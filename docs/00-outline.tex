\chapter{Introduction}

% \section{Motivation}

% \textcolor{red}{CO2 emissions}

% \textcolor{red}{Emissions by energy source}

% \section{Research Objectives}

% \section{Reproducing this work}
% Talk about \texttt{snakemake}, show the DAG... 

\chapter{Literature Review}
\label{chapter:lit-review}
% Every year, world leaders meet to discuss plans to address climate change at the
COP summit. In 1995, world leaders established a set of targets with the Kyoto
Protocol \cite{united_nations_kyoto_1998} and again with the 2016 Paris Climate
Agreement \cite{united_nations_paris_2015}. Every few years, the United Nations
releases a report from the \ac{ipcc} assessing the current impacts of climate
change and forecasting future scenarios. Most of the world understands that
anthropogenic climate change is an existential threat to society. Indeed, many
studies in the \ac{esom} literature begin with a statement about the urgency of
climate change. This chapter reviews the extant literature for both quantitative
and qualitative analyses of the problem considered in this thesis --- primarily
bridging the gap between feasibility or planning studies to address the climate
crisis and the current pattern of missed targets and growing carbon emissions.
First, I draw from the risk assessment literature to characterize and situate
the problem of climate change and demonstrate the necessity of a holistic
analysis. Second, I build upon the central issue of disproportionality of
climate change risk by reviewing the energy and environmental justice
literature. Third, I develop an encompassing definition of an ``energy system''
using technical and social perspectives. Finally, I review the energy system
literature for gaps in conventional modeling practices and identify previous
attempts to incorporate social science and justice concepts into energy system
models.

% The first section reviews methods and attempts to model energy systems for
% planning purposes. The second section reviews the ways social movements help
% or hinder various energy projects, how governments succeed or fail to achieve
% their energy goals from this perspective, and why some communities favor or
% disfavor energy projects.

\input{2-literature/23-social-movements}
% \input{2-literature/26-nuclear-energy} 
% \input{2-literature/27-renewables}
% \input{2-literature/28-procedures}
\input{2-literature/25-energy-justice}
\input{2-literature/21-esoms}
\input{2-literature/22-moop}
\textcolor{red}{\section{Hyper-local energy system modeling}}
\textcolor{red}{To the literature review, I want to add a section discussing 
the research on modeling at the municipal level. These papers start with the 
following: 
\cite{mckenna_combining_2018,johannsen_municipal_2023,ben_amer_too_2020}.}
\section{Modeling and Quantifying Energy Justice}

The dearth of studies that incorporate energy justice into \acp{esom} highlights
the challenge of combining these techniques. The literature on energy justice
and socio-technical transitions tend to derogate modeling efforts as cold and
calculating \cite{sovacool_energy_2015,sovacool_energy_2016}, and most models do
not account for energy justice in either equations or analysis. However, there
have been some notable attempts to bridge this gap. The following studies by
Patrizio et al. (2020) \cite{patrizio_socially_2020} and Neumann \& Brown (2021)
\cite{neumann_near-optimal_2021} explicitly use \acp{esom} in their analyses.
Although the works by Chapman et al. (2018) \cite{chapman_prioritizing_2018} and
Mayfield et al. (2019) \cite{mayfield_quantifying_2019} do not use \acp{esom} as
described in Section \ref{section:esoms}, these contributions quantify some
features of their respective energy systems and how they relate to notions of
energy justice and equity.

Patrizio et al. (2020) conducted a technology-agnostic `social equity' scenario
that maximized the \ac{gva} of several countries' energy systems rather than
minimizing the total cost \cite{patrizio_socially_2020}. \Ac{gva} is also
distinct from social welfare because it measures contributions to \ac{gdp} from
individual producers rather than maximizing surplus. This metric enables
sector-specific analysis of the impacts of energy infrastructure on employment
and sales. Equity, in this context, is identical to socioeconomic development as
measured by \ac{gdp}. Using this definition of equity, the researchers looked at
a socio-technical transition for three countries: Spain, the United Kingdom, and
Poland. They found that a 100\% renewable energy system would reduce labor
compensation by 50-60\% in the UK and Poland but could increase benefits in
Spain. They argue this is due to the outsourcing of manufacturing and mining
jobs in the former cases, while Spain has enough domestic resources to
accommodate the transition. The researchers did not analyze possible shifts in
power dynamics related to the energy systems, but they did identify that there
is no one-size-fits-all solution to achieving net-zero carbon emissions.

Neumann \& Brown (2021) performed a detailed analysis of the European energy
system considering the expansion of transmission networks and energy producers
for a 100\% renewable energy system under cost minimization
\cite{neumann_near-optimal_2021}. They also used a novel formulation of \ac{mga}
to identify the boundaries of the feasible space for each technology within
different levels of tolerance. This study uses Lorenz curves and Gini
coefficients to measure the uniformity of the distribution of energy production
and consumption. In other words, the most equitable distribution of energy
resources would accord with energy consumption \cite{neumann_near-optimal_2021}.
The researchers conclude that wind power and greater transmission capacity are
associated with less regional equity, while solar power and storage technologies
lead to a more even distribution of the power supply. This is useful for
measuring the distribution of energy benefits from the energy system but does
not consider the distribution of costs nor consider regional preferences. 

Chapman et al. (2018) looked at the energy justice implications of transitioning
coal plants to renewable energy projects for the nearby communities
\cite{chapman_prioritizing_2018}. They measure distributional justice with
``relative equity'' and ``policy burden.'' Relative equity accounts for factors
such as \ac{ghg} reduction, employment, electricity cost, and health impacts.
Policy burden is a weighted value according to the income level of each
community. These two quantities were plotted together to identify a retirement
schedule that maximizes equity outcomes and ensures that burdens are borne by
the most capable communities \cite{chapman_prioritizing_2018}. Additionally, the
researchers argue that by using equity measures to inform policy choices, those
policy decisions are more procedurally just. However, this neglects meaningful
participation and may or may not address decision-making transparency
\cite{sovacool_energy_2015}. Further, this study does not consider how replacing
dispatchable suppliers with \ac{vre} will affect the availability and
affordability of electricity \cite{sovacool_energy_2015}. This latter challenge
could be addressed by incorporating methods from the \ac{esom} literature. The
former issue of decision-making transparency is one of the motivations for this
thesis.

Mayfield et al. (2019) quantified the social equity implications for the
expansion of natural gas infrastructure in Appalachia using spatial and temporal
metrics such as job-years generated by greater gas development, premature deaths
caused by air pollution, changes in poverty and income, and the distribution of
these various benefits along regional, racial, and economic lines. Additionally,
they identified some of the intergenerational equity impacts of climate change
and expanded gas infrastructure. 

\subsection{Enabling Procedural Justice Through Energy Models}

Traditionally, \acp{esom} are used to inform policy-makers \cite{li_open_2020}
in order to infuse policy choices with an appearance of objectivity. Indeed,
some of the studies reviewed in the previous section argue that this infusion
will lead to greater procedural and recognitional justice outcomes as long as
the policies maximize some measure of energy justice
\cite{chapman_prioritizing_2018, heffron_resolving_2015}. However, these types
of detailed analyses may also be used to dismiss concerns or opposition from the
public due to insufficient `technical expertise' \cite{johnson_dakota_2021,
susser_better_2022}. Further, without meaningful participation from the affected
public, this approach further entrenches procedural injustices. To credit the
energy modeling community, there is significant awareness of the importance of
transparency and repeatability in the space \cite{decarolis_case_2012,
pfenninger_energy_2014, pfenninger_openmod_2022, forster_open_2022,
hilpert_open_2018}. Yet these two goals are challenged by the computational
resources required to run the more complex and detailed models, as well as the
learning curve necessary to understand and modify the model inputs themselves.
There has been some effort to reduce this learning curve and make modeling
itself more accessible. Frameworks such as METIS, EnergyRT, and \ac{pygen} all
emphasize reproducibility, user-friendliness, and a shallower learning curve
\cite{sakellaris_metis_2018, lugovoy_energyrt_2022, dotson_python_2021}. The
creators of METIS state their goal is to ``close the gap between modelers and
policy-makers, enabling policy-makers to become modelers''
\cite{sakellaris_metis_2018}. However, these frameworks do not offer
computational resources to run their models. The \ac{temoa} project offers
limited cloud computing capabilities, free of charge
\cite{temoa_project_temoa_2023}. However, the responsibility for creating an
input file still falls to the user, which can be overwhelming even for
experienced modelers. For this reason, many municipalities hire external
organizations to run models since cities and towns typically do not have the
capacity to do it themselves \cite{ben_amer_too_2020, johannsen_designing_2021}.
Additionally, researchers identified gaps between the needs of modelers and
users, such as the inflexibility of models to reflect multiple policy options.
``Many models include a carbon price, often labelled as `climate policy,' as the
only explicit policy instrument'' \cite{susser_better_2022}. Despite broad
agreement about the need to incorporate stakeholders the ``process'' of
modeling, there have been surprisingly limited efforts so far
\cite{ben_amer_too_2020,johannsen_designing_2021,susser_better_2022} Finally,
it's not clear that perfectly accessible and transparent modeling tools will
translate to more procedurally just policy-making. The next sections outline
some methods used to address this challenge.

\subsection{\Acl{pve}}

Even if the public could use modeling tools, their testimony may still be
dismissed due to a `lack of expertise.' However, the public has preferences that
should be incorporated into decision-making. Additionally, community members are
frequently able to assess trade-offs when presented with them. \Acf{pve} is one
method for translating community preferences into just policy outcomes.
Researchers in the Netherlands developed this method to enhance democratic
participation and infuse policies with genuine feedback from constituents
\cite{mouter_introduction_2019}. They observed that a common method of assessing
social impacts is \ac{wtp}, which is the maximum price an individual is willing
to pay for a good or service, yet individual purchasing habits do not
necessarily reflect their views on public policy due to the relative salience of
moral considerations \cite{mouter_introduction_2019}. With \ac{pve},
participants can allocate a specific amount of the public budget for certain
policies, including levying or reducing taxes for greater or lesser government
spending \cite{mouter_introduction_2019}. Researchers applied \ac{pve} in three
different settings, mobility and transportation \cite{mouter_contrasting_2021},
flood risk projects (i.e., a climate hazard \textit{infrastructure} response)
\cite{dekker_economics_2019}, and with a phaseout of natural gas
\cite{mouter_including_2021}. Importantly, the studies also measured the impact
of these interventions and found that \ac{pve} enables participation from people
that do not typically participate (recognition), the results were useful for
decision-making and participation was meaningful for the majority of subjects
\cite{mouter_including_2021}. Although previous applications of \ac{pve} focused
on economic policy levers, this approach offers a promising pathway toward
identifying equitable and just energy mixes for the future. The next section
reviews some efforts to enhance outcomes via direct collaboration between energy
modelers and communities themselves.

\subsection{Participatory modeling processes}

Although \ac{pve} offers one method to elicit stakeholder input, researchers are
beginning to explore transdisciplinary methods for engaging members of the
public. McGookin et al. 2021 \cite{mcgookin_participatory_2021} systematically
reviewed (``the review'') existing literature on participatory energy modeling.
The review found a major difference in goals and participants for studies in
either national or local contexts. Studies at the national scale sought
impactful policy outcomes and engaged with policy technical experts, while
localized projects aimed to materially improve the studied communities
\cite{mcgookin_participatory_2021}. However, even in the latter cases, only a
fraction of the reviewed studies included non-academic stakeholders. This
suggests that local knowledge is still an under-utilized resource and that
researchers need to do more to identify communities' actual priorities.

The review offered three overarching benefits of participatory modeling. First,
involving more stakeholders improves perceptions of legitimacy and robustness
\cite{mcgookin_participatory_2021}. Discussion and tradeoff analysis lead to
solutions which are more socially and politically feasible. This corroborates
literature from social movement theory which show that support for energy policies
and projects is conditioned on genuine public participation rather than
\ac{nimby} sentiments \cite{summers_influencing_2020,ottinger_procedural_2014,
walker_procedural_2017,barragan-contreras_procedural_2022,gonyo_resident_2021,konisky_proximity_2021}.
McGookin et al. also indicate that transparency in the model assumptions builds
trust.
Second, participatory methods build capacity for individuals and communities.
Non-academic participants benefit from learning about new solutions, while
policymakers and modelers gain new perspectives and a better understanding of a
community's reality, improving decision quality. Communities benefit through the
formation of new social networks and stronger relationships among stakeholder
groups. The benefit to communities found in the review aligns with the
understanding of energy systems as a ``gathering force'' that can produce new
polities and social identities \cite{bridge_energy_2018}. Further, the review
notes that one of the keys for enhancing decision-making at both local and
national levels ``was the insight into systems thinking and trade-offs or
cause-effect relationships that participants gained from the methods used in
studies.'' \ac{mcda}, a class of methods for analyzing tradeoffs in different
scenarios, was the most popular tool found in the review
\cite{mcgookin_participatory_2021}. Lastly, the authors note that consensus
building through controversial discussions, collective learning, and jointly
owned solutions was effective for building trust. Both national and local
studies found increased commitments for stakeholders to implement co-created
decisions \cite{mcgookin_participatory_2021}. These findings support greater
participation in a ``deliberative democracy'' \cite{dryzek_deliberative_2013}.
While the review does not explicitly identify these benefits with energy
justice. The benefits of particpatory energy modeling evinced by the literature
are consistent with both procedural and recognition justice.



% \textcolor{red}{This enumeration will be good for the following chapter where
% I introduce my theoretic framework --- \textit{why} should we engage
% stakeholders in modeling?} The authors proposed a conceptual framework for
% successful integration of participatory methods. \begin{enumerate} \item
% Stakeholder engagement must occur before drawing conclusions from a
% quantitative analysis. \textcolor{red}{I'm extending this conceptual framework
% by describing precisely where in the modeling process stakeholder input can
% and should take place, and in what forms.} \item There should be an iterative
% process where stakeholders can shape the process in addition to evaluating the
% results. \item Lastly, the ideal process should include stakeholder input at
% every stage. \end{enumerate}

\subsubsection{Persistent gaps}
Although there have been attempts to integrate energy justice and energy modeling
there are still gaps in the literature. These gaps include:
\begin{enumerate}
    \item Lack of awareness about the role of normativity in energy modeling.
    Only one paper in the review by McGookin et al. 2021 investigated the
    normative assertion that renewable energy is the ``cornerstone'' of
    decarbonization \cite{zelt_long-term_2019}.
    \item Lack of understanding the normative premises that inform the rationale
    for participatory methods. For example, there is little to no discussion
    about disproportionality in the relevant literature. Only one paper mentions
    ``procedural justice'' indicating that it's not yet a salient term among the
    modeling community \cite{knudsen_local_2015}.
    \item There is a dearth of studies considering participatory modeling in a
    U.S. context.
    \item Studies reviewing model development focus on skill-based barriers
    rather than structural barriers to energy modeling at the municipal level.
    \item Although some studies in the McGookin review
    \cite{mcgookin_participatory_2021} included non-academic participants, the
    connection between the study objectives and actionable policies is unclear.
    How does modeling fit into the policymaking process?
\end{enumerate}

In summary, climate change is a multi-dimensional existential threat to society.
Transitioning to a zero-carbon economy by decarbonizing our energy systems may
prevent the worst outcomes of climate change. However, energy systems do not
only transport electrons and gas but also mediate socio-political power.
Therefore this transition must be done equitably in order to avoid entrenching
further injustices. The existing energy system modeling tools and literature
routinely ignore the social dimensions of these systems and forego true
trade-off analysis. Additionally, it's unclear whether improving these modeling
practices will correspond to just energy policy outcomes. This thesis attempts
to bridge the gap between energy system modeling and energy justice by
developing a novel framework that allows multiple, and perhaps non-economic,
objectives and is designed for transparency and usability by non-modelers to
inform energy policy decisions. A framework such as the one developed in this
thesis may be used in conjunction with a policy process like \ac{pve} or
participatory modeling to fully enclose the triumvirate of energy justice
tenets: distribution, procedure, and recognition.



\chapter{\acf{osier}}
\label{chapter:osier}
% \chapter{\acf{osier}}
\label{chapter:osier}

This chapter introduces \acf{osier}, a novel open-source energy system modeling
framework for \acl{moo} \cite{dotson_osier_nodate}. There are currently no
\acp{esom} that enable \ac{moo} and \ac{osier} fills that gap. Figure 
\ref{fig:osier_flow} illustrates the flow of data into and within \ac{osier}.

\begin{figure}[H]
    \centering
    \includegraphics[width=\columnwidth]{figures/osier_flow}
    \caption{The flow of data into and within \ac{osier}}
    \label{fig:osier_flow}
\end{figure}

Technology data, objectives, constraints, and a dispatch model are all features
within \ac{osier}, while \ac{pymoo} drives the optimization of these objectives.
The dispatch model is independently executable for inspecting specific test
cases and mapping solutions from other solvers onto \ac{osier}'s objective
space. The next section elaborates on the dispatch model's formulation.

\textcolor{red}{This chapter is mostly written but there are a few things that
should be added.}

\textcolor{red}{\begin{enumerate}
    \item Update to the \ac{mga} algorithm.
    \item Add description of the ``greedy'' farthest-first traversal algorithm.
    \item Move the description of \ac{pygen} and \ac{temoa} to either a
    different section or a different chapter since those pieces do not belong to
    \ac{osier} itself.
    \item Add a description of the ``simplified approach'' to capacity expansion
    that will be added to the model and used the example chapter.
    \item Remove the description of model data.
    \item Add a description of the hypothetical \ac{osier} procedure, as
    proposed in the prelim.
\end{enumerate}}





\section{Inputs}

% \textcolor{red}{General thoughts:}

% Accurate input data are essential but curating data represents a challenging step in the
% modeling process. \ac{osier} attempts to lower this barrier by providing a
% variety of technology data from ``reliable'' sources.

% This section connects to the normative and descriptive portions of modeling. 

This section describes the input data and parameters that users must provide to run an \ac{osier} simulation. Broadly, \ac{osier} needs
technology data and some objectives to optimize. 

\subsection{Technology Data}
\ac{osier} only requires marginal costs and technology names in order to run successfully.
These data are needed to run the dispatch model. \ac{osier} also accepts operational data, such
as ramp rates and storage capacities. Additionally, \ac{osier} will understand any bespoke 
piece of data (e.g., ``popularity,'' ``technology-readiness score,'' or anything else) that
might be needed for a user-defined objective. All of these data are passed to \ac{osier}
through an \texttt{osier.Technology} object. Code listing \ref{listing:user-defined-technologies}
shows how users can create a simple technology object.

\begin{listing}[!ht]
    \caption{A basic technology object in \ac{osier}.}
    \label{listing:user-defined-technologies}
    \begin{minted}
    [ frame=lines, framesep=2mm, baselinestretch=1.2, bgcolor=LightGray,
    fontsize=\footnotesize, linenos ] {python} 
    from osier import ThermalTechnology

    fusion = ThermalTechnology(technology_name="Fusion",
                               dispatchable=True,
                               renewable=False,
                               fuel_cost=10*(GWh)**-1,
                               lifecycle_co2_rate=0.0,
                               )
    \end{minted}
    \end{listing}

\subsection{Objectives}
\label{section:osier_objectives}

There are many possible objectives to optimize. This section summarizes a few of
them and how they may be calculated in \ac{osier}. Due to \ac{pymoo}'s
structure, all objectives are minimized. Therefore, if users wish to maximize
some quantity, it must be reformatted with a reversal in sign to be an equivalent minimization objective.

\subsubsection{Per-unit-capacity}

Some quantities of interest depend on the \textit{capacity} of each technology.
For example, land use of different energy producers is often reported as a land
density $\text{km}^2/\text{MW}$. A generalized specific \textit{density} with respect to power 
may be $\text{unit}/\text{MW}$. The objective function for these quantities reads
\begin{align}
    \mathcal{K} &= \sum_g^G \textbf{CAP}_g \kappa_g ,
    \intertext{Where}
    \kappa &= \text{the power density of the \textit{g-th} technology} \quad \left[\frac{-}{MW}\right].
\end{align}

Table \ref{tab:objectives-per-capacity} lists some example objectives that could be
minimized or maximized.

\begin{table}[h]
    \centering
    \caption{Example objectives on a per-unit-capacity basis.}
    \begin{tabular}{cc}
       \toprule
       Quantity  & Units (per MW)\\
       \midrule
        Land Use & $\left[\text{km$^2$}\right]$\\
        Employment & $\left[\text{jobs}\right]$\\
        Capital Cost & $\left[\text{\$}\right]$\\
        Fixed O\&M Cost & $\left[\text{\$ / year}\right]$\\
        \bottomrule
    \end{tabular}
    \label{tab:objectives-per-capacity}
\end{table}

\subsubsection{Per-unit-energy}

Some quantities of interest depend on the \textit{amount of energy produced} by
each technology. For example, carbon emissions only occur when a coal or natural
gas plant burns fuel. A generalized specific \textit{intensity} with respect to energy 
may be in $\text{unit}/\text{MWh}$. The objective function for these quantities reads
\begin{align}
    \mathcal{E} &= \sum_g^G \xi_g \sum_t^T x_{g,t},
    \intertext{where}
    \xi_g &= \text{the energy density of the \textit{g-th} technology}\quad
    \left[\frac{-}{MWh}\right].
\end{align}

\begin{table}[h]
    \centering
    \caption{Example objectives on a per-unit-energy basis.}
    \begin{tabular}{cc}
       \toprule
       Quantity  & Units (per MWh)\\
       \midrule
        \acs{ghg} Emissions & $\left[\text{kg}\right]$ \\
        Water Use & $\left[\text{L}\right]$\\
        ``Safety'' & $\left[\text{deaths}\right]$\\
        Fuel Cost & $\left[\text{\$}\right]$\\
        Variable O\&M Cost & $\left[\text{\$}\right]$\\
        \bottomrule
    \end{tabular}
    \label{tab:objectives-per-energy}
\end{table}

\subsubsection{Reliability and Predictability}

Reliability has many definitions in the literature and it also depends heavily
on the dispatch method. A hierarchical flow, which dispatches energy based on a
set of rules (as opposed to true cost minimization), may simply report the
fraction of hours when electricity demand was not met by the model
\cite{donado_hyres_2020,bilil_multiobjective_2014,kamjoo_multi-objective_2016,riou_multi-objective_2021}.
\acs{lp} or \acs{milp} formulations typically have an energy balance constraint requiring 
electricity demand to be satisfied at all times, or within some specified tolerance. 
Thus reliability may be translated into a cost by determining consumers'
\ac{wtp} for electricity \cite{gorman_quest_2022, najafi_value_2021}. However,
this thesis relates system reliability to price volatility and net demand
predictability. Since the price of electricity is determined by matching supply
and demand, the price will spike when supply and demand are out of phase. For
instance, geopolitics may cause the supply of natural gas to drop, increasing
the spot price of electricity. Or, more commonly, the availability of solar and
wind resources may fall unexpectedly, leading to a greater demand for backup
energy. Both of those examples are difficult to predict; otherwise, fuel
reserves could be deployed, avoiding the price shock. Thus, I propose that
measuring the predictability and volatility of an energy system is an
appropriate proxy for reliability. Additionally, minimal price volatility is
considered an aspect of energy justice \cite{sovacool_energy_2015,
van_uffelen_revisiting_2022}.

In this thesis, I measure the  predictability of hourly electricity prices and
net demand using a measure from complexity science, \ac{wpe}
\cite{fadlallah_weighted-permutation_2013}. Permutation entropy, the precursor
to \ac{wpe}, is essentially the Shannon entropy for particular sequences of
values called \textit{motifs} \cite{bandt_permutation_2002}. \ac{wpe} expands on this
concept by weighting each instance of a motif by its variance
\cite{fadlallah_weighted-permutation_2013,garland_model-free_2014}. \ac{wpe} is
defined as
\begin{align}
    H_w(m) &= -\sum_{\pi \in \Pi} P_w(\pi)\log_2(P_w(\pi))
    \intertext{where}
    \pi &= \text{a particular motif,}\nonumber\\
    P_w &= \text{the probability of a given motif, $\pi$,}\nonumber\\
    &= \frac{\mathlarger{\sum\limits_{j\leq N}} w\left(x_j^{(m, \tau)}\right)\cdot\delta\left(\phi\left(x_j^{(m, \tau)}\right), \pi_i\right)}{\mathlarger{\sum\limits_{j\leq N}} w\left(x_j^{(m, \tau)}\right)}
    \intertext{and}
    w\left(x_j^{(m, \tau)}\right) &= \text{the weight of a particular vector}\nonumber\\
      &= \frac{1}{m}{\sum_{j}^m} \left(x_j^{(m,\tau)} - \Bar{x}\right)^2,\\\
     \phi(\cdot) &= \text{the ordinal pattern of a vector,}\nonumber\\
     \delta(\cdot) &= \text{Kronecker delta,}\nonumber\\
     m &= \text{the embedding dimension,}\nonumber\\
     \tau &= \text{the time delay}\nonumber.
\end{align}

There are other reliability metrics in the literature, frequently employing some
variation on the ``spread'' of data through standard deviation  or mean squared
error \cite{galvani_optimal_2021, galvani_unified_2014,
delsole_predictability_2004}. However, these metrics are unbounded and do not
contain any information about the underlying dynamics that produce a certain
distribution. Whereas \ac{wpe} can indicate a theoretical ceiling on
predictability \cite{garland_model-free_2014}. Importantly, \ac{wpe} works for
systems where the underlying dynamics are unknown. The Hurst exponent is another
measure of predictability, but it too has drawbacks, such as computational
expense and a stationarity requirement \cite{mesa_hurst_1993,
chandrasekaran_investigation_2019}. This thesis uses the \ac{wpe} implementation
I contributed to the open source package \texttt{PyEntropy}
\cite{donets_pyentropy_2023}.

\subsubsection{User-defined Objectives}

A key feature of \ac{osier} is the ability for users to define their own
objectives relatively easily. This feature is required
because modelers cannot know \textit{a priori} every objective that users might
be interested in optimizing. While \ac{osier} ships with some standard objective
functions, allowing users to create their own objectives makes every model
bespoke. With requisite user-supplied data \textit{any quantitative metric may be used as an objective in
\ac{osier}.} Every objective function has at least two arguments, the list of
technologies used in the model and the solved dispatch model. Users will never
have to pass these arguments manually since \ac{osier} will automatically call
the function during a simulation. One example of a user-defined objective might
be technology readiness. This objective is independent from the energy produced
and could be weighted by the capacity but is not a per-unit-capacity objective.
The values of the readiness parameter must be passed to each \texttt{Technology}
object, which can be accessed at run-time. Code listing
\ref{listing:user-defined-objective} shows the basic approach to creating a new
objective. 

\begin{listing}[!ht]
\caption{The fundamental way to create a novel objective in \ac{osier}.}
\label{listing:user-defined-objective}
\begin{minted}
[ frame=lines, framesep=2mm, baselinestretch=1.2, bgcolor=LightGray,
fontsize=\footnotesize, linenos ] {python} 

nuclear.readiness = 9
fusion.readiness = 3

technology_list = [nuclear, fusion]

def osier_objective(technology_list, solved_dispatch_model): 
    """ 
        Calculate the capacity-weighted technology readiness 
        score for this energy mix. 
    """

    total_capacity = np.array([t.capacity for t in technology_list]).sum()
    
    objective_value = np.array([t.readiness*t.capacity 
                                for t in technology_list]).sum()

    return objective_value / total_capacity
\end{minted}
\end{listing}
\noindent
Importantly, because all technologies in \ac{osier} are Python objects, users
can add attributes at will. Such as the technology readiness level as shown in
Code listing \ref{listing:user-defined-objective}. 

\subsection{Constraints}
\label{section:constraints}

Besides the physical constraints defined in Section \ref{section:dispatch},
\ac{osier} does not have any default constraints. This is because each
additional constraint corresponds to an additional assumption and will affect
the trade-off analysis that makes \ac{moo} so powerful. However, there are some
circumstances where the optimal solutions are still
infeasible in practice. For instance, if a community wants to determine the best
energy mix according to their unique objectives, this community might not have
the budget for even a least-cost solution because the capital requirements are
too high. Therefore, they must constrain the capital cost for their modeling
problem. Thus, \ac{osier} enables the following:
\begin{enumerate}
    \item Users may define their own constraints.
    \item Any objective function may be transformed into a constraint.
\end{enumerate}
This feature makes \ac{osier} unique among \acp{esom}.
Single-objective \acp{esom} can never account for unique situations such as the
one suggested above, nor any other bespoke considerations. In the case above,
the capital cost may constrain the problem while still minimizing the total
cost. The solutions under these conditions will have a higher total cost but
could be achievable in the near term due to meeting capital cost requirements.
\input{3-osier/32-genetic-algorithms}
\section{Dispatch Model}
\ac{osier} offers two models for dispatching energy. The first is a merit order
dispatch model that optimally dispatches electricity according to the marginal
cost of generation with perfect foresight. This model is formulated as a
\acf{lp} and written in \ac{pyomo}. Similarly, the second dispatch model follows
merit-order but electricity is dispatched according to a hierarchy of rules and
is myopic. This approach facilitates parallelization and reduces the overhead
cost of problem setup. However, dispatch solutions may be sub-optimal compared 
to the \ac{lp} formulation.

\subsection{Economic/Merit-order dispatch}
\label{section:merit_order}

 The economic dispatch model minimizes the generation cost subject to physical
 constraints but does not optimize capacity investments. The complete set of
 equations for the model is detailed below.
\begin{align}
    \intertext{Minimize: }
    \label{eq:dispatch-objective}
    &\left(\sum_t^T\sum_g^G \left[C_{g,t}^{fuel} + C_{g,t}^{vom}\right]x_{g,t}
    \right)+\left(\sum_t^T\sum_g^S x_{g,t}c_{g,t}\pi\right)\\
    \intertext{such that,}
    \intertext{1. The generation meets demand, less the amount of energy stored or curtailed, 
    within a user-specified tolerance (undersupply and oversupply),}
    \left[\sum_g^Gx_{g,t}-\sum_g^S c_{g,t}\right] &\geq \left(1-\text{undersupply}\right)\text{D}_t\quad \forall \quad t \in T, S, \\
    \left[\sum_g^Gx_{g,t}-\sum_g^S c_{g,t}\right] &\leq \left(1+\text{oversupply}\right)\text{D}_t \quad \forall \quad t \in T, S,
    \intertext{2. A generator's production, $x_{g}$ does not exceed its capacity at any time, $t$}
    x_{g,t} &\leq \textbf{CAP}_{g}\Delta \tau \quad \forall \quad g,t \in G,T
    \intertext{3. A generator's ramping rate is never exceeded,}
    \frac{x_{r,t} - x_{r,t-1}}{\Delta \tau} = \Delta P_{r,t} &\leq
        \rho^{up}_g\textbf{CAP}_g\Delta\tau \quad \forall \quad r,t
        \in R, T,\\
    \frac{x_{r,t} - x_{r,t-1}}{\Delta \tau} = \Delta P_{r,t} &\leq
        -\rho^{down}_g\textbf{CAP}_g\Delta\tau \quad \forall \quad r,t
        \in R, T,
    \intertext{4. Storage capacity for each storage technology is never exceeded}
    \textbf{SOC}_{s,t} &\leq \textbf{CAP}^S_{s} \quad \forall \quad s,t \in S,T,
    \intertext{5. Storage discharge cannot exceed stored energy.}
    x_{s,t} &\leq \textbf{SOC}_{s,t} \quad \forall \quad s,t \in S,T,
    \intertext{6. Storage charge rate cannot exceed unit capacity}
    c_{s,t} &\leq \textbf{CAP}_{s}\Delta \tau \quad \forall \quad s,t \in S,T,
    \intertext{where,}
    G &= \text{ the set of all generating technologies},\nonumber\\
    R &= \text{ the set of all ramping technologies}, \quad R \subset G,\nonumber\\
    S &= \text{ the set of all storage technologies}, \quad S \subset G,\nonumber\\
    T &= \text{ the set of all time periods in the model},\nonumber\\
    D_t &= \text{ the demand at each time period, \textit{t}},\nonumber\\
    \textbf{CAP}_g &= \text{ the capacity of the \textit{g-th} technology}\quad \left[MW\right],\nonumber\\
    \textbf{CAP}^S_g &= \text{ the storage capacity of the \textit{g-th} technology}\quad \left[MWh\right],\nonumber\\
    \textbf{SOC}_{s,t} &= \text{ the state of charge of the \textit{g-th} technology at time \textit{t}}\quad \left[MWh\right]\nonumber,\\
    \Delta\tau &= t_{i+1} - t_i \quad \forall \quad t_i \in T \quad \left[h\right],\nonumber\\
    x_{g,t} &= \text{ the energy produced by generator, \textit{g}, at time, \textit{t}}\quad \left[MWh\right]\nonumber,\\
    c_{s,t} &= \text{ the energy stored by storage technology, \textit{s}, at time, \textit{t}}\quad \left[MWh\right]\nonumber,\\
    \rho_g &= \text{ the up/down ramp rate for technology, \textit{g}} \quad \left[-\right]\nonumber,\\
    \pi &= \text{ A small penalty for simultaneous charging and discharging.}\nonumber
\end{align}
The second term in the objective function, Equation \ref{eq:dispatch-objective},
represents a minor penalty to prevent the unphysical behavior of simultaneous
charging and discharging from storage technologies. I used this approach because
constraining this behavior requires a binary variable that makes the problem
non-convex and therefore requires a more sophisticated solver. A small but
sufficiently large $\pi$ will always nullify the penalty term. This dispatch
model reflects the minimum physical constraints for an energy system without
considering fine-scale operational details such as frequency control. Equation
\ref{eq:dispatch-objective} assumes that the retail cost for generating
electricity is identical to the marginal cost of producing electricity. 

\subsection{Hierarchical dispatch}

Similar to the \ac{lp} formulation described in Section
\ref{section:merit_order}, the hierarchical dispatch follows a set of rules to
dispatch energy. Technologies are first sorted by their marginal cost. If two
technologies have the same marginal cost the tie is broken by comparing the 
technology efficiencies. Then, model loops through each value in the provided 
electricity demand time series. A second loop runs through the sorted list of 
technology objects and calculates the power output for each technology. There
are three broad types of technologies in \ac{osier}: Standard technologies,
ramping technologies, and storage technologies. The power output for a standard
technology is given by

\begin{align}
    x_{g,t} &= \text{max}\left(0, \text{min}\left(D_t, \textbf{CAP}_g\right)\right)\label{eq:std_tech_output}.
\end{align}

\noindent
This equation guarantees that a standard technology will have a power output between
zero and its rated capacity. In addition to Equation \ref{eq:std_tech_output},
a ``ramping'' technology will calculate the maximum and minimum attainable power
levels given its ramp rates and the problem's time resolution. The power output
is given by

\begin{align}
    x_{g,t} &= \begin{cases}
         \text{max}\left(D_t, P_{min}\right) & D_t < x_{g,t-1}\\
         \text{min}\left(D_t, P_{max}\right) & x_{g,t-1} \leq D_t \leq \textbf{CAP}_g \\
         P_{max} & x_{g,t-1} \leq \textbf{CAP}_g \leq D_t
     \end{cases}
     \intertext{where}
     P_{min} &= \text{max}\left(0, \left(x_{g,t-1}-\rho_g^{down}\Delta \tau\right)\right),\nonumber\\
     P_{max} &= \text{min}\left(\left(x_{g,t-1}+\rho_g^{up}\Delta \tau\right), \textbf{CAP}_g\right).\nonumber
\end{align}

\textcolor{red}{$\rho_g$ is not the same $rho_g$ as in the LP formulation! Maybe
choose another variable and explain its meaning... It should be the "ramp up/down"
which is the ramp rate times the capacity.}

Finally, a storage technology has a charging and discharging mode, depending on
whether the current demand level is positive or negative. A storage technology
extends Equation \ref{eq:std_tech_output} by checking that the storage unit has enough
\textit{storage} capacity. Thus the power output is given by

\begin{align}
    x_{g,t} &= \begin{cases}
        -\frac{E_{in}}{\Delta \tau} & D_t < 0\\
        \frac{E_{out}}{\Delta \tau} & D_t \geq 0\\
    \end{cases}
    \intertext{where}
    E_{in} &= \text{min}\left(\left(\textbf{CAP}^S_g-\textbf{SOC}_{g,t}\right), P_{in}\Delta \tau\right),\nonumber \\
    E_{out} &= \text{min}\left(P_{out}\Delta \tau, \textbf{SOC}_{g,t}\right),\nonumber \\
    P_{out} &= \text{max}\left(0, \text{min}\left(D_t, \textbf{CAP}_{g}\right)\right),\nonumber\\
    P_{in} &= \text{min}\left(\lvert\text{min}\left(0, D_t\right)\rvert, \textbf{CAP}_g\right).\nonumber
\end{align}

\noindent
Since this algorithm is myopic, solution optimality is not guaranteed. Higher
penetration of renewable energy and energy storage increases the influence of
this myopia due to the unpredictability of these resources. Thus, this approach
may be more ``realistic'' than an truly optimal dispatch. Additionally, this
method is faster from reduced problem setup. Figure \ref{fig:hierarchy_algorithm} illustrates the flow of the
algorithm.

\begin{figure}[H]
    \centering
    \begin{tikzpicture}[node distance=1.7cm]
            \tikzstyle{every node}=[font=\small] 
            \node (1) [lbblock]{\textbf{Sort technologies by marginal cost}}; 
            \node(2) [lbblock, below of =1] {\textbf{Start dispatch loop}};
            \node(3) [lbblock, below of=2] {\textbf{Calculate power output for
            current technology}};
            \node(4) [lbblock, below of=3] {\textbf{Decrement current demand
            value \\ by power output}};
            \node(5) [lbblock, below of=4] {\textbf{Reached last technology?}};
            \node (7) [lbblock, below of=5] {\textbf{Reached the end of the
            demand time series?}}; 
            \node (8) [loblock,below of=7] {\textbf{Done}}; 
            \draw [arrow] (1) -- (2); 
            \draw [arrow] (2) -- (3);
            \draw [arrow] (3) -- (4);
            \draw [arrow] (4) -- (5);
            \draw [arrow] (5) -- node[anchor=east]{yes} (7); 
            \draw [arrow] (5) -- ([shift={(0.5cm,0cm)}]5.east) -- node[anchor=west] 
            {no} ([shift={(0.5cm,0cm)}]3.east)--(3);
            \draw [arrow] (7) -- (8); 
            \draw [arrow] (7) -- node[anchor=east]{yes} (8); 
            \draw [arrow] (7) -- ([shift={(1.15cm,0cm)}]7.east) -- node[anchor=west] 
            {no} ([shift={(1.15cm,0cm)}]2.east)--(2);
    \end{tikzpicture}
    \caption{The hierarchical dispatch flow.}
    \label{fig:hierarchy_algorithm}
\end{figure}
\section{\acs{mga} with \acl{moo}}
\label{section:mga-moo}

\ac{osier} addresses structural uncertainty in two ways. First, it uses 
\ac{moo} to identify tradeoffs. Second, since structural uncertainty will
persist regardless of the number of modeled objectives, \ac{osier} offers
a high-dimensional version of the classic \ac{mga} approach to handling
structural uncertainty.
This thesis applies some ideas from \ac{mga} to the analysis of the sub-optimal
space from a \acl{moo} problem. Due to their iterative process, \acp{ga}
naturally generate many samples in a problem's feasible space. However, this
does not lead to a ``limited set'' of solutions but rather a potentially
unbounded set. Some literature developed \acp{ga} that directly use \ac{mga} in
the iterative process
\cite{zechman_evolutionary_2004,zechman_evolutionary_2013}. However, existing
Python libraries such as \ac{pymoo} and \ac{deap} do not implement these
methods, and the challenge is not an inability to sample the sub-optimal space,
but rather to provide a comprehensible subset of solutions. The algorithm I
developed in this thesis to search the near-feasible space is the following:

\begin{enumerate}
    \item Obtain a set of Pareto-optimal solutions \textit{using any \ac{ga}}.
    \item Decide on a slack value (e.g., 10\% or 0.1), which represents an
    acceptable deviation from the Pareto front.
    \item Create a ``near-feasible front'' where the coordinates of each point
    are multiplied by unity plus the slack value. This is equivalent to relaxing
    the objective functions and converting them to a constraint. 
    \item Every individual is checked if all of its coordinates are
    \begin{itemize}
        \item below all of the coordinates for at least one point on the
        near-feasible front and
        \item above all of the coordinates for at least one point on the Pareto
        front.
    \end{itemize}  
    \item Lastly, the set of interior points may be sampled either randomly or
    with a farthest-first-traversal algorithm to restrict the number of analyzed
    solutions.
\end{enumerate}
\noindent
Figure \ref{fig:nd-mga} and Figure \ref{fig:3d-mga} show ``near-feasible
fronts'' and interior points with 20 percent slack for a 2-D and 3-D Pareto
front, respectively. Figure \ref{fig:nd-mga} shows clearly that only points
within the near-optimal space (gray) are considered. Illustrating this behavior
in three dimensions (and above) is considerably more difficult. The 3-D interior
points should be covered by both surfaces, obstructing their view. Figure
\ref{fig:3d-mga} shows that this is the case in three panels. First, a top view
of an opaque Pareto front (green) where no interior points can be observed.
Second, the same view with a translucent Pareto front, revealing interior points
and the near-optimal front (blue). Finally, the view from underneath the
near-optimal front once again obscures the interior points, except for two near
the edges of the sub-optimal space. The tested points are omitted for clarity.

\begin{figure}[h]
  \centering
  \resizebox{0.6\columnwidth}{!}{%% Creator: Matplotlib, PGF backend
%%
%% To include the figure in your LaTeX document, write
%%   \input{<filename>.pgf}
%%
%% Make sure the required packages are loaded in your preamble
%%   \usepackage{pgf}
%%
%% Also ensure that all the required font packages are loaded; for instance,
%% the lmodern package is sometimes necessary when using math font.
%%   \usepackage{lmodern}
%%
%% Figures using additional raster images can only be included by \input if
%% they are in the same directory as the main LaTeX file. For loading figures
%% from other directories you can use the `import` package
%%   \usepackage{import}
%%
%% and then include the figures with
%%   \import{<path to file>}{<filename>.pgf}
%%
%% Matplotlib used the following preamble
%%   \def\mathdefault#1{#1}
%%   \everymath=\expandafter{\the\everymath\displaystyle}
%%   \IfFileExists{scrextend.sty}{
%%     \usepackage[fontsize=10.000000pt]{scrextend}
%%   }{
%%     \renewcommand{\normalsize}{\fontsize{10.000000}{12.000000}\selectfont}
%%     \normalsize
%%   }
%%   
%%   \makeatletter\@ifpackageloaded{underscore}{}{\usepackage[strings]{underscore}}\makeatother
%%
\begingroup%
\makeatletter%
\begin{pgfpicture}%
\pgfpathrectangle{\pgfpointorigin}{\pgfqpoint{6.951690in}{5.397777in}}%
\pgfusepath{use as bounding box, clip}%
\begin{pgfscope}%
\pgfsetbuttcap%
\pgfsetmiterjoin%
\definecolor{currentfill}{rgb}{1.000000,1.000000,1.000000}%
\pgfsetfillcolor{currentfill}%
\pgfsetlinewidth{0.000000pt}%
\definecolor{currentstroke}{rgb}{0.000000,0.000000,0.000000}%
\pgfsetstrokecolor{currentstroke}%
\pgfsetdash{}{0pt}%
\pgfpathmoveto{\pgfqpoint{0.000000in}{0.000000in}}%
\pgfpathlineto{\pgfqpoint{6.951690in}{0.000000in}}%
\pgfpathlineto{\pgfqpoint{6.951690in}{5.397777in}}%
\pgfpathlineto{\pgfqpoint{0.000000in}{5.397777in}}%
\pgfpathlineto{\pgfqpoint{0.000000in}{0.000000in}}%
\pgfpathclose%
\pgfusepath{fill}%
\end{pgfscope}%
\begin{pgfscope}%
\pgfsetbuttcap%
\pgfsetmiterjoin%
\definecolor{currentfill}{rgb}{1.000000,1.000000,1.000000}%
\pgfsetfillcolor{currentfill}%
\pgfsetlinewidth{0.000000pt}%
\definecolor{currentstroke}{rgb}{0.000000,0.000000,0.000000}%
\pgfsetstrokecolor{currentstroke}%
\pgfsetstrokeopacity{0.000000}%
\pgfsetdash{}{0pt}%
\pgfpathmoveto{\pgfqpoint{0.626386in}{0.608332in}}%
\pgfpathlineto{\pgfqpoint{6.826386in}{0.608332in}}%
\pgfpathlineto{\pgfqpoint{6.826386in}{5.228333in}}%
\pgfpathlineto{\pgfqpoint{0.626386in}{5.228333in}}%
\pgfpathlineto{\pgfqpoint{0.626386in}{0.608332in}}%
\pgfpathclose%
\pgfusepath{fill}%
\end{pgfscope}%
\begin{pgfscope}%
\pgfpathrectangle{\pgfqpoint{0.626386in}{0.608332in}}{\pgfqpoint{6.200000in}{4.620000in}}%
\pgfusepath{clip}%
\pgfsetbuttcap%
\pgfsetroundjoin%
\definecolor{currentfill}{rgb}{0.121569,0.466667,0.705882}%
\pgfsetfillcolor{currentfill}%
\pgfsetlinewidth{1.003750pt}%
\definecolor{currentstroke}{rgb}{0.121569,0.466667,0.705882}%
\pgfsetstrokecolor{currentstroke}%
\pgfsetdash{}{0pt}%
\pgfsys@defobject{currentmarker}{\pgfqpoint{-0.012028in}{-0.012028in}}{\pgfqpoint{0.012028in}{0.012028in}}{%
\pgfpathmoveto{\pgfqpoint{0.000000in}{-0.012028in}}%
\pgfpathcurveto{\pgfqpoint{0.003190in}{-0.012028in}}{\pgfqpoint{0.006250in}{-0.010761in}}{\pgfqpoint{0.008505in}{-0.008505in}}%
\pgfpathcurveto{\pgfqpoint{0.010761in}{-0.006250in}}{\pgfqpoint{0.012028in}{-0.003190in}}{\pgfqpoint{0.012028in}{0.000000in}}%
\pgfpathcurveto{\pgfqpoint{0.012028in}{0.003190in}}{\pgfqpoint{0.010761in}{0.006250in}}{\pgfqpoint{0.008505in}{0.008505in}}%
\pgfpathcurveto{\pgfqpoint{0.006250in}{0.010761in}}{\pgfqpoint{0.003190in}{0.012028in}}{\pgfqpoint{0.000000in}{0.012028in}}%
\pgfpathcurveto{\pgfqpoint{-0.003190in}{0.012028in}}{\pgfqpoint{-0.006250in}{0.010761in}}{\pgfqpoint{-0.008505in}{0.008505in}}%
\pgfpathcurveto{\pgfqpoint{-0.010761in}{0.006250in}}{\pgfqpoint{-0.012028in}{0.003190in}}{\pgfqpoint{-0.012028in}{0.000000in}}%
\pgfpathcurveto{\pgfqpoint{-0.012028in}{-0.003190in}}{\pgfqpoint{-0.010761in}{-0.006250in}}{\pgfqpoint{-0.008505in}{-0.008505in}}%
\pgfpathcurveto{\pgfqpoint{-0.006250in}{-0.010761in}}{\pgfqpoint{-0.003190in}{-0.012028in}}{\pgfqpoint{0.000000in}{-0.012028in}}%
\pgfpathlineto{\pgfqpoint{0.000000in}{-0.012028in}}%
\pgfpathclose%
\pgfusepath{stroke,fill}%
}%
\begin{pgfscope}%
\pgfsys@transformshift{6.748714in}{2.274941in}%
\pgfsys@useobject{currentmarker}{}%
\end{pgfscope}%
\begin{pgfscope}%
\pgfsys@transformshift{6.413646in}{1.630257in}%
\pgfsys@useobject{currentmarker}{}%
\end{pgfscope}%
\begin{pgfscope}%
\pgfsys@transformshift{2.626608in}{0.849035in}%
\pgfsys@useobject{currentmarker}{}%
\end{pgfscope}%
\begin{pgfscope}%
\pgfsys@transformshift{2.141870in}{1.939487in}%
\pgfsys@useobject{currentmarker}{}%
\end{pgfscope}%
\begin{pgfscope}%
\pgfsys@transformshift{6.669603in}{1.640906in}%
\pgfsys@useobject{currentmarker}{}%
\end{pgfscope}%
\begin{pgfscope}%
\pgfsys@transformshift{3.390781in}{3.524423in}%
\pgfsys@useobject{currentmarker}{}%
\end{pgfscope}%
\begin{pgfscope}%
\pgfsys@transformshift{6.039894in}{4.905623in}%
\pgfsys@useobject{currentmarker}{}%
\end{pgfscope}%
\begin{pgfscope}%
\pgfsys@transformshift{4.856800in}{3.796453in}%
\pgfsys@useobject{currentmarker}{}%
\end{pgfscope}%
\begin{pgfscope}%
\pgfsys@transformshift{5.238397in}{1.418422in}%
\pgfsys@useobject{currentmarker}{}%
\end{pgfscope}%
\begin{pgfscope}%
\pgfsys@transformshift{1.704963in}{4.941840in}%
\pgfsys@useobject{currentmarker}{}%
\end{pgfscope}%
\begin{pgfscope}%
\pgfsys@transformshift{1.003358in}{3.926010in}%
\pgfsys@useobject{currentmarker}{}%
\end{pgfscope}%
\begin{pgfscope}%
\pgfsys@transformshift{4.833963in}{3.530664in}%
\pgfsys@useobject{currentmarker}{}%
\end{pgfscope}%
\begin{pgfscope}%
\pgfsys@transformshift{1.003350in}{5.525871in}%
\pgfsys@useobject{currentmarker}{}%
\end{pgfscope}%
\begin{pgfscope}%
\pgfsys@transformshift{3.377903in}{3.104026in}%
\pgfsys@useobject{currentmarker}{}%
\end{pgfscope}%
\begin{pgfscope}%
\pgfsys@transformshift{0.646020in}{1.573541in}%
\pgfsys@useobject{currentmarker}{}%
\end{pgfscope}%
\begin{pgfscope}%
\pgfsys@transformshift{6.007732in}{2.520309in}%
\pgfsys@useobject{currentmarker}{}%
\end{pgfscope}%
\begin{pgfscope}%
\pgfsys@transformshift{5.238781in}{5.222709in}%
\pgfsys@useobject{currentmarker}{}%
\end{pgfscope}%
\begin{pgfscope}%
\pgfsys@transformshift{1.588421in}{5.604699in}%
\pgfsys@useobject{currentmarker}{}%
\end{pgfscope}%
\begin{pgfscope}%
\pgfsys@transformshift{1.769311in}{5.321012in}%
\pgfsys@useobject{currentmarker}{}%
\end{pgfscope}%
\begin{pgfscope}%
\pgfsys@transformshift{1.171002in}{2.753761in}%
\pgfsys@useobject{currentmarker}{}%
\end{pgfscope}%
\begin{pgfscope}%
\pgfsys@transformshift{5.822807in}{1.735579in}%
\pgfsys@useobject{currentmarker}{}%
\end{pgfscope}%
\begin{pgfscope}%
\pgfsys@transformshift{6.074184in}{5.518512in}%
\pgfsys@useobject{currentmarker}{}%
\end{pgfscope}%
\begin{pgfscope}%
\pgfsys@transformshift{5.902569in}{2.648563in}%
\pgfsys@useobject{currentmarker}{}%
\end{pgfscope}%
\begin{pgfscope}%
\pgfsys@transformshift{2.948874in}{2.832420in}%
\pgfsys@useobject{currentmarker}{}%
\end{pgfscope}%
\begin{pgfscope}%
\pgfsys@transformshift{1.204215in}{1.439799in}%
\pgfsys@useobject{currentmarker}{}%
\end{pgfscope}%
\begin{pgfscope}%
\pgfsys@transformshift{3.989788in}{4.195533in}%
\pgfsys@useobject{currentmarker}{}%
\end{pgfscope}%
\begin{pgfscope}%
\pgfsys@transformshift{3.440330in}{1.868584in}%
\pgfsys@useobject{currentmarker}{}%
\end{pgfscope}%
\begin{pgfscope}%
\pgfsys@transformshift{6.564798in}{4.292381in}%
\pgfsys@useobject{currentmarker}{}%
\end{pgfscope}%
\begin{pgfscope}%
\pgfsys@transformshift{3.585366in}{2.694274in}%
\pgfsys@useobject{currentmarker}{}%
\end{pgfscope}%
\begin{pgfscope}%
\pgfsys@transformshift{5.304717in}{4.860155in}%
\pgfsys@useobject{currentmarker}{}%
\end{pgfscope}%
\begin{pgfscope}%
\pgfsys@transformshift{2.466166in}{4.195984in}%
\pgfsys@useobject{currentmarker}{}%
\end{pgfscope}%
\begin{pgfscope}%
\pgfsys@transformshift{5.696446in}{5.032539in}%
\pgfsys@useobject{currentmarker}{}%
\end{pgfscope}%
\begin{pgfscope}%
\pgfsys@transformshift{0.782050in}{2.994974in}%
\pgfsys@useobject{currentmarker}{}%
\end{pgfscope}%
\begin{pgfscope}%
\pgfsys@transformshift{4.467561in}{1.731753in}%
\pgfsys@useobject{currentmarker}{}%
\end{pgfscope}%
\begin{pgfscope}%
\pgfsys@transformshift{3.460521in}{3.096552in}%
\pgfsys@useobject{currentmarker}{}%
\end{pgfscope}%
\begin{pgfscope}%
\pgfsys@transformshift{6.823525in}{0.239116in}%
\pgfsys@useobject{currentmarker}{}%
\end{pgfscope}%
\begin{pgfscope}%
\pgfsys@transformshift{1.888138in}{1.604054in}%
\pgfsys@useobject{currentmarker}{}%
\end{pgfscope}%
\begin{pgfscope}%
\pgfsys@transformshift{2.762305in}{0.494936in}%
\pgfsys@useobject{currentmarker}{}%
\end{pgfscope}%
\begin{pgfscope}%
\pgfsys@transformshift{2.672978in}{1.285705in}%
\pgfsys@useobject{currentmarker}{}%
\end{pgfscope}%
\begin{pgfscope}%
\pgfsys@transformshift{5.824222in}{3.672674in}%
\pgfsys@useobject{currentmarker}{}%
\end{pgfscope}%
\begin{pgfscope}%
\pgfsys@transformshift{1.161961in}{1.404314in}%
\pgfsys@useobject{currentmarker}{}%
\end{pgfscope}%
\begin{pgfscope}%
\pgfsys@transformshift{4.896258in}{0.937813in}%
\pgfsys@useobject{currentmarker}{}%
\end{pgfscope}%
\begin{pgfscope}%
\pgfsys@transformshift{6.622180in}{0.949313in}%
\pgfsys@useobject{currentmarker}{}%
\end{pgfscope}%
\begin{pgfscope}%
\pgfsys@transformshift{5.626688in}{2.917418in}%
\pgfsys@useobject{currentmarker}{}%
\end{pgfscope}%
\begin{pgfscope}%
\pgfsys@transformshift{2.536411in}{1.480725in}%
\pgfsys@useobject{currentmarker}{}%
\end{pgfscope}%
\begin{pgfscope}%
\pgfsys@transformshift{6.224103in}{2.260474in}%
\pgfsys@useobject{currentmarker}{}%
\end{pgfscope}%
\begin{pgfscope}%
\pgfsys@transformshift{1.104948in}{0.390510in}%
\pgfsys@useobject{currentmarker}{}%
\end{pgfscope}%
\begin{pgfscope}%
\pgfsys@transformshift{3.519955in}{4.960386in}%
\pgfsys@useobject{currentmarker}{}%
\end{pgfscope}%
\begin{pgfscope}%
\pgfsys@transformshift{2.824140in}{4.686724in}%
\pgfsys@useobject{currentmarker}{}%
\end{pgfscope}%
\begin{pgfscope}%
\pgfsys@transformshift{6.563792in}{0.571309in}%
\pgfsys@useobject{currentmarker}{}%
\end{pgfscope}%
\begin{pgfscope}%
\pgfsys@transformshift{2.493362in}{3.574964in}%
\pgfsys@useobject{currentmarker}{}%
\end{pgfscope}%
\begin{pgfscope}%
\pgfsys@transformshift{2.517917in}{1.157796in}%
\pgfsys@useobject{currentmarker}{}%
\end{pgfscope}%
\begin{pgfscope}%
\pgfsys@transformshift{3.732402in}{4.344262in}%
\pgfsys@useobject{currentmarker}{}%
\end{pgfscope}%
\begin{pgfscope}%
\pgfsys@transformshift{4.934404in}{1.556609in}%
\pgfsys@useobject{currentmarker}{}%
\end{pgfscope}%
\begin{pgfscope}%
\pgfsys@transformshift{4.525612in}{1.923829in}%
\pgfsys@useobject{currentmarker}{}%
\end{pgfscope}%
\begin{pgfscope}%
\pgfsys@transformshift{2.364084in}{5.265262in}%
\pgfsys@useobject{currentmarker}{}%
\end{pgfscope}%
\begin{pgfscope}%
\pgfsys@transformshift{2.378996in}{5.368462in}%
\pgfsys@useobject{currentmarker}{}%
\end{pgfscope}%
\begin{pgfscope}%
\pgfsys@transformshift{1.892708in}{3.282919in}%
\pgfsys@useobject{currentmarker}{}%
\end{pgfscope}%
\begin{pgfscope}%
\pgfsys@transformshift{3.010271in}{4.602870in}%
\pgfsys@useobject{currentmarker}{}%
\end{pgfscope}%
\begin{pgfscope}%
\pgfsys@transformshift{1.425941in}{5.382087in}%
\pgfsys@useobject{currentmarker}{}%
\end{pgfscope}%
\begin{pgfscope}%
\pgfsys@transformshift{5.194955in}{1.161596in}%
\pgfsys@useobject{currentmarker}{}%
\end{pgfscope}%
\begin{pgfscope}%
\pgfsys@transformshift{4.968865in}{0.711625in}%
\pgfsys@useobject{currentmarker}{}%
\end{pgfscope}%
\begin{pgfscope}%
\pgfsys@transformshift{4.114124in}{0.764184in}%
\pgfsys@useobject{currentmarker}{}%
\end{pgfscope}%
\begin{pgfscope}%
\pgfsys@transformshift{0.900272in}{2.342915in}%
\pgfsys@useobject{currentmarker}{}%
\end{pgfscope}%
\begin{pgfscope}%
\pgfsys@transformshift{4.958177in}{4.161509in}%
\pgfsys@useobject{currentmarker}{}%
\end{pgfscope}%
\begin{pgfscope}%
\pgfsys@transformshift{4.820132in}{3.424780in}%
\pgfsys@useobject{currentmarker}{}%
\end{pgfscope}%
\begin{pgfscope}%
\pgfsys@transformshift{6.136298in}{3.131568in}%
\pgfsys@useobject{currentmarker}{}%
\end{pgfscope}%
\begin{pgfscope}%
\pgfsys@transformshift{1.239574in}{0.975273in}%
\pgfsys@useobject{currentmarker}{}%
\end{pgfscope}%
\begin{pgfscope}%
\pgfsys@transformshift{6.658326in}{4.548452in}%
\pgfsys@useobject{currentmarker}{}%
\end{pgfscope}%
\begin{pgfscope}%
\pgfsys@transformshift{0.668721in}{5.352014in}%
\pgfsys@useobject{currentmarker}{}%
\end{pgfscope}%
\begin{pgfscope}%
\pgfsys@transformshift{4.734825in}{5.160915in}%
\pgfsys@useobject{currentmarker}{}%
\end{pgfscope}%
\begin{pgfscope}%
\pgfsys@transformshift{1.582845in}{3.408881in}%
\pgfsys@useobject{currentmarker}{}%
\end{pgfscope}%
\begin{pgfscope}%
\pgfsys@transformshift{3.784181in}{1.152152in}%
\pgfsys@useobject{currentmarker}{}%
\end{pgfscope}%
\begin{pgfscope}%
\pgfsys@transformshift{1.020356in}{3.821881in}%
\pgfsys@useobject{currentmarker}{}%
\end{pgfscope}%
\begin{pgfscope}%
\pgfsys@transformshift{0.826380in}{1.502212in}%
\pgfsys@useobject{currentmarker}{}%
\end{pgfscope}%
\begin{pgfscope}%
\pgfsys@transformshift{3.857296in}{3.483551in}%
\pgfsys@useobject{currentmarker}{}%
\end{pgfscope}%
\begin{pgfscope}%
\pgfsys@transformshift{2.400085in}{4.793404in}%
\pgfsys@useobject{currentmarker}{}%
\end{pgfscope}%
\begin{pgfscope}%
\pgfsys@transformshift{6.234743in}{4.140911in}%
\pgfsys@useobject{currentmarker}{}%
\end{pgfscope}%
\begin{pgfscope}%
\pgfsys@transformshift{6.434302in}{0.527551in}%
\pgfsys@useobject{currentmarker}{}%
\end{pgfscope}%
\begin{pgfscope}%
\pgfsys@transformshift{0.940729in}{0.880528in}%
\pgfsys@useobject{currentmarker}{}%
\end{pgfscope}%
\begin{pgfscope}%
\pgfsys@transformshift{1.391633in}{3.096166in}%
\pgfsys@useobject{currentmarker}{}%
\end{pgfscope}%
\begin{pgfscope}%
\pgfsys@transformshift{6.350284in}{1.199255in}%
\pgfsys@useobject{currentmarker}{}%
\end{pgfscope}%
\begin{pgfscope}%
\pgfsys@transformshift{6.841479in}{4.993467in}%
\pgfsys@useobject{currentmarker}{}%
\end{pgfscope}%
\begin{pgfscope}%
\pgfsys@transformshift{2.650795in}{1.247392in}%
\pgfsys@useobject{currentmarker}{}%
\end{pgfscope}%
\begin{pgfscope}%
\pgfsys@transformshift{3.740935in}{1.917027in}%
\pgfsys@useobject{currentmarker}{}%
\end{pgfscope}%
\begin{pgfscope}%
\pgfsys@transformshift{0.816347in}{0.710485in}%
\pgfsys@useobject{currentmarker}{}%
\end{pgfscope}%
\begin{pgfscope}%
\pgfsys@transformshift{4.318591in}{3.655900in}%
\pgfsys@useobject{currentmarker}{}%
\end{pgfscope}%
\begin{pgfscope}%
\pgfsys@transformshift{6.683803in}{0.287542in}%
\pgfsys@useobject{currentmarker}{}%
\end{pgfscope}%
\begin{pgfscope}%
\pgfsys@transformshift{5.222113in}{3.649505in}%
\pgfsys@useobject{currentmarker}{}%
\end{pgfscope}%
\begin{pgfscope}%
\pgfsys@transformshift{3.771139in}{0.864515in}%
\pgfsys@useobject{currentmarker}{}%
\end{pgfscope}%
\begin{pgfscope}%
\pgfsys@transformshift{6.004284in}{4.027861in}%
\pgfsys@useobject{currentmarker}{}%
\end{pgfscope}%
\begin{pgfscope}%
\pgfsys@transformshift{4.190676in}{5.543986in}%
\pgfsys@useobject{currentmarker}{}%
\end{pgfscope}%
\begin{pgfscope}%
\pgfsys@transformshift{1.192751in}{4.162065in}%
\pgfsys@useobject{currentmarker}{}%
\end{pgfscope}%
\begin{pgfscope}%
\pgfsys@transformshift{3.689329in}{1.732570in}%
\pgfsys@useobject{currentmarker}{}%
\end{pgfscope}%
\begin{pgfscope}%
\pgfsys@transformshift{6.175975in}{4.600810in}%
\pgfsys@useobject{currentmarker}{}%
\end{pgfscope}%
\begin{pgfscope}%
\pgfsys@transformshift{3.381021in}{3.697299in}%
\pgfsys@useobject{currentmarker}{}%
\end{pgfscope}%
\begin{pgfscope}%
\pgfsys@transformshift{4.581980in}{0.252491in}%
\pgfsys@useobject{currentmarker}{}%
\end{pgfscope}%
\begin{pgfscope}%
\pgfsys@transformshift{2.015825in}{1.386537in}%
\pgfsys@useobject{currentmarker}{}%
\end{pgfscope}%
\begin{pgfscope}%
\pgfsys@transformshift{2.007116in}{4.238541in}%
\pgfsys@useobject{currentmarker}{}%
\end{pgfscope}%
\begin{pgfscope}%
\pgfsys@transformshift{6.528407in}{1.656070in}%
\pgfsys@useobject{currentmarker}{}%
\end{pgfscope}%
\begin{pgfscope}%
\pgfsys@transformshift{5.310581in}{0.217860in}%
\pgfsys@useobject{currentmarker}{}%
\end{pgfscope}%
\begin{pgfscope}%
\pgfsys@transformshift{5.699741in}{4.595809in}%
\pgfsys@useobject{currentmarker}{}%
\end{pgfscope}%
\begin{pgfscope}%
\pgfsys@transformshift{0.968218in}{0.744828in}%
\pgfsys@useobject{currentmarker}{}%
\end{pgfscope}%
\begin{pgfscope}%
\pgfsys@transformshift{4.094222in}{4.356993in}%
\pgfsys@useobject{currentmarker}{}%
\end{pgfscope}%
\begin{pgfscope}%
\pgfsys@transformshift{5.208142in}{3.919384in}%
\pgfsys@useobject{currentmarker}{}%
\end{pgfscope}%
\begin{pgfscope}%
\pgfsys@transformshift{5.406600in}{3.467528in}%
\pgfsys@useobject{currentmarker}{}%
\end{pgfscope}%
\begin{pgfscope}%
\pgfsys@transformshift{2.102448in}{0.359253in}%
\pgfsys@useobject{currentmarker}{}%
\end{pgfscope}%
\begin{pgfscope}%
\pgfsys@transformshift{6.341119in}{2.459282in}%
\pgfsys@useobject{currentmarker}{}%
\end{pgfscope}%
\begin{pgfscope}%
\pgfsys@transformshift{5.161335in}{1.653424in}%
\pgfsys@useobject{currentmarker}{}%
\end{pgfscope}%
\begin{pgfscope}%
\pgfsys@transformshift{6.771717in}{1.175442in}%
\pgfsys@useobject{currentmarker}{}%
\end{pgfscope}%
\begin{pgfscope}%
\pgfsys@transformshift{0.920489in}{2.894691in}%
\pgfsys@useobject{currentmarker}{}%
\end{pgfscope}%
\begin{pgfscope}%
\pgfsys@transformshift{2.949004in}{3.773060in}%
\pgfsys@useobject{currentmarker}{}%
\end{pgfscope}%
\begin{pgfscope}%
\pgfsys@transformshift{0.626675in}{0.984625in}%
\pgfsys@useobject{currentmarker}{}%
\end{pgfscope}%
\begin{pgfscope}%
\pgfsys@transformshift{6.010775in}{3.921228in}%
\pgfsys@useobject{currentmarker}{}%
\end{pgfscope}%
\begin{pgfscope}%
\pgfsys@transformshift{2.707698in}{0.341599in}%
\pgfsys@useobject{currentmarker}{}%
\end{pgfscope}%
\begin{pgfscope}%
\pgfsys@transformshift{0.890930in}{2.326007in}%
\pgfsys@useobject{currentmarker}{}%
\end{pgfscope}%
\begin{pgfscope}%
\pgfsys@transformshift{6.537541in}{5.204043in}%
\pgfsys@useobject{currentmarker}{}%
\end{pgfscope}%
\begin{pgfscope}%
\pgfsys@transformshift{2.599215in}{2.231556in}%
\pgfsys@useobject{currentmarker}{}%
\end{pgfscope}%
\begin{pgfscope}%
\pgfsys@transformshift{2.298475in}{0.978419in}%
\pgfsys@useobject{currentmarker}{}%
\end{pgfscope}%
\begin{pgfscope}%
\pgfsys@transformshift{4.334250in}{2.725847in}%
\pgfsys@useobject{currentmarker}{}%
\end{pgfscope}%
\begin{pgfscope}%
\pgfsys@transformshift{5.862556in}{2.901862in}%
\pgfsys@useobject{currentmarker}{}%
\end{pgfscope}%
\begin{pgfscope}%
\pgfsys@transformshift{0.737427in}{2.588836in}%
\pgfsys@useobject{currentmarker}{}%
\end{pgfscope}%
\begin{pgfscope}%
\pgfsys@transformshift{1.466590in}{2.014927in}%
\pgfsys@useobject{currentmarker}{}%
\end{pgfscope}%
\begin{pgfscope}%
\pgfsys@transformshift{3.768743in}{3.162637in}%
\pgfsys@useobject{currentmarker}{}%
\end{pgfscope}%
\begin{pgfscope}%
\pgfsys@transformshift{4.981965in}{5.452028in}%
\pgfsys@useobject{currentmarker}{}%
\end{pgfscope}%
\begin{pgfscope}%
\pgfsys@transformshift{2.976413in}{4.315850in}%
\pgfsys@useobject{currentmarker}{}%
\end{pgfscope}%
\begin{pgfscope}%
\pgfsys@transformshift{3.939099in}{4.910376in}%
\pgfsys@useobject{currentmarker}{}%
\end{pgfscope}%
\begin{pgfscope}%
\pgfsys@transformshift{4.142002in}{2.285504in}%
\pgfsys@useobject{currentmarker}{}%
\end{pgfscope}%
\begin{pgfscope}%
\pgfsys@transformshift{3.760338in}{3.088483in}%
\pgfsys@useobject{currentmarker}{}%
\end{pgfscope}%
\begin{pgfscope}%
\pgfsys@transformshift{6.804641in}{0.539977in}%
\pgfsys@useobject{currentmarker}{}%
\end{pgfscope}%
\begin{pgfscope}%
\pgfsys@transformshift{0.907503in}{1.263762in}%
\pgfsys@useobject{currentmarker}{}%
\end{pgfscope}%
\begin{pgfscope}%
\pgfsys@transformshift{0.650888in}{0.782229in}%
\pgfsys@useobject{currentmarker}{}%
\end{pgfscope}%
\begin{pgfscope}%
\pgfsys@transformshift{3.098873in}{1.320918in}%
\pgfsys@useobject{currentmarker}{}%
\end{pgfscope}%
\begin{pgfscope}%
\pgfsys@transformshift{3.647529in}{4.805754in}%
\pgfsys@useobject{currentmarker}{}%
\end{pgfscope}%
\begin{pgfscope}%
\pgfsys@transformshift{4.628549in}{4.381591in}%
\pgfsys@useobject{currentmarker}{}%
\end{pgfscope}%
\begin{pgfscope}%
\pgfsys@transformshift{5.832577in}{0.897172in}%
\pgfsys@useobject{currentmarker}{}%
\end{pgfscope}%
\begin{pgfscope}%
\pgfsys@transformshift{0.903870in}{2.644883in}%
\pgfsys@useobject{currentmarker}{}%
\end{pgfscope}%
\begin{pgfscope}%
\pgfsys@transformshift{3.550935in}{5.430657in}%
\pgfsys@useobject{currentmarker}{}%
\end{pgfscope}%
\begin{pgfscope}%
\pgfsys@transformshift{2.242411in}{2.622652in}%
\pgfsys@useobject{currentmarker}{}%
\end{pgfscope}%
\begin{pgfscope}%
\pgfsys@transformshift{4.130464in}{4.830717in}%
\pgfsys@useobject{currentmarker}{}%
\end{pgfscope}%
\begin{pgfscope}%
\pgfsys@transformshift{1.411996in}{5.257843in}%
\pgfsys@useobject{currentmarker}{}%
\end{pgfscope}%
\begin{pgfscope}%
\pgfsys@transformshift{4.260224in}{2.722876in}%
\pgfsys@useobject{currentmarker}{}%
\end{pgfscope}%
\begin{pgfscope}%
\pgfsys@transformshift{3.405747in}{0.478012in}%
\pgfsys@useobject{currentmarker}{}%
\end{pgfscope}%
\begin{pgfscope}%
\pgfsys@transformshift{2.350355in}{3.579143in}%
\pgfsys@useobject{currentmarker}{}%
\end{pgfscope}%
\begin{pgfscope}%
\pgfsys@transformshift{6.387106in}{5.215306in}%
\pgfsys@useobject{currentmarker}{}%
\end{pgfscope}%
\begin{pgfscope}%
\pgfsys@transformshift{2.231490in}{3.123346in}%
\pgfsys@useobject{currentmarker}{}%
\end{pgfscope}%
\begin{pgfscope}%
\pgfsys@transformshift{2.544036in}{2.207787in}%
\pgfsys@useobject{currentmarker}{}%
\end{pgfscope}%
\begin{pgfscope}%
\pgfsys@transformshift{2.827197in}{2.561162in}%
\pgfsys@useobject{currentmarker}{}%
\end{pgfscope}%
\begin{pgfscope}%
\pgfsys@transformshift{0.646834in}{2.681178in}%
\pgfsys@useobject{currentmarker}{}%
\end{pgfscope}%
\begin{pgfscope}%
\pgfsys@transformshift{2.594401in}{1.640905in}%
\pgfsys@useobject{currentmarker}{}%
\end{pgfscope}%
\begin{pgfscope}%
\pgfsys@transformshift{4.214490in}{2.877842in}%
\pgfsys@useobject{currentmarker}{}%
\end{pgfscope}%
\begin{pgfscope}%
\pgfsys@transformshift{5.137671in}{2.729692in}%
\pgfsys@useobject{currentmarker}{}%
\end{pgfscope}%
\begin{pgfscope}%
\pgfsys@transformshift{5.961491in}{4.495294in}%
\pgfsys@useobject{currentmarker}{}%
\end{pgfscope}%
\begin{pgfscope}%
\pgfsys@transformshift{4.017877in}{5.354591in}%
\pgfsys@useobject{currentmarker}{}%
\end{pgfscope}%
\begin{pgfscope}%
\pgfsys@transformshift{4.678181in}{2.499123in}%
\pgfsys@useobject{currentmarker}{}%
\end{pgfscope}%
\begin{pgfscope}%
\pgfsys@transformshift{2.045123in}{1.090396in}%
\pgfsys@useobject{currentmarker}{}%
\end{pgfscope}%
\begin{pgfscope}%
\pgfsys@transformshift{2.850108in}{1.542976in}%
\pgfsys@useobject{currentmarker}{}%
\end{pgfscope}%
\begin{pgfscope}%
\pgfsys@transformshift{3.559126in}{4.430788in}%
\pgfsys@useobject{currentmarker}{}%
\end{pgfscope}%
\begin{pgfscope}%
\pgfsys@transformshift{0.868814in}{4.461062in}%
\pgfsys@useobject{currentmarker}{}%
\end{pgfscope}%
\begin{pgfscope}%
\pgfsys@transformshift{1.680316in}{3.410044in}%
\pgfsys@useobject{currentmarker}{}%
\end{pgfscope}%
\begin{pgfscope}%
\pgfsys@transformshift{0.654024in}{4.739106in}%
\pgfsys@useobject{currentmarker}{}%
\end{pgfscope}%
\begin{pgfscope}%
\pgfsys@transformshift{5.124198in}{1.177305in}%
\pgfsys@useobject{currentmarker}{}%
\end{pgfscope}%
\begin{pgfscope}%
\pgfsys@transformshift{6.386361in}{2.124473in}%
\pgfsys@useobject{currentmarker}{}%
\end{pgfscope}%
\begin{pgfscope}%
\pgfsys@transformshift{4.465022in}{1.382993in}%
\pgfsys@useobject{currentmarker}{}%
\end{pgfscope}%
\begin{pgfscope}%
\pgfsys@transformshift{3.751574in}{3.923941in}%
\pgfsys@useobject{currentmarker}{}%
\end{pgfscope}%
\begin{pgfscope}%
\pgfsys@transformshift{1.865799in}{1.980264in}%
\pgfsys@useobject{currentmarker}{}%
\end{pgfscope}%
\begin{pgfscope}%
\pgfsys@transformshift{3.344360in}{3.226466in}%
\pgfsys@useobject{currentmarker}{}%
\end{pgfscope}%
\begin{pgfscope}%
\pgfsys@transformshift{2.451884in}{2.779128in}%
\pgfsys@useobject{currentmarker}{}%
\end{pgfscope}%
\begin{pgfscope}%
\pgfsys@transformshift{1.279225in}{2.588906in}%
\pgfsys@useobject{currentmarker}{}%
\end{pgfscope}%
\begin{pgfscope}%
\pgfsys@transformshift{0.759148in}{1.180563in}%
\pgfsys@useobject{currentmarker}{}%
\end{pgfscope}%
\begin{pgfscope}%
\pgfsys@transformshift{6.349859in}{1.331878in}%
\pgfsys@useobject{currentmarker}{}%
\end{pgfscope}%
\begin{pgfscope}%
\pgfsys@transformshift{2.908829in}{1.191898in}%
\pgfsys@useobject{currentmarker}{}%
\end{pgfscope}%
\begin{pgfscope}%
\pgfsys@transformshift{4.294240in}{3.406530in}%
\pgfsys@useobject{currentmarker}{}%
\end{pgfscope}%
\begin{pgfscope}%
\pgfsys@transformshift{6.223995in}{4.193718in}%
\pgfsys@useobject{currentmarker}{}%
\end{pgfscope}%
\begin{pgfscope}%
\pgfsys@transformshift{6.541431in}{1.266260in}%
\pgfsys@useobject{currentmarker}{}%
\end{pgfscope}%
\begin{pgfscope}%
\pgfsys@transformshift{2.137565in}{4.917569in}%
\pgfsys@useobject{currentmarker}{}%
\end{pgfscope}%
\begin{pgfscope}%
\pgfsys@transformshift{6.080457in}{0.934593in}%
\pgfsys@useobject{currentmarker}{}%
\end{pgfscope}%
\begin{pgfscope}%
\pgfsys@transformshift{2.901099in}{0.280599in}%
\pgfsys@useobject{currentmarker}{}%
\end{pgfscope}%
\begin{pgfscope}%
\pgfsys@transformshift{2.090748in}{2.087960in}%
\pgfsys@useobject{currentmarker}{}%
\end{pgfscope}%
\begin{pgfscope}%
\pgfsys@transformshift{2.560285in}{0.866113in}%
\pgfsys@useobject{currentmarker}{}%
\end{pgfscope}%
\begin{pgfscope}%
\pgfsys@transformshift{6.601782in}{2.153191in}%
\pgfsys@useobject{currentmarker}{}%
\end{pgfscope}%
\begin{pgfscope}%
\pgfsys@transformshift{3.588372in}{3.986237in}%
\pgfsys@useobject{currentmarker}{}%
\end{pgfscope}%
\begin{pgfscope}%
\pgfsys@transformshift{5.736898in}{5.045594in}%
\pgfsys@useobject{currentmarker}{}%
\end{pgfscope}%
\begin{pgfscope}%
\pgfsys@transformshift{4.309654in}{5.016243in}%
\pgfsys@useobject{currentmarker}{}%
\end{pgfscope}%
\begin{pgfscope}%
\pgfsys@transformshift{4.099087in}{4.534596in}%
\pgfsys@useobject{currentmarker}{}%
\end{pgfscope}%
\begin{pgfscope}%
\pgfsys@transformshift{2.145398in}{4.283886in}%
\pgfsys@useobject{currentmarker}{}%
\end{pgfscope}%
\begin{pgfscope}%
\pgfsys@transformshift{2.250270in}{3.345020in}%
\pgfsys@useobject{currentmarker}{}%
\end{pgfscope}%
\begin{pgfscope}%
\pgfsys@transformshift{6.491615in}{3.548322in}%
\pgfsys@useobject{currentmarker}{}%
\end{pgfscope}%
\begin{pgfscope}%
\pgfsys@transformshift{1.068840in}{5.064836in}%
\pgfsys@useobject{currentmarker}{}%
\end{pgfscope}%
\begin{pgfscope}%
\pgfsys@transformshift{3.883485in}{0.664182in}%
\pgfsys@useobject{currentmarker}{}%
\end{pgfscope}%
\begin{pgfscope}%
\pgfsys@transformshift{2.971665in}{4.192784in}%
\pgfsys@useobject{currentmarker}{}%
\end{pgfscope}%
\begin{pgfscope}%
\pgfsys@transformshift{1.209223in}{3.839503in}%
\pgfsys@useobject{currentmarker}{}%
\end{pgfscope}%
\begin{pgfscope}%
\pgfsys@transformshift{6.871656in}{1.235874in}%
\pgfsys@useobject{currentmarker}{}%
\end{pgfscope}%
\begin{pgfscope}%
\pgfsys@transformshift{4.623291in}{5.231267in}%
\pgfsys@useobject{currentmarker}{}%
\end{pgfscope}%
\begin{pgfscope}%
\pgfsys@transformshift{5.719362in}{1.486985in}%
\pgfsys@useobject{currentmarker}{}%
\end{pgfscope}%
\begin{pgfscope}%
\pgfsys@transformshift{4.762970in}{4.961946in}%
\pgfsys@useobject{currentmarker}{}%
\end{pgfscope}%
\begin{pgfscope}%
\pgfsys@transformshift{4.886365in}{1.147861in}%
\pgfsys@useobject{currentmarker}{}%
\end{pgfscope}%
\begin{pgfscope}%
\pgfsys@transformshift{5.786494in}{4.789347in}%
\pgfsys@useobject{currentmarker}{}%
\end{pgfscope}%
\begin{pgfscope}%
\pgfsys@transformshift{6.112708in}{0.869915in}%
\pgfsys@useobject{currentmarker}{}%
\end{pgfscope}%
\begin{pgfscope}%
\pgfsys@transformshift{3.058098in}{3.065259in}%
\pgfsys@useobject{currentmarker}{}%
\end{pgfscope}%
\begin{pgfscope}%
\pgfsys@transformshift{1.419679in}{2.355247in}%
\pgfsys@useobject{currentmarker}{}%
\end{pgfscope}%
\begin{pgfscope}%
\pgfsys@transformshift{6.365833in}{4.315201in}%
\pgfsys@useobject{currentmarker}{}%
\end{pgfscope}%
\begin{pgfscope}%
\pgfsys@transformshift{1.021013in}{4.198028in}%
\pgfsys@useobject{currentmarker}{}%
\end{pgfscope}%
\begin{pgfscope}%
\pgfsys@transformshift{3.973729in}{5.516341in}%
\pgfsys@useobject{currentmarker}{}%
\end{pgfscope}%
\begin{pgfscope}%
\pgfsys@transformshift{4.944733in}{5.586916in}%
\pgfsys@useobject{currentmarker}{}%
\end{pgfscope}%
\begin{pgfscope}%
\pgfsys@transformshift{3.107162in}{4.467529in}%
\pgfsys@useobject{currentmarker}{}%
\end{pgfscope}%
\begin{pgfscope}%
\pgfsys@transformshift{5.668124in}{0.924856in}%
\pgfsys@useobject{currentmarker}{}%
\end{pgfscope}%
\begin{pgfscope}%
\pgfsys@transformshift{1.849565in}{3.214984in}%
\pgfsys@useobject{currentmarker}{}%
\end{pgfscope}%
\begin{pgfscope}%
\pgfsys@transformshift{5.738722in}{1.604631in}%
\pgfsys@useobject{currentmarker}{}%
\end{pgfscope}%
\begin{pgfscope}%
\pgfsys@transformshift{0.750418in}{1.855157in}%
\pgfsys@useobject{currentmarker}{}%
\end{pgfscope}%
\begin{pgfscope}%
\pgfsys@transformshift{2.910595in}{3.519489in}%
\pgfsys@useobject{currentmarker}{}%
\end{pgfscope}%
\begin{pgfscope}%
\pgfsys@transformshift{1.803510in}{1.618952in}%
\pgfsys@useobject{currentmarker}{}%
\end{pgfscope}%
\begin{pgfscope}%
\pgfsys@transformshift{4.522746in}{0.923843in}%
\pgfsys@useobject{currentmarker}{}%
\end{pgfscope}%
\begin{pgfscope}%
\pgfsys@transformshift{3.232716in}{2.542740in}%
\pgfsys@useobject{currentmarker}{}%
\end{pgfscope}%
\begin{pgfscope}%
\pgfsys@transformshift{1.440932in}{1.007042in}%
\pgfsys@useobject{currentmarker}{}%
\end{pgfscope}%
\begin{pgfscope}%
\pgfsys@transformshift{3.097251in}{4.226800in}%
\pgfsys@useobject{currentmarker}{}%
\end{pgfscope}%
\begin{pgfscope}%
\pgfsys@transformshift{4.639182in}{4.513124in}%
\pgfsys@useobject{currentmarker}{}%
\end{pgfscope}%
\begin{pgfscope}%
\pgfsys@transformshift{4.729855in}{0.475843in}%
\pgfsys@useobject{currentmarker}{}%
\end{pgfscope}%
\begin{pgfscope}%
\pgfsys@transformshift{5.652688in}{1.824843in}%
\pgfsys@useobject{currentmarker}{}%
\end{pgfscope}%
\begin{pgfscope}%
\pgfsys@transformshift{3.226216in}{1.426385in}%
\pgfsys@useobject{currentmarker}{}%
\end{pgfscope}%
\begin{pgfscope}%
\pgfsys@transformshift{5.167749in}{4.568793in}%
\pgfsys@useobject{currentmarker}{}%
\end{pgfscope}%
\begin{pgfscope}%
\pgfsys@transformshift{6.010328in}{1.606239in}%
\pgfsys@useobject{currentmarker}{}%
\end{pgfscope}%
\begin{pgfscope}%
\pgfsys@transformshift{4.470339in}{1.975688in}%
\pgfsys@useobject{currentmarker}{}%
\end{pgfscope}%
\begin{pgfscope}%
\pgfsys@transformshift{5.772001in}{0.915793in}%
\pgfsys@useobject{currentmarker}{}%
\end{pgfscope}%
\begin{pgfscope}%
\pgfsys@transformshift{5.528211in}{3.267157in}%
\pgfsys@useobject{currentmarker}{}%
\end{pgfscope}%
\begin{pgfscope}%
\pgfsys@transformshift{1.920141in}{0.399767in}%
\pgfsys@useobject{currentmarker}{}%
\end{pgfscope}%
\begin{pgfscope}%
\pgfsys@transformshift{1.233618in}{4.969010in}%
\pgfsys@useobject{currentmarker}{}%
\end{pgfscope}%
\begin{pgfscope}%
\pgfsys@transformshift{5.169037in}{4.325928in}%
\pgfsys@useobject{currentmarker}{}%
\end{pgfscope}%
\begin{pgfscope}%
\pgfsys@transformshift{4.070491in}{2.299736in}%
\pgfsys@useobject{currentmarker}{}%
\end{pgfscope}%
\begin{pgfscope}%
\pgfsys@transformshift{3.110229in}{0.525900in}%
\pgfsys@useobject{currentmarker}{}%
\end{pgfscope}%
\begin{pgfscope}%
\pgfsys@transformshift{4.423411in}{3.893190in}%
\pgfsys@useobject{currentmarker}{}%
\end{pgfscope}%
\begin{pgfscope}%
\pgfsys@transformshift{1.296334in}{4.701201in}%
\pgfsys@useobject{currentmarker}{}%
\end{pgfscope}%
\begin{pgfscope}%
\pgfsys@transformshift{4.933271in}{5.359361in}%
\pgfsys@useobject{currentmarker}{}%
\end{pgfscope}%
\begin{pgfscope}%
\pgfsys@transformshift{6.616783in}{4.755727in}%
\pgfsys@useobject{currentmarker}{}%
\end{pgfscope}%
\begin{pgfscope}%
\pgfsys@transformshift{3.774014in}{2.483044in}%
\pgfsys@useobject{currentmarker}{}%
\end{pgfscope}%
\begin{pgfscope}%
\pgfsys@transformshift{0.714621in}{2.239258in}%
\pgfsys@useobject{currentmarker}{}%
\end{pgfscope}%
\begin{pgfscope}%
\pgfsys@transformshift{1.455364in}{0.361922in}%
\pgfsys@useobject{currentmarker}{}%
\end{pgfscope}%
\begin{pgfscope}%
\pgfsys@transformshift{4.502330in}{3.441172in}%
\pgfsys@useobject{currentmarker}{}%
\end{pgfscope}%
\begin{pgfscope}%
\pgfsys@transformshift{5.177961in}{3.455854in}%
\pgfsys@useobject{currentmarker}{}%
\end{pgfscope}%
\begin{pgfscope}%
\pgfsys@transformshift{4.047061in}{3.342995in}%
\pgfsys@useobject{currentmarker}{}%
\end{pgfscope}%
\begin{pgfscope}%
\pgfsys@transformshift{3.295777in}{4.783609in}%
\pgfsys@useobject{currentmarker}{}%
\end{pgfscope}%
\begin{pgfscope}%
\pgfsys@transformshift{1.063543in}{0.413376in}%
\pgfsys@useobject{currentmarker}{}%
\end{pgfscope}%
\begin{pgfscope}%
\pgfsys@transformshift{0.783832in}{2.739907in}%
\pgfsys@useobject{currentmarker}{}%
\end{pgfscope}%
\begin{pgfscope}%
\pgfsys@transformshift{5.078126in}{4.404576in}%
\pgfsys@useobject{currentmarker}{}%
\end{pgfscope}%
\begin{pgfscope}%
\pgfsys@transformshift{4.136988in}{4.658145in}%
\pgfsys@useobject{currentmarker}{}%
\end{pgfscope}%
\begin{pgfscope}%
\pgfsys@transformshift{5.736855in}{1.292936in}%
\pgfsys@useobject{currentmarker}{}%
\end{pgfscope}%
\begin{pgfscope}%
\pgfsys@transformshift{1.983580in}{4.886320in}%
\pgfsys@useobject{currentmarker}{}%
\end{pgfscope}%
\begin{pgfscope}%
\pgfsys@transformshift{6.873462in}{1.295461in}%
\pgfsys@useobject{currentmarker}{}%
\end{pgfscope}%
\begin{pgfscope}%
\pgfsys@transformshift{4.409221in}{1.675381in}%
\pgfsys@useobject{currentmarker}{}%
\end{pgfscope}%
\begin{pgfscope}%
\pgfsys@transformshift{4.047554in}{5.293867in}%
\pgfsys@useobject{currentmarker}{}%
\end{pgfscope}%
\begin{pgfscope}%
\pgfsys@transformshift{1.998868in}{0.481306in}%
\pgfsys@useobject{currentmarker}{}%
\end{pgfscope}%
\begin{pgfscope}%
\pgfsys@transformshift{4.261880in}{4.741855in}%
\pgfsys@useobject{currentmarker}{}%
\end{pgfscope}%
\begin{pgfscope}%
\pgfsys@transformshift{3.995381in}{3.332257in}%
\pgfsys@useobject{currentmarker}{}%
\end{pgfscope}%
\begin{pgfscope}%
\pgfsys@transformshift{4.523514in}{2.854564in}%
\pgfsys@useobject{currentmarker}{}%
\end{pgfscope}%
\begin{pgfscope}%
\pgfsys@transformshift{1.056228in}{2.867376in}%
\pgfsys@useobject{currentmarker}{}%
\end{pgfscope}%
\begin{pgfscope}%
\pgfsys@transformshift{4.359336in}{0.514376in}%
\pgfsys@useobject{currentmarker}{}%
\end{pgfscope}%
\begin{pgfscope}%
\pgfsys@transformshift{4.808662in}{4.480181in}%
\pgfsys@useobject{currentmarker}{}%
\end{pgfscope}%
\begin{pgfscope}%
\pgfsys@transformshift{6.488879in}{4.097878in}%
\pgfsys@useobject{currentmarker}{}%
\end{pgfscope}%
\begin{pgfscope}%
\pgfsys@transformshift{6.287794in}{4.801690in}%
\pgfsys@useobject{currentmarker}{}%
\end{pgfscope}%
\begin{pgfscope}%
\pgfsys@transformshift{4.998063in}{2.004002in}%
\pgfsys@useobject{currentmarker}{}%
\end{pgfscope}%
\begin{pgfscope}%
\pgfsys@transformshift{0.985787in}{5.511862in}%
\pgfsys@useobject{currentmarker}{}%
\end{pgfscope}%
\begin{pgfscope}%
\pgfsys@transformshift{4.566527in}{5.375120in}%
\pgfsys@useobject{currentmarker}{}%
\end{pgfscope}%
\begin{pgfscope}%
\pgfsys@transformshift{1.608226in}{1.416583in}%
\pgfsys@useobject{currentmarker}{}%
\end{pgfscope}%
\begin{pgfscope}%
\pgfsys@transformshift{4.195638in}{1.749215in}%
\pgfsys@useobject{currentmarker}{}%
\end{pgfscope}%
\begin{pgfscope}%
\pgfsys@transformshift{0.763440in}{3.773630in}%
\pgfsys@useobject{currentmarker}{}%
\end{pgfscope}%
\begin{pgfscope}%
\pgfsys@transformshift{1.843265in}{4.852228in}%
\pgfsys@useobject{currentmarker}{}%
\end{pgfscope}%
\begin{pgfscope}%
\pgfsys@transformshift{1.275953in}{2.071698in}%
\pgfsys@useobject{currentmarker}{}%
\end{pgfscope}%
\begin{pgfscope}%
\pgfsys@transformshift{2.621490in}{1.658213in}%
\pgfsys@useobject{currentmarker}{}%
\end{pgfscope}%
\begin{pgfscope}%
\pgfsys@transformshift{4.138927in}{3.125188in}%
\pgfsys@useobject{currentmarker}{}%
\end{pgfscope}%
\begin{pgfscope}%
\pgfsys@transformshift{1.196307in}{2.321273in}%
\pgfsys@useobject{currentmarker}{}%
\end{pgfscope}%
\begin{pgfscope}%
\pgfsys@transformshift{3.514381in}{4.916843in}%
\pgfsys@useobject{currentmarker}{}%
\end{pgfscope}%
\begin{pgfscope}%
\pgfsys@transformshift{5.521781in}{4.691254in}%
\pgfsys@useobject{currentmarker}{}%
\end{pgfscope}%
\begin{pgfscope}%
\pgfsys@transformshift{4.495839in}{2.987808in}%
\pgfsys@useobject{currentmarker}{}%
\end{pgfscope}%
\begin{pgfscope}%
\pgfsys@transformshift{6.548739in}{2.904807in}%
\pgfsys@useobject{currentmarker}{}%
\end{pgfscope}%
\begin{pgfscope}%
\pgfsys@transformshift{3.592530in}{3.343022in}%
\pgfsys@useobject{currentmarker}{}%
\end{pgfscope}%
\begin{pgfscope}%
\pgfsys@transformshift{2.527701in}{5.010536in}%
\pgfsys@useobject{currentmarker}{}%
\end{pgfscope}%
\begin{pgfscope}%
\pgfsys@transformshift{3.904015in}{3.203000in}%
\pgfsys@useobject{currentmarker}{}%
\end{pgfscope}%
\begin{pgfscope}%
\pgfsys@transformshift{2.004440in}{3.186267in}%
\pgfsys@useobject{currentmarker}{}%
\end{pgfscope}%
\begin{pgfscope}%
\pgfsys@transformshift{3.898349in}{1.531822in}%
\pgfsys@useobject{currentmarker}{}%
\end{pgfscope}%
\begin{pgfscope}%
\pgfsys@transformshift{3.307557in}{0.739362in}%
\pgfsys@useobject{currentmarker}{}%
\end{pgfscope}%
\begin{pgfscope}%
\pgfsys@transformshift{2.013122in}{5.165292in}%
\pgfsys@useobject{currentmarker}{}%
\end{pgfscope}%
\begin{pgfscope}%
\pgfsys@transformshift{6.131114in}{4.719844in}%
\pgfsys@useobject{currentmarker}{}%
\end{pgfscope}%
\begin{pgfscope}%
\pgfsys@transformshift{5.146627in}{1.968083in}%
\pgfsys@useobject{currentmarker}{}%
\end{pgfscope}%
\begin{pgfscope}%
\pgfsys@transformshift{1.105166in}{4.174348in}%
\pgfsys@useobject{currentmarker}{}%
\end{pgfscope}%
\begin{pgfscope}%
\pgfsys@transformshift{1.727846in}{3.988167in}%
\pgfsys@useobject{currentmarker}{}%
\end{pgfscope}%
\begin{pgfscope}%
\pgfsys@transformshift{5.477223in}{2.268647in}%
\pgfsys@useobject{currentmarker}{}%
\end{pgfscope}%
\begin{pgfscope}%
\pgfsys@transformshift{0.959381in}{3.622332in}%
\pgfsys@useobject{currentmarker}{}%
\end{pgfscope}%
\begin{pgfscope}%
\pgfsys@transformshift{3.248512in}{0.542400in}%
\pgfsys@useobject{currentmarker}{}%
\end{pgfscope}%
\begin{pgfscope}%
\pgfsys@transformshift{3.129105in}{5.246287in}%
\pgfsys@useobject{currentmarker}{}%
\end{pgfscope}%
\begin{pgfscope}%
\pgfsys@transformshift{1.943354in}{4.893697in}%
\pgfsys@useobject{currentmarker}{}%
\end{pgfscope}%
\begin{pgfscope}%
\pgfsys@transformshift{3.587215in}{3.580868in}%
\pgfsys@useobject{currentmarker}{}%
\end{pgfscope}%
\begin{pgfscope}%
\pgfsys@transformshift{2.342935in}{2.097338in}%
\pgfsys@useobject{currentmarker}{}%
\end{pgfscope}%
\begin{pgfscope}%
\pgfsys@transformshift{1.021497in}{0.690499in}%
\pgfsys@useobject{currentmarker}{}%
\end{pgfscope}%
\begin{pgfscope}%
\pgfsys@transformshift{3.817014in}{1.576762in}%
\pgfsys@useobject{currentmarker}{}%
\end{pgfscope}%
\begin{pgfscope}%
\pgfsys@transformshift{5.442389in}{2.100678in}%
\pgfsys@useobject{currentmarker}{}%
\end{pgfscope}%
\begin{pgfscope}%
\pgfsys@transformshift{6.058288in}{4.673716in}%
\pgfsys@useobject{currentmarker}{}%
\end{pgfscope}%
\begin{pgfscope}%
\pgfsys@transformshift{1.979398in}{0.840928in}%
\pgfsys@useobject{currentmarker}{}%
\end{pgfscope}%
\begin{pgfscope}%
\pgfsys@transformshift{2.648712in}{3.247655in}%
\pgfsys@useobject{currentmarker}{}%
\end{pgfscope}%
\begin{pgfscope}%
\pgfsys@transformshift{5.949050in}{5.533644in}%
\pgfsys@useobject{currentmarker}{}%
\end{pgfscope}%
\begin{pgfscope}%
\pgfsys@transformshift{5.550495in}{1.991920in}%
\pgfsys@useobject{currentmarker}{}%
\end{pgfscope}%
\begin{pgfscope}%
\pgfsys@transformshift{6.755225in}{5.103245in}%
\pgfsys@useobject{currentmarker}{}%
\end{pgfscope}%
\begin{pgfscope}%
\pgfsys@transformshift{3.865731in}{2.334149in}%
\pgfsys@useobject{currentmarker}{}%
\end{pgfscope}%
\begin{pgfscope}%
\pgfsys@transformshift{2.976125in}{5.561992in}%
\pgfsys@useobject{currentmarker}{}%
\end{pgfscope}%
\begin{pgfscope}%
\pgfsys@transformshift{3.081797in}{3.709330in}%
\pgfsys@useobject{currentmarker}{}%
\end{pgfscope}%
\begin{pgfscope}%
\pgfsys@transformshift{1.382073in}{0.821854in}%
\pgfsys@useobject{currentmarker}{}%
\end{pgfscope}%
\begin{pgfscope}%
\pgfsys@transformshift{5.184168in}{4.949794in}%
\pgfsys@useobject{currentmarker}{}%
\end{pgfscope}%
\begin{pgfscope}%
\pgfsys@transformshift{4.142106in}{4.831008in}%
\pgfsys@useobject{currentmarker}{}%
\end{pgfscope}%
\begin{pgfscope}%
\pgfsys@transformshift{0.655059in}{2.256314in}%
\pgfsys@useobject{currentmarker}{}%
\end{pgfscope}%
\begin{pgfscope}%
\pgfsys@transformshift{0.802744in}{2.327255in}%
\pgfsys@useobject{currentmarker}{}%
\end{pgfscope}%
\begin{pgfscope}%
\pgfsys@transformshift{2.579353in}{2.170771in}%
\pgfsys@useobject{currentmarker}{}%
\end{pgfscope}%
\begin{pgfscope}%
\pgfsys@transformshift{1.585746in}{4.271890in}%
\pgfsys@useobject{currentmarker}{}%
\end{pgfscope}%
\begin{pgfscope}%
\pgfsys@transformshift{6.865968in}{0.230952in}%
\pgfsys@useobject{currentmarker}{}%
\end{pgfscope}%
\begin{pgfscope}%
\pgfsys@transformshift{4.662028in}{2.002836in}%
\pgfsys@useobject{currentmarker}{}%
\end{pgfscope}%
\begin{pgfscope}%
\pgfsys@transformshift{4.027602in}{2.510529in}%
\pgfsys@useobject{currentmarker}{}%
\end{pgfscope}%
\begin{pgfscope}%
\pgfsys@transformshift{6.703161in}{1.286896in}%
\pgfsys@useobject{currentmarker}{}%
\end{pgfscope}%
\begin{pgfscope}%
\pgfsys@transformshift{4.412214in}{4.151845in}%
\pgfsys@useobject{currentmarker}{}%
\end{pgfscope}%
\begin{pgfscope}%
\pgfsys@transformshift{1.445357in}{3.910361in}%
\pgfsys@useobject{currentmarker}{}%
\end{pgfscope}%
\begin{pgfscope}%
\pgfsys@transformshift{5.808650in}{2.612827in}%
\pgfsys@useobject{currentmarker}{}%
\end{pgfscope}%
\begin{pgfscope}%
\pgfsys@transformshift{1.220703in}{0.896912in}%
\pgfsys@useobject{currentmarker}{}%
\end{pgfscope}%
\begin{pgfscope}%
\pgfsys@transformshift{3.603961in}{4.178180in}%
\pgfsys@useobject{currentmarker}{}%
\end{pgfscope}%
\begin{pgfscope}%
\pgfsys@transformshift{4.539430in}{0.774317in}%
\pgfsys@useobject{currentmarker}{}%
\end{pgfscope}%
\begin{pgfscope}%
\pgfsys@transformshift{2.716342in}{1.761126in}%
\pgfsys@useobject{currentmarker}{}%
\end{pgfscope}%
\begin{pgfscope}%
\pgfsys@transformshift{2.797144in}{0.838678in}%
\pgfsys@useobject{currentmarker}{}%
\end{pgfscope}%
\begin{pgfscope}%
\pgfsys@transformshift{3.762384in}{2.095525in}%
\pgfsys@useobject{currentmarker}{}%
\end{pgfscope}%
\begin{pgfscope}%
\pgfsys@transformshift{5.164025in}{0.672611in}%
\pgfsys@useobject{currentmarker}{}%
\end{pgfscope}%
\begin{pgfscope}%
\pgfsys@transformshift{1.870542in}{0.457542in}%
\pgfsys@useobject{currentmarker}{}%
\end{pgfscope}%
\begin{pgfscope}%
\pgfsys@transformshift{2.910121in}{4.505037in}%
\pgfsys@useobject{currentmarker}{}%
\end{pgfscope}%
\begin{pgfscope}%
\pgfsys@transformshift{1.411858in}{2.496224in}%
\pgfsys@useobject{currentmarker}{}%
\end{pgfscope}%
\begin{pgfscope}%
\pgfsys@transformshift{4.895000in}{0.989972in}%
\pgfsys@useobject{currentmarker}{}%
\end{pgfscope}%
\begin{pgfscope}%
\pgfsys@transformshift{0.738002in}{0.308557in}%
\pgfsys@useobject{currentmarker}{}%
\end{pgfscope}%
\begin{pgfscope}%
\pgfsys@transformshift{4.221218in}{1.777841in}%
\pgfsys@useobject{currentmarker}{}%
\end{pgfscope}%
\begin{pgfscope}%
\pgfsys@transformshift{4.841482in}{1.341076in}%
\pgfsys@useobject{currentmarker}{}%
\end{pgfscope}%
\begin{pgfscope}%
\pgfsys@transformshift{3.825316in}{5.064443in}%
\pgfsys@useobject{currentmarker}{}%
\end{pgfscope}%
\begin{pgfscope}%
\pgfsys@transformshift{4.349185in}{3.651545in}%
\pgfsys@useobject{currentmarker}{}%
\end{pgfscope}%
\begin{pgfscope}%
\pgfsys@transformshift{6.614449in}{2.078768in}%
\pgfsys@useobject{currentmarker}{}%
\end{pgfscope}%
\begin{pgfscope}%
\pgfsys@transformshift{3.440025in}{3.712999in}%
\pgfsys@useobject{currentmarker}{}%
\end{pgfscope}%
\begin{pgfscope}%
\pgfsys@transformshift{3.524073in}{3.416601in}%
\pgfsys@useobject{currentmarker}{}%
\end{pgfscope}%
\begin{pgfscope}%
\pgfsys@transformshift{0.656534in}{3.158901in}%
\pgfsys@useobject{currentmarker}{}%
\end{pgfscope}%
\begin{pgfscope}%
\pgfsys@transformshift{2.233883in}{2.221251in}%
\pgfsys@useobject{currentmarker}{}%
\end{pgfscope}%
\begin{pgfscope}%
\pgfsys@transformshift{5.568420in}{1.991570in}%
\pgfsys@useobject{currentmarker}{}%
\end{pgfscope}%
\begin{pgfscope}%
\pgfsys@transformshift{6.063286in}{3.907894in}%
\pgfsys@useobject{currentmarker}{}%
\end{pgfscope}%
\begin{pgfscope}%
\pgfsys@transformshift{5.342023in}{3.563392in}%
\pgfsys@useobject{currentmarker}{}%
\end{pgfscope}%
\begin{pgfscope}%
\pgfsys@transformshift{6.739944in}{5.157110in}%
\pgfsys@useobject{currentmarker}{}%
\end{pgfscope}%
\begin{pgfscope}%
\pgfsys@transformshift{6.204109in}{4.336997in}%
\pgfsys@useobject{currentmarker}{}%
\end{pgfscope}%
\begin{pgfscope}%
\pgfsys@transformshift{0.896520in}{0.404529in}%
\pgfsys@useobject{currentmarker}{}%
\end{pgfscope}%
\begin{pgfscope}%
\pgfsys@transformshift{5.692888in}{5.529288in}%
\pgfsys@useobject{currentmarker}{}%
\end{pgfscope}%
\begin{pgfscope}%
\pgfsys@transformshift{4.756665in}{2.321836in}%
\pgfsys@useobject{currentmarker}{}%
\end{pgfscope}%
\begin{pgfscope}%
\pgfsys@transformshift{4.620765in}{1.241046in}%
\pgfsys@useobject{currentmarker}{}%
\end{pgfscope}%
\begin{pgfscope}%
\pgfsys@transformshift{3.198662in}{2.226083in}%
\pgfsys@useobject{currentmarker}{}%
\end{pgfscope}%
\begin{pgfscope}%
\pgfsys@transformshift{2.414933in}{1.108528in}%
\pgfsys@useobject{currentmarker}{}%
\end{pgfscope}%
\begin{pgfscope}%
\pgfsys@transformshift{6.506654in}{5.601735in}%
\pgfsys@useobject{currentmarker}{}%
\end{pgfscope}%
\begin{pgfscope}%
\pgfsys@transformshift{4.142466in}{3.244470in}%
\pgfsys@useobject{currentmarker}{}%
\end{pgfscope}%
\begin{pgfscope}%
\pgfsys@transformshift{6.658868in}{3.905016in}%
\pgfsys@useobject{currentmarker}{}%
\end{pgfscope}%
\begin{pgfscope}%
\pgfsys@transformshift{0.934661in}{1.771176in}%
\pgfsys@useobject{currentmarker}{}%
\end{pgfscope}%
\begin{pgfscope}%
\pgfsys@transformshift{3.154612in}{2.342803in}%
\pgfsys@useobject{currentmarker}{}%
\end{pgfscope}%
\begin{pgfscope}%
\pgfsys@transformshift{2.002356in}{2.343359in}%
\pgfsys@useobject{currentmarker}{}%
\end{pgfscope}%
\begin{pgfscope}%
\pgfsys@transformshift{5.090432in}{4.517353in}%
\pgfsys@useobject{currentmarker}{}%
\end{pgfscope}%
\begin{pgfscope}%
\pgfsys@transformshift{6.080457in}{1.202822in}%
\pgfsys@useobject{currentmarker}{}%
\end{pgfscope}%
\begin{pgfscope}%
\pgfsys@transformshift{3.323354in}{1.022890in}%
\pgfsys@useobject{currentmarker}{}%
\end{pgfscope}%
\begin{pgfscope}%
\pgfsys@transformshift{2.354026in}{0.531485in}%
\pgfsys@useobject{currentmarker}{}%
\end{pgfscope}%
\begin{pgfscope}%
\pgfsys@transformshift{4.593579in}{1.800996in}%
\pgfsys@useobject{currentmarker}{}%
\end{pgfscope}%
\begin{pgfscope}%
\pgfsys@transformshift{6.272463in}{1.782078in}%
\pgfsys@useobject{currentmarker}{}%
\end{pgfscope}%
\begin{pgfscope}%
\pgfsys@transformshift{1.152253in}{2.650657in}%
\pgfsys@useobject{currentmarker}{}%
\end{pgfscope}%
\begin{pgfscope}%
\pgfsys@transformshift{1.944140in}{3.576537in}%
\pgfsys@useobject{currentmarker}{}%
\end{pgfscope}%
\begin{pgfscope}%
\pgfsys@transformshift{3.994197in}{5.242610in}%
\pgfsys@useobject{currentmarker}{}%
\end{pgfscope}%
\begin{pgfscope}%
\pgfsys@transformshift{2.653452in}{4.274385in}%
\pgfsys@useobject{currentmarker}{}%
\end{pgfscope}%
\begin{pgfscope}%
\pgfsys@transformshift{5.626037in}{0.483404in}%
\pgfsys@useobject{currentmarker}{}%
\end{pgfscope}%
\begin{pgfscope}%
\pgfsys@transformshift{2.255638in}{4.246827in}%
\pgfsys@useobject{currentmarker}{}%
\end{pgfscope}%
\begin{pgfscope}%
\pgfsys@transformshift{4.652513in}{3.859054in}%
\pgfsys@useobject{currentmarker}{}%
\end{pgfscope}%
\begin{pgfscope}%
\pgfsys@transformshift{3.277927in}{0.472024in}%
\pgfsys@useobject{currentmarker}{}%
\end{pgfscope}%
\begin{pgfscope}%
\pgfsys@transformshift{2.053643in}{2.791624in}%
\pgfsys@useobject{currentmarker}{}%
\end{pgfscope}%
\begin{pgfscope}%
\pgfsys@transformshift{3.166837in}{4.829729in}%
\pgfsys@useobject{currentmarker}{}%
\end{pgfscope}%
\begin{pgfscope}%
\pgfsys@transformshift{5.976142in}{5.391060in}%
\pgfsys@useobject{currentmarker}{}%
\end{pgfscope}%
\begin{pgfscope}%
\pgfsys@transformshift{6.825375in}{5.032743in}%
\pgfsys@useobject{currentmarker}{}%
\end{pgfscope}%
\begin{pgfscope}%
\pgfsys@transformshift{4.646199in}{2.772443in}%
\pgfsys@useobject{currentmarker}{}%
\end{pgfscope}%
\begin{pgfscope}%
\pgfsys@transformshift{1.824536in}{0.631393in}%
\pgfsys@useobject{currentmarker}{}%
\end{pgfscope}%
\begin{pgfscope}%
\pgfsys@transformshift{4.222637in}{3.845652in}%
\pgfsys@useobject{currentmarker}{}%
\end{pgfscope}%
\begin{pgfscope}%
\pgfsys@transformshift{4.493681in}{0.668277in}%
\pgfsys@useobject{currentmarker}{}%
\end{pgfscope}%
\begin{pgfscope}%
\pgfsys@transformshift{5.758562in}{0.396976in}%
\pgfsys@useobject{currentmarker}{}%
\end{pgfscope}%
\begin{pgfscope}%
\pgfsys@transformshift{4.139112in}{0.785824in}%
\pgfsys@useobject{currentmarker}{}%
\end{pgfscope}%
\begin{pgfscope}%
\pgfsys@transformshift{6.889347in}{3.604728in}%
\pgfsys@useobject{currentmarker}{}%
\end{pgfscope}%
\begin{pgfscope}%
\pgfsys@transformshift{2.493739in}{4.962428in}%
\pgfsys@useobject{currentmarker}{}%
\end{pgfscope}%
\begin{pgfscope}%
\pgfsys@transformshift{4.887080in}{1.597741in}%
\pgfsys@useobject{currentmarker}{}%
\end{pgfscope}%
\begin{pgfscope}%
\pgfsys@transformshift{6.166293in}{5.525876in}%
\pgfsys@useobject{currentmarker}{}%
\end{pgfscope}%
\begin{pgfscope}%
\pgfsys@transformshift{3.989516in}{3.609263in}%
\pgfsys@useobject{currentmarker}{}%
\end{pgfscope}%
\begin{pgfscope}%
\pgfsys@transformshift{3.344430in}{0.694691in}%
\pgfsys@useobject{currentmarker}{}%
\end{pgfscope}%
\begin{pgfscope}%
\pgfsys@transformshift{6.075360in}{0.974671in}%
\pgfsys@useobject{currentmarker}{}%
\end{pgfscope}%
\begin{pgfscope}%
\pgfsys@transformshift{4.786806in}{2.685359in}%
\pgfsys@useobject{currentmarker}{}%
\end{pgfscope}%
\begin{pgfscope}%
\pgfsys@transformshift{2.275116in}{4.214604in}%
\pgfsys@useobject{currentmarker}{}%
\end{pgfscope}%
\begin{pgfscope}%
\pgfsys@transformshift{5.820529in}{4.658387in}%
\pgfsys@useobject{currentmarker}{}%
\end{pgfscope}%
\begin{pgfscope}%
\pgfsys@transformshift{1.870196in}{3.240343in}%
\pgfsys@useobject{currentmarker}{}%
\end{pgfscope}%
\begin{pgfscope}%
\pgfsys@transformshift{1.672379in}{4.284370in}%
\pgfsys@useobject{currentmarker}{}%
\end{pgfscope}%
\begin{pgfscope}%
\pgfsys@transformshift{2.302752in}{4.962718in}%
\pgfsys@useobject{currentmarker}{}%
\end{pgfscope}%
\begin{pgfscope}%
\pgfsys@transformshift{0.673143in}{0.893077in}%
\pgfsys@useobject{currentmarker}{}%
\end{pgfscope}%
\begin{pgfscope}%
\pgfsys@transformshift{1.263833in}{1.611700in}%
\pgfsys@useobject{currentmarker}{}%
\end{pgfscope}%
\begin{pgfscope}%
\pgfsys@transformshift{4.282282in}{4.761581in}%
\pgfsys@useobject{currentmarker}{}%
\end{pgfscope}%
\begin{pgfscope}%
\pgfsys@transformshift{4.622965in}{3.963335in}%
\pgfsys@useobject{currentmarker}{}%
\end{pgfscope}%
\begin{pgfscope}%
\pgfsys@transformshift{1.301933in}{1.347133in}%
\pgfsys@useobject{currentmarker}{}%
\end{pgfscope}%
\begin{pgfscope}%
\pgfsys@transformshift{5.734062in}{3.964222in}%
\pgfsys@useobject{currentmarker}{}%
\end{pgfscope}%
\begin{pgfscope}%
\pgfsys@transformshift{6.543641in}{1.069784in}%
\pgfsys@useobject{currentmarker}{}%
\end{pgfscope}%
\begin{pgfscope}%
\pgfsys@transformshift{1.455448in}{1.633203in}%
\pgfsys@useobject{currentmarker}{}%
\end{pgfscope}%
\begin{pgfscope}%
\pgfsys@transformshift{5.604056in}{2.970101in}%
\pgfsys@useobject{currentmarker}{}%
\end{pgfscope}%
\begin{pgfscope}%
\pgfsys@transformshift{4.489020in}{1.154589in}%
\pgfsys@useobject{currentmarker}{}%
\end{pgfscope}%
\begin{pgfscope}%
\pgfsys@transformshift{4.421605in}{3.229968in}%
\pgfsys@useobject{currentmarker}{}%
\end{pgfscope}%
\begin{pgfscope}%
\pgfsys@transformshift{4.858586in}{2.609125in}%
\pgfsys@useobject{currentmarker}{}%
\end{pgfscope}%
\begin{pgfscope}%
\pgfsys@transformshift{4.648877in}{2.516551in}%
\pgfsys@useobject{currentmarker}{}%
\end{pgfscope}%
\begin{pgfscope}%
\pgfsys@transformshift{4.130094in}{1.571152in}%
\pgfsys@useobject{currentmarker}{}%
\end{pgfscope}%
\begin{pgfscope}%
\pgfsys@transformshift{0.896027in}{1.189427in}%
\pgfsys@useobject{currentmarker}{}%
\end{pgfscope}%
\begin{pgfscope}%
\pgfsys@transformshift{1.847530in}{0.612260in}%
\pgfsys@useobject{currentmarker}{}%
\end{pgfscope}%
\begin{pgfscope}%
\pgfsys@transformshift{6.351608in}{0.972429in}%
\pgfsys@useobject{currentmarker}{}%
\end{pgfscope}%
\begin{pgfscope}%
\pgfsys@transformshift{6.134735in}{5.352141in}%
\pgfsys@useobject{currentmarker}{}%
\end{pgfscope}%
\begin{pgfscope}%
\pgfsys@transformshift{6.182954in}{5.595977in}%
\pgfsys@useobject{currentmarker}{}%
\end{pgfscope}%
\begin{pgfscope}%
\pgfsys@transformshift{6.046383in}{4.049364in}%
\pgfsys@useobject{currentmarker}{}%
\end{pgfscope}%
\begin{pgfscope}%
\pgfsys@transformshift{5.741789in}{2.214999in}%
\pgfsys@useobject{currentmarker}{}%
\end{pgfscope}%
\begin{pgfscope}%
\pgfsys@transformshift{2.504074in}{4.901585in}%
\pgfsys@useobject{currentmarker}{}%
\end{pgfscope}%
\begin{pgfscope}%
\pgfsys@transformshift{4.351859in}{0.591350in}%
\pgfsys@useobject{currentmarker}{}%
\end{pgfscope}%
\begin{pgfscope}%
\pgfsys@transformshift{3.783058in}{2.728852in}%
\pgfsys@useobject{currentmarker}{}%
\end{pgfscope}%
\begin{pgfscope}%
\pgfsys@transformshift{6.770911in}{3.250488in}%
\pgfsys@useobject{currentmarker}{}%
\end{pgfscope}%
\begin{pgfscope}%
\pgfsys@transformshift{6.342129in}{0.486091in}%
\pgfsys@useobject{currentmarker}{}%
\end{pgfscope}%
\begin{pgfscope}%
\pgfsys@transformshift{2.417289in}{3.706067in}%
\pgfsys@useobject{currentmarker}{}%
\end{pgfscope}%
\begin{pgfscope}%
\pgfsys@transformshift{4.438185in}{3.896967in}%
\pgfsys@useobject{currentmarker}{}%
\end{pgfscope}%
\begin{pgfscope}%
\pgfsys@transformshift{1.602751in}{2.048253in}%
\pgfsys@useobject{currentmarker}{}%
\end{pgfscope}%
\begin{pgfscope}%
\pgfsys@transformshift{5.482158in}{0.576145in}%
\pgfsys@useobject{currentmarker}{}%
\end{pgfscope}%
\begin{pgfscope}%
\pgfsys@transformshift{2.180598in}{2.896309in}%
\pgfsys@useobject{currentmarker}{}%
\end{pgfscope}%
\begin{pgfscope}%
\pgfsys@transformshift{1.404368in}{0.948406in}%
\pgfsys@useobject{currentmarker}{}%
\end{pgfscope}%
\begin{pgfscope}%
\pgfsys@transformshift{4.519508in}{2.969509in}%
\pgfsys@useobject{currentmarker}{}%
\end{pgfscope}%
\begin{pgfscope}%
\pgfsys@transformshift{2.527479in}{4.051990in}%
\pgfsys@useobject{currentmarker}{}%
\end{pgfscope}%
\begin{pgfscope}%
\pgfsys@transformshift{3.248174in}{0.965477in}%
\pgfsys@useobject{currentmarker}{}%
\end{pgfscope}%
\begin{pgfscope}%
\pgfsys@transformshift{3.839753in}{0.806067in}%
\pgfsys@useobject{currentmarker}{}%
\end{pgfscope}%
\begin{pgfscope}%
\pgfsys@transformshift{6.845770in}{3.650208in}%
\pgfsys@useobject{currentmarker}{}%
\end{pgfscope}%
\begin{pgfscope}%
\pgfsys@transformshift{1.617632in}{3.341566in}%
\pgfsys@useobject{currentmarker}{}%
\end{pgfscope}%
\begin{pgfscope}%
\pgfsys@transformshift{2.896925in}{4.283457in}%
\pgfsys@useobject{currentmarker}{}%
\end{pgfscope}%
\begin{pgfscope}%
\pgfsys@transformshift{6.327943in}{3.286432in}%
\pgfsys@useobject{currentmarker}{}%
\end{pgfscope}%
\begin{pgfscope}%
\pgfsys@transformshift{3.189071in}{0.912943in}%
\pgfsys@useobject{currentmarker}{}%
\end{pgfscope}%
\begin{pgfscope}%
\pgfsys@transformshift{2.362344in}{2.960396in}%
\pgfsys@useobject{currentmarker}{}%
\end{pgfscope}%
\begin{pgfscope}%
\pgfsys@transformshift{2.429558in}{1.668869in}%
\pgfsys@useobject{currentmarker}{}%
\end{pgfscope}%
\begin{pgfscope}%
\pgfsys@transformshift{3.106341in}{1.308808in}%
\pgfsys@useobject{currentmarker}{}%
\end{pgfscope}%
\begin{pgfscope}%
\pgfsys@transformshift{3.037818in}{3.917910in}%
\pgfsys@useobject{currentmarker}{}%
\end{pgfscope}%
\begin{pgfscope}%
\pgfsys@transformshift{5.310423in}{0.359517in}%
\pgfsys@useobject{currentmarker}{}%
\end{pgfscope}%
\begin{pgfscope}%
\pgfsys@transformshift{5.540275in}{4.582562in}%
\pgfsys@useobject{currentmarker}{}%
\end{pgfscope}%
\begin{pgfscope}%
\pgfsys@transformshift{4.445540in}{2.355851in}%
\pgfsys@useobject{currentmarker}{}%
\end{pgfscope}%
\begin{pgfscope}%
\pgfsys@transformshift{6.144699in}{0.842884in}%
\pgfsys@useobject{currentmarker}{}%
\end{pgfscope}%
\begin{pgfscope}%
\pgfsys@transformshift{5.587596in}{4.316849in}%
\pgfsys@useobject{currentmarker}{}%
\end{pgfscope}%
\begin{pgfscope}%
\pgfsys@transformshift{5.860039in}{4.270332in}%
\pgfsys@useobject{currentmarker}{}%
\end{pgfscope}%
\begin{pgfscope}%
\pgfsys@transformshift{1.327747in}{0.342544in}%
\pgfsys@useobject{currentmarker}{}%
\end{pgfscope}%
\begin{pgfscope}%
\pgfsys@transformshift{5.284133in}{3.990551in}%
\pgfsys@useobject{currentmarker}{}%
\end{pgfscope}%
\begin{pgfscope}%
\pgfsys@transformshift{3.993120in}{0.899747in}%
\pgfsys@useobject{currentmarker}{}%
\end{pgfscope}%
\begin{pgfscope}%
\pgfsys@transformshift{1.820555in}{3.037024in}%
\pgfsys@useobject{currentmarker}{}%
\end{pgfscope}%
\begin{pgfscope}%
\pgfsys@transformshift{4.976421in}{5.467875in}%
\pgfsys@useobject{currentmarker}{}%
\end{pgfscope}%
\begin{pgfscope}%
\pgfsys@transformshift{4.518831in}{5.000925in}%
\pgfsys@useobject{currentmarker}{}%
\end{pgfscope}%
\begin{pgfscope}%
\pgfsys@transformshift{5.758990in}{0.597983in}%
\pgfsys@useobject{currentmarker}{}%
\end{pgfscope}%
\begin{pgfscope}%
\pgfsys@transformshift{5.322391in}{5.195192in}%
\pgfsys@useobject{currentmarker}{}%
\end{pgfscope}%
\begin{pgfscope}%
\pgfsys@transformshift{6.191893in}{3.808835in}%
\pgfsys@useobject{currentmarker}{}%
\end{pgfscope}%
\begin{pgfscope}%
\pgfsys@transformshift{4.112982in}{3.072140in}%
\pgfsys@useobject{currentmarker}{}%
\end{pgfscope}%
\begin{pgfscope}%
\pgfsys@transformshift{4.346249in}{2.073142in}%
\pgfsys@useobject{currentmarker}{}%
\end{pgfscope}%
\begin{pgfscope}%
\pgfsys@transformshift{5.031335in}{4.367788in}%
\pgfsys@useobject{currentmarker}{}%
\end{pgfscope}%
\begin{pgfscope}%
\pgfsys@transformshift{2.320682in}{1.898677in}%
\pgfsys@useobject{currentmarker}{}%
\end{pgfscope}%
\begin{pgfscope}%
\pgfsys@transformshift{2.639656in}{4.074777in}%
\pgfsys@useobject{currentmarker}{}%
\end{pgfscope}%
\begin{pgfscope}%
\pgfsys@transformshift{4.591237in}{3.148890in}%
\pgfsys@useobject{currentmarker}{}%
\end{pgfscope}%
\begin{pgfscope}%
\pgfsys@transformshift{5.707718in}{0.457513in}%
\pgfsys@useobject{currentmarker}{}%
\end{pgfscope}%
\begin{pgfscope}%
\pgfsys@transformshift{2.585349in}{5.586581in}%
\pgfsys@useobject{currentmarker}{}%
\end{pgfscope}%
\begin{pgfscope}%
\pgfsys@transformshift{1.537944in}{5.101799in}%
\pgfsys@useobject{currentmarker}{}%
\end{pgfscope}%
\begin{pgfscope}%
\pgfsys@transformshift{5.052617in}{2.372156in}%
\pgfsys@useobject{currentmarker}{}%
\end{pgfscope}%
\begin{pgfscope}%
\pgfsys@transformshift{3.531247in}{3.288660in}%
\pgfsys@useobject{currentmarker}{}%
\end{pgfscope}%
\begin{pgfscope}%
\pgfsys@transformshift{5.727482in}{2.427963in}%
\pgfsys@useobject{currentmarker}{}%
\end{pgfscope}%
\begin{pgfscope}%
\pgfsys@transformshift{1.090843in}{5.060215in}%
\pgfsys@useobject{currentmarker}{}%
\end{pgfscope}%
\begin{pgfscope}%
\pgfsys@transformshift{0.917944in}{4.411994in}%
\pgfsys@useobject{currentmarker}{}%
\end{pgfscope}%
\begin{pgfscope}%
\pgfsys@transformshift{5.944407in}{4.296100in}%
\pgfsys@useobject{currentmarker}{}%
\end{pgfscope}%
\begin{pgfscope}%
\pgfsys@transformshift{2.400886in}{0.241725in}%
\pgfsys@useobject{currentmarker}{}%
\end{pgfscope}%
\begin{pgfscope}%
\pgfsys@transformshift{3.933951in}{2.423825in}%
\pgfsys@useobject{currentmarker}{}%
\end{pgfscope}%
\begin{pgfscope}%
\pgfsys@transformshift{2.953373in}{0.403868in}%
\pgfsys@useobject{currentmarker}{}%
\end{pgfscope}%
\begin{pgfscope}%
\pgfsys@transformshift{3.684811in}{5.370307in}%
\pgfsys@useobject{currentmarker}{}%
\end{pgfscope}%
\begin{pgfscope}%
\pgfsys@transformshift{4.527017in}{2.440840in}%
\pgfsys@useobject{currentmarker}{}%
\end{pgfscope}%
\begin{pgfscope}%
\pgfsys@transformshift{1.597930in}{3.973651in}%
\pgfsys@useobject{currentmarker}{}%
\end{pgfscope}%
\begin{pgfscope}%
\pgfsys@transformshift{4.244140in}{5.201478in}%
\pgfsys@useobject{currentmarker}{}%
\end{pgfscope}%
\begin{pgfscope}%
\pgfsys@transformshift{2.731012in}{2.446474in}%
\pgfsys@useobject{currentmarker}{}%
\end{pgfscope}%
\begin{pgfscope}%
\pgfsys@transformshift{1.721705in}{3.275660in}%
\pgfsys@useobject{currentmarker}{}%
\end{pgfscope}%
\begin{pgfscope}%
\pgfsys@transformshift{6.883942in}{1.299839in}%
\pgfsys@useobject{currentmarker}{}%
\end{pgfscope}%
\begin{pgfscope}%
\pgfsys@transformshift{1.465171in}{0.970937in}%
\pgfsys@useobject{currentmarker}{}%
\end{pgfscope}%
\begin{pgfscope}%
\pgfsys@transformshift{2.667479in}{5.139195in}%
\pgfsys@useobject{currentmarker}{}%
\end{pgfscope}%
\begin{pgfscope}%
\pgfsys@transformshift{2.701238in}{2.707072in}%
\pgfsys@useobject{currentmarker}{}%
\end{pgfscope}%
\begin{pgfscope}%
\pgfsys@transformshift{6.547016in}{2.283006in}%
\pgfsys@useobject{currentmarker}{}%
\end{pgfscope}%
\begin{pgfscope}%
\pgfsys@transformshift{3.266648in}{5.347777in}%
\pgfsys@useobject{currentmarker}{}%
\end{pgfscope}%
\begin{pgfscope}%
\pgfsys@transformshift{2.378545in}{1.229894in}%
\pgfsys@useobject{currentmarker}{}%
\end{pgfscope}%
\begin{pgfscope}%
\pgfsys@transformshift{5.194616in}{2.453481in}%
\pgfsys@useobject{currentmarker}{}%
\end{pgfscope}%
\begin{pgfscope}%
\pgfsys@transformshift{4.251196in}{5.418875in}%
\pgfsys@useobject{currentmarker}{}%
\end{pgfscope}%
\begin{pgfscope}%
\pgfsys@transformshift{5.395600in}{2.679137in}%
\pgfsys@useobject{currentmarker}{}%
\end{pgfscope}%
\begin{pgfscope}%
\pgfsys@transformshift{1.518682in}{4.800976in}%
\pgfsys@useobject{currentmarker}{}%
\end{pgfscope}%
\begin{pgfscope}%
\pgfsys@transformshift{4.342434in}{0.678690in}%
\pgfsys@useobject{currentmarker}{}%
\end{pgfscope}%
\begin{pgfscope}%
\pgfsys@transformshift{6.703472in}{2.313088in}%
\pgfsys@useobject{currentmarker}{}%
\end{pgfscope}%
\begin{pgfscope}%
\pgfsys@transformshift{0.850509in}{5.345405in}%
\pgfsys@useobject{currentmarker}{}%
\end{pgfscope}%
\begin{pgfscope}%
\pgfsys@transformshift{4.473622in}{2.677575in}%
\pgfsys@useobject{currentmarker}{}%
\end{pgfscope}%
\begin{pgfscope}%
\pgfsys@transformshift{3.497820in}{5.397189in}%
\pgfsys@useobject{currentmarker}{}%
\end{pgfscope}%
\begin{pgfscope}%
\pgfsys@transformshift{2.507566in}{2.142928in}%
\pgfsys@useobject{currentmarker}{}%
\end{pgfscope}%
\begin{pgfscope}%
\pgfsys@transformshift{1.572349in}{1.477577in}%
\pgfsys@useobject{currentmarker}{}%
\end{pgfscope}%
\begin{pgfscope}%
\pgfsys@transformshift{0.918109in}{2.102665in}%
\pgfsys@useobject{currentmarker}{}%
\end{pgfscope}%
\begin{pgfscope}%
\pgfsys@transformshift{5.551334in}{2.724328in}%
\pgfsys@useobject{currentmarker}{}%
\end{pgfscope}%
\begin{pgfscope}%
\pgfsys@transformshift{3.804229in}{3.856476in}%
\pgfsys@useobject{currentmarker}{}%
\end{pgfscope}%
\begin{pgfscope}%
\pgfsys@transformshift{3.405157in}{1.312488in}%
\pgfsys@useobject{currentmarker}{}%
\end{pgfscope}%
\begin{pgfscope}%
\pgfsys@transformshift{6.210793in}{3.665697in}%
\pgfsys@useobject{currentmarker}{}%
\end{pgfscope}%
\begin{pgfscope}%
\pgfsys@transformshift{5.999147in}{3.448027in}%
\pgfsys@useobject{currentmarker}{}%
\end{pgfscope}%
\begin{pgfscope}%
\pgfsys@transformshift{3.648746in}{5.011550in}%
\pgfsys@useobject{currentmarker}{}%
\end{pgfscope}%
\begin{pgfscope}%
\pgfsys@transformshift{0.730912in}{5.317873in}%
\pgfsys@useobject{currentmarker}{}%
\end{pgfscope}%
\begin{pgfscope}%
\pgfsys@transformshift{1.721779in}{0.648734in}%
\pgfsys@useobject{currentmarker}{}%
\end{pgfscope}%
\begin{pgfscope}%
\pgfsys@transformshift{2.297439in}{3.943681in}%
\pgfsys@useobject{currentmarker}{}%
\end{pgfscope}%
\begin{pgfscope}%
\pgfsys@transformshift{4.635825in}{4.998243in}%
\pgfsys@useobject{currentmarker}{}%
\end{pgfscope}%
\begin{pgfscope}%
\pgfsys@transformshift{1.290406in}{1.104941in}%
\pgfsys@useobject{currentmarker}{}%
\end{pgfscope}%
\begin{pgfscope}%
\pgfsys@transformshift{0.954170in}{5.152561in}%
\pgfsys@useobject{currentmarker}{}%
\end{pgfscope}%
\begin{pgfscope}%
\pgfsys@transformshift{3.740532in}{1.895624in}%
\pgfsys@useobject{currentmarker}{}%
\end{pgfscope}%
\begin{pgfscope}%
\pgfsys@transformshift{1.671660in}{5.497159in}%
\pgfsys@useobject{currentmarker}{}%
\end{pgfscope}%
\begin{pgfscope}%
\pgfsys@transformshift{4.359756in}{3.536926in}%
\pgfsys@useobject{currentmarker}{}%
\end{pgfscope}%
\begin{pgfscope}%
\pgfsys@transformshift{4.913462in}{0.591345in}%
\pgfsys@useobject{currentmarker}{}%
\end{pgfscope}%
\begin{pgfscope}%
\pgfsys@transformshift{5.375068in}{1.908884in}%
\pgfsys@useobject{currentmarker}{}%
\end{pgfscope}%
\begin{pgfscope}%
\pgfsys@transformshift{4.581745in}{0.846596in}%
\pgfsys@useobject{currentmarker}{}%
\end{pgfscope}%
\begin{pgfscope}%
\pgfsys@transformshift{5.444748in}{5.453287in}%
\pgfsys@useobject{currentmarker}{}%
\end{pgfscope}%
\begin{pgfscope}%
\pgfsys@transformshift{3.442235in}{3.630797in}%
\pgfsys@useobject{currentmarker}{}%
\end{pgfscope}%
\begin{pgfscope}%
\pgfsys@transformshift{1.615806in}{4.731700in}%
\pgfsys@useobject{currentmarker}{}%
\end{pgfscope}%
\begin{pgfscope}%
\pgfsys@transformshift{3.854399in}{3.314784in}%
\pgfsys@useobject{currentmarker}{}%
\end{pgfscope}%
\begin{pgfscope}%
\pgfsys@transformshift{2.055857in}{1.570616in}%
\pgfsys@useobject{currentmarker}{}%
\end{pgfscope}%
\begin{pgfscope}%
\pgfsys@transformshift{1.436347in}{3.105986in}%
\pgfsys@useobject{currentmarker}{}%
\end{pgfscope}%
\begin{pgfscope}%
\pgfsys@transformshift{4.726454in}{3.149538in}%
\pgfsys@useobject{currentmarker}{}%
\end{pgfscope}%
\begin{pgfscope}%
\pgfsys@transformshift{4.935666in}{4.874458in}%
\pgfsys@useobject{currentmarker}{}%
\end{pgfscope}%
\begin{pgfscope}%
\pgfsys@transformshift{4.924599in}{2.981236in}%
\pgfsys@useobject{currentmarker}{}%
\end{pgfscope}%
\begin{pgfscope}%
\pgfsys@transformshift{5.646671in}{0.236834in}%
\pgfsys@useobject{currentmarker}{}%
\end{pgfscope}%
\begin{pgfscope}%
\pgfsys@transformshift{2.508063in}{5.385651in}%
\pgfsys@useobject{currentmarker}{}%
\end{pgfscope}%
\begin{pgfscope}%
\pgfsys@transformshift{4.318884in}{4.682130in}%
\pgfsys@useobject{currentmarker}{}%
\end{pgfscope}%
\begin{pgfscope}%
\pgfsys@transformshift{3.453664in}{5.476330in}%
\pgfsys@useobject{currentmarker}{}%
\end{pgfscope}%
\begin{pgfscope}%
\pgfsys@transformshift{6.008581in}{1.218234in}%
\pgfsys@useobject{currentmarker}{}%
\end{pgfscope}%
\begin{pgfscope}%
\pgfsys@transformshift{2.172534in}{1.988399in}%
\pgfsys@useobject{currentmarker}{}%
\end{pgfscope}%
\begin{pgfscope}%
\pgfsys@transformshift{4.574985in}{2.093693in}%
\pgfsys@useobject{currentmarker}{}%
\end{pgfscope}%
\begin{pgfscope}%
\pgfsys@transformshift{2.821949in}{0.915201in}%
\pgfsys@useobject{currentmarker}{}%
\end{pgfscope}%
\begin{pgfscope}%
\pgfsys@transformshift{2.291114in}{1.022960in}%
\pgfsys@useobject{currentmarker}{}%
\end{pgfscope}%
\begin{pgfscope}%
\pgfsys@transformshift{3.249111in}{3.451440in}%
\pgfsys@useobject{currentmarker}{}%
\end{pgfscope}%
\begin{pgfscope}%
\pgfsys@transformshift{3.418379in}{4.944464in}%
\pgfsys@useobject{currentmarker}{}%
\end{pgfscope}%
\begin{pgfscope}%
\pgfsys@transformshift{6.721855in}{4.742230in}%
\pgfsys@useobject{currentmarker}{}%
\end{pgfscope}%
\begin{pgfscope}%
\pgfsys@transformshift{3.195830in}{2.139203in}%
\pgfsys@useobject{currentmarker}{}%
\end{pgfscope}%
\begin{pgfscope}%
\pgfsys@transformshift{4.613722in}{3.598510in}%
\pgfsys@useobject{currentmarker}{}%
\end{pgfscope}%
\begin{pgfscope}%
\pgfsys@transformshift{4.571450in}{1.755305in}%
\pgfsys@useobject{currentmarker}{}%
\end{pgfscope}%
\begin{pgfscope}%
\pgfsys@transformshift{2.687764in}{4.679773in}%
\pgfsys@useobject{currentmarker}{}%
\end{pgfscope}%
\begin{pgfscope}%
\pgfsys@transformshift{4.919162in}{1.245477in}%
\pgfsys@useobject{currentmarker}{}%
\end{pgfscope}%
\begin{pgfscope}%
\pgfsys@transformshift{5.478505in}{3.023083in}%
\pgfsys@useobject{currentmarker}{}%
\end{pgfscope}%
\begin{pgfscope}%
\pgfsys@transformshift{5.286372in}{2.644489in}%
\pgfsys@useobject{currentmarker}{}%
\end{pgfscope}%
\begin{pgfscope}%
\pgfsys@transformshift{1.932215in}{3.910854in}%
\pgfsys@useobject{currentmarker}{}%
\end{pgfscope}%
\begin{pgfscope}%
\pgfsys@transformshift{5.826780in}{3.067734in}%
\pgfsys@useobject{currentmarker}{}%
\end{pgfscope}%
\begin{pgfscope}%
\pgfsys@transformshift{6.173009in}{5.541503in}%
\pgfsys@useobject{currentmarker}{}%
\end{pgfscope}%
\begin{pgfscope}%
\pgfsys@transformshift{4.563118in}{4.624819in}%
\pgfsys@useobject{currentmarker}{}%
\end{pgfscope}%
\begin{pgfscope}%
\pgfsys@transformshift{6.081851in}{5.283892in}%
\pgfsys@useobject{currentmarker}{}%
\end{pgfscope}%
\begin{pgfscope}%
\pgfsys@transformshift{3.816489in}{4.911076in}%
\pgfsys@useobject{currentmarker}{}%
\end{pgfscope}%
\begin{pgfscope}%
\pgfsys@transformshift{2.570919in}{0.951181in}%
\pgfsys@useobject{currentmarker}{}%
\end{pgfscope}%
\begin{pgfscope}%
\pgfsys@transformshift{6.077710in}{2.312357in}%
\pgfsys@useobject{currentmarker}{}%
\end{pgfscope}%
\begin{pgfscope}%
\pgfsys@transformshift{3.339419in}{1.969108in}%
\pgfsys@useobject{currentmarker}{}%
\end{pgfscope}%
\begin{pgfscope}%
\pgfsys@transformshift{6.293994in}{0.283070in}%
\pgfsys@useobject{currentmarker}{}%
\end{pgfscope}%
\begin{pgfscope}%
\pgfsys@transformshift{3.755276in}{2.885846in}%
\pgfsys@useobject{currentmarker}{}%
\end{pgfscope}%
\begin{pgfscope}%
\pgfsys@transformshift{6.775032in}{5.131919in}%
\pgfsys@useobject{currentmarker}{}%
\end{pgfscope}%
\begin{pgfscope}%
\pgfsys@transformshift{5.354899in}{1.732918in}%
\pgfsys@useobject{currentmarker}{}%
\end{pgfscope}%
\begin{pgfscope}%
\pgfsys@transformshift{5.345581in}{2.511129in}%
\pgfsys@useobject{currentmarker}{}%
\end{pgfscope}%
\begin{pgfscope}%
\pgfsys@transformshift{2.387974in}{3.463126in}%
\pgfsys@useobject{currentmarker}{}%
\end{pgfscope}%
\begin{pgfscope}%
\pgfsys@transformshift{6.222673in}{3.024721in}%
\pgfsys@useobject{currentmarker}{}%
\end{pgfscope}%
\begin{pgfscope}%
\pgfsys@transformshift{5.708974in}{3.028273in}%
\pgfsys@useobject{currentmarker}{}%
\end{pgfscope}%
\begin{pgfscope}%
\pgfsys@transformshift{2.121233in}{4.570489in}%
\pgfsys@useobject{currentmarker}{}%
\end{pgfscope}%
\begin{pgfscope}%
\pgfsys@transformshift{6.588359in}{2.987668in}%
\pgfsys@useobject{currentmarker}{}%
\end{pgfscope}%
\begin{pgfscope}%
\pgfsys@transformshift{0.660766in}{3.380144in}%
\pgfsys@useobject{currentmarker}{}%
\end{pgfscope}%
\begin{pgfscope}%
\pgfsys@transformshift{0.638422in}{0.377692in}%
\pgfsys@useobject{currentmarker}{}%
\end{pgfscope}%
\begin{pgfscope}%
\pgfsys@transformshift{4.883686in}{2.024349in}%
\pgfsys@useobject{currentmarker}{}%
\end{pgfscope}%
\begin{pgfscope}%
\pgfsys@transformshift{2.740257in}{3.626682in}%
\pgfsys@useobject{currentmarker}{}%
\end{pgfscope}%
\begin{pgfscope}%
\pgfsys@transformshift{0.959435in}{5.629592in}%
\pgfsys@useobject{currentmarker}{}%
\end{pgfscope}%
\begin{pgfscope}%
\pgfsys@transformshift{1.541526in}{4.402371in}%
\pgfsys@useobject{currentmarker}{}%
\end{pgfscope}%
\begin{pgfscope}%
\pgfsys@transformshift{6.276516in}{3.780366in}%
\pgfsys@useobject{currentmarker}{}%
\end{pgfscope}%
\begin{pgfscope}%
\pgfsys@transformshift{4.656231in}{5.099656in}%
\pgfsys@useobject{currentmarker}{}%
\end{pgfscope}%
\begin{pgfscope}%
\pgfsys@transformshift{1.403655in}{3.799710in}%
\pgfsys@useobject{currentmarker}{}%
\end{pgfscope}%
\begin{pgfscope}%
\pgfsys@transformshift{3.077661in}{4.281679in}%
\pgfsys@useobject{currentmarker}{}%
\end{pgfscope}%
\begin{pgfscope}%
\pgfsys@transformshift{3.978488in}{3.094930in}%
\pgfsys@useobject{currentmarker}{}%
\end{pgfscope}%
\begin{pgfscope}%
\pgfsys@transformshift{6.308239in}{1.314485in}%
\pgfsys@useobject{currentmarker}{}%
\end{pgfscope}%
\begin{pgfscope}%
\pgfsys@transformshift{5.665058in}{0.916599in}%
\pgfsys@useobject{currentmarker}{}%
\end{pgfscope}%
\begin{pgfscope}%
\pgfsys@transformshift{4.307675in}{1.537587in}%
\pgfsys@useobject{currentmarker}{}%
\end{pgfscope}%
\begin{pgfscope}%
\pgfsys@transformshift{1.783796in}{5.632632in}%
\pgfsys@useobject{currentmarker}{}%
\end{pgfscope}%
\begin{pgfscope}%
\pgfsys@transformshift{4.003026in}{4.943363in}%
\pgfsys@useobject{currentmarker}{}%
\end{pgfscope}%
\begin{pgfscope}%
\pgfsys@transformshift{5.169143in}{4.312399in}%
\pgfsys@useobject{currentmarker}{}%
\end{pgfscope}%
\begin{pgfscope}%
\pgfsys@transformshift{4.726654in}{2.312508in}%
\pgfsys@useobject{currentmarker}{}%
\end{pgfscope}%
\begin{pgfscope}%
\pgfsys@transformshift{2.266604in}{2.484720in}%
\pgfsys@useobject{currentmarker}{}%
\end{pgfscope}%
\begin{pgfscope}%
\pgfsys@transformshift{4.237641in}{4.877722in}%
\pgfsys@useobject{currentmarker}{}%
\end{pgfscope}%
\begin{pgfscope}%
\pgfsys@transformshift{6.127742in}{2.376750in}%
\pgfsys@useobject{currentmarker}{}%
\end{pgfscope}%
\begin{pgfscope}%
\pgfsys@transformshift{0.685222in}{2.435419in}%
\pgfsys@useobject{currentmarker}{}%
\end{pgfscope}%
\begin{pgfscope}%
\pgfsys@transformshift{3.995462in}{2.932399in}%
\pgfsys@useobject{currentmarker}{}%
\end{pgfscope}%
\begin{pgfscope}%
\pgfsys@transformshift{1.199929in}{4.919594in}%
\pgfsys@useobject{currentmarker}{}%
\end{pgfscope}%
\begin{pgfscope}%
\pgfsys@transformshift{4.786246in}{5.141406in}%
\pgfsys@useobject{currentmarker}{}%
\end{pgfscope}%
\begin{pgfscope}%
\pgfsys@transformshift{4.097274in}{4.381858in}%
\pgfsys@useobject{currentmarker}{}%
\end{pgfscope}%
\begin{pgfscope}%
\pgfsys@transformshift{1.456272in}{2.142124in}%
\pgfsys@useobject{currentmarker}{}%
\end{pgfscope}%
\begin{pgfscope}%
\pgfsys@transformshift{0.983666in}{3.334171in}%
\pgfsys@useobject{currentmarker}{}%
\end{pgfscope}%
\begin{pgfscope}%
\pgfsys@transformshift{4.540991in}{1.982214in}%
\pgfsys@useobject{currentmarker}{}%
\end{pgfscope}%
\begin{pgfscope}%
\pgfsys@transformshift{2.887081in}{2.480684in}%
\pgfsys@useobject{currentmarker}{}%
\end{pgfscope}%
\begin{pgfscope}%
\pgfsys@transformshift{1.974107in}{4.422461in}%
\pgfsys@useobject{currentmarker}{}%
\end{pgfscope}%
\begin{pgfscope}%
\pgfsys@transformshift{3.742329in}{1.808660in}%
\pgfsys@useobject{currentmarker}{}%
\end{pgfscope}%
\begin{pgfscope}%
\pgfsys@transformshift{5.625123in}{5.136882in}%
\pgfsys@useobject{currentmarker}{}%
\end{pgfscope}%
\begin{pgfscope}%
\pgfsys@transformshift{4.970478in}{2.101681in}%
\pgfsys@useobject{currentmarker}{}%
\end{pgfscope}%
\begin{pgfscope}%
\pgfsys@transformshift{4.449601in}{2.118021in}%
\pgfsys@useobject{currentmarker}{}%
\end{pgfscope}%
\begin{pgfscope}%
\pgfsys@transformshift{4.545370in}{1.625933in}%
\pgfsys@useobject{currentmarker}{}%
\end{pgfscope}%
\begin{pgfscope}%
\pgfsys@transformshift{1.701117in}{3.302885in}%
\pgfsys@useobject{currentmarker}{}%
\end{pgfscope}%
\begin{pgfscope}%
\pgfsys@transformshift{3.020819in}{0.698542in}%
\pgfsys@useobject{currentmarker}{}%
\end{pgfscope}%
\begin{pgfscope}%
\pgfsys@transformshift{2.882987in}{3.637747in}%
\pgfsys@useobject{currentmarker}{}%
\end{pgfscope}%
\begin{pgfscope}%
\pgfsys@transformshift{6.173915in}{5.168388in}%
\pgfsys@useobject{currentmarker}{}%
\end{pgfscope}%
\begin{pgfscope}%
\pgfsys@transformshift{6.016951in}{5.527557in}%
\pgfsys@useobject{currentmarker}{}%
\end{pgfscope}%
\begin{pgfscope}%
\pgfsys@transformshift{2.444148in}{5.354277in}%
\pgfsys@useobject{currentmarker}{}%
\end{pgfscope}%
\begin{pgfscope}%
\pgfsys@transformshift{3.541478in}{2.148269in}%
\pgfsys@useobject{currentmarker}{}%
\end{pgfscope}%
\begin{pgfscope}%
\pgfsys@transformshift{2.937770in}{5.471810in}%
\pgfsys@useobject{currentmarker}{}%
\end{pgfscope}%
\begin{pgfscope}%
\pgfsys@transformshift{6.844005in}{5.483905in}%
\pgfsys@useobject{currentmarker}{}%
\end{pgfscope}%
\begin{pgfscope}%
\pgfsys@transformshift{5.545303in}{0.229241in}%
\pgfsys@useobject{currentmarker}{}%
\end{pgfscope}%
\begin{pgfscope}%
\pgfsys@transformshift{2.322194in}{2.891862in}%
\pgfsys@useobject{currentmarker}{}%
\end{pgfscope}%
\begin{pgfscope}%
\pgfsys@transformshift{2.813031in}{1.667116in}%
\pgfsys@useobject{currentmarker}{}%
\end{pgfscope}%
\begin{pgfscope}%
\pgfsys@transformshift{6.512908in}{2.472316in}%
\pgfsys@useobject{currentmarker}{}%
\end{pgfscope}%
\begin{pgfscope}%
\pgfsys@transformshift{2.649575in}{3.713479in}%
\pgfsys@useobject{currentmarker}{}%
\end{pgfscope}%
\begin{pgfscope}%
\pgfsys@transformshift{2.685982in}{5.150939in}%
\pgfsys@useobject{currentmarker}{}%
\end{pgfscope}%
\begin{pgfscope}%
\pgfsys@transformshift{4.611924in}{2.043274in}%
\pgfsys@useobject{currentmarker}{}%
\end{pgfscope}%
\begin{pgfscope}%
\pgfsys@transformshift{1.008702in}{0.812180in}%
\pgfsys@useobject{currentmarker}{}%
\end{pgfscope}%
\begin{pgfscope}%
\pgfsys@transformshift{4.631452in}{4.610566in}%
\pgfsys@useobject{currentmarker}{}%
\end{pgfscope}%
\begin{pgfscope}%
\pgfsys@transformshift{3.566628in}{0.476345in}%
\pgfsys@useobject{currentmarker}{}%
\end{pgfscope}%
\begin{pgfscope}%
\pgfsys@transformshift{3.859542in}{1.606297in}%
\pgfsys@useobject{currentmarker}{}%
\end{pgfscope}%
\begin{pgfscope}%
\pgfsys@transformshift{3.126543in}{3.326074in}%
\pgfsys@useobject{currentmarker}{}%
\end{pgfscope}%
\begin{pgfscope}%
\pgfsys@transformshift{2.071983in}{0.819441in}%
\pgfsys@useobject{currentmarker}{}%
\end{pgfscope}%
\begin{pgfscope}%
\pgfsys@transformshift{2.557020in}{3.165861in}%
\pgfsys@useobject{currentmarker}{}%
\end{pgfscope}%
\begin{pgfscope}%
\pgfsys@transformshift{4.660763in}{3.961684in}%
\pgfsys@useobject{currentmarker}{}%
\end{pgfscope}%
\begin{pgfscope}%
\pgfsys@transformshift{6.095763in}{0.327096in}%
\pgfsys@useobject{currentmarker}{}%
\end{pgfscope}%
\begin{pgfscope}%
\pgfsys@transformshift{4.777143in}{5.321530in}%
\pgfsys@useobject{currentmarker}{}%
\end{pgfscope}%
\begin{pgfscope}%
\pgfsys@transformshift{5.019800in}{2.678550in}%
\pgfsys@useobject{currentmarker}{}%
\end{pgfscope}%
\begin{pgfscope}%
\pgfsys@transformshift{1.096184in}{4.373993in}%
\pgfsys@useobject{currentmarker}{}%
\end{pgfscope}%
\begin{pgfscope}%
\pgfsys@transformshift{4.817608in}{1.764063in}%
\pgfsys@useobject{currentmarker}{}%
\end{pgfscope}%
\begin{pgfscope}%
\pgfsys@transformshift{2.388446in}{2.056302in}%
\pgfsys@useobject{currentmarker}{}%
\end{pgfscope}%
\begin{pgfscope}%
\pgfsys@transformshift{6.638673in}{1.337319in}%
\pgfsys@useobject{currentmarker}{}%
\end{pgfscope}%
\begin{pgfscope}%
\pgfsys@transformshift{3.917515in}{4.306589in}%
\pgfsys@useobject{currentmarker}{}%
\end{pgfscope}%
\begin{pgfscope}%
\pgfsys@transformshift{5.262884in}{2.769688in}%
\pgfsys@useobject{currentmarker}{}%
\end{pgfscope}%
\begin{pgfscope}%
\pgfsys@transformshift{2.540800in}{2.195987in}%
\pgfsys@useobject{currentmarker}{}%
\end{pgfscope}%
\begin{pgfscope}%
\pgfsys@transformshift{2.278667in}{4.273157in}%
\pgfsys@useobject{currentmarker}{}%
\end{pgfscope}%
\begin{pgfscope}%
\pgfsys@transformshift{1.600388in}{5.157514in}%
\pgfsys@useobject{currentmarker}{}%
\end{pgfscope}%
\begin{pgfscope}%
\pgfsys@transformshift{1.256587in}{4.185291in}%
\pgfsys@useobject{currentmarker}{}%
\end{pgfscope}%
\begin{pgfscope}%
\pgfsys@transformshift{1.675103in}{2.367521in}%
\pgfsys@useobject{currentmarker}{}%
\end{pgfscope}%
\begin{pgfscope}%
\pgfsys@transformshift{1.140633in}{1.975087in}%
\pgfsys@useobject{currentmarker}{}%
\end{pgfscope}%
\begin{pgfscope}%
\pgfsys@transformshift{6.150431in}{1.526901in}%
\pgfsys@useobject{currentmarker}{}%
\end{pgfscope}%
\begin{pgfscope}%
\pgfsys@transformshift{2.754860in}{3.534710in}%
\pgfsys@useobject{currentmarker}{}%
\end{pgfscope}%
\begin{pgfscope}%
\pgfsys@transformshift{5.688736in}{3.249390in}%
\pgfsys@useobject{currentmarker}{}%
\end{pgfscope}%
\begin{pgfscope}%
\pgfsys@transformshift{3.462161in}{1.070390in}%
\pgfsys@useobject{currentmarker}{}%
\end{pgfscope}%
\begin{pgfscope}%
\pgfsys@transformshift{4.573148in}{2.042096in}%
\pgfsys@useobject{currentmarker}{}%
\end{pgfscope}%
\begin{pgfscope}%
\pgfsys@transformshift{1.477976in}{1.126700in}%
\pgfsys@useobject{currentmarker}{}%
\end{pgfscope}%
\begin{pgfscope}%
\pgfsys@transformshift{2.059752in}{5.603854in}%
\pgfsys@useobject{currentmarker}{}%
\end{pgfscope}%
\begin{pgfscope}%
\pgfsys@transformshift{3.761870in}{2.737511in}%
\pgfsys@useobject{currentmarker}{}%
\end{pgfscope}%
\begin{pgfscope}%
\pgfsys@transformshift{2.387502in}{2.618844in}%
\pgfsys@useobject{currentmarker}{}%
\end{pgfscope}%
\begin{pgfscope}%
\pgfsys@transformshift{1.177883in}{1.628458in}%
\pgfsys@useobject{currentmarker}{}%
\end{pgfscope}%
\begin{pgfscope}%
\pgfsys@transformshift{1.514227in}{1.497437in}%
\pgfsys@useobject{currentmarker}{}%
\end{pgfscope}%
\begin{pgfscope}%
\pgfsys@transformshift{1.995204in}{3.074388in}%
\pgfsys@useobject{currentmarker}{}%
\end{pgfscope}%
\begin{pgfscope}%
\pgfsys@transformshift{2.698425in}{5.229061in}%
\pgfsys@useobject{currentmarker}{}%
\end{pgfscope}%
\begin{pgfscope}%
\pgfsys@transformshift{2.848675in}{1.640935in}%
\pgfsys@useobject{currentmarker}{}%
\end{pgfscope}%
\begin{pgfscope}%
\pgfsys@transformshift{4.312361in}{4.661572in}%
\pgfsys@useobject{currentmarker}{}%
\end{pgfscope}%
\begin{pgfscope}%
\pgfsys@transformshift{6.711979in}{4.754123in}%
\pgfsys@useobject{currentmarker}{}%
\end{pgfscope}%
\begin{pgfscope}%
\pgfsys@transformshift{4.555861in}{1.916473in}%
\pgfsys@useobject{currentmarker}{}%
\end{pgfscope}%
\begin{pgfscope}%
\pgfsys@transformshift{4.456666in}{4.816556in}%
\pgfsys@useobject{currentmarker}{}%
\end{pgfscope}%
\begin{pgfscope}%
\pgfsys@transformshift{1.955766in}{4.913128in}%
\pgfsys@useobject{currentmarker}{}%
\end{pgfscope}%
\begin{pgfscope}%
\pgfsys@transformshift{4.702284in}{0.728509in}%
\pgfsys@useobject{currentmarker}{}%
\end{pgfscope}%
\begin{pgfscope}%
\pgfsys@transformshift{5.486862in}{1.687828in}%
\pgfsys@useobject{currentmarker}{}%
\end{pgfscope}%
\begin{pgfscope}%
\pgfsys@transformshift{6.662081in}{3.638517in}%
\pgfsys@useobject{currentmarker}{}%
\end{pgfscope}%
\begin{pgfscope}%
\pgfsys@transformshift{2.027746in}{5.261369in}%
\pgfsys@useobject{currentmarker}{}%
\end{pgfscope}%
\begin{pgfscope}%
\pgfsys@transformshift{3.231916in}{0.236334in}%
\pgfsys@useobject{currentmarker}{}%
\end{pgfscope}%
\begin{pgfscope}%
\pgfsys@transformshift{6.714576in}{3.312566in}%
\pgfsys@useobject{currentmarker}{}%
\end{pgfscope}%
\begin{pgfscope}%
\pgfsys@transformshift{2.461026in}{2.860383in}%
\pgfsys@useobject{currentmarker}{}%
\end{pgfscope}%
\begin{pgfscope}%
\pgfsys@transformshift{4.920990in}{3.284783in}%
\pgfsys@useobject{currentmarker}{}%
\end{pgfscope}%
\begin{pgfscope}%
\pgfsys@transformshift{0.633466in}{3.601195in}%
\pgfsys@useobject{currentmarker}{}%
\end{pgfscope}%
\begin{pgfscope}%
\pgfsys@transformshift{6.410720in}{4.771303in}%
\pgfsys@useobject{currentmarker}{}%
\end{pgfscope}%
\begin{pgfscope}%
\pgfsys@transformshift{0.737807in}{5.192324in}%
\pgfsys@useobject{currentmarker}{}%
\end{pgfscope}%
\begin{pgfscope}%
\pgfsys@transformshift{3.030773in}{1.004290in}%
\pgfsys@useobject{currentmarker}{}%
\end{pgfscope}%
\begin{pgfscope}%
\pgfsys@transformshift{5.721762in}{3.184300in}%
\pgfsys@useobject{currentmarker}{}%
\end{pgfscope}%
\begin{pgfscope}%
\pgfsys@transformshift{6.672696in}{1.345201in}%
\pgfsys@useobject{currentmarker}{}%
\end{pgfscope}%
\begin{pgfscope}%
\pgfsys@transformshift{4.262420in}{2.709797in}%
\pgfsys@useobject{currentmarker}{}%
\end{pgfscope}%
\begin{pgfscope}%
\pgfsys@transformshift{0.910133in}{1.575668in}%
\pgfsys@useobject{currentmarker}{}%
\end{pgfscope}%
\begin{pgfscope}%
\pgfsys@transformshift{3.702260in}{3.494981in}%
\pgfsys@useobject{currentmarker}{}%
\end{pgfscope}%
\begin{pgfscope}%
\pgfsys@transformshift{3.848048in}{4.539957in}%
\pgfsys@useobject{currentmarker}{}%
\end{pgfscope}%
\begin{pgfscope}%
\pgfsys@transformshift{0.823245in}{1.736091in}%
\pgfsys@useobject{currentmarker}{}%
\end{pgfscope}%
\begin{pgfscope}%
\pgfsys@transformshift{1.944397in}{5.536021in}%
\pgfsys@useobject{currentmarker}{}%
\end{pgfscope}%
\begin{pgfscope}%
\pgfsys@transformshift{5.748313in}{2.938554in}%
\pgfsys@useobject{currentmarker}{}%
\end{pgfscope}%
\begin{pgfscope}%
\pgfsys@transformshift{0.850308in}{2.263331in}%
\pgfsys@useobject{currentmarker}{}%
\end{pgfscope}%
\begin{pgfscope}%
\pgfsys@transformshift{5.533335in}{2.051328in}%
\pgfsys@useobject{currentmarker}{}%
\end{pgfscope}%
\begin{pgfscope}%
\pgfsys@transformshift{4.920908in}{1.178994in}%
\pgfsys@useobject{currentmarker}{}%
\end{pgfscope}%
\begin{pgfscope}%
\pgfsys@transformshift{3.078704in}{4.308053in}%
\pgfsys@useobject{currentmarker}{}%
\end{pgfscope}%
\begin{pgfscope}%
\pgfsys@transformshift{2.622395in}{4.775882in}%
\pgfsys@useobject{currentmarker}{}%
\end{pgfscope}%
\begin{pgfscope}%
\pgfsys@transformshift{3.070403in}{0.511672in}%
\pgfsys@useobject{currentmarker}{}%
\end{pgfscope}%
\begin{pgfscope}%
\pgfsys@transformshift{2.010159in}{4.587327in}%
\pgfsys@useobject{currentmarker}{}%
\end{pgfscope}%
\begin{pgfscope}%
\pgfsys@transformshift{4.615435in}{1.512341in}%
\pgfsys@useobject{currentmarker}{}%
\end{pgfscope}%
\begin{pgfscope}%
\pgfsys@transformshift{3.283848in}{1.923445in}%
\pgfsys@useobject{currentmarker}{}%
\end{pgfscope}%
\begin{pgfscope}%
\pgfsys@transformshift{3.516931in}{2.372846in}%
\pgfsys@useobject{currentmarker}{}%
\end{pgfscope}%
\begin{pgfscope}%
\pgfsys@transformshift{2.730136in}{4.537345in}%
\pgfsys@useobject{currentmarker}{}%
\end{pgfscope}%
\begin{pgfscope}%
\pgfsys@transformshift{4.197092in}{1.232488in}%
\pgfsys@useobject{currentmarker}{}%
\end{pgfscope}%
\begin{pgfscope}%
\pgfsys@transformshift{1.084735in}{5.441939in}%
\pgfsys@useobject{currentmarker}{}%
\end{pgfscope}%
\begin{pgfscope}%
\pgfsys@transformshift{5.055315in}{4.079259in}%
\pgfsys@useobject{currentmarker}{}%
\end{pgfscope}%
\begin{pgfscope}%
\pgfsys@transformshift{4.998613in}{2.542947in}%
\pgfsys@useobject{currentmarker}{}%
\end{pgfscope}%
\begin{pgfscope}%
\pgfsys@transformshift{3.753514in}{3.592069in}%
\pgfsys@useobject{currentmarker}{}%
\end{pgfscope}%
\begin{pgfscope}%
\pgfsys@transformshift{5.009085in}{1.180972in}%
\pgfsys@useobject{currentmarker}{}%
\end{pgfscope}%
\begin{pgfscope}%
\pgfsys@transformshift{3.916301in}{0.346932in}%
\pgfsys@useobject{currentmarker}{}%
\end{pgfscope}%
\begin{pgfscope}%
\pgfsys@transformshift{3.295930in}{0.419960in}%
\pgfsys@useobject{currentmarker}{}%
\end{pgfscope}%
\begin{pgfscope}%
\pgfsys@transformshift{2.941804in}{4.369959in}%
\pgfsys@useobject{currentmarker}{}%
\end{pgfscope}%
\begin{pgfscope}%
\pgfsys@transformshift{4.725180in}{3.528098in}%
\pgfsys@useobject{currentmarker}{}%
\end{pgfscope}%
\begin{pgfscope}%
\pgfsys@transformshift{2.788343in}{2.740159in}%
\pgfsys@useobject{currentmarker}{}%
\end{pgfscope}%
\begin{pgfscope}%
\pgfsys@transformshift{1.878187in}{1.349408in}%
\pgfsys@useobject{currentmarker}{}%
\end{pgfscope}%
\begin{pgfscope}%
\pgfsys@transformshift{1.664014in}{1.692202in}%
\pgfsys@useobject{currentmarker}{}%
\end{pgfscope}%
\begin{pgfscope}%
\pgfsys@transformshift{6.188220in}{4.797263in}%
\pgfsys@useobject{currentmarker}{}%
\end{pgfscope}%
\begin{pgfscope}%
\pgfsys@transformshift{6.450629in}{4.934804in}%
\pgfsys@useobject{currentmarker}{}%
\end{pgfscope}%
\begin{pgfscope}%
\pgfsys@transformshift{1.258554in}{4.447746in}%
\pgfsys@useobject{currentmarker}{}%
\end{pgfscope}%
\begin{pgfscope}%
\pgfsys@transformshift{3.528620in}{4.236590in}%
\pgfsys@useobject{currentmarker}{}%
\end{pgfscope}%
\begin{pgfscope}%
\pgfsys@transformshift{5.959390in}{4.101922in}%
\pgfsys@useobject{currentmarker}{}%
\end{pgfscope}%
\begin{pgfscope}%
\pgfsys@transformshift{5.097255in}{5.072753in}%
\pgfsys@useobject{currentmarker}{}%
\end{pgfscope}%
\begin{pgfscope}%
\pgfsys@transformshift{2.059371in}{1.976888in}%
\pgfsys@useobject{currentmarker}{}%
\end{pgfscope}%
\begin{pgfscope}%
\pgfsys@transformshift{6.460961in}{0.415587in}%
\pgfsys@useobject{currentmarker}{}%
\end{pgfscope}%
\begin{pgfscope}%
\pgfsys@transformshift{3.758362in}{5.370915in}%
\pgfsys@useobject{currentmarker}{}%
\end{pgfscope}%
\begin{pgfscope}%
\pgfsys@transformshift{3.653366in}{1.443312in}%
\pgfsys@useobject{currentmarker}{}%
\end{pgfscope}%
\begin{pgfscope}%
\pgfsys@transformshift{6.462242in}{4.811799in}%
\pgfsys@useobject{currentmarker}{}%
\end{pgfscope}%
\begin{pgfscope}%
\pgfsys@transformshift{1.189272in}{5.328609in}%
\pgfsys@useobject{currentmarker}{}%
\end{pgfscope}%
\begin{pgfscope}%
\pgfsys@transformshift{0.783360in}{3.137870in}%
\pgfsys@useobject{currentmarker}{}%
\end{pgfscope}%
\begin{pgfscope}%
\pgfsys@transformshift{3.754773in}{4.837084in}%
\pgfsys@useobject{currentmarker}{}%
\end{pgfscope}%
\begin{pgfscope}%
\pgfsys@transformshift{5.897790in}{3.739800in}%
\pgfsys@useobject{currentmarker}{}%
\end{pgfscope}%
\begin{pgfscope}%
\pgfsys@transformshift{3.670624in}{1.499253in}%
\pgfsys@useobject{currentmarker}{}%
\end{pgfscope}%
\begin{pgfscope}%
\pgfsys@transformshift{2.928574in}{1.683345in}%
\pgfsys@useobject{currentmarker}{}%
\end{pgfscope}%
\begin{pgfscope}%
\pgfsys@transformshift{4.163611in}{2.548539in}%
\pgfsys@useobject{currentmarker}{}%
\end{pgfscope}%
\begin{pgfscope}%
\pgfsys@transformshift{2.994857in}{2.965365in}%
\pgfsys@useobject{currentmarker}{}%
\end{pgfscope}%
\begin{pgfscope}%
\pgfsys@transformshift{1.687544in}{3.910411in}%
\pgfsys@useobject{currentmarker}{}%
\end{pgfscope}%
\begin{pgfscope}%
\pgfsys@transformshift{4.119993in}{3.466925in}%
\pgfsys@useobject{currentmarker}{}%
\end{pgfscope}%
\begin{pgfscope}%
\pgfsys@transformshift{2.882721in}{1.955034in}%
\pgfsys@useobject{currentmarker}{}%
\end{pgfscope}%
\begin{pgfscope}%
\pgfsys@transformshift{3.997017in}{3.791505in}%
\pgfsys@useobject{currentmarker}{}%
\end{pgfscope}%
\begin{pgfscope}%
\pgfsys@transformshift{6.666083in}{4.144635in}%
\pgfsys@useobject{currentmarker}{}%
\end{pgfscope}%
\begin{pgfscope}%
\pgfsys@transformshift{1.630066in}{2.230214in}%
\pgfsys@useobject{currentmarker}{}%
\end{pgfscope}%
\begin{pgfscope}%
\pgfsys@transformshift{3.940720in}{0.753932in}%
\pgfsys@useobject{currentmarker}{}%
\end{pgfscope}%
\begin{pgfscope}%
\pgfsys@transformshift{0.671154in}{1.163744in}%
\pgfsys@useobject{currentmarker}{}%
\end{pgfscope}%
\begin{pgfscope}%
\pgfsys@transformshift{3.659008in}{2.878941in}%
\pgfsys@useobject{currentmarker}{}%
\end{pgfscope}%
\begin{pgfscope}%
\pgfsys@transformshift{3.796370in}{0.920318in}%
\pgfsys@useobject{currentmarker}{}%
\end{pgfscope}%
\begin{pgfscope}%
\pgfsys@transformshift{5.193854in}{3.388259in}%
\pgfsys@useobject{currentmarker}{}%
\end{pgfscope}%
\begin{pgfscope}%
\pgfsys@transformshift{1.879139in}{4.064168in}%
\pgfsys@useobject{currentmarker}{}%
\end{pgfscope}%
\begin{pgfscope}%
\pgfsys@transformshift{5.683591in}{4.909548in}%
\pgfsys@useobject{currentmarker}{}%
\end{pgfscope}%
\begin{pgfscope}%
\pgfsys@transformshift{6.267351in}{4.062387in}%
\pgfsys@useobject{currentmarker}{}%
\end{pgfscope}%
\begin{pgfscope}%
\pgfsys@transformshift{5.605500in}{0.456673in}%
\pgfsys@useobject{currentmarker}{}%
\end{pgfscope}%
\begin{pgfscope}%
\pgfsys@transformshift{5.238620in}{2.651835in}%
\pgfsys@useobject{currentmarker}{}%
\end{pgfscope}%
\begin{pgfscope}%
\pgfsys@transformshift{1.665680in}{2.554861in}%
\pgfsys@useobject{currentmarker}{}%
\end{pgfscope}%
\begin{pgfscope}%
\pgfsys@transformshift{3.973932in}{3.794709in}%
\pgfsys@useobject{currentmarker}{}%
\end{pgfscope}%
\begin{pgfscope}%
\pgfsys@transformshift{1.749542in}{0.304289in}%
\pgfsys@useobject{currentmarker}{}%
\end{pgfscope}%
\begin{pgfscope}%
\pgfsys@transformshift{5.210576in}{5.541205in}%
\pgfsys@useobject{currentmarker}{}%
\end{pgfscope}%
\begin{pgfscope}%
\pgfsys@transformshift{3.657671in}{3.972159in}%
\pgfsys@useobject{currentmarker}{}%
\end{pgfscope}%
\begin{pgfscope}%
\pgfsys@transformshift{4.365775in}{1.854875in}%
\pgfsys@useobject{currentmarker}{}%
\end{pgfscope}%
\begin{pgfscope}%
\pgfsys@transformshift{4.581518in}{3.919286in}%
\pgfsys@useobject{currentmarker}{}%
\end{pgfscope}%
\begin{pgfscope}%
\pgfsys@transformshift{6.044030in}{3.977402in}%
\pgfsys@useobject{currentmarker}{}%
\end{pgfscope}%
\begin{pgfscope}%
\pgfsys@transformshift{3.257253in}{4.176009in}%
\pgfsys@useobject{currentmarker}{}%
\end{pgfscope}%
\begin{pgfscope}%
\pgfsys@transformshift{0.728425in}{2.306781in}%
\pgfsys@useobject{currentmarker}{}%
\end{pgfscope}%
\begin{pgfscope}%
\pgfsys@transformshift{0.699212in}{0.644844in}%
\pgfsys@useobject{currentmarker}{}%
\end{pgfscope}%
\begin{pgfscope}%
\pgfsys@transformshift{2.824735in}{3.322079in}%
\pgfsys@useobject{currentmarker}{}%
\end{pgfscope}%
\begin{pgfscope}%
\pgfsys@transformshift{5.575909in}{4.168814in}%
\pgfsys@useobject{currentmarker}{}%
\end{pgfscope}%
\begin{pgfscope}%
\pgfsys@transformshift{2.694993in}{0.745092in}%
\pgfsys@useobject{currentmarker}{}%
\end{pgfscope}%
\begin{pgfscope}%
\pgfsys@transformshift{2.827246in}{2.093276in}%
\pgfsys@useobject{currentmarker}{}%
\end{pgfscope}%
\begin{pgfscope}%
\pgfsys@transformshift{4.860671in}{1.087019in}%
\pgfsys@useobject{currentmarker}{}%
\end{pgfscope}%
\begin{pgfscope}%
\pgfsys@transformshift{2.873589in}{1.278336in}%
\pgfsys@useobject{currentmarker}{}%
\end{pgfscope}%
\begin{pgfscope}%
\pgfsys@transformshift{1.536825in}{0.259268in}%
\pgfsys@useobject{currentmarker}{}%
\end{pgfscope}%
\begin{pgfscope}%
\pgfsys@transformshift{2.764352in}{0.681245in}%
\pgfsys@useobject{currentmarker}{}%
\end{pgfscope}%
\begin{pgfscope}%
\pgfsys@transformshift{6.472715in}{3.118065in}%
\pgfsys@useobject{currentmarker}{}%
\end{pgfscope}%
\begin{pgfscope}%
\pgfsys@transformshift{5.451502in}{0.558177in}%
\pgfsys@useobject{currentmarker}{}%
\end{pgfscope}%
\begin{pgfscope}%
\pgfsys@transformshift{4.521637in}{4.721217in}%
\pgfsys@useobject{currentmarker}{}%
\end{pgfscope}%
\begin{pgfscope}%
\pgfsys@transformshift{5.541218in}{1.493272in}%
\pgfsys@useobject{currentmarker}{}%
\end{pgfscope}%
\begin{pgfscope}%
\pgfsys@transformshift{4.253324in}{3.800007in}%
\pgfsys@useobject{currentmarker}{}%
\end{pgfscope}%
\begin{pgfscope}%
\pgfsys@transformshift{5.895094in}{2.687622in}%
\pgfsys@useobject{currentmarker}{}%
\end{pgfscope}%
\begin{pgfscope}%
\pgfsys@transformshift{6.081673in}{1.079201in}%
\pgfsys@useobject{currentmarker}{}%
\end{pgfscope}%
\begin{pgfscope}%
\pgfsys@transformshift{2.484298in}{1.217911in}%
\pgfsys@useobject{currentmarker}{}%
\end{pgfscope}%
\begin{pgfscope}%
\pgfsys@transformshift{5.425424in}{5.125628in}%
\pgfsys@useobject{currentmarker}{}%
\end{pgfscope}%
\begin{pgfscope}%
\pgfsys@transformshift{2.759490in}{2.627771in}%
\pgfsys@useobject{currentmarker}{}%
\end{pgfscope}%
\begin{pgfscope}%
\pgfsys@transformshift{1.602078in}{5.464114in}%
\pgfsys@useobject{currentmarker}{}%
\end{pgfscope}%
\begin{pgfscope}%
\pgfsys@transformshift{2.916437in}{4.833342in}%
\pgfsys@useobject{currentmarker}{}%
\end{pgfscope}%
\begin{pgfscope}%
\pgfsys@transformshift{5.260685in}{3.002060in}%
\pgfsys@useobject{currentmarker}{}%
\end{pgfscope}%
\begin{pgfscope}%
\pgfsys@transformshift{1.180553in}{0.604540in}%
\pgfsys@useobject{currentmarker}{}%
\end{pgfscope}%
\begin{pgfscope}%
\pgfsys@transformshift{5.687892in}{4.590760in}%
\pgfsys@useobject{currentmarker}{}%
\end{pgfscope}%
\begin{pgfscope}%
\pgfsys@transformshift{0.761716in}{5.353630in}%
\pgfsys@useobject{currentmarker}{}%
\end{pgfscope}%
\begin{pgfscope}%
\pgfsys@transformshift{3.249295in}{1.538378in}%
\pgfsys@useobject{currentmarker}{}%
\end{pgfscope}%
\begin{pgfscope}%
\pgfsys@transformshift{5.306802in}{3.756128in}%
\pgfsys@useobject{currentmarker}{}%
\end{pgfscope}%
\begin{pgfscope}%
\pgfsys@transformshift{5.910930in}{2.450328in}%
\pgfsys@useobject{currentmarker}{}%
\end{pgfscope}%
\begin{pgfscope}%
\pgfsys@transformshift{4.532737in}{4.248805in}%
\pgfsys@useobject{currentmarker}{}%
\end{pgfscope}%
\begin{pgfscope}%
\pgfsys@transformshift{6.161062in}{4.110244in}%
\pgfsys@useobject{currentmarker}{}%
\end{pgfscope}%
\begin{pgfscope}%
\pgfsys@transformshift{0.787277in}{2.876043in}%
\pgfsys@useobject{currentmarker}{}%
\end{pgfscope}%
\begin{pgfscope}%
\pgfsys@transformshift{6.267104in}{3.324397in}%
\pgfsys@useobject{currentmarker}{}%
\end{pgfscope}%
\begin{pgfscope}%
\pgfsys@transformshift{2.836078in}{5.207461in}%
\pgfsys@useobject{currentmarker}{}%
\end{pgfscope}%
\begin{pgfscope}%
\pgfsys@transformshift{1.628938in}{3.940938in}%
\pgfsys@useobject{currentmarker}{}%
\end{pgfscope}%
\begin{pgfscope}%
\pgfsys@transformshift{1.324555in}{2.256327in}%
\pgfsys@useobject{currentmarker}{}%
\end{pgfscope}%
\begin{pgfscope}%
\pgfsys@transformshift{6.296936in}{0.536252in}%
\pgfsys@useobject{currentmarker}{}%
\end{pgfscope}%
\begin{pgfscope}%
\pgfsys@transformshift{1.336231in}{0.735668in}%
\pgfsys@useobject{currentmarker}{}%
\end{pgfscope}%
\begin{pgfscope}%
\pgfsys@transformshift{6.633723in}{4.558885in}%
\pgfsys@useobject{currentmarker}{}%
\end{pgfscope}%
\begin{pgfscope}%
\pgfsys@transformshift{0.911925in}{5.373046in}%
\pgfsys@useobject{currentmarker}{}%
\end{pgfscope}%
\begin{pgfscope}%
\pgfsys@transformshift{5.570592in}{2.806925in}%
\pgfsys@useobject{currentmarker}{}%
\end{pgfscope}%
\begin{pgfscope}%
\pgfsys@transformshift{5.204466in}{4.173171in}%
\pgfsys@useobject{currentmarker}{}%
\end{pgfscope}%
\begin{pgfscope}%
\pgfsys@transformshift{2.411060in}{1.472078in}%
\pgfsys@useobject{currentmarker}{}%
\end{pgfscope}%
\begin{pgfscope}%
\pgfsys@transformshift{4.028813in}{2.990967in}%
\pgfsys@useobject{currentmarker}{}%
\end{pgfscope}%
\begin{pgfscope}%
\pgfsys@transformshift{2.181372in}{1.436556in}%
\pgfsys@useobject{currentmarker}{}%
\end{pgfscope}%
\begin{pgfscope}%
\pgfsys@transformshift{2.042943in}{1.212113in}%
\pgfsys@useobject{currentmarker}{}%
\end{pgfscope}%
\begin{pgfscope}%
\pgfsys@transformshift{3.470396in}{1.039351in}%
\pgfsys@useobject{currentmarker}{}%
\end{pgfscope}%
\begin{pgfscope}%
\pgfsys@transformshift{4.279956in}{1.609999in}%
\pgfsys@useobject{currentmarker}{}%
\end{pgfscope}%
\begin{pgfscope}%
\pgfsys@transformshift{5.419420in}{5.133857in}%
\pgfsys@useobject{currentmarker}{}%
\end{pgfscope}%
\begin{pgfscope}%
\pgfsys@transformshift{1.923650in}{2.529221in}%
\pgfsys@useobject{currentmarker}{}%
\end{pgfscope}%
\begin{pgfscope}%
\pgfsys@transformshift{6.254481in}{2.411960in}%
\pgfsys@useobject{currentmarker}{}%
\end{pgfscope}%
\begin{pgfscope}%
\pgfsys@transformshift{1.386280in}{0.596923in}%
\pgfsys@useobject{currentmarker}{}%
\end{pgfscope}%
\begin{pgfscope}%
\pgfsys@transformshift{1.367933in}{5.261716in}%
\pgfsys@useobject{currentmarker}{}%
\end{pgfscope}%
\begin{pgfscope}%
\pgfsys@transformshift{5.902000in}{1.579038in}%
\pgfsys@useobject{currentmarker}{}%
\end{pgfscope}%
\begin{pgfscope}%
\pgfsys@transformshift{3.800574in}{4.788179in}%
\pgfsys@useobject{currentmarker}{}%
\end{pgfscope}%
\begin{pgfscope}%
\pgfsys@transformshift{3.371569in}{3.312704in}%
\pgfsys@useobject{currentmarker}{}%
\end{pgfscope}%
\begin{pgfscope}%
\pgfsys@transformshift{6.073693in}{4.359344in}%
\pgfsys@useobject{currentmarker}{}%
\end{pgfscope}%
\begin{pgfscope}%
\pgfsys@transformshift{4.193810in}{4.467048in}%
\pgfsys@useobject{currentmarker}{}%
\end{pgfscope}%
\begin{pgfscope}%
\pgfsys@transformshift{2.599471in}{4.124679in}%
\pgfsys@useobject{currentmarker}{}%
\end{pgfscope}%
\begin{pgfscope}%
\pgfsys@transformshift{4.514961in}{4.936874in}%
\pgfsys@useobject{currentmarker}{}%
\end{pgfscope}%
\begin{pgfscope}%
\pgfsys@transformshift{3.272950in}{3.195707in}%
\pgfsys@useobject{currentmarker}{}%
\end{pgfscope}%
\begin{pgfscope}%
\pgfsys@transformshift{1.510721in}{2.354600in}%
\pgfsys@useobject{currentmarker}{}%
\end{pgfscope}%
\begin{pgfscope}%
\pgfsys@transformshift{3.264662in}{0.622165in}%
\pgfsys@useobject{currentmarker}{}%
\end{pgfscope}%
\begin{pgfscope}%
\pgfsys@transformshift{3.625515in}{3.140835in}%
\pgfsys@useobject{currentmarker}{}%
\end{pgfscope}%
\begin{pgfscope}%
\pgfsys@transformshift{2.563533in}{1.864619in}%
\pgfsys@useobject{currentmarker}{}%
\end{pgfscope}%
\begin{pgfscope}%
\pgfsys@transformshift{4.278420in}{1.568148in}%
\pgfsys@useobject{currentmarker}{}%
\end{pgfscope}%
\begin{pgfscope}%
\pgfsys@transformshift{6.133139in}{1.593437in}%
\pgfsys@useobject{currentmarker}{}%
\end{pgfscope}%
\begin{pgfscope}%
\pgfsys@transformshift{4.848546in}{3.568652in}%
\pgfsys@useobject{currentmarker}{}%
\end{pgfscope}%
\begin{pgfscope}%
\pgfsys@transformshift{6.723860in}{3.848760in}%
\pgfsys@useobject{currentmarker}{}%
\end{pgfscope}%
\begin{pgfscope}%
\pgfsys@transformshift{1.457826in}{4.531374in}%
\pgfsys@useobject{currentmarker}{}%
\end{pgfscope}%
\begin{pgfscope}%
\pgfsys@transformshift{6.297737in}{1.887891in}%
\pgfsys@useobject{currentmarker}{}%
\end{pgfscope}%
\begin{pgfscope}%
\pgfsys@transformshift{2.346430in}{0.797244in}%
\pgfsys@useobject{currentmarker}{}%
\end{pgfscope}%
\begin{pgfscope}%
\pgfsys@transformshift{6.648502in}{3.096227in}%
\pgfsys@useobject{currentmarker}{}%
\end{pgfscope}%
\begin{pgfscope}%
\pgfsys@transformshift{4.173255in}{3.519895in}%
\pgfsys@useobject{currentmarker}{}%
\end{pgfscope}%
\begin{pgfscope}%
\pgfsys@transformshift{2.165725in}{1.381293in}%
\pgfsys@useobject{currentmarker}{}%
\end{pgfscope}%
\begin{pgfscope}%
\pgfsys@transformshift{2.721770in}{1.510947in}%
\pgfsys@useobject{currentmarker}{}%
\end{pgfscope}%
\begin{pgfscope}%
\pgfsys@transformshift{3.548642in}{1.282580in}%
\pgfsys@useobject{currentmarker}{}%
\end{pgfscope}%
\begin{pgfscope}%
\pgfsys@transformshift{2.475107in}{4.994021in}%
\pgfsys@useobject{currentmarker}{}%
\end{pgfscope}%
\begin{pgfscope}%
\pgfsys@transformshift{1.470327in}{2.482215in}%
\pgfsys@useobject{currentmarker}{}%
\end{pgfscope}%
\begin{pgfscope}%
\pgfsys@transformshift{6.669199in}{5.298142in}%
\pgfsys@useobject{currentmarker}{}%
\end{pgfscope}%
\begin{pgfscope}%
\pgfsys@transformshift{0.827830in}{1.636520in}%
\pgfsys@useobject{currentmarker}{}%
\end{pgfscope}%
\begin{pgfscope}%
\pgfsys@transformshift{5.032274in}{4.734021in}%
\pgfsys@useobject{currentmarker}{}%
\end{pgfscope}%
\begin{pgfscope}%
\pgfsys@transformshift{4.672835in}{2.191141in}%
\pgfsys@useobject{currentmarker}{}%
\end{pgfscope}%
\begin{pgfscope}%
\pgfsys@transformshift{3.169599in}{2.938631in}%
\pgfsys@useobject{currentmarker}{}%
\end{pgfscope}%
\begin{pgfscope}%
\pgfsys@transformshift{2.953476in}{1.578938in}%
\pgfsys@useobject{currentmarker}{}%
\end{pgfscope}%
\begin{pgfscope}%
\pgfsys@transformshift{2.521981in}{4.095214in}%
\pgfsys@useobject{currentmarker}{}%
\end{pgfscope}%
\begin{pgfscope}%
\pgfsys@transformshift{2.205016in}{0.421058in}%
\pgfsys@useobject{currentmarker}{}%
\end{pgfscope}%
\begin{pgfscope}%
\pgfsys@transformshift{5.352028in}{4.064159in}%
\pgfsys@useobject{currentmarker}{}%
\end{pgfscope}%
\begin{pgfscope}%
\pgfsys@transformshift{4.636411in}{5.380365in}%
\pgfsys@useobject{currentmarker}{}%
\end{pgfscope}%
\begin{pgfscope}%
\pgfsys@transformshift{2.770858in}{1.945693in}%
\pgfsys@useobject{currentmarker}{}%
\end{pgfscope}%
\begin{pgfscope}%
\pgfsys@transformshift{0.782278in}{5.103194in}%
\pgfsys@useobject{currentmarker}{}%
\end{pgfscope}%
\begin{pgfscope}%
\pgfsys@transformshift{1.775147in}{2.990035in}%
\pgfsys@useobject{currentmarker}{}%
\end{pgfscope}%
\begin{pgfscope}%
\pgfsys@transformshift{2.358027in}{3.214200in}%
\pgfsys@useobject{currentmarker}{}%
\end{pgfscope}%
\begin{pgfscope}%
\pgfsys@transformshift{3.922695in}{4.973715in}%
\pgfsys@useobject{currentmarker}{}%
\end{pgfscope}%
\begin{pgfscope}%
\pgfsys@transformshift{3.978582in}{4.103172in}%
\pgfsys@useobject{currentmarker}{}%
\end{pgfscope}%
\begin{pgfscope}%
\pgfsys@transformshift{4.368506in}{5.467180in}%
\pgfsys@useobject{currentmarker}{}%
\end{pgfscope}%
\begin{pgfscope}%
\pgfsys@transformshift{3.193502in}{1.561359in}%
\pgfsys@useobject{currentmarker}{}%
\end{pgfscope}%
\begin{pgfscope}%
\pgfsys@transformshift{6.288622in}{3.614347in}%
\pgfsys@useobject{currentmarker}{}%
\end{pgfscope}%
\begin{pgfscope}%
\pgfsys@transformshift{4.084287in}{3.312912in}%
\pgfsys@useobject{currentmarker}{}%
\end{pgfscope}%
\begin{pgfscope}%
\pgfsys@transformshift{1.347858in}{3.946126in}%
\pgfsys@useobject{currentmarker}{}%
\end{pgfscope}%
\begin{pgfscope}%
\pgfsys@transformshift{4.769809in}{3.591916in}%
\pgfsys@useobject{currentmarker}{}%
\end{pgfscope}%
\begin{pgfscope}%
\pgfsys@transformshift{3.773972in}{0.287757in}%
\pgfsys@useobject{currentmarker}{}%
\end{pgfscope}%
\begin{pgfscope}%
\pgfsys@transformshift{2.176257in}{4.675706in}%
\pgfsys@useobject{currentmarker}{}%
\end{pgfscope}%
\begin{pgfscope}%
\pgfsys@transformshift{4.512409in}{3.576521in}%
\pgfsys@useobject{currentmarker}{}%
\end{pgfscope}%
\begin{pgfscope}%
\pgfsys@transformshift{2.101242in}{4.547340in}%
\pgfsys@useobject{currentmarker}{}%
\end{pgfscope}%
\begin{pgfscope}%
\pgfsys@transformshift{1.753081in}{0.789429in}%
\pgfsys@useobject{currentmarker}{}%
\end{pgfscope}%
\begin{pgfscope}%
\pgfsys@transformshift{4.632241in}{0.212409in}%
\pgfsys@useobject{currentmarker}{}%
\end{pgfscope}%
\begin{pgfscope}%
\pgfsys@transformshift{0.632041in}{2.836243in}%
\pgfsys@useobject{currentmarker}{}%
\end{pgfscope}%
\begin{pgfscope}%
\pgfsys@transformshift{4.936656in}{4.301994in}%
\pgfsys@useobject{currentmarker}{}%
\end{pgfscope}%
\begin{pgfscope}%
\pgfsys@transformshift{6.689590in}{3.328316in}%
\pgfsys@useobject{currentmarker}{}%
\end{pgfscope}%
\begin{pgfscope}%
\pgfsys@transformshift{4.217622in}{4.612652in}%
\pgfsys@useobject{currentmarker}{}%
\end{pgfscope}%
\begin{pgfscope}%
\pgfsys@transformshift{2.619341in}{1.072985in}%
\pgfsys@useobject{currentmarker}{}%
\end{pgfscope}%
\begin{pgfscope}%
\pgfsys@transformshift{5.300125in}{5.145135in}%
\pgfsys@useobject{currentmarker}{}%
\end{pgfscope}%
\begin{pgfscope}%
\pgfsys@transformshift{3.358836in}{5.209199in}%
\pgfsys@useobject{currentmarker}{}%
\end{pgfscope}%
\begin{pgfscope}%
\pgfsys@transformshift{2.044740in}{2.143881in}%
\pgfsys@useobject{currentmarker}{}%
\end{pgfscope}%
\begin{pgfscope}%
\pgfsys@transformshift{5.852632in}{5.298387in}%
\pgfsys@useobject{currentmarker}{}%
\end{pgfscope}%
\begin{pgfscope}%
\pgfsys@transformshift{4.560976in}{5.361356in}%
\pgfsys@useobject{currentmarker}{}%
\end{pgfscope}%
\begin{pgfscope}%
\pgfsys@transformshift{2.888138in}{1.740370in}%
\pgfsys@useobject{currentmarker}{}%
\end{pgfscope}%
\begin{pgfscope}%
\pgfsys@transformshift{3.882673in}{5.078881in}%
\pgfsys@useobject{currentmarker}{}%
\end{pgfscope}%
\begin{pgfscope}%
\pgfsys@transformshift{1.316389in}{2.227657in}%
\pgfsys@useobject{currentmarker}{}%
\end{pgfscope}%
\begin{pgfscope}%
\pgfsys@transformshift{5.300434in}{3.557414in}%
\pgfsys@useobject{currentmarker}{}%
\end{pgfscope}%
\begin{pgfscope}%
\pgfsys@transformshift{1.980597in}{1.453799in}%
\pgfsys@useobject{currentmarker}{}%
\end{pgfscope}%
\begin{pgfscope}%
\pgfsys@transformshift{0.695441in}{3.262340in}%
\pgfsys@useobject{currentmarker}{}%
\end{pgfscope}%
\begin{pgfscope}%
\pgfsys@transformshift{6.706101in}{0.873051in}%
\pgfsys@useobject{currentmarker}{}%
\end{pgfscope}%
\begin{pgfscope}%
\pgfsys@transformshift{5.474031in}{1.664002in}%
\pgfsys@useobject{currentmarker}{}%
\end{pgfscope}%
\begin{pgfscope}%
\pgfsys@transformshift{4.718805in}{5.538386in}%
\pgfsys@useobject{currentmarker}{}%
\end{pgfscope}%
\begin{pgfscope}%
\pgfsys@transformshift{4.886382in}{2.207520in}%
\pgfsys@useobject{currentmarker}{}%
\end{pgfscope}%
\begin{pgfscope}%
\pgfsys@transformshift{2.099852in}{3.710927in}%
\pgfsys@useobject{currentmarker}{}%
\end{pgfscope}%
\begin{pgfscope}%
\pgfsys@transformshift{6.633788in}{4.370551in}%
\pgfsys@useobject{currentmarker}{}%
\end{pgfscope}%
\begin{pgfscope}%
\pgfsys@transformshift{6.665277in}{4.003804in}%
\pgfsys@useobject{currentmarker}{}%
\end{pgfscope}%
\begin{pgfscope}%
\pgfsys@transformshift{6.761160in}{5.179835in}%
\pgfsys@useobject{currentmarker}{}%
\end{pgfscope}%
\begin{pgfscope}%
\pgfsys@transformshift{0.733463in}{4.564859in}%
\pgfsys@useobject{currentmarker}{}%
\end{pgfscope}%
\begin{pgfscope}%
\pgfsys@transformshift{0.844439in}{3.911732in}%
\pgfsys@useobject{currentmarker}{}%
\end{pgfscope}%
\begin{pgfscope}%
\pgfsys@transformshift{5.381436in}{3.192049in}%
\pgfsys@useobject{currentmarker}{}%
\end{pgfscope}%
\begin{pgfscope}%
\pgfsys@transformshift{3.807021in}{0.858638in}%
\pgfsys@useobject{currentmarker}{}%
\end{pgfscope}%
\begin{pgfscope}%
\pgfsys@transformshift{2.391056in}{1.264711in}%
\pgfsys@useobject{currentmarker}{}%
\end{pgfscope}%
\begin{pgfscope}%
\pgfsys@transformshift{2.948632in}{0.706713in}%
\pgfsys@useobject{currentmarker}{}%
\end{pgfscope}%
\begin{pgfscope}%
\pgfsys@transformshift{3.242636in}{0.463250in}%
\pgfsys@useobject{currentmarker}{}%
\end{pgfscope}%
\begin{pgfscope}%
\pgfsys@transformshift{0.758804in}{5.598900in}%
\pgfsys@useobject{currentmarker}{}%
\end{pgfscope}%
\begin{pgfscope}%
\pgfsys@transformshift{2.299802in}{0.900808in}%
\pgfsys@useobject{currentmarker}{}%
\end{pgfscope}%
\begin{pgfscope}%
\pgfsys@transformshift{3.295763in}{1.279436in}%
\pgfsys@useobject{currentmarker}{}%
\end{pgfscope}%
\begin{pgfscope}%
\pgfsys@transformshift{6.553575in}{2.015242in}%
\pgfsys@useobject{currentmarker}{}%
\end{pgfscope}%
\begin{pgfscope}%
\pgfsys@transformshift{4.901427in}{2.011591in}%
\pgfsys@useobject{currentmarker}{}%
\end{pgfscope}%
\begin{pgfscope}%
\pgfsys@transformshift{3.869181in}{0.528128in}%
\pgfsys@useobject{currentmarker}{}%
\end{pgfscope}%
\begin{pgfscope}%
\pgfsys@transformshift{0.816007in}{3.698963in}%
\pgfsys@useobject{currentmarker}{}%
\end{pgfscope}%
\begin{pgfscope}%
\pgfsys@transformshift{3.384732in}{5.596170in}%
\pgfsys@useobject{currentmarker}{}%
\end{pgfscope}%
\begin{pgfscope}%
\pgfsys@transformshift{1.883288in}{5.003959in}%
\pgfsys@useobject{currentmarker}{}%
\end{pgfscope}%
\begin{pgfscope}%
\pgfsys@transformshift{1.174030in}{4.249498in}%
\pgfsys@useobject{currentmarker}{}%
\end{pgfscope}%
\begin{pgfscope}%
\pgfsys@transformshift{3.011708in}{4.125365in}%
\pgfsys@useobject{currentmarker}{}%
\end{pgfscope}%
\begin{pgfscope}%
\pgfsys@transformshift{2.049231in}{1.517104in}%
\pgfsys@useobject{currentmarker}{}%
\end{pgfscope}%
\begin{pgfscope}%
\pgfsys@transformshift{6.343334in}{1.037194in}%
\pgfsys@useobject{currentmarker}{}%
\end{pgfscope}%
\begin{pgfscope}%
\pgfsys@transformshift{6.000403in}{3.813646in}%
\pgfsys@useobject{currentmarker}{}%
\end{pgfscope}%
\begin{pgfscope}%
\pgfsys@transformshift{3.776016in}{5.515685in}%
\pgfsys@useobject{currentmarker}{}%
\end{pgfscope}%
\begin{pgfscope}%
\pgfsys@transformshift{4.647949in}{0.315952in}%
\pgfsys@useobject{currentmarker}{}%
\end{pgfscope}%
\begin{pgfscope}%
\pgfsys@transformshift{4.495272in}{5.387673in}%
\pgfsys@useobject{currentmarker}{}%
\end{pgfscope}%
\begin{pgfscope}%
\pgfsys@transformshift{4.731555in}{2.352484in}%
\pgfsys@useobject{currentmarker}{}%
\end{pgfscope}%
\begin{pgfscope}%
\pgfsys@transformshift{1.649823in}{1.237897in}%
\pgfsys@useobject{currentmarker}{}%
\end{pgfscope}%
\begin{pgfscope}%
\pgfsys@transformshift{1.616021in}{4.337931in}%
\pgfsys@useobject{currentmarker}{}%
\end{pgfscope}%
\begin{pgfscope}%
\pgfsys@transformshift{5.747019in}{4.866110in}%
\pgfsys@useobject{currentmarker}{}%
\end{pgfscope}%
\begin{pgfscope}%
\pgfsys@transformshift{0.883956in}{0.861120in}%
\pgfsys@useobject{currentmarker}{}%
\end{pgfscope}%
\begin{pgfscope}%
\pgfsys@transformshift{2.898364in}{4.087447in}%
\pgfsys@useobject{currentmarker}{}%
\end{pgfscope}%
\begin{pgfscope}%
\pgfsys@transformshift{3.156722in}{2.291898in}%
\pgfsys@useobject{currentmarker}{}%
\end{pgfscope}%
\begin{pgfscope}%
\pgfsys@transformshift{2.298231in}{4.039262in}%
\pgfsys@useobject{currentmarker}{}%
\end{pgfscope}%
\begin{pgfscope}%
\pgfsys@transformshift{1.889708in}{3.986759in}%
\pgfsys@useobject{currentmarker}{}%
\end{pgfscope}%
\begin{pgfscope}%
\pgfsys@transformshift{5.893202in}{3.444348in}%
\pgfsys@useobject{currentmarker}{}%
\end{pgfscope}%
\begin{pgfscope}%
\pgfsys@transformshift{5.171030in}{4.730581in}%
\pgfsys@useobject{currentmarker}{}%
\end{pgfscope}%
\begin{pgfscope}%
\pgfsys@transformshift{6.623684in}{4.946985in}%
\pgfsys@useobject{currentmarker}{}%
\end{pgfscope}%
\begin{pgfscope}%
\pgfsys@transformshift{2.202053in}{3.748833in}%
\pgfsys@useobject{currentmarker}{}%
\end{pgfscope}%
\begin{pgfscope}%
\pgfsys@transformshift{0.806785in}{2.518087in}%
\pgfsys@useobject{currentmarker}{}%
\end{pgfscope}%
\begin{pgfscope}%
\pgfsys@transformshift{6.119562in}{2.890613in}%
\pgfsys@useobject{currentmarker}{}%
\end{pgfscope}%
\begin{pgfscope}%
\pgfsys@transformshift{4.361458in}{2.021060in}%
\pgfsys@useobject{currentmarker}{}%
\end{pgfscope}%
\begin{pgfscope}%
\pgfsys@transformshift{4.079672in}{4.818521in}%
\pgfsys@useobject{currentmarker}{}%
\end{pgfscope}%
\begin{pgfscope}%
\pgfsys@transformshift{5.987951in}{1.287495in}%
\pgfsys@useobject{currentmarker}{}%
\end{pgfscope}%
\begin{pgfscope}%
\pgfsys@transformshift{4.502762in}{1.680264in}%
\pgfsys@useobject{currentmarker}{}%
\end{pgfscope}%
\begin{pgfscope}%
\pgfsys@transformshift{6.715823in}{0.617697in}%
\pgfsys@useobject{currentmarker}{}%
\end{pgfscope}%
\begin{pgfscope}%
\pgfsys@transformshift{4.070039in}{4.771017in}%
\pgfsys@useobject{currentmarker}{}%
\end{pgfscope}%
\begin{pgfscope}%
\pgfsys@transformshift{5.057400in}{2.521292in}%
\pgfsys@useobject{currentmarker}{}%
\end{pgfscope}%
\begin{pgfscope}%
\pgfsys@transformshift{4.514795in}{3.566155in}%
\pgfsys@useobject{currentmarker}{}%
\end{pgfscope}%
\begin{pgfscope}%
\pgfsys@transformshift{6.362345in}{1.868520in}%
\pgfsys@useobject{currentmarker}{}%
\end{pgfscope}%
\begin{pgfscope}%
\pgfsys@transformshift{1.841212in}{0.507564in}%
\pgfsys@useobject{currentmarker}{}%
\end{pgfscope}%
\begin{pgfscope}%
\pgfsys@transformshift{1.487338in}{4.673985in}%
\pgfsys@useobject{currentmarker}{}%
\end{pgfscope}%
\begin{pgfscope}%
\pgfsys@transformshift{6.597316in}{3.374792in}%
\pgfsys@useobject{currentmarker}{}%
\end{pgfscope}%
\begin{pgfscope}%
\pgfsys@transformshift{4.460198in}{3.902970in}%
\pgfsys@useobject{currentmarker}{}%
\end{pgfscope}%
\begin{pgfscope}%
\pgfsys@transformshift{1.318192in}{2.419967in}%
\pgfsys@useobject{currentmarker}{}%
\end{pgfscope}%
\begin{pgfscope}%
\pgfsys@transformshift{1.613930in}{0.413143in}%
\pgfsys@useobject{currentmarker}{}%
\end{pgfscope}%
\begin{pgfscope}%
\pgfsys@transformshift{5.257206in}{1.360614in}%
\pgfsys@useobject{currentmarker}{}%
\end{pgfscope}%
\begin{pgfscope}%
\pgfsys@transformshift{1.440141in}{1.772643in}%
\pgfsys@useobject{currentmarker}{}%
\end{pgfscope}%
\begin{pgfscope}%
\pgfsys@transformshift{2.953166in}{3.772788in}%
\pgfsys@useobject{currentmarker}{}%
\end{pgfscope}%
\begin{pgfscope}%
\pgfsys@transformshift{4.214887in}{4.188086in}%
\pgfsys@useobject{currentmarker}{}%
\end{pgfscope}%
\begin{pgfscope}%
\pgfsys@transformshift{4.789466in}{2.974967in}%
\pgfsys@useobject{currentmarker}{}%
\end{pgfscope}%
\begin{pgfscope}%
\pgfsys@transformshift{5.488053in}{3.383650in}%
\pgfsys@useobject{currentmarker}{}%
\end{pgfscope}%
\begin{pgfscope}%
\pgfsys@transformshift{4.578398in}{2.975727in}%
\pgfsys@useobject{currentmarker}{}%
\end{pgfscope}%
\begin{pgfscope}%
\pgfsys@transformshift{5.907316in}{1.755087in}%
\pgfsys@useobject{currentmarker}{}%
\end{pgfscope}%
\begin{pgfscope}%
\pgfsys@transformshift{0.794864in}{1.561967in}%
\pgfsys@useobject{currentmarker}{}%
\end{pgfscope}%
\begin{pgfscope}%
\pgfsys@transformshift{4.611415in}{0.866224in}%
\pgfsys@useobject{currentmarker}{}%
\end{pgfscope}%
\begin{pgfscope}%
\pgfsys@transformshift{1.024983in}{5.547819in}%
\pgfsys@useobject{currentmarker}{}%
\end{pgfscope}%
\begin{pgfscope}%
\pgfsys@transformshift{2.096234in}{0.799507in}%
\pgfsys@useobject{currentmarker}{}%
\end{pgfscope}%
\begin{pgfscope}%
\pgfsys@transformshift{1.569828in}{0.385262in}%
\pgfsys@useobject{currentmarker}{}%
\end{pgfscope}%
\begin{pgfscope}%
\pgfsys@transformshift{2.777036in}{5.405926in}%
\pgfsys@useobject{currentmarker}{}%
\end{pgfscope}%
\begin{pgfscope}%
\pgfsys@transformshift{2.785162in}{1.086089in}%
\pgfsys@useobject{currentmarker}{}%
\end{pgfscope}%
\begin{pgfscope}%
\pgfsys@transformshift{3.728300in}{1.947660in}%
\pgfsys@useobject{currentmarker}{}%
\end{pgfscope}%
\begin{pgfscope}%
\pgfsys@transformshift{5.521787in}{2.578131in}%
\pgfsys@useobject{currentmarker}{}%
\end{pgfscope}%
\begin{pgfscope}%
\pgfsys@transformshift{3.465046in}{3.551128in}%
\pgfsys@useobject{currentmarker}{}%
\end{pgfscope}%
\begin{pgfscope}%
\pgfsys@transformshift{6.422339in}{4.732718in}%
\pgfsys@useobject{currentmarker}{}%
\end{pgfscope}%
\begin{pgfscope}%
\pgfsys@transformshift{1.621743in}{4.623190in}%
\pgfsys@useobject{currentmarker}{}%
\end{pgfscope}%
\begin{pgfscope}%
\pgfsys@transformshift{0.890611in}{4.445590in}%
\pgfsys@useobject{currentmarker}{}%
\end{pgfscope}%
\begin{pgfscope}%
\pgfsys@transformshift{5.947438in}{4.060954in}%
\pgfsys@useobject{currentmarker}{}%
\end{pgfscope}%
\begin{pgfscope}%
\pgfsys@transformshift{2.512667in}{5.565949in}%
\pgfsys@useobject{currentmarker}{}%
\end{pgfscope}%
\begin{pgfscope}%
\pgfsys@transformshift{4.067232in}{5.013550in}%
\pgfsys@useobject{currentmarker}{}%
\end{pgfscope}%
\begin{pgfscope}%
\pgfsys@transformshift{2.448968in}{2.405912in}%
\pgfsys@useobject{currentmarker}{}%
\end{pgfscope}%
\begin{pgfscope}%
\pgfsys@transformshift{4.072426in}{5.401527in}%
\pgfsys@useobject{currentmarker}{}%
\end{pgfscope}%
\begin{pgfscope}%
\pgfsys@transformshift{0.857677in}{4.708843in}%
\pgfsys@useobject{currentmarker}{}%
\end{pgfscope}%
\begin{pgfscope}%
\pgfsys@transformshift{6.282338in}{2.736996in}%
\pgfsys@useobject{currentmarker}{}%
\end{pgfscope}%
\begin{pgfscope}%
\pgfsys@transformshift{5.093364in}{5.042830in}%
\pgfsys@useobject{currentmarker}{}%
\end{pgfscope}%
\begin{pgfscope}%
\pgfsys@transformshift{3.571786in}{1.453709in}%
\pgfsys@useobject{currentmarker}{}%
\end{pgfscope}%
\begin{pgfscope}%
\pgfsys@transformshift{1.562124in}{0.839372in}%
\pgfsys@useobject{currentmarker}{}%
\end{pgfscope}%
\begin{pgfscope}%
\pgfsys@transformshift{0.816111in}{3.179639in}%
\pgfsys@useobject{currentmarker}{}%
\end{pgfscope}%
\begin{pgfscope}%
\pgfsys@transformshift{2.160495in}{0.629547in}%
\pgfsys@useobject{currentmarker}{}%
\end{pgfscope}%
\begin{pgfscope}%
\pgfsys@transformshift{4.651263in}{3.714401in}%
\pgfsys@useobject{currentmarker}{}%
\end{pgfscope}%
\begin{pgfscope}%
\pgfsys@transformshift{4.577791in}{2.654746in}%
\pgfsys@useobject{currentmarker}{}%
\end{pgfscope}%
\begin{pgfscope}%
\pgfsys@transformshift{6.728982in}{2.689800in}%
\pgfsys@useobject{currentmarker}{}%
\end{pgfscope}%
\begin{pgfscope}%
\pgfsys@transformshift{4.891599in}{4.563249in}%
\pgfsys@useobject{currentmarker}{}%
\end{pgfscope}%
\begin{pgfscope}%
\pgfsys@transformshift{6.740999in}{3.849346in}%
\pgfsys@useobject{currentmarker}{}%
\end{pgfscope}%
\begin{pgfscope}%
\pgfsys@transformshift{4.099947in}{4.262015in}%
\pgfsys@useobject{currentmarker}{}%
\end{pgfscope}%
\begin{pgfscope}%
\pgfsys@transformshift{1.034388in}{0.344090in}%
\pgfsys@useobject{currentmarker}{}%
\end{pgfscope}%
\begin{pgfscope}%
\pgfsys@transformshift{4.420099in}{4.245184in}%
\pgfsys@useobject{currentmarker}{}%
\end{pgfscope}%
\begin{pgfscope}%
\pgfsys@transformshift{5.747745in}{4.768645in}%
\pgfsys@useobject{currentmarker}{}%
\end{pgfscope}%
\begin{pgfscope}%
\pgfsys@transformshift{3.239791in}{1.440911in}%
\pgfsys@useobject{currentmarker}{}%
\end{pgfscope}%
\begin{pgfscope}%
\pgfsys@transformshift{6.718639in}{1.029180in}%
\pgfsys@useobject{currentmarker}{}%
\end{pgfscope}%
\begin{pgfscope}%
\pgfsys@transformshift{6.744954in}{3.132132in}%
\pgfsys@useobject{currentmarker}{}%
\end{pgfscope}%
\begin{pgfscope}%
\pgfsys@transformshift{2.462530in}{4.768330in}%
\pgfsys@useobject{currentmarker}{}%
\end{pgfscope}%
\begin{pgfscope}%
\pgfsys@transformshift{4.109290in}{4.745653in}%
\pgfsys@useobject{currentmarker}{}%
\end{pgfscope}%
\begin{pgfscope}%
\pgfsys@transformshift{3.063840in}{4.490096in}%
\pgfsys@useobject{currentmarker}{}%
\end{pgfscope}%
\begin{pgfscope}%
\pgfsys@transformshift{3.621973in}{5.508516in}%
\pgfsys@useobject{currentmarker}{}%
\end{pgfscope}%
\begin{pgfscope}%
\pgfsys@transformshift{4.459954in}{4.106209in}%
\pgfsys@useobject{currentmarker}{}%
\end{pgfscope}%
\begin{pgfscope}%
\pgfsys@transformshift{6.170199in}{2.008509in}%
\pgfsys@useobject{currentmarker}{}%
\end{pgfscope}%
\begin{pgfscope}%
\pgfsys@transformshift{4.627067in}{0.828684in}%
\pgfsys@useobject{currentmarker}{}%
\end{pgfscope}%
\begin{pgfscope}%
\pgfsys@transformshift{4.432043in}{2.914979in}%
\pgfsys@useobject{currentmarker}{}%
\end{pgfscope}%
\begin{pgfscope}%
\pgfsys@transformshift{4.727994in}{4.194037in}%
\pgfsys@useobject{currentmarker}{}%
\end{pgfscope}%
\begin{pgfscope}%
\pgfsys@transformshift{6.576946in}{3.399158in}%
\pgfsys@useobject{currentmarker}{}%
\end{pgfscope}%
\begin{pgfscope}%
\pgfsys@transformshift{5.099498in}{0.575336in}%
\pgfsys@useobject{currentmarker}{}%
\end{pgfscope}%
\begin{pgfscope}%
\pgfsys@transformshift{3.780532in}{2.528300in}%
\pgfsys@useobject{currentmarker}{}%
\end{pgfscope}%
\begin{pgfscope}%
\pgfsys@transformshift{4.323725in}{2.496238in}%
\pgfsys@useobject{currentmarker}{}%
\end{pgfscope}%
\begin{pgfscope}%
\pgfsys@transformshift{3.260696in}{1.952726in}%
\pgfsys@useobject{currentmarker}{}%
\end{pgfscope}%
\begin{pgfscope}%
\pgfsys@transformshift{1.045688in}{1.410939in}%
\pgfsys@useobject{currentmarker}{}%
\end{pgfscope}%
\begin{pgfscope}%
\pgfsys@transformshift{5.821127in}{4.339093in}%
\pgfsys@useobject{currentmarker}{}%
\end{pgfscope}%
\begin{pgfscope}%
\pgfsys@transformshift{5.566394in}{4.342515in}%
\pgfsys@useobject{currentmarker}{}%
\end{pgfscope}%
\begin{pgfscope}%
\pgfsys@transformshift{3.639947in}{2.206803in}%
\pgfsys@useobject{currentmarker}{}%
\end{pgfscope}%
\begin{pgfscope}%
\pgfsys@transformshift{5.561833in}{4.281859in}%
\pgfsys@useobject{currentmarker}{}%
\end{pgfscope}%
\begin{pgfscope}%
\pgfsys@transformshift{1.847167in}{0.855194in}%
\pgfsys@useobject{currentmarker}{}%
\end{pgfscope}%
\begin{pgfscope}%
\pgfsys@transformshift{6.708460in}{3.253407in}%
\pgfsys@useobject{currentmarker}{}%
\end{pgfscope}%
\begin{pgfscope}%
\pgfsys@transformshift{3.909865in}{2.742576in}%
\pgfsys@useobject{currentmarker}{}%
\end{pgfscope}%
\begin{pgfscope}%
\pgfsys@transformshift{2.269464in}{3.395422in}%
\pgfsys@useobject{currentmarker}{}%
\end{pgfscope}%
\begin{pgfscope}%
\pgfsys@transformshift{6.875543in}{4.279453in}%
\pgfsys@useobject{currentmarker}{}%
\end{pgfscope}%
\begin{pgfscope}%
\pgfsys@transformshift{3.861631in}{2.252622in}%
\pgfsys@useobject{currentmarker}{}%
\end{pgfscope}%
\begin{pgfscope}%
\pgfsys@transformshift{2.729439in}{2.717699in}%
\pgfsys@useobject{currentmarker}{}%
\end{pgfscope}%
\end{pgfscope}%
\begin{pgfscope}%
\pgfpathrectangle{\pgfqpoint{0.626386in}{0.608332in}}{\pgfqpoint{6.200000in}{4.620000in}}%
\pgfusepath{clip}%
\pgfsetbuttcap%
\pgfsetroundjoin%
\definecolor{currentfill}{rgb}{0.839216,0.152941,0.156863}%
\pgfsetfillcolor{currentfill}%
\pgfsetlinewidth{1.003750pt}%
\definecolor{currentstroke}{rgb}{0.839216,0.152941,0.156863}%
\pgfsetstrokecolor{currentstroke}%
\pgfsetdash{}{0pt}%
\pgfsys@defobject{currentmarker}{\pgfqpoint{-0.031056in}{-0.031056in}}{\pgfqpoint{0.031056in}{0.031056in}}{%
\pgfpathmoveto{\pgfqpoint{0.000000in}{-0.031056in}}%
\pgfpathcurveto{\pgfqpoint{0.008236in}{-0.031056in}}{\pgfqpoint{0.016136in}{-0.027784in}}{\pgfqpoint{0.021960in}{-0.021960in}}%
\pgfpathcurveto{\pgfqpoint{0.027784in}{-0.016136in}}{\pgfqpoint{0.031056in}{-0.008236in}}{\pgfqpoint{0.031056in}{0.000000in}}%
\pgfpathcurveto{\pgfqpoint{0.031056in}{0.008236in}}{\pgfqpoint{0.027784in}{0.016136in}}{\pgfqpoint{0.021960in}{0.021960in}}%
\pgfpathcurveto{\pgfqpoint{0.016136in}{0.027784in}}{\pgfqpoint{0.008236in}{0.031056in}}{\pgfqpoint{0.000000in}{0.031056in}}%
\pgfpathcurveto{\pgfqpoint{-0.008236in}{0.031056in}}{\pgfqpoint{-0.016136in}{0.027784in}}{\pgfqpoint{-0.021960in}{0.021960in}}%
\pgfpathcurveto{\pgfqpoint{-0.027784in}{0.016136in}}{\pgfqpoint{-0.031056in}{0.008236in}}{\pgfqpoint{-0.031056in}{0.000000in}}%
\pgfpathcurveto{\pgfqpoint{-0.031056in}{-0.008236in}}{\pgfqpoint{-0.027784in}{-0.016136in}}{\pgfqpoint{-0.021960in}{-0.021960in}}%
\pgfpathcurveto{\pgfqpoint{-0.016136in}{-0.027784in}}{\pgfqpoint{-0.008236in}{-0.031056in}}{\pgfqpoint{0.000000in}{-0.031056in}}%
\pgfpathlineto{\pgfqpoint{0.000000in}{-0.031056in}}%
\pgfpathclose%
\pgfusepath{stroke,fill}%
}%
\begin{pgfscope}%
\pgfsys@transformshift{1.171002in}{2.753761in}%
\pgfsys@useobject{currentmarker}{}%
\end{pgfscope}%
\begin{pgfscope}%
\pgfsys@transformshift{2.672978in}{1.285705in}%
\pgfsys@useobject{currentmarker}{}%
\end{pgfscope}%
\begin{pgfscope}%
\pgfsys@transformshift{2.536411in}{1.480725in}%
\pgfsys@useobject{currentmarker}{}%
\end{pgfscope}%
\begin{pgfscope}%
\pgfsys@transformshift{6.563792in}{0.571309in}%
\pgfsys@useobject{currentmarker}{}%
\end{pgfscope}%
\begin{pgfscope}%
\pgfsys@transformshift{4.968865in}{0.711625in}%
\pgfsys@useobject{currentmarker}{}%
\end{pgfscope}%
\begin{pgfscope}%
\pgfsys@transformshift{4.114124in}{0.764184in}%
\pgfsys@useobject{currentmarker}{}%
\end{pgfscope}%
\begin{pgfscope}%
\pgfsys@transformshift{2.650795in}{1.247392in}%
\pgfsys@useobject{currentmarker}{}%
\end{pgfscope}%
\begin{pgfscope}%
\pgfsys@transformshift{3.771139in}{0.864515in}%
\pgfsys@useobject{currentmarker}{}%
\end{pgfscope}%
\begin{pgfscope}%
\pgfsys@transformshift{3.098873in}{1.320918in}%
\pgfsys@useobject{currentmarker}{}%
\end{pgfscope}%
\begin{pgfscope}%
\pgfsys@transformshift{2.594401in}{1.640905in}%
\pgfsys@useobject{currentmarker}{}%
\end{pgfscope}%
\begin{pgfscope}%
\pgfsys@transformshift{1.865799in}{1.980264in}%
\pgfsys@useobject{currentmarker}{}%
\end{pgfscope}%
\begin{pgfscope}%
\pgfsys@transformshift{1.279225in}{2.588906in}%
\pgfsys@useobject{currentmarker}{}%
\end{pgfscope}%
\begin{pgfscope}%
\pgfsys@transformshift{2.908829in}{1.191898in}%
\pgfsys@useobject{currentmarker}{}%
\end{pgfscope}%
\begin{pgfscope}%
\pgfsys@transformshift{1.419679in}{2.355247in}%
\pgfsys@useobject{currentmarker}{}%
\end{pgfscope}%
\begin{pgfscope}%
\pgfsys@transformshift{1.056228in}{2.867376in}%
\pgfsys@useobject{currentmarker}{}%
\end{pgfscope}%
\begin{pgfscope}%
\pgfsys@transformshift{0.763440in}{3.773630in}%
\pgfsys@useobject{currentmarker}{}%
\end{pgfscope}%
\begin{pgfscope}%
\pgfsys@transformshift{4.539430in}{0.774317in}%
\pgfsys@useobject{currentmarker}{}%
\end{pgfscope}%
\begin{pgfscope}%
\pgfsys@transformshift{5.164025in}{0.672611in}%
\pgfsys@useobject{currentmarker}{}%
\end{pgfscope}%
\begin{pgfscope}%
\pgfsys@transformshift{1.411858in}{2.496224in}%
\pgfsys@useobject{currentmarker}{}%
\end{pgfscope}%
\begin{pgfscope}%
\pgfsys@transformshift{3.323354in}{1.022890in}%
\pgfsys@useobject{currentmarker}{}%
\end{pgfscope}%
\begin{pgfscope}%
\pgfsys@transformshift{1.152253in}{2.650657in}%
\pgfsys@useobject{currentmarker}{}%
\end{pgfscope}%
\begin{pgfscope}%
\pgfsys@transformshift{4.493681in}{0.668277in}%
\pgfsys@useobject{currentmarker}{}%
\end{pgfscope}%
\begin{pgfscope}%
\pgfsys@transformshift{4.139112in}{0.785824in}%
\pgfsys@useobject{currentmarker}{}%
\end{pgfscope}%
\begin{pgfscope}%
\pgfsys@transformshift{1.602751in}{2.048253in}%
\pgfsys@useobject{currentmarker}{}%
\end{pgfscope}%
\begin{pgfscope}%
\pgfsys@transformshift{5.482158in}{0.576145in}%
\pgfsys@useobject{currentmarker}{}%
\end{pgfscope}%
\begin{pgfscope}%
\pgfsys@transformshift{3.248174in}{0.965477in}%
\pgfsys@useobject{currentmarker}{}%
\end{pgfscope}%
\begin{pgfscope}%
\pgfsys@transformshift{3.839753in}{0.806067in}%
\pgfsys@useobject{currentmarker}{}%
\end{pgfscope}%
\begin{pgfscope}%
\pgfsys@transformshift{2.429558in}{1.668869in}%
\pgfsys@useobject{currentmarker}{}%
\end{pgfscope}%
\begin{pgfscope}%
\pgfsys@transformshift{3.106341in}{1.308808in}%
\pgfsys@useobject{currentmarker}{}%
\end{pgfscope}%
\begin{pgfscope}%
\pgfsys@transformshift{3.993120in}{0.899747in}%
\pgfsys@useobject{currentmarker}{}%
\end{pgfscope}%
\begin{pgfscope}%
\pgfsys@transformshift{5.758990in}{0.597983in}%
\pgfsys@useobject{currentmarker}{}%
\end{pgfscope}%
\begin{pgfscope}%
\pgfsys@transformshift{4.342434in}{0.678690in}%
\pgfsys@useobject{currentmarker}{}%
\end{pgfscope}%
\begin{pgfscope}%
\pgfsys@transformshift{4.913462in}{0.591345in}%
\pgfsys@useobject{currentmarker}{}%
\end{pgfscope}%
\begin{pgfscope}%
\pgfsys@transformshift{1.456272in}{2.142124in}%
\pgfsys@useobject{currentmarker}{}%
\end{pgfscope}%
\begin{pgfscope}%
\pgfsys@transformshift{0.983666in}{3.334171in}%
\pgfsys@useobject{currentmarker}{}%
\end{pgfscope}%
\begin{pgfscope}%
\pgfsys@transformshift{1.675103in}{2.367521in}%
\pgfsys@useobject{currentmarker}{}%
\end{pgfscope}%
\begin{pgfscope}%
\pgfsys@transformshift{3.462161in}{1.070390in}%
\pgfsys@useobject{currentmarker}{}%
\end{pgfscope}%
\begin{pgfscope}%
\pgfsys@transformshift{4.702284in}{0.728509in}%
\pgfsys@useobject{currentmarker}{}%
\end{pgfscope}%
\begin{pgfscope}%
\pgfsys@transformshift{2.059371in}{1.976888in}%
\pgfsys@useobject{currentmarker}{}%
\end{pgfscope}%
\begin{pgfscope}%
\pgfsys@transformshift{1.630066in}{2.230214in}%
\pgfsys@useobject{currentmarker}{}%
\end{pgfscope}%
\begin{pgfscope}%
\pgfsys@transformshift{3.940720in}{0.753932in}%
\pgfsys@useobject{currentmarker}{}%
\end{pgfscope}%
\begin{pgfscope}%
\pgfsys@transformshift{3.796370in}{0.920318in}%
\pgfsys@useobject{currentmarker}{}%
\end{pgfscope}%
\begin{pgfscope}%
\pgfsys@transformshift{2.873589in}{1.278336in}%
\pgfsys@useobject{currentmarker}{}%
\end{pgfscope}%
\begin{pgfscope}%
\pgfsys@transformshift{5.451502in}{0.558177in}%
\pgfsys@useobject{currentmarker}{}%
\end{pgfscope}%
\begin{pgfscope}%
\pgfsys@transformshift{1.324555in}{2.256327in}%
\pgfsys@useobject{currentmarker}{}%
\end{pgfscope}%
\begin{pgfscope}%
\pgfsys@transformshift{2.411060in}{1.472078in}%
\pgfsys@useobject{currentmarker}{}%
\end{pgfscope}%
\begin{pgfscope}%
\pgfsys@transformshift{3.470396in}{1.039351in}%
\pgfsys@useobject{currentmarker}{}%
\end{pgfscope}%
\begin{pgfscope}%
\pgfsys@transformshift{1.510721in}{2.354600in}%
\pgfsys@useobject{currentmarker}{}%
\end{pgfscope}%
\begin{pgfscope}%
\pgfsys@transformshift{2.721770in}{1.510947in}%
\pgfsys@useobject{currentmarker}{}%
\end{pgfscope}%
\begin{pgfscope}%
\pgfsys@transformshift{1.470327in}{2.482215in}%
\pgfsys@useobject{currentmarker}{}%
\end{pgfscope}%
\begin{pgfscope}%
\pgfsys@transformshift{3.807021in}{0.858638in}%
\pgfsys@useobject{currentmarker}{}%
\end{pgfscope}%
\begin{pgfscope}%
\pgfsys@transformshift{0.816007in}{3.698963in}%
\pgfsys@useobject{currentmarker}{}%
\end{pgfscope}%
\begin{pgfscope}%
\pgfsys@transformshift{1.318192in}{2.419967in}%
\pgfsys@useobject{currentmarker}{}%
\end{pgfscope}%
\begin{pgfscope}%
\pgfsys@transformshift{5.099498in}{0.575336in}%
\pgfsys@useobject{currentmarker}{}%
\end{pgfscope}%
\end{pgfscope}%
\begin{pgfscope}%
\pgfpathrectangle{\pgfqpoint{0.626386in}{0.608332in}}{\pgfqpoint{6.200000in}{4.620000in}}%
\pgfusepath{clip}%
\pgfsetbuttcap%
\pgfsetmiterjoin%
\definecolor{currentfill}{rgb}{0.501961,0.501961,0.501961}%
\pgfsetfillcolor{currentfill}%
\pgfsetfillopacity{0.200000}%
\pgfsetlinewidth{1.003750pt}%
\definecolor{currentstroke}{rgb}{0.501961,0.501961,0.501961}%
\pgfsetstrokecolor{currentstroke}%
\pgfsetstrokeopacity{0.200000}%
\pgfsetdash{}{0pt}%
\pgfpathmoveto{\pgfqpoint{0.626386in}{4.391376in}}%
\pgfpathlineto{\pgfqpoint{0.627161in}{4.307262in}}%
\pgfpathlineto{\pgfqpoint{0.629487in}{4.224002in}}%
\pgfpathlineto{\pgfqpoint{0.633363in}{4.141596in}}%
\pgfpathlineto{\pgfqpoint{0.638790in}{4.060043in}}%
\pgfpathlineto{\pgfqpoint{0.645767in}{3.979345in}}%
\pgfpathlineto{\pgfqpoint{0.654294in}{3.899501in}}%
\pgfpathlineto{\pgfqpoint{0.664372in}{3.820510in}}%
\pgfpathlineto{\pgfqpoint{0.676001in}{3.742374in}}%
\pgfpathlineto{\pgfqpoint{0.689180in}{3.665091in}}%
\pgfpathlineto{\pgfqpoint{0.703909in}{3.588663in}}%
\pgfpathlineto{\pgfqpoint{0.720189in}{3.513088in}}%
\pgfpathlineto{\pgfqpoint{0.738019in}{3.438368in}}%
\pgfpathlineto{\pgfqpoint{0.757400in}{3.364501in}}%
\pgfpathlineto{\pgfqpoint{0.778331in}{3.291488in}}%
\pgfpathlineto{\pgfqpoint{0.800813in}{3.219329in}}%
\pgfpathlineto{\pgfqpoint{0.824845in}{3.148025in}}%
\pgfpathlineto{\pgfqpoint{0.850428in}{3.077574in}}%
\pgfpathlineto{\pgfqpoint{0.877561in}{3.007977in}}%
\pgfpathlineto{\pgfqpoint{0.906244in}{2.939234in}}%
\pgfpathlineto{\pgfqpoint{0.936478in}{2.871345in}}%
\pgfpathlineto{\pgfqpoint{0.968263in}{2.804310in}}%
\pgfpathlineto{\pgfqpoint{1.001598in}{2.738129in}}%
\pgfpathlineto{\pgfqpoint{1.036483in}{2.672801in}}%
\pgfpathlineto{\pgfqpoint{1.072919in}{2.608328in}}%
\pgfpathlineto{\pgfqpoint{1.110905in}{2.544709in}}%
\pgfpathlineto{\pgfqpoint{1.150442in}{2.481943in}}%
\pgfpathlineto{\pgfqpoint{1.191529in}{2.420032in}}%
\pgfpathlineto{\pgfqpoint{1.234167in}{2.358975in}}%
\pgfpathlineto{\pgfqpoint{1.278355in}{2.298771in}}%
\pgfpathlineto{\pgfqpoint{1.324094in}{2.239422in}}%
\pgfpathlineto{\pgfqpoint{1.371383in}{2.180926in}}%
\pgfpathlineto{\pgfqpoint{1.420223in}{2.123284in}}%
\pgfpathlineto{\pgfqpoint{1.470613in}{2.066497in}}%
\pgfpathlineto{\pgfqpoint{1.522553in}{2.010563in}}%
\pgfpathlineto{\pgfqpoint{1.576044in}{1.955483in}}%
\pgfpathlineto{\pgfqpoint{1.631085in}{1.901257in}}%
\pgfpathlineto{\pgfqpoint{1.687677in}{1.847886in}}%
\pgfpathlineto{\pgfqpoint{1.745820in}{1.795368in}}%
\pgfpathlineto{\pgfqpoint{1.805512in}{1.743704in}}%
\pgfpathlineto{\pgfqpoint{1.866756in}{1.692894in}}%
\pgfpathlineto{\pgfqpoint{1.929549in}{1.642937in}}%
\pgfpathlineto{\pgfqpoint{1.993893in}{1.593835in}}%
\pgfpathlineto{\pgfqpoint{2.059788in}{1.545587in}}%
\pgfpathlineto{\pgfqpoint{2.127233in}{1.498193in}}%
\pgfpathlineto{\pgfqpoint{2.196229in}{1.451653in}}%
\pgfpathlineto{\pgfqpoint{2.266775in}{1.405966in}}%
\pgfpathlineto{\pgfqpoint{2.338871in}{1.361134in}}%
\pgfpathlineto{\pgfqpoint{2.412518in}{1.317156in}}%
\pgfpathlineto{\pgfqpoint{2.487716in}{1.274031in}}%
\pgfpathlineto{\pgfqpoint{2.564464in}{1.231760in}}%
\pgfpathlineto{\pgfqpoint{2.642762in}{1.190344in}}%
\pgfpathlineto{\pgfqpoint{2.722611in}{1.149781in}}%
\pgfpathlineto{\pgfqpoint{2.804010in}{1.110073in}}%
\pgfpathlineto{\pgfqpoint{2.886960in}{1.071218in}}%
\pgfpathlineto{\pgfqpoint{2.971460in}{1.033217in}}%
\pgfpathlineto{\pgfqpoint{3.057510in}{0.996070in}}%
\pgfpathlineto{\pgfqpoint{3.145112in}{0.959777in}}%
\pgfpathlineto{\pgfqpoint{3.234263in}{0.924338in}}%
\pgfpathlineto{\pgfqpoint{3.324965in}{0.889753in}}%
\pgfpathlineto{\pgfqpoint{3.389449in}{0.866091in}}%
\pgfpathlineto{\pgfqpoint{3.436350in}{0.849652in}}%
\pgfpathlineto{\pgfqpoint{3.484027in}{0.833640in}}%
\pgfpathlineto{\pgfqpoint{3.532479in}{0.818056in}}%
\pgfpathlineto{\pgfqpoint{3.581706in}{0.802898in}}%
\pgfpathlineto{\pgfqpoint{3.631709in}{0.788168in}}%
\pgfpathlineto{\pgfqpoint{3.682486in}{0.773864in}}%
\pgfpathlineto{\pgfqpoint{3.734039in}{0.759987in}}%
\pgfpathlineto{\pgfqpoint{3.786367in}{0.746537in}}%
\pgfpathlineto{\pgfqpoint{3.839470in}{0.733515in}}%
\pgfpathlineto{\pgfqpoint{3.893349in}{0.720919in}}%
\pgfpathlineto{\pgfqpoint{3.948003in}{0.708750in}}%
\pgfpathlineto{\pgfqpoint{4.003432in}{0.697008in}}%
\pgfpathlineto{\pgfqpoint{4.059636in}{0.685694in}}%
\pgfpathlineto{\pgfqpoint{4.116616in}{0.674806in}}%
\pgfpathlineto{\pgfqpoint{4.174370in}{0.664345in}}%
\pgfpathlineto{\pgfqpoint{4.232900in}{0.654311in}}%
\pgfpathlineto{\pgfqpoint{4.292205in}{0.644704in}}%
\pgfpathlineto{\pgfqpoint{4.352286in}{0.635524in}}%
\pgfpathlineto{\pgfqpoint{4.413141in}{0.626771in}}%
\pgfpathlineto{\pgfqpoint{4.474772in}{0.618445in}}%
\pgfpathlineto{\pgfqpoint{4.537178in}{0.610546in}}%
\pgfpathlineto{\pgfqpoint{4.600360in}{0.603074in}}%
\pgfpathlineto{\pgfqpoint{4.664316in}{0.596029in}}%
\pgfpathlineto{\pgfqpoint{4.729048in}{0.589411in}}%
\pgfpathlineto{\pgfqpoint{4.794555in}{0.583220in}}%
\pgfpathlineto{\pgfqpoint{4.860837in}{0.577455in}}%
\pgfpathlineto{\pgfqpoint{4.927895in}{0.572118in}}%
\pgfpathlineto{\pgfqpoint{4.995728in}{0.567208in}}%
\pgfpathlineto{\pgfqpoint{5.064335in}{0.562725in}}%
\pgfpathlineto{\pgfqpoint{5.133719in}{0.558668in}}%
\pgfpathlineto{\pgfqpoint{5.203877in}{0.555039in}}%
\pgfpathlineto{\pgfqpoint{5.274811in}{0.551837in}}%
\pgfpathlineto{\pgfqpoint{5.346520in}{0.549062in}}%
\pgfpathlineto{\pgfqpoint{5.419004in}{0.546713in}}%
\pgfpathlineto{\pgfqpoint{5.492263in}{0.544792in}}%
\pgfpathlineto{\pgfqpoint{5.566298in}{0.543297in}}%
\pgfpathlineto{\pgfqpoint{5.641107in}{0.542230in}}%
\pgfpathlineto{\pgfqpoint{5.716692in}{0.541589in}}%
\pgfpathlineto{\pgfqpoint{5.793053in}{0.541376in}}%
\pgfpathlineto{\pgfqpoint{6.826386in}{0.608332in}}%
\pgfpathlineto{\pgfqpoint{6.734754in}{0.608589in}}%
\pgfpathlineto{\pgfqpoint{6.644052in}{0.609357in}}%
\pgfpathlineto{\pgfqpoint{6.554280in}{0.610638in}}%
\pgfpathlineto{\pgfqpoint{6.465438in}{0.612431in}}%
\pgfpathlineto{\pgfqpoint{6.377527in}{0.614737in}}%
\pgfpathlineto{\pgfqpoint{6.290546in}{0.617555in}}%
\pgfpathlineto{\pgfqpoint{6.204496in}{0.620886in}}%
\pgfpathlineto{\pgfqpoint{6.119375in}{0.624728in}}%
\pgfpathlineto{\pgfqpoint{6.035185in}{0.629083in}}%
\pgfpathlineto{\pgfqpoint{5.951925in}{0.633951in}}%
\pgfpathlineto{\pgfqpoint{5.869596in}{0.639331in}}%
\pgfpathlineto{\pgfqpoint{5.788197in}{0.645223in}}%
\pgfpathlineto{\pgfqpoint{5.707728in}{0.651628in}}%
\pgfpathlineto{\pgfqpoint{5.628189in}{0.658545in}}%
\pgfpathlineto{\pgfqpoint{5.549580in}{0.665974in}}%
\pgfpathlineto{\pgfqpoint{5.471902in}{0.673916in}}%
\pgfpathlineto{\pgfqpoint{5.395154in}{0.682370in}}%
\pgfpathlineto{\pgfqpoint{5.319337in}{0.691336in}}%
\pgfpathlineto{\pgfqpoint{5.244450in}{0.700815in}}%
\pgfpathlineto{\pgfqpoint{5.170493in}{0.710807in}}%
\pgfpathlineto{\pgfqpoint{5.097466in}{0.721310in}}%
\pgfpathlineto{\pgfqpoint{5.025369in}{0.732326in}}%
\pgfpathlineto{\pgfqpoint{4.954203in}{0.743854in}}%
\pgfpathlineto{\pgfqpoint{4.883967in}{0.755895in}}%
\pgfpathlineto{\pgfqpoint{4.814661in}{0.768448in}}%
\pgfpathlineto{\pgfqpoint{4.746286in}{0.781514in}}%
\pgfpathlineto{\pgfqpoint{4.678841in}{0.795091in}}%
\pgfpathlineto{\pgfqpoint{4.612326in}{0.809182in}}%
\pgfpathlineto{\pgfqpoint{4.546742in}{0.823784in}}%
\pgfpathlineto{\pgfqpoint{4.482087in}{0.838899in}}%
\pgfpathlineto{\pgfqpoint{4.418363in}{0.854526in}}%
\pgfpathlineto{\pgfqpoint{4.355570in}{0.870666in}}%
\pgfpathlineto{\pgfqpoint{4.293706in}{0.887318in}}%
\pgfpathlineto{\pgfqpoint{4.232773in}{0.904482in}}%
\pgfpathlineto{\pgfqpoint{4.172770in}{0.922159in}}%
\pgfpathlineto{\pgfqpoint{4.113698in}{0.940348in}}%
\pgfpathlineto{\pgfqpoint{4.055555in}{0.959050in}}%
\pgfpathlineto{\pgfqpoint{3.998343in}{0.978264in}}%
\pgfpathlineto{\pgfqpoint{3.942061in}{0.997990in}}%
\pgfpathlineto{\pgfqpoint{3.864681in}{1.026385in}}%
\pgfpathlineto{\pgfqpoint{3.755839in}{1.067887in}}%
\pgfpathlineto{\pgfqpoint{3.648857in}{1.110414in}}%
\pgfpathlineto{\pgfqpoint{3.543735in}{1.153966in}}%
\pgfpathlineto{\pgfqpoint{3.440475in}{1.198542in}}%
\pgfpathlineto{\pgfqpoint{3.339074in}{1.244143in}}%
\pgfpathlineto{\pgfqpoint{3.239535in}{1.290768in}}%
\pgfpathlineto{\pgfqpoint{3.141856in}{1.338419in}}%
\pgfpathlineto{\pgfqpoint{3.046037in}{1.387094in}}%
\pgfpathlineto{\pgfqpoint{2.952079in}{1.436794in}}%
\pgfpathlineto{\pgfqpoint{2.859982in}{1.487519in}}%
\pgfpathlineto{\pgfqpoint{2.769745in}{1.539268in}}%
\pgfpathlineto{\pgfqpoint{2.681368in}{1.592042in}}%
\pgfpathlineto{\pgfqpoint{2.594853in}{1.645841in}}%
\pgfpathlineto{\pgfqpoint{2.510197in}{1.700665in}}%
\pgfpathlineto{\pgfqpoint{2.427403in}{1.756513in}}%
\pgfpathlineto{\pgfqpoint{2.346469in}{1.813386in}}%
\pgfpathlineto{\pgfqpoint{2.267395in}{1.871284in}}%
\pgfpathlineto{\pgfqpoint{2.190182in}{1.930206in}}%
\pgfpathlineto{\pgfqpoint{2.114830in}{1.990154in}}%
\pgfpathlineto{\pgfqpoint{2.041338in}{2.051126in}}%
\pgfpathlineto{\pgfqpoint{1.969706in}{2.113122in}}%
\pgfpathlineto{\pgfqpoint{1.899935in}{2.176144in}}%
\pgfpathlineto{\pgfqpoint{1.832025in}{2.240190in}}%
\pgfpathlineto{\pgfqpoint{1.765976in}{2.305261in}}%
\pgfpathlineto{\pgfqpoint{1.701786in}{2.371357in}}%
\pgfpathlineto{\pgfqpoint{1.639458in}{2.438477in}}%
\pgfpathlineto{\pgfqpoint{1.578990in}{2.506623in}}%
\pgfpathlineto{\pgfqpoint{1.520382in}{2.575793in}}%
\pgfpathlineto{\pgfqpoint{1.463635in}{2.645987in}}%
\pgfpathlineto{\pgfqpoint{1.408749in}{2.717207in}}%
\pgfpathlineto{\pgfqpoint{1.355723in}{2.789451in}}%
\pgfpathlineto{\pgfqpoint{1.304558in}{2.862720in}}%
\pgfpathlineto{\pgfqpoint{1.255253in}{2.937014in}}%
\pgfpathlineto{\pgfqpoint{1.207809in}{3.012332in}}%
\pgfpathlineto{\pgfqpoint{1.162226in}{3.088675in}}%
\pgfpathlineto{\pgfqpoint{1.118503in}{3.166043in}}%
\pgfpathlineto{\pgfqpoint{1.076640in}{3.244436in}}%
\pgfpathlineto{\pgfqpoint{1.036638in}{3.323853in}}%
\pgfpathlineto{\pgfqpoint{0.998497in}{3.404295in}}%
\pgfpathlineto{\pgfqpoint{0.962216in}{3.485762in}}%
\pgfpathlineto{\pgfqpoint{0.927796in}{3.568253in}}%
\pgfpathlineto{\pgfqpoint{0.895236in}{3.651770in}}%
\pgfpathlineto{\pgfqpoint{0.864537in}{3.736311in}}%
\pgfpathlineto{\pgfqpoint{0.835698in}{3.821877in}}%
\pgfpathlineto{\pgfqpoint{0.808720in}{3.908467in}}%
\pgfpathlineto{\pgfqpoint{0.783603in}{3.996082in}}%
\pgfpathlineto{\pgfqpoint{0.760346in}{4.084722in}}%
\pgfpathlineto{\pgfqpoint{0.738950in}{4.174387in}}%
\pgfpathlineto{\pgfqpoint{0.719414in}{4.265077in}}%
\pgfpathlineto{\pgfqpoint{0.701738in}{4.356791in}}%
\pgfpathlineto{\pgfqpoint{0.685924in}{4.449530in}}%
\pgfpathlineto{\pgfqpoint{0.671970in}{4.543294in}}%
\pgfpathlineto{\pgfqpoint{0.659876in}{4.638082in}}%
\pgfpathlineto{\pgfqpoint{0.649643in}{4.733895in}}%
\pgfpathlineto{\pgfqpoint{0.641270in}{4.830733in}}%
\pgfpathlineto{\pgfqpoint{0.634758in}{4.928596in}}%
\pgfpathlineto{\pgfqpoint{0.630107in}{5.027483in}}%
\pgfpathlineto{\pgfqpoint{0.627316in}{5.127396in}}%
\pgfpathlineto{\pgfqpoint{0.626386in}{5.228333in}}%
\pgfpathlineto{\pgfqpoint{0.626386in}{4.391376in}}%
\pgfpathclose%
\pgfusepath{stroke,fill}%
\end{pgfscope}%
\begin{pgfscope}%
\pgfpathrectangle{\pgfqpoint{0.626386in}{0.608332in}}{\pgfqpoint{6.200000in}{4.620000in}}%
\pgfusepath{clip}%
\pgfsetbuttcap%
\pgfsetroundjoin%
\definecolor{currentfill}{rgb}{1.000000,1.000000,1.000000}%
\pgfsetfillcolor{currentfill}%
\pgfsetlinewidth{1.003750pt}%
\definecolor{currentstroke}{rgb}{1.000000,1.000000,1.000000}%
\pgfsetstrokecolor{currentstroke}%
\pgfsetdash{}{0pt}%
\pgfsys@defobject{currentmarker}{\pgfqpoint{0.626386in}{0.206593in}}{\pgfqpoint{5.689719in}{4.307680in}}{%
\pgfpathmoveto{\pgfqpoint{0.626386in}{0.206593in}}%
\pgfpathlineto{\pgfqpoint{0.626386in}{4.307680in}}%
\pgfpathlineto{\pgfqpoint{0.627146in}{4.225249in}}%
\pgfpathlineto{\pgfqpoint{0.629425in}{4.143654in}}%
\pgfpathlineto{\pgfqpoint{0.633224in}{4.062896in}}%
\pgfpathlineto{\pgfqpoint{0.638542in}{3.982974in}}%
\pgfpathlineto{\pgfqpoint{0.645379in}{3.903890in}}%
\pgfpathlineto{\pgfqpoint{0.653736in}{3.825643in}}%
\pgfpathlineto{\pgfqpoint{0.663613in}{3.748232in}}%
\pgfpathlineto{\pgfqpoint{0.675008in}{3.671658in}}%
\pgfpathlineto{\pgfqpoint{0.687924in}{3.595921in}}%
\pgfpathlineto{\pgfqpoint{0.702359in}{3.521021in}}%
\pgfpathlineto{\pgfqpoint{0.718313in}{3.446958in}}%
\pgfpathlineto{\pgfqpoint{0.735787in}{3.373732in}}%
\pgfpathlineto{\pgfqpoint{0.754780in}{3.301343in}}%
\pgfpathlineto{\pgfqpoint{0.775292in}{3.229790in}}%
\pgfpathlineto{\pgfqpoint{0.797324in}{3.159075in}}%
\pgfpathlineto{\pgfqpoint{0.820876in}{3.089196in}}%
\pgfpathlineto{\pgfqpoint{0.845947in}{3.020154in}}%
\pgfpathlineto{\pgfqpoint{0.872537in}{2.951949in}}%
\pgfpathlineto{\pgfqpoint{0.900647in}{2.884581in}}%
\pgfpathlineto{\pgfqpoint{0.930277in}{2.818050in}}%
\pgfpathlineto{\pgfqpoint{0.961425in}{2.752355in}}%
\pgfpathlineto{\pgfqpoint{0.994094in}{2.687498in}}%
\pgfpathlineto{\pgfqpoint{1.028281in}{2.623477in}}%
\pgfpathlineto{\pgfqpoint{1.063988in}{2.560293in}}%
\pgfpathlineto{\pgfqpoint{1.101215in}{2.497947in}}%
\pgfpathlineto{\pgfqpoint{1.139961in}{2.436436in}}%
\pgfpathlineto{\pgfqpoint{1.180227in}{2.375763in}}%
\pgfpathlineto{\pgfqpoint{1.222011in}{2.315927in}}%
\pgfpathlineto{\pgfqpoint{1.265316in}{2.256928in}}%
\pgfpathlineto{\pgfqpoint{1.310140in}{2.198765in}}%
\pgfpathlineto{\pgfqpoint{1.356483in}{2.141439in}}%
\pgfpathlineto{\pgfqpoint{1.404346in}{2.084951in}}%
\pgfpathlineto{\pgfqpoint{1.453728in}{2.029299in}}%
\pgfpathlineto{\pgfqpoint{1.504630in}{1.974484in}}%
\pgfpathlineto{\pgfqpoint{1.557051in}{1.920505in}}%
\pgfpathlineto{\pgfqpoint{1.610991in}{1.867364in}}%
\pgfpathlineto{\pgfqpoint{1.666451in}{1.815060in}}%
\pgfpathlineto{\pgfqpoint{1.723431in}{1.763592in}}%
\pgfpathlineto{\pgfqpoint{1.781930in}{1.712961in}}%
\pgfpathlineto{\pgfqpoint{1.841948in}{1.663168in}}%
\pgfpathlineto{\pgfqpoint{1.903486in}{1.614211in}}%
\pgfpathlineto{\pgfqpoint{1.966543in}{1.566091in}}%
\pgfpathlineto{\pgfqpoint{2.031120in}{1.518807in}}%
\pgfpathlineto{\pgfqpoint{2.097216in}{1.472361in}}%
\pgfpathlineto{\pgfqpoint{2.164832in}{1.426751in}}%
\pgfpathlineto{\pgfqpoint{2.233967in}{1.381979in}}%
\pgfpathlineto{\pgfqpoint{2.304622in}{1.338043in}}%
\pgfpathlineto{\pgfqpoint{2.376796in}{1.294944in}}%
\pgfpathlineto{\pgfqpoint{2.450489in}{1.252682in}}%
\pgfpathlineto{\pgfqpoint{2.525702in}{1.211257in}}%
\pgfpathlineto{\pgfqpoint{2.602434in}{1.170669in}}%
\pgfpathlineto{\pgfqpoint{2.680686in}{1.130918in}}%
\pgfpathlineto{\pgfqpoint{2.760457in}{1.092003in}}%
\pgfpathlineto{\pgfqpoint{2.841748in}{1.053925in}}%
\pgfpathlineto{\pgfqpoint{2.924558in}{1.016685in}}%
\pgfpathlineto{\pgfqpoint{3.008888in}{0.980281in}}%
\pgfpathlineto{\pgfqpoint{3.094737in}{0.944714in}}%
\pgfpathlineto{\pgfqpoint{3.182106in}{0.909984in}}%
\pgfpathlineto{\pgfqpoint{3.270994in}{0.876090in}}%
\pgfpathlineto{\pgfqpoint{3.334188in}{0.852901in}}%
\pgfpathlineto{\pgfqpoint{3.380151in}{0.836791in}}%
\pgfpathlineto{\pgfqpoint{3.426874in}{0.821099in}}%
\pgfpathlineto{\pgfqpoint{3.474357in}{0.805827in}}%
\pgfpathlineto{\pgfqpoint{3.522600in}{0.790972in}}%
\pgfpathlineto{\pgfqpoint{3.571602in}{0.776536in}}%
\pgfpathlineto{\pgfqpoint{3.621364in}{0.762518in}}%
\pgfpathlineto{\pgfqpoint{3.671886in}{0.748919in}}%
\pgfpathlineto{\pgfqpoint{3.723168in}{0.735739in}}%
\pgfpathlineto{\pgfqpoint{3.775209in}{0.722976in}}%
\pgfpathlineto{\pgfqpoint{3.828010in}{0.710632in}}%
\pgfpathlineto{\pgfqpoint{3.881570in}{0.698707in}}%
\pgfpathlineto{\pgfqpoint{3.935891in}{0.687200in}}%
\pgfpathlineto{\pgfqpoint{3.990971in}{0.676112in}}%
\pgfpathlineto{\pgfqpoint{4.046811in}{0.665441in}}%
\pgfpathlineto{\pgfqpoint{4.103411in}{0.655190in}}%
\pgfpathlineto{\pgfqpoint{4.160770in}{0.645357in}}%
\pgfpathlineto{\pgfqpoint{4.218889in}{0.635942in}}%
\pgfpathlineto{\pgfqpoint{4.277768in}{0.626945in}}%
\pgfpathlineto{\pgfqpoint{4.337406in}{0.618367in}}%
\pgfpathlineto{\pgfqpoint{4.397805in}{0.610208in}}%
\pgfpathlineto{\pgfqpoint{4.458963in}{0.602467in}}%
\pgfpathlineto{\pgfqpoint{4.520880in}{0.595144in}}%
\pgfpathlineto{\pgfqpoint{4.583558in}{0.588240in}}%
\pgfpathlineto{\pgfqpoint{4.646995in}{0.581754in}}%
\pgfpathlineto{\pgfqpoint{4.711192in}{0.575687in}}%
\pgfpathlineto{\pgfqpoint{4.776148in}{0.570038in}}%
\pgfpathlineto{\pgfqpoint{4.841865in}{0.564808in}}%
\pgfpathlineto{\pgfqpoint{4.908341in}{0.559996in}}%
\pgfpathlineto{\pgfqpoint{4.975576in}{0.555602in}}%
\pgfpathlineto{\pgfqpoint{5.043572in}{0.551627in}}%
\pgfpathlineto{\pgfqpoint{5.112327in}{0.548070in}}%
\pgfpathlineto{\pgfqpoint{5.181842in}{0.544932in}}%
\pgfpathlineto{\pgfqpoint{5.252117in}{0.542212in}}%
\pgfpathlineto{\pgfqpoint{5.323151in}{0.539911in}}%
\pgfpathlineto{\pgfqpoint{5.394945in}{0.538028in}}%
\pgfpathlineto{\pgfqpoint{5.467499in}{0.536563in}}%
\pgfpathlineto{\pgfqpoint{5.540813in}{0.535517in}}%
\pgfpathlineto{\pgfqpoint{5.614886in}{0.534890in}}%
\pgfpathlineto{\pgfqpoint{5.689719in}{0.534680in}}%
\pgfpathlineto{\pgfqpoint{5.689719in}{0.206593in}}%
\pgfpathlineto{\pgfqpoint{5.689719in}{0.206593in}}%
\pgfpathlineto{\pgfqpoint{5.614886in}{0.206593in}}%
\pgfpathlineto{\pgfqpoint{5.540813in}{0.206593in}}%
\pgfpathlineto{\pgfqpoint{5.467499in}{0.206593in}}%
\pgfpathlineto{\pgfqpoint{5.394945in}{0.206593in}}%
\pgfpathlineto{\pgfqpoint{5.323151in}{0.206593in}}%
\pgfpathlineto{\pgfqpoint{5.252117in}{0.206593in}}%
\pgfpathlineto{\pgfqpoint{5.181842in}{0.206593in}}%
\pgfpathlineto{\pgfqpoint{5.112327in}{0.206593in}}%
\pgfpathlineto{\pgfqpoint{5.043572in}{0.206593in}}%
\pgfpathlineto{\pgfqpoint{4.975576in}{0.206593in}}%
\pgfpathlineto{\pgfqpoint{4.908341in}{0.206593in}}%
\pgfpathlineto{\pgfqpoint{4.841865in}{0.206593in}}%
\pgfpathlineto{\pgfqpoint{4.776148in}{0.206593in}}%
\pgfpathlineto{\pgfqpoint{4.711192in}{0.206593in}}%
\pgfpathlineto{\pgfqpoint{4.646995in}{0.206593in}}%
\pgfpathlineto{\pgfqpoint{4.583558in}{0.206593in}}%
\pgfpathlineto{\pgfqpoint{4.520880in}{0.206593in}}%
\pgfpathlineto{\pgfqpoint{4.458963in}{0.206593in}}%
\pgfpathlineto{\pgfqpoint{4.397805in}{0.206593in}}%
\pgfpathlineto{\pgfqpoint{4.337406in}{0.206593in}}%
\pgfpathlineto{\pgfqpoint{4.277768in}{0.206593in}}%
\pgfpathlineto{\pgfqpoint{4.218889in}{0.206593in}}%
\pgfpathlineto{\pgfqpoint{4.160770in}{0.206593in}}%
\pgfpathlineto{\pgfqpoint{4.103411in}{0.206593in}}%
\pgfpathlineto{\pgfqpoint{4.046811in}{0.206593in}}%
\pgfpathlineto{\pgfqpoint{3.990971in}{0.206593in}}%
\pgfpathlineto{\pgfqpoint{3.935891in}{0.206593in}}%
\pgfpathlineto{\pgfqpoint{3.881570in}{0.206593in}}%
\pgfpathlineto{\pgfqpoint{3.828010in}{0.206593in}}%
\pgfpathlineto{\pgfqpoint{3.775209in}{0.206593in}}%
\pgfpathlineto{\pgfqpoint{3.723168in}{0.206593in}}%
\pgfpathlineto{\pgfqpoint{3.671886in}{0.206593in}}%
\pgfpathlineto{\pgfqpoint{3.621364in}{0.206593in}}%
\pgfpathlineto{\pgfqpoint{3.571602in}{0.206593in}}%
\pgfpathlineto{\pgfqpoint{3.522600in}{0.206593in}}%
\pgfpathlineto{\pgfqpoint{3.474357in}{0.206593in}}%
\pgfpathlineto{\pgfqpoint{3.426874in}{0.206593in}}%
\pgfpathlineto{\pgfqpoint{3.380151in}{0.206593in}}%
\pgfpathlineto{\pgfqpoint{3.334188in}{0.206593in}}%
\pgfpathlineto{\pgfqpoint{3.270994in}{0.206593in}}%
\pgfpathlineto{\pgfqpoint{3.182106in}{0.206593in}}%
\pgfpathlineto{\pgfqpoint{3.094737in}{0.206593in}}%
\pgfpathlineto{\pgfqpoint{3.008888in}{0.206593in}}%
\pgfpathlineto{\pgfqpoint{2.924558in}{0.206593in}}%
\pgfpathlineto{\pgfqpoint{2.841748in}{0.206593in}}%
\pgfpathlineto{\pgfqpoint{2.760457in}{0.206593in}}%
\pgfpathlineto{\pgfqpoint{2.680686in}{0.206593in}}%
\pgfpathlineto{\pgfqpoint{2.602434in}{0.206593in}}%
\pgfpathlineto{\pgfqpoint{2.525702in}{0.206593in}}%
\pgfpathlineto{\pgfqpoint{2.450489in}{0.206593in}}%
\pgfpathlineto{\pgfqpoint{2.376796in}{0.206593in}}%
\pgfpathlineto{\pgfqpoint{2.304622in}{0.206593in}}%
\pgfpathlineto{\pgfqpoint{2.233967in}{0.206593in}}%
\pgfpathlineto{\pgfqpoint{2.164832in}{0.206593in}}%
\pgfpathlineto{\pgfqpoint{2.097216in}{0.206593in}}%
\pgfpathlineto{\pgfqpoint{2.031120in}{0.206593in}}%
\pgfpathlineto{\pgfqpoint{1.966543in}{0.206593in}}%
\pgfpathlineto{\pgfqpoint{1.903486in}{0.206593in}}%
\pgfpathlineto{\pgfqpoint{1.841948in}{0.206593in}}%
\pgfpathlineto{\pgfqpoint{1.781930in}{0.206593in}}%
\pgfpathlineto{\pgfqpoint{1.723431in}{0.206593in}}%
\pgfpathlineto{\pgfqpoint{1.666451in}{0.206593in}}%
\pgfpathlineto{\pgfqpoint{1.610991in}{0.206593in}}%
\pgfpathlineto{\pgfqpoint{1.557051in}{0.206593in}}%
\pgfpathlineto{\pgfqpoint{1.504630in}{0.206593in}}%
\pgfpathlineto{\pgfqpoint{1.453728in}{0.206593in}}%
\pgfpathlineto{\pgfqpoint{1.404346in}{0.206593in}}%
\pgfpathlineto{\pgfqpoint{1.356483in}{0.206593in}}%
\pgfpathlineto{\pgfqpoint{1.310140in}{0.206593in}}%
\pgfpathlineto{\pgfqpoint{1.265316in}{0.206593in}}%
\pgfpathlineto{\pgfqpoint{1.222011in}{0.206593in}}%
\pgfpathlineto{\pgfqpoint{1.180227in}{0.206593in}}%
\pgfpathlineto{\pgfqpoint{1.139961in}{0.206593in}}%
\pgfpathlineto{\pgfqpoint{1.101215in}{0.206593in}}%
\pgfpathlineto{\pgfqpoint{1.063988in}{0.206593in}}%
\pgfpathlineto{\pgfqpoint{1.028281in}{0.206593in}}%
\pgfpathlineto{\pgfqpoint{0.994094in}{0.206593in}}%
\pgfpathlineto{\pgfqpoint{0.961425in}{0.206593in}}%
\pgfpathlineto{\pgfqpoint{0.930277in}{0.206593in}}%
\pgfpathlineto{\pgfqpoint{0.900647in}{0.206593in}}%
\pgfpathlineto{\pgfqpoint{0.872537in}{0.206593in}}%
\pgfpathlineto{\pgfqpoint{0.845947in}{0.206593in}}%
\pgfpathlineto{\pgfqpoint{0.820876in}{0.206593in}}%
\pgfpathlineto{\pgfqpoint{0.797324in}{0.206593in}}%
\pgfpathlineto{\pgfqpoint{0.775292in}{0.206593in}}%
\pgfpathlineto{\pgfqpoint{0.754780in}{0.206593in}}%
\pgfpathlineto{\pgfqpoint{0.735787in}{0.206593in}}%
\pgfpathlineto{\pgfqpoint{0.718313in}{0.206593in}}%
\pgfpathlineto{\pgfqpoint{0.702359in}{0.206593in}}%
\pgfpathlineto{\pgfqpoint{0.687924in}{0.206593in}}%
\pgfpathlineto{\pgfqpoint{0.675008in}{0.206593in}}%
\pgfpathlineto{\pgfqpoint{0.663613in}{0.206593in}}%
\pgfpathlineto{\pgfqpoint{0.653736in}{0.206593in}}%
\pgfpathlineto{\pgfqpoint{0.645379in}{0.206593in}}%
\pgfpathlineto{\pgfqpoint{0.638542in}{0.206593in}}%
\pgfpathlineto{\pgfqpoint{0.633224in}{0.206593in}}%
\pgfpathlineto{\pgfqpoint{0.629425in}{0.206593in}}%
\pgfpathlineto{\pgfqpoint{0.627146in}{0.206593in}}%
\pgfpathlineto{\pgfqpoint{0.626386in}{0.206593in}}%
\pgfpathlineto{\pgfqpoint{0.626386in}{0.206593in}}%
\pgfpathclose%
\pgfusepath{stroke,fill}%
}%
\begin{pgfscope}%
\pgfsys@transformshift{0.000000in}{0.000000in}%
\pgfsys@useobject{currentmarker}{}%
\end{pgfscope}%
\end{pgfscope}%
\begin{pgfscope}%
\pgfsetbuttcap%
\pgfsetroundjoin%
\definecolor{currentfill}{rgb}{0.000000,0.000000,0.000000}%
\pgfsetfillcolor{currentfill}%
\pgfsetlinewidth{0.803000pt}%
\definecolor{currentstroke}{rgb}{0.000000,0.000000,0.000000}%
\pgfsetstrokecolor{currentstroke}%
\pgfsetdash{}{0pt}%
\pgfsys@defobject{currentmarker}{\pgfqpoint{0.000000in}{-0.048611in}}{\pgfqpoint{0.000000in}{0.000000in}}{%
\pgfpathmoveto{\pgfqpoint{0.000000in}{0.000000in}}%
\pgfpathlineto{\pgfqpoint{0.000000in}{-0.048611in}}%
\pgfusepath{stroke,fill}%
}%
\begin{pgfscope}%
\pgfsys@transformshift{0.626386in}{0.608332in}%
\pgfsys@useobject{currentmarker}{}%
\end{pgfscope}%
\end{pgfscope}%
\begin{pgfscope}%
\definecolor{textcolor}{rgb}{0.000000,0.000000,0.000000}%
\pgfsetstrokecolor{textcolor}%
\pgfsetfillcolor{textcolor}%
\pgftext[x=0.626386in,y=0.511110in,,top]{\color{textcolor}{\rmfamily\fontsize{14.000000}{16.800000}\selectfont\catcode`\^=\active\def^{\ifmmode\sp\else\^{}\fi}\catcode`\%=\active\def%{\%}$\mathdefault{0}$}}%
\end{pgfscope}%
\begin{pgfscope}%
\pgfsetbuttcap%
\pgfsetroundjoin%
\definecolor{currentfill}{rgb}{0.000000,0.000000,0.000000}%
\pgfsetfillcolor{currentfill}%
\pgfsetlinewidth{0.803000pt}%
\definecolor{currentstroke}{rgb}{0.000000,0.000000,0.000000}%
\pgfsetstrokecolor{currentstroke}%
\pgfsetdash{}{0pt}%
\pgfsys@defobject{currentmarker}{\pgfqpoint{0.000000in}{-0.048611in}}{\pgfqpoint{0.000000in}{0.000000in}}{%
\pgfpathmoveto{\pgfqpoint{0.000000in}{0.000000in}}%
\pgfpathlineto{\pgfqpoint{0.000000in}{-0.048611in}}%
\pgfusepath{stroke,fill}%
}%
\begin{pgfscope}%
\pgfsys@transformshift{1.386190in}{0.608332in}%
\pgfsys@useobject{currentmarker}{}%
\end{pgfscope}%
\end{pgfscope}%
\begin{pgfscope}%
\definecolor{textcolor}{rgb}{0.000000,0.000000,0.000000}%
\pgfsetstrokecolor{textcolor}%
\pgfsetfillcolor{textcolor}%
\pgftext[x=1.386190in,y=0.511110in,,top]{\color{textcolor}{\rmfamily\fontsize{14.000000}{16.800000}\selectfont\catcode`\^=\active\def^{\ifmmode\sp\else\^{}\fi}\catcode`\%=\active\def%{\%}$\mathdefault{20}$}}%
\end{pgfscope}%
\begin{pgfscope}%
\pgfsetbuttcap%
\pgfsetroundjoin%
\definecolor{currentfill}{rgb}{0.000000,0.000000,0.000000}%
\pgfsetfillcolor{currentfill}%
\pgfsetlinewidth{0.803000pt}%
\definecolor{currentstroke}{rgb}{0.000000,0.000000,0.000000}%
\pgfsetstrokecolor{currentstroke}%
\pgfsetdash{}{0pt}%
\pgfsys@defobject{currentmarker}{\pgfqpoint{0.000000in}{-0.048611in}}{\pgfqpoint{0.000000in}{0.000000in}}{%
\pgfpathmoveto{\pgfqpoint{0.000000in}{0.000000in}}%
\pgfpathlineto{\pgfqpoint{0.000000in}{-0.048611in}}%
\pgfusepath{stroke,fill}%
}%
\begin{pgfscope}%
\pgfsys@transformshift{2.145994in}{0.608332in}%
\pgfsys@useobject{currentmarker}{}%
\end{pgfscope}%
\end{pgfscope}%
\begin{pgfscope}%
\definecolor{textcolor}{rgb}{0.000000,0.000000,0.000000}%
\pgfsetstrokecolor{textcolor}%
\pgfsetfillcolor{textcolor}%
\pgftext[x=2.145994in,y=0.511110in,,top]{\color{textcolor}{\rmfamily\fontsize{14.000000}{16.800000}\selectfont\catcode`\^=\active\def^{\ifmmode\sp\else\^{}\fi}\catcode`\%=\active\def%{\%}$\mathdefault{40}$}}%
\end{pgfscope}%
\begin{pgfscope}%
\pgfsetbuttcap%
\pgfsetroundjoin%
\definecolor{currentfill}{rgb}{0.000000,0.000000,0.000000}%
\pgfsetfillcolor{currentfill}%
\pgfsetlinewidth{0.803000pt}%
\definecolor{currentstroke}{rgb}{0.000000,0.000000,0.000000}%
\pgfsetstrokecolor{currentstroke}%
\pgfsetdash{}{0pt}%
\pgfsys@defobject{currentmarker}{\pgfqpoint{0.000000in}{-0.048611in}}{\pgfqpoint{0.000000in}{0.000000in}}{%
\pgfpathmoveto{\pgfqpoint{0.000000in}{0.000000in}}%
\pgfpathlineto{\pgfqpoint{0.000000in}{-0.048611in}}%
\pgfusepath{stroke,fill}%
}%
\begin{pgfscope}%
\pgfsys@transformshift{2.905798in}{0.608332in}%
\pgfsys@useobject{currentmarker}{}%
\end{pgfscope}%
\end{pgfscope}%
\begin{pgfscope}%
\definecolor{textcolor}{rgb}{0.000000,0.000000,0.000000}%
\pgfsetstrokecolor{textcolor}%
\pgfsetfillcolor{textcolor}%
\pgftext[x=2.905798in,y=0.511110in,,top]{\color{textcolor}{\rmfamily\fontsize{14.000000}{16.800000}\selectfont\catcode`\^=\active\def^{\ifmmode\sp\else\^{}\fi}\catcode`\%=\active\def%{\%}$\mathdefault{60}$}}%
\end{pgfscope}%
\begin{pgfscope}%
\pgfsetbuttcap%
\pgfsetroundjoin%
\definecolor{currentfill}{rgb}{0.000000,0.000000,0.000000}%
\pgfsetfillcolor{currentfill}%
\pgfsetlinewidth{0.803000pt}%
\definecolor{currentstroke}{rgb}{0.000000,0.000000,0.000000}%
\pgfsetstrokecolor{currentstroke}%
\pgfsetdash{}{0pt}%
\pgfsys@defobject{currentmarker}{\pgfqpoint{0.000000in}{-0.048611in}}{\pgfqpoint{0.000000in}{0.000000in}}{%
\pgfpathmoveto{\pgfqpoint{0.000000in}{0.000000in}}%
\pgfpathlineto{\pgfqpoint{0.000000in}{-0.048611in}}%
\pgfusepath{stroke,fill}%
}%
\begin{pgfscope}%
\pgfsys@transformshift{3.665602in}{0.608332in}%
\pgfsys@useobject{currentmarker}{}%
\end{pgfscope}%
\end{pgfscope}%
\begin{pgfscope}%
\definecolor{textcolor}{rgb}{0.000000,0.000000,0.000000}%
\pgfsetstrokecolor{textcolor}%
\pgfsetfillcolor{textcolor}%
\pgftext[x=3.665602in,y=0.511110in,,top]{\color{textcolor}{\rmfamily\fontsize{14.000000}{16.800000}\selectfont\catcode`\^=\active\def^{\ifmmode\sp\else\^{}\fi}\catcode`\%=\active\def%{\%}$\mathdefault{80}$}}%
\end{pgfscope}%
\begin{pgfscope}%
\pgfsetbuttcap%
\pgfsetroundjoin%
\definecolor{currentfill}{rgb}{0.000000,0.000000,0.000000}%
\pgfsetfillcolor{currentfill}%
\pgfsetlinewidth{0.803000pt}%
\definecolor{currentstroke}{rgb}{0.000000,0.000000,0.000000}%
\pgfsetstrokecolor{currentstroke}%
\pgfsetdash{}{0pt}%
\pgfsys@defobject{currentmarker}{\pgfqpoint{0.000000in}{-0.048611in}}{\pgfqpoint{0.000000in}{0.000000in}}{%
\pgfpathmoveto{\pgfqpoint{0.000000in}{0.000000in}}%
\pgfpathlineto{\pgfqpoint{0.000000in}{-0.048611in}}%
\pgfusepath{stroke,fill}%
}%
\begin{pgfscope}%
\pgfsys@transformshift{4.425406in}{0.608332in}%
\pgfsys@useobject{currentmarker}{}%
\end{pgfscope}%
\end{pgfscope}%
\begin{pgfscope}%
\definecolor{textcolor}{rgb}{0.000000,0.000000,0.000000}%
\pgfsetstrokecolor{textcolor}%
\pgfsetfillcolor{textcolor}%
\pgftext[x=4.425406in,y=0.511110in,,top]{\color{textcolor}{\rmfamily\fontsize{14.000000}{16.800000}\selectfont\catcode`\^=\active\def^{\ifmmode\sp\else\^{}\fi}\catcode`\%=\active\def%{\%}$\mathdefault{100}$}}%
\end{pgfscope}%
\begin{pgfscope}%
\pgfsetbuttcap%
\pgfsetroundjoin%
\definecolor{currentfill}{rgb}{0.000000,0.000000,0.000000}%
\pgfsetfillcolor{currentfill}%
\pgfsetlinewidth{0.803000pt}%
\definecolor{currentstroke}{rgb}{0.000000,0.000000,0.000000}%
\pgfsetstrokecolor{currentstroke}%
\pgfsetdash{}{0pt}%
\pgfsys@defobject{currentmarker}{\pgfqpoint{0.000000in}{-0.048611in}}{\pgfqpoint{0.000000in}{0.000000in}}{%
\pgfpathmoveto{\pgfqpoint{0.000000in}{0.000000in}}%
\pgfpathlineto{\pgfqpoint{0.000000in}{-0.048611in}}%
\pgfusepath{stroke,fill}%
}%
\begin{pgfscope}%
\pgfsys@transformshift{5.185210in}{0.608332in}%
\pgfsys@useobject{currentmarker}{}%
\end{pgfscope}%
\end{pgfscope}%
\begin{pgfscope}%
\definecolor{textcolor}{rgb}{0.000000,0.000000,0.000000}%
\pgfsetstrokecolor{textcolor}%
\pgfsetfillcolor{textcolor}%
\pgftext[x=5.185210in,y=0.511110in,,top]{\color{textcolor}{\rmfamily\fontsize{14.000000}{16.800000}\selectfont\catcode`\^=\active\def^{\ifmmode\sp\else\^{}\fi}\catcode`\%=\active\def%{\%}$\mathdefault{120}$}}%
\end{pgfscope}%
\begin{pgfscope}%
\pgfsetbuttcap%
\pgfsetroundjoin%
\definecolor{currentfill}{rgb}{0.000000,0.000000,0.000000}%
\pgfsetfillcolor{currentfill}%
\pgfsetlinewidth{0.803000pt}%
\definecolor{currentstroke}{rgb}{0.000000,0.000000,0.000000}%
\pgfsetstrokecolor{currentstroke}%
\pgfsetdash{}{0pt}%
\pgfsys@defobject{currentmarker}{\pgfqpoint{0.000000in}{-0.048611in}}{\pgfqpoint{0.000000in}{0.000000in}}{%
\pgfpathmoveto{\pgfqpoint{0.000000in}{0.000000in}}%
\pgfpathlineto{\pgfqpoint{0.000000in}{-0.048611in}}%
\pgfusepath{stroke,fill}%
}%
\begin{pgfscope}%
\pgfsys@transformshift{5.945013in}{0.608332in}%
\pgfsys@useobject{currentmarker}{}%
\end{pgfscope}%
\end{pgfscope}%
\begin{pgfscope}%
\definecolor{textcolor}{rgb}{0.000000,0.000000,0.000000}%
\pgfsetstrokecolor{textcolor}%
\pgfsetfillcolor{textcolor}%
\pgftext[x=5.945013in,y=0.511110in,,top]{\color{textcolor}{\rmfamily\fontsize{14.000000}{16.800000}\selectfont\catcode`\^=\active\def^{\ifmmode\sp\else\^{}\fi}\catcode`\%=\active\def%{\%}$\mathdefault{140}$}}%
\end{pgfscope}%
\begin{pgfscope}%
\pgfsetbuttcap%
\pgfsetroundjoin%
\definecolor{currentfill}{rgb}{0.000000,0.000000,0.000000}%
\pgfsetfillcolor{currentfill}%
\pgfsetlinewidth{0.803000pt}%
\definecolor{currentstroke}{rgb}{0.000000,0.000000,0.000000}%
\pgfsetstrokecolor{currentstroke}%
\pgfsetdash{}{0pt}%
\pgfsys@defobject{currentmarker}{\pgfqpoint{0.000000in}{-0.048611in}}{\pgfqpoint{0.000000in}{0.000000in}}{%
\pgfpathmoveto{\pgfqpoint{0.000000in}{0.000000in}}%
\pgfpathlineto{\pgfqpoint{0.000000in}{-0.048611in}}%
\pgfusepath{stroke,fill}%
}%
\begin{pgfscope}%
\pgfsys@transformshift{6.704817in}{0.608332in}%
\pgfsys@useobject{currentmarker}{}%
\end{pgfscope}%
\end{pgfscope}%
\begin{pgfscope}%
\definecolor{textcolor}{rgb}{0.000000,0.000000,0.000000}%
\pgfsetstrokecolor{textcolor}%
\pgfsetfillcolor{textcolor}%
\pgftext[x=6.704817in,y=0.511110in,,top]{\color{textcolor}{\rmfamily\fontsize{14.000000}{16.800000}\selectfont\catcode`\^=\active\def^{\ifmmode\sp\else\^{}\fi}\catcode`\%=\active\def%{\%}$\mathdefault{160}$}}%
\end{pgfscope}%
\begin{pgfscope}%
\definecolor{textcolor}{rgb}{0.000000,0.000000,0.000000}%
\pgfsetstrokecolor{textcolor}%
\pgfsetfillcolor{textcolor}%
\pgftext[x=3.726386in,y=0.277777in,,top]{\color{textcolor}{\rmfamily\fontsize{14.000000}{16.800000}\selectfont\catcode`\^=\active\def^{\ifmmode\sp\else\^{}\fi}\catcode`\%=\active\def%{\%}f1}}%
\end{pgfscope}%
\begin{pgfscope}%
\pgfsetbuttcap%
\pgfsetroundjoin%
\definecolor{currentfill}{rgb}{0.000000,0.000000,0.000000}%
\pgfsetfillcolor{currentfill}%
\pgfsetlinewidth{0.803000pt}%
\definecolor{currentstroke}{rgb}{0.000000,0.000000,0.000000}%
\pgfsetstrokecolor{currentstroke}%
\pgfsetdash{}{0pt}%
\pgfsys@defobject{currentmarker}{\pgfqpoint{-0.048611in}{0.000000in}}{\pgfqpoint{-0.000000in}{0.000000in}}{%
\pgfpathmoveto{\pgfqpoint{-0.000000in}{0.000000in}}%
\pgfpathlineto{\pgfqpoint{-0.048611in}{0.000000in}}%
\pgfusepath{stroke,fill}%
}%
\begin{pgfscope}%
\pgfsys@transformshift{0.626386in}{1.043550in}%
\pgfsys@useobject{currentmarker}{}%
\end{pgfscope}%
\end{pgfscope}%
\begin{pgfscope}%
\definecolor{textcolor}{rgb}{0.000000,0.000000,0.000000}%
\pgfsetstrokecolor{textcolor}%
\pgfsetfillcolor{textcolor}%
\pgftext[x=0.333333in, y=0.974106in, left, base]{\color{textcolor}{\rmfamily\fontsize{14.000000}{16.800000}\selectfont\catcode`\^=\active\def^{\ifmmode\sp\else\^{}\fi}\catcode`\%=\active\def%{\%}$\mathdefault{10}$}}%
\end{pgfscope}%
\begin{pgfscope}%
\pgfsetbuttcap%
\pgfsetroundjoin%
\definecolor{currentfill}{rgb}{0.000000,0.000000,0.000000}%
\pgfsetfillcolor{currentfill}%
\pgfsetlinewidth{0.803000pt}%
\definecolor{currentstroke}{rgb}{0.000000,0.000000,0.000000}%
\pgfsetstrokecolor{currentstroke}%
\pgfsetdash{}{0pt}%
\pgfsys@defobject{currentmarker}{\pgfqpoint{-0.048611in}{0.000000in}}{\pgfqpoint{-0.000000in}{0.000000in}}{%
\pgfpathmoveto{\pgfqpoint{-0.000000in}{0.000000in}}%
\pgfpathlineto{\pgfqpoint{-0.048611in}{0.000000in}}%
\pgfusepath{stroke,fill}%
}%
\begin{pgfscope}%
\pgfsys@transformshift{0.626386in}{1.880506in}%
\pgfsys@useobject{currentmarker}{}%
\end{pgfscope}%
\end{pgfscope}%
\begin{pgfscope}%
\definecolor{textcolor}{rgb}{0.000000,0.000000,0.000000}%
\pgfsetstrokecolor{textcolor}%
\pgfsetfillcolor{textcolor}%
\pgftext[x=0.333333in, y=1.811062in, left, base]{\color{textcolor}{\rmfamily\fontsize{14.000000}{16.800000}\selectfont\catcode`\^=\active\def^{\ifmmode\sp\else\^{}\fi}\catcode`\%=\active\def%{\%}$\mathdefault{20}$}}%
\end{pgfscope}%
\begin{pgfscope}%
\pgfsetbuttcap%
\pgfsetroundjoin%
\definecolor{currentfill}{rgb}{0.000000,0.000000,0.000000}%
\pgfsetfillcolor{currentfill}%
\pgfsetlinewidth{0.803000pt}%
\definecolor{currentstroke}{rgb}{0.000000,0.000000,0.000000}%
\pgfsetstrokecolor{currentstroke}%
\pgfsetdash{}{0pt}%
\pgfsys@defobject{currentmarker}{\pgfqpoint{-0.048611in}{0.000000in}}{\pgfqpoint{-0.000000in}{0.000000in}}{%
\pgfpathmoveto{\pgfqpoint{-0.000000in}{0.000000in}}%
\pgfpathlineto{\pgfqpoint{-0.048611in}{0.000000in}}%
\pgfusepath{stroke,fill}%
}%
\begin{pgfscope}%
\pgfsys@transformshift{0.626386in}{2.717463in}%
\pgfsys@useobject{currentmarker}{}%
\end{pgfscope}%
\end{pgfscope}%
\begin{pgfscope}%
\definecolor{textcolor}{rgb}{0.000000,0.000000,0.000000}%
\pgfsetstrokecolor{textcolor}%
\pgfsetfillcolor{textcolor}%
\pgftext[x=0.333333in, y=2.648019in, left, base]{\color{textcolor}{\rmfamily\fontsize{14.000000}{16.800000}\selectfont\catcode`\^=\active\def^{\ifmmode\sp\else\^{}\fi}\catcode`\%=\active\def%{\%}$\mathdefault{30}$}}%
\end{pgfscope}%
\begin{pgfscope}%
\pgfsetbuttcap%
\pgfsetroundjoin%
\definecolor{currentfill}{rgb}{0.000000,0.000000,0.000000}%
\pgfsetfillcolor{currentfill}%
\pgfsetlinewidth{0.803000pt}%
\definecolor{currentstroke}{rgb}{0.000000,0.000000,0.000000}%
\pgfsetstrokecolor{currentstroke}%
\pgfsetdash{}{0pt}%
\pgfsys@defobject{currentmarker}{\pgfqpoint{-0.048611in}{0.000000in}}{\pgfqpoint{-0.000000in}{0.000000in}}{%
\pgfpathmoveto{\pgfqpoint{-0.000000in}{0.000000in}}%
\pgfpathlineto{\pgfqpoint{-0.048611in}{0.000000in}}%
\pgfusepath{stroke,fill}%
}%
\begin{pgfscope}%
\pgfsys@transformshift{0.626386in}{3.554419in}%
\pgfsys@useobject{currentmarker}{}%
\end{pgfscope}%
\end{pgfscope}%
\begin{pgfscope}%
\definecolor{textcolor}{rgb}{0.000000,0.000000,0.000000}%
\pgfsetstrokecolor{textcolor}%
\pgfsetfillcolor{textcolor}%
\pgftext[x=0.333333in, y=3.484975in, left, base]{\color{textcolor}{\rmfamily\fontsize{14.000000}{16.800000}\selectfont\catcode`\^=\active\def^{\ifmmode\sp\else\^{}\fi}\catcode`\%=\active\def%{\%}$\mathdefault{40}$}}%
\end{pgfscope}%
\begin{pgfscope}%
\pgfsetbuttcap%
\pgfsetroundjoin%
\definecolor{currentfill}{rgb}{0.000000,0.000000,0.000000}%
\pgfsetfillcolor{currentfill}%
\pgfsetlinewidth{0.803000pt}%
\definecolor{currentstroke}{rgb}{0.000000,0.000000,0.000000}%
\pgfsetstrokecolor{currentstroke}%
\pgfsetdash{}{0pt}%
\pgfsys@defobject{currentmarker}{\pgfqpoint{-0.048611in}{0.000000in}}{\pgfqpoint{-0.000000in}{0.000000in}}{%
\pgfpathmoveto{\pgfqpoint{-0.000000in}{0.000000in}}%
\pgfpathlineto{\pgfqpoint{-0.048611in}{0.000000in}}%
\pgfusepath{stroke,fill}%
}%
\begin{pgfscope}%
\pgfsys@transformshift{0.626386in}{4.391376in}%
\pgfsys@useobject{currentmarker}{}%
\end{pgfscope}%
\end{pgfscope}%
\begin{pgfscope}%
\definecolor{textcolor}{rgb}{0.000000,0.000000,0.000000}%
\pgfsetstrokecolor{textcolor}%
\pgfsetfillcolor{textcolor}%
\pgftext[x=0.333333in, y=4.321932in, left, base]{\color{textcolor}{\rmfamily\fontsize{14.000000}{16.800000}\selectfont\catcode`\^=\active\def^{\ifmmode\sp\else\^{}\fi}\catcode`\%=\active\def%{\%}$\mathdefault{50}$}}%
\end{pgfscope}%
\begin{pgfscope}%
\pgfsetbuttcap%
\pgfsetroundjoin%
\definecolor{currentfill}{rgb}{0.000000,0.000000,0.000000}%
\pgfsetfillcolor{currentfill}%
\pgfsetlinewidth{0.803000pt}%
\definecolor{currentstroke}{rgb}{0.000000,0.000000,0.000000}%
\pgfsetstrokecolor{currentstroke}%
\pgfsetdash{}{0pt}%
\pgfsys@defobject{currentmarker}{\pgfqpoint{-0.048611in}{0.000000in}}{\pgfqpoint{-0.000000in}{0.000000in}}{%
\pgfpathmoveto{\pgfqpoint{-0.000000in}{0.000000in}}%
\pgfpathlineto{\pgfqpoint{-0.048611in}{0.000000in}}%
\pgfusepath{stroke,fill}%
}%
\begin{pgfscope}%
\pgfsys@transformshift{0.626386in}{5.228333in}%
\pgfsys@useobject{currentmarker}{}%
\end{pgfscope}%
\end{pgfscope}%
\begin{pgfscope}%
\definecolor{textcolor}{rgb}{0.000000,0.000000,0.000000}%
\pgfsetstrokecolor{textcolor}%
\pgfsetfillcolor{textcolor}%
\pgftext[x=0.333333in, y=5.158888in, left, base]{\color{textcolor}{\rmfamily\fontsize{14.000000}{16.800000}\selectfont\catcode`\^=\active\def^{\ifmmode\sp\else\^{}\fi}\catcode`\%=\active\def%{\%}$\mathdefault{60}$}}%
\end{pgfscope}%
\begin{pgfscope}%
\definecolor{textcolor}{rgb}{0.000000,0.000000,0.000000}%
\pgfsetstrokecolor{textcolor}%
\pgfsetfillcolor{textcolor}%
\pgftext[x=0.277777in,y=2.918332in,,bottom,rotate=90.000000]{\color{textcolor}{\rmfamily\fontsize{14.000000}{16.800000}\selectfont\catcode`\^=\active\def^{\ifmmode\sp\else\^{}\fi}\catcode`\%=\active\def%{\%}f2}}%
\end{pgfscope}%
\begin{pgfscope}%
\pgfpathrectangle{\pgfqpoint{0.626386in}{0.608332in}}{\pgfqpoint{6.200000in}{4.620000in}}%
\pgfusepath{clip}%
\pgfsetrectcap%
\pgfsetroundjoin%
\pgfsetlinewidth{3.011250pt}%
\definecolor{currentstroke}{rgb}{0.000000,0.000000,0.000000}%
\pgfsetstrokecolor{currentstroke}%
\pgfsetdash{}{0pt}%
\pgfpathmoveto{\pgfqpoint{0.626386in}{4.391376in}}%
\pgfpathlineto{\pgfqpoint{0.627161in}{4.307262in}}%
\pgfpathlineto{\pgfqpoint{0.629487in}{4.224002in}}%
\pgfpathlineto{\pgfqpoint{0.633363in}{4.141596in}}%
\pgfpathlineto{\pgfqpoint{0.638790in}{4.060043in}}%
\pgfpathlineto{\pgfqpoint{0.645767in}{3.979345in}}%
\pgfpathlineto{\pgfqpoint{0.654294in}{3.899501in}}%
\pgfpathlineto{\pgfqpoint{0.664372in}{3.820510in}}%
\pgfpathlineto{\pgfqpoint{0.676001in}{3.742374in}}%
\pgfpathlineto{\pgfqpoint{0.689180in}{3.665091in}}%
\pgfpathlineto{\pgfqpoint{0.703909in}{3.588663in}}%
\pgfpathlineto{\pgfqpoint{0.720189in}{3.513088in}}%
\pgfpathlineto{\pgfqpoint{0.738019in}{3.438368in}}%
\pgfpathlineto{\pgfqpoint{0.757400in}{3.364501in}}%
\pgfpathlineto{\pgfqpoint{0.778331in}{3.291488in}}%
\pgfpathlineto{\pgfqpoint{0.800813in}{3.219329in}}%
\pgfpathlineto{\pgfqpoint{0.824845in}{3.148025in}}%
\pgfpathlineto{\pgfqpoint{0.850428in}{3.077574in}}%
\pgfpathlineto{\pgfqpoint{0.877561in}{3.007977in}}%
\pgfpathlineto{\pgfqpoint{0.906244in}{2.939234in}}%
\pgfpathlineto{\pgfqpoint{0.936478in}{2.871345in}}%
\pgfpathlineto{\pgfqpoint{0.968263in}{2.804310in}}%
\pgfpathlineto{\pgfqpoint{1.001598in}{2.738129in}}%
\pgfpathlineto{\pgfqpoint{1.036483in}{2.672801in}}%
\pgfpathlineto{\pgfqpoint{1.072919in}{2.608328in}}%
\pgfpathlineto{\pgfqpoint{1.110905in}{2.544709in}}%
\pgfpathlineto{\pgfqpoint{1.150442in}{2.481943in}}%
\pgfpathlineto{\pgfqpoint{1.191529in}{2.420032in}}%
\pgfpathlineto{\pgfqpoint{1.234167in}{2.358975in}}%
\pgfpathlineto{\pgfqpoint{1.278355in}{2.298771in}}%
\pgfpathlineto{\pgfqpoint{1.324094in}{2.239422in}}%
\pgfpathlineto{\pgfqpoint{1.371383in}{2.180926in}}%
\pgfpathlineto{\pgfqpoint{1.420223in}{2.123284in}}%
\pgfpathlineto{\pgfqpoint{1.470613in}{2.066497in}}%
\pgfpathlineto{\pgfqpoint{1.522553in}{2.010563in}}%
\pgfpathlineto{\pgfqpoint{1.576044in}{1.955483in}}%
\pgfpathlineto{\pgfqpoint{1.631085in}{1.901257in}}%
\pgfpathlineto{\pgfqpoint{1.687677in}{1.847886in}}%
\pgfpathlineto{\pgfqpoint{1.745820in}{1.795368in}}%
\pgfpathlineto{\pgfqpoint{1.805512in}{1.743704in}}%
\pgfpathlineto{\pgfqpoint{1.866756in}{1.692894in}}%
\pgfpathlineto{\pgfqpoint{1.929549in}{1.642937in}}%
\pgfpathlineto{\pgfqpoint{1.993893in}{1.593835in}}%
\pgfpathlineto{\pgfqpoint{2.059788in}{1.545587in}}%
\pgfpathlineto{\pgfqpoint{2.127233in}{1.498193in}}%
\pgfpathlineto{\pgfqpoint{2.196229in}{1.451653in}}%
\pgfpathlineto{\pgfqpoint{2.266775in}{1.405966in}}%
\pgfpathlineto{\pgfqpoint{2.338871in}{1.361134in}}%
\pgfpathlineto{\pgfqpoint{2.412518in}{1.317156in}}%
\pgfpathlineto{\pgfqpoint{2.487716in}{1.274031in}}%
\pgfpathlineto{\pgfqpoint{2.564464in}{1.231760in}}%
\pgfpathlineto{\pgfqpoint{2.642762in}{1.190344in}}%
\pgfpathlineto{\pgfqpoint{2.722611in}{1.149781in}}%
\pgfpathlineto{\pgfqpoint{2.804010in}{1.110073in}}%
\pgfpathlineto{\pgfqpoint{2.886960in}{1.071218in}}%
\pgfpathlineto{\pgfqpoint{2.971460in}{1.033217in}}%
\pgfpathlineto{\pgfqpoint{3.057510in}{0.996070in}}%
\pgfpathlineto{\pgfqpoint{3.145112in}{0.959777in}}%
\pgfpathlineto{\pgfqpoint{3.234263in}{0.924338in}}%
\pgfpathlineto{\pgfqpoint{3.324965in}{0.889753in}}%
\pgfpathlineto{\pgfqpoint{3.389449in}{0.866091in}}%
\pgfpathlineto{\pgfqpoint{3.436350in}{0.849652in}}%
\pgfpathlineto{\pgfqpoint{3.484027in}{0.833640in}}%
\pgfpathlineto{\pgfqpoint{3.532479in}{0.818056in}}%
\pgfpathlineto{\pgfqpoint{3.581706in}{0.802898in}}%
\pgfpathlineto{\pgfqpoint{3.631709in}{0.788168in}}%
\pgfpathlineto{\pgfqpoint{3.682486in}{0.773864in}}%
\pgfpathlineto{\pgfqpoint{3.734039in}{0.759987in}}%
\pgfpathlineto{\pgfqpoint{3.786367in}{0.746537in}}%
\pgfpathlineto{\pgfqpoint{3.839470in}{0.733515in}}%
\pgfpathlineto{\pgfqpoint{3.893349in}{0.720919in}}%
\pgfpathlineto{\pgfqpoint{3.948003in}{0.708750in}}%
\pgfpathlineto{\pgfqpoint{4.003432in}{0.697008in}}%
\pgfpathlineto{\pgfqpoint{4.059636in}{0.685694in}}%
\pgfpathlineto{\pgfqpoint{4.116616in}{0.674806in}}%
\pgfpathlineto{\pgfqpoint{4.174370in}{0.664345in}}%
\pgfpathlineto{\pgfqpoint{4.232900in}{0.654311in}}%
\pgfpathlineto{\pgfqpoint{4.292205in}{0.644704in}}%
\pgfpathlineto{\pgfqpoint{4.352286in}{0.635524in}}%
\pgfpathlineto{\pgfqpoint{4.413141in}{0.626771in}}%
\pgfpathlineto{\pgfqpoint{4.474772in}{0.618445in}}%
\pgfpathlineto{\pgfqpoint{4.537178in}{0.610546in}}%
\pgfpathlineto{\pgfqpoint{4.569987in}{0.606666in}}%
\pgfusepath{stroke}%
\end{pgfscope}%
\begin{pgfscope}%
\pgfpathrectangle{\pgfqpoint{0.626386in}{0.608332in}}{\pgfqpoint{6.200000in}{4.620000in}}%
\pgfusepath{clip}%
\pgfsetrectcap%
\pgfsetroundjoin%
\pgfsetlinewidth{1.003750pt}%
\definecolor{currentstroke}{rgb}{0.000000,0.000000,0.000000}%
\pgfsetstrokecolor{currentstroke}%
\pgfsetstrokeopacity{0.200000}%
\pgfsetdash{}{0pt}%
\pgfpathmoveto{\pgfqpoint{0.626386in}{5.228333in}}%
\pgfpathlineto{\pgfqpoint{0.627316in}{5.127396in}}%
\pgfpathlineto{\pgfqpoint{0.630107in}{5.027483in}}%
\pgfpathlineto{\pgfqpoint{0.634758in}{4.928596in}}%
\pgfpathlineto{\pgfqpoint{0.641270in}{4.830733in}}%
\pgfpathlineto{\pgfqpoint{0.649643in}{4.733895in}}%
\pgfpathlineto{\pgfqpoint{0.659876in}{4.638082in}}%
\pgfpathlineto{\pgfqpoint{0.671970in}{4.543294in}}%
\pgfpathlineto{\pgfqpoint{0.685924in}{4.449530in}}%
\pgfpathlineto{\pgfqpoint{0.701738in}{4.356791in}}%
\pgfpathlineto{\pgfqpoint{0.719414in}{4.265077in}}%
\pgfpathlineto{\pgfqpoint{0.738950in}{4.174387in}}%
\pgfpathlineto{\pgfqpoint{0.760346in}{4.084722in}}%
\pgfpathlineto{\pgfqpoint{0.783603in}{3.996082in}}%
\pgfpathlineto{\pgfqpoint{0.808720in}{3.908467in}}%
\pgfpathlineto{\pgfqpoint{0.835698in}{3.821877in}}%
\pgfpathlineto{\pgfqpoint{0.864537in}{3.736311in}}%
\pgfpathlineto{\pgfqpoint{0.895236in}{3.651770in}}%
\pgfpathlineto{\pgfqpoint{0.927796in}{3.568253in}}%
\pgfpathlineto{\pgfqpoint{0.962216in}{3.485762in}}%
\pgfpathlineto{\pgfqpoint{0.998497in}{3.404295in}}%
\pgfpathlineto{\pgfqpoint{1.036638in}{3.323853in}}%
\pgfpathlineto{\pgfqpoint{1.076640in}{3.244436in}}%
\pgfpathlineto{\pgfqpoint{1.118503in}{3.166043in}}%
\pgfpathlineto{\pgfqpoint{1.162226in}{3.088675in}}%
\pgfpathlineto{\pgfqpoint{1.207809in}{3.012332in}}%
\pgfpathlineto{\pgfqpoint{1.255253in}{2.937014in}}%
\pgfpathlineto{\pgfqpoint{1.304558in}{2.862720in}}%
\pgfpathlineto{\pgfqpoint{1.355723in}{2.789451in}}%
\pgfpathlineto{\pgfqpoint{1.408749in}{2.717207in}}%
\pgfpathlineto{\pgfqpoint{1.463635in}{2.645987in}}%
\pgfpathlineto{\pgfqpoint{1.520382in}{2.575793in}}%
\pgfpathlineto{\pgfqpoint{1.578990in}{2.506623in}}%
\pgfpathlineto{\pgfqpoint{1.639458in}{2.438477in}}%
\pgfpathlineto{\pgfqpoint{1.701786in}{2.371357in}}%
\pgfpathlineto{\pgfqpoint{1.765976in}{2.305261in}}%
\pgfpathlineto{\pgfqpoint{1.832025in}{2.240190in}}%
\pgfpathlineto{\pgfqpoint{1.899935in}{2.176144in}}%
\pgfpathlineto{\pgfqpoint{1.969706in}{2.113122in}}%
\pgfpathlineto{\pgfqpoint{2.041338in}{2.051126in}}%
\pgfpathlineto{\pgfqpoint{2.114830in}{1.990154in}}%
\pgfpathlineto{\pgfqpoint{2.190182in}{1.930206in}}%
\pgfpathlineto{\pgfqpoint{2.267395in}{1.871284in}}%
\pgfpathlineto{\pgfqpoint{2.346469in}{1.813386in}}%
\pgfpathlineto{\pgfqpoint{2.427403in}{1.756513in}}%
\pgfpathlineto{\pgfqpoint{2.510197in}{1.700665in}}%
\pgfpathlineto{\pgfqpoint{2.594853in}{1.645841in}}%
\pgfpathlineto{\pgfqpoint{2.681368in}{1.592042in}}%
\pgfpathlineto{\pgfqpoint{2.769745in}{1.539268in}}%
\pgfpathlineto{\pgfqpoint{2.859982in}{1.487519in}}%
\pgfpathlineto{\pgfqpoint{2.952079in}{1.436794in}}%
\pgfpathlineto{\pgfqpoint{3.046037in}{1.387094in}}%
\pgfpathlineto{\pgfqpoint{3.141856in}{1.338419in}}%
\pgfpathlineto{\pgfqpoint{3.239535in}{1.290768in}}%
\pgfpathlineto{\pgfqpoint{3.339074in}{1.244143in}}%
\pgfpathlineto{\pgfqpoint{3.440475in}{1.198542in}}%
\pgfpathlineto{\pgfqpoint{3.543735in}{1.153966in}}%
\pgfpathlineto{\pgfqpoint{3.648857in}{1.110414in}}%
\pgfpathlineto{\pgfqpoint{3.755839in}{1.067887in}}%
\pgfpathlineto{\pgfqpoint{3.864681in}{1.026385in}}%
\pgfpathlineto{\pgfqpoint{3.942061in}{0.997990in}}%
\pgfpathlineto{\pgfqpoint{3.998343in}{0.978264in}}%
\pgfpathlineto{\pgfqpoint{4.055555in}{0.959050in}}%
\pgfpathlineto{\pgfqpoint{4.113698in}{0.940348in}}%
\pgfpathlineto{\pgfqpoint{4.172770in}{0.922159in}}%
\pgfpathlineto{\pgfqpoint{4.232773in}{0.904482in}}%
\pgfpathlineto{\pgfqpoint{4.293706in}{0.887318in}}%
\pgfpathlineto{\pgfqpoint{4.355570in}{0.870666in}}%
\pgfpathlineto{\pgfqpoint{4.418363in}{0.854526in}}%
\pgfpathlineto{\pgfqpoint{4.482087in}{0.838899in}}%
\pgfpathlineto{\pgfqpoint{4.546742in}{0.823784in}}%
\pgfpathlineto{\pgfqpoint{4.612326in}{0.809182in}}%
\pgfpathlineto{\pgfqpoint{4.678841in}{0.795091in}}%
\pgfpathlineto{\pgfqpoint{4.746286in}{0.781514in}}%
\pgfpathlineto{\pgfqpoint{4.814661in}{0.768448in}}%
\pgfpathlineto{\pgfqpoint{4.883967in}{0.755895in}}%
\pgfpathlineto{\pgfqpoint{4.954203in}{0.743854in}}%
\pgfpathlineto{\pgfqpoint{5.025369in}{0.732326in}}%
\pgfpathlineto{\pgfqpoint{5.097466in}{0.721310in}}%
\pgfpathlineto{\pgfqpoint{5.170493in}{0.710807in}}%
\pgfpathlineto{\pgfqpoint{5.244450in}{0.700815in}}%
\pgfpathlineto{\pgfqpoint{5.319337in}{0.691336in}}%
\pgfpathlineto{\pgfqpoint{5.395154in}{0.682370in}}%
\pgfpathlineto{\pgfqpoint{5.471902in}{0.673916in}}%
\pgfpathlineto{\pgfqpoint{5.549580in}{0.665974in}}%
\pgfpathlineto{\pgfqpoint{5.628189in}{0.658545in}}%
\pgfpathlineto{\pgfqpoint{5.707728in}{0.651628in}}%
\pgfpathlineto{\pgfqpoint{5.788197in}{0.645223in}}%
\pgfpathlineto{\pgfqpoint{5.869596in}{0.639331in}}%
\pgfpathlineto{\pgfqpoint{5.951925in}{0.633951in}}%
\pgfpathlineto{\pgfqpoint{6.035185in}{0.629083in}}%
\pgfpathlineto{\pgfqpoint{6.119375in}{0.624728in}}%
\pgfpathlineto{\pgfqpoint{6.204496in}{0.620886in}}%
\pgfpathlineto{\pgfqpoint{6.290546in}{0.617555in}}%
\pgfpathlineto{\pgfqpoint{6.377527in}{0.614737in}}%
\pgfpathlineto{\pgfqpoint{6.465438in}{0.612431in}}%
\pgfpathlineto{\pgfqpoint{6.554280in}{0.610638in}}%
\pgfpathlineto{\pgfqpoint{6.644052in}{0.609357in}}%
\pgfpathlineto{\pgfqpoint{6.734754in}{0.608589in}}%
\pgfpathlineto{\pgfqpoint{6.826386in}{0.608332in}}%
\pgfusepath{stroke}%
\end{pgfscope}%
\begin{pgfscope}%
\pgfsetrectcap%
\pgfsetmiterjoin%
\pgfsetlinewidth{0.803000pt}%
\definecolor{currentstroke}{rgb}{0.000000,0.000000,0.000000}%
\pgfsetstrokecolor{currentstroke}%
\pgfsetdash{}{0pt}%
\pgfpathmoveto{\pgfqpoint{0.626386in}{0.608332in}}%
\pgfpathlineto{\pgfqpoint{0.626386in}{5.228333in}}%
\pgfusepath{stroke}%
\end{pgfscope}%
\begin{pgfscope}%
\pgfsetrectcap%
\pgfsetmiterjoin%
\pgfsetlinewidth{0.803000pt}%
\definecolor{currentstroke}{rgb}{0.000000,0.000000,0.000000}%
\pgfsetstrokecolor{currentstroke}%
\pgfsetdash{}{0pt}%
\pgfpathmoveto{\pgfqpoint{6.826386in}{0.608332in}}%
\pgfpathlineto{\pgfqpoint{6.826386in}{5.228333in}}%
\pgfusepath{stroke}%
\end{pgfscope}%
\begin{pgfscope}%
\pgfsetrectcap%
\pgfsetmiterjoin%
\pgfsetlinewidth{0.803000pt}%
\definecolor{currentstroke}{rgb}{0.000000,0.000000,0.000000}%
\pgfsetstrokecolor{currentstroke}%
\pgfsetdash{}{0pt}%
\pgfpathmoveto{\pgfqpoint{0.626386in}{0.608332in}}%
\pgfpathlineto{\pgfqpoint{6.826386in}{0.608332in}}%
\pgfusepath{stroke}%
\end{pgfscope}%
\begin{pgfscope}%
\pgfsetrectcap%
\pgfsetmiterjoin%
\pgfsetlinewidth{0.803000pt}%
\definecolor{currentstroke}{rgb}{0.000000,0.000000,0.000000}%
\pgfsetstrokecolor{currentstroke}%
\pgfsetdash{}{0pt}%
\pgfpathmoveto{\pgfqpoint{0.626386in}{5.228333in}}%
\pgfpathlineto{\pgfqpoint{6.826386in}{5.228333in}}%
\pgfusepath{stroke}%
\end{pgfscope}%
\begin{pgfscope}%
\pgfsetbuttcap%
\pgfsetmiterjoin%
\definecolor{currentfill}{rgb}{0.300000,0.300000,0.300000}%
\pgfsetfillcolor{currentfill}%
\pgfsetfillopacity{0.500000}%
\pgfsetlinewidth{1.003750pt}%
\definecolor{currentstroke}{rgb}{0.300000,0.300000,0.300000}%
\pgfsetstrokecolor{currentstroke}%
\pgfsetstrokeopacity{0.500000}%
\pgfsetdash{}{0pt}%
\pgfpathmoveto{\pgfqpoint{4.314927in}{4.220000in}}%
\pgfpathlineto{\pgfqpoint{6.718053in}{4.220000in}}%
\pgfpathquadraticcurveto{\pgfqpoint{6.756942in}{4.220000in}}{\pgfqpoint{6.756942in}{4.258889in}}%
\pgfpathlineto{\pgfqpoint{6.756942in}{5.064444in}}%
\pgfpathquadraticcurveto{\pgfqpoint{6.756942in}{5.103333in}}{\pgfqpoint{6.718053in}{5.103333in}}%
\pgfpathlineto{\pgfqpoint{4.314927in}{5.103333in}}%
\pgfpathquadraticcurveto{\pgfqpoint{4.276038in}{5.103333in}}{\pgfqpoint{4.276038in}{5.064444in}}%
\pgfpathlineto{\pgfqpoint{4.276038in}{4.258889in}}%
\pgfpathquadraticcurveto{\pgfqpoint{4.276038in}{4.220000in}}{\pgfqpoint{4.314927in}{4.220000in}}%
\pgfpathlineto{\pgfqpoint{4.314927in}{4.220000in}}%
\pgfpathclose%
\pgfusepath{stroke,fill}%
\end{pgfscope}%
\begin{pgfscope}%
\pgfsetbuttcap%
\pgfsetmiterjoin%
\definecolor{currentfill}{rgb}{1.000000,1.000000,1.000000}%
\pgfsetfillcolor{currentfill}%
\pgfsetlinewidth{1.003750pt}%
\definecolor{currentstroke}{rgb}{0.800000,0.800000,0.800000}%
\pgfsetstrokecolor{currentstroke}%
\pgfsetdash{}{0pt}%
\pgfpathmoveto{\pgfqpoint{4.287149in}{4.247778in}}%
\pgfpathlineto{\pgfqpoint{6.690275in}{4.247778in}}%
\pgfpathquadraticcurveto{\pgfqpoint{6.729164in}{4.247778in}}{\pgfqpoint{6.729164in}{4.286667in}}%
\pgfpathlineto{\pgfqpoint{6.729164in}{5.092221in}}%
\pgfpathquadraticcurveto{\pgfqpoint{6.729164in}{5.131110in}}{\pgfqpoint{6.690275in}{5.131110in}}%
\pgfpathlineto{\pgfqpoint{4.287149in}{5.131110in}}%
\pgfpathquadraticcurveto{\pgfqpoint{4.248260in}{5.131110in}}{\pgfqpoint{4.248260in}{5.092221in}}%
\pgfpathlineto{\pgfqpoint{4.248260in}{4.286667in}}%
\pgfpathquadraticcurveto{\pgfqpoint{4.248260in}{4.247778in}}{\pgfqpoint{4.287149in}{4.247778in}}%
\pgfpathlineto{\pgfqpoint{4.287149in}{4.247778in}}%
\pgfpathclose%
\pgfusepath{stroke,fill}%
\end{pgfscope}%
\begin{pgfscope}%
\pgfsetrectcap%
\pgfsetroundjoin%
\pgfsetlinewidth{3.011250pt}%
\definecolor{currentstroke}{rgb}{0.000000,0.000000,0.000000}%
\pgfsetstrokecolor{currentstroke}%
\pgfsetdash{}{0pt}%
\pgfpathmoveto{\pgfqpoint{4.326038in}{4.982499in}}%
\pgfpathlineto{\pgfqpoint{4.520482in}{4.982499in}}%
\pgfpathlineto{\pgfqpoint{4.714927in}{4.982499in}}%
\pgfusepath{stroke}%
\end{pgfscope}%
\begin{pgfscope}%
\definecolor{textcolor}{rgb}{0.000000,0.000000,0.000000}%
\pgfsetstrokecolor{textcolor}%
\pgfsetfillcolor{textcolor}%
\pgftext[x=4.870482in,y=4.914444in,left,base]{\color{textcolor}{\rmfamily\fontsize{14.000000}{16.800000}\selectfont\catcode`\^=\active\def^{\ifmmode\sp\else\^{}\fi}\catcode`\%=\active\def%{\%}Pareto Front}}%
\end{pgfscope}%
\begin{pgfscope}%
\pgfsetbuttcap%
\pgfsetroundjoin%
\definecolor{currentfill}{rgb}{0.121569,0.466667,0.705882}%
\pgfsetfillcolor{currentfill}%
\pgfsetlinewidth{1.003750pt}%
\definecolor{currentstroke}{rgb}{0.121569,0.466667,0.705882}%
\pgfsetstrokecolor{currentstroke}%
\pgfsetdash{}{0pt}%
\pgfsys@defobject{currentmarker}{\pgfqpoint{-0.012028in}{-0.012028in}}{\pgfqpoint{0.012028in}{0.012028in}}{%
\pgfpathmoveto{\pgfqpoint{0.000000in}{-0.012028in}}%
\pgfpathcurveto{\pgfqpoint{0.003190in}{-0.012028in}}{\pgfqpoint{0.006250in}{-0.010761in}}{\pgfqpoint{0.008505in}{-0.008505in}}%
\pgfpathcurveto{\pgfqpoint{0.010761in}{-0.006250in}}{\pgfqpoint{0.012028in}{-0.003190in}}{\pgfqpoint{0.012028in}{0.000000in}}%
\pgfpathcurveto{\pgfqpoint{0.012028in}{0.003190in}}{\pgfqpoint{0.010761in}{0.006250in}}{\pgfqpoint{0.008505in}{0.008505in}}%
\pgfpathcurveto{\pgfqpoint{0.006250in}{0.010761in}}{\pgfqpoint{0.003190in}{0.012028in}}{\pgfqpoint{0.000000in}{0.012028in}}%
\pgfpathcurveto{\pgfqpoint{-0.003190in}{0.012028in}}{\pgfqpoint{-0.006250in}{0.010761in}}{\pgfqpoint{-0.008505in}{0.008505in}}%
\pgfpathcurveto{\pgfqpoint{-0.010761in}{0.006250in}}{\pgfqpoint{-0.012028in}{0.003190in}}{\pgfqpoint{-0.012028in}{0.000000in}}%
\pgfpathcurveto{\pgfqpoint{-0.012028in}{-0.003190in}}{\pgfqpoint{-0.010761in}{-0.006250in}}{\pgfqpoint{-0.008505in}{-0.008505in}}%
\pgfpathcurveto{\pgfqpoint{-0.006250in}{-0.010761in}}{\pgfqpoint{-0.003190in}{-0.012028in}}{\pgfqpoint{0.000000in}{-0.012028in}}%
\pgfpathlineto{\pgfqpoint{0.000000in}{-0.012028in}}%
\pgfpathclose%
\pgfusepath{stroke,fill}%
}%
\begin{pgfscope}%
\pgfsys@transformshift{4.520482in}{4.690486in}%
\pgfsys@useobject{currentmarker}{}%
\end{pgfscope}%
\end{pgfscope}%
\begin{pgfscope}%
\definecolor{textcolor}{rgb}{0.000000,0.000000,0.000000}%
\pgfsetstrokecolor{textcolor}%
\pgfsetfillcolor{textcolor}%
\pgftext[x=4.870482in,y=4.639444in,left,base]{\color{textcolor}{\rmfamily\fontsize{14.000000}{16.800000}\selectfont\catcode`\^=\active\def^{\ifmmode\sp\else\^{}\fi}\catcode`\%=\active\def%{\%}Tested points}}%
\end{pgfscope}%
\begin{pgfscope}%
\pgfsetbuttcap%
\pgfsetroundjoin%
\definecolor{currentfill}{rgb}{0.839216,0.152941,0.156863}%
\pgfsetfillcolor{currentfill}%
\pgfsetlinewidth{1.003750pt}%
\definecolor{currentstroke}{rgb}{0.839216,0.152941,0.156863}%
\pgfsetstrokecolor{currentstroke}%
\pgfsetdash{}{0pt}%
\pgfsys@defobject{currentmarker}{\pgfqpoint{-0.031056in}{-0.031056in}}{\pgfqpoint{0.031056in}{0.031056in}}{%
\pgfpathmoveto{\pgfqpoint{0.000000in}{-0.031056in}}%
\pgfpathcurveto{\pgfqpoint{0.008236in}{-0.031056in}}{\pgfqpoint{0.016136in}{-0.027784in}}{\pgfqpoint{0.021960in}{-0.021960in}}%
\pgfpathcurveto{\pgfqpoint{0.027784in}{-0.016136in}}{\pgfqpoint{0.031056in}{-0.008236in}}{\pgfqpoint{0.031056in}{0.000000in}}%
\pgfpathcurveto{\pgfqpoint{0.031056in}{0.008236in}}{\pgfqpoint{0.027784in}{0.016136in}}{\pgfqpoint{0.021960in}{0.021960in}}%
\pgfpathcurveto{\pgfqpoint{0.016136in}{0.027784in}}{\pgfqpoint{0.008236in}{0.031056in}}{\pgfqpoint{0.000000in}{0.031056in}}%
\pgfpathcurveto{\pgfqpoint{-0.008236in}{0.031056in}}{\pgfqpoint{-0.016136in}{0.027784in}}{\pgfqpoint{-0.021960in}{0.021960in}}%
\pgfpathcurveto{\pgfqpoint{-0.027784in}{0.016136in}}{\pgfqpoint{-0.031056in}{0.008236in}}{\pgfqpoint{-0.031056in}{0.000000in}}%
\pgfpathcurveto{\pgfqpoint{-0.031056in}{-0.008236in}}{\pgfqpoint{-0.027784in}{-0.016136in}}{\pgfqpoint{-0.021960in}{-0.021960in}}%
\pgfpathcurveto{\pgfqpoint{-0.016136in}{-0.027784in}}{\pgfqpoint{-0.008236in}{-0.031056in}}{\pgfqpoint{0.000000in}{-0.031056in}}%
\pgfpathlineto{\pgfqpoint{0.000000in}{-0.031056in}}%
\pgfpathclose%
\pgfusepath{stroke,fill}%
}%
\begin{pgfscope}%
\pgfsys@transformshift{4.520482in}{4.415486in}%
\pgfsys@useobject{currentmarker}{}%
\end{pgfscope}%
\end{pgfscope}%
\begin{pgfscope}%
\definecolor{textcolor}{rgb}{0.000000,0.000000,0.000000}%
\pgfsetstrokecolor{textcolor}%
\pgfsetfillcolor{textcolor}%
\pgftext[x=4.870482in,y=4.364445in,left,base]{\color{textcolor}{\rmfamily\fontsize{14.000000}{16.800000}\selectfont\catcode`\^=\active\def^{\ifmmode\sp\else\^{}\fi}\catcode`\%=\active\def%{\%}Alternative solutions}}%
\end{pgfscope}%
\end{pgfpicture}%
\makeatother%
\endgroup%
}
  \caption{All of the alternative points inside the near-feasible space selected
  using the algorithm described in Section \ref{section:mga-moo}.}
  \label{fig:nd-mga}
\end{figure}

\begin{figure}[H]
  \centering
  \resizebox{1\columnwidth}{!}{%% Creator: Matplotlib, PGF backend
%%
%% To include the figure in your LaTeX document, write
%%   \input{<filename>.pgf}
%%
%% Make sure the required packages are loaded in your preamble
%%   \usepackage{pgf}
%%
%% Also ensure that all the required font packages are loaded; for instance,
%% the lmodern package is sometimes necessary when using math font.
%%   \usepackage{lmodern}
%%
%% Figures using additional raster images can only be included by \input if
%% they are in the same directory as the main LaTeX file. For loading figures
%% from other directories you can use the `import` package
%%   \usepackage{import}
%%
%% and then include the figures with
%%   \import{<path to file>}{<filename>.pgf}
%%
%% Matplotlib used the following preamble
%%   \def\mathdefault#1{#1}
%%   \everymath=\expandafter{\the\everymath\displaystyle}
%%   \IfFileExists{scrextend.sty}{
%%     \usepackage[fontsize=10.000000pt]{scrextend}
%%   }{
%%     \renewcommand{\normalsize}{\fontsize{10.000000}{12.000000}\selectfont}
%%     \normalsize
%%   }
%%   
%%   \makeatletter\@ifpackageloaded{underscore}{}{\usepackage[strings]{underscore}}\makeatother
%%
\begingroup%
\makeatletter%
\begin{pgfpicture}%
\pgfpathrectangle{\pgfpointorigin}{\pgfqpoint{9.654231in}{3.182465in}}%
\pgfusepath{use as bounding box, clip}%
\begin{pgfscope}%
\pgfsetbuttcap%
\pgfsetmiterjoin%
\definecolor{currentfill}{rgb}{1.000000,1.000000,1.000000}%
\pgfsetfillcolor{currentfill}%
\pgfsetlinewidth{0.000000pt}%
\definecolor{currentstroke}{rgb}{0.000000,0.000000,0.000000}%
\pgfsetstrokecolor{currentstroke}%
\pgfsetdash{}{0pt}%
\pgfpathmoveto{\pgfqpoint{0.000000in}{0.000000in}}%
\pgfpathlineto{\pgfqpoint{9.654231in}{0.000000in}}%
\pgfpathlineto{\pgfqpoint{9.654231in}{3.182465in}}%
\pgfpathlineto{\pgfqpoint{0.000000in}{3.182465in}}%
\pgfpathlineto{\pgfqpoint{0.000000in}{0.000000in}}%
\pgfpathclose%
\pgfusepath{fill}%
\end{pgfscope}%
\begin{pgfscope}%
\pgfsetbuttcap%
\pgfsetmiterjoin%
\definecolor{currentfill}{rgb}{1.000000,1.000000,1.000000}%
\pgfsetfillcolor{currentfill}%
\pgfsetlinewidth{0.000000pt}%
\definecolor{currentstroke}{rgb}{0.000000,0.000000,0.000000}%
\pgfsetstrokecolor{currentstroke}%
\pgfsetstrokeopacity{0.000000}%
\pgfsetdash{}{0pt}%
\pgfpathmoveto{\pgfqpoint{3.536584in}{0.147348in}}%
\pgfpathlineto{\pgfqpoint{6.271879in}{0.147348in}}%
\pgfpathlineto{\pgfqpoint{6.271879in}{2.882642in}}%
\pgfpathlineto{\pgfqpoint{3.536584in}{2.882642in}}%
\pgfpathlineto{\pgfqpoint{3.536584in}{0.147348in}}%
\pgfpathclose%
\pgfusepath{fill}%
\end{pgfscope}%
\begin{pgfscope}%
\pgfsetbuttcap%
\pgfsetmiterjoin%
\definecolor{currentfill}{rgb}{0.950000,0.950000,0.950000}%
\pgfsetfillcolor{currentfill}%
\pgfsetfillopacity{0.500000}%
\pgfsetlinewidth{1.003750pt}%
\definecolor{currentstroke}{rgb}{0.950000,0.950000,0.950000}%
\pgfsetstrokecolor{currentstroke}%
\pgfsetstrokeopacity{0.500000}%
\pgfsetdash{}{0pt}%
\pgfpathmoveto{\pgfqpoint{4.941195in}{1.930798in}}%
\pgfpathlineto{\pgfqpoint{6.147731in}{1.099507in}}%
\pgfpathlineto{\pgfqpoint{6.226389in}{2.033906in}}%
\pgfpathlineto{\pgfqpoint{4.941195in}{2.862146in}}%
\pgfusepath{stroke,fill}%
\end{pgfscope}%
\begin{pgfscope}%
\pgfsetbuttcap%
\pgfsetmiterjoin%
\definecolor{currentfill}{rgb}{0.900000,0.900000,0.900000}%
\pgfsetfillcolor{currentfill}%
\pgfsetfillopacity{0.500000}%
\pgfsetlinewidth{1.003750pt}%
\definecolor{currentstroke}{rgb}{0.900000,0.900000,0.900000}%
\pgfsetstrokecolor{currentstroke}%
\pgfsetstrokeopacity{0.500000}%
\pgfsetdash{}{0pt}%
\pgfpathmoveto{\pgfqpoint{4.941195in}{1.930798in}}%
\pgfpathlineto{\pgfqpoint{3.734658in}{1.099507in}}%
\pgfpathlineto{\pgfqpoint{3.656001in}{2.033906in}}%
\pgfpathlineto{\pgfqpoint{4.941195in}{2.862146in}}%
\pgfusepath{stroke,fill}%
\end{pgfscope}%
\begin{pgfscope}%
\pgfsetbuttcap%
\pgfsetmiterjoin%
\definecolor{currentfill}{rgb}{0.925000,0.925000,0.925000}%
\pgfsetfillcolor{currentfill}%
\pgfsetfillopacity{0.500000}%
\pgfsetlinewidth{1.003750pt}%
\definecolor{currentstroke}{rgb}{0.925000,0.925000,0.925000}%
\pgfsetstrokecolor{currentstroke}%
\pgfsetstrokeopacity{0.500000}%
\pgfsetdash{}{0pt}%
\pgfpathmoveto{\pgfqpoint{4.941195in}{1.930798in}}%
\pgfpathlineto{\pgfqpoint{3.734658in}{1.099507in}}%
\pgfpathlineto{\pgfqpoint{4.941195in}{0.166408in}}%
\pgfpathlineto{\pgfqpoint{6.147731in}{1.099507in}}%
\pgfusepath{stroke,fill}%
\end{pgfscope}%
\begin{pgfscope}%
\pgfsetbuttcap%
\pgfsetroundjoin%
\pgfsetlinewidth{0.803000pt}%
\definecolor{currentstroke}{rgb}{0.690196,0.690196,0.690196}%
\pgfsetstrokecolor{currentstroke}%
\pgfsetdash{}{0pt}%
\pgfpathmoveto{\pgfqpoint{6.075116in}{1.043348in}}%
\pgfpathlineto{\pgfqpoint{4.868351in}{1.880609in}}%
\pgfpathlineto{\pgfqpoint{4.863861in}{2.812308in}}%
\pgfusepath{stroke}%
\end{pgfscope}%
\begin{pgfscope}%
\pgfsetbuttcap%
\pgfsetroundjoin%
\pgfsetlinewidth{0.803000pt}%
\definecolor{currentstroke}{rgb}{0.690196,0.690196,0.690196}%
\pgfsetstrokecolor{currentstroke}%
\pgfsetdash{}{0pt}%
\pgfpathmoveto{\pgfqpoint{5.845306in}{0.865621in}}%
\pgfpathlineto{\pgfqpoint{4.638013in}{1.721909in}}%
\pgfpathlineto{\pgfqpoint{4.619106in}{2.654576in}}%
\pgfusepath{stroke}%
\end{pgfscope}%
\begin{pgfscope}%
\pgfsetbuttcap%
\pgfsetroundjoin%
\pgfsetlinewidth{0.803000pt}%
\definecolor{currentstroke}{rgb}{0.690196,0.690196,0.690196}%
\pgfsetstrokecolor{currentstroke}%
\pgfsetdash{}{0pt}%
\pgfpathmoveto{\pgfqpoint{5.610093in}{0.683714in}}%
\pgfpathlineto{\pgfqpoint{4.402562in}{1.559686in}}%
\pgfpathlineto{\pgfqpoint{4.368575in}{2.493122in}}%
\pgfusepath{stroke}%
\end{pgfscope}%
\begin{pgfscope}%
\pgfsetbuttcap%
\pgfsetroundjoin%
\pgfsetlinewidth{0.803000pt}%
\definecolor{currentstroke}{rgb}{0.690196,0.690196,0.690196}%
\pgfsetstrokecolor{currentstroke}%
\pgfsetdash{}{0pt}%
\pgfpathmoveto{\pgfqpoint{5.369283in}{0.497478in}}%
\pgfpathlineto{\pgfqpoint{4.161826in}{1.393821in}}%
\pgfpathlineto{\pgfqpoint{4.112061in}{2.327813in}}%
\pgfusepath{stroke}%
\end{pgfscope}%
\begin{pgfscope}%
\pgfsetbuttcap%
\pgfsetroundjoin%
\pgfsetlinewidth{0.803000pt}%
\definecolor{currentstroke}{rgb}{0.690196,0.690196,0.690196}%
\pgfsetstrokecolor{currentstroke}%
\pgfsetdash{}{0pt}%
\pgfpathmoveto{\pgfqpoint{5.122674in}{0.306758in}}%
\pgfpathlineto{\pgfqpoint{3.915624in}{1.224191in}}%
\pgfpathlineto{\pgfqpoint{3.849347in}{2.158507in}}%
\pgfusepath{stroke}%
\end{pgfscope}%
\begin{pgfscope}%
\pgfsetbuttcap%
\pgfsetroundjoin%
\pgfsetlinewidth{0.803000pt}%
\definecolor{currentstroke}{rgb}{0.690196,0.690196,0.690196}%
\pgfsetstrokecolor{currentstroke}%
\pgfsetdash{}{0pt}%
\pgfpathmoveto{\pgfqpoint{5.018529in}{2.812308in}}%
\pgfpathlineto{\pgfqpoint{5.014039in}{1.880609in}}%
\pgfpathlineto{\pgfqpoint{3.807274in}{1.043348in}}%
\pgfusepath{stroke}%
\end{pgfscope}%
\begin{pgfscope}%
\pgfsetbuttcap%
\pgfsetroundjoin%
\pgfsetlinewidth{0.803000pt}%
\definecolor{currentstroke}{rgb}{0.690196,0.690196,0.690196}%
\pgfsetstrokecolor{currentstroke}%
\pgfsetdash{}{0pt}%
\pgfpathmoveto{\pgfqpoint{5.263284in}{2.654576in}}%
\pgfpathlineto{\pgfqpoint{5.244376in}{1.721909in}}%
\pgfpathlineto{\pgfqpoint{4.037084in}{0.865621in}}%
\pgfusepath{stroke}%
\end{pgfscope}%
\begin{pgfscope}%
\pgfsetbuttcap%
\pgfsetroundjoin%
\pgfsetlinewidth{0.803000pt}%
\definecolor{currentstroke}{rgb}{0.690196,0.690196,0.690196}%
\pgfsetstrokecolor{currentstroke}%
\pgfsetdash{}{0pt}%
\pgfpathmoveto{\pgfqpoint{5.513815in}{2.493122in}}%
\pgfpathlineto{\pgfqpoint{5.479827in}{1.559686in}}%
\pgfpathlineto{\pgfqpoint{4.272297in}{0.683714in}}%
\pgfusepath{stroke}%
\end{pgfscope}%
\begin{pgfscope}%
\pgfsetbuttcap%
\pgfsetroundjoin%
\pgfsetlinewidth{0.803000pt}%
\definecolor{currentstroke}{rgb}{0.690196,0.690196,0.690196}%
\pgfsetstrokecolor{currentstroke}%
\pgfsetdash{}{0pt}%
\pgfpathmoveto{\pgfqpoint{5.770329in}{2.327813in}}%
\pgfpathlineto{\pgfqpoint{5.720564in}{1.393821in}}%
\pgfpathlineto{\pgfqpoint{4.513107in}{0.497478in}}%
\pgfusepath{stroke}%
\end{pgfscope}%
\begin{pgfscope}%
\pgfsetbuttcap%
\pgfsetroundjoin%
\pgfsetlinewidth{0.803000pt}%
\definecolor{currentstroke}{rgb}{0.690196,0.690196,0.690196}%
\pgfsetstrokecolor{currentstroke}%
\pgfsetdash{}{0pt}%
\pgfpathmoveto{\pgfqpoint{6.033043in}{2.158507in}}%
\pgfpathlineto{\pgfqpoint{5.966766in}{1.224191in}}%
\pgfpathlineto{\pgfqpoint{4.759716in}{0.306758in}}%
\pgfusepath{stroke}%
\end{pgfscope}%
\begin{pgfscope}%
\pgfsetbuttcap%
\pgfsetroundjoin%
\pgfsetlinewidth{0.803000pt}%
\definecolor{currentstroke}{rgb}{0.690196,0.690196,0.690196}%
\pgfsetstrokecolor{currentstroke}%
\pgfsetdash{}{0pt}%
\pgfpathmoveto{\pgfqpoint{3.729941in}{1.155548in}}%
\pgfpathlineto{\pgfqpoint{4.941195in}{1.986842in}}%
\pgfpathlineto{\pgfqpoint{6.152449in}{1.155548in}}%
\pgfusepath{stroke}%
\end{pgfscope}%
\begin{pgfscope}%
\pgfsetbuttcap%
\pgfsetroundjoin%
\pgfsetlinewidth{0.803000pt}%
\definecolor{currentstroke}{rgb}{0.690196,0.690196,0.690196}%
\pgfsetstrokecolor{currentstroke}%
\pgfsetdash{}{0pt}%
\pgfpathmoveto{\pgfqpoint{3.714997in}{1.333067in}}%
\pgfpathlineto{\pgfqpoint{4.941195in}{2.164215in}}%
\pgfpathlineto{\pgfqpoint{6.167392in}{1.333067in}}%
\pgfusepath{stroke}%
\end{pgfscope}%
\begin{pgfscope}%
\pgfsetbuttcap%
\pgfsetroundjoin%
\pgfsetlinewidth{0.803000pt}%
\definecolor{currentstroke}{rgb}{0.690196,0.690196,0.690196}%
\pgfsetstrokecolor{currentstroke}%
\pgfsetdash{}{0pt}%
\pgfpathmoveto{\pgfqpoint{3.699681in}{1.515021in}}%
\pgfpathlineto{\pgfqpoint{4.941195in}{2.345771in}}%
\pgfpathlineto{\pgfqpoint{6.182709in}{1.515021in}}%
\pgfusepath{stroke}%
\end{pgfscope}%
\begin{pgfscope}%
\pgfsetbuttcap%
\pgfsetroundjoin%
\pgfsetlinewidth{0.803000pt}%
\definecolor{currentstroke}{rgb}{0.690196,0.690196,0.690196}%
\pgfsetstrokecolor{currentstroke}%
\pgfsetdash{}{0pt}%
\pgfpathmoveto{\pgfqpoint{3.683976in}{1.701578in}}%
\pgfpathlineto{\pgfqpoint{4.941195in}{2.531660in}}%
\pgfpathlineto{\pgfqpoint{6.198413in}{1.701578in}}%
\pgfusepath{stroke}%
\end{pgfscope}%
\begin{pgfscope}%
\pgfsetbuttcap%
\pgfsetroundjoin%
\pgfsetlinewidth{0.803000pt}%
\definecolor{currentstroke}{rgb}{0.690196,0.690196,0.690196}%
\pgfsetstrokecolor{currentstroke}%
\pgfsetdash{}{0pt}%
\pgfpathmoveto{\pgfqpoint{3.667870in}{1.892915in}}%
\pgfpathlineto{\pgfqpoint{4.941195in}{2.722038in}}%
\pgfpathlineto{\pgfqpoint{6.214520in}{1.892915in}}%
\pgfusepath{stroke}%
\end{pgfscope}%
\begin{pgfscope}%
\pgfsetrectcap%
\pgfsetroundjoin%
\pgfsetlinewidth{0.803000pt}%
\definecolor{currentstroke}{rgb}{0.000000,0.000000,0.000000}%
\pgfsetstrokecolor{currentstroke}%
\pgfsetdash{}{0pt}%
\pgfpathmoveto{\pgfqpoint{6.147731in}{1.099507in}}%
\pgfpathlineto{\pgfqpoint{4.941195in}{0.166408in}}%
\pgfusepath{stroke}%
\end{pgfscope}%
\begin{pgfscope}%
\pgfsetrectcap%
\pgfsetroundjoin%
\pgfsetlinewidth{0.803000pt}%
\definecolor{currentstroke}{rgb}{0.000000,0.000000,0.000000}%
\pgfsetstrokecolor{currentstroke}%
\pgfsetdash{}{0pt}%
\pgfpathmoveto{\pgfqpoint{6.064907in}{1.050431in}}%
\pgfpathlineto{\pgfqpoint{6.095561in}{1.029163in}}%
\pgfusepath{stroke}%
\end{pgfscope}%
\begin{pgfscope}%
\pgfsetrectcap%
\pgfsetroundjoin%
\pgfsetlinewidth{0.803000pt}%
\definecolor{currentstroke}{rgb}{0.000000,0.000000,0.000000}%
\pgfsetstrokecolor{currentstroke}%
\pgfsetdash{}{0pt}%
\pgfpathmoveto{\pgfqpoint{5.835087in}{0.872869in}}%
\pgfpathlineto{\pgfqpoint{5.865774in}{0.851104in}}%
\pgfusepath{stroke}%
\end{pgfscope}%
\begin{pgfscope}%
\pgfsetrectcap%
\pgfsetroundjoin%
\pgfsetlinewidth{0.803000pt}%
\definecolor{currentstroke}{rgb}{0.000000,0.000000,0.000000}%
\pgfsetstrokecolor{currentstroke}%
\pgfsetdash{}{0pt}%
\pgfpathmoveto{\pgfqpoint{5.599865in}{0.691133in}}%
\pgfpathlineto{\pgfqpoint{5.630578in}{0.668853in}}%
\pgfusepath{stroke}%
\end{pgfscope}%
\begin{pgfscope}%
\pgfsetrectcap%
\pgfsetroundjoin%
\pgfsetlinewidth{0.803000pt}%
\definecolor{currentstroke}{rgb}{0.000000,0.000000,0.000000}%
\pgfsetstrokecolor{currentstroke}%
\pgfsetdash{}{0pt}%
\pgfpathmoveto{\pgfqpoint{5.359049in}{0.505076in}}%
\pgfpathlineto{\pgfqpoint{5.389781in}{0.482262in}}%
\pgfusepath{stroke}%
\end{pgfscope}%
\begin{pgfscope}%
\pgfsetrectcap%
\pgfsetroundjoin%
\pgfsetlinewidth{0.803000pt}%
\definecolor{currentstroke}{rgb}{0.000000,0.000000,0.000000}%
\pgfsetstrokecolor{currentstroke}%
\pgfsetdash{}{0pt}%
\pgfpathmoveto{\pgfqpoint{5.112437in}{0.314540in}}%
\pgfpathlineto{\pgfqpoint{5.143179in}{0.291173in}}%
\pgfusepath{stroke}%
\end{pgfscope}%
\begin{pgfscope}%
\definecolor{textcolor}{rgb}{0.000000,0.000000,0.000000}%
\pgfsetstrokecolor{textcolor}%
\pgfsetfillcolor{textcolor}%
\pgftext[x=5.840237in,y=0.241958in,,]{\color{textcolor}{\rmfamily\fontsize{14.000000}{16.800000}\selectfont\catcode`\^=\active\def^{\ifmmode\sp\else\^{}\fi}\catcode`\%=\active\def%{\%}f1}}%
\end{pgfscope}%
\begin{pgfscope}%
\pgfsetrectcap%
\pgfsetroundjoin%
\pgfsetlinewidth{0.803000pt}%
\definecolor{currentstroke}{rgb}{0.000000,0.000000,0.000000}%
\pgfsetstrokecolor{currentstroke}%
\pgfsetdash{}{0pt}%
\pgfpathmoveto{\pgfqpoint{3.734658in}{1.099507in}}%
\pgfpathlineto{\pgfqpoint{4.941195in}{0.166408in}}%
\pgfusepath{stroke}%
\end{pgfscope}%
\begin{pgfscope}%
\pgfsetrectcap%
\pgfsetroundjoin%
\pgfsetlinewidth{0.803000pt}%
\definecolor{currentstroke}{rgb}{0.000000,0.000000,0.000000}%
\pgfsetstrokecolor{currentstroke}%
\pgfsetdash{}{0pt}%
\pgfpathmoveto{\pgfqpoint{3.817483in}{1.050431in}}%
\pgfpathlineto{\pgfqpoint{3.786829in}{1.029163in}}%
\pgfusepath{stroke}%
\end{pgfscope}%
\begin{pgfscope}%
\pgfsetrectcap%
\pgfsetroundjoin%
\pgfsetlinewidth{0.803000pt}%
\definecolor{currentstroke}{rgb}{0.000000,0.000000,0.000000}%
\pgfsetstrokecolor{currentstroke}%
\pgfsetdash{}{0pt}%
\pgfpathmoveto{\pgfqpoint{4.047303in}{0.872869in}}%
\pgfpathlineto{\pgfqpoint{4.016616in}{0.851104in}}%
\pgfusepath{stroke}%
\end{pgfscope}%
\begin{pgfscope}%
\pgfsetrectcap%
\pgfsetroundjoin%
\pgfsetlinewidth{0.803000pt}%
\definecolor{currentstroke}{rgb}{0.000000,0.000000,0.000000}%
\pgfsetstrokecolor{currentstroke}%
\pgfsetdash{}{0pt}%
\pgfpathmoveto{\pgfqpoint{4.282525in}{0.691133in}}%
\pgfpathlineto{\pgfqpoint{4.251812in}{0.668853in}}%
\pgfusepath{stroke}%
\end{pgfscope}%
\begin{pgfscope}%
\pgfsetrectcap%
\pgfsetroundjoin%
\pgfsetlinewidth{0.803000pt}%
\definecolor{currentstroke}{rgb}{0.000000,0.000000,0.000000}%
\pgfsetstrokecolor{currentstroke}%
\pgfsetdash{}{0pt}%
\pgfpathmoveto{\pgfqpoint{4.523341in}{0.505076in}}%
\pgfpathlineto{\pgfqpoint{4.492609in}{0.482262in}}%
\pgfusepath{stroke}%
\end{pgfscope}%
\begin{pgfscope}%
\pgfsetrectcap%
\pgfsetroundjoin%
\pgfsetlinewidth{0.803000pt}%
\definecolor{currentstroke}{rgb}{0.000000,0.000000,0.000000}%
\pgfsetstrokecolor{currentstroke}%
\pgfsetdash{}{0pt}%
\pgfpathmoveto{\pgfqpoint{4.769953in}{0.314540in}}%
\pgfpathlineto{\pgfqpoint{4.739210in}{0.291173in}}%
\pgfusepath{stroke}%
\end{pgfscope}%
\begin{pgfscope}%
\definecolor{textcolor}{rgb}{0.000000,0.000000,0.000000}%
\pgfsetstrokecolor{textcolor}%
\pgfsetfillcolor{textcolor}%
\pgftext[x=4.042153in,y=0.241958in,,]{\color{textcolor}{\rmfamily\fontsize{14.000000}{16.800000}\selectfont\catcode`\^=\active\def^{\ifmmode\sp\else\^{}\fi}\catcode`\%=\active\def%{\%}f2}}%
\end{pgfscope}%
\begin{pgfscope}%
\pgfsetrectcap%
\pgfsetroundjoin%
\pgfsetlinewidth{0.803000pt}%
\definecolor{currentstroke}{rgb}{0.000000,0.000000,0.000000}%
\pgfsetstrokecolor{currentstroke}%
\pgfsetdash{}{0pt}%
\pgfpathmoveto{\pgfqpoint{3.734658in}{1.099507in}}%
\pgfpathlineto{\pgfqpoint{3.656001in}{2.033906in}}%
\pgfusepath{stroke}%
\end{pgfscope}%
\begin{pgfscope}%
\pgfsetrectcap%
\pgfsetroundjoin%
\pgfsetlinewidth{0.803000pt}%
\definecolor{currentstroke}{rgb}{0.000000,0.000000,0.000000}%
\pgfsetstrokecolor{currentstroke}%
\pgfsetdash{}{0pt}%
\pgfpathmoveto{\pgfqpoint{3.740188in}{1.162580in}}%
\pgfpathlineto{\pgfqpoint{3.709419in}{1.141464in}}%
\pgfusepath{stroke}%
\end{pgfscope}%
\begin{pgfscope}%
\pgfsetrectcap%
\pgfsetroundjoin%
\pgfsetlinewidth{0.803000pt}%
\definecolor{currentstroke}{rgb}{0.000000,0.000000,0.000000}%
\pgfsetstrokecolor{currentstroke}%
\pgfsetdash{}{0pt}%
\pgfpathmoveto{\pgfqpoint{3.725377in}{1.340103in}}%
\pgfpathlineto{\pgfqpoint{3.694208in}{1.318976in}}%
\pgfusepath{stroke}%
\end{pgfscope}%
\begin{pgfscope}%
\pgfsetrectcap%
\pgfsetroundjoin%
\pgfsetlinewidth{0.803000pt}%
\definecolor{currentstroke}{rgb}{0.000000,0.000000,0.000000}%
\pgfsetstrokecolor{currentstroke}%
\pgfsetdash{}{0pt}%
\pgfpathmoveto{\pgfqpoint{3.710198in}{1.522058in}}%
\pgfpathlineto{\pgfqpoint{3.678617in}{1.500926in}}%
\pgfusepath{stroke}%
\end{pgfscope}%
\begin{pgfscope}%
\pgfsetrectcap%
\pgfsetroundjoin%
\pgfsetlinewidth{0.803000pt}%
\definecolor{currentstroke}{rgb}{0.000000,0.000000,0.000000}%
\pgfsetstrokecolor{currentstroke}%
\pgfsetdash{}{0pt}%
\pgfpathmoveto{\pgfqpoint{3.694634in}{1.708614in}}%
\pgfpathlineto{\pgfqpoint{3.662631in}{1.687484in}}%
\pgfusepath{stroke}%
\end{pgfscope}%
\begin{pgfscope}%
\pgfsetrectcap%
\pgfsetroundjoin%
\pgfsetlinewidth{0.803000pt}%
\definecolor{currentstroke}{rgb}{0.000000,0.000000,0.000000}%
\pgfsetstrokecolor{currentstroke}%
\pgfsetdash{}{0pt}%
\pgfpathmoveto{\pgfqpoint{3.678672in}{1.899948in}}%
\pgfpathlineto{\pgfqpoint{3.646235in}{1.878827in}}%
\pgfusepath{stroke}%
\end{pgfscope}%
\begin{pgfscope}%
\definecolor{textcolor}{rgb}{0.000000,0.000000,0.000000}%
\pgfsetstrokecolor{textcolor}%
\pgfsetfillcolor{textcolor}%
\pgftext[x=3.138409in,y=1.551958in,,]{\color{textcolor}{\rmfamily\fontsize{14.000000}{16.800000}\selectfont\catcode`\^=\active\def^{\ifmmode\sp\else\^{}\fi}\catcode`\%=\active\def%{\%}f3}}%
\end{pgfscope}%
\begin{pgfscope}%
\pgfpathrectangle{\pgfqpoint{3.536584in}{0.147348in}}{\pgfqpoint{2.735294in}{2.735294in}}%
\pgfusepath{clip}%
\pgfsetbuttcap%
\pgfsetroundjoin%
\definecolor{currentfill}{rgb}{0.050070,0.192203,0.290728}%
\pgfsetfillcolor{currentfill}%
\pgfsetlinewidth{0.000000pt}%
\definecolor{currentstroke}{rgb}{0.000000,0.000000,0.000000}%
\pgfsetstrokecolor{currentstroke}%
\pgfsetdash{}{0pt}%
\pgfpathmoveto{\pgfqpoint{5.897888in}{1.291641in}}%
\pgfpathlineto{\pgfqpoint{5.810696in}{1.165011in}}%
\pgfpathlineto{\pgfqpoint{5.897793in}{1.225003in}}%
\pgfpathlineto{\pgfqpoint{5.897888in}{1.291641in}}%
\pgfpathclose%
\pgfusepath{fill}%
\end{pgfscope}%
\begin{pgfscope}%
\pgfpathrectangle{\pgfqpoint{3.536584in}{0.147348in}}{\pgfqpoint{2.735294in}{2.735294in}}%
\pgfusepath{clip}%
\pgfsetbuttcap%
\pgfsetroundjoin%
\definecolor{currentfill}{rgb}{0.050070,0.192203,0.290728}%
\pgfsetfillcolor{currentfill}%
\pgfsetlinewidth{0.000000pt}%
\definecolor{currentstroke}{rgb}{0.000000,0.000000,0.000000}%
\pgfsetstrokecolor{currentstroke}%
\pgfsetdash{}{0pt}%
\pgfpathmoveto{\pgfqpoint{4.071693in}{1.165011in}}%
\pgfpathlineto{\pgfqpoint{3.984502in}{1.291641in}}%
\pgfpathlineto{\pgfqpoint{3.984596in}{1.225003in}}%
\pgfpathlineto{\pgfqpoint{4.071693in}{1.165011in}}%
\pgfpathclose%
\pgfusepath{fill}%
\end{pgfscope}%
\begin{pgfscope}%
\pgfpathrectangle{\pgfqpoint{3.536584in}{0.147348in}}{\pgfqpoint{2.735294in}{2.735294in}}%
\pgfusepath{clip}%
\pgfsetbuttcap%
\pgfsetroundjoin%
\definecolor{currentfill}{rgb}{0.090605,0.347808,0.526096}%
\pgfsetfillcolor{currentfill}%
\pgfsetlinewidth{0.000000pt}%
\definecolor{currentstroke}{rgb}{0.000000,0.000000,0.000000}%
\pgfsetstrokecolor{currentstroke}%
\pgfsetdash{}{0pt}%
\pgfpathmoveto{\pgfqpoint{4.854003in}{2.561465in}}%
\pgfpathlineto{\pgfqpoint{5.028387in}{2.561465in}}%
\pgfpathlineto{\pgfqpoint{4.941195in}{2.621838in}}%
\pgfpathlineto{\pgfqpoint{4.854003in}{2.561465in}}%
\pgfpathclose%
\pgfusepath{fill}%
\end{pgfscope}%
\begin{pgfscope}%
\pgfpathrectangle{\pgfqpoint{3.536584in}{0.147348in}}{\pgfqpoint{2.735294in}{2.735294in}}%
\pgfusepath{clip}%
\pgfsetbuttcap%
\pgfsetroundjoin%
\definecolor{currentfill}{rgb}{0.047548,0.182523,0.276086}%
\pgfsetfillcolor{currentfill}%
\pgfsetlinewidth{0.000000pt}%
\definecolor{currentstroke}{rgb}{0.000000,0.000000,0.000000}%
\pgfsetstrokecolor{currentstroke}%
\pgfsetdash{}{0pt}%
\pgfpathmoveto{\pgfqpoint{5.810696in}{1.165011in}}%
\pgfpathlineto{\pgfqpoint{5.801292in}{1.232896in}}%
\pgfpathlineto{\pgfqpoint{5.697728in}{1.101844in}}%
\pgfpathlineto{\pgfqpoint{5.810696in}{1.165011in}}%
\pgfpathclose%
\pgfusepath{fill}%
\end{pgfscope}%
\begin{pgfscope}%
\pgfpathrectangle{\pgfqpoint{3.536584in}{0.147348in}}{\pgfqpoint{2.735294in}{2.735294in}}%
\pgfusepath{clip}%
\pgfsetbuttcap%
\pgfsetroundjoin%
\definecolor{currentfill}{rgb}{0.047548,0.182523,0.276086}%
\pgfsetfillcolor{currentfill}%
\pgfsetlinewidth{0.000000pt}%
\definecolor{currentstroke}{rgb}{0.000000,0.000000,0.000000}%
\pgfsetstrokecolor{currentstroke}%
\pgfsetdash{}{0pt}%
\pgfpathmoveto{\pgfqpoint{4.184662in}{1.101844in}}%
\pgfpathlineto{\pgfqpoint{4.081098in}{1.232896in}}%
\pgfpathlineto{\pgfqpoint{4.071693in}{1.165011in}}%
\pgfpathlineto{\pgfqpoint{4.184662in}{1.101844in}}%
\pgfpathclose%
\pgfusepath{fill}%
\end{pgfscope}%
\begin{pgfscope}%
\pgfpathrectangle{\pgfqpoint{3.536584in}{0.147348in}}{\pgfqpoint{2.735294in}{2.735294in}}%
\pgfusepath{clip}%
\pgfsetbuttcap%
\pgfsetroundjoin%
\definecolor{currentfill}{rgb}{0.048960,0.187944,0.284285}%
\pgfsetfillcolor{currentfill}%
\pgfsetlinewidth{0.000000pt}%
\definecolor{currentstroke}{rgb}{0.000000,0.000000,0.000000}%
\pgfsetstrokecolor{currentstroke}%
\pgfsetdash{}{0pt}%
\pgfpathmoveto{\pgfqpoint{3.984502in}{1.291641in}}%
\pgfpathlineto{\pgfqpoint{4.071693in}{1.165011in}}%
\pgfpathlineto{\pgfqpoint{4.070905in}{1.618388in}}%
\pgfpathlineto{\pgfqpoint{3.984502in}{1.291641in}}%
\pgfpathclose%
\pgfusepath{fill}%
\end{pgfscope}%
\begin{pgfscope}%
\pgfpathrectangle{\pgfqpoint{3.536584in}{0.147348in}}{\pgfqpoint{2.735294in}{2.735294in}}%
\pgfusepath{clip}%
\pgfsetbuttcap%
\pgfsetroundjoin%
\definecolor{currentfill}{rgb}{0.048960,0.187944,0.284285}%
\pgfsetfillcolor{currentfill}%
\pgfsetlinewidth{0.000000pt}%
\definecolor{currentstroke}{rgb}{0.000000,0.000000,0.000000}%
\pgfsetstrokecolor{currentstroke}%
\pgfsetdash{}{0pt}%
\pgfpathmoveto{\pgfqpoint{5.897888in}{1.291641in}}%
\pgfpathlineto{\pgfqpoint{5.811485in}{1.618388in}}%
\pgfpathlineto{\pgfqpoint{5.810696in}{1.165011in}}%
\pgfpathlineto{\pgfqpoint{5.897888in}{1.291641in}}%
\pgfpathclose%
\pgfusepath{fill}%
\end{pgfscope}%
\begin{pgfscope}%
\pgfpathrectangle{\pgfqpoint{3.536584in}{0.147348in}}{\pgfqpoint{2.735294in}{2.735294in}}%
\pgfusepath{clip}%
\pgfsetbuttcap%
\pgfsetroundjoin%
\definecolor{currentfill}{rgb}{0.070254,0.269685,0.407928}%
\pgfsetfillcolor{currentfill}%
\pgfsetlinewidth{0.000000pt}%
\definecolor{currentstroke}{rgb}{0.000000,0.000000,0.000000}%
\pgfsetstrokecolor{currentstroke}%
\pgfsetdash{}{0pt}%
\pgfpathmoveto{\pgfqpoint{5.801292in}{1.232896in}}%
\pgfpathlineto{\pgfqpoint{5.810696in}{1.165011in}}%
\pgfpathlineto{\pgfqpoint{5.811485in}{1.618388in}}%
\pgfpathlineto{\pgfqpoint{5.801292in}{1.232896in}}%
\pgfpathclose%
\pgfusepath{fill}%
\end{pgfscope}%
\begin{pgfscope}%
\pgfpathrectangle{\pgfqpoint{3.536584in}{0.147348in}}{\pgfqpoint{2.735294in}{2.735294in}}%
\pgfusepath{clip}%
\pgfsetbuttcap%
\pgfsetroundjoin%
\definecolor{currentfill}{rgb}{0.070254,0.269685,0.407928}%
\pgfsetfillcolor{currentfill}%
\pgfsetlinewidth{0.000000pt}%
\definecolor{currentstroke}{rgb}{0.000000,0.000000,0.000000}%
\pgfsetstrokecolor{currentstroke}%
\pgfsetdash{}{0pt}%
\pgfpathmoveto{\pgfqpoint{4.070905in}{1.618388in}}%
\pgfpathlineto{\pgfqpoint{4.071693in}{1.165011in}}%
\pgfpathlineto{\pgfqpoint{4.081098in}{1.232896in}}%
\pgfpathlineto{\pgfqpoint{4.070905in}{1.618388in}}%
\pgfpathclose%
\pgfusepath{fill}%
\end{pgfscope}%
\begin{pgfscope}%
\pgfpathrectangle{\pgfqpoint{3.536584in}{0.147348in}}{\pgfqpoint{2.735294in}{2.735294in}}%
\pgfusepath{clip}%
\pgfsetbuttcap%
\pgfsetroundjoin%
\definecolor{currentfill}{rgb}{0.044978,0.172658,0.261163}%
\pgfsetfillcolor{currentfill}%
\pgfsetlinewidth{0.000000pt}%
\definecolor{currentstroke}{rgb}{0.000000,0.000000,0.000000}%
\pgfsetstrokecolor{currentstroke}%
\pgfsetdash{}{0pt}%
\pgfpathmoveto{\pgfqpoint{4.328767in}{1.039603in}}%
\pgfpathlineto{\pgfqpoint{4.209060in}{1.171211in}}%
\pgfpathlineto{\pgfqpoint{4.184662in}{1.101844in}}%
\pgfpathlineto{\pgfqpoint{4.328767in}{1.039603in}}%
\pgfpathclose%
\pgfusepath{fill}%
\end{pgfscope}%
\begin{pgfscope}%
\pgfpathrectangle{\pgfqpoint{3.536584in}{0.147348in}}{\pgfqpoint{2.735294in}{2.735294in}}%
\pgfusepath{clip}%
\pgfsetbuttcap%
\pgfsetroundjoin%
\definecolor{currentfill}{rgb}{0.044978,0.172658,0.261163}%
\pgfsetfillcolor{currentfill}%
\pgfsetlinewidth{0.000000pt}%
\definecolor{currentstroke}{rgb}{0.000000,0.000000,0.000000}%
\pgfsetstrokecolor{currentstroke}%
\pgfsetdash{}{0pt}%
\pgfpathmoveto{\pgfqpoint{5.697728in}{1.101844in}}%
\pgfpathlineto{\pgfqpoint{5.673330in}{1.171211in}}%
\pgfpathlineto{\pgfqpoint{5.553623in}{1.039603in}}%
\pgfpathlineto{\pgfqpoint{5.697728in}{1.101844in}}%
\pgfpathclose%
\pgfusepath{fill}%
\end{pgfscope}%
\begin{pgfscope}%
\pgfpathrectangle{\pgfqpoint{3.536584in}{0.147348in}}{\pgfqpoint{2.735294in}{2.735294in}}%
\pgfusepath{clip}%
\pgfsetbuttcap%
\pgfsetroundjoin%
\definecolor{currentfill}{rgb}{0.081954,0.314596,0.475860}%
\pgfsetfillcolor{currentfill}%
\pgfsetlinewidth{0.000000pt}%
\definecolor{currentstroke}{rgb}{0.000000,0.000000,0.000000}%
\pgfsetstrokecolor{currentstroke}%
\pgfsetdash{}{0pt}%
\pgfpathmoveto{\pgfqpoint{5.028387in}{2.561465in}}%
\pgfpathlineto{\pgfqpoint{4.854003in}{2.561465in}}%
\pgfpathlineto{\pgfqpoint{4.816407in}{2.124855in}}%
\pgfpathlineto{\pgfqpoint{5.028387in}{2.561465in}}%
\pgfpathclose%
\pgfusepath{fill}%
\end{pgfscope}%
\begin{pgfscope}%
\pgfpathrectangle{\pgfqpoint{3.536584in}{0.147348in}}{\pgfqpoint{2.735294in}{2.735294in}}%
\pgfusepath{clip}%
\pgfsetbuttcap%
\pgfsetroundjoin%
\definecolor{currentfill}{rgb}{0.047247,0.181368,0.274339}%
\pgfsetfillcolor{currentfill}%
\pgfsetlinewidth{0.000000pt}%
\definecolor{currentstroke}{rgb}{0.000000,0.000000,0.000000}%
\pgfsetstrokecolor{currentstroke}%
\pgfsetdash{}{0pt}%
\pgfpathmoveto{\pgfqpoint{5.664465in}{1.589448in}}%
\pgfpathlineto{\pgfqpoint{5.697728in}{1.101844in}}%
\pgfpathlineto{\pgfqpoint{5.801292in}{1.232896in}}%
\pgfpathlineto{\pgfqpoint{5.664465in}{1.589448in}}%
\pgfpathclose%
\pgfusepath{fill}%
\end{pgfscope}%
\begin{pgfscope}%
\pgfpathrectangle{\pgfqpoint{3.536584in}{0.147348in}}{\pgfqpoint{2.735294in}{2.735294in}}%
\pgfusepath{clip}%
\pgfsetbuttcap%
\pgfsetroundjoin%
\definecolor{currentfill}{rgb}{0.047247,0.181368,0.274339}%
\pgfsetfillcolor{currentfill}%
\pgfsetlinewidth{0.000000pt}%
\definecolor{currentstroke}{rgb}{0.000000,0.000000,0.000000}%
\pgfsetstrokecolor{currentstroke}%
\pgfsetdash{}{0pt}%
\pgfpathmoveto{\pgfqpoint{4.081098in}{1.232896in}}%
\pgfpathlineto{\pgfqpoint{4.184662in}{1.101844in}}%
\pgfpathlineto{\pgfqpoint{4.217925in}{1.589448in}}%
\pgfpathlineto{\pgfqpoint{4.081098in}{1.232896in}}%
\pgfpathclose%
\pgfusepath{fill}%
\end{pgfscope}%
\begin{pgfscope}%
\pgfpathrectangle{\pgfqpoint{3.536584in}{0.147348in}}{\pgfqpoint{2.735294in}{2.735294in}}%
\pgfusepath{clip}%
\pgfsetbuttcap%
\pgfsetroundjoin%
\definecolor{currentfill}{rgb}{0.067179,0.257880,0.390071}%
\pgfsetfillcolor{currentfill}%
\pgfsetlinewidth{0.000000pt}%
\definecolor{currentstroke}{rgb}{0.000000,0.000000,0.000000}%
\pgfsetstrokecolor{currentstroke}%
\pgfsetdash{}{0pt}%
\pgfpathmoveto{\pgfqpoint{4.217925in}{1.589448in}}%
\pgfpathlineto{\pgfqpoint{4.184662in}{1.101844in}}%
\pgfpathlineto{\pgfqpoint{4.209060in}{1.171211in}}%
\pgfpathlineto{\pgfqpoint{4.217925in}{1.589448in}}%
\pgfpathclose%
\pgfusepath{fill}%
\end{pgfscope}%
\begin{pgfscope}%
\pgfpathrectangle{\pgfqpoint{3.536584in}{0.147348in}}{\pgfqpoint{2.735294in}{2.735294in}}%
\pgfusepath{clip}%
\pgfsetbuttcap%
\pgfsetroundjoin%
\definecolor{currentfill}{rgb}{0.067179,0.257880,0.390071}%
\pgfsetfillcolor{currentfill}%
\pgfsetlinewidth{0.000000pt}%
\definecolor{currentstroke}{rgb}{0.000000,0.000000,0.000000}%
\pgfsetstrokecolor{currentstroke}%
\pgfsetdash{}{0pt}%
\pgfpathmoveto{\pgfqpoint{5.673330in}{1.171211in}}%
\pgfpathlineto{\pgfqpoint{5.697728in}{1.101844in}}%
\pgfpathlineto{\pgfqpoint{5.664465in}{1.589448in}}%
\pgfpathlineto{\pgfqpoint{5.673330in}{1.171211in}}%
\pgfpathclose%
\pgfusepath{fill}%
\end{pgfscope}%
\begin{pgfscope}%
\pgfpathrectangle{\pgfqpoint{3.536584in}{0.147348in}}{\pgfqpoint{2.735294in}{2.735294in}}%
\pgfusepath{clip}%
\pgfsetbuttcap%
\pgfsetroundjoin%
\definecolor{currentfill}{rgb}{0.042579,0.163449,0.247234}%
\pgfsetfillcolor{currentfill}%
\pgfsetlinewidth{0.000000pt}%
\definecolor{currentstroke}{rgb}{0.000000,0.000000,0.000000}%
\pgfsetstrokecolor{currentstroke}%
\pgfsetdash{}{0pt}%
\pgfpathmoveto{\pgfqpoint{4.328767in}{1.039603in}}%
\pgfpathlineto{\pgfqpoint{4.506598in}{0.985051in}}%
\pgfpathlineto{\pgfqpoint{4.374765in}{1.111775in}}%
\pgfpathlineto{\pgfqpoint{4.328767in}{1.039603in}}%
\pgfpathclose%
\pgfusepath{fill}%
\end{pgfscope}%
\begin{pgfscope}%
\pgfpathrectangle{\pgfqpoint{3.536584in}{0.147348in}}{\pgfqpoint{2.735294in}{2.735294in}}%
\pgfusepath{clip}%
\pgfsetbuttcap%
\pgfsetroundjoin%
\definecolor{currentfill}{rgb}{0.042579,0.163449,0.247234}%
\pgfsetfillcolor{currentfill}%
\pgfsetlinewidth{0.000000pt}%
\definecolor{currentstroke}{rgb}{0.000000,0.000000,0.000000}%
\pgfsetstrokecolor{currentstroke}%
\pgfsetdash{}{0pt}%
\pgfpathmoveto{\pgfqpoint{5.507624in}{1.111775in}}%
\pgfpathlineto{\pgfqpoint{5.375791in}{0.985051in}}%
\pgfpathlineto{\pgfqpoint{5.553623in}{1.039603in}}%
\pgfpathlineto{\pgfqpoint{5.507624in}{1.111775in}}%
\pgfpathclose%
\pgfusepath{fill}%
\end{pgfscope}%
\begin{pgfscope}%
\pgfpathrectangle{\pgfqpoint{3.536584in}{0.147348in}}{\pgfqpoint{2.735294in}{2.735294in}}%
\pgfusepath{clip}%
\pgfsetbuttcap%
\pgfsetroundjoin%
\definecolor{currentfill}{rgb}{0.052493,0.201505,0.304798}%
\pgfsetfillcolor{currentfill}%
\pgfsetlinewidth{0.000000pt}%
\definecolor{currentstroke}{rgb}{0.000000,0.000000,0.000000}%
\pgfsetstrokecolor{currentstroke}%
\pgfsetdash{}{0pt}%
\pgfpathmoveto{\pgfqpoint{5.664465in}{1.589448in}}%
\pgfpathlineto{\pgfqpoint{5.801292in}{1.232896in}}%
\pgfpathlineto{\pgfqpoint{5.811485in}{1.618388in}}%
\pgfpathlineto{\pgfqpoint{5.664465in}{1.589448in}}%
\pgfpathclose%
\pgfusepath{fill}%
\end{pgfscope}%
\begin{pgfscope}%
\pgfpathrectangle{\pgfqpoint{3.536584in}{0.147348in}}{\pgfqpoint{2.735294in}{2.735294in}}%
\pgfusepath{clip}%
\pgfsetbuttcap%
\pgfsetroundjoin%
\definecolor{currentfill}{rgb}{0.052493,0.201505,0.304798}%
\pgfsetfillcolor{currentfill}%
\pgfsetlinewidth{0.000000pt}%
\definecolor{currentstroke}{rgb}{0.000000,0.000000,0.000000}%
\pgfsetstrokecolor{currentstroke}%
\pgfsetdash{}{0pt}%
\pgfpathmoveto{\pgfqpoint{4.070905in}{1.618388in}}%
\pgfpathlineto{\pgfqpoint{4.081098in}{1.232896in}}%
\pgfpathlineto{\pgfqpoint{4.217925in}{1.589448in}}%
\pgfpathlineto{\pgfqpoint{4.070905in}{1.618388in}}%
\pgfpathclose%
\pgfusepath{fill}%
\end{pgfscope}%
\begin{pgfscope}%
\pgfpathrectangle{\pgfqpoint{3.536584in}{0.147348in}}{\pgfqpoint{2.735294in}{2.735294in}}%
\pgfusepath{clip}%
\pgfsetbuttcap%
\pgfsetroundjoin%
\definecolor{currentfill}{rgb}{0.082280,0.315849,0.477755}%
\pgfsetfillcolor{currentfill}%
\pgfsetlinewidth{0.000000pt}%
\definecolor{currentstroke}{rgb}{0.000000,0.000000,0.000000}%
\pgfsetstrokecolor{currentstroke}%
\pgfsetdash{}{0pt}%
\pgfpathmoveto{\pgfqpoint{5.376890in}{2.253326in}}%
\pgfpathlineto{\pgfqpoint{5.028387in}{2.561465in}}%
\pgfpathlineto{\pgfqpoint{5.065982in}{2.124855in}}%
\pgfpathlineto{\pgfqpoint{5.376890in}{2.253326in}}%
\pgfpathclose%
\pgfusepath{fill}%
\end{pgfscope}%
\begin{pgfscope}%
\pgfpathrectangle{\pgfqpoint{3.536584in}{0.147348in}}{\pgfqpoint{2.735294in}{2.735294in}}%
\pgfusepath{clip}%
\pgfsetbuttcap%
\pgfsetroundjoin%
\definecolor{currentfill}{rgb}{0.082280,0.315849,0.477755}%
\pgfsetfillcolor{currentfill}%
\pgfsetlinewidth{0.000000pt}%
\definecolor{currentstroke}{rgb}{0.000000,0.000000,0.000000}%
\pgfsetstrokecolor{currentstroke}%
\pgfsetdash{}{0pt}%
\pgfpathmoveto{\pgfqpoint{4.816407in}{2.124855in}}%
\pgfpathlineto{\pgfqpoint{4.854003in}{2.561465in}}%
\pgfpathlineto{\pgfqpoint{4.505500in}{2.253326in}}%
\pgfpathlineto{\pgfqpoint{4.816407in}{2.124855in}}%
\pgfpathclose%
\pgfusepath{fill}%
\end{pgfscope}%
\begin{pgfscope}%
\pgfpathrectangle{\pgfqpoint{3.536584in}{0.147348in}}{\pgfqpoint{2.735294in}{2.735294in}}%
\pgfusepath{clip}%
\pgfsetbuttcap%
\pgfsetroundjoin%
\definecolor{currentfill}{rgb}{0.045702,0.175435,0.265364}%
\pgfsetfillcolor{currentfill}%
\pgfsetlinewidth{0.000000pt}%
\definecolor{currentstroke}{rgb}{0.000000,0.000000,0.000000}%
\pgfsetstrokecolor{currentstroke}%
\pgfsetdash{}{0pt}%
\pgfpathmoveto{\pgfqpoint{4.209060in}{1.171211in}}%
\pgfpathlineto{\pgfqpoint{4.328767in}{1.039603in}}%
\pgfpathlineto{\pgfqpoint{4.414548in}{1.562087in}}%
\pgfpathlineto{\pgfqpoint{4.209060in}{1.171211in}}%
\pgfpathclose%
\pgfusepath{fill}%
\end{pgfscope}%
\begin{pgfscope}%
\pgfpathrectangle{\pgfqpoint{3.536584in}{0.147348in}}{\pgfqpoint{2.735294in}{2.735294in}}%
\pgfusepath{clip}%
\pgfsetbuttcap%
\pgfsetroundjoin%
\definecolor{currentfill}{rgb}{0.045702,0.175435,0.265364}%
\pgfsetfillcolor{currentfill}%
\pgfsetlinewidth{0.000000pt}%
\definecolor{currentstroke}{rgb}{0.000000,0.000000,0.000000}%
\pgfsetstrokecolor{currentstroke}%
\pgfsetdash{}{0pt}%
\pgfpathmoveto{\pgfqpoint{5.467842in}{1.562087in}}%
\pgfpathlineto{\pgfqpoint{5.553623in}{1.039603in}}%
\pgfpathlineto{\pgfqpoint{5.673330in}{1.171211in}}%
\pgfpathlineto{\pgfqpoint{5.467842in}{1.562087in}}%
\pgfpathclose%
\pgfusepath{fill}%
\end{pgfscope}%
\begin{pgfscope}%
\pgfpathrectangle{\pgfqpoint{3.536584in}{0.147348in}}{\pgfqpoint{2.735294in}{2.735294in}}%
\pgfusepath{clip}%
\pgfsetbuttcap%
\pgfsetroundjoin%
\definecolor{currentfill}{rgb}{0.040669,0.156116,0.236142}%
\pgfsetfillcolor{currentfill}%
\pgfsetlinewidth{0.000000pt}%
\definecolor{currentstroke}{rgb}{0.000000,0.000000,0.000000}%
\pgfsetstrokecolor{currentstroke}%
\pgfsetdash{}{0pt}%
\pgfpathmoveto{\pgfqpoint{4.579939in}{1.063020in}}%
\pgfpathlineto{\pgfqpoint{4.506598in}{0.985051in}}%
\pgfpathlineto{\pgfqpoint{4.714698in}{0.946905in}}%
\pgfpathlineto{\pgfqpoint{4.579939in}{1.063020in}}%
\pgfpathclose%
\pgfusepath{fill}%
\end{pgfscope}%
\begin{pgfscope}%
\pgfpathrectangle{\pgfqpoint{3.536584in}{0.147348in}}{\pgfqpoint{2.735294in}{2.735294in}}%
\pgfusepath{clip}%
\pgfsetbuttcap%
\pgfsetroundjoin%
\definecolor{currentfill}{rgb}{0.040669,0.156116,0.236142}%
\pgfsetfillcolor{currentfill}%
\pgfsetlinewidth{0.000000pt}%
\definecolor{currentstroke}{rgb}{0.000000,0.000000,0.000000}%
\pgfsetstrokecolor{currentstroke}%
\pgfsetdash{}{0pt}%
\pgfpathmoveto{\pgfqpoint{5.167692in}{0.946905in}}%
\pgfpathlineto{\pgfqpoint{5.375791in}{0.985051in}}%
\pgfpathlineto{\pgfqpoint{5.302451in}{1.063020in}}%
\pgfpathlineto{\pgfqpoint{5.167692in}{0.946905in}}%
\pgfpathclose%
\pgfusepath{fill}%
\end{pgfscope}%
\begin{pgfscope}%
\pgfpathrectangle{\pgfqpoint{3.536584in}{0.147348in}}{\pgfqpoint{2.735294in}{2.735294in}}%
\pgfusepath{clip}%
\pgfsetbuttcap%
\pgfsetroundjoin%
\definecolor{currentfill}{rgb}{0.063981,0.245604,0.371502}%
\pgfsetfillcolor{currentfill}%
\pgfsetlinewidth{0.000000pt}%
\definecolor{currentstroke}{rgb}{0.000000,0.000000,0.000000}%
\pgfsetstrokecolor{currentstroke}%
\pgfsetdash{}{0pt}%
\pgfpathmoveto{\pgfqpoint{4.414548in}{1.562087in}}%
\pgfpathlineto{\pgfqpoint{4.328767in}{1.039603in}}%
\pgfpathlineto{\pgfqpoint{4.374765in}{1.111775in}}%
\pgfpathlineto{\pgfqpoint{4.414548in}{1.562087in}}%
\pgfpathclose%
\pgfusepath{fill}%
\end{pgfscope}%
\begin{pgfscope}%
\pgfpathrectangle{\pgfqpoint{3.536584in}{0.147348in}}{\pgfqpoint{2.735294in}{2.735294in}}%
\pgfusepath{clip}%
\pgfsetbuttcap%
\pgfsetroundjoin%
\definecolor{currentfill}{rgb}{0.063981,0.245604,0.371502}%
\pgfsetfillcolor{currentfill}%
\pgfsetlinewidth{0.000000pt}%
\definecolor{currentstroke}{rgb}{0.000000,0.000000,0.000000}%
\pgfsetstrokecolor{currentstroke}%
\pgfsetdash{}{0pt}%
\pgfpathmoveto{\pgfqpoint{5.507624in}{1.111775in}}%
\pgfpathlineto{\pgfqpoint{5.553623in}{1.039603in}}%
\pgfpathlineto{\pgfqpoint{5.467842in}{1.562087in}}%
\pgfpathlineto{\pgfqpoint{5.507624in}{1.111775in}}%
\pgfpathclose%
\pgfusepath{fill}%
\end{pgfscope}%
\begin{pgfscope}%
\pgfpathrectangle{\pgfqpoint{3.536584in}{0.147348in}}{\pgfqpoint{2.735294in}{2.735294in}}%
\pgfusepath{clip}%
\pgfsetbuttcap%
\pgfsetroundjoin%
\definecolor{currentfill}{rgb}{0.060942,0.233938,0.353856}%
\pgfsetfillcolor{currentfill}%
\pgfsetlinewidth{0.000000pt}%
\definecolor{currentstroke}{rgb}{0.000000,0.000000,0.000000}%
\pgfsetstrokecolor{currentstroke}%
\pgfsetdash{}{0pt}%
\pgfpathmoveto{\pgfqpoint{4.217925in}{1.589448in}}%
\pgfpathlineto{\pgfqpoint{4.147150in}{1.772344in}}%
\pgfpathlineto{\pgfqpoint{4.070905in}{1.618388in}}%
\pgfpathlineto{\pgfqpoint{4.217925in}{1.589448in}}%
\pgfpathclose%
\pgfusepath{fill}%
\end{pgfscope}%
\begin{pgfscope}%
\pgfpathrectangle{\pgfqpoint{3.536584in}{0.147348in}}{\pgfqpoint{2.735294in}{2.735294in}}%
\pgfusepath{clip}%
\pgfsetbuttcap%
\pgfsetroundjoin%
\definecolor{currentfill}{rgb}{0.060942,0.233938,0.353856}%
\pgfsetfillcolor{currentfill}%
\pgfsetlinewidth{0.000000pt}%
\definecolor{currentstroke}{rgb}{0.000000,0.000000,0.000000}%
\pgfsetstrokecolor{currentstroke}%
\pgfsetdash{}{0pt}%
\pgfpathmoveto{\pgfqpoint{5.811485in}{1.618388in}}%
\pgfpathlineto{\pgfqpoint{5.735240in}{1.772344in}}%
\pgfpathlineto{\pgfqpoint{5.664465in}{1.589448in}}%
\pgfpathlineto{\pgfqpoint{5.811485in}{1.618388in}}%
\pgfpathclose%
\pgfusepath{fill}%
\end{pgfscope}%
\begin{pgfscope}%
\pgfpathrectangle{\pgfqpoint{3.536584in}{0.147348in}}{\pgfqpoint{2.735294in}{2.735294in}}%
\pgfusepath{clip}%
\pgfsetbuttcap%
\pgfsetroundjoin%
\definecolor{currentfill}{rgb}{0.081954,0.314596,0.475860}%
\pgfsetfillcolor{currentfill}%
\pgfsetlinewidth{0.000000pt}%
\definecolor{currentstroke}{rgb}{0.000000,0.000000,0.000000}%
\pgfsetstrokecolor{currentstroke}%
\pgfsetdash{}{0pt}%
\pgfpathmoveto{\pgfqpoint{4.816407in}{2.124855in}}%
\pgfpathlineto{\pgfqpoint{5.065982in}{2.124855in}}%
\pgfpathlineto{\pgfqpoint{5.028387in}{2.561465in}}%
\pgfpathlineto{\pgfqpoint{4.816407in}{2.124855in}}%
\pgfpathclose%
\pgfusepath{fill}%
\end{pgfscope}%
\begin{pgfscope}%
\pgfpathrectangle{\pgfqpoint{3.536584in}{0.147348in}}{\pgfqpoint{2.735294in}{2.735294in}}%
\pgfusepath{clip}%
\pgfsetbuttcap%
\pgfsetroundjoin%
\definecolor{currentfill}{rgb}{0.039595,0.151995,0.229908}%
\pgfsetfillcolor{currentfill}%
\pgfsetlinewidth{0.000000pt}%
\definecolor{currentstroke}{rgb}{0.000000,0.000000,0.000000}%
\pgfsetstrokecolor{currentstroke}%
\pgfsetdash{}{0pt}%
\pgfpathmoveto{\pgfqpoint{4.941195in}{0.933095in}}%
\pgfpathlineto{\pgfqpoint{4.816678in}{1.035006in}}%
\pgfpathlineto{\pgfqpoint{4.714698in}{0.946905in}}%
\pgfpathlineto{\pgfqpoint{4.941195in}{0.933095in}}%
\pgfpathclose%
\pgfusepath{fill}%
\end{pgfscope}%
\begin{pgfscope}%
\pgfpathrectangle{\pgfqpoint{3.536584in}{0.147348in}}{\pgfqpoint{2.735294in}{2.735294in}}%
\pgfusepath{clip}%
\pgfsetbuttcap%
\pgfsetroundjoin%
\definecolor{currentfill}{rgb}{0.039595,0.151995,0.229908}%
\pgfsetfillcolor{currentfill}%
\pgfsetlinewidth{0.000000pt}%
\definecolor{currentstroke}{rgb}{0.000000,0.000000,0.000000}%
\pgfsetstrokecolor{currentstroke}%
\pgfsetdash{}{0pt}%
\pgfpathmoveto{\pgfqpoint{5.167692in}{0.946905in}}%
\pgfpathlineto{\pgfqpoint{5.065712in}{1.035006in}}%
\pgfpathlineto{\pgfqpoint{4.941195in}{0.933095in}}%
\pgfpathlineto{\pgfqpoint{5.167692in}{0.946905in}}%
\pgfpathclose%
\pgfusepath{fill}%
\end{pgfscope}%
\begin{pgfscope}%
\pgfpathrectangle{\pgfqpoint{3.536584in}{0.147348in}}{\pgfqpoint{2.735294in}{2.735294in}}%
\pgfusepath{clip}%
\pgfsetbuttcap%
\pgfsetroundjoin%
\definecolor{currentfill}{rgb}{0.075436,0.289576,0.438014}%
\pgfsetfillcolor{currentfill}%
\pgfsetlinewidth{0.000000pt}%
\definecolor{currentstroke}{rgb}{0.000000,0.000000,0.000000}%
\pgfsetstrokecolor{currentstroke}%
\pgfsetdash{}{0pt}%
\pgfpathmoveto{\pgfqpoint{5.508743in}{2.103280in}}%
\pgfpathlineto{\pgfqpoint{5.376890in}{2.253326in}}%
\pgfpathlineto{\pgfqpoint{5.303210in}{2.116981in}}%
\pgfpathlineto{\pgfqpoint{5.508743in}{2.103280in}}%
\pgfpathclose%
\pgfusepath{fill}%
\end{pgfscope}%
\begin{pgfscope}%
\pgfpathrectangle{\pgfqpoint{3.536584in}{0.147348in}}{\pgfqpoint{2.735294in}{2.735294in}}%
\pgfusepath{clip}%
\pgfsetbuttcap%
\pgfsetroundjoin%
\definecolor{currentfill}{rgb}{0.075436,0.289576,0.438014}%
\pgfsetfillcolor{currentfill}%
\pgfsetlinewidth{0.000000pt}%
\definecolor{currentstroke}{rgb}{0.000000,0.000000,0.000000}%
\pgfsetstrokecolor{currentstroke}%
\pgfsetdash{}{0pt}%
\pgfpathmoveto{\pgfqpoint{4.579180in}{2.116981in}}%
\pgfpathlineto{\pgfqpoint{4.505500in}{2.253326in}}%
\pgfpathlineto{\pgfqpoint{4.373646in}{2.103280in}}%
\pgfpathlineto{\pgfqpoint{4.579180in}{2.116981in}}%
\pgfpathclose%
\pgfusepath{fill}%
\end{pgfscope}%
\begin{pgfscope}%
\pgfpathrectangle{\pgfqpoint{3.536584in}{0.147348in}}{\pgfqpoint{2.735294in}{2.735294in}}%
\pgfusepath{clip}%
\pgfsetbuttcap%
\pgfsetroundjoin%
\definecolor{currentfill}{rgb}{0.062760,0.240916,0.364410}%
\pgfsetfillcolor{currentfill}%
\pgfsetlinewidth{0.000000pt}%
\definecolor{currentstroke}{rgb}{0.000000,0.000000,0.000000}%
\pgfsetstrokecolor{currentstroke}%
\pgfsetdash{}{0pt}%
\pgfpathmoveto{\pgfqpoint{4.217925in}{1.589448in}}%
\pgfpathlineto{\pgfqpoint{4.373646in}{2.103280in}}%
\pgfpathlineto{\pgfqpoint{4.147150in}{1.772344in}}%
\pgfpathlineto{\pgfqpoint{4.217925in}{1.589448in}}%
\pgfpathclose%
\pgfusepath{fill}%
\end{pgfscope}%
\begin{pgfscope}%
\pgfpathrectangle{\pgfqpoint{3.536584in}{0.147348in}}{\pgfqpoint{2.735294in}{2.735294in}}%
\pgfusepath{clip}%
\pgfsetbuttcap%
\pgfsetroundjoin%
\definecolor{currentfill}{rgb}{0.062760,0.240916,0.364410}%
\pgfsetfillcolor{currentfill}%
\pgfsetlinewidth{0.000000pt}%
\definecolor{currentstroke}{rgb}{0.000000,0.000000,0.000000}%
\pgfsetstrokecolor{currentstroke}%
\pgfsetdash{}{0pt}%
\pgfpathmoveto{\pgfqpoint{5.735240in}{1.772344in}}%
\pgfpathlineto{\pgfqpoint{5.508743in}{2.103280in}}%
\pgfpathlineto{\pgfqpoint{5.664465in}{1.589448in}}%
\pgfpathlineto{\pgfqpoint{5.735240in}{1.772344in}}%
\pgfpathclose%
\pgfusepath{fill}%
\end{pgfscope}%
\begin{pgfscope}%
\pgfpathrectangle{\pgfqpoint{3.536584in}{0.147348in}}{\pgfqpoint{2.735294in}{2.735294in}}%
\pgfusepath{clip}%
\pgfsetbuttcap%
\pgfsetroundjoin%
\definecolor{currentfill}{rgb}{0.043508,0.167016,0.252629}%
\pgfsetfillcolor{currentfill}%
\pgfsetlinewidth{0.000000pt}%
\definecolor{currentstroke}{rgb}{0.000000,0.000000,0.000000}%
\pgfsetstrokecolor{currentstroke}%
\pgfsetdash{}{0pt}%
\pgfpathmoveto{\pgfqpoint{4.374765in}{1.111775in}}%
\pgfpathlineto{\pgfqpoint{4.506598in}{0.985051in}}%
\pgfpathlineto{\pgfqpoint{4.538549in}{1.360106in}}%
\pgfpathlineto{\pgfqpoint{4.374765in}{1.111775in}}%
\pgfpathclose%
\pgfusepath{fill}%
\end{pgfscope}%
\begin{pgfscope}%
\pgfpathrectangle{\pgfqpoint{3.536584in}{0.147348in}}{\pgfqpoint{2.735294in}{2.735294in}}%
\pgfusepath{clip}%
\pgfsetbuttcap%
\pgfsetroundjoin%
\definecolor{currentfill}{rgb}{0.043508,0.167016,0.252629}%
\pgfsetfillcolor{currentfill}%
\pgfsetlinewidth{0.000000pt}%
\definecolor{currentstroke}{rgb}{0.000000,0.000000,0.000000}%
\pgfsetstrokecolor{currentstroke}%
\pgfsetdash{}{0pt}%
\pgfpathmoveto{\pgfqpoint{5.343841in}{1.360106in}}%
\pgfpathlineto{\pgfqpoint{5.375791in}{0.985051in}}%
\pgfpathlineto{\pgfqpoint{5.507624in}{1.111775in}}%
\pgfpathlineto{\pgfqpoint{5.343841in}{1.360106in}}%
\pgfpathclose%
\pgfusepath{fill}%
\end{pgfscope}%
\begin{pgfscope}%
\pgfpathrectangle{\pgfqpoint{3.536584in}{0.147348in}}{\pgfqpoint{2.735294in}{2.735294in}}%
\pgfusepath{clip}%
\pgfsetbuttcap%
\pgfsetroundjoin%
\definecolor{currentfill}{rgb}{0.050011,0.191979,0.290388}%
\pgfsetfillcolor{currentfill}%
\pgfsetlinewidth{0.000000pt}%
\definecolor{currentstroke}{rgb}{0.000000,0.000000,0.000000}%
\pgfsetstrokecolor{currentstroke}%
\pgfsetdash{}{0pt}%
\pgfpathmoveto{\pgfqpoint{4.217925in}{1.589448in}}%
\pgfpathlineto{\pgfqpoint{4.209060in}{1.171211in}}%
\pgfpathlineto{\pgfqpoint{4.414548in}{1.562087in}}%
\pgfpathlineto{\pgfqpoint{4.217925in}{1.589448in}}%
\pgfpathclose%
\pgfusepath{fill}%
\end{pgfscope}%
\begin{pgfscope}%
\pgfpathrectangle{\pgfqpoint{3.536584in}{0.147348in}}{\pgfqpoint{2.735294in}{2.735294in}}%
\pgfusepath{clip}%
\pgfsetbuttcap%
\pgfsetroundjoin%
\definecolor{currentfill}{rgb}{0.050011,0.191979,0.290388}%
\pgfsetfillcolor{currentfill}%
\pgfsetlinewidth{0.000000pt}%
\definecolor{currentstroke}{rgb}{0.000000,0.000000,0.000000}%
\pgfsetstrokecolor{currentstroke}%
\pgfsetdash{}{0pt}%
\pgfpathmoveto{\pgfqpoint{5.467842in}{1.562087in}}%
\pgfpathlineto{\pgfqpoint{5.673330in}{1.171211in}}%
\pgfpathlineto{\pgfqpoint{5.664465in}{1.589448in}}%
\pgfpathlineto{\pgfqpoint{5.467842in}{1.562087in}}%
\pgfpathclose%
\pgfusepath{fill}%
\end{pgfscope}%
\begin{pgfscope}%
\pgfpathrectangle{\pgfqpoint{3.536584in}{0.147348in}}{\pgfqpoint{2.735294in}{2.735294in}}%
\pgfusepath{clip}%
\pgfsetbuttcap%
\pgfsetroundjoin%
\definecolor{currentfill}{rgb}{0.049941,0.191710,0.289982}%
\pgfsetfillcolor{currentfill}%
\pgfsetlinewidth{0.000000pt}%
\definecolor{currentstroke}{rgb}{0.000000,0.000000,0.000000}%
\pgfsetstrokecolor{currentstroke}%
\pgfsetdash{}{0pt}%
\pgfpathmoveto{\pgfqpoint{5.302451in}{1.063020in}}%
\pgfpathlineto{\pgfqpoint{5.375791in}{0.985051in}}%
\pgfpathlineto{\pgfqpoint{5.343841in}{1.360106in}}%
\pgfpathlineto{\pgfqpoint{5.302451in}{1.063020in}}%
\pgfpathclose%
\pgfusepath{fill}%
\end{pgfscope}%
\begin{pgfscope}%
\pgfpathrectangle{\pgfqpoint{3.536584in}{0.147348in}}{\pgfqpoint{2.735294in}{2.735294in}}%
\pgfusepath{clip}%
\pgfsetbuttcap%
\pgfsetroundjoin%
\definecolor{currentfill}{rgb}{0.049941,0.191710,0.289982}%
\pgfsetfillcolor{currentfill}%
\pgfsetlinewidth{0.000000pt}%
\definecolor{currentstroke}{rgb}{0.000000,0.000000,0.000000}%
\pgfsetstrokecolor{currentstroke}%
\pgfsetdash{}{0pt}%
\pgfpathmoveto{\pgfqpoint{4.538549in}{1.360106in}}%
\pgfpathlineto{\pgfqpoint{4.506598in}{0.985051in}}%
\pgfpathlineto{\pgfqpoint{4.579939in}{1.063020in}}%
\pgfpathlineto{\pgfqpoint{4.538549in}{1.360106in}}%
\pgfpathclose%
\pgfusepath{fill}%
\end{pgfscope}%
\begin{pgfscope}%
\pgfpathrectangle{\pgfqpoint{3.536584in}{0.147348in}}{\pgfqpoint{2.735294in}{2.735294in}}%
\pgfusepath{clip}%
\pgfsetbuttcap%
\pgfsetroundjoin%
\definecolor{currentfill}{rgb}{0.078663,0.301965,0.456754}%
\pgfsetfillcolor{currentfill}%
\pgfsetlinewidth{0.000000pt}%
\definecolor{currentstroke}{rgb}{0.000000,0.000000,0.000000}%
\pgfsetstrokecolor{currentstroke}%
\pgfsetdash{}{0pt}%
\pgfpathmoveto{\pgfqpoint{4.816407in}{2.124855in}}%
\pgfpathlineto{\pgfqpoint{4.505500in}{2.253326in}}%
\pgfpathlineto{\pgfqpoint{4.579180in}{2.116981in}}%
\pgfpathlineto{\pgfqpoint{4.816407in}{2.124855in}}%
\pgfpathclose%
\pgfusepath{fill}%
\end{pgfscope}%
\begin{pgfscope}%
\pgfpathrectangle{\pgfqpoint{3.536584in}{0.147348in}}{\pgfqpoint{2.735294in}{2.735294in}}%
\pgfusepath{clip}%
\pgfsetbuttcap%
\pgfsetroundjoin%
\definecolor{currentfill}{rgb}{0.078663,0.301965,0.456754}%
\pgfsetfillcolor{currentfill}%
\pgfsetlinewidth{0.000000pt}%
\definecolor{currentstroke}{rgb}{0.000000,0.000000,0.000000}%
\pgfsetstrokecolor{currentstroke}%
\pgfsetdash{}{0pt}%
\pgfpathmoveto{\pgfqpoint{5.303210in}{2.116981in}}%
\pgfpathlineto{\pgfqpoint{5.376890in}{2.253326in}}%
\pgfpathlineto{\pgfqpoint{5.065982in}{2.124855in}}%
\pgfpathlineto{\pgfqpoint{5.303210in}{2.116981in}}%
\pgfpathclose%
\pgfusepath{fill}%
\end{pgfscope}%
\begin{pgfscope}%
\pgfpathrectangle{\pgfqpoint{3.536584in}{0.147348in}}{\pgfqpoint{2.735294in}{2.735294in}}%
\pgfusepath{clip}%
\pgfsetbuttcap%
\pgfsetroundjoin%
\definecolor{currentfill}{rgb}{0.064954,0.249341,0.377155}%
\pgfsetfillcolor{currentfill}%
\pgfsetlinewidth{0.000000pt}%
\definecolor{currentstroke}{rgb}{0.000000,0.000000,0.000000}%
\pgfsetstrokecolor{currentstroke}%
\pgfsetdash{}{0pt}%
\pgfpathmoveto{\pgfqpoint{4.414548in}{1.562087in}}%
\pgfpathlineto{\pgfqpoint{4.373646in}{2.103280in}}%
\pgfpathlineto{\pgfqpoint{4.217925in}{1.589448in}}%
\pgfpathlineto{\pgfqpoint{4.414548in}{1.562087in}}%
\pgfpathclose%
\pgfusepath{fill}%
\end{pgfscope}%
\begin{pgfscope}%
\pgfpathrectangle{\pgfqpoint{3.536584in}{0.147348in}}{\pgfqpoint{2.735294in}{2.735294in}}%
\pgfusepath{clip}%
\pgfsetbuttcap%
\pgfsetroundjoin%
\definecolor{currentfill}{rgb}{0.064954,0.249341,0.377155}%
\pgfsetfillcolor{currentfill}%
\pgfsetlinewidth{0.000000pt}%
\definecolor{currentstroke}{rgb}{0.000000,0.000000,0.000000}%
\pgfsetstrokecolor{currentstroke}%
\pgfsetdash{}{0pt}%
\pgfpathmoveto{\pgfqpoint{5.664465in}{1.589448in}}%
\pgfpathlineto{\pgfqpoint{5.508743in}{2.103280in}}%
\pgfpathlineto{\pgfqpoint{5.467842in}{1.562087in}}%
\pgfpathlineto{\pgfqpoint{5.664465in}{1.589448in}}%
\pgfpathclose%
\pgfusepath{fill}%
\end{pgfscope}%
\begin{pgfscope}%
\pgfpathrectangle{\pgfqpoint{3.536584in}{0.147348in}}{\pgfqpoint{2.735294in}{2.735294in}}%
\pgfusepath{clip}%
\pgfsetbuttcap%
\pgfsetroundjoin%
\definecolor{currentfill}{rgb}{0.042669,0.163794,0.247755}%
\pgfsetfillcolor{currentfill}%
\pgfsetlinewidth{0.000000pt}%
\definecolor{currentstroke}{rgb}{0.000000,0.000000,0.000000}%
\pgfsetstrokecolor{currentstroke}%
\pgfsetdash{}{0pt}%
\pgfpathmoveto{\pgfqpoint{4.714698in}{0.946905in}}%
\pgfpathlineto{\pgfqpoint{4.801285in}{1.337788in}}%
\pgfpathlineto{\pgfqpoint{4.579939in}{1.063020in}}%
\pgfpathlineto{\pgfqpoint{4.714698in}{0.946905in}}%
\pgfpathclose%
\pgfusepath{fill}%
\end{pgfscope}%
\begin{pgfscope}%
\pgfpathrectangle{\pgfqpoint{3.536584in}{0.147348in}}{\pgfqpoint{2.735294in}{2.735294in}}%
\pgfusepath{clip}%
\pgfsetbuttcap%
\pgfsetroundjoin%
\definecolor{currentfill}{rgb}{0.042669,0.163794,0.247755}%
\pgfsetfillcolor{currentfill}%
\pgfsetlinewidth{0.000000pt}%
\definecolor{currentstroke}{rgb}{0.000000,0.000000,0.000000}%
\pgfsetstrokecolor{currentstroke}%
\pgfsetdash{}{0pt}%
\pgfpathmoveto{\pgfqpoint{5.302451in}{1.063020in}}%
\pgfpathlineto{\pgfqpoint{5.081105in}{1.337788in}}%
\pgfpathlineto{\pgfqpoint{5.167692in}{0.946905in}}%
\pgfpathlineto{\pgfqpoint{5.302451in}{1.063020in}}%
\pgfpathclose%
\pgfusepath{fill}%
\end{pgfscope}%
\begin{pgfscope}%
\pgfpathrectangle{\pgfqpoint{3.536584in}{0.147348in}}{\pgfqpoint{2.735294in}{2.735294in}}%
\pgfusepath{clip}%
\pgfsetbuttcap%
\pgfsetroundjoin%
\definecolor{currentfill}{rgb}{0.068541,0.263111,0.397982}%
\pgfsetfillcolor{currentfill}%
\pgfsetlinewidth{0.000000pt}%
\definecolor{currentstroke}{rgb}{0.000000,0.000000,0.000000}%
\pgfsetstrokecolor{currentstroke}%
\pgfsetdash{}{0pt}%
\pgfpathmoveto{\pgfqpoint{4.579180in}{2.116981in}}%
\pgfpathlineto{\pgfqpoint{4.373646in}{2.103280in}}%
\pgfpathlineto{\pgfqpoint{4.677670in}{1.946079in}}%
\pgfpathlineto{\pgfqpoint{4.579180in}{2.116981in}}%
\pgfpathclose%
\pgfusepath{fill}%
\end{pgfscope}%
\begin{pgfscope}%
\pgfpathrectangle{\pgfqpoint{3.536584in}{0.147348in}}{\pgfqpoint{2.735294in}{2.735294in}}%
\pgfusepath{clip}%
\pgfsetbuttcap%
\pgfsetroundjoin%
\definecolor{currentfill}{rgb}{0.068541,0.263111,0.397982}%
\pgfsetfillcolor{currentfill}%
\pgfsetlinewidth{0.000000pt}%
\definecolor{currentstroke}{rgb}{0.000000,0.000000,0.000000}%
\pgfsetstrokecolor{currentstroke}%
\pgfsetdash{}{0pt}%
\pgfpathmoveto{\pgfqpoint{5.204720in}{1.946079in}}%
\pgfpathlineto{\pgfqpoint{5.508743in}{2.103280in}}%
\pgfpathlineto{\pgfqpoint{5.303210in}{2.116981in}}%
\pgfpathlineto{\pgfqpoint{5.204720in}{1.946079in}}%
\pgfpathclose%
\pgfusepath{fill}%
\end{pgfscope}%
\begin{pgfscope}%
\pgfpathrectangle{\pgfqpoint{3.536584in}{0.147348in}}{\pgfqpoint{2.735294in}{2.735294in}}%
\pgfusepath{clip}%
\pgfsetbuttcap%
\pgfsetroundjoin%
\definecolor{currentfill}{rgb}{0.047555,0.182548,0.276123}%
\pgfsetfillcolor{currentfill}%
\pgfsetlinewidth{0.000000pt}%
\definecolor{currentstroke}{rgb}{0.000000,0.000000,0.000000}%
\pgfsetstrokecolor{currentstroke}%
\pgfsetdash{}{0pt}%
\pgfpathmoveto{\pgfqpoint{4.714698in}{0.946905in}}%
\pgfpathlineto{\pgfqpoint{4.816678in}{1.035006in}}%
\pgfpathlineto{\pgfqpoint{4.801285in}{1.337788in}}%
\pgfpathlineto{\pgfqpoint{4.714698in}{0.946905in}}%
\pgfpathclose%
\pgfusepath{fill}%
\end{pgfscope}%
\begin{pgfscope}%
\pgfpathrectangle{\pgfqpoint{3.536584in}{0.147348in}}{\pgfqpoint{2.735294in}{2.735294in}}%
\pgfusepath{clip}%
\pgfsetbuttcap%
\pgfsetroundjoin%
\definecolor{currentfill}{rgb}{0.047555,0.182548,0.276123}%
\pgfsetfillcolor{currentfill}%
\pgfsetlinewidth{0.000000pt}%
\definecolor{currentstroke}{rgb}{0.000000,0.000000,0.000000}%
\pgfsetstrokecolor{currentstroke}%
\pgfsetdash{}{0pt}%
\pgfpathmoveto{\pgfqpoint{5.081105in}{1.337788in}}%
\pgfpathlineto{\pgfqpoint{5.065712in}{1.035006in}}%
\pgfpathlineto{\pgfqpoint{5.167692in}{0.946905in}}%
\pgfpathlineto{\pgfqpoint{5.081105in}{1.337788in}}%
\pgfpathclose%
\pgfusepath{fill}%
\end{pgfscope}%
\begin{pgfscope}%
\pgfpathrectangle{\pgfqpoint{3.536584in}{0.147348in}}{\pgfqpoint{2.735294in}{2.735294in}}%
\pgfusepath{clip}%
\pgfsetbuttcap%
\pgfsetroundjoin%
\definecolor{currentfill}{rgb}{0.046101,0.176968,0.267683}%
\pgfsetfillcolor{currentfill}%
\pgfsetlinewidth{0.000000pt}%
\definecolor{currentstroke}{rgb}{0.000000,0.000000,0.000000}%
\pgfsetstrokecolor{currentstroke}%
\pgfsetdash{}{0pt}%
\pgfpathmoveto{\pgfqpoint{4.941195in}{0.933095in}}%
\pgfpathlineto{\pgfqpoint{4.801285in}{1.337788in}}%
\pgfpathlineto{\pgfqpoint{4.816678in}{1.035006in}}%
\pgfpathlineto{\pgfqpoint{4.941195in}{0.933095in}}%
\pgfpathclose%
\pgfusepath{fill}%
\end{pgfscope}%
\begin{pgfscope}%
\pgfpathrectangle{\pgfqpoint{3.536584in}{0.147348in}}{\pgfqpoint{2.735294in}{2.735294in}}%
\pgfusepath{clip}%
\pgfsetbuttcap%
\pgfsetroundjoin%
\definecolor{currentfill}{rgb}{0.046101,0.176968,0.267683}%
\pgfsetfillcolor{currentfill}%
\pgfsetlinewidth{0.000000pt}%
\definecolor{currentstroke}{rgb}{0.000000,0.000000,0.000000}%
\pgfsetstrokecolor{currentstroke}%
\pgfsetdash{}{0pt}%
\pgfpathmoveto{\pgfqpoint{5.065712in}{1.035006in}}%
\pgfpathlineto{\pgfqpoint{5.081105in}{1.337788in}}%
\pgfpathlineto{\pgfqpoint{4.941195in}{0.933095in}}%
\pgfpathlineto{\pgfqpoint{5.065712in}{1.035006in}}%
\pgfpathclose%
\pgfusepath{fill}%
\end{pgfscope}%
\begin{pgfscope}%
\pgfpathrectangle{\pgfqpoint{3.536584in}{0.147348in}}{\pgfqpoint{2.735294in}{2.735294in}}%
\pgfusepath{clip}%
\pgfsetbuttcap%
\pgfsetroundjoin%
\definecolor{currentfill}{rgb}{0.051850,0.199036,0.301063}%
\pgfsetfillcolor{currentfill}%
\pgfsetlinewidth{0.000000pt}%
\definecolor{currentstroke}{rgb}{0.000000,0.000000,0.000000}%
\pgfsetstrokecolor{currentstroke}%
\pgfsetdash{}{0pt}%
\pgfpathmoveto{\pgfqpoint{4.374765in}{1.111775in}}%
\pgfpathlineto{\pgfqpoint{4.538549in}{1.360106in}}%
\pgfpathlineto{\pgfqpoint{4.414548in}{1.562087in}}%
\pgfpathlineto{\pgfqpoint{4.374765in}{1.111775in}}%
\pgfpathclose%
\pgfusepath{fill}%
\end{pgfscope}%
\begin{pgfscope}%
\pgfpathrectangle{\pgfqpoint{3.536584in}{0.147348in}}{\pgfqpoint{2.735294in}{2.735294in}}%
\pgfusepath{clip}%
\pgfsetbuttcap%
\pgfsetroundjoin%
\definecolor{currentfill}{rgb}{0.051850,0.199036,0.301063}%
\pgfsetfillcolor{currentfill}%
\pgfsetlinewidth{0.000000pt}%
\definecolor{currentstroke}{rgb}{0.000000,0.000000,0.000000}%
\pgfsetstrokecolor{currentstroke}%
\pgfsetdash{}{0pt}%
\pgfpathmoveto{\pgfqpoint{5.467842in}{1.562087in}}%
\pgfpathlineto{\pgfqpoint{5.343841in}{1.360106in}}%
\pgfpathlineto{\pgfqpoint{5.507624in}{1.111775in}}%
\pgfpathlineto{\pgfqpoint{5.467842in}{1.562087in}}%
\pgfpathclose%
\pgfusepath{fill}%
\end{pgfscope}%
\begin{pgfscope}%
\pgfpathrectangle{\pgfqpoint{3.536584in}{0.147348in}}{\pgfqpoint{2.735294in}{2.735294in}}%
\pgfusepath{clip}%
\pgfsetbuttcap%
\pgfsetroundjoin%
\definecolor{currentfill}{rgb}{0.064759,0.248590,0.376018}%
\pgfsetfillcolor{currentfill}%
\pgfsetlinewidth{0.000000pt}%
\definecolor{currentstroke}{rgb}{0.000000,0.000000,0.000000}%
\pgfsetstrokecolor{currentstroke}%
\pgfsetdash{}{0pt}%
\pgfpathmoveto{\pgfqpoint{5.508743in}{2.103280in}}%
\pgfpathlineto{\pgfqpoint{5.204720in}{1.946079in}}%
\pgfpathlineto{\pgfqpoint{5.467842in}{1.562087in}}%
\pgfpathlineto{\pgfqpoint{5.508743in}{2.103280in}}%
\pgfpathclose%
\pgfusepath{fill}%
\end{pgfscope}%
\begin{pgfscope}%
\pgfpathrectangle{\pgfqpoint{3.536584in}{0.147348in}}{\pgfqpoint{2.735294in}{2.735294in}}%
\pgfusepath{clip}%
\pgfsetbuttcap%
\pgfsetroundjoin%
\definecolor{currentfill}{rgb}{0.064759,0.248590,0.376018}%
\pgfsetfillcolor{currentfill}%
\pgfsetlinewidth{0.000000pt}%
\definecolor{currentstroke}{rgb}{0.000000,0.000000,0.000000}%
\pgfsetstrokecolor{currentstroke}%
\pgfsetdash{}{0pt}%
\pgfpathmoveto{\pgfqpoint{4.414548in}{1.562087in}}%
\pgfpathlineto{\pgfqpoint{4.677670in}{1.946079in}}%
\pgfpathlineto{\pgfqpoint{4.373646in}{2.103280in}}%
\pgfpathlineto{\pgfqpoint{4.414548in}{1.562087in}}%
\pgfpathclose%
\pgfusepath{fill}%
\end{pgfscope}%
\begin{pgfscope}%
\pgfpathrectangle{\pgfqpoint{3.536584in}{0.147348in}}{\pgfqpoint{2.735294in}{2.735294in}}%
\pgfusepath{clip}%
\pgfsetbuttcap%
\pgfsetroundjoin%
\definecolor{currentfill}{rgb}{0.071694,0.275212,0.416288}%
\pgfsetfillcolor{currentfill}%
\pgfsetlinewidth{0.000000pt}%
\definecolor{currentstroke}{rgb}{0.000000,0.000000,0.000000}%
\pgfsetstrokecolor{currentstroke}%
\pgfsetdash{}{0pt}%
\pgfpathmoveto{\pgfqpoint{4.677670in}{1.946079in}}%
\pgfpathlineto{\pgfqpoint{4.816407in}{2.124855in}}%
\pgfpathlineto{\pgfqpoint{4.579180in}{2.116981in}}%
\pgfpathlineto{\pgfqpoint{4.677670in}{1.946079in}}%
\pgfpathclose%
\pgfusepath{fill}%
\end{pgfscope}%
\begin{pgfscope}%
\pgfpathrectangle{\pgfqpoint{3.536584in}{0.147348in}}{\pgfqpoint{2.735294in}{2.735294in}}%
\pgfusepath{clip}%
\pgfsetbuttcap%
\pgfsetroundjoin%
\definecolor{currentfill}{rgb}{0.071694,0.275212,0.416288}%
\pgfsetfillcolor{currentfill}%
\pgfsetlinewidth{0.000000pt}%
\definecolor{currentstroke}{rgb}{0.000000,0.000000,0.000000}%
\pgfsetstrokecolor{currentstroke}%
\pgfsetdash{}{0pt}%
\pgfpathmoveto{\pgfqpoint{5.303210in}{2.116981in}}%
\pgfpathlineto{\pgfqpoint{5.065982in}{2.124855in}}%
\pgfpathlineto{\pgfqpoint{5.204720in}{1.946079in}}%
\pgfpathlineto{\pgfqpoint{5.303210in}{2.116981in}}%
\pgfpathclose%
\pgfusepath{fill}%
\end{pgfscope}%
\begin{pgfscope}%
\pgfpathrectangle{\pgfqpoint{3.536584in}{0.147348in}}{\pgfqpoint{2.735294in}{2.735294in}}%
\pgfusepath{clip}%
\pgfsetbuttcap%
\pgfsetroundjoin%
\definecolor{currentfill}{rgb}{0.071636,0.274990,0.415951}%
\pgfsetfillcolor{currentfill}%
\pgfsetlinewidth{0.000000pt}%
\definecolor{currentstroke}{rgb}{0.000000,0.000000,0.000000}%
\pgfsetstrokecolor{currentstroke}%
\pgfsetdash{}{0pt}%
\pgfpathmoveto{\pgfqpoint{4.941195in}{1.947266in}}%
\pgfpathlineto{\pgfqpoint{5.065982in}{2.124855in}}%
\pgfpathlineto{\pgfqpoint{4.816407in}{2.124855in}}%
\pgfpathlineto{\pgfqpoint{4.941195in}{1.947266in}}%
\pgfpathclose%
\pgfusepath{fill}%
\end{pgfscope}%
\begin{pgfscope}%
\pgfpathrectangle{\pgfqpoint{3.536584in}{0.147348in}}{\pgfqpoint{2.735294in}{2.735294in}}%
\pgfusepath{clip}%
\pgfsetbuttcap%
\pgfsetroundjoin%
\definecolor{currentfill}{rgb}{0.045820,0.175891,0.266053}%
\pgfsetfillcolor{currentfill}%
\pgfsetlinewidth{0.000000pt}%
\definecolor{currentstroke}{rgb}{0.000000,0.000000,0.000000}%
\pgfsetstrokecolor{currentstroke}%
\pgfsetdash{}{0pt}%
\pgfpathmoveto{\pgfqpoint{4.579939in}{1.063020in}}%
\pgfpathlineto{\pgfqpoint{4.661262in}{1.541647in}}%
\pgfpathlineto{\pgfqpoint{4.538549in}{1.360106in}}%
\pgfpathlineto{\pgfqpoint{4.579939in}{1.063020in}}%
\pgfpathclose%
\pgfusepath{fill}%
\end{pgfscope}%
\begin{pgfscope}%
\pgfpathrectangle{\pgfqpoint{3.536584in}{0.147348in}}{\pgfqpoint{2.735294in}{2.735294in}}%
\pgfusepath{clip}%
\pgfsetbuttcap%
\pgfsetroundjoin%
\definecolor{currentfill}{rgb}{0.045820,0.175891,0.266053}%
\pgfsetfillcolor{currentfill}%
\pgfsetlinewidth{0.000000pt}%
\definecolor{currentstroke}{rgb}{0.000000,0.000000,0.000000}%
\pgfsetstrokecolor{currentstroke}%
\pgfsetdash{}{0pt}%
\pgfpathmoveto{\pgfqpoint{5.343841in}{1.360106in}}%
\pgfpathlineto{\pgfqpoint{5.221128in}{1.541647in}}%
\pgfpathlineto{\pgfqpoint{5.302451in}{1.063020in}}%
\pgfpathlineto{\pgfqpoint{5.343841in}{1.360106in}}%
\pgfpathclose%
\pgfusepath{fill}%
\end{pgfscope}%
\begin{pgfscope}%
\pgfpathrectangle{\pgfqpoint{3.536584in}{0.147348in}}{\pgfqpoint{2.735294in}{2.735294in}}%
\pgfusepath{clip}%
\pgfsetbuttcap%
\pgfsetroundjoin%
\definecolor{currentfill}{rgb}{0.046814,0.179706,0.271825}%
\pgfsetfillcolor{currentfill}%
\pgfsetlinewidth{0.000000pt}%
\definecolor{currentstroke}{rgb}{0.000000,0.000000,0.000000}%
\pgfsetstrokecolor{currentstroke}%
\pgfsetdash{}{0pt}%
\pgfpathmoveto{\pgfqpoint{4.941195in}{1.533948in}}%
\pgfpathlineto{\pgfqpoint{4.941195in}{0.933095in}}%
\pgfpathlineto{\pgfqpoint{5.081105in}{1.337788in}}%
\pgfpathlineto{\pgfqpoint{4.941195in}{1.533948in}}%
\pgfpathclose%
\pgfusepath{fill}%
\end{pgfscope}%
\begin{pgfscope}%
\pgfpathrectangle{\pgfqpoint{3.536584in}{0.147348in}}{\pgfqpoint{2.735294in}{2.735294in}}%
\pgfusepath{clip}%
\pgfsetbuttcap%
\pgfsetroundjoin%
\definecolor{currentfill}{rgb}{0.046814,0.179706,0.271825}%
\pgfsetfillcolor{currentfill}%
\pgfsetlinewidth{0.000000pt}%
\definecolor{currentstroke}{rgb}{0.000000,0.000000,0.000000}%
\pgfsetstrokecolor{currentstroke}%
\pgfsetdash{}{0pt}%
\pgfpathmoveto{\pgfqpoint{4.801285in}{1.337788in}}%
\pgfpathlineto{\pgfqpoint{4.941195in}{0.933095in}}%
\pgfpathlineto{\pgfqpoint{4.941195in}{1.533948in}}%
\pgfpathlineto{\pgfqpoint{4.801285in}{1.337788in}}%
\pgfpathclose%
\pgfusepath{fill}%
\end{pgfscope}%
\begin{pgfscope}%
\pgfpathrectangle{\pgfqpoint{3.536584in}{0.147348in}}{\pgfqpoint{2.735294in}{2.735294in}}%
\pgfusepath{clip}%
\pgfsetbuttcap%
\pgfsetroundjoin%
\definecolor{currentfill}{rgb}{0.069261,0.265872,0.402159}%
\pgfsetfillcolor{currentfill}%
\pgfsetlinewidth{0.000000pt}%
\definecolor{currentstroke}{rgb}{0.000000,0.000000,0.000000}%
\pgfsetstrokecolor{currentstroke}%
\pgfsetdash{}{0pt}%
\pgfpathmoveto{\pgfqpoint{5.204720in}{1.946079in}}%
\pgfpathlineto{\pgfqpoint{5.065982in}{2.124855in}}%
\pgfpathlineto{\pgfqpoint{4.941195in}{1.947266in}}%
\pgfpathlineto{\pgfqpoint{5.204720in}{1.946079in}}%
\pgfpathclose%
\pgfusepath{fill}%
\end{pgfscope}%
\begin{pgfscope}%
\pgfpathrectangle{\pgfqpoint{3.536584in}{0.147348in}}{\pgfqpoint{2.735294in}{2.735294in}}%
\pgfusepath{clip}%
\pgfsetbuttcap%
\pgfsetroundjoin%
\definecolor{currentfill}{rgb}{0.069261,0.265872,0.402159}%
\pgfsetfillcolor{currentfill}%
\pgfsetlinewidth{0.000000pt}%
\definecolor{currentstroke}{rgb}{0.000000,0.000000,0.000000}%
\pgfsetstrokecolor{currentstroke}%
\pgfsetdash{}{0pt}%
\pgfpathmoveto{\pgfqpoint{4.941195in}{1.947266in}}%
\pgfpathlineto{\pgfqpoint{4.816407in}{2.124855in}}%
\pgfpathlineto{\pgfqpoint{4.677670in}{1.946079in}}%
\pgfpathlineto{\pgfqpoint{4.941195in}{1.947266in}}%
\pgfpathclose%
\pgfusepath{fill}%
\end{pgfscope}%
\begin{pgfscope}%
\pgfpathrectangle{\pgfqpoint{3.536584in}{0.147348in}}{\pgfqpoint{2.735294in}{2.735294in}}%
\pgfusepath{clip}%
\pgfsetbuttcap%
\pgfsetroundjoin%
\definecolor{currentfill}{rgb}{0.049465,0.189883,0.287218}%
\pgfsetfillcolor{currentfill}%
\pgfsetlinewidth{0.000000pt}%
\definecolor{currentstroke}{rgb}{0.000000,0.000000,0.000000}%
\pgfsetstrokecolor{currentstroke}%
\pgfsetdash{}{0pt}%
\pgfpathmoveto{\pgfqpoint{4.579939in}{1.063020in}}%
\pgfpathlineto{\pgfqpoint{4.801285in}{1.337788in}}%
\pgfpathlineto{\pgfqpoint{4.661262in}{1.541647in}}%
\pgfpathlineto{\pgfqpoint{4.579939in}{1.063020in}}%
\pgfpathclose%
\pgfusepath{fill}%
\end{pgfscope}%
\begin{pgfscope}%
\pgfpathrectangle{\pgfqpoint{3.536584in}{0.147348in}}{\pgfqpoint{2.735294in}{2.735294in}}%
\pgfusepath{clip}%
\pgfsetbuttcap%
\pgfsetroundjoin%
\definecolor{currentfill}{rgb}{0.049465,0.189883,0.287218}%
\pgfsetfillcolor{currentfill}%
\pgfsetlinewidth{0.000000pt}%
\definecolor{currentstroke}{rgb}{0.000000,0.000000,0.000000}%
\pgfsetstrokecolor{currentstroke}%
\pgfsetdash{}{0pt}%
\pgfpathmoveto{\pgfqpoint{5.221128in}{1.541647in}}%
\pgfpathlineto{\pgfqpoint{5.081105in}{1.337788in}}%
\pgfpathlineto{\pgfqpoint{5.302451in}{1.063020in}}%
\pgfpathlineto{\pgfqpoint{5.221128in}{1.541647in}}%
\pgfpathclose%
\pgfusepath{fill}%
\end{pgfscope}%
\begin{pgfscope}%
\pgfpathrectangle{\pgfqpoint{3.536584in}{0.147348in}}{\pgfqpoint{2.735294in}{2.735294in}}%
\pgfusepath{clip}%
\pgfsetbuttcap%
\pgfsetroundjoin%
\definecolor{currentfill}{rgb}{0.061576,0.236373,0.357539}%
\pgfsetfillcolor{currentfill}%
\pgfsetlinewidth{0.000000pt}%
\definecolor{currentstroke}{rgb}{0.000000,0.000000,0.000000}%
\pgfsetstrokecolor{currentstroke}%
\pgfsetdash{}{0pt}%
\pgfpathmoveto{\pgfqpoint{4.661262in}{1.541647in}}%
\pgfpathlineto{\pgfqpoint{4.677670in}{1.946079in}}%
\pgfpathlineto{\pgfqpoint{4.414548in}{1.562087in}}%
\pgfpathlineto{\pgfqpoint{4.661262in}{1.541647in}}%
\pgfpathclose%
\pgfusepath{fill}%
\end{pgfscope}%
\begin{pgfscope}%
\pgfpathrectangle{\pgfqpoint{3.536584in}{0.147348in}}{\pgfqpoint{2.735294in}{2.735294in}}%
\pgfusepath{clip}%
\pgfsetbuttcap%
\pgfsetroundjoin%
\definecolor{currentfill}{rgb}{0.061576,0.236373,0.357539}%
\pgfsetfillcolor{currentfill}%
\pgfsetlinewidth{0.000000pt}%
\definecolor{currentstroke}{rgb}{0.000000,0.000000,0.000000}%
\pgfsetstrokecolor{currentstroke}%
\pgfsetdash{}{0pt}%
\pgfpathmoveto{\pgfqpoint{5.467842in}{1.562087in}}%
\pgfpathlineto{\pgfqpoint{5.204720in}{1.946079in}}%
\pgfpathlineto{\pgfqpoint{5.221128in}{1.541647in}}%
\pgfpathlineto{\pgfqpoint{5.467842in}{1.562087in}}%
\pgfpathclose%
\pgfusepath{fill}%
\end{pgfscope}%
\begin{pgfscope}%
\pgfpathrectangle{\pgfqpoint{3.536584in}{0.147348in}}{\pgfqpoint{2.735294in}{2.735294in}}%
\pgfusepath{clip}%
\pgfsetbuttcap%
\pgfsetroundjoin%
\definecolor{currentfill}{rgb}{0.053541,0.205528,0.310883}%
\pgfsetfillcolor{currentfill}%
\pgfsetlinewidth{0.000000pt}%
\definecolor{currentstroke}{rgb}{0.000000,0.000000,0.000000}%
\pgfsetstrokecolor{currentstroke}%
\pgfsetdash{}{0pt}%
\pgfpathmoveto{\pgfqpoint{4.414548in}{1.562087in}}%
\pgfpathlineto{\pgfqpoint{4.538549in}{1.360106in}}%
\pgfpathlineto{\pgfqpoint{4.661262in}{1.541647in}}%
\pgfpathlineto{\pgfqpoint{4.414548in}{1.562087in}}%
\pgfpathclose%
\pgfusepath{fill}%
\end{pgfscope}%
\begin{pgfscope}%
\pgfpathrectangle{\pgfqpoint{3.536584in}{0.147348in}}{\pgfqpoint{2.735294in}{2.735294in}}%
\pgfusepath{clip}%
\pgfsetbuttcap%
\pgfsetroundjoin%
\definecolor{currentfill}{rgb}{0.053541,0.205528,0.310883}%
\pgfsetfillcolor{currentfill}%
\pgfsetlinewidth{0.000000pt}%
\definecolor{currentstroke}{rgb}{0.000000,0.000000,0.000000}%
\pgfsetstrokecolor{currentstroke}%
\pgfsetdash{}{0pt}%
\pgfpathmoveto{\pgfqpoint{5.221128in}{1.541647in}}%
\pgfpathlineto{\pgfqpoint{5.343841in}{1.360106in}}%
\pgfpathlineto{\pgfqpoint{5.467842in}{1.562087in}}%
\pgfpathlineto{\pgfqpoint{5.221128in}{1.541647in}}%
\pgfpathclose%
\pgfusepath{fill}%
\end{pgfscope}%
\begin{pgfscope}%
\pgfpathrectangle{\pgfqpoint{3.536584in}{0.147348in}}{\pgfqpoint{2.735294in}{2.735294in}}%
\pgfusepath{clip}%
\pgfsetbuttcap%
\pgfsetroundjoin%
\definecolor{currentfill}{rgb}{0.060634,0.232757,0.352069}%
\pgfsetfillcolor{currentfill}%
\pgfsetlinewidth{0.000000pt}%
\definecolor{currentstroke}{rgb}{0.000000,0.000000,0.000000}%
\pgfsetstrokecolor{currentstroke}%
\pgfsetdash{}{0pt}%
\pgfpathmoveto{\pgfqpoint{4.677670in}{1.946079in}}%
\pgfpathlineto{\pgfqpoint{4.661262in}{1.541647in}}%
\pgfpathlineto{\pgfqpoint{4.941195in}{1.947266in}}%
\pgfpathlineto{\pgfqpoint{4.677670in}{1.946079in}}%
\pgfpathclose%
\pgfusepath{fill}%
\end{pgfscope}%
\begin{pgfscope}%
\pgfpathrectangle{\pgfqpoint{3.536584in}{0.147348in}}{\pgfqpoint{2.735294in}{2.735294in}}%
\pgfusepath{clip}%
\pgfsetbuttcap%
\pgfsetroundjoin%
\definecolor{currentfill}{rgb}{0.060634,0.232757,0.352069}%
\pgfsetfillcolor{currentfill}%
\pgfsetlinewidth{0.000000pt}%
\definecolor{currentstroke}{rgb}{0.000000,0.000000,0.000000}%
\pgfsetstrokecolor{currentstroke}%
\pgfsetdash{}{0pt}%
\pgfpathmoveto{\pgfqpoint{4.941195in}{1.947266in}}%
\pgfpathlineto{\pgfqpoint{5.221128in}{1.541647in}}%
\pgfpathlineto{\pgfqpoint{5.204720in}{1.946079in}}%
\pgfpathlineto{\pgfqpoint{4.941195in}{1.947266in}}%
\pgfpathclose%
\pgfusepath{fill}%
\end{pgfscope}%
\begin{pgfscope}%
\pgfpathrectangle{\pgfqpoint{3.536584in}{0.147348in}}{\pgfqpoint{2.735294in}{2.735294in}}%
\pgfusepath{clip}%
\pgfsetbuttcap%
\pgfsetroundjoin%
\definecolor{currentfill}{rgb}{0.060773,0.233289,0.352874}%
\pgfsetfillcolor{currentfill}%
\pgfsetlinewidth{0.000000pt}%
\definecolor{currentstroke}{rgb}{0.000000,0.000000,0.000000}%
\pgfsetstrokecolor{currentstroke}%
\pgfsetdash{}{0pt}%
\pgfpathmoveto{\pgfqpoint{4.941195in}{1.533948in}}%
\pgfpathlineto{\pgfqpoint{4.941195in}{1.947266in}}%
\pgfpathlineto{\pgfqpoint{4.661262in}{1.541647in}}%
\pgfpathlineto{\pgfqpoint{4.941195in}{1.533948in}}%
\pgfpathclose%
\pgfusepath{fill}%
\end{pgfscope}%
\begin{pgfscope}%
\pgfpathrectangle{\pgfqpoint{3.536584in}{0.147348in}}{\pgfqpoint{2.735294in}{2.735294in}}%
\pgfusepath{clip}%
\pgfsetbuttcap%
\pgfsetroundjoin%
\definecolor{currentfill}{rgb}{0.060773,0.233289,0.352874}%
\pgfsetfillcolor{currentfill}%
\pgfsetlinewidth{0.000000pt}%
\definecolor{currentstroke}{rgb}{0.000000,0.000000,0.000000}%
\pgfsetstrokecolor{currentstroke}%
\pgfsetdash{}{0pt}%
\pgfpathmoveto{\pgfqpoint{5.221128in}{1.541647in}}%
\pgfpathlineto{\pgfqpoint{4.941195in}{1.947266in}}%
\pgfpathlineto{\pgfqpoint{4.941195in}{1.533948in}}%
\pgfpathlineto{\pgfqpoint{5.221128in}{1.541647in}}%
\pgfpathclose%
\pgfusepath{fill}%
\end{pgfscope}%
\begin{pgfscope}%
\pgfpathrectangle{\pgfqpoint{3.536584in}{0.147348in}}{\pgfqpoint{2.735294in}{2.735294in}}%
\pgfusepath{clip}%
\pgfsetbuttcap%
\pgfsetroundjoin%
\definecolor{currentfill}{rgb}{0.052607,0.201942,0.305459}%
\pgfsetfillcolor{currentfill}%
\pgfsetlinewidth{0.000000pt}%
\definecolor{currentstroke}{rgb}{0.000000,0.000000,0.000000}%
\pgfsetstrokecolor{currentstroke}%
\pgfsetdash{}{0pt}%
\pgfpathmoveto{\pgfqpoint{4.661262in}{1.541647in}}%
\pgfpathlineto{\pgfqpoint{4.801285in}{1.337788in}}%
\pgfpathlineto{\pgfqpoint{4.941195in}{1.533948in}}%
\pgfpathlineto{\pgfqpoint{4.661262in}{1.541647in}}%
\pgfpathclose%
\pgfusepath{fill}%
\end{pgfscope}%
\begin{pgfscope}%
\pgfpathrectangle{\pgfqpoint{3.536584in}{0.147348in}}{\pgfqpoint{2.735294in}{2.735294in}}%
\pgfusepath{clip}%
\pgfsetbuttcap%
\pgfsetroundjoin%
\definecolor{currentfill}{rgb}{0.052607,0.201942,0.305459}%
\pgfsetfillcolor{currentfill}%
\pgfsetlinewidth{0.000000pt}%
\definecolor{currentstroke}{rgb}{0.000000,0.000000,0.000000}%
\pgfsetstrokecolor{currentstroke}%
\pgfsetdash{}{0pt}%
\pgfpathmoveto{\pgfqpoint{4.941195in}{1.533948in}}%
\pgfpathlineto{\pgfqpoint{5.081105in}{1.337788in}}%
\pgfpathlineto{\pgfqpoint{5.221128in}{1.541647in}}%
\pgfpathlineto{\pgfqpoint{4.941195in}{1.533948in}}%
\pgfpathclose%
\pgfusepath{fill}%
\end{pgfscope}%
\begin{pgfscope}%
\pgfpathrectangle{\pgfqpoint{3.536584in}{0.147348in}}{\pgfqpoint{2.735294in}{2.735294in}}%
\pgfusepath{clip}%
\pgfsetbuttcap%
\pgfsetroundjoin%
\definecolor{currentfill}{rgb}{0.839216,0.152941,0.156863}%
\pgfsetfillcolor{currentfill}%
\pgfsetfillopacity{0.300000}%
\pgfsetlinewidth{1.003750pt}%
\definecolor{currentstroke}{rgb}{0.839216,0.152941,0.156863}%
\pgfsetstrokecolor{currentstroke}%
\pgfsetstrokeopacity{0.300000}%
\pgfsetdash{}{0pt}%
\pgfpathmoveto{\pgfqpoint{4.347722in}{0.944929in}}%
\pgfpathcurveto{\pgfqpoint{4.357810in}{0.944929in}}{\pgfqpoint{4.367485in}{0.948937in}}{\pgfqpoint{4.374618in}{0.956070in}}%
\pgfpathcurveto{\pgfqpoint{4.381751in}{0.963203in}}{\pgfqpoint{4.385759in}{0.972878in}}{\pgfqpoint{4.385759in}{0.982966in}}%
\pgfpathcurveto{\pgfqpoint{4.385759in}{0.993053in}}{\pgfqpoint{4.381751in}{1.002729in}}{\pgfqpoint{4.374618in}{1.009861in}}%
\pgfpathcurveto{\pgfqpoint{4.367485in}{1.016994in}}{\pgfqpoint{4.357810in}{1.021002in}}{\pgfqpoint{4.347722in}{1.021002in}}%
\pgfpathcurveto{\pgfqpoint{4.337635in}{1.021002in}}{\pgfqpoint{4.327959in}{1.016994in}}{\pgfqpoint{4.320827in}{1.009861in}}%
\pgfpathcurveto{\pgfqpoint{4.313694in}{1.002729in}}{\pgfqpoint{4.309686in}{0.993053in}}{\pgfqpoint{4.309686in}{0.982966in}}%
\pgfpathcurveto{\pgfqpoint{4.309686in}{0.972878in}}{\pgfqpoint{4.313694in}{0.963203in}}{\pgfqpoint{4.320827in}{0.956070in}}%
\pgfpathcurveto{\pgfqpoint{4.327959in}{0.948937in}}{\pgfqpoint{4.337635in}{0.944929in}}{\pgfqpoint{4.347722in}{0.944929in}}%
\pgfpathlineto{\pgfqpoint{4.347722in}{0.944929in}}%
\pgfpathclose%
\pgfusepath{stroke,fill}%
\end{pgfscope}%
\begin{pgfscope}%
\pgfpathrectangle{\pgfqpoint{3.536584in}{0.147348in}}{\pgfqpoint{2.735294in}{2.735294in}}%
\pgfusepath{clip}%
\pgfsetbuttcap%
\pgfsetroundjoin%
\definecolor{currentfill}{rgb}{0.839216,0.152941,0.156863}%
\pgfsetfillcolor{currentfill}%
\pgfsetfillopacity{0.383610}%
\pgfsetlinewidth{1.003750pt}%
\definecolor{currentstroke}{rgb}{0.839216,0.152941,0.156863}%
\pgfsetstrokecolor{currentstroke}%
\pgfsetstrokeopacity{0.383610}%
\pgfsetdash{}{0pt}%
\pgfpathmoveto{\pgfqpoint{4.327859in}{1.009596in}}%
\pgfpathcurveto{\pgfqpoint{4.337946in}{1.009596in}}{\pgfqpoint{4.347622in}{1.013604in}}{\pgfqpoint{4.354755in}{1.020737in}}%
\pgfpathcurveto{\pgfqpoint{4.361888in}{1.027870in}}{\pgfqpoint{4.365895in}{1.037545in}}{\pgfqpoint{4.365895in}{1.047633in}}%
\pgfpathcurveto{\pgfqpoint{4.365895in}{1.057720in}}{\pgfqpoint{4.361888in}{1.067396in}}{\pgfqpoint{4.354755in}{1.074528in}}%
\pgfpathcurveto{\pgfqpoint{4.347622in}{1.081661in}}{\pgfqpoint{4.337946in}{1.085669in}}{\pgfqpoint{4.327859in}{1.085669in}}%
\pgfpathcurveto{\pgfqpoint{4.317772in}{1.085669in}}{\pgfqpoint{4.308096in}{1.081661in}}{\pgfqpoint{4.300963in}{1.074528in}}%
\pgfpathcurveto{\pgfqpoint{4.293831in}{1.067396in}}{\pgfqpoint{4.289823in}{1.057720in}}{\pgfqpoint{4.289823in}{1.047633in}}%
\pgfpathcurveto{\pgfqpoint{4.289823in}{1.037545in}}{\pgfqpoint{4.293831in}{1.027870in}}{\pgfqpoint{4.300963in}{1.020737in}}%
\pgfpathcurveto{\pgfqpoint{4.308096in}{1.013604in}}{\pgfqpoint{4.317772in}{1.009596in}}{\pgfqpoint{4.327859in}{1.009596in}}%
\pgfpathlineto{\pgfqpoint{4.327859in}{1.009596in}}%
\pgfpathclose%
\pgfusepath{stroke,fill}%
\end{pgfscope}%
\begin{pgfscope}%
\pgfpathrectangle{\pgfqpoint{3.536584in}{0.147348in}}{\pgfqpoint{2.735294in}{2.735294in}}%
\pgfusepath{clip}%
\pgfsetbuttcap%
\pgfsetroundjoin%
\definecolor{currentfill}{rgb}{0.839216,0.152941,0.156863}%
\pgfsetfillcolor{currentfill}%
\pgfsetfillopacity{0.457533}%
\pgfsetlinewidth{1.003750pt}%
\definecolor{currentstroke}{rgb}{0.839216,0.152941,0.156863}%
\pgfsetstrokecolor{currentstroke}%
\pgfsetstrokeopacity{0.457533}%
\pgfsetdash{}{0pt}%
\pgfpathmoveto{\pgfqpoint{4.483875in}{0.962534in}}%
\pgfpathcurveto{\pgfqpoint{4.493962in}{0.962534in}}{\pgfqpoint{4.503637in}{0.966542in}}{\pgfqpoint{4.510770in}{0.973675in}}%
\pgfpathcurveto{\pgfqpoint{4.517903in}{0.980808in}}{\pgfqpoint{4.521911in}{0.990483in}}{\pgfqpoint{4.521911in}{1.000570in}}%
\pgfpathcurveto{\pgfqpoint{4.521911in}{1.010658in}}{\pgfqpoint{4.517903in}{1.020333in}}{\pgfqpoint{4.510770in}{1.027466in}}%
\pgfpathcurveto{\pgfqpoint{4.503637in}{1.034599in}}{\pgfqpoint{4.493962in}{1.038607in}}{\pgfqpoint{4.483875in}{1.038607in}}%
\pgfpathcurveto{\pgfqpoint{4.473787in}{1.038607in}}{\pgfqpoint{4.464112in}{1.034599in}}{\pgfqpoint{4.456979in}{1.027466in}}%
\pgfpathcurveto{\pgfqpoint{4.449846in}{1.020333in}}{\pgfqpoint{4.445838in}{1.010658in}}{\pgfqpoint{4.445838in}{1.000570in}}%
\pgfpathcurveto{\pgfqpoint{4.445838in}{0.990483in}}{\pgfqpoint{4.449846in}{0.980808in}}{\pgfqpoint{4.456979in}{0.973675in}}%
\pgfpathcurveto{\pgfqpoint{4.464112in}{0.966542in}}{\pgfqpoint{4.473787in}{0.962534in}}{\pgfqpoint{4.483875in}{0.962534in}}%
\pgfpathlineto{\pgfqpoint{4.483875in}{0.962534in}}%
\pgfpathclose%
\pgfusepath{stroke,fill}%
\end{pgfscope}%
\begin{pgfscope}%
\pgfpathrectangle{\pgfqpoint{3.536584in}{0.147348in}}{\pgfqpoint{2.735294in}{2.735294in}}%
\pgfusepath{clip}%
\pgfsetbuttcap%
\pgfsetroundjoin%
\definecolor{currentfill}{rgb}{0.839216,0.152941,0.156863}%
\pgfsetfillcolor{currentfill}%
\pgfsetfillopacity{0.492303}%
\pgfsetlinewidth{1.003750pt}%
\definecolor{currentstroke}{rgb}{0.839216,0.152941,0.156863}%
\pgfsetstrokecolor{currentstroke}%
\pgfsetstrokeopacity{0.492303}%
\pgfsetdash{}{0pt}%
\pgfpathmoveto{\pgfqpoint{4.160550in}{1.511653in}}%
\pgfpathcurveto{\pgfqpoint{4.170638in}{1.511653in}}{\pgfqpoint{4.180313in}{1.515661in}}{\pgfqpoint{4.187446in}{1.522794in}}%
\pgfpathcurveto{\pgfqpoint{4.194579in}{1.529927in}}{\pgfqpoint{4.198587in}{1.539602in}}{\pgfqpoint{4.198587in}{1.549689in}}%
\pgfpathcurveto{\pgfqpoint{4.198587in}{1.559777in}}{\pgfqpoint{4.194579in}{1.569452in}}{\pgfqpoint{4.187446in}{1.576585in}}%
\pgfpathcurveto{\pgfqpoint{4.180313in}{1.583718in}}{\pgfqpoint{4.170638in}{1.587726in}}{\pgfqpoint{4.160550in}{1.587726in}}%
\pgfpathcurveto{\pgfqpoint{4.150463in}{1.587726in}}{\pgfqpoint{4.140788in}{1.583718in}}{\pgfqpoint{4.133655in}{1.576585in}}%
\pgfpathcurveto{\pgfqpoint{4.126522in}{1.569452in}}{\pgfqpoint{4.122514in}{1.559777in}}{\pgfqpoint{4.122514in}{1.549689in}}%
\pgfpathcurveto{\pgfqpoint{4.122514in}{1.539602in}}{\pgfqpoint{4.126522in}{1.529927in}}{\pgfqpoint{4.133655in}{1.522794in}}%
\pgfpathcurveto{\pgfqpoint{4.140788in}{1.515661in}}{\pgfqpoint{4.150463in}{1.511653in}}{\pgfqpoint{4.160550in}{1.511653in}}%
\pgfpathlineto{\pgfqpoint{4.160550in}{1.511653in}}%
\pgfpathclose%
\pgfusepath{stroke,fill}%
\end{pgfscope}%
\begin{pgfscope}%
\pgfpathrectangle{\pgfqpoint{3.536584in}{0.147348in}}{\pgfqpoint{2.735294in}{2.735294in}}%
\pgfusepath{clip}%
\pgfsetbuttcap%
\pgfsetroundjoin%
\definecolor{currentfill}{rgb}{0.839216,0.152941,0.156863}%
\pgfsetfillcolor{currentfill}%
\pgfsetfillopacity{0.498590}%
\pgfsetlinewidth{1.003750pt}%
\definecolor{currentstroke}{rgb}{0.839216,0.152941,0.156863}%
\pgfsetstrokecolor{currentstroke}%
\pgfsetstrokeopacity{0.498590}%
\pgfsetdash{}{0pt}%
\pgfpathmoveto{\pgfqpoint{5.807826in}{1.621636in}}%
\pgfpathcurveto{\pgfqpoint{5.817913in}{1.621636in}}{\pgfqpoint{5.827588in}{1.625643in}}{\pgfqpoint{5.834721in}{1.632776in}}%
\pgfpathcurveto{\pgfqpoint{5.841854in}{1.639909in}}{\pgfqpoint{5.845862in}{1.649585in}}{\pgfqpoint{5.845862in}{1.659672in}}%
\pgfpathcurveto{\pgfqpoint{5.845862in}{1.669759in}}{\pgfqpoint{5.841854in}{1.679435in}}{\pgfqpoint{5.834721in}{1.686568in}}%
\pgfpathcurveto{\pgfqpoint{5.827588in}{1.693701in}}{\pgfqpoint{5.817913in}{1.697708in}}{\pgfqpoint{5.807826in}{1.697708in}}%
\pgfpathcurveto{\pgfqpoint{5.797738in}{1.697708in}}{\pgfqpoint{5.788063in}{1.693701in}}{\pgfqpoint{5.780930in}{1.686568in}}%
\pgfpathcurveto{\pgfqpoint{5.773797in}{1.679435in}}{\pgfqpoint{5.769789in}{1.669759in}}{\pgfqpoint{5.769789in}{1.659672in}}%
\pgfpathcurveto{\pgfqpoint{5.769789in}{1.649585in}}{\pgfqpoint{5.773797in}{1.639909in}}{\pgfqpoint{5.780930in}{1.632776in}}%
\pgfpathcurveto{\pgfqpoint{5.788063in}{1.625643in}}{\pgfqpoint{5.797738in}{1.621636in}}{\pgfqpoint{5.807826in}{1.621636in}}%
\pgfpathlineto{\pgfqpoint{5.807826in}{1.621636in}}%
\pgfpathclose%
\pgfusepath{stroke,fill}%
\end{pgfscope}%
\begin{pgfscope}%
\pgfpathrectangle{\pgfqpoint{3.536584in}{0.147348in}}{\pgfqpoint{2.735294in}{2.735294in}}%
\pgfusepath{clip}%
\pgfsetbuttcap%
\pgfsetroundjoin%
\definecolor{currentfill}{rgb}{0.839216,0.152941,0.156863}%
\pgfsetfillcolor{currentfill}%
\pgfsetfillopacity{0.612876}%
\pgfsetlinewidth{1.003750pt}%
\definecolor{currentstroke}{rgb}{0.839216,0.152941,0.156863}%
\pgfsetstrokecolor{currentstroke}%
\pgfsetstrokeopacity{0.612876}%
\pgfsetdash{}{0pt}%
\pgfpathmoveto{\pgfqpoint{4.496123in}{0.979547in}}%
\pgfpathcurveto{\pgfqpoint{4.506210in}{0.979547in}}{\pgfqpoint{4.515886in}{0.983554in}}{\pgfqpoint{4.523019in}{0.990687in}}%
\pgfpathcurveto{\pgfqpoint{4.530151in}{0.997820in}}{\pgfqpoint{4.534159in}{1.007495in}}{\pgfqpoint{4.534159in}{1.017583in}}%
\pgfpathcurveto{\pgfqpoint{4.534159in}{1.027670in}}{\pgfqpoint{4.530151in}{1.037346in}}{\pgfqpoint{4.523019in}{1.044479in}}%
\pgfpathcurveto{\pgfqpoint{4.515886in}{1.051611in}}{\pgfqpoint{4.506210in}{1.055619in}}{\pgfqpoint{4.496123in}{1.055619in}}%
\pgfpathcurveto{\pgfqpoint{4.486036in}{1.055619in}}{\pgfqpoint{4.476360in}{1.051611in}}{\pgfqpoint{4.469227in}{1.044479in}}%
\pgfpathcurveto{\pgfqpoint{4.462094in}{1.037346in}}{\pgfqpoint{4.458087in}{1.027670in}}{\pgfqpoint{4.458087in}{1.017583in}}%
\pgfpathcurveto{\pgfqpoint{4.458087in}{1.007495in}}{\pgfqpoint{4.462094in}{0.997820in}}{\pgfqpoint{4.469227in}{0.990687in}}%
\pgfpathcurveto{\pgfqpoint{4.476360in}{0.983554in}}{\pgfqpoint{4.486036in}{0.979547in}}{\pgfqpoint{4.496123in}{0.979547in}}%
\pgfpathlineto{\pgfqpoint{4.496123in}{0.979547in}}%
\pgfpathclose%
\pgfusepath{stroke,fill}%
\end{pgfscope}%
\begin{pgfscope}%
\pgfpathrectangle{\pgfqpoint{3.536584in}{0.147348in}}{\pgfqpoint{2.735294in}{2.735294in}}%
\pgfusepath{clip}%
\pgfsetbuttcap%
\pgfsetroundjoin%
\definecolor{currentfill}{rgb}{0.839216,0.152941,0.156863}%
\pgfsetfillcolor{currentfill}%
\pgfsetfillopacity{0.625674}%
\pgfsetlinewidth{1.003750pt}%
\definecolor{currentstroke}{rgb}{0.839216,0.152941,0.156863}%
\pgfsetstrokecolor{currentstroke}%
\pgfsetstrokeopacity{0.625674}%
\pgfsetdash{}{0pt}%
\pgfpathmoveto{\pgfqpoint{5.375023in}{2.164239in}}%
\pgfpathcurveto{\pgfqpoint{5.385111in}{2.164239in}}{\pgfqpoint{5.394786in}{2.168247in}}{\pgfqpoint{5.401919in}{2.175380in}}%
\pgfpathcurveto{\pgfqpoint{5.409052in}{2.182513in}}{\pgfqpoint{5.413059in}{2.192188in}}{\pgfqpoint{5.413059in}{2.202275in}}%
\pgfpathcurveto{\pgfqpoint{5.413059in}{2.212363in}}{\pgfqpoint{5.409052in}{2.222038in}}{\pgfqpoint{5.401919in}{2.229171in}}%
\pgfpathcurveto{\pgfqpoint{5.394786in}{2.236304in}}{\pgfqpoint{5.385111in}{2.240312in}}{\pgfqpoint{5.375023in}{2.240312in}}%
\pgfpathcurveto{\pgfqpoint{5.364936in}{2.240312in}}{\pgfqpoint{5.355260in}{2.236304in}}{\pgfqpoint{5.348127in}{2.229171in}}%
\pgfpathcurveto{\pgfqpoint{5.340995in}{2.222038in}}{\pgfqpoint{5.336987in}{2.212363in}}{\pgfqpoint{5.336987in}{2.202275in}}%
\pgfpathcurveto{\pgfqpoint{5.336987in}{2.192188in}}{\pgfqpoint{5.340995in}{2.182513in}}{\pgfqpoint{5.348127in}{2.175380in}}%
\pgfpathcurveto{\pgfqpoint{5.355260in}{2.168247in}}{\pgfqpoint{5.364936in}{2.164239in}}{\pgfqpoint{5.375023in}{2.164239in}}%
\pgfpathlineto{\pgfqpoint{5.375023in}{2.164239in}}%
\pgfpathclose%
\pgfusepath{stroke,fill}%
\end{pgfscope}%
\begin{pgfscope}%
\pgfpathrectangle{\pgfqpoint{3.536584in}{0.147348in}}{\pgfqpoint{2.735294in}{2.735294in}}%
\pgfusepath{clip}%
\pgfsetbuttcap%
\pgfsetroundjoin%
\definecolor{currentfill}{rgb}{0.839216,0.152941,0.156863}%
\pgfsetfillcolor{currentfill}%
\pgfsetfillopacity{0.631635}%
\pgfsetlinewidth{1.003750pt}%
\definecolor{currentstroke}{rgb}{0.839216,0.152941,0.156863}%
\pgfsetstrokecolor{currentstroke}%
\pgfsetstrokeopacity{0.631635}%
\pgfsetdash{}{0pt}%
\pgfpathmoveto{\pgfqpoint{5.366125in}{2.104757in}}%
\pgfpathcurveto{\pgfqpoint{5.376212in}{2.104757in}}{\pgfqpoint{5.385888in}{2.108765in}}{\pgfqpoint{5.393021in}{2.115898in}}%
\pgfpathcurveto{\pgfqpoint{5.400154in}{2.123030in}}{\pgfqpoint{5.404161in}{2.132706in}}{\pgfqpoint{5.404161in}{2.142793in}}%
\pgfpathcurveto{\pgfqpoint{5.404161in}{2.152881in}}{\pgfqpoint{5.400154in}{2.162556in}}{\pgfqpoint{5.393021in}{2.169689in}}%
\pgfpathcurveto{\pgfqpoint{5.385888in}{2.176822in}}{\pgfqpoint{5.376212in}{2.180830in}}{\pgfqpoint{5.366125in}{2.180830in}}%
\pgfpathcurveto{\pgfqpoint{5.356038in}{2.180830in}}{\pgfqpoint{5.346362in}{2.176822in}}{\pgfqpoint{5.339229in}{2.169689in}}%
\pgfpathcurveto{\pgfqpoint{5.332097in}{2.162556in}}{\pgfqpoint{5.328089in}{2.152881in}}{\pgfqpoint{5.328089in}{2.142793in}}%
\pgfpathcurveto{\pgfqpoint{5.328089in}{2.132706in}}{\pgfqpoint{5.332097in}{2.123030in}}{\pgfqpoint{5.339229in}{2.115898in}}%
\pgfpathcurveto{\pgfqpoint{5.346362in}{2.108765in}}{\pgfqpoint{5.356038in}{2.104757in}}{\pgfqpoint{5.366125in}{2.104757in}}%
\pgfpathlineto{\pgfqpoint{5.366125in}{2.104757in}}%
\pgfpathclose%
\pgfusepath{stroke,fill}%
\end{pgfscope}%
\begin{pgfscope}%
\pgfpathrectangle{\pgfqpoint{3.536584in}{0.147348in}}{\pgfqpoint{2.735294in}{2.735294in}}%
\pgfusepath{clip}%
\pgfsetbuttcap%
\pgfsetroundjoin%
\definecolor{currentfill}{rgb}{0.839216,0.152941,0.156863}%
\pgfsetfillcolor{currentfill}%
\pgfsetfillopacity{0.634032}%
\pgfsetlinewidth{1.003750pt}%
\definecolor{currentstroke}{rgb}{0.839216,0.152941,0.156863}%
\pgfsetstrokecolor{currentstroke}%
\pgfsetstrokeopacity{0.634032}%
\pgfsetdash{}{0pt}%
\pgfpathmoveto{\pgfqpoint{5.637120in}{1.533362in}}%
\pgfpathcurveto{\pgfqpoint{5.647207in}{1.533362in}}{\pgfqpoint{5.656883in}{1.537370in}}{\pgfqpoint{5.664016in}{1.544503in}}%
\pgfpathcurveto{\pgfqpoint{5.671149in}{1.551635in}}{\pgfqpoint{5.675156in}{1.561311in}}{\pgfqpoint{5.675156in}{1.571398in}}%
\pgfpathcurveto{\pgfqpoint{5.675156in}{1.581486in}}{\pgfqpoint{5.671149in}{1.591161in}}{\pgfqpoint{5.664016in}{1.598294in}}%
\pgfpathcurveto{\pgfqpoint{5.656883in}{1.605427in}}{\pgfqpoint{5.647207in}{1.609435in}}{\pgfqpoint{5.637120in}{1.609435in}}%
\pgfpathcurveto{\pgfqpoint{5.627033in}{1.609435in}}{\pgfqpoint{5.617357in}{1.605427in}}{\pgfqpoint{5.610224in}{1.598294in}}%
\pgfpathcurveto{\pgfqpoint{5.603091in}{1.591161in}}{\pgfqpoint{5.599084in}{1.581486in}}{\pgfqpoint{5.599084in}{1.571398in}}%
\pgfpathcurveto{\pgfqpoint{5.599084in}{1.561311in}}{\pgfqpoint{5.603091in}{1.551635in}}{\pgfqpoint{5.610224in}{1.544503in}}%
\pgfpathcurveto{\pgfqpoint{5.617357in}{1.537370in}}{\pgfqpoint{5.627033in}{1.533362in}}{\pgfqpoint{5.637120in}{1.533362in}}%
\pgfpathlineto{\pgfqpoint{5.637120in}{1.533362in}}%
\pgfpathclose%
\pgfusepath{stroke,fill}%
\end{pgfscope}%
\begin{pgfscope}%
\pgfpathrectangle{\pgfqpoint{3.536584in}{0.147348in}}{\pgfqpoint{2.735294in}{2.735294in}}%
\pgfusepath{clip}%
\pgfsetbuttcap%
\pgfsetroundjoin%
\definecolor{currentfill}{rgb}{0.839216,0.152941,0.156863}%
\pgfsetfillcolor{currentfill}%
\pgfsetfillopacity{0.652064}%
\pgfsetlinewidth{1.003750pt}%
\definecolor{currentstroke}{rgb}{0.839216,0.152941,0.156863}%
\pgfsetstrokecolor{currentstroke}%
\pgfsetstrokeopacity{0.652064}%
\pgfsetdash{}{0pt}%
\pgfpathmoveto{\pgfqpoint{4.750408in}{0.952320in}}%
\pgfpathcurveto{\pgfqpoint{4.760495in}{0.952320in}}{\pgfqpoint{4.770171in}{0.956328in}}{\pgfqpoint{4.777304in}{0.963461in}}%
\pgfpathcurveto{\pgfqpoint{4.784437in}{0.970593in}}{\pgfqpoint{4.788444in}{0.980269in}}{\pgfqpoint{4.788444in}{0.990356in}}%
\pgfpathcurveto{\pgfqpoint{4.788444in}{1.000444in}}{\pgfqpoint{4.784437in}{1.010119in}}{\pgfqpoint{4.777304in}{1.017252in}}%
\pgfpathcurveto{\pgfqpoint{4.770171in}{1.024385in}}{\pgfqpoint{4.760495in}{1.028393in}}{\pgfqpoint{4.750408in}{1.028393in}}%
\pgfpathcurveto{\pgfqpoint{4.740321in}{1.028393in}}{\pgfqpoint{4.730645in}{1.024385in}}{\pgfqpoint{4.723512in}{1.017252in}}%
\pgfpathcurveto{\pgfqpoint{4.716379in}{1.010119in}}{\pgfqpoint{4.712372in}{1.000444in}}{\pgfqpoint{4.712372in}{0.990356in}}%
\pgfpathcurveto{\pgfqpoint{4.712372in}{0.980269in}}{\pgfqpoint{4.716379in}{0.970593in}}{\pgfqpoint{4.723512in}{0.963461in}}%
\pgfpathcurveto{\pgfqpoint{4.730645in}{0.956328in}}{\pgfqpoint{4.740321in}{0.952320in}}{\pgfqpoint{4.750408in}{0.952320in}}%
\pgfpathlineto{\pgfqpoint{4.750408in}{0.952320in}}%
\pgfpathclose%
\pgfusepath{stroke,fill}%
\end{pgfscope}%
\begin{pgfscope}%
\pgfpathrectangle{\pgfqpoint{3.536584in}{0.147348in}}{\pgfqpoint{2.735294in}{2.735294in}}%
\pgfusepath{clip}%
\pgfsetbuttcap%
\pgfsetroundjoin%
\definecolor{currentfill}{rgb}{0.839216,0.152941,0.156863}%
\pgfsetfillcolor{currentfill}%
\pgfsetfillopacity{0.738747}%
\pgfsetlinewidth{1.003750pt}%
\definecolor{currentstroke}{rgb}{0.839216,0.152941,0.156863}%
\pgfsetstrokecolor{currentstroke}%
\pgfsetstrokeopacity{0.738747}%
\pgfsetdash{}{0pt}%
\pgfpathmoveto{\pgfqpoint{4.140523in}{1.507716in}}%
\pgfpathcurveto{\pgfqpoint{4.150611in}{1.507716in}}{\pgfqpoint{4.160286in}{1.511723in}}{\pgfqpoint{4.167419in}{1.518856in}}%
\pgfpathcurveto{\pgfqpoint{4.174552in}{1.525989in}}{\pgfqpoint{4.178560in}{1.535665in}}{\pgfqpoint{4.178560in}{1.545752in}}%
\pgfpathcurveto{\pgfqpoint{4.178560in}{1.555839in}}{\pgfqpoint{4.174552in}{1.565515in}}{\pgfqpoint{4.167419in}{1.572648in}}%
\pgfpathcurveto{\pgfqpoint{4.160286in}{1.579780in}}{\pgfqpoint{4.150611in}{1.583788in}}{\pgfqpoint{4.140523in}{1.583788in}}%
\pgfpathcurveto{\pgfqpoint{4.130436in}{1.583788in}}{\pgfqpoint{4.120761in}{1.579780in}}{\pgfqpoint{4.113628in}{1.572648in}}%
\pgfpathcurveto{\pgfqpoint{4.106495in}{1.565515in}}{\pgfqpoint{4.102487in}{1.555839in}}{\pgfqpoint{4.102487in}{1.545752in}}%
\pgfpathcurveto{\pgfqpoint{4.102487in}{1.535665in}}{\pgfqpoint{4.106495in}{1.525989in}}{\pgfqpoint{4.113628in}{1.518856in}}%
\pgfpathcurveto{\pgfqpoint{4.120761in}{1.511723in}}{\pgfqpoint{4.130436in}{1.507716in}}{\pgfqpoint{4.140523in}{1.507716in}}%
\pgfpathlineto{\pgfqpoint{4.140523in}{1.507716in}}%
\pgfpathclose%
\pgfusepath{stroke,fill}%
\end{pgfscope}%
\begin{pgfscope}%
\pgfpathrectangle{\pgfqpoint{3.536584in}{0.147348in}}{\pgfqpoint{2.735294in}{2.735294in}}%
\pgfusepath{clip}%
\pgfsetbuttcap%
\pgfsetroundjoin%
\definecolor{currentfill}{rgb}{0.839216,0.152941,0.156863}%
\pgfsetfillcolor{currentfill}%
\pgfsetfillopacity{0.791813}%
\pgfsetlinewidth{1.003750pt}%
\definecolor{currentstroke}{rgb}{0.839216,0.152941,0.156863}%
\pgfsetstrokecolor{currentstroke}%
\pgfsetstrokeopacity{0.791813}%
\pgfsetdash{}{0pt}%
\pgfpathmoveto{\pgfqpoint{4.633661in}{2.014183in}}%
\pgfpathcurveto{\pgfqpoint{4.643748in}{2.014183in}}{\pgfqpoint{4.653424in}{2.018191in}}{\pgfqpoint{4.660557in}{2.025323in}}%
\pgfpathcurveto{\pgfqpoint{4.667690in}{2.032456in}}{\pgfqpoint{4.671697in}{2.042132in}}{\pgfqpoint{4.671697in}{2.052219in}}%
\pgfpathcurveto{\pgfqpoint{4.671697in}{2.062307in}}{\pgfqpoint{4.667690in}{2.071982in}}{\pgfqpoint{4.660557in}{2.079115in}}%
\pgfpathcurveto{\pgfqpoint{4.653424in}{2.086248in}}{\pgfqpoint{4.643748in}{2.090255in}}{\pgfqpoint{4.633661in}{2.090255in}}%
\pgfpathcurveto{\pgfqpoint{4.623574in}{2.090255in}}{\pgfqpoint{4.613898in}{2.086248in}}{\pgfqpoint{4.606765in}{2.079115in}}%
\pgfpathcurveto{\pgfqpoint{4.599632in}{2.071982in}}{\pgfqpoint{4.595625in}{2.062307in}}{\pgfqpoint{4.595625in}{2.052219in}}%
\pgfpathcurveto{\pgfqpoint{4.595625in}{2.042132in}}{\pgfqpoint{4.599632in}{2.032456in}}{\pgfqpoint{4.606765in}{2.025323in}}%
\pgfpathcurveto{\pgfqpoint{4.613898in}{2.018191in}}{\pgfqpoint{4.623574in}{2.014183in}}{\pgfqpoint{4.633661in}{2.014183in}}%
\pgfpathlineto{\pgfqpoint{4.633661in}{2.014183in}}%
\pgfpathclose%
\pgfusepath{stroke,fill}%
\end{pgfscope}%
\begin{pgfscope}%
\pgfpathrectangle{\pgfqpoint{3.536584in}{0.147348in}}{\pgfqpoint{2.735294in}{2.735294in}}%
\pgfusepath{clip}%
\pgfsetbuttcap%
\pgfsetroundjoin%
\definecolor{currentfill}{rgb}{0.839216,0.152941,0.156863}%
\pgfsetfillcolor{currentfill}%
\pgfsetfillopacity{0.864233}%
\pgfsetlinewidth{1.003750pt}%
\definecolor{currentstroke}{rgb}{0.839216,0.152941,0.156863}%
\pgfsetstrokecolor{currentstroke}%
\pgfsetstrokeopacity{0.864233}%
\pgfsetdash{}{0pt}%
\pgfpathmoveto{\pgfqpoint{5.299988in}{1.926870in}}%
\pgfpathcurveto{\pgfqpoint{5.310075in}{1.926870in}}{\pgfqpoint{5.319750in}{1.930878in}}{\pgfqpoint{5.326883in}{1.938011in}}%
\pgfpathcurveto{\pgfqpoint{5.334016in}{1.945143in}}{\pgfqpoint{5.338024in}{1.954819in}}{\pgfqpoint{5.338024in}{1.964906in}}%
\pgfpathcurveto{\pgfqpoint{5.338024in}{1.974994in}}{\pgfqpoint{5.334016in}{1.984669in}}{\pgfqpoint{5.326883in}{1.991802in}}%
\pgfpathcurveto{\pgfqpoint{5.319750in}{1.998935in}}{\pgfqpoint{5.310075in}{2.002943in}}{\pgfqpoint{5.299988in}{2.002943in}}%
\pgfpathcurveto{\pgfqpoint{5.289900in}{2.002943in}}{\pgfqpoint{5.280225in}{1.998935in}}{\pgfqpoint{5.273092in}{1.991802in}}%
\pgfpathcurveto{\pgfqpoint{5.265959in}{1.984669in}}{\pgfqpoint{5.261951in}{1.974994in}}{\pgfqpoint{5.261951in}{1.964906in}}%
\pgfpathcurveto{\pgfqpoint{5.261951in}{1.954819in}}{\pgfqpoint{5.265959in}{1.945143in}}{\pgfqpoint{5.273092in}{1.938011in}}%
\pgfpathcurveto{\pgfqpoint{5.280225in}{1.930878in}}{\pgfqpoint{5.289900in}{1.926870in}}{\pgfqpoint{5.299988in}{1.926870in}}%
\pgfpathlineto{\pgfqpoint{5.299988in}{1.926870in}}%
\pgfpathclose%
\pgfusepath{stroke,fill}%
\end{pgfscope}%
\begin{pgfscope}%
\pgfpathrectangle{\pgfqpoint{3.536584in}{0.147348in}}{\pgfqpoint{2.735294in}{2.735294in}}%
\pgfusepath{clip}%
\pgfsetbuttcap%
\pgfsetroundjoin%
\definecolor{currentfill}{rgb}{0.839216,0.152941,0.156863}%
\pgfsetfillcolor{currentfill}%
\pgfsetfillopacity{0.929084}%
\pgfsetlinewidth{1.003750pt}%
\definecolor{currentstroke}{rgb}{0.839216,0.152941,0.156863}%
\pgfsetstrokecolor{currentstroke}%
\pgfsetstrokeopacity{0.929084}%
\pgfsetdash{}{0pt}%
\pgfpathmoveto{\pgfqpoint{5.359400in}{1.288920in}}%
\pgfpathcurveto{\pgfqpoint{5.369488in}{1.288920in}}{\pgfqpoint{5.379163in}{1.292928in}}{\pgfqpoint{5.386296in}{1.300060in}}%
\pgfpathcurveto{\pgfqpoint{5.393429in}{1.307193in}}{\pgfqpoint{5.397437in}{1.316869in}}{\pgfqpoint{5.397437in}{1.326956in}}%
\pgfpathcurveto{\pgfqpoint{5.397437in}{1.337043in}}{\pgfqpoint{5.393429in}{1.346719in}}{\pgfqpoint{5.386296in}{1.353852in}}%
\pgfpathcurveto{\pgfqpoint{5.379163in}{1.360985in}}{\pgfqpoint{5.369488in}{1.364992in}}{\pgfqpoint{5.359400in}{1.364992in}}%
\pgfpathcurveto{\pgfqpoint{5.349313in}{1.364992in}}{\pgfqpoint{5.339637in}{1.360985in}}{\pgfqpoint{5.332505in}{1.353852in}}%
\pgfpathcurveto{\pgfqpoint{5.325372in}{1.346719in}}{\pgfqpoint{5.321364in}{1.337043in}}{\pgfqpoint{5.321364in}{1.326956in}}%
\pgfpathcurveto{\pgfqpoint{5.321364in}{1.316869in}}{\pgfqpoint{5.325372in}{1.307193in}}{\pgfqpoint{5.332505in}{1.300060in}}%
\pgfpathcurveto{\pgfqpoint{5.339637in}{1.292928in}}{\pgfqpoint{5.349313in}{1.288920in}}{\pgfqpoint{5.359400in}{1.288920in}}%
\pgfpathlineto{\pgfqpoint{5.359400in}{1.288920in}}%
\pgfpathclose%
\pgfusepath{stroke,fill}%
\end{pgfscope}%
\begin{pgfscope}%
\pgfpathrectangle{\pgfqpoint{3.536584in}{0.147348in}}{\pgfqpoint{2.735294in}{2.735294in}}%
\pgfusepath{clip}%
\pgfsetbuttcap%
\pgfsetroundjoin%
\definecolor{currentfill}{rgb}{0.839216,0.152941,0.156863}%
\pgfsetfillcolor{currentfill}%
\pgfsetlinewidth{1.003750pt}%
\definecolor{currentstroke}{rgb}{0.839216,0.152941,0.156863}%
\pgfsetstrokecolor{currentstroke}%
\pgfsetdash{}{0pt}%
\pgfpathmoveto{\pgfqpoint{5.239366in}{1.506712in}}%
\pgfpathcurveto{\pgfqpoint{5.249453in}{1.506712in}}{\pgfqpoint{5.259128in}{1.510720in}}{\pgfqpoint{5.266261in}{1.517853in}}%
\pgfpathcurveto{\pgfqpoint{5.273394in}{1.524986in}}{\pgfqpoint{5.277402in}{1.534661in}}{\pgfqpoint{5.277402in}{1.544748in}}%
\pgfpathcurveto{\pgfqpoint{5.277402in}{1.554836in}}{\pgfqpoint{5.273394in}{1.564511in}}{\pgfqpoint{5.266261in}{1.571644in}}%
\pgfpathcurveto{\pgfqpoint{5.259128in}{1.578777in}}{\pgfqpoint{5.249453in}{1.582785in}}{\pgfqpoint{5.239366in}{1.582785in}}%
\pgfpathcurveto{\pgfqpoint{5.229278in}{1.582785in}}{\pgfqpoint{5.219603in}{1.578777in}}{\pgfqpoint{5.212470in}{1.571644in}}%
\pgfpathcurveto{\pgfqpoint{5.205337in}{1.564511in}}{\pgfqpoint{5.201329in}{1.554836in}}{\pgfqpoint{5.201329in}{1.544748in}}%
\pgfpathcurveto{\pgfqpoint{5.201329in}{1.534661in}}{\pgfqpoint{5.205337in}{1.524986in}}{\pgfqpoint{5.212470in}{1.517853in}}%
\pgfpathcurveto{\pgfqpoint{5.219603in}{1.510720in}}{\pgfqpoint{5.229278in}{1.506712in}}{\pgfqpoint{5.239366in}{1.506712in}}%
\pgfpathlineto{\pgfqpoint{5.239366in}{1.506712in}}%
\pgfpathclose%
\pgfusepath{stroke,fill}%
\end{pgfscope}%
\begin{pgfscope}%
\pgfpathrectangle{\pgfqpoint{3.536584in}{0.147348in}}{\pgfqpoint{2.735294in}{2.735294in}}%
\pgfusepath{clip}%
\pgfsetbuttcap%
\pgfsetroundjoin%
\definecolor{currentfill}{rgb}{0.071067,0.258424,0.071067}%
\pgfsetfillcolor{currentfill}%
\pgfsetfillopacity{0.200000}%
\pgfsetlinewidth{0.000000pt}%
\definecolor{currentstroke}{rgb}{0.000000,0.000000,0.000000}%
\pgfsetstrokecolor{currentstroke}%
\pgfsetdash{}{0pt}%
\pgfpathmoveto{\pgfqpoint{3.979947in}{1.088921in}}%
\pgfpathlineto{\pgfqpoint{3.883527in}{1.228895in}}%
\pgfpathlineto{\pgfqpoint{3.884101in}{1.155548in}}%
\pgfpathlineto{\pgfqpoint{3.979947in}{1.088921in}}%
\pgfpathclose%
\pgfusepath{fill}%
\end{pgfscope}%
\begin{pgfscope}%
\pgfpathrectangle{\pgfqpoint{3.536584in}{0.147348in}}{\pgfqpoint{2.735294in}{2.735294in}}%
\pgfusepath{clip}%
\pgfsetbuttcap%
\pgfsetroundjoin%
\definecolor{currentfill}{rgb}{0.071067,0.258424,0.071067}%
\pgfsetfillcolor{currentfill}%
\pgfsetfillopacity{0.200000}%
\pgfsetlinewidth{0.000000pt}%
\definecolor{currentstroke}{rgb}{0.000000,0.000000,0.000000}%
\pgfsetstrokecolor{currentstroke}%
\pgfsetdash{}{0pt}%
\pgfpathmoveto{\pgfqpoint{5.998863in}{1.228895in}}%
\pgfpathlineto{\pgfqpoint{5.902443in}{1.088921in}}%
\pgfpathlineto{\pgfqpoint{5.998289in}{1.155548in}}%
\pgfpathlineto{\pgfqpoint{5.998863in}{1.228895in}}%
\pgfpathclose%
\pgfusepath{fill}%
\end{pgfscope}%
\begin{pgfscope}%
\pgfpathrectangle{\pgfqpoint{3.536584in}{0.147348in}}{\pgfqpoint{2.735294in}{2.735294in}}%
\pgfusepath{clip}%
\pgfsetbuttcap%
\pgfsetroundjoin%
\definecolor{currentfill}{rgb}{0.128601,0.467641,0.128601}%
\pgfsetfillcolor{currentfill}%
\pgfsetfillopacity{0.200000}%
\pgfsetlinewidth{0.000000pt}%
\definecolor{currentstroke}{rgb}{0.000000,0.000000,0.000000}%
\pgfsetstrokecolor{currentstroke}%
\pgfsetdash{}{0pt}%
\pgfpathmoveto{\pgfqpoint{4.844774in}{2.632943in}}%
\pgfpathlineto{\pgfqpoint{5.037616in}{2.632943in}}%
\pgfpathlineto{\pgfqpoint{4.941195in}{2.699366in}}%
\pgfpathlineto{\pgfqpoint{4.844774in}{2.632943in}}%
\pgfpathclose%
\pgfusepath{fill}%
\end{pgfscope}%
\begin{pgfscope}%
\pgfpathrectangle{\pgfqpoint{3.536584in}{0.147348in}}{\pgfqpoint{2.735294in}{2.735294in}}%
\pgfusepath{clip}%
\pgfsetbuttcap%
\pgfsetroundjoin%
\definecolor{currentfill}{rgb}{0.067488,0.245410,0.067488}%
\pgfsetfillcolor{currentfill}%
\pgfsetfillopacity{0.200000}%
\pgfsetlinewidth{0.000000pt}%
\definecolor{currentstroke}{rgb}{0.000000,0.000000,0.000000}%
\pgfsetstrokecolor{currentstroke}%
\pgfsetdash{}{0pt}%
\pgfpathmoveto{\pgfqpoint{4.104467in}{1.018707in}}%
\pgfpathlineto{\pgfqpoint{3.989844in}{1.163589in}}%
\pgfpathlineto{\pgfqpoint{3.979947in}{1.088921in}}%
\pgfpathlineto{\pgfqpoint{4.104467in}{1.018707in}}%
\pgfpathclose%
\pgfusepath{fill}%
\end{pgfscope}%
\begin{pgfscope}%
\pgfpathrectangle{\pgfqpoint{3.536584in}{0.147348in}}{\pgfqpoint{2.735294in}{2.735294in}}%
\pgfusepath{clip}%
\pgfsetbuttcap%
\pgfsetroundjoin%
\definecolor{currentfill}{rgb}{0.067488,0.245410,0.067488}%
\pgfsetfillcolor{currentfill}%
\pgfsetfillopacity{0.200000}%
\pgfsetlinewidth{0.000000pt}%
\definecolor{currentstroke}{rgb}{0.000000,0.000000,0.000000}%
\pgfsetstrokecolor{currentstroke}%
\pgfsetdash{}{0pt}%
\pgfpathmoveto{\pgfqpoint{5.902443in}{1.088921in}}%
\pgfpathlineto{\pgfqpoint{5.892546in}{1.163589in}}%
\pgfpathlineto{\pgfqpoint{5.777922in}{1.018707in}}%
\pgfpathlineto{\pgfqpoint{5.902443in}{1.088921in}}%
\pgfpathclose%
\pgfusepath{fill}%
\end{pgfscope}%
\begin{pgfscope}%
\pgfpathrectangle{\pgfqpoint{3.536584in}{0.147348in}}{\pgfqpoint{2.735294in}{2.735294in}}%
\pgfusepath{clip}%
\pgfsetbuttcap%
\pgfsetroundjoin%
\definecolor{currentfill}{rgb}{0.069492,0.252698,0.069492}%
\pgfsetfillcolor{currentfill}%
\pgfsetfillopacity{0.200000}%
\pgfsetlinewidth{0.000000pt}%
\definecolor{currentstroke}{rgb}{0.000000,0.000000,0.000000}%
\pgfsetstrokecolor{currentstroke}%
\pgfsetdash{}{0pt}%
\pgfpathmoveto{\pgfqpoint{3.883527in}{1.228895in}}%
\pgfpathlineto{\pgfqpoint{3.979947in}{1.088921in}}%
\pgfpathlineto{\pgfqpoint{3.977739in}{1.589726in}}%
\pgfpathlineto{\pgfqpoint{3.883527in}{1.228895in}}%
\pgfpathclose%
\pgfusepath{fill}%
\end{pgfscope}%
\begin{pgfscope}%
\pgfpathrectangle{\pgfqpoint{3.536584in}{0.147348in}}{\pgfqpoint{2.735294in}{2.735294in}}%
\pgfusepath{clip}%
\pgfsetbuttcap%
\pgfsetroundjoin%
\definecolor{currentfill}{rgb}{0.069492,0.252698,0.069492}%
\pgfsetfillcolor{currentfill}%
\pgfsetfillopacity{0.200000}%
\pgfsetlinewidth{0.000000pt}%
\definecolor{currentstroke}{rgb}{0.000000,0.000000,0.000000}%
\pgfsetstrokecolor{currentstroke}%
\pgfsetdash{}{0pt}%
\pgfpathmoveto{\pgfqpoint{5.998863in}{1.228895in}}%
\pgfpathlineto{\pgfqpoint{5.904651in}{1.589726in}}%
\pgfpathlineto{\pgfqpoint{5.902443in}{1.088921in}}%
\pgfpathlineto{\pgfqpoint{5.998863in}{1.228895in}}%
\pgfpathclose%
\pgfusepath{fill}%
\end{pgfscope}%
\begin{pgfscope}%
\pgfpathrectangle{\pgfqpoint{3.536584in}{0.147348in}}{\pgfqpoint{2.735294in}{2.735294in}}%
\pgfusepath{clip}%
\pgfsetbuttcap%
\pgfsetroundjoin%
\definecolor{currentfill}{rgb}{0.099716,0.362602,0.099716}%
\pgfsetfillcolor{currentfill}%
\pgfsetfillopacity{0.200000}%
\pgfsetlinewidth{0.000000pt}%
\definecolor{currentstroke}{rgb}{0.000000,0.000000,0.000000}%
\pgfsetstrokecolor{currentstroke}%
\pgfsetdash{}{0pt}%
\pgfpathmoveto{\pgfqpoint{3.977739in}{1.589726in}}%
\pgfpathlineto{\pgfqpoint{3.979947in}{1.088921in}}%
\pgfpathlineto{\pgfqpoint{3.989844in}{1.163589in}}%
\pgfpathlineto{\pgfqpoint{3.977739in}{1.589726in}}%
\pgfpathclose%
\pgfusepath{fill}%
\end{pgfscope}%
\begin{pgfscope}%
\pgfpathrectangle{\pgfqpoint{3.536584in}{0.147348in}}{\pgfqpoint{2.735294in}{2.735294in}}%
\pgfusepath{clip}%
\pgfsetbuttcap%
\pgfsetroundjoin%
\definecolor{currentfill}{rgb}{0.099716,0.362602,0.099716}%
\pgfsetfillcolor{currentfill}%
\pgfsetfillopacity{0.200000}%
\pgfsetlinewidth{0.000000pt}%
\definecolor{currentstroke}{rgb}{0.000000,0.000000,0.000000}%
\pgfsetstrokecolor{currentstroke}%
\pgfsetdash{}{0pt}%
\pgfpathmoveto{\pgfqpoint{5.892546in}{1.163589in}}%
\pgfpathlineto{\pgfqpoint{5.902443in}{1.088921in}}%
\pgfpathlineto{\pgfqpoint{5.904651in}{1.589726in}}%
\pgfpathlineto{\pgfqpoint{5.892546in}{1.163589in}}%
\pgfpathclose%
\pgfusepath{fill}%
\end{pgfscope}%
\begin{pgfscope}%
\pgfpathrectangle{\pgfqpoint{3.536584in}{0.147348in}}{\pgfqpoint{2.735294in}{2.735294in}}%
\pgfusepath{clip}%
\pgfsetbuttcap%
\pgfsetroundjoin%
\definecolor{currentfill}{rgb}{0.063840,0.232145,0.063840}%
\pgfsetfillcolor{currentfill}%
\pgfsetfillopacity{0.200000}%
\pgfsetlinewidth{0.000000pt}%
\definecolor{currentstroke}{rgb}{0.000000,0.000000,0.000000}%
\pgfsetstrokecolor{currentstroke}%
\pgfsetdash{}{0pt}%
\pgfpathmoveto{\pgfqpoint{4.263554in}{0.949464in}}%
\pgfpathlineto{\pgfqpoint{4.130959in}{1.094945in}}%
\pgfpathlineto{\pgfqpoint{4.104467in}{1.018707in}}%
\pgfpathlineto{\pgfqpoint{4.263554in}{0.949464in}}%
\pgfpathclose%
\pgfusepath{fill}%
\end{pgfscope}%
\begin{pgfscope}%
\pgfpathrectangle{\pgfqpoint{3.536584in}{0.147348in}}{\pgfqpoint{2.735294in}{2.735294in}}%
\pgfusepath{clip}%
\pgfsetbuttcap%
\pgfsetroundjoin%
\definecolor{currentfill}{rgb}{0.063840,0.232145,0.063840}%
\pgfsetfillcolor{currentfill}%
\pgfsetfillopacity{0.200000}%
\pgfsetlinewidth{0.000000pt}%
\definecolor{currentstroke}{rgb}{0.000000,0.000000,0.000000}%
\pgfsetstrokecolor{currentstroke}%
\pgfsetdash{}{0pt}%
\pgfpathmoveto{\pgfqpoint{5.777922in}{1.018707in}}%
\pgfpathlineto{\pgfqpoint{5.751431in}{1.094945in}}%
\pgfpathlineto{\pgfqpoint{5.618836in}{0.949464in}}%
\pgfpathlineto{\pgfqpoint{5.777922in}{1.018707in}}%
\pgfpathclose%
\pgfusepath{fill}%
\end{pgfscope}%
\begin{pgfscope}%
\pgfpathrectangle{\pgfqpoint{3.536584in}{0.147348in}}{\pgfqpoint{2.735294in}{2.735294in}}%
\pgfusepath{clip}%
\pgfsetbuttcap%
\pgfsetroundjoin%
\definecolor{currentfill}{rgb}{0.116321,0.422987,0.116321}%
\pgfsetfillcolor{currentfill}%
\pgfsetfillopacity{0.200000}%
\pgfsetlinewidth{0.000000pt}%
\definecolor{currentstroke}{rgb}{0.000000,0.000000,0.000000}%
\pgfsetstrokecolor{currentstroke}%
\pgfsetdash{}{0pt}%
\pgfpathmoveto{\pgfqpoint{5.037616in}{2.632943in}}%
\pgfpathlineto{\pgfqpoint{4.844774in}{2.632943in}}%
\pgfpathlineto{\pgfqpoint{4.802903in}{2.150689in}}%
\pgfpathlineto{\pgfqpoint{5.037616in}{2.632943in}}%
\pgfpathclose%
\pgfusepath{fill}%
\end{pgfscope}%
\begin{pgfscope}%
\pgfpathrectangle{\pgfqpoint{3.536584in}{0.147348in}}{\pgfqpoint{2.735294in}{2.735294in}}%
\pgfusepath{clip}%
\pgfsetbuttcap%
\pgfsetroundjoin%
\definecolor{currentfill}{rgb}{0.067061,0.243857,0.067061}%
\pgfsetfillcolor{currentfill}%
\pgfsetfillopacity{0.200000}%
\pgfsetlinewidth{0.000000pt}%
\definecolor{currentstroke}{rgb}{0.000000,0.000000,0.000000}%
\pgfsetstrokecolor{currentstroke}%
\pgfsetdash{}{0pt}%
\pgfpathmoveto{\pgfqpoint{3.989844in}{1.163589in}}%
\pgfpathlineto{\pgfqpoint{4.104467in}{1.018707in}}%
\pgfpathlineto{\pgfqpoint{4.139943in}{1.557460in}}%
\pgfpathlineto{\pgfqpoint{3.989844in}{1.163589in}}%
\pgfpathclose%
\pgfusepath{fill}%
\end{pgfscope}%
\begin{pgfscope}%
\pgfpathrectangle{\pgfqpoint{3.536584in}{0.147348in}}{\pgfqpoint{2.735294in}{2.735294in}}%
\pgfusepath{clip}%
\pgfsetbuttcap%
\pgfsetroundjoin%
\definecolor{currentfill}{rgb}{0.067061,0.243857,0.067061}%
\pgfsetfillcolor{currentfill}%
\pgfsetfillopacity{0.200000}%
\pgfsetlinewidth{0.000000pt}%
\definecolor{currentstroke}{rgb}{0.000000,0.000000,0.000000}%
\pgfsetstrokecolor{currentstroke}%
\pgfsetdash{}{0pt}%
\pgfpathmoveto{\pgfqpoint{5.742447in}{1.557460in}}%
\pgfpathlineto{\pgfqpoint{5.777922in}{1.018707in}}%
\pgfpathlineto{\pgfqpoint{5.892546in}{1.163589in}}%
\pgfpathlineto{\pgfqpoint{5.742447in}{1.557460in}}%
\pgfpathclose%
\pgfusepath{fill}%
\end{pgfscope}%
\begin{pgfscope}%
\pgfpathrectangle{\pgfqpoint{3.536584in}{0.147348in}}{\pgfqpoint{2.735294in}{2.735294in}}%
\pgfusepath{clip}%
\pgfsetbuttcap%
\pgfsetroundjoin%
\definecolor{currentfill}{rgb}{0.095351,0.346729,0.095351}%
\pgfsetfillcolor{currentfill}%
\pgfsetfillopacity{0.200000}%
\pgfsetlinewidth{0.000000pt}%
\definecolor{currentstroke}{rgb}{0.000000,0.000000,0.000000}%
\pgfsetstrokecolor{currentstroke}%
\pgfsetdash{}{0pt}%
\pgfpathmoveto{\pgfqpoint{4.139943in}{1.557460in}}%
\pgfpathlineto{\pgfqpoint{4.104467in}{1.018707in}}%
\pgfpathlineto{\pgfqpoint{4.130959in}{1.094945in}}%
\pgfpathlineto{\pgfqpoint{4.139943in}{1.557460in}}%
\pgfpathclose%
\pgfusepath{fill}%
\end{pgfscope}%
\begin{pgfscope}%
\pgfpathrectangle{\pgfqpoint{3.536584in}{0.147348in}}{\pgfqpoint{2.735294in}{2.735294in}}%
\pgfusepath{clip}%
\pgfsetbuttcap%
\pgfsetroundjoin%
\definecolor{currentfill}{rgb}{0.095351,0.346729,0.095351}%
\pgfsetfillcolor{currentfill}%
\pgfsetfillopacity{0.200000}%
\pgfsetlinewidth{0.000000pt}%
\definecolor{currentstroke}{rgb}{0.000000,0.000000,0.000000}%
\pgfsetstrokecolor{currentstroke}%
\pgfsetdash{}{0pt}%
\pgfpathmoveto{\pgfqpoint{5.751431in}{1.094945in}}%
\pgfpathlineto{\pgfqpoint{5.777922in}{1.018707in}}%
\pgfpathlineto{\pgfqpoint{5.742447in}{1.557460in}}%
\pgfpathlineto{\pgfqpoint{5.751431in}{1.094945in}}%
\pgfpathclose%
\pgfusepath{fill}%
\end{pgfscope}%
\begin{pgfscope}%
\pgfpathrectangle{\pgfqpoint{3.536584in}{0.147348in}}{\pgfqpoint{2.735294in}{2.735294in}}%
\pgfusepath{clip}%
\pgfsetbuttcap%
\pgfsetroundjoin%
\definecolor{currentfill}{rgb}{0.060435,0.219763,0.060435}%
\pgfsetfillcolor{currentfill}%
\pgfsetfillopacity{0.200000}%
\pgfsetlinewidth{0.000000pt}%
\definecolor{currentstroke}{rgb}{0.000000,0.000000,0.000000}%
\pgfsetstrokecolor{currentstroke}%
\pgfsetdash{}{0pt}%
\pgfpathmoveto{\pgfqpoint{5.568365in}{1.028737in}}%
\pgfpathlineto{\pgfqpoint{5.422251in}{0.888724in}}%
\pgfpathlineto{\pgfqpoint{5.618836in}{0.949464in}}%
\pgfpathlineto{\pgfqpoint{5.568365in}{1.028737in}}%
\pgfpathclose%
\pgfusepath{fill}%
\end{pgfscope}%
\begin{pgfscope}%
\pgfpathrectangle{\pgfqpoint{3.536584in}{0.147348in}}{\pgfqpoint{2.735294in}{2.735294in}}%
\pgfusepath{clip}%
\pgfsetbuttcap%
\pgfsetroundjoin%
\definecolor{currentfill}{rgb}{0.060435,0.219763,0.060435}%
\pgfsetfillcolor{currentfill}%
\pgfsetfillopacity{0.200000}%
\pgfsetlinewidth{0.000000pt}%
\definecolor{currentstroke}{rgb}{0.000000,0.000000,0.000000}%
\pgfsetstrokecolor{currentstroke}%
\pgfsetdash{}{0pt}%
\pgfpathmoveto{\pgfqpoint{4.263554in}{0.949464in}}%
\pgfpathlineto{\pgfqpoint{4.460139in}{0.888724in}}%
\pgfpathlineto{\pgfqpoint{4.314024in}{1.028737in}}%
\pgfpathlineto{\pgfqpoint{4.263554in}{0.949464in}}%
\pgfpathclose%
\pgfusepath{fill}%
\end{pgfscope}%
\begin{pgfscope}%
\pgfpathrectangle{\pgfqpoint{3.536584in}{0.147348in}}{\pgfqpoint{2.735294in}{2.735294in}}%
\pgfusepath{clip}%
\pgfsetbuttcap%
\pgfsetroundjoin%
\definecolor{currentfill}{rgb}{0.074506,0.270932,0.074506}%
\pgfsetfillcolor{currentfill}%
\pgfsetfillopacity{0.200000}%
\pgfsetlinewidth{0.000000pt}%
\definecolor{currentstroke}{rgb}{0.000000,0.000000,0.000000}%
\pgfsetstrokecolor{currentstroke}%
\pgfsetdash{}{0pt}%
\pgfpathmoveto{\pgfqpoint{3.977739in}{1.589726in}}%
\pgfpathlineto{\pgfqpoint{3.989844in}{1.163589in}}%
\pgfpathlineto{\pgfqpoint{4.139943in}{1.557460in}}%
\pgfpathlineto{\pgfqpoint{3.977739in}{1.589726in}}%
\pgfpathclose%
\pgfusepath{fill}%
\end{pgfscope}%
\begin{pgfscope}%
\pgfpathrectangle{\pgfqpoint{3.536584in}{0.147348in}}{\pgfqpoint{2.735294in}{2.735294in}}%
\pgfusepath{clip}%
\pgfsetbuttcap%
\pgfsetroundjoin%
\definecolor{currentfill}{rgb}{0.074506,0.270932,0.074506}%
\pgfsetfillcolor{currentfill}%
\pgfsetfillopacity{0.200000}%
\pgfsetlinewidth{0.000000pt}%
\definecolor{currentstroke}{rgb}{0.000000,0.000000,0.000000}%
\pgfsetstrokecolor{currentstroke}%
\pgfsetdash{}{0pt}%
\pgfpathmoveto{\pgfqpoint{5.742447in}{1.557460in}}%
\pgfpathlineto{\pgfqpoint{5.892546in}{1.163589in}}%
\pgfpathlineto{\pgfqpoint{5.904651in}{1.589726in}}%
\pgfpathlineto{\pgfqpoint{5.742447in}{1.557460in}}%
\pgfpathclose%
\pgfusepath{fill}%
\end{pgfscope}%
\begin{pgfscope}%
\pgfpathrectangle{\pgfqpoint{3.536584in}{0.147348in}}{\pgfqpoint{2.735294in}{2.735294in}}%
\pgfusepath{clip}%
\pgfsetbuttcap%
\pgfsetroundjoin%
\definecolor{currentfill}{rgb}{0.116785,0.424671,0.116785}%
\pgfsetfillcolor{currentfill}%
\pgfsetfillopacity{0.200000}%
\pgfsetlinewidth{0.000000pt}%
\definecolor{currentstroke}{rgb}{0.000000,0.000000,0.000000}%
\pgfsetstrokecolor{currentstroke}%
\pgfsetdash{}{0pt}%
\pgfpathmoveto{\pgfqpoint{5.423597in}{2.292690in}}%
\pgfpathlineto{\pgfqpoint{5.037616in}{2.632943in}}%
\pgfpathlineto{\pgfqpoint{5.079487in}{2.150689in}}%
\pgfpathlineto{\pgfqpoint{5.423597in}{2.292690in}}%
\pgfpathclose%
\pgfusepath{fill}%
\end{pgfscope}%
\begin{pgfscope}%
\pgfpathrectangle{\pgfqpoint{3.536584in}{0.147348in}}{\pgfqpoint{2.735294in}{2.735294in}}%
\pgfusepath{clip}%
\pgfsetbuttcap%
\pgfsetroundjoin%
\definecolor{currentfill}{rgb}{0.116785,0.424671,0.116785}%
\pgfsetfillcolor{currentfill}%
\pgfsetfillopacity{0.200000}%
\pgfsetlinewidth{0.000000pt}%
\definecolor{currentstroke}{rgb}{0.000000,0.000000,0.000000}%
\pgfsetstrokecolor{currentstroke}%
\pgfsetdash{}{0pt}%
\pgfpathmoveto{\pgfqpoint{4.802903in}{2.150689in}}%
\pgfpathlineto{\pgfqpoint{4.844774in}{2.632943in}}%
\pgfpathlineto{\pgfqpoint{4.458793in}{2.292690in}}%
\pgfpathlineto{\pgfqpoint{4.802903in}{2.150689in}}%
\pgfpathclose%
\pgfusepath{fill}%
\end{pgfscope}%
\begin{pgfscope}%
\pgfpathrectangle{\pgfqpoint{3.536584in}{0.147348in}}{\pgfqpoint{2.735294in}{2.735294in}}%
\pgfusepath{clip}%
\pgfsetbuttcap%
\pgfsetroundjoin%
\definecolor{currentfill}{rgb}{0.064867,0.235879,0.064867}%
\pgfsetfillcolor{currentfill}%
\pgfsetfillopacity{0.200000}%
\pgfsetlinewidth{0.000000pt}%
\definecolor{currentstroke}{rgb}{0.000000,0.000000,0.000000}%
\pgfsetstrokecolor{currentstroke}%
\pgfsetdash{}{0pt}%
\pgfpathmoveto{\pgfqpoint{4.130959in}{1.094945in}}%
\pgfpathlineto{\pgfqpoint{4.263554in}{0.949464in}}%
\pgfpathlineto{\pgfqpoint{4.357383in}{1.526914in}}%
\pgfpathlineto{\pgfqpoint{4.130959in}{1.094945in}}%
\pgfpathclose%
\pgfusepath{fill}%
\end{pgfscope}%
\begin{pgfscope}%
\pgfpathrectangle{\pgfqpoint{3.536584in}{0.147348in}}{\pgfqpoint{2.735294in}{2.735294in}}%
\pgfusepath{clip}%
\pgfsetbuttcap%
\pgfsetroundjoin%
\definecolor{currentfill}{rgb}{0.064867,0.235879,0.064867}%
\pgfsetfillcolor{currentfill}%
\pgfsetfillopacity{0.200000}%
\pgfsetlinewidth{0.000000pt}%
\definecolor{currentstroke}{rgb}{0.000000,0.000000,0.000000}%
\pgfsetstrokecolor{currentstroke}%
\pgfsetdash{}{0pt}%
\pgfpathmoveto{\pgfqpoint{5.525007in}{1.526914in}}%
\pgfpathlineto{\pgfqpoint{5.618836in}{0.949464in}}%
\pgfpathlineto{\pgfqpoint{5.751431in}{1.094945in}}%
\pgfpathlineto{\pgfqpoint{5.525007in}{1.526914in}}%
\pgfpathclose%
\pgfusepath{fill}%
\end{pgfscope}%
\begin{pgfscope}%
\pgfpathrectangle{\pgfqpoint{3.536584in}{0.147348in}}{\pgfqpoint{2.735294in}{2.735294in}}%
\pgfusepath{clip}%
\pgfsetbuttcap%
\pgfsetroundjoin%
\definecolor{currentfill}{rgb}{0.057724,0.209904,0.057724}%
\pgfsetfillcolor{currentfill}%
\pgfsetfillopacity{0.200000}%
\pgfsetlinewidth{0.000000pt}%
\definecolor{currentstroke}{rgb}{0.000000,0.000000,0.000000}%
\pgfsetstrokecolor{currentstroke}%
\pgfsetdash{}{0pt}%
\pgfpathmoveto{\pgfqpoint{4.541034in}{0.974376in}}%
\pgfpathlineto{\pgfqpoint{4.460139in}{0.888724in}}%
\pgfpathlineto{\pgfqpoint{4.690418in}{0.846223in}}%
\pgfpathlineto{\pgfqpoint{4.541034in}{0.974376in}}%
\pgfpathclose%
\pgfusepath{fill}%
\end{pgfscope}%
\begin{pgfscope}%
\pgfpathrectangle{\pgfqpoint{3.536584in}{0.147348in}}{\pgfqpoint{2.735294in}{2.735294in}}%
\pgfusepath{clip}%
\pgfsetbuttcap%
\pgfsetroundjoin%
\definecolor{currentfill}{rgb}{0.057724,0.209904,0.057724}%
\pgfsetfillcolor{currentfill}%
\pgfsetfillopacity{0.200000}%
\pgfsetlinewidth{0.000000pt}%
\definecolor{currentstroke}{rgb}{0.000000,0.000000,0.000000}%
\pgfsetstrokecolor{currentstroke}%
\pgfsetdash{}{0pt}%
\pgfpathmoveto{\pgfqpoint{5.191972in}{0.846223in}}%
\pgfpathlineto{\pgfqpoint{5.422251in}{0.888724in}}%
\pgfpathlineto{\pgfqpoint{5.341356in}{0.974376in}}%
\pgfpathlineto{\pgfqpoint{5.191972in}{0.846223in}}%
\pgfpathclose%
\pgfusepath{fill}%
\end{pgfscope}%
\begin{pgfscope}%
\pgfpathrectangle{\pgfqpoint{3.536584in}{0.147348in}}{\pgfqpoint{2.735294in}{2.735294in}}%
\pgfusepath{clip}%
\pgfsetbuttcap%
\pgfsetroundjoin%
\definecolor{currentfill}{rgb}{0.086498,0.314539,0.086498}%
\pgfsetfillcolor{currentfill}%
\pgfsetfillopacity{0.200000}%
\pgfsetlinewidth{0.000000pt}%
\definecolor{currentstroke}{rgb}{0.000000,0.000000,0.000000}%
\pgfsetstrokecolor{currentstroke}%
\pgfsetdash{}{0pt}%
\pgfpathmoveto{\pgfqpoint{4.139943in}{1.557460in}}%
\pgfpathlineto{\pgfqpoint{4.061876in}{1.760124in}}%
\pgfpathlineto{\pgfqpoint{3.977739in}{1.589726in}}%
\pgfpathlineto{\pgfqpoint{4.139943in}{1.557460in}}%
\pgfpathclose%
\pgfusepath{fill}%
\end{pgfscope}%
\begin{pgfscope}%
\pgfpathrectangle{\pgfqpoint{3.536584in}{0.147348in}}{\pgfqpoint{2.735294in}{2.735294in}}%
\pgfusepath{clip}%
\pgfsetbuttcap%
\pgfsetroundjoin%
\definecolor{currentfill}{rgb}{0.086498,0.314539,0.086498}%
\pgfsetfillcolor{currentfill}%
\pgfsetfillopacity{0.200000}%
\pgfsetlinewidth{0.000000pt}%
\definecolor{currentstroke}{rgb}{0.000000,0.000000,0.000000}%
\pgfsetstrokecolor{currentstroke}%
\pgfsetdash{}{0pt}%
\pgfpathmoveto{\pgfqpoint{5.904651in}{1.589726in}}%
\pgfpathlineto{\pgfqpoint{5.820514in}{1.760124in}}%
\pgfpathlineto{\pgfqpoint{5.742447in}{1.557460in}}%
\pgfpathlineto{\pgfqpoint{5.904651in}{1.589726in}}%
\pgfpathclose%
\pgfusepath{fill}%
\end{pgfscope}%
\begin{pgfscope}%
\pgfpathrectangle{\pgfqpoint{3.536584in}{0.147348in}}{\pgfqpoint{2.735294in}{2.735294in}}%
\pgfusepath{clip}%
\pgfsetbuttcap%
\pgfsetroundjoin%
\definecolor{currentfill}{rgb}{0.090812,0.330224,0.090812}%
\pgfsetfillcolor{currentfill}%
\pgfsetfillopacity{0.200000}%
\pgfsetlinewidth{0.000000pt}%
\definecolor{currentstroke}{rgb}{0.000000,0.000000,0.000000}%
\pgfsetstrokecolor{currentstroke}%
\pgfsetdash{}{0pt}%
\pgfpathmoveto{\pgfqpoint{4.357383in}{1.526914in}}%
\pgfpathlineto{\pgfqpoint{4.263554in}{0.949464in}}%
\pgfpathlineto{\pgfqpoint{4.314024in}{1.028737in}}%
\pgfpathlineto{\pgfqpoint{4.357383in}{1.526914in}}%
\pgfpathclose%
\pgfusepath{fill}%
\end{pgfscope}%
\begin{pgfscope}%
\pgfpathrectangle{\pgfqpoint{3.536584in}{0.147348in}}{\pgfqpoint{2.735294in}{2.735294in}}%
\pgfusepath{clip}%
\pgfsetbuttcap%
\pgfsetroundjoin%
\definecolor{currentfill}{rgb}{0.090812,0.330224,0.090812}%
\pgfsetfillcolor{currentfill}%
\pgfsetfillopacity{0.200000}%
\pgfsetlinewidth{0.000000pt}%
\definecolor{currentstroke}{rgb}{0.000000,0.000000,0.000000}%
\pgfsetstrokecolor{currentstroke}%
\pgfsetdash{}{0pt}%
\pgfpathmoveto{\pgfqpoint{5.568365in}{1.028737in}}%
\pgfpathlineto{\pgfqpoint{5.618836in}{0.949464in}}%
\pgfpathlineto{\pgfqpoint{5.525007in}{1.526914in}}%
\pgfpathlineto{\pgfqpoint{5.568365in}{1.028737in}}%
\pgfpathclose%
\pgfusepath{fill}%
\end{pgfscope}%
\begin{pgfscope}%
\pgfpathrectangle{\pgfqpoint{3.536584in}{0.147348in}}{\pgfqpoint{2.735294in}{2.735294in}}%
\pgfusepath{clip}%
\pgfsetbuttcap%
\pgfsetroundjoin%
\definecolor{currentfill}{rgb}{0.116321,0.422987,0.116321}%
\pgfsetfillcolor{currentfill}%
\pgfsetfillopacity{0.200000}%
\pgfsetlinewidth{0.000000pt}%
\definecolor{currentstroke}{rgb}{0.000000,0.000000,0.000000}%
\pgfsetstrokecolor{currentstroke}%
\pgfsetdash{}{0pt}%
\pgfpathmoveto{\pgfqpoint{4.802903in}{2.150689in}}%
\pgfpathlineto{\pgfqpoint{5.079487in}{2.150689in}}%
\pgfpathlineto{\pgfqpoint{5.037616in}{2.632943in}}%
\pgfpathlineto{\pgfqpoint{4.802903in}{2.150689in}}%
\pgfpathclose%
\pgfusepath{fill}%
\end{pgfscope}%
\begin{pgfscope}%
\pgfpathrectangle{\pgfqpoint{3.536584in}{0.147348in}}{\pgfqpoint{2.735294in}{2.735294in}}%
\pgfusepath{clip}%
\pgfsetbuttcap%
\pgfsetroundjoin%
\definecolor{currentfill}{rgb}{0.056200,0.204363,0.056200}%
\pgfsetfillcolor{currentfill}%
\pgfsetfillopacity{0.200000}%
\pgfsetlinewidth{0.000000pt}%
\definecolor{currentstroke}{rgb}{0.000000,0.000000,0.000000}%
\pgfsetstrokecolor{currentstroke}%
\pgfsetdash{}{0pt}%
\pgfpathmoveto{\pgfqpoint{4.941195in}{0.830831in}}%
\pgfpathlineto{\pgfqpoint{4.803235in}{0.943120in}}%
\pgfpathlineto{\pgfqpoint{4.690418in}{0.846223in}}%
\pgfpathlineto{\pgfqpoint{4.941195in}{0.830831in}}%
\pgfpathclose%
\pgfusepath{fill}%
\end{pgfscope}%
\begin{pgfscope}%
\pgfpathrectangle{\pgfqpoint{3.536584in}{0.147348in}}{\pgfqpoint{2.735294in}{2.735294in}}%
\pgfusepath{clip}%
\pgfsetbuttcap%
\pgfsetroundjoin%
\definecolor{currentfill}{rgb}{0.056200,0.204363,0.056200}%
\pgfsetfillcolor{currentfill}%
\pgfsetfillopacity{0.200000}%
\pgfsetlinewidth{0.000000pt}%
\definecolor{currentstroke}{rgb}{0.000000,0.000000,0.000000}%
\pgfsetstrokecolor{currentstroke}%
\pgfsetdash{}{0pt}%
\pgfpathmoveto{\pgfqpoint{5.191972in}{0.846223in}}%
\pgfpathlineto{\pgfqpoint{5.079154in}{0.943120in}}%
\pgfpathlineto{\pgfqpoint{4.941195in}{0.830831in}}%
\pgfpathlineto{\pgfqpoint{5.191972in}{0.846223in}}%
\pgfpathclose%
\pgfusepath{fill}%
\end{pgfscope}%
\begin{pgfscope}%
\pgfpathrectangle{\pgfqpoint{3.536584in}{0.147348in}}{\pgfqpoint{2.735294in}{2.735294in}}%
\pgfusepath{clip}%
\pgfsetbuttcap%
\pgfsetroundjoin%
\definecolor{currentfill}{rgb}{0.107070,0.389346,0.107070}%
\pgfsetfillcolor{currentfill}%
\pgfsetfillopacity{0.200000}%
\pgfsetlinewidth{0.000000pt}%
\definecolor{currentstroke}{rgb}{0.000000,0.000000,0.000000}%
\pgfsetstrokecolor{currentstroke}%
\pgfsetdash{}{0pt}%
\pgfpathmoveto{\pgfqpoint{4.540103in}{2.141900in}}%
\pgfpathlineto{\pgfqpoint{4.458793in}{2.292690in}}%
\pgfpathlineto{\pgfqpoint{4.312652in}{2.126617in}}%
\pgfpathlineto{\pgfqpoint{4.540103in}{2.141900in}}%
\pgfpathclose%
\pgfusepath{fill}%
\end{pgfscope}%
\begin{pgfscope}%
\pgfpathrectangle{\pgfqpoint{3.536584in}{0.147348in}}{\pgfqpoint{2.735294in}{2.735294in}}%
\pgfusepath{clip}%
\pgfsetbuttcap%
\pgfsetroundjoin%
\definecolor{currentfill}{rgb}{0.107070,0.389346,0.107070}%
\pgfsetfillcolor{currentfill}%
\pgfsetfillopacity{0.200000}%
\pgfsetlinewidth{0.000000pt}%
\definecolor{currentstroke}{rgb}{0.000000,0.000000,0.000000}%
\pgfsetstrokecolor{currentstroke}%
\pgfsetdash{}{0pt}%
\pgfpathmoveto{\pgfqpoint{5.569738in}{2.126617in}}%
\pgfpathlineto{\pgfqpoint{5.423597in}{2.292690in}}%
\pgfpathlineto{\pgfqpoint{5.342287in}{2.141900in}}%
\pgfpathlineto{\pgfqpoint{5.569738in}{2.126617in}}%
\pgfpathclose%
\pgfusepath{fill}%
\end{pgfscope}%
\begin{pgfscope}%
\pgfpathrectangle{\pgfqpoint{3.536584in}{0.147348in}}{\pgfqpoint{2.735294in}{2.735294in}}%
\pgfusepath{clip}%
\pgfsetbuttcap%
\pgfsetroundjoin%
\definecolor{currentfill}{rgb}{0.089078,0.323920,0.089078}%
\pgfsetfillcolor{currentfill}%
\pgfsetfillopacity{0.200000}%
\pgfsetlinewidth{0.000000pt}%
\definecolor{currentstroke}{rgb}{0.000000,0.000000,0.000000}%
\pgfsetstrokecolor{currentstroke}%
\pgfsetdash{}{0pt}%
\pgfpathmoveto{\pgfqpoint{4.139943in}{1.557460in}}%
\pgfpathlineto{\pgfqpoint{4.312652in}{2.126617in}}%
\pgfpathlineto{\pgfqpoint{4.061876in}{1.760124in}}%
\pgfpathlineto{\pgfqpoint{4.139943in}{1.557460in}}%
\pgfpathclose%
\pgfusepath{fill}%
\end{pgfscope}%
\begin{pgfscope}%
\pgfpathrectangle{\pgfqpoint{3.536584in}{0.147348in}}{\pgfqpoint{2.735294in}{2.735294in}}%
\pgfusepath{clip}%
\pgfsetbuttcap%
\pgfsetroundjoin%
\definecolor{currentfill}{rgb}{0.089078,0.323920,0.089078}%
\pgfsetfillcolor{currentfill}%
\pgfsetfillopacity{0.200000}%
\pgfsetlinewidth{0.000000pt}%
\definecolor{currentstroke}{rgb}{0.000000,0.000000,0.000000}%
\pgfsetstrokecolor{currentstroke}%
\pgfsetdash{}{0pt}%
\pgfpathmoveto{\pgfqpoint{5.820514in}{1.760124in}}%
\pgfpathlineto{\pgfqpoint{5.569738in}{2.126617in}}%
\pgfpathlineto{\pgfqpoint{5.742447in}{1.557460in}}%
\pgfpathlineto{\pgfqpoint{5.820514in}{1.760124in}}%
\pgfpathclose%
\pgfusepath{fill}%
\end{pgfscope}%
\begin{pgfscope}%
\pgfpathrectangle{\pgfqpoint{3.536584in}{0.147348in}}{\pgfqpoint{2.735294in}{2.735294in}}%
\pgfusepath{clip}%
\pgfsetbuttcap%
\pgfsetroundjoin%
\definecolor{currentfill}{rgb}{0.061754,0.224559,0.061754}%
\pgfsetfillcolor{currentfill}%
\pgfsetfillopacity{0.200000}%
\pgfsetlinewidth{0.000000pt}%
\definecolor{currentstroke}{rgb}{0.000000,0.000000,0.000000}%
\pgfsetstrokecolor{currentstroke}%
\pgfsetdash{}{0pt}%
\pgfpathmoveto{\pgfqpoint{4.314024in}{1.028737in}}%
\pgfpathlineto{\pgfqpoint{4.460139in}{0.888724in}}%
\pgfpathlineto{\pgfqpoint{4.494794in}{1.302943in}}%
\pgfpathlineto{\pgfqpoint{4.314024in}{1.028737in}}%
\pgfpathclose%
\pgfusepath{fill}%
\end{pgfscope}%
\begin{pgfscope}%
\pgfpathrectangle{\pgfqpoint{3.536584in}{0.147348in}}{\pgfqpoint{2.735294in}{2.735294in}}%
\pgfusepath{clip}%
\pgfsetbuttcap%
\pgfsetroundjoin%
\definecolor{currentfill}{rgb}{0.061754,0.224559,0.061754}%
\pgfsetfillcolor{currentfill}%
\pgfsetfillopacity{0.200000}%
\pgfsetlinewidth{0.000000pt}%
\definecolor{currentstroke}{rgb}{0.000000,0.000000,0.000000}%
\pgfsetstrokecolor{currentstroke}%
\pgfsetdash{}{0pt}%
\pgfpathmoveto{\pgfqpoint{5.387596in}{1.302943in}}%
\pgfpathlineto{\pgfqpoint{5.422251in}{0.888724in}}%
\pgfpathlineto{\pgfqpoint{5.568365in}{1.028737in}}%
\pgfpathlineto{\pgfqpoint{5.387596in}{1.302943in}}%
\pgfpathclose%
\pgfusepath{fill}%
\end{pgfscope}%
\begin{pgfscope}%
\pgfpathrectangle{\pgfqpoint{3.536584in}{0.147348in}}{\pgfqpoint{2.735294in}{2.735294in}}%
\pgfusepath{clip}%
\pgfsetbuttcap%
\pgfsetroundjoin%
\definecolor{currentfill}{rgb}{0.070984,0.258123,0.070984}%
\pgfsetfillcolor{currentfill}%
\pgfsetfillopacity{0.200000}%
\pgfsetlinewidth{0.000000pt}%
\definecolor{currentstroke}{rgb}{0.000000,0.000000,0.000000}%
\pgfsetstrokecolor{currentstroke}%
\pgfsetdash{}{0pt}%
\pgfpathmoveto{\pgfqpoint{4.139943in}{1.557460in}}%
\pgfpathlineto{\pgfqpoint{4.130959in}{1.094945in}}%
\pgfpathlineto{\pgfqpoint{4.357383in}{1.526914in}}%
\pgfpathlineto{\pgfqpoint{4.139943in}{1.557460in}}%
\pgfpathclose%
\pgfusepath{fill}%
\end{pgfscope}%
\begin{pgfscope}%
\pgfpathrectangle{\pgfqpoint{3.536584in}{0.147348in}}{\pgfqpoint{2.735294in}{2.735294in}}%
\pgfusepath{clip}%
\pgfsetbuttcap%
\pgfsetroundjoin%
\definecolor{currentfill}{rgb}{0.070984,0.258123,0.070984}%
\pgfsetfillcolor{currentfill}%
\pgfsetfillopacity{0.200000}%
\pgfsetlinewidth{0.000000pt}%
\definecolor{currentstroke}{rgb}{0.000000,0.000000,0.000000}%
\pgfsetstrokecolor{currentstroke}%
\pgfsetdash{}{0pt}%
\pgfpathmoveto{\pgfqpoint{5.525007in}{1.526914in}}%
\pgfpathlineto{\pgfqpoint{5.751431in}{1.094945in}}%
\pgfpathlineto{\pgfqpoint{5.742447in}{1.557460in}}%
\pgfpathlineto{\pgfqpoint{5.525007in}{1.526914in}}%
\pgfpathclose%
\pgfusepath{fill}%
\end{pgfscope}%
\begin{pgfscope}%
\pgfpathrectangle{\pgfqpoint{3.536584in}{0.147348in}}{\pgfqpoint{2.735294in}{2.735294in}}%
\pgfusepath{clip}%
\pgfsetbuttcap%
\pgfsetroundjoin%
\definecolor{currentfill}{rgb}{0.070885,0.257762,0.070885}%
\pgfsetfillcolor{currentfill}%
\pgfsetfillopacity{0.200000}%
\pgfsetlinewidth{0.000000pt}%
\definecolor{currentstroke}{rgb}{0.000000,0.000000,0.000000}%
\pgfsetstrokecolor{currentstroke}%
\pgfsetdash{}{0pt}%
\pgfpathmoveto{\pgfqpoint{5.341356in}{0.974376in}}%
\pgfpathlineto{\pgfqpoint{5.422251in}{0.888724in}}%
\pgfpathlineto{\pgfqpoint{5.387596in}{1.302943in}}%
\pgfpathlineto{\pgfqpoint{5.341356in}{0.974376in}}%
\pgfpathclose%
\pgfusepath{fill}%
\end{pgfscope}%
\begin{pgfscope}%
\pgfpathrectangle{\pgfqpoint{3.536584in}{0.147348in}}{\pgfqpoint{2.735294in}{2.735294in}}%
\pgfusepath{clip}%
\pgfsetbuttcap%
\pgfsetroundjoin%
\definecolor{currentfill}{rgb}{0.070885,0.257762,0.070885}%
\pgfsetfillcolor{currentfill}%
\pgfsetfillopacity{0.200000}%
\pgfsetlinewidth{0.000000pt}%
\definecolor{currentstroke}{rgb}{0.000000,0.000000,0.000000}%
\pgfsetstrokecolor{currentstroke}%
\pgfsetdash{}{0pt}%
\pgfpathmoveto{\pgfqpoint{4.494794in}{1.302943in}}%
\pgfpathlineto{\pgfqpoint{4.460139in}{0.888724in}}%
\pgfpathlineto{\pgfqpoint{4.541034in}{0.974376in}}%
\pgfpathlineto{\pgfqpoint{4.494794in}{1.302943in}}%
\pgfpathclose%
\pgfusepath{fill}%
\end{pgfscope}%
\begin{pgfscope}%
\pgfpathrectangle{\pgfqpoint{3.536584in}{0.147348in}}{\pgfqpoint{2.735294in}{2.735294in}}%
\pgfusepath{clip}%
\pgfsetbuttcap%
\pgfsetroundjoin%
\definecolor{currentfill}{rgb}{0.111651,0.406004,0.111651}%
\pgfsetfillcolor{currentfill}%
\pgfsetfillopacity{0.200000}%
\pgfsetlinewidth{0.000000pt}%
\definecolor{currentstroke}{rgb}{0.000000,0.000000,0.000000}%
\pgfsetstrokecolor{currentstroke}%
\pgfsetdash{}{0pt}%
\pgfpathmoveto{\pgfqpoint{5.342287in}{2.141900in}}%
\pgfpathlineto{\pgfqpoint{5.423597in}{2.292690in}}%
\pgfpathlineto{\pgfqpoint{5.079487in}{2.150689in}}%
\pgfpathlineto{\pgfqpoint{5.342287in}{2.141900in}}%
\pgfpathclose%
\pgfusepath{fill}%
\end{pgfscope}%
\begin{pgfscope}%
\pgfpathrectangle{\pgfqpoint{3.536584in}{0.147348in}}{\pgfqpoint{2.735294in}{2.735294in}}%
\pgfusepath{clip}%
\pgfsetbuttcap%
\pgfsetroundjoin%
\definecolor{currentfill}{rgb}{0.111651,0.406004,0.111651}%
\pgfsetfillcolor{currentfill}%
\pgfsetfillopacity{0.200000}%
\pgfsetlinewidth{0.000000pt}%
\definecolor{currentstroke}{rgb}{0.000000,0.000000,0.000000}%
\pgfsetstrokecolor{currentstroke}%
\pgfsetdash{}{0pt}%
\pgfpathmoveto{\pgfqpoint{4.802903in}{2.150689in}}%
\pgfpathlineto{\pgfqpoint{4.458793in}{2.292690in}}%
\pgfpathlineto{\pgfqpoint{4.540103in}{2.141900in}}%
\pgfpathlineto{\pgfqpoint{4.802903in}{2.150689in}}%
\pgfpathclose%
\pgfusepath{fill}%
\end{pgfscope}%
\begin{pgfscope}%
\pgfpathrectangle{\pgfqpoint{3.536584in}{0.147348in}}{\pgfqpoint{2.735294in}{2.735294in}}%
\pgfusepath{clip}%
\pgfsetbuttcap%
\pgfsetroundjoin%
\definecolor{currentfill}{rgb}{0.092193,0.335248,0.092193}%
\pgfsetfillcolor{currentfill}%
\pgfsetfillopacity{0.200000}%
\pgfsetlinewidth{0.000000pt}%
\definecolor{currentstroke}{rgb}{0.000000,0.000000,0.000000}%
\pgfsetstrokecolor{currentstroke}%
\pgfsetdash{}{0pt}%
\pgfpathmoveto{\pgfqpoint{4.357383in}{1.526914in}}%
\pgfpathlineto{\pgfqpoint{4.312652in}{2.126617in}}%
\pgfpathlineto{\pgfqpoint{4.139943in}{1.557460in}}%
\pgfpathlineto{\pgfqpoint{4.357383in}{1.526914in}}%
\pgfpathclose%
\pgfusepath{fill}%
\end{pgfscope}%
\begin{pgfscope}%
\pgfpathrectangle{\pgfqpoint{3.536584in}{0.147348in}}{\pgfqpoint{2.735294in}{2.735294in}}%
\pgfusepath{clip}%
\pgfsetbuttcap%
\pgfsetroundjoin%
\definecolor{currentfill}{rgb}{0.092193,0.335248,0.092193}%
\pgfsetfillcolor{currentfill}%
\pgfsetfillopacity{0.200000}%
\pgfsetlinewidth{0.000000pt}%
\definecolor{currentstroke}{rgb}{0.000000,0.000000,0.000000}%
\pgfsetstrokecolor{currentstroke}%
\pgfsetdash{}{0pt}%
\pgfpathmoveto{\pgfqpoint{5.742447in}{1.557460in}}%
\pgfpathlineto{\pgfqpoint{5.569738in}{2.126617in}}%
\pgfpathlineto{\pgfqpoint{5.525007in}{1.526914in}}%
\pgfpathlineto{\pgfqpoint{5.742447in}{1.557460in}}%
\pgfpathclose%
\pgfusepath{fill}%
\end{pgfscope}%
\begin{pgfscope}%
\pgfpathrectangle{\pgfqpoint{3.536584in}{0.147348in}}{\pgfqpoint{2.735294in}{2.735294in}}%
\pgfusepath{clip}%
\pgfsetbuttcap%
\pgfsetroundjoin%
\definecolor{currentfill}{rgb}{0.060562,0.220227,0.060562}%
\pgfsetfillcolor{currentfill}%
\pgfsetfillopacity{0.200000}%
\pgfsetlinewidth{0.000000pt}%
\definecolor{currentstroke}{rgb}{0.000000,0.000000,0.000000}%
\pgfsetstrokecolor{currentstroke}%
\pgfsetdash{}{0pt}%
\pgfpathmoveto{\pgfqpoint{5.341356in}{0.974376in}}%
\pgfpathlineto{\pgfqpoint{5.096361in}{1.277996in}}%
\pgfpathlineto{\pgfqpoint{5.191972in}{0.846223in}}%
\pgfpathlineto{\pgfqpoint{5.341356in}{0.974376in}}%
\pgfpathclose%
\pgfusepath{fill}%
\end{pgfscope}%
\begin{pgfscope}%
\pgfpathrectangle{\pgfqpoint{3.536584in}{0.147348in}}{\pgfqpoint{2.735294in}{2.735294in}}%
\pgfusepath{clip}%
\pgfsetbuttcap%
\pgfsetroundjoin%
\definecolor{currentfill}{rgb}{0.060562,0.220227,0.060562}%
\pgfsetfillcolor{currentfill}%
\pgfsetfillopacity{0.200000}%
\pgfsetlinewidth{0.000000pt}%
\definecolor{currentstroke}{rgb}{0.000000,0.000000,0.000000}%
\pgfsetstrokecolor{currentstroke}%
\pgfsetdash{}{0pt}%
\pgfpathmoveto{\pgfqpoint{4.690418in}{0.846223in}}%
\pgfpathlineto{\pgfqpoint{4.786029in}{1.277996in}}%
\pgfpathlineto{\pgfqpoint{4.541034in}{0.974376in}}%
\pgfpathlineto{\pgfqpoint{4.690418in}{0.846223in}}%
\pgfpathclose%
\pgfusepath{fill}%
\end{pgfscope}%
\begin{pgfscope}%
\pgfpathrectangle{\pgfqpoint{3.536584in}{0.147348in}}{\pgfqpoint{2.735294in}{2.735294in}}%
\pgfusepath{clip}%
\pgfsetbuttcap%
\pgfsetroundjoin%
\definecolor{currentfill}{rgb}{0.097285,0.353762,0.097285}%
\pgfsetfillcolor{currentfill}%
\pgfsetfillopacity{0.200000}%
\pgfsetlinewidth{0.000000pt}%
\definecolor{currentstroke}{rgb}{0.000000,0.000000,0.000000}%
\pgfsetstrokecolor{currentstroke}%
\pgfsetdash{}{0pt}%
\pgfpathmoveto{\pgfqpoint{4.540103in}{2.141900in}}%
\pgfpathlineto{\pgfqpoint{4.312652in}{2.126617in}}%
\pgfpathlineto{\pgfqpoint{4.649042in}{1.952591in}}%
\pgfpathlineto{\pgfqpoint{4.540103in}{2.141900in}}%
\pgfpathclose%
\pgfusepath{fill}%
\end{pgfscope}%
\begin{pgfscope}%
\pgfpathrectangle{\pgfqpoint{3.536584in}{0.147348in}}{\pgfqpoint{2.735294in}{2.735294in}}%
\pgfusepath{clip}%
\pgfsetbuttcap%
\pgfsetroundjoin%
\definecolor{currentfill}{rgb}{0.097285,0.353762,0.097285}%
\pgfsetfillcolor{currentfill}%
\pgfsetfillopacity{0.200000}%
\pgfsetlinewidth{0.000000pt}%
\definecolor{currentstroke}{rgb}{0.000000,0.000000,0.000000}%
\pgfsetstrokecolor{currentstroke}%
\pgfsetdash{}{0pt}%
\pgfpathmoveto{\pgfqpoint{5.233348in}{1.952591in}}%
\pgfpathlineto{\pgfqpoint{5.569738in}{2.126617in}}%
\pgfpathlineto{\pgfqpoint{5.342287in}{2.141900in}}%
\pgfpathlineto{\pgfqpoint{5.233348in}{1.952591in}}%
\pgfpathclose%
\pgfusepath{fill}%
\end{pgfscope}%
\begin{pgfscope}%
\pgfpathrectangle{\pgfqpoint{3.536584in}{0.147348in}}{\pgfqpoint{2.735294in}{2.735294in}}%
\pgfusepath{clip}%
\pgfsetbuttcap%
\pgfsetroundjoin%
\definecolor{currentfill}{rgb}{0.067497,0.245443,0.067497}%
\pgfsetfillcolor{currentfill}%
\pgfsetfillopacity{0.200000}%
\pgfsetlinewidth{0.000000pt}%
\definecolor{currentstroke}{rgb}{0.000000,0.000000,0.000000}%
\pgfsetstrokecolor{currentstroke}%
\pgfsetdash{}{0pt}%
\pgfpathmoveto{\pgfqpoint{4.690418in}{0.846223in}}%
\pgfpathlineto{\pgfqpoint{4.803235in}{0.943120in}}%
\pgfpathlineto{\pgfqpoint{4.786029in}{1.277996in}}%
\pgfpathlineto{\pgfqpoint{4.690418in}{0.846223in}}%
\pgfpathclose%
\pgfusepath{fill}%
\end{pgfscope}%
\begin{pgfscope}%
\pgfpathrectangle{\pgfqpoint{3.536584in}{0.147348in}}{\pgfqpoint{2.735294in}{2.735294in}}%
\pgfusepath{clip}%
\pgfsetbuttcap%
\pgfsetroundjoin%
\definecolor{currentfill}{rgb}{0.067497,0.245443,0.067497}%
\pgfsetfillcolor{currentfill}%
\pgfsetfillopacity{0.200000}%
\pgfsetlinewidth{0.000000pt}%
\definecolor{currentstroke}{rgb}{0.000000,0.000000,0.000000}%
\pgfsetstrokecolor{currentstroke}%
\pgfsetdash{}{0pt}%
\pgfpathmoveto{\pgfqpoint{5.096361in}{1.277996in}}%
\pgfpathlineto{\pgfqpoint{5.079154in}{0.943120in}}%
\pgfpathlineto{\pgfqpoint{5.191972in}{0.846223in}}%
\pgfpathlineto{\pgfqpoint{5.096361in}{1.277996in}}%
\pgfpathclose%
\pgfusepath{fill}%
\end{pgfscope}%
\begin{pgfscope}%
\pgfpathrectangle{\pgfqpoint{3.536584in}{0.147348in}}{\pgfqpoint{2.735294in}{2.735294in}}%
\pgfusepath{clip}%
\pgfsetbuttcap%
\pgfsetroundjoin%
\definecolor{currentfill}{rgb}{0.065434,0.237940,0.065434}%
\pgfsetfillcolor{currentfill}%
\pgfsetfillopacity{0.200000}%
\pgfsetlinewidth{0.000000pt}%
\definecolor{currentstroke}{rgb}{0.000000,0.000000,0.000000}%
\pgfsetstrokecolor{currentstroke}%
\pgfsetdash{}{0pt}%
\pgfpathmoveto{\pgfqpoint{4.941195in}{0.830831in}}%
\pgfpathlineto{\pgfqpoint{4.786029in}{1.277996in}}%
\pgfpathlineto{\pgfqpoint{4.803235in}{0.943120in}}%
\pgfpathlineto{\pgfqpoint{4.941195in}{0.830831in}}%
\pgfpathclose%
\pgfusepath{fill}%
\end{pgfscope}%
\begin{pgfscope}%
\pgfpathrectangle{\pgfqpoint{3.536584in}{0.147348in}}{\pgfqpoint{2.735294in}{2.735294in}}%
\pgfusepath{clip}%
\pgfsetbuttcap%
\pgfsetroundjoin%
\definecolor{currentfill}{rgb}{0.065434,0.237940,0.065434}%
\pgfsetfillcolor{currentfill}%
\pgfsetfillopacity{0.200000}%
\pgfsetlinewidth{0.000000pt}%
\definecolor{currentstroke}{rgb}{0.000000,0.000000,0.000000}%
\pgfsetstrokecolor{currentstroke}%
\pgfsetdash{}{0pt}%
\pgfpathmoveto{\pgfqpoint{5.079154in}{0.943120in}}%
\pgfpathlineto{\pgfqpoint{5.096361in}{1.277996in}}%
\pgfpathlineto{\pgfqpoint{4.941195in}{0.830831in}}%
\pgfpathlineto{\pgfqpoint{5.079154in}{0.943120in}}%
\pgfpathclose%
\pgfusepath{fill}%
\end{pgfscope}%
\begin{pgfscope}%
\pgfpathrectangle{\pgfqpoint{3.536584in}{0.147348in}}{\pgfqpoint{2.735294in}{2.735294in}}%
\pgfusepath{clip}%
\pgfsetbuttcap%
\pgfsetroundjoin%
\definecolor{currentfill}{rgb}{0.073593,0.267612,0.073593}%
\pgfsetfillcolor{currentfill}%
\pgfsetfillopacity{0.200000}%
\pgfsetlinewidth{0.000000pt}%
\definecolor{currentstroke}{rgb}{0.000000,0.000000,0.000000}%
\pgfsetstrokecolor{currentstroke}%
\pgfsetdash{}{0pt}%
\pgfpathmoveto{\pgfqpoint{4.314024in}{1.028737in}}%
\pgfpathlineto{\pgfqpoint{4.494794in}{1.302943in}}%
\pgfpathlineto{\pgfqpoint{4.357383in}{1.526914in}}%
\pgfpathlineto{\pgfqpoint{4.314024in}{1.028737in}}%
\pgfpathclose%
\pgfusepath{fill}%
\end{pgfscope}%
\begin{pgfscope}%
\pgfpathrectangle{\pgfqpoint{3.536584in}{0.147348in}}{\pgfqpoint{2.735294in}{2.735294in}}%
\pgfusepath{clip}%
\pgfsetbuttcap%
\pgfsetroundjoin%
\definecolor{currentfill}{rgb}{0.073593,0.267612,0.073593}%
\pgfsetfillcolor{currentfill}%
\pgfsetfillopacity{0.200000}%
\pgfsetlinewidth{0.000000pt}%
\definecolor{currentstroke}{rgb}{0.000000,0.000000,0.000000}%
\pgfsetstrokecolor{currentstroke}%
\pgfsetdash{}{0pt}%
\pgfpathmoveto{\pgfqpoint{5.525007in}{1.526914in}}%
\pgfpathlineto{\pgfqpoint{5.387596in}{1.302943in}}%
\pgfpathlineto{\pgfqpoint{5.568365in}{1.028737in}}%
\pgfpathlineto{\pgfqpoint{5.525007in}{1.526914in}}%
\pgfpathclose%
\pgfusepath{fill}%
\end{pgfscope}%
\begin{pgfscope}%
\pgfpathrectangle{\pgfqpoint{3.536584in}{0.147348in}}{\pgfqpoint{2.735294in}{2.735294in}}%
\pgfusepath{clip}%
\pgfsetbuttcap%
\pgfsetroundjoin%
\definecolor{currentfill}{rgb}{0.091915,0.334238,0.091915}%
\pgfsetfillcolor{currentfill}%
\pgfsetfillopacity{0.200000}%
\pgfsetlinewidth{0.000000pt}%
\definecolor{currentstroke}{rgb}{0.000000,0.000000,0.000000}%
\pgfsetstrokecolor{currentstroke}%
\pgfsetdash{}{0pt}%
\pgfpathmoveto{\pgfqpoint{4.357383in}{1.526914in}}%
\pgfpathlineto{\pgfqpoint{4.649042in}{1.952591in}}%
\pgfpathlineto{\pgfqpoint{4.312652in}{2.126617in}}%
\pgfpathlineto{\pgfqpoint{4.357383in}{1.526914in}}%
\pgfpathclose%
\pgfusepath{fill}%
\end{pgfscope}%
\begin{pgfscope}%
\pgfpathrectangle{\pgfqpoint{3.536584in}{0.147348in}}{\pgfqpoint{2.735294in}{2.735294in}}%
\pgfusepath{clip}%
\pgfsetbuttcap%
\pgfsetroundjoin%
\definecolor{currentfill}{rgb}{0.091915,0.334238,0.091915}%
\pgfsetfillcolor{currentfill}%
\pgfsetfillopacity{0.200000}%
\pgfsetlinewidth{0.000000pt}%
\definecolor{currentstroke}{rgb}{0.000000,0.000000,0.000000}%
\pgfsetstrokecolor{currentstroke}%
\pgfsetdash{}{0pt}%
\pgfpathmoveto{\pgfqpoint{5.569738in}{2.126617in}}%
\pgfpathlineto{\pgfqpoint{5.233348in}{1.952591in}}%
\pgfpathlineto{\pgfqpoint{5.525007in}{1.526914in}}%
\pgfpathlineto{\pgfqpoint{5.569738in}{2.126617in}}%
\pgfpathclose%
\pgfusepath{fill}%
\end{pgfscope}%
\begin{pgfscope}%
\pgfpathrectangle{\pgfqpoint{3.536584in}{0.147348in}}{\pgfqpoint{2.735294in}{2.735294in}}%
\pgfusepath{clip}%
\pgfsetbuttcap%
\pgfsetroundjoin%
\definecolor{currentfill}{rgb}{0.101759,0.370033,0.101759}%
\pgfsetfillcolor{currentfill}%
\pgfsetfillopacity{0.200000}%
\pgfsetlinewidth{0.000000pt}%
\definecolor{currentstroke}{rgb}{0.000000,0.000000,0.000000}%
\pgfsetstrokecolor{currentstroke}%
\pgfsetdash{}{0pt}%
\pgfpathmoveto{\pgfqpoint{4.649042in}{1.952591in}}%
\pgfpathlineto{\pgfqpoint{4.802903in}{2.150689in}}%
\pgfpathlineto{\pgfqpoint{4.540103in}{2.141900in}}%
\pgfpathlineto{\pgfqpoint{4.649042in}{1.952591in}}%
\pgfpathclose%
\pgfusepath{fill}%
\end{pgfscope}%
\begin{pgfscope}%
\pgfpathrectangle{\pgfqpoint{3.536584in}{0.147348in}}{\pgfqpoint{2.735294in}{2.735294in}}%
\pgfusepath{clip}%
\pgfsetbuttcap%
\pgfsetroundjoin%
\definecolor{currentfill}{rgb}{0.101759,0.370033,0.101759}%
\pgfsetfillcolor{currentfill}%
\pgfsetfillopacity{0.200000}%
\pgfsetlinewidth{0.000000pt}%
\definecolor{currentstroke}{rgb}{0.000000,0.000000,0.000000}%
\pgfsetstrokecolor{currentstroke}%
\pgfsetdash{}{0pt}%
\pgfpathmoveto{\pgfqpoint{5.342287in}{2.141900in}}%
\pgfpathlineto{\pgfqpoint{5.079487in}{2.150689in}}%
\pgfpathlineto{\pgfqpoint{5.233348in}{1.952591in}}%
\pgfpathlineto{\pgfqpoint{5.342287in}{2.141900in}}%
\pgfpathclose%
\pgfusepath{fill}%
\end{pgfscope}%
\begin{pgfscope}%
\pgfpathrectangle{\pgfqpoint{3.536584in}{0.147348in}}{\pgfqpoint{2.735294in}{2.735294in}}%
\pgfusepath{clip}%
\pgfsetbuttcap%
\pgfsetroundjoin%
\definecolor{currentfill}{rgb}{0.101677,0.369734,0.101677}%
\pgfsetfillcolor{currentfill}%
\pgfsetfillopacity{0.200000}%
\pgfsetlinewidth{0.000000pt}%
\definecolor{currentstroke}{rgb}{0.000000,0.000000,0.000000}%
\pgfsetstrokecolor{currentstroke}%
\pgfsetdash{}{0pt}%
\pgfpathmoveto{\pgfqpoint{4.941195in}{1.953918in}}%
\pgfpathlineto{\pgfqpoint{5.079487in}{2.150689in}}%
\pgfpathlineto{\pgfqpoint{4.802903in}{2.150689in}}%
\pgfpathlineto{\pgfqpoint{4.941195in}{1.953918in}}%
\pgfpathclose%
\pgfusepath{fill}%
\end{pgfscope}%
\begin{pgfscope}%
\pgfpathrectangle{\pgfqpoint{3.536584in}{0.147348in}}{\pgfqpoint{2.735294in}{2.735294in}}%
\pgfusepath{clip}%
\pgfsetbuttcap%
\pgfsetroundjoin%
\definecolor{currentfill}{rgb}{0.065035,0.236492,0.065035}%
\pgfsetfillcolor{currentfill}%
\pgfsetfillopacity{0.200000}%
\pgfsetlinewidth{0.000000pt}%
\definecolor{currentstroke}{rgb}{0.000000,0.000000,0.000000}%
\pgfsetstrokecolor{currentstroke}%
\pgfsetdash{}{0pt}%
\pgfpathmoveto{\pgfqpoint{4.541034in}{0.974376in}}%
\pgfpathlineto{\pgfqpoint{4.630724in}{1.504068in}}%
\pgfpathlineto{\pgfqpoint{4.494794in}{1.302943in}}%
\pgfpathlineto{\pgfqpoint{4.541034in}{0.974376in}}%
\pgfpathclose%
\pgfusepath{fill}%
\end{pgfscope}%
\begin{pgfscope}%
\pgfpathrectangle{\pgfqpoint{3.536584in}{0.147348in}}{\pgfqpoint{2.735294in}{2.735294in}}%
\pgfusepath{clip}%
\pgfsetbuttcap%
\pgfsetroundjoin%
\definecolor{currentfill}{rgb}{0.065035,0.236492,0.065035}%
\pgfsetfillcolor{currentfill}%
\pgfsetfillopacity{0.200000}%
\pgfsetlinewidth{0.000000pt}%
\definecolor{currentstroke}{rgb}{0.000000,0.000000,0.000000}%
\pgfsetstrokecolor{currentstroke}%
\pgfsetdash{}{0pt}%
\pgfpathmoveto{\pgfqpoint{5.387596in}{1.302943in}}%
\pgfpathlineto{\pgfqpoint{5.251666in}{1.504068in}}%
\pgfpathlineto{\pgfqpoint{5.341356in}{0.974376in}}%
\pgfpathlineto{\pgfqpoint{5.387596in}{1.302943in}}%
\pgfpathclose%
\pgfusepath{fill}%
\end{pgfscope}%
\begin{pgfscope}%
\pgfpathrectangle{\pgfqpoint{3.536584in}{0.147348in}}{\pgfqpoint{2.735294in}{2.735294in}}%
\pgfusepath{clip}%
\pgfsetbuttcap%
\pgfsetroundjoin%
\definecolor{currentfill}{rgb}{0.066446,0.241622,0.066446}%
\pgfsetfillcolor{currentfill}%
\pgfsetfillopacity{0.200000}%
\pgfsetlinewidth{0.000000pt}%
\definecolor{currentstroke}{rgb}{0.000000,0.000000,0.000000}%
\pgfsetstrokecolor{currentstroke}%
\pgfsetdash{}{0pt}%
\pgfpathmoveto{\pgfqpoint{4.786029in}{1.277996in}}%
\pgfpathlineto{\pgfqpoint{4.941195in}{0.830831in}}%
\pgfpathlineto{\pgfqpoint{4.941195in}{1.495457in}}%
\pgfpathlineto{\pgfqpoint{4.786029in}{1.277996in}}%
\pgfpathclose%
\pgfusepath{fill}%
\end{pgfscope}%
\begin{pgfscope}%
\pgfpathrectangle{\pgfqpoint{3.536584in}{0.147348in}}{\pgfqpoint{2.735294in}{2.735294in}}%
\pgfusepath{clip}%
\pgfsetbuttcap%
\pgfsetroundjoin%
\definecolor{currentfill}{rgb}{0.066446,0.241622,0.066446}%
\pgfsetfillcolor{currentfill}%
\pgfsetfillopacity{0.200000}%
\pgfsetlinewidth{0.000000pt}%
\definecolor{currentstroke}{rgb}{0.000000,0.000000,0.000000}%
\pgfsetstrokecolor{currentstroke}%
\pgfsetdash{}{0pt}%
\pgfpathmoveto{\pgfqpoint{4.941195in}{1.495457in}}%
\pgfpathlineto{\pgfqpoint{4.941195in}{0.830831in}}%
\pgfpathlineto{\pgfqpoint{5.096361in}{1.277996in}}%
\pgfpathlineto{\pgfqpoint{4.941195in}{1.495457in}}%
\pgfpathclose%
\pgfusepath{fill}%
\end{pgfscope}%
\begin{pgfscope}%
\pgfpathrectangle{\pgfqpoint{3.536584in}{0.147348in}}{\pgfqpoint{2.735294in}{2.735294in}}%
\pgfusepath{clip}%
\pgfsetbuttcap%
\pgfsetroundjoin%
\definecolor{currentfill}{rgb}{0.098306,0.357475,0.098306}%
\pgfsetfillcolor{currentfill}%
\pgfsetfillopacity{0.200000}%
\pgfsetlinewidth{0.000000pt}%
\definecolor{currentstroke}{rgb}{0.000000,0.000000,0.000000}%
\pgfsetstrokecolor{currentstroke}%
\pgfsetdash{}{0pt}%
\pgfpathmoveto{\pgfqpoint{4.941195in}{1.953918in}}%
\pgfpathlineto{\pgfqpoint{4.802903in}{2.150689in}}%
\pgfpathlineto{\pgfqpoint{4.649042in}{1.952591in}}%
\pgfpathlineto{\pgfqpoint{4.941195in}{1.953918in}}%
\pgfpathclose%
\pgfusepath{fill}%
\end{pgfscope}%
\begin{pgfscope}%
\pgfpathrectangle{\pgfqpoint{3.536584in}{0.147348in}}{\pgfqpoint{2.735294in}{2.735294in}}%
\pgfusepath{clip}%
\pgfsetbuttcap%
\pgfsetroundjoin%
\definecolor{currentfill}{rgb}{0.098306,0.357475,0.098306}%
\pgfsetfillcolor{currentfill}%
\pgfsetfillopacity{0.200000}%
\pgfsetlinewidth{0.000000pt}%
\definecolor{currentstroke}{rgb}{0.000000,0.000000,0.000000}%
\pgfsetstrokecolor{currentstroke}%
\pgfsetdash{}{0pt}%
\pgfpathmoveto{\pgfqpoint{5.233348in}{1.952591in}}%
\pgfpathlineto{\pgfqpoint{5.079487in}{2.150689in}}%
\pgfpathlineto{\pgfqpoint{4.941195in}{1.953918in}}%
\pgfpathlineto{\pgfqpoint{5.233348in}{1.952591in}}%
\pgfpathclose%
\pgfusepath{fill}%
\end{pgfscope}%
\begin{pgfscope}%
\pgfpathrectangle{\pgfqpoint{3.536584in}{0.147348in}}{\pgfqpoint{2.735294in}{2.735294in}}%
\pgfusepath{clip}%
\pgfsetbuttcap%
\pgfsetroundjoin%
\definecolor{currentfill}{rgb}{0.070209,0.255305,0.070209}%
\pgfsetfillcolor{currentfill}%
\pgfsetfillopacity{0.200000}%
\pgfsetlinewidth{0.000000pt}%
\definecolor{currentstroke}{rgb}{0.000000,0.000000,0.000000}%
\pgfsetstrokecolor{currentstroke}%
\pgfsetdash{}{0pt}%
\pgfpathmoveto{\pgfqpoint{4.541034in}{0.974376in}}%
\pgfpathlineto{\pgfqpoint{4.786029in}{1.277996in}}%
\pgfpathlineto{\pgfqpoint{4.630724in}{1.504068in}}%
\pgfpathlineto{\pgfqpoint{4.541034in}{0.974376in}}%
\pgfpathclose%
\pgfusepath{fill}%
\end{pgfscope}%
\begin{pgfscope}%
\pgfpathrectangle{\pgfqpoint{3.536584in}{0.147348in}}{\pgfqpoint{2.735294in}{2.735294in}}%
\pgfusepath{clip}%
\pgfsetbuttcap%
\pgfsetroundjoin%
\definecolor{currentfill}{rgb}{0.070209,0.255305,0.070209}%
\pgfsetfillcolor{currentfill}%
\pgfsetfillopacity{0.200000}%
\pgfsetlinewidth{0.000000pt}%
\definecolor{currentstroke}{rgb}{0.000000,0.000000,0.000000}%
\pgfsetstrokecolor{currentstroke}%
\pgfsetdash{}{0pt}%
\pgfpathmoveto{\pgfqpoint{5.251666in}{1.504068in}}%
\pgfpathlineto{\pgfqpoint{5.096361in}{1.277996in}}%
\pgfpathlineto{\pgfqpoint{5.341356in}{0.974376in}}%
\pgfpathlineto{\pgfqpoint{5.251666in}{1.504068in}}%
\pgfpathclose%
\pgfusepath{fill}%
\end{pgfscope}%
\begin{pgfscope}%
\pgfpathrectangle{\pgfqpoint{3.536584in}{0.147348in}}{\pgfqpoint{2.735294in}{2.735294in}}%
\pgfusepath{clip}%
\pgfsetbuttcap%
\pgfsetroundjoin%
\definecolor{currentfill}{rgb}{0.087398,0.317812,0.087398}%
\pgfsetfillcolor{currentfill}%
\pgfsetfillopacity{0.200000}%
\pgfsetlinewidth{0.000000pt}%
\definecolor{currentstroke}{rgb}{0.000000,0.000000,0.000000}%
\pgfsetstrokecolor{currentstroke}%
\pgfsetdash{}{0pt}%
\pgfpathmoveto{\pgfqpoint{5.525007in}{1.526914in}}%
\pgfpathlineto{\pgfqpoint{5.233348in}{1.952591in}}%
\pgfpathlineto{\pgfqpoint{5.251666in}{1.504068in}}%
\pgfpathlineto{\pgfqpoint{5.525007in}{1.526914in}}%
\pgfpathclose%
\pgfusepath{fill}%
\end{pgfscope}%
\begin{pgfscope}%
\pgfpathrectangle{\pgfqpoint{3.536584in}{0.147348in}}{\pgfqpoint{2.735294in}{2.735294in}}%
\pgfusepath{clip}%
\pgfsetbuttcap%
\pgfsetroundjoin%
\definecolor{currentfill}{rgb}{0.087398,0.317812,0.087398}%
\pgfsetfillcolor{currentfill}%
\pgfsetfillopacity{0.200000}%
\pgfsetlinewidth{0.000000pt}%
\definecolor{currentstroke}{rgb}{0.000000,0.000000,0.000000}%
\pgfsetstrokecolor{currentstroke}%
\pgfsetdash{}{0pt}%
\pgfpathmoveto{\pgfqpoint{4.630724in}{1.504068in}}%
\pgfpathlineto{\pgfqpoint{4.649042in}{1.952591in}}%
\pgfpathlineto{\pgfqpoint{4.357383in}{1.526914in}}%
\pgfpathlineto{\pgfqpoint{4.630724in}{1.504068in}}%
\pgfpathclose%
\pgfusepath{fill}%
\end{pgfscope}%
\begin{pgfscope}%
\pgfpathrectangle{\pgfqpoint{3.536584in}{0.147348in}}{\pgfqpoint{2.735294in}{2.735294in}}%
\pgfusepath{clip}%
\pgfsetbuttcap%
\pgfsetroundjoin%
\definecolor{currentfill}{rgb}{0.075994,0.276341,0.075994}%
\pgfsetfillcolor{currentfill}%
\pgfsetfillopacity{0.200000}%
\pgfsetlinewidth{0.000000pt}%
\definecolor{currentstroke}{rgb}{0.000000,0.000000,0.000000}%
\pgfsetstrokecolor{currentstroke}%
\pgfsetdash{}{0pt}%
\pgfpathmoveto{\pgfqpoint{4.357383in}{1.526914in}}%
\pgfpathlineto{\pgfqpoint{4.494794in}{1.302943in}}%
\pgfpathlineto{\pgfqpoint{4.630724in}{1.504068in}}%
\pgfpathlineto{\pgfqpoint{4.357383in}{1.526914in}}%
\pgfpathclose%
\pgfusepath{fill}%
\end{pgfscope}%
\begin{pgfscope}%
\pgfpathrectangle{\pgfqpoint{3.536584in}{0.147348in}}{\pgfqpoint{2.735294in}{2.735294in}}%
\pgfusepath{clip}%
\pgfsetbuttcap%
\pgfsetroundjoin%
\definecolor{currentfill}{rgb}{0.075994,0.276341,0.075994}%
\pgfsetfillcolor{currentfill}%
\pgfsetfillopacity{0.200000}%
\pgfsetlinewidth{0.000000pt}%
\definecolor{currentstroke}{rgb}{0.000000,0.000000,0.000000}%
\pgfsetstrokecolor{currentstroke}%
\pgfsetdash{}{0pt}%
\pgfpathmoveto{\pgfqpoint{5.251666in}{1.504068in}}%
\pgfpathlineto{\pgfqpoint{5.387596in}{1.302943in}}%
\pgfpathlineto{\pgfqpoint{5.525007in}{1.526914in}}%
\pgfpathlineto{\pgfqpoint{5.251666in}{1.504068in}}%
\pgfpathclose%
\pgfusepath{fill}%
\end{pgfscope}%
\begin{pgfscope}%
\pgfpathrectangle{\pgfqpoint{3.536584in}{0.147348in}}{\pgfqpoint{2.735294in}{2.735294in}}%
\pgfusepath{clip}%
\pgfsetbuttcap%
\pgfsetroundjoin%
\definecolor{currentfill}{rgb}{0.086061,0.312950,0.086061}%
\pgfsetfillcolor{currentfill}%
\pgfsetfillopacity{0.200000}%
\pgfsetlinewidth{0.000000pt}%
\definecolor{currentstroke}{rgb}{0.000000,0.000000,0.000000}%
\pgfsetstrokecolor{currentstroke}%
\pgfsetdash{}{0pt}%
\pgfpathmoveto{\pgfqpoint{4.649042in}{1.952591in}}%
\pgfpathlineto{\pgfqpoint{4.630724in}{1.504068in}}%
\pgfpathlineto{\pgfqpoint{4.941195in}{1.953918in}}%
\pgfpathlineto{\pgfqpoint{4.649042in}{1.952591in}}%
\pgfpathclose%
\pgfusepath{fill}%
\end{pgfscope}%
\begin{pgfscope}%
\pgfpathrectangle{\pgfqpoint{3.536584in}{0.147348in}}{\pgfqpoint{2.735294in}{2.735294in}}%
\pgfusepath{clip}%
\pgfsetbuttcap%
\pgfsetroundjoin%
\definecolor{currentfill}{rgb}{0.086061,0.312950,0.086061}%
\pgfsetfillcolor{currentfill}%
\pgfsetfillopacity{0.200000}%
\pgfsetlinewidth{0.000000pt}%
\definecolor{currentstroke}{rgb}{0.000000,0.000000,0.000000}%
\pgfsetstrokecolor{currentstroke}%
\pgfsetdash{}{0pt}%
\pgfpathmoveto{\pgfqpoint{4.941195in}{1.953918in}}%
\pgfpathlineto{\pgfqpoint{5.251666in}{1.504068in}}%
\pgfpathlineto{\pgfqpoint{5.233348in}{1.952591in}}%
\pgfpathlineto{\pgfqpoint{4.941195in}{1.953918in}}%
\pgfpathclose%
\pgfusepath{fill}%
\end{pgfscope}%
\begin{pgfscope}%
\pgfpathrectangle{\pgfqpoint{3.536584in}{0.147348in}}{\pgfqpoint{2.735294in}{2.735294in}}%
\pgfusepath{clip}%
\pgfsetbuttcap%
\pgfsetroundjoin%
\definecolor{currentfill}{rgb}{0.086258,0.313666,0.086258}%
\pgfsetfillcolor{currentfill}%
\pgfsetfillopacity{0.200000}%
\pgfsetlinewidth{0.000000pt}%
\definecolor{currentstroke}{rgb}{0.000000,0.000000,0.000000}%
\pgfsetstrokecolor{currentstroke}%
\pgfsetdash{}{0pt}%
\pgfpathmoveto{\pgfqpoint{4.941195in}{1.495457in}}%
\pgfpathlineto{\pgfqpoint{4.941195in}{1.953918in}}%
\pgfpathlineto{\pgfqpoint{4.630724in}{1.504068in}}%
\pgfpathlineto{\pgfqpoint{4.941195in}{1.495457in}}%
\pgfpathclose%
\pgfusepath{fill}%
\end{pgfscope}%
\begin{pgfscope}%
\pgfpathrectangle{\pgfqpoint{3.536584in}{0.147348in}}{\pgfqpoint{2.735294in}{2.735294in}}%
\pgfusepath{clip}%
\pgfsetbuttcap%
\pgfsetroundjoin%
\definecolor{currentfill}{rgb}{0.086258,0.313666,0.086258}%
\pgfsetfillcolor{currentfill}%
\pgfsetfillopacity{0.200000}%
\pgfsetlinewidth{0.000000pt}%
\definecolor{currentstroke}{rgb}{0.000000,0.000000,0.000000}%
\pgfsetstrokecolor{currentstroke}%
\pgfsetdash{}{0pt}%
\pgfpathmoveto{\pgfqpoint{5.251666in}{1.504068in}}%
\pgfpathlineto{\pgfqpoint{4.941195in}{1.953918in}}%
\pgfpathlineto{\pgfqpoint{4.941195in}{1.495457in}}%
\pgfpathlineto{\pgfqpoint{5.251666in}{1.504068in}}%
\pgfpathclose%
\pgfusepath{fill}%
\end{pgfscope}%
\begin{pgfscope}%
\pgfpathrectangle{\pgfqpoint{3.536584in}{0.147348in}}{\pgfqpoint{2.735294in}{2.735294in}}%
\pgfusepath{clip}%
\pgfsetbuttcap%
\pgfsetroundjoin%
\definecolor{currentfill}{rgb}{0.074668,0.271519,0.074668}%
\pgfsetfillcolor{currentfill}%
\pgfsetfillopacity{0.200000}%
\pgfsetlinewidth{0.000000pt}%
\definecolor{currentstroke}{rgb}{0.000000,0.000000,0.000000}%
\pgfsetstrokecolor{currentstroke}%
\pgfsetdash{}{0pt}%
\pgfpathmoveto{\pgfqpoint{4.630724in}{1.504068in}}%
\pgfpathlineto{\pgfqpoint{4.786029in}{1.277996in}}%
\pgfpathlineto{\pgfqpoint{4.941195in}{1.495457in}}%
\pgfpathlineto{\pgfqpoint{4.630724in}{1.504068in}}%
\pgfpathclose%
\pgfusepath{fill}%
\end{pgfscope}%
\begin{pgfscope}%
\pgfpathrectangle{\pgfqpoint{3.536584in}{0.147348in}}{\pgfqpoint{2.735294in}{2.735294in}}%
\pgfusepath{clip}%
\pgfsetbuttcap%
\pgfsetroundjoin%
\definecolor{currentfill}{rgb}{0.074668,0.271519,0.074668}%
\pgfsetfillcolor{currentfill}%
\pgfsetfillopacity{0.200000}%
\pgfsetlinewidth{0.000000pt}%
\definecolor{currentstroke}{rgb}{0.000000,0.000000,0.000000}%
\pgfsetstrokecolor{currentstroke}%
\pgfsetdash{}{0pt}%
\pgfpathmoveto{\pgfqpoint{4.941195in}{1.495457in}}%
\pgfpathlineto{\pgfqpoint{5.096361in}{1.277996in}}%
\pgfpathlineto{\pgfqpoint{5.251666in}{1.504068in}}%
\pgfpathlineto{\pgfqpoint{4.941195in}{1.495457in}}%
\pgfpathclose%
\pgfusepath{fill}%
\end{pgfscope}%
\begin{pgfscope}%
\pgfsetbuttcap%
\pgfsetmiterjoin%
\definecolor{currentfill}{rgb}{1.000000,1.000000,1.000000}%
\pgfsetfillcolor{currentfill}%
\pgfsetlinewidth{0.000000pt}%
\definecolor{currentstroke}{rgb}{0.000000,0.000000,0.000000}%
\pgfsetstrokecolor{currentstroke}%
\pgfsetstrokeopacity{0.000000}%
\pgfsetdash{}{0pt}%
\pgfpathmoveto{\pgfqpoint{6.818937in}{0.147348in}}%
\pgfpathlineto{\pgfqpoint{9.554231in}{0.147348in}}%
\pgfpathlineto{\pgfqpoint{9.554231in}{2.882642in}}%
\pgfpathlineto{\pgfqpoint{6.818937in}{2.882642in}}%
\pgfpathlineto{\pgfqpoint{6.818937in}{0.147348in}}%
\pgfpathclose%
\pgfusepath{fill}%
\end{pgfscope}%
\begin{pgfscope}%
\pgfsetbuttcap%
\pgfsetmiterjoin%
\definecolor{currentfill}{rgb}{0.950000,0.950000,0.950000}%
\pgfsetfillcolor{currentfill}%
\pgfsetfillopacity{0.500000}%
\pgfsetlinewidth{1.003750pt}%
\definecolor{currentstroke}{rgb}{0.950000,0.950000,0.950000}%
\pgfsetstrokecolor{currentstroke}%
\pgfsetstrokeopacity{0.500000}%
\pgfsetdash{}{0pt}%
\pgfpathmoveto{\pgfqpoint{9.508741in}{1.070011in}}%
\pgfpathlineto{\pgfqpoint{8.223548in}{0.241771in}}%
\pgfpathlineto{\pgfqpoint{8.223548in}{1.173119in}}%
\pgfpathlineto{\pgfqpoint{9.430084in}{2.004410in}}%
\pgfusepath{stroke,fill}%
\end{pgfscope}%
\begin{pgfscope}%
\pgfsetbuttcap%
\pgfsetmiterjoin%
\definecolor{currentfill}{rgb}{0.900000,0.900000,0.900000}%
\pgfsetfillcolor{currentfill}%
\pgfsetfillopacity{0.500000}%
\pgfsetlinewidth{1.003750pt}%
\definecolor{currentstroke}{rgb}{0.900000,0.900000,0.900000}%
\pgfsetstrokecolor{currentstroke}%
\pgfsetstrokeopacity{0.500000}%
\pgfsetdash{}{0pt}%
\pgfpathmoveto{\pgfqpoint{6.938354in}{1.070011in}}%
\pgfpathlineto{\pgfqpoint{8.223548in}{0.241771in}}%
\pgfpathlineto{\pgfqpoint{8.223548in}{1.173119in}}%
\pgfpathlineto{\pgfqpoint{7.017011in}{2.004410in}}%
\pgfusepath{stroke,fill}%
\end{pgfscope}%
\begin{pgfscope}%
\pgfsetbuttcap%
\pgfsetmiterjoin%
\definecolor{currentfill}{rgb}{0.925000,0.925000,0.925000}%
\pgfsetfillcolor{currentfill}%
\pgfsetfillopacity{0.500000}%
\pgfsetlinewidth{1.003750pt}%
\definecolor{currentstroke}{rgb}{0.925000,0.925000,0.925000}%
\pgfsetstrokecolor{currentstroke}%
\pgfsetstrokeopacity{0.500000}%
\pgfsetdash{}{0pt}%
\pgfpathmoveto{\pgfqpoint{8.223548in}{2.937509in}}%
\pgfpathlineto{\pgfqpoint{9.430084in}{2.004410in}}%
\pgfpathlineto{\pgfqpoint{8.223548in}{1.173119in}}%
\pgfpathlineto{\pgfqpoint{7.017011in}{2.004410in}}%
\pgfusepath{stroke,fill}%
\end{pgfscope}%
\begin{pgfscope}%
\pgfsetbuttcap%
\pgfsetroundjoin%
\pgfsetlinewidth{0.803000pt}%
\definecolor{currentstroke}{rgb}{0.690196,0.690196,0.690196}%
\pgfsetstrokecolor{currentstroke}%
\pgfsetdash{}{0pt}%
\pgfpathmoveto{\pgfqpoint{8.304713in}{2.874739in}}%
\pgfpathlineto{\pgfqpoint{7.097924in}{1.948662in}}%
\pgfpathlineto{\pgfqpoint{7.024828in}{1.014283in}}%
\pgfusepath{stroke}%
\end{pgfscope}%
\begin{pgfscope}%
\pgfsetbuttcap%
\pgfsetroundjoin%
\pgfsetlinewidth{0.803000pt}%
\definecolor{currentstroke}{rgb}{0.690196,0.690196,0.690196}%
\pgfsetstrokecolor{currentstroke}%
\pgfsetdash{}{0pt}%
\pgfpathmoveto{\pgfqpoint{8.553700in}{2.682179in}}%
\pgfpathlineto{\pgfqpoint{7.346364in}{1.777489in}}%
\pgfpathlineto{\pgfqpoint{7.290085in}{0.843339in}}%
\pgfusepath{stroke}%
\end{pgfscope}%
\begin{pgfscope}%
\pgfsetbuttcap%
\pgfsetroundjoin%
\pgfsetlinewidth{0.803000pt}%
\definecolor{currentstroke}{rgb}{0.690196,0.690196,0.690196}%
\pgfsetstrokecolor{currentstroke}%
\pgfsetdash{}{0pt}%
\pgfpathmoveto{\pgfqpoint{8.796805in}{2.494169in}}%
\pgfpathlineto{\pgfqpoint{7.589265in}{1.610133in}}%
\pgfpathlineto{\pgfqpoint{7.549052in}{0.676448in}}%
\pgfusepath{stroke}%
\end{pgfscope}%
\begin{pgfscope}%
\pgfsetbuttcap%
\pgfsetroundjoin%
\pgfsetlinewidth{0.803000pt}%
\definecolor{currentstroke}{rgb}{0.690196,0.690196,0.690196}%
\pgfsetstrokecolor{currentstroke}%
\pgfsetdash{}{0pt}%
\pgfpathmoveto{\pgfqpoint{9.034233in}{2.310549in}}%
\pgfpathlineto{\pgfqpoint{7.826809in}{1.446468in}}%
\pgfpathlineto{\pgfqpoint{7.801951in}{0.513468in}}%
\pgfusepath{stroke}%
\end{pgfscope}%
\begin{pgfscope}%
\pgfsetbuttcap%
\pgfsetroundjoin%
\pgfsetlinewidth{0.803000pt}%
\definecolor{currentstroke}{rgb}{0.690196,0.690196,0.690196}%
\pgfsetstrokecolor{currentstroke}%
\pgfsetdash{}{0pt}%
\pgfpathmoveto{\pgfqpoint{9.266181in}{2.131167in}}%
\pgfpathlineto{\pgfqpoint{8.059172in}{1.286373in}}%
\pgfpathlineto{\pgfqpoint{8.048992in}{0.354263in}}%
\pgfusepath{stroke}%
\end{pgfscope}%
\begin{pgfscope}%
\pgfsetbuttcap%
\pgfsetroundjoin%
\pgfsetlinewidth{0.803000pt}%
\definecolor{currentstroke}{rgb}{0.690196,0.690196,0.690196}%
\pgfsetstrokecolor{currentstroke}%
\pgfsetdash{}{0pt}%
\pgfpathmoveto{\pgfqpoint{9.422268in}{1.014283in}}%
\pgfpathlineto{\pgfqpoint{9.349172in}{1.948662in}}%
\pgfpathlineto{\pgfqpoint{8.142383in}{2.874739in}}%
\pgfusepath{stroke}%
\end{pgfscope}%
\begin{pgfscope}%
\pgfsetbuttcap%
\pgfsetroundjoin%
\pgfsetlinewidth{0.803000pt}%
\definecolor{currentstroke}{rgb}{0.690196,0.690196,0.690196}%
\pgfsetstrokecolor{currentstroke}%
\pgfsetdash{}{0pt}%
\pgfpathmoveto{\pgfqpoint{9.157011in}{0.843339in}}%
\pgfpathlineto{\pgfqpoint{9.100731in}{1.777489in}}%
\pgfpathlineto{\pgfqpoint{7.893396in}{2.682179in}}%
\pgfusepath{stroke}%
\end{pgfscope}%
\begin{pgfscope}%
\pgfsetbuttcap%
\pgfsetroundjoin%
\pgfsetlinewidth{0.803000pt}%
\definecolor{currentstroke}{rgb}{0.690196,0.690196,0.690196}%
\pgfsetstrokecolor{currentstroke}%
\pgfsetdash{}{0pt}%
\pgfpathmoveto{\pgfqpoint{8.898044in}{0.676448in}}%
\pgfpathlineto{\pgfqpoint{8.857831in}{1.610133in}}%
\pgfpathlineto{\pgfqpoint{7.650291in}{2.494169in}}%
\pgfusepath{stroke}%
\end{pgfscope}%
\begin{pgfscope}%
\pgfsetbuttcap%
\pgfsetroundjoin%
\pgfsetlinewidth{0.803000pt}%
\definecolor{currentstroke}{rgb}{0.690196,0.690196,0.690196}%
\pgfsetstrokecolor{currentstroke}%
\pgfsetdash{}{0pt}%
\pgfpathmoveto{\pgfqpoint{8.645145in}{0.513468in}}%
\pgfpathlineto{\pgfqpoint{8.620287in}{1.446468in}}%
\pgfpathlineto{\pgfqpoint{7.412863in}{2.310549in}}%
\pgfusepath{stroke}%
\end{pgfscope}%
\begin{pgfscope}%
\pgfsetbuttcap%
\pgfsetroundjoin%
\pgfsetlinewidth{0.803000pt}%
\definecolor{currentstroke}{rgb}{0.690196,0.690196,0.690196}%
\pgfsetstrokecolor{currentstroke}%
\pgfsetdash{}{0pt}%
\pgfpathmoveto{\pgfqpoint{8.398104in}{0.354263in}}%
\pgfpathlineto{\pgfqpoint{8.387924in}{1.286373in}}%
\pgfpathlineto{\pgfqpoint{7.180914in}{2.131167in}}%
\pgfusepath{stroke}%
\end{pgfscope}%
\begin{pgfscope}%
\pgfsetbuttcap%
\pgfsetroundjoin%
\pgfsetlinewidth{0.803000pt}%
\definecolor{currentstroke}{rgb}{0.690196,0.690196,0.690196}%
\pgfsetstrokecolor{currentstroke}%
\pgfsetdash{}{0pt}%
\pgfpathmoveto{\pgfqpoint{6.943664in}{1.133087in}}%
\pgfpathlineto{\pgfqpoint{8.223548in}{0.304434in}}%
\pgfpathlineto{\pgfqpoint{9.503432in}{1.133087in}}%
\pgfusepath{stroke}%
\end{pgfscope}%
\begin{pgfscope}%
\pgfsetbuttcap%
\pgfsetroundjoin%
\pgfsetlinewidth{0.803000pt}%
\definecolor{currentstroke}{rgb}{0.690196,0.690196,0.690196}%
\pgfsetstrokecolor{currentstroke}%
\pgfsetdash{}{0pt}%
\pgfpathmoveto{\pgfqpoint{6.959936in}{1.326388in}}%
\pgfpathlineto{\pgfqpoint{8.223548in}{0.496654in}}%
\pgfpathlineto{\pgfqpoint{9.487160in}{1.326388in}}%
\pgfusepath{stroke}%
\end{pgfscope}%
\begin{pgfscope}%
\pgfsetbuttcap%
\pgfsetroundjoin%
\pgfsetlinewidth{0.803000pt}%
\definecolor{currentstroke}{rgb}{0.690196,0.690196,0.690196}%
\pgfsetstrokecolor{currentstroke}%
\pgfsetdash{}{0pt}%
\pgfpathmoveto{\pgfqpoint{6.975799in}{1.514835in}}%
\pgfpathlineto{\pgfqpoint{8.223548in}{0.684319in}}%
\pgfpathlineto{\pgfqpoint{9.471296in}{1.514835in}}%
\pgfusepath{stroke}%
\end{pgfscope}%
\begin{pgfscope}%
\pgfsetbuttcap%
\pgfsetroundjoin%
\pgfsetlinewidth{0.803000pt}%
\definecolor{currentstroke}{rgb}{0.690196,0.690196,0.690196}%
\pgfsetstrokecolor{currentstroke}%
\pgfsetdash{}{0pt}%
\pgfpathmoveto{\pgfqpoint{6.991269in}{1.698610in}}%
\pgfpathlineto{\pgfqpoint{8.223548in}{0.867589in}}%
\pgfpathlineto{\pgfqpoint{9.455826in}{1.698610in}}%
\pgfusepath{stroke}%
\end{pgfscope}%
\begin{pgfscope}%
\pgfsetbuttcap%
\pgfsetroundjoin%
\pgfsetlinewidth{0.803000pt}%
\definecolor{currentstroke}{rgb}{0.690196,0.690196,0.690196}%
\pgfsetstrokecolor{currentstroke}%
\pgfsetdash{}{0pt}%
\pgfpathmoveto{\pgfqpoint{7.006360in}{1.877883in}}%
\pgfpathlineto{\pgfqpoint{8.223548in}{1.046618in}}%
\pgfpathlineto{\pgfqpoint{9.440735in}{1.877883in}}%
\pgfusepath{stroke}%
\end{pgfscope}%
\begin{pgfscope}%
\pgfsetrectcap%
\pgfsetroundjoin%
\pgfsetlinewidth{0.803000pt}%
\definecolor{currentstroke}{rgb}{0.000000,0.000000,0.000000}%
\pgfsetstrokecolor{currentstroke}%
\pgfsetdash{}{0pt}%
\pgfpathmoveto{\pgfqpoint{9.430084in}{2.004410in}}%
\pgfpathlineto{\pgfqpoint{8.223548in}{2.937509in}}%
\pgfusepath{stroke}%
\end{pgfscope}%
\begin{pgfscope}%
\pgfsetrectcap%
\pgfsetroundjoin%
\pgfsetlinewidth{0.803000pt}%
\definecolor{currentstroke}{rgb}{0.000000,0.000000,0.000000}%
\pgfsetstrokecolor{currentstroke}%
\pgfsetdash{}{0pt}%
\pgfpathmoveto{\pgfqpoint{8.294475in}{2.866882in}}%
\pgfpathlineto{\pgfqpoint{8.325219in}{2.890475in}}%
\pgfusepath{stroke}%
\end{pgfscope}%
\begin{pgfscope}%
\pgfsetrectcap%
\pgfsetroundjoin%
\pgfsetlinewidth{0.803000pt}%
\definecolor{currentstroke}{rgb}{0.000000,0.000000,0.000000}%
\pgfsetstrokecolor{currentstroke}%
\pgfsetdash{}{0pt}%
\pgfpathmoveto{\pgfqpoint{8.543464in}{2.674509in}}%
\pgfpathlineto{\pgfqpoint{8.574201in}{2.697542in}}%
\pgfusepath{stroke}%
\end{pgfscope}%
\begin{pgfscope}%
\pgfsetrectcap%
\pgfsetroundjoin%
\pgfsetlinewidth{0.803000pt}%
\definecolor{currentstroke}{rgb}{0.000000,0.000000,0.000000}%
\pgfsetstrokecolor{currentstroke}%
\pgfsetdash{}{0pt}%
\pgfpathmoveto{\pgfqpoint{8.786574in}{2.486679in}}%
\pgfpathlineto{\pgfqpoint{8.817296in}{2.509171in}}%
\pgfusepath{stroke}%
\end{pgfscope}%
\begin{pgfscope}%
\pgfsetrectcap%
\pgfsetroundjoin%
\pgfsetlinewidth{0.803000pt}%
\definecolor{currentstroke}{rgb}{0.000000,0.000000,0.000000}%
\pgfsetstrokecolor{currentstroke}%
\pgfsetdash{}{0pt}%
\pgfpathmoveto{\pgfqpoint{9.024010in}{2.303233in}}%
\pgfpathlineto{\pgfqpoint{9.054708in}{2.325202in}}%
\pgfusepath{stroke}%
\end{pgfscope}%
\begin{pgfscope}%
\pgfsetrectcap%
\pgfsetroundjoin%
\pgfsetlinewidth{0.803000pt}%
\definecolor{currentstroke}{rgb}{0.000000,0.000000,0.000000}%
\pgfsetstrokecolor{currentstroke}%
\pgfsetdash{}{0pt}%
\pgfpathmoveto{\pgfqpoint{9.255968in}{2.124019in}}%
\pgfpathlineto{\pgfqpoint{9.286636in}{2.145484in}}%
\pgfusepath{stroke}%
\end{pgfscope}%
\begin{pgfscope}%
\definecolor{textcolor}{rgb}{0.000000,0.000000,0.000000}%
\pgfsetstrokecolor{textcolor}%
\pgfsetfillcolor{textcolor}%
\pgftext[x=9.122590in,y=2.861959in,,]{\color{textcolor}{\rmfamily\fontsize{14.000000}{16.800000}\selectfont\catcode`\^=\active\def^{\ifmmode\sp\else\^{}\fi}\catcode`\%=\active\def%{\%}f1}}%
\end{pgfscope}%
\begin{pgfscope}%
\pgfsetrectcap%
\pgfsetroundjoin%
\pgfsetlinewidth{0.803000pt}%
\definecolor{currentstroke}{rgb}{0.000000,0.000000,0.000000}%
\pgfsetstrokecolor{currentstroke}%
\pgfsetdash{}{0pt}%
\pgfpathmoveto{\pgfqpoint{7.017011in}{2.004410in}}%
\pgfpathlineto{\pgfqpoint{8.223548in}{2.937509in}}%
\pgfusepath{stroke}%
\end{pgfscope}%
\begin{pgfscope}%
\pgfsetrectcap%
\pgfsetroundjoin%
\pgfsetlinewidth{0.803000pt}%
\definecolor{currentstroke}{rgb}{0.000000,0.000000,0.000000}%
\pgfsetstrokecolor{currentstroke}%
\pgfsetdash{}{0pt}%
\pgfpathmoveto{\pgfqpoint{8.152621in}{2.866882in}}%
\pgfpathlineto{\pgfqpoint{8.121876in}{2.890475in}}%
\pgfusepath{stroke}%
\end{pgfscope}%
\begin{pgfscope}%
\pgfsetrectcap%
\pgfsetroundjoin%
\pgfsetlinewidth{0.803000pt}%
\definecolor{currentstroke}{rgb}{0.000000,0.000000,0.000000}%
\pgfsetstrokecolor{currentstroke}%
\pgfsetdash{}{0pt}%
\pgfpathmoveto{\pgfqpoint{7.903631in}{2.674509in}}%
\pgfpathlineto{\pgfqpoint{7.872894in}{2.697542in}}%
\pgfusepath{stroke}%
\end{pgfscope}%
\begin{pgfscope}%
\pgfsetrectcap%
\pgfsetroundjoin%
\pgfsetlinewidth{0.803000pt}%
\definecolor{currentstroke}{rgb}{0.000000,0.000000,0.000000}%
\pgfsetstrokecolor{currentstroke}%
\pgfsetdash{}{0pt}%
\pgfpathmoveto{\pgfqpoint{7.660522in}{2.486679in}}%
\pgfpathlineto{\pgfqpoint{7.629800in}{2.509171in}}%
\pgfusepath{stroke}%
\end{pgfscope}%
\begin{pgfscope}%
\pgfsetrectcap%
\pgfsetroundjoin%
\pgfsetlinewidth{0.803000pt}%
\definecolor{currentstroke}{rgb}{0.000000,0.000000,0.000000}%
\pgfsetstrokecolor{currentstroke}%
\pgfsetdash{}{0pt}%
\pgfpathmoveto{\pgfqpoint{7.423086in}{2.303233in}}%
\pgfpathlineto{\pgfqpoint{7.392387in}{2.325202in}}%
\pgfusepath{stroke}%
\end{pgfscope}%
\begin{pgfscope}%
\pgfsetrectcap%
\pgfsetroundjoin%
\pgfsetlinewidth{0.803000pt}%
\definecolor{currentstroke}{rgb}{0.000000,0.000000,0.000000}%
\pgfsetstrokecolor{currentstroke}%
\pgfsetdash{}{0pt}%
\pgfpathmoveto{\pgfqpoint{7.191128in}{2.124019in}}%
\pgfpathlineto{\pgfqpoint{7.160460in}{2.145484in}}%
\pgfusepath{stroke}%
\end{pgfscope}%
\begin{pgfscope}%
\definecolor{textcolor}{rgb}{0.000000,0.000000,0.000000}%
\pgfsetstrokecolor{textcolor}%
\pgfsetfillcolor{textcolor}%
\pgftext[x=7.324506in,y=2.861959in,,]{\color{textcolor}{\rmfamily\fontsize{14.000000}{16.800000}\selectfont\catcode`\^=\active\def^{\ifmmode\sp\else\^{}\fi}\catcode`\%=\active\def%{\%}f2}}%
\end{pgfscope}%
\begin{pgfscope}%
\pgfsetrectcap%
\pgfsetroundjoin%
\pgfsetlinewidth{0.803000pt}%
\definecolor{currentstroke}{rgb}{0.000000,0.000000,0.000000}%
\pgfsetstrokecolor{currentstroke}%
\pgfsetdash{}{0pt}%
\pgfpathmoveto{\pgfqpoint{7.017011in}{2.004410in}}%
\pgfpathlineto{\pgfqpoint{6.938354in}{1.070011in}}%
\pgfusepath{stroke}%
\end{pgfscope}%
\begin{pgfscope}%
\pgfsetrectcap%
\pgfsetroundjoin%
\pgfsetlinewidth{0.803000pt}%
\definecolor{currentstroke}{rgb}{0.000000,0.000000,0.000000}%
\pgfsetstrokecolor{currentstroke}%
\pgfsetdash{}{0pt}%
\pgfpathmoveto{\pgfqpoint{6.954524in}{1.126056in}}%
\pgfpathlineto{\pgfqpoint{6.921911in}{1.147171in}}%
\pgfusepath{stroke}%
\end{pgfscope}%
\begin{pgfscope}%
\pgfsetrectcap%
\pgfsetroundjoin%
\pgfsetlinewidth{0.803000pt}%
\definecolor{currentstroke}{rgb}{0.000000,0.000000,0.000000}%
\pgfsetstrokecolor{currentstroke}%
\pgfsetdash{}{0pt}%
\pgfpathmoveto{\pgfqpoint{6.970651in}{1.319352in}}%
\pgfpathlineto{\pgfqpoint{6.938476in}{1.340480in}}%
\pgfusepath{stroke}%
\end{pgfscope}%
\begin{pgfscope}%
\pgfsetrectcap%
\pgfsetroundjoin%
\pgfsetlinewidth{0.803000pt}%
\definecolor{currentstroke}{rgb}{0.000000,0.000000,0.000000}%
\pgfsetstrokecolor{currentstroke}%
\pgfsetdash{}{0pt}%
\pgfpathmoveto{\pgfqpoint{6.986372in}{1.507798in}}%
\pgfpathlineto{\pgfqpoint{6.954624in}{1.528930in}}%
\pgfusepath{stroke}%
\end{pgfscope}%
\begin{pgfscope}%
\pgfsetrectcap%
\pgfsetroundjoin%
\pgfsetlinewidth{0.803000pt}%
\definecolor{currentstroke}{rgb}{0.000000,0.000000,0.000000}%
\pgfsetstrokecolor{currentstroke}%
\pgfsetdash{}{0pt}%
\pgfpathmoveto{\pgfqpoint{7.001704in}{1.691573in}}%
\pgfpathlineto{\pgfqpoint{6.970371in}{1.712703in}}%
\pgfusepath{stroke}%
\end{pgfscope}%
\begin{pgfscope}%
\pgfsetrectcap%
\pgfsetroundjoin%
\pgfsetlinewidth{0.803000pt}%
\definecolor{currentstroke}{rgb}{0.000000,0.000000,0.000000}%
\pgfsetstrokecolor{currentstroke}%
\pgfsetdash{}{0pt}%
\pgfpathmoveto{\pgfqpoint{7.016660in}{1.870849in}}%
\pgfpathlineto{\pgfqpoint{6.985733in}{1.891971in}}%
\pgfusepath{stroke}%
\end{pgfscope}%
\begin{pgfscope}%
\definecolor{textcolor}{rgb}{0.000000,0.000000,0.000000}%
\pgfsetstrokecolor{textcolor}%
\pgfsetfillcolor{textcolor}%
\pgftext[x=6.420762in,y=1.551958in,,]{\color{textcolor}{\rmfamily\fontsize{14.000000}{16.800000}\selectfont\catcode`\^=\active\def^{\ifmmode\sp\else\^{}\fi}\catcode`\%=\active\def%{\%}f3}}%
\end{pgfscope}%
\begin{pgfscope}%
\pgfpathrectangle{\pgfqpoint{6.818937in}{0.147348in}}{\pgfqpoint{2.735294in}{2.735294in}}%
\pgfusepath{clip}%
\pgfsetbuttcap%
\pgfsetroundjoin%
\definecolor{currentfill}{rgb}{0.839216,0.152941,0.156863}%
\pgfsetfillcolor{currentfill}%
\pgfsetfillopacity{0.300000}%
\pgfsetlinewidth{1.003750pt}%
\definecolor{currentstroke}{rgb}{0.839216,0.152941,0.156863}%
\pgfsetstrokecolor{currentstroke}%
\pgfsetstrokeopacity{0.300000}%
\pgfsetdash{}{0pt}%
\pgfpathmoveto{\pgfqpoint{7.927926in}{1.506774in}}%
\pgfpathcurveto{\pgfqpoint{7.938014in}{1.506774in}}{\pgfqpoint{7.947689in}{1.510782in}}{\pgfqpoint{7.954822in}{1.517914in}}%
\pgfpathcurveto{\pgfqpoint{7.961955in}{1.525047in}}{\pgfqpoint{7.965963in}{1.534723in}}{\pgfqpoint{7.965963in}{1.544810in}}%
\pgfpathcurveto{\pgfqpoint{7.965963in}{1.554897in}}{\pgfqpoint{7.961955in}{1.564573in}}{\pgfqpoint{7.954822in}{1.571706in}}%
\pgfpathcurveto{\pgfqpoint{7.947689in}{1.578839in}}{\pgfqpoint{7.938014in}{1.582846in}}{\pgfqpoint{7.927926in}{1.582846in}}%
\pgfpathcurveto{\pgfqpoint{7.917839in}{1.582846in}}{\pgfqpoint{7.908164in}{1.578839in}}{\pgfqpoint{7.901031in}{1.571706in}}%
\pgfpathcurveto{\pgfqpoint{7.893898in}{1.564573in}}{\pgfqpoint{7.889890in}{1.554897in}}{\pgfqpoint{7.889890in}{1.544810in}}%
\pgfpathcurveto{\pgfqpoint{7.889890in}{1.534723in}}{\pgfqpoint{7.893898in}{1.525047in}}{\pgfqpoint{7.901031in}{1.517914in}}%
\pgfpathcurveto{\pgfqpoint{7.908164in}{1.510782in}}{\pgfqpoint{7.917839in}{1.506774in}}{\pgfqpoint{7.927926in}{1.506774in}}%
\pgfpathlineto{\pgfqpoint{7.927926in}{1.506774in}}%
\pgfpathclose%
\pgfusepath{stroke,fill}%
\end{pgfscope}%
\begin{pgfscope}%
\pgfpathrectangle{\pgfqpoint{6.818937in}{0.147348in}}{\pgfqpoint{2.735294in}{2.735294in}}%
\pgfusepath{clip}%
\pgfsetbuttcap%
\pgfsetroundjoin%
\definecolor{currentfill}{rgb}{0.839216,0.152941,0.156863}%
\pgfsetfillcolor{currentfill}%
\pgfsetfillopacity{0.368519}%
\pgfsetlinewidth{1.003750pt}%
\definecolor{currentstroke}{rgb}{0.839216,0.152941,0.156863}%
\pgfsetstrokecolor{currentstroke}%
\pgfsetstrokeopacity{0.368519}%
\pgfsetdash{}{0pt}%
\pgfpathmoveto{\pgfqpoint{7.807338in}{1.289993in}}%
\pgfpathcurveto{\pgfqpoint{7.817425in}{1.289993in}}{\pgfqpoint{7.827101in}{1.294001in}}{\pgfqpoint{7.834233in}{1.301134in}}%
\pgfpathcurveto{\pgfqpoint{7.841366in}{1.308267in}}{\pgfqpoint{7.845374in}{1.317942in}}{\pgfqpoint{7.845374in}{1.328030in}}%
\pgfpathcurveto{\pgfqpoint{7.845374in}{1.338117in}}{\pgfqpoint{7.841366in}{1.347793in}}{\pgfqpoint{7.834233in}{1.354925in}}%
\pgfpathcurveto{\pgfqpoint{7.827101in}{1.362058in}}{\pgfqpoint{7.817425in}{1.366066in}}{\pgfqpoint{7.807338in}{1.366066in}}%
\pgfpathcurveto{\pgfqpoint{7.797250in}{1.366066in}}{\pgfqpoint{7.787575in}{1.362058in}}{\pgfqpoint{7.780442in}{1.354925in}}%
\pgfpathcurveto{\pgfqpoint{7.773309in}{1.347793in}}{\pgfqpoint{7.769301in}{1.338117in}}{\pgfqpoint{7.769301in}{1.328030in}}%
\pgfpathcurveto{\pgfqpoint{7.769301in}{1.317942in}}{\pgfqpoint{7.773309in}{1.308267in}}{\pgfqpoint{7.780442in}{1.301134in}}%
\pgfpathcurveto{\pgfqpoint{7.787575in}{1.294001in}}{\pgfqpoint{7.797250in}{1.289993in}}{\pgfqpoint{7.807338in}{1.289993in}}%
\pgfpathlineto{\pgfqpoint{7.807338in}{1.289993in}}%
\pgfpathclose%
\pgfusepath{stroke,fill}%
\end{pgfscope}%
\begin{pgfscope}%
\pgfpathrectangle{\pgfqpoint{6.818937in}{0.147348in}}{\pgfqpoint{2.735294in}{2.735294in}}%
\pgfusepath{clip}%
\pgfsetbuttcap%
\pgfsetroundjoin%
\definecolor{currentfill}{rgb}{0.839216,0.152941,0.156863}%
\pgfsetfillcolor{currentfill}%
\pgfsetfillopacity{0.431635}%
\pgfsetlinewidth{1.003750pt}%
\definecolor{currentstroke}{rgb}{0.839216,0.152941,0.156863}%
\pgfsetstrokecolor{currentstroke}%
\pgfsetstrokeopacity{0.431635}%
\pgfsetdash{}{0pt}%
\pgfpathmoveto{\pgfqpoint{7.865218in}{1.926337in}}%
\pgfpathcurveto{\pgfqpoint{7.875306in}{1.926337in}}{\pgfqpoint{7.884981in}{1.930345in}}{\pgfqpoint{7.892114in}{1.937478in}}%
\pgfpathcurveto{\pgfqpoint{7.899247in}{1.944610in}}{\pgfqpoint{7.903255in}{1.954286in}}{\pgfqpoint{7.903255in}{1.964373in}}%
\pgfpathcurveto{\pgfqpoint{7.903255in}{1.974461in}}{\pgfqpoint{7.899247in}{1.984136in}}{\pgfqpoint{7.892114in}{1.991269in}}%
\pgfpathcurveto{\pgfqpoint{7.884981in}{1.998402in}}{\pgfqpoint{7.875306in}{2.002410in}}{\pgfqpoint{7.865218in}{2.002410in}}%
\pgfpathcurveto{\pgfqpoint{7.855131in}{2.002410in}}{\pgfqpoint{7.845455in}{1.998402in}}{\pgfqpoint{7.838323in}{1.991269in}}%
\pgfpathcurveto{\pgfqpoint{7.831190in}{1.984136in}}{\pgfqpoint{7.827182in}{1.974461in}}{\pgfqpoint{7.827182in}{1.964373in}}%
\pgfpathcurveto{\pgfqpoint{7.827182in}{1.954286in}}{\pgfqpoint{7.831190in}{1.944610in}}{\pgfqpoint{7.838323in}{1.937478in}}%
\pgfpathcurveto{\pgfqpoint{7.845455in}{1.930345in}}{\pgfqpoint{7.855131in}{1.926337in}}{\pgfqpoint{7.865218in}{1.926337in}}%
\pgfpathlineto{\pgfqpoint{7.865218in}{1.926337in}}%
\pgfpathclose%
\pgfusepath{stroke,fill}%
\end{pgfscope}%
\begin{pgfscope}%
\pgfpathrectangle{\pgfqpoint{6.818937in}{0.147348in}}{\pgfqpoint{2.735294in}{2.735294in}}%
\pgfusepath{clip}%
\pgfsetbuttcap%
\pgfsetroundjoin%
\definecolor{currentfill}{rgb}{0.839216,0.152941,0.156863}%
\pgfsetfillcolor{currentfill}%
\pgfsetfillopacity{0.502644}%
\pgfsetlinewidth{1.003750pt}%
\definecolor{currentstroke}{rgb}{0.839216,0.152941,0.156863}%
\pgfsetstrokecolor{currentstroke}%
\pgfsetstrokeopacity{0.502644}%
\pgfsetdash{}{0pt}%
\pgfpathmoveto{\pgfqpoint{8.531889in}{2.015496in}}%
\pgfpathcurveto{\pgfqpoint{8.541976in}{2.015496in}}{\pgfqpoint{8.551652in}{2.019504in}}{\pgfqpoint{8.558785in}{2.026637in}}%
\pgfpathcurveto{\pgfqpoint{8.565918in}{2.033769in}}{\pgfqpoint{8.569925in}{2.043445in}}{\pgfqpoint{8.569925in}{2.053532in}}%
\pgfpathcurveto{\pgfqpoint{8.569925in}{2.063620in}}{\pgfqpoint{8.565918in}{2.073295in}}{\pgfqpoint{8.558785in}{2.080428in}}%
\pgfpathcurveto{\pgfqpoint{8.551652in}{2.087561in}}{\pgfqpoint{8.541976in}{2.091569in}}{\pgfqpoint{8.531889in}{2.091569in}}%
\pgfpathcurveto{\pgfqpoint{8.521802in}{2.091569in}}{\pgfqpoint{8.512126in}{2.087561in}}{\pgfqpoint{8.504993in}{2.080428in}}%
\pgfpathcurveto{\pgfqpoint{8.497860in}{2.073295in}}{\pgfqpoint{8.493853in}{2.063620in}}{\pgfqpoint{8.493853in}{2.053532in}}%
\pgfpathcurveto{\pgfqpoint{8.493853in}{2.043445in}}{\pgfqpoint{8.497860in}{2.033769in}}{\pgfqpoint{8.504993in}{2.026637in}}%
\pgfpathcurveto{\pgfqpoint{8.512126in}{2.019504in}}{\pgfqpoint{8.521802in}{2.015496in}}{\pgfqpoint{8.531889in}{2.015496in}}%
\pgfpathlineto{\pgfqpoint{8.531889in}{2.015496in}}%
\pgfpathclose%
\pgfusepath{stroke,fill}%
\end{pgfscope}%
\begin{pgfscope}%
\pgfpathrectangle{\pgfqpoint{6.818937in}{0.147348in}}{\pgfqpoint{2.735294in}{2.735294in}}%
\pgfusepath{clip}%
\pgfsetbuttcap%
\pgfsetroundjoin%
\definecolor{currentfill}{rgb}{0.839216,0.152941,0.156863}%
\pgfsetfillcolor{currentfill}%
\pgfsetfillopacity{0.555030}%
\pgfsetlinewidth{1.003750pt}%
\definecolor{currentstroke}{rgb}{0.839216,0.152941,0.156863}%
\pgfsetstrokecolor{currentstroke}%
\pgfsetstrokeopacity{0.555030}%
\pgfsetdash{}{0pt}%
\pgfpathmoveto{\pgfqpoint{9.028634in}{1.507681in}}%
\pgfpathcurveto{\pgfqpoint{9.038721in}{1.507681in}}{\pgfqpoint{9.048397in}{1.511689in}}{\pgfqpoint{9.055530in}{1.518822in}}%
\pgfpathcurveto{\pgfqpoint{9.062663in}{1.525955in}}{\pgfqpoint{9.066670in}{1.535630in}}{\pgfqpoint{9.066670in}{1.545718in}}%
\pgfpathcurveto{\pgfqpoint{9.066670in}{1.555805in}}{\pgfqpoint{9.062663in}{1.565481in}}{\pgfqpoint{9.055530in}{1.572613in}}%
\pgfpathcurveto{\pgfqpoint{9.048397in}{1.579746in}}{\pgfqpoint{9.038721in}{1.583754in}}{\pgfqpoint{9.028634in}{1.583754in}}%
\pgfpathcurveto{\pgfqpoint{9.018547in}{1.583754in}}{\pgfqpoint{9.008871in}{1.579746in}}{\pgfqpoint{9.001738in}{1.572613in}}%
\pgfpathcurveto{\pgfqpoint{8.994605in}{1.565481in}}{\pgfqpoint{8.990598in}{1.555805in}}{\pgfqpoint{8.990598in}{1.545718in}}%
\pgfpathcurveto{\pgfqpoint{8.990598in}{1.535630in}}{\pgfqpoint{8.994605in}{1.525955in}}{\pgfqpoint{9.001738in}{1.518822in}}%
\pgfpathcurveto{\pgfqpoint{9.008871in}{1.511689in}}{\pgfqpoint{9.018547in}{1.507681in}}{\pgfqpoint{9.028634in}{1.507681in}}%
\pgfpathlineto{\pgfqpoint{9.028634in}{1.507681in}}%
\pgfpathclose%
\pgfusepath{stroke,fill}%
\end{pgfscope}%
\begin{pgfscope}%
\pgfpathrectangle{\pgfqpoint{6.818937in}{0.147348in}}{\pgfqpoint{2.735294in}{2.735294in}}%
\pgfusepath{clip}%
\pgfsetbuttcap%
\pgfsetroundjoin%
\definecolor{currentfill}{rgb}{0.839216,0.152941,0.156863}%
\pgfsetfillcolor{currentfill}%
\pgfsetfillopacity{0.641254}%
\pgfsetlinewidth{1.003750pt}%
\definecolor{currentstroke}{rgb}{0.839216,0.152941,0.156863}%
\pgfsetstrokecolor{currentstroke}%
\pgfsetstrokeopacity{0.641254}%
\pgfsetdash{}{0pt}%
\pgfpathmoveto{\pgfqpoint{8.416294in}{0.946553in}}%
\pgfpathcurveto{\pgfqpoint{8.426381in}{0.946553in}}{\pgfqpoint{8.436057in}{0.950561in}}{\pgfqpoint{8.443190in}{0.957694in}}%
\pgfpathcurveto{\pgfqpoint{8.450322in}{0.964827in}}{\pgfqpoint{8.454330in}{0.974502in}}{\pgfqpoint{8.454330in}{0.984590in}}%
\pgfpathcurveto{\pgfqpoint{8.454330in}{0.994677in}}{\pgfqpoint{8.450322in}{1.004352in}}{\pgfqpoint{8.443190in}{1.011485in}}%
\pgfpathcurveto{\pgfqpoint{8.436057in}{1.018618in}}{\pgfqpoint{8.426381in}{1.022626in}}{\pgfqpoint{8.416294in}{1.022626in}}%
\pgfpathcurveto{\pgfqpoint{8.406207in}{1.022626in}}{\pgfqpoint{8.396531in}{1.018618in}}{\pgfqpoint{8.389398in}{1.011485in}}%
\pgfpathcurveto{\pgfqpoint{8.382265in}{1.004352in}}{\pgfqpoint{8.378258in}{0.994677in}}{\pgfqpoint{8.378258in}{0.984590in}}%
\pgfpathcurveto{\pgfqpoint{8.378258in}{0.974502in}}{\pgfqpoint{8.382265in}{0.964827in}}{\pgfqpoint{8.389398in}{0.957694in}}%
\pgfpathcurveto{\pgfqpoint{8.396531in}{0.950561in}}{\pgfqpoint{8.406207in}{0.946553in}}{\pgfqpoint{8.416294in}{0.946553in}}%
\pgfpathlineto{\pgfqpoint{8.416294in}{0.946553in}}%
\pgfpathclose%
\pgfusepath{stroke,fill}%
\end{pgfscope}%
\begin{pgfscope}%
\pgfpathrectangle{\pgfqpoint{6.818937in}{0.147348in}}{\pgfqpoint{2.735294in}{2.735294in}}%
\pgfusepath{clip}%
\pgfsetbuttcap%
\pgfsetroundjoin%
\definecolor{currentfill}{rgb}{0.839216,0.152941,0.156863}%
\pgfsetfillcolor{currentfill}%
\pgfsetfillopacity{0.659293}%
\pgfsetlinewidth{1.003750pt}%
\definecolor{currentstroke}{rgb}{0.839216,0.152941,0.156863}%
\pgfsetstrokecolor{currentstroke}%
\pgfsetstrokeopacity{0.659293}%
\pgfsetdash{}{0pt}%
\pgfpathmoveto{\pgfqpoint{7.519784in}{1.533581in}}%
\pgfpathcurveto{\pgfqpoint{7.529872in}{1.533581in}}{\pgfqpoint{7.539547in}{1.537589in}}{\pgfqpoint{7.546680in}{1.544722in}}%
\pgfpathcurveto{\pgfqpoint{7.553813in}{1.551854in}}{\pgfqpoint{7.557821in}{1.561530in}}{\pgfqpoint{7.557821in}{1.571617in}}%
\pgfpathcurveto{\pgfqpoint{7.557821in}{1.581705in}}{\pgfqpoint{7.553813in}{1.591380in}}{\pgfqpoint{7.546680in}{1.598513in}}%
\pgfpathcurveto{\pgfqpoint{7.539547in}{1.605646in}}{\pgfqpoint{7.529872in}{1.609654in}}{\pgfqpoint{7.519784in}{1.609654in}}%
\pgfpathcurveto{\pgfqpoint{7.509697in}{1.609654in}}{\pgfqpoint{7.500022in}{1.605646in}}{\pgfqpoint{7.492889in}{1.598513in}}%
\pgfpathcurveto{\pgfqpoint{7.485756in}{1.591380in}}{\pgfqpoint{7.481748in}{1.581705in}}{\pgfqpoint{7.481748in}{1.571617in}}%
\pgfpathcurveto{\pgfqpoint{7.481748in}{1.561530in}}{\pgfqpoint{7.485756in}{1.551854in}}{\pgfqpoint{7.492889in}{1.544722in}}%
\pgfpathcurveto{\pgfqpoint{7.500022in}{1.537589in}}{\pgfqpoint{7.509697in}{1.533581in}}{\pgfqpoint{7.519784in}{1.533581in}}%
\pgfpathlineto{\pgfqpoint{7.519784in}{1.533581in}}%
\pgfpathclose%
\pgfusepath{stroke,fill}%
\end{pgfscope}%
\begin{pgfscope}%
\pgfpathrectangle{\pgfqpoint{6.818937in}{0.147348in}}{\pgfqpoint{2.735294in}{2.735294in}}%
\pgfusepath{clip}%
\pgfsetbuttcap%
\pgfsetroundjoin%
\definecolor{currentfill}{rgb}{0.839216,0.152941,0.156863}%
\pgfsetfillcolor{currentfill}%
\pgfsetfillopacity{0.661693}%
\pgfsetlinewidth{1.003750pt}%
\definecolor{currentstroke}{rgb}{0.839216,0.152941,0.156863}%
\pgfsetstrokecolor{currentstroke}%
\pgfsetstrokeopacity{0.661693}%
\pgfsetdash{}{0pt}%
\pgfpathmoveto{\pgfqpoint{7.793775in}{2.111490in}}%
\pgfpathcurveto{\pgfqpoint{7.803863in}{2.111490in}}{\pgfqpoint{7.813538in}{2.115498in}}{\pgfqpoint{7.820671in}{2.122631in}}%
\pgfpathcurveto{\pgfqpoint{7.827804in}{2.129763in}}{\pgfqpoint{7.831812in}{2.139439in}}{\pgfqpoint{7.831812in}{2.149526in}}%
\pgfpathcurveto{\pgfqpoint{7.831812in}{2.159614in}}{\pgfqpoint{7.827804in}{2.169289in}}{\pgfqpoint{7.820671in}{2.176422in}}%
\pgfpathcurveto{\pgfqpoint{7.813538in}{2.183555in}}{\pgfqpoint{7.803863in}{2.187563in}}{\pgfqpoint{7.793775in}{2.187563in}}%
\pgfpathcurveto{\pgfqpoint{7.783688in}{2.187563in}}{\pgfqpoint{7.774013in}{2.183555in}}{\pgfqpoint{7.766880in}{2.176422in}}%
\pgfpathcurveto{\pgfqpoint{7.759747in}{2.169289in}}{\pgfqpoint{7.755739in}{2.159614in}}{\pgfqpoint{7.755739in}{2.149526in}}%
\pgfpathcurveto{\pgfqpoint{7.755739in}{2.139439in}}{\pgfqpoint{7.759747in}{2.129763in}}{\pgfqpoint{7.766880in}{2.122631in}}%
\pgfpathcurveto{\pgfqpoint{7.774013in}{2.115498in}}{\pgfqpoint{7.783688in}{2.111490in}}{\pgfqpoint{7.793775in}{2.111490in}}%
\pgfpathlineto{\pgfqpoint{7.793775in}{2.111490in}}%
\pgfpathclose%
\pgfusepath{stroke,fill}%
\end{pgfscope}%
\begin{pgfscope}%
\pgfpathrectangle{\pgfqpoint{6.818937in}{0.147348in}}{\pgfqpoint{2.735294in}{2.735294in}}%
\pgfusepath{clip}%
\pgfsetbuttcap%
\pgfsetroundjoin%
\definecolor{currentfill}{rgb}{0.839216,0.152941,0.156863}%
\pgfsetfillcolor{currentfill}%
\pgfsetfillopacity{0.667667}%
\pgfsetlinewidth{1.003750pt}%
\definecolor{currentstroke}{rgb}{0.839216,0.152941,0.156863}%
\pgfsetstrokecolor{currentstroke}%
\pgfsetstrokeopacity{0.667667}%
\pgfsetdash{}{0pt}%
\pgfpathmoveto{\pgfqpoint{7.784633in}{2.171864in}}%
\pgfpathcurveto{\pgfqpoint{7.794720in}{2.171864in}}{\pgfqpoint{7.804396in}{2.175872in}}{\pgfqpoint{7.811529in}{2.183005in}}%
\pgfpathcurveto{\pgfqpoint{7.818661in}{2.190137in}}{\pgfqpoint{7.822669in}{2.199813in}}{\pgfqpoint{7.822669in}{2.209900in}}%
\pgfpathcurveto{\pgfqpoint{7.822669in}{2.219988in}}{\pgfqpoint{7.818661in}{2.229663in}}{\pgfqpoint{7.811529in}{2.236796in}}%
\pgfpathcurveto{\pgfqpoint{7.804396in}{2.243929in}}{\pgfqpoint{7.794720in}{2.247937in}}{\pgfqpoint{7.784633in}{2.247937in}}%
\pgfpathcurveto{\pgfqpoint{7.774546in}{2.247937in}}{\pgfqpoint{7.764870in}{2.243929in}}{\pgfqpoint{7.757737in}{2.236796in}}%
\pgfpathcurveto{\pgfqpoint{7.750604in}{2.229663in}}{\pgfqpoint{7.746597in}{2.219988in}}{\pgfqpoint{7.746597in}{2.209900in}}%
\pgfpathcurveto{\pgfqpoint{7.746597in}{2.199813in}}{\pgfqpoint{7.750604in}{2.190137in}}{\pgfqpoint{7.757737in}{2.183005in}}%
\pgfpathcurveto{\pgfqpoint{7.764870in}{2.175872in}}{\pgfqpoint{7.774546in}{2.171864in}}{\pgfqpoint{7.784633in}{2.171864in}}%
\pgfpathlineto{\pgfqpoint{7.784633in}{2.171864in}}%
\pgfpathclose%
\pgfusepath{stroke,fill}%
\end{pgfscope}%
\begin{pgfscope}%
\pgfpathrectangle{\pgfqpoint{6.818937in}{0.147348in}}{\pgfqpoint{2.735294in}{2.735294in}}%
\pgfusepath{clip}%
\pgfsetbuttcap%
\pgfsetroundjoin%
\definecolor{currentfill}{rgb}{0.839216,0.152941,0.156863}%
\pgfsetfillcolor{currentfill}%
\pgfsetfillopacity{0.680503}%
\pgfsetlinewidth{1.003750pt}%
\definecolor{currentstroke}{rgb}{0.839216,0.152941,0.156863}%
\pgfsetstrokecolor{currentstroke}%
\pgfsetstrokeopacity{0.680503}%
\pgfsetdash{}{0pt}%
\pgfpathmoveto{\pgfqpoint{8.674153in}{0.972903in}}%
\pgfpathcurveto{\pgfqpoint{8.684241in}{0.972903in}}{\pgfqpoint{8.693916in}{0.976910in}}{\pgfqpoint{8.701049in}{0.984043in}}%
\pgfpathcurveto{\pgfqpoint{8.708182in}{0.991176in}}{\pgfqpoint{8.712190in}{1.000852in}}{\pgfqpoint{8.712190in}{1.010939in}}%
\pgfpathcurveto{\pgfqpoint{8.712190in}{1.021026in}}{\pgfqpoint{8.708182in}{1.030702in}}{\pgfqpoint{8.701049in}{1.037835in}}%
\pgfpathcurveto{\pgfqpoint{8.693916in}{1.044968in}}{\pgfqpoint{8.684241in}{1.048975in}}{\pgfqpoint{8.674153in}{1.048975in}}%
\pgfpathcurveto{\pgfqpoint{8.664066in}{1.048975in}}{\pgfqpoint{8.654390in}{1.044968in}}{\pgfqpoint{8.647258in}{1.037835in}}%
\pgfpathcurveto{\pgfqpoint{8.640125in}{1.030702in}}{\pgfqpoint{8.636117in}{1.021026in}}{\pgfqpoint{8.636117in}{1.010939in}}%
\pgfpathcurveto{\pgfqpoint{8.636117in}{1.000852in}}{\pgfqpoint{8.640125in}{0.991176in}}{\pgfqpoint{8.647258in}{0.984043in}}%
\pgfpathcurveto{\pgfqpoint{8.654390in}{0.976910in}}{\pgfqpoint{8.664066in}{0.972903in}}{\pgfqpoint{8.674153in}{0.972903in}}%
\pgfpathlineto{\pgfqpoint{8.674153in}{0.972903in}}%
\pgfpathclose%
\pgfusepath{stroke,fill}%
\end{pgfscope}%
\begin{pgfscope}%
\pgfpathrectangle{\pgfqpoint{6.818937in}{0.147348in}}{\pgfqpoint{2.735294in}{2.735294in}}%
\pgfusepath{clip}%
\pgfsetbuttcap%
\pgfsetroundjoin%
\definecolor{currentfill}{rgb}{0.839216,0.152941,0.156863}%
\pgfsetfillcolor{currentfill}%
\pgfsetfillopacity{0.795933}%
\pgfsetlinewidth{1.003750pt}%
\definecolor{currentstroke}{rgb}{0.839216,0.152941,0.156863}%
\pgfsetstrokecolor{currentstroke}%
\pgfsetstrokeopacity{0.795933}%
\pgfsetdash{}{0pt}%
\pgfpathmoveto{\pgfqpoint{7.340626in}{1.623661in}}%
\pgfpathcurveto{\pgfqpoint{7.350714in}{1.623661in}}{\pgfqpoint{7.360389in}{1.627668in}}{\pgfqpoint{7.367522in}{1.634801in}}%
\pgfpathcurveto{\pgfqpoint{7.374655in}{1.641934in}}{\pgfqpoint{7.378663in}{1.651609in}}{\pgfqpoint{7.378663in}{1.661697in}}%
\pgfpathcurveto{\pgfqpoint{7.378663in}{1.671784in}}{\pgfqpoint{7.374655in}{1.681460in}}{\pgfqpoint{7.367522in}{1.688593in}}%
\pgfpathcurveto{\pgfqpoint{7.360389in}{1.695725in}}{\pgfqpoint{7.350714in}{1.699733in}}{\pgfqpoint{7.340626in}{1.699733in}}%
\pgfpathcurveto{\pgfqpoint{7.330539in}{1.699733in}}{\pgfqpoint{7.320863in}{1.695725in}}{\pgfqpoint{7.313731in}{1.688593in}}%
\pgfpathcurveto{\pgfqpoint{7.306598in}{1.681460in}}{\pgfqpoint{7.302590in}{1.671784in}}{\pgfqpoint{7.302590in}{1.661697in}}%
\pgfpathcurveto{\pgfqpoint{7.302590in}{1.651609in}}{\pgfqpoint{7.306598in}{1.641934in}}{\pgfqpoint{7.313731in}{1.634801in}}%
\pgfpathcurveto{\pgfqpoint{7.320863in}{1.627668in}}{\pgfqpoint{7.330539in}{1.623661in}}{\pgfqpoint{7.340626in}{1.623661in}}%
\pgfpathlineto{\pgfqpoint{7.340626in}{1.623661in}}%
\pgfpathclose%
\pgfusepath{stroke,fill}%
\end{pgfscope}%
\begin{pgfscope}%
\pgfpathrectangle{\pgfqpoint{6.818937in}{0.147348in}}{\pgfqpoint{2.735294in}{2.735294in}}%
\pgfusepath{clip}%
\pgfsetbuttcap%
\pgfsetroundjoin%
\definecolor{currentfill}{rgb}{0.839216,0.152941,0.156863}%
\pgfsetfillcolor{currentfill}%
\pgfsetfillopacity{0.802324}%
\pgfsetlinewidth{1.003750pt}%
\definecolor{currentstroke}{rgb}{0.839216,0.152941,0.156863}%
\pgfsetstrokecolor{currentstroke}%
\pgfsetstrokeopacity{0.802324}%
\pgfsetdash{}{0pt}%
\pgfpathmoveto{\pgfqpoint{9.019142in}{1.511610in}}%
\pgfpathcurveto{\pgfqpoint{9.029229in}{1.511610in}}{\pgfqpoint{9.038905in}{1.515617in}}{\pgfqpoint{9.046038in}{1.522750in}}%
\pgfpathcurveto{\pgfqpoint{9.053171in}{1.529883in}}{\pgfqpoint{9.057178in}{1.539559in}}{\pgfqpoint{9.057178in}{1.549646in}}%
\pgfpathcurveto{\pgfqpoint{9.057178in}{1.559733in}}{\pgfqpoint{9.053171in}{1.569409in}}{\pgfqpoint{9.046038in}{1.576542in}}%
\pgfpathcurveto{\pgfqpoint{9.038905in}{1.583675in}}{\pgfqpoint{9.029229in}{1.587682in}}{\pgfqpoint{9.019142in}{1.587682in}}%
\pgfpathcurveto{\pgfqpoint{9.009055in}{1.587682in}}{\pgfqpoint{8.999379in}{1.583675in}}{\pgfqpoint{8.992246in}{1.576542in}}%
\pgfpathcurveto{\pgfqpoint{8.985114in}{1.569409in}}{\pgfqpoint{8.981106in}{1.559733in}}{\pgfqpoint{8.981106in}{1.549646in}}%
\pgfpathcurveto{\pgfqpoint{8.981106in}{1.539559in}}{\pgfqpoint{8.985114in}{1.529883in}}{\pgfqpoint{8.992246in}{1.522750in}}%
\pgfpathcurveto{\pgfqpoint{8.999379in}{1.515617in}}{\pgfqpoint{9.009055in}{1.511610in}}{\pgfqpoint{9.019142in}{1.511610in}}%
\pgfpathlineto{\pgfqpoint{9.019142in}{1.511610in}}%
\pgfpathclose%
\pgfusepath{stroke,fill}%
\end{pgfscope}%
\begin{pgfscope}%
\pgfpathrectangle{\pgfqpoint{6.818937in}{0.147348in}}{\pgfqpoint{2.735294in}{2.735294in}}%
\pgfusepath{clip}%
\pgfsetbuttcap%
\pgfsetroundjoin%
\definecolor{currentfill}{rgb}{0.839216,0.152941,0.156863}%
\pgfsetfillcolor{currentfill}%
\pgfsetfillopacity{0.837755}%
\pgfsetlinewidth{1.003750pt}%
\definecolor{currentstroke}{rgb}{0.839216,0.152941,0.156863}%
\pgfsetstrokecolor{currentstroke}%
\pgfsetstrokeopacity{0.837755}%
\pgfsetdash{}{0pt}%
\pgfpathmoveto{\pgfqpoint{8.690520in}{0.950898in}}%
\pgfpathcurveto{\pgfqpoint{8.700607in}{0.950898in}}{\pgfqpoint{8.710282in}{0.954905in}}{\pgfqpoint{8.717415in}{0.962038in}}%
\pgfpathcurveto{\pgfqpoint{8.724548in}{0.969171in}}{\pgfqpoint{8.728556in}{0.978846in}}{\pgfqpoint{8.728556in}{0.988934in}}%
\pgfpathcurveto{\pgfqpoint{8.728556in}{0.999021in}}{\pgfqpoint{8.724548in}{1.008697in}}{\pgfqpoint{8.717415in}{1.015830in}}%
\pgfpathcurveto{\pgfqpoint{8.710282in}{1.022962in}}{\pgfqpoint{8.700607in}{1.026970in}}{\pgfqpoint{8.690520in}{1.026970in}}%
\pgfpathcurveto{\pgfqpoint{8.680432in}{1.026970in}}{\pgfqpoint{8.670757in}{1.022962in}}{\pgfqpoint{8.663624in}{1.015830in}}%
\pgfpathcurveto{\pgfqpoint{8.656491in}{1.008697in}}{\pgfqpoint{8.652483in}{0.999021in}}{\pgfqpoint{8.652483in}{0.988934in}}%
\pgfpathcurveto{\pgfqpoint{8.652483in}{0.978846in}}{\pgfqpoint{8.656491in}{0.969171in}}{\pgfqpoint{8.663624in}{0.962038in}}%
\pgfpathcurveto{\pgfqpoint{8.670757in}{0.954905in}}{\pgfqpoint{8.680432in}{0.950898in}}{\pgfqpoint{8.690520in}{0.950898in}}%
\pgfpathlineto{\pgfqpoint{8.690520in}{0.950898in}}%
\pgfpathclose%
\pgfusepath{stroke,fill}%
\end{pgfscope}%
\begin{pgfscope}%
\pgfpathrectangle{\pgfqpoint{6.818937in}{0.147348in}}{\pgfqpoint{2.735294in}{2.735294in}}%
\pgfusepath{clip}%
\pgfsetbuttcap%
\pgfsetroundjoin%
\definecolor{currentfill}{rgb}{0.839216,0.152941,0.156863}%
\pgfsetfillcolor{currentfill}%
\pgfsetfillopacity{0.913537}%
\pgfsetlinewidth{1.003750pt}%
\definecolor{currentstroke}{rgb}{0.839216,0.152941,0.156863}%
\pgfsetstrokecolor{currentstroke}%
\pgfsetstrokeopacity{0.913537}%
\pgfsetdash{}{0pt}%
\pgfpathmoveto{\pgfqpoint{8.852391in}{0.996845in}}%
\pgfpathcurveto{\pgfqpoint{8.862478in}{0.996845in}}{\pgfqpoint{8.872154in}{1.000853in}}{\pgfqpoint{8.879286in}{1.007986in}}%
\pgfpathcurveto{\pgfqpoint{8.886419in}{1.015119in}}{\pgfqpoint{8.890427in}{1.024794in}}{\pgfqpoint{8.890427in}{1.034882in}}%
\pgfpathcurveto{\pgfqpoint{8.890427in}{1.044969in}}{\pgfqpoint{8.886419in}{1.054645in}}{\pgfqpoint{8.879286in}{1.061777in}}%
\pgfpathcurveto{\pgfqpoint{8.872154in}{1.068910in}}{\pgfqpoint{8.862478in}{1.072918in}}{\pgfqpoint{8.852391in}{1.072918in}}%
\pgfpathcurveto{\pgfqpoint{8.842303in}{1.072918in}}{\pgfqpoint{8.832628in}{1.068910in}}{\pgfqpoint{8.825495in}{1.061777in}}%
\pgfpathcurveto{\pgfqpoint{8.818362in}{1.054645in}}{\pgfqpoint{8.814354in}{1.044969in}}{\pgfqpoint{8.814354in}{1.034882in}}%
\pgfpathcurveto{\pgfqpoint{8.814354in}{1.024794in}}{\pgfqpoint{8.818362in}{1.015119in}}{\pgfqpoint{8.825495in}{1.007986in}}%
\pgfpathcurveto{\pgfqpoint{8.832628in}{1.000853in}}{\pgfqpoint{8.842303in}{0.996845in}}{\pgfqpoint{8.852391in}{0.996845in}}%
\pgfpathlineto{\pgfqpoint{8.852391in}{0.996845in}}%
\pgfpathclose%
\pgfusepath{stroke,fill}%
\end{pgfscope}%
\begin{pgfscope}%
\pgfpathrectangle{\pgfqpoint{6.818937in}{0.147348in}}{\pgfqpoint{2.735294in}{2.735294in}}%
\pgfusepath{clip}%
\pgfsetbuttcap%
\pgfsetroundjoin%
\definecolor{currentfill}{rgb}{0.839216,0.152941,0.156863}%
\pgfsetfillcolor{currentfill}%
\pgfsetlinewidth{1.003750pt}%
\definecolor{currentstroke}{rgb}{0.839216,0.152941,0.156863}%
\pgfsetstrokecolor{currentstroke}%
\pgfsetdash{}{0pt}%
\pgfpathmoveto{\pgfqpoint{8.834855in}{0.927831in}}%
\pgfpathcurveto{\pgfqpoint{8.844942in}{0.927831in}}{\pgfqpoint{8.854618in}{0.931838in}}{\pgfqpoint{8.861751in}{0.938971in}}%
\pgfpathcurveto{\pgfqpoint{8.868883in}{0.946104in}}{\pgfqpoint{8.872891in}{0.955779in}}{\pgfqpoint{8.872891in}{0.965867in}}%
\pgfpathcurveto{\pgfqpoint{8.872891in}{0.975954in}}{\pgfqpoint{8.868883in}{0.985630in}}{\pgfqpoint{8.861751in}{0.992763in}}%
\pgfpathcurveto{\pgfqpoint{8.854618in}{0.999895in}}{\pgfqpoint{8.844942in}{1.003903in}}{\pgfqpoint{8.834855in}{1.003903in}}%
\pgfpathcurveto{\pgfqpoint{8.824768in}{1.003903in}}{\pgfqpoint{8.815092in}{0.999895in}}{\pgfqpoint{8.807959in}{0.992763in}}%
\pgfpathcurveto{\pgfqpoint{8.800826in}{0.985630in}}{\pgfqpoint{8.796819in}{0.975954in}}{\pgfqpoint{8.796819in}{0.965867in}}%
\pgfpathcurveto{\pgfqpoint{8.796819in}{0.955779in}}{\pgfqpoint{8.800826in}{0.946104in}}{\pgfqpoint{8.807959in}{0.938971in}}%
\pgfpathcurveto{\pgfqpoint{8.815092in}{0.931838in}}{\pgfqpoint{8.824768in}{0.927831in}}{\pgfqpoint{8.834855in}{0.927831in}}%
\pgfpathlineto{\pgfqpoint{8.834855in}{0.927831in}}%
\pgfpathclose%
\pgfusepath{stroke,fill}%
\end{pgfscope}%
\begin{pgfscope}%
\pgfpathrectangle{\pgfqpoint{6.818937in}{0.147348in}}{\pgfqpoint{2.735294in}{2.735294in}}%
\pgfusepath{clip}%
\pgfsetbuttcap%
\pgfsetroundjoin%
\definecolor{currentfill}{rgb}{0.074668,0.271519,0.074668}%
\pgfsetfillcolor{currentfill}%
\pgfsetfillopacity{0.200000}%
\pgfsetlinewidth{0.000000pt}%
\definecolor{currentstroke}{rgb}{0.000000,0.000000,0.000000}%
\pgfsetstrokecolor{currentstroke}%
\pgfsetdash{}{0pt}%
\pgfpathmoveto{\pgfqpoint{8.527601in}{1.505058in}}%
\pgfpathlineto{\pgfqpoint{8.375641in}{1.283420in}}%
\pgfpathlineto{\pgfqpoint{8.223548in}{1.496830in}}%
\pgfpathlineto{\pgfqpoint{8.527601in}{1.505058in}}%
\pgfpathclose%
\pgfusepath{fill}%
\end{pgfscope}%
\begin{pgfscope}%
\pgfpathrectangle{\pgfqpoint{6.818937in}{0.147348in}}{\pgfqpoint{2.735294in}{2.735294in}}%
\pgfusepath{clip}%
\pgfsetbuttcap%
\pgfsetroundjoin%
\definecolor{currentfill}{rgb}{0.074668,0.271519,0.074668}%
\pgfsetfillcolor{currentfill}%
\pgfsetfillopacity{0.200000}%
\pgfsetlinewidth{0.000000pt}%
\definecolor{currentstroke}{rgb}{0.000000,0.000000,0.000000}%
\pgfsetstrokecolor{currentstroke}%
\pgfsetdash{}{0pt}%
\pgfpathmoveto{\pgfqpoint{8.223548in}{1.496830in}}%
\pgfpathlineto{\pgfqpoint{8.071454in}{1.283420in}}%
\pgfpathlineto{\pgfqpoint{7.919495in}{1.505058in}}%
\pgfpathlineto{\pgfqpoint{8.223548in}{1.496830in}}%
\pgfpathclose%
\pgfusepath{fill}%
\end{pgfscope}%
\begin{pgfscope}%
\pgfpathrectangle{\pgfqpoint{6.818937in}{0.147348in}}{\pgfqpoint{2.735294in}{2.735294in}}%
\pgfusepath{clip}%
\pgfsetbuttcap%
\pgfsetroundjoin%
\definecolor{currentfill}{rgb}{0.086258,0.313666,0.086258}%
\pgfsetfillcolor{currentfill}%
\pgfsetfillopacity{0.200000}%
\pgfsetlinewidth{0.000000pt}%
\definecolor{currentstroke}{rgb}{0.000000,0.000000,0.000000}%
\pgfsetstrokecolor{currentstroke}%
\pgfsetdash{}{0pt}%
\pgfpathmoveto{\pgfqpoint{7.919495in}{1.505058in}}%
\pgfpathlineto{\pgfqpoint{8.223548in}{1.947612in}}%
\pgfpathlineto{\pgfqpoint{8.223548in}{1.496830in}}%
\pgfpathlineto{\pgfqpoint{7.919495in}{1.505058in}}%
\pgfpathclose%
\pgfusepath{fill}%
\end{pgfscope}%
\begin{pgfscope}%
\pgfpathrectangle{\pgfqpoint{6.818937in}{0.147348in}}{\pgfqpoint{2.735294in}{2.735294in}}%
\pgfusepath{clip}%
\pgfsetbuttcap%
\pgfsetroundjoin%
\definecolor{currentfill}{rgb}{0.086258,0.313666,0.086258}%
\pgfsetfillcolor{currentfill}%
\pgfsetfillopacity{0.200000}%
\pgfsetlinewidth{0.000000pt}%
\definecolor{currentstroke}{rgb}{0.000000,0.000000,0.000000}%
\pgfsetstrokecolor{currentstroke}%
\pgfsetdash{}{0pt}%
\pgfpathmoveto{\pgfqpoint{8.223548in}{1.496830in}}%
\pgfpathlineto{\pgfqpoint{8.223548in}{1.947612in}}%
\pgfpathlineto{\pgfqpoint{8.527601in}{1.505058in}}%
\pgfpathlineto{\pgfqpoint{8.223548in}{1.496830in}}%
\pgfpathclose%
\pgfusepath{fill}%
\end{pgfscope}%
\begin{pgfscope}%
\pgfpathrectangle{\pgfqpoint{6.818937in}{0.147348in}}{\pgfqpoint{2.735294in}{2.735294in}}%
\pgfusepath{clip}%
\pgfsetbuttcap%
\pgfsetroundjoin%
\definecolor{currentfill}{rgb}{0.086061,0.312950,0.086061}%
\pgfsetfillcolor{currentfill}%
\pgfsetfillopacity{0.200000}%
\pgfsetlinewidth{0.000000pt}%
\definecolor{currentstroke}{rgb}{0.000000,0.000000,0.000000}%
\pgfsetstrokecolor{currentstroke}%
\pgfsetdash{}{0pt}%
\pgfpathmoveto{\pgfqpoint{8.223548in}{1.947612in}}%
\pgfpathlineto{\pgfqpoint{7.919495in}{1.505058in}}%
\pgfpathlineto{\pgfqpoint{7.935071in}{1.947550in}}%
\pgfpathlineto{\pgfqpoint{8.223548in}{1.947612in}}%
\pgfpathclose%
\pgfusepath{fill}%
\end{pgfscope}%
\begin{pgfscope}%
\pgfpathrectangle{\pgfqpoint{6.818937in}{0.147348in}}{\pgfqpoint{2.735294in}{2.735294in}}%
\pgfusepath{clip}%
\pgfsetbuttcap%
\pgfsetroundjoin%
\definecolor{currentfill}{rgb}{0.086061,0.312950,0.086061}%
\pgfsetfillcolor{currentfill}%
\pgfsetfillopacity{0.200000}%
\pgfsetlinewidth{0.000000pt}%
\definecolor{currentstroke}{rgb}{0.000000,0.000000,0.000000}%
\pgfsetstrokecolor{currentstroke}%
\pgfsetdash{}{0pt}%
\pgfpathmoveto{\pgfqpoint{8.512025in}{1.947550in}}%
\pgfpathlineto{\pgfqpoint{8.527601in}{1.505058in}}%
\pgfpathlineto{\pgfqpoint{8.223548in}{1.947612in}}%
\pgfpathlineto{\pgfqpoint{8.512025in}{1.947550in}}%
\pgfpathclose%
\pgfusepath{fill}%
\end{pgfscope}%
\begin{pgfscope}%
\pgfpathrectangle{\pgfqpoint{6.818937in}{0.147348in}}{\pgfqpoint{2.735294in}{2.735294in}}%
\pgfusepath{clip}%
\pgfsetbuttcap%
\pgfsetroundjoin%
\definecolor{currentfill}{rgb}{0.075994,0.276341,0.075994}%
\pgfsetfillcolor{currentfill}%
\pgfsetfillopacity{0.200000}%
\pgfsetlinewidth{0.000000pt}%
\definecolor{currentstroke}{rgb}{0.000000,0.000000,0.000000}%
\pgfsetstrokecolor{currentstroke}%
\pgfsetdash{}{0pt}%
\pgfpathmoveto{\pgfqpoint{8.800986in}{1.527187in}}%
\pgfpathlineto{\pgfqpoint{8.664078in}{1.306218in}}%
\pgfpathlineto{\pgfqpoint{8.527601in}{1.505058in}}%
\pgfpathlineto{\pgfqpoint{8.800986in}{1.527187in}}%
\pgfpathclose%
\pgfusepath{fill}%
\end{pgfscope}%
\begin{pgfscope}%
\pgfpathrectangle{\pgfqpoint{6.818937in}{0.147348in}}{\pgfqpoint{2.735294in}{2.735294in}}%
\pgfusepath{clip}%
\pgfsetbuttcap%
\pgfsetroundjoin%
\definecolor{currentfill}{rgb}{0.075994,0.276341,0.075994}%
\pgfsetfillcolor{currentfill}%
\pgfsetfillopacity{0.200000}%
\pgfsetlinewidth{0.000000pt}%
\definecolor{currentstroke}{rgb}{0.000000,0.000000,0.000000}%
\pgfsetstrokecolor{currentstroke}%
\pgfsetdash{}{0pt}%
\pgfpathmoveto{\pgfqpoint{7.919495in}{1.505058in}}%
\pgfpathlineto{\pgfqpoint{7.783018in}{1.306218in}}%
\pgfpathlineto{\pgfqpoint{7.646110in}{1.527187in}}%
\pgfpathlineto{\pgfqpoint{7.919495in}{1.505058in}}%
\pgfpathclose%
\pgfusepath{fill}%
\end{pgfscope}%
\begin{pgfscope}%
\pgfpathrectangle{\pgfqpoint{6.818937in}{0.147348in}}{\pgfqpoint{2.735294in}{2.735294in}}%
\pgfusepath{clip}%
\pgfsetbuttcap%
\pgfsetroundjoin%
\definecolor{currentfill}{rgb}{0.087398,0.317812,0.087398}%
\pgfsetfillcolor{currentfill}%
\pgfsetfillopacity{0.200000}%
\pgfsetlinewidth{0.000000pt}%
\definecolor{currentstroke}{rgb}{0.000000,0.000000,0.000000}%
\pgfsetstrokecolor{currentstroke}%
\pgfsetdash{}{0pt}%
\pgfpathmoveto{\pgfqpoint{8.527601in}{1.505058in}}%
\pgfpathlineto{\pgfqpoint{8.512025in}{1.947550in}}%
\pgfpathlineto{\pgfqpoint{8.800986in}{1.527187in}}%
\pgfpathlineto{\pgfqpoint{8.527601in}{1.505058in}}%
\pgfpathclose%
\pgfusepath{fill}%
\end{pgfscope}%
\begin{pgfscope}%
\pgfpathrectangle{\pgfqpoint{6.818937in}{0.147348in}}{\pgfqpoint{2.735294in}{2.735294in}}%
\pgfusepath{clip}%
\pgfsetbuttcap%
\pgfsetroundjoin%
\definecolor{currentfill}{rgb}{0.087398,0.317812,0.087398}%
\pgfsetfillcolor{currentfill}%
\pgfsetfillopacity{0.200000}%
\pgfsetlinewidth{0.000000pt}%
\definecolor{currentstroke}{rgb}{0.000000,0.000000,0.000000}%
\pgfsetstrokecolor{currentstroke}%
\pgfsetdash{}{0pt}%
\pgfpathmoveto{\pgfqpoint{7.646110in}{1.527187in}}%
\pgfpathlineto{\pgfqpoint{7.935071in}{1.947550in}}%
\pgfpathlineto{\pgfqpoint{7.919495in}{1.505058in}}%
\pgfpathlineto{\pgfqpoint{7.646110in}{1.527187in}}%
\pgfpathclose%
\pgfusepath{fill}%
\end{pgfscope}%
\begin{pgfscope}%
\pgfpathrectangle{\pgfqpoint{6.818937in}{0.147348in}}{\pgfqpoint{2.735294in}{2.735294in}}%
\pgfusepath{clip}%
\pgfsetbuttcap%
\pgfsetroundjoin%
\definecolor{currentfill}{rgb}{0.070209,0.255305,0.070209}%
\pgfsetfillcolor{currentfill}%
\pgfsetfillopacity{0.200000}%
\pgfsetlinewidth{0.000000pt}%
\definecolor{currentstroke}{rgb}{0.000000,0.000000,0.000000}%
\pgfsetstrokecolor{currentstroke}%
\pgfsetdash{}{0pt}%
\pgfpathmoveto{\pgfqpoint{8.625616in}{0.971623in}}%
\pgfpathlineto{\pgfqpoint{8.375641in}{1.283420in}}%
\pgfpathlineto{\pgfqpoint{8.527601in}{1.505058in}}%
\pgfpathlineto{\pgfqpoint{8.625616in}{0.971623in}}%
\pgfpathclose%
\pgfusepath{fill}%
\end{pgfscope}%
\begin{pgfscope}%
\pgfpathrectangle{\pgfqpoint{6.818937in}{0.147348in}}{\pgfqpoint{2.735294in}{2.735294in}}%
\pgfusepath{clip}%
\pgfsetbuttcap%
\pgfsetroundjoin%
\definecolor{currentfill}{rgb}{0.070209,0.255305,0.070209}%
\pgfsetfillcolor{currentfill}%
\pgfsetfillopacity{0.200000}%
\pgfsetlinewidth{0.000000pt}%
\definecolor{currentstroke}{rgb}{0.000000,0.000000,0.000000}%
\pgfsetstrokecolor{currentstroke}%
\pgfsetdash{}{0pt}%
\pgfpathmoveto{\pgfqpoint{7.919495in}{1.505058in}}%
\pgfpathlineto{\pgfqpoint{8.071454in}{1.283420in}}%
\pgfpathlineto{\pgfqpoint{7.821480in}{0.971623in}}%
\pgfpathlineto{\pgfqpoint{7.919495in}{1.505058in}}%
\pgfpathclose%
\pgfusepath{fill}%
\end{pgfscope}%
\begin{pgfscope}%
\pgfpathrectangle{\pgfqpoint{6.818937in}{0.147348in}}{\pgfqpoint{2.735294in}{2.735294in}}%
\pgfusepath{clip}%
\pgfsetbuttcap%
\pgfsetroundjoin%
\definecolor{currentfill}{rgb}{0.098306,0.357475,0.098306}%
\pgfsetfillcolor{currentfill}%
\pgfsetfillopacity{0.200000}%
\pgfsetlinewidth{0.000000pt}%
\definecolor{currentstroke}{rgb}{0.000000,0.000000,0.000000}%
\pgfsetstrokecolor{currentstroke}%
\pgfsetdash{}{0pt}%
\pgfpathmoveto{\pgfqpoint{8.223548in}{1.947612in}}%
\pgfpathlineto{\pgfqpoint{8.361157in}{2.147735in}}%
\pgfpathlineto{\pgfqpoint{8.512025in}{1.947550in}}%
\pgfpathlineto{\pgfqpoint{8.223548in}{1.947612in}}%
\pgfpathclose%
\pgfusepath{fill}%
\end{pgfscope}%
\begin{pgfscope}%
\pgfpathrectangle{\pgfqpoint{6.818937in}{0.147348in}}{\pgfqpoint{2.735294in}{2.735294in}}%
\pgfusepath{clip}%
\pgfsetbuttcap%
\pgfsetroundjoin%
\definecolor{currentfill}{rgb}{0.098306,0.357475,0.098306}%
\pgfsetfillcolor{currentfill}%
\pgfsetfillopacity{0.200000}%
\pgfsetlinewidth{0.000000pt}%
\definecolor{currentstroke}{rgb}{0.000000,0.000000,0.000000}%
\pgfsetstrokecolor{currentstroke}%
\pgfsetdash{}{0pt}%
\pgfpathmoveto{\pgfqpoint{7.935071in}{1.947550in}}%
\pgfpathlineto{\pgfqpoint{8.085938in}{2.147735in}}%
\pgfpathlineto{\pgfqpoint{8.223548in}{1.947612in}}%
\pgfpathlineto{\pgfqpoint{7.935071in}{1.947550in}}%
\pgfpathclose%
\pgfusepath{fill}%
\end{pgfscope}%
\begin{pgfscope}%
\pgfpathrectangle{\pgfqpoint{6.818937in}{0.147348in}}{\pgfqpoint{2.735294in}{2.735294in}}%
\pgfusepath{clip}%
\pgfsetbuttcap%
\pgfsetroundjoin%
\definecolor{currentfill}{rgb}{0.066446,0.241622,0.066446}%
\pgfsetfillcolor{currentfill}%
\pgfsetfillopacity{0.200000}%
\pgfsetlinewidth{0.000000pt}%
\definecolor{currentstroke}{rgb}{0.000000,0.000000,0.000000}%
\pgfsetstrokecolor{currentstroke}%
\pgfsetdash{}{0pt}%
\pgfpathmoveto{\pgfqpoint{8.375641in}{1.283420in}}%
\pgfpathlineto{\pgfqpoint{8.223548in}{0.822079in}}%
\pgfpathlineto{\pgfqpoint{8.223548in}{1.496830in}}%
\pgfpathlineto{\pgfqpoint{8.375641in}{1.283420in}}%
\pgfpathclose%
\pgfusepath{fill}%
\end{pgfscope}%
\begin{pgfscope}%
\pgfpathrectangle{\pgfqpoint{6.818937in}{0.147348in}}{\pgfqpoint{2.735294in}{2.735294in}}%
\pgfusepath{clip}%
\pgfsetbuttcap%
\pgfsetroundjoin%
\definecolor{currentfill}{rgb}{0.066446,0.241622,0.066446}%
\pgfsetfillcolor{currentfill}%
\pgfsetfillopacity{0.200000}%
\pgfsetlinewidth{0.000000pt}%
\definecolor{currentstroke}{rgb}{0.000000,0.000000,0.000000}%
\pgfsetstrokecolor{currentstroke}%
\pgfsetdash{}{0pt}%
\pgfpathmoveto{\pgfqpoint{8.223548in}{1.496830in}}%
\pgfpathlineto{\pgfqpoint{8.223548in}{0.822079in}}%
\pgfpathlineto{\pgfqpoint{8.071454in}{1.283420in}}%
\pgfpathlineto{\pgfqpoint{8.223548in}{1.496830in}}%
\pgfpathclose%
\pgfusepath{fill}%
\end{pgfscope}%
\begin{pgfscope}%
\pgfpathrectangle{\pgfqpoint{6.818937in}{0.147348in}}{\pgfqpoint{2.735294in}{2.735294in}}%
\pgfusepath{clip}%
\pgfsetbuttcap%
\pgfsetroundjoin%
\definecolor{currentfill}{rgb}{0.065035,0.236492,0.065035}%
\pgfsetfillcolor{currentfill}%
\pgfsetfillopacity{0.200000}%
\pgfsetlinewidth{0.000000pt}%
\definecolor{currentstroke}{rgb}{0.000000,0.000000,0.000000}%
\pgfsetstrokecolor{currentstroke}%
\pgfsetdash{}{0pt}%
\pgfpathmoveto{\pgfqpoint{8.625616in}{0.971623in}}%
\pgfpathlineto{\pgfqpoint{8.527601in}{1.505058in}}%
\pgfpathlineto{\pgfqpoint{8.664078in}{1.306218in}}%
\pgfpathlineto{\pgfqpoint{8.625616in}{0.971623in}}%
\pgfpathclose%
\pgfusepath{fill}%
\end{pgfscope}%
\begin{pgfscope}%
\pgfpathrectangle{\pgfqpoint{6.818937in}{0.147348in}}{\pgfqpoint{2.735294in}{2.735294in}}%
\pgfusepath{clip}%
\pgfsetbuttcap%
\pgfsetroundjoin%
\definecolor{currentfill}{rgb}{0.065035,0.236492,0.065035}%
\pgfsetfillcolor{currentfill}%
\pgfsetfillopacity{0.200000}%
\pgfsetlinewidth{0.000000pt}%
\definecolor{currentstroke}{rgb}{0.000000,0.000000,0.000000}%
\pgfsetstrokecolor{currentstroke}%
\pgfsetdash{}{0pt}%
\pgfpathmoveto{\pgfqpoint{7.783018in}{1.306218in}}%
\pgfpathlineto{\pgfqpoint{7.919495in}{1.505058in}}%
\pgfpathlineto{\pgfqpoint{7.821480in}{0.971623in}}%
\pgfpathlineto{\pgfqpoint{7.783018in}{1.306218in}}%
\pgfpathclose%
\pgfusepath{fill}%
\end{pgfscope}%
\begin{pgfscope}%
\pgfpathrectangle{\pgfqpoint{6.818937in}{0.147348in}}{\pgfqpoint{2.735294in}{2.735294in}}%
\pgfusepath{clip}%
\pgfsetbuttcap%
\pgfsetroundjoin%
\definecolor{currentfill}{rgb}{0.101677,0.369734,0.101677}%
\pgfsetfillcolor{currentfill}%
\pgfsetfillopacity{0.200000}%
\pgfsetlinewidth{0.000000pt}%
\definecolor{currentstroke}{rgb}{0.000000,0.000000,0.000000}%
\pgfsetstrokecolor{currentstroke}%
\pgfsetdash{}{0pt}%
\pgfpathmoveto{\pgfqpoint{8.223548in}{1.947612in}}%
\pgfpathlineto{\pgfqpoint{8.085938in}{2.147735in}}%
\pgfpathlineto{\pgfqpoint{8.361157in}{2.147735in}}%
\pgfpathlineto{\pgfqpoint{8.223548in}{1.947612in}}%
\pgfpathclose%
\pgfusepath{fill}%
\end{pgfscope}%
\begin{pgfscope}%
\pgfpathrectangle{\pgfqpoint{6.818937in}{0.147348in}}{\pgfqpoint{2.735294in}{2.735294in}}%
\pgfusepath{clip}%
\pgfsetbuttcap%
\pgfsetroundjoin%
\definecolor{currentfill}{rgb}{0.101759,0.370033,0.101759}%
\pgfsetfillcolor{currentfill}%
\pgfsetfillopacity{0.200000}%
\pgfsetlinewidth{0.000000pt}%
\definecolor{currentstroke}{rgb}{0.000000,0.000000,0.000000}%
\pgfsetstrokecolor{currentstroke}%
\pgfsetdash{}{0pt}%
\pgfpathmoveto{\pgfqpoint{8.512025in}{1.947550in}}%
\pgfpathlineto{\pgfqpoint{8.361157in}{2.147735in}}%
\pgfpathlineto{\pgfqpoint{8.624680in}{2.141959in}}%
\pgfpathlineto{\pgfqpoint{8.512025in}{1.947550in}}%
\pgfpathclose%
\pgfusepath{fill}%
\end{pgfscope}%
\begin{pgfscope}%
\pgfpathrectangle{\pgfqpoint{6.818937in}{0.147348in}}{\pgfqpoint{2.735294in}{2.735294in}}%
\pgfusepath{clip}%
\pgfsetbuttcap%
\pgfsetroundjoin%
\definecolor{currentfill}{rgb}{0.101759,0.370033,0.101759}%
\pgfsetfillcolor{currentfill}%
\pgfsetfillopacity{0.200000}%
\pgfsetlinewidth{0.000000pt}%
\definecolor{currentstroke}{rgb}{0.000000,0.000000,0.000000}%
\pgfsetstrokecolor{currentstroke}%
\pgfsetdash{}{0pt}%
\pgfpathmoveto{\pgfqpoint{7.822416in}{2.141959in}}%
\pgfpathlineto{\pgfqpoint{8.085938in}{2.147735in}}%
\pgfpathlineto{\pgfqpoint{7.935071in}{1.947550in}}%
\pgfpathlineto{\pgfqpoint{7.822416in}{2.141959in}}%
\pgfpathclose%
\pgfusepath{fill}%
\end{pgfscope}%
\begin{pgfscope}%
\pgfpathrectangle{\pgfqpoint{6.818937in}{0.147348in}}{\pgfqpoint{2.735294in}{2.735294in}}%
\pgfusepath{clip}%
\pgfsetbuttcap%
\pgfsetroundjoin%
\definecolor{currentfill}{rgb}{0.091915,0.334238,0.091915}%
\pgfsetfillcolor{currentfill}%
\pgfsetfillopacity{0.200000}%
\pgfsetlinewidth{0.000000pt}%
\definecolor{currentstroke}{rgb}{0.000000,0.000000,0.000000}%
\pgfsetstrokecolor{currentstroke}%
\pgfsetdash{}{0pt}%
\pgfpathmoveto{\pgfqpoint{8.800986in}{1.527187in}}%
\pgfpathlineto{\pgfqpoint{8.512025in}{1.947550in}}%
\pgfpathlineto{\pgfqpoint{8.857732in}{2.131775in}}%
\pgfpathlineto{\pgfqpoint{8.800986in}{1.527187in}}%
\pgfpathclose%
\pgfusepath{fill}%
\end{pgfscope}%
\begin{pgfscope}%
\pgfpathrectangle{\pgfqpoint{6.818937in}{0.147348in}}{\pgfqpoint{2.735294in}{2.735294in}}%
\pgfusepath{clip}%
\pgfsetbuttcap%
\pgfsetroundjoin%
\definecolor{currentfill}{rgb}{0.091915,0.334238,0.091915}%
\pgfsetfillcolor{currentfill}%
\pgfsetfillopacity{0.200000}%
\pgfsetlinewidth{0.000000pt}%
\definecolor{currentstroke}{rgb}{0.000000,0.000000,0.000000}%
\pgfsetstrokecolor{currentstroke}%
\pgfsetdash{}{0pt}%
\pgfpathmoveto{\pgfqpoint{7.589364in}{2.131775in}}%
\pgfpathlineto{\pgfqpoint{7.935071in}{1.947550in}}%
\pgfpathlineto{\pgfqpoint{7.646110in}{1.527187in}}%
\pgfpathlineto{\pgfqpoint{7.589364in}{2.131775in}}%
\pgfpathclose%
\pgfusepath{fill}%
\end{pgfscope}%
\begin{pgfscope}%
\pgfpathrectangle{\pgfqpoint{6.818937in}{0.147348in}}{\pgfqpoint{2.735294in}{2.735294in}}%
\pgfusepath{clip}%
\pgfsetbuttcap%
\pgfsetroundjoin%
\definecolor{currentfill}{rgb}{0.073593,0.267612,0.073593}%
\pgfsetfillcolor{currentfill}%
\pgfsetfillopacity{0.200000}%
\pgfsetlinewidth{0.000000pt}%
\definecolor{currentstroke}{rgb}{0.000000,0.000000,0.000000}%
\pgfsetstrokecolor{currentstroke}%
\pgfsetdash{}{0pt}%
\pgfpathmoveto{\pgfqpoint{8.859135in}{1.021716in}}%
\pgfpathlineto{\pgfqpoint{8.664078in}{1.306218in}}%
\pgfpathlineto{\pgfqpoint{8.800986in}{1.527187in}}%
\pgfpathlineto{\pgfqpoint{8.859135in}{1.021716in}}%
\pgfpathclose%
\pgfusepath{fill}%
\end{pgfscope}%
\begin{pgfscope}%
\pgfpathrectangle{\pgfqpoint{6.818937in}{0.147348in}}{\pgfqpoint{2.735294in}{2.735294in}}%
\pgfusepath{clip}%
\pgfsetbuttcap%
\pgfsetroundjoin%
\definecolor{currentfill}{rgb}{0.073593,0.267612,0.073593}%
\pgfsetfillcolor{currentfill}%
\pgfsetfillopacity{0.200000}%
\pgfsetlinewidth{0.000000pt}%
\definecolor{currentstroke}{rgb}{0.000000,0.000000,0.000000}%
\pgfsetstrokecolor{currentstroke}%
\pgfsetdash{}{0pt}%
\pgfpathmoveto{\pgfqpoint{7.646110in}{1.527187in}}%
\pgfpathlineto{\pgfqpoint{7.783018in}{1.306218in}}%
\pgfpathlineto{\pgfqpoint{7.587961in}{1.021716in}}%
\pgfpathlineto{\pgfqpoint{7.646110in}{1.527187in}}%
\pgfpathclose%
\pgfusepath{fill}%
\end{pgfscope}%
\begin{pgfscope}%
\pgfpathrectangle{\pgfqpoint{6.818937in}{0.147348in}}{\pgfqpoint{2.735294in}{2.735294in}}%
\pgfusepath{clip}%
\pgfsetbuttcap%
\pgfsetroundjoin%
\definecolor{currentfill}{rgb}{0.065434,0.237940,0.065434}%
\pgfsetfillcolor{currentfill}%
\pgfsetfillopacity{0.200000}%
\pgfsetlinewidth{0.000000pt}%
\definecolor{currentstroke}{rgb}{0.000000,0.000000,0.000000}%
\pgfsetstrokecolor{currentstroke}%
\pgfsetdash{}{0pt}%
\pgfpathmoveto{\pgfqpoint{8.223548in}{0.822079in}}%
\pgfpathlineto{\pgfqpoint{8.375641in}{1.283420in}}%
\pgfpathlineto{\pgfqpoint{8.361488in}{0.943207in}}%
\pgfpathlineto{\pgfqpoint{8.223548in}{0.822079in}}%
\pgfpathclose%
\pgfusepath{fill}%
\end{pgfscope}%
\begin{pgfscope}%
\pgfpathrectangle{\pgfqpoint{6.818937in}{0.147348in}}{\pgfqpoint{2.735294in}{2.735294in}}%
\pgfusepath{clip}%
\pgfsetbuttcap%
\pgfsetroundjoin%
\definecolor{currentfill}{rgb}{0.065434,0.237940,0.065434}%
\pgfsetfillcolor{currentfill}%
\pgfsetfillopacity{0.200000}%
\pgfsetlinewidth{0.000000pt}%
\definecolor{currentstroke}{rgb}{0.000000,0.000000,0.000000}%
\pgfsetstrokecolor{currentstroke}%
\pgfsetdash{}{0pt}%
\pgfpathmoveto{\pgfqpoint{8.085608in}{0.943207in}}%
\pgfpathlineto{\pgfqpoint{8.071454in}{1.283420in}}%
\pgfpathlineto{\pgfqpoint{8.223548in}{0.822079in}}%
\pgfpathlineto{\pgfqpoint{8.085608in}{0.943207in}}%
\pgfpathclose%
\pgfusepath{fill}%
\end{pgfscope}%
\begin{pgfscope}%
\pgfpathrectangle{\pgfqpoint{6.818937in}{0.147348in}}{\pgfqpoint{2.735294in}{2.735294in}}%
\pgfusepath{clip}%
\pgfsetbuttcap%
\pgfsetroundjoin%
\definecolor{currentfill}{rgb}{0.067497,0.245443,0.067497}%
\pgfsetfillcolor{currentfill}%
\pgfsetfillopacity{0.200000}%
\pgfsetlinewidth{0.000000pt}%
\definecolor{currentstroke}{rgb}{0.000000,0.000000,0.000000}%
\pgfsetstrokecolor{currentstroke}%
\pgfsetdash{}{0pt}%
\pgfpathmoveto{\pgfqpoint{8.477877in}{0.836228in}}%
\pgfpathlineto{\pgfqpoint{8.361488in}{0.943207in}}%
\pgfpathlineto{\pgfqpoint{8.375641in}{1.283420in}}%
\pgfpathlineto{\pgfqpoint{8.477877in}{0.836228in}}%
\pgfpathclose%
\pgfusepath{fill}%
\end{pgfscope}%
\begin{pgfscope}%
\pgfpathrectangle{\pgfqpoint{6.818937in}{0.147348in}}{\pgfqpoint{2.735294in}{2.735294in}}%
\pgfusepath{clip}%
\pgfsetbuttcap%
\pgfsetroundjoin%
\definecolor{currentfill}{rgb}{0.067497,0.245443,0.067497}%
\pgfsetfillcolor{currentfill}%
\pgfsetfillopacity{0.200000}%
\pgfsetlinewidth{0.000000pt}%
\definecolor{currentstroke}{rgb}{0.000000,0.000000,0.000000}%
\pgfsetstrokecolor{currentstroke}%
\pgfsetdash{}{0pt}%
\pgfpathmoveto{\pgfqpoint{8.071454in}{1.283420in}}%
\pgfpathlineto{\pgfqpoint{8.085608in}{0.943207in}}%
\pgfpathlineto{\pgfqpoint{7.969219in}{0.836228in}}%
\pgfpathlineto{\pgfqpoint{8.071454in}{1.283420in}}%
\pgfpathclose%
\pgfusepath{fill}%
\end{pgfscope}%
\begin{pgfscope}%
\pgfpathrectangle{\pgfqpoint{6.818937in}{0.147348in}}{\pgfqpoint{2.735294in}{2.735294in}}%
\pgfusepath{clip}%
\pgfsetbuttcap%
\pgfsetroundjoin%
\definecolor{currentfill}{rgb}{0.097285,0.353762,0.097285}%
\pgfsetfillcolor{currentfill}%
\pgfsetfillopacity{0.200000}%
\pgfsetlinewidth{0.000000pt}%
\definecolor{currentstroke}{rgb}{0.000000,0.000000,0.000000}%
\pgfsetstrokecolor{currentstroke}%
\pgfsetdash{}{0pt}%
\pgfpathmoveto{\pgfqpoint{8.624680in}{2.141959in}}%
\pgfpathlineto{\pgfqpoint{8.857732in}{2.131775in}}%
\pgfpathlineto{\pgfqpoint{8.512025in}{1.947550in}}%
\pgfpathlineto{\pgfqpoint{8.624680in}{2.141959in}}%
\pgfpathclose%
\pgfusepath{fill}%
\end{pgfscope}%
\begin{pgfscope}%
\pgfpathrectangle{\pgfqpoint{6.818937in}{0.147348in}}{\pgfqpoint{2.735294in}{2.735294in}}%
\pgfusepath{clip}%
\pgfsetbuttcap%
\pgfsetroundjoin%
\definecolor{currentfill}{rgb}{0.097285,0.353762,0.097285}%
\pgfsetfillcolor{currentfill}%
\pgfsetfillopacity{0.200000}%
\pgfsetlinewidth{0.000000pt}%
\definecolor{currentstroke}{rgb}{0.000000,0.000000,0.000000}%
\pgfsetstrokecolor{currentstroke}%
\pgfsetdash{}{0pt}%
\pgfpathmoveto{\pgfqpoint{7.935071in}{1.947550in}}%
\pgfpathlineto{\pgfqpoint{7.589364in}{2.131775in}}%
\pgfpathlineto{\pgfqpoint{7.822416in}{2.141959in}}%
\pgfpathlineto{\pgfqpoint{7.935071in}{1.947550in}}%
\pgfpathclose%
\pgfusepath{fill}%
\end{pgfscope}%
\begin{pgfscope}%
\pgfpathrectangle{\pgfqpoint{6.818937in}{0.147348in}}{\pgfqpoint{2.735294in}{2.735294in}}%
\pgfusepath{clip}%
\pgfsetbuttcap%
\pgfsetroundjoin%
\definecolor{currentfill}{rgb}{0.060562,0.220227,0.060562}%
\pgfsetfillcolor{currentfill}%
\pgfsetfillopacity{0.200000}%
\pgfsetlinewidth{0.000000pt}%
\definecolor{currentstroke}{rgb}{0.000000,0.000000,0.000000}%
\pgfsetstrokecolor{currentstroke}%
\pgfsetdash{}{0pt}%
\pgfpathmoveto{\pgfqpoint{8.477877in}{0.836228in}}%
\pgfpathlineto{\pgfqpoint{8.375641in}{1.283420in}}%
\pgfpathlineto{\pgfqpoint{8.625616in}{0.971623in}}%
\pgfpathlineto{\pgfqpoint{8.477877in}{0.836228in}}%
\pgfpathclose%
\pgfusepath{fill}%
\end{pgfscope}%
\begin{pgfscope}%
\pgfpathrectangle{\pgfqpoint{6.818937in}{0.147348in}}{\pgfqpoint{2.735294in}{2.735294in}}%
\pgfusepath{clip}%
\pgfsetbuttcap%
\pgfsetroundjoin%
\definecolor{currentfill}{rgb}{0.060562,0.220227,0.060562}%
\pgfsetfillcolor{currentfill}%
\pgfsetfillopacity{0.200000}%
\pgfsetlinewidth{0.000000pt}%
\definecolor{currentstroke}{rgb}{0.000000,0.000000,0.000000}%
\pgfsetstrokecolor{currentstroke}%
\pgfsetdash{}{0pt}%
\pgfpathmoveto{\pgfqpoint{7.821480in}{0.971623in}}%
\pgfpathlineto{\pgfqpoint{8.071454in}{1.283420in}}%
\pgfpathlineto{\pgfqpoint{7.969219in}{0.836228in}}%
\pgfpathlineto{\pgfqpoint{7.821480in}{0.971623in}}%
\pgfpathclose%
\pgfusepath{fill}%
\end{pgfscope}%
\begin{pgfscope}%
\pgfpathrectangle{\pgfqpoint{6.818937in}{0.147348in}}{\pgfqpoint{2.735294in}{2.735294in}}%
\pgfusepath{clip}%
\pgfsetbuttcap%
\pgfsetroundjoin%
\definecolor{currentfill}{rgb}{0.092193,0.335248,0.092193}%
\pgfsetfillcolor{currentfill}%
\pgfsetfillopacity{0.200000}%
\pgfsetlinewidth{0.000000pt}%
\definecolor{currentstroke}{rgb}{0.000000,0.000000,0.000000}%
\pgfsetstrokecolor{currentstroke}%
\pgfsetdash{}{0pt}%
\pgfpathmoveto{\pgfqpoint{8.800986in}{1.527187in}}%
\pgfpathlineto{\pgfqpoint{8.857732in}{2.131775in}}%
\pgfpathlineto{\pgfqpoint{9.026747in}{1.557473in}}%
\pgfpathlineto{\pgfqpoint{8.800986in}{1.527187in}}%
\pgfpathclose%
\pgfusepath{fill}%
\end{pgfscope}%
\begin{pgfscope}%
\pgfpathrectangle{\pgfqpoint{6.818937in}{0.147348in}}{\pgfqpoint{2.735294in}{2.735294in}}%
\pgfusepath{clip}%
\pgfsetbuttcap%
\pgfsetroundjoin%
\definecolor{currentfill}{rgb}{0.092193,0.335248,0.092193}%
\pgfsetfillcolor{currentfill}%
\pgfsetfillopacity{0.200000}%
\pgfsetlinewidth{0.000000pt}%
\definecolor{currentstroke}{rgb}{0.000000,0.000000,0.000000}%
\pgfsetstrokecolor{currentstroke}%
\pgfsetdash{}{0pt}%
\pgfpathmoveto{\pgfqpoint{7.420349in}{1.557473in}}%
\pgfpathlineto{\pgfqpoint{7.589364in}{2.131775in}}%
\pgfpathlineto{\pgfqpoint{7.646110in}{1.527187in}}%
\pgfpathlineto{\pgfqpoint{7.420349in}{1.557473in}}%
\pgfpathclose%
\pgfusepath{fill}%
\end{pgfscope}%
\begin{pgfscope}%
\pgfpathrectangle{\pgfqpoint{6.818937in}{0.147348in}}{\pgfqpoint{2.735294in}{2.735294in}}%
\pgfusepath{clip}%
\pgfsetbuttcap%
\pgfsetroundjoin%
\definecolor{currentfill}{rgb}{0.111651,0.406004,0.111651}%
\pgfsetfillcolor{currentfill}%
\pgfsetfillopacity{0.200000}%
\pgfsetlinewidth{0.000000pt}%
\definecolor{currentstroke}{rgb}{0.000000,0.000000,0.000000}%
\pgfsetstrokecolor{currentstroke}%
\pgfsetdash{}{0pt}%
\pgfpathmoveto{\pgfqpoint{7.822416in}{2.141959in}}%
\pgfpathlineto{\pgfqpoint{7.734361in}{2.303109in}}%
\pgfpathlineto{\pgfqpoint{8.085938in}{2.147735in}}%
\pgfpathlineto{\pgfqpoint{7.822416in}{2.141959in}}%
\pgfpathclose%
\pgfusepath{fill}%
\end{pgfscope}%
\begin{pgfscope}%
\pgfpathrectangle{\pgfqpoint{6.818937in}{0.147348in}}{\pgfqpoint{2.735294in}{2.735294in}}%
\pgfusepath{clip}%
\pgfsetbuttcap%
\pgfsetroundjoin%
\definecolor{currentfill}{rgb}{0.111651,0.406004,0.111651}%
\pgfsetfillcolor{currentfill}%
\pgfsetfillopacity{0.200000}%
\pgfsetlinewidth{0.000000pt}%
\definecolor{currentstroke}{rgb}{0.000000,0.000000,0.000000}%
\pgfsetstrokecolor{currentstroke}%
\pgfsetdash{}{0pt}%
\pgfpathmoveto{\pgfqpoint{8.361157in}{2.147735in}}%
\pgfpathlineto{\pgfqpoint{8.712735in}{2.303109in}}%
\pgfpathlineto{\pgfqpoint{8.624680in}{2.141959in}}%
\pgfpathlineto{\pgfqpoint{8.361157in}{2.147735in}}%
\pgfpathclose%
\pgfusepath{fill}%
\end{pgfscope}%
\begin{pgfscope}%
\pgfpathrectangle{\pgfqpoint{6.818937in}{0.147348in}}{\pgfqpoint{2.735294in}{2.735294in}}%
\pgfusepath{clip}%
\pgfsetbuttcap%
\pgfsetroundjoin%
\definecolor{currentfill}{rgb}{0.070885,0.257762,0.070885}%
\pgfsetfillcolor{currentfill}%
\pgfsetfillopacity{0.200000}%
\pgfsetlinewidth{0.000000pt}%
\definecolor{currentstroke}{rgb}{0.000000,0.000000,0.000000}%
\pgfsetstrokecolor{currentstroke}%
\pgfsetdash{}{0pt}%
\pgfpathmoveto{\pgfqpoint{8.664078in}{1.306218in}}%
\pgfpathlineto{\pgfqpoint{8.714128in}{0.875593in}}%
\pgfpathlineto{\pgfqpoint{8.625616in}{0.971623in}}%
\pgfpathlineto{\pgfqpoint{8.664078in}{1.306218in}}%
\pgfpathclose%
\pgfusepath{fill}%
\end{pgfscope}%
\begin{pgfscope}%
\pgfpathrectangle{\pgfqpoint{6.818937in}{0.147348in}}{\pgfqpoint{2.735294in}{2.735294in}}%
\pgfusepath{clip}%
\pgfsetbuttcap%
\pgfsetroundjoin%
\definecolor{currentfill}{rgb}{0.070885,0.257762,0.070885}%
\pgfsetfillcolor{currentfill}%
\pgfsetfillopacity{0.200000}%
\pgfsetlinewidth{0.000000pt}%
\definecolor{currentstroke}{rgb}{0.000000,0.000000,0.000000}%
\pgfsetstrokecolor{currentstroke}%
\pgfsetdash{}{0pt}%
\pgfpathmoveto{\pgfqpoint{7.821480in}{0.971623in}}%
\pgfpathlineto{\pgfqpoint{7.732968in}{0.875593in}}%
\pgfpathlineto{\pgfqpoint{7.783018in}{1.306218in}}%
\pgfpathlineto{\pgfqpoint{7.821480in}{0.971623in}}%
\pgfpathclose%
\pgfusepath{fill}%
\end{pgfscope}%
\begin{pgfscope}%
\pgfpathrectangle{\pgfqpoint{6.818937in}{0.147348in}}{\pgfqpoint{2.735294in}{2.735294in}}%
\pgfusepath{clip}%
\pgfsetbuttcap%
\pgfsetroundjoin%
\definecolor{currentfill}{rgb}{0.070984,0.258123,0.070984}%
\pgfsetfillcolor{currentfill}%
\pgfsetfillopacity{0.200000}%
\pgfsetlinewidth{0.000000pt}%
\definecolor{currentstroke}{rgb}{0.000000,0.000000,0.000000}%
\pgfsetstrokecolor{currentstroke}%
\pgfsetdash{}{0pt}%
\pgfpathmoveto{\pgfqpoint{9.026747in}{1.557473in}}%
\pgfpathlineto{\pgfqpoint{9.053362in}{1.083902in}}%
\pgfpathlineto{\pgfqpoint{8.800986in}{1.527187in}}%
\pgfpathlineto{\pgfqpoint{9.026747in}{1.557473in}}%
\pgfpathclose%
\pgfusepath{fill}%
\end{pgfscope}%
\begin{pgfscope}%
\pgfpathrectangle{\pgfqpoint{6.818937in}{0.147348in}}{\pgfqpoint{2.735294in}{2.735294in}}%
\pgfusepath{clip}%
\pgfsetbuttcap%
\pgfsetroundjoin%
\definecolor{currentfill}{rgb}{0.070984,0.258123,0.070984}%
\pgfsetfillcolor{currentfill}%
\pgfsetfillopacity{0.200000}%
\pgfsetlinewidth{0.000000pt}%
\definecolor{currentstroke}{rgb}{0.000000,0.000000,0.000000}%
\pgfsetstrokecolor{currentstroke}%
\pgfsetdash{}{0pt}%
\pgfpathmoveto{\pgfqpoint{7.646110in}{1.527187in}}%
\pgfpathlineto{\pgfqpoint{7.393734in}{1.083902in}}%
\pgfpathlineto{\pgfqpoint{7.420349in}{1.557473in}}%
\pgfpathlineto{\pgfqpoint{7.646110in}{1.527187in}}%
\pgfpathclose%
\pgfusepath{fill}%
\end{pgfscope}%
\begin{pgfscope}%
\pgfpathrectangle{\pgfqpoint{6.818937in}{0.147348in}}{\pgfqpoint{2.735294in}{2.735294in}}%
\pgfusepath{clip}%
\pgfsetbuttcap%
\pgfsetroundjoin%
\definecolor{currentfill}{rgb}{0.061754,0.224559,0.061754}%
\pgfsetfillcolor{currentfill}%
\pgfsetfillopacity{0.200000}%
\pgfsetlinewidth{0.000000pt}%
\definecolor{currentstroke}{rgb}{0.000000,0.000000,0.000000}%
\pgfsetstrokecolor{currentstroke}%
\pgfsetdash{}{0pt}%
\pgfpathmoveto{\pgfqpoint{8.859135in}{1.021716in}}%
\pgfpathlineto{\pgfqpoint{8.714128in}{0.875593in}}%
\pgfpathlineto{\pgfqpoint{8.664078in}{1.306218in}}%
\pgfpathlineto{\pgfqpoint{8.859135in}{1.021716in}}%
\pgfpathclose%
\pgfusepath{fill}%
\end{pgfscope}%
\begin{pgfscope}%
\pgfpathrectangle{\pgfqpoint{6.818937in}{0.147348in}}{\pgfqpoint{2.735294in}{2.735294in}}%
\pgfusepath{clip}%
\pgfsetbuttcap%
\pgfsetroundjoin%
\definecolor{currentfill}{rgb}{0.061754,0.224559,0.061754}%
\pgfsetfillcolor{currentfill}%
\pgfsetfillopacity{0.200000}%
\pgfsetlinewidth{0.000000pt}%
\definecolor{currentstroke}{rgb}{0.000000,0.000000,0.000000}%
\pgfsetstrokecolor{currentstroke}%
\pgfsetdash{}{0pt}%
\pgfpathmoveto{\pgfqpoint{7.783018in}{1.306218in}}%
\pgfpathlineto{\pgfqpoint{7.732968in}{0.875593in}}%
\pgfpathlineto{\pgfqpoint{7.587961in}{1.021716in}}%
\pgfpathlineto{\pgfqpoint{7.783018in}{1.306218in}}%
\pgfpathclose%
\pgfusepath{fill}%
\end{pgfscope}%
\begin{pgfscope}%
\pgfpathrectangle{\pgfqpoint{6.818937in}{0.147348in}}{\pgfqpoint{2.735294in}{2.735294in}}%
\pgfusepath{clip}%
\pgfsetbuttcap%
\pgfsetroundjoin%
\definecolor{currentfill}{rgb}{0.089078,0.323920,0.089078}%
\pgfsetfillcolor{currentfill}%
\pgfsetfillopacity{0.200000}%
\pgfsetlinewidth{0.000000pt}%
\definecolor{currentstroke}{rgb}{0.000000,0.000000,0.000000}%
\pgfsetstrokecolor{currentstroke}%
\pgfsetdash{}{0pt}%
\pgfpathmoveto{\pgfqpoint{9.026747in}{1.557473in}}%
\pgfpathlineto{\pgfqpoint{8.857732in}{2.131775in}}%
\pgfpathlineto{\pgfqpoint{9.112059in}{1.762300in}}%
\pgfpathlineto{\pgfqpoint{9.026747in}{1.557473in}}%
\pgfpathclose%
\pgfusepath{fill}%
\end{pgfscope}%
\begin{pgfscope}%
\pgfpathrectangle{\pgfqpoint{6.818937in}{0.147348in}}{\pgfqpoint{2.735294in}{2.735294in}}%
\pgfusepath{clip}%
\pgfsetbuttcap%
\pgfsetroundjoin%
\definecolor{currentfill}{rgb}{0.089078,0.323920,0.089078}%
\pgfsetfillcolor{currentfill}%
\pgfsetfillopacity{0.200000}%
\pgfsetlinewidth{0.000000pt}%
\definecolor{currentstroke}{rgb}{0.000000,0.000000,0.000000}%
\pgfsetstrokecolor{currentstroke}%
\pgfsetdash{}{0pt}%
\pgfpathmoveto{\pgfqpoint{7.335037in}{1.762300in}}%
\pgfpathlineto{\pgfqpoint{7.589364in}{2.131775in}}%
\pgfpathlineto{\pgfqpoint{7.420349in}{1.557473in}}%
\pgfpathlineto{\pgfqpoint{7.335037in}{1.762300in}}%
\pgfpathclose%
\pgfusepath{fill}%
\end{pgfscope}%
\begin{pgfscope}%
\pgfpathrectangle{\pgfqpoint{6.818937in}{0.147348in}}{\pgfqpoint{2.735294in}{2.735294in}}%
\pgfusepath{clip}%
\pgfsetbuttcap%
\pgfsetroundjoin%
\definecolor{currentfill}{rgb}{0.107070,0.389346,0.107070}%
\pgfsetfillcolor{currentfill}%
\pgfsetfillopacity{0.200000}%
\pgfsetlinewidth{0.000000pt}%
\definecolor{currentstroke}{rgb}{0.000000,0.000000,0.000000}%
\pgfsetstrokecolor{currentstroke}%
\pgfsetdash{}{0pt}%
\pgfpathmoveto{\pgfqpoint{8.624680in}{2.141959in}}%
\pgfpathlineto{\pgfqpoint{8.712735in}{2.303109in}}%
\pgfpathlineto{\pgfqpoint{8.857732in}{2.131775in}}%
\pgfpathlineto{\pgfqpoint{8.624680in}{2.141959in}}%
\pgfpathclose%
\pgfusepath{fill}%
\end{pgfscope}%
\begin{pgfscope}%
\pgfpathrectangle{\pgfqpoint{6.818937in}{0.147348in}}{\pgfqpoint{2.735294in}{2.735294in}}%
\pgfusepath{clip}%
\pgfsetbuttcap%
\pgfsetroundjoin%
\definecolor{currentfill}{rgb}{0.107070,0.389346,0.107070}%
\pgfsetfillcolor{currentfill}%
\pgfsetfillopacity{0.200000}%
\pgfsetlinewidth{0.000000pt}%
\definecolor{currentstroke}{rgb}{0.000000,0.000000,0.000000}%
\pgfsetstrokecolor{currentstroke}%
\pgfsetdash{}{0pt}%
\pgfpathmoveto{\pgfqpoint{7.589364in}{2.131775in}}%
\pgfpathlineto{\pgfqpoint{7.734361in}{2.303109in}}%
\pgfpathlineto{\pgfqpoint{7.822416in}{2.141959in}}%
\pgfpathlineto{\pgfqpoint{7.589364in}{2.131775in}}%
\pgfpathclose%
\pgfusepath{fill}%
\end{pgfscope}%
\begin{pgfscope}%
\pgfpathrectangle{\pgfqpoint{6.818937in}{0.147348in}}{\pgfqpoint{2.735294in}{2.735294in}}%
\pgfusepath{clip}%
\pgfsetbuttcap%
\pgfsetroundjoin%
\definecolor{currentfill}{rgb}{0.056200,0.204363,0.056200}%
\pgfsetfillcolor{currentfill}%
\pgfsetfillopacity{0.200000}%
\pgfsetlinewidth{0.000000pt}%
\definecolor{currentstroke}{rgb}{0.000000,0.000000,0.000000}%
\pgfsetstrokecolor{currentstroke}%
\pgfsetdash{}{0pt}%
\pgfpathmoveto{\pgfqpoint{7.969219in}{0.836228in}}%
\pgfpathlineto{\pgfqpoint{8.085608in}{0.943207in}}%
\pgfpathlineto{\pgfqpoint{8.223548in}{0.822079in}}%
\pgfpathlineto{\pgfqpoint{7.969219in}{0.836228in}}%
\pgfpathclose%
\pgfusepath{fill}%
\end{pgfscope}%
\begin{pgfscope}%
\pgfpathrectangle{\pgfqpoint{6.818937in}{0.147348in}}{\pgfqpoint{2.735294in}{2.735294in}}%
\pgfusepath{clip}%
\pgfsetbuttcap%
\pgfsetroundjoin%
\definecolor{currentfill}{rgb}{0.056200,0.204363,0.056200}%
\pgfsetfillcolor{currentfill}%
\pgfsetfillopacity{0.200000}%
\pgfsetlinewidth{0.000000pt}%
\definecolor{currentstroke}{rgb}{0.000000,0.000000,0.000000}%
\pgfsetstrokecolor{currentstroke}%
\pgfsetdash{}{0pt}%
\pgfpathmoveto{\pgfqpoint{8.223548in}{0.822079in}}%
\pgfpathlineto{\pgfqpoint{8.361488in}{0.943207in}}%
\pgfpathlineto{\pgfqpoint{8.477877in}{0.836228in}}%
\pgfpathlineto{\pgfqpoint{8.223548in}{0.822079in}}%
\pgfpathclose%
\pgfusepath{fill}%
\end{pgfscope}%
\begin{pgfscope}%
\pgfpathrectangle{\pgfqpoint{6.818937in}{0.147348in}}{\pgfqpoint{2.735294in}{2.735294in}}%
\pgfusepath{clip}%
\pgfsetbuttcap%
\pgfsetroundjoin%
\definecolor{currentfill}{rgb}{0.086498,0.314539,0.086498}%
\pgfsetfillcolor{currentfill}%
\pgfsetfillopacity{0.200000}%
\pgfsetlinewidth{0.000000pt}%
\definecolor{currentstroke}{rgb}{0.000000,0.000000,0.000000}%
\pgfsetstrokecolor{currentstroke}%
\pgfsetdash{}{0pt}%
\pgfpathmoveto{\pgfqpoint{9.026747in}{1.557473in}}%
\pgfpathlineto{\pgfqpoint{9.112059in}{1.762300in}}%
\pgfpathlineto{\pgfqpoint{9.203313in}{1.590365in}}%
\pgfpathlineto{\pgfqpoint{9.026747in}{1.557473in}}%
\pgfpathclose%
\pgfusepath{fill}%
\end{pgfscope}%
\begin{pgfscope}%
\pgfpathrectangle{\pgfqpoint{6.818937in}{0.147348in}}{\pgfqpoint{2.735294in}{2.735294in}}%
\pgfusepath{clip}%
\pgfsetbuttcap%
\pgfsetroundjoin%
\definecolor{currentfill}{rgb}{0.086498,0.314539,0.086498}%
\pgfsetfillcolor{currentfill}%
\pgfsetfillopacity{0.200000}%
\pgfsetlinewidth{0.000000pt}%
\definecolor{currentstroke}{rgb}{0.000000,0.000000,0.000000}%
\pgfsetstrokecolor{currentstroke}%
\pgfsetdash{}{0pt}%
\pgfpathmoveto{\pgfqpoint{7.243783in}{1.590365in}}%
\pgfpathlineto{\pgfqpoint{7.335037in}{1.762300in}}%
\pgfpathlineto{\pgfqpoint{7.420349in}{1.557473in}}%
\pgfpathlineto{\pgfqpoint{7.243783in}{1.590365in}}%
\pgfpathclose%
\pgfusepath{fill}%
\end{pgfscope}%
\begin{pgfscope}%
\pgfpathrectangle{\pgfqpoint{6.818937in}{0.147348in}}{\pgfqpoint{2.735294in}{2.735294in}}%
\pgfusepath{clip}%
\pgfsetbuttcap%
\pgfsetroundjoin%
\definecolor{currentfill}{rgb}{0.090812,0.330224,0.090812}%
\pgfsetfillcolor{currentfill}%
\pgfsetfillopacity{0.200000}%
\pgfsetlinewidth{0.000000pt}%
\definecolor{currentstroke}{rgb}{0.000000,0.000000,0.000000}%
\pgfsetstrokecolor{currentstroke}%
\pgfsetdash{}{0pt}%
\pgfpathmoveto{\pgfqpoint{8.800986in}{1.527187in}}%
\pgfpathlineto{\pgfqpoint{8.920136in}{0.932617in}}%
\pgfpathlineto{\pgfqpoint{8.859135in}{1.021716in}}%
\pgfpathlineto{\pgfqpoint{8.800986in}{1.527187in}}%
\pgfpathclose%
\pgfusepath{fill}%
\end{pgfscope}%
\begin{pgfscope}%
\pgfpathrectangle{\pgfqpoint{6.818937in}{0.147348in}}{\pgfqpoint{2.735294in}{2.735294in}}%
\pgfusepath{clip}%
\pgfsetbuttcap%
\pgfsetroundjoin%
\definecolor{currentfill}{rgb}{0.090812,0.330224,0.090812}%
\pgfsetfillcolor{currentfill}%
\pgfsetfillopacity{0.200000}%
\pgfsetlinewidth{0.000000pt}%
\definecolor{currentstroke}{rgb}{0.000000,0.000000,0.000000}%
\pgfsetstrokecolor{currentstroke}%
\pgfsetdash{}{0pt}%
\pgfpathmoveto{\pgfqpoint{7.587961in}{1.021716in}}%
\pgfpathlineto{\pgfqpoint{7.526959in}{0.932617in}}%
\pgfpathlineto{\pgfqpoint{7.646110in}{1.527187in}}%
\pgfpathlineto{\pgfqpoint{7.587961in}{1.021716in}}%
\pgfpathclose%
\pgfusepath{fill}%
\end{pgfscope}%
\begin{pgfscope}%
\pgfpathrectangle{\pgfqpoint{6.818937in}{0.147348in}}{\pgfqpoint{2.735294in}{2.735294in}}%
\pgfusepath{clip}%
\pgfsetbuttcap%
\pgfsetroundjoin%
\definecolor{currentfill}{rgb}{0.116321,0.422987,0.116321}%
\pgfsetfillcolor{currentfill}%
\pgfsetfillopacity{0.200000}%
\pgfsetlinewidth{0.000000pt}%
\definecolor{currentstroke}{rgb}{0.000000,0.000000,0.000000}%
\pgfsetstrokecolor{currentstroke}%
\pgfsetdash{}{0pt}%
\pgfpathmoveto{\pgfqpoint{8.361157in}{2.147735in}}%
\pgfpathlineto{\pgfqpoint{8.085938in}{2.147735in}}%
\pgfpathlineto{\pgfqpoint{8.123207in}{2.676888in}}%
\pgfpathlineto{\pgfqpoint{8.361157in}{2.147735in}}%
\pgfpathclose%
\pgfusepath{fill}%
\end{pgfscope}%
\begin{pgfscope}%
\pgfpathrectangle{\pgfqpoint{6.818937in}{0.147348in}}{\pgfqpoint{2.735294in}{2.735294in}}%
\pgfusepath{clip}%
\pgfsetbuttcap%
\pgfsetroundjoin%
\definecolor{currentfill}{rgb}{0.057724,0.209904,0.057724}%
\pgfsetfillcolor{currentfill}%
\pgfsetfillopacity{0.200000}%
\pgfsetlinewidth{0.000000pt}%
\definecolor{currentstroke}{rgb}{0.000000,0.000000,0.000000}%
\pgfsetstrokecolor{currentstroke}%
\pgfsetdash{}{0pt}%
\pgfpathmoveto{\pgfqpoint{7.969219in}{0.836228in}}%
\pgfpathlineto{\pgfqpoint{7.732968in}{0.875593in}}%
\pgfpathlineto{\pgfqpoint{7.821480in}{0.971623in}}%
\pgfpathlineto{\pgfqpoint{7.969219in}{0.836228in}}%
\pgfpathclose%
\pgfusepath{fill}%
\end{pgfscope}%
\begin{pgfscope}%
\pgfpathrectangle{\pgfqpoint{6.818937in}{0.147348in}}{\pgfqpoint{2.735294in}{2.735294in}}%
\pgfusepath{clip}%
\pgfsetbuttcap%
\pgfsetroundjoin%
\definecolor{currentfill}{rgb}{0.057724,0.209904,0.057724}%
\pgfsetfillcolor{currentfill}%
\pgfsetfillopacity{0.200000}%
\pgfsetlinewidth{0.000000pt}%
\definecolor{currentstroke}{rgb}{0.000000,0.000000,0.000000}%
\pgfsetstrokecolor{currentstroke}%
\pgfsetdash{}{0pt}%
\pgfpathmoveto{\pgfqpoint{8.625616in}{0.971623in}}%
\pgfpathlineto{\pgfqpoint{8.714128in}{0.875593in}}%
\pgfpathlineto{\pgfqpoint{8.477877in}{0.836228in}}%
\pgfpathlineto{\pgfqpoint{8.625616in}{0.971623in}}%
\pgfpathclose%
\pgfusepath{fill}%
\end{pgfscope}%
\begin{pgfscope}%
\pgfpathrectangle{\pgfqpoint{6.818937in}{0.147348in}}{\pgfqpoint{2.735294in}{2.735294in}}%
\pgfusepath{clip}%
\pgfsetbuttcap%
\pgfsetroundjoin%
\definecolor{currentfill}{rgb}{0.064867,0.235879,0.064867}%
\pgfsetfillcolor{currentfill}%
\pgfsetfillopacity{0.200000}%
\pgfsetlinewidth{0.000000pt}%
\definecolor{currentstroke}{rgb}{0.000000,0.000000,0.000000}%
\pgfsetstrokecolor{currentstroke}%
\pgfsetdash{}{0pt}%
\pgfpathmoveto{\pgfqpoint{9.053362in}{1.083902in}}%
\pgfpathlineto{\pgfqpoint{8.920136in}{0.932617in}}%
\pgfpathlineto{\pgfqpoint{8.800986in}{1.527187in}}%
\pgfpathlineto{\pgfqpoint{9.053362in}{1.083902in}}%
\pgfpathclose%
\pgfusepath{fill}%
\end{pgfscope}%
\begin{pgfscope}%
\pgfpathrectangle{\pgfqpoint{6.818937in}{0.147348in}}{\pgfqpoint{2.735294in}{2.735294in}}%
\pgfusepath{clip}%
\pgfsetbuttcap%
\pgfsetroundjoin%
\definecolor{currentfill}{rgb}{0.064867,0.235879,0.064867}%
\pgfsetfillcolor{currentfill}%
\pgfsetfillopacity{0.200000}%
\pgfsetlinewidth{0.000000pt}%
\definecolor{currentstroke}{rgb}{0.000000,0.000000,0.000000}%
\pgfsetstrokecolor{currentstroke}%
\pgfsetdash{}{0pt}%
\pgfpathmoveto{\pgfqpoint{7.646110in}{1.527187in}}%
\pgfpathlineto{\pgfqpoint{7.526959in}{0.932617in}}%
\pgfpathlineto{\pgfqpoint{7.393734in}{1.083902in}}%
\pgfpathlineto{\pgfqpoint{7.646110in}{1.527187in}}%
\pgfpathclose%
\pgfusepath{fill}%
\end{pgfscope}%
\begin{pgfscope}%
\pgfpathrectangle{\pgfqpoint{6.818937in}{0.147348in}}{\pgfqpoint{2.735294in}{2.735294in}}%
\pgfusepath{clip}%
\pgfsetbuttcap%
\pgfsetroundjoin%
\definecolor{currentfill}{rgb}{0.116785,0.424671,0.116785}%
\pgfsetfillcolor{currentfill}%
\pgfsetfillopacity{0.200000}%
\pgfsetlinewidth{0.000000pt}%
\definecolor{currentstroke}{rgb}{0.000000,0.000000,0.000000}%
\pgfsetstrokecolor{currentstroke}%
\pgfsetdash{}{0pt}%
\pgfpathmoveto{\pgfqpoint{8.361157in}{2.147735in}}%
\pgfpathlineto{\pgfqpoint{8.323888in}{2.676888in}}%
\pgfpathlineto{\pgfqpoint{8.712735in}{2.303109in}}%
\pgfpathlineto{\pgfqpoint{8.361157in}{2.147735in}}%
\pgfpathclose%
\pgfusepath{fill}%
\end{pgfscope}%
\begin{pgfscope}%
\pgfpathrectangle{\pgfqpoint{6.818937in}{0.147348in}}{\pgfqpoint{2.735294in}{2.735294in}}%
\pgfusepath{clip}%
\pgfsetbuttcap%
\pgfsetroundjoin%
\definecolor{currentfill}{rgb}{0.116785,0.424671,0.116785}%
\pgfsetfillcolor{currentfill}%
\pgfsetfillopacity{0.200000}%
\pgfsetlinewidth{0.000000pt}%
\definecolor{currentstroke}{rgb}{0.000000,0.000000,0.000000}%
\pgfsetstrokecolor{currentstroke}%
\pgfsetdash{}{0pt}%
\pgfpathmoveto{\pgfqpoint{7.734361in}{2.303109in}}%
\pgfpathlineto{\pgfqpoint{8.123207in}{2.676888in}}%
\pgfpathlineto{\pgfqpoint{8.085938in}{2.147735in}}%
\pgfpathlineto{\pgfqpoint{7.734361in}{2.303109in}}%
\pgfpathclose%
\pgfusepath{fill}%
\end{pgfscope}%
\begin{pgfscope}%
\pgfpathrectangle{\pgfqpoint{6.818937in}{0.147348in}}{\pgfqpoint{2.735294in}{2.735294in}}%
\pgfusepath{clip}%
\pgfsetbuttcap%
\pgfsetroundjoin%
\definecolor{currentfill}{rgb}{0.074506,0.270932,0.074506}%
\pgfsetfillcolor{currentfill}%
\pgfsetfillopacity{0.200000}%
\pgfsetlinewidth{0.000000pt}%
\definecolor{currentstroke}{rgb}{0.000000,0.000000,0.000000}%
\pgfsetstrokecolor{currentstroke}%
\pgfsetdash{}{0pt}%
\pgfpathmoveto{\pgfqpoint{9.203313in}{1.590365in}}%
\pgfpathlineto{\pgfqpoint{9.208715in}{1.149785in}}%
\pgfpathlineto{\pgfqpoint{9.026747in}{1.557473in}}%
\pgfpathlineto{\pgfqpoint{9.203313in}{1.590365in}}%
\pgfpathclose%
\pgfusepath{fill}%
\end{pgfscope}%
\begin{pgfscope}%
\pgfpathrectangle{\pgfqpoint{6.818937in}{0.147348in}}{\pgfqpoint{2.735294in}{2.735294in}}%
\pgfusepath{clip}%
\pgfsetbuttcap%
\pgfsetroundjoin%
\definecolor{currentfill}{rgb}{0.074506,0.270932,0.074506}%
\pgfsetfillcolor{currentfill}%
\pgfsetfillopacity{0.200000}%
\pgfsetlinewidth{0.000000pt}%
\definecolor{currentstroke}{rgb}{0.000000,0.000000,0.000000}%
\pgfsetstrokecolor{currentstroke}%
\pgfsetdash{}{0pt}%
\pgfpathmoveto{\pgfqpoint{7.420349in}{1.557473in}}%
\pgfpathlineto{\pgfqpoint{7.238381in}{1.149785in}}%
\pgfpathlineto{\pgfqpoint{7.243783in}{1.590365in}}%
\pgfpathlineto{\pgfqpoint{7.420349in}{1.557473in}}%
\pgfpathclose%
\pgfusepath{fill}%
\end{pgfscope}%
\begin{pgfscope}%
\pgfpathrectangle{\pgfqpoint{6.818937in}{0.147348in}}{\pgfqpoint{2.735294in}{2.735294in}}%
\pgfusepath{clip}%
\pgfsetbuttcap%
\pgfsetroundjoin%
\definecolor{currentfill}{rgb}{0.060435,0.219763,0.060435}%
\pgfsetfillcolor{currentfill}%
\pgfsetfillopacity{0.200000}%
\pgfsetlinewidth{0.000000pt}%
\definecolor{currentstroke}{rgb}{0.000000,0.000000,0.000000}%
\pgfsetstrokecolor{currentstroke}%
\pgfsetdash{}{0pt}%
\pgfpathmoveto{\pgfqpoint{8.920136in}{0.932617in}}%
\pgfpathlineto{\pgfqpoint{8.714128in}{0.875593in}}%
\pgfpathlineto{\pgfqpoint{8.859135in}{1.021716in}}%
\pgfpathlineto{\pgfqpoint{8.920136in}{0.932617in}}%
\pgfpathclose%
\pgfusepath{fill}%
\end{pgfscope}%
\begin{pgfscope}%
\pgfpathrectangle{\pgfqpoint{6.818937in}{0.147348in}}{\pgfqpoint{2.735294in}{2.735294in}}%
\pgfusepath{clip}%
\pgfsetbuttcap%
\pgfsetroundjoin%
\definecolor{currentfill}{rgb}{0.060435,0.219763,0.060435}%
\pgfsetfillcolor{currentfill}%
\pgfsetfillopacity{0.200000}%
\pgfsetlinewidth{0.000000pt}%
\definecolor{currentstroke}{rgb}{0.000000,0.000000,0.000000}%
\pgfsetstrokecolor{currentstroke}%
\pgfsetdash{}{0pt}%
\pgfpathmoveto{\pgfqpoint{7.587961in}{1.021716in}}%
\pgfpathlineto{\pgfqpoint{7.732968in}{0.875593in}}%
\pgfpathlineto{\pgfqpoint{7.526959in}{0.932617in}}%
\pgfpathlineto{\pgfqpoint{7.587961in}{1.021716in}}%
\pgfpathclose%
\pgfusepath{fill}%
\end{pgfscope}%
\begin{pgfscope}%
\pgfpathrectangle{\pgfqpoint{6.818937in}{0.147348in}}{\pgfqpoint{2.735294in}{2.735294in}}%
\pgfusepath{clip}%
\pgfsetbuttcap%
\pgfsetroundjoin%
\definecolor{currentfill}{rgb}{0.095351,0.346729,0.095351}%
\pgfsetfillcolor{currentfill}%
\pgfsetfillopacity{0.200000}%
\pgfsetlinewidth{0.000000pt}%
\definecolor{currentstroke}{rgb}{0.000000,0.000000,0.000000}%
\pgfsetstrokecolor{currentstroke}%
\pgfsetdash{}{0pt}%
\pgfpathmoveto{\pgfqpoint{9.026747in}{1.557473in}}%
\pgfpathlineto{\pgfqpoint{9.091593in}{0.998749in}}%
\pgfpathlineto{\pgfqpoint{9.053362in}{1.083902in}}%
\pgfpathlineto{\pgfqpoint{9.026747in}{1.557473in}}%
\pgfpathclose%
\pgfusepath{fill}%
\end{pgfscope}%
\begin{pgfscope}%
\pgfpathrectangle{\pgfqpoint{6.818937in}{0.147348in}}{\pgfqpoint{2.735294in}{2.735294in}}%
\pgfusepath{clip}%
\pgfsetbuttcap%
\pgfsetroundjoin%
\definecolor{currentfill}{rgb}{0.095351,0.346729,0.095351}%
\pgfsetfillcolor{currentfill}%
\pgfsetfillopacity{0.200000}%
\pgfsetlinewidth{0.000000pt}%
\definecolor{currentstroke}{rgb}{0.000000,0.000000,0.000000}%
\pgfsetstrokecolor{currentstroke}%
\pgfsetdash{}{0pt}%
\pgfpathmoveto{\pgfqpoint{7.393734in}{1.083902in}}%
\pgfpathlineto{\pgfqpoint{7.355503in}{0.998749in}}%
\pgfpathlineto{\pgfqpoint{7.420349in}{1.557473in}}%
\pgfpathlineto{\pgfqpoint{7.393734in}{1.083902in}}%
\pgfpathclose%
\pgfusepath{fill}%
\end{pgfscope}%
\begin{pgfscope}%
\pgfpathrectangle{\pgfqpoint{6.818937in}{0.147348in}}{\pgfqpoint{2.735294in}{2.735294in}}%
\pgfusepath{clip}%
\pgfsetbuttcap%
\pgfsetroundjoin%
\definecolor{currentfill}{rgb}{0.067061,0.243857,0.067061}%
\pgfsetfillcolor{currentfill}%
\pgfsetfillopacity{0.200000}%
\pgfsetlinewidth{0.000000pt}%
\definecolor{currentstroke}{rgb}{0.000000,0.000000,0.000000}%
\pgfsetstrokecolor{currentstroke}%
\pgfsetdash{}{0pt}%
\pgfpathmoveto{\pgfqpoint{9.208715in}{1.149785in}}%
\pgfpathlineto{\pgfqpoint{9.091593in}{0.998749in}}%
\pgfpathlineto{\pgfqpoint{9.026747in}{1.557473in}}%
\pgfpathlineto{\pgfqpoint{9.208715in}{1.149785in}}%
\pgfpathclose%
\pgfusepath{fill}%
\end{pgfscope}%
\begin{pgfscope}%
\pgfpathrectangle{\pgfqpoint{6.818937in}{0.147348in}}{\pgfqpoint{2.735294in}{2.735294in}}%
\pgfusepath{clip}%
\pgfsetbuttcap%
\pgfsetroundjoin%
\definecolor{currentfill}{rgb}{0.067061,0.243857,0.067061}%
\pgfsetfillcolor{currentfill}%
\pgfsetfillopacity{0.200000}%
\pgfsetlinewidth{0.000000pt}%
\definecolor{currentstroke}{rgb}{0.000000,0.000000,0.000000}%
\pgfsetstrokecolor{currentstroke}%
\pgfsetdash{}{0pt}%
\pgfpathmoveto{\pgfqpoint{7.420349in}{1.557473in}}%
\pgfpathlineto{\pgfqpoint{7.355503in}{0.998749in}}%
\pgfpathlineto{\pgfqpoint{7.238381in}{1.149785in}}%
\pgfpathlineto{\pgfqpoint{7.420349in}{1.557473in}}%
\pgfpathclose%
\pgfusepath{fill}%
\end{pgfscope}%
\begin{pgfscope}%
\pgfpathrectangle{\pgfqpoint{6.818937in}{0.147348in}}{\pgfqpoint{2.735294in}{2.735294in}}%
\pgfusepath{clip}%
\pgfsetbuttcap%
\pgfsetroundjoin%
\definecolor{currentfill}{rgb}{0.116321,0.422987,0.116321}%
\pgfsetfillcolor{currentfill}%
\pgfsetfillopacity{0.200000}%
\pgfsetlinewidth{0.000000pt}%
\definecolor{currentstroke}{rgb}{0.000000,0.000000,0.000000}%
\pgfsetstrokecolor{currentstroke}%
\pgfsetdash{}{0pt}%
\pgfpathmoveto{\pgfqpoint{8.123207in}{2.676888in}}%
\pgfpathlineto{\pgfqpoint{8.323888in}{2.676888in}}%
\pgfpathlineto{\pgfqpoint{8.361157in}{2.147735in}}%
\pgfpathlineto{\pgfqpoint{8.123207in}{2.676888in}}%
\pgfpathclose%
\pgfusepath{fill}%
\end{pgfscope}%
\begin{pgfscope}%
\pgfpathrectangle{\pgfqpoint{6.818937in}{0.147348in}}{\pgfqpoint{2.735294in}{2.735294in}}%
\pgfusepath{clip}%
\pgfsetbuttcap%
\pgfsetroundjoin%
\definecolor{currentfill}{rgb}{0.063840,0.232145,0.063840}%
\pgfsetfillcolor{currentfill}%
\pgfsetfillopacity{0.200000}%
\pgfsetlinewidth{0.000000pt}%
\definecolor{currentstroke}{rgb}{0.000000,0.000000,0.000000}%
\pgfsetstrokecolor{currentstroke}%
\pgfsetdash{}{0pt}%
\pgfpathmoveto{\pgfqpoint{8.920136in}{0.932617in}}%
\pgfpathlineto{\pgfqpoint{9.053362in}{1.083902in}}%
\pgfpathlineto{\pgfqpoint{9.091593in}{0.998749in}}%
\pgfpathlineto{\pgfqpoint{8.920136in}{0.932617in}}%
\pgfpathclose%
\pgfusepath{fill}%
\end{pgfscope}%
\begin{pgfscope}%
\pgfpathrectangle{\pgfqpoint{6.818937in}{0.147348in}}{\pgfqpoint{2.735294in}{2.735294in}}%
\pgfusepath{clip}%
\pgfsetbuttcap%
\pgfsetroundjoin%
\definecolor{currentfill}{rgb}{0.063840,0.232145,0.063840}%
\pgfsetfillcolor{currentfill}%
\pgfsetfillopacity{0.200000}%
\pgfsetlinewidth{0.000000pt}%
\definecolor{currentstroke}{rgb}{0.000000,0.000000,0.000000}%
\pgfsetstrokecolor{currentstroke}%
\pgfsetdash{}{0pt}%
\pgfpathmoveto{\pgfqpoint{7.355503in}{0.998749in}}%
\pgfpathlineto{\pgfqpoint{7.393734in}{1.083902in}}%
\pgfpathlineto{\pgfqpoint{7.526959in}{0.932617in}}%
\pgfpathlineto{\pgfqpoint{7.355503in}{0.998749in}}%
\pgfpathclose%
\pgfusepath{fill}%
\end{pgfscope}%
\begin{pgfscope}%
\pgfpathrectangle{\pgfqpoint{6.818937in}{0.147348in}}{\pgfqpoint{2.735294in}{2.735294in}}%
\pgfusepath{clip}%
\pgfsetbuttcap%
\pgfsetroundjoin%
\definecolor{currentfill}{rgb}{0.099716,0.362602,0.099716}%
\pgfsetfillcolor{currentfill}%
\pgfsetfillopacity{0.200000}%
\pgfsetlinewidth{0.000000pt}%
\definecolor{currentstroke}{rgb}{0.000000,0.000000,0.000000}%
\pgfsetstrokecolor{currentstroke}%
\pgfsetdash{}{0pt}%
\pgfpathmoveto{\pgfqpoint{9.203313in}{1.590365in}}%
\pgfpathlineto{\pgfqpoint{9.230175in}{1.067062in}}%
\pgfpathlineto{\pgfqpoint{9.208715in}{1.149785in}}%
\pgfpathlineto{\pgfqpoint{9.203313in}{1.590365in}}%
\pgfpathclose%
\pgfusepath{fill}%
\end{pgfscope}%
\begin{pgfscope}%
\pgfpathrectangle{\pgfqpoint{6.818937in}{0.147348in}}{\pgfqpoint{2.735294in}{2.735294in}}%
\pgfusepath{clip}%
\pgfsetbuttcap%
\pgfsetroundjoin%
\definecolor{currentfill}{rgb}{0.099716,0.362602,0.099716}%
\pgfsetfillcolor{currentfill}%
\pgfsetfillopacity{0.200000}%
\pgfsetlinewidth{0.000000pt}%
\definecolor{currentstroke}{rgb}{0.000000,0.000000,0.000000}%
\pgfsetstrokecolor{currentstroke}%
\pgfsetdash{}{0pt}%
\pgfpathmoveto{\pgfqpoint{7.238381in}{1.149785in}}%
\pgfpathlineto{\pgfqpoint{7.216921in}{1.067062in}}%
\pgfpathlineto{\pgfqpoint{7.243783in}{1.590365in}}%
\pgfpathlineto{\pgfqpoint{7.238381in}{1.149785in}}%
\pgfpathclose%
\pgfusepath{fill}%
\end{pgfscope}%
\begin{pgfscope}%
\pgfpathrectangle{\pgfqpoint{6.818937in}{0.147348in}}{\pgfqpoint{2.735294in}{2.735294in}}%
\pgfusepath{clip}%
\pgfsetbuttcap%
\pgfsetroundjoin%
\definecolor{currentfill}{rgb}{0.069492,0.252698,0.069492}%
\pgfsetfillcolor{currentfill}%
\pgfsetfillopacity{0.200000}%
\pgfsetlinewidth{0.000000pt}%
\definecolor{currentstroke}{rgb}{0.000000,0.000000,0.000000}%
\pgfsetstrokecolor{currentstroke}%
\pgfsetdash{}{0pt}%
\pgfpathmoveto{\pgfqpoint{9.330515in}{1.213837in}}%
\pgfpathlineto{\pgfqpoint{9.230175in}{1.067062in}}%
\pgfpathlineto{\pgfqpoint{9.203313in}{1.590365in}}%
\pgfpathlineto{\pgfqpoint{9.330515in}{1.213837in}}%
\pgfpathclose%
\pgfusepath{fill}%
\end{pgfscope}%
\begin{pgfscope}%
\pgfpathrectangle{\pgfqpoint{6.818937in}{0.147348in}}{\pgfqpoint{2.735294in}{2.735294in}}%
\pgfusepath{clip}%
\pgfsetbuttcap%
\pgfsetroundjoin%
\definecolor{currentfill}{rgb}{0.069492,0.252698,0.069492}%
\pgfsetfillcolor{currentfill}%
\pgfsetfillopacity{0.200000}%
\pgfsetlinewidth{0.000000pt}%
\definecolor{currentstroke}{rgb}{0.000000,0.000000,0.000000}%
\pgfsetstrokecolor{currentstroke}%
\pgfsetdash{}{0pt}%
\pgfpathmoveto{\pgfqpoint{7.116581in}{1.213837in}}%
\pgfpathlineto{\pgfqpoint{7.243783in}{1.590365in}}%
\pgfpathlineto{\pgfqpoint{7.216921in}{1.067062in}}%
\pgfpathlineto{\pgfqpoint{7.116581in}{1.213837in}}%
\pgfpathclose%
\pgfusepath{fill}%
\end{pgfscope}%
\begin{pgfscope}%
\pgfpathrectangle{\pgfqpoint{6.818937in}{0.147348in}}{\pgfqpoint{2.735294in}{2.735294in}}%
\pgfusepath{clip}%
\pgfsetbuttcap%
\pgfsetroundjoin%
\definecolor{currentfill}{rgb}{0.067488,0.245410,0.067488}%
\pgfsetfillcolor{currentfill}%
\pgfsetfillopacity{0.200000}%
\pgfsetlinewidth{0.000000pt}%
\definecolor{currentstroke}{rgb}{0.000000,0.000000,0.000000}%
\pgfsetstrokecolor{currentstroke}%
\pgfsetdash{}{0pt}%
\pgfpathmoveto{\pgfqpoint{9.091593in}{0.998749in}}%
\pgfpathlineto{\pgfqpoint{9.208715in}{1.149785in}}%
\pgfpathlineto{\pgfqpoint{9.230175in}{1.067062in}}%
\pgfpathlineto{\pgfqpoint{9.091593in}{0.998749in}}%
\pgfpathclose%
\pgfusepath{fill}%
\end{pgfscope}%
\begin{pgfscope}%
\pgfpathrectangle{\pgfqpoint{6.818937in}{0.147348in}}{\pgfqpoint{2.735294in}{2.735294in}}%
\pgfusepath{clip}%
\pgfsetbuttcap%
\pgfsetroundjoin%
\definecolor{currentfill}{rgb}{0.067488,0.245410,0.067488}%
\pgfsetfillcolor{currentfill}%
\pgfsetfillopacity{0.200000}%
\pgfsetlinewidth{0.000000pt}%
\definecolor{currentstroke}{rgb}{0.000000,0.000000,0.000000}%
\pgfsetstrokecolor{currentstroke}%
\pgfsetdash{}{0pt}%
\pgfpathmoveto{\pgfqpoint{7.216921in}{1.067062in}}%
\pgfpathlineto{\pgfqpoint{7.238381in}{1.149785in}}%
\pgfpathlineto{\pgfqpoint{7.355503in}{0.998749in}}%
\pgfpathlineto{\pgfqpoint{7.216921in}{1.067062in}}%
\pgfpathclose%
\pgfusepath{fill}%
\end{pgfscope}%
\begin{pgfscope}%
\pgfpathrectangle{\pgfqpoint{6.818937in}{0.147348in}}{\pgfqpoint{2.735294in}{2.735294in}}%
\pgfusepath{clip}%
\pgfsetbuttcap%
\pgfsetroundjoin%
\definecolor{currentfill}{rgb}{0.128601,0.467641,0.128601}%
\pgfsetfillcolor{currentfill}%
\pgfsetfillopacity{0.200000}%
\pgfsetlinewidth{0.000000pt}%
\definecolor{currentstroke}{rgb}{0.000000,0.000000,0.000000}%
\pgfsetstrokecolor{currentstroke}%
\pgfsetdash{}{0pt}%
\pgfpathmoveto{\pgfqpoint{8.323888in}{2.676888in}}%
\pgfpathlineto{\pgfqpoint{8.123207in}{2.676888in}}%
\pgfpathlineto{\pgfqpoint{8.223548in}{2.756756in}}%
\pgfpathlineto{\pgfqpoint{8.323888in}{2.676888in}}%
\pgfpathclose%
\pgfusepath{fill}%
\end{pgfscope}%
\begin{pgfscope}%
\pgfpathrectangle{\pgfqpoint{6.818937in}{0.147348in}}{\pgfqpoint{2.735294in}{2.735294in}}%
\pgfusepath{clip}%
\pgfsetbuttcap%
\pgfsetroundjoin%
\definecolor{currentfill}{rgb}{0.071067,0.258424,0.071067}%
\pgfsetfillcolor{currentfill}%
\pgfsetfillopacity{0.200000}%
\pgfsetlinewidth{0.000000pt}%
\definecolor{currentstroke}{rgb}{0.000000,0.000000,0.000000}%
\pgfsetstrokecolor{currentstroke}%
\pgfsetdash{}{0pt}%
\pgfpathmoveto{\pgfqpoint{9.230175in}{1.067062in}}%
\pgfpathlineto{\pgfqpoint{9.330515in}{1.213837in}}%
\pgfpathlineto{\pgfqpoint{9.340537in}{1.133087in}}%
\pgfpathlineto{\pgfqpoint{9.230175in}{1.067062in}}%
\pgfpathclose%
\pgfusepath{fill}%
\end{pgfscope}%
\begin{pgfscope}%
\pgfpathrectangle{\pgfqpoint{6.818937in}{0.147348in}}{\pgfqpoint{2.735294in}{2.735294in}}%
\pgfusepath{clip}%
\pgfsetbuttcap%
\pgfsetroundjoin%
\definecolor{currentfill}{rgb}{0.071067,0.258424,0.071067}%
\pgfsetfillcolor{currentfill}%
\pgfsetfillopacity{0.200000}%
\pgfsetlinewidth{0.000000pt}%
\definecolor{currentstroke}{rgb}{0.000000,0.000000,0.000000}%
\pgfsetstrokecolor{currentstroke}%
\pgfsetdash{}{0pt}%
\pgfpathmoveto{\pgfqpoint{7.116581in}{1.213837in}}%
\pgfpathlineto{\pgfqpoint{7.216921in}{1.067062in}}%
\pgfpathlineto{\pgfqpoint{7.106558in}{1.133087in}}%
\pgfpathlineto{\pgfqpoint{7.116581in}{1.213837in}}%
\pgfpathclose%
\pgfusepath{fill}%
\end{pgfscope}%
\begin{pgfscope}%
\pgfpathrectangle{\pgfqpoint{6.818937in}{0.147348in}}{\pgfqpoint{2.735294in}{2.735294in}}%
\pgfusepath{clip}%
\pgfsetbuttcap%
\pgfsetroundjoin%
\definecolor{currentfill}{rgb}{0.052607,0.201942,0.305459}%
\pgfsetfillcolor{currentfill}%
\pgfsetlinewidth{0.000000pt}%
\definecolor{currentstroke}{rgb}{0.000000,0.000000,0.000000}%
\pgfsetstrokecolor{currentstroke}%
\pgfsetdash{}{0pt}%
\pgfpathmoveto{\pgfqpoint{8.502215in}{1.541694in}}%
\pgfpathlineto{\pgfqpoint{8.362938in}{1.338584in}}%
\pgfpathlineto{\pgfqpoint{8.223548in}{1.534090in}}%
\pgfpathlineto{\pgfqpoint{8.502215in}{1.541694in}}%
\pgfpathclose%
\pgfusepath{fill}%
\end{pgfscope}%
\begin{pgfscope}%
\pgfpathrectangle{\pgfqpoint{6.818937in}{0.147348in}}{\pgfqpoint{2.735294in}{2.735294in}}%
\pgfusepath{clip}%
\pgfsetbuttcap%
\pgfsetroundjoin%
\definecolor{currentfill}{rgb}{0.052607,0.201942,0.305459}%
\pgfsetfillcolor{currentfill}%
\pgfsetlinewidth{0.000000pt}%
\definecolor{currentstroke}{rgb}{0.000000,0.000000,0.000000}%
\pgfsetstrokecolor{currentstroke}%
\pgfsetdash{}{0pt}%
\pgfpathmoveto{\pgfqpoint{8.223548in}{1.534090in}}%
\pgfpathlineto{\pgfqpoint{8.084158in}{1.338584in}}%
\pgfpathlineto{\pgfqpoint{7.944881in}{1.541694in}}%
\pgfpathlineto{\pgfqpoint{8.223548in}{1.534090in}}%
\pgfpathclose%
\pgfusepath{fill}%
\end{pgfscope}%
\begin{pgfscope}%
\pgfpathrectangle{\pgfqpoint{6.818937in}{0.147348in}}{\pgfqpoint{2.735294in}{2.735294in}}%
\pgfusepath{clip}%
\pgfsetbuttcap%
\pgfsetroundjoin%
\definecolor{currentfill}{rgb}{0.060773,0.233289,0.352874}%
\pgfsetfillcolor{currentfill}%
\pgfsetlinewidth{0.000000pt}%
\definecolor{currentstroke}{rgb}{0.000000,0.000000,0.000000}%
\pgfsetstrokecolor{currentstroke}%
\pgfsetdash{}{0pt}%
\pgfpathmoveto{\pgfqpoint{8.223548in}{1.534090in}}%
\pgfpathlineto{\pgfqpoint{8.223548in}{1.947297in}}%
\pgfpathlineto{\pgfqpoint{8.502215in}{1.541694in}}%
\pgfpathlineto{\pgfqpoint{8.223548in}{1.534090in}}%
\pgfpathclose%
\pgfusepath{fill}%
\end{pgfscope}%
\begin{pgfscope}%
\pgfpathrectangle{\pgfqpoint{6.818937in}{0.147348in}}{\pgfqpoint{2.735294in}{2.735294in}}%
\pgfusepath{clip}%
\pgfsetbuttcap%
\pgfsetroundjoin%
\definecolor{currentfill}{rgb}{0.060773,0.233289,0.352874}%
\pgfsetfillcolor{currentfill}%
\pgfsetlinewidth{0.000000pt}%
\definecolor{currentstroke}{rgb}{0.000000,0.000000,0.000000}%
\pgfsetstrokecolor{currentstroke}%
\pgfsetdash{}{0pt}%
\pgfpathmoveto{\pgfqpoint{7.944881in}{1.541694in}}%
\pgfpathlineto{\pgfqpoint{8.223548in}{1.947297in}}%
\pgfpathlineto{\pgfqpoint{8.223548in}{1.534090in}}%
\pgfpathlineto{\pgfqpoint{7.944881in}{1.541694in}}%
\pgfpathclose%
\pgfusepath{fill}%
\end{pgfscope}%
\begin{pgfscope}%
\pgfpathrectangle{\pgfqpoint{6.818937in}{0.147348in}}{\pgfqpoint{2.735294in}{2.735294in}}%
\pgfusepath{clip}%
\pgfsetbuttcap%
\pgfsetroundjoin%
\definecolor{currentfill}{rgb}{0.060634,0.232757,0.352069}%
\pgfsetfillcolor{currentfill}%
\pgfsetlinewidth{0.000000pt}%
\definecolor{currentstroke}{rgb}{0.000000,0.000000,0.000000}%
\pgfsetstrokecolor{currentstroke}%
\pgfsetdash{}{0pt}%
\pgfpathmoveto{\pgfqpoint{8.487849in}{1.947240in}}%
\pgfpathlineto{\pgfqpoint{8.502215in}{1.541694in}}%
\pgfpathlineto{\pgfqpoint{8.223548in}{1.947297in}}%
\pgfpathlineto{\pgfqpoint{8.487849in}{1.947240in}}%
\pgfpathclose%
\pgfusepath{fill}%
\end{pgfscope}%
\begin{pgfscope}%
\pgfpathrectangle{\pgfqpoint{6.818937in}{0.147348in}}{\pgfqpoint{2.735294in}{2.735294in}}%
\pgfusepath{clip}%
\pgfsetbuttcap%
\pgfsetroundjoin%
\definecolor{currentfill}{rgb}{0.060634,0.232757,0.352069}%
\pgfsetfillcolor{currentfill}%
\pgfsetlinewidth{0.000000pt}%
\definecolor{currentstroke}{rgb}{0.000000,0.000000,0.000000}%
\pgfsetstrokecolor{currentstroke}%
\pgfsetdash{}{0pt}%
\pgfpathmoveto{\pgfqpoint{8.223548in}{1.947297in}}%
\pgfpathlineto{\pgfqpoint{7.944881in}{1.541694in}}%
\pgfpathlineto{\pgfqpoint{7.959246in}{1.947240in}}%
\pgfpathlineto{\pgfqpoint{8.223548in}{1.947297in}}%
\pgfpathclose%
\pgfusepath{fill}%
\end{pgfscope}%
\begin{pgfscope}%
\pgfpathrectangle{\pgfqpoint{6.818937in}{0.147348in}}{\pgfqpoint{2.735294in}{2.735294in}}%
\pgfusepath{clip}%
\pgfsetbuttcap%
\pgfsetroundjoin%
\definecolor{currentfill}{rgb}{0.053541,0.205528,0.310883}%
\pgfsetfillcolor{currentfill}%
\pgfsetlinewidth{0.000000pt}%
\definecolor{currentstroke}{rgb}{0.000000,0.000000,0.000000}%
\pgfsetstrokecolor{currentstroke}%
\pgfsetdash{}{0pt}%
\pgfpathmoveto{\pgfqpoint{8.752556in}{1.562133in}}%
\pgfpathlineto{\pgfqpoint{8.627169in}{1.359642in}}%
\pgfpathlineto{\pgfqpoint{8.502215in}{1.541694in}}%
\pgfpathlineto{\pgfqpoint{8.752556in}{1.562133in}}%
\pgfpathclose%
\pgfusepath{fill}%
\end{pgfscope}%
\begin{pgfscope}%
\pgfpathrectangle{\pgfqpoint{6.818937in}{0.147348in}}{\pgfqpoint{2.735294in}{2.735294in}}%
\pgfusepath{clip}%
\pgfsetbuttcap%
\pgfsetroundjoin%
\definecolor{currentfill}{rgb}{0.053541,0.205528,0.310883}%
\pgfsetfillcolor{currentfill}%
\pgfsetlinewidth{0.000000pt}%
\definecolor{currentstroke}{rgb}{0.000000,0.000000,0.000000}%
\pgfsetstrokecolor{currentstroke}%
\pgfsetdash{}{0pt}%
\pgfpathmoveto{\pgfqpoint{7.944881in}{1.541694in}}%
\pgfpathlineto{\pgfqpoint{7.819927in}{1.359642in}}%
\pgfpathlineto{\pgfqpoint{7.694540in}{1.562133in}}%
\pgfpathlineto{\pgfqpoint{7.944881in}{1.541694in}}%
\pgfpathclose%
\pgfusepath{fill}%
\end{pgfscope}%
\begin{pgfscope}%
\pgfpathrectangle{\pgfqpoint{6.818937in}{0.147348in}}{\pgfqpoint{2.735294in}{2.735294in}}%
\pgfusepath{clip}%
\pgfsetbuttcap%
\pgfsetroundjoin%
\definecolor{currentfill}{rgb}{0.061576,0.236373,0.357539}%
\pgfsetfillcolor{currentfill}%
\pgfsetlinewidth{0.000000pt}%
\definecolor{currentstroke}{rgb}{0.000000,0.000000,0.000000}%
\pgfsetstrokecolor{currentstroke}%
\pgfsetdash{}{0pt}%
\pgfpathmoveto{\pgfqpoint{7.694540in}{1.562133in}}%
\pgfpathlineto{\pgfqpoint{7.959246in}{1.947240in}}%
\pgfpathlineto{\pgfqpoint{7.944881in}{1.541694in}}%
\pgfpathlineto{\pgfqpoint{7.694540in}{1.562133in}}%
\pgfpathclose%
\pgfusepath{fill}%
\end{pgfscope}%
\begin{pgfscope}%
\pgfpathrectangle{\pgfqpoint{6.818937in}{0.147348in}}{\pgfqpoint{2.735294in}{2.735294in}}%
\pgfusepath{clip}%
\pgfsetbuttcap%
\pgfsetroundjoin%
\definecolor{currentfill}{rgb}{0.061576,0.236373,0.357539}%
\pgfsetfillcolor{currentfill}%
\pgfsetlinewidth{0.000000pt}%
\definecolor{currentstroke}{rgb}{0.000000,0.000000,0.000000}%
\pgfsetstrokecolor{currentstroke}%
\pgfsetdash{}{0pt}%
\pgfpathmoveto{\pgfqpoint{8.502215in}{1.541694in}}%
\pgfpathlineto{\pgfqpoint{8.487849in}{1.947240in}}%
\pgfpathlineto{\pgfqpoint{8.752556in}{1.562133in}}%
\pgfpathlineto{\pgfqpoint{8.502215in}{1.541694in}}%
\pgfpathclose%
\pgfusepath{fill}%
\end{pgfscope}%
\begin{pgfscope}%
\pgfpathrectangle{\pgfqpoint{6.818937in}{0.147348in}}{\pgfqpoint{2.735294in}{2.735294in}}%
\pgfusepath{clip}%
\pgfsetbuttcap%
\pgfsetroundjoin%
\definecolor{currentfill}{rgb}{0.049465,0.189883,0.287218}%
\pgfsetfillcolor{currentfill}%
\pgfsetlinewidth{0.000000pt}%
\definecolor{currentstroke}{rgb}{0.000000,0.000000,0.000000}%
\pgfsetstrokecolor{currentstroke}%
\pgfsetdash{}{0pt}%
\pgfpathmoveto{\pgfqpoint{8.591651in}{1.053752in}}%
\pgfpathlineto{\pgfqpoint{8.362938in}{1.338584in}}%
\pgfpathlineto{\pgfqpoint{8.502215in}{1.541694in}}%
\pgfpathlineto{\pgfqpoint{8.591651in}{1.053752in}}%
\pgfpathclose%
\pgfusepath{fill}%
\end{pgfscope}%
\begin{pgfscope}%
\pgfpathrectangle{\pgfqpoint{6.818937in}{0.147348in}}{\pgfqpoint{2.735294in}{2.735294in}}%
\pgfusepath{clip}%
\pgfsetbuttcap%
\pgfsetroundjoin%
\definecolor{currentfill}{rgb}{0.049465,0.189883,0.287218}%
\pgfsetfillcolor{currentfill}%
\pgfsetlinewidth{0.000000pt}%
\definecolor{currentstroke}{rgb}{0.000000,0.000000,0.000000}%
\pgfsetstrokecolor{currentstroke}%
\pgfsetdash{}{0pt}%
\pgfpathmoveto{\pgfqpoint{7.944881in}{1.541694in}}%
\pgfpathlineto{\pgfqpoint{8.084158in}{1.338584in}}%
\pgfpathlineto{\pgfqpoint{7.855445in}{1.053752in}}%
\pgfpathlineto{\pgfqpoint{7.944881in}{1.541694in}}%
\pgfpathclose%
\pgfusepath{fill}%
\end{pgfscope}%
\begin{pgfscope}%
\pgfpathrectangle{\pgfqpoint{6.818937in}{0.147348in}}{\pgfqpoint{2.735294in}{2.735294in}}%
\pgfusepath{clip}%
\pgfsetbuttcap%
\pgfsetroundjoin%
\definecolor{currentfill}{rgb}{0.069261,0.265872,0.402159}%
\pgfsetfillcolor{currentfill}%
\pgfsetlinewidth{0.000000pt}%
\definecolor{currentstroke}{rgb}{0.000000,0.000000,0.000000}%
\pgfsetstrokecolor{currentstroke}%
\pgfsetdash{}{0pt}%
\pgfpathmoveto{\pgfqpoint{8.223548in}{1.947297in}}%
\pgfpathlineto{\pgfqpoint{8.349584in}{2.130589in}}%
\pgfpathlineto{\pgfqpoint{8.487849in}{1.947240in}}%
\pgfpathlineto{\pgfqpoint{8.223548in}{1.947297in}}%
\pgfpathclose%
\pgfusepath{fill}%
\end{pgfscope}%
\begin{pgfscope}%
\pgfpathrectangle{\pgfqpoint{6.818937in}{0.147348in}}{\pgfqpoint{2.735294in}{2.735294in}}%
\pgfusepath{clip}%
\pgfsetbuttcap%
\pgfsetroundjoin%
\definecolor{currentfill}{rgb}{0.069261,0.265872,0.402159}%
\pgfsetfillcolor{currentfill}%
\pgfsetlinewidth{0.000000pt}%
\definecolor{currentstroke}{rgb}{0.000000,0.000000,0.000000}%
\pgfsetstrokecolor{currentstroke}%
\pgfsetdash{}{0pt}%
\pgfpathmoveto{\pgfqpoint{7.959246in}{1.947240in}}%
\pgfpathlineto{\pgfqpoint{8.097511in}{2.130589in}}%
\pgfpathlineto{\pgfqpoint{8.223548in}{1.947297in}}%
\pgfpathlineto{\pgfqpoint{7.959246in}{1.947240in}}%
\pgfpathclose%
\pgfusepath{fill}%
\end{pgfscope}%
\begin{pgfscope}%
\pgfpathrectangle{\pgfqpoint{6.818937in}{0.147348in}}{\pgfqpoint{2.735294in}{2.735294in}}%
\pgfusepath{clip}%
\pgfsetbuttcap%
\pgfsetroundjoin%
\definecolor{currentfill}{rgb}{0.046814,0.179706,0.271825}%
\pgfsetfillcolor{currentfill}%
\pgfsetlinewidth{0.000000pt}%
\definecolor{currentstroke}{rgb}{0.000000,0.000000,0.000000}%
\pgfsetstrokecolor{currentstroke}%
\pgfsetdash{}{0pt}%
\pgfpathmoveto{\pgfqpoint{8.362938in}{1.338584in}}%
\pgfpathlineto{\pgfqpoint{8.223548in}{0.917160in}}%
\pgfpathlineto{\pgfqpoint{8.223548in}{1.534090in}}%
\pgfpathlineto{\pgfqpoint{8.362938in}{1.338584in}}%
\pgfpathclose%
\pgfusepath{fill}%
\end{pgfscope}%
\begin{pgfscope}%
\pgfpathrectangle{\pgfqpoint{6.818937in}{0.147348in}}{\pgfqpoint{2.735294in}{2.735294in}}%
\pgfusepath{clip}%
\pgfsetbuttcap%
\pgfsetroundjoin%
\definecolor{currentfill}{rgb}{0.046814,0.179706,0.271825}%
\pgfsetfillcolor{currentfill}%
\pgfsetlinewidth{0.000000pt}%
\definecolor{currentstroke}{rgb}{0.000000,0.000000,0.000000}%
\pgfsetstrokecolor{currentstroke}%
\pgfsetdash{}{0pt}%
\pgfpathmoveto{\pgfqpoint{8.223548in}{1.534090in}}%
\pgfpathlineto{\pgfqpoint{8.223548in}{0.917160in}}%
\pgfpathlineto{\pgfqpoint{8.084158in}{1.338584in}}%
\pgfpathlineto{\pgfqpoint{8.223548in}{1.534090in}}%
\pgfpathclose%
\pgfusepath{fill}%
\end{pgfscope}%
\begin{pgfscope}%
\pgfpathrectangle{\pgfqpoint{6.818937in}{0.147348in}}{\pgfqpoint{2.735294in}{2.735294in}}%
\pgfusepath{clip}%
\pgfsetbuttcap%
\pgfsetroundjoin%
\definecolor{currentfill}{rgb}{0.045820,0.175891,0.266053}%
\pgfsetfillcolor{currentfill}%
\pgfsetlinewidth{0.000000pt}%
\definecolor{currentstroke}{rgb}{0.000000,0.000000,0.000000}%
\pgfsetstrokecolor{currentstroke}%
\pgfsetdash{}{0pt}%
\pgfpathmoveto{\pgfqpoint{8.591651in}{1.053752in}}%
\pgfpathlineto{\pgfqpoint{8.502215in}{1.541694in}}%
\pgfpathlineto{\pgfqpoint{8.627169in}{1.359642in}}%
\pgfpathlineto{\pgfqpoint{8.591651in}{1.053752in}}%
\pgfpathclose%
\pgfusepath{fill}%
\end{pgfscope}%
\begin{pgfscope}%
\pgfpathrectangle{\pgfqpoint{6.818937in}{0.147348in}}{\pgfqpoint{2.735294in}{2.735294in}}%
\pgfusepath{clip}%
\pgfsetbuttcap%
\pgfsetroundjoin%
\definecolor{currentfill}{rgb}{0.045820,0.175891,0.266053}%
\pgfsetfillcolor{currentfill}%
\pgfsetlinewidth{0.000000pt}%
\definecolor{currentstroke}{rgb}{0.000000,0.000000,0.000000}%
\pgfsetstrokecolor{currentstroke}%
\pgfsetdash{}{0pt}%
\pgfpathmoveto{\pgfqpoint{7.819927in}{1.359642in}}%
\pgfpathlineto{\pgfqpoint{7.944881in}{1.541694in}}%
\pgfpathlineto{\pgfqpoint{7.855445in}{1.053752in}}%
\pgfpathlineto{\pgfqpoint{7.819927in}{1.359642in}}%
\pgfpathclose%
\pgfusepath{fill}%
\end{pgfscope}%
\begin{pgfscope}%
\pgfpathrectangle{\pgfqpoint{6.818937in}{0.147348in}}{\pgfqpoint{2.735294in}{2.735294in}}%
\pgfusepath{clip}%
\pgfsetbuttcap%
\pgfsetroundjoin%
\definecolor{currentfill}{rgb}{0.071636,0.274990,0.415951}%
\pgfsetfillcolor{currentfill}%
\pgfsetlinewidth{0.000000pt}%
\definecolor{currentstroke}{rgb}{0.000000,0.000000,0.000000}%
\pgfsetstrokecolor{currentstroke}%
\pgfsetdash{}{0pt}%
\pgfpathmoveto{\pgfqpoint{8.223548in}{1.947297in}}%
\pgfpathlineto{\pgfqpoint{8.097511in}{2.130589in}}%
\pgfpathlineto{\pgfqpoint{8.349584in}{2.130589in}}%
\pgfpathlineto{\pgfqpoint{8.223548in}{1.947297in}}%
\pgfpathclose%
\pgfusepath{fill}%
\end{pgfscope}%
\begin{pgfscope}%
\pgfpathrectangle{\pgfqpoint{6.818937in}{0.147348in}}{\pgfqpoint{2.735294in}{2.735294in}}%
\pgfusepath{clip}%
\pgfsetbuttcap%
\pgfsetroundjoin%
\definecolor{currentfill}{rgb}{0.071694,0.275212,0.416288}%
\pgfsetfillcolor{currentfill}%
\pgfsetlinewidth{0.000000pt}%
\definecolor{currentstroke}{rgb}{0.000000,0.000000,0.000000}%
\pgfsetstrokecolor{currentstroke}%
\pgfsetdash{}{0pt}%
\pgfpathmoveto{\pgfqpoint{8.487849in}{1.947240in}}%
\pgfpathlineto{\pgfqpoint{8.349584in}{2.130589in}}%
\pgfpathlineto{\pgfqpoint{8.590867in}{2.125260in}}%
\pgfpathlineto{\pgfqpoint{8.487849in}{1.947240in}}%
\pgfpathclose%
\pgfusepath{fill}%
\end{pgfscope}%
\begin{pgfscope}%
\pgfpathrectangle{\pgfqpoint{6.818937in}{0.147348in}}{\pgfqpoint{2.735294in}{2.735294in}}%
\pgfusepath{clip}%
\pgfsetbuttcap%
\pgfsetroundjoin%
\definecolor{currentfill}{rgb}{0.071694,0.275212,0.416288}%
\pgfsetfillcolor{currentfill}%
\pgfsetlinewidth{0.000000pt}%
\definecolor{currentstroke}{rgb}{0.000000,0.000000,0.000000}%
\pgfsetstrokecolor{currentstroke}%
\pgfsetdash{}{0pt}%
\pgfpathmoveto{\pgfqpoint{7.856229in}{2.125260in}}%
\pgfpathlineto{\pgfqpoint{8.097511in}{2.130589in}}%
\pgfpathlineto{\pgfqpoint{7.959246in}{1.947240in}}%
\pgfpathlineto{\pgfqpoint{7.856229in}{2.125260in}}%
\pgfpathclose%
\pgfusepath{fill}%
\end{pgfscope}%
\begin{pgfscope}%
\pgfpathrectangle{\pgfqpoint{6.818937in}{0.147348in}}{\pgfqpoint{2.735294in}{2.735294in}}%
\pgfusepath{clip}%
\pgfsetbuttcap%
\pgfsetroundjoin%
\definecolor{currentfill}{rgb}{0.064759,0.248590,0.376018}%
\pgfsetfillcolor{currentfill}%
\pgfsetlinewidth{0.000000pt}%
\definecolor{currentstroke}{rgb}{0.000000,0.000000,0.000000}%
\pgfsetstrokecolor{currentstroke}%
\pgfsetdash{}{0pt}%
\pgfpathmoveto{\pgfqpoint{8.752556in}{1.562133in}}%
\pgfpathlineto{\pgfqpoint{8.487849in}{1.947240in}}%
\pgfpathlineto{\pgfqpoint{8.804056in}{2.115869in}}%
\pgfpathlineto{\pgfqpoint{8.752556in}{1.562133in}}%
\pgfpathclose%
\pgfusepath{fill}%
\end{pgfscope}%
\begin{pgfscope}%
\pgfpathrectangle{\pgfqpoint{6.818937in}{0.147348in}}{\pgfqpoint{2.735294in}{2.735294in}}%
\pgfusepath{clip}%
\pgfsetbuttcap%
\pgfsetroundjoin%
\definecolor{currentfill}{rgb}{0.064759,0.248590,0.376018}%
\pgfsetfillcolor{currentfill}%
\pgfsetlinewidth{0.000000pt}%
\definecolor{currentstroke}{rgb}{0.000000,0.000000,0.000000}%
\pgfsetstrokecolor{currentstroke}%
\pgfsetdash{}{0pt}%
\pgfpathmoveto{\pgfqpoint{7.643040in}{2.115869in}}%
\pgfpathlineto{\pgfqpoint{7.959246in}{1.947240in}}%
\pgfpathlineto{\pgfqpoint{7.694540in}{1.562133in}}%
\pgfpathlineto{\pgfqpoint{7.643040in}{2.115869in}}%
\pgfpathclose%
\pgfusepath{fill}%
\end{pgfscope}%
\begin{pgfscope}%
\pgfpathrectangle{\pgfqpoint{6.818937in}{0.147348in}}{\pgfqpoint{2.735294in}{2.735294in}}%
\pgfusepath{clip}%
\pgfsetbuttcap%
\pgfsetroundjoin%
\definecolor{currentfill}{rgb}{0.051850,0.199036,0.301063}%
\pgfsetfillcolor{currentfill}%
\pgfsetlinewidth{0.000000pt}%
\definecolor{currentstroke}{rgb}{0.000000,0.000000,0.000000}%
\pgfsetstrokecolor{currentstroke}%
\pgfsetdash{}{0pt}%
\pgfpathmoveto{\pgfqpoint{8.805232in}{1.099921in}}%
\pgfpathlineto{\pgfqpoint{8.627169in}{1.359642in}}%
\pgfpathlineto{\pgfqpoint{8.752556in}{1.562133in}}%
\pgfpathlineto{\pgfqpoint{8.805232in}{1.099921in}}%
\pgfpathclose%
\pgfusepath{fill}%
\end{pgfscope}%
\begin{pgfscope}%
\pgfpathrectangle{\pgfqpoint{6.818937in}{0.147348in}}{\pgfqpoint{2.735294in}{2.735294in}}%
\pgfusepath{clip}%
\pgfsetbuttcap%
\pgfsetroundjoin%
\definecolor{currentfill}{rgb}{0.051850,0.199036,0.301063}%
\pgfsetfillcolor{currentfill}%
\pgfsetlinewidth{0.000000pt}%
\definecolor{currentstroke}{rgb}{0.000000,0.000000,0.000000}%
\pgfsetstrokecolor{currentstroke}%
\pgfsetdash{}{0pt}%
\pgfpathmoveto{\pgfqpoint{7.694540in}{1.562133in}}%
\pgfpathlineto{\pgfqpoint{7.819927in}{1.359642in}}%
\pgfpathlineto{\pgfqpoint{7.641864in}{1.099921in}}%
\pgfpathlineto{\pgfqpoint{7.694540in}{1.562133in}}%
\pgfpathclose%
\pgfusepath{fill}%
\end{pgfscope}%
\begin{pgfscope}%
\pgfpathrectangle{\pgfqpoint{6.818937in}{0.147348in}}{\pgfqpoint{2.735294in}{2.735294in}}%
\pgfusepath{clip}%
\pgfsetbuttcap%
\pgfsetroundjoin%
\definecolor{currentfill}{rgb}{0.046101,0.176968,0.267683}%
\pgfsetfillcolor{currentfill}%
\pgfsetlinewidth{0.000000pt}%
\definecolor{currentstroke}{rgb}{0.000000,0.000000,0.000000}%
\pgfsetstrokecolor{currentstroke}%
\pgfsetdash{}{0pt}%
\pgfpathmoveto{\pgfqpoint{8.223548in}{0.917160in}}%
\pgfpathlineto{\pgfqpoint{8.362938in}{1.338584in}}%
\pgfpathlineto{\pgfqpoint{8.349861in}{1.027548in}}%
\pgfpathlineto{\pgfqpoint{8.223548in}{0.917160in}}%
\pgfpathclose%
\pgfusepath{fill}%
\end{pgfscope}%
\begin{pgfscope}%
\pgfpathrectangle{\pgfqpoint{6.818937in}{0.147348in}}{\pgfqpoint{2.735294in}{2.735294in}}%
\pgfusepath{clip}%
\pgfsetbuttcap%
\pgfsetroundjoin%
\definecolor{currentfill}{rgb}{0.046101,0.176968,0.267683}%
\pgfsetfillcolor{currentfill}%
\pgfsetlinewidth{0.000000pt}%
\definecolor{currentstroke}{rgb}{0.000000,0.000000,0.000000}%
\pgfsetstrokecolor{currentstroke}%
\pgfsetdash{}{0pt}%
\pgfpathmoveto{\pgfqpoint{8.097234in}{1.027548in}}%
\pgfpathlineto{\pgfqpoint{8.084158in}{1.338584in}}%
\pgfpathlineto{\pgfqpoint{8.223548in}{0.917160in}}%
\pgfpathlineto{\pgfqpoint{8.097234in}{1.027548in}}%
\pgfpathclose%
\pgfusepath{fill}%
\end{pgfscope}%
\begin{pgfscope}%
\pgfpathrectangle{\pgfqpoint{6.818937in}{0.147348in}}{\pgfqpoint{2.735294in}{2.735294in}}%
\pgfusepath{clip}%
\pgfsetbuttcap%
\pgfsetroundjoin%
\definecolor{currentfill}{rgb}{0.047555,0.182548,0.276123}%
\pgfsetfillcolor{currentfill}%
\pgfsetlinewidth{0.000000pt}%
\definecolor{currentstroke}{rgb}{0.000000,0.000000,0.000000}%
\pgfsetstrokecolor{currentstroke}%
\pgfsetdash{}{0pt}%
\pgfpathmoveto{\pgfqpoint{8.456300in}{0.930196in}}%
\pgfpathlineto{\pgfqpoint{8.349861in}{1.027548in}}%
\pgfpathlineto{\pgfqpoint{8.362938in}{1.338584in}}%
\pgfpathlineto{\pgfqpoint{8.456300in}{0.930196in}}%
\pgfpathclose%
\pgfusepath{fill}%
\end{pgfscope}%
\begin{pgfscope}%
\pgfpathrectangle{\pgfqpoint{6.818937in}{0.147348in}}{\pgfqpoint{2.735294in}{2.735294in}}%
\pgfusepath{clip}%
\pgfsetbuttcap%
\pgfsetroundjoin%
\definecolor{currentfill}{rgb}{0.047555,0.182548,0.276123}%
\pgfsetfillcolor{currentfill}%
\pgfsetlinewidth{0.000000pt}%
\definecolor{currentstroke}{rgb}{0.000000,0.000000,0.000000}%
\pgfsetstrokecolor{currentstroke}%
\pgfsetdash{}{0pt}%
\pgfpathmoveto{\pgfqpoint{8.084158in}{1.338584in}}%
\pgfpathlineto{\pgfqpoint{8.097234in}{1.027548in}}%
\pgfpathlineto{\pgfqpoint{7.990796in}{0.930196in}}%
\pgfpathlineto{\pgfqpoint{8.084158in}{1.338584in}}%
\pgfpathclose%
\pgfusepath{fill}%
\end{pgfscope}%
\begin{pgfscope}%
\pgfpathrectangle{\pgfqpoint{6.818937in}{0.147348in}}{\pgfqpoint{2.735294in}{2.735294in}}%
\pgfusepath{clip}%
\pgfsetbuttcap%
\pgfsetroundjoin%
\definecolor{currentfill}{rgb}{0.068541,0.263111,0.397982}%
\pgfsetfillcolor{currentfill}%
\pgfsetlinewidth{0.000000pt}%
\definecolor{currentstroke}{rgb}{0.000000,0.000000,0.000000}%
\pgfsetstrokecolor{currentstroke}%
\pgfsetdash{}{0pt}%
\pgfpathmoveto{\pgfqpoint{8.590867in}{2.125260in}}%
\pgfpathlineto{\pgfqpoint{8.804056in}{2.115869in}}%
\pgfpathlineto{\pgfqpoint{8.487849in}{1.947240in}}%
\pgfpathlineto{\pgfqpoint{8.590867in}{2.125260in}}%
\pgfpathclose%
\pgfusepath{fill}%
\end{pgfscope}%
\begin{pgfscope}%
\pgfpathrectangle{\pgfqpoint{6.818937in}{0.147348in}}{\pgfqpoint{2.735294in}{2.735294in}}%
\pgfusepath{clip}%
\pgfsetbuttcap%
\pgfsetroundjoin%
\definecolor{currentfill}{rgb}{0.068541,0.263111,0.397982}%
\pgfsetfillcolor{currentfill}%
\pgfsetlinewidth{0.000000pt}%
\definecolor{currentstroke}{rgb}{0.000000,0.000000,0.000000}%
\pgfsetstrokecolor{currentstroke}%
\pgfsetdash{}{0pt}%
\pgfpathmoveto{\pgfqpoint{7.959246in}{1.947240in}}%
\pgfpathlineto{\pgfqpoint{7.643040in}{2.115869in}}%
\pgfpathlineto{\pgfqpoint{7.856229in}{2.125260in}}%
\pgfpathlineto{\pgfqpoint{7.959246in}{1.947240in}}%
\pgfpathclose%
\pgfusepath{fill}%
\end{pgfscope}%
\begin{pgfscope}%
\pgfpathrectangle{\pgfqpoint{6.818937in}{0.147348in}}{\pgfqpoint{2.735294in}{2.735294in}}%
\pgfusepath{clip}%
\pgfsetbuttcap%
\pgfsetroundjoin%
\definecolor{currentfill}{rgb}{0.042669,0.163794,0.247755}%
\pgfsetfillcolor{currentfill}%
\pgfsetlinewidth{0.000000pt}%
\definecolor{currentstroke}{rgb}{0.000000,0.000000,0.000000}%
\pgfsetstrokecolor{currentstroke}%
\pgfsetdash{}{0pt}%
\pgfpathmoveto{\pgfqpoint{8.456300in}{0.930196in}}%
\pgfpathlineto{\pgfqpoint{8.362938in}{1.338584in}}%
\pgfpathlineto{\pgfqpoint{8.591651in}{1.053752in}}%
\pgfpathlineto{\pgfqpoint{8.456300in}{0.930196in}}%
\pgfpathclose%
\pgfusepath{fill}%
\end{pgfscope}%
\begin{pgfscope}%
\pgfpathrectangle{\pgfqpoint{6.818937in}{0.147348in}}{\pgfqpoint{2.735294in}{2.735294in}}%
\pgfusepath{clip}%
\pgfsetbuttcap%
\pgfsetroundjoin%
\definecolor{currentfill}{rgb}{0.042669,0.163794,0.247755}%
\pgfsetfillcolor{currentfill}%
\pgfsetlinewidth{0.000000pt}%
\definecolor{currentstroke}{rgb}{0.000000,0.000000,0.000000}%
\pgfsetstrokecolor{currentstroke}%
\pgfsetdash{}{0pt}%
\pgfpathmoveto{\pgfqpoint{7.855445in}{1.053752in}}%
\pgfpathlineto{\pgfqpoint{8.084158in}{1.338584in}}%
\pgfpathlineto{\pgfqpoint{7.990796in}{0.930196in}}%
\pgfpathlineto{\pgfqpoint{7.855445in}{1.053752in}}%
\pgfpathclose%
\pgfusepath{fill}%
\end{pgfscope}%
\begin{pgfscope}%
\pgfpathrectangle{\pgfqpoint{6.818937in}{0.147348in}}{\pgfqpoint{2.735294in}{2.735294in}}%
\pgfusepath{clip}%
\pgfsetbuttcap%
\pgfsetroundjoin%
\definecolor{currentfill}{rgb}{0.064954,0.249341,0.377155}%
\pgfsetfillcolor{currentfill}%
\pgfsetlinewidth{0.000000pt}%
\definecolor{currentstroke}{rgb}{0.000000,0.000000,0.000000}%
\pgfsetstrokecolor{currentstroke}%
\pgfsetdash{}{0pt}%
\pgfpathmoveto{\pgfqpoint{8.752556in}{1.562133in}}%
\pgfpathlineto{\pgfqpoint{8.804056in}{2.115869in}}%
\pgfpathlineto{\pgfqpoint{8.958969in}{1.590078in}}%
\pgfpathlineto{\pgfqpoint{8.752556in}{1.562133in}}%
\pgfpathclose%
\pgfusepath{fill}%
\end{pgfscope}%
\begin{pgfscope}%
\pgfpathrectangle{\pgfqpoint{6.818937in}{0.147348in}}{\pgfqpoint{2.735294in}{2.735294in}}%
\pgfusepath{clip}%
\pgfsetbuttcap%
\pgfsetroundjoin%
\definecolor{currentfill}{rgb}{0.064954,0.249341,0.377155}%
\pgfsetfillcolor{currentfill}%
\pgfsetlinewidth{0.000000pt}%
\definecolor{currentstroke}{rgb}{0.000000,0.000000,0.000000}%
\pgfsetstrokecolor{currentstroke}%
\pgfsetdash{}{0pt}%
\pgfpathmoveto{\pgfqpoint{7.488127in}{1.590078in}}%
\pgfpathlineto{\pgfqpoint{7.643040in}{2.115869in}}%
\pgfpathlineto{\pgfqpoint{7.694540in}{1.562133in}}%
\pgfpathlineto{\pgfqpoint{7.488127in}{1.590078in}}%
\pgfpathclose%
\pgfusepath{fill}%
\end{pgfscope}%
\begin{pgfscope}%
\pgfpathrectangle{\pgfqpoint{6.818937in}{0.147348in}}{\pgfqpoint{2.735294in}{2.735294in}}%
\pgfusepath{clip}%
\pgfsetbuttcap%
\pgfsetroundjoin%
\definecolor{currentfill}{rgb}{0.078663,0.301965,0.456754}%
\pgfsetfillcolor{currentfill}%
\pgfsetlinewidth{0.000000pt}%
\definecolor{currentstroke}{rgb}{0.000000,0.000000,0.000000}%
\pgfsetstrokecolor{currentstroke}%
\pgfsetdash{}{0pt}%
\pgfpathmoveto{\pgfqpoint{7.856229in}{2.125260in}}%
\pgfpathlineto{\pgfqpoint{7.775860in}{2.272632in}}%
\pgfpathlineto{\pgfqpoint{8.097511in}{2.130589in}}%
\pgfpathlineto{\pgfqpoint{7.856229in}{2.125260in}}%
\pgfpathclose%
\pgfusepath{fill}%
\end{pgfscope}%
\begin{pgfscope}%
\pgfpathrectangle{\pgfqpoint{6.818937in}{0.147348in}}{\pgfqpoint{2.735294in}{2.735294in}}%
\pgfusepath{clip}%
\pgfsetbuttcap%
\pgfsetroundjoin%
\definecolor{currentfill}{rgb}{0.078663,0.301965,0.456754}%
\pgfsetfillcolor{currentfill}%
\pgfsetlinewidth{0.000000pt}%
\definecolor{currentstroke}{rgb}{0.000000,0.000000,0.000000}%
\pgfsetstrokecolor{currentstroke}%
\pgfsetdash{}{0pt}%
\pgfpathmoveto{\pgfqpoint{8.349584in}{2.130589in}}%
\pgfpathlineto{\pgfqpoint{8.671236in}{2.272632in}}%
\pgfpathlineto{\pgfqpoint{8.590867in}{2.125260in}}%
\pgfpathlineto{\pgfqpoint{8.349584in}{2.130589in}}%
\pgfpathclose%
\pgfusepath{fill}%
\end{pgfscope}%
\begin{pgfscope}%
\pgfpathrectangle{\pgfqpoint{6.818937in}{0.147348in}}{\pgfqpoint{2.735294in}{2.735294in}}%
\pgfusepath{clip}%
\pgfsetbuttcap%
\pgfsetroundjoin%
\definecolor{currentfill}{rgb}{0.049941,0.191710,0.289982}%
\pgfsetfillcolor{currentfill}%
\pgfsetlinewidth{0.000000pt}%
\definecolor{currentstroke}{rgb}{0.000000,0.000000,0.000000}%
\pgfsetstrokecolor{currentstroke}%
\pgfsetdash{}{0pt}%
\pgfpathmoveto{\pgfqpoint{7.855445in}{1.053752in}}%
\pgfpathlineto{\pgfqpoint{7.774694in}{0.966453in}}%
\pgfpathlineto{\pgfqpoint{7.819927in}{1.359642in}}%
\pgfpathlineto{\pgfqpoint{7.855445in}{1.053752in}}%
\pgfpathclose%
\pgfusepath{fill}%
\end{pgfscope}%
\begin{pgfscope}%
\pgfpathrectangle{\pgfqpoint{6.818937in}{0.147348in}}{\pgfqpoint{2.735294in}{2.735294in}}%
\pgfusepath{clip}%
\pgfsetbuttcap%
\pgfsetroundjoin%
\definecolor{currentfill}{rgb}{0.049941,0.191710,0.289982}%
\pgfsetfillcolor{currentfill}%
\pgfsetlinewidth{0.000000pt}%
\definecolor{currentstroke}{rgb}{0.000000,0.000000,0.000000}%
\pgfsetstrokecolor{currentstroke}%
\pgfsetdash{}{0pt}%
\pgfpathmoveto{\pgfqpoint{8.627169in}{1.359642in}}%
\pgfpathlineto{\pgfqpoint{8.672401in}{0.966453in}}%
\pgfpathlineto{\pgfqpoint{8.591651in}{1.053752in}}%
\pgfpathlineto{\pgfqpoint{8.627169in}{1.359642in}}%
\pgfpathclose%
\pgfusepath{fill}%
\end{pgfscope}%
\begin{pgfscope}%
\pgfpathrectangle{\pgfqpoint{6.818937in}{0.147348in}}{\pgfqpoint{2.735294in}{2.735294in}}%
\pgfusepath{clip}%
\pgfsetbuttcap%
\pgfsetroundjoin%
\definecolor{currentfill}{rgb}{0.050011,0.191979,0.290388}%
\pgfsetfillcolor{currentfill}%
\pgfsetlinewidth{0.000000pt}%
\definecolor{currentstroke}{rgb}{0.000000,0.000000,0.000000}%
\pgfsetstrokecolor{currentstroke}%
\pgfsetdash{}{0pt}%
\pgfpathmoveto{\pgfqpoint{8.958969in}{1.590078in}}%
\pgfpathlineto{\pgfqpoint{8.982643in}{1.157190in}}%
\pgfpathlineto{\pgfqpoint{8.752556in}{1.562133in}}%
\pgfpathlineto{\pgfqpoint{8.958969in}{1.590078in}}%
\pgfpathclose%
\pgfusepath{fill}%
\end{pgfscope}%
\begin{pgfscope}%
\pgfpathrectangle{\pgfqpoint{6.818937in}{0.147348in}}{\pgfqpoint{2.735294in}{2.735294in}}%
\pgfusepath{clip}%
\pgfsetbuttcap%
\pgfsetroundjoin%
\definecolor{currentfill}{rgb}{0.050011,0.191979,0.290388}%
\pgfsetfillcolor{currentfill}%
\pgfsetlinewidth{0.000000pt}%
\definecolor{currentstroke}{rgb}{0.000000,0.000000,0.000000}%
\pgfsetstrokecolor{currentstroke}%
\pgfsetdash{}{0pt}%
\pgfpathmoveto{\pgfqpoint{7.694540in}{1.562133in}}%
\pgfpathlineto{\pgfqpoint{7.464453in}{1.157190in}}%
\pgfpathlineto{\pgfqpoint{7.488127in}{1.590078in}}%
\pgfpathlineto{\pgfqpoint{7.694540in}{1.562133in}}%
\pgfpathclose%
\pgfusepath{fill}%
\end{pgfscope}%
\begin{pgfscope}%
\pgfpathrectangle{\pgfqpoint{6.818937in}{0.147348in}}{\pgfqpoint{2.735294in}{2.735294in}}%
\pgfusepath{clip}%
\pgfsetbuttcap%
\pgfsetroundjoin%
\definecolor{currentfill}{rgb}{0.043508,0.167016,0.252629}%
\pgfsetfillcolor{currentfill}%
\pgfsetlinewidth{0.000000pt}%
\definecolor{currentstroke}{rgb}{0.000000,0.000000,0.000000}%
\pgfsetstrokecolor{currentstroke}%
\pgfsetdash{}{0pt}%
\pgfpathmoveto{\pgfqpoint{8.805232in}{1.099921in}}%
\pgfpathlineto{\pgfqpoint{8.672401in}{0.966453in}}%
\pgfpathlineto{\pgfqpoint{8.627169in}{1.359642in}}%
\pgfpathlineto{\pgfqpoint{8.805232in}{1.099921in}}%
\pgfpathclose%
\pgfusepath{fill}%
\end{pgfscope}%
\begin{pgfscope}%
\pgfpathrectangle{\pgfqpoint{6.818937in}{0.147348in}}{\pgfqpoint{2.735294in}{2.735294in}}%
\pgfusepath{clip}%
\pgfsetbuttcap%
\pgfsetroundjoin%
\definecolor{currentfill}{rgb}{0.043508,0.167016,0.252629}%
\pgfsetfillcolor{currentfill}%
\pgfsetlinewidth{0.000000pt}%
\definecolor{currentstroke}{rgb}{0.000000,0.000000,0.000000}%
\pgfsetstrokecolor{currentstroke}%
\pgfsetdash{}{0pt}%
\pgfpathmoveto{\pgfqpoint{7.819927in}{1.359642in}}%
\pgfpathlineto{\pgfqpoint{7.774694in}{0.966453in}}%
\pgfpathlineto{\pgfqpoint{7.641864in}{1.099921in}}%
\pgfpathlineto{\pgfqpoint{7.819927in}{1.359642in}}%
\pgfpathclose%
\pgfusepath{fill}%
\end{pgfscope}%
\begin{pgfscope}%
\pgfpathrectangle{\pgfqpoint{6.818937in}{0.147348in}}{\pgfqpoint{2.735294in}{2.735294in}}%
\pgfusepath{clip}%
\pgfsetbuttcap%
\pgfsetroundjoin%
\definecolor{currentfill}{rgb}{0.062760,0.240916,0.364410}%
\pgfsetfillcolor{currentfill}%
\pgfsetlinewidth{0.000000pt}%
\definecolor{currentstroke}{rgb}{0.000000,0.000000,0.000000}%
\pgfsetstrokecolor{currentstroke}%
\pgfsetdash{}{0pt}%
\pgfpathmoveto{\pgfqpoint{8.958969in}{1.590078in}}%
\pgfpathlineto{\pgfqpoint{8.804056in}{2.115869in}}%
\pgfpathlineto{\pgfqpoint{9.036807in}{1.777676in}}%
\pgfpathlineto{\pgfqpoint{8.958969in}{1.590078in}}%
\pgfpathclose%
\pgfusepath{fill}%
\end{pgfscope}%
\begin{pgfscope}%
\pgfpathrectangle{\pgfqpoint{6.818937in}{0.147348in}}{\pgfqpoint{2.735294in}{2.735294in}}%
\pgfusepath{clip}%
\pgfsetbuttcap%
\pgfsetroundjoin%
\definecolor{currentfill}{rgb}{0.062760,0.240916,0.364410}%
\pgfsetfillcolor{currentfill}%
\pgfsetlinewidth{0.000000pt}%
\definecolor{currentstroke}{rgb}{0.000000,0.000000,0.000000}%
\pgfsetstrokecolor{currentstroke}%
\pgfsetdash{}{0pt}%
\pgfpathmoveto{\pgfqpoint{7.410289in}{1.777676in}}%
\pgfpathlineto{\pgfqpoint{7.643040in}{2.115869in}}%
\pgfpathlineto{\pgfqpoint{7.488127in}{1.590078in}}%
\pgfpathlineto{\pgfqpoint{7.410289in}{1.777676in}}%
\pgfpathclose%
\pgfusepath{fill}%
\end{pgfscope}%
\begin{pgfscope}%
\pgfpathrectangle{\pgfqpoint{6.818937in}{0.147348in}}{\pgfqpoint{2.735294in}{2.735294in}}%
\pgfusepath{clip}%
\pgfsetbuttcap%
\pgfsetroundjoin%
\definecolor{currentfill}{rgb}{0.075436,0.289576,0.438014}%
\pgfsetfillcolor{currentfill}%
\pgfsetlinewidth{0.000000pt}%
\definecolor{currentstroke}{rgb}{0.000000,0.000000,0.000000}%
\pgfsetstrokecolor{currentstroke}%
\pgfsetdash{}{0pt}%
\pgfpathmoveto{\pgfqpoint{8.590867in}{2.125260in}}%
\pgfpathlineto{\pgfqpoint{8.671236in}{2.272632in}}%
\pgfpathlineto{\pgfqpoint{8.804056in}{2.115869in}}%
\pgfpathlineto{\pgfqpoint{8.590867in}{2.125260in}}%
\pgfpathclose%
\pgfusepath{fill}%
\end{pgfscope}%
\begin{pgfscope}%
\pgfpathrectangle{\pgfqpoint{6.818937in}{0.147348in}}{\pgfqpoint{2.735294in}{2.735294in}}%
\pgfusepath{clip}%
\pgfsetbuttcap%
\pgfsetroundjoin%
\definecolor{currentfill}{rgb}{0.075436,0.289576,0.438014}%
\pgfsetfillcolor{currentfill}%
\pgfsetlinewidth{0.000000pt}%
\definecolor{currentstroke}{rgb}{0.000000,0.000000,0.000000}%
\pgfsetstrokecolor{currentstroke}%
\pgfsetdash{}{0pt}%
\pgfpathmoveto{\pgfqpoint{7.643040in}{2.115869in}}%
\pgfpathlineto{\pgfqpoint{7.775860in}{2.272632in}}%
\pgfpathlineto{\pgfqpoint{7.856229in}{2.125260in}}%
\pgfpathlineto{\pgfqpoint{7.643040in}{2.115869in}}%
\pgfpathclose%
\pgfusepath{fill}%
\end{pgfscope}%
\begin{pgfscope}%
\pgfpathrectangle{\pgfqpoint{6.818937in}{0.147348in}}{\pgfqpoint{2.735294in}{2.735294in}}%
\pgfusepath{clip}%
\pgfsetbuttcap%
\pgfsetroundjoin%
\definecolor{currentfill}{rgb}{0.039595,0.151995,0.229908}%
\pgfsetfillcolor{currentfill}%
\pgfsetlinewidth{0.000000pt}%
\definecolor{currentstroke}{rgb}{0.000000,0.000000,0.000000}%
\pgfsetstrokecolor{currentstroke}%
\pgfsetdash{}{0pt}%
\pgfpathmoveto{\pgfqpoint{8.223548in}{0.917160in}}%
\pgfpathlineto{\pgfqpoint{8.349861in}{1.027548in}}%
\pgfpathlineto{\pgfqpoint{8.456300in}{0.930196in}}%
\pgfpathlineto{\pgfqpoint{8.223548in}{0.917160in}}%
\pgfpathclose%
\pgfusepath{fill}%
\end{pgfscope}%
\begin{pgfscope}%
\pgfpathrectangle{\pgfqpoint{6.818937in}{0.147348in}}{\pgfqpoint{2.735294in}{2.735294in}}%
\pgfusepath{clip}%
\pgfsetbuttcap%
\pgfsetroundjoin%
\definecolor{currentfill}{rgb}{0.039595,0.151995,0.229908}%
\pgfsetfillcolor{currentfill}%
\pgfsetlinewidth{0.000000pt}%
\definecolor{currentstroke}{rgb}{0.000000,0.000000,0.000000}%
\pgfsetstrokecolor{currentstroke}%
\pgfsetdash{}{0pt}%
\pgfpathmoveto{\pgfqpoint{7.990796in}{0.930196in}}%
\pgfpathlineto{\pgfqpoint{8.097234in}{1.027548in}}%
\pgfpathlineto{\pgfqpoint{8.223548in}{0.917160in}}%
\pgfpathlineto{\pgfqpoint{7.990796in}{0.930196in}}%
\pgfpathclose%
\pgfusepath{fill}%
\end{pgfscope}%
\begin{pgfscope}%
\pgfpathrectangle{\pgfqpoint{6.818937in}{0.147348in}}{\pgfqpoint{2.735294in}{2.735294in}}%
\pgfusepath{clip}%
\pgfsetbuttcap%
\pgfsetroundjoin%
\definecolor{currentfill}{rgb}{0.060942,0.233938,0.353856}%
\pgfsetfillcolor{currentfill}%
\pgfsetlinewidth{0.000000pt}%
\definecolor{currentstroke}{rgb}{0.000000,0.000000,0.000000}%
\pgfsetstrokecolor{currentstroke}%
\pgfsetdash{}{0pt}%
\pgfpathmoveto{\pgfqpoint{7.327008in}{1.620392in}}%
\pgfpathlineto{\pgfqpoint{7.410289in}{1.777676in}}%
\pgfpathlineto{\pgfqpoint{7.488127in}{1.590078in}}%
\pgfpathlineto{\pgfqpoint{7.327008in}{1.620392in}}%
\pgfpathclose%
\pgfusepath{fill}%
\end{pgfscope}%
\begin{pgfscope}%
\pgfpathrectangle{\pgfqpoint{6.818937in}{0.147348in}}{\pgfqpoint{2.735294in}{2.735294in}}%
\pgfusepath{clip}%
\pgfsetbuttcap%
\pgfsetroundjoin%
\definecolor{currentfill}{rgb}{0.060942,0.233938,0.353856}%
\pgfsetfillcolor{currentfill}%
\pgfsetlinewidth{0.000000pt}%
\definecolor{currentstroke}{rgb}{0.000000,0.000000,0.000000}%
\pgfsetstrokecolor{currentstroke}%
\pgfsetdash{}{0pt}%
\pgfpathmoveto{\pgfqpoint{8.958969in}{1.590078in}}%
\pgfpathlineto{\pgfqpoint{9.036807in}{1.777676in}}%
\pgfpathlineto{\pgfqpoint{9.120088in}{1.620392in}}%
\pgfpathlineto{\pgfqpoint{8.958969in}{1.590078in}}%
\pgfpathclose%
\pgfusepath{fill}%
\end{pgfscope}%
\begin{pgfscope}%
\pgfpathrectangle{\pgfqpoint{6.818937in}{0.147348in}}{\pgfqpoint{2.735294in}{2.735294in}}%
\pgfusepath{clip}%
\pgfsetbuttcap%
\pgfsetroundjoin%
\definecolor{currentfill}{rgb}{0.063981,0.245604,0.371502}%
\pgfsetfillcolor{currentfill}%
\pgfsetlinewidth{0.000000pt}%
\definecolor{currentstroke}{rgb}{0.000000,0.000000,0.000000}%
\pgfsetstrokecolor{currentstroke}%
\pgfsetdash{}{0pt}%
\pgfpathmoveto{\pgfqpoint{7.641864in}{1.099921in}}%
\pgfpathlineto{\pgfqpoint{7.586426in}{1.018945in}}%
\pgfpathlineto{\pgfqpoint{7.694540in}{1.562133in}}%
\pgfpathlineto{\pgfqpoint{7.641864in}{1.099921in}}%
\pgfpathclose%
\pgfusepath{fill}%
\end{pgfscope}%
\begin{pgfscope}%
\pgfpathrectangle{\pgfqpoint{6.818937in}{0.147348in}}{\pgfqpoint{2.735294in}{2.735294in}}%
\pgfusepath{clip}%
\pgfsetbuttcap%
\pgfsetroundjoin%
\definecolor{currentfill}{rgb}{0.063981,0.245604,0.371502}%
\pgfsetfillcolor{currentfill}%
\pgfsetlinewidth{0.000000pt}%
\definecolor{currentstroke}{rgb}{0.000000,0.000000,0.000000}%
\pgfsetstrokecolor{currentstroke}%
\pgfsetdash{}{0pt}%
\pgfpathmoveto{\pgfqpoint{8.752556in}{1.562133in}}%
\pgfpathlineto{\pgfqpoint{8.860669in}{1.018945in}}%
\pgfpathlineto{\pgfqpoint{8.805232in}{1.099921in}}%
\pgfpathlineto{\pgfqpoint{8.752556in}{1.562133in}}%
\pgfpathclose%
\pgfusepath{fill}%
\end{pgfscope}%
\begin{pgfscope}%
\pgfpathrectangle{\pgfqpoint{6.818937in}{0.147348in}}{\pgfqpoint{2.735294in}{2.735294in}}%
\pgfusepath{clip}%
\pgfsetbuttcap%
\pgfsetroundjoin%
\definecolor{currentfill}{rgb}{0.081954,0.314596,0.475860}%
\pgfsetfillcolor{currentfill}%
\pgfsetlinewidth{0.000000pt}%
\definecolor{currentstroke}{rgb}{0.000000,0.000000,0.000000}%
\pgfsetstrokecolor{currentstroke}%
\pgfsetdash{}{0pt}%
\pgfpathmoveto{\pgfqpoint{8.349584in}{2.130589in}}%
\pgfpathlineto{\pgfqpoint{8.097511in}{2.130589in}}%
\pgfpathlineto{\pgfqpoint{8.131822in}{2.613951in}}%
\pgfpathlineto{\pgfqpoint{8.349584in}{2.130589in}}%
\pgfpathclose%
\pgfusepath{fill}%
\end{pgfscope}%
\begin{pgfscope}%
\pgfpathrectangle{\pgfqpoint{6.818937in}{0.147348in}}{\pgfqpoint{2.735294in}{2.735294in}}%
\pgfusepath{clip}%
\pgfsetbuttcap%
\pgfsetroundjoin%
\definecolor{currentfill}{rgb}{0.040669,0.156116,0.236142}%
\pgfsetfillcolor{currentfill}%
\pgfsetlinewidth{0.000000pt}%
\definecolor{currentstroke}{rgb}{0.000000,0.000000,0.000000}%
\pgfsetstrokecolor{currentstroke}%
\pgfsetdash{}{0pt}%
\pgfpathmoveto{\pgfqpoint{8.591651in}{1.053752in}}%
\pgfpathlineto{\pgfqpoint{8.672401in}{0.966453in}}%
\pgfpathlineto{\pgfqpoint{8.456300in}{0.930196in}}%
\pgfpathlineto{\pgfqpoint{8.591651in}{1.053752in}}%
\pgfpathclose%
\pgfusepath{fill}%
\end{pgfscope}%
\begin{pgfscope}%
\pgfpathrectangle{\pgfqpoint{6.818937in}{0.147348in}}{\pgfqpoint{2.735294in}{2.735294in}}%
\pgfusepath{clip}%
\pgfsetbuttcap%
\pgfsetroundjoin%
\definecolor{currentfill}{rgb}{0.040669,0.156116,0.236142}%
\pgfsetfillcolor{currentfill}%
\pgfsetlinewidth{0.000000pt}%
\definecolor{currentstroke}{rgb}{0.000000,0.000000,0.000000}%
\pgfsetstrokecolor{currentstroke}%
\pgfsetdash{}{0pt}%
\pgfpathmoveto{\pgfqpoint{7.990796in}{0.930196in}}%
\pgfpathlineto{\pgfqpoint{7.774694in}{0.966453in}}%
\pgfpathlineto{\pgfqpoint{7.855445in}{1.053752in}}%
\pgfpathlineto{\pgfqpoint{7.990796in}{0.930196in}}%
\pgfpathclose%
\pgfusepath{fill}%
\end{pgfscope}%
\begin{pgfscope}%
\pgfpathrectangle{\pgfqpoint{6.818937in}{0.147348in}}{\pgfqpoint{2.735294in}{2.735294in}}%
\pgfusepath{clip}%
\pgfsetbuttcap%
\pgfsetroundjoin%
\definecolor{currentfill}{rgb}{0.045702,0.175435,0.265364}%
\pgfsetfillcolor{currentfill}%
\pgfsetlinewidth{0.000000pt}%
\definecolor{currentstroke}{rgb}{0.000000,0.000000,0.000000}%
\pgfsetstrokecolor{currentstroke}%
\pgfsetdash{}{0pt}%
\pgfpathmoveto{\pgfqpoint{8.982643in}{1.157190in}}%
\pgfpathlineto{\pgfqpoint{8.860669in}{1.018945in}}%
\pgfpathlineto{\pgfqpoint{8.752556in}{1.562133in}}%
\pgfpathlineto{\pgfqpoint{8.982643in}{1.157190in}}%
\pgfpathclose%
\pgfusepath{fill}%
\end{pgfscope}%
\begin{pgfscope}%
\pgfpathrectangle{\pgfqpoint{6.818937in}{0.147348in}}{\pgfqpoint{2.735294in}{2.735294in}}%
\pgfusepath{clip}%
\pgfsetbuttcap%
\pgfsetroundjoin%
\definecolor{currentfill}{rgb}{0.045702,0.175435,0.265364}%
\pgfsetfillcolor{currentfill}%
\pgfsetlinewidth{0.000000pt}%
\definecolor{currentstroke}{rgb}{0.000000,0.000000,0.000000}%
\pgfsetstrokecolor{currentstroke}%
\pgfsetdash{}{0pt}%
\pgfpathmoveto{\pgfqpoint{7.694540in}{1.562133in}}%
\pgfpathlineto{\pgfqpoint{7.586426in}{1.018945in}}%
\pgfpathlineto{\pgfqpoint{7.464453in}{1.157190in}}%
\pgfpathlineto{\pgfqpoint{7.694540in}{1.562133in}}%
\pgfpathclose%
\pgfusepath{fill}%
\end{pgfscope}%
\begin{pgfscope}%
\pgfpathrectangle{\pgfqpoint{6.818937in}{0.147348in}}{\pgfqpoint{2.735294in}{2.735294in}}%
\pgfusepath{clip}%
\pgfsetbuttcap%
\pgfsetroundjoin%
\definecolor{currentfill}{rgb}{0.082280,0.315849,0.477755}%
\pgfsetfillcolor{currentfill}%
\pgfsetlinewidth{0.000000pt}%
\definecolor{currentstroke}{rgb}{0.000000,0.000000,0.000000}%
\pgfsetstrokecolor{currentstroke}%
\pgfsetdash{}{0pt}%
\pgfpathmoveto{\pgfqpoint{8.349584in}{2.130589in}}%
\pgfpathlineto{\pgfqpoint{8.315273in}{2.613951in}}%
\pgfpathlineto{\pgfqpoint{8.671236in}{2.272632in}}%
\pgfpathlineto{\pgfqpoint{8.349584in}{2.130589in}}%
\pgfpathclose%
\pgfusepath{fill}%
\end{pgfscope}%
\begin{pgfscope}%
\pgfpathrectangle{\pgfqpoint{6.818937in}{0.147348in}}{\pgfqpoint{2.735294in}{2.735294in}}%
\pgfusepath{clip}%
\pgfsetbuttcap%
\pgfsetroundjoin%
\definecolor{currentfill}{rgb}{0.082280,0.315849,0.477755}%
\pgfsetfillcolor{currentfill}%
\pgfsetlinewidth{0.000000pt}%
\definecolor{currentstroke}{rgb}{0.000000,0.000000,0.000000}%
\pgfsetstrokecolor{currentstroke}%
\pgfsetdash{}{0pt}%
\pgfpathmoveto{\pgfqpoint{7.775860in}{2.272632in}}%
\pgfpathlineto{\pgfqpoint{8.131822in}{2.613951in}}%
\pgfpathlineto{\pgfqpoint{8.097511in}{2.130589in}}%
\pgfpathlineto{\pgfqpoint{7.775860in}{2.272632in}}%
\pgfpathclose%
\pgfusepath{fill}%
\end{pgfscope}%
\begin{pgfscope}%
\pgfpathrectangle{\pgfqpoint{6.818937in}{0.147348in}}{\pgfqpoint{2.735294in}{2.735294in}}%
\pgfusepath{clip}%
\pgfsetbuttcap%
\pgfsetroundjoin%
\definecolor{currentfill}{rgb}{0.052493,0.201505,0.304798}%
\pgfsetfillcolor{currentfill}%
\pgfsetlinewidth{0.000000pt}%
\definecolor{currentstroke}{rgb}{0.000000,0.000000,0.000000}%
\pgfsetstrokecolor{currentstroke}%
\pgfsetdash{}{0pt}%
\pgfpathmoveto{\pgfqpoint{9.120088in}{1.620392in}}%
\pgfpathlineto{\pgfqpoint{9.124324in}{1.217806in}}%
\pgfpathlineto{\pgfqpoint{8.958969in}{1.590078in}}%
\pgfpathlineto{\pgfqpoint{9.120088in}{1.620392in}}%
\pgfpathclose%
\pgfusepath{fill}%
\end{pgfscope}%
\begin{pgfscope}%
\pgfpathrectangle{\pgfqpoint{6.818937in}{0.147348in}}{\pgfqpoint{2.735294in}{2.735294in}}%
\pgfusepath{clip}%
\pgfsetbuttcap%
\pgfsetroundjoin%
\definecolor{currentfill}{rgb}{0.052493,0.201505,0.304798}%
\pgfsetfillcolor{currentfill}%
\pgfsetlinewidth{0.000000pt}%
\definecolor{currentstroke}{rgb}{0.000000,0.000000,0.000000}%
\pgfsetstrokecolor{currentstroke}%
\pgfsetdash{}{0pt}%
\pgfpathmoveto{\pgfqpoint{7.488127in}{1.590078in}}%
\pgfpathlineto{\pgfqpoint{7.322771in}{1.217806in}}%
\pgfpathlineto{\pgfqpoint{7.327008in}{1.620392in}}%
\pgfpathlineto{\pgfqpoint{7.488127in}{1.590078in}}%
\pgfpathclose%
\pgfusepath{fill}%
\end{pgfscope}%
\begin{pgfscope}%
\pgfpathrectangle{\pgfqpoint{6.818937in}{0.147348in}}{\pgfqpoint{2.735294in}{2.735294in}}%
\pgfusepath{clip}%
\pgfsetbuttcap%
\pgfsetroundjoin%
\definecolor{currentfill}{rgb}{0.042579,0.163449,0.247234}%
\pgfsetfillcolor{currentfill}%
\pgfsetlinewidth{0.000000pt}%
\definecolor{currentstroke}{rgb}{0.000000,0.000000,0.000000}%
\pgfsetstrokecolor{currentstroke}%
\pgfsetdash{}{0pt}%
\pgfpathmoveto{\pgfqpoint{8.860669in}{1.018945in}}%
\pgfpathlineto{\pgfqpoint{8.672401in}{0.966453in}}%
\pgfpathlineto{\pgfqpoint{8.805232in}{1.099921in}}%
\pgfpathlineto{\pgfqpoint{8.860669in}{1.018945in}}%
\pgfpathclose%
\pgfusepath{fill}%
\end{pgfscope}%
\begin{pgfscope}%
\pgfpathrectangle{\pgfqpoint{6.818937in}{0.147348in}}{\pgfqpoint{2.735294in}{2.735294in}}%
\pgfusepath{clip}%
\pgfsetbuttcap%
\pgfsetroundjoin%
\definecolor{currentfill}{rgb}{0.042579,0.163449,0.247234}%
\pgfsetfillcolor{currentfill}%
\pgfsetlinewidth{0.000000pt}%
\definecolor{currentstroke}{rgb}{0.000000,0.000000,0.000000}%
\pgfsetstrokecolor{currentstroke}%
\pgfsetdash{}{0pt}%
\pgfpathmoveto{\pgfqpoint{7.641864in}{1.099921in}}%
\pgfpathlineto{\pgfqpoint{7.774694in}{0.966453in}}%
\pgfpathlineto{\pgfqpoint{7.586426in}{1.018945in}}%
\pgfpathlineto{\pgfqpoint{7.641864in}{1.099921in}}%
\pgfpathclose%
\pgfusepath{fill}%
\end{pgfscope}%
\begin{pgfscope}%
\pgfpathrectangle{\pgfqpoint{6.818937in}{0.147348in}}{\pgfqpoint{2.735294in}{2.735294in}}%
\pgfusepath{clip}%
\pgfsetbuttcap%
\pgfsetroundjoin%
\definecolor{currentfill}{rgb}{0.067179,0.257880,0.390071}%
\pgfsetfillcolor{currentfill}%
\pgfsetlinewidth{0.000000pt}%
\definecolor{currentstroke}{rgb}{0.000000,0.000000,0.000000}%
\pgfsetstrokecolor{currentstroke}%
\pgfsetdash{}{0pt}%
\pgfpathmoveto{\pgfqpoint{8.958969in}{1.590078in}}%
\pgfpathlineto{\pgfqpoint{9.017173in}{1.079776in}}%
\pgfpathlineto{\pgfqpoint{8.982643in}{1.157190in}}%
\pgfpathlineto{\pgfqpoint{8.958969in}{1.590078in}}%
\pgfpathclose%
\pgfusepath{fill}%
\end{pgfscope}%
\begin{pgfscope}%
\pgfpathrectangle{\pgfqpoint{6.818937in}{0.147348in}}{\pgfqpoint{2.735294in}{2.735294in}}%
\pgfusepath{clip}%
\pgfsetbuttcap%
\pgfsetroundjoin%
\definecolor{currentfill}{rgb}{0.067179,0.257880,0.390071}%
\pgfsetfillcolor{currentfill}%
\pgfsetlinewidth{0.000000pt}%
\definecolor{currentstroke}{rgb}{0.000000,0.000000,0.000000}%
\pgfsetstrokecolor{currentstroke}%
\pgfsetdash{}{0pt}%
\pgfpathmoveto{\pgfqpoint{7.464453in}{1.157190in}}%
\pgfpathlineto{\pgfqpoint{7.429923in}{1.079776in}}%
\pgfpathlineto{\pgfqpoint{7.488127in}{1.590078in}}%
\pgfpathlineto{\pgfqpoint{7.464453in}{1.157190in}}%
\pgfpathclose%
\pgfusepath{fill}%
\end{pgfscope}%
\begin{pgfscope}%
\pgfpathrectangle{\pgfqpoint{6.818937in}{0.147348in}}{\pgfqpoint{2.735294in}{2.735294in}}%
\pgfusepath{clip}%
\pgfsetbuttcap%
\pgfsetroundjoin%
\definecolor{currentfill}{rgb}{0.047247,0.181368,0.274339}%
\pgfsetfillcolor{currentfill}%
\pgfsetlinewidth{0.000000pt}%
\definecolor{currentstroke}{rgb}{0.000000,0.000000,0.000000}%
\pgfsetstrokecolor{currentstroke}%
\pgfsetdash{}{0pt}%
\pgfpathmoveto{\pgfqpoint{9.124324in}{1.217806in}}%
\pgfpathlineto{\pgfqpoint{9.017173in}{1.079776in}}%
\pgfpathlineto{\pgfqpoint{8.958969in}{1.590078in}}%
\pgfpathlineto{\pgfqpoint{9.124324in}{1.217806in}}%
\pgfpathclose%
\pgfusepath{fill}%
\end{pgfscope}%
\begin{pgfscope}%
\pgfpathrectangle{\pgfqpoint{6.818937in}{0.147348in}}{\pgfqpoint{2.735294in}{2.735294in}}%
\pgfusepath{clip}%
\pgfsetbuttcap%
\pgfsetroundjoin%
\definecolor{currentfill}{rgb}{0.047247,0.181368,0.274339}%
\pgfsetfillcolor{currentfill}%
\pgfsetlinewidth{0.000000pt}%
\definecolor{currentstroke}{rgb}{0.000000,0.000000,0.000000}%
\pgfsetstrokecolor{currentstroke}%
\pgfsetdash{}{0pt}%
\pgfpathmoveto{\pgfqpoint{7.488127in}{1.590078in}}%
\pgfpathlineto{\pgfqpoint{7.429923in}{1.079776in}}%
\pgfpathlineto{\pgfqpoint{7.322771in}{1.217806in}}%
\pgfpathlineto{\pgfqpoint{7.488127in}{1.590078in}}%
\pgfpathclose%
\pgfusepath{fill}%
\end{pgfscope}%
\begin{pgfscope}%
\pgfpathrectangle{\pgfqpoint{6.818937in}{0.147348in}}{\pgfqpoint{2.735294in}{2.735294in}}%
\pgfusepath{clip}%
\pgfsetbuttcap%
\pgfsetroundjoin%
\definecolor{currentfill}{rgb}{0.081954,0.314596,0.475860}%
\pgfsetfillcolor{currentfill}%
\pgfsetlinewidth{0.000000pt}%
\definecolor{currentstroke}{rgb}{0.000000,0.000000,0.000000}%
\pgfsetstrokecolor{currentstroke}%
\pgfsetdash{}{0pt}%
\pgfpathmoveto{\pgfqpoint{8.131822in}{2.613951in}}%
\pgfpathlineto{\pgfqpoint{8.315273in}{2.613951in}}%
\pgfpathlineto{\pgfqpoint{8.349584in}{2.130589in}}%
\pgfpathlineto{\pgfqpoint{8.131822in}{2.613951in}}%
\pgfpathclose%
\pgfusepath{fill}%
\end{pgfscope}%
\begin{pgfscope}%
\pgfpathrectangle{\pgfqpoint{6.818937in}{0.147348in}}{\pgfqpoint{2.735294in}{2.735294in}}%
\pgfusepath{clip}%
\pgfsetbuttcap%
\pgfsetroundjoin%
\definecolor{currentfill}{rgb}{0.044978,0.172658,0.261163}%
\pgfsetfillcolor{currentfill}%
\pgfsetlinewidth{0.000000pt}%
\definecolor{currentstroke}{rgb}{0.000000,0.000000,0.000000}%
\pgfsetstrokecolor{currentstroke}%
\pgfsetdash{}{0pt}%
\pgfpathmoveto{\pgfqpoint{8.860669in}{1.018945in}}%
\pgfpathlineto{\pgfqpoint{8.982643in}{1.157190in}}%
\pgfpathlineto{\pgfqpoint{9.017173in}{1.079776in}}%
\pgfpathlineto{\pgfqpoint{8.860669in}{1.018945in}}%
\pgfpathclose%
\pgfusepath{fill}%
\end{pgfscope}%
\begin{pgfscope}%
\pgfpathrectangle{\pgfqpoint{6.818937in}{0.147348in}}{\pgfqpoint{2.735294in}{2.735294in}}%
\pgfusepath{clip}%
\pgfsetbuttcap%
\pgfsetroundjoin%
\definecolor{currentfill}{rgb}{0.044978,0.172658,0.261163}%
\pgfsetfillcolor{currentfill}%
\pgfsetlinewidth{0.000000pt}%
\definecolor{currentstroke}{rgb}{0.000000,0.000000,0.000000}%
\pgfsetstrokecolor{currentstroke}%
\pgfsetdash{}{0pt}%
\pgfpathmoveto{\pgfqpoint{7.429923in}{1.079776in}}%
\pgfpathlineto{\pgfqpoint{7.464453in}{1.157190in}}%
\pgfpathlineto{\pgfqpoint{7.586426in}{1.018945in}}%
\pgfpathlineto{\pgfqpoint{7.429923in}{1.079776in}}%
\pgfpathclose%
\pgfusepath{fill}%
\end{pgfscope}%
\begin{pgfscope}%
\pgfpathrectangle{\pgfqpoint{6.818937in}{0.147348in}}{\pgfqpoint{2.735294in}{2.735294in}}%
\pgfusepath{clip}%
\pgfsetbuttcap%
\pgfsetroundjoin%
\definecolor{currentfill}{rgb}{0.070254,0.269685,0.407928}%
\pgfsetfillcolor{currentfill}%
\pgfsetlinewidth{0.000000pt}%
\definecolor{currentstroke}{rgb}{0.000000,0.000000,0.000000}%
\pgfsetstrokecolor{currentstroke}%
\pgfsetdash{}{0pt}%
\pgfpathmoveto{\pgfqpoint{9.120088in}{1.620392in}}%
\pgfpathlineto{\pgfqpoint{9.143495in}{1.142562in}}%
\pgfpathlineto{\pgfqpoint{9.124324in}{1.217806in}}%
\pgfpathlineto{\pgfqpoint{9.120088in}{1.620392in}}%
\pgfpathclose%
\pgfusepath{fill}%
\end{pgfscope}%
\begin{pgfscope}%
\pgfpathrectangle{\pgfqpoint{6.818937in}{0.147348in}}{\pgfqpoint{2.735294in}{2.735294in}}%
\pgfusepath{clip}%
\pgfsetbuttcap%
\pgfsetroundjoin%
\definecolor{currentfill}{rgb}{0.070254,0.269685,0.407928}%
\pgfsetfillcolor{currentfill}%
\pgfsetlinewidth{0.000000pt}%
\definecolor{currentstroke}{rgb}{0.000000,0.000000,0.000000}%
\pgfsetstrokecolor{currentstroke}%
\pgfsetdash{}{0pt}%
\pgfpathmoveto{\pgfqpoint{7.322771in}{1.217806in}}%
\pgfpathlineto{\pgfqpoint{7.303601in}{1.142562in}}%
\pgfpathlineto{\pgfqpoint{7.327008in}{1.620392in}}%
\pgfpathlineto{\pgfqpoint{7.322771in}{1.217806in}}%
\pgfpathclose%
\pgfusepath{fill}%
\end{pgfscope}%
\begin{pgfscope}%
\pgfpathrectangle{\pgfqpoint{6.818937in}{0.147348in}}{\pgfqpoint{2.735294in}{2.735294in}}%
\pgfusepath{clip}%
\pgfsetbuttcap%
\pgfsetroundjoin%
\definecolor{currentfill}{rgb}{0.048960,0.187944,0.284285}%
\pgfsetfillcolor{currentfill}%
\pgfsetlinewidth{0.000000pt}%
\definecolor{currentstroke}{rgb}{0.000000,0.000000,0.000000}%
\pgfsetstrokecolor{currentstroke}%
\pgfsetdash{}{0pt}%
\pgfpathmoveto{\pgfqpoint{9.235220in}{1.276682in}}%
\pgfpathlineto{\pgfqpoint{9.143495in}{1.142562in}}%
\pgfpathlineto{\pgfqpoint{9.120088in}{1.620392in}}%
\pgfpathlineto{\pgfqpoint{9.235220in}{1.276682in}}%
\pgfpathclose%
\pgfusepath{fill}%
\end{pgfscope}%
\begin{pgfscope}%
\pgfpathrectangle{\pgfqpoint{6.818937in}{0.147348in}}{\pgfqpoint{2.735294in}{2.735294in}}%
\pgfusepath{clip}%
\pgfsetbuttcap%
\pgfsetroundjoin%
\definecolor{currentfill}{rgb}{0.048960,0.187944,0.284285}%
\pgfsetfillcolor{currentfill}%
\pgfsetlinewidth{0.000000pt}%
\definecolor{currentstroke}{rgb}{0.000000,0.000000,0.000000}%
\pgfsetstrokecolor{currentstroke}%
\pgfsetdash{}{0pt}%
\pgfpathmoveto{\pgfqpoint{7.211876in}{1.276682in}}%
\pgfpathlineto{\pgfqpoint{7.327008in}{1.620392in}}%
\pgfpathlineto{\pgfqpoint{7.303601in}{1.142562in}}%
\pgfpathlineto{\pgfqpoint{7.211876in}{1.276682in}}%
\pgfpathclose%
\pgfusepath{fill}%
\end{pgfscope}%
\begin{pgfscope}%
\pgfpathrectangle{\pgfqpoint{6.818937in}{0.147348in}}{\pgfqpoint{2.735294in}{2.735294in}}%
\pgfusepath{clip}%
\pgfsetbuttcap%
\pgfsetroundjoin%
\definecolor{currentfill}{rgb}{0.047548,0.182523,0.276086}%
\pgfsetfillcolor{currentfill}%
\pgfsetlinewidth{0.000000pt}%
\definecolor{currentstroke}{rgb}{0.000000,0.000000,0.000000}%
\pgfsetstrokecolor{currentstroke}%
\pgfsetdash{}{0pt}%
\pgfpathmoveto{\pgfqpoint{9.017173in}{1.079776in}}%
\pgfpathlineto{\pgfqpoint{9.124324in}{1.217806in}}%
\pgfpathlineto{\pgfqpoint{9.143495in}{1.142562in}}%
\pgfpathlineto{\pgfqpoint{9.017173in}{1.079776in}}%
\pgfpathclose%
\pgfusepath{fill}%
\end{pgfscope}%
\begin{pgfscope}%
\pgfpathrectangle{\pgfqpoint{6.818937in}{0.147348in}}{\pgfqpoint{2.735294in}{2.735294in}}%
\pgfusepath{clip}%
\pgfsetbuttcap%
\pgfsetroundjoin%
\definecolor{currentfill}{rgb}{0.047548,0.182523,0.276086}%
\pgfsetfillcolor{currentfill}%
\pgfsetlinewidth{0.000000pt}%
\definecolor{currentstroke}{rgb}{0.000000,0.000000,0.000000}%
\pgfsetstrokecolor{currentstroke}%
\pgfsetdash{}{0pt}%
\pgfpathmoveto{\pgfqpoint{7.303601in}{1.142562in}}%
\pgfpathlineto{\pgfqpoint{7.322771in}{1.217806in}}%
\pgfpathlineto{\pgfqpoint{7.429923in}{1.079776in}}%
\pgfpathlineto{\pgfqpoint{7.303601in}{1.142562in}}%
\pgfpathclose%
\pgfusepath{fill}%
\end{pgfscope}%
\begin{pgfscope}%
\pgfpathrectangle{\pgfqpoint{6.818937in}{0.147348in}}{\pgfqpoint{2.735294in}{2.735294in}}%
\pgfusepath{clip}%
\pgfsetbuttcap%
\pgfsetroundjoin%
\definecolor{currentfill}{rgb}{0.090605,0.347808,0.526096}%
\pgfsetfillcolor{currentfill}%
\pgfsetlinewidth{0.000000pt}%
\definecolor{currentstroke}{rgb}{0.000000,0.000000,0.000000}%
\pgfsetstrokecolor{currentstroke}%
\pgfsetdash{}{0pt}%
\pgfpathmoveto{\pgfqpoint{8.315273in}{2.613951in}}%
\pgfpathlineto{\pgfqpoint{8.131822in}{2.613951in}}%
\pgfpathlineto{\pgfqpoint{8.223548in}{2.686670in}}%
\pgfpathlineto{\pgfqpoint{8.315273in}{2.613951in}}%
\pgfpathclose%
\pgfusepath{fill}%
\end{pgfscope}%
\begin{pgfscope}%
\pgfpathrectangle{\pgfqpoint{6.818937in}{0.147348in}}{\pgfqpoint{2.735294in}{2.735294in}}%
\pgfusepath{clip}%
\pgfsetbuttcap%
\pgfsetroundjoin%
\definecolor{currentfill}{rgb}{0.050070,0.192203,0.290728}%
\pgfsetfillcolor{currentfill}%
\pgfsetlinewidth{0.000000pt}%
\definecolor{currentstroke}{rgb}{0.000000,0.000000,0.000000}%
\pgfsetstrokecolor{currentstroke}%
\pgfsetdash{}{0pt}%
\pgfpathmoveto{\pgfqpoint{9.143495in}{1.142562in}}%
\pgfpathlineto{\pgfqpoint{9.235220in}{1.276682in}}%
\pgfpathlineto{\pgfqpoint{9.243948in}{1.203196in}}%
\pgfpathlineto{\pgfqpoint{9.143495in}{1.142562in}}%
\pgfpathclose%
\pgfusepath{fill}%
\end{pgfscope}%
\begin{pgfscope}%
\pgfpathrectangle{\pgfqpoint{6.818937in}{0.147348in}}{\pgfqpoint{2.735294in}{2.735294in}}%
\pgfusepath{clip}%
\pgfsetbuttcap%
\pgfsetroundjoin%
\definecolor{currentfill}{rgb}{0.050070,0.192203,0.290728}%
\pgfsetfillcolor{currentfill}%
\pgfsetlinewidth{0.000000pt}%
\definecolor{currentstroke}{rgb}{0.000000,0.000000,0.000000}%
\pgfsetstrokecolor{currentstroke}%
\pgfsetdash{}{0pt}%
\pgfpathmoveto{\pgfqpoint{7.211876in}{1.276682in}}%
\pgfpathlineto{\pgfqpoint{7.303601in}{1.142562in}}%
\pgfpathlineto{\pgfqpoint{7.203147in}{1.203196in}}%
\pgfpathlineto{\pgfqpoint{7.211876in}{1.276682in}}%
\pgfpathclose%
\pgfusepath{fill}%
\end{pgfscope}%
\begin{pgfscope}%
\pgfsetbuttcap%
\pgfsetmiterjoin%
\definecolor{currentfill}{rgb}{1.000000,1.000000,1.000000}%
\pgfsetfillcolor{currentfill}%
\pgfsetlinewidth{0.000000pt}%
\definecolor{currentstroke}{rgb}{0.000000,0.000000,0.000000}%
\pgfsetstrokecolor{currentstroke}%
\pgfsetstrokeopacity{0.000000}%
\pgfsetdash{}{0pt}%
\pgfpathmoveto{\pgfqpoint{0.254231in}{0.147348in}}%
\pgfpathlineto{\pgfqpoint{2.989526in}{0.147348in}}%
\pgfpathlineto{\pgfqpoint{2.989526in}{2.882642in}}%
\pgfpathlineto{\pgfqpoint{0.254231in}{2.882642in}}%
\pgfpathlineto{\pgfqpoint{0.254231in}{0.147348in}}%
\pgfpathclose%
\pgfusepath{fill}%
\end{pgfscope}%
\begin{pgfscope}%
\pgfsetbuttcap%
\pgfsetmiterjoin%
\definecolor{currentfill}{rgb}{0.950000,0.950000,0.950000}%
\pgfsetfillcolor{currentfill}%
\pgfsetfillopacity{0.500000}%
\pgfsetlinewidth{1.003750pt}%
\definecolor{currentstroke}{rgb}{0.950000,0.950000,0.950000}%
\pgfsetstrokecolor{currentstroke}%
\pgfsetstrokeopacity{0.500000}%
\pgfsetdash{}{0pt}%
\pgfpathmoveto{\pgfqpoint{1.658842in}{1.930798in}}%
\pgfpathlineto{\pgfqpoint{2.865378in}{1.099507in}}%
\pgfpathlineto{\pgfqpoint{2.944036in}{2.033906in}}%
\pgfpathlineto{\pgfqpoint{1.658842in}{2.862146in}}%
\pgfusepath{stroke,fill}%
\end{pgfscope}%
\begin{pgfscope}%
\pgfsetbuttcap%
\pgfsetmiterjoin%
\definecolor{currentfill}{rgb}{0.900000,0.900000,0.900000}%
\pgfsetfillcolor{currentfill}%
\pgfsetfillopacity{0.500000}%
\pgfsetlinewidth{1.003750pt}%
\definecolor{currentstroke}{rgb}{0.900000,0.900000,0.900000}%
\pgfsetstrokecolor{currentstroke}%
\pgfsetstrokeopacity{0.500000}%
\pgfsetdash{}{0pt}%
\pgfpathmoveto{\pgfqpoint{1.658842in}{1.930798in}}%
\pgfpathlineto{\pgfqpoint{0.452305in}{1.099507in}}%
\pgfpathlineto{\pgfqpoint{0.373648in}{2.033906in}}%
\pgfpathlineto{\pgfqpoint{1.658842in}{2.862146in}}%
\pgfusepath{stroke,fill}%
\end{pgfscope}%
\begin{pgfscope}%
\pgfsetbuttcap%
\pgfsetmiterjoin%
\definecolor{currentfill}{rgb}{0.925000,0.925000,0.925000}%
\pgfsetfillcolor{currentfill}%
\pgfsetfillopacity{0.500000}%
\pgfsetlinewidth{1.003750pt}%
\definecolor{currentstroke}{rgb}{0.925000,0.925000,0.925000}%
\pgfsetstrokecolor{currentstroke}%
\pgfsetstrokeopacity{0.500000}%
\pgfsetdash{}{0pt}%
\pgfpathmoveto{\pgfqpoint{1.658842in}{1.930798in}}%
\pgfpathlineto{\pgfqpoint{0.452305in}{1.099507in}}%
\pgfpathlineto{\pgfqpoint{1.658842in}{0.166408in}}%
\pgfpathlineto{\pgfqpoint{2.865378in}{1.099507in}}%
\pgfusepath{stroke,fill}%
\end{pgfscope}%
\begin{pgfscope}%
\pgfsetbuttcap%
\pgfsetroundjoin%
\pgfsetlinewidth{0.803000pt}%
\definecolor{currentstroke}{rgb}{0.690196,0.690196,0.690196}%
\pgfsetstrokecolor{currentstroke}%
\pgfsetdash{}{0pt}%
\pgfpathmoveto{\pgfqpoint{2.792763in}{1.043348in}}%
\pgfpathlineto{\pgfqpoint{1.585998in}{1.880609in}}%
\pgfpathlineto{\pgfqpoint{1.581508in}{2.812308in}}%
\pgfusepath{stroke}%
\end{pgfscope}%
\begin{pgfscope}%
\pgfsetbuttcap%
\pgfsetroundjoin%
\pgfsetlinewidth{0.803000pt}%
\definecolor{currentstroke}{rgb}{0.690196,0.690196,0.690196}%
\pgfsetstrokecolor{currentstroke}%
\pgfsetdash{}{0pt}%
\pgfpathmoveto{\pgfqpoint{2.562953in}{0.865621in}}%
\pgfpathlineto{\pgfqpoint{1.355660in}{1.721909in}}%
\pgfpathlineto{\pgfqpoint{1.336753in}{2.654576in}}%
\pgfusepath{stroke}%
\end{pgfscope}%
\begin{pgfscope}%
\pgfsetbuttcap%
\pgfsetroundjoin%
\pgfsetlinewidth{0.803000pt}%
\definecolor{currentstroke}{rgb}{0.690196,0.690196,0.690196}%
\pgfsetstrokecolor{currentstroke}%
\pgfsetdash{}{0pt}%
\pgfpathmoveto{\pgfqpoint{2.327740in}{0.683714in}}%
\pgfpathlineto{\pgfqpoint{1.120209in}{1.559686in}}%
\pgfpathlineto{\pgfqpoint{1.086222in}{2.493122in}}%
\pgfusepath{stroke}%
\end{pgfscope}%
\begin{pgfscope}%
\pgfsetbuttcap%
\pgfsetroundjoin%
\pgfsetlinewidth{0.803000pt}%
\definecolor{currentstroke}{rgb}{0.690196,0.690196,0.690196}%
\pgfsetstrokecolor{currentstroke}%
\pgfsetdash{}{0pt}%
\pgfpathmoveto{\pgfqpoint{2.086930in}{0.497478in}}%
\pgfpathlineto{\pgfqpoint{0.879473in}{1.393821in}}%
\pgfpathlineto{\pgfqpoint{0.829708in}{2.327813in}}%
\pgfusepath{stroke}%
\end{pgfscope}%
\begin{pgfscope}%
\pgfsetbuttcap%
\pgfsetroundjoin%
\pgfsetlinewidth{0.803000pt}%
\definecolor{currentstroke}{rgb}{0.690196,0.690196,0.690196}%
\pgfsetstrokecolor{currentstroke}%
\pgfsetdash{}{0pt}%
\pgfpathmoveto{\pgfqpoint{1.840321in}{0.306758in}}%
\pgfpathlineto{\pgfqpoint{0.633271in}{1.224191in}}%
\pgfpathlineto{\pgfqpoint{0.566994in}{2.158507in}}%
\pgfusepath{stroke}%
\end{pgfscope}%
\begin{pgfscope}%
\pgfsetbuttcap%
\pgfsetroundjoin%
\pgfsetlinewidth{0.803000pt}%
\definecolor{currentstroke}{rgb}{0.690196,0.690196,0.690196}%
\pgfsetstrokecolor{currentstroke}%
\pgfsetdash{}{0pt}%
\pgfpathmoveto{\pgfqpoint{1.736176in}{2.812308in}}%
\pgfpathlineto{\pgfqpoint{1.731686in}{1.880609in}}%
\pgfpathlineto{\pgfqpoint{0.524921in}{1.043348in}}%
\pgfusepath{stroke}%
\end{pgfscope}%
\begin{pgfscope}%
\pgfsetbuttcap%
\pgfsetroundjoin%
\pgfsetlinewidth{0.803000pt}%
\definecolor{currentstroke}{rgb}{0.690196,0.690196,0.690196}%
\pgfsetstrokecolor{currentstroke}%
\pgfsetdash{}{0pt}%
\pgfpathmoveto{\pgfqpoint{1.980931in}{2.654576in}}%
\pgfpathlineto{\pgfqpoint{1.962024in}{1.721909in}}%
\pgfpathlineto{\pgfqpoint{0.754731in}{0.865621in}}%
\pgfusepath{stroke}%
\end{pgfscope}%
\begin{pgfscope}%
\pgfsetbuttcap%
\pgfsetroundjoin%
\pgfsetlinewidth{0.803000pt}%
\definecolor{currentstroke}{rgb}{0.690196,0.690196,0.690196}%
\pgfsetstrokecolor{currentstroke}%
\pgfsetdash{}{0pt}%
\pgfpathmoveto{\pgfqpoint{2.231462in}{2.493122in}}%
\pgfpathlineto{\pgfqpoint{2.197475in}{1.559686in}}%
\pgfpathlineto{\pgfqpoint{0.989944in}{0.683714in}}%
\pgfusepath{stroke}%
\end{pgfscope}%
\begin{pgfscope}%
\pgfsetbuttcap%
\pgfsetroundjoin%
\pgfsetlinewidth{0.803000pt}%
\definecolor{currentstroke}{rgb}{0.690196,0.690196,0.690196}%
\pgfsetstrokecolor{currentstroke}%
\pgfsetdash{}{0pt}%
\pgfpathmoveto{\pgfqpoint{2.487976in}{2.327813in}}%
\pgfpathlineto{\pgfqpoint{2.438211in}{1.393821in}}%
\pgfpathlineto{\pgfqpoint{1.230754in}{0.497478in}}%
\pgfusepath{stroke}%
\end{pgfscope}%
\begin{pgfscope}%
\pgfsetbuttcap%
\pgfsetroundjoin%
\pgfsetlinewidth{0.803000pt}%
\definecolor{currentstroke}{rgb}{0.690196,0.690196,0.690196}%
\pgfsetstrokecolor{currentstroke}%
\pgfsetdash{}{0pt}%
\pgfpathmoveto{\pgfqpoint{2.750690in}{2.158507in}}%
\pgfpathlineto{\pgfqpoint{2.684413in}{1.224191in}}%
\pgfpathlineto{\pgfqpoint{1.477363in}{0.306758in}}%
\pgfusepath{stroke}%
\end{pgfscope}%
\begin{pgfscope}%
\pgfsetbuttcap%
\pgfsetroundjoin%
\pgfsetlinewidth{0.803000pt}%
\definecolor{currentstroke}{rgb}{0.690196,0.690196,0.690196}%
\pgfsetstrokecolor{currentstroke}%
\pgfsetdash{}{0pt}%
\pgfpathmoveto{\pgfqpoint{0.447588in}{1.155548in}}%
\pgfpathlineto{\pgfqpoint{1.658842in}{1.986842in}}%
\pgfpathlineto{\pgfqpoint{2.870096in}{1.155548in}}%
\pgfusepath{stroke}%
\end{pgfscope}%
\begin{pgfscope}%
\pgfsetbuttcap%
\pgfsetroundjoin%
\pgfsetlinewidth{0.803000pt}%
\definecolor{currentstroke}{rgb}{0.690196,0.690196,0.690196}%
\pgfsetstrokecolor{currentstroke}%
\pgfsetdash{}{0pt}%
\pgfpathmoveto{\pgfqpoint{0.432645in}{1.333067in}}%
\pgfpathlineto{\pgfqpoint{1.658842in}{2.164215in}}%
\pgfpathlineto{\pgfqpoint{2.885039in}{1.333067in}}%
\pgfusepath{stroke}%
\end{pgfscope}%
\begin{pgfscope}%
\pgfsetbuttcap%
\pgfsetroundjoin%
\pgfsetlinewidth{0.803000pt}%
\definecolor{currentstroke}{rgb}{0.690196,0.690196,0.690196}%
\pgfsetstrokecolor{currentstroke}%
\pgfsetdash{}{0pt}%
\pgfpathmoveto{\pgfqpoint{0.417328in}{1.515021in}}%
\pgfpathlineto{\pgfqpoint{1.658842in}{2.345771in}}%
\pgfpathlineto{\pgfqpoint{2.900356in}{1.515021in}}%
\pgfusepath{stroke}%
\end{pgfscope}%
\begin{pgfscope}%
\pgfsetbuttcap%
\pgfsetroundjoin%
\pgfsetlinewidth{0.803000pt}%
\definecolor{currentstroke}{rgb}{0.690196,0.690196,0.690196}%
\pgfsetstrokecolor{currentstroke}%
\pgfsetdash{}{0pt}%
\pgfpathmoveto{\pgfqpoint{0.401624in}{1.701578in}}%
\pgfpathlineto{\pgfqpoint{1.658842in}{2.531660in}}%
\pgfpathlineto{\pgfqpoint{2.916060in}{1.701578in}}%
\pgfusepath{stroke}%
\end{pgfscope}%
\begin{pgfscope}%
\pgfsetbuttcap%
\pgfsetroundjoin%
\pgfsetlinewidth{0.803000pt}%
\definecolor{currentstroke}{rgb}{0.690196,0.690196,0.690196}%
\pgfsetstrokecolor{currentstroke}%
\pgfsetdash{}{0pt}%
\pgfpathmoveto{\pgfqpoint{0.385517in}{1.892915in}}%
\pgfpathlineto{\pgfqpoint{1.658842in}{2.722038in}}%
\pgfpathlineto{\pgfqpoint{2.932167in}{1.892915in}}%
\pgfusepath{stroke}%
\end{pgfscope}%
\begin{pgfscope}%
\pgfsetrectcap%
\pgfsetroundjoin%
\pgfsetlinewidth{0.803000pt}%
\definecolor{currentstroke}{rgb}{0.000000,0.000000,0.000000}%
\pgfsetstrokecolor{currentstroke}%
\pgfsetdash{}{0pt}%
\pgfpathmoveto{\pgfqpoint{2.865378in}{1.099507in}}%
\pgfpathlineto{\pgfqpoint{1.658842in}{0.166408in}}%
\pgfusepath{stroke}%
\end{pgfscope}%
\begin{pgfscope}%
\pgfsetrectcap%
\pgfsetroundjoin%
\pgfsetlinewidth{0.803000pt}%
\definecolor{currentstroke}{rgb}{0.000000,0.000000,0.000000}%
\pgfsetstrokecolor{currentstroke}%
\pgfsetdash{}{0pt}%
\pgfpathmoveto{\pgfqpoint{2.782554in}{1.050431in}}%
\pgfpathlineto{\pgfqpoint{2.813208in}{1.029163in}}%
\pgfusepath{stroke}%
\end{pgfscope}%
\begin{pgfscope}%
\pgfsetrectcap%
\pgfsetroundjoin%
\pgfsetlinewidth{0.803000pt}%
\definecolor{currentstroke}{rgb}{0.000000,0.000000,0.000000}%
\pgfsetstrokecolor{currentstroke}%
\pgfsetdash{}{0pt}%
\pgfpathmoveto{\pgfqpoint{2.552734in}{0.872869in}}%
\pgfpathlineto{\pgfqpoint{2.583421in}{0.851104in}}%
\pgfusepath{stroke}%
\end{pgfscope}%
\begin{pgfscope}%
\pgfsetrectcap%
\pgfsetroundjoin%
\pgfsetlinewidth{0.803000pt}%
\definecolor{currentstroke}{rgb}{0.000000,0.000000,0.000000}%
\pgfsetstrokecolor{currentstroke}%
\pgfsetdash{}{0pt}%
\pgfpathmoveto{\pgfqpoint{2.317512in}{0.691133in}}%
\pgfpathlineto{\pgfqpoint{2.348225in}{0.668853in}}%
\pgfusepath{stroke}%
\end{pgfscope}%
\begin{pgfscope}%
\pgfsetrectcap%
\pgfsetroundjoin%
\pgfsetlinewidth{0.803000pt}%
\definecolor{currentstroke}{rgb}{0.000000,0.000000,0.000000}%
\pgfsetstrokecolor{currentstroke}%
\pgfsetdash{}{0pt}%
\pgfpathmoveto{\pgfqpoint{2.076696in}{0.505076in}}%
\pgfpathlineto{\pgfqpoint{2.107428in}{0.482262in}}%
\pgfusepath{stroke}%
\end{pgfscope}%
\begin{pgfscope}%
\pgfsetrectcap%
\pgfsetroundjoin%
\pgfsetlinewidth{0.803000pt}%
\definecolor{currentstroke}{rgb}{0.000000,0.000000,0.000000}%
\pgfsetstrokecolor{currentstroke}%
\pgfsetdash{}{0pt}%
\pgfpathmoveto{\pgfqpoint{1.830084in}{0.314540in}}%
\pgfpathlineto{\pgfqpoint{1.860826in}{0.291173in}}%
\pgfusepath{stroke}%
\end{pgfscope}%
\begin{pgfscope}%
\definecolor{textcolor}{rgb}{0.000000,0.000000,0.000000}%
\pgfsetstrokecolor{textcolor}%
\pgfsetfillcolor{textcolor}%
\pgftext[x=2.557884in,y=0.241958in,,]{\color{textcolor}{\rmfamily\fontsize{14.000000}{16.800000}\selectfont\catcode`\^=\active\def^{\ifmmode\sp\else\^{}\fi}\catcode`\%=\active\def%{\%}f1}}%
\end{pgfscope}%
\begin{pgfscope}%
\pgfsetrectcap%
\pgfsetroundjoin%
\pgfsetlinewidth{0.803000pt}%
\definecolor{currentstroke}{rgb}{0.000000,0.000000,0.000000}%
\pgfsetstrokecolor{currentstroke}%
\pgfsetdash{}{0pt}%
\pgfpathmoveto{\pgfqpoint{0.452305in}{1.099507in}}%
\pgfpathlineto{\pgfqpoint{1.658842in}{0.166408in}}%
\pgfusepath{stroke}%
\end{pgfscope}%
\begin{pgfscope}%
\pgfsetrectcap%
\pgfsetroundjoin%
\pgfsetlinewidth{0.803000pt}%
\definecolor{currentstroke}{rgb}{0.000000,0.000000,0.000000}%
\pgfsetstrokecolor{currentstroke}%
\pgfsetdash{}{0pt}%
\pgfpathmoveto{\pgfqpoint{0.535130in}{1.050431in}}%
\pgfpathlineto{\pgfqpoint{0.504476in}{1.029163in}}%
\pgfusepath{stroke}%
\end{pgfscope}%
\begin{pgfscope}%
\pgfsetrectcap%
\pgfsetroundjoin%
\pgfsetlinewidth{0.803000pt}%
\definecolor{currentstroke}{rgb}{0.000000,0.000000,0.000000}%
\pgfsetstrokecolor{currentstroke}%
\pgfsetdash{}{0pt}%
\pgfpathmoveto{\pgfqpoint{0.764950in}{0.872869in}}%
\pgfpathlineto{\pgfqpoint{0.734263in}{0.851104in}}%
\pgfusepath{stroke}%
\end{pgfscope}%
\begin{pgfscope}%
\pgfsetrectcap%
\pgfsetroundjoin%
\pgfsetlinewidth{0.803000pt}%
\definecolor{currentstroke}{rgb}{0.000000,0.000000,0.000000}%
\pgfsetstrokecolor{currentstroke}%
\pgfsetdash{}{0pt}%
\pgfpathmoveto{\pgfqpoint{1.000172in}{0.691133in}}%
\pgfpathlineto{\pgfqpoint{0.969459in}{0.668853in}}%
\pgfusepath{stroke}%
\end{pgfscope}%
\begin{pgfscope}%
\pgfsetrectcap%
\pgfsetroundjoin%
\pgfsetlinewidth{0.803000pt}%
\definecolor{currentstroke}{rgb}{0.000000,0.000000,0.000000}%
\pgfsetstrokecolor{currentstroke}%
\pgfsetdash{}{0pt}%
\pgfpathmoveto{\pgfqpoint{1.240988in}{0.505076in}}%
\pgfpathlineto{\pgfqpoint{1.210256in}{0.482262in}}%
\pgfusepath{stroke}%
\end{pgfscope}%
\begin{pgfscope}%
\pgfsetrectcap%
\pgfsetroundjoin%
\pgfsetlinewidth{0.803000pt}%
\definecolor{currentstroke}{rgb}{0.000000,0.000000,0.000000}%
\pgfsetstrokecolor{currentstroke}%
\pgfsetdash{}{0pt}%
\pgfpathmoveto{\pgfqpoint{1.487600in}{0.314540in}}%
\pgfpathlineto{\pgfqpoint{1.456858in}{0.291173in}}%
\pgfusepath{stroke}%
\end{pgfscope}%
\begin{pgfscope}%
\definecolor{textcolor}{rgb}{0.000000,0.000000,0.000000}%
\pgfsetstrokecolor{textcolor}%
\pgfsetfillcolor{textcolor}%
\pgftext[x=0.759800in,y=0.241958in,,]{\color{textcolor}{\rmfamily\fontsize{14.000000}{16.800000}\selectfont\catcode`\^=\active\def^{\ifmmode\sp\else\^{}\fi}\catcode`\%=\active\def%{\%}f2}}%
\end{pgfscope}%
\begin{pgfscope}%
\pgfsetrectcap%
\pgfsetroundjoin%
\pgfsetlinewidth{0.803000pt}%
\definecolor{currentstroke}{rgb}{0.000000,0.000000,0.000000}%
\pgfsetstrokecolor{currentstroke}%
\pgfsetdash{}{0pt}%
\pgfpathmoveto{\pgfqpoint{0.452305in}{1.099507in}}%
\pgfpathlineto{\pgfqpoint{0.373648in}{2.033906in}}%
\pgfusepath{stroke}%
\end{pgfscope}%
\begin{pgfscope}%
\pgfsetrectcap%
\pgfsetroundjoin%
\pgfsetlinewidth{0.803000pt}%
\definecolor{currentstroke}{rgb}{0.000000,0.000000,0.000000}%
\pgfsetstrokecolor{currentstroke}%
\pgfsetdash{}{0pt}%
\pgfpathmoveto{\pgfqpoint{0.457835in}{1.162580in}}%
\pgfpathlineto{\pgfqpoint{0.427066in}{1.141464in}}%
\pgfusepath{stroke}%
\end{pgfscope}%
\begin{pgfscope}%
\pgfsetrectcap%
\pgfsetroundjoin%
\pgfsetlinewidth{0.803000pt}%
\definecolor{currentstroke}{rgb}{0.000000,0.000000,0.000000}%
\pgfsetstrokecolor{currentstroke}%
\pgfsetdash{}{0pt}%
\pgfpathmoveto{\pgfqpoint{0.443025in}{1.340103in}}%
\pgfpathlineto{\pgfqpoint{0.411855in}{1.318976in}}%
\pgfusepath{stroke}%
\end{pgfscope}%
\begin{pgfscope}%
\pgfsetrectcap%
\pgfsetroundjoin%
\pgfsetlinewidth{0.803000pt}%
\definecolor{currentstroke}{rgb}{0.000000,0.000000,0.000000}%
\pgfsetstrokecolor{currentstroke}%
\pgfsetdash{}{0pt}%
\pgfpathmoveto{\pgfqpoint{0.427845in}{1.522058in}}%
\pgfpathlineto{\pgfqpoint{0.396264in}{1.500926in}}%
\pgfusepath{stroke}%
\end{pgfscope}%
\begin{pgfscope}%
\pgfsetrectcap%
\pgfsetroundjoin%
\pgfsetlinewidth{0.803000pt}%
\definecolor{currentstroke}{rgb}{0.000000,0.000000,0.000000}%
\pgfsetstrokecolor{currentstroke}%
\pgfsetdash{}{0pt}%
\pgfpathmoveto{\pgfqpoint{0.412281in}{1.708614in}}%
\pgfpathlineto{\pgfqpoint{0.380278in}{1.687484in}}%
\pgfusepath{stroke}%
\end{pgfscope}%
\begin{pgfscope}%
\pgfsetrectcap%
\pgfsetroundjoin%
\pgfsetlinewidth{0.803000pt}%
\definecolor{currentstroke}{rgb}{0.000000,0.000000,0.000000}%
\pgfsetstrokecolor{currentstroke}%
\pgfsetdash{}{0pt}%
\pgfpathmoveto{\pgfqpoint{0.396319in}{1.899948in}}%
\pgfpathlineto{\pgfqpoint{0.363882in}{1.878827in}}%
\pgfusepath{stroke}%
\end{pgfscope}%
\begin{pgfscope}%
\definecolor{textcolor}{rgb}{0.000000,0.000000,0.000000}%
\pgfsetstrokecolor{textcolor}%
\pgfsetfillcolor{textcolor}%
\pgftext[x=-0.143944in,y=1.551958in,,]{\color{textcolor}{\rmfamily\fontsize{14.000000}{16.800000}\selectfont\catcode`\^=\active\def^{\ifmmode\sp\else\^{}\fi}\catcode`\%=\active\def%{\%}f3}}%
\end{pgfscope}%
\begin{pgfscope}%
\pgfpathrectangle{\pgfqpoint{0.254231in}{0.147348in}}{\pgfqpoint{2.735294in}{2.735294in}}%
\pgfusepath{clip}%
\pgfsetbuttcap%
\pgfsetroundjoin%
\definecolor{currentfill}{rgb}{0.050070,0.192203,0.290728}%
\pgfsetfillcolor{currentfill}%
\pgfsetlinewidth{0.000000pt}%
\definecolor{currentstroke}{rgb}{0.000000,0.000000,0.000000}%
\pgfsetstrokecolor{currentstroke}%
\pgfsetdash{}{0pt}%
\pgfpathmoveto{\pgfqpoint{2.615535in}{1.291641in}}%
\pgfpathlineto{\pgfqpoint{2.528344in}{1.165011in}}%
\pgfpathlineto{\pgfqpoint{2.615440in}{1.225003in}}%
\pgfpathlineto{\pgfqpoint{2.615535in}{1.291641in}}%
\pgfpathclose%
\pgfusepath{fill}%
\end{pgfscope}%
\begin{pgfscope}%
\pgfpathrectangle{\pgfqpoint{0.254231in}{0.147348in}}{\pgfqpoint{2.735294in}{2.735294in}}%
\pgfusepath{clip}%
\pgfsetbuttcap%
\pgfsetroundjoin%
\definecolor{currentfill}{rgb}{0.050070,0.192203,0.290728}%
\pgfsetfillcolor{currentfill}%
\pgfsetlinewidth{0.000000pt}%
\definecolor{currentstroke}{rgb}{0.000000,0.000000,0.000000}%
\pgfsetstrokecolor{currentstroke}%
\pgfsetdash{}{0pt}%
\pgfpathmoveto{\pgfqpoint{0.789340in}{1.165011in}}%
\pgfpathlineto{\pgfqpoint{0.702149in}{1.291641in}}%
\pgfpathlineto{\pgfqpoint{0.702244in}{1.225003in}}%
\pgfpathlineto{\pgfqpoint{0.789340in}{1.165011in}}%
\pgfpathclose%
\pgfusepath{fill}%
\end{pgfscope}%
\begin{pgfscope}%
\pgfpathrectangle{\pgfqpoint{0.254231in}{0.147348in}}{\pgfqpoint{2.735294in}{2.735294in}}%
\pgfusepath{clip}%
\pgfsetbuttcap%
\pgfsetroundjoin%
\definecolor{currentfill}{rgb}{0.090605,0.347808,0.526096}%
\pgfsetfillcolor{currentfill}%
\pgfsetlinewidth{0.000000pt}%
\definecolor{currentstroke}{rgb}{0.000000,0.000000,0.000000}%
\pgfsetstrokecolor{currentstroke}%
\pgfsetdash{}{0pt}%
\pgfpathmoveto{\pgfqpoint{1.571650in}{2.561465in}}%
\pgfpathlineto{\pgfqpoint{1.746034in}{2.561465in}}%
\pgfpathlineto{\pgfqpoint{1.658842in}{2.621838in}}%
\pgfpathlineto{\pgfqpoint{1.571650in}{2.561465in}}%
\pgfpathclose%
\pgfusepath{fill}%
\end{pgfscope}%
\begin{pgfscope}%
\pgfpathrectangle{\pgfqpoint{0.254231in}{0.147348in}}{\pgfqpoint{2.735294in}{2.735294in}}%
\pgfusepath{clip}%
\pgfsetbuttcap%
\pgfsetroundjoin%
\definecolor{currentfill}{rgb}{0.047548,0.182523,0.276086}%
\pgfsetfillcolor{currentfill}%
\pgfsetlinewidth{0.000000pt}%
\definecolor{currentstroke}{rgb}{0.000000,0.000000,0.000000}%
\pgfsetstrokecolor{currentstroke}%
\pgfsetdash{}{0pt}%
\pgfpathmoveto{\pgfqpoint{2.528344in}{1.165011in}}%
\pgfpathlineto{\pgfqpoint{2.518939in}{1.232896in}}%
\pgfpathlineto{\pgfqpoint{2.415375in}{1.101844in}}%
\pgfpathlineto{\pgfqpoint{2.528344in}{1.165011in}}%
\pgfpathclose%
\pgfusepath{fill}%
\end{pgfscope}%
\begin{pgfscope}%
\pgfpathrectangle{\pgfqpoint{0.254231in}{0.147348in}}{\pgfqpoint{2.735294in}{2.735294in}}%
\pgfusepath{clip}%
\pgfsetbuttcap%
\pgfsetroundjoin%
\definecolor{currentfill}{rgb}{0.047548,0.182523,0.276086}%
\pgfsetfillcolor{currentfill}%
\pgfsetlinewidth{0.000000pt}%
\definecolor{currentstroke}{rgb}{0.000000,0.000000,0.000000}%
\pgfsetstrokecolor{currentstroke}%
\pgfsetdash{}{0pt}%
\pgfpathmoveto{\pgfqpoint{0.902309in}{1.101844in}}%
\pgfpathlineto{\pgfqpoint{0.798745in}{1.232896in}}%
\pgfpathlineto{\pgfqpoint{0.789340in}{1.165011in}}%
\pgfpathlineto{\pgfqpoint{0.902309in}{1.101844in}}%
\pgfpathclose%
\pgfusepath{fill}%
\end{pgfscope}%
\begin{pgfscope}%
\pgfpathrectangle{\pgfqpoint{0.254231in}{0.147348in}}{\pgfqpoint{2.735294in}{2.735294in}}%
\pgfusepath{clip}%
\pgfsetbuttcap%
\pgfsetroundjoin%
\definecolor{currentfill}{rgb}{0.048960,0.187944,0.284285}%
\pgfsetfillcolor{currentfill}%
\pgfsetlinewidth{0.000000pt}%
\definecolor{currentstroke}{rgb}{0.000000,0.000000,0.000000}%
\pgfsetstrokecolor{currentstroke}%
\pgfsetdash{}{0pt}%
\pgfpathmoveto{\pgfqpoint{0.702149in}{1.291641in}}%
\pgfpathlineto{\pgfqpoint{0.789340in}{1.165011in}}%
\pgfpathlineto{\pgfqpoint{0.788552in}{1.618388in}}%
\pgfpathlineto{\pgfqpoint{0.702149in}{1.291641in}}%
\pgfpathclose%
\pgfusepath{fill}%
\end{pgfscope}%
\begin{pgfscope}%
\pgfpathrectangle{\pgfqpoint{0.254231in}{0.147348in}}{\pgfqpoint{2.735294in}{2.735294in}}%
\pgfusepath{clip}%
\pgfsetbuttcap%
\pgfsetroundjoin%
\definecolor{currentfill}{rgb}{0.048960,0.187944,0.284285}%
\pgfsetfillcolor{currentfill}%
\pgfsetlinewidth{0.000000pt}%
\definecolor{currentstroke}{rgb}{0.000000,0.000000,0.000000}%
\pgfsetstrokecolor{currentstroke}%
\pgfsetdash{}{0pt}%
\pgfpathmoveto{\pgfqpoint{2.615535in}{1.291641in}}%
\pgfpathlineto{\pgfqpoint{2.529132in}{1.618388in}}%
\pgfpathlineto{\pgfqpoint{2.528344in}{1.165011in}}%
\pgfpathlineto{\pgfqpoint{2.615535in}{1.291641in}}%
\pgfpathclose%
\pgfusepath{fill}%
\end{pgfscope}%
\begin{pgfscope}%
\pgfpathrectangle{\pgfqpoint{0.254231in}{0.147348in}}{\pgfqpoint{2.735294in}{2.735294in}}%
\pgfusepath{clip}%
\pgfsetbuttcap%
\pgfsetroundjoin%
\definecolor{currentfill}{rgb}{0.070254,0.269685,0.407928}%
\pgfsetfillcolor{currentfill}%
\pgfsetlinewidth{0.000000pt}%
\definecolor{currentstroke}{rgb}{0.000000,0.000000,0.000000}%
\pgfsetstrokecolor{currentstroke}%
\pgfsetdash{}{0pt}%
\pgfpathmoveto{\pgfqpoint{2.518939in}{1.232896in}}%
\pgfpathlineto{\pgfqpoint{2.528344in}{1.165011in}}%
\pgfpathlineto{\pgfqpoint{2.529132in}{1.618388in}}%
\pgfpathlineto{\pgfqpoint{2.518939in}{1.232896in}}%
\pgfpathclose%
\pgfusepath{fill}%
\end{pgfscope}%
\begin{pgfscope}%
\pgfpathrectangle{\pgfqpoint{0.254231in}{0.147348in}}{\pgfqpoint{2.735294in}{2.735294in}}%
\pgfusepath{clip}%
\pgfsetbuttcap%
\pgfsetroundjoin%
\definecolor{currentfill}{rgb}{0.070254,0.269685,0.407928}%
\pgfsetfillcolor{currentfill}%
\pgfsetlinewidth{0.000000pt}%
\definecolor{currentstroke}{rgb}{0.000000,0.000000,0.000000}%
\pgfsetstrokecolor{currentstroke}%
\pgfsetdash{}{0pt}%
\pgfpathmoveto{\pgfqpoint{0.788552in}{1.618388in}}%
\pgfpathlineto{\pgfqpoint{0.789340in}{1.165011in}}%
\pgfpathlineto{\pgfqpoint{0.798745in}{1.232896in}}%
\pgfpathlineto{\pgfqpoint{0.788552in}{1.618388in}}%
\pgfpathclose%
\pgfusepath{fill}%
\end{pgfscope}%
\begin{pgfscope}%
\pgfpathrectangle{\pgfqpoint{0.254231in}{0.147348in}}{\pgfqpoint{2.735294in}{2.735294in}}%
\pgfusepath{clip}%
\pgfsetbuttcap%
\pgfsetroundjoin%
\definecolor{currentfill}{rgb}{0.044978,0.172658,0.261163}%
\pgfsetfillcolor{currentfill}%
\pgfsetlinewidth{0.000000pt}%
\definecolor{currentstroke}{rgb}{0.000000,0.000000,0.000000}%
\pgfsetstrokecolor{currentstroke}%
\pgfsetdash{}{0pt}%
\pgfpathmoveto{\pgfqpoint{1.046414in}{1.039603in}}%
\pgfpathlineto{\pgfqpoint{0.926707in}{1.171211in}}%
\pgfpathlineto{\pgfqpoint{0.902309in}{1.101844in}}%
\pgfpathlineto{\pgfqpoint{1.046414in}{1.039603in}}%
\pgfpathclose%
\pgfusepath{fill}%
\end{pgfscope}%
\begin{pgfscope}%
\pgfpathrectangle{\pgfqpoint{0.254231in}{0.147348in}}{\pgfqpoint{2.735294in}{2.735294in}}%
\pgfusepath{clip}%
\pgfsetbuttcap%
\pgfsetroundjoin%
\definecolor{currentfill}{rgb}{0.044978,0.172658,0.261163}%
\pgfsetfillcolor{currentfill}%
\pgfsetlinewidth{0.000000pt}%
\definecolor{currentstroke}{rgb}{0.000000,0.000000,0.000000}%
\pgfsetstrokecolor{currentstroke}%
\pgfsetdash{}{0pt}%
\pgfpathmoveto{\pgfqpoint{2.415375in}{1.101844in}}%
\pgfpathlineto{\pgfqpoint{2.390977in}{1.171211in}}%
\pgfpathlineto{\pgfqpoint{2.271270in}{1.039603in}}%
\pgfpathlineto{\pgfqpoint{2.415375in}{1.101844in}}%
\pgfpathclose%
\pgfusepath{fill}%
\end{pgfscope}%
\begin{pgfscope}%
\pgfpathrectangle{\pgfqpoint{0.254231in}{0.147348in}}{\pgfqpoint{2.735294in}{2.735294in}}%
\pgfusepath{clip}%
\pgfsetbuttcap%
\pgfsetroundjoin%
\definecolor{currentfill}{rgb}{0.081954,0.314596,0.475860}%
\pgfsetfillcolor{currentfill}%
\pgfsetlinewidth{0.000000pt}%
\definecolor{currentstroke}{rgb}{0.000000,0.000000,0.000000}%
\pgfsetstrokecolor{currentstroke}%
\pgfsetdash{}{0pt}%
\pgfpathmoveto{\pgfqpoint{1.746034in}{2.561465in}}%
\pgfpathlineto{\pgfqpoint{1.571650in}{2.561465in}}%
\pgfpathlineto{\pgfqpoint{1.534055in}{2.124855in}}%
\pgfpathlineto{\pgfqpoint{1.746034in}{2.561465in}}%
\pgfpathclose%
\pgfusepath{fill}%
\end{pgfscope}%
\begin{pgfscope}%
\pgfpathrectangle{\pgfqpoint{0.254231in}{0.147348in}}{\pgfqpoint{2.735294in}{2.735294in}}%
\pgfusepath{clip}%
\pgfsetbuttcap%
\pgfsetroundjoin%
\definecolor{currentfill}{rgb}{0.047247,0.181368,0.274339}%
\pgfsetfillcolor{currentfill}%
\pgfsetlinewidth{0.000000pt}%
\definecolor{currentstroke}{rgb}{0.000000,0.000000,0.000000}%
\pgfsetstrokecolor{currentstroke}%
\pgfsetdash{}{0pt}%
\pgfpathmoveto{\pgfqpoint{2.382112in}{1.589448in}}%
\pgfpathlineto{\pgfqpoint{2.415375in}{1.101844in}}%
\pgfpathlineto{\pgfqpoint{2.518939in}{1.232896in}}%
\pgfpathlineto{\pgfqpoint{2.382112in}{1.589448in}}%
\pgfpathclose%
\pgfusepath{fill}%
\end{pgfscope}%
\begin{pgfscope}%
\pgfpathrectangle{\pgfqpoint{0.254231in}{0.147348in}}{\pgfqpoint{2.735294in}{2.735294in}}%
\pgfusepath{clip}%
\pgfsetbuttcap%
\pgfsetroundjoin%
\definecolor{currentfill}{rgb}{0.047247,0.181368,0.274339}%
\pgfsetfillcolor{currentfill}%
\pgfsetlinewidth{0.000000pt}%
\definecolor{currentstroke}{rgb}{0.000000,0.000000,0.000000}%
\pgfsetstrokecolor{currentstroke}%
\pgfsetdash{}{0pt}%
\pgfpathmoveto{\pgfqpoint{0.798745in}{1.232896in}}%
\pgfpathlineto{\pgfqpoint{0.902309in}{1.101844in}}%
\pgfpathlineto{\pgfqpoint{0.935572in}{1.589448in}}%
\pgfpathlineto{\pgfqpoint{0.798745in}{1.232896in}}%
\pgfpathclose%
\pgfusepath{fill}%
\end{pgfscope}%
\begin{pgfscope}%
\pgfpathrectangle{\pgfqpoint{0.254231in}{0.147348in}}{\pgfqpoint{2.735294in}{2.735294in}}%
\pgfusepath{clip}%
\pgfsetbuttcap%
\pgfsetroundjoin%
\definecolor{currentfill}{rgb}{0.067179,0.257880,0.390071}%
\pgfsetfillcolor{currentfill}%
\pgfsetlinewidth{0.000000pt}%
\definecolor{currentstroke}{rgb}{0.000000,0.000000,0.000000}%
\pgfsetstrokecolor{currentstroke}%
\pgfsetdash{}{0pt}%
\pgfpathmoveto{\pgfqpoint{0.935572in}{1.589448in}}%
\pgfpathlineto{\pgfqpoint{0.902309in}{1.101844in}}%
\pgfpathlineto{\pgfqpoint{0.926707in}{1.171211in}}%
\pgfpathlineto{\pgfqpoint{0.935572in}{1.589448in}}%
\pgfpathclose%
\pgfusepath{fill}%
\end{pgfscope}%
\begin{pgfscope}%
\pgfpathrectangle{\pgfqpoint{0.254231in}{0.147348in}}{\pgfqpoint{2.735294in}{2.735294in}}%
\pgfusepath{clip}%
\pgfsetbuttcap%
\pgfsetroundjoin%
\definecolor{currentfill}{rgb}{0.067179,0.257880,0.390071}%
\pgfsetfillcolor{currentfill}%
\pgfsetlinewidth{0.000000pt}%
\definecolor{currentstroke}{rgb}{0.000000,0.000000,0.000000}%
\pgfsetstrokecolor{currentstroke}%
\pgfsetdash{}{0pt}%
\pgfpathmoveto{\pgfqpoint{2.390977in}{1.171211in}}%
\pgfpathlineto{\pgfqpoint{2.415375in}{1.101844in}}%
\pgfpathlineto{\pgfqpoint{2.382112in}{1.589448in}}%
\pgfpathlineto{\pgfqpoint{2.390977in}{1.171211in}}%
\pgfpathclose%
\pgfusepath{fill}%
\end{pgfscope}%
\begin{pgfscope}%
\pgfpathrectangle{\pgfqpoint{0.254231in}{0.147348in}}{\pgfqpoint{2.735294in}{2.735294in}}%
\pgfusepath{clip}%
\pgfsetbuttcap%
\pgfsetroundjoin%
\definecolor{currentfill}{rgb}{0.042579,0.163449,0.247234}%
\pgfsetfillcolor{currentfill}%
\pgfsetlinewidth{0.000000pt}%
\definecolor{currentstroke}{rgb}{0.000000,0.000000,0.000000}%
\pgfsetstrokecolor{currentstroke}%
\pgfsetdash{}{0pt}%
\pgfpathmoveto{\pgfqpoint{1.046414in}{1.039603in}}%
\pgfpathlineto{\pgfqpoint{1.224246in}{0.985051in}}%
\pgfpathlineto{\pgfqpoint{1.092413in}{1.111775in}}%
\pgfpathlineto{\pgfqpoint{1.046414in}{1.039603in}}%
\pgfpathclose%
\pgfusepath{fill}%
\end{pgfscope}%
\begin{pgfscope}%
\pgfpathrectangle{\pgfqpoint{0.254231in}{0.147348in}}{\pgfqpoint{2.735294in}{2.735294in}}%
\pgfusepath{clip}%
\pgfsetbuttcap%
\pgfsetroundjoin%
\definecolor{currentfill}{rgb}{0.042579,0.163449,0.247234}%
\pgfsetfillcolor{currentfill}%
\pgfsetlinewidth{0.000000pt}%
\definecolor{currentstroke}{rgb}{0.000000,0.000000,0.000000}%
\pgfsetstrokecolor{currentstroke}%
\pgfsetdash{}{0pt}%
\pgfpathmoveto{\pgfqpoint{2.225271in}{1.111775in}}%
\pgfpathlineto{\pgfqpoint{2.093438in}{0.985051in}}%
\pgfpathlineto{\pgfqpoint{2.271270in}{1.039603in}}%
\pgfpathlineto{\pgfqpoint{2.225271in}{1.111775in}}%
\pgfpathclose%
\pgfusepath{fill}%
\end{pgfscope}%
\begin{pgfscope}%
\pgfpathrectangle{\pgfqpoint{0.254231in}{0.147348in}}{\pgfqpoint{2.735294in}{2.735294in}}%
\pgfusepath{clip}%
\pgfsetbuttcap%
\pgfsetroundjoin%
\definecolor{currentfill}{rgb}{0.052493,0.201505,0.304798}%
\pgfsetfillcolor{currentfill}%
\pgfsetlinewidth{0.000000pt}%
\definecolor{currentstroke}{rgb}{0.000000,0.000000,0.000000}%
\pgfsetstrokecolor{currentstroke}%
\pgfsetdash{}{0pt}%
\pgfpathmoveto{\pgfqpoint{2.382112in}{1.589448in}}%
\pgfpathlineto{\pgfqpoint{2.518939in}{1.232896in}}%
\pgfpathlineto{\pgfqpoint{2.529132in}{1.618388in}}%
\pgfpathlineto{\pgfqpoint{2.382112in}{1.589448in}}%
\pgfpathclose%
\pgfusepath{fill}%
\end{pgfscope}%
\begin{pgfscope}%
\pgfpathrectangle{\pgfqpoint{0.254231in}{0.147348in}}{\pgfqpoint{2.735294in}{2.735294in}}%
\pgfusepath{clip}%
\pgfsetbuttcap%
\pgfsetroundjoin%
\definecolor{currentfill}{rgb}{0.052493,0.201505,0.304798}%
\pgfsetfillcolor{currentfill}%
\pgfsetlinewidth{0.000000pt}%
\definecolor{currentstroke}{rgb}{0.000000,0.000000,0.000000}%
\pgfsetstrokecolor{currentstroke}%
\pgfsetdash{}{0pt}%
\pgfpathmoveto{\pgfqpoint{0.788552in}{1.618388in}}%
\pgfpathlineto{\pgfqpoint{0.798745in}{1.232896in}}%
\pgfpathlineto{\pgfqpoint{0.935572in}{1.589448in}}%
\pgfpathlineto{\pgfqpoint{0.788552in}{1.618388in}}%
\pgfpathclose%
\pgfusepath{fill}%
\end{pgfscope}%
\begin{pgfscope}%
\pgfpathrectangle{\pgfqpoint{0.254231in}{0.147348in}}{\pgfqpoint{2.735294in}{2.735294in}}%
\pgfusepath{clip}%
\pgfsetbuttcap%
\pgfsetroundjoin%
\definecolor{currentfill}{rgb}{0.082280,0.315849,0.477755}%
\pgfsetfillcolor{currentfill}%
\pgfsetlinewidth{0.000000pt}%
\definecolor{currentstroke}{rgb}{0.000000,0.000000,0.000000}%
\pgfsetstrokecolor{currentstroke}%
\pgfsetdash{}{0pt}%
\pgfpathmoveto{\pgfqpoint{2.094537in}{2.253326in}}%
\pgfpathlineto{\pgfqpoint{1.746034in}{2.561465in}}%
\pgfpathlineto{\pgfqpoint{1.783629in}{2.124855in}}%
\pgfpathlineto{\pgfqpoint{2.094537in}{2.253326in}}%
\pgfpathclose%
\pgfusepath{fill}%
\end{pgfscope}%
\begin{pgfscope}%
\pgfpathrectangle{\pgfqpoint{0.254231in}{0.147348in}}{\pgfqpoint{2.735294in}{2.735294in}}%
\pgfusepath{clip}%
\pgfsetbuttcap%
\pgfsetroundjoin%
\definecolor{currentfill}{rgb}{0.082280,0.315849,0.477755}%
\pgfsetfillcolor{currentfill}%
\pgfsetlinewidth{0.000000pt}%
\definecolor{currentstroke}{rgb}{0.000000,0.000000,0.000000}%
\pgfsetstrokecolor{currentstroke}%
\pgfsetdash{}{0pt}%
\pgfpathmoveto{\pgfqpoint{1.534055in}{2.124855in}}%
\pgfpathlineto{\pgfqpoint{1.571650in}{2.561465in}}%
\pgfpathlineto{\pgfqpoint{1.223147in}{2.253326in}}%
\pgfpathlineto{\pgfqpoint{1.534055in}{2.124855in}}%
\pgfpathclose%
\pgfusepath{fill}%
\end{pgfscope}%
\begin{pgfscope}%
\pgfpathrectangle{\pgfqpoint{0.254231in}{0.147348in}}{\pgfqpoint{2.735294in}{2.735294in}}%
\pgfusepath{clip}%
\pgfsetbuttcap%
\pgfsetroundjoin%
\definecolor{currentfill}{rgb}{0.045702,0.175435,0.265364}%
\pgfsetfillcolor{currentfill}%
\pgfsetlinewidth{0.000000pt}%
\definecolor{currentstroke}{rgb}{0.000000,0.000000,0.000000}%
\pgfsetstrokecolor{currentstroke}%
\pgfsetdash{}{0pt}%
\pgfpathmoveto{\pgfqpoint{0.926707in}{1.171211in}}%
\pgfpathlineto{\pgfqpoint{1.046414in}{1.039603in}}%
\pgfpathlineto{\pgfqpoint{1.132195in}{1.562087in}}%
\pgfpathlineto{\pgfqpoint{0.926707in}{1.171211in}}%
\pgfpathclose%
\pgfusepath{fill}%
\end{pgfscope}%
\begin{pgfscope}%
\pgfpathrectangle{\pgfqpoint{0.254231in}{0.147348in}}{\pgfqpoint{2.735294in}{2.735294in}}%
\pgfusepath{clip}%
\pgfsetbuttcap%
\pgfsetroundjoin%
\definecolor{currentfill}{rgb}{0.045702,0.175435,0.265364}%
\pgfsetfillcolor{currentfill}%
\pgfsetlinewidth{0.000000pt}%
\definecolor{currentstroke}{rgb}{0.000000,0.000000,0.000000}%
\pgfsetstrokecolor{currentstroke}%
\pgfsetdash{}{0pt}%
\pgfpathmoveto{\pgfqpoint{2.185489in}{1.562087in}}%
\pgfpathlineto{\pgfqpoint{2.271270in}{1.039603in}}%
\pgfpathlineto{\pgfqpoint{2.390977in}{1.171211in}}%
\pgfpathlineto{\pgfqpoint{2.185489in}{1.562087in}}%
\pgfpathclose%
\pgfusepath{fill}%
\end{pgfscope}%
\begin{pgfscope}%
\pgfpathrectangle{\pgfqpoint{0.254231in}{0.147348in}}{\pgfqpoint{2.735294in}{2.735294in}}%
\pgfusepath{clip}%
\pgfsetbuttcap%
\pgfsetroundjoin%
\definecolor{currentfill}{rgb}{0.040669,0.156116,0.236142}%
\pgfsetfillcolor{currentfill}%
\pgfsetlinewidth{0.000000pt}%
\definecolor{currentstroke}{rgb}{0.000000,0.000000,0.000000}%
\pgfsetstrokecolor{currentstroke}%
\pgfsetdash{}{0pt}%
\pgfpathmoveto{\pgfqpoint{1.297586in}{1.063020in}}%
\pgfpathlineto{\pgfqpoint{1.224246in}{0.985051in}}%
\pgfpathlineto{\pgfqpoint{1.432345in}{0.946905in}}%
\pgfpathlineto{\pgfqpoint{1.297586in}{1.063020in}}%
\pgfpathclose%
\pgfusepath{fill}%
\end{pgfscope}%
\begin{pgfscope}%
\pgfpathrectangle{\pgfqpoint{0.254231in}{0.147348in}}{\pgfqpoint{2.735294in}{2.735294in}}%
\pgfusepath{clip}%
\pgfsetbuttcap%
\pgfsetroundjoin%
\definecolor{currentfill}{rgb}{0.040669,0.156116,0.236142}%
\pgfsetfillcolor{currentfill}%
\pgfsetlinewidth{0.000000pt}%
\definecolor{currentstroke}{rgb}{0.000000,0.000000,0.000000}%
\pgfsetstrokecolor{currentstroke}%
\pgfsetdash{}{0pt}%
\pgfpathmoveto{\pgfqpoint{1.885339in}{0.946905in}}%
\pgfpathlineto{\pgfqpoint{2.093438in}{0.985051in}}%
\pgfpathlineto{\pgfqpoint{2.020098in}{1.063020in}}%
\pgfpathlineto{\pgfqpoint{1.885339in}{0.946905in}}%
\pgfpathclose%
\pgfusepath{fill}%
\end{pgfscope}%
\begin{pgfscope}%
\pgfpathrectangle{\pgfqpoint{0.254231in}{0.147348in}}{\pgfqpoint{2.735294in}{2.735294in}}%
\pgfusepath{clip}%
\pgfsetbuttcap%
\pgfsetroundjoin%
\definecolor{currentfill}{rgb}{0.063981,0.245604,0.371502}%
\pgfsetfillcolor{currentfill}%
\pgfsetlinewidth{0.000000pt}%
\definecolor{currentstroke}{rgb}{0.000000,0.000000,0.000000}%
\pgfsetstrokecolor{currentstroke}%
\pgfsetdash{}{0pt}%
\pgfpathmoveto{\pgfqpoint{1.132195in}{1.562087in}}%
\pgfpathlineto{\pgfqpoint{1.046414in}{1.039603in}}%
\pgfpathlineto{\pgfqpoint{1.092413in}{1.111775in}}%
\pgfpathlineto{\pgfqpoint{1.132195in}{1.562087in}}%
\pgfpathclose%
\pgfusepath{fill}%
\end{pgfscope}%
\begin{pgfscope}%
\pgfpathrectangle{\pgfqpoint{0.254231in}{0.147348in}}{\pgfqpoint{2.735294in}{2.735294in}}%
\pgfusepath{clip}%
\pgfsetbuttcap%
\pgfsetroundjoin%
\definecolor{currentfill}{rgb}{0.063981,0.245604,0.371502}%
\pgfsetfillcolor{currentfill}%
\pgfsetlinewidth{0.000000pt}%
\definecolor{currentstroke}{rgb}{0.000000,0.000000,0.000000}%
\pgfsetstrokecolor{currentstroke}%
\pgfsetdash{}{0pt}%
\pgfpathmoveto{\pgfqpoint{2.225271in}{1.111775in}}%
\pgfpathlineto{\pgfqpoint{2.271270in}{1.039603in}}%
\pgfpathlineto{\pgfqpoint{2.185489in}{1.562087in}}%
\pgfpathlineto{\pgfqpoint{2.225271in}{1.111775in}}%
\pgfpathclose%
\pgfusepath{fill}%
\end{pgfscope}%
\begin{pgfscope}%
\pgfpathrectangle{\pgfqpoint{0.254231in}{0.147348in}}{\pgfqpoint{2.735294in}{2.735294in}}%
\pgfusepath{clip}%
\pgfsetbuttcap%
\pgfsetroundjoin%
\definecolor{currentfill}{rgb}{0.060942,0.233938,0.353856}%
\pgfsetfillcolor{currentfill}%
\pgfsetlinewidth{0.000000pt}%
\definecolor{currentstroke}{rgb}{0.000000,0.000000,0.000000}%
\pgfsetstrokecolor{currentstroke}%
\pgfsetdash{}{0pt}%
\pgfpathmoveto{\pgfqpoint{0.935572in}{1.589448in}}%
\pgfpathlineto{\pgfqpoint{0.864797in}{1.772344in}}%
\pgfpathlineto{\pgfqpoint{0.788552in}{1.618388in}}%
\pgfpathlineto{\pgfqpoint{0.935572in}{1.589448in}}%
\pgfpathclose%
\pgfusepath{fill}%
\end{pgfscope}%
\begin{pgfscope}%
\pgfpathrectangle{\pgfqpoint{0.254231in}{0.147348in}}{\pgfqpoint{2.735294in}{2.735294in}}%
\pgfusepath{clip}%
\pgfsetbuttcap%
\pgfsetroundjoin%
\definecolor{currentfill}{rgb}{0.060942,0.233938,0.353856}%
\pgfsetfillcolor{currentfill}%
\pgfsetlinewidth{0.000000pt}%
\definecolor{currentstroke}{rgb}{0.000000,0.000000,0.000000}%
\pgfsetstrokecolor{currentstroke}%
\pgfsetdash{}{0pt}%
\pgfpathmoveto{\pgfqpoint{2.529132in}{1.618388in}}%
\pgfpathlineto{\pgfqpoint{2.452887in}{1.772344in}}%
\pgfpathlineto{\pgfqpoint{2.382112in}{1.589448in}}%
\pgfpathlineto{\pgfqpoint{2.529132in}{1.618388in}}%
\pgfpathclose%
\pgfusepath{fill}%
\end{pgfscope}%
\begin{pgfscope}%
\pgfpathrectangle{\pgfqpoint{0.254231in}{0.147348in}}{\pgfqpoint{2.735294in}{2.735294in}}%
\pgfusepath{clip}%
\pgfsetbuttcap%
\pgfsetroundjoin%
\definecolor{currentfill}{rgb}{0.081954,0.314596,0.475860}%
\pgfsetfillcolor{currentfill}%
\pgfsetlinewidth{0.000000pt}%
\definecolor{currentstroke}{rgb}{0.000000,0.000000,0.000000}%
\pgfsetstrokecolor{currentstroke}%
\pgfsetdash{}{0pt}%
\pgfpathmoveto{\pgfqpoint{1.534055in}{2.124855in}}%
\pgfpathlineto{\pgfqpoint{1.783629in}{2.124855in}}%
\pgfpathlineto{\pgfqpoint{1.746034in}{2.561465in}}%
\pgfpathlineto{\pgfqpoint{1.534055in}{2.124855in}}%
\pgfpathclose%
\pgfusepath{fill}%
\end{pgfscope}%
\begin{pgfscope}%
\pgfpathrectangle{\pgfqpoint{0.254231in}{0.147348in}}{\pgfqpoint{2.735294in}{2.735294in}}%
\pgfusepath{clip}%
\pgfsetbuttcap%
\pgfsetroundjoin%
\definecolor{currentfill}{rgb}{0.039595,0.151995,0.229908}%
\pgfsetfillcolor{currentfill}%
\pgfsetlinewidth{0.000000pt}%
\definecolor{currentstroke}{rgb}{0.000000,0.000000,0.000000}%
\pgfsetstrokecolor{currentstroke}%
\pgfsetdash{}{0pt}%
\pgfpathmoveto{\pgfqpoint{1.658842in}{0.933095in}}%
\pgfpathlineto{\pgfqpoint{1.534325in}{1.035006in}}%
\pgfpathlineto{\pgfqpoint{1.432345in}{0.946905in}}%
\pgfpathlineto{\pgfqpoint{1.658842in}{0.933095in}}%
\pgfpathclose%
\pgfusepath{fill}%
\end{pgfscope}%
\begin{pgfscope}%
\pgfpathrectangle{\pgfqpoint{0.254231in}{0.147348in}}{\pgfqpoint{2.735294in}{2.735294in}}%
\pgfusepath{clip}%
\pgfsetbuttcap%
\pgfsetroundjoin%
\definecolor{currentfill}{rgb}{0.039595,0.151995,0.229908}%
\pgfsetfillcolor{currentfill}%
\pgfsetlinewidth{0.000000pt}%
\definecolor{currentstroke}{rgb}{0.000000,0.000000,0.000000}%
\pgfsetstrokecolor{currentstroke}%
\pgfsetdash{}{0pt}%
\pgfpathmoveto{\pgfqpoint{1.885339in}{0.946905in}}%
\pgfpathlineto{\pgfqpoint{1.783359in}{1.035006in}}%
\pgfpathlineto{\pgfqpoint{1.658842in}{0.933095in}}%
\pgfpathlineto{\pgfqpoint{1.885339in}{0.946905in}}%
\pgfpathclose%
\pgfusepath{fill}%
\end{pgfscope}%
\begin{pgfscope}%
\pgfpathrectangle{\pgfqpoint{0.254231in}{0.147348in}}{\pgfqpoint{2.735294in}{2.735294in}}%
\pgfusepath{clip}%
\pgfsetbuttcap%
\pgfsetroundjoin%
\definecolor{currentfill}{rgb}{0.075436,0.289576,0.438014}%
\pgfsetfillcolor{currentfill}%
\pgfsetlinewidth{0.000000pt}%
\definecolor{currentstroke}{rgb}{0.000000,0.000000,0.000000}%
\pgfsetstrokecolor{currentstroke}%
\pgfsetdash{}{0pt}%
\pgfpathmoveto{\pgfqpoint{2.226391in}{2.103280in}}%
\pgfpathlineto{\pgfqpoint{2.094537in}{2.253326in}}%
\pgfpathlineto{\pgfqpoint{2.020857in}{2.116981in}}%
\pgfpathlineto{\pgfqpoint{2.226391in}{2.103280in}}%
\pgfpathclose%
\pgfusepath{fill}%
\end{pgfscope}%
\begin{pgfscope}%
\pgfpathrectangle{\pgfqpoint{0.254231in}{0.147348in}}{\pgfqpoint{2.735294in}{2.735294in}}%
\pgfusepath{clip}%
\pgfsetbuttcap%
\pgfsetroundjoin%
\definecolor{currentfill}{rgb}{0.075436,0.289576,0.438014}%
\pgfsetfillcolor{currentfill}%
\pgfsetlinewidth{0.000000pt}%
\definecolor{currentstroke}{rgb}{0.000000,0.000000,0.000000}%
\pgfsetstrokecolor{currentstroke}%
\pgfsetdash{}{0pt}%
\pgfpathmoveto{\pgfqpoint{1.296827in}{2.116981in}}%
\pgfpathlineto{\pgfqpoint{1.223147in}{2.253326in}}%
\pgfpathlineto{\pgfqpoint{1.091293in}{2.103280in}}%
\pgfpathlineto{\pgfqpoint{1.296827in}{2.116981in}}%
\pgfpathclose%
\pgfusepath{fill}%
\end{pgfscope}%
\begin{pgfscope}%
\pgfpathrectangle{\pgfqpoint{0.254231in}{0.147348in}}{\pgfqpoint{2.735294in}{2.735294in}}%
\pgfusepath{clip}%
\pgfsetbuttcap%
\pgfsetroundjoin%
\definecolor{currentfill}{rgb}{0.062760,0.240916,0.364410}%
\pgfsetfillcolor{currentfill}%
\pgfsetlinewidth{0.000000pt}%
\definecolor{currentstroke}{rgb}{0.000000,0.000000,0.000000}%
\pgfsetstrokecolor{currentstroke}%
\pgfsetdash{}{0pt}%
\pgfpathmoveto{\pgfqpoint{0.935572in}{1.589448in}}%
\pgfpathlineto{\pgfqpoint{1.091293in}{2.103280in}}%
\pgfpathlineto{\pgfqpoint{0.864797in}{1.772344in}}%
\pgfpathlineto{\pgfqpoint{0.935572in}{1.589448in}}%
\pgfpathclose%
\pgfusepath{fill}%
\end{pgfscope}%
\begin{pgfscope}%
\pgfpathrectangle{\pgfqpoint{0.254231in}{0.147348in}}{\pgfqpoint{2.735294in}{2.735294in}}%
\pgfusepath{clip}%
\pgfsetbuttcap%
\pgfsetroundjoin%
\definecolor{currentfill}{rgb}{0.062760,0.240916,0.364410}%
\pgfsetfillcolor{currentfill}%
\pgfsetlinewidth{0.000000pt}%
\definecolor{currentstroke}{rgb}{0.000000,0.000000,0.000000}%
\pgfsetstrokecolor{currentstroke}%
\pgfsetdash{}{0pt}%
\pgfpathmoveto{\pgfqpoint{2.452887in}{1.772344in}}%
\pgfpathlineto{\pgfqpoint{2.226391in}{2.103280in}}%
\pgfpathlineto{\pgfqpoint{2.382112in}{1.589448in}}%
\pgfpathlineto{\pgfqpoint{2.452887in}{1.772344in}}%
\pgfpathclose%
\pgfusepath{fill}%
\end{pgfscope}%
\begin{pgfscope}%
\pgfpathrectangle{\pgfqpoint{0.254231in}{0.147348in}}{\pgfqpoint{2.735294in}{2.735294in}}%
\pgfusepath{clip}%
\pgfsetbuttcap%
\pgfsetroundjoin%
\definecolor{currentfill}{rgb}{0.043508,0.167016,0.252629}%
\pgfsetfillcolor{currentfill}%
\pgfsetlinewidth{0.000000pt}%
\definecolor{currentstroke}{rgb}{0.000000,0.000000,0.000000}%
\pgfsetstrokecolor{currentstroke}%
\pgfsetdash{}{0pt}%
\pgfpathmoveto{\pgfqpoint{1.092413in}{1.111775in}}%
\pgfpathlineto{\pgfqpoint{1.224246in}{0.985051in}}%
\pgfpathlineto{\pgfqpoint{1.256196in}{1.360106in}}%
\pgfpathlineto{\pgfqpoint{1.092413in}{1.111775in}}%
\pgfpathclose%
\pgfusepath{fill}%
\end{pgfscope}%
\begin{pgfscope}%
\pgfpathrectangle{\pgfqpoint{0.254231in}{0.147348in}}{\pgfqpoint{2.735294in}{2.735294in}}%
\pgfusepath{clip}%
\pgfsetbuttcap%
\pgfsetroundjoin%
\definecolor{currentfill}{rgb}{0.043508,0.167016,0.252629}%
\pgfsetfillcolor{currentfill}%
\pgfsetlinewidth{0.000000pt}%
\definecolor{currentstroke}{rgb}{0.000000,0.000000,0.000000}%
\pgfsetstrokecolor{currentstroke}%
\pgfsetdash{}{0pt}%
\pgfpathmoveto{\pgfqpoint{2.061488in}{1.360106in}}%
\pgfpathlineto{\pgfqpoint{2.093438in}{0.985051in}}%
\pgfpathlineto{\pgfqpoint{2.225271in}{1.111775in}}%
\pgfpathlineto{\pgfqpoint{2.061488in}{1.360106in}}%
\pgfpathclose%
\pgfusepath{fill}%
\end{pgfscope}%
\begin{pgfscope}%
\pgfpathrectangle{\pgfqpoint{0.254231in}{0.147348in}}{\pgfqpoint{2.735294in}{2.735294in}}%
\pgfusepath{clip}%
\pgfsetbuttcap%
\pgfsetroundjoin%
\definecolor{currentfill}{rgb}{0.050011,0.191979,0.290388}%
\pgfsetfillcolor{currentfill}%
\pgfsetlinewidth{0.000000pt}%
\definecolor{currentstroke}{rgb}{0.000000,0.000000,0.000000}%
\pgfsetstrokecolor{currentstroke}%
\pgfsetdash{}{0pt}%
\pgfpathmoveto{\pgfqpoint{0.935572in}{1.589448in}}%
\pgfpathlineto{\pgfqpoint{0.926707in}{1.171211in}}%
\pgfpathlineto{\pgfqpoint{1.132195in}{1.562087in}}%
\pgfpathlineto{\pgfqpoint{0.935572in}{1.589448in}}%
\pgfpathclose%
\pgfusepath{fill}%
\end{pgfscope}%
\begin{pgfscope}%
\pgfpathrectangle{\pgfqpoint{0.254231in}{0.147348in}}{\pgfqpoint{2.735294in}{2.735294in}}%
\pgfusepath{clip}%
\pgfsetbuttcap%
\pgfsetroundjoin%
\definecolor{currentfill}{rgb}{0.050011,0.191979,0.290388}%
\pgfsetfillcolor{currentfill}%
\pgfsetlinewidth{0.000000pt}%
\definecolor{currentstroke}{rgb}{0.000000,0.000000,0.000000}%
\pgfsetstrokecolor{currentstroke}%
\pgfsetdash{}{0pt}%
\pgfpathmoveto{\pgfqpoint{2.185489in}{1.562087in}}%
\pgfpathlineto{\pgfqpoint{2.390977in}{1.171211in}}%
\pgfpathlineto{\pgfqpoint{2.382112in}{1.589448in}}%
\pgfpathlineto{\pgfqpoint{2.185489in}{1.562087in}}%
\pgfpathclose%
\pgfusepath{fill}%
\end{pgfscope}%
\begin{pgfscope}%
\pgfpathrectangle{\pgfqpoint{0.254231in}{0.147348in}}{\pgfqpoint{2.735294in}{2.735294in}}%
\pgfusepath{clip}%
\pgfsetbuttcap%
\pgfsetroundjoin%
\definecolor{currentfill}{rgb}{0.049941,0.191710,0.289982}%
\pgfsetfillcolor{currentfill}%
\pgfsetlinewidth{0.000000pt}%
\definecolor{currentstroke}{rgb}{0.000000,0.000000,0.000000}%
\pgfsetstrokecolor{currentstroke}%
\pgfsetdash{}{0pt}%
\pgfpathmoveto{\pgfqpoint{2.020098in}{1.063020in}}%
\pgfpathlineto{\pgfqpoint{2.093438in}{0.985051in}}%
\pgfpathlineto{\pgfqpoint{2.061488in}{1.360106in}}%
\pgfpathlineto{\pgfqpoint{2.020098in}{1.063020in}}%
\pgfpathclose%
\pgfusepath{fill}%
\end{pgfscope}%
\begin{pgfscope}%
\pgfpathrectangle{\pgfqpoint{0.254231in}{0.147348in}}{\pgfqpoint{2.735294in}{2.735294in}}%
\pgfusepath{clip}%
\pgfsetbuttcap%
\pgfsetroundjoin%
\definecolor{currentfill}{rgb}{0.049941,0.191710,0.289982}%
\pgfsetfillcolor{currentfill}%
\pgfsetlinewidth{0.000000pt}%
\definecolor{currentstroke}{rgb}{0.000000,0.000000,0.000000}%
\pgfsetstrokecolor{currentstroke}%
\pgfsetdash{}{0pt}%
\pgfpathmoveto{\pgfqpoint{1.256196in}{1.360106in}}%
\pgfpathlineto{\pgfqpoint{1.224246in}{0.985051in}}%
\pgfpathlineto{\pgfqpoint{1.297586in}{1.063020in}}%
\pgfpathlineto{\pgfqpoint{1.256196in}{1.360106in}}%
\pgfpathclose%
\pgfusepath{fill}%
\end{pgfscope}%
\begin{pgfscope}%
\pgfpathrectangle{\pgfqpoint{0.254231in}{0.147348in}}{\pgfqpoint{2.735294in}{2.735294in}}%
\pgfusepath{clip}%
\pgfsetbuttcap%
\pgfsetroundjoin%
\definecolor{currentfill}{rgb}{0.078663,0.301965,0.456754}%
\pgfsetfillcolor{currentfill}%
\pgfsetlinewidth{0.000000pt}%
\definecolor{currentstroke}{rgb}{0.000000,0.000000,0.000000}%
\pgfsetstrokecolor{currentstroke}%
\pgfsetdash{}{0pt}%
\pgfpathmoveto{\pgfqpoint{1.534055in}{2.124855in}}%
\pgfpathlineto{\pgfqpoint{1.223147in}{2.253326in}}%
\pgfpathlineto{\pgfqpoint{1.296827in}{2.116981in}}%
\pgfpathlineto{\pgfqpoint{1.534055in}{2.124855in}}%
\pgfpathclose%
\pgfusepath{fill}%
\end{pgfscope}%
\begin{pgfscope}%
\pgfpathrectangle{\pgfqpoint{0.254231in}{0.147348in}}{\pgfqpoint{2.735294in}{2.735294in}}%
\pgfusepath{clip}%
\pgfsetbuttcap%
\pgfsetroundjoin%
\definecolor{currentfill}{rgb}{0.078663,0.301965,0.456754}%
\pgfsetfillcolor{currentfill}%
\pgfsetlinewidth{0.000000pt}%
\definecolor{currentstroke}{rgb}{0.000000,0.000000,0.000000}%
\pgfsetstrokecolor{currentstroke}%
\pgfsetdash{}{0pt}%
\pgfpathmoveto{\pgfqpoint{2.020857in}{2.116981in}}%
\pgfpathlineto{\pgfqpoint{2.094537in}{2.253326in}}%
\pgfpathlineto{\pgfqpoint{1.783629in}{2.124855in}}%
\pgfpathlineto{\pgfqpoint{2.020857in}{2.116981in}}%
\pgfpathclose%
\pgfusepath{fill}%
\end{pgfscope}%
\begin{pgfscope}%
\pgfpathrectangle{\pgfqpoint{0.254231in}{0.147348in}}{\pgfqpoint{2.735294in}{2.735294in}}%
\pgfusepath{clip}%
\pgfsetbuttcap%
\pgfsetroundjoin%
\definecolor{currentfill}{rgb}{0.064954,0.249341,0.377155}%
\pgfsetfillcolor{currentfill}%
\pgfsetlinewidth{0.000000pt}%
\definecolor{currentstroke}{rgb}{0.000000,0.000000,0.000000}%
\pgfsetstrokecolor{currentstroke}%
\pgfsetdash{}{0pt}%
\pgfpathmoveto{\pgfqpoint{1.132195in}{1.562087in}}%
\pgfpathlineto{\pgfqpoint{1.091293in}{2.103280in}}%
\pgfpathlineto{\pgfqpoint{0.935572in}{1.589448in}}%
\pgfpathlineto{\pgfqpoint{1.132195in}{1.562087in}}%
\pgfpathclose%
\pgfusepath{fill}%
\end{pgfscope}%
\begin{pgfscope}%
\pgfpathrectangle{\pgfqpoint{0.254231in}{0.147348in}}{\pgfqpoint{2.735294in}{2.735294in}}%
\pgfusepath{clip}%
\pgfsetbuttcap%
\pgfsetroundjoin%
\definecolor{currentfill}{rgb}{0.064954,0.249341,0.377155}%
\pgfsetfillcolor{currentfill}%
\pgfsetlinewidth{0.000000pt}%
\definecolor{currentstroke}{rgb}{0.000000,0.000000,0.000000}%
\pgfsetstrokecolor{currentstroke}%
\pgfsetdash{}{0pt}%
\pgfpathmoveto{\pgfqpoint{2.382112in}{1.589448in}}%
\pgfpathlineto{\pgfqpoint{2.226391in}{2.103280in}}%
\pgfpathlineto{\pgfqpoint{2.185489in}{1.562087in}}%
\pgfpathlineto{\pgfqpoint{2.382112in}{1.589448in}}%
\pgfpathclose%
\pgfusepath{fill}%
\end{pgfscope}%
\begin{pgfscope}%
\pgfpathrectangle{\pgfqpoint{0.254231in}{0.147348in}}{\pgfqpoint{2.735294in}{2.735294in}}%
\pgfusepath{clip}%
\pgfsetbuttcap%
\pgfsetroundjoin%
\definecolor{currentfill}{rgb}{0.042669,0.163794,0.247755}%
\pgfsetfillcolor{currentfill}%
\pgfsetlinewidth{0.000000pt}%
\definecolor{currentstroke}{rgb}{0.000000,0.000000,0.000000}%
\pgfsetstrokecolor{currentstroke}%
\pgfsetdash{}{0pt}%
\pgfpathmoveto{\pgfqpoint{1.432345in}{0.946905in}}%
\pgfpathlineto{\pgfqpoint{1.518932in}{1.337788in}}%
\pgfpathlineto{\pgfqpoint{1.297586in}{1.063020in}}%
\pgfpathlineto{\pgfqpoint{1.432345in}{0.946905in}}%
\pgfpathclose%
\pgfusepath{fill}%
\end{pgfscope}%
\begin{pgfscope}%
\pgfpathrectangle{\pgfqpoint{0.254231in}{0.147348in}}{\pgfqpoint{2.735294in}{2.735294in}}%
\pgfusepath{clip}%
\pgfsetbuttcap%
\pgfsetroundjoin%
\definecolor{currentfill}{rgb}{0.042669,0.163794,0.247755}%
\pgfsetfillcolor{currentfill}%
\pgfsetlinewidth{0.000000pt}%
\definecolor{currentstroke}{rgb}{0.000000,0.000000,0.000000}%
\pgfsetstrokecolor{currentstroke}%
\pgfsetdash{}{0pt}%
\pgfpathmoveto{\pgfqpoint{2.020098in}{1.063020in}}%
\pgfpathlineto{\pgfqpoint{1.798752in}{1.337788in}}%
\pgfpathlineto{\pgfqpoint{1.885339in}{0.946905in}}%
\pgfpathlineto{\pgfqpoint{2.020098in}{1.063020in}}%
\pgfpathclose%
\pgfusepath{fill}%
\end{pgfscope}%
\begin{pgfscope}%
\pgfpathrectangle{\pgfqpoint{0.254231in}{0.147348in}}{\pgfqpoint{2.735294in}{2.735294in}}%
\pgfusepath{clip}%
\pgfsetbuttcap%
\pgfsetroundjoin%
\definecolor{currentfill}{rgb}{0.068541,0.263111,0.397982}%
\pgfsetfillcolor{currentfill}%
\pgfsetlinewidth{0.000000pt}%
\definecolor{currentstroke}{rgb}{0.000000,0.000000,0.000000}%
\pgfsetstrokecolor{currentstroke}%
\pgfsetdash{}{0pt}%
\pgfpathmoveto{\pgfqpoint{1.296827in}{2.116981in}}%
\pgfpathlineto{\pgfqpoint{1.091293in}{2.103280in}}%
\pgfpathlineto{\pgfqpoint{1.395317in}{1.946079in}}%
\pgfpathlineto{\pgfqpoint{1.296827in}{2.116981in}}%
\pgfpathclose%
\pgfusepath{fill}%
\end{pgfscope}%
\begin{pgfscope}%
\pgfpathrectangle{\pgfqpoint{0.254231in}{0.147348in}}{\pgfqpoint{2.735294in}{2.735294in}}%
\pgfusepath{clip}%
\pgfsetbuttcap%
\pgfsetroundjoin%
\definecolor{currentfill}{rgb}{0.068541,0.263111,0.397982}%
\pgfsetfillcolor{currentfill}%
\pgfsetlinewidth{0.000000pt}%
\definecolor{currentstroke}{rgb}{0.000000,0.000000,0.000000}%
\pgfsetstrokecolor{currentstroke}%
\pgfsetdash{}{0pt}%
\pgfpathmoveto{\pgfqpoint{1.922367in}{1.946079in}}%
\pgfpathlineto{\pgfqpoint{2.226391in}{2.103280in}}%
\pgfpathlineto{\pgfqpoint{2.020857in}{2.116981in}}%
\pgfpathlineto{\pgfqpoint{1.922367in}{1.946079in}}%
\pgfpathclose%
\pgfusepath{fill}%
\end{pgfscope}%
\begin{pgfscope}%
\pgfpathrectangle{\pgfqpoint{0.254231in}{0.147348in}}{\pgfqpoint{2.735294in}{2.735294in}}%
\pgfusepath{clip}%
\pgfsetbuttcap%
\pgfsetroundjoin%
\definecolor{currentfill}{rgb}{0.047555,0.182548,0.276123}%
\pgfsetfillcolor{currentfill}%
\pgfsetlinewidth{0.000000pt}%
\definecolor{currentstroke}{rgb}{0.000000,0.000000,0.000000}%
\pgfsetstrokecolor{currentstroke}%
\pgfsetdash{}{0pt}%
\pgfpathmoveto{\pgfqpoint{1.432345in}{0.946905in}}%
\pgfpathlineto{\pgfqpoint{1.534325in}{1.035006in}}%
\pgfpathlineto{\pgfqpoint{1.518932in}{1.337788in}}%
\pgfpathlineto{\pgfqpoint{1.432345in}{0.946905in}}%
\pgfpathclose%
\pgfusepath{fill}%
\end{pgfscope}%
\begin{pgfscope}%
\pgfpathrectangle{\pgfqpoint{0.254231in}{0.147348in}}{\pgfqpoint{2.735294in}{2.735294in}}%
\pgfusepath{clip}%
\pgfsetbuttcap%
\pgfsetroundjoin%
\definecolor{currentfill}{rgb}{0.047555,0.182548,0.276123}%
\pgfsetfillcolor{currentfill}%
\pgfsetlinewidth{0.000000pt}%
\definecolor{currentstroke}{rgb}{0.000000,0.000000,0.000000}%
\pgfsetstrokecolor{currentstroke}%
\pgfsetdash{}{0pt}%
\pgfpathmoveto{\pgfqpoint{1.798752in}{1.337788in}}%
\pgfpathlineto{\pgfqpoint{1.783359in}{1.035006in}}%
\pgfpathlineto{\pgfqpoint{1.885339in}{0.946905in}}%
\pgfpathlineto{\pgfqpoint{1.798752in}{1.337788in}}%
\pgfpathclose%
\pgfusepath{fill}%
\end{pgfscope}%
\begin{pgfscope}%
\pgfpathrectangle{\pgfqpoint{0.254231in}{0.147348in}}{\pgfqpoint{2.735294in}{2.735294in}}%
\pgfusepath{clip}%
\pgfsetbuttcap%
\pgfsetroundjoin%
\definecolor{currentfill}{rgb}{0.046101,0.176968,0.267683}%
\pgfsetfillcolor{currentfill}%
\pgfsetlinewidth{0.000000pt}%
\definecolor{currentstroke}{rgb}{0.000000,0.000000,0.000000}%
\pgfsetstrokecolor{currentstroke}%
\pgfsetdash{}{0pt}%
\pgfpathmoveto{\pgfqpoint{1.658842in}{0.933095in}}%
\pgfpathlineto{\pgfqpoint{1.518932in}{1.337788in}}%
\pgfpathlineto{\pgfqpoint{1.534325in}{1.035006in}}%
\pgfpathlineto{\pgfqpoint{1.658842in}{0.933095in}}%
\pgfpathclose%
\pgfusepath{fill}%
\end{pgfscope}%
\begin{pgfscope}%
\pgfpathrectangle{\pgfqpoint{0.254231in}{0.147348in}}{\pgfqpoint{2.735294in}{2.735294in}}%
\pgfusepath{clip}%
\pgfsetbuttcap%
\pgfsetroundjoin%
\definecolor{currentfill}{rgb}{0.046101,0.176968,0.267683}%
\pgfsetfillcolor{currentfill}%
\pgfsetlinewidth{0.000000pt}%
\definecolor{currentstroke}{rgb}{0.000000,0.000000,0.000000}%
\pgfsetstrokecolor{currentstroke}%
\pgfsetdash{}{0pt}%
\pgfpathmoveto{\pgfqpoint{1.783359in}{1.035006in}}%
\pgfpathlineto{\pgfqpoint{1.798752in}{1.337788in}}%
\pgfpathlineto{\pgfqpoint{1.658842in}{0.933095in}}%
\pgfpathlineto{\pgfqpoint{1.783359in}{1.035006in}}%
\pgfpathclose%
\pgfusepath{fill}%
\end{pgfscope}%
\begin{pgfscope}%
\pgfpathrectangle{\pgfqpoint{0.254231in}{0.147348in}}{\pgfqpoint{2.735294in}{2.735294in}}%
\pgfusepath{clip}%
\pgfsetbuttcap%
\pgfsetroundjoin%
\definecolor{currentfill}{rgb}{0.051850,0.199036,0.301063}%
\pgfsetfillcolor{currentfill}%
\pgfsetlinewidth{0.000000pt}%
\definecolor{currentstroke}{rgb}{0.000000,0.000000,0.000000}%
\pgfsetstrokecolor{currentstroke}%
\pgfsetdash{}{0pt}%
\pgfpathmoveto{\pgfqpoint{1.092413in}{1.111775in}}%
\pgfpathlineto{\pgfqpoint{1.256196in}{1.360106in}}%
\pgfpathlineto{\pgfqpoint{1.132195in}{1.562087in}}%
\pgfpathlineto{\pgfqpoint{1.092413in}{1.111775in}}%
\pgfpathclose%
\pgfusepath{fill}%
\end{pgfscope}%
\begin{pgfscope}%
\pgfpathrectangle{\pgfqpoint{0.254231in}{0.147348in}}{\pgfqpoint{2.735294in}{2.735294in}}%
\pgfusepath{clip}%
\pgfsetbuttcap%
\pgfsetroundjoin%
\definecolor{currentfill}{rgb}{0.051850,0.199036,0.301063}%
\pgfsetfillcolor{currentfill}%
\pgfsetlinewidth{0.000000pt}%
\definecolor{currentstroke}{rgb}{0.000000,0.000000,0.000000}%
\pgfsetstrokecolor{currentstroke}%
\pgfsetdash{}{0pt}%
\pgfpathmoveto{\pgfqpoint{2.185489in}{1.562087in}}%
\pgfpathlineto{\pgfqpoint{2.061488in}{1.360106in}}%
\pgfpathlineto{\pgfqpoint{2.225271in}{1.111775in}}%
\pgfpathlineto{\pgfqpoint{2.185489in}{1.562087in}}%
\pgfpathclose%
\pgfusepath{fill}%
\end{pgfscope}%
\begin{pgfscope}%
\pgfpathrectangle{\pgfqpoint{0.254231in}{0.147348in}}{\pgfqpoint{2.735294in}{2.735294in}}%
\pgfusepath{clip}%
\pgfsetbuttcap%
\pgfsetroundjoin%
\definecolor{currentfill}{rgb}{0.064759,0.248590,0.376018}%
\pgfsetfillcolor{currentfill}%
\pgfsetlinewidth{0.000000pt}%
\definecolor{currentstroke}{rgb}{0.000000,0.000000,0.000000}%
\pgfsetstrokecolor{currentstroke}%
\pgfsetdash{}{0pt}%
\pgfpathmoveto{\pgfqpoint{2.226391in}{2.103280in}}%
\pgfpathlineto{\pgfqpoint{1.922367in}{1.946079in}}%
\pgfpathlineto{\pgfqpoint{2.185489in}{1.562087in}}%
\pgfpathlineto{\pgfqpoint{2.226391in}{2.103280in}}%
\pgfpathclose%
\pgfusepath{fill}%
\end{pgfscope}%
\begin{pgfscope}%
\pgfpathrectangle{\pgfqpoint{0.254231in}{0.147348in}}{\pgfqpoint{2.735294in}{2.735294in}}%
\pgfusepath{clip}%
\pgfsetbuttcap%
\pgfsetroundjoin%
\definecolor{currentfill}{rgb}{0.064759,0.248590,0.376018}%
\pgfsetfillcolor{currentfill}%
\pgfsetlinewidth{0.000000pt}%
\definecolor{currentstroke}{rgb}{0.000000,0.000000,0.000000}%
\pgfsetstrokecolor{currentstroke}%
\pgfsetdash{}{0pt}%
\pgfpathmoveto{\pgfqpoint{1.132195in}{1.562087in}}%
\pgfpathlineto{\pgfqpoint{1.395317in}{1.946079in}}%
\pgfpathlineto{\pgfqpoint{1.091293in}{2.103280in}}%
\pgfpathlineto{\pgfqpoint{1.132195in}{1.562087in}}%
\pgfpathclose%
\pgfusepath{fill}%
\end{pgfscope}%
\begin{pgfscope}%
\pgfpathrectangle{\pgfqpoint{0.254231in}{0.147348in}}{\pgfqpoint{2.735294in}{2.735294in}}%
\pgfusepath{clip}%
\pgfsetbuttcap%
\pgfsetroundjoin%
\definecolor{currentfill}{rgb}{0.071694,0.275212,0.416288}%
\pgfsetfillcolor{currentfill}%
\pgfsetlinewidth{0.000000pt}%
\definecolor{currentstroke}{rgb}{0.000000,0.000000,0.000000}%
\pgfsetstrokecolor{currentstroke}%
\pgfsetdash{}{0pt}%
\pgfpathmoveto{\pgfqpoint{1.395317in}{1.946079in}}%
\pgfpathlineto{\pgfqpoint{1.534055in}{2.124855in}}%
\pgfpathlineto{\pgfqpoint{1.296827in}{2.116981in}}%
\pgfpathlineto{\pgfqpoint{1.395317in}{1.946079in}}%
\pgfpathclose%
\pgfusepath{fill}%
\end{pgfscope}%
\begin{pgfscope}%
\pgfpathrectangle{\pgfqpoint{0.254231in}{0.147348in}}{\pgfqpoint{2.735294in}{2.735294in}}%
\pgfusepath{clip}%
\pgfsetbuttcap%
\pgfsetroundjoin%
\definecolor{currentfill}{rgb}{0.071694,0.275212,0.416288}%
\pgfsetfillcolor{currentfill}%
\pgfsetlinewidth{0.000000pt}%
\definecolor{currentstroke}{rgb}{0.000000,0.000000,0.000000}%
\pgfsetstrokecolor{currentstroke}%
\pgfsetdash{}{0pt}%
\pgfpathmoveto{\pgfqpoint{2.020857in}{2.116981in}}%
\pgfpathlineto{\pgfqpoint{1.783629in}{2.124855in}}%
\pgfpathlineto{\pgfqpoint{1.922367in}{1.946079in}}%
\pgfpathlineto{\pgfqpoint{2.020857in}{2.116981in}}%
\pgfpathclose%
\pgfusepath{fill}%
\end{pgfscope}%
\begin{pgfscope}%
\pgfpathrectangle{\pgfqpoint{0.254231in}{0.147348in}}{\pgfqpoint{2.735294in}{2.735294in}}%
\pgfusepath{clip}%
\pgfsetbuttcap%
\pgfsetroundjoin%
\definecolor{currentfill}{rgb}{0.071636,0.274990,0.415951}%
\pgfsetfillcolor{currentfill}%
\pgfsetlinewidth{0.000000pt}%
\definecolor{currentstroke}{rgb}{0.000000,0.000000,0.000000}%
\pgfsetstrokecolor{currentstroke}%
\pgfsetdash{}{0pt}%
\pgfpathmoveto{\pgfqpoint{1.658842in}{1.947266in}}%
\pgfpathlineto{\pgfqpoint{1.783629in}{2.124855in}}%
\pgfpathlineto{\pgfqpoint{1.534055in}{2.124855in}}%
\pgfpathlineto{\pgfqpoint{1.658842in}{1.947266in}}%
\pgfpathclose%
\pgfusepath{fill}%
\end{pgfscope}%
\begin{pgfscope}%
\pgfpathrectangle{\pgfqpoint{0.254231in}{0.147348in}}{\pgfqpoint{2.735294in}{2.735294in}}%
\pgfusepath{clip}%
\pgfsetbuttcap%
\pgfsetroundjoin%
\definecolor{currentfill}{rgb}{0.045820,0.175891,0.266053}%
\pgfsetfillcolor{currentfill}%
\pgfsetlinewidth{0.000000pt}%
\definecolor{currentstroke}{rgb}{0.000000,0.000000,0.000000}%
\pgfsetstrokecolor{currentstroke}%
\pgfsetdash{}{0pt}%
\pgfpathmoveto{\pgfqpoint{1.297586in}{1.063020in}}%
\pgfpathlineto{\pgfqpoint{1.378909in}{1.541647in}}%
\pgfpathlineto{\pgfqpoint{1.256196in}{1.360106in}}%
\pgfpathlineto{\pgfqpoint{1.297586in}{1.063020in}}%
\pgfpathclose%
\pgfusepath{fill}%
\end{pgfscope}%
\begin{pgfscope}%
\pgfpathrectangle{\pgfqpoint{0.254231in}{0.147348in}}{\pgfqpoint{2.735294in}{2.735294in}}%
\pgfusepath{clip}%
\pgfsetbuttcap%
\pgfsetroundjoin%
\definecolor{currentfill}{rgb}{0.045820,0.175891,0.266053}%
\pgfsetfillcolor{currentfill}%
\pgfsetlinewidth{0.000000pt}%
\definecolor{currentstroke}{rgb}{0.000000,0.000000,0.000000}%
\pgfsetstrokecolor{currentstroke}%
\pgfsetdash{}{0pt}%
\pgfpathmoveto{\pgfqpoint{2.061488in}{1.360106in}}%
\pgfpathlineto{\pgfqpoint{1.938775in}{1.541647in}}%
\pgfpathlineto{\pgfqpoint{2.020098in}{1.063020in}}%
\pgfpathlineto{\pgfqpoint{2.061488in}{1.360106in}}%
\pgfpathclose%
\pgfusepath{fill}%
\end{pgfscope}%
\begin{pgfscope}%
\pgfpathrectangle{\pgfqpoint{0.254231in}{0.147348in}}{\pgfqpoint{2.735294in}{2.735294in}}%
\pgfusepath{clip}%
\pgfsetbuttcap%
\pgfsetroundjoin%
\definecolor{currentfill}{rgb}{0.046814,0.179706,0.271825}%
\pgfsetfillcolor{currentfill}%
\pgfsetlinewidth{0.000000pt}%
\definecolor{currentstroke}{rgb}{0.000000,0.000000,0.000000}%
\pgfsetstrokecolor{currentstroke}%
\pgfsetdash{}{0pt}%
\pgfpathmoveto{\pgfqpoint{1.658842in}{1.533948in}}%
\pgfpathlineto{\pgfqpoint{1.658842in}{0.933095in}}%
\pgfpathlineto{\pgfqpoint{1.798752in}{1.337788in}}%
\pgfpathlineto{\pgfqpoint{1.658842in}{1.533948in}}%
\pgfpathclose%
\pgfusepath{fill}%
\end{pgfscope}%
\begin{pgfscope}%
\pgfpathrectangle{\pgfqpoint{0.254231in}{0.147348in}}{\pgfqpoint{2.735294in}{2.735294in}}%
\pgfusepath{clip}%
\pgfsetbuttcap%
\pgfsetroundjoin%
\definecolor{currentfill}{rgb}{0.046814,0.179706,0.271825}%
\pgfsetfillcolor{currentfill}%
\pgfsetlinewidth{0.000000pt}%
\definecolor{currentstroke}{rgb}{0.000000,0.000000,0.000000}%
\pgfsetstrokecolor{currentstroke}%
\pgfsetdash{}{0pt}%
\pgfpathmoveto{\pgfqpoint{1.518932in}{1.337788in}}%
\pgfpathlineto{\pgfqpoint{1.658842in}{0.933095in}}%
\pgfpathlineto{\pgfqpoint{1.658842in}{1.533948in}}%
\pgfpathlineto{\pgfqpoint{1.518932in}{1.337788in}}%
\pgfpathclose%
\pgfusepath{fill}%
\end{pgfscope}%
\begin{pgfscope}%
\pgfpathrectangle{\pgfqpoint{0.254231in}{0.147348in}}{\pgfqpoint{2.735294in}{2.735294in}}%
\pgfusepath{clip}%
\pgfsetbuttcap%
\pgfsetroundjoin%
\definecolor{currentfill}{rgb}{0.069261,0.265872,0.402159}%
\pgfsetfillcolor{currentfill}%
\pgfsetlinewidth{0.000000pt}%
\definecolor{currentstroke}{rgb}{0.000000,0.000000,0.000000}%
\pgfsetstrokecolor{currentstroke}%
\pgfsetdash{}{0pt}%
\pgfpathmoveto{\pgfqpoint{1.922367in}{1.946079in}}%
\pgfpathlineto{\pgfqpoint{1.783629in}{2.124855in}}%
\pgfpathlineto{\pgfqpoint{1.658842in}{1.947266in}}%
\pgfpathlineto{\pgfqpoint{1.922367in}{1.946079in}}%
\pgfpathclose%
\pgfusepath{fill}%
\end{pgfscope}%
\begin{pgfscope}%
\pgfpathrectangle{\pgfqpoint{0.254231in}{0.147348in}}{\pgfqpoint{2.735294in}{2.735294in}}%
\pgfusepath{clip}%
\pgfsetbuttcap%
\pgfsetroundjoin%
\definecolor{currentfill}{rgb}{0.069261,0.265872,0.402159}%
\pgfsetfillcolor{currentfill}%
\pgfsetlinewidth{0.000000pt}%
\definecolor{currentstroke}{rgb}{0.000000,0.000000,0.000000}%
\pgfsetstrokecolor{currentstroke}%
\pgfsetdash{}{0pt}%
\pgfpathmoveto{\pgfqpoint{1.658842in}{1.947266in}}%
\pgfpathlineto{\pgfqpoint{1.534055in}{2.124855in}}%
\pgfpathlineto{\pgfqpoint{1.395317in}{1.946079in}}%
\pgfpathlineto{\pgfqpoint{1.658842in}{1.947266in}}%
\pgfpathclose%
\pgfusepath{fill}%
\end{pgfscope}%
\begin{pgfscope}%
\pgfpathrectangle{\pgfqpoint{0.254231in}{0.147348in}}{\pgfqpoint{2.735294in}{2.735294in}}%
\pgfusepath{clip}%
\pgfsetbuttcap%
\pgfsetroundjoin%
\definecolor{currentfill}{rgb}{0.049465,0.189883,0.287218}%
\pgfsetfillcolor{currentfill}%
\pgfsetlinewidth{0.000000pt}%
\definecolor{currentstroke}{rgb}{0.000000,0.000000,0.000000}%
\pgfsetstrokecolor{currentstroke}%
\pgfsetdash{}{0pt}%
\pgfpathmoveto{\pgfqpoint{1.297586in}{1.063020in}}%
\pgfpathlineto{\pgfqpoint{1.518932in}{1.337788in}}%
\pgfpathlineto{\pgfqpoint{1.378909in}{1.541647in}}%
\pgfpathlineto{\pgfqpoint{1.297586in}{1.063020in}}%
\pgfpathclose%
\pgfusepath{fill}%
\end{pgfscope}%
\begin{pgfscope}%
\pgfpathrectangle{\pgfqpoint{0.254231in}{0.147348in}}{\pgfqpoint{2.735294in}{2.735294in}}%
\pgfusepath{clip}%
\pgfsetbuttcap%
\pgfsetroundjoin%
\definecolor{currentfill}{rgb}{0.049465,0.189883,0.287218}%
\pgfsetfillcolor{currentfill}%
\pgfsetlinewidth{0.000000pt}%
\definecolor{currentstroke}{rgb}{0.000000,0.000000,0.000000}%
\pgfsetstrokecolor{currentstroke}%
\pgfsetdash{}{0pt}%
\pgfpathmoveto{\pgfqpoint{1.938775in}{1.541647in}}%
\pgfpathlineto{\pgfqpoint{1.798752in}{1.337788in}}%
\pgfpathlineto{\pgfqpoint{2.020098in}{1.063020in}}%
\pgfpathlineto{\pgfqpoint{1.938775in}{1.541647in}}%
\pgfpathclose%
\pgfusepath{fill}%
\end{pgfscope}%
\begin{pgfscope}%
\pgfpathrectangle{\pgfqpoint{0.254231in}{0.147348in}}{\pgfqpoint{2.735294in}{2.735294in}}%
\pgfusepath{clip}%
\pgfsetbuttcap%
\pgfsetroundjoin%
\definecolor{currentfill}{rgb}{0.061576,0.236373,0.357539}%
\pgfsetfillcolor{currentfill}%
\pgfsetlinewidth{0.000000pt}%
\definecolor{currentstroke}{rgb}{0.000000,0.000000,0.000000}%
\pgfsetstrokecolor{currentstroke}%
\pgfsetdash{}{0pt}%
\pgfpathmoveto{\pgfqpoint{1.378909in}{1.541647in}}%
\pgfpathlineto{\pgfqpoint{1.395317in}{1.946079in}}%
\pgfpathlineto{\pgfqpoint{1.132195in}{1.562087in}}%
\pgfpathlineto{\pgfqpoint{1.378909in}{1.541647in}}%
\pgfpathclose%
\pgfusepath{fill}%
\end{pgfscope}%
\begin{pgfscope}%
\pgfpathrectangle{\pgfqpoint{0.254231in}{0.147348in}}{\pgfqpoint{2.735294in}{2.735294in}}%
\pgfusepath{clip}%
\pgfsetbuttcap%
\pgfsetroundjoin%
\definecolor{currentfill}{rgb}{0.061576,0.236373,0.357539}%
\pgfsetfillcolor{currentfill}%
\pgfsetlinewidth{0.000000pt}%
\definecolor{currentstroke}{rgb}{0.000000,0.000000,0.000000}%
\pgfsetstrokecolor{currentstroke}%
\pgfsetdash{}{0pt}%
\pgfpathmoveto{\pgfqpoint{2.185489in}{1.562087in}}%
\pgfpathlineto{\pgfqpoint{1.922367in}{1.946079in}}%
\pgfpathlineto{\pgfqpoint{1.938775in}{1.541647in}}%
\pgfpathlineto{\pgfqpoint{2.185489in}{1.562087in}}%
\pgfpathclose%
\pgfusepath{fill}%
\end{pgfscope}%
\begin{pgfscope}%
\pgfpathrectangle{\pgfqpoint{0.254231in}{0.147348in}}{\pgfqpoint{2.735294in}{2.735294in}}%
\pgfusepath{clip}%
\pgfsetbuttcap%
\pgfsetroundjoin%
\definecolor{currentfill}{rgb}{0.053541,0.205528,0.310883}%
\pgfsetfillcolor{currentfill}%
\pgfsetlinewidth{0.000000pt}%
\definecolor{currentstroke}{rgb}{0.000000,0.000000,0.000000}%
\pgfsetstrokecolor{currentstroke}%
\pgfsetdash{}{0pt}%
\pgfpathmoveto{\pgfqpoint{1.132195in}{1.562087in}}%
\pgfpathlineto{\pgfqpoint{1.256196in}{1.360106in}}%
\pgfpathlineto{\pgfqpoint{1.378909in}{1.541647in}}%
\pgfpathlineto{\pgfqpoint{1.132195in}{1.562087in}}%
\pgfpathclose%
\pgfusepath{fill}%
\end{pgfscope}%
\begin{pgfscope}%
\pgfpathrectangle{\pgfqpoint{0.254231in}{0.147348in}}{\pgfqpoint{2.735294in}{2.735294in}}%
\pgfusepath{clip}%
\pgfsetbuttcap%
\pgfsetroundjoin%
\definecolor{currentfill}{rgb}{0.053541,0.205528,0.310883}%
\pgfsetfillcolor{currentfill}%
\pgfsetlinewidth{0.000000pt}%
\definecolor{currentstroke}{rgb}{0.000000,0.000000,0.000000}%
\pgfsetstrokecolor{currentstroke}%
\pgfsetdash{}{0pt}%
\pgfpathmoveto{\pgfqpoint{1.938775in}{1.541647in}}%
\pgfpathlineto{\pgfqpoint{2.061488in}{1.360106in}}%
\pgfpathlineto{\pgfqpoint{2.185489in}{1.562087in}}%
\pgfpathlineto{\pgfqpoint{1.938775in}{1.541647in}}%
\pgfpathclose%
\pgfusepath{fill}%
\end{pgfscope}%
\begin{pgfscope}%
\pgfpathrectangle{\pgfqpoint{0.254231in}{0.147348in}}{\pgfqpoint{2.735294in}{2.735294in}}%
\pgfusepath{clip}%
\pgfsetbuttcap%
\pgfsetroundjoin%
\definecolor{currentfill}{rgb}{0.060634,0.232757,0.352069}%
\pgfsetfillcolor{currentfill}%
\pgfsetlinewidth{0.000000pt}%
\definecolor{currentstroke}{rgb}{0.000000,0.000000,0.000000}%
\pgfsetstrokecolor{currentstroke}%
\pgfsetdash{}{0pt}%
\pgfpathmoveto{\pgfqpoint{1.395317in}{1.946079in}}%
\pgfpathlineto{\pgfqpoint{1.378909in}{1.541647in}}%
\pgfpathlineto{\pgfqpoint{1.658842in}{1.947266in}}%
\pgfpathlineto{\pgfqpoint{1.395317in}{1.946079in}}%
\pgfpathclose%
\pgfusepath{fill}%
\end{pgfscope}%
\begin{pgfscope}%
\pgfpathrectangle{\pgfqpoint{0.254231in}{0.147348in}}{\pgfqpoint{2.735294in}{2.735294in}}%
\pgfusepath{clip}%
\pgfsetbuttcap%
\pgfsetroundjoin%
\definecolor{currentfill}{rgb}{0.060634,0.232757,0.352069}%
\pgfsetfillcolor{currentfill}%
\pgfsetlinewidth{0.000000pt}%
\definecolor{currentstroke}{rgb}{0.000000,0.000000,0.000000}%
\pgfsetstrokecolor{currentstroke}%
\pgfsetdash{}{0pt}%
\pgfpathmoveto{\pgfqpoint{1.658842in}{1.947266in}}%
\pgfpathlineto{\pgfqpoint{1.938775in}{1.541647in}}%
\pgfpathlineto{\pgfqpoint{1.922367in}{1.946079in}}%
\pgfpathlineto{\pgfqpoint{1.658842in}{1.947266in}}%
\pgfpathclose%
\pgfusepath{fill}%
\end{pgfscope}%
\begin{pgfscope}%
\pgfpathrectangle{\pgfqpoint{0.254231in}{0.147348in}}{\pgfqpoint{2.735294in}{2.735294in}}%
\pgfusepath{clip}%
\pgfsetbuttcap%
\pgfsetroundjoin%
\definecolor{currentfill}{rgb}{0.060773,0.233289,0.352874}%
\pgfsetfillcolor{currentfill}%
\pgfsetlinewidth{0.000000pt}%
\definecolor{currentstroke}{rgb}{0.000000,0.000000,0.000000}%
\pgfsetstrokecolor{currentstroke}%
\pgfsetdash{}{0pt}%
\pgfpathmoveto{\pgfqpoint{1.658842in}{1.533948in}}%
\pgfpathlineto{\pgfqpoint{1.658842in}{1.947266in}}%
\pgfpathlineto{\pgfqpoint{1.378909in}{1.541647in}}%
\pgfpathlineto{\pgfqpoint{1.658842in}{1.533948in}}%
\pgfpathclose%
\pgfusepath{fill}%
\end{pgfscope}%
\begin{pgfscope}%
\pgfpathrectangle{\pgfqpoint{0.254231in}{0.147348in}}{\pgfqpoint{2.735294in}{2.735294in}}%
\pgfusepath{clip}%
\pgfsetbuttcap%
\pgfsetroundjoin%
\definecolor{currentfill}{rgb}{0.060773,0.233289,0.352874}%
\pgfsetfillcolor{currentfill}%
\pgfsetlinewidth{0.000000pt}%
\definecolor{currentstroke}{rgb}{0.000000,0.000000,0.000000}%
\pgfsetstrokecolor{currentstroke}%
\pgfsetdash{}{0pt}%
\pgfpathmoveto{\pgfqpoint{1.938775in}{1.541647in}}%
\pgfpathlineto{\pgfqpoint{1.658842in}{1.947266in}}%
\pgfpathlineto{\pgfqpoint{1.658842in}{1.533948in}}%
\pgfpathlineto{\pgfqpoint{1.938775in}{1.541647in}}%
\pgfpathclose%
\pgfusepath{fill}%
\end{pgfscope}%
\begin{pgfscope}%
\pgfpathrectangle{\pgfqpoint{0.254231in}{0.147348in}}{\pgfqpoint{2.735294in}{2.735294in}}%
\pgfusepath{clip}%
\pgfsetbuttcap%
\pgfsetroundjoin%
\definecolor{currentfill}{rgb}{0.052607,0.201942,0.305459}%
\pgfsetfillcolor{currentfill}%
\pgfsetlinewidth{0.000000pt}%
\definecolor{currentstroke}{rgb}{0.000000,0.000000,0.000000}%
\pgfsetstrokecolor{currentstroke}%
\pgfsetdash{}{0pt}%
\pgfpathmoveto{\pgfqpoint{1.378909in}{1.541647in}}%
\pgfpathlineto{\pgfqpoint{1.518932in}{1.337788in}}%
\pgfpathlineto{\pgfqpoint{1.658842in}{1.533948in}}%
\pgfpathlineto{\pgfqpoint{1.378909in}{1.541647in}}%
\pgfpathclose%
\pgfusepath{fill}%
\end{pgfscope}%
\begin{pgfscope}%
\pgfpathrectangle{\pgfqpoint{0.254231in}{0.147348in}}{\pgfqpoint{2.735294in}{2.735294in}}%
\pgfusepath{clip}%
\pgfsetbuttcap%
\pgfsetroundjoin%
\definecolor{currentfill}{rgb}{0.052607,0.201942,0.305459}%
\pgfsetfillcolor{currentfill}%
\pgfsetlinewidth{0.000000pt}%
\definecolor{currentstroke}{rgb}{0.000000,0.000000,0.000000}%
\pgfsetstrokecolor{currentstroke}%
\pgfsetdash{}{0pt}%
\pgfpathmoveto{\pgfqpoint{1.658842in}{1.533948in}}%
\pgfpathlineto{\pgfqpoint{1.798752in}{1.337788in}}%
\pgfpathlineto{\pgfqpoint{1.938775in}{1.541647in}}%
\pgfpathlineto{\pgfqpoint{1.658842in}{1.533948in}}%
\pgfpathclose%
\pgfusepath{fill}%
\end{pgfscope}%
\begin{pgfscope}%
\pgfpathrectangle{\pgfqpoint{0.254231in}{0.147348in}}{\pgfqpoint{2.735294in}{2.735294in}}%
\pgfusepath{clip}%
\pgfsetbuttcap%
\pgfsetroundjoin%
\definecolor{currentfill}{rgb}{0.839216,0.152941,0.156863}%
\pgfsetfillcolor{currentfill}%
\pgfsetfillopacity{0.300000}%
\pgfsetlinewidth{1.003750pt}%
\definecolor{currentstroke}{rgb}{0.839216,0.152941,0.156863}%
\pgfsetstrokecolor{currentstroke}%
\pgfsetstrokeopacity{0.300000}%
\pgfsetdash{}{0pt}%
\pgfpathmoveto{\pgfqpoint{1.065369in}{0.944929in}}%
\pgfpathcurveto{\pgfqpoint{1.075457in}{0.944929in}}{\pgfqpoint{1.085132in}{0.948937in}}{\pgfqpoint{1.092265in}{0.956070in}}%
\pgfpathcurveto{\pgfqpoint{1.099398in}{0.963203in}}{\pgfqpoint{1.103406in}{0.972878in}}{\pgfqpoint{1.103406in}{0.982966in}}%
\pgfpathcurveto{\pgfqpoint{1.103406in}{0.993053in}}{\pgfqpoint{1.099398in}{1.002729in}}{\pgfqpoint{1.092265in}{1.009861in}}%
\pgfpathcurveto{\pgfqpoint{1.085132in}{1.016994in}}{\pgfqpoint{1.075457in}{1.021002in}}{\pgfqpoint{1.065369in}{1.021002in}}%
\pgfpathcurveto{\pgfqpoint{1.055282in}{1.021002in}}{\pgfqpoint{1.045607in}{1.016994in}}{\pgfqpoint{1.038474in}{1.009861in}}%
\pgfpathcurveto{\pgfqpoint{1.031341in}{1.002729in}}{\pgfqpoint{1.027333in}{0.993053in}}{\pgfqpoint{1.027333in}{0.982966in}}%
\pgfpathcurveto{\pgfqpoint{1.027333in}{0.972878in}}{\pgfqpoint{1.031341in}{0.963203in}}{\pgfqpoint{1.038474in}{0.956070in}}%
\pgfpathcurveto{\pgfqpoint{1.045607in}{0.948937in}}{\pgfqpoint{1.055282in}{0.944929in}}{\pgfqpoint{1.065369in}{0.944929in}}%
\pgfpathlineto{\pgfqpoint{1.065369in}{0.944929in}}%
\pgfpathclose%
\pgfusepath{stroke,fill}%
\end{pgfscope}%
\begin{pgfscope}%
\pgfpathrectangle{\pgfqpoint{0.254231in}{0.147348in}}{\pgfqpoint{2.735294in}{2.735294in}}%
\pgfusepath{clip}%
\pgfsetbuttcap%
\pgfsetroundjoin%
\definecolor{currentfill}{rgb}{0.839216,0.152941,0.156863}%
\pgfsetfillcolor{currentfill}%
\pgfsetfillopacity{0.383610}%
\pgfsetlinewidth{1.003750pt}%
\definecolor{currentstroke}{rgb}{0.839216,0.152941,0.156863}%
\pgfsetstrokecolor{currentstroke}%
\pgfsetstrokeopacity{0.383610}%
\pgfsetdash{}{0pt}%
\pgfpathmoveto{\pgfqpoint{1.045506in}{1.009596in}}%
\pgfpathcurveto{\pgfqpoint{1.055594in}{1.009596in}}{\pgfqpoint{1.065269in}{1.013604in}}{\pgfqpoint{1.072402in}{1.020737in}}%
\pgfpathcurveto{\pgfqpoint{1.079535in}{1.027870in}}{\pgfqpoint{1.083543in}{1.037545in}}{\pgfqpoint{1.083543in}{1.047633in}}%
\pgfpathcurveto{\pgfqpoint{1.083543in}{1.057720in}}{\pgfqpoint{1.079535in}{1.067396in}}{\pgfqpoint{1.072402in}{1.074528in}}%
\pgfpathcurveto{\pgfqpoint{1.065269in}{1.081661in}}{\pgfqpoint{1.055594in}{1.085669in}}{\pgfqpoint{1.045506in}{1.085669in}}%
\pgfpathcurveto{\pgfqpoint{1.035419in}{1.085669in}}{\pgfqpoint{1.025743in}{1.081661in}}{\pgfqpoint{1.018610in}{1.074528in}}%
\pgfpathcurveto{\pgfqpoint{1.011478in}{1.067396in}}{\pgfqpoint{1.007470in}{1.057720in}}{\pgfqpoint{1.007470in}{1.047633in}}%
\pgfpathcurveto{\pgfqpoint{1.007470in}{1.037545in}}{\pgfqpoint{1.011478in}{1.027870in}}{\pgfqpoint{1.018610in}{1.020737in}}%
\pgfpathcurveto{\pgfqpoint{1.025743in}{1.013604in}}{\pgfqpoint{1.035419in}{1.009596in}}{\pgfqpoint{1.045506in}{1.009596in}}%
\pgfpathlineto{\pgfqpoint{1.045506in}{1.009596in}}%
\pgfpathclose%
\pgfusepath{stroke,fill}%
\end{pgfscope}%
\begin{pgfscope}%
\pgfpathrectangle{\pgfqpoint{0.254231in}{0.147348in}}{\pgfqpoint{2.735294in}{2.735294in}}%
\pgfusepath{clip}%
\pgfsetbuttcap%
\pgfsetroundjoin%
\definecolor{currentfill}{rgb}{0.839216,0.152941,0.156863}%
\pgfsetfillcolor{currentfill}%
\pgfsetfillopacity{0.457533}%
\pgfsetlinewidth{1.003750pt}%
\definecolor{currentstroke}{rgb}{0.839216,0.152941,0.156863}%
\pgfsetstrokecolor{currentstroke}%
\pgfsetstrokeopacity{0.457533}%
\pgfsetdash{}{0pt}%
\pgfpathmoveto{\pgfqpoint{1.201522in}{0.962534in}}%
\pgfpathcurveto{\pgfqpoint{1.211609in}{0.962534in}}{\pgfqpoint{1.221285in}{0.966542in}}{\pgfqpoint{1.228417in}{0.973675in}}%
\pgfpathcurveto{\pgfqpoint{1.235550in}{0.980808in}}{\pgfqpoint{1.239558in}{0.990483in}}{\pgfqpoint{1.239558in}{1.000570in}}%
\pgfpathcurveto{\pgfqpoint{1.239558in}{1.010658in}}{\pgfqpoint{1.235550in}{1.020333in}}{\pgfqpoint{1.228417in}{1.027466in}}%
\pgfpathcurveto{\pgfqpoint{1.221285in}{1.034599in}}{\pgfqpoint{1.211609in}{1.038607in}}{\pgfqpoint{1.201522in}{1.038607in}}%
\pgfpathcurveto{\pgfqpoint{1.191434in}{1.038607in}}{\pgfqpoint{1.181759in}{1.034599in}}{\pgfqpoint{1.174626in}{1.027466in}}%
\pgfpathcurveto{\pgfqpoint{1.167493in}{1.020333in}}{\pgfqpoint{1.163485in}{1.010658in}}{\pgfqpoint{1.163485in}{1.000570in}}%
\pgfpathcurveto{\pgfqpoint{1.163485in}{0.990483in}}{\pgfqpoint{1.167493in}{0.980808in}}{\pgfqpoint{1.174626in}{0.973675in}}%
\pgfpathcurveto{\pgfqpoint{1.181759in}{0.966542in}}{\pgfqpoint{1.191434in}{0.962534in}}{\pgfqpoint{1.201522in}{0.962534in}}%
\pgfpathlineto{\pgfqpoint{1.201522in}{0.962534in}}%
\pgfpathclose%
\pgfusepath{stroke,fill}%
\end{pgfscope}%
\begin{pgfscope}%
\pgfpathrectangle{\pgfqpoint{0.254231in}{0.147348in}}{\pgfqpoint{2.735294in}{2.735294in}}%
\pgfusepath{clip}%
\pgfsetbuttcap%
\pgfsetroundjoin%
\definecolor{currentfill}{rgb}{0.839216,0.152941,0.156863}%
\pgfsetfillcolor{currentfill}%
\pgfsetfillopacity{0.492303}%
\pgfsetlinewidth{1.003750pt}%
\definecolor{currentstroke}{rgb}{0.839216,0.152941,0.156863}%
\pgfsetstrokecolor{currentstroke}%
\pgfsetstrokeopacity{0.492303}%
\pgfsetdash{}{0pt}%
\pgfpathmoveto{\pgfqpoint{0.878198in}{1.511653in}}%
\pgfpathcurveto{\pgfqpoint{0.888285in}{1.511653in}}{\pgfqpoint{0.897960in}{1.515661in}}{\pgfqpoint{0.905093in}{1.522794in}}%
\pgfpathcurveto{\pgfqpoint{0.912226in}{1.529927in}}{\pgfqpoint{0.916234in}{1.539602in}}{\pgfqpoint{0.916234in}{1.549689in}}%
\pgfpathcurveto{\pgfqpoint{0.916234in}{1.559777in}}{\pgfqpoint{0.912226in}{1.569452in}}{\pgfqpoint{0.905093in}{1.576585in}}%
\pgfpathcurveto{\pgfqpoint{0.897960in}{1.583718in}}{\pgfqpoint{0.888285in}{1.587726in}}{\pgfqpoint{0.878198in}{1.587726in}}%
\pgfpathcurveto{\pgfqpoint{0.868110in}{1.587726in}}{\pgfqpoint{0.858435in}{1.583718in}}{\pgfqpoint{0.851302in}{1.576585in}}%
\pgfpathcurveto{\pgfqpoint{0.844169in}{1.569452in}}{\pgfqpoint{0.840161in}{1.559777in}}{\pgfqpoint{0.840161in}{1.549689in}}%
\pgfpathcurveto{\pgfqpoint{0.840161in}{1.539602in}}{\pgfqpoint{0.844169in}{1.529927in}}{\pgfqpoint{0.851302in}{1.522794in}}%
\pgfpathcurveto{\pgfqpoint{0.858435in}{1.515661in}}{\pgfqpoint{0.868110in}{1.511653in}}{\pgfqpoint{0.878198in}{1.511653in}}%
\pgfpathlineto{\pgfqpoint{0.878198in}{1.511653in}}%
\pgfpathclose%
\pgfusepath{stroke,fill}%
\end{pgfscope}%
\begin{pgfscope}%
\pgfpathrectangle{\pgfqpoint{0.254231in}{0.147348in}}{\pgfqpoint{2.735294in}{2.735294in}}%
\pgfusepath{clip}%
\pgfsetbuttcap%
\pgfsetroundjoin%
\definecolor{currentfill}{rgb}{0.839216,0.152941,0.156863}%
\pgfsetfillcolor{currentfill}%
\pgfsetfillopacity{0.498590}%
\pgfsetlinewidth{1.003750pt}%
\definecolor{currentstroke}{rgb}{0.839216,0.152941,0.156863}%
\pgfsetstrokecolor{currentstroke}%
\pgfsetstrokeopacity{0.498590}%
\pgfsetdash{}{0pt}%
\pgfpathmoveto{\pgfqpoint{2.525473in}{1.621636in}}%
\pgfpathcurveto{\pgfqpoint{2.535560in}{1.621636in}}{\pgfqpoint{2.545236in}{1.625643in}}{\pgfqpoint{2.552368in}{1.632776in}}%
\pgfpathcurveto{\pgfqpoint{2.559501in}{1.639909in}}{\pgfqpoint{2.563509in}{1.649585in}}{\pgfqpoint{2.563509in}{1.659672in}}%
\pgfpathcurveto{\pgfqpoint{2.563509in}{1.669759in}}{\pgfqpoint{2.559501in}{1.679435in}}{\pgfqpoint{2.552368in}{1.686568in}}%
\pgfpathcurveto{\pgfqpoint{2.545236in}{1.693701in}}{\pgfqpoint{2.535560in}{1.697708in}}{\pgfqpoint{2.525473in}{1.697708in}}%
\pgfpathcurveto{\pgfqpoint{2.515385in}{1.697708in}}{\pgfqpoint{2.505710in}{1.693701in}}{\pgfqpoint{2.498577in}{1.686568in}}%
\pgfpathcurveto{\pgfqpoint{2.491444in}{1.679435in}}{\pgfqpoint{2.487436in}{1.669759in}}{\pgfqpoint{2.487436in}{1.659672in}}%
\pgfpathcurveto{\pgfqpoint{2.487436in}{1.649585in}}{\pgfqpoint{2.491444in}{1.639909in}}{\pgfqpoint{2.498577in}{1.632776in}}%
\pgfpathcurveto{\pgfqpoint{2.505710in}{1.625643in}}{\pgfqpoint{2.515385in}{1.621636in}}{\pgfqpoint{2.525473in}{1.621636in}}%
\pgfpathlineto{\pgfqpoint{2.525473in}{1.621636in}}%
\pgfpathclose%
\pgfusepath{stroke,fill}%
\end{pgfscope}%
\begin{pgfscope}%
\pgfpathrectangle{\pgfqpoint{0.254231in}{0.147348in}}{\pgfqpoint{2.735294in}{2.735294in}}%
\pgfusepath{clip}%
\pgfsetbuttcap%
\pgfsetroundjoin%
\definecolor{currentfill}{rgb}{0.839216,0.152941,0.156863}%
\pgfsetfillcolor{currentfill}%
\pgfsetfillopacity{0.612876}%
\pgfsetlinewidth{1.003750pt}%
\definecolor{currentstroke}{rgb}{0.839216,0.152941,0.156863}%
\pgfsetstrokecolor{currentstroke}%
\pgfsetstrokeopacity{0.612876}%
\pgfsetdash{}{0pt}%
\pgfpathmoveto{\pgfqpoint{1.213770in}{0.979547in}}%
\pgfpathcurveto{\pgfqpoint{1.223857in}{0.979547in}}{\pgfqpoint{1.233533in}{0.983554in}}{\pgfqpoint{1.240666in}{0.990687in}}%
\pgfpathcurveto{\pgfqpoint{1.247798in}{0.997820in}}{\pgfqpoint{1.251806in}{1.007495in}}{\pgfqpoint{1.251806in}{1.017583in}}%
\pgfpathcurveto{\pgfqpoint{1.251806in}{1.027670in}}{\pgfqpoint{1.247798in}{1.037346in}}{\pgfqpoint{1.240666in}{1.044479in}}%
\pgfpathcurveto{\pgfqpoint{1.233533in}{1.051611in}}{\pgfqpoint{1.223857in}{1.055619in}}{\pgfqpoint{1.213770in}{1.055619in}}%
\pgfpathcurveto{\pgfqpoint{1.203683in}{1.055619in}}{\pgfqpoint{1.194007in}{1.051611in}}{\pgfqpoint{1.186874in}{1.044479in}}%
\pgfpathcurveto{\pgfqpoint{1.179741in}{1.037346in}}{\pgfqpoint{1.175734in}{1.027670in}}{\pgfqpoint{1.175734in}{1.017583in}}%
\pgfpathcurveto{\pgfqpoint{1.175734in}{1.007495in}}{\pgfqpoint{1.179741in}{0.997820in}}{\pgfqpoint{1.186874in}{0.990687in}}%
\pgfpathcurveto{\pgfqpoint{1.194007in}{0.983554in}}{\pgfqpoint{1.203683in}{0.979547in}}{\pgfqpoint{1.213770in}{0.979547in}}%
\pgfpathlineto{\pgfqpoint{1.213770in}{0.979547in}}%
\pgfpathclose%
\pgfusepath{stroke,fill}%
\end{pgfscope}%
\begin{pgfscope}%
\pgfpathrectangle{\pgfqpoint{0.254231in}{0.147348in}}{\pgfqpoint{2.735294in}{2.735294in}}%
\pgfusepath{clip}%
\pgfsetbuttcap%
\pgfsetroundjoin%
\definecolor{currentfill}{rgb}{0.839216,0.152941,0.156863}%
\pgfsetfillcolor{currentfill}%
\pgfsetfillopacity{0.625674}%
\pgfsetlinewidth{1.003750pt}%
\definecolor{currentstroke}{rgb}{0.839216,0.152941,0.156863}%
\pgfsetstrokecolor{currentstroke}%
\pgfsetstrokeopacity{0.625674}%
\pgfsetdash{}{0pt}%
\pgfpathmoveto{\pgfqpoint{2.092670in}{2.164239in}}%
\pgfpathcurveto{\pgfqpoint{2.102758in}{2.164239in}}{\pgfqpoint{2.112433in}{2.168247in}}{\pgfqpoint{2.119566in}{2.175380in}}%
\pgfpathcurveto{\pgfqpoint{2.126699in}{2.182513in}}{\pgfqpoint{2.130707in}{2.192188in}}{\pgfqpoint{2.130707in}{2.202275in}}%
\pgfpathcurveto{\pgfqpoint{2.130707in}{2.212363in}}{\pgfqpoint{2.126699in}{2.222038in}}{\pgfqpoint{2.119566in}{2.229171in}}%
\pgfpathcurveto{\pgfqpoint{2.112433in}{2.236304in}}{\pgfqpoint{2.102758in}{2.240312in}}{\pgfqpoint{2.092670in}{2.240312in}}%
\pgfpathcurveto{\pgfqpoint{2.082583in}{2.240312in}}{\pgfqpoint{2.072907in}{2.236304in}}{\pgfqpoint{2.065775in}{2.229171in}}%
\pgfpathcurveto{\pgfqpoint{2.058642in}{2.222038in}}{\pgfqpoint{2.054634in}{2.212363in}}{\pgfqpoint{2.054634in}{2.202275in}}%
\pgfpathcurveto{\pgfqpoint{2.054634in}{2.192188in}}{\pgfqpoint{2.058642in}{2.182513in}}{\pgfqpoint{2.065775in}{2.175380in}}%
\pgfpathcurveto{\pgfqpoint{2.072907in}{2.168247in}}{\pgfqpoint{2.082583in}{2.164239in}}{\pgfqpoint{2.092670in}{2.164239in}}%
\pgfpathlineto{\pgfqpoint{2.092670in}{2.164239in}}%
\pgfpathclose%
\pgfusepath{stroke,fill}%
\end{pgfscope}%
\begin{pgfscope}%
\pgfpathrectangle{\pgfqpoint{0.254231in}{0.147348in}}{\pgfqpoint{2.735294in}{2.735294in}}%
\pgfusepath{clip}%
\pgfsetbuttcap%
\pgfsetroundjoin%
\definecolor{currentfill}{rgb}{0.839216,0.152941,0.156863}%
\pgfsetfillcolor{currentfill}%
\pgfsetfillopacity{0.631635}%
\pgfsetlinewidth{1.003750pt}%
\definecolor{currentstroke}{rgb}{0.839216,0.152941,0.156863}%
\pgfsetstrokecolor{currentstroke}%
\pgfsetstrokeopacity{0.631635}%
\pgfsetdash{}{0pt}%
\pgfpathmoveto{\pgfqpoint{2.083772in}{2.104757in}}%
\pgfpathcurveto{\pgfqpoint{2.093859in}{2.104757in}}{\pgfqpoint{2.103535in}{2.108765in}}{\pgfqpoint{2.110668in}{2.115898in}}%
\pgfpathcurveto{\pgfqpoint{2.117801in}{2.123030in}}{\pgfqpoint{2.121808in}{2.132706in}}{\pgfqpoint{2.121808in}{2.142793in}}%
\pgfpathcurveto{\pgfqpoint{2.121808in}{2.152881in}}{\pgfqpoint{2.117801in}{2.162556in}}{\pgfqpoint{2.110668in}{2.169689in}}%
\pgfpathcurveto{\pgfqpoint{2.103535in}{2.176822in}}{\pgfqpoint{2.093859in}{2.180830in}}{\pgfqpoint{2.083772in}{2.180830in}}%
\pgfpathcurveto{\pgfqpoint{2.073685in}{2.180830in}}{\pgfqpoint{2.064009in}{2.176822in}}{\pgfqpoint{2.056876in}{2.169689in}}%
\pgfpathcurveto{\pgfqpoint{2.049744in}{2.162556in}}{\pgfqpoint{2.045736in}{2.152881in}}{\pgfqpoint{2.045736in}{2.142793in}}%
\pgfpathcurveto{\pgfqpoint{2.045736in}{2.132706in}}{\pgfqpoint{2.049744in}{2.123030in}}{\pgfqpoint{2.056876in}{2.115898in}}%
\pgfpathcurveto{\pgfqpoint{2.064009in}{2.108765in}}{\pgfqpoint{2.073685in}{2.104757in}}{\pgfqpoint{2.083772in}{2.104757in}}%
\pgfpathlineto{\pgfqpoint{2.083772in}{2.104757in}}%
\pgfpathclose%
\pgfusepath{stroke,fill}%
\end{pgfscope}%
\begin{pgfscope}%
\pgfpathrectangle{\pgfqpoint{0.254231in}{0.147348in}}{\pgfqpoint{2.735294in}{2.735294in}}%
\pgfusepath{clip}%
\pgfsetbuttcap%
\pgfsetroundjoin%
\definecolor{currentfill}{rgb}{0.839216,0.152941,0.156863}%
\pgfsetfillcolor{currentfill}%
\pgfsetfillopacity{0.634032}%
\pgfsetlinewidth{1.003750pt}%
\definecolor{currentstroke}{rgb}{0.839216,0.152941,0.156863}%
\pgfsetstrokecolor{currentstroke}%
\pgfsetstrokeopacity{0.634032}%
\pgfsetdash{}{0pt}%
\pgfpathmoveto{\pgfqpoint{2.354767in}{1.533362in}}%
\pgfpathcurveto{\pgfqpoint{2.364854in}{1.533362in}}{\pgfqpoint{2.374530in}{1.537370in}}{\pgfqpoint{2.381663in}{1.544503in}}%
\pgfpathcurveto{\pgfqpoint{2.388796in}{1.551635in}}{\pgfqpoint{2.392803in}{1.561311in}}{\pgfqpoint{2.392803in}{1.571398in}}%
\pgfpathcurveto{\pgfqpoint{2.392803in}{1.581486in}}{\pgfqpoint{2.388796in}{1.591161in}}{\pgfqpoint{2.381663in}{1.598294in}}%
\pgfpathcurveto{\pgfqpoint{2.374530in}{1.605427in}}{\pgfqpoint{2.364854in}{1.609435in}}{\pgfqpoint{2.354767in}{1.609435in}}%
\pgfpathcurveto{\pgfqpoint{2.344680in}{1.609435in}}{\pgfqpoint{2.335004in}{1.605427in}}{\pgfqpoint{2.327871in}{1.598294in}}%
\pgfpathcurveto{\pgfqpoint{2.320739in}{1.591161in}}{\pgfqpoint{2.316731in}{1.581486in}}{\pgfqpoint{2.316731in}{1.571398in}}%
\pgfpathcurveto{\pgfqpoint{2.316731in}{1.561311in}}{\pgfqpoint{2.320739in}{1.551635in}}{\pgfqpoint{2.327871in}{1.544503in}}%
\pgfpathcurveto{\pgfqpoint{2.335004in}{1.537370in}}{\pgfqpoint{2.344680in}{1.533362in}}{\pgfqpoint{2.354767in}{1.533362in}}%
\pgfpathlineto{\pgfqpoint{2.354767in}{1.533362in}}%
\pgfpathclose%
\pgfusepath{stroke,fill}%
\end{pgfscope}%
\begin{pgfscope}%
\pgfpathrectangle{\pgfqpoint{0.254231in}{0.147348in}}{\pgfqpoint{2.735294in}{2.735294in}}%
\pgfusepath{clip}%
\pgfsetbuttcap%
\pgfsetroundjoin%
\definecolor{currentfill}{rgb}{0.839216,0.152941,0.156863}%
\pgfsetfillcolor{currentfill}%
\pgfsetfillopacity{0.652064}%
\pgfsetlinewidth{1.003750pt}%
\definecolor{currentstroke}{rgb}{0.839216,0.152941,0.156863}%
\pgfsetstrokecolor{currentstroke}%
\pgfsetstrokeopacity{0.652064}%
\pgfsetdash{}{0pt}%
\pgfpathmoveto{\pgfqpoint{1.468055in}{0.952320in}}%
\pgfpathcurveto{\pgfqpoint{1.478142in}{0.952320in}}{\pgfqpoint{1.487818in}{0.956328in}}{\pgfqpoint{1.494951in}{0.963461in}}%
\pgfpathcurveto{\pgfqpoint{1.502084in}{0.970593in}}{\pgfqpoint{1.506091in}{0.980269in}}{\pgfqpoint{1.506091in}{0.990356in}}%
\pgfpathcurveto{\pgfqpoint{1.506091in}{1.000444in}}{\pgfqpoint{1.502084in}{1.010119in}}{\pgfqpoint{1.494951in}{1.017252in}}%
\pgfpathcurveto{\pgfqpoint{1.487818in}{1.024385in}}{\pgfqpoint{1.478142in}{1.028393in}}{\pgfqpoint{1.468055in}{1.028393in}}%
\pgfpathcurveto{\pgfqpoint{1.457968in}{1.028393in}}{\pgfqpoint{1.448292in}{1.024385in}}{\pgfqpoint{1.441159in}{1.017252in}}%
\pgfpathcurveto{\pgfqpoint{1.434027in}{1.010119in}}{\pgfqpoint{1.430019in}{1.000444in}}{\pgfqpoint{1.430019in}{0.990356in}}%
\pgfpathcurveto{\pgfqpoint{1.430019in}{0.980269in}}{\pgfqpoint{1.434027in}{0.970593in}}{\pgfqpoint{1.441159in}{0.963461in}}%
\pgfpathcurveto{\pgfqpoint{1.448292in}{0.956328in}}{\pgfqpoint{1.457968in}{0.952320in}}{\pgfqpoint{1.468055in}{0.952320in}}%
\pgfpathlineto{\pgfqpoint{1.468055in}{0.952320in}}%
\pgfpathclose%
\pgfusepath{stroke,fill}%
\end{pgfscope}%
\begin{pgfscope}%
\pgfpathrectangle{\pgfqpoint{0.254231in}{0.147348in}}{\pgfqpoint{2.735294in}{2.735294in}}%
\pgfusepath{clip}%
\pgfsetbuttcap%
\pgfsetroundjoin%
\definecolor{currentfill}{rgb}{0.839216,0.152941,0.156863}%
\pgfsetfillcolor{currentfill}%
\pgfsetfillopacity{0.738747}%
\pgfsetlinewidth{1.003750pt}%
\definecolor{currentstroke}{rgb}{0.839216,0.152941,0.156863}%
\pgfsetstrokecolor{currentstroke}%
\pgfsetstrokeopacity{0.738747}%
\pgfsetdash{}{0pt}%
\pgfpathmoveto{\pgfqpoint{0.858170in}{1.507716in}}%
\pgfpathcurveto{\pgfqpoint{0.868258in}{1.507716in}}{\pgfqpoint{0.877933in}{1.511723in}}{\pgfqpoint{0.885066in}{1.518856in}}%
\pgfpathcurveto{\pgfqpoint{0.892199in}{1.525989in}}{\pgfqpoint{0.896207in}{1.535665in}}{\pgfqpoint{0.896207in}{1.545752in}}%
\pgfpathcurveto{\pgfqpoint{0.896207in}{1.555839in}}{\pgfqpoint{0.892199in}{1.565515in}}{\pgfqpoint{0.885066in}{1.572648in}}%
\pgfpathcurveto{\pgfqpoint{0.877933in}{1.579780in}}{\pgfqpoint{0.868258in}{1.583788in}}{\pgfqpoint{0.858170in}{1.583788in}}%
\pgfpathcurveto{\pgfqpoint{0.848083in}{1.583788in}}{\pgfqpoint{0.838408in}{1.579780in}}{\pgfqpoint{0.831275in}{1.572648in}}%
\pgfpathcurveto{\pgfqpoint{0.824142in}{1.565515in}}{\pgfqpoint{0.820134in}{1.555839in}}{\pgfqpoint{0.820134in}{1.545752in}}%
\pgfpathcurveto{\pgfqpoint{0.820134in}{1.535665in}}{\pgfqpoint{0.824142in}{1.525989in}}{\pgfqpoint{0.831275in}{1.518856in}}%
\pgfpathcurveto{\pgfqpoint{0.838408in}{1.511723in}}{\pgfqpoint{0.848083in}{1.507716in}}{\pgfqpoint{0.858170in}{1.507716in}}%
\pgfpathlineto{\pgfqpoint{0.858170in}{1.507716in}}%
\pgfpathclose%
\pgfusepath{stroke,fill}%
\end{pgfscope}%
\begin{pgfscope}%
\pgfpathrectangle{\pgfqpoint{0.254231in}{0.147348in}}{\pgfqpoint{2.735294in}{2.735294in}}%
\pgfusepath{clip}%
\pgfsetbuttcap%
\pgfsetroundjoin%
\definecolor{currentfill}{rgb}{0.839216,0.152941,0.156863}%
\pgfsetfillcolor{currentfill}%
\pgfsetfillopacity{0.791813}%
\pgfsetlinewidth{1.003750pt}%
\definecolor{currentstroke}{rgb}{0.839216,0.152941,0.156863}%
\pgfsetstrokecolor{currentstroke}%
\pgfsetstrokeopacity{0.791813}%
\pgfsetdash{}{0pt}%
\pgfpathmoveto{\pgfqpoint{1.351308in}{2.014183in}}%
\pgfpathcurveto{\pgfqpoint{1.361395in}{2.014183in}}{\pgfqpoint{1.371071in}{2.018191in}}{\pgfqpoint{1.378204in}{2.025323in}}%
\pgfpathcurveto{\pgfqpoint{1.385337in}{2.032456in}}{\pgfqpoint{1.389344in}{2.042132in}}{\pgfqpoint{1.389344in}{2.052219in}}%
\pgfpathcurveto{\pgfqpoint{1.389344in}{2.062307in}}{\pgfqpoint{1.385337in}{2.071982in}}{\pgfqpoint{1.378204in}{2.079115in}}%
\pgfpathcurveto{\pgfqpoint{1.371071in}{2.086248in}}{\pgfqpoint{1.361395in}{2.090255in}}{\pgfqpoint{1.351308in}{2.090255in}}%
\pgfpathcurveto{\pgfqpoint{1.341221in}{2.090255in}}{\pgfqpoint{1.331545in}{2.086248in}}{\pgfqpoint{1.324412in}{2.079115in}}%
\pgfpathcurveto{\pgfqpoint{1.317280in}{2.071982in}}{\pgfqpoint{1.313272in}{2.062307in}}{\pgfqpoint{1.313272in}{2.052219in}}%
\pgfpathcurveto{\pgfqpoint{1.313272in}{2.042132in}}{\pgfqpoint{1.317280in}{2.032456in}}{\pgfqpoint{1.324412in}{2.025323in}}%
\pgfpathcurveto{\pgfqpoint{1.331545in}{2.018191in}}{\pgfqpoint{1.341221in}{2.014183in}}{\pgfqpoint{1.351308in}{2.014183in}}%
\pgfpathlineto{\pgfqpoint{1.351308in}{2.014183in}}%
\pgfpathclose%
\pgfusepath{stroke,fill}%
\end{pgfscope}%
\begin{pgfscope}%
\pgfpathrectangle{\pgfqpoint{0.254231in}{0.147348in}}{\pgfqpoint{2.735294in}{2.735294in}}%
\pgfusepath{clip}%
\pgfsetbuttcap%
\pgfsetroundjoin%
\definecolor{currentfill}{rgb}{0.839216,0.152941,0.156863}%
\pgfsetfillcolor{currentfill}%
\pgfsetfillopacity{0.864233}%
\pgfsetlinewidth{1.003750pt}%
\definecolor{currentstroke}{rgb}{0.839216,0.152941,0.156863}%
\pgfsetstrokecolor{currentstroke}%
\pgfsetstrokeopacity{0.864233}%
\pgfsetdash{}{0pt}%
\pgfpathmoveto{\pgfqpoint{2.017635in}{1.926870in}}%
\pgfpathcurveto{\pgfqpoint{2.027722in}{1.926870in}}{\pgfqpoint{2.037397in}{1.930878in}}{\pgfqpoint{2.044530in}{1.938011in}}%
\pgfpathcurveto{\pgfqpoint{2.051663in}{1.945143in}}{\pgfqpoint{2.055671in}{1.954819in}}{\pgfqpoint{2.055671in}{1.964906in}}%
\pgfpathcurveto{\pgfqpoint{2.055671in}{1.974994in}}{\pgfqpoint{2.051663in}{1.984669in}}{\pgfqpoint{2.044530in}{1.991802in}}%
\pgfpathcurveto{\pgfqpoint{2.037397in}{1.998935in}}{\pgfqpoint{2.027722in}{2.002943in}}{\pgfqpoint{2.017635in}{2.002943in}}%
\pgfpathcurveto{\pgfqpoint{2.007547in}{2.002943in}}{\pgfqpoint{1.997872in}{1.998935in}}{\pgfqpoint{1.990739in}{1.991802in}}%
\pgfpathcurveto{\pgfqpoint{1.983606in}{1.984669in}}{\pgfqpoint{1.979598in}{1.974994in}}{\pgfqpoint{1.979598in}{1.964906in}}%
\pgfpathcurveto{\pgfqpoint{1.979598in}{1.954819in}}{\pgfqpoint{1.983606in}{1.945143in}}{\pgfqpoint{1.990739in}{1.938011in}}%
\pgfpathcurveto{\pgfqpoint{1.997872in}{1.930878in}}{\pgfqpoint{2.007547in}{1.926870in}}{\pgfqpoint{2.017635in}{1.926870in}}%
\pgfpathlineto{\pgfqpoint{2.017635in}{1.926870in}}%
\pgfpathclose%
\pgfusepath{stroke,fill}%
\end{pgfscope}%
\begin{pgfscope}%
\pgfpathrectangle{\pgfqpoint{0.254231in}{0.147348in}}{\pgfqpoint{2.735294in}{2.735294in}}%
\pgfusepath{clip}%
\pgfsetbuttcap%
\pgfsetroundjoin%
\definecolor{currentfill}{rgb}{0.839216,0.152941,0.156863}%
\pgfsetfillcolor{currentfill}%
\pgfsetfillopacity{0.929084}%
\pgfsetlinewidth{1.003750pt}%
\definecolor{currentstroke}{rgb}{0.839216,0.152941,0.156863}%
\pgfsetstrokecolor{currentstroke}%
\pgfsetstrokeopacity{0.929084}%
\pgfsetdash{}{0pt}%
\pgfpathmoveto{\pgfqpoint{2.077047in}{1.288920in}}%
\pgfpathcurveto{\pgfqpoint{2.087135in}{1.288920in}}{\pgfqpoint{2.096810in}{1.292928in}}{\pgfqpoint{2.103943in}{1.300060in}}%
\pgfpathcurveto{\pgfqpoint{2.111076in}{1.307193in}}{\pgfqpoint{2.115084in}{1.316869in}}{\pgfqpoint{2.115084in}{1.326956in}}%
\pgfpathcurveto{\pgfqpoint{2.115084in}{1.337043in}}{\pgfqpoint{2.111076in}{1.346719in}}{\pgfqpoint{2.103943in}{1.353852in}}%
\pgfpathcurveto{\pgfqpoint{2.096810in}{1.360985in}}{\pgfqpoint{2.087135in}{1.364992in}}{\pgfqpoint{2.077047in}{1.364992in}}%
\pgfpathcurveto{\pgfqpoint{2.066960in}{1.364992in}}{\pgfqpoint{2.057285in}{1.360985in}}{\pgfqpoint{2.050152in}{1.353852in}}%
\pgfpathcurveto{\pgfqpoint{2.043019in}{1.346719in}}{\pgfqpoint{2.039011in}{1.337043in}}{\pgfqpoint{2.039011in}{1.326956in}}%
\pgfpathcurveto{\pgfqpoint{2.039011in}{1.316869in}}{\pgfqpoint{2.043019in}{1.307193in}}{\pgfqpoint{2.050152in}{1.300060in}}%
\pgfpathcurveto{\pgfqpoint{2.057285in}{1.292928in}}{\pgfqpoint{2.066960in}{1.288920in}}{\pgfqpoint{2.077047in}{1.288920in}}%
\pgfpathlineto{\pgfqpoint{2.077047in}{1.288920in}}%
\pgfpathclose%
\pgfusepath{stroke,fill}%
\end{pgfscope}%
\begin{pgfscope}%
\pgfpathrectangle{\pgfqpoint{0.254231in}{0.147348in}}{\pgfqpoint{2.735294in}{2.735294in}}%
\pgfusepath{clip}%
\pgfsetbuttcap%
\pgfsetroundjoin%
\definecolor{currentfill}{rgb}{0.839216,0.152941,0.156863}%
\pgfsetfillcolor{currentfill}%
\pgfsetlinewidth{1.003750pt}%
\definecolor{currentstroke}{rgb}{0.839216,0.152941,0.156863}%
\pgfsetstrokecolor{currentstroke}%
\pgfsetdash{}{0pt}%
\pgfpathmoveto{\pgfqpoint{1.957013in}{1.506712in}}%
\pgfpathcurveto{\pgfqpoint{1.967100in}{1.506712in}}{\pgfqpoint{1.976776in}{1.510720in}}{\pgfqpoint{1.983908in}{1.517853in}}%
\pgfpathcurveto{\pgfqpoint{1.991041in}{1.524986in}}{\pgfqpoint{1.995049in}{1.534661in}}{\pgfqpoint{1.995049in}{1.544748in}}%
\pgfpathcurveto{\pgfqpoint{1.995049in}{1.554836in}}{\pgfqpoint{1.991041in}{1.564511in}}{\pgfqpoint{1.983908in}{1.571644in}}%
\pgfpathcurveto{\pgfqpoint{1.976776in}{1.578777in}}{\pgfqpoint{1.967100in}{1.582785in}}{\pgfqpoint{1.957013in}{1.582785in}}%
\pgfpathcurveto{\pgfqpoint{1.946925in}{1.582785in}}{\pgfqpoint{1.937250in}{1.578777in}}{\pgfqpoint{1.930117in}{1.571644in}}%
\pgfpathcurveto{\pgfqpoint{1.922984in}{1.564511in}}{\pgfqpoint{1.918976in}{1.554836in}}{\pgfqpoint{1.918976in}{1.544748in}}%
\pgfpathcurveto{\pgfqpoint{1.918976in}{1.534661in}}{\pgfqpoint{1.922984in}{1.524986in}}{\pgfqpoint{1.930117in}{1.517853in}}%
\pgfpathcurveto{\pgfqpoint{1.937250in}{1.510720in}}{\pgfqpoint{1.946925in}{1.506712in}}{\pgfqpoint{1.957013in}{1.506712in}}%
\pgfpathlineto{\pgfqpoint{1.957013in}{1.506712in}}%
\pgfpathclose%
\pgfusepath{stroke,fill}%
\end{pgfscope}%
\begin{pgfscope}%
\pgfpathrectangle{\pgfqpoint{0.254231in}{0.147348in}}{\pgfqpoint{2.735294in}{2.735294in}}%
\pgfusepath{clip}%
\pgfsetbuttcap%
\pgfsetroundjoin%
\definecolor{currentfill}{rgb}{0.071067,0.258424,0.071067}%
\pgfsetfillcolor{currentfill}%
\pgfsetlinewidth{0.000000pt}%
\definecolor{currentstroke}{rgb}{0.000000,0.000000,0.000000}%
\pgfsetstrokecolor{currentstroke}%
\pgfsetdash{}{0pt}%
\pgfpathmoveto{\pgfqpoint{0.697594in}{1.088921in}}%
\pgfpathlineto{\pgfqpoint{0.601174in}{1.228895in}}%
\pgfpathlineto{\pgfqpoint{0.601748in}{1.155548in}}%
\pgfpathlineto{\pgfqpoint{0.697594in}{1.088921in}}%
\pgfpathclose%
\pgfusepath{fill}%
\end{pgfscope}%
\begin{pgfscope}%
\pgfpathrectangle{\pgfqpoint{0.254231in}{0.147348in}}{\pgfqpoint{2.735294in}{2.735294in}}%
\pgfusepath{clip}%
\pgfsetbuttcap%
\pgfsetroundjoin%
\definecolor{currentfill}{rgb}{0.071067,0.258424,0.071067}%
\pgfsetfillcolor{currentfill}%
\pgfsetlinewidth{0.000000pt}%
\definecolor{currentstroke}{rgb}{0.000000,0.000000,0.000000}%
\pgfsetstrokecolor{currentstroke}%
\pgfsetdash{}{0pt}%
\pgfpathmoveto{\pgfqpoint{2.716510in}{1.228895in}}%
\pgfpathlineto{\pgfqpoint{2.620090in}{1.088921in}}%
\pgfpathlineto{\pgfqpoint{2.715936in}{1.155548in}}%
\pgfpathlineto{\pgfqpoint{2.716510in}{1.228895in}}%
\pgfpathclose%
\pgfusepath{fill}%
\end{pgfscope}%
\begin{pgfscope}%
\pgfpathrectangle{\pgfqpoint{0.254231in}{0.147348in}}{\pgfqpoint{2.735294in}{2.735294in}}%
\pgfusepath{clip}%
\pgfsetbuttcap%
\pgfsetroundjoin%
\definecolor{currentfill}{rgb}{0.128601,0.467641,0.128601}%
\pgfsetfillcolor{currentfill}%
\pgfsetlinewidth{0.000000pt}%
\definecolor{currentstroke}{rgb}{0.000000,0.000000,0.000000}%
\pgfsetstrokecolor{currentstroke}%
\pgfsetdash{}{0pt}%
\pgfpathmoveto{\pgfqpoint{1.562421in}{2.632943in}}%
\pgfpathlineto{\pgfqpoint{1.755263in}{2.632943in}}%
\pgfpathlineto{\pgfqpoint{1.658842in}{2.699366in}}%
\pgfpathlineto{\pgfqpoint{1.562421in}{2.632943in}}%
\pgfpathclose%
\pgfusepath{fill}%
\end{pgfscope}%
\begin{pgfscope}%
\pgfpathrectangle{\pgfqpoint{0.254231in}{0.147348in}}{\pgfqpoint{2.735294in}{2.735294in}}%
\pgfusepath{clip}%
\pgfsetbuttcap%
\pgfsetroundjoin%
\definecolor{currentfill}{rgb}{0.067488,0.245410,0.067488}%
\pgfsetfillcolor{currentfill}%
\pgfsetlinewidth{0.000000pt}%
\definecolor{currentstroke}{rgb}{0.000000,0.000000,0.000000}%
\pgfsetstrokecolor{currentstroke}%
\pgfsetdash{}{0pt}%
\pgfpathmoveto{\pgfqpoint{0.822114in}{1.018707in}}%
\pgfpathlineto{\pgfqpoint{0.707491in}{1.163589in}}%
\pgfpathlineto{\pgfqpoint{0.697594in}{1.088921in}}%
\pgfpathlineto{\pgfqpoint{0.822114in}{1.018707in}}%
\pgfpathclose%
\pgfusepath{fill}%
\end{pgfscope}%
\begin{pgfscope}%
\pgfpathrectangle{\pgfqpoint{0.254231in}{0.147348in}}{\pgfqpoint{2.735294in}{2.735294in}}%
\pgfusepath{clip}%
\pgfsetbuttcap%
\pgfsetroundjoin%
\definecolor{currentfill}{rgb}{0.067488,0.245410,0.067488}%
\pgfsetfillcolor{currentfill}%
\pgfsetlinewidth{0.000000pt}%
\definecolor{currentstroke}{rgb}{0.000000,0.000000,0.000000}%
\pgfsetstrokecolor{currentstroke}%
\pgfsetdash{}{0pt}%
\pgfpathmoveto{\pgfqpoint{2.620090in}{1.088921in}}%
\pgfpathlineto{\pgfqpoint{2.610193in}{1.163589in}}%
\pgfpathlineto{\pgfqpoint{2.495569in}{1.018707in}}%
\pgfpathlineto{\pgfqpoint{2.620090in}{1.088921in}}%
\pgfpathclose%
\pgfusepath{fill}%
\end{pgfscope}%
\begin{pgfscope}%
\pgfpathrectangle{\pgfqpoint{0.254231in}{0.147348in}}{\pgfqpoint{2.735294in}{2.735294in}}%
\pgfusepath{clip}%
\pgfsetbuttcap%
\pgfsetroundjoin%
\definecolor{currentfill}{rgb}{0.069492,0.252698,0.069492}%
\pgfsetfillcolor{currentfill}%
\pgfsetlinewidth{0.000000pt}%
\definecolor{currentstroke}{rgb}{0.000000,0.000000,0.000000}%
\pgfsetstrokecolor{currentstroke}%
\pgfsetdash{}{0pt}%
\pgfpathmoveto{\pgfqpoint{0.601174in}{1.228895in}}%
\pgfpathlineto{\pgfqpoint{0.697594in}{1.088921in}}%
\pgfpathlineto{\pgfqpoint{0.695386in}{1.589726in}}%
\pgfpathlineto{\pgfqpoint{0.601174in}{1.228895in}}%
\pgfpathclose%
\pgfusepath{fill}%
\end{pgfscope}%
\begin{pgfscope}%
\pgfpathrectangle{\pgfqpoint{0.254231in}{0.147348in}}{\pgfqpoint{2.735294in}{2.735294in}}%
\pgfusepath{clip}%
\pgfsetbuttcap%
\pgfsetroundjoin%
\definecolor{currentfill}{rgb}{0.069492,0.252698,0.069492}%
\pgfsetfillcolor{currentfill}%
\pgfsetlinewidth{0.000000pt}%
\definecolor{currentstroke}{rgb}{0.000000,0.000000,0.000000}%
\pgfsetstrokecolor{currentstroke}%
\pgfsetdash{}{0pt}%
\pgfpathmoveto{\pgfqpoint{2.716510in}{1.228895in}}%
\pgfpathlineto{\pgfqpoint{2.622298in}{1.589726in}}%
\pgfpathlineto{\pgfqpoint{2.620090in}{1.088921in}}%
\pgfpathlineto{\pgfqpoint{2.716510in}{1.228895in}}%
\pgfpathclose%
\pgfusepath{fill}%
\end{pgfscope}%
\begin{pgfscope}%
\pgfpathrectangle{\pgfqpoint{0.254231in}{0.147348in}}{\pgfqpoint{2.735294in}{2.735294in}}%
\pgfusepath{clip}%
\pgfsetbuttcap%
\pgfsetroundjoin%
\definecolor{currentfill}{rgb}{0.099716,0.362602,0.099716}%
\pgfsetfillcolor{currentfill}%
\pgfsetlinewidth{0.000000pt}%
\definecolor{currentstroke}{rgb}{0.000000,0.000000,0.000000}%
\pgfsetstrokecolor{currentstroke}%
\pgfsetdash{}{0pt}%
\pgfpathmoveto{\pgfqpoint{0.695386in}{1.589726in}}%
\pgfpathlineto{\pgfqpoint{0.697594in}{1.088921in}}%
\pgfpathlineto{\pgfqpoint{0.707491in}{1.163589in}}%
\pgfpathlineto{\pgfqpoint{0.695386in}{1.589726in}}%
\pgfpathclose%
\pgfusepath{fill}%
\end{pgfscope}%
\begin{pgfscope}%
\pgfpathrectangle{\pgfqpoint{0.254231in}{0.147348in}}{\pgfqpoint{2.735294in}{2.735294in}}%
\pgfusepath{clip}%
\pgfsetbuttcap%
\pgfsetroundjoin%
\definecolor{currentfill}{rgb}{0.099716,0.362602,0.099716}%
\pgfsetfillcolor{currentfill}%
\pgfsetlinewidth{0.000000pt}%
\definecolor{currentstroke}{rgb}{0.000000,0.000000,0.000000}%
\pgfsetstrokecolor{currentstroke}%
\pgfsetdash{}{0pt}%
\pgfpathmoveto{\pgfqpoint{2.610193in}{1.163589in}}%
\pgfpathlineto{\pgfqpoint{2.620090in}{1.088921in}}%
\pgfpathlineto{\pgfqpoint{2.622298in}{1.589726in}}%
\pgfpathlineto{\pgfqpoint{2.610193in}{1.163589in}}%
\pgfpathclose%
\pgfusepath{fill}%
\end{pgfscope}%
\begin{pgfscope}%
\pgfpathrectangle{\pgfqpoint{0.254231in}{0.147348in}}{\pgfqpoint{2.735294in}{2.735294in}}%
\pgfusepath{clip}%
\pgfsetbuttcap%
\pgfsetroundjoin%
\definecolor{currentfill}{rgb}{0.063840,0.232145,0.063840}%
\pgfsetfillcolor{currentfill}%
\pgfsetlinewidth{0.000000pt}%
\definecolor{currentstroke}{rgb}{0.000000,0.000000,0.000000}%
\pgfsetstrokecolor{currentstroke}%
\pgfsetdash{}{0pt}%
\pgfpathmoveto{\pgfqpoint{0.981201in}{0.949464in}}%
\pgfpathlineto{\pgfqpoint{0.848606in}{1.094945in}}%
\pgfpathlineto{\pgfqpoint{0.822114in}{1.018707in}}%
\pgfpathlineto{\pgfqpoint{0.981201in}{0.949464in}}%
\pgfpathclose%
\pgfusepath{fill}%
\end{pgfscope}%
\begin{pgfscope}%
\pgfpathrectangle{\pgfqpoint{0.254231in}{0.147348in}}{\pgfqpoint{2.735294in}{2.735294in}}%
\pgfusepath{clip}%
\pgfsetbuttcap%
\pgfsetroundjoin%
\definecolor{currentfill}{rgb}{0.063840,0.232145,0.063840}%
\pgfsetfillcolor{currentfill}%
\pgfsetlinewidth{0.000000pt}%
\definecolor{currentstroke}{rgb}{0.000000,0.000000,0.000000}%
\pgfsetstrokecolor{currentstroke}%
\pgfsetdash{}{0pt}%
\pgfpathmoveto{\pgfqpoint{2.495569in}{1.018707in}}%
\pgfpathlineto{\pgfqpoint{2.469078in}{1.094945in}}%
\pgfpathlineto{\pgfqpoint{2.336483in}{0.949464in}}%
\pgfpathlineto{\pgfqpoint{2.495569in}{1.018707in}}%
\pgfpathclose%
\pgfusepath{fill}%
\end{pgfscope}%
\begin{pgfscope}%
\pgfpathrectangle{\pgfqpoint{0.254231in}{0.147348in}}{\pgfqpoint{2.735294in}{2.735294in}}%
\pgfusepath{clip}%
\pgfsetbuttcap%
\pgfsetroundjoin%
\definecolor{currentfill}{rgb}{0.116321,0.422987,0.116321}%
\pgfsetfillcolor{currentfill}%
\pgfsetlinewidth{0.000000pt}%
\definecolor{currentstroke}{rgb}{0.000000,0.000000,0.000000}%
\pgfsetstrokecolor{currentstroke}%
\pgfsetdash{}{0pt}%
\pgfpathmoveto{\pgfqpoint{1.755263in}{2.632943in}}%
\pgfpathlineto{\pgfqpoint{1.562421in}{2.632943in}}%
\pgfpathlineto{\pgfqpoint{1.520550in}{2.150689in}}%
\pgfpathlineto{\pgfqpoint{1.755263in}{2.632943in}}%
\pgfpathclose%
\pgfusepath{fill}%
\end{pgfscope}%
\begin{pgfscope}%
\pgfpathrectangle{\pgfqpoint{0.254231in}{0.147348in}}{\pgfqpoint{2.735294in}{2.735294in}}%
\pgfusepath{clip}%
\pgfsetbuttcap%
\pgfsetroundjoin%
\definecolor{currentfill}{rgb}{0.067061,0.243857,0.067061}%
\pgfsetfillcolor{currentfill}%
\pgfsetlinewidth{0.000000pt}%
\definecolor{currentstroke}{rgb}{0.000000,0.000000,0.000000}%
\pgfsetstrokecolor{currentstroke}%
\pgfsetdash{}{0pt}%
\pgfpathmoveto{\pgfqpoint{0.707491in}{1.163589in}}%
\pgfpathlineto{\pgfqpoint{0.822114in}{1.018707in}}%
\pgfpathlineto{\pgfqpoint{0.857590in}{1.557460in}}%
\pgfpathlineto{\pgfqpoint{0.707491in}{1.163589in}}%
\pgfpathclose%
\pgfusepath{fill}%
\end{pgfscope}%
\begin{pgfscope}%
\pgfpathrectangle{\pgfqpoint{0.254231in}{0.147348in}}{\pgfqpoint{2.735294in}{2.735294in}}%
\pgfusepath{clip}%
\pgfsetbuttcap%
\pgfsetroundjoin%
\definecolor{currentfill}{rgb}{0.067061,0.243857,0.067061}%
\pgfsetfillcolor{currentfill}%
\pgfsetlinewidth{0.000000pt}%
\definecolor{currentstroke}{rgb}{0.000000,0.000000,0.000000}%
\pgfsetstrokecolor{currentstroke}%
\pgfsetdash{}{0pt}%
\pgfpathmoveto{\pgfqpoint{2.460094in}{1.557460in}}%
\pgfpathlineto{\pgfqpoint{2.495569in}{1.018707in}}%
\pgfpathlineto{\pgfqpoint{2.610193in}{1.163589in}}%
\pgfpathlineto{\pgfqpoint{2.460094in}{1.557460in}}%
\pgfpathclose%
\pgfusepath{fill}%
\end{pgfscope}%
\begin{pgfscope}%
\pgfpathrectangle{\pgfqpoint{0.254231in}{0.147348in}}{\pgfqpoint{2.735294in}{2.735294in}}%
\pgfusepath{clip}%
\pgfsetbuttcap%
\pgfsetroundjoin%
\definecolor{currentfill}{rgb}{0.095351,0.346729,0.095351}%
\pgfsetfillcolor{currentfill}%
\pgfsetlinewidth{0.000000pt}%
\definecolor{currentstroke}{rgb}{0.000000,0.000000,0.000000}%
\pgfsetstrokecolor{currentstroke}%
\pgfsetdash{}{0pt}%
\pgfpathmoveto{\pgfqpoint{0.857590in}{1.557460in}}%
\pgfpathlineto{\pgfqpoint{0.822114in}{1.018707in}}%
\pgfpathlineto{\pgfqpoint{0.848606in}{1.094945in}}%
\pgfpathlineto{\pgfqpoint{0.857590in}{1.557460in}}%
\pgfpathclose%
\pgfusepath{fill}%
\end{pgfscope}%
\begin{pgfscope}%
\pgfpathrectangle{\pgfqpoint{0.254231in}{0.147348in}}{\pgfqpoint{2.735294in}{2.735294in}}%
\pgfusepath{clip}%
\pgfsetbuttcap%
\pgfsetroundjoin%
\definecolor{currentfill}{rgb}{0.095351,0.346729,0.095351}%
\pgfsetfillcolor{currentfill}%
\pgfsetlinewidth{0.000000pt}%
\definecolor{currentstroke}{rgb}{0.000000,0.000000,0.000000}%
\pgfsetstrokecolor{currentstroke}%
\pgfsetdash{}{0pt}%
\pgfpathmoveto{\pgfqpoint{2.469078in}{1.094945in}}%
\pgfpathlineto{\pgfqpoint{2.495569in}{1.018707in}}%
\pgfpathlineto{\pgfqpoint{2.460094in}{1.557460in}}%
\pgfpathlineto{\pgfqpoint{2.469078in}{1.094945in}}%
\pgfpathclose%
\pgfusepath{fill}%
\end{pgfscope}%
\begin{pgfscope}%
\pgfpathrectangle{\pgfqpoint{0.254231in}{0.147348in}}{\pgfqpoint{2.735294in}{2.735294in}}%
\pgfusepath{clip}%
\pgfsetbuttcap%
\pgfsetroundjoin%
\definecolor{currentfill}{rgb}{0.060435,0.219763,0.060435}%
\pgfsetfillcolor{currentfill}%
\pgfsetlinewidth{0.000000pt}%
\definecolor{currentstroke}{rgb}{0.000000,0.000000,0.000000}%
\pgfsetstrokecolor{currentstroke}%
\pgfsetdash{}{0pt}%
\pgfpathmoveto{\pgfqpoint{2.286012in}{1.028737in}}%
\pgfpathlineto{\pgfqpoint{2.139898in}{0.888724in}}%
\pgfpathlineto{\pgfqpoint{2.336483in}{0.949464in}}%
\pgfpathlineto{\pgfqpoint{2.286012in}{1.028737in}}%
\pgfpathclose%
\pgfusepath{fill}%
\end{pgfscope}%
\begin{pgfscope}%
\pgfpathrectangle{\pgfqpoint{0.254231in}{0.147348in}}{\pgfqpoint{2.735294in}{2.735294in}}%
\pgfusepath{clip}%
\pgfsetbuttcap%
\pgfsetroundjoin%
\definecolor{currentfill}{rgb}{0.060435,0.219763,0.060435}%
\pgfsetfillcolor{currentfill}%
\pgfsetlinewidth{0.000000pt}%
\definecolor{currentstroke}{rgb}{0.000000,0.000000,0.000000}%
\pgfsetstrokecolor{currentstroke}%
\pgfsetdash{}{0pt}%
\pgfpathmoveto{\pgfqpoint{0.981201in}{0.949464in}}%
\pgfpathlineto{\pgfqpoint{1.177786in}{0.888724in}}%
\pgfpathlineto{\pgfqpoint{1.031671in}{1.028737in}}%
\pgfpathlineto{\pgfqpoint{0.981201in}{0.949464in}}%
\pgfpathclose%
\pgfusepath{fill}%
\end{pgfscope}%
\begin{pgfscope}%
\pgfpathrectangle{\pgfqpoint{0.254231in}{0.147348in}}{\pgfqpoint{2.735294in}{2.735294in}}%
\pgfusepath{clip}%
\pgfsetbuttcap%
\pgfsetroundjoin%
\definecolor{currentfill}{rgb}{0.074506,0.270932,0.074506}%
\pgfsetfillcolor{currentfill}%
\pgfsetlinewidth{0.000000pt}%
\definecolor{currentstroke}{rgb}{0.000000,0.000000,0.000000}%
\pgfsetstrokecolor{currentstroke}%
\pgfsetdash{}{0pt}%
\pgfpathmoveto{\pgfqpoint{0.695386in}{1.589726in}}%
\pgfpathlineto{\pgfqpoint{0.707491in}{1.163589in}}%
\pgfpathlineto{\pgfqpoint{0.857590in}{1.557460in}}%
\pgfpathlineto{\pgfqpoint{0.695386in}{1.589726in}}%
\pgfpathclose%
\pgfusepath{fill}%
\end{pgfscope}%
\begin{pgfscope}%
\pgfpathrectangle{\pgfqpoint{0.254231in}{0.147348in}}{\pgfqpoint{2.735294in}{2.735294in}}%
\pgfusepath{clip}%
\pgfsetbuttcap%
\pgfsetroundjoin%
\definecolor{currentfill}{rgb}{0.074506,0.270932,0.074506}%
\pgfsetfillcolor{currentfill}%
\pgfsetlinewidth{0.000000pt}%
\definecolor{currentstroke}{rgb}{0.000000,0.000000,0.000000}%
\pgfsetstrokecolor{currentstroke}%
\pgfsetdash{}{0pt}%
\pgfpathmoveto{\pgfqpoint{2.460094in}{1.557460in}}%
\pgfpathlineto{\pgfqpoint{2.610193in}{1.163589in}}%
\pgfpathlineto{\pgfqpoint{2.622298in}{1.589726in}}%
\pgfpathlineto{\pgfqpoint{2.460094in}{1.557460in}}%
\pgfpathclose%
\pgfusepath{fill}%
\end{pgfscope}%
\begin{pgfscope}%
\pgfpathrectangle{\pgfqpoint{0.254231in}{0.147348in}}{\pgfqpoint{2.735294in}{2.735294in}}%
\pgfusepath{clip}%
\pgfsetbuttcap%
\pgfsetroundjoin%
\definecolor{currentfill}{rgb}{0.116785,0.424671,0.116785}%
\pgfsetfillcolor{currentfill}%
\pgfsetlinewidth{0.000000pt}%
\definecolor{currentstroke}{rgb}{0.000000,0.000000,0.000000}%
\pgfsetstrokecolor{currentstroke}%
\pgfsetdash{}{0pt}%
\pgfpathmoveto{\pgfqpoint{2.141244in}{2.292690in}}%
\pgfpathlineto{\pgfqpoint{1.755263in}{2.632943in}}%
\pgfpathlineto{\pgfqpoint{1.797134in}{2.150689in}}%
\pgfpathlineto{\pgfqpoint{2.141244in}{2.292690in}}%
\pgfpathclose%
\pgfusepath{fill}%
\end{pgfscope}%
\begin{pgfscope}%
\pgfpathrectangle{\pgfqpoint{0.254231in}{0.147348in}}{\pgfqpoint{2.735294in}{2.735294in}}%
\pgfusepath{clip}%
\pgfsetbuttcap%
\pgfsetroundjoin%
\definecolor{currentfill}{rgb}{0.116785,0.424671,0.116785}%
\pgfsetfillcolor{currentfill}%
\pgfsetlinewidth{0.000000pt}%
\definecolor{currentstroke}{rgb}{0.000000,0.000000,0.000000}%
\pgfsetstrokecolor{currentstroke}%
\pgfsetdash{}{0pt}%
\pgfpathmoveto{\pgfqpoint{1.520550in}{2.150689in}}%
\pgfpathlineto{\pgfqpoint{1.562421in}{2.632943in}}%
\pgfpathlineto{\pgfqpoint{1.176440in}{2.292690in}}%
\pgfpathlineto{\pgfqpoint{1.520550in}{2.150689in}}%
\pgfpathclose%
\pgfusepath{fill}%
\end{pgfscope}%
\begin{pgfscope}%
\pgfpathrectangle{\pgfqpoint{0.254231in}{0.147348in}}{\pgfqpoint{2.735294in}{2.735294in}}%
\pgfusepath{clip}%
\pgfsetbuttcap%
\pgfsetroundjoin%
\definecolor{currentfill}{rgb}{0.064867,0.235879,0.064867}%
\pgfsetfillcolor{currentfill}%
\pgfsetlinewidth{0.000000pt}%
\definecolor{currentstroke}{rgb}{0.000000,0.000000,0.000000}%
\pgfsetstrokecolor{currentstroke}%
\pgfsetdash{}{0pt}%
\pgfpathmoveto{\pgfqpoint{0.848606in}{1.094945in}}%
\pgfpathlineto{\pgfqpoint{0.981201in}{0.949464in}}%
\pgfpathlineto{\pgfqpoint{1.075030in}{1.526914in}}%
\pgfpathlineto{\pgfqpoint{0.848606in}{1.094945in}}%
\pgfpathclose%
\pgfusepath{fill}%
\end{pgfscope}%
\begin{pgfscope}%
\pgfpathrectangle{\pgfqpoint{0.254231in}{0.147348in}}{\pgfqpoint{2.735294in}{2.735294in}}%
\pgfusepath{clip}%
\pgfsetbuttcap%
\pgfsetroundjoin%
\definecolor{currentfill}{rgb}{0.064867,0.235879,0.064867}%
\pgfsetfillcolor{currentfill}%
\pgfsetlinewidth{0.000000pt}%
\definecolor{currentstroke}{rgb}{0.000000,0.000000,0.000000}%
\pgfsetstrokecolor{currentstroke}%
\pgfsetdash{}{0pt}%
\pgfpathmoveto{\pgfqpoint{2.242654in}{1.526914in}}%
\pgfpathlineto{\pgfqpoint{2.336483in}{0.949464in}}%
\pgfpathlineto{\pgfqpoint{2.469078in}{1.094945in}}%
\pgfpathlineto{\pgfqpoint{2.242654in}{1.526914in}}%
\pgfpathclose%
\pgfusepath{fill}%
\end{pgfscope}%
\begin{pgfscope}%
\pgfpathrectangle{\pgfqpoint{0.254231in}{0.147348in}}{\pgfqpoint{2.735294in}{2.735294in}}%
\pgfusepath{clip}%
\pgfsetbuttcap%
\pgfsetroundjoin%
\definecolor{currentfill}{rgb}{0.057724,0.209904,0.057724}%
\pgfsetfillcolor{currentfill}%
\pgfsetlinewidth{0.000000pt}%
\definecolor{currentstroke}{rgb}{0.000000,0.000000,0.000000}%
\pgfsetstrokecolor{currentstroke}%
\pgfsetdash{}{0pt}%
\pgfpathmoveto{\pgfqpoint{1.258681in}{0.974376in}}%
\pgfpathlineto{\pgfqpoint{1.177786in}{0.888724in}}%
\pgfpathlineto{\pgfqpoint{1.408065in}{0.846223in}}%
\pgfpathlineto{\pgfqpoint{1.258681in}{0.974376in}}%
\pgfpathclose%
\pgfusepath{fill}%
\end{pgfscope}%
\begin{pgfscope}%
\pgfpathrectangle{\pgfqpoint{0.254231in}{0.147348in}}{\pgfqpoint{2.735294in}{2.735294in}}%
\pgfusepath{clip}%
\pgfsetbuttcap%
\pgfsetroundjoin%
\definecolor{currentfill}{rgb}{0.057724,0.209904,0.057724}%
\pgfsetfillcolor{currentfill}%
\pgfsetlinewidth{0.000000pt}%
\definecolor{currentstroke}{rgb}{0.000000,0.000000,0.000000}%
\pgfsetstrokecolor{currentstroke}%
\pgfsetdash{}{0pt}%
\pgfpathmoveto{\pgfqpoint{1.909619in}{0.846223in}}%
\pgfpathlineto{\pgfqpoint{2.139898in}{0.888724in}}%
\pgfpathlineto{\pgfqpoint{2.059003in}{0.974376in}}%
\pgfpathlineto{\pgfqpoint{1.909619in}{0.846223in}}%
\pgfpathclose%
\pgfusepath{fill}%
\end{pgfscope}%
\begin{pgfscope}%
\pgfpathrectangle{\pgfqpoint{0.254231in}{0.147348in}}{\pgfqpoint{2.735294in}{2.735294in}}%
\pgfusepath{clip}%
\pgfsetbuttcap%
\pgfsetroundjoin%
\definecolor{currentfill}{rgb}{0.086498,0.314539,0.086498}%
\pgfsetfillcolor{currentfill}%
\pgfsetlinewidth{0.000000pt}%
\definecolor{currentstroke}{rgb}{0.000000,0.000000,0.000000}%
\pgfsetstrokecolor{currentstroke}%
\pgfsetdash{}{0pt}%
\pgfpathmoveto{\pgfqpoint{0.857590in}{1.557460in}}%
\pgfpathlineto{\pgfqpoint{0.779523in}{1.760124in}}%
\pgfpathlineto{\pgfqpoint{0.695386in}{1.589726in}}%
\pgfpathlineto{\pgfqpoint{0.857590in}{1.557460in}}%
\pgfpathclose%
\pgfusepath{fill}%
\end{pgfscope}%
\begin{pgfscope}%
\pgfpathrectangle{\pgfqpoint{0.254231in}{0.147348in}}{\pgfqpoint{2.735294in}{2.735294in}}%
\pgfusepath{clip}%
\pgfsetbuttcap%
\pgfsetroundjoin%
\definecolor{currentfill}{rgb}{0.086498,0.314539,0.086498}%
\pgfsetfillcolor{currentfill}%
\pgfsetlinewidth{0.000000pt}%
\definecolor{currentstroke}{rgb}{0.000000,0.000000,0.000000}%
\pgfsetstrokecolor{currentstroke}%
\pgfsetdash{}{0pt}%
\pgfpathmoveto{\pgfqpoint{2.622298in}{1.589726in}}%
\pgfpathlineto{\pgfqpoint{2.538161in}{1.760124in}}%
\pgfpathlineto{\pgfqpoint{2.460094in}{1.557460in}}%
\pgfpathlineto{\pgfqpoint{2.622298in}{1.589726in}}%
\pgfpathclose%
\pgfusepath{fill}%
\end{pgfscope}%
\begin{pgfscope}%
\pgfpathrectangle{\pgfqpoint{0.254231in}{0.147348in}}{\pgfqpoint{2.735294in}{2.735294in}}%
\pgfusepath{clip}%
\pgfsetbuttcap%
\pgfsetroundjoin%
\definecolor{currentfill}{rgb}{0.090812,0.330224,0.090812}%
\pgfsetfillcolor{currentfill}%
\pgfsetlinewidth{0.000000pt}%
\definecolor{currentstroke}{rgb}{0.000000,0.000000,0.000000}%
\pgfsetstrokecolor{currentstroke}%
\pgfsetdash{}{0pt}%
\pgfpathmoveto{\pgfqpoint{1.075030in}{1.526914in}}%
\pgfpathlineto{\pgfqpoint{0.981201in}{0.949464in}}%
\pgfpathlineto{\pgfqpoint{1.031671in}{1.028737in}}%
\pgfpathlineto{\pgfqpoint{1.075030in}{1.526914in}}%
\pgfpathclose%
\pgfusepath{fill}%
\end{pgfscope}%
\begin{pgfscope}%
\pgfpathrectangle{\pgfqpoint{0.254231in}{0.147348in}}{\pgfqpoint{2.735294in}{2.735294in}}%
\pgfusepath{clip}%
\pgfsetbuttcap%
\pgfsetroundjoin%
\definecolor{currentfill}{rgb}{0.090812,0.330224,0.090812}%
\pgfsetfillcolor{currentfill}%
\pgfsetlinewidth{0.000000pt}%
\definecolor{currentstroke}{rgb}{0.000000,0.000000,0.000000}%
\pgfsetstrokecolor{currentstroke}%
\pgfsetdash{}{0pt}%
\pgfpathmoveto{\pgfqpoint{2.286012in}{1.028737in}}%
\pgfpathlineto{\pgfqpoint{2.336483in}{0.949464in}}%
\pgfpathlineto{\pgfqpoint{2.242654in}{1.526914in}}%
\pgfpathlineto{\pgfqpoint{2.286012in}{1.028737in}}%
\pgfpathclose%
\pgfusepath{fill}%
\end{pgfscope}%
\begin{pgfscope}%
\pgfpathrectangle{\pgfqpoint{0.254231in}{0.147348in}}{\pgfqpoint{2.735294in}{2.735294in}}%
\pgfusepath{clip}%
\pgfsetbuttcap%
\pgfsetroundjoin%
\definecolor{currentfill}{rgb}{0.116321,0.422987,0.116321}%
\pgfsetfillcolor{currentfill}%
\pgfsetlinewidth{0.000000pt}%
\definecolor{currentstroke}{rgb}{0.000000,0.000000,0.000000}%
\pgfsetstrokecolor{currentstroke}%
\pgfsetdash{}{0pt}%
\pgfpathmoveto{\pgfqpoint{1.520550in}{2.150689in}}%
\pgfpathlineto{\pgfqpoint{1.797134in}{2.150689in}}%
\pgfpathlineto{\pgfqpoint{1.755263in}{2.632943in}}%
\pgfpathlineto{\pgfqpoint{1.520550in}{2.150689in}}%
\pgfpathclose%
\pgfusepath{fill}%
\end{pgfscope}%
\begin{pgfscope}%
\pgfpathrectangle{\pgfqpoint{0.254231in}{0.147348in}}{\pgfqpoint{2.735294in}{2.735294in}}%
\pgfusepath{clip}%
\pgfsetbuttcap%
\pgfsetroundjoin%
\definecolor{currentfill}{rgb}{0.056200,0.204363,0.056200}%
\pgfsetfillcolor{currentfill}%
\pgfsetlinewidth{0.000000pt}%
\definecolor{currentstroke}{rgb}{0.000000,0.000000,0.000000}%
\pgfsetstrokecolor{currentstroke}%
\pgfsetdash{}{0pt}%
\pgfpathmoveto{\pgfqpoint{1.658842in}{0.830831in}}%
\pgfpathlineto{\pgfqpoint{1.520882in}{0.943120in}}%
\pgfpathlineto{\pgfqpoint{1.408065in}{0.846223in}}%
\pgfpathlineto{\pgfqpoint{1.658842in}{0.830831in}}%
\pgfpathclose%
\pgfusepath{fill}%
\end{pgfscope}%
\begin{pgfscope}%
\pgfpathrectangle{\pgfqpoint{0.254231in}{0.147348in}}{\pgfqpoint{2.735294in}{2.735294in}}%
\pgfusepath{clip}%
\pgfsetbuttcap%
\pgfsetroundjoin%
\definecolor{currentfill}{rgb}{0.056200,0.204363,0.056200}%
\pgfsetfillcolor{currentfill}%
\pgfsetlinewidth{0.000000pt}%
\definecolor{currentstroke}{rgb}{0.000000,0.000000,0.000000}%
\pgfsetstrokecolor{currentstroke}%
\pgfsetdash{}{0pt}%
\pgfpathmoveto{\pgfqpoint{1.909619in}{0.846223in}}%
\pgfpathlineto{\pgfqpoint{1.796802in}{0.943120in}}%
\pgfpathlineto{\pgfqpoint{1.658842in}{0.830831in}}%
\pgfpathlineto{\pgfqpoint{1.909619in}{0.846223in}}%
\pgfpathclose%
\pgfusepath{fill}%
\end{pgfscope}%
\begin{pgfscope}%
\pgfpathrectangle{\pgfqpoint{0.254231in}{0.147348in}}{\pgfqpoint{2.735294in}{2.735294in}}%
\pgfusepath{clip}%
\pgfsetbuttcap%
\pgfsetroundjoin%
\definecolor{currentfill}{rgb}{0.107070,0.389346,0.107070}%
\pgfsetfillcolor{currentfill}%
\pgfsetlinewidth{0.000000pt}%
\definecolor{currentstroke}{rgb}{0.000000,0.000000,0.000000}%
\pgfsetstrokecolor{currentstroke}%
\pgfsetdash{}{0pt}%
\pgfpathmoveto{\pgfqpoint{1.257750in}{2.141900in}}%
\pgfpathlineto{\pgfqpoint{1.176440in}{2.292690in}}%
\pgfpathlineto{\pgfqpoint{1.030299in}{2.126617in}}%
\pgfpathlineto{\pgfqpoint{1.257750in}{2.141900in}}%
\pgfpathclose%
\pgfusepath{fill}%
\end{pgfscope}%
\begin{pgfscope}%
\pgfpathrectangle{\pgfqpoint{0.254231in}{0.147348in}}{\pgfqpoint{2.735294in}{2.735294in}}%
\pgfusepath{clip}%
\pgfsetbuttcap%
\pgfsetroundjoin%
\definecolor{currentfill}{rgb}{0.107070,0.389346,0.107070}%
\pgfsetfillcolor{currentfill}%
\pgfsetlinewidth{0.000000pt}%
\definecolor{currentstroke}{rgb}{0.000000,0.000000,0.000000}%
\pgfsetstrokecolor{currentstroke}%
\pgfsetdash{}{0pt}%
\pgfpathmoveto{\pgfqpoint{2.287385in}{2.126617in}}%
\pgfpathlineto{\pgfqpoint{2.141244in}{2.292690in}}%
\pgfpathlineto{\pgfqpoint{2.059934in}{2.141900in}}%
\pgfpathlineto{\pgfqpoint{2.287385in}{2.126617in}}%
\pgfpathclose%
\pgfusepath{fill}%
\end{pgfscope}%
\begin{pgfscope}%
\pgfpathrectangle{\pgfqpoint{0.254231in}{0.147348in}}{\pgfqpoint{2.735294in}{2.735294in}}%
\pgfusepath{clip}%
\pgfsetbuttcap%
\pgfsetroundjoin%
\definecolor{currentfill}{rgb}{0.089078,0.323920,0.089078}%
\pgfsetfillcolor{currentfill}%
\pgfsetlinewidth{0.000000pt}%
\definecolor{currentstroke}{rgb}{0.000000,0.000000,0.000000}%
\pgfsetstrokecolor{currentstroke}%
\pgfsetdash{}{0pt}%
\pgfpathmoveto{\pgfqpoint{0.857590in}{1.557460in}}%
\pgfpathlineto{\pgfqpoint{1.030299in}{2.126617in}}%
\pgfpathlineto{\pgfqpoint{0.779523in}{1.760124in}}%
\pgfpathlineto{\pgfqpoint{0.857590in}{1.557460in}}%
\pgfpathclose%
\pgfusepath{fill}%
\end{pgfscope}%
\begin{pgfscope}%
\pgfpathrectangle{\pgfqpoint{0.254231in}{0.147348in}}{\pgfqpoint{2.735294in}{2.735294in}}%
\pgfusepath{clip}%
\pgfsetbuttcap%
\pgfsetroundjoin%
\definecolor{currentfill}{rgb}{0.089078,0.323920,0.089078}%
\pgfsetfillcolor{currentfill}%
\pgfsetlinewidth{0.000000pt}%
\definecolor{currentstroke}{rgb}{0.000000,0.000000,0.000000}%
\pgfsetstrokecolor{currentstroke}%
\pgfsetdash{}{0pt}%
\pgfpathmoveto{\pgfqpoint{2.538161in}{1.760124in}}%
\pgfpathlineto{\pgfqpoint{2.287385in}{2.126617in}}%
\pgfpathlineto{\pgfqpoint{2.460094in}{1.557460in}}%
\pgfpathlineto{\pgfqpoint{2.538161in}{1.760124in}}%
\pgfpathclose%
\pgfusepath{fill}%
\end{pgfscope}%
\begin{pgfscope}%
\pgfpathrectangle{\pgfqpoint{0.254231in}{0.147348in}}{\pgfqpoint{2.735294in}{2.735294in}}%
\pgfusepath{clip}%
\pgfsetbuttcap%
\pgfsetroundjoin%
\definecolor{currentfill}{rgb}{0.061754,0.224559,0.061754}%
\pgfsetfillcolor{currentfill}%
\pgfsetlinewidth{0.000000pt}%
\definecolor{currentstroke}{rgb}{0.000000,0.000000,0.000000}%
\pgfsetstrokecolor{currentstroke}%
\pgfsetdash{}{0pt}%
\pgfpathmoveto{\pgfqpoint{1.031671in}{1.028737in}}%
\pgfpathlineto{\pgfqpoint{1.177786in}{0.888724in}}%
\pgfpathlineto{\pgfqpoint{1.212441in}{1.302943in}}%
\pgfpathlineto{\pgfqpoint{1.031671in}{1.028737in}}%
\pgfpathclose%
\pgfusepath{fill}%
\end{pgfscope}%
\begin{pgfscope}%
\pgfpathrectangle{\pgfqpoint{0.254231in}{0.147348in}}{\pgfqpoint{2.735294in}{2.735294in}}%
\pgfusepath{clip}%
\pgfsetbuttcap%
\pgfsetroundjoin%
\definecolor{currentfill}{rgb}{0.061754,0.224559,0.061754}%
\pgfsetfillcolor{currentfill}%
\pgfsetlinewidth{0.000000pt}%
\definecolor{currentstroke}{rgb}{0.000000,0.000000,0.000000}%
\pgfsetstrokecolor{currentstroke}%
\pgfsetdash{}{0pt}%
\pgfpathmoveto{\pgfqpoint{2.105243in}{1.302943in}}%
\pgfpathlineto{\pgfqpoint{2.139898in}{0.888724in}}%
\pgfpathlineto{\pgfqpoint{2.286012in}{1.028737in}}%
\pgfpathlineto{\pgfqpoint{2.105243in}{1.302943in}}%
\pgfpathclose%
\pgfusepath{fill}%
\end{pgfscope}%
\begin{pgfscope}%
\pgfpathrectangle{\pgfqpoint{0.254231in}{0.147348in}}{\pgfqpoint{2.735294in}{2.735294in}}%
\pgfusepath{clip}%
\pgfsetbuttcap%
\pgfsetroundjoin%
\definecolor{currentfill}{rgb}{0.070984,0.258123,0.070984}%
\pgfsetfillcolor{currentfill}%
\pgfsetlinewidth{0.000000pt}%
\definecolor{currentstroke}{rgb}{0.000000,0.000000,0.000000}%
\pgfsetstrokecolor{currentstroke}%
\pgfsetdash{}{0pt}%
\pgfpathmoveto{\pgfqpoint{0.857590in}{1.557460in}}%
\pgfpathlineto{\pgfqpoint{0.848606in}{1.094945in}}%
\pgfpathlineto{\pgfqpoint{1.075030in}{1.526914in}}%
\pgfpathlineto{\pgfqpoint{0.857590in}{1.557460in}}%
\pgfpathclose%
\pgfusepath{fill}%
\end{pgfscope}%
\begin{pgfscope}%
\pgfpathrectangle{\pgfqpoint{0.254231in}{0.147348in}}{\pgfqpoint{2.735294in}{2.735294in}}%
\pgfusepath{clip}%
\pgfsetbuttcap%
\pgfsetroundjoin%
\definecolor{currentfill}{rgb}{0.070984,0.258123,0.070984}%
\pgfsetfillcolor{currentfill}%
\pgfsetlinewidth{0.000000pt}%
\definecolor{currentstroke}{rgb}{0.000000,0.000000,0.000000}%
\pgfsetstrokecolor{currentstroke}%
\pgfsetdash{}{0pt}%
\pgfpathmoveto{\pgfqpoint{2.242654in}{1.526914in}}%
\pgfpathlineto{\pgfqpoint{2.469078in}{1.094945in}}%
\pgfpathlineto{\pgfqpoint{2.460094in}{1.557460in}}%
\pgfpathlineto{\pgfqpoint{2.242654in}{1.526914in}}%
\pgfpathclose%
\pgfusepath{fill}%
\end{pgfscope}%
\begin{pgfscope}%
\pgfpathrectangle{\pgfqpoint{0.254231in}{0.147348in}}{\pgfqpoint{2.735294in}{2.735294in}}%
\pgfusepath{clip}%
\pgfsetbuttcap%
\pgfsetroundjoin%
\definecolor{currentfill}{rgb}{0.070885,0.257762,0.070885}%
\pgfsetfillcolor{currentfill}%
\pgfsetlinewidth{0.000000pt}%
\definecolor{currentstroke}{rgb}{0.000000,0.000000,0.000000}%
\pgfsetstrokecolor{currentstroke}%
\pgfsetdash{}{0pt}%
\pgfpathmoveto{\pgfqpoint{2.059003in}{0.974376in}}%
\pgfpathlineto{\pgfqpoint{2.139898in}{0.888724in}}%
\pgfpathlineto{\pgfqpoint{2.105243in}{1.302943in}}%
\pgfpathlineto{\pgfqpoint{2.059003in}{0.974376in}}%
\pgfpathclose%
\pgfusepath{fill}%
\end{pgfscope}%
\begin{pgfscope}%
\pgfpathrectangle{\pgfqpoint{0.254231in}{0.147348in}}{\pgfqpoint{2.735294in}{2.735294in}}%
\pgfusepath{clip}%
\pgfsetbuttcap%
\pgfsetroundjoin%
\definecolor{currentfill}{rgb}{0.070885,0.257762,0.070885}%
\pgfsetfillcolor{currentfill}%
\pgfsetlinewidth{0.000000pt}%
\definecolor{currentstroke}{rgb}{0.000000,0.000000,0.000000}%
\pgfsetstrokecolor{currentstroke}%
\pgfsetdash{}{0pt}%
\pgfpathmoveto{\pgfqpoint{1.212441in}{1.302943in}}%
\pgfpathlineto{\pgfqpoint{1.177786in}{0.888724in}}%
\pgfpathlineto{\pgfqpoint{1.258681in}{0.974376in}}%
\pgfpathlineto{\pgfqpoint{1.212441in}{1.302943in}}%
\pgfpathclose%
\pgfusepath{fill}%
\end{pgfscope}%
\begin{pgfscope}%
\pgfpathrectangle{\pgfqpoint{0.254231in}{0.147348in}}{\pgfqpoint{2.735294in}{2.735294in}}%
\pgfusepath{clip}%
\pgfsetbuttcap%
\pgfsetroundjoin%
\definecolor{currentfill}{rgb}{0.111651,0.406004,0.111651}%
\pgfsetfillcolor{currentfill}%
\pgfsetlinewidth{0.000000pt}%
\definecolor{currentstroke}{rgb}{0.000000,0.000000,0.000000}%
\pgfsetstrokecolor{currentstroke}%
\pgfsetdash{}{0pt}%
\pgfpathmoveto{\pgfqpoint{2.059934in}{2.141900in}}%
\pgfpathlineto{\pgfqpoint{2.141244in}{2.292690in}}%
\pgfpathlineto{\pgfqpoint{1.797134in}{2.150689in}}%
\pgfpathlineto{\pgfqpoint{2.059934in}{2.141900in}}%
\pgfpathclose%
\pgfusepath{fill}%
\end{pgfscope}%
\begin{pgfscope}%
\pgfpathrectangle{\pgfqpoint{0.254231in}{0.147348in}}{\pgfqpoint{2.735294in}{2.735294in}}%
\pgfusepath{clip}%
\pgfsetbuttcap%
\pgfsetroundjoin%
\definecolor{currentfill}{rgb}{0.111651,0.406004,0.111651}%
\pgfsetfillcolor{currentfill}%
\pgfsetlinewidth{0.000000pt}%
\definecolor{currentstroke}{rgb}{0.000000,0.000000,0.000000}%
\pgfsetstrokecolor{currentstroke}%
\pgfsetdash{}{0pt}%
\pgfpathmoveto{\pgfqpoint{1.520550in}{2.150689in}}%
\pgfpathlineto{\pgfqpoint{1.176440in}{2.292690in}}%
\pgfpathlineto{\pgfqpoint{1.257750in}{2.141900in}}%
\pgfpathlineto{\pgfqpoint{1.520550in}{2.150689in}}%
\pgfpathclose%
\pgfusepath{fill}%
\end{pgfscope}%
\begin{pgfscope}%
\pgfpathrectangle{\pgfqpoint{0.254231in}{0.147348in}}{\pgfqpoint{2.735294in}{2.735294in}}%
\pgfusepath{clip}%
\pgfsetbuttcap%
\pgfsetroundjoin%
\definecolor{currentfill}{rgb}{0.092193,0.335248,0.092193}%
\pgfsetfillcolor{currentfill}%
\pgfsetlinewidth{0.000000pt}%
\definecolor{currentstroke}{rgb}{0.000000,0.000000,0.000000}%
\pgfsetstrokecolor{currentstroke}%
\pgfsetdash{}{0pt}%
\pgfpathmoveto{\pgfqpoint{1.075030in}{1.526914in}}%
\pgfpathlineto{\pgfqpoint{1.030299in}{2.126617in}}%
\pgfpathlineto{\pgfqpoint{0.857590in}{1.557460in}}%
\pgfpathlineto{\pgfqpoint{1.075030in}{1.526914in}}%
\pgfpathclose%
\pgfusepath{fill}%
\end{pgfscope}%
\begin{pgfscope}%
\pgfpathrectangle{\pgfqpoint{0.254231in}{0.147348in}}{\pgfqpoint{2.735294in}{2.735294in}}%
\pgfusepath{clip}%
\pgfsetbuttcap%
\pgfsetroundjoin%
\definecolor{currentfill}{rgb}{0.092193,0.335248,0.092193}%
\pgfsetfillcolor{currentfill}%
\pgfsetlinewidth{0.000000pt}%
\definecolor{currentstroke}{rgb}{0.000000,0.000000,0.000000}%
\pgfsetstrokecolor{currentstroke}%
\pgfsetdash{}{0pt}%
\pgfpathmoveto{\pgfqpoint{2.460094in}{1.557460in}}%
\pgfpathlineto{\pgfqpoint{2.287385in}{2.126617in}}%
\pgfpathlineto{\pgfqpoint{2.242654in}{1.526914in}}%
\pgfpathlineto{\pgfqpoint{2.460094in}{1.557460in}}%
\pgfpathclose%
\pgfusepath{fill}%
\end{pgfscope}%
\begin{pgfscope}%
\pgfpathrectangle{\pgfqpoint{0.254231in}{0.147348in}}{\pgfqpoint{2.735294in}{2.735294in}}%
\pgfusepath{clip}%
\pgfsetbuttcap%
\pgfsetroundjoin%
\definecolor{currentfill}{rgb}{0.060562,0.220227,0.060562}%
\pgfsetfillcolor{currentfill}%
\pgfsetlinewidth{0.000000pt}%
\definecolor{currentstroke}{rgb}{0.000000,0.000000,0.000000}%
\pgfsetstrokecolor{currentstroke}%
\pgfsetdash{}{0pt}%
\pgfpathmoveto{\pgfqpoint{2.059003in}{0.974376in}}%
\pgfpathlineto{\pgfqpoint{1.814008in}{1.277996in}}%
\pgfpathlineto{\pgfqpoint{1.909619in}{0.846223in}}%
\pgfpathlineto{\pgfqpoint{2.059003in}{0.974376in}}%
\pgfpathclose%
\pgfusepath{fill}%
\end{pgfscope}%
\begin{pgfscope}%
\pgfpathrectangle{\pgfqpoint{0.254231in}{0.147348in}}{\pgfqpoint{2.735294in}{2.735294in}}%
\pgfusepath{clip}%
\pgfsetbuttcap%
\pgfsetroundjoin%
\definecolor{currentfill}{rgb}{0.060562,0.220227,0.060562}%
\pgfsetfillcolor{currentfill}%
\pgfsetlinewidth{0.000000pt}%
\definecolor{currentstroke}{rgb}{0.000000,0.000000,0.000000}%
\pgfsetstrokecolor{currentstroke}%
\pgfsetdash{}{0pt}%
\pgfpathmoveto{\pgfqpoint{1.408065in}{0.846223in}}%
\pgfpathlineto{\pgfqpoint{1.503676in}{1.277996in}}%
\pgfpathlineto{\pgfqpoint{1.258681in}{0.974376in}}%
\pgfpathlineto{\pgfqpoint{1.408065in}{0.846223in}}%
\pgfpathclose%
\pgfusepath{fill}%
\end{pgfscope}%
\begin{pgfscope}%
\pgfpathrectangle{\pgfqpoint{0.254231in}{0.147348in}}{\pgfqpoint{2.735294in}{2.735294in}}%
\pgfusepath{clip}%
\pgfsetbuttcap%
\pgfsetroundjoin%
\definecolor{currentfill}{rgb}{0.097285,0.353762,0.097285}%
\pgfsetfillcolor{currentfill}%
\pgfsetlinewidth{0.000000pt}%
\definecolor{currentstroke}{rgb}{0.000000,0.000000,0.000000}%
\pgfsetstrokecolor{currentstroke}%
\pgfsetdash{}{0pt}%
\pgfpathmoveto{\pgfqpoint{1.257750in}{2.141900in}}%
\pgfpathlineto{\pgfqpoint{1.030299in}{2.126617in}}%
\pgfpathlineto{\pgfqpoint{1.366689in}{1.952591in}}%
\pgfpathlineto{\pgfqpoint{1.257750in}{2.141900in}}%
\pgfpathclose%
\pgfusepath{fill}%
\end{pgfscope}%
\begin{pgfscope}%
\pgfpathrectangle{\pgfqpoint{0.254231in}{0.147348in}}{\pgfqpoint{2.735294in}{2.735294in}}%
\pgfusepath{clip}%
\pgfsetbuttcap%
\pgfsetroundjoin%
\definecolor{currentfill}{rgb}{0.097285,0.353762,0.097285}%
\pgfsetfillcolor{currentfill}%
\pgfsetlinewidth{0.000000pt}%
\definecolor{currentstroke}{rgb}{0.000000,0.000000,0.000000}%
\pgfsetstrokecolor{currentstroke}%
\pgfsetdash{}{0pt}%
\pgfpathmoveto{\pgfqpoint{1.950995in}{1.952591in}}%
\pgfpathlineto{\pgfqpoint{2.287385in}{2.126617in}}%
\pgfpathlineto{\pgfqpoint{2.059934in}{2.141900in}}%
\pgfpathlineto{\pgfqpoint{1.950995in}{1.952591in}}%
\pgfpathclose%
\pgfusepath{fill}%
\end{pgfscope}%
\begin{pgfscope}%
\pgfpathrectangle{\pgfqpoint{0.254231in}{0.147348in}}{\pgfqpoint{2.735294in}{2.735294in}}%
\pgfusepath{clip}%
\pgfsetbuttcap%
\pgfsetroundjoin%
\definecolor{currentfill}{rgb}{0.067497,0.245443,0.067497}%
\pgfsetfillcolor{currentfill}%
\pgfsetlinewidth{0.000000pt}%
\definecolor{currentstroke}{rgb}{0.000000,0.000000,0.000000}%
\pgfsetstrokecolor{currentstroke}%
\pgfsetdash{}{0pt}%
\pgfpathmoveto{\pgfqpoint{1.408065in}{0.846223in}}%
\pgfpathlineto{\pgfqpoint{1.520882in}{0.943120in}}%
\pgfpathlineto{\pgfqpoint{1.503676in}{1.277996in}}%
\pgfpathlineto{\pgfqpoint{1.408065in}{0.846223in}}%
\pgfpathclose%
\pgfusepath{fill}%
\end{pgfscope}%
\begin{pgfscope}%
\pgfpathrectangle{\pgfqpoint{0.254231in}{0.147348in}}{\pgfqpoint{2.735294in}{2.735294in}}%
\pgfusepath{clip}%
\pgfsetbuttcap%
\pgfsetroundjoin%
\definecolor{currentfill}{rgb}{0.067497,0.245443,0.067497}%
\pgfsetfillcolor{currentfill}%
\pgfsetlinewidth{0.000000pt}%
\definecolor{currentstroke}{rgb}{0.000000,0.000000,0.000000}%
\pgfsetstrokecolor{currentstroke}%
\pgfsetdash{}{0pt}%
\pgfpathmoveto{\pgfqpoint{1.814008in}{1.277996in}}%
\pgfpathlineto{\pgfqpoint{1.796802in}{0.943120in}}%
\pgfpathlineto{\pgfqpoint{1.909619in}{0.846223in}}%
\pgfpathlineto{\pgfqpoint{1.814008in}{1.277996in}}%
\pgfpathclose%
\pgfusepath{fill}%
\end{pgfscope}%
\begin{pgfscope}%
\pgfpathrectangle{\pgfqpoint{0.254231in}{0.147348in}}{\pgfqpoint{2.735294in}{2.735294in}}%
\pgfusepath{clip}%
\pgfsetbuttcap%
\pgfsetroundjoin%
\definecolor{currentfill}{rgb}{0.065434,0.237940,0.065434}%
\pgfsetfillcolor{currentfill}%
\pgfsetlinewidth{0.000000pt}%
\definecolor{currentstroke}{rgb}{0.000000,0.000000,0.000000}%
\pgfsetstrokecolor{currentstroke}%
\pgfsetdash{}{0pt}%
\pgfpathmoveto{\pgfqpoint{1.658842in}{0.830831in}}%
\pgfpathlineto{\pgfqpoint{1.503676in}{1.277996in}}%
\pgfpathlineto{\pgfqpoint{1.520882in}{0.943120in}}%
\pgfpathlineto{\pgfqpoint{1.658842in}{0.830831in}}%
\pgfpathclose%
\pgfusepath{fill}%
\end{pgfscope}%
\begin{pgfscope}%
\pgfpathrectangle{\pgfqpoint{0.254231in}{0.147348in}}{\pgfqpoint{2.735294in}{2.735294in}}%
\pgfusepath{clip}%
\pgfsetbuttcap%
\pgfsetroundjoin%
\definecolor{currentfill}{rgb}{0.065434,0.237940,0.065434}%
\pgfsetfillcolor{currentfill}%
\pgfsetlinewidth{0.000000pt}%
\definecolor{currentstroke}{rgb}{0.000000,0.000000,0.000000}%
\pgfsetstrokecolor{currentstroke}%
\pgfsetdash{}{0pt}%
\pgfpathmoveto{\pgfqpoint{1.796802in}{0.943120in}}%
\pgfpathlineto{\pgfqpoint{1.814008in}{1.277996in}}%
\pgfpathlineto{\pgfqpoint{1.658842in}{0.830831in}}%
\pgfpathlineto{\pgfqpoint{1.796802in}{0.943120in}}%
\pgfpathclose%
\pgfusepath{fill}%
\end{pgfscope}%
\begin{pgfscope}%
\pgfpathrectangle{\pgfqpoint{0.254231in}{0.147348in}}{\pgfqpoint{2.735294in}{2.735294in}}%
\pgfusepath{clip}%
\pgfsetbuttcap%
\pgfsetroundjoin%
\definecolor{currentfill}{rgb}{0.073593,0.267612,0.073593}%
\pgfsetfillcolor{currentfill}%
\pgfsetlinewidth{0.000000pt}%
\definecolor{currentstroke}{rgb}{0.000000,0.000000,0.000000}%
\pgfsetstrokecolor{currentstroke}%
\pgfsetdash{}{0pt}%
\pgfpathmoveto{\pgfqpoint{1.031671in}{1.028737in}}%
\pgfpathlineto{\pgfqpoint{1.212441in}{1.302943in}}%
\pgfpathlineto{\pgfqpoint{1.075030in}{1.526914in}}%
\pgfpathlineto{\pgfqpoint{1.031671in}{1.028737in}}%
\pgfpathclose%
\pgfusepath{fill}%
\end{pgfscope}%
\begin{pgfscope}%
\pgfpathrectangle{\pgfqpoint{0.254231in}{0.147348in}}{\pgfqpoint{2.735294in}{2.735294in}}%
\pgfusepath{clip}%
\pgfsetbuttcap%
\pgfsetroundjoin%
\definecolor{currentfill}{rgb}{0.073593,0.267612,0.073593}%
\pgfsetfillcolor{currentfill}%
\pgfsetlinewidth{0.000000pt}%
\definecolor{currentstroke}{rgb}{0.000000,0.000000,0.000000}%
\pgfsetstrokecolor{currentstroke}%
\pgfsetdash{}{0pt}%
\pgfpathmoveto{\pgfqpoint{2.242654in}{1.526914in}}%
\pgfpathlineto{\pgfqpoint{2.105243in}{1.302943in}}%
\pgfpathlineto{\pgfqpoint{2.286012in}{1.028737in}}%
\pgfpathlineto{\pgfqpoint{2.242654in}{1.526914in}}%
\pgfpathclose%
\pgfusepath{fill}%
\end{pgfscope}%
\begin{pgfscope}%
\pgfpathrectangle{\pgfqpoint{0.254231in}{0.147348in}}{\pgfqpoint{2.735294in}{2.735294in}}%
\pgfusepath{clip}%
\pgfsetbuttcap%
\pgfsetroundjoin%
\definecolor{currentfill}{rgb}{0.091915,0.334238,0.091915}%
\pgfsetfillcolor{currentfill}%
\pgfsetlinewidth{0.000000pt}%
\definecolor{currentstroke}{rgb}{0.000000,0.000000,0.000000}%
\pgfsetstrokecolor{currentstroke}%
\pgfsetdash{}{0pt}%
\pgfpathmoveto{\pgfqpoint{1.075030in}{1.526914in}}%
\pgfpathlineto{\pgfqpoint{1.366689in}{1.952591in}}%
\pgfpathlineto{\pgfqpoint{1.030299in}{2.126617in}}%
\pgfpathlineto{\pgfqpoint{1.075030in}{1.526914in}}%
\pgfpathclose%
\pgfusepath{fill}%
\end{pgfscope}%
\begin{pgfscope}%
\pgfpathrectangle{\pgfqpoint{0.254231in}{0.147348in}}{\pgfqpoint{2.735294in}{2.735294in}}%
\pgfusepath{clip}%
\pgfsetbuttcap%
\pgfsetroundjoin%
\definecolor{currentfill}{rgb}{0.091915,0.334238,0.091915}%
\pgfsetfillcolor{currentfill}%
\pgfsetlinewidth{0.000000pt}%
\definecolor{currentstroke}{rgb}{0.000000,0.000000,0.000000}%
\pgfsetstrokecolor{currentstroke}%
\pgfsetdash{}{0pt}%
\pgfpathmoveto{\pgfqpoint{2.287385in}{2.126617in}}%
\pgfpathlineto{\pgfqpoint{1.950995in}{1.952591in}}%
\pgfpathlineto{\pgfqpoint{2.242654in}{1.526914in}}%
\pgfpathlineto{\pgfqpoint{2.287385in}{2.126617in}}%
\pgfpathclose%
\pgfusepath{fill}%
\end{pgfscope}%
\begin{pgfscope}%
\pgfpathrectangle{\pgfqpoint{0.254231in}{0.147348in}}{\pgfqpoint{2.735294in}{2.735294in}}%
\pgfusepath{clip}%
\pgfsetbuttcap%
\pgfsetroundjoin%
\definecolor{currentfill}{rgb}{0.101759,0.370033,0.101759}%
\pgfsetfillcolor{currentfill}%
\pgfsetlinewidth{0.000000pt}%
\definecolor{currentstroke}{rgb}{0.000000,0.000000,0.000000}%
\pgfsetstrokecolor{currentstroke}%
\pgfsetdash{}{0pt}%
\pgfpathmoveto{\pgfqpoint{1.366689in}{1.952591in}}%
\pgfpathlineto{\pgfqpoint{1.520550in}{2.150689in}}%
\pgfpathlineto{\pgfqpoint{1.257750in}{2.141900in}}%
\pgfpathlineto{\pgfqpoint{1.366689in}{1.952591in}}%
\pgfpathclose%
\pgfusepath{fill}%
\end{pgfscope}%
\begin{pgfscope}%
\pgfpathrectangle{\pgfqpoint{0.254231in}{0.147348in}}{\pgfqpoint{2.735294in}{2.735294in}}%
\pgfusepath{clip}%
\pgfsetbuttcap%
\pgfsetroundjoin%
\definecolor{currentfill}{rgb}{0.101759,0.370033,0.101759}%
\pgfsetfillcolor{currentfill}%
\pgfsetlinewidth{0.000000pt}%
\definecolor{currentstroke}{rgb}{0.000000,0.000000,0.000000}%
\pgfsetstrokecolor{currentstroke}%
\pgfsetdash{}{0pt}%
\pgfpathmoveto{\pgfqpoint{2.059934in}{2.141900in}}%
\pgfpathlineto{\pgfqpoint{1.797134in}{2.150689in}}%
\pgfpathlineto{\pgfqpoint{1.950995in}{1.952591in}}%
\pgfpathlineto{\pgfqpoint{2.059934in}{2.141900in}}%
\pgfpathclose%
\pgfusepath{fill}%
\end{pgfscope}%
\begin{pgfscope}%
\pgfpathrectangle{\pgfqpoint{0.254231in}{0.147348in}}{\pgfqpoint{2.735294in}{2.735294in}}%
\pgfusepath{clip}%
\pgfsetbuttcap%
\pgfsetroundjoin%
\definecolor{currentfill}{rgb}{0.101677,0.369734,0.101677}%
\pgfsetfillcolor{currentfill}%
\pgfsetlinewidth{0.000000pt}%
\definecolor{currentstroke}{rgb}{0.000000,0.000000,0.000000}%
\pgfsetstrokecolor{currentstroke}%
\pgfsetdash{}{0pt}%
\pgfpathmoveto{\pgfqpoint{1.658842in}{1.953918in}}%
\pgfpathlineto{\pgfqpoint{1.797134in}{2.150689in}}%
\pgfpathlineto{\pgfqpoint{1.520550in}{2.150689in}}%
\pgfpathlineto{\pgfqpoint{1.658842in}{1.953918in}}%
\pgfpathclose%
\pgfusepath{fill}%
\end{pgfscope}%
\begin{pgfscope}%
\pgfpathrectangle{\pgfqpoint{0.254231in}{0.147348in}}{\pgfqpoint{2.735294in}{2.735294in}}%
\pgfusepath{clip}%
\pgfsetbuttcap%
\pgfsetroundjoin%
\definecolor{currentfill}{rgb}{0.065035,0.236492,0.065035}%
\pgfsetfillcolor{currentfill}%
\pgfsetlinewidth{0.000000pt}%
\definecolor{currentstroke}{rgb}{0.000000,0.000000,0.000000}%
\pgfsetstrokecolor{currentstroke}%
\pgfsetdash{}{0pt}%
\pgfpathmoveto{\pgfqpoint{1.258681in}{0.974376in}}%
\pgfpathlineto{\pgfqpoint{1.348371in}{1.504068in}}%
\pgfpathlineto{\pgfqpoint{1.212441in}{1.302943in}}%
\pgfpathlineto{\pgfqpoint{1.258681in}{0.974376in}}%
\pgfpathclose%
\pgfusepath{fill}%
\end{pgfscope}%
\begin{pgfscope}%
\pgfpathrectangle{\pgfqpoint{0.254231in}{0.147348in}}{\pgfqpoint{2.735294in}{2.735294in}}%
\pgfusepath{clip}%
\pgfsetbuttcap%
\pgfsetroundjoin%
\definecolor{currentfill}{rgb}{0.065035,0.236492,0.065035}%
\pgfsetfillcolor{currentfill}%
\pgfsetlinewidth{0.000000pt}%
\definecolor{currentstroke}{rgb}{0.000000,0.000000,0.000000}%
\pgfsetstrokecolor{currentstroke}%
\pgfsetdash{}{0pt}%
\pgfpathmoveto{\pgfqpoint{2.105243in}{1.302943in}}%
\pgfpathlineto{\pgfqpoint{1.969313in}{1.504068in}}%
\pgfpathlineto{\pgfqpoint{2.059003in}{0.974376in}}%
\pgfpathlineto{\pgfqpoint{2.105243in}{1.302943in}}%
\pgfpathclose%
\pgfusepath{fill}%
\end{pgfscope}%
\begin{pgfscope}%
\pgfpathrectangle{\pgfqpoint{0.254231in}{0.147348in}}{\pgfqpoint{2.735294in}{2.735294in}}%
\pgfusepath{clip}%
\pgfsetbuttcap%
\pgfsetroundjoin%
\definecolor{currentfill}{rgb}{0.066446,0.241622,0.066446}%
\pgfsetfillcolor{currentfill}%
\pgfsetlinewidth{0.000000pt}%
\definecolor{currentstroke}{rgb}{0.000000,0.000000,0.000000}%
\pgfsetstrokecolor{currentstroke}%
\pgfsetdash{}{0pt}%
\pgfpathmoveto{\pgfqpoint{1.503676in}{1.277996in}}%
\pgfpathlineto{\pgfqpoint{1.658842in}{0.830831in}}%
\pgfpathlineto{\pgfqpoint{1.658842in}{1.495457in}}%
\pgfpathlineto{\pgfqpoint{1.503676in}{1.277996in}}%
\pgfpathclose%
\pgfusepath{fill}%
\end{pgfscope}%
\begin{pgfscope}%
\pgfpathrectangle{\pgfqpoint{0.254231in}{0.147348in}}{\pgfqpoint{2.735294in}{2.735294in}}%
\pgfusepath{clip}%
\pgfsetbuttcap%
\pgfsetroundjoin%
\definecolor{currentfill}{rgb}{0.066446,0.241622,0.066446}%
\pgfsetfillcolor{currentfill}%
\pgfsetlinewidth{0.000000pt}%
\definecolor{currentstroke}{rgb}{0.000000,0.000000,0.000000}%
\pgfsetstrokecolor{currentstroke}%
\pgfsetdash{}{0pt}%
\pgfpathmoveto{\pgfqpoint{1.658842in}{1.495457in}}%
\pgfpathlineto{\pgfqpoint{1.658842in}{0.830831in}}%
\pgfpathlineto{\pgfqpoint{1.814008in}{1.277996in}}%
\pgfpathlineto{\pgfqpoint{1.658842in}{1.495457in}}%
\pgfpathclose%
\pgfusepath{fill}%
\end{pgfscope}%
\begin{pgfscope}%
\pgfpathrectangle{\pgfqpoint{0.254231in}{0.147348in}}{\pgfqpoint{2.735294in}{2.735294in}}%
\pgfusepath{clip}%
\pgfsetbuttcap%
\pgfsetroundjoin%
\definecolor{currentfill}{rgb}{0.098306,0.357475,0.098306}%
\pgfsetfillcolor{currentfill}%
\pgfsetlinewidth{0.000000pt}%
\definecolor{currentstroke}{rgb}{0.000000,0.000000,0.000000}%
\pgfsetstrokecolor{currentstroke}%
\pgfsetdash{}{0pt}%
\pgfpathmoveto{\pgfqpoint{1.658842in}{1.953918in}}%
\pgfpathlineto{\pgfqpoint{1.520550in}{2.150689in}}%
\pgfpathlineto{\pgfqpoint{1.366689in}{1.952591in}}%
\pgfpathlineto{\pgfqpoint{1.658842in}{1.953918in}}%
\pgfpathclose%
\pgfusepath{fill}%
\end{pgfscope}%
\begin{pgfscope}%
\pgfpathrectangle{\pgfqpoint{0.254231in}{0.147348in}}{\pgfqpoint{2.735294in}{2.735294in}}%
\pgfusepath{clip}%
\pgfsetbuttcap%
\pgfsetroundjoin%
\definecolor{currentfill}{rgb}{0.098306,0.357475,0.098306}%
\pgfsetfillcolor{currentfill}%
\pgfsetlinewidth{0.000000pt}%
\definecolor{currentstroke}{rgb}{0.000000,0.000000,0.000000}%
\pgfsetstrokecolor{currentstroke}%
\pgfsetdash{}{0pt}%
\pgfpathmoveto{\pgfqpoint{1.950995in}{1.952591in}}%
\pgfpathlineto{\pgfqpoint{1.797134in}{2.150689in}}%
\pgfpathlineto{\pgfqpoint{1.658842in}{1.953918in}}%
\pgfpathlineto{\pgfqpoint{1.950995in}{1.952591in}}%
\pgfpathclose%
\pgfusepath{fill}%
\end{pgfscope}%
\begin{pgfscope}%
\pgfpathrectangle{\pgfqpoint{0.254231in}{0.147348in}}{\pgfqpoint{2.735294in}{2.735294in}}%
\pgfusepath{clip}%
\pgfsetbuttcap%
\pgfsetroundjoin%
\definecolor{currentfill}{rgb}{0.070209,0.255305,0.070209}%
\pgfsetfillcolor{currentfill}%
\pgfsetlinewidth{0.000000pt}%
\definecolor{currentstroke}{rgb}{0.000000,0.000000,0.000000}%
\pgfsetstrokecolor{currentstroke}%
\pgfsetdash{}{0pt}%
\pgfpathmoveto{\pgfqpoint{1.258681in}{0.974376in}}%
\pgfpathlineto{\pgfqpoint{1.503676in}{1.277996in}}%
\pgfpathlineto{\pgfqpoint{1.348371in}{1.504068in}}%
\pgfpathlineto{\pgfqpoint{1.258681in}{0.974376in}}%
\pgfpathclose%
\pgfusepath{fill}%
\end{pgfscope}%
\begin{pgfscope}%
\pgfpathrectangle{\pgfqpoint{0.254231in}{0.147348in}}{\pgfqpoint{2.735294in}{2.735294in}}%
\pgfusepath{clip}%
\pgfsetbuttcap%
\pgfsetroundjoin%
\definecolor{currentfill}{rgb}{0.070209,0.255305,0.070209}%
\pgfsetfillcolor{currentfill}%
\pgfsetlinewidth{0.000000pt}%
\definecolor{currentstroke}{rgb}{0.000000,0.000000,0.000000}%
\pgfsetstrokecolor{currentstroke}%
\pgfsetdash{}{0pt}%
\pgfpathmoveto{\pgfqpoint{1.969313in}{1.504068in}}%
\pgfpathlineto{\pgfqpoint{1.814008in}{1.277996in}}%
\pgfpathlineto{\pgfqpoint{2.059003in}{0.974376in}}%
\pgfpathlineto{\pgfqpoint{1.969313in}{1.504068in}}%
\pgfpathclose%
\pgfusepath{fill}%
\end{pgfscope}%
\begin{pgfscope}%
\pgfpathrectangle{\pgfqpoint{0.254231in}{0.147348in}}{\pgfqpoint{2.735294in}{2.735294in}}%
\pgfusepath{clip}%
\pgfsetbuttcap%
\pgfsetroundjoin%
\definecolor{currentfill}{rgb}{0.087398,0.317812,0.087398}%
\pgfsetfillcolor{currentfill}%
\pgfsetlinewidth{0.000000pt}%
\definecolor{currentstroke}{rgb}{0.000000,0.000000,0.000000}%
\pgfsetstrokecolor{currentstroke}%
\pgfsetdash{}{0pt}%
\pgfpathmoveto{\pgfqpoint{2.242654in}{1.526914in}}%
\pgfpathlineto{\pgfqpoint{1.950995in}{1.952591in}}%
\pgfpathlineto{\pgfqpoint{1.969313in}{1.504068in}}%
\pgfpathlineto{\pgfqpoint{2.242654in}{1.526914in}}%
\pgfpathclose%
\pgfusepath{fill}%
\end{pgfscope}%
\begin{pgfscope}%
\pgfpathrectangle{\pgfqpoint{0.254231in}{0.147348in}}{\pgfqpoint{2.735294in}{2.735294in}}%
\pgfusepath{clip}%
\pgfsetbuttcap%
\pgfsetroundjoin%
\definecolor{currentfill}{rgb}{0.087398,0.317812,0.087398}%
\pgfsetfillcolor{currentfill}%
\pgfsetlinewidth{0.000000pt}%
\definecolor{currentstroke}{rgb}{0.000000,0.000000,0.000000}%
\pgfsetstrokecolor{currentstroke}%
\pgfsetdash{}{0pt}%
\pgfpathmoveto{\pgfqpoint{1.348371in}{1.504068in}}%
\pgfpathlineto{\pgfqpoint{1.366689in}{1.952591in}}%
\pgfpathlineto{\pgfqpoint{1.075030in}{1.526914in}}%
\pgfpathlineto{\pgfqpoint{1.348371in}{1.504068in}}%
\pgfpathclose%
\pgfusepath{fill}%
\end{pgfscope}%
\begin{pgfscope}%
\pgfpathrectangle{\pgfqpoint{0.254231in}{0.147348in}}{\pgfqpoint{2.735294in}{2.735294in}}%
\pgfusepath{clip}%
\pgfsetbuttcap%
\pgfsetroundjoin%
\definecolor{currentfill}{rgb}{0.075994,0.276341,0.075994}%
\pgfsetfillcolor{currentfill}%
\pgfsetlinewidth{0.000000pt}%
\definecolor{currentstroke}{rgb}{0.000000,0.000000,0.000000}%
\pgfsetstrokecolor{currentstroke}%
\pgfsetdash{}{0pt}%
\pgfpathmoveto{\pgfqpoint{1.075030in}{1.526914in}}%
\pgfpathlineto{\pgfqpoint{1.212441in}{1.302943in}}%
\pgfpathlineto{\pgfqpoint{1.348371in}{1.504068in}}%
\pgfpathlineto{\pgfqpoint{1.075030in}{1.526914in}}%
\pgfpathclose%
\pgfusepath{fill}%
\end{pgfscope}%
\begin{pgfscope}%
\pgfpathrectangle{\pgfqpoint{0.254231in}{0.147348in}}{\pgfqpoint{2.735294in}{2.735294in}}%
\pgfusepath{clip}%
\pgfsetbuttcap%
\pgfsetroundjoin%
\definecolor{currentfill}{rgb}{0.075994,0.276341,0.075994}%
\pgfsetfillcolor{currentfill}%
\pgfsetlinewidth{0.000000pt}%
\definecolor{currentstroke}{rgb}{0.000000,0.000000,0.000000}%
\pgfsetstrokecolor{currentstroke}%
\pgfsetdash{}{0pt}%
\pgfpathmoveto{\pgfqpoint{1.969313in}{1.504068in}}%
\pgfpathlineto{\pgfqpoint{2.105243in}{1.302943in}}%
\pgfpathlineto{\pgfqpoint{2.242654in}{1.526914in}}%
\pgfpathlineto{\pgfqpoint{1.969313in}{1.504068in}}%
\pgfpathclose%
\pgfusepath{fill}%
\end{pgfscope}%
\begin{pgfscope}%
\pgfpathrectangle{\pgfqpoint{0.254231in}{0.147348in}}{\pgfqpoint{2.735294in}{2.735294in}}%
\pgfusepath{clip}%
\pgfsetbuttcap%
\pgfsetroundjoin%
\definecolor{currentfill}{rgb}{0.086061,0.312950,0.086061}%
\pgfsetfillcolor{currentfill}%
\pgfsetlinewidth{0.000000pt}%
\definecolor{currentstroke}{rgb}{0.000000,0.000000,0.000000}%
\pgfsetstrokecolor{currentstroke}%
\pgfsetdash{}{0pt}%
\pgfpathmoveto{\pgfqpoint{1.366689in}{1.952591in}}%
\pgfpathlineto{\pgfqpoint{1.348371in}{1.504068in}}%
\pgfpathlineto{\pgfqpoint{1.658842in}{1.953918in}}%
\pgfpathlineto{\pgfqpoint{1.366689in}{1.952591in}}%
\pgfpathclose%
\pgfusepath{fill}%
\end{pgfscope}%
\begin{pgfscope}%
\pgfpathrectangle{\pgfqpoint{0.254231in}{0.147348in}}{\pgfqpoint{2.735294in}{2.735294in}}%
\pgfusepath{clip}%
\pgfsetbuttcap%
\pgfsetroundjoin%
\definecolor{currentfill}{rgb}{0.086061,0.312950,0.086061}%
\pgfsetfillcolor{currentfill}%
\pgfsetlinewidth{0.000000pt}%
\definecolor{currentstroke}{rgb}{0.000000,0.000000,0.000000}%
\pgfsetstrokecolor{currentstroke}%
\pgfsetdash{}{0pt}%
\pgfpathmoveto{\pgfqpoint{1.658842in}{1.953918in}}%
\pgfpathlineto{\pgfqpoint{1.969313in}{1.504068in}}%
\pgfpathlineto{\pgfqpoint{1.950995in}{1.952591in}}%
\pgfpathlineto{\pgfqpoint{1.658842in}{1.953918in}}%
\pgfpathclose%
\pgfusepath{fill}%
\end{pgfscope}%
\begin{pgfscope}%
\pgfpathrectangle{\pgfqpoint{0.254231in}{0.147348in}}{\pgfqpoint{2.735294in}{2.735294in}}%
\pgfusepath{clip}%
\pgfsetbuttcap%
\pgfsetroundjoin%
\definecolor{currentfill}{rgb}{0.086258,0.313666,0.086258}%
\pgfsetfillcolor{currentfill}%
\pgfsetlinewidth{0.000000pt}%
\definecolor{currentstroke}{rgb}{0.000000,0.000000,0.000000}%
\pgfsetstrokecolor{currentstroke}%
\pgfsetdash{}{0pt}%
\pgfpathmoveto{\pgfqpoint{1.658842in}{1.495457in}}%
\pgfpathlineto{\pgfqpoint{1.658842in}{1.953918in}}%
\pgfpathlineto{\pgfqpoint{1.348371in}{1.504068in}}%
\pgfpathlineto{\pgfqpoint{1.658842in}{1.495457in}}%
\pgfpathclose%
\pgfusepath{fill}%
\end{pgfscope}%
\begin{pgfscope}%
\pgfpathrectangle{\pgfqpoint{0.254231in}{0.147348in}}{\pgfqpoint{2.735294in}{2.735294in}}%
\pgfusepath{clip}%
\pgfsetbuttcap%
\pgfsetroundjoin%
\definecolor{currentfill}{rgb}{0.086258,0.313666,0.086258}%
\pgfsetfillcolor{currentfill}%
\pgfsetlinewidth{0.000000pt}%
\definecolor{currentstroke}{rgb}{0.000000,0.000000,0.000000}%
\pgfsetstrokecolor{currentstroke}%
\pgfsetdash{}{0pt}%
\pgfpathmoveto{\pgfqpoint{1.969313in}{1.504068in}}%
\pgfpathlineto{\pgfqpoint{1.658842in}{1.953918in}}%
\pgfpathlineto{\pgfqpoint{1.658842in}{1.495457in}}%
\pgfpathlineto{\pgfqpoint{1.969313in}{1.504068in}}%
\pgfpathclose%
\pgfusepath{fill}%
\end{pgfscope}%
\begin{pgfscope}%
\pgfpathrectangle{\pgfqpoint{0.254231in}{0.147348in}}{\pgfqpoint{2.735294in}{2.735294in}}%
\pgfusepath{clip}%
\pgfsetbuttcap%
\pgfsetroundjoin%
\definecolor{currentfill}{rgb}{0.074668,0.271519,0.074668}%
\pgfsetfillcolor{currentfill}%
\pgfsetlinewidth{0.000000pt}%
\definecolor{currentstroke}{rgb}{0.000000,0.000000,0.000000}%
\pgfsetstrokecolor{currentstroke}%
\pgfsetdash{}{0pt}%
\pgfpathmoveto{\pgfqpoint{1.348371in}{1.504068in}}%
\pgfpathlineto{\pgfqpoint{1.503676in}{1.277996in}}%
\pgfpathlineto{\pgfqpoint{1.658842in}{1.495457in}}%
\pgfpathlineto{\pgfqpoint{1.348371in}{1.504068in}}%
\pgfpathclose%
\pgfusepath{fill}%
\end{pgfscope}%
\begin{pgfscope}%
\pgfpathrectangle{\pgfqpoint{0.254231in}{0.147348in}}{\pgfqpoint{2.735294in}{2.735294in}}%
\pgfusepath{clip}%
\pgfsetbuttcap%
\pgfsetroundjoin%
\definecolor{currentfill}{rgb}{0.074668,0.271519,0.074668}%
\pgfsetfillcolor{currentfill}%
\pgfsetlinewidth{0.000000pt}%
\definecolor{currentstroke}{rgb}{0.000000,0.000000,0.000000}%
\pgfsetstrokecolor{currentstroke}%
\pgfsetdash{}{0pt}%
\pgfpathmoveto{\pgfqpoint{1.658842in}{1.495457in}}%
\pgfpathlineto{\pgfqpoint{1.814008in}{1.277996in}}%
\pgfpathlineto{\pgfqpoint{1.969313in}{1.504068in}}%
\pgfpathlineto{\pgfqpoint{1.658842in}{1.495457in}}%
\pgfpathclose%
\pgfusepath{fill}%
\end{pgfscope}%
\end{pgfpicture}%
\makeatother%
\endgroup%
}
  \caption{From left to right: An opaque Pareto front; a translucent Pareto front 
  showing the interior points above a sub-optimal front; and the sub-optimal front 
  hiding the interior points from a different angle.}
  \label{fig:3d-mga}
\end{figure}

\subsection{Farthest First Traversal}
Previous studies emphasized that a key benefit of \ac{mga} is
obtaining a set of near-optimal solutions that are maximally different in design
space \cite{decarolis_modelling_2016, yue_review_2018-1}. Instead of
constraining a linear program to guarantee a maximally-different solution set.
\ac{osier} implements a ``greedy'' search algorithm in decision space to find
solutions \cite{hochbaum_best_1985}. The algorithm is
\begin{enumerate}
  \item Calculate a distance matrix $\mathcal{D}$ containing the distances among
  all points. 
  \item Choose an initial point (either randomly or a specified point).
  \item Step towards the point that has the maximum distance from the current
  point that has not already been visited.
  \item Continue until the desired number of points has been reached.
\end{enumerate}

Figure \ref{fig:mga-fft} demonstrates \ac{mga} with
``farthest-first-traversal.'' The left plot in Figure \ref{fig:mga-fft} shows
the objective space for the same problem as in Figure \ref{fig:nd-mga}. The plot
on the right shows the corresponding design space. Both plots show the Pareto
front as red dots. The colored dots represent the points selected by the ``farthest-first-traversal''
algorithm. These points are connected by similarly colored arrows in the design space plot. The arrow
color corresponds to the color of the next selected point (i.e., the next farthest point).

\begin{figure}[H]
  \centering
  \resizebox{1\columnwidth}{!}{%% Creator: Matplotlib, PGF backend
%%
%% To include the figure in your LaTeX document, write
%%   \input{<filename>.pgf}
%%
%% Make sure the required packages are loaded in your preamble
%%   \usepackage{pgf}
%%
%% Also ensure that all the required font packages are loaded; for instance,
%% the lmodern package is sometimes necessary when using math font.
%%   \usepackage{lmodern}
%%
%% Figures using additional raster images can only be included by \input if
%% they are in the same directory as the main LaTeX file. For loading figures
%% from other directories you can use the `import` package
%%   \usepackage{import}
%%
%% and then include the figures with
%%   \import{<path to file>}{<filename>.pgf}
%%
%% Matplotlib used the following preamble
%%   \def\mathdefault#1{#1}
%%   \everymath=\expandafter{\the\everymath\displaystyle}
%%   \IfFileExists{scrextend.sty}{
%%     \usepackage[fontsize=10.000000pt]{scrextend}
%%   }{
%%     \renewcommand{\normalsize}{\fontsize{10.000000}{12.000000}\selectfont}
%%     \normalsize
%%   }
%%   
%%   \makeatletter\@ifpackageloaded{underscore}{}{\usepackage[strings]{underscore}}\makeatother
%%
\begingroup%
\makeatletter%
\begin{pgfpicture}%
\pgfpathrectangle{\pgfpointorigin}{\pgfqpoint{13.900000in}{5.900000in}}%
\pgfusepath{use as bounding box, clip}%
\begin{pgfscope}%
\pgfsetbuttcap%
\pgfsetmiterjoin%
\definecolor{currentfill}{rgb}{1.000000,1.000000,1.000000}%
\pgfsetfillcolor{currentfill}%
\pgfsetlinewidth{0.000000pt}%
\definecolor{currentstroke}{rgb}{0.000000,0.000000,0.000000}%
\pgfsetstrokecolor{currentstroke}%
\pgfsetdash{}{0pt}%
\pgfpathmoveto{\pgfqpoint{0.000000in}{0.000000in}}%
\pgfpathlineto{\pgfqpoint{13.900000in}{0.000000in}}%
\pgfpathlineto{\pgfqpoint{13.900000in}{5.900000in}}%
\pgfpathlineto{\pgfqpoint{0.000000in}{5.900000in}}%
\pgfpathlineto{\pgfqpoint{0.000000in}{0.000000in}}%
\pgfpathclose%
\pgfusepath{fill}%
\end{pgfscope}%
\begin{pgfscope}%
\pgfsetbuttcap%
\pgfsetmiterjoin%
\definecolor{currentfill}{rgb}{1.000000,1.000000,1.000000}%
\pgfsetfillcolor{currentfill}%
\pgfsetlinewidth{0.000000pt}%
\definecolor{currentstroke}{rgb}{0.000000,0.000000,0.000000}%
\pgfsetstrokecolor{currentstroke}%
\pgfsetstrokeopacity{0.000000}%
\pgfsetdash{}{0pt}%
\pgfpathmoveto{\pgfqpoint{0.393053in}{0.375000in}}%
\pgfpathlineto{\pgfqpoint{6.749886in}{0.375000in}}%
\pgfpathlineto{\pgfqpoint{6.749886in}{5.550000in}}%
\pgfpathlineto{\pgfqpoint{0.393053in}{5.550000in}}%
\pgfpathlineto{\pgfqpoint{0.393053in}{0.375000in}}%
\pgfpathclose%
\pgfusepath{fill}%
\end{pgfscope}%
\begin{pgfscope}%
\pgfpathrectangle{\pgfqpoint{0.393053in}{0.375000in}}{\pgfqpoint{6.356833in}{5.175000in}}%
\pgfusepath{clip}%
\pgfsetbuttcap%
\pgfsetroundjoin%
\pgfsetlinewidth{1.003750pt}%
\definecolor{currentstroke}{rgb}{0.827451,0.827451,0.827451}%
\pgfsetstrokecolor{currentstroke}%
\pgfsetdash{}{0pt}%
\pgfpathmoveto{\pgfqpoint{1.092331in}{2.381241in}}%
\pgfpathcurveto{\pgfqpoint{1.103381in}{2.381241in}}{\pgfqpoint{1.113980in}{2.385631in}}{\pgfqpoint{1.121794in}{2.393444in}}%
\pgfpathcurveto{\pgfqpoint{1.129608in}{2.401258in}}{\pgfqpoint{1.133998in}{2.411857in}}{\pgfqpoint{1.133998in}{2.422907in}}%
\pgfpathcurveto{\pgfqpoint{1.133998in}{2.433957in}}{\pgfqpoint{1.129608in}{2.444556in}}{\pgfqpoint{1.121794in}{2.452370in}}%
\pgfpathcurveto{\pgfqpoint{1.113980in}{2.460184in}}{\pgfqpoint{1.103381in}{2.464574in}}{\pgfqpoint{1.092331in}{2.464574in}}%
\pgfpathcurveto{\pgfqpoint{1.081281in}{2.464574in}}{\pgfqpoint{1.070682in}{2.460184in}}{\pgfqpoint{1.062868in}{2.452370in}}%
\pgfpathcurveto{\pgfqpoint{1.055055in}{2.444556in}}{\pgfqpoint{1.050664in}{2.433957in}}{\pgfqpoint{1.050664in}{2.422907in}}%
\pgfpathcurveto{\pgfqpoint{1.050664in}{2.411857in}}{\pgfqpoint{1.055055in}{2.401258in}}{\pgfqpoint{1.062868in}{2.393444in}}%
\pgfpathcurveto{\pgfqpoint{1.070682in}{2.385631in}}{\pgfqpoint{1.081281in}{2.381241in}}{\pgfqpoint{1.092331in}{2.381241in}}%
\pgfpathlineto{\pgfqpoint{1.092331in}{2.381241in}}%
\pgfpathclose%
\pgfusepath{stroke}%
\end{pgfscope}%
\begin{pgfscope}%
\pgfpathrectangle{\pgfqpoint{0.393053in}{0.375000in}}{\pgfqpoint{6.356833in}{5.175000in}}%
\pgfusepath{clip}%
\pgfsetbuttcap%
\pgfsetroundjoin%
\pgfsetlinewidth{1.003750pt}%
\definecolor{currentstroke}{rgb}{0.827451,0.827451,0.827451}%
\pgfsetstrokecolor{currentstroke}%
\pgfsetdash{}{0pt}%
\pgfpathmoveto{\pgfqpoint{2.036490in}{1.340063in}}%
\pgfpathcurveto{\pgfqpoint{2.047540in}{1.340063in}}{\pgfqpoint{2.058139in}{1.344454in}}{\pgfqpoint{2.065953in}{1.352267in}}%
\pgfpathcurveto{\pgfqpoint{2.073767in}{1.360081in}}{\pgfqpoint{2.078157in}{1.370680in}}{\pgfqpoint{2.078157in}{1.381730in}}%
\pgfpathcurveto{\pgfqpoint{2.078157in}{1.392780in}}{\pgfqpoint{2.073767in}{1.403379in}}{\pgfqpoint{2.065953in}{1.411193in}}%
\pgfpathcurveto{\pgfqpoint{2.058139in}{1.419007in}}{\pgfqpoint{2.047540in}{1.423397in}}{\pgfqpoint{2.036490in}{1.423397in}}%
\pgfpathcurveto{\pgfqpoint{2.025440in}{1.423397in}}{\pgfqpoint{2.014841in}{1.419007in}}{\pgfqpoint{2.007027in}{1.411193in}}%
\pgfpathcurveto{\pgfqpoint{1.999214in}{1.403379in}}{\pgfqpoint{1.994824in}{1.392780in}}{\pgfqpoint{1.994824in}{1.381730in}}%
\pgfpathcurveto{\pgfqpoint{1.994824in}{1.370680in}}{\pgfqpoint{1.999214in}{1.360081in}}{\pgfqpoint{2.007027in}{1.352267in}}%
\pgfpathcurveto{\pgfqpoint{2.014841in}{1.344454in}}{\pgfqpoint{2.025440in}{1.340063in}}{\pgfqpoint{2.036490in}{1.340063in}}%
\pgfpathlineto{\pgfqpoint{2.036490in}{1.340063in}}%
\pgfpathclose%
\pgfusepath{stroke}%
\end{pgfscope}%
\begin{pgfscope}%
\pgfpathrectangle{\pgfqpoint{0.393053in}{0.375000in}}{\pgfqpoint{6.356833in}{5.175000in}}%
\pgfusepath{clip}%
\pgfsetbuttcap%
\pgfsetroundjoin%
\pgfsetlinewidth{1.003750pt}%
\definecolor{currentstroke}{rgb}{0.827451,0.827451,0.827451}%
\pgfsetstrokecolor{currentstroke}%
\pgfsetdash{}{0pt}%
\pgfpathmoveto{\pgfqpoint{1.616394in}{1.795125in}}%
\pgfpathcurveto{\pgfqpoint{1.627444in}{1.795125in}}{\pgfqpoint{1.638043in}{1.799515in}}{\pgfqpoint{1.645857in}{1.807329in}}%
\pgfpathcurveto{\pgfqpoint{1.653671in}{1.815142in}}{\pgfqpoint{1.658061in}{1.825742in}}{\pgfqpoint{1.658061in}{1.836792in}}%
\pgfpathcurveto{\pgfqpoint{1.658061in}{1.847842in}}{\pgfqpoint{1.653671in}{1.858441in}}{\pgfqpoint{1.645857in}{1.866254in}}%
\pgfpathcurveto{\pgfqpoint{1.638043in}{1.874068in}}{\pgfqpoint{1.627444in}{1.878458in}}{\pgfqpoint{1.616394in}{1.878458in}}%
\pgfpathcurveto{\pgfqpoint{1.605344in}{1.878458in}}{\pgfqpoint{1.594745in}{1.874068in}}{\pgfqpoint{1.586931in}{1.866254in}}%
\pgfpathcurveto{\pgfqpoint{1.579118in}{1.858441in}}{\pgfqpoint{1.574728in}{1.847842in}}{\pgfqpoint{1.574728in}{1.836792in}}%
\pgfpathcurveto{\pgfqpoint{1.574728in}{1.825742in}}{\pgfqpoint{1.579118in}{1.815142in}}{\pgfqpoint{1.586931in}{1.807329in}}%
\pgfpathcurveto{\pgfqpoint{1.594745in}{1.799515in}}{\pgfqpoint{1.605344in}{1.795125in}}{\pgfqpoint{1.616394in}{1.795125in}}%
\pgfpathlineto{\pgfqpoint{1.616394in}{1.795125in}}%
\pgfpathclose%
\pgfusepath{stroke}%
\end{pgfscope}%
\begin{pgfscope}%
\pgfpathrectangle{\pgfqpoint{0.393053in}{0.375000in}}{\pgfqpoint{6.356833in}{5.175000in}}%
\pgfusepath{clip}%
\pgfsetbuttcap%
\pgfsetroundjoin%
\pgfsetlinewidth{1.003750pt}%
\definecolor{currentstroke}{rgb}{0.827451,0.827451,0.827451}%
\pgfsetstrokecolor{currentstroke}%
\pgfsetdash{}{0pt}%
\pgfpathmoveto{\pgfqpoint{1.388153in}{1.896293in}}%
\pgfpathcurveto{\pgfqpoint{1.399204in}{1.896293in}}{\pgfqpoint{1.409803in}{1.900683in}}{\pgfqpoint{1.417616in}{1.908497in}}%
\pgfpathcurveto{\pgfqpoint{1.425430in}{1.916310in}}{\pgfqpoint{1.429820in}{1.926909in}}{\pgfqpoint{1.429820in}{1.937959in}}%
\pgfpathcurveto{\pgfqpoint{1.429820in}{1.949010in}}{\pgfqpoint{1.425430in}{1.959609in}}{\pgfqpoint{1.417616in}{1.967422in}}%
\pgfpathcurveto{\pgfqpoint{1.409803in}{1.975236in}}{\pgfqpoint{1.399204in}{1.979626in}}{\pgfqpoint{1.388153in}{1.979626in}}%
\pgfpathcurveto{\pgfqpoint{1.377103in}{1.979626in}}{\pgfqpoint{1.366504in}{1.975236in}}{\pgfqpoint{1.358691in}{1.967422in}}%
\pgfpathcurveto{\pgfqpoint{1.350877in}{1.959609in}}{\pgfqpoint{1.346487in}{1.949010in}}{\pgfqpoint{1.346487in}{1.937959in}}%
\pgfpathcurveto{\pgfqpoint{1.346487in}{1.926909in}}{\pgfqpoint{1.350877in}{1.916310in}}{\pgfqpoint{1.358691in}{1.908497in}}%
\pgfpathcurveto{\pgfqpoint{1.366504in}{1.900683in}}{\pgfqpoint{1.377103in}{1.896293in}}{\pgfqpoint{1.388153in}{1.896293in}}%
\pgfpathlineto{\pgfqpoint{1.388153in}{1.896293in}}%
\pgfpathclose%
\pgfusepath{stroke}%
\end{pgfscope}%
\begin{pgfscope}%
\pgfpathrectangle{\pgfqpoint{0.393053in}{0.375000in}}{\pgfqpoint{6.356833in}{5.175000in}}%
\pgfusepath{clip}%
\pgfsetbuttcap%
\pgfsetroundjoin%
\pgfsetlinewidth{1.003750pt}%
\definecolor{currentstroke}{rgb}{0.827451,0.827451,0.827451}%
\pgfsetstrokecolor{currentstroke}%
\pgfsetdash{}{0pt}%
\pgfpathmoveto{\pgfqpoint{3.880256in}{0.594158in}}%
\pgfpathcurveto{\pgfqpoint{3.891307in}{0.594158in}}{\pgfqpoint{3.901906in}{0.598548in}}{\pgfqpoint{3.909719in}{0.606362in}}%
\pgfpathcurveto{\pgfqpoint{3.917533in}{0.614176in}}{\pgfqpoint{3.921923in}{0.624775in}}{\pgfqpoint{3.921923in}{0.635825in}}%
\pgfpathcurveto{\pgfqpoint{3.921923in}{0.646875in}}{\pgfqpoint{3.917533in}{0.657474in}}{\pgfqpoint{3.909719in}{0.665288in}}%
\pgfpathcurveto{\pgfqpoint{3.901906in}{0.673101in}}{\pgfqpoint{3.891307in}{0.677492in}}{\pgfqpoint{3.880256in}{0.677492in}}%
\pgfpathcurveto{\pgfqpoint{3.869206in}{0.677492in}}{\pgfqpoint{3.858607in}{0.673101in}}{\pgfqpoint{3.850794in}{0.665288in}}%
\pgfpathcurveto{\pgfqpoint{3.842980in}{0.657474in}}{\pgfqpoint{3.838590in}{0.646875in}}{\pgfqpoint{3.838590in}{0.635825in}}%
\pgfpathcurveto{\pgfqpoint{3.838590in}{0.624775in}}{\pgfqpoint{3.842980in}{0.614176in}}{\pgfqpoint{3.850794in}{0.606362in}}%
\pgfpathcurveto{\pgfqpoint{3.858607in}{0.598548in}}{\pgfqpoint{3.869206in}{0.594158in}}{\pgfqpoint{3.880256in}{0.594158in}}%
\pgfpathlineto{\pgfqpoint{3.880256in}{0.594158in}}%
\pgfpathclose%
\pgfusepath{stroke}%
\end{pgfscope}%
\begin{pgfscope}%
\pgfpathrectangle{\pgfqpoint{0.393053in}{0.375000in}}{\pgfqpoint{6.356833in}{5.175000in}}%
\pgfusepath{clip}%
\pgfsetbuttcap%
\pgfsetroundjoin%
\pgfsetlinewidth{1.003750pt}%
\definecolor{currentstroke}{rgb}{0.827451,0.827451,0.827451}%
\pgfsetstrokecolor{currentstroke}%
\pgfsetdash{}{0pt}%
\pgfpathmoveto{\pgfqpoint{1.463696in}{1.814748in}}%
\pgfpathcurveto{\pgfqpoint{1.474746in}{1.814748in}}{\pgfqpoint{1.485345in}{1.819138in}}{\pgfqpoint{1.493159in}{1.826952in}}%
\pgfpathcurveto{\pgfqpoint{1.500972in}{1.834766in}}{\pgfqpoint{1.505362in}{1.845365in}}{\pgfqpoint{1.505362in}{1.856415in}}%
\pgfpathcurveto{\pgfqpoint{1.505362in}{1.867465in}}{\pgfqpoint{1.500972in}{1.878064in}}{\pgfqpoint{1.493159in}{1.885878in}}%
\pgfpathcurveto{\pgfqpoint{1.485345in}{1.893691in}}{\pgfqpoint{1.474746in}{1.898081in}}{\pgfqpoint{1.463696in}{1.898081in}}%
\pgfpathcurveto{\pgfqpoint{1.452646in}{1.898081in}}{\pgfqpoint{1.442047in}{1.893691in}}{\pgfqpoint{1.434233in}{1.885878in}}%
\pgfpathcurveto{\pgfqpoint{1.426419in}{1.878064in}}{\pgfqpoint{1.422029in}{1.867465in}}{\pgfqpoint{1.422029in}{1.856415in}}%
\pgfpathcurveto{\pgfqpoint{1.422029in}{1.845365in}}{\pgfqpoint{1.426419in}{1.834766in}}{\pgfqpoint{1.434233in}{1.826952in}}%
\pgfpathcurveto{\pgfqpoint{1.442047in}{1.819138in}}{\pgfqpoint{1.452646in}{1.814748in}}{\pgfqpoint{1.463696in}{1.814748in}}%
\pgfpathlineto{\pgfqpoint{1.463696in}{1.814748in}}%
\pgfpathclose%
\pgfusepath{stroke}%
\end{pgfscope}%
\begin{pgfscope}%
\pgfpathrectangle{\pgfqpoint{0.393053in}{0.375000in}}{\pgfqpoint{6.356833in}{5.175000in}}%
\pgfusepath{clip}%
\pgfsetbuttcap%
\pgfsetroundjoin%
\pgfsetlinewidth{1.003750pt}%
\definecolor{currentstroke}{rgb}{0.827451,0.827451,0.827451}%
\pgfsetstrokecolor{currentstroke}%
\pgfsetdash{}{0pt}%
\pgfpathmoveto{\pgfqpoint{0.610193in}{3.339690in}}%
\pgfpathcurveto{\pgfqpoint{0.621243in}{3.339690in}}{\pgfqpoint{0.631842in}{3.344080in}}{\pgfqpoint{0.639655in}{3.351894in}}%
\pgfpathcurveto{\pgfqpoint{0.647469in}{3.359707in}}{\pgfqpoint{0.651859in}{3.370306in}}{\pgfqpoint{0.651859in}{3.381357in}}%
\pgfpathcurveto{\pgfqpoint{0.651859in}{3.392407in}}{\pgfqpoint{0.647469in}{3.403006in}}{\pgfqpoint{0.639655in}{3.410819in}}%
\pgfpathcurveto{\pgfqpoint{0.631842in}{3.418633in}}{\pgfqpoint{0.621243in}{3.423023in}}{\pgfqpoint{0.610193in}{3.423023in}}%
\pgfpathcurveto{\pgfqpoint{0.599143in}{3.423023in}}{\pgfqpoint{0.588544in}{3.418633in}}{\pgfqpoint{0.580730in}{3.410819in}}%
\pgfpathcurveto{\pgfqpoint{0.572916in}{3.403006in}}{\pgfqpoint{0.568526in}{3.392407in}}{\pgfqpoint{0.568526in}{3.381357in}}%
\pgfpathcurveto{\pgfqpoint{0.568526in}{3.370306in}}{\pgfqpoint{0.572916in}{3.359707in}}{\pgfqpoint{0.580730in}{3.351894in}}%
\pgfpathcurveto{\pgfqpoint{0.588544in}{3.344080in}}{\pgfqpoint{0.599143in}{3.339690in}}{\pgfqpoint{0.610193in}{3.339690in}}%
\pgfpathlineto{\pgfqpoint{0.610193in}{3.339690in}}%
\pgfpathclose%
\pgfusepath{stroke}%
\end{pgfscope}%
\begin{pgfscope}%
\pgfpathrectangle{\pgfqpoint{0.393053in}{0.375000in}}{\pgfqpoint{6.356833in}{5.175000in}}%
\pgfusepath{clip}%
\pgfsetbuttcap%
\pgfsetroundjoin%
\pgfsetlinewidth{1.003750pt}%
\definecolor{currentstroke}{rgb}{0.827451,0.827451,0.827451}%
\pgfsetstrokecolor{currentstroke}%
\pgfsetdash{}{0pt}%
\pgfpathmoveto{\pgfqpoint{1.909651in}{1.444209in}}%
\pgfpathcurveto{\pgfqpoint{1.920701in}{1.444209in}}{\pgfqpoint{1.931300in}{1.448599in}}{\pgfqpoint{1.939114in}{1.456413in}}%
\pgfpathcurveto{\pgfqpoint{1.946927in}{1.464226in}}{\pgfqpoint{1.951318in}{1.474825in}}{\pgfqpoint{1.951318in}{1.485875in}}%
\pgfpathcurveto{\pgfqpoint{1.951318in}{1.496926in}}{\pgfqpoint{1.946927in}{1.507525in}}{\pgfqpoint{1.939114in}{1.515338in}}%
\pgfpathcurveto{\pgfqpoint{1.931300in}{1.523152in}}{\pgfqpoint{1.920701in}{1.527542in}}{\pgfqpoint{1.909651in}{1.527542in}}%
\pgfpathcurveto{\pgfqpoint{1.898601in}{1.527542in}}{\pgfqpoint{1.888002in}{1.523152in}}{\pgfqpoint{1.880188in}{1.515338in}}%
\pgfpathcurveto{\pgfqpoint{1.872375in}{1.507525in}}{\pgfqpoint{1.867984in}{1.496926in}}{\pgfqpoint{1.867984in}{1.485875in}}%
\pgfpathcurveto{\pgfqpoint{1.867984in}{1.474825in}}{\pgfqpoint{1.872375in}{1.464226in}}{\pgfqpoint{1.880188in}{1.456413in}}%
\pgfpathcurveto{\pgfqpoint{1.888002in}{1.448599in}}{\pgfqpoint{1.898601in}{1.444209in}}{\pgfqpoint{1.909651in}{1.444209in}}%
\pgfpathlineto{\pgfqpoint{1.909651in}{1.444209in}}%
\pgfpathclose%
\pgfusepath{stroke}%
\end{pgfscope}%
\begin{pgfscope}%
\pgfpathrectangle{\pgfqpoint{0.393053in}{0.375000in}}{\pgfqpoint{6.356833in}{5.175000in}}%
\pgfusepath{clip}%
\pgfsetbuttcap%
\pgfsetroundjoin%
\pgfsetlinewidth{1.003750pt}%
\definecolor{currentstroke}{rgb}{0.827451,0.827451,0.827451}%
\pgfsetstrokecolor{currentstroke}%
\pgfsetdash{}{0pt}%
\pgfpathmoveto{\pgfqpoint{1.253491in}{2.045656in}}%
\pgfpathcurveto{\pgfqpoint{1.264541in}{2.045656in}}{\pgfqpoint{1.275140in}{2.050046in}}{\pgfqpoint{1.282953in}{2.057860in}}%
\pgfpathcurveto{\pgfqpoint{1.290767in}{2.065674in}}{\pgfqpoint{1.295157in}{2.076273in}}{\pgfqpoint{1.295157in}{2.087323in}}%
\pgfpathcurveto{\pgfqpoint{1.295157in}{2.098373in}}{\pgfqpoint{1.290767in}{2.108972in}}{\pgfqpoint{1.282953in}{2.116786in}}%
\pgfpathcurveto{\pgfqpoint{1.275140in}{2.124599in}}{\pgfqpoint{1.264541in}{2.128990in}}{\pgfqpoint{1.253491in}{2.128990in}}%
\pgfpathcurveto{\pgfqpoint{1.242440in}{2.128990in}}{\pgfqpoint{1.231841in}{2.124599in}}{\pgfqpoint{1.224028in}{2.116786in}}%
\pgfpathcurveto{\pgfqpoint{1.216214in}{2.108972in}}{\pgfqpoint{1.211824in}{2.098373in}}{\pgfqpoint{1.211824in}{2.087323in}}%
\pgfpathcurveto{\pgfqpoint{1.211824in}{2.076273in}}{\pgfqpoint{1.216214in}{2.065674in}}{\pgfqpoint{1.224028in}{2.057860in}}%
\pgfpathcurveto{\pgfqpoint{1.231841in}{2.050046in}}{\pgfqpoint{1.242440in}{2.045656in}}{\pgfqpoint{1.253491in}{2.045656in}}%
\pgfpathlineto{\pgfqpoint{1.253491in}{2.045656in}}%
\pgfpathclose%
\pgfusepath{stroke}%
\end{pgfscope}%
\begin{pgfscope}%
\pgfpathrectangle{\pgfqpoint{0.393053in}{0.375000in}}{\pgfqpoint{6.356833in}{5.175000in}}%
\pgfusepath{clip}%
\pgfsetbuttcap%
\pgfsetroundjoin%
\pgfsetlinewidth{1.003750pt}%
\definecolor{currentstroke}{rgb}{0.827451,0.827451,0.827451}%
\pgfsetstrokecolor{currentstroke}%
\pgfsetdash{}{0pt}%
\pgfpathmoveto{\pgfqpoint{2.352374in}{1.159706in}}%
\pgfpathcurveto{\pgfqpoint{2.363424in}{1.159706in}}{\pgfqpoint{2.374023in}{1.164096in}}{\pgfqpoint{2.381836in}{1.171910in}}%
\pgfpathcurveto{\pgfqpoint{2.389650in}{1.179724in}}{\pgfqpoint{2.394040in}{1.190323in}}{\pgfqpoint{2.394040in}{1.201373in}}%
\pgfpathcurveto{\pgfqpoint{2.394040in}{1.212423in}}{\pgfqpoint{2.389650in}{1.223022in}}{\pgfqpoint{2.381836in}{1.230835in}}%
\pgfpathcurveto{\pgfqpoint{2.374023in}{1.238649in}}{\pgfqpoint{2.363424in}{1.243039in}}{\pgfqpoint{2.352374in}{1.243039in}}%
\pgfpathcurveto{\pgfqpoint{2.341324in}{1.243039in}}{\pgfqpoint{2.330725in}{1.238649in}}{\pgfqpoint{2.322911in}{1.230835in}}%
\pgfpathcurveto{\pgfqpoint{2.315097in}{1.223022in}}{\pgfqpoint{2.310707in}{1.212423in}}{\pgfqpoint{2.310707in}{1.201373in}}%
\pgfpathcurveto{\pgfqpoint{2.310707in}{1.190323in}}{\pgfqpoint{2.315097in}{1.179724in}}{\pgfqpoint{2.322911in}{1.171910in}}%
\pgfpathcurveto{\pgfqpoint{2.330725in}{1.164096in}}{\pgfqpoint{2.341324in}{1.159706in}}{\pgfqpoint{2.352374in}{1.159706in}}%
\pgfpathlineto{\pgfqpoint{2.352374in}{1.159706in}}%
\pgfpathclose%
\pgfusepath{stroke}%
\end{pgfscope}%
\begin{pgfscope}%
\pgfpathrectangle{\pgfqpoint{0.393053in}{0.375000in}}{\pgfqpoint{6.356833in}{5.175000in}}%
\pgfusepath{clip}%
\pgfsetbuttcap%
\pgfsetroundjoin%
\pgfsetlinewidth{1.003750pt}%
\definecolor{currentstroke}{rgb}{0.827451,0.827451,0.827451}%
\pgfsetstrokecolor{currentstroke}%
\pgfsetdash{}{0pt}%
\pgfpathmoveto{\pgfqpoint{0.455840in}{3.833482in}}%
\pgfpathcurveto{\pgfqpoint{0.466890in}{3.833482in}}{\pgfqpoint{0.477489in}{3.837873in}}{\pgfqpoint{0.485303in}{3.845686in}}%
\pgfpathcurveto{\pgfqpoint{0.493117in}{3.853500in}}{\pgfqpoint{0.497507in}{3.864099in}}{\pgfqpoint{0.497507in}{3.875149in}}%
\pgfpathcurveto{\pgfqpoint{0.497507in}{3.886199in}}{\pgfqpoint{0.493117in}{3.896798in}}{\pgfqpoint{0.485303in}{3.904612in}}%
\pgfpathcurveto{\pgfqpoint{0.477489in}{3.912425in}}{\pgfqpoint{0.466890in}{3.916816in}}{\pgfqpoint{0.455840in}{3.916816in}}%
\pgfpathcurveto{\pgfqpoint{0.444790in}{3.916816in}}{\pgfqpoint{0.434191in}{3.912425in}}{\pgfqpoint{0.426377in}{3.904612in}}%
\pgfpathcurveto{\pgfqpoint{0.418564in}{3.896798in}}{\pgfqpoint{0.414173in}{3.886199in}}{\pgfqpoint{0.414173in}{3.875149in}}%
\pgfpathcurveto{\pgfqpoint{0.414173in}{3.864099in}}{\pgfqpoint{0.418564in}{3.853500in}}{\pgfqpoint{0.426377in}{3.845686in}}%
\pgfpathcurveto{\pgfqpoint{0.434191in}{3.837873in}}{\pgfqpoint{0.444790in}{3.833482in}}{\pgfqpoint{0.455840in}{3.833482in}}%
\pgfpathlineto{\pgfqpoint{0.455840in}{3.833482in}}%
\pgfpathclose%
\pgfusepath{stroke}%
\end{pgfscope}%
\begin{pgfscope}%
\pgfpathrectangle{\pgfqpoint{0.393053in}{0.375000in}}{\pgfqpoint{6.356833in}{5.175000in}}%
\pgfusepath{clip}%
\pgfsetbuttcap%
\pgfsetroundjoin%
\pgfsetlinewidth{1.003750pt}%
\definecolor{currentstroke}{rgb}{0.827451,0.827451,0.827451}%
\pgfsetstrokecolor{currentstroke}%
\pgfsetdash{}{0pt}%
\pgfpathmoveto{\pgfqpoint{1.186674in}{2.128077in}}%
\pgfpathcurveto{\pgfqpoint{1.197724in}{2.128077in}}{\pgfqpoint{1.208323in}{2.132467in}}{\pgfqpoint{1.216137in}{2.140281in}}%
\pgfpathcurveto{\pgfqpoint{1.223950in}{2.148095in}}{\pgfqpoint{1.228341in}{2.158694in}}{\pgfqpoint{1.228341in}{2.169744in}}%
\pgfpathcurveto{\pgfqpoint{1.228341in}{2.180794in}}{\pgfqpoint{1.223950in}{2.191393in}}{\pgfqpoint{1.216137in}{2.199207in}}%
\pgfpathcurveto{\pgfqpoint{1.208323in}{2.207020in}}{\pgfqpoint{1.197724in}{2.211411in}}{\pgfqpoint{1.186674in}{2.211411in}}%
\pgfpathcurveto{\pgfqpoint{1.175624in}{2.211411in}}{\pgfqpoint{1.165025in}{2.207020in}}{\pgfqpoint{1.157211in}{2.199207in}}%
\pgfpathcurveto{\pgfqpoint{1.149397in}{2.191393in}}{\pgfqpoint{1.145007in}{2.180794in}}{\pgfqpoint{1.145007in}{2.169744in}}%
\pgfpathcurveto{\pgfqpoint{1.145007in}{2.158694in}}{\pgfqpoint{1.149397in}{2.148095in}}{\pgfqpoint{1.157211in}{2.140281in}}%
\pgfpathcurveto{\pgfqpoint{1.165025in}{2.132467in}}{\pgfqpoint{1.175624in}{2.128077in}}{\pgfqpoint{1.186674in}{2.128077in}}%
\pgfpathlineto{\pgfqpoint{1.186674in}{2.128077in}}%
\pgfpathclose%
\pgfusepath{stroke}%
\end{pgfscope}%
\begin{pgfscope}%
\pgfpathrectangle{\pgfqpoint{0.393053in}{0.375000in}}{\pgfqpoint{6.356833in}{5.175000in}}%
\pgfusepath{clip}%
\pgfsetbuttcap%
\pgfsetroundjoin%
\pgfsetlinewidth{1.003750pt}%
\definecolor{currentstroke}{rgb}{0.827451,0.827451,0.827451}%
\pgfsetstrokecolor{currentstroke}%
\pgfsetdash{}{0pt}%
\pgfpathmoveto{\pgfqpoint{3.097658in}{0.775315in}}%
\pgfpathcurveto{\pgfqpoint{3.108708in}{0.775315in}}{\pgfqpoint{3.119307in}{0.779705in}}{\pgfqpoint{3.127121in}{0.787519in}}%
\pgfpathcurveto{\pgfqpoint{3.134935in}{0.795332in}}{\pgfqpoint{3.139325in}{0.805931in}}{\pgfqpoint{3.139325in}{0.816981in}}%
\pgfpathcurveto{\pgfqpoint{3.139325in}{0.828032in}}{\pgfqpoint{3.134935in}{0.838631in}}{\pgfqpoint{3.127121in}{0.846444in}}%
\pgfpathcurveto{\pgfqpoint{3.119307in}{0.854258in}}{\pgfqpoint{3.108708in}{0.858648in}}{\pgfqpoint{3.097658in}{0.858648in}}%
\pgfpathcurveto{\pgfqpoint{3.086608in}{0.858648in}}{\pgfqpoint{3.076009in}{0.854258in}}{\pgfqpoint{3.068195in}{0.846444in}}%
\pgfpathcurveto{\pgfqpoint{3.060382in}{0.838631in}}{\pgfqpoint{3.055992in}{0.828032in}}{\pgfqpoint{3.055992in}{0.816981in}}%
\pgfpathcurveto{\pgfqpoint{3.055992in}{0.805931in}}{\pgfqpoint{3.060382in}{0.795332in}}{\pgfqpoint{3.068195in}{0.787519in}}%
\pgfpathcurveto{\pgfqpoint{3.076009in}{0.779705in}}{\pgfqpoint{3.086608in}{0.775315in}}{\pgfqpoint{3.097658in}{0.775315in}}%
\pgfpathlineto{\pgfqpoint{3.097658in}{0.775315in}}%
\pgfpathclose%
\pgfusepath{stroke}%
\end{pgfscope}%
\begin{pgfscope}%
\pgfpathrectangle{\pgfqpoint{0.393053in}{0.375000in}}{\pgfqpoint{6.356833in}{5.175000in}}%
\pgfusepath{clip}%
\pgfsetbuttcap%
\pgfsetroundjoin%
\pgfsetlinewidth{1.003750pt}%
\definecolor{currentstroke}{rgb}{0.827451,0.827451,0.827451}%
\pgfsetstrokecolor{currentstroke}%
\pgfsetdash{}{0pt}%
\pgfpathmoveto{\pgfqpoint{4.283632in}{0.443541in}}%
\pgfpathcurveto{\pgfqpoint{4.294682in}{0.443541in}}{\pgfqpoint{4.305281in}{0.447931in}}{\pgfqpoint{4.313095in}{0.455745in}}%
\pgfpathcurveto{\pgfqpoint{4.320908in}{0.463559in}}{\pgfqpoint{4.325299in}{0.474158in}}{\pgfqpoint{4.325299in}{0.485208in}}%
\pgfpathcurveto{\pgfqpoint{4.325299in}{0.496258in}}{\pgfqpoint{4.320908in}{0.506857in}}{\pgfqpoint{4.313095in}{0.514670in}}%
\pgfpathcurveto{\pgfqpoint{4.305281in}{0.522484in}}{\pgfqpoint{4.294682in}{0.526874in}}{\pgfqpoint{4.283632in}{0.526874in}}%
\pgfpathcurveto{\pgfqpoint{4.272582in}{0.526874in}}{\pgfqpoint{4.261983in}{0.522484in}}{\pgfqpoint{4.254169in}{0.514670in}}%
\pgfpathcurveto{\pgfqpoint{4.246356in}{0.506857in}}{\pgfqpoint{4.241965in}{0.496258in}}{\pgfqpoint{4.241965in}{0.485208in}}%
\pgfpathcurveto{\pgfqpoint{4.241965in}{0.474158in}}{\pgfqpoint{4.246356in}{0.463559in}}{\pgfqpoint{4.254169in}{0.455745in}}%
\pgfpathcurveto{\pgfqpoint{4.261983in}{0.447931in}}{\pgfqpoint{4.272582in}{0.443541in}}{\pgfqpoint{4.283632in}{0.443541in}}%
\pgfpathlineto{\pgfqpoint{4.283632in}{0.443541in}}%
\pgfpathclose%
\pgfusepath{stroke}%
\end{pgfscope}%
\begin{pgfscope}%
\pgfpathrectangle{\pgfqpoint{0.393053in}{0.375000in}}{\pgfqpoint{6.356833in}{5.175000in}}%
\pgfusepath{clip}%
\pgfsetbuttcap%
\pgfsetroundjoin%
\pgfsetlinewidth{1.003750pt}%
\definecolor{currentstroke}{rgb}{0.827451,0.827451,0.827451}%
\pgfsetstrokecolor{currentstroke}%
\pgfsetdash{}{0pt}%
\pgfpathmoveto{\pgfqpoint{2.582769in}{1.048231in}}%
\pgfpathcurveto{\pgfqpoint{2.593819in}{1.048231in}}{\pgfqpoint{2.604418in}{1.052621in}}{\pgfqpoint{2.612231in}{1.060435in}}%
\pgfpathcurveto{\pgfqpoint{2.620045in}{1.068248in}}{\pgfqpoint{2.624435in}{1.078848in}}{\pgfqpoint{2.624435in}{1.089898in}}%
\pgfpathcurveto{\pgfqpoint{2.624435in}{1.100948in}}{\pgfqpoint{2.620045in}{1.111547in}}{\pgfqpoint{2.612231in}{1.119360in}}%
\pgfpathcurveto{\pgfqpoint{2.604418in}{1.127174in}}{\pgfqpoint{2.593819in}{1.131564in}}{\pgfqpoint{2.582769in}{1.131564in}}%
\pgfpathcurveto{\pgfqpoint{2.571718in}{1.131564in}}{\pgfqpoint{2.561119in}{1.127174in}}{\pgfqpoint{2.553306in}{1.119360in}}%
\pgfpathcurveto{\pgfqpoint{2.545492in}{1.111547in}}{\pgfqpoint{2.541102in}{1.100948in}}{\pgfqpoint{2.541102in}{1.089898in}}%
\pgfpathcurveto{\pgfqpoint{2.541102in}{1.078848in}}{\pgfqpoint{2.545492in}{1.068248in}}{\pgfqpoint{2.553306in}{1.060435in}}%
\pgfpathcurveto{\pgfqpoint{2.561119in}{1.052621in}}{\pgfqpoint{2.571718in}{1.048231in}}{\pgfqpoint{2.582769in}{1.048231in}}%
\pgfpathlineto{\pgfqpoint{2.582769in}{1.048231in}}%
\pgfpathclose%
\pgfusepath{stroke}%
\end{pgfscope}%
\begin{pgfscope}%
\pgfpathrectangle{\pgfqpoint{0.393053in}{0.375000in}}{\pgfqpoint{6.356833in}{5.175000in}}%
\pgfusepath{clip}%
\pgfsetbuttcap%
\pgfsetroundjoin%
\pgfsetlinewidth{1.003750pt}%
\definecolor{currentstroke}{rgb}{0.827451,0.827451,0.827451}%
\pgfsetstrokecolor{currentstroke}%
\pgfsetdash{}{0pt}%
\pgfpathmoveto{\pgfqpoint{0.474470in}{3.727925in}}%
\pgfpathcurveto{\pgfqpoint{0.485520in}{3.727925in}}{\pgfqpoint{0.496119in}{3.732315in}}{\pgfqpoint{0.503933in}{3.740129in}}%
\pgfpathcurveto{\pgfqpoint{0.511746in}{3.747943in}}{\pgfqpoint{0.516137in}{3.758542in}}{\pgfqpoint{0.516137in}{3.769592in}}%
\pgfpathcurveto{\pgfqpoint{0.516137in}{3.780642in}}{\pgfqpoint{0.511746in}{3.791241in}}{\pgfqpoint{0.503933in}{3.799055in}}%
\pgfpathcurveto{\pgfqpoint{0.496119in}{3.806868in}}{\pgfqpoint{0.485520in}{3.811259in}}{\pgfqpoint{0.474470in}{3.811259in}}%
\pgfpathcurveto{\pgfqpoint{0.463420in}{3.811259in}}{\pgfqpoint{0.452821in}{3.806868in}}{\pgfqpoint{0.445007in}{3.799055in}}%
\pgfpathcurveto{\pgfqpoint{0.437194in}{3.791241in}}{\pgfqpoint{0.432803in}{3.780642in}}{\pgfqpoint{0.432803in}{3.769592in}}%
\pgfpathcurveto{\pgfqpoint{0.432803in}{3.758542in}}{\pgfqpoint{0.437194in}{3.747943in}}{\pgfqpoint{0.445007in}{3.740129in}}%
\pgfpathcurveto{\pgfqpoint{0.452821in}{3.732315in}}{\pgfqpoint{0.463420in}{3.727925in}}{\pgfqpoint{0.474470in}{3.727925in}}%
\pgfpathlineto{\pgfqpoint{0.474470in}{3.727925in}}%
\pgfpathclose%
\pgfusepath{stroke}%
\end{pgfscope}%
\begin{pgfscope}%
\pgfpathrectangle{\pgfqpoint{0.393053in}{0.375000in}}{\pgfqpoint{6.356833in}{5.175000in}}%
\pgfusepath{clip}%
\pgfsetbuttcap%
\pgfsetroundjoin%
\pgfsetlinewidth{1.003750pt}%
\definecolor{currentstroke}{rgb}{0.827451,0.827451,0.827451}%
\pgfsetstrokecolor{currentstroke}%
\pgfsetdash{}{0pt}%
\pgfpathmoveto{\pgfqpoint{1.493790in}{1.813114in}}%
\pgfpathcurveto{\pgfqpoint{1.504840in}{1.813114in}}{\pgfqpoint{1.515439in}{1.817505in}}{\pgfqpoint{1.523252in}{1.825318in}}%
\pgfpathcurveto{\pgfqpoint{1.531066in}{1.833132in}}{\pgfqpoint{1.535456in}{1.843731in}}{\pgfqpoint{1.535456in}{1.854781in}}%
\pgfpathcurveto{\pgfqpoint{1.535456in}{1.865831in}}{\pgfqpoint{1.531066in}{1.876430in}}{\pgfqpoint{1.523252in}{1.884244in}}%
\pgfpathcurveto{\pgfqpoint{1.515439in}{1.892058in}}{\pgfqpoint{1.504840in}{1.896448in}}{\pgfqpoint{1.493790in}{1.896448in}}%
\pgfpathcurveto{\pgfqpoint{1.482740in}{1.896448in}}{\pgfqpoint{1.472140in}{1.892058in}}{\pgfqpoint{1.464327in}{1.884244in}}%
\pgfpathcurveto{\pgfqpoint{1.456513in}{1.876430in}}{\pgfqpoint{1.452123in}{1.865831in}}{\pgfqpoint{1.452123in}{1.854781in}}%
\pgfpathcurveto{\pgfqpoint{1.452123in}{1.843731in}}{\pgfqpoint{1.456513in}{1.833132in}}{\pgfqpoint{1.464327in}{1.825318in}}%
\pgfpathcurveto{\pgfqpoint{1.472140in}{1.817505in}}{\pgfqpoint{1.482740in}{1.813114in}}{\pgfqpoint{1.493790in}{1.813114in}}%
\pgfpathlineto{\pgfqpoint{1.493790in}{1.813114in}}%
\pgfpathclose%
\pgfusepath{stroke}%
\end{pgfscope}%
\begin{pgfscope}%
\pgfpathrectangle{\pgfqpoint{0.393053in}{0.375000in}}{\pgfqpoint{6.356833in}{5.175000in}}%
\pgfusepath{clip}%
\pgfsetbuttcap%
\pgfsetroundjoin%
\pgfsetlinewidth{1.003750pt}%
\definecolor{currentstroke}{rgb}{0.827451,0.827451,0.827451}%
\pgfsetstrokecolor{currentstroke}%
\pgfsetdash{}{0pt}%
\pgfpathmoveto{\pgfqpoint{1.687372in}{1.621104in}}%
\pgfpathcurveto{\pgfqpoint{1.698422in}{1.621104in}}{\pgfqpoint{1.709022in}{1.625494in}}{\pgfqpoint{1.716835in}{1.633308in}}%
\pgfpathcurveto{\pgfqpoint{1.724649in}{1.641122in}}{\pgfqpoint{1.729039in}{1.651721in}}{\pgfqpoint{1.729039in}{1.662771in}}%
\pgfpathcurveto{\pgfqpoint{1.729039in}{1.673821in}}{\pgfqpoint{1.724649in}{1.684420in}}{\pgfqpoint{1.716835in}{1.692234in}}%
\pgfpathcurveto{\pgfqpoint{1.709022in}{1.700047in}}{\pgfqpoint{1.698422in}{1.704437in}}{\pgfqpoint{1.687372in}{1.704437in}}%
\pgfpathcurveto{\pgfqpoint{1.676322in}{1.704437in}}{\pgfqpoint{1.665723in}{1.700047in}}{\pgfqpoint{1.657910in}{1.692234in}}%
\pgfpathcurveto{\pgfqpoint{1.650096in}{1.684420in}}{\pgfqpoint{1.645706in}{1.673821in}}{\pgfqpoint{1.645706in}{1.662771in}}%
\pgfpathcurveto{\pgfqpoint{1.645706in}{1.651721in}}{\pgfqpoint{1.650096in}{1.641122in}}{\pgfqpoint{1.657910in}{1.633308in}}%
\pgfpathcurveto{\pgfqpoint{1.665723in}{1.625494in}}{\pgfqpoint{1.676322in}{1.621104in}}{\pgfqpoint{1.687372in}{1.621104in}}%
\pgfpathlineto{\pgfqpoint{1.687372in}{1.621104in}}%
\pgfpathclose%
\pgfusepath{stroke}%
\end{pgfscope}%
\begin{pgfscope}%
\pgfpathrectangle{\pgfqpoint{0.393053in}{0.375000in}}{\pgfqpoint{6.356833in}{5.175000in}}%
\pgfusepath{clip}%
\pgfsetbuttcap%
\pgfsetroundjoin%
\pgfsetlinewidth{1.003750pt}%
\definecolor{currentstroke}{rgb}{0.827451,0.827451,0.827451}%
\pgfsetstrokecolor{currentstroke}%
\pgfsetdash{}{0pt}%
\pgfpathmoveto{\pgfqpoint{3.592196in}{0.693461in}}%
\pgfpathcurveto{\pgfqpoint{3.603246in}{0.693461in}}{\pgfqpoint{3.613845in}{0.697851in}}{\pgfqpoint{3.621659in}{0.705665in}}%
\pgfpathcurveto{\pgfqpoint{3.629472in}{0.713479in}}{\pgfqpoint{3.633862in}{0.724078in}}{\pgfqpoint{3.633862in}{0.735128in}}%
\pgfpathcurveto{\pgfqpoint{3.633862in}{0.746178in}}{\pgfqpoint{3.629472in}{0.756777in}}{\pgfqpoint{3.621659in}{0.764591in}}%
\pgfpathcurveto{\pgfqpoint{3.613845in}{0.772404in}}{\pgfqpoint{3.603246in}{0.776794in}}{\pgfqpoint{3.592196in}{0.776794in}}%
\pgfpathcurveto{\pgfqpoint{3.581146in}{0.776794in}}{\pgfqpoint{3.570547in}{0.772404in}}{\pgfqpoint{3.562733in}{0.764591in}}%
\pgfpathcurveto{\pgfqpoint{3.554919in}{0.756777in}}{\pgfqpoint{3.550529in}{0.746178in}}{\pgfqpoint{3.550529in}{0.735128in}}%
\pgfpathcurveto{\pgfqpoint{3.550529in}{0.724078in}}{\pgfqpoint{3.554919in}{0.713479in}}{\pgfqpoint{3.562733in}{0.705665in}}%
\pgfpathcurveto{\pgfqpoint{3.570547in}{0.697851in}}{\pgfqpoint{3.581146in}{0.693461in}}{\pgfqpoint{3.592196in}{0.693461in}}%
\pgfpathlineto{\pgfqpoint{3.592196in}{0.693461in}}%
\pgfpathclose%
\pgfusepath{stroke}%
\end{pgfscope}%
\begin{pgfscope}%
\pgfpathrectangle{\pgfqpoint{0.393053in}{0.375000in}}{\pgfqpoint{6.356833in}{5.175000in}}%
\pgfusepath{clip}%
\pgfsetbuttcap%
\pgfsetroundjoin%
\pgfsetlinewidth{1.003750pt}%
\definecolor{currentstroke}{rgb}{0.827451,0.827451,0.827451}%
\pgfsetstrokecolor{currentstroke}%
\pgfsetdash{}{0pt}%
\pgfpathmoveto{\pgfqpoint{0.947650in}{2.625936in}}%
\pgfpathcurveto{\pgfqpoint{0.958700in}{2.625936in}}{\pgfqpoint{0.969299in}{2.630326in}}{\pgfqpoint{0.977112in}{2.638140in}}%
\pgfpathcurveto{\pgfqpoint{0.984926in}{2.645953in}}{\pgfqpoint{0.989316in}{2.656552in}}{\pgfqpoint{0.989316in}{2.667602in}}%
\pgfpathcurveto{\pgfqpoint{0.989316in}{2.678652in}}{\pgfqpoint{0.984926in}{2.689252in}}{\pgfqpoint{0.977112in}{2.697065in}}%
\pgfpathcurveto{\pgfqpoint{0.969299in}{2.704879in}}{\pgfqpoint{0.958700in}{2.709269in}}{\pgfqpoint{0.947650in}{2.709269in}}%
\pgfpathcurveto{\pgfqpoint{0.936600in}{2.709269in}}{\pgfqpoint{0.926000in}{2.704879in}}{\pgfqpoint{0.918187in}{2.697065in}}%
\pgfpathcurveto{\pgfqpoint{0.910373in}{2.689252in}}{\pgfqpoint{0.905983in}{2.678652in}}{\pgfqpoint{0.905983in}{2.667602in}}%
\pgfpathcurveto{\pgfqpoint{0.905983in}{2.656552in}}{\pgfqpoint{0.910373in}{2.645953in}}{\pgfqpoint{0.918187in}{2.638140in}}%
\pgfpathcurveto{\pgfqpoint{0.926000in}{2.630326in}}{\pgfqpoint{0.936600in}{2.625936in}}{\pgfqpoint{0.947650in}{2.625936in}}%
\pgfpathlineto{\pgfqpoint{0.947650in}{2.625936in}}%
\pgfpathclose%
\pgfusepath{stroke}%
\end{pgfscope}%
\begin{pgfscope}%
\pgfpathrectangle{\pgfqpoint{0.393053in}{0.375000in}}{\pgfqpoint{6.356833in}{5.175000in}}%
\pgfusepath{clip}%
\pgfsetbuttcap%
\pgfsetroundjoin%
\pgfsetlinewidth{1.003750pt}%
\definecolor{currentstroke}{rgb}{0.827451,0.827451,0.827451}%
\pgfsetstrokecolor{currentstroke}%
\pgfsetdash{}{0pt}%
\pgfpathmoveto{\pgfqpoint{2.796329in}{0.903359in}}%
\pgfpathcurveto{\pgfqpoint{2.807379in}{0.903359in}}{\pgfqpoint{2.817978in}{0.907749in}}{\pgfqpoint{2.825792in}{0.915563in}}%
\pgfpathcurveto{\pgfqpoint{2.833605in}{0.923376in}}{\pgfqpoint{2.837995in}{0.933975in}}{\pgfqpoint{2.837995in}{0.945025in}}%
\pgfpathcurveto{\pgfqpoint{2.837995in}{0.956076in}}{\pgfqpoint{2.833605in}{0.966675in}}{\pgfqpoint{2.825792in}{0.974488in}}%
\pgfpathcurveto{\pgfqpoint{2.817978in}{0.982302in}}{\pgfqpoint{2.807379in}{0.986692in}}{\pgfqpoint{2.796329in}{0.986692in}}%
\pgfpathcurveto{\pgfqpoint{2.785279in}{0.986692in}}{\pgfqpoint{2.774680in}{0.982302in}}{\pgfqpoint{2.766866in}{0.974488in}}%
\pgfpathcurveto{\pgfqpoint{2.759052in}{0.966675in}}{\pgfqpoint{2.754662in}{0.956076in}}{\pgfqpoint{2.754662in}{0.945025in}}%
\pgfpathcurveto{\pgfqpoint{2.754662in}{0.933975in}}{\pgfqpoint{2.759052in}{0.923376in}}{\pgfqpoint{2.766866in}{0.915563in}}%
\pgfpathcurveto{\pgfqpoint{2.774680in}{0.907749in}}{\pgfqpoint{2.785279in}{0.903359in}}{\pgfqpoint{2.796329in}{0.903359in}}%
\pgfpathlineto{\pgfqpoint{2.796329in}{0.903359in}}%
\pgfpathclose%
\pgfusepath{stroke}%
\end{pgfscope}%
\begin{pgfscope}%
\pgfpathrectangle{\pgfqpoint{0.393053in}{0.375000in}}{\pgfqpoint{6.356833in}{5.175000in}}%
\pgfusepath{clip}%
\pgfsetbuttcap%
\pgfsetroundjoin%
\pgfsetlinewidth{1.003750pt}%
\definecolor{currentstroke}{rgb}{0.827451,0.827451,0.827451}%
\pgfsetstrokecolor{currentstroke}%
\pgfsetdash{}{0pt}%
\pgfpathmoveto{\pgfqpoint{1.671050in}{1.714593in}}%
\pgfpathcurveto{\pgfqpoint{1.682100in}{1.714593in}}{\pgfqpoint{1.692699in}{1.718984in}}{\pgfqpoint{1.700512in}{1.726797in}}%
\pgfpathcurveto{\pgfqpoint{1.708326in}{1.734611in}}{\pgfqpoint{1.712716in}{1.745210in}}{\pgfqpoint{1.712716in}{1.756260in}}%
\pgfpathcurveto{\pgfqpoint{1.712716in}{1.767310in}}{\pgfqpoint{1.708326in}{1.777909in}}{\pgfqpoint{1.700512in}{1.785723in}}%
\pgfpathcurveto{\pgfqpoint{1.692699in}{1.793537in}}{\pgfqpoint{1.682100in}{1.797927in}}{\pgfqpoint{1.671050in}{1.797927in}}%
\pgfpathcurveto{\pgfqpoint{1.659999in}{1.797927in}}{\pgfqpoint{1.649400in}{1.793537in}}{\pgfqpoint{1.641587in}{1.785723in}}%
\pgfpathcurveto{\pgfqpoint{1.633773in}{1.777909in}}{\pgfqpoint{1.629383in}{1.767310in}}{\pgfqpoint{1.629383in}{1.756260in}}%
\pgfpathcurveto{\pgfqpoint{1.629383in}{1.745210in}}{\pgfqpoint{1.633773in}{1.734611in}}{\pgfqpoint{1.641587in}{1.726797in}}%
\pgfpathcurveto{\pgfqpoint{1.649400in}{1.718984in}}{\pgfqpoint{1.659999in}{1.714593in}}{\pgfqpoint{1.671050in}{1.714593in}}%
\pgfpathlineto{\pgfqpoint{1.671050in}{1.714593in}}%
\pgfpathclose%
\pgfusepath{stroke}%
\end{pgfscope}%
\begin{pgfscope}%
\pgfpathrectangle{\pgfqpoint{0.393053in}{0.375000in}}{\pgfqpoint{6.356833in}{5.175000in}}%
\pgfusepath{clip}%
\pgfsetbuttcap%
\pgfsetroundjoin%
\pgfsetlinewidth{1.003750pt}%
\definecolor{currentstroke}{rgb}{0.827451,0.827451,0.827451}%
\pgfsetstrokecolor{currentstroke}%
\pgfsetdash{}{0pt}%
\pgfpathmoveto{\pgfqpoint{2.498084in}{1.056522in}}%
\pgfpathcurveto{\pgfqpoint{2.509134in}{1.056522in}}{\pgfqpoint{2.519733in}{1.060912in}}{\pgfqpoint{2.527547in}{1.068726in}}%
\pgfpathcurveto{\pgfqpoint{2.535360in}{1.076539in}}{\pgfqpoint{2.539750in}{1.087138in}}{\pgfqpoint{2.539750in}{1.098188in}}%
\pgfpathcurveto{\pgfqpoint{2.539750in}{1.109239in}}{\pgfqpoint{2.535360in}{1.119838in}}{\pgfqpoint{2.527547in}{1.127651in}}%
\pgfpathcurveto{\pgfqpoint{2.519733in}{1.135465in}}{\pgfqpoint{2.509134in}{1.139855in}}{\pgfqpoint{2.498084in}{1.139855in}}%
\pgfpathcurveto{\pgfqpoint{2.487034in}{1.139855in}}{\pgfqpoint{2.476435in}{1.135465in}}{\pgfqpoint{2.468621in}{1.127651in}}%
\pgfpathcurveto{\pgfqpoint{2.460807in}{1.119838in}}{\pgfqpoint{2.456417in}{1.109239in}}{\pgfqpoint{2.456417in}{1.098188in}}%
\pgfpathcurveto{\pgfqpoint{2.456417in}{1.087138in}}{\pgfqpoint{2.460807in}{1.076539in}}{\pgfqpoint{2.468621in}{1.068726in}}%
\pgfpathcurveto{\pgfqpoint{2.476435in}{1.060912in}}{\pgfqpoint{2.487034in}{1.056522in}}{\pgfqpoint{2.498084in}{1.056522in}}%
\pgfpathlineto{\pgfqpoint{2.498084in}{1.056522in}}%
\pgfpathclose%
\pgfusepath{stroke}%
\end{pgfscope}%
\begin{pgfscope}%
\pgfpathrectangle{\pgfqpoint{0.393053in}{0.375000in}}{\pgfqpoint{6.356833in}{5.175000in}}%
\pgfusepath{clip}%
\pgfsetbuttcap%
\pgfsetroundjoin%
\pgfsetlinewidth{1.003750pt}%
\definecolor{currentstroke}{rgb}{0.827451,0.827451,0.827451}%
\pgfsetstrokecolor{currentstroke}%
\pgfsetdash{}{0pt}%
\pgfpathmoveto{\pgfqpoint{0.868151in}{2.818774in}}%
\pgfpathcurveto{\pgfqpoint{0.879201in}{2.818774in}}{\pgfqpoint{0.889800in}{2.823165in}}{\pgfqpoint{0.897614in}{2.830978in}}%
\pgfpathcurveto{\pgfqpoint{0.905428in}{2.838792in}}{\pgfqpoint{0.909818in}{2.849391in}}{\pgfqpoint{0.909818in}{2.860441in}}%
\pgfpathcurveto{\pgfqpoint{0.909818in}{2.871491in}}{\pgfqpoint{0.905428in}{2.882090in}}{\pgfqpoint{0.897614in}{2.889904in}}%
\pgfpathcurveto{\pgfqpoint{0.889800in}{2.897717in}}{\pgfqpoint{0.879201in}{2.902108in}}{\pgfqpoint{0.868151in}{2.902108in}}%
\pgfpathcurveto{\pgfqpoint{0.857101in}{2.902108in}}{\pgfqpoint{0.846502in}{2.897717in}}{\pgfqpoint{0.838688in}{2.889904in}}%
\pgfpathcurveto{\pgfqpoint{0.830875in}{2.882090in}}{\pgfqpoint{0.826484in}{2.871491in}}{\pgfqpoint{0.826484in}{2.860441in}}%
\pgfpathcurveto{\pgfqpoint{0.826484in}{2.849391in}}{\pgfqpoint{0.830875in}{2.838792in}}{\pgfqpoint{0.838688in}{2.830978in}}%
\pgfpathcurveto{\pgfqpoint{0.846502in}{2.823165in}}{\pgfqpoint{0.857101in}{2.818774in}}{\pgfqpoint{0.868151in}{2.818774in}}%
\pgfpathlineto{\pgfqpoint{0.868151in}{2.818774in}}%
\pgfpathclose%
\pgfusepath{stroke}%
\end{pgfscope}%
\begin{pgfscope}%
\pgfpathrectangle{\pgfqpoint{0.393053in}{0.375000in}}{\pgfqpoint{6.356833in}{5.175000in}}%
\pgfusepath{clip}%
\pgfsetbuttcap%
\pgfsetroundjoin%
\pgfsetlinewidth{1.003750pt}%
\definecolor{currentstroke}{rgb}{0.827451,0.827451,0.827451}%
\pgfsetstrokecolor{currentstroke}%
\pgfsetdash{}{0pt}%
\pgfpathmoveto{\pgfqpoint{2.437419in}{1.106016in}}%
\pgfpathcurveto{\pgfqpoint{2.448469in}{1.106016in}}{\pgfqpoint{2.459069in}{1.110406in}}{\pgfqpoint{2.466882in}{1.118219in}}%
\pgfpathcurveto{\pgfqpoint{2.474696in}{1.126033in}}{\pgfqpoint{2.479086in}{1.136632in}}{\pgfqpoint{2.479086in}{1.147682in}}%
\pgfpathcurveto{\pgfqpoint{2.479086in}{1.158732in}}{\pgfqpoint{2.474696in}{1.169331in}}{\pgfqpoint{2.466882in}{1.177145in}}%
\pgfpathcurveto{\pgfqpoint{2.459069in}{1.184959in}}{\pgfqpoint{2.448469in}{1.189349in}}{\pgfqpoint{2.437419in}{1.189349in}}%
\pgfpathcurveto{\pgfqpoint{2.426369in}{1.189349in}}{\pgfqpoint{2.415770in}{1.184959in}}{\pgfqpoint{2.407957in}{1.177145in}}%
\pgfpathcurveto{\pgfqpoint{2.400143in}{1.169331in}}{\pgfqpoint{2.395753in}{1.158732in}}{\pgfqpoint{2.395753in}{1.147682in}}%
\pgfpathcurveto{\pgfqpoint{2.395753in}{1.136632in}}{\pgfqpoint{2.400143in}{1.126033in}}{\pgfqpoint{2.407957in}{1.118219in}}%
\pgfpathcurveto{\pgfqpoint{2.415770in}{1.110406in}}{\pgfqpoint{2.426369in}{1.106016in}}{\pgfqpoint{2.437419in}{1.106016in}}%
\pgfpathlineto{\pgfqpoint{2.437419in}{1.106016in}}%
\pgfpathclose%
\pgfusepath{stroke}%
\end{pgfscope}%
\begin{pgfscope}%
\pgfpathrectangle{\pgfqpoint{0.393053in}{0.375000in}}{\pgfqpoint{6.356833in}{5.175000in}}%
\pgfusepath{clip}%
\pgfsetbuttcap%
\pgfsetroundjoin%
\pgfsetlinewidth{1.003750pt}%
\definecolor{currentstroke}{rgb}{0.827451,0.827451,0.827451}%
\pgfsetstrokecolor{currentstroke}%
\pgfsetdash{}{0pt}%
\pgfpathmoveto{\pgfqpoint{0.921910in}{2.777569in}}%
\pgfpathcurveto{\pgfqpoint{0.932960in}{2.777569in}}{\pgfqpoint{0.943559in}{2.781959in}}{\pgfqpoint{0.951373in}{2.789773in}}%
\pgfpathcurveto{\pgfqpoint{0.959186in}{2.797586in}}{\pgfqpoint{0.963577in}{2.808185in}}{\pgfqpoint{0.963577in}{2.819235in}}%
\pgfpathcurveto{\pgfqpoint{0.963577in}{2.830286in}}{\pgfqpoint{0.959186in}{2.840885in}}{\pgfqpoint{0.951373in}{2.848698in}}%
\pgfpathcurveto{\pgfqpoint{0.943559in}{2.856512in}}{\pgfqpoint{0.932960in}{2.860902in}}{\pgfqpoint{0.921910in}{2.860902in}}%
\pgfpathcurveto{\pgfqpoint{0.910860in}{2.860902in}}{\pgfqpoint{0.900261in}{2.856512in}}{\pgfqpoint{0.892447in}{2.848698in}}%
\pgfpathcurveto{\pgfqpoint{0.884633in}{2.840885in}}{\pgfqpoint{0.880243in}{2.830286in}}{\pgfqpoint{0.880243in}{2.819235in}}%
\pgfpathcurveto{\pgfqpoint{0.880243in}{2.808185in}}{\pgfqpoint{0.884633in}{2.797586in}}{\pgfqpoint{0.892447in}{2.789773in}}%
\pgfpathcurveto{\pgfqpoint{0.900261in}{2.781959in}}{\pgfqpoint{0.910860in}{2.777569in}}{\pgfqpoint{0.921910in}{2.777569in}}%
\pgfpathlineto{\pgfqpoint{0.921910in}{2.777569in}}%
\pgfpathclose%
\pgfusepath{stroke}%
\end{pgfscope}%
\begin{pgfscope}%
\pgfpathrectangle{\pgfqpoint{0.393053in}{0.375000in}}{\pgfqpoint{6.356833in}{5.175000in}}%
\pgfusepath{clip}%
\pgfsetbuttcap%
\pgfsetroundjoin%
\pgfsetlinewidth{1.003750pt}%
\definecolor{currentstroke}{rgb}{0.827451,0.827451,0.827451}%
\pgfsetstrokecolor{currentstroke}%
\pgfsetdash{}{0pt}%
\pgfpathmoveto{\pgfqpoint{3.649696in}{0.625044in}}%
\pgfpathcurveto{\pgfqpoint{3.660746in}{0.625044in}}{\pgfqpoint{3.671345in}{0.629434in}}{\pgfqpoint{3.679159in}{0.637248in}}%
\pgfpathcurveto{\pgfqpoint{3.686972in}{0.645061in}}{\pgfqpoint{3.691363in}{0.655660in}}{\pgfqpoint{3.691363in}{0.666710in}}%
\pgfpathcurveto{\pgfqpoint{3.691363in}{0.677761in}}{\pgfqpoint{3.686972in}{0.688360in}}{\pgfqpoint{3.679159in}{0.696173in}}%
\pgfpathcurveto{\pgfqpoint{3.671345in}{0.703987in}}{\pgfqpoint{3.660746in}{0.708377in}}{\pgfqpoint{3.649696in}{0.708377in}}%
\pgfpathcurveto{\pgfqpoint{3.638646in}{0.708377in}}{\pgfqpoint{3.628047in}{0.703987in}}{\pgfqpoint{3.620233in}{0.696173in}}%
\pgfpathcurveto{\pgfqpoint{3.612420in}{0.688360in}}{\pgfqpoint{3.608029in}{0.677761in}}{\pgfqpoint{3.608029in}{0.666710in}}%
\pgfpathcurveto{\pgfqpoint{3.608029in}{0.655660in}}{\pgfqpoint{3.612420in}{0.645061in}}{\pgfqpoint{3.620233in}{0.637248in}}%
\pgfpathcurveto{\pgfqpoint{3.628047in}{0.629434in}}{\pgfqpoint{3.638646in}{0.625044in}}{\pgfqpoint{3.649696in}{0.625044in}}%
\pgfpathlineto{\pgfqpoint{3.649696in}{0.625044in}}%
\pgfpathclose%
\pgfusepath{stroke}%
\end{pgfscope}%
\begin{pgfscope}%
\pgfpathrectangle{\pgfqpoint{0.393053in}{0.375000in}}{\pgfqpoint{6.356833in}{5.175000in}}%
\pgfusepath{clip}%
\pgfsetbuttcap%
\pgfsetroundjoin%
\pgfsetlinewidth{1.003750pt}%
\definecolor{currentstroke}{rgb}{0.827451,0.827451,0.827451}%
\pgfsetstrokecolor{currentstroke}%
\pgfsetdash{}{0pt}%
\pgfpathmoveto{\pgfqpoint{0.762565in}{3.034222in}}%
\pgfpathcurveto{\pgfqpoint{0.773615in}{3.034222in}}{\pgfqpoint{0.784214in}{3.038612in}}{\pgfqpoint{0.792028in}{3.046426in}}%
\pgfpathcurveto{\pgfqpoint{0.799841in}{3.054240in}}{\pgfqpoint{0.804232in}{3.064839in}}{\pgfqpoint{0.804232in}{3.075889in}}%
\pgfpathcurveto{\pgfqpoint{0.804232in}{3.086939in}}{\pgfqpoint{0.799841in}{3.097538in}}{\pgfqpoint{0.792028in}{3.105352in}}%
\pgfpathcurveto{\pgfqpoint{0.784214in}{3.113165in}}{\pgfqpoint{0.773615in}{3.117556in}}{\pgfqpoint{0.762565in}{3.117556in}}%
\pgfpathcurveto{\pgfqpoint{0.751515in}{3.117556in}}{\pgfqpoint{0.740916in}{3.113165in}}{\pgfqpoint{0.733102in}{3.105352in}}%
\pgfpathcurveto{\pgfqpoint{0.725289in}{3.097538in}}{\pgfqpoint{0.720898in}{3.086939in}}{\pgfqpoint{0.720898in}{3.075889in}}%
\pgfpathcurveto{\pgfqpoint{0.720898in}{3.064839in}}{\pgfqpoint{0.725289in}{3.054240in}}{\pgfqpoint{0.733102in}{3.046426in}}%
\pgfpathcurveto{\pgfqpoint{0.740916in}{3.038612in}}{\pgfqpoint{0.751515in}{3.034222in}}{\pgfqpoint{0.762565in}{3.034222in}}%
\pgfpathlineto{\pgfqpoint{0.762565in}{3.034222in}}%
\pgfpathclose%
\pgfusepath{stroke}%
\end{pgfscope}%
\begin{pgfscope}%
\pgfpathrectangle{\pgfqpoint{0.393053in}{0.375000in}}{\pgfqpoint{6.356833in}{5.175000in}}%
\pgfusepath{clip}%
\pgfsetbuttcap%
\pgfsetroundjoin%
\pgfsetlinewidth{1.003750pt}%
\definecolor{currentstroke}{rgb}{0.827451,0.827451,0.827451}%
\pgfsetstrokecolor{currentstroke}%
\pgfsetdash{}{0pt}%
\pgfpathmoveto{\pgfqpoint{1.457810in}{1.831739in}}%
\pgfpathcurveto{\pgfqpoint{1.468860in}{1.831739in}}{\pgfqpoint{1.479459in}{1.836129in}}{\pgfqpoint{1.487273in}{1.843943in}}%
\pgfpathcurveto{\pgfqpoint{1.495086in}{1.851756in}}{\pgfqpoint{1.499477in}{1.862355in}}{\pgfqpoint{1.499477in}{1.873406in}}%
\pgfpathcurveto{\pgfqpoint{1.499477in}{1.884456in}}{\pgfqpoint{1.495086in}{1.895055in}}{\pgfqpoint{1.487273in}{1.902868in}}%
\pgfpathcurveto{\pgfqpoint{1.479459in}{1.910682in}}{\pgfqpoint{1.468860in}{1.915072in}}{\pgfqpoint{1.457810in}{1.915072in}}%
\pgfpathcurveto{\pgfqpoint{1.446760in}{1.915072in}}{\pgfqpoint{1.436161in}{1.910682in}}{\pgfqpoint{1.428347in}{1.902868in}}%
\pgfpathcurveto{\pgfqpoint{1.420533in}{1.895055in}}{\pgfqpoint{1.416143in}{1.884456in}}{\pgfqpoint{1.416143in}{1.873406in}}%
\pgfpathcurveto{\pgfqpoint{1.416143in}{1.862355in}}{\pgfqpoint{1.420533in}{1.851756in}}{\pgfqpoint{1.428347in}{1.843943in}}%
\pgfpathcurveto{\pgfqpoint{1.436161in}{1.836129in}}{\pgfqpoint{1.446760in}{1.831739in}}{\pgfqpoint{1.457810in}{1.831739in}}%
\pgfpathlineto{\pgfqpoint{1.457810in}{1.831739in}}%
\pgfpathclose%
\pgfusepath{stroke}%
\end{pgfscope}%
\begin{pgfscope}%
\pgfpathrectangle{\pgfqpoint{0.393053in}{0.375000in}}{\pgfqpoint{6.356833in}{5.175000in}}%
\pgfusepath{clip}%
\pgfsetbuttcap%
\pgfsetroundjoin%
\pgfsetlinewidth{1.003750pt}%
\definecolor{currentstroke}{rgb}{0.827451,0.827451,0.827451}%
\pgfsetstrokecolor{currentstroke}%
\pgfsetdash{}{0pt}%
\pgfpathmoveto{\pgfqpoint{0.418752in}{4.107793in}}%
\pgfpathcurveto{\pgfqpoint{0.429802in}{4.107793in}}{\pgfqpoint{0.440401in}{4.112183in}}{\pgfqpoint{0.448215in}{4.119996in}}%
\pgfpathcurveto{\pgfqpoint{0.456028in}{4.127810in}}{\pgfqpoint{0.460419in}{4.138409in}}{\pgfqpoint{0.460419in}{4.149459in}}%
\pgfpathcurveto{\pgfqpoint{0.460419in}{4.160509in}}{\pgfqpoint{0.456028in}{4.171108in}}{\pgfqpoint{0.448215in}{4.178922in}}%
\pgfpathcurveto{\pgfqpoint{0.440401in}{4.186736in}}{\pgfqpoint{0.429802in}{4.191126in}}{\pgfqpoint{0.418752in}{4.191126in}}%
\pgfpathcurveto{\pgfqpoint{0.407702in}{4.191126in}}{\pgfqpoint{0.397103in}{4.186736in}}{\pgfqpoint{0.389289in}{4.178922in}}%
\pgfpathcurveto{\pgfqpoint{0.381476in}{4.171108in}}{\pgfqpoint{0.377085in}{4.160509in}}{\pgfqpoint{0.377085in}{4.149459in}}%
\pgfpathcurveto{\pgfqpoint{0.377085in}{4.138409in}}{\pgfqpoint{0.381476in}{4.127810in}}{\pgfqpoint{0.389289in}{4.119996in}}%
\pgfpathcurveto{\pgfqpoint{0.397103in}{4.112183in}}{\pgfqpoint{0.407702in}{4.107793in}}{\pgfqpoint{0.418752in}{4.107793in}}%
\pgfpathlineto{\pgfqpoint{0.418752in}{4.107793in}}%
\pgfpathclose%
\pgfusepath{stroke}%
\end{pgfscope}%
\begin{pgfscope}%
\pgfpathrectangle{\pgfqpoint{0.393053in}{0.375000in}}{\pgfqpoint{6.356833in}{5.175000in}}%
\pgfusepath{clip}%
\pgfsetbuttcap%
\pgfsetroundjoin%
\pgfsetlinewidth{1.003750pt}%
\definecolor{currentstroke}{rgb}{0.827451,0.827451,0.827451}%
\pgfsetstrokecolor{currentstroke}%
\pgfsetdash{}{0pt}%
\pgfpathmoveto{\pgfqpoint{1.306751in}{1.985193in}}%
\pgfpathcurveto{\pgfqpoint{1.317801in}{1.985193in}}{\pgfqpoint{1.328400in}{1.989583in}}{\pgfqpoint{1.336214in}{1.997397in}}%
\pgfpathcurveto{\pgfqpoint{1.344027in}{2.005210in}}{\pgfqpoint{1.348418in}{2.015809in}}{\pgfqpoint{1.348418in}{2.026859in}}%
\pgfpathcurveto{\pgfqpoint{1.348418in}{2.037909in}}{\pgfqpoint{1.344027in}{2.048509in}}{\pgfqpoint{1.336214in}{2.056322in}}%
\pgfpathcurveto{\pgfqpoint{1.328400in}{2.064136in}}{\pgfqpoint{1.317801in}{2.068526in}}{\pgfqpoint{1.306751in}{2.068526in}}%
\pgfpathcurveto{\pgfqpoint{1.295701in}{2.068526in}}{\pgfqpoint{1.285102in}{2.064136in}}{\pgfqpoint{1.277288in}{2.056322in}}%
\pgfpathcurveto{\pgfqpoint{1.269475in}{2.048509in}}{\pgfqpoint{1.265084in}{2.037909in}}{\pgfqpoint{1.265084in}{2.026859in}}%
\pgfpathcurveto{\pgfqpoint{1.265084in}{2.015809in}}{\pgfqpoint{1.269475in}{2.005210in}}{\pgfqpoint{1.277288in}{1.997397in}}%
\pgfpathcurveto{\pgfqpoint{1.285102in}{1.989583in}}{\pgfqpoint{1.295701in}{1.985193in}}{\pgfqpoint{1.306751in}{1.985193in}}%
\pgfpathlineto{\pgfqpoint{1.306751in}{1.985193in}}%
\pgfpathclose%
\pgfusepath{stroke}%
\end{pgfscope}%
\begin{pgfscope}%
\pgfpathrectangle{\pgfqpoint{0.393053in}{0.375000in}}{\pgfqpoint{6.356833in}{5.175000in}}%
\pgfusepath{clip}%
\pgfsetbuttcap%
\pgfsetroundjoin%
\pgfsetlinewidth{1.003750pt}%
\definecolor{currentstroke}{rgb}{0.827451,0.827451,0.827451}%
\pgfsetstrokecolor{currentstroke}%
\pgfsetdash{}{0pt}%
\pgfpathmoveto{\pgfqpoint{0.537744in}{3.497149in}}%
\pgfpathcurveto{\pgfqpoint{0.548795in}{3.497149in}}{\pgfqpoint{0.559394in}{3.501540in}}{\pgfqpoint{0.567207in}{3.509353in}}%
\pgfpathcurveto{\pgfqpoint{0.575021in}{3.517167in}}{\pgfqpoint{0.579411in}{3.527766in}}{\pgfqpoint{0.579411in}{3.538816in}}%
\pgfpathcurveto{\pgfqpoint{0.579411in}{3.549866in}}{\pgfqpoint{0.575021in}{3.560465in}}{\pgfqpoint{0.567207in}{3.568279in}}%
\pgfpathcurveto{\pgfqpoint{0.559394in}{3.576092in}}{\pgfqpoint{0.548795in}{3.580483in}}{\pgfqpoint{0.537744in}{3.580483in}}%
\pgfpathcurveto{\pgfqpoint{0.526694in}{3.580483in}}{\pgfqpoint{0.516095in}{3.576092in}}{\pgfqpoint{0.508282in}{3.568279in}}%
\pgfpathcurveto{\pgfqpoint{0.500468in}{3.560465in}}{\pgfqpoint{0.496078in}{3.549866in}}{\pgfqpoint{0.496078in}{3.538816in}}%
\pgfpathcurveto{\pgfqpoint{0.496078in}{3.527766in}}{\pgfqpoint{0.500468in}{3.517167in}}{\pgfqpoint{0.508282in}{3.509353in}}%
\pgfpathcurveto{\pgfqpoint{0.516095in}{3.501540in}}{\pgfqpoint{0.526694in}{3.497149in}}{\pgfqpoint{0.537744in}{3.497149in}}%
\pgfpathlineto{\pgfqpoint{0.537744in}{3.497149in}}%
\pgfpathclose%
\pgfusepath{stroke}%
\end{pgfscope}%
\begin{pgfscope}%
\pgfpathrectangle{\pgfqpoint{0.393053in}{0.375000in}}{\pgfqpoint{6.356833in}{5.175000in}}%
\pgfusepath{clip}%
\pgfsetbuttcap%
\pgfsetroundjoin%
\pgfsetlinewidth{1.003750pt}%
\definecolor{currentstroke}{rgb}{0.827451,0.827451,0.827451}%
\pgfsetstrokecolor{currentstroke}%
\pgfsetdash{}{0pt}%
\pgfpathmoveto{\pgfqpoint{1.255595in}{2.043062in}}%
\pgfpathcurveto{\pgfqpoint{1.266645in}{2.043062in}}{\pgfqpoint{1.277244in}{2.047452in}}{\pgfqpoint{1.285058in}{2.055266in}}%
\pgfpathcurveto{\pgfqpoint{1.292872in}{2.063079in}}{\pgfqpoint{1.297262in}{2.073678in}}{\pgfqpoint{1.297262in}{2.084728in}}%
\pgfpathcurveto{\pgfqpoint{1.297262in}{2.095779in}}{\pgfqpoint{1.292872in}{2.106378in}}{\pgfqpoint{1.285058in}{2.114191in}}%
\pgfpathcurveto{\pgfqpoint{1.277244in}{2.122005in}}{\pgfqpoint{1.266645in}{2.126395in}}{\pgfqpoint{1.255595in}{2.126395in}}%
\pgfpathcurveto{\pgfqpoint{1.244545in}{2.126395in}}{\pgfqpoint{1.233946in}{2.122005in}}{\pgfqpoint{1.226132in}{2.114191in}}%
\pgfpathcurveto{\pgfqpoint{1.218319in}{2.106378in}}{\pgfqpoint{1.213928in}{2.095779in}}{\pgfqpoint{1.213928in}{2.084728in}}%
\pgfpathcurveto{\pgfqpoint{1.213928in}{2.073678in}}{\pgfqpoint{1.218319in}{2.063079in}}{\pgfqpoint{1.226132in}{2.055266in}}%
\pgfpathcurveto{\pgfqpoint{1.233946in}{2.047452in}}{\pgfqpoint{1.244545in}{2.043062in}}{\pgfqpoint{1.255595in}{2.043062in}}%
\pgfpathlineto{\pgfqpoint{1.255595in}{2.043062in}}%
\pgfpathclose%
\pgfusepath{stroke}%
\end{pgfscope}%
\begin{pgfscope}%
\pgfpathrectangle{\pgfqpoint{0.393053in}{0.375000in}}{\pgfqpoint{6.356833in}{5.175000in}}%
\pgfusepath{clip}%
\pgfsetbuttcap%
\pgfsetroundjoin%
\pgfsetlinewidth{1.003750pt}%
\definecolor{currentstroke}{rgb}{0.827451,0.827451,0.827451}%
\pgfsetstrokecolor{currentstroke}%
\pgfsetdash{}{0pt}%
\pgfpathmoveto{\pgfqpoint{0.580211in}{3.378166in}}%
\pgfpathcurveto{\pgfqpoint{0.591261in}{3.378166in}}{\pgfqpoint{0.601860in}{3.382556in}}{\pgfqpoint{0.609674in}{3.390370in}}%
\pgfpathcurveto{\pgfqpoint{0.617487in}{3.398183in}}{\pgfqpoint{0.621878in}{3.408782in}}{\pgfqpoint{0.621878in}{3.419833in}}%
\pgfpathcurveto{\pgfqpoint{0.621878in}{3.430883in}}{\pgfqpoint{0.617487in}{3.441482in}}{\pgfqpoint{0.609674in}{3.449295in}}%
\pgfpathcurveto{\pgfqpoint{0.601860in}{3.457109in}}{\pgfqpoint{0.591261in}{3.461499in}}{\pgfqpoint{0.580211in}{3.461499in}}%
\pgfpathcurveto{\pgfqpoint{0.569161in}{3.461499in}}{\pgfqpoint{0.558562in}{3.457109in}}{\pgfqpoint{0.550748in}{3.449295in}}%
\pgfpathcurveto{\pgfqpoint{0.542935in}{3.441482in}}{\pgfqpoint{0.538544in}{3.430883in}}{\pgfqpoint{0.538544in}{3.419833in}}%
\pgfpathcurveto{\pgfqpoint{0.538544in}{3.408782in}}{\pgfqpoint{0.542935in}{3.398183in}}{\pgfqpoint{0.550748in}{3.390370in}}%
\pgfpathcurveto{\pgfqpoint{0.558562in}{3.382556in}}{\pgfqpoint{0.569161in}{3.378166in}}{\pgfqpoint{0.580211in}{3.378166in}}%
\pgfpathlineto{\pgfqpoint{0.580211in}{3.378166in}}%
\pgfpathclose%
\pgfusepath{stroke}%
\end{pgfscope}%
\begin{pgfscope}%
\pgfpathrectangle{\pgfqpoint{0.393053in}{0.375000in}}{\pgfqpoint{6.356833in}{5.175000in}}%
\pgfusepath{clip}%
\pgfsetbuttcap%
\pgfsetroundjoin%
\pgfsetlinewidth{1.003750pt}%
\definecolor{currentstroke}{rgb}{0.827451,0.827451,0.827451}%
\pgfsetstrokecolor{currentstroke}%
\pgfsetdash{}{0pt}%
\pgfpathmoveto{\pgfqpoint{3.893477in}{0.565646in}}%
\pgfpathcurveto{\pgfqpoint{3.904527in}{0.565646in}}{\pgfqpoint{3.915126in}{0.570036in}}{\pgfqpoint{3.922939in}{0.577850in}}%
\pgfpathcurveto{\pgfqpoint{3.930753in}{0.585664in}}{\pgfqpoint{3.935143in}{0.596263in}}{\pgfqpoint{3.935143in}{0.607313in}}%
\pgfpathcurveto{\pgfqpoint{3.935143in}{0.618363in}}{\pgfqpoint{3.930753in}{0.628962in}}{\pgfqpoint{3.922939in}{0.636776in}}%
\pgfpathcurveto{\pgfqpoint{3.915126in}{0.644589in}}{\pgfqpoint{3.904527in}{0.648980in}}{\pgfqpoint{3.893477in}{0.648980in}}%
\pgfpathcurveto{\pgfqpoint{3.882426in}{0.648980in}}{\pgfqpoint{3.871827in}{0.644589in}}{\pgfqpoint{3.864014in}{0.636776in}}%
\pgfpathcurveto{\pgfqpoint{3.856200in}{0.628962in}}{\pgfqpoint{3.851810in}{0.618363in}}{\pgfqpoint{3.851810in}{0.607313in}}%
\pgfpathcurveto{\pgfqpoint{3.851810in}{0.596263in}}{\pgfqpoint{3.856200in}{0.585664in}}{\pgfqpoint{3.864014in}{0.577850in}}%
\pgfpathcurveto{\pgfqpoint{3.871827in}{0.570036in}}{\pgfqpoint{3.882426in}{0.565646in}}{\pgfqpoint{3.893477in}{0.565646in}}%
\pgfpathlineto{\pgfqpoint{3.893477in}{0.565646in}}%
\pgfpathclose%
\pgfusepath{stroke}%
\end{pgfscope}%
\begin{pgfscope}%
\pgfpathrectangle{\pgfqpoint{0.393053in}{0.375000in}}{\pgfqpoint{6.356833in}{5.175000in}}%
\pgfusepath{clip}%
\pgfsetbuttcap%
\pgfsetroundjoin%
\pgfsetlinewidth{1.003750pt}%
\definecolor{currentstroke}{rgb}{0.827451,0.827451,0.827451}%
\pgfsetstrokecolor{currentstroke}%
\pgfsetdash{}{0pt}%
\pgfpathmoveto{\pgfqpoint{4.898856in}{0.391247in}}%
\pgfpathcurveto{\pgfqpoint{4.909906in}{0.391247in}}{\pgfqpoint{4.920505in}{0.395637in}}{\pgfqpoint{4.928319in}{0.403450in}}%
\pgfpathcurveto{\pgfqpoint{4.936132in}{0.411264in}}{\pgfqpoint{4.940523in}{0.421863in}}{\pgfqpoint{4.940523in}{0.432913in}}%
\pgfpathcurveto{\pgfqpoint{4.940523in}{0.443963in}}{\pgfqpoint{4.936132in}{0.454562in}}{\pgfqpoint{4.928319in}{0.462376in}}%
\pgfpathcurveto{\pgfqpoint{4.920505in}{0.470190in}}{\pgfqpoint{4.909906in}{0.474580in}}{\pgfqpoint{4.898856in}{0.474580in}}%
\pgfpathcurveto{\pgfqpoint{4.887806in}{0.474580in}}{\pgfqpoint{4.877207in}{0.470190in}}{\pgfqpoint{4.869393in}{0.462376in}}%
\pgfpathcurveto{\pgfqpoint{4.861580in}{0.454562in}}{\pgfqpoint{4.857189in}{0.443963in}}{\pgfqpoint{4.857189in}{0.432913in}}%
\pgfpathcurveto{\pgfqpoint{4.857189in}{0.421863in}}{\pgfqpoint{4.861580in}{0.411264in}}{\pgfqpoint{4.869393in}{0.403450in}}%
\pgfpathcurveto{\pgfqpoint{4.877207in}{0.395637in}}{\pgfqpoint{4.887806in}{0.391247in}}{\pgfqpoint{4.898856in}{0.391247in}}%
\pgfpathlineto{\pgfqpoint{4.898856in}{0.391247in}}%
\pgfpathclose%
\pgfusepath{stroke}%
\end{pgfscope}%
\begin{pgfscope}%
\pgfpathrectangle{\pgfqpoint{0.393053in}{0.375000in}}{\pgfqpoint{6.356833in}{5.175000in}}%
\pgfusepath{clip}%
\pgfsetbuttcap%
\pgfsetroundjoin%
\pgfsetlinewidth{1.003750pt}%
\definecolor{currentstroke}{rgb}{0.827451,0.827451,0.827451}%
\pgfsetstrokecolor{currentstroke}%
\pgfsetdash{}{0pt}%
\pgfpathmoveto{\pgfqpoint{1.843211in}{1.515821in}}%
\pgfpathcurveto{\pgfqpoint{1.854261in}{1.515821in}}{\pgfqpoint{1.864860in}{1.520211in}}{\pgfqpoint{1.872674in}{1.528025in}}%
\pgfpathcurveto{\pgfqpoint{1.880487in}{1.535838in}}{\pgfqpoint{1.884878in}{1.546437in}}{\pgfqpoint{1.884878in}{1.557487in}}%
\pgfpathcurveto{\pgfqpoint{1.884878in}{1.568537in}}{\pgfqpoint{1.880487in}{1.579136in}}{\pgfqpoint{1.872674in}{1.586950in}}%
\pgfpathcurveto{\pgfqpoint{1.864860in}{1.594764in}}{\pgfqpoint{1.854261in}{1.599154in}}{\pgfqpoint{1.843211in}{1.599154in}}%
\pgfpathcurveto{\pgfqpoint{1.832161in}{1.599154in}}{\pgfqpoint{1.821562in}{1.594764in}}{\pgfqpoint{1.813748in}{1.586950in}}%
\pgfpathcurveto{\pgfqpoint{1.805934in}{1.579136in}}{\pgfqpoint{1.801544in}{1.568537in}}{\pgfqpoint{1.801544in}{1.557487in}}%
\pgfpathcurveto{\pgfqpoint{1.801544in}{1.546437in}}{\pgfqpoint{1.805934in}{1.535838in}}{\pgfqpoint{1.813748in}{1.528025in}}%
\pgfpathcurveto{\pgfqpoint{1.821562in}{1.520211in}}{\pgfqpoint{1.832161in}{1.515821in}}{\pgfqpoint{1.843211in}{1.515821in}}%
\pgfpathlineto{\pgfqpoint{1.843211in}{1.515821in}}%
\pgfpathclose%
\pgfusepath{stroke}%
\end{pgfscope}%
\begin{pgfscope}%
\pgfpathrectangle{\pgfqpoint{0.393053in}{0.375000in}}{\pgfqpoint{6.356833in}{5.175000in}}%
\pgfusepath{clip}%
\pgfsetbuttcap%
\pgfsetroundjoin%
\pgfsetlinewidth{1.003750pt}%
\definecolor{currentstroke}{rgb}{0.827451,0.827451,0.827451}%
\pgfsetstrokecolor{currentstroke}%
\pgfsetdash{}{0pt}%
\pgfpathmoveto{\pgfqpoint{1.680974in}{1.628980in}}%
\pgfpathcurveto{\pgfqpoint{1.692024in}{1.628980in}}{\pgfqpoint{1.702623in}{1.633370in}}{\pgfqpoint{1.710437in}{1.641184in}}%
\pgfpathcurveto{\pgfqpoint{1.718250in}{1.648997in}}{\pgfqpoint{1.722641in}{1.659596in}}{\pgfqpoint{1.722641in}{1.670646in}}%
\pgfpathcurveto{\pgfqpoint{1.722641in}{1.681697in}}{\pgfqpoint{1.718250in}{1.692296in}}{\pgfqpoint{1.710437in}{1.700109in}}%
\pgfpathcurveto{\pgfqpoint{1.702623in}{1.707923in}}{\pgfqpoint{1.692024in}{1.712313in}}{\pgfqpoint{1.680974in}{1.712313in}}%
\pgfpathcurveto{\pgfqpoint{1.669924in}{1.712313in}}{\pgfqpoint{1.659325in}{1.707923in}}{\pgfqpoint{1.651511in}{1.700109in}}%
\pgfpathcurveto{\pgfqpoint{1.643698in}{1.692296in}}{\pgfqpoint{1.639307in}{1.681697in}}{\pgfqpoint{1.639307in}{1.670646in}}%
\pgfpathcurveto{\pgfqpoint{1.639307in}{1.659596in}}{\pgfqpoint{1.643698in}{1.648997in}}{\pgfqpoint{1.651511in}{1.641184in}}%
\pgfpathcurveto{\pgfqpoint{1.659325in}{1.633370in}}{\pgfqpoint{1.669924in}{1.628980in}}{\pgfqpoint{1.680974in}{1.628980in}}%
\pgfpathlineto{\pgfqpoint{1.680974in}{1.628980in}}%
\pgfpathclose%
\pgfusepath{stroke}%
\end{pgfscope}%
\begin{pgfscope}%
\pgfpathrectangle{\pgfqpoint{0.393053in}{0.375000in}}{\pgfqpoint{6.356833in}{5.175000in}}%
\pgfusepath{clip}%
\pgfsetbuttcap%
\pgfsetroundjoin%
\pgfsetlinewidth{1.003750pt}%
\definecolor{currentstroke}{rgb}{0.827451,0.827451,0.827451}%
\pgfsetstrokecolor{currentstroke}%
\pgfsetdash{}{0pt}%
\pgfpathmoveto{\pgfqpoint{0.652600in}{3.123417in}}%
\pgfpathcurveto{\pgfqpoint{0.663650in}{3.123417in}}{\pgfqpoint{0.674249in}{3.127808in}}{\pgfqpoint{0.682062in}{3.135621in}}%
\pgfpathcurveto{\pgfqpoint{0.689876in}{3.143435in}}{\pgfqpoint{0.694266in}{3.154034in}}{\pgfqpoint{0.694266in}{3.165084in}}%
\pgfpathcurveto{\pgfqpoint{0.694266in}{3.176134in}}{\pgfqpoint{0.689876in}{3.186733in}}{\pgfqpoint{0.682062in}{3.194547in}}%
\pgfpathcurveto{\pgfqpoint{0.674249in}{3.202360in}}{\pgfqpoint{0.663650in}{3.206751in}}{\pgfqpoint{0.652600in}{3.206751in}}%
\pgfpathcurveto{\pgfqpoint{0.641549in}{3.206751in}}{\pgfqpoint{0.630950in}{3.202360in}}{\pgfqpoint{0.623137in}{3.194547in}}%
\pgfpathcurveto{\pgfqpoint{0.615323in}{3.186733in}}{\pgfqpoint{0.610933in}{3.176134in}}{\pgfqpoint{0.610933in}{3.165084in}}%
\pgfpathcurveto{\pgfqpoint{0.610933in}{3.154034in}}{\pgfqpoint{0.615323in}{3.143435in}}{\pgfqpoint{0.623137in}{3.135621in}}%
\pgfpathcurveto{\pgfqpoint{0.630950in}{3.127808in}}{\pgfqpoint{0.641549in}{3.123417in}}{\pgfqpoint{0.652600in}{3.123417in}}%
\pgfpathlineto{\pgfqpoint{0.652600in}{3.123417in}}%
\pgfpathclose%
\pgfusepath{stroke}%
\end{pgfscope}%
\begin{pgfscope}%
\pgfpathrectangle{\pgfqpoint{0.393053in}{0.375000in}}{\pgfqpoint{6.356833in}{5.175000in}}%
\pgfusepath{clip}%
\pgfsetbuttcap%
\pgfsetroundjoin%
\pgfsetlinewidth{1.003750pt}%
\definecolor{currentstroke}{rgb}{0.827451,0.827451,0.827451}%
\pgfsetstrokecolor{currentstroke}%
\pgfsetdash{}{0pt}%
\pgfpathmoveto{\pgfqpoint{4.229679in}{0.486406in}}%
\pgfpathcurveto{\pgfqpoint{4.240729in}{0.486406in}}{\pgfqpoint{4.251328in}{0.490796in}}{\pgfqpoint{4.259142in}{0.498610in}}%
\pgfpathcurveto{\pgfqpoint{4.266956in}{0.506423in}}{\pgfqpoint{4.271346in}{0.517022in}}{\pgfqpoint{4.271346in}{0.528072in}}%
\pgfpathcurveto{\pgfqpoint{4.271346in}{0.539123in}}{\pgfqpoint{4.266956in}{0.549722in}}{\pgfqpoint{4.259142in}{0.557535in}}%
\pgfpathcurveto{\pgfqpoint{4.251328in}{0.565349in}}{\pgfqpoint{4.240729in}{0.569739in}}{\pgfqpoint{4.229679in}{0.569739in}}%
\pgfpathcurveto{\pgfqpoint{4.218629in}{0.569739in}}{\pgfqpoint{4.208030in}{0.565349in}}{\pgfqpoint{4.200216in}{0.557535in}}%
\pgfpathcurveto{\pgfqpoint{4.192403in}{0.549722in}}{\pgfqpoint{4.188013in}{0.539123in}}{\pgfqpoint{4.188013in}{0.528072in}}%
\pgfpathcurveto{\pgfqpoint{4.188013in}{0.517022in}}{\pgfqpoint{4.192403in}{0.506423in}}{\pgfqpoint{4.200216in}{0.498610in}}%
\pgfpathcurveto{\pgfqpoint{4.208030in}{0.490796in}}{\pgfqpoint{4.218629in}{0.486406in}}{\pgfqpoint{4.229679in}{0.486406in}}%
\pgfpathlineto{\pgfqpoint{4.229679in}{0.486406in}}%
\pgfpathclose%
\pgfusepath{stroke}%
\end{pgfscope}%
\begin{pgfscope}%
\pgfpathrectangle{\pgfqpoint{0.393053in}{0.375000in}}{\pgfqpoint{6.356833in}{5.175000in}}%
\pgfusepath{clip}%
\pgfsetbuttcap%
\pgfsetroundjoin%
\pgfsetlinewidth{1.003750pt}%
\definecolor{currentstroke}{rgb}{0.827451,0.827451,0.827451}%
\pgfsetstrokecolor{currentstroke}%
\pgfsetdash{}{0pt}%
\pgfpathmoveto{\pgfqpoint{2.204630in}{1.411646in}}%
\pgfpathcurveto{\pgfqpoint{2.215680in}{1.411646in}}{\pgfqpoint{2.226279in}{1.416036in}}{\pgfqpoint{2.234093in}{1.423850in}}%
\pgfpathcurveto{\pgfqpoint{2.241907in}{1.431663in}}{\pgfqpoint{2.246297in}{1.442262in}}{\pgfqpoint{2.246297in}{1.453313in}}%
\pgfpathcurveto{\pgfqpoint{2.246297in}{1.464363in}}{\pgfqpoint{2.241907in}{1.474962in}}{\pgfqpoint{2.234093in}{1.482775in}}%
\pgfpathcurveto{\pgfqpoint{2.226279in}{1.490589in}}{\pgfqpoint{2.215680in}{1.494979in}}{\pgfqpoint{2.204630in}{1.494979in}}%
\pgfpathcurveto{\pgfqpoint{2.193580in}{1.494979in}}{\pgfqpoint{2.182981in}{1.490589in}}{\pgfqpoint{2.175167in}{1.482775in}}%
\pgfpathcurveto{\pgfqpoint{2.167354in}{1.474962in}}{\pgfqpoint{2.162963in}{1.464363in}}{\pgfqpoint{2.162963in}{1.453313in}}%
\pgfpathcurveto{\pgfqpoint{2.162963in}{1.442262in}}{\pgfqpoint{2.167354in}{1.431663in}}{\pgfqpoint{2.175167in}{1.423850in}}%
\pgfpathcurveto{\pgfqpoint{2.182981in}{1.416036in}}{\pgfqpoint{2.193580in}{1.411646in}}{\pgfqpoint{2.204630in}{1.411646in}}%
\pgfpathlineto{\pgfqpoint{2.204630in}{1.411646in}}%
\pgfpathclose%
\pgfusepath{stroke}%
\end{pgfscope}%
\begin{pgfscope}%
\pgfpathrectangle{\pgfqpoint{0.393053in}{0.375000in}}{\pgfqpoint{6.356833in}{5.175000in}}%
\pgfusepath{clip}%
\pgfsetbuttcap%
\pgfsetroundjoin%
\pgfsetlinewidth{1.003750pt}%
\definecolor{currentstroke}{rgb}{0.827451,0.827451,0.827451}%
\pgfsetstrokecolor{currentstroke}%
\pgfsetdash{}{0pt}%
\pgfpathmoveto{\pgfqpoint{1.252210in}{2.215695in}}%
\pgfpathcurveto{\pgfqpoint{1.263260in}{2.215695in}}{\pgfqpoint{1.273859in}{2.220085in}}{\pgfqpoint{1.281673in}{2.227899in}}%
\pgfpathcurveto{\pgfqpoint{1.289486in}{2.235712in}}{\pgfqpoint{1.293877in}{2.246311in}}{\pgfqpoint{1.293877in}{2.257361in}}%
\pgfpathcurveto{\pgfqpoint{1.293877in}{2.268412in}}{\pgfqpoint{1.289486in}{2.279011in}}{\pgfqpoint{1.281673in}{2.286824in}}%
\pgfpathcurveto{\pgfqpoint{1.273859in}{2.294638in}}{\pgfqpoint{1.263260in}{2.299028in}}{\pgfqpoint{1.252210in}{2.299028in}}%
\pgfpathcurveto{\pgfqpoint{1.241160in}{2.299028in}}{\pgfqpoint{1.230561in}{2.294638in}}{\pgfqpoint{1.222747in}{2.286824in}}%
\pgfpathcurveto{\pgfqpoint{1.214934in}{2.279011in}}{\pgfqpoint{1.210543in}{2.268412in}}{\pgfqpoint{1.210543in}{2.257361in}}%
\pgfpathcurveto{\pgfqpoint{1.210543in}{2.246311in}}{\pgfqpoint{1.214934in}{2.235712in}}{\pgfqpoint{1.222747in}{2.227899in}}%
\pgfpathcurveto{\pgfqpoint{1.230561in}{2.220085in}}{\pgfqpoint{1.241160in}{2.215695in}}{\pgfqpoint{1.252210in}{2.215695in}}%
\pgfpathlineto{\pgfqpoint{1.252210in}{2.215695in}}%
\pgfpathclose%
\pgfusepath{stroke}%
\end{pgfscope}%
\begin{pgfscope}%
\pgfpathrectangle{\pgfqpoint{0.393053in}{0.375000in}}{\pgfqpoint{6.356833in}{5.175000in}}%
\pgfusepath{clip}%
\pgfsetbuttcap%
\pgfsetroundjoin%
\pgfsetlinewidth{1.003750pt}%
\definecolor{currentstroke}{rgb}{0.827451,0.827451,0.827451}%
\pgfsetstrokecolor{currentstroke}%
\pgfsetdash{}{0pt}%
\pgfpathmoveto{\pgfqpoint{3.448233in}{0.853167in}}%
\pgfpathcurveto{\pgfqpoint{3.459284in}{0.853167in}}{\pgfqpoint{3.469883in}{0.857557in}}{\pgfqpoint{3.477696in}{0.865370in}}%
\pgfpathcurveto{\pgfqpoint{3.485510in}{0.873184in}}{\pgfqpoint{3.489900in}{0.883783in}}{\pgfqpoint{3.489900in}{0.894833in}}%
\pgfpathcurveto{\pgfqpoint{3.489900in}{0.905883in}}{\pgfqpoint{3.485510in}{0.916482in}}{\pgfqpoint{3.477696in}{0.924296in}}%
\pgfpathcurveto{\pgfqpoint{3.469883in}{0.932110in}}{\pgfqpoint{3.459284in}{0.936500in}}{\pgfqpoint{3.448233in}{0.936500in}}%
\pgfpathcurveto{\pgfqpoint{3.437183in}{0.936500in}}{\pgfqpoint{3.426584in}{0.932110in}}{\pgfqpoint{3.418771in}{0.924296in}}%
\pgfpathcurveto{\pgfqpoint{3.410957in}{0.916482in}}{\pgfqpoint{3.406567in}{0.905883in}}{\pgfqpoint{3.406567in}{0.894833in}}%
\pgfpathcurveto{\pgfqpoint{3.406567in}{0.883783in}}{\pgfqpoint{3.410957in}{0.873184in}}{\pgfqpoint{3.418771in}{0.865370in}}%
\pgfpathcurveto{\pgfqpoint{3.426584in}{0.857557in}}{\pgfqpoint{3.437183in}{0.853167in}}{\pgfqpoint{3.448233in}{0.853167in}}%
\pgfpathlineto{\pgfqpoint{3.448233in}{0.853167in}}%
\pgfpathclose%
\pgfusepath{stroke}%
\end{pgfscope}%
\begin{pgfscope}%
\pgfpathrectangle{\pgfqpoint{0.393053in}{0.375000in}}{\pgfqpoint{6.356833in}{5.175000in}}%
\pgfusepath{clip}%
\pgfsetbuttcap%
\pgfsetroundjoin%
\pgfsetlinewidth{1.003750pt}%
\definecolor{currentstroke}{rgb}{0.827451,0.827451,0.827451}%
\pgfsetstrokecolor{currentstroke}%
\pgfsetdash{}{0pt}%
\pgfpathmoveto{\pgfqpoint{3.530781in}{0.847740in}}%
\pgfpathcurveto{\pgfqpoint{3.541831in}{0.847740in}}{\pgfqpoint{3.552430in}{0.852130in}}{\pgfqpoint{3.560244in}{0.859944in}}%
\pgfpathcurveto{\pgfqpoint{3.568057in}{0.867757in}}{\pgfqpoint{3.572448in}{0.878356in}}{\pgfqpoint{3.572448in}{0.889407in}}%
\pgfpathcurveto{\pgfqpoint{3.572448in}{0.900457in}}{\pgfqpoint{3.568057in}{0.911056in}}{\pgfqpoint{3.560244in}{0.918869in}}%
\pgfpathcurveto{\pgfqpoint{3.552430in}{0.926683in}}{\pgfqpoint{3.541831in}{0.931073in}}{\pgfqpoint{3.530781in}{0.931073in}}%
\pgfpathcurveto{\pgfqpoint{3.519731in}{0.931073in}}{\pgfqpoint{3.509132in}{0.926683in}}{\pgfqpoint{3.501318in}{0.918869in}}%
\pgfpathcurveto{\pgfqpoint{3.493504in}{0.911056in}}{\pgfqpoint{3.489114in}{0.900457in}}{\pgfqpoint{3.489114in}{0.889407in}}%
\pgfpathcurveto{\pgfqpoint{3.489114in}{0.878356in}}{\pgfqpoint{3.493504in}{0.867757in}}{\pgfqpoint{3.501318in}{0.859944in}}%
\pgfpathcurveto{\pgfqpoint{3.509132in}{0.852130in}}{\pgfqpoint{3.519731in}{0.847740in}}{\pgfqpoint{3.530781in}{0.847740in}}%
\pgfpathlineto{\pgfqpoint{3.530781in}{0.847740in}}%
\pgfpathclose%
\pgfusepath{stroke}%
\end{pgfscope}%
\begin{pgfscope}%
\pgfpathrectangle{\pgfqpoint{0.393053in}{0.375000in}}{\pgfqpoint{6.356833in}{5.175000in}}%
\pgfusepath{clip}%
\pgfsetbuttcap%
\pgfsetroundjoin%
\pgfsetlinewidth{1.003750pt}%
\definecolor{currentstroke}{rgb}{0.827451,0.827451,0.827451}%
\pgfsetstrokecolor{currentstroke}%
\pgfsetdash{}{0pt}%
\pgfpathmoveto{\pgfqpoint{0.970121in}{2.719437in}}%
\pgfpathcurveto{\pgfqpoint{0.981171in}{2.719437in}}{\pgfqpoint{0.991770in}{2.723827in}}{\pgfqpoint{0.999584in}{2.731641in}}%
\pgfpathcurveto{\pgfqpoint{1.007397in}{2.739454in}}{\pgfqpoint{1.011788in}{2.750053in}}{\pgfqpoint{1.011788in}{2.761103in}}%
\pgfpathcurveto{\pgfqpoint{1.011788in}{2.772154in}}{\pgfqpoint{1.007397in}{2.782753in}}{\pgfqpoint{0.999584in}{2.790566in}}%
\pgfpathcurveto{\pgfqpoint{0.991770in}{2.798380in}}{\pgfqpoint{0.981171in}{2.802770in}}{\pgfqpoint{0.970121in}{2.802770in}}%
\pgfpathcurveto{\pgfqpoint{0.959071in}{2.802770in}}{\pgfqpoint{0.948472in}{2.798380in}}{\pgfqpoint{0.940658in}{2.790566in}}%
\pgfpathcurveto{\pgfqpoint{0.932845in}{2.782753in}}{\pgfqpoint{0.928454in}{2.772154in}}{\pgfqpoint{0.928454in}{2.761103in}}%
\pgfpathcurveto{\pgfqpoint{0.928454in}{2.750053in}}{\pgfqpoint{0.932845in}{2.739454in}}{\pgfqpoint{0.940658in}{2.731641in}}%
\pgfpathcurveto{\pgfqpoint{0.948472in}{2.723827in}}{\pgfqpoint{0.959071in}{2.719437in}}{\pgfqpoint{0.970121in}{2.719437in}}%
\pgfpathlineto{\pgfqpoint{0.970121in}{2.719437in}}%
\pgfpathclose%
\pgfusepath{stroke}%
\end{pgfscope}%
\begin{pgfscope}%
\pgfpathrectangle{\pgfqpoint{0.393053in}{0.375000in}}{\pgfqpoint{6.356833in}{5.175000in}}%
\pgfusepath{clip}%
\pgfsetbuttcap%
\pgfsetroundjoin%
\pgfsetlinewidth{1.003750pt}%
\definecolor{currentstroke}{rgb}{0.827451,0.827451,0.827451}%
\pgfsetstrokecolor{currentstroke}%
\pgfsetdash{}{0pt}%
\pgfpathmoveto{\pgfqpoint{2.975443in}{0.952053in}}%
\pgfpathcurveto{\pgfqpoint{2.986493in}{0.952053in}}{\pgfqpoint{2.997092in}{0.956443in}}{\pgfqpoint{3.004906in}{0.964257in}}%
\pgfpathcurveto{\pgfqpoint{3.012719in}{0.972071in}}{\pgfqpoint{3.017110in}{0.982670in}}{\pgfqpoint{3.017110in}{0.993720in}}%
\pgfpathcurveto{\pgfqpoint{3.017110in}{1.004770in}}{\pgfqpoint{3.012719in}{1.015369in}}{\pgfqpoint{3.004906in}{1.023183in}}%
\pgfpathcurveto{\pgfqpoint{2.997092in}{1.030996in}}{\pgfqpoint{2.986493in}{1.035386in}}{\pgfqpoint{2.975443in}{1.035386in}}%
\pgfpathcurveto{\pgfqpoint{2.964393in}{1.035386in}}{\pgfqpoint{2.953794in}{1.030996in}}{\pgfqpoint{2.945980in}{1.023183in}}%
\pgfpathcurveto{\pgfqpoint{2.938166in}{1.015369in}}{\pgfqpoint{2.933776in}{1.004770in}}{\pgfqpoint{2.933776in}{0.993720in}}%
\pgfpathcurveto{\pgfqpoint{2.933776in}{0.982670in}}{\pgfqpoint{2.938166in}{0.972071in}}{\pgfqpoint{2.945980in}{0.964257in}}%
\pgfpathcurveto{\pgfqpoint{2.953794in}{0.956443in}}{\pgfqpoint{2.964393in}{0.952053in}}{\pgfqpoint{2.975443in}{0.952053in}}%
\pgfpathlineto{\pgfqpoint{2.975443in}{0.952053in}}%
\pgfpathclose%
\pgfusepath{stroke}%
\end{pgfscope}%
\begin{pgfscope}%
\pgfpathrectangle{\pgfqpoint{0.393053in}{0.375000in}}{\pgfqpoint{6.356833in}{5.175000in}}%
\pgfusepath{clip}%
\pgfsetbuttcap%
\pgfsetroundjoin%
\pgfsetlinewidth{1.003750pt}%
\definecolor{currentstroke}{rgb}{0.827451,0.827451,0.827451}%
\pgfsetstrokecolor{currentstroke}%
\pgfsetdash{}{0pt}%
\pgfpathmoveto{\pgfqpoint{3.073650in}{0.935575in}}%
\pgfpathcurveto{\pgfqpoint{3.084700in}{0.935575in}}{\pgfqpoint{3.095299in}{0.939965in}}{\pgfqpoint{3.103113in}{0.947779in}}%
\pgfpathcurveto{\pgfqpoint{3.110927in}{0.955592in}}{\pgfqpoint{3.115317in}{0.966191in}}{\pgfqpoint{3.115317in}{0.977242in}}%
\pgfpathcurveto{\pgfqpoint{3.115317in}{0.988292in}}{\pgfqpoint{3.110927in}{0.998891in}}{\pgfqpoint{3.103113in}{1.006704in}}%
\pgfpathcurveto{\pgfqpoint{3.095299in}{1.014518in}}{\pgfqpoint{3.084700in}{1.018908in}}{\pgfqpoint{3.073650in}{1.018908in}}%
\pgfpathcurveto{\pgfqpoint{3.062600in}{1.018908in}}{\pgfqpoint{3.052001in}{1.014518in}}{\pgfqpoint{3.044187in}{1.006704in}}%
\pgfpathcurveto{\pgfqpoint{3.036374in}{0.998891in}}{\pgfqpoint{3.031984in}{0.988292in}}{\pgfqpoint{3.031984in}{0.977242in}}%
\pgfpathcurveto{\pgfqpoint{3.031984in}{0.966191in}}{\pgfqpoint{3.036374in}{0.955592in}}{\pgfqpoint{3.044187in}{0.947779in}}%
\pgfpathcurveto{\pgfqpoint{3.052001in}{0.939965in}}{\pgfqpoint{3.062600in}{0.935575in}}{\pgfqpoint{3.073650in}{0.935575in}}%
\pgfpathlineto{\pgfqpoint{3.073650in}{0.935575in}}%
\pgfpathclose%
\pgfusepath{stroke}%
\end{pgfscope}%
\begin{pgfscope}%
\pgfpathrectangle{\pgfqpoint{0.393053in}{0.375000in}}{\pgfqpoint{6.356833in}{5.175000in}}%
\pgfusepath{clip}%
\pgfsetbuttcap%
\pgfsetroundjoin%
\pgfsetlinewidth{1.003750pt}%
\definecolor{currentstroke}{rgb}{0.827451,0.827451,0.827451}%
\pgfsetstrokecolor{currentstroke}%
\pgfsetdash{}{0pt}%
\pgfpathmoveto{\pgfqpoint{2.578528in}{1.236168in}}%
\pgfpathcurveto{\pgfqpoint{2.589578in}{1.236168in}}{\pgfqpoint{2.600177in}{1.240558in}}{\pgfqpoint{2.607991in}{1.248372in}}%
\pgfpathcurveto{\pgfqpoint{2.615805in}{1.256185in}}{\pgfqpoint{2.620195in}{1.266784in}}{\pgfqpoint{2.620195in}{1.277835in}}%
\pgfpathcurveto{\pgfqpoint{2.620195in}{1.288885in}}{\pgfqpoint{2.615805in}{1.299484in}}{\pgfqpoint{2.607991in}{1.307297in}}%
\pgfpathcurveto{\pgfqpoint{2.600177in}{1.315111in}}{\pgfqpoint{2.589578in}{1.319501in}}{\pgfqpoint{2.578528in}{1.319501in}}%
\pgfpathcurveto{\pgfqpoint{2.567478in}{1.319501in}}{\pgfqpoint{2.556879in}{1.315111in}}{\pgfqpoint{2.549065in}{1.307297in}}%
\pgfpathcurveto{\pgfqpoint{2.541252in}{1.299484in}}{\pgfqpoint{2.536861in}{1.288885in}}{\pgfqpoint{2.536861in}{1.277835in}}%
\pgfpathcurveto{\pgfqpoint{2.536861in}{1.266784in}}{\pgfqpoint{2.541252in}{1.256185in}}{\pgfqpoint{2.549065in}{1.248372in}}%
\pgfpathcurveto{\pgfqpoint{2.556879in}{1.240558in}}{\pgfqpoint{2.567478in}{1.236168in}}{\pgfqpoint{2.578528in}{1.236168in}}%
\pgfpathlineto{\pgfqpoint{2.578528in}{1.236168in}}%
\pgfpathclose%
\pgfusepath{stroke}%
\end{pgfscope}%
\begin{pgfscope}%
\pgfpathrectangle{\pgfqpoint{0.393053in}{0.375000in}}{\pgfqpoint{6.356833in}{5.175000in}}%
\pgfusepath{clip}%
\pgfsetbuttcap%
\pgfsetroundjoin%
\pgfsetlinewidth{1.003750pt}%
\definecolor{currentstroke}{rgb}{0.827451,0.827451,0.827451}%
\pgfsetstrokecolor{currentstroke}%
\pgfsetdash{}{0pt}%
\pgfpathmoveto{\pgfqpoint{3.662350in}{0.729635in}}%
\pgfpathcurveto{\pgfqpoint{3.673400in}{0.729635in}}{\pgfqpoint{3.683999in}{0.734025in}}{\pgfqpoint{3.691813in}{0.741839in}}%
\pgfpathcurveto{\pgfqpoint{3.699626in}{0.749652in}}{\pgfqpoint{3.704016in}{0.760251in}}{\pgfqpoint{3.704016in}{0.771301in}}%
\pgfpathcurveto{\pgfqpoint{3.704016in}{0.782352in}}{\pgfqpoint{3.699626in}{0.792951in}}{\pgfqpoint{3.691813in}{0.800764in}}%
\pgfpathcurveto{\pgfqpoint{3.683999in}{0.808578in}}{\pgfqpoint{3.673400in}{0.812968in}}{\pgfqpoint{3.662350in}{0.812968in}}%
\pgfpathcurveto{\pgfqpoint{3.651300in}{0.812968in}}{\pgfqpoint{3.640701in}{0.808578in}}{\pgfqpoint{3.632887in}{0.800764in}}%
\pgfpathcurveto{\pgfqpoint{3.625073in}{0.792951in}}{\pgfqpoint{3.620683in}{0.782352in}}{\pgfqpoint{3.620683in}{0.771301in}}%
\pgfpathcurveto{\pgfqpoint{3.620683in}{0.760251in}}{\pgfqpoint{3.625073in}{0.749652in}}{\pgfqpoint{3.632887in}{0.741839in}}%
\pgfpathcurveto{\pgfqpoint{3.640701in}{0.734025in}}{\pgfqpoint{3.651300in}{0.729635in}}{\pgfqpoint{3.662350in}{0.729635in}}%
\pgfpathlineto{\pgfqpoint{3.662350in}{0.729635in}}%
\pgfpathclose%
\pgfusepath{stroke}%
\end{pgfscope}%
\begin{pgfscope}%
\pgfpathrectangle{\pgfqpoint{0.393053in}{0.375000in}}{\pgfqpoint{6.356833in}{5.175000in}}%
\pgfusepath{clip}%
\pgfsetbuttcap%
\pgfsetroundjoin%
\pgfsetlinewidth{1.003750pt}%
\definecolor{currentstroke}{rgb}{0.827451,0.827451,0.827451}%
\pgfsetstrokecolor{currentstroke}%
\pgfsetdash{}{0pt}%
\pgfpathmoveto{\pgfqpoint{1.260728in}{2.046542in}}%
\pgfpathcurveto{\pgfqpoint{1.271778in}{2.046542in}}{\pgfqpoint{1.282377in}{2.050933in}}{\pgfqpoint{1.290191in}{2.058746in}}%
\pgfpathcurveto{\pgfqpoint{1.298005in}{2.066560in}}{\pgfqpoint{1.302395in}{2.077159in}}{\pgfqpoint{1.302395in}{2.088209in}}%
\pgfpathcurveto{\pgfqpoint{1.302395in}{2.099259in}}{\pgfqpoint{1.298005in}{2.109858in}}{\pgfqpoint{1.290191in}{2.117672in}}%
\pgfpathcurveto{\pgfqpoint{1.282377in}{2.125486in}}{\pgfqpoint{1.271778in}{2.129876in}}{\pgfqpoint{1.260728in}{2.129876in}}%
\pgfpathcurveto{\pgfqpoint{1.249678in}{2.129876in}}{\pgfqpoint{1.239079in}{2.125486in}}{\pgfqpoint{1.231265in}{2.117672in}}%
\pgfpathcurveto{\pgfqpoint{1.223452in}{2.109858in}}{\pgfqpoint{1.219061in}{2.099259in}}{\pgfqpoint{1.219061in}{2.088209in}}%
\pgfpathcurveto{\pgfqpoint{1.219061in}{2.077159in}}{\pgfqpoint{1.223452in}{2.066560in}}{\pgfqpoint{1.231265in}{2.058746in}}%
\pgfpathcurveto{\pgfqpoint{1.239079in}{2.050933in}}{\pgfqpoint{1.249678in}{2.046542in}}{\pgfqpoint{1.260728in}{2.046542in}}%
\pgfpathlineto{\pgfqpoint{1.260728in}{2.046542in}}%
\pgfpathclose%
\pgfusepath{stroke}%
\end{pgfscope}%
\begin{pgfscope}%
\pgfpathrectangle{\pgfqpoint{0.393053in}{0.375000in}}{\pgfqpoint{6.356833in}{5.175000in}}%
\pgfusepath{clip}%
\pgfsetbuttcap%
\pgfsetroundjoin%
\pgfsetlinewidth{1.003750pt}%
\definecolor{currentstroke}{rgb}{0.827451,0.827451,0.827451}%
\pgfsetstrokecolor{currentstroke}%
\pgfsetdash{}{0pt}%
\pgfpathmoveto{\pgfqpoint{1.274887in}{2.046521in}}%
\pgfpathcurveto{\pgfqpoint{1.285937in}{2.046521in}}{\pgfqpoint{1.296536in}{2.050911in}}{\pgfqpoint{1.304350in}{2.058725in}}%
\pgfpathcurveto{\pgfqpoint{1.312164in}{2.066538in}}{\pgfqpoint{1.316554in}{2.077137in}}{\pgfqpoint{1.316554in}{2.088187in}}%
\pgfpathcurveto{\pgfqpoint{1.316554in}{2.099237in}}{\pgfqpoint{1.312164in}{2.109836in}}{\pgfqpoint{1.304350in}{2.117650in}}%
\pgfpathcurveto{\pgfqpoint{1.296536in}{2.125464in}}{\pgfqpoint{1.285937in}{2.129854in}}{\pgfqpoint{1.274887in}{2.129854in}}%
\pgfpathcurveto{\pgfqpoint{1.263837in}{2.129854in}}{\pgfqpoint{1.253238in}{2.125464in}}{\pgfqpoint{1.245425in}{2.117650in}}%
\pgfpathcurveto{\pgfqpoint{1.237611in}{2.109836in}}{\pgfqpoint{1.233221in}{2.099237in}}{\pgfqpoint{1.233221in}{2.088187in}}%
\pgfpathcurveto{\pgfqpoint{1.233221in}{2.077137in}}{\pgfqpoint{1.237611in}{2.066538in}}{\pgfqpoint{1.245425in}{2.058725in}}%
\pgfpathcurveto{\pgfqpoint{1.253238in}{2.050911in}}{\pgfqpoint{1.263837in}{2.046521in}}{\pgfqpoint{1.274887in}{2.046521in}}%
\pgfpathlineto{\pgfqpoint{1.274887in}{2.046521in}}%
\pgfpathclose%
\pgfusepath{stroke}%
\end{pgfscope}%
\begin{pgfscope}%
\pgfpathrectangle{\pgfqpoint{0.393053in}{0.375000in}}{\pgfqpoint{6.356833in}{5.175000in}}%
\pgfusepath{clip}%
\pgfsetbuttcap%
\pgfsetroundjoin%
\pgfsetlinewidth{1.003750pt}%
\definecolor{currentstroke}{rgb}{0.827451,0.827451,0.827451}%
\pgfsetstrokecolor{currentstroke}%
\pgfsetdash{}{0pt}%
\pgfpathmoveto{\pgfqpoint{5.128133in}{0.470706in}}%
\pgfpathcurveto{\pgfqpoint{5.139183in}{0.470706in}}{\pgfqpoint{5.149782in}{0.475096in}}{\pgfqpoint{5.157596in}{0.482910in}}%
\pgfpathcurveto{\pgfqpoint{5.165409in}{0.490723in}}{\pgfqpoint{5.169800in}{0.501322in}}{\pgfqpoint{5.169800in}{0.512372in}}%
\pgfpathcurveto{\pgfqpoint{5.169800in}{0.523422in}}{\pgfqpoint{5.165409in}{0.534021in}}{\pgfqpoint{5.157596in}{0.541835in}}%
\pgfpathcurveto{\pgfqpoint{5.149782in}{0.549649in}}{\pgfqpoint{5.139183in}{0.554039in}}{\pgfqpoint{5.128133in}{0.554039in}}%
\pgfpathcurveto{\pgfqpoint{5.117083in}{0.554039in}}{\pgfqpoint{5.106484in}{0.549649in}}{\pgfqpoint{5.098670in}{0.541835in}}%
\pgfpathcurveto{\pgfqpoint{5.090857in}{0.534021in}}{\pgfqpoint{5.086466in}{0.523422in}}{\pgfqpoint{5.086466in}{0.512372in}}%
\pgfpathcurveto{\pgfqpoint{5.086466in}{0.501322in}}{\pgfqpoint{5.090857in}{0.490723in}}{\pgfqpoint{5.098670in}{0.482910in}}%
\pgfpathcurveto{\pgfqpoint{5.106484in}{0.475096in}}{\pgfqpoint{5.117083in}{0.470706in}}{\pgfqpoint{5.128133in}{0.470706in}}%
\pgfpathlineto{\pgfqpoint{5.128133in}{0.470706in}}%
\pgfpathclose%
\pgfusepath{stroke}%
\end{pgfscope}%
\begin{pgfscope}%
\pgfpathrectangle{\pgfqpoint{0.393053in}{0.375000in}}{\pgfqpoint{6.356833in}{5.175000in}}%
\pgfusepath{clip}%
\pgfsetbuttcap%
\pgfsetroundjoin%
\pgfsetlinewidth{1.003750pt}%
\definecolor{currentstroke}{rgb}{0.827451,0.827451,0.827451}%
\pgfsetstrokecolor{currentstroke}%
\pgfsetdash{}{0pt}%
\pgfpathmoveto{\pgfqpoint{1.893927in}{1.548232in}}%
\pgfpathcurveto{\pgfqpoint{1.904977in}{1.548232in}}{\pgfqpoint{1.915576in}{1.552622in}}{\pgfqpoint{1.923390in}{1.560435in}}%
\pgfpathcurveto{\pgfqpoint{1.931203in}{1.568249in}}{\pgfqpoint{1.935594in}{1.578848in}}{\pgfqpoint{1.935594in}{1.589898in}}%
\pgfpathcurveto{\pgfqpoint{1.935594in}{1.600948in}}{\pgfqpoint{1.931203in}{1.611547in}}{\pgfqpoint{1.923390in}{1.619361in}}%
\pgfpathcurveto{\pgfqpoint{1.915576in}{1.627175in}}{\pgfqpoint{1.904977in}{1.631565in}}{\pgfqpoint{1.893927in}{1.631565in}}%
\pgfpathcurveto{\pgfqpoint{1.882877in}{1.631565in}}{\pgfqpoint{1.872278in}{1.627175in}}{\pgfqpoint{1.864464in}{1.619361in}}%
\pgfpathcurveto{\pgfqpoint{1.856650in}{1.611547in}}{\pgfqpoint{1.852260in}{1.600948in}}{\pgfqpoint{1.852260in}{1.589898in}}%
\pgfpathcurveto{\pgfqpoint{1.852260in}{1.578848in}}{\pgfqpoint{1.856650in}{1.568249in}}{\pgfqpoint{1.864464in}{1.560435in}}%
\pgfpathcurveto{\pgfqpoint{1.872278in}{1.552622in}}{\pgfqpoint{1.882877in}{1.548232in}}{\pgfqpoint{1.893927in}{1.548232in}}%
\pgfpathlineto{\pgfqpoint{1.893927in}{1.548232in}}%
\pgfpathclose%
\pgfusepath{stroke}%
\end{pgfscope}%
\begin{pgfscope}%
\pgfpathrectangle{\pgfqpoint{0.393053in}{0.375000in}}{\pgfqpoint{6.356833in}{5.175000in}}%
\pgfusepath{clip}%
\pgfsetbuttcap%
\pgfsetroundjoin%
\pgfsetlinewidth{1.003750pt}%
\definecolor{currentstroke}{rgb}{0.827451,0.827451,0.827451}%
\pgfsetstrokecolor{currentstroke}%
\pgfsetdash{}{0pt}%
\pgfpathmoveto{\pgfqpoint{1.760243in}{1.825127in}}%
\pgfpathcurveto{\pgfqpoint{1.771293in}{1.825127in}}{\pgfqpoint{1.781892in}{1.829517in}}{\pgfqpoint{1.789706in}{1.837331in}}%
\pgfpathcurveto{\pgfqpoint{1.797520in}{1.845144in}}{\pgfqpoint{1.801910in}{1.855743in}}{\pgfqpoint{1.801910in}{1.866793in}}%
\pgfpathcurveto{\pgfqpoint{1.801910in}{1.877843in}}{\pgfqpoint{1.797520in}{1.888442in}}{\pgfqpoint{1.789706in}{1.896256in}}%
\pgfpathcurveto{\pgfqpoint{1.781892in}{1.904070in}}{\pgfqpoint{1.771293in}{1.908460in}}{\pgfqpoint{1.760243in}{1.908460in}}%
\pgfpathcurveto{\pgfqpoint{1.749193in}{1.908460in}}{\pgfqpoint{1.738594in}{1.904070in}}{\pgfqpoint{1.730781in}{1.896256in}}%
\pgfpathcurveto{\pgfqpoint{1.722967in}{1.888442in}}{\pgfqpoint{1.718577in}{1.877843in}}{\pgfqpoint{1.718577in}{1.866793in}}%
\pgfpathcurveto{\pgfqpoint{1.718577in}{1.855743in}}{\pgfqpoint{1.722967in}{1.845144in}}{\pgfqpoint{1.730781in}{1.837331in}}%
\pgfpathcurveto{\pgfqpoint{1.738594in}{1.829517in}}{\pgfqpoint{1.749193in}{1.825127in}}{\pgfqpoint{1.760243in}{1.825127in}}%
\pgfpathlineto{\pgfqpoint{1.760243in}{1.825127in}}%
\pgfpathclose%
\pgfusepath{stroke}%
\end{pgfscope}%
\begin{pgfscope}%
\pgfpathrectangle{\pgfqpoint{0.393053in}{0.375000in}}{\pgfqpoint{6.356833in}{5.175000in}}%
\pgfusepath{clip}%
\pgfsetbuttcap%
\pgfsetroundjoin%
\pgfsetlinewidth{1.003750pt}%
\definecolor{currentstroke}{rgb}{0.827451,0.827451,0.827451}%
\pgfsetstrokecolor{currentstroke}%
\pgfsetdash{}{0pt}%
\pgfpathmoveto{\pgfqpoint{1.712338in}{2.041709in}}%
\pgfpathcurveto{\pgfqpoint{1.723388in}{2.041709in}}{\pgfqpoint{1.733987in}{2.046100in}}{\pgfqpoint{1.741801in}{2.053913in}}%
\pgfpathcurveto{\pgfqpoint{1.749614in}{2.061727in}}{\pgfqpoint{1.754004in}{2.072326in}}{\pgfqpoint{1.754004in}{2.083376in}}%
\pgfpathcurveto{\pgfqpoint{1.754004in}{2.094426in}}{\pgfqpoint{1.749614in}{2.105025in}}{\pgfqpoint{1.741801in}{2.112839in}}%
\pgfpathcurveto{\pgfqpoint{1.733987in}{2.120652in}}{\pgfqpoint{1.723388in}{2.125043in}}{\pgfqpoint{1.712338in}{2.125043in}}%
\pgfpathcurveto{\pgfqpoint{1.701288in}{2.125043in}}{\pgfqpoint{1.690689in}{2.120652in}}{\pgfqpoint{1.682875in}{2.112839in}}%
\pgfpathcurveto{\pgfqpoint{1.675061in}{2.105025in}}{\pgfqpoint{1.670671in}{2.094426in}}{\pgfqpoint{1.670671in}{2.083376in}}%
\pgfpathcurveto{\pgfqpoint{1.670671in}{2.072326in}}{\pgfqpoint{1.675061in}{2.061727in}}{\pgfqpoint{1.682875in}{2.053913in}}%
\pgfpathcurveto{\pgfqpoint{1.690689in}{2.046100in}}{\pgfqpoint{1.701288in}{2.041709in}}{\pgfqpoint{1.712338in}{2.041709in}}%
\pgfpathlineto{\pgfqpoint{1.712338in}{2.041709in}}%
\pgfpathclose%
\pgfusepath{stroke}%
\end{pgfscope}%
\begin{pgfscope}%
\pgfpathrectangle{\pgfqpoint{0.393053in}{0.375000in}}{\pgfqpoint{6.356833in}{5.175000in}}%
\pgfusepath{clip}%
\pgfsetbuttcap%
\pgfsetroundjoin%
\pgfsetlinewidth{1.003750pt}%
\definecolor{currentstroke}{rgb}{0.827451,0.827451,0.827451}%
\pgfsetstrokecolor{currentstroke}%
\pgfsetdash{}{0pt}%
\pgfpathmoveto{\pgfqpoint{1.864918in}{1.670322in}}%
\pgfpathcurveto{\pgfqpoint{1.875968in}{1.670322in}}{\pgfqpoint{1.886567in}{1.674712in}}{\pgfqpoint{1.894381in}{1.682526in}}%
\pgfpathcurveto{\pgfqpoint{1.902195in}{1.690339in}}{\pgfqpoint{1.906585in}{1.700938in}}{\pgfqpoint{1.906585in}{1.711988in}}%
\pgfpathcurveto{\pgfqpoint{1.906585in}{1.723038in}}{\pgfqpoint{1.902195in}{1.733637in}}{\pgfqpoint{1.894381in}{1.741451in}}%
\pgfpathcurveto{\pgfqpoint{1.886567in}{1.749265in}}{\pgfqpoint{1.875968in}{1.753655in}}{\pgfqpoint{1.864918in}{1.753655in}}%
\pgfpathcurveto{\pgfqpoint{1.853868in}{1.753655in}}{\pgfqpoint{1.843269in}{1.749265in}}{\pgfqpoint{1.835456in}{1.741451in}}%
\pgfpathcurveto{\pgfqpoint{1.827642in}{1.733637in}}{\pgfqpoint{1.823252in}{1.723038in}}{\pgfqpoint{1.823252in}{1.711988in}}%
\pgfpathcurveto{\pgfqpoint{1.823252in}{1.700938in}}{\pgfqpoint{1.827642in}{1.690339in}}{\pgfqpoint{1.835456in}{1.682526in}}%
\pgfpathcurveto{\pgfqpoint{1.843269in}{1.674712in}}{\pgfqpoint{1.853868in}{1.670322in}}{\pgfqpoint{1.864918in}{1.670322in}}%
\pgfpathlineto{\pgfqpoint{1.864918in}{1.670322in}}%
\pgfpathclose%
\pgfusepath{stroke}%
\end{pgfscope}%
\begin{pgfscope}%
\pgfpathrectangle{\pgfqpoint{0.393053in}{0.375000in}}{\pgfqpoint{6.356833in}{5.175000in}}%
\pgfusepath{clip}%
\pgfsetbuttcap%
\pgfsetroundjoin%
\pgfsetlinewidth{1.003750pt}%
\definecolor{currentstroke}{rgb}{0.827451,0.827451,0.827451}%
\pgfsetstrokecolor{currentstroke}%
\pgfsetdash{}{0pt}%
\pgfpathmoveto{\pgfqpoint{0.669503in}{3.263449in}}%
\pgfpathcurveto{\pgfqpoint{0.680554in}{3.263449in}}{\pgfqpoint{0.691153in}{3.267840in}}{\pgfqpoint{0.698966in}{3.275653in}}%
\pgfpathcurveto{\pgfqpoint{0.706780in}{3.283467in}}{\pgfqpoint{0.711170in}{3.294066in}}{\pgfqpoint{0.711170in}{3.305116in}}%
\pgfpathcurveto{\pgfqpoint{0.711170in}{3.316166in}}{\pgfqpoint{0.706780in}{3.326765in}}{\pgfqpoint{0.698966in}{3.334579in}}%
\pgfpathcurveto{\pgfqpoint{0.691153in}{3.342392in}}{\pgfqpoint{0.680554in}{3.346783in}}{\pgfqpoint{0.669503in}{3.346783in}}%
\pgfpathcurveto{\pgfqpoint{0.658453in}{3.346783in}}{\pgfqpoint{0.647854in}{3.342392in}}{\pgfqpoint{0.640041in}{3.334579in}}%
\pgfpathcurveto{\pgfqpoint{0.632227in}{3.326765in}}{\pgfqpoint{0.627837in}{3.316166in}}{\pgfqpoint{0.627837in}{3.305116in}}%
\pgfpathcurveto{\pgfqpoint{0.627837in}{3.294066in}}{\pgfqpoint{0.632227in}{3.283467in}}{\pgfqpoint{0.640041in}{3.275653in}}%
\pgfpathcurveto{\pgfqpoint{0.647854in}{3.267840in}}{\pgfqpoint{0.658453in}{3.263449in}}{\pgfqpoint{0.669503in}{3.263449in}}%
\pgfpathlineto{\pgfqpoint{0.669503in}{3.263449in}}%
\pgfpathclose%
\pgfusepath{stroke}%
\end{pgfscope}%
\begin{pgfscope}%
\pgfpathrectangle{\pgfqpoint{0.393053in}{0.375000in}}{\pgfqpoint{6.356833in}{5.175000in}}%
\pgfusepath{clip}%
\pgfsetbuttcap%
\pgfsetroundjoin%
\pgfsetlinewidth{1.003750pt}%
\definecolor{currentstroke}{rgb}{0.827451,0.827451,0.827451}%
\pgfsetstrokecolor{currentstroke}%
\pgfsetdash{}{0pt}%
\pgfpathmoveto{\pgfqpoint{4.942541in}{0.603946in}}%
\pgfpathcurveto{\pgfqpoint{4.953591in}{0.603946in}}{\pgfqpoint{4.964190in}{0.608336in}}{\pgfqpoint{4.972003in}{0.616150in}}%
\pgfpathcurveto{\pgfqpoint{4.979817in}{0.623963in}}{\pgfqpoint{4.984207in}{0.634563in}}{\pgfqpoint{4.984207in}{0.645613in}}%
\pgfpathcurveto{\pgfqpoint{4.984207in}{0.656663in}}{\pgfqpoint{4.979817in}{0.667262in}}{\pgfqpoint{4.972003in}{0.675075in}}%
\pgfpathcurveto{\pgfqpoint{4.964190in}{0.682889in}}{\pgfqpoint{4.953591in}{0.687279in}}{\pgfqpoint{4.942541in}{0.687279in}}%
\pgfpathcurveto{\pgfqpoint{4.931490in}{0.687279in}}{\pgfqpoint{4.920891in}{0.682889in}}{\pgfqpoint{4.913078in}{0.675075in}}%
\pgfpathcurveto{\pgfqpoint{4.905264in}{0.667262in}}{\pgfqpoint{4.900874in}{0.656663in}}{\pgfqpoint{4.900874in}{0.645613in}}%
\pgfpathcurveto{\pgfqpoint{4.900874in}{0.634563in}}{\pgfqpoint{4.905264in}{0.623963in}}{\pgfqpoint{4.913078in}{0.616150in}}%
\pgfpathcurveto{\pgfqpoint{4.920891in}{0.608336in}}{\pgfqpoint{4.931490in}{0.603946in}}{\pgfqpoint{4.942541in}{0.603946in}}%
\pgfpathlineto{\pgfqpoint{4.942541in}{0.603946in}}%
\pgfpathclose%
\pgfusepath{stroke}%
\end{pgfscope}%
\begin{pgfscope}%
\pgfpathrectangle{\pgfqpoint{0.393053in}{0.375000in}}{\pgfqpoint{6.356833in}{5.175000in}}%
\pgfusepath{clip}%
\pgfsetbuttcap%
\pgfsetroundjoin%
\pgfsetlinewidth{1.003750pt}%
\definecolor{currentstroke}{rgb}{0.827451,0.827451,0.827451}%
\pgfsetstrokecolor{currentstroke}%
\pgfsetdash{}{0pt}%
\pgfpathmoveto{\pgfqpoint{2.263432in}{1.528542in}}%
\pgfpathcurveto{\pgfqpoint{2.274482in}{1.528542in}}{\pgfqpoint{2.285081in}{1.532932in}}{\pgfqpoint{2.292895in}{1.540746in}}%
\pgfpathcurveto{\pgfqpoint{2.300708in}{1.548560in}}{\pgfqpoint{2.305098in}{1.559159in}}{\pgfqpoint{2.305098in}{1.570209in}}%
\pgfpathcurveto{\pgfqpoint{2.305098in}{1.581259in}}{\pgfqpoint{2.300708in}{1.591858in}}{\pgfqpoint{2.292895in}{1.599672in}}%
\pgfpathcurveto{\pgfqpoint{2.285081in}{1.607485in}}{\pgfqpoint{2.274482in}{1.611875in}}{\pgfqpoint{2.263432in}{1.611875in}}%
\pgfpathcurveto{\pgfqpoint{2.252382in}{1.611875in}}{\pgfqpoint{2.241783in}{1.607485in}}{\pgfqpoint{2.233969in}{1.599672in}}%
\pgfpathcurveto{\pgfqpoint{2.226155in}{1.591858in}}{\pgfqpoint{2.221765in}{1.581259in}}{\pgfqpoint{2.221765in}{1.570209in}}%
\pgfpathcurveto{\pgfqpoint{2.221765in}{1.559159in}}{\pgfqpoint{2.226155in}{1.548560in}}{\pgfqpoint{2.233969in}{1.540746in}}%
\pgfpathcurveto{\pgfqpoint{2.241783in}{1.532932in}}{\pgfqpoint{2.252382in}{1.528542in}}{\pgfqpoint{2.263432in}{1.528542in}}%
\pgfpathlineto{\pgfqpoint{2.263432in}{1.528542in}}%
\pgfpathclose%
\pgfusepath{stroke}%
\end{pgfscope}%
\begin{pgfscope}%
\pgfpathrectangle{\pgfqpoint{0.393053in}{0.375000in}}{\pgfqpoint{6.356833in}{5.175000in}}%
\pgfusepath{clip}%
\pgfsetbuttcap%
\pgfsetroundjoin%
\pgfsetlinewidth{1.003750pt}%
\definecolor{currentstroke}{rgb}{0.827451,0.827451,0.827451}%
\pgfsetstrokecolor{currentstroke}%
\pgfsetdash{}{0pt}%
\pgfpathmoveto{\pgfqpoint{2.282585in}{1.436760in}}%
\pgfpathcurveto{\pgfqpoint{2.293635in}{1.436760in}}{\pgfqpoint{2.304234in}{1.441150in}}{\pgfqpoint{2.312048in}{1.448963in}}%
\pgfpathcurveto{\pgfqpoint{2.319861in}{1.456777in}}{\pgfqpoint{2.324252in}{1.467376in}}{\pgfqpoint{2.324252in}{1.478426in}}%
\pgfpathcurveto{\pgfqpoint{2.324252in}{1.489476in}}{\pgfqpoint{2.319861in}{1.500075in}}{\pgfqpoint{2.312048in}{1.507889in}}%
\pgfpathcurveto{\pgfqpoint{2.304234in}{1.515703in}}{\pgfqpoint{2.293635in}{1.520093in}}{\pgfqpoint{2.282585in}{1.520093in}}%
\pgfpathcurveto{\pgfqpoint{2.271535in}{1.520093in}}{\pgfqpoint{2.260936in}{1.515703in}}{\pgfqpoint{2.253122in}{1.507889in}}%
\pgfpathcurveto{\pgfqpoint{2.245309in}{1.500075in}}{\pgfqpoint{2.240918in}{1.489476in}}{\pgfqpoint{2.240918in}{1.478426in}}%
\pgfpathcurveto{\pgfqpoint{2.240918in}{1.467376in}}{\pgfqpoint{2.245309in}{1.456777in}}{\pgfqpoint{2.253122in}{1.448963in}}%
\pgfpathcurveto{\pgfqpoint{2.260936in}{1.441150in}}{\pgfqpoint{2.271535in}{1.436760in}}{\pgfqpoint{2.282585in}{1.436760in}}%
\pgfpathlineto{\pgfqpoint{2.282585in}{1.436760in}}%
\pgfpathclose%
\pgfusepath{stroke}%
\end{pgfscope}%
\begin{pgfscope}%
\pgfpathrectangle{\pgfqpoint{0.393053in}{0.375000in}}{\pgfqpoint{6.356833in}{5.175000in}}%
\pgfusepath{clip}%
\pgfsetbuttcap%
\pgfsetroundjoin%
\pgfsetlinewidth{1.003750pt}%
\definecolor{currentstroke}{rgb}{0.827451,0.827451,0.827451}%
\pgfsetstrokecolor{currentstroke}%
\pgfsetdash{}{0pt}%
\pgfpathmoveto{\pgfqpoint{0.977927in}{3.008457in}}%
\pgfpathcurveto{\pgfqpoint{0.988977in}{3.008457in}}{\pgfqpoint{0.999576in}{3.012848in}}{\pgfqpoint{1.007390in}{3.020661in}}%
\pgfpathcurveto{\pgfqpoint{1.015204in}{3.028475in}}{\pgfqpoint{1.019594in}{3.039074in}}{\pgfqpoint{1.019594in}{3.050124in}}%
\pgfpathcurveto{\pgfqpoint{1.019594in}{3.061174in}}{\pgfqpoint{1.015204in}{3.071773in}}{\pgfqpoint{1.007390in}{3.079587in}}%
\pgfpathcurveto{\pgfqpoint{0.999576in}{3.087400in}}{\pgfqpoint{0.988977in}{3.091791in}}{\pgfqpoint{0.977927in}{3.091791in}}%
\pgfpathcurveto{\pgfqpoint{0.966877in}{3.091791in}}{\pgfqpoint{0.956278in}{3.087400in}}{\pgfqpoint{0.948465in}{3.079587in}}%
\pgfpathcurveto{\pgfqpoint{0.940651in}{3.071773in}}{\pgfqpoint{0.936261in}{3.061174in}}{\pgfqpoint{0.936261in}{3.050124in}}%
\pgfpathcurveto{\pgfqpoint{0.936261in}{3.039074in}}{\pgfqpoint{0.940651in}{3.028475in}}{\pgfqpoint{0.948465in}{3.020661in}}%
\pgfpathcurveto{\pgfqpoint{0.956278in}{3.012848in}}{\pgfqpoint{0.966877in}{3.008457in}}{\pgfqpoint{0.977927in}{3.008457in}}%
\pgfpathlineto{\pgfqpoint{0.977927in}{3.008457in}}%
\pgfpathclose%
\pgfusepath{stroke}%
\end{pgfscope}%
\begin{pgfscope}%
\pgfpathrectangle{\pgfqpoint{0.393053in}{0.375000in}}{\pgfqpoint{6.356833in}{5.175000in}}%
\pgfusepath{clip}%
\pgfsetbuttcap%
\pgfsetroundjoin%
\pgfsetlinewidth{1.003750pt}%
\definecolor{currentstroke}{rgb}{0.827451,0.827451,0.827451}%
\pgfsetstrokecolor{currentstroke}%
\pgfsetdash{}{0pt}%
\pgfpathmoveto{\pgfqpoint{1.177335in}{2.903141in}}%
\pgfpathcurveto{\pgfqpoint{1.188385in}{2.903141in}}{\pgfqpoint{1.198984in}{2.907532in}}{\pgfqpoint{1.206798in}{2.915345in}}%
\pgfpathcurveto{\pgfqpoint{1.214611in}{2.923159in}}{\pgfqpoint{1.219002in}{2.933758in}}{\pgfqpoint{1.219002in}{2.944808in}}%
\pgfpathcurveto{\pgfqpoint{1.219002in}{2.955858in}}{\pgfqpoint{1.214611in}{2.966457in}}{\pgfqpoint{1.206798in}{2.974271in}}%
\pgfpathcurveto{\pgfqpoint{1.198984in}{2.982084in}}{\pgfqpoint{1.188385in}{2.986475in}}{\pgfqpoint{1.177335in}{2.986475in}}%
\pgfpathcurveto{\pgfqpoint{1.166285in}{2.986475in}}{\pgfqpoint{1.155686in}{2.982084in}}{\pgfqpoint{1.147872in}{2.974271in}}%
\pgfpathcurveto{\pgfqpoint{1.140059in}{2.966457in}}{\pgfqpoint{1.135668in}{2.955858in}}{\pgfqpoint{1.135668in}{2.944808in}}%
\pgfpathcurveto{\pgfqpoint{1.135668in}{2.933758in}}{\pgfqpoint{1.140059in}{2.923159in}}{\pgfqpoint{1.147872in}{2.915345in}}%
\pgfpathcurveto{\pgfqpoint{1.155686in}{2.907532in}}{\pgfqpoint{1.166285in}{2.903141in}}{\pgfqpoint{1.177335in}{2.903141in}}%
\pgfpathlineto{\pgfqpoint{1.177335in}{2.903141in}}%
\pgfpathclose%
\pgfusepath{stroke}%
\end{pgfscope}%
\begin{pgfscope}%
\pgfpathrectangle{\pgfqpoint{0.393053in}{0.375000in}}{\pgfqpoint{6.356833in}{5.175000in}}%
\pgfusepath{clip}%
\pgfsetbuttcap%
\pgfsetroundjoin%
\pgfsetlinewidth{1.003750pt}%
\definecolor{currentstroke}{rgb}{0.827451,0.827451,0.827451}%
\pgfsetstrokecolor{currentstroke}%
\pgfsetdash{}{0pt}%
\pgfpathmoveto{\pgfqpoint{3.053346in}{1.090590in}}%
\pgfpathcurveto{\pgfqpoint{3.064396in}{1.090590in}}{\pgfqpoint{3.074995in}{1.094981in}}{\pgfqpoint{3.082809in}{1.102794in}}%
\pgfpathcurveto{\pgfqpoint{3.090622in}{1.110608in}}{\pgfqpoint{3.095013in}{1.121207in}}{\pgfqpoint{3.095013in}{1.132257in}}%
\pgfpathcurveto{\pgfqpoint{3.095013in}{1.143307in}}{\pgfqpoint{3.090622in}{1.153906in}}{\pgfqpoint{3.082809in}{1.161720in}}%
\pgfpathcurveto{\pgfqpoint{3.074995in}{1.169533in}}{\pgfqpoint{3.064396in}{1.173924in}}{\pgfqpoint{3.053346in}{1.173924in}}%
\pgfpathcurveto{\pgfqpoint{3.042296in}{1.173924in}}{\pgfqpoint{3.031697in}{1.169533in}}{\pgfqpoint{3.023883in}{1.161720in}}%
\pgfpathcurveto{\pgfqpoint{3.016070in}{1.153906in}}{\pgfqpoint{3.011679in}{1.143307in}}{\pgfqpoint{3.011679in}{1.132257in}}%
\pgfpathcurveto{\pgfqpoint{3.011679in}{1.121207in}}{\pgfqpoint{3.016070in}{1.110608in}}{\pgfqpoint{3.023883in}{1.102794in}}%
\pgfpathcurveto{\pgfqpoint{3.031697in}{1.094981in}}{\pgfqpoint{3.042296in}{1.090590in}}{\pgfqpoint{3.053346in}{1.090590in}}%
\pgfpathlineto{\pgfqpoint{3.053346in}{1.090590in}}%
\pgfpathclose%
\pgfusepath{stroke}%
\end{pgfscope}%
\begin{pgfscope}%
\pgfpathrectangle{\pgfqpoint{0.393053in}{0.375000in}}{\pgfqpoint{6.356833in}{5.175000in}}%
\pgfusepath{clip}%
\pgfsetbuttcap%
\pgfsetroundjoin%
\pgfsetlinewidth{1.003750pt}%
\definecolor{currentstroke}{rgb}{0.827451,0.827451,0.827451}%
\pgfsetstrokecolor{currentstroke}%
\pgfsetdash{}{0pt}%
\pgfpathmoveto{\pgfqpoint{3.318513in}{1.024145in}}%
\pgfpathcurveto{\pgfqpoint{3.329563in}{1.024145in}}{\pgfqpoint{3.340162in}{1.028535in}}{\pgfqpoint{3.347975in}{1.036349in}}%
\pgfpathcurveto{\pgfqpoint{3.355789in}{1.044162in}}{\pgfqpoint{3.360179in}{1.054761in}}{\pgfqpoint{3.360179in}{1.065811in}}%
\pgfpathcurveto{\pgfqpoint{3.360179in}{1.076861in}}{\pgfqpoint{3.355789in}{1.087461in}}{\pgfqpoint{3.347975in}{1.095274in}}%
\pgfpathcurveto{\pgfqpoint{3.340162in}{1.103088in}}{\pgfqpoint{3.329563in}{1.107478in}}{\pgfqpoint{3.318513in}{1.107478in}}%
\pgfpathcurveto{\pgfqpoint{3.307462in}{1.107478in}}{\pgfqpoint{3.296863in}{1.103088in}}{\pgfqpoint{3.289050in}{1.095274in}}%
\pgfpathcurveto{\pgfqpoint{3.281236in}{1.087461in}}{\pgfqpoint{3.276846in}{1.076861in}}{\pgfqpoint{3.276846in}{1.065811in}}%
\pgfpathcurveto{\pgfqpoint{3.276846in}{1.054761in}}{\pgfqpoint{3.281236in}{1.044162in}}{\pgfqpoint{3.289050in}{1.036349in}}%
\pgfpathcurveto{\pgfqpoint{3.296863in}{1.028535in}}{\pgfqpoint{3.307462in}{1.024145in}}{\pgfqpoint{3.318513in}{1.024145in}}%
\pgfpathlineto{\pgfqpoint{3.318513in}{1.024145in}}%
\pgfpathclose%
\pgfusepath{stroke}%
\end{pgfscope}%
\begin{pgfscope}%
\pgfpathrectangle{\pgfqpoint{0.393053in}{0.375000in}}{\pgfqpoint{6.356833in}{5.175000in}}%
\pgfusepath{clip}%
\pgfsetbuttcap%
\pgfsetroundjoin%
\pgfsetlinewidth{1.003750pt}%
\definecolor{currentstroke}{rgb}{0.827451,0.827451,0.827451}%
\pgfsetstrokecolor{currentstroke}%
\pgfsetdash{}{0pt}%
\pgfpathmoveto{\pgfqpoint{4.338031in}{0.752683in}}%
\pgfpathcurveto{\pgfqpoint{4.349081in}{0.752683in}}{\pgfqpoint{4.359680in}{0.757073in}}{\pgfqpoint{4.367493in}{0.764887in}}%
\pgfpathcurveto{\pgfqpoint{4.375307in}{0.772700in}}{\pgfqpoint{4.379697in}{0.783300in}}{\pgfqpoint{4.379697in}{0.794350in}}%
\pgfpathcurveto{\pgfqpoint{4.379697in}{0.805400in}}{\pgfqpoint{4.375307in}{0.815999in}}{\pgfqpoint{4.367493in}{0.823812in}}%
\pgfpathcurveto{\pgfqpoint{4.359680in}{0.831626in}}{\pgfqpoint{4.349081in}{0.836016in}}{\pgfqpoint{4.338031in}{0.836016in}}%
\pgfpathcurveto{\pgfqpoint{4.326981in}{0.836016in}}{\pgfqpoint{4.316381in}{0.831626in}}{\pgfqpoint{4.308568in}{0.823812in}}%
\pgfpathcurveto{\pgfqpoint{4.300754in}{0.815999in}}{\pgfqpoint{4.296364in}{0.805400in}}{\pgfqpoint{4.296364in}{0.794350in}}%
\pgfpathcurveto{\pgfqpoint{4.296364in}{0.783300in}}{\pgfqpoint{4.300754in}{0.772700in}}{\pgfqpoint{4.308568in}{0.764887in}}%
\pgfpathcurveto{\pgfqpoint{4.316381in}{0.757073in}}{\pgfqpoint{4.326981in}{0.752683in}}{\pgfqpoint{4.338031in}{0.752683in}}%
\pgfpathlineto{\pgfqpoint{4.338031in}{0.752683in}}%
\pgfpathclose%
\pgfusepath{stroke}%
\end{pgfscope}%
\begin{pgfscope}%
\pgfpathrectangle{\pgfqpoint{0.393053in}{0.375000in}}{\pgfqpoint{6.356833in}{5.175000in}}%
\pgfusepath{clip}%
\pgfsetbuttcap%
\pgfsetroundjoin%
\pgfsetlinewidth{1.003750pt}%
\definecolor{currentstroke}{rgb}{0.827451,0.827451,0.827451}%
\pgfsetstrokecolor{currentstroke}%
\pgfsetdash{}{0pt}%
\pgfpathmoveto{\pgfqpoint{3.925280in}{0.780281in}}%
\pgfpathcurveto{\pgfqpoint{3.936330in}{0.780281in}}{\pgfqpoint{3.946929in}{0.784671in}}{\pgfqpoint{3.954742in}{0.792484in}}%
\pgfpathcurveto{\pgfqpoint{3.962556in}{0.800298in}}{\pgfqpoint{3.966946in}{0.810897in}}{\pgfqpoint{3.966946in}{0.821947in}}%
\pgfpathcurveto{\pgfqpoint{3.966946in}{0.832997in}}{\pgfqpoint{3.962556in}{0.843596in}}{\pgfqpoint{3.954742in}{0.851410in}}%
\pgfpathcurveto{\pgfqpoint{3.946929in}{0.859224in}}{\pgfqpoint{3.936330in}{0.863614in}}{\pgfqpoint{3.925280in}{0.863614in}}%
\pgfpathcurveto{\pgfqpoint{3.914229in}{0.863614in}}{\pgfqpoint{3.903630in}{0.859224in}}{\pgfqpoint{3.895817in}{0.851410in}}%
\pgfpathcurveto{\pgfqpoint{3.888003in}{0.843596in}}{\pgfqpoint{3.883613in}{0.832997in}}{\pgfqpoint{3.883613in}{0.821947in}}%
\pgfpathcurveto{\pgfqpoint{3.883613in}{0.810897in}}{\pgfqpoint{3.888003in}{0.800298in}}{\pgfqpoint{3.895817in}{0.792484in}}%
\pgfpathcurveto{\pgfqpoint{3.903630in}{0.784671in}}{\pgfqpoint{3.914229in}{0.780281in}}{\pgfqpoint{3.925280in}{0.780281in}}%
\pgfpathlineto{\pgfqpoint{3.925280in}{0.780281in}}%
\pgfpathclose%
\pgfusepath{stroke}%
\end{pgfscope}%
\begin{pgfscope}%
\pgfpathrectangle{\pgfqpoint{0.393053in}{0.375000in}}{\pgfqpoint{6.356833in}{5.175000in}}%
\pgfusepath{clip}%
\pgfsetbuttcap%
\pgfsetroundjoin%
\pgfsetlinewidth{1.003750pt}%
\definecolor{currentstroke}{rgb}{0.827451,0.827451,0.827451}%
\pgfsetstrokecolor{currentstroke}%
\pgfsetdash{}{0pt}%
\pgfpathmoveto{\pgfqpoint{1.350353in}{2.107036in}}%
\pgfpathcurveto{\pgfqpoint{1.361403in}{2.107036in}}{\pgfqpoint{1.372002in}{2.111426in}}{\pgfqpoint{1.379815in}{2.119240in}}%
\pgfpathcurveto{\pgfqpoint{1.387629in}{2.127053in}}{\pgfqpoint{1.392019in}{2.137652in}}{\pgfqpoint{1.392019in}{2.148703in}}%
\pgfpathcurveto{\pgfqpoint{1.392019in}{2.159753in}}{\pgfqpoint{1.387629in}{2.170352in}}{\pgfqpoint{1.379815in}{2.178165in}}%
\pgfpathcurveto{\pgfqpoint{1.372002in}{2.185979in}}{\pgfqpoint{1.361403in}{2.190369in}}{\pgfqpoint{1.350353in}{2.190369in}}%
\pgfpathcurveto{\pgfqpoint{1.339302in}{2.190369in}}{\pgfqpoint{1.328703in}{2.185979in}}{\pgfqpoint{1.320890in}{2.178165in}}%
\pgfpathcurveto{\pgfqpoint{1.313076in}{2.170352in}}{\pgfqpoint{1.308686in}{2.159753in}}{\pgfqpoint{1.308686in}{2.148703in}}%
\pgfpathcurveto{\pgfqpoint{1.308686in}{2.137652in}}{\pgfqpoint{1.313076in}{2.127053in}}{\pgfqpoint{1.320890in}{2.119240in}}%
\pgfpathcurveto{\pgfqpoint{1.328703in}{2.111426in}}{\pgfqpoint{1.339302in}{2.107036in}}{\pgfqpoint{1.350353in}{2.107036in}}%
\pgfpathlineto{\pgfqpoint{1.350353in}{2.107036in}}%
\pgfpathclose%
\pgfusepath{stroke}%
\end{pgfscope}%
\begin{pgfscope}%
\pgfpathrectangle{\pgfqpoint{0.393053in}{0.375000in}}{\pgfqpoint{6.356833in}{5.175000in}}%
\pgfusepath{clip}%
\pgfsetbuttcap%
\pgfsetroundjoin%
\pgfsetlinewidth{1.003750pt}%
\definecolor{currentstroke}{rgb}{0.827451,0.827451,0.827451}%
\pgfsetstrokecolor{currentstroke}%
\pgfsetdash{}{0pt}%
\pgfpathmoveto{\pgfqpoint{1.461271in}{2.053176in}}%
\pgfpathcurveto{\pgfqpoint{1.472321in}{2.053176in}}{\pgfqpoint{1.482920in}{2.057566in}}{\pgfqpoint{1.490733in}{2.065380in}}%
\pgfpathcurveto{\pgfqpoint{1.498547in}{2.073193in}}{\pgfqpoint{1.502937in}{2.083792in}}{\pgfqpoint{1.502937in}{2.094842in}}%
\pgfpathcurveto{\pgfqpoint{1.502937in}{2.105893in}}{\pgfqpoint{1.498547in}{2.116492in}}{\pgfqpoint{1.490733in}{2.124305in}}%
\pgfpathcurveto{\pgfqpoint{1.482920in}{2.132119in}}{\pgfqpoint{1.472321in}{2.136509in}}{\pgfqpoint{1.461271in}{2.136509in}}%
\pgfpathcurveto{\pgfqpoint{1.450220in}{2.136509in}}{\pgfqpoint{1.439621in}{2.132119in}}{\pgfqpoint{1.431808in}{2.124305in}}%
\pgfpathcurveto{\pgfqpoint{1.423994in}{2.116492in}}{\pgfqpoint{1.419604in}{2.105893in}}{\pgfqpoint{1.419604in}{2.094842in}}%
\pgfpathcurveto{\pgfqpoint{1.419604in}{2.083792in}}{\pgfqpoint{1.423994in}{2.073193in}}{\pgfqpoint{1.431808in}{2.065380in}}%
\pgfpathcurveto{\pgfqpoint{1.439621in}{2.057566in}}{\pgfqpoint{1.450220in}{2.053176in}}{\pgfqpoint{1.461271in}{2.053176in}}%
\pgfpathlineto{\pgfqpoint{1.461271in}{2.053176in}}%
\pgfpathclose%
\pgfusepath{stroke}%
\end{pgfscope}%
\begin{pgfscope}%
\pgfpathrectangle{\pgfqpoint{0.393053in}{0.375000in}}{\pgfqpoint{6.356833in}{5.175000in}}%
\pgfusepath{clip}%
\pgfsetbuttcap%
\pgfsetroundjoin%
\pgfsetlinewidth{1.003750pt}%
\definecolor{currentstroke}{rgb}{0.827451,0.827451,0.827451}%
\pgfsetstrokecolor{currentstroke}%
\pgfsetdash{}{0pt}%
\pgfpathmoveto{\pgfqpoint{1.286931in}{2.268304in}}%
\pgfpathcurveto{\pgfqpoint{1.297981in}{2.268304in}}{\pgfqpoint{1.308580in}{2.272695in}}{\pgfqpoint{1.316394in}{2.280508in}}%
\pgfpathcurveto{\pgfqpoint{1.324208in}{2.288322in}}{\pgfqpoint{1.328598in}{2.298921in}}{\pgfqpoint{1.328598in}{2.309971in}}%
\pgfpathcurveto{\pgfqpoint{1.328598in}{2.321021in}}{\pgfqpoint{1.324208in}{2.331620in}}{\pgfqpoint{1.316394in}{2.339434in}}%
\pgfpathcurveto{\pgfqpoint{1.308580in}{2.347247in}}{\pgfqpoint{1.297981in}{2.351638in}}{\pgfqpoint{1.286931in}{2.351638in}}%
\pgfpathcurveto{\pgfqpoint{1.275881in}{2.351638in}}{\pgfqpoint{1.265282in}{2.347247in}}{\pgfqpoint{1.257469in}{2.339434in}}%
\pgfpathcurveto{\pgfqpoint{1.249655in}{2.331620in}}{\pgfqpoint{1.245265in}{2.321021in}}{\pgfqpoint{1.245265in}{2.309971in}}%
\pgfpathcurveto{\pgfqpoint{1.245265in}{2.298921in}}{\pgfqpoint{1.249655in}{2.288322in}}{\pgfqpoint{1.257469in}{2.280508in}}%
\pgfpathcurveto{\pgfqpoint{1.265282in}{2.272695in}}{\pgfqpoint{1.275881in}{2.268304in}}{\pgfqpoint{1.286931in}{2.268304in}}%
\pgfpathlineto{\pgfqpoint{1.286931in}{2.268304in}}%
\pgfpathclose%
\pgfusepath{stroke}%
\end{pgfscope}%
\begin{pgfscope}%
\pgfpathrectangle{\pgfqpoint{0.393053in}{0.375000in}}{\pgfqpoint{6.356833in}{5.175000in}}%
\pgfusepath{clip}%
\pgfsetbuttcap%
\pgfsetroundjoin%
\pgfsetlinewidth{1.003750pt}%
\definecolor{currentstroke}{rgb}{0.827451,0.827451,0.827451}%
\pgfsetstrokecolor{currentstroke}%
\pgfsetdash{}{0pt}%
\pgfpathmoveto{\pgfqpoint{5.137809in}{0.487765in}}%
\pgfpathcurveto{\pgfqpoint{5.148859in}{0.487765in}}{\pgfqpoint{5.159458in}{0.492155in}}{\pgfqpoint{5.167271in}{0.499969in}}%
\pgfpathcurveto{\pgfqpoint{5.175085in}{0.507783in}}{\pgfqpoint{5.179475in}{0.518382in}}{\pgfqpoint{5.179475in}{0.529432in}}%
\pgfpathcurveto{\pgfqpoint{5.179475in}{0.540482in}}{\pgfqpoint{5.175085in}{0.551081in}}{\pgfqpoint{5.167271in}{0.558895in}}%
\pgfpathcurveto{\pgfqpoint{5.159458in}{0.566708in}}{\pgfqpoint{5.148859in}{0.571098in}}{\pgfqpoint{5.137809in}{0.571098in}}%
\pgfpathcurveto{\pgfqpoint{5.126758in}{0.571098in}}{\pgfqpoint{5.116159in}{0.566708in}}{\pgfqpoint{5.108346in}{0.558895in}}%
\pgfpathcurveto{\pgfqpoint{5.100532in}{0.551081in}}{\pgfqpoint{5.096142in}{0.540482in}}{\pgfqpoint{5.096142in}{0.529432in}}%
\pgfpathcurveto{\pgfqpoint{5.096142in}{0.518382in}}{\pgfqpoint{5.100532in}{0.507783in}}{\pgfqpoint{5.108346in}{0.499969in}}%
\pgfpathcurveto{\pgfqpoint{5.116159in}{0.492155in}}{\pgfqpoint{5.126758in}{0.487765in}}{\pgfqpoint{5.137809in}{0.487765in}}%
\pgfpathlineto{\pgfqpoint{5.137809in}{0.487765in}}%
\pgfpathclose%
\pgfusepath{stroke}%
\end{pgfscope}%
\begin{pgfscope}%
\pgfpathrectangle{\pgfqpoint{0.393053in}{0.375000in}}{\pgfqpoint{6.356833in}{5.175000in}}%
\pgfusepath{clip}%
\pgfsetbuttcap%
\pgfsetroundjoin%
\pgfsetlinewidth{1.003750pt}%
\definecolor{currentstroke}{rgb}{0.827451,0.827451,0.827451}%
\pgfsetstrokecolor{currentstroke}%
\pgfsetdash{}{0pt}%
\pgfpathmoveto{\pgfqpoint{1.806681in}{1.831268in}}%
\pgfpathcurveto{\pgfqpoint{1.817731in}{1.831268in}}{\pgfqpoint{1.828330in}{1.835658in}}{\pgfqpoint{1.836144in}{1.843472in}}%
\pgfpathcurveto{\pgfqpoint{1.843957in}{1.851285in}}{\pgfqpoint{1.848348in}{1.861884in}}{\pgfqpoint{1.848348in}{1.872935in}}%
\pgfpathcurveto{\pgfqpoint{1.848348in}{1.883985in}}{\pgfqpoint{1.843957in}{1.894584in}}{\pgfqpoint{1.836144in}{1.902397in}}%
\pgfpathcurveto{\pgfqpoint{1.828330in}{1.910211in}}{\pgfqpoint{1.817731in}{1.914601in}}{\pgfqpoint{1.806681in}{1.914601in}}%
\pgfpathcurveto{\pgfqpoint{1.795631in}{1.914601in}}{\pgfqpoint{1.785032in}{1.910211in}}{\pgfqpoint{1.777218in}{1.902397in}}%
\pgfpathcurveto{\pgfqpoint{1.769405in}{1.894584in}}{\pgfqpoint{1.765014in}{1.883985in}}{\pgfqpoint{1.765014in}{1.872935in}}%
\pgfpathcurveto{\pgfqpoint{1.765014in}{1.861884in}}{\pgfqpoint{1.769405in}{1.851285in}}{\pgfqpoint{1.777218in}{1.843472in}}%
\pgfpathcurveto{\pgfqpoint{1.785032in}{1.835658in}}{\pgfqpoint{1.795631in}{1.831268in}}{\pgfqpoint{1.806681in}{1.831268in}}%
\pgfpathlineto{\pgfqpoint{1.806681in}{1.831268in}}%
\pgfpathclose%
\pgfusepath{stroke}%
\end{pgfscope}%
\begin{pgfscope}%
\pgfpathrectangle{\pgfqpoint{0.393053in}{0.375000in}}{\pgfqpoint{6.356833in}{5.175000in}}%
\pgfusepath{clip}%
\pgfsetbuttcap%
\pgfsetroundjoin%
\pgfsetlinewidth{1.003750pt}%
\definecolor{currentstroke}{rgb}{0.827451,0.827451,0.827451}%
\pgfsetstrokecolor{currentstroke}%
\pgfsetdash{}{0pt}%
\pgfpathmoveto{\pgfqpoint{1.983774in}{1.752458in}}%
\pgfpathcurveto{\pgfqpoint{1.994824in}{1.752458in}}{\pgfqpoint{2.005423in}{1.756849in}}{\pgfqpoint{2.013237in}{1.764662in}}%
\pgfpathcurveto{\pgfqpoint{2.021051in}{1.772476in}}{\pgfqpoint{2.025441in}{1.783075in}}{\pgfqpoint{2.025441in}{1.794125in}}%
\pgfpathcurveto{\pgfqpoint{2.025441in}{1.805175in}}{\pgfqpoint{2.021051in}{1.815774in}}{\pgfqpoint{2.013237in}{1.823588in}}%
\pgfpathcurveto{\pgfqpoint{2.005423in}{1.831401in}}{\pgfqpoint{1.994824in}{1.835792in}}{\pgfqpoint{1.983774in}{1.835792in}}%
\pgfpathcurveto{\pgfqpoint{1.972724in}{1.835792in}}{\pgfqpoint{1.962125in}{1.831401in}}{\pgfqpoint{1.954311in}{1.823588in}}%
\pgfpathcurveto{\pgfqpoint{1.946498in}{1.815774in}}{\pgfqpoint{1.942108in}{1.805175in}}{\pgfqpoint{1.942108in}{1.794125in}}%
\pgfpathcurveto{\pgfqpoint{1.942108in}{1.783075in}}{\pgfqpoint{1.946498in}{1.772476in}}{\pgfqpoint{1.954311in}{1.764662in}}%
\pgfpathcurveto{\pgfqpoint{1.962125in}{1.756849in}}{\pgfqpoint{1.972724in}{1.752458in}}{\pgfqpoint{1.983774in}{1.752458in}}%
\pgfpathlineto{\pgfqpoint{1.983774in}{1.752458in}}%
\pgfpathclose%
\pgfusepath{stroke}%
\end{pgfscope}%
\begin{pgfscope}%
\pgfpathrectangle{\pgfqpoint{0.393053in}{0.375000in}}{\pgfqpoint{6.356833in}{5.175000in}}%
\pgfusepath{clip}%
\pgfsetbuttcap%
\pgfsetroundjoin%
\pgfsetlinewidth{1.003750pt}%
\definecolor{currentstroke}{rgb}{0.827451,0.827451,0.827451}%
\pgfsetstrokecolor{currentstroke}%
\pgfsetdash{}{0pt}%
\pgfpathmoveto{\pgfqpoint{2.916369in}{1.691886in}}%
\pgfpathcurveto{\pgfqpoint{2.927419in}{1.691886in}}{\pgfqpoint{2.938018in}{1.696276in}}{\pgfqpoint{2.945832in}{1.704090in}}%
\pgfpathcurveto{\pgfqpoint{2.953645in}{1.711903in}}{\pgfqpoint{2.958035in}{1.722502in}}{\pgfqpoint{2.958035in}{1.733553in}}%
\pgfpathcurveto{\pgfqpoint{2.958035in}{1.744603in}}{\pgfqpoint{2.953645in}{1.755202in}}{\pgfqpoint{2.945832in}{1.763015in}}%
\pgfpathcurveto{\pgfqpoint{2.938018in}{1.770829in}}{\pgfqpoint{2.927419in}{1.775219in}}{\pgfqpoint{2.916369in}{1.775219in}}%
\pgfpathcurveto{\pgfqpoint{2.905319in}{1.775219in}}{\pgfqpoint{2.894720in}{1.770829in}}{\pgfqpoint{2.886906in}{1.763015in}}%
\pgfpathcurveto{\pgfqpoint{2.879092in}{1.755202in}}{\pgfqpoint{2.874702in}{1.744603in}}{\pgfqpoint{2.874702in}{1.733553in}}%
\pgfpathcurveto{\pgfqpoint{2.874702in}{1.722502in}}{\pgfqpoint{2.879092in}{1.711903in}}{\pgfqpoint{2.886906in}{1.704090in}}%
\pgfpathcurveto{\pgfqpoint{2.894720in}{1.696276in}}{\pgfqpoint{2.905319in}{1.691886in}}{\pgfqpoint{2.916369in}{1.691886in}}%
\pgfpathlineto{\pgfqpoint{2.916369in}{1.691886in}}%
\pgfpathclose%
\pgfusepath{stroke}%
\end{pgfscope}%
\begin{pgfscope}%
\pgfpathrectangle{\pgfqpoint{0.393053in}{0.375000in}}{\pgfqpoint{6.356833in}{5.175000in}}%
\pgfusepath{clip}%
\pgfsetbuttcap%
\pgfsetroundjoin%
\pgfsetlinewidth{1.003750pt}%
\definecolor{currentstroke}{rgb}{0.827451,0.827451,0.827451}%
\pgfsetstrokecolor{currentstroke}%
\pgfsetdash{}{0pt}%
\pgfpathmoveto{\pgfqpoint{3.360869in}{1.081733in}}%
\pgfpathcurveto{\pgfqpoint{3.371919in}{1.081733in}}{\pgfqpoint{3.382518in}{1.086123in}}{\pgfqpoint{3.390332in}{1.093936in}}%
\pgfpathcurveto{\pgfqpoint{3.398145in}{1.101750in}}{\pgfqpoint{3.402535in}{1.112349in}}{\pgfqpoint{3.402535in}{1.123399in}}%
\pgfpathcurveto{\pgfqpoint{3.402535in}{1.134449in}}{\pgfqpoint{3.398145in}{1.145048in}}{\pgfqpoint{3.390332in}{1.152862in}}%
\pgfpathcurveto{\pgfqpoint{3.382518in}{1.160676in}}{\pgfqpoint{3.371919in}{1.165066in}}{\pgfqpoint{3.360869in}{1.165066in}}%
\pgfpathcurveto{\pgfqpoint{3.349819in}{1.165066in}}{\pgfqpoint{3.339220in}{1.160676in}}{\pgfqpoint{3.331406in}{1.152862in}}%
\pgfpathcurveto{\pgfqpoint{3.323592in}{1.145048in}}{\pgfqpoint{3.319202in}{1.134449in}}{\pgfqpoint{3.319202in}{1.123399in}}%
\pgfpathcurveto{\pgfqpoint{3.319202in}{1.112349in}}{\pgfqpoint{3.323592in}{1.101750in}}{\pgfqpoint{3.331406in}{1.093936in}}%
\pgfpathcurveto{\pgfqpoint{3.339220in}{1.086123in}}{\pgfqpoint{3.349819in}{1.081733in}}{\pgfqpoint{3.360869in}{1.081733in}}%
\pgfpathlineto{\pgfqpoint{3.360869in}{1.081733in}}%
\pgfpathclose%
\pgfusepath{stroke}%
\end{pgfscope}%
\begin{pgfscope}%
\pgfpathrectangle{\pgfqpoint{0.393053in}{0.375000in}}{\pgfqpoint{6.356833in}{5.175000in}}%
\pgfusepath{clip}%
\pgfsetbuttcap%
\pgfsetroundjoin%
\pgfsetlinewidth{1.003750pt}%
\definecolor{currentstroke}{rgb}{0.827451,0.827451,0.827451}%
\pgfsetstrokecolor{currentstroke}%
\pgfsetdash{}{0pt}%
\pgfpathmoveto{\pgfqpoint{4.459424in}{0.831339in}}%
\pgfpathcurveto{\pgfqpoint{4.470474in}{0.831339in}}{\pgfqpoint{4.481073in}{0.835729in}}{\pgfqpoint{4.488887in}{0.843543in}}%
\pgfpathcurveto{\pgfqpoint{4.496700in}{0.851357in}}{\pgfqpoint{4.501091in}{0.861956in}}{\pgfqpoint{4.501091in}{0.873006in}}%
\pgfpathcurveto{\pgfqpoint{4.501091in}{0.884056in}}{\pgfqpoint{4.496700in}{0.894655in}}{\pgfqpoint{4.488887in}{0.902469in}}%
\pgfpathcurveto{\pgfqpoint{4.481073in}{0.910282in}}{\pgfqpoint{4.470474in}{0.914673in}}{\pgfqpoint{4.459424in}{0.914673in}}%
\pgfpathcurveto{\pgfqpoint{4.448374in}{0.914673in}}{\pgfqpoint{4.437775in}{0.910282in}}{\pgfqpoint{4.429961in}{0.902469in}}%
\pgfpathcurveto{\pgfqpoint{4.422148in}{0.894655in}}{\pgfqpoint{4.417757in}{0.884056in}}{\pgfqpoint{4.417757in}{0.873006in}}%
\pgfpathcurveto{\pgfqpoint{4.417757in}{0.861956in}}{\pgfqpoint{4.422148in}{0.851357in}}{\pgfqpoint{4.429961in}{0.843543in}}%
\pgfpathcurveto{\pgfqpoint{4.437775in}{0.835729in}}{\pgfqpoint{4.448374in}{0.831339in}}{\pgfqpoint{4.459424in}{0.831339in}}%
\pgfpathlineto{\pgfqpoint{4.459424in}{0.831339in}}%
\pgfpathclose%
\pgfusepath{stroke}%
\end{pgfscope}%
\begin{pgfscope}%
\pgfpathrectangle{\pgfqpoint{0.393053in}{0.375000in}}{\pgfqpoint{6.356833in}{5.175000in}}%
\pgfusepath{clip}%
\pgfsetbuttcap%
\pgfsetroundjoin%
\pgfsetlinewidth{1.003750pt}%
\definecolor{currentstroke}{rgb}{0.827451,0.827451,0.827451}%
\pgfsetstrokecolor{currentstroke}%
\pgfsetdash{}{0pt}%
\pgfpathmoveto{\pgfqpoint{4.302682in}{0.880322in}}%
\pgfpathcurveto{\pgfqpoint{4.313732in}{0.880322in}}{\pgfqpoint{4.324331in}{0.884712in}}{\pgfqpoint{4.332144in}{0.892526in}}%
\pgfpathcurveto{\pgfqpoint{4.339958in}{0.900339in}}{\pgfqpoint{4.344348in}{0.910938in}}{\pgfqpoint{4.344348in}{0.921988in}}%
\pgfpathcurveto{\pgfqpoint{4.344348in}{0.933039in}}{\pgfqpoint{4.339958in}{0.943638in}}{\pgfqpoint{4.332144in}{0.951451in}}%
\pgfpathcurveto{\pgfqpoint{4.324331in}{0.959265in}}{\pgfqpoint{4.313732in}{0.963655in}}{\pgfqpoint{4.302682in}{0.963655in}}%
\pgfpathcurveto{\pgfqpoint{4.291632in}{0.963655in}}{\pgfqpoint{4.281033in}{0.959265in}}{\pgfqpoint{4.273219in}{0.951451in}}%
\pgfpathcurveto{\pgfqpoint{4.265405in}{0.943638in}}{\pgfqpoint{4.261015in}{0.933039in}}{\pgfqpoint{4.261015in}{0.921988in}}%
\pgfpathcurveto{\pgfqpoint{4.261015in}{0.910938in}}{\pgfqpoint{4.265405in}{0.900339in}}{\pgfqpoint{4.273219in}{0.892526in}}%
\pgfpathcurveto{\pgfqpoint{4.281033in}{0.884712in}}{\pgfqpoint{4.291632in}{0.880322in}}{\pgfqpoint{4.302682in}{0.880322in}}%
\pgfpathlineto{\pgfqpoint{4.302682in}{0.880322in}}%
\pgfpathclose%
\pgfusepath{stroke}%
\end{pgfscope}%
\begin{pgfscope}%
\pgfpathrectangle{\pgfqpoint{0.393053in}{0.375000in}}{\pgfqpoint{6.356833in}{5.175000in}}%
\pgfusepath{clip}%
\pgfsetbuttcap%
\pgfsetroundjoin%
\pgfsetlinewidth{1.003750pt}%
\definecolor{currentstroke}{rgb}{0.827451,0.827451,0.827451}%
\pgfsetstrokecolor{currentstroke}%
\pgfsetdash{}{0pt}%
\pgfpathmoveto{\pgfqpoint{4.803817in}{0.795027in}}%
\pgfpathcurveto{\pgfqpoint{4.814867in}{0.795027in}}{\pgfqpoint{4.825466in}{0.799417in}}{\pgfqpoint{4.833280in}{0.807231in}}%
\pgfpathcurveto{\pgfqpoint{4.841093in}{0.815045in}}{\pgfqpoint{4.845483in}{0.825644in}}{\pgfqpoint{4.845483in}{0.836694in}}%
\pgfpathcurveto{\pgfqpoint{4.845483in}{0.847744in}}{\pgfqpoint{4.841093in}{0.858343in}}{\pgfqpoint{4.833280in}{0.866156in}}%
\pgfpathcurveto{\pgfqpoint{4.825466in}{0.873970in}}{\pgfqpoint{4.814867in}{0.878360in}}{\pgfqpoint{4.803817in}{0.878360in}}%
\pgfpathcurveto{\pgfqpoint{4.792767in}{0.878360in}}{\pgfqpoint{4.782168in}{0.873970in}}{\pgfqpoint{4.774354in}{0.866156in}}%
\pgfpathcurveto{\pgfqpoint{4.766540in}{0.858343in}}{\pgfqpoint{4.762150in}{0.847744in}}{\pgfqpoint{4.762150in}{0.836694in}}%
\pgfpathcurveto{\pgfqpoint{4.762150in}{0.825644in}}{\pgfqpoint{4.766540in}{0.815045in}}{\pgfqpoint{4.774354in}{0.807231in}}%
\pgfpathcurveto{\pgfqpoint{4.782168in}{0.799417in}}{\pgfqpoint{4.792767in}{0.795027in}}{\pgfqpoint{4.803817in}{0.795027in}}%
\pgfpathlineto{\pgfqpoint{4.803817in}{0.795027in}}%
\pgfpathclose%
\pgfusepath{stroke}%
\end{pgfscope}%
\begin{pgfscope}%
\pgfpathrectangle{\pgfqpoint{0.393053in}{0.375000in}}{\pgfqpoint{6.356833in}{5.175000in}}%
\pgfusepath{clip}%
\pgfsetbuttcap%
\pgfsetroundjoin%
\pgfsetlinewidth{1.003750pt}%
\definecolor{currentstroke}{rgb}{0.827451,0.827451,0.827451}%
\pgfsetstrokecolor{currentstroke}%
\pgfsetdash{}{0pt}%
\pgfpathmoveto{\pgfqpoint{1.355945in}{2.392613in}}%
\pgfpathcurveto{\pgfqpoint{1.366995in}{2.392613in}}{\pgfqpoint{1.377594in}{2.397003in}}{\pgfqpoint{1.385408in}{2.404817in}}%
\pgfpathcurveto{\pgfqpoint{1.393221in}{2.412631in}}{\pgfqpoint{1.397612in}{2.423230in}}{\pgfqpoint{1.397612in}{2.434280in}}%
\pgfpathcurveto{\pgfqpoint{1.397612in}{2.445330in}}{\pgfqpoint{1.393221in}{2.455929in}}{\pgfqpoint{1.385408in}{2.463742in}}%
\pgfpathcurveto{\pgfqpoint{1.377594in}{2.471556in}}{\pgfqpoint{1.366995in}{2.475946in}}{\pgfqpoint{1.355945in}{2.475946in}}%
\pgfpathcurveto{\pgfqpoint{1.344895in}{2.475946in}}{\pgfqpoint{1.334296in}{2.471556in}}{\pgfqpoint{1.326482in}{2.463742in}}%
\pgfpathcurveto{\pgfqpoint{1.318669in}{2.455929in}}{\pgfqpoint{1.314278in}{2.445330in}}{\pgfqpoint{1.314278in}{2.434280in}}%
\pgfpathcurveto{\pgfqpoint{1.314278in}{2.423230in}}{\pgfqpoint{1.318669in}{2.412631in}}{\pgfqpoint{1.326482in}{2.404817in}}%
\pgfpathcurveto{\pgfqpoint{1.334296in}{2.397003in}}{\pgfqpoint{1.344895in}{2.392613in}}{\pgfqpoint{1.355945in}{2.392613in}}%
\pgfpathlineto{\pgfqpoint{1.355945in}{2.392613in}}%
\pgfpathclose%
\pgfusepath{stroke}%
\end{pgfscope}%
\begin{pgfscope}%
\pgfpathrectangle{\pgfqpoint{0.393053in}{0.375000in}}{\pgfqpoint{6.356833in}{5.175000in}}%
\pgfusepath{clip}%
\pgfsetbuttcap%
\pgfsetroundjoin%
\pgfsetlinewidth{1.003750pt}%
\definecolor{currentstroke}{rgb}{0.827451,0.827451,0.827451}%
\pgfsetstrokecolor{currentstroke}%
\pgfsetdash{}{0pt}%
\pgfpathmoveto{\pgfqpoint{1.293215in}{2.631475in}}%
\pgfpathcurveto{\pgfqpoint{1.304265in}{2.631475in}}{\pgfqpoint{1.314864in}{2.635866in}}{\pgfqpoint{1.322678in}{2.643679in}}%
\pgfpathcurveto{\pgfqpoint{1.330491in}{2.651493in}}{\pgfqpoint{1.334882in}{2.662092in}}{\pgfqpoint{1.334882in}{2.673142in}}%
\pgfpathcurveto{\pgfqpoint{1.334882in}{2.684192in}}{\pgfqpoint{1.330491in}{2.694791in}}{\pgfqpoint{1.322678in}{2.702605in}}%
\pgfpathcurveto{\pgfqpoint{1.314864in}{2.710418in}}{\pgfqpoint{1.304265in}{2.714809in}}{\pgfqpoint{1.293215in}{2.714809in}}%
\pgfpathcurveto{\pgfqpoint{1.282165in}{2.714809in}}{\pgfqpoint{1.271566in}{2.710418in}}{\pgfqpoint{1.263752in}{2.702605in}}%
\pgfpathcurveto{\pgfqpoint{1.255939in}{2.694791in}}{\pgfqpoint{1.251548in}{2.684192in}}{\pgfqpoint{1.251548in}{2.673142in}}%
\pgfpathcurveto{\pgfqpoint{1.251548in}{2.662092in}}{\pgfqpoint{1.255939in}{2.651493in}}{\pgfqpoint{1.263752in}{2.643679in}}%
\pgfpathcurveto{\pgfqpoint{1.271566in}{2.635866in}}{\pgfqpoint{1.282165in}{2.631475in}}{\pgfqpoint{1.293215in}{2.631475in}}%
\pgfpathlineto{\pgfqpoint{1.293215in}{2.631475in}}%
\pgfpathclose%
\pgfusepath{stroke}%
\end{pgfscope}%
\begin{pgfscope}%
\pgfpathrectangle{\pgfqpoint{0.393053in}{0.375000in}}{\pgfqpoint{6.356833in}{5.175000in}}%
\pgfusepath{clip}%
\pgfsetbuttcap%
\pgfsetroundjoin%
\pgfsetlinewidth{1.003750pt}%
\definecolor{currentstroke}{rgb}{0.827451,0.827451,0.827451}%
\pgfsetstrokecolor{currentstroke}%
\pgfsetdash{}{0pt}%
\pgfpathmoveto{\pgfqpoint{5.273810in}{0.561810in}}%
\pgfpathcurveto{\pgfqpoint{5.284861in}{0.561810in}}{\pgfqpoint{5.295460in}{0.566200in}}{\pgfqpoint{5.303273in}{0.574014in}}%
\pgfpathcurveto{\pgfqpoint{5.311087in}{0.581827in}}{\pgfqpoint{5.315477in}{0.592426in}}{\pgfqpoint{5.315477in}{0.603477in}}%
\pgfpathcurveto{\pgfqpoint{5.315477in}{0.614527in}}{\pgfqpoint{5.311087in}{0.625126in}}{\pgfqpoint{5.303273in}{0.632939in}}%
\pgfpathcurveto{\pgfqpoint{5.295460in}{0.640753in}}{\pgfqpoint{5.284861in}{0.645143in}}{\pgfqpoint{5.273810in}{0.645143in}}%
\pgfpathcurveto{\pgfqpoint{5.262760in}{0.645143in}}{\pgfqpoint{5.252161in}{0.640753in}}{\pgfqpoint{5.244348in}{0.632939in}}%
\pgfpathcurveto{\pgfqpoint{5.236534in}{0.625126in}}{\pgfqpoint{5.232144in}{0.614527in}}{\pgfqpoint{5.232144in}{0.603477in}}%
\pgfpathcurveto{\pgfqpoint{5.232144in}{0.592426in}}{\pgfqpoint{5.236534in}{0.581827in}}{\pgfqpoint{5.244348in}{0.574014in}}%
\pgfpathcurveto{\pgfqpoint{5.252161in}{0.566200in}}{\pgfqpoint{5.262760in}{0.561810in}}{\pgfqpoint{5.273810in}{0.561810in}}%
\pgfpathlineto{\pgfqpoint{5.273810in}{0.561810in}}%
\pgfpathclose%
\pgfusepath{stroke}%
\end{pgfscope}%
\begin{pgfscope}%
\pgfpathrectangle{\pgfqpoint{0.393053in}{0.375000in}}{\pgfqpoint{6.356833in}{5.175000in}}%
\pgfusepath{clip}%
\pgfsetbuttcap%
\pgfsetroundjoin%
\pgfsetlinewidth{1.003750pt}%
\definecolor{currentstroke}{rgb}{0.827451,0.827451,0.827451}%
\pgfsetstrokecolor{currentstroke}%
\pgfsetdash{}{0pt}%
\pgfpathmoveto{\pgfqpoint{1.859283in}{2.358637in}}%
\pgfpathcurveto{\pgfqpoint{1.870333in}{2.358637in}}{\pgfqpoint{1.880932in}{2.363027in}}{\pgfqpoint{1.888745in}{2.370841in}}%
\pgfpathcurveto{\pgfqpoint{1.896559in}{2.378654in}}{\pgfqpoint{1.900949in}{2.389253in}}{\pgfqpoint{1.900949in}{2.400304in}}%
\pgfpathcurveto{\pgfqpoint{1.900949in}{2.411354in}}{\pgfqpoint{1.896559in}{2.421953in}}{\pgfqpoint{1.888745in}{2.429766in}}%
\pgfpathcurveto{\pgfqpoint{1.880932in}{2.437580in}}{\pgfqpoint{1.870333in}{2.441970in}}{\pgfqpoint{1.859283in}{2.441970in}}%
\pgfpathcurveto{\pgfqpoint{1.848233in}{2.441970in}}{\pgfqpoint{1.837634in}{2.437580in}}{\pgfqpoint{1.829820in}{2.429766in}}%
\pgfpathcurveto{\pgfqpoint{1.822006in}{2.421953in}}{\pgfqpoint{1.817616in}{2.411354in}}{\pgfqpoint{1.817616in}{2.400304in}}%
\pgfpathcurveto{\pgfqpoint{1.817616in}{2.389253in}}{\pgfqpoint{1.822006in}{2.378654in}}{\pgfqpoint{1.829820in}{2.370841in}}%
\pgfpathcurveto{\pgfqpoint{1.837634in}{2.363027in}}{\pgfqpoint{1.848233in}{2.358637in}}{\pgfqpoint{1.859283in}{2.358637in}}%
\pgfpathlineto{\pgfqpoint{1.859283in}{2.358637in}}%
\pgfpathclose%
\pgfusepath{stroke}%
\end{pgfscope}%
\begin{pgfscope}%
\pgfpathrectangle{\pgfqpoint{0.393053in}{0.375000in}}{\pgfqpoint{6.356833in}{5.175000in}}%
\pgfusepath{clip}%
\pgfsetbuttcap%
\pgfsetroundjoin%
\pgfsetlinewidth{1.003750pt}%
\definecolor{currentstroke}{rgb}{0.827451,0.827451,0.827451}%
\pgfsetstrokecolor{currentstroke}%
\pgfsetdash{}{0pt}%
\pgfpathmoveto{\pgfqpoint{2.405446in}{2.076936in}}%
\pgfpathcurveto{\pgfqpoint{2.416496in}{2.076936in}}{\pgfqpoint{2.427095in}{2.081326in}}{\pgfqpoint{2.434909in}{2.089140in}}%
\pgfpathcurveto{\pgfqpoint{2.442722in}{2.096954in}}{\pgfqpoint{2.447113in}{2.107553in}}{\pgfqpoint{2.447113in}{2.118603in}}%
\pgfpathcurveto{\pgfqpoint{2.447113in}{2.129653in}}{\pgfqpoint{2.442722in}{2.140252in}}{\pgfqpoint{2.434909in}{2.148066in}}%
\pgfpathcurveto{\pgfqpoint{2.427095in}{2.155879in}}{\pgfqpoint{2.416496in}{2.160269in}}{\pgfqpoint{2.405446in}{2.160269in}}%
\pgfpathcurveto{\pgfqpoint{2.394396in}{2.160269in}}{\pgfqpoint{2.383797in}{2.155879in}}{\pgfqpoint{2.375983in}{2.148066in}}%
\pgfpathcurveto{\pgfqpoint{2.368170in}{2.140252in}}{\pgfqpoint{2.363779in}{2.129653in}}{\pgfqpoint{2.363779in}{2.118603in}}%
\pgfpathcurveto{\pgfqpoint{2.363779in}{2.107553in}}{\pgfqpoint{2.368170in}{2.096954in}}{\pgfqpoint{2.375983in}{2.089140in}}%
\pgfpathcurveto{\pgfqpoint{2.383797in}{2.081326in}}{\pgfqpoint{2.394396in}{2.076936in}}{\pgfqpoint{2.405446in}{2.076936in}}%
\pgfpathlineto{\pgfqpoint{2.405446in}{2.076936in}}%
\pgfpathclose%
\pgfusepath{stroke}%
\end{pgfscope}%
\begin{pgfscope}%
\pgfpathrectangle{\pgfqpoint{0.393053in}{0.375000in}}{\pgfqpoint{6.356833in}{5.175000in}}%
\pgfusepath{clip}%
\pgfsetbuttcap%
\pgfsetroundjoin%
\pgfsetlinewidth{1.003750pt}%
\definecolor{currentstroke}{rgb}{0.827451,0.827451,0.827451}%
\pgfsetstrokecolor{currentstroke}%
\pgfsetdash{}{0pt}%
\pgfpathmoveto{\pgfqpoint{2.530188in}{2.066339in}}%
\pgfpathcurveto{\pgfqpoint{2.541238in}{2.066339in}}{\pgfqpoint{2.551837in}{2.070730in}}{\pgfqpoint{2.559651in}{2.078543in}}%
\pgfpathcurveto{\pgfqpoint{2.567465in}{2.086357in}}{\pgfqpoint{2.571855in}{2.096956in}}{\pgfqpoint{2.571855in}{2.108006in}}%
\pgfpathcurveto{\pgfqpoint{2.571855in}{2.119056in}}{\pgfqpoint{2.567465in}{2.129655in}}{\pgfqpoint{2.559651in}{2.137469in}}%
\pgfpathcurveto{\pgfqpoint{2.551837in}{2.145282in}}{\pgfqpoint{2.541238in}{2.149673in}}{\pgfqpoint{2.530188in}{2.149673in}}%
\pgfpathcurveto{\pgfqpoint{2.519138in}{2.149673in}}{\pgfqpoint{2.508539in}{2.145282in}}{\pgfqpoint{2.500726in}{2.137469in}}%
\pgfpathcurveto{\pgfqpoint{2.492912in}{2.129655in}}{\pgfqpoint{2.488522in}{2.119056in}}{\pgfqpoint{2.488522in}{2.108006in}}%
\pgfpathcurveto{\pgfqpoint{2.488522in}{2.096956in}}{\pgfqpoint{2.492912in}{2.086357in}}{\pgfqpoint{2.500726in}{2.078543in}}%
\pgfpathcurveto{\pgfqpoint{2.508539in}{2.070730in}}{\pgfqpoint{2.519138in}{2.066339in}}{\pgfqpoint{2.530188in}{2.066339in}}%
\pgfpathlineto{\pgfqpoint{2.530188in}{2.066339in}}%
\pgfpathclose%
\pgfusepath{stroke}%
\end{pgfscope}%
\begin{pgfscope}%
\pgfpathrectangle{\pgfqpoint{0.393053in}{0.375000in}}{\pgfqpoint{6.356833in}{5.175000in}}%
\pgfusepath{clip}%
\pgfsetbuttcap%
\pgfsetroundjoin%
\pgfsetlinewidth{1.003750pt}%
\definecolor{currentstroke}{rgb}{0.827451,0.827451,0.827451}%
\pgfsetstrokecolor{currentstroke}%
\pgfsetdash{}{0pt}%
\pgfpathmoveto{\pgfqpoint{2.047664in}{2.235271in}}%
\pgfpathcurveto{\pgfqpoint{2.058714in}{2.235271in}}{\pgfqpoint{2.069313in}{2.239662in}}{\pgfqpoint{2.077127in}{2.247475in}}%
\pgfpathcurveto{\pgfqpoint{2.084940in}{2.255289in}}{\pgfqpoint{2.089331in}{2.265888in}}{\pgfqpoint{2.089331in}{2.276938in}}%
\pgfpathcurveto{\pgfqpoint{2.089331in}{2.287988in}}{\pgfqpoint{2.084940in}{2.298587in}}{\pgfqpoint{2.077127in}{2.306401in}}%
\pgfpathcurveto{\pgfqpoint{2.069313in}{2.314214in}}{\pgfqpoint{2.058714in}{2.318605in}}{\pgfqpoint{2.047664in}{2.318605in}}%
\pgfpathcurveto{\pgfqpoint{2.036614in}{2.318605in}}{\pgfqpoint{2.026015in}{2.314214in}}{\pgfqpoint{2.018201in}{2.306401in}}%
\pgfpathcurveto{\pgfqpoint{2.010388in}{2.298587in}}{\pgfqpoint{2.005997in}{2.287988in}}{\pgfqpoint{2.005997in}{2.276938in}}%
\pgfpathcurveto{\pgfqpoint{2.005997in}{2.265888in}}{\pgfqpoint{2.010388in}{2.255289in}}{\pgfqpoint{2.018201in}{2.247475in}}%
\pgfpathcurveto{\pgfqpoint{2.026015in}{2.239662in}}{\pgfqpoint{2.036614in}{2.235271in}}{\pgfqpoint{2.047664in}{2.235271in}}%
\pgfpathlineto{\pgfqpoint{2.047664in}{2.235271in}}%
\pgfpathclose%
\pgfusepath{stroke}%
\end{pgfscope}%
\begin{pgfscope}%
\pgfpathrectangle{\pgfqpoint{0.393053in}{0.375000in}}{\pgfqpoint{6.356833in}{5.175000in}}%
\pgfusepath{clip}%
\pgfsetbuttcap%
\pgfsetroundjoin%
\pgfsetlinewidth{1.003750pt}%
\definecolor{currentstroke}{rgb}{0.827451,0.827451,0.827451}%
\pgfsetstrokecolor{currentstroke}%
\pgfsetdash{}{0pt}%
\pgfpathmoveto{\pgfqpoint{2.979019in}{1.983017in}}%
\pgfpathcurveto{\pgfqpoint{2.990069in}{1.983017in}}{\pgfqpoint{3.000668in}{1.987407in}}{\pgfqpoint{3.008482in}{1.995221in}}%
\pgfpathcurveto{\pgfqpoint{3.016296in}{2.003035in}}{\pgfqpoint{3.020686in}{2.013634in}}{\pgfqpoint{3.020686in}{2.024684in}}%
\pgfpathcurveto{\pgfqpoint{3.020686in}{2.035734in}}{\pgfqpoint{3.016296in}{2.046333in}}{\pgfqpoint{3.008482in}{2.054147in}}%
\pgfpathcurveto{\pgfqpoint{3.000668in}{2.061960in}}{\pgfqpoint{2.990069in}{2.066351in}}{\pgfqpoint{2.979019in}{2.066351in}}%
\pgfpathcurveto{\pgfqpoint{2.967969in}{2.066351in}}{\pgfqpoint{2.957370in}{2.061960in}}{\pgfqpoint{2.949556in}{2.054147in}}%
\pgfpathcurveto{\pgfqpoint{2.941743in}{2.046333in}}{\pgfqpoint{2.937353in}{2.035734in}}{\pgfqpoint{2.937353in}{2.024684in}}%
\pgfpathcurveto{\pgfqpoint{2.937353in}{2.013634in}}{\pgfqpoint{2.941743in}{2.003035in}}{\pgfqpoint{2.949556in}{1.995221in}}%
\pgfpathcurveto{\pgfqpoint{2.957370in}{1.987407in}}{\pgfqpoint{2.967969in}{1.983017in}}{\pgfqpoint{2.979019in}{1.983017in}}%
\pgfpathlineto{\pgfqpoint{2.979019in}{1.983017in}}%
\pgfpathclose%
\pgfusepath{stroke}%
\end{pgfscope}%
\begin{pgfscope}%
\pgfpathrectangle{\pgfqpoint{0.393053in}{0.375000in}}{\pgfqpoint{6.356833in}{5.175000in}}%
\pgfusepath{clip}%
\pgfsetbuttcap%
\pgfsetroundjoin%
\pgfsetlinewidth{1.003750pt}%
\definecolor{currentstroke}{rgb}{0.827451,0.827451,0.827451}%
\pgfsetstrokecolor{currentstroke}%
\pgfsetdash{}{0pt}%
\pgfpathmoveto{\pgfqpoint{3.666311in}{1.262542in}}%
\pgfpathcurveto{\pgfqpoint{3.677361in}{1.262542in}}{\pgfqpoint{3.687960in}{1.266933in}}{\pgfqpoint{3.695774in}{1.274746in}}%
\pgfpathcurveto{\pgfqpoint{3.703588in}{1.282560in}}{\pgfqpoint{3.707978in}{1.293159in}}{\pgfqpoint{3.707978in}{1.304209in}}%
\pgfpathcurveto{\pgfqpoint{3.707978in}{1.315259in}}{\pgfqpoint{3.703588in}{1.325858in}}{\pgfqpoint{3.695774in}{1.333672in}}%
\pgfpathcurveto{\pgfqpoint{3.687960in}{1.341486in}}{\pgfqpoint{3.677361in}{1.345876in}}{\pgfqpoint{3.666311in}{1.345876in}}%
\pgfpathcurveto{\pgfqpoint{3.655261in}{1.345876in}}{\pgfqpoint{3.644662in}{1.341486in}}{\pgfqpoint{3.636848in}{1.333672in}}%
\pgfpathcurveto{\pgfqpoint{3.629035in}{1.325858in}}{\pgfqpoint{3.624644in}{1.315259in}}{\pgfqpoint{3.624644in}{1.304209in}}%
\pgfpathcurveto{\pgfqpoint{3.624644in}{1.293159in}}{\pgfqpoint{3.629035in}{1.282560in}}{\pgfqpoint{3.636848in}{1.274746in}}%
\pgfpathcurveto{\pgfqpoint{3.644662in}{1.266933in}}{\pgfqpoint{3.655261in}{1.262542in}}{\pgfqpoint{3.666311in}{1.262542in}}%
\pgfpathlineto{\pgfqpoint{3.666311in}{1.262542in}}%
\pgfpathclose%
\pgfusepath{stroke}%
\end{pgfscope}%
\begin{pgfscope}%
\pgfpathrectangle{\pgfqpoint{0.393053in}{0.375000in}}{\pgfqpoint{6.356833in}{5.175000in}}%
\pgfusepath{clip}%
\pgfsetbuttcap%
\pgfsetroundjoin%
\pgfsetlinewidth{1.003750pt}%
\definecolor{currentstroke}{rgb}{0.827451,0.827451,0.827451}%
\pgfsetstrokecolor{currentstroke}%
\pgfsetdash{}{0pt}%
\pgfpathmoveto{\pgfqpoint{3.407815in}{1.595979in}}%
\pgfpathcurveto{\pgfqpoint{3.418865in}{1.595979in}}{\pgfqpoint{3.429464in}{1.600369in}}{\pgfqpoint{3.437278in}{1.608183in}}%
\pgfpathcurveto{\pgfqpoint{3.445091in}{1.615996in}}{\pgfqpoint{3.449481in}{1.626595in}}{\pgfqpoint{3.449481in}{1.637645in}}%
\pgfpathcurveto{\pgfqpoint{3.449481in}{1.648695in}}{\pgfqpoint{3.445091in}{1.659294in}}{\pgfqpoint{3.437278in}{1.667108in}}%
\pgfpathcurveto{\pgfqpoint{3.429464in}{1.674922in}}{\pgfqpoint{3.418865in}{1.679312in}}{\pgfqpoint{3.407815in}{1.679312in}}%
\pgfpathcurveto{\pgfqpoint{3.396765in}{1.679312in}}{\pgfqpoint{3.386166in}{1.674922in}}{\pgfqpoint{3.378352in}{1.667108in}}%
\pgfpathcurveto{\pgfqpoint{3.370538in}{1.659294in}}{\pgfqpoint{3.366148in}{1.648695in}}{\pgfqpoint{3.366148in}{1.637645in}}%
\pgfpathcurveto{\pgfqpoint{3.366148in}{1.626595in}}{\pgfqpoint{3.370538in}{1.615996in}}{\pgfqpoint{3.378352in}{1.608183in}}%
\pgfpathcurveto{\pgfqpoint{3.386166in}{1.600369in}}{\pgfqpoint{3.396765in}{1.595979in}}{\pgfqpoint{3.407815in}{1.595979in}}%
\pgfpathlineto{\pgfqpoint{3.407815in}{1.595979in}}%
\pgfpathclose%
\pgfusepath{stroke}%
\end{pgfscope}%
\begin{pgfscope}%
\pgfpathrectangle{\pgfqpoint{0.393053in}{0.375000in}}{\pgfqpoint{6.356833in}{5.175000in}}%
\pgfusepath{clip}%
\pgfsetbuttcap%
\pgfsetroundjoin%
\pgfsetlinewidth{1.003750pt}%
\definecolor{currentstroke}{rgb}{0.827451,0.827451,0.827451}%
\pgfsetstrokecolor{currentstroke}%
\pgfsetdash{}{0pt}%
\pgfpathmoveto{\pgfqpoint{4.630868in}{0.989199in}}%
\pgfpathcurveto{\pgfqpoint{4.641918in}{0.989199in}}{\pgfqpoint{4.652517in}{0.993589in}}{\pgfqpoint{4.660331in}{1.001403in}}%
\pgfpathcurveto{\pgfqpoint{4.668145in}{1.009216in}}{\pgfqpoint{4.672535in}{1.019815in}}{\pgfqpoint{4.672535in}{1.030866in}}%
\pgfpathcurveto{\pgfqpoint{4.672535in}{1.041916in}}{\pgfqpoint{4.668145in}{1.052515in}}{\pgfqpoint{4.660331in}{1.060328in}}%
\pgfpathcurveto{\pgfqpoint{4.652517in}{1.068142in}}{\pgfqpoint{4.641918in}{1.072532in}}{\pgfqpoint{4.630868in}{1.072532in}}%
\pgfpathcurveto{\pgfqpoint{4.619818in}{1.072532in}}{\pgfqpoint{4.609219in}{1.068142in}}{\pgfqpoint{4.601406in}{1.060328in}}%
\pgfpathcurveto{\pgfqpoint{4.593592in}{1.052515in}}{\pgfqpoint{4.589202in}{1.041916in}}{\pgfqpoint{4.589202in}{1.030866in}}%
\pgfpathcurveto{\pgfqpoint{4.589202in}{1.019815in}}{\pgfqpoint{4.593592in}{1.009216in}}{\pgfqpoint{4.601406in}{1.001403in}}%
\pgfpathcurveto{\pgfqpoint{4.609219in}{0.993589in}}{\pgfqpoint{4.619818in}{0.989199in}}{\pgfqpoint{4.630868in}{0.989199in}}%
\pgfpathlineto{\pgfqpoint{4.630868in}{0.989199in}}%
\pgfpathclose%
\pgfusepath{stroke}%
\end{pgfscope}%
\begin{pgfscope}%
\pgfpathrectangle{\pgfqpoint{0.393053in}{0.375000in}}{\pgfqpoint{6.356833in}{5.175000in}}%
\pgfusepath{clip}%
\pgfsetbuttcap%
\pgfsetroundjoin%
\pgfsetlinewidth{1.003750pt}%
\definecolor{currentstroke}{rgb}{0.827451,0.827451,0.827451}%
\pgfsetstrokecolor{currentstroke}%
\pgfsetdash{}{0pt}%
\pgfpathmoveto{\pgfqpoint{1.398867in}{2.806411in}}%
\pgfpathcurveto{\pgfqpoint{1.409917in}{2.806411in}}{\pgfqpoint{1.420516in}{2.810801in}}{\pgfqpoint{1.428329in}{2.818615in}}%
\pgfpathcurveto{\pgfqpoint{1.436143in}{2.826429in}}{\pgfqpoint{1.440533in}{2.837028in}}{\pgfqpoint{1.440533in}{2.848078in}}%
\pgfpathcurveto{\pgfqpoint{1.440533in}{2.859128in}}{\pgfqpoint{1.436143in}{2.869727in}}{\pgfqpoint{1.428329in}{2.877541in}}%
\pgfpathcurveto{\pgfqpoint{1.420516in}{2.885354in}}{\pgfqpoint{1.409917in}{2.889745in}}{\pgfqpoint{1.398867in}{2.889745in}}%
\pgfpathcurveto{\pgfqpoint{1.387816in}{2.889745in}}{\pgfqpoint{1.377217in}{2.885354in}}{\pgfqpoint{1.369404in}{2.877541in}}%
\pgfpathcurveto{\pgfqpoint{1.361590in}{2.869727in}}{\pgfqpoint{1.357200in}{2.859128in}}{\pgfqpoint{1.357200in}{2.848078in}}%
\pgfpathcurveto{\pgfqpoint{1.357200in}{2.837028in}}{\pgfqpoint{1.361590in}{2.826429in}}{\pgfqpoint{1.369404in}{2.818615in}}%
\pgfpathcurveto{\pgfqpoint{1.377217in}{2.810801in}}{\pgfqpoint{1.387816in}{2.806411in}}{\pgfqpoint{1.398867in}{2.806411in}}%
\pgfpathlineto{\pgfqpoint{1.398867in}{2.806411in}}%
\pgfpathclose%
\pgfusepath{stroke}%
\end{pgfscope}%
\begin{pgfscope}%
\pgfpathrectangle{\pgfqpoint{0.393053in}{0.375000in}}{\pgfqpoint{6.356833in}{5.175000in}}%
\pgfusepath{clip}%
\pgfsetbuttcap%
\pgfsetroundjoin%
\pgfsetlinewidth{1.003750pt}%
\definecolor{currentstroke}{rgb}{0.827451,0.827451,0.827451}%
\pgfsetstrokecolor{currentstroke}%
\pgfsetdash{}{0pt}%
\pgfpathmoveto{\pgfqpoint{1.665651in}{2.690508in}}%
\pgfpathcurveto{\pgfqpoint{1.676701in}{2.690508in}}{\pgfqpoint{1.687300in}{2.694898in}}{\pgfqpoint{1.695113in}{2.702712in}}%
\pgfpathcurveto{\pgfqpoint{1.702927in}{2.710525in}}{\pgfqpoint{1.707317in}{2.721124in}}{\pgfqpoint{1.707317in}{2.732174in}}%
\pgfpathcurveto{\pgfqpoint{1.707317in}{2.743225in}}{\pgfqpoint{1.702927in}{2.753824in}}{\pgfqpoint{1.695113in}{2.761637in}}%
\pgfpathcurveto{\pgfqpoint{1.687300in}{2.769451in}}{\pgfqpoint{1.676701in}{2.773841in}}{\pgfqpoint{1.665651in}{2.773841in}}%
\pgfpathcurveto{\pgfqpoint{1.654600in}{2.773841in}}{\pgfqpoint{1.644001in}{2.769451in}}{\pgfqpoint{1.636188in}{2.761637in}}%
\pgfpathcurveto{\pgfqpoint{1.628374in}{2.753824in}}{\pgfqpoint{1.623984in}{2.743225in}}{\pgfqpoint{1.623984in}{2.732174in}}%
\pgfpathcurveto{\pgfqpoint{1.623984in}{2.721124in}}{\pgfqpoint{1.628374in}{2.710525in}}{\pgfqpoint{1.636188in}{2.702712in}}%
\pgfpathcurveto{\pgfqpoint{1.644001in}{2.694898in}}{\pgfqpoint{1.654600in}{2.690508in}}{\pgfqpoint{1.665651in}{2.690508in}}%
\pgfpathlineto{\pgfqpoint{1.665651in}{2.690508in}}%
\pgfpathclose%
\pgfusepath{stroke}%
\end{pgfscope}%
\begin{pgfscope}%
\pgfpathrectangle{\pgfqpoint{0.393053in}{0.375000in}}{\pgfqpoint{6.356833in}{5.175000in}}%
\pgfusepath{clip}%
\pgfsetbuttcap%
\pgfsetroundjoin%
\pgfsetlinewidth{1.003750pt}%
\definecolor{currentstroke}{rgb}{0.827451,0.827451,0.827451}%
\pgfsetstrokecolor{currentstroke}%
\pgfsetdash{}{0pt}%
\pgfpathmoveto{\pgfqpoint{3.554528in}{1.660486in}}%
\pgfpathcurveto{\pgfqpoint{3.565578in}{1.660486in}}{\pgfqpoint{3.576177in}{1.664877in}}{\pgfqpoint{3.583991in}{1.672690in}}%
\pgfpathcurveto{\pgfqpoint{3.591804in}{1.680504in}}{\pgfqpoint{3.596195in}{1.691103in}}{\pgfqpoint{3.596195in}{1.702153in}}%
\pgfpathcurveto{\pgfqpoint{3.596195in}{1.713203in}}{\pgfqpoint{3.591804in}{1.723802in}}{\pgfqpoint{3.583991in}{1.731616in}}%
\pgfpathcurveto{\pgfqpoint{3.576177in}{1.739429in}}{\pgfqpoint{3.565578in}{1.743820in}}{\pgfqpoint{3.554528in}{1.743820in}}%
\pgfpathcurveto{\pgfqpoint{3.543478in}{1.743820in}}{\pgfqpoint{3.532879in}{1.739429in}}{\pgfqpoint{3.525065in}{1.731616in}}%
\pgfpathcurveto{\pgfqpoint{3.517252in}{1.723802in}}{\pgfqpoint{3.512861in}{1.713203in}}{\pgfqpoint{3.512861in}{1.702153in}}%
\pgfpathcurveto{\pgfqpoint{3.512861in}{1.691103in}}{\pgfqpoint{3.517252in}{1.680504in}}{\pgfqpoint{3.525065in}{1.672690in}}%
\pgfpathcurveto{\pgfqpoint{3.532879in}{1.664877in}}{\pgfqpoint{3.543478in}{1.660486in}}{\pgfqpoint{3.554528in}{1.660486in}}%
\pgfpathlineto{\pgfqpoint{3.554528in}{1.660486in}}%
\pgfpathclose%
\pgfusepath{stroke}%
\end{pgfscope}%
\begin{pgfscope}%
\pgfpathrectangle{\pgfqpoint{0.393053in}{0.375000in}}{\pgfqpoint{6.356833in}{5.175000in}}%
\pgfusepath{clip}%
\pgfsetbuttcap%
\pgfsetroundjoin%
\pgfsetlinewidth{1.003750pt}%
\definecolor{currentstroke}{rgb}{0.827451,0.827451,0.827451}%
\pgfsetstrokecolor{currentstroke}%
\pgfsetdash{}{0pt}%
\pgfpathmoveto{\pgfqpoint{4.130219in}{1.886699in}}%
\pgfpathcurveto{\pgfqpoint{4.141270in}{1.886699in}}{\pgfqpoint{4.151869in}{1.891089in}}{\pgfqpoint{4.159682in}{1.898903in}}%
\pgfpathcurveto{\pgfqpoint{4.167496in}{1.906716in}}{\pgfqpoint{4.171886in}{1.917315in}}{\pgfqpoint{4.171886in}{1.928365in}}%
\pgfpathcurveto{\pgfqpoint{4.171886in}{1.939416in}}{\pgfqpoint{4.167496in}{1.950015in}}{\pgfqpoint{4.159682in}{1.957828in}}%
\pgfpathcurveto{\pgfqpoint{4.151869in}{1.965642in}}{\pgfqpoint{4.141270in}{1.970032in}}{\pgfqpoint{4.130219in}{1.970032in}}%
\pgfpathcurveto{\pgfqpoint{4.119169in}{1.970032in}}{\pgfqpoint{4.108570in}{1.965642in}}{\pgfqpoint{4.100757in}{1.957828in}}%
\pgfpathcurveto{\pgfqpoint{4.092943in}{1.950015in}}{\pgfqpoint{4.088553in}{1.939416in}}{\pgfqpoint{4.088553in}{1.928365in}}%
\pgfpathcurveto{\pgfqpoint{4.088553in}{1.917315in}}{\pgfqpoint{4.092943in}{1.906716in}}{\pgfqpoint{4.100757in}{1.898903in}}%
\pgfpathcurveto{\pgfqpoint{4.108570in}{1.891089in}}{\pgfqpoint{4.119169in}{1.886699in}}{\pgfqpoint{4.130219in}{1.886699in}}%
\pgfpathlineto{\pgfqpoint{4.130219in}{1.886699in}}%
\pgfpathclose%
\pgfusepath{stroke}%
\end{pgfscope}%
\begin{pgfscope}%
\pgfpathrectangle{\pgfqpoint{0.393053in}{0.375000in}}{\pgfqpoint{6.356833in}{5.175000in}}%
\pgfusepath{clip}%
\pgfsetbuttcap%
\pgfsetroundjoin%
\pgfsetlinewidth{1.003750pt}%
\definecolor{currentstroke}{rgb}{0.827451,0.827451,0.827451}%
\pgfsetstrokecolor{currentstroke}%
\pgfsetdash{}{0pt}%
\pgfpathmoveto{\pgfqpoint{4.369789in}{1.803195in}}%
\pgfpathcurveto{\pgfqpoint{4.380839in}{1.803195in}}{\pgfqpoint{4.391439in}{1.807585in}}{\pgfqpoint{4.399252in}{1.815399in}}%
\pgfpathcurveto{\pgfqpoint{4.407066in}{1.823213in}}{\pgfqpoint{4.411456in}{1.833812in}}{\pgfqpoint{4.411456in}{1.844862in}}%
\pgfpathcurveto{\pgfqpoint{4.411456in}{1.855912in}}{\pgfqpoint{4.407066in}{1.866511in}}{\pgfqpoint{4.399252in}{1.874325in}}%
\pgfpathcurveto{\pgfqpoint{4.391439in}{1.882138in}}{\pgfqpoint{4.380839in}{1.886528in}}{\pgfqpoint{4.369789in}{1.886528in}}%
\pgfpathcurveto{\pgfqpoint{4.358739in}{1.886528in}}{\pgfqpoint{4.348140in}{1.882138in}}{\pgfqpoint{4.340327in}{1.874325in}}%
\pgfpathcurveto{\pgfqpoint{4.332513in}{1.866511in}}{\pgfqpoint{4.328123in}{1.855912in}}{\pgfqpoint{4.328123in}{1.844862in}}%
\pgfpathcurveto{\pgfqpoint{4.328123in}{1.833812in}}{\pgfqpoint{4.332513in}{1.823213in}}{\pgfqpoint{4.340327in}{1.815399in}}%
\pgfpathcurveto{\pgfqpoint{4.348140in}{1.807585in}}{\pgfqpoint{4.358739in}{1.803195in}}{\pgfqpoint{4.369789in}{1.803195in}}%
\pgfpathlineto{\pgfqpoint{4.369789in}{1.803195in}}%
\pgfpathclose%
\pgfusepath{stroke}%
\end{pgfscope}%
\begin{pgfscope}%
\pgfpathrectangle{\pgfqpoint{0.393053in}{0.375000in}}{\pgfqpoint{6.356833in}{5.175000in}}%
\pgfusepath{clip}%
\pgfsetbuttcap%
\pgfsetroundjoin%
\pgfsetlinewidth{1.003750pt}%
\definecolor{currentstroke}{rgb}{0.827451,0.827451,0.827451}%
\pgfsetstrokecolor{currentstroke}%
\pgfsetdash{}{0pt}%
\pgfpathmoveto{\pgfqpoint{0.522483in}{3.797326in}}%
\pgfpathcurveto{\pgfqpoint{0.533533in}{3.797326in}}{\pgfqpoint{0.544133in}{3.801716in}}{\pgfqpoint{0.551946in}{3.809530in}}%
\pgfpathcurveto{\pgfqpoint{0.559760in}{3.817344in}}{\pgfqpoint{0.564150in}{3.827943in}}{\pgfqpoint{0.564150in}{3.838993in}}%
\pgfpathcurveto{\pgfqpoint{0.564150in}{3.850043in}}{\pgfqpoint{0.559760in}{3.860642in}}{\pgfqpoint{0.551946in}{3.868456in}}%
\pgfpathcurveto{\pgfqpoint{0.544133in}{3.876269in}}{\pgfqpoint{0.533533in}{3.880660in}}{\pgfqpoint{0.522483in}{3.880660in}}%
\pgfpathcurveto{\pgfqpoint{0.511433in}{3.880660in}}{\pgfqpoint{0.500834in}{3.876269in}}{\pgfqpoint{0.493021in}{3.868456in}}%
\pgfpathcurveto{\pgfqpoint{0.485207in}{3.860642in}}{\pgfqpoint{0.480817in}{3.850043in}}{\pgfqpoint{0.480817in}{3.838993in}}%
\pgfpathcurveto{\pgfqpoint{0.480817in}{3.827943in}}{\pgfqpoint{0.485207in}{3.817344in}}{\pgfqpoint{0.493021in}{3.809530in}}%
\pgfpathcurveto{\pgfqpoint{0.500834in}{3.801716in}}{\pgfqpoint{0.511433in}{3.797326in}}{\pgfqpoint{0.522483in}{3.797326in}}%
\pgfpathlineto{\pgfqpoint{0.522483in}{3.797326in}}%
\pgfpathclose%
\pgfusepath{stroke}%
\end{pgfscope}%
\begin{pgfscope}%
\pgfpathrectangle{\pgfqpoint{0.393053in}{0.375000in}}{\pgfqpoint{6.356833in}{5.175000in}}%
\pgfusepath{clip}%
\pgfsetbuttcap%
\pgfsetroundjoin%
\pgfsetlinewidth{1.003750pt}%
\definecolor{currentstroke}{rgb}{0.827451,0.827451,0.827451}%
\pgfsetstrokecolor{currentstroke}%
\pgfsetdash{}{0pt}%
\pgfpathmoveto{\pgfqpoint{0.851347in}{3.094066in}}%
\pgfpathcurveto{\pgfqpoint{0.862397in}{3.094066in}}{\pgfqpoint{0.872996in}{3.098456in}}{\pgfqpoint{0.880810in}{3.106269in}}%
\pgfpathcurveto{\pgfqpoint{0.888623in}{3.114083in}}{\pgfqpoint{0.893014in}{3.124682in}}{\pgfqpoint{0.893014in}{3.135732in}}%
\pgfpathcurveto{\pgfqpoint{0.893014in}{3.146782in}}{\pgfqpoint{0.888623in}{3.157381in}}{\pgfqpoint{0.880810in}{3.165195in}}%
\pgfpathcurveto{\pgfqpoint{0.872996in}{3.173009in}}{\pgfqpoint{0.862397in}{3.177399in}}{\pgfqpoint{0.851347in}{3.177399in}}%
\pgfpathcurveto{\pgfqpoint{0.840297in}{3.177399in}}{\pgfqpoint{0.829698in}{3.173009in}}{\pgfqpoint{0.821884in}{3.165195in}}%
\pgfpathcurveto{\pgfqpoint{0.814071in}{3.157381in}}{\pgfqpoint{0.809680in}{3.146782in}}{\pgfqpoint{0.809680in}{3.135732in}}%
\pgfpathcurveto{\pgfqpoint{0.809680in}{3.124682in}}{\pgfqpoint{0.814071in}{3.114083in}}{\pgfqpoint{0.821884in}{3.106269in}}%
\pgfpathcurveto{\pgfqpoint{0.829698in}{3.098456in}}{\pgfqpoint{0.840297in}{3.094066in}}{\pgfqpoint{0.851347in}{3.094066in}}%
\pgfpathlineto{\pgfqpoint{0.851347in}{3.094066in}}%
\pgfpathclose%
\pgfusepath{stroke}%
\end{pgfscope}%
\begin{pgfscope}%
\pgfpathrectangle{\pgfqpoint{0.393053in}{0.375000in}}{\pgfqpoint{6.356833in}{5.175000in}}%
\pgfusepath{clip}%
\pgfsetbuttcap%
\pgfsetroundjoin%
\pgfsetlinewidth{1.003750pt}%
\definecolor{currentstroke}{rgb}{0.827451,0.827451,0.827451}%
\pgfsetstrokecolor{currentstroke}%
\pgfsetdash{}{0pt}%
\pgfpathmoveto{\pgfqpoint{1.046962in}{2.844710in}}%
\pgfpathcurveto{\pgfqpoint{1.058012in}{2.844710in}}{\pgfqpoint{1.068611in}{2.849100in}}{\pgfqpoint{1.076425in}{2.856914in}}%
\pgfpathcurveto{\pgfqpoint{1.084238in}{2.864727in}}{\pgfqpoint{1.088629in}{2.875326in}}{\pgfqpoint{1.088629in}{2.886376in}}%
\pgfpathcurveto{\pgfqpoint{1.088629in}{2.897427in}}{\pgfqpoint{1.084238in}{2.908026in}}{\pgfqpoint{1.076425in}{2.915839in}}%
\pgfpathcurveto{\pgfqpoint{1.068611in}{2.923653in}}{\pgfqpoint{1.058012in}{2.928043in}}{\pgfqpoint{1.046962in}{2.928043in}}%
\pgfpathcurveto{\pgfqpoint{1.035912in}{2.928043in}}{\pgfqpoint{1.025313in}{2.923653in}}{\pgfqpoint{1.017499in}{2.915839in}}%
\pgfpathcurveto{\pgfqpoint{1.009686in}{2.908026in}}{\pgfqpoint{1.005295in}{2.897427in}}{\pgfqpoint{1.005295in}{2.886376in}}%
\pgfpathcurveto{\pgfqpoint{1.005295in}{2.875326in}}{\pgfqpoint{1.009686in}{2.864727in}}{\pgfqpoint{1.017499in}{2.856914in}}%
\pgfpathcurveto{\pgfqpoint{1.025313in}{2.849100in}}{\pgfqpoint{1.035912in}{2.844710in}}{\pgfqpoint{1.046962in}{2.844710in}}%
\pgfpathlineto{\pgfqpoint{1.046962in}{2.844710in}}%
\pgfpathclose%
\pgfusepath{stroke}%
\end{pgfscope}%
\begin{pgfscope}%
\pgfpathrectangle{\pgfqpoint{0.393053in}{0.375000in}}{\pgfqpoint{6.356833in}{5.175000in}}%
\pgfusepath{clip}%
\pgfsetbuttcap%
\pgfsetroundjoin%
\pgfsetlinewidth{1.003750pt}%
\definecolor{currentstroke}{rgb}{0.827451,0.827451,0.827451}%
\pgfsetstrokecolor{currentstroke}%
\pgfsetdash{}{0pt}%
\pgfpathmoveto{\pgfqpoint{0.901021in}{3.096155in}}%
\pgfpathcurveto{\pgfqpoint{0.912071in}{3.096155in}}{\pgfqpoint{0.922670in}{3.100545in}}{\pgfqpoint{0.930484in}{3.108359in}}%
\pgfpathcurveto{\pgfqpoint{0.938297in}{3.116173in}}{\pgfqpoint{0.942688in}{3.126772in}}{\pgfqpoint{0.942688in}{3.137822in}}%
\pgfpathcurveto{\pgfqpoint{0.942688in}{3.148872in}}{\pgfqpoint{0.938297in}{3.159471in}}{\pgfqpoint{0.930484in}{3.167285in}}%
\pgfpathcurveto{\pgfqpoint{0.922670in}{3.175098in}}{\pgfqpoint{0.912071in}{3.179488in}}{\pgfqpoint{0.901021in}{3.179488in}}%
\pgfpathcurveto{\pgfqpoint{0.889971in}{3.179488in}}{\pgfqpoint{0.879372in}{3.175098in}}{\pgfqpoint{0.871558in}{3.167285in}}%
\pgfpathcurveto{\pgfqpoint{0.863745in}{3.159471in}}{\pgfqpoint{0.859354in}{3.148872in}}{\pgfqpoint{0.859354in}{3.137822in}}%
\pgfpathcurveto{\pgfqpoint{0.859354in}{3.126772in}}{\pgfqpoint{0.863745in}{3.116173in}}{\pgfqpoint{0.871558in}{3.108359in}}%
\pgfpathcurveto{\pgfqpoint{0.879372in}{3.100545in}}{\pgfqpoint{0.889971in}{3.096155in}}{\pgfqpoint{0.901021in}{3.096155in}}%
\pgfpathlineto{\pgfqpoint{0.901021in}{3.096155in}}%
\pgfpathclose%
\pgfusepath{stroke}%
\end{pgfscope}%
\begin{pgfscope}%
\pgfpathrectangle{\pgfqpoint{0.393053in}{0.375000in}}{\pgfqpoint{6.356833in}{5.175000in}}%
\pgfusepath{clip}%
\pgfsetbuttcap%
\pgfsetroundjoin%
\pgfsetlinewidth{1.003750pt}%
\definecolor{currentstroke}{rgb}{0.827451,0.827451,0.827451}%
\pgfsetstrokecolor{currentstroke}%
\pgfsetdash{}{0pt}%
\pgfpathmoveto{\pgfqpoint{1.053407in}{2.878421in}}%
\pgfpathcurveto{\pgfqpoint{1.064458in}{2.878421in}}{\pgfqpoint{1.075057in}{2.882811in}}{\pgfqpoint{1.082870in}{2.890625in}}%
\pgfpathcurveto{\pgfqpoint{1.090684in}{2.898439in}}{\pgfqpoint{1.095074in}{2.909038in}}{\pgfqpoint{1.095074in}{2.920088in}}%
\pgfpathcurveto{\pgfqpoint{1.095074in}{2.931138in}}{\pgfqpoint{1.090684in}{2.941737in}}{\pgfqpoint{1.082870in}{2.949551in}}%
\pgfpathcurveto{\pgfqpoint{1.075057in}{2.957364in}}{\pgfqpoint{1.064458in}{2.961755in}}{\pgfqpoint{1.053407in}{2.961755in}}%
\pgfpathcurveto{\pgfqpoint{1.042357in}{2.961755in}}{\pgfqpoint{1.031758in}{2.957364in}}{\pgfqpoint{1.023945in}{2.949551in}}%
\pgfpathcurveto{\pgfqpoint{1.016131in}{2.941737in}}{\pgfqpoint{1.011741in}{2.931138in}}{\pgfqpoint{1.011741in}{2.920088in}}%
\pgfpathcurveto{\pgfqpoint{1.011741in}{2.909038in}}{\pgfqpoint{1.016131in}{2.898439in}}{\pgfqpoint{1.023945in}{2.890625in}}%
\pgfpathcurveto{\pgfqpoint{1.031758in}{2.882811in}}{\pgfqpoint{1.042357in}{2.878421in}}{\pgfqpoint{1.053407in}{2.878421in}}%
\pgfpathlineto{\pgfqpoint{1.053407in}{2.878421in}}%
\pgfpathclose%
\pgfusepath{stroke}%
\end{pgfscope}%
\begin{pgfscope}%
\pgfpathrectangle{\pgfqpoint{0.393053in}{0.375000in}}{\pgfqpoint{6.356833in}{5.175000in}}%
\pgfusepath{clip}%
\pgfsetbuttcap%
\pgfsetroundjoin%
\pgfsetlinewidth{1.003750pt}%
\definecolor{currentstroke}{rgb}{0.827451,0.827451,0.827451}%
\pgfsetstrokecolor{currentstroke}%
\pgfsetdash{}{0pt}%
\pgfpathmoveto{\pgfqpoint{1.737782in}{2.203176in}}%
\pgfpathcurveto{\pgfqpoint{1.748832in}{2.203176in}}{\pgfqpoint{1.759431in}{2.207566in}}{\pgfqpoint{1.767245in}{2.215380in}}%
\pgfpathcurveto{\pgfqpoint{1.775058in}{2.223193in}}{\pgfqpoint{1.779449in}{2.233792in}}{\pgfqpoint{1.779449in}{2.244843in}}%
\pgfpathcurveto{\pgfqpoint{1.779449in}{2.255893in}}{\pgfqpoint{1.775058in}{2.266492in}}{\pgfqpoint{1.767245in}{2.274305in}}%
\pgfpathcurveto{\pgfqpoint{1.759431in}{2.282119in}}{\pgfqpoint{1.748832in}{2.286509in}}{\pgfqpoint{1.737782in}{2.286509in}}%
\pgfpathcurveto{\pgfqpoint{1.726732in}{2.286509in}}{\pgfqpoint{1.716133in}{2.282119in}}{\pgfqpoint{1.708319in}{2.274305in}}%
\pgfpathcurveto{\pgfqpoint{1.700506in}{2.266492in}}{\pgfqpoint{1.696115in}{2.255893in}}{\pgfqpoint{1.696115in}{2.244843in}}%
\pgfpathcurveto{\pgfqpoint{1.696115in}{2.233792in}}{\pgfqpoint{1.700506in}{2.223193in}}{\pgfqpoint{1.708319in}{2.215380in}}%
\pgfpathcurveto{\pgfqpoint{1.716133in}{2.207566in}}{\pgfqpoint{1.726732in}{2.203176in}}{\pgfqpoint{1.737782in}{2.203176in}}%
\pgfpathlineto{\pgfqpoint{1.737782in}{2.203176in}}%
\pgfpathclose%
\pgfusepath{stroke}%
\end{pgfscope}%
\begin{pgfscope}%
\pgfpathrectangle{\pgfqpoint{0.393053in}{0.375000in}}{\pgfqpoint{6.356833in}{5.175000in}}%
\pgfusepath{clip}%
\pgfsetbuttcap%
\pgfsetroundjoin%
\pgfsetlinewidth{1.003750pt}%
\definecolor{currentstroke}{rgb}{0.827451,0.827451,0.827451}%
\pgfsetstrokecolor{currentstroke}%
\pgfsetdash{}{0pt}%
\pgfpathmoveto{\pgfqpoint{1.553604in}{2.718990in}}%
\pgfpathcurveto{\pgfqpoint{1.564654in}{2.718990in}}{\pgfqpoint{1.575253in}{2.723381in}}{\pgfqpoint{1.583066in}{2.731194in}}%
\pgfpathcurveto{\pgfqpoint{1.590880in}{2.739008in}}{\pgfqpoint{1.595270in}{2.749607in}}{\pgfqpoint{1.595270in}{2.760657in}}%
\pgfpathcurveto{\pgfqpoint{1.595270in}{2.771707in}}{\pgfqpoint{1.590880in}{2.782306in}}{\pgfqpoint{1.583066in}{2.790120in}}%
\pgfpathcurveto{\pgfqpoint{1.575253in}{2.797934in}}{\pgfqpoint{1.564654in}{2.802324in}}{\pgfqpoint{1.553604in}{2.802324in}}%
\pgfpathcurveto{\pgfqpoint{1.542553in}{2.802324in}}{\pgfqpoint{1.531954in}{2.797934in}}{\pgfqpoint{1.524141in}{2.790120in}}%
\pgfpathcurveto{\pgfqpoint{1.516327in}{2.782306in}}{\pgfqpoint{1.511937in}{2.771707in}}{\pgfqpoint{1.511937in}{2.760657in}}%
\pgfpathcurveto{\pgfqpoint{1.511937in}{2.749607in}}{\pgfqpoint{1.516327in}{2.739008in}}{\pgfqpoint{1.524141in}{2.731194in}}%
\pgfpathcurveto{\pgfqpoint{1.531954in}{2.723381in}}{\pgfqpoint{1.542553in}{2.718990in}}{\pgfqpoint{1.553604in}{2.718990in}}%
\pgfpathlineto{\pgfqpoint{1.553604in}{2.718990in}}%
\pgfpathclose%
\pgfusepath{stroke}%
\end{pgfscope}%
\begin{pgfscope}%
\pgfpathrectangle{\pgfqpoint{0.393053in}{0.375000in}}{\pgfqpoint{6.356833in}{5.175000in}}%
\pgfusepath{clip}%
\pgfsetbuttcap%
\pgfsetroundjoin%
\pgfsetlinewidth{1.003750pt}%
\definecolor{currentstroke}{rgb}{0.827451,0.827451,0.827451}%
\pgfsetstrokecolor{currentstroke}%
\pgfsetdash{}{0pt}%
\pgfpathmoveto{\pgfqpoint{1.092331in}{2.381241in}}%
\pgfpathcurveto{\pgfqpoint{1.103381in}{2.381241in}}{\pgfqpoint{1.113980in}{2.385631in}}{\pgfqpoint{1.121794in}{2.393444in}}%
\pgfpathcurveto{\pgfqpoint{1.129608in}{2.401258in}}{\pgfqpoint{1.133998in}{2.411857in}}{\pgfqpoint{1.133998in}{2.422907in}}%
\pgfpathcurveto{\pgfqpoint{1.133998in}{2.433957in}}{\pgfqpoint{1.129608in}{2.444556in}}{\pgfqpoint{1.121794in}{2.452370in}}%
\pgfpathcurveto{\pgfqpoint{1.113980in}{2.460184in}}{\pgfqpoint{1.103381in}{2.464574in}}{\pgfqpoint{1.092331in}{2.464574in}}%
\pgfpathcurveto{\pgfqpoint{1.081281in}{2.464574in}}{\pgfqpoint{1.070682in}{2.460184in}}{\pgfqpoint{1.062868in}{2.452370in}}%
\pgfpathcurveto{\pgfqpoint{1.055055in}{2.444556in}}{\pgfqpoint{1.050664in}{2.433957in}}{\pgfqpoint{1.050664in}{2.422907in}}%
\pgfpathcurveto{\pgfqpoint{1.050664in}{2.411857in}}{\pgfqpoint{1.055055in}{2.401258in}}{\pgfqpoint{1.062868in}{2.393444in}}%
\pgfpathcurveto{\pgfqpoint{1.070682in}{2.385631in}}{\pgfqpoint{1.081281in}{2.381241in}}{\pgfqpoint{1.092331in}{2.381241in}}%
\pgfpathlineto{\pgfqpoint{1.092331in}{2.381241in}}%
\pgfpathclose%
\pgfusepath{stroke}%
\end{pgfscope}%
\begin{pgfscope}%
\pgfpathrectangle{\pgfqpoint{0.393053in}{0.375000in}}{\pgfqpoint{6.356833in}{5.175000in}}%
\pgfusepath{clip}%
\pgfsetbuttcap%
\pgfsetroundjoin%
\pgfsetlinewidth{1.003750pt}%
\definecolor{currentstroke}{rgb}{0.827451,0.827451,0.827451}%
\pgfsetstrokecolor{currentstroke}%
\pgfsetdash{}{0pt}%
\pgfpathmoveto{\pgfqpoint{2.036490in}{1.340063in}}%
\pgfpathcurveto{\pgfqpoint{2.047540in}{1.340063in}}{\pgfqpoint{2.058139in}{1.344454in}}{\pgfqpoint{2.065953in}{1.352267in}}%
\pgfpathcurveto{\pgfqpoint{2.073767in}{1.360081in}}{\pgfqpoint{2.078157in}{1.370680in}}{\pgfqpoint{2.078157in}{1.381730in}}%
\pgfpathcurveto{\pgfqpoint{2.078157in}{1.392780in}}{\pgfqpoint{2.073767in}{1.403379in}}{\pgfqpoint{2.065953in}{1.411193in}}%
\pgfpathcurveto{\pgfqpoint{2.058139in}{1.419007in}}{\pgfqpoint{2.047540in}{1.423397in}}{\pgfqpoint{2.036490in}{1.423397in}}%
\pgfpathcurveto{\pgfqpoint{2.025440in}{1.423397in}}{\pgfqpoint{2.014841in}{1.419007in}}{\pgfqpoint{2.007027in}{1.411193in}}%
\pgfpathcurveto{\pgfqpoint{1.999214in}{1.403379in}}{\pgfqpoint{1.994824in}{1.392780in}}{\pgfqpoint{1.994824in}{1.381730in}}%
\pgfpathcurveto{\pgfqpoint{1.994824in}{1.370680in}}{\pgfqpoint{1.999214in}{1.360081in}}{\pgfqpoint{2.007027in}{1.352267in}}%
\pgfpathcurveto{\pgfqpoint{2.014841in}{1.344454in}}{\pgfqpoint{2.025440in}{1.340063in}}{\pgfqpoint{2.036490in}{1.340063in}}%
\pgfpathlineto{\pgfqpoint{2.036490in}{1.340063in}}%
\pgfpathclose%
\pgfusepath{stroke}%
\end{pgfscope}%
\begin{pgfscope}%
\pgfpathrectangle{\pgfqpoint{0.393053in}{0.375000in}}{\pgfqpoint{6.356833in}{5.175000in}}%
\pgfusepath{clip}%
\pgfsetbuttcap%
\pgfsetroundjoin%
\pgfsetlinewidth{1.003750pt}%
\definecolor{currentstroke}{rgb}{0.827451,0.827451,0.827451}%
\pgfsetstrokecolor{currentstroke}%
\pgfsetdash{}{0pt}%
\pgfpathmoveto{\pgfqpoint{1.616394in}{1.795125in}}%
\pgfpathcurveto{\pgfqpoint{1.627444in}{1.795125in}}{\pgfqpoint{1.638043in}{1.799515in}}{\pgfqpoint{1.645857in}{1.807329in}}%
\pgfpathcurveto{\pgfqpoint{1.653671in}{1.815142in}}{\pgfqpoint{1.658061in}{1.825742in}}{\pgfqpoint{1.658061in}{1.836792in}}%
\pgfpathcurveto{\pgfqpoint{1.658061in}{1.847842in}}{\pgfqpoint{1.653671in}{1.858441in}}{\pgfqpoint{1.645857in}{1.866254in}}%
\pgfpathcurveto{\pgfqpoint{1.638043in}{1.874068in}}{\pgfqpoint{1.627444in}{1.878458in}}{\pgfqpoint{1.616394in}{1.878458in}}%
\pgfpathcurveto{\pgfqpoint{1.605344in}{1.878458in}}{\pgfqpoint{1.594745in}{1.874068in}}{\pgfqpoint{1.586931in}{1.866254in}}%
\pgfpathcurveto{\pgfqpoint{1.579118in}{1.858441in}}{\pgfqpoint{1.574728in}{1.847842in}}{\pgfqpoint{1.574728in}{1.836792in}}%
\pgfpathcurveto{\pgfqpoint{1.574728in}{1.825742in}}{\pgfqpoint{1.579118in}{1.815142in}}{\pgfqpoint{1.586931in}{1.807329in}}%
\pgfpathcurveto{\pgfqpoint{1.594745in}{1.799515in}}{\pgfqpoint{1.605344in}{1.795125in}}{\pgfqpoint{1.616394in}{1.795125in}}%
\pgfpathlineto{\pgfqpoint{1.616394in}{1.795125in}}%
\pgfpathclose%
\pgfusepath{stroke}%
\end{pgfscope}%
\begin{pgfscope}%
\pgfpathrectangle{\pgfqpoint{0.393053in}{0.375000in}}{\pgfqpoint{6.356833in}{5.175000in}}%
\pgfusepath{clip}%
\pgfsetbuttcap%
\pgfsetroundjoin%
\pgfsetlinewidth{1.003750pt}%
\definecolor{currentstroke}{rgb}{0.827451,0.827451,0.827451}%
\pgfsetstrokecolor{currentstroke}%
\pgfsetdash{}{0pt}%
\pgfpathmoveto{\pgfqpoint{1.388153in}{1.896293in}}%
\pgfpathcurveto{\pgfqpoint{1.399204in}{1.896293in}}{\pgfqpoint{1.409803in}{1.900683in}}{\pgfqpoint{1.417616in}{1.908497in}}%
\pgfpathcurveto{\pgfqpoint{1.425430in}{1.916310in}}{\pgfqpoint{1.429820in}{1.926909in}}{\pgfqpoint{1.429820in}{1.937959in}}%
\pgfpathcurveto{\pgfqpoint{1.429820in}{1.949010in}}{\pgfqpoint{1.425430in}{1.959609in}}{\pgfqpoint{1.417616in}{1.967422in}}%
\pgfpathcurveto{\pgfqpoint{1.409803in}{1.975236in}}{\pgfqpoint{1.399204in}{1.979626in}}{\pgfqpoint{1.388153in}{1.979626in}}%
\pgfpathcurveto{\pgfqpoint{1.377103in}{1.979626in}}{\pgfqpoint{1.366504in}{1.975236in}}{\pgfqpoint{1.358691in}{1.967422in}}%
\pgfpathcurveto{\pgfqpoint{1.350877in}{1.959609in}}{\pgfqpoint{1.346487in}{1.949010in}}{\pgfqpoint{1.346487in}{1.937959in}}%
\pgfpathcurveto{\pgfqpoint{1.346487in}{1.926909in}}{\pgfqpoint{1.350877in}{1.916310in}}{\pgfqpoint{1.358691in}{1.908497in}}%
\pgfpathcurveto{\pgfqpoint{1.366504in}{1.900683in}}{\pgfqpoint{1.377103in}{1.896293in}}{\pgfqpoint{1.388153in}{1.896293in}}%
\pgfpathlineto{\pgfqpoint{1.388153in}{1.896293in}}%
\pgfpathclose%
\pgfusepath{stroke}%
\end{pgfscope}%
\begin{pgfscope}%
\pgfpathrectangle{\pgfqpoint{0.393053in}{0.375000in}}{\pgfqpoint{6.356833in}{5.175000in}}%
\pgfusepath{clip}%
\pgfsetbuttcap%
\pgfsetroundjoin%
\pgfsetlinewidth{1.003750pt}%
\definecolor{currentstroke}{rgb}{0.827451,0.827451,0.827451}%
\pgfsetstrokecolor{currentstroke}%
\pgfsetdash{}{0pt}%
\pgfpathmoveto{\pgfqpoint{3.880256in}{0.594158in}}%
\pgfpathcurveto{\pgfqpoint{3.891307in}{0.594158in}}{\pgfqpoint{3.901906in}{0.598548in}}{\pgfqpoint{3.909719in}{0.606362in}}%
\pgfpathcurveto{\pgfqpoint{3.917533in}{0.614176in}}{\pgfqpoint{3.921923in}{0.624775in}}{\pgfqpoint{3.921923in}{0.635825in}}%
\pgfpathcurveto{\pgfqpoint{3.921923in}{0.646875in}}{\pgfqpoint{3.917533in}{0.657474in}}{\pgfqpoint{3.909719in}{0.665288in}}%
\pgfpathcurveto{\pgfqpoint{3.901906in}{0.673101in}}{\pgfqpoint{3.891307in}{0.677492in}}{\pgfqpoint{3.880256in}{0.677492in}}%
\pgfpathcurveto{\pgfqpoint{3.869206in}{0.677492in}}{\pgfqpoint{3.858607in}{0.673101in}}{\pgfqpoint{3.850794in}{0.665288in}}%
\pgfpathcurveto{\pgfqpoint{3.842980in}{0.657474in}}{\pgfqpoint{3.838590in}{0.646875in}}{\pgfqpoint{3.838590in}{0.635825in}}%
\pgfpathcurveto{\pgfqpoint{3.838590in}{0.624775in}}{\pgfqpoint{3.842980in}{0.614176in}}{\pgfqpoint{3.850794in}{0.606362in}}%
\pgfpathcurveto{\pgfqpoint{3.858607in}{0.598548in}}{\pgfqpoint{3.869206in}{0.594158in}}{\pgfqpoint{3.880256in}{0.594158in}}%
\pgfpathlineto{\pgfqpoint{3.880256in}{0.594158in}}%
\pgfpathclose%
\pgfusepath{stroke}%
\end{pgfscope}%
\begin{pgfscope}%
\pgfpathrectangle{\pgfqpoint{0.393053in}{0.375000in}}{\pgfqpoint{6.356833in}{5.175000in}}%
\pgfusepath{clip}%
\pgfsetbuttcap%
\pgfsetroundjoin%
\pgfsetlinewidth{1.003750pt}%
\definecolor{currentstroke}{rgb}{0.827451,0.827451,0.827451}%
\pgfsetstrokecolor{currentstroke}%
\pgfsetdash{}{0pt}%
\pgfpathmoveto{\pgfqpoint{1.463696in}{1.814748in}}%
\pgfpathcurveto{\pgfqpoint{1.474746in}{1.814748in}}{\pgfqpoint{1.485345in}{1.819138in}}{\pgfqpoint{1.493159in}{1.826952in}}%
\pgfpathcurveto{\pgfqpoint{1.500972in}{1.834766in}}{\pgfqpoint{1.505362in}{1.845365in}}{\pgfqpoint{1.505362in}{1.856415in}}%
\pgfpathcurveto{\pgfqpoint{1.505362in}{1.867465in}}{\pgfqpoint{1.500972in}{1.878064in}}{\pgfqpoint{1.493159in}{1.885878in}}%
\pgfpathcurveto{\pgfqpoint{1.485345in}{1.893691in}}{\pgfqpoint{1.474746in}{1.898081in}}{\pgfqpoint{1.463696in}{1.898081in}}%
\pgfpathcurveto{\pgfqpoint{1.452646in}{1.898081in}}{\pgfqpoint{1.442047in}{1.893691in}}{\pgfqpoint{1.434233in}{1.885878in}}%
\pgfpathcurveto{\pgfqpoint{1.426419in}{1.878064in}}{\pgfqpoint{1.422029in}{1.867465in}}{\pgfqpoint{1.422029in}{1.856415in}}%
\pgfpathcurveto{\pgfqpoint{1.422029in}{1.845365in}}{\pgfqpoint{1.426419in}{1.834766in}}{\pgfqpoint{1.434233in}{1.826952in}}%
\pgfpathcurveto{\pgfqpoint{1.442047in}{1.819138in}}{\pgfqpoint{1.452646in}{1.814748in}}{\pgfqpoint{1.463696in}{1.814748in}}%
\pgfpathlineto{\pgfqpoint{1.463696in}{1.814748in}}%
\pgfpathclose%
\pgfusepath{stroke}%
\end{pgfscope}%
\begin{pgfscope}%
\pgfpathrectangle{\pgfqpoint{0.393053in}{0.375000in}}{\pgfqpoint{6.356833in}{5.175000in}}%
\pgfusepath{clip}%
\pgfsetbuttcap%
\pgfsetroundjoin%
\pgfsetlinewidth{1.003750pt}%
\definecolor{currentstroke}{rgb}{0.827451,0.827451,0.827451}%
\pgfsetstrokecolor{currentstroke}%
\pgfsetdash{}{0pt}%
\pgfpathmoveto{\pgfqpoint{1.909651in}{1.444209in}}%
\pgfpathcurveto{\pgfqpoint{1.920701in}{1.444209in}}{\pgfqpoint{1.931300in}{1.448599in}}{\pgfqpoint{1.939114in}{1.456413in}}%
\pgfpathcurveto{\pgfqpoint{1.946927in}{1.464226in}}{\pgfqpoint{1.951318in}{1.474825in}}{\pgfqpoint{1.951318in}{1.485875in}}%
\pgfpathcurveto{\pgfqpoint{1.951318in}{1.496926in}}{\pgfqpoint{1.946927in}{1.507525in}}{\pgfqpoint{1.939114in}{1.515338in}}%
\pgfpathcurveto{\pgfqpoint{1.931300in}{1.523152in}}{\pgfqpoint{1.920701in}{1.527542in}}{\pgfqpoint{1.909651in}{1.527542in}}%
\pgfpathcurveto{\pgfqpoint{1.898601in}{1.527542in}}{\pgfqpoint{1.888002in}{1.523152in}}{\pgfqpoint{1.880188in}{1.515338in}}%
\pgfpathcurveto{\pgfqpoint{1.872375in}{1.507525in}}{\pgfqpoint{1.867984in}{1.496926in}}{\pgfqpoint{1.867984in}{1.485875in}}%
\pgfpathcurveto{\pgfqpoint{1.867984in}{1.474825in}}{\pgfqpoint{1.872375in}{1.464226in}}{\pgfqpoint{1.880188in}{1.456413in}}%
\pgfpathcurveto{\pgfqpoint{1.888002in}{1.448599in}}{\pgfqpoint{1.898601in}{1.444209in}}{\pgfqpoint{1.909651in}{1.444209in}}%
\pgfpathlineto{\pgfqpoint{1.909651in}{1.444209in}}%
\pgfpathclose%
\pgfusepath{stroke}%
\end{pgfscope}%
\begin{pgfscope}%
\pgfpathrectangle{\pgfqpoint{0.393053in}{0.375000in}}{\pgfqpoint{6.356833in}{5.175000in}}%
\pgfusepath{clip}%
\pgfsetbuttcap%
\pgfsetroundjoin%
\pgfsetlinewidth{1.003750pt}%
\definecolor{currentstroke}{rgb}{0.827451,0.827451,0.827451}%
\pgfsetstrokecolor{currentstroke}%
\pgfsetdash{}{0pt}%
\pgfpathmoveto{\pgfqpoint{1.253491in}{2.045656in}}%
\pgfpathcurveto{\pgfqpoint{1.264541in}{2.045656in}}{\pgfqpoint{1.275140in}{2.050046in}}{\pgfqpoint{1.282953in}{2.057860in}}%
\pgfpathcurveto{\pgfqpoint{1.290767in}{2.065674in}}{\pgfqpoint{1.295157in}{2.076273in}}{\pgfqpoint{1.295157in}{2.087323in}}%
\pgfpathcurveto{\pgfqpoint{1.295157in}{2.098373in}}{\pgfqpoint{1.290767in}{2.108972in}}{\pgfqpoint{1.282953in}{2.116786in}}%
\pgfpathcurveto{\pgfqpoint{1.275140in}{2.124599in}}{\pgfqpoint{1.264541in}{2.128990in}}{\pgfqpoint{1.253491in}{2.128990in}}%
\pgfpathcurveto{\pgfqpoint{1.242440in}{2.128990in}}{\pgfqpoint{1.231841in}{2.124599in}}{\pgfqpoint{1.224028in}{2.116786in}}%
\pgfpathcurveto{\pgfqpoint{1.216214in}{2.108972in}}{\pgfqpoint{1.211824in}{2.098373in}}{\pgfqpoint{1.211824in}{2.087323in}}%
\pgfpathcurveto{\pgfqpoint{1.211824in}{2.076273in}}{\pgfqpoint{1.216214in}{2.065674in}}{\pgfqpoint{1.224028in}{2.057860in}}%
\pgfpathcurveto{\pgfqpoint{1.231841in}{2.050046in}}{\pgfqpoint{1.242440in}{2.045656in}}{\pgfqpoint{1.253491in}{2.045656in}}%
\pgfpathlineto{\pgfqpoint{1.253491in}{2.045656in}}%
\pgfpathclose%
\pgfusepath{stroke}%
\end{pgfscope}%
\begin{pgfscope}%
\pgfpathrectangle{\pgfqpoint{0.393053in}{0.375000in}}{\pgfqpoint{6.356833in}{5.175000in}}%
\pgfusepath{clip}%
\pgfsetbuttcap%
\pgfsetroundjoin%
\pgfsetlinewidth{1.003750pt}%
\definecolor{currentstroke}{rgb}{0.827451,0.827451,0.827451}%
\pgfsetstrokecolor{currentstroke}%
\pgfsetdash{}{0pt}%
\pgfpathmoveto{\pgfqpoint{2.352374in}{1.159706in}}%
\pgfpathcurveto{\pgfqpoint{2.363424in}{1.159706in}}{\pgfqpoint{2.374023in}{1.164096in}}{\pgfqpoint{2.381836in}{1.171910in}}%
\pgfpathcurveto{\pgfqpoint{2.389650in}{1.179724in}}{\pgfqpoint{2.394040in}{1.190323in}}{\pgfqpoint{2.394040in}{1.201373in}}%
\pgfpathcurveto{\pgfqpoint{2.394040in}{1.212423in}}{\pgfqpoint{2.389650in}{1.223022in}}{\pgfqpoint{2.381836in}{1.230835in}}%
\pgfpathcurveto{\pgfqpoint{2.374023in}{1.238649in}}{\pgfqpoint{2.363424in}{1.243039in}}{\pgfqpoint{2.352374in}{1.243039in}}%
\pgfpathcurveto{\pgfqpoint{2.341324in}{1.243039in}}{\pgfqpoint{2.330725in}{1.238649in}}{\pgfqpoint{2.322911in}{1.230835in}}%
\pgfpathcurveto{\pgfqpoint{2.315097in}{1.223022in}}{\pgfqpoint{2.310707in}{1.212423in}}{\pgfqpoint{2.310707in}{1.201373in}}%
\pgfpathcurveto{\pgfqpoint{2.310707in}{1.190323in}}{\pgfqpoint{2.315097in}{1.179724in}}{\pgfqpoint{2.322911in}{1.171910in}}%
\pgfpathcurveto{\pgfqpoint{2.330725in}{1.164096in}}{\pgfqpoint{2.341324in}{1.159706in}}{\pgfqpoint{2.352374in}{1.159706in}}%
\pgfpathlineto{\pgfqpoint{2.352374in}{1.159706in}}%
\pgfpathclose%
\pgfusepath{stroke}%
\end{pgfscope}%
\begin{pgfscope}%
\pgfpathrectangle{\pgfqpoint{0.393053in}{0.375000in}}{\pgfqpoint{6.356833in}{5.175000in}}%
\pgfusepath{clip}%
\pgfsetbuttcap%
\pgfsetroundjoin%
\pgfsetlinewidth{1.003750pt}%
\definecolor{currentstroke}{rgb}{0.827451,0.827451,0.827451}%
\pgfsetstrokecolor{currentstroke}%
\pgfsetdash{}{0pt}%
\pgfpathmoveto{\pgfqpoint{0.455840in}{3.833482in}}%
\pgfpathcurveto{\pgfqpoint{0.466890in}{3.833482in}}{\pgfqpoint{0.477489in}{3.837873in}}{\pgfqpoint{0.485303in}{3.845686in}}%
\pgfpathcurveto{\pgfqpoint{0.493117in}{3.853500in}}{\pgfqpoint{0.497507in}{3.864099in}}{\pgfqpoint{0.497507in}{3.875149in}}%
\pgfpathcurveto{\pgfqpoint{0.497507in}{3.886199in}}{\pgfqpoint{0.493117in}{3.896798in}}{\pgfqpoint{0.485303in}{3.904612in}}%
\pgfpathcurveto{\pgfqpoint{0.477489in}{3.912425in}}{\pgfqpoint{0.466890in}{3.916816in}}{\pgfqpoint{0.455840in}{3.916816in}}%
\pgfpathcurveto{\pgfqpoint{0.444790in}{3.916816in}}{\pgfqpoint{0.434191in}{3.912425in}}{\pgfqpoint{0.426377in}{3.904612in}}%
\pgfpathcurveto{\pgfqpoint{0.418564in}{3.896798in}}{\pgfqpoint{0.414173in}{3.886199in}}{\pgfqpoint{0.414173in}{3.875149in}}%
\pgfpathcurveto{\pgfqpoint{0.414173in}{3.864099in}}{\pgfqpoint{0.418564in}{3.853500in}}{\pgfqpoint{0.426377in}{3.845686in}}%
\pgfpathcurveto{\pgfqpoint{0.434191in}{3.837873in}}{\pgfqpoint{0.444790in}{3.833482in}}{\pgfqpoint{0.455840in}{3.833482in}}%
\pgfpathlineto{\pgfqpoint{0.455840in}{3.833482in}}%
\pgfpathclose%
\pgfusepath{stroke}%
\end{pgfscope}%
\begin{pgfscope}%
\pgfpathrectangle{\pgfqpoint{0.393053in}{0.375000in}}{\pgfqpoint{6.356833in}{5.175000in}}%
\pgfusepath{clip}%
\pgfsetbuttcap%
\pgfsetroundjoin%
\pgfsetlinewidth{1.003750pt}%
\definecolor{currentstroke}{rgb}{0.827451,0.827451,0.827451}%
\pgfsetstrokecolor{currentstroke}%
\pgfsetdash{}{0pt}%
\pgfpathmoveto{\pgfqpoint{1.186674in}{2.128077in}}%
\pgfpathcurveto{\pgfqpoint{1.197724in}{2.128077in}}{\pgfqpoint{1.208323in}{2.132467in}}{\pgfqpoint{1.216137in}{2.140281in}}%
\pgfpathcurveto{\pgfqpoint{1.223950in}{2.148095in}}{\pgfqpoint{1.228341in}{2.158694in}}{\pgfqpoint{1.228341in}{2.169744in}}%
\pgfpathcurveto{\pgfqpoint{1.228341in}{2.180794in}}{\pgfqpoint{1.223950in}{2.191393in}}{\pgfqpoint{1.216137in}{2.199207in}}%
\pgfpathcurveto{\pgfqpoint{1.208323in}{2.207020in}}{\pgfqpoint{1.197724in}{2.211411in}}{\pgfqpoint{1.186674in}{2.211411in}}%
\pgfpathcurveto{\pgfqpoint{1.175624in}{2.211411in}}{\pgfqpoint{1.165025in}{2.207020in}}{\pgfqpoint{1.157211in}{2.199207in}}%
\pgfpathcurveto{\pgfqpoint{1.149397in}{2.191393in}}{\pgfqpoint{1.145007in}{2.180794in}}{\pgfqpoint{1.145007in}{2.169744in}}%
\pgfpathcurveto{\pgfqpoint{1.145007in}{2.158694in}}{\pgfqpoint{1.149397in}{2.148095in}}{\pgfqpoint{1.157211in}{2.140281in}}%
\pgfpathcurveto{\pgfqpoint{1.165025in}{2.132467in}}{\pgfqpoint{1.175624in}{2.128077in}}{\pgfqpoint{1.186674in}{2.128077in}}%
\pgfpathlineto{\pgfqpoint{1.186674in}{2.128077in}}%
\pgfpathclose%
\pgfusepath{stroke}%
\end{pgfscope}%
\begin{pgfscope}%
\pgfpathrectangle{\pgfqpoint{0.393053in}{0.375000in}}{\pgfqpoint{6.356833in}{5.175000in}}%
\pgfusepath{clip}%
\pgfsetbuttcap%
\pgfsetroundjoin%
\pgfsetlinewidth{1.003750pt}%
\definecolor{currentstroke}{rgb}{0.827451,0.827451,0.827451}%
\pgfsetstrokecolor{currentstroke}%
\pgfsetdash{}{0pt}%
\pgfpathmoveto{\pgfqpoint{3.097658in}{0.775315in}}%
\pgfpathcurveto{\pgfqpoint{3.108708in}{0.775315in}}{\pgfqpoint{3.119307in}{0.779705in}}{\pgfqpoint{3.127121in}{0.787519in}}%
\pgfpathcurveto{\pgfqpoint{3.134935in}{0.795332in}}{\pgfqpoint{3.139325in}{0.805931in}}{\pgfqpoint{3.139325in}{0.816981in}}%
\pgfpathcurveto{\pgfqpoint{3.139325in}{0.828032in}}{\pgfqpoint{3.134935in}{0.838631in}}{\pgfqpoint{3.127121in}{0.846444in}}%
\pgfpathcurveto{\pgfqpoint{3.119307in}{0.854258in}}{\pgfqpoint{3.108708in}{0.858648in}}{\pgfqpoint{3.097658in}{0.858648in}}%
\pgfpathcurveto{\pgfqpoint{3.086608in}{0.858648in}}{\pgfqpoint{3.076009in}{0.854258in}}{\pgfqpoint{3.068195in}{0.846444in}}%
\pgfpathcurveto{\pgfqpoint{3.060382in}{0.838631in}}{\pgfqpoint{3.055992in}{0.828032in}}{\pgfqpoint{3.055992in}{0.816981in}}%
\pgfpathcurveto{\pgfqpoint{3.055992in}{0.805931in}}{\pgfqpoint{3.060382in}{0.795332in}}{\pgfqpoint{3.068195in}{0.787519in}}%
\pgfpathcurveto{\pgfqpoint{3.076009in}{0.779705in}}{\pgfqpoint{3.086608in}{0.775315in}}{\pgfqpoint{3.097658in}{0.775315in}}%
\pgfpathlineto{\pgfqpoint{3.097658in}{0.775315in}}%
\pgfpathclose%
\pgfusepath{stroke}%
\end{pgfscope}%
\begin{pgfscope}%
\pgfpathrectangle{\pgfqpoint{0.393053in}{0.375000in}}{\pgfqpoint{6.356833in}{5.175000in}}%
\pgfusepath{clip}%
\pgfsetbuttcap%
\pgfsetroundjoin%
\pgfsetlinewidth{1.003750pt}%
\definecolor{currentstroke}{rgb}{0.827451,0.827451,0.827451}%
\pgfsetstrokecolor{currentstroke}%
\pgfsetdash{}{0pt}%
\pgfpathmoveto{\pgfqpoint{4.283632in}{0.443541in}}%
\pgfpathcurveto{\pgfqpoint{4.294682in}{0.443541in}}{\pgfqpoint{4.305281in}{0.447931in}}{\pgfqpoint{4.313095in}{0.455745in}}%
\pgfpathcurveto{\pgfqpoint{4.320908in}{0.463559in}}{\pgfqpoint{4.325299in}{0.474158in}}{\pgfqpoint{4.325299in}{0.485208in}}%
\pgfpathcurveto{\pgfqpoint{4.325299in}{0.496258in}}{\pgfqpoint{4.320908in}{0.506857in}}{\pgfqpoint{4.313095in}{0.514670in}}%
\pgfpathcurveto{\pgfqpoint{4.305281in}{0.522484in}}{\pgfqpoint{4.294682in}{0.526874in}}{\pgfqpoint{4.283632in}{0.526874in}}%
\pgfpathcurveto{\pgfqpoint{4.272582in}{0.526874in}}{\pgfqpoint{4.261983in}{0.522484in}}{\pgfqpoint{4.254169in}{0.514670in}}%
\pgfpathcurveto{\pgfqpoint{4.246356in}{0.506857in}}{\pgfqpoint{4.241965in}{0.496258in}}{\pgfqpoint{4.241965in}{0.485208in}}%
\pgfpathcurveto{\pgfqpoint{4.241965in}{0.474158in}}{\pgfqpoint{4.246356in}{0.463559in}}{\pgfqpoint{4.254169in}{0.455745in}}%
\pgfpathcurveto{\pgfqpoint{4.261983in}{0.447931in}}{\pgfqpoint{4.272582in}{0.443541in}}{\pgfqpoint{4.283632in}{0.443541in}}%
\pgfpathlineto{\pgfqpoint{4.283632in}{0.443541in}}%
\pgfpathclose%
\pgfusepath{stroke}%
\end{pgfscope}%
\begin{pgfscope}%
\pgfpathrectangle{\pgfqpoint{0.393053in}{0.375000in}}{\pgfqpoint{6.356833in}{5.175000in}}%
\pgfusepath{clip}%
\pgfsetbuttcap%
\pgfsetroundjoin%
\pgfsetlinewidth{1.003750pt}%
\definecolor{currentstroke}{rgb}{0.827451,0.827451,0.827451}%
\pgfsetstrokecolor{currentstroke}%
\pgfsetdash{}{0pt}%
\pgfpathmoveto{\pgfqpoint{2.582769in}{1.048231in}}%
\pgfpathcurveto{\pgfqpoint{2.593819in}{1.048231in}}{\pgfqpoint{2.604418in}{1.052621in}}{\pgfqpoint{2.612231in}{1.060435in}}%
\pgfpathcurveto{\pgfqpoint{2.620045in}{1.068248in}}{\pgfqpoint{2.624435in}{1.078848in}}{\pgfqpoint{2.624435in}{1.089898in}}%
\pgfpathcurveto{\pgfqpoint{2.624435in}{1.100948in}}{\pgfqpoint{2.620045in}{1.111547in}}{\pgfqpoint{2.612231in}{1.119360in}}%
\pgfpathcurveto{\pgfqpoint{2.604418in}{1.127174in}}{\pgfqpoint{2.593819in}{1.131564in}}{\pgfqpoint{2.582769in}{1.131564in}}%
\pgfpathcurveto{\pgfqpoint{2.571718in}{1.131564in}}{\pgfqpoint{2.561119in}{1.127174in}}{\pgfqpoint{2.553306in}{1.119360in}}%
\pgfpathcurveto{\pgfqpoint{2.545492in}{1.111547in}}{\pgfqpoint{2.541102in}{1.100948in}}{\pgfqpoint{2.541102in}{1.089898in}}%
\pgfpathcurveto{\pgfqpoint{2.541102in}{1.078848in}}{\pgfqpoint{2.545492in}{1.068248in}}{\pgfqpoint{2.553306in}{1.060435in}}%
\pgfpathcurveto{\pgfqpoint{2.561119in}{1.052621in}}{\pgfqpoint{2.571718in}{1.048231in}}{\pgfqpoint{2.582769in}{1.048231in}}%
\pgfpathlineto{\pgfqpoint{2.582769in}{1.048231in}}%
\pgfpathclose%
\pgfusepath{stroke}%
\end{pgfscope}%
\begin{pgfscope}%
\pgfpathrectangle{\pgfqpoint{0.393053in}{0.375000in}}{\pgfqpoint{6.356833in}{5.175000in}}%
\pgfusepath{clip}%
\pgfsetbuttcap%
\pgfsetroundjoin%
\pgfsetlinewidth{1.003750pt}%
\definecolor{currentstroke}{rgb}{0.827451,0.827451,0.827451}%
\pgfsetstrokecolor{currentstroke}%
\pgfsetdash{}{0pt}%
\pgfpathmoveto{\pgfqpoint{0.474470in}{3.727925in}}%
\pgfpathcurveto{\pgfqpoint{0.485520in}{3.727925in}}{\pgfqpoint{0.496119in}{3.732315in}}{\pgfqpoint{0.503933in}{3.740129in}}%
\pgfpathcurveto{\pgfqpoint{0.511746in}{3.747943in}}{\pgfqpoint{0.516137in}{3.758542in}}{\pgfqpoint{0.516137in}{3.769592in}}%
\pgfpathcurveto{\pgfqpoint{0.516137in}{3.780642in}}{\pgfqpoint{0.511746in}{3.791241in}}{\pgfqpoint{0.503933in}{3.799055in}}%
\pgfpathcurveto{\pgfqpoint{0.496119in}{3.806868in}}{\pgfqpoint{0.485520in}{3.811259in}}{\pgfqpoint{0.474470in}{3.811259in}}%
\pgfpathcurveto{\pgfqpoint{0.463420in}{3.811259in}}{\pgfqpoint{0.452821in}{3.806868in}}{\pgfqpoint{0.445007in}{3.799055in}}%
\pgfpathcurveto{\pgfqpoint{0.437194in}{3.791241in}}{\pgfqpoint{0.432803in}{3.780642in}}{\pgfqpoint{0.432803in}{3.769592in}}%
\pgfpathcurveto{\pgfqpoint{0.432803in}{3.758542in}}{\pgfqpoint{0.437194in}{3.747943in}}{\pgfqpoint{0.445007in}{3.740129in}}%
\pgfpathcurveto{\pgfqpoint{0.452821in}{3.732315in}}{\pgfqpoint{0.463420in}{3.727925in}}{\pgfqpoint{0.474470in}{3.727925in}}%
\pgfpathlineto{\pgfqpoint{0.474470in}{3.727925in}}%
\pgfpathclose%
\pgfusepath{stroke}%
\end{pgfscope}%
\begin{pgfscope}%
\pgfpathrectangle{\pgfqpoint{0.393053in}{0.375000in}}{\pgfqpoint{6.356833in}{5.175000in}}%
\pgfusepath{clip}%
\pgfsetbuttcap%
\pgfsetroundjoin%
\pgfsetlinewidth{1.003750pt}%
\definecolor{currentstroke}{rgb}{0.827451,0.827451,0.827451}%
\pgfsetstrokecolor{currentstroke}%
\pgfsetdash{}{0pt}%
\pgfpathmoveto{\pgfqpoint{1.493790in}{1.813114in}}%
\pgfpathcurveto{\pgfqpoint{1.504840in}{1.813114in}}{\pgfqpoint{1.515439in}{1.817505in}}{\pgfqpoint{1.523252in}{1.825318in}}%
\pgfpathcurveto{\pgfqpoint{1.531066in}{1.833132in}}{\pgfqpoint{1.535456in}{1.843731in}}{\pgfqpoint{1.535456in}{1.854781in}}%
\pgfpathcurveto{\pgfqpoint{1.535456in}{1.865831in}}{\pgfqpoint{1.531066in}{1.876430in}}{\pgfqpoint{1.523252in}{1.884244in}}%
\pgfpathcurveto{\pgfqpoint{1.515439in}{1.892058in}}{\pgfqpoint{1.504840in}{1.896448in}}{\pgfqpoint{1.493790in}{1.896448in}}%
\pgfpathcurveto{\pgfqpoint{1.482740in}{1.896448in}}{\pgfqpoint{1.472140in}{1.892058in}}{\pgfqpoint{1.464327in}{1.884244in}}%
\pgfpathcurveto{\pgfqpoint{1.456513in}{1.876430in}}{\pgfqpoint{1.452123in}{1.865831in}}{\pgfqpoint{1.452123in}{1.854781in}}%
\pgfpathcurveto{\pgfqpoint{1.452123in}{1.843731in}}{\pgfqpoint{1.456513in}{1.833132in}}{\pgfqpoint{1.464327in}{1.825318in}}%
\pgfpathcurveto{\pgfqpoint{1.472140in}{1.817505in}}{\pgfqpoint{1.482740in}{1.813114in}}{\pgfqpoint{1.493790in}{1.813114in}}%
\pgfpathlineto{\pgfqpoint{1.493790in}{1.813114in}}%
\pgfpathclose%
\pgfusepath{stroke}%
\end{pgfscope}%
\begin{pgfscope}%
\pgfpathrectangle{\pgfqpoint{0.393053in}{0.375000in}}{\pgfqpoint{6.356833in}{5.175000in}}%
\pgfusepath{clip}%
\pgfsetbuttcap%
\pgfsetroundjoin%
\pgfsetlinewidth{1.003750pt}%
\definecolor{currentstroke}{rgb}{0.827451,0.827451,0.827451}%
\pgfsetstrokecolor{currentstroke}%
\pgfsetdash{}{0pt}%
\pgfpathmoveto{\pgfqpoint{1.687372in}{1.621104in}}%
\pgfpathcurveto{\pgfqpoint{1.698422in}{1.621104in}}{\pgfqpoint{1.709022in}{1.625494in}}{\pgfqpoint{1.716835in}{1.633308in}}%
\pgfpathcurveto{\pgfqpoint{1.724649in}{1.641122in}}{\pgfqpoint{1.729039in}{1.651721in}}{\pgfqpoint{1.729039in}{1.662771in}}%
\pgfpathcurveto{\pgfqpoint{1.729039in}{1.673821in}}{\pgfqpoint{1.724649in}{1.684420in}}{\pgfqpoint{1.716835in}{1.692234in}}%
\pgfpathcurveto{\pgfqpoint{1.709022in}{1.700047in}}{\pgfqpoint{1.698422in}{1.704437in}}{\pgfqpoint{1.687372in}{1.704437in}}%
\pgfpathcurveto{\pgfqpoint{1.676322in}{1.704437in}}{\pgfqpoint{1.665723in}{1.700047in}}{\pgfqpoint{1.657910in}{1.692234in}}%
\pgfpathcurveto{\pgfqpoint{1.650096in}{1.684420in}}{\pgfqpoint{1.645706in}{1.673821in}}{\pgfqpoint{1.645706in}{1.662771in}}%
\pgfpathcurveto{\pgfqpoint{1.645706in}{1.651721in}}{\pgfqpoint{1.650096in}{1.641122in}}{\pgfqpoint{1.657910in}{1.633308in}}%
\pgfpathcurveto{\pgfqpoint{1.665723in}{1.625494in}}{\pgfqpoint{1.676322in}{1.621104in}}{\pgfqpoint{1.687372in}{1.621104in}}%
\pgfpathlineto{\pgfqpoint{1.687372in}{1.621104in}}%
\pgfpathclose%
\pgfusepath{stroke}%
\end{pgfscope}%
\begin{pgfscope}%
\pgfpathrectangle{\pgfqpoint{0.393053in}{0.375000in}}{\pgfqpoint{6.356833in}{5.175000in}}%
\pgfusepath{clip}%
\pgfsetbuttcap%
\pgfsetroundjoin%
\pgfsetlinewidth{1.003750pt}%
\definecolor{currentstroke}{rgb}{0.827451,0.827451,0.827451}%
\pgfsetstrokecolor{currentstroke}%
\pgfsetdash{}{0pt}%
\pgfpathmoveto{\pgfqpoint{3.592196in}{0.693461in}}%
\pgfpathcurveto{\pgfqpoint{3.603246in}{0.693461in}}{\pgfqpoint{3.613845in}{0.697851in}}{\pgfqpoint{3.621659in}{0.705665in}}%
\pgfpathcurveto{\pgfqpoint{3.629472in}{0.713479in}}{\pgfqpoint{3.633862in}{0.724078in}}{\pgfqpoint{3.633862in}{0.735128in}}%
\pgfpathcurveto{\pgfqpoint{3.633862in}{0.746178in}}{\pgfqpoint{3.629472in}{0.756777in}}{\pgfqpoint{3.621659in}{0.764591in}}%
\pgfpathcurveto{\pgfqpoint{3.613845in}{0.772404in}}{\pgfqpoint{3.603246in}{0.776794in}}{\pgfqpoint{3.592196in}{0.776794in}}%
\pgfpathcurveto{\pgfqpoint{3.581146in}{0.776794in}}{\pgfqpoint{3.570547in}{0.772404in}}{\pgfqpoint{3.562733in}{0.764591in}}%
\pgfpathcurveto{\pgfqpoint{3.554919in}{0.756777in}}{\pgfqpoint{3.550529in}{0.746178in}}{\pgfqpoint{3.550529in}{0.735128in}}%
\pgfpathcurveto{\pgfqpoint{3.550529in}{0.724078in}}{\pgfqpoint{3.554919in}{0.713479in}}{\pgfqpoint{3.562733in}{0.705665in}}%
\pgfpathcurveto{\pgfqpoint{3.570547in}{0.697851in}}{\pgfqpoint{3.581146in}{0.693461in}}{\pgfqpoint{3.592196in}{0.693461in}}%
\pgfpathlineto{\pgfqpoint{3.592196in}{0.693461in}}%
\pgfpathclose%
\pgfusepath{stroke}%
\end{pgfscope}%
\begin{pgfscope}%
\pgfpathrectangle{\pgfqpoint{0.393053in}{0.375000in}}{\pgfqpoint{6.356833in}{5.175000in}}%
\pgfusepath{clip}%
\pgfsetbuttcap%
\pgfsetroundjoin%
\pgfsetlinewidth{1.003750pt}%
\definecolor{currentstroke}{rgb}{0.827451,0.827451,0.827451}%
\pgfsetstrokecolor{currentstroke}%
\pgfsetdash{}{0pt}%
\pgfpathmoveto{\pgfqpoint{0.947650in}{2.625936in}}%
\pgfpathcurveto{\pgfqpoint{0.958700in}{2.625936in}}{\pgfqpoint{0.969299in}{2.630326in}}{\pgfqpoint{0.977112in}{2.638140in}}%
\pgfpathcurveto{\pgfqpoint{0.984926in}{2.645953in}}{\pgfqpoint{0.989316in}{2.656552in}}{\pgfqpoint{0.989316in}{2.667602in}}%
\pgfpathcurveto{\pgfqpoint{0.989316in}{2.678652in}}{\pgfqpoint{0.984926in}{2.689252in}}{\pgfqpoint{0.977112in}{2.697065in}}%
\pgfpathcurveto{\pgfqpoint{0.969299in}{2.704879in}}{\pgfqpoint{0.958700in}{2.709269in}}{\pgfqpoint{0.947650in}{2.709269in}}%
\pgfpathcurveto{\pgfqpoint{0.936600in}{2.709269in}}{\pgfqpoint{0.926000in}{2.704879in}}{\pgfqpoint{0.918187in}{2.697065in}}%
\pgfpathcurveto{\pgfqpoint{0.910373in}{2.689252in}}{\pgfqpoint{0.905983in}{2.678652in}}{\pgfqpoint{0.905983in}{2.667602in}}%
\pgfpathcurveto{\pgfqpoint{0.905983in}{2.656552in}}{\pgfqpoint{0.910373in}{2.645953in}}{\pgfqpoint{0.918187in}{2.638140in}}%
\pgfpathcurveto{\pgfqpoint{0.926000in}{2.630326in}}{\pgfqpoint{0.936600in}{2.625936in}}{\pgfqpoint{0.947650in}{2.625936in}}%
\pgfpathlineto{\pgfqpoint{0.947650in}{2.625936in}}%
\pgfpathclose%
\pgfusepath{stroke}%
\end{pgfscope}%
\begin{pgfscope}%
\pgfpathrectangle{\pgfqpoint{0.393053in}{0.375000in}}{\pgfqpoint{6.356833in}{5.175000in}}%
\pgfusepath{clip}%
\pgfsetbuttcap%
\pgfsetroundjoin%
\pgfsetlinewidth{1.003750pt}%
\definecolor{currentstroke}{rgb}{0.827451,0.827451,0.827451}%
\pgfsetstrokecolor{currentstroke}%
\pgfsetdash{}{0pt}%
\pgfpathmoveto{\pgfqpoint{2.796329in}{0.903359in}}%
\pgfpathcurveto{\pgfqpoint{2.807379in}{0.903359in}}{\pgfqpoint{2.817978in}{0.907749in}}{\pgfqpoint{2.825792in}{0.915563in}}%
\pgfpathcurveto{\pgfqpoint{2.833605in}{0.923376in}}{\pgfqpoint{2.837995in}{0.933975in}}{\pgfqpoint{2.837995in}{0.945025in}}%
\pgfpathcurveto{\pgfqpoint{2.837995in}{0.956076in}}{\pgfqpoint{2.833605in}{0.966675in}}{\pgfqpoint{2.825792in}{0.974488in}}%
\pgfpathcurveto{\pgfqpoint{2.817978in}{0.982302in}}{\pgfqpoint{2.807379in}{0.986692in}}{\pgfqpoint{2.796329in}{0.986692in}}%
\pgfpathcurveto{\pgfqpoint{2.785279in}{0.986692in}}{\pgfqpoint{2.774680in}{0.982302in}}{\pgfqpoint{2.766866in}{0.974488in}}%
\pgfpathcurveto{\pgfqpoint{2.759052in}{0.966675in}}{\pgfqpoint{2.754662in}{0.956076in}}{\pgfqpoint{2.754662in}{0.945025in}}%
\pgfpathcurveto{\pgfqpoint{2.754662in}{0.933975in}}{\pgfqpoint{2.759052in}{0.923376in}}{\pgfqpoint{2.766866in}{0.915563in}}%
\pgfpathcurveto{\pgfqpoint{2.774680in}{0.907749in}}{\pgfqpoint{2.785279in}{0.903359in}}{\pgfqpoint{2.796329in}{0.903359in}}%
\pgfpathlineto{\pgfqpoint{2.796329in}{0.903359in}}%
\pgfpathclose%
\pgfusepath{stroke}%
\end{pgfscope}%
\begin{pgfscope}%
\pgfpathrectangle{\pgfqpoint{0.393053in}{0.375000in}}{\pgfqpoint{6.356833in}{5.175000in}}%
\pgfusepath{clip}%
\pgfsetbuttcap%
\pgfsetroundjoin%
\pgfsetlinewidth{1.003750pt}%
\definecolor{currentstroke}{rgb}{0.827451,0.827451,0.827451}%
\pgfsetstrokecolor{currentstroke}%
\pgfsetdash{}{0pt}%
\pgfpathmoveto{\pgfqpoint{1.671050in}{1.714593in}}%
\pgfpathcurveto{\pgfqpoint{1.682100in}{1.714593in}}{\pgfqpoint{1.692699in}{1.718984in}}{\pgfqpoint{1.700512in}{1.726797in}}%
\pgfpathcurveto{\pgfqpoint{1.708326in}{1.734611in}}{\pgfqpoint{1.712716in}{1.745210in}}{\pgfqpoint{1.712716in}{1.756260in}}%
\pgfpathcurveto{\pgfqpoint{1.712716in}{1.767310in}}{\pgfqpoint{1.708326in}{1.777909in}}{\pgfqpoint{1.700512in}{1.785723in}}%
\pgfpathcurveto{\pgfqpoint{1.692699in}{1.793537in}}{\pgfqpoint{1.682100in}{1.797927in}}{\pgfqpoint{1.671050in}{1.797927in}}%
\pgfpathcurveto{\pgfqpoint{1.659999in}{1.797927in}}{\pgfqpoint{1.649400in}{1.793537in}}{\pgfqpoint{1.641587in}{1.785723in}}%
\pgfpathcurveto{\pgfqpoint{1.633773in}{1.777909in}}{\pgfqpoint{1.629383in}{1.767310in}}{\pgfqpoint{1.629383in}{1.756260in}}%
\pgfpathcurveto{\pgfqpoint{1.629383in}{1.745210in}}{\pgfqpoint{1.633773in}{1.734611in}}{\pgfqpoint{1.641587in}{1.726797in}}%
\pgfpathcurveto{\pgfqpoint{1.649400in}{1.718984in}}{\pgfqpoint{1.659999in}{1.714593in}}{\pgfqpoint{1.671050in}{1.714593in}}%
\pgfpathlineto{\pgfqpoint{1.671050in}{1.714593in}}%
\pgfpathclose%
\pgfusepath{stroke}%
\end{pgfscope}%
\begin{pgfscope}%
\pgfpathrectangle{\pgfqpoint{0.393053in}{0.375000in}}{\pgfqpoint{6.356833in}{5.175000in}}%
\pgfusepath{clip}%
\pgfsetbuttcap%
\pgfsetroundjoin%
\pgfsetlinewidth{1.003750pt}%
\definecolor{currentstroke}{rgb}{0.827451,0.827451,0.827451}%
\pgfsetstrokecolor{currentstroke}%
\pgfsetdash{}{0pt}%
\pgfpathmoveto{\pgfqpoint{2.498084in}{1.056522in}}%
\pgfpathcurveto{\pgfqpoint{2.509134in}{1.056522in}}{\pgfqpoint{2.519733in}{1.060912in}}{\pgfqpoint{2.527547in}{1.068726in}}%
\pgfpathcurveto{\pgfqpoint{2.535360in}{1.076539in}}{\pgfqpoint{2.539750in}{1.087138in}}{\pgfqpoint{2.539750in}{1.098188in}}%
\pgfpathcurveto{\pgfqpoint{2.539750in}{1.109239in}}{\pgfqpoint{2.535360in}{1.119838in}}{\pgfqpoint{2.527547in}{1.127651in}}%
\pgfpathcurveto{\pgfqpoint{2.519733in}{1.135465in}}{\pgfqpoint{2.509134in}{1.139855in}}{\pgfqpoint{2.498084in}{1.139855in}}%
\pgfpathcurveto{\pgfqpoint{2.487034in}{1.139855in}}{\pgfqpoint{2.476435in}{1.135465in}}{\pgfqpoint{2.468621in}{1.127651in}}%
\pgfpathcurveto{\pgfqpoint{2.460807in}{1.119838in}}{\pgfqpoint{2.456417in}{1.109239in}}{\pgfqpoint{2.456417in}{1.098188in}}%
\pgfpathcurveto{\pgfqpoint{2.456417in}{1.087138in}}{\pgfqpoint{2.460807in}{1.076539in}}{\pgfqpoint{2.468621in}{1.068726in}}%
\pgfpathcurveto{\pgfqpoint{2.476435in}{1.060912in}}{\pgfqpoint{2.487034in}{1.056522in}}{\pgfqpoint{2.498084in}{1.056522in}}%
\pgfpathlineto{\pgfqpoint{2.498084in}{1.056522in}}%
\pgfpathclose%
\pgfusepath{stroke}%
\end{pgfscope}%
\begin{pgfscope}%
\pgfpathrectangle{\pgfqpoint{0.393053in}{0.375000in}}{\pgfqpoint{6.356833in}{5.175000in}}%
\pgfusepath{clip}%
\pgfsetbuttcap%
\pgfsetroundjoin%
\pgfsetlinewidth{1.003750pt}%
\definecolor{currentstroke}{rgb}{0.827451,0.827451,0.827451}%
\pgfsetstrokecolor{currentstroke}%
\pgfsetdash{}{0pt}%
\pgfpathmoveto{\pgfqpoint{0.868151in}{2.818774in}}%
\pgfpathcurveto{\pgfqpoint{0.879201in}{2.818774in}}{\pgfqpoint{0.889800in}{2.823165in}}{\pgfqpoint{0.897614in}{2.830978in}}%
\pgfpathcurveto{\pgfqpoint{0.905428in}{2.838792in}}{\pgfqpoint{0.909818in}{2.849391in}}{\pgfqpoint{0.909818in}{2.860441in}}%
\pgfpathcurveto{\pgfqpoint{0.909818in}{2.871491in}}{\pgfqpoint{0.905428in}{2.882090in}}{\pgfqpoint{0.897614in}{2.889904in}}%
\pgfpathcurveto{\pgfqpoint{0.889800in}{2.897717in}}{\pgfqpoint{0.879201in}{2.902108in}}{\pgfqpoint{0.868151in}{2.902108in}}%
\pgfpathcurveto{\pgfqpoint{0.857101in}{2.902108in}}{\pgfqpoint{0.846502in}{2.897717in}}{\pgfqpoint{0.838688in}{2.889904in}}%
\pgfpathcurveto{\pgfqpoint{0.830875in}{2.882090in}}{\pgfqpoint{0.826484in}{2.871491in}}{\pgfqpoint{0.826484in}{2.860441in}}%
\pgfpathcurveto{\pgfqpoint{0.826484in}{2.849391in}}{\pgfqpoint{0.830875in}{2.838792in}}{\pgfqpoint{0.838688in}{2.830978in}}%
\pgfpathcurveto{\pgfqpoint{0.846502in}{2.823165in}}{\pgfqpoint{0.857101in}{2.818774in}}{\pgfqpoint{0.868151in}{2.818774in}}%
\pgfpathlineto{\pgfqpoint{0.868151in}{2.818774in}}%
\pgfpathclose%
\pgfusepath{stroke}%
\end{pgfscope}%
\begin{pgfscope}%
\pgfpathrectangle{\pgfqpoint{0.393053in}{0.375000in}}{\pgfqpoint{6.356833in}{5.175000in}}%
\pgfusepath{clip}%
\pgfsetbuttcap%
\pgfsetroundjoin%
\pgfsetlinewidth{1.003750pt}%
\definecolor{currentstroke}{rgb}{0.827451,0.827451,0.827451}%
\pgfsetstrokecolor{currentstroke}%
\pgfsetdash{}{0pt}%
\pgfpathmoveto{\pgfqpoint{2.437419in}{1.106016in}}%
\pgfpathcurveto{\pgfqpoint{2.448469in}{1.106016in}}{\pgfqpoint{2.459069in}{1.110406in}}{\pgfqpoint{2.466882in}{1.118219in}}%
\pgfpathcurveto{\pgfqpoint{2.474696in}{1.126033in}}{\pgfqpoint{2.479086in}{1.136632in}}{\pgfqpoint{2.479086in}{1.147682in}}%
\pgfpathcurveto{\pgfqpoint{2.479086in}{1.158732in}}{\pgfqpoint{2.474696in}{1.169331in}}{\pgfqpoint{2.466882in}{1.177145in}}%
\pgfpathcurveto{\pgfqpoint{2.459069in}{1.184959in}}{\pgfqpoint{2.448469in}{1.189349in}}{\pgfqpoint{2.437419in}{1.189349in}}%
\pgfpathcurveto{\pgfqpoint{2.426369in}{1.189349in}}{\pgfqpoint{2.415770in}{1.184959in}}{\pgfqpoint{2.407957in}{1.177145in}}%
\pgfpathcurveto{\pgfqpoint{2.400143in}{1.169331in}}{\pgfqpoint{2.395753in}{1.158732in}}{\pgfqpoint{2.395753in}{1.147682in}}%
\pgfpathcurveto{\pgfqpoint{2.395753in}{1.136632in}}{\pgfqpoint{2.400143in}{1.126033in}}{\pgfqpoint{2.407957in}{1.118219in}}%
\pgfpathcurveto{\pgfqpoint{2.415770in}{1.110406in}}{\pgfqpoint{2.426369in}{1.106016in}}{\pgfqpoint{2.437419in}{1.106016in}}%
\pgfpathlineto{\pgfqpoint{2.437419in}{1.106016in}}%
\pgfpathclose%
\pgfusepath{stroke}%
\end{pgfscope}%
\begin{pgfscope}%
\pgfpathrectangle{\pgfqpoint{0.393053in}{0.375000in}}{\pgfqpoint{6.356833in}{5.175000in}}%
\pgfusepath{clip}%
\pgfsetbuttcap%
\pgfsetroundjoin%
\pgfsetlinewidth{1.003750pt}%
\definecolor{currentstroke}{rgb}{0.827451,0.827451,0.827451}%
\pgfsetstrokecolor{currentstroke}%
\pgfsetdash{}{0pt}%
\pgfpathmoveto{\pgfqpoint{0.921910in}{2.777569in}}%
\pgfpathcurveto{\pgfqpoint{0.932960in}{2.777569in}}{\pgfqpoint{0.943559in}{2.781959in}}{\pgfqpoint{0.951373in}{2.789773in}}%
\pgfpathcurveto{\pgfqpoint{0.959186in}{2.797586in}}{\pgfqpoint{0.963577in}{2.808185in}}{\pgfqpoint{0.963577in}{2.819235in}}%
\pgfpathcurveto{\pgfqpoint{0.963577in}{2.830286in}}{\pgfqpoint{0.959186in}{2.840885in}}{\pgfqpoint{0.951373in}{2.848698in}}%
\pgfpathcurveto{\pgfqpoint{0.943559in}{2.856512in}}{\pgfqpoint{0.932960in}{2.860902in}}{\pgfqpoint{0.921910in}{2.860902in}}%
\pgfpathcurveto{\pgfqpoint{0.910860in}{2.860902in}}{\pgfqpoint{0.900261in}{2.856512in}}{\pgfqpoint{0.892447in}{2.848698in}}%
\pgfpathcurveto{\pgfqpoint{0.884633in}{2.840885in}}{\pgfqpoint{0.880243in}{2.830286in}}{\pgfqpoint{0.880243in}{2.819235in}}%
\pgfpathcurveto{\pgfqpoint{0.880243in}{2.808185in}}{\pgfqpoint{0.884633in}{2.797586in}}{\pgfqpoint{0.892447in}{2.789773in}}%
\pgfpathcurveto{\pgfqpoint{0.900261in}{2.781959in}}{\pgfqpoint{0.910860in}{2.777569in}}{\pgfqpoint{0.921910in}{2.777569in}}%
\pgfpathlineto{\pgfqpoint{0.921910in}{2.777569in}}%
\pgfpathclose%
\pgfusepath{stroke}%
\end{pgfscope}%
\begin{pgfscope}%
\pgfpathrectangle{\pgfqpoint{0.393053in}{0.375000in}}{\pgfqpoint{6.356833in}{5.175000in}}%
\pgfusepath{clip}%
\pgfsetbuttcap%
\pgfsetroundjoin%
\pgfsetlinewidth{1.003750pt}%
\definecolor{currentstroke}{rgb}{0.827451,0.827451,0.827451}%
\pgfsetstrokecolor{currentstroke}%
\pgfsetdash{}{0pt}%
\pgfpathmoveto{\pgfqpoint{3.649696in}{0.625044in}}%
\pgfpathcurveto{\pgfqpoint{3.660746in}{0.625044in}}{\pgfqpoint{3.671345in}{0.629434in}}{\pgfqpoint{3.679159in}{0.637248in}}%
\pgfpathcurveto{\pgfqpoint{3.686972in}{0.645061in}}{\pgfqpoint{3.691363in}{0.655660in}}{\pgfqpoint{3.691363in}{0.666710in}}%
\pgfpathcurveto{\pgfqpoint{3.691363in}{0.677761in}}{\pgfqpoint{3.686972in}{0.688360in}}{\pgfqpoint{3.679159in}{0.696173in}}%
\pgfpathcurveto{\pgfqpoint{3.671345in}{0.703987in}}{\pgfqpoint{3.660746in}{0.708377in}}{\pgfqpoint{3.649696in}{0.708377in}}%
\pgfpathcurveto{\pgfqpoint{3.638646in}{0.708377in}}{\pgfqpoint{3.628047in}{0.703987in}}{\pgfqpoint{3.620233in}{0.696173in}}%
\pgfpathcurveto{\pgfqpoint{3.612420in}{0.688360in}}{\pgfqpoint{3.608029in}{0.677761in}}{\pgfqpoint{3.608029in}{0.666710in}}%
\pgfpathcurveto{\pgfqpoint{3.608029in}{0.655660in}}{\pgfqpoint{3.612420in}{0.645061in}}{\pgfqpoint{3.620233in}{0.637248in}}%
\pgfpathcurveto{\pgfqpoint{3.628047in}{0.629434in}}{\pgfqpoint{3.638646in}{0.625044in}}{\pgfqpoint{3.649696in}{0.625044in}}%
\pgfpathlineto{\pgfqpoint{3.649696in}{0.625044in}}%
\pgfpathclose%
\pgfusepath{stroke}%
\end{pgfscope}%
\begin{pgfscope}%
\pgfpathrectangle{\pgfqpoint{0.393053in}{0.375000in}}{\pgfqpoint{6.356833in}{5.175000in}}%
\pgfusepath{clip}%
\pgfsetbuttcap%
\pgfsetroundjoin%
\pgfsetlinewidth{1.003750pt}%
\definecolor{currentstroke}{rgb}{0.827451,0.827451,0.827451}%
\pgfsetstrokecolor{currentstroke}%
\pgfsetdash{}{0pt}%
\pgfpathmoveto{\pgfqpoint{0.762565in}{3.034222in}}%
\pgfpathcurveto{\pgfqpoint{0.773615in}{3.034222in}}{\pgfqpoint{0.784214in}{3.038612in}}{\pgfqpoint{0.792028in}{3.046426in}}%
\pgfpathcurveto{\pgfqpoint{0.799841in}{3.054240in}}{\pgfqpoint{0.804232in}{3.064839in}}{\pgfqpoint{0.804232in}{3.075889in}}%
\pgfpathcurveto{\pgfqpoint{0.804232in}{3.086939in}}{\pgfqpoint{0.799841in}{3.097538in}}{\pgfqpoint{0.792028in}{3.105352in}}%
\pgfpathcurveto{\pgfqpoint{0.784214in}{3.113165in}}{\pgfqpoint{0.773615in}{3.117556in}}{\pgfqpoint{0.762565in}{3.117556in}}%
\pgfpathcurveto{\pgfqpoint{0.751515in}{3.117556in}}{\pgfqpoint{0.740916in}{3.113165in}}{\pgfqpoint{0.733102in}{3.105352in}}%
\pgfpathcurveto{\pgfqpoint{0.725289in}{3.097538in}}{\pgfqpoint{0.720898in}{3.086939in}}{\pgfqpoint{0.720898in}{3.075889in}}%
\pgfpathcurveto{\pgfqpoint{0.720898in}{3.064839in}}{\pgfqpoint{0.725289in}{3.054240in}}{\pgfqpoint{0.733102in}{3.046426in}}%
\pgfpathcurveto{\pgfqpoint{0.740916in}{3.038612in}}{\pgfqpoint{0.751515in}{3.034222in}}{\pgfqpoint{0.762565in}{3.034222in}}%
\pgfpathlineto{\pgfqpoint{0.762565in}{3.034222in}}%
\pgfpathclose%
\pgfusepath{stroke}%
\end{pgfscope}%
\begin{pgfscope}%
\pgfpathrectangle{\pgfqpoint{0.393053in}{0.375000in}}{\pgfqpoint{6.356833in}{5.175000in}}%
\pgfusepath{clip}%
\pgfsetbuttcap%
\pgfsetroundjoin%
\pgfsetlinewidth{1.003750pt}%
\definecolor{currentstroke}{rgb}{0.827451,0.827451,0.827451}%
\pgfsetstrokecolor{currentstroke}%
\pgfsetdash{}{0pt}%
\pgfpathmoveto{\pgfqpoint{1.457810in}{1.831739in}}%
\pgfpathcurveto{\pgfqpoint{1.468860in}{1.831739in}}{\pgfqpoint{1.479459in}{1.836129in}}{\pgfqpoint{1.487273in}{1.843943in}}%
\pgfpathcurveto{\pgfqpoint{1.495086in}{1.851756in}}{\pgfqpoint{1.499477in}{1.862355in}}{\pgfqpoint{1.499477in}{1.873406in}}%
\pgfpathcurveto{\pgfqpoint{1.499477in}{1.884456in}}{\pgfqpoint{1.495086in}{1.895055in}}{\pgfqpoint{1.487273in}{1.902868in}}%
\pgfpathcurveto{\pgfqpoint{1.479459in}{1.910682in}}{\pgfqpoint{1.468860in}{1.915072in}}{\pgfqpoint{1.457810in}{1.915072in}}%
\pgfpathcurveto{\pgfqpoint{1.446760in}{1.915072in}}{\pgfqpoint{1.436161in}{1.910682in}}{\pgfqpoint{1.428347in}{1.902868in}}%
\pgfpathcurveto{\pgfqpoint{1.420533in}{1.895055in}}{\pgfqpoint{1.416143in}{1.884456in}}{\pgfqpoint{1.416143in}{1.873406in}}%
\pgfpathcurveto{\pgfqpoint{1.416143in}{1.862355in}}{\pgfqpoint{1.420533in}{1.851756in}}{\pgfqpoint{1.428347in}{1.843943in}}%
\pgfpathcurveto{\pgfqpoint{1.436161in}{1.836129in}}{\pgfqpoint{1.446760in}{1.831739in}}{\pgfqpoint{1.457810in}{1.831739in}}%
\pgfpathlineto{\pgfqpoint{1.457810in}{1.831739in}}%
\pgfpathclose%
\pgfusepath{stroke}%
\end{pgfscope}%
\begin{pgfscope}%
\pgfpathrectangle{\pgfqpoint{0.393053in}{0.375000in}}{\pgfqpoint{6.356833in}{5.175000in}}%
\pgfusepath{clip}%
\pgfsetbuttcap%
\pgfsetroundjoin%
\pgfsetlinewidth{1.003750pt}%
\definecolor{currentstroke}{rgb}{0.827451,0.827451,0.827451}%
\pgfsetstrokecolor{currentstroke}%
\pgfsetdash{}{0pt}%
\pgfpathmoveto{\pgfqpoint{0.418752in}{4.107793in}}%
\pgfpathcurveto{\pgfqpoint{0.429802in}{4.107793in}}{\pgfqpoint{0.440401in}{4.112183in}}{\pgfqpoint{0.448215in}{4.119996in}}%
\pgfpathcurveto{\pgfqpoint{0.456028in}{4.127810in}}{\pgfqpoint{0.460419in}{4.138409in}}{\pgfqpoint{0.460419in}{4.149459in}}%
\pgfpathcurveto{\pgfqpoint{0.460419in}{4.160509in}}{\pgfqpoint{0.456028in}{4.171108in}}{\pgfqpoint{0.448215in}{4.178922in}}%
\pgfpathcurveto{\pgfqpoint{0.440401in}{4.186736in}}{\pgfqpoint{0.429802in}{4.191126in}}{\pgfqpoint{0.418752in}{4.191126in}}%
\pgfpathcurveto{\pgfqpoint{0.407702in}{4.191126in}}{\pgfqpoint{0.397103in}{4.186736in}}{\pgfqpoint{0.389289in}{4.178922in}}%
\pgfpathcurveto{\pgfqpoint{0.381476in}{4.171108in}}{\pgfqpoint{0.377085in}{4.160509in}}{\pgfqpoint{0.377085in}{4.149459in}}%
\pgfpathcurveto{\pgfqpoint{0.377085in}{4.138409in}}{\pgfqpoint{0.381476in}{4.127810in}}{\pgfqpoint{0.389289in}{4.119996in}}%
\pgfpathcurveto{\pgfqpoint{0.397103in}{4.112183in}}{\pgfqpoint{0.407702in}{4.107793in}}{\pgfqpoint{0.418752in}{4.107793in}}%
\pgfpathlineto{\pgfqpoint{0.418752in}{4.107793in}}%
\pgfpathclose%
\pgfusepath{stroke}%
\end{pgfscope}%
\begin{pgfscope}%
\pgfpathrectangle{\pgfqpoint{0.393053in}{0.375000in}}{\pgfqpoint{6.356833in}{5.175000in}}%
\pgfusepath{clip}%
\pgfsetbuttcap%
\pgfsetroundjoin%
\pgfsetlinewidth{1.003750pt}%
\definecolor{currentstroke}{rgb}{0.827451,0.827451,0.827451}%
\pgfsetstrokecolor{currentstroke}%
\pgfsetdash{}{0pt}%
\pgfpathmoveto{\pgfqpoint{1.306751in}{1.985193in}}%
\pgfpathcurveto{\pgfqpoint{1.317801in}{1.985193in}}{\pgfqpoint{1.328400in}{1.989583in}}{\pgfqpoint{1.336214in}{1.997397in}}%
\pgfpathcurveto{\pgfqpoint{1.344027in}{2.005210in}}{\pgfqpoint{1.348418in}{2.015809in}}{\pgfqpoint{1.348418in}{2.026859in}}%
\pgfpathcurveto{\pgfqpoint{1.348418in}{2.037909in}}{\pgfqpoint{1.344027in}{2.048509in}}{\pgfqpoint{1.336214in}{2.056322in}}%
\pgfpathcurveto{\pgfqpoint{1.328400in}{2.064136in}}{\pgfqpoint{1.317801in}{2.068526in}}{\pgfqpoint{1.306751in}{2.068526in}}%
\pgfpathcurveto{\pgfqpoint{1.295701in}{2.068526in}}{\pgfqpoint{1.285102in}{2.064136in}}{\pgfqpoint{1.277288in}{2.056322in}}%
\pgfpathcurveto{\pgfqpoint{1.269475in}{2.048509in}}{\pgfqpoint{1.265084in}{2.037909in}}{\pgfqpoint{1.265084in}{2.026859in}}%
\pgfpathcurveto{\pgfqpoint{1.265084in}{2.015809in}}{\pgfqpoint{1.269475in}{2.005210in}}{\pgfqpoint{1.277288in}{1.997397in}}%
\pgfpathcurveto{\pgfqpoint{1.285102in}{1.989583in}}{\pgfqpoint{1.295701in}{1.985193in}}{\pgfqpoint{1.306751in}{1.985193in}}%
\pgfpathlineto{\pgfqpoint{1.306751in}{1.985193in}}%
\pgfpathclose%
\pgfusepath{stroke}%
\end{pgfscope}%
\begin{pgfscope}%
\pgfpathrectangle{\pgfqpoint{0.393053in}{0.375000in}}{\pgfqpoint{6.356833in}{5.175000in}}%
\pgfusepath{clip}%
\pgfsetbuttcap%
\pgfsetroundjoin%
\pgfsetlinewidth{1.003750pt}%
\definecolor{currentstroke}{rgb}{0.827451,0.827451,0.827451}%
\pgfsetstrokecolor{currentstroke}%
\pgfsetdash{}{0pt}%
\pgfpathmoveto{\pgfqpoint{0.537744in}{3.497149in}}%
\pgfpathcurveto{\pgfqpoint{0.548795in}{3.497149in}}{\pgfqpoint{0.559394in}{3.501540in}}{\pgfqpoint{0.567207in}{3.509353in}}%
\pgfpathcurveto{\pgfqpoint{0.575021in}{3.517167in}}{\pgfqpoint{0.579411in}{3.527766in}}{\pgfqpoint{0.579411in}{3.538816in}}%
\pgfpathcurveto{\pgfqpoint{0.579411in}{3.549866in}}{\pgfqpoint{0.575021in}{3.560465in}}{\pgfqpoint{0.567207in}{3.568279in}}%
\pgfpathcurveto{\pgfqpoint{0.559394in}{3.576092in}}{\pgfqpoint{0.548795in}{3.580483in}}{\pgfqpoint{0.537744in}{3.580483in}}%
\pgfpathcurveto{\pgfqpoint{0.526694in}{3.580483in}}{\pgfqpoint{0.516095in}{3.576092in}}{\pgfqpoint{0.508282in}{3.568279in}}%
\pgfpathcurveto{\pgfqpoint{0.500468in}{3.560465in}}{\pgfqpoint{0.496078in}{3.549866in}}{\pgfqpoint{0.496078in}{3.538816in}}%
\pgfpathcurveto{\pgfqpoint{0.496078in}{3.527766in}}{\pgfqpoint{0.500468in}{3.517167in}}{\pgfqpoint{0.508282in}{3.509353in}}%
\pgfpathcurveto{\pgfqpoint{0.516095in}{3.501540in}}{\pgfqpoint{0.526694in}{3.497149in}}{\pgfqpoint{0.537744in}{3.497149in}}%
\pgfpathlineto{\pgfqpoint{0.537744in}{3.497149in}}%
\pgfpathclose%
\pgfusepath{stroke}%
\end{pgfscope}%
\begin{pgfscope}%
\pgfpathrectangle{\pgfqpoint{0.393053in}{0.375000in}}{\pgfqpoint{6.356833in}{5.175000in}}%
\pgfusepath{clip}%
\pgfsetbuttcap%
\pgfsetroundjoin%
\pgfsetlinewidth{1.003750pt}%
\definecolor{currentstroke}{rgb}{0.827451,0.827451,0.827451}%
\pgfsetstrokecolor{currentstroke}%
\pgfsetdash{}{0pt}%
\pgfpathmoveto{\pgfqpoint{1.255595in}{2.043062in}}%
\pgfpathcurveto{\pgfqpoint{1.266645in}{2.043062in}}{\pgfqpoint{1.277244in}{2.047452in}}{\pgfqpoint{1.285058in}{2.055266in}}%
\pgfpathcurveto{\pgfqpoint{1.292872in}{2.063079in}}{\pgfqpoint{1.297262in}{2.073678in}}{\pgfqpoint{1.297262in}{2.084728in}}%
\pgfpathcurveto{\pgfqpoint{1.297262in}{2.095779in}}{\pgfqpoint{1.292872in}{2.106378in}}{\pgfqpoint{1.285058in}{2.114191in}}%
\pgfpathcurveto{\pgfqpoint{1.277244in}{2.122005in}}{\pgfqpoint{1.266645in}{2.126395in}}{\pgfqpoint{1.255595in}{2.126395in}}%
\pgfpathcurveto{\pgfqpoint{1.244545in}{2.126395in}}{\pgfqpoint{1.233946in}{2.122005in}}{\pgfqpoint{1.226132in}{2.114191in}}%
\pgfpathcurveto{\pgfqpoint{1.218319in}{2.106378in}}{\pgfqpoint{1.213928in}{2.095779in}}{\pgfqpoint{1.213928in}{2.084728in}}%
\pgfpathcurveto{\pgfqpoint{1.213928in}{2.073678in}}{\pgfqpoint{1.218319in}{2.063079in}}{\pgfqpoint{1.226132in}{2.055266in}}%
\pgfpathcurveto{\pgfqpoint{1.233946in}{2.047452in}}{\pgfqpoint{1.244545in}{2.043062in}}{\pgfqpoint{1.255595in}{2.043062in}}%
\pgfpathlineto{\pgfqpoint{1.255595in}{2.043062in}}%
\pgfpathclose%
\pgfusepath{stroke}%
\end{pgfscope}%
\begin{pgfscope}%
\pgfpathrectangle{\pgfqpoint{0.393053in}{0.375000in}}{\pgfqpoint{6.356833in}{5.175000in}}%
\pgfusepath{clip}%
\pgfsetbuttcap%
\pgfsetroundjoin%
\pgfsetlinewidth{1.003750pt}%
\definecolor{currentstroke}{rgb}{0.827451,0.827451,0.827451}%
\pgfsetstrokecolor{currentstroke}%
\pgfsetdash{}{0pt}%
\pgfpathmoveto{\pgfqpoint{0.580211in}{3.378166in}}%
\pgfpathcurveto{\pgfqpoint{0.591261in}{3.378166in}}{\pgfqpoint{0.601860in}{3.382556in}}{\pgfqpoint{0.609674in}{3.390370in}}%
\pgfpathcurveto{\pgfqpoint{0.617487in}{3.398183in}}{\pgfqpoint{0.621878in}{3.408782in}}{\pgfqpoint{0.621878in}{3.419833in}}%
\pgfpathcurveto{\pgfqpoint{0.621878in}{3.430883in}}{\pgfqpoint{0.617487in}{3.441482in}}{\pgfqpoint{0.609674in}{3.449295in}}%
\pgfpathcurveto{\pgfqpoint{0.601860in}{3.457109in}}{\pgfqpoint{0.591261in}{3.461499in}}{\pgfqpoint{0.580211in}{3.461499in}}%
\pgfpathcurveto{\pgfqpoint{0.569161in}{3.461499in}}{\pgfqpoint{0.558562in}{3.457109in}}{\pgfqpoint{0.550748in}{3.449295in}}%
\pgfpathcurveto{\pgfqpoint{0.542935in}{3.441482in}}{\pgfqpoint{0.538544in}{3.430883in}}{\pgfqpoint{0.538544in}{3.419833in}}%
\pgfpathcurveto{\pgfqpoint{0.538544in}{3.408782in}}{\pgfqpoint{0.542935in}{3.398183in}}{\pgfqpoint{0.550748in}{3.390370in}}%
\pgfpathcurveto{\pgfqpoint{0.558562in}{3.382556in}}{\pgfqpoint{0.569161in}{3.378166in}}{\pgfqpoint{0.580211in}{3.378166in}}%
\pgfpathlineto{\pgfqpoint{0.580211in}{3.378166in}}%
\pgfpathclose%
\pgfusepath{stroke}%
\end{pgfscope}%
\begin{pgfscope}%
\pgfpathrectangle{\pgfqpoint{0.393053in}{0.375000in}}{\pgfqpoint{6.356833in}{5.175000in}}%
\pgfusepath{clip}%
\pgfsetbuttcap%
\pgfsetroundjoin%
\pgfsetlinewidth{1.003750pt}%
\definecolor{currentstroke}{rgb}{0.827451,0.827451,0.827451}%
\pgfsetstrokecolor{currentstroke}%
\pgfsetdash{}{0pt}%
\pgfpathmoveto{\pgfqpoint{3.893477in}{0.565646in}}%
\pgfpathcurveto{\pgfqpoint{3.904527in}{0.565646in}}{\pgfqpoint{3.915126in}{0.570036in}}{\pgfqpoint{3.922939in}{0.577850in}}%
\pgfpathcurveto{\pgfqpoint{3.930753in}{0.585664in}}{\pgfqpoint{3.935143in}{0.596263in}}{\pgfqpoint{3.935143in}{0.607313in}}%
\pgfpathcurveto{\pgfqpoint{3.935143in}{0.618363in}}{\pgfqpoint{3.930753in}{0.628962in}}{\pgfqpoint{3.922939in}{0.636776in}}%
\pgfpathcurveto{\pgfqpoint{3.915126in}{0.644589in}}{\pgfqpoint{3.904527in}{0.648980in}}{\pgfqpoint{3.893477in}{0.648980in}}%
\pgfpathcurveto{\pgfqpoint{3.882426in}{0.648980in}}{\pgfqpoint{3.871827in}{0.644589in}}{\pgfqpoint{3.864014in}{0.636776in}}%
\pgfpathcurveto{\pgfqpoint{3.856200in}{0.628962in}}{\pgfqpoint{3.851810in}{0.618363in}}{\pgfqpoint{3.851810in}{0.607313in}}%
\pgfpathcurveto{\pgfqpoint{3.851810in}{0.596263in}}{\pgfqpoint{3.856200in}{0.585664in}}{\pgfqpoint{3.864014in}{0.577850in}}%
\pgfpathcurveto{\pgfqpoint{3.871827in}{0.570036in}}{\pgfqpoint{3.882426in}{0.565646in}}{\pgfqpoint{3.893477in}{0.565646in}}%
\pgfpathlineto{\pgfqpoint{3.893477in}{0.565646in}}%
\pgfpathclose%
\pgfusepath{stroke}%
\end{pgfscope}%
\begin{pgfscope}%
\pgfpathrectangle{\pgfqpoint{0.393053in}{0.375000in}}{\pgfqpoint{6.356833in}{5.175000in}}%
\pgfusepath{clip}%
\pgfsetbuttcap%
\pgfsetroundjoin%
\pgfsetlinewidth{1.003750pt}%
\definecolor{currentstroke}{rgb}{0.827451,0.827451,0.827451}%
\pgfsetstrokecolor{currentstroke}%
\pgfsetdash{}{0pt}%
\pgfpathmoveto{\pgfqpoint{4.898856in}{0.391247in}}%
\pgfpathcurveto{\pgfqpoint{4.909906in}{0.391247in}}{\pgfqpoint{4.920505in}{0.395637in}}{\pgfqpoint{4.928319in}{0.403450in}}%
\pgfpathcurveto{\pgfqpoint{4.936132in}{0.411264in}}{\pgfqpoint{4.940523in}{0.421863in}}{\pgfqpoint{4.940523in}{0.432913in}}%
\pgfpathcurveto{\pgfqpoint{4.940523in}{0.443963in}}{\pgfqpoint{4.936132in}{0.454562in}}{\pgfqpoint{4.928319in}{0.462376in}}%
\pgfpathcurveto{\pgfqpoint{4.920505in}{0.470190in}}{\pgfqpoint{4.909906in}{0.474580in}}{\pgfqpoint{4.898856in}{0.474580in}}%
\pgfpathcurveto{\pgfqpoint{4.887806in}{0.474580in}}{\pgfqpoint{4.877207in}{0.470190in}}{\pgfqpoint{4.869393in}{0.462376in}}%
\pgfpathcurveto{\pgfqpoint{4.861580in}{0.454562in}}{\pgfqpoint{4.857189in}{0.443963in}}{\pgfqpoint{4.857189in}{0.432913in}}%
\pgfpathcurveto{\pgfqpoint{4.857189in}{0.421863in}}{\pgfqpoint{4.861580in}{0.411264in}}{\pgfqpoint{4.869393in}{0.403450in}}%
\pgfpathcurveto{\pgfqpoint{4.877207in}{0.395637in}}{\pgfqpoint{4.887806in}{0.391247in}}{\pgfqpoint{4.898856in}{0.391247in}}%
\pgfpathlineto{\pgfqpoint{4.898856in}{0.391247in}}%
\pgfpathclose%
\pgfusepath{stroke}%
\end{pgfscope}%
\begin{pgfscope}%
\pgfpathrectangle{\pgfqpoint{0.393053in}{0.375000in}}{\pgfqpoint{6.356833in}{5.175000in}}%
\pgfusepath{clip}%
\pgfsetbuttcap%
\pgfsetroundjoin%
\pgfsetlinewidth{1.003750pt}%
\definecolor{currentstroke}{rgb}{0.827451,0.827451,0.827451}%
\pgfsetstrokecolor{currentstroke}%
\pgfsetdash{}{0pt}%
\pgfpathmoveto{\pgfqpoint{1.843211in}{1.515821in}}%
\pgfpathcurveto{\pgfqpoint{1.854261in}{1.515821in}}{\pgfqpoint{1.864860in}{1.520211in}}{\pgfqpoint{1.872674in}{1.528025in}}%
\pgfpathcurveto{\pgfqpoint{1.880487in}{1.535838in}}{\pgfqpoint{1.884878in}{1.546437in}}{\pgfqpoint{1.884878in}{1.557487in}}%
\pgfpathcurveto{\pgfqpoint{1.884878in}{1.568537in}}{\pgfqpoint{1.880487in}{1.579136in}}{\pgfqpoint{1.872674in}{1.586950in}}%
\pgfpathcurveto{\pgfqpoint{1.864860in}{1.594764in}}{\pgfqpoint{1.854261in}{1.599154in}}{\pgfqpoint{1.843211in}{1.599154in}}%
\pgfpathcurveto{\pgfqpoint{1.832161in}{1.599154in}}{\pgfqpoint{1.821562in}{1.594764in}}{\pgfqpoint{1.813748in}{1.586950in}}%
\pgfpathcurveto{\pgfqpoint{1.805934in}{1.579136in}}{\pgfqpoint{1.801544in}{1.568537in}}{\pgfqpoint{1.801544in}{1.557487in}}%
\pgfpathcurveto{\pgfqpoint{1.801544in}{1.546437in}}{\pgfqpoint{1.805934in}{1.535838in}}{\pgfqpoint{1.813748in}{1.528025in}}%
\pgfpathcurveto{\pgfqpoint{1.821562in}{1.520211in}}{\pgfqpoint{1.832161in}{1.515821in}}{\pgfqpoint{1.843211in}{1.515821in}}%
\pgfpathlineto{\pgfqpoint{1.843211in}{1.515821in}}%
\pgfpathclose%
\pgfusepath{stroke}%
\end{pgfscope}%
\begin{pgfscope}%
\pgfpathrectangle{\pgfqpoint{0.393053in}{0.375000in}}{\pgfqpoint{6.356833in}{5.175000in}}%
\pgfusepath{clip}%
\pgfsetbuttcap%
\pgfsetroundjoin%
\pgfsetlinewidth{1.003750pt}%
\definecolor{currentstroke}{rgb}{0.827451,0.827451,0.827451}%
\pgfsetstrokecolor{currentstroke}%
\pgfsetdash{}{0pt}%
\pgfpathmoveto{\pgfqpoint{1.680974in}{1.628980in}}%
\pgfpathcurveto{\pgfqpoint{1.692024in}{1.628980in}}{\pgfqpoint{1.702623in}{1.633370in}}{\pgfqpoint{1.710437in}{1.641184in}}%
\pgfpathcurveto{\pgfqpoint{1.718250in}{1.648997in}}{\pgfqpoint{1.722641in}{1.659596in}}{\pgfqpoint{1.722641in}{1.670646in}}%
\pgfpathcurveto{\pgfqpoint{1.722641in}{1.681697in}}{\pgfqpoint{1.718250in}{1.692296in}}{\pgfqpoint{1.710437in}{1.700109in}}%
\pgfpathcurveto{\pgfqpoint{1.702623in}{1.707923in}}{\pgfqpoint{1.692024in}{1.712313in}}{\pgfqpoint{1.680974in}{1.712313in}}%
\pgfpathcurveto{\pgfqpoint{1.669924in}{1.712313in}}{\pgfqpoint{1.659325in}{1.707923in}}{\pgfqpoint{1.651511in}{1.700109in}}%
\pgfpathcurveto{\pgfqpoint{1.643698in}{1.692296in}}{\pgfqpoint{1.639307in}{1.681697in}}{\pgfqpoint{1.639307in}{1.670646in}}%
\pgfpathcurveto{\pgfqpoint{1.639307in}{1.659596in}}{\pgfqpoint{1.643698in}{1.648997in}}{\pgfqpoint{1.651511in}{1.641184in}}%
\pgfpathcurveto{\pgfqpoint{1.659325in}{1.633370in}}{\pgfqpoint{1.669924in}{1.628980in}}{\pgfqpoint{1.680974in}{1.628980in}}%
\pgfpathlineto{\pgfqpoint{1.680974in}{1.628980in}}%
\pgfpathclose%
\pgfusepath{stroke}%
\end{pgfscope}%
\begin{pgfscope}%
\pgfpathrectangle{\pgfqpoint{0.393053in}{0.375000in}}{\pgfqpoint{6.356833in}{5.175000in}}%
\pgfusepath{clip}%
\pgfsetbuttcap%
\pgfsetroundjoin%
\pgfsetlinewidth{1.003750pt}%
\definecolor{currentstroke}{rgb}{0.827451,0.827451,0.827451}%
\pgfsetstrokecolor{currentstroke}%
\pgfsetdash{}{0pt}%
\pgfpathmoveto{\pgfqpoint{0.652600in}{3.123417in}}%
\pgfpathcurveto{\pgfqpoint{0.663650in}{3.123417in}}{\pgfqpoint{0.674249in}{3.127808in}}{\pgfqpoint{0.682062in}{3.135621in}}%
\pgfpathcurveto{\pgfqpoint{0.689876in}{3.143435in}}{\pgfqpoint{0.694266in}{3.154034in}}{\pgfqpoint{0.694266in}{3.165084in}}%
\pgfpathcurveto{\pgfqpoint{0.694266in}{3.176134in}}{\pgfqpoint{0.689876in}{3.186733in}}{\pgfqpoint{0.682062in}{3.194547in}}%
\pgfpathcurveto{\pgfqpoint{0.674249in}{3.202360in}}{\pgfqpoint{0.663650in}{3.206751in}}{\pgfqpoint{0.652600in}{3.206751in}}%
\pgfpathcurveto{\pgfqpoint{0.641549in}{3.206751in}}{\pgfqpoint{0.630950in}{3.202360in}}{\pgfqpoint{0.623137in}{3.194547in}}%
\pgfpathcurveto{\pgfqpoint{0.615323in}{3.186733in}}{\pgfqpoint{0.610933in}{3.176134in}}{\pgfqpoint{0.610933in}{3.165084in}}%
\pgfpathcurveto{\pgfqpoint{0.610933in}{3.154034in}}{\pgfqpoint{0.615323in}{3.143435in}}{\pgfqpoint{0.623137in}{3.135621in}}%
\pgfpathcurveto{\pgfqpoint{0.630950in}{3.127808in}}{\pgfqpoint{0.641549in}{3.123417in}}{\pgfqpoint{0.652600in}{3.123417in}}%
\pgfpathlineto{\pgfqpoint{0.652600in}{3.123417in}}%
\pgfpathclose%
\pgfusepath{stroke}%
\end{pgfscope}%
\begin{pgfscope}%
\pgfpathrectangle{\pgfqpoint{0.393053in}{0.375000in}}{\pgfqpoint{6.356833in}{5.175000in}}%
\pgfusepath{clip}%
\pgfsetbuttcap%
\pgfsetroundjoin%
\pgfsetlinewidth{1.003750pt}%
\definecolor{currentstroke}{rgb}{0.827451,0.827451,0.827451}%
\pgfsetstrokecolor{currentstroke}%
\pgfsetdash{}{0pt}%
\pgfpathmoveto{\pgfqpoint{0.878661in}{2.805605in}}%
\pgfpathcurveto{\pgfqpoint{0.889711in}{2.805605in}}{\pgfqpoint{0.900310in}{2.809995in}}{\pgfqpoint{0.908123in}{2.817809in}}%
\pgfpathcurveto{\pgfqpoint{0.915937in}{2.825623in}}{\pgfqpoint{0.920327in}{2.836222in}}{\pgfqpoint{0.920327in}{2.847272in}}%
\pgfpathcurveto{\pgfqpoint{0.920327in}{2.858322in}}{\pgfqpoint{0.915937in}{2.868921in}}{\pgfqpoint{0.908123in}{2.876735in}}%
\pgfpathcurveto{\pgfqpoint{0.900310in}{2.884548in}}{\pgfqpoint{0.889711in}{2.888938in}}{\pgfqpoint{0.878661in}{2.888938in}}%
\pgfpathcurveto{\pgfqpoint{0.867610in}{2.888938in}}{\pgfqpoint{0.857011in}{2.884548in}}{\pgfqpoint{0.849198in}{2.876735in}}%
\pgfpathcurveto{\pgfqpoint{0.841384in}{2.868921in}}{\pgfqpoint{0.836994in}{2.858322in}}{\pgfqpoint{0.836994in}{2.847272in}}%
\pgfpathcurveto{\pgfqpoint{0.836994in}{2.836222in}}{\pgfqpoint{0.841384in}{2.825623in}}{\pgfqpoint{0.849198in}{2.817809in}}%
\pgfpathcurveto{\pgfqpoint{0.857011in}{2.809995in}}{\pgfqpoint{0.867610in}{2.805605in}}{\pgfqpoint{0.878661in}{2.805605in}}%
\pgfpathlineto{\pgfqpoint{0.878661in}{2.805605in}}%
\pgfpathclose%
\pgfusepath{stroke}%
\end{pgfscope}%
\begin{pgfscope}%
\pgfpathrectangle{\pgfqpoint{0.393053in}{0.375000in}}{\pgfqpoint{6.356833in}{5.175000in}}%
\pgfusepath{clip}%
\pgfsetbuttcap%
\pgfsetroundjoin%
\pgfsetlinewidth{1.003750pt}%
\definecolor{currentstroke}{rgb}{0.827451,0.827451,0.827451}%
\pgfsetstrokecolor{currentstroke}%
\pgfsetdash{}{0pt}%
\pgfpathmoveto{\pgfqpoint{1.006155in}{2.578482in}}%
\pgfpathcurveto{\pgfqpoint{1.017205in}{2.578482in}}{\pgfqpoint{1.027804in}{2.582873in}}{\pgfqpoint{1.035618in}{2.590686in}}%
\pgfpathcurveto{\pgfqpoint{1.043431in}{2.598500in}}{\pgfqpoint{1.047821in}{2.609099in}}{\pgfqpoint{1.047821in}{2.620149in}}%
\pgfpathcurveto{\pgfqpoint{1.047821in}{2.631199in}}{\pgfqpoint{1.043431in}{2.641798in}}{\pgfqpoint{1.035618in}{2.649612in}}%
\pgfpathcurveto{\pgfqpoint{1.027804in}{2.657426in}}{\pgfqpoint{1.017205in}{2.661816in}}{\pgfqpoint{1.006155in}{2.661816in}}%
\pgfpathcurveto{\pgfqpoint{0.995105in}{2.661816in}}{\pgfqpoint{0.984506in}{2.657426in}}{\pgfqpoint{0.976692in}{2.649612in}}%
\pgfpathcurveto{\pgfqpoint{0.968878in}{2.641798in}}{\pgfqpoint{0.964488in}{2.631199in}}{\pgfqpoint{0.964488in}{2.620149in}}%
\pgfpathcurveto{\pgfqpoint{0.964488in}{2.609099in}}{\pgfqpoint{0.968878in}{2.598500in}}{\pgfqpoint{0.976692in}{2.590686in}}%
\pgfpathcurveto{\pgfqpoint{0.984506in}{2.582873in}}{\pgfqpoint{0.995105in}{2.578482in}}{\pgfqpoint{1.006155in}{2.578482in}}%
\pgfpathlineto{\pgfqpoint{1.006155in}{2.578482in}}%
\pgfpathclose%
\pgfusepath{stroke}%
\end{pgfscope}%
\begin{pgfscope}%
\pgfpathrectangle{\pgfqpoint{0.393053in}{0.375000in}}{\pgfqpoint{6.356833in}{5.175000in}}%
\pgfusepath{clip}%
\pgfsetbuttcap%
\pgfsetroundjoin%
\pgfsetlinewidth{1.003750pt}%
\definecolor{currentstroke}{rgb}{0.827451,0.827451,0.827451}%
\pgfsetstrokecolor{currentstroke}%
\pgfsetdash{}{0pt}%
\pgfpathmoveto{\pgfqpoint{4.927239in}{0.379634in}}%
\pgfpathcurveto{\pgfqpoint{4.938289in}{0.379634in}}{\pgfqpoint{4.948888in}{0.384025in}}{\pgfqpoint{4.956702in}{0.391838in}}%
\pgfpathcurveto{\pgfqpoint{4.964516in}{0.399652in}}{\pgfqpoint{4.968906in}{0.410251in}}{\pgfqpoint{4.968906in}{0.421301in}}%
\pgfpathcurveto{\pgfqpoint{4.968906in}{0.432351in}}{\pgfqpoint{4.964516in}{0.442950in}}{\pgfqpoint{4.956702in}{0.450764in}}%
\pgfpathcurveto{\pgfqpoint{4.948888in}{0.458578in}}{\pgfqpoint{4.938289in}{0.462968in}}{\pgfqpoint{4.927239in}{0.462968in}}%
\pgfpathcurveto{\pgfqpoint{4.916189in}{0.462968in}}{\pgfqpoint{4.905590in}{0.458578in}}{\pgfqpoint{4.897776in}{0.450764in}}%
\pgfpathcurveto{\pgfqpoint{4.889963in}{0.442950in}}{\pgfqpoint{4.885572in}{0.432351in}}{\pgfqpoint{4.885572in}{0.421301in}}%
\pgfpathcurveto{\pgfqpoint{4.885572in}{0.410251in}}{\pgfqpoint{4.889963in}{0.399652in}}{\pgfqpoint{4.897776in}{0.391838in}}%
\pgfpathcurveto{\pgfqpoint{4.905590in}{0.384025in}}{\pgfqpoint{4.916189in}{0.379634in}}{\pgfqpoint{4.927239in}{0.379634in}}%
\pgfpathlineto{\pgfqpoint{4.927239in}{0.379634in}}%
\pgfpathclose%
\pgfusepath{stroke}%
\end{pgfscope}%
\begin{pgfscope}%
\pgfpathrectangle{\pgfqpoint{0.393053in}{0.375000in}}{\pgfqpoint{6.356833in}{5.175000in}}%
\pgfusepath{clip}%
\pgfsetbuttcap%
\pgfsetroundjoin%
\pgfsetlinewidth{1.003750pt}%
\definecolor{currentstroke}{rgb}{0.827451,0.827451,0.827451}%
\pgfsetstrokecolor{currentstroke}%
\pgfsetdash{}{0pt}%
\pgfpathmoveto{\pgfqpoint{1.291595in}{1.999205in}}%
\pgfpathcurveto{\pgfqpoint{1.302646in}{1.999205in}}{\pgfqpoint{1.313245in}{2.003595in}}{\pgfqpoint{1.321058in}{2.011409in}}%
\pgfpathcurveto{\pgfqpoint{1.328872in}{2.019222in}}{\pgfqpoint{1.333262in}{2.029821in}}{\pgfqpoint{1.333262in}{2.040872in}}%
\pgfpathcurveto{\pgfqpoint{1.333262in}{2.051922in}}{\pgfqpoint{1.328872in}{2.062521in}}{\pgfqpoint{1.321058in}{2.070334in}}%
\pgfpathcurveto{\pgfqpoint{1.313245in}{2.078148in}}{\pgfqpoint{1.302646in}{2.082538in}}{\pgfqpoint{1.291595in}{2.082538in}}%
\pgfpathcurveto{\pgfqpoint{1.280545in}{2.082538in}}{\pgfqpoint{1.269946in}{2.078148in}}{\pgfqpoint{1.262133in}{2.070334in}}%
\pgfpathcurveto{\pgfqpoint{1.254319in}{2.062521in}}{\pgfqpoint{1.249929in}{2.051922in}}{\pgfqpoint{1.249929in}{2.040872in}}%
\pgfpathcurveto{\pgfqpoint{1.249929in}{2.029821in}}{\pgfqpoint{1.254319in}{2.019222in}}{\pgfqpoint{1.262133in}{2.011409in}}%
\pgfpathcurveto{\pgfqpoint{1.269946in}{2.003595in}}{\pgfqpoint{1.280545in}{1.999205in}}{\pgfqpoint{1.291595in}{1.999205in}}%
\pgfpathlineto{\pgfqpoint{1.291595in}{1.999205in}}%
\pgfpathclose%
\pgfusepath{stroke}%
\end{pgfscope}%
\begin{pgfscope}%
\pgfpathrectangle{\pgfqpoint{0.393053in}{0.375000in}}{\pgfqpoint{6.356833in}{5.175000in}}%
\pgfusepath{clip}%
\pgfsetbuttcap%
\pgfsetroundjoin%
\pgfsetlinewidth{1.003750pt}%
\definecolor{currentstroke}{rgb}{0.827451,0.827451,0.827451}%
\pgfsetstrokecolor{currentstroke}%
\pgfsetdash{}{0pt}%
\pgfpathmoveto{\pgfqpoint{1.702004in}{1.599430in}}%
\pgfpathcurveto{\pgfqpoint{1.713054in}{1.599430in}}{\pgfqpoint{1.723653in}{1.603820in}}{\pgfqpoint{1.731467in}{1.611633in}}%
\pgfpathcurveto{\pgfqpoint{1.739281in}{1.619447in}}{\pgfqpoint{1.743671in}{1.630046in}}{\pgfqpoint{1.743671in}{1.641096in}}%
\pgfpathcurveto{\pgfqpoint{1.743671in}{1.652146in}}{\pgfqpoint{1.739281in}{1.662745in}}{\pgfqpoint{1.731467in}{1.670559in}}%
\pgfpathcurveto{\pgfqpoint{1.723653in}{1.678373in}}{\pgfqpoint{1.713054in}{1.682763in}}{\pgfqpoint{1.702004in}{1.682763in}}%
\pgfpathcurveto{\pgfqpoint{1.690954in}{1.682763in}}{\pgfqpoint{1.680355in}{1.678373in}}{\pgfqpoint{1.672541in}{1.670559in}}%
\pgfpathcurveto{\pgfqpoint{1.664728in}{1.662745in}}{\pgfqpoint{1.660338in}{1.652146in}}{\pgfqpoint{1.660338in}{1.641096in}}%
\pgfpathcurveto{\pgfqpoint{1.660338in}{1.630046in}}{\pgfqpoint{1.664728in}{1.619447in}}{\pgfqpoint{1.672541in}{1.611633in}}%
\pgfpathcurveto{\pgfqpoint{1.680355in}{1.603820in}}{\pgfqpoint{1.690954in}{1.599430in}}{\pgfqpoint{1.702004in}{1.599430in}}%
\pgfpathlineto{\pgfqpoint{1.702004in}{1.599430in}}%
\pgfpathclose%
\pgfusepath{stroke}%
\end{pgfscope}%
\begin{pgfscope}%
\pgfpathrectangle{\pgfqpoint{0.393053in}{0.375000in}}{\pgfqpoint{6.356833in}{5.175000in}}%
\pgfusepath{clip}%
\pgfsetbuttcap%
\pgfsetroundjoin%
\pgfsetlinewidth{1.003750pt}%
\definecolor{currentstroke}{rgb}{0.827451,0.827451,0.827451}%
\pgfsetstrokecolor{currentstroke}%
\pgfsetdash{}{0pt}%
\pgfpathmoveto{\pgfqpoint{1.008971in}{2.512431in}}%
\pgfpathcurveto{\pgfqpoint{1.020021in}{2.512431in}}{\pgfqpoint{1.030620in}{2.516821in}}{\pgfqpoint{1.038434in}{2.524635in}}%
\pgfpathcurveto{\pgfqpoint{1.046247in}{2.532448in}}{\pgfqpoint{1.050637in}{2.543047in}}{\pgfqpoint{1.050637in}{2.554097in}}%
\pgfpathcurveto{\pgfqpoint{1.050637in}{2.565148in}}{\pgfqpoint{1.046247in}{2.575747in}}{\pgfqpoint{1.038434in}{2.583560in}}%
\pgfpathcurveto{\pgfqpoint{1.030620in}{2.591374in}}{\pgfqpoint{1.020021in}{2.595764in}}{\pgfqpoint{1.008971in}{2.595764in}}%
\pgfpathcurveto{\pgfqpoint{0.997921in}{2.595764in}}{\pgfqpoint{0.987322in}{2.591374in}}{\pgfqpoint{0.979508in}{2.583560in}}%
\pgfpathcurveto{\pgfqpoint{0.971694in}{2.575747in}}{\pgfqpoint{0.967304in}{2.565148in}}{\pgfqpoint{0.967304in}{2.554097in}}%
\pgfpathcurveto{\pgfqpoint{0.967304in}{2.543047in}}{\pgfqpoint{0.971694in}{2.532448in}}{\pgfqpoint{0.979508in}{2.524635in}}%
\pgfpathcurveto{\pgfqpoint{0.987322in}{2.516821in}}{\pgfqpoint{0.997921in}{2.512431in}}{\pgfqpoint{1.008971in}{2.512431in}}%
\pgfpathlineto{\pgfqpoint{1.008971in}{2.512431in}}%
\pgfpathclose%
\pgfusepath{stroke}%
\end{pgfscope}%
\begin{pgfscope}%
\pgfpathrectangle{\pgfqpoint{0.393053in}{0.375000in}}{\pgfqpoint{6.356833in}{5.175000in}}%
\pgfusepath{clip}%
\pgfsetbuttcap%
\pgfsetroundjoin%
\pgfsetlinewidth{1.003750pt}%
\definecolor{currentstroke}{rgb}{0.827451,0.827451,0.827451}%
\pgfsetstrokecolor{currentstroke}%
\pgfsetdash{}{0pt}%
\pgfpathmoveto{\pgfqpoint{0.583953in}{3.273995in}}%
\pgfpathcurveto{\pgfqpoint{0.595003in}{3.273995in}}{\pgfqpoint{0.605602in}{3.278385in}}{\pgfqpoint{0.613416in}{3.286198in}}%
\pgfpathcurveto{\pgfqpoint{0.621229in}{3.294012in}}{\pgfqpoint{0.625620in}{3.304611in}}{\pgfqpoint{0.625620in}{3.315661in}}%
\pgfpathcurveto{\pgfqpoint{0.625620in}{3.326711in}}{\pgfqpoint{0.621229in}{3.337310in}}{\pgfqpoint{0.613416in}{3.345124in}}%
\pgfpathcurveto{\pgfqpoint{0.605602in}{3.352938in}}{\pgfqpoint{0.595003in}{3.357328in}}{\pgfqpoint{0.583953in}{3.357328in}}%
\pgfpathcurveto{\pgfqpoint{0.572903in}{3.357328in}}{\pgfqpoint{0.562304in}{3.352938in}}{\pgfqpoint{0.554490in}{3.345124in}}%
\pgfpathcurveto{\pgfqpoint{0.546677in}{3.337310in}}{\pgfqpoint{0.542286in}{3.326711in}}{\pgfqpoint{0.542286in}{3.315661in}}%
\pgfpathcurveto{\pgfqpoint{0.542286in}{3.304611in}}{\pgfqpoint{0.546677in}{3.294012in}}{\pgfqpoint{0.554490in}{3.286198in}}%
\pgfpathcurveto{\pgfqpoint{0.562304in}{3.278385in}}{\pgfqpoint{0.572903in}{3.273995in}}{\pgfqpoint{0.583953in}{3.273995in}}%
\pgfpathlineto{\pgfqpoint{0.583953in}{3.273995in}}%
\pgfpathclose%
\pgfusepath{stroke}%
\end{pgfscope}%
\begin{pgfscope}%
\pgfpathrectangle{\pgfqpoint{0.393053in}{0.375000in}}{\pgfqpoint{6.356833in}{5.175000in}}%
\pgfusepath{clip}%
\pgfsetbuttcap%
\pgfsetroundjoin%
\pgfsetlinewidth{1.003750pt}%
\definecolor{currentstroke}{rgb}{0.827451,0.827451,0.827451}%
\pgfsetstrokecolor{currentstroke}%
\pgfsetdash{}{0pt}%
\pgfpathmoveto{\pgfqpoint{1.224501in}{2.087879in}}%
\pgfpathcurveto{\pgfqpoint{1.235551in}{2.087879in}}{\pgfqpoint{1.246150in}{2.092269in}}{\pgfqpoint{1.253963in}{2.100083in}}%
\pgfpathcurveto{\pgfqpoint{1.261777in}{2.107896in}}{\pgfqpoint{1.266167in}{2.118495in}}{\pgfqpoint{1.266167in}{2.129546in}}%
\pgfpathcurveto{\pgfqpoint{1.266167in}{2.140596in}}{\pgfqpoint{1.261777in}{2.151195in}}{\pgfqpoint{1.253963in}{2.159008in}}%
\pgfpathcurveto{\pgfqpoint{1.246150in}{2.166822in}}{\pgfqpoint{1.235551in}{2.171212in}}{\pgfqpoint{1.224501in}{2.171212in}}%
\pgfpathcurveto{\pgfqpoint{1.213451in}{2.171212in}}{\pgfqpoint{1.202852in}{2.166822in}}{\pgfqpoint{1.195038in}{2.159008in}}%
\pgfpathcurveto{\pgfqpoint{1.187224in}{2.151195in}}{\pgfqpoint{1.182834in}{2.140596in}}{\pgfqpoint{1.182834in}{2.129546in}}%
\pgfpathcurveto{\pgfqpoint{1.182834in}{2.118495in}}{\pgfqpoint{1.187224in}{2.107896in}}{\pgfqpoint{1.195038in}{2.100083in}}%
\pgfpathcurveto{\pgfqpoint{1.202852in}{2.092269in}}{\pgfqpoint{1.213451in}{2.087879in}}{\pgfqpoint{1.224501in}{2.087879in}}%
\pgfpathlineto{\pgfqpoint{1.224501in}{2.087879in}}%
\pgfpathclose%
\pgfusepath{stroke}%
\end{pgfscope}%
\begin{pgfscope}%
\pgfpathrectangle{\pgfqpoint{0.393053in}{0.375000in}}{\pgfqpoint{6.356833in}{5.175000in}}%
\pgfusepath{clip}%
\pgfsetbuttcap%
\pgfsetroundjoin%
\pgfsetlinewidth{1.003750pt}%
\definecolor{currentstroke}{rgb}{0.827451,0.827451,0.827451}%
\pgfsetstrokecolor{currentstroke}%
\pgfsetdash{}{0pt}%
\pgfpathmoveto{\pgfqpoint{4.169452in}{0.483485in}}%
\pgfpathcurveto{\pgfqpoint{4.180503in}{0.483485in}}{\pgfqpoint{4.191102in}{0.487876in}}{\pgfqpoint{4.198915in}{0.495689in}}%
\pgfpathcurveto{\pgfqpoint{4.206729in}{0.503503in}}{\pgfqpoint{4.211119in}{0.514102in}}{\pgfqpoint{4.211119in}{0.525152in}}%
\pgfpathcurveto{\pgfqpoint{4.211119in}{0.536202in}}{\pgfqpoint{4.206729in}{0.546801in}}{\pgfqpoint{4.198915in}{0.554615in}}%
\pgfpathcurveto{\pgfqpoint{4.191102in}{0.562429in}}{\pgfqpoint{4.180503in}{0.566819in}}{\pgfqpoint{4.169452in}{0.566819in}}%
\pgfpathcurveto{\pgfqpoint{4.158402in}{0.566819in}}{\pgfqpoint{4.147803in}{0.562429in}}{\pgfqpoint{4.139990in}{0.554615in}}%
\pgfpathcurveto{\pgfqpoint{4.132176in}{0.546801in}}{\pgfqpoint{4.127786in}{0.536202in}}{\pgfqpoint{4.127786in}{0.525152in}}%
\pgfpathcurveto{\pgfqpoint{4.127786in}{0.514102in}}{\pgfqpoint{4.132176in}{0.503503in}}{\pgfqpoint{4.139990in}{0.495689in}}%
\pgfpathcurveto{\pgfqpoint{4.147803in}{0.487876in}}{\pgfqpoint{4.158402in}{0.483485in}}{\pgfqpoint{4.169452in}{0.483485in}}%
\pgfpathlineto{\pgfqpoint{4.169452in}{0.483485in}}%
\pgfpathclose%
\pgfusepath{stroke}%
\end{pgfscope}%
\begin{pgfscope}%
\pgfpathrectangle{\pgfqpoint{0.393053in}{0.375000in}}{\pgfqpoint{6.356833in}{5.175000in}}%
\pgfusepath{clip}%
\pgfsetbuttcap%
\pgfsetroundjoin%
\pgfsetlinewidth{1.003750pt}%
\definecolor{currentstroke}{rgb}{0.827451,0.827451,0.827451}%
\pgfsetstrokecolor{currentstroke}%
\pgfsetdash{}{0pt}%
\pgfpathmoveto{\pgfqpoint{1.195477in}{2.118618in}}%
\pgfpathcurveto{\pgfqpoint{1.206527in}{2.118618in}}{\pgfqpoint{1.217126in}{2.123008in}}{\pgfqpoint{1.224940in}{2.130821in}}%
\pgfpathcurveto{\pgfqpoint{1.232753in}{2.138635in}}{\pgfqpoint{1.237144in}{2.149234in}}{\pgfqpoint{1.237144in}{2.160284in}}%
\pgfpathcurveto{\pgfqpoint{1.237144in}{2.171334in}}{\pgfqpoint{1.232753in}{2.181933in}}{\pgfqpoint{1.224940in}{2.189747in}}%
\pgfpathcurveto{\pgfqpoint{1.217126in}{2.197561in}}{\pgfqpoint{1.206527in}{2.201951in}}{\pgfqpoint{1.195477in}{2.201951in}}%
\pgfpathcurveto{\pgfqpoint{1.184427in}{2.201951in}}{\pgfqpoint{1.173828in}{2.197561in}}{\pgfqpoint{1.166014in}{2.189747in}}%
\pgfpathcurveto{\pgfqpoint{1.158201in}{2.181933in}}{\pgfqpoint{1.153810in}{2.171334in}}{\pgfqpoint{1.153810in}{2.160284in}}%
\pgfpathcurveto{\pgfqpoint{1.153810in}{2.149234in}}{\pgfqpoint{1.158201in}{2.138635in}}{\pgfqpoint{1.166014in}{2.130821in}}%
\pgfpathcurveto{\pgfqpoint{1.173828in}{2.123008in}}{\pgfqpoint{1.184427in}{2.118618in}}{\pgfqpoint{1.195477in}{2.118618in}}%
\pgfpathlineto{\pgfqpoint{1.195477in}{2.118618in}}%
\pgfpathclose%
\pgfusepath{stroke}%
\end{pgfscope}%
\begin{pgfscope}%
\pgfpathrectangle{\pgfqpoint{0.393053in}{0.375000in}}{\pgfqpoint{6.356833in}{5.175000in}}%
\pgfusepath{clip}%
\pgfsetbuttcap%
\pgfsetroundjoin%
\pgfsetlinewidth{1.003750pt}%
\definecolor{currentstroke}{rgb}{0.827451,0.827451,0.827451}%
\pgfsetstrokecolor{currentstroke}%
\pgfsetdash{}{0pt}%
\pgfpathmoveto{\pgfqpoint{3.256902in}{0.730828in}}%
\pgfpathcurveto{\pgfqpoint{3.267953in}{0.730828in}}{\pgfqpoint{3.278552in}{0.735218in}}{\pgfqpoint{3.286365in}{0.743032in}}%
\pgfpathcurveto{\pgfqpoint{3.294179in}{0.750846in}}{\pgfqpoint{3.298569in}{0.761445in}}{\pgfqpoint{3.298569in}{0.772495in}}%
\pgfpathcurveto{\pgfqpoint{3.298569in}{0.783545in}}{\pgfqpoint{3.294179in}{0.794144in}}{\pgfqpoint{3.286365in}{0.801958in}}%
\pgfpathcurveto{\pgfqpoint{3.278552in}{0.809771in}}{\pgfqpoint{3.267953in}{0.814162in}}{\pgfqpoint{3.256902in}{0.814162in}}%
\pgfpathcurveto{\pgfqpoint{3.245852in}{0.814162in}}{\pgfqpoint{3.235253in}{0.809771in}}{\pgfqpoint{3.227440in}{0.801958in}}%
\pgfpathcurveto{\pgfqpoint{3.219626in}{0.794144in}}{\pgfqpoint{3.215236in}{0.783545in}}{\pgfqpoint{3.215236in}{0.772495in}}%
\pgfpathcurveto{\pgfqpoint{3.215236in}{0.761445in}}{\pgfqpoint{3.219626in}{0.750846in}}{\pgfqpoint{3.227440in}{0.743032in}}%
\pgfpathcurveto{\pgfqpoint{3.235253in}{0.735218in}}{\pgfqpoint{3.245852in}{0.730828in}}{\pgfqpoint{3.256902in}{0.730828in}}%
\pgfpathlineto{\pgfqpoint{3.256902in}{0.730828in}}%
\pgfpathclose%
\pgfusepath{stroke}%
\end{pgfscope}%
\begin{pgfscope}%
\pgfpathrectangle{\pgfqpoint{0.393053in}{0.375000in}}{\pgfqpoint{6.356833in}{5.175000in}}%
\pgfusepath{clip}%
\pgfsetbuttcap%
\pgfsetroundjoin%
\pgfsetlinewidth{1.003750pt}%
\definecolor{currentstroke}{rgb}{0.827451,0.827451,0.827451}%
\pgfsetstrokecolor{currentstroke}%
\pgfsetdash{}{0pt}%
\pgfpathmoveto{\pgfqpoint{1.980937in}{1.385443in}}%
\pgfpathcurveto{\pgfqpoint{1.991987in}{1.385443in}}{\pgfqpoint{2.002586in}{1.389833in}}{\pgfqpoint{2.010399in}{1.397647in}}%
\pgfpathcurveto{\pgfqpoint{2.018213in}{1.405460in}}{\pgfqpoint{2.022603in}{1.416059in}}{\pgfqpoint{2.022603in}{1.427109in}}%
\pgfpathcurveto{\pgfqpoint{2.022603in}{1.438160in}}{\pgfqpoint{2.018213in}{1.448759in}}{\pgfqpoint{2.010399in}{1.456572in}}%
\pgfpathcurveto{\pgfqpoint{2.002586in}{1.464386in}}{\pgfqpoint{1.991987in}{1.468776in}}{\pgfqpoint{1.980937in}{1.468776in}}%
\pgfpathcurveto{\pgfqpoint{1.969886in}{1.468776in}}{\pgfqpoint{1.959287in}{1.464386in}}{\pgfqpoint{1.951474in}{1.456572in}}%
\pgfpathcurveto{\pgfqpoint{1.943660in}{1.448759in}}{\pgfqpoint{1.939270in}{1.438160in}}{\pgfqpoint{1.939270in}{1.427109in}}%
\pgfpathcurveto{\pgfqpoint{1.939270in}{1.416059in}}{\pgfqpoint{1.943660in}{1.405460in}}{\pgfqpoint{1.951474in}{1.397647in}}%
\pgfpathcurveto{\pgfqpoint{1.959287in}{1.389833in}}{\pgfqpoint{1.969886in}{1.385443in}}{\pgfqpoint{1.980937in}{1.385443in}}%
\pgfpathlineto{\pgfqpoint{1.980937in}{1.385443in}}%
\pgfpathclose%
\pgfusepath{stroke}%
\end{pgfscope}%
\begin{pgfscope}%
\pgfpathrectangle{\pgfqpoint{0.393053in}{0.375000in}}{\pgfqpoint{6.356833in}{5.175000in}}%
\pgfusepath{clip}%
\pgfsetbuttcap%
\pgfsetroundjoin%
\pgfsetlinewidth{1.003750pt}%
\definecolor{currentstroke}{rgb}{0.827451,0.827451,0.827451}%
\pgfsetstrokecolor{currentstroke}%
\pgfsetdash{}{0pt}%
\pgfpathmoveto{\pgfqpoint{2.124283in}{1.277669in}}%
\pgfpathcurveto{\pgfqpoint{2.135334in}{1.277669in}}{\pgfqpoint{2.145933in}{1.282059in}}{\pgfqpoint{2.153746in}{1.289873in}}%
\pgfpathcurveto{\pgfqpoint{2.161560in}{1.297686in}}{\pgfqpoint{2.165950in}{1.308286in}}{\pgfqpoint{2.165950in}{1.319336in}}%
\pgfpathcurveto{\pgfqpoint{2.165950in}{1.330386in}}{\pgfqpoint{2.161560in}{1.340985in}}{\pgfqpoint{2.153746in}{1.348798in}}%
\pgfpathcurveto{\pgfqpoint{2.145933in}{1.356612in}}{\pgfqpoint{2.135334in}{1.361002in}}{\pgfqpoint{2.124283in}{1.361002in}}%
\pgfpathcurveto{\pgfqpoint{2.113233in}{1.361002in}}{\pgfqpoint{2.102634in}{1.356612in}}{\pgfqpoint{2.094821in}{1.348798in}}%
\pgfpathcurveto{\pgfqpoint{2.087007in}{1.340985in}}{\pgfqpoint{2.082617in}{1.330386in}}{\pgfqpoint{2.082617in}{1.319336in}}%
\pgfpathcurveto{\pgfqpoint{2.082617in}{1.308286in}}{\pgfqpoint{2.087007in}{1.297686in}}{\pgfqpoint{2.094821in}{1.289873in}}%
\pgfpathcurveto{\pgfqpoint{2.102634in}{1.282059in}}{\pgfqpoint{2.113233in}{1.277669in}}{\pgfqpoint{2.124283in}{1.277669in}}%
\pgfpathlineto{\pgfqpoint{2.124283in}{1.277669in}}%
\pgfpathclose%
\pgfusepath{stroke}%
\end{pgfscope}%
\begin{pgfscope}%
\pgfpathrectangle{\pgfqpoint{0.393053in}{0.375000in}}{\pgfqpoint{6.356833in}{5.175000in}}%
\pgfusepath{clip}%
\pgfsetbuttcap%
\pgfsetroundjoin%
\pgfsetlinewidth{1.003750pt}%
\definecolor{currentstroke}{rgb}{0.827451,0.827451,0.827451}%
\pgfsetstrokecolor{currentstroke}%
\pgfsetdash{}{0pt}%
\pgfpathmoveto{\pgfqpoint{1.050681in}{2.479907in}}%
\pgfpathcurveto{\pgfqpoint{1.061731in}{2.479907in}}{\pgfqpoint{1.072330in}{2.484297in}}{\pgfqpoint{1.080144in}{2.492111in}}%
\pgfpathcurveto{\pgfqpoint{1.087957in}{2.499925in}}{\pgfqpoint{1.092347in}{2.510524in}}{\pgfqpoint{1.092347in}{2.521574in}}%
\pgfpathcurveto{\pgfqpoint{1.092347in}{2.532624in}}{\pgfqpoint{1.087957in}{2.543223in}}{\pgfqpoint{1.080144in}{2.551037in}}%
\pgfpathcurveto{\pgfqpoint{1.072330in}{2.558850in}}{\pgfqpoint{1.061731in}{2.563240in}}{\pgfqpoint{1.050681in}{2.563240in}}%
\pgfpathcurveto{\pgfqpoint{1.039631in}{2.563240in}}{\pgfqpoint{1.029032in}{2.558850in}}{\pgfqpoint{1.021218in}{2.551037in}}%
\pgfpathcurveto{\pgfqpoint{1.013404in}{2.543223in}}{\pgfqpoint{1.009014in}{2.532624in}}{\pgfqpoint{1.009014in}{2.521574in}}%
\pgfpathcurveto{\pgfqpoint{1.009014in}{2.510524in}}{\pgfqpoint{1.013404in}{2.499925in}}{\pgfqpoint{1.021218in}{2.492111in}}%
\pgfpathcurveto{\pgfqpoint{1.029032in}{2.484297in}}{\pgfqpoint{1.039631in}{2.479907in}}{\pgfqpoint{1.050681in}{2.479907in}}%
\pgfpathlineto{\pgfqpoint{1.050681in}{2.479907in}}%
\pgfpathclose%
\pgfusepath{stroke}%
\end{pgfscope}%
\begin{pgfscope}%
\pgfpathrectangle{\pgfqpoint{0.393053in}{0.375000in}}{\pgfqpoint{6.356833in}{5.175000in}}%
\pgfusepath{clip}%
\pgfsetbuttcap%
\pgfsetroundjoin%
\pgfsetlinewidth{1.003750pt}%
\definecolor{currentstroke}{rgb}{0.827451,0.827451,0.827451}%
\pgfsetstrokecolor{currentstroke}%
\pgfsetdash{}{0pt}%
\pgfpathmoveto{\pgfqpoint{0.499793in}{3.640317in}}%
\pgfpathcurveto{\pgfqpoint{0.510843in}{3.640317in}}{\pgfqpoint{0.521442in}{3.644707in}}{\pgfqpoint{0.529256in}{3.652521in}}%
\pgfpathcurveto{\pgfqpoint{0.537070in}{3.660334in}}{\pgfqpoint{0.541460in}{3.670934in}}{\pgfqpoint{0.541460in}{3.681984in}}%
\pgfpathcurveto{\pgfqpoint{0.541460in}{3.693034in}}{\pgfqpoint{0.537070in}{3.703633in}}{\pgfqpoint{0.529256in}{3.711446in}}%
\pgfpathcurveto{\pgfqpoint{0.521442in}{3.719260in}}{\pgfqpoint{0.510843in}{3.723650in}}{\pgfqpoint{0.499793in}{3.723650in}}%
\pgfpathcurveto{\pgfqpoint{0.488743in}{3.723650in}}{\pgfqpoint{0.478144in}{3.719260in}}{\pgfqpoint{0.470331in}{3.711446in}}%
\pgfpathcurveto{\pgfqpoint{0.462517in}{3.703633in}}{\pgfqpoint{0.458127in}{3.693034in}}{\pgfqpoint{0.458127in}{3.681984in}}%
\pgfpathcurveto{\pgfqpoint{0.458127in}{3.670934in}}{\pgfqpoint{0.462517in}{3.660334in}}{\pgfqpoint{0.470331in}{3.652521in}}%
\pgfpathcurveto{\pgfqpoint{0.478144in}{3.644707in}}{\pgfqpoint{0.488743in}{3.640317in}}{\pgfqpoint{0.499793in}{3.640317in}}%
\pgfpathlineto{\pgfqpoint{0.499793in}{3.640317in}}%
\pgfpathclose%
\pgfusepath{stroke}%
\end{pgfscope}%
\begin{pgfscope}%
\pgfpathrectangle{\pgfqpoint{0.393053in}{0.375000in}}{\pgfqpoint{6.356833in}{5.175000in}}%
\pgfusepath{clip}%
\pgfsetbuttcap%
\pgfsetroundjoin%
\pgfsetlinewidth{1.003750pt}%
\definecolor{currentstroke}{rgb}{0.827451,0.827451,0.827451}%
\pgfsetstrokecolor{currentstroke}%
\pgfsetdash{}{0pt}%
\pgfpathmoveto{\pgfqpoint{0.646474in}{3.151616in}}%
\pgfpathcurveto{\pgfqpoint{0.657525in}{3.151616in}}{\pgfqpoint{0.668124in}{3.156006in}}{\pgfqpoint{0.675937in}{3.163820in}}%
\pgfpathcurveto{\pgfqpoint{0.683751in}{3.171634in}}{\pgfqpoint{0.688141in}{3.182233in}}{\pgfqpoint{0.688141in}{3.193283in}}%
\pgfpathcurveto{\pgfqpoint{0.688141in}{3.204333in}}{\pgfqpoint{0.683751in}{3.214932in}}{\pgfqpoint{0.675937in}{3.222745in}}%
\pgfpathcurveto{\pgfqpoint{0.668124in}{3.230559in}}{\pgfqpoint{0.657525in}{3.234949in}}{\pgfqpoint{0.646474in}{3.234949in}}%
\pgfpathcurveto{\pgfqpoint{0.635424in}{3.234949in}}{\pgfqpoint{0.624825in}{3.230559in}}{\pgfqpoint{0.617012in}{3.222745in}}%
\pgfpathcurveto{\pgfqpoint{0.609198in}{3.214932in}}{\pgfqpoint{0.604808in}{3.204333in}}{\pgfqpoint{0.604808in}{3.193283in}}%
\pgfpathcurveto{\pgfqpoint{0.604808in}{3.182233in}}{\pgfqpoint{0.609198in}{3.171634in}}{\pgfqpoint{0.617012in}{3.163820in}}%
\pgfpathcurveto{\pgfqpoint{0.624825in}{3.156006in}}{\pgfqpoint{0.635424in}{3.151616in}}{\pgfqpoint{0.646474in}{3.151616in}}%
\pgfpathlineto{\pgfqpoint{0.646474in}{3.151616in}}%
\pgfpathclose%
\pgfusepath{stroke}%
\end{pgfscope}%
\begin{pgfscope}%
\pgfpathrectangle{\pgfqpoint{0.393053in}{0.375000in}}{\pgfqpoint{6.356833in}{5.175000in}}%
\pgfusepath{clip}%
\pgfsetbuttcap%
\pgfsetroundjoin%
\pgfsetlinewidth{1.003750pt}%
\definecolor{currentstroke}{rgb}{0.827451,0.827451,0.827451}%
\pgfsetstrokecolor{currentstroke}%
\pgfsetdash{}{0pt}%
\pgfpathmoveto{\pgfqpoint{3.695118in}{0.605158in}}%
\pgfpathcurveto{\pgfqpoint{3.706168in}{0.605158in}}{\pgfqpoint{3.716767in}{0.609548in}}{\pgfqpoint{3.724581in}{0.617362in}}%
\pgfpathcurveto{\pgfqpoint{3.732394in}{0.625175in}}{\pgfqpoint{3.736785in}{0.635775in}}{\pgfqpoint{3.736785in}{0.646825in}}%
\pgfpathcurveto{\pgfqpoint{3.736785in}{0.657875in}}{\pgfqpoint{3.732394in}{0.668474in}}{\pgfqpoint{3.724581in}{0.676287in}}%
\pgfpathcurveto{\pgfqpoint{3.716767in}{0.684101in}}{\pgfqpoint{3.706168in}{0.688491in}}{\pgfqpoint{3.695118in}{0.688491in}}%
\pgfpathcurveto{\pgfqpoint{3.684068in}{0.688491in}}{\pgfqpoint{3.673469in}{0.684101in}}{\pgfqpoint{3.665655in}{0.676287in}}%
\pgfpathcurveto{\pgfqpoint{3.657842in}{0.668474in}}{\pgfqpoint{3.653451in}{0.657875in}}{\pgfqpoint{3.653451in}{0.646825in}}%
\pgfpathcurveto{\pgfqpoint{3.653451in}{0.635775in}}{\pgfqpoint{3.657842in}{0.625175in}}{\pgfqpoint{3.665655in}{0.617362in}}%
\pgfpathcurveto{\pgfqpoint{3.673469in}{0.609548in}}{\pgfqpoint{3.684068in}{0.605158in}}{\pgfqpoint{3.695118in}{0.605158in}}%
\pgfpathlineto{\pgfqpoint{3.695118in}{0.605158in}}%
\pgfpathclose%
\pgfusepath{stroke}%
\end{pgfscope}%
\begin{pgfscope}%
\pgfpathrectangle{\pgfqpoint{0.393053in}{0.375000in}}{\pgfqpoint{6.356833in}{5.175000in}}%
\pgfusepath{clip}%
\pgfsetbuttcap%
\pgfsetroundjoin%
\pgfsetlinewidth{1.003750pt}%
\definecolor{currentstroke}{rgb}{0.827451,0.827451,0.827451}%
\pgfsetstrokecolor{currentstroke}%
\pgfsetdash{}{0pt}%
\pgfpathmoveto{\pgfqpoint{1.365917in}{1.953195in}}%
\pgfpathcurveto{\pgfqpoint{1.376967in}{1.953195in}}{\pgfqpoint{1.387566in}{1.957586in}}{\pgfqpoint{1.395380in}{1.965399in}}%
\pgfpathcurveto{\pgfqpoint{1.403193in}{1.973213in}}{\pgfqpoint{1.407584in}{1.983812in}}{\pgfqpoint{1.407584in}{1.994862in}}%
\pgfpathcurveto{\pgfqpoint{1.407584in}{2.005912in}}{\pgfqpoint{1.403193in}{2.016511in}}{\pgfqpoint{1.395380in}{2.024325in}}%
\pgfpathcurveto{\pgfqpoint{1.387566in}{2.032138in}}{\pgfqpoint{1.376967in}{2.036529in}}{\pgfqpoint{1.365917in}{2.036529in}}%
\pgfpathcurveto{\pgfqpoint{1.354867in}{2.036529in}}{\pgfqpoint{1.344268in}{2.032138in}}{\pgfqpoint{1.336454in}{2.024325in}}%
\pgfpathcurveto{\pgfqpoint{1.328641in}{2.016511in}}{\pgfqpoint{1.324250in}{2.005912in}}{\pgfqpoint{1.324250in}{1.994862in}}%
\pgfpathcurveto{\pgfqpoint{1.324250in}{1.983812in}}{\pgfqpoint{1.328641in}{1.973213in}}{\pgfqpoint{1.336454in}{1.965399in}}%
\pgfpathcurveto{\pgfqpoint{1.344268in}{1.957586in}}{\pgfqpoint{1.354867in}{1.953195in}}{\pgfqpoint{1.365917in}{1.953195in}}%
\pgfpathlineto{\pgfqpoint{1.365917in}{1.953195in}}%
\pgfpathclose%
\pgfusepath{stroke}%
\end{pgfscope}%
\begin{pgfscope}%
\pgfpathrectangle{\pgfqpoint{0.393053in}{0.375000in}}{\pgfqpoint{6.356833in}{5.175000in}}%
\pgfusepath{clip}%
\pgfsetbuttcap%
\pgfsetroundjoin%
\pgfsetlinewidth{1.003750pt}%
\definecolor{currentstroke}{rgb}{0.827451,0.827451,0.827451}%
\pgfsetstrokecolor{currentstroke}%
\pgfsetdash{}{0pt}%
\pgfpathmoveto{\pgfqpoint{2.287965in}{1.196822in}}%
\pgfpathcurveto{\pgfqpoint{2.299015in}{1.196822in}}{\pgfqpoint{2.309614in}{1.201212in}}{\pgfqpoint{2.317427in}{1.209026in}}%
\pgfpathcurveto{\pgfqpoint{2.325241in}{1.216839in}}{\pgfqpoint{2.329631in}{1.227438in}}{\pgfqpoint{2.329631in}{1.238488in}}%
\pgfpathcurveto{\pgfqpoint{2.329631in}{1.249539in}}{\pgfqpoint{2.325241in}{1.260138in}}{\pgfqpoint{2.317427in}{1.267951in}}%
\pgfpathcurveto{\pgfqpoint{2.309614in}{1.275765in}}{\pgfqpoint{2.299015in}{1.280155in}}{\pgfqpoint{2.287965in}{1.280155in}}%
\pgfpathcurveto{\pgfqpoint{2.276915in}{1.280155in}}{\pgfqpoint{2.266316in}{1.275765in}}{\pgfqpoint{2.258502in}{1.267951in}}%
\pgfpathcurveto{\pgfqpoint{2.250688in}{1.260138in}}{\pgfqpoint{2.246298in}{1.249539in}}{\pgfqpoint{2.246298in}{1.238488in}}%
\pgfpathcurveto{\pgfqpoint{2.246298in}{1.227438in}}{\pgfqpoint{2.250688in}{1.216839in}}{\pgfqpoint{2.258502in}{1.209026in}}%
\pgfpathcurveto{\pgfqpoint{2.266316in}{1.201212in}}{\pgfqpoint{2.276915in}{1.196822in}}{\pgfqpoint{2.287965in}{1.196822in}}%
\pgfpathlineto{\pgfqpoint{2.287965in}{1.196822in}}%
\pgfpathclose%
\pgfusepath{stroke}%
\end{pgfscope}%
\begin{pgfscope}%
\pgfpathrectangle{\pgfqpoint{0.393053in}{0.375000in}}{\pgfqpoint{6.356833in}{5.175000in}}%
\pgfusepath{clip}%
\pgfsetbuttcap%
\pgfsetroundjoin%
\pgfsetlinewidth{1.003750pt}%
\definecolor{currentstroke}{rgb}{0.827451,0.827451,0.827451}%
\pgfsetstrokecolor{currentstroke}%
\pgfsetdash{}{0pt}%
\pgfpathmoveto{\pgfqpoint{2.591408in}{1.044742in}}%
\pgfpathcurveto{\pgfqpoint{2.602459in}{1.044742in}}{\pgfqpoint{2.613058in}{1.049132in}}{\pgfqpoint{2.620871in}{1.056945in}}%
\pgfpathcurveto{\pgfqpoint{2.628685in}{1.064759in}}{\pgfqpoint{2.633075in}{1.075358in}}{\pgfqpoint{2.633075in}{1.086408in}}%
\pgfpathcurveto{\pgfqpoint{2.633075in}{1.097458in}}{\pgfqpoint{2.628685in}{1.108057in}}{\pgfqpoint{2.620871in}{1.115871in}}%
\pgfpathcurveto{\pgfqpoint{2.613058in}{1.123685in}}{\pgfqpoint{2.602459in}{1.128075in}}{\pgfqpoint{2.591408in}{1.128075in}}%
\pgfpathcurveto{\pgfqpoint{2.580358in}{1.128075in}}{\pgfqpoint{2.569759in}{1.123685in}}{\pgfqpoint{2.561946in}{1.115871in}}%
\pgfpathcurveto{\pgfqpoint{2.554132in}{1.108057in}}{\pgfqpoint{2.549742in}{1.097458in}}{\pgfqpoint{2.549742in}{1.086408in}}%
\pgfpathcurveto{\pgfqpoint{2.549742in}{1.075358in}}{\pgfqpoint{2.554132in}{1.064759in}}{\pgfqpoint{2.561946in}{1.056945in}}%
\pgfpathcurveto{\pgfqpoint{2.569759in}{1.049132in}}{\pgfqpoint{2.580358in}{1.044742in}}{\pgfqpoint{2.591408in}{1.044742in}}%
\pgfpathlineto{\pgfqpoint{2.591408in}{1.044742in}}%
\pgfpathclose%
\pgfusepath{stroke}%
\end{pgfscope}%
\begin{pgfscope}%
\pgfpathrectangle{\pgfqpoint{0.393053in}{0.375000in}}{\pgfqpoint{6.356833in}{5.175000in}}%
\pgfusepath{clip}%
\pgfsetbuttcap%
\pgfsetroundjoin%
\pgfsetlinewidth{1.003750pt}%
\definecolor{currentstroke}{rgb}{0.827451,0.827451,0.827451}%
\pgfsetstrokecolor{currentstroke}%
\pgfsetdash{}{0pt}%
\pgfpathmoveto{\pgfqpoint{1.858747in}{1.496619in}}%
\pgfpathcurveto{\pgfqpoint{1.869797in}{1.496619in}}{\pgfqpoint{1.880396in}{1.501009in}}{\pgfqpoint{1.888209in}{1.508823in}}%
\pgfpathcurveto{\pgfqpoint{1.896023in}{1.516636in}}{\pgfqpoint{1.900413in}{1.527236in}}{\pgfqpoint{1.900413in}{1.538286in}}%
\pgfpathcurveto{\pgfqpoint{1.900413in}{1.549336in}}{\pgfqpoint{1.896023in}{1.559935in}}{\pgfqpoint{1.888209in}{1.567748in}}%
\pgfpathcurveto{\pgfqpoint{1.880396in}{1.575562in}}{\pgfqpoint{1.869797in}{1.579952in}}{\pgfqpoint{1.858747in}{1.579952in}}%
\pgfpathcurveto{\pgfqpoint{1.847696in}{1.579952in}}{\pgfqpoint{1.837097in}{1.575562in}}{\pgfqpoint{1.829284in}{1.567748in}}%
\pgfpathcurveto{\pgfqpoint{1.821470in}{1.559935in}}{\pgfqpoint{1.817080in}{1.549336in}}{\pgfqpoint{1.817080in}{1.538286in}}%
\pgfpathcurveto{\pgfqpoint{1.817080in}{1.527236in}}{\pgfqpoint{1.821470in}{1.516636in}}{\pgfqpoint{1.829284in}{1.508823in}}%
\pgfpathcurveto{\pgfqpoint{1.837097in}{1.501009in}}{\pgfqpoint{1.847696in}{1.496619in}}{\pgfqpoint{1.858747in}{1.496619in}}%
\pgfpathlineto{\pgfqpoint{1.858747in}{1.496619in}}%
\pgfpathclose%
\pgfusepath{stroke}%
\end{pgfscope}%
\begin{pgfscope}%
\pgfpathrectangle{\pgfqpoint{0.393053in}{0.375000in}}{\pgfqpoint{6.356833in}{5.175000in}}%
\pgfusepath{clip}%
\pgfsetbuttcap%
\pgfsetroundjoin%
\pgfsetlinewidth{1.003750pt}%
\definecolor{currentstroke}{rgb}{0.827451,0.827451,0.827451}%
\pgfsetstrokecolor{currentstroke}%
\pgfsetdash{}{0pt}%
\pgfpathmoveto{\pgfqpoint{1.790986in}{1.580405in}}%
\pgfpathcurveto{\pgfqpoint{1.802036in}{1.580405in}}{\pgfqpoint{1.812635in}{1.584795in}}{\pgfqpoint{1.820449in}{1.592608in}}%
\pgfpathcurveto{\pgfqpoint{1.828263in}{1.600422in}}{\pgfqpoint{1.832653in}{1.611021in}}{\pgfqpoint{1.832653in}{1.622071in}}%
\pgfpathcurveto{\pgfqpoint{1.832653in}{1.633121in}}{\pgfqpoint{1.828263in}{1.643720in}}{\pgfqpoint{1.820449in}{1.651534in}}%
\pgfpathcurveto{\pgfqpoint{1.812635in}{1.659348in}}{\pgfqpoint{1.802036in}{1.663738in}}{\pgfqpoint{1.790986in}{1.663738in}}%
\pgfpathcurveto{\pgfqpoint{1.779936in}{1.663738in}}{\pgfqpoint{1.769337in}{1.659348in}}{\pgfqpoint{1.761523in}{1.651534in}}%
\pgfpathcurveto{\pgfqpoint{1.753710in}{1.643720in}}{\pgfqpoint{1.749319in}{1.633121in}}{\pgfqpoint{1.749319in}{1.622071in}}%
\pgfpathcurveto{\pgfqpoint{1.749319in}{1.611021in}}{\pgfqpoint{1.753710in}{1.600422in}}{\pgfqpoint{1.761523in}{1.592608in}}%
\pgfpathcurveto{\pgfqpoint{1.769337in}{1.584795in}}{\pgfqpoint{1.779936in}{1.580405in}}{\pgfqpoint{1.790986in}{1.580405in}}%
\pgfpathlineto{\pgfqpoint{1.790986in}{1.580405in}}%
\pgfpathclose%
\pgfusepath{stroke}%
\end{pgfscope}%
\begin{pgfscope}%
\pgfpathrectangle{\pgfqpoint{0.393053in}{0.375000in}}{\pgfqpoint{6.356833in}{5.175000in}}%
\pgfusepath{clip}%
\pgfsetbuttcap%
\pgfsetroundjoin%
\pgfsetlinewidth{1.003750pt}%
\definecolor{currentstroke}{rgb}{0.827451,0.827451,0.827451}%
\pgfsetstrokecolor{currentstroke}%
\pgfsetdash{}{0pt}%
\pgfpathmoveto{\pgfqpoint{0.430992in}{4.025736in}}%
\pgfpathcurveto{\pgfqpoint{0.442043in}{4.025736in}}{\pgfqpoint{0.452642in}{4.030127in}}{\pgfqpoint{0.460455in}{4.037940in}}%
\pgfpathcurveto{\pgfqpoint{0.468269in}{4.045754in}}{\pgfqpoint{0.472659in}{4.056353in}}{\pgfqpoint{0.472659in}{4.067403in}}%
\pgfpathcurveto{\pgfqpoint{0.472659in}{4.078453in}}{\pgfqpoint{0.468269in}{4.089052in}}{\pgfqpoint{0.460455in}{4.096866in}}%
\pgfpathcurveto{\pgfqpoint{0.452642in}{4.104679in}}{\pgfqpoint{0.442043in}{4.109070in}}{\pgfqpoint{0.430992in}{4.109070in}}%
\pgfpathcurveto{\pgfqpoint{0.419942in}{4.109070in}}{\pgfqpoint{0.409343in}{4.104679in}}{\pgfqpoint{0.401530in}{4.096866in}}%
\pgfpathcurveto{\pgfqpoint{0.393716in}{4.089052in}}{\pgfqpoint{0.389326in}{4.078453in}}{\pgfqpoint{0.389326in}{4.067403in}}%
\pgfpathcurveto{\pgfqpoint{0.389326in}{4.056353in}}{\pgfqpoint{0.393716in}{4.045754in}}{\pgfqpoint{0.401530in}{4.037940in}}%
\pgfpathcurveto{\pgfqpoint{0.409343in}{4.030127in}}{\pgfqpoint{0.419942in}{4.025736in}}{\pgfqpoint{0.430992in}{4.025736in}}%
\pgfpathlineto{\pgfqpoint{0.430992in}{4.025736in}}%
\pgfpathclose%
\pgfusepath{stroke}%
\end{pgfscope}%
\begin{pgfscope}%
\pgfpathrectangle{\pgfqpoint{0.393053in}{0.375000in}}{\pgfqpoint{6.356833in}{5.175000in}}%
\pgfusepath{clip}%
\pgfsetbuttcap%
\pgfsetroundjoin%
\pgfsetlinewidth{1.003750pt}%
\definecolor{currentstroke}{rgb}{0.827451,0.827451,0.827451}%
\pgfsetstrokecolor{currentstroke}%
\pgfsetdash{}{0pt}%
\pgfpathmoveto{\pgfqpoint{2.905386in}{0.884940in}}%
\pgfpathcurveto{\pgfqpoint{2.916436in}{0.884940in}}{\pgfqpoint{2.927035in}{0.889330in}}{\pgfqpoint{2.934849in}{0.897144in}}%
\pgfpathcurveto{\pgfqpoint{2.942663in}{0.904957in}}{\pgfqpoint{2.947053in}{0.915556in}}{\pgfqpoint{2.947053in}{0.926606in}}%
\pgfpathcurveto{\pgfqpoint{2.947053in}{0.937657in}}{\pgfqpoint{2.942663in}{0.948256in}}{\pgfqpoint{2.934849in}{0.956069in}}%
\pgfpathcurveto{\pgfqpoint{2.927035in}{0.963883in}}{\pgfqpoint{2.916436in}{0.968273in}}{\pgfqpoint{2.905386in}{0.968273in}}%
\pgfpathcurveto{\pgfqpoint{2.894336in}{0.968273in}}{\pgfqpoint{2.883737in}{0.963883in}}{\pgfqpoint{2.875923in}{0.956069in}}%
\pgfpathcurveto{\pgfqpoint{2.868110in}{0.948256in}}{\pgfqpoint{2.863720in}{0.937657in}}{\pgfqpoint{2.863720in}{0.926606in}}%
\pgfpathcurveto{\pgfqpoint{2.863720in}{0.915556in}}{\pgfqpoint{2.868110in}{0.904957in}}{\pgfqpoint{2.875923in}{0.897144in}}%
\pgfpathcurveto{\pgfqpoint{2.883737in}{0.889330in}}{\pgfqpoint{2.894336in}{0.884940in}}{\pgfqpoint{2.905386in}{0.884940in}}%
\pgfpathlineto{\pgfqpoint{2.905386in}{0.884940in}}%
\pgfpathclose%
\pgfusepath{stroke}%
\end{pgfscope}%
\begin{pgfscope}%
\pgfpathrectangle{\pgfqpoint{0.393053in}{0.375000in}}{\pgfqpoint{6.356833in}{5.175000in}}%
\pgfusepath{clip}%
\pgfsetbuttcap%
\pgfsetroundjoin%
\pgfsetlinewidth{1.003750pt}%
\definecolor{currentstroke}{rgb}{0.827451,0.827451,0.827451}%
\pgfsetstrokecolor{currentstroke}%
\pgfsetdash{}{0pt}%
\pgfpathmoveto{\pgfqpoint{4.173270in}{0.457379in}}%
\pgfpathcurveto{\pgfqpoint{4.184320in}{0.457379in}}{\pgfqpoint{4.194919in}{0.461770in}}{\pgfqpoint{4.202733in}{0.469583in}}%
\pgfpathcurveto{\pgfqpoint{4.210547in}{0.477397in}}{\pgfqpoint{4.214937in}{0.487996in}}{\pgfqpoint{4.214937in}{0.499046in}}%
\pgfpathcurveto{\pgfqpoint{4.214937in}{0.510096in}}{\pgfqpoint{4.210547in}{0.520695in}}{\pgfqpoint{4.202733in}{0.528509in}}%
\pgfpathcurveto{\pgfqpoint{4.194919in}{0.536322in}}{\pgfqpoint{4.184320in}{0.540713in}}{\pgfqpoint{4.173270in}{0.540713in}}%
\pgfpathcurveto{\pgfqpoint{4.162220in}{0.540713in}}{\pgfqpoint{4.151621in}{0.536322in}}{\pgfqpoint{4.143807in}{0.528509in}}%
\pgfpathcurveto{\pgfqpoint{4.135994in}{0.520695in}}{\pgfqpoint{4.131604in}{0.510096in}}{\pgfqpoint{4.131604in}{0.499046in}}%
\pgfpathcurveto{\pgfqpoint{4.131604in}{0.487996in}}{\pgfqpoint{4.135994in}{0.477397in}}{\pgfqpoint{4.143807in}{0.469583in}}%
\pgfpathcurveto{\pgfqpoint{4.151621in}{0.461770in}}{\pgfqpoint{4.162220in}{0.457379in}}{\pgfqpoint{4.173270in}{0.457379in}}%
\pgfpathlineto{\pgfqpoint{4.173270in}{0.457379in}}%
\pgfpathclose%
\pgfusepath{stroke}%
\end{pgfscope}%
\begin{pgfscope}%
\pgfpathrectangle{\pgfqpoint{0.393053in}{0.375000in}}{\pgfqpoint{6.356833in}{5.175000in}}%
\pgfusepath{clip}%
\pgfsetbuttcap%
\pgfsetroundjoin%
\pgfsetlinewidth{1.003750pt}%
\definecolor{currentstroke}{rgb}{0.827451,0.827451,0.827451}%
\pgfsetstrokecolor{currentstroke}%
\pgfsetdash{}{0pt}%
\pgfpathmoveto{\pgfqpoint{0.435142in}{3.936498in}}%
\pgfpathcurveto{\pgfqpoint{0.446192in}{3.936498in}}{\pgfqpoint{0.456791in}{3.940888in}}{\pgfqpoint{0.464605in}{3.948702in}}%
\pgfpathcurveto{\pgfqpoint{0.472419in}{3.956516in}}{\pgfqpoint{0.476809in}{3.967115in}}{\pgfqpoint{0.476809in}{3.978165in}}%
\pgfpathcurveto{\pgfqpoint{0.476809in}{3.989215in}}{\pgfqpoint{0.472419in}{3.999814in}}{\pgfqpoint{0.464605in}{4.007627in}}%
\pgfpathcurveto{\pgfqpoint{0.456791in}{4.015441in}}{\pgfqpoint{0.446192in}{4.019831in}}{\pgfqpoint{0.435142in}{4.019831in}}%
\pgfpathcurveto{\pgfqpoint{0.424092in}{4.019831in}}{\pgfqpoint{0.413493in}{4.015441in}}{\pgfqpoint{0.405679in}{4.007627in}}%
\pgfpathcurveto{\pgfqpoint{0.397866in}{3.999814in}}{\pgfqpoint{0.393475in}{3.989215in}}{\pgfqpoint{0.393475in}{3.978165in}}%
\pgfpathcurveto{\pgfqpoint{0.393475in}{3.967115in}}{\pgfqpoint{0.397866in}{3.956516in}}{\pgfqpoint{0.405679in}{3.948702in}}%
\pgfpathcurveto{\pgfqpoint{0.413493in}{3.940888in}}{\pgfqpoint{0.424092in}{3.936498in}}{\pgfqpoint{0.435142in}{3.936498in}}%
\pgfpathlineto{\pgfqpoint{0.435142in}{3.936498in}}%
\pgfpathclose%
\pgfusepath{stroke}%
\end{pgfscope}%
\begin{pgfscope}%
\pgfpathrectangle{\pgfqpoint{0.393053in}{0.375000in}}{\pgfqpoint{6.356833in}{5.175000in}}%
\pgfusepath{clip}%
\pgfsetbuttcap%
\pgfsetroundjoin%
\pgfsetlinewidth{1.003750pt}%
\definecolor{currentstroke}{rgb}{0.827451,0.827451,0.827451}%
\pgfsetstrokecolor{currentstroke}%
\pgfsetdash{}{0pt}%
\pgfpathmoveto{\pgfqpoint{1.618902in}{1.791320in}}%
\pgfpathcurveto{\pgfqpoint{1.629952in}{1.791320in}}{\pgfqpoint{1.640551in}{1.795710in}}{\pgfqpoint{1.648364in}{1.803524in}}%
\pgfpathcurveto{\pgfqpoint{1.656178in}{1.811338in}}{\pgfqpoint{1.660568in}{1.821937in}}{\pgfqpoint{1.660568in}{1.832987in}}%
\pgfpathcurveto{\pgfqpoint{1.660568in}{1.844037in}}{\pgfqpoint{1.656178in}{1.854636in}}{\pgfqpoint{1.648364in}{1.862450in}}%
\pgfpathcurveto{\pgfqpoint{1.640551in}{1.870263in}}{\pgfqpoint{1.629952in}{1.874653in}}{\pgfqpoint{1.618902in}{1.874653in}}%
\pgfpathcurveto{\pgfqpoint{1.607851in}{1.874653in}}{\pgfqpoint{1.597252in}{1.870263in}}{\pgfqpoint{1.589439in}{1.862450in}}%
\pgfpathcurveto{\pgfqpoint{1.581625in}{1.854636in}}{\pgfqpoint{1.577235in}{1.844037in}}{\pgfqpoint{1.577235in}{1.832987in}}%
\pgfpathcurveto{\pgfqpoint{1.577235in}{1.821937in}}{\pgfqpoint{1.581625in}{1.811338in}}{\pgfqpoint{1.589439in}{1.803524in}}%
\pgfpathcurveto{\pgfqpoint{1.597252in}{1.795710in}}{\pgfqpoint{1.607851in}{1.791320in}}{\pgfqpoint{1.618902in}{1.791320in}}%
\pgfpathlineto{\pgfqpoint{1.618902in}{1.791320in}}%
\pgfpathclose%
\pgfusepath{stroke}%
\end{pgfscope}%
\begin{pgfscope}%
\pgfpathrectangle{\pgfqpoint{0.393053in}{0.375000in}}{\pgfqpoint{6.356833in}{5.175000in}}%
\pgfusepath{clip}%
\pgfsetbuttcap%
\pgfsetroundjoin%
\pgfsetlinewidth{1.003750pt}%
\definecolor{currentstroke}{rgb}{0.827451,0.827451,0.827451}%
\pgfsetstrokecolor{currentstroke}%
\pgfsetdash{}{0pt}%
\pgfpathmoveto{\pgfqpoint{2.514644in}{1.050767in}}%
\pgfpathcurveto{\pgfqpoint{2.525694in}{1.050767in}}{\pgfqpoint{2.536293in}{1.055158in}}{\pgfqpoint{2.544106in}{1.062971in}}%
\pgfpathcurveto{\pgfqpoint{2.551920in}{1.070785in}}{\pgfqpoint{2.556310in}{1.081384in}}{\pgfqpoint{2.556310in}{1.092434in}}%
\pgfpathcurveto{\pgfqpoint{2.556310in}{1.103484in}}{\pgfqpoint{2.551920in}{1.114083in}}{\pgfqpoint{2.544106in}{1.121897in}}%
\pgfpathcurveto{\pgfqpoint{2.536293in}{1.129711in}}{\pgfqpoint{2.525694in}{1.134101in}}{\pgfqpoint{2.514644in}{1.134101in}}%
\pgfpathcurveto{\pgfqpoint{2.503593in}{1.134101in}}{\pgfqpoint{2.492994in}{1.129711in}}{\pgfqpoint{2.485181in}{1.121897in}}%
\pgfpathcurveto{\pgfqpoint{2.477367in}{1.114083in}}{\pgfqpoint{2.472977in}{1.103484in}}{\pgfqpoint{2.472977in}{1.092434in}}%
\pgfpathcurveto{\pgfqpoint{2.472977in}{1.081384in}}{\pgfqpoint{2.477367in}{1.070785in}}{\pgfqpoint{2.485181in}{1.062971in}}%
\pgfpathcurveto{\pgfqpoint{2.492994in}{1.055158in}}{\pgfqpoint{2.503593in}{1.050767in}}{\pgfqpoint{2.514644in}{1.050767in}}%
\pgfpathlineto{\pgfqpoint{2.514644in}{1.050767in}}%
\pgfpathclose%
\pgfusepath{stroke}%
\end{pgfscope}%
\begin{pgfscope}%
\pgfpathrectangle{\pgfqpoint{0.393053in}{0.375000in}}{\pgfqpoint{6.356833in}{5.175000in}}%
\pgfusepath{clip}%
\pgfsetbuttcap%
\pgfsetroundjoin%
\pgfsetlinewidth{1.003750pt}%
\definecolor{currentstroke}{rgb}{0.827451,0.827451,0.827451}%
\pgfsetstrokecolor{currentstroke}%
\pgfsetdash{}{0pt}%
\pgfpathmoveto{\pgfqpoint{3.694432in}{0.605429in}}%
\pgfpathcurveto{\pgfqpoint{3.705482in}{0.605429in}}{\pgfqpoint{3.716081in}{0.609820in}}{\pgfqpoint{3.723895in}{0.617633in}}%
\pgfpathcurveto{\pgfqpoint{3.731708in}{0.625447in}}{\pgfqpoint{3.736098in}{0.636046in}}{\pgfqpoint{3.736098in}{0.647096in}}%
\pgfpathcurveto{\pgfqpoint{3.736098in}{0.658146in}}{\pgfqpoint{3.731708in}{0.668745in}}{\pgfqpoint{3.723895in}{0.676559in}}%
\pgfpathcurveto{\pgfqpoint{3.716081in}{0.684372in}}{\pgfqpoint{3.705482in}{0.688763in}}{\pgfqpoint{3.694432in}{0.688763in}}%
\pgfpathcurveto{\pgfqpoint{3.683382in}{0.688763in}}{\pgfqpoint{3.672783in}{0.684372in}}{\pgfqpoint{3.664969in}{0.676559in}}%
\pgfpathcurveto{\pgfqpoint{3.657155in}{0.668745in}}{\pgfqpoint{3.652765in}{0.658146in}}{\pgfqpoint{3.652765in}{0.647096in}}%
\pgfpathcurveto{\pgfqpoint{3.652765in}{0.636046in}}{\pgfqpoint{3.657155in}{0.625447in}}{\pgfqpoint{3.664969in}{0.617633in}}%
\pgfpathcurveto{\pgfqpoint{3.672783in}{0.609820in}}{\pgfqpoint{3.683382in}{0.605429in}}{\pgfqpoint{3.694432in}{0.605429in}}%
\pgfpathlineto{\pgfqpoint{3.694432in}{0.605429in}}%
\pgfpathclose%
\pgfusepath{stroke}%
\end{pgfscope}%
\begin{pgfscope}%
\pgfpathrectangle{\pgfqpoint{0.393053in}{0.375000in}}{\pgfqpoint{6.356833in}{5.175000in}}%
\pgfusepath{clip}%
\pgfsetbuttcap%
\pgfsetroundjoin%
\pgfsetlinewidth{1.003750pt}%
\definecolor{currentstroke}{rgb}{0.827451,0.827451,0.827451}%
\pgfsetstrokecolor{currentstroke}%
\pgfsetdash{}{0pt}%
\pgfpathmoveto{\pgfqpoint{1.393668in}{1.940253in}}%
\pgfpathcurveto{\pgfqpoint{1.404718in}{1.940253in}}{\pgfqpoint{1.415317in}{1.944643in}}{\pgfqpoint{1.423131in}{1.952457in}}%
\pgfpathcurveto{\pgfqpoint{1.430944in}{1.960270in}}{\pgfqpoint{1.435335in}{1.970869in}}{\pgfqpoint{1.435335in}{1.981919in}}%
\pgfpathcurveto{\pgfqpoint{1.435335in}{1.992969in}}{\pgfqpoint{1.430944in}{2.003568in}}{\pgfqpoint{1.423131in}{2.011382in}}%
\pgfpathcurveto{\pgfqpoint{1.415317in}{2.019196in}}{\pgfqpoint{1.404718in}{2.023586in}}{\pgfqpoint{1.393668in}{2.023586in}}%
\pgfpathcurveto{\pgfqpoint{1.382618in}{2.023586in}}{\pgfqpoint{1.372019in}{2.019196in}}{\pgfqpoint{1.364205in}{2.011382in}}%
\pgfpathcurveto{\pgfqpoint{1.356392in}{2.003568in}}{\pgfqpoint{1.352001in}{1.992969in}}{\pgfqpoint{1.352001in}{1.981919in}}%
\pgfpathcurveto{\pgfqpoint{1.352001in}{1.970869in}}{\pgfqpoint{1.356392in}{1.960270in}}{\pgfqpoint{1.364205in}{1.952457in}}%
\pgfpathcurveto{\pgfqpoint{1.372019in}{1.944643in}}{\pgfqpoint{1.382618in}{1.940253in}}{\pgfqpoint{1.393668in}{1.940253in}}%
\pgfpathlineto{\pgfqpoint{1.393668in}{1.940253in}}%
\pgfpathclose%
\pgfusepath{stroke}%
\end{pgfscope}%
\begin{pgfscope}%
\pgfpathrectangle{\pgfqpoint{0.393053in}{0.375000in}}{\pgfqpoint{6.356833in}{5.175000in}}%
\pgfusepath{clip}%
\pgfsetbuttcap%
\pgfsetroundjoin%
\pgfsetlinewidth{1.003750pt}%
\definecolor{currentstroke}{rgb}{0.827451,0.827451,0.827451}%
\pgfsetstrokecolor{currentstroke}%
\pgfsetdash{}{0pt}%
\pgfpathmoveto{\pgfqpoint{3.110740in}{0.870644in}}%
\pgfpathcurveto{\pgfqpoint{3.121790in}{0.870644in}}{\pgfqpoint{3.132389in}{0.875034in}}{\pgfqpoint{3.140202in}{0.882848in}}%
\pgfpathcurveto{\pgfqpoint{3.148016in}{0.890661in}}{\pgfqpoint{3.152406in}{0.901261in}}{\pgfqpoint{3.152406in}{0.912311in}}%
\pgfpathcurveto{\pgfqpoint{3.152406in}{0.923361in}}{\pgfqpoint{3.148016in}{0.933960in}}{\pgfqpoint{3.140202in}{0.941773in}}%
\pgfpathcurveto{\pgfqpoint{3.132389in}{0.949587in}}{\pgfqpoint{3.121790in}{0.953977in}}{\pgfqpoint{3.110740in}{0.953977in}}%
\pgfpathcurveto{\pgfqpoint{3.099690in}{0.953977in}}{\pgfqpoint{3.089090in}{0.949587in}}{\pgfqpoint{3.081277in}{0.941773in}}%
\pgfpathcurveto{\pgfqpoint{3.073463in}{0.933960in}}{\pgfqpoint{3.069073in}{0.923361in}}{\pgfqpoint{3.069073in}{0.912311in}}%
\pgfpathcurveto{\pgfqpoint{3.069073in}{0.901261in}}{\pgfqpoint{3.073463in}{0.890661in}}{\pgfqpoint{3.081277in}{0.882848in}}%
\pgfpathcurveto{\pgfqpoint{3.089090in}{0.875034in}}{\pgfqpoint{3.099690in}{0.870644in}}{\pgfqpoint{3.110740in}{0.870644in}}%
\pgfpathlineto{\pgfqpoint{3.110740in}{0.870644in}}%
\pgfpathclose%
\pgfusepath{stroke}%
\end{pgfscope}%
\begin{pgfscope}%
\pgfpathrectangle{\pgfqpoint{0.393053in}{0.375000in}}{\pgfqpoint{6.356833in}{5.175000in}}%
\pgfusepath{clip}%
\pgfsetbuttcap%
\pgfsetroundjoin%
\pgfsetlinewidth{1.003750pt}%
\definecolor{currentstroke}{rgb}{0.827451,0.827451,0.827451}%
\pgfsetstrokecolor{currentstroke}%
\pgfsetdash{}{0pt}%
\pgfpathmoveto{\pgfqpoint{0.970121in}{2.719437in}}%
\pgfpathcurveto{\pgfqpoint{0.981171in}{2.719437in}}{\pgfqpoint{0.991770in}{2.723827in}}{\pgfqpoint{0.999584in}{2.731641in}}%
\pgfpathcurveto{\pgfqpoint{1.007397in}{2.739454in}}{\pgfqpoint{1.011788in}{2.750053in}}{\pgfqpoint{1.011788in}{2.761103in}}%
\pgfpathcurveto{\pgfqpoint{1.011788in}{2.772154in}}{\pgfqpoint{1.007397in}{2.782753in}}{\pgfqpoint{0.999584in}{2.790566in}}%
\pgfpathcurveto{\pgfqpoint{0.991770in}{2.798380in}}{\pgfqpoint{0.981171in}{2.802770in}}{\pgfqpoint{0.970121in}{2.802770in}}%
\pgfpathcurveto{\pgfqpoint{0.959071in}{2.802770in}}{\pgfqpoint{0.948472in}{2.798380in}}{\pgfqpoint{0.940658in}{2.790566in}}%
\pgfpathcurveto{\pgfqpoint{0.932845in}{2.782753in}}{\pgfqpoint{0.928454in}{2.772154in}}{\pgfqpoint{0.928454in}{2.761103in}}%
\pgfpathcurveto{\pgfqpoint{0.928454in}{2.750053in}}{\pgfqpoint{0.932845in}{2.739454in}}{\pgfqpoint{0.940658in}{2.731641in}}%
\pgfpathcurveto{\pgfqpoint{0.948472in}{2.723827in}}{\pgfqpoint{0.959071in}{2.719437in}}{\pgfqpoint{0.970121in}{2.719437in}}%
\pgfpathlineto{\pgfqpoint{0.970121in}{2.719437in}}%
\pgfpathclose%
\pgfusepath{stroke}%
\end{pgfscope}%
\begin{pgfscope}%
\pgfpathrectangle{\pgfqpoint{0.393053in}{0.375000in}}{\pgfqpoint{6.356833in}{5.175000in}}%
\pgfusepath{clip}%
\pgfsetbuttcap%
\pgfsetroundjoin%
\pgfsetlinewidth{1.003750pt}%
\definecolor{currentstroke}{rgb}{0.827451,0.827451,0.827451}%
\pgfsetstrokecolor{currentstroke}%
\pgfsetdash{}{0pt}%
\pgfpathmoveto{\pgfqpoint{2.834525in}{0.919593in}}%
\pgfpathcurveto{\pgfqpoint{2.845575in}{0.919593in}}{\pgfqpoint{2.856174in}{0.923983in}}{\pgfqpoint{2.863987in}{0.931797in}}%
\pgfpathcurveto{\pgfqpoint{2.871801in}{0.939610in}}{\pgfqpoint{2.876191in}{0.950209in}}{\pgfqpoint{2.876191in}{0.961260in}}%
\pgfpathcurveto{\pgfqpoint{2.876191in}{0.972310in}}{\pgfqpoint{2.871801in}{0.982909in}}{\pgfqpoint{2.863987in}{0.990722in}}%
\pgfpathcurveto{\pgfqpoint{2.856174in}{0.998536in}}{\pgfqpoint{2.845575in}{1.002926in}}{\pgfqpoint{2.834525in}{1.002926in}}%
\pgfpathcurveto{\pgfqpoint{2.823475in}{1.002926in}}{\pgfqpoint{2.812876in}{0.998536in}}{\pgfqpoint{2.805062in}{0.990722in}}%
\pgfpathcurveto{\pgfqpoint{2.797248in}{0.982909in}}{\pgfqpoint{2.792858in}{0.972310in}}{\pgfqpoint{2.792858in}{0.961260in}}%
\pgfpathcurveto{\pgfqpoint{2.792858in}{0.950209in}}{\pgfqpoint{2.797248in}{0.939610in}}{\pgfqpoint{2.805062in}{0.931797in}}%
\pgfpathcurveto{\pgfqpoint{2.812876in}{0.923983in}}{\pgfqpoint{2.823475in}{0.919593in}}{\pgfqpoint{2.834525in}{0.919593in}}%
\pgfpathlineto{\pgfqpoint{2.834525in}{0.919593in}}%
\pgfpathclose%
\pgfusepath{stroke}%
\end{pgfscope}%
\begin{pgfscope}%
\pgfpathrectangle{\pgfqpoint{0.393053in}{0.375000in}}{\pgfqpoint{6.356833in}{5.175000in}}%
\pgfusepath{clip}%
\pgfsetbuttcap%
\pgfsetroundjoin%
\pgfsetlinewidth{1.003750pt}%
\definecolor{currentstroke}{rgb}{0.827451,0.827451,0.827451}%
\pgfsetstrokecolor{currentstroke}%
\pgfsetdash{}{0pt}%
\pgfpathmoveto{\pgfqpoint{3.662350in}{0.729635in}}%
\pgfpathcurveto{\pgfqpoint{3.673400in}{0.729635in}}{\pgfqpoint{3.683999in}{0.734025in}}{\pgfqpoint{3.691813in}{0.741839in}}%
\pgfpathcurveto{\pgfqpoint{3.699626in}{0.749652in}}{\pgfqpoint{3.704016in}{0.760251in}}{\pgfqpoint{3.704016in}{0.771301in}}%
\pgfpathcurveto{\pgfqpoint{3.704016in}{0.782352in}}{\pgfqpoint{3.699626in}{0.792951in}}{\pgfqpoint{3.691813in}{0.800764in}}%
\pgfpathcurveto{\pgfqpoint{3.683999in}{0.808578in}}{\pgfqpoint{3.673400in}{0.812968in}}{\pgfqpoint{3.662350in}{0.812968in}}%
\pgfpathcurveto{\pgfqpoint{3.651300in}{0.812968in}}{\pgfqpoint{3.640701in}{0.808578in}}{\pgfqpoint{3.632887in}{0.800764in}}%
\pgfpathcurveto{\pgfqpoint{3.625073in}{0.792951in}}{\pgfqpoint{3.620683in}{0.782352in}}{\pgfqpoint{3.620683in}{0.771301in}}%
\pgfpathcurveto{\pgfqpoint{3.620683in}{0.760251in}}{\pgfqpoint{3.625073in}{0.749652in}}{\pgfqpoint{3.632887in}{0.741839in}}%
\pgfpathcurveto{\pgfqpoint{3.640701in}{0.734025in}}{\pgfqpoint{3.651300in}{0.729635in}}{\pgfqpoint{3.662350in}{0.729635in}}%
\pgfpathlineto{\pgfqpoint{3.662350in}{0.729635in}}%
\pgfpathclose%
\pgfusepath{stroke}%
\end{pgfscope}%
\begin{pgfscope}%
\pgfpathrectangle{\pgfqpoint{0.393053in}{0.375000in}}{\pgfqpoint{6.356833in}{5.175000in}}%
\pgfusepath{clip}%
\pgfsetbuttcap%
\pgfsetroundjoin%
\pgfsetlinewidth{1.003750pt}%
\definecolor{currentstroke}{rgb}{0.827451,0.827451,0.827451}%
\pgfsetstrokecolor{currentstroke}%
\pgfsetdash{}{0pt}%
\pgfpathmoveto{\pgfqpoint{3.668244in}{0.630867in}}%
\pgfpathcurveto{\pgfqpoint{3.679294in}{0.630867in}}{\pgfqpoint{3.689893in}{0.635257in}}{\pgfqpoint{3.697707in}{0.643071in}}%
\pgfpathcurveto{\pgfqpoint{3.705521in}{0.650884in}}{\pgfqpoint{3.709911in}{0.661483in}}{\pgfqpoint{3.709911in}{0.672534in}}%
\pgfpathcurveto{\pgfqpoint{3.709911in}{0.683584in}}{\pgfqpoint{3.705521in}{0.694183in}}{\pgfqpoint{3.697707in}{0.701996in}}%
\pgfpathcurveto{\pgfqpoint{3.689893in}{0.709810in}}{\pgfqpoint{3.679294in}{0.714200in}}{\pgfqpoint{3.668244in}{0.714200in}}%
\pgfpathcurveto{\pgfqpoint{3.657194in}{0.714200in}}{\pgfqpoint{3.646595in}{0.709810in}}{\pgfqpoint{3.638781in}{0.701996in}}%
\pgfpathcurveto{\pgfqpoint{3.630968in}{0.694183in}}{\pgfqpoint{3.626578in}{0.683584in}}{\pgfqpoint{3.626578in}{0.672534in}}%
\pgfpathcurveto{\pgfqpoint{3.626578in}{0.661483in}}{\pgfqpoint{3.630968in}{0.650884in}}{\pgfqpoint{3.638781in}{0.643071in}}%
\pgfpathcurveto{\pgfqpoint{3.646595in}{0.635257in}}{\pgfqpoint{3.657194in}{0.630867in}}{\pgfqpoint{3.668244in}{0.630867in}}%
\pgfpathlineto{\pgfqpoint{3.668244in}{0.630867in}}%
\pgfpathclose%
\pgfusepath{stroke}%
\end{pgfscope}%
\begin{pgfscope}%
\pgfpathrectangle{\pgfqpoint{0.393053in}{0.375000in}}{\pgfqpoint{6.356833in}{5.175000in}}%
\pgfusepath{clip}%
\pgfsetbuttcap%
\pgfsetroundjoin%
\pgfsetlinewidth{1.003750pt}%
\definecolor{currentstroke}{rgb}{0.827451,0.827451,0.827451}%
\pgfsetstrokecolor{currentstroke}%
\pgfsetdash{}{0pt}%
\pgfpathmoveto{\pgfqpoint{1.260728in}{2.046542in}}%
\pgfpathcurveto{\pgfqpoint{1.271778in}{2.046542in}}{\pgfqpoint{1.282377in}{2.050933in}}{\pgfqpoint{1.290191in}{2.058746in}}%
\pgfpathcurveto{\pgfqpoint{1.298005in}{2.066560in}}{\pgfqpoint{1.302395in}{2.077159in}}{\pgfqpoint{1.302395in}{2.088209in}}%
\pgfpathcurveto{\pgfqpoint{1.302395in}{2.099259in}}{\pgfqpoint{1.298005in}{2.109858in}}{\pgfqpoint{1.290191in}{2.117672in}}%
\pgfpathcurveto{\pgfqpoint{1.282377in}{2.125486in}}{\pgfqpoint{1.271778in}{2.129876in}}{\pgfqpoint{1.260728in}{2.129876in}}%
\pgfpathcurveto{\pgfqpoint{1.249678in}{2.129876in}}{\pgfqpoint{1.239079in}{2.125486in}}{\pgfqpoint{1.231265in}{2.117672in}}%
\pgfpathcurveto{\pgfqpoint{1.223452in}{2.109858in}}{\pgfqpoint{1.219061in}{2.099259in}}{\pgfqpoint{1.219061in}{2.088209in}}%
\pgfpathcurveto{\pgfqpoint{1.219061in}{2.077159in}}{\pgfqpoint{1.223452in}{2.066560in}}{\pgfqpoint{1.231265in}{2.058746in}}%
\pgfpathcurveto{\pgfqpoint{1.239079in}{2.050933in}}{\pgfqpoint{1.249678in}{2.046542in}}{\pgfqpoint{1.260728in}{2.046542in}}%
\pgfpathlineto{\pgfqpoint{1.260728in}{2.046542in}}%
\pgfpathclose%
\pgfusepath{stroke}%
\end{pgfscope}%
\begin{pgfscope}%
\pgfpathrectangle{\pgfqpoint{0.393053in}{0.375000in}}{\pgfqpoint{6.356833in}{5.175000in}}%
\pgfusepath{clip}%
\pgfsetbuttcap%
\pgfsetroundjoin%
\pgfsetlinewidth{1.003750pt}%
\definecolor{currentstroke}{rgb}{0.827451,0.827451,0.827451}%
\pgfsetstrokecolor{currentstroke}%
\pgfsetdash{}{0pt}%
\pgfpathmoveto{\pgfqpoint{1.274887in}{2.046521in}}%
\pgfpathcurveto{\pgfqpoint{1.285937in}{2.046521in}}{\pgfqpoint{1.296536in}{2.050911in}}{\pgfqpoint{1.304350in}{2.058725in}}%
\pgfpathcurveto{\pgfqpoint{1.312164in}{2.066538in}}{\pgfqpoint{1.316554in}{2.077137in}}{\pgfqpoint{1.316554in}{2.088187in}}%
\pgfpathcurveto{\pgfqpoint{1.316554in}{2.099237in}}{\pgfqpoint{1.312164in}{2.109836in}}{\pgfqpoint{1.304350in}{2.117650in}}%
\pgfpathcurveto{\pgfqpoint{1.296536in}{2.125464in}}{\pgfqpoint{1.285937in}{2.129854in}}{\pgfqpoint{1.274887in}{2.129854in}}%
\pgfpathcurveto{\pgfqpoint{1.263837in}{2.129854in}}{\pgfqpoint{1.253238in}{2.125464in}}{\pgfqpoint{1.245425in}{2.117650in}}%
\pgfpathcurveto{\pgfqpoint{1.237611in}{2.109836in}}{\pgfqpoint{1.233221in}{2.099237in}}{\pgfqpoint{1.233221in}{2.088187in}}%
\pgfpathcurveto{\pgfqpoint{1.233221in}{2.077137in}}{\pgfqpoint{1.237611in}{2.066538in}}{\pgfqpoint{1.245425in}{2.058725in}}%
\pgfpathcurveto{\pgfqpoint{1.253238in}{2.050911in}}{\pgfqpoint{1.263837in}{2.046521in}}{\pgfqpoint{1.274887in}{2.046521in}}%
\pgfpathlineto{\pgfqpoint{1.274887in}{2.046521in}}%
\pgfpathclose%
\pgfusepath{stroke}%
\end{pgfscope}%
\begin{pgfscope}%
\pgfpathrectangle{\pgfqpoint{0.393053in}{0.375000in}}{\pgfqpoint{6.356833in}{5.175000in}}%
\pgfusepath{clip}%
\pgfsetbuttcap%
\pgfsetroundjoin%
\pgfsetlinewidth{1.003750pt}%
\definecolor{currentstroke}{rgb}{0.827451,0.827451,0.827451}%
\pgfsetstrokecolor{currentstroke}%
\pgfsetdash{}{0pt}%
\pgfpathmoveto{\pgfqpoint{3.923522in}{0.572263in}}%
\pgfpathcurveto{\pgfqpoint{3.934572in}{0.572263in}}{\pgfqpoint{3.945171in}{0.576653in}}{\pgfqpoint{3.952985in}{0.584467in}}%
\pgfpathcurveto{\pgfqpoint{3.960798in}{0.592281in}}{\pgfqpoint{3.965188in}{0.602880in}}{\pgfqpoint{3.965188in}{0.613930in}}%
\pgfpathcurveto{\pgfqpoint{3.965188in}{0.624980in}}{\pgfqpoint{3.960798in}{0.635579in}}{\pgfqpoint{3.952985in}{0.643393in}}%
\pgfpathcurveto{\pgfqpoint{3.945171in}{0.651206in}}{\pgfqpoint{3.934572in}{0.655597in}}{\pgfqpoint{3.923522in}{0.655597in}}%
\pgfpathcurveto{\pgfqpoint{3.912472in}{0.655597in}}{\pgfqpoint{3.901873in}{0.651206in}}{\pgfqpoint{3.894059in}{0.643393in}}%
\pgfpathcurveto{\pgfqpoint{3.886245in}{0.635579in}}{\pgfqpoint{3.881855in}{0.624980in}}{\pgfqpoint{3.881855in}{0.613930in}}%
\pgfpathcurveto{\pgfqpoint{3.881855in}{0.602880in}}{\pgfqpoint{3.886245in}{0.592281in}}{\pgfqpoint{3.894059in}{0.584467in}}%
\pgfpathcurveto{\pgfqpoint{3.901873in}{0.576653in}}{\pgfqpoint{3.912472in}{0.572263in}}{\pgfqpoint{3.923522in}{0.572263in}}%
\pgfpathlineto{\pgfqpoint{3.923522in}{0.572263in}}%
\pgfpathclose%
\pgfusepath{stroke}%
\end{pgfscope}%
\begin{pgfscope}%
\pgfpathrectangle{\pgfqpoint{0.393053in}{0.375000in}}{\pgfqpoint{6.356833in}{5.175000in}}%
\pgfusepath{clip}%
\pgfsetbuttcap%
\pgfsetroundjoin%
\pgfsetlinewidth{1.003750pt}%
\definecolor{currentstroke}{rgb}{0.827451,0.827451,0.827451}%
\pgfsetstrokecolor{currentstroke}%
\pgfsetdash{}{0pt}%
\pgfpathmoveto{\pgfqpoint{0.832747in}{3.136019in}}%
\pgfpathcurveto{\pgfqpoint{0.843797in}{3.136019in}}{\pgfqpoint{0.854396in}{3.140409in}}{\pgfqpoint{0.862210in}{3.148223in}}%
\pgfpathcurveto{\pgfqpoint{0.870024in}{3.156036in}}{\pgfqpoint{0.874414in}{3.166635in}}{\pgfqpoint{0.874414in}{3.177685in}}%
\pgfpathcurveto{\pgfqpoint{0.874414in}{3.188735in}}{\pgfqpoint{0.870024in}{3.199334in}}{\pgfqpoint{0.862210in}{3.207148in}}%
\pgfpathcurveto{\pgfqpoint{0.854396in}{3.214962in}}{\pgfqpoint{0.843797in}{3.219352in}}{\pgfqpoint{0.832747in}{3.219352in}}%
\pgfpathcurveto{\pgfqpoint{0.821697in}{3.219352in}}{\pgfqpoint{0.811098in}{3.214962in}}{\pgfqpoint{0.803284in}{3.207148in}}%
\pgfpathcurveto{\pgfqpoint{0.795471in}{3.199334in}}{\pgfqpoint{0.791081in}{3.188735in}}{\pgfqpoint{0.791081in}{3.177685in}}%
\pgfpathcurveto{\pgfqpoint{0.791081in}{3.166635in}}{\pgfqpoint{0.795471in}{3.156036in}}{\pgfqpoint{0.803284in}{3.148223in}}%
\pgfpathcurveto{\pgfqpoint{0.811098in}{3.140409in}}{\pgfqpoint{0.821697in}{3.136019in}}{\pgfqpoint{0.832747in}{3.136019in}}%
\pgfpathlineto{\pgfqpoint{0.832747in}{3.136019in}}%
\pgfpathclose%
\pgfusepath{stroke}%
\end{pgfscope}%
\begin{pgfscope}%
\pgfpathrectangle{\pgfqpoint{0.393053in}{0.375000in}}{\pgfqpoint{6.356833in}{5.175000in}}%
\pgfusepath{clip}%
\pgfsetbuttcap%
\pgfsetroundjoin%
\pgfsetlinewidth{1.003750pt}%
\definecolor{currentstroke}{rgb}{0.827451,0.827451,0.827451}%
\pgfsetstrokecolor{currentstroke}%
\pgfsetdash{}{0pt}%
\pgfpathmoveto{\pgfqpoint{5.564780in}{0.432326in}}%
\pgfpathcurveto{\pgfqpoint{5.575830in}{0.432326in}}{\pgfqpoint{5.586429in}{0.436716in}}{\pgfqpoint{5.594243in}{0.444530in}}%
\pgfpathcurveto{\pgfqpoint{5.602057in}{0.452343in}}{\pgfqpoint{5.606447in}{0.462942in}}{\pgfqpoint{5.606447in}{0.473992in}}%
\pgfpathcurveto{\pgfqpoint{5.606447in}{0.485043in}}{\pgfqpoint{5.602057in}{0.495642in}}{\pgfqpoint{5.594243in}{0.503455in}}%
\pgfpathcurveto{\pgfqpoint{5.586429in}{0.511269in}}{\pgfqpoint{5.575830in}{0.515659in}}{\pgfqpoint{5.564780in}{0.515659in}}%
\pgfpathcurveto{\pgfqpoint{5.553730in}{0.515659in}}{\pgfqpoint{5.543131in}{0.511269in}}{\pgfqpoint{5.535317in}{0.503455in}}%
\pgfpathcurveto{\pgfqpoint{5.527504in}{0.495642in}}{\pgfqpoint{5.523114in}{0.485043in}}{\pgfqpoint{5.523114in}{0.473992in}}%
\pgfpathcurveto{\pgfqpoint{5.523114in}{0.462942in}}{\pgfqpoint{5.527504in}{0.452343in}}{\pgfqpoint{5.535317in}{0.444530in}}%
\pgfpathcurveto{\pgfqpoint{5.543131in}{0.436716in}}{\pgfqpoint{5.553730in}{0.432326in}}{\pgfqpoint{5.564780in}{0.432326in}}%
\pgfpathlineto{\pgfqpoint{5.564780in}{0.432326in}}%
\pgfpathclose%
\pgfusepath{stroke}%
\end{pgfscope}%
\begin{pgfscope}%
\pgfpathrectangle{\pgfqpoint{0.393053in}{0.375000in}}{\pgfqpoint{6.356833in}{5.175000in}}%
\pgfusepath{clip}%
\pgfsetbuttcap%
\pgfsetroundjoin%
\pgfsetlinewidth{1.003750pt}%
\definecolor{currentstroke}{rgb}{0.827451,0.827451,0.827451}%
\pgfsetstrokecolor{currentstroke}%
\pgfsetdash{}{0pt}%
\pgfpathmoveto{\pgfqpoint{1.331869in}{2.038531in}}%
\pgfpathcurveto{\pgfqpoint{1.342919in}{2.038531in}}{\pgfqpoint{1.353518in}{2.042922in}}{\pgfqpoint{1.361331in}{2.050735in}}%
\pgfpathcurveto{\pgfqpoint{1.369145in}{2.058549in}}{\pgfqpoint{1.373535in}{2.069148in}}{\pgfqpoint{1.373535in}{2.080198in}}%
\pgfpathcurveto{\pgfqpoint{1.373535in}{2.091248in}}{\pgfqpoint{1.369145in}{2.101847in}}{\pgfqpoint{1.361331in}{2.109661in}}%
\pgfpathcurveto{\pgfqpoint{1.353518in}{2.117475in}}{\pgfqpoint{1.342919in}{2.121865in}}{\pgfqpoint{1.331869in}{2.121865in}}%
\pgfpathcurveto{\pgfqpoint{1.320818in}{2.121865in}}{\pgfqpoint{1.310219in}{2.117475in}}{\pgfqpoint{1.302406in}{2.109661in}}%
\pgfpathcurveto{\pgfqpoint{1.294592in}{2.101847in}}{\pgfqpoint{1.290202in}{2.091248in}}{\pgfqpoint{1.290202in}{2.080198in}}%
\pgfpathcurveto{\pgfqpoint{1.290202in}{2.069148in}}{\pgfqpoint{1.294592in}{2.058549in}}{\pgfqpoint{1.302406in}{2.050735in}}%
\pgfpathcurveto{\pgfqpoint{1.310219in}{2.042922in}}{\pgfqpoint{1.320818in}{2.038531in}}{\pgfqpoint{1.331869in}{2.038531in}}%
\pgfpathlineto{\pgfqpoint{1.331869in}{2.038531in}}%
\pgfpathclose%
\pgfusepath{stroke}%
\end{pgfscope}%
\begin{pgfscope}%
\pgfpathrectangle{\pgfqpoint{0.393053in}{0.375000in}}{\pgfqpoint{6.356833in}{5.175000in}}%
\pgfusepath{clip}%
\pgfsetbuttcap%
\pgfsetroundjoin%
\pgfsetlinewidth{1.003750pt}%
\definecolor{currentstroke}{rgb}{0.827451,0.827451,0.827451}%
\pgfsetstrokecolor{currentstroke}%
\pgfsetdash{}{0pt}%
\pgfpathmoveto{\pgfqpoint{1.707565in}{1.606245in}}%
\pgfpathcurveto{\pgfqpoint{1.718615in}{1.606245in}}{\pgfqpoint{1.729214in}{1.610635in}}{\pgfqpoint{1.737028in}{1.618449in}}%
\pgfpathcurveto{\pgfqpoint{1.744842in}{1.626263in}}{\pgfqpoint{1.749232in}{1.636862in}}{\pgfqpoint{1.749232in}{1.647912in}}%
\pgfpathcurveto{\pgfqpoint{1.749232in}{1.658962in}}{\pgfqpoint{1.744842in}{1.669561in}}{\pgfqpoint{1.737028in}{1.677375in}}%
\pgfpathcurveto{\pgfqpoint{1.729214in}{1.685188in}}{\pgfqpoint{1.718615in}{1.689578in}}{\pgfqpoint{1.707565in}{1.689578in}}%
\pgfpathcurveto{\pgfqpoint{1.696515in}{1.689578in}}{\pgfqpoint{1.685916in}{1.685188in}}{\pgfqpoint{1.678102in}{1.677375in}}%
\pgfpathcurveto{\pgfqpoint{1.670289in}{1.669561in}}{\pgfqpoint{1.665899in}{1.658962in}}{\pgfqpoint{1.665899in}{1.647912in}}%
\pgfpathcurveto{\pgfqpoint{1.665899in}{1.636862in}}{\pgfqpoint{1.670289in}{1.626263in}}{\pgfqpoint{1.678102in}{1.618449in}}%
\pgfpathcurveto{\pgfqpoint{1.685916in}{1.610635in}}{\pgfqpoint{1.696515in}{1.606245in}}{\pgfqpoint{1.707565in}{1.606245in}}%
\pgfpathlineto{\pgfqpoint{1.707565in}{1.606245in}}%
\pgfpathclose%
\pgfusepath{stroke}%
\end{pgfscope}%
\begin{pgfscope}%
\pgfpathrectangle{\pgfqpoint{0.393053in}{0.375000in}}{\pgfqpoint{6.356833in}{5.175000in}}%
\pgfusepath{clip}%
\pgfsetbuttcap%
\pgfsetroundjoin%
\pgfsetlinewidth{1.003750pt}%
\definecolor{currentstroke}{rgb}{0.827451,0.827451,0.827451}%
\pgfsetstrokecolor{currentstroke}%
\pgfsetdash{}{0pt}%
\pgfpathmoveto{\pgfqpoint{0.610193in}{3.339690in}}%
\pgfpathcurveto{\pgfqpoint{0.621243in}{3.339690in}}{\pgfqpoint{0.631842in}{3.344080in}}{\pgfqpoint{0.639655in}{3.351894in}}%
\pgfpathcurveto{\pgfqpoint{0.647469in}{3.359707in}}{\pgfqpoint{0.651859in}{3.370306in}}{\pgfqpoint{0.651859in}{3.381357in}}%
\pgfpathcurveto{\pgfqpoint{0.651859in}{3.392407in}}{\pgfqpoint{0.647469in}{3.403006in}}{\pgfqpoint{0.639655in}{3.410819in}}%
\pgfpathcurveto{\pgfqpoint{0.631842in}{3.418633in}}{\pgfqpoint{0.621243in}{3.423023in}}{\pgfqpoint{0.610193in}{3.423023in}}%
\pgfpathcurveto{\pgfqpoint{0.599143in}{3.423023in}}{\pgfqpoint{0.588544in}{3.418633in}}{\pgfqpoint{0.580730in}{3.410819in}}%
\pgfpathcurveto{\pgfqpoint{0.572916in}{3.403006in}}{\pgfqpoint{0.568526in}{3.392407in}}{\pgfqpoint{0.568526in}{3.381357in}}%
\pgfpathcurveto{\pgfqpoint{0.568526in}{3.370306in}}{\pgfqpoint{0.572916in}{3.359707in}}{\pgfqpoint{0.580730in}{3.351894in}}%
\pgfpathcurveto{\pgfqpoint{0.588544in}{3.344080in}}{\pgfqpoint{0.599143in}{3.339690in}}{\pgfqpoint{0.610193in}{3.339690in}}%
\pgfpathlineto{\pgfqpoint{0.610193in}{3.339690in}}%
\pgfpathclose%
\pgfusepath{stroke}%
\end{pgfscope}%
\begin{pgfscope}%
\pgfpathrectangle{\pgfqpoint{0.393053in}{0.375000in}}{\pgfqpoint{6.356833in}{5.175000in}}%
\pgfusepath{clip}%
\pgfsetbuttcap%
\pgfsetroundjoin%
\pgfsetlinewidth{1.003750pt}%
\definecolor{currentstroke}{rgb}{0.827451,0.827451,0.827451}%
\pgfsetstrokecolor{currentstroke}%
\pgfsetdash{}{0pt}%
\pgfpathmoveto{\pgfqpoint{1.252210in}{2.215695in}}%
\pgfpathcurveto{\pgfqpoint{1.263260in}{2.215695in}}{\pgfqpoint{1.273859in}{2.220085in}}{\pgfqpoint{1.281673in}{2.227899in}}%
\pgfpathcurveto{\pgfqpoint{1.289486in}{2.235712in}}{\pgfqpoint{1.293877in}{2.246311in}}{\pgfqpoint{1.293877in}{2.257361in}}%
\pgfpathcurveto{\pgfqpoint{1.293877in}{2.268412in}}{\pgfqpoint{1.289486in}{2.279011in}}{\pgfqpoint{1.281673in}{2.286824in}}%
\pgfpathcurveto{\pgfqpoint{1.273859in}{2.294638in}}{\pgfqpoint{1.263260in}{2.299028in}}{\pgfqpoint{1.252210in}{2.299028in}}%
\pgfpathcurveto{\pgfqpoint{1.241160in}{2.299028in}}{\pgfqpoint{1.230561in}{2.294638in}}{\pgfqpoint{1.222747in}{2.286824in}}%
\pgfpathcurveto{\pgfqpoint{1.214934in}{2.279011in}}{\pgfqpoint{1.210543in}{2.268412in}}{\pgfqpoint{1.210543in}{2.257361in}}%
\pgfpathcurveto{\pgfqpoint{1.210543in}{2.246311in}}{\pgfqpoint{1.214934in}{2.235712in}}{\pgfqpoint{1.222747in}{2.227899in}}%
\pgfpathcurveto{\pgfqpoint{1.230561in}{2.220085in}}{\pgfqpoint{1.241160in}{2.215695in}}{\pgfqpoint{1.252210in}{2.215695in}}%
\pgfpathlineto{\pgfqpoint{1.252210in}{2.215695in}}%
\pgfpathclose%
\pgfusepath{stroke}%
\end{pgfscope}%
\begin{pgfscope}%
\pgfpathrectangle{\pgfqpoint{0.393053in}{0.375000in}}{\pgfqpoint{6.356833in}{5.175000in}}%
\pgfusepath{clip}%
\pgfsetbuttcap%
\pgfsetroundjoin%
\pgfsetlinewidth{1.003750pt}%
\definecolor{currentstroke}{rgb}{0.827451,0.827451,0.827451}%
\pgfsetstrokecolor{currentstroke}%
\pgfsetdash{}{0pt}%
\pgfpathmoveto{\pgfqpoint{1.234766in}{2.234095in}}%
\pgfpathcurveto{\pgfqpoint{1.245817in}{2.234095in}}{\pgfqpoint{1.256416in}{2.238485in}}{\pgfqpoint{1.264229in}{2.246299in}}%
\pgfpathcurveto{\pgfqpoint{1.272043in}{2.254113in}}{\pgfqpoint{1.276433in}{2.264712in}}{\pgfqpoint{1.276433in}{2.275762in}}%
\pgfpathcurveto{\pgfqpoint{1.276433in}{2.286812in}}{\pgfqpoint{1.272043in}{2.297411in}}{\pgfqpoint{1.264229in}{2.305224in}}%
\pgfpathcurveto{\pgfqpoint{1.256416in}{2.313038in}}{\pgfqpoint{1.245817in}{2.317428in}}{\pgfqpoint{1.234766in}{2.317428in}}%
\pgfpathcurveto{\pgfqpoint{1.223716in}{2.317428in}}{\pgfqpoint{1.213117in}{2.313038in}}{\pgfqpoint{1.205304in}{2.305224in}}%
\pgfpathcurveto{\pgfqpoint{1.197490in}{2.297411in}}{\pgfqpoint{1.193100in}{2.286812in}}{\pgfqpoint{1.193100in}{2.275762in}}%
\pgfpathcurveto{\pgfqpoint{1.193100in}{2.264712in}}{\pgfqpoint{1.197490in}{2.254113in}}{\pgfqpoint{1.205304in}{2.246299in}}%
\pgfpathcurveto{\pgfqpoint{1.213117in}{2.238485in}}{\pgfqpoint{1.223716in}{2.234095in}}{\pgfqpoint{1.234766in}{2.234095in}}%
\pgfpathlineto{\pgfqpoint{1.234766in}{2.234095in}}%
\pgfpathclose%
\pgfusepath{stroke}%
\end{pgfscope}%
\begin{pgfscope}%
\pgfpathrectangle{\pgfqpoint{0.393053in}{0.375000in}}{\pgfqpoint{6.356833in}{5.175000in}}%
\pgfusepath{clip}%
\pgfsetbuttcap%
\pgfsetroundjoin%
\pgfsetlinewidth{1.003750pt}%
\definecolor{currentstroke}{rgb}{0.827451,0.827451,0.827451}%
\pgfsetstrokecolor{currentstroke}%
\pgfsetdash{}{0pt}%
\pgfpathmoveto{\pgfqpoint{3.282049in}{0.765764in}}%
\pgfpathcurveto{\pgfqpoint{3.293099in}{0.765764in}}{\pgfqpoint{3.303698in}{0.770154in}}{\pgfqpoint{3.311512in}{0.777968in}}%
\pgfpathcurveto{\pgfqpoint{3.319326in}{0.785782in}}{\pgfqpoint{3.323716in}{0.796381in}}{\pgfqpoint{3.323716in}{0.807431in}}%
\pgfpathcurveto{\pgfqpoint{3.323716in}{0.818481in}}{\pgfqpoint{3.319326in}{0.829080in}}{\pgfqpoint{3.311512in}{0.836894in}}%
\pgfpathcurveto{\pgfqpoint{3.303698in}{0.844707in}}{\pgfqpoint{3.293099in}{0.849097in}}{\pgfqpoint{3.282049in}{0.849097in}}%
\pgfpathcurveto{\pgfqpoint{3.270999in}{0.849097in}}{\pgfqpoint{3.260400in}{0.844707in}}{\pgfqpoint{3.252587in}{0.836894in}}%
\pgfpathcurveto{\pgfqpoint{3.244773in}{0.829080in}}{\pgfqpoint{3.240383in}{0.818481in}}{\pgfqpoint{3.240383in}{0.807431in}}%
\pgfpathcurveto{\pgfqpoint{3.240383in}{0.796381in}}{\pgfqpoint{3.244773in}{0.785782in}}{\pgfqpoint{3.252587in}{0.777968in}}%
\pgfpathcurveto{\pgfqpoint{3.260400in}{0.770154in}}{\pgfqpoint{3.270999in}{0.765764in}}{\pgfqpoint{3.282049in}{0.765764in}}%
\pgfpathlineto{\pgfqpoint{3.282049in}{0.765764in}}%
\pgfpathclose%
\pgfusepath{stroke}%
\end{pgfscope}%
\begin{pgfscope}%
\pgfpathrectangle{\pgfqpoint{0.393053in}{0.375000in}}{\pgfqpoint{6.356833in}{5.175000in}}%
\pgfusepath{clip}%
\pgfsetbuttcap%
\pgfsetroundjoin%
\pgfsetlinewidth{1.003750pt}%
\definecolor{currentstroke}{rgb}{0.827451,0.827451,0.827451}%
\pgfsetstrokecolor{currentstroke}%
\pgfsetdash{}{0pt}%
\pgfpathmoveto{\pgfqpoint{3.611658in}{0.763879in}}%
\pgfpathcurveto{\pgfqpoint{3.622708in}{0.763879in}}{\pgfqpoint{3.633307in}{0.768269in}}{\pgfqpoint{3.641121in}{0.776083in}}%
\pgfpathcurveto{\pgfqpoint{3.648934in}{0.783896in}}{\pgfqpoint{3.653325in}{0.794496in}}{\pgfqpoint{3.653325in}{0.805546in}}%
\pgfpathcurveto{\pgfqpoint{3.653325in}{0.816596in}}{\pgfqpoint{3.648934in}{0.827195in}}{\pgfqpoint{3.641121in}{0.835008in}}%
\pgfpathcurveto{\pgfqpoint{3.633307in}{0.842822in}}{\pgfqpoint{3.622708in}{0.847212in}}{\pgfqpoint{3.611658in}{0.847212in}}%
\pgfpathcurveto{\pgfqpoint{3.600608in}{0.847212in}}{\pgfqpoint{3.590009in}{0.842822in}}{\pgfqpoint{3.582195in}{0.835008in}}%
\pgfpathcurveto{\pgfqpoint{3.574381in}{0.827195in}}{\pgfqpoint{3.569991in}{0.816596in}}{\pgfqpoint{3.569991in}{0.805546in}}%
\pgfpathcurveto{\pgfqpoint{3.569991in}{0.794496in}}{\pgfqpoint{3.574381in}{0.783896in}}{\pgfqpoint{3.582195in}{0.776083in}}%
\pgfpathcurveto{\pgfqpoint{3.590009in}{0.768269in}}{\pgfqpoint{3.600608in}{0.763879in}}{\pgfqpoint{3.611658in}{0.763879in}}%
\pgfpathlineto{\pgfqpoint{3.611658in}{0.763879in}}%
\pgfpathclose%
\pgfusepath{stroke}%
\end{pgfscope}%
\begin{pgfscope}%
\pgfpathrectangle{\pgfqpoint{0.393053in}{0.375000in}}{\pgfqpoint{6.356833in}{5.175000in}}%
\pgfusepath{clip}%
\pgfsetbuttcap%
\pgfsetroundjoin%
\pgfsetlinewidth{1.003750pt}%
\definecolor{currentstroke}{rgb}{0.827451,0.827451,0.827451}%
\pgfsetstrokecolor{currentstroke}%
\pgfsetdash{}{0pt}%
\pgfpathmoveto{\pgfqpoint{2.044607in}{1.403018in}}%
\pgfpathcurveto{\pgfqpoint{2.055657in}{1.403018in}}{\pgfqpoint{2.066256in}{1.407409in}}{\pgfqpoint{2.074070in}{1.415222in}}%
\pgfpathcurveto{\pgfqpoint{2.081883in}{1.423036in}}{\pgfqpoint{2.086273in}{1.433635in}}{\pgfqpoint{2.086273in}{1.444685in}}%
\pgfpathcurveto{\pgfqpoint{2.086273in}{1.455735in}}{\pgfqpoint{2.081883in}{1.466334in}}{\pgfqpoint{2.074070in}{1.474148in}}%
\pgfpathcurveto{\pgfqpoint{2.066256in}{1.481961in}}{\pgfqpoint{2.055657in}{1.486352in}}{\pgfqpoint{2.044607in}{1.486352in}}%
\pgfpathcurveto{\pgfqpoint{2.033557in}{1.486352in}}{\pgfqpoint{2.022958in}{1.481961in}}{\pgfqpoint{2.015144in}{1.474148in}}%
\pgfpathcurveto{\pgfqpoint{2.007330in}{1.466334in}}{\pgfqpoint{2.002940in}{1.455735in}}{\pgfqpoint{2.002940in}{1.444685in}}%
\pgfpathcurveto{\pgfqpoint{2.002940in}{1.433635in}}{\pgfqpoint{2.007330in}{1.423036in}}{\pgfqpoint{2.015144in}{1.415222in}}%
\pgfpathcurveto{\pgfqpoint{2.022958in}{1.407409in}}{\pgfqpoint{2.033557in}{1.403018in}}{\pgfqpoint{2.044607in}{1.403018in}}%
\pgfpathlineto{\pgfqpoint{2.044607in}{1.403018in}}%
\pgfpathclose%
\pgfusepath{stroke}%
\end{pgfscope}%
\begin{pgfscope}%
\pgfpathrectangle{\pgfqpoint{0.393053in}{0.375000in}}{\pgfqpoint{6.356833in}{5.175000in}}%
\pgfusepath{clip}%
\pgfsetbuttcap%
\pgfsetroundjoin%
\pgfsetlinewidth{1.003750pt}%
\definecolor{currentstroke}{rgb}{0.827451,0.827451,0.827451}%
\pgfsetstrokecolor{currentstroke}%
\pgfsetdash{}{0pt}%
\pgfpathmoveto{\pgfqpoint{0.669503in}{3.263449in}}%
\pgfpathcurveto{\pgfqpoint{0.680554in}{3.263449in}}{\pgfqpoint{0.691153in}{3.267840in}}{\pgfqpoint{0.698966in}{3.275653in}}%
\pgfpathcurveto{\pgfqpoint{0.706780in}{3.283467in}}{\pgfqpoint{0.711170in}{3.294066in}}{\pgfqpoint{0.711170in}{3.305116in}}%
\pgfpathcurveto{\pgfqpoint{0.711170in}{3.316166in}}{\pgfqpoint{0.706780in}{3.326765in}}{\pgfqpoint{0.698966in}{3.334579in}}%
\pgfpathcurveto{\pgfqpoint{0.691153in}{3.342392in}}{\pgfqpoint{0.680554in}{3.346783in}}{\pgfqpoint{0.669503in}{3.346783in}}%
\pgfpathcurveto{\pgfqpoint{0.658453in}{3.346783in}}{\pgfqpoint{0.647854in}{3.342392in}}{\pgfqpoint{0.640041in}{3.334579in}}%
\pgfpathcurveto{\pgfqpoint{0.632227in}{3.326765in}}{\pgfqpoint{0.627837in}{3.316166in}}{\pgfqpoint{0.627837in}{3.305116in}}%
\pgfpathcurveto{\pgfqpoint{0.627837in}{3.294066in}}{\pgfqpoint{0.632227in}{3.283467in}}{\pgfqpoint{0.640041in}{3.275653in}}%
\pgfpathcurveto{\pgfqpoint{0.647854in}{3.267840in}}{\pgfqpoint{0.658453in}{3.263449in}}{\pgfqpoint{0.669503in}{3.263449in}}%
\pgfpathlineto{\pgfqpoint{0.669503in}{3.263449in}}%
\pgfpathclose%
\pgfusepath{stroke}%
\end{pgfscope}%
\begin{pgfscope}%
\pgfpathrectangle{\pgfqpoint{0.393053in}{0.375000in}}{\pgfqpoint{6.356833in}{5.175000in}}%
\pgfusepath{clip}%
\pgfsetbuttcap%
\pgfsetroundjoin%
\pgfsetlinewidth{1.003750pt}%
\definecolor{currentstroke}{rgb}{0.827451,0.827451,0.827451}%
\pgfsetstrokecolor{currentstroke}%
\pgfsetdash{}{0pt}%
\pgfpathmoveto{\pgfqpoint{2.440737in}{1.390559in}}%
\pgfpathcurveto{\pgfqpoint{2.451787in}{1.390559in}}{\pgfqpoint{2.462386in}{1.394949in}}{\pgfqpoint{2.470200in}{1.402763in}}%
\pgfpathcurveto{\pgfqpoint{2.478013in}{1.410577in}}{\pgfqpoint{2.482404in}{1.421176in}}{\pgfqpoint{2.482404in}{1.432226in}}%
\pgfpathcurveto{\pgfqpoint{2.482404in}{1.443276in}}{\pgfqpoint{2.478013in}{1.453875in}}{\pgfqpoint{2.470200in}{1.461689in}}%
\pgfpathcurveto{\pgfqpoint{2.462386in}{1.469502in}}{\pgfqpoint{2.451787in}{1.473892in}}{\pgfqpoint{2.440737in}{1.473892in}}%
\pgfpathcurveto{\pgfqpoint{2.429687in}{1.473892in}}{\pgfqpoint{2.419088in}{1.469502in}}{\pgfqpoint{2.411274in}{1.461689in}}%
\pgfpathcurveto{\pgfqpoint{2.403461in}{1.453875in}}{\pgfqpoint{2.399070in}{1.443276in}}{\pgfqpoint{2.399070in}{1.432226in}}%
\pgfpathcurveto{\pgfqpoint{2.399070in}{1.421176in}}{\pgfqpoint{2.403461in}{1.410577in}}{\pgfqpoint{2.411274in}{1.402763in}}%
\pgfpathcurveto{\pgfqpoint{2.419088in}{1.394949in}}{\pgfqpoint{2.429687in}{1.390559in}}{\pgfqpoint{2.440737in}{1.390559in}}%
\pgfpathlineto{\pgfqpoint{2.440737in}{1.390559in}}%
\pgfpathclose%
\pgfusepath{stroke}%
\end{pgfscope}%
\begin{pgfscope}%
\pgfpathrectangle{\pgfqpoint{0.393053in}{0.375000in}}{\pgfqpoint{6.356833in}{5.175000in}}%
\pgfusepath{clip}%
\pgfsetbuttcap%
\pgfsetroundjoin%
\pgfsetlinewidth{1.003750pt}%
\definecolor{currentstroke}{rgb}{0.827451,0.827451,0.827451}%
\pgfsetstrokecolor{currentstroke}%
\pgfsetdash{}{0pt}%
\pgfpathmoveto{\pgfqpoint{2.787676in}{1.049368in}}%
\pgfpathcurveto{\pgfqpoint{2.798726in}{1.049368in}}{\pgfqpoint{2.809325in}{1.053758in}}{\pgfqpoint{2.817139in}{1.061572in}}%
\pgfpathcurveto{\pgfqpoint{2.824953in}{1.069386in}}{\pgfqpoint{2.829343in}{1.079985in}}{\pgfqpoint{2.829343in}{1.091035in}}%
\pgfpathcurveto{\pgfqpoint{2.829343in}{1.102085in}}{\pgfqpoint{2.824953in}{1.112684in}}{\pgfqpoint{2.817139in}{1.120498in}}%
\pgfpathcurveto{\pgfqpoint{2.809325in}{1.128311in}}{\pgfqpoint{2.798726in}{1.132701in}}{\pgfqpoint{2.787676in}{1.132701in}}%
\pgfpathcurveto{\pgfqpoint{2.776626in}{1.132701in}}{\pgfqpoint{2.766027in}{1.128311in}}{\pgfqpoint{2.758213in}{1.120498in}}%
\pgfpathcurveto{\pgfqpoint{2.750400in}{1.112684in}}{\pgfqpoint{2.746009in}{1.102085in}}{\pgfqpoint{2.746009in}{1.091035in}}%
\pgfpathcurveto{\pgfqpoint{2.746009in}{1.079985in}}{\pgfqpoint{2.750400in}{1.069386in}}{\pgfqpoint{2.758213in}{1.061572in}}%
\pgfpathcurveto{\pgfqpoint{2.766027in}{1.053758in}}{\pgfqpoint{2.776626in}{1.049368in}}{\pgfqpoint{2.787676in}{1.049368in}}%
\pgfpathlineto{\pgfqpoint{2.787676in}{1.049368in}}%
\pgfpathclose%
\pgfusepath{stroke}%
\end{pgfscope}%
\begin{pgfscope}%
\pgfpathrectangle{\pgfqpoint{0.393053in}{0.375000in}}{\pgfqpoint{6.356833in}{5.175000in}}%
\pgfusepath{clip}%
\pgfsetbuttcap%
\pgfsetroundjoin%
\pgfsetlinewidth{1.003750pt}%
\definecolor{currentstroke}{rgb}{0.827451,0.827451,0.827451}%
\pgfsetstrokecolor{currentstroke}%
\pgfsetdash{}{0pt}%
\pgfpathmoveto{\pgfqpoint{1.893927in}{1.548232in}}%
\pgfpathcurveto{\pgfqpoint{1.904977in}{1.548232in}}{\pgfqpoint{1.915576in}{1.552622in}}{\pgfqpoint{1.923390in}{1.560435in}}%
\pgfpathcurveto{\pgfqpoint{1.931203in}{1.568249in}}{\pgfqpoint{1.935594in}{1.578848in}}{\pgfqpoint{1.935594in}{1.589898in}}%
\pgfpathcurveto{\pgfqpoint{1.935594in}{1.600948in}}{\pgfqpoint{1.931203in}{1.611547in}}{\pgfqpoint{1.923390in}{1.619361in}}%
\pgfpathcurveto{\pgfqpoint{1.915576in}{1.627175in}}{\pgfqpoint{1.904977in}{1.631565in}}{\pgfqpoint{1.893927in}{1.631565in}}%
\pgfpathcurveto{\pgfqpoint{1.882877in}{1.631565in}}{\pgfqpoint{1.872278in}{1.627175in}}{\pgfqpoint{1.864464in}{1.619361in}}%
\pgfpathcurveto{\pgfqpoint{1.856650in}{1.611547in}}{\pgfqpoint{1.852260in}{1.600948in}}{\pgfqpoint{1.852260in}{1.589898in}}%
\pgfpathcurveto{\pgfqpoint{1.852260in}{1.578848in}}{\pgfqpoint{1.856650in}{1.568249in}}{\pgfqpoint{1.864464in}{1.560435in}}%
\pgfpathcurveto{\pgfqpoint{1.872278in}{1.552622in}}{\pgfqpoint{1.882877in}{1.548232in}}{\pgfqpoint{1.893927in}{1.548232in}}%
\pgfpathlineto{\pgfqpoint{1.893927in}{1.548232in}}%
\pgfpathclose%
\pgfusepath{stroke}%
\end{pgfscope}%
\begin{pgfscope}%
\pgfpathrectangle{\pgfqpoint{0.393053in}{0.375000in}}{\pgfqpoint{6.356833in}{5.175000in}}%
\pgfusepath{clip}%
\pgfsetbuttcap%
\pgfsetroundjoin%
\pgfsetlinewidth{1.003750pt}%
\definecolor{currentstroke}{rgb}{0.827451,0.827451,0.827451}%
\pgfsetstrokecolor{currentstroke}%
\pgfsetdash{}{0pt}%
\pgfpathmoveto{\pgfqpoint{1.868551in}{1.561864in}}%
\pgfpathcurveto{\pgfqpoint{1.879601in}{1.561864in}}{\pgfqpoint{1.890200in}{1.566254in}}{\pgfqpoint{1.898014in}{1.574068in}}%
\pgfpathcurveto{\pgfqpoint{1.905828in}{1.581882in}}{\pgfqpoint{1.910218in}{1.592481in}}{\pgfqpoint{1.910218in}{1.603531in}}%
\pgfpathcurveto{\pgfqpoint{1.910218in}{1.614581in}}{\pgfqpoint{1.905828in}{1.625180in}}{\pgfqpoint{1.898014in}{1.632993in}}%
\pgfpathcurveto{\pgfqpoint{1.890200in}{1.640807in}}{\pgfqpoint{1.879601in}{1.645197in}}{\pgfqpoint{1.868551in}{1.645197in}}%
\pgfpathcurveto{\pgfqpoint{1.857501in}{1.645197in}}{\pgfqpoint{1.846902in}{1.640807in}}{\pgfqpoint{1.839089in}{1.632993in}}%
\pgfpathcurveto{\pgfqpoint{1.831275in}{1.625180in}}{\pgfqpoint{1.826885in}{1.614581in}}{\pgfqpoint{1.826885in}{1.603531in}}%
\pgfpathcurveto{\pgfqpoint{1.826885in}{1.592481in}}{\pgfqpoint{1.831275in}{1.581882in}}{\pgfqpoint{1.839089in}{1.574068in}}%
\pgfpathcurveto{\pgfqpoint{1.846902in}{1.566254in}}{\pgfqpoint{1.857501in}{1.561864in}}{\pgfqpoint{1.868551in}{1.561864in}}%
\pgfpathlineto{\pgfqpoint{1.868551in}{1.561864in}}%
\pgfpathclose%
\pgfusepath{stroke}%
\end{pgfscope}%
\begin{pgfscope}%
\pgfpathrectangle{\pgfqpoint{0.393053in}{0.375000in}}{\pgfqpoint{6.356833in}{5.175000in}}%
\pgfusepath{clip}%
\pgfsetbuttcap%
\pgfsetroundjoin%
\pgfsetlinewidth{1.003750pt}%
\definecolor{currentstroke}{rgb}{0.827451,0.827451,0.827451}%
\pgfsetstrokecolor{currentstroke}%
\pgfsetdash{}{0pt}%
\pgfpathmoveto{\pgfqpoint{4.229679in}{0.486406in}}%
\pgfpathcurveto{\pgfqpoint{4.240729in}{0.486406in}}{\pgfqpoint{4.251328in}{0.490796in}}{\pgfqpoint{4.259142in}{0.498610in}}%
\pgfpathcurveto{\pgfqpoint{4.266956in}{0.506423in}}{\pgfqpoint{4.271346in}{0.517022in}}{\pgfqpoint{4.271346in}{0.528072in}}%
\pgfpathcurveto{\pgfqpoint{4.271346in}{0.539123in}}{\pgfqpoint{4.266956in}{0.549722in}}{\pgfqpoint{4.259142in}{0.557535in}}%
\pgfpathcurveto{\pgfqpoint{4.251328in}{0.565349in}}{\pgfqpoint{4.240729in}{0.569739in}}{\pgfqpoint{4.229679in}{0.569739in}}%
\pgfpathcurveto{\pgfqpoint{4.218629in}{0.569739in}}{\pgfqpoint{4.208030in}{0.565349in}}{\pgfqpoint{4.200216in}{0.557535in}}%
\pgfpathcurveto{\pgfqpoint{4.192403in}{0.549722in}}{\pgfqpoint{4.188013in}{0.539123in}}{\pgfqpoint{4.188013in}{0.528072in}}%
\pgfpathcurveto{\pgfqpoint{4.188013in}{0.517022in}}{\pgfqpoint{4.192403in}{0.506423in}}{\pgfqpoint{4.200216in}{0.498610in}}%
\pgfpathcurveto{\pgfqpoint{4.208030in}{0.490796in}}{\pgfqpoint{4.218629in}{0.486406in}}{\pgfqpoint{4.229679in}{0.486406in}}%
\pgfpathlineto{\pgfqpoint{4.229679in}{0.486406in}}%
\pgfpathclose%
\pgfusepath{stroke}%
\end{pgfscope}%
\begin{pgfscope}%
\pgfpathrectangle{\pgfqpoint{0.393053in}{0.375000in}}{\pgfqpoint{6.356833in}{5.175000in}}%
\pgfusepath{clip}%
\pgfsetbuttcap%
\pgfsetroundjoin%
\pgfsetlinewidth{1.003750pt}%
\definecolor{currentstroke}{rgb}{0.827451,0.827451,0.827451}%
\pgfsetstrokecolor{currentstroke}%
\pgfsetdash{}{0pt}%
\pgfpathmoveto{\pgfqpoint{5.128133in}{0.470706in}}%
\pgfpathcurveto{\pgfqpoint{5.139183in}{0.470706in}}{\pgfqpoint{5.149782in}{0.475096in}}{\pgfqpoint{5.157596in}{0.482910in}}%
\pgfpathcurveto{\pgfqpoint{5.165409in}{0.490723in}}{\pgfqpoint{5.169800in}{0.501322in}}{\pgfqpoint{5.169800in}{0.512372in}}%
\pgfpathcurveto{\pgfqpoint{5.169800in}{0.523422in}}{\pgfqpoint{5.165409in}{0.534021in}}{\pgfqpoint{5.157596in}{0.541835in}}%
\pgfpathcurveto{\pgfqpoint{5.149782in}{0.549649in}}{\pgfqpoint{5.139183in}{0.554039in}}{\pgfqpoint{5.128133in}{0.554039in}}%
\pgfpathcurveto{\pgfqpoint{5.117083in}{0.554039in}}{\pgfqpoint{5.106484in}{0.549649in}}{\pgfqpoint{5.098670in}{0.541835in}}%
\pgfpathcurveto{\pgfqpoint{5.090857in}{0.534021in}}{\pgfqpoint{5.086466in}{0.523422in}}{\pgfqpoint{5.086466in}{0.512372in}}%
\pgfpathcurveto{\pgfqpoint{5.086466in}{0.501322in}}{\pgfqpoint{5.090857in}{0.490723in}}{\pgfqpoint{5.098670in}{0.482910in}}%
\pgfpathcurveto{\pgfqpoint{5.106484in}{0.475096in}}{\pgfqpoint{5.117083in}{0.470706in}}{\pgfqpoint{5.128133in}{0.470706in}}%
\pgfpathlineto{\pgfqpoint{5.128133in}{0.470706in}}%
\pgfpathclose%
\pgfusepath{stroke}%
\end{pgfscope}%
\begin{pgfscope}%
\pgfpathrectangle{\pgfqpoint{0.393053in}{0.375000in}}{\pgfqpoint{6.356833in}{5.175000in}}%
\pgfusepath{clip}%
\pgfsetbuttcap%
\pgfsetroundjoin%
\pgfsetlinewidth{1.003750pt}%
\definecolor{currentstroke}{rgb}{0.827451,0.827451,0.827451}%
\pgfsetstrokecolor{currentstroke}%
\pgfsetdash{}{0pt}%
\pgfpathmoveto{\pgfqpoint{5.135829in}{0.466814in}}%
\pgfpathcurveto{\pgfqpoint{5.146879in}{0.466814in}}{\pgfqpoint{5.157478in}{0.471204in}}{\pgfqpoint{5.165292in}{0.479018in}}%
\pgfpathcurveto{\pgfqpoint{5.173106in}{0.486832in}}{\pgfqpoint{5.177496in}{0.497431in}}{\pgfqpoint{5.177496in}{0.508481in}}%
\pgfpathcurveto{\pgfqpoint{5.177496in}{0.519531in}}{\pgfqpoint{5.173106in}{0.530130in}}{\pgfqpoint{5.165292in}{0.537944in}}%
\pgfpathcurveto{\pgfqpoint{5.157478in}{0.545757in}}{\pgfqpoint{5.146879in}{0.550147in}}{\pgfqpoint{5.135829in}{0.550147in}}%
\pgfpathcurveto{\pgfqpoint{5.124779in}{0.550147in}}{\pgfqpoint{5.114180in}{0.545757in}}{\pgfqpoint{5.106367in}{0.537944in}}%
\pgfpathcurveto{\pgfqpoint{5.098553in}{0.530130in}}{\pgfqpoint{5.094163in}{0.519531in}}{\pgfqpoint{5.094163in}{0.508481in}}%
\pgfpathcurveto{\pgfqpoint{5.094163in}{0.497431in}}{\pgfqpoint{5.098553in}{0.486832in}}{\pgfqpoint{5.106367in}{0.479018in}}%
\pgfpathcurveto{\pgfqpoint{5.114180in}{0.471204in}}{\pgfqpoint{5.124779in}{0.466814in}}{\pgfqpoint{5.135829in}{0.466814in}}%
\pgfpathlineto{\pgfqpoint{5.135829in}{0.466814in}}%
\pgfpathclose%
\pgfusepath{stroke}%
\end{pgfscope}%
\begin{pgfscope}%
\pgfpathrectangle{\pgfqpoint{0.393053in}{0.375000in}}{\pgfqpoint{6.356833in}{5.175000in}}%
\pgfusepath{clip}%
\pgfsetbuttcap%
\pgfsetroundjoin%
\pgfsetlinewidth{1.003750pt}%
\definecolor{currentstroke}{rgb}{0.827451,0.827451,0.827451}%
\pgfsetstrokecolor{currentstroke}%
\pgfsetdash{}{0pt}%
\pgfpathmoveto{\pgfqpoint{4.213992in}{0.486413in}}%
\pgfpathcurveto{\pgfqpoint{4.225042in}{0.486413in}}{\pgfqpoint{4.235641in}{0.490803in}}{\pgfqpoint{4.243454in}{0.498616in}}%
\pgfpathcurveto{\pgfqpoint{4.251268in}{0.506430in}}{\pgfqpoint{4.255658in}{0.517029in}}{\pgfqpoint{4.255658in}{0.528079in}}%
\pgfpathcurveto{\pgfqpoint{4.255658in}{0.539129in}}{\pgfqpoint{4.251268in}{0.549728in}}{\pgfqpoint{4.243454in}{0.557542in}}%
\pgfpathcurveto{\pgfqpoint{4.235641in}{0.565356in}}{\pgfqpoint{4.225042in}{0.569746in}}{\pgfqpoint{4.213992in}{0.569746in}}%
\pgfpathcurveto{\pgfqpoint{4.202942in}{0.569746in}}{\pgfqpoint{4.192343in}{0.565356in}}{\pgfqpoint{4.184529in}{0.557542in}}%
\pgfpathcurveto{\pgfqpoint{4.176715in}{0.549728in}}{\pgfqpoint{4.172325in}{0.539129in}}{\pgfqpoint{4.172325in}{0.528079in}}%
\pgfpathcurveto{\pgfqpoint{4.172325in}{0.517029in}}{\pgfqpoint{4.176715in}{0.506430in}}{\pgfqpoint{4.184529in}{0.498616in}}%
\pgfpathcurveto{\pgfqpoint{4.192343in}{0.490803in}}{\pgfqpoint{4.202942in}{0.486413in}}{\pgfqpoint{4.213992in}{0.486413in}}%
\pgfpathlineto{\pgfqpoint{4.213992in}{0.486413in}}%
\pgfpathclose%
\pgfusepath{stroke}%
\end{pgfscope}%
\begin{pgfscope}%
\pgfpathrectangle{\pgfqpoint{0.393053in}{0.375000in}}{\pgfqpoint{6.356833in}{5.175000in}}%
\pgfusepath{clip}%
\pgfsetbuttcap%
\pgfsetroundjoin%
\pgfsetlinewidth{1.003750pt}%
\definecolor{currentstroke}{rgb}{0.827451,0.827451,0.827451}%
\pgfsetstrokecolor{currentstroke}%
\pgfsetdash{}{0pt}%
\pgfpathmoveto{\pgfqpoint{2.578528in}{1.236168in}}%
\pgfpathcurveto{\pgfqpoint{2.589578in}{1.236168in}}{\pgfqpoint{2.600177in}{1.240558in}}{\pgfqpoint{2.607991in}{1.248372in}}%
\pgfpathcurveto{\pgfqpoint{2.615805in}{1.256185in}}{\pgfqpoint{2.620195in}{1.266784in}}{\pgfqpoint{2.620195in}{1.277835in}}%
\pgfpathcurveto{\pgfqpoint{2.620195in}{1.288885in}}{\pgfqpoint{2.615805in}{1.299484in}}{\pgfqpoint{2.607991in}{1.307297in}}%
\pgfpathcurveto{\pgfqpoint{2.600177in}{1.315111in}}{\pgfqpoint{2.589578in}{1.319501in}}{\pgfqpoint{2.578528in}{1.319501in}}%
\pgfpathcurveto{\pgfqpoint{2.567478in}{1.319501in}}{\pgfqpoint{2.556879in}{1.315111in}}{\pgfqpoint{2.549065in}{1.307297in}}%
\pgfpathcurveto{\pgfqpoint{2.541252in}{1.299484in}}{\pgfqpoint{2.536861in}{1.288885in}}{\pgfqpoint{2.536861in}{1.277835in}}%
\pgfpathcurveto{\pgfqpoint{2.536861in}{1.266784in}}{\pgfqpoint{2.541252in}{1.256185in}}{\pgfqpoint{2.549065in}{1.248372in}}%
\pgfpathcurveto{\pgfqpoint{2.556879in}{1.240558in}}{\pgfqpoint{2.567478in}{1.236168in}}{\pgfqpoint{2.578528in}{1.236168in}}%
\pgfpathlineto{\pgfqpoint{2.578528in}{1.236168in}}%
\pgfpathclose%
\pgfusepath{stroke}%
\end{pgfscope}%
\begin{pgfscope}%
\pgfpathrectangle{\pgfqpoint{0.393053in}{0.375000in}}{\pgfqpoint{6.356833in}{5.175000in}}%
\pgfusepath{clip}%
\pgfsetbuttcap%
\pgfsetroundjoin%
\pgfsetlinewidth{1.003750pt}%
\definecolor{currentstroke}{rgb}{0.827451,0.827451,0.827451}%
\pgfsetstrokecolor{currentstroke}%
\pgfsetdash{}{0pt}%
\pgfpathmoveto{\pgfqpoint{2.665438in}{1.111968in}}%
\pgfpathcurveto{\pgfqpoint{2.676489in}{1.111968in}}{\pgfqpoint{2.687088in}{1.116358in}}{\pgfqpoint{2.694901in}{1.124171in}}%
\pgfpathcurveto{\pgfqpoint{2.702715in}{1.131985in}}{\pgfqpoint{2.707105in}{1.142584in}}{\pgfqpoint{2.707105in}{1.153634in}}%
\pgfpathcurveto{\pgfqpoint{2.707105in}{1.164684in}}{\pgfqpoint{2.702715in}{1.175283in}}{\pgfqpoint{2.694901in}{1.183097in}}%
\pgfpathcurveto{\pgfqpoint{2.687088in}{1.190911in}}{\pgfqpoint{2.676489in}{1.195301in}}{\pgfqpoint{2.665438in}{1.195301in}}%
\pgfpathcurveto{\pgfqpoint{2.654388in}{1.195301in}}{\pgfqpoint{2.643789in}{1.190911in}}{\pgfqpoint{2.635976in}{1.183097in}}%
\pgfpathcurveto{\pgfqpoint{2.628162in}{1.175283in}}{\pgfqpoint{2.623772in}{1.164684in}}{\pgfqpoint{2.623772in}{1.153634in}}%
\pgfpathcurveto{\pgfqpoint{2.623772in}{1.142584in}}{\pgfqpoint{2.628162in}{1.131985in}}{\pgfqpoint{2.635976in}{1.124171in}}%
\pgfpathcurveto{\pgfqpoint{2.643789in}{1.116358in}}{\pgfqpoint{2.654388in}{1.111968in}}{\pgfqpoint{2.665438in}{1.111968in}}%
\pgfpathlineto{\pgfqpoint{2.665438in}{1.111968in}}%
\pgfpathclose%
\pgfusepath{stroke}%
\end{pgfscope}%
\begin{pgfscope}%
\pgfpathrectangle{\pgfqpoint{0.393053in}{0.375000in}}{\pgfqpoint{6.356833in}{5.175000in}}%
\pgfusepath{clip}%
\pgfsetbuttcap%
\pgfsetroundjoin%
\pgfsetlinewidth{1.003750pt}%
\definecolor{currentstroke}{rgb}{0.827451,0.827451,0.827451}%
\pgfsetstrokecolor{currentstroke}%
\pgfsetdash{}{0pt}%
\pgfpathmoveto{\pgfqpoint{0.977927in}{3.008457in}}%
\pgfpathcurveto{\pgfqpoint{0.988977in}{3.008457in}}{\pgfqpoint{0.999576in}{3.012848in}}{\pgfqpoint{1.007390in}{3.020661in}}%
\pgfpathcurveto{\pgfqpoint{1.015204in}{3.028475in}}{\pgfqpoint{1.019594in}{3.039074in}}{\pgfqpoint{1.019594in}{3.050124in}}%
\pgfpathcurveto{\pgfqpoint{1.019594in}{3.061174in}}{\pgfqpoint{1.015204in}{3.071773in}}{\pgfqpoint{1.007390in}{3.079587in}}%
\pgfpathcurveto{\pgfqpoint{0.999576in}{3.087400in}}{\pgfqpoint{0.988977in}{3.091791in}}{\pgfqpoint{0.977927in}{3.091791in}}%
\pgfpathcurveto{\pgfqpoint{0.966877in}{3.091791in}}{\pgfqpoint{0.956278in}{3.087400in}}{\pgfqpoint{0.948465in}{3.079587in}}%
\pgfpathcurveto{\pgfqpoint{0.940651in}{3.071773in}}{\pgfqpoint{0.936261in}{3.061174in}}{\pgfqpoint{0.936261in}{3.050124in}}%
\pgfpathcurveto{\pgfqpoint{0.936261in}{3.039074in}}{\pgfqpoint{0.940651in}{3.028475in}}{\pgfqpoint{0.948465in}{3.020661in}}%
\pgfpathcurveto{\pgfqpoint{0.956278in}{3.012848in}}{\pgfqpoint{0.966877in}{3.008457in}}{\pgfqpoint{0.977927in}{3.008457in}}%
\pgfpathlineto{\pgfqpoint{0.977927in}{3.008457in}}%
\pgfpathclose%
\pgfusepath{stroke}%
\end{pgfscope}%
\begin{pgfscope}%
\pgfpathrectangle{\pgfqpoint{0.393053in}{0.375000in}}{\pgfqpoint{6.356833in}{5.175000in}}%
\pgfusepath{clip}%
\pgfsetbuttcap%
\pgfsetroundjoin%
\pgfsetlinewidth{1.003750pt}%
\definecolor{currentstroke}{rgb}{0.827451,0.827451,0.827451}%
\pgfsetstrokecolor{currentstroke}%
\pgfsetdash{}{0pt}%
\pgfpathmoveto{\pgfqpoint{5.137809in}{0.487765in}}%
\pgfpathcurveto{\pgfqpoint{5.148859in}{0.487765in}}{\pgfqpoint{5.159458in}{0.492155in}}{\pgfqpoint{5.167271in}{0.499969in}}%
\pgfpathcurveto{\pgfqpoint{5.175085in}{0.507783in}}{\pgfqpoint{5.179475in}{0.518382in}}{\pgfqpoint{5.179475in}{0.529432in}}%
\pgfpathcurveto{\pgfqpoint{5.179475in}{0.540482in}}{\pgfqpoint{5.175085in}{0.551081in}}{\pgfqpoint{5.167271in}{0.558895in}}%
\pgfpathcurveto{\pgfqpoint{5.159458in}{0.566708in}}{\pgfqpoint{5.148859in}{0.571098in}}{\pgfqpoint{5.137809in}{0.571098in}}%
\pgfpathcurveto{\pgfqpoint{5.126758in}{0.571098in}}{\pgfqpoint{5.116159in}{0.566708in}}{\pgfqpoint{5.108346in}{0.558895in}}%
\pgfpathcurveto{\pgfqpoint{5.100532in}{0.551081in}}{\pgfqpoint{5.096142in}{0.540482in}}{\pgfqpoint{5.096142in}{0.529432in}}%
\pgfpathcurveto{\pgfqpoint{5.096142in}{0.518382in}}{\pgfqpoint{5.100532in}{0.507783in}}{\pgfqpoint{5.108346in}{0.499969in}}%
\pgfpathcurveto{\pgfqpoint{5.116159in}{0.492155in}}{\pgfqpoint{5.126758in}{0.487765in}}{\pgfqpoint{5.137809in}{0.487765in}}%
\pgfpathlineto{\pgfqpoint{5.137809in}{0.487765in}}%
\pgfpathclose%
\pgfusepath{stroke}%
\end{pgfscope}%
\begin{pgfscope}%
\pgfpathrectangle{\pgfqpoint{0.393053in}{0.375000in}}{\pgfqpoint{6.356833in}{5.175000in}}%
\pgfusepath{clip}%
\pgfsetbuttcap%
\pgfsetroundjoin%
\pgfsetlinewidth{1.003750pt}%
\definecolor{currentstroke}{rgb}{0.827451,0.827451,0.827451}%
\pgfsetstrokecolor{currentstroke}%
\pgfsetdash{}{0pt}%
\pgfpathmoveto{\pgfqpoint{3.811752in}{0.656898in}}%
\pgfpathcurveto{\pgfqpoint{3.822802in}{0.656898in}}{\pgfqpoint{3.833401in}{0.661288in}}{\pgfqpoint{3.841215in}{0.669102in}}%
\pgfpathcurveto{\pgfqpoint{3.849028in}{0.676915in}}{\pgfqpoint{3.853419in}{0.687514in}}{\pgfqpoint{3.853419in}{0.698564in}}%
\pgfpathcurveto{\pgfqpoint{3.853419in}{0.709615in}}{\pgfqpoint{3.849028in}{0.720214in}}{\pgfqpoint{3.841215in}{0.728027in}}%
\pgfpathcurveto{\pgfqpoint{3.833401in}{0.735841in}}{\pgfqpoint{3.822802in}{0.740231in}}{\pgfqpoint{3.811752in}{0.740231in}}%
\pgfpathcurveto{\pgfqpoint{3.800702in}{0.740231in}}{\pgfqpoint{3.790103in}{0.735841in}}{\pgfqpoint{3.782289in}{0.728027in}}%
\pgfpathcurveto{\pgfqpoint{3.774476in}{0.720214in}}{\pgfqpoint{3.770085in}{0.709615in}}{\pgfqpoint{3.770085in}{0.698564in}}%
\pgfpathcurveto{\pgfqpoint{3.770085in}{0.687514in}}{\pgfqpoint{3.774476in}{0.676915in}}{\pgfqpoint{3.782289in}{0.669102in}}%
\pgfpathcurveto{\pgfqpoint{3.790103in}{0.661288in}}{\pgfqpoint{3.800702in}{0.656898in}}{\pgfqpoint{3.811752in}{0.656898in}}%
\pgfpathlineto{\pgfqpoint{3.811752in}{0.656898in}}%
\pgfpathclose%
\pgfusepath{stroke}%
\end{pgfscope}%
\begin{pgfscope}%
\pgfpathrectangle{\pgfqpoint{0.393053in}{0.375000in}}{\pgfqpoint{6.356833in}{5.175000in}}%
\pgfusepath{clip}%
\pgfsetbuttcap%
\pgfsetroundjoin%
\pgfsetlinewidth{1.003750pt}%
\definecolor{currentstroke}{rgb}{0.827451,0.827451,0.827451}%
\pgfsetstrokecolor{currentstroke}%
\pgfsetdash{}{0pt}%
\pgfpathmoveto{\pgfqpoint{2.036490in}{1.340063in}}%
\pgfpathcurveto{\pgfqpoint{2.047540in}{1.340063in}}{\pgfqpoint{2.058139in}{1.344454in}}{\pgfqpoint{2.065953in}{1.352267in}}%
\pgfpathcurveto{\pgfqpoint{2.073767in}{1.360081in}}{\pgfqpoint{2.078157in}{1.370680in}}{\pgfqpoint{2.078157in}{1.381730in}}%
\pgfpathcurveto{\pgfqpoint{2.078157in}{1.392780in}}{\pgfqpoint{2.073767in}{1.403379in}}{\pgfqpoint{2.065953in}{1.411193in}}%
\pgfpathcurveto{\pgfqpoint{2.058139in}{1.419007in}}{\pgfqpoint{2.047540in}{1.423397in}}{\pgfqpoint{2.036490in}{1.423397in}}%
\pgfpathcurveto{\pgfqpoint{2.025440in}{1.423397in}}{\pgfqpoint{2.014841in}{1.419007in}}{\pgfqpoint{2.007027in}{1.411193in}}%
\pgfpathcurveto{\pgfqpoint{1.999214in}{1.403379in}}{\pgfqpoint{1.994824in}{1.392780in}}{\pgfqpoint{1.994824in}{1.381730in}}%
\pgfpathcurveto{\pgfqpoint{1.994824in}{1.370680in}}{\pgfqpoint{1.999214in}{1.360081in}}{\pgfqpoint{2.007027in}{1.352267in}}%
\pgfpathcurveto{\pgfqpoint{2.014841in}{1.344454in}}{\pgfqpoint{2.025440in}{1.340063in}}{\pgfqpoint{2.036490in}{1.340063in}}%
\pgfpathlineto{\pgfqpoint{2.036490in}{1.340063in}}%
\pgfpathclose%
\pgfusepath{stroke}%
\end{pgfscope}%
\begin{pgfscope}%
\pgfpathrectangle{\pgfqpoint{0.393053in}{0.375000in}}{\pgfqpoint{6.356833in}{5.175000in}}%
\pgfusepath{clip}%
\pgfsetbuttcap%
\pgfsetroundjoin%
\pgfsetlinewidth{1.003750pt}%
\definecolor{currentstroke}{rgb}{0.827451,0.827451,0.827451}%
\pgfsetstrokecolor{currentstroke}%
\pgfsetdash{}{0pt}%
\pgfpathmoveto{\pgfqpoint{1.388153in}{1.896293in}}%
\pgfpathcurveto{\pgfqpoint{1.399204in}{1.896293in}}{\pgfqpoint{1.409803in}{1.900683in}}{\pgfqpoint{1.417616in}{1.908497in}}%
\pgfpathcurveto{\pgfqpoint{1.425430in}{1.916310in}}{\pgfqpoint{1.429820in}{1.926909in}}{\pgfqpoint{1.429820in}{1.937959in}}%
\pgfpathcurveto{\pgfqpoint{1.429820in}{1.949010in}}{\pgfqpoint{1.425430in}{1.959609in}}{\pgfqpoint{1.417616in}{1.967422in}}%
\pgfpathcurveto{\pgfqpoint{1.409803in}{1.975236in}}{\pgfqpoint{1.399204in}{1.979626in}}{\pgfqpoint{1.388153in}{1.979626in}}%
\pgfpathcurveto{\pgfqpoint{1.377103in}{1.979626in}}{\pgfqpoint{1.366504in}{1.975236in}}{\pgfqpoint{1.358691in}{1.967422in}}%
\pgfpathcurveto{\pgfqpoint{1.350877in}{1.959609in}}{\pgfqpoint{1.346487in}{1.949010in}}{\pgfqpoint{1.346487in}{1.937959in}}%
\pgfpathcurveto{\pgfqpoint{1.346487in}{1.926909in}}{\pgfqpoint{1.350877in}{1.916310in}}{\pgfqpoint{1.358691in}{1.908497in}}%
\pgfpathcurveto{\pgfqpoint{1.366504in}{1.900683in}}{\pgfqpoint{1.377103in}{1.896293in}}{\pgfqpoint{1.388153in}{1.896293in}}%
\pgfpathlineto{\pgfqpoint{1.388153in}{1.896293in}}%
\pgfpathclose%
\pgfusepath{stroke}%
\end{pgfscope}%
\begin{pgfscope}%
\pgfpathrectangle{\pgfqpoint{0.393053in}{0.375000in}}{\pgfqpoint{6.356833in}{5.175000in}}%
\pgfusepath{clip}%
\pgfsetbuttcap%
\pgfsetroundjoin%
\pgfsetlinewidth{1.003750pt}%
\definecolor{currentstroke}{rgb}{0.827451,0.827451,0.827451}%
\pgfsetstrokecolor{currentstroke}%
\pgfsetdash{}{0pt}%
\pgfpathmoveto{\pgfqpoint{1.463696in}{1.814748in}}%
\pgfpathcurveto{\pgfqpoint{1.474746in}{1.814748in}}{\pgfqpoint{1.485345in}{1.819138in}}{\pgfqpoint{1.493159in}{1.826952in}}%
\pgfpathcurveto{\pgfqpoint{1.500972in}{1.834766in}}{\pgfqpoint{1.505362in}{1.845365in}}{\pgfqpoint{1.505362in}{1.856415in}}%
\pgfpathcurveto{\pgfqpoint{1.505362in}{1.867465in}}{\pgfqpoint{1.500972in}{1.878064in}}{\pgfqpoint{1.493159in}{1.885878in}}%
\pgfpathcurveto{\pgfqpoint{1.485345in}{1.893691in}}{\pgfqpoint{1.474746in}{1.898081in}}{\pgfqpoint{1.463696in}{1.898081in}}%
\pgfpathcurveto{\pgfqpoint{1.452646in}{1.898081in}}{\pgfqpoint{1.442047in}{1.893691in}}{\pgfqpoint{1.434233in}{1.885878in}}%
\pgfpathcurveto{\pgfqpoint{1.426419in}{1.878064in}}{\pgfqpoint{1.422029in}{1.867465in}}{\pgfqpoint{1.422029in}{1.856415in}}%
\pgfpathcurveto{\pgfqpoint{1.422029in}{1.845365in}}{\pgfqpoint{1.426419in}{1.834766in}}{\pgfqpoint{1.434233in}{1.826952in}}%
\pgfpathcurveto{\pgfqpoint{1.442047in}{1.819138in}}{\pgfqpoint{1.452646in}{1.814748in}}{\pgfqpoint{1.463696in}{1.814748in}}%
\pgfpathlineto{\pgfqpoint{1.463696in}{1.814748in}}%
\pgfpathclose%
\pgfusepath{stroke}%
\end{pgfscope}%
\begin{pgfscope}%
\pgfpathrectangle{\pgfqpoint{0.393053in}{0.375000in}}{\pgfqpoint{6.356833in}{5.175000in}}%
\pgfusepath{clip}%
\pgfsetbuttcap%
\pgfsetroundjoin%
\pgfsetlinewidth{1.003750pt}%
\definecolor{currentstroke}{rgb}{0.827451,0.827451,0.827451}%
\pgfsetstrokecolor{currentstroke}%
\pgfsetdash{}{0pt}%
\pgfpathmoveto{\pgfqpoint{1.909651in}{1.444209in}}%
\pgfpathcurveto{\pgfqpoint{1.920701in}{1.444209in}}{\pgfqpoint{1.931300in}{1.448599in}}{\pgfqpoint{1.939114in}{1.456413in}}%
\pgfpathcurveto{\pgfqpoint{1.946927in}{1.464226in}}{\pgfqpoint{1.951318in}{1.474825in}}{\pgfqpoint{1.951318in}{1.485875in}}%
\pgfpathcurveto{\pgfqpoint{1.951318in}{1.496926in}}{\pgfqpoint{1.946927in}{1.507525in}}{\pgfqpoint{1.939114in}{1.515338in}}%
\pgfpathcurveto{\pgfqpoint{1.931300in}{1.523152in}}{\pgfqpoint{1.920701in}{1.527542in}}{\pgfqpoint{1.909651in}{1.527542in}}%
\pgfpathcurveto{\pgfqpoint{1.898601in}{1.527542in}}{\pgfqpoint{1.888002in}{1.523152in}}{\pgfqpoint{1.880188in}{1.515338in}}%
\pgfpathcurveto{\pgfqpoint{1.872375in}{1.507525in}}{\pgfqpoint{1.867984in}{1.496926in}}{\pgfqpoint{1.867984in}{1.485875in}}%
\pgfpathcurveto{\pgfqpoint{1.867984in}{1.474825in}}{\pgfqpoint{1.872375in}{1.464226in}}{\pgfqpoint{1.880188in}{1.456413in}}%
\pgfpathcurveto{\pgfqpoint{1.888002in}{1.448599in}}{\pgfqpoint{1.898601in}{1.444209in}}{\pgfqpoint{1.909651in}{1.444209in}}%
\pgfpathlineto{\pgfqpoint{1.909651in}{1.444209in}}%
\pgfpathclose%
\pgfusepath{stroke}%
\end{pgfscope}%
\begin{pgfscope}%
\pgfpathrectangle{\pgfqpoint{0.393053in}{0.375000in}}{\pgfqpoint{6.356833in}{5.175000in}}%
\pgfusepath{clip}%
\pgfsetbuttcap%
\pgfsetroundjoin%
\pgfsetlinewidth{1.003750pt}%
\definecolor{currentstroke}{rgb}{0.827451,0.827451,0.827451}%
\pgfsetstrokecolor{currentstroke}%
\pgfsetdash{}{0pt}%
\pgfpathmoveto{\pgfqpoint{1.253491in}{2.045656in}}%
\pgfpathcurveto{\pgfqpoint{1.264541in}{2.045656in}}{\pgfqpoint{1.275140in}{2.050046in}}{\pgfqpoint{1.282953in}{2.057860in}}%
\pgfpathcurveto{\pgfqpoint{1.290767in}{2.065674in}}{\pgfqpoint{1.295157in}{2.076273in}}{\pgfqpoint{1.295157in}{2.087323in}}%
\pgfpathcurveto{\pgfqpoint{1.295157in}{2.098373in}}{\pgfqpoint{1.290767in}{2.108972in}}{\pgfqpoint{1.282953in}{2.116786in}}%
\pgfpathcurveto{\pgfqpoint{1.275140in}{2.124599in}}{\pgfqpoint{1.264541in}{2.128990in}}{\pgfqpoint{1.253491in}{2.128990in}}%
\pgfpathcurveto{\pgfqpoint{1.242440in}{2.128990in}}{\pgfqpoint{1.231841in}{2.124599in}}{\pgfqpoint{1.224028in}{2.116786in}}%
\pgfpathcurveto{\pgfqpoint{1.216214in}{2.108972in}}{\pgfqpoint{1.211824in}{2.098373in}}{\pgfqpoint{1.211824in}{2.087323in}}%
\pgfpathcurveto{\pgfqpoint{1.211824in}{2.076273in}}{\pgfqpoint{1.216214in}{2.065674in}}{\pgfqpoint{1.224028in}{2.057860in}}%
\pgfpathcurveto{\pgfqpoint{1.231841in}{2.050046in}}{\pgfqpoint{1.242440in}{2.045656in}}{\pgfqpoint{1.253491in}{2.045656in}}%
\pgfpathlineto{\pgfqpoint{1.253491in}{2.045656in}}%
\pgfpathclose%
\pgfusepath{stroke}%
\end{pgfscope}%
\begin{pgfscope}%
\pgfpathrectangle{\pgfqpoint{0.393053in}{0.375000in}}{\pgfqpoint{6.356833in}{5.175000in}}%
\pgfusepath{clip}%
\pgfsetbuttcap%
\pgfsetroundjoin%
\pgfsetlinewidth{1.003750pt}%
\definecolor{currentstroke}{rgb}{0.827451,0.827451,0.827451}%
\pgfsetstrokecolor{currentstroke}%
\pgfsetdash{}{0pt}%
\pgfpathmoveto{\pgfqpoint{2.352374in}{1.159706in}}%
\pgfpathcurveto{\pgfqpoint{2.363424in}{1.159706in}}{\pgfqpoint{2.374023in}{1.164096in}}{\pgfqpoint{2.381836in}{1.171910in}}%
\pgfpathcurveto{\pgfqpoint{2.389650in}{1.179724in}}{\pgfqpoint{2.394040in}{1.190323in}}{\pgfqpoint{2.394040in}{1.201373in}}%
\pgfpathcurveto{\pgfqpoint{2.394040in}{1.212423in}}{\pgfqpoint{2.389650in}{1.223022in}}{\pgfqpoint{2.381836in}{1.230835in}}%
\pgfpathcurveto{\pgfqpoint{2.374023in}{1.238649in}}{\pgfqpoint{2.363424in}{1.243039in}}{\pgfqpoint{2.352374in}{1.243039in}}%
\pgfpathcurveto{\pgfqpoint{2.341324in}{1.243039in}}{\pgfqpoint{2.330725in}{1.238649in}}{\pgfqpoint{2.322911in}{1.230835in}}%
\pgfpathcurveto{\pgfqpoint{2.315097in}{1.223022in}}{\pgfqpoint{2.310707in}{1.212423in}}{\pgfqpoint{2.310707in}{1.201373in}}%
\pgfpathcurveto{\pgfqpoint{2.310707in}{1.190323in}}{\pgfqpoint{2.315097in}{1.179724in}}{\pgfqpoint{2.322911in}{1.171910in}}%
\pgfpathcurveto{\pgfqpoint{2.330725in}{1.164096in}}{\pgfqpoint{2.341324in}{1.159706in}}{\pgfqpoint{2.352374in}{1.159706in}}%
\pgfpathlineto{\pgfqpoint{2.352374in}{1.159706in}}%
\pgfpathclose%
\pgfusepath{stroke}%
\end{pgfscope}%
\begin{pgfscope}%
\pgfpathrectangle{\pgfqpoint{0.393053in}{0.375000in}}{\pgfqpoint{6.356833in}{5.175000in}}%
\pgfusepath{clip}%
\pgfsetbuttcap%
\pgfsetroundjoin%
\pgfsetlinewidth{1.003750pt}%
\definecolor{currentstroke}{rgb}{0.827451,0.827451,0.827451}%
\pgfsetstrokecolor{currentstroke}%
\pgfsetdash{}{0pt}%
\pgfpathmoveto{\pgfqpoint{0.455840in}{3.833482in}}%
\pgfpathcurveto{\pgfqpoint{0.466890in}{3.833482in}}{\pgfqpoint{0.477489in}{3.837873in}}{\pgfqpoint{0.485303in}{3.845686in}}%
\pgfpathcurveto{\pgfqpoint{0.493117in}{3.853500in}}{\pgfqpoint{0.497507in}{3.864099in}}{\pgfqpoint{0.497507in}{3.875149in}}%
\pgfpathcurveto{\pgfqpoint{0.497507in}{3.886199in}}{\pgfqpoint{0.493117in}{3.896798in}}{\pgfqpoint{0.485303in}{3.904612in}}%
\pgfpathcurveto{\pgfqpoint{0.477489in}{3.912425in}}{\pgfqpoint{0.466890in}{3.916816in}}{\pgfqpoint{0.455840in}{3.916816in}}%
\pgfpathcurveto{\pgfqpoint{0.444790in}{3.916816in}}{\pgfqpoint{0.434191in}{3.912425in}}{\pgfqpoint{0.426377in}{3.904612in}}%
\pgfpathcurveto{\pgfqpoint{0.418564in}{3.896798in}}{\pgfqpoint{0.414173in}{3.886199in}}{\pgfqpoint{0.414173in}{3.875149in}}%
\pgfpathcurveto{\pgfqpoint{0.414173in}{3.864099in}}{\pgfqpoint{0.418564in}{3.853500in}}{\pgfqpoint{0.426377in}{3.845686in}}%
\pgfpathcurveto{\pgfqpoint{0.434191in}{3.837873in}}{\pgfqpoint{0.444790in}{3.833482in}}{\pgfqpoint{0.455840in}{3.833482in}}%
\pgfpathlineto{\pgfqpoint{0.455840in}{3.833482in}}%
\pgfpathclose%
\pgfusepath{stroke}%
\end{pgfscope}%
\begin{pgfscope}%
\pgfpathrectangle{\pgfqpoint{0.393053in}{0.375000in}}{\pgfqpoint{6.356833in}{5.175000in}}%
\pgfusepath{clip}%
\pgfsetbuttcap%
\pgfsetroundjoin%
\pgfsetlinewidth{1.003750pt}%
\definecolor{currentstroke}{rgb}{0.827451,0.827451,0.827451}%
\pgfsetstrokecolor{currentstroke}%
\pgfsetdash{}{0pt}%
\pgfpathmoveto{\pgfqpoint{1.186674in}{2.128077in}}%
\pgfpathcurveto{\pgfqpoint{1.197724in}{2.128077in}}{\pgfqpoint{1.208323in}{2.132467in}}{\pgfqpoint{1.216137in}{2.140281in}}%
\pgfpathcurveto{\pgfqpoint{1.223950in}{2.148095in}}{\pgfqpoint{1.228341in}{2.158694in}}{\pgfqpoint{1.228341in}{2.169744in}}%
\pgfpathcurveto{\pgfqpoint{1.228341in}{2.180794in}}{\pgfqpoint{1.223950in}{2.191393in}}{\pgfqpoint{1.216137in}{2.199207in}}%
\pgfpathcurveto{\pgfqpoint{1.208323in}{2.207020in}}{\pgfqpoint{1.197724in}{2.211411in}}{\pgfqpoint{1.186674in}{2.211411in}}%
\pgfpathcurveto{\pgfqpoint{1.175624in}{2.211411in}}{\pgfqpoint{1.165025in}{2.207020in}}{\pgfqpoint{1.157211in}{2.199207in}}%
\pgfpathcurveto{\pgfqpoint{1.149397in}{2.191393in}}{\pgfqpoint{1.145007in}{2.180794in}}{\pgfqpoint{1.145007in}{2.169744in}}%
\pgfpathcurveto{\pgfqpoint{1.145007in}{2.158694in}}{\pgfqpoint{1.149397in}{2.148095in}}{\pgfqpoint{1.157211in}{2.140281in}}%
\pgfpathcurveto{\pgfqpoint{1.165025in}{2.132467in}}{\pgfqpoint{1.175624in}{2.128077in}}{\pgfqpoint{1.186674in}{2.128077in}}%
\pgfpathlineto{\pgfqpoint{1.186674in}{2.128077in}}%
\pgfpathclose%
\pgfusepath{stroke}%
\end{pgfscope}%
\begin{pgfscope}%
\pgfpathrectangle{\pgfqpoint{0.393053in}{0.375000in}}{\pgfqpoint{6.356833in}{5.175000in}}%
\pgfusepath{clip}%
\pgfsetbuttcap%
\pgfsetroundjoin%
\pgfsetlinewidth{1.003750pt}%
\definecolor{currentstroke}{rgb}{0.827451,0.827451,0.827451}%
\pgfsetstrokecolor{currentstroke}%
\pgfsetdash{}{0pt}%
\pgfpathmoveto{\pgfqpoint{3.097658in}{0.775315in}}%
\pgfpathcurveto{\pgfqpoint{3.108708in}{0.775315in}}{\pgfqpoint{3.119307in}{0.779705in}}{\pgfqpoint{3.127121in}{0.787519in}}%
\pgfpathcurveto{\pgfqpoint{3.134935in}{0.795332in}}{\pgfqpoint{3.139325in}{0.805931in}}{\pgfqpoint{3.139325in}{0.816981in}}%
\pgfpathcurveto{\pgfqpoint{3.139325in}{0.828032in}}{\pgfqpoint{3.134935in}{0.838631in}}{\pgfqpoint{3.127121in}{0.846444in}}%
\pgfpathcurveto{\pgfqpoint{3.119307in}{0.854258in}}{\pgfqpoint{3.108708in}{0.858648in}}{\pgfqpoint{3.097658in}{0.858648in}}%
\pgfpathcurveto{\pgfqpoint{3.086608in}{0.858648in}}{\pgfqpoint{3.076009in}{0.854258in}}{\pgfqpoint{3.068195in}{0.846444in}}%
\pgfpathcurveto{\pgfqpoint{3.060382in}{0.838631in}}{\pgfqpoint{3.055992in}{0.828032in}}{\pgfqpoint{3.055992in}{0.816981in}}%
\pgfpathcurveto{\pgfqpoint{3.055992in}{0.805931in}}{\pgfqpoint{3.060382in}{0.795332in}}{\pgfqpoint{3.068195in}{0.787519in}}%
\pgfpathcurveto{\pgfqpoint{3.076009in}{0.779705in}}{\pgfqpoint{3.086608in}{0.775315in}}{\pgfqpoint{3.097658in}{0.775315in}}%
\pgfpathlineto{\pgfqpoint{3.097658in}{0.775315in}}%
\pgfpathclose%
\pgfusepath{stroke}%
\end{pgfscope}%
\begin{pgfscope}%
\pgfpathrectangle{\pgfqpoint{0.393053in}{0.375000in}}{\pgfqpoint{6.356833in}{5.175000in}}%
\pgfusepath{clip}%
\pgfsetbuttcap%
\pgfsetroundjoin%
\pgfsetlinewidth{1.003750pt}%
\definecolor{currentstroke}{rgb}{0.827451,0.827451,0.827451}%
\pgfsetstrokecolor{currentstroke}%
\pgfsetdash{}{0pt}%
\pgfpathmoveto{\pgfqpoint{4.283632in}{0.443541in}}%
\pgfpathcurveto{\pgfqpoint{4.294682in}{0.443541in}}{\pgfqpoint{4.305281in}{0.447931in}}{\pgfqpoint{4.313095in}{0.455745in}}%
\pgfpathcurveto{\pgfqpoint{4.320908in}{0.463559in}}{\pgfqpoint{4.325299in}{0.474158in}}{\pgfqpoint{4.325299in}{0.485208in}}%
\pgfpathcurveto{\pgfqpoint{4.325299in}{0.496258in}}{\pgfqpoint{4.320908in}{0.506857in}}{\pgfqpoint{4.313095in}{0.514670in}}%
\pgfpathcurveto{\pgfqpoint{4.305281in}{0.522484in}}{\pgfqpoint{4.294682in}{0.526874in}}{\pgfqpoint{4.283632in}{0.526874in}}%
\pgfpathcurveto{\pgfqpoint{4.272582in}{0.526874in}}{\pgfqpoint{4.261983in}{0.522484in}}{\pgfqpoint{4.254169in}{0.514670in}}%
\pgfpathcurveto{\pgfqpoint{4.246356in}{0.506857in}}{\pgfqpoint{4.241965in}{0.496258in}}{\pgfqpoint{4.241965in}{0.485208in}}%
\pgfpathcurveto{\pgfqpoint{4.241965in}{0.474158in}}{\pgfqpoint{4.246356in}{0.463559in}}{\pgfqpoint{4.254169in}{0.455745in}}%
\pgfpathcurveto{\pgfqpoint{4.261983in}{0.447931in}}{\pgfqpoint{4.272582in}{0.443541in}}{\pgfqpoint{4.283632in}{0.443541in}}%
\pgfpathlineto{\pgfqpoint{4.283632in}{0.443541in}}%
\pgfpathclose%
\pgfusepath{stroke}%
\end{pgfscope}%
\begin{pgfscope}%
\pgfpathrectangle{\pgfqpoint{0.393053in}{0.375000in}}{\pgfqpoint{6.356833in}{5.175000in}}%
\pgfusepath{clip}%
\pgfsetbuttcap%
\pgfsetroundjoin%
\pgfsetlinewidth{1.003750pt}%
\definecolor{currentstroke}{rgb}{0.827451,0.827451,0.827451}%
\pgfsetstrokecolor{currentstroke}%
\pgfsetdash{}{0pt}%
\pgfpathmoveto{\pgfqpoint{2.582769in}{1.048231in}}%
\pgfpathcurveto{\pgfqpoint{2.593819in}{1.048231in}}{\pgfqpoint{2.604418in}{1.052621in}}{\pgfqpoint{2.612231in}{1.060435in}}%
\pgfpathcurveto{\pgfqpoint{2.620045in}{1.068248in}}{\pgfqpoint{2.624435in}{1.078848in}}{\pgfqpoint{2.624435in}{1.089898in}}%
\pgfpathcurveto{\pgfqpoint{2.624435in}{1.100948in}}{\pgfqpoint{2.620045in}{1.111547in}}{\pgfqpoint{2.612231in}{1.119360in}}%
\pgfpathcurveto{\pgfqpoint{2.604418in}{1.127174in}}{\pgfqpoint{2.593819in}{1.131564in}}{\pgfqpoint{2.582769in}{1.131564in}}%
\pgfpathcurveto{\pgfqpoint{2.571718in}{1.131564in}}{\pgfqpoint{2.561119in}{1.127174in}}{\pgfqpoint{2.553306in}{1.119360in}}%
\pgfpathcurveto{\pgfqpoint{2.545492in}{1.111547in}}{\pgfqpoint{2.541102in}{1.100948in}}{\pgfqpoint{2.541102in}{1.089898in}}%
\pgfpathcurveto{\pgfqpoint{2.541102in}{1.078848in}}{\pgfqpoint{2.545492in}{1.068248in}}{\pgfqpoint{2.553306in}{1.060435in}}%
\pgfpathcurveto{\pgfqpoint{2.561119in}{1.052621in}}{\pgfqpoint{2.571718in}{1.048231in}}{\pgfqpoint{2.582769in}{1.048231in}}%
\pgfpathlineto{\pgfqpoint{2.582769in}{1.048231in}}%
\pgfpathclose%
\pgfusepath{stroke}%
\end{pgfscope}%
\begin{pgfscope}%
\pgfpathrectangle{\pgfqpoint{0.393053in}{0.375000in}}{\pgfqpoint{6.356833in}{5.175000in}}%
\pgfusepath{clip}%
\pgfsetbuttcap%
\pgfsetroundjoin%
\pgfsetlinewidth{1.003750pt}%
\definecolor{currentstroke}{rgb}{0.827451,0.827451,0.827451}%
\pgfsetstrokecolor{currentstroke}%
\pgfsetdash{}{0pt}%
\pgfpathmoveto{\pgfqpoint{0.474470in}{3.727925in}}%
\pgfpathcurveto{\pgfqpoint{0.485520in}{3.727925in}}{\pgfqpoint{0.496119in}{3.732315in}}{\pgfqpoint{0.503933in}{3.740129in}}%
\pgfpathcurveto{\pgfqpoint{0.511746in}{3.747943in}}{\pgfqpoint{0.516137in}{3.758542in}}{\pgfqpoint{0.516137in}{3.769592in}}%
\pgfpathcurveto{\pgfqpoint{0.516137in}{3.780642in}}{\pgfqpoint{0.511746in}{3.791241in}}{\pgfqpoint{0.503933in}{3.799055in}}%
\pgfpathcurveto{\pgfqpoint{0.496119in}{3.806868in}}{\pgfqpoint{0.485520in}{3.811259in}}{\pgfqpoint{0.474470in}{3.811259in}}%
\pgfpathcurveto{\pgfqpoint{0.463420in}{3.811259in}}{\pgfqpoint{0.452821in}{3.806868in}}{\pgfqpoint{0.445007in}{3.799055in}}%
\pgfpathcurveto{\pgfqpoint{0.437194in}{3.791241in}}{\pgfqpoint{0.432803in}{3.780642in}}{\pgfqpoint{0.432803in}{3.769592in}}%
\pgfpathcurveto{\pgfqpoint{0.432803in}{3.758542in}}{\pgfqpoint{0.437194in}{3.747943in}}{\pgfqpoint{0.445007in}{3.740129in}}%
\pgfpathcurveto{\pgfqpoint{0.452821in}{3.732315in}}{\pgfqpoint{0.463420in}{3.727925in}}{\pgfqpoint{0.474470in}{3.727925in}}%
\pgfpathlineto{\pgfqpoint{0.474470in}{3.727925in}}%
\pgfpathclose%
\pgfusepath{stroke}%
\end{pgfscope}%
\begin{pgfscope}%
\pgfpathrectangle{\pgfqpoint{0.393053in}{0.375000in}}{\pgfqpoint{6.356833in}{5.175000in}}%
\pgfusepath{clip}%
\pgfsetbuttcap%
\pgfsetroundjoin%
\pgfsetlinewidth{1.003750pt}%
\definecolor{currentstroke}{rgb}{0.827451,0.827451,0.827451}%
\pgfsetstrokecolor{currentstroke}%
\pgfsetdash{}{0pt}%
\pgfpathmoveto{\pgfqpoint{1.493790in}{1.813114in}}%
\pgfpathcurveto{\pgfqpoint{1.504840in}{1.813114in}}{\pgfqpoint{1.515439in}{1.817505in}}{\pgfqpoint{1.523252in}{1.825318in}}%
\pgfpathcurveto{\pgfqpoint{1.531066in}{1.833132in}}{\pgfqpoint{1.535456in}{1.843731in}}{\pgfqpoint{1.535456in}{1.854781in}}%
\pgfpathcurveto{\pgfqpoint{1.535456in}{1.865831in}}{\pgfqpoint{1.531066in}{1.876430in}}{\pgfqpoint{1.523252in}{1.884244in}}%
\pgfpathcurveto{\pgfqpoint{1.515439in}{1.892058in}}{\pgfqpoint{1.504840in}{1.896448in}}{\pgfqpoint{1.493790in}{1.896448in}}%
\pgfpathcurveto{\pgfqpoint{1.482740in}{1.896448in}}{\pgfqpoint{1.472140in}{1.892058in}}{\pgfqpoint{1.464327in}{1.884244in}}%
\pgfpathcurveto{\pgfqpoint{1.456513in}{1.876430in}}{\pgfqpoint{1.452123in}{1.865831in}}{\pgfqpoint{1.452123in}{1.854781in}}%
\pgfpathcurveto{\pgfqpoint{1.452123in}{1.843731in}}{\pgfqpoint{1.456513in}{1.833132in}}{\pgfqpoint{1.464327in}{1.825318in}}%
\pgfpathcurveto{\pgfqpoint{1.472140in}{1.817505in}}{\pgfqpoint{1.482740in}{1.813114in}}{\pgfqpoint{1.493790in}{1.813114in}}%
\pgfpathlineto{\pgfqpoint{1.493790in}{1.813114in}}%
\pgfpathclose%
\pgfusepath{stroke}%
\end{pgfscope}%
\begin{pgfscope}%
\pgfpathrectangle{\pgfqpoint{0.393053in}{0.375000in}}{\pgfqpoint{6.356833in}{5.175000in}}%
\pgfusepath{clip}%
\pgfsetbuttcap%
\pgfsetroundjoin%
\pgfsetlinewidth{1.003750pt}%
\definecolor{currentstroke}{rgb}{0.827451,0.827451,0.827451}%
\pgfsetstrokecolor{currentstroke}%
\pgfsetdash{}{0pt}%
\pgfpathmoveto{\pgfqpoint{1.687372in}{1.621104in}}%
\pgfpathcurveto{\pgfqpoint{1.698422in}{1.621104in}}{\pgfqpoint{1.709022in}{1.625494in}}{\pgfqpoint{1.716835in}{1.633308in}}%
\pgfpathcurveto{\pgfqpoint{1.724649in}{1.641122in}}{\pgfqpoint{1.729039in}{1.651721in}}{\pgfqpoint{1.729039in}{1.662771in}}%
\pgfpathcurveto{\pgfqpoint{1.729039in}{1.673821in}}{\pgfqpoint{1.724649in}{1.684420in}}{\pgfqpoint{1.716835in}{1.692234in}}%
\pgfpathcurveto{\pgfqpoint{1.709022in}{1.700047in}}{\pgfqpoint{1.698422in}{1.704437in}}{\pgfqpoint{1.687372in}{1.704437in}}%
\pgfpathcurveto{\pgfqpoint{1.676322in}{1.704437in}}{\pgfqpoint{1.665723in}{1.700047in}}{\pgfqpoint{1.657910in}{1.692234in}}%
\pgfpathcurveto{\pgfqpoint{1.650096in}{1.684420in}}{\pgfqpoint{1.645706in}{1.673821in}}{\pgfqpoint{1.645706in}{1.662771in}}%
\pgfpathcurveto{\pgfqpoint{1.645706in}{1.651721in}}{\pgfqpoint{1.650096in}{1.641122in}}{\pgfqpoint{1.657910in}{1.633308in}}%
\pgfpathcurveto{\pgfqpoint{1.665723in}{1.625494in}}{\pgfqpoint{1.676322in}{1.621104in}}{\pgfqpoint{1.687372in}{1.621104in}}%
\pgfpathlineto{\pgfqpoint{1.687372in}{1.621104in}}%
\pgfpathclose%
\pgfusepath{stroke}%
\end{pgfscope}%
\begin{pgfscope}%
\pgfpathrectangle{\pgfqpoint{0.393053in}{0.375000in}}{\pgfqpoint{6.356833in}{5.175000in}}%
\pgfusepath{clip}%
\pgfsetbuttcap%
\pgfsetroundjoin%
\pgfsetlinewidth{1.003750pt}%
\definecolor{currentstroke}{rgb}{0.827451,0.827451,0.827451}%
\pgfsetstrokecolor{currentstroke}%
\pgfsetdash{}{0pt}%
\pgfpathmoveto{\pgfqpoint{0.947650in}{2.625936in}}%
\pgfpathcurveto{\pgfqpoint{0.958700in}{2.625936in}}{\pgfqpoint{0.969299in}{2.630326in}}{\pgfqpoint{0.977112in}{2.638140in}}%
\pgfpathcurveto{\pgfqpoint{0.984926in}{2.645953in}}{\pgfqpoint{0.989316in}{2.656552in}}{\pgfqpoint{0.989316in}{2.667602in}}%
\pgfpathcurveto{\pgfqpoint{0.989316in}{2.678652in}}{\pgfqpoint{0.984926in}{2.689252in}}{\pgfqpoint{0.977112in}{2.697065in}}%
\pgfpathcurveto{\pgfqpoint{0.969299in}{2.704879in}}{\pgfqpoint{0.958700in}{2.709269in}}{\pgfqpoint{0.947650in}{2.709269in}}%
\pgfpathcurveto{\pgfqpoint{0.936600in}{2.709269in}}{\pgfqpoint{0.926000in}{2.704879in}}{\pgfqpoint{0.918187in}{2.697065in}}%
\pgfpathcurveto{\pgfqpoint{0.910373in}{2.689252in}}{\pgfqpoint{0.905983in}{2.678652in}}{\pgfqpoint{0.905983in}{2.667602in}}%
\pgfpathcurveto{\pgfqpoint{0.905983in}{2.656552in}}{\pgfqpoint{0.910373in}{2.645953in}}{\pgfqpoint{0.918187in}{2.638140in}}%
\pgfpathcurveto{\pgfqpoint{0.926000in}{2.630326in}}{\pgfqpoint{0.936600in}{2.625936in}}{\pgfqpoint{0.947650in}{2.625936in}}%
\pgfpathlineto{\pgfqpoint{0.947650in}{2.625936in}}%
\pgfpathclose%
\pgfusepath{stroke}%
\end{pgfscope}%
\begin{pgfscope}%
\pgfpathrectangle{\pgfqpoint{0.393053in}{0.375000in}}{\pgfqpoint{6.356833in}{5.175000in}}%
\pgfusepath{clip}%
\pgfsetbuttcap%
\pgfsetroundjoin%
\pgfsetlinewidth{1.003750pt}%
\definecolor{currentstroke}{rgb}{0.827451,0.827451,0.827451}%
\pgfsetstrokecolor{currentstroke}%
\pgfsetdash{}{0pt}%
\pgfpathmoveto{\pgfqpoint{2.796329in}{0.903359in}}%
\pgfpathcurveto{\pgfqpoint{2.807379in}{0.903359in}}{\pgfqpoint{2.817978in}{0.907749in}}{\pgfqpoint{2.825792in}{0.915563in}}%
\pgfpathcurveto{\pgfqpoint{2.833605in}{0.923376in}}{\pgfqpoint{2.837995in}{0.933975in}}{\pgfqpoint{2.837995in}{0.945025in}}%
\pgfpathcurveto{\pgfqpoint{2.837995in}{0.956076in}}{\pgfqpoint{2.833605in}{0.966675in}}{\pgfqpoint{2.825792in}{0.974488in}}%
\pgfpathcurveto{\pgfqpoint{2.817978in}{0.982302in}}{\pgfqpoint{2.807379in}{0.986692in}}{\pgfqpoint{2.796329in}{0.986692in}}%
\pgfpathcurveto{\pgfqpoint{2.785279in}{0.986692in}}{\pgfqpoint{2.774680in}{0.982302in}}{\pgfqpoint{2.766866in}{0.974488in}}%
\pgfpathcurveto{\pgfqpoint{2.759052in}{0.966675in}}{\pgfqpoint{2.754662in}{0.956076in}}{\pgfqpoint{2.754662in}{0.945025in}}%
\pgfpathcurveto{\pgfqpoint{2.754662in}{0.933975in}}{\pgfqpoint{2.759052in}{0.923376in}}{\pgfqpoint{2.766866in}{0.915563in}}%
\pgfpathcurveto{\pgfqpoint{2.774680in}{0.907749in}}{\pgfqpoint{2.785279in}{0.903359in}}{\pgfqpoint{2.796329in}{0.903359in}}%
\pgfpathlineto{\pgfqpoint{2.796329in}{0.903359in}}%
\pgfpathclose%
\pgfusepath{stroke}%
\end{pgfscope}%
\begin{pgfscope}%
\pgfpathrectangle{\pgfqpoint{0.393053in}{0.375000in}}{\pgfqpoint{6.356833in}{5.175000in}}%
\pgfusepath{clip}%
\pgfsetbuttcap%
\pgfsetroundjoin%
\pgfsetlinewidth{1.003750pt}%
\definecolor{currentstroke}{rgb}{0.827451,0.827451,0.827451}%
\pgfsetstrokecolor{currentstroke}%
\pgfsetdash{}{0pt}%
\pgfpathmoveto{\pgfqpoint{2.498084in}{1.056522in}}%
\pgfpathcurveto{\pgfqpoint{2.509134in}{1.056522in}}{\pgfqpoint{2.519733in}{1.060912in}}{\pgfqpoint{2.527547in}{1.068726in}}%
\pgfpathcurveto{\pgfqpoint{2.535360in}{1.076539in}}{\pgfqpoint{2.539750in}{1.087138in}}{\pgfqpoint{2.539750in}{1.098188in}}%
\pgfpathcurveto{\pgfqpoint{2.539750in}{1.109239in}}{\pgfqpoint{2.535360in}{1.119838in}}{\pgfqpoint{2.527547in}{1.127651in}}%
\pgfpathcurveto{\pgfqpoint{2.519733in}{1.135465in}}{\pgfqpoint{2.509134in}{1.139855in}}{\pgfqpoint{2.498084in}{1.139855in}}%
\pgfpathcurveto{\pgfqpoint{2.487034in}{1.139855in}}{\pgfqpoint{2.476435in}{1.135465in}}{\pgfqpoint{2.468621in}{1.127651in}}%
\pgfpathcurveto{\pgfqpoint{2.460807in}{1.119838in}}{\pgfqpoint{2.456417in}{1.109239in}}{\pgfqpoint{2.456417in}{1.098188in}}%
\pgfpathcurveto{\pgfqpoint{2.456417in}{1.087138in}}{\pgfqpoint{2.460807in}{1.076539in}}{\pgfqpoint{2.468621in}{1.068726in}}%
\pgfpathcurveto{\pgfqpoint{2.476435in}{1.060912in}}{\pgfqpoint{2.487034in}{1.056522in}}{\pgfqpoint{2.498084in}{1.056522in}}%
\pgfpathlineto{\pgfqpoint{2.498084in}{1.056522in}}%
\pgfpathclose%
\pgfusepath{stroke}%
\end{pgfscope}%
\begin{pgfscope}%
\pgfpathrectangle{\pgfqpoint{0.393053in}{0.375000in}}{\pgfqpoint{6.356833in}{5.175000in}}%
\pgfusepath{clip}%
\pgfsetbuttcap%
\pgfsetroundjoin%
\pgfsetlinewidth{1.003750pt}%
\definecolor{currentstroke}{rgb}{0.827451,0.827451,0.827451}%
\pgfsetstrokecolor{currentstroke}%
\pgfsetdash{}{0pt}%
\pgfpathmoveto{\pgfqpoint{0.868151in}{2.818774in}}%
\pgfpathcurveto{\pgfqpoint{0.879201in}{2.818774in}}{\pgfqpoint{0.889800in}{2.823165in}}{\pgfqpoint{0.897614in}{2.830978in}}%
\pgfpathcurveto{\pgfqpoint{0.905428in}{2.838792in}}{\pgfqpoint{0.909818in}{2.849391in}}{\pgfqpoint{0.909818in}{2.860441in}}%
\pgfpathcurveto{\pgfqpoint{0.909818in}{2.871491in}}{\pgfqpoint{0.905428in}{2.882090in}}{\pgfqpoint{0.897614in}{2.889904in}}%
\pgfpathcurveto{\pgfqpoint{0.889800in}{2.897717in}}{\pgfqpoint{0.879201in}{2.902108in}}{\pgfqpoint{0.868151in}{2.902108in}}%
\pgfpathcurveto{\pgfqpoint{0.857101in}{2.902108in}}{\pgfqpoint{0.846502in}{2.897717in}}{\pgfqpoint{0.838688in}{2.889904in}}%
\pgfpathcurveto{\pgfqpoint{0.830875in}{2.882090in}}{\pgfqpoint{0.826484in}{2.871491in}}{\pgfqpoint{0.826484in}{2.860441in}}%
\pgfpathcurveto{\pgfqpoint{0.826484in}{2.849391in}}{\pgfqpoint{0.830875in}{2.838792in}}{\pgfqpoint{0.838688in}{2.830978in}}%
\pgfpathcurveto{\pgfqpoint{0.846502in}{2.823165in}}{\pgfqpoint{0.857101in}{2.818774in}}{\pgfqpoint{0.868151in}{2.818774in}}%
\pgfpathlineto{\pgfqpoint{0.868151in}{2.818774in}}%
\pgfpathclose%
\pgfusepath{stroke}%
\end{pgfscope}%
\begin{pgfscope}%
\pgfpathrectangle{\pgfqpoint{0.393053in}{0.375000in}}{\pgfqpoint{6.356833in}{5.175000in}}%
\pgfusepath{clip}%
\pgfsetbuttcap%
\pgfsetroundjoin%
\pgfsetlinewidth{1.003750pt}%
\definecolor{currentstroke}{rgb}{0.827451,0.827451,0.827451}%
\pgfsetstrokecolor{currentstroke}%
\pgfsetdash{}{0pt}%
\pgfpathmoveto{\pgfqpoint{2.437419in}{1.106016in}}%
\pgfpathcurveto{\pgfqpoint{2.448469in}{1.106016in}}{\pgfqpoint{2.459069in}{1.110406in}}{\pgfqpoint{2.466882in}{1.118219in}}%
\pgfpathcurveto{\pgfqpoint{2.474696in}{1.126033in}}{\pgfqpoint{2.479086in}{1.136632in}}{\pgfqpoint{2.479086in}{1.147682in}}%
\pgfpathcurveto{\pgfqpoint{2.479086in}{1.158732in}}{\pgfqpoint{2.474696in}{1.169331in}}{\pgfqpoint{2.466882in}{1.177145in}}%
\pgfpathcurveto{\pgfqpoint{2.459069in}{1.184959in}}{\pgfqpoint{2.448469in}{1.189349in}}{\pgfqpoint{2.437419in}{1.189349in}}%
\pgfpathcurveto{\pgfqpoint{2.426369in}{1.189349in}}{\pgfqpoint{2.415770in}{1.184959in}}{\pgfqpoint{2.407957in}{1.177145in}}%
\pgfpathcurveto{\pgfqpoint{2.400143in}{1.169331in}}{\pgfqpoint{2.395753in}{1.158732in}}{\pgfqpoint{2.395753in}{1.147682in}}%
\pgfpathcurveto{\pgfqpoint{2.395753in}{1.136632in}}{\pgfqpoint{2.400143in}{1.126033in}}{\pgfqpoint{2.407957in}{1.118219in}}%
\pgfpathcurveto{\pgfqpoint{2.415770in}{1.110406in}}{\pgfqpoint{2.426369in}{1.106016in}}{\pgfqpoint{2.437419in}{1.106016in}}%
\pgfpathlineto{\pgfqpoint{2.437419in}{1.106016in}}%
\pgfpathclose%
\pgfusepath{stroke}%
\end{pgfscope}%
\begin{pgfscope}%
\pgfpathrectangle{\pgfqpoint{0.393053in}{0.375000in}}{\pgfqpoint{6.356833in}{5.175000in}}%
\pgfusepath{clip}%
\pgfsetbuttcap%
\pgfsetroundjoin%
\pgfsetlinewidth{1.003750pt}%
\definecolor{currentstroke}{rgb}{0.827451,0.827451,0.827451}%
\pgfsetstrokecolor{currentstroke}%
\pgfsetdash{}{0pt}%
\pgfpathmoveto{\pgfqpoint{0.921910in}{2.777569in}}%
\pgfpathcurveto{\pgfqpoint{0.932960in}{2.777569in}}{\pgfqpoint{0.943559in}{2.781959in}}{\pgfqpoint{0.951373in}{2.789773in}}%
\pgfpathcurveto{\pgfqpoint{0.959186in}{2.797586in}}{\pgfqpoint{0.963577in}{2.808185in}}{\pgfqpoint{0.963577in}{2.819235in}}%
\pgfpathcurveto{\pgfqpoint{0.963577in}{2.830286in}}{\pgfqpoint{0.959186in}{2.840885in}}{\pgfqpoint{0.951373in}{2.848698in}}%
\pgfpathcurveto{\pgfqpoint{0.943559in}{2.856512in}}{\pgfqpoint{0.932960in}{2.860902in}}{\pgfqpoint{0.921910in}{2.860902in}}%
\pgfpathcurveto{\pgfqpoint{0.910860in}{2.860902in}}{\pgfqpoint{0.900261in}{2.856512in}}{\pgfqpoint{0.892447in}{2.848698in}}%
\pgfpathcurveto{\pgfqpoint{0.884633in}{2.840885in}}{\pgfqpoint{0.880243in}{2.830286in}}{\pgfqpoint{0.880243in}{2.819235in}}%
\pgfpathcurveto{\pgfqpoint{0.880243in}{2.808185in}}{\pgfqpoint{0.884633in}{2.797586in}}{\pgfqpoint{0.892447in}{2.789773in}}%
\pgfpathcurveto{\pgfqpoint{0.900261in}{2.781959in}}{\pgfqpoint{0.910860in}{2.777569in}}{\pgfqpoint{0.921910in}{2.777569in}}%
\pgfpathlineto{\pgfqpoint{0.921910in}{2.777569in}}%
\pgfpathclose%
\pgfusepath{stroke}%
\end{pgfscope}%
\begin{pgfscope}%
\pgfpathrectangle{\pgfqpoint{0.393053in}{0.375000in}}{\pgfqpoint{6.356833in}{5.175000in}}%
\pgfusepath{clip}%
\pgfsetbuttcap%
\pgfsetroundjoin%
\pgfsetlinewidth{1.003750pt}%
\definecolor{currentstroke}{rgb}{0.827451,0.827451,0.827451}%
\pgfsetstrokecolor{currentstroke}%
\pgfsetdash{}{0pt}%
\pgfpathmoveto{\pgfqpoint{3.649696in}{0.625044in}}%
\pgfpathcurveto{\pgfqpoint{3.660746in}{0.625044in}}{\pgfqpoint{3.671345in}{0.629434in}}{\pgfqpoint{3.679159in}{0.637248in}}%
\pgfpathcurveto{\pgfqpoint{3.686972in}{0.645061in}}{\pgfqpoint{3.691363in}{0.655660in}}{\pgfqpoint{3.691363in}{0.666710in}}%
\pgfpathcurveto{\pgfqpoint{3.691363in}{0.677761in}}{\pgfqpoint{3.686972in}{0.688360in}}{\pgfqpoint{3.679159in}{0.696173in}}%
\pgfpathcurveto{\pgfqpoint{3.671345in}{0.703987in}}{\pgfqpoint{3.660746in}{0.708377in}}{\pgfqpoint{3.649696in}{0.708377in}}%
\pgfpathcurveto{\pgfqpoint{3.638646in}{0.708377in}}{\pgfqpoint{3.628047in}{0.703987in}}{\pgfqpoint{3.620233in}{0.696173in}}%
\pgfpathcurveto{\pgfqpoint{3.612420in}{0.688360in}}{\pgfqpoint{3.608029in}{0.677761in}}{\pgfqpoint{3.608029in}{0.666710in}}%
\pgfpathcurveto{\pgfqpoint{3.608029in}{0.655660in}}{\pgfqpoint{3.612420in}{0.645061in}}{\pgfqpoint{3.620233in}{0.637248in}}%
\pgfpathcurveto{\pgfqpoint{3.628047in}{0.629434in}}{\pgfqpoint{3.638646in}{0.625044in}}{\pgfqpoint{3.649696in}{0.625044in}}%
\pgfpathlineto{\pgfqpoint{3.649696in}{0.625044in}}%
\pgfpathclose%
\pgfusepath{stroke}%
\end{pgfscope}%
\begin{pgfscope}%
\pgfpathrectangle{\pgfqpoint{0.393053in}{0.375000in}}{\pgfqpoint{6.356833in}{5.175000in}}%
\pgfusepath{clip}%
\pgfsetbuttcap%
\pgfsetroundjoin%
\pgfsetlinewidth{1.003750pt}%
\definecolor{currentstroke}{rgb}{0.827451,0.827451,0.827451}%
\pgfsetstrokecolor{currentstroke}%
\pgfsetdash{}{0pt}%
\pgfpathmoveto{\pgfqpoint{1.457810in}{1.831739in}}%
\pgfpathcurveto{\pgfqpoint{1.468860in}{1.831739in}}{\pgfqpoint{1.479459in}{1.836129in}}{\pgfqpoint{1.487273in}{1.843943in}}%
\pgfpathcurveto{\pgfqpoint{1.495086in}{1.851756in}}{\pgfqpoint{1.499477in}{1.862355in}}{\pgfqpoint{1.499477in}{1.873406in}}%
\pgfpathcurveto{\pgfqpoint{1.499477in}{1.884456in}}{\pgfqpoint{1.495086in}{1.895055in}}{\pgfqpoint{1.487273in}{1.902868in}}%
\pgfpathcurveto{\pgfqpoint{1.479459in}{1.910682in}}{\pgfqpoint{1.468860in}{1.915072in}}{\pgfqpoint{1.457810in}{1.915072in}}%
\pgfpathcurveto{\pgfqpoint{1.446760in}{1.915072in}}{\pgfqpoint{1.436161in}{1.910682in}}{\pgfqpoint{1.428347in}{1.902868in}}%
\pgfpathcurveto{\pgfqpoint{1.420533in}{1.895055in}}{\pgfqpoint{1.416143in}{1.884456in}}{\pgfqpoint{1.416143in}{1.873406in}}%
\pgfpathcurveto{\pgfqpoint{1.416143in}{1.862355in}}{\pgfqpoint{1.420533in}{1.851756in}}{\pgfqpoint{1.428347in}{1.843943in}}%
\pgfpathcurveto{\pgfqpoint{1.436161in}{1.836129in}}{\pgfqpoint{1.446760in}{1.831739in}}{\pgfqpoint{1.457810in}{1.831739in}}%
\pgfpathlineto{\pgfqpoint{1.457810in}{1.831739in}}%
\pgfpathclose%
\pgfusepath{stroke}%
\end{pgfscope}%
\begin{pgfscope}%
\pgfpathrectangle{\pgfqpoint{0.393053in}{0.375000in}}{\pgfqpoint{6.356833in}{5.175000in}}%
\pgfusepath{clip}%
\pgfsetbuttcap%
\pgfsetroundjoin%
\pgfsetlinewidth{1.003750pt}%
\definecolor{currentstroke}{rgb}{0.827451,0.827451,0.827451}%
\pgfsetstrokecolor{currentstroke}%
\pgfsetdash{}{0pt}%
\pgfpathmoveto{\pgfqpoint{0.418752in}{4.107793in}}%
\pgfpathcurveto{\pgfqpoint{0.429802in}{4.107793in}}{\pgfqpoint{0.440401in}{4.112183in}}{\pgfqpoint{0.448215in}{4.119996in}}%
\pgfpathcurveto{\pgfqpoint{0.456028in}{4.127810in}}{\pgfqpoint{0.460419in}{4.138409in}}{\pgfqpoint{0.460419in}{4.149459in}}%
\pgfpathcurveto{\pgfqpoint{0.460419in}{4.160509in}}{\pgfqpoint{0.456028in}{4.171108in}}{\pgfqpoint{0.448215in}{4.178922in}}%
\pgfpathcurveto{\pgfqpoint{0.440401in}{4.186736in}}{\pgfqpoint{0.429802in}{4.191126in}}{\pgfqpoint{0.418752in}{4.191126in}}%
\pgfpathcurveto{\pgfqpoint{0.407702in}{4.191126in}}{\pgfqpoint{0.397103in}{4.186736in}}{\pgfqpoint{0.389289in}{4.178922in}}%
\pgfpathcurveto{\pgfqpoint{0.381476in}{4.171108in}}{\pgfqpoint{0.377085in}{4.160509in}}{\pgfqpoint{0.377085in}{4.149459in}}%
\pgfpathcurveto{\pgfqpoint{0.377085in}{4.138409in}}{\pgfqpoint{0.381476in}{4.127810in}}{\pgfqpoint{0.389289in}{4.119996in}}%
\pgfpathcurveto{\pgfqpoint{0.397103in}{4.112183in}}{\pgfqpoint{0.407702in}{4.107793in}}{\pgfqpoint{0.418752in}{4.107793in}}%
\pgfpathlineto{\pgfqpoint{0.418752in}{4.107793in}}%
\pgfpathclose%
\pgfusepath{stroke}%
\end{pgfscope}%
\begin{pgfscope}%
\pgfpathrectangle{\pgfqpoint{0.393053in}{0.375000in}}{\pgfqpoint{6.356833in}{5.175000in}}%
\pgfusepath{clip}%
\pgfsetbuttcap%
\pgfsetroundjoin%
\pgfsetlinewidth{1.003750pt}%
\definecolor{currentstroke}{rgb}{0.827451,0.827451,0.827451}%
\pgfsetstrokecolor{currentstroke}%
\pgfsetdash{}{0pt}%
\pgfpathmoveto{\pgfqpoint{1.306751in}{1.985193in}}%
\pgfpathcurveto{\pgfqpoint{1.317801in}{1.985193in}}{\pgfqpoint{1.328400in}{1.989583in}}{\pgfqpoint{1.336214in}{1.997397in}}%
\pgfpathcurveto{\pgfqpoint{1.344027in}{2.005210in}}{\pgfqpoint{1.348418in}{2.015809in}}{\pgfqpoint{1.348418in}{2.026859in}}%
\pgfpathcurveto{\pgfqpoint{1.348418in}{2.037909in}}{\pgfqpoint{1.344027in}{2.048509in}}{\pgfqpoint{1.336214in}{2.056322in}}%
\pgfpathcurveto{\pgfqpoint{1.328400in}{2.064136in}}{\pgfqpoint{1.317801in}{2.068526in}}{\pgfqpoint{1.306751in}{2.068526in}}%
\pgfpathcurveto{\pgfqpoint{1.295701in}{2.068526in}}{\pgfqpoint{1.285102in}{2.064136in}}{\pgfqpoint{1.277288in}{2.056322in}}%
\pgfpathcurveto{\pgfqpoint{1.269475in}{2.048509in}}{\pgfqpoint{1.265084in}{2.037909in}}{\pgfqpoint{1.265084in}{2.026859in}}%
\pgfpathcurveto{\pgfqpoint{1.265084in}{2.015809in}}{\pgfqpoint{1.269475in}{2.005210in}}{\pgfqpoint{1.277288in}{1.997397in}}%
\pgfpathcurveto{\pgfqpoint{1.285102in}{1.989583in}}{\pgfqpoint{1.295701in}{1.985193in}}{\pgfqpoint{1.306751in}{1.985193in}}%
\pgfpathlineto{\pgfqpoint{1.306751in}{1.985193in}}%
\pgfpathclose%
\pgfusepath{stroke}%
\end{pgfscope}%
\begin{pgfscope}%
\pgfpathrectangle{\pgfqpoint{0.393053in}{0.375000in}}{\pgfqpoint{6.356833in}{5.175000in}}%
\pgfusepath{clip}%
\pgfsetbuttcap%
\pgfsetroundjoin%
\pgfsetlinewidth{1.003750pt}%
\definecolor{currentstroke}{rgb}{0.827451,0.827451,0.827451}%
\pgfsetstrokecolor{currentstroke}%
\pgfsetdash{}{0pt}%
\pgfpathmoveto{\pgfqpoint{1.255595in}{2.043062in}}%
\pgfpathcurveto{\pgfqpoint{1.266645in}{2.043062in}}{\pgfqpoint{1.277244in}{2.047452in}}{\pgfqpoint{1.285058in}{2.055266in}}%
\pgfpathcurveto{\pgfqpoint{1.292872in}{2.063079in}}{\pgfqpoint{1.297262in}{2.073678in}}{\pgfqpoint{1.297262in}{2.084728in}}%
\pgfpathcurveto{\pgfqpoint{1.297262in}{2.095779in}}{\pgfqpoint{1.292872in}{2.106378in}}{\pgfqpoint{1.285058in}{2.114191in}}%
\pgfpathcurveto{\pgfqpoint{1.277244in}{2.122005in}}{\pgfqpoint{1.266645in}{2.126395in}}{\pgfqpoint{1.255595in}{2.126395in}}%
\pgfpathcurveto{\pgfqpoint{1.244545in}{2.126395in}}{\pgfqpoint{1.233946in}{2.122005in}}{\pgfqpoint{1.226132in}{2.114191in}}%
\pgfpathcurveto{\pgfqpoint{1.218319in}{2.106378in}}{\pgfqpoint{1.213928in}{2.095779in}}{\pgfqpoint{1.213928in}{2.084728in}}%
\pgfpathcurveto{\pgfqpoint{1.213928in}{2.073678in}}{\pgfqpoint{1.218319in}{2.063079in}}{\pgfqpoint{1.226132in}{2.055266in}}%
\pgfpathcurveto{\pgfqpoint{1.233946in}{2.047452in}}{\pgfqpoint{1.244545in}{2.043062in}}{\pgfqpoint{1.255595in}{2.043062in}}%
\pgfpathlineto{\pgfqpoint{1.255595in}{2.043062in}}%
\pgfpathclose%
\pgfusepath{stroke}%
\end{pgfscope}%
\begin{pgfscope}%
\pgfpathrectangle{\pgfqpoint{0.393053in}{0.375000in}}{\pgfqpoint{6.356833in}{5.175000in}}%
\pgfusepath{clip}%
\pgfsetbuttcap%
\pgfsetroundjoin%
\pgfsetlinewidth{1.003750pt}%
\definecolor{currentstroke}{rgb}{0.827451,0.827451,0.827451}%
\pgfsetstrokecolor{currentstroke}%
\pgfsetdash{}{0pt}%
\pgfpathmoveto{\pgfqpoint{0.580211in}{3.378166in}}%
\pgfpathcurveto{\pgfqpoint{0.591261in}{3.378166in}}{\pgfqpoint{0.601860in}{3.382556in}}{\pgfqpoint{0.609674in}{3.390370in}}%
\pgfpathcurveto{\pgfqpoint{0.617487in}{3.398183in}}{\pgfqpoint{0.621878in}{3.408782in}}{\pgfqpoint{0.621878in}{3.419833in}}%
\pgfpathcurveto{\pgfqpoint{0.621878in}{3.430883in}}{\pgfqpoint{0.617487in}{3.441482in}}{\pgfqpoint{0.609674in}{3.449295in}}%
\pgfpathcurveto{\pgfqpoint{0.601860in}{3.457109in}}{\pgfqpoint{0.591261in}{3.461499in}}{\pgfqpoint{0.580211in}{3.461499in}}%
\pgfpathcurveto{\pgfqpoint{0.569161in}{3.461499in}}{\pgfqpoint{0.558562in}{3.457109in}}{\pgfqpoint{0.550748in}{3.449295in}}%
\pgfpathcurveto{\pgfqpoint{0.542935in}{3.441482in}}{\pgfqpoint{0.538544in}{3.430883in}}{\pgfqpoint{0.538544in}{3.419833in}}%
\pgfpathcurveto{\pgfqpoint{0.538544in}{3.408782in}}{\pgfqpoint{0.542935in}{3.398183in}}{\pgfqpoint{0.550748in}{3.390370in}}%
\pgfpathcurveto{\pgfqpoint{0.558562in}{3.382556in}}{\pgfqpoint{0.569161in}{3.378166in}}{\pgfqpoint{0.580211in}{3.378166in}}%
\pgfpathlineto{\pgfqpoint{0.580211in}{3.378166in}}%
\pgfpathclose%
\pgfusepath{stroke}%
\end{pgfscope}%
\begin{pgfscope}%
\pgfpathrectangle{\pgfqpoint{0.393053in}{0.375000in}}{\pgfqpoint{6.356833in}{5.175000in}}%
\pgfusepath{clip}%
\pgfsetbuttcap%
\pgfsetroundjoin%
\pgfsetlinewidth{1.003750pt}%
\definecolor{currentstroke}{rgb}{0.827451,0.827451,0.827451}%
\pgfsetstrokecolor{currentstroke}%
\pgfsetdash{}{0pt}%
\pgfpathmoveto{\pgfqpoint{3.893477in}{0.565646in}}%
\pgfpathcurveto{\pgfqpoint{3.904527in}{0.565646in}}{\pgfqpoint{3.915126in}{0.570036in}}{\pgfqpoint{3.922939in}{0.577850in}}%
\pgfpathcurveto{\pgfqpoint{3.930753in}{0.585664in}}{\pgfqpoint{3.935143in}{0.596263in}}{\pgfqpoint{3.935143in}{0.607313in}}%
\pgfpathcurveto{\pgfqpoint{3.935143in}{0.618363in}}{\pgfqpoint{3.930753in}{0.628962in}}{\pgfqpoint{3.922939in}{0.636776in}}%
\pgfpathcurveto{\pgfqpoint{3.915126in}{0.644589in}}{\pgfqpoint{3.904527in}{0.648980in}}{\pgfqpoint{3.893477in}{0.648980in}}%
\pgfpathcurveto{\pgfqpoint{3.882426in}{0.648980in}}{\pgfqpoint{3.871827in}{0.644589in}}{\pgfqpoint{3.864014in}{0.636776in}}%
\pgfpathcurveto{\pgfqpoint{3.856200in}{0.628962in}}{\pgfqpoint{3.851810in}{0.618363in}}{\pgfqpoint{3.851810in}{0.607313in}}%
\pgfpathcurveto{\pgfqpoint{3.851810in}{0.596263in}}{\pgfqpoint{3.856200in}{0.585664in}}{\pgfqpoint{3.864014in}{0.577850in}}%
\pgfpathcurveto{\pgfqpoint{3.871827in}{0.570036in}}{\pgfqpoint{3.882426in}{0.565646in}}{\pgfqpoint{3.893477in}{0.565646in}}%
\pgfpathlineto{\pgfqpoint{3.893477in}{0.565646in}}%
\pgfpathclose%
\pgfusepath{stroke}%
\end{pgfscope}%
\begin{pgfscope}%
\pgfpathrectangle{\pgfqpoint{0.393053in}{0.375000in}}{\pgfqpoint{6.356833in}{5.175000in}}%
\pgfusepath{clip}%
\pgfsetbuttcap%
\pgfsetroundjoin%
\pgfsetlinewidth{1.003750pt}%
\definecolor{currentstroke}{rgb}{0.827451,0.827451,0.827451}%
\pgfsetstrokecolor{currentstroke}%
\pgfsetdash{}{0pt}%
\pgfpathmoveto{\pgfqpoint{4.898856in}{0.391247in}}%
\pgfpathcurveto{\pgfqpoint{4.909906in}{0.391247in}}{\pgfqpoint{4.920505in}{0.395637in}}{\pgfqpoint{4.928319in}{0.403450in}}%
\pgfpathcurveto{\pgfqpoint{4.936132in}{0.411264in}}{\pgfqpoint{4.940523in}{0.421863in}}{\pgfqpoint{4.940523in}{0.432913in}}%
\pgfpathcurveto{\pgfqpoint{4.940523in}{0.443963in}}{\pgfqpoint{4.936132in}{0.454562in}}{\pgfqpoint{4.928319in}{0.462376in}}%
\pgfpathcurveto{\pgfqpoint{4.920505in}{0.470190in}}{\pgfqpoint{4.909906in}{0.474580in}}{\pgfqpoint{4.898856in}{0.474580in}}%
\pgfpathcurveto{\pgfqpoint{4.887806in}{0.474580in}}{\pgfqpoint{4.877207in}{0.470190in}}{\pgfqpoint{4.869393in}{0.462376in}}%
\pgfpathcurveto{\pgfqpoint{4.861580in}{0.454562in}}{\pgfqpoint{4.857189in}{0.443963in}}{\pgfqpoint{4.857189in}{0.432913in}}%
\pgfpathcurveto{\pgfqpoint{4.857189in}{0.421863in}}{\pgfqpoint{4.861580in}{0.411264in}}{\pgfqpoint{4.869393in}{0.403450in}}%
\pgfpathcurveto{\pgfqpoint{4.877207in}{0.395637in}}{\pgfqpoint{4.887806in}{0.391247in}}{\pgfqpoint{4.898856in}{0.391247in}}%
\pgfpathlineto{\pgfqpoint{4.898856in}{0.391247in}}%
\pgfpathclose%
\pgfusepath{stroke}%
\end{pgfscope}%
\begin{pgfscope}%
\pgfpathrectangle{\pgfqpoint{0.393053in}{0.375000in}}{\pgfqpoint{6.356833in}{5.175000in}}%
\pgfusepath{clip}%
\pgfsetbuttcap%
\pgfsetroundjoin%
\pgfsetlinewidth{1.003750pt}%
\definecolor{currentstroke}{rgb}{0.827451,0.827451,0.827451}%
\pgfsetstrokecolor{currentstroke}%
\pgfsetdash{}{0pt}%
\pgfpathmoveto{\pgfqpoint{1.843211in}{1.515821in}}%
\pgfpathcurveto{\pgfqpoint{1.854261in}{1.515821in}}{\pgfqpoint{1.864860in}{1.520211in}}{\pgfqpoint{1.872674in}{1.528025in}}%
\pgfpathcurveto{\pgfqpoint{1.880487in}{1.535838in}}{\pgfqpoint{1.884878in}{1.546437in}}{\pgfqpoint{1.884878in}{1.557487in}}%
\pgfpathcurveto{\pgfqpoint{1.884878in}{1.568537in}}{\pgfqpoint{1.880487in}{1.579136in}}{\pgfqpoint{1.872674in}{1.586950in}}%
\pgfpathcurveto{\pgfqpoint{1.864860in}{1.594764in}}{\pgfqpoint{1.854261in}{1.599154in}}{\pgfqpoint{1.843211in}{1.599154in}}%
\pgfpathcurveto{\pgfqpoint{1.832161in}{1.599154in}}{\pgfqpoint{1.821562in}{1.594764in}}{\pgfqpoint{1.813748in}{1.586950in}}%
\pgfpathcurveto{\pgfqpoint{1.805934in}{1.579136in}}{\pgfqpoint{1.801544in}{1.568537in}}{\pgfqpoint{1.801544in}{1.557487in}}%
\pgfpathcurveto{\pgfqpoint{1.801544in}{1.546437in}}{\pgfqpoint{1.805934in}{1.535838in}}{\pgfqpoint{1.813748in}{1.528025in}}%
\pgfpathcurveto{\pgfqpoint{1.821562in}{1.520211in}}{\pgfqpoint{1.832161in}{1.515821in}}{\pgfqpoint{1.843211in}{1.515821in}}%
\pgfpathlineto{\pgfqpoint{1.843211in}{1.515821in}}%
\pgfpathclose%
\pgfusepath{stroke}%
\end{pgfscope}%
\begin{pgfscope}%
\pgfpathrectangle{\pgfqpoint{0.393053in}{0.375000in}}{\pgfqpoint{6.356833in}{5.175000in}}%
\pgfusepath{clip}%
\pgfsetbuttcap%
\pgfsetroundjoin%
\pgfsetlinewidth{1.003750pt}%
\definecolor{currentstroke}{rgb}{0.827451,0.827451,0.827451}%
\pgfsetstrokecolor{currentstroke}%
\pgfsetdash{}{0pt}%
\pgfpathmoveto{\pgfqpoint{1.680974in}{1.628980in}}%
\pgfpathcurveto{\pgfqpoint{1.692024in}{1.628980in}}{\pgfqpoint{1.702623in}{1.633370in}}{\pgfqpoint{1.710437in}{1.641184in}}%
\pgfpathcurveto{\pgfqpoint{1.718250in}{1.648997in}}{\pgfqpoint{1.722641in}{1.659596in}}{\pgfqpoint{1.722641in}{1.670646in}}%
\pgfpathcurveto{\pgfqpoint{1.722641in}{1.681697in}}{\pgfqpoint{1.718250in}{1.692296in}}{\pgfqpoint{1.710437in}{1.700109in}}%
\pgfpathcurveto{\pgfqpoint{1.702623in}{1.707923in}}{\pgfqpoint{1.692024in}{1.712313in}}{\pgfqpoint{1.680974in}{1.712313in}}%
\pgfpathcurveto{\pgfqpoint{1.669924in}{1.712313in}}{\pgfqpoint{1.659325in}{1.707923in}}{\pgfqpoint{1.651511in}{1.700109in}}%
\pgfpathcurveto{\pgfqpoint{1.643698in}{1.692296in}}{\pgfqpoint{1.639307in}{1.681697in}}{\pgfqpoint{1.639307in}{1.670646in}}%
\pgfpathcurveto{\pgfqpoint{1.639307in}{1.659596in}}{\pgfqpoint{1.643698in}{1.648997in}}{\pgfqpoint{1.651511in}{1.641184in}}%
\pgfpathcurveto{\pgfqpoint{1.659325in}{1.633370in}}{\pgfqpoint{1.669924in}{1.628980in}}{\pgfqpoint{1.680974in}{1.628980in}}%
\pgfpathlineto{\pgfqpoint{1.680974in}{1.628980in}}%
\pgfpathclose%
\pgfusepath{stroke}%
\end{pgfscope}%
\begin{pgfscope}%
\pgfpathrectangle{\pgfqpoint{0.393053in}{0.375000in}}{\pgfqpoint{6.356833in}{5.175000in}}%
\pgfusepath{clip}%
\pgfsetbuttcap%
\pgfsetroundjoin%
\pgfsetlinewidth{1.003750pt}%
\definecolor{currentstroke}{rgb}{0.827451,0.827451,0.827451}%
\pgfsetstrokecolor{currentstroke}%
\pgfsetdash{}{0pt}%
\pgfpathmoveto{\pgfqpoint{0.652600in}{3.123417in}}%
\pgfpathcurveto{\pgfqpoint{0.663650in}{3.123417in}}{\pgfqpoint{0.674249in}{3.127808in}}{\pgfqpoint{0.682062in}{3.135621in}}%
\pgfpathcurveto{\pgfqpoint{0.689876in}{3.143435in}}{\pgfqpoint{0.694266in}{3.154034in}}{\pgfqpoint{0.694266in}{3.165084in}}%
\pgfpathcurveto{\pgfqpoint{0.694266in}{3.176134in}}{\pgfqpoint{0.689876in}{3.186733in}}{\pgfqpoint{0.682062in}{3.194547in}}%
\pgfpathcurveto{\pgfqpoint{0.674249in}{3.202360in}}{\pgfqpoint{0.663650in}{3.206751in}}{\pgfqpoint{0.652600in}{3.206751in}}%
\pgfpathcurveto{\pgfqpoint{0.641549in}{3.206751in}}{\pgfqpoint{0.630950in}{3.202360in}}{\pgfqpoint{0.623137in}{3.194547in}}%
\pgfpathcurveto{\pgfqpoint{0.615323in}{3.186733in}}{\pgfqpoint{0.610933in}{3.176134in}}{\pgfqpoint{0.610933in}{3.165084in}}%
\pgfpathcurveto{\pgfqpoint{0.610933in}{3.154034in}}{\pgfqpoint{0.615323in}{3.143435in}}{\pgfqpoint{0.623137in}{3.135621in}}%
\pgfpathcurveto{\pgfqpoint{0.630950in}{3.127808in}}{\pgfqpoint{0.641549in}{3.123417in}}{\pgfqpoint{0.652600in}{3.123417in}}%
\pgfpathlineto{\pgfqpoint{0.652600in}{3.123417in}}%
\pgfpathclose%
\pgfusepath{stroke}%
\end{pgfscope}%
\begin{pgfscope}%
\pgfpathrectangle{\pgfqpoint{0.393053in}{0.375000in}}{\pgfqpoint{6.356833in}{5.175000in}}%
\pgfusepath{clip}%
\pgfsetbuttcap%
\pgfsetroundjoin%
\pgfsetlinewidth{1.003750pt}%
\definecolor{currentstroke}{rgb}{0.827451,0.827451,0.827451}%
\pgfsetstrokecolor{currentstroke}%
\pgfsetdash{}{0pt}%
\pgfpathmoveto{\pgfqpoint{0.878661in}{2.805605in}}%
\pgfpathcurveto{\pgfqpoint{0.889711in}{2.805605in}}{\pgfqpoint{0.900310in}{2.809995in}}{\pgfqpoint{0.908123in}{2.817809in}}%
\pgfpathcurveto{\pgfqpoint{0.915937in}{2.825623in}}{\pgfqpoint{0.920327in}{2.836222in}}{\pgfqpoint{0.920327in}{2.847272in}}%
\pgfpathcurveto{\pgfqpoint{0.920327in}{2.858322in}}{\pgfqpoint{0.915937in}{2.868921in}}{\pgfqpoint{0.908123in}{2.876735in}}%
\pgfpathcurveto{\pgfqpoint{0.900310in}{2.884548in}}{\pgfqpoint{0.889711in}{2.888938in}}{\pgfqpoint{0.878661in}{2.888938in}}%
\pgfpathcurveto{\pgfqpoint{0.867610in}{2.888938in}}{\pgfqpoint{0.857011in}{2.884548in}}{\pgfqpoint{0.849198in}{2.876735in}}%
\pgfpathcurveto{\pgfqpoint{0.841384in}{2.868921in}}{\pgfqpoint{0.836994in}{2.858322in}}{\pgfqpoint{0.836994in}{2.847272in}}%
\pgfpathcurveto{\pgfqpoint{0.836994in}{2.836222in}}{\pgfqpoint{0.841384in}{2.825623in}}{\pgfqpoint{0.849198in}{2.817809in}}%
\pgfpathcurveto{\pgfqpoint{0.857011in}{2.809995in}}{\pgfqpoint{0.867610in}{2.805605in}}{\pgfqpoint{0.878661in}{2.805605in}}%
\pgfpathlineto{\pgfqpoint{0.878661in}{2.805605in}}%
\pgfpathclose%
\pgfusepath{stroke}%
\end{pgfscope}%
\begin{pgfscope}%
\pgfpathrectangle{\pgfqpoint{0.393053in}{0.375000in}}{\pgfqpoint{6.356833in}{5.175000in}}%
\pgfusepath{clip}%
\pgfsetbuttcap%
\pgfsetroundjoin%
\pgfsetlinewidth{1.003750pt}%
\definecolor{currentstroke}{rgb}{0.827451,0.827451,0.827451}%
\pgfsetstrokecolor{currentstroke}%
\pgfsetdash{}{0pt}%
\pgfpathmoveto{\pgfqpoint{4.927239in}{0.379634in}}%
\pgfpathcurveto{\pgfqpoint{4.938289in}{0.379634in}}{\pgfqpoint{4.948888in}{0.384025in}}{\pgfqpoint{4.956702in}{0.391838in}}%
\pgfpathcurveto{\pgfqpoint{4.964516in}{0.399652in}}{\pgfqpoint{4.968906in}{0.410251in}}{\pgfqpoint{4.968906in}{0.421301in}}%
\pgfpathcurveto{\pgfqpoint{4.968906in}{0.432351in}}{\pgfqpoint{4.964516in}{0.442950in}}{\pgfqpoint{4.956702in}{0.450764in}}%
\pgfpathcurveto{\pgfqpoint{4.948888in}{0.458578in}}{\pgfqpoint{4.938289in}{0.462968in}}{\pgfqpoint{4.927239in}{0.462968in}}%
\pgfpathcurveto{\pgfqpoint{4.916189in}{0.462968in}}{\pgfqpoint{4.905590in}{0.458578in}}{\pgfqpoint{4.897776in}{0.450764in}}%
\pgfpathcurveto{\pgfqpoint{4.889963in}{0.442950in}}{\pgfqpoint{4.885572in}{0.432351in}}{\pgfqpoint{4.885572in}{0.421301in}}%
\pgfpathcurveto{\pgfqpoint{4.885572in}{0.410251in}}{\pgfqpoint{4.889963in}{0.399652in}}{\pgfqpoint{4.897776in}{0.391838in}}%
\pgfpathcurveto{\pgfqpoint{4.905590in}{0.384025in}}{\pgfqpoint{4.916189in}{0.379634in}}{\pgfqpoint{4.927239in}{0.379634in}}%
\pgfpathlineto{\pgfqpoint{4.927239in}{0.379634in}}%
\pgfpathclose%
\pgfusepath{stroke}%
\end{pgfscope}%
\begin{pgfscope}%
\pgfpathrectangle{\pgfqpoint{0.393053in}{0.375000in}}{\pgfqpoint{6.356833in}{5.175000in}}%
\pgfusepath{clip}%
\pgfsetbuttcap%
\pgfsetroundjoin%
\pgfsetlinewidth{1.003750pt}%
\definecolor{currentstroke}{rgb}{0.827451,0.827451,0.827451}%
\pgfsetstrokecolor{currentstroke}%
\pgfsetdash{}{0pt}%
\pgfpathmoveto{\pgfqpoint{1.291595in}{1.999205in}}%
\pgfpathcurveto{\pgfqpoint{1.302646in}{1.999205in}}{\pgfqpoint{1.313245in}{2.003595in}}{\pgfqpoint{1.321058in}{2.011409in}}%
\pgfpathcurveto{\pgfqpoint{1.328872in}{2.019222in}}{\pgfqpoint{1.333262in}{2.029821in}}{\pgfqpoint{1.333262in}{2.040872in}}%
\pgfpathcurveto{\pgfqpoint{1.333262in}{2.051922in}}{\pgfqpoint{1.328872in}{2.062521in}}{\pgfqpoint{1.321058in}{2.070334in}}%
\pgfpathcurveto{\pgfqpoint{1.313245in}{2.078148in}}{\pgfqpoint{1.302646in}{2.082538in}}{\pgfqpoint{1.291595in}{2.082538in}}%
\pgfpathcurveto{\pgfqpoint{1.280545in}{2.082538in}}{\pgfqpoint{1.269946in}{2.078148in}}{\pgfqpoint{1.262133in}{2.070334in}}%
\pgfpathcurveto{\pgfqpoint{1.254319in}{2.062521in}}{\pgfqpoint{1.249929in}{2.051922in}}{\pgfqpoint{1.249929in}{2.040872in}}%
\pgfpathcurveto{\pgfqpoint{1.249929in}{2.029821in}}{\pgfqpoint{1.254319in}{2.019222in}}{\pgfqpoint{1.262133in}{2.011409in}}%
\pgfpathcurveto{\pgfqpoint{1.269946in}{2.003595in}}{\pgfqpoint{1.280545in}{1.999205in}}{\pgfqpoint{1.291595in}{1.999205in}}%
\pgfpathlineto{\pgfqpoint{1.291595in}{1.999205in}}%
\pgfpathclose%
\pgfusepath{stroke}%
\end{pgfscope}%
\begin{pgfscope}%
\pgfpathrectangle{\pgfqpoint{0.393053in}{0.375000in}}{\pgfqpoint{6.356833in}{5.175000in}}%
\pgfusepath{clip}%
\pgfsetbuttcap%
\pgfsetroundjoin%
\pgfsetlinewidth{1.003750pt}%
\definecolor{currentstroke}{rgb}{0.827451,0.827451,0.827451}%
\pgfsetstrokecolor{currentstroke}%
\pgfsetdash{}{0pt}%
\pgfpathmoveto{\pgfqpoint{1.702004in}{1.599430in}}%
\pgfpathcurveto{\pgfqpoint{1.713054in}{1.599430in}}{\pgfqpoint{1.723653in}{1.603820in}}{\pgfqpoint{1.731467in}{1.611633in}}%
\pgfpathcurveto{\pgfqpoint{1.739281in}{1.619447in}}{\pgfqpoint{1.743671in}{1.630046in}}{\pgfqpoint{1.743671in}{1.641096in}}%
\pgfpathcurveto{\pgfqpoint{1.743671in}{1.652146in}}{\pgfqpoint{1.739281in}{1.662745in}}{\pgfqpoint{1.731467in}{1.670559in}}%
\pgfpathcurveto{\pgfqpoint{1.723653in}{1.678373in}}{\pgfqpoint{1.713054in}{1.682763in}}{\pgfqpoint{1.702004in}{1.682763in}}%
\pgfpathcurveto{\pgfqpoint{1.690954in}{1.682763in}}{\pgfqpoint{1.680355in}{1.678373in}}{\pgfqpoint{1.672541in}{1.670559in}}%
\pgfpathcurveto{\pgfqpoint{1.664728in}{1.662745in}}{\pgfqpoint{1.660338in}{1.652146in}}{\pgfqpoint{1.660338in}{1.641096in}}%
\pgfpathcurveto{\pgfqpoint{1.660338in}{1.630046in}}{\pgfqpoint{1.664728in}{1.619447in}}{\pgfqpoint{1.672541in}{1.611633in}}%
\pgfpathcurveto{\pgfqpoint{1.680355in}{1.603820in}}{\pgfqpoint{1.690954in}{1.599430in}}{\pgfqpoint{1.702004in}{1.599430in}}%
\pgfpathlineto{\pgfqpoint{1.702004in}{1.599430in}}%
\pgfpathclose%
\pgfusepath{stroke}%
\end{pgfscope}%
\begin{pgfscope}%
\pgfpathrectangle{\pgfqpoint{0.393053in}{0.375000in}}{\pgfqpoint{6.356833in}{5.175000in}}%
\pgfusepath{clip}%
\pgfsetbuttcap%
\pgfsetroundjoin%
\pgfsetlinewidth{1.003750pt}%
\definecolor{currentstroke}{rgb}{0.827451,0.827451,0.827451}%
\pgfsetstrokecolor{currentstroke}%
\pgfsetdash{}{0pt}%
\pgfpathmoveto{\pgfqpoint{0.583953in}{3.273995in}}%
\pgfpathcurveto{\pgfqpoint{0.595003in}{3.273995in}}{\pgfqpoint{0.605602in}{3.278385in}}{\pgfqpoint{0.613416in}{3.286198in}}%
\pgfpathcurveto{\pgfqpoint{0.621229in}{3.294012in}}{\pgfqpoint{0.625620in}{3.304611in}}{\pgfqpoint{0.625620in}{3.315661in}}%
\pgfpathcurveto{\pgfqpoint{0.625620in}{3.326711in}}{\pgfqpoint{0.621229in}{3.337310in}}{\pgfqpoint{0.613416in}{3.345124in}}%
\pgfpathcurveto{\pgfqpoint{0.605602in}{3.352938in}}{\pgfqpoint{0.595003in}{3.357328in}}{\pgfqpoint{0.583953in}{3.357328in}}%
\pgfpathcurveto{\pgfqpoint{0.572903in}{3.357328in}}{\pgfqpoint{0.562304in}{3.352938in}}{\pgfqpoint{0.554490in}{3.345124in}}%
\pgfpathcurveto{\pgfqpoint{0.546677in}{3.337310in}}{\pgfqpoint{0.542286in}{3.326711in}}{\pgfqpoint{0.542286in}{3.315661in}}%
\pgfpathcurveto{\pgfqpoint{0.542286in}{3.304611in}}{\pgfqpoint{0.546677in}{3.294012in}}{\pgfqpoint{0.554490in}{3.286198in}}%
\pgfpathcurveto{\pgfqpoint{0.562304in}{3.278385in}}{\pgfqpoint{0.572903in}{3.273995in}}{\pgfqpoint{0.583953in}{3.273995in}}%
\pgfpathlineto{\pgfqpoint{0.583953in}{3.273995in}}%
\pgfpathclose%
\pgfusepath{stroke}%
\end{pgfscope}%
\begin{pgfscope}%
\pgfpathrectangle{\pgfqpoint{0.393053in}{0.375000in}}{\pgfqpoint{6.356833in}{5.175000in}}%
\pgfusepath{clip}%
\pgfsetbuttcap%
\pgfsetroundjoin%
\pgfsetlinewidth{1.003750pt}%
\definecolor{currentstroke}{rgb}{0.827451,0.827451,0.827451}%
\pgfsetstrokecolor{currentstroke}%
\pgfsetdash{}{0pt}%
\pgfpathmoveto{\pgfqpoint{1.224501in}{2.087879in}}%
\pgfpathcurveto{\pgfqpoint{1.235551in}{2.087879in}}{\pgfqpoint{1.246150in}{2.092269in}}{\pgfqpoint{1.253963in}{2.100083in}}%
\pgfpathcurveto{\pgfqpoint{1.261777in}{2.107896in}}{\pgfqpoint{1.266167in}{2.118495in}}{\pgfqpoint{1.266167in}{2.129546in}}%
\pgfpathcurveto{\pgfqpoint{1.266167in}{2.140596in}}{\pgfqpoint{1.261777in}{2.151195in}}{\pgfqpoint{1.253963in}{2.159008in}}%
\pgfpathcurveto{\pgfqpoint{1.246150in}{2.166822in}}{\pgfqpoint{1.235551in}{2.171212in}}{\pgfqpoint{1.224501in}{2.171212in}}%
\pgfpathcurveto{\pgfqpoint{1.213451in}{2.171212in}}{\pgfqpoint{1.202852in}{2.166822in}}{\pgfqpoint{1.195038in}{2.159008in}}%
\pgfpathcurveto{\pgfqpoint{1.187224in}{2.151195in}}{\pgfqpoint{1.182834in}{2.140596in}}{\pgfqpoint{1.182834in}{2.129546in}}%
\pgfpathcurveto{\pgfqpoint{1.182834in}{2.118495in}}{\pgfqpoint{1.187224in}{2.107896in}}{\pgfqpoint{1.195038in}{2.100083in}}%
\pgfpathcurveto{\pgfqpoint{1.202852in}{2.092269in}}{\pgfqpoint{1.213451in}{2.087879in}}{\pgfqpoint{1.224501in}{2.087879in}}%
\pgfpathlineto{\pgfqpoint{1.224501in}{2.087879in}}%
\pgfpathclose%
\pgfusepath{stroke}%
\end{pgfscope}%
\begin{pgfscope}%
\pgfpathrectangle{\pgfqpoint{0.393053in}{0.375000in}}{\pgfqpoint{6.356833in}{5.175000in}}%
\pgfusepath{clip}%
\pgfsetbuttcap%
\pgfsetroundjoin%
\pgfsetlinewidth{1.003750pt}%
\definecolor{currentstroke}{rgb}{0.827451,0.827451,0.827451}%
\pgfsetstrokecolor{currentstroke}%
\pgfsetdash{}{0pt}%
\pgfpathmoveto{\pgfqpoint{4.169452in}{0.483485in}}%
\pgfpathcurveto{\pgfqpoint{4.180503in}{0.483485in}}{\pgfqpoint{4.191102in}{0.487876in}}{\pgfqpoint{4.198915in}{0.495689in}}%
\pgfpathcurveto{\pgfqpoint{4.206729in}{0.503503in}}{\pgfqpoint{4.211119in}{0.514102in}}{\pgfqpoint{4.211119in}{0.525152in}}%
\pgfpathcurveto{\pgfqpoint{4.211119in}{0.536202in}}{\pgfqpoint{4.206729in}{0.546801in}}{\pgfqpoint{4.198915in}{0.554615in}}%
\pgfpathcurveto{\pgfqpoint{4.191102in}{0.562429in}}{\pgfqpoint{4.180503in}{0.566819in}}{\pgfqpoint{4.169452in}{0.566819in}}%
\pgfpathcurveto{\pgfqpoint{4.158402in}{0.566819in}}{\pgfqpoint{4.147803in}{0.562429in}}{\pgfqpoint{4.139990in}{0.554615in}}%
\pgfpathcurveto{\pgfqpoint{4.132176in}{0.546801in}}{\pgfqpoint{4.127786in}{0.536202in}}{\pgfqpoint{4.127786in}{0.525152in}}%
\pgfpathcurveto{\pgfqpoint{4.127786in}{0.514102in}}{\pgfqpoint{4.132176in}{0.503503in}}{\pgfqpoint{4.139990in}{0.495689in}}%
\pgfpathcurveto{\pgfqpoint{4.147803in}{0.487876in}}{\pgfqpoint{4.158402in}{0.483485in}}{\pgfqpoint{4.169452in}{0.483485in}}%
\pgfpathlineto{\pgfqpoint{4.169452in}{0.483485in}}%
\pgfpathclose%
\pgfusepath{stroke}%
\end{pgfscope}%
\begin{pgfscope}%
\pgfpathrectangle{\pgfqpoint{0.393053in}{0.375000in}}{\pgfqpoint{6.356833in}{5.175000in}}%
\pgfusepath{clip}%
\pgfsetbuttcap%
\pgfsetroundjoin%
\pgfsetlinewidth{1.003750pt}%
\definecolor{currentstroke}{rgb}{0.827451,0.827451,0.827451}%
\pgfsetstrokecolor{currentstroke}%
\pgfsetdash{}{0pt}%
\pgfpathmoveto{\pgfqpoint{1.195477in}{2.118618in}}%
\pgfpathcurveto{\pgfqpoint{1.206527in}{2.118618in}}{\pgfqpoint{1.217126in}{2.123008in}}{\pgfqpoint{1.224940in}{2.130821in}}%
\pgfpathcurveto{\pgfqpoint{1.232753in}{2.138635in}}{\pgfqpoint{1.237144in}{2.149234in}}{\pgfqpoint{1.237144in}{2.160284in}}%
\pgfpathcurveto{\pgfqpoint{1.237144in}{2.171334in}}{\pgfqpoint{1.232753in}{2.181933in}}{\pgfqpoint{1.224940in}{2.189747in}}%
\pgfpathcurveto{\pgfqpoint{1.217126in}{2.197561in}}{\pgfqpoint{1.206527in}{2.201951in}}{\pgfqpoint{1.195477in}{2.201951in}}%
\pgfpathcurveto{\pgfqpoint{1.184427in}{2.201951in}}{\pgfqpoint{1.173828in}{2.197561in}}{\pgfqpoint{1.166014in}{2.189747in}}%
\pgfpathcurveto{\pgfqpoint{1.158201in}{2.181933in}}{\pgfqpoint{1.153810in}{2.171334in}}{\pgfqpoint{1.153810in}{2.160284in}}%
\pgfpathcurveto{\pgfqpoint{1.153810in}{2.149234in}}{\pgfqpoint{1.158201in}{2.138635in}}{\pgfqpoint{1.166014in}{2.130821in}}%
\pgfpathcurveto{\pgfqpoint{1.173828in}{2.123008in}}{\pgfqpoint{1.184427in}{2.118618in}}{\pgfqpoint{1.195477in}{2.118618in}}%
\pgfpathlineto{\pgfqpoint{1.195477in}{2.118618in}}%
\pgfpathclose%
\pgfusepath{stroke}%
\end{pgfscope}%
\begin{pgfscope}%
\pgfpathrectangle{\pgfqpoint{0.393053in}{0.375000in}}{\pgfqpoint{6.356833in}{5.175000in}}%
\pgfusepath{clip}%
\pgfsetbuttcap%
\pgfsetroundjoin%
\pgfsetlinewidth{1.003750pt}%
\definecolor{currentstroke}{rgb}{0.827451,0.827451,0.827451}%
\pgfsetstrokecolor{currentstroke}%
\pgfsetdash{}{0pt}%
\pgfpathmoveto{\pgfqpoint{3.256902in}{0.730828in}}%
\pgfpathcurveto{\pgfqpoint{3.267953in}{0.730828in}}{\pgfqpoint{3.278552in}{0.735218in}}{\pgfqpoint{3.286365in}{0.743032in}}%
\pgfpathcurveto{\pgfqpoint{3.294179in}{0.750846in}}{\pgfqpoint{3.298569in}{0.761445in}}{\pgfqpoint{3.298569in}{0.772495in}}%
\pgfpathcurveto{\pgfqpoint{3.298569in}{0.783545in}}{\pgfqpoint{3.294179in}{0.794144in}}{\pgfqpoint{3.286365in}{0.801958in}}%
\pgfpathcurveto{\pgfqpoint{3.278552in}{0.809771in}}{\pgfqpoint{3.267953in}{0.814162in}}{\pgfqpoint{3.256902in}{0.814162in}}%
\pgfpathcurveto{\pgfqpoint{3.245852in}{0.814162in}}{\pgfqpoint{3.235253in}{0.809771in}}{\pgfqpoint{3.227440in}{0.801958in}}%
\pgfpathcurveto{\pgfqpoint{3.219626in}{0.794144in}}{\pgfqpoint{3.215236in}{0.783545in}}{\pgfqpoint{3.215236in}{0.772495in}}%
\pgfpathcurveto{\pgfqpoint{3.215236in}{0.761445in}}{\pgfqpoint{3.219626in}{0.750846in}}{\pgfqpoint{3.227440in}{0.743032in}}%
\pgfpathcurveto{\pgfqpoint{3.235253in}{0.735218in}}{\pgfqpoint{3.245852in}{0.730828in}}{\pgfqpoint{3.256902in}{0.730828in}}%
\pgfpathlineto{\pgfqpoint{3.256902in}{0.730828in}}%
\pgfpathclose%
\pgfusepath{stroke}%
\end{pgfscope}%
\begin{pgfscope}%
\pgfpathrectangle{\pgfqpoint{0.393053in}{0.375000in}}{\pgfqpoint{6.356833in}{5.175000in}}%
\pgfusepath{clip}%
\pgfsetbuttcap%
\pgfsetroundjoin%
\pgfsetlinewidth{1.003750pt}%
\definecolor{currentstroke}{rgb}{0.827451,0.827451,0.827451}%
\pgfsetstrokecolor{currentstroke}%
\pgfsetdash{}{0pt}%
\pgfpathmoveto{\pgfqpoint{1.980937in}{1.385443in}}%
\pgfpathcurveto{\pgfqpoint{1.991987in}{1.385443in}}{\pgfqpoint{2.002586in}{1.389833in}}{\pgfqpoint{2.010399in}{1.397647in}}%
\pgfpathcurveto{\pgfqpoint{2.018213in}{1.405460in}}{\pgfqpoint{2.022603in}{1.416059in}}{\pgfqpoint{2.022603in}{1.427109in}}%
\pgfpathcurveto{\pgfqpoint{2.022603in}{1.438160in}}{\pgfqpoint{2.018213in}{1.448759in}}{\pgfqpoint{2.010399in}{1.456572in}}%
\pgfpathcurveto{\pgfqpoint{2.002586in}{1.464386in}}{\pgfqpoint{1.991987in}{1.468776in}}{\pgfqpoint{1.980937in}{1.468776in}}%
\pgfpathcurveto{\pgfqpoint{1.969886in}{1.468776in}}{\pgfqpoint{1.959287in}{1.464386in}}{\pgfqpoint{1.951474in}{1.456572in}}%
\pgfpathcurveto{\pgfqpoint{1.943660in}{1.448759in}}{\pgfqpoint{1.939270in}{1.438160in}}{\pgfqpoint{1.939270in}{1.427109in}}%
\pgfpathcurveto{\pgfqpoint{1.939270in}{1.416059in}}{\pgfqpoint{1.943660in}{1.405460in}}{\pgfqpoint{1.951474in}{1.397647in}}%
\pgfpathcurveto{\pgfqpoint{1.959287in}{1.389833in}}{\pgfqpoint{1.969886in}{1.385443in}}{\pgfqpoint{1.980937in}{1.385443in}}%
\pgfpathlineto{\pgfqpoint{1.980937in}{1.385443in}}%
\pgfpathclose%
\pgfusepath{stroke}%
\end{pgfscope}%
\begin{pgfscope}%
\pgfpathrectangle{\pgfqpoint{0.393053in}{0.375000in}}{\pgfqpoint{6.356833in}{5.175000in}}%
\pgfusepath{clip}%
\pgfsetbuttcap%
\pgfsetroundjoin%
\pgfsetlinewidth{1.003750pt}%
\definecolor{currentstroke}{rgb}{0.827451,0.827451,0.827451}%
\pgfsetstrokecolor{currentstroke}%
\pgfsetdash{}{0pt}%
\pgfpathmoveto{\pgfqpoint{2.124283in}{1.277669in}}%
\pgfpathcurveto{\pgfqpoint{2.135334in}{1.277669in}}{\pgfqpoint{2.145933in}{1.282059in}}{\pgfqpoint{2.153746in}{1.289873in}}%
\pgfpathcurveto{\pgfqpoint{2.161560in}{1.297686in}}{\pgfqpoint{2.165950in}{1.308286in}}{\pgfqpoint{2.165950in}{1.319336in}}%
\pgfpathcurveto{\pgfqpoint{2.165950in}{1.330386in}}{\pgfqpoint{2.161560in}{1.340985in}}{\pgfqpoint{2.153746in}{1.348798in}}%
\pgfpathcurveto{\pgfqpoint{2.145933in}{1.356612in}}{\pgfqpoint{2.135334in}{1.361002in}}{\pgfqpoint{2.124283in}{1.361002in}}%
\pgfpathcurveto{\pgfqpoint{2.113233in}{1.361002in}}{\pgfqpoint{2.102634in}{1.356612in}}{\pgfqpoint{2.094821in}{1.348798in}}%
\pgfpathcurveto{\pgfqpoint{2.087007in}{1.340985in}}{\pgfqpoint{2.082617in}{1.330386in}}{\pgfqpoint{2.082617in}{1.319336in}}%
\pgfpathcurveto{\pgfqpoint{2.082617in}{1.308286in}}{\pgfqpoint{2.087007in}{1.297686in}}{\pgfqpoint{2.094821in}{1.289873in}}%
\pgfpathcurveto{\pgfqpoint{2.102634in}{1.282059in}}{\pgfqpoint{2.113233in}{1.277669in}}{\pgfqpoint{2.124283in}{1.277669in}}%
\pgfpathlineto{\pgfqpoint{2.124283in}{1.277669in}}%
\pgfpathclose%
\pgfusepath{stroke}%
\end{pgfscope}%
\begin{pgfscope}%
\pgfpathrectangle{\pgfqpoint{0.393053in}{0.375000in}}{\pgfqpoint{6.356833in}{5.175000in}}%
\pgfusepath{clip}%
\pgfsetbuttcap%
\pgfsetroundjoin%
\pgfsetlinewidth{1.003750pt}%
\definecolor{currentstroke}{rgb}{0.827451,0.827451,0.827451}%
\pgfsetstrokecolor{currentstroke}%
\pgfsetdash{}{0pt}%
\pgfpathmoveto{\pgfqpoint{1.050681in}{2.479907in}}%
\pgfpathcurveto{\pgfqpoint{1.061731in}{2.479907in}}{\pgfqpoint{1.072330in}{2.484297in}}{\pgfqpoint{1.080144in}{2.492111in}}%
\pgfpathcurveto{\pgfqpoint{1.087957in}{2.499925in}}{\pgfqpoint{1.092347in}{2.510524in}}{\pgfqpoint{1.092347in}{2.521574in}}%
\pgfpathcurveto{\pgfqpoint{1.092347in}{2.532624in}}{\pgfqpoint{1.087957in}{2.543223in}}{\pgfqpoint{1.080144in}{2.551037in}}%
\pgfpathcurveto{\pgfqpoint{1.072330in}{2.558850in}}{\pgfqpoint{1.061731in}{2.563240in}}{\pgfqpoint{1.050681in}{2.563240in}}%
\pgfpathcurveto{\pgfqpoint{1.039631in}{2.563240in}}{\pgfqpoint{1.029032in}{2.558850in}}{\pgfqpoint{1.021218in}{2.551037in}}%
\pgfpathcurveto{\pgfqpoint{1.013404in}{2.543223in}}{\pgfqpoint{1.009014in}{2.532624in}}{\pgfqpoint{1.009014in}{2.521574in}}%
\pgfpathcurveto{\pgfqpoint{1.009014in}{2.510524in}}{\pgfqpoint{1.013404in}{2.499925in}}{\pgfqpoint{1.021218in}{2.492111in}}%
\pgfpathcurveto{\pgfqpoint{1.029032in}{2.484297in}}{\pgfqpoint{1.039631in}{2.479907in}}{\pgfqpoint{1.050681in}{2.479907in}}%
\pgfpathlineto{\pgfqpoint{1.050681in}{2.479907in}}%
\pgfpathclose%
\pgfusepath{stroke}%
\end{pgfscope}%
\begin{pgfscope}%
\pgfpathrectangle{\pgfqpoint{0.393053in}{0.375000in}}{\pgfqpoint{6.356833in}{5.175000in}}%
\pgfusepath{clip}%
\pgfsetbuttcap%
\pgfsetroundjoin%
\pgfsetlinewidth{1.003750pt}%
\definecolor{currentstroke}{rgb}{0.827451,0.827451,0.827451}%
\pgfsetstrokecolor{currentstroke}%
\pgfsetdash{}{0pt}%
\pgfpathmoveto{\pgfqpoint{0.499793in}{3.640317in}}%
\pgfpathcurveto{\pgfqpoint{0.510843in}{3.640317in}}{\pgfqpoint{0.521442in}{3.644707in}}{\pgfqpoint{0.529256in}{3.652521in}}%
\pgfpathcurveto{\pgfqpoint{0.537070in}{3.660334in}}{\pgfqpoint{0.541460in}{3.670934in}}{\pgfqpoint{0.541460in}{3.681984in}}%
\pgfpathcurveto{\pgfqpoint{0.541460in}{3.693034in}}{\pgfqpoint{0.537070in}{3.703633in}}{\pgfqpoint{0.529256in}{3.711446in}}%
\pgfpathcurveto{\pgfqpoint{0.521442in}{3.719260in}}{\pgfqpoint{0.510843in}{3.723650in}}{\pgfqpoint{0.499793in}{3.723650in}}%
\pgfpathcurveto{\pgfqpoint{0.488743in}{3.723650in}}{\pgfqpoint{0.478144in}{3.719260in}}{\pgfqpoint{0.470331in}{3.711446in}}%
\pgfpathcurveto{\pgfqpoint{0.462517in}{3.703633in}}{\pgfqpoint{0.458127in}{3.693034in}}{\pgfqpoint{0.458127in}{3.681984in}}%
\pgfpathcurveto{\pgfqpoint{0.458127in}{3.670934in}}{\pgfqpoint{0.462517in}{3.660334in}}{\pgfqpoint{0.470331in}{3.652521in}}%
\pgfpathcurveto{\pgfqpoint{0.478144in}{3.644707in}}{\pgfqpoint{0.488743in}{3.640317in}}{\pgfqpoint{0.499793in}{3.640317in}}%
\pgfpathlineto{\pgfqpoint{0.499793in}{3.640317in}}%
\pgfpathclose%
\pgfusepath{stroke}%
\end{pgfscope}%
\begin{pgfscope}%
\pgfpathrectangle{\pgfqpoint{0.393053in}{0.375000in}}{\pgfqpoint{6.356833in}{5.175000in}}%
\pgfusepath{clip}%
\pgfsetbuttcap%
\pgfsetroundjoin%
\pgfsetlinewidth{1.003750pt}%
\definecolor{currentstroke}{rgb}{0.827451,0.827451,0.827451}%
\pgfsetstrokecolor{currentstroke}%
\pgfsetdash{}{0pt}%
\pgfpathmoveto{\pgfqpoint{0.646474in}{3.151616in}}%
\pgfpathcurveto{\pgfqpoint{0.657525in}{3.151616in}}{\pgfqpoint{0.668124in}{3.156006in}}{\pgfqpoint{0.675937in}{3.163820in}}%
\pgfpathcurveto{\pgfqpoint{0.683751in}{3.171634in}}{\pgfqpoint{0.688141in}{3.182233in}}{\pgfqpoint{0.688141in}{3.193283in}}%
\pgfpathcurveto{\pgfqpoint{0.688141in}{3.204333in}}{\pgfqpoint{0.683751in}{3.214932in}}{\pgfqpoint{0.675937in}{3.222745in}}%
\pgfpathcurveto{\pgfqpoint{0.668124in}{3.230559in}}{\pgfqpoint{0.657525in}{3.234949in}}{\pgfqpoint{0.646474in}{3.234949in}}%
\pgfpathcurveto{\pgfqpoint{0.635424in}{3.234949in}}{\pgfqpoint{0.624825in}{3.230559in}}{\pgfqpoint{0.617012in}{3.222745in}}%
\pgfpathcurveto{\pgfqpoint{0.609198in}{3.214932in}}{\pgfqpoint{0.604808in}{3.204333in}}{\pgfqpoint{0.604808in}{3.193283in}}%
\pgfpathcurveto{\pgfqpoint{0.604808in}{3.182233in}}{\pgfqpoint{0.609198in}{3.171634in}}{\pgfqpoint{0.617012in}{3.163820in}}%
\pgfpathcurveto{\pgfqpoint{0.624825in}{3.156006in}}{\pgfqpoint{0.635424in}{3.151616in}}{\pgfqpoint{0.646474in}{3.151616in}}%
\pgfpathlineto{\pgfqpoint{0.646474in}{3.151616in}}%
\pgfpathclose%
\pgfusepath{stroke}%
\end{pgfscope}%
\begin{pgfscope}%
\pgfpathrectangle{\pgfqpoint{0.393053in}{0.375000in}}{\pgfqpoint{6.356833in}{5.175000in}}%
\pgfusepath{clip}%
\pgfsetbuttcap%
\pgfsetroundjoin%
\pgfsetlinewidth{1.003750pt}%
\definecolor{currentstroke}{rgb}{0.827451,0.827451,0.827451}%
\pgfsetstrokecolor{currentstroke}%
\pgfsetdash{}{0pt}%
\pgfpathmoveto{\pgfqpoint{3.695118in}{0.605158in}}%
\pgfpathcurveto{\pgfqpoint{3.706168in}{0.605158in}}{\pgfqpoint{3.716767in}{0.609548in}}{\pgfqpoint{3.724581in}{0.617362in}}%
\pgfpathcurveto{\pgfqpoint{3.732394in}{0.625175in}}{\pgfqpoint{3.736785in}{0.635775in}}{\pgfqpoint{3.736785in}{0.646825in}}%
\pgfpathcurveto{\pgfqpoint{3.736785in}{0.657875in}}{\pgfqpoint{3.732394in}{0.668474in}}{\pgfqpoint{3.724581in}{0.676287in}}%
\pgfpathcurveto{\pgfqpoint{3.716767in}{0.684101in}}{\pgfqpoint{3.706168in}{0.688491in}}{\pgfqpoint{3.695118in}{0.688491in}}%
\pgfpathcurveto{\pgfqpoint{3.684068in}{0.688491in}}{\pgfqpoint{3.673469in}{0.684101in}}{\pgfqpoint{3.665655in}{0.676287in}}%
\pgfpathcurveto{\pgfqpoint{3.657842in}{0.668474in}}{\pgfqpoint{3.653451in}{0.657875in}}{\pgfqpoint{3.653451in}{0.646825in}}%
\pgfpathcurveto{\pgfqpoint{3.653451in}{0.635775in}}{\pgfqpoint{3.657842in}{0.625175in}}{\pgfqpoint{3.665655in}{0.617362in}}%
\pgfpathcurveto{\pgfqpoint{3.673469in}{0.609548in}}{\pgfqpoint{3.684068in}{0.605158in}}{\pgfqpoint{3.695118in}{0.605158in}}%
\pgfpathlineto{\pgfqpoint{3.695118in}{0.605158in}}%
\pgfpathclose%
\pgfusepath{stroke}%
\end{pgfscope}%
\begin{pgfscope}%
\pgfpathrectangle{\pgfqpoint{0.393053in}{0.375000in}}{\pgfqpoint{6.356833in}{5.175000in}}%
\pgfusepath{clip}%
\pgfsetbuttcap%
\pgfsetroundjoin%
\pgfsetlinewidth{1.003750pt}%
\definecolor{currentstroke}{rgb}{0.827451,0.827451,0.827451}%
\pgfsetstrokecolor{currentstroke}%
\pgfsetdash{}{0pt}%
\pgfpathmoveto{\pgfqpoint{1.365917in}{1.953195in}}%
\pgfpathcurveto{\pgfqpoint{1.376967in}{1.953195in}}{\pgfqpoint{1.387566in}{1.957586in}}{\pgfqpoint{1.395380in}{1.965399in}}%
\pgfpathcurveto{\pgfqpoint{1.403193in}{1.973213in}}{\pgfqpoint{1.407584in}{1.983812in}}{\pgfqpoint{1.407584in}{1.994862in}}%
\pgfpathcurveto{\pgfqpoint{1.407584in}{2.005912in}}{\pgfqpoint{1.403193in}{2.016511in}}{\pgfqpoint{1.395380in}{2.024325in}}%
\pgfpathcurveto{\pgfqpoint{1.387566in}{2.032138in}}{\pgfqpoint{1.376967in}{2.036529in}}{\pgfqpoint{1.365917in}{2.036529in}}%
\pgfpathcurveto{\pgfqpoint{1.354867in}{2.036529in}}{\pgfqpoint{1.344268in}{2.032138in}}{\pgfqpoint{1.336454in}{2.024325in}}%
\pgfpathcurveto{\pgfqpoint{1.328641in}{2.016511in}}{\pgfqpoint{1.324250in}{2.005912in}}{\pgfqpoint{1.324250in}{1.994862in}}%
\pgfpathcurveto{\pgfqpoint{1.324250in}{1.983812in}}{\pgfqpoint{1.328641in}{1.973213in}}{\pgfqpoint{1.336454in}{1.965399in}}%
\pgfpathcurveto{\pgfqpoint{1.344268in}{1.957586in}}{\pgfqpoint{1.354867in}{1.953195in}}{\pgfqpoint{1.365917in}{1.953195in}}%
\pgfpathlineto{\pgfqpoint{1.365917in}{1.953195in}}%
\pgfpathclose%
\pgfusepath{stroke}%
\end{pgfscope}%
\begin{pgfscope}%
\pgfpathrectangle{\pgfqpoint{0.393053in}{0.375000in}}{\pgfqpoint{6.356833in}{5.175000in}}%
\pgfusepath{clip}%
\pgfsetbuttcap%
\pgfsetroundjoin%
\pgfsetlinewidth{1.003750pt}%
\definecolor{currentstroke}{rgb}{0.827451,0.827451,0.827451}%
\pgfsetstrokecolor{currentstroke}%
\pgfsetdash{}{0pt}%
\pgfpathmoveto{\pgfqpoint{2.287965in}{1.196822in}}%
\pgfpathcurveto{\pgfqpoint{2.299015in}{1.196822in}}{\pgfqpoint{2.309614in}{1.201212in}}{\pgfqpoint{2.317427in}{1.209026in}}%
\pgfpathcurveto{\pgfqpoint{2.325241in}{1.216839in}}{\pgfqpoint{2.329631in}{1.227438in}}{\pgfqpoint{2.329631in}{1.238488in}}%
\pgfpathcurveto{\pgfqpoint{2.329631in}{1.249539in}}{\pgfqpoint{2.325241in}{1.260138in}}{\pgfqpoint{2.317427in}{1.267951in}}%
\pgfpathcurveto{\pgfqpoint{2.309614in}{1.275765in}}{\pgfqpoint{2.299015in}{1.280155in}}{\pgfqpoint{2.287965in}{1.280155in}}%
\pgfpathcurveto{\pgfqpoint{2.276915in}{1.280155in}}{\pgfqpoint{2.266316in}{1.275765in}}{\pgfqpoint{2.258502in}{1.267951in}}%
\pgfpathcurveto{\pgfqpoint{2.250688in}{1.260138in}}{\pgfqpoint{2.246298in}{1.249539in}}{\pgfqpoint{2.246298in}{1.238488in}}%
\pgfpathcurveto{\pgfqpoint{2.246298in}{1.227438in}}{\pgfqpoint{2.250688in}{1.216839in}}{\pgfqpoint{2.258502in}{1.209026in}}%
\pgfpathcurveto{\pgfqpoint{2.266316in}{1.201212in}}{\pgfqpoint{2.276915in}{1.196822in}}{\pgfqpoint{2.287965in}{1.196822in}}%
\pgfpathlineto{\pgfqpoint{2.287965in}{1.196822in}}%
\pgfpathclose%
\pgfusepath{stroke}%
\end{pgfscope}%
\begin{pgfscope}%
\pgfpathrectangle{\pgfqpoint{0.393053in}{0.375000in}}{\pgfqpoint{6.356833in}{5.175000in}}%
\pgfusepath{clip}%
\pgfsetbuttcap%
\pgfsetroundjoin%
\pgfsetlinewidth{1.003750pt}%
\definecolor{currentstroke}{rgb}{0.827451,0.827451,0.827451}%
\pgfsetstrokecolor{currentstroke}%
\pgfsetdash{}{0pt}%
\pgfpathmoveto{\pgfqpoint{1.858747in}{1.496619in}}%
\pgfpathcurveto{\pgfqpoint{1.869797in}{1.496619in}}{\pgfqpoint{1.880396in}{1.501009in}}{\pgfqpoint{1.888209in}{1.508823in}}%
\pgfpathcurveto{\pgfqpoint{1.896023in}{1.516636in}}{\pgfqpoint{1.900413in}{1.527236in}}{\pgfqpoint{1.900413in}{1.538286in}}%
\pgfpathcurveto{\pgfqpoint{1.900413in}{1.549336in}}{\pgfqpoint{1.896023in}{1.559935in}}{\pgfqpoint{1.888209in}{1.567748in}}%
\pgfpathcurveto{\pgfqpoint{1.880396in}{1.575562in}}{\pgfqpoint{1.869797in}{1.579952in}}{\pgfqpoint{1.858747in}{1.579952in}}%
\pgfpathcurveto{\pgfqpoint{1.847696in}{1.579952in}}{\pgfqpoint{1.837097in}{1.575562in}}{\pgfqpoint{1.829284in}{1.567748in}}%
\pgfpathcurveto{\pgfqpoint{1.821470in}{1.559935in}}{\pgfqpoint{1.817080in}{1.549336in}}{\pgfqpoint{1.817080in}{1.538286in}}%
\pgfpathcurveto{\pgfqpoint{1.817080in}{1.527236in}}{\pgfqpoint{1.821470in}{1.516636in}}{\pgfqpoint{1.829284in}{1.508823in}}%
\pgfpathcurveto{\pgfqpoint{1.837097in}{1.501009in}}{\pgfqpoint{1.847696in}{1.496619in}}{\pgfqpoint{1.858747in}{1.496619in}}%
\pgfpathlineto{\pgfqpoint{1.858747in}{1.496619in}}%
\pgfpathclose%
\pgfusepath{stroke}%
\end{pgfscope}%
\begin{pgfscope}%
\pgfpathrectangle{\pgfqpoint{0.393053in}{0.375000in}}{\pgfqpoint{6.356833in}{5.175000in}}%
\pgfusepath{clip}%
\pgfsetbuttcap%
\pgfsetroundjoin%
\pgfsetlinewidth{1.003750pt}%
\definecolor{currentstroke}{rgb}{0.827451,0.827451,0.827451}%
\pgfsetstrokecolor{currentstroke}%
\pgfsetdash{}{0pt}%
\pgfpathmoveto{\pgfqpoint{0.430992in}{4.025736in}}%
\pgfpathcurveto{\pgfqpoint{0.442043in}{4.025736in}}{\pgfqpoint{0.452642in}{4.030127in}}{\pgfqpoint{0.460455in}{4.037940in}}%
\pgfpathcurveto{\pgfqpoint{0.468269in}{4.045754in}}{\pgfqpoint{0.472659in}{4.056353in}}{\pgfqpoint{0.472659in}{4.067403in}}%
\pgfpathcurveto{\pgfqpoint{0.472659in}{4.078453in}}{\pgfqpoint{0.468269in}{4.089052in}}{\pgfqpoint{0.460455in}{4.096866in}}%
\pgfpathcurveto{\pgfqpoint{0.452642in}{4.104679in}}{\pgfqpoint{0.442043in}{4.109070in}}{\pgfqpoint{0.430992in}{4.109070in}}%
\pgfpathcurveto{\pgfqpoint{0.419942in}{4.109070in}}{\pgfqpoint{0.409343in}{4.104679in}}{\pgfqpoint{0.401530in}{4.096866in}}%
\pgfpathcurveto{\pgfqpoint{0.393716in}{4.089052in}}{\pgfqpoint{0.389326in}{4.078453in}}{\pgfqpoint{0.389326in}{4.067403in}}%
\pgfpathcurveto{\pgfqpoint{0.389326in}{4.056353in}}{\pgfqpoint{0.393716in}{4.045754in}}{\pgfqpoint{0.401530in}{4.037940in}}%
\pgfpathcurveto{\pgfqpoint{0.409343in}{4.030127in}}{\pgfqpoint{0.419942in}{4.025736in}}{\pgfqpoint{0.430992in}{4.025736in}}%
\pgfpathlineto{\pgfqpoint{0.430992in}{4.025736in}}%
\pgfpathclose%
\pgfusepath{stroke}%
\end{pgfscope}%
\begin{pgfscope}%
\pgfpathrectangle{\pgfqpoint{0.393053in}{0.375000in}}{\pgfqpoint{6.356833in}{5.175000in}}%
\pgfusepath{clip}%
\pgfsetbuttcap%
\pgfsetroundjoin%
\pgfsetlinewidth{1.003750pt}%
\definecolor{currentstroke}{rgb}{0.827451,0.827451,0.827451}%
\pgfsetstrokecolor{currentstroke}%
\pgfsetdash{}{0pt}%
\pgfpathmoveto{\pgfqpoint{2.905386in}{0.884940in}}%
\pgfpathcurveto{\pgfqpoint{2.916436in}{0.884940in}}{\pgfqpoint{2.927035in}{0.889330in}}{\pgfqpoint{2.934849in}{0.897144in}}%
\pgfpathcurveto{\pgfqpoint{2.942663in}{0.904957in}}{\pgfqpoint{2.947053in}{0.915556in}}{\pgfqpoint{2.947053in}{0.926606in}}%
\pgfpathcurveto{\pgfqpoint{2.947053in}{0.937657in}}{\pgfqpoint{2.942663in}{0.948256in}}{\pgfqpoint{2.934849in}{0.956069in}}%
\pgfpathcurveto{\pgfqpoint{2.927035in}{0.963883in}}{\pgfqpoint{2.916436in}{0.968273in}}{\pgfqpoint{2.905386in}{0.968273in}}%
\pgfpathcurveto{\pgfqpoint{2.894336in}{0.968273in}}{\pgfqpoint{2.883737in}{0.963883in}}{\pgfqpoint{2.875923in}{0.956069in}}%
\pgfpathcurveto{\pgfqpoint{2.868110in}{0.948256in}}{\pgfqpoint{2.863720in}{0.937657in}}{\pgfqpoint{2.863720in}{0.926606in}}%
\pgfpathcurveto{\pgfqpoint{2.863720in}{0.915556in}}{\pgfqpoint{2.868110in}{0.904957in}}{\pgfqpoint{2.875923in}{0.897144in}}%
\pgfpathcurveto{\pgfqpoint{2.883737in}{0.889330in}}{\pgfqpoint{2.894336in}{0.884940in}}{\pgfqpoint{2.905386in}{0.884940in}}%
\pgfpathlineto{\pgfqpoint{2.905386in}{0.884940in}}%
\pgfpathclose%
\pgfusepath{stroke}%
\end{pgfscope}%
\begin{pgfscope}%
\pgfpathrectangle{\pgfqpoint{0.393053in}{0.375000in}}{\pgfqpoint{6.356833in}{5.175000in}}%
\pgfusepath{clip}%
\pgfsetbuttcap%
\pgfsetroundjoin%
\pgfsetlinewidth{1.003750pt}%
\definecolor{currentstroke}{rgb}{0.827451,0.827451,0.827451}%
\pgfsetstrokecolor{currentstroke}%
\pgfsetdash{}{0pt}%
\pgfpathmoveto{\pgfqpoint{4.173270in}{0.457379in}}%
\pgfpathcurveto{\pgfqpoint{4.184320in}{0.457379in}}{\pgfqpoint{4.194919in}{0.461770in}}{\pgfqpoint{4.202733in}{0.469583in}}%
\pgfpathcurveto{\pgfqpoint{4.210547in}{0.477397in}}{\pgfqpoint{4.214937in}{0.487996in}}{\pgfqpoint{4.214937in}{0.499046in}}%
\pgfpathcurveto{\pgfqpoint{4.214937in}{0.510096in}}{\pgfqpoint{4.210547in}{0.520695in}}{\pgfqpoint{4.202733in}{0.528509in}}%
\pgfpathcurveto{\pgfqpoint{4.194919in}{0.536322in}}{\pgfqpoint{4.184320in}{0.540713in}}{\pgfqpoint{4.173270in}{0.540713in}}%
\pgfpathcurveto{\pgfqpoint{4.162220in}{0.540713in}}{\pgfqpoint{4.151621in}{0.536322in}}{\pgfqpoint{4.143807in}{0.528509in}}%
\pgfpathcurveto{\pgfqpoint{4.135994in}{0.520695in}}{\pgfqpoint{4.131604in}{0.510096in}}{\pgfqpoint{4.131604in}{0.499046in}}%
\pgfpathcurveto{\pgfqpoint{4.131604in}{0.487996in}}{\pgfqpoint{4.135994in}{0.477397in}}{\pgfqpoint{4.143807in}{0.469583in}}%
\pgfpathcurveto{\pgfqpoint{4.151621in}{0.461770in}}{\pgfqpoint{4.162220in}{0.457379in}}{\pgfqpoint{4.173270in}{0.457379in}}%
\pgfpathlineto{\pgfqpoint{4.173270in}{0.457379in}}%
\pgfpathclose%
\pgfusepath{stroke}%
\end{pgfscope}%
\begin{pgfscope}%
\pgfpathrectangle{\pgfqpoint{0.393053in}{0.375000in}}{\pgfqpoint{6.356833in}{5.175000in}}%
\pgfusepath{clip}%
\pgfsetbuttcap%
\pgfsetroundjoin%
\pgfsetlinewidth{1.003750pt}%
\definecolor{currentstroke}{rgb}{0.827451,0.827451,0.827451}%
\pgfsetstrokecolor{currentstroke}%
\pgfsetdash{}{0pt}%
\pgfpathmoveto{\pgfqpoint{0.435142in}{3.936498in}}%
\pgfpathcurveto{\pgfqpoint{0.446192in}{3.936498in}}{\pgfqpoint{0.456791in}{3.940888in}}{\pgfqpoint{0.464605in}{3.948702in}}%
\pgfpathcurveto{\pgfqpoint{0.472419in}{3.956516in}}{\pgfqpoint{0.476809in}{3.967115in}}{\pgfqpoint{0.476809in}{3.978165in}}%
\pgfpathcurveto{\pgfqpoint{0.476809in}{3.989215in}}{\pgfqpoint{0.472419in}{3.999814in}}{\pgfqpoint{0.464605in}{4.007627in}}%
\pgfpathcurveto{\pgfqpoint{0.456791in}{4.015441in}}{\pgfqpoint{0.446192in}{4.019831in}}{\pgfqpoint{0.435142in}{4.019831in}}%
\pgfpathcurveto{\pgfqpoint{0.424092in}{4.019831in}}{\pgfqpoint{0.413493in}{4.015441in}}{\pgfqpoint{0.405679in}{4.007627in}}%
\pgfpathcurveto{\pgfqpoint{0.397866in}{3.999814in}}{\pgfqpoint{0.393475in}{3.989215in}}{\pgfqpoint{0.393475in}{3.978165in}}%
\pgfpathcurveto{\pgfqpoint{0.393475in}{3.967115in}}{\pgfqpoint{0.397866in}{3.956516in}}{\pgfqpoint{0.405679in}{3.948702in}}%
\pgfpathcurveto{\pgfqpoint{0.413493in}{3.940888in}}{\pgfqpoint{0.424092in}{3.936498in}}{\pgfqpoint{0.435142in}{3.936498in}}%
\pgfpathlineto{\pgfqpoint{0.435142in}{3.936498in}}%
\pgfpathclose%
\pgfusepath{stroke}%
\end{pgfscope}%
\begin{pgfscope}%
\pgfpathrectangle{\pgfqpoint{0.393053in}{0.375000in}}{\pgfqpoint{6.356833in}{5.175000in}}%
\pgfusepath{clip}%
\pgfsetbuttcap%
\pgfsetroundjoin%
\pgfsetlinewidth{1.003750pt}%
\definecolor{currentstroke}{rgb}{0.827451,0.827451,0.827451}%
\pgfsetstrokecolor{currentstroke}%
\pgfsetdash{}{0pt}%
\pgfpathmoveto{\pgfqpoint{2.514644in}{1.050767in}}%
\pgfpathcurveto{\pgfqpoint{2.525694in}{1.050767in}}{\pgfqpoint{2.536293in}{1.055158in}}{\pgfqpoint{2.544106in}{1.062971in}}%
\pgfpathcurveto{\pgfqpoint{2.551920in}{1.070785in}}{\pgfqpoint{2.556310in}{1.081384in}}{\pgfqpoint{2.556310in}{1.092434in}}%
\pgfpathcurveto{\pgfqpoint{2.556310in}{1.103484in}}{\pgfqpoint{2.551920in}{1.114083in}}{\pgfqpoint{2.544106in}{1.121897in}}%
\pgfpathcurveto{\pgfqpoint{2.536293in}{1.129711in}}{\pgfqpoint{2.525694in}{1.134101in}}{\pgfqpoint{2.514644in}{1.134101in}}%
\pgfpathcurveto{\pgfqpoint{2.503593in}{1.134101in}}{\pgfqpoint{2.492994in}{1.129711in}}{\pgfqpoint{2.485181in}{1.121897in}}%
\pgfpathcurveto{\pgfqpoint{2.477367in}{1.114083in}}{\pgfqpoint{2.472977in}{1.103484in}}{\pgfqpoint{2.472977in}{1.092434in}}%
\pgfpathcurveto{\pgfqpoint{2.472977in}{1.081384in}}{\pgfqpoint{2.477367in}{1.070785in}}{\pgfqpoint{2.485181in}{1.062971in}}%
\pgfpathcurveto{\pgfqpoint{2.492994in}{1.055158in}}{\pgfqpoint{2.503593in}{1.050767in}}{\pgfqpoint{2.514644in}{1.050767in}}%
\pgfpathlineto{\pgfqpoint{2.514644in}{1.050767in}}%
\pgfpathclose%
\pgfusepath{stroke}%
\end{pgfscope}%
\begin{pgfscope}%
\pgfpathrectangle{\pgfqpoint{0.393053in}{0.375000in}}{\pgfqpoint{6.356833in}{5.175000in}}%
\pgfusepath{clip}%
\pgfsetbuttcap%
\pgfsetroundjoin%
\pgfsetlinewidth{1.003750pt}%
\definecolor{currentstroke}{rgb}{0.827451,0.827451,0.827451}%
\pgfsetstrokecolor{currentstroke}%
\pgfsetdash{}{0pt}%
\pgfpathmoveto{\pgfqpoint{3.694432in}{0.605429in}}%
\pgfpathcurveto{\pgfqpoint{3.705482in}{0.605429in}}{\pgfqpoint{3.716081in}{0.609820in}}{\pgfqpoint{3.723895in}{0.617633in}}%
\pgfpathcurveto{\pgfqpoint{3.731708in}{0.625447in}}{\pgfqpoint{3.736098in}{0.636046in}}{\pgfqpoint{3.736098in}{0.647096in}}%
\pgfpathcurveto{\pgfqpoint{3.736098in}{0.658146in}}{\pgfqpoint{3.731708in}{0.668745in}}{\pgfqpoint{3.723895in}{0.676559in}}%
\pgfpathcurveto{\pgfqpoint{3.716081in}{0.684372in}}{\pgfqpoint{3.705482in}{0.688763in}}{\pgfqpoint{3.694432in}{0.688763in}}%
\pgfpathcurveto{\pgfqpoint{3.683382in}{0.688763in}}{\pgfqpoint{3.672783in}{0.684372in}}{\pgfqpoint{3.664969in}{0.676559in}}%
\pgfpathcurveto{\pgfqpoint{3.657155in}{0.668745in}}{\pgfqpoint{3.652765in}{0.658146in}}{\pgfqpoint{3.652765in}{0.647096in}}%
\pgfpathcurveto{\pgfqpoint{3.652765in}{0.636046in}}{\pgfqpoint{3.657155in}{0.625447in}}{\pgfqpoint{3.664969in}{0.617633in}}%
\pgfpathcurveto{\pgfqpoint{3.672783in}{0.609820in}}{\pgfqpoint{3.683382in}{0.605429in}}{\pgfqpoint{3.694432in}{0.605429in}}%
\pgfpathlineto{\pgfqpoint{3.694432in}{0.605429in}}%
\pgfpathclose%
\pgfusepath{stroke}%
\end{pgfscope}%
\begin{pgfscope}%
\pgfpathrectangle{\pgfqpoint{0.393053in}{0.375000in}}{\pgfqpoint{6.356833in}{5.175000in}}%
\pgfusepath{clip}%
\pgfsetbuttcap%
\pgfsetroundjoin%
\pgfsetlinewidth{1.003750pt}%
\definecolor{currentstroke}{rgb}{0.827451,0.827451,0.827451}%
\pgfsetstrokecolor{currentstroke}%
\pgfsetdash{}{0pt}%
\pgfpathmoveto{\pgfqpoint{2.145018in}{1.265722in}}%
\pgfpathcurveto{\pgfqpoint{2.156069in}{1.265722in}}{\pgfqpoint{2.166668in}{1.270112in}}{\pgfqpoint{2.174481in}{1.277926in}}%
\pgfpathcurveto{\pgfqpoint{2.182295in}{1.285739in}}{\pgfqpoint{2.186685in}{1.296338in}}{\pgfqpoint{2.186685in}{1.307389in}}%
\pgfpathcurveto{\pgfqpoint{2.186685in}{1.318439in}}{\pgfqpoint{2.182295in}{1.329038in}}{\pgfqpoint{2.174481in}{1.336851in}}%
\pgfpathcurveto{\pgfqpoint{2.166668in}{1.344665in}}{\pgfqpoint{2.156069in}{1.349055in}}{\pgfqpoint{2.145018in}{1.349055in}}%
\pgfpathcurveto{\pgfqpoint{2.133968in}{1.349055in}}{\pgfqpoint{2.123369in}{1.344665in}}{\pgfqpoint{2.115556in}{1.336851in}}%
\pgfpathcurveto{\pgfqpoint{2.107742in}{1.329038in}}{\pgfqpoint{2.103352in}{1.318439in}}{\pgfqpoint{2.103352in}{1.307389in}}%
\pgfpathcurveto{\pgfqpoint{2.103352in}{1.296338in}}{\pgfqpoint{2.107742in}{1.285739in}}{\pgfqpoint{2.115556in}{1.277926in}}%
\pgfpathcurveto{\pgfqpoint{2.123369in}{1.270112in}}{\pgfqpoint{2.133968in}{1.265722in}}{\pgfqpoint{2.145018in}{1.265722in}}%
\pgfpathlineto{\pgfqpoint{2.145018in}{1.265722in}}%
\pgfpathclose%
\pgfusepath{stroke}%
\end{pgfscope}%
\begin{pgfscope}%
\pgfpathrectangle{\pgfqpoint{0.393053in}{0.375000in}}{\pgfqpoint{6.356833in}{5.175000in}}%
\pgfusepath{clip}%
\pgfsetbuttcap%
\pgfsetroundjoin%
\pgfsetlinewidth{1.003750pt}%
\definecolor{currentstroke}{rgb}{0.827451,0.827451,0.827451}%
\pgfsetstrokecolor{currentstroke}%
\pgfsetdash{}{0pt}%
\pgfpathmoveto{\pgfqpoint{5.017789in}{0.367635in}}%
\pgfpathcurveto{\pgfqpoint{5.028839in}{0.367635in}}{\pgfqpoint{5.039438in}{0.372025in}}{\pgfqpoint{5.047252in}{0.379839in}}%
\pgfpathcurveto{\pgfqpoint{5.055065in}{0.387652in}}{\pgfqpoint{5.059456in}{0.398251in}}{\pgfqpoint{5.059456in}{0.409302in}}%
\pgfpathcurveto{\pgfqpoint{5.059456in}{0.420352in}}{\pgfqpoint{5.055065in}{0.430951in}}{\pgfqpoint{5.047252in}{0.438764in}}%
\pgfpathcurveto{\pgfqpoint{5.039438in}{0.446578in}}{\pgfqpoint{5.028839in}{0.450968in}}{\pgfqpoint{5.017789in}{0.450968in}}%
\pgfpathcurveto{\pgfqpoint{5.006739in}{0.450968in}}{\pgfqpoint{4.996140in}{0.446578in}}{\pgfqpoint{4.988326in}{0.438764in}}%
\pgfpathcurveto{\pgfqpoint{4.980512in}{0.430951in}}{\pgfqpoint{4.976122in}{0.420352in}}{\pgfqpoint{4.976122in}{0.409302in}}%
\pgfpathcurveto{\pgfqpoint{4.976122in}{0.398251in}}{\pgfqpoint{4.980512in}{0.387652in}}{\pgfqpoint{4.988326in}{0.379839in}}%
\pgfpathcurveto{\pgfqpoint{4.996140in}{0.372025in}}{\pgfqpoint{5.006739in}{0.367635in}}{\pgfqpoint{5.017789in}{0.367635in}}%
\pgfusepath{stroke}%
\end{pgfscope}%
\begin{pgfscope}%
\pgfpathrectangle{\pgfqpoint{0.393053in}{0.375000in}}{\pgfqpoint{6.356833in}{5.175000in}}%
\pgfusepath{clip}%
\pgfsetbuttcap%
\pgfsetroundjoin%
\pgfsetlinewidth{1.003750pt}%
\definecolor{currentstroke}{rgb}{0.827451,0.827451,0.827451}%
\pgfsetstrokecolor{currentstroke}%
\pgfsetdash{}{0pt}%
\pgfpathmoveto{\pgfqpoint{1.123487in}{2.207492in}}%
\pgfpathcurveto{\pgfqpoint{1.134537in}{2.207492in}}{\pgfqpoint{1.145136in}{2.211882in}}{\pgfqpoint{1.152950in}{2.219696in}}%
\pgfpathcurveto{\pgfqpoint{1.160763in}{2.227510in}}{\pgfqpoint{1.165153in}{2.238109in}}{\pgfqpoint{1.165153in}{2.249159in}}%
\pgfpathcurveto{\pgfqpoint{1.165153in}{2.260209in}}{\pgfqpoint{1.160763in}{2.270808in}}{\pgfqpoint{1.152950in}{2.278622in}}%
\pgfpathcurveto{\pgfqpoint{1.145136in}{2.286435in}}{\pgfqpoint{1.134537in}{2.290825in}}{\pgfqpoint{1.123487in}{2.290825in}}%
\pgfpathcurveto{\pgfqpoint{1.112437in}{2.290825in}}{\pgfqpoint{1.101838in}{2.286435in}}{\pgfqpoint{1.094024in}{2.278622in}}%
\pgfpathcurveto{\pgfqpoint{1.086210in}{2.270808in}}{\pgfqpoint{1.081820in}{2.260209in}}{\pgfqpoint{1.081820in}{2.249159in}}%
\pgfpathcurveto{\pgfqpoint{1.081820in}{2.238109in}}{\pgfqpoint{1.086210in}{2.227510in}}{\pgfqpoint{1.094024in}{2.219696in}}%
\pgfpathcurveto{\pgfqpoint{1.101838in}{2.211882in}}{\pgfqpoint{1.112437in}{2.207492in}}{\pgfqpoint{1.123487in}{2.207492in}}%
\pgfpathlineto{\pgfqpoint{1.123487in}{2.207492in}}%
\pgfpathclose%
\pgfusepath{stroke}%
\end{pgfscope}%
\begin{pgfscope}%
\pgfpathrectangle{\pgfqpoint{0.393053in}{0.375000in}}{\pgfqpoint{6.356833in}{5.175000in}}%
\pgfusepath{clip}%
\pgfsetbuttcap%
\pgfsetroundjoin%
\pgfsetlinewidth{1.003750pt}%
\definecolor{currentstroke}{rgb}{0.827451,0.827451,0.827451}%
\pgfsetstrokecolor{currentstroke}%
\pgfsetdash{}{0pt}%
\pgfpathmoveto{\pgfqpoint{2.043944in}{1.333426in}}%
\pgfpathcurveto{\pgfqpoint{2.054995in}{1.333426in}}{\pgfqpoint{2.065594in}{1.337816in}}{\pgfqpoint{2.073407in}{1.345629in}}%
\pgfpathcurveto{\pgfqpoint{2.081221in}{1.353443in}}{\pgfqpoint{2.085611in}{1.364042in}}{\pgfqpoint{2.085611in}{1.375092in}}%
\pgfpathcurveto{\pgfqpoint{2.085611in}{1.386142in}}{\pgfqpoint{2.081221in}{1.396741in}}{\pgfqpoint{2.073407in}{1.404555in}}%
\pgfpathcurveto{\pgfqpoint{2.065594in}{1.412369in}}{\pgfqpoint{2.054995in}{1.416759in}}{\pgfqpoint{2.043944in}{1.416759in}}%
\pgfpathcurveto{\pgfqpoint{2.032894in}{1.416759in}}{\pgfqpoint{2.022295in}{1.412369in}}{\pgfqpoint{2.014482in}{1.404555in}}%
\pgfpathcurveto{\pgfqpoint{2.006668in}{1.396741in}}{\pgfqpoint{2.002278in}{1.386142in}}{\pgfqpoint{2.002278in}{1.375092in}}%
\pgfpathcurveto{\pgfqpoint{2.002278in}{1.364042in}}{\pgfqpoint{2.006668in}{1.353443in}}{\pgfqpoint{2.014482in}{1.345629in}}%
\pgfpathcurveto{\pgfqpoint{2.022295in}{1.337816in}}{\pgfqpoint{2.032894in}{1.333426in}}{\pgfqpoint{2.043944in}{1.333426in}}%
\pgfpathlineto{\pgfqpoint{2.043944in}{1.333426in}}%
\pgfpathclose%
\pgfusepath{stroke}%
\end{pgfscope}%
\begin{pgfscope}%
\pgfpathrectangle{\pgfqpoint{0.393053in}{0.375000in}}{\pgfqpoint{6.356833in}{5.175000in}}%
\pgfusepath{clip}%
\pgfsetbuttcap%
\pgfsetroundjoin%
\pgfsetlinewidth{1.003750pt}%
\definecolor{currentstroke}{rgb}{0.827451,0.827451,0.827451}%
\pgfsetstrokecolor{currentstroke}%
\pgfsetdash{}{0pt}%
\pgfpathmoveto{\pgfqpoint{0.527354in}{3.462173in}}%
\pgfpathcurveto{\pgfqpoint{0.538404in}{3.462173in}}{\pgfqpoint{0.549003in}{3.466563in}}{\pgfqpoint{0.556817in}{3.474377in}}%
\pgfpathcurveto{\pgfqpoint{0.564631in}{3.482190in}}{\pgfqpoint{0.569021in}{3.492790in}}{\pgfqpoint{0.569021in}{3.503840in}}%
\pgfpathcurveto{\pgfqpoint{0.569021in}{3.514890in}}{\pgfqpoint{0.564631in}{3.525489in}}{\pgfqpoint{0.556817in}{3.533302in}}%
\pgfpathcurveto{\pgfqpoint{0.549003in}{3.541116in}}{\pgfqpoint{0.538404in}{3.545506in}}{\pgfqpoint{0.527354in}{3.545506in}}%
\pgfpathcurveto{\pgfqpoint{0.516304in}{3.545506in}}{\pgfqpoint{0.505705in}{3.541116in}}{\pgfqpoint{0.497891in}{3.533302in}}%
\pgfpathcurveto{\pgfqpoint{0.490078in}{3.525489in}}{\pgfqpoint{0.485687in}{3.514890in}}{\pgfqpoint{0.485687in}{3.503840in}}%
\pgfpathcurveto{\pgfqpoint{0.485687in}{3.492790in}}{\pgfqpoint{0.490078in}{3.482190in}}{\pgfqpoint{0.497891in}{3.474377in}}%
\pgfpathcurveto{\pgfqpoint{0.505705in}{3.466563in}}{\pgfqpoint{0.516304in}{3.462173in}}{\pgfqpoint{0.527354in}{3.462173in}}%
\pgfpathlineto{\pgfqpoint{0.527354in}{3.462173in}}%
\pgfpathclose%
\pgfusepath{stroke}%
\end{pgfscope}%
\begin{pgfscope}%
\pgfpathrectangle{\pgfqpoint{0.393053in}{0.375000in}}{\pgfqpoint{6.356833in}{5.175000in}}%
\pgfusepath{clip}%
\pgfsetbuttcap%
\pgfsetroundjoin%
\pgfsetlinewidth{1.003750pt}%
\definecolor{currentstroke}{rgb}{0.827451,0.827451,0.827451}%
\pgfsetstrokecolor{currentstroke}%
\pgfsetdash{}{0pt}%
\pgfpathmoveto{\pgfqpoint{1.614675in}{1.708400in}}%
\pgfpathcurveto{\pgfqpoint{1.625725in}{1.708400in}}{\pgfqpoint{1.636324in}{1.712790in}}{\pgfqpoint{1.644137in}{1.720604in}}%
\pgfpathcurveto{\pgfqpoint{1.651951in}{1.728417in}}{\pgfqpoint{1.656341in}{1.739016in}}{\pgfqpoint{1.656341in}{1.750066in}}%
\pgfpathcurveto{\pgfqpoint{1.656341in}{1.761117in}}{\pgfqpoint{1.651951in}{1.771716in}}{\pgfqpoint{1.644137in}{1.779529in}}%
\pgfpathcurveto{\pgfqpoint{1.636324in}{1.787343in}}{\pgfqpoint{1.625725in}{1.791733in}}{\pgfqpoint{1.614675in}{1.791733in}}%
\pgfpathcurveto{\pgfqpoint{1.603625in}{1.791733in}}{\pgfqpoint{1.593025in}{1.787343in}}{\pgfqpoint{1.585212in}{1.779529in}}%
\pgfpathcurveto{\pgfqpoint{1.577398in}{1.771716in}}{\pgfqpoint{1.573008in}{1.761117in}}{\pgfqpoint{1.573008in}{1.750066in}}%
\pgfpathcurveto{\pgfqpoint{1.573008in}{1.739016in}}{\pgfqpoint{1.577398in}{1.728417in}}{\pgfqpoint{1.585212in}{1.720604in}}%
\pgfpathcurveto{\pgfqpoint{1.593025in}{1.712790in}}{\pgfqpoint{1.603625in}{1.708400in}}{\pgfqpoint{1.614675in}{1.708400in}}%
\pgfpathlineto{\pgfqpoint{1.614675in}{1.708400in}}%
\pgfpathclose%
\pgfusepath{stroke}%
\end{pgfscope}%
\begin{pgfscope}%
\pgfpathrectangle{\pgfqpoint{0.393053in}{0.375000in}}{\pgfqpoint{6.356833in}{5.175000in}}%
\pgfusepath{clip}%
\pgfsetbuttcap%
\pgfsetroundjoin%
\pgfsetlinewidth{1.003750pt}%
\definecolor{currentstroke}{rgb}{0.827451,0.827451,0.827451}%
\pgfsetstrokecolor{currentstroke}%
\pgfsetdash{}{0pt}%
\pgfpathmoveto{\pgfqpoint{1.882270in}{1.458195in}}%
\pgfpathcurveto{\pgfqpoint{1.893321in}{1.458195in}}{\pgfqpoint{1.903920in}{1.462585in}}{\pgfqpoint{1.911733in}{1.470399in}}%
\pgfpathcurveto{\pgfqpoint{1.919547in}{1.478213in}}{\pgfqpoint{1.923937in}{1.488812in}}{\pgfqpoint{1.923937in}{1.499862in}}%
\pgfpathcurveto{\pgfqpoint{1.923937in}{1.510912in}}{\pgfqpoint{1.919547in}{1.521511in}}{\pgfqpoint{1.911733in}{1.529325in}}%
\pgfpathcurveto{\pgfqpoint{1.903920in}{1.537138in}}{\pgfqpoint{1.893321in}{1.541529in}}{\pgfqpoint{1.882270in}{1.541529in}}%
\pgfpathcurveto{\pgfqpoint{1.871220in}{1.541529in}}{\pgfqpoint{1.860621in}{1.537138in}}{\pgfqpoint{1.852808in}{1.529325in}}%
\pgfpathcurveto{\pgfqpoint{1.844994in}{1.521511in}}{\pgfqpoint{1.840604in}{1.510912in}}{\pgfqpoint{1.840604in}{1.499862in}}%
\pgfpathcurveto{\pgfqpoint{1.840604in}{1.488812in}}{\pgfqpoint{1.844994in}{1.478213in}}{\pgfqpoint{1.852808in}{1.470399in}}%
\pgfpathcurveto{\pgfqpoint{1.860621in}{1.462585in}}{\pgfqpoint{1.871220in}{1.458195in}}{\pgfqpoint{1.882270in}{1.458195in}}%
\pgfpathlineto{\pgfqpoint{1.882270in}{1.458195in}}%
\pgfpathclose%
\pgfusepath{stroke}%
\end{pgfscope}%
\begin{pgfscope}%
\pgfpathrectangle{\pgfqpoint{0.393053in}{0.375000in}}{\pgfqpoint{6.356833in}{5.175000in}}%
\pgfusepath{clip}%
\pgfsetbuttcap%
\pgfsetroundjoin%
\pgfsetlinewidth{1.003750pt}%
\definecolor{currentstroke}{rgb}{0.827451,0.827451,0.827451}%
\pgfsetstrokecolor{currentstroke}%
\pgfsetdash{}{0pt}%
\pgfpathmoveto{\pgfqpoint{0.997743in}{2.491784in}}%
\pgfpathcurveto{\pgfqpoint{1.008793in}{2.491784in}}{\pgfqpoint{1.019392in}{2.496174in}}{\pgfqpoint{1.027206in}{2.503988in}}%
\pgfpathcurveto{\pgfqpoint{1.035019in}{2.511802in}}{\pgfqpoint{1.039410in}{2.522401in}}{\pgfqpoint{1.039410in}{2.533451in}}%
\pgfpathcurveto{\pgfqpoint{1.039410in}{2.544501in}}{\pgfqpoint{1.035019in}{2.555100in}}{\pgfqpoint{1.027206in}{2.562914in}}%
\pgfpathcurveto{\pgfqpoint{1.019392in}{2.570727in}}{\pgfqpoint{1.008793in}{2.575118in}}{\pgfqpoint{0.997743in}{2.575118in}}%
\pgfpathcurveto{\pgfqpoint{0.986693in}{2.575118in}}{\pgfqpoint{0.976094in}{2.570727in}}{\pgfqpoint{0.968280in}{2.562914in}}%
\pgfpathcurveto{\pgfqpoint{0.960466in}{2.555100in}}{\pgfqpoint{0.956076in}{2.544501in}}{\pgfqpoint{0.956076in}{2.533451in}}%
\pgfpathcurveto{\pgfqpoint{0.956076in}{2.522401in}}{\pgfqpoint{0.960466in}{2.511802in}}{\pgfqpoint{0.968280in}{2.503988in}}%
\pgfpathcurveto{\pgfqpoint{0.976094in}{2.496174in}}{\pgfqpoint{0.986693in}{2.491784in}}{\pgfqpoint{0.997743in}{2.491784in}}%
\pgfpathlineto{\pgfqpoint{0.997743in}{2.491784in}}%
\pgfpathclose%
\pgfusepath{stroke}%
\end{pgfscope}%
\begin{pgfscope}%
\pgfpathrectangle{\pgfqpoint{0.393053in}{0.375000in}}{\pgfqpoint{6.356833in}{5.175000in}}%
\pgfusepath{clip}%
\pgfsetbuttcap%
\pgfsetroundjoin%
\pgfsetlinewidth{1.003750pt}%
\definecolor{currentstroke}{rgb}{0.827451,0.827451,0.827451}%
\pgfsetstrokecolor{currentstroke}%
\pgfsetdash{}{0pt}%
\pgfpathmoveto{\pgfqpoint{0.423946in}{4.069027in}}%
\pgfpathcurveto{\pgfqpoint{0.434996in}{4.069027in}}{\pgfqpoint{0.445595in}{4.073417in}}{\pgfqpoint{0.453408in}{4.081231in}}%
\pgfpathcurveto{\pgfqpoint{0.461222in}{4.089045in}}{\pgfqpoint{0.465612in}{4.099644in}}{\pgfqpoint{0.465612in}{4.110694in}}%
\pgfpathcurveto{\pgfqpoint{0.465612in}{4.121744in}}{\pgfqpoint{0.461222in}{4.132343in}}{\pgfqpoint{0.453408in}{4.140157in}}%
\pgfpathcurveto{\pgfqpoint{0.445595in}{4.147970in}}{\pgfqpoint{0.434996in}{4.152361in}}{\pgfqpoint{0.423946in}{4.152361in}}%
\pgfpathcurveto{\pgfqpoint{0.412896in}{4.152361in}}{\pgfqpoint{0.402297in}{4.147970in}}{\pgfqpoint{0.394483in}{4.140157in}}%
\pgfpathcurveto{\pgfqpoint{0.386669in}{4.132343in}}{\pgfqpoint{0.382279in}{4.121744in}}{\pgfqpoint{0.382279in}{4.110694in}}%
\pgfpathcurveto{\pgfqpoint{0.382279in}{4.099644in}}{\pgfqpoint{0.386669in}{4.089045in}}{\pgfqpoint{0.394483in}{4.081231in}}%
\pgfpathcurveto{\pgfqpoint{0.402297in}{4.073417in}}{\pgfqpoint{0.412896in}{4.069027in}}{\pgfqpoint{0.423946in}{4.069027in}}%
\pgfpathlineto{\pgfqpoint{0.423946in}{4.069027in}}%
\pgfpathclose%
\pgfusepath{stroke}%
\end{pgfscope}%
\begin{pgfscope}%
\pgfpathrectangle{\pgfqpoint{0.393053in}{0.375000in}}{\pgfqpoint{6.356833in}{5.175000in}}%
\pgfusepath{clip}%
\pgfsetbuttcap%
\pgfsetroundjoin%
\pgfsetlinewidth{1.003750pt}%
\definecolor{currentstroke}{rgb}{0.827451,0.827451,0.827451}%
\pgfsetstrokecolor{currentstroke}%
\pgfsetdash{}{0pt}%
\pgfpathmoveto{\pgfqpoint{1.256856in}{2.039646in}}%
\pgfpathcurveto{\pgfqpoint{1.267906in}{2.039646in}}{\pgfqpoint{1.278505in}{2.044036in}}{\pgfqpoint{1.286319in}{2.051849in}}%
\pgfpathcurveto{\pgfqpoint{1.294132in}{2.059663in}}{\pgfqpoint{1.298523in}{2.070262in}}{\pgfqpoint{1.298523in}{2.081312in}}%
\pgfpathcurveto{\pgfqpoint{1.298523in}{2.092362in}}{\pgfqpoint{1.294132in}{2.102961in}}{\pgfqpoint{1.286319in}{2.110775in}}%
\pgfpathcurveto{\pgfqpoint{1.278505in}{2.118589in}}{\pgfqpoint{1.267906in}{2.122979in}}{\pgfqpoint{1.256856in}{2.122979in}}%
\pgfpathcurveto{\pgfqpoint{1.245806in}{2.122979in}}{\pgfqpoint{1.235207in}{2.118589in}}{\pgfqpoint{1.227393in}{2.110775in}}%
\pgfpathcurveto{\pgfqpoint{1.219580in}{2.102961in}}{\pgfqpoint{1.215189in}{2.092362in}}{\pgfqpoint{1.215189in}{2.081312in}}%
\pgfpathcurveto{\pgfqpoint{1.215189in}{2.070262in}}{\pgfqpoint{1.219580in}{2.059663in}}{\pgfqpoint{1.227393in}{2.051849in}}%
\pgfpathcurveto{\pgfqpoint{1.235207in}{2.044036in}}{\pgfqpoint{1.245806in}{2.039646in}}{\pgfqpoint{1.256856in}{2.039646in}}%
\pgfpathlineto{\pgfqpoint{1.256856in}{2.039646in}}%
\pgfpathclose%
\pgfusepath{stroke}%
\end{pgfscope}%
\begin{pgfscope}%
\pgfpathrectangle{\pgfqpoint{0.393053in}{0.375000in}}{\pgfqpoint{6.356833in}{5.175000in}}%
\pgfusepath{clip}%
\pgfsetbuttcap%
\pgfsetroundjoin%
\pgfsetlinewidth{1.003750pt}%
\definecolor{currentstroke}{rgb}{0.827451,0.827451,0.827451}%
\pgfsetstrokecolor{currentstroke}%
\pgfsetdash{}{0pt}%
\pgfpathmoveto{\pgfqpoint{2.477308in}{1.087549in}}%
\pgfpathcurveto{\pgfqpoint{2.488358in}{1.087549in}}{\pgfqpoint{2.498957in}{1.091940in}}{\pgfqpoint{2.506771in}{1.099753in}}%
\pgfpathcurveto{\pgfqpoint{2.514584in}{1.107567in}}{\pgfqpoint{2.518974in}{1.118166in}}{\pgfqpoint{2.518974in}{1.129216in}}%
\pgfpathcurveto{\pgfqpoint{2.518974in}{1.140266in}}{\pgfqpoint{2.514584in}{1.150865in}}{\pgfqpoint{2.506771in}{1.158679in}}%
\pgfpathcurveto{\pgfqpoint{2.498957in}{1.166492in}}{\pgfqpoint{2.488358in}{1.170883in}}{\pgfqpoint{2.477308in}{1.170883in}}%
\pgfpathcurveto{\pgfqpoint{2.466258in}{1.170883in}}{\pgfqpoint{2.455659in}{1.166492in}}{\pgfqpoint{2.447845in}{1.158679in}}%
\pgfpathcurveto{\pgfqpoint{2.440031in}{1.150865in}}{\pgfqpoint{2.435641in}{1.140266in}}{\pgfqpoint{2.435641in}{1.129216in}}%
\pgfpathcurveto{\pgfqpoint{2.435641in}{1.118166in}}{\pgfqpoint{2.440031in}{1.107567in}}{\pgfqpoint{2.447845in}{1.099753in}}%
\pgfpathcurveto{\pgfqpoint{2.455659in}{1.091940in}}{\pgfqpoint{2.466258in}{1.087549in}}{\pgfqpoint{2.477308in}{1.087549in}}%
\pgfpathlineto{\pgfqpoint{2.477308in}{1.087549in}}%
\pgfpathclose%
\pgfusepath{stroke}%
\end{pgfscope}%
\begin{pgfscope}%
\pgfpathrectangle{\pgfqpoint{0.393053in}{0.375000in}}{\pgfqpoint{6.356833in}{5.175000in}}%
\pgfusepath{clip}%
\pgfsetbuttcap%
\pgfsetroundjoin%
\pgfsetlinewidth{1.003750pt}%
\definecolor{currentstroke}{rgb}{0.827451,0.827451,0.827451}%
\pgfsetstrokecolor{currentstroke}%
\pgfsetdash{}{0pt}%
\pgfpathmoveto{\pgfqpoint{4.176160in}{0.456257in}}%
\pgfpathcurveto{\pgfqpoint{4.187210in}{0.456257in}}{\pgfqpoint{4.197809in}{0.460647in}}{\pgfqpoint{4.205623in}{0.468461in}}%
\pgfpathcurveto{\pgfqpoint{4.213436in}{0.476274in}}{\pgfqpoint{4.217827in}{0.486873in}}{\pgfqpoint{4.217827in}{0.497923in}}%
\pgfpathcurveto{\pgfqpoint{4.217827in}{0.508973in}}{\pgfqpoint{4.213436in}{0.519572in}}{\pgfqpoint{4.205623in}{0.527386in}}%
\pgfpathcurveto{\pgfqpoint{4.197809in}{0.535200in}}{\pgfqpoint{4.187210in}{0.539590in}}{\pgfqpoint{4.176160in}{0.539590in}}%
\pgfpathcurveto{\pgfqpoint{4.165110in}{0.539590in}}{\pgfqpoint{4.154511in}{0.535200in}}{\pgfqpoint{4.146697in}{0.527386in}}%
\pgfpathcurveto{\pgfqpoint{4.138883in}{0.519572in}}{\pgfqpoint{4.134493in}{0.508973in}}{\pgfqpoint{4.134493in}{0.497923in}}%
\pgfpathcurveto{\pgfqpoint{4.134493in}{0.486873in}}{\pgfqpoint{4.138883in}{0.476274in}}{\pgfqpoint{4.146697in}{0.468461in}}%
\pgfpathcurveto{\pgfqpoint{4.154511in}{0.460647in}}{\pgfqpoint{4.165110in}{0.456257in}}{\pgfqpoint{4.176160in}{0.456257in}}%
\pgfpathlineto{\pgfqpoint{4.176160in}{0.456257in}}%
\pgfpathclose%
\pgfusepath{stroke}%
\end{pgfscope}%
\begin{pgfscope}%
\pgfpathrectangle{\pgfqpoint{0.393053in}{0.375000in}}{\pgfqpoint{6.356833in}{5.175000in}}%
\pgfusepath{clip}%
\pgfsetbuttcap%
\pgfsetroundjoin%
\pgfsetlinewidth{1.003750pt}%
\definecolor{currentstroke}{rgb}{0.827451,0.827451,0.827451}%
\pgfsetstrokecolor{currentstroke}%
\pgfsetdash{}{0pt}%
\pgfpathmoveto{\pgfqpoint{1.041837in}{2.483497in}}%
\pgfpathcurveto{\pgfqpoint{1.052887in}{2.483497in}}{\pgfqpoint{1.063486in}{2.487887in}}{\pgfqpoint{1.071300in}{2.495701in}}%
\pgfpathcurveto{\pgfqpoint{1.079113in}{2.503514in}}{\pgfqpoint{1.083503in}{2.514113in}}{\pgfqpoint{1.083503in}{2.525164in}}%
\pgfpathcurveto{\pgfqpoint{1.083503in}{2.536214in}}{\pgfqpoint{1.079113in}{2.546813in}}{\pgfqpoint{1.071300in}{2.554626in}}%
\pgfpathcurveto{\pgfqpoint{1.063486in}{2.562440in}}{\pgfqpoint{1.052887in}{2.566830in}}{\pgfqpoint{1.041837in}{2.566830in}}%
\pgfpathcurveto{\pgfqpoint{1.030787in}{2.566830in}}{\pgfqpoint{1.020188in}{2.562440in}}{\pgfqpoint{1.012374in}{2.554626in}}%
\pgfpathcurveto{\pgfqpoint{1.004560in}{2.546813in}}{\pgfqpoint{1.000170in}{2.536214in}}{\pgfqpoint{1.000170in}{2.525164in}}%
\pgfpathcurveto{\pgfqpoint{1.000170in}{2.514113in}}{\pgfqpoint{1.004560in}{2.503514in}}{\pgfqpoint{1.012374in}{2.495701in}}%
\pgfpathcurveto{\pgfqpoint{1.020188in}{2.487887in}}{\pgfqpoint{1.030787in}{2.483497in}}{\pgfqpoint{1.041837in}{2.483497in}}%
\pgfpathlineto{\pgfqpoint{1.041837in}{2.483497in}}%
\pgfpathclose%
\pgfusepath{stroke}%
\end{pgfscope}%
\begin{pgfscope}%
\pgfpathrectangle{\pgfqpoint{0.393053in}{0.375000in}}{\pgfqpoint{6.356833in}{5.175000in}}%
\pgfusepath{clip}%
\pgfsetbuttcap%
\pgfsetroundjoin%
\pgfsetlinewidth{1.003750pt}%
\definecolor{currentstroke}{rgb}{0.827451,0.827451,0.827451}%
\pgfsetstrokecolor{currentstroke}%
\pgfsetdash{}{0pt}%
\pgfpathmoveto{\pgfqpoint{0.396398in}{4.410130in}}%
\pgfpathcurveto{\pgfqpoint{0.407448in}{4.410130in}}{\pgfqpoint{0.418047in}{4.414521in}}{\pgfqpoint{0.425861in}{4.422334in}}%
\pgfpathcurveto{\pgfqpoint{0.433674in}{4.430148in}}{\pgfqpoint{0.438064in}{4.440747in}}{\pgfqpoint{0.438064in}{4.451797in}}%
\pgfpathcurveto{\pgfqpoint{0.438064in}{4.462847in}}{\pgfqpoint{0.433674in}{4.473446in}}{\pgfqpoint{0.425861in}{4.481260in}}%
\pgfpathcurveto{\pgfqpoint{0.418047in}{4.489073in}}{\pgfqpoint{0.407448in}{4.493464in}}{\pgfqpoint{0.396398in}{4.493464in}}%
\pgfpathcurveto{\pgfqpoint{0.385348in}{4.493464in}}{\pgfqpoint{0.374749in}{4.489073in}}{\pgfqpoint{0.366935in}{4.481260in}}%
\pgfpathcurveto{\pgfqpoint{0.359121in}{4.473446in}}{\pgfqpoint{0.354731in}{4.462847in}}{\pgfqpoint{0.354731in}{4.451797in}}%
\pgfpathcurveto{\pgfqpoint{0.354731in}{4.440747in}}{\pgfqpoint{0.359121in}{4.430148in}}{\pgfqpoint{0.366935in}{4.422334in}}%
\pgfpathcurveto{\pgfqpoint{0.374749in}{4.414521in}}{\pgfqpoint{0.385348in}{4.410130in}}{\pgfqpoint{0.396398in}{4.410130in}}%
\pgfpathlineto{\pgfqpoint{0.396398in}{4.410130in}}%
\pgfpathclose%
\pgfusepath{stroke}%
\end{pgfscope}%
\begin{pgfscope}%
\pgfpathrectangle{\pgfqpoint{0.393053in}{0.375000in}}{\pgfqpoint{6.356833in}{5.175000in}}%
\pgfusepath{clip}%
\pgfsetbuttcap%
\pgfsetroundjoin%
\pgfsetlinewidth{1.003750pt}%
\definecolor{currentstroke}{rgb}{0.827451,0.827451,0.827451}%
\pgfsetstrokecolor{currentstroke}%
\pgfsetdash{}{0pt}%
\pgfpathmoveto{\pgfqpoint{0.931735in}{2.733617in}}%
\pgfpathcurveto{\pgfqpoint{0.942785in}{2.733617in}}{\pgfqpoint{0.953384in}{2.738008in}}{\pgfqpoint{0.961197in}{2.745821in}}%
\pgfpathcurveto{\pgfqpoint{0.969011in}{2.753635in}}{\pgfqpoint{0.973401in}{2.764234in}}{\pgfqpoint{0.973401in}{2.775284in}}%
\pgfpathcurveto{\pgfqpoint{0.973401in}{2.786334in}}{\pgfqpoint{0.969011in}{2.796933in}}{\pgfqpoint{0.961197in}{2.804747in}}%
\pgfpathcurveto{\pgfqpoint{0.953384in}{2.812560in}}{\pgfqpoint{0.942785in}{2.816951in}}{\pgfqpoint{0.931735in}{2.816951in}}%
\pgfpathcurveto{\pgfqpoint{0.920684in}{2.816951in}}{\pgfqpoint{0.910085in}{2.812560in}}{\pgfqpoint{0.902272in}{2.804747in}}%
\pgfpathcurveto{\pgfqpoint{0.894458in}{2.796933in}}{\pgfqpoint{0.890068in}{2.786334in}}{\pgfqpoint{0.890068in}{2.775284in}}%
\pgfpathcurveto{\pgfqpoint{0.890068in}{2.764234in}}{\pgfqpoint{0.894458in}{2.753635in}}{\pgfqpoint{0.902272in}{2.745821in}}%
\pgfpathcurveto{\pgfqpoint{0.910085in}{2.738008in}}{\pgfqpoint{0.920684in}{2.733617in}}{\pgfqpoint{0.931735in}{2.733617in}}%
\pgfpathlineto{\pgfqpoint{0.931735in}{2.733617in}}%
\pgfpathclose%
\pgfusepath{stroke}%
\end{pgfscope}%
\begin{pgfscope}%
\pgfpathrectangle{\pgfqpoint{0.393053in}{0.375000in}}{\pgfqpoint{6.356833in}{5.175000in}}%
\pgfusepath{clip}%
\pgfsetbuttcap%
\pgfsetroundjoin%
\pgfsetlinewidth{1.003750pt}%
\definecolor{currentstroke}{rgb}{0.827451,0.827451,0.827451}%
\pgfsetstrokecolor{currentstroke}%
\pgfsetdash{}{0pt}%
\pgfpathmoveto{\pgfqpoint{3.748752in}{0.577226in}}%
\pgfpathcurveto{\pgfqpoint{3.759802in}{0.577226in}}{\pgfqpoint{3.770401in}{0.581616in}}{\pgfqpoint{3.778214in}{0.589430in}}%
\pgfpathcurveto{\pgfqpoint{3.786028in}{0.597244in}}{\pgfqpoint{3.790418in}{0.607843in}}{\pgfqpoint{3.790418in}{0.618893in}}%
\pgfpathcurveto{\pgfqpoint{3.790418in}{0.629943in}}{\pgfqpoint{3.786028in}{0.640542in}}{\pgfqpoint{3.778214in}{0.648356in}}%
\pgfpathcurveto{\pgfqpoint{3.770401in}{0.656169in}}{\pgfqpoint{3.759802in}{0.660559in}}{\pgfqpoint{3.748752in}{0.660559in}}%
\pgfpathcurveto{\pgfqpoint{3.737702in}{0.660559in}}{\pgfqpoint{3.727103in}{0.656169in}}{\pgfqpoint{3.719289in}{0.648356in}}%
\pgfpathcurveto{\pgfqpoint{3.711475in}{0.640542in}}{\pgfqpoint{3.707085in}{0.629943in}}{\pgfqpoint{3.707085in}{0.618893in}}%
\pgfpathcurveto{\pgfqpoint{3.707085in}{0.607843in}}{\pgfqpoint{3.711475in}{0.597244in}}{\pgfqpoint{3.719289in}{0.589430in}}%
\pgfpathcurveto{\pgfqpoint{3.727103in}{0.581616in}}{\pgfqpoint{3.737702in}{0.577226in}}{\pgfqpoint{3.748752in}{0.577226in}}%
\pgfpathlineto{\pgfqpoint{3.748752in}{0.577226in}}%
\pgfpathclose%
\pgfusepath{stroke}%
\end{pgfscope}%
\begin{pgfscope}%
\pgfpathrectangle{\pgfqpoint{0.393053in}{0.375000in}}{\pgfqpoint{6.356833in}{5.175000in}}%
\pgfusepath{clip}%
\pgfsetbuttcap%
\pgfsetroundjoin%
\pgfsetlinewidth{1.003750pt}%
\definecolor{currentstroke}{rgb}{0.827451,0.827451,0.827451}%
\pgfsetstrokecolor{currentstroke}%
\pgfsetdash{}{0pt}%
\pgfpathmoveto{\pgfqpoint{0.612319in}{3.180269in}}%
\pgfpathcurveto{\pgfqpoint{0.623369in}{3.180269in}}{\pgfqpoint{0.633968in}{3.184660in}}{\pgfqpoint{0.641781in}{3.192473in}}%
\pgfpathcurveto{\pgfqpoint{0.649595in}{3.200287in}}{\pgfqpoint{0.653985in}{3.210886in}}{\pgfqpoint{0.653985in}{3.221936in}}%
\pgfpathcurveto{\pgfqpoint{0.653985in}{3.232986in}}{\pgfqpoint{0.649595in}{3.243585in}}{\pgfqpoint{0.641781in}{3.251399in}}%
\pgfpathcurveto{\pgfqpoint{0.633968in}{3.259212in}}{\pgfqpoint{0.623369in}{3.263603in}}{\pgfqpoint{0.612319in}{3.263603in}}%
\pgfpathcurveto{\pgfqpoint{0.601269in}{3.263603in}}{\pgfqpoint{0.590669in}{3.259212in}}{\pgfqpoint{0.582856in}{3.251399in}}%
\pgfpathcurveto{\pgfqpoint{0.575042in}{3.243585in}}{\pgfqpoint{0.570652in}{3.232986in}}{\pgfqpoint{0.570652in}{3.221936in}}%
\pgfpathcurveto{\pgfqpoint{0.570652in}{3.210886in}}{\pgfqpoint{0.575042in}{3.200287in}}{\pgfqpoint{0.582856in}{3.192473in}}%
\pgfpathcurveto{\pgfqpoint{0.590669in}{3.184660in}}{\pgfqpoint{0.601269in}{3.180269in}}{\pgfqpoint{0.612319in}{3.180269in}}%
\pgfpathlineto{\pgfqpoint{0.612319in}{3.180269in}}%
\pgfpathclose%
\pgfusepath{stroke}%
\end{pgfscope}%
\begin{pgfscope}%
\pgfpathrectangle{\pgfqpoint{0.393053in}{0.375000in}}{\pgfqpoint{6.356833in}{5.175000in}}%
\pgfusepath{clip}%
\pgfsetbuttcap%
\pgfsetroundjoin%
\pgfsetlinewidth{1.003750pt}%
\definecolor{currentstroke}{rgb}{0.827451,0.827451,0.827451}%
\pgfsetstrokecolor{currentstroke}%
\pgfsetdash{}{0pt}%
\pgfpathmoveto{\pgfqpoint{4.534846in}{0.422259in}}%
\pgfpathcurveto{\pgfqpoint{4.545896in}{0.422259in}}{\pgfqpoint{4.556495in}{0.426649in}}{\pgfqpoint{4.564308in}{0.434463in}}%
\pgfpathcurveto{\pgfqpoint{4.572122in}{0.442276in}}{\pgfqpoint{4.576512in}{0.452875in}}{\pgfqpoint{4.576512in}{0.463926in}}%
\pgfpathcurveto{\pgfqpoint{4.576512in}{0.474976in}}{\pgfqpoint{4.572122in}{0.485575in}}{\pgfqpoint{4.564308in}{0.493388in}}%
\pgfpathcurveto{\pgfqpoint{4.556495in}{0.501202in}}{\pgfqpoint{4.545896in}{0.505592in}}{\pgfqpoint{4.534846in}{0.505592in}}%
\pgfpathcurveto{\pgfqpoint{4.523795in}{0.505592in}}{\pgfqpoint{4.513196in}{0.501202in}}{\pgfqpoint{4.505383in}{0.493388in}}%
\pgfpathcurveto{\pgfqpoint{4.497569in}{0.485575in}}{\pgfqpoint{4.493179in}{0.474976in}}{\pgfqpoint{4.493179in}{0.463926in}}%
\pgfpathcurveto{\pgfqpoint{4.493179in}{0.452875in}}{\pgfqpoint{4.497569in}{0.442276in}}{\pgfqpoint{4.505383in}{0.434463in}}%
\pgfpathcurveto{\pgfqpoint{4.513196in}{0.426649in}}{\pgfqpoint{4.523795in}{0.422259in}}{\pgfqpoint{4.534846in}{0.422259in}}%
\pgfpathlineto{\pgfqpoint{4.534846in}{0.422259in}}%
\pgfpathclose%
\pgfusepath{stroke}%
\end{pgfscope}%
\begin{pgfscope}%
\pgfpathrectangle{\pgfqpoint{0.393053in}{0.375000in}}{\pgfqpoint{6.356833in}{5.175000in}}%
\pgfusepath{clip}%
\pgfsetbuttcap%
\pgfsetroundjoin%
\pgfsetlinewidth{1.003750pt}%
\definecolor{currentstroke}{rgb}{0.827451,0.827451,0.827451}%
\pgfsetstrokecolor{currentstroke}%
\pgfsetdash{}{0pt}%
\pgfpathmoveto{\pgfqpoint{2.585362in}{1.005073in}}%
\pgfpathcurveto{\pgfqpoint{2.596412in}{1.005073in}}{\pgfqpoint{2.607011in}{1.009463in}}{\pgfqpoint{2.614824in}{1.017277in}}%
\pgfpathcurveto{\pgfqpoint{2.622638in}{1.025090in}}{\pgfqpoint{2.627028in}{1.035689in}}{\pgfqpoint{2.627028in}{1.046740in}}%
\pgfpathcurveto{\pgfqpoint{2.627028in}{1.057790in}}{\pgfqpoint{2.622638in}{1.068389in}}{\pgfqpoint{2.614824in}{1.076202in}}%
\pgfpathcurveto{\pgfqpoint{2.607011in}{1.084016in}}{\pgfqpoint{2.596412in}{1.088406in}}{\pgfqpoint{2.585362in}{1.088406in}}%
\pgfpathcurveto{\pgfqpoint{2.574311in}{1.088406in}}{\pgfqpoint{2.563712in}{1.084016in}}{\pgfqpoint{2.555899in}{1.076202in}}%
\pgfpathcurveto{\pgfqpoint{2.548085in}{1.068389in}}{\pgfqpoint{2.543695in}{1.057790in}}{\pgfqpoint{2.543695in}{1.046740in}}%
\pgfpathcurveto{\pgfqpoint{2.543695in}{1.035689in}}{\pgfqpoint{2.548085in}{1.025090in}}{\pgfqpoint{2.555899in}{1.017277in}}%
\pgfpathcurveto{\pgfqpoint{2.563712in}{1.009463in}}{\pgfqpoint{2.574311in}{1.005073in}}{\pgfqpoint{2.585362in}{1.005073in}}%
\pgfpathlineto{\pgfqpoint{2.585362in}{1.005073in}}%
\pgfpathclose%
\pgfusepath{stroke}%
\end{pgfscope}%
\begin{pgfscope}%
\pgfpathrectangle{\pgfqpoint{0.393053in}{0.375000in}}{\pgfqpoint{6.356833in}{5.175000in}}%
\pgfusepath{clip}%
\pgfsetbuttcap%
\pgfsetroundjoin%
\pgfsetlinewidth{1.003750pt}%
\definecolor{currentstroke}{rgb}{0.827451,0.827451,0.827451}%
\pgfsetstrokecolor{currentstroke}%
\pgfsetdash{}{0pt}%
\pgfpathmoveto{\pgfqpoint{3.083461in}{0.827183in}}%
\pgfpathcurveto{\pgfqpoint{3.094511in}{0.827183in}}{\pgfqpoint{3.105110in}{0.831573in}}{\pgfqpoint{3.112924in}{0.839387in}}%
\pgfpathcurveto{\pgfqpoint{3.120738in}{0.847200in}}{\pgfqpoint{3.125128in}{0.857799in}}{\pgfqpoint{3.125128in}{0.868849in}}%
\pgfpathcurveto{\pgfqpoint{3.125128in}{0.879900in}}{\pgfqpoint{3.120738in}{0.890499in}}{\pgfqpoint{3.112924in}{0.898312in}}%
\pgfpathcurveto{\pgfqpoint{3.105110in}{0.906126in}}{\pgfqpoint{3.094511in}{0.910516in}}{\pgfqpoint{3.083461in}{0.910516in}}%
\pgfpathcurveto{\pgfqpoint{3.072411in}{0.910516in}}{\pgfqpoint{3.061812in}{0.906126in}}{\pgfqpoint{3.053999in}{0.898312in}}%
\pgfpathcurveto{\pgfqpoint{3.046185in}{0.890499in}}{\pgfqpoint{3.041795in}{0.879900in}}{\pgfqpoint{3.041795in}{0.868849in}}%
\pgfpathcurveto{\pgfqpoint{3.041795in}{0.857799in}}{\pgfqpoint{3.046185in}{0.847200in}}{\pgfqpoint{3.053999in}{0.839387in}}%
\pgfpathcurveto{\pgfqpoint{3.061812in}{0.831573in}}{\pgfqpoint{3.072411in}{0.827183in}}{\pgfqpoint{3.083461in}{0.827183in}}%
\pgfpathlineto{\pgfqpoint{3.083461in}{0.827183in}}%
\pgfpathclose%
\pgfusepath{stroke}%
\end{pgfscope}%
\begin{pgfscope}%
\pgfpathrectangle{\pgfqpoint{0.393053in}{0.375000in}}{\pgfqpoint{6.356833in}{5.175000in}}%
\pgfusepath{clip}%
\pgfsetbuttcap%
\pgfsetroundjoin%
\pgfsetlinewidth{1.003750pt}%
\definecolor{currentstroke}{rgb}{0.827451,0.827451,0.827451}%
\pgfsetstrokecolor{currentstroke}%
\pgfsetdash{}{0pt}%
\pgfpathmoveto{\pgfqpoint{1.387587in}{1.896348in}}%
\pgfpathcurveto{\pgfqpoint{1.398637in}{1.896348in}}{\pgfqpoint{1.409236in}{1.900738in}}{\pgfqpoint{1.417050in}{1.908552in}}%
\pgfpathcurveto{\pgfqpoint{1.424863in}{1.916366in}}{\pgfqpoint{1.429254in}{1.926965in}}{\pgfqpoint{1.429254in}{1.938015in}}%
\pgfpathcurveto{\pgfqpoint{1.429254in}{1.949065in}}{\pgfqpoint{1.424863in}{1.959664in}}{\pgfqpoint{1.417050in}{1.967477in}}%
\pgfpathcurveto{\pgfqpoint{1.409236in}{1.975291in}}{\pgfqpoint{1.398637in}{1.979681in}}{\pgfqpoint{1.387587in}{1.979681in}}%
\pgfpathcurveto{\pgfqpoint{1.376537in}{1.979681in}}{\pgfqpoint{1.365938in}{1.975291in}}{\pgfqpoint{1.358124in}{1.967477in}}%
\pgfpathcurveto{\pgfqpoint{1.350311in}{1.959664in}}{\pgfqpoint{1.345920in}{1.949065in}}{\pgfqpoint{1.345920in}{1.938015in}}%
\pgfpathcurveto{\pgfqpoint{1.345920in}{1.926965in}}{\pgfqpoint{1.350311in}{1.916366in}}{\pgfqpoint{1.358124in}{1.908552in}}%
\pgfpathcurveto{\pgfqpoint{1.365938in}{1.900738in}}{\pgfqpoint{1.376537in}{1.896348in}}{\pgfqpoint{1.387587in}{1.896348in}}%
\pgfpathlineto{\pgfqpoint{1.387587in}{1.896348in}}%
\pgfpathclose%
\pgfusepath{stroke}%
\end{pgfscope}%
\begin{pgfscope}%
\pgfpathrectangle{\pgfqpoint{0.393053in}{0.375000in}}{\pgfqpoint{6.356833in}{5.175000in}}%
\pgfusepath{clip}%
\pgfsetbuttcap%
\pgfsetroundjoin%
\pgfsetlinewidth{1.003750pt}%
\definecolor{currentstroke}{rgb}{0.827451,0.827451,0.827451}%
\pgfsetstrokecolor{currentstroke}%
\pgfsetdash{}{0pt}%
\pgfpathmoveto{\pgfqpoint{3.142172in}{0.759581in}}%
\pgfpathcurveto{\pgfqpoint{3.153222in}{0.759581in}}{\pgfqpoint{3.163821in}{0.763971in}}{\pgfqpoint{3.171635in}{0.771785in}}%
\pgfpathcurveto{\pgfqpoint{3.179449in}{0.779598in}}{\pgfqpoint{3.183839in}{0.790197in}}{\pgfqpoint{3.183839in}{0.801247in}}%
\pgfpathcurveto{\pgfqpoint{3.183839in}{0.812297in}}{\pgfqpoint{3.179449in}{0.822896in}}{\pgfqpoint{3.171635in}{0.830710in}}%
\pgfpathcurveto{\pgfqpoint{3.163821in}{0.838524in}}{\pgfqpoint{3.153222in}{0.842914in}}{\pgfqpoint{3.142172in}{0.842914in}}%
\pgfpathcurveto{\pgfqpoint{3.131122in}{0.842914in}}{\pgfqpoint{3.120523in}{0.838524in}}{\pgfqpoint{3.112710in}{0.830710in}}%
\pgfpathcurveto{\pgfqpoint{3.104896in}{0.822896in}}{\pgfqpoint{3.100506in}{0.812297in}}{\pgfqpoint{3.100506in}{0.801247in}}%
\pgfpathcurveto{\pgfqpoint{3.100506in}{0.790197in}}{\pgfqpoint{3.104896in}{0.779598in}}{\pgfqpoint{3.112710in}{0.771785in}}%
\pgfpathcurveto{\pgfqpoint{3.120523in}{0.763971in}}{\pgfqpoint{3.131122in}{0.759581in}}{\pgfqpoint{3.142172in}{0.759581in}}%
\pgfpathlineto{\pgfqpoint{3.142172in}{0.759581in}}%
\pgfpathclose%
\pgfusepath{stroke}%
\end{pgfscope}%
\begin{pgfscope}%
\pgfpathrectangle{\pgfqpoint{0.393053in}{0.375000in}}{\pgfqpoint{6.356833in}{5.175000in}}%
\pgfusepath{clip}%
\pgfsetbuttcap%
\pgfsetroundjoin%
\pgfsetlinewidth{1.003750pt}%
\definecolor{currentstroke}{rgb}{0.827451,0.827451,0.827451}%
\pgfsetstrokecolor{currentstroke}%
\pgfsetdash{}{0pt}%
\pgfpathmoveto{\pgfqpoint{1.075039in}{2.317616in}}%
\pgfpathcurveto{\pgfqpoint{1.086090in}{2.317616in}}{\pgfqpoint{1.096689in}{2.322006in}}{\pgfqpoint{1.104502in}{2.329819in}}%
\pgfpathcurveto{\pgfqpoint{1.112316in}{2.337633in}}{\pgfqpoint{1.116706in}{2.348232in}}{\pgfqpoint{1.116706in}{2.359282in}}%
\pgfpathcurveto{\pgfqpoint{1.116706in}{2.370332in}}{\pgfqpoint{1.112316in}{2.380931in}}{\pgfqpoint{1.104502in}{2.388745in}}%
\pgfpathcurveto{\pgfqpoint{1.096689in}{2.396559in}}{\pgfqpoint{1.086090in}{2.400949in}}{\pgfqpoint{1.075039in}{2.400949in}}%
\pgfpathcurveto{\pgfqpoint{1.063989in}{2.400949in}}{\pgfqpoint{1.053390in}{2.396559in}}{\pgfqpoint{1.045577in}{2.388745in}}%
\pgfpathcurveto{\pgfqpoint{1.037763in}{2.380931in}}{\pgfqpoint{1.033373in}{2.370332in}}{\pgfqpoint{1.033373in}{2.359282in}}%
\pgfpathcurveto{\pgfqpoint{1.033373in}{2.348232in}}{\pgfqpoint{1.037763in}{2.337633in}}{\pgfqpoint{1.045577in}{2.329819in}}%
\pgfpathcurveto{\pgfqpoint{1.053390in}{2.322006in}}{\pgfqpoint{1.063989in}{2.317616in}}{\pgfqpoint{1.075039in}{2.317616in}}%
\pgfpathlineto{\pgfqpoint{1.075039in}{2.317616in}}%
\pgfpathclose%
\pgfusepath{stroke}%
\end{pgfscope}%
\begin{pgfscope}%
\pgfpathrectangle{\pgfqpoint{0.393053in}{0.375000in}}{\pgfqpoint{6.356833in}{5.175000in}}%
\pgfusepath{clip}%
\pgfsetbuttcap%
\pgfsetroundjoin%
\pgfsetlinewidth{1.003750pt}%
\definecolor{currentstroke}{rgb}{0.827451,0.827451,0.827451}%
\pgfsetstrokecolor{currentstroke}%
\pgfsetdash{}{0pt}%
\pgfpathmoveto{\pgfqpoint{0.483847in}{3.664437in}}%
\pgfpathcurveto{\pgfqpoint{0.494897in}{3.664437in}}{\pgfqpoint{0.505496in}{3.668827in}}{\pgfqpoint{0.513309in}{3.676640in}}%
\pgfpathcurveto{\pgfqpoint{0.521123in}{3.684454in}}{\pgfqpoint{0.525513in}{3.695053in}}{\pgfqpoint{0.525513in}{3.706103in}}%
\pgfpathcurveto{\pgfqpoint{0.525513in}{3.717153in}}{\pgfqpoint{0.521123in}{3.727752in}}{\pgfqpoint{0.513309in}{3.735566in}}%
\pgfpathcurveto{\pgfqpoint{0.505496in}{3.743380in}}{\pgfqpoint{0.494897in}{3.747770in}}{\pgfqpoint{0.483847in}{3.747770in}}%
\pgfpathcurveto{\pgfqpoint{0.472797in}{3.747770in}}{\pgfqpoint{0.462198in}{3.743380in}}{\pgfqpoint{0.454384in}{3.735566in}}%
\pgfpathcurveto{\pgfqpoint{0.446570in}{3.727752in}}{\pgfqpoint{0.442180in}{3.717153in}}{\pgfqpoint{0.442180in}{3.706103in}}%
\pgfpathcurveto{\pgfqpoint{0.442180in}{3.695053in}}{\pgfqpoint{0.446570in}{3.684454in}}{\pgfqpoint{0.454384in}{3.676640in}}%
\pgfpathcurveto{\pgfqpoint{0.462198in}{3.668827in}}{\pgfqpoint{0.472797in}{3.664437in}}{\pgfqpoint{0.483847in}{3.664437in}}%
\pgfpathlineto{\pgfqpoint{0.483847in}{3.664437in}}%
\pgfpathclose%
\pgfusepath{stroke}%
\end{pgfscope}%
\begin{pgfscope}%
\pgfpathrectangle{\pgfqpoint{0.393053in}{0.375000in}}{\pgfqpoint{6.356833in}{5.175000in}}%
\pgfusepath{clip}%
\pgfsetbuttcap%
\pgfsetroundjoin%
\pgfsetlinewidth{1.003750pt}%
\definecolor{currentstroke}{rgb}{0.827451,0.827451,0.827451}%
\pgfsetstrokecolor{currentstroke}%
\pgfsetdash{}{0pt}%
\pgfpathmoveto{\pgfqpoint{2.061096in}{1.323279in}}%
\pgfpathcurveto{\pgfqpoint{2.072146in}{1.323279in}}{\pgfqpoint{2.082745in}{1.327669in}}{\pgfqpoint{2.090559in}{1.335483in}}%
\pgfpathcurveto{\pgfqpoint{2.098372in}{1.343297in}}{\pgfqpoint{2.102763in}{1.353896in}}{\pgfqpoint{2.102763in}{1.364946in}}%
\pgfpathcurveto{\pgfqpoint{2.102763in}{1.375996in}}{\pgfqpoint{2.098372in}{1.386595in}}{\pgfqpoint{2.090559in}{1.394408in}}%
\pgfpathcurveto{\pgfqpoint{2.082745in}{1.402222in}}{\pgfqpoint{2.072146in}{1.406612in}}{\pgfqpoint{2.061096in}{1.406612in}}%
\pgfpathcurveto{\pgfqpoint{2.050046in}{1.406612in}}{\pgfqpoint{2.039447in}{1.402222in}}{\pgfqpoint{2.031633in}{1.394408in}}%
\pgfpathcurveto{\pgfqpoint{2.023819in}{1.386595in}}{\pgfqpoint{2.019429in}{1.375996in}}{\pgfqpoint{2.019429in}{1.364946in}}%
\pgfpathcurveto{\pgfqpoint{2.019429in}{1.353896in}}{\pgfqpoint{2.023819in}{1.343297in}}{\pgfqpoint{2.031633in}{1.335483in}}%
\pgfpathcurveto{\pgfqpoint{2.039447in}{1.327669in}}{\pgfqpoint{2.050046in}{1.323279in}}{\pgfqpoint{2.061096in}{1.323279in}}%
\pgfpathlineto{\pgfqpoint{2.061096in}{1.323279in}}%
\pgfpathclose%
\pgfusepath{stroke}%
\end{pgfscope}%
\begin{pgfscope}%
\pgfpathrectangle{\pgfqpoint{0.393053in}{0.375000in}}{\pgfqpoint{6.356833in}{5.175000in}}%
\pgfusepath{clip}%
\pgfsetbuttcap%
\pgfsetroundjoin%
\pgfsetlinewidth{1.003750pt}%
\definecolor{currentstroke}{rgb}{0.827451,0.827451,0.827451}%
\pgfsetstrokecolor{currentstroke}%
\pgfsetdash{}{0pt}%
\pgfpathmoveto{\pgfqpoint{0.672197in}{3.023099in}}%
\pgfpathcurveto{\pgfqpoint{0.683247in}{3.023099in}}{\pgfqpoint{0.693846in}{3.027489in}}{\pgfqpoint{0.701660in}{3.035303in}}%
\pgfpathcurveto{\pgfqpoint{0.709473in}{3.043116in}}{\pgfqpoint{0.713863in}{3.053715in}}{\pgfqpoint{0.713863in}{3.064766in}}%
\pgfpathcurveto{\pgfqpoint{0.713863in}{3.075816in}}{\pgfqpoint{0.709473in}{3.086415in}}{\pgfqpoint{0.701660in}{3.094228in}}%
\pgfpathcurveto{\pgfqpoint{0.693846in}{3.102042in}}{\pgfqpoint{0.683247in}{3.106432in}}{\pgfqpoint{0.672197in}{3.106432in}}%
\pgfpathcurveto{\pgfqpoint{0.661147in}{3.106432in}}{\pgfqpoint{0.650548in}{3.102042in}}{\pgfqpoint{0.642734in}{3.094228in}}%
\pgfpathcurveto{\pgfqpoint{0.634920in}{3.086415in}}{\pgfqpoint{0.630530in}{3.075816in}}{\pgfqpoint{0.630530in}{3.064766in}}%
\pgfpathcurveto{\pgfqpoint{0.630530in}{3.053715in}}{\pgfqpoint{0.634920in}{3.043116in}}{\pgfqpoint{0.642734in}{3.035303in}}%
\pgfpathcurveto{\pgfqpoint{0.650548in}{3.027489in}}{\pgfqpoint{0.661147in}{3.023099in}}{\pgfqpoint{0.672197in}{3.023099in}}%
\pgfpathlineto{\pgfqpoint{0.672197in}{3.023099in}}%
\pgfpathclose%
\pgfusepath{stroke}%
\end{pgfscope}%
\begin{pgfscope}%
\pgfpathrectangle{\pgfqpoint{0.393053in}{0.375000in}}{\pgfqpoint{6.356833in}{5.175000in}}%
\pgfusepath{clip}%
\pgfsetbuttcap%
\pgfsetroundjoin%
\pgfsetlinewidth{1.003750pt}%
\definecolor{currentstroke}{rgb}{0.827451,0.827451,0.827451}%
\pgfsetstrokecolor{currentstroke}%
\pgfsetdash{}{0pt}%
\pgfpathmoveto{\pgfqpoint{5.388083in}{0.344848in}}%
\pgfpathcurveto{\pgfqpoint{5.399133in}{0.344848in}}{\pgfqpoint{5.409732in}{0.349238in}}{\pgfqpoint{5.417546in}{0.357052in}}%
\pgfpathcurveto{\pgfqpoint{5.425360in}{0.364865in}}{\pgfqpoint{5.429750in}{0.375464in}}{\pgfqpoint{5.429750in}{0.386514in}}%
\pgfpathcurveto{\pgfqpoint{5.429750in}{0.397565in}}{\pgfqpoint{5.425360in}{0.408164in}}{\pgfqpoint{5.417546in}{0.415977in}}%
\pgfpathcurveto{\pgfqpoint{5.409732in}{0.423791in}}{\pgfqpoint{5.399133in}{0.428181in}}{\pgfqpoint{5.388083in}{0.428181in}}%
\pgfpathcurveto{\pgfqpoint{5.377033in}{0.428181in}}{\pgfqpoint{5.366434in}{0.423791in}}{\pgfqpoint{5.358620in}{0.415977in}}%
\pgfpathcurveto{\pgfqpoint{5.350807in}{0.408164in}}{\pgfqpoint{5.346417in}{0.397565in}}{\pgfqpoint{5.346417in}{0.386514in}}%
\pgfpathcurveto{\pgfqpoint{5.346417in}{0.375464in}}{\pgfqpoint{5.350807in}{0.364865in}}{\pgfqpoint{5.358620in}{0.357052in}}%
\pgfpathcurveto{\pgfqpoint{5.366434in}{0.349238in}}{\pgfqpoint{5.377033in}{0.344848in}}{\pgfqpoint{5.388083in}{0.344848in}}%
\pgfusepath{stroke}%
\end{pgfscope}%
\begin{pgfscope}%
\pgfpathrectangle{\pgfqpoint{0.393053in}{0.375000in}}{\pgfqpoint{6.356833in}{5.175000in}}%
\pgfusepath{clip}%
\pgfsetbuttcap%
\pgfsetroundjoin%
\pgfsetlinewidth{1.003750pt}%
\definecolor{currentstroke}{rgb}{0.827451,0.827451,0.827451}%
\pgfsetstrokecolor{currentstroke}%
\pgfsetdash{}{0pt}%
\pgfpathmoveto{\pgfqpoint{4.852716in}{0.410760in}}%
\pgfpathcurveto{\pgfqpoint{4.863766in}{0.410760in}}{\pgfqpoint{4.874365in}{0.415150in}}{\pgfqpoint{4.882179in}{0.422963in}}%
\pgfpathcurveto{\pgfqpoint{4.889992in}{0.430777in}}{\pgfqpoint{4.894383in}{0.441376in}}{\pgfqpoint{4.894383in}{0.452426in}}%
\pgfpathcurveto{\pgfqpoint{4.894383in}{0.463476in}}{\pgfqpoint{4.889992in}{0.474075in}}{\pgfqpoint{4.882179in}{0.481889in}}%
\pgfpathcurveto{\pgfqpoint{4.874365in}{0.489703in}}{\pgfqpoint{4.863766in}{0.494093in}}{\pgfqpoint{4.852716in}{0.494093in}}%
\pgfpathcurveto{\pgfqpoint{4.841666in}{0.494093in}}{\pgfqpoint{4.831067in}{0.489703in}}{\pgfqpoint{4.823253in}{0.481889in}}%
\pgfpathcurveto{\pgfqpoint{4.815439in}{0.474075in}}{\pgfqpoint{4.811049in}{0.463476in}}{\pgfqpoint{4.811049in}{0.452426in}}%
\pgfpathcurveto{\pgfqpoint{4.811049in}{0.441376in}}{\pgfqpoint{4.815439in}{0.430777in}}{\pgfqpoint{4.823253in}{0.422963in}}%
\pgfpathcurveto{\pgfqpoint{4.831067in}{0.415150in}}{\pgfqpoint{4.841666in}{0.410760in}}{\pgfqpoint{4.852716in}{0.410760in}}%
\pgfpathlineto{\pgfqpoint{4.852716in}{0.410760in}}%
\pgfpathclose%
\pgfusepath{stroke}%
\end{pgfscope}%
\begin{pgfscope}%
\pgfpathrectangle{\pgfqpoint{0.393053in}{0.375000in}}{\pgfqpoint{6.356833in}{5.175000in}}%
\pgfusepath{clip}%
\pgfsetbuttcap%
\pgfsetroundjoin%
\pgfsetlinewidth{1.003750pt}%
\definecolor{currentstroke}{rgb}{0.827451,0.827451,0.827451}%
\pgfsetstrokecolor{currentstroke}%
\pgfsetdash{}{0pt}%
\pgfpathmoveto{\pgfqpoint{2.076849in}{1.314285in}}%
\pgfpathcurveto{\pgfqpoint{2.087899in}{1.314285in}}{\pgfqpoint{2.098498in}{1.318675in}}{\pgfqpoint{2.106312in}{1.326489in}}%
\pgfpathcurveto{\pgfqpoint{2.114125in}{1.334302in}}{\pgfqpoint{2.118516in}{1.344901in}}{\pgfqpoint{2.118516in}{1.355952in}}%
\pgfpathcurveto{\pgfqpoint{2.118516in}{1.367002in}}{\pgfqpoint{2.114125in}{1.377601in}}{\pgfqpoint{2.106312in}{1.385414in}}%
\pgfpathcurveto{\pgfqpoint{2.098498in}{1.393228in}}{\pgfqpoint{2.087899in}{1.397618in}}{\pgfqpoint{2.076849in}{1.397618in}}%
\pgfpathcurveto{\pgfqpoint{2.065799in}{1.397618in}}{\pgfqpoint{2.055200in}{1.393228in}}{\pgfqpoint{2.047386in}{1.385414in}}%
\pgfpathcurveto{\pgfqpoint{2.039573in}{1.377601in}}{\pgfqpoint{2.035182in}{1.367002in}}{\pgfqpoint{2.035182in}{1.355952in}}%
\pgfpathcurveto{\pgfqpoint{2.035182in}{1.344901in}}{\pgfqpoint{2.039573in}{1.334302in}}{\pgfqpoint{2.047386in}{1.326489in}}%
\pgfpathcurveto{\pgfqpoint{2.055200in}{1.318675in}}{\pgfqpoint{2.065799in}{1.314285in}}{\pgfqpoint{2.076849in}{1.314285in}}%
\pgfpathlineto{\pgfqpoint{2.076849in}{1.314285in}}%
\pgfpathclose%
\pgfusepath{stroke}%
\end{pgfscope}%
\begin{pgfscope}%
\pgfpathrectangle{\pgfqpoint{0.393053in}{0.375000in}}{\pgfqpoint{6.356833in}{5.175000in}}%
\pgfusepath{clip}%
\pgfsetbuttcap%
\pgfsetroundjoin%
\pgfsetlinewidth{1.003750pt}%
\definecolor{currentstroke}{rgb}{0.827451,0.827451,0.827451}%
\pgfsetstrokecolor{currentstroke}%
\pgfsetdash{}{0pt}%
\pgfpathmoveto{\pgfqpoint{0.611155in}{3.254921in}}%
\pgfpathcurveto{\pgfqpoint{0.622206in}{3.254921in}}{\pgfqpoint{0.632805in}{3.259311in}}{\pgfqpoint{0.640618in}{3.267124in}}%
\pgfpathcurveto{\pgfqpoint{0.648432in}{3.274938in}}{\pgfqpoint{0.652822in}{3.285537in}}{\pgfqpoint{0.652822in}{3.296587in}}%
\pgfpathcurveto{\pgfqpoint{0.652822in}{3.307637in}}{\pgfqpoint{0.648432in}{3.318236in}}{\pgfqpoint{0.640618in}{3.326050in}}%
\pgfpathcurveto{\pgfqpoint{0.632805in}{3.333864in}}{\pgfqpoint{0.622206in}{3.338254in}}{\pgfqpoint{0.611155in}{3.338254in}}%
\pgfpathcurveto{\pgfqpoint{0.600105in}{3.338254in}}{\pgfqpoint{0.589506in}{3.333864in}}{\pgfqpoint{0.581693in}{3.326050in}}%
\pgfpathcurveto{\pgfqpoint{0.573879in}{3.318236in}}{\pgfqpoint{0.569489in}{3.307637in}}{\pgfqpoint{0.569489in}{3.296587in}}%
\pgfpathcurveto{\pgfqpoint{0.569489in}{3.285537in}}{\pgfqpoint{0.573879in}{3.274938in}}{\pgfqpoint{0.581693in}{3.267124in}}%
\pgfpathcurveto{\pgfqpoint{0.589506in}{3.259311in}}{\pgfqpoint{0.600105in}{3.254921in}}{\pgfqpoint{0.611155in}{3.254921in}}%
\pgfpathlineto{\pgfqpoint{0.611155in}{3.254921in}}%
\pgfpathclose%
\pgfusepath{stroke}%
\end{pgfscope}%
\begin{pgfscope}%
\pgfpathrectangle{\pgfqpoint{0.393053in}{0.375000in}}{\pgfqpoint{6.356833in}{5.175000in}}%
\pgfusepath{clip}%
\pgfsetbuttcap%
\pgfsetroundjoin%
\pgfsetlinewidth{1.003750pt}%
\definecolor{currentstroke}{rgb}{0.827451,0.827451,0.827451}%
\pgfsetstrokecolor{currentstroke}%
\pgfsetdash{}{0pt}%
\pgfpathmoveto{\pgfqpoint{1.768411in}{1.566057in}}%
\pgfpathcurveto{\pgfqpoint{1.779461in}{1.566057in}}{\pgfqpoint{1.790060in}{1.570447in}}{\pgfqpoint{1.797874in}{1.578261in}}%
\pgfpathcurveto{\pgfqpoint{1.805687in}{1.586074in}}{\pgfqpoint{1.810077in}{1.596673in}}{\pgfqpoint{1.810077in}{1.607723in}}%
\pgfpathcurveto{\pgfqpoint{1.810077in}{1.618773in}}{\pgfqpoint{1.805687in}{1.629373in}}{\pgfqpoint{1.797874in}{1.637186in}}%
\pgfpathcurveto{\pgfqpoint{1.790060in}{1.645000in}}{\pgfqpoint{1.779461in}{1.649390in}}{\pgfqpoint{1.768411in}{1.649390in}}%
\pgfpathcurveto{\pgfqpoint{1.757361in}{1.649390in}}{\pgfqpoint{1.746762in}{1.645000in}}{\pgfqpoint{1.738948in}{1.637186in}}%
\pgfpathcurveto{\pgfqpoint{1.731134in}{1.629373in}}{\pgfqpoint{1.726744in}{1.618773in}}{\pgfqpoint{1.726744in}{1.607723in}}%
\pgfpathcurveto{\pgfqpoint{1.726744in}{1.596673in}}{\pgfqpoint{1.731134in}{1.586074in}}{\pgfqpoint{1.738948in}{1.578261in}}%
\pgfpathcurveto{\pgfqpoint{1.746762in}{1.570447in}}{\pgfqpoint{1.757361in}{1.566057in}}{\pgfqpoint{1.768411in}{1.566057in}}%
\pgfpathlineto{\pgfqpoint{1.768411in}{1.566057in}}%
\pgfpathclose%
\pgfusepath{stroke}%
\end{pgfscope}%
\begin{pgfscope}%
\pgfpathrectangle{\pgfqpoint{0.393053in}{0.375000in}}{\pgfqpoint{6.356833in}{5.175000in}}%
\pgfusepath{clip}%
\pgfsetbuttcap%
\pgfsetroundjoin%
\pgfsetlinewidth{1.003750pt}%
\definecolor{currentstroke}{rgb}{0.827451,0.827451,0.827451}%
\pgfsetstrokecolor{currentstroke}%
\pgfsetdash{}{0pt}%
\pgfpathmoveto{\pgfqpoint{0.765086in}{2.832905in}}%
\pgfpathcurveto{\pgfqpoint{0.776136in}{2.832905in}}{\pgfqpoint{0.786735in}{2.837296in}}{\pgfqpoint{0.794549in}{2.845109in}}%
\pgfpathcurveto{\pgfqpoint{0.802362in}{2.852923in}}{\pgfqpoint{0.806753in}{2.863522in}}{\pgfqpoint{0.806753in}{2.874572in}}%
\pgfpathcurveto{\pgfqpoint{0.806753in}{2.885622in}}{\pgfqpoint{0.802362in}{2.896221in}}{\pgfqpoint{0.794549in}{2.904035in}}%
\pgfpathcurveto{\pgfqpoint{0.786735in}{2.911849in}}{\pgfqpoint{0.776136in}{2.916239in}}{\pgfqpoint{0.765086in}{2.916239in}}%
\pgfpathcurveto{\pgfqpoint{0.754036in}{2.916239in}}{\pgfqpoint{0.743437in}{2.911849in}}{\pgfqpoint{0.735623in}{2.904035in}}%
\pgfpathcurveto{\pgfqpoint{0.727810in}{2.896221in}}{\pgfqpoint{0.723419in}{2.885622in}}{\pgfqpoint{0.723419in}{2.874572in}}%
\pgfpathcurveto{\pgfqpoint{0.723419in}{2.863522in}}{\pgfqpoint{0.727810in}{2.852923in}}{\pgfqpoint{0.735623in}{2.845109in}}%
\pgfpathcurveto{\pgfqpoint{0.743437in}{2.837296in}}{\pgfqpoint{0.754036in}{2.832905in}}{\pgfqpoint{0.765086in}{2.832905in}}%
\pgfpathlineto{\pgfqpoint{0.765086in}{2.832905in}}%
\pgfpathclose%
\pgfusepath{stroke}%
\end{pgfscope}%
\begin{pgfscope}%
\pgfpathrectangle{\pgfqpoint{0.393053in}{0.375000in}}{\pgfqpoint{6.356833in}{5.175000in}}%
\pgfusepath{clip}%
\pgfsetbuttcap%
\pgfsetroundjoin%
\pgfsetlinewidth{1.003750pt}%
\definecolor{currentstroke}{rgb}{0.827451,0.827451,0.827451}%
\pgfsetstrokecolor{currentstroke}%
\pgfsetdash{}{0pt}%
\pgfpathmoveto{\pgfqpoint{3.463913in}{0.638701in}}%
\pgfpathcurveto{\pgfqpoint{3.474963in}{0.638701in}}{\pgfqpoint{3.485562in}{0.643091in}}{\pgfqpoint{3.493376in}{0.650905in}}%
\pgfpathcurveto{\pgfqpoint{3.501190in}{0.658718in}}{\pgfqpoint{3.505580in}{0.669317in}}{\pgfqpoint{3.505580in}{0.680367in}}%
\pgfpathcurveto{\pgfqpoint{3.505580in}{0.691418in}}{\pgfqpoint{3.501190in}{0.702017in}}{\pgfqpoint{3.493376in}{0.709830in}}%
\pgfpathcurveto{\pgfqpoint{3.485562in}{0.717644in}}{\pgfqpoint{3.474963in}{0.722034in}}{\pgfqpoint{3.463913in}{0.722034in}}%
\pgfpathcurveto{\pgfqpoint{3.452863in}{0.722034in}}{\pgfqpoint{3.442264in}{0.717644in}}{\pgfqpoint{3.434451in}{0.709830in}}%
\pgfpathcurveto{\pgfqpoint{3.426637in}{0.702017in}}{\pgfqpoint{3.422247in}{0.691418in}}{\pgfqpoint{3.422247in}{0.680367in}}%
\pgfpathcurveto{\pgfqpoint{3.422247in}{0.669317in}}{\pgfqpoint{3.426637in}{0.658718in}}{\pgfqpoint{3.434451in}{0.650905in}}%
\pgfpathcurveto{\pgfqpoint{3.442264in}{0.643091in}}{\pgfqpoint{3.452863in}{0.638701in}}{\pgfqpoint{3.463913in}{0.638701in}}%
\pgfpathlineto{\pgfqpoint{3.463913in}{0.638701in}}%
\pgfpathclose%
\pgfusepath{stroke}%
\end{pgfscope}%
\begin{pgfscope}%
\pgfpathrectangle{\pgfqpoint{0.393053in}{0.375000in}}{\pgfqpoint{6.356833in}{5.175000in}}%
\pgfusepath{clip}%
\pgfsetbuttcap%
\pgfsetroundjoin%
\pgfsetlinewidth{1.003750pt}%
\definecolor{currentstroke}{rgb}{0.827451,0.827451,0.827451}%
\pgfsetstrokecolor{currentstroke}%
\pgfsetdash{}{0pt}%
\pgfpathmoveto{\pgfqpoint{1.375643in}{1.906128in}}%
\pgfpathcurveto{\pgfqpoint{1.386694in}{1.906128in}}{\pgfqpoint{1.397293in}{1.910518in}}{\pgfqpoint{1.405106in}{1.918331in}}%
\pgfpathcurveto{\pgfqpoint{1.412920in}{1.926145in}}{\pgfqpoint{1.417310in}{1.936744in}}{\pgfqpoint{1.417310in}{1.947794in}}%
\pgfpathcurveto{\pgfqpoint{1.417310in}{1.958844in}}{\pgfqpoint{1.412920in}{1.969443in}}{\pgfqpoint{1.405106in}{1.977257in}}%
\pgfpathcurveto{\pgfqpoint{1.397293in}{1.985071in}}{\pgfqpoint{1.386694in}{1.989461in}}{\pgfqpoint{1.375643in}{1.989461in}}%
\pgfpathcurveto{\pgfqpoint{1.364593in}{1.989461in}}{\pgfqpoint{1.353994in}{1.985071in}}{\pgfqpoint{1.346181in}{1.977257in}}%
\pgfpathcurveto{\pgfqpoint{1.338367in}{1.969443in}}{\pgfqpoint{1.333977in}{1.958844in}}{\pgfqpoint{1.333977in}{1.947794in}}%
\pgfpathcurveto{\pgfqpoint{1.333977in}{1.936744in}}{\pgfqpoint{1.338367in}{1.926145in}}{\pgfqpoint{1.346181in}{1.918331in}}%
\pgfpathcurveto{\pgfqpoint{1.353994in}{1.910518in}}{\pgfqpoint{1.364593in}{1.906128in}}{\pgfqpoint{1.375643in}{1.906128in}}%
\pgfpathlineto{\pgfqpoint{1.375643in}{1.906128in}}%
\pgfpathclose%
\pgfusepath{stroke}%
\end{pgfscope}%
\begin{pgfscope}%
\pgfpathrectangle{\pgfqpoint{0.393053in}{0.375000in}}{\pgfqpoint{6.356833in}{5.175000in}}%
\pgfusepath{clip}%
\pgfsetbuttcap%
\pgfsetroundjoin%
\pgfsetlinewidth{1.003750pt}%
\definecolor{currentstroke}{rgb}{0.827451,0.827451,0.827451}%
\pgfsetstrokecolor{currentstroke}%
\pgfsetdash{}{0pt}%
\pgfpathmoveto{\pgfqpoint{0.538696in}{3.440120in}}%
\pgfpathcurveto{\pgfqpoint{0.549746in}{3.440120in}}{\pgfqpoint{0.560345in}{3.444510in}}{\pgfqpoint{0.568159in}{3.452323in}}%
\pgfpathcurveto{\pgfqpoint{0.575972in}{3.460137in}}{\pgfqpoint{0.580363in}{3.470736in}}{\pgfqpoint{0.580363in}{3.481786in}}%
\pgfpathcurveto{\pgfqpoint{0.580363in}{3.492836in}}{\pgfqpoint{0.575972in}{3.503435in}}{\pgfqpoint{0.568159in}{3.511249in}}%
\pgfpathcurveto{\pgfqpoint{0.560345in}{3.519063in}}{\pgfqpoint{0.549746in}{3.523453in}}{\pgfqpoint{0.538696in}{3.523453in}}%
\pgfpathcurveto{\pgfqpoint{0.527646in}{3.523453in}}{\pgfqpoint{0.517047in}{3.519063in}}{\pgfqpoint{0.509233in}{3.511249in}}%
\pgfpathcurveto{\pgfqpoint{0.501420in}{3.503435in}}{\pgfqpoint{0.497029in}{3.492836in}}{\pgfqpoint{0.497029in}{3.481786in}}%
\pgfpathcurveto{\pgfqpoint{0.497029in}{3.470736in}}{\pgfqpoint{0.501420in}{3.460137in}}{\pgfqpoint{0.509233in}{3.452323in}}%
\pgfpathcurveto{\pgfqpoint{0.517047in}{3.444510in}}{\pgfqpoint{0.527646in}{3.440120in}}{\pgfqpoint{0.538696in}{3.440120in}}%
\pgfpathlineto{\pgfqpoint{0.538696in}{3.440120in}}%
\pgfpathclose%
\pgfusepath{stroke}%
\end{pgfscope}%
\begin{pgfscope}%
\pgfpathrectangle{\pgfqpoint{0.393053in}{0.375000in}}{\pgfqpoint{6.356833in}{5.175000in}}%
\pgfusepath{clip}%
\pgfsetbuttcap%
\pgfsetroundjoin%
\pgfsetlinewidth{1.003750pt}%
\definecolor{currentstroke}{rgb}{0.827451,0.827451,0.827451}%
\pgfsetstrokecolor{currentstroke}%
\pgfsetdash{}{0pt}%
\pgfpathmoveto{\pgfqpoint{3.397541in}{0.662535in}}%
\pgfpathcurveto{\pgfqpoint{3.408591in}{0.662535in}}{\pgfqpoint{3.419190in}{0.666925in}}{\pgfqpoint{3.427004in}{0.674738in}}%
\pgfpathcurveto{\pgfqpoint{3.434817in}{0.682552in}}{\pgfqpoint{3.439208in}{0.693151in}}{\pgfqpoint{3.439208in}{0.704201in}}%
\pgfpathcurveto{\pgfqpoint{3.439208in}{0.715251in}}{\pgfqpoint{3.434817in}{0.725850in}}{\pgfqpoint{3.427004in}{0.733664in}}%
\pgfpathcurveto{\pgfqpoint{3.419190in}{0.741478in}}{\pgfqpoint{3.408591in}{0.745868in}}{\pgfqpoint{3.397541in}{0.745868in}}%
\pgfpathcurveto{\pgfqpoint{3.386491in}{0.745868in}}{\pgfqpoint{3.375892in}{0.741478in}}{\pgfqpoint{3.368078in}{0.733664in}}%
\pgfpathcurveto{\pgfqpoint{3.360265in}{0.725850in}}{\pgfqpoint{3.355874in}{0.715251in}}{\pgfqpoint{3.355874in}{0.704201in}}%
\pgfpathcurveto{\pgfqpoint{3.355874in}{0.693151in}}{\pgfqpoint{3.360265in}{0.682552in}}{\pgfqpoint{3.368078in}{0.674738in}}%
\pgfpathcurveto{\pgfqpoint{3.375892in}{0.666925in}}{\pgfqpoint{3.386491in}{0.662535in}}{\pgfqpoint{3.397541in}{0.662535in}}%
\pgfpathlineto{\pgfqpoint{3.397541in}{0.662535in}}%
\pgfpathclose%
\pgfusepath{stroke}%
\end{pgfscope}%
\begin{pgfscope}%
\pgfpathrectangle{\pgfqpoint{0.393053in}{0.375000in}}{\pgfqpoint{6.356833in}{5.175000in}}%
\pgfusepath{clip}%
\pgfsetbuttcap%
\pgfsetroundjoin%
\pgfsetlinewidth{1.003750pt}%
\definecolor{currentstroke}{rgb}{0.827451,0.827451,0.827451}%
\pgfsetstrokecolor{currentstroke}%
\pgfsetdash{}{0pt}%
\pgfpathmoveto{\pgfqpoint{0.531017in}{3.497297in}}%
\pgfpathcurveto{\pgfqpoint{0.542067in}{3.497297in}}{\pgfqpoint{0.552666in}{3.501687in}}{\pgfqpoint{0.560480in}{3.509501in}}%
\pgfpathcurveto{\pgfqpoint{0.568293in}{3.517314in}}{\pgfqpoint{0.572683in}{3.527913in}}{\pgfqpoint{0.572683in}{3.538964in}}%
\pgfpathcurveto{\pgfqpoint{0.572683in}{3.550014in}}{\pgfqpoint{0.568293in}{3.560613in}}{\pgfqpoint{0.560480in}{3.568426in}}%
\pgfpathcurveto{\pgfqpoint{0.552666in}{3.576240in}}{\pgfqpoint{0.542067in}{3.580630in}}{\pgfqpoint{0.531017in}{3.580630in}}%
\pgfpathcurveto{\pgfqpoint{0.519967in}{3.580630in}}{\pgfqpoint{0.509368in}{3.576240in}}{\pgfqpoint{0.501554in}{3.568426in}}%
\pgfpathcurveto{\pgfqpoint{0.493740in}{3.560613in}}{\pgfqpoint{0.489350in}{3.550014in}}{\pgfqpoint{0.489350in}{3.538964in}}%
\pgfpathcurveto{\pgfqpoint{0.489350in}{3.527913in}}{\pgfqpoint{0.493740in}{3.517314in}}{\pgfqpoint{0.501554in}{3.509501in}}%
\pgfpathcurveto{\pgfqpoint{0.509368in}{3.501687in}}{\pgfqpoint{0.519967in}{3.497297in}}{\pgfqpoint{0.531017in}{3.497297in}}%
\pgfpathlineto{\pgfqpoint{0.531017in}{3.497297in}}%
\pgfpathclose%
\pgfusepath{stroke}%
\end{pgfscope}%
\begin{pgfscope}%
\pgfpathrectangle{\pgfqpoint{0.393053in}{0.375000in}}{\pgfqpoint{6.356833in}{5.175000in}}%
\pgfusepath{clip}%
\pgfsetbuttcap%
\pgfsetroundjoin%
\pgfsetlinewidth{1.003750pt}%
\definecolor{currentstroke}{rgb}{0.827451,0.827451,0.827451}%
\pgfsetstrokecolor{currentstroke}%
\pgfsetdash{}{0pt}%
\pgfpathmoveto{\pgfqpoint{5.473312in}{0.379973in}}%
\pgfpathcurveto{\pgfqpoint{5.484362in}{0.379973in}}{\pgfqpoint{5.494961in}{0.384363in}}{\pgfqpoint{5.502774in}{0.392177in}}%
\pgfpathcurveto{\pgfqpoint{5.510588in}{0.399990in}}{\pgfqpoint{5.514978in}{0.410589in}}{\pgfqpoint{5.514978in}{0.421639in}}%
\pgfpathcurveto{\pgfqpoint{5.514978in}{0.432689in}}{\pgfqpoint{5.510588in}{0.443289in}}{\pgfqpoint{5.502774in}{0.451102in}}%
\pgfpathcurveto{\pgfqpoint{5.494961in}{0.458916in}}{\pgfqpoint{5.484362in}{0.463306in}}{\pgfqpoint{5.473312in}{0.463306in}}%
\pgfpathcurveto{\pgfqpoint{5.462261in}{0.463306in}}{\pgfqpoint{5.451662in}{0.458916in}}{\pgfqpoint{5.443849in}{0.451102in}}%
\pgfpathcurveto{\pgfqpoint{5.436035in}{0.443289in}}{\pgfqpoint{5.431645in}{0.432689in}}{\pgfqpoint{5.431645in}{0.421639in}}%
\pgfpathcurveto{\pgfqpoint{5.431645in}{0.410589in}}{\pgfqpoint{5.436035in}{0.399990in}}{\pgfqpoint{5.443849in}{0.392177in}}%
\pgfpathcurveto{\pgfqpoint{5.451662in}{0.384363in}}{\pgfqpoint{5.462261in}{0.379973in}}{\pgfqpoint{5.473312in}{0.379973in}}%
\pgfpathlineto{\pgfqpoint{5.473312in}{0.379973in}}%
\pgfpathclose%
\pgfusepath{stroke}%
\end{pgfscope}%
\begin{pgfscope}%
\pgfpathrectangle{\pgfqpoint{0.393053in}{0.375000in}}{\pgfqpoint{6.356833in}{5.175000in}}%
\pgfusepath{clip}%
\pgfsetbuttcap%
\pgfsetroundjoin%
\pgfsetlinewidth{1.003750pt}%
\definecolor{currentstroke}{rgb}{0.827451,0.827451,0.827451}%
\pgfsetstrokecolor{currentstroke}%
\pgfsetdash{}{0pt}%
\pgfpathmoveto{\pgfqpoint{4.727163in}{0.448145in}}%
\pgfpathcurveto{\pgfqpoint{4.738213in}{0.448145in}}{\pgfqpoint{4.748812in}{0.452535in}}{\pgfqpoint{4.756626in}{0.460349in}}%
\pgfpathcurveto{\pgfqpoint{4.764439in}{0.468162in}}{\pgfqpoint{4.768830in}{0.478761in}}{\pgfqpoint{4.768830in}{0.489811in}}%
\pgfpathcurveto{\pgfqpoint{4.768830in}{0.500861in}}{\pgfqpoint{4.764439in}{0.511460in}}{\pgfqpoint{4.756626in}{0.519274in}}%
\pgfpathcurveto{\pgfqpoint{4.748812in}{0.527088in}}{\pgfqpoint{4.738213in}{0.531478in}}{\pgfqpoint{4.727163in}{0.531478in}}%
\pgfpathcurveto{\pgfqpoint{4.716113in}{0.531478in}}{\pgfqpoint{4.705514in}{0.527088in}}{\pgfqpoint{4.697700in}{0.519274in}}%
\pgfpathcurveto{\pgfqpoint{4.689887in}{0.511460in}}{\pgfqpoint{4.685496in}{0.500861in}}{\pgfqpoint{4.685496in}{0.489811in}}%
\pgfpathcurveto{\pgfqpoint{4.685496in}{0.478761in}}{\pgfqpoint{4.689887in}{0.468162in}}{\pgfqpoint{4.697700in}{0.460349in}}%
\pgfpathcurveto{\pgfqpoint{4.705514in}{0.452535in}}{\pgfqpoint{4.716113in}{0.448145in}}{\pgfqpoint{4.727163in}{0.448145in}}%
\pgfpathlineto{\pgfqpoint{4.727163in}{0.448145in}}%
\pgfpathclose%
\pgfusepath{stroke}%
\end{pgfscope}%
\begin{pgfscope}%
\pgfpathrectangle{\pgfqpoint{0.393053in}{0.375000in}}{\pgfqpoint{6.356833in}{5.175000in}}%
\pgfusepath{clip}%
\pgfsetbuttcap%
\pgfsetroundjoin%
\pgfsetlinewidth{1.003750pt}%
\definecolor{currentstroke}{rgb}{0.827451,0.827451,0.827451}%
\pgfsetstrokecolor{currentstroke}%
\pgfsetdash{}{0pt}%
\pgfpathmoveto{\pgfqpoint{2.253356in}{1.299787in}}%
\pgfpathcurveto{\pgfqpoint{2.264406in}{1.299787in}}{\pgfqpoint{2.275005in}{1.304178in}}{\pgfqpoint{2.282819in}{1.311991in}}%
\pgfpathcurveto{\pgfqpoint{2.290632in}{1.319805in}}{\pgfqpoint{2.295023in}{1.330404in}}{\pgfqpoint{2.295023in}{1.341454in}}%
\pgfpathcurveto{\pgfqpoint{2.295023in}{1.352504in}}{\pgfqpoint{2.290632in}{1.363103in}}{\pgfqpoint{2.282819in}{1.370917in}}%
\pgfpathcurveto{\pgfqpoint{2.275005in}{1.378730in}}{\pgfqpoint{2.264406in}{1.383121in}}{\pgfqpoint{2.253356in}{1.383121in}}%
\pgfpathcurveto{\pgfqpoint{2.242306in}{1.383121in}}{\pgfqpoint{2.231707in}{1.378730in}}{\pgfqpoint{2.223893in}{1.370917in}}%
\pgfpathcurveto{\pgfqpoint{2.216080in}{1.363103in}}{\pgfqpoint{2.211689in}{1.352504in}}{\pgfqpoint{2.211689in}{1.341454in}}%
\pgfpathcurveto{\pgfqpoint{2.211689in}{1.330404in}}{\pgfqpoint{2.216080in}{1.319805in}}{\pgfqpoint{2.223893in}{1.311991in}}%
\pgfpathcurveto{\pgfqpoint{2.231707in}{1.304178in}}{\pgfqpoint{2.242306in}{1.299787in}}{\pgfqpoint{2.253356in}{1.299787in}}%
\pgfpathlineto{\pgfqpoint{2.253356in}{1.299787in}}%
\pgfpathclose%
\pgfusepath{stroke}%
\end{pgfscope}%
\begin{pgfscope}%
\pgfpathrectangle{\pgfqpoint{0.393053in}{0.375000in}}{\pgfqpoint{6.356833in}{5.175000in}}%
\pgfusepath{clip}%
\pgfsetbuttcap%
\pgfsetroundjoin%
\pgfsetlinewidth{1.003750pt}%
\definecolor{currentstroke}{rgb}{0.827451,0.827451,0.827451}%
\pgfsetstrokecolor{currentstroke}%
\pgfsetdash{}{0pt}%
\pgfpathmoveto{\pgfqpoint{2.574273in}{1.118285in}}%
\pgfpathcurveto{\pgfqpoint{2.585323in}{1.118285in}}{\pgfqpoint{2.595922in}{1.122676in}}{\pgfqpoint{2.603735in}{1.130489in}}%
\pgfpathcurveto{\pgfqpoint{2.611549in}{1.138303in}}{\pgfqpoint{2.615939in}{1.148902in}}{\pgfqpoint{2.615939in}{1.159952in}}%
\pgfpathcurveto{\pgfqpoint{2.615939in}{1.171002in}}{\pgfqpoint{2.611549in}{1.181601in}}{\pgfqpoint{2.603735in}{1.189415in}}%
\pgfpathcurveto{\pgfqpoint{2.595922in}{1.197228in}}{\pgfqpoint{2.585323in}{1.201619in}}{\pgfqpoint{2.574273in}{1.201619in}}%
\pgfpathcurveto{\pgfqpoint{2.563222in}{1.201619in}}{\pgfqpoint{2.552623in}{1.197228in}}{\pgfqpoint{2.544810in}{1.189415in}}%
\pgfpathcurveto{\pgfqpoint{2.536996in}{1.181601in}}{\pgfqpoint{2.532606in}{1.171002in}}{\pgfqpoint{2.532606in}{1.159952in}}%
\pgfpathcurveto{\pgfqpoint{2.532606in}{1.148902in}}{\pgfqpoint{2.536996in}{1.138303in}}{\pgfqpoint{2.544810in}{1.130489in}}%
\pgfpathcurveto{\pgfqpoint{2.552623in}{1.122676in}}{\pgfqpoint{2.563222in}{1.118285in}}{\pgfqpoint{2.574273in}{1.118285in}}%
\pgfpathlineto{\pgfqpoint{2.574273in}{1.118285in}}%
\pgfpathclose%
\pgfusepath{stroke}%
\end{pgfscope}%
\begin{pgfscope}%
\pgfpathrectangle{\pgfqpoint{0.393053in}{0.375000in}}{\pgfqpoint{6.356833in}{5.175000in}}%
\pgfusepath{clip}%
\pgfsetbuttcap%
\pgfsetroundjoin%
\pgfsetlinewidth{1.003750pt}%
\definecolor{currentstroke}{rgb}{0.827451,0.827451,0.827451}%
\pgfsetstrokecolor{currentstroke}%
\pgfsetdash{}{0pt}%
\pgfpathmoveto{\pgfqpoint{1.092331in}{2.381241in}}%
\pgfpathcurveto{\pgfqpoint{1.103381in}{2.381241in}}{\pgfqpoint{1.113980in}{2.385631in}}{\pgfqpoint{1.121794in}{2.393444in}}%
\pgfpathcurveto{\pgfqpoint{1.129608in}{2.401258in}}{\pgfqpoint{1.133998in}{2.411857in}}{\pgfqpoint{1.133998in}{2.422907in}}%
\pgfpathcurveto{\pgfqpoint{1.133998in}{2.433957in}}{\pgfqpoint{1.129608in}{2.444556in}}{\pgfqpoint{1.121794in}{2.452370in}}%
\pgfpathcurveto{\pgfqpoint{1.113980in}{2.460184in}}{\pgfqpoint{1.103381in}{2.464574in}}{\pgfqpoint{1.092331in}{2.464574in}}%
\pgfpathcurveto{\pgfqpoint{1.081281in}{2.464574in}}{\pgfqpoint{1.070682in}{2.460184in}}{\pgfqpoint{1.062868in}{2.452370in}}%
\pgfpathcurveto{\pgfqpoint{1.055055in}{2.444556in}}{\pgfqpoint{1.050664in}{2.433957in}}{\pgfqpoint{1.050664in}{2.422907in}}%
\pgfpathcurveto{\pgfqpoint{1.050664in}{2.411857in}}{\pgfqpoint{1.055055in}{2.401258in}}{\pgfqpoint{1.062868in}{2.393444in}}%
\pgfpathcurveto{\pgfqpoint{1.070682in}{2.385631in}}{\pgfqpoint{1.081281in}{2.381241in}}{\pgfqpoint{1.092331in}{2.381241in}}%
\pgfpathlineto{\pgfqpoint{1.092331in}{2.381241in}}%
\pgfpathclose%
\pgfusepath{stroke}%
\end{pgfscope}%
\begin{pgfscope}%
\pgfpathrectangle{\pgfqpoint{0.393053in}{0.375000in}}{\pgfqpoint{6.356833in}{5.175000in}}%
\pgfusepath{clip}%
\pgfsetbuttcap%
\pgfsetroundjoin%
\pgfsetlinewidth{1.003750pt}%
\definecolor{currentstroke}{rgb}{0.827451,0.827451,0.827451}%
\pgfsetstrokecolor{currentstroke}%
\pgfsetdash{}{0pt}%
\pgfpathmoveto{\pgfqpoint{0.669503in}{3.263449in}}%
\pgfpathcurveto{\pgfqpoint{0.680554in}{3.263449in}}{\pgfqpoint{0.691153in}{3.267840in}}{\pgfqpoint{0.698966in}{3.275653in}}%
\pgfpathcurveto{\pgfqpoint{0.706780in}{3.283467in}}{\pgfqpoint{0.711170in}{3.294066in}}{\pgfqpoint{0.711170in}{3.305116in}}%
\pgfpathcurveto{\pgfqpoint{0.711170in}{3.316166in}}{\pgfqpoint{0.706780in}{3.326765in}}{\pgfqpoint{0.698966in}{3.334579in}}%
\pgfpathcurveto{\pgfqpoint{0.691153in}{3.342392in}}{\pgfqpoint{0.680554in}{3.346783in}}{\pgfqpoint{0.669503in}{3.346783in}}%
\pgfpathcurveto{\pgfqpoint{0.658453in}{3.346783in}}{\pgfqpoint{0.647854in}{3.342392in}}{\pgfqpoint{0.640041in}{3.334579in}}%
\pgfpathcurveto{\pgfqpoint{0.632227in}{3.326765in}}{\pgfqpoint{0.627837in}{3.316166in}}{\pgfqpoint{0.627837in}{3.305116in}}%
\pgfpathcurveto{\pgfqpoint{0.627837in}{3.294066in}}{\pgfqpoint{0.632227in}{3.283467in}}{\pgfqpoint{0.640041in}{3.275653in}}%
\pgfpathcurveto{\pgfqpoint{0.647854in}{3.267840in}}{\pgfqpoint{0.658453in}{3.263449in}}{\pgfqpoint{0.669503in}{3.263449in}}%
\pgfpathlineto{\pgfqpoint{0.669503in}{3.263449in}}%
\pgfpathclose%
\pgfusepath{stroke}%
\end{pgfscope}%
\begin{pgfscope}%
\pgfpathrectangle{\pgfqpoint{0.393053in}{0.375000in}}{\pgfqpoint{6.356833in}{5.175000in}}%
\pgfusepath{clip}%
\pgfsetbuttcap%
\pgfsetroundjoin%
\pgfsetlinewidth{1.003750pt}%
\definecolor{currentstroke}{rgb}{0.827451,0.827451,0.827451}%
\pgfsetstrokecolor{currentstroke}%
\pgfsetdash{}{0pt}%
\pgfpathmoveto{\pgfqpoint{2.834525in}{0.919593in}}%
\pgfpathcurveto{\pgfqpoint{2.845575in}{0.919593in}}{\pgfqpoint{2.856174in}{0.923983in}}{\pgfqpoint{2.863987in}{0.931797in}}%
\pgfpathcurveto{\pgfqpoint{2.871801in}{0.939610in}}{\pgfqpoint{2.876191in}{0.950209in}}{\pgfqpoint{2.876191in}{0.961260in}}%
\pgfpathcurveto{\pgfqpoint{2.876191in}{0.972310in}}{\pgfqpoint{2.871801in}{0.982909in}}{\pgfqpoint{2.863987in}{0.990722in}}%
\pgfpathcurveto{\pgfqpoint{2.856174in}{0.998536in}}{\pgfqpoint{2.845575in}{1.002926in}}{\pgfqpoint{2.834525in}{1.002926in}}%
\pgfpathcurveto{\pgfqpoint{2.823475in}{1.002926in}}{\pgfqpoint{2.812876in}{0.998536in}}{\pgfqpoint{2.805062in}{0.990722in}}%
\pgfpathcurveto{\pgfqpoint{2.797248in}{0.982909in}}{\pgfqpoint{2.792858in}{0.972310in}}{\pgfqpoint{2.792858in}{0.961260in}}%
\pgfpathcurveto{\pgfqpoint{2.792858in}{0.950209in}}{\pgfqpoint{2.797248in}{0.939610in}}{\pgfqpoint{2.805062in}{0.931797in}}%
\pgfpathcurveto{\pgfqpoint{2.812876in}{0.923983in}}{\pgfqpoint{2.823475in}{0.919593in}}{\pgfqpoint{2.834525in}{0.919593in}}%
\pgfpathlineto{\pgfqpoint{2.834525in}{0.919593in}}%
\pgfpathclose%
\pgfusepath{stroke}%
\end{pgfscope}%
\begin{pgfscope}%
\pgfpathrectangle{\pgfqpoint{0.393053in}{0.375000in}}{\pgfqpoint{6.356833in}{5.175000in}}%
\pgfusepath{clip}%
\pgfsetbuttcap%
\pgfsetroundjoin%
\pgfsetlinewidth{1.003750pt}%
\definecolor{currentstroke}{rgb}{0.827451,0.827451,0.827451}%
\pgfsetstrokecolor{currentstroke}%
\pgfsetdash{}{0pt}%
\pgfpathmoveto{\pgfqpoint{4.229679in}{0.486406in}}%
\pgfpathcurveto{\pgfqpoint{4.240729in}{0.486406in}}{\pgfqpoint{4.251328in}{0.490796in}}{\pgfqpoint{4.259142in}{0.498610in}}%
\pgfpathcurveto{\pgfqpoint{4.266956in}{0.506423in}}{\pgfqpoint{4.271346in}{0.517022in}}{\pgfqpoint{4.271346in}{0.528072in}}%
\pgfpathcurveto{\pgfqpoint{4.271346in}{0.539123in}}{\pgfqpoint{4.266956in}{0.549722in}}{\pgfqpoint{4.259142in}{0.557535in}}%
\pgfpathcurveto{\pgfqpoint{4.251328in}{0.565349in}}{\pgfqpoint{4.240729in}{0.569739in}}{\pgfqpoint{4.229679in}{0.569739in}}%
\pgfpathcurveto{\pgfqpoint{4.218629in}{0.569739in}}{\pgfqpoint{4.208030in}{0.565349in}}{\pgfqpoint{4.200216in}{0.557535in}}%
\pgfpathcurveto{\pgfqpoint{4.192403in}{0.549722in}}{\pgfqpoint{4.188013in}{0.539123in}}{\pgfqpoint{4.188013in}{0.528072in}}%
\pgfpathcurveto{\pgfqpoint{4.188013in}{0.517022in}}{\pgfqpoint{4.192403in}{0.506423in}}{\pgfqpoint{4.200216in}{0.498610in}}%
\pgfpathcurveto{\pgfqpoint{4.208030in}{0.490796in}}{\pgfqpoint{4.218629in}{0.486406in}}{\pgfqpoint{4.229679in}{0.486406in}}%
\pgfpathlineto{\pgfqpoint{4.229679in}{0.486406in}}%
\pgfpathclose%
\pgfusepath{stroke}%
\end{pgfscope}%
\begin{pgfscope}%
\pgfpathrectangle{\pgfqpoint{0.393053in}{0.375000in}}{\pgfqpoint{6.356833in}{5.175000in}}%
\pgfusepath{clip}%
\pgfsetbuttcap%
\pgfsetroundjoin%
\pgfsetlinewidth{1.003750pt}%
\definecolor{currentstroke}{rgb}{0.827451,0.827451,0.827451}%
\pgfsetstrokecolor{currentstroke}%
\pgfsetdash{}{0pt}%
\pgfpathmoveto{\pgfqpoint{0.396398in}{4.410130in}}%
\pgfpathcurveto{\pgfqpoint{0.407448in}{4.410130in}}{\pgfqpoint{0.418047in}{4.414521in}}{\pgfqpoint{0.425861in}{4.422334in}}%
\pgfpathcurveto{\pgfqpoint{0.433674in}{4.430148in}}{\pgfqpoint{0.438064in}{4.440747in}}{\pgfqpoint{0.438064in}{4.451797in}}%
\pgfpathcurveto{\pgfqpoint{0.438064in}{4.462847in}}{\pgfqpoint{0.433674in}{4.473446in}}{\pgfqpoint{0.425861in}{4.481260in}}%
\pgfpathcurveto{\pgfqpoint{0.418047in}{4.489073in}}{\pgfqpoint{0.407448in}{4.493464in}}{\pgfqpoint{0.396398in}{4.493464in}}%
\pgfpathcurveto{\pgfqpoint{0.385348in}{4.493464in}}{\pgfqpoint{0.374749in}{4.489073in}}{\pgfqpoint{0.366935in}{4.481260in}}%
\pgfpathcurveto{\pgfqpoint{0.359121in}{4.473446in}}{\pgfqpoint{0.354731in}{4.462847in}}{\pgfqpoint{0.354731in}{4.451797in}}%
\pgfpathcurveto{\pgfqpoint{0.354731in}{4.440747in}}{\pgfqpoint{0.359121in}{4.430148in}}{\pgfqpoint{0.366935in}{4.422334in}}%
\pgfpathcurveto{\pgfqpoint{0.374749in}{4.414521in}}{\pgfqpoint{0.385348in}{4.410130in}}{\pgfqpoint{0.396398in}{4.410130in}}%
\pgfpathlineto{\pgfqpoint{0.396398in}{4.410130in}}%
\pgfpathclose%
\pgfusepath{stroke}%
\end{pgfscope}%
\begin{pgfscope}%
\pgfpathrectangle{\pgfqpoint{0.393053in}{0.375000in}}{\pgfqpoint{6.356833in}{5.175000in}}%
\pgfusepath{clip}%
\pgfsetbuttcap%
\pgfsetroundjoin%
\pgfsetlinewidth{1.003750pt}%
\definecolor{currentstroke}{rgb}{0.827451,0.827451,0.827451}%
\pgfsetstrokecolor{currentstroke}%
\pgfsetdash{}{0pt}%
\pgfpathmoveto{\pgfqpoint{5.388083in}{0.344848in}}%
\pgfpathcurveto{\pgfqpoint{5.399133in}{0.344848in}}{\pgfqpoint{5.409732in}{0.349238in}}{\pgfqpoint{5.417546in}{0.357052in}}%
\pgfpathcurveto{\pgfqpoint{5.425360in}{0.364865in}}{\pgfqpoint{5.429750in}{0.375464in}}{\pgfqpoint{5.429750in}{0.386514in}}%
\pgfpathcurveto{\pgfqpoint{5.429750in}{0.397565in}}{\pgfqpoint{5.425360in}{0.408164in}}{\pgfqpoint{5.417546in}{0.415977in}}%
\pgfpathcurveto{\pgfqpoint{5.409732in}{0.423791in}}{\pgfqpoint{5.399133in}{0.428181in}}{\pgfqpoint{5.388083in}{0.428181in}}%
\pgfpathcurveto{\pgfqpoint{5.377033in}{0.428181in}}{\pgfqpoint{5.366434in}{0.423791in}}{\pgfqpoint{5.358620in}{0.415977in}}%
\pgfpathcurveto{\pgfqpoint{5.350807in}{0.408164in}}{\pgfqpoint{5.346417in}{0.397565in}}{\pgfqpoint{5.346417in}{0.386514in}}%
\pgfpathcurveto{\pgfqpoint{5.346417in}{0.375464in}}{\pgfqpoint{5.350807in}{0.364865in}}{\pgfqpoint{5.358620in}{0.357052in}}%
\pgfpathcurveto{\pgfqpoint{5.366434in}{0.349238in}}{\pgfqpoint{5.377033in}{0.344848in}}{\pgfqpoint{5.388083in}{0.344848in}}%
\pgfusepath{stroke}%
\end{pgfscope}%
\begin{pgfscope}%
\pgfpathrectangle{\pgfqpoint{0.393053in}{0.375000in}}{\pgfqpoint{6.356833in}{5.175000in}}%
\pgfusepath{clip}%
\pgfsetbuttcap%
\pgfsetroundjoin%
\pgfsetlinewidth{1.003750pt}%
\definecolor{currentstroke}{rgb}{0.827451,0.827451,0.827451}%
\pgfsetstrokecolor{currentstroke}%
\pgfsetdash{}{0pt}%
\pgfpathmoveto{\pgfqpoint{3.954008in}{0.515189in}}%
\pgfpathcurveto{\pgfqpoint{3.965058in}{0.515189in}}{\pgfqpoint{3.975657in}{0.519579in}}{\pgfqpoint{3.983471in}{0.527393in}}%
\pgfpathcurveto{\pgfqpoint{3.991284in}{0.535206in}}{\pgfqpoint{3.995675in}{0.545805in}}{\pgfqpoint{3.995675in}{0.556855in}}%
\pgfpathcurveto{\pgfqpoint{3.995675in}{0.567905in}}{\pgfqpoint{3.991284in}{0.578504in}}{\pgfqpoint{3.983471in}{0.586318in}}%
\pgfpathcurveto{\pgfqpoint{3.975657in}{0.594132in}}{\pgfqpoint{3.965058in}{0.598522in}}{\pgfqpoint{3.954008in}{0.598522in}}%
\pgfpathcurveto{\pgfqpoint{3.942958in}{0.598522in}}{\pgfqpoint{3.932359in}{0.594132in}}{\pgfqpoint{3.924545in}{0.586318in}}%
\pgfpathcurveto{\pgfqpoint{3.916732in}{0.578504in}}{\pgfqpoint{3.912341in}{0.567905in}}{\pgfqpoint{3.912341in}{0.556855in}}%
\pgfpathcurveto{\pgfqpoint{3.912341in}{0.545805in}}{\pgfqpoint{3.916732in}{0.535206in}}{\pgfqpoint{3.924545in}{0.527393in}}%
\pgfpathcurveto{\pgfqpoint{3.932359in}{0.519579in}}{\pgfqpoint{3.942958in}{0.515189in}}{\pgfqpoint{3.954008in}{0.515189in}}%
\pgfpathlineto{\pgfqpoint{3.954008in}{0.515189in}}%
\pgfpathclose%
\pgfusepath{stroke}%
\end{pgfscope}%
\begin{pgfscope}%
\pgfpathrectangle{\pgfqpoint{0.393053in}{0.375000in}}{\pgfqpoint{6.356833in}{5.175000in}}%
\pgfusepath{clip}%
\pgfsetbuttcap%
\pgfsetroundjoin%
\pgfsetlinewidth{1.003750pt}%
\definecolor{currentstroke}{rgb}{0.827451,0.827451,0.827451}%
\pgfsetstrokecolor{currentstroke}%
\pgfsetdash{}{0pt}%
\pgfpathmoveto{\pgfqpoint{0.418752in}{4.107793in}}%
\pgfpathcurveto{\pgfqpoint{0.429802in}{4.107793in}}{\pgfqpoint{0.440401in}{4.112183in}}{\pgfqpoint{0.448215in}{4.119996in}}%
\pgfpathcurveto{\pgfqpoint{0.456028in}{4.127810in}}{\pgfqpoint{0.460419in}{4.138409in}}{\pgfqpoint{0.460419in}{4.149459in}}%
\pgfpathcurveto{\pgfqpoint{0.460419in}{4.160509in}}{\pgfqpoint{0.456028in}{4.171108in}}{\pgfqpoint{0.448215in}{4.178922in}}%
\pgfpathcurveto{\pgfqpoint{0.440401in}{4.186736in}}{\pgfqpoint{0.429802in}{4.191126in}}{\pgfqpoint{0.418752in}{4.191126in}}%
\pgfpathcurveto{\pgfqpoint{0.407702in}{4.191126in}}{\pgfqpoint{0.397103in}{4.186736in}}{\pgfqpoint{0.389289in}{4.178922in}}%
\pgfpathcurveto{\pgfqpoint{0.381476in}{4.171108in}}{\pgfqpoint{0.377085in}{4.160509in}}{\pgfqpoint{0.377085in}{4.149459in}}%
\pgfpathcurveto{\pgfqpoint{0.377085in}{4.138409in}}{\pgfqpoint{0.381476in}{4.127810in}}{\pgfqpoint{0.389289in}{4.119996in}}%
\pgfpathcurveto{\pgfqpoint{0.397103in}{4.112183in}}{\pgfqpoint{0.407702in}{4.107793in}}{\pgfqpoint{0.418752in}{4.107793in}}%
\pgfpathlineto{\pgfqpoint{0.418752in}{4.107793in}}%
\pgfpathclose%
\pgfusepath{stroke}%
\end{pgfscope}%
\begin{pgfscope}%
\pgfpathrectangle{\pgfqpoint{0.393053in}{0.375000in}}{\pgfqpoint{6.356833in}{5.175000in}}%
\pgfusepath{clip}%
\pgfsetbuttcap%
\pgfsetroundjoin%
\pgfsetlinewidth{1.003750pt}%
\definecolor{currentstroke}{rgb}{0.827451,0.827451,0.827451}%
\pgfsetstrokecolor{currentstroke}%
\pgfsetdash{}{0pt}%
\pgfpathmoveto{\pgfqpoint{0.672197in}{3.023099in}}%
\pgfpathcurveto{\pgfqpoint{0.683247in}{3.023099in}}{\pgfqpoint{0.693846in}{3.027489in}}{\pgfqpoint{0.701660in}{3.035303in}}%
\pgfpathcurveto{\pgfqpoint{0.709473in}{3.043116in}}{\pgfqpoint{0.713863in}{3.053715in}}{\pgfqpoint{0.713863in}{3.064766in}}%
\pgfpathcurveto{\pgfqpoint{0.713863in}{3.075816in}}{\pgfqpoint{0.709473in}{3.086415in}}{\pgfqpoint{0.701660in}{3.094228in}}%
\pgfpathcurveto{\pgfqpoint{0.693846in}{3.102042in}}{\pgfqpoint{0.683247in}{3.106432in}}{\pgfqpoint{0.672197in}{3.106432in}}%
\pgfpathcurveto{\pgfqpoint{0.661147in}{3.106432in}}{\pgfqpoint{0.650548in}{3.102042in}}{\pgfqpoint{0.642734in}{3.094228in}}%
\pgfpathcurveto{\pgfqpoint{0.634920in}{3.086415in}}{\pgfqpoint{0.630530in}{3.075816in}}{\pgfqpoint{0.630530in}{3.064766in}}%
\pgfpathcurveto{\pgfqpoint{0.630530in}{3.053715in}}{\pgfqpoint{0.634920in}{3.043116in}}{\pgfqpoint{0.642734in}{3.035303in}}%
\pgfpathcurveto{\pgfqpoint{0.650548in}{3.027489in}}{\pgfqpoint{0.661147in}{3.023099in}}{\pgfqpoint{0.672197in}{3.023099in}}%
\pgfpathlineto{\pgfqpoint{0.672197in}{3.023099in}}%
\pgfpathclose%
\pgfusepath{stroke}%
\end{pgfscope}%
\begin{pgfscope}%
\pgfpathrectangle{\pgfqpoint{0.393053in}{0.375000in}}{\pgfqpoint{6.356833in}{5.175000in}}%
\pgfusepath{clip}%
\pgfsetbuttcap%
\pgfsetroundjoin%
\pgfsetlinewidth{1.003750pt}%
\definecolor{currentstroke}{rgb}{0.827451,0.827451,0.827451}%
\pgfsetstrokecolor{currentstroke}%
\pgfsetdash{}{0pt}%
\pgfpathmoveto{\pgfqpoint{5.142473in}{0.360415in}}%
\pgfpathcurveto{\pgfqpoint{5.153523in}{0.360415in}}{\pgfqpoint{5.164122in}{0.364805in}}{\pgfqpoint{5.171936in}{0.372618in}}%
\pgfpathcurveto{\pgfqpoint{5.179749in}{0.380432in}}{\pgfqpoint{5.184140in}{0.391031in}}{\pgfqpoint{5.184140in}{0.402081in}}%
\pgfpathcurveto{\pgfqpoint{5.184140in}{0.413131in}}{\pgfqpoint{5.179749in}{0.423730in}}{\pgfqpoint{5.171936in}{0.431544in}}%
\pgfpathcurveto{\pgfqpoint{5.164122in}{0.439358in}}{\pgfqpoint{5.153523in}{0.443748in}}{\pgfqpoint{5.142473in}{0.443748in}}%
\pgfpathcurveto{\pgfqpoint{5.131423in}{0.443748in}}{\pgfqpoint{5.120824in}{0.439358in}}{\pgfqpoint{5.113010in}{0.431544in}}%
\pgfpathcurveto{\pgfqpoint{5.105196in}{0.423730in}}{\pgfqpoint{5.100806in}{0.413131in}}{\pgfqpoint{5.100806in}{0.402081in}}%
\pgfpathcurveto{\pgfqpoint{5.100806in}{0.391031in}}{\pgfqpoint{5.105196in}{0.380432in}}{\pgfqpoint{5.113010in}{0.372618in}}%
\pgfpathcurveto{\pgfqpoint{5.120824in}{0.364805in}}{\pgfqpoint{5.131423in}{0.360415in}}{\pgfqpoint{5.142473in}{0.360415in}}%
\pgfusepath{stroke}%
\end{pgfscope}%
\begin{pgfscope}%
\pgfpathrectangle{\pgfqpoint{0.393053in}{0.375000in}}{\pgfqpoint{6.356833in}{5.175000in}}%
\pgfusepath{clip}%
\pgfsetbuttcap%
\pgfsetroundjoin%
\pgfsetlinewidth{1.003750pt}%
\definecolor{currentstroke}{rgb}{0.827451,0.827451,0.827451}%
\pgfsetstrokecolor{currentstroke}%
\pgfsetdash{}{0pt}%
\pgfpathmoveto{\pgfqpoint{4.655370in}{0.409439in}}%
\pgfpathcurveto{\pgfqpoint{4.666420in}{0.409439in}}{\pgfqpoint{4.677019in}{0.413829in}}{\pgfqpoint{4.684833in}{0.421642in}}%
\pgfpathcurveto{\pgfqpoint{4.692646in}{0.429456in}}{\pgfqpoint{4.697036in}{0.440055in}}{\pgfqpoint{4.697036in}{0.451105in}}%
\pgfpathcurveto{\pgfqpoint{4.697036in}{0.462155in}}{\pgfqpoint{4.692646in}{0.472754in}}{\pgfqpoint{4.684833in}{0.480568in}}%
\pgfpathcurveto{\pgfqpoint{4.677019in}{0.488382in}}{\pgfqpoint{4.666420in}{0.492772in}}{\pgfqpoint{4.655370in}{0.492772in}}%
\pgfpathcurveto{\pgfqpoint{4.644320in}{0.492772in}}{\pgfqpoint{4.633721in}{0.488382in}}{\pgfqpoint{4.625907in}{0.480568in}}%
\pgfpathcurveto{\pgfqpoint{4.618093in}{0.472754in}}{\pgfqpoint{4.613703in}{0.462155in}}{\pgfqpoint{4.613703in}{0.451105in}}%
\pgfpathcurveto{\pgfqpoint{4.613703in}{0.440055in}}{\pgfqpoint{4.618093in}{0.429456in}}{\pgfqpoint{4.625907in}{0.421642in}}%
\pgfpathcurveto{\pgfqpoint{4.633721in}{0.413829in}}{\pgfqpoint{4.644320in}{0.409439in}}{\pgfqpoint{4.655370in}{0.409439in}}%
\pgfpathlineto{\pgfqpoint{4.655370in}{0.409439in}}%
\pgfpathclose%
\pgfusepath{stroke}%
\end{pgfscope}%
\begin{pgfscope}%
\pgfpathrectangle{\pgfqpoint{0.393053in}{0.375000in}}{\pgfqpoint{6.356833in}{5.175000in}}%
\pgfusepath{clip}%
\pgfsetbuttcap%
\pgfsetroundjoin%
\pgfsetlinewidth{1.003750pt}%
\definecolor{currentstroke}{rgb}{0.827451,0.827451,0.827451}%
\pgfsetstrokecolor{currentstroke}%
\pgfsetdash{}{0pt}%
\pgfpathmoveto{\pgfqpoint{0.765086in}{2.832905in}}%
\pgfpathcurveto{\pgfqpoint{0.776136in}{2.832905in}}{\pgfqpoint{0.786735in}{2.837296in}}{\pgfqpoint{0.794549in}{2.845109in}}%
\pgfpathcurveto{\pgfqpoint{0.802362in}{2.852923in}}{\pgfqpoint{0.806753in}{2.863522in}}{\pgfqpoint{0.806753in}{2.874572in}}%
\pgfpathcurveto{\pgfqpoint{0.806753in}{2.885622in}}{\pgfqpoint{0.802362in}{2.896221in}}{\pgfqpoint{0.794549in}{2.904035in}}%
\pgfpathcurveto{\pgfqpoint{0.786735in}{2.911849in}}{\pgfqpoint{0.776136in}{2.916239in}}{\pgfqpoint{0.765086in}{2.916239in}}%
\pgfpathcurveto{\pgfqpoint{0.754036in}{2.916239in}}{\pgfqpoint{0.743437in}{2.911849in}}{\pgfqpoint{0.735623in}{2.904035in}}%
\pgfpathcurveto{\pgfqpoint{0.727810in}{2.896221in}}{\pgfqpoint{0.723419in}{2.885622in}}{\pgfqpoint{0.723419in}{2.874572in}}%
\pgfpathcurveto{\pgfqpoint{0.723419in}{2.863522in}}{\pgfqpoint{0.727810in}{2.852923in}}{\pgfqpoint{0.735623in}{2.845109in}}%
\pgfpathcurveto{\pgfqpoint{0.743437in}{2.837296in}}{\pgfqpoint{0.754036in}{2.832905in}}{\pgfqpoint{0.765086in}{2.832905in}}%
\pgfpathlineto{\pgfqpoint{0.765086in}{2.832905in}}%
\pgfpathclose%
\pgfusepath{stroke}%
\end{pgfscope}%
\begin{pgfscope}%
\pgfpathrectangle{\pgfqpoint{0.393053in}{0.375000in}}{\pgfqpoint{6.356833in}{5.175000in}}%
\pgfusepath{clip}%
\pgfsetbuttcap%
\pgfsetroundjoin%
\pgfsetlinewidth{1.003750pt}%
\definecolor{currentstroke}{rgb}{0.827451,0.827451,0.827451}%
\pgfsetstrokecolor{currentstroke}%
\pgfsetdash{}{0pt}%
\pgfpathmoveto{\pgfqpoint{0.911466in}{2.525583in}}%
\pgfpathcurveto{\pgfqpoint{0.922516in}{2.525583in}}{\pgfqpoint{0.933116in}{2.529974in}}{\pgfqpoint{0.940929in}{2.537787in}}%
\pgfpathcurveto{\pgfqpoint{0.948743in}{2.545601in}}{\pgfqpoint{0.953133in}{2.556200in}}{\pgfqpoint{0.953133in}{2.567250in}}%
\pgfpathcurveto{\pgfqpoint{0.953133in}{2.578300in}}{\pgfqpoint{0.948743in}{2.588899in}}{\pgfqpoint{0.940929in}{2.596713in}}%
\pgfpathcurveto{\pgfqpoint{0.933116in}{2.604526in}}{\pgfqpoint{0.922516in}{2.608917in}}{\pgfqpoint{0.911466in}{2.608917in}}%
\pgfpathcurveto{\pgfqpoint{0.900416in}{2.608917in}}{\pgfqpoint{0.889817in}{2.604526in}}{\pgfqpoint{0.882004in}{2.596713in}}%
\pgfpathcurveto{\pgfqpoint{0.874190in}{2.588899in}}{\pgfqpoint{0.869800in}{2.578300in}}{\pgfqpoint{0.869800in}{2.567250in}}%
\pgfpathcurveto{\pgfqpoint{0.869800in}{2.556200in}}{\pgfqpoint{0.874190in}{2.545601in}}{\pgfqpoint{0.882004in}{2.537787in}}%
\pgfpathcurveto{\pgfqpoint{0.889817in}{2.529974in}}{\pgfqpoint{0.900416in}{2.525583in}}{\pgfqpoint{0.911466in}{2.525583in}}%
\pgfpathlineto{\pgfqpoint{0.911466in}{2.525583in}}%
\pgfpathclose%
\pgfusepath{stroke}%
\end{pgfscope}%
\begin{pgfscope}%
\pgfpathrectangle{\pgfqpoint{0.393053in}{0.375000in}}{\pgfqpoint{6.356833in}{5.175000in}}%
\pgfusepath{clip}%
\pgfsetbuttcap%
\pgfsetroundjoin%
\pgfsetlinewidth{1.003750pt}%
\definecolor{currentstroke}{rgb}{0.827451,0.827451,0.827451}%
\pgfsetstrokecolor{currentstroke}%
\pgfsetdash{}{0pt}%
\pgfpathmoveto{\pgfqpoint{1.614675in}{1.708400in}}%
\pgfpathcurveto{\pgfqpoint{1.625725in}{1.708400in}}{\pgfqpoint{1.636324in}{1.712790in}}{\pgfqpoint{1.644137in}{1.720604in}}%
\pgfpathcurveto{\pgfqpoint{1.651951in}{1.728417in}}{\pgfqpoint{1.656341in}{1.739016in}}{\pgfqpoint{1.656341in}{1.750066in}}%
\pgfpathcurveto{\pgfqpoint{1.656341in}{1.761117in}}{\pgfqpoint{1.651951in}{1.771716in}}{\pgfqpoint{1.644137in}{1.779529in}}%
\pgfpathcurveto{\pgfqpoint{1.636324in}{1.787343in}}{\pgfqpoint{1.625725in}{1.791733in}}{\pgfqpoint{1.614675in}{1.791733in}}%
\pgfpathcurveto{\pgfqpoint{1.603625in}{1.791733in}}{\pgfqpoint{1.593025in}{1.787343in}}{\pgfqpoint{1.585212in}{1.779529in}}%
\pgfpathcurveto{\pgfqpoint{1.577398in}{1.771716in}}{\pgfqpoint{1.573008in}{1.761117in}}{\pgfqpoint{1.573008in}{1.750066in}}%
\pgfpathcurveto{\pgfqpoint{1.573008in}{1.739016in}}{\pgfqpoint{1.577398in}{1.728417in}}{\pgfqpoint{1.585212in}{1.720604in}}%
\pgfpathcurveto{\pgfqpoint{1.593025in}{1.712790in}}{\pgfqpoint{1.603625in}{1.708400in}}{\pgfqpoint{1.614675in}{1.708400in}}%
\pgfpathlineto{\pgfqpoint{1.614675in}{1.708400in}}%
\pgfpathclose%
\pgfusepath{stroke}%
\end{pgfscope}%
\begin{pgfscope}%
\pgfpathrectangle{\pgfqpoint{0.393053in}{0.375000in}}{\pgfqpoint{6.356833in}{5.175000in}}%
\pgfusepath{clip}%
\pgfsetbuttcap%
\pgfsetroundjoin%
\pgfsetlinewidth{1.003750pt}%
\definecolor{currentstroke}{rgb}{0.827451,0.827451,0.827451}%
\pgfsetstrokecolor{currentstroke}%
\pgfsetdash{}{0pt}%
\pgfpathmoveto{\pgfqpoint{0.849130in}{2.701883in}}%
\pgfpathcurveto{\pgfqpoint{0.860180in}{2.701883in}}{\pgfqpoint{0.870779in}{2.706274in}}{\pgfqpoint{0.878592in}{2.714087in}}%
\pgfpathcurveto{\pgfqpoint{0.886406in}{2.721901in}}{\pgfqpoint{0.890796in}{2.732500in}}{\pgfqpoint{0.890796in}{2.743550in}}%
\pgfpathcurveto{\pgfqpoint{0.890796in}{2.754600in}}{\pgfqpoint{0.886406in}{2.765199in}}{\pgfqpoint{0.878592in}{2.773013in}}%
\pgfpathcurveto{\pgfqpoint{0.870779in}{2.780826in}}{\pgfqpoint{0.860180in}{2.785217in}}{\pgfqpoint{0.849130in}{2.785217in}}%
\pgfpathcurveto{\pgfqpoint{0.838079in}{2.785217in}}{\pgfqpoint{0.827480in}{2.780826in}}{\pgfqpoint{0.819667in}{2.773013in}}%
\pgfpathcurveto{\pgfqpoint{0.811853in}{2.765199in}}{\pgfqpoint{0.807463in}{2.754600in}}{\pgfqpoint{0.807463in}{2.743550in}}%
\pgfpathcurveto{\pgfqpoint{0.807463in}{2.732500in}}{\pgfqpoint{0.811853in}{2.721901in}}{\pgfqpoint{0.819667in}{2.714087in}}%
\pgfpathcurveto{\pgfqpoint{0.827480in}{2.706274in}}{\pgfqpoint{0.838079in}{2.701883in}}{\pgfqpoint{0.849130in}{2.701883in}}%
\pgfpathlineto{\pgfqpoint{0.849130in}{2.701883in}}%
\pgfpathclose%
\pgfusepath{stroke}%
\end{pgfscope}%
\begin{pgfscope}%
\pgfpathrectangle{\pgfqpoint{0.393053in}{0.375000in}}{\pgfqpoint{6.356833in}{5.175000in}}%
\pgfusepath{clip}%
\pgfsetbuttcap%
\pgfsetroundjoin%
\pgfsetlinewidth{1.003750pt}%
\definecolor{currentstroke}{rgb}{0.827451,0.827451,0.827451}%
\pgfsetstrokecolor{currentstroke}%
\pgfsetdash{}{0pt}%
\pgfpathmoveto{\pgfqpoint{4.898856in}{0.391247in}}%
\pgfpathcurveto{\pgfqpoint{4.909906in}{0.391247in}}{\pgfqpoint{4.920505in}{0.395637in}}{\pgfqpoint{4.928319in}{0.403450in}}%
\pgfpathcurveto{\pgfqpoint{4.936132in}{0.411264in}}{\pgfqpoint{4.940523in}{0.421863in}}{\pgfqpoint{4.940523in}{0.432913in}}%
\pgfpathcurveto{\pgfqpoint{4.940523in}{0.443963in}}{\pgfqpoint{4.936132in}{0.454562in}}{\pgfqpoint{4.928319in}{0.462376in}}%
\pgfpathcurveto{\pgfqpoint{4.920505in}{0.470190in}}{\pgfqpoint{4.909906in}{0.474580in}}{\pgfqpoint{4.898856in}{0.474580in}}%
\pgfpathcurveto{\pgfqpoint{4.887806in}{0.474580in}}{\pgfqpoint{4.877207in}{0.470190in}}{\pgfqpoint{4.869393in}{0.462376in}}%
\pgfpathcurveto{\pgfqpoint{4.861580in}{0.454562in}}{\pgfqpoint{4.857189in}{0.443963in}}{\pgfqpoint{4.857189in}{0.432913in}}%
\pgfpathcurveto{\pgfqpoint{4.857189in}{0.421863in}}{\pgfqpoint{4.861580in}{0.411264in}}{\pgfqpoint{4.869393in}{0.403450in}}%
\pgfpathcurveto{\pgfqpoint{4.877207in}{0.395637in}}{\pgfqpoint{4.887806in}{0.391247in}}{\pgfqpoint{4.898856in}{0.391247in}}%
\pgfpathlineto{\pgfqpoint{4.898856in}{0.391247in}}%
\pgfpathclose%
\pgfusepath{stroke}%
\end{pgfscope}%
\begin{pgfscope}%
\pgfpathrectangle{\pgfqpoint{0.393053in}{0.375000in}}{\pgfqpoint{6.356833in}{5.175000in}}%
\pgfusepath{clip}%
\pgfsetbuttcap%
\pgfsetroundjoin%
\pgfsetlinewidth{1.003750pt}%
\definecolor{currentstroke}{rgb}{0.827451,0.827451,0.827451}%
\pgfsetstrokecolor{currentstroke}%
\pgfsetdash{}{0pt}%
\pgfpathmoveto{\pgfqpoint{0.455840in}{3.833482in}}%
\pgfpathcurveto{\pgfqpoint{0.466890in}{3.833482in}}{\pgfqpoint{0.477489in}{3.837873in}}{\pgfqpoint{0.485303in}{3.845686in}}%
\pgfpathcurveto{\pgfqpoint{0.493117in}{3.853500in}}{\pgfqpoint{0.497507in}{3.864099in}}{\pgfqpoint{0.497507in}{3.875149in}}%
\pgfpathcurveto{\pgfqpoint{0.497507in}{3.886199in}}{\pgfqpoint{0.493117in}{3.896798in}}{\pgfqpoint{0.485303in}{3.904612in}}%
\pgfpathcurveto{\pgfqpoint{0.477489in}{3.912425in}}{\pgfqpoint{0.466890in}{3.916816in}}{\pgfqpoint{0.455840in}{3.916816in}}%
\pgfpathcurveto{\pgfqpoint{0.444790in}{3.916816in}}{\pgfqpoint{0.434191in}{3.912425in}}{\pgfqpoint{0.426377in}{3.904612in}}%
\pgfpathcurveto{\pgfqpoint{0.418564in}{3.896798in}}{\pgfqpoint{0.414173in}{3.886199in}}{\pgfqpoint{0.414173in}{3.875149in}}%
\pgfpathcurveto{\pgfqpoint{0.414173in}{3.864099in}}{\pgfqpoint{0.418564in}{3.853500in}}{\pgfqpoint{0.426377in}{3.845686in}}%
\pgfpathcurveto{\pgfqpoint{0.434191in}{3.837873in}}{\pgfqpoint{0.444790in}{3.833482in}}{\pgfqpoint{0.455840in}{3.833482in}}%
\pgfpathlineto{\pgfqpoint{0.455840in}{3.833482in}}%
\pgfpathclose%
\pgfusepath{stroke}%
\end{pgfscope}%
\begin{pgfscope}%
\pgfpathrectangle{\pgfqpoint{0.393053in}{0.375000in}}{\pgfqpoint{6.356833in}{5.175000in}}%
\pgfusepath{clip}%
\pgfsetbuttcap%
\pgfsetroundjoin%
\pgfsetlinewidth{1.003750pt}%
\definecolor{currentstroke}{rgb}{0.827451,0.827451,0.827451}%
\pgfsetstrokecolor{currentstroke}%
\pgfsetdash{}{0pt}%
\pgfpathmoveto{\pgfqpoint{4.169452in}{0.483485in}}%
\pgfpathcurveto{\pgfqpoint{4.180503in}{0.483485in}}{\pgfqpoint{4.191102in}{0.487876in}}{\pgfqpoint{4.198915in}{0.495689in}}%
\pgfpathcurveto{\pgfqpoint{4.206729in}{0.503503in}}{\pgfqpoint{4.211119in}{0.514102in}}{\pgfqpoint{4.211119in}{0.525152in}}%
\pgfpathcurveto{\pgfqpoint{4.211119in}{0.536202in}}{\pgfqpoint{4.206729in}{0.546801in}}{\pgfqpoint{4.198915in}{0.554615in}}%
\pgfpathcurveto{\pgfqpoint{4.191102in}{0.562429in}}{\pgfqpoint{4.180503in}{0.566819in}}{\pgfqpoint{4.169452in}{0.566819in}}%
\pgfpathcurveto{\pgfqpoint{4.158402in}{0.566819in}}{\pgfqpoint{4.147803in}{0.562429in}}{\pgfqpoint{4.139990in}{0.554615in}}%
\pgfpathcurveto{\pgfqpoint{4.132176in}{0.546801in}}{\pgfqpoint{4.127786in}{0.536202in}}{\pgfqpoint{4.127786in}{0.525152in}}%
\pgfpathcurveto{\pgfqpoint{4.127786in}{0.514102in}}{\pgfqpoint{4.132176in}{0.503503in}}{\pgfqpoint{4.139990in}{0.495689in}}%
\pgfpathcurveto{\pgfqpoint{4.147803in}{0.487876in}}{\pgfqpoint{4.158402in}{0.483485in}}{\pgfqpoint{4.169452in}{0.483485in}}%
\pgfpathlineto{\pgfqpoint{4.169452in}{0.483485in}}%
\pgfpathclose%
\pgfusepath{stroke}%
\end{pgfscope}%
\begin{pgfscope}%
\pgfpathrectangle{\pgfqpoint{0.393053in}{0.375000in}}{\pgfqpoint{6.356833in}{5.175000in}}%
\pgfusepath{clip}%
\pgfsetbuttcap%
\pgfsetroundjoin%
\pgfsetlinewidth{1.003750pt}%
\definecolor{currentstroke}{rgb}{0.827451,0.827451,0.827451}%
\pgfsetstrokecolor{currentstroke}%
\pgfsetdash{}{0pt}%
\pgfpathmoveto{\pgfqpoint{4.449412in}{0.418377in}}%
\pgfpathcurveto{\pgfqpoint{4.460462in}{0.418377in}}{\pgfqpoint{4.471061in}{0.422767in}}{\pgfqpoint{4.478875in}{0.430581in}}%
\pgfpathcurveto{\pgfqpoint{4.486688in}{0.438394in}}{\pgfqpoint{4.491079in}{0.448993in}}{\pgfqpoint{4.491079in}{0.460044in}}%
\pgfpathcurveto{\pgfqpoint{4.491079in}{0.471094in}}{\pgfqpoint{4.486688in}{0.481693in}}{\pgfqpoint{4.478875in}{0.489506in}}%
\pgfpathcurveto{\pgfqpoint{4.471061in}{0.497320in}}{\pgfqpoint{4.460462in}{0.501710in}}{\pgfqpoint{4.449412in}{0.501710in}}%
\pgfpathcurveto{\pgfqpoint{4.438362in}{0.501710in}}{\pgfqpoint{4.427763in}{0.497320in}}{\pgfqpoint{4.419949in}{0.489506in}}%
\pgfpathcurveto{\pgfqpoint{4.412136in}{0.481693in}}{\pgfqpoint{4.407745in}{0.471094in}}{\pgfqpoint{4.407745in}{0.460044in}}%
\pgfpathcurveto{\pgfqpoint{4.407745in}{0.448993in}}{\pgfqpoint{4.412136in}{0.438394in}}{\pgfqpoint{4.419949in}{0.430581in}}%
\pgfpathcurveto{\pgfqpoint{4.427763in}{0.422767in}}{\pgfqpoint{4.438362in}{0.418377in}}{\pgfqpoint{4.449412in}{0.418377in}}%
\pgfpathlineto{\pgfqpoint{4.449412in}{0.418377in}}%
\pgfpathclose%
\pgfusepath{stroke}%
\end{pgfscope}%
\begin{pgfscope}%
\pgfpathrectangle{\pgfqpoint{0.393053in}{0.375000in}}{\pgfqpoint{6.356833in}{5.175000in}}%
\pgfusepath{clip}%
\pgfsetbuttcap%
\pgfsetroundjoin%
\pgfsetlinewidth{1.003750pt}%
\definecolor{currentstroke}{rgb}{0.827451,0.827451,0.827451}%
\pgfsetstrokecolor{currentstroke}%
\pgfsetdash{}{0pt}%
\pgfpathmoveto{\pgfqpoint{2.693530in}{0.950710in}}%
\pgfpathcurveto{\pgfqpoint{2.704580in}{0.950710in}}{\pgfqpoint{2.715179in}{0.955100in}}{\pgfqpoint{2.722993in}{0.962914in}}%
\pgfpathcurveto{\pgfqpoint{2.730806in}{0.970727in}}{\pgfqpoint{2.735197in}{0.981326in}}{\pgfqpoint{2.735197in}{0.992376in}}%
\pgfpathcurveto{\pgfqpoint{2.735197in}{1.003427in}}{\pgfqpoint{2.730806in}{1.014026in}}{\pgfqpoint{2.722993in}{1.021839in}}%
\pgfpathcurveto{\pgfqpoint{2.715179in}{1.029653in}}{\pgfqpoint{2.704580in}{1.034043in}}{\pgfqpoint{2.693530in}{1.034043in}}%
\pgfpathcurveto{\pgfqpoint{2.682480in}{1.034043in}}{\pgfqpoint{2.671881in}{1.029653in}}{\pgfqpoint{2.664067in}{1.021839in}}%
\pgfpathcurveto{\pgfqpoint{2.656254in}{1.014026in}}{\pgfqpoint{2.651863in}{1.003427in}}{\pgfqpoint{2.651863in}{0.992376in}}%
\pgfpathcurveto{\pgfqpoint{2.651863in}{0.981326in}}{\pgfqpoint{2.656254in}{0.970727in}}{\pgfqpoint{2.664067in}{0.962914in}}%
\pgfpathcurveto{\pgfqpoint{2.671881in}{0.955100in}}{\pgfqpoint{2.682480in}{0.950710in}}{\pgfqpoint{2.693530in}{0.950710in}}%
\pgfpathlineto{\pgfqpoint{2.693530in}{0.950710in}}%
\pgfpathclose%
\pgfusepath{stroke}%
\end{pgfscope}%
\begin{pgfscope}%
\pgfpathrectangle{\pgfqpoint{0.393053in}{0.375000in}}{\pgfqpoint{6.356833in}{5.175000in}}%
\pgfusepath{clip}%
\pgfsetbuttcap%
\pgfsetroundjoin%
\pgfsetlinewidth{1.003750pt}%
\definecolor{currentstroke}{rgb}{0.827451,0.827451,0.827451}%
\pgfsetstrokecolor{currentstroke}%
\pgfsetdash{}{0pt}%
\pgfpathmoveto{\pgfqpoint{3.774127in}{0.547432in}}%
\pgfpathcurveto{\pgfqpoint{3.785177in}{0.547432in}}{\pgfqpoint{3.795776in}{0.551822in}}{\pgfqpoint{3.803589in}{0.559636in}}%
\pgfpathcurveto{\pgfqpoint{3.811403in}{0.567450in}}{\pgfqpoint{3.815793in}{0.578049in}}{\pgfqpoint{3.815793in}{0.589099in}}%
\pgfpathcurveto{\pgfqpoint{3.815793in}{0.600149in}}{\pgfqpoint{3.811403in}{0.610748in}}{\pgfqpoint{3.803589in}{0.618562in}}%
\pgfpathcurveto{\pgfqpoint{3.795776in}{0.626375in}}{\pgfqpoint{3.785177in}{0.630766in}}{\pgfqpoint{3.774127in}{0.630766in}}%
\pgfpathcurveto{\pgfqpoint{3.763077in}{0.630766in}}{\pgfqpoint{3.752478in}{0.626375in}}{\pgfqpoint{3.744664in}{0.618562in}}%
\pgfpathcurveto{\pgfqpoint{3.736850in}{0.610748in}}{\pgfqpoint{3.732460in}{0.600149in}}{\pgfqpoint{3.732460in}{0.589099in}}%
\pgfpathcurveto{\pgfqpoint{3.732460in}{0.578049in}}{\pgfqpoint{3.736850in}{0.567450in}}{\pgfqpoint{3.744664in}{0.559636in}}%
\pgfpathcurveto{\pgfqpoint{3.752478in}{0.551822in}}{\pgfqpoint{3.763077in}{0.547432in}}{\pgfqpoint{3.774127in}{0.547432in}}%
\pgfpathlineto{\pgfqpoint{3.774127in}{0.547432in}}%
\pgfpathclose%
\pgfusepath{stroke}%
\end{pgfscope}%
\begin{pgfscope}%
\pgfpathrectangle{\pgfqpoint{0.393053in}{0.375000in}}{\pgfqpoint{6.356833in}{5.175000in}}%
\pgfusepath{clip}%
\pgfsetbuttcap%
\pgfsetroundjoin%
\pgfsetlinewidth{1.003750pt}%
\definecolor{currentstroke}{rgb}{0.827451,0.827451,0.827451}%
\pgfsetstrokecolor{currentstroke}%
\pgfsetdash{}{0pt}%
\pgfpathmoveto{\pgfqpoint{0.911294in}{2.681526in}}%
\pgfpathcurveto{\pgfqpoint{0.922344in}{2.681526in}}{\pgfqpoint{0.932943in}{2.685916in}}{\pgfqpoint{0.940757in}{2.693730in}}%
\pgfpathcurveto{\pgfqpoint{0.948571in}{2.701544in}}{\pgfqpoint{0.952961in}{2.712143in}}{\pgfqpoint{0.952961in}{2.723193in}}%
\pgfpathcurveto{\pgfqpoint{0.952961in}{2.734243in}}{\pgfqpoint{0.948571in}{2.744842in}}{\pgfqpoint{0.940757in}{2.752655in}}%
\pgfpathcurveto{\pgfqpoint{0.932943in}{2.760469in}}{\pgfqpoint{0.922344in}{2.764859in}}{\pgfqpoint{0.911294in}{2.764859in}}%
\pgfpathcurveto{\pgfqpoint{0.900244in}{2.764859in}}{\pgfqpoint{0.889645in}{2.760469in}}{\pgfqpoint{0.881831in}{2.752655in}}%
\pgfpathcurveto{\pgfqpoint{0.874018in}{2.744842in}}{\pgfqpoint{0.869627in}{2.734243in}}{\pgfqpoint{0.869627in}{2.723193in}}%
\pgfpathcurveto{\pgfqpoint{0.869627in}{2.712143in}}{\pgfqpoint{0.874018in}{2.701544in}}{\pgfqpoint{0.881831in}{2.693730in}}%
\pgfpathcurveto{\pgfqpoint{0.889645in}{2.685916in}}{\pgfqpoint{0.900244in}{2.681526in}}{\pgfqpoint{0.911294in}{2.681526in}}%
\pgfpathlineto{\pgfqpoint{0.911294in}{2.681526in}}%
\pgfpathclose%
\pgfusepath{stroke}%
\end{pgfscope}%
\begin{pgfscope}%
\pgfpathrectangle{\pgfqpoint{0.393053in}{0.375000in}}{\pgfqpoint{6.356833in}{5.175000in}}%
\pgfusepath{clip}%
\pgfsetbuttcap%
\pgfsetroundjoin%
\pgfsetlinewidth{1.003750pt}%
\definecolor{currentstroke}{rgb}{0.827451,0.827451,0.827451}%
\pgfsetstrokecolor{currentstroke}%
\pgfsetdash{}{0pt}%
\pgfpathmoveto{\pgfqpoint{3.649696in}{0.625044in}}%
\pgfpathcurveto{\pgfqpoint{3.660746in}{0.625044in}}{\pgfqpoint{3.671345in}{0.629434in}}{\pgfqpoint{3.679159in}{0.637248in}}%
\pgfpathcurveto{\pgfqpoint{3.686972in}{0.645061in}}{\pgfqpoint{3.691363in}{0.655660in}}{\pgfqpoint{3.691363in}{0.666710in}}%
\pgfpathcurveto{\pgfqpoint{3.691363in}{0.677761in}}{\pgfqpoint{3.686972in}{0.688360in}}{\pgfqpoint{3.679159in}{0.696173in}}%
\pgfpathcurveto{\pgfqpoint{3.671345in}{0.703987in}}{\pgfqpoint{3.660746in}{0.708377in}}{\pgfqpoint{3.649696in}{0.708377in}}%
\pgfpathcurveto{\pgfqpoint{3.638646in}{0.708377in}}{\pgfqpoint{3.628047in}{0.703987in}}{\pgfqpoint{3.620233in}{0.696173in}}%
\pgfpathcurveto{\pgfqpoint{3.612420in}{0.688360in}}{\pgfqpoint{3.608029in}{0.677761in}}{\pgfqpoint{3.608029in}{0.666710in}}%
\pgfpathcurveto{\pgfqpoint{3.608029in}{0.655660in}}{\pgfqpoint{3.612420in}{0.645061in}}{\pgfqpoint{3.620233in}{0.637248in}}%
\pgfpathcurveto{\pgfqpoint{3.628047in}{0.629434in}}{\pgfqpoint{3.638646in}{0.625044in}}{\pgfqpoint{3.649696in}{0.625044in}}%
\pgfpathlineto{\pgfqpoint{3.649696in}{0.625044in}}%
\pgfpathclose%
\pgfusepath{stroke}%
\end{pgfscope}%
\begin{pgfscope}%
\pgfpathrectangle{\pgfqpoint{0.393053in}{0.375000in}}{\pgfqpoint{6.356833in}{5.175000in}}%
\pgfusepath{clip}%
\pgfsetbuttcap%
\pgfsetroundjoin%
\pgfsetlinewidth{1.003750pt}%
\definecolor{currentstroke}{rgb}{0.827451,0.827451,0.827451}%
\pgfsetstrokecolor{currentstroke}%
\pgfsetdash{}{0pt}%
\pgfpathmoveto{\pgfqpoint{2.953536in}{0.827530in}}%
\pgfpathcurveto{\pgfqpoint{2.964586in}{0.827530in}}{\pgfqpoint{2.975185in}{0.831920in}}{\pgfqpoint{2.982999in}{0.839734in}}%
\pgfpathcurveto{\pgfqpoint{2.990812in}{0.847548in}}{\pgfqpoint{2.995203in}{0.858147in}}{\pgfqpoint{2.995203in}{0.869197in}}%
\pgfpathcurveto{\pgfqpoint{2.995203in}{0.880247in}}{\pgfqpoint{2.990812in}{0.890846in}}{\pgfqpoint{2.982999in}{0.898659in}}%
\pgfpathcurveto{\pgfqpoint{2.975185in}{0.906473in}}{\pgfqpoint{2.964586in}{0.910863in}}{\pgfqpoint{2.953536in}{0.910863in}}%
\pgfpathcurveto{\pgfqpoint{2.942486in}{0.910863in}}{\pgfqpoint{2.931887in}{0.906473in}}{\pgfqpoint{2.924073in}{0.898659in}}%
\pgfpathcurveto{\pgfqpoint{2.916260in}{0.890846in}}{\pgfqpoint{2.911869in}{0.880247in}}{\pgfqpoint{2.911869in}{0.869197in}}%
\pgfpathcurveto{\pgfqpoint{2.911869in}{0.858147in}}{\pgfqpoint{2.916260in}{0.847548in}}{\pgfqpoint{2.924073in}{0.839734in}}%
\pgfpathcurveto{\pgfqpoint{2.931887in}{0.831920in}}{\pgfqpoint{2.942486in}{0.827530in}}{\pgfqpoint{2.953536in}{0.827530in}}%
\pgfpathlineto{\pgfqpoint{2.953536in}{0.827530in}}%
\pgfpathclose%
\pgfusepath{stroke}%
\end{pgfscope}%
\begin{pgfscope}%
\pgfpathrectangle{\pgfqpoint{0.393053in}{0.375000in}}{\pgfqpoint{6.356833in}{5.175000in}}%
\pgfusepath{clip}%
\pgfsetbuttcap%
\pgfsetroundjoin%
\pgfsetlinewidth{1.003750pt}%
\definecolor{currentstroke}{rgb}{0.827451,0.827451,0.827451}%
\pgfsetstrokecolor{currentstroke}%
\pgfsetdash{}{0pt}%
\pgfpathmoveto{\pgfqpoint{1.019447in}{2.455876in}}%
\pgfpathcurveto{\pgfqpoint{1.030497in}{2.455876in}}{\pgfqpoint{1.041096in}{2.460267in}}{\pgfqpoint{1.048910in}{2.468080in}}%
\pgfpathcurveto{\pgfqpoint{1.056723in}{2.475894in}}{\pgfqpoint{1.061113in}{2.486493in}}{\pgfqpoint{1.061113in}{2.497543in}}%
\pgfpathcurveto{\pgfqpoint{1.061113in}{2.508593in}}{\pgfqpoint{1.056723in}{2.519192in}}{\pgfqpoint{1.048910in}{2.527006in}}%
\pgfpathcurveto{\pgfqpoint{1.041096in}{2.534819in}}{\pgfqpoint{1.030497in}{2.539210in}}{\pgfqpoint{1.019447in}{2.539210in}}%
\pgfpathcurveto{\pgfqpoint{1.008397in}{2.539210in}}{\pgfqpoint{0.997798in}{2.534819in}}{\pgfqpoint{0.989984in}{2.527006in}}%
\pgfpathcurveto{\pgfqpoint{0.982170in}{2.519192in}}{\pgfqpoint{0.977780in}{2.508593in}}{\pgfqpoint{0.977780in}{2.497543in}}%
\pgfpathcurveto{\pgfqpoint{0.977780in}{2.486493in}}{\pgfqpoint{0.982170in}{2.475894in}}{\pgfqpoint{0.989984in}{2.468080in}}%
\pgfpathcurveto{\pgfqpoint{0.997798in}{2.460267in}}{\pgfqpoint{1.008397in}{2.455876in}}{\pgfqpoint{1.019447in}{2.455876in}}%
\pgfpathlineto{\pgfqpoint{1.019447in}{2.455876in}}%
\pgfpathclose%
\pgfusepath{stroke}%
\end{pgfscope}%
\begin{pgfscope}%
\pgfpathrectangle{\pgfqpoint{0.393053in}{0.375000in}}{\pgfqpoint{6.356833in}{5.175000in}}%
\pgfusepath{clip}%
\pgfsetbuttcap%
\pgfsetroundjoin%
\pgfsetlinewidth{1.003750pt}%
\definecolor{currentstroke}{rgb}{0.827451,0.827451,0.827451}%
\pgfsetstrokecolor{currentstroke}%
\pgfsetdash{}{0pt}%
\pgfpathmoveto{\pgfqpoint{2.352374in}{1.159706in}}%
\pgfpathcurveto{\pgfqpoint{2.363424in}{1.159706in}}{\pgfqpoint{2.374023in}{1.164096in}}{\pgfqpoint{2.381836in}{1.171910in}}%
\pgfpathcurveto{\pgfqpoint{2.389650in}{1.179724in}}{\pgfqpoint{2.394040in}{1.190323in}}{\pgfqpoint{2.394040in}{1.201373in}}%
\pgfpathcurveto{\pgfqpoint{2.394040in}{1.212423in}}{\pgfqpoint{2.389650in}{1.223022in}}{\pgfqpoint{2.381836in}{1.230835in}}%
\pgfpathcurveto{\pgfqpoint{2.374023in}{1.238649in}}{\pgfqpoint{2.363424in}{1.243039in}}{\pgfqpoint{2.352374in}{1.243039in}}%
\pgfpathcurveto{\pgfqpoint{2.341324in}{1.243039in}}{\pgfqpoint{2.330725in}{1.238649in}}{\pgfqpoint{2.322911in}{1.230835in}}%
\pgfpathcurveto{\pgfqpoint{2.315097in}{1.223022in}}{\pgfqpoint{2.310707in}{1.212423in}}{\pgfqpoint{2.310707in}{1.201373in}}%
\pgfpathcurveto{\pgfqpoint{2.310707in}{1.190323in}}{\pgfqpoint{2.315097in}{1.179724in}}{\pgfqpoint{2.322911in}{1.171910in}}%
\pgfpathcurveto{\pgfqpoint{2.330725in}{1.164096in}}{\pgfqpoint{2.341324in}{1.159706in}}{\pgfqpoint{2.352374in}{1.159706in}}%
\pgfpathlineto{\pgfqpoint{2.352374in}{1.159706in}}%
\pgfpathclose%
\pgfusepath{stroke}%
\end{pgfscope}%
\begin{pgfscope}%
\pgfpathrectangle{\pgfqpoint{0.393053in}{0.375000in}}{\pgfqpoint{6.356833in}{5.175000in}}%
\pgfusepath{clip}%
\pgfsetbuttcap%
\pgfsetroundjoin%
\pgfsetlinewidth{1.003750pt}%
\definecolor{currentstroke}{rgb}{0.827451,0.827451,0.827451}%
\pgfsetstrokecolor{currentstroke}%
\pgfsetdash{}{0pt}%
\pgfpathmoveto{\pgfqpoint{1.536415in}{1.768054in}}%
\pgfpathcurveto{\pgfqpoint{1.547466in}{1.768054in}}{\pgfqpoint{1.558065in}{1.772444in}}{\pgfqpoint{1.565878in}{1.780258in}}%
\pgfpathcurveto{\pgfqpoint{1.573692in}{1.788071in}}{\pgfqpoint{1.578082in}{1.798670in}}{\pgfqpoint{1.578082in}{1.809720in}}%
\pgfpathcurveto{\pgfqpoint{1.578082in}{1.820770in}}{\pgfqpoint{1.573692in}{1.831370in}}{\pgfqpoint{1.565878in}{1.839183in}}%
\pgfpathcurveto{\pgfqpoint{1.558065in}{1.846997in}}{\pgfqpoint{1.547466in}{1.851387in}}{\pgfqpoint{1.536415in}{1.851387in}}%
\pgfpathcurveto{\pgfqpoint{1.525365in}{1.851387in}}{\pgfqpoint{1.514766in}{1.846997in}}{\pgfqpoint{1.506953in}{1.839183in}}%
\pgfpathcurveto{\pgfqpoint{1.499139in}{1.831370in}}{\pgfqpoint{1.494749in}{1.820770in}}{\pgfqpoint{1.494749in}{1.809720in}}%
\pgfpathcurveto{\pgfqpoint{1.494749in}{1.798670in}}{\pgfqpoint{1.499139in}{1.788071in}}{\pgfqpoint{1.506953in}{1.780258in}}%
\pgfpathcurveto{\pgfqpoint{1.514766in}{1.772444in}}{\pgfqpoint{1.525365in}{1.768054in}}{\pgfqpoint{1.536415in}{1.768054in}}%
\pgfpathlineto{\pgfqpoint{1.536415in}{1.768054in}}%
\pgfpathclose%
\pgfusepath{stroke}%
\end{pgfscope}%
\begin{pgfscope}%
\pgfpathrectangle{\pgfqpoint{0.393053in}{0.375000in}}{\pgfqpoint{6.356833in}{5.175000in}}%
\pgfusepath{clip}%
\pgfsetbuttcap%
\pgfsetroundjoin%
\pgfsetlinewidth{1.003750pt}%
\definecolor{currentstroke}{rgb}{0.827451,0.827451,0.827451}%
\pgfsetstrokecolor{currentstroke}%
\pgfsetdash{}{0pt}%
\pgfpathmoveto{\pgfqpoint{0.580211in}{3.378166in}}%
\pgfpathcurveto{\pgfqpoint{0.591261in}{3.378166in}}{\pgfqpoint{0.601860in}{3.382556in}}{\pgfqpoint{0.609674in}{3.390370in}}%
\pgfpathcurveto{\pgfqpoint{0.617487in}{3.398183in}}{\pgfqpoint{0.621878in}{3.408782in}}{\pgfqpoint{0.621878in}{3.419833in}}%
\pgfpathcurveto{\pgfqpoint{0.621878in}{3.430883in}}{\pgfqpoint{0.617487in}{3.441482in}}{\pgfqpoint{0.609674in}{3.449295in}}%
\pgfpathcurveto{\pgfqpoint{0.601860in}{3.457109in}}{\pgfqpoint{0.591261in}{3.461499in}}{\pgfqpoint{0.580211in}{3.461499in}}%
\pgfpathcurveto{\pgfqpoint{0.569161in}{3.461499in}}{\pgfqpoint{0.558562in}{3.457109in}}{\pgfqpoint{0.550748in}{3.449295in}}%
\pgfpathcurveto{\pgfqpoint{0.542935in}{3.441482in}}{\pgfqpoint{0.538544in}{3.430883in}}{\pgfqpoint{0.538544in}{3.419833in}}%
\pgfpathcurveto{\pgfqpoint{0.538544in}{3.408782in}}{\pgfqpoint{0.542935in}{3.398183in}}{\pgfqpoint{0.550748in}{3.390370in}}%
\pgfpathcurveto{\pgfqpoint{0.558562in}{3.382556in}}{\pgfqpoint{0.569161in}{3.378166in}}{\pgfqpoint{0.580211in}{3.378166in}}%
\pgfpathlineto{\pgfqpoint{0.580211in}{3.378166in}}%
\pgfpathclose%
\pgfusepath{stroke}%
\end{pgfscope}%
\begin{pgfscope}%
\pgfpathrectangle{\pgfqpoint{0.393053in}{0.375000in}}{\pgfqpoint{6.356833in}{5.175000in}}%
\pgfusepath{clip}%
\pgfsetbuttcap%
\pgfsetroundjoin%
\pgfsetlinewidth{1.003750pt}%
\definecolor{currentstroke}{rgb}{0.827451,0.827451,0.827451}%
\pgfsetstrokecolor{currentstroke}%
\pgfsetdash{}{0pt}%
\pgfpathmoveto{\pgfqpoint{0.773700in}{2.823025in}}%
\pgfpathcurveto{\pgfqpoint{0.784750in}{2.823025in}}{\pgfqpoint{0.795350in}{2.827415in}}{\pgfqpoint{0.803163in}{2.835229in}}%
\pgfpathcurveto{\pgfqpoint{0.810977in}{2.843042in}}{\pgfqpoint{0.815367in}{2.853641in}}{\pgfqpoint{0.815367in}{2.864691in}}%
\pgfpathcurveto{\pgfqpoint{0.815367in}{2.875742in}}{\pgfqpoint{0.810977in}{2.886341in}}{\pgfqpoint{0.803163in}{2.894154in}}%
\pgfpathcurveto{\pgfqpoint{0.795350in}{2.901968in}}{\pgfqpoint{0.784750in}{2.906358in}}{\pgfqpoint{0.773700in}{2.906358in}}%
\pgfpathcurveto{\pgfqpoint{0.762650in}{2.906358in}}{\pgfqpoint{0.752051in}{2.901968in}}{\pgfqpoint{0.744238in}{2.894154in}}%
\pgfpathcurveto{\pgfqpoint{0.736424in}{2.886341in}}{\pgfqpoint{0.732034in}{2.875742in}}{\pgfqpoint{0.732034in}{2.864691in}}%
\pgfpathcurveto{\pgfqpoint{0.732034in}{2.853641in}}{\pgfqpoint{0.736424in}{2.843042in}}{\pgfqpoint{0.744238in}{2.835229in}}%
\pgfpathcurveto{\pgfqpoint{0.752051in}{2.827415in}}{\pgfqpoint{0.762650in}{2.823025in}}{\pgfqpoint{0.773700in}{2.823025in}}%
\pgfpathlineto{\pgfqpoint{0.773700in}{2.823025in}}%
\pgfpathclose%
\pgfusepath{stroke}%
\end{pgfscope}%
\begin{pgfscope}%
\pgfpathrectangle{\pgfqpoint{0.393053in}{0.375000in}}{\pgfqpoint{6.356833in}{5.175000in}}%
\pgfusepath{clip}%
\pgfsetbuttcap%
\pgfsetroundjoin%
\pgfsetlinewidth{1.003750pt}%
\definecolor{currentstroke}{rgb}{0.827451,0.827451,0.827451}%
\pgfsetstrokecolor{currentstroke}%
\pgfsetdash{}{0pt}%
\pgfpathmoveto{\pgfqpoint{1.768411in}{1.566057in}}%
\pgfpathcurveto{\pgfqpoint{1.779461in}{1.566057in}}{\pgfqpoint{1.790060in}{1.570447in}}{\pgfqpoint{1.797874in}{1.578261in}}%
\pgfpathcurveto{\pgfqpoint{1.805687in}{1.586074in}}{\pgfqpoint{1.810077in}{1.596673in}}{\pgfqpoint{1.810077in}{1.607723in}}%
\pgfpathcurveto{\pgfqpoint{1.810077in}{1.618773in}}{\pgfqpoint{1.805687in}{1.629373in}}{\pgfqpoint{1.797874in}{1.637186in}}%
\pgfpathcurveto{\pgfqpoint{1.790060in}{1.645000in}}{\pgfqpoint{1.779461in}{1.649390in}}{\pgfqpoint{1.768411in}{1.649390in}}%
\pgfpathcurveto{\pgfqpoint{1.757361in}{1.649390in}}{\pgfqpoint{1.746762in}{1.645000in}}{\pgfqpoint{1.738948in}{1.637186in}}%
\pgfpathcurveto{\pgfqpoint{1.731134in}{1.629373in}}{\pgfqpoint{1.726744in}{1.618773in}}{\pgfqpoint{1.726744in}{1.607723in}}%
\pgfpathcurveto{\pgfqpoint{1.726744in}{1.596673in}}{\pgfqpoint{1.731134in}{1.586074in}}{\pgfqpoint{1.738948in}{1.578261in}}%
\pgfpathcurveto{\pgfqpoint{1.746762in}{1.570447in}}{\pgfqpoint{1.757361in}{1.566057in}}{\pgfqpoint{1.768411in}{1.566057in}}%
\pgfpathlineto{\pgfqpoint{1.768411in}{1.566057in}}%
\pgfpathclose%
\pgfusepath{stroke}%
\end{pgfscope}%
\begin{pgfscope}%
\pgfpathrectangle{\pgfqpoint{0.393053in}{0.375000in}}{\pgfqpoint{6.356833in}{5.175000in}}%
\pgfusepath{clip}%
\pgfsetbuttcap%
\pgfsetroundjoin%
\pgfsetlinewidth{1.003750pt}%
\definecolor{currentstroke}{rgb}{0.827451,0.827451,0.827451}%
\pgfsetstrokecolor{currentstroke}%
\pgfsetdash{}{0pt}%
\pgfpathmoveto{\pgfqpoint{2.796329in}{0.903359in}}%
\pgfpathcurveto{\pgfqpoint{2.807379in}{0.903359in}}{\pgfqpoint{2.817978in}{0.907749in}}{\pgfqpoint{2.825792in}{0.915563in}}%
\pgfpathcurveto{\pgfqpoint{2.833605in}{0.923376in}}{\pgfqpoint{2.837995in}{0.933975in}}{\pgfqpoint{2.837995in}{0.945025in}}%
\pgfpathcurveto{\pgfqpoint{2.837995in}{0.956076in}}{\pgfqpoint{2.833605in}{0.966675in}}{\pgfqpoint{2.825792in}{0.974488in}}%
\pgfpathcurveto{\pgfqpoint{2.817978in}{0.982302in}}{\pgfqpoint{2.807379in}{0.986692in}}{\pgfqpoint{2.796329in}{0.986692in}}%
\pgfpathcurveto{\pgfqpoint{2.785279in}{0.986692in}}{\pgfqpoint{2.774680in}{0.982302in}}{\pgfqpoint{2.766866in}{0.974488in}}%
\pgfpathcurveto{\pgfqpoint{2.759052in}{0.966675in}}{\pgfqpoint{2.754662in}{0.956076in}}{\pgfqpoint{2.754662in}{0.945025in}}%
\pgfpathcurveto{\pgfqpoint{2.754662in}{0.933975in}}{\pgfqpoint{2.759052in}{0.923376in}}{\pgfqpoint{2.766866in}{0.915563in}}%
\pgfpathcurveto{\pgfqpoint{2.774680in}{0.907749in}}{\pgfqpoint{2.785279in}{0.903359in}}{\pgfqpoint{2.796329in}{0.903359in}}%
\pgfpathlineto{\pgfqpoint{2.796329in}{0.903359in}}%
\pgfpathclose%
\pgfusepath{stroke}%
\end{pgfscope}%
\begin{pgfscope}%
\pgfpathrectangle{\pgfqpoint{0.393053in}{0.375000in}}{\pgfqpoint{6.356833in}{5.175000in}}%
\pgfusepath{clip}%
\pgfsetbuttcap%
\pgfsetroundjoin%
\pgfsetlinewidth{1.003750pt}%
\definecolor{currentstroke}{rgb}{0.827451,0.827451,0.827451}%
\pgfsetstrokecolor{currentstroke}%
\pgfsetdash{}{0pt}%
\pgfpathmoveto{\pgfqpoint{5.017789in}{0.367635in}}%
\pgfpathcurveto{\pgfqpoint{5.028839in}{0.367635in}}{\pgfqpoint{5.039438in}{0.372025in}}{\pgfqpoint{5.047252in}{0.379839in}}%
\pgfpathcurveto{\pgfqpoint{5.055065in}{0.387652in}}{\pgfqpoint{5.059456in}{0.398251in}}{\pgfqpoint{5.059456in}{0.409302in}}%
\pgfpathcurveto{\pgfqpoint{5.059456in}{0.420352in}}{\pgfqpoint{5.055065in}{0.430951in}}{\pgfqpoint{5.047252in}{0.438764in}}%
\pgfpathcurveto{\pgfqpoint{5.039438in}{0.446578in}}{\pgfqpoint{5.028839in}{0.450968in}}{\pgfqpoint{5.017789in}{0.450968in}}%
\pgfpathcurveto{\pgfqpoint{5.006739in}{0.450968in}}{\pgfqpoint{4.996140in}{0.446578in}}{\pgfqpoint{4.988326in}{0.438764in}}%
\pgfpathcurveto{\pgfqpoint{4.980512in}{0.430951in}}{\pgfqpoint{4.976122in}{0.420352in}}{\pgfqpoint{4.976122in}{0.409302in}}%
\pgfpathcurveto{\pgfqpoint{4.976122in}{0.398251in}}{\pgfqpoint{4.980512in}{0.387652in}}{\pgfqpoint{4.988326in}{0.379839in}}%
\pgfpathcurveto{\pgfqpoint{4.996140in}{0.372025in}}{\pgfqpoint{5.006739in}{0.367635in}}{\pgfqpoint{5.017789in}{0.367635in}}%
\pgfusepath{stroke}%
\end{pgfscope}%
\begin{pgfscope}%
\pgfpathrectangle{\pgfqpoint{0.393053in}{0.375000in}}{\pgfqpoint{6.356833in}{5.175000in}}%
\pgfusepath{clip}%
\pgfsetbuttcap%
\pgfsetroundjoin%
\pgfsetlinewidth{1.003750pt}%
\definecolor{currentstroke}{rgb}{0.827451,0.827451,0.827451}%
\pgfsetstrokecolor{currentstroke}%
\pgfsetdash{}{0pt}%
\pgfpathmoveto{\pgfqpoint{0.474470in}{3.727925in}}%
\pgfpathcurveto{\pgfqpoint{0.485520in}{3.727925in}}{\pgfqpoint{0.496119in}{3.732315in}}{\pgfqpoint{0.503933in}{3.740129in}}%
\pgfpathcurveto{\pgfqpoint{0.511746in}{3.747943in}}{\pgfqpoint{0.516137in}{3.758542in}}{\pgfqpoint{0.516137in}{3.769592in}}%
\pgfpathcurveto{\pgfqpoint{0.516137in}{3.780642in}}{\pgfqpoint{0.511746in}{3.791241in}}{\pgfqpoint{0.503933in}{3.799055in}}%
\pgfpathcurveto{\pgfqpoint{0.496119in}{3.806868in}}{\pgfqpoint{0.485520in}{3.811259in}}{\pgfqpoint{0.474470in}{3.811259in}}%
\pgfpathcurveto{\pgfqpoint{0.463420in}{3.811259in}}{\pgfqpoint{0.452821in}{3.806868in}}{\pgfqpoint{0.445007in}{3.799055in}}%
\pgfpathcurveto{\pgfqpoint{0.437194in}{3.791241in}}{\pgfqpoint{0.432803in}{3.780642in}}{\pgfqpoint{0.432803in}{3.769592in}}%
\pgfpathcurveto{\pgfqpoint{0.432803in}{3.758542in}}{\pgfqpoint{0.437194in}{3.747943in}}{\pgfqpoint{0.445007in}{3.740129in}}%
\pgfpathcurveto{\pgfqpoint{0.452821in}{3.732315in}}{\pgfqpoint{0.463420in}{3.727925in}}{\pgfqpoint{0.474470in}{3.727925in}}%
\pgfpathlineto{\pgfqpoint{0.474470in}{3.727925in}}%
\pgfpathclose%
\pgfusepath{stroke}%
\end{pgfscope}%
\begin{pgfscope}%
\pgfpathrectangle{\pgfqpoint{0.393053in}{0.375000in}}{\pgfqpoint{6.356833in}{5.175000in}}%
\pgfusepath{clip}%
\pgfsetbuttcap%
\pgfsetroundjoin%
\pgfsetlinewidth{1.003750pt}%
\definecolor{currentstroke}{rgb}{0.827451,0.827451,0.827451}%
\pgfsetstrokecolor{currentstroke}%
\pgfsetdash{}{0pt}%
\pgfpathmoveto{\pgfqpoint{1.063961in}{2.328288in}}%
\pgfpathcurveto{\pgfqpoint{1.075011in}{2.328288in}}{\pgfqpoint{1.085610in}{2.332678in}}{\pgfqpoint{1.093424in}{2.340492in}}%
\pgfpathcurveto{\pgfqpoint{1.101237in}{2.348306in}}{\pgfqpoint{1.105627in}{2.358905in}}{\pgfqpoint{1.105627in}{2.369955in}}%
\pgfpathcurveto{\pgfqpoint{1.105627in}{2.381005in}}{\pgfqpoint{1.101237in}{2.391604in}}{\pgfqpoint{1.093424in}{2.399418in}}%
\pgfpathcurveto{\pgfqpoint{1.085610in}{2.407231in}}{\pgfqpoint{1.075011in}{2.411621in}}{\pgfqpoint{1.063961in}{2.411621in}}%
\pgfpathcurveto{\pgfqpoint{1.052911in}{2.411621in}}{\pgfqpoint{1.042312in}{2.407231in}}{\pgfqpoint{1.034498in}{2.399418in}}%
\pgfpathcurveto{\pgfqpoint{1.026684in}{2.391604in}}{\pgfqpoint{1.022294in}{2.381005in}}{\pgfqpoint{1.022294in}{2.369955in}}%
\pgfpathcurveto{\pgfqpoint{1.022294in}{2.358905in}}{\pgfqpoint{1.026684in}{2.348306in}}{\pgfqpoint{1.034498in}{2.340492in}}%
\pgfpathcurveto{\pgfqpoint{1.042312in}{2.332678in}}{\pgfqpoint{1.052911in}{2.328288in}}{\pgfqpoint{1.063961in}{2.328288in}}%
\pgfpathlineto{\pgfqpoint{1.063961in}{2.328288in}}%
\pgfpathclose%
\pgfusepath{stroke}%
\end{pgfscope}%
\begin{pgfscope}%
\pgfpathrectangle{\pgfqpoint{0.393053in}{0.375000in}}{\pgfqpoint{6.356833in}{5.175000in}}%
\pgfusepath{clip}%
\pgfsetbuttcap%
\pgfsetroundjoin%
\pgfsetlinewidth{1.003750pt}%
\definecolor{currentstroke}{rgb}{0.827451,0.827451,0.827451}%
\pgfsetstrokecolor{currentstroke}%
\pgfsetdash{}{0pt}%
\pgfpathmoveto{\pgfqpoint{3.335111in}{0.673028in}}%
\pgfpathcurveto{\pgfqpoint{3.346161in}{0.673028in}}{\pgfqpoint{3.356760in}{0.677418in}}{\pgfqpoint{3.364573in}{0.685232in}}%
\pgfpathcurveto{\pgfqpoint{3.372387in}{0.693045in}}{\pgfqpoint{3.376777in}{0.703644in}}{\pgfqpoint{3.376777in}{0.714695in}}%
\pgfpathcurveto{\pgfqpoint{3.376777in}{0.725745in}}{\pgfqpoint{3.372387in}{0.736344in}}{\pgfqpoint{3.364573in}{0.744157in}}%
\pgfpathcurveto{\pgfqpoint{3.356760in}{0.751971in}}{\pgfqpoint{3.346161in}{0.756361in}}{\pgfqpoint{3.335111in}{0.756361in}}%
\pgfpathcurveto{\pgfqpoint{3.324061in}{0.756361in}}{\pgfqpoint{3.313461in}{0.751971in}}{\pgfqpoint{3.305648in}{0.744157in}}%
\pgfpathcurveto{\pgfqpoint{3.297834in}{0.736344in}}{\pgfqpoint{3.293444in}{0.725745in}}{\pgfqpoint{3.293444in}{0.714695in}}%
\pgfpathcurveto{\pgfqpoint{3.293444in}{0.703644in}}{\pgfqpoint{3.297834in}{0.693045in}}{\pgfqpoint{3.305648in}{0.685232in}}%
\pgfpathcurveto{\pgfqpoint{3.313461in}{0.677418in}}{\pgfqpoint{3.324061in}{0.673028in}}{\pgfqpoint{3.335111in}{0.673028in}}%
\pgfpathlineto{\pgfqpoint{3.335111in}{0.673028in}}%
\pgfpathclose%
\pgfusepath{stroke}%
\end{pgfscope}%
\begin{pgfscope}%
\pgfpathrectangle{\pgfqpoint{0.393053in}{0.375000in}}{\pgfqpoint{6.356833in}{5.175000in}}%
\pgfusepath{clip}%
\pgfsetbuttcap%
\pgfsetroundjoin%
\pgfsetlinewidth{1.003750pt}%
\definecolor{currentstroke}{rgb}{0.827451,0.827451,0.827451}%
\pgfsetstrokecolor{currentstroke}%
\pgfsetdash{}{0pt}%
\pgfpathmoveto{\pgfqpoint{0.583953in}{3.273995in}}%
\pgfpathcurveto{\pgfqpoint{0.595003in}{3.273995in}}{\pgfqpoint{0.605602in}{3.278385in}}{\pgfqpoint{0.613416in}{3.286198in}}%
\pgfpathcurveto{\pgfqpoint{0.621229in}{3.294012in}}{\pgfqpoint{0.625620in}{3.304611in}}{\pgfqpoint{0.625620in}{3.315661in}}%
\pgfpathcurveto{\pgfqpoint{0.625620in}{3.326711in}}{\pgfqpoint{0.621229in}{3.337310in}}{\pgfqpoint{0.613416in}{3.345124in}}%
\pgfpathcurveto{\pgfqpoint{0.605602in}{3.352938in}}{\pgfqpoint{0.595003in}{3.357328in}}{\pgfqpoint{0.583953in}{3.357328in}}%
\pgfpathcurveto{\pgfqpoint{0.572903in}{3.357328in}}{\pgfqpoint{0.562304in}{3.352938in}}{\pgfqpoint{0.554490in}{3.345124in}}%
\pgfpathcurveto{\pgfqpoint{0.546677in}{3.337310in}}{\pgfqpoint{0.542286in}{3.326711in}}{\pgfqpoint{0.542286in}{3.315661in}}%
\pgfpathcurveto{\pgfqpoint{0.542286in}{3.304611in}}{\pgfqpoint{0.546677in}{3.294012in}}{\pgfqpoint{0.554490in}{3.286198in}}%
\pgfpathcurveto{\pgfqpoint{0.562304in}{3.278385in}}{\pgfqpoint{0.572903in}{3.273995in}}{\pgfqpoint{0.583953in}{3.273995in}}%
\pgfpathlineto{\pgfqpoint{0.583953in}{3.273995in}}%
\pgfpathclose%
\pgfusepath{stroke}%
\end{pgfscope}%
\begin{pgfscope}%
\pgfpathrectangle{\pgfqpoint{0.393053in}{0.375000in}}{\pgfqpoint{6.356833in}{5.175000in}}%
\pgfusepath{clip}%
\pgfsetbuttcap%
\pgfsetroundjoin%
\pgfsetlinewidth{1.003750pt}%
\definecolor{currentstroke}{rgb}{0.827451,0.827451,0.827451}%
\pgfsetstrokecolor{currentstroke}%
\pgfsetdash{}{0pt}%
\pgfpathmoveto{\pgfqpoint{2.230001in}{1.209659in}}%
\pgfpathcurveto{\pgfqpoint{2.241051in}{1.209659in}}{\pgfqpoint{2.251650in}{1.214049in}}{\pgfqpoint{2.259463in}{1.221862in}}%
\pgfpathcurveto{\pgfqpoint{2.267277in}{1.229676in}}{\pgfqpoint{2.271667in}{1.240275in}}{\pgfqpoint{2.271667in}{1.251325in}}%
\pgfpathcurveto{\pgfqpoint{2.271667in}{1.262375in}}{\pgfqpoint{2.267277in}{1.272974in}}{\pgfqpoint{2.259463in}{1.280788in}}%
\pgfpathcurveto{\pgfqpoint{2.251650in}{1.288602in}}{\pgfqpoint{2.241051in}{1.292992in}}{\pgfqpoint{2.230001in}{1.292992in}}%
\pgfpathcurveto{\pgfqpoint{2.218950in}{1.292992in}}{\pgfqpoint{2.208351in}{1.288602in}}{\pgfqpoint{2.200538in}{1.280788in}}%
\pgfpathcurveto{\pgfqpoint{2.192724in}{1.272974in}}{\pgfqpoint{2.188334in}{1.262375in}}{\pgfqpoint{2.188334in}{1.251325in}}%
\pgfpathcurveto{\pgfqpoint{2.188334in}{1.240275in}}{\pgfqpoint{2.192724in}{1.229676in}}{\pgfqpoint{2.200538in}{1.221862in}}%
\pgfpathcurveto{\pgfqpoint{2.208351in}{1.214049in}}{\pgfqpoint{2.218950in}{1.209659in}}{\pgfqpoint{2.230001in}{1.209659in}}%
\pgfpathlineto{\pgfqpoint{2.230001in}{1.209659in}}%
\pgfpathclose%
\pgfusepath{stroke}%
\end{pgfscope}%
\begin{pgfscope}%
\pgfpathrectangle{\pgfqpoint{0.393053in}{0.375000in}}{\pgfqpoint{6.356833in}{5.175000in}}%
\pgfusepath{clip}%
\pgfsetbuttcap%
\pgfsetroundjoin%
\pgfsetlinewidth{1.003750pt}%
\definecolor{currentstroke}{rgb}{0.827451,0.827451,0.827451}%
\pgfsetstrokecolor{currentstroke}%
\pgfsetdash{}{0pt}%
\pgfpathmoveto{\pgfqpoint{4.537650in}{0.413917in}}%
\pgfpathcurveto{\pgfqpoint{4.548700in}{0.413917in}}{\pgfqpoint{4.559300in}{0.418307in}}{\pgfqpoint{4.567113in}{0.426121in}}%
\pgfpathcurveto{\pgfqpoint{4.574927in}{0.433935in}}{\pgfqpoint{4.579317in}{0.444534in}}{\pgfqpoint{4.579317in}{0.455584in}}%
\pgfpathcurveto{\pgfqpoint{4.579317in}{0.466634in}}{\pgfqpoint{4.574927in}{0.477233in}}{\pgfqpoint{4.567113in}{0.485047in}}%
\pgfpathcurveto{\pgfqpoint{4.559300in}{0.492860in}}{\pgfqpoint{4.548700in}{0.497250in}}{\pgfqpoint{4.537650in}{0.497250in}}%
\pgfpathcurveto{\pgfqpoint{4.526600in}{0.497250in}}{\pgfqpoint{4.516001in}{0.492860in}}{\pgfqpoint{4.508188in}{0.485047in}}%
\pgfpathcurveto{\pgfqpoint{4.500374in}{0.477233in}}{\pgfqpoint{4.495984in}{0.466634in}}{\pgfqpoint{4.495984in}{0.455584in}}%
\pgfpathcurveto{\pgfqpoint{4.495984in}{0.444534in}}{\pgfqpoint{4.500374in}{0.433935in}}{\pgfqpoint{4.508188in}{0.426121in}}%
\pgfpathcurveto{\pgfqpoint{4.516001in}{0.418307in}}{\pgfqpoint{4.526600in}{0.413917in}}{\pgfqpoint{4.537650in}{0.413917in}}%
\pgfpathlineto{\pgfqpoint{4.537650in}{0.413917in}}%
\pgfpathclose%
\pgfusepath{stroke}%
\end{pgfscope}%
\begin{pgfscope}%
\pgfpathrectangle{\pgfqpoint{0.393053in}{0.375000in}}{\pgfqpoint{6.356833in}{5.175000in}}%
\pgfusepath{clip}%
\pgfsetbuttcap%
\pgfsetroundjoin%
\pgfsetlinewidth{1.003750pt}%
\definecolor{currentstroke}{rgb}{0.827451,0.827451,0.827451}%
\pgfsetstrokecolor{currentstroke}%
\pgfsetdash{}{0pt}%
\pgfpathmoveto{\pgfqpoint{3.463913in}{0.638701in}}%
\pgfpathcurveto{\pgfqpoint{3.474963in}{0.638701in}}{\pgfqpoint{3.485562in}{0.643091in}}{\pgfqpoint{3.493376in}{0.650905in}}%
\pgfpathcurveto{\pgfqpoint{3.501190in}{0.658718in}}{\pgfqpoint{3.505580in}{0.669317in}}{\pgfqpoint{3.505580in}{0.680367in}}%
\pgfpathcurveto{\pgfqpoint{3.505580in}{0.691418in}}{\pgfqpoint{3.501190in}{0.702017in}}{\pgfqpoint{3.493376in}{0.709830in}}%
\pgfpathcurveto{\pgfqpoint{3.485562in}{0.717644in}}{\pgfqpoint{3.474963in}{0.722034in}}{\pgfqpoint{3.463913in}{0.722034in}}%
\pgfpathcurveto{\pgfqpoint{3.452863in}{0.722034in}}{\pgfqpoint{3.442264in}{0.717644in}}{\pgfqpoint{3.434451in}{0.709830in}}%
\pgfpathcurveto{\pgfqpoint{3.426637in}{0.702017in}}{\pgfqpoint{3.422247in}{0.691418in}}{\pgfqpoint{3.422247in}{0.680367in}}%
\pgfpathcurveto{\pgfqpoint{3.422247in}{0.669317in}}{\pgfqpoint{3.426637in}{0.658718in}}{\pgfqpoint{3.434451in}{0.650905in}}%
\pgfpathcurveto{\pgfqpoint{3.442264in}{0.643091in}}{\pgfqpoint{3.452863in}{0.638701in}}{\pgfqpoint{3.463913in}{0.638701in}}%
\pgfpathlineto{\pgfqpoint{3.463913in}{0.638701in}}%
\pgfpathclose%
\pgfusepath{stroke}%
\end{pgfscope}%
\begin{pgfscope}%
\pgfpathrectangle{\pgfqpoint{0.393053in}{0.375000in}}{\pgfqpoint{6.356833in}{5.175000in}}%
\pgfusepath{clip}%
\pgfsetbuttcap%
\pgfsetroundjoin%
\pgfsetlinewidth{1.003750pt}%
\definecolor{currentstroke}{rgb}{0.827451,0.827451,0.827451}%
\pgfsetstrokecolor{currentstroke}%
\pgfsetdash{}{0pt}%
\pgfpathmoveto{\pgfqpoint{2.437419in}{1.106016in}}%
\pgfpathcurveto{\pgfqpoint{2.448469in}{1.106016in}}{\pgfqpoint{2.459069in}{1.110406in}}{\pgfqpoint{2.466882in}{1.118219in}}%
\pgfpathcurveto{\pgfqpoint{2.474696in}{1.126033in}}{\pgfqpoint{2.479086in}{1.136632in}}{\pgfqpoint{2.479086in}{1.147682in}}%
\pgfpathcurveto{\pgfqpoint{2.479086in}{1.158732in}}{\pgfqpoint{2.474696in}{1.169331in}}{\pgfqpoint{2.466882in}{1.177145in}}%
\pgfpathcurveto{\pgfqpoint{2.459069in}{1.184959in}}{\pgfqpoint{2.448469in}{1.189349in}}{\pgfqpoint{2.437419in}{1.189349in}}%
\pgfpathcurveto{\pgfqpoint{2.426369in}{1.189349in}}{\pgfqpoint{2.415770in}{1.184959in}}{\pgfqpoint{2.407957in}{1.177145in}}%
\pgfpathcurveto{\pgfqpoint{2.400143in}{1.169331in}}{\pgfqpoint{2.395753in}{1.158732in}}{\pgfqpoint{2.395753in}{1.147682in}}%
\pgfpathcurveto{\pgfqpoint{2.395753in}{1.136632in}}{\pgfqpoint{2.400143in}{1.126033in}}{\pgfqpoint{2.407957in}{1.118219in}}%
\pgfpathcurveto{\pgfqpoint{2.415770in}{1.110406in}}{\pgfqpoint{2.426369in}{1.106016in}}{\pgfqpoint{2.437419in}{1.106016in}}%
\pgfpathlineto{\pgfqpoint{2.437419in}{1.106016in}}%
\pgfpathclose%
\pgfusepath{stroke}%
\end{pgfscope}%
\begin{pgfscope}%
\pgfpathrectangle{\pgfqpoint{0.393053in}{0.375000in}}{\pgfqpoint{6.356833in}{5.175000in}}%
\pgfusepath{clip}%
\pgfsetbuttcap%
\pgfsetroundjoin%
\pgfsetlinewidth{1.003750pt}%
\definecolor{currentstroke}{rgb}{0.827451,0.827451,0.827451}%
\pgfsetstrokecolor{currentstroke}%
\pgfsetdash{}{0pt}%
\pgfpathmoveto{\pgfqpoint{0.435142in}{3.936498in}}%
\pgfpathcurveto{\pgfqpoint{0.446192in}{3.936498in}}{\pgfqpoint{0.456791in}{3.940888in}}{\pgfqpoint{0.464605in}{3.948702in}}%
\pgfpathcurveto{\pgfqpoint{0.472419in}{3.956516in}}{\pgfqpoint{0.476809in}{3.967115in}}{\pgfqpoint{0.476809in}{3.978165in}}%
\pgfpathcurveto{\pgfqpoint{0.476809in}{3.989215in}}{\pgfqpoint{0.472419in}{3.999814in}}{\pgfqpoint{0.464605in}{4.007627in}}%
\pgfpathcurveto{\pgfqpoint{0.456791in}{4.015441in}}{\pgfqpoint{0.446192in}{4.019831in}}{\pgfqpoint{0.435142in}{4.019831in}}%
\pgfpathcurveto{\pgfqpoint{0.424092in}{4.019831in}}{\pgfqpoint{0.413493in}{4.015441in}}{\pgfqpoint{0.405679in}{4.007627in}}%
\pgfpathcurveto{\pgfqpoint{0.397866in}{3.999814in}}{\pgfqpoint{0.393475in}{3.989215in}}{\pgfqpoint{0.393475in}{3.978165in}}%
\pgfpathcurveto{\pgfqpoint{0.393475in}{3.967115in}}{\pgfqpoint{0.397866in}{3.956516in}}{\pgfqpoint{0.405679in}{3.948702in}}%
\pgfpathcurveto{\pgfqpoint{0.413493in}{3.940888in}}{\pgfqpoint{0.424092in}{3.936498in}}{\pgfqpoint{0.435142in}{3.936498in}}%
\pgfpathlineto{\pgfqpoint{0.435142in}{3.936498in}}%
\pgfpathclose%
\pgfusepath{stroke}%
\end{pgfscope}%
\begin{pgfscope}%
\pgfpathrectangle{\pgfqpoint{0.393053in}{0.375000in}}{\pgfqpoint{6.356833in}{5.175000in}}%
\pgfusepath{clip}%
\pgfsetbuttcap%
\pgfsetroundjoin%
\pgfsetlinewidth{1.003750pt}%
\definecolor{currentstroke}{rgb}{0.827451,0.827451,0.827451}%
\pgfsetstrokecolor{currentstroke}%
\pgfsetdash{}{0pt}%
\pgfpathmoveto{\pgfqpoint{3.053820in}{0.786671in}}%
\pgfpathcurveto{\pgfqpoint{3.064870in}{0.786671in}}{\pgfqpoint{3.075469in}{0.791061in}}{\pgfqpoint{3.083282in}{0.798875in}}%
\pgfpathcurveto{\pgfqpoint{3.091096in}{0.806688in}}{\pgfqpoint{3.095486in}{0.817287in}}{\pgfqpoint{3.095486in}{0.828338in}}%
\pgfpathcurveto{\pgfqpoint{3.095486in}{0.839388in}}{\pgfqpoint{3.091096in}{0.849987in}}{\pgfqpoint{3.083282in}{0.857800in}}%
\pgfpathcurveto{\pgfqpoint{3.075469in}{0.865614in}}{\pgfqpoint{3.064870in}{0.870004in}}{\pgfqpoint{3.053820in}{0.870004in}}%
\pgfpathcurveto{\pgfqpoint{3.042770in}{0.870004in}}{\pgfqpoint{3.032171in}{0.865614in}}{\pgfqpoint{3.024357in}{0.857800in}}%
\pgfpathcurveto{\pgfqpoint{3.016543in}{0.849987in}}{\pgfqpoint{3.012153in}{0.839388in}}{\pgfqpoint{3.012153in}{0.828338in}}%
\pgfpathcurveto{\pgfqpoint{3.012153in}{0.817287in}}{\pgfqpoint{3.016543in}{0.806688in}}{\pgfqpoint{3.024357in}{0.798875in}}%
\pgfpathcurveto{\pgfqpoint{3.032171in}{0.791061in}}{\pgfqpoint{3.042770in}{0.786671in}}{\pgfqpoint{3.053820in}{0.786671in}}%
\pgfpathlineto{\pgfqpoint{3.053820in}{0.786671in}}%
\pgfpathclose%
\pgfusepath{stroke}%
\end{pgfscope}%
\begin{pgfscope}%
\pgfpathrectangle{\pgfqpoint{0.393053in}{0.375000in}}{\pgfqpoint{6.356833in}{5.175000in}}%
\pgfusepath{clip}%
\pgfsetbuttcap%
\pgfsetroundjoin%
\pgfsetlinewidth{1.003750pt}%
\definecolor{currentstroke}{rgb}{0.827451,0.827451,0.827451}%
\pgfsetstrokecolor{currentstroke}%
\pgfsetdash{}{0pt}%
\pgfpathmoveto{\pgfqpoint{3.256902in}{0.730828in}}%
\pgfpathcurveto{\pgfqpoint{3.267953in}{0.730828in}}{\pgfqpoint{3.278552in}{0.735218in}}{\pgfqpoint{3.286365in}{0.743032in}}%
\pgfpathcurveto{\pgfqpoint{3.294179in}{0.750846in}}{\pgfqpoint{3.298569in}{0.761445in}}{\pgfqpoint{3.298569in}{0.772495in}}%
\pgfpathcurveto{\pgfqpoint{3.298569in}{0.783545in}}{\pgfqpoint{3.294179in}{0.794144in}}{\pgfqpoint{3.286365in}{0.801958in}}%
\pgfpathcurveto{\pgfqpoint{3.278552in}{0.809771in}}{\pgfqpoint{3.267953in}{0.814162in}}{\pgfqpoint{3.256902in}{0.814162in}}%
\pgfpathcurveto{\pgfqpoint{3.245852in}{0.814162in}}{\pgfqpoint{3.235253in}{0.809771in}}{\pgfqpoint{3.227440in}{0.801958in}}%
\pgfpathcurveto{\pgfqpoint{3.219626in}{0.794144in}}{\pgfqpoint{3.215236in}{0.783545in}}{\pgfqpoint{3.215236in}{0.772495in}}%
\pgfpathcurveto{\pgfqpoint{3.215236in}{0.761445in}}{\pgfqpoint{3.219626in}{0.750846in}}{\pgfqpoint{3.227440in}{0.743032in}}%
\pgfpathcurveto{\pgfqpoint{3.235253in}{0.735218in}}{\pgfqpoint{3.245852in}{0.730828in}}{\pgfqpoint{3.256902in}{0.730828in}}%
\pgfpathlineto{\pgfqpoint{3.256902in}{0.730828in}}%
\pgfpathclose%
\pgfusepath{stroke}%
\end{pgfscope}%
\begin{pgfscope}%
\pgfpathrectangle{\pgfqpoint{0.393053in}{0.375000in}}{\pgfqpoint{6.356833in}{5.175000in}}%
\pgfusepath{clip}%
\pgfsetbuttcap%
\pgfsetroundjoin%
\pgfsetlinewidth{1.003750pt}%
\definecolor{currentstroke}{rgb}{0.827451,0.827451,0.827451}%
\pgfsetstrokecolor{currentstroke}%
\pgfsetdash{}{0pt}%
\pgfpathmoveto{\pgfqpoint{4.284366in}{0.438620in}}%
\pgfpathcurveto{\pgfqpoint{4.295416in}{0.438620in}}{\pgfqpoint{4.306015in}{0.443011in}}{\pgfqpoint{4.313828in}{0.450824in}}%
\pgfpathcurveto{\pgfqpoint{4.321642in}{0.458638in}}{\pgfqpoint{4.326032in}{0.469237in}}{\pgfqpoint{4.326032in}{0.480287in}}%
\pgfpathcurveto{\pgfqpoint{4.326032in}{0.491337in}}{\pgfqpoint{4.321642in}{0.501936in}}{\pgfqpoint{4.313828in}{0.509750in}}%
\pgfpathcurveto{\pgfqpoint{4.306015in}{0.517563in}}{\pgfqpoint{4.295416in}{0.521954in}}{\pgfqpoint{4.284366in}{0.521954in}}%
\pgfpathcurveto{\pgfqpoint{4.273315in}{0.521954in}}{\pgfqpoint{4.262716in}{0.517563in}}{\pgfqpoint{4.254903in}{0.509750in}}%
\pgfpathcurveto{\pgfqpoint{4.247089in}{0.501936in}}{\pgfqpoint{4.242699in}{0.491337in}}{\pgfqpoint{4.242699in}{0.480287in}}%
\pgfpathcurveto{\pgfqpoint{4.242699in}{0.469237in}}{\pgfqpoint{4.247089in}{0.458638in}}{\pgfqpoint{4.254903in}{0.450824in}}%
\pgfpathcurveto{\pgfqpoint{4.262716in}{0.443011in}}{\pgfqpoint{4.273315in}{0.438620in}}{\pgfqpoint{4.284366in}{0.438620in}}%
\pgfpathlineto{\pgfqpoint{4.284366in}{0.438620in}}%
\pgfpathclose%
\pgfusepath{stroke}%
\end{pgfscope}%
\begin{pgfscope}%
\pgfpathrectangle{\pgfqpoint{0.393053in}{0.375000in}}{\pgfqpoint{6.356833in}{5.175000in}}%
\pgfusepath{clip}%
\pgfsetbuttcap%
\pgfsetroundjoin%
\pgfsetlinewidth{1.003750pt}%
\definecolor{currentstroke}{rgb}{0.827451,0.827451,0.827451}%
\pgfsetstrokecolor{currentstroke}%
\pgfsetdash{}{0pt}%
\pgfpathmoveto{\pgfqpoint{0.997743in}{2.491784in}}%
\pgfpathcurveto{\pgfqpoint{1.008793in}{2.491784in}}{\pgfqpoint{1.019392in}{2.496174in}}{\pgfqpoint{1.027206in}{2.503988in}}%
\pgfpathcurveto{\pgfqpoint{1.035019in}{2.511802in}}{\pgfqpoint{1.039410in}{2.522401in}}{\pgfqpoint{1.039410in}{2.533451in}}%
\pgfpathcurveto{\pgfqpoint{1.039410in}{2.544501in}}{\pgfqpoint{1.035019in}{2.555100in}}{\pgfqpoint{1.027206in}{2.562914in}}%
\pgfpathcurveto{\pgfqpoint{1.019392in}{2.570727in}}{\pgfqpoint{1.008793in}{2.575118in}}{\pgfqpoint{0.997743in}{2.575118in}}%
\pgfpathcurveto{\pgfqpoint{0.986693in}{2.575118in}}{\pgfqpoint{0.976094in}{2.570727in}}{\pgfqpoint{0.968280in}{2.562914in}}%
\pgfpathcurveto{\pgfqpoint{0.960466in}{2.555100in}}{\pgfqpoint{0.956076in}{2.544501in}}{\pgfqpoint{0.956076in}{2.533451in}}%
\pgfpathcurveto{\pgfqpoint{0.956076in}{2.522401in}}{\pgfqpoint{0.960466in}{2.511802in}}{\pgfqpoint{0.968280in}{2.503988in}}%
\pgfpathcurveto{\pgfqpoint{0.976094in}{2.496174in}}{\pgfqpoint{0.986693in}{2.491784in}}{\pgfqpoint{0.997743in}{2.491784in}}%
\pgfpathlineto{\pgfqpoint{0.997743in}{2.491784in}}%
\pgfpathclose%
\pgfusepath{stroke}%
\end{pgfscope}%
\begin{pgfscope}%
\pgfpathrectangle{\pgfqpoint{0.393053in}{0.375000in}}{\pgfqpoint{6.356833in}{5.175000in}}%
\pgfusepath{clip}%
\pgfsetbuttcap%
\pgfsetroundjoin%
\pgfsetlinewidth{1.003750pt}%
\definecolor{currentstroke}{rgb}{0.827451,0.827451,0.827451}%
\pgfsetstrokecolor{currentstroke}%
\pgfsetdash{}{0pt}%
\pgfpathmoveto{\pgfqpoint{1.123487in}{2.207492in}}%
\pgfpathcurveto{\pgfqpoint{1.134537in}{2.207492in}}{\pgfqpoint{1.145136in}{2.211882in}}{\pgfqpoint{1.152950in}{2.219696in}}%
\pgfpathcurveto{\pgfqpoint{1.160763in}{2.227510in}}{\pgfqpoint{1.165153in}{2.238109in}}{\pgfqpoint{1.165153in}{2.249159in}}%
\pgfpathcurveto{\pgfqpoint{1.165153in}{2.260209in}}{\pgfqpoint{1.160763in}{2.270808in}}{\pgfqpoint{1.152950in}{2.278622in}}%
\pgfpathcurveto{\pgfqpoint{1.145136in}{2.286435in}}{\pgfqpoint{1.134537in}{2.290825in}}{\pgfqpoint{1.123487in}{2.290825in}}%
\pgfpathcurveto{\pgfqpoint{1.112437in}{2.290825in}}{\pgfqpoint{1.101838in}{2.286435in}}{\pgfqpoint{1.094024in}{2.278622in}}%
\pgfpathcurveto{\pgfqpoint{1.086210in}{2.270808in}}{\pgfqpoint{1.081820in}{2.260209in}}{\pgfqpoint{1.081820in}{2.249159in}}%
\pgfpathcurveto{\pgfqpoint{1.081820in}{2.238109in}}{\pgfqpoint{1.086210in}{2.227510in}}{\pgfqpoint{1.094024in}{2.219696in}}%
\pgfpathcurveto{\pgfqpoint{1.101838in}{2.211882in}}{\pgfqpoint{1.112437in}{2.207492in}}{\pgfqpoint{1.123487in}{2.207492in}}%
\pgfpathlineto{\pgfqpoint{1.123487in}{2.207492in}}%
\pgfpathclose%
\pgfusepath{stroke}%
\end{pgfscope}%
\begin{pgfscope}%
\pgfpathrectangle{\pgfqpoint{0.393053in}{0.375000in}}{\pgfqpoint{6.356833in}{5.175000in}}%
\pgfusepath{clip}%
\pgfsetbuttcap%
\pgfsetroundjoin%
\pgfsetlinewidth{1.003750pt}%
\definecolor{currentstroke}{rgb}{0.827451,0.827451,0.827451}%
\pgfsetstrokecolor{currentstroke}%
\pgfsetdash{}{0pt}%
\pgfpathmoveto{\pgfqpoint{2.145018in}{1.265722in}}%
\pgfpathcurveto{\pgfqpoint{2.156069in}{1.265722in}}{\pgfqpoint{2.166668in}{1.270112in}}{\pgfqpoint{2.174481in}{1.277926in}}%
\pgfpathcurveto{\pgfqpoint{2.182295in}{1.285739in}}{\pgfqpoint{2.186685in}{1.296338in}}{\pgfqpoint{2.186685in}{1.307389in}}%
\pgfpathcurveto{\pgfqpoint{2.186685in}{1.318439in}}{\pgfqpoint{2.182295in}{1.329038in}}{\pgfqpoint{2.174481in}{1.336851in}}%
\pgfpathcurveto{\pgfqpoint{2.166668in}{1.344665in}}{\pgfqpoint{2.156069in}{1.349055in}}{\pgfqpoint{2.145018in}{1.349055in}}%
\pgfpathcurveto{\pgfqpoint{2.133968in}{1.349055in}}{\pgfqpoint{2.123369in}{1.344665in}}{\pgfqpoint{2.115556in}{1.336851in}}%
\pgfpathcurveto{\pgfqpoint{2.107742in}{1.329038in}}{\pgfqpoint{2.103352in}{1.318439in}}{\pgfqpoint{2.103352in}{1.307389in}}%
\pgfpathcurveto{\pgfqpoint{2.103352in}{1.296338in}}{\pgfqpoint{2.107742in}{1.285739in}}{\pgfqpoint{2.115556in}{1.277926in}}%
\pgfpathcurveto{\pgfqpoint{2.123369in}{1.270112in}}{\pgfqpoint{2.133968in}{1.265722in}}{\pgfqpoint{2.145018in}{1.265722in}}%
\pgfpathlineto{\pgfqpoint{2.145018in}{1.265722in}}%
\pgfpathclose%
\pgfusepath{stroke}%
\end{pgfscope}%
\begin{pgfscope}%
\pgfpathrectangle{\pgfqpoint{0.393053in}{0.375000in}}{\pgfqpoint{6.356833in}{5.175000in}}%
\pgfusepath{clip}%
\pgfsetbuttcap%
\pgfsetroundjoin%
\pgfsetlinewidth{1.003750pt}%
\definecolor{currentstroke}{rgb}{0.827451,0.827451,0.827451}%
\pgfsetstrokecolor{currentstroke}%
\pgfsetdash{}{0pt}%
\pgfpathmoveto{\pgfqpoint{1.078224in}{2.311634in}}%
\pgfpathcurveto{\pgfqpoint{1.089274in}{2.311634in}}{\pgfqpoint{1.099873in}{2.316024in}}{\pgfqpoint{1.107686in}{2.323838in}}%
\pgfpathcurveto{\pgfqpoint{1.115500in}{2.331651in}}{\pgfqpoint{1.119890in}{2.342250in}}{\pgfqpoint{1.119890in}{2.353300in}}%
\pgfpathcurveto{\pgfqpoint{1.119890in}{2.364351in}}{\pgfqpoint{1.115500in}{2.374950in}}{\pgfqpoint{1.107686in}{2.382763in}}%
\pgfpathcurveto{\pgfqpoint{1.099873in}{2.390577in}}{\pgfqpoint{1.089274in}{2.394967in}}{\pgfqpoint{1.078224in}{2.394967in}}%
\pgfpathcurveto{\pgfqpoint{1.067173in}{2.394967in}}{\pgfqpoint{1.056574in}{2.390577in}}{\pgfqpoint{1.048761in}{2.382763in}}%
\pgfpathcurveto{\pgfqpoint{1.040947in}{2.374950in}}{\pgfqpoint{1.036557in}{2.364351in}}{\pgfqpoint{1.036557in}{2.353300in}}%
\pgfpathcurveto{\pgfqpoint{1.036557in}{2.342250in}}{\pgfqpoint{1.040947in}{2.331651in}}{\pgfqpoint{1.048761in}{2.323838in}}%
\pgfpathcurveto{\pgfqpoint{1.056574in}{2.316024in}}{\pgfqpoint{1.067173in}{2.311634in}}{\pgfqpoint{1.078224in}{2.311634in}}%
\pgfpathlineto{\pgfqpoint{1.078224in}{2.311634in}}%
\pgfpathclose%
\pgfusepath{stroke}%
\end{pgfscope}%
\begin{pgfscope}%
\pgfpathrectangle{\pgfqpoint{0.393053in}{0.375000in}}{\pgfqpoint{6.356833in}{5.175000in}}%
\pgfusepath{clip}%
\pgfsetbuttcap%
\pgfsetroundjoin%
\pgfsetlinewidth{1.003750pt}%
\definecolor{currentstroke}{rgb}{0.827451,0.827451,0.827451}%
\pgfsetstrokecolor{currentstroke}%
\pgfsetdash{}{0pt}%
\pgfpathmoveto{\pgfqpoint{0.651937in}{3.079342in}}%
\pgfpathcurveto{\pgfqpoint{0.662987in}{3.079342in}}{\pgfqpoint{0.673586in}{3.083732in}}{\pgfqpoint{0.681400in}{3.091546in}}%
\pgfpathcurveto{\pgfqpoint{0.689213in}{3.099359in}}{\pgfqpoint{0.693604in}{3.109958in}}{\pgfqpoint{0.693604in}{3.121009in}}%
\pgfpathcurveto{\pgfqpoint{0.693604in}{3.132059in}}{\pgfqpoint{0.689213in}{3.142658in}}{\pgfqpoint{0.681400in}{3.150471in}}%
\pgfpathcurveto{\pgfqpoint{0.673586in}{3.158285in}}{\pgfqpoint{0.662987in}{3.162675in}}{\pgfqpoint{0.651937in}{3.162675in}}%
\pgfpathcurveto{\pgfqpoint{0.640887in}{3.162675in}}{\pgfqpoint{0.630288in}{3.158285in}}{\pgfqpoint{0.622474in}{3.150471in}}%
\pgfpathcurveto{\pgfqpoint{0.614660in}{3.142658in}}{\pgfqpoint{0.610270in}{3.132059in}}{\pgfqpoint{0.610270in}{3.121009in}}%
\pgfpathcurveto{\pgfqpoint{0.610270in}{3.109958in}}{\pgfqpoint{0.614660in}{3.099359in}}{\pgfqpoint{0.622474in}{3.091546in}}%
\pgfpathcurveto{\pgfqpoint{0.630288in}{3.083732in}}{\pgfqpoint{0.640887in}{3.079342in}}{\pgfqpoint{0.651937in}{3.079342in}}%
\pgfpathlineto{\pgfqpoint{0.651937in}{3.079342in}}%
\pgfpathclose%
\pgfusepath{stroke}%
\end{pgfscope}%
\begin{pgfscope}%
\pgfpathrectangle{\pgfqpoint{0.393053in}{0.375000in}}{\pgfqpoint{6.356833in}{5.175000in}}%
\pgfusepath{clip}%
\pgfsetbuttcap%
\pgfsetroundjoin%
\pgfsetlinewidth{1.003750pt}%
\definecolor{currentstroke}{rgb}{0.827451,0.827451,0.827451}%
\pgfsetstrokecolor{currentstroke}%
\pgfsetdash{}{0pt}%
\pgfpathmoveto{\pgfqpoint{1.680974in}{1.628980in}}%
\pgfpathcurveto{\pgfqpoint{1.692024in}{1.628980in}}{\pgfqpoint{1.702623in}{1.633370in}}{\pgfqpoint{1.710437in}{1.641184in}}%
\pgfpathcurveto{\pgfqpoint{1.718250in}{1.648997in}}{\pgfqpoint{1.722641in}{1.659596in}}{\pgfqpoint{1.722641in}{1.670646in}}%
\pgfpathcurveto{\pgfqpoint{1.722641in}{1.681697in}}{\pgfqpoint{1.718250in}{1.692296in}}{\pgfqpoint{1.710437in}{1.700109in}}%
\pgfpathcurveto{\pgfqpoint{1.702623in}{1.707923in}}{\pgfqpoint{1.692024in}{1.712313in}}{\pgfqpoint{1.680974in}{1.712313in}}%
\pgfpathcurveto{\pgfqpoint{1.669924in}{1.712313in}}{\pgfqpoint{1.659325in}{1.707923in}}{\pgfqpoint{1.651511in}{1.700109in}}%
\pgfpathcurveto{\pgfqpoint{1.643698in}{1.692296in}}{\pgfqpoint{1.639307in}{1.681697in}}{\pgfqpoint{1.639307in}{1.670646in}}%
\pgfpathcurveto{\pgfqpoint{1.639307in}{1.659596in}}{\pgfqpoint{1.643698in}{1.648997in}}{\pgfqpoint{1.651511in}{1.641184in}}%
\pgfpathcurveto{\pgfqpoint{1.659325in}{1.633370in}}{\pgfqpoint{1.669924in}{1.628980in}}{\pgfqpoint{1.680974in}{1.628980in}}%
\pgfpathlineto{\pgfqpoint{1.680974in}{1.628980in}}%
\pgfpathclose%
\pgfusepath{stroke}%
\end{pgfscope}%
\begin{pgfscope}%
\pgfpathrectangle{\pgfqpoint{0.393053in}{0.375000in}}{\pgfqpoint{6.356833in}{5.175000in}}%
\pgfusepath{clip}%
\pgfsetbuttcap%
\pgfsetroundjoin%
\pgfsetlinewidth{1.003750pt}%
\definecolor{currentstroke}{rgb}{0.827451,0.827451,0.827451}%
\pgfsetstrokecolor{currentstroke}%
\pgfsetdash{}{0pt}%
\pgfpathmoveto{\pgfqpoint{1.843211in}{1.515821in}}%
\pgfpathcurveto{\pgfqpoint{1.854261in}{1.515821in}}{\pgfqpoint{1.864860in}{1.520211in}}{\pgfqpoint{1.872674in}{1.528025in}}%
\pgfpathcurveto{\pgfqpoint{1.880487in}{1.535838in}}{\pgfqpoint{1.884878in}{1.546437in}}{\pgfqpoint{1.884878in}{1.557487in}}%
\pgfpathcurveto{\pgfqpoint{1.884878in}{1.568537in}}{\pgfqpoint{1.880487in}{1.579136in}}{\pgfqpoint{1.872674in}{1.586950in}}%
\pgfpathcurveto{\pgfqpoint{1.864860in}{1.594764in}}{\pgfqpoint{1.854261in}{1.599154in}}{\pgfqpoint{1.843211in}{1.599154in}}%
\pgfpathcurveto{\pgfqpoint{1.832161in}{1.599154in}}{\pgfqpoint{1.821562in}{1.594764in}}{\pgfqpoint{1.813748in}{1.586950in}}%
\pgfpathcurveto{\pgfqpoint{1.805934in}{1.579136in}}{\pgfqpoint{1.801544in}{1.568537in}}{\pgfqpoint{1.801544in}{1.557487in}}%
\pgfpathcurveto{\pgfqpoint{1.801544in}{1.546437in}}{\pgfqpoint{1.805934in}{1.535838in}}{\pgfqpoint{1.813748in}{1.528025in}}%
\pgfpathcurveto{\pgfqpoint{1.821562in}{1.520211in}}{\pgfqpoint{1.832161in}{1.515821in}}{\pgfqpoint{1.843211in}{1.515821in}}%
\pgfpathlineto{\pgfqpoint{1.843211in}{1.515821in}}%
\pgfpathclose%
\pgfusepath{stroke}%
\end{pgfscope}%
\begin{pgfscope}%
\pgfpathrectangle{\pgfqpoint{0.393053in}{0.375000in}}{\pgfqpoint{6.356833in}{5.175000in}}%
\pgfusepath{clip}%
\pgfsetbuttcap%
\pgfsetroundjoin%
\pgfsetlinewidth{1.003750pt}%
\definecolor{currentstroke}{rgb}{0.827451,0.827451,0.827451}%
\pgfsetstrokecolor{currentstroke}%
\pgfsetdash{}{0pt}%
\pgfpathmoveto{\pgfqpoint{2.632835in}{1.000527in}}%
\pgfpathcurveto{\pgfqpoint{2.643885in}{1.000527in}}{\pgfqpoint{2.654484in}{1.004918in}}{\pgfqpoint{2.662297in}{1.012731in}}%
\pgfpathcurveto{\pgfqpoint{2.670111in}{1.020545in}}{\pgfqpoint{2.674501in}{1.031144in}}{\pgfqpoint{2.674501in}{1.042194in}}%
\pgfpathcurveto{\pgfqpoint{2.674501in}{1.053244in}}{\pgfqpoint{2.670111in}{1.063843in}}{\pgfqpoint{2.662297in}{1.071657in}}%
\pgfpathcurveto{\pgfqpoint{2.654484in}{1.079471in}}{\pgfqpoint{2.643885in}{1.083861in}}{\pgfqpoint{2.632835in}{1.083861in}}%
\pgfpathcurveto{\pgfqpoint{2.621785in}{1.083861in}}{\pgfqpoint{2.611186in}{1.079471in}}{\pgfqpoint{2.603372in}{1.071657in}}%
\pgfpathcurveto{\pgfqpoint{2.595558in}{1.063843in}}{\pgfqpoint{2.591168in}{1.053244in}}{\pgfqpoint{2.591168in}{1.042194in}}%
\pgfpathcurveto{\pgfqpoint{2.591168in}{1.031144in}}{\pgfqpoint{2.595558in}{1.020545in}}{\pgfqpoint{2.603372in}{1.012731in}}%
\pgfpathcurveto{\pgfqpoint{2.611186in}{1.004918in}}{\pgfqpoint{2.621785in}{1.000527in}}{\pgfqpoint{2.632835in}{1.000527in}}%
\pgfpathlineto{\pgfqpoint{2.632835in}{1.000527in}}%
\pgfpathclose%
\pgfusepath{stroke}%
\end{pgfscope}%
\begin{pgfscope}%
\pgfpathrectangle{\pgfqpoint{0.393053in}{0.375000in}}{\pgfqpoint{6.356833in}{5.175000in}}%
\pgfusepath{clip}%
\pgfsetbuttcap%
\pgfsetroundjoin%
\pgfsetlinewidth{1.003750pt}%
\definecolor{currentstroke}{rgb}{0.827451,0.827451,0.827451}%
\pgfsetstrokecolor{currentstroke}%
\pgfsetdash{}{0pt}%
\pgfpathmoveto{\pgfqpoint{0.429666in}{3.982254in}}%
\pgfpathcurveto{\pgfqpoint{0.440716in}{3.982254in}}{\pgfqpoint{0.451315in}{3.986644in}}{\pgfqpoint{0.459129in}{3.994458in}}%
\pgfpathcurveto{\pgfqpoint{0.466942in}{4.002271in}}{\pgfqpoint{0.471332in}{4.012870in}}{\pgfqpoint{0.471332in}{4.023920in}}%
\pgfpathcurveto{\pgfqpoint{0.471332in}{4.034970in}}{\pgfqpoint{0.466942in}{4.045570in}}{\pgfqpoint{0.459129in}{4.053383in}}%
\pgfpathcurveto{\pgfqpoint{0.451315in}{4.061197in}}{\pgfqpoint{0.440716in}{4.065587in}}{\pgfqpoint{0.429666in}{4.065587in}}%
\pgfpathcurveto{\pgfqpoint{0.418616in}{4.065587in}}{\pgfqpoint{0.408017in}{4.061197in}}{\pgfqpoint{0.400203in}{4.053383in}}%
\pgfpathcurveto{\pgfqpoint{0.392389in}{4.045570in}}{\pgfqpoint{0.387999in}{4.034970in}}{\pgfqpoint{0.387999in}{4.023920in}}%
\pgfpathcurveto{\pgfqpoint{0.387999in}{4.012870in}}{\pgfqpoint{0.392389in}{4.002271in}}{\pgfqpoint{0.400203in}{3.994458in}}%
\pgfpathcurveto{\pgfqpoint{0.408017in}{3.986644in}}{\pgfqpoint{0.418616in}{3.982254in}}{\pgfqpoint{0.429666in}{3.982254in}}%
\pgfpathlineto{\pgfqpoint{0.429666in}{3.982254in}}%
\pgfpathclose%
\pgfusepath{stroke}%
\end{pgfscope}%
\begin{pgfscope}%
\pgfpathrectangle{\pgfqpoint{0.393053in}{0.375000in}}{\pgfqpoint{6.356833in}{5.175000in}}%
\pgfusepath{clip}%
\pgfsetbuttcap%
\pgfsetroundjoin%
\pgfsetlinewidth{1.003750pt}%
\definecolor{currentstroke}{rgb}{0.827451,0.827451,0.827451}%
\pgfsetstrokecolor{currentstroke}%
\pgfsetdash{}{0pt}%
\pgfpathmoveto{\pgfqpoint{2.905386in}{0.884940in}}%
\pgfpathcurveto{\pgfqpoint{2.916436in}{0.884940in}}{\pgfqpoint{2.927035in}{0.889330in}}{\pgfqpoint{2.934849in}{0.897144in}}%
\pgfpathcurveto{\pgfqpoint{2.942663in}{0.904957in}}{\pgfqpoint{2.947053in}{0.915556in}}{\pgfqpoint{2.947053in}{0.926606in}}%
\pgfpathcurveto{\pgfqpoint{2.947053in}{0.937657in}}{\pgfqpoint{2.942663in}{0.948256in}}{\pgfqpoint{2.934849in}{0.956069in}}%
\pgfpathcurveto{\pgfqpoint{2.927035in}{0.963883in}}{\pgfqpoint{2.916436in}{0.968273in}}{\pgfqpoint{2.905386in}{0.968273in}}%
\pgfpathcurveto{\pgfqpoint{2.894336in}{0.968273in}}{\pgfqpoint{2.883737in}{0.963883in}}{\pgfqpoint{2.875923in}{0.956069in}}%
\pgfpathcurveto{\pgfqpoint{2.868110in}{0.948256in}}{\pgfqpoint{2.863720in}{0.937657in}}{\pgfqpoint{2.863720in}{0.926606in}}%
\pgfpathcurveto{\pgfqpoint{2.863720in}{0.915556in}}{\pgfqpoint{2.868110in}{0.904957in}}{\pgfqpoint{2.875923in}{0.897144in}}%
\pgfpathcurveto{\pgfqpoint{2.883737in}{0.889330in}}{\pgfqpoint{2.894336in}{0.884940in}}{\pgfqpoint{2.905386in}{0.884940in}}%
\pgfpathlineto{\pgfqpoint{2.905386in}{0.884940in}}%
\pgfpathclose%
\pgfusepath{stroke}%
\end{pgfscope}%
\begin{pgfscope}%
\pgfpathrectangle{\pgfqpoint{0.393053in}{0.375000in}}{\pgfqpoint{6.356833in}{5.175000in}}%
\pgfusepath{clip}%
\pgfsetbuttcap%
\pgfsetroundjoin%
\pgfsetlinewidth{1.003750pt}%
\definecolor{currentstroke}{rgb}{0.827451,0.827451,0.827451}%
\pgfsetstrokecolor{currentstroke}%
\pgfsetdash{}{0pt}%
\pgfpathmoveto{\pgfqpoint{0.519167in}{3.558335in}}%
\pgfpathcurveto{\pgfqpoint{0.530217in}{3.558335in}}{\pgfqpoint{0.540816in}{3.562726in}}{\pgfqpoint{0.548630in}{3.570539in}}%
\pgfpathcurveto{\pgfqpoint{0.556443in}{3.578353in}}{\pgfqpoint{0.560834in}{3.588952in}}{\pgfqpoint{0.560834in}{3.600002in}}%
\pgfpathcurveto{\pgfqpoint{0.560834in}{3.611052in}}{\pgfqpoint{0.556443in}{3.621651in}}{\pgfqpoint{0.548630in}{3.629465in}}%
\pgfpathcurveto{\pgfqpoint{0.540816in}{3.637278in}}{\pgfqpoint{0.530217in}{3.641669in}}{\pgfqpoint{0.519167in}{3.641669in}}%
\pgfpathcurveto{\pgfqpoint{0.508117in}{3.641669in}}{\pgfqpoint{0.497518in}{3.637278in}}{\pgfqpoint{0.489704in}{3.629465in}}%
\pgfpathcurveto{\pgfqpoint{0.481891in}{3.621651in}}{\pgfqpoint{0.477500in}{3.611052in}}{\pgfqpoint{0.477500in}{3.600002in}}%
\pgfpathcurveto{\pgfqpoint{0.477500in}{3.588952in}}{\pgfqpoint{0.481891in}{3.578353in}}{\pgfqpoint{0.489704in}{3.570539in}}%
\pgfpathcurveto{\pgfqpoint{0.497518in}{3.562726in}}{\pgfqpoint{0.508117in}{3.558335in}}{\pgfqpoint{0.519167in}{3.558335in}}%
\pgfpathlineto{\pgfqpoint{0.519167in}{3.558335in}}%
\pgfpathclose%
\pgfusepath{stroke}%
\end{pgfscope}%
\begin{pgfscope}%
\pgfpathrectangle{\pgfqpoint{0.393053in}{0.375000in}}{\pgfqpoint{6.356833in}{5.175000in}}%
\pgfusepath{clip}%
\pgfsetbuttcap%
\pgfsetroundjoin%
\pgfsetlinewidth{1.003750pt}%
\definecolor{currentstroke}{rgb}{0.827451,0.827451,0.827451}%
\pgfsetstrokecolor{currentstroke}%
\pgfsetdash{}{0pt}%
\pgfpathmoveto{\pgfqpoint{1.186674in}{2.128077in}}%
\pgfpathcurveto{\pgfqpoint{1.197724in}{2.128077in}}{\pgfqpoint{1.208323in}{2.132467in}}{\pgfqpoint{1.216137in}{2.140281in}}%
\pgfpathcurveto{\pgfqpoint{1.223950in}{2.148095in}}{\pgfqpoint{1.228341in}{2.158694in}}{\pgfqpoint{1.228341in}{2.169744in}}%
\pgfpathcurveto{\pgfqpoint{1.228341in}{2.180794in}}{\pgfqpoint{1.223950in}{2.191393in}}{\pgfqpoint{1.216137in}{2.199207in}}%
\pgfpathcurveto{\pgfqpoint{1.208323in}{2.207020in}}{\pgfqpoint{1.197724in}{2.211411in}}{\pgfqpoint{1.186674in}{2.211411in}}%
\pgfpathcurveto{\pgfqpoint{1.175624in}{2.211411in}}{\pgfqpoint{1.165025in}{2.207020in}}{\pgfqpoint{1.157211in}{2.199207in}}%
\pgfpathcurveto{\pgfqpoint{1.149397in}{2.191393in}}{\pgfqpoint{1.145007in}{2.180794in}}{\pgfqpoint{1.145007in}{2.169744in}}%
\pgfpathcurveto{\pgfqpoint{1.145007in}{2.158694in}}{\pgfqpoint{1.149397in}{2.148095in}}{\pgfqpoint{1.157211in}{2.140281in}}%
\pgfpathcurveto{\pgfqpoint{1.165025in}{2.132467in}}{\pgfqpoint{1.175624in}{2.128077in}}{\pgfqpoint{1.186674in}{2.128077in}}%
\pgfpathlineto{\pgfqpoint{1.186674in}{2.128077in}}%
\pgfpathclose%
\pgfusepath{stroke}%
\end{pgfscope}%
\begin{pgfscope}%
\pgfpathrectangle{\pgfqpoint{0.393053in}{0.375000in}}{\pgfqpoint{6.356833in}{5.175000in}}%
\pgfusepath{clip}%
\pgfsetbuttcap%
\pgfsetroundjoin%
\pgfsetlinewidth{1.003750pt}%
\definecolor{currentstroke}{rgb}{0.827451,0.827451,0.827451}%
\pgfsetstrokecolor{currentstroke}%
\pgfsetdash{}{0pt}%
\pgfpathmoveto{\pgfqpoint{1.354968in}{1.928632in}}%
\pgfpathcurveto{\pgfqpoint{1.366018in}{1.928632in}}{\pgfqpoint{1.376617in}{1.933022in}}{\pgfqpoint{1.384431in}{1.940836in}}%
\pgfpathcurveto{\pgfqpoint{1.392245in}{1.948650in}}{\pgfqpoint{1.396635in}{1.959249in}}{\pgfqpoint{1.396635in}{1.970299in}}%
\pgfpathcurveto{\pgfqpoint{1.396635in}{1.981349in}}{\pgfqpoint{1.392245in}{1.991948in}}{\pgfqpoint{1.384431in}{1.999762in}}%
\pgfpathcurveto{\pgfqpoint{1.376617in}{2.007575in}}{\pgfqpoint{1.366018in}{2.011966in}}{\pgfqpoint{1.354968in}{2.011966in}}%
\pgfpathcurveto{\pgfqpoint{1.343918in}{2.011966in}}{\pgfqpoint{1.333319in}{2.007575in}}{\pgfqpoint{1.325505in}{1.999762in}}%
\pgfpathcurveto{\pgfqpoint{1.317692in}{1.991948in}}{\pgfqpoint{1.313302in}{1.981349in}}{\pgfqpoint{1.313302in}{1.970299in}}%
\pgfpathcurveto{\pgfqpoint{1.313302in}{1.959249in}}{\pgfqpoint{1.317692in}{1.948650in}}{\pgfqpoint{1.325505in}{1.940836in}}%
\pgfpathcurveto{\pgfqpoint{1.333319in}{1.933022in}}{\pgfqpoint{1.343918in}{1.928632in}}{\pgfqpoint{1.354968in}{1.928632in}}%
\pgfpathlineto{\pgfqpoint{1.354968in}{1.928632in}}%
\pgfpathclose%
\pgfusepath{stroke}%
\end{pgfscope}%
\begin{pgfscope}%
\pgfpathrectangle{\pgfqpoint{0.393053in}{0.375000in}}{\pgfqpoint{6.356833in}{5.175000in}}%
\pgfusepath{clip}%
\pgfsetbuttcap%
\pgfsetroundjoin%
\pgfsetlinewidth{1.003750pt}%
\definecolor{currentstroke}{rgb}{0.827451,0.827451,0.827451}%
\pgfsetstrokecolor{currentstroke}%
\pgfsetdash{}{0pt}%
\pgfpathmoveto{\pgfqpoint{0.423946in}{4.069027in}}%
\pgfpathcurveto{\pgfqpoint{0.434996in}{4.069027in}}{\pgfqpoint{0.445595in}{4.073417in}}{\pgfqpoint{0.453408in}{4.081231in}}%
\pgfpathcurveto{\pgfqpoint{0.461222in}{4.089045in}}{\pgfqpoint{0.465612in}{4.099644in}}{\pgfqpoint{0.465612in}{4.110694in}}%
\pgfpathcurveto{\pgfqpoint{0.465612in}{4.121744in}}{\pgfqpoint{0.461222in}{4.132343in}}{\pgfqpoint{0.453408in}{4.140157in}}%
\pgfpathcurveto{\pgfqpoint{0.445595in}{4.147970in}}{\pgfqpoint{0.434996in}{4.152361in}}{\pgfqpoint{0.423946in}{4.152361in}}%
\pgfpathcurveto{\pgfqpoint{0.412896in}{4.152361in}}{\pgfqpoint{0.402297in}{4.147970in}}{\pgfqpoint{0.394483in}{4.140157in}}%
\pgfpathcurveto{\pgfqpoint{0.386669in}{4.132343in}}{\pgfqpoint{0.382279in}{4.121744in}}{\pgfqpoint{0.382279in}{4.110694in}}%
\pgfpathcurveto{\pgfqpoint{0.382279in}{4.099644in}}{\pgfqpoint{0.386669in}{4.089045in}}{\pgfqpoint{0.394483in}{4.081231in}}%
\pgfpathcurveto{\pgfqpoint{0.402297in}{4.073417in}}{\pgfqpoint{0.412896in}{4.069027in}}{\pgfqpoint{0.423946in}{4.069027in}}%
\pgfpathlineto{\pgfqpoint{0.423946in}{4.069027in}}%
\pgfpathclose%
\pgfusepath{stroke}%
\end{pgfscope}%
\begin{pgfscope}%
\pgfpathrectangle{\pgfqpoint{0.393053in}{0.375000in}}{\pgfqpoint{6.356833in}{5.175000in}}%
\pgfusepath{clip}%
\pgfsetbuttcap%
\pgfsetroundjoin%
\pgfsetlinewidth{1.003750pt}%
\definecolor{currentstroke}{rgb}{0.827451,0.827451,0.827451}%
\pgfsetstrokecolor{currentstroke}%
\pgfsetdash{}{0pt}%
\pgfpathmoveto{\pgfqpoint{0.646474in}{3.151616in}}%
\pgfpathcurveto{\pgfqpoint{0.657525in}{3.151616in}}{\pgfqpoint{0.668124in}{3.156006in}}{\pgfqpoint{0.675937in}{3.163820in}}%
\pgfpathcurveto{\pgfqpoint{0.683751in}{3.171634in}}{\pgfqpoint{0.688141in}{3.182233in}}{\pgfqpoint{0.688141in}{3.193283in}}%
\pgfpathcurveto{\pgfqpoint{0.688141in}{3.204333in}}{\pgfqpoint{0.683751in}{3.214932in}}{\pgfqpoint{0.675937in}{3.222745in}}%
\pgfpathcurveto{\pgfqpoint{0.668124in}{3.230559in}}{\pgfqpoint{0.657525in}{3.234949in}}{\pgfqpoint{0.646474in}{3.234949in}}%
\pgfpathcurveto{\pgfqpoint{0.635424in}{3.234949in}}{\pgfqpoint{0.624825in}{3.230559in}}{\pgfqpoint{0.617012in}{3.222745in}}%
\pgfpathcurveto{\pgfqpoint{0.609198in}{3.214932in}}{\pgfqpoint{0.604808in}{3.204333in}}{\pgfqpoint{0.604808in}{3.193283in}}%
\pgfpathcurveto{\pgfqpoint{0.604808in}{3.182233in}}{\pgfqpoint{0.609198in}{3.171634in}}{\pgfqpoint{0.617012in}{3.163820in}}%
\pgfpathcurveto{\pgfqpoint{0.624825in}{3.156006in}}{\pgfqpoint{0.635424in}{3.151616in}}{\pgfqpoint{0.646474in}{3.151616in}}%
\pgfpathlineto{\pgfqpoint{0.646474in}{3.151616in}}%
\pgfpathclose%
\pgfusepath{stroke}%
\end{pgfscope}%
\begin{pgfscope}%
\pgfpathrectangle{\pgfqpoint{0.393053in}{0.375000in}}{\pgfqpoint{6.356833in}{5.175000in}}%
\pgfusepath{clip}%
\pgfsetbuttcap%
\pgfsetroundjoin%
\pgfsetlinewidth{1.003750pt}%
\definecolor{currentstroke}{rgb}{0.827451,0.827451,0.827451}%
\pgfsetstrokecolor{currentstroke}%
\pgfsetdash{}{0pt}%
\pgfpathmoveto{\pgfqpoint{1.130293in}{2.198725in}}%
\pgfpathcurveto{\pgfqpoint{1.141343in}{2.198725in}}{\pgfqpoint{1.151942in}{2.203115in}}{\pgfqpoint{1.159755in}{2.210929in}}%
\pgfpathcurveto{\pgfqpoint{1.167569in}{2.218743in}}{\pgfqpoint{1.171959in}{2.229342in}}{\pgfqpoint{1.171959in}{2.240392in}}%
\pgfpathcurveto{\pgfqpoint{1.171959in}{2.251442in}}{\pgfqpoint{1.167569in}{2.262041in}}{\pgfqpoint{1.159755in}{2.269854in}}%
\pgfpathcurveto{\pgfqpoint{1.151942in}{2.277668in}}{\pgfqpoint{1.141343in}{2.282058in}}{\pgfqpoint{1.130293in}{2.282058in}}%
\pgfpathcurveto{\pgfqpoint{1.119243in}{2.282058in}}{\pgfqpoint{1.108644in}{2.277668in}}{\pgfqpoint{1.100830in}{2.269854in}}%
\pgfpathcurveto{\pgfqpoint{1.093016in}{2.262041in}}{\pgfqpoint{1.088626in}{2.251442in}}{\pgfqpoint{1.088626in}{2.240392in}}%
\pgfpathcurveto{\pgfqpoint{1.088626in}{2.229342in}}{\pgfqpoint{1.093016in}{2.218743in}}{\pgfqpoint{1.100830in}{2.210929in}}%
\pgfpathcurveto{\pgfqpoint{1.108644in}{2.203115in}}{\pgfqpoint{1.119243in}{2.198725in}}{\pgfqpoint{1.130293in}{2.198725in}}%
\pgfpathlineto{\pgfqpoint{1.130293in}{2.198725in}}%
\pgfpathclose%
\pgfusepath{stroke}%
\end{pgfscope}%
\begin{pgfscope}%
\pgfpathrectangle{\pgfqpoint{0.393053in}{0.375000in}}{\pgfqpoint{6.356833in}{5.175000in}}%
\pgfusepath{clip}%
\pgfsetbuttcap%
\pgfsetroundjoin%
\pgfsetlinewidth{1.003750pt}%
\definecolor{currentstroke}{rgb}{0.827451,0.827451,0.827451}%
\pgfsetstrokecolor{currentstroke}%
\pgfsetdash{}{0pt}%
\pgfpathmoveto{\pgfqpoint{1.980937in}{1.385443in}}%
\pgfpathcurveto{\pgfqpoint{1.991987in}{1.385443in}}{\pgfqpoint{2.002586in}{1.389833in}}{\pgfqpoint{2.010399in}{1.397647in}}%
\pgfpathcurveto{\pgfqpoint{2.018213in}{1.405460in}}{\pgfqpoint{2.022603in}{1.416059in}}{\pgfqpoint{2.022603in}{1.427109in}}%
\pgfpathcurveto{\pgfqpoint{2.022603in}{1.438160in}}{\pgfqpoint{2.018213in}{1.448759in}}{\pgfqpoint{2.010399in}{1.456572in}}%
\pgfpathcurveto{\pgfqpoint{2.002586in}{1.464386in}}{\pgfqpoint{1.991987in}{1.468776in}}{\pgfqpoint{1.980937in}{1.468776in}}%
\pgfpathcurveto{\pgfqpoint{1.969886in}{1.468776in}}{\pgfqpoint{1.959287in}{1.464386in}}{\pgfqpoint{1.951474in}{1.456572in}}%
\pgfpathcurveto{\pgfqpoint{1.943660in}{1.448759in}}{\pgfqpoint{1.939270in}{1.438160in}}{\pgfqpoint{1.939270in}{1.427109in}}%
\pgfpathcurveto{\pgfqpoint{1.939270in}{1.416059in}}{\pgfqpoint{1.943660in}{1.405460in}}{\pgfqpoint{1.951474in}{1.397647in}}%
\pgfpathcurveto{\pgfqpoint{1.959287in}{1.389833in}}{\pgfqpoint{1.969886in}{1.385443in}}{\pgfqpoint{1.980937in}{1.385443in}}%
\pgfpathlineto{\pgfqpoint{1.980937in}{1.385443in}}%
\pgfpathclose%
\pgfusepath{stroke}%
\end{pgfscope}%
\begin{pgfscope}%
\pgfpathrectangle{\pgfqpoint{0.393053in}{0.375000in}}{\pgfqpoint{6.356833in}{5.175000in}}%
\pgfusepath{clip}%
\pgfsetbuttcap%
\pgfsetroundjoin%
\pgfsetlinewidth{1.003750pt}%
\definecolor{currentstroke}{rgb}{0.827451,0.827451,0.827451}%
\pgfsetstrokecolor{currentstroke}%
\pgfsetdash{}{0pt}%
\pgfpathmoveto{\pgfqpoint{0.538696in}{3.440120in}}%
\pgfpathcurveto{\pgfqpoint{0.549746in}{3.440120in}}{\pgfqpoint{0.560345in}{3.444510in}}{\pgfqpoint{0.568159in}{3.452323in}}%
\pgfpathcurveto{\pgfqpoint{0.575972in}{3.460137in}}{\pgfqpoint{0.580363in}{3.470736in}}{\pgfqpoint{0.580363in}{3.481786in}}%
\pgfpathcurveto{\pgfqpoint{0.580363in}{3.492836in}}{\pgfqpoint{0.575972in}{3.503435in}}{\pgfqpoint{0.568159in}{3.511249in}}%
\pgfpathcurveto{\pgfqpoint{0.560345in}{3.519063in}}{\pgfqpoint{0.549746in}{3.523453in}}{\pgfqpoint{0.538696in}{3.523453in}}%
\pgfpathcurveto{\pgfqpoint{0.527646in}{3.523453in}}{\pgfqpoint{0.517047in}{3.519063in}}{\pgfqpoint{0.509233in}{3.511249in}}%
\pgfpathcurveto{\pgfqpoint{0.501420in}{3.503435in}}{\pgfqpoint{0.497029in}{3.492836in}}{\pgfqpoint{0.497029in}{3.481786in}}%
\pgfpathcurveto{\pgfqpoint{0.497029in}{3.470736in}}{\pgfqpoint{0.501420in}{3.460137in}}{\pgfqpoint{0.509233in}{3.452323in}}%
\pgfpathcurveto{\pgfqpoint{0.517047in}{3.444510in}}{\pgfqpoint{0.527646in}{3.440120in}}{\pgfqpoint{0.538696in}{3.440120in}}%
\pgfpathlineto{\pgfqpoint{0.538696in}{3.440120in}}%
\pgfpathclose%
\pgfusepath{stroke}%
\end{pgfscope}%
\begin{pgfscope}%
\pgfpathrectangle{\pgfqpoint{0.393053in}{0.375000in}}{\pgfqpoint{6.356833in}{5.175000in}}%
\pgfusepath{clip}%
\pgfsetbuttcap%
\pgfsetroundjoin%
\pgfsetlinewidth{1.003750pt}%
\definecolor{currentstroke}{rgb}{0.827451,0.827451,0.827451}%
\pgfsetstrokecolor{currentstroke}%
\pgfsetdash{}{0pt}%
\pgfpathmoveto{\pgfqpoint{2.585362in}{1.005073in}}%
\pgfpathcurveto{\pgfqpoint{2.596412in}{1.005073in}}{\pgfqpoint{2.607011in}{1.009463in}}{\pgfqpoint{2.614824in}{1.017277in}}%
\pgfpathcurveto{\pgfqpoint{2.622638in}{1.025090in}}{\pgfqpoint{2.627028in}{1.035689in}}{\pgfqpoint{2.627028in}{1.046740in}}%
\pgfpathcurveto{\pgfqpoint{2.627028in}{1.057790in}}{\pgfqpoint{2.622638in}{1.068389in}}{\pgfqpoint{2.614824in}{1.076202in}}%
\pgfpathcurveto{\pgfqpoint{2.607011in}{1.084016in}}{\pgfqpoint{2.596412in}{1.088406in}}{\pgfqpoint{2.585362in}{1.088406in}}%
\pgfpathcurveto{\pgfqpoint{2.574311in}{1.088406in}}{\pgfqpoint{2.563712in}{1.084016in}}{\pgfqpoint{2.555899in}{1.076202in}}%
\pgfpathcurveto{\pgfqpoint{2.548085in}{1.068389in}}{\pgfqpoint{2.543695in}{1.057790in}}{\pgfqpoint{2.543695in}{1.046740in}}%
\pgfpathcurveto{\pgfqpoint{2.543695in}{1.035689in}}{\pgfqpoint{2.548085in}{1.025090in}}{\pgfqpoint{2.555899in}{1.017277in}}%
\pgfpathcurveto{\pgfqpoint{2.563712in}{1.009463in}}{\pgfqpoint{2.574311in}{1.005073in}}{\pgfqpoint{2.585362in}{1.005073in}}%
\pgfpathlineto{\pgfqpoint{2.585362in}{1.005073in}}%
\pgfpathclose%
\pgfusepath{stroke}%
\end{pgfscope}%
\begin{pgfscope}%
\pgfpathrectangle{\pgfqpoint{0.393053in}{0.375000in}}{\pgfqpoint{6.356833in}{5.175000in}}%
\pgfusepath{clip}%
\pgfsetbuttcap%
\pgfsetroundjoin%
\pgfsetlinewidth{1.003750pt}%
\definecolor{currentstroke}{rgb}{0.827451,0.827451,0.827451}%
\pgfsetstrokecolor{currentstroke}%
\pgfsetdash{}{0pt}%
\pgfpathmoveto{\pgfqpoint{1.306751in}{1.985193in}}%
\pgfpathcurveto{\pgfqpoint{1.317801in}{1.985193in}}{\pgfqpoint{1.328400in}{1.989583in}}{\pgfqpoint{1.336214in}{1.997397in}}%
\pgfpathcurveto{\pgfqpoint{1.344027in}{2.005210in}}{\pgfqpoint{1.348418in}{2.015809in}}{\pgfqpoint{1.348418in}{2.026859in}}%
\pgfpathcurveto{\pgfqpoint{1.348418in}{2.037909in}}{\pgfqpoint{1.344027in}{2.048509in}}{\pgfqpoint{1.336214in}{2.056322in}}%
\pgfpathcurveto{\pgfqpoint{1.328400in}{2.064136in}}{\pgfqpoint{1.317801in}{2.068526in}}{\pgfqpoint{1.306751in}{2.068526in}}%
\pgfpathcurveto{\pgfqpoint{1.295701in}{2.068526in}}{\pgfqpoint{1.285102in}{2.064136in}}{\pgfqpoint{1.277288in}{2.056322in}}%
\pgfpathcurveto{\pgfqpoint{1.269475in}{2.048509in}}{\pgfqpoint{1.265084in}{2.037909in}}{\pgfqpoint{1.265084in}{2.026859in}}%
\pgfpathcurveto{\pgfqpoint{1.265084in}{2.015809in}}{\pgfqpoint{1.269475in}{2.005210in}}{\pgfqpoint{1.277288in}{1.997397in}}%
\pgfpathcurveto{\pgfqpoint{1.285102in}{1.989583in}}{\pgfqpoint{1.295701in}{1.985193in}}{\pgfqpoint{1.306751in}{1.985193in}}%
\pgfpathlineto{\pgfqpoint{1.306751in}{1.985193in}}%
\pgfpathclose%
\pgfusepath{stroke}%
\end{pgfscope}%
\begin{pgfscope}%
\pgfpathrectangle{\pgfqpoint{0.393053in}{0.375000in}}{\pgfqpoint{6.356833in}{5.175000in}}%
\pgfusepath{clip}%
\pgfsetbuttcap%
\pgfsetroundjoin%
\pgfsetlinewidth{1.003750pt}%
\definecolor{currentstroke}{rgb}{0.827451,0.827451,0.827451}%
\pgfsetstrokecolor{currentstroke}%
\pgfsetdash{}{0pt}%
\pgfpathmoveto{\pgfqpoint{1.418682in}{1.883025in}}%
\pgfpathcurveto{\pgfqpoint{1.429732in}{1.883025in}}{\pgfqpoint{1.440331in}{1.887415in}}{\pgfqpoint{1.448145in}{1.895229in}}%
\pgfpathcurveto{\pgfqpoint{1.455958in}{1.903043in}}{\pgfqpoint{1.460349in}{1.913642in}}{\pgfqpoint{1.460349in}{1.924692in}}%
\pgfpathcurveto{\pgfqpoint{1.460349in}{1.935742in}}{\pgfqpoint{1.455958in}{1.946341in}}{\pgfqpoint{1.448145in}{1.954155in}}%
\pgfpathcurveto{\pgfqpoint{1.440331in}{1.961968in}}{\pgfqpoint{1.429732in}{1.966358in}}{\pgfqpoint{1.418682in}{1.966358in}}%
\pgfpathcurveto{\pgfqpoint{1.407632in}{1.966358in}}{\pgfqpoint{1.397033in}{1.961968in}}{\pgfqpoint{1.389219in}{1.954155in}}%
\pgfpathcurveto{\pgfqpoint{1.381406in}{1.946341in}}{\pgfqpoint{1.377015in}{1.935742in}}{\pgfqpoint{1.377015in}{1.924692in}}%
\pgfpathcurveto{\pgfqpoint{1.377015in}{1.913642in}}{\pgfqpoint{1.381406in}{1.903043in}}{\pgfqpoint{1.389219in}{1.895229in}}%
\pgfpathcurveto{\pgfqpoint{1.397033in}{1.887415in}}{\pgfqpoint{1.407632in}{1.883025in}}{\pgfqpoint{1.418682in}{1.883025in}}%
\pgfpathlineto{\pgfqpoint{1.418682in}{1.883025in}}%
\pgfpathclose%
\pgfusepath{stroke}%
\end{pgfscope}%
\begin{pgfscope}%
\pgfpathrectangle{\pgfqpoint{0.393053in}{0.375000in}}{\pgfqpoint{6.356833in}{5.175000in}}%
\pgfusepath{clip}%
\pgfsetbuttcap%
\pgfsetroundjoin%
\pgfsetlinewidth{1.003750pt}%
\definecolor{currentstroke}{rgb}{0.827451,0.827451,0.827451}%
\pgfsetstrokecolor{currentstroke}%
\pgfsetdash{}{0pt}%
\pgfpathmoveto{\pgfqpoint{1.702004in}{1.599430in}}%
\pgfpathcurveto{\pgfqpoint{1.713054in}{1.599430in}}{\pgfqpoint{1.723653in}{1.603820in}}{\pgfqpoint{1.731467in}{1.611633in}}%
\pgfpathcurveto{\pgfqpoint{1.739281in}{1.619447in}}{\pgfqpoint{1.743671in}{1.630046in}}{\pgfqpoint{1.743671in}{1.641096in}}%
\pgfpathcurveto{\pgfqpoint{1.743671in}{1.652146in}}{\pgfqpoint{1.739281in}{1.662745in}}{\pgfqpoint{1.731467in}{1.670559in}}%
\pgfpathcurveto{\pgfqpoint{1.723653in}{1.678373in}}{\pgfqpoint{1.713054in}{1.682763in}}{\pgfqpoint{1.702004in}{1.682763in}}%
\pgfpathcurveto{\pgfqpoint{1.690954in}{1.682763in}}{\pgfqpoint{1.680355in}{1.678373in}}{\pgfqpoint{1.672541in}{1.670559in}}%
\pgfpathcurveto{\pgfqpoint{1.664728in}{1.662745in}}{\pgfqpoint{1.660338in}{1.652146in}}{\pgfqpoint{1.660338in}{1.641096in}}%
\pgfpathcurveto{\pgfqpoint{1.660338in}{1.630046in}}{\pgfqpoint{1.664728in}{1.619447in}}{\pgfqpoint{1.672541in}{1.611633in}}%
\pgfpathcurveto{\pgfqpoint{1.680355in}{1.603820in}}{\pgfqpoint{1.690954in}{1.599430in}}{\pgfqpoint{1.702004in}{1.599430in}}%
\pgfpathlineto{\pgfqpoint{1.702004in}{1.599430in}}%
\pgfpathclose%
\pgfusepath{stroke}%
\end{pgfscope}%
\begin{pgfscope}%
\pgfpathrectangle{\pgfqpoint{0.393053in}{0.375000in}}{\pgfqpoint{6.356833in}{5.175000in}}%
\pgfusepath{clip}%
\pgfsetbuttcap%
\pgfsetroundjoin%
\pgfsetlinewidth{1.003750pt}%
\definecolor{currentstroke}{rgb}{0.827451,0.827451,0.827451}%
\pgfsetstrokecolor{currentstroke}%
\pgfsetdash{}{0pt}%
\pgfpathmoveto{\pgfqpoint{4.927239in}{0.379634in}}%
\pgfpathcurveto{\pgfqpoint{4.938289in}{0.379634in}}{\pgfqpoint{4.948888in}{0.384025in}}{\pgfqpoint{4.956702in}{0.391838in}}%
\pgfpathcurveto{\pgfqpoint{4.964516in}{0.399652in}}{\pgfqpoint{4.968906in}{0.410251in}}{\pgfqpoint{4.968906in}{0.421301in}}%
\pgfpathcurveto{\pgfqpoint{4.968906in}{0.432351in}}{\pgfqpoint{4.964516in}{0.442950in}}{\pgfqpoint{4.956702in}{0.450764in}}%
\pgfpathcurveto{\pgfqpoint{4.948888in}{0.458578in}}{\pgfqpoint{4.938289in}{0.462968in}}{\pgfqpoint{4.927239in}{0.462968in}}%
\pgfpathcurveto{\pgfqpoint{4.916189in}{0.462968in}}{\pgfqpoint{4.905590in}{0.458578in}}{\pgfqpoint{4.897776in}{0.450764in}}%
\pgfpathcurveto{\pgfqpoint{4.889963in}{0.442950in}}{\pgfqpoint{4.885572in}{0.432351in}}{\pgfqpoint{4.885572in}{0.421301in}}%
\pgfpathcurveto{\pgfqpoint{4.885572in}{0.410251in}}{\pgfqpoint{4.889963in}{0.399652in}}{\pgfqpoint{4.897776in}{0.391838in}}%
\pgfpathcurveto{\pgfqpoint{4.905590in}{0.384025in}}{\pgfqpoint{4.916189in}{0.379634in}}{\pgfqpoint{4.927239in}{0.379634in}}%
\pgfpathlineto{\pgfqpoint{4.927239in}{0.379634in}}%
\pgfpathclose%
\pgfusepath{stroke}%
\end{pgfscope}%
\begin{pgfscope}%
\pgfpathrectangle{\pgfqpoint{0.393053in}{0.375000in}}{\pgfqpoint{6.356833in}{5.175000in}}%
\pgfusepath{clip}%
\pgfsetbuttcap%
\pgfsetroundjoin%
\pgfsetlinewidth{1.003750pt}%
\definecolor{currentstroke}{rgb}{0.827451,0.827451,0.827451}%
\pgfsetstrokecolor{currentstroke}%
\pgfsetdash{}{0pt}%
\pgfpathmoveto{\pgfqpoint{0.601770in}{3.214331in}}%
\pgfpathcurveto{\pgfqpoint{0.612820in}{3.214331in}}{\pgfqpoint{0.623419in}{3.218722in}}{\pgfqpoint{0.631233in}{3.226535in}}%
\pgfpathcurveto{\pgfqpoint{0.639046in}{3.234349in}}{\pgfqpoint{0.643437in}{3.244948in}}{\pgfqpoint{0.643437in}{3.255998in}}%
\pgfpathcurveto{\pgfqpoint{0.643437in}{3.267048in}}{\pgfqpoint{0.639046in}{3.277647in}}{\pgfqpoint{0.631233in}{3.285461in}}%
\pgfpathcurveto{\pgfqpoint{0.623419in}{3.293275in}}{\pgfqpoint{0.612820in}{3.297665in}}{\pgfqpoint{0.601770in}{3.297665in}}%
\pgfpathcurveto{\pgfqpoint{0.590720in}{3.297665in}}{\pgfqpoint{0.580121in}{3.293275in}}{\pgfqpoint{0.572307in}{3.285461in}}%
\pgfpathcurveto{\pgfqpoint{0.564494in}{3.277647in}}{\pgfqpoint{0.560103in}{3.267048in}}{\pgfqpoint{0.560103in}{3.255998in}}%
\pgfpathcurveto{\pgfqpoint{0.560103in}{3.244948in}}{\pgfqpoint{0.564494in}{3.234349in}}{\pgfqpoint{0.572307in}{3.226535in}}%
\pgfpathcurveto{\pgfqpoint{0.580121in}{3.218722in}}{\pgfqpoint{0.590720in}{3.214331in}}{\pgfqpoint{0.601770in}{3.214331in}}%
\pgfpathlineto{\pgfqpoint{0.601770in}{3.214331in}}%
\pgfpathclose%
\pgfusepath{stroke}%
\end{pgfscope}%
\begin{pgfscope}%
\pgfpathrectangle{\pgfqpoint{0.393053in}{0.375000in}}{\pgfqpoint{6.356833in}{5.175000in}}%
\pgfusepath{clip}%
\pgfsetbuttcap%
\pgfsetroundjoin%
\pgfsetlinewidth{1.003750pt}%
\definecolor{currentstroke}{rgb}{0.827451,0.827451,0.827451}%
\pgfsetstrokecolor{currentstroke}%
\pgfsetdash{}{0pt}%
\pgfpathmoveto{\pgfqpoint{3.397541in}{0.662535in}}%
\pgfpathcurveto{\pgfqpoint{3.408591in}{0.662535in}}{\pgfqpoint{3.419190in}{0.666925in}}{\pgfqpoint{3.427004in}{0.674738in}}%
\pgfpathcurveto{\pgfqpoint{3.434817in}{0.682552in}}{\pgfqpoint{3.439208in}{0.693151in}}{\pgfqpoint{3.439208in}{0.704201in}}%
\pgfpathcurveto{\pgfqpoint{3.439208in}{0.715251in}}{\pgfqpoint{3.434817in}{0.725850in}}{\pgfqpoint{3.427004in}{0.733664in}}%
\pgfpathcurveto{\pgfqpoint{3.419190in}{0.741478in}}{\pgfqpoint{3.408591in}{0.745868in}}{\pgfqpoint{3.397541in}{0.745868in}}%
\pgfpathcurveto{\pgfqpoint{3.386491in}{0.745868in}}{\pgfqpoint{3.375892in}{0.741478in}}{\pgfqpoint{3.368078in}{0.733664in}}%
\pgfpathcurveto{\pgfqpoint{3.360265in}{0.725850in}}{\pgfqpoint{3.355874in}{0.715251in}}{\pgfqpoint{3.355874in}{0.704201in}}%
\pgfpathcurveto{\pgfqpoint{3.355874in}{0.693151in}}{\pgfqpoint{3.360265in}{0.682552in}}{\pgfqpoint{3.368078in}{0.674738in}}%
\pgfpathcurveto{\pgfqpoint{3.375892in}{0.666925in}}{\pgfqpoint{3.386491in}{0.662535in}}{\pgfqpoint{3.397541in}{0.662535in}}%
\pgfpathlineto{\pgfqpoint{3.397541in}{0.662535in}}%
\pgfpathclose%
\pgfusepath{stroke}%
\end{pgfscope}%
\begin{pgfscope}%
\pgfpathrectangle{\pgfqpoint{0.393053in}{0.375000in}}{\pgfqpoint{6.356833in}{5.175000in}}%
\pgfusepath{clip}%
\pgfsetbuttcap%
\pgfsetroundjoin%
\pgfsetlinewidth{1.003750pt}%
\definecolor{currentstroke}{rgb}{0.827451,0.827451,0.827451}%
\pgfsetstrokecolor{currentstroke}%
\pgfsetdash{}{0pt}%
\pgfpathmoveto{\pgfqpoint{1.951105in}{1.400378in}}%
\pgfpathcurveto{\pgfqpoint{1.962155in}{1.400378in}}{\pgfqpoint{1.972754in}{1.404768in}}{\pgfqpoint{1.980568in}{1.412582in}}%
\pgfpathcurveto{\pgfqpoint{1.988382in}{1.420396in}}{\pgfqpoint{1.992772in}{1.430995in}}{\pgfqpoint{1.992772in}{1.442045in}}%
\pgfpathcurveto{\pgfqpoint{1.992772in}{1.453095in}}{\pgfqpoint{1.988382in}{1.463694in}}{\pgfqpoint{1.980568in}{1.471507in}}%
\pgfpathcurveto{\pgfqpoint{1.972754in}{1.479321in}}{\pgfqpoint{1.962155in}{1.483711in}}{\pgfqpoint{1.951105in}{1.483711in}}%
\pgfpathcurveto{\pgfqpoint{1.940055in}{1.483711in}}{\pgfqpoint{1.929456in}{1.479321in}}{\pgfqpoint{1.921642in}{1.471507in}}%
\pgfpathcurveto{\pgfqpoint{1.913829in}{1.463694in}}{\pgfqpoint{1.909439in}{1.453095in}}{\pgfqpoint{1.909439in}{1.442045in}}%
\pgfpathcurveto{\pgfqpoint{1.909439in}{1.430995in}}{\pgfqpoint{1.913829in}{1.420396in}}{\pgfqpoint{1.921642in}{1.412582in}}%
\pgfpathcurveto{\pgfqpoint{1.929456in}{1.404768in}}{\pgfqpoint{1.940055in}{1.400378in}}{\pgfqpoint{1.951105in}{1.400378in}}%
\pgfpathlineto{\pgfqpoint{1.951105in}{1.400378in}}%
\pgfpathclose%
\pgfusepath{stroke}%
\end{pgfscope}%
\begin{pgfscope}%
\pgfpathrectangle{\pgfqpoint{0.393053in}{0.375000in}}{\pgfqpoint{6.356833in}{5.175000in}}%
\pgfusepath{clip}%
\pgfsetbuttcap%
\pgfsetroundjoin%
\pgfsetlinewidth{1.003750pt}%
\definecolor{currentstroke}{rgb}{0.827451,0.827451,0.827451}%
\pgfsetstrokecolor{currentstroke}%
\pgfsetdash{}{0pt}%
\pgfpathmoveto{\pgfqpoint{1.224501in}{2.087879in}}%
\pgfpathcurveto{\pgfqpoint{1.235551in}{2.087879in}}{\pgfqpoint{1.246150in}{2.092269in}}{\pgfqpoint{1.253963in}{2.100083in}}%
\pgfpathcurveto{\pgfqpoint{1.261777in}{2.107896in}}{\pgfqpoint{1.266167in}{2.118495in}}{\pgfqpoint{1.266167in}{2.129546in}}%
\pgfpathcurveto{\pgfqpoint{1.266167in}{2.140596in}}{\pgfqpoint{1.261777in}{2.151195in}}{\pgfqpoint{1.253963in}{2.159008in}}%
\pgfpathcurveto{\pgfqpoint{1.246150in}{2.166822in}}{\pgfqpoint{1.235551in}{2.171212in}}{\pgfqpoint{1.224501in}{2.171212in}}%
\pgfpathcurveto{\pgfqpoint{1.213451in}{2.171212in}}{\pgfqpoint{1.202852in}{2.166822in}}{\pgfqpoint{1.195038in}{2.159008in}}%
\pgfpathcurveto{\pgfqpoint{1.187224in}{2.151195in}}{\pgfqpoint{1.182834in}{2.140596in}}{\pgfqpoint{1.182834in}{2.129546in}}%
\pgfpathcurveto{\pgfqpoint{1.182834in}{2.118495in}}{\pgfqpoint{1.187224in}{2.107896in}}{\pgfqpoint{1.195038in}{2.100083in}}%
\pgfpathcurveto{\pgfqpoint{1.202852in}{2.092269in}}{\pgfqpoint{1.213451in}{2.087879in}}{\pgfqpoint{1.224501in}{2.087879in}}%
\pgfpathlineto{\pgfqpoint{1.224501in}{2.087879in}}%
\pgfpathclose%
\pgfusepath{stroke}%
\end{pgfscope}%
\begin{pgfscope}%
\pgfpathrectangle{\pgfqpoint{0.393053in}{0.375000in}}{\pgfqpoint{6.356833in}{5.175000in}}%
\pgfusepath{clip}%
\pgfsetbuttcap%
\pgfsetroundjoin%
\pgfsetlinewidth{1.003750pt}%
\definecolor{currentstroke}{rgb}{0.827451,0.827451,0.827451}%
\pgfsetstrokecolor{currentstroke}%
\pgfsetdash{}{0pt}%
\pgfpathmoveto{\pgfqpoint{2.870546in}{0.897562in}}%
\pgfpathcurveto{\pgfqpoint{2.881596in}{0.897562in}}{\pgfqpoint{2.892195in}{0.901953in}}{\pgfqpoint{2.900009in}{0.909766in}}%
\pgfpathcurveto{\pgfqpoint{2.907823in}{0.917580in}}{\pgfqpoint{2.912213in}{0.928179in}}{\pgfqpoint{2.912213in}{0.939229in}}%
\pgfpathcurveto{\pgfqpoint{2.912213in}{0.950279in}}{\pgfqpoint{2.907823in}{0.960878in}}{\pgfqpoint{2.900009in}{0.968692in}}%
\pgfpathcurveto{\pgfqpoint{2.892195in}{0.976505in}}{\pgfqpoint{2.881596in}{0.980896in}}{\pgfqpoint{2.870546in}{0.980896in}}%
\pgfpathcurveto{\pgfqpoint{2.859496in}{0.980896in}}{\pgfqpoint{2.848897in}{0.976505in}}{\pgfqpoint{2.841083in}{0.968692in}}%
\pgfpathcurveto{\pgfqpoint{2.833270in}{0.960878in}}{\pgfqpoint{2.828880in}{0.950279in}}{\pgfqpoint{2.828880in}{0.939229in}}%
\pgfpathcurveto{\pgfqpoint{2.828880in}{0.928179in}}{\pgfqpoint{2.833270in}{0.917580in}}{\pgfqpoint{2.841083in}{0.909766in}}%
\pgfpathcurveto{\pgfqpoint{2.848897in}{0.901953in}}{\pgfqpoint{2.859496in}{0.897562in}}{\pgfqpoint{2.870546in}{0.897562in}}%
\pgfpathlineto{\pgfqpoint{2.870546in}{0.897562in}}%
\pgfpathclose%
\pgfusepath{stroke}%
\end{pgfscope}%
\begin{pgfscope}%
\pgfpathrectangle{\pgfqpoint{0.393053in}{0.375000in}}{\pgfqpoint{6.356833in}{5.175000in}}%
\pgfusepath{clip}%
\pgfsetbuttcap%
\pgfsetroundjoin%
\pgfsetlinewidth{1.003750pt}%
\definecolor{currentstroke}{rgb}{0.827451,0.827451,0.827451}%
\pgfsetstrokecolor{currentstroke}%
\pgfsetdash{}{0pt}%
\pgfpathmoveto{\pgfqpoint{0.483847in}{3.664437in}}%
\pgfpathcurveto{\pgfqpoint{0.494897in}{3.664437in}}{\pgfqpoint{0.505496in}{3.668827in}}{\pgfqpoint{0.513309in}{3.676640in}}%
\pgfpathcurveto{\pgfqpoint{0.521123in}{3.684454in}}{\pgfqpoint{0.525513in}{3.695053in}}{\pgfqpoint{0.525513in}{3.706103in}}%
\pgfpathcurveto{\pgfqpoint{0.525513in}{3.717153in}}{\pgfqpoint{0.521123in}{3.727752in}}{\pgfqpoint{0.513309in}{3.735566in}}%
\pgfpathcurveto{\pgfqpoint{0.505496in}{3.743380in}}{\pgfqpoint{0.494897in}{3.747770in}}{\pgfqpoint{0.483847in}{3.747770in}}%
\pgfpathcurveto{\pgfqpoint{0.472797in}{3.747770in}}{\pgfqpoint{0.462198in}{3.743380in}}{\pgfqpoint{0.454384in}{3.735566in}}%
\pgfpathcurveto{\pgfqpoint{0.446570in}{3.727752in}}{\pgfqpoint{0.442180in}{3.717153in}}{\pgfqpoint{0.442180in}{3.706103in}}%
\pgfpathcurveto{\pgfqpoint{0.442180in}{3.695053in}}{\pgfqpoint{0.446570in}{3.684454in}}{\pgfqpoint{0.454384in}{3.676640in}}%
\pgfpathcurveto{\pgfqpoint{0.462198in}{3.668827in}}{\pgfqpoint{0.472797in}{3.664437in}}{\pgfqpoint{0.483847in}{3.664437in}}%
\pgfpathlineto{\pgfqpoint{0.483847in}{3.664437in}}%
\pgfpathclose%
\pgfusepath{stroke}%
\end{pgfscope}%
\begin{pgfscope}%
\pgfpathrectangle{\pgfqpoint{0.393053in}{0.375000in}}{\pgfqpoint{6.356833in}{5.175000in}}%
\pgfusepath{clip}%
\pgfsetbuttcap%
\pgfsetroundjoin%
\pgfsetlinewidth{1.003750pt}%
\definecolor{currentstroke}{rgb}{0.827451,0.827451,0.827451}%
\pgfsetstrokecolor{currentstroke}%
\pgfsetdash{}{0pt}%
\pgfpathmoveto{\pgfqpoint{3.142172in}{0.759581in}}%
\pgfpathcurveto{\pgfqpoint{3.153222in}{0.759581in}}{\pgfqpoint{3.163821in}{0.763971in}}{\pgfqpoint{3.171635in}{0.771785in}}%
\pgfpathcurveto{\pgfqpoint{3.179449in}{0.779598in}}{\pgfqpoint{3.183839in}{0.790197in}}{\pgfqpoint{3.183839in}{0.801247in}}%
\pgfpathcurveto{\pgfqpoint{3.183839in}{0.812297in}}{\pgfqpoint{3.179449in}{0.822896in}}{\pgfqpoint{3.171635in}{0.830710in}}%
\pgfpathcurveto{\pgfqpoint{3.163821in}{0.838524in}}{\pgfqpoint{3.153222in}{0.842914in}}{\pgfqpoint{3.142172in}{0.842914in}}%
\pgfpathcurveto{\pgfqpoint{3.131122in}{0.842914in}}{\pgfqpoint{3.120523in}{0.838524in}}{\pgfqpoint{3.112710in}{0.830710in}}%
\pgfpathcurveto{\pgfqpoint{3.104896in}{0.822896in}}{\pgfqpoint{3.100506in}{0.812297in}}{\pgfqpoint{3.100506in}{0.801247in}}%
\pgfpathcurveto{\pgfqpoint{3.100506in}{0.790197in}}{\pgfqpoint{3.104896in}{0.779598in}}{\pgfqpoint{3.112710in}{0.771785in}}%
\pgfpathcurveto{\pgfqpoint{3.120523in}{0.763971in}}{\pgfqpoint{3.131122in}{0.759581in}}{\pgfqpoint{3.142172in}{0.759581in}}%
\pgfpathlineto{\pgfqpoint{3.142172in}{0.759581in}}%
\pgfpathclose%
\pgfusepath{stroke}%
\end{pgfscope}%
\begin{pgfscope}%
\pgfpathrectangle{\pgfqpoint{0.393053in}{0.375000in}}{\pgfqpoint{6.356833in}{5.175000in}}%
\pgfusepath{clip}%
\pgfsetbuttcap%
\pgfsetroundjoin%
\pgfsetlinewidth{1.003750pt}%
\definecolor{currentstroke}{rgb}{0.827451,0.827451,0.827451}%
\pgfsetstrokecolor{currentstroke}%
\pgfsetdash{}{0pt}%
\pgfpathmoveto{\pgfqpoint{1.493790in}{1.813114in}}%
\pgfpathcurveto{\pgfqpoint{1.504840in}{1.813114in}}{\pgfqpoint{1.515439in}{1.817505in}}{\pgfqpoint{1.523252in}{1.825318in}}%
\pgfpathcurveto{\pgfqpoint{1.531066in}{1.833132in}}{\pgfqpoint{1.535456in}{1.843731in}}{\pgfqpoint{1.535456in}{1.854781in}}%
\pgfpathcurveto{\pgfqpoint{1.535456in}{1.865831in}}{\pgfqpoint{1.531066in}{1.876430in}}{\pgfqpoint{1.523252in}{1.884244in}}%
\pgfpathcurveto{\pgfqpoint{1.515439in}{1.892058in}}{\pgfqpoint{1.504840in}{1.896448in}}{\pgfqpoint{1.493790in}{1.896448in}}%
\pgfpathcurveto{\pgfqpoint{1.482740in}{1.896448in}}{\pgfqpoint{1.472140in}{1.892058in}}{\pgfqpoint{1.464327in}{1.884244in}}%
\pgfpathcurveto{\pgfqpoint{1.456513in}{1.876430in}}{\pgfqpoint{1.452123in}{1.865831in}}{\pgfqpoint{1.452123in}{1.854781in}}%
\pgfpathcurveto{\pgfqpoint{1.452123in}{1.843731in}}{\pgfqpoint{1.456513in}{1.833132in}}{\pgfqpoint{1.464327in}{1.825318in}}%
\pgfpathcurveto{\pgfqpoint{1.472140in}{1.817505in}}{\pgfqpoint{1.482740in}{1.813114in}}{\pgfqpoint{1.493790in}{1.813114in}}%
\pgfpathlineto{\pgfqpoint{1.493790in}{1.813114in}}%
\pgfpathclose%
\pgfusepath{stroke}%
\end{pgfscope}%
\begin{pgfscope}%
\pgfpathrectangle{\pgfqpoint{0.393053in}{0.375000in}}{\pgfqpoint{6.356833in}{5.175000in}}%
\pgfusepath{clip}%
\pgfsetbuttcap%
\pgfsetroundjoin%
\pgfsetlinewidth{1.003750pt}%
\definecolor{currentstroke}{rgb}{0.827451,0.827451,0.827451}%
\pgfsetstrokecolor{currentstroke}%
\pgfsetdash{}{0pt}%
\pgfpathmoveto{\pgfqpoint{4.283632in}{0.443541in}}%
\pgfpathcurveto{\pgfqpoint{4.294682in}{0.443541in}}{\pgfqpoint{4.305281in}{0.447931in}}{\pgfqpoint{4.313095in}{0.455745in}}%
\pgfpathcurveto{\pgfqpoint{4.320908in}{0.463559in}}{\pgfqpoint{4.325299in}{0.474158in}}{\pgfqpoint{4.325299in}{0.485208in}}%
\pgfpathcurveto{\pgfqpoint{4.325299in}{0.496258in}}{\pgfqpoint{4.320908in}{0.506857in}}{\pgfqpoint{4.313095in}{0.514670in}}%
\pgfpathcurveto{\pgfqpoint{4.305281in}{0.522484in}}{\pgfqpoint{4.294682in}{0.526874in}}{\pgfqpoint{4.283632in}{0.526874in}}%
\pgfpathcurveto{\pgfqpoint{4.272582in}{0.526874in}}{\pgfqpoint{4.261983in}{0.522484in}}{\pgfqpoint{4.254169in}{0.514670in}}%
\pgfpathcurveto{\pgfqpoint{4.246356in}{0.506857in}}{\pgfqpoint{4.241965in}{0.496258in}}{\pgfqpoint{4.241965in}{0.485208in}}%
\pgfpathcurveto{\pgfqpoint{4.241965in}{0.474158in}}{\pgfqpoint{4.246356in}{0.463559in}}{\pgfqpoint{4.254169in}{0.455745in}}%
\pgfpathcurveto{\pgfqpoint{4.261983in}{0.447931in}}{\pgfqpoint{4.272582in}{0.443541in}}{\pgfqpoint{4.283632in}{0.443541in}}%
\pgfpathlineto{\pgfqpoint{4.283632in}{0.443541in}}%
\pgfpathclose%
\pgfusepath{stroke}%
\end{pgfscope}%
\begin{pgfscope}%
\pgfpathrectangle{\pgfqpoint{0.393053in}{0.375000in}}{\pgfqpoint{6.356833in}{5.175000in}}%
\pgfusepath{clip}%
\pgfsetbuttcap%
\pgfsetroundjoin%
\pgfsetlinewidth{1.003750pt}%
\definecolor{currentstroke}{rgb}{0.827451,0.827451,0.827451}%
\pgfsetstrokecolor{currentstroke}%
\pgfsetdash{}{0pt}%
\pgfpathmoveto{\pgfqpoint{1.457810in}{1.831739in}}%
\pgfpathcurveto{\pgfqpoint{1.468860in}{1.831739in}}{\pgfqpoint{1.479459in}{1.836129in}}{\pgfqpoint{1.487273in}{1.843943in}}%
\pgfpathcurveto{\pgfqpoint{1.495086in}{1.851756in}}{\pgfqpoint{1.499477in}{1.862355in}}{\pgfqpoint{1.499477in}{1.873406in}}%
\pgfpathcurveto{\pgfqpoint{1.499477in}{1.884456in}}{\pgfqpoint{1.495086in}{1.895055in}}{\pgfqpoint{1.487273in}{1.902868in}}%
\pgfpathcurveto{\pgfqpoint{1.479459in}{1.910682in}}{\pgfqpoint{1.468860in}{1.915072in}}{\pgfqpoint{1.457810in}{1.915072in}}%
\pgfpathcurveto{\pgfqpoint{1.446760in}{1.915072in}}{\pgfqpoint{1.436161in}{1.910682in}}{\pgfqpoint{1.428347in}{1.902868in}}%
\pgfpathcurveto{\pgfqpoint{1.420533in}{1.895055in}}{\pgfqpoint{1.416143in}{1.884456in}}{\pgfqpoint{1.416143in}{1.873406in}}%
\pgfpathcurveto{\pgfqpoint{1.416143in}{1.862355in}}{\pgfqpoint{1.420533in}{1.851756in}}{\pgfqpoint{1.428347in}{1.843943in}}%
\pgfpathcurveto{\pgfqpoint{1.436161in}{1.836129in}}{\pgfqpoint{1.446760in}{1.831739in}}{\pgfqpoint{1.457810in}{1.831739in}}%
\pgfpathlineto{\pgfqpoint{1.457810in}{1.831739in}}%
\pgfpathclose%
\pgfusepath{stroke}%
\end{pgfscope}%
\begin{pgfscope}%
\pgfpathrectangle{\pgfqpoint{0.393053in}{0.375000in}}{\pgfqpoint{6.356833in}{5.175000in}}%
\pgfusepath{clip}%
\pgfsetbuttcap%
\pgfsetroundjoin%
\pgfsetlinewidth{1.003750pt}%
\definecolor{currentstroke}{rgb}{0.827451,0.827451,0.827451}%
\pgfsetstrokecolor{currentstroke}%
\pgfsetdash{}{0pt}%
\pgfpathmoveto{\pgfqpoint{4.176160in}{0.456257in}}%
\pgfpathcurveto{\pgfqpoint{4.187210in}{0.456257in}}{\pgfqpoint{4.197809in}{0.460647in}}{\pgfqpoint{4.205623in}{0.468461in}}%
\pgfpathcurveto{\pgfqpoint{4.213436in}{0.476274in}}{\pgfqpoint{4.217827in}{0.486873in}}{\pgfqpoint{4.217827in}{0.497923in}}%
\pgfpathcurveto{\pgfqpoint{4.217827in}{0.508973in}}{\pgfqpoint{4.213436in}{0.519572in}}{\pgfqpoint{4.205623in}{0.527386in}}%
\pgfpathcurveto{\pgfqpoint{4.197809in}{0.535200in}}{\pgfqpoint{4.187210in}{0.539590in}}{\pgfqpoint{4.176160in}{0.539590in}}%
\pgfpathcurveto{\pgfqpoint{4.165110in}{0.539590in}}{\pgfqpoint{4.154511in}{0.535200in}}{\pgfqpoint{4.146697in}{0.527386in}}%
\pgfpathcurveto{\pgfqpoint{4.138883in}{0.519572in}}{\pgfqpoint{4.134493in}{0.508973in}}{\pgfqpoint{4.134493in}{0.497923in}}%
\pgfpathcurveto{\pgfqpoint{4.134493in}{0.486873in}}{\pgfqpoint{4.138883in}{0.476274in}}{\pgfqpoint{4.146697in}{0.468461in}}%
\pgfpathcurveto{\pgfqpoint{4.154511in}{0.460647in}}{\pgfqpoint{4.165110in}{0.456257in}}{\pgfqpoint{4.176160in}{0.456257in}}%
\pgfpathlineto{\pgfqpoint{4.176160in}{0.456257in}}%
\pgfpathclose%
\pgfusepath{stroke}%
\end{pgfscope}%
\begin{pgfscope}%
\pgfpathrectangle{\pgfqpoint{0.393053in}{0.375000in}}{\pgfqpoint{6.356833in}{5.175000in}}%
\pgfusepath{clip}%
\pgfsetbuttcap%
\pgfsetroundjoin%
\pgfsetlinewidth{1.003750pt}%
\definecolor{currentstroke}{rgb}{0.827451,0.827451,0.827451}%
\pgfsetstrokecolor{currentstroke}%
\pgfsetdash{}{0pt}%
\pgfpathmoveto{\pgfqpoint{2.036490in}{1.340063in}}%
\pgfpathcurveto{\pgfqpoint{2.047540in}{1.340063in}}{\pgfqpoint{2.058139in}{1.344454in}}{\pgfqpoint{2.065953in}{1.352267in}}%
\pgfpathcurveto{\pgfqpoint{2.073767in}{1.360081in}}{\pgfqpoint{2.078157in}{1.370680in}}{\pgfqpoint{2.078157in}{1.381730in}}%
\pgfpathcurveto{\pgfqpoint{2.078157in}{1.392780in}}{\pgfqpoint{2.073767in}{1.403379in}}{\pgfqpoint{2.065953in}{1.411193in}}%
\pgfpathcurveto{\pgfqpoint{2.058139in}{1.419007in}}{\pgfqpoint{2.047540in}{1.423397in}}{\pgfqpoint{2.036490in}{1.423397in}}%
\pgfpathcurveto{\pgfqpoint{2.025440in}{1.423397in}}{\pgfqpoint{2.014841in}{1.419007in}}{\pgfqpoint{2.007027in}{1.411193in}}%
\pgfpathcurveto{\pgfqpoint{1.999214in}{1.403379in}}{\pgfqpoint{1.994824in}{1.392780in}}{\pgfqpoint{1.994824in}{1.381730in}}%
\pgfpathcurveto{\pgfqpoint{1.994824in}{1.370680in}}{\pgfqpoint{1.999214in}{1.360081in}}{\pgfqpoint{2.007027in}{1.352267in}}%
\pgfpathcurveto{\pgfqpoint{2.014841in}{1.344454in}}{\pgfqpoint{2.025440in}{1.340063in}}{\pgfqpoint{2.036490in}{1.340063in}}%
\pgfpathlineto{\pgfqpoint{2.036490in}{1.340063in}}%
\pgfpathclose%
\pgfusepath{stroke}%
\end{pgfscope}%
\begin{pgfscope}%
\pgfpathrectangle{\pgfqpoint{0.393053in}{0.375000in}}{\pgfqpoint{6.356833in}{5.175000in}}%
\pgfusepath{clip}%
\pgfsetbuttcap%
\pgfsetroundjoin%
\pgfsetlinewidth{1.003750pt}%
\definecolor{currentstroke}{rgb}{0.827451,0.827451,0.827451}%
\pgfsetstrokecolor{currentstroke}%
\pgfsetdash{}{0pt}%
\pgfpathmoveto{\pgfqpoint{2.525895in}{1.037541in}}%
\pgfpathcurveto{\pgfqpoint{2.536945in}{1.037541in}}{\pgfqpoint{2.547544in}{1.041932in}}{\pgfqpoint{2.555358in}{1.049745in}}%
\pgfpathcurveto{\pgfqpoint{2.563172in}{1.057559in}}{\pgfqpoint{2.567562in}{1.068158in}}{\pgfqpoint{2.567562in}{1.079208in}}%
\pgfpathcurveto{\pgfqpoint{2.567562in}{1.090258in}}{\pgfqpoint{2.563172in}{1.100857in}}{\pgfqpoint{2.555358in}{1.108671in}}%
\pgfpathcurveto{\pgfqpoint{2.547544in}{1.116484in}}{\pgfqpoint{2.536945in}{1.120875in}}{\pgfqpoint{2.525895in}{1.120875in}}%
\pgfpathcurveto{\pgfqpoint{2.514845in}{1.120875in}}{\pgfqpoint{2.504246in}{1.116484in}}{\pgfqpoint{2.496432in}{1.108671in}}%
\pgfpathcurveto{\pgfqpoint{2.488619in}{1.100857in}}{\pgfqpoint{2.484228in}{1.090258in}}{\pgfqpoint{2.484228in}{1.079208in}}%
\pgfpathcurveto{\pgfqpoint{2.484228in}{1.068158in}}{\pgfqpoint{2.488619in}{1.057559in}}{\pgfqpoint{2.496432in}{1.049745in}}%
\pgfpathcurveto{\pgfqpoint{2.504246in}{1.041932in}}{\pgfqpoint{2.514845in}{1.037541in}}{\pgfqpoint{2.525895in}{1.037541in}}%
\pgfpathlineto{\pgfqpoint{2.525895in}{1.037541in}}%
\pgfpathclose%
\pgfusepath{stroke}%
\end{pgfscope}%
\begin{pgfscope}%
\pgfpathrectangle{\pgfqpoint{0.393053in}{0.375000in}}{\pgfqpoint{6.356833in}{5.175000in}}%
\pgfusepath{clip}%
\pgfsetbuttcap%
\pgfsetroundjoin%
\pgfsetlinewidth{1.003750pt}%
\definecolor{currentstroke}{rgb}{0.827451,0.827451,0.827451}%
\pgfsetstrokecolor{currentstroke}%
\pgfsetdash{}{0pt}%
\pgfpathmoveto{\pgfqpoint{0.525326in}{3.470591in}}%
\pgfpathcurveto{\pgfqpoint{0.536376in}{3.470591in}}{\pgfqpoint{0.546975in}{3.474982in}}{\pgfqpoint{0.554788in}{3.482795in}}%
\pgfpathcurveto{\pgfqpoint{0.562602in}{3.490609in}}{\pgfqpoint{0.566992in}{3.501208in}}{\pgfqpoint{0.566992in}{3.512258in}}%
\pgfpathcurveto{\pgfqpoint{0.566992in}{3.523308in}}{\pgfqpoint{0.562602in}{3.533907in}}{\pgfqpoint{0.554788in}{3.541721in}}%
\pgfpathcurveto{\pgfqpoint{0.546975in}{3.549534in}}{\pgfqpoint{0.536376in}{3.553925in}}{\pgfqpoint{0.525326in}{3.553925in}}%
\pgfpathcurveto{\pgfqpoint{0.514275in}{3.553925in}}{\pgfqpoint{0.503676in}{3.549534in}}{\pgfqpoint{0.495863in}{3.541721in}}%
\pgfpathcurveto{\pgfqpoint{0.488049in}{3.533907in}}{\pgfqpoint{0.483659in}{3.523308in}}{\pgfqpoint{0.483659in}{3.512258in}}%
\pgfpathcurveto{\pgfqpoint{0.483659in}{3.501208in}}{\pgfqpoint{0.488049in}{3.490609in}}{\pgfqpoint{0.495863in}{3.482795in}}%
\pgfpathcurveto{\pgfqpoint{0.503676in}{3.474982in}}{\pgfqpoint{0.514275in}{3.470591in}}{\pgfqpoint{0.525326in}{3.470591in}}%
\pgfpathlineto{\pgfqpoint{0.525326in}{3.470591in}}%
\pgfpathclose%
\pgfusepath{stroke}%
\end{pgfscope}%
\begin{pgfscope}%
\pgfpathrectangle{\pgfqpoint{0.393053in}{0.375000in}}{\pgfqpoint{6.356833in}{5.175000in}}%
\pgfusepath{clip}%
\pgfsetbuttcap%
\pgfsetroundjoin%
\pgfsetlinewidth{1.003750pt}%
\definecolor{currentstroke}{rgb}{0.827451,0.827451,0.827451}%
\pgfsetstrokecolor{currentstroke}%
\pgfsetdash{}{0pt}%
\pgfpathmoveto{\pgfqpoint{0.612319in}{3.180269in}}%
\pgfpathcurveto{\pgfqpoint{0.623369in}{3.180269in}}{\pgfqpoint{0.633968in}{3.184660in}}{\pgfqpoint{0.641781in}{3.192473in}}%
\pgfpathcurveto{\pgfqpoint{0.649595in}{3.200287in}}{\pgfqpoint{0.653985in}{3.210886in}}{\pgfqpoint{0.653985in}{3.221936in}}%
\pgfpathcurveto{\pgfqpoint{0.653985in}{3.232986in}}{\pgfqpoint{0.649595in}{3.243585in}}{\pgfqpoint{0.641781in}{3.251399in}}%
\pgfpathcurveto{\pgfqpoint{0.633968in}{3.259212in}}{\pgfqpoint{0.623369in}{3.263603in}}{\pgfqpoint{0.612319in}{3.263603in}}%
\pgfpathcurveto{\pgfqpoint{0.601269in}{3.263603in}}{\pgfqpoint{0.590669in}{3.259212in}}{\pgfqpoint{0.582856in}{3.251399in}}%
\pgfpathcurveto{\pgfqpoint{0.575042in}{3.243585in}}{\pgfqpoint{0.570652in}{3.232986in}}{\pgfqpoint{0.570652in}{3.221936in}}%
\pgfpathcurveto{\pgfqpoint{0.570652in}{3.210886in}}{\pgfqpoint{0.575042in}{3.200287in}}{\pgfqpoint{0.582856in}{3.192473in}}%
\pgfpathcurveto{\pgfqpoint{0.590669in}{3.184660in}}{\pgfqpoint{0.601269in}{3.180269in}}{\pgfqpoint{0.612319in}{3.180269in}}%
\pgfpathlineto{\pgfqpoint{0.612319in}{3.180269in}}%
\pgfpathclose%
\pgfusepath{stroke}%
\end{pgfscope}%
\begin{pgfscope}%
\pgfpathrectangle{\pgfqpoint{0.393053in}{0.375000in}}{\pgfqpoint{6.356833in}{5.175000in}}%
\pgfusepath{clip}%
\pgfsetbuttcap%
\pgfsetroundjoin%
\pgfsetlinewidth{1.003750pt}%
\definecolor{currentstroke}{rgb}{0.827451,0.827451,0.827451}%
\pgfsetstrokecolor{currentstroke}%
\pgfsetdash{}{0pt}%
\pgfpathmoveto{\pgfqpoint{3.097658in}{0.775315in}}%
\pgfpathcurveto{\pgfqpoint{3.108708in}{0.775315in}}{\pgfqpoint{3.119307in}{0.779705in}}{\pgfqpoint{3.127121in}{0.787519in}}%
\pgfpathcurveto{\pgfqpoint{3.134935in}{0.795332in}}{\pgfqpoint{3.139325in}{0.805931in}}{\pgfqpoint{3.139325in}{0.816981in}}%
\pgfpathcurveto{\pgfqpoint{3.139325in}{0.828032in}}{\pgfqpoint{3.134935in}{0.838631in}}{\pgfqpoint{3.127121in}{0.846444in}}%
\pgfpathcurveto{\pgfqpoint{3.119307in}{0.854258in}}{\pgfqpoint{3.108708in}{0.858648in}}{\pgfqpoint{3.097658in}{0.858648in}}%
\pgfpathcurveto{\pgfqpoint{3.086608in}{0.858648in}}{\pgfqpoint{3.076009in}{0.854258in}}{\pgfqpoint{3.068195in}{0.846444in}}%
\pgfpathcurveto{\pgfqpoint{3.060382in}{0.838631in}}{\pgfqpoint{3.055992in}{0.828032in}}{\pgfqpoint{3.055992in}{0.816981in}}%
\pgfpathcurveto{\pgfqpoint{3.055992in}{0.805931in}}{\pgfqpoint{3.060382in}{0.795332in}}{\pgfqpoint{3.068195in}{0.787519in}}%
\pgfpathcurveto{\pgfqpoint{3.076009in}{0.779705in}}{\pgfqpoint{3.086608in}{0.775315in}}{\pgfqpoint{3.097658in}{0.775315in}}%
\pgfpathlineto{\pgfqpoint{3.097658in}{0.775315in}}%
\pgfpathclose%
\pgfusepath{stroke}%
\end{pgfscope}%
\begin{pgfscope}%
\pgfpathrectangle{\pgfqpoint{0.393053in}{0.375000in}}{\pgfqpoint{6.356833in}{5.175000in}}%
\pgfusepath{clip}%
\pgfsetbuttcap%
\pgfsetroundjoin%
\pgfsetlinewidth{1.003750pt}%
\definecolor{currentstroke}{rgb}{0.827451,0.827451,0.827451}%
\pgfsetstrokecolor{currentstroke}%
\pgfsetdash{}{0pt}%
\pgfpathmoveto{\pgfqpoint{2.477308in}{1.087549in}}%
\pgfpathcurveto{\pgfqpoint{2.488358in}{1.087549in}}{\pgfqpoint{2.498957in}{1.091940in}}{\pgfqpoint{2.506771in}{1.099753in}}%
\pgfpathcurveto{\pgfqpoint{2.514584in}{1.107567in}}{\pgfqpoint{2.518974in}{1.118166in}}{\pgfqpoint{2.518974in}{1.129216in}}%
\pgfpathcurveto{\pgfqpoint{2.518974in}{1.140266in}}{\pgfqpoint{2.514584in}{1.150865in}}{\pgfqpoint{2.506771in}{1.158679in}}%
\pgfpathcurveto{\pgfqpoint{2.498957in}{1.166492in}}{\pgfqpoint{2.488358in}{1.170883in}}{\pgfqpoint{2.477308in}{1.170883in}}%
\pgfpathcurveto{\pgfqpoint{2.466258in}{1.170883in}}{\pgfqpoint{2.455659in}{1.166492in}}{\pgfqpoint{2.447845in}{1.158679in}}%
\pgfpathcurveto{\pgfqpoint{2.440031in}{1.150865in}}{\pgfqpoint{2.435641in}{1.140266in}}{\pgfqpoint{2.435641in}{1.129216in}}%
\pgfpathcurveto{\pgfqpoint{2.435641in}{1.118166in}}{\pgfqpoint{2.440031in}{1.107567in}}{\pgfqpoint{2.447845in}{1.099753in}}%
\pgfpathcurveto{\pgfqpoint{2.455659in}{1.091940in}}{\pgfqpoint{2.466258in}{1.087549in}}{\pgfqpoint{2.477308in}{1.087549in}}%
\pgfpathlineto{\pgfqpoint{2.477308in}{1.087549in}}%
\pgfpathclose%
\pgfusepath{stroke}%
\end{pgfscope}%
\begin{pgfscope}%
\pgfpathrectangle{\pgfqpoint{0.393053in}{0.375000in}}{\pgfqpoint{6.356833in}{5.175000in}}%
\pgfusepath{clip}%
\pgfsetbuttcap%
\pgfsetroundjoin%
\pgfsetlinewidth{1.003750pt}%
\definecolor{currentstroke}{rgb}{0.827451,0.827451,0.827451}%
\pgfsetstrokecolor{currentstroke}%
\pgfsetdash{}{0pt}%
\pgfpathmoveto{\pgfqpoint{2.287965in}{1.196822in}}%
\pgfpathcurveto{\pgfqpoint{2.299015in}{1.196822in}}{\pgfqpoint{2.309614in}{1.201212in}}{\pgfqpoint{2.317427in}{1.209026in}}%
\pgfpathcurveto{\pgfqpoint{2.325241in}{1.216839in}}{\pgfqpoint{2.329631in}{1.227438in}}{\pgfqpoint{2.329631in}{1.238488in}}%
\pgfpathcurveto{\pgfqpoint{2.329631in}{1.249539in}}{\pgfqpoint{2.325241in}{1.260138in}}{\pgfqpoint{2.317427in}{1.267951in}}%
\pgfpathcurveto{\pgfqpoint{2.309614in}{1.275765in}}{\pgfqpoint{2.299015in}{1.280155in}}{\pgfqpoint{2.287965in}{1.280155in}}%
\pgfpathcurveto{\pgfqpoint{2.276915in}{1.280155in}}{\pgfqpoint{2.266316in}{1.275765in}}{\pgfqpoint{2.258502in}{1.267951in}}%
\pgfpathcurveto{\pgfqpoint{2.250688in}{1.260138in}}{\pgfqpoint{2.246298in}{1.249539in}}{\pgfqpoint{2.246298in}{1.238488in}}%
\pgfpathcurveto{\pgfqpoint{2.246298in}{1.227438in}}{\pgfqpoint{2.250688in}{1.216839in}}{\pgfqpoint{2.258502in}{1.209026in}}%
\pgfpathcurveto{\pgfqpoint{2.266316in}{1.201212in}}{\pgfqpoint{2.276915in}{1.196822in}}{\pgfqpoint{2.287965in}{1.196822in}}%
\pgfpathlineto{\pgfqpoint{2.287965in}{1.196822in}}%
\pgfpathclose%
\pgfusepath{stroke}%
\end{pgfscope}%
\begin{pgfscope}%
\pgfpathrectangle{\pgfqpoint{0.393053in}{0.375000in}}{\pgfqpoint{6.356833in}{5.175000in}}%
\pgfusepath{clip}%
\pgfsetbuttcap%
\pgfsetroundjoin%
\pgfsetlinewidth{1.003750pt}%
\definecolor{currentstroke}{rgb}{0.827451,0.827451,0.827451}%
\pgfsetstrokecolor{currentstroke}%
\pgfsetdash{}{0pt}%
\pgfpathmoveto{\pgfqpoint{0.508412in}{3.594376in}}%
\pgfpathcurveto{\pgfqpoint{0.519462in}{3.594376in}}{\pgfqpoint{0.530061in}{3.598766in}}{\pgfqpoint{0.537874in}{3.606580in}}%
\pgfpathcurveto{\pgfqpoint{0.545688in}{3.614393in}}{\pgfqpoint{0.550078in}{3.624992in}}{\pgfqpoint{0.550078in}{3.636042in}}%
\pgfpathcurveto{\pgfqpoint{0.550078in}{3.647092in}}{\pgfqpoint{0.545688in}{3.657692in}}{\pgfqpoint{0.537874in}{3.665505in}}%
\pgfpathcurveto{\pgfqpoint{0.530061in}{3.673319in}}{\pgfqpoint{0.519462in}{3.677709in}}{\pgfqpoint{0.508412in}{3.677709in}}%
\pgfpathcurveto{\pgfqpoint{0.497361in}{3.677709in}}{\pgfqpoint{0.486762in}{3.673319in}}{\pgfqpoint{0.478949in}{3.665505in}}%
\pgfpathcurveto{\pgfqpoint{0.471135in}{3.657692in}}{\pgfqpoint{0.466745in}{3.647092in}}{\pgfqpoint{0.466745in}{3.636042in}}%
\pgfpathcurveto{\pgfqpoint{0.466745in}{3.624992in}}{\pgfqpoint{0.471135in}{3.614393in}}{\pgfqpoint{0.478949in}{3.606580in}}%
\pgfpathcurveto{\pgfqpoint{0.486762in}{3.598766in}}{\pgfqpoint{0.497361in}{3.594376in}}{\pgfqpoint{0.508412in}{3.594376in}}%
\pgfpathlineto{\pgfqpoint{0.508412in}{3.594376in}}%
\pgfpathclose%
\pgfusepath{stroke}%
\end{pgfscope}%
\begin{pgfscope}%
\pgfpathrectangle{\pgfqpoint{0.393053in}{0.375000in}}{\pgfqpoint{6.356833in}{5.175000in}}%
\pgfusepath{clip}%
\pgfsetbuttcap%
\pgfsetroundjoin%
\pgfsetlinewidth{1.003750pt}%
\definecolor{currentstroke}{rgb}{0.827451,0.827451,0.827451}%
\pgfsetstrokecolor{currentstroke}%
\pgfsetdash{}{0pt}%
\pgfpathmoveto{\pgfqpoint{1.291595in}{1.999205in}}%
\pgfpathcurveto{\pgfqpoint{1.302646in}{1.999205in}}{\pgfqpoint{1.313245in}{2.003595in}}{\pgfqpoint{1.321058in}{2.011409in}}%
\pgfpathcurveto{\pgfqpoint{1.328872in}{2.019222in}}{\pgfqpoint{1.333262in}{2.029821in}}{\pgfqpoint{1.333262in}{2.040872in}}%
\pgfpathcurveto{\pgfqpoint{1.333262in}{2.051922in}}{\pgfqpoint{1.328872in}{2.062521in}}{\pgfqpoint{1.321058in}{2.070334in}}%
\pgfpathcurveto{\pgfqpoint{1.313245in}{2.078148in}}{\pgfqpoint{1.302646in}{2.082538in}}{\pgfqpoint{1.291595in}{2.082538in}}%
\pgfpathcurveto{\pgfqpoint{1.280545in}{2.082538in}}{\pgfqpoint{1.269946in}{2.078148in}}{\pgfqpoint{1.262133in}{2.070334in}}%
\pgfpathcurveto{\pgfqpoint{1.254319in}{2.062521in}}{\pgfqpoint{1.249929in}{2.051922in}}{\pgfqpoint{1.249929in}{2.040872in}}%
\pgfpathcurveto{\pgfqpoint{1.249929in}{2.029821in}}{\pgfqpoint{1.254319in}{2.019222in}}{\pgfqpoint{1.262133in}{2.011409in}}%
\pgfpathcurveto{\pgfqpoint{1.269946in}{2.003595in}}{\pgfqpoint{1.280545in}{1.999205in}}{\pgfqpoint{1.291595in}{1.999205in}}%
\pgfpathlineto{\pgfqpoint{1.291595in}{1.999205in}}%
\pgfpathclose%
\pgfusepath{stroke}%
\end{pgfscope}%
\begin{pgfscope}%
\pgfpathrectangle{\pgfqpoint{0.393053in}{0.375000in}}{\pgfqpoint{6.356833in}{5.175000in}}%
\pgfusepath{clip}%
\pgfsetbuttcap%
\pgfsetroundjoin%
\pgfsetlinewidth{1.003750pt}%
\definecolor{currentstroke}{rgb}{0.827451,0.827451,0.827451}%
\pgfsetstrokecolor{currentstroke}%
\pgfsetdash{}{0pt}%
\pgfpathmoveto{\pgfqpoint{1.882270in}{1.458195in}}%
\pgfpathcurveto{\pgfqpoint{1.893321in}{1.458195in}}{\pgfqpoint{1.903920in}{1.462585in}}{\pgfqpoint{1.911733in}{1.470399in}}%
\pgfpathcurveto{\pgfqpoint{1.919547in}{1.478213in}}{\pgfqpoint{1.923937in}{1.488812in}}{\pgfqpoint{1.923937in}{1.499862in}}%
\pgfpathcurveto{\pgfqpoint{1.923937in}{1.510912in}}{\pgfqpoint{1.919547in}{1.521511in}}{\pgfqpoint{1.911733in}{1.529325in}}%
\pgfpathcurveto{\pgfqpoint{1.903920in}{1.537138in}}{\pgfqpoint{1.893321in}{1.541529in}}{\pgfqpoint{1.882270in}{1.541529in}}%
\pgfpathcurveto{\pgfqpoint{1.871220in}{1.541529in}}{\pgfqpoint{1.860621in}{1.537138in}}{\pgfqpoint{1.852808in}{1.529325in}}%
\pgfpathcurveto{\pgfqpoint{1.844994in}{1.521511in}}{\pgfqpoint{1.840604in}{1.510912in}}{\pgfqpoint{1.840604in}{1.499862in}}%
\pgfpathcurveto{\pgfqpoint{1.840604in}{1.488812in}}{\pgfqpoint{1.844994in}{1.478213in}}{\pgfqpoint{1.852808in}{1.470399in}}%
\pgfpathcurveto{\pgfqpoint{1.860621in}{1.462585in}}{\pgfqpoint{1.871220in}{1.458195in}}{\pgfqpoint{1.882270in}{1.458195in}}%
\pgfpathlineto{\pgfqpoint{1.882270in}{1.458195in}}%
\pgfpathclose%
\pgfusepath{stroke}%
\end{pgfscope}%
\begin{pgfscope}%
\pgfpathrectangle{\pgfqpoint{0.393053in}{0.375000in}}{\pgfqpoint{6.356833in}{5.175000in}}%
\pgfusepath{clip}%
\pgfsetbuttcap%
\pgfsetroundjoin%
\pgfsetlinewidth{1.003750pt}%
\definecolor{currentstroke}{rgb}{0.827451,0.827451,0.827451}%
\pgfsetstrokecolor{currentstroke}%
\pgfsetdash{}{0pt}%
\pgfpathmoveto{\pgfqpoint{1.858747in}{1.496619in}}%
\pgfpathcurveto{\pgfqpoint{1.869797in}{1.496619in}}{\pgfqpoint{1.880396in}{1.501009in}}{\pgfqpoint{1.888209in}{1.508823in}}%
\pgfpathcurveto{\pgfqpoint{1.896023in}{1.516636in}}{\pgfqpoint{1.900413in}{1.527236in}}{\pgfqpoint{1.900413in}{1.538286in}}%
\pgfpathcurveto{\pgfqpoint{1.900413in}{1.549336in}}{\pgfqpoint{1.896023in}{1.559935in}}{\pgfqpoint{1.888209in}{1.567748in}}%
\pgfpathcurveto{\pgfqpoint{1.880396in}{1.575562in}}{\pgfqpoint{1.869797in}{1.579952in}}{\pgfqpoint{1.858747in}{1.579952in}}%
\pgfpathcurveto{\pgfqpoint{1.847696in}{1.579952in}}{\pgfqpoint{1.837097in}{1.575562in}}{\pgfqpoint{1.829284in}{1.567748in}}%
\pgfpathcurveto{\pgfqpoint{1.821470in}{1.559935in}}{\pgfqpoint{1.817080in}{1.549336in}}{\pgfqpoint{1.817080in}{1.538286in}}%
\pgfpathcurveto{\pgfqpoint{1.817080in}{1.527236in}}{\pgfqpoint{1.821470in}{1.516636in}}{\pgfqpoint{1.829284in}{1.508823in}}%
\pgfpathcurveto{\pgfqpoint{1.837097in}{1.501009in}}{\pgfqpoint{1.847696in}{1.496619in}}{\pgfqpoint{1.858747in}{1.496619in}}%
\pgfpathlineto{\pgfqpoint{1.858747in}{1.496619in}}%
\pgfpathclose%
\pgfusepath{stroke}%
\end{pgfscope}%
\begin{pgfscope}%
\pgfpathrectangle{\pgfqpoint{0.393053in}{0.375000in}}{\pgfqpoint{6.356833in}{5.175000in}}%
\pgfusepath{clip}%
\pgfsetbuttcap%
\pgfsetroundjoin%
\pgfsetlinewidth{1.003750pt}%
\definecolor{currentstroke}{rgb}{0.827451,0.827451,0.827451}%
\pgfsetstrokecolor{currentstroke}%
\pgfsetdash{}{0pt}%
\pgfpathmoveto{\pgfqpoint{2.124283in}{1.277669in}}%
\pgfpathcurveto{\pgfqpoint{2.135334in}{1.277669in}}{\pgfqpoint{2.145933in}{1.282059in}}{\pgfqpoint{2.153746in}{1.289873in}}%
\pgfpathcurveto{\pgfqpoint{2.161560in}{1.297686in}}{\pgfqpoint{2.165950in}{1.308286in}}{\pgfqpoint{2.165950in}{1.319336in}}%
\pgfpathcurveto{\pgfqpoint{2.165950in}{1.330386in}}{\pgfqpoint{2.161560in}{1.340985in}}{\pgfqpoint{2.153746in}{1.348798in}}%
\pgfpathcurveto{\pgfqpoint{2.145933in}{1.356612in}}{\pgfqpoint{2.135334in}{1.361002in}}{\pgfqpoint{2.124283in}{1.361002in}}%
\pgfpathcurveto{\pgfqpoint{2.113233in}{1.361002in}}{\pgfqpoint{2.102634in}{1.356612in}}{\pgfqpoint{2.094821in}{1.348798in}}%
\pgfpathcurveto{\pgfqpoint{2.087007in}{1.340985in}}{\pgfqpoint{2.082617in}{1.330386in}}{\pgfqpoint{2.082617in}{1.319336in}}%
\pgfpathcurveto{\pgfqpoint{2.082617in}{1.308286in}}{\pgfqpoint{2.087007in}{1.297686in}}{\pgfqpoint{2.094821in}{1.289873in}}%
\pgfpathcurveto{\pgfqpoint{2.102634in}{1.282059in}}{\pgfqpoint{2.113233in}{1.277669in}}{\pgfqpoint{2.124283in}{1.277669in}}%
\pgfpathlineto{\pgfqpoint{2.124283in}{1.277669in}}%
\pgfpathclose%
\pgfusepath{stroke}%
\end{pgfscope}%
\begin{pgfscope}%
\pgfpathrectangle{\pgfqpoint{0.393053in}{0.375000in}}{\pgfqpoint{6.356833in}{5.175000in}}%
\pgfusepath{clip}%
\pgfsetbuttcap%
\pgfsetroundjoin%
\pgfsetlinewidth{1.003750pt}%
\definecolor{currentstroke}{rgb}{0.827451,0.827451,0.827451}%
\pgfsetstrokecolor{currentstroke}%
\pgfsetdash{}{0pt}%
\pgfpathmoveto{\pgfqpoint{3.184302in}{0.739704in}}%
\pgfpathcurveto{\pgfqpoint{3.195352in}{0.739704in}}{\pgfqpoint{3.205951in}{0.744095in}}{\pgfqpoint{3.213765in}{0.751908in}}%
\pgfpathcurveto{\pgfqpoint{3.221578in}{0.759722in}}{\pgfqpoint{3.225969in}{0.770321in}}{\pgfqpoint{3.225969in}{0.781371in}}%
\pgfpathcurveto{\pgfqpoint{3.225969in}{0.792421in}}{\pgfqpoint{3.221578in}{0.803020in}}{\pgfqpoint{3.213765in}{0.810834in}}%
\pgfpathcurveto{\pgfqpoint{3.205951in}{0.818647in}}{\pgfqpoint{3.195352in}{0.823038in}}{\pgfqpoint{3.184302in}{0.823038in}}%
\pgfpathcurveto{\pgfqpoint{3.173252in}{0.823038in}}{\pgfqpoint{3.162653in}{0.818647in}}{\pgfqpoint{3.154839in}{0.810834in}}%
\pgfpathcurveto{\pgfqpoint{3.147026in}{0.803020in}}{\pgfqpoint{3.142635in}{0.792421in}}{\pgfqpoint{3.142635in}{0.781371in}}%
\pgfpathcurveto{\pgfqpoint{3.142635in}{0.770321in}}{\pgfqpoint{3.147026in}{0.759722in}}{\pgfqpoint{3.154839in}{0.751908in}}%
\pgfpathcurveto{\pgfqpoint{3.162653in}{0.744095in}}{\pgfqpoint{3.173252in}{0.739704in}}{\pgfqpoint{3.184302in}{0.739704in}}%
\pgfpathlineto{\pgfqpoint{3.184302in}{0.739704in}}%
\pgfpathclose%
\pgfusepath{stroke}%
\end{pgfscope}%
\begin{pgfscope}%
\pgfpathrectangle{\pgfqpoint{0.393053in}{0.375000in}}{\pgfqpoint{6.356833in}{5.175000in}}%
\pgfusepath{clip}%
\pgfsetbuttcap%
\pgfsetroundjoin%
\pgfsetlinewidth{1.003750pt}%
\definecolor{currentstroke}{rgb}{0.827451,0.827451,0.827451}%
\pgfsetstrokecolor{currentstroke}%
\pgfsetdash{}{0pt}%
\pgfpathmoveto{\pgfqpoint{1.911625in}{1.440877in}}%
\pgfpathcurveto{\pgfqpoint{1.922675in}{1.440877in}}{\pgfqpoint{1.933274in}{1.445267in}}{\pgfqpoint{1.941087in}{1.453081in}}%
\pgfpathcurveto{\pgfqpoint{1.948901in}{1.460894in}}{\pgfqpoint{1.953291in}{1.471493in}}{\pgfqpoint{1.953291in}{1.482543in}}%
\pgfpathcurveto{\pgfqpoint{1.953291in}{1.493594in}}{\pgfqpoint{1.948901in}{1.504193in}}{\pgfqpoint{1.941087in}{1.512006in}}%
\pgfpathcurveto{\pgfqpoint{1.933274in}{1.519820in}}{\pgfqpoint{1.922675in}{1.524210in}}{\pgfqpoint{1.911625in}{1.524210in}}%
\pgfpathcurveto{\pgfqpoint{1.900574in}{1.524210in}}{\pgfqpoint{1.889975in}{1.519820in}}{\pgfqpoint{1.882162in}{1.512006in}}%
\pgfpathcurveto{\pgfqpoint{1.874348in}{1.504193in}}{\pgfqpoint{1.869958in}{1.493594in}}{\pgfqpoint{1.869958in}{1.482543in}}%
\pgfpathcurveto{\pgfqpoint{1.869958in}{1.471493in}}{\pgfqpoint{1.874348in}{1.460894in}}{\pgfqpoint{1.882162in}{1.453081in}}%
\pgfpathcurveto{\pgfqpoint{1.889975in}{1.445267in}}{\pgfqpoint{1.900574in}{1.440877in}}{\pgfqpoint{1.911625in}{1.440877in}}%
\pgfpathlineto{\pgfqpoint{1.911625in}{1.440877in}}%
\pgfpathclose%
\pgfusepath{stroke}%
\end{pgfscope}%
\begin{pgfscope}%
\pgfpathrectangle{\pgfqpoint{0.393053in}{0.375000in}}{\pgfqpoint{6.356833in}{5.175000in}}%
\pgfusepath{clip}%
\pgfsetbuttcap%
\pgfsetroundjoin%
\pgfsetlinewidth{1.003750pt}%
\definecolor{currentstroke}{rgb}{0.827451,0.827451,0.827451}%
\pgfsetstrokecolor{currentstroke}%
\pgfsetdash{}{0pt}%
\pgfpathmoveto{\pgfqpoint{2.103640in}{1.312371in}}%
\pgfpathcurveto{\pgfqpoint{2.114690in}{1.312371in}}{\pgfqpoint{2.125289in}{1.316762in}}{\pgfqpoint{2.133103in}{1.324575in}}%
\pgfpathcurveto{\pgfqpoint{2.140916in}{1.332389in}}{\pgfqpoint{2.145307in}{1.342988in}}{\pgfqpoint{2.145307in}{1.354038in}}%
\pgfpathcurveto{\pgfqpoint{2.145307in}{1.365088in}}{\pgfqpoint{2.140916in}{1.375687in}}{\pgfqpoint{2.133103in}{1.383501in}}%
\pgfpathcurveto{\pgfqpoint{2.125289in}{1.391314in}}{\pgfqpoint{2.114690in}{1.395705in}}{\pgfqpoint{2.103640in}{1.395705in}}%
\pgfpathcurveto{\pgfqpoint{2.092590in}{1.395705in}}{\pgfqpoint{2.081991in}{1.391314in}}{\pgfqpoint{2.074177in}{1.383501in}}%
\pgfpathcurveto{\pgfqpoint{2.066364in}{1.375687in}}{\pgfqpoint{2.061973in}{1.365088in}}{\pgfqpoint{2.061973in}{1.354038in}}%
\pgfpathcurveto{\pgfqpoint{2.061973in}{1.342988in}}{\pgfqpoint{2.066364in}{1.332389in}}{\pgfqpoint{2.074177in}{1.324575in}}%
\pgfpathcurveto{\pgfqpoint{2.081991in}{1.316762in}}{\pgfqpoint{2.092590in}{1.312371in}}{\pgfqpoint{2.103640in}{1.312371in}}%
\pgfpathlineto{\pgfqpoint{2.103640in}{1.312371in}}%
\pgfpathclose%
\pgfusepath{stroke}%
\end{pgfscope}%
\begin{pgfscope}%
\pgfpathrectangle{\pgfqpoint{0.393053in}{0.375000in}}{\pgfqpoint{6.356833in}{5.175000in}}%
\pgfusepath{clip}%
\pgfsetbuttcap%
\pgfsetroundjoin%
\pgfsetlinewidth{1.003750pt}%
\definecolor{currentstroke}{rgb}{0.827451,0.827451,0.827451}%
\pgfsetstrokecolor{currentstroke}%
\pgfsetdash{}{0pt}%
\pgfpathmoveto{\pgfqpoint{1.256856in}{2.039646in}}%
\pgfpathcurveto{\pgfqpoint{1.267906in}{2.039646in}}{\pgfqpoint{1.278505in}{2.044036in}}{\pgfqpoint{1.286319in}{2.051849in}}%
\pgfpathcurveto{\pgfqpoint{1.294132in}{2.059663in}}{\pgfqpoint{1.298523in}{2.070262in}}{\pgfqpoint{1.298523in}{2.081312in}}%
\pgfpathcurveto{\pgfqpoint{1.298523in}{2.092362in}}{\pgfqpoint{1.294132in}{2.102961in}}{\pgfqpoint{1.286319in}{2.110775in}}%
\pgfpathcurveto{\pgfqpoint{1.278505in}{2.118589in}}{\pgfqpoint{1.267906in}{2.122979in}}{\pgfqpoint{1.256856in}{2.122979in}}%
\pgfpathcurveto{\pgfqpoint{1.245806in}{2.122979in}}{\pgfqpoint{1.235207in}{2.118589in}}{\pgfqpoint{1.227393in}{2.110775in}}%
\pgfpathcurveto{\pgfqpoint{1.219580in}{2.102961in}}{\pgfqpoint{1.215189in}{2.092362in}}{\pgfqpoint{1.215189in}{2.081312in}}%
\pgfpathcurveto{\pgfqpoint{1.215189in}{2.070262in}}{\pgfqpoint{1.219580in}{2.059663in}}{\pgfqpoint{1.227393in}{2.051849in}}%
\pgfpathcurveto{\pgfqpoint{1.235207in}{2.044036in}}{\pgfqpoint{1.245806in}{2.039646in}}{\pgfqpoint{1.256856in}{2.039646in}}%
\pgfpathlineto{\pgfqpoint{1.256856in}{2.039646in}}%
\pgfpathclose%
\pgfusepath{stroke}%
\end{pgfscope}%
\begin{pgfscope}%
\pgfpathrectangle{\pgfqpoint{0.393053in}{0.375000in}}{\pgfqpoint{6.356833in}{5.175000in}}%
\pgfusepath{clip}%
\pgfsetbuttcap%
\pgfsetroundjoin%
\pgfsetlinewidth{1.003750pt}%
\definecolor{currentstroke}{rgb}{0.827451,0.827451,0.827451}%
\pgfsetstrokecolor{currentstroke}%
\pgfsetdash{}{0pt}%
\pgfpathmoveto{\pgfqpoint{3.746487in}{0.578247in}}%
\pgfpathcurveto{\pgfqpoint{3.757537in}{0.578247in}}{\pgfqpoint{3.768136in}{0.582637in}}{\pgfqpoint{3.775950in}{0.590451in}}%
\pgfpathcurveto{\pgfqpoint{3.783763in}{0.598264in}}{\pgfqpoint{3.788154in}{0.608863in}}{\pgfqpoint{3.788154in}{0.619914in}}%
\pgfpathcurveto{\pgfqpoint{3.788154in}{0.630964in}}{\pgfqpoint{3.783763in}{0.641563in}}{\pgfqpoint{3.775950in}{0.649376in}}%
\pgfpathcurveto{\pgfqpoint{3.768136in}{0.657190in}}{\pgfqpoint{3.757537in}{0.661580in}}{\pgfqpoint{3.746487in}{0.661580in}}%
\pgfpathcurveto{\pgfqpoint{3.735437in}{0.661580in}}{\pgfqpoint{3.724838in}{0.657190in}}{\pgfqpoint{3.717024in}{0.649376in}}%
\pgfpathcurveto{\pgfqpoint{3.709211in}{0.641563in}}{\pgfqpoint{3.704820in}{0.630964in}}{\pgfqpoint{3.704820in}{0.619914in}}%
\pgfpathcurveto{\pgfqpoint{3.704820in}{0.608863in}}{\pgfqpoint{3.709211in}{0.598264in}}{\pgfqpoint{3.717024in}{0.590451in}}%
\pgfpathcurveto{\pgfqpoint{3.724838in}{0.582637in}}{\pgfqpoint{3.735437in}{0.578247in}}{\pgfqpoint{3.746487in}{0.578247in}}%
\pgfpathlineto{\pgfqpoint{3.746487in}{0.578247in}}%
\pgfpathclose%
\pgfusepath{stroke}%
\end{pgfscope}%
\begin{pgfscope}%
\pgfpathrectangle{\pgfqpoint{0.393053in}{0.375000in}}{\pgfqpoint{6.356833in}{5.175000in}}%
\pgfusepath{clip}%
\pgfsetbuttcap%
\pgfsetroundjoin%
\pgfsetlinewidth{1.003750pt}%
\definecolor{currentstroke}{rgb}{0.827451,0.827451,0.827451}%
\pgfsetstrokecolor{currentstroke}%
\pgfsetdash{}{0pt}%
\pgfpathmoveto{\pgfqpoint{1.195439in}{2.114255in}}%
\pgfpathcurveto{\pgfqpoint{1.206489in}{2.114255in}}{\pgfqpoint{1.217089in}{2.118645in}}{\pgfqpoint{1.224902in}{2.126458in}}%
\pgfpathcurveto{\pgfqpoint{1.232716in}{2.134272in}}{\pgfqpoint{1.237106in}{2.144871in}}{\pgfqpoint{1.237106in}{2.155921in}}%
\pgfpathcurveto{\pgfqpoint{1.237106in}{2.166971in}}{\pgfqpoint{1.232716in}{2.177570in}}{\pgfqpoint{1.224902in}{2.185384in}}%
\pgfpathcurveto{\pgfqpoint{1.217089in}{2.193198in}}{\pgfqpoint{1.206489in}{2.197588in}}{\pgfqpoint{1.195439in}{2.197588in}}%
\pgfpathcurveto{\pgfqpoint{1.184389in}{2.197588in}}{\pgfqpoint{1.173790in}{2.193198in}}{\pgfqpoint{1.165977in}{2.185384in}}%
\pgfpathcurveto{\pgfqpoint{1.158163in}{2.177570in}}{\pgfqpoint{1.153773in}{2.166971in}}{\pgfqpoint{1.153773in}{2.155921in}}%
\pgfpathcurveto{\pgfqpoint{1.153773in}{2.144871in}}{\pgfqpoint{1.158163in}{2.134272in}}{\pgfqpoint{1.165977in}{2.126458in}}%
\pgfpathcurveto{\pgfqpoint{1.173790in}{2.118645in}}{\pgfqpoint{1.184389in}{2.114255in}}{\pgfqpoint{1.195439in}{2.114255in}}%
\pgfpathlineto{\pgfqpoint{1.195439in}{2.114255in}}%
\pgfpathclose%
\pgfusepath{stroke}%
\end{pgfscope}%
\begin{pgfscope}%
\pgfpathrectangle{\pgfqpoint{0.393053in}{0.375000in}}{\pgfqpoint{6.356833in}{5.175000in}}%
\pgfusepath{clip}%
\pgfsetbuttcap%
\pgfsetroundjoin%
\pgfsetlinewidth{1.003750pt}%
\definecolor{currentstroke}{rgb}{0.827451,0.827451,0.827451}%
\pgfsetstrokecolor{currentstroke}%
\pgfsetdash{}{0pt}%
\pgfpathmoveto{\pgfqpoint{3.695118in}{0.605158in}}%
\pgfpathcurveto{\pgfqpoint{3.706168in}{0.605158in}}{\pgfqpoint{3.716767in}{0.609548in}}{\pgfqpoint{3.724581in}{0.617362in}}%
\pgfpathcurveto{\pgfqpoint{3.732394in}{0.625175in}}{\pgfqpoint{3.736785in}{0.635775in}}{\pgfqpoint{3.736785in}{0.646825in}}%
\pgfpathcurveto{\pgfqpoint{3.736785in}{0.657875in}}{\pgfqpoint{3.732394in}{0.668474in}}{\pgfqpoint{3.724581in}{0.676287in}}%
\pgfpathcurveto{\pgfqpoint{3.716767in}{0.684101in}}{\pgfqpoint{3.706168in}{0.688491in}}{\pgfqpoint{3.695118in}{0.688491in}}%
\pgfpathcurveto{\pgfqpoint{3.684068in}{0.688491in}}{\pgfqpoint{3.673469in}{0.684101in}}{\pgfqpoint{3.665655in}{0.676287in}}%
\pgfpathcurveto{\pgfqpoint{3.657842in}{0.668474in}}{\pgfqpoint{3.653451in}{0.657875in}}{\pgfqpoint{3.653451in}{0.646825in}}%
\pgfpathcurveto{\pgfqpoint{3.653451in}{0.635775in}}{\pgfqpoint{3.657842in}{0.625175in}}{\pgfqpoint{3.665655in}{0.617362in}}%
\pgfpathcurveto{\pgfqpoint{3.673469in}{0.609548in}}{\pgfqpoint{3.684068in}{0.605158in}}{\pgfqpoint{3.695118in}{0.605158in}}%
\pgfpathlineto{\pgfqpoint{3.695118in}{0.605158in}}%
\pgfpathclose%
\pgfusepath{stroke}%
\end{pgfscope}%
\begin{pgfscope}%
\pgfpathrectangle{\pgfqpoint{0.393053in}{0.375000in}}{\pgfqpoint{6.356833in}{5.175000in}}%
\pgfusepath{clip}%
\pgfsetbuttcap%
\pgfsetroundjoin%
\pgfsetlinewidth{1.003750pt}%
\definecolor{currentstroke}{rgb}{0.827451,0.827451,0.827451}%
\pgfsetstrokecolor{currentstroke}%
\pgfsetdash{}{0pt}%
\pgfpathmoveto{\pgfqpoint{3.210188in}{0.736681in}}%
\pgfpathcurveto{\pgfqpoint{3.221238in}{0.736681in}}{\pgfqpoint{3.231837in}{0.741071in}}{\pgfqpoint{3.239651in}{0.748885in}}%
\pgfpathcurveto{\pgfqpoint{3.247464in}{0.756699in}}{\pgfqpoint{3.251855in}{0.767298in}}{\pgfqpoint{3.251855in}{0.778348in}}%
\pgfpathcurveto{\pgfqpoint{3.251855in}{0.789398in}}{\pgfqpoint{3.247464in}{0.799997in}}{\pgfqpoint{3.239651in}{0.807810in}}%
\pgfpathcurveto{\pgfqpoint{3.231837in}{0.815624in}}{\pgfqpoint{3.221238in}{0.820014in}}{\pgfqpoint{3.210188in}{0.820014in}}%
\pgfpathcurveto{\pgfqpoint{3.199138in}{0.820014in}}{\pgfqpoint{3.188539in}{0.815624in}}{\pgfqpoint{3.180725in}{0.807810in}}%
\pgfpathcurveto{\pgfqpoint{3.172912in}{0.799997in}}{\pgfqpoint{3.168521in}{0.789398in}}{\pgfqpoint{3.168521in}{0.778348in}}%
\pgfpathcurveto{\pgfqpoint{3.168521in}{0.767298in}}{\pgfqpoint{3.172912in}{0.756699in}}{\pgfqpoint{3.180725in}{0.748885in}}%
\pgfpathcurveto{\pgfqpoint{3.188539in}{0.741071in}}{\pgfqpoint{3.199138in}{0.736681in}}{\pgfqpoint{3.210188in}{0.736681in}}%
\pgfpathlineto{\pgfqpoint{3.210188in}{0.736681in}}%
\pgfpathclose%
\pgfusepath{stroke}%
\end{pgfscope}%
\begin{pgfscope}%
\pgfpathrectangle{\pgfqpoint{0.393053in}{0.375000in}}{\pgfqpoint{6.356833in}{5.175000in}}%
\pgfusepath{clip}%
\pgfsetbuttcap%
\pgfsetroundjoin%
\pgfsetlinewidth{1.003750pt}%
\definecolor{currentstroke}{rgb}{0.827451,0.827451,0.827451}%
\pgfsetstrokecolor{currentstroke}%
\pgfsetdash{}{0pt}%
\pgfpathmoveto{\pgfqpoint{2.498084in}{1.056522in}}%
\pgfpathcurveto{\pgfqpoint{2.509134in}{1.056522in}}{\pgfqpoint{2.519733in}{1.060912in}}{\pgfqpoint{2.527547in}{1.068726in}}%
\pgfpathcurveto{\pgfqpoint{2.535360in}{1.076539in}}{\pgfqpoint{2.539750in}{1.087138in}}{\pgfqpoint{2.539750in}{1.098188in}}%
\pgfpathcurveto{\pgfqpoint{2.539750in}{1.109239in}}{\pgfqpoint{2.535360in}{1.119838in}}{\pgfqpoint{2.527547in}{1.127651in}}%
\pgfpathcurveto{\pgfqpoint{2.519733in}{1.135465in}}{\pgfqpoint{2.509134in}{1.139855in}}{\pgfqpoint{2.498084in}{1.139855in}}%
\pgfpathcurveto{\pgfqpoint{2.487034in}{1.139855in}}{\pgfqpoint{2.476435in}{1.135465in}}{\pgfqpoint{2.468621in}{1.127651in}}%
\pgfpathcurveto{\pgfqpoint{2.460807in}{1.119838in}}{\pgfqpoint{2.456417in}{1.109239in}}{\pgfqpoint{2.456417in}{1.098188in}}%
\pgfpathcurveto{\pgfqpoint{2.456417in}{1.087138in}}{\pgfqpoint{2.460807in}{1.076539in}}{\pgfqpoint{2.468621in}{1.068726in}}%
\pgfpathcurveto{\pgfqpoint{2.476435in}{1.060912in}}{\pgfqpoint{2.487034in}{1.056522in}}{\pgfqpoint{2.498084in}{1.056522in}}%
\pgfpathlineto{\pgfqpoint{2.498084in}{1.056522in}}%
\pgfpathclose%
\pgfusepath{stroke}%
\end{pgfscope}%
\begin{pgfscope}%
\pgfpathrectangle{\pgfqpoint{0.393053in}{0.375000in}}{\pgfqpoint{6.356833in}{5.175000in}}%
\pgfusepath{clip}%
\pgfsetbuttcap%
\pgfsetroundjoin%
\pgfsetlinewidth{1.003750pt}%
\definecolor{currentstroke}{rgb}{0.827451,0.827451,0.827451}%
\pgfsetstrokecolor{currentstroke}%
\pgfsetdash{}{0pt}%
\pgfpathmoveto{\pgfqpoint{3.440404in}{0.644692in}}%
\pgfpathcurveto{\pgfqpoint{3.451454in}{0.644692in}}{\pgfqpoint{3.462053in}{0.649082in}}{\pgfqpoint{3.469866in}{0.656896in}}%
\pgfpathcurveto{\pgfqpoint{3.477680in}{0.664710in}}{\pgfqpoint{3.482070in}{0.675309in}}{\pgfqpoint{3.482070in}{0.686359in}}%
\pgfpathcurveto{\pgfqpoint{3.482070in}{0.697409in}}{\pgfqpoint{3.477680in}{0.708008in}}{\pgfqpoint{3.469866in}{0.715822in}}%
\pgfpathcurveto{\pgfqpoint{3.462053in}{0.723635in}}{\pgfqpoint{3.451454in}{0.728025in}}{\pgfqpoint{3.440404in}{0.728025in}}%
\pgfpathcurveto{\pgfqpoint{3.429353in}{0.728025in}}{\pgfqpoint{3.418754in}{0.723635in}}{\pgfqpoint{3.410941in}{0.715822in}}%
\pgfpathcurveto{\pgfqpoint{3.403127in}{0.708008in}}{\pgfqpoint{3.398737in}{0.697409in}}{\pgfqpoint{3.398737in}{0.686359in}}%
\pgfpathcurveto{\pgfqpoint{3.398737in}{0.675309in}}{\pgfqpoint{3.403127in}{0.664710in}}{\pgfqpoint{3.410941in}{0.656896in}}%
\pgfpathcurveto{\pgfqpoint{3.418754in}{0.649082in}}{\pgfqpoint{3.429353in}{0.644692in}}{\pgfqpoint{3.440404in}{0.644692in}}%
\pgfpathlineto{\pgfqpoint{3.440404in}{0.644692in}}%
\pgfpathclose%
\pgfusepath{stroke}%
\end{pgfscope}%
\begin{pgfscope}%
\pgfpathrectangle{\pgfqpoint{0.393053in}{0.375000in}}{\pgfqpoint{6.356833in}{5.175000in}}%
\pgfusepath{clip}%
\pgfsetbuttcap%
\pgfsetroundjoin%
\pgfsetlinewidth{1.003750pt}%
\definecolor{currentstroke}{rgb}{0.827451,0.827451,0.827451}%
\pgfsetstrokecolor{currentstroke}%
\pgfsetdash{}{0pt}%
\pgfpathmoveto{\pgfqpoint{1.250251in}{2.047500in}}%
\pgfpathcurveto{\pgfqpoint{1.261301in}{2.047500in}}{\pgfqpoint{1.271900in}{2.051890in}}{\pgfqpoint{1.279714in}{2.059704in}}%
\pgfpathcurveto{\pgfqpoint{1.287527in}{2.067518in}}{\pgfqpoint{1.291918in}{2.078117in}}{\pgfqpoint{1.291918in}{2.089167in}}%
\pgfpathcurveto{\pgfqpoint{1.291918in}{2.100217in}}{\pgfqpoint{1.287527in}{2.110816in}}{\pgfqpoint{1.279714in}{2.118630in}}%
\pgfpathcurveto{\pgfqpoint{1.271900in}{2.126443in}}{\pgfqpoint{1.261301in}{2.130833in}}{\pgfqpoint{1.250251in}{2.130833in}}%
\pgfpathcurveto{\pgfqpoint{1.239201in}{2.130833in}}{\pgfqpoint{1.228602in}{2.126443in}}{\pgfqpoint{1.220788in}{2.118630in}}%
\pgfpathcurveto{\pgfqpoint{1.212974in}{2.110816in}}{\pgfqpoint{1.208584in}{2.100217in}}{\pgfqpoint{1.208584in}{2.089167in}}%
\pgfpathcurveto{\pgfqpoint{1.208584in}{2.078117in}}{\pgfqpoint{1.212974in}{2.067518in}}{\pgfqpoint{1.220788in}{2.059704in}}%
\pgfpathcurveto{\pgfqpoint{1.228602in}{2.051890in}}{\pgfqpoint{1.239201in}{2.047500in}}{\pgfqpoint{1.250251in}{2.047500in}}%
\pgfpathlineto{\pgfqpoint{1.250251in}{2.047500in}}%
\pgfpathclose%
\pgfusepath{stroke}%
\end{pgfscope}%
\begin{pgfscope}%
\pgfpathrectangle{\pgfqpoint{0.393053in}{0.375000in}}{\pgfqpoint{6.356833in}{5.175000in}}%
\pgfusepath{clip}%
\pgfsetbuttcap%
\pgfsetroundjoin%
\pgfsetlinewidth{1.003750pt}%
\definecolor{currentstroke}{rgb}{0.827451,0.827451,0.827451}%
\pgfsetstrokecolor{currentstroke}%
\pgfsetdash{}{0pt}%
\pgfpathmoveto{\pgfqpoint{2.280509in}{1.199195in}}%
\pgfpathcurveto{\pgfqpoint{2.291559in}{1.199195in}}{\pgfqpoint{2.302158in}{1.203585in}}{\pgfqpoint{2.309972in}{1.211399in}}%
\pgfpathcurveto{\pgfqpoint{2.317785in}{1.219212in}}{\pgfqpoint{2.322176in}{1.229811in}}{\pgfqpoint{2.322176in}{1.240862in}}%
\pgfpathcurveto{\pgfqpoint{2.322176in}{1.251912in}}{\pgfqpoint{2.317785in}{1.262511in}}{\pgfqpoint{2.309972in}{1.270324in}}%
\pgfpathcurveto{\pgfqpoint{2.302158in}{1.278138in}}{\pgfqpoint{2.291559in}{1.282528in}}{\pgfqpoint{2.280509in}{1.282528in}}%
\pgfpathcurveto{\pgfqpoint{2.269459in}{1.282528in}}{\pgfqpoint{2.258860in}{1.278138in}}{\pgfqpoint{2.251046in}{1.270324in}}%
\pgfpathcurveto{\pgfqpoint{2.243232in}{1.262511in}}{\pgfqpoint{2.238842in}{1.251912in}}{\pgfqpoint{2.238842in}{1.240862in}}%
\pgfpathcurveto{\pgfqpoint{2.238842in}{1.229811in}}{\pgfqpoint{2.243232in}{1.219212in}}{\pgfqpoint{2.251046in}{1.211399in}}%
\pgfpathcurveto{\pgfqpoint{2.258860in}{1.203585in}}{\pgfqpoint{2.269459in}{1.199195in}}{\pgfqpoint{2.280509in}{1.199195in}}%
\pgfpathlineto{\pgfqpoint{2.280509in}{1.199195in}}%
\pgfpathclose%
\pgfusepath{stroke}%
\end{pgfscope}%
\begin{pgfscope}%
\pgfpathrectangle{\pgfqpoint{0.393053in}{0.375000in}}{\pgfqpoint{6.356833in}{5.175000in}}%
\pgfusepath{clip}%
\pgfsetbuttcap%
\pgfsetroundjoin%
\pgfsetlinewidth{1.003750pt}%
\definecolor{currentstroke}{rgb}{0.827451,0.827451,0.827451}%
\pgfsetstrokecolor{currentstroke}%
\pgfsetdash{}{0pt}%
\pgfpathmoveto{\pgfqpoint{1.375643in}{1.906128in}}%
\pgfpathcurveto{\pgfqpoint{1.386694in}{1.906128in}}{\pgfqpoint{1.397293in}{1.910518in}}{\pgfqpoint{1.405106in}{1.918331in}}%
\pgfpathcurveto{\pgfqpoint{1.412920in}{1.926145in}}{\pgfqpoint{1.417310in}{1.936744in}}{\pgfqpoint{1.417310in}{1.947794in}}%
\pgfpathcurveto{\pgfqpoint{1.417310in}{1.958844in}}{\pgfqpoint{1.412920in}{1.969443in}}{\pgfqpoint{1.405106in}{1.977257in}}%
\pgfpathcurveto{\pgfqpoint{1.397293in}{1.985071in}}{\pgfqpoint{1.386694in}{1.989461in}}{\pgfqpoint{1.375643in}{1.989461in}}%
\pgfpathcurveto{\pgfqpoint{1.364593in}{1.989461in}}{\pgfqpoint{1.353994in}{1.985071in}}{\pgfqpoint{1.346181in}{1.977257in}}%
\pgfpathcurveto{\pgfqpoint{1.338367in}{1.969443in}}{\pgfqpoint{1.333977in}{1.958844in}}{\pgfqpoint{1.333977in}{1.947794in}}%
\pgfpathcurveto{\pgfqpoint{1.333977in}{1.936744in}}{\pgfqpoint{1.338367in}{1.926145in}}{\pgfqpoint{1.346181in}{1.918331in}}%
\pgfpathcurveto{\pgfqpoint{1.353994in}{1.910518in}}{\pgfqpoint{1.364593in}{1.906128in}}{\pgfqpoint{1.375643in}{1.906128in}}%
\pgfpathlineto{\pgfqpoint{1.375643in}{1.906128in}}%
\pgfpathclose%
\pgfusepath{stroke}%
\end{pgfscope}%
\begin{pgfscope}%
\pgfpathrectangle{\pgfqpoint{0.393053in}{0.375000in}}{\pgfqpoint{6.356833in}{5.175000in}}%
\pgfusepath{clip}%
\pgfsetbuttcap%
\pgfsetroundjoin%
\pgfsetlinewidth{1.003750pt}%
\definecolor{currentstroke}{rgb}{0.827451,0.827451,0.827451}%
\pgfsetstrokecolor{currentstroke}%
\pgfsetdash{}{0pt}%
\pgfpathmoveto{\pgfqpoint{3.694432in}{0.605429in}}%
\pgfpathcurveto{\pgfqpoint{3.705482in}{0.605429in}}{\pgfqpoint{3.716081in}{0.609820in}}{\pgfqpoint{3.723895in}{0.617633in}}%
\pgfpathcurveto{\pgfqpoint{3.731708in}{0.625447in}}{\pgfqpoint{3.736098in}{0.636046in}}{\pgfqpoint{3.736098in}{0.647096in}}%
\pgfpathcurveto{\pgfqpoint{3.736098in}{0.658146in}}{\pgfqpoint{3.731708in}{0.668745in}}{\pgfqpoint{3.723895in}{0.676559in}}%
\pgfpathcurveto{\pgfqpoint{3.716081in}{0.684372in}}{\pgfqpoint{3.705482in}{0.688763in}}{\pgfqpoint{3.694432in}{0.688763in}}%
\pgfpathcurveto{\pgfqpoint{3.683382in}{0.688763in}}{\pgfqpoint{3.672783in}{0.684372in}}{\pgfqpoint{3.664969in}{0.676559in}}%
\pgfpathcurveto{\pgfqpoint{3.657155in}{0.668745in}}{\pgfqpoint{3.652765in}{0.658146in}}{\pgfqpoint{3.652765in}{0.647096in}}%
\pgfpathcurveto{\pgfqpoint{3.652765in}{0.636046in}}{\pgfqpoint{3.657155in}{0.625447in}}{\pgfqpoint{3.664969in}{0.617633in}}%
\pgfpathcurveto{\pgfqpoint{3.672783in}{0.609820in}}{\pgfqpoint{3.683382in}{0.605429in}}{\pgfqpoint{3.694432in}{0.605429in}}%
\pgfpathlineto{\pgfqpoint{3.694432in}{0.605429in}}%
\pgfpathclose%
\pgfusepath{stroke}%
\end{pgfscope}%
\begin{pgfscope}%
\pgfpathrectangle{\pgfqpoint{0.393053in}{0.375000in}}{\pgfqpoint{6.356833in}{5.175000in}}%
\pgfusepath{clip}%
\pgfsetbuttcap%
\pgfsetroundjoin%
\pgfsetlinewidth{1.003750pt}%
\definecolor{currentstroke}{rgb}{0.827451,0.827451,0.827451}%
\pgfsetstrokecolor{currentstroke}%
\pgfsetdash{}{0pt}%
\pgfpathmoveto{\pgfqpoint{0.396398in}{4.410130in}}%
\pgfpathcurveto{\pgfqpoint{0.407448in}{4.410130in}}{\pgfqpoint{0.418047in}{4.414521in}}{\pgfqpoint{0.425861in}{4.422334in}}%
\pgfpathcurveto{\pgfqpoint{0.433674in}{4.430148in}}{\pgfqpoint{0.438064in}{4.440747in}}{\pgfqpoint{0.438064in}{4.451797in}}%
\pgfpathcurveto{\pgfqpoint{0.438064in}{4.462847in}}{\pgfqpoint{0.433674in}{4.473446in}}{\pgfqpoint{0.425861in}{4.481260in}}%
\pgfpathcurveto{\pgfqpoint{0.418047in}{4.489073in}}{\pgfqpoint{0.407448in}{4.493464in}}{\pgfqpoint{0.396398in}{4.493464in}}%
\pgfpathcurveto{\pgfqpoint{0.385348in}{4.493464in}}{\pgfqpoint{0.374749in}{4.489073in}}{\pgfqpoint{0.366935in}{4.481260in}}%
\pgfpathcurveto{\pgfqpoint{0.359121in}{4.473446in}}{\pgfqpoint{0.354731in}{4.462847in}}{\pgfqpoint{0.354731in}{4.451797in}}%
\pgfpathcurveto{\pgfqpoint{0.354731in}{4.440747in}}{\pgfqpoint{0.359121in}{4.430148in}}{\pgfqpoint{0.366935in}{4.422334in}}%
\pgfpathcurveto{\pgfqpoint{0.374749in}{4.414521in}}{\pgfqpoint{0.385348in}{4.410130in}}{\pgfqpoint{0.396398in}{4.410130in}}%
\pgfpathlineto{\pgfqpoint{0.396398in}{4.410130in}}%
\pgfpathclose%
\pgfusepath{stroke}%
\end{pgfscope}%
\begin{pgfscope}%
\pgfpathrectangle{\pgfqpoint{0.393053in}{0.375000in}}{\pgfqpoint{6.356833in}{5.175000in}}%
\pgfusepath{clip}%
\pgfsetbuttcap%
\pgfsetroundjoin%
\pgfsetlinewidth{1.003750pt}%
\definecolor{currentstroke}{rgb}{0.827451,0.827451,0.827451}%
\pgfsetstrokecolor{currentstroke}%
\pgfsetdash{}{0pt}%
\pgfpathmoveto{\pgfqpoint{5.710861in}{0.339184in}}%
\pgfpathcurveto{\pgfqpoint{5.721911in}{0.339184in}}{\pgfqpoint{5.732510in}{0.343574in}}{\pgfqpoint{5.740323in}{0.351388in}}%
\pgfpathcurveto{\pgfqpoint{5.748137in}{0.359201in}}{\pgfqpoint{5.752527in}{0.369800in}}{\pgfqpoint{5.752527in}{0.380851in}}%
\pgfpathcurveto{\pgfqpoint{5.752527in}{0.391901in}}{\pgfqpoint{5.748137in}{0.402500in}}{\pgfqpoint{5.740323in}{0.410313in}}%
\pgfpathcurveto{\pgfqpoint{5.732510in}{0.418127in}}{\pgfqpoint{5.721911in}{0.422517in}}{\pgfqpoint{5.710861in}{0.422517in}}%
\pgfpathcurveto{\pgfqpoint{5.699810in}{0.422517in}}{\pgfqpoint{5.689211in}{0.418127in}}{\pgfqpoint{5.681398in}{0.410313in}}%
\pgfpathcurveto{\pgfqpoint{5.673584in}{0.402500in}}{\pgfqpoint{5.669194in}{0.391901in}}{\pgfqpoint{5.669194in}{0.380851in}}%
\pgfpathcurveto{\pgfqpoint{5.669194in}{0.369800in}}{\pgfqpoint{5.673584in}{0.359201in}}{\pgfqpoint{5.681398in}{0.351388in}}%
\pgfpathcurveto{\pgfqpoint{5.689211in}{0.343574in}}{\pgfqpoint{5.699810in}{0.339184in}}{\pgfqpoint{5.710861in}{0.339184in}}%
\pgfusepath{stroke}%
\end{pgfscope}%
\begin{pgfscope}%
\pgfpathrectangle{\pgfqpoint{0.393053in}{0.375000in}}{\pgfqpoint{6.356833in}{5.175000in}}%
\pgfusepath{clip}%
\pgfsetbuttcap%
\pgfsetroundjoin%
\pgfsetlinewidth{1.003750pt}%
\definecolor{currentstroke}{rgb}{0.827451,0.827451,0.827451}%
\pgfsetstrokecolor{currentstroke}%
\pgfsetdash{}{0pt}%
\pgfpathmoveto{\pgfqpoint{5.339131in}{0.342839in}}%
\pgfpathcurveto{\pgfqpoint{5.350181in}{0.342839in}}{\pgfqpoint{5.360780in}{0.347229in}}{\pgfqpoint{5.368594in}{0.355043in}}%
\pgfpathcurveto{\pgfqpoint{5.376407in}{0.362857in}}{\pgfqpoint{5.380798in}{0.373456in}}{\pgfqpoint{5.380798in}{0.384506in}}%
\pgfpathcurveto{\pgfqpoint{5.380798in}{0.395556in}}{\pgfqpoint{5.376407in}{0.406155in}}{\pgfqpoint{5.368594in}{0.413968in}}%
\pgfpathcurveto{\pgfqpoint{5.360780in}{0.421782in}}{\pgfqpoint{5.350181in}{0.426172in}}{\pgfqpoint{5.339131in}{0.426172in}}%
\pgfpathcurveto{\pgfqpoint{5.328081in}{0.426172in}}{\pgfqpoint{5.317482in}{0.421782in}}{\pgfqpoint{5.309668in}{0.413968in}}%
\pgfpathcurveto{\pgfqpoint{5.301855in}{0.406155in}}{\pgfqpoint{5.297464in}{0.395556in}}{\pgfqpoint{5.297464in}{0.384506in}}%
\pgfpathcurveto{\pgfqpoint{5.297464in}{0.373456in}}{\pgfqpoint{5.301855in}{0.362857in}}{\pgfqpoint{5.309668in}{0.355043in}}%
\pgfpathcurveto{\pgfqpoint{5.317482in}{0.347229in}}{\pgfqpoint{5.328081in}{0.342839in}}{\pgfqpoint{5.339131in}{0.342839in}}%
\pgfusepath{stroke}%
\end{pgfscope}%
\begin{pgfscope}%
\pgfpathrectangle{\pgfqpoint{0.393053in}{0.375000in}}{\pgfqpoint{6.356833in}{5.175000in}}%
\pgfusepath{clip}%
\pgfsetbuttcap%
\pgfsetroundjoin%
\pgfsetlinewidth{1.003750pt}%
\definecolor{currentstroke}{rgb}{0.827451,0.827451,0.827451}%
\pgfsetstrokecolor{currentstroke}%
\pgfsetdash{}{0pt}%
\pgfpathmoveto{\pgfqpoint{4.655370in}{0.409439in}}%
\pgfpathcurveto{\pgfqpoint{4.666420in}{0.409439in}}{\pgfqpoint{4.677019in}{0.413829in}}{\pgfqpoint{4.684833in}{0.421642in}}%
\pgfpathcurveto{\pgfqpoint{4.692646in}{0.429456in}}{\pgfqpoint{4.697036in}{0.440055in}}{\pgfqpoint{4.697036in}{0.451105in}}%
\pgfpathcurveto{\pgfqpoint{4.697036in}{0.462155in}}{\pgfqpoint{4.692646in}{0.472754in}}{\pgfqpoint{4.684833in}{0.480568in}}%
\pgfpathcurveto{\pgfqpoint{4.677019in}{0.488382in}}{\pgfqpoint{4.666420in}{0.492772in}}{\pgfqpoint{4.655370in}{0.492772in}}%
\pgfpathcurveto{\pgfqpoint{4.644320in}{0.492772in}}{\pgfqpoint{4.633721in}{0.488382in}}{\pgfqpoint{4.625907in}{0.480568in}}%
\pgfpathcurveto{\pgfqpoint{4.618093in}{0.472754in}}{\pgfqpoint{4.613703in}{0.462155in}}{\pgfqpoint{4.613703in}{0.451105in}}%
\pgfpathcurveto{\pgfqpoint{4.613703in}{0.440055in}}{\pgfqpoint{4.618093in}{0.429456in}}{\pgfqpoint{4.625907in}{0.421642in}}%
\pgfpathcurveto{\pgfqpoint{4.633721in}{0.413829in}}{\pgfqpoint{4.644320in}{0.409439in}}{\pgfqpoint{4.655370in}{0.409439in}}%
\pgfpathlineto{\pgfqpoint{4.655370in}{0.409439in}}%
\pgfpathclose%
\pgfusepath{stroke}%
\end{pgfscope}%
\begin{pgfscope}%
\pgfpathrectangle{\pgfqpoint{0.393053in}{0.375000in}}{\pgfqpoint{6.356833in}{5.175000in}}%
\pgfusepath{clip}%
\pgfsetbuttcap%
\pgfsetroundjoin%
\pgfsetlinewidth{1.003750pt}%
\definecolor{currentstroke}{rgb}{0.827451,0.827451,0.827451}%
\pgfsetstrokecolor{currentstroke}%
\pgfsetdash{}{0pt}%
\pgfpathmoveto{\pgfqpoint{0.399634in}{4.323598in}}%
\pgfpathcurveto{\pgfqpoint{0.410684in}{4.323598in}}{\pgfqpoint{0.421283in}{4.327988in}}{\pgfqpoint{0.429097in}{4.335802in}}%
\pgfpathcurveto{\pgfqpoint{0.436910in}{4.343616in}}{\pgfqpoint{0.441301in}{4.354215in}}{\pgfqpoint{0.441301in}{4.365265in}}%
\pgfpathcurveto{\pgfqpoint{0.441301in}{4.376315in}}{\pgfqpoint{0.436910in}{4.386914in}}{\pgfqpoint{0.429097in}{4.394727in}}%
\pgfpathcurveto{\pgfqpoint{0.421283in}{4.402541in}}{\pgfqpoint{0.410684in}{4.406931in}}{\pgfqpoint{0.399634in}{4.406931in}}%
\pgfpathcurveto{\pgfqpoint{0.388584in}{4.406931in}}{\pgfqpoint{0.377985in}{4.402541in}}{\pgfqpoint{0.370171in}{4.394727in}}%
\pgfpathcurveto{\pgfqpoint{0.362357in}{4.386914in}}{\pgfqpoint{0.357967in}{4.376315in}}{\pgfqpoint{0.357967in}{4.365265in}}%
\pgfpathcurveto{\pgfqpoint{0.357967in}{4.354215in}}{\pgfqpoint{0.362357in}{4.343616in}}{\pgfqpoint{0.370171in}{4.335802in}}%
\pgfpathcurveto{\pgfqpoint{0.377985in}{4.327988in}}{\pgfqpoint{0.388584in}{4.323598in}}{\pgfqpoint{0.399634in}{4.323598in}}%
\pgfpathlineto{\pgfqpoint{0.399634in}{4.323598in}}%
\pgfpathclose%
\pgfusepath{stroke}%
\end{pgfscope}%
\begin{pgfscope}%
\pgfpathrectangle{\pgfqpoint{0.393053in}{0.375000in}}{\pgfqpoint{6.356833in}{5.175000in}}%
\pgfusepath{clip}%
\pgfsetbuttcap%
\pgfsetroundjoin%
\pgfsetlinewidth{1.003750pt}%
\definecolor{currentstroke}{rgb}{0.827451,0.827451,0.827451}%
\pgfsetstrokecolor{currentstroke}%
\pgfsetdash{}{0pt}%
\pgfpathmoveto{\pgfqpoint{3.774127in}{0.547432in}}%
\pgfpathcurveto{\pgfqpoint{3.785177in}{0.547432in}}{\pgfqpoint{3.795776in}{0.551822in}}{\pgfqpoint{3.803589in}{0.559636in}}%
\pgfpathcurveto{\pgfqpoint{3.811403in}{0.567450in}}{\pgfqpoint{3.815793in}{0.578049in}}{\pgfqpoint{3.815793in}{0.589099in}}%
\pgfpathcurveto{\pgfqpoint{3.815793in}{0.600149in}}{\pgfqpoint{3.811403in}{0.610748in}}{\pgfqpoint{3.803589in}{0.618562in}}%
\pgfpathcurveto{\pgfqpoint{3.795776in}{0.626375in}}{\pgfqpoint{3.785177in}{0.630766in}}{\pgfqpoint{3.774127in}{0.630766in}}%
\pgfpathcurveto{\pgfqpoint{3.763077in}{0.630766in}}{\pgfqpoint{3.752478in}{0.626375in}}{\pgfqpoint{3.744664in}{0.618562in}}%
\pgfpathcurveto{\pgfqpoint{3.736850in}{0.610748in}}{\pgfqpoint{3.732460in}{0.600149in}}{\pgfqpoint{3.732460in}{0.589099in}}%
\pgfpathcurveto{\pgfqpoint{3.732460in}{0.578049in}}{\pgfqpoint{3.736850in}{0.567450in}}{\pgfqpoint{3.744664in}{0.559636in}}%
\pgfpathcurveto{\pgfqpoint{3.752478in}{0.551822in}}{\pgfqpoint{3.763077in}{0.547432in}}{\pgfqpoint{3.774127in}{0.547432in}}%
\pgfpathlineto{\pgfqpoint{3.774127in}{0.547432in}}%
\pgfpathclose%
\pgfusepath{stroke}%
\end{pgfscope}%
\begin{pgfscope}%
\pgfpathrectangle{\pgfqpoint{0.393053in}{0.375000in}}{\pgfqpoint{6.356833in}{5.175000in}}%
\pgfusepath{clip}%
\pgfsetbuttcap%
\pgfsetroundjoin%
\pgfsetlinewidth{1.003750pt}%
\definecolor{currentstroke}{rgb}{0.827451,0.827451,0.827451}%
\pgfsetstrokecolor{currentstroke}%
\pgfsetdash{}{0pt}%
\pgfpathmoveto{\pgfqpoint{5.142473in}{0.360415in}}%
\pgfpathcurveto{\pgfqpoint{5.153523in}{0.360415in}}{\pgfqpoint{5.164122in}{0.364805in}}{\pgfqpoint{5.171936in}{0.372618in}}%
\pgfpathcurveto{\pgfqpoint{5.179749in}{0.380432in}}{\pgfqpoint{5.184140in}{0.391031in}}{\pgfqpoint{5.184140in}{0.402081in}}%
\pgfpathcurveto{\pgfqpoint{5.184140in}{0.413131in}}{\pgfqpoint{5.179749in}{0.423730in}}{\pgfqpoint{5.171936in}{0.431544in}}%
\pgfpathcurveto{\pgfqpoint{5.164122in}{0.439358in}}{\pgfqpoint{5.153523in}{0.443748in}}{\pgfqpoint{5.142473in}{0.443748in}}%
\pgfpathcurveto{\pgfqpoint{5.131423in}{0.443748in}}{\pgfqpoint{5.120824in}{0.439358in}}{\pgfqpoint{5.113010in}{0.431544in}}%
\pgfpathcurveto{\pgfqpoint{5.105196in}{0.423730in}}{\pgfqpoint{5.100806in}{0.413131in}}{\pgfqpoint{5.100806in}{0.402081in}}%
\pgfpathcurveto{\pgfqpoint{5.100806in}{0.391031in}}{\pgfqpoint{5.105196in}{0.380432in}}{\pgfqpoint{5.113010in}{0.372618in}}%
\pgfpathcurveto{\pgfqpoint{5.120824in}{0.364805in}}{\pgfqpoint{5.131423in}{0.360415in}}{\pgfqpoint{5.142473in}{0.360415in}}%
\pgfusepath{stroke}%
\end{pgfscope}%
\begin{pgfscope}%
\pgfpathrectangle{\pgfqpoint{0.393053in}{0.375000in}}{\pgfqpoint{6.356833in}{5.175000in}}%
\pgfusepath{clip}%
\pgfsetbuttcap%
\pgfsetroundjoin%
\pgfsetlinewidth{1.003750pt}%
\definecolor{currentstroke}{rgb}{0.827451,0.827451,0.827451}%
\pgfsetstrokecolor{currentstroke}%
\pgfsetdash{}{0pt}%
\pgfpathmoveto{\pgfqpoint{0.743892in}{2.889747in}}%
\pgfpathcurveto{\pgfqpoint{0.754942in}{2.889747in}}{\pgfqpoint{0.765541in}{2.894137in}}{\pgfqpoint{0.773355in}{2.901951in}}%
\pgfpathcurveto{\pgfqpoint{0.781168in}{2.909765in}}{\pgfqpoint{0.785558in}{2.920364in}}{\pgfqpoint{0.785558in}{2.931414in}}%
\pgfpathcurveto{\pgfqpoint{0.785558in}{2.942464in}}{\pgfqpoint{0.781168in}{2.953063in}}{\pgfqpoint{0.773355in}{2.960877in}}%
\pgfpathcurveto{\pgfqpoint{0.765541in}{2.968690in}}{\pgfqpoint{0.754942in}{2.973081in}}{\pgfqpoint{0.743892in}{2.973081in}}%
\pgfpathcurveto{\pgfqpoint{0.732842in}{2.973081in}}{\pgfqpoint{0.722243in}{2.968690in}}{\pgfqpoint{0.714429in}{2.960877in}}%
\pgfpathcurveto{\pgfqpoint{0.706615in}{2.953063in}}{\pgfqpoint{0.702225in}{2.942464in}}{\pgfqpoint{0.702225in}{2.931414in}}%
\pgfpathcurveto{\pgfqpoint{0.702225in}{2.920364in}}{\pgfqpoint{0.706615in}{2.909765in}}{\pgfqpoint{0.714429in}{2.901951in}}%
\pgfpathcurveto{\pgfqpoint{0.722243in}{2.894137in}}{\pgfqpoint{0.732842in}{2.889747in}}{\pgfqpoint{0.743892in}{2.889747in}}%
\pgfpathlineto{\pgfqpoint{0.743892in}{2.889747in}}%
\pgfpathclose%
\pgfusepath{stroke}%
\end{pgfscope}%
\begin{pgfscope}%
\pgfpathrectangle{\pgfqpoint{0.393053in}{0.375000in}}{\pgfqpoint{6.356833in}{5.175000in}}%
\pgfusepath{clip}%
\pgfsetbuttcap%
\pgfsetroundjoin%
\pgfsetlinewidth{1.003750pt}%
\definecolor{currentstroke}{rgb}{0.827451,0.827451,0.827451}%
\pgfsetstrokecolor{currentstroke}%
\pgfsetdash{}{0pt}%
\pgfpathmoveto{\pgfqpoint{0.672197in}{3.023099in}}%
\pgfpathcurveto{\pgfqpoint{0.683247in}{3.023099in}}{\pgfqpoint{0.693846in}{3.027489in}}{\pgfqpoint{0.701660in}{3.035303in}}%
\pgfpathcurveto{\pgfqpoint{0.709473in}{3.043116in}}{\pgfqpoint{0.713863in}{3.053715in}}{\pgfqpoint{0.713863in}{3.064766in}}%
\pgfpathcurveto{\pgfqpoint{0.713863in}{3.075816in}}{\pgfqpoint{0.709473in}{3.086415in}}{\pgfqpoint{0.701660in}{3.094228in}}%
\pgfpathcurveto{\pgfqpoint{0.693846in}{3.102042in}}{\pgfqpoint{0.683247in}{3.106432in}}{\pgfqpoint{0.672197in}{3.106432in}}%
\pgfpathcurveto{\pgfqpoint{0.661147in}{3.106432in}}{\pgfqpoint{0.650548in}{3.102042in}}{\pgfqpoint{0.642734in}{3.094228in}}%
\pgfpathcurveto{\pgfqpoint{0.634920in}{3.086415in}}{\pgfqpoint{0.630530in}{3.075816in}}{\pgfqpoint{0.630530in}{3.064766in}}%
\pgfpathcurveto{\pgfqpoint{0.630530in}{3.053715in}}{\pgfqpoint{0.634920in}{3.043116in}}{\pgfqpoint{0.642734in}{3.035303in}}%
\pgfpathcurveto{\pgfqpoint{0.650548in}{3.027489in}}{\pgfqpoint{0.661147in}{3.023099in}}{\pgfqpoint{0.672197in}{3.023099in}}%
\pgfpathlineto{\pgfqpoint{0.672197in}{3.023099in}}%
\pgfpathclose%
\pgfusepath{stroke}%
\end{pgfscope}%
\begin{pgfscope}%
\pgfpathrectangle{\pgfqpoint{0.393053in}{0.375000in}}{\pgfqpoint{6.356833in}{5.175000in}}%
\pgfusepath{clip}%
\pgfsetbuttcap%
\pgfsetroundjoin%
\pgfsetlinewidth{1.003750pt}%
\definecolor{currentstroke}{rgb}{0.827451,0.827451,0.827451}%
\pgfsetstrokecolor{currentstroke}%
\pgfsetdash{}{0pt}%
\pgfpathmoveto{\pgfqpoint{0.418752in}{4.107793in}}%
\pgfpathcurveto{\pgfqpoint{0.429802in}{4.107793in}}{\pgfqpoint{0.440401in}{4.112183in}}{\pgfqpoint{0.448215in}{4.119996in}}%
\pgfpathcurveto{\pgfqpoint{0.456028in}{4.127810in}}{\pgfqpoint{0.460419in}{4.138409in}}{\pgfqpoint{0.460419in}{4.149459in}}%
\pgfpathcurveto{\pgfqpoint{0.460419in}{4.160509in}}{\pgfqpoint{0.456028in}{4.171108in}}{\pgfqpoint{0.448215in}{4.178922in}}%
\pgfpathcurveto{\pgfqpoint{0.440401in}{4.186736in}}{\pgfqpoint{0.429802in}{4.191126in}}{\pgfqpoint{0.418752in}{4.191126in}}%
\pgfpathcurveto{\pgfqpoint{0.407702in}{4.191126in}}{\pgfqpoint{0.397103in}{4.186736in}}{\pgfqpoint{0.389289in}{4.178922in}}%
\pgfpathcurveto{\pgfqpoint{0.381476in}{4.171108in}}{\pgfqpoint{0.377085in}{4.160509in}}{\pgfqpoint{0.377085in}{4.149459in}}%
\pgfpathcurveto{\pgfqpoint{0.377085in}{4.138409in}}{\pgfqpoint{0.381476in}{4.127810in}}{\pgfqpoint{0.389289in}{4.119996in}}%
\pgfpathcurveto{\pgfqpoint{0.397103in}{4.112183in}}{\pgfqpoint{0.407702in}{4.107793in}}{\pgfqpoint{0.418752in}{4.107793in}}%
\pgfpathlineto{\pgfqpoint{0.418752in}{4.107793in}}%
\pgfpathclose%
\pgfusepath{stroke}%
\end{pgfscope}%
\begin{pgfscope}%
\pgfpathrectangle{\pgfqpoint{0.393053in}{0.375000in}}{\pgfqpoint{6.356833in}{5.175000in}}%
\pgfusepath{clip}%
\pgfsetbuttcap%
\pgfsetroundjoin%
\pgfsetlinewidth{1.003750pt}%
\definecolor{currentstroke}{rgb}{0.827451,0.827451,0.827451}%
\pgfsetstrokecolor{currentstroke}%
\pgfsetdash{}{0pt}%
\pgfpathmoveto{\pgfqpoint{0.849130in}{2.701883in}}%
\pgfpathcurveto{\pgfqpoint{0.860180in}{2.701883in}}{\pgfqpoint{0.870779in}{2.706274in}}{\pgfqpoint{0.878592in}{2.714087in}}%
\pgfpathcurveto{\pgfqpoint{0.886406in}{2.721901in}}{\pgfqpoint{0.890796in}{2.732500in}}{\pgfqpoint{0.890796in}{2.743550in}}%
\pgfpathcurveto{\pgfqpoint{0.890796in}{2.754600in}}{\pgfqpoint{0.886406in}{2.765199in}}{\pgfqpoint{0.878592in}{2.773013in}}%
\pgfpathcurveto{\pgfqpoint{0.870779in}{2.780826in}}{\pgfqpoint{0.860180in}{2.785217in}}{\pgfqpoint{0.849130in}{2.785217in}}%
\pgfpathcurveto{\pgfqpoint{0.838079in}{2.785217in}}{\pgfqpoint{0.827480in}{2.780826in}}{\pgfqpoint{0.819667in}{2.773013in}}%
\pgfpathcurveto{\pgfqpoint{0.811853in}{2.765199in}}{\pgfqpoint{0.807463in}{2.754600in}}{\pgfqpoint{0.807463in}{2.743550in}}%
\pgfpathcurveto{\pgfqpoint{0.807463in}{2.732500in}}{\pgfqpoint{0.811853in}{2.721901in}}{\pgfqpoint{0.819667in}{2.714087in}}%
\pgfpathcurveto{\pgfqpoint{0.827480in}{2.706274in}}{\pgfqpoint{0.838079in}{2.701883in}}{\pgfqpoint{0.849130in}{2.701883in}}%
\pgfpathlineto{\pgfqpoint{0.849130in}{2.701883in}}%
\pgfpathclose%
\pgfusepath{stroke}%
\end{pgfscope}%
\begin{pgfscope}%
\pgfpathrectangle{\pgfqpoint{0.393053in}{0.375000in}}{\pgfqpoint{6.356833in}{5.175000in}}%
\pgfusepath{clip}%
\pgfsetbuttcap%
\pgfsetroundjoin%
\pgfsetlinewidth{1.003750pt}%
\definecolor{currentstroke}{rgb}{0.827451,0.827451,0.827451}%
\pgfsetstrokecolor{currentstroke}%
\pgfsetdash{}{0pt}%
\pgfpathmoveto{\pgfqpoint{4.898856in}{0.391247in}}%
\pgfpathcurveto{\pgfqpoint{4.909906in}{0.391247in}}{\pgfqpoint{4.920505in}{0.395637in}}{\pgfqpoint{4.928319in}{0.403450in}}%
\pgfpathcurveto{\pgfqpoint{4.936132in}{0.411264in}}{\pgfqpoint{4.940523in}{0.421863in}}{\pgfqpoint{4.940523in}{0.432913in}}%
\pgfpathcurveto{\pgfqpoint{4.940523in}{0.443963in}}{\pgfqpoint{4.936132in}{0.454562in}}{\pgfqpoint{4.928319in}{0.462376in}}%
\pgfpathcurveto{\pgfqpoint{4.920505in}{0.470190in}}{\pgfqpoint{4.909906in}{0.474580in}}{\pgfqpoint{4.898856in}{0.474580in}}%
\pgfpathcurveto{\pgfqpoint{4.887806in}{0.474580in}}{\pgfqpoint{4.877207in}{0.470190in}}{\pgfqpoint{4.869393in}{0.462376in}}%
\pgfpathcurveto{\pgfqpoint{4.861580in}{0.454562in}}{\pgfqpoint{4.857189in}{0.443963in}}{\pgfqpoint{4.857189in}{0.432913in}}%
\pgfpathcurveto{\pgfqpoint{4.857189in}{0.421863in}}{\pgfqpoint{4.861580in}{0.411264in}}{\pgfqpoint{4.869393in}{0.403450in}}%
\pgfpathcurveto{\pgfqpoint{4.877207in}{0.395637in}}{\pgfqpoint{4.887806in}{0.391247in}}{\pgfqpoint{4.898856in}{0.391247in}}%
\pgfpathlineto{\pgfqpoint{4.898856in}{0.391247in}}%
\pgfpathclose%
\pgfusepath{stroke}%
\end{pgfscope}%
\begin{pgfscope}%
\pgfpathrectangle{\pgfqpoint{0.393053in}{0.375000in}}{\pgfqpoint{6.356833in}{5.175000in}}%
\pgfusepath{clip}%
\pgfsetbuttcap%
\pgfsetroundjoin%
\pgfsetlinewidth{1.003750pt}%
\definecolor{currentstroke}{rgb}{0.827451,0.827451,0.827451}%
\pgfsetstrokecolor{currentstroke}%
\pgfsetdash{}{0pt}%
\pgfpathmoveto{\pgfqpoint{2.953536in}{0.827530in}}%
\pgfpathcurveto{\pgfqpoint{2.964586in}{0.827530in}}{\pgfqpoint{2.975185in}{0.831920in}}{\pgfqpoint{2.982999in}{0.839734in}}%
\pgfpathcurveto{\pgfqpoint{2.990812in}{0.847548in}}{\pgfqpoint{2.995203in}{0.858147in}}{\pgfqpoint{2.995203in}{0.869197in}}%
\pgfpathcurveto{\pgfqpoint{2.995203in}{0.880247in}}{\pgfqpoint{2.990812in}{0.890846in}}{\pgfqpoint{2.982999in}{0.898659in}}%
\pgfpathcurveto{\pgfqpoint{2.975185in}{0.906473in}}{\pgfqpoint{2.964586in}{0.910863in}}{\pgfqpoint{2.953536in}{0.910863in}}%
\pgfpathcurveto{\pgfqpoint{2.942486in}{0.910863in}}{\pgfqpoint{2.931887in}{0.906473in}}{\pgfqpoint{2.924073in}{0.898659in}}%
\pgfpathcurveto{\pgfqpoint{2.916260in}{0.890846in}}{\pgfqpoint{2.911869in}{0.880247in}}{\pgfqpoint{2.911869in}{0.869197in}}%
\pgfpathcurveto{\pgfqpoint{2.911869in}{0.858147in}}{\pgfqpoint{2.916260in}{0.847548in}}{\pgfqpoint{2.924073in}{0.839734in}}%
\pgfpathcurveto{\pgfqpoint{2.931887in}{0.831920in}}{\pgfqpoint{2.942486in}{0.827530in}}{\pgfqpoint{2.953536in}{0.827530in}}%
\pgfpathlineto{\pgfqpoint{2.953536in}{0.827530in}}%
\pgfpathclose%
\pgfusepath{stroke}%
\end{pgfscope}%
\begin{pgfscope}%
\pgfpathrectangle{\pgfqpoint{0.393053in}{0.375000in}}{\pgfqpoint{6.356833in}{5.175000in}}%
\pgfusepath{clip}%
\pgfsetbuttcap%
\pgfsetroundjoin%
\pgfsetlinewidth{1.003750pt}%
\definecolor{currentstroke}{rgb}{0.827451,0.827451,0.827451}%
\pgfsetstrokecolor{currentstroke}%
\pgfsetdash{}{0pt}%
\pgfpathmoveto{\pgfqpoint{3.649696in}{0.625044in}}%
\pgfpathcurveto{\pgfqpoint{3.660746in}{0.625044in}}{\pgfqpoint{3.671345in}{0.629434in}}{\pgfqpoint{3.679159in}{0.637248in}}%
\pgfpathcurveto{\pgfqpoint{3.686972in}{0.645061in}}{\pgfqpoint{3.691363in}{0.655660in}}{\pgfqpoint{3.691363in}{0.666710in}}%
\pgfpathcurveto{\pgfqpoint{3.691363in}{0.677761in}}{\pgfqpoint{3.686972in}{0.688360in}}{\pgfqpoint{3.679159in}{0.696173in}}%
\pgfpathcurveto{\pgfqpoint{3.671345in}{0.703987in}}{\pgfqpoint{3.660746in}{0.708377in}}{\pgfqpoint{3.649696in}{0.708377in}}%
\pgfpathcurveto{\pgfqpoint{3.638646in}{0.708377in}}{\pgfqpoint{3.628047in}{0.703987in}}{\pgfqpoint{3.620233in}{0.696173in}}%
\pgfpathcurveto{\pgfqpoint{3.612420in}{0.688360in}}{\pgfqpoint{3.608029in}{0.677761in}}{\pgfqpoint{3.608029in}{0.666710in}}%
\pgfpathcurveto{\pgfqpoint{3.608029in}{0.655660in}}{\pgfqpoint{3.612420in}{0.645061in}}{\pgfqpoint{3.620233in}{0.637248in}}%
\pgfpathcurveto{\pgfqpoint{3.628047in}{0.629434in}}{\pgfqpoint{3.638646in}{0.625044in}}{\pgfqpoint{3.649696in}{0.625044in}}%
\pgfpathlineto{\pgfqpoint{3.649696in}{0.625044in}}%
\pgfpathclose%
\pgfusepath{stroke}%
\end{pgfscope}%
\begin{pgfscope}%
\pgfpathrectangle{\pgfqpoint{0.393053in}{0.375000in}}{\pgfqpoint{6.356833in}{5.175000in}}%
\pgfusepath{clip}%
\pgfsetbuttcap%
\pgfsetroundjoin%
\pgfsetlinewidth{1.003750pt}%
\definecolor{currentstroke}{rgb}{0.827451,0.827451,0.827451}%
\pgfsetstrokecolor{currentstroke}%
\pgfsetdash{}{0pt}%
\pgfpathmoveto{\pgfqpoint{4.065017in}{0.476607in}}%
\pgfpathcurveto{\pgfqpoint{4.076067in}{0.476607in}}{\pgfqpoint{4.086666in}{0.480997in}}{\pgfqpoint{4.094480in}{0.488811in}}%
\pgfpathcurveto{\pgfqpoint{4.102293in}{0.496625in}}{\pgfqpoint{4.106684in}{0.507224in}}{\pgfqpoint{4.106684in}{0.518274in}}%
\pgfpathcurveto{\pgfqpoint{4.106684in}{0.529324in}}{\pgfqpoint{4.102293in}{0.539923in}}{\pgfqpoint{4.094480in}{0.547737in}}%
\pgfpathcurveto{\pgfqpoint{4.086666in}{0.555550in}}{\pgfqpoint{4.076067in}{0.559941in}}{\pgfqpoint{4.065017in}{0.559941in}}%
\pgfpathcurveto{\pgfqpoint{4.053967in}{0.559941in}}{\pgfqpoint{4.043368in}{0.555550in}}{\pgfqpoint{4.035554in}{0.547737in}}%
\pgfpathcurveto{\pgfqpoint{4.027741in}{0.539923in}}{\pgfqpoint{4.023350in}{0.529324in}}{\pgfqpoint{4.023350in}{0.518274in}}%
\pgfpathcurveto{\pgfqpoint{4.023350in}{0.507224in}}{\pgfqpoint{4.027741in}{0.496625in}}{\pgfqpoint{4.035554in}{0.488811in}}%
\pgfpathcurveto{\pgfqpoint{4.043368in}{0.480997in}}{\pgfqpoint{4.053967in}{0.476607in}}{\pgfqpoint{4.065017in}{0.476607in}}%
\pgfpathlineto{\pgfqpoint{4.065017in}{0.476607in}}%
\pgfpathclose%
\pgfusepath{stroke}%
\end{pgfscope}%
\begin{pgfscope}%
\pgfpathrectangle{\pgfqpoint{0.393053in}{0.375000in}}{\pgfqpoint{6.356833in}{5.175000in}}%
\pgfusepath{clip}%
\pgfsetbuttcap%
\pgfsetroundjoin%
\pgfsetlinewidth{1.003750pt}%
\definecolor{currentstroke}{rgb}{0.827451,0.827451,0.827451}%
\pgfsetstrokecolor{currentstroke}%
\pgfsetdash{}{0pt}%
\pgfpathmoveto{\pgfqpoint{1.614675in}{1.708400in}}%
\pgfpathcurveto{\pgfqpoint{1.625725in}{1.708400in}}{\pgfqpoint{1.636324in}{1.712790in}}{\pgfqpoint{1.644137in}{1.720604in}}%
\pgfpathcurveto{\pgfqpoint{1.651951in}{1.728417in}}{\pgfqpoint{1.656341in}{1.739016in}}{\pgfqpoint{1.656341in}{1.750066in}}%
\pgfpathcurveto{\pgfqpoint{1.656341in}{1.761117in}}{\pgfqpoint{1.651951in}{1.771716in}}{\pgfqpoint{1.644137in}{1.779529in}}%
\pgfpathcurveto{\pgfqpoint{1.636324in}{1.787343in}}{\pgfqpoint{1.625725in}{1.791733in}}{\pgfqpoint{1.614675in}{1.791733in}}%
\pgfpathcurveto{\pgfqpoint{1.603625in}{1.791733in}}{\pgfqpoint{1.593025in}{1.787343in}}{\pgfqpoint{1.585212in}{1.779529in}}%
\pgfpathcurveto{\pgfqpoint{1.577398in}{1.771716in}}{\pgfqpoint{1.573008in}{1.761117in}}{\pgfqpoint{1.573008in}{1.750066in}}%
\pgfpathcurveto{\pgfqpoint{1.573008in}{1.739016in}}{\pgfqpoint{1.577398in}{1.728417in}}{\pgfqpoint{1.585212in}{1.720604in}}%
\pgfpathcurveto{\pgfqpoint{1.593025in}{1.712790in}}{\pgfqpoint{1.603625in}{1.708400in}}{\pgfqpoint{1.614675in}{1.708400in}}%
\pgfpathlineto{\pgfqpoint{1.614675in}{1.708400in}}%
\pgfpathclose%
\pgfusepath{stroke}%
\end{pgfscope}%
\begin{pgfscope}%
\pgfpathrectangle{\pgfqpoint{0.393053in}{0.375000in}}{\pgfqpoint{6.356833in}{5.175000in}}%
\pgfusepath{clip}%
\pgfsetbuttcap%
\pgfsetroundjoin%
\pgfsetlinewidth{1.003750pt}%
\definecolor{currentstroke}{rgb}{0.827451,0.827451,0.827451}%
\pgfsetstrokecolor{currentstroke}%
\pgfsetdash{}{0pt}%
\pgfpathmoveto{\pgfqpoint{4.176160in}{0.456257in}}%
\pgfpathcurveto{\pgfqpoint{4.187210in}{0.456257in}}{\pgfqpoint{4.197809in}{0.460647in}}{\pgfqpoint{4.205623in}{0.468461in}}%
\pgfpathcurveto{\pgfqpoint{4.213436in}{0.476274in}}{\pgfqpoint{4.217827in}{0.486873in}}{\pgfqpoint{4.217827in}{0.497923in}}%
\pgfpathcurveto{\pgfqpoint{4.217827in}{0.508973in}}{\pgfqpoint{4.213436in}{0.519572in}}{\pgfqpoint{4.205623in}{0.527386in}}%
\pgfpathcurveto{\pgfqpoint{4.197809in}{0.535200in}}{\pgfqpoint{4.187210in}{0.539590in}}{\pgfqpoint{4.176160in}{0.539590in}}%
\pgfpathcurveto{\pgfqpoint{4.165110in}{0.539590in}}{\pgfqpoint{4.154511in}{0.535200in}}{\pgfqpoint{4.146697in}{0.527386in}}%
\pgfpathcurveto{\pgfqpoint{4.138883in}{0.519572in}}{\pgfqpoint{4.134493in}{0.508973in}}{\pgfqpoint{4.134493in}{0.497923in}}%
\pgfpathcurveto{\pgfqpoint{4.134493in}{0.486873in}}{\pgfqpoint{4.138883in}{0.476274in}}{\pgfqpoint{4.146697in}{0.468461in}}%
\pgfpathcurveto{\pgfqpoint{4.154511in}{0.460647in}}{\pgfqpoint{4.165110in}{0.456257in}}{\pgfqpoint{4.176160in}{0.456257in}}%
\pgfpathlineto{\pgfqpoint{4.176160in}{0.456257in}}%
\pgfpathclose%
\pgfusepath{stroke}%
\end{pgfscope}%
\begin{pgfscope}%
\pgfpathrectangle{\pgfqpoint{0.393053in}{0.375000in}}{\pgfqpoint{6.356833in}{5.175000in}}%
\pgfusepath{clip}%
\pgfsetbuttcap%
\pgfsetroundjoin%
\pgfsetlinewidth{1.003750pt}%
\definecolor{currentstroke}{rgb}{0.827451,0.827451,0.827451}%
\pgfsetstrokecolor{currentstroke}%
\pgfsetdash{}{0pt}%
\pgfpathmoveto{\pgfqpoint{2.862547in}{0.868774in}}%
\pgfpathcurveto{\pgfqpoint{2.873597in}{0.868774in}}{\pgfqpoint{2.884196in}{0.873164in}}{\pgfqpoint{2.892010in}{0.880977in}}%
\pgfpathcurveto{\pgfqpoint{2.899824in}{0.888791in}}{\pgfqpoint{2.904214in}{0.899390in}}{\pgfqpoint{2.904214in}{0.910440in}}%
\pgfpathcurveto{\pgfqpoint{2.904214in}{0.921490in}}{\pgfqpoint{2.899824in}{0.932089in}}{\pgfqpoint{2.892010in}{0.939903in}}%
\pgfpathcurveto{\pgfqpoint{2.884196in}{0.947717in}}{\pgfqpoint{2.873597in}{0.952107in}}{\pgfqpoint{2.862547in}{0.952107in}}%
\pgfpathcurveto{\pgfqpoint{2.851497in}{0.952107in}}{\pgfqpoint{2.840898in}{0.947717in}}{\pgfqpoint{2.833084in}{0.939903in}}%
\pgfpathcurveto{\pgfqpoint{2.825271in}{0.932089in}}{\pgfqpoint{2.820880in}{0.921490in}}{\pgfqpoint{2.820880in}{0.910440in}}%
\pgfpathcurveto{\pgfqpoint{2.820880in}{0.899390in}}{\pgfqpoint{2.825271in}{0.888791in}}{\pgfqpoint{2.833084in}{0.880977in}}%
\pgfpathcurveto{\pgfqpoint{2.840898in}{0.873164in}}{\pgfqpoint{2.851497in}{0.868774in}}{\pgfqpoint{2.862547in}{0.868774in}}%
\pgfpathlineto{\pgfqpoint{2.862547in}{0.868774in}}%
\pgfpathclose%
\pgfusepath{stroke}%
\end{pgfscope}%
\begin{pgfscope}%
\pgfpathrectangle{\pgfqpoint{0.393053in}{0.375000in}}{\pgfqpoint{6.356833in}{5.175000in}}%
\pgfusepath{clip}%
\pgfsetbuttcap%
\pgfsetroundjoin%
\pgfsetlinewidth{1.003750pt}%
\definecolor{currentstroke}{rgb}{0.827451,0.827451,0.827451}%
\pgfsetstrokecolor{currentstroke}%
\pgfsetdash{}{0pt}%
\pgfpathmoveto{\pgfqpoint{0.773700in}{2.823025in}}%
\pgfpathcurveto{\pgfqpoint{0.784750in}{2.823025in}}{\pgfqpoint{0.795350in}{2.827415in}}{\pgfqpoint{0.803163in}{2.835229in}}%
\pgfpathcurveto{\pgfqpoint{0.810977in}{2.843042in}}{\pgfqpoint{0.815367in}{2.853641in}}{\pgfqpoint{0.815367in}{2.864691in}}%
\pgfpathcurveto{\pgfqpoint{0.815367in}{2.875742in}}{\pgfqpoint{0.810977in}{2.886341in}}{\pgfqpoint{0.803163in}{2.894154in}}%
\pgfpathcurveto{\pgfqpoint{0.795350in}{2.901968in}}{\pgfqpoint{0.784750in}{2.906358in}}{\pgfqpoint{0.773700in}{2.906358in}}%
\pgfpathcurveto{\pgfqpoint{0.762650in}{2.906358in}}{\pgfqpoint{0.752051in}{2.901968in}}{\pgfqpoint{0.744238in}{2.894154in}}%
\pgfpathcurveto{\pgfqpoint{0.736424in}{2.886341in}}{\pgfqpoint{0.732034in}{2.875742in}}{\pgfqpoint{0.732034in}{2.864691in}}%
\pgfpathcurveto{\pgfqpoint{0.732034in}{2.853641in}}{\pgfqpoint{0.736424in}{2.843042in}}{\pgfqpoint{0.744238in}{2.835229in}}%
\pgfpathcurveto{\pgfqpoint{0.752051in}{2.827415in}}{\pgfqpoint{0.762650in}{2.823025in}}{\pgfqpoint{0.773700in}{2.823025in}}%
\pgfpathlineto{\pgfqpoint{0.773700in}{2.823025in}}%
\pgfpathclose%
\pgfusepath{stroke}%
\end{pgfscope}%
\begin{pgfscope}%
\pgfpathrectangle{\pgfqpoint{0.393053in}{0.375000in}}{\pgfqpoint{6.356833in}{5.175000in}}%
\pgfusepath{clip}%
\pgfsetbuttcap%
\pgfsetroundjoin%
\pgfsetlinewidth{1.003750pt}%
\definecolor{currentstroke}{rgb}{0.827451,0.827451,0.827451}%
\pgfsetstrokecolor{currentstroke}%
\pgfsetdash{}{0pt}%
\pgfpathmoveto{\pgfqpoint{0.580211in}{3.378166in}}%
\pgfpathcurveto{\pgfqpoint{0.591261in}{3.378166in}}{\pgfqpoint{0.601860in}{3.382556in}}{\pgfqpoint{0.609674in}{3.390370in}}%
\pgfpathcurveto{\pgfqpoint{0.617487in}{3.398183in}}{\pgfqpoint{0.621878in}{3.408782in}}{\pgfqpoint{0.621878in}{3.419833in}}%
\pgfpathcurveto{\pgfqpoint{0.621878in}{3.430883in}}{\pgfqpoint{0.617487in}{3.441482in}}{\pgfqpoint{0.609674in}{3.449295in}}%
\pgfpathcurveto{\pgfqpoint{0.601860in}{3.457109in}}{\pgfqpoint{0.591261in}{3.461499in}}{\pgfqpoint{0.580211in}{3.461499in}}%
\pgfpathcurveto{\pgfqpoint{0.569161in}{3.461499in}}{\pgfqpoint{0.558562in}{3.457109in}}{\pgfqpoint{0.550748in}{3.449295in}}%
\pgfpathcurveto{\pgfqpoint{0.542935in}{3.441482in}}{\pgfqpoint{0.538544in}{3.430883in}}{\pgfqpoint{0.538544in}{3.419833in}}%
\pgfpathcurveto{\pgfqpoint{0.538544in}{3.408782in}}{\pgfqpoint{0.542935in}{3.398183in}}{\pgfqpoint{0.550748in}{3.390370in}}%
\pgfpathcurveto{\pgfqpoint{0.558562in}{3.382556in}}{\pgfqpoint{0.569161in}{3.378166in}}{\pgfqpoint{0.580211in}{3.378166in}}%
\pgfpathlineto{\pgfqpoint{0.580211in}{3.378166in}}%
\pgfpathclose%
\pgfusepath{stroke}%
\end{pgfscope}%
\begin{pgfscope}%
\pgfpathrectangle{\pgfqpoint{0.393053in}{0.375000in}}{\pgfqpoint{6.356833in}{5.175000in}}%
\pgfusepath{clip}%
\pgfsetbuttcap%
\pgfsetroundjoin%
\pgfsetlinewidth{1.003750pt}%
\definecolor{currentstroke}{rgb}{0.827451,0.827451,0.827451}%
\pgfsetstrokecolor{currentstroke}%
\pgfsetdash{}{0pt}%
\pgfpathmoveto{\pgfqpoint{5.017789in}{0.367635in}}%
\pgfpathcurveto{\pgfqpoint{5.028839in}{0.367635in}}{\pgfqpoint{5.039438in}{0.372025in}}{\pgfqpoint{5.047252in}{0.379839in}}%
\pgfpathcurveto{\pgfqpoint{5.055065in}{0.387652in}}{\pgfqpoint{5.059456in}{0.398251in}}{\pgfqpoint{5.059456in}{0.409302in}}%
\pgfpathcurveto{\pgfqpoint{5.059456in}{0.420352in}}{\pgfqpoint{5.055065in}{0.430951in}}{\pgfqpoint{5.047252in}{0.438764in}}%
\pgfpathcurveto{\pgfqpoint{5.039438in}{0.446578in}}{\pgfqpoint{5.028839in}{0.450968in}}{\pgfqpoint{5.017789in}{0.450968in}}%
\pgfpathcurveto{\pgfqpoint{5.006739in}{0.450968in}}{\pgfqpoint{4.996140in}{0.446578in}}{\pgfqpoint{4.988326in}{0.438764in}}%
\pgfpathcurveto{\pgfqpoint{4.980512in}{0.430951in}}{\pgfqpoint{4.976122in}{0.420352in}}{\pgfqpoint{4.976122in}{0.409302in}}%
\pgfpathcurveto{\pgfqpoint{4.976122in}{0.398251in}}{\pgfqpoint{4.980512in}{0.387652in}}{\pgfqpoint{4.988326in}{0.379839in}}%
\pgfpathcurveto{\pgfqpoint{4.996140in}{0.372025in}}{\pgfqpoint{5.006739in}{0.367635in}}{\pgfqpoint{5.017789in}{0.367635in}}%
\pgfusepath{stroke}%
\end{pgfscope}%
\begin{pgfscope}%
\pgfpathrectangle{\pgfqpoint{0.393053in}{0.375000in}}{\pgfqpoint{6.356833in}{5.175000in}}%
\pgfusepath{clip}%
\pgfsetbuttcap%
\pgfsetroundjoin%
\pgfsetlinewidth{1.003750pt}%
\definecolor{currentstroke}{rgb}{0.827451,0.827451,0.827451}%
\pgfsetstrokecolor{currentstroke}%
\pgfsetdash{}{0pt}%
\pgfpathmoveto{\pgfqpoint{0.583953in}{3.273995in}}%
\pgfpathcurveto{\pgfqpoint{0.595003in}{3.273995in}}{\pgfqpoint{0.605602in}{3.278385in}}{\pgfqpoint{0.613416in}{3.286198in}}%
\pgfpathcurveto{\pgfqpoint{0.621229in}{3.294012in}}{\pgfqpoint{0.625620in}{3.304611in}}{\pgfqpoint{0.625620in}{3.315661in}}%
\pgfpathcurveto{\pgfqpoint{0.625620in}{3.326711in}}{\pgfqpoint{0.621229in}{3.337310in}}{\pgfqpoint{0.613416in}{3.345124in}}%
\pgfpathcurveto{\pgfqpoint{0.605602in}{3.352938in}}{\pgfqpoint{0.595003in}{3.357328in}}{\pgfqpoint{0.583953in}{3.357328in}}%
\pgfpathcurveto{\pgfqpoint{0.572903in}{3.357328in}}{\pgfqpoint{0.562304in}{3.352938in}}{\pgfqpoint{0.554490in}{3.345124in}}%
\pgfpathcurveto{\pgfqpoint{0.546677in}{3.337310in}}{\pgfqpoint{0.542286in}{3.326711in}}{\pgfqpoint{0.542286in}{3.315661in}}%
\pgfpathcurveto{\pgfqpoint{0.542286in}{3.304611in}}{\pgfqpoint{0.546677in}{3.294012in}}{\pgfqpoint{0.554490in}{3.286198in}}%
\pgfpathcurveto{\pgfqpoint{0.562304in}{3.278385in}}{\pgfqpoint{0.572903in}{3.273995in}}{\pgfqpoint{0.583953in}{3.273995in}}%
\pgfpathlineto{\pgfqpoint{0.583953in}{3.273995in}}%
\pgfpathclose%
\pgfusepath{stroke}%
\end{pgfscope}%
\begin{pgfscope}%
\pgfpathrectangle{\pgfqpoint{0.393053in}{0.375000in}}{\pgfqpoint{6.356833in}{5.175000in}}%
\pgfusepath{clip}%
\pgfsetbuttcap%
\pgfsetroundjoin%
\pgfsetlinewidth{1.003750pt}%
\definecolor{currentstroke}{rgb}{0.827451,0.827451,0.827451}%
\pgfsetstrokecolor{currentstroke}%
\pgfsetdash{}{0pt}%
\pgfpathmoveto{\pgfqpoint{0.905111in}{2.683698in}}%
\pgfpathcurveto{\pgfqpoint{0.916161in}{2.683698in}}{\pgfqpoint{0.926760in}{2.688088in}}{\pgfqpoint{0.934574in}{2.695902in}}%
\pgfpathcurveto{\pgfqpoint{0.942388in}{2.703715in}}{\pgfqpoint{0.946778in}{2.714314in}}{\pgfqpoint{0.946778in}{2.725364in}}%
\pgfpathcurveto{\pgfqpoint{0.946778in}{2.736415in}}{\pgfqpoint{0.942388in}{2.747014in}}{\pgfqpoint{0.934574in}{2.754827in}}%
\pgfpathcurveto{\pgfqpoint{0.926760in}{2.762641in}}{\pgfqpoint{0.916161in}{2.767031in}}{\pgfqpoint{0.905111in}{2.767031in}}%
\pgfpathcurveto{\pgfqpoint{0.894061in}{2.767031in}}{\pgfqpoint{0.883462in}{2.762641in}}{\pgfqpoint{0.875649in}{2.754827in}}%
\pgfpathcurveto{\pgfqpoint{0.867835in}{2.747014in}}{\pgfqpoint{0.863445in}{2.736415in}}{\pgfqpoint{0.863445in}{2.725364in}}%
\pgfpathcurveto{\pgfqpoint{0.863445in}{2.714314in}}{\pgfqpoint{0.867835in}{2.703715in}}{\pgfqpoint{0.875649in}{2.695902in}}%
\pgfpathcurveto{\pgfqpoint{0.883462in}{2.688088in}}{\pgfqpoint{0.894061in}{2.683698in}}{\pgfqpoint{0.905111in}{2.683698in}}%
\pgfpathlineto{\pgfqpoint{0.905111in}{2.683698in}}%
\pgfpathclose%
\pgfusepath{stroke}%
\end{pgfscope}%
\begin{pgfscope}%
\pgfpathrectangle{\pgfqpoint{0.393053in}{0.375000in}}{\pgfqpoint{6.356833in}{5.175000in}}%
\pgfusepath{clip}%
\pgfsetbuttcap%
\pgfsetroundjoin%
\pgfsetlinewidth{1.003750pt}%
\definecolor{currentstroke}{rgb}{0.827451,0.827451,0.827451}%
\pgfsetstrokecolor{currentstroke}%
\pgfsetdash{}{0pt}%
\pgfpathmoveto{\pgfqpoint{3.954008in}{0.515189in}}%
\pgfpathcurveto{\pgfqpoint{3.965058in}{0.515189in}}{\pgfqpoint{3.975657in}{0.519579in}}{\pgfqpoint{3.983471in}{0.527393in}}%
\pgfpathcurveto{\pgfqpoint{3.991284in}{0.535206in}}{\pgfqpoint{3.995675in}{0.545805in}}{\pgfqpoint{3.995675in}{0.556855in}}%
\pgfpathcurveto{\pgfqpoint{3.995675in}{0.567905in}}{\pgfqpoint{3.991284in}{0.578504in}}{\pgfqpoint{3.983471in}{0.586318in}}%
\pgfpathcurveto{\pgfqpoint{3.975657in}{0.594132in}}{\pgfqpoint{3.965058in}{0.598522in}}{\pgfqpoint{3.954008in}{0.598522in}}%
\pgfpathcurveto{\pgfqpoint{3.942958in}{0.598522in}}{\pgfqpoint{3.932359in}{0.594132in}}{\pgfqpoint{3.924545in}{0.586318in}}%
\pgfpathcurveto{\pgfqpoint{3.916732in}{0.578504in}}{\pgfqpoint{3.912341in}{0.567905in}}{\pgfqpoint{3.912341in}{0.556855in}}%
\pgfpathcurveto{\pgfqpoint{3.912341in}{0.545805in}}{\pgfqpoint{3.916732in}{0.535206in}}{\pgfqpoint{3.924545in}{0.527393in}}%
\pgfpathcurveto{\pgfqpoint{3.932359in}{0.519579in}}{\pgfqpoint{3.942958in}{0.515189in}}{\pgfqpoint{3.954008in}{0.515189in}}%
\pgfpathlineto{\pgfqpoint{3.954008in}{0.515189in}}%
\pgfpathclose%
\pgfusepath{stroke}%
\end{pgfscope}%
\begin{pgfscope}%
\pgfpathrectangle{\pgfqpoint{0.393053in}{0.375000in}}{\pgfqpoint{6.356833in}{5.175000in}}%
\pgfusepath{clip}%
\pgfsetbuttcap%
\pgfsetroundjoin%
\pgfsetlinewidth{1.003750pt}%
\definecolor{currentstroke}{rgb}{0.827451,0.827451,0.827451}%
\pgfsetstrokecolor{currentstroke}%
\pgfsetdash{}{0pt}%
\pgfpathmoveto{\pgfqpoint{3.677938in}{0.575928in}}%
\pgfpathcurveto{\pgfqpoint{3.688988in}{0.575928in}}{\pgfqpoint{3.699587in}{0.580318in}}{\pgfqpoint{3.707401in}{0.588132in}}%
\pgfpathcurveto{\pgfqpoint{3.715214in}{0.595945in}}{\pgfqpoint{3.719604in}{0.606544in}}{\pgfqpoint{3.719604in}{0.617595in}}%
\pgfpathcurveto{\pgfqpoint{3.719604in}{0.628645in}}{\pgfqpoint{3.715214in}{0.639244in}}{\pgfqpoint{3.707401in}{0.647057in}}%
\pgfpathcurveto{\pgfqpoint{3.699587in}{0.654871in}}{\pgfqpoint{3.688988in}{0.659261in}}{\pgfqpoint{3.677938in}{0.659261in}}%
\pgfpathcurveto{\pgfqpoint{3.666888in}{0.659261in}}{\pgfqpoint{3.656289in}{0.654871in}}{\pgfqpoint{3.648475in}{0.647057in}}%
\pgfpathcurveto{\pgfqpoint{3.640661in}{0.639244in}}{\pgfqpoint{3.636271in}{0.628645in}}{\pgfqpoint{3.636271in}{0.617595in}}%
\pgfpathcurveto{\pgfqpoint{3.636271in}{0.606544in}}{\pgfqpoint{3.640661in}{0.595945in}}{\pgfqpoint{3.648475in}{0.588132in}}%
\pgfpathcurveto{\pgfqpoint{3.656289in}{0.580318in}}{\pgfqpoint{3.666888in}{0.575928in}}{\pgfqpoint{3.677938in}{0.575928in}}%
\pgfpathlineto{\pgfqpoint{3.677938in}{0.575928in}}%
\pgfpathclose%
\pgfusepath{stroke}%
\end{pgfscope}%
\begin{pgfscope}%
\pgfpathrectangle{\pgfqpoint{0.393053in}{0.375000in}}{\pgfqpoint{6.356833in}{5.175000in}}%
\pgfusepath{clip}%
\pgfsetbuttcap%
\pgfsetroundjoin%
\pgfsetlinewidth{1.003750pt}%
\definecolor{currentstroke}{rgb}{0.827451,0.827451,0.827451}%
\pgfsetstrokecolor{currentstroke}%
\pgfsetdash{}{0pt}%
\pgfpathmoveto{\pgfqpoint{2.230001in}{1.209659in}}%
\pgfpathcurveto{\pgfqpoint{2.241051in}{1.209659in}}{\pgfqpoint{2.251650in}{1.214049in}}{\pgfqpoint{2.259463in}{1.221862in}}%
\pgfpathcurveto{\pgfqpoint{2.267277in}{1.229676in}}{\pgfqpoint{2.271667in}{1.240275in}}{\pgfqpoint{2.271667in}{1.251325in}}%
\pgfpathcurveto{\pgfqpoint{2.271667in}{1.262375in}}{\pgfqpoint{2.267277in}{1.272974in}}{\pgfqpoint{2.259463in}{1.280788in}}%
\pgfpathcurveto{\pgfqpoint{2.251650in}{1.288602in}}{\pgfqpoint{2.241051in}{1.292992in}}{\pgfqpoint{2.230001in}{1.292992in}}%
\pgfpathcurveto{\pgfqpoint{2.218950in}{1.292992in}}{\pgfqpoint{2.208351in}{1.288602in}}{\pgfqpoint{2.200538in}{1.280788in}}%
\pgfpathcurveto{\pgfqpoint{2.192724in}{1.272974in}}{\pgfqpoint{2.188334in}{1.262375in}}{\pgfqpoint{2.188334in}{1.251325in}}%
\pgfpathcurveto{\pgfqpoint{2.188334in}{1.240275in}}{\pgfqpoint{2.192724in}{1.229676in}}{\pgfqpoint{2.200538in}{1.221862in}}%
\pgfpathcurveto{\pgfqpoint{2.208351in}{1.214049in}}{\pgfqpoint{2.218950in}{1.209659in}}{\pgfqpoint{2.230001in}{1.209659in}}%
\pgfpathlineto{\pgfqpoint{2.230001in}{1.209659in}}%
\pgfpathclose%
\pgfusepath{stroke}%
\end{pgfscope}%
\begin{pgfscope}%
\pgfpathrectangle{\pgfqpoint{0.393053in}{0.375000in}}{\pgfqpoint{6.356833in}{5.175000in}}%
\pgfusepath{clip}%
\pgfsetbuttcap%
\pgfsetroundjoin%
\pgfsetlinewidth{1.003750pt}%
\definecolor{currentstroke}{rgb}{0.827451,0.827451,0.827451}%
\pgfsetstrokecolor{currentstroke}%
\pgfsetdash{}{0pt}%
\pgfpathmoveto{\pgfqpoint{1.019447in}{2.455876in}}%
\pgfpathcurveto{\pgfqpoint{1.030497in}{2.455876in}}{\pgfqpoint{1.041096in}{2.460267in}}{\pgfqpoint{1.048910in}{2.468080in}}%
\pgfpathcurveto{\pgfqpoint{1.056723in}{2.475894in}}{\pgfqpoint{1.061113in}{2.486493in}}{\pgfqpoint{1.061113in}{2.497543in}}%
\pgfpathcurveto{\pgfqpoint{1.061113in}{2.508593in}}{\pgfqpoint{1.056723in}{2.519192in}}{\pgfqpoint{1.048910in}{2.527006in}}%
\pgfpathcurveto{\pgfqpoint{1.041096in}{2.534819in}}{\pgfqpoint{1.030497in}{2.539210in}}{\pgfqpoint{1.019447in}{2.539210in}}%
\pgfpathcurveto{\pgfqpoint{1.008397in}{2.539210in}}{\pgfqpoint{0.997798in}{2.534819in}}{\pgfqpoint{0.989984in}{2.527006in}}%
\pgfpathcurveto{\pgfqpoint{0.982170in}{2.519192in}}{\pgfqpoint{0.977780in}{2.508593in}}{\pgfqpoint{0.977780in}{2.497543in}}%
\pgfpathcurveto{\pgfqpoint{0.977780in}{2.486493in}}{\pgfqpoint{0.982170in}{2.475894in}}{\pgfqpoint{0.989984in}{2.468080in}}%
\pgfpathcurveto{\pgfqpoint{0.997798in}{2.460267in}}{\pgfqpoint{1.008397in}{2.455876in}}{\pgfqpoint{1.019447in}{2.455876in}}%
\pgfpathlineto{\pgfqpoint{1.019447in}{2.455876in}}%
\pgfpathclose%
\pgfusepath{stroke}%
\end{pgfscope}%
\begin{pgfscope}%
\pgfpathrectangle{\pgfqpoint{0.393053in}{0.375000in}}{\pgfqpoint{6.356833in}{5.175000in}}%
\pgfusepath{clip}%
\pgfsetbuttcap%
\pgfsetroundjoin%
\pgfsetlinewidth{1.003750pt}%
\definecolor{currentstroke}{rgb}{0.827451,0.827451,0.827451}%
\pgfsetstrokecolor{currentstroke}%
\pgfsetdash{}{0pt}%
\pgfpathmoveto{\pgfqpoint{0.911257in}{2.570368in}}%
\pgfpathcurveto{\pgfqpoint{0.922307in}{2.570368in}}{\pgfqpoint{0.932906in}{2.574759in}}{\pgfqpoint{0.940720in}{2.582572in}}%
\pgfpathcurveto{\pgfqpoint{0.948533in}{2.590386in}}{\pgfqpoint{0.952923in}{2.600985in}}{\pgfqpoint{0.952923in}{2.612035in}}%
\pgfpathcurveto{\pgfqpoint{0.952923in}{2.623085in}}{\pgfqpoint{0.948533in}{2.633684in}}{\pgfqpoint{0.940720in}{2.641498in}}%
\pgfpathcurveto{\pgfqpoint{0.932906in}{2.649311in}}{\pgfqpoint{0.922307in}{2.653702in}}{\pgfqpoint{0.911257in}{2.653702in}}%
\pgfpathcurveto{\pgfqpoint{0.900207in}{2.653702in}}{\pgfqpoint{0.889608in}{2.649311in}}{\pgfqpoint{0.881794in}{2.641498in}}%
\pgfpathcurveto{\pgfqpoint{0.873980in}{2.633684in}}{\pgfqpoint{0.869590in}{2.623085in}}{\pgfqpoint{0.869590in}{2.612035in}}%
\pgfpathcurveto{\pgfqpoint{0.869590in}{2.600985in}}{\pgfqpoint{0.873980in}{2.590386in}}{\pgfqpoint{0.881794in}{2.582572in}}%
\pgfpathcurveto{\pgfqpoint{0.889608in}{2.574759in}}{\pgfqpoint{0.900207in}{2.570368in}}{\pgfqpoint{0.911257in}{2.570368in}}%
\pgfpathlineto{\pgfqpoint{0.911257in}{2.570368in}}%
\pgfpathclose%
\pgfusepath{stroke}%
\end{pgfscope}%
\begin{pgfscope}%
\pgfpathrectangle{\pgfqpoint{0.393053in}{0.375000in}}{\pgfqpoint{6.356833in}{5.175000in}}%
\pgfusepath{clip}%
\pgfsetbuttcap%
\pgfsetroundjoin%
\pgfsetlinewidth{1.003750pt}%
\definecolor{currentstroke}{rgb}{0.827451,0.827451,0.827451}%
\pgfsetstrokecolor{currentstroke}%
\pgfsetdash{}{0pt}%
\pgfpathmoveto{\pgfqpoint{3.053820in}{0.786671in}}%
\pgfpathcurveto{\pgfqpoint{3.064870in}{0.786671in}}{\pgfqpoint{3.075469in}{0.791061in}}{\pgfqpoint{3.083282in}{0.798875in}}%
\pgfpathcurveto{\pgfqpoint{3.091096in}{0.806688in}}{\pgfqpoint{3.095486in}{0.817287in}}{\pgfqpoint{3.095486in}{0.828338in}}%
\pgfpathcurveto{\pgfqpoint{3.095486in}{0.839388in}}{\pgfqpoint{3.091096in}{0.849987in}}{\pgfqpoint{3.083282in}{0.857800in}}%
\pgfpathcurveto{\pgfqpoint{3.075469in}{0.865614in}}{\pgfqpoint{3.064870in}{0.870004in}}{\pgfqpoint{3.053820in}{0.870004in}}%
\pgfpathcurveto{\pgfqpoint{3.042770in}{0.870004in}}{\pgfqpoint{3.032171in}{0.865614in}}{\pgfqpoint{3.024357in}{0.857800in}}%
\pgfpathcurveto{\pgfqpoint{3.016543in}{0.849987in}}{\pgfqpoint{3.012153in}{0.839388in}}{\pgfqpoint{3.012153in}{0.828338in}}%
\pgfpathcurveto{\pgfqpoint{3.012153in}{0.817287in}}{\pgfqpoint{3.016543in}{0.806688in}}{\pgfqpoint{3.024357in}{0.798875in}}%
\pgfpathcurveto{\pgfqpoint{3.032171in}{0.791061in}}{\pgfqpoint{3.042770in}{0.786671in}}{\pgfqpoint{3.053820in}{0.786671in}}%
\pgfpathlineto{\pgfqpoint{3.053820in}{0.786671in}}%
\pgfpathclose%
\pgfusepath{stroke}%
\end{pgfscope}%
\begin{pgfscope}%
\pgfpathrectangle{\pgfqpoint{0.393053in}{0.375000in}}{\pgfqpoint{6.356833in}{5.175000in}}%
\pgfusepath{clip}%
\pgfsetbuttcap%
\pgfsetroundjoin%
\pgfsetlinewidth{1.003750pt}%
\definecolor{currentstroke}{rgb}{0.827451,0.827451,0.827451}%
\pgfsetstrokecolor{currentstroke}%
\pgfsetdash{}{0pt}%
\pgfpathmoveto{\pgfqpoint{1.040104in}{2.359499in}}%
\pgfpathcurveto{\pgfqpoint{1.051154in}{2.359499in}}{\pgfqpoint{1.061753in}{2.363889in}}{\pgfqpoint{1.069567in}{2.371703in}}%
\pgfpathcurveto{\pgfqpoint{1.077381in}{2.379516in}}{\pgfqpoint{1.081771in}{2.390115in}}{\pgfqpoint{1.081771in}{2.401165in}}%
\pgfpathcurveto{\pgfqpoint{1.081771in}{2.412215in}}{\pgfqpoint{1.077381in}{2.422814in}}{\pgfqpoint{1.069567in}{2.430628in}}%
\pgfpathcurveto{\pgfqpoint{1.061753in}{2.438442in}}{\pgfqpoint{1.051154in}{2.442832in}}{\pgfqpoint{1.040104in}{2.442832in}}%
\pgfpathcurveto{\pgfqpoint{1.029054in}{2.442832in}}{\pgfqpoint{1.018455in}{2.438442in}}{\pgfqpoint{1.010641in}{2.430628in}}%
\pgfpathcurveto{\pgfqpoint{1.002828in}{2.422814in}}{\pgfqpoint{0.998437in}{2.412215in}}{\pgfqpoint{0.998437in}{2.401165in}}%
\pgfpathcurveto{\pgfqpoint{0.998437in}{2.390115in}}{\pgfqpoint{1.002828in}{2.379516in}}{\pgfqpoint{1.010641in}{2.371703in}}%
\pgfpathcurveto{\pgfqpoint{1.018455in}{2.363889in}}{\pgfqpoint{1.029054in}{2.359499in}}{\pgfqpoint{1.040104in}{2.359499in}}%
\pgfpathlineto{\pgfqpoint{1.040104in}{2.359499in}}%
\pgfpathclose%
\pgfusepath{stroke}%
\end{pgfscope}%
\begin{pgfscope}%
\pgfpathrectangle{\pgfqpoint{0.393053in}{0.375000in}}{\pgfqpoint{6.356833in}{5.175000in}}%
\pgfusepath{clip}%
\pgfsetbuttcap%
\pgfsetroundjoin%
\pgfsetlinewidth{1.003750pt}%
\definecolor{currentstroke}{rgb}{0.827451,0.827451,0.827451}%
\pgfsetstrokecolor{currentstroke}%
\pgfsetdash{}{0pt}%
\pgfpathmoveto{\pgfqpoint{3.464896in}{0.638283in}}%
\pgfpathcurveto{\pgfqpoint{3.475946in}{0.638283in}}{\pgfqpoint{3.486545in}{0.642673in}}{\pgfqpoint{3.494359in}{0.650486in}}%
\pgfpathcurveto{\pgfqpoint{3.502173in}{0.658300in}}{\pgfqpoint{3.506563in}{0.668899in}}{\pgfqpoint{3.506563in}{0.679949in}}%
\pgfpathcurveto{\pgfqpoint{3.506563in}{0.690999in}}{\pgfqpoint{3.502173in}{0.701598in}}{\pgfqpoint{3.494359in}{0.709412in}}%
\pgfpathcurveto{\pgfqpoint{3.486545in}{0.717226in}}{\pgfqpoint{3.475946in}{0.721616in}}{\pgfqpoint{3.464896in}{0.721616in}}%
\pgfpathcurveto{\pgfqpoint{3.453846in}{0.721616in}}{\pgfqpoint{3.443247in}{0.717226in}}{\pgfqpoint{3.435433in}{0.709412in}}%
\pgfpathcurveto{\pgfqpoint{3.427620in}{0.701598in}}{\pgfqpoint{3.423230in}{0.690999in}}{\pgfqpoint{3.423230in}{0.679949in}}%
\pgfpathcurveto{\pgfqpoint{3.423230in}{0.668899in}}{\pgfqpoint{3.427620in}{0.658300in}}{\pgfqpoint{3.435433in}{0.650486in}}%
\pgfpathcurveto{\pgfqpoint{3.443247in}{0.642673in}}{\pgfqpoint{3.453846in}{0.638283in}}{\pgfqpoint{3.464896in}{0.638283in}}%
\pgfpathlineto{\pgfqpoint{3.464896in}{0.638283in}}%
\pgfpathclose%
\pgfusepath{stroke}%
\end{pgfscope}%
\begin{pgfscope}%
\pgfpathrectangle{\pgfqpoint{0.393053in}{0.375000in}}{\pgfqpoint{6.356833in}{5.175000in}}%
\pgfusepath{clip}%
\pgfsetbuttcap%
\pgfsetroundjoin%
\pgfsetlinewidth{1.003750pt}%
\definecolor{currentstroke}{rgb}{0.827451,0.827451,0.827451}%
\pgfsetstrokecolor{currentstroke}%
\pgfsetdash{}{0pt}%
\pgfpathmoveto{\pgfqpoint{4.537650in}{0.413917in}}%
\pgfpathcurveto{\pgfqpoint{4.548700in}{0.413917in}}{\pgfqpoint{4.559300in}{0.418307in}}{\pgfqpoint{4.567113in}{0.426121in}}%
\pgfpathcurveto{\pgfqpoint{4.574927in}{0.433935in}}{\pgfqpoint{4.579317in}{0.444534in}}{\pgfqpoint{4.579317in}{0.455584in}}%
\pgfpathcurveto{\pgfqpoint{4.579317in}{0.466634in}}{\pgfqpoint{4.574927in}{0.477233in}}{\pgfqpoint{4.567113in}{0.485047in}}%
\pgfpathcurveto{\pgfqpoint{4.559300in}{0.492860in}}{\pgfqpoint{4.548700in}{0.497250in}}{\pgfqpoint{4.537650in}{0.497250in}}%
\pgfpathcurveto{\pgfqpoint{4.526600in}{0.497250in}}{\pgfqpoint{4.516001in}{0.492860in}}{\pgfqpoint{4.508188in}{0.485047in}}%
\pgfpathcurveto{\pgfqpoint{4.500374in}{0.477233in}}{\pgfqpoint{4.495984in}{0.466634in}}{\pgfqpoint{4.495984in}{0.455584in}}%
\pgfpathcurveto{\pgfqpoint{4.495984in}{0.444534in}}{\pgfqpoint{4.500374in}{0.433935in}}{\pgfqpoint{4.508188in}{0.426121in}}%
\pgfpathcurveto{\pgfqpoint{4.516001in}{0.418307in}}{\pgfqpoint{4.526600in}{0.413917in}}{\pgfqpoint{4.537650in}{0.413917in}}%
\pgfpathlineto{\pgfqpoint{4.537650in}{0.413917in}}%
\pgfpathclose%
\pgfusepath{stroke}%
\end{pgfscope}%
\begin{pgfscope}%
\pgfpathrectangle{\pgfqpoint{0.393053in}{0.375000in}}{\pgfqpoint{6.356833in}{5.175000in}}%
\pgfusepath{clip}%
\pgfsetbuttcap%
\pgfsetroundjoin%
\pgfsetlinewidth{1.003750pt}%
\definecolor{currentstroke}{rgb}{0.827451,0.827451,0.827451}%
\pgfsetstrokecolor{currentstroke}%
\pgfsetdash{}{0pt}%
\pgfpathmoveto{\pgfqpoint{0.997743in}{2.491784in}}%
\pgfpathcurveto{\pgfqpoint{1.008793in}{2.491784in}}{\pgfqpoint{1.019392in}{2.496174in}}{\pgfqpoint{1.027206in}{2.503988in}}%
\pgfpathcurveto{\pgfqpoint{1.035019in}{2.511802in}}{\pgfqpoint{1.039410in}{2.522401in}}{\pgfqpoint{1.039410in}{2.533451in}}%
\pgfpathcurveto{\pgfqpoint{1.039410in}{2.544501in}}{\pgfqpoint{1.035019in}{2.555100in}}{\pgfqpoint{1.027206in}{2.562914in}}%
\pgfpathcurveto{\pgfqpoint{1.019392in}{2.570727in}}{\pgfqpoint{1.008793in}{2.575118in}}{\pgfqpoint{0.997743in}{2.575118in}}%
\pgfpathcurveto{\pgfqpoint{0.986693in}{2.575118in}}{\pgfqpoint{0.976094in}{2.570727in}}{\pgfqpoint{0.968280in}{2.562914in}}%
\pgfpathcurveto{\pgfqpoint{0.960466in}{2.555100in}}{\pgfqpoint{0.956076in}{2.544501in}}{\pgfqpoint{0.956076in}{2.533451in}}%
\pgfpathcurveto{\pgfqpoint{0.956076in}{2.522401in}}{\pgfqpoint{0.960466in}{2.511802in}}{\pgfqpoint{0.968280in}{2.503988in}}%
\pgfpathcurveto{\pgfqpoint{0.976094in}{2.496174in}}{\pgfqpoint{0.986693in}{2.491784in}}{\pgfqpoint{0.997743in}{2.491784in}}%
\pgfpathlineto{\pgfqpoint{0.997743in}{2.491784in}}%
\pgfpathclose%
\pgfusepath{stroke}%
\end{pgfscope}%
\begin{pgfscope}%
\pgfpathrectangle{\pgfqpoint{0.393053in}{0.375000in}}{\pgfqpoint{6.356833in}{5.175000in}}%
\pgfusepath{clip}%
\pgfsetbuttcap%
\pgfsetroundjoin%
\pgfsetlinewidth{1.003750pt}%
\definecolor{currentstroke}{rgb}{0.827451,0.827451,0.827451}%
\pgfsetstrokecolor{currentstroke}%
\pgfsetdash{}{0pt}%
\pgfpathmoveto{\pgfqpoint{0.474470in}{3.727925in}}%
\pgfpathcurveto{\pgfqpoint{0.485520in}{3.727925in}}{\pgfqpoint{0.496119in}{3.732315in}}{\pgfqpoint{0.503933in}{3.740129in}}%
\pgfpathcurveto{\pgfqpoint{0.511746in}{3.747943in}}{\pgfqpoint{0.516137in}{3.758542in}}{\pgfqpoint{0.516137in}{3.769592in}}%
\pgfpathcurveto{\pgfqpoint{0.516137in}{3.780642in}}{\pgfqpoint{0.511746in}{3.791241in}}{\pgfqpoint{0.503933in}{3.799055in}}%
\pgfpathcurveto{\pgfqpoint{0.496119in}{3.806868in}}{\pgfqpoint{0.485520in}{3.811259in}}{\pgfqpoint{0.474470in}{3.811259in}}%
\pgfpathcurveto{\pgfqpoint{0.463420in}{3.811259in}}{\pgfqpoint{0.452821in}{3.806868in}}{\pgfqpoint{0.445007in}{3.799055in}}%
\pgfpathcurveto{\pgfqpoint{0.437194in}{3.791241in}}{\pgfqpoint{0.432803in}{3.780642in}}{\pgfqpoint{0.432803in}{3.769592in}}%
\pgfpathcurveto{\pgfqpoint{0.432803in}{3.758542in}}{\pgfqpoint{0.437194in}{3.747943in}}{\pgfqpoint{0.445007in}{3.740129in}}%
\pgfpathcurveto{\pgfqpoint{0.452821in}{3.732315in}}{\pgfqpoint{0.463420in}{3.727925in}}{\pgfqpoint{0.474470in}{3.727925in}}%
\pgfpathlineto{\pgfqpoint{0.474470in}{3.727925in}}%
\pgfpathclose%
\pgfusepath{stroke}%
\end{pgfscope}%
\begin{pgfscope}%
\pgfpathrectangle{\pgfqpoint{0.393053in}{0.375000in}}{\pgfqpoint{6.356833in}{5.175000in}}%
\pgfusepath{clip}%
\pgfsetbuttcap%
\pgfsetroundjoin%
\pgfsetlinewidth{1.003750pt}%
\definecolor{currentstroke}{rgb}{0.827451,0.827451,0.827451}%
\pgfsetstrokecolor{currentstroke}%
\pgfsetdash{}{0pt}%
\pgfpathmoveto{\pgfqpoint{2.145018in}{1.265722in}}%
\pgfpathcurveto{\pgfqpoint{2.156069in}{1.265722in}}{\pgfqpoint{2.166668in}{1.270112in}}{\pgfqpoint{2.174481in}{1.277926in}}%
\pgfpathcurveto{\pgfqpoint{2.182295in}{1.285739in}}{\pgfqpoint{2.186685in}{1.296338in}}{\pgfqpoint{2.186685in}{1.307389in}}%
\pgfpathcurveto{\pgfqpoint{2.186685in}{1.318439in}}{\pgfqpoint{2.182295in}{1.329038in}}{\pgfqpoint{2.174481in}{1.336851in}}%
\pgfpathcurveto{\pgfqpoint{2.166668in}{1.344665in}}{\pgfqpoint{2.156069in}{1.349055in}}{\pgfqpoint{2.145018in}{1.349055in}}%
\pgfpathcurveto{\pgfqpoint{2.133968in}{1.349055in}}{\pgfqpoint{2.123369in}{1.344665in}}{\pgfqpoint{2.115556in}{1.336851in}}%
\pgfpathcurveto{\pgfqpoint{2.107742in}{1.329038in}}{\pgfqpoint{2.103352in}{1.318439in}}{\pgfqpoint{2.103352in}{1.307389in}}%
\pgfpathcurveto{\pgfqpoint{2.103352in}{1.296338in}}{\pgfqpoint{2.107742in}{1.285739in}}{\pgfqpoint{2.115556in}{1.277926in}}%
\pgfpathcurveto{\pgfqpoint{2.123369in}{1.270112in}}{\pgfqpoint{2.133968in}{1.265722in}}{\pgfqpoint{2.145018in}{1.265722in}}%
\pgfpathlineto{\pgfqpoint{2.145018in}{1.265722in}}%
\pgfpathclose%
\pgfusepath{stroke}%
\end{pgfscope}%
\begin{pgfscope}%
\pgfpathrectangle{\pgfqpoint{0.393053in}{0.375000in}}{\pgfqpoint{6.356833in}{5.175000in}}%
\pgfusepath{clip}%
\pgfsetbuttcap%
\pgfsetroundjoin%
\pgfsetlinewidth{1.003750pt}%
\definecolor{currentstroke}{rgb}{0.827451,0.827451,0.827451}%
\pgfsetstrokecolor{currentstroke}%
\pgfsetdash{}{0pt}%
\pgfpathmoveto{\pgfqpoint{0.911466in}{2.525583in}}%
\pgfpathcurveto{\pgfqpoint{0.922516in}{2.525583in}}{\pgfqpoint{0.933116in}{2.529974in}}{\pgfqpoint{0.940929in}{2.537787in}}%
\pgfpathcurveto{\pgfqpoint{0.948743in}{2.545601in}}{\pgfqpoint{0.953133in}{2.556200in}}{\pgfqpoint{0.953133in}{2.567250in}}%
\pgfpathcurveto{\pgfqpoint{0.953133in}{2.578300in}}{\pgfqpoint{0.948743in}{2.588899in}}{\pgfqpoint{0.940929in}{2.596713in}}%
\pgfpathcurveto{\pgfqpoint{0.933116in}{2.604526in}}{\pgfqpoint{0.922516in}{2.608917in}}{\pgfqpoint{0.911466in}{2.608917in}}%
\pgfpathcurveto{\pgfqpoint{0.900416in}{2.608917in}}{\pgfqpoint{0.889817in}{2.604526in}}{\pgfqpoint{0.882004in}{2.596713in}}%
\pgfpathcurveto{\pgfqpoint{0.874190in}{2.588899in}}{\pgfqpoint{0.869800in}{2.578300in}}{\pgfqpoint{0.869800in}{2.567250in}}%
\pgfpathcurveto{\pgfqpoint{0.869800in}{2.556200in}}{\pgfqpoint{0.874190in}{2.545601in}}{\pgfqpoint{0.882004in}{2.537787in}}%
\pgfpathcurveto{\pgfqpoint{0.889817in}{2.529974in}}{\pgfqpoint{0.900416in}{2.525583in}}{\pgfqpoint{0.911466in}{2.525583in}}%
\pgfpathlineto{\pgfqpoint{0.911466in}{2.525583in}}%
\pgfpathclose%
\pgfusepath{stroke}%
\end{pgfscope}%
\begin{pgfscope}%
\pgfpathrectangle{\pgfqpoint{0.393053in}{0.375000in}}{\pgfqpoint{6.356833in}{5.175000in}}%
\pgfusepath{clip}%
\pgfsetbuttcap%
\pgfsetroundjoin%
\pgfsetlinewidth{1.003750pt}%
\definecolor{currentstroke}{rgb}{0.827451,0.827451,0.827451}%
\pgfsetstrokecolor{currentstroke}%
\pgfsetdash{}{0pt}%
\pgfpathmoveto{\pgfqpoint{0.455840in}{3.833482in}}%
\pgfpathcurveto{\pgfqpoint{0.466890in}{3.833482in}}{\pgfqpoint{0.477489in}{3.837873in}}{\pgfqpoint{0.485303in}{3.845686in}}%
\pgfpathcurveto{\pgfqpoint{0.493117in}{3.853500in}}{\pgfqpoint{0.497507in}{3.864099in}}{\pgfqpoint{0.497507in}{3.875149in}}%
\pgfpathcurveto{\pgfqpoint{0.497507in}{3.886199in}}{\pgfqpoint{0.493117in}{3.896798in}}{\pgfqpoint{0.485303in}{3.904612in}}%
\pgfpathcurveto{\pgfqpoint{0.477489in}{3.912425in}}{\pgfqpoint{0.466890in}{3.916816in}}{\pgfqpoint{0.455840in}{3.916816in}}%
\pgfpathcurveto{\pgfqpoint{0.444790in}{3.916816in}}{\pgfqpoint{0.434191in}{3.912425in}}{\pgfqpoint{0.426377in}{3.904612in}}%
\pgfpathcurveto{\pgfqpoint{0.418564in}{3.896798in}}{\pgfqpoint{0.414173in}{3.886199in}}{\pgfqpoint{0.414173in}{3.875149in}}%
\pgfpathcurveto{\pgfqpoint{0.414173in}{3.864099in}}{\pgfqpoint{0.418564in}{3.853500in}}{\pgfqpoint{0.426377in}{3.845686in}}%
\pgfpathcurveto{\pgfqpoint{0.434191in}{3.837873in}}{\pgfqpoint{0.444790in}{3.833482in}}{\pgfqpoint{0.455840in}{3.833482in}}%
\pgfpathlineto{\pgfqpoint{0.455840in}{3.833482in}}%
\pgfpathclose%
\pgfusepath{stroke}%
\end{pgfscope}%
\begin{pgfscope}%
\pgfpathrectangle{\pgfqpoint{0.393053in}{0.375000in}}{\pgfqpoint{6.356833in}{5.175000in}}%
\pgfusepath{clip}%
\pgfsetbuttcap%
\pgfsetroundjoin%
\pgfsetlinewidth{1.003750pt}%
\definecolor{currentstroke}{rgb}{0.827451,0.827451,0.827451}%
\pgfsetstrokecolor{currentstroke}%
\pgfsetdash{}{0pt}%
\pgfpathmoveto{\pgfqpoint{0.651937in}{3.079342in}}%
\pgfpathcurveto{\pgfqpoint{0.662987in}{3.079342in}}{\pgfqpoint{0.673586in}{3.083732in}}{\pgfqpoint{0.681400in}{3.091546in}}%
\pgfpathcurveto{\pgfqpoint{0.689213in}{3.099359in}}{\pgfqpoint{0.693604in}{3.109958in}}{\pgfqpoint{0.693604in}{3.121009in}}%
\pgfpathcurveto{\pgfqpoint{0.693604in}{3.132059in}}{\pgfqpoint{0.689213in}{3.142658in}}{\pgfqpoint{0.681400in}{3.150471in}}%
\pgfpathcurveto{\pgfqpoint{0.673586in}{3.158285in}}{\pgfqpoint{0.662987in}{3.162675in}}{\pgfqpoint{0.651937in}{3.162675in}}%
\pgfpathcurveto{\pgfqpoint{0.640887in}{3.162675in}}{\pgfqpoint{0.630288in}{3.158285in}}{\pgfqpoint{0.622474in}{3.150471in}}%
\pgfpathcurveto{\pgfqpoint{0.614660in}{3.142658in}}{\pgfqpoint{0.610270in}{3.132059in}}{\pgfqpoint{0.610270in}{3.121009in}}%
\pgfpathcurveto{\pgfqpoint{0.610270in}{3.109958in}}{\pgfqpoint{0.614660in}{3.099359in}}{\pgfqpoint{0.622474in}{3.091546in}}%
\pgfpathcurveto{\pgfqpoint{0.630288in}{3.083732in}}{\pgfqpoint{0.640887in}{3.079342in}}{\pgfqpoint{0.651937in}{3.079342in}}%
\pgfpathlineto{\pgfqpoint{0.651937in}{3.079342in}}%
\pgfpathclose%
\pgfusepath{stroke}%
\end{pgfscope}%
\begin{pgfscope}%
\pgfpathrectangle{\pgfqpoint{0.393053in}{0.375000in}}{\pgfqpoint{6.356833in}{5.175000in}}%
\pgfusepath{clip}%
\pgfsetbuttcap%
\pgfsetroundjoin%
\pgfsetlinewidth{1.003750pt}%
\definecolor{currentstroke}{rgb}{0.827451,0.827451,0.827451}%
\pgfsetstrokecolor{currentstroke}%
\pgfsetdash{}{0pt}%
\pgfpathmoveto{\pgfqpoint{2.036490in}{1.340063in}}%
\pgfpathcurveto{\pgfqpoint{2.047540in}{1.340063in}}{\pgfqpoint{2.058139in}{1.344454in}}{\pgfqpoint{2.065953in}{1.352267in}}%
\pgfpathcurveto{\pgfqpoint{2.073767in}{1.360081in}}{\pgfqpoint{2.078157in}{1.370680in}}{\pgfqpoint{2.078157in}{1.381730in}}%
\pgfpathcurveto{\pgfqpoint{2.078157in}{1.392780in}}{\pgfqpoint{2.073767in}{1.403379in}}{\pgfqpoint{2.065953in}{1.411193in}}%
\pgfpathcurveto{\pgfqpoint{2.058139in}{1.419007in}}{\pgfqpoint{2.047540in}{1.423397in}}{\pgfqpoint{2.036490in}{1.423397in}}%
\pgfpathcurveto{\pgfqpoint{2.025440in}{1.423397in}}{\pgfqpoint{2.014841in}{1.419007in}}{\pgfqpoint{2.007027in}{1.411193in}}%
\pgfpathcurveto{\pgfqpoint{1.999214in}{1.403379in}}{\pgfqpoint{1.994824in}{1.392780in}}{\pgfqpoint{1.994824in}{1.381730in}}%
\pgfpathcurveto{\pgfqpoint{1.994824in}{1.370680in}}{\pgfqpoint{1.999214in}{1.360081in}}{\pgfqpoint{2.007027in}{1.352267in}}%
\pgfpathcurveto{\pgfqpoint{2.014841in}{1.344454in}}{\pgfqpoint{2.025440in}{1.340063in}}{\pgfqpoint{2.036490in}{1.340063in}}%
\pgfpathlineto{\pgfqpoint{2.036490in}{1.340063in}}%
\pgfpathclose%
\pgfusepath{stroke}%
\end{pgfscope}%
\begin{pgfscope}%
\pgfpathrectangle{\pgfqpoint{0.393053in}{0.375000in}}{\pgfqpoint{6.356833in}{5.175000in}}%
\pgfusepath{clip}%
\pgfsetbuttcap%
\pgfsetroundjoin%
\pgfsetlinewidth{1.003750pt}%
\definecolor{currentstroke}{rgb}{0.827451,0.827451,0.827451}%
\pgfsetstrokecolor{currentstroke}%
\pgfsetdash{}{0pt}%
\pgfpathmoveto{\pgfqpoint{2.629511in}{0.982910in}}%
\pgfpathcurveto{\pgfqpoint{2.640562in}{0.982910in}}{\pgfqpoint{2.651161in}{0.987300in}}{\pgfqpoint{2.658974in}{0.995114in}}%
\pgfpathcurveto{\pgfqpoint{2.666788in}{1.002928in}}{\pgfqpoint{2.671178in}{1.013527in}}{\pgfqpoint{2.671178in}{1.024577in}}%
\pgfpathcurveto{\pgfqpoint{2.671178in}{1.035627in}}{\pgfqpoint{2.666788in}{1.046226in}}{\pgfqpoint{2.658974in}{1.054040in}}%
\pgfpathcurveto{\pgfqpoint{2.651161in}{1.061853in}}{\pgfqpoint{2.640562in}{1.066243in}}{\pgfqpoint{2.629511in}{1.066243in}}%
\pgfpathcurveto{\pgfqpoint{2.618461in}{1.066243in}}{\pgfqpoint{2.607862in}{1.061853in}}{\pgfqpoint{2.600049in}{1.054040in}}%
\pgfpathcurveto{\pgfqpoint{2.592235in}{1.046226in}}{\pgfqpoint{2.587845in}{1.035627in}}{\pgfqpoint{2.587845in}{1.024577in}}%
\pgfpathcurveto{\pgfqpoint{2.587845in}{1.013527in}}{\pgfqpoint{2.592235in}{1.002928in}}{\pgfqpoint{2.600049in}{0.995114in}}%
\pgfpathcurveto{\pgfqpoint{2.607862in}{0.987300in}}{\pgfqpoint{2.618461in}{0.982910in}}{\pgfqpoint{2.629511in}{0.982910in}}%
\pgfpathlineto{\pgfqpoint{2.629511in}{0.982910in}}%
\pgfpathclose%
\pgfusepath{stroke}%
\end{pgfscope}%
\begin{pgfscope}%
\pgfpathrectangle{\pgfqpoint{0.393053in}{0.375000in}}{\pgfqpoint{6.356833in}{5.175000in}}%
\pgfusepath{clip}%
\pgfsetbuttcap%
\pgfsetroundjoin%
\pgfsetlinewidth{1.003750pt}%
\definecolor{currentstroke}{rgb}{0.827451,0.827451,0.827451}%
\pgfsetstrokecolor{currentstroke}%
\pgfsetdash{}{0pt}%
\pgfpathmoveto{\pgfqpoint{4.310929in}{0.438129in}}%
\pgfpathcurveto{\pgfqpoint{4.321979in}{0.438129in}}{\pgfqpoint{4.332578in}{0.442519in}}{\pgfqpoint{4.340392in}{0.450333in}}%
\pgfpathcurveto{\pgfqpoint{4.348205in}{0.458147in}}{\pgfqpoint{4.352595in}{0.468746in}}{\pgfqpoint{4.352595in}{0.479796in}}%
\pgfpathcurveto{\pgfqpoint{4.352595in}{0.490846in}}{\pgfqpoint{4.348205in}{0.501445in}}{\pgfqpoint{4.340392in}{0.509258in}}%
\pgfpathcurveto{\pgfqpoint{4.332578in}{0.517072in}}{\pgfqpoint{4.321979in}{0.521462in}}{\pgfqpoint{4.310929in}{0.521462in}}%
\pgfpathcurveto{\pgfqpoint{4.299879in}{0.521462in}}{\pgfqpoint{4.289280in}{0.517072in}}{\pgfqpoint{4.281466in}{0.509258in}}%
\pgfpathcurveto{\pgfqpoint{4.273652in}{0.501445in}}{\pgfqpoint{4.269262in}{0.490846in}}{\pgfqpoint{4.269262in}{0.479796in}}%
\pgfpathcurveto{\pgfqpoint{4.269262in}{0.468746in}}{\pgfqpoint{4.273652in}{0.458147in}}{\pgfqpoint{4.281466in}{0.450333in}}%
\pgfpathcurveto{\pgfqpoint{4.289280in}{0.442519in}}{\pgfqpoint{4.299879in}{0.438129in}}{\pgfqpoint{4.310929in}{0.438129in}}%
\pgfpathlineto{\pgfqpoint{4.310929in}{0.438129in}}%
\pgfpathclose%
\pgfusepath{stroke}%
\end{pgfscope}%
\begin{pgfscope}%
\pgfpathrectangle{\pgfqpoint{0.393053in}{0.375000in}}{\pgfqpoint{6.356833in}{5.175000in}}%
\pgfusepath{clip}%
\pgfsetbuttcap%
\pgfsetroundjoin%
\pgfsetlinewidth{1.003750pt}%
\definecolor{currentstroke}{rgb}{0.827451,0.827451,0.827451}%
\pgfsetstrokecolor{currentstroke}%
\pgfsetdash{}{0pt}%
\pgfpathmoveto{\pgfqpoint{0.519167in}{3.558335in}}%
\pgfpathcurveto{\pgfqpoint{0.530217in}{3.558335in}}{\pgfqpoint{0.540816in}{3.562726in}}{\pgfqpoint{0.548630in}{3.570539in}}%
\pgfpathcurveto{\pgfqpoint{0.556443in}{3.578353in}}{\pgfqpoint{0.560834in}{3.588952in}}{\pgfqpoint{0.560834in}{3.600002in}}%
\pgfpathcurveto{\pgfqpoint{0.560834in}{3.611052in}}{\pgfqpoint{0.556443in}{3.621651in}}{\pgfqpoint{0.548630in}{3.629465in}}%
\pgfpathcurveto{\pgfqpoint{0.540816in}{3.637278in}}{\pgfqpoint{0.530217in}{3.641669in}}{\pgfqpoint{0.519167in}{3.641669in}}%
\pgfpathcurveto{\pgfqpoint{0.508117in}{3.641669in}}{\pgfqpoint{0.497518in}{3.637278in}}{\pgfqpoint{0.489704in}{3.629465in}}%
\pgfpathcurveto{\pgfqpoint{0.481891in}{3.621651in}}{\pgfqpoint{0.477500in}{3.611052in}}{\pgfqpoint{0.477500in}{3.600002in}}%
\pgfpathcurveto{\pgfqpoint{0.477500in}{3.588952in}}{\pgfqpoint{0.481891in}{3.578353in}}{\pgfqpoint{0.489704in}{3.570539in}}%
\pgfpathcurveto{\pgfqpoint{0.497518in}{3.562726in}}{\pgfqpoint{0.508117in}{3.558335in}}{\pgfqpoint{0.519167in}{3.558335in}}%
\pgfpathlineto{\pgfqpoint{0.519167in}{3.558335in}}%
\pgfpathclose%
\pgfusepath{stroke}%
\end{pgfscope}%
\begin{pgfscope}%
\pgfpathrectangle{\pgfqpoint{0.393053in}{0.375000in}}{\pgfqpoint{6.356833in}{5.175000in}}%
\pgfusepath{clip}%
\pgfsetbuttcap%
\pgfsetroundjoin%
\pgfsetlinewidth{1.003750pt}%
\definecolor{currentstroke}{rgb}{0.827451,0.827451,0.827451}%
\pgfsetstrokecolor{currentstroke}%
\pgfsetdash{}{0pt}%
\pgfpathmoveto{\pgfqpoint{1.680974in}{1.628980in}}%
\pgfpathcurveto{\pgfqpoint{1.692024in}{1.628980in}}{\pgfqpoint{1.702623in}{1.633370in}}{\pgfqpoint{1.710437in}{1.641184in}}%
\pgfpathcurveto{\pgfqpoint{1.718250in}{1.648997in}}{\pgfqpoint{1.722641in}{1.659596in}}{\pgfqpoint{1.722641in}{1.670646in}}%
\pgfpathcurveto{\pgfqpoint{1.722641in}{1.681697in}}{\pgfqpoint{1.718250in}{1.692296in}}{\pgfqpoint{1.710437in}{1.700109in}}%
\pgfpathcurveto{\pgfqpoint{1.702623in}{1.707923in}}{\pgfqpoint{1.692024in}{1.712313in}}{\pgfqpoint{1.680974in}{1.712313in}}%
\pgfpathcurveto{\pgfqpoint{1.669924in}{1.712313in}}{\pgfqpoint{1.659325in}{1.707923in}}{\pgfqpoint{1.651511in}{1.700109in}}%
\pgfpathcurveto{\pgfqpoint{1.643698in}{1.692296in}}{\pgfqpoint{1.639307in}{1.681697in}}{\pgfqpoint{1.639307in}{1.670646in}}%
\pgfpathcurveto{\pgfqpoint{1.639307in}{1.659596in}}{\pgfqpoint{1.643698in}{1.648997in}}{\pgfqpoint{1.651511in}{1.641184in}}%
\pgfpathcurveto{\pgfqpoint{1.659325in}{1.633370in}}{\pgfqpoint{1.669924in}{1.628980in}}{\pgfqpoint{1.680974in}{1.628980in}}%
\pgfpathlineto{\pgfqpoint{1.680974in}{1.628980in}}%
\pgfpathclose%
\pgfusepath{stroke}%
\end{pgfscope}%
\begin{pgfscope}%
\pgfpathrectangle{\pgfqpoint{0.393053in}{0.375000in}}{\pgfqpoint{6.356833in}{5.175000in}}%
\pgfusepath{clip}%
\pgfsetbuttcap%
\pgfsetroundjoin%
\pgfsetlinewidth{1.003750pt}%
\definecolor{currentstroke}{rgb}{0.827451,0.827451,0.827451}%
\pgfsetstrokecolor{currentstroke}%
\pgfsetdash{}{0pt}%
\pgfpathmoveto{\pgfqpoint{0.508412in}{3.594376in}}%
\pgfpathcurveto{\pgfqpoint{0.519462in}{3.594376in}}{\pgfqpoint{0.530061in}{3.598766in}}{\pgfqpoint{0.537874in}{3.606580in}}%
\pgfpathcurveto{\pgfqpoint{0.545688in}{3.614393in}}{\pgfqpoint{0.550078in}{3.624992in}}{\pgfqpoint{0.550078in}{3.636042in}}%
\pgfpathcurveto{\pgfqpoint{0.550078in}{3.647092in}}{\pgfqpoint{0.545688in}{3.657692in}}{\pgfqpoint{0.537874in}{3.665505in}}%
\pgfpathcurveto{\pgfqpoint{0.530061in}{3.673319in}}{\pgfqpoint{0.519462in}{3.677709in}}{\pgfqpoint{0.508412in}{3.677709in}}%
\pgfpathcurveto{\pgfqpoint{0.497361in}{3.677709in}}{\pgfqpoint{0.486762in}{3.673319in}}{\pgfqpoint{0.478949in}{3.665505in}}%
\pgfpathcurveto{\pgfqpoint{0.471135in}{3.657692in}}{\pgfqpoint{0.466745in}{3.647092in}}{\pgfqpoint{0.466745in}{3.636042in}}%
\pgfpathcurveto{\pgfqpoint{0.466745in}{3.624992in}}{\pgfqpoint{0.471135in}{3.614393in}}{\pgfqpoint{0.478949in}{3.606580in}}%
\pgfpathcurveto{\pgfqpoint{0.486762in}{3.598766in}}{\pgfqpoint{0.497361in}{3.594376in}}{\pgfqpoint{0.508412in}{3.594376in}}%
\pgfpathlineto{\pgfqpoint{0.508412in}{3.594376in}}%
\pgfpathclose%
\pgfusepath{stroke}%
\end{pgfscope}%
\begin{pgfscope}%
\pgfpathrectangle{\pgfqpoint{0.393053in}{0.375000in}}{\pgfqpoint{6.356833in}{5.175000in}}%
\pgfusepath{clip}%
\pgfsetbuttcap%
\pgfsetroundjoin%
\pgfsetlinewidth{1.003750pt}%
\definecolor{currentstroke}{rgb}{0.827451,0.827451,0.827451}%
\pgfsetstrokecolor{currentstroke}%
\pgfsetdash{}{0pt}%
\pgfpathmoveto{\pgfqpoint{1.354968in}{1.928632in}}%
\pgfpathcurveto{\pgfqpoint{1.366018in}{1.928632in}}{\pgfqpoint{1.376617in}{1.933022in}}{\pgfqpoint{1.384431in}{1.940836in}}%
\pgfpathcurveto{\pgfqpoint{1.392245in}{1.948650in}}{\pgfqpoint{1.396635in}{1.959249in}}{\pgfqpoint{1.396635in}{1.970299in}}%
\pgfpathcurveto{\pgfqpoint{1.396635in}{1.981349in}}{\pgfqpoint{1.392245in}{1.991948in}}{\pgfqpoint{1.384431in}{1.999762in}}%
\pgfpathcurveto{\pgfqpoint{1.376617in}{2.007575in}}{\pgfqpoint{1.366018in}{2.011966in}}{\pgfqpoint{1.354968in}{2.011966in}}%
\pgfpathcurveto{\pgfqpoint{1.343918in}{2.011966in}}{\pgfqpoint{1.333319in}{2.007575in}}{\pgfqpoint{1.325505in}{1.999762in}}%
\pgfpathcurveto{\pgfqpoint{1.317692in}{1.991948in}}{\pgfqpoint{1.313302in}{1.981349in}}{\pgfqpoint{1.313302in}{1.970299in}}%
\pgfpathcurveto{\pgfqpoint{1.313302in}{1.959249in}}{\pgfqpoint{1.317692in}{1.948650in}}{\pgfqpoint{1.325505in}{1.940836in}}%
\pgfpathcurveto{\pgfqpoint{1.333319in}{1.933022in}}{\pgfqpoint{1.343918in}{1.928632in}}{\pgfqpoint{1.354968in}{1.928632in}}%
\pgfpathlineto{\pgfqpoint{1.354968in}{1.928632in}}%
\pgfpathclose%
\pgfusepath{stroke}%
\end{pgfscope}%
\begin{pgfscope}%
\pgfpathrectangle{\pgfqpoint{0.393053in}{0.375000in}}{\pgfqpoint{6.356833in}{5.175000in}}%
\pgfusepath{clip}%
\pgfsetbuttcap%
\pgfsetroundjoin%
\pgfsetlinewidth{1.003750pt}%
\definecolor{currentstroke}{rgb}{0.827451,0.827451,0.827451}%
\pgfsetstrokecolor{currentstroke}%
\pgfsetdash{}{0pt}%
\pgfpathmoveto{\pgfqpoint{0.646039in}{3.145596in}}%
\pgfpathcurveto{\pgfqpoint{0.657089in}{3.145596in}}{\pgfqpoint{0.667688in}{3.149986in}}{\pgfqpoint{0.675502in}{3.157800in}}%
\pgfpathcurveto{\pgfqpoint{0.683316in}{3.165613in}}{\pgfqpoint{0.687706in}{3.176212in}}{\pgfqpoint{0.687706in}{3.187263in}}%
\pgfpathcurveto{\pgfqpoint{0.687706in}{3.198313in}}{\pgfqpoint{0.683316in}{3.208912in}}{\pgfqpoint{0.675502in}{3.216725in}}%
\pgfpathcurveto{\pgfqpoint{0.667688in}{3.224539in}}{\pgfqpoint{0.657089in}{3.228929in}}{\pgfqpoint{0.646039in}{3.228929in}}%
\pgfpathcurveto{\pgfqpoint{0.634989in}{3.228929in}}{\pgfqpoint{0.624390in}{3.224539in}}{\pgfqpoint{0.616576in}{3.216725in}}%
\pgfpathcurveto{\pgfqpoint{0.608763in}{3.208912in}}{\pgfqpoint{0.604373in}{3.198313in}}{\pgfqpoint{0.604373in}{3.187263in}}%
\pgfpathcurveto{\pgfqpoint{0.604373in}{3.176212in}}{\pgfqpoint{0.608763in}{3.165613in}}{\pgfqpoint{0.616576in}{3.157800in}}%
\pgfpathcurveto{\pgfqpoint{0.624390in}{3.149986in}}{\pgfqpoint{0.634989in}{3.145596in}}{\pgfqpoint{0.646039in}{3.145596in}}%
\pgfpathlineto{\pgfqpoint{0.646039in}{3.145596in}}%
\pgfpathclose%
\pgfusepath{stroke}%
\end{pgfscope}%
\begin{pgfscope}%
\pgfpathrectangle{\pgfqpoint{0.393053in}{0.375000in}}{\pgfqpoint{6.356833in}{5.175000in}}%
\pgfusepath{clip}%
\pgfsetbuttcap%
\pgfsetroundjoin%
\pgfsetlinewidth{1.003750pt}%
\definecolor{currentstroke}{rgb}{0.827451,0.827451,0.827451}%
\pgfsetstrokecolor{currentstroke}%
\pgfsetdash{}{0pt}%
\pgfpathmoveto{\pgfqpoint{1.843211in}{1.515821in}}%
\pgfpathcurveto{\pgfqpoint{1.854261in}{1.515821in}}{\pgfqpoint{1.864860in}{1.520211in}}{\pgfqpoint{1.872674in}{1.528025in}}%
\pgfpathcurveto{\pgfqpoint{1.880487in}{1.535838in}}{\pgfqpoint{1.884878in}{1.546437in}}{\pgfqpoint{1.884878in}{1.557487in}}%
\pgfpathcurveto{\pgfqpoint{1.884878in}{1.568537in}}{\pgfqpoint{1.880487in}{1.579136in}}{\pgfqpoint{1.872674in}{1.586950in}}%
\pgfpathcurveto{\pgfqpoint{1.864860in}{1.594764in}}{\pgfqpoint{1.854261in}{1.599154in}}{\pgfqpoint{1.843211in}{1.599154in}}%
\pgfpathcurveto{\pgfqpoint{1.832161in}{1.599154in}}{\pgfqpoint{1.821562in}{1.594764in}}{\pgfqpoint{1.813748in}{1.586950in}}%
\pgfpathcurveto{\pgfqpoint{1.805934in}{1.579136in}}{\pgfqpoint{1.801544in}{1.568537in}}{\pgfqpoint{1.801544in}{1.557487in}}%
\pgfpathcurveto{\pgfqpoint{1.801544in}{1.546437in}}{\pgfqpoint{1.805934in}{1.535838in}}{\pgfqpoint{1.813748in}{1.528025in}}%
\pgfpathcurveto{\pgfqpoint{1.821562in}{1.520211in}}{\pgfqpoint{1.832161in}{1.515821in}}{\pgfqpoint{1.843211in}{1.515821in}}%
\pgfpathlineto{\pgfqpoint{1.843211in}{1.515821in}}%
\pgfpathclose%
\pgfusepath{stroke}%
\end{pgfscope}%
\begin{pgfscope}%
\pgfpathrectangle{\pgfqpoint{0.393053in}{0.375000in}}{\pgfqpoint{6.356833in}{5.175000in}}%
\pgfusepath{clip}%
\pgfsetbuttcap%
\pgfsetroundjoin%
\pgfsetlinewidth{1.003750pt}%
\definecolor{currentstroke}{rgb}{0.827451,0.827451,0.827451}%
\pgfsetstrokecolor{currentstroke}%
\pgfsetdash{}{0pt}%
\pgfpathmoveto{\pgfqpoint{2.103640in}{1.312371in}}%
\pgfpathcurveto{\pgfqpoint{2.114690in}{1.312371in}}{\pgfqpoint{2.125289in}{1.316762in}}{\pgfqpoint{2.133103in}{1.324575in}}%
\pgfpathcurveto{\pgfqpoint{2.140916in}{1.332389in}}{\pgfqpoint{2.145307in}{1.342988in}}{\pgfqpoint{2.145307in}{1.354038in}}%
\pgfpathcurveto{\pgfqpoint{2.145307in}{1.365088in}}{\pgfqpoint{2.140916in}{1.375687in}}{\pgfqpoint{2.133103in}{1.383501in}}%
\pgfpathcurveto{\pgfqpoint{2.125289in}{1.391314in}}{\pgfqpoint{2.114690in}{1.395705in}}{\pgfqpoint{2.103640in}{1.395705in}}%
\pgfpathcurveto{\pgfqpoint{2.092590in}{1.395705in}}{\pgfqpoint{2.081991in}{1.391314in}}{\pgfqpoint{2.074177in}{1.383501in}}%
\pgfpathcurveto{\pgfqpoint{2.066364in}{1.375687in}}{\pgfqpoint{2.061973in}{1.365088in}}{\pgfqpoint{2.061973in}{1.354038in}}%
\pgfpathcurveto{\pgfqpoint{2.061973in}{1.342988in}}{\pgfqpoint{2.066364in}{1.332389in}}{\pgfqpoint{2.074177in}{1.324575in}}%
\pgfpathcurveto{\pgfqpoint{2.081991in}{1.316762in}}{\pgfqpoint{2.092590in}{1.312371in}}{\pgfqpoint{2.103640in}{1.312371in}}%
\pgfpathlineto{\pgfqpoint{2.103640in}{1.312371in}}%
\pgfpathclose%
\pgfusepath{stroke}%
\end{pgfscope}%
\begin{pgfscope}%
\pgfpathrectangle{\pgfqpoint{0.393053in}{0.375000in}}{\pgfqpoint{6.356833in}{5.175000in}}%
\pgfusepath{clip}%
\pgfsetbuttcap%
\pgfsetroundjoin%
\pgfsetlinewidth{1.003750pt}%
\definecolor{currentstroke}{rgb}{0.827451,0.827451,0.827451}%
\pgfsetstrokecolor{currentstroke}%
\pgfsetdash{}{0pt}%
\pgfpathmoveto{\pgfqpoint{3.256902in}{0.730828in}}%
\pgfpathcurveto{\pgfqpoint{3.267953in}{0.730828in}}{\pgfqpoint{3.278552in}{0.735218in}}{\pgfqpoint{3.286365in}{0.743032in}}%
\pgfpathcurveto{\pgfqpoint{3.294179in}{0.750846in}}{\pgfqpoint{3.298569in}{0.761445in}}{\pgfqpoint{3.298569in}{0.772495in}}%
\pgfpathcurveto{\pgfqpoint{3.298569in}{0.783545in}}{\pgfqpoint{3.294179in}{0.794144in}}{\pgfqpoint{3.286365in}{0.801958in}}%
\pgfpathcurveto{\pgfqpoint{3.278552in}{0.809771in}}{\pgfqpoint{3.267953in}{0.814162in}}{\pgfqpoint{3.256902in}{0.814162in}}%
\pgfpathcurveto{\pgfqpoint{3.245852in}{0.814162in}}{\pgfqpoint{3.235253in}{0.809771in}}{\pgfqpoint{3.227440in}{0.801958in}}%
\pgfpathcurveto{\pgfqpoint{3.219626in}{0.794144in}}{\pgfqpoint{3.215236in}{0.783545in}}{\pgfqpoint{3.215236in}{0.772495in}}%
\pgfpathcurveto{\pgfqpoint{3.215236in}{0.761445in}}{\pgfqpoint{3.219626in}{0.750846in}}{\pgfqpoint{3.227440in}{0.743032in}}%
\pgfpathcurveto{\pgfqpoint{3.235253in}{0.735218in}}{\pgfqpoint{3.245852in}{0.730828in}}{\pgfqpoint{3.256902in}{0.730828in}}%
\pgfpathlineto{\pgfqpoint{3.256902in}{0.730828in}}%
\pgfpathclose%
\pgfusepath{stroke}%
\end{pgfscope}%
\begin{pgfscope}%
\pgfpathrectangle{\pgfqpoint{0.393053in}{0.375000in}}{\pgfqpoint{6.356833in}{5.175000in}}%
\pgfusepath{clip}%
\pgfsetbuttcap%
\pgfsetroundjoin%
\pgfsetlinewidth{1.003750pt}%
\definecolor{currentstroke}{rgb}{0.827451,0.827451,0.827451}%
\pgfsetstrokecolor{currentstroke}%
\pgfsetdash{}{0pt}%
\pgfpathmoveto{\pgfqpoint{0.483847in}{3.664437in}}%
\pgfpathcurveto{\pgfqpoint{0.494897in}{3.664437in}}{\pgfqpoint{0.505496in}{3.668827in}}{\pgfqpoint{0.513309in}{3.676640in}}%
\pgfpathcurveto{\pgfqpoint{0.521123in}{3.684454in}}{\pgfqpoint{0.525513in}{3.695053in}}{\pgfqpoint{0.525513in}{3.706103in}}%
\pgfpathcurveto{\pgfqpoint{0.525513in}{3.717153in}}{\pgfqpoint{0.521123in}{3.727752in}}{\pgfqpoint{0.513309in}{3.735566in}}%
\pgfpathcurveto{\pgfqpoint{0.505496in}{3.743380in}}{\pgfqpoint{0.494897in}{3.747770in}}{\pgfqpoint{0.483847in}{3.747770in}}%
\pgfpathcurveto{\pgfqpoint{0.472797in}{3.747770in}}{\pgfqpoint{0.462198in}{3.743380in}}{\pgfqpoint{0.454384in}{3.735566in}}%
\pgfpathcurveto{\pgfqpoint{0.446570in}{3.727752in}}{\pgfqpoint{0.442180in}{3.717153in}}{\pgfqpoint{0.442180in}{3.706103in}}%
\pgfpathcurveto{\pgfqpoint{0.442180in}{3.695053in}}{\pgfqpoint{0.446570in}{3.684454in}}{\pgfqpoint{0.454384in}{3.676640in}}%
\pgfpathcurveto{\pgfqpoint{0.462198in}{3.668827in}}{\pgfqpoint{0.472797in}{3.664437in}}{\pgfqpoint{0.483847in}{3.664437in}}%
\pgfpathlineto{\pgfqpoint{0.483847in}{3.664437in}}%
\pgfpathclose%
\pgfusepath{stroke}%
\end{pgfscope}%
\begin{pgfscope}%
\pgfpathrectangle{\pgfqpoint{0.393053in}{0.375000in}}{\pgfqpoint{6.356833in}{5.175000in}}%
\pgfusepath{clip}%
\pgfsetbuttcap%
\pgfsetroundjoin%
\pgfsetlinewidth{1.003750pt}%
\definecolor{currentstroke}{rgb}{0.827451,0.827451,0.827451}%
\pgfsetstrokecolor{currentstroke}%
\pgfsetdash{}{0pt}%
\pgfpathmoveto{\pgfqpoint{1.702004in}{1.599430in}}%
\pgfpathcurveto{\pgfqpoint{1.713054in}{1.599430in}}{\pgfqpoint{1.723653in}{1.603820in}}{\pgfqpoint{1.731467in}{1.611633in}}%
\pgfpathcurveto{\pgfqpoint{1.739281in}{1.619447in}}{\pgfqpoint{1.743671in}{1.630046in}}{\pgfqpoint{1.743671in}{1.641096in}}%
\pgfpathcurveto{\pgfqpoint{1.743671in}{1.652146in}}{\pgfqpoint{1.739281in}{1.662745in}}{\pgfqpoint{1.731467in}{1.670559in}}%
\pgfpathcurveto{\pgfqpoint{1.723653in}{1.678373in}}{\pgfqpoint{1.713054in}{1.682763in}}{\pgfqpoint{1.702004in}{1.682763in}}%
\pgfpathcurveto{\pgfqpoint{1.690954in}{1.682763in}}{\pgfqpoint{1.680355in}{1.678373in}}{\pgfqpoint{1.672541in}{1.670559in}}%
\pgfpathcurveto{\pgfqpoint{1.664728in}{1.662745in}}{\pgfqpoint{1.660338in}{1.652146in}}{\pgfqpoint{1.660338in}{1.641096in}}%
\pgfpathcurveto{\pgfqpoint{1.660338in}{1.630046in}}{\pgfqpoint{1.664728in}{1.619447in}}{\pgfqpoint{1.672541in}{1.611633in}}%
\pgfpathcurveto{\pgfqpoint{1.680355in}{1.603820in}}{\pgfqpoint{1.690954in}{1.599430in}}{\pgfqpoint{1.702004in}{1.599430in}}%
\pgfpathlineto{\pgfqpoint{1.702004in}{1.599430in}}%
\pgfpathclose%
\pgfusepath{stroke}%
\end{pgfscope}%
\begin{pgfscope}%
\pgfpathrectangle{\pgfqpoint{0.393053in}{0.375000in}}{\pgfqpoint{6.356833in}{5.175000in}}%
\pgfusepath{clip}%
\pgfsetbuttcap%
\pgfsetroundjoin%
\pgfsetlinewidth{1.003750pt}%
\definecolor{currentstroke}{rgb}{0.827451,0.827451,0.827451}%
\pgfsetstrokecolor{currentstroke}%
\pgfsetdash{}{0pt}%
\pgfpathmoveto{\pgfqpoint{2.796329in}{0.903359in}}%
\pgfpathcurveto{\pgfqpoint{2.807379in}{0.903359in}}{\pgfqpoint{2.817978in}{0.907749in}}{\pgfqpoint{2.825792in}{0.915563in}}%
\pgfpathcurveto{\pgfqpoint{2.833605in}{0.923376in}}{\pgfqpoint{2.837995in}{0.933975in}}{\pgfqpoint{2.837995in}{0.945025in}}%
\pgfpathcurveto{\pgfqpoint{2.837995in}{0.956076in}}{\pgfqpoint{2.833605in}{0.966675in}}{\pgfqpoint{2.825792in}{0.974488in}}%
\pgfpathcurveto{\pgfqpoint{2.817978in}{0.982302in}}{\pgfqpoint{2.807379in}{0.986692in}}{\pgfqpoint{2.796329in}{0.986692in}}%
\pgfpathcurveto{\pgfqpoint{2.785279in}{0.986692in}}{\pgfqpoint{2.774680in}{0.982302in}}{\pgfqpoint{2.766866in}{0.974488in}}%
\pgfpathcurveto{\pgfqpoint{2.759052in}{0.966675in}}{\pgfqpoint{2.754662in}{0.956076in}}{\pgfqpoint{2.754662in}{0.945025in}}%
\pgfpathcurveto{\pgfqpoint{2.754662in}{0.933975in}}{\pgfqpoint{2.759052in}{0.923376in}}{\pgfqpoint{2.766866in}{0.915563in}}%
\pgfpathcurveto{\pgfqpoint{2.774680in}{0.907749in}}{\pgfqpoint{2.785279in}{0.903359in}}{\pgfqpoint{2.796329in}{0.903359in}}%
\pgfpathlineto{\pgfqpoint{2.796329in}{0.903359in}}%
\pgfpathclose%
\pgfusepath{stroke}%
\end{pgfscope}%
\begin{pgfscope}%
\pgfpathrectangle{\pgfqpoint{0.393053in}{0.375000in}}{\pgfqpoint{6.356833in}{5.175000in}}%
\pgfusepath{clip}%
\pgfsetbuttcap%
\pgfsetroundjoin%
\pgfsetlinewidth{1.003750pt}%
\definecolor{currentstroke}{rgb}{0.827451,0.827451,0.827451}%
\pgfsetstrokecolor{currentstroke}%
\pgfsetdash{}{0pt}%
\pgfpathmoveto{\pgfqpoint{1.493790in}{1.813114in}}%
\pgfpathcurveto{\pgfqpoint{1.504840in}{1.813114in}}{\pgfqpoint{1.515439in}{1.817505in}}{\pgfqpoint{1.523252in}{1.825318in}}%
\pgfpathcurveto{\pgfqpoint{1.531066in}{1.833132in}}{\pgfqpoint{1.535456in}{1.843731in}}{\pgfqpoint{1.535456in}{1.854781in}}%
\pgfpathcurveto{\pgfqpoint{1.535456in}{1.865831in}}{\pgfqpoint{1.531066in}{1.876430in}}{\pgfqpoint{1.523252in}{1.884244in}}%
\pgfpathcurveto{\pgfqpoint{1.515439in}{1.892058in}}{\pgfqpoint{1.504840in}{1.896448in}}{\pgfqpoint{1.493790in}{1.896448in}}%
\pgfpathcurveto{\pgfqpoint{1.482740in}{1.896448in}}{\pgfqpoint{1.472140in}{1.892058in}}{\pgfqpoint{1.464327in}{1.884244in}}%
\pgfpathcurveto{\pgfqpoint{1.456513in}{1.876430in}}{\pgfqpoint{1.452123in}{1.865831in}}{\pgfqpoint{1.452123in}{1.854781in}}%
\pgfpathcurveto{\pgfqpoint{1.452123in}{1.843731in}}{\pgfqpoint{1.456513in}{1.833132in}}{\pgfqpoint{1.464327in}{1.825318in}}%
\pgfpathcurveto{\pgfqpoint{1.472140in}{1.817505in}}{\pgfqpoint{1.482740in}{1.813114in}}{\pgfqpoint{1.493790in}{1.813114in}}%
\pgfpathlineto{\pgfqpoint{1.493790in}{1.813114in}}%
\pgfpathclose%
\pgfusepath{stroke}%
\end{pgfscope}%
\begin{pgfscope}%
\pgfpathrectangle{\pgfqpoint{0.393053in}{0.375000in}}{\pgfqpoint{6.356833in}{5.175000in}}%
\pgfusepath{clip}%
\pgfsetbuttcap%
\pgfsetroundjoin%
\pgfsetlinewidth{1.003750pt}%
\definecolor{currentstroke}{rgb}{0.827451,0.827451,0.827451}%
\pgfsetstrokecolor{currentstroke}%
\pgfsetdash{}{0pt}%
\pgfpathmoveto{\pgfqpoint{1.536415in}{1.768054in}}%
\pgfpathcurveto{\pgfqpoint{1.547466in}{1.768054in}}{\pgfqpoint{1.558065in}{1.772444in}}{\pgfqpoint{1.565878in}{1.780258in}}%
\pgfpathcurveto{\pgfqpoint{1.573692in}{1.788071in}}{\pgfqpoint{1.578082in}{1.798670in}}{\pgfqpoint{1.578082in}{1.809720in}}%
\pgfpathcurveto{\pgfqpoint{1.578082in}{1.820770in}}{\pgfqpoint{1.573692in}{1.831370in}}{\pgfqpoint{1.565878in}{1.839183in}}%
\pgfpathcurveto{\pgfqpoint{1.558065in}{1.846997in}}{\pgfqpoint{1.547466in}{1.851387in}}{\pgfqpoint{1.536415in}{1.851387in}}%
\pgfpathcurveto{\pgfqpoint{1.525365in}{1.851387in}}{\pgfqpoint{1.514766in}{1.846997in}}{\pgfqpoint{1.506953in}{1.839183in}}%
\pgfpathcurveto{\pgfqpoint{1.499139in}{1.831370in}}{\pgfqpoint{1.494749in}{1.820770in}}{\pgfqpoint{1.494749in}{1.809720in}}%
\pgfpathcurveto{\pgfqpoint{1.494749in}{1.798670in}}{\pgfqpoint{1.499139in}{1.788071in}}{\pgfqpoint{1.506953in}{1.780258in}}%
\pgfpathcurveto{\pgfqpoint{1.514766in}{1.772444in}}{\pgfqpoint{1.525365in}{1.768054in}}{\pgfqpoint{1.536415in}{1.768054in}}%
\pgfpathlineto{\pgfqpoint{1.536415in}{1.768054in}}%
\pgfpathclose%
\pgfusepath{stroke}%
\end{pgfscope}%
\begin{pgfscope}%
\pgfpathrectangle{\pgfqpoint{0.393053in}{0.375000in}}{\pgfqpoint{6.356833in}{5.175000in}}%
\pgfusepath{clip}%
\pgfsetbuttcap%
\pgfsetroundjoin%
\pgfsetlinewidth{1.003750pt}%
\definecolor{currentstroke}{rgb}{0.827451,0.827451,0.827451}%
\pgfsetstrokecolor{currentstroke}%
\pgfsetdash{}{0pt}%
\pgfpathmoveto{\pgfqpoint{0.537535in}{3.433752in}}%
\pgfpathcurveto{\pgfqpoint{0.548585in}{3.433752in}}{\pgfqpoint{0.559184in}{3.438142in}}{\pgfqpoint{0.566998in}{3.445955in}}%
\pgfpathcurveto{\pgfqpoint{0.574811in}{3.453769in}}{\pgfqpoint{0.579202in}{3.464368in}}{\pgfqpoint{0.579202in}{3.475418in}}%
\pgfpathcurveto{\pgfqpoint{0.579202in}{3.486468in}}{\pgfqpoint{0.574811in}{3.497067in}}{\pgfqpoint{0.566998in}{3.504881in}}%
\pgfpathcurveto{\pgfqpoint{0.559184in}{3.512695in}}{\pgfqpoint{0.548585in}{3.517085in}}{\pgfqpoint{0.537535in}{3.517085in}}%
\pgfpathcurveto{\pgfqpoint{0.526485in}{3.517085in}}{\pgfqpoint{0.515886in}{3.512695in}}{\pgfqpoint{0.508072in}{3.504881in}}%
\pgfpathcurveto{\pgfqpoint{0.500259in}{3.497067in}}{\pgfqpoint{0.495868in}{3.486468in}}{\pgfqpoint{0.495868in}{3.475418in}}%
\pgfpathcurveto{\pgfqpoint{0.495868in}{3.464368in}}{\pgfqpoint{0.500259in}{3.453769in}}{\pgfqpoint{0.508072in}{3.445955in}}%
\pgfpathcurveto{\pgfqpoint{0.515886in}{3.438142in}}{\pgfqpoint{0.526485in}{3.433752in}}{\pgfqpoint{0.537535in}{3.433752in}}%
\pgfpathlineto{\pgfqpoint{0.537535in}{3.433752in}}%
\pgfpathclose%
\pgfusepath{stroke}%
\end{pgfscope}%
\begin{pgfscope}%
\pgfpathrectangle{\pgfqpoint{0.393053in}{0.375000in}}{\pgfqpoint{6.356833in}{5.175000in}}%
\pgfusepath{clip}%
\pgfsetbuttcap%
\pgfsetroundjoin%
\pgfsetlinewidth{1.003750pt}%
\definecolor{currentstroke}{rgb}{0.827451,0.827451,0.827451}%
\pgfsetstrokecolor{currentstroke}%
\pgfsetdash{}{0pt}%
\pgfpathmoveto{\pgfqpoint{1.576544in}{1.754527in}}%
\pgfpathcurveto{\pgfqpoint{1.587594in}{1.754527in}}{\pgfqpoint{1.598193in}{1.758917in}}{\pgfqpoint{1.606007in}{1.766731in}}%
\pgfpathcurveto{\pgfqpoint{1.613821in}{1.774544in}}{\pgfqpoint{1.618211in}{1.785143in}}{\pgfqpoint{1.618211in}{1.796194in}}%
\pgfpathcurveto{\pgfqpoint{1.618211in}{1.807244in}}{\pgfqpoint{1.613821in}{1.817843in}}{\pgfqpoint{1.606007in}{1.825656in}}%
\pgfpathcurveto{\pgfqpoint{1.598193in}{1.833470in}}{\pgfqpoint{1.587594in}{1.837860in}}{\pgfqpoint{1.576544in}{1.837860in}}%
\pgfpathcurveto{\pgfqpoint{1.565494in}{1.837860in}}{\pgfqpoint{1.554895in}{1.833470in}}{\pgfqpoint{1.547082in}{1.825656in}}%
\pgfpathcurveto{\pgfqpoint{1.539268in}{1.817843in}}{\pgfqpoint{1.534878in}{1.807244in}}{\pgfqpoint{1.534878in}{1.796194in}}%
\pgfpathcurveto{\pgfqpoint{1.534878in}{1.785143in}}{\pgfqpoint{1.539268in}{1.774544in}}{\pgfqpoint{1.547082in}{1.766731in}}%
\pgfpathcurveto{\pgfqpoint{1.554895in}{1.758917in}}{\pgfqpoint{1.565494in}{1.754527in}}{\pgfqpoint{1.576544in}{1.754527in}}%
\pgfpathlineto{\pgfqpoint{1.576544in}{1.754527in}}%
\pgfpathclose%
\pgfusepath{stroke}%
\end{pgfscope}%
\begin{pgfscope}%
\pgfpathrectangle{\pgfqpoint{0.393053in}{0.375000in}}{\pgfqpoint{6.356833in}{5.175000in}}%
\pgfusepath{clip}%
\pgfsetbuttcap%
\pgfsetroundjoin%
\pgfsetlinewidth{1.003750pt}%
\definecolor{currentstroke}{rgb}{0.827451,0.827451,0.827451}%
\pgfsetstrokecolor{currentstroke}%
\pgfsetdash{}{0pt}%
\pgfpathmoveto{\pgfqpoint{1.306751in}{1.985193in}}%
\pgfpathcurveto{\pgfqpoint{1.317801in}{1.985193in}}{\pgfqpoint{1.328400in}{1.989583in}}{\pgfqpoint{1.336214in}{1.997397in}}%
\pgfpathcurveto{\pgfqpoint{1.344027in}{2.005210in}}{\pgfqpoint{1.348418in}{2.015809in}}{\pgfqpoint{1.348418in}{2.026859in}}%
\pgfpathcurveto{\pgfqpoint{1.348418in}{2.037909in}}{\pgfqpoint{1.344027in}{2.048509in}}{\pgfqpoint{1.336214in}{2.056322in}}%
\pgfpathcurveto{\pgfqpoint{1.328400in}{2.064136in}}{\pgfqpoint{1.317801in}{2.068526in}}{\pgfqpoint{1.306751in}{2.068526in}}%
\pgfpathcurveto{\pgfqpoint{1.295701in}{2.068526in}}{\pgfqpoint{1.285102in}{2.064136in}}{\pgfqpoint{1.277288in}{2.056322in}}%
\pgfpathcurveto{\pgfqpoint{1.269475in}{2.048509in}}{\pgfqpoint{1.265084in}{2.037909in}}{\pgfqpoint{1.265084in}{2.026859in}}%
\pgfpathcurveto{\pgfqpoint{1.265084in}{2.015809in}}{\pgfqpoint{1.269475in}{2.005210in}}{\pgfqpoint{1.277288in}{1.997397in}}%
\pgfpathcurveto{\pgfqpoint{1.285102in}{1.989583in}}{\pgfqpoint{1.295701in}{1.985193in}}{\pgfqpoint{1.306751in}{1.985193in}}%
\pgfpathlineto{\pgfqpoint{1.306751in}{1.985193in}}%
\pgfpathclose%
\pgfusepath{stroke}%
\end{pgfscope}%
\begin{pgfscope}%
\pgfpathrectangle{\pgfqpoint{0.393053in}{0.375000in}}{\pgfqpoint{6.356833in}{5.175000in}}%
\pgfusepath{clip}%
\pgfsetbuttcap%
\pgfsetroundjoin%
\pgfsetlinewidth{1.003750pt}%
\definecolor{currentstroke}{rgb}{0.827451,0.827451,0.827451}%
\pgfsetstrokecolor{currentstroke}%
\pgfsetdash{}{0pt}%
\pgfpathmoveto{\pgfqpoint{0.452170in}{3.855614in}}%
\pgfpathcurveto{\pgfqpoint{0.463220in}{3.855614in}}{\pgfqpoint{0.473819in}{3.860005in}}{\pgfqpoint{0.481633in}{3.867818in}}%
\pgfpathcurveto{\pgfqpoint{0.489447in}{3.875632in}}{\pgfqpoint{0.493837in}{3.886231in}}{\pgfqpoint{0.493837in}{3.897281in}}%
\pgfpathcurveto{\pgfqpoint{0.493837in}{3.908331in}}{\pgfqpoint{0.489447in}{3.918930in}}{\pgfqpoint{0.481633in}{3.926744in}}%
\pgfpathcurveto{\pgfqpoint{0.473819in}{3.934557in}}{\pgfqpoint{0.463220in}{3.938948in}}{\pgfqpoint{0.452170in}{3.938948in}}%
\pgfpathcurveto{\pgfqpoint{0.441120in}{3.938948in}}{\pgfqpoint{0.430521in}{3.934557in}}{\pgfqpoint{0.422707in}{3.926744in}}%
\pgfpathcurveto{\pgfqpoint{0.414894in}{3.918930in}}{\pgfqpoint{0.410503in}{3.908331in}}{\pgfqpoint{0.410503in}{3.897281in}}%
\pgfpathcurveto{\pgfqpoint{0.410503in}{3.886231in}}{\pgfqpoint{0.414894in}{3.875632in}}{\pgfqpoint{0.422707in}{3.867818in}}%
\pgfpathcurveto{\pgfqpoint{0.430521in}{3.860005in}}{\pgfqpoint{0.441120in}{3.855614in}}{\pgfqpoint{0.452170in}{3.855614in}}%
\pgfpathlineto{\pgfqpoint{0.452170in}{3.855614in}}%
\pgfpathclose%
\pgfusepath{stroke}%
\end{pgfscope}%
\begin{pgfscope}%
\pgfpathrectangle{\pgfqpoint{0.393053in}{0.375000in}}{\pgfqpoint{6.356833in}{5.175000in}}%
\pgfusepath{clip}%
\pgfsetbuttcap%
\pgfsetroundjoin%
\pgfsetlinewidth{1.003750pt}%
\definecolor{currentstroke}{rgb}{0.827451,0.827451,0.827451}%
\pgfsetstrokecolor{currentstroke}%
\pgfsetdash{}{0pt}%
\pgfpathmoveto{\pgfqpoint{2.366777in}{1.146192in}}%
\pgfpathcurveto{\pgfqpoint{2.377827in}{1.146192in}}{\pgfqpoint{2.388426in}{1.150582in}}{\pgfqpoint{2.396240in}{1.158396in}}%
\pgfpathcurveto{\pgfqpoint{2.404053in}{1.166210in}}{\pgfqpoint{2.408444in}{1.176809in}}{\pgfqpoint{2.408444in}{1.187859in}}%
\pgfpathcurveto{\pgfqpoint{2.408444in}{1.198909in}}{\pgfqpoint{2.404053in}{1.209508in}}{\pgfqpoint{2.396240in}{1.217322in}}%
\pgfpathcurveto{\pgfqpoint{2.388426in}{1.225135in}}{\pgfqpoint{2.377827in}{1.229525in}}{\pgfqpoint{2.366777in}{1.229525in}}%
\pgfpathcurveto{\pgfqpoint{2.355727in}{1.229525in}}{\pgfqpoint{2.345128in}{1.225135in}}{\pgfqpoint{2.337314in}{1.217322in}}%
\pgfpathcurveto{\pgfqpoint{2.329501in}{1.209508in}}{\pgfqpoint{2.325110in}{1.198909in}}{\pgfqpoint{2.325110in}{1.187859in}}%
\pgfpathcurveto{\pgfqpoint{2.325110in}{1.176809in}}{\pgfqpoint{2.329501in}{1.166210in}}{\pgfqpoint{2.337314in}{1.158396in}}%
\pgfpathcurveto{\pgfqpoint{2.345128in}{1.150582in}}{\pgfqpoint{2.355727in}{1.146192in}}{\pgfqpoint{2.366777in}{1.146192in}}%
\pgfpathlineto{\pgfqpoint{2.366777in}{1.146192in}}%
\pgfpathclose%
\pgfusepath{stroke}%
\end{pgfscope}%
\begin{pgfscope}%
\pgfpathrectangle{\pgfqpoint{0.393053in}{0.375000in}}{\pgfqpoint{6.356833in}{5.175000in}}%
\pgfusepath{clip}%
\pgfsetbuttcap%
\pgfsetroundjoin%
\pgfsetlinewidth{1.003750pt}%
\definecolor{currentstroke}{rgb}{0.827451,0.827451,0.827451}%
\pgfsetstrokecolor{currentstroke}%
\pgfsetdash{}{0pt}%
\pgfpathmoveto{\pgfqpoint{2.693530in}{0.950710in}}%
\pgfpathcurveto{\pgfqpoint{2.704580in}{0.950710in}}{\pgfqpoint{2.715179in}{0.955100in}}{\pgfqpoint{2.722993in}{0.962914in}}%
\pgfpathcurveto{\pgfqpoint{2.730806in}{0.970727in}}{\pgfqpoint{2.735197in}{0.981326in}}{\pgfqpoint{2.735197in}{0.992376in}}%
\pgfpathcurveto{\pgfqpoint{2.735197in}{1.003427in}}{\pgfqpoint{2.730806in}{1.014026in}}{\pgfqpoint{2.722993in}{1.021839in}}%
\pgfpathcurveto{\pgfqpoint{2.715179in}{1.029653in}}{\pgfqpoint{2.704580in}{1.034043in}}{\pgfqpoint{2.693530in}{1.034043in}}%
\pgfpathcurveto{\pgfqpoint{2.682480in}{1.034043in}}{\pgfqpoint{2.671881in}{1.029653in}}{\pgfqpoint{2.664067in}{1.021839in}}%
\pgfpathcurveto{\pgfqpoint{2.656254in}{1.014026in}}{\pgfqpoint{2.651863in}{1.003427in}}{\pgfqpoint{2.651863in}{0.992376in}}%
\pgfpathcurveto{\pgfqpoint{2.651863in}{0.981326in}}{\pgfqpoint{2.656254in}{0.970727in}}{\pgfqpoint{2.664067in}{0.962914in}}%
\pgfpathcurveto{\pgfqpoint{2.671881in}{0.955100in}}{\pgfqpoint{2.682480in}{0.950710in}}{\pgfqpoint{2.693530in}{0.950710in}}%
\pgfpathlineto{\pgfqpoint{2.693530in}{0.950710in}}%
\pgfpathclose%
\pgfusepath{stroke}%
\end{pgfscope}%
\begin{pgfscope}%
\pgfpathrectangle{\pgfqpoint{0.393053in}{0.375000in}}{\pgfqpoint{6.356833in}{5.175000in}}%
\pgfusepath{clip}%
\pgfsetbuttcap%
\pgfsetroundjoin%
\pgfsetlinewidth{1.003750pt}%
\definecolor{currentstroke}{rgb}{0.827451,0.827451,0.827451}%
\pgfsetstrokecolor{currentstroke}%
\pgfsetdash{}{0pt}%
\pgfpathmoveto{\pgfqpoint{3.308571in}{0.683593in}}%
\pgfpathcurveto{\pgfqpoint{3.319621in}{0.683593in}}{\pgfqpoint{3.330220in}{0.687984in}}{\pgfqpoint{3.338034in}{0.695797in}}%
\pgfpathcurveto{\pgfqpoint{3.345848in}{0.703611in}}{\pgfqpoint{3.350238in}{0.714210in}}{\pgfqpoint{3.350238in}{0.725260in}}%
\pgfpathcurveto{\pgfqpoint{3.350238in}{0.736310in}}{\pgfqpoint{3.345848in}{0.746909in}}{\pgfqpoint{3.338034in}{0.754723in}}%
\pgfpathcurveto{\pgfqpoint{3.330220in}{0.762537in}}{\pgfqpoint{3.319621in}{0.766927in}}{\pgfqpoint{3.308571in}{0.766927in}}%
\pgfpathcurveto{\pgfqpoint{3.297521in}{0.766927in}}{\pgfqpoint{3.286922in}{0.762537in}}{\pgfqpoint{3.279108in}{0.754723in}}%
\pgfpathcurveto{\pgfqpoint{3.271295in}{0.746909in}}{\pgfqpoint{3.266904in}{0.736310in}}{\pgfqpoint{3.266904in}{0.725260in}}%
\pgfpathcurveto{\pgfqpoint{3.266904in}{0.714210in}}{\pgfqpoint{3.271295in}{0.703611in}}{\pgfqpoint{3.279108in}{0.695797in}}%
\pgfpathcurveto{\pgfqpoint{3.286922in}{0.687984in}}{\pgfqpoint{3.297521in}{0.683593in}}{\pgfqpoint{3.308571in}{0.683593in}}%
\pgfpathlineto{\pgfqpoint{3.308571in}{0.683593in}}%
\pgfpathclose%
\pgfusepath{stroke}%
\end{pgfscope}%
\begin{pgfscope}%
\pgfpathrectangle{\pgfqpoint{0.393053in}{0.375000in}}{\pgfqpoint{6.356833in}{5.175000in}}%
\pgfusepath{clip}%
\pgfsetbuttcap%
\pgfsetroundjoin%
\pgfsetlinewidth{1.003750pt}%
\definecolor{currentstroke}{rgb}{0.827451,0.827451,0.827451}%
\pgfsetstrokecolor{currentstroke}%
\pgfsetdash{}{0pt}%
\pgfpathmoveto{\pgfqpoint{4.445943in}{0.425176in}}%
\pgfpathcurveto{\pgfqpoint{4.456993in}{0.425176in}}{\pgfqpoint{4.467592in}{0.429567in}}{\pgfqpoint{4.475406in}{0.437380in}}%
\pgfpathcurveto{\pgfqpoint{4.483219in}{0.445194in}}{\pgfqpoint{4.487610in}{0.455793in}}{\pgfqpoint{4.487610in}{0.466843in}}%
\pgfpathcurveto{\pgfqpoint{4.487610in}{0.477893in}}{\pgfqpoint{4.483219in}{0.488492in}}{\pgfqpoint{4.475406in}{0.496306in}}%
\pgfpathcurveto{\pgfqpoint{4.467592in}{0.504119in}}{\pgfqpoint{4.456993in}{0.508510in}}{\pgfqpoint{4.445943in}{0.508510in}}%
\pgfpathcurveto{\pgfqpoint{4.434893in}{0.508510in}}{\pgfqpoint{4.424294in}{0.504119in}}{\pgfqpoint{4.416480in}{0.496306in}}%
\pgfpathcurveto{\pgfqpoint{4.408667in}{0.488492in}}{\pgfqpoint{4.404276in}{0.477893in}}{\pgfqpoint{4.404276in}{0.466843in}}%
\pgfpathcurveto{\pgfqpoint{4.404276in}{0.455793in}}{\pgfqpoint{4.408667in}{0.445194in}}{\pgfqpoint{4.416480in}{0.437380in}}%
\pgfpathcurveto{\pgfqpoint{4.424294in}{0.429567in}}{\pgfqpoint{4.434893in}{0.425176in}}{\pgfqpoint{4.445943in}{0.425176in}}%
\pgfpathlineto{\pgfqpoint{4.445943in}{0.425176in}}%
\pgfpathclose%
\pgfusepath{stroke}%
\end{pgfscope}%
\begin{pgfscope}%
\pgfpathrectangle{\pgfqpoint{0.393053in}{0.375000in}}{\pgfqpoint{6.356833in}{5.175000in}}%
\pgfusepath{clip}%
\pgfsetbuttcap%
\pgfsetroundjoin%
\pgfsetlinewidth{1.003750pt}%
\definecolor{currentstroke}{rgb}{0.827451,0.827451,0.827451}%
\pgfsetstrokecolor{currentstroke}%
\pgfsetdash{}{0pt}%
\pgfpathmoveto{\pgfqpoint{3.967155in}{0.512418in}}%
\pgfpathcurveto{\pgfqpoint{3.978205in}{0.512418in}}{\pgfqpoint{3.988804in}{0.516808in}}{\pgfqpoint{3.996618in}{0.524622in}}%
\pgfpathcurveto{\pgfqpoint{4.004431in}{0.532435in}}{\pgfqpoint{4.008822in}{0.543034in}}{\pgfqpoint{4.008822in}{0.554084in}}%
\pgfpathcurveto{\pgfqpoint{4.008822in}{0.565134in}}{\pgfqpoint{4.004431in}{0.575733in}}{\pgfqpoint{3.996618in}{0.583547in}}%
\pgfpathcurveto{\pgfqpoint{3.988804in}{0.591361in}}{\pgfqpoint{3.978205in}{0.595751in}}{\pgfqpoint{3.967155in}{0.595751in}}%
\pgfpathcurveto{\pgfqpoint{3.956105in}{0.595751in}}{\pgfqpoint{3.945506in}{0.591361in}}{\pgfqpoint{3.937692in}{0.583547in}}%
\pgfpathcurveto{\pgfqpoint{3.929879in}{0.575733in}}{\pgfqpoint{3.925488in}{0.565134in}}{\pgfqpoint{3.925488in}{0.554084in}}%
\pgfpathcurveto{\pgfqpoint{3.925488in}{0.543034in}}{\pgfqpoint{3.929879in}{0.532435in}}{\pgfqpoint{3.937692in}{0.524622in}}%
\pgfpathcurveto{\pgfqpoint{3.945506in}{0.516808in}}{\pgfqpoint{3.956105in}{0.512418in}}{\pgfqpoint{3.967155in}{0.512418in}}%
\pgfpathlineto{\pgfqpoint{3.967155in}{0.512418in}}%
\pgfpathclose%
\pgfusepath{stroke}%
\end{pgfscope}%
\begin{pgfscope}%
\pgfpathrectangle{\pgfqpoint{0.393053in}{0.375000in}}{\pgfqpoint{6.356833in}{5.175000in}}%
\pgfusepath{clip}%
\pgfsetbuttcap%
\pgfsetroundjoin%
\pgfsetlinewidth{1.003750pt}%
\definecolor{currentstroke}{rgb}{0.827451,0.827451,0.827451}%
\pgfsetstrokecolor{currentstroke}%
\pgfsetdash{}{0pt}%
\pgfpathmoveto{\pgfqpoint{4.927239in}{0.379634in}}%
\pgfpathcurveto{\pgfqpoint{4.938289in}{0.379634in}}{\pgfqpoint{4.948888in}{0.384025in}}{\pgfqpoint{4.956702in}{0.391838in}}%
\pgfpathcurveto{\pgfqpoint{4.964516in}{0.399652in}}{\pgfqpoint{4.968906in}{0.410251in}}{\pgfqpoint{4.968906in}{0.421301in}}%
\pgfpathcurveto{\pgfqpoint{4.968906in}{0.432351in}}{\pgfqpoint{4.964516in}{0.442950in}}{\pgfqpoint{4.956702in}{0.450764in}}%
\pgfpathcurveto{\pgfqpoint{4.948888in}{0.458578in}}{\pgfqpoint{4.938289in}{0.462968in}}{\pgfqpoint{4.927239in}{0.462968in}}%
\pgfpathcurveto{\pgfqpoint{4.916189in}{0.462968in}}{\pgfqpoint{4.905590in}{0.458578in}}{\pgfqpoint{4.897776in}{0.450764in}}%
\pgfpathcurveto{\pgfqpoint{4.889963in}{0.442950in}}{\pgfqpoint{4.885572in}{0.432351in}}{\pgfqpoint{4.885572in}{0.421301in}}%
\pgfpathcurveto{\pgfqpoint{4.885572in}{0.410251in}}{\pgfqpoint{4.889963in}{0.399652in}}{\pgfqpoint{4.897776in}{0.391838in}}%
\pgfpathcurveto{\pgfqpoint{4.905590in}{0.384025in}}{\pgfqpoint{4.916189in}{0.379634in}}{\pgfqpoint{4.927239in}{0.379634in}}%
\pgfpathlineto{\pgfqpoint{4.927239in}{0.379634in}}%
\pgfpathclose%
\pgfusepath{stroke}%
\end{pgfscope}%
\begin{pgfscope}%
\pgfpathrectangle{\pgfqpoint{0.393053in}{0.375000in}}{\pgfqpoint{6.356833in}{5.175000in}}%
\pgfusepath{clip}%
\pgfsetbuttcap%
\pgfsetroundjoin%
\pgfsetlinewidth{1.003750pt}%
\definecolor{currentstroke}{rgb}{0.827451,0.827451,0.827451}%
\pgfsetstrokecolor{currentstroke}%
\pgfsetdash{}{0pt}%
\pgfpathmoveto{\pgfqpoint{2.352374in}{1.159706in}}%
\pgfpathcurveto{\pgfqpoint{2.363424in}{1.159706in}}{\pgfqpoint{2.374023in}{1.164096in}}{\pgfqpoint{2.381836in}{1.171910in}}%
\pgfpathcurveto{\pgfqpoint{2.389650in}{1.179724in}}{\pgfqpoint{2.394040in}{1.190323in}}{\pgfqpoint{2.394040in}{1.201373in}}%
\pgfpathcurveto{\pgfqpoint{2.394040in}{1.212423in}}{\pgfqpoint{2.389650in}{1.223022in}}{\pgfqpoint{2.381836in}{1.230835in}}%
\pgfpathcurveto{\pgfqpoint{2.374023in}{1.238649in}}{\pgfqpoint{2.363424in}{1.243039in}}{\pgfqpoint{2.352374in}{1.243039in}}%
\pgfpathcurveto{\pgfqpoint{2.341324in}{1.243039in}}{\pgfqpoint{2.330725in}{1.238649in}}{\pgfqpoint{2.322911in}{1.230835in}}%
\pgfpathcurveto{\pgfqpoint{2.315097in}{1.223022in}}{\pgfqpoint{2.310707in}{1.212423in}}{\pgfqpoint{2.310707in}{1.201373in}}%
\pgfpathcurveto{\pgfqpoint{2.310707in}{1.190323in}}{\pgfqpoint{2.315097in}{1.179724in}}{\pgfqpoint{2.322911in}{1.171910in}}%
\pgfpathcurveto{\pgfqpoint{2.330725in}{1.164096in}}{\pgfqpoint{2.341324in}{1.159706in}}{\pgfqpoint{2.352374in}{1.159706in}}%
\pgfpathlineto{\pgfqpoint{2.352374in}{1.159706in}}%
\pgfpathclose%
\pgfusepath{stroke}%
\end{pgfscope}%
\begin{pgfscope}%
\pgfpathrectangle{\pgfqpoint{0.393053in}{0.375000in}}{\pgfqpoint{6.356833in}{5.175000in}}%
\pgfusepath{clip}%
\pgfsetbuttcap%
\pgfsetroundjoin%
\pgfsetlinewidth{1.003750pt}%
\definecolor{currentstroke}{rgb}{0.827451,0.827451,0.827451}%
\pgfsetstrokecolor{currentstroke}%
\pgfsetdash{}{0pt}%
\pgfpathmoveto{\pgfqpoint{1.768437in}{1.558678in}}%
\pgfpathcurveto{\pgfqpoint{1.779488in}{1.558678in}}{\pgfqpoint{1.790087in}{1.563068in}}{\pgfqpoint{1.797900in}{1.570882in}}%
\pgfpathcurveto{\pgfqpoint{1.805714in}{1.578696in}}{\pgfqpoint{1.810104in}{1.589295in}}{\pgfqpoint{1.810104in}{1.600345in}}%
\pgfpathcurveto{\pgfqpoint{1.810104in}{1.611395in}}{\pgfqpoint{1.805714in}{1.621994in}}{\pgfqpoint{1.797900in}{1.629807in}}%
\pgfpathcurveto{\pgfqpoint{1.790087in}{1.637621in}}{\pgfqpoint{1.779488in}{1.642011in}}{\pgfqpoint{1.768437in}{1.642011in}}%
\pgfpathcurveto{\pgfqpoint{1.757387in}{1.642011in}}{\pgfqpoint{1.746788in}{1.637621in}}{\pgfqpoint{1.738975in}{1.629807in}}%
\pgfpathcurveto{\pgfqpoint{1.731161in}{1.621994in}}{\pgfqpoint{1.726771in}{1.611395in}}{\pgfqpoint{1.726771in}{1.600345in}}%
\pgfpathcurveto{\pgfqpoint{1.726771in}{1.589295in}}{\pgfqpoint{1.731161in}{1.578696in}}{\pgfqpoint{1.738975in}{1.570882in}}%
\pgfpathcurveto{\pgfqpoint{1.746788in}{1.563068in}}{\pgfqpoint{1.757387in}{1.558678in}}{\pgfqpoint{1.768437in}{1.558678in}}%
\pgfpathlineto{\pgfqpoint{1.768437in}{1.558678in}}%
\pgfpathclose%
\pgfusepath{stroke}%
\end{pgfscope}%
\begin{pgfscope}%
\pgfpathrectangle{\pgfqpoint{0.393053in}{0.375000in}}{\pgfqpoint{6.356833in}{5.175000in}}%
\pgfusepath{clip}%
\pgfsetbuttcap%
\pgfsetroundjoin%
\pgfsetlinewidth{1.003750pt}%
\definecolor{currentstroke}{rgb}{0.827451,0.827451,0.827451}%
\pgfsetstrokecolor{currentstroke}%
\pgfsetdash{}{0pt}%
\pgfpathmoveto{\pgfqpoint{0.525326in}{3.470591in}}%
\pgfpathcurveto{\pgfqpoint{0.536376in}{3.470591in}}{\pgfqpoint{0.546975in}{3.474982in}}{\pgfqpoint{0.554788in}{3.482795in}}%
\pgfpathcurveto{\pgfqpoint{0.562602in}{3.490609in}}{\pgfqpoint{0.566992in}{3.501208in}}{\pgfqpoint{0.566992in}{3.512258in}}%
\pgfpathcurveto{\pgfqpoint{0.566992in}{3.523308in}}{\pgfqpoint{0.562602in}{3.533907in}}{\pgfqpoint{0.554788in}{3.541721in}}%
\pgfpathcurveto{\pgfqpoint{0.546975in}{3.549534in}}{\pgfqpoint{0.536376in}{3.553925in}}{\pgfqpoint{0.525326in}{3.553925in}}%
\pgfpathcurveto{\pgfqpoint{0.514275in}{3.553925in}}{\pgfqpoint{0.503676in}{3.549534in}}{\pgfqpoint{0.495863in}{3.541721in}}%
\pgfpathcurveto{\pgfqpoint{0.488049in}{3.533907in}}{\pgfqpoint{0.483659in}{3.523308in}}{\pgfqpoint{0.483659in}{3.512258in}}%
\pgfpathcurveto{\pgfqpoint{0.483659in}{3.501208in}}{\pgfqpoint{0.488049in}{3.490609in}}{\pgfqpoint{0.495863in}{3.482795in}}%
\pgfpathcurveto{\pgfqpoint{0.503676in}{3.474982in}}{\pgfqpoint{0.514275in}{3.470591in}}{\pgfqpoint{0.525326in}{3.470591in}}%
\pgfpathlineto{\pgfqpoint{0.525326in}{3.470591in}}%
\pgfpathclose%
\pgfusepath{stroke}%
\end{pgfscope}%
\begin{pgfscope}%
\pgfpathrectangle{\pgfqpoint{0.393053in}{0.375000in}}{\pgfqpoint{6.356833in}{5.175000in}}%
\pgfusepath{clip}%
\pgfsetbuttcap%
\pgfsetroundjoin%
\pgfsetlinewidth{1.003750pt}%
\definecolor{currentstroke}{rgb}{0.827451,0.827451,0.827451}%
\pgfsetstrokecolor{currentstroke}%
\pgfsetdash{}{0pt}%
\pgfpathmoveto{\pgfqpoint{4.275388in}{0.441212in}}%
\pgfpathcurveto{\pgfqpoint{4.286438in}{0.441212in}}{\pgfqpoint{4.297038in}{0.445602in}}{\pgfqpoint{4.304851in}{0.453416in}}%
\pgfpathcurveto{\pgfqpoint{4.312665in}{0.461229in}}{\pgfqpoint{4.317055in}{0.471828in}}{\pgfqpoint{4.317055in}{0.482878in}}%
\pgfpathcurveto{\pgfqpoint{4.317055in}{0.493928in}}{\pgfqpoint{4.312665in}{0.504527in}}{\pgfqpoint{4.304851in}{0.512341in}}%
\pgfpathcurveto{\pgfqpoint{4.297038in}{0.520155in}}{\pgfqpoint{4.286438in}{0.524545in}}{\pgfqpoint{4.275388in}{0.524545in}}%
\pgfpathcurveto{\pgfqpoint{4.264338in}{0.524545in}}{\pgfqpoint{4.253739in}{0.520155in}}{\pgfqpoint{4.245926in}{0.512341in}}%
\pgfpathcurveto{\pgfqpoint{4.238112in}{0.504527in}}{\pgfqpoint{4.233722in}{0.493928in}}{\pgfqpoint{4.233722in}{0.482878in}}%
\pgfpathcurveto{\pgfqpoint{4.233722in}{0.471828in}}{\pgfqpoint{4.238112in}{0.461229in}}{\pgfqpoint{4.245926in}{0.453416in}}%
\pgfpathcurveto{\pgfqpoint{4.253739in}{0.445602in}}{\pgfqpoint{4.264338in}{0.441212in}}{\pgfqpoint{4.275388in}{0.441212in}}%
\pgfpathlineto{\pgfqpoint{4.275388in}{0.441212in}}%
\pgfpathclose%
\pgfusepath{stroke}%
\end{pgfscope}%
\begin{pgfscope}%
\pgfpathrectangle{\pgfqpoint{0.393053in}{0.375000in}}{\pgfqpoint{6.356833in}{5.175000in}}%
\pgfusepath{clip}%
\pgfsetbuttcap%
\pgfsetroundjoin%
\pgfsetlinewidth{1.003750pt}%
\definecolor{currentstroke}{rgb}{0.827451,0.827451,0.827451}%
\pgfsetstrokecolor{currentstroke}%
\pgfsetdash{}{0pt}%
\pgfpathmoveto{\pgfqpoint{1.911625in}{1.440877in}}%
\pgfpathcurveto{\pgfqpoint{1.922675in}{1.440877in}}{\pgfqpoint{1.933274in}{1.445267in}}{\pgfqpoint{1.941087in}{1.453081in}}%
\pgfpathcurveto{\pgfqpoint{1.948901in}{1.460894in}}{\pgfqpoint{1.953291in}{1.471493in}}{\pgfqpoint{1.953291in}{1.482543in}}%
\pgfpathcurveto{\pgfqpoint{1.953291in}{1.493594in}}{\pgfqpoint{1.948901in}{1.504193in}}{\pgfqpoint{1.941087in}{1.512006in}}%
\pgfpathcurveto{\pgfqpoint{1.933274in}{1.519820in}}{\pgfqpoint{1.922675in}{1.524210in}}{\pgfqpoint{1.911625in}{1.524210in}}%
\pgfpathcurveto{\pgfqpoint{1.900574in}{1.524210in}}{\pgfqpoint{1.889975in}{1.519820in}}{\pgfqpoint{1.882162in}{1.512006in}}%
\pgfpathcurveto{\pgfqpoint{1.874348in}{1.504193in}}{\pgfqpoint{1.869958in}{1.493594in}}{\pgfqpoint{1.869958in}{1.482543in}}%
\pgfpathcurveto{\pgfqpoint{1.869958in}{1.471493in}}{\pgfqpoint{1.874348in}{1.460894in}}{\pgfqpoint{1.882162in}{1.453081in}}%
\pgfpathcurveto{\pgfqpoint{1.889975in}{1.445267in}}{\pgfqpoint{1.900574in}{1.440877in}}{\pgfqpoint{1.911625in}{1.440877in}}%
\pgfpathlineto{\pgfqpoint{1.911625in}{1.440877in}}%
\pgfpathclose%
\pgfusepath{stroke}%
\end{pgfscope}%
\begin{pgfscope}%
\pgfpathrectangle{\pgfqpoint{0.393053in}{0.375000in}}{\pgfqpoint{6.356833in}{5.175000in}}%
\pgfusepath{clip}%
\pgfsetbuttcap%
\pgfsetroundjoin%
\pgfsetlinewidth{1.003750pt}%
\definecolor{currentstroke}{rgb}{0.827451,0.827451,0.827451}%
\pgfsetstrokecolor{currentstroke}%
\pgfsetdash{}{0pt}%
\pgfpathmoveto{\pgfqpoint{0.435142in}{3.936498in}}%
\pgfpathcurveto{\pgfqpoint{0.446192in}{3.936498in}}{\pgfqpoint{0.456791in}{3.940888in}}{\pgfqpoint{0.464605in}{3.948702in}}%
\pgfpathcurveto{\pgfqpoint{0.472419in}{3.956516in}}{\pgfqpoint{0.476809in}{3.967115in}}{\pgfqpoint{0.476809in}{3.978165in}}%
\pgfpathcurveto{\pgfqpoint{0.476809in}{3.989215in}}{\pgfqpoint{0.472419in}{3.999814in}}{\pgfqpoint{0.464605in}{4.007627in}}%
\pgfpathcurveto{\pgfqpoint{0.456791in}{4.015441in}}{\pgfqpoint{0.446192in}{4.019831in}}{\pgfqpoint{0.435142in}{4.019831in}}%
\pgfpathcurveto{\pgfqpoint{0.424092in}{4.019831in}}{\pgfqpoint{0.413493in}{4.015441in}}{\pgfqpoint{0.405679in}{4.007627in}}%
\pgfpathcurveto{\pgfqpoint{0.397866in}{3.999814in}}{\pgfqpoint{0.393475in}{3.989215in}}{\pgfqpoint{0.393475in}{3.978165in}}%
\pgfpathcurveto{\pgfqpoint{0.393475in}{3.967115in}}{\pgfqpoint{0.397866in}{3.956516in}}{\pgfqpoint{0.405679in}{3.948702in}}%
\pgfpathcurveto{\pgfqpoint{0.413493in}{3.940888in}}{\pgfqpoint{0.424092in}{3.936498in}}{\pgfqpoint{0.435142in}{3.936498in}}%
\pgfpathlineto{\pgfqpoint{0.435142in}{3.936498in}}%
\pgfpathclose%
\pgfusepath{stroke}%
\end{pgfscope}%
\begin{pgfscope}%
\pgfpathrectangle{\pgfqpoint{0.393053in}{0.375000in}}{\pgfqpoint{6.356833in}{5.175000in}}%
\pgfusepath{clip}%
\pgfsetbuttcap%
\pgfsetroundjoin%
\pgfsetlinewidth{1.003750pt}%
\definecolor{currentstroke}{rgb}{0.827451,0.827451,0.827451}%
\pgfsetstrokecolor{currentstroke}%
\pgfsetdash{}{0pt}%
\pgfpathmoveto{\pgfqpoint{2.437419in}{1.106016in}}%
\pgfpathcurveto{\pgfqpoint{2.448469in}{1.106016in}}{\pgfqpoint{2.459069in}{1.110406in}}{\pgfqpoint{2.466882in}{1.118219in}}%
\pgfpathcurveto{\pgfqpoint{2.474696in}{1.126033in}}{\pgfqpoint{2.479086in}{1.136632in}}{\pgfqpoint{2.479086in}{1.147682in}}%
\pgfpathcurveto{\pgfqpoint{2.479086in}{1.158732in}}{\pgfqpoint{2.474696in}{1.169331in}}{\pgfqpoint{2.466882in}{1.177145in}}%
\pgfpathcurveto{\pgfqpoint{2.459069in}{1.184959in}}{\pgfqpoint{2.448469in}{1.189349in}}{\pgfqpoint{2.437419in}{1.189349in}}%
\pgfpathcurveto{\pgfqpoint{2.426369in}{1.189349in}}{\pgfqpoint{2.415770in}{1.184959in}}{\pgfqpoint{2.407957in}{1.177145in}}%
\pgfpathcurveto{\pgfqpoint{2.400143in}{1.169331in}}{\pgfqpoint{2.395753in}{1.158732in}}{\pgfqpoint{2.395753in}{1.147682in}}%
\pgfpathcurveto{\pgfqpoint{2.395753in}{1.136632in}}{\pgfqpoint{2.400143in}{1.126033in}}{\pgfqpoint{2.407957in}{1.118219in}}%
\pgfpathcurveto{\pgfqpoint{2.415770in}{1.110406in}}{\pgfqpoint{2.426369in}{1.106016in}}{\pgfqpoint{2.437419in}{1.106016in}}%
\pgfpathlineto{\pgfqpoint{2.437419in}{1.106016in}}%
\pgfpathclose%
\pgfusepath{stroke}%
\end{pgfscope}%
\begin{pgfscope}%
\pgfpathrectangle{\pgfqpoint{0.393053in}{0.375000in}}{\pgfqpoint{6.356833in}{5.175000in}}%
\pgfusepath{clip}%
\pgfsetbuttcap%
\pgfsetroundjoin%
\pgfsetlinewidth{1.003750pt}%
\definecolor{currentstroke}{rgb}{0.827451,0.827451,0.827451}%
\pgfsetstrokecolor{currentstroke}%
\pgfsetdash{}{0pt}%
\pgfpathmoveto{\pgfqpoint{1.882270in}{1.458195in}}%
\pgfpathcurveto{\pgfqpoint{1.893321in}{1.458195in}}{\pgfqpoint{1.903920in}{1.462585in}}{\pgfqpoint{1.911733in}{1.470399in}}%
\pgfpathcurveto{\pgfqpoint{1.919547in}{1.478213in}}{\pgfqpoint{1.923937in}{1.488812in}}{\pgfqpoint{1.923937in}{1.499862in}}%
\pgfpathcurveto{\pgfqpoint{1.923937in}{1.510912in}}{\pgfqpoint{1.919547in}{1.521511in}}{\pgfqpoint{1.911733in}{1.529325in}}%
\pgfpathcurveto{\pgfqpoint{1.903920in}{1.537138in}}{\pgfqpoint{1.893321in}{1.541529in}}{\pgfqpoint{1.882270in}{1.541529in}}%
\pgfpathcurveto{\pgfqpoint{1.871220in}{1.541529in}}{\pgfqpoint{1.860621in}{1.537138in}}{\pgfqpoint{1.852808in}{1.529325in}}%
\pgfpathcurveto{\pgfqpoint{1.844994in}{1.521511in}}{\pgfqpoint{1.840604in}{1.510912in}}{\pgfqpoint{1.840604in}{1.499862in}}%
\pgfpathcurveto{\pgfqpoint{1.840604in}{1.488812in}}{\pgfqpoint{1.844994in}{1.478213in}}{\pgfqpoint{1.852808in}{1.470399in}}%
\pgfpathcurveto{\pgfqpoint{1.860621in}{1.462585in}}{\pgfqpoint{1.871220in}{1.458195in}}{\pgfqpoint{1.882270in}{1.458195in}}%
\pgfpathlineto{\pgfqpoint{1.882270in}{1.458195in}}%
\pgfpathclose%
\pgfusepath{stroke}%
\end{pgfscope}%
\begin{pgfscope}%
\pgfpathrectangle{\pgfqpoint{0.393053in}{0.375000in}}{\pgfqpoint{6.356833in}{5.175000in}}%
\pgfusepath{clip}%
\pgfsetbuttcap%
\pgfsetroundjoin%
\pgfsetlinewidth{1.003750pt}%
\definecolor{currentstroke}{rgb}{0.827451,0.827451,0.827451}%
\pgfsetstrokecolor{currentstroke}%
\pgfsetdash{}{0pt}%
\pgfpathmoveto{\pgfqpoint{2.287965in}{1.196822in}}%
\pgfpathcurveto{\pgfqpoint{2.299015in}{1.196822in}}{\pgfqpoint{2.309614in}{1.201212in}}{\pgfqpoint{2.317427in}{1.209026in}}%
\pgfpathcurveto{\pgfqpoint{2.325241in}{1.216839in}}{\pgfqpoint{2.329631in}{1.227438in}}{\pgfqpoint{2.329631in}{1.238488in}}%
\pgfpathcurveto{\pgfqpoint{2.329631in}{1.249539in}}{\pgfqpoint{2.325241in}{1.260138in}}{\pgfqpoint{2.317427in}{1.267951in}}%
\pgfpathcurveto{\pgfqpoint{2.309614in}{1.275765in}}{\pgfqpoint{2.299015in}{1.280155in}}{\pgfqpoint{2.287965in}{1.280155in}}%
\pgfpathcurveto{\pgfqpoint{2.276915in}{1.280155in}}{\pgfqpoint{2.266316in}{1.275765in}}{\pgfqpoint{2.258502in}{1.267951in}}%
\pgfpathcurveto{\pgfqpoint{2.250688in}{1.260138in}}{\pgfqpoint{2.246298in}{1.249539in}}{\pgfqpoint{2.246298in}{1.238488in}}%
\pgfpathcurveto{\pgfqpoint{2.246298in}{1.227438in}}{\pgfqpoint{2.250688in}{1.216839in}}{\pgfqpoint{2.258502in}{1.209026in}}%
\pgfpathcurveto{\pgfqpoint{2.266316in}{1.201212in}}{\pgfqpoint{2.276915in}{1.196822in}}{\pgfqpoint{2.287965in}{1.196822in}}%
\pgfpathlineto{\pgfqpoint{2.287965in}{1.196822in}}%
\pgfpathclose%
\pgfusepath{stroke}%
\end{pgfscope}%
\begin{pgfscope}%
\pgfpathrectangle{\pgfqpoint{0.393053in}{0.375000in}}{\pgfqpoint{6.356833in}{5.175000in}}%
\pgfusepath{clip}%
\pgfsetbuttcap%
\pgfsetroundjoin%
\pgfsetlinewidth{1.003750pt}%
\definecolor{currentstroke}{rgb}{0.827451,0.827451,0.827451}%
\pgfsetstrokecolor{currentstroke}%
\pgfsetdash{}{0pt}%
\pgfpathmoveto{\pgfqpoint{0.429666in}{3.982254in}}%
\pgfpathcurveto{\pgfqpoint{0.440716in}{3.982254in}}{\pgfqpoint{0.451315in}{3.986644in}}{\pgfqpoint{0.459129in}{3.994458in}}%
\pgfpathcurveto{\pgfqpoint{0.466942in}{4.002271in}}{\pgfqpoint{0.471332in}{4.012870in}}{\pgfqpoint{0.471332in}{4.023920in}}%
\pgfpathcurveto{\pgfqpoint{0.471332in}{4.034970in}}{\pgfqpoint{0.466942in}{4.045570in}}{\pgfqpoint{0.459129in}{4.053383in}}%
\pgfpathcurveto{\pgfqpoint{0.451315in}{4.061197in}}{\pgfqpoint{0.440716in}{4.065587in}}{\pgfqpoint{0.429666in}{4.065587in}}%
\pgfpathcurveto{\pgfqpoint{0.418616in}{4.065587in}}{\pgfqpoint{0.408017in}{4.061197in}}{\pgfqpoint{0.400203in}{4.053383in}}%
\pgfpathcurveto{\pgfqpoint{0.392389in}{4.045570in}}{\pgfqpoint{0.387999in}{4.034970in}}{\pgfqpoint{0.387999in}{4.023920in}}%
\pgfpathcurveto{\pgfqpoint{0.387999in}{4.012870in}}{\pgfqpoint{0.392389in}{4.002271in}}{\pgfqpoint{0.400203in}{3.994458in}}%
\pgfpathcurveto{\pgfqpoint{0.408017in}{3.986644in}}{\pgfqpoint{0.418616in}{3.982254in}}{\pgfqpoint{0.429666in}{3.982254in}}%
\pgfpathlineto{\pgfqpoint{0.429666in}{3.982254in}}%
\pgfpathclose%
\pgfusepath{stroke}%
\end{pgfscope}%
\begin{pgfscope}%
\pgfpathrectangle{\pgfqpoint{0.393053in}{0.375000in}}{\pgfqpoint{6.356833in}{5.175000in}}%
\pgfusepath{clip}%
\pgfsetbuttcap%
\pgfsetroundjoin%
\pgfsetlinewidth{1.003750pt}%
\definecolor{currentstroke}{rgb}{0.827451,0.827451,0.827451}%
\pgfsetstrokecolor{currentstroke}%
\pgfsetdash{}{0pt}%
\pgfpathmoveto{\pgfqpoint{0.601770in}{3.214331in}}%
\pgfpathcurveto{\pgfqpoint{0.612820in}{3.214331in}}{\pgfqpoint{0.623419in}{3.218722in}}{\pgfqpoint{0.631233in}{3.226535in}}%
\pgfpathcurveto{\pgfqpoint{0.639046in}{3.234349in}}{\pgfqpoint{0.643437in}{3.244948in}}{\pgfqpoint{0.643437in}{3.255998in}}%
\pgfpathcurveto{\pgfqpoint{0.643437in}{3.267048in}}{\pgfqpoint{0.639046in}{3.277647in}}{\pgfqpoint{0.631233in}{3.285461in}}%
\pgfpathcurveto{\pgfqpoint{0.623419in}{3.293275in}}{\pgfqpoint{0.612820in}{3.297665in}}{\pgfqpoint{0.601770in}{3.297665in}}%
\pgfpathcurveto{\pgfqpoint{0.590720in}{3.297665in}}{\pgfqpoint{0.580121in}{3.293275in}}{\pgfqpoint{0.572307in}{3.285461in}}%
\pgfpathcurveto{\pgfqpoint{0.564494in}{3.277647in}}{\pgfqpoint{0.560103in}{3.267048in}}{\pgfqpoint{0.560103in}{3.255998in}}%
\pgfpathcurveto{\pgfqpoint{0.560103in}{3.244948in}}{\pgfqpoint{0.564494in}{3.234349in}}{\pgfqpoint{0.572307in}{3.226535in}}%
\pgfpathcurveto{\pgfqpoint{0.580121in}{3.218722in}}{\pgfqpoint{0.590720in}{3.214331in}}{\pgfqpoint{0.601770in}{3.214331in}}%
\pgfpathlineto{\pgfqpoint{0.601770in}{3.214331in}}%
\pgfpathclose%
\pgfusepath{stroke}%
\end{pgfscope}%
\begin{pgfscope}%
\pgfpathrectangle{\pgfqpoint{0.393053in}{0.375000in}}{\pgfqpoint{6.356833in}{5.175000in}}%
\pgfusepath{clip}%
\pgfsetbuttcap%
\pgfsetroundjoin%
\pgfsetlinewidth{1.003750pt}%
\definecolor{currentstroke}{rgb}{0.827451,0.827451,0.827451}%
\pgfsetstrokecolor{currentstroke}%
\pgfsetdash{}{0pt}%
\pgfpathmoveto{\pgfqpoint{2.477308in}{1.087549in}}%
\pgfpathcurveto{\pgfqpoint{2.488358in}{1.087549in}}{\pgfqpoint{2.498957in}{1.091940in}}{\pgfqpoint{2.506771in}{1.099753in}}%
\pgfpathcurveto{\pgfqpoint{2.514584in}{1.107567in}}{\pgfqpoint{2.518974in}{1.118166in}}{\pgfqpoint{2.518974in}{1.129216in}}%
\pgfpathcurveto{\pgfqpoint{2.518974in}{1.140266in}}{\pgfqpoint{2.514584in}{1.150865in}}{\pgfqpoint{2.506771in}{1.158679in}}%
\pgfpathcurveto{\pgfqpoint{2.498957in}{1.166492in}}{\pgfqpoint{2.488358in}{1.170883in}}{\pgfqpoint{2.477308in}{1.170883in}}%
\pgfpathcurveto{\pgfqpoint{2.466258in}{1.170883in}}{\pgfqpoint{2.455659in}{1.166492in}}{\pgfqpoint{2.447845in}{1.158679in}}%
\pgfpathcurveto{\pgfqpoint{2.440031in}{1.150865in}}{\pgfqpoint{2.435641in}{1.140266in}}{\pgfqpoint{2.435641in}{1.129216in}}%
\pgfpathcurveto{\pgfqpoint{2.435641in}{1.118166in}}{\pgfqpoint{2.440031in}{1.107567in}}{\pgfqpoint{2.447845in}{1.099753in}}%
\pgfpathcurveto{\pgfqpoint{2.455659in}{1.091940in}}{\pgfqpoint{2.466258in}{1.087549in}}{\pgfqpoint{2.477308in}{1.087549in}}%
\pgfpathlineto{\pgfqpoint{2.477308in}{1.087549in}}%
\pgfpathclose%
\pgfusepath{stroke}%
\end{pgfscope}%
\begin{pgfscope}%
\pgfpathrectangle{\pgfqpoint{0.393053in}{0.375000in}}{\pgfqpoint{6.356833in}{5.175000in}}%
\pgfusepath{clip}%
\pgfsetbuttcap%
\pgfsetroundjoin%
\pgfsetlinewidth{1.003750pt}%
\definecolor{currentstroke}{rgb}{0.827451,0.827451,0.827451}%
\pgfsetstrokecolor{currentstroke}%
\pgfsetdash{}{0pt}%
\pgfpathmoveto{\pgfqpoint{1.167227in}{2.196399in}}%
\pgfpathcurveto{\pgfqpoint{1.178278in}{2.196399in}}{\pgfqpoint{1.188877in}{2.200789in}}{\pgfqpoint{1.196690in}{2.208603in}}%
\pgfpathcurveto{\pgfqpoint{1.204504in}{2.216416in}}{\pgfqpoint{1.208894in}{2.227016in}}{\pgfqpoint{1.208894in}{2.238066in}}%
\pgfpathcurveto{\pgfqpoint{1.208894in}{2.249116in}}{\pgfqpoint{1.204504in}{2.259715in}}{\pgfqpoint{1.196690in}{2.267528in}}%
\pgfpathcurveto{\pgfqpoint{1.188877in}{2.275342in}}{\pgfqpoint{1.178278in}{2.279732in}}{\pgfqpoint{1.167227in}{2.279732in}}%
\pgfpathcurveto{\pgfqpoint{1.156177in}{2.279732in}}{\pgfqpoint{1.145578in}{2.275342in}}{\pgfqpoint{1.137765in}{2.267528in}}%
\pgfpathcurveto{\pgfqpoint{1.129951in}{2.259715in}}{\pgfqpoint{1.125561in}{2.249116in}}{\pgfqpoint{1.125561in}{2.238066in}}%
\pgfpathcurveto{\pgfqpoint{1.125561in}{2.227016in}}{\pgfqpoint{1.129951in}{2.216416in}}{\pgfqpoint{1.137765in}{2.208603in}}%
\pgfpathcurveto{\pgfqpoint{1.145578in}{2.200789in}}{\pgfqpoint{1.156177in}{2.196399in}}{\pgfqpoint{1.167227in}{2.196399in}}%
\pgfpathlineto{\pgfqpoint{1.167227in}{2.196399in}}%
\pgfpathclose%
\pgfusepath{stroke}%
\end{pgfscope}%
\begin{pgfscope}%
\pgfpathrectangle{\pgfqpoint{0.393053in}{0.375000in}}{\pgfqpoint{6.356833in}{5.175000in}}%
\pgfusepath{clip}%
\pgfsetbuttcap%
\pgfsetroundjoin%
\pgfsetlinewidth{1.003750pt}%
\definecolor{currentstroke}{rgb}{0.827451,0.827451,0.827451}%
\pgfsetstrokecolor{currentstroke}%
\pgfsetdash{}{0pt}%
\pgfpathmoveto{\pgfqpoint{1.291595in}{1.999205in}}%
\pgfpathcurveto{\pgfqpoint{1.302646in}{1.999205in}}{\pgfqpoint{1.313245in}{2.003595in}}{\pgfqpoint{1.321058in}{2.011409in}}%
\pgfpathcurveto{\pgfqpoint{1.328872in}{2.019222in}}{\pgfqpoint{1.333262in}{2.029821in}}{\pgfqpoint{1.333262in}{2.040872in}}%
\pgfpathcurveto{\pgfqpoint{1.333262in}{2.051922in}}{\pgfqpoint{1.328872in}{2.062521in}}{\pgfqpoint{1.321058in}{2.070334in}}%
\pgfpathcurveto{\pgfqpoint{1.313245in}{2.078148in}}{\pgfqpoint{1.302646in}{2.082538in}}{\pgfqpoint{1.291595in}{2.082538in}}%
\pgfpathcurveto{\pgfqpoint{1.280545in}{2.082538in}}{\pgfqpoint{1.269946in}{2.078148in}}{\pgfqpoint{1.262133in}{2.070334in}}%
\pgfpathcurveto{\pgfqpoint{1.254319in}{2.062521in}}{\pgfqpoint{1.249929in}{2.051922in}}{\pgfqpoint{1.249929in}{2.040872in}}%
\pgfpathcurveto{\pgfqpoint{1.249929in}{2.029821in}}{\pgfqpoint{1.254319in}{2.019222in}}{\pgfqpoint{1.262133in}{2.011409in}}%
\pgfpathcurveto{\pgfqpoint{1.269946in}{2.003595in}}{\pgfqpoint{1.280545in}{1.999205in}}{\pgfqpoint{1.291595in}{1.999205in}}%
\pgfpathlineto{\pgfqpoint{1.291595in}{1.999205in}}%
\pgfpathclose%
\pgfusepath{stroke}%
\end{pgfscope}%
\begin{pgfscope}%
\pgfpathrectangle{\pgfqpoint{0.393053in}{0.375000in}}{\pgfqpoint{6.356833in}{5.175000in}}%
\pgfusepath{clip}%
\pgfsetbuttcap%
\pgfsetroundjoin%
\pgfsetlinewidth{1.003750pt}%
\definecolor{currentstroke}{rgb}{0.827451,0.827451,0.827451}%
\pgfsetstrokecolor{currentstroke}%
\pgfsetdash{}{0pt}%
\pgfpathmoveto{\pgfqpoint{2.585362in}{1.005073in}}%
\pgfpathcurveto{\pgfqpoint{2.596412in}{1.005073in}}{\pgfqpoint{2.607011in}{1.009463in}}{\pgfqpoint{2.614824in}{1.017277in}}%
\pgfpathcurveto{\pgfqpoint{2.622638in}{1.025090in}}{\pgfqpoint{2.627028in}{1.035689in}}{\pgfqpoint{2.627028in}{1.046740in}}%
\pgfpathcurveto{\pgfqpoint{2.627028in}{1.057790in}}{\pgfqpoint{2.622638in}{1.068389in}}{\pgfqpoint{2.614824in}{1.076202in}}%
\pgfpathcurveto{\pgfqpoint{2.607011in}{1.084016in}}{\pgfqpoint{2.596412in}{1.088406in}}{\pgfqpoint{2.585362in}{1.088406in}}%
\pgfpathcurveto{\pgfqpoint{2.574311in}{1.088406in}}{\pgfqpoint{2.563712in}{1.084016in}}{\pgfqpoint{2.555899in}{1.076202in}}%
\pgfpathcurveto{\pgfqpoint{2.548085in}{1.068389in}}{\pgfqpoint{2.543695in}{1.057790in}}{\pgfqpoint{2.543695in}{1.046740in}}%
\pgfpathcurveto{\pgfqpoint{2.543695in}{1.035689in}}{\pgfqpoint{2.548085in}{1.025090in}}{\pgfqpoint{2.555899in}{1.017277in}}%
\pgfpathcurveto{\pgfqpoint{2.563712in}{1.009463in}}{\pgfqpoint{2.574311in}{1.005073in}}{\pgfqpoint{2.585362in}{1.005073in}}%
\pgfpathlineto{\pgfqpoint{2.585362in}{1.005073in}}%
\pgfpathclose%
\pgfusepath{stroke}%
\end{pgfscope}%
\begin{pgfscope}%
\pgfpathrectangle{\pgfqpoint{0.393053in}{0.375000in}}{\pgfqpoint{6.356833in}{5.175000in}}%
\pgfusepath{clip}%
\pgfsetbuttcap%
\pgfsetroundjoin%
\pgfsetlinewidth{1.003750pt}%
\definecolor{currentstroke}{rgb}{0.827451,0.827451,0.827451}%
\pgfsetstrokecolor{currentstroke}%
\pgfsetdash{}{0pt}%
\pgfpathmoveto{\pgfqpoint{1.768411in}{1.566057in}}%
\pgfpathcurveto{\pgfqpoint{1.779461in}{1.566057in}}{\pgfqpoint{1.790060in}{1.570447in}}{\pgfqpoint{1.797874in}{1.578261in}}%
\pgfpathcurveto{\pgfqpoint{1.805687in}{1.586074in}}{\pgfqpoint{1.810077in}{1.596673in}}{\pgfqpoint{1.810077in}{1.607723in}}%
\pgfpathcurveto{\pgfqpoint{1.810077in}{1.618773in}}{\pgfqpoint{1.805687in}{1.629373in}}{\pgfqpoint{1.797874in}{1.637186in}}%
\pgfpathcurveto{\pgfqpoint{1.790060in}{1.645000in}}{\pgfqpoint{1.779461in}{1.649390in}}{\pgfqpoint{1.768411in}{1.649390in}}%
\pgfpathcurveto{\pgfqpoint{1.757361in}{1.649390in}}{\pgfqpoint{1.746762in}{1.645000in}}{\pgfqpoint{1.738948in}{1.637186in}}%
\pgfpathcurveto{\pgfqpoint{1.731134in}{1.629373in}}{\pgfqpoint{1.726744in}{1.618773in}}{\pgfqpoint{1.726744in}{1.607723in}}%
\pgfpathcurveto{\pgfqpoint{1.726744in}{1.596673in}}{\pgfqpoint{1.731134in}{1.586074in}}{\pgfqpoint{1.738948in}{1.578261in}}%
\pgfpathcurveto{\pgfqpoint{1.746762in}{1.570447in}}{\pgfqpoint{1.757361in}{1.566057in}}{\pgfqpoint{1.768411in}{1.566057in}}%
\pgfpathlineto{\pgfqpoint{1.768411in}{1.566057in}}%
\pgfpathclose%
\pgfusepath{stroke}%
\end{pgfscope}%
\begin{pgfscope}%
\pgfpathrectangle{\pgfqpoint{0.393053in}{0.375000in}}{\pgfqpoint{6.356833in}{5.175000in}}%
\pgfusepath{clip}%
\pgfsetbuttcap%
\pgfsetroundjoin%
\pgfsetlinewidth{1.003750pt}%
\definecolor{currentstroke}{rgb}{0.827451,0.827451,0.827451}%
\pgfsetstrokecolor{currentstroke}%
\pgfsetdash{}{0pt}%
\pgfpathmoveto{\pgfqpoint{1.375643in}{1.906128in}}%
\pgfpathcurveto{\pgfqpoint{1.386694in}{1.906128in}}{\pgfqpoint{1.397293in}{1.910518in}}{\pgfqpoint{1.405106in}{1.918331in}}%
\pgfpathcurveto{\pgfqpoint{1.412920in}{1.926145in}}{\pgfqpoint{1.417310in}{1.936744in}}{\pgfqpoint{1.417310in}{1.947794in}}%
\pgfpathcurveto{\pgfqpoint{1.417310in}{1.958844in}}{\pgfqpoint{1.412920in}{1.969443in}}{\pgfqpoint{1.405106in}{1.977257in}}%
\pgfpathcurveto{\pgfqpoint{1.397293in}{1.985071in}}{\pgfqpoint{1.386694in}{1.989461in}}{\pgfqpoint{1.375643in}{1.989461in}}%
\pgfpathcurveto{\pgfqpoint{1.364593in}{1.989461in}}{\pgfqpoint{1.353994in}{1.985071in}}{\pgfqpoint{1.346181in}{1.977257in}}%
\pgfpathcurveto{\pgfqpoint{1.338367in}{1.969443in}}{\pgfqpoint{1.333977in}{1.958844in}}{\pgfqpoint{1.333977in}{1.947794in}}%
\pgfpathcurveto{\pgfqpoint{1.333977in}{1.936744in}}{\pgfqpoint{1.338367in}{1.926145in}}{\pgfqpoint{1.346181in}{1.918331in}}%
\pgfpathcurveto{\pgfqpoint{1.353994in}{1.910518in}}{\pgfqpoint{1.364593in}{1.906128in}}{\pgfqpoint{1.375643in}{1.906128in}}%
\pgfpathlineto{\pgfqpoint{1.375643in}{1.906128in}}%
\pgfpathclose%
\pgfusepath{stroke}%
\end{pgfscope}%
\begin{pgfscope}%
\pgfpathrectangle{\pgfqpoint{0.393053in}{0.375000in}}{\pgfqpoint{6.356833in}{5.175000in}}%
\pgfusepath{clip}%
\pgfsetbuttcap%
\pgfsetroundjoin%
\pgfsetlinewidth{1.003750pt}%
\definecolor{currentstroke}{rgb}{0.827451,0.827451,0.827451}%
\pgfsetstrokecolor{currentstroke}%
\pgfsetdash{}{0pt}%
\pgfpathmoveto{\pgfqpoint{0.424181in}{4.027299in}}%
\pgfpathcurveto{\pgfqpoint{0.435231in}{4.027299in}}{\pgfqpoint{0.445830in}{4.031689in}}{\pgfqpoint{0.453644in}{4.039503in}}%
\pgfpathcurveto{\pgfqpoint{0.461458in}{4.047316in}}{\pgfqpoint{0.465848in}{4.057915in}}{\pgfqpoint{0.465848in}{4.068966in}}%
\pgfpathcurveto{\pgfqpoint{0.465848in}{4.080016in}}{\pgfqpoint{0.461458in}{4.090615in}}{\pgfqpoint{0.453644in}{4.098428in}}%
\pgfpathcurveto{\pgfqpoint{0.445830in}{4.106242in}}{\pgfqpoint{0.435231in}{4.110632in}}{\pgfqpoint{0.424181in}{4.110632in}}%
\pgfpathcurveto{\pgfqpoint{0.413131in}{4.110632in}}{\pgfqpoint{0.402532in}{4.106242in}}{\pgfqpoint{0.394719in}{4.098428in}}%
\pgfpathcurveto{\pgfqpoint{0.386905in}{4.090615in}}{\pgfqpoint{0.382515in}{4.080016in}}{\pgfqpoint{0.382515in}{4.068966in}}%
\pgfpathcurveto{\pgfqpoint{0.382515in}{4.057915in}}{\pgfqpoint{0.386905in}{4.047316in}}{\pgfqpoint{0.394719in}{4.039503in}}%
\pgfpathcurveto{\pgfqpoint{0.402532in}{4.031689in}}{\pgfqpoint{0.413131in}{4.027299in}}{\pgfqpoint{0.424181in}{4.027299in}}%
\pgfpathlineto{\pgfqpoint{0.424181in}{4.027299in}}%
\pgfpathclose%
\pgfusepath{stroke}%
\end{pgfscope}%
\begin{pgfscope}%
\pgfpathrectangle{\pgfqpoint{0.393053in}{0.375000in}}{\pgfqpoint{6.356833in}{5.175000in}}%
\pgfusepath{clip}%
\pgfsetbuttcap%
\pgfsetroundjoin%
\pgfsetlinewidth{1.003750pt}%
\definecolor{currentstroke}{rgb}{0.827451,0.827451,0.827451}%
\pgfsetstrokecolor{currentstroke}%
\pgfsetdash{}{0pt}%
\pgfpathmoveto{\pgfqpoint{0.765086in}{2.832905in}}%
\pgfpathcurveto{\pgfqpoint{0.776136in}{2.832905in}}{\pgfqpoint{0.786735in}{2.837296in}}{\pgfqpoint{0.794549in}{2.845109in}}%
\pgfpathcurveto{\pgfqpoint{0.802362in}{2.852923in}}{\pgfqpoint{0.806753in}{2.863522in}}{\pgfqpoint{0.806753in}{2.874572in}}%
\pgfpathcurveto{\pgfqpoint{0.806753in}{2.885622in}}{\pgfqpoint{0.802362in}{2.896221in}}{\pgfqpoint{0.794549in}{2.904035in}}%
\pgfpathcurveto{\pgfqpoint{0.786735in}{2.911849in}}{\pgfqpoint{0.776136in}{2.916239in}}{\pgfqpoint{0.765086in}{2.916239in}}%
\pgfpathcurveto{\pgfqpoint{0.754036in}{2.916239in}}{\pgfqpoint{0.743437in}{2.911849in}}{\pgfqpoint{0.735623in}{2.904035in}}%
\pgfpathcurveto{\pgfqpoint{0.727810in}{2.896221in}}{\pgfqpoint{0.723419in}{2.885622in}}{\pgfqpoint{0.723419in}{2.874572in}}%
\pgfpathcurveto{\pgfqpoint{0.723419in}{2.863522in}}{\pgfqpoint{0.727810in}{2.852923in}}{\pgfqpoint{0.735623in}{2.845109in}}%
\pgfpathcurveto{\pgfqpoint{0.743437in}{2.837296in}}{\pgfqpoint{0.754036in}{2.832905in}}{\pgfqpoint{0.765086in}{2.832905in}}%
\pgfpathlineto{\pgfqpoint{0.765086in}{2.832905in}}%
\pgfpathclose%
\pgfusepath{stroke}%
\end{pgfscope}%
\begin{pgfscope}%
\pgfpathrectangle{\pgfqpoint{0.393053in}{0.375000in}}{\pgfqpoint{6.356833in}{5.175000in}}%
\pgfusepath{clip}%
\pgfsetbuttcap%
\pgfsetroundjoin%
\pgfsetlinewidth{1.003750pt}%
\definecolor{currentstroke}{rgb}{0.827451,0.827451,0.827451}%
\pgfsetstrokecolor{currentstroke}%
\pgfsetdash{}{0pt}%
\pgfpathmoveto{\pgfqpoint{0.397999in}{4.365942in}}%
\pgfpathcurveto{\pgfqpoint{0.409049in}{4.365942in}}{\pgfqpoint{0.419648in}{4.370333in}}{\pgfqpoint{0.427461in}{4.378146in}}%
\pgfpathcurveto{\pgfqpoint{0.435275in}{4.385960in}}{\pgfqpoint{0.439665in}{4.396559in}}{\pgfqpoint{0.439665in}{4.407609in}}%
\pgfpathcurveto{\pgfqpoint{0.439665in}{4.418659in}}{\pgfqpoint{0.435275in}{4.429258in}}{\pgfqpoint{0.427461in}{4.437072in}}%
\pgfpathcurveto{\pgfqpoint{0.419648in}{4.444885in}}{\pgfqpoint{0.409049in}{4.449276in}}{\pgfqpoint{0.397999in}{4.449276in}}%
\pgfpathcurveto{\pgfqpoint{0.386948in}{4.449276in}}{\pgfqpoint{0.376349in}{4.444885in}}{\pgfqpoint{0.368536in}{4.437072in}}%
\pgfpathcurveto{\pgfqpoint{0.360722in}{4.429258in}}{\pgfqpoint{0.356332in}{4.418659in}}{\pgfqpoint{0.356332in}{4.407609in}}%
\pgfpathcurveto{\pgfqpoint{0.356332in}{4.396559in}}{\pgfqpoint{0.360722in}{4.385960in}}{\pgfqpoint{0.368536in}{4.378146in}}%
\pgfpathcurveto{\pgfqpoint{0.376349in}{4.370333in}}{\pgfqpoint{0.386948in}{4.365942in}}{\pgfqpoint{0.397999in}{4.365942in}}%
\pgfpathlineto{\pgfqpoint{0.397999in}{4.365942in}}%
\pgfpathclose%
\pgfusepath{stroke}%
\end{pgfscope}%
\begin{pgfscope}%
\pgfpathrectangle{\pgfqpoint{0.393053in}{0.375000in}}{\pgfqpoint{6.356833in}{5.175000in}}%
\pgfusepath{clip}%
\pgfsetbuttcap%
\pgfsetroundjoin%
\pgfsetlinewidth{1.003750pt}%
\definecolor{currentstroke}{rgb}{0.827451,0.827451,0.827451}%
\pgfsetstrokecolor{currentstroke}%
\pgfsetdash{}{0pt}%
\pgfpathmoveto{\pgfqpoint{2.000452in}{1.370955in}}%
\pgfpathcurveto{\pgfqpoint{2.011502in}{1.370955in}}{\pgfqpoint{2.022101in}{1.375345in}}{\pgfqpoint{2.029914in}{1.383159in}}%
\pgfpathcurveto{\pgfqpoint{2.037728in}{1.390972in}}{\pgfqpoint{2.042118in}{1.401571in}}{\pgfqpoint{2.042118in}{1.412622in}}%
\pgfpathcurveto{\pgfqpoint{2.042118in}{1.423672in}}{\pgfqpoint{2.037728in}{1.434271in}}{\pgfqpoint{2.029914in}{1.442084in}}%
\pgfpathcurveto{\pgfqpoint{2.022101in}{1.449898in}}{\pgfqpoint{2.011502in}{1.454288in}}{\pgfqpoint{2.000452in}{1.454288in}}%
\pgfpathcurveto{\pgfqpoint{1.989401in}{1.454288in}}{\pgfqpoint{1.978802in}{1.449898in}}{\pgfqpoint{1.970989in}{1.442084in}}%
\pgfpathcurveto{\pgfqpoint{1.963175in}{1.434271in}}{\pgfqpoint{1.958785in}{1.423672in}}{\pgfqpoint{1.958785in}{1.412622in}}%
\pgfpathcurveto{\pgfqpoint{1.958785in}{1.401571in}}{\pgfqpoint{1.963175in}{1.390972in}}{\pgfqpoint{1.970989in}{1.383159in}}%
\pgfpathcurveto{\pgfqpoint{1.978802in}{1.375345in}}{\pgfqpoint{1.989401in}{1.370955in}}{\pgfqpoint{2.000452in}{1.370955in}}%
\pgfpathlineto{\pgfqpoint{2.000452in}{1.370955in}}%
\pgfpathclose%
\pgfusepath{stroke}%
\end{pgfscope}%
\begin{pgfscope}%
\pgfpathrectangle{\pgfqpoint{0.393053in}{0.375000in}}{\pgfqpoint{6.356833in}{5.175000in}}%
\pgfusepath{clip}%
\pgfsetbuttcap%
\pgfsetroundjoin%
\pgfsetlinewidth{1.003750pt}%
\definecolor{currentstroke}{rgb}{0.827451,0.827451,0.827451}%
\pgfsetstrokecolor{currentstroke}%
\pgfsetdash{}{0pt}%
\pgfpathmoveto{\pgfqpoint{1.858747in}{1.496619in}}%
\pgfpathcurveto{\pgfqpoint{1.869797in}{1.496619in}}{\pgfqpoint{1.880396in}{1.501009in}}{\pgfqpoint{1.888209in}{1.508823in}}%
\pgfpathcurveto{\pgfqpoint{1.896023in}{1.516636in}}{\pgfqpoint{1.900413in}{1.527236in}}{\pgfqpoint{1.900413in}{1.538286in}}%
\pgfpathcurveto{\pgfqpoint{1.900413in}{1.549336in}}{\pgfqpoint{1.896023in}{1.559935in}}{\pgfqpoint{1.888209in}{1.567748in}}%
\pgfpathcurveto{\pgfqpoint{1.880396in}{1.575562in}}{\pgfqpoint{1.869797in}{1.579952in}}{\pgfqpoint{1.858747in}{1.579952in}}%
\pgfpathcurveto{\pgfqpoint{1.847696in}{1.579952in}}{\pgfqpoint{1.837097in}{1.575562in}}{\pgfqpoint{1.829284in}{1.567748in}}%
\pgfpathcurveto{\pgfqpoint{1.821470in}{1.559935in}}{\pgfqpoint{1.817080in}{1.549336in}}{\pgfqpoint{1.817080in}{1.538286in}}%
\pgfpathcurveto{\pgfqpoint{1.817080in}{1.527236in}}{\pgfqpoint{1.821470in}{1.516636in}}{\pgfqpoint{1.829284in}{1.508823in}}%
\pgfpathcurveto{\pgfqpoint{1.837097in}{1.501009in}}{\pgfqpoint{1.847696in}{1.496619in}}{\pgfqpoint{1.858747in}{1.496619in}}%
\pgfpathlineto{\pgfqpoint{1.858747in}{1.496619in}}%
\pgfpathclose%
\pgfusepath{stroke}%
\end{pgfscope}%
\begin{pgfscope}%
\pgfpathrectangle{\pgfqpoint{0.393053in}{0.375000in}}{\pgfqpoint{6.356833in}{5.175000in}}%
\pgfusepath{clip}%
\pgfsetbuttcap%
\pgfsetroundjoin%
\pgfsetlinewidth{1.003750pt}%
\definecolor{currentstroke}{rgb}{0.827451,0.827451,0.827451}%
\pgfsetstrokecolor{currentstroke}%
\pgfsetdash{}{0pt}%
\pgfpathmoveto{\pgfqpoint{2.498084in}{1.056522in}}%
\pgfpathcurveto{\pgfqpoint{2.509134in}{1.056522in}}{\pgfqpoint{2.519733in}{1.060912in}}{\pgfqpoint{2.527547in}{1.068726in}}%
\pgfpathcurveto{\pgfqpoint{2.535360in}{1.076539in}}{\pgfqpoint{2.539750in}{1.087138in}}{\pgfqpoint{2.539750in}{1.098188in}}%
\pgfpathcurveto{\pgfqpoint{2.539750in}{1.109239in}}{\pgfqpoint{2.535360in}{1.119838in}}{\pgfqpoint{2.527547in}{1.127651in}}%
\pgfpathcurveto{\pgfqpoint{2.519733in}{1.135465in}}{\pgfqpoint{2.509134in}{1.139855in}}{\pgfqpoint{2.498084in}{1.139855in}}%
\pgfpathcurveto{\pgfqpoint{2.487034in}{1.139855in}}{\pgfqpoint{2.476435in}{1.135465in}}{\pgfqpoint{2.468621in}{1.127651in}}%
\pgfpathcurveto{\pgfqpoint{2.460807in}{1.119838in}}{\pgfqpoint{2.456417in}{1.109239in}}{\pgfqpoint{2.456417in}{1.098188in}}%
\pgfpathcurveto{\pgfqpoint{2.456417in}{1.087138in}}{\pgfqpoint{2.460807in}{1.076539in}}{\pgfqpoint{2.468621in}{1.068726in}}%
\pgfpathcurveto{\pgfqpoint{2.476435in}{1.060912in}}{\pgfqpoint{2.487034in}{1.056522in}}{\pgfqpoint{2.498084in}{1.056522in}}%
\pgfpathlineto{\pgfqpoint{2.498084in}{1.056522in}}%
\pgfpathclose%
\pgfusepath{stroke}%
\end{pgfscope}%
\begin{pgfscope}%
\pgfpathrectangle{\pgfqpoint{0.393053in}{0.375000in}}{\pgfqpoint{6.356833in}{5.175000in}}%
\pgfusepath{clip}%
\pgfsetbuttcap%
\pgfsetroundjoin%
\pgfsetlinewidth{1.003750pt}%
\definecolor{currentstroke}{rgb}{0.827451,0.827451,0.827451}%
\pgfsetstrokecolor{currentstroke}%
\pgfsetdash{}{0pt}%
\pgfpathmoveto{\pgfqpoint{1.078224in}{2.311634in}}%
\pgfpathcurveto{\pgfqpoint{1.089274in}{2.311634in}}{\pgfqpoint{1.099873in}{2.316024in}}{\pgfqpoint{1.107686in}{2.323838in}}%
\pgfpathcurveto{\pgfqpoint{1.115500in}{2.331651in}}{\pgfqpoint{1.119890in}{2.342250in}}{\pgfqpoint{1.119890in}{2.353300in}}%
\pgfpathcurveto{\pgfqpoint{1.119890in}{2.364351in}}{\pgfqpoint{1.115500in}{2.374950in}}{\pgfqpoint{1.107686in}{2.382763in}}%
\pgfpathcurveto{\pgfqpoint{1.099873in}{2.390577in}}{\pgfqpoint{1.089274in}{2.394967in}}{\pgfqpoint{1.078224in}{2.394967in}}%
\pgfpathcurveto{\pgfqpoint{1.067173in}{2.394967in}}{\pgfqpoint{1.056574in}{2.390577in}}{\pgfqpoint{1.048761in}{2.382763in}}%
\pgfpathcurveto{\pgfqpoint{1.040947in}{2.374950in}}{\pgfqpoint{1.036557in}{2.364351in}}{\pgfqpoint{1.036557in}{2.353300in}}%
\pgfpathcurveto{\pgfqpoint{1.036557in}{2.342250in}}{\pgfqpoint{1.040947in}{2.331651in}}{\pgfqpoint{1.048761in}{2.323838in}}%
\pgfpathcurveto{\pgfqpoint{1.056574in}{2.316024in}}{\pgfqpoint{1.067173in}{2.311634in}}{\pgfqpoint{1.078224in}{2.311634in}}%
\pgfpathlineto{\pgfqpoint{1.078224in}{2.311634in}}%
\pgfpathclose%
\pgfusepath{stroke}%
\end{pgfscope}%
\begin{pgfscope}%
\pgfpathrectangle{\pgfqpoint{0.393053in}{0.375000in}}{\pgfqpoint{6.356833in}{5.175000in}}%
\pgfusepath{clip}%
\pgfsetbuttcap%
\pgfsetroundjoin%
\pgfsetlinewidth{1.003750pt}%
\definecolor{currentstroke}{rgb}{0.827451,0.827451,0.827451}%
\pgfsetstrokecolor{currentstroke}%
\pgfsetdash{}{0pt}%
\pgfpathmoveto{\pgfqpoint{1.176673in}{2.141426in}}%
\pgfpathcurveto{\pgfqpoint{1.187723in}{2.141426in}}{\pgfqpoint{1.198322in}{2.145816in}}{\pgfqpoint{1.206136in}{2.153630in}}%
\pgfpathcurveto{\pgfqpoint{1.213949in}{2.161444in}}{\pgfqpoint{1.218340in}{2.172043in}}{\pgfqpoint{1.218340in}{2.183093in}}%
\pgfpathcurveto{\pgfqpoint{1.218340in}{2.194143in}}{\pgfqpoint{1.213949in}{2.204742in}}{\pgfqpoint{1.206136in}{2.212556in}}%
\pgfpathcurveto{\pgfqpoint{1.198322in}{2.220369in}}{\pgfqpoint{1.187723in}{2.224759in}}{\pgfqpoint{1.176673in}{2.224759in}}%
\pgfpathcurveto{\pgfqpoint{1.165623in}{2.224759in}}{\pgfqpoint{1.155024in}{2.220369in}}{\pgfqpoint{1.147210in}{2.212556in}}%
\pgfpathcurveto{\pgfqpoint{1.139396in}{2.204742in}}{\pgfqpoint{1.135006in}{2.194143in}}{\pgfqpoint{1.135006in}{2.183093in}}%
\pgfpathcurveto{\pgfqpoint{1.135006in}{2.172043in}}{\pgfqpoint{1.139396in}{2.161444in}}{\pgfqpoint{1.147210in}{2.153630in}}%
\pgfpathcurveto{\pgfqpoint{1.155024in}{2.145816in}}{\pgfqpoint{1.165623in}{2.141426in}}{\pgfqpoint{1.176673in}{2.141426in}}%
\pgfpathlineto{\pgfqpoint{1.176673in}{2.141426in}}%
\pgfpathclose%
\pgfusepath{stroke}%
\end{pgfscope}%
\begin{pgfscope}%
\pgfpathrectangle{\pgfqpoint{0.393053in}{0.375000in}}{\pgfqpoint{6.356833in}{5.175000in}}%
\pgfusepath{clip}%
\pgfsetbuttcap%
\pgfsetroundjoin%
\pgfsetlinewidth{1.003750pt}%
\definecolor{currentstroke}{rgb}{0.827451,0.827451,0.827451}%
\pgfsetstrokecolor{currentstroke}%
\pgfsetdash{}{0pt}%
\pgfpathmoveto{\pgfqpoint{2.525895in}{1.037541in}}%
\pgfpathcurveto{\pgfqpoint{2.536945in}{1.037541in}}{\pgfqpoint{2.547544in}{1.041932in}}{\pgfqpoint{2.555358in}{1.049745in}}%
\pgfpathcurveto{\pgfqpoint{2.563172in}{1.057559in}}{\pgfqpoint{2.567562in}{1.068158in}}{\pgfqpoint{2.567562in}{1.079208in}}%
\pgfpathcurveto{\pgfqpoint{2.567562in}{1.090258in}}{\pgfqpoint{2.563172in}{1.100857in}}{\pgfqpoint{2.555358in}{1.108671in}}%
\pgfpathcurveto{\pgfqpoint{2.547544in}{1.116484in}}{\pgfqpoint{2.536945in}{1.120875in}}{\pgfqpoint{2.525895in}{1.120875in}}%
\pgfpathcurveto{\pgfqpoint{2.514845in}{1.120875in}}{\pgfqpoint{2.504246in}{1.116484in}}{\pgfqpoint{2.496432in}{1.108671in}}%
\pgfpathcurveto{\pgfqpoint{2.488619in}{1.100857in}}{\pgfqpoint{2.484228in}{1.090258in}}{\pgfqpoint{2.484228in}{1.079208in}}%
\pgfpathcurveto{\pgfqpoint{2.484228in}{1.068158in}}{\pgfqpoint{2.488619in}{1.057559in}}{\pgfqpoint{2.496432in}{1.049745in}}%
\pgfpathcurveto{\pgfqpoint{2.504246in}{1.041932in}}{\pgfqpoint{2.514845in}{1.037541in}}{\pgfqpoint{2.525895in}{1.037541in}}%
\pgfpathlineto{\pgfqpoint{2.525895in}{1.037541in}}%
\pgfpathclose%
\pgfusepath{stroke}%
\end{pgfscope}%
\begin{pgfscope}%
\pgfpathrectangle{\pgfqpoint{0.393053in}{0.375000in}}{\pgfqpoint{6.356833in}{5.175000in}}%
\pgfusepath{clip}%
\pgfsetbuttcap%
\pgfsetroundjoin%
\pgfsetlinewidth{1.003750pt}%
\definecolor{currentstroke}{rgb}{0.827451,0.827451,0.827451}%
\pgfsetstrokecolor{currentstroke}%
\pgfsetdash{}{0pt}%
\pgfpathmoveto{\pgfqpoint{1.093823in}{2.259190in}}%
\pgfpathcurveto{\pgfqpoint{1.104873in}{2.259190in}}{\pgfqpoint{1.115472in}{2.263580in}}{\pgfqpoint{1.123286in}{2.271394in}}%
\pgfpathcurveto{\pgfqpoint{1.131099in}{2.279207in}}{\pgfqpoint{1.135489in}{2.289806in}}{\pgfqpoint{1.135489in}{2.300856in}}%
\pgfpathcurveto{\pgfqpoint{1.135489in}{2.311906in}}{\pgfqpoint{1.131099in}{2.322505in}}{\pgfqpoint{1.123286in}{2.330319in}}%
\pgfpathcurveto{\pgfqpoint{1.115472in}{2.338133in}}{\pgfqpoint{1.104873in}{2.342523in}}{\pgfqpoint{1.093823in}{2.342523in}}%
\pgfpathcurveto{\pgfqpoint{1.082773in}{2.342523in}}{\pgfqpoint{1.072174in}{2.338133in}}{\pgfqpoint{1.064360in}{2.330319in}}%
\pgfpathcurveto{\pgfqpoint{1.056546in}{2.322505in}}{\pgfqpoint{1.052156in}{2.311906in}}{\pgfqpoint{1.052156in}{2.300856in}}%
\pgfpathcurveto{\pgfqpoint{1.052156in}{2.289806in}}{\pgfqpoint{1.056546in}{2.279207in}}{\pgfqpoint{1.064360in}{2.271394in}}%
\pgfpathcurveto{\pgfqpoint{1.072174in}{2.263580in}}{\pgfqpoint{1.082773in}{2.259190in}}{\pgfqpoint{1.093823in}{2.259190in}}%
\pgfpathlineto{\pgfqpoint{1.093823in}{2.259190in}}%
\pgfpathclose%
\pgfusepath{stroke}%
\end{pgfscope}%
\begin{pgfscope}%
\pgfpathrectangle{\pgfqpoint{0.393053in}{0.375000in}}{\pgfqpoint{6.356833in}{5.175000in}}%
\pgfusepath{clip}%
\pgfsetbuttcap%
\pgfsetroundjoin%
\pgfsetlinewidth{1.003750pt}%
\definecolor{currentstroke}{rgb}{0.827451,0.827451,0.827451}%
\pgfsetstrokecolor{currentstroke}%
\pgfsetdash{}{0pt}%
\pgfpathmoveto{\pgfqpoint{0.612319in}{3.180269in}}%
\pgfpathcurveto{\pgfqpoint{0.623369in}{3.180269in}}{\pgfqpoint{0.633968in}{3.184660in}}{\pgfqpoint{0.641781in}{3.192473in}}%
\pgfpathcurveto{\pgfqpoint{0.649595in}{3.200287in}}{\pgfqpoint{0.653985in}{3.210886in}}{\pgfqpoint{0.653985in}{3.221936in}}%
\pgfpathcurveto{\pgfqpoint{0.653985in}{3.232986in}}{\pgfqpoint{0.649595in}{3.243585in}}{\pgfqpoint{0.641781in}{3.251399in}}%
\pgfpathcurveto{\pgfqpoint{0.633968in}{3.259212in}}{\pgfqpoint{0.623369in}{3.263603in}}{\pgfqpoint{0.612319in}{3.263603in}}%
\pgfpathcurveto{\pgfqpoint{0.601269in}{3.263603in}}{\pgfqpoint{0.590669in}{3.259212in}}{\pgfqpoint{0.582856in}{3.251399in}}%
\pgfpathcurveto{\pgfqpoint{0.575042in}{3.243585in}}{\pgfqpoint{0.570652in}{3.232986in}}{\pgfqpoint{0.570652in}{3.221936in}}%
\pgfpathcurveto{\pgfqpoint{0.570652in}{3.210886in}}{\pgfqpoint{0.575042in}{3.200287in}}{\pgfqpoint{0.582856in}{3.192473in}}%
\pgfpathcurveto{\pgfqpoint{0.590669in}{3.184660in}}{\pgfqpoint{0.601269in}{3.180269in}}{\pgfqpoint{0.612319in}{3.180269in}}%
\pgfpathlineto{\pgfqpoint{0.612319in}{3.180269in}}%
\pgfpathclose%
\pgfusepath{stroke}%
\end{pgfscope}%
\begin{pgfscope}%
\pgfpathrectangle{\pgfqpoint{0.393053in}{0.375000in}}{\pgfqpoint{6.356833in}{5.175000in}}%
\pgfusepath{clip}%
\pgfsetbuttcap%
\pgfsetroundjoin%
\pgfsetlinewidth{1.003750pt}%
\definecolor{currentstroke}{rgb}{0.827451,0.827451,0.827451}%
\pgfsetstrokecolor{currentstroke}%
\pgfsetdash{}{0pt}%
\pgfpathmoveto{\pgfqpoint{1.256856in}{2.039646in}}%
\pgfpathcurveto{\pgfqpoint{1.267906in}{2.039646in}}{\pgfqpoint{1.278505in}{2.044036in}}{\pgfqpoint{1.286319in}{2.051849in}}%
\pgfpathcurveto{\pgfqpoint{1.294132in}{2.059663in}}{\pgfqpoint{1.298523in}{2.070262in}}{\pgfqpoint{1.298523in}{2.081312in}}%
\pgfpathcurveto{\pgfqpoint{1.298523in}{2.092362in}}{\pgfqpoint{1.294132in}{2.102961in}}{\pgfqpoint{1.286319in}{2.110775in}}%
\pgfpathcurveto{\pgfqpoint{1.278505in}{2.118589in}}{\pgfqpoint{1.267906in}{2.122979in}}{\pgfqpoint{1.256856in}{2.122979in}}%
\pgfpathcurveto{\pgfqpoint{1.245806in}{2.122979in}}{\pgfqpoint{1.235207in}{2.118589in}}{\pgfqpoint{1.227393in}{2.110775in}}%
\pgfpathcurveto{\pgfqpoint{1.219580in}{2.102961in}}{\pgfqpoint{1.215189in}{2.092362in}}{\pgfqpoint{1.215189in}{2.081312in}}%
\pgfpathcurveto{\pgfqpoint{1.215189in}{2.070262in}}{\pgfqpoint{1.219580in}{2.059663in}}{\pgfqpoint{1.227393in}{2.051849in}}%
\pgfpathcurveto{\pgfqpoint{1.235207in}{2.044036in}}{\pgfqpoint{1.245806in}{2.039646in}}{\pgfqpoint{1.256856in}{2.039646in}}%
\pgfpathlineto{\pgfqpoint{1.256856in}{2.039646in}}%
\pgfpathclose%
\pgfusepath{stroke}%
\end{pgfscope}%
\begin{pgfscope}%
\pgfpathrectangle{\pgfqpoint{0.393053in}{0.375000in}}{\pgfqpoint{6.356833in}{5.175000in}}%
\pgfusepath{clip}%
\pgfsetbuttcap%
\pgfsetroundjoin%
\pgfsetlinewidth{1.003750pt}%
\definecolor{currentstroke}{rgb}{0.827451,0.827451,0.827451}%
\pgfsetstrokecolor{currentstroke}%
\pgfsetdash{}{0pt}%
\pgfpathmoveto{\pgfqpoint{2.124283in}{1.277669in}}%
\pgfpathcurveto{\pgfqpoint{2.135334in}{1.277669in}}{\pgfqpoint{2.145933in}{1.282059in}}{\pgfqpoint{2.153746in}{1.289873in}}%
\pgfpathcurveto{\pgfqpoint{2.161560in}{1.297686in}}{\pgfqpoint{2.165950in}{1.308286in}}{\pgfqpoint{2.165950in}{1.319336in}}%
\pgfpathcurveto{\pgfqpoint{2.165950in}{1.330386in}}{\pgfqpoint{2.161560in}{1.340985in}}{\pgfqpoint{2.153746in}{1.348798in}}%
\pgfpathcurveto{\pgfqpoint{2.145933in}{1.356612in}}{\pgfqpoint{2.135334in}{1.361002in}}{\pgfqpoint{2.124283in}{1.361002in}}%
\pgfpathcurveto{\pgfqpoint{2.113233in}{1.361002in}}{\pgfqpoint{2.102634in}{1.356612in}}{\pgfqpoint{2.094821in}{1.348798in}}%
\pgfpathcurveto{\pgfqpoint{2.087007in}{1.340985in}}{\pgfqpoint{2.082617in}{1.330386in}}{\pgfqpoint{2.082617in}{1.319336in}}%
\pgfpathcurveto{\pgfqpoint{2.082617in}{1.308286in}}{\pgfqpoint{2.087007in}{1.297686in}}{\pgfqpoint{2.094821in}{1.289873in}}%
\pgfpathcurveto{\pgfqpoint{2.102634in}{1.282059in}}{\pgfqpoint{2.113233in}{1.277669in}}{\pgfqpoint{2.124283in}{1.277669in}}%
\pgfpathlineto{\pgfqpoint{2.124283in}{1.277669in}}%
\pgfpathclose%
\pgfusepath{stroke}%
\end{pgfscope}%
\begin{pgfscope}%
\pgfpathrectangle{\pgfqpoint{0.393053in}{0.375000in}}{\pgfqpoint{6.356833in}{5.175000in}}%
\pgfusepath{clip}%
\pgfsetbuttcap%
\pgfsetroundjoin%
\pgfsetlinewidth{1.003750pt}%
\definecolor{currentstroke}{rgb}{0.827451,0.827451,0.827451}%
\pgfsetstrokecolor{currentstroke}%
\pgfsetdash{}{0pt}%
\pgfpathmoveto{\pgfqpoint{1.457810in}{1.831739in}}%
\pgfpathcurveto{\pgfqpoint{1.468860in}{1.831739in}}{\pgfqpoint{1.479459in}{1.836129in}}{\pgfqpoint{1.487273in}{1.843943in}}%
\pgfpathcurveto{\pgfqpoint{1.495086in}{1.851756in}}{\pgfqpoint{1.499477in}{1.862355in}}{\pgfqpoint{1.499477in}{1.873406in}}%
\pgfpathcurveto{\pgfqpoint{1.499477in}{1.884456in}}{\pgfqpoint{1.495086in}{1.895055in}}{\pgfqpoint{1.487273in}{1.902868in}}%
\pgfpathcurveto{\pgfqpoint{1.479459in}{1.910682in}}{\pgfqpoint{1.468860in}{1.915072in}}{\pgfqpoint{1.457810in}{1.915072in}}%
\pgfpathcurveto{\pgfqpoint{1.446760in}{1.915072in}}{\pgfqpoint{1.436161in}{1.910682in}}{\pgfqpoint{1.428347in}{1.902868in}}%
\pgfpathcurveto{\pgfqpoint{1.420533in}{1.895055in}}{\pgfqpoint{1.416143in}{1.884456in}}{\pgfqpoint{1.416143in}{1.873406in}}%
\pgfpathcurveto{\pgfqpoint{1.416143in}{1.862355in}}{\pgfqpoint{1.420533in}{1.851756in}}{\pgfqpoint{1.428347in}{1.843943in}}%
\pgfpathcurveto{\pgfqpoint{1.436161in}{1.836129in}}{\pgfqpoint{1.446760in}{1.831739in}}{\pgfqpoint{1.457810in}{1.831739in}}%
\pgfpathlineto{\pgfqpoint{1.457810in}{1.831739in}}%
\pgfpathclose%
\pgfusepath{stroke}%
\end{pgfscope}%
\begin{pgfscope}%
\pgfpathrectangle{\pgfqpoint{0.393053in}{0.375000in}}{\pgfqpoint{6.356833in}{5.175000in}}%
\pgfusepath{clip}%
\pgfsetbuttcap%
\pgfsetroundjoin%
\pgfsetlinewidth{1.003750pt}%
\definecolor{currentstroke}{rgb}{0.827451,0.827451,0.827451}%
\pgfsetstrokecolor{currentstroke}%
\pgfsetdash{}{0pt}%
\pgfpathmoveto{\pgfqpoint{2.578507in}{1.035661in}}%
\pgfpathcurveto{\pgfqpoint{2.589557in}{1.035661in}}{\pgfqpoint{2.600156in}{1.040051in}}{\pgfqpoint{2.607970in}{1.047865in}}%
\pgfpathcurveto{\pgfqpoint{2.615783in}{1.055678in}}{\pgfqpoint{2.620173in}{1.066277in}}{\pgfqpoint{2.620173in}{1.077327in}}%
\pgfpathcurveto{\pgfqpoint{2.620173in}{1.088378in}}{\pgfqpoint{2.615783in}{1.098977in}}{\pgfqpoint{2.607970in}{1.106790in}}%
\pgfpathcurveto{\pgfqpoint{2.600156in}{1.114604in}}{\pgfqpoint{2.589557in}{1.118994in}}{\pgfqpoint{2.578507in}{1.118994in}}%
\pgfpathcurveto{\pgfqpoint{2.567457in}{1.118994in}}{\pgfqpoint{2.556858in}{1.114604in}}{\pgfqpoint{2.549044in}{1.106790in}}%
\pgfpathcurveto{\pgfqpoint{2.541230in}{1.098977in}}{\pgfqpoint{2.536840in}{1.088378in}}{\pgfqpoint{2.536840in}{1.077327in}}%
\pgfpathcurveto{\pgfqpoint{2.536840in}{1.066277in}}{\pgfqpoint{2.541230in}{1.055678in}}{\pgfqpoint{2.549044in}{1.047865in}}%
\pgfpathcurveto{\pgfqpoint{2.556858in}{1.040051in}}{\pgfqpoint{2.567457in}{1.035661in}}{\pgfqpoint{2.578507in}{1.035661in}}%
\pgfpathlineto{\pgfqpoint{2.578507in}{1.035661in}}%
\pgfpathclose%
\pgfusepath{stroke}%
\end{pgfscope}%
\begin{pgfscope}%
\pgfpathrectangle{\pgfqpoint{0.393053in}{0.375000in}}{\pgfqpoint{6.356833in}{5.175000in}}%
\pgfusepath{clip}%
\pgfsetbuttcap%
\pgfsetroundjoin%
\pgfsetlinewidth{1.003750pt}%
\definecolor{currentstroke}{rgb}{0.827451,0.827451,0.827451}%
\pgfsetstrokecolor{currentstroke}%
\pgfsetdash{}{0pt}%
\pgfpathmoveto{\pgfqpoint{3.184302in}{0.739704in}}%
\pgfpathcurveto{\pgfqpoint{3.195352in}{0.739704in}}{\pgfqpoint{3.205951in}{0.744095in}}{\pgfqpoint{3.213765in}{0.751908in}}%
\pgfpathcurveto{\pgfqpoint{3.221578in}{0.759722in}}{\pgfqpoint{3.225969in}{0.770321in}}{\pgfqpoint{3.225969in}{0.781371in}}%
\pgfpathcurveto{\pgfqpoint{3.225969in}{0.792421in}}{\pgfqpoint{3.221578in}{0.803020in}}{\pgfqpoint{3.213765in}{0.810834in}}%
\pgfpathcurveto{\pgfqpoint{3.205951in}{0.818647in}}{\pgfqpoint{3.195352in}{0.823038in}}{\pgfqpoint{3.184302in}{0.823038in}}%
\pgfpathcurveto{\pgfqpoint{3.173252in}{0.823038in}}{\pgfqpoint{3.162653in}{0.818647in}}{\pgfqpoint{3.154839in}{0.810834in}}%
\pgfpathcurveto{\pgfqpoint{3.147026in}{0.803020in}}{\pgfqpoint{3.142635in}{0.792421in}}{\pgfqpoint{3.142635in}{0.781371in}}%
\pgfpathcurveto{\pgfqpoint{3.142635in}{0.770321in}}{\pgfqpoint{3.147026in}{0.759722in}}{\pgfqpoint{3.154839in}{0.751908in}}%
\pgfpathcurveto{\pgfqpoint{3.162653in}{0.744095in}}{\pgfqpoint{3.173252in}{0.739704in}}{\pgfqpoint{3.184302in}{0.739704in}}%
\pgfpathlineto{\pgfqpoint{3.184302in}{0.739704in}}%
\pgfpathclose%
\pgfusepath{stroke}%
\end{pgfscope}%
\begin{pgfscope}%
\pgfpathrectangle{\pgfqpoint{0.393053in}{0.375000in}}{\pgfqpoint{6.356833in}{5.175000in}}%
\pgfusepath{clip}%
\pgfsetbuttcap%
\pgfsetroundjoin%
\pgfsetlinewidth{1.003750pt}%
\definecolor{currentstroke}{rgb}{0.827451,0.827451,0.827451}%
\pgfsetstrokecolor{currentstroke}%
\pgfsetdash{}{0pt}%
\pgfpathmoveto{\pgfqpoint{3.440404in}{0.644692in}}%
\pgfpathcurveto{\pgfqpoint{3.451454in}{0.644692in}}{\pgfqpoint{3.462053in}{0.649082in}}{\pgfqpoint{3.469866in}{0.656896in}}%
\pgfpathcurveto{\pgfqpoint{3.477680in}{0.664710in}}{\pgfqpoint{3.482070in}{0.675309in}}{\pgfqpoint{3.482070in}{0.686359in}}%
\pgfpathcurveto{\pgfqpoint{3.482070in}{0.697409in}}{\pgfqpoint{3.477680in}{0.708008in}}{\pgfqpoint{3.469866in}{0.715822in}}%
\pgfpathcurveto{\pgfqpoint{3.462053in}{0.723635in}}{\pgfqpoint{3.451454in}{0.728025in}}{\pgfqpoint{3.440404in}{0.728025in}}%
\pgfpathcurveto{\pgfqpoint{3.429353in}{0.728025in}}{\pgfqpoint{3.418754in}{0.723635in}}{\pgfqpoint{3.410941in}{0.715822in}}%
\pgfpathcurveto{\pgfqpoint{3.403127in}{0.708008in}}{\pgfqpoint{3.398737in}{0.697409in}}{\pgfqpoint{3.398737in}{0.686359in}}%
\pgfpathcurveto{\pgfqpoint{3.398737in}{0.675309in}}{\pgfqpoint{3.403127in}{0.664710in}}{\pgfqpoint{3.410941in}{0.656896in}}%
\pgfpathcurveto{\pgfqpoint{3.418754in}{0.649082in}}{\pgfqpoint{3.429353in}{0.644692in}}{\pgfqpoint{3.440404in}{0.644692in}}%
\pgfpathlineto{\pgfqpoint{3.440404in}{0.644692in}}%
\pgfpathclose%
\pgfusepath{stroke}%
\end{pgfscope}%
\begin{pgfscope}%
\pgfpathrectangle{\pgfqpoint{0.393053in}{0.375000in}}{\pgfqpoint{6.356833in}{5.175000in}}%
\pgfusepath{clip}%
\pgfsetbuttcap%
\pgfsetroundjoin%
\pgfsetlinewidth{1.003750pt}%
\definecolor{currentstroke}{rgb}{0.827451,0.827451,0.827451}%
\pgfsetstrokecolor{currentstroke}%
\pgfsetdash{}{0pt}%
\pgfpathmoveto{\pgfqpoint{1.115030in}{2.257910in}}%
\pgfpathcurveto{\pgfqpoint{1.126080in}{2.257910in}}{\pgfqpoint{1.136679in}{2.262300in}}{\pgfqpoint{1.144493in}{2.270114in}}%
\pgfpathcurveto{\pgfqpoint{1.152307in}{2.277927in}}{\pgfqpoint{1.156697in}{2.288526in}}{\pgfqpoint{1.156697in}{2.299576in}}%
\pgfpathcurveto{\pgfqpoint{1.156697in}{2.310626in}}{\pgfqpoint{1.152307in}{2.321226in}}{\pgfqpoint{1.144493in}{2.329039in}}%
\pgfpathcurveto{\pgfqpoint{1.136679in}{2.336853in}}{\pgfqpoint{1.126080in}{2.341243in}}{\pgfqpoint{1.115030in}{2.341243in}}%
\pgfpathcurveto{\pgfqpoint{1.103980in}{2.341243in}}{\pgfqpoint{1.093381in}{2.336853in}}{\pgfqpoint{1.085567in}{2.329039in}}%
\pgfpathcurveto{\pgfqpoint{1.077754in}{2.321226in}}{\pgfqpoint{1.073364in}{2.310626in}}{\pgfqpoint{1.073364in}{2.299576in}}%
\pgfpathcurveto{\pgfqpoint{1.073364in}{2.288526in}}{\pgfqpoint{1.077754in}{2.277927in}}{\pgfqpoint{1.085567in}{2.270114in}}%
\pgfpathcurveto{\pgfqpoint{1.093381in}{2.262300in}}{\pgfqpoint{1.103980in}{2.257910in}}{\pgfqpoint{1.115030in}{2.257910in}}%
\pgfpathlineto{\pgfqpoint{1.115030in}{2.257910in}}%
\pgfpathclose%
\pgfusepath{stroke}%
\end{pgfscope}%
\begin{pgfscope}%
\pgfpathrectangle{\pgfqpoint{0.393053in}{0.375000in}}{\pgfqpoint{6.356833in}{5.175000in}}%
\pgfusepath{clip}%
\pgfsetbuttcap%
\pgfsetroundjoin%
\pgfsetlinewidth{1.003750pt}%
\definecolor{currentstroke}{rgb}{0.827451,0.827451,0.827451}%
\pgfsetstrokecolor{currentstroke}%
\pgfsetdash{}{0pt}%
\pgfpathmoveto{\pgfqpoint{3.397541in}{0.662535in}}%
\pgfpathcurveto{\pgfqpoint{3.408591in}{0.662535in}}{\pgfqpoint{3.419190in}{0.666925in}}{\pgfqpoint{3.427004in}{0.674738in}}%
\pgfpathcurveto{\pgfqpoint{3.434817in}{0.682552in}}{\pgfqpoint{3.439208in}{0.693151in}}{\pgfqpoint{3.439208in}{0.704201in}}%
\pgfpathcurveto{\pgfqpoint{3.439208in}{0.715251in}}{\pgfqpoint{3.434817in}{0.725850in}}{\pgfqpoint{3.427004in}{0.733664in}}%
\pgfpathcurveto{\pgfqpoint{3.419190in}{0.741478in}}{\pgfqpoint{3.408591in}{0.745868in}}{\pgfqpoint{3.397541in}{0.745868in}}%
\pgfpathcurveto{\pgfqpoint{3.386491in}{0.745868in}}{\pgfqpoint{3.375892in}{0.741478in}}{\pgfqpoint{3.368078in}{0.733664in}}%
\pgfpathcurveto{\pgfqpoint{3.360265in}{0.725850in}}{\pgfqpoint{3.355874in}{0.715251in}}{\pgfqpoint{3.355874in}{0.704201in}}%
\pgfpathcurveto{\pgfqpoint{3.355874in}{0.693151in}}{\pgfqpoint{3.360265in}{0.682552in}}{\pgfqpoint{3.368078in}{0.674738in}}%
\pgfpathcurveto{\pgfqpoint{3.375892in}{0.666925in}}{\pgfqpoint{3.386491in}{0.662535in}}{\pgfqpoint{3.397541in}{0.662535in}}%
\pgfpathlineto{\pgfqpoint{3.397541in}{0.662535in}}%
\pgfpathclose%
\pgfusepath{stroke}%
\end{pgfscope}%
\begin{pgfscope}%
\pgfpathrectangle{\pgfqpoint{0.393053in}{0.375000in}}{\pgfqpoint{6.356833in}{5.175000in}}%
\pgfusepath{clip}%
\pgfsetbuttcap%
\pgfsetroundjoin%
\pgfsetlinewidth{1.003750pt}%
\definecolor{currentstroke}{rgb}{0.827451,0.827451,0.827451}%
\pgfsetstrokecolor{currentstroke}%
\pgfsetdash{}{0pt}%
\pgfpathmoveto{\pgfqpoint{0.393825in}{4.495013in}}%
\pgfpathcurveto{\pgfqpoint{0.404875in}{4.495013in}}{\pgfqpoint{0.415474in}{4.499403in}}{\pgfqpoint{0.423288in}{4.507217in}}%
\pgfpathcurveto{\pgfqpoint{0.431101in}{4.515030in}}{\pgfqpoint{0.435492in}{4.525629in}}{\pgfqpoint{0.435492in}{4.536679in}}%
\pgfpathcurveto{\pgfqpoint{0.435492in}{4.547730in}}{\pgfqpoint{0.431101in}{4.558329in}}{\pgfqpoint{0.423288in}{4.566142in}}%
\pgfpathcurveto{\pgfqpoint{0.415474in}{4.573956in}}{\pgfqpoint{0.404875in}{4.578346in}}{\pgfqpoint{0.393825in}{4.578346in}}%
\pgfpathcurveto{\pgfqpoint{0.382775in}{4.578346in}}{\pgfqpoint{0.372176in}{4.573956in}}{\pgfqpoint{0.364362in}{4.566142in}}%
\pgfpathcurveto{\pgfqpoint{0.356548in}{4.558329in}}{\pgfqpoint{0.352158in}{4.547730in}}{\pgfqpoint{0.352158in}{4.536679in}}%
\pgfpathcurveto{\pgfqpoint{0.352158in}{4.525629in}}{\pgfqpoint{0.356548in}{4.515030in}}{\pgfqpoint{0.364362in}{4.507217in}}%
\pgfpathcurveto{\pgfqpoint{0.372176in}{4.499403in}}{\pgfqpoint{0.382775in}{4.495013in}}{\pgfqpoint{0.393825in}{4.495013in}}%
\pgfpathlineto{\pgfqpoint{0.393825in}{4.495013in}}%
\pgfpathclose%
\pgfusepath{stroke}%
\end{pgfscope}%
\begin{pgfscope}%
\pgfpathrectangle{\pgfqpoint{0.393053in}{0.375000in}}{\pgfqpoint{6.356833in}{5.175000in}}%
\pgfusepath{clip}%
\pgfsetbuttcap%
\pgfsetroundjoin%
\pgfsetlinewidth{1.003750pt}%
\definecolor{currentstroke}{rgb}{0.827451,0.827451,0.827451}%
\pgfsetstrokecolor{currentstroke}%
\pgfsetdash{}{0pt}%
\pgfpathmoveto{\pgfqpoint{5.707928in}{0.337346in}}%
\pgfpathcurveto{\pgfqpoint{5.718979in}{0.337346in}}{\pgfqpoint{5.729578in}{0.341736in}}{\pgfqpoint{5.737391in}{0.349550in}}%
\pgfpathcurveto{\pgfqpoint{5.745205in}{0.357363in}}{\pgfqpoint{5.749595in}{0.367962in}}{\pgfqpoint{5.749595in}{0.379012in}}%
\pgfpathcurveto{\pgfqpoint{5.749595in}{0.390063in}}{\pgfqpoint{5.745205in}{0.400662in}}{\pgfqpoint{5.737391in}{0.408475in}}%
\pgfpathcurveto{\pgfqpoint{5.729578in}{0.416289in}}{\pgfqpoint{5.718979in}{0.420679in}}{\pgfqpoint{5.707928in}{0.420679in}}%
\pgfpathcurveto{\pgfqpoint{5.696878in}{0.420679in}}{\pgfqpoint{5.686279in}{0.416289in}}{\pgfqpoint{5.678466in}{0.408475in}}%
\pgfpathcurveto{\pgfqpoint{5.670652in}{0.400662in}}{\pgfqpoint{5.666262in}{0.390063in}}{\pgfqpoint{5.666262in}{0.379012in}}%
\pgfpathcurveto{\pgfqpoint{5.666262in}{0.367962in}}{\pgfqpoint{5.670652in}{0.357363in}}{\pgfqpoint{5.678466in}{0.349550in}}%
\pgfpathcurveto{\pgfqpoint{5.686279in}{0.341736in}}{\pgfqpoint{5.696878in}{0.337346in}}{\pgfqpoint{5.707928in}{0.337346in}}%
\pgfusepath{stroke}%
\end{pgfscope}%
\begin{pgfscope}%
\pgfpathrectangle{\pgfqpoint{0.393053in}{0.375000in}}{\pgfqpoint{6.356833in}{5.175000in}}%
\pgfusepath{clip}%
\pgfsetbuttcap%
\pgfsetroundjoin%
\pgfsetlinewidth{1.003750pt}%
\definecolor{currentstroke}{rgb}{0.827451,0.827451,0.827451}%
\pgfsetstrokecolor{currentstroke}%
\pgfsetdash{}{0pt}%
\pgfpathmoveto{\pgfqpoint{4.655370in}{0.409439in}}%
\pgfpathcurveto{\pgfqpoint{4.666420in}{0.409439in}}{\pgfqpoint{4.677019in}{0.413829in}}{\pgfqpoint{4.684833in}{0.421642in}}%
\pgfpathcurveto{\pgfqpoint{4.692646in}{0.429456in}}{\pgfqpoint{4.697036in}{0.440055in}}{\pgfqpoint{4.697036in}{0.451105in}}%
\pgfpathcurveto{\pgfqpoint{4.697036in}{0.462155in}}{\pgfqpoint{4.692646in}{0.472754in}}{\pgfqpoint{4.684833in}{0.480568in}}%
\pgfpathcurveto{\pgfqpoint{4.677019in}{0.488382in}}{\pgfqpoint{4.666420in}{0.492772in}}{\pgfqpoint{4.655370in}{0.492772in}}%
\pgfpathcurveto{\pgfqpoint{4.644320in}{0.492772in}}{\pgfqpoint{4.633721in}{0.488382in}}{\pgfqpoint{4.625907in}{0.480568in}}%
\pgfpathcurveto{\pgfqpoint{4.618093in}{0.472754in}}{\pgfqpoint{4.613703in}{0.462155in}}{\pgfqpoint{4.613703in}{0.451105in}}%
\pgfpathcurveto{\pgfqpoint{4.613703in}{0.440055in}}{\pgfqpoint{4.618093in}{0.429456in}}{\pgfqpoint{4.625907in}{0.421642in}}%
\pgfpathcurveto{\pgfqpoint{4.633721in}{0.413829in}}{\pgfqpoint{4.644320in}{0.409439in}}{\pgfqpoint{4.655370in}{0.409439in}}%
\pgfpathlineto{\pgfqpoint{4.655370in}{0.409439in}}%
\pgfpathclose%
\pgfusepath{stroke}%
\end{pgfscope}%
\begin{pgfscope}%
\pgfpathrectangle{\pgfqpoint{0.393053in}{0.375000in}}{\pgfqpoint{6.356833in}{5.175000in}}%
\pgfusepath{clip}%
\pgfsetbuttcap%
\pgfsetroundjoin%
\pgfsetlinewidth{1.003750pt}%
\definecolor{currentstroke}{rgb}{0.827451,0.827451,0.827451}%
\pgfsetstrokecolor{currentstroke}%
\pgfsetdash{}{0pt}%
\pgfpathmoveto{\pgfqpoint{5.524477in}{0.341503in}}%
\pgfpathcurveto{\pgfqpoint{5.535528in}{0.341503in}}{\pgfqpoint{5.546127in}{0.345894in}}{\pgfqpoint{5.553940in}{0.353707in}}%
\pgfpathcurveto{\pgfqpoint{5.561754in}{0.361521in}}{\pgfqpoint{5.566144in}{0.372120in}}{\pgfqpoint{5.566144in}{0.383170in}}%
\pgfpathcurveto{\pgfqpoint{5.566144in}{0.394220in}}{\pgfqpoint{5.561754in}{0.404819in}}{\pgfqpoint{5.553940in}{0.412633in}}%
\pgfpathcurveto{\pgfqpoint{5.546127in}{0.420446in}}{\pgfqpoint{5.535528in}{0.424837in}}{\pgfqpoint{5.524477in}{0.424837in}}%
\pgfpathcurveto{\pgfqpoint{5.513427in}{0.424837in}}{\pgfqpoint{5.502828in}{0.420446in}}{\pgfqpoint{5.495015in}{0.412633in}}%
\pgfpathcurveto{\pgfqpoint{5.487201in}{0.404819in}}{\pgfqpoint{5.482811in}{0.394220in}}{\pgfqpoint{5.482811in}{0.383170in}}%
\pgfpathcurveto{\pgfqpoint{5.482811in}{0.372120in}}{\pgfqpoint{5.487201in}{0.361521in}}{\pgfqpoint{5.495015in}{0.353707in}}%
\pgfpathcurveto{\pgfqpoint{5.502828in}{0.345894in}}{\pgfqpoint{5.513427in}{0.341503in}}{\pgfqpoint{5.524477in}{0.341503in}}%
\pgfusepath{stroke}%
\end{pgfscope}%
\begin{pgfscope}%
\pgfpathrectangle{\pgfqpoint{0.393053in}{0.375000in}}{\pgfqpoint{6.356833in}{5.175000in}}%
\pgfusepath{clip}%
\pgfsetbuttcap%
\pgfsetroundjoin%
\pgfsetlinewidth{1.003750pt}%
\definecolor{currentstroke}{rgb}{0.827451,0.827451,0.827451}%
\pgfsetstrokecolor{currentstroke}%
\pgfsetdash{}{0pt}%
\pgfpathmoveto{\pgfqpoint{5.142473in}{0.360415in}}%
\pgfpathcurveto{\pgfqpoint{5.153523in}{0.360415in}}{\pgfqpoint{5.164122in}{0.364805in}}{\pgfqpoint{5.171936in}{0.372618in}}%
\pgfpathcurveto{\pgfqpoint{5.179749in}{0.380432in}}{\pgfqpoint{5.184140in}{0.391031in}}{\pgfqpoint{5.184140in}{0.402081in}}%
\pgfpathcurveto{\pgfqpoint{5.184140in}{0.413131in}}{\pgfqpoint{5.179749in}{0.423730in}}{\pgfqpoint{5.171936in}{0.431544in}}%
\pgfpathcurveto{\pgfqpoint{5.164122in}{0.439358in}}{\pgfqpoint{5.153523in}{0.443748in}}{\pgfqpoint{5.142473in}{0.443748in}}%
\pgfpathcurveto{\pgfqpoint{5.131423in}{0.443748in}}{\pgfqpoint{5.120824in}{0.439358in}}{\pgfqpoint{5.113010in}{0.431544in}}%
\pgfpathcurveto{\pgfqpoint{5.105196in}{0.423730in}}{\pgfqpoint{5.100806in}{0.413131in}}{\pgfqpoint{5.100806in}{0.402081in}}%
\pgfpathcurveto{\pgfqpoint{5.100806in}{0.391031in}}{\pgfqpoint{5.105196in}{0.380432in}}{\pgfqpoint{5.113010in}{0.372618in}}%
\pgfpathcurveto{\pgfqpoint{5.120824in}{0.364805in}}{\pgfqpoint{5.131423in}{0.360415in}}{\pgfqpoint{5.142473in}{0.360415in}}%
\pgfusepath{stroke}%
\end{pgfscope}%
\begin{pgfscope}%
\pgfpathrectangle{\pgfqpoint{0.393053in}{0.375000in}}{\pgfqpoint{6.356833in}{5.175000in}}%
\pgfusepath{clip}%
\pgfsetbuttcap%
\pgfsetroundjoin%
\pgfsetlinewidth{1.003750pt}%
\definecolor{currentstroke}{rgb}{0.827451,0.827451,0.827451}%
\pgfsetstrokecolor{currentstroke}%
\pgfsetdash{}{0pt}%
\pgfpathmoveto{\pgfqpoint{0.399634in}{4.323598in}}%
\pgfpathcurveto{\pgfqpoint{0.410684in}{4.323598in}}{\pgfqpoint{0.421283in}{4.327988in}}{\pgfqpoint{0.429097in}{4.335802in}}%
\pgfpathcurveto{\pgfqpoint{0.436910in}{4.343616in}}{\pgfqpoint{0.441301in}{4.354215in}}{\pgfqpoint{0.441301in}{4.365265in}}%
\pgfpathcurveto{\pgfqpoint{0.441301in}{4.376315in}}{\pgfqpoint{0.436910in}{4.386914in}}{\pgfqpoint{0.429097in}{4.394727in}}%
\pgfpathcurveto{\pgfqpoint{0.421283in}{4.402541in}}{\pgfqpoint{0.410684in}{4.406931in}}{\pgfqpoint{0.399634in}{4.406931in}}%
\pgfpathcurveto{\pgfqpoint{0.388584in}{4.406931in}}{\pgfqpoint{0.377985in}{4.402541in}}{\pgfqpoint{0.370171in}{4.394727in}}%
\pgfpathcurveto{\pgfqpoint{0.362357in}{4.386914in}}{\pgfqpoint{0.357967in}{4.376315in}}{\pgfqpoint{0.357967in}{4.365265in}}%
\pgfpathcurveto{\pgfqpoint{0.357967in}{4.354215in}}{\pgfqpoint{0.362357in}{4.343616in}}{\pgfqpoint{0.370171in}{4.335802in}}%
\pgfpathcurveto{\pgfqpoint{0.377985in}{4.327988in}}{\pgfqpoint{0.388584in}{4.323598in}}{\pgfqpoint{0.399634in}{4.323598in}}%
\pgfpathlineto{\pgfqpoint{0.399634in}{4.323598in}}%
\pgfpathclose%
\pgfusepath{stroke}%
\end{pgfscope}%
\begin{pgfscope}%
\pgfpathrectangle{\pgfqpoint{0.393053in}{0.375000in}}{\pgfqpoint{6.356833in}{5.175000in}}%
\pgfusepath{clip}%
\pgfsetbuttcap%
\pgfsetroundjoin%
\pgfsetlinewidth{1.003750pt}%
\definecolor{currentstroke}{rgb}{0.827451,0.827451,0.827451}%
\pgfsetstrokecolor{currentstroke}%
\pgfsetdash{}{0pt}%
\pgfpathmoveto{\pgfqpoint{4.898856in}{0.391247in}}%
\pgfpathcurveto{\pgfqpoint{4.909906in}{0.391247in}}{\pgfqpoint{4.920505in}{0.395637in}}{\pgfqpoint{4.928319in}{0.403450in}}%
\pgfpathcurveto{\pgfqpoint{4.936132in}{0.411264in}}{\pgfqpoint{4.940523in}{0.421863in}}{\pgfqpoint{4.940523in}{0.432913in}}%
\pgfpathcurveto{\pgfqpoint{4.940523in}{0.443963in}}{\pgfqpoint{4.936132in}{0.454562in}}{\pgfqpoint{4.928319in}{0.462376in}}%
\pgfpathcurveto{\pgfqpoint{4.920505in}{0.470190in}}{\pgfqpoint{4.909906in}{0.474580in}}{\pgfqpoint{4.898856in}{0.474580in}}%
\pgfpathcurveto{\pgfqpoint{4.887806in}{0.474580in}}{\pgfqpoint{4.877207in}{0.470190in}}{\pgfqpoint{4.869393in}{0.462376in}}%
\pgfpathcurveto{\pgfqpoint{4.861580in}{0.454562in}}{\pgfqpoint{4.857189in}{0.443963in}}{\pgfqpoint{4.857189in}{0.432913in}}%
\pgfpathcurveto{\pgfqpoint{4.857189in}{0.421863in}}{\pgfqpoint{4.861580in}{0.411264in}}{\pgfqpoint{4.869393in}{0.403450in}}%
\pgfpathcurveto{\pgfqpoint{4.877207in}{0.395637in}}{\pgfqpoint{4.887806in}{0.391247in}}{\pgfqpoint{4.898856in}{0.391247in}}%
\pgfpathlineto{\pgfqpoint{4.898856in}{0.391247in}}%
\pgfpathclose%
\pgfusepath{stroke}%
\end{pgfscope}%
\begin{pgfscope}%
\pgfpathrectangle{\pgfqpoint{0.393053in}{0.375000in}}{\pgfqpoint{6.356833in}{5.175000in}}%
\pgfusepath{clip}%
\pgfsetbuttcap%
\pgfsetroundjoin%
\pgfsetlinewidth{1.003750pt}%
\definecolor{currentstroke}{rgb}{0.827451,0.827451,0.827451}%
\pgfsetstrokecolor{currentstroke}%
\pgfsetdash{}{0pt}%
\pgfpathmoveto{\pgfqpoint{0.418752in}{4.107793in}}%
\pgfpathcurveto{\pgfqpoint{0.429802in}{4.107793in}}{\pgfqpoint{0.440401in}{4.112183in}}{\pgfqpoint{0.448215in}{4.119996in}}%
\pgfpathcurveto{\pgfqpoint{0.456028in}{4.127810in}}{\pgfqpoint{0.460419in}{4.138409in}}{\pgfqpoint{0.460419in}{4.149459in}}%
\pgfpathcurveto{\pgfqpoint{0.460419in}{4.160509in}}{\pgfqpoint{0.456028in}{4.171108in}}{\pgfqpoint{0.448215in}{4.178922in}}%
\pgfpathcurveto{\pgfqpoint{0.440401in}{4.186736in}}{\pgfqpoint{0.429802in}{4.191126in}}{\pgfqpoint{0.418752in}{4.191126in}}%
\pgfpathcurveto{\pgfqpoint{0.407702in}{4.191126in}}{\pgfqpoint{0.397103in}{4.186736in}}{\pgfqpoint{0.389289in}{4.178922in}}%
\pgfpathcurveto{\pgfqpoint{0.381476in}{4.171108in}}{\pgfqpoint{0.377085in}{4.160509in}}{\pgfqpoint{0.377085in}{4.149459in}}%
\pgfpathcurveto{\pgfqpoint{0.377085in}{4.138409in}}{\pgfqpoint{0.381476in}{4.127810in}}{\pgfqpoint{0.389289in}{4.119996in}}%
\pgfpathcurveto{\pgfqpoint{0.397103in}{4.112183in}}{\pgfqpoint{0.407702in}{4.107793in}}{\pgfqpoint{0.418752in}{4.107793in}}%
\pgfpathlineto{\pgfqpoint{0.418752in}{4.107793in}}%
\pgfpathclose%
\pgfusepath{stroke}%
\end{pgfscope}%
\begin{pgfscope}%
\pgfpathrectangle{\pgfqpoint{0.393053in}{0.375000in}}{\pgfqpoint{6.356833in}{5.175000in}}%
\pgfusepath{clip}%
\pgfsetbuttcap%
\pgfsetroundjoin%
\pgfsetlinewidth{1.003750pt}%
\definecolor{currentstroke}{rgb}{0.827451,0.827451,0.827451}%
\pgfsetstrokecolor{currentstroke}%
\pgfsetdash{}{0pt}%
\pgfpathmoveto{\pgfqpoint{3.649696in}{0.625044in}}%
\pgfpathcurveto{\pgfqpoint{3.660746in}{0.625044in}}{\pgfqpoint{3.671345in}{0.629434in}}{\pgfqpoint{3.679159in}{0.637248in}}%
\pgfpathcurveto{\pgfqpoint{3.686972in}{0.645061in}}{\pgfqpoint{3.691363in}{0.655660in}}{\pgfqpoint{3.691363in}{0.666710in}}%
\pgfpathcurveto{\pgfqpoint{3.691363in}{0.677761in}}{\pgfqpoint{3.686972in}{0.688360in}}{\pgfqpoint{3.679159in}{0.696173in}}%
\pgfpathcurveto{\pgfqpoint{3.671345in}{0.703987in}}{\pgfqpoint{3.660746in}{0.708377in}}{\pgfqpoint{3.649696in}{0.708377in}}%
\pgfpathcurveto{\pgfqpoint{3.638646in}{0.708377in}}{\pgfqpoint{3.628047in}{0.703987in}}{\pgfqpoint{3.620233in}{0.696173in}}%
\pgfpathcurveto{\pgfqpoint{3.612420in}{0.688360in}}{\pgfqpoint{3.608029in}{0.677761in}}{\pgfqpoint{3.608029in}{0.666710in}}%
\pgfpathcurveto{\pgfqpoint{3.608029in}{0.655660in}}{\pgfqpoint{3.612420in}{0.645061in}}{\pgfqpoint{3.620233in}{0.637248in}}%
\pgfpathcurveto{\pgfqpoint{3.628047in}{0.629434in}}{\pgfqpoint{3.638646in}{0.625044in}}{\pgfqpoint{3.649696in}{0.625044in}}%
\pgfpathlineto{\pgfqpoint{3.649696in}{0.625044in}}%
\pgfpathclose%
\pgfusepath{stroke}%
\end{pgfscope}%
\begin{pgfscope}%
\pgfpathrectangle{\pgfqpoint{0.393053in}{0.375000in}}{\pgfqpoint{6.356833in}{5.175000in}}%
\pgfusepath{clip}%
\pgfsetbuttcap%
\pgfsetroundjoin%
\pgfsetlinewidth{1.003750pt}%
\definecolor{currentstroke}{rgb}{0.827451,0.827451,0.827451}%
\pgfsetstrokecolor{currentstroke}%
\pgfsetdash{}{0pt}%
\pgfpathmoveto{\pgfqpoint{2.693530in}{0.950710in}}%
\pgfpathcurveto{\pgfqpoint{2.704580in}{0.950710in}}{\pgfqpoint{2.715179in}{0.955100in}}{\pgfqpoint{2.722993in}{0.962914in}}%
\pgfpathcurveto{\pgfqpoint{2.730806in}{0.970727in}}{\pgfqpoint{2.735197in}{0.981326in}}{\pgfqpoint{2.735197in}{0.992376in}}%
\pgfpathcurveto{\pgfqpoint{2.735197in}{1.003427in}}{\pgfqpoint{2.730806in}{1.014026in}}{\pgfqpoint{2.722993in}{1.021839in}}%
\pgfpathcurveto{\pgfqpoint{2.715179in}{1.029653in}}{\pgfqpoint{2.704580in}{1.034043in}}{\pgfqpoint{2.693530in}{1.034043in}}%
\pgfpathcurveto{\pgfqpoint{2.682480in}{1.034043in}}{\pgfqpoint{2.671881in}{1.029653in}}{\pgfqpoint{2.664067in}{1.021839in}}%
\pgfpathcurveto{\pgfqpoint{2.656254in}{1.014026in}}{\pgfqpoint{2.651863in}{1.003427in}}{\pgfqpoint{2.651863in}{0.992376in}}%
\pgfpathcurveto{\pgfqpoint{2.651863in}{0.981326in}}{\pgfqpoint{2.656254in}{0.970727in}}{\pgfqpoint{2.664067in}{0.962914in}}%
\pgfpathcurveto{\pgfqpoint{2.671881in}{0.955100in}}{\pgfqpoint{2.682480in}{0.950710in}}{\pgfqpoint{2.693530in}{0.950710in}}%
\pgfpathlineto{\pgfqpoint{2.693530in}{0.950710in}}%
\pgfpathclose%
\pgfusepath{stroke}%
\end{pgfscope}%
\begin{pgfscope}%
\pgfpathrectangle{\pgfqpoint{0.393053in}{0.375000in}}{\pgfqpoint{6.356833in}{5.175000in}}%
\pgfusepath{clip}%
\pgfsetbuttcap%
\pgfsetroundjoin%
\pgfsetlinewidth{1.003750pt}%
\definecolor{currentstroke}{rgb}{0.827451,0.827451,0.827451}%
\pgfsetstrokecolor{currentstroke}%
\pgfsetdash{}{0pt}%
\pgfpathmoveto{\pgfqpoint{3.184302in}{0.739704in}}%
\pgfpathcurveto{\pgfqpoint{3.195352in}{0.739704in}}{\pgfqpoint{3.205951in}{0.744095in}}{\pgfqpoint{3.213765in}{0.751908in}}%
\pgfpathcurveto{\pgfqpoint{3.221578in}{0.759722in}}{\pgfqpoint{3.225969in}{0.770321in}}{\pgfqpoint{3.225969in}{0.781371in}}%
\pgfpathcurveto{\pgfqpoint{3.225969in}{0.792421in}}{\pgfqpoint{3.221578in}{0.803020in}}{\pgfqpoint{3.213765in}{0.810834in}}%
\pgfpathcurveto{\pgfqpoint{3.205951in}{0.818647in}}{\pgfqpoint{3.195352in}{0.823038in}}{\pgfqpoint{3.184302in}{0.823038in}}%
\pgfpathcurveto{\pgfqpoint{3.173252in}{0.823038in}}{\pgfqpoint{3.162653in}{0.818647in}}{\pgfqpoint{3.154839in}{0.810834in}}%
\pgfpathcurveto{\pgfqpoint{3.147026in}{0.803020in}}{\pgfqpoint{3.142635in}{0.792421in}}{\pgfqpoint{3.142635in}{0.781371in}}%
\pgfpathcurveto{\pgfqpoint{3.142635in}{0.770321in}}{\pgfqpoint{3.147026in}{0.759722in}}{\pgfqpoint{3.154839in}{0.751908in}}%
\pgfpathcurveto{\pgfqpoint{3.162653in}{0.744095in}}{\pgfqpoint{3.173252in}{0.739704in}}{\pgfqpoint{3.184302in}{0.739704in}}%
\pgfpathlineto{\pgfqpoint{3.184302in}{0.739704in}}%
\pgfpathclose%
\pgfusepath{stroke}%
\end{pgfscope}%
\begin{pgfscope}%
\pgfpathrectangle{\pgfqpoint{0.393053in}{0.375000in}}{\pgfqpoint{6.356833in}{5.175000in}}%
\pgfusepath{clip}%
\pgfsetbuttcap%
\pgfsetroundjoin%
\pgfsetlinewidth{1.003750pt}%
\definecolor{currentstroke}{rgb}{0.827451,0.827451,0.827451}%
\pgfsetstrokecolor{currentstroke}%
\pgfsetdash{}{0pt}%
\pgfpathmoveto{\pgfqpoint{0.474470in}{3.727925in}}%
\pgfpathcurveto{\pgfqpoint{0.485520in}{3.727925in}}{\pgfqpoint{0.496119in}{3.732315in}}{\pgfqpoint{0.503933in}{3.740129in}}%
\pgfpathcurveto{\pgfqpoint{0.511746in}{3.747943in}}{\pgfqpoint{0.516137in}{3.758542in}}{\pgfqpoint{0.516137in}{3.769592in}}%
\pgfpathcurveto{\pgfqpoint{0.516137in}{3.780642in}}{\pgfqpoint{0.511746in}{3.791241in}}{\pgfqpoint{0.503933in}{3.799055in}}%
\pgfpathcurveto{\pgfqpoint{0.496119in}{3.806868in}}{\pgfqpoint{0.485520in}{3.811259in}}{\pgfqpoint{0.474470in}{3.811259in}}%
\pgfpathcurveto{\pgfqpoint{0.463420in}{3.811259in}}{\pgfqpoint{0.452821in}{3.806868in}}{\pgfqpoint{0.445007in}{3.799055in}}%
\pgfpathcurveto{\pgfqpoint{0.437194in}{3.791241in}}{\pgfqpoint{0.432803in}{3.780642in}}{\pgfqpoint{0.432803in}{3.769592in}}%
\pgfpathcurveto{\pgfqpoint{0.432803in}{3.758542in}}{\pgfqpoint{0.437194in}{3.747943in}}{\pgfqpoint{0.445007in}{3.740129in}}%
\pgfpathcurveto{\pgfqpoint{0.452821in}{3.732315in}}{\pgfqpoint{0.463420in}{3.727925in}}{\pgfqpoint{0.474470in}{3.727925in}}%
\pgfpathlineto{\pgfqpoint{0.474470in}{3.727925in}}%
\pgfpathclose%
\pgfusepath{stroke}%
\end{pgfscope}%
\begin{pgfscope}%
\pgfpathrectangle{\pgfqpoint{0.393053in}{0.375000in}}{\pgfqpoint{6.356833in}{5.175000in}}%
\pgfusepath{clip}%
\pgfsetbuttcap%
\pgfsetroundjoin%
\pgfsetlinewidth{1.003750pt}%
\definecolor{currentstroke}{rgb}{0.827451,0.827451,0.827451}%
\pgfsetstrokecolor{currentstroke}%
\pgfsetdash{}{0pt}%
\pgfpathmoveto{\pgfqpoint{1.040104in}{2.359499in}}%
\pgfpathcurveto{\pgfqpoint{1.051154in}{2.359499in}}{\pgfqpoint{1.061753in}{2.363889in}}{\pgfqpoint{1.069567in}{2.371703in}}%
\pgfpathcurveto{\pgfqpoint{1.077381in}{2.379516in}}{\pgfqpoint{1.081771in}{2.390115in}}{\pgfqpoint{1.081771in}{2.401165in}}%
\pgfpathcurveto{\pgfqpoint{1.081771in}{2.412215in}}{\pgfqpoint{1.077381in}{2.422814in}}{\pgfqpoint{1.069567in}{2.430628in}}%
\pgfpathcurveto{\pgfqpoint{1.061753in}{2.438442in}}{\pgfqpoint{1.051154in}{2.442832in}}{\pgfqpoint{1.040104in}{2.442832in}}%
\pgfpathcurveto{\pgfqpoint{1.029054in}{2.442832in}}{\pgfqpoint{1.018455in}{2.438442in}}{\pgfqpoint{1.010641in}{2.430628in}}%
\pgfpathcurveto{\pgfqpoint{1.002828in}{2.422814in}}{\pgfqpoint{0.998437in}{2.412215in}}{\pgfqpoint{0.998437in}{2.401165in}}%
\pgfpathcurveto{\pgfqpoint{0.998437in}{2.390115in}}{\pgfqpoint{1.002828in}{2.379516in}}{\pgfqpoint{1.010641in}{2.371703in}}%
\pgfpathcurveto{\pgfqpoint{1.018455in}{2.363889in}}{\pgfqpoint{1.029054in}{2.359499in}}{\pgfqpoint{1.040104in}{2.359499in}}%
\pgfpathlineto{\pgfqpoint{1.040104in}{2.359499in}}%
\pgfpathclose%
\pgfusepath{stroke}%
\end{pgfscope}%
\begin{pgfscope}%
\pgfpathrectangle{\pgfqpoint{0.393053in}{0.375000in}}{\pgfqpoint{6.356833in}{5.175000in}}%
\pgfusepath{clip}%
\pgfsetbuttcap%
\pgfsetroundjoin%
\pgfsetlinewidth{1.003750pt}%
\definecolor{currentstroke}{rgb}{0.827451,0.827451,0.827451}%
\pgfsetstrokecolor{currentstroke}%
\pgfsetdash{}{0pt}%
\pgfpathmoveto{\pgfqpoint{1.457810in}{1.831739in}}%
\pgfpathcurveto{\pgfqpoint{1.468860in}{1.831739in}}{\pgfqpoint{1.479459in}{1.836129in}}{\pgfqpoint{1.487273in}{1.843943in}}%
\pgfpathcurveto{\pgfqpoint{1.495086in}{1.851756in}}{\pgfqpoint{1.499477in}{1.862355in}}{\pgfqpoint{1.499477in}{1.873406in}}%
\pgfpathcurveto{\pgfqpoint{1.499477in}{1.884456in}}{\pgfqpoint{1.495086in}{1.895055in}}{\pgfqpoint{1.487273in}{1.902868in}}%
\pgfpathcurveto{\pgfqpoint{1.479459in}{1.910682in}}{\pgfqpoint{1.468860in}{1.915072in}}{\pgfqpoint{1.457810in}{1.915072in}}%
\pgfpathcurveto{\pgfqpoint{1.446760in}{1.915072in}}{\pgfqpoint{1.436161in}{1.910682in}}{\pgfqpoint{1.428347in}{1.902868in}}%
\pgfpathcurveto{\pgfqpoint{1.420533in}{1.895055in}}{\pgfqpoint{1.416143in}{1.884456in}}{\pgfqpoint{1.416143in}{1.873406in}}%
\pgfpathcurveto{\pgfqpoint{1.416143in}{1.862355in}}{\pgfqpoint{1.420533in}{1.851756in}}{\pgfqpoint{1.428347in}{1.843943in}}%
\pgfpathcurveto{\pgfqpoint{1.436161in}{1.836129in}}{\pgfqpoint{1.446760in}{1.831739in}}{\pgfqpoint{1.457810in}{1.831739in}}%
\pgfpathlineto{\pgfqpoint{1.457810in}{1.831739in}}%
\pgfpathclose%
\pgfusepath{stroke}%
\end{pgfscope}%
\begin{pgfscope}%
\pgfpathrectangle{\pgfqpoint{0.393053in}{0.375000in}}{\pgfqpoint{6.356833in}{5.175000in}}%
\pgfusepath{clip}%
\pgfsetbuttcap%
\pgfsetroundjoin%
\pgfsetlinewidth{1.003750pt}%
\definecolor{currentstroke}{rgb}{0.827451,0.827451,0.827451}%
\pgfsetstrokecolor{currentstroke}%
\pgfsetdash{}{0pt}%
\pgfpathmoveto{\pgfqpoint{3.464896in}{0.638283in}}%
\pgfpathcurveto{\pgfqpoint{3.475946in}{0.638283in}}{\pgfqpoint{3.486545in}{0.642673in}}{\pgfqpoint{3.494359in}{0.650486in}}%
\pgfpathcurveto{\pgfqpoint{3.502173in}{0.658300in}}{\pgfqpoint{3.506563in}{0.668899in}}{\pgfqpoint{3.506563in}{0.679949in}}%
\pgfpathcurveto{\pgfqpoint{3.506563in}{0.690999in}}{\pgfqpoint{3.502173in}{0.701598in}}{\pgfqpoint{3.494359in}{0.709412in}}%
\pgfpathcurveto{\pgfqpoint{3.486545in}{0.717226in}}{\pgfqpoint{3.475946in}{0.721616in}}{\pgfqpoint{3.464896in}{0.721616in}}%
\pgfpathcurveto{\pgfqpoint{3.453846in}{0.721616in}}{\pgfqpoint{3.443247in}{0.717226in}}{\pgfqpoint{3.435433in}{0.709412in}}%
\pgfpathcurveto{\pgfqpoint{3.427620in}{0.701598in}}{\pgfqpoint{3.423230in}{0.690999in}}{\pgfqpoint{3.423230in}{0.679949in}}%
\pgfpathcurveto{\pgfqpoint{3.423230in}{0.668899in}}{\pgfqpoint{3.427620in}{0.658300in}}{\pgfqpoint{3.435433in}{0.650486in}}%
\pgfpathcurveto{\pgfqpoint{3.443247in}{0.642673in}}{\pgfqpoint{3.453846in}{0.638283in}}{\pgfqpoint{3.464896in}{0.638283in}}%
\pgfpathlineto{\pgfqpoint{3.464896in}{0.638283in}}%
\pgfpathclose%
\pgfusepath{stroke}%
\end{pgfscope}%
\begin{pgfscope}%
\pgfpathrectangle{\pgfqpoint{0.393053in}{0.375000in}}{\pgfqpoint{6.356833in}{5.175000in}}%
\pgfusepath{clip}%
\pgfsetbuttcap%
\pgfsetroundjoin%
\pgfsetlinewidth{1.003750pt}%
\definecolor{currentstroke}{rgb}{0.827451,0.827451,0.827451}%
\pgfsetstrokecolor{currentstroke}%
\pgfsetdash{}{0pt}%
\pgfpathmoveto{\pgfqpoint{2.230001in}{1.209659in}}%
\pgfpathcurveto{\pgfqpoint{2.241051in}{1.209659in}}{\pgfqpoint{2.251650in}{1.214049in}}{\pgfqpoint{2.259463in}{1.221862in}}%
\pgfpathcurveto{\pgfqpoint{2.267277in}{1.229676in}}{\pgfqpoint{2.271667in}{1.240275in}}{\pgfqpoint{2.271667in}{1.251325in}}%
\pgfpathcurveto{\pgfqpoint{2.271667in}{1.262375in}}{\pgfqpoint{2.267277in}{1.272974in}}{\pgfqpoint{2.259463in}{1.280788in}}%
\pgfpathcurveto{\pgfqpoint{2.251650in}{1.288602in}}{\pgfqpoint{2.241051in}{1.292992in}}{\pgfqpoint{2.230001in}{1.292992in}}%
\pgfpathcurveto{\pgfqpoint{2.218950in}{1.292992in}}{\pgfqpoint{2.208351in}{1.288602in}}{\pgfqpoint{2.200538in}{1.280788in}}%
\pgfpathcurveto{\pgfqpoint{2.192724in}{1.272974in}}{\pgfqpoint{2.188334in}{1.262375in}}{\pgfqpoint{2.188334in}{1.251325in}}%
\pgfpathcurveto{\pgfqpoint{2.188334in}{1.240275in}}{\pgfqpoint{2.192724in}{1.229676in}}{\pgfqpoint{2.200538in}{1.221862in}}%
\pgfpathcurveto{\pgfqpoint{2.208351in}{1.214049in}}{\pgfqpoint{2.218950in}{1.209659in}}{\pgfqpoint{2.230001in}{1.209659in}}%
\pgfpathlineto{\pgfqpoint{2.230001in}{1.209659in}}%
\pgfpathclose%
\pgfusepath{stroke}%
\end{pgfscope}%
\begin{pgfscope}%
\pgfpathrectangle{\pgfqpoint{0.393053in}{0.375000in}}{\pgfqpoint{6.356833in}{5.175000in}}%
\pgfusepath{clip}%
\pgfsetbuttcap%
\pgfsetroundjoin%
\pgfsetlinewidth{1.003750pt}%
\definecolor{currentstroke}{rgb}{0.827451,0.827451,0.827451}%
\pgfsetstrokecolor{currentstroke}%
\pgfsetdash{}{0pt}%
\pgfpathmoveto{\pgfqpoint{0.806869in}{2.717160in}}%
\pgfpathcurveto{\pgfqpoint{0.817919in}{2.717160in}}{\pgfqpoint{0.828518in}{2.721550in}}{\pgfqpoint{0.836332in}{2.729363in}}%
\pgfpathcurveto{\pgfqpoint{0.844145in}{2.737177in}}{\pgfqpoint{0.848536in}{2.747776in}}{\pgfqpoint{0.848536in}{2.758826in}}%
\pgfpathcurveto{\pgfqpoint{0.848536in}{2.769876in}}{\pgfqpoint{0.844145in}{2.780475in}}{\pgfqpoint{0.836332in}{2.788289in}}%
\pgfpathcurveto{\pgfqpoint{0.828518in}{2.796103in}}{\pgfqpoint{0.817919in}{2.800493in}}{\pgfqpoint{0.806869in}{2.800493in}}%
\pgfpathcurveto{\pgfqpoint{0.795819in}{2.800493in}}{\pgfqpoint{0.785220in}{2.796103in}}{\pgfqpoint{0.777406in}{2.788289in}}%
\pgfpathcurveto{\pgfqpoint{0.769593in}{2.780475in}}{\pgfqpoint{0.765202in}{2.769876in}}{\pgfqpoint{0.765202in}{2.758826in}}%
\pgfpathcurveto{\pgfqpoint{0.765202in}{2.747776in}}{\pgfqpoint{0.769593in}{2.737177in}}{\pgfqpoint{0.777406in}{2.729363in}}%
\pgfpathcurveto{\pgfqpoint{0.785220in}{2.721550in}}{\pgfqpoint{0.795819in}{2.717160in}}{\pgfqpoint{0.806869in}{2.717160in}}%
\pgfpathlineto{\pgfqpoint{0.806869in}{2.717160in}}%
\pgfpathclose%
\pgfusepath{stroke}%
\end{pgfscope}%
\begin{pgfscope}%
\pgfpathrectangle{\pgfqpoint{0.393053in}{0.375000in}}{\pgfqpoint{6.356833in}{5.175000in}}%
\pgfusepath{clip}%
\pgfsetbuttcap%
\pgfsetroundjoin%
\pgfsetlinewidth{1.003750pt}%
\definecolor{currentstroke}{rgb}{0.827451,0.827451,0.827451}%
\pgfsetstrokecolor{currentstroke}%
\pgfsetdash{}{0pt}%
\pgfpathmoveto{\pgfqpoint{0.905111in}{2.683698in}}%
\pgfpathcurveto{\pgfqpoint{0.916161in}{2.683698in}}{\pgfqpoint{0.926760in}{2.688088in}}{\pgfqpoint{0.934574in}{2.695902in}}%
\pgfpathcurveto{\pgfqpoint{0.942388in}{2.703715in}}{\pgfqpoint{0.946778in}{2.714314in}}{\pgfqpoint{0.946778in}{2.725364in}}%
\pgfpathcurveto{\pgfqpoint{0.946778in}{2.736415in}}{\pgfqpoint{0.942388in}{2.747014in}}{\pgfqpoint{0.934574in}{2.754827in}}%
\pgfpathcurveto{\pgfqpoint{0.926760in}{2.762641in}}{\pgfqpoint{0.916161in}{2.767031in}}{\pgfqpoint{0.905111in}{2.767031in}}%
\pgfpathcurveto{\pgfqpoint{0.894061in}{2.767031in}}{\pgfqpoint{0.883462in}{2.762641in}}{\pgfqpoint{0.875649in}{2.754827in}}%
\pgfpathcurveto{\pgfqpoint{0.867835in}{2.747014in}}{\pgfqpoint{0.863445in}{2.736415in}}{\pgfqpoint{0.863445in}{2.725364in}}%
\pgfpathcurveto{\pgfqpoint{0.863445in}{2.714314in}}{\pgfqpoint{0.867835in}{2.703715in}}{\pgfqpoint{0.875649in}{2.695902in}}%
\pgfpathcurveto{\pgfqpoint{0.883462in}{2.688088in}}{\pgfqpoint{0.894061in}{2.683698in}}{\pgfqpoint{0.905111in}{2.683698in}}%
\pgfpathlineto{\pgfqpoint{0.905111in}{2.683698in}}%
\pgfpathclose%
\pgfusepath{stroke}%
\end{pgfscope}%
\begin{pgfscope}%
\pgfpathrectangle{\pgfqpoint{0.393053in}{0.375000in}}{\pgfqpoint{6.356833in}{5.175000in}}%
\pgfusepath{clip}%
\pgfsetbuttcap%
\pgfsetroundjoin%
\pgfsetlinewidth{1.003750pt}%
\definecolor{currentstroke}{rgb}{0.827451,0.827451,0.827451}%
\pgfsetstrokecolor{currentstroke}%
\pgfsetdash{}{0pt}%
\pgfpathmoveto{\pgfqpoint{3.308571in}{0.683593in}}%
\pgfpathcurveto{\pgfqpoint{3.319621in}{0.683593in}}{\pgfqpoint{3.330220in}{0.687984in}}{\pgfqpoint{3.338034in}{0.695797in}}%
\pgfpathcurveto{\pgfqpoint{3.345848in}{0.703611in}}{\pgfqpoint{3.350238in}{0.714210in}}{\pgfqpoint{3.350238in}{0.725260in}}%
\pgfpathcurveto{\pgfqpoint{3.350238in}{0.736310in}}{\pgfqpoint{3.345848in}{0.746909in}}{\pgfqpoint{3.338034in}{0.754723in}}%
\pgfpathcurveto{\pgfqpoint{3.330220in}{0.762537in}}{\pgfqpoint{3.319621in}{0.766927in}}{\pgfqpoint{3.308571in}{0.766927in}}%
\pgfpathcurveto{\pgfqpoint{3.297521in}{0.766927in}}{\pgfqpoint{3.286922in}{0.762537in}}{\pgfqpoint{3.279108in}{0.754723in}}%
\pgfpathcurveto{\pgfqpoint{3.271295in}{0.746909in}}{\pgfqpoint{3.266904in}{0.736310in}}{\pgfqpoint{3.266904in}{0.725260in}}%
\pgfpathcurveto{\pgfqpoint{3.266904in}{0.714210in}}{\pgfqpoint{3.271295in}{0.703611in}}{\pgfqpoint{3.279108in}{0.695797in}}%
\pgfpathcurveto{\pgfqpoint{3.286922in}{0.687984in}}{\pgfqpoint{3.297521in}{0.683593in}}{\pgfqpoint{3.308571in}{0.683593in}}%
\pgfpathlineto{\pgfqpoint{3.308571in}{0.683593in}}%
\pgfpathclose%
\pgfusepath{stroke}%
\end{pgfscope}%
\begin{pgfscope}%
\pgfpathrectangle{\pgfqpoint{0.393053in}{0.375000in}}{\pgfqpoint{6.356833in}{5.175000in}}%
\pgfusepath{clip}%
\pgfsetbuttcap%
\pgfsetroundjoin%
\pgfsetlinewidth{1.003750pt}%
\definecolor{currentstroke}{rgb}{0.827451,0.827451,0.827451}%
\pgfsetstrokecolor{currentstroke}%
\pgfsetdash{}{0pt}%
\pgfpathmoveto{\pgfqpoint{3.844148in}{0.542510in}}%
\pgfpathcurveto{\pgfqpoint{3.855198in}{0.542510in}}{\pgfqpoint{3.865797in}{0.546901in}}{\pgfqpoint{3.873610in}{0.554714in}}%
\pgfpathcurveto{\pgfqpoint{3.881424in}{0.562528in}}{\pgfqpoint{3.885814in}{0.573127in}}{\pgfqpoint{3.885814in}{0.584177in}}%
\pgfpathcurveto{\pgfqpoint{3.885814in}{0.595227in}}{\pgfqpoint{3.881424in}{0.605826in}}{\pgfqpoint{3.873610in}{0.613640in}}%
\pgfpathcurveto{\pgfqpoint{3.865797in}{0.621453in}}{\pgfqpoint{3.855198in}{0.625844in}}{\pgfqpoint{3.844148in}{0.625844in}}%
\pgfpathcurveto{\pgfqpoint{3.833097in}{0.625844in}}{\pgfqpoint{3.822498in}{0.621453in}}{\pgfqpoint{3.814685in}{0.613640in}}%
\pgfpathcurveto{\pgfqpoint{3.806871in}{0.605826in}}{\pgfqpoint{3.802481in}{0.595227in}}{\pgfqpoint{3.802481in}{0.584177in}}%
\pgfpathcurveto{\pgfqpoint{3.802481in}{0.573127in}}{\pgfqpoint{3.806871in}{0.562528in}}{\pgfqpoint{3.814685in}{0.554714in}}%
\pgfpathcurveto{\pgfqpoint{3.822498in}{0.546901in}}{\pgfqpoint{3.833097in}{0.542510in}}{\pgfqpoint{3.844148in}{0.542510in}}%
\pgfpathlineto{\pgfqpoint{3.844148in}{0.542510in}}%
\pgfpathclose%
\pgfusepath{stroke}%
\end{pgfscope}%
\begin{pgfscope}%
\pgfpathrectangle{\pgfqpoint{0.393053in}{0.375000in}}{\pgfqpoint{6.356833in}{5.175000in}}%
\pgfusepath{clip}%
\pgfsetbuttcap%
\pgfsetroundjoin%
\pgfsetlinewidth{1.003750pt}%
\definecolor{currentstroke}{rgb}{0.827451,0.827451,0.827451}%
\pgfsetstrokecolor{currentstroke}%
\pgfsetdash{}{0pt}%
\pgfpathmoveto{\pgfqpoint{5.339131in}{0.342839in}}%
\pgfpathcurveto{\pgfqpoint{5.350181in}{0.342839in}}{\pgfqpoint{5.360780in}{0.347229in}}{\pgfqpoint{5.368594in}{0.355043in}}%
\pgfpathcurveto{\pgfqpoint{5.376407in}{0.362857in}}{\pgfqpoint{5.380798in}{0.373456in}}{\pgfqpoint{5.380798in}{0.384506in}}%
\pgfpathcurveto{\pgfqpoint{5.380798in}{0.395556in}}{\pgfqpoint{5.376407in}{0.406155in}}{\pgfqpoint{5.368594in}{0.413968in}}%
\pgfpathcurveto{\pgfqpoint{5.360780in}{0.421782in}}{\pgfqpoint{5.350181in}{0.426172in}}{\pgfqpoint{5.339131in}{0.426172in}}%
\pgfpathcurveto{\pgfqpoint{5.328081in}{0.426172in}}{\pgfqpoint{5.317482in}{0.421782in}}{\pgfqpoint{5.309668in}{0.413968in}}%
\pgfpathcurveto{\pgfqpoint{5.301855in}{0.406155in}}{\pgfqpoint{5.297464in}{0.395556in}}{\pgfqpoint{5.297464in}{0.384506in}}%
\pgfpathcurveto{\pgfqpoint{5.297464in}{0.373456in}}{\pgfqpoint{5.301855in}{0.362857in}}{\pgfqpoint{5.309668in}{0.355043in}}%
\pgfpathcurveto{\pgfqpoint{5.317482in}{0.347229in}}{\pgfqpoint{5.328081in}{0.342839in}}{\pgfqpoint{5.339131in}{0.342839in}}%
\pgfusepath{stroke}%
\end{pgfscope}%
\begin{pgfscope}%
\pgfpathrectangle{\pgfqpoint{0.393053in}{0.375000in}}{\pgfqpoint{6.356833in}{5.175000in}}%
\pgfusepath{clip}%
\pgfsetbuttcap%
\pgfsetroundjoin%
\pgfsetlinewidth{1.003750pt}%
\definecolor{currentstroke}{rgb}{0.827451,0.827451,0.827451}%
\pgfsetstrokecolor{currentstroke}%
\pgfsetdash{}{0pt}%
\pgfpathmoveto{\pgfqpoint{4.537650in}{0.413917in}}%
\pgfpathcurveto{\pgfqpoint{4.548700in}{0.413917in}}{\pgfqpoint{4.559300in}{0.418307in}}{\pgfqpoint{4.567113in}{0.426121in}}%
\pgfpathcurveto{\pgfqpoint{4.574927in}{0.433935in}}{\pgfqpoint{4.579317in}{0.444534in}}{\pgfqpoint{4.579317in}{0.455584in}}%
\pgfpathcurveto{\pgfqpoint{4.579317in}{0.466634in}}{\pgfqpoint{4.574927in}{0.477233in}}{\pgfqpoint{4.567113in}{0.485047in}}%
\pgfpathcurveto{\pgfqpoint{4.559300in}{0.492860in}}{\pgfqpoint{4.548700in}{0.497250in}}{\pgfqpoint{4.537650in}{0.497250in}}%
\pgfpathcurveto{\pgfqpoint{4.526600in}{0.497250in}}{\pgfqpoint{4.516001in}{0.492860in}}{\pgfqpoint{4.508188in}{0.485047in}}%
\pgfpathcurveto{\pgfqpoint{4.500374in}{0.477233in}}{\pgfqpoint{4.495984in}{0.466634in}}{\pgfqpoint{4.495984in}{0.455584in}}%
\pgfpathcurveto{\pgfqpoint{4.495984in}{0.444534in}}{\pgfqpoint{4.500374in}{0.433935in}}{\pgfqpoint{4.508188in}{0.426121in}}%
\pgfpathcurveto{\pgfqpoint{4.516001in}{0.418307in}}{\pgfqpoint{4.526600in}{0.413917in}}{\pgfqpoint{4.537650in}{0.413917in}}%
\pgfpathlineto{\pgfqpoint{4.537650in}{0.413917in}}%
\pgfpathclose%
\pgfusepath{stroke}%
\end{pgfscope}%
\begin{pgfscope}%
\pgfpathrectangle{\pgfqpoint{0.393053in}{0.375000in}}{\pgfqpoint{6.356833in}{5.175000in}}%
\pgfusepath{clip}%
\pgfsetbuttcap%
\pgfsetroundjoin%
\pgfsetlinewidth{1.003750pt}%
\definecolor{currentstroke}{rgb}{0.827451,0.827451,0.827451}%
\pgfsetstrokecolor{currentstroke}%
\pgfsetdash{}{0pt}%
\pgfpathmoveto{\pgfqpoint{4.065017in}{0.476607in}}%
\pgfpathcurveto{\pgfqpoint{4.076067in}{0.476607in}}{\pgfqpoint{4.086666in}{0.480997in}}{\pgfqpoint{4.094480in}{0.488811in}}%
\pgfpathcurveto{\pgfqpoint{4.102293in}{0.496625in}}{\pgfqpoint{4.106684in}{0.507224in}}{\pgfqpoint{4.106684in}{0.518274in}}%
\pgfpathcurveto{\pgfqpoint{4.106684in}{0.529324in}}{\pgfqpoint{4.102293in}{0.539923in}}{\pgfqpoint{4.094480in}{0.547737in}}%
\pgfpathcurveto{\pgfqpoint{4.086666in}{0.555550in}}{\pgfqpoint{4.076067in}{0.559941in}}{\pgfqpoint{4.065017in}{0.559941in}}%
\pgfpathcurveto{\pgfqpoint{4.053967in}{0.559941in}}{\pgfqpoint{4.043368in}{0.555550in}}{\pgfqpoint{4.035554in}{0.547737in}}%
\pgfpathcurveto{\pgfqpoint{4.027741in}{0.539923in}}{\pgfqpoint{4.023350in}{0.529324in}}{\pgfqpoint{4.023350in}{0.518274in}}%
\pgfpathcurveto{\pgfqpoint{4.023350in}{0.507224in}}{\pgfqpoint{4.027741in}{0.496625in}}{\pgfqpoint{4.035554in}{0.488811in}}%
\pgfpathcurveto{\pgfqpoint{4.043368in}{0.480997in}}{\pgfqpoint{4.053967in}{0.476607in}}{\pgfqpoint{4.065017in}{0.476607in}}%
\pgfpathlineto{\pgfqpoint{4.065017in}{0.476607in}}%
\pgfpathclose%
\pgfusepath{stroke}%
\end{pgfscope}%
\begin{pgfscope}%
\pgfpathrectangle{\pgfqpoint{0.393053in}{0.375000in}}{\pgfqpoint{6.356833in}{5.175000in}}%
\pgfusepath{clip}%
\pgfsetbuttcap%
\pgfsetroundjoin%
\pgfsetlinewidth{1.003750pt}%
\definecolor{currentstroke}{rgb}{0.827451,0.827451,0.827451}%
\pgfsetstrokecolor{currentstroke}%
\pgfsetdash{}{0pt}%
\pgfpathmoveto{\pgfqpoint{2.000452in}{1.370955in}}%
\pgfpathcurveto{\pgfqpoint{2.011502in}{1.370955in}}{\pgfqpoint{2.022101in}{1.375345in}}{\pgfqpoint{2.029914in}{1.383159in}}%
\pgfpathcurveto{\pgfqpoint{2.037728in}{1.390972in}}{\pgfqpoint{2.042118in}{1.401571in}}{\pgfqpoint{2.042118in}{1.412622in}}%
\pgfpathcurveto{\pgfqpoint{2.042118in}{1.423672in}}{\pgfqpoint{2.037728in}{1.434271in}}{\pgfqpoint{2.029914in}{1.442084in}}%
\pgfpathcurveto{\pgfqpoint{2.022101in}{1.449898in}}{\pgfqpoint{2.011502in}{1.454288in}}{\pgfqpoint{2.000452in}{1.454288in}}%
\pgfpathcurveto{\pgfqpoint{1.989401in}{1.454288in}}{\pgfqpoint{1.978802in}{1.449898in}}{\pgfqpoint{1.970989in}{1.442084in}}%
\pgfpathcurveto{\pgfqpoint{1.963175in}{1.434271in}}{\pgfqpoint{1.958785in}{1.423672in}}{\pgfqpoint{1.958785in}{1.412622in}}%
\pgfpathcurveto{\pgfqpoint{1.958785in}{1.401571in}}{\pgfqpoint{1.963175in}{1.390972in}}{\pgfqpoint{1.970989in}{1.383159in}}%
\pgfpathcurveto{\pgfqpoint{1.978802in}{1.375345in}}{\pgfqpoint{1.989401in}{1.370955in}}{\pgfqpoint{2.000452in}{1.370955in}}%
\pgfpathlineto{\pgfqpoint{2.000452in}{1.370955in}}%
\pgfpathclose%
\pgfusepath{stroke}%
\end{pgfscope}%
\begin{pgfscope}%
\pgfpathrectangle{\pgfqpoint{0.393053in}{0.375000in}}{\pgfqpoint{6.356833in}{5.175000in}}%
\pgfusepath{clip}%
\pgfsetbuttcap%
\pgfsetroundjoin%
\pgfsetlinewidth{1.003750pt}%
\definecolor{currentstroke}{rgb}{0.827451,0.827451,0.827451}%
\pgfsetstrokecolor{currentstroke}%
\pgfsetdash{}{0pt}%
\pgfpathmoveto{\pgfqpoint{1.019447in}{2.455876in}}%
\pgfpathcurveto{\pgfqpoint{1.030497in}{2.455876in}}{\pgfqpoint{1.041096in}{2.460267in}}{\pgfqpoint{1.048910in}{2.468080in}}%
\pgfpathcurveto{\pgfqpoint{1.056723in}{2.475894in}}{\pgfqpoint{1.061113in}{2.486493in}}{\pgfqpoint{1.061113in}{2.497543in}}%
\pgfpathcurveto{\pgfqpoint{1.061113in}{2.508593in}}{\pgfqpoint{1.056723in}{2.519192in}}{\pgfqpoint{1.048910in}{2.527006in}}%
\pgfpathcurveto{\pgfqpoint{1.041096in}{2.534819in}}{\pgfqpoint{1.030497in}{2.539210in}}{\pgfqpoint{1.019447in}{2.539210in}}%
\pgfpathcurveto{\pgfqpoint{1.008397in}{2.539210in}}{\pgfqpoint{0.997798in}{2.534819in}}{\pgfqpoint{0.989984in}{2.527006in}}%
\pgfpathcurveto{\pgfqpoint{0.982170in}{2.519192in}}{\pgfqpoint{0.977780in}{2.508593in}}{\pgfqpoint{0.977780in}{2.497543in}}%
\pgfpathcurveto{\pgfqpoint{0.977780in}{2.486493in}}{\pgfqpoint{0.982170in}{2.475894in}}{\pgfqpoint{0.989984in}{2.468080in}}%
\pgfpathcurveto{\pgfqpoint{0.997798in}{2.460267in}}{\pgfqpoint{1.008397in}{2.455876in}}{\pgfqpoint{1.019447in}{2.455876in}}%
\pgfpathlineto{\pgfqpoint{1.019447in}{2.455876in}}%
\pgfpathclose%
\pgfusepath{stroke}%
\end{pgfscope}%
\begin{pgfscope}%
\pgfpathrectangle{\pgfqpoint{0.393053in}{0.375000in}}{\pgfqpoint{6.356833in}{5.175000in}}%
\pgfusepath{clip}%
\pgfsetbuttcap%
\pgfsetroundjoin%
\pgfsetlinewidth{1.003750pt}%
\definecolor{currentstroke}{rgb}{0.827451,0.827451,0.827451}%
\pgfsetstrokecolor{currentstroke}%
\pgfsetdash{}{0pt}%
\pgfpathmoveto{\pgfqpoint{1.234935in}{2.069821in}}%
\pgfpathcurveto{\pgfqpoint{1.245986in}{2.069821in}}{\pgfqpoint{1.256585in}{2.074212in}}{\pgfqpoint{1.264398in}{2.082025in}}%
\pgfpathcurveto{\pgfqpoint{1.272212in}{2.089839in}}{\pgfqpoint{1.276602in}{2.100438in}}{\pgfqpoint{1.276602in}{2.111488in}}%
\pgfpathcurveto{\pgfqpoint{1.276602in}{2.122538in}}{\pgfqpoint{1.272212in}{2.133137in}}{\pgfqpoint{1.264398in}{2.140951in}}%
\pgfpathcurveto{\pgfqpoint{1.256585in}{2.148764in}}{\pgfqpoint{1.245986in}{2.153155in}}{\pgfqpoint{1.234935in}{2.153155in}}%
\pgfpathcurveto{\pgfqpoint{1.223885in}{2.153155in}}{\pgfqpoint{1.213286in}{2.148764in}}{\pgfqpoint{1.205473in}{2.140951in}}%
\pgfpathcurveto{\pgfqpoint{1.197659in}{2.133137in}}{\pgfqpoint{1.193269in}{2.122538in}}{\pgfqpoint{1.193269in}{2.111488in}}%
\pgfpathcurveto{\pgfqpoint{1.193269in}{2.100438in}}{\pgfqpoint{1.197659in}{2.089839in}}{\pgfqpoint{1.205473in}{2.082025in}}%
\pgfpathcurveto{\pgfqpoint{1.213286in}{2.074212in}}{\pgfqpoint{1.223885in}{2.069821in}}{\pgfqpoint{1.234935in}{2.069821in}}%
\pgfpathlineto{\pgfqpoint{1.234935in}{2.069821in}}%
\pgfpathclose%
\pgfusepath{stroke}%
\end{pgfscope}%
\begin{pgfscope}%
\pgfpathrectangle{\pgfqpoint{0.393053in}{0.375000in}}{\pgfqpoint{6.356833in}{5.175000in}}%
\pgfusepath{clip}%
\pgfsetbuttcap%
\pgfsetroundjoin%
\pgfsetlinewidth{1.003750pt}%
\definecolor{currentstroke}{rgb}{0.827451,0.827451,0.827451}%
\pgfsetstrokecolor{currentstroke}%
\pgfsetdash{}{0pt}%
\pgfpathmoveto{\pgfqpoint{0.911257in}{2.570368in}}%
\pgfpathcurveto{\pgfqpoint{0.922307in}{2.570368in}}{\pgfqpoint{0.932906in}{2.574759in}}{\pgfqpoint{0.940720in}{2.582572in}}%
\pgfpathcurveto{\pgfqpoint{0.948533in}{2.590386in}}{\pgfqpoint{0.952923in}{2.600985in}}{\pgfqpoint{0.952923in}{2.612035in}}%
\pgfpathcurveto{\pgfqpoint{0.952923in}{2.623085in}}{\pgfqpoint{0.948533in}{2.633684in}}{\pgfqpoint{0.940720in}{2.641498in}}%
\pgfpathcurveto{\pgfqpoint{0.932906in}{2.649311in}}{\pgfqpoint{0.922307in}{2.653702in}}{\pgfqpoint{0.911257in}{2.653702in}}%
\pgfpathcurveto{\pgfqpoint{0.900207in}{2.653702in}}{\pgfqpoint{0.889608in}{2.649311in}}{\pgfqpoint{0.881794in}{2.641498in}}%
\pgfpathcurveto{\pgfqpoint{0.873980in}{2.633684in}}{\pgfqpoint{0.869590in}{2.623085in}}{\pgfqpoint{0.869590in}{2.612035in}}%
\pgfpathcurveto{\pgfqpoint{0.869590in}{2.600985in}}{\pgfqpoint{0.873980in}{2.590386in}}{\pgfqpoint{0.881794in}{2.582572in}}%
\pgfpathcurveto{\pgfqpoint{0.889608in}{2.574759in}}{\pgfqpoint{0.900207in}{2.570368in}}{\pgfqpoint{0.911257in}{2.570368in}}%
\pgfpathlineto{\pgfqpoint{0.911257in}{2.570368in}}%
\pgfpathclose%
\pgfusepath{stroke}%
\end{pgfscope}%
\begin{pgfscope}%
\pgfpathrectangle{\pgfqpoint{0.393053in}{0.375000in}}{\pgfqpoint{6.356833in}{5.175000in}}%
\pgfusepath{clip}%
\pgfsetbuttcap%
\pgfsetroundjoin%
\pgfsetlinewidth{1.003750pt}%
\definecolor{currentstroke}{rgb}{0.827451,0.827451,0.827451}%
\pgfsetstrokecolor{currentstroke}%
\pgfsetdash{}{0pt}%
\pgfpathmoveto{\pgfqpoint{1.375643in}{1.906128in}}%
\pgfpathcurveto{\pgfqpoint{1.386694in}{1.906128in}}{\pgfqpoint{1.397293in}{1.910518in}}{\pgfqpoint{1.405106in}{1.918331in}}%
\pgfpathcurveto{\pgfqpoint{1.412920in}{1.926145in}}{\pgfqpoint{1.417310in}{1.936744in}}{\pgfqpoint{1.417310in}{1.947794in}}%
\pgfpathcurveto{\pgfqpoint{1.417310in}{1.958844in}}{\pgfqpoint{1.412920in}{1.969443in}}{\pgfqpoint{1.405106in}{1.977257in}}%
\pgfpathcurveto{\pgfqpoint{1.397293in}{1.985071in}}{\pgfqpoint{1.386694in}{1.989461in}}{\pgfqpoint{1.375643in}{1.989461in}}%
\pgfpathcurveto{\pgfqpoint{1.364593in}{1.989461in}}{\pgfqpoint{1.353994in}{1.985071in}}{\pgfqpoint{1.346181in}{1.977257in}}%
\pgfpathcurveto{\pgfqpoint{1.338367in}{1.969443in}}{\pgfqpoint{1.333977in}{1.958844in}}{\pgfqpoint{1.333977in}{1.947794in}}%
\pgfpathcurveto{\pgfqpoint{1.333977in}{1.936744in}}{\pgfqpoint{1.338367in}{1.926145in}}{\pgfqpoint{1.346181in}{1.918331in}}%
\pgfpathcurveto{\pgfqpoint{1.353994in}{1.910518in}}{\pgfqpoint{1.364593in}{1.906128in}}{\pgfqpoint{1.375643in}{1.906128in}}%
\pgfpathlineto{\pgfqpoint{1.375643in}{1.906128in}}%
\pgfpathclose%
\pgfusepath{stroke}%
\end{pgfscope}%
\begin{pgfscope}%
\pgfpathrectangle{\pgfqpoint{0.393053in}{0.375000in}}{\pgfqpoint{6.356833in}{5.175000in}}%
\pgfusepath{clip}%
\pgfsetbuttcap%
\pgfsetroundjoin%
\pgfsetlinewidth{1.003750pt}%
\definecolor{currentstroke}{rgb}{0.827451,0.827451,0.827451}%
\pgfsetstrokecolor{currentstroke}%
\pgfsetdash{}{0pt}%
\pgfpathmoveto{\pgfqpoint{0.483847in}{3.664437in}}%
\pgfpathcurveto{\pgfqpoint{0.494897in}{3.664437in}}{\pgfqpoint{0.505496in}{3.668827in}}{\pgfqpoint{0.513309in}{3.676640in}}%
\pgfpathcurveto{\pgfqpoint{0.521123in}{3.684454in}}{\pgfqpoint{0.525513in}{3.695053in}}{\pgfqpoint{0.525513in}{3.706103in}}%
\pgfpathcurveto{\pgfqpoint{0.525513in}{3.717153in}}{\pgfqpoint{0.521123in}{3.727752in}}{\pgfqpoint{0.513309in}{3.735566in}}%
\pgfpathcurveto{\pgfqpoint{0.505496in}{3.743380in}}{\pgfqpoint{0.494897in}{3.747770in}}{\pgfqpoint{0.483847in}{3.747770in}}%
\pgfpathcurveto{\pgfqpoint{0.472797in}{3.747770in}}{\pgfqpoint{0.462198in}{3.743380in}}{\pgfqpoint{0.454384in}{3.735566in}}%
\pgfpathcurveto{\pgfqpoint{0.446570in}{3.727752in}}{\pgfqpoint{0.442180in}{3.717153in}}{\pgfqpoint{0.442180in}{3.706103in}}%
\pgfpathcurveto{\pgfqpoint{0.442180in}{3.695053in}}{\pgfqpoint{0.446570in}{3.684454in}}{\pgfqpoint{0.454384in}{3.676640in}}%
\pgfpathcurveto{\pgfqpoint{0.462198in}{3.668827in}}{\pgfqpoint{0.472797in}{3.664437in}}{\pgfqpoint{0.483847in}{3.664437in}}%
\pgfpathlineto{\pgfqpoint{0.483847in}{3.664437in}}%
\pgfpathclose%
\pgfusepath{stroke}%
\end{pgfscope}%
\begin{pgfscope}%
\pgfpathrectangle{\pgfqpoint{0.393053in}{0.375000in}}{\pgfqpoint{6.356833in}{5.175000in}}%
\pgfusepath{clip}%
\pgfsetbuttcap%
\pgfsetroundjoin%
\pgfsetlinewidth{1.003750pt}%
\definecolor{currentstroke}{rgb}{0.827451,0.827451,0.827451}%
\pgfsetstrokecolor{currentstroke}%
\pgfsetdash{}{0pt}%
\pgfpathmoveto{\pgfqpoint{3.256902in}{0.730828in}}%
\pgfpathcurveto{\pgfqpoint{3.267953in}{0.730828in}}{\pgfqpoint{3.278552in}{0.735218in}}{\pgfqpoint{3.286365in}{0.743032in}}%
\pgfpathcurveto{\pgfqpoint{3.294179in}{0.750846in}}{\pgfqpoint{3.298569in}{0.761445in}}{\pgfqpoint{3.298569in}{0.772495in}}%
\pgfpathcurveto{\pgfqpoint{3.298569in}{0.783545in}}{\pgfqpoint{3.294179in}{0.794144in}}{\pgfqpoint{3.286365in}{0.801958in}}%
\pgfpathcurveto{\pgfqpoint{3.278552in}{0.809771in}}{\pgfqpoint{3.267953in}{0.814162in}}{\pgfqpoint{3.256902in}{0.814162in}}%
\pgfpathcurveto{\pgfqpoint{3.245852in}{0.814162in}}{\pgfqpoint{3.235253in}{0.809771in}}{\pgfqpoint{3.227440in}{0.801958in}}%
\pgfpathcurveto{\pgfqpoint{3.219626in}{0.794144in}}{\pgfqpoint{3.215236in}{0.783545in}}{\pgfqpoint{3.215236in}{0.772495in}}%
\pgfpathcurveto{\pgfqpoint{3.215236in}{0.761445in}}{\pgfqpoint{3.219626in}{0.750846in}}{\pgfqpoint{3.227440in}{0.743032in}}%
\pgfpathcurveto{\pgfqpoint{3.235253in}{0.735218in}}{\pgfqpoint{3.245852in}{0.730828in}}{\pgfqpoint{3.256902in}{0.730828in}}%
\pgfpathlineto{\pgfqpoint{3.256902in}{0.730828in}}%
\pgfpathclose%
\pgfusepath{stroke}%
\end{pgfscope}%
\begin{pgfscope}%
\pgfpathrectangle{\pgfqpoint{0.393053in}{0.375000in}}{\pgfqpoint{6.356833in}{5.175000in}}%
\pgfusepath{clip}%
\pgfsetbuttcap%
\pgfsetroundjoin%
\pgfsetlinewidth{1.003750pt}%
\definecolor{currentstroke}{rgb}{0.827451,0.827451,0.827451}%
\pgfsetstrokecolor{currentstroke}%
\pgfsetdash{}{0pt}%
\pgfpathmoveto{\pgfqpoint{2.145018in}{1.265722in}}%
\pgfpathcurveto{\pgfqpoint{2.156069in}{1.265722in}}{\pgfqpoint{2.166668in}{1.270112in}}{\pgfqpoint{2.174481in}{1.277926in}}%
\pgfpathcurveto{\pgfqpoint{2.182295in}{1.285739in}}{\pgfqpoint{2.186685in}{1.296338in}}{\pgfqpoint{2.186685in}{1.307389in}}%
\pgfpathcurveto{\pgfqpoint{2.186685in}{1.318439in}}{\pgfqpoint{2.182295in}{1.329038in}}{\pgfqpoint{2.174481in}{1.336851in}}%
\pgfpathcurveto{\pgfqpoint{2.166668in}{1.344665in}}{\pgfqpoint{2.156069in}{1.349055in}}{\pgfqpoint{2.145018in}{1.349055in}}%
\pgfpathcurveto{\pgfqpoint{2.133968in}{1.349055in}}{\pgfqpoint{2.123369in}{1.344665in}}{\pgfqpoint{2.115556in}{1.336851in}}%
\pgfpathcurveto{\pgfqpoint{2.107742in}{1.329038in}}{\pgfqpoint{2.103352in}{1.318439in}}{\pgfqpoint{2.103352in}{1.307389in}}%
\pgfpathcurveto{\pgfqpoint{2.103352in}{1.296338in}}{\pgfqpoint{2.107742in}{1.285739in}}{\pgfqpoint{2.115556in}{1.277926in}}%
\pgfpathcurveto{\pgfqpoint{2.123369in}{1.270112in}}{\pgfqpoint{2.133968in}{1.265722in}}{\pgfqpoint{2.145018in}{1.265722in}}%
\pgfpathlineto{\pgfqpoint{2.145018in}{1.265722in}}%
\pgfpathclose%
\pgfusepath{stroke}%
\end{pgfscope}%
\begin{pgfscope}%
\pgfpathrectangle{\pgfqpoint{0.393053in}{0.375000in}}{\pgfqpoint{6.356833in}{5.175000in}}%
\pgfusepath{clip}%
\pgfsetbuttcap%
\pgfsetroundjoin%
\pgfsetlinewidth{1.003750pt}%
\definecolor{currentstroke}{rgb}{0.827451,0.827451,0.827451}%
\pgfsetstrokecolor{currentstroke}%
\pgfsetdash{}{0pt}%
\pgfpathmoveto{\pgfqpoint{1.167227in}{2.196399in}}%
\pgfpathcurveto{\pgfqpoint{1.178278in}{2.196399in}}{\pgfqpoint{1.188877in}{2.200789in}}{\pgfqpoint{1.196690in}{2.208603in}}%
\pgfpathcurveto{\pgfqpoint{1.204504in}{2.216416in}}{\pgfqpoint{1.208894in}{2.227016in}}{\pgfqpoint{1.208894in}{2.238066in}}%
\pgfpathcurveto{\pgfqpoint{1.208894in}{2.249116in}}{\pgfqpoint{1.204504in}{2.259715in}}{\pgfqpoint{1.196690in}{2.267528in}}%
\pgfpathcurveto{\pgfqpoint{1.188877in}{2.275342in}}{\pgfqpoint{1.178278in}{2.279732in}}{\pgfqpoint{1.167227in}{2.279732in}}%
\pgfpathcurveto{\pgfqpoint{1.156177in}{2.279732in}}{\pgfqpoint{1.145578in}{2.275342in}}{\pgfqpoint{1.137765in}{2.267528in}}%
\pgfpathcurveto{\pgfqpoint{1.129951in}{2.259715in}}{\pgfqpoint{1.125561in}{2.249116in}}{\pgfqpoint{1.125561in}{2.238066in}}%
\pgfpathcurveto{\pgfqpoint{1.125561in}{2.227016in}}{\pgfqpoint{1.129951in}{2.216416in}}{\pgfqpoint{1.137765in}{2.208603in}}%
\pgfpathcurveto{\pgfqpoint{1.145578in}{2.200789in}}{\pgfqpoint{1.156177in}{2.196399in}}{\pgfqpoint{1.167227in}{2.196399in}}%
\pgfpathlineto{\pgfqpoint{1.167227in}{2.196399in}}%
\pgfpathclose%
\pgfusepath{stroke}%
\end{pgfscope}%
\begin{pgfscope}%
\pgfpathrectangle{\pgfqpoint{0.393053in}{0.375000in}}{\pgfqpoint{6.356833in}{5.175000in}}%
\pgfusepath{clip}%
\pgfsetbuttcap%
\pgfsetroundjoin%
\pgfsetlinewidth{1.003750pt}%
\definecolor{currentstroke}{rgb}{0.827451,0.827451,0.827451}%
\pgfsetstrokecolor{currentstroke}%
\pgfsetdash{}{0pt}%
\pgfpathmoveto{\pgfqpoint{2.862547in}{0.868774in}}%
\pgfpathcurveto{\pgfqpoint{2.873597in}{0.868774in}}{\pgfqpoint{2.884196in}{0.873164in}}{\pgfqpoint{2.892010in}{0.880977in}}%
\pgfpathcurveto{\pgfqpoint{2.899824in}{0.888791in}}{\pgfqpoint{2.904214in}{0.899390in}}{\pgfqpoint{2.904214in}{0.910440in}}%
\pgfpathcurveto{\pgfqpoint{2.904214in}{0.921490in}}{\pgfqpoint{2.899824in}{0.932089in}}{\pgfqpoint{2.892010in}{0.939903in}}%
\pgfpathcurveto{\pgfqpoint{2.884196in}{0.947717in}}{\pgfqpoint{2.873597in}{0.952107in}}{\pgfqpoint{2.862547in}{0.952107in}}%
\pgfpathcurveto{\pgfqpoint{2.851497in}{0.952107in}}{\pgfqpoint{2.840898in}{0.947717in}}{\pgfqpoint{2.833084in}{0.939903in}}%
\pgfpathcurveto{\pgfqpoint{2.825271in}{0.932089in}}{\pgfqpoint{2.820880in}{0.921490in}}{\pgfqpoint{2.820880in}{0.910440in}}%
\pgfpathcurveto{\pgfqpoint{2.820880in}{0.899390in}}{\pgfqpoint{2.825271in}{0.888791in}}{\pgfqpoint{2.833084in}{0.880977in}}%
\pgfpathcurveto{\pgfqpoint{2.840898in}{0.873164in}}{\pgfqpoint{2.851497in}{0.868774in}}{\pgfqpoint{2.862547in}{0.868774in}}%
\pgfpathlineto{\pgfqpoint{2.862547in}{0.868774in}}%
\pgfpathclose%
\pgfusepath{stroke}%
\end{pgfscope}%
\begin{pgfscope}%
\pgfpathrectangle{\pgfqpoint{0.393053in}{0.375000in}}{\pgfqpoint{6.356833in}{5.175000in}}%
\pgfusepath{clip}%
\pgfsetbuttcap%
\pgfsetroundjoin%
\pgfsetlinewidth{1.003750pt}%
\definecolor{currentstroke}{rgb}{0.827451,0.827451,0.827451}%
\pgfsetstrokecolor{currentstroke}%
\pgfsetdash{}{0pt}%
\pgfpathmoveto{\pgfqpoint{4.310929in}{0.438129in}}%
\pgfpathcurveto{\pgfqpoint{4.321979in}{0.438129in}}{\pgfqpoint{4.332578in}{0.442519in}}{\pgfqpoint{4.340392in}{0.450333in}}%
\pgfpathcurveto{\pgfqpoint{4.348205in}{0.458147in}}{\pgfqpoint{4.352595in}{0.468746in}}{\pgfqpoint{4.352595in}{0.479796in}}%
\pgfpathcurveto{\pgfqpoint{4.352595in}{0.490846in}}{\pgfqpoint{4.348205in}{0.501445in}}{\pgfqpoint{4.340392in}{0.509258in}}%
\pgfpathcurveto{\pgfqpoint{4.332578in}{0.517072in}}{\pgfqpoint{4.321979in}{0.521462in}}{\pgfqpoint{4.310929in}{0.521462in}}%
\pgfpathcurveto{\pgfqpoint{4.299879in}{0.521462in}}{\pgfqpoint{4.289280in}{0.517072in}}{\pgfqpoint{4.281466in}{0.509258in}}%
\pgfpathcurveto{\pgfqpoint{4.273652in}{0.501445in}}{\pgfqpoint{4.269262in}{0.490846in}}{\pgfqpoint{4.269262in}{0.479796in}}%
\pgfpathcurveto{\pgfqpoint{4.269262in}{0.468746in}}{\pgfqpoint{4.273652in}{0.458147in}}{\pgfqpoint{4.281466in}{0.450333in}}%
\pgfpathcurveto{\pgfqpoint{4.289280in}{0.442519in}}{\pgfqpoint{4.299879in}{0.438129in}}{\pgfqpoint{4.310929in}{0.438129in}}%
\pgfpathlineto{\pgfqpoint{4.310929in}{0.438129in}}%
\pgfpathclose%
\pgfusepath{stroke}%
\end{pgfscope}%
\begin{pgfscope}%
\pgfpathrectangle{\pgfqpoint{0.393053in}{0.375000in}}{\pgfqpoint{6.356833in}{5.175000in}}%
\pgfusepath{clip}%
\pgfsetbuttcap%
\pgfsetroundjoin%
\pgfsetlinewidth{1.003750pt}%
\definecolor{currentstroke}{rgb}{0.827451,0.827451,0.827451}%
\pgfsetstrokecolor{currentstroke}%
\pgfsetdash{}{0pt}%
\pgfpathmoveto{\pgfqpoint{3.077370in}{0.783906in}}%
\pgfpathcurveto{\pgfqpoint{3.088420in}{0.783906in}}{\pgfqpoint{3.099019in}{0.788296in}}{\pgfqpoint{3.106832in}{0.796110in}}%
\pgfpathcurveto{\pgfqpoint{3.114646in}{0.803924in}}{\pgfqpoint{3.119036in}{0.814523in}}{\pgfqpoint{3.119036in}{0.825573in}}%
\pgfpathcurveto{\pgfqpoint{3.119036in}{0.836623in}}{\pgfqpoint{3.114646in}{0.847222in}}{\pgfqpoint{3.106832in}{0.855036in}}%
\pgfpathcurveto{\pgfqpoint{3.099019in}{0.862849in}}{\pgfqpoint{3.088420in}{0.867239in}}{\pgfqpoint{3.077370in}{0.867239in}}%
\pgfpathcurveto{\pgfqpoint{3.066319in}{0.867239in}}{\pgfqpoint{3.055720in}{0.862849in}}{\pgfqpoint{3.047907in}{0.855036in}}%
\pgfpathcurveto{\pgfqpoint{3.040093in}{0.847222in}}{\pgfqpoint{3.035703in}{0.836623in}}{\pgfqpoint{3.035703in}{0.825573in}}%
\pgfpathcurveto{\pgfqpoint{3.035703in}{0.814523in}}{\pgfqpoint{3.040093in}{0.803924in}}{\pgfqpoint{3.047907in}{0.796110in}}%
\pgfpathcurveto{\pgfqpoint{3.055720in}{0.788296in}}{\pgfqpoint{3.066319in}{0.783906in}}{\pgfqpoint{3.077370in}{0.783906in}}%
\pgfpathlineto{\pgfqpoint{3.077370in}{0.783906in}}%
\pgfpathclose%
\pgfusepath{stroke}%
\end{pgfscope}%
\begin{pgfscope}%
\pgfpathrectangle{\pgfqpoint{0.393053in}{0.375000in}}{\pgfqpoint{6.356833in}{5.175000in}}%
\pgfusepath{clip}%
\pgfsetbuttcap%
\pgfsetroundjoin%
\pgfsetlinewidth{1.003750pt}%
\definecolor{currentstroke}{rgb}{0.827451,0.827451,0.827451}%
\pgfsetstrokecolor{currentstroke}%
\pgfsetdash{}{0pt}%
\pgfpathmoveto{\pgfqpoint{0.773700in}{2.823025in}}%
\pgfpathcurveto{\pgfqpoint{0.784750in}{2.823025in}}{\pgfqpoint{0.795350in}{2.827415in}}{\pgfqpoint{0.803163in}{2.835229in}}%
\pgfpathcurveto{\pgfqpoint{0.810977in}{2.843042in}}{\pgfqpoint{0.815367in}{2.853641in}}{\pgfqpoint{0.815367in}{2.864691in}}%
\pgfpathcurveto{\pgfqpoint{0.815367in}{2.875742in}}{\pgfqpoint{0.810977in}{2.886341in}}{\pgfqpoint{0.803163in}{2.894154in}}%
\pgfpathcurveto{\pgfqpoint{0.795350in}{2.901968in}}{\pgfqpoint{0.784750in}{2.906358in}}{\pgfqpoint{0.773700in}{2.906358in}}%
\pgfpathcurveto{\pgfqpoint{0.762650in}{2.906358in}}{\pgfqpoint{0.752051in}{2.901968in}}{\pgfqpoint{0.744238in}{2.894154in}}%
\pgfpathcurveto{\pgfqpoint{0.736424in}{2.886341in}}{\pgfqpoint{0.732034in}{2.875742in}}{\pgfqpoint{0.732034in}{2.864691in}}%
\pgfpathcurveto{\pgfqpoint{0.732034in}{2.853641in}}{\pgfqpoint{0.736424in}{2.843042in}}{\pgfqpoint{0.744238in}{2.835229in}}%
\pgfpathcurveto{\pgfqpoint{0.752051in}{2.827415in}}{\pgfqpoint{0.762650in}{2.823025in}}{\pgfqpoint{0.773700in}{2.823025in}}%
\pgfpathlineto{\pgfqpoint{0.773700in}{2.823025in}}%
\pgfpathclose%
\pgfusepath{stroke}%
\end{pgfscope}%
\begin{pgfscope}%
\pgfpathrectangle{\pgfqpoint{0.393053in}{0.375000in}}{\pgfqpoint{6.356833in}{5.175000in}}%
\pgfusepath{clip}%
\pgfsetbuttcap%
\pgfsetroundjoin%
\pgfsetlinewidth{1.003750pt}%
\definecolor{currentstroke}{rgb}{0.827451,0.827451,0.827451}%
\pgfsetstrokecolor{currentstroke}%
\pgfsetdash{}{0pt}%
\pgfpathmoveto{\pgfqpoint{5.340211in}{0.342419in}}%
\pgfpathcurveto{\pgfqpoint{5.351261in}{0.342419in}}{\pgfqpoint{5.361860in}{0.346810in}}{\pgfqpoint{5.369673in}{0.354623in}}%
\pgfpathcurveto{\pgfqpoint{5.377487in}{0.362437in}}{\pgfqpoint{5.381877in}{0.373036in}}{\pgfqpoint{5.381877in}{0.384086in}}%
\pgfpathcurveto{\pgfqpoint{5.381877in}{0.395136in}}{\pgfqpoint{5.377487in}{0.405735in}}{\pgfqpoint{5.369673in}{0.413549in}}%
\pgfpathcurveto{\pgfqpoint{5.361860in}{0.421362in}}{\pgfqpoint{5.351261in}{0.425753in}}{\pgfqpoint{5.340211in}{0.425753in}}%
\pgfpathcurveto{\pgfqpoint{5.329160in}{0.425753in}}{\pgfqpoint{5.318561in}{0.421362in}}{\pgfqpoint{5.310748in}{0.413549in}}%
\pgfpathcurveto{\pgfqpoint{5.302934in}{0.405735in}}{\pgfqpoint{5.298544in}{0.395136in}}{\pgfqpoint{5.298544in}{0.384086in}}%
\pgfpathcurveto{\pgfqpoint{5.298544in}{0.373036in}}{\pgfqpoint{5.302934in}{0.362437in}}{\pgfqpoint{5.310748in}{0.354623in}}%
\pgfpathcurveto{\pgfqpoint{5.318561in}{0.346810in}}{\pgfqpoint{5.329160in}{0.342419in}}{\pgfqpoint{5.340211in}{0.342419in}}%
\pgfusepath{stroke}%
\end{pgfscope}%
\begin{pgfscope}%
\pgfpathrectangle{\pgfqpoint{0.393053in}{0.375000in}}{\pgfqpoint{6.356833in}{5.175000in}}%
\pgfusepath{clip}%
\pgfsetbuttcap%
\pgfsetroundjoin%
\pgfsetlinewidth{1.003750pt}%
\definecolor{currentstroke}{rgb}{0.827451,0.827451,0.827451}%
\pgfsetstrokecolor{currentstroke}%
\pgfsetdash{}{0pt}%
\pgfpathmoveto{\pgfqpoint{1.078224in}{2.311634in}}%
\pgfpathcurveto{\pgfqpoint{1.089274in}{2.311634in}}{\pgfqpoint{1.099873in}{2.316024in}}{\pgfqpoint{1.107686in}{2.323838in}}%
\pgfpathcurveto{\pgfqpoint{1.115500in}{2.331651in}}{\pgfqpoint{1.119890in}{2.342250in}}{\pgfqpoint{1.119890in}{2.353300in}}%
\pgfpathcurveto{\pgfqpoint{1.119890in}{2.364351in}}{\pgfqpoint{1.115500in}{2.374950in}}{\pgfqpoint{1.107686in}{2.382763in}}%
\pgfpathcurveto{\pgfqpoint{1.099873in}{2.390577in}}{\pgfqpoint{1.089274in}{2.394967in}}{\pgfqpoint{1.078224in}{2.394967in}}%
\pgfpathcurveto{\pgfqpoint{1.067173in}{2.394967in}}{\pgfqpoint{1.056574in}{2.390577in}}{\pgfqpoint{1.048761in}{2.382763in}}%
\pgfpathcurveto{\pgfqpoint{1.040947in}{2.374950in}}{\pgfqpoint{1.036557in}{2.364351in}}{\pgfqpoint{1.036557in}{2.353300in}}%
\pgfpathcurveto{\pgfqpoint{1.036557in}{2.342250in}}{\pgfqpoint{1.040947in}{2.331651in}}{\pgfqpoint{1.048761in}{2.323838in}}%
\pgfpathcurveto{\pgfqpoint{1.056574in}{2.316024in}}{\pgfqpoint{1.067173in}{2.311634in}}{\pgfqpoint{1.078224in}{2.311634in}}%
\pgfpathlineto{\pgfqpoint{1.078224in}{2.311634in}}%
\pgfpathclose%
\pgfusepath{stroke}%
\end{pgfscope}%
\begin{pgfscope}%
\pgfpathrectangle{\pgfqpoint{0.393053in}{0.375000in}}{\pgfqpoint{6.356833in}{5.175000in}}%
\pgfusepath{clip}%
\pgfsetbuttcap%
\pgfsetroundjoin%
\pgfsetlinewidth{1.003750pt}%
\definecolor{currentstroke}{rgb}{0.827451,0.827451,0.827451}%
\pgfsetstrokecolor{currentstroke}%
\pgfsetdash{}{0pt}%
\pgfpathmoveto{\pgfqpoint{3.397541in}{0.662535in}}%
\pgfpathcurveto{\pgfqpoint{3.408591in}{0.662535in}}{\pgfqpoint{3.419190in}{0.666925in}}{\pgfqpoint{3.427004in}{0.674738in}}%
\pgfpathcurveto{\pgfqpoint{3.434817in}{0.682552in}}{\pgfqpoint{3.439208in}{0.693151in}}{\pgfqpoint{3.439208in}{0.704201in}}%
\pgfpathcurveto{\pgfqpoint{3.439208in}{0.715251in}}{\pgfqpoint{3.434817in}{0.725850in}}{\pgfqpoint{3.427004in}{0.733664in}}%
\pgfpathcurveto{\pgfqpoint{3.419190in}{0.741478in}}{\pgfqpoint{3.408591in}{0.745868in}}{\pgfqpoint{3.397541in}{0.745868in}}%
\pgfpathcurveto{\pgfqpoint{3.386491in}{0.745868in}}{\pgfqpoint{3.375892in}{0.741478in}}{\pgfqpoint{3.368078in}{0.733664in}}%
\pgfpathcurveto{\pgfqpoint{3.360265in}{0.725850in}}{\pgfqpoint{3.355874in}{0.715251in}}{\pgfqpoint{3.355874in}{0.704201in}}%
\pgfpathcurveto{\pgfqpoint{3.355874in}{0.693151in}}{\pgfqpoint{3.360265in}{0.682552in}}{\pgfqpoint{3.368078in}{0.674738in}}%
\pgfpathcurveto{\pgfqpoint{3.375892in}{0.666925in}}{\pgfqpoint{3.386491in}{0.662535in}}{\pgfqpoint{3.397541in}{0.662535in}}%
\pgfpathlineto{\pgfqpoint{3.397541in}{0.662535in}}%
\pgfpathclose%
\pgfusepath{stroke}%
\end{pgfscope}%
\begin{pgfscope}%
\pgfpathrectangle{\pgfqpoint{0.393053in}{0.375000in}}{\pgfqpoint{6.356833in}{5.175000in}}%
\pgfusepath{clip}%
\pgfsetbuttcap%
\pgfsetroundjoin%
\pgfsetlinewidth{1.003750pt}%
\definecolor{currentstroke}{rgb}{0.827451,0.827451,0.827451}%
\pgfsetstrokecolor{currentstroke}%
\pgfsetdash{}{0pt}%
\pgfpathmoveto{\pgfqpoint{1.934673in}{1.425939in}}%
\pgfpathcurveto{\pgfqpoint{1.945723in}{1.425939in}}{\pgfqpoint{1.956322in}{1.430329in}}{\pgfqpoint{1.964136in}{1.438142in}}%
\pgfpathcurveto{\pgfqpoint{1.971949in}{1.445956in}}{\pgfqpoint{1.976340in}{1.456555in}}{\pgfqpoint{1.976340in}{1.467605in}}%
\pgfpathcurveto{\pgfqpoint{1.976340in}{1.478655in}}{\pgfqpoint{1.971949in}{1.489254in}}{\pgfqpoint{1.964136in}{1.497068in}}%
\pgfpathcurveto{\pgfqpoint{1.956322in}{1.504882in}}{\pgfqpoint{1.945723in}{1.509272in}}{\pgfqpoint{1.934673in}{1.509272in}}%
\pgfpathcurveto{\pgfqpoint{1.923623in}{1.509272in}}{\pgfqpoint{1.913024in}{1.504882in}}{\pgfqpoint{1.905210in}{1.497068in}}%
\pgfpathcurveto{\pgfqpoint{1.897396in}{1.489254in}}{\pgfqpoint{1.893006in}{1.478655in}}{\pgfqpoint{1.893006in}{1.467605in}}%
\pgfpathcurveto{\pgfqpoint{1.893006in}{1.456555in}}{\pgfqpoint{1.897396in}{1.445956in}}{\pgfqpoint{1.905210in}{1.438142in}}%
\pgfpathcurveto{\pgfqpoint{1.913024in}{1.430329in}}{\pgfqpoint{1.923623in}{1.425939in}}{\pgfqpoint{1.934673in}{1.425939in}}%
\pgfpathlineto{\pgfqpoint{1.934673in}{1.425939in}}%
\pgfpathclose%
\pgfusepath{stroke}%
\end{pgfscope}%
\begin{pgfscope}%
\pgfpathrectangle{\pgfqpoint{0.393053in}{0.375000in}}{\pgfqpoint{6.356833in}{5.175000in}}%
\pgfusepath{clip}%
\pgfsetbuttcap%
\pgfsetroundjoin%
\pgfsetlinewidth{1.003750pt}%
\definecolor{currentstroke}{rgb}{0.827451,0.827451,0.827451}%
\pgfsetstrokecolor{currentstroke}%
\pgfsetdash{}{0pt}%
\pgfpathmoveto{\pgfqpoint{2.036490in}{1.340063in}}%
\pgfpathcurveto{\pgfqpoint{2.047540in}{1.340063in}}{\pgfqpoint{2.058139in}{1.344454in}}{\pgfqpoint{2.065953in}{1.352267in}}%
\pgfpathcurveto{\pgfqpoint{2.073767in}{1.360081in}}{\pgfqpoint{2.078157in}{1.370680in}}{\pgfqpoint{2.078157in}{1.381730in}}%
\pgfpathcurveto{\pgfqpoint{2.078157in}{1.392780in}}{\pgfqpoint{2.073767in}{1.403379in}}{\pgfqpoint{2.065953in}{1.411193in}}%
\pgfpathcurveto{\pgfqpoint{2.058139in}{1.419007in}}{\pgfqpoint{2.047540in}{1.423397in}}{\pgfqpoint{2.036490in}{1.423397in}}%
\pgfpathcurveto{\pgfqpoint{2.025440in}{1.423397in}}{\pgfqpoint{2.014841in}{1.419007in}}{\pgfqpoint{2.007027in}{1.411193in}}%
\pgfpathcurveto{\pgfqpoint{1.999214in}{1.403379in}}{\pgfqpoint{1.994824in}{1.392780in}}{\pgfqpoint{1.994824in}{1.381730in}}%
\pgfpathcurveto{\pgfqpoint{1.994824in}{1.370680in}}{\pgfqpoint{1.999214in}{1.360081in}}{\pgfqpoint{2.007027in}{1.352267in}}%
\pgfpathcurveto{\pgfqpoint{2.014841in}{1.344454in}}{\pgfqpoint{2.025440in}{1.340063in}}{\pgfqpoint{2.036490in}{1.340063in}}%
\pgfpathlineto{\pgfqpoint{2.036490in}{1.340063in}}%
\pgfpathclose%
\pgfusepath{stroke}%
\end{pgfscope}%
\begin{pgfscope}%
\pgfpathrectangle{\pgfqpoint{0.393053in}{0.375000in}}{\pgfqpoint{6.356833in}{5.175000in}}%
\pgfusepath{clip}%
\pgfsetbuttcap%
\pgfsetroundjoin%
\pgfsetlinewidth{1.003750pt}%
\definecolor{currentstroke}{rgb}{0.827451,0.827451,0.827451}%
\pgfsetstrokecolor{currentstroke}%
\pgfsetdash{}{0pt}%
\pgfpathmoveto{\pgfqpoint{2.629511in}{0.982910in}}%
\pgfpathcurveto{\pgfqpoint{2.640562in}{0.982910in}}{\pgfqpoint{2.651161in}{0.987300in}}{\pgfqpoint{2.658974in}{0.995114in}}%
\pgfpathcurveto{\pgfqpoint{2.666788in}{1.002928in}}{\pgfqpoint{2.671178in}{1.013527in}}{\pgfqpoint{2.671178in}{1.024577in}}%
\pgfpathcurveto{\pgfqpoint{2.671178in}{1.035627in}}{\pgfqpoint{2.666788in}{1.046226in}}{\pgfqpoint{2.658974in}{1.054040in}}%
\pgfpathcurveto{\pgfqpoint{2.651161in}{1.061853in}}{\pgfqpoint{2.640562in}{1.066243in}}{\pgfqpoint{2.629511in}{1.066243in}}%
\pgfpathcurveto{\pgfqpoint{2.618461in}{1.066243in}}{\pgfqpoint{2.607862in}{1.061853in}}{\pgfqpoint{2.600049in}{1.054040in}}%
\pgfpathcurveto{\pgfqpoint{2.592235in}{1.046226in}}{\pgfqpoint{2.587845in}{1.035627in}}{\pgfqpoint{2.587845in}{1.024577in}}%
\pgfpathcurveto{\pgfqpoint{2.587845in}{1.013527in}}{\pgfqpoint{2.592235in}{1.002928in}}{\pgfqpoint{2.600049in}{0.995114in}}%
\pgfpathcurveto{\pgfqpoint{2.607862in}{0.987300in}}{\pgfqpoint{2.618461in}{0.982910in}}{\pgfqpoint{2.629511in}{0.982910in}}%
\pgfpathlineto{\pgfqpoint{2.629511in}{0.982910in}}%
\pgfpathclose%
\pgfusepath{stroke}%
\end{pgfscope}%
\begin{pgfscope}%
\pgfpathrectangle{\pgfqpoint{0.393053in}{0.375000in}}{\pgfqpoint{6.356833in}{5.175000in}}%
\pgfusepath{clip}%
\pgfsetbuttcap%
\pgfsetroundjoin%
\pgfsetlinewidth{1.003750pt}%
\definecolor{currentstroke}{rgb}{0.827451,0.827451,0.827451}%
\pgfsetstrokecolor{currentstroke}%
\pgfsetdash{}{0pt}%
\pgfpathmoveto{\pgfqpoint{0.525326in}{3.470591in}}%
\pgfpathcurveto{\pgfqpoint{0.536376in}{3.470591in}}{\pgfqpoint{0.546975in}{3.474982in}}{\pgfqpoint{0.554788in}{3.482795in}}%
\pgfpathcurveto{\pgfqpoint{0.562602in}{3.490609in}}{\pgfqpoint{0.566992in}{3.501208in}}{\pgfqpoint{0.566992in}{3.512258in}}%
\pgfpathcurveto{\pgfqpoint{0.566992in}{3.523308in}}{\pgfqpoint{0.562602in}{3.533907in}}{\pgfqpoint{0.554788in}{3.541721in}}%
\pgfpathcurveto{\pgfqpoint{0.546975in}{3.549534in}}{\pgfqpoint{0.536376in}{3.553925in}}{\pgfqpoint{0.525326in}{3.553925in}}%
\pgfpathcurveto{\pgfqpoint{0.514275in}{3.553925in}}{\pgfqpoint{0.503676in}{3.549534in}}{\pgfqpoint{0.495863in}{3.541721in}}%
\pgfpathcurveto{\pgfqpoint{0.488049in}{3.533907in}}{\pgfqpoint{0.483659in}{3.523308in}}{\pgfqpoint{0.483659in}{3.512258in}}%
\pgfpathcurveto{\pgfqpoint{0.483659in}{3.501208in}}{\pgfqpoint{0.488049in}{3.490609in}}{\pgfqpoint{0.495863in}{3.482795in}}%
\pgfpathcurveto{\pgfqpoint{0.503676in}{3.474982in}}{\pgfqpoint{0.514275in}{3.470591in}}{\pgfqpoint{0.525326in}{3.470591in}}%
\pgfpathlineto{\pgfqpoint{0.525326in}{3.470591in}}%
\pgfpathclose%
\pgfusepath{stroke}%
\end{pgfscope}%
\begin{pgfscope}%
\pgfpathrectangle{\pgfqpoint{0.393053in}{0.375000in}}{\pgfqpoint{6.356833in}{5.175000in}}%
\pgfusepath{clip}%
\pgfsetbuttcap%
\pgfsetroundjoin%
\pgfsetlinewidth{1.003750pt}%
\definecolor{currentstroke}{rgb}{0.827451,0.827451,0.827451}%
\pgfsetstrokecolor{currentstroke}%
\pgfsetdash{}{0pt}%
\pgfpathmoveto{\pgfqpoint{3.677938in}{0.575928in}}%
\pgfpathcurveto{\pgfqpoint{3.688988in}{0.575928in}}{\pgfqpoint{3.699587in}{0.580318in}}{\pgfqpoint{3.707401in}{0.588132in}}%
\pgfpathcurveto{\pgfqpoint{3.715214in}{0.595945in}}{\pgfqpoint{3.719604in}{0.606544in}}{\pgfqpoint{3.719604in}{0.617595in}}%
\pgfpathcurveto{\pgfqpoint{3.719604in}{0.628645in}}{\pgfqpoint{3.715214in}{0.639244in}}{\pgfqpoint{3.707401in}{0.647057in}}%
\pgfpathcurveto{\pgfqpoint{3.699587in}{0.654871in}}{\pgfqpoint{3.688988in}{0.659261in}}{\pgfqpoint{3.677938in}{0.659261in}}%
\pgfpathcurveto{\pgfqpoint{3.666888in}{0.659261in}}{\pgfqpoint{3.656289in}{0.654871in}}{\pgfqpoint{3.648475in}{0.647057in}}%
\pgfpathcurveto{\pgfqpoint{3.640661in}{0.639244in}}{\pgfqpoint{3.636271in}{0.628645in}}{\pgfqpoint{3.636271in}{0.617595in}}%
\pgfpathcurveto{\pgfqpoint{3.636271in}{0.606544in}}{\pgfqpoint{3.640661in}{0.595945in}}{\pgfqpoint{3.648475in}{0.588132in}}%
\pgfpathcurveto{\pgfqpoint{3.656289in}{0.580318in}}{\pgfqpoint{3.666888in}{0.575928in}}{\pgfqpoint{3.677938in}{0.575928in}}%
\pgfpathlineto{\pgfqpoint{3.677938in}{0.575928in}}%
\pgfpathclose%
\pgfusepath{stroke}%
\end{pgfscope}%
\begin{pgfscope}%
\pgfpathrectangle{\pgfqpoint{0.393053in}{0.375000in}}{\pgfqpoint{6.356833in}{5.175000in}}%
\pgfusepath{clip}%
\pgfsetbuttcap%
\pgfsetroundjoin%
\pgfsetlinewidth{1.003750pt}%
\definecolor{currentstroke}{rgb}{0.827451,0.827451,0.827451}%
\pgfsetstrokecolor{currentstroke}%
\pgfsetdash{}{0pt}%
\pgfpathmoveto{\pgfqpoint{1.843211in}{1.515821in}}%
\pgfpathcurveto{\pgfqpoint{1.854261in}{1.515821in}}{\pgfqpoint{1.864860in}{1.520211in}}{\pgfqpoint{1.872674in}{1.528025in}}%
\pgfpathcurveto{\pgfqpoint{1.880487in}{1.535838in}}{\pgfqpoint{1.884878in}{1.546437in}}{\pgfqpoint{1.884878in}{1.557487in}}%
\pgfpathcurveto{\pgfqpoint{1.884878in}{1.568537in}}{\pgfqpoint{1.880487in}{1.579136in}}{\pgfqpoint{1.872674in}{1.586950in}}%
\pgfpathcurveto{\pgfqpoint{1.864860in}{1.594764in}}{\pgfqpoint{1.854261in}{1.599154in}}{\pgfqpoint{1.843211in}{1.599154in}}%
\pgfpathcurveto{\pgfqpoint{1.832161in}{1.599154in}}{\pgfqpoint{1.821562in}{1.594764in}}{\pgfqpoint{1.813748in}{1.586950in}}%
\pgfpathcurveto{\pgfqpoint{1.805934in}{1.579136in}}{\pgfqpoint{1.801544in}{1.568537in}}{\pgfqpoint{1.801544in}{1.557487in}}%
\pgfpathcurveto{\pgfqpoint{1.801544in}{1.546437in}}{\pgfqpoint{1.805934in}{1.535838in}}{\pgfqpoint{1.813748in}{1.528025in}}%
\pgfpathcurveto{\pgfqpoint{1.821562in}{1.520211in}}{\pgfqpoint{1.832161in}{1.515821in}}{\pgfqpoint{1.843211in}{1.515821in}}%
\pgfpathlineto{\pgfqpoint{1.843211in}{1.515821in}}%
\pgfpathclose%
\pgfusepath{stroke}%
\end{pgfscope}%
\begin{pgfscope}%
\pgfpathrectangle{\pgfqpoint{0.393053in}{0.375000in}}{\pgfqpoint{6.356833in}{5.175000in}}%
\pgfusepath{clip}%
\pgfsetbuttcap%
\pgfsetroundjoin%
\pgfsetlinewidth{1.003750pt}%
\definecolor{currentstroke}{rgb}{0.827451,0.827451,0.827451}%
\pgfsetstrokecolor{currentstroke}%
\pgfsetdash{}{0pt}%
\pgfpathmoveto{\pgfqpoint{0.424181in}{4.027299in}}%
\pgfpathcurveto{\pgfqpoint{0.435231in}{4.027299in}}{\pgfqpoint{0.445830in}{4.031689in}}{\pgfqpoint{0.453644in}{4.039503in}}%
\pgfpathcurveto{\pgfqpoint{0.461458in}{4.047316in}}{\pgfqpoint{0.465848in}{4.057915in}}{\pgfqpoint{0.465848in}{4.068966in}}%
\pgfpathcurveto{\pgfqpoint{0.465848in}{4.080016in}}{\pgfqpoint{0.461458in}{4.090615in}}{\pgfqpoint{0.453644in}{4.098428in}}%
\pgfpathcurveto{\pgfqpoint{0.445830in}{4.106242in}}{\pgfqpoint{0.435231in}{4.110632in}}{\pgfqpoint{0.424181in}{4.110632in}}%
\pgfpathcurveto{\pgfqpoint{0.413131in}{4.110632in}}{\pgfqpoint{0.402532in}{4.106242in}}{\pgfqpoint{0.394719in}{4.098428in}}%
\pgfpathcurveto{\pgfqpoint{0.386905in}{4.090615in}}{\pgfqpoint{0.382515in}{4.080016in}}{\pgfqpoint{0.382515in}{4.068966in}}%
\pgfpathcurveto{\pgfqpoint{0.382515in}{4.057915in}}{\pgfqpoint{0.386905in}{4.047316in}}{\pgfqpoint{0.394719in}{4.039503in}}%
\pgfpathcurveto{\pgfqpoint{0.402532in}{4.031689in}}{\pgfqpoint{0.413131in}{4.027299in}}{\pgfqpoint{0.424181in}{4.027299in}}%
\pgfpathlineto{\pgfqpoint{0.424181in}{4.027299in}}%
\pgfpathclose%
\pgfusepath{stroke}%
\end{pgfscope}%
\begin{pgfscope}%
\pgfpathrectangle{\pgfqpoint{0.393053in}{0.375000in}}{\pgfqpoint{6.356833in}{5.175000in}}%
\pgfusepath{clip}%
\pgfsetbuttcap%
\pgfsetroundjoin%
\pgfsetlinewidth{1.003750pt}%
\definecolor{currentstroke}{rgb}{0.827451,0.827451,0.827451}%
\pgfsetstrokecolor{currentstroke}%
\pgfsetdash{}{0pt}%
\pgfpathmoveto{\pgfqpoint{0.455840in}{3.833482in}}%
\pgfpathcurveto{\pgfqpoint{0.466890in}{3.833482in}}{\pgfqpoint{0.477489in}{3.837873in}}{\pgfqpoint{0.485303in}{3.845686in}}%
\pgfpathcurveto{\pgfqpoint{0.493117in}{3.853500in}}{\pgfqpoint{0.497507in}{3.864099in}}{\pgfqpoint{0.497507in}{3.875149in}}%
\pgfpathcurveto{\pgfqpoint{0.497507in}{3.886199in}}{\pgfqpoint{0.493117in}{3.896798in}}{\pgfqpoint{0.485303in}{3.904612in}}%
\pgfpathcurveto{\pgfqpoint{0.477489in}{3.912425in}}{\pgfqpoint{0.466890in}{3.916816in}}{\pgfqpoint{0.455840in}{3.916816in}}%
\pgfpathcurveto{\pgfqpoint{0.444790in}{3.916816in}}{\pgfqpoint{0.434191in}{3.912425in}}{\pgfqpoint{0.426377in}{3.904612in}}%
\pgfpathcurveto{\pgfqpoint{0.418564in}{3.896798in}}{\pgfqpoint{0.414173in}{3.886199in}}{\pgfqpoint{0.414173in}{3.875149in}}%
\pgfpathcurveto{\pgfqpoint{0.414173in}{3.864099in}}{\pgfqpoint{0.418564in}{3.853500in}}{\pgfqpoint{0.426377in}{3.845686in}}%
\pgfpathcurveto{\pgfqpoint{0.434191in}{3.837873in}}{\pgfqpoint{0.444790in}{3.833482in}}{\pgfqpoint{0.455840in}{3.833482in}}%
\pgfpathlineto{\pgfqpoint{0.455840in}{3.833482in}}%
\pgfpathclose%
\pgfusepath{stroke}%
\end{pgfscope}%
\begin{pgfscope}%
\pgfpathrectangle{\pgfqpoint{0.393053in}{0.375000in}}{\pgfqpoint{6.356833in}{5.175000in}}%
\pgfusepath{clip}%
\pgfsetbuttcap%
\pgfsetroundjoin%
\pgfsetlinewidth{1.003750pt}%
\definecolor{currentstroke}{rgb}{0.827451,0.827451,0.827451}%
\pgfsetstrokecolor{currentstroke}%
\pgfsetdash{}{0pt}%
\pgfpathmoveto{\pgfqpoint{0.396398in}{4.410130in}}%
\pgfpathcurveto{\pgfqpoint{0.407448in}{4.410130in}}{\pgfqpoint{0.418047in}{4.414521in}}{\pgfqpoint{0.425861in}{4.422334in}}%
\pgfpathcurveto{\pgfqpoint{0.433674in}{4.430148in}}{\pgfqpoint{0.438064in}{4.440747in}}{\pgfqpoint{0.438064in}{4.451797in}}%
\pgfpathcurveto{\pgfqpoint{0.438064in}{4.462847in}}{\pgfqpoint{0.433674in}{4.473446in}}{\pgfqpoint{0.425861in}{4.481260in}}%
\pgfpathcurveto{\pgfqpoint{0.418047in}{4.489073in}}{\pgfqpoint{0.407448in}{4.493464in}}{\pgfqpoint{0.396398in}{4.493464in}}%
\pgfpathcurveto{\pgfqpoint{0.385348in}{4.493464in}}{\pgfqpoint{0.374749in}{4.489073in}}{\pgfqpoint{0.366935in}{4.481260in}}%
\pgfpathcurveto{\pgfqpoint{0.359121in}{4.473446in}}{\pgfqpoint{0.354731in}{4.462847in}}{\pgfqpoint{0.354731in}{4.451797in}}%
\pgfpathcurveto{\pgfqpoint{0.354731in}{4.440747in}}{\pgfqpoint{0.359121in}{4.430148in}}{\pgfqpoint{0.366935in}{4.422334in}}%
\pgfpathcurveto{\pgfqpoint{0.374749in}{4.414521in}}{\pgfqpoint{0.385348in}{4.410130in}}{\pgfqpoint{0.396398in}{4.410130in}}%
\pgfpathlineto{\pgfqpoint{0.396398in}{4.410130in}}%
\pgfpathclose%
\pgfusepath{stroke}%
\end{pgfscope}%
\begin{pgfscope}%
\pgfpathrectangle{\pgfqpoint{0.393053in}{0.375000in}}{\pgfqpoint{6.356833in}{5.175000in}}%
\pgfusepath{clip}%
\pgfsetbuttcap%
\pgfsetroundjoin%
\pgfsetlinewidth{1.003750pt}%
\definecolor{currentstroke}{rgb}{0.827451,0.827451,0.827451}%
\pgfsetstrokecolor{currentstroke}%
\pgfsetdash{}{0pt}%
\pgfpathmoveto{\pgfqpoint{0.635911in}{3.119895in}}%
\pgfpathcurveto{\pgfqpoint{0.646961in}{3.119895in}}{\pgfqpoint{0.657560in}{3.124285in}}{\pgfqpoint{0.665374in}{3.132098in}}%
\pgfpathcurveto{\pgfqpoint{0.673187in}{3.139912in}}{\pgfqpoint{0.677577in}{3.150511in}}{\pgfqpoint{0.677577in}{3.161561in}}%
\pgfpathcurveto{\pgfqpoint{0.677577in}{3.172611in}}{\pgfqpoint{0.673187in}{3.183210in}}{\pgfqpoint{0.665374in}{3.191024in}}%
\pgfpathcurveto{\pgfqpoint{0.657560in}{3.198838in}}{\pgfqpoint{0.646961in}{3.203228in}}{\pgfqpoint{0.635911in}{3.203228in}}%
\pgfpathcurveto{\pgfqpoint{0.624861in}{3.203228in}}{\pgfqpoint{0.614262in}{3.198838in}}{\pgfqpoint{0.606448in}{3.191024in}}%
\pgfpathcurveto{\pgfqpoint{0.598634in}{3.183210in}}{\pgfqpoint{0.594244in}{3.172611in}}{\pgfqpoint{0.594244in}{3.161561in}}%
\pgfpathcurveto{\pgfqpoint{0.594244in}{3.150511in}}{\pgfqpoint{0.598634in}{3.139912in}}{\pgfqpoint{0.606448in}{3.132098in}}%
\pgfpathcurveto{\pgfqpoint{0.614262in}{3.124285in}}{\pgfqpoint{0.624861in}{3.119895in}}{\pgfqpoint{0.635911in}{3.119895in}}%
\pgfpathlineto{\pgfqpoint{0.635911in}{3.119895in}}%
\pgfpathclose%
\pgfusepath{stroke}%
\end{pgfscope}%
\begin{pgfscope}%
\pgfpathrectangle{\pgfqpoint{0.393053in}{0.375000in}}{\pgfqpoint{6.356833in}{5.175000in}}%
\pgfusepath{clip}%
\pgfsetbuttcap%
\pgfsetroundjoin%
\pgfsetlinewidth{1.003750pt}%
\definecolor{currentstroke}{rgb}{0.827451,0.827451,0.827451}%
\pgfsetstrokecolor{currentstroke}%
\pgfsetdash{}{0pt}%
\pgfpathmoveto{\pgfqpoint{1.702004in}{1.599430in}}%
\pgfpathcurveto{\pgfqpoint{1.713054in}{1.599430in}}{\pgfqpoint{1.723653in}{1.603820in}}{\pgfqpoint{1.731467in}{1.611633in}}%
\pgfpathcurveto{\pgfqpoint{1.739281in}{1.619447in}}{\pgfqpoint{1.743671in}{1.630046in}}{\pgfqpoint{1.743671in}{1.641096in}}%
\pgfpathcurveto{\pgfqpoint{1.743671in}{1.652146in}}{\pgfqpoint{1.739281in}{1.662745in}}{\pgfqpoint{1.731467in}{1.670559in}}%
\pgfpathcurveto{\pgfqpoint{1.723653in}{1.678373in}}{\pgfqpoint{1.713054in}{1.682763in}}{\pgfqpoint{1.702004in}{1.682763in}}%
\pgfpathcurveto{\pgfqpoint{1.690954in}{1.682763in}}{\pgfqpoint{1.680355in}{1.678373in}}{\pgfqpoint{1.672541in}{1.670559in}}%
\pgfpathcurveto{\pgfqpoint{1.664728in}{1.662745in}}{\pgfqpoint{1.660338in}{1.652146in}}{\pgfqpoint{1.660338in}{1.641096in}}%
\pgfpathcurveto{\pgfqpoint{1.660338in}{1.630046in}}{\pgfqpoint{1.664728in}{1.619447in}}{\pgfqpoint{1.672541in}{1.611633in}}%
\pgfpathcurveto{\pgfqpoint{1.680355in}{1.603820in}}{\pgfqpoint{1.690954in}{1.599430in}}{\pgfqpoint{1.702004in}{1.599430in}}%
\pgfpathlineto{\pgfqpoint{1.702004in}{1.599430in}}%
\pgfpathclose%
\pgfusepath{stroke}%
\end{pgfscope}%
\begin{pgfscope}%
\pgfpathrectangle{\pgfqpoint{0.393053in}{0.375000in}}{\pgfqpoint{6.356833in}{5.175000in}}%
\pgfusepath{clip}%
\pgfsetbuttcap%
\pgfsetroundjoin%
\pgfsetlinewidth{1.003750pt}%
\definecolor{currentstroke}{rgb}{0.827451,0.827451,0.827451}%
\pgfsetstrokecolor{currentstroke}%
\pgfsetdash{}{0pt}%
\pgfpathmoveto{\pgfqpoint{1.176673in}{2.141426in}}%
\pgfpathcurveto{\pgfqpoint{1.187723in}{2.141426in}}{\pgfqpoint{1.198322in}{2.145816in}}{\pgfqpoint{1.206136in}{2.153630in}}%
\pgfpathcurveto{\pgfqpoint{1.213949in}{2.161444in}}{\pgfqpoint{1.218340in}{2.172043in}}{\pgfqpoint{1.218340in}{2.183093in}}%
\pgfpathcurveto{\pgfqpoint{1.218340in}{2.194143in}}{\pgfqpoint{1.213949in}{2.204742in}}{\pgfqpoint{1.206136in}{2.212556in}}%
\pgfpathcurveto{\pgfqpoint{1.198322in}{2.220369in}}{\pgfqpoint{1.187723in}{2.224759in}}{\pgfqpoint{1.176673in}{2.224759in}}%
\pgfpathcurveto{\pgfqpoint{1.165623in}{2.224759in}}{\pgfqpoint{1.155024in}{2.220369in}}{\pgfqpoint{1.147210in}{2.212556in}}%
\pgfpathcurveto{\pgfqpoint{1.139396in}{2.204742in}}{\pgfqpoint{1.135006in}{2.194143in}}{\pgfqpoint{1.135006in}{2.183093in}}%
\pgfpathcurveto{\pgfqpoint{1.135006in}{2.172043in}}{\pgfqpoint{1.139396in}{2.161444in}}{\pgfqpoint{1.147210in}{2.153630in}}%
\pgfpathcurveto{\pgfqpoint{1.155024in}{2.145816in}}{\pgfqpoint{1.165623in}{2.141426in}}{\pgfqpoint{1.176673in}{2.141426in}}%
\pgfpathlineto{\pgfqpoint{1.176673in}{2.141426in}}%
\pgfpathclose%
\pgfusepath{stroke}%
\end{pgfscope}%
\begin{pgfscope}%
\pgfpathrectangle{\pgfqpoint{0.393053in}{0.375000in}}{\pgfqpoint{6.356833in}{5.175000in}}%
\pgfusepath{clip}%
\pgfsetbuttcap%
\pgfsetroundjoin%
\pgfsetlinewidth{1.003750pt}%
\definecolor{currentstroke}{rgb}{0.827451,0.827451,0.827451}%
\pgfsetstrokecolor{currentstroke}%
\pgfsetdash{}{0pt}%
\pgfpathmoveto{\pgfqpoint{2.103640in}{1.312371in}}%
\pgfpathcurveto{\pgfqpoint{2.114690in}{1.312371in}}{\pgfqpoint{2.125289in}{1.316762in}}{\pgfqpoint{2.133103in}{1.324575in}}%
\pgfpathcurveto{\pgfqpoint{2.140916in}{1.332389in}}{\pgfqpoint{2.145307in}{1.342988in}}{\pgfqpoint{2.145307in}{1.354038in}}%
\pgfpathcurveto{\pgfqpoint{2.145307in}{1.365088in}}{\pgfqpoint{2.140916in}{1.375687in}}{\pgfqpoint{2.133103in}{1.383501in}}%
\pgfpathcurveto{\pgfqpoint{2.125289in}{1.391314in}}{\pgfqpoint{2.114690in}{1.395705in}}{\pgfqpoint{2.103640in}{1.395705in}}%
\pgfpathcurveto{\pgfqpoint{2.092590in}{1.395705in}}{\pgfqpoint{2.081991in}{1.391314in}}{\pgfqpoint{2.074177in}{1.383501in}}%
\pgfpathcurveto{\pgfqpoint{2.066364in}{1.375687in}}{\pgfqpoint{2.061973in}{1.365088in}}{\pgfqpoint{2.061973in}{1.354038in}}%
\pgfpathcurveto{\pgfqpoint{2.061973in}{1.342988in}}{\pgfqpoint{2.066364in}{1.332389in}}{\pgfqpoint{2.074177in}{1.324575in}}%
\pgfpathcurveto{\pgfqpoint{2.081991in}{1.316762in}}{\pgfqpoint{2.092590in}{1.312371in}}{\pgfqpoint{2.103640in}{1.312371in}}%
\pgfpathlineto{\pgfqpoint{2.103640in}{1.312371in}}%
\pgfpathclose%
\pgfusepath{stroke}%
\end{pgfscope}%
\begin{pgfscope}%
\pgfpathrectangle{\pgfqpoint{0.393053in}{0.375000in}}{\pgfqpoint{6.356833in}{5.175000in}}%
\pgfusepath{clip}%
\pgfsetbuttcap%
\pgfsetroundjoin%
\pgfsetlinewidth{1.003750pt}%
\definecolor{currentstroke}{rgb}{0.827451,0.827451,0.827451}%
\pgfsetstrokecolor{currentstroke}%
\pgfsetdash{}{0pt}%
\pgfpathmoveto{\pgfqpoint{2.785690in}{0.904937in}}%
\pgfpathcurveto{\pgfqpoint{2.796740in}{0.904937in}}{\pgfqpoint{2.807339in}{0.909327in}}{\pgfqpoint{2.815153in}{0.917140in}}%
\pgfpathcurveto{\pgfqpoint{2.822966in}{0.924954in}}{\pgfqpoint{2.827357in}{0.935553in}}{\pgfqpoint{2.827357in}{0.946603in}}%
\pgfpathcurveto{\pgfqpoint{2.827357in}{0.957653in}}{\pgfqpoint{2.822966in}{0.968252in}}{\pgfqpoint{2.815153in}{0.976066in}}%
\pgfpathcurveto{\pgfqpoint{2.807339in}{0.983880in}}{\pgfqpoint{2.796740in}{0.988270in}}{\pgfqpoint{2.785690in}{0.988270in}}%
\pgfpathcurveto{\pgfqpoint{2.774640in}{0.988270in}}{\pgfqpoint{2.764041in}{0.983880in}}{\pgfqpoint{2.756227in}{0.976066in}}%
\pgfpathcurveto{\pgfqpoint{2.748414in}{0.968252in}}{\pgfqpoint{2.744023in}{0.957653in}}{\pgfqpoint{2.744023in}{0.946603in}}%
\pgfpathcurveto{\pgfqpoint{2.744023in}{0.935553in}}{\pgfqpoint{2.748414in}{0.924954in}}{\pgfqpoint{2.756227in}{0.917140in}}%
\pgfpathcurveto{\pgfqpoint{2.764041in}{0.909327in}}{\pgfqpoint{2.774640in}{0.904937in}}{\pgfqpoint{2.785690in}{0.904937in}}%
\pgfpathlineto{\pgfqpoint{2.785690in}{0.904937in}}%
\pgfpathclose%
\pgfusepath{stroke}%
\end{pgfscope}%
\begin{pgfscope}%
\pgfpathrectangle{\pgfqpoint{0.393053in}{0.375000in}}{\pgfqpoint{6.356833in}{5.175000in}}%
\pgfusepath{clip}%
\pgfsetbuttcap%
\pgfsetroundjoin%
\pgfsetlinewidth{1.003750pt}%
\definecolor{currentstroke}{rgb}{0.827451,0.827451,0.827451}%
\pgfsetstrokecolor{currentstroke}%
\pgfsetdash{}{0pt}%
\pgfpathmoveto{\pgfqpoint{4.445943in}{0.425176in}}%
\pgfpathcurveto{\pgfqpoint{4.456993in}{0.425176in}}{\pgfqpoint{4.467592in}{0.429567in}}{\pgfqpoint{4.475406in}{0.437380in}}%
\pgfpathcurveto{\pgfqpoint{4.483219in}{0.445194in}}{\pgfqpoint{4.487610in}{0.455793in}}{\pgfqpoint{4.487610in}{0.466843in}}%
\pgfpathcurveto{\pgfqpoint{4.487610in}{0.477893in}}{\pgfqpoint{4.483219in}{0.488492in}}{\pgfqpoint{4.475406in}{0.496306in}}%
\pgfpathcurveto{\pgfqpoint{4.467592in}{0.504119in}}{\pgfqpoint{4.456993in}{0.508510in}}{\pgfqpoint{4.445943in}{0.508510in}}%
\pgfpathcurveto{\pgfqpoint{4.434893in}{0.508510in}}{\pgfqpoint{4.424294in}{0.504119in}}{\pgfqpoint{4.416480in}{0.496306in}}%
\pgfpathcurveto{\pgfqpoint{4.408667in}{0.488492in}}{\pgfqpoint{4.404276in}{0.477893in}}{\pgfqpoint{4.404276in}{0.466843in}}%
\pgfpathcurveto{\pgfqpoint{4.404276in}{0.455793in}}{\pgfqpoint{4.408667in}{0.445194in}}{\pgfqpoint{4.416480in}{0.437380in}}%
\pgfpathcurveto{\pgfqpoint{4.424294in}{0.429567in}}{\pgfqpoint{4.434893in}{0.425176in}}{\pgfqpoint{4.445943in}{0.425176in}}%
\pgfpathlineto{\pgfqpoint{4.445943in}{0.425176in}}%
\pgfpathclose%
\pgfusepath{stroke}%
\end{pgfscope}%
\begin{pgfscope}%
\pgfpathrectangle{\pgfqpoint{0.393053in}{0.375000in}}{\pgfqpoint{6.356833in}{5.175000in}}%
\pgfusepath{clip}%
\pgfsetbuttcap%
\pgfsetroundjoin%
\pgfsetlinewidth{1.003750pt}%
\definecolor{currentstroke}{rgb}{0.827451,0.827451,0.827451}%
\pgfsetstrokecolor{currentstroke}%
\pgfsetdash{}{0pt}%
\pgfpathmoveto{\pgfqpoint{3.954008in}{0.515189in}}%
\pgfpathcurveto{\pgfqpoint{3.965058in}{0.515189in}}{\pgfqpoint{3.975657in}{0.519579in}}{\pgfqpoint{3.983471in}{0.527393in}}%
\pgfpathcurveto{\pgfqpoint{3.991284in}{0.535206in}}{\pgfqpoint{3.995675in}{0.545805in}}{\pgfqpoint{3.995675in}{0.556855in}}%
\pgfpathcurveto{\pgfqpoint{3.995675in}{0.567905in}}{\pgfqpoint{3.991284in}{0.578504in}}{\pgfqpoint{3.983471in}{0.586318in}}%
\pgfpathcurveto{\pgfqpoint{3.975657in}{0.594132in}}{\pgfqpoint{3.965058in}{0.598522in}}{\pgfqpoint{3.954008in}{0.598522in}}%
\pgfpathcurveto{\pgfqpoint{3.942958in}{0.598522in}}{\pgfqpoint{3.932359in}{0.594132in}}{\pgfqpoint{3.924545in}{0.586318in}}%
\pgfpathcurveto{\pgfqpoint{3.916732in}{0.578504in}}{\pgfqpoint{3.912341in}{0.567905in}}{\pgfqpoint{3.912341in}{0.556855in}}%
\pgfpathcurveto{\pgfqpoint{3.912341in}{0.545805in}}{\pgfqpoint{3.916732in}{0.535206in}}{\pgfqpoint{3.924545in}{0.527393in}}%
\pgfpathcurveto{\pgfqpoint{3.932359in}{0.519579in}}{\pgfqpoint{3.942958in}{0.515189in}}{\pgfqpoint{3.954008in}{0.515189in}}%
\pgfpathlineto{\pgfqpoint{3.954008in}{0.515189in}}%
\pgfpathclose%
\pgfusepath{stroke}%
\end{pgfscope}%
\begin{pgfscope}%
\pgfpathrectangle{\pgfqpoint{0.393053in}{0.375000in}}{\pgfqpoint{6.356833in}{5.175000in}}%
\pgfusepath{clip}%
\pgfsetbuttcap%
\pgfsetroundjoin%
\pgfsetlinewidth{1.003750pt}%
\definecolor{currentstroke}{rgb}{0.827451,0.827451,0.827451}%
\pgfsetstrokecolor{currentstroke}%
\pgfsetdash{}{0pt}%
\pgfpathmoveto{\pgfqpoint{3.967155in}{0.512418in}}%
\pgfpathcurveto{\pgfqpoint{3.978205in}{0.512418in}}{\pgfqpoint{3.988804in}{0.516808in}}{\pgfqpoint{3.996618in}{0.524622in}}%
\pgfpathcurveto{\pgfqpoint{4.004431in}{0.532435in}}{\pgfqpoint{4.008822in}{0.543034in}}{\pgfqpoint{4.008822in}{0.554084in}}%
\pgfpathcurveto{\pgfqpoint{4.008822in}{0.565134in}}{\pgfqpoint{4.004431in}{0.575733in}}{\pgfqpoint{3.996618in}{0.583547in}}%
\pgfpathcurveto{\pgfqpoint{3.988804in}{0.591361in}}{\pgfqpoint{3.978205in}{0.595751in}}{\pgfqpoint{3.967155in}{0.595751in}}%
\pgfpathcurveto{\pgfqpoint{3.956105in}{0.595751in}}{\pgfqpoint{3.945506in}{0.591361in}}{\pgfqpoint{3.937692in}{0.583547in}}%
\pgfpathcurveto{\pgfqpoint{3.929879in}{0.575733in}}{\pgfqpoint{3.925488in}{0.565134in}}{\pgfqpoint{3.925488in}{0.554084in}}%
\pgfpathcurveto{\pgfqpoint{3.925488in}{0.543034in}}{\pgfqpoint{3.929879in}{0.532435in}}{\pgfqpoint{3.937692in}{0.524622in}}%
\pgfpathcurveto{\pgfqpoint{3.945506in}{0.516808in}}{\pgfqpoint{3.956105in}{0.512418in}}{\pgfqpoint{3.967155in}{0.512418in}}%
\pgfpathlineto{\pgfqpoint{3.967155in}{0.512418in}}%
\pgfpathclose%
\pgfusepath{stroke}%
\end{pgfscope}%
\begin{pgfscope}%
\pgfpathrectangle{\pgfqpoint{0.393053in}{0.375000in}}{\pgfqpoint{6.356833in}{5.175000in}}%
\pgfusepath{clip}%
\pgfsetbuttcap%
\pgfsetroundjoin%
\pgfsetlinewidth{1.003750pt}%
\definecolor{currentstroke}{rgb}{0.827451,0.827451,0.827451}%
\pgfsetstrokecolor{currentstroke}%
\pgfsetdash{}{0pt}%
\pgfpathmoveto{\pgfqpoint{0.651937in}{3.079342in}}%
\pgfpathcurveto{\pgfqpoint{0.662987in}{3.079342in}}{\pgfqpoint{0.673586in}{3.083732in}}{\pgfqpoint{0.681400in}{3.091546in}}%
\pgfpathcurveto{\pgfqpoint{0.689213in}{3.099359in}}{\pgfqpoint{0.693604in}{3.109958in}}{\pgfqpoint{0.693604in}{3.121009in}}%
\pgfpathcurveto{\pgfqpoint{0.693604in}{3.132059in}}{\pgfqpoint{0.689213in}{3.142658in}}{\pgfqpoint{0.681400in}{3.150471in}}%
\pgfpathcurveto{\pgfqpoint{0.673586in}{3.158285in}}{\pgfqpoint{0.662987in}{3.162675in}}{\pgfqpoint{0.651937in}{3.162675in}}%
\pgfpathcurveto{\pgfqpoint{0.640887in}{3.162675in}}{\pgfqpoint{0.630288in}{3.158285in}}{\pgfqpoint{0.622474in}{3.150471in}}%
\pgfpathcurveto{\pgfqpoint{0.614660in}{3.142658in}}{\pgfqpoint{0.610270in}{3.132059in}}{\pgfqpoint{0.610270in}{3.121009in}}%
\pgfpathcurveto{\pgfqpoint{0.610270in}{3.109958in}}{\pgfqpoint{0.614660in}{3.099359in}}{\pgfqpoint{0.622474in}{3.091546in}}%
\pgfpathcurveto{\pgfqpoint{0.630288in}{3.083732in}}{\pgfqpoint{0.640887in}{3.079342in}}{\pgfqpoint{0.651937in}{3.079342in}}%
\pgfpathlineto{\pgfqpoint{0.651937in}{3.079342in}}%
\pgfpathclose%
\pgfusepath{stroke}%
\end{pgfscope}%
\begin{pgfscope}%
\pgfpathrectangle{\pgfqpoint{0.393053in}{0.375000in}}{\pgfqpoint{6.356833in}{5.175000in}}%
\pgfusepath{clip}%
\pgfsetbuttcap%
\pgfsetroundjoin%
\pgfsetlinewidth{1.003750pt}%
\definecolor{currentstroke}{rgb}{0.827451,0.827451,0.827451}%
\pgfsetstrokecolor{currentstroke}%
\pgfsetdash{}{0pt}%
\pgfpathmoveto{\pgfqpoint{1.493790in}{1.813114in}}%
\pgfpathcurveto{\pgfqpoint{1.504840in}{1.813114in}}{\pgfqpoint{1.515439in}{1.817505in}}{\pgfqpoint{1.523252in}{1.825318in}}%
\pgfpathcurveto{\pgfqpoint{1.531066in}{1.833132in}}{\pgfqpoint{1.535456in}{1.843731in}}{\pgfqpoint{1.535456in}{1.854781in}}%
\pgfpathcurveto{\pgfqpoint{1.535456in}{1.865831in}}{\pgfqpoint{1.531066in}{1.876430in}}{\pgfqpoint{1.523252in}{1.884244in}}%
\pgfpathcurveto{\pgfqpoint{1.515439in}{1.892058in}}{\pgfqpoint{1.504840in}{1.896448in}}{\pgfqpoint{1.493790in}{1.896448in}}%
\pgfpathcurveto{\pgfqpoint{1.482740in}{1.896448in}}{\pgfqpoint{1.472140in}{1.892058in}}{\pgfqpoint{1.464327in}{1.884244in}}%
\pgfpathcurveto{\pgfqpoint{1.456513in}{1.876430in}}{\pgfqpoint{1.452123in}{1.865831in}}{\pgfqpoint{1.452123in}{1.854781in}}%
\pgfpathcurveto{\pgfqpoint{1.452123in}{1.843731in}}{\pgfqpoint{1.456513in}{1.833132in}}{\pgfqpoint{1.464327in}{1.825318in}}%
\pgfpathcurveto{\pgfqpoint{1.472140in}{1.817505in}}{\pgfqpoint{1.482740in}{1.813114in}}{\pgfqpoint{1.493790in}{1.813114in}}%
\pgfpathlineto{\pgfqpoint{1.493790in}{1.813114in}}%
\pgfpathclose%
\pgfusepath{stroke}%
\end{pgfscope}%
\begin{pgfscope}%
\pgfpathrectangle{\pgfqpoint{0.393053in}{0.375000in}}{\pgfqpoint{6.356833in}{5.175000in}}%
\pgfusepath{clip}%
\pgfsetbuttcap%
\pgfsetroundjoin%
\pgfsetlinewidth{1.003750pt}%
\definecolor{currentstroke}{rgb}{0.827451,0.827451,0.827451}%
\pgfsetstrokecolor{currentstroke}%
\pgfsetdash{}{0pt}%
\pgfpathmoveto{\pgfqpoint{3.053820in}{0.786671in}}%
\pgfpathcurveto{\pgfqpoint{3.064870in}{0.786671in}}{\pgfqpoint{3.075469in}{0.791061in}}{\pgfqpoint{3.083282in}{0.798875in}}%
\pgfpathcurveto{\pgfqpoint{3.091096in}{0.806688in}}{\pgfqpoint{3.095486in}{0.817287in}}{\pgfqpoint{3.095486in}{0.828338in}}%
\pgfpathcurveto{\pgfqpoint{3.095486in}{0.839388in}}{\pgfqpoint{3.091096in}{0.849987in}}{\pgfqpoint{3.083282in}{0.857800in}}%
\pgfpathcurveto{\pgfqpoint{3.075469in}{0.865614in}}{\pgfqpoint{3.064870in}{0.870004in}}{\pgfqpoint{3.053820in}{0.870004in}}%
\pgfpathcurveto{\pgfqpoint{3.042770in}{0.870004in}}{\pgfqpoint{3.032171in}{0.865614in}}{\pgfqpoint{3.024357in}{0.857800in}}%
\pgfpathcurveto{\pgfqpoint{3.016543in}{0.849987in}}{\pgfqpoint{3.012153in}{0.839388in}}{\pgfqpoint{3.012153in}{0.828338in}}%
\pgfpathcurveto{\pgfqpoint{3.012153in}{0.817287in}}{\pgfqpoint{3.016543in}{0.806688in}}{\pgfqpoint{3.024357in}{0.798875in}}%
\pgfpathcurveto{\pgfqpoint{3.032171in}{0.791061in}}{\pgfqpoint{3.042770in}{0.786671in}}{\pgfqpoint{3.053820in}{0.786671in}}%
\pgfpathlineto{\pgfqpoint{3.053820in}{0.786671in}}%
\pgfpathclose%
\pgfusepath{stroke}%
\end{pgfscope}%
\begin{pgfscope}%
\pgfpathrectangle{\pgfqpoint{0.393053in}{0.375000in}}{\pgfqpoint{6.356833in}{5.175000in}}%
\pgfusepath{clip}%
\pgfsetbuttcap%
\pgfsetroundjoin%
\pgfsetlinewidth{1.003750pt}%
\definecolor{currentstroke}{rgb}{0.827451,0.827451,0.827451}%
\pgfsetstrokecolor{currentstroke}%
\pgfsetdash{}{0pt}%
\pgfpathmoveto{\pgfqpoint{4.275388in}{0.441212in}}%
\pgfpathcurveto{\pgfqpoint{4.286438in}{0.441212in}}{\pgfqpoint{4.297038in}{0.445602in}}{\pgfqpoint{4.304851in}{0.453416in}}%
\pgfpathcurveto{\pgfqpoint{4.312665in}{0.461229in}}{\pgfqpoint{4.317055in}{0.471828in}}{\pgfqpoint{4.317055in}{0.482878in}}%
\pgfpathcurveto{\pgfqpoint{4.317055in}{0.493928in}}{\pgfqpoint{4.312665in}{0.504527in}}{\pgfqpoint{4.304851in}{0.512341in}}%
\pgfpathcurveto{\pgfqpoint{4.297038in}{0.520155in}}{\pgfqpoint{4.286438in}{0.524545in}}{\pgfqpoint{4.275388in}{0.524545in}}%
\pgfpathcurveto{\pgfqpoint{4.264338in}{0.524545in}}{\pgfqpoint{4.253739in}{0.520155in}}{\pgfqpoint{4.245926in}{0.512341in}}%
\pgfpathcurveto{\pgfqpoint{4.238112in}{0.504527in}}{\pgfqpoint{4.233722in}{0.493928in}}{\pgfqpoint{4.233722in}{0.482878in}}%
\pgfpathcurveto{\pgfqpoint{4.233722in}{0.471828in}}{\pgfqpoint{4.238112in}{0.461229in}}{\pgfqpoint{4.245926in}{0.453416in}}%
\pgfpathcurveto{\pgfqpoint{4.253739in}{0.445602in}}{\pgfqpoint{4.264338in}{0.441212in}}{\pgfqpoint{4.275388in}{0.441212in}}%
\pgfpathlineto{\pgfqpoint{4.275388in}{0.441212in}}%
\pgfpathclose%
\pgfusepath{stroke}%
\end{pgfscope}%
\begin{pgfscope}%
\pgfpathrectangle{\pgfqpoint{0.393053in}{0.375000in}}{\pgfqpoint{6.356833in}{5.175000in}}%
\pgfusepath{clip}%
\pgfsetbuttcap%
\pgfsetroundjoin%
\pgfsetlinewidth{1.003750pt}%
\definecolor{currentstroke}{rgb}{0.827451,0.827451,0.827451}%
\pgfsetstrokecolor{currentstroke}%
\pgfsetdash{}{0pt}%
\pgfpathmoveto{\pgfqpoint{0.519167in}{3.558335in}}%
\pgfpathcurveto{\pgfqpoint{0.530217in}{3.558335in}}{\pgfqpoint{0.540816in}{3.562726in}}{\pgfqpoint{0.548630in}{3.570539in}}%
\pgfpathcurveto{\pgfqpoint{0.556443in}{3.578353in}}{\pgfqpoint{0.560834in}{3.588952in}}{\pgfqpoint{0.560834in}{3.600002in}}%
\pgfpathcurveto{\pgfqpoint{0.560834in}{3.611052in}}{\pgfqpoint{0.556443in}{3.621651in}}{\pgfqpoint{0.548630in}{3.629465in}}%
\pgfpathcurveto{\pgfqpoint{0.540816in}{3.637278in}}{\pgfqpoint{0.530217in}{3.641669in}}{\pgfqpoint{0.519167in}{3.641669in}}%
\pgfpathcurveto{\pgfqpoint{0.508117in}{3.641669in}}{\pgfqpoint{0.497518in}{3.637278in}}{\pgfqpoint{0.489704in}{3.629465in}}%
\pgfpathcurveto{\pgfqpoint{0.481891in}{3.621651in}}{\pgfqpoint{0.477500in}{3.611052in}}{\pgfqpoint{0.477500in}{3.600002in}}%
\pgfpathcurveto{\pgfqpoint{0.477500in}{3.588952in}}{\pgfqpoint{0.481891in}{3.578353in}}{\pgfqpoint{0.489704in}{3.570539in}}%
\pgfpathcurveto{\pgfqpoint{0.497518in}{3.562726in}}{\pgfqpoint{0.508117in}{3.558335in}}{\pgfqpoint{0.519167in}{3.558335in}}%
\pgfpathlineto{\pgfqpoint{0.519167in}{3.558335in}}%
\pgfpathclose%
\pgfusepath{stroke}%
\end{pgfscope}%
\begin{pgfscope}%
\pgfpathrectangle{\pgfqpoint{0.393053in}{0.375000in}}{\pgfqpoint{6.356833in}{5.175000in}}%
\pgfusepath{clip}%
\pgfsetbuttcap%
\pgfsetroundjoin%
\pgfsetlinewidth{1.003750pt}%
\definecolor{currentstroke}{rgb}{0.827451,0.827451,0.827451}%
\pgfsetstrokecolor{currentstroke}%
\pgfsetdash{}{0pt}%
\pgfpathmoveto{\pgfqpoint{1.536415in}{1.768054in}}%
\pgfpathcurveto{\pgfqpoint{1.547466in}{1.768054in}}{\pgfqpoint{1.558065in}{1.772444in}}{\pgfqpoint{1.565878in}{1.780258in}}%
\pgfpathcurveto{\pgfqpoint{1.573692in}{1.788071in}}{\pgfqpoint{1.578082in}{1.798670in}}{\pgfqpoint{1.578082in}{1.809720in}}%
\pgfpathcurveto{\pgfqpoint{1.578082in}{1.820770in}}{\pgfqpoint{1.573692in}{1.831370in}}{\pgfqpoint{1.565878in}{1.839183in}}%
\pgfpathcurveto{\pgfqpoint{1.558065in}{1.846997in}}{\pgfqpoint{1.547466in}{1.851387in}}{\pgfqpoint{1.536415in}{1.851387in}}%
\pgfpathcurveto{\pgfqpoint{1.525365in}{1.851387in}}{\pgfqpoint{1.514766in}{1.846997in}}{\pgfqpoint{1.506953in}{1.839183in}}%
\pgfpathcurveto{\pgfqpoint{1.499139in}{1.831370in}}{\pgfqpoint{1.494749in}{1.820770in}}{\pgfqpoint{1.494749in}{1.809720in}}%
\pgfpathcurveto{\pgfqpoint{1.494749in}{1.798670in}}{\pgfqpoint{1.499139in}{1.788071in}}{\pgfqpoint{1.506953in}{1.780258in}}%
\pgfpathcurveto{\pgfqpoint{1.514766in}{1.772444in}}{\pgfqpoint{1.525365in}{1.768054in}}{\pgfqpoint{1.536415in}{1.768054in}}%
\pgfpathlineto{\pgfqpoint{1.536415in}{1.768054in}}%
\pgfpathclose%
\pgfusepath{stroke}%
\end{pgfscope}%
\begin{pgfscope}%
\pgfpathrectangle{\pgfqpoint{0.393053in}{0.375000in}}{\pgfqpoint{6.356833in}{5.175000in}}%
\pgfusepath{clip}%
\pgfsetbuttcap%
\pgfsetroundjoin%
\pgfsetlinewidth{1.003750pt}%
\definecolor{currentstroke}{rgb}{0.827451,0.827451,0.827451}%
\pgfsetstrokecolor{currentstroke}%
\pgfsetdash{}{0pt}%
\pgfpathmoveto{\pgfqpoint{2.287965in}{1.196822in}}%
\pgfpathcurveto{\pgfqpoint{2.299015in}{1.196822in}}{\pgfqpoint{2.309614in}{1.201212in}}{\pgfqpoint{2.317427in}{1.209026in}}%
\pgfpathcurveto{\pgfqpoint{2.325241in}{1.216839in}}{\pgfqpoint{2.329631in}{1.227438in}}{\pgfqpoint{2.329631in}{1.238488in}}%
\pgfpathcurveto{\pgfqpoint{2.329631in}{1.249539in}}{\pgfqpoint{2.325241in}{1.260138in}}{\pgfqpoint{2.317427in}{1.267951in}}%
\pgfpathcurveto{\pgfqpoint{2.309614in}{1.275765in}}{\pgfqpoint{2.299015in}{1.280155in}}{\pgfqpoint{2.287965in}{1.280155in}}%
\pgfpathcurveto{\pgfqpoint{2.276915in}{1.280155in}}{\pgfqpoint{2.266316in}{1.275765in}}{\pgfqpoint{2.258502in}{1.267951in}}%
\pgfpathcurveto{\pgfqpoint{2.250688in}{1.260138in}}{\pgfqpoint{2.246298in}{1.249539in}}{\pgfqpoint{2.246298in}{1.238488in}}%
\pgfpathcurveto{\pgfqpoint{2.246298in}{1.227438in}}{\pgfqpoint{2.250688in}{1.216839in}}{\pgfqpoint{2.258502in}{1.209026in}}%
\pgfpathcurveto{\pgfqpoint{2.266316in}{1.201212in}}{\pgfqpoint{2.276915in}{1.196822in}}{\pgfqpoint{2.287965in}{1.196822in}}%
\pgfpathlineto{\pgfqpoint{2.287965in}{1.196822in}}%
\pgfpathclose%
\pgfusepath{stroke}%
\end{pgfscope}%
\begin{pgfscope}%
\pgfpathrectangle{\pgfqpoint{0.393053in}{0.375000in}}{\pgfqpoint{6.356833in}{5.175000in}}%
\pgfusepath{clip}%
\pgfsetbuttcap%
\pgfsetroundjoin%
\pgfsetlinewidth{1.003750pt}%
\definecolor{currentstroke}{rgb}{0.827451,0.827451,0.827451}%
\pgfsetstrokecolor{currentstroke}%
\pgfsetdash{}{0pt}%
\pgfpathmoveto{\pgfqpoint{0.612319in}{3.180269in}}%
\pgfpathcurveto{\pgfqpoint{0.623369in}{3.180269in}}{\pgfqpoint{0.633968in}{3.184660in}}{\pgfqpoint{0.641781in}{3.192473in}}%
\pgfpathcurveto{\pgfqpoint{0.649595in}{3.200287in}}{\pgfqpoint{0.653985in}{3.210886in}}{\pgfqpoint{0.653985in}{3.221936in}}%
\pgfpathcurveto{\pgfqpoint{0.653985in}{3.232986in}}{\pgfqpoint{0.649595in}{3.243585in}}{\pgfqpoint{0.641781in}{3.251399in}}%
\pgfpathcurveto{\pgfqpoint{0.633968in}{3.259212in}}{\pgfqpoint{0.623369in}{3.263603in}}{\pgfqpoint{0.612319in}{3.263603in}}%
\pgfpathcurveto{\pgfqpoint{0.601269in}{3.263603in}}{\pgfqpoint{0.590669in}{3.259212in}}{\pgfqpoint{0.582856in}{3.251399in}}%
\pgfpathcurveto{\pgfqpoint{0.575042in}{3.243585in}}{\pgfqpoint{0.570652in}{3.232986in}}{\pgfqpoint{0.570652in}{3.221936in}}%
\pgfpathcurveto{\pgfqpoint{0.570652in}{3.210886in}}{\pgfqpoint{0.575042in}{3.200287in}}{\pgfqpoint{0.582856in}{3.192473in}}%
\pgfpathcurveto{\pgfqpoint{0.590669in}{3.184660in}}{\pgfqpoint{0.601269in}{3.180269in}}{\pgfqpoint{0.612319in}{3.180269in}}%
\pgfpathlineto{\pgfqpoint{0.612319in}{3.180269in}}%
\pgfpathclose%
\pgfusepath{stroke}%
\end{pgfscope}%
\begin{pgfscope}%
\pgfpathrectangle{\pgfqpoint{0.393053in}{0.375000in}}{\pgfqpoint{6.356833in}{5.175000in}}%
\pgfusepath{clip}%
\pgfsetbuttcap%
\pgfsetroundjoin%
\pgfsetlinewidth{1.003750pt}%
\definecolor{currentstroke}{rgb}{0.827451,0.827451,0.827451}%
\pgfsetstrokecolor{currentstroke}%
\pgfsetdash{}{0pt}%
\pgfpathmoveto{\pgfqpoint{1.576544in}{1.754527in}}%
\pgfpathcurveto{\pgfqpoint{1.587594in}{1.754527in}}{\pgfqpoint{1.598193in}{1.758917in}}{\pgfqpoint{1.606007in}{1.766731in}}%
\pgfpathcurveto{\pgfqpoint{1.613821in}{1.774544in}}{\pgfqpoint{1.618211in}{1.785143in}}{\pgfqpoint{1.618211in}{1.796194in}}%
\pgfpathcurveto{\pgfqpoint{1.618211in}{1.807244in}}{\pgfqpoint{1.613821in}{1.817843in}}{\pgfqpoint{1.606007in}{1.825656in}}%
\pgfpathcurveto{\pgfqpoint{1.598193in}{1.833470in}}{\pgfqpoint{1.587594in}{1.837860in}}{\pgfqpoint{1.576544in}{1.837860in}}%
\pgfpathcurveto{\pgfqpoint{1.565494in}{1.837860in}}{\pgfqpoint{1.554895in}{1.833470in}}{\pgfqpoint{1.547082in}{1.825656in}}%
\pgfpathcurveto{\pgfqpoint{1.539268in}{1.817843in}}{\pgfqpoint{1.534878in}{1.807244in}}{\pgfqpoint{1.534878in}{1.796194in}}%
\pgfpathcurveto{\pgfqpoint{1.534878in}{1.785143in}}{\pgfqpoint{1.539268in}{1.774544in}}{\pgfqpoint{1.547082in}{1.766731in}}%
\pgfpathcurveto{\pgfqpoint{1.554895in}{1.758917in}}{\pgfqpoint{1.565494in}{1.754527in}}{\pgfqpoint{1.576544in}{1.754527in}}%
\pgfpathlineto{\pgfqpoint{1.576544in}{1.754527in}}%
\pgfpathclose%
\pgfusepath{stroke}%
\end{pgfscope}%
\begin{pgfscope}%
\pgfpathrectangle{\pgfqpoint{0.393053in}{0.375000in}}{\pgfqpoint{6.356833in}{5.175000in}}%
\pgfusepath{clip}%
\pgfsetbuttcap%
\pgfsetroundjoin%
\pgfsetlinewidth{1.003750pt}%
\definecolor{currentstroke}{rgb}{0.827451,0.827451,0.827451}%
\pgfsetstrokecolor{currentstroke}%
\pgfsetdash{}{0pt}%
\pgfpathmoveto{\pgfqpoint{0.435142in}{3.936498in}}%
\pgfpathcurveto{\pgfqpoint{0.446192in}{3.936498in}}{\pgfqpoint{0.456791in}{3.940888in}}{\pgfqpoint{0.464605in}{3.948702in}}%
\pgfpathcurveto{\pgfqpoint{0.472419in}{3.956516in}}{\pgfqpoint{0.476809in}{3.967115in}}{\pgfqpoint{0.476809in}{3.978165in}}%
\pgfpathcurveto{\pgfqpoint{0.476809in}{3.989215in}}{\pgfqpoint{0.472419in}{3.999814in}}{\pgfqpoint{0.464605in}{4.007627in}}%
\pgfpathcurveto{\pgfqpoint{0.456791in}{4.015441in}}{\pgfqpoint{0.446192in}{4.019831in}}{\pgfqpoint{0.435142in}{4.019831in}}%
\pgfpathcurveto{\pgfqpoint{0.424092in}{4.019831in}}{\pgfqpoint{0.413493in}{4.015441in}}{\pgfqpoint{0.405679in}{4.007627in}}%
\pgfpathcurveto{\pgfqpoint{0.397866in}{3.999814in}}{\pgfqpoint{0.393475in}{3.989215in}}{\pgfqpoint{0.393475in}{3.978165in}}%
\pgfpathcurveto{\pgfqpoint{0.393475in}{3.967115in}}{\pgfqpoint{0.397866in}{3.956516in}}{\pgfqpoint{0.405679in}{3.948702in}}%
\pgfpathcurveto{\pgfqpoint{0.413493in}{3.940888in}}{\pgfqpoint{0.424092in}{3.936498in}}{\pgfqpoint{0.435142in}{3.936498in}}%
\pgfpathlineto{\pgfqpoint{0.435142in}{3.936498in}}%
\pgfpathclose%
\pgfusepath{stroke}%
\end{pgfscope}%
\begin{pgfscope}%
\pgfpathrectangle{\pgfqpoint{0.393053in}{0.375000in}}{\pgfqpoint{6.356833in}{5.175000in}}%
\pgfusepath{clip}%
\pgfsetbuttcap%
\pgfsetroundjoin%
\pgfsetlinewidth{1.003750pt}%
\definecolor{currentstroke}{rgb}{0.827451,0.827451,0.827451}%
\pgfsetstrokecolor{currentstroke}%
\pgfsetdash{}{0pt}%
\pgfpathmoveto{\pgfqpoint{1.115030in}{2.257910in}}%
\pgfpathcurveto{\pgfqpoint{1.126080in}{2.257910in}}{\pgfqpoint{1.136679in}{2.262300in}}{\pgfqpoint{1.144493in}{2.270114in}}%
\pgfpathcurveto{\pgfqpoint{1.152307in}{2.277927in}}{\pgfqpoint{1.156697in}{2.288526in}}{\pgfqpoint{1.156697in}{2.299576in}}%
\pgfpathcurveto{\pgfqpoint{1.156697in}{2.310626in}}{\pgfqpoint{1.152307in}{2.321226in}}{\pgfqpoint{1.144493in}{2.329039in}}%
\pgfpathcurveto{\pgfqpoint{1.136679in}{2.336853in}}{\pgfqpoint{1.126080in}{2.341243in}}{\pgfqpoint{1.115030in}{2.341243in}}%
\pgfpathcurveto{\pgfqpoint{1.103980in}{2.341243in}}{\pgfqpoint{1.093381in}{2.336853in}}{\pgfqpoint{1.085567in}{2.329039in}}%
\pgfpathcurveto{\pgfqpoint{1.077754in}{2.321226in}}{\pgfqpoint{1.073364in}{2.310626in}}{\pgfqpoint{1.073364in}{2.299576in}}%
\pgfpathcurveto{\pgfqpoint{1.073364in}{2.288526in}}{\pgfqpoint{1.077754in}{2.277927in}}{\pgfqpoint{1.085567in}{2.270114in}}%
\pgfpathcurveto{\pgfqpoint{1.093381in}{2.262300in}}{\pgfqpoint{1.103980in}{2.257910in}}{\pgfqpoint{1.115030in}{2.257910in}}%
\pgfpathlineto{\pgfqpoint{1.115030in}{2.257910in}}%
\pgfpathclose%
\pgfusepath{stroke}%
\end{pgfscope}%
\begin{pgfscope}%
\pgfpathrectangle{\pgfqpoint{0.393053in}{0.375000in}}{\pgfqpoint{6.356833in}{5.175000in}}%
\pgfusepath{clip}%
\pgfsetbuttcap%
\pgfsetroundjoin%
\pgfsetlinewidth{1.003750pt}%
\definecolor{currentstroke}{rgb}{0.827451,0.827451,0.827451}%
\pgfsetstrokecolor{currentstroke}%
\pgfsetdash{}{0pt}%
\pgfpathmoveto{\pgfqpoint{1.306751in}{1.985193in}}%
\pgfpathcurveto{\pgfqpoint{1.317801in}{1.985193in}}{\pgfqpoint{1.328400in}{1.989583in}}{\pgfqpoint{1.336214in}{1.997397in}}%
\pgfpathcurveto{\pgfqpoint{1.344027in}{2.005210in}}{\pgfqpoint{1.348418in}{2.015809in}}{\pgfqpoint{1.348418in}{2.026859in}}%
\pgfpathcurveto{\pgfqpoint{1.348418in}{2.037909in}}{\pgfqpoint{1.344027in}{2.048509in}}{\pgfqpoint{1.336214in}{2.056322in}}%
\pgfpathcurveto{\pgfqpoint{1.328400in}{2.064136in}}{\pgfqpoint{1.317801in}{2.068526in}}{\pgfqpoint{1.306751in}{2.068526in}}%
\pgfpathcurveto{\pgfqpoint{1.295701in}{2.068526in}}{\pgfqpoint{1.285102in}{2.064136in}}{\pgfqpoint{1.277288in}{2.056322in}}%
\pgfpathcurveto{\pgfqpoint{1.269475in}{2.048509in}}{\pgfqpoint{1.265084in}{2.037909in}}{\pgfqpoint{1.265084in}{2.026859in}}%
\pgfpathcurveto{\pgfqpoint{1.265084in}{2.015809in}}{\pgfqpoint{1.269475in}{2.005210in}}{\pgfqpoint{1.277288in}{1.997397in}}%
\pgfpathcurveto{\pgfqpoint{1.285102in}{1.989583in}}{\pgfqpoint{1.295701in}{1.985193in}}{\pgfqpoint{1.306751in}{1.985193in}}%
\pgfpathlineto{\pgfqpoint{1.306751in}{1.985193in}}%
\pgfpathclose%
\pgfusepath{stroke}%
\end{pgfscope}%
\begin{pgfscope}%
\pgfpathrectangle{\pgfqpoint{0.393053in}{0.375000in}}{\pgfqpoint{6.356833in}{5.175000in}}%
\pgfusepath{clip}%
\pgfsetbuttcap%
\pgfsetroundjoin%
\pgfsetlinewidth{1.003750pt}%
\definecolor{currentstroke}{rgb}{0.827451,0.827451,0.827451}%
\pgfsetstrokecolor{currentstroke}%
\pgfsetdash{}{0pt}%
\pgfpathmoveto{\pgfqpoint{2.966901in}{0.821848in}}%
\pgfpathcurveto{\pgfqpoint{2.977951in}{0.821848in}}{\pgfqpoint{2.988550in}{0.826238in}}{\pgfqpoint{2.996364in}{0.834052in}}%
\pgfpathcurveto{\pgfqpoint{3.004178in}{0.841865in}}{\pgfqpoint{3.008568in}{0.852464in}}{\pgfqpoint{3.008568in}{0.863514in}}%
\pgfpathcurveto{\pgfqpoint{3.008568in}{0.874564in}}{\pgfqpoint{3.004178in}{0.885163in}}{\pgfqpoint{2.996364in}{0.892977in}}%
\pgfpathcurveto{\pgfqpoint{2.988550in}{0.900791in}}{\pgfqpoint{2.977951in}{0.905181in}}{\pgfqpoint{2.966901in}{0.905181in}}%
\pgfpathcurveto{\pgfqpoint{2.955851in}{0.905181in}}{\pgfqpoint{2.945252in}{0.900791in}}{\pgfqpoint{2.937438in}{0.892977in}}%
\pgfpathcurveto{\pgfqpoint{2.929625in}{0.885163in}}{\pgfqpoint{2.925235in}{0.874564in}}{\pgfqpoint{2.925235in}{0.863514in}}%
\pgfpathcurveto{\pgfqpoint{2.925235in}{0.852464in}}{\pgfqpoint{2.929625in}{0.841865in}}{\pgfqpoint{2.937438in}{0.834052in}}%
\pgfpathcurveto{\pgfqpoint{2.945252in}{0.826238in}}{\pgfqpoint{2.955851in}{0.821848in}}{\pgfqpoint{2.966901in}{0.821848in}}%
\pgfpathlineto{\pgfqpoint{2.966901in}{0.821848in}}%
\pgfpathclose%
\pgfusepath{stroke}%
\end{pgfscope}%
\begin{pgfscope}%
\pgfpathrectangle{\pgfqpoint{0.393053in}{0.375000in}}{\pgfqpoint{6.356833in}{5.175000in}}%
\pgfusepath{clip}%
\pgfsetbuttcap%
\pgfsetroundjoin%
\pgfsetlinewidth{1.003750pt}%
\definecolor{currentstroke}{rgb}{0.827451,0.827451,0.827451}%
\pgfsetstrokecolor{currentstroke}%
\pgfsetdash{}{0pt}%
\pgfpathmoveto{\pgfqpoint{0.672197in}{3.023099in}}%
\pgfpathcurveto{\pgfqpoint{0.683247in}{3.023099in}}{\pgfqpoint{0.693846in}{3.027489in}}{\pgfqpoint{0.701660in}{3.035303in}}%
\pgfpathcurveto{\pgfqpoint{0.709473in}{3.043116in}}{\pgfqpoint{0.713863in}{3.053715in}}{\pgfqpoint{0.713863in}{3.064766in}}%
\pgfpathcurveto{\pgfqpoint{0.713863in}{3.075816in}}{\pgfqpoint{0.709473in}{3.086415in}}{\pgfqpoint{0.701660in}{3.094228in}}%
\pgfpathcurveto{\pgfqpoint{0.693846in}{3.102042in}}{\pgfqpoint{0.683247in}{3.106432in}}{\pgfqpoint{0.672197in}{3.106432in}}%
\pgfpathcurveto{\pgfqpoint{0.661147in}{3.106432in}}{\pgfqpoint{0.650548in}{3.102042in}}{\pgfqpoint{0.642734in}{3.094228in}}%
\pgfpathcurveto{\pgfqpoint{0.634920in}{3.086415in}}{\pgfqpoint{0.630530in}{3.075816in}}{\pgfqpoint{0.630530in}{3.064766in}}%
\pgfpathcurveto{\pgfqpoint{0.630530in}{3.053715in}}{\pgfqpoint{0.634920in}{3.043116in}}{\pgfqpoint{0.642734in}{3.035303in}}%
\pgfpathcurveto{\pgfqpoint{0.650548in}{3.027489in}}{\pgfqpoint{0.661147in}{3.023099in}}{\pgfqpoint{0.672197in}{3.023099in}}%
\pgfpathlineto{\pgfqpoint{0.672197in}{3.023099in}}%
\pgfpathclose%
\pgfusepath{stroke}%
\end{pgfscope}%
\begin{pgfscope}%
\pgfpathrectangle{\pgfqpoint{0.393053in}{0.375000in}}{\pgfqpoint{6.356833in}{5.175000in}}%
\pgfusepath{clip}%
\pgfsetbuttcap%
\pgfsetroundjoin%
\pgfsetlinewidth{1.003750pt}%
\definecolor{currentstroke}{rgb}{0.827451,0.827451,0.827451}%
\pgfsetstrokecolor{currentstroke}%
\pgfsetdash{}{0pt}%
\pgfpathmoveto{\pgfqpoint{0.601770in}{3.214331in}}%
\pgfpathcurveto{\pgfqpoint{0.612820in}{3.214331in}}{\pgfqpoint{0.623419in}{3.218722in}}{\pgfqpoint{0.631233in}{3.226535in}}%
\pgfpathcurveto{\pgfqpoint{0.639046in}{3.234349in}}{\pgfqpoint{0.643437in}{3.244948in}}{\pgfqpoint{0.643437in}{3.255998in}}%
\pgfpathcurveto{\pgfqpoint{0.643437in}{3.267048in}}{\pgfqpoint{0.639046in}{3.277647in}}{\pgfqpoint{0.631233in}{3.285461in}}%
\pgfpathcurveto{\pgfqpoint{0.623419in}{3.293275in}}{\pgfqpoint{0.612820in}{3.297665in}}{\pgfqpoint{0.601770in}{3.297665in}}%
\pgfpathcurveto{\pgfqpoint{0.590720in}{3.297665in}}{\pgfqpoint{0.580121in}{3.293275in}}{\pgfqpoint{0.572307in}{3.285461in}}%
\pgfpathcurveto{\pgfqpoint{0.564494in}{3.277647in}}{\pgfqpoint{0.560103in}{3.267048in}}{\pgfqpoint{0.560103in}{3.255998in}}%
\pgfpathcurveto{\pgfqpoint{0.560103in}{3.244948in}}{\pgfqpoint{0.564494in}{3.234349in}}{\pgfqpoint{0.572307in}{3.226535in}}%
\pgfpathcurveto{\pgfqpoint{0.580121in}{3.218722in}}{\pgfqpoint{0.590720in}{3.214331in}}{\pgfqpoint{0.601770in}{3.214331in}}%
\pgfpathlineto{\pgfqpoint{0.601770in}{3.214331in}}%
\pgfpathclose%
\pgfusepath{stroke}%
\end{pgfscope}%
\begin{pgfscope}%
\pgfpathrectangle{\pgfqpoint{0.393053in}{0.375000in}}{\pgfqpoint{6.356833in}{5.175000in}}%
\pgfusepath{clip}%
\pgfsetbuttcap%
\pgfsetroundjoin%
\pgfsetlinewidth{1.003750pt}%
\definecolor{currentstroke}{rgb}{0.827451,0.827451,0.827451}%
\pgfsetstrokecolor{currentstroke}%
\pgfsetdash{}{0pt}%
\pgfpathmoveto{\pgfqpoint{5.017789in}{0.367635in}}%
\pgfpathcurveto{\pgfqpoint{5.028839in}{0.367635in}}{\pgfqpoint{5.039438in}{0.372025in}}{\pgfqpoint{5.047252in}{0.379839in}}%
\pgfpathcurveto{\pgfqpoint{5.055065in}{0.387652in}}{\pgfqpoint{5.059456in}{0.398251in}}{\pgfqpoint{5.059456in}{0.409302in}}%
\pgfpathcurveto{\pgfqpoint{5.059456in}{0.420352in}}{\pgfqpoint{5.055065in}{0.430951in}}{\pgfqpoint{5.047252in}{0.438764in}}%
\pgfpathcurveto{\pgfqpoint{5.039438in}{0.446578in}}{\pgfqpoint{5.028839in}{0.450968in}}{\pgfqpoint{5.017789in}{0.450968in}}%
\pgfpathcurveto{\pgfqpoint{5.006739in}{0.450968in}}{\pgfqpoint{4.996140in}{0.446578in}}{\pgfqpoint{4.988326in}{0.438764in}}%
\pgfpathcurveto{\pgfqpoint{4.980512in}{0.430951in}}{\pgfqpoint{4.976122in}{0.420352in}}{\pgfqpoint{4.976122in}{0.409302in}}%
\pgfpathcurveto{\pgfqpoint{4.976122in}{0.398251in}}{\pgfqpoint{4.980512in}{0.387652in}}{\pgfqpoint{4.988326in}{0.379839in}}%
\pgfpathcurveto{\pgfqpoint{4.996140in}{0.372025in}}{\pgfqpoint{5.006739in}{0.367635in}}{\pgfqpoint{5.017789in}{0.367635in}}%
\pgfusepath{stroke}%
\end{pgfscope}%
\begin{pgfscope}%
\pgfpathrectangle{\pgfqpoint{0.393053in}{0.375000in}}{\pgfqpoint{6.356833in}{5.175000in}}%
\pgfusepath{clip}%
\pgfsetbuttcap%
\pgfsetroundjoin%
\pgfsetlinewidth{1.003750pt}%
\definecolor{currentstroke}{rgb}{0.827451,0.827451,0.827451}%
\pgfsetstrokecolor{currentstroke}%
\pgfsetdash{}{0pt}%
\pgfpathmoveto{\pgfqpoint{1.256856in}{2.039646in}}%
\pgfpathcurveto{\pgfqpoint{1.267906in}{2.039646in}}{\pgfqpoint{1.278505in}{2.044036in}}{\pgfqpoint{1.286319in}{2.051849in}}%
\pgfpathcurveto{\pgfqpoint{1.294132in}{2.059663in}}{\pgfqpoint{1.298523in}{2.070262in}}{\pgfqpoint{1.298523in}{2.081312in}}%
\pgfpathcurveto{\pgfqpoint{1.298523in}{2.092362in}}{\pgfqpoint{1.294132in}{2.102961in}}{\pgfqpoint{1.286319in}{2.110775in}}%
\pgfpathcurveto{\pgfqpoint{1.278505in}{2.118589in}}{\pgfqpoint{1.267906in}{2.122979in}}{\pgfqpoint{1.256856in}{2.122979in}}%
\pgfpathcurveto{\pgfqpoint{1.245806in}{2.122979in}}{\pgfqpoint{1.235207in}{2.118589in}}{\pgfqpoint{1.227393in}{2.110775in}}%
\pgfpathcurveto{\pgfqpoint{1.219580in}{2.102961in}}{\pgfqpoint{1.215189in}{2.092362in}}{\pgfqpoint{1.215189in}{2.081312in}}%
\pgfpathcurveto{\pgfqpoint{1.215189in}{2.070262in}}{\pgfqpoint{1.219580in}{2.059663in}}{\pgfqpoint{1.227393in}{2.051849in}}%
\pgfpathcurveto{\pgfqpoint{1.235207in}{2.044036in}}{\pgfqpoint{1.245806in}{2.039646in}}{\pgfqpoint{1.256856in}{2.039646in}}%
\pgfpathlineto{\pgfqpoint{1.256856in}{2.039646in}}%
\pgfpathclose%
\pgfusepath{stroke}%
\end{pgfscope}%
\begin{pgfscope}%
\pgfpathrectangle{\pgfqpoint{0.393053in}{0.375000in}}{\pgfqpoint{6.356833in}{5.175000in}}%
\pgfusepath{clip}%
\pgfsetbuttcap%
\pgfsetroundjoin%
\pgfsetlinewidth{1.003750pt}%
\definecolor{currentstroke}{rgb}{0.827451,0.827451,0.827451}%
\pgfsetstrokecolor{currentstroke}%
\pgfsetdash{}{0pt}%
\pgfpathmoveto{\pgfqpoint{2.585362in}{1.005073in}}%
\pgfpathcurveto{\pgfqpoint{2.596412in}{1.005073in}}{\pgfqpoint{2.607011in}{1.009463in}}{\pgfqpoint{2.614824in}{1.017277in}}%
\pgfpathcurveto{\pgfqpoint{2.622638in}{1.025090in}}{\pgfqpoint{2.627028in}{1.035689in}}{\pgfqpoint{2.627028in}{1.046740in}}%
\pgfpathcurveto{\pgfqpoint{2.627028in}{1.057790in}}{\pgfqpoint{2.622638in}{1.068389in}}{\pgfqpoint{2.614824in}{1.076202in}}%
\pgfpathcurveto{\pgfqpoint{2.607011in}{1.084016in}}{\pgfqpoint{2.596412in}{1.088406in}}{\pgfqpoint{2.585362in}{1.088406in}}%
\pgfpathcurveto{\pgfqpoint{2.574311in}{1.088406in}}{\pgfqpoint{2.563712in}{1.084016in}}{\pgfqpoint{2.555899in}{1.076202in}}%
\pgfpathcurveto{\pgfqpoint{2.548085in}{1.068389in}}{\pgfqpoint{2.543695in}{1.057790in}}{\pgfqpoint{2.543695in}{1.046740in}}%
\pgfpathcurveto{\pgfqpoint{2.543695in}{1.035689in}}{\pgfqpoint{2.548085in}{1.025090in}}{\pgfqpoint{2.555899in}{1.017277in}}%
\pgfpathcurveto{\pgfqpoint{2.563712in}{1.009463in}}{\pgfqpoint{2.574311in}{1.005073in}}{\pgfqpoint{2.585362in}{1.005073in}}%
\pgfpathlineto{\pgfqpoint{2.585362in}{1.005073in}}%
\pgfpathclose%
\pgfusepath{stroke}%
\end{pgfscope}%
\begin{pgfscope}%
\pgfpathrectangle{\pgfqpoint{0.393053in}{0.375000in}}{\pgfqpoint{6.356833in}{5.175000in}}%
\pgfusepath{clip}%
\pgfsetbuttcap%
\pgfsetroundjoin%
\pgfsetlinewidth{1.003750pt}%
\definecolor{currentstroke}{rgb}{0.827451,0.827451,0.827451}%
\pgfsetstrokecolor{currentstroke}%
\pgfsetdash{}{0pt}%
\pgfpathmoveto{\pgfqpoint{2.920897in}{0.851071in}}%
\pgfpathcurveto{\pgfqpoint{2.931947in}{0.851071in}}{\pgfqpoint{2.942546in}{0.855462in}}{\pgfqpoint{2.950360in}{0.863275in}}%
\pgfpathcurveto{\pgfqpoint{2.958173in}{0.871089in}}{\pgfqpoint{2.962563in}{0.881688in}}{\pgfqpoint{2.962563in}{0.892738in}}%
\pgfpathcurveto{\pgfqpoint{2.962563in}{0.903788in}}{\pgfqpoint{2.958173in}{0.914387in}}{\pgfqpoint{2.950360in}{0.922201in}}%
\pgfpathcurveto{\pgfqpoint{2.942546in}{0.930014in}}{\pgfqpoint{2.931947in}{0.934405in}}{\pgfqpoint{2.920897in}{0.934405in}}%
\pgfpathcurveto{\pgfqpoint{2.909847in}{0.934405in}}{\pgfqpoint{2.899248in}{0.930014in}}{\pgfqpoint{2.891434in}{0.922201in}}%
\pgfpathcurveto{\pgfqpoint{2.883620in}{0.914387in}}{\pgfqpoint{2.879230in}{0.903788in}}{\pgfqpoint{2.879230in}{0.892738in}}%
\pgfpathcurveto{\pgfqpoint{2.879230in}{0.881688in}}{\pgfqpoint{2.883620in}{0.871089in}}{\pgfqpoint{2.891434in}{0.863275in}}%
\pgfpathcurveto{\pgfqpoint{2.899248in}{0.855462in}}{\pgfqpoint{2.909847in}{0.851071in}}{\pgfqpoint{2.920897in}{0.851071in}}%
\pgfpathlineto{\pgfqpoint{2.920897in}{0.851071in}}%
\pgfpathclose%
\pgfusepath{stroke}%
\end{pgfscope}%
\begin{pgfscope}%
\pgfpathrectangle{\pgfqpoint{0.393053in}{0.375000in}}{\pgfqpoint{6.356833in}{5.175000in}}%
\pgfusepath{clip}%
\pgfsetbuttcap%
\pgfsetroundjoin%
\pgfsetlinewidth{1.003750pt}%
\definecolor{currentstroke}{rgb}{0.827451,0.827451,0.827451}%
\pgfsetstrokecolor{currentstroke}%
\pgfsetdash{}{0pt}%
\pgfpathmoveto{\pgfqpoint{0.997743in}{2.491784in}}%
\pgfpathcurveto{\pgfqpoint{1.008793in}{2.491784in}}{\pgfqpoint{1.019392in}{2.496174in}}{\pgfqpoint{1.027206in}{2.503988in}}%
\pgfpathcurveto{\pgfqpoint{1.035019in}{2.511802in}}{\pgfqpoint{1.039410in}{2.522401in}}{\pgfqpoint{1.039410in}{2.533451in}}%
\pgfpathcurveto{\pgfqpoint{1.039410in}{2.544501in}}{\pgfqpoint{1.035019in}{2.555100in}}{\pgfqpoint{1.027206in}{2.562914in}}%
\pgfpathcurveto{\pgfqpoint{1.019392in}{2.570727in}}{\pgfqpoint{1.008793in}{2.575118in}}{\pgfqpoint{0.997743in}{2.575118in}}%
\pgfpathcurveto{\pgfqpoint{0.986693in}{2.575118in}}{\pgfqpoint{0.976094in}{2.570727in}}{\pgfqpoint{0.968280in}{2.562914in}}%
\pgfpathcurveto{\pgfqpoint{0.960466in}{2.555100in}}{\pgfqpoint{0.956076in}{2.544501in}}{\pgfqpoint{0.956076in}{2.533451in}}%
\pgfpathcurveto{\pgfqpoint{0.956076in}{2.522401in}}{\pgfqpoint{0.960466in}{2.511802in}}{\pgfqpoint{0.968280in}{2.503988in}}%
\pgfpathcurveto{\pgfqpoint{0.976094in}{2.496174in}}{\pgfqpoint{0.986693in}{2.491784in}}{\pgfqpoint{0.997743in}{2.491784in}}%
\pgfpathlineto{\pgfqpoint{0.997743in}{2.491784in}}%
\pgfpathclose%
\pgfusepath{stroke}%
\end{pgfscope}%
\begin{pgfscope}%
\pgfpathrectangle{\pgfqpoint{0.393053in}{0.375000in}}{\pgfqpoint{6.356833in}{5.175000in}}%
\pgfusepath{clip}%
\pgfsetbuttcap%
\pgfsetroundjoin%
\pgfsetlinewidth{1.003750pt}%
\definecolor{currentstroke}{rgb}{0.827451,0.827451,0.827451}%
\pgfsetstrokecolor{currentstroke}%
\pgfsetdash{}{0pt}%
\pgfpathmoveto{\pgfqpoint{0.849130in}{2.701883in}}%
\pgfpathcurveto{\pgfqpoint{0.860180in}{2.701883in}}{\pgfqpoint{0.870779in}{2.706274in}}{\pgfqpoint{0.878592in}{2.714087in}}%
\pgfpathcurveto{\pgfqpoint{0.886406in}{2.721901in}}{\pgfqpoint{0.890796in}{2.732500in}}{\pgfqpoint{0.890796in}{2.743550in}}%
\pgfpathcurveto{\pgfqpoint{0.890796in}{2.754600in}}{\pgfqpoint{0.886406in}{2.765199in}}{\pgfqpoint{0.878592in}{2.773013in}}%
\pgfpathcurveto{\pgfqpoint{0.870779in}{2.780826in}}{\pgfqpoint{0.860180in}{2.785217in}}{\pgfqpoint{0.849130in}{2.785217in}}%
\pgfpathcurveto{\pgfqpoint{0.838079in}{2.785217in}}{\pgfqpoint{0.827480in}{2.780826in}}{\pgfqpoint{0.819667in}{2.773013in}}%
\pgfpathcurveto{\pgfqpoint{0.811853in}{2.765199in}}{\pgfqpoint{0.807463in}{2.754600in}}{\pgfqpoint{0.807463in}{2.743550in}}%
\pgfpathcurveto{\pgfqpoint{0.807463in}{2.732500in}}{\pgfqpoint{0.811853in}{2.721901in}}{\pgfqpoint{0.819667in}{2.714087in}}%
\pgfpathcurveto{\pgfqpoint{0.827480in}{2.706274in}}{\pgfqpoint{0.838079in}{2.701883in}}{\pgfqpoint{0.849130in}{2.701883in}}%
\pgfpathlineto{\pgfqpoint{0.849130in}{2.701883in}}%
\pgfpathclose%
\pgfusepath{stroke}%
\end{pgfscope}%
\begin{pgfscope}%
\pgfpathrectangle{\pgfqpoint{0.393053in}{0.375000in}}{\pgfqpoint{6.356833in}{5.175000in}}%
\pgfusepath{clip}%
\pgfsetbuttcap%
\pgfsetroundjoin%
\pgfsetlinewidth{1.003750pt}%
\definecolor{currentstroke}{rgb}{0.827451,0.827451,0.827451}%
\pgfsetstrokecolor{currentstroke}%
\pgfsetdash{}{0pt}%
\pgfpathmoveto{\pgfqpoint{1.354968in}{1.928632in}}%
\pgfpathcurveto{\pgfqpoint{1.366018in}{1.928632in}}{\pgfqpoint{1.376617in}{1.933022in}}{\pgfqpoint{1.384431in}{1.940836in}}%
\pgfpathcurveto{\pgfqpoint{1.392245in}{1.948650in}}{\pgfqpoint{1.396635in}{1.959249in}}{\pgfqpoint{1.396635in}{1.970299in}}%
\pgfpathcurveto{\pgfqpoint{1.396635in}{1.981349in}}{\pgfqpoint{1.392245in}{1.991948in}}{\pgfqpoint{1.384431in}{1.999762in}}%
\pgfpathcurveto{\pgfqpoint{1.376617in}{2.007575in}}{\pgfqpoint{1.366018in}{2.011966in}}{\pgfqpoint{1.354968in}{2.011966in}}%
\pgfpathcurveto{\pgfqpoint{1.343918in}{2.011966in}}{\pgfqpoint{1.333319in}{2.007575in}}{\pgfqpoint{1.325505in}{1.999762in}}%
\pgfpathcurveto{\pgfqpoint{1.317692in}{1.991948in}}{\pgfqpoint{1.313302in}{1.981349in}}{\pgfqpoint{1.313302in}{1.970299in}}%
\pgfpathcurveto{\pgfqpoint{1.313302in}{1.959249in}}{\pgfqpoint{1.317692in}{1.948650in}}{\pgfqpoint{1.325505in}{1.940836in}}%
\pgfpathcurveto{\pgfqpoint{1.333319in}{1.933022in}}{\pgfqpoint{1.343918in}{1.928632in}}{\pgfqpoint{1.354968in}{1.928632in}}%
\pgfpathlineto{\pgfqpoint{1.354968in}{1.928632in}}%
\pgfpathclose%
\pgfusepath{stroke}%
\end{pgfscope}%
\begin{pgfscope}%
\pgfpathrectangle{\pgfqpoint{0.393053in}{0.375000in}}{\pgfqpoint{6.356833in}{5.175000in}}%
\pgfusepath{clip}%
\pgfsetbuttcap%
\pgfsetroundjoin%
\pgfsetlinewidth{1.003750pt}%
\definecolor{currentstroke}{rgb}{0.827451,0.827451,0.827451}%
\pgfsetstrokecolor{currentstroke}%
\pgfsetdash{}{0pt}%
\pgfpathmoveto{\pgfqpoint{1.768437in}{1.558678in}}%
\pgfpathcurveto{\pgfqpoint{1.779488in}{1.558678in}}{\pgfqpoint{1.790087in}{1.563068in}}{\pgfqpoint{1.797900in}{1.570882in}}%
\pgfpathcurveto{\pgfqpoint{1.805714in}{1.578696in}}{\pgfqpoint{1.810104in}{1.589295in}}{\pgfqpoint{1.810104in}{1.600345in}}%
\pgfpathcurveto{\pgfqpoint{1.810104in}{1.611395in}}{\pgfqpoint{1.805714in}{1.621994in}}{\pgfqpoint{1.797900in}{1.629807in}}%
\pgfpathcurveto{\pgfqpoint{1.790087in}{1.637621in}}{\pgfqpoint{1.779488in}{1.642011in}}{\pgfqpoint{1.768437in}{1.642011in}}%
\pgfpathcurveto{\pgfqpoint{1.757387in}{1.642011in}}{\pgfqpoint{1.746788in}{1.637621in}}{\pgfqpoint{1.738975in}{1.629807in}}%
\pgfpathcurveto{\pgfqpoint{1.731161in}{1.621994in}}{\pgfqpoint{1.726771in}{1.611395in}}{\pgfqpoint{1.726771in}{1.600345in}}%
\pgfpathcurveto{\pgfqpoint{1.726771in}{1.589295in}}{\pgfqpoint{1.731161in}{1.578696in}}{\pgfqpoint{1.738975in}{1.570882in}}%
\pgfpathcurveto{\pgfqpoint{1.746788in}{1.563068in}}{\pgfqpoint{1.757387in}{1.558678in}}{\pgfqpoint{1.768437in}{1.558678in}}%
\pgfpathlineto{\pgfqpoint{1.768437in}{1.558678in}}%
\pgfpathclose%
\pgfusepath{stroke}%
\end{pgfscope}%
\begin{pgfscope}%
\pgfpathrectangle{\pgfqpoint{0.393053in}{0.375000in}}{\pgfqpoint{6.356833in}{5.175000in}}%
\pgfusepath{clip}%
\pgfsetbuttcap%
\pgfsetroundjoin%
\pgfsetlinewidth{1.003750pt}%
\definecolor{currentstroke}{rgb}{0.827451,0.827451,0.827451}%
\pgfsetstrokecolor{currentstroke}%
\pgfsetdash{}{0pt}%
\pgfpathmoveto{\pgfqpoint{0.580834in}{3.297265in}}%
\pgfpathcurveto{\pgfqpoint{0.591884in}{3.297265in}}{\pgfqpoint{0.602483in}{3.301655in}}{\pgfqpoint{0.610297in}{3.309469in}}%
\pgfpathcurveto{\pgfqpoint{0.618111in}{3.317282in}}{\pgfqpoint{0.622501in}{3.327881in}}{\pgfqpoint{0.622501in}{3.338932in}}%
\pgfpathcurveto{\pgfqpoint{0.622501in}{3.349982in}}{\pgfqpoint{0.618111in}{3.360581in}}{\pgfqpoint{0.610297in}{3.368394in}}%
\pgfpathcurveto{\pgfqpoint{0.602483in}{3.376208in}}{\pgfqpoint{0.591884in}{3.380598in}}{\pgfqpoint{0.580834in}{3.380598in}}%
\pgfpathcurveto{\pgfqpoint{0.569784in}{3.380598in}}{\pgfqpoint{0.559185in}{3.376208in}}{\pgfqpoint{0.551371in}{3.368394in}}%
\pgfpathcurveto{\pgfqpoint{0.543558in}{3.360581in}}{\pgfqpoint{0.539167in}{3.349982in}}{\pgfqpoint{0.539167in}{3.338932in}}%
\pgfpathcurveto{\pgfqpoint{0.539167in}{3.327881in}}{\pgfqpoint{0.543558in}{3.317282in}}{\pgfqpoint{0.551371in}{3.309469in}}%
\pgfpathcurveto{\pgfqpoint{0.559185in}{3.301655in}}{\pgfqpoint{0.569784in}{3.297265in}}{\pgfqpoint{0.580834in}{3.297265in}}%
\pgfpathlineto{\pgfqpoint{0.580834in}{3.297265in}}%
\pgfpathclose%
\pgfusepath{stroke}%
\end{pgfscope}%
\begin{pgfscope}%
\pgfpathrectangle{\pgfqpoint{0.393053in}{0.375000in}}{\pgfqpoint{6.356833in}{5.175000in}}%
\pgfusepath{clip}%
\pgfsetbuttcap%
\pgfsetroundjoin%
\pgfsetlinewidth{1.003750pt}%
\definecolor{currentstroke}{rgb}{0.827451,0.827451,0.827451}%
\pgfsetstrokecolor{currentstroke}%
\pgfsetdash{}{0pt}%
\pgfpathmoveto{\pgfqpoint{0.452170in}{3.855614in}}%
\pgfpathcurveto{\pgfqpoint{0.463220in}{3.855614in}}{\pgfqpoint{0.473819in}{3.860005in}}{\pgfqpoint{0.481633in}{3.867818in}}%
\pgfpathcurveto{\pgfqpoint{0.489447in}{3.875632in}}{\pgfqpoint{0.493837in}{3.886231in}}{\pgfqpoint{0.493837in}{3.897281in}}%
\pgfpathcurveto{\pgfqpoint{0.493837in}{3.908331in}}{\pgfqpoint{0.489447in}{3.918930in}}{\pgfqpoint{0.481633in}{3.926744in}}%
\pgfpathcurveto{\pgfqpoint{0.473819in}{3.934557in}}{\pgfqpoint{0.463220in}{3.938948in}}{\pgfqpoint{0.452170in}{3.938948in}}%
\pgfpathcurveto{\pgfqpoint{0.441120in}{3.938948in}}{\pgfqpoint{0.430521in}{3.934557in}}{\pgfqpoint{0.422707in}{3.926744in}}%
\pgfpathcurveto{\pgfqpoint{0.414894in}{3.918930in}}{\pgfqpoint{0.410503in}{3.908331in}}{\pgfqpoint{0.410503in}{3.897281in}}%
\pgfpathcurveto{\pgfqpoint{0.410503in}{3.886231in}}{\pgfqpoint{0.414894in}{3.875632in}}{\pgfqpoint{0.422707in}{3.867818in}}%
\pgfpathcurveto{\pgfqpoint{0.430521in}{3.860005in}}{\pgfqpoint{0.441120in}{3.855614in}}{\pgfqpoint{0.452170in}{3.855614in}}%
\pgfpathlineto{\pgfqpoint{0.452170in}{3.855614in}}%
\pgfpathclose%
\pgfusepath{stroke}%
\end{pgfscope}%
\begin{pgfscope}%
\pgfpathrectangle{\pgfqpoint{0.393053in}{0.375000in}}{\pgfqpoint{6.356833in}{5.175000in}}%
\pgfusepath{clip}%
\pgfsetbuttcap%
\pgfsetroundjoin%
\pgfsetlinewidth{1.003750pt}%
\definecolor{currentstroke}{rgb}{0.827451,0.827451,0.827451}%
\pgfsetstrokecolor{currentstroke}%
\pgfsetdash{}{0pt}%
\pgfpathmoveto{\pgfqpoint{4.176160in}{0.456257in}}%
\pgfpathcurveto{\pgfqpoint{4.187210in}{0.456257in}}{\pgfqpoint{4.197809in}{0.460647in}}{\pgfqpoint{4.205623in}{0.468461in}}%
\pgfpathcurveto{\pgfqpoint{4.213436in}{0.476274in}}{\pgfqpoint{4.217827in}{0.486873in}}{\pgfqpoint{4.217827in}{0.497923in}}%
\pgfpathcurveto{\pgfqpoint{4.217827in}{0.508973in}}{\pgfqpoint{4.213436in}{0.519572in}}{\pgfqpoint{4.205623in}{0.527386in}}%
\pgfpathcurveto{\pgfqpoint{4.197809in}{0.535200in}}{\pgfqpoint{4.187210in}{0.539590in}}{\pgfqpoint{4.176160in}{0.539590in}}%
\pgfpathcurveto{\pgfqpoint{4.165110in}{0.539590in}}{\pgfqpoint{4.154511in}{0.535200in}}{\pgfqpoint{4.146697in}{0.527386in}}%
\pgfpathcurveto{\pgfqpoint{4.138883in}{0.519572in}}{\pgfqpoint{4.134493in}{0.508973in}}{\pgfqpoint{4.134493in}{0.497923in}}%
\pgfpathcurveto{\pgfqpoint{4.134493in}{0.486873in}}{\pgfqpoint{4.138883in}{0.476274in}}{\pgfqpoint{4.146697in}{0.468461in}}%
\pgfpathcurveto{\pgfqpoint{4.154511in}{0.460647in}}{\pgfqpoint{4.165110in}{0.456257in}}{\pgfqpoint{4.176160in}{0.456257in}}%
\pgfpathlineto{\pgfqpoint{4.176160in}{0.456257in}}%
\pgfpathclose%
\pgfusepath{stroke}%
\end{pgfscope}%
\begin{pgfscope}%
\pgfpathrectangle{\pgfqpoint{0.393053in}{0.375000in}}{\pgfqpoint{6.356833in}{5.175000in}}%
\pgfusepath{clip}%
\pgfsetbuttcap%
\pgfsetroundjoin%
\pgfsetlinewidth{1.003750pt}%
\definecolor{currentstroke}{rgb}{0.827451,0.827451,0.827451}%
\pgfsetstrokecolor{currentstroke}%
\pgfsetdash{}{0pt}%
\pgfpathmoveto{\pgfqpoint{4.927239in}{0.379634in}}%
\pgfpathcurveto{\pgfqpoint{4.938289in}{0.379634in}}{\pgfqpoint{4.948888in}{0.384025in}}{\pgfqpoint{4.956702in}{0.391838in}}%
\pgfpathcurveto{\pgfqpoint{4.964516in}{0.399652in}}{\pgfqpoint{4.968906in}{0.410251in}}{\pgfqpoint{4.968906in}{0.421301in}}%
\pgfpathcurveto{\pgfqpoint{4.968906in}{0.432351in}}{\pgfqpoint{4.964516in}{0.442950in}}{\pgfqpoint{4.956702in}{0.450764in}}%
\pgfpathcurveto{\pgfqpoint{4.948888in}{0.458578in}}{\pgfqpoint{4.938289in}{0.462968in}}{\pgfqpoint{4.927239in}{0.462968in}}%
\pgfpathcurveto{\pgfqpoint{4.916189in}{0.462968in}}{\pgfqpoint{4.905590in}{0.458578in}}{\pgfqpoint{4.897776in}{0.450764in}}%
\pgfpathcurveto{\pgfqpoint{4.889963in}{0.442950in}}{\pgfqpoint{4.885572in}{0.432351in}}{\pgfqpoint{4.885572in}{0.421301in}}%
\pgfpathcurveto{\pgfqpoint{4.885572in}{0.410251in}}{\pgfqpoint{4.889963in}{0.399652in}}{\pgfqpoint{4.897776in}{0.391838in}}%
\pgfpathcurveto{\pgfqpoint{4.905590in}{0.384025in}}{\pgfqpoint{4.916189in}{0.379634in}}{\pgfqpoint{4.927239in}{0.379634in}}%
\pgfpathlineto{\pgfqpoint{4.927239in}{0.379634in}}%
\pgfpathclose%
\pgfusepath{stroke}%
\end{pgfscope}%
\begin{pgfscope}%
\pgfpathrectangle{\pgfqpoint{0.393053in}{0.375000in}}{\pgfqpoint{6.356833in}{5.175000in}}%
\pgfusepath{clip}%
\pgfsetbuttcap%
\pgfsetroundjoin%
\pgfsetlinewidth{1.003750pt}%
\definecolor{currentstroke}{rgb}{0.827451,0.827451,0.827451}%
\pgfsetstrokecolor{currentstroke}%
\pgfsetdash{}{0pt}%
\pgfpathmoveto{\pgfqpoint{3.741594in}{0.563993in}}%
\pgfpathcurveto{\pgfqpoint{3.752644in}{0.563993in}}{\pgfqpoint{3.763243in}{0.568383in}}{\pgfqpoint{3.771057in}{0.576197in}}%
\pgfpathcurveto{\pgfqpoint{3.778871in}{0.584010in}}{\pgfqpoint{3.783261in}{0.594609in}}{\pgfqpoint{3.783261in}{0.605660in}}%
\pgfpathcurveto{\pgfqpoint{3.783261in}{0.616710in}}{\pgfqpoint{3.778871in}{0.627309in}}{\pgfqpoint{3.771057in}{0.635122in}}%
\pgfpathcurveto{\pgfqpoint{3.763243in}{0.642936in}}{\pgfqpoint{3.752644in}{0.647326in}}{\pgfqpoint{3.741594in}{0.647326in}}%
\pgfpathcurveto{\pgfqpoint{3.730544in}{0.647326in}}{\pgfqpoint{3.719945in}{0.642936in}}{\pgfqpoint{3.712131in}{0.635122in}}%
\pgfpathcurveto{\pgfqpoint{3.704318in}{0.627309in}}{\pgfqpoint{3.699928in}{0.616710in}}{\pgfqpoint{3.699928in}{0.605660in}}%
\pgfpathcurveto{\pgfqpoint{3.699928in}{0.594609in}}{\pgfqpoint{3.704318in}{0.584010in}}{\pgfqpoint{3.712131in}{0.576197in}}%
\pgfpathcurveto{\pgfqpoint{3.719945in}{0.568383in}}{\pgfqpoint{3.730544in}{0.563993in}}{\pgfqpoint{3.741594in}{0.563993in}}%
\pgfpathlineto{\pgfqpoint{3.741594in}{0.563993in}}%
\pgfpathclose%
\pgfusepath{stroke}%
\end{pgfscope}%
\begin{pgfscope}%
\pgfpathrectangle{\pgfqpoint{0.393053in}{0.375000in}}{\pgfqpoint{6.356833in}{5.175000in}}%
\pgfusepath{clip}%
\pgfsetbuttcap%
\pgfsetroundjoin%
\pgfsetlinewidth{1.003750pt}%
\definecolor{currentstroke}{rgb}{0.827451,0.827451,0.827451}%
\pgfsetstrokecolor{currentstroke}%
\pgfsetdash{}{0pt}%
\pgfpathmoveto{\pgfqpoint{0.743892in}{2.889747in}}%
\pgfpathcurveto{\pgfqpoint{0.754942in}{2.889747in}}{\pgfqpoint{0.765541in}{2.894137in}}{\pgfqpoint{0.773355in}{2.901951in}}%
\pgfpathcurveto{\pgfqpoint{0.781168in}{2.909765in}}{\pgfqpoint{0.785558in}{2.920364in}}{\pgfqpoint{0.785558in}{2.931414in}}%
\pgfpathcurveto{\pgfqpoint{0.785558in}{2.942464in}}{\pgfqpoint{0.781168in}{2.953063in}}{\pgfqpoint{0.773355in}{2.960877in}}%
\pgfpathcurveto{\pgfqpoint{0.765541in}{2.968690in}}{\pgfqpoint{0.754942in}{2.973081in}}{\pgfqpoint{0.743892in}{2.973081in}}%
\pgfpathcurveto{\pgfqpoint{0.732842in}{2.973081in}}{\pgfqpoint{0.722243in}{2.968690in}}{\pgfqpoint{0.714429in}{2.960877in}}%
\pgfpathcurveto{\pgfqpoint{0.706615in}{2.953063in}}{\pgfqpoint{0.702225in}{2.942464in}}{\pgfqpoint{0.702225in}{2.931414in}}%
\pgfpathcurveto{\pgfqpoint{0.702225in}{2.920364in}}{\pgfqpoint{0.706615in}{2.909765in}}{\pgfqpoint{0.714429in}{2.901951in}}%
\pgfpathcurveto{\pgfqpoint{0.722243in}{2.894137in}}{\pgfqpoint{0.732842in}{2.889747in}}{\pgfqpoint{0.743892in}{2.889747in}}%
\pgfpathlineto{\pgfqpoint{0.743892in}{2.889747in}}%
\pgfpathclose%
\pgfusepath{stroke}%
\end{pgfscope}%
\begin{pgfscope}%
\pgfpathrectangle{\pgfqpoint{0.393053in}{0.375000in}}{\pgfqpoint{6.356833in}{5.175000in}}%
\pgfusepath{clip}%
\pgfsetbuttcap%
\pgfsetroundjoin%
\pgfsetlinewidth{1.003750pt}%
\definecolor{currentstroke}{rgb}{0.827451,0.827451,0.827451}%
\pgfsetstrokecolor{currentstroke}%
\pgfsetdash{}{0pt}%
\pgfpathmoveto{\pgfqpoint{0.508412in}{3.594376in}}%
\pgfpathcurveto{\pgfqpoint{0.519462in}{3.594376in}}{\pgfqpoint{0.530061in}{3.598766in}}{\pgfqpoint{0.537874in}{3.606580in}}%
\pgfpathcurveto{\pgfqpoint{0.545688in}{3.614393in}}{\pgfqpoint{0.550078in}{3.624992in}}{\pgfqpoint{0.550078in}{3.636042in}}%
\pgfpathcurveto{\pgfqpoint{0.550078in}{3.647092in}}{\pgfqpoint{0.545688in}{3.657692in}}{\pgfqpoint{0.537874in}{3.665505in}}%
\pgfpathcurveto{\pgfqpoint{0.530061in}{3.673319in}}{\pgfqpoint{0.519462in}{3.677709in}}{\pgfqpoint{0.508412in}{3.677709in}}%
\pgfpathcurveto{\pgfqpoint{0.497361in}{3.677709in}}{\pgfqpoint{0.486762in}{3.673319in}}{\pgfqpoint{0.478949in}{3.665505in}}%
\pgfpathcurveto{\pgfqpoint{0.471135in}{3.657692in}}{\pgfqpoint{0.466745in}{3.647092in}}{\pgfqpoint{0.466745in}{3.636042in}}%
\pgfpathcurveto{\pgfqpoint{0.466745in}{3.624992in}}{\pgfqpoint{0.471135in}{3.614393in}}{\pgfqpoint{0.478949in}{3.606580in}}%
\pgfpathcurveto{\pgfqpoint{0.486762in}{3.598766in}}{\pgfqpoint{0.497361in}{3.594376in}}{\pgfqpoint{0.508412in}{3.594376in}}%
\pgfpathlineto{\pgfqpoint{0.508412in}{3.594376in}}%
\pgfpathclose%
\pgfusepath{stroke}%
\end{pgfscope}%
\begin{pgfscope}%
\pgfpathrectangle{\pgfqpoint{0.393053in}{0.375000in}}{\pgfqpoint{6.356833in}{5.175000in}}%
\pgfusepath{clip}%
\pgfsetbuttcap%
\pgfsetroundjoin%
\pgfsetlinewidth{1.003750pt}%
\definecolor{currentstroke}{rgb}{0.827451,0.827451,0.827451}%
\pgfsetstrokecolor{currentstroke}%
\pgfsetdash{}{0pt}%
\pgfpathmoveto{\pgfqpoint{0.732342in}{2.910264in}}%
\pgfpathcurveto{\pgfqpoint{0.743392in}{2.910264in}}{\pgfqpoint{0.753991in}{2.914654in}}{\pgfqpoint{0.761805in}{2.922468in}}%
\pgfpathcurveto{\pgfqpoint{0.769618in}{2.930282in}}{\pgfqpoint{0.774008in}{2.940881in}}{\pgfqpoint{0.774008in}{2.951931in}}%
\pgfpathcurveto{\pgfqpoint{0.774008in}{2.962981in}}{\pgfqpoint{0.769618in}{2.973580in}}{\pgfqpoint{0.761805in}{2.981394in}}%
\pgfpathcurveto{\pgfqpoint{0.753991in}{2.989207in}}{\pgfqpoint{0.743392in}{2.993598in}}{\pgfqpoint{0.732342in}{2.993598in}}%
\pgfpathcurveto{\pgfqpoint{0.721292in}{2.993598in}}{\pgfqpoint{0.710693in}{2.989207in}}{\pgfqpoint{0.702879in}{2.981394in}}%
\pgfpathcurveto{\pgfqpoint{0.695065in}{2.973580in}}{\pgfqpoint{0.690675in}{2.962981in}}{\pgfqpoint{0.690675in}{2.951931in}}%
\pgfpathcurveto{\pgfqpoint{0.690675in}{2.940881in}}{\pgfqpoint{0.695065in}{2.930282in}}{\pgfqpoint{0.702879in}{2.922468in}}%
\pgfpathcurveto{\pgfqpoint{0.710693in}{2.914654in}}{\pgfqpoint{0.721292in}{2.910264in}}{\pgfqpoint{0.732342in}{2.910264in}}%
\pgfpathlineto{\pgfqpoint{0.732342in}{2.910264in}}%
\pgfpathclose%
\pgfusepath{stroke}%
\end{pgfscope}%
\begin{pgfscope}%
\pgfpathrectangle{\pgfqpoint{0.393053in}{0.375000in}}{\pgfqpoint{6.356833in}{5.175000in}}%
\pgfusepath{clip}%
\pgfsetbuttcap%
\pgfsetroundjoin%
\pgfsetlinewidth{1.003750pt}%
\definecolor{currentstroke}{rgb}{0.827451,0.827451,0.827451}%
\pgfsetstrokecolor{currentstroke}%
\pgfsetdash{}{0pt}%
\pgfpathmoveto{\pgfqpoint{3.774127in}{0.547432in}}%
\pgfpathcurveto{\pgfqpoint{3.785177in}{0.547432in}}{\pgfqpoint{3.795776in}{0.551822in}}{\pgfqpoint{3.803589in}{0.559636in}}%
\pgfpathcurveto{\pgfqpoint{3.811403in}{0.567450in}}{\pgfqpoint{3.815793in}{0.578049in}}{\pgfqpoint{3.815793in}{0.589099in}}%
\pgfpathcurveto{\pgfqpoint{3.815793in}{0.600149in}}{\pgfqpoint{3.811403in}{0.610748in}}{\pgfqpoint{3.803589in}{0.618562in}}%
\pgfpathcurveto{\pgfqpoint{3.795776in}{0.626375in}}{\pgfqpoint{3.785177in}{0.630766in}}{\pgfqpoint{3.774127in}{0.630766in}}%
\pgfpathcurveto{\pgfqpoint{3.763077in}{0.630766in}}{\pgfqpoint{3.752478in}{0.626375in}}{\pgfqpoint{3.744664in}{0.618562in}}%
\pgfpathcurveto{\pgfqpoint{3.736850in}{0.610748in}}{\pgfqpoint{3.732460in}{0.600149in}}{\pgfqpoint{3.732460in}{0.589099in}}%
\pgfpathcurveto{\pgfqpoint{3.732460in}{0.578049in}}{\pgfqpoint{3.736850in}{0.567450in}}{\pgfqpoint{3.744664in}{0.559636in}}%
\pgfpathcurveto{\pgfqpoint{3.752478in}{0.551822in}}{\pgfqpoint{3.763077in}{0.547432in}}{\pgfqpoint{3.774127in}{0.547432in}}%
\pgfpathlineto{\pgfqpoint{3.774127in}{0.547432in}}%
\pgfpathclose%
\pgfusepath{stroke}%
\end{pgfscope}%
\begin{pgfscope}%
\pgfpathrectangle{\pgfqpoint{0.393053in}{0.375000in}}{\pgfqpoint{6.356833in}{5.175000in}}%
\pgfusepath{clip}%
\pgfsetbuttcap%
\pgfsetroundjoin%
\pgfsetlinewidth{1.003750pt}%
\definecolor{currentstroke}{rgb}{0.827451,0.827451,0.827451}%
\pgfsetstrokecolor{currentstroke}%
\pgfsetdash{}{0pt}%
\pgfpathmoveto{\pgfqpoint{0.961220in}{2.522545in}}%
\pgfpathcurveto{\pgfqpoint{0.972270in}{2.522545in}}{\pgfqpoint{0.982869in}{2.526936in}}{\pgfqpoint{0.990683in}{2.534749in}}%
\pgfpathcurveto{\pgfqpoint{0.998497in}{2.542563in}}{\pgfqpoint{1.002887in}{2.553162in}}{\pgfqpoint{1.002887in}{2.564212in}}%
\pgfpathcurveto{\pgfqpoint{1.002887in}{2.575262in}}{\pgfqpoint{0.998497in}{2.585861in}}{\pgfqpoint{0.990683in}{2.593675in}}%
\pgfpathcurveto{\pgfqpoint{0.982869in}{2.601488in}}{\pgfqpoint{0.972270in}{2.605879in}}{\pgfqpoint{0.961220in}{2.605879in}}%
\pgfpathcurveto{\pgfqpoint{0.950170in}{2.605879in}}{\pgfqpoint{0.939571in}{2.601488in}}{\pgfqpoint{0.931757in}{2.593675in}}%
\pgfpathcurveto{\pgfqpoint{0.923944in}{2.585861in}}{\pgfqpoint{0.919554in}{2.575262in}}{\pgfqpoint{0.919554in}{2.564212in}}%
\pgfpathcurveto{\pgfqpoint{0.919554in}{2.553162in}}{\pgfqpoint{0.923944in}{2.542563in}}{\pgfqpoint{0.931757in}{2.534749in}}%
\pgfpathcurveto{\pgfqpoint{0.939571in}{2.526936in}}{\pgfqpoint{0.950170in}{2.522545in}}{\pgfqpoint{0.961220in}{2.522545in}}%
\pgfpathlineto{\pgfqpoint{0.961220in}{2.522545in}}%
\pgfpathclose%
\pgfusepath{stroke}%
\end{pgfscope}%
\begin{pgfscope}%
\pgfpathrectangle{\pgfqpoint{0.393053in}{0.375000in}}{\pgfqpoint{6.356833in}{5.175000in}}%
\pgfusepath{clip}%
\pgfsetbuttcap%
\pgfsetroundjoin%
\pgfsetlinewidth{1.003750pt}%
\definecolor{currentstroke}{rgb}{0.827451,0.827451,0.827451}%
\pgfsetstrokecolor{currentstroke}%
\pgfsetdash{}{0pt}%
\pgfpathmoveto{\pgfqpoint{0.583953in}{3.273995in}}%
\pgfpathcurveto{\pgfqpoint{0.595003in}{3.273995in}}{\pgfqpoint{0.605602in}{3.278385in}}{\pgfqpoint{0.613416in}{3.286198in}}%
\pgfpathcurveto{\pgfqpoint{0.621229in}{3.294012in}}{\pgfqpoint{0.625620in}{3.304611in}}{\pgfqpoint{0.625620in}{3.315661in}}%
\pgfpathcurveto{\pgfqpoint{0.625620in}{3.326711in}}{\pgfqpoint{0.621229in}{3.337310in}}{\pgfqpoint{0.613416in}{3.345124in}}%
\pgfpathcurveto{\pgfqpoint{0.605602in}{3.352938in}}{\pgfqpoint{0.595003in}{3.357328in}}{\pgfqpoint{0.583953in}{3.357328in}}%
\pgfpathcurveto{\pgfqpoint{0.572903in}{3.357328in}}{\pgfqpoint{0.562304in}{3.352938in}}{\pgfqpoint{0.554490in}{3.345124in}}%
\pgfpathcurveto{\pgfqpoint{0.546677in}{3.337310in}}{\pgfqpoint{0.542286in}{3.326711in}}{\pgfqpoint{0.542286in}{3.315661in}}%
\pgfpathcurveto{\pgfqpoint{0.542286in}{3.304611in}}{\pgfqpoint{0.546677in}{3.294012in}}{\pgfqpoint{0.554490in}{3.286198in}}%
\pgfpathcurveto{\pgfqpoint{0.562304in}{3.278385in}}{\pgfqpoint{0.572903in}{3.273995in}}{\pgfqpoint{0.583953in}{3.273995in}}%
\pgfpathlineto{\pgfqpoint{0.583953in}{3.273995in}}%
\pgfpathclose%
\pgfusepath{stroke}%
\end{pgfscope}%
\begin{pgfscope}%
\pgfpathrectangle{\pgfqpoint{0.393053in}{0.375000in}}{\pgfqpoint{6.356833in}{5.175000in}}%
\pgfusepath{clip}%
\pgfsetbuttcap%
\pgfsetroundjoin%
\pgfsetlinewidth{1.003750pt}%
\definecolor{currentstroke}{rgb}{0.827451,0.827451,0.827451}%
\pgfsetstrokecolor{currentstroke}%
\pgfsetdash{}{0pt}%
\pgfpathmoveto{\pgfqpoint{1.658477in}{1.641127in}}%
\pgfpathcurveto{\pgfqpoint{1.669527in}{1.641127in}}{\pgfqpoint{1.680126in}{1.645517in}}{\pgfqpoint{1.687939in}{1.653331in}}%
\pgfpathcurveto{\pgfqpoint{1.695753in}{1.661144in}}{\pgfqpoint{1.700143in}{1.671743in}}{\pgfqpoint{1.700143in}{1.682794in}}%
\pgfpathcurveto{\pgfqpoint{1.700143in}{1.693844in}}{\pgfqpoint{1.695753in}{1.704443in}}{\pgfqpoint{1.687939in}{1.712256in}}%
\pgfpathcurveto{\pgfqpoint{1.680126in}{1.720070in}}{\pgfqpoint{1.669527in}{1.724460in}}{\pgfqpoint{1.658477in}{1.724460in}}%
\pgfpathcurveto{\pgfqpoint{1.647427in}{1.724460in}}{\pgfqpoint{1.636828in}{1.720070in}}{\pgfqpoint{1.629014in}{1.712256in}}%
\pgfpathcurveto{\pgfqpoint{1.621200in}{1.704443in}}{\pgfqpoint{1.616810in}{1.693844in}}{\pgfqpoint{1.616810in}{1.682794in}}%
\pgfpathcurveto{\pgfqpoint{1.616810in}{1.671743in}}{\pgfqpoint{1.621200in}{1.661144in}}{\pgfqpoint{1.629014in}{1.653331in}}%
\pgfpathcurveto{\pgfqpoint{1.636828in}{1.645517in}}{\pgfqpoint{1.647427in}{1.641127in}}{\pgfqpoint{1.658477in}{1.641127in}}%
\pgfpathlineto{\pgfqpoint{1.658477in}{1.641127in}}%
\pgfpathclose%
\pgfusepath{stroke}%
\end{pgfscope}%
\begin{pgfscope}%
\pgfpathrectangle{\pgfqpoint{0.393053in}{0.375000in}}{\pgfqpoint{6.356833in}{5.175000in}}%
\pgfusepath{clip}%
\pgfsetbuttcap%
\pgfsetroundjoin%
\pgfsetlinewidth{1.003750pt}%
\definecolor{currentstroke}{rgb}{0.827451,0.827451,0.827451}%
\pgfsetstrokecolor{currentstroke}%
\pgfsetdash{}{0pt}%
\pgfpathmoveto{\pgfqpoint{1.882270in}{1.458195in}}%
\pgfpathcurveto{\pgfqpoint{1.893321in}{1.458195in}}{\pgfqpoint{1.903920in}{1.462585in}}{\pgfqpoint{1.911733in}{1.470399in}}%
\pgfpathcurveto{\pgfqpoint{1.919547in}{1.478213in}}{\pgfqpoint{1.923937in}{1.488812in}}{\pgfqpoint{1.923937in}{1.499862in}}%
\pgfpathcurveto{\pgfqpoint{1.923937in}{1.510912in}}{\pgfqpoint{1.919547in}{1.521511in}}{\pgfqpoint{1.911733in}{1.529325in}}%
\pgfpathcurveto{\pgfqpoint{1.903920in}{1.537138in}}{\pgfqpoint{1.893321in}{1.541529in}}{\pgfqpoint{1.882270in}{1.541529in}}%
\pgfpathcurveto{\pgfqpoint{1.871220in}{1.541529in}}{\pgfqpoint{1.860621in}{1.537138in}}{\pgfqpoint{1.852808in}{1.529325in}}%
\pgfpathcurveto{\pgfqpoint{1.844994in}{1.521511in}}{\pgfqpoint{1.840604in}{1.510912in}}{\pgfqpoint{1.840604in}{1.499862in}}%
\pgfpathcurveto{\pgfqpoint{1.840604in}{1.488812in}}{\pgfqpoint{1.844994in}{1.478213in}}{\pgfqpoint{1.852808in}{1.470399in}}%
\pgfpathcurveto{\pgfqpoint{1.860621in}{1.462585in}}{\pgfqpoint{1.871220in}{1.458195in}}{\pgfqpoint{1.882270in}{1.458195in}}%
\pgfpathlineto{\pgfqpoint{1.882270in}{1.458195in}}%
\pgfpathclose%
\pgfusepath{stroke}%
\end{pgfscope}%
\begin{pgfscope}%
\pgfpathrectangle{\pgfqpoint{0.393053in}{0.375000in}}{\pgfqpoint{6.356833in}{5.175000in}}%
\pgfusepath{clip}%
\pgfsetbuttcap%
\pgfsetroundjoin%
\pgfsetlinewidth{1.003750pt}%
\definecolor{currentstroke}{rgb}{0.827451,0.827451,0.827451}%
\pgfsetstrokecolor{currentstroke}%
\pgfsetdash{}{0pt}%
\pgfpathmoveto{\pgfqpoint{2.477308in}{1.087549in}}%
\pgfpathcurveto{\pgfqpoint{2.488358in}{1.087549in}}{\pgfqpoint{2.498957in}{1.091940in}}{\pgfqpoint{2.506771in}{1.099753in}}%
\pgfpathcurveto{\pgfqpoint{2.514584in}{1.107567in}}{\pgfqpoint{2.518974in}{1.118166in}}{\pgfqpoint{2.518974in}{1.129216in}}%
\pgfpathcurveto{\pgfqpoint{2.518974in}{1.140266in}}{\pgfqpoint{2.514584in}{1.150865in}}{\pgfqpoint{2.506771in}{1.158679in}}%
\pgfpathcurveto{\pgfqpoint{2.498957in}{1.166492in}}{\pgfqpoint{2.488358in}{1.170883in}}{\pgfqpoint{2.477308in}{1.170883in}}%
\pgfpathcurveto{\pgfqpoint{2.466258in}{1.170883in}}{\pgfqpoint{2.455659in}{1.166492in}}{\pgfqpoint{2.447845in}{1.158679in}}%
\pgfpathcurveto{\pgfqpoint{2.440031in}{1.150865in}}{\pgfqpoint{2.435641in}{1.140266in}}{\pgfqpoint{2.435641in}{1.129216in}}%
\pgfpathcurveto{\pgfqpoint{2.435641in}{1.118166in}}{\pgfqpoint{2.440031in}{1.107567in}}{\pgfqpoint{2.447845in}{1.099753in}}%
\pgfpathcurveto{\pgfqpoint{2.455659in}{1.091940in}}{\pgfqpoint{2.466258in}{1.087549in}}{\pgfqpoint{2.477308in}{1.087549in}}%
\pgfpathlineto{\pgfqpoint{2.477308in}{1.087549in}}%
\pgfpathclose%
\pgfusepath{stroke}%
\end{pgfscope}%
\begin{pgfscope}%
\pgfpathrectangle{\pgfqpoint{0.393053in}{0.375000in}}{\pgfqpoint{6.356833in}{5.175000in}}%
\pgfusepath{clip}%
\pgfsetbuttcap%
\pgfsetroundjoin%
\pgfsetlinewidth{1.003750pt}%
\definecolor{currentstroke}{rgb}{0.827451,0.827451,0.827451}%
\pgfsetstrokecolor{currentstroke}%
\pgfsetdash{}{0pt}%
\pgfpathmoveto{\pgfqpoint{2.796329in}{0.903359in}}%
\pgfpathcurveto{\pgfqpoint{2.807379in}{0.903359in}}{\pgfqpoint{2.817978in}{0.907749in}}{\pgfqpoint{2.825792in}{0.915563in}}%
\pgfpathcurveto{\pgfqpoint{2.833605in}{0.923376in}}{\pgfqpoint{2.837995in}{0.933975in}}{\pgfqpoint{2.837995in}{0.945025in}}%
\pgfpathcurveto{\pgfqpoint{2.837995in}{0.956076in}}{\pgfqpoint{2.833605in}{0.966675in}}{\pgfqpoint{2.825792in}{0.974488in}}%
\pgfpathcurveto{\pgfqpoint{2.817978in}{0.982302in}}{\pgfqpoint{2.807379in}{0.986692in}}{\pgfqpoint{2.796329in}{0.986692in}}%
\pgfpathcurveto{\pgfqpoint{2.785279in}{0.986692in}}{\pgfqpoint{2.774680in}{0.982302in}}{\pgfqpoint{2.766866in}{0.974488in}}%
\pgfpathcurveto{\pgfqpoint{2.759052in}{0.966675in}}{\pgfqpoint{2.754662in}{0.956076in}}{\pgfqpoint{2.754662in}{0.945025in}}%
\pgfpathcurveto{\pgfqpoint{2.754662in}{0.933975in}}{\pgfqpoint{2.759052in}{0.923376in}}{\pgfqpoint{2.766866in}{0.915563in}}%
\pgfpathcurveto{\pgfqpoint{2.774680in}{0.907749in}}{\pgfqpoint{2.785279in}{0.903359in}}{\pgfqpoint{2.796329in}{0.903359in}}%
\pgfpathlineto{\pgfqpoint{2.796329in}{0.903359in}}%
\pgfpathclose%
\pgfusepath{stroke}%
\end{pgfscope}%
\begin{pgfscope}%
\pgfpathrectangle{\pgfqpoint{0.393053in}{0.375000in}}{\pgfqpoint{6.356833in}{5.175000in}}%
\pgfusepath{clip}%
\pgfsetbuttcap%
\pgfsetroundjoin%
\pgfsetlinewidth{1.003750pt}%
\definecolor{currentstroke}{rgb}{0.827451,0.827451,0.827451}%
\pgfsetstrokecolor{currentstroke}%
\pgfsetdash{}{0pt}%
\pgfpathmoveto{\pgfqpoint{1.614675in}{1.708400in}}%
\pgfpathcurveto{\pgfqpoint{1.625725in}{1.708400in}}{\pgfqpoint{1.636324in}{1.712790in}}{\pgfqpoint{1.644137in}{1.720604in}}%
\pgfpathcurveto{\pgfqpoint{1.651951in}{1.728417in}}{\pgfqpoint{1.656341in}{1.739016in}}{\pgfqpoint{1.656341in}{1.750066in}}%
\pgfpathcurveto{\pgfqpoint{1.656341in}{1.761117in}}{\pgfqpoint{1.651951in}{1.771716in}}{\pgfqpoint{1.644137in}{1.779529in}}%
\pgfpathcurveto{\pgfqpoint{1.636324in}{1.787343in}}{\pgfqpoint{1.625725in}{1.791733in}}{\pgfqpoint{1.614675in}{1.791733in}}%
\pgfpathcurveto{\pgfqpoint{1.603625in}{1.791733in}}{\pgfqpoint{1.593025in}{1.787343in}}{\pgfqpoint{1.585212in}{1.779529in}}%
\pgfpathcurveto{\pgfqpoint{1.577398in}{1.771716in}}{\pgfqpoint{1.573008in}{1.761117in}}{\pgfqpoint{1.573008in}{1.750066in}}%
\pgfpathcurveto{\pgfqpoint{1.573008in}{1.739016in}}{\pgfqpoint{1.577398in}{1.728417in}}{\pgfqpoint{1.585212in}{1.720604in}}%
\pgfpathcurveto{\pgfqpoint{1.593025in}{1.712790in}}{\pgfqpoint{1.603625in}{1.708400in}}{\pgfqpoint{1.614675in}{1.708400in}}%
\pgfpathlineto{\pgfqpoint{1.614675in}{1.708400in}}%
\pgfpathclose%
\pgfusepath{stroke}%
\end{pgfscope}%
\begin{pgfscope}%
\pgfpathrectangle{\pgfqpoint{0.393053in}{0.375000in}}{\pgfqpoint{6.356833in}{5.175000in}}%
\pgfusepath{clip}%
\pgfsetbuttcap%
\pgfsetroundjoin%
\pgfsetlinewidth{1.003750pt}%
\definecolor{currentstroke}{rgb}{0.827451,0.827451,0.827451}%
\pgfsetstrokecolor{currentstroke}%
\pgfsetdash{}{0pt}%
\pgfpathmoveto{\pgfqpoint{2.416040in}{1.117802in}}%
\pgfpathcurveto{\pgfqpoint{2.427090in}{1.117802in}}{\pgfqpoint{2.437689in}{1.122193in}}{\pgfqpoint{2.445503in}{1.130006in}}%
\pgfpathcurveto{\pgfqpoint{2.453317in}{1.137820in}}{\pgfqpoint{2.457707in}{1.148419in}}{\pgfqpoint{2.457707in}{1.159469in}}%
\pgfpathcurveto{\pgfqpoint{2.457707in}{1.170519in}}{\pgfqpoint{2.453317in}{1.181118in}}{\pgfqpoint{2.445503in}{1.188932in}}%
\pgfpathcurveto{\pgfqpoint{2.437689in}{1.196745in}}{\pgfqpoint{2.427090in}{1.201136in}}{\pgfqpoint{2.416040in}{1.201136in}}%
\pgfpathcurveto{\pgfqpoint{2.404990in}{1.201136in}}{\pgfqpoint{2.394391in}{1.196745in}}{\pgfqpoint{2.386577in}{1.188932in}}%
\pgfpathcurveto{\pgfqpoint{2.378764in}{1.181118in}}{\pgfqpoint{2.374373in}{1.170519in}}{\pgfqpoint{2.374373in}{1.159469in}}%
\pgfpathcurveto{\pgfqpoint{2.374373in}{1.148419in}}{\pgfqpoint{2.378764in}{1.137820in}}{\pgfqpoint{2.386577in}{1.130006in}}%
\pgfpathcurveto{\pgfqpoint{2.394391in}{1.122193in}}{\pgfqpoint{2.404990in}{1.117802in}}{\pgfqpoint{2.416040in}{1.117802in}}%
\pgfpathlineto{\pgfqpoint{2.416040in}{1.117802in}}%
\pgfpathclose%
\pgfusepath{stroke}%
\end{pgfscope}%
\begin{pgfscope}%
\pgfpathrectangle{\pgfqpoint{0.393053in}{0.375000in}}{\pgfqpoint{6.356833in}{5.175000in}}%
\pgfusepath{clip}%
\pgfsetbuttcap%
\pgfsetroundjoin%
\pgfsetlinewidth{1.003750pt}%
\definecolor{currentstroke}{rgb}{0.827451,0.827451,0.827451}%
\pgfsetstrokecolor{currentstroke}%
\pgfsetdash{}{0pt}%
\pgfpathmoveto{\pgfqpoint{0.580211in}{3.378166in}}%
\pgfpathcurveto{\pgfqpoint{0.591261in}{3.378166in}}{\pgfqpoint{0.601860in}{3.382556in}}{\pgfqpoint{0.609674in}{3.390370in}}%
\pgfpathcurveto{\pgfqpoint{0.617487in}{3.398183in}}{\pgfqpoint{0.621878in}{3.408782in}}{\pgfqpoint{0.621878in}{3.419833in}}%
\pgfpathcurveto{\pgfqpoint{0.621878in}{3.430883in}}{\pgfqpoint{0.617487in}{3.441482in}}{\pgfqpoint{0.609674in}{3.449295in}}%
\pgfpathcurveto{\pgfqpoint{0.601860in}{3.457109in}}{\pgfqpoint{0.591261in}{3.461499in}}{\pgfqpoint{0.580211in}{3.461499in}}%
\pgfpathcurveto{\pgfqpoint{0.569161in}{3.461499in}}{\pgfqpoint{0.558562in}{3.457109in}}{\pgfqpoint{0.550748in}{3.449295in}}%
\pgfpathcurveto{\pgfqpoint{0.542935in}{3.441482in}}{\pgfqpoint{0.538544in}{3.430883in}}{\pgfqpoint{0.538544in}{3.419833in}}%
\pgfpathcurveto{\pgfqpoint{0.538544in}{3.408782in}}{\pgfqpoint{0.542935in}{3.398183in}}{\pgfqpoint{0.550748in}{3.390370in}}%
\pgfpathcurveto{\pgfqpoint{0.558562in}{3.382556in}}{\pgfqpoint{0.569161in}{3.378166in}}{\pgfqpoint{0.580211in}{3.378166in}}%
\pgfpathlineto{\pgfqpoint{0.580211in}{3.378166in}}%
\pgfpathclose%
\pgfusepath{stroke}%
\end{pgfscope}%
\begin{pgfscope}%
\pgfpathrectangle{\pgfqpoint{0.393053in}{0.375000in}}{\pgfqpoint{6.356833in}{5.175000in}}%
\pgfusepath{clip}%
\pgfsetbuttcap%
\pgfsetroundjoin%
\pgfsetlinewidth{1.003750pt}%
\definecolor{currentstroke}{rgb}{0.827451,0.827451,0.827451}%
\pgfsetstrokecolor{currentstroke}%
\pgfsetdash{}{0pt}%
\pgfpathmoveto{\pgfqpoint{0.707419in}{2.963452in}}%
\pgfpathcurveto{\pgfqpoint{0.718469in}{2.963452in}}{\pgfqpoint{0.729068in}{2.967843in}}{\pgfqpoint{0.736882in}{2.975656in}}%
\pgfpathcurveto{\pgfqpoint{0.744695in}{2.983470in}}{\pgfqpoint{0.749086in}{2.994069in}}{\pgfqpoint{0.749086in}{3.005119in}}%
\pgfpathcurveto{\pgfqpoint{0.749086in}{3.016169in}}{\pgfqpoint{0.744695in}{3.026768in}}{\pgfqpoint{0.736882in}{3.034582in}}%
\pgfpathcurveto{\pgfqpoint{0.729068in}{3.042395in}}{\pgfqpoint{0.718469in}{3.046786in}}{\pgfqpoint{0.707419in}{3.046786in}}%
\pgfpathcurveto{\pgfqpoint{0.696369in}{3.046786in}}{\pgfqpoint{0.685770in}{3.042395in}}{\pgfqpoint{0.677956in}{3.034582in}}%
\pgfpathcurveto{\pgfqpoint{0.670142in}{3.026768in}}{\pgfqpoint{0.665752in}{3.016169in}}{\pgfqpoint{0.665752in}{3.005119in}}%
\pgfpathcurveto{\pgfqpoint{0.665752in}{2.994069in}}{\pgfqpoint{0.670142in}{2.983470in}}{\pgfqpoint{0.677956in}{2.975656in}}%
\pgfpathcurveto{\pgfqpoint{0.685770in}{2.967843in}}{\pgfqpoint{0.696369in}{2.963452in}}{\pgfqpoint{0.707419in}{2.963452in}}%
\pgfpathlineto{\pgfqpoint{0.707419in}{2.963452in}}%
\pgfpathclose%
\pgfusepath{stroke}%
\end{pgfscope}%
\begin{pgfscope}%
\pgfpathrectangle{\pgfqpoint{0.393053in}{0.375000in}}{\pgfqpoint{6.356833in}{5.175000in}}%
\pgfusepath{clip}%
\pgfsetbuttcap%
\pgfsetroundjoin%
\pgfsetlinewidth{1.003750pt}%
\definecolor{currentstroke}{rgb}{0.827451,0.827451,0.827451}%
\pgfsetstrokecolor{currentstroke}%
\pgfsetdash{}{0pt}%
\pgfpathmoveto{\pgfqpoint{1.291595in}{1.999205in}}%
\pgfpathcurveto{\pgfqpoint{1.302646in}{1.999205in}}{\pgfqpoint{1.313245in}{2.003595in}}{\pgfqpoint{1.321058in}{2.011409in}}%
\pgfpathcurveto{\pgfqpoint{1.328872in}{2.019222in}}{\pgfqpoint{1.333262in}{2.029821in}}{\pgfqpoint{1.333262in}{2.040872in}}%
\pgfpathcurveto{\pgfqpoint{1.333262in}{2.051922in}}{\pgfqpoint{1.328872in}{2.062521in}}{\pgfqpoint{1.321058in}{2.070334in}}%
\pgfpathcurveto{\pgfqpoint{1.313245in}{2.078148in}}{\pgfqpoint{1.302646in}{2.082538in}}{\pgfqpoint{1.291595in}{2.082538in}}%
\pgfpathcurveto{\pgfqpoint{1.280545in}{2.082538in}}{\pgfqpoint{1.269946in}{2.078148in}}{\pgfqpoint{1.262133in}{2.070334in}}%
\pgfpathcurveto{\pgfqpoint{1.254319in}{2.062521in}}{\pgfqpoint{1.249929in}{2.051922in}}{\pgfqpoint{1.249929in}{2.040872in}}%
\pgfpathcurveto{\pgfqpoint{1.249929in}{2.029821in}}{\pgfqpoint{1.254319in}{2.019222in}}{\pgfqpoint{1.262133in}{2.011409in}}%
\pgfpathcurveto{\pgfqpoint{1.269946in}{2.003595in}}{\pgfqpoint{1.280545in}{1.999205in}}{\pgfqpoint{1.291595in}{1.999205in}}%
\pgfpathlineto{\pgfqpoint{1.291595in}{1.999205in}}%
\pgfpathclose%
\pgfusepath{stroke}%
\end{pgfscope}%
\begin{pgfscope}%
\pgfpathrectangle{\pgfqpoint{0.393053in}{0.375000in}}{\pgfqpoint{6.356833in}{5.175000in}}%
\pgfusepath{clip}%
\pgfsetbuttcap%
\pgfsetroundjoin%
\pgfsetlinewidth{1.003750pt}%
\definecolor{currentstroke}{rgb}{0.827451,0.827451,0.827451}%
\pgfsetstrokecolor{currentstroke}%
\pgfsetdash{}{0pt}%
\pgfpathmoveto{\pgfqpoint{4.122151in}{0.465889in}}%
\pgfpathcurveto{\pgfqpoint{4.133202in}{0.465889in}}{\pgfqpoint{4.143801in}{0.470279in}}{\pgfqpoint{4.151614in}{0.478093in}}%
\pgfpathcurveto{\pgfqpoint{4.159428in}{0.485907in}}{\pgfqpoint{4.163818in}{0.496506in}}{\pgfqpoint{4.163818in}{0.507556in}}%
\pgfpathcurveto{\pgfqpoint{4.163818in}{0.518606in}}{\pgfqpoint{4.159428in}{0.529205in}}{\pgfqpoint{4.151614in}{0.537019in}}%
\pgfpathcurveto{\pgfqpoint{4.143801in}{0.544832in}}{\pgfqpoint{4.133202in}{0.549222in}}{\pgfqpoint{4.122151in}{0.549222in}}%
\pgfpathcurveto{\pgfqpoint{4.111101in}{0.549222in}}{\pgfqpoint{4.100502in}{0.544832in}}{\pgfqpoint{4.092689in}{0.537019in}}%
\pgfpathcurveto{\pgfqpoint{4.084875in}{0.529205in}}{\pgfqpoint{4.080485in}{0.518606in}}{\pgfqpoint{4.080485in}{0.507556in}}%
\pgfpathcurveto{\pgfqpoint{4.080485in}{0.496506in}}{\pgfqpoint{4.084875in}{0.485907in}}{\pgfqpoint{4.092689in}{0.478093in}}%
\pgfpathcurveto{\pgfqpoint{4.100502in}{0.470279in}}{\pgfqpoint{4.111101in}{0.465889in}}{\pgfqpoint{4.122151in}{0.465889in}}%
\pgfpathlineto{\pgfqpoint{4.122151in}{0.465889in}}%
\pgfpathclose%
\pgfusepath{stroke}%
\end{pgfscope}%
\begin{pgfscope}%
\pgfpathrectangle{\pgfqpoint{0.393053in}{0.375000in}}{\pgfqpoint{6.356833in}{5.175000in}}%
\pgfusepath{clip}%
\pgfsetbuttcap%
\pgfsetroundjoin%
\pgfsetlinewidth{1.003750pt}%
\definecolor{currentstroke}{rgb}{0.827451,0.827451,0.827451}%
\pgfsetstrokecolor{currentstroke}%
\pgfsetdash{}{0pt}%
\pgfpathmoveto{\pgfqpoint{0.393415in}{4.523360in}}%
\pgfpathcurveto{\pgfqpoint{0.404465in}{4.523360in}}{\pgfqpoint{0.415064in}{4.527750in}}{\pgfqpoint{0.422878in}{4.535564in}}%
\pgfpathcurveto{\pgfqpoint{0.430691in}{4.543377in}}{\pgfqpoint{0.435082in}{4.553976in}}{\pgfqpoint{0.435082in}{4.565026in}}%
\pgfpathcurveto{\pgfqpoint{0.435082in}{4.576077in}}{\pgfqpoint{0.430691in}{4.586676in}}{\pgfqpoint{0.422878in}{4.594489in}}%
\pgfpathcurveto{\pgfqpoint{0.415064in}{4.602303in}}{\pgfqpoint{0.404465in}{4.606693in}}{\pgfqpoint{0.393415in}{4.606693in}}%
\pgfpathcurveto{\pgfqpoint{0.382365in}{4.606693in}}{\pgfqpoint{0.371766in}{4.602303in}}{\pgfqpoint{0.363952in}{4.594489in}}%
\pgfpathcurveto{\pgfqpoint{0.356139in}{4.586676in}}{\pgfqpoint{0.351748in}{4.576077in}}{\pgfqpoint{0.351748in}{4.565026in}}%
\pgfpathcurveto{\pgfqpoint{0.351748in}{4.553976in}}{\pgfqpoint{0.356139in}{4.543377in}}{\pgfqpoint{0.363952in}{4.535564in}}%
\pgfpathcurveto{\pgfqpoint{0.371766in}{4.527750in}}{\pgfqpoint{0.382365in}{4.523360in}}{\pgfqpoint{0.393415in}{4.523360in}}%
\pgfpathlineto{\pgfqpoint{0.393415in}{4.523360in}}%
\pgfpathclose%
\pgfusepath{stroke}%
\end{pgfscope}%
\begin{pgfscope}%
\pgfpathrectangle{\pgfqpoint{0.393053in}{0.375000in}}{\pgfqpoint{6.356833in}{5.175000in}}%
\pgfusepath{clip}%
\pgfsetbuttcap%
\pgfsetroundjoin%
\pgfsetlinewidth{1.003750pt}%
\definecolor{currentstroke}{rgb}{0.827451,0.827451,0.827451}%
\pgfsetstrokecolor{currentstroke}%
\pgfsetdash{}{0pt}%
\pgfpathmoveto{\pgfqpoint{5.707928in}{0.337346in}}%
\pgfpathcurveto{\pgfqpoint{5.718979in}{0.337346in}}{\pgfqpoint{5.729578in}{0.341736in}}{\pgfqpoint{5.737391in}{0.349550in}}%
\pgfpathcurveto{\pgfqpoint{5.745205in}{0.357363in}}{\pgfqpoint{5.749595in}{0.367962in}}{\pgfqpoint{5.749595in}{0.379012in}}%
\pgfpathcurveto{\pgfqpoint{5.749595in}{0.390063in}}{\pgfqpoint{5.745205in}{0.400662in}}{\pgfqpoint{5.737391in}{0.408475in}}%
\pgfpathcurveto{\pgfqpoint{5.729578in}{0.416289in}}{\pgfqpoint{5.718979in}{0.420679in}}{\pgfqpoint{5.707928in}{0.420679in}}%
\pgfpathcurveto{\pgfqpoint{5.696878in}{0.420679in}}{\pgfqpoint{5.686279in}{0.416289in}}{\pgfqpoint{5.678466in}{0.408475in}}%
\pgfpathcurveto{\pgfqpoint{5.670652in}{0.400662in}}{\pgfqpoint{5.666262in}{0.390063in}}{\pgfqpoint{5.666262in}{0.379012in}}%
\pgfpathcurveto{\pgfqpoint{5.666262in}{0.367962in}}{\pgfqpoint{5.670652in}{0.357363in}}{\pgfqpoint{5.678466in}{0.349550in}}%
\pgfpathcurveto{\pgfqpoint{5.686279in}{0.341736in}}{\pgfqpoint{5.696878in}{0.337346in}}{\pgfqpoint{5.707928in}{0.337346in}}%
\pgfusepath{stroke}%
\end{pgfscope}%
\begin{pgfscope}%
\pgfpathrectangle{\pgfqpoint{0.393053in}{0.375000in}}{\pgfqpoint{6.356833in}{5.175000in}}%
\pgfusepath{clip}%
\pgfsetbuttcap%
\pgfsetroundjoin%
\pgfsetlinewidth{1.003750pt}%
\definecolor{currentstroke}{rgb}{0.827451,0.827451,0.827451}%
\pgfsetstrokecolor{currentstroke}%
\pgfsetdash{}{0pt}%
\pgfpathmoveto{\pgfqpoint{5.524477in}{0.341503in}}%
\pgfpathcurveto{\pgfqpoint{5.535528in}{0.341503in}}{\pgfqpoint{5.546127in}{0.345894in}}{\pgfqpoint{5.553940in}{0.353707in}}%
\pgfpathcurveto{\pgfqpoint{5.561754in}{0.361521in}}{\pgfqpoint{5.566144in}{0.372120in}}{\pgfqpoint{5.566144in}{0.383170in}}%
\pgfpathcurveto{\pgfqpoint{5.566144in}{0.394220in}}{\pgfqpoint{5.561754in}{0.404819in}}{\pgfqpoint{5.553940in}{0.412633in}}%
\pgfpathcurveto{\pgfqpoint{5.546127in}{0.420446in}}{\pgfqpoint{5.535528in}{0.424837in}}{\pgfqpoint{5.524477in}{0.424837in}}%
\pgfpathcurveto{\pgfqpoint{5.513427in}{0.424837in}}{\pgfqpoint{5.502828in}{0.420446in}}{\pgfqpoint{5.495015in}{0.412633in}}%
\pgfpathcurveto{\pgfqpoint{5.487201in}{0.404819in}}{\pgfqpoint{5.482811in}{0.394220in}}{\pgfqpoint{5.482811in}{0.383170in}}%
\pgfpathcurveto{\pgfqpoint{5.482811in}{0.372120in}}{\pgfqpoint{5.487201in}{0.361521in}}{\pgfqpoint{5.495015in}{0.353707in}}%
\pgfpathcurveto{\pgfqpoint{5.502828in}{0.345894in}}{\pgfqpoint{5.513427in}{0.341503in}}{\pgfqpoint{5.524477in}{0.341503in}}%
\pgfusepath{stroke}%
\end{pgfscope}%
\begin{pgfscope}%
\pgfpathrectangle{\pgfqpoint{0.393053in}{0.375000in}}{\pgfqpoint{6.356833in}{5.175000in}}%
\pgfusepath{clip}%
\pgfsetbuttcap%
\pgfsetroundjoin%
\pgfsetlinewidth{1.003750pt}%
\definecolor{currentstroke}{rgb}{0.827451,0.827451,0.827451}%
\pgfsetstrokecolor{currentstroke}%
\pgfsetdash{}{0pt}%
\pgfpathmoveto{\pgfqpoint{5.142473in}{0.360415in}}%
\pgfpathcurveto{\pgfqpoint{5.153523in}{0.360415in}}{\pgfqpoint{5.164122in}{0.364805in}}{\pgfqpoint{5.171936in}{0.372618in}}%
\pgfpathcurveto{\pgfqpoint{5.179749in}{0.380432in}}{\pgfqpoint{5.184140in}{0.391031in}}{\pgfqpoint{5.184140in}{0.402081in}}%
\pgfpathcurveto{\pgfqpoint{5.184140in}{0.413131in}}{\pgfqpoint{5.179749in}{0.423730in}}{\pgfqpoint{5.171936in}{0.431544in}}%
\pgfpathcurveto{\pgfqpoint{5.164122in}{0.439358in}}{\pgfqpoint{5.153523in}{0.443748in}}{\pgfqpoint{5.142473in}{0.443748in}}%
\pgfpathcurveto{\pgfqpoint{5.131423in}{0.443748in}}{\pgfqpoint{5.120824in}{0.439358in}}{\pgfqpoint{5.113010in}{0.431544in}}%
\pgfpathcurveto{\pgfqpoint{5.105196in}{0.423730in}}{\pgfqpoint{5.100806in}{0.413131in}}{\pgfqpoint{5.100806in}{0.402081in}}%
\pgfpathcurveto{\pgfqpoint{5.100806in}{0.391031in}}{\pgfqpoint{5.105196in}{0.380432in}}{\pgfqpoint{5.113010in}{0.372618in}}%
\pgfpathcurveto{\pgfqpoint{5.120824in}{0.364805in}}{\pgfqpoint{5.131423in}{0.360415in}}{\pgfqpoint{5.142473in}{0.360415in}}%
\pgfusepath{stroke}%
\end{pgfscope}%
\begin{pgfscope}%
\pgfpathrectangle{\pgfqpoint{0.393053in}{0.375000in}}{\pgfqpoint{6.356833in}{5.175000in}}%
\pgfusepath{clip}%
\pgfsetbuttcap%
\pgfsetroundjoin%
\pgfsetlinewidth{1.003750pt}%
\definecolor{currentstroke}{rgb}{0.827451,0.827451,0.827451}%
\pgfsetstrokecolor{currentstroke}%
\pgfsetdash{}{0pt}%
\pgfpathmoveto{\pgfqpoint{3.464896in}{0.638283in}}%
\pgfpathcurveto{\pgfqpoint{3.475946in}{0.638283in}}{\pgfqpoint{3.486545in}{0.642673in}}{\pgfqpoint{3.494359in}{0.650486in}}%
\pgfpathcurveto{\pgfqpoint{3.502173in}{0.658300in}}{\pgfqpoint{3.506563in}{0.668899in}}{\pgfqpoint{3.506563in}{0.679949in}}%
\pgfpathcurveto{\pgfqpoint{3.506563in}{0.690999in}}{\pgfqpoint{3.502173in}{0.701598in}}{\pgfqpoint{3.494359in}{0.709412in}}%
\pgfpathcurveto{\pgfqpoint{3.486545in}{0.717226in}}{\pgfqpoint{3.475946in}{0.721616in}}{\pgfqpoint{3.464896in}{0.721616in}}%
\pgfpathcurveto{\pgfqpoint{3.453846in}{0.721616in}}{\pgfqpoint{3.443247in}{0.717226in}}{\pgfqpoint{3.435433in}{0.709412in}}%
\pgfpathcurveto{\pgfqpoint{3.427620in}{0.701598in}}{\pgfqpoint{3.423230in}{0.690999in}}{\pgfqpoint{3.423230in}{0.679949in}}%
\pgfpathcurveto{\pgfqpoint{3.423230in}{0.668899in}}{\pgfqpoint{3.427620in}{0.658300in}}{\pgfqpoint{3.435433in}{0.650486in}}%
\pgfpathcurveto{\pgfqpoint{3.443247in}{0.642673in}}{\pgfqpoint{3.453846in}{0.638283in}}{\pgfqpoint{3.464896in}{0.638283in}}%
\pgfpathlineto{\pgfqpoint{3.464896in}{0.638283in}}%
\pgfpathclose%
\pgfusepath{stroke}%
\end{pgfscope}%
\begin{pgfscope}%
\pgfpathrectangle{\pgfqpoint{0.393053in}{0.375000in}}{\pgfqpoint{6.356833in}{5.175000in}}%
\pgfusepath{clip}%
\pgfsetbuttcap%
\pgfsetroundjoin%
\pgfsetlinewidth{1.003750pt}%
\definecolor{currentstroke}{rgb}{0.827451,0.827451,0.827451}%
\pgfsetstrokecolor{currentstroke}%
\pgfsetdash{}{0pt}%
\pgfpathmoveto{\pgfqpoint{4.659727in}{0.389065in}}%
\pgfpathcurveto{\pgfqpoint{4.670777in}{0.389065in}}{\pgfqpoint{4.681376in}{0.393455in}}{\pgfqpoint{4.689190in}{0.401269in}}%
\pgfpathcurveto{\pgfqpoint{4.697004in}{0.409082in}}{\pgfqpoint{4.701394in}{0.419682in}}{\pgfqpoint{4.701394in}{0.430732in}}%
\pgfpathcurveto{\pgfqpoint{4.701394in}{0.441782in}}{\pgfqpoint{4.697004in}{0.452381in}}{\pgfqpoint{4.689190in}{0.460194in}}%
\pgfpathcurveto{\pgfqpoint{4.681376in}{0.468008in}}{\pgfqpoint{4.670777in}{0.472398in}}{\pgfqpoint{4.659727in}{0.472398in}}%
\pgfpathcurveto{\pgfqpoint{4.648677in}{0.472398in}}{\pgfqpoint{4.638078in}{0.468008in}}{\pgfqpoint{4.630264in}{0.460194in}}%
\pgfpathcurveto{\pgfqpoint{4.622451in}{0.452381in}}{\pgfqpoint{4.618060in}{0.441782in}}{\pgfqpoint{4.618060in}{0.430732in}}%
\pgfpathcurveto{\pgfqpoint{4.618060in}{0.419682in}}{\pgfqpoint{4.622451in}{0.409082in}}{\pgfqpoint{4.630264in}{0.401269in}}%
\pgfpathcurveto{\pgfqpoint{4.638078in}{0.393455in}}{\pgfqpoint{4.648677in}{0.389065in}}{\pgfqpoint{4.659727in}{0.389065in}}%
\pgfpathlineto{\pgfqpoint{4.659727in}{0.389065in}}%
\pgfpathclose%
\pgfusepath{stroke}%
\end{pgfscope}%
\begin{pgfscope}%
\pgfpathrectangle{\pgfqpoint{0.393053in}{0.375000in}}{\pgfqpoint{6.356833in}{5.175000in}}%
\pgfusepath{clip}%
\pgfsetbuttcap%
\pgfsetroundjoin%
\pgfsetlinewidth{1.003750pt}%
\definecolor{currentstroke}{rgb}{0.827451,0.827451,0.827451}%
\pgfsetstrokecolor{currentstroke}%
\pgfsetdash{}{0pt}%
\pgfpathmoveto{\pgfqpoint{4.927239in}{0.379634in}}%
\pgfpathcurveto{\pgfqpoint{4.938289in}{0.379634in}}{\pgfqpoint{4.948888in}{0.384025in}}{\pgfqpoint{4.956702in}{0.391838in}}%
\pgfpathcurveto{\pgfqpoint{4.964516in}{0.399652in}}{\pgfqpoint{4.968906in}{0.410251in}}{\pgfqpoint{4.968906in}{0.421301in}}%
\pgfpathcurveto{\pgfqpoint{4.968906in}{0.432351in}}{\pgfqpoint{4.964516in}{0.442950in}}{\pgfqpoint{4.956702in}{0.450764in}}%
\pgfpathcurveto{\pgfqpoint{4.948888in}{0.458578in}}{\pgfqpoint{4.938289in}{0.462968in}}{\pgfqpoint{4.927239in}{0.462968in}}%
\pgfpathcurveto{\pgfqpoint{4.916189in}{0.462968in}}{\pgfqpoint{4.905590in}{0.458578in}}{\pgfqpoint{4.897776in}{0.450764in}}%
\pgfpathcurveto{\pgfqpoint{4.889963in}{0.442950in}}{\pgfqpoint{4.885572in}{0.432351in}}{\pgfqpoint{4.885572in}{0.421301in}}%
\pgfpathcurveto{\pgfqpoint{4.885572in}{0.410251in}}{\pgfqpoint{4.889963in}{0.399652in}}{\pgfqpoint{4.897776in}{0.391838in}}%
\pgfpathcurveto{\pgfqpoint{4.905590in}{0.384025in}}{\pgfqpoint{4.916189in}{0.379634in}}{\pgfqpoint{4.927239in}{0.379634in}}%
\pgfpathlineto{\pgfqpoint{4.927239in}{0.379634in}}%
\pgfpathclose%
\pgfusepath{stroke}%
\end{pgfscope}%
\begin{pgfscope}%
\pgfpathrectangle{\pgfqpoint{0.393053in}{0.375000in}}{\pgfqpoint{6.356833in}{5.175000in}}%
\pgfusepath{clip}%
\pgfsetbuttcap%
\pgfsetroundjoin%
\pgfsetlinewidth{1.003750pt}%
\definecolor{currentstroke}{rgb}{0.827451,0.827451,0.827451}%
\pgfsetstrokecolor{currentstroke}%
\pgfsetdash{}{0pt}%
\pgfpathmoveto{\pgfqpoint{3.637654in}{0.581771in}}%
\pgfpathcurveto{\pgfqpoint{3.648705in}{0.581771in}}{\pgfqpoint{3.659304in}{0.586161in}}{\pgfqpoint{3.667117in}{0.593974in}}%
\pgfpathcurveto{\pgfqpoint{3.674931in}{0.601788in}}{\pgfqpoint{3.679321in}{0.612387in}}{\pgfqpoint{3.679321in}{0.623437in}}%
\pgfpathcurveto{\pgfqpoint{3.679321in}{0.634487in}}{\pgfqpoint{3.674931in}{0.645086in}}{\pgfqpoint{3.667117in}{0.652900in}}%
\pgfpathcurveto{\pgfqpoint{3.659304in}{0.660714in}}{\pgfqpoint{3.648705in}{0.665104in}}{\pgfqpoint{3.637654in}{0.665104in}}%
\pgfpathcurveto{\pgfqpoint{3.626604in}{0.665104in}}{\pgfqpoint{3.616005in}{0.660714in}}{\pgfqpoint{3.608192in}{0.652900in}}%
\pgfpathcurveto{\pgfqpoint{3.600378in}{0.645086in}}{\pgfqpoint{3.595988in}{0.634487in}}{\pgfqpoint{3.595988in}{0.623437in}}%
\pgfpathcurveto{\pgfqpoint{3.595988in}{0.612387in}}{\pgfqpoint{3.600378in}{0.601788in}}{\pgfqpoint{3.608192in}{0.593974in}}%
\pgfpathcurveto{\pgfqpoint{3.616005in}{0.586161in}}{\pgfqpoint{3.626604in}{0.581771in}}{\pgfqpoint{3.637654in}{0.581771in}}%
\pgfpathlineto{\pgfqpoint{3.637654in}{0.581771in}}%
\pgfpathclose%
\pgfusepath{stroke}%
\end{pgfscope}%
\begin{pgfscope}%
\pgfpathrectangle{\pgfqpoint{0.393053in}{0.375000in}}{\pgfqpoint{6.356833in}{5.175000in}}%
\pgfusepath{clip}%
\pgfsetbuttcap%
\pgfsetroundjoin%
\pgfsetlinewidth{1.003750pt}%
\definecolor{currentstroke}{rgb}{0.827451,0.827451,0.827451}%
\pgfsetstrokecolor{currentstroke}%
\pgfsetdash{}{0pt}%
\pgfpathmoveto{\pgfqpoint{0.580211in}{3.378166in}}%
\pgfpathcurveto{\pgfqpoint{0.591261in}{3.378166in}}{\pgfqpoint{0.601860in}{3.382556in}}{\pgfqpoint{0.609674in}{3.390370in}}%
\pgfpathcurveto{\pgfqpoint{0.617487in}{3.398183in}}{\pgfqpoint{0.621878in}{3.408782in}}{\pgfqpoint{0.621878in}{3.419833in}}%
\pgfpathcurveto{\pgfqpoint{0.621878in}{3.430883in}}{\pgfqpoint{0.617487in}{3.441482in}}{\pgfqpoint{0.609674in}{3.449295in}}%
\pgfpathcurveto{\pgfqpoint{0.601860in}{3.457109in}}{\pgfqpoint{0.591261in}{3.461499in}}{\pgfqpoint{0.580211in}{3.461499in}}%
\pgfpathcurveto{\pgfqpoint{0.569161in}{3.461499in}}{\pgfqpoint{0.558562in}{3.457109in}}{\pgfqpoint{0.550748in}{3.449295in}}%
\pgfpathcurveto{\pgfqpoint{0.542935in}{3.441482in}}{\pgfqpoint{0.538544in}{3.430883in}}{\pgfqpoint{0.538544in}{3.419833in}}%
\pgfpathcurveto{\pgfqpoint{0.538544in}{3.408782in}}{\pgfqpoint{0.542935in}{3.398183in}}{\pgfqpoint{0.550748in}{3.390370in}}%
\pgfpathcurveto{\pgfqpoint{0.558562in}{3.382556in}}{\pgfqpoint{0.569161in}{3.378166in}}{\pgfqpoint{0.580211in}{3.378166in}}%
\pgfpathlineto{\pgfqpoint{0.580211in}{3.378166in}}%
\pgfpathclose%
\pgfusepath{stroke}%
\end{pgfscope}%
\begin{pgfscope}%
\pgfpathrectangle{\pgfqpoint{0.393053in}{0.375000in}}{\pgfqpoint{6.356833in}{5.175000in}}%
\pgfusepath{clip}%
\pgfsetbuttcap%
\pgfsetroundjoin%
\pgfsetlinewidth{1.003750pt}%
\definecolor{currentstroke}{rgb}{0.827451,0.827451,0.827451}%
\pgfsetstrokecolor{currentstroke}%
\pgfsetdash{}{0pt}%
\pgfpathmoveto{\pgfqpoint{2.287965in}{1.196822in}}%
\pgfpathcurveto{\pgfqpoint{2.299015in}{1.196822in}}{\pgfqpoint{2.309614in}{1.201212in}}{\pgfqpoint{2.317427in}{1.209026in}}%
\pgfpathcurveto{\pgfqpoint{2.325241in}{1.216839in}}{\pgfqpoint{2.329631in}{1.227438in}}{\pgfqpoint{2.329631in}{1.238488in}}%
\pgfpathcurveto{\pgfqpoint{2.329631in}{1.249539in}}{\pgfqpoint{2.325241in}{1.260138in}}{\pgfqpoint{2.317427in}{1.267951in}}%
\pgfpathcurveto{\pgfqpoint{2.309614in}{1.275765in}}{\pgfqpoint{2.299015in}{1.280155in}}{\pgfqpoint{2.287965in}{1.280155in}}%
\pgfpathcurveto{\pgfqpoint{2.276915in}{1.280155in}}{\pgfqpoint{2.266316in}{1.275765in}}{\pgfqpoint{2.258502in}{1.267951in}}%
\pgfpathcurveto{\pgfqpoint{2.250688in}{1.260138in}}{\pgfqpoint{2.246298in}{1.249539in}}{\pgfqpoint{2.246298in}{1.238488in}}%
\pgfpathcurveto{\pgfqpoint{2.246298in}{1.227438in}}{\pgfqpoint{2.250688in}{1.216839in}}{\pgfqpoint{2.258502in}{1.209026in}}%
\pgfpathcurveto{\pgfqpoint{2.266316in}{1.201212in}}{\pgfqpoint{2.276915in}{1.196822in}}{\pgfqpoint{2.287965in}{1.196822in}}%
\pgfpathlineto{\pgfqpoint{2.287965in}{1.196822in}}%
\pgfpathclose%
\pgfusepath{stroke}%
\end{pgfscope}%
\begin{pgfscope}%
\pgfpathrectangle{\pgfqpoint{0.393053in}{0.375000in}}{\pgfqpoint{6.356833in}{5.175000in}}%
\pgfusepath{clip}%
\pgfsetbuttcap%
\pgfsetroundjoin%
\pgfsetlinewidth{1.003750pt}%
\definecolor{currentstroke}{rgb}{0.827451,0.827451,0.827451}%
\pgfsetstrokecolor{currentstroke}%
\pgfsetdash{}{0pt}%
\pgfpathmoveto{\pgfqpoint{4.445943in}{0.425176in}}%
\pgfpathcurveto{\pgfqpoint{4.456993in}{0.425176in}}{\pgfqpoint{4.467592in}{0.429567in}}{\pgfqpoint{4.475406in}{0.437380in}}%
\pgfpathcurveto{\pgfqpoint{4.483219in}{0.445194in}}{\pgfqpoint{4.487610in}{0.455793in}}{\pgfqpoint{4.487610in}{0.466843in}}%
\pgfpathcurveto{\pgfqpoint{4.487610in}{0.477893in}}{\pgfqpoint{4.483219in}{0.488492in}}{\pgfqpoint{4.475406in}{0.496306in}}%
\pgfpathcurveto{\pgfqpoint{4.467592in}{0.504119in}}{\pgfqpoint{4.456993in}{0.508510in}}{\pgfqpoint{4.445943in}{0.508510in}}%
\pgfpathcurveto{\pgfqpoint{4.434893in}{0.508510in}}{\pgfqpoint{4.424294in}{0.504119in}}{\pgfqpoint{4.416480in}{0.496306in}}%
\pgfpathcurveto{\pgfqpoint{4.408667in}{0.488492in}}{\pgfqpoint{4.404276in}{0.477893in}}{\pgfqpoint{4.404276in}{0.466843in}}%
\pgfpathcurveto{\pgfqpoint{4.404276in}{0.455793in}}{\pgfqpoint{4.408667in}{0.445194in}}{\pgfqpoint{4.416480in}{0.437380in}}%
\pgfpathcurveto{\pgfqpoint{4.424294in}{0.429567in}}{\pgfqpoint{4.434893in}{0.425176in}}{\pgfqpoint{4.445943in}{0.425176in}}%
\pgfpathlineto{\pgfqpoint{4.445943in}{0.425176in}}%
\pgfpathclose%
\pgfusepath{stroke}%
\end{pgfscope}%
\begin{pgfscope}%
\pgfpathrectangle{\pgfqpoint{0.393053in}{0.375000in}}{\pgfqpoint{6.356833in}{5.175000in}}%
\pgfusepath{clip}%
\pgfsetbuttcap%
\pgfsetroundjoin%
\pgfsetlinewidth{1.003750pt}%
\definecolor{currentstroke}{rgb}{0.827451,0.827451,0.827451}%
\pgfsetstrokecolor{currentstroke}%
\pgfsetdash{}{0pt}%
\pgfpathmoveto{\pgfqpoint{0.525326in}{3.470591in}}%
\pgfpathcurveto{\pgfqpoint{0.536376in}{3.470591in}}{\pgfqpoint{0.546975in}{3.474982in}}{\pgfqpoint{0.554788in}{3.482795in}}%
\pgfpathcurveto{\pgfqpoint{0.562602in}{3.490609in}}{\pgfqpoint{0.566992in}{3.501208in}}{\pgfqpoint{0.566992in}{3.512258in}}%
\pgfpathcurveto{\pgfqpoint{0.566992in}{3.523308in}}{\pgfqpoint{0.562602in}{3.533907in}}{\pgfqpoint{0.554788in}{3.541721in}}%
\pgfpathcurveto{\pgfqpoint{0.546975in}{3.549534in}}{\pgfqpoint{0.536376in}{3.553925in}}{\pgfqpoint{0.525326in}{3.553925in}}%
\pgfpathcurveto{\pgfqpoint{0.514275in}{3.553925in}}{\pgfqpoint{0.503676in}{3.549534in}}{\pgfqpoint{0.495863in}{3.541721in}}%
\pgfpathcurveto{\pgfqpoint{0.488049in}{3.533907in}}{\pgfqpoint{0.483659in}{3.523308in}}{\pgfqpoint{0.483659in}{3.512258in}}%
\pgfpathcurveto{\pgfqpoint{0.483659in}{3.501208in}}{\pgfqpoint{0.488049in}{3.490609in}}{\pgfqpoint{0.495863in}{3.482795in}}%
\pgfpathcurveto{\pgfqpoint{0.503676in}{3.474982in}}{\pgfqpoint{0.514275in}{3.470591in}}{\pgfqpoint{0.525326in}{3.470591in}}%
\pgfpathlineto{\pgfqpoint{0.525326in}{3.470591in}}%
\pgfpathclose%
\pgfusepath{stroke}%
\end{pgfscope}%
\begin{pgfscope}%
\pgfpathrectangle{\pgfqpoint{0.393053in}{0.375000in}}{\pgfqpoint{6.356833in}{5.175000in}}%
\pgfusepath{clip}%
\pgfsetbuttcap%
\pgfsetroundjoin%
\pgfsetlinewidth{1.003750pt}%
\definecolor{currentstroke}{rgb}{0.827451,0.827451,0.827451}%
\pgfsetstrokecolor{currentstroke}%
\pgfsetdash{}{0pt}%
\pgfpathmoveto{\pgfqpoint{1.768437in}{1.558678in}}%
\pgfpathcurveto{\pgfqpoint{1.779488in}{1.558678in}}{\pgfqpoint{1.790087in}{1.563068in}}{\pgfqpoint{1.797900in}{1.570882in}}%
\pgfpathcurveto{\pgfqpoint{1.805714in}{1.578696in}}{\pgfqpoint{1.810104in}{1.589295in}}{\pgfqpoint{1.810104in}{1.600345in}}%
\pgfpathcurveto{\pgfqpoint{1.810104in}{1.611395in}}{\pgfqpoint{1.805714in}{1.621994in}}{\pgfqpoint{1.797900in}{1.629807in}}%
\pgfpathcurveto{\pgfqpoint{1.790087in}{1.637621in}}{\pgfqpoint{1.779488in}{1.642011in}}{\pgfqpoint{1.768437in}{1.642011in}}%
\pgfpathcurveto{\pgfqpoint{1.757387in}{1.642011in}}{\pgfqpoint{1.746788in}{1.637621in}}{\pgfqpoint{1.738975in}{1.629807in}}%
\pgfpathcurveto{\pgfqpoint{1.731161in}{1.621994in}}{\pgfqpoint{1.726771in}{1.611395in}}{\pgfqpoint{1.726771in}{1.600345in}}%
\pgfpathcurveto{\pgfqpoint{1.726771in}{1.589295in}}{\pgfqpoint{1.731161in}{1.578696in}}{\pgfqpoint{1.738975in}{1.570882in}}%
\pgfpathcurveto{\pgfqpoint{1.746788in}{1.563068in}}{\pgfqpoint{1.757387in}{1.558678in}}{\pgfqpoint{1.768437in}{1.558678in}}%
\pgfpathlineto{\pgfqpoint{1.768437in}{1.558678in}}%
\pgfpathclose%
\pgfusepath{stroke}%
\end{pgfscope}%
\begin{pgfscope}%
\pgfpathrectangle{\pgfqpoint{0.393053in}{0.375000in}}{\pgfqpoint{6.356833in}{5.175000in}}%
\pgfusepath{clip}%
\pgfsetbuttcap%
\pgfsetroundjoin%
\pgfsetlinewidth{1.003750pt}%
\definecolor{currentstroke}{rgb}{0.827451,0.827451,0.827451}%
\pgfsetstrokecolor{currentstroke}%
\pgfsetdash{}{0pt}%
\pgfpathmoveto{\pgfqpoint{0.435142in}{3.936498in}}%
\pgfpathcurveto{\pgfqpoint{0.446192in}{3.936498in}}{\pgfqpoint{0.456791in}{3.940888in}}{\pgfqpoint{0.464605in}{3.948702in}}%
\pgfpathcurveto{\pgfqpoint{0.472419in}{3.956516in}}{\pgfqpoint{0.476809in}{3.967115in}}{\pgfqpoint{0.476809in}{3.978165in}}%
\pgfpathcurveto{\pgfqpoint{0.476809in}{3.989215in}}{\pgfqpoint{0.472419in}{3.999814in}}{\pgfqpoint{0.464605in}{4.007627in}}%
\pgfpathcurveto{\pgfqpoint{0.456791in}{4.015441in}}{\pgfqpoint{0.446192in}{4.019831in}}{\pgfqpoint{0.435142in}{4.019831in}}%
\pgfpathcurveto{\pgfqpoint{0.424092in}{4.019831in}}{\pgfqpoint{0.413493in}{4.015441in}}{\pgfqpoint{0.405679in}{4.007627in}}%
\pgfpathcurveto{\pgfqpoint{0.397866in}{3.999814in}}{\pgfqpoint{0.393475in}{3.989215in}}{\pgfqpoint{0.393475in}{3.978165in}}%
\pgfpathcurveto{\pgfqpoint{0.393475in}{3.967115in}}{\pgfqpoint{0.397866in}{3.956516in}}{\pgfqpoint{0.405679in}{3.948702in}}%
\pgfpathcurveto{\pgfqpoint{0.413493in}{3.940888in}}{\pgfqpoint{0.424092in}{3.936498in}}{\pgfqpoint{0.435142in}{3.936498in}}%
\pgfpathlineto{\pgfqpoint{0.435142in}{3.936498in}}%
\pgfpathclose%
\pgfusepath{stroke}%
\end{pgfscope}%
\begin{pgfscope}%
\pgfpathrectangle{\pgfqpoint{0.393053in}{0.375000in}}{\pgfqpoint{6.356833in}{5.175000in}}%
\pgfusepath{clip}%
\pgfsetbuttcap%
\pgfsetroundjoin%
\pgfsetlinewidth{1.003750pt}%
\definecolor{currentstroke}{rgb}{0.827451,0.827451,0.827451}%
\pgfsetstrokecolor{currentstroke}%
\pgfsetdash{}{0pt}%
\pgfpathmoveto{\pgfqpoint{1.843211in}{1.515821in}}%
\pgfpathcurveto{\pgfqpoint{1.854261in}{1.515821in}}{\pgfqpoint{1.864860in}{1.520211in}}{\pgfqpoint{1.872674in}{1.528025in}}%
\pgfpathcurveto{\pgfqpoint{1.880487in}{1.535838in}}{\pgfqpoint{1.884878in}{1.546437in}}{\pgfqpoint{1.884878in}{1.557487in}}%
\pgfpathcurveto{\pgfqpoint{1.884878in}{1.568537in}}{\pgfqpoint{1.880487in}{1.579136in}}{\pgfqpoint{1.872674in}{1.586950in}}%
\pgfpathcurveto{\pgfqpoint{1.864860in}{1.594764in}}{\pgfqpoint{1.854261in}{1.599154in}}{\pgfqpoint{1.843211in}{1.599154in}}%
\pgfpathcurveto{\pgfqpoint{1.832161in}{1.599154in}}{\pgfqpoint{1.821562in}{1.594764in}}{\pgfqpoint{1.813748in}{1.586950in}}%
\pgfpathcurveto{\pgfqpoint{1.805934in}{1.579136in}}{\pgfqpoint{1.801544in}{1.568537in}}{\pgfqpoint{1.801544in}{1.557487in}}%
\pgfpathcurveto{\pgfqpoint{1.801544in}{1.546437in}}{\pgfqpoint{1.805934in}{1.535838in}}{\pgfqpoint{1.813748in}{1.528025in}}%
\pgfpathcurveto{\pgfqpoint{1.821562in}{1.520211in}}{\pgfqpoint{1.832161in}{1.515821in}}{\pgfqpoint{1.843211in}{1.515821in}}%
\pgfpathlineto{\pgfqpoint{1.843211in}{1.515821in}}%
\pgfpathclose%
\pgfusepath{stroke}%
\end{pgfscope}%
\begin{pgfscope}%
\pgfpathrectangle{\pgfqpoint{0.393053in}{0.375000in}}{\pgfqpoint{6.356833in}{5.175000in}}%
\pgfusepath{clip}%
\pgfsetbuttcap%
\pgfsetroundjoin%
\pgfsetlinewidth{1.003750pt}%
\definecolor{currentstroke}{rgb}{0.827451,0.827451,0.827451}%
\pgfsetstrokecolor{currentstroke}%
\pgfsetdash{}{0pt}%
\pgfpathmoveto{\pgfqpoint{0.806869in}{2.717160in}}%
\pgfpathcurveto{\pgfqpoint{0.817919in}{2.717160in}}{\pgfqpoint{0.828518in}{2.721550in}}{\pgfqpoint{0.836332in}{2.729363in}}%
\pgfpathcurveto{\pgfqpoint{0.844145in}{2.737177in}}{\pgfqpoint{0.848536in}{2.747776in}}{\pgfqpoint{0.848536in}{2.758826in}}%
\pgfpathcurveto{\pgfqpoint{0.848536in}{2.769876in}}{\pgfqpoint{0.844145in}{2.780475in}}{\pgfqpoint{0.836332in}{2.788289in}}%
\pgfpathcurveto{\pgfqpoint{0.828518in}{2.796103in}}{\pgfqpoint{0.817919in}{2.800493in}}{\pgfqpoint{0.806869in}{2.800493in}}%
\pgfpathcurveto{\pgfqpoint{0.795819in}{2.800493in}}{\pgfqpoint{0.785220in}{2.796103in}}{\pgfqpoint{0.777406in}{2.788289in}}%
\pgfpathcurveto{\pgfqpoint{0.769593in}{2.780475in}}{\pgfqpoint{0.765202in}{2.769876in}}{\pgfqpoint{0.765202in}{2.758826in}}%
\pgfpathcurveto{\pgfqpoint{0.765202in}{2.747776in}}{\pgfqpoint{0.769593in}{2.737177in}}{\pgfqpoint{0.777406in}{2.729363in}}%
\pgfpathcurveto{\pgfqpoint{0.785220in}{2.721550in}}{\pgfqpoint{0.795819in}{2.717160in}}{\pgfqpoint{0.806869in}{2.717160in}}%
\pgfpathlineto{\pgfqpoint{0.806869in}{2.717160in}}%
\pgfpathclose%
\pgfusepath{stroke}%
\end{pgfscope}%
\begin{pgfscope}%
\pgfpathrectangle{\pgfqpoint{0.393053in}{0.375000in}}{\pgfqpoint{6.356833in}{5.175000in}}%
\pgfusepath{clip}%
\pgfsetbuttcap%
\pgfsetroundjoin%
\pgfsetlinewidth{1.003750pt}%
\definecolor{currentstroke}{rgb}{0.827451,0.827451,0.827451}%
\pgfsetstrokecolor{currentstroke}%
\pgfsetdash{}{0pt}%
\pgfpathmoveto{\pgfqpoint{0.424181in}{4.027299in}}%
\pgfpathcurveto{\pgfqpoint{0.435231in}{4.027299in}}{\pgfqpoint{0.445830in}{4.031689in}}{\pgfqpoint{0.453644in}{4.039503in}}%
\pgfpathcurveto{\pgfqpoint{0.461458in}{4.047316in}}{\pgfqpoint{0.465848in}{4.057915in}}{\pgfqpoint{0.465848in}{4.068966in}}%
\pgfpathcurveto{\pgfqpoint{0.465848in}{4.080016in}}{\pgfqpoint{0.461458in}{4.090615in}}{\pgfqpoint{0.453644in}{4.098428in}}%
\pgfpathcurveto{\pgfqpoint{0.445830in}{4.106242in}}{\pgfqpoint{0.435231in}{4.110632in}}{\pgfqpoint{0.424181in}{4.110632in}}%
\pgfpathcurveto{\pgfqpoint{0.413131in}{4.110632in}}{\pgfqpoint{0.402532in}{4.106242in}}{\pgfqpoint{0.394719in}{4.098428in}}%
\pgfpathcurveto{\pgfqpoint{0.386905in}{4.090615in}}{\pgfqpoint{0.382515in}{4.080016in}}{\pgfqpoint{0.382515in}{4.068966in}}%
\pgfpathcurveto{\pgfqpoint{0.382515in}{4.057915in}}{\pgfqpoint{0.386905in}{4.047316in}}{\pgfqpoint{0.394719in}{4.039503in}}%
\pgfpathcurveto{\pgfqpoint{0.402532in}{4.031689in}}{\pgfqpoint{0.413131in}{4.027299in}}{\pgfqpoint{0.424181in}{4.027299in}}%
\pgfpathlineto{\pgfqpoint{0.424181in}{4.027299in}}%
\pgfpathclose%
\pgfusepath{stroke}%
\end{pgfscope}%
\begin{pgfscope}%
\pgfpathrectangle{\pgfqpoint{0.393053in}{0.375000in}}{\pgfqpoint{6.356833in}{5.175000in}}%
\pgfusepath{clip}%
\pgfsetbuttcap%
\pgfsetroundjoin%
\pgfsetlinewidth{1.003750pt}%
\definecolor{currentstroke}{rgb}{0.827451,0.827451,0.827451}%
\pgfsetstrokecolor{currentstroke}%
\pgfsetdash{}{0pt}%
\pgfpathmoveto{\pgfqpoint{0.848540in}{2.695419in}}%
\pgfpathcurveto{\pgfqpoint{0.859590in}{2.695419in}}{\pgfqpoint{0.870189in}{2.699809in}}{\pgfqpoint{0.878002in}{2.707623in}}%
\pgfpathcurveto{\pgfqpoint{0.885816in}{2.715436in}}{\pgfqpoint{0.890206in}{2.726035in}}{\pgfqpoint{0.890206in}{2.737085in}}%
\pgfpathcurveto{\pgfqpoint{0.890206in}{2.748135in}}{\pgfqpoint{0.885816in}{2.758735in}}{\pgfqpoint{0.878002in}{2.766548in}}%
\pgfpathcurveto{\pgfqpoint{0.870189in}{2.774362in}}{\pgfqpoint{0.859590in}{2.778752in}}{\pgfqpoint{0.848540in}{2.778752in}}%
\pgfpathcurveto{\pgfqpoint{0.837489in}{2.778752in}}{\pgfqpoint{0.826890in}{2.774362in}}{\pgfqpoint{0.819077in}{2.766548in}}%
\pgfpathcurveto{\pgfqpoint{0.811263in}{2.758735in}}{\pgfqpoint{0.806873in}{2.748135in}}{\pgfqpoint{0.806873in}{2.737085in}}%
\pgfpathcurveto{\pgfqpoint{0.806873in}{2.726035in}}{\pgfqpoint{0.811263in}{2.715436in}}{\pgfqpoint{0.819077in}{2.707623in}}%
\pgfpathcurveto{\pgfqpoint{0.826890in}{2.699809in}}{\pgfqpoint{0.837489in}{2.695419in}}{\pgfqpoint{0.848540in}{2.695419in}}%
\pgfpathlineto{\pgfqpoint{0.848540in}{2.695419in}}%
\pgfpathclose%
\pgfusepath{stroke}%
\end{pgfscope}%
\begin{pgfscope}%
\pgfpathrectangle{\pgfqpoint{0.393053in}{0.375000in}}{\pgfqpoint{6.356833in}{5.175000in}}%
\pgfusepath{clip}%
\pgfsetbuttcap%
\pgfsetroundjoin%
\pgfsetlinewidth{1.003750pt}%
\definecolor{currentstroke}{rgb}{0.827451,0.827451,0.827451}%
\pgfsetstrokecolor{currentstroke}%
\pgfsetdash{}{0pt}%
\pgfpathmoveto{\pgfqpoint{0.773700in}{2.823025in}}%
\pgfpathcurveto{\pgfqpoint{0.784750in}{2.823025in}}{\pgfqpoint{0.795350in}{2.827415in}}{\pgfqpoint{0.803163in}{2.835229in}}%
\pgfpathcurveto{\pgfqpoint{0.810977in}{2.843042in}}{\pgfqpoint{0.815367in}{2.853641in}}{\pgfqpoint{0.815367in}{2.864691in}}%
\pgfpathcurveto{\pgfqpoint{0.815367in}{2.875742in}}{\pgfqpoint{0.810977in}{2.886341in}}{\pgfqpoint{0.803163in}{2.894154in}}%
\pgfpathcurveto{\pgfqpoint{0.795350in}{2.901968in}}{\pgfqpoint{0.784750in}{2.906358in}}{\pgfqpoint{0.773700in}{2.906358in}}%
\pgfpathcurveto{\pgfqpoint{0.762650in}{2.906358in}}{\pgfqpoint{0.752051in}{2.901968in}}{\pgfqpoint{0.744238in}{2.894154in}}%
\pgfpathcurveto{\pgfqpoint{0.736424in}{2.886341in}}{\pgfqpoint{0.732034in}{2.875742in}}{\pgfqpoint{0.732034in}{2.864691in}}%
\pgfpathcurveto{\pgfqpoint{0.732034in}{2.853641in}}{\pgfqpoint{0.736424in}{2.843042in}}{\pgfqpoint{0.744238in}{2.835229in}}%
\pgfpathcurveto{\pgfqpoint{0.752051in}{2.827415in}}{\pgfqpoint{0.762650in}{2.823025in}}{\pgfqpoint{0.773700in}{2.823025in}}%
\pgfpathlineto{\pgfqpoint{0.773700in}{2.823025in}}%
\pgfpathclose%
\pgfusepath{stroke}%
\end{pgfscope}%
\begin{pgfscope}%
\pgfpathrectangle{\pgfqpoint{0.393053in}{0.375000in}}{\pgfqpoint{6.356833in}{5.175000in}}%
\pgfusepath{clip}%
\pgfsetbuttcap%
\pgfsetroundjoin%
\pgfsetlinewidth{1.003750pt}%
\definecolor{currentstroke}{rgb}{0.827451,0.827451,0.827451}%
\pgfsetstrokecolor{currentstroke}%
\pgfsetdash{}{0pt}%
\pgfpathmoveto{\pgfqpoint{4.537650in}{0.413917in}}%
\pgfpathcurveto{\pgfqpoint{4.548700in}{0.413917in}}{\pgfqpoint{4.559300in}{0.418307in}}{\pgfqpoint{4.567113in}{0.426121in}}%
\pgfpathcurveto{\pgfqpoint{4.574927in}{0.433935in}}{\pgfqpoint{4.579317in}{0.444534in}}{\pgfqpoint{4.579317in}{0.455584in}}%
\pgfpathcurveto{\pgfqpoint{4.579317in}{0.466634in}}{\pgfqpoint{4.574927in}{0.477233in}}{\pgfqpoint{4.567113in}{0.485047in}}%
\pgfpathcurveto{\pgfqpoint{4.559300in}{0.492860in}}{\pgfqpoint{4.548700in}{0.497250in}}{\pgfqpoint{4.537650in}{0.497250in}}%
\pgfpathcurveto{\pgfqpoint{4.526600in}{0.497250in}}{\pgfqpoint{4.516001in}{0.492860in}}{\pgfqpoint{4.508188in}{0.485047in}}%
\pgfpathcurveto{\pgfqpoint{4.500374in}{0.477233in}}{\pgfqpoint{4.495984in}{0.466634in}}{\pgfqpoint{4.495984in}{0.455584in}}%
\pgfpathcurveto{\pgfqpoint{4.495984in}{0.444534in}}{\pgfqpoint{4.500374in}{0.433935in}}{\pgfqpoint{4.508188in}{0.426121in}}%
\pgfpathcurveto{\pgfqpoint{4.516001in}{0.418307in}}{\pgfqpoint{4.526600in}{0.413917in}}{\pgfqpoint{4.537650in}{0.413917in}}%
\pgfpathlineto{\pgfqpoint{4.537650in}{0.413917in}}%
\pgfpathclose%
\pgfusepath{stroke}%
\end{pgfscope}%
\begin{pgfscope}%
\pgfpathrectangle{\pgfqpoint{0.393053in}{0.375000in}}{\pgfqpoint{6.356833in}{5.175000in}}%
\pgfusepath{clip}%
\pgfsetbuttcap%
\pgfsetroundjoin%
\pgfsetlinewidth{1.003750pt}%
\definecolor{currentstroke}{rgb}{0.827451,0.827451,0.827451}%
\pgfsetstrokecolor{currentstroke}%
\pgfsetdash{}{0pt}%
\pgfpathmoveto{\pgfqpoint{1.934673in}{1.425939in}}%
\pgfpathcurveto{\pgfqpoint{1.945723in}{1.425939in}}{\pgfqpoint{1.956322in}{1.430329in}}{\pgfqpoint{1.964136in}{1.438142in}}%
\pgfpathcurveto{\pgfqpoint{1.971949in}{1.445956in}}{\pgfqpoint{1.976340in}{1.456555in}}{\pgfqpoint{1.976340in}{1.467605in}}%
\pgfpathcurveto{\pgfqpoint{1.976340in}{1.478655in}}{\pgfqpoint{1.971949in}{1.489254in}}{\pgfqpoint{1.964136in}{1.497068in}}%
\pgfpathcurveto{\pgfqpoint{1.956322in}{1.504882in}}{\pgfqpoint{1.945723in}{1.509272in}}{\pgfqpoint{1.934673in}{1.509272in}}%
\pgfpathcurveto{\pgfqpoint{1.923623in}{1.509272in}}{\pgfqpoint{1.913024in}{1.504882in}}{\pgfqpoint{1.905210in}{1.497068in}}%
\pgfpathcurveto{\pgfqpoint{1.897396in}{1.489254in}}{\pgfqpoint{1.893006in}{1.478655in}}{\pgfqpoint{1.893006in}{1.467605in}}%
\pgfpathcurveto{\pgfqpoint{1.893006in}{1.456555in}}{\pgfqpoint{1.897396in}{1.445956in}}{\pgfqpoint{1.905210in}{1.438142in}}%
\pgfpathcurveto{\pgfqpoint{1.913024in}{1.430329in}}{\pgfqpoint{1.923623in}{1.425939in}}{\pgfqpoint{1.934673in}{1.425939in}}%
\pgfpathlineto{\pgfqpoint{1.934673in}{1.425939in}}%
\pgfpathclose%
\pgfusepath{stroke}%
\end{pgfscope}%
\begin{pgfscope}%
\pgfpathrectangle{\pgfqpoint{0.393053in}{0.375000in}}{\pgfqpoint{6.356833in}{5.175000in}}%
\pgfusepath{clip}%
\pgfsetbuttcap%
\pgfsetroundjoin%
\pgfsetlinewidth{1.003750pt}%
\definecolor{currentstroke}{rgb}{0.827451,0.827451,0.827451}%
\pgfsetstrokecolor{currentstroke}%
\pgfsetdash{}{0pt}%
\pgfpathmoveto{\pgfqpoint{2.376136in}{1.121818in}}%
\pgfpathcurveto{\pgfqpoint{2.387186in}{1.121818in}}{\pgfqpoint{2.397785in}{1.126208in}}{\pgfqpoint{2.405598in}{1.134022in}}%
\pgfpathcurveto{\pgfqpoint{2.413412in}{1.141835in}}{\pgfqpoint{2.417802in}{1.152434in}}{\pgfqpoint{2.417802in}{1.163485in}}%
\pgfpathcurveto{\pgfqpoint{2.417802in}{1.174535in}}{\pgfqpoint{2.413412in}{1.185134in}}{\pgfqpoint{2.405598in}{1.192947in}}%
\pgfpathcurveto{\pgfqpoint{2.397785in}{1.200761in}}{\pgfqpoint{2.387186in}{1.205151in}}{\pgfqpoint{2.376136in}{1.205151in}}%
\pgfpathcurveto{\pgfqpoint{2.365086in}{1.205151in}}{\pgfqpoint{2.354487in}{1.200761in}}{\pgfqpoint{2.346673in}{1.192947in}}%
\pgfpathcurveto{\pgfqpoint{2.338859in}{1.185134in}}{\pgfqpoint{2.334469in}{1.174535in}}{\pgfqpoint{2.334469in}{1.163485in}}%
\pgfpathcurveto{\pgfqpoint{2.334469in}{1.152434in}}{\pgfqpoint{2.338859in}{1.141835in}}{\pgfqpoint{2.346673in}{1.134022in}}%
\pgfpathcurveto{\pgfqpoint{2.354487in}{1.126208in}}{\pgfqpoint{2.365086in}{1.121818in}}{\pgfqpoint{2.376136in}{1.121818in}}%
\pgfpathlineto{\pgfqpoint{2.376136in}{1.121818in}}%
\pgfpathclose%
\pgfusepath{stroke}%
\end{pgfscope}%
\begin{pgfscope}%
\pgfpathrectangle{\pgfqpoint{0.393053in}{0.375000in}}{\pgfqpoint{6.356833in}{5.175000in}}%
\pgfusepath{clip}%
\pgfsetbuttcap%
\pgfsetroundjoin%
\pgfsetlinewidth{1.003750pt}%
\definecolor{currentstroke}{rgb}{0.827451,0.827451,0.827451}%
\pgfsetstrokecolor{currentstroke}%
\pgfsetdash{}{0pt}%
\pgfpathmoveto{\pgfqpoint{5.017789in}{0.367635in}}%
\pgfpathcurveto{\pgfqpoint{5.028839in}{0.367635in}}{\pgfqpoint{5.039438in}{0.372025in}}{\pgfqpoint{5.047252in}{0.379839in}}%
\pgfpathcurveto{\pgfqpoint{5.055065in}{0.387652in}}{\pgfqpoint{5.059456in}{0.398251in}}{\pgfqpoint{5.059456in}{0.409302in}}%
\pgfpathcurveto{\pgfqpoint{5.059456in}{0.420352in}}{\pgfqpoint{5.055065in}{0.430951in}}{\pgfqpoint{5.047252in}{0.438764in}}%
\pgfpathcurveto{\pgfqpoint{5.039438in}{0.446578in}}{\pgfqpoint{5.028839in}{0.450968in}}{\pgfqpoint{5.017789in}{0.450968in}}%
\pgfpathcurveto{\pgfqpoint{5.006739in}{0.450968in}}{\pgfqpoint{4.996140in}{0.446578in}}{\pgfqpoint{4.988326in}{0.438764in}}%
\pgfpathcurveto{\pgfqpoint{4.980512in}{0.430951in}}{\pgfqpoint{4.976122in}{0.420352in}}{\pgfqpoint{4.976122in}{0.409302in}}%
\pgfpathcurveto{\pgfqpoint{4.976122in}{0.398251in}}{\pgfqpoint{4.980512in}{0.387652in}}{\pgfqpoint{4.988326in}{0.379839in}}%
\pgfpathcurveto{\pgfqpoint{4.996140in}{0.372025in}}{\pgfqpoint{5.006739in}{0.367635in}}{\pgfqpoint{5.017789in}{0.367635in}}%
\pgfusepath{stroke}%
\end{pgfscope}%
\begin{pgfscope}%
\pgfpathrectangle{\pgfqpoint{0.393053in}{0.375000in}}{\pgfqpoint{6.356833in}{5.175000in}}%
\pgfusepath{clip}%
\pgfsetbuttcap%
\pgfsetroundjoin%
\pgfsetlinewidth{1.003750pt}%
\definecolor{currentstroke}{rgb}{0.827451,0.827451,0.827451}%
\pgfsetstrokecolor{currentstroke}%
\pgfsetdash{}{0pt}%
\pgfpathmoveto{\pgfqpoint{0.396398in}{4.410130in}}%
\pgfpathcurveto{\pgfqpoint{0.407448in}{4.410130in}}{\pgfqpoint{0.418047in}{4.414521in}}{\pgfqpoint{0.425861in}{4.422334in}}%
\pgfpathcurveto{\pgfqpoint{0.433674in}{4.430148in}}{\pgfqpoint{0.438064in}{4.440747in}}{\pgfqpoint{0.438064in}{4.451797in}}%
\pgfpathcurveto{\pgfqpoint{0.438064in}{4.462847in}}{\pgfqpoint{0.433674in}{4.473446in}}{\pgfqpoint{0.425861in}{4.481260in}}%
\pgfpathcurveto{\pgfqpoint{0.418047in}{4.489073in}}{\pgfqpoint{0.407448in}{4.493464in}}{\pgfqpoint{0.396398in}{4.493464in}}%
\pgfpathcurveto{\pgfqpoint{0.385348in}{4.493464in}}{\pgfqpoint{0.374749in}{4.489073in}}{\pgfqpoint{0.366935in}{4.481260in}}%
\pgfpathcurveto{\pgfqpoint{0.359121in}{4.473446in}}{\pgfqpoint{0.354731in}{4.462847in}}{\pgfqpoint{0.354731in}{4.451797in}}%
\pgfpathcurveto{\pgfqpoint{0.354731in}{4.440747in}}{\pgfqpoint{0.359121in}{4.430148in}}{\pgfqpoint{0.366935in}{4.422334in}}%
\pgfpathcurveto{\pgfqpoint{0.374749in}{4.414521in}}{\pgfqpoint{0.385348in}{4.410130in}}{\pgfqpoint{0.396398in}{4.410130in}}%
\pgfpathlineto{\pgfqpoint{0.396398in}{4.410130in}}%
\pgfpathclose%
\pgfusepath{stroke}%
\end{pgfscope}%
\begin{pgfscope}%
\pgfpathrectangle{\pgfqpoint{0.393053in}{0.375000in}}{\pgfqpoint{6.356833in}{5.175000in}}%
\pgfusepath{clip}%
\pgfsetbuttcap%
\pgfsetroundjoin%
\pgfsetlinewidth{1.003750pt}%
\definecolor{currentstroke}{rgb}{0.827451,0.827451,0.827451}%
\pgfsetstrokecolor{currentstroke}%
\pgfsetdash{}{0pt}%
\pgfpathmoveto{\pgfqpoint{0.875375in}{2.588406in}}%
\pgfpathcurveto{\pgfqpoint{0.886425in}{2.588406in}}{\pgfqpoint{0.897024in}{2.592796in}}{\pgfqpoint{0.904838in}{2.600609in}}%
\pgfpathcurveto{\pgfqpoint{0.912652in}{2.608423in}}{\pgfqpoint{0.917042in}{2.619022in}}{\pgfqpoint{0.917042in}{2.630072in}}%
\pgfpathcurveto{\pgfqpoint{0.917042in}{2.641122in}}{\pgfqpoint{0.912652in}{2.651721in}}{\pgfqpoint{0.904838in}{2.659535in}}%
\pgfpathcurveto{\pgfqpoint{0.897024in}{2.667349in}}{\pgfqpoint{0.886425in}{2.671739in}}{\pgfqpoint{0.875375in}{2.671739in}}%
\pgfpathcurveto{\pgfqpoint{0.864325in}{2.671739in}}{\pgfqpoint{0.853726in}{2.667349in}}{\pgfqpoint{0.845912in}{2.659535in}}%
\pgfpathcurveto{\pgfqpoint{0.838099in}{2.651721in}}{\pgfqpoint{0.833708in}{2.641122in}}{\pgfqpoint{0.833708in}{2.630072in}}%
\pgfpathcurveto{\pgfqpoint{0.833708in}{2.619022in}}{\pgfqpoint{0.838099in}{2.608423in}}{\pgfqpoint{0.845912in}{2.600609in}}%
\pgfpathcurveto{\pgfqpoint{0.853726in}{2.592796in}}{\pgfqpoint{0.864325in}{2.588406in}}{\pgfqpoint{0.875375in}{2.588406in}}%
\pgfpathlineto{\pgfqpoint{0.875375in}{2.588406in}}%
\pgfpathclose%
\pgfusepath{stroke}%
\end{pgfscope}%
\begin{pgfscope}%
\pgfpathrectangle{\pgfqpoint{0.393053in}{0.375000in}}{\pgfqpoint{6.356833in}{5.175000in}}%
\pgfusepath{clip}%
\pgfsetbuttcap%
\pgfsetroundjoin%
\pgfsetlinewidth{1.003750pt}%
\definecolor{currentstroke}{rgb}{0.827451,0.827451,0.827451}%
\pgfsetstrokecolor{currentstroke}%
\pgfsetdash{}{0pt}%
\pgfpathmoveto{\pgfqpoint{5.339131in}{0.342839in}}%
\pgfpathcurveto{\pgfqpoint{5.350181in}{0.342839in}}{\pgfqpoint{5.360780in}{0.347229in}}{\pgfqpoint{5.368594in}{0.355043in}}%
\pgfpathcurveto{\pgfqpoint{5.376407in}{0.362857in}}{\pgfqpoint{5.380798in}{0.373456in}}{\pgfqpoint{5.380798in}{0.384506in}}%
\pgfpathcurveto{\pgfqpoint{5.380798in}{0.395556in}}{\pgfqpoint{5.376407in}{0.406155in}}{\pgfqpoint{5.368594in}{0.413968in}}%
\pgfpathcurveto{\pgfqpoint{5.360780in}{0.421782in}}{\pgfqpoint{5.350181in}{0.426172in}}{\pgfqpoint{5.339131in}{0.426172in}}%
\pgfpathcurveto{\pgfqpoint{5.328081in}{0.426172in}}{\pgfqpoint{5.317482in}{0.421782in}}{\pgfqpoint{5.309668in}{0.413968in}}%
\pgfpathcurveto{\pgfqpoint{5.301855in}{0.406155in}}{\pgfqpoint{5.297464in}{0.395556in}}{\pgfqpoint{5.297464in}{0.384506in}}%
\pgfpathcurveto{\pgfqpoint{5.297464in}{0.373456in}}{\pgfqpoint{5.301855in}{0.362857in}}{\pgfqpoint{5.309668in}{0.355043in}}%
\pgfpathcurveto{\pgfqpoint{5.317482in}{0.347229in}}{\pgfqpoint{5.328081in}{0.342839in}}{\pgfqpoint{5.339131in}{0.342839in}}%
\pgfusepath{stroke}%
\end{pgfscope}%
\begin{pgfscope}%
\pgfpathrectangle{\pgfqpoint{0.393053in}{0.375000in}}{\pgfqpoint{6.356833in}{5.175000in}}%
\pgfusepath{clip}%
\pgfsetbuttcap%
\pgfsetroundjoin%
\pgfsetlinewidth{1.003750pt}%
\definecolor{currentstroke}{rgb}{0.827451,0.827451,0.827451}%
\pgfsetstrokecolor{currentstroke}%
\pgfsetdash{}{0pt}%
\pgfpathmoveto{\pgfqpoint{1.702004in}{1.599430in}}%
\pgfpathcurveto{\pgfqpoint{1.713054in}{1.599430in}}{\pgfqpoint{1.723653in}{1.603820in}}{\pgfqpoint{1.731467in}{1.611633in}}%
\pgfpathcurveto{\pgfqpoint{1.739281in}{1.619447in}}{\pgfqpoint{1.743671in}{1.630046in}}{\pgfqpoint{1.743671in}{1.641096in}}%
\pgfpathcurveto{\pgfqpoint{1.743671in}{1.652146in}}{\pgfqpoint{1.739281in}{1.662745in}}{\pgfqpoint{1.731467in}{1.670559in}}%
\pgfpathcurveto{\pgfqpoint{1.723653in}{1.678373in}}{\pgfqpoint{1.713054in}{1.682763in}}{\pgfqpoint{1.702004in}{1.682763in}}%
\pgfpathcurveto{\pgfqpoint{1.690954in}{1.682763in}}{\pgfqpoint{1.680355in}{1.678373in}}{\pgfqpoint{1.672541in}{1.670559in}}%
\pgfpathcurveto{\pgfqpoint{1.664728in}{1.662745in}}{\pgfqpoint{1.660338in}{1.652146in}}{\pgfqpoint{1.660338in}{1.641096in}}%
\pgfpathcurveto{\pgfqpoint{1.660338in}{1.630046in}}{\pgfqpoint{1.664728in}{1.619447in}}{\pgfqpoint{1.672541in}{1.611633in}}%
\pgfpathcurveto{\pgfqpoint{1.680355in}{1.603820in}}{\pgfqpoint{1.690954in}{1.599430in}}{\pgfqpoint{1.702004in}{1.599430in}}%
\pgfpathlineto{\pgfqpoint{1.702004in}{1.599430in}}%
\pgfpathclose%
\pgfusepath{stroke}%
\end{pgfscope}%
\begin{pgfscope}%
\pgfpathrectangle{\pgfqpoint{0.393053in}{0.375000in}}{\pgfqpoint{6.356833in}{5.175000in}}%
\pgfusepath{clip}%
\pgfsetbuttcap%
\pgfsetroundjoin%
\pgfsetlinewidth{1.003750pt}%
\definecolor{currentstroke}{rgb}{0.827451,0.827451,0.827451}%
\pgfsetstrokecolor{currentstroke}%
\pgfsetdash{}{0pt}%
\pgfpathmoveto{\pgfqpoint{3.397541in}{0.662535in}}%
\pgfpathcurveto{\pgfqpoint{3.408591in}{0.662535in}}{\pgfqpoint{3.419190in}{0.666925in}}{\pgfqpoint{3.427004in}{0.674738in}}%
\pgfpathcurveto{\pgfqpoint{3.434817in}{0.682552in}}{\pgfqpoint{3.439208in}{0.693151in}}{\pgfqpoint{3.439208in}{0.704201in}}%
\pgfpathcurveto{\pgfqpoint{3.439208in}{0.715251in}}{\pgfqpoint{3.434817in}{0.725850in}}{\pgfqpoint{3.427004in}{0.733664in}}%
\pgfpathcurveto{\pgfqpoint{3.419190in}{0.741478in}}{\pgfqpoint{3.408591in}{0.745868in}}{\pgfqpoint{3.397541in}{0.745868in}}%
\pgfpathcurveto{\pgfqpoint{3.386491in}{0.745868in}}{\pgfqpoint{3.375892in}{0.741478in}}{\pgfqpoint{3.368078in}{0.733664in}}%
\pgfpathcurveto{\pgfqpoint{3.360265in}{0.725850in}}{\pgfqpoint{3.355874in}{0.715251in}}{\pgfqpoint{3.355874in}{0.704201in}}%
\pgfpathcurveto{\pgfqpoint{3.355874in}{0.693151in}}{\pgfqpoint{3.360265in}{0.682552in}}{\pgfqpoint{3.368078in}{0.674738in}}%
\pgfpathcurveto{\pgfqpoint{3.375892in}{0.666925in}}{\pgfqpoint{3.386491in}{0.662535in}}{\pgfqpoint{3.397541in}{0.662535in}}%
\pgfpathlineto{\pgfqpoint{3.397541in}{0.662535in}}%
\pgfpathclose%
\pgfusepath{stroke}%
\end{pgfscope}%
\begin{pgfscope}%
\pgfpathrectangle{\pgfqpoint{0.393053in}{0.375000in}}{\pgfqpoint{6.356833in}{5.175000in}}%
\pgfusepath{clip}%
\pgfsetbuttcap%
\pgfsetroundjoin%
\pgfsetlinewidth{1.003750pt}%
\definecolor{currentstroke}{rgb}{0.827451,0.827451,0.827451}%
\pgfsetstrokecolor{currentstroke}%
\pgfsetdash{}{0pt}%
\pgfpathmoveto{\pgfqpoint{1.162345in}{2.161331in}}%
\pgfpathcurveto{\pgfqpoint{1.173395in}{2.161331in}}{\pgfqpoint{1.183994in}{2.165721in}}{\pgfqpoint{1.191807in}{2.173535in}}%
\pgfpathcurveto{\pgfqpoint{1.199621in}{2.181348in}}{\pgfqpoint{1.204011in}{2.191948in}}{\pgfqpoint{1.204011in}{2.202998in}}%
\pgfpathcurveto{\pgfqpoint{1.204011in}{2.214048in}}{\pgfqpoint{1.199621in}{2.224647in}}{\pgfqpoint{1.191807in}{2.232460in}}%
\pgfpathcurveto{\pgfqpoint{1.183994in}{2.240274in}}{\pgfqpoint{1.173395in}{2.244664in}}{\pgfqpoint{1.162345in}{2.244664in}}%
\pgfpathcurveto{\pgfqpoint{1.151294in}{2.244664in}}{\pgfqpoint{1.140695in}{2.240274in}}{\pgfqpoint{1.132882in}{2.232460in}}%
\pgfpathcurveto{\pgfqpoint{1.125068in}{2.224647in}}{\pgfqpoint{1.120678in}{2.214048in}}{\pgfqpoint{1.120678in}{2.202998in}}%
\pgfpathcurveto{\pgfqpoint{1.120678in}{2.191948in}}{\pgfqpoint{1.125068in}{2.181348in}}{\pgfqpoint{1.132882in}{2.173535in}}%
\pgfpathcurveto{\pgfqpoint{1.140695in}{2.165721in}}{\pgfqpoint{1.151294in}{2.161331in}}{\pgfqpoint{1.162345in}{2.161331in}}%
\pgfpathlineto{\pgfqpoint{1.162345in}{2.161331in}}%
\pgfpathclose%
\pgfusepath{stroke}%
\end{pgfscope}%
\begin{pgfscope}%
\pgfpathrectangle{\pgfqpoint{0.393053in}{0.375000in}}{\pgfqpoint{6.356833in}{5.175000in}}%
\pgfusepath{clip}%
\pgfsetbuttcap%
\pgfsetroundjoin%
\pgfsetlinewidth{1.003750pt}%
\definecolor{currentstroke}{rgb}{0.827451,0.827451,0.827451}%
\pgfsetstrokecolor{currentstroke}%
\pgfsetdash{}{0pt}%
\pgfpathmoveto{\pgfqpoint{1.234935in}{2.069821in}}%
\pgfpathcurveto{\pgfqpoint{1.245986in}{2.069821in}}{\pgfqpoint{1.256585in}{2.074212in}}{\pgfqpoint{1.264398in}{2.082025in}}%
\pgfpathcurveto{\pgfqpoint{1.272212in}{2.089839in}}{\pgfqpoint{1.276602in}{2.100438in}}{\pgfqpoint{1.276602in}{2.111488in}}%
\pgfpathcurveto{\pgfqpoint{1.276602in}{2.122538in}}{\pgfqpoint{1.272212in}{2.133137in}}{\pgfqpoint{1.264398in}{2.140951in}}%
\pgfpathcurveto{\pgfqpoint{1.256585in}{2.148764in}}{\pgfqpoint{1.245986in}{2.153155in}}{\pgfqpoint{1.234935in}{2.153155in}}%
\pgfpathcurveto{\pgfqpoint{1.223885in}{2.153155in}}{\pgfqpoint{1.213286in}{2.148764in}}{\pgfqpoint{1.205473in}{2.140951in}}%
\pgfpathcurveto{\pgfqpoint{1.197659in}{2.133137in}}{\pgfqpoint{1.193269in}{2.122538in}}{\pgfqpoint{1.193269in}{2.111488in}}%
\pgfpathcurveto{\pgfqpoint{1.193269in}{2.100438in}}{\pgfqpoint{1.197659in}{2.089839in}}{\pgfqpoint{1.205473in}{2.082025in}}%
\pgfpathcurveto{\pgfqpoint{1.213286in}{2.074212in}}{\pgfqpoint{1.223885in}{2.069821in}}{\pgfqpoint{1.234935in}{2.069821in}}%
\pgfpathlineto{\pgfqpoint{1.234935in}{2.069821in}}%
\pgfpathclose%
\pgfusepath{stroke}%
\end{pgfscope}%
\begin{pgfscope}%
\pgfpathrectangle{\pgfqpoint{0.393053in}{0.375000in}}{\pgfqpoint{6.356833in}{5.175000in}}%
\pgfusepath{clip}%
\pgfsetbuttcap%
\pgfsetroundjoin%
\pgfsetlinewidth{1.003750pt}%
\definecolor{currentstroke}{rgb}{0.827451,0.827451,0.827451}%
\pgfsetstrokecolor{currentstroke}%
\pgfsetdash{}{0pt}%
\pgfpathmoveto{\pgfqpoint{1.882270in}{1.458195in}}%
\pgfpathcurveto{\pgfqpoint{1.893321in}{1.458195in}}{\pgfqpoint{1.903920in}{1.462585in}}{\pgfqpoint{1.911733in}{1.470399in}}%
\pgfpathcurveto{\pgfqpoint{1.919547in}{1.478213in}}{\pgfqpoint{1.923937in}{1.488812in}}{\pgfqpoint{1.923937in}{1.499862in}}%
\pgfpathcurveto{\pgfqpoint{1.923937in}{1.510912in}}{\pgfqpoint{1.919547in}{1.521511in}}{\pgfqpoint{1.911733in}{1.529325in}}%
\pgfpathcurveto{\pgfqpoint{1.903920in}{1.537138in}}{\pgfqpoint{1.893321in}{1.541529in}}{\pgfqpoint{1.882270in}{1.541529in}}%
\pgfpathcurveto{\pgfqpoint{1.871220in}{1.541529in}}{\pgfqpoint{1.860621in}{1.537138in}}{\pgfqpoint{1.852808in}{1.529325in}}%
\pgfpathcurveto{\pgfqpoint{1.844994in}{1.521511in}}{\pgfqpoint{1.840604in}{1.510912in}}{\pgfqpoint{1.840604in}{1.499862in}}%
\pgfpathcurveto{\pgfqpoint{1.840604in}{1.488812in}}{\pgfqpoint{1.844994in}{1.478213in}}{\pgfqpoint{1.852808in}{1.470399in}}%
\pgfpathcurveto{\pgfqpoint{1.860621in}{1.462585in}}{\pgfqpoint{1.871220in}{1.458195in}}{\pgfqpoint{1.882270in}{1.458195in}}%
\pgfpathlineto{\pgfqpoint{1.882270in}{1.458195in}}%
\pgfpathclose%
\pgfusepath{stroke}%
\end{pgfscope}%
\begin{pgfscope}%
\pgfpathrectangle{\pgfqpoint{0.393053in}{0.375000in}}{\pgfqpoint{6.356833in}{5.175000in}}%
\pgfusepath{clip}%
\pgfsetbuttcap%
\pgfsetroundjoin%
\pgfsetlinewidth{1.003750pt}%
\definecolor{currentstroke}{rgb}{0.827451,0.827451,0.827451}%
\pgfsetstrokecolor{currentstroke}%
\pgfsetdash{}{0pt}%
\pgfpathmoveto{\pgfqpoint{1.115030in}{2.257910in}}%
\pgfpathcurveto{\pgfqpoint{1.126080in}{2.257910in}}{\pgfqpoint{1.136679in}{2.262300in}}{\pgfqpoint{1.144493in}{2.270114in}}%
\pgfpathcurveto{\pgfqpoint{1.152307in}{2.277927in}}{\pgfqpoint{1.156697in}{2.288526in}}{\pgfqpoint{1.156697in}{2.299576in}}%
\pgfpathcurveto{\pgfqpoint{1.156697in}{2.310626in}}{\pgfqpoint{1.152307in}{2.321226in}}{\pgfqpoint{1.144493in}{2.329039in}}%
\pgfpathcurveto{\pgfqpoint{1.136679in}{2.336853in}}{\pgfqpoint{1.126080in}{2.341243in}}{\pgfqpoint{1.115030in}{2.341243in}}%
\pgfpathcurveto{\pgfqpoint{1.103980in}{2.341243in}}{\pgfqpoint{1.093381in}{2.336853in}}{\pgfqpoint{1.085567in}{2.329039in}}%
\pgfpathcurveto{\pgfqpoint{1.077754in}{2.321226in}}{\pgfqpoint{1.073364in}{2.310626in}}{\pgfqpoint{1.073364in}{2.299576in}}%
\pgfpathcurveto{\pgfqpoint{1.073364in}{2.288526in}}{\pgfqpoint{1.077754in}{2.277927in}}{\pgfqpoint{1.085567in}{2.270114in}}%
\pgfpathcurveto{\pgfqpoint{1.093381in}{2.262300in}}{\pgfqpoint{1.103980in}{2.257910in}}{\pgfqpoint{1.115030in}{2.257910in}}%
\pgfpathlineto{\pgfqpoint{1.115030in}{2.257910in}}%
\pgfpathclose%
\pgfusepath{stroke}%
\end{pgfscope}%
\begin{pgfscope}%
\pgfpathrectangle{\pgfqpoint{0.393053in}{0.375000in}}{\pgfqpoint{6.356833in}{5.175000in}}%
\pgfusepath{clip}%
\pgfsetbuttcap%
\pgfsetroundjoin%
\pgfsetlinewidth{1.003750pt}%
\definecolor{currentstroke}{rgb}{0.827451,0.827451,0.827451}%
\pgfsetstrokecolor{currentstroke}%
\pgfsetdash{}{0pt}%
\pgfpathmoveto{\pgfqpoint{2.103640in}{1.312371in}}%
\pgfpathcurveto{\pgfqpoint{2.114690in}{1.312371in}}{\pgfqpoint{2.125289in}{1.316762in}}{\pgfqpoint{2.133103in}{1.324575in}}%
\pgfpathcurveto{\pgfqpoint{2.140916in}{1.332389in}}{\pgfqpoint{2.145307in}{1.342988in}}{\pgfqpoint{2.145307in}{1.354038in}}%
\pgfpathcurveto{\pgfqpoint{2.145307in}{1.365088in}}{\pgfqpoint{2.140916in}{1.375687in}}{\pgfqpoint{2.133103in}{1.383501in}}%
\pgfpathcurveto{\pgfqpoint{2.125289in}{1.391314in}}{\pgfqpoint{2.114690in}{1.395705in}}{\pgfqpoint{2.103640in}{1.395705in}}%
\pgfpathcurveto{\pgfqpoint{2.092590in}{1.395705in}}{\pgfqpoint{2.081991in}{1.391314in}}{\pgfqpoint{2.074177in}{1.383501in}}%
\pgfpathcurveto{\pgfqpoint{2.066364in}{1.375687in}}{\pgfqpoint{2.061973in}{1.365088in}}{\pgfqpoint{2.061973in}{1.354038in}}%
\pgfpathcurveto{\pgfqpoint{2.061973in}{1.342988in}}{\pgfqpoint{2.066364in}{1.332389in}}{\pgfqpoint{2.074177in}{1.324575in}}%
\pgfpathcurveto{\pgfqpoint{2.081991in}{1.316762in}}{\pgfqpoint{2.092590in}{1.312371in}}{\pgfqpoint{2.103640in}{1.312371in}}%
\pgfpathlineto{\pgfqpoint{2.103640in}{1.312371in}}%
\pgfpathclose%
\pgfusepath{stroke}%
\end{pgfscope}%
\begin{pgfscope}%
\pgfpathrectangle{\pgfqpoint{0.393053in}{0.375000in}}{\pgfqpoint{6.356833in}{5.175000in}}%
\pgfusepath{clip}%
\pgfsetbuttcap%
\pgfsetroundjoin%
\pgfsetlinewidth{1.003750pt}%
\definecolor{currentstroke}{rgb}{0.827451,0.827451,0.827451}%
\pgfsetstrokecolor{currentstroke}%
\pgfsetdash{}{0pt}%
\pgfpathmoveto{\pgfqpoint{0.672197in}{3.023099in}}%
\pgfpathcurveto{\pgfqpoint{0.683247in}{3.023099in}}{\pgfqpoint{0.693846in}{3.027489in}}{\pgfqpoint{0.701660in}{3.035303in}}%
\pgfpathcurveto{\pgfqpoint{0.709473in}{3.043116in}}{\pgfqpoint{0.713863in}{3.053715in}}{\pgfqpoint{0.713863in}{3.064766in}}%
\pgfpathcurveto{\pgfqpoint{0.713863in}{3.075816in}}{\pgfqpoint{0.709473in}{3.086415in}}{\pgfqpoint{0.701660in}{3.094228in}}%
\pgfpathcurveto{\pgfqpoint{0.693846in}{3.102042in}}{\pgfqpoint{0.683247in}{3.106432in}}{\pgfqpoint{0.672197in}{3.106432in}}%
\pgfpathcurveto{\pgfqpoint{0.661147in}{3.106432in}}{\pgfqpoint{0.650548in}{3.102042in}}{\pgfqpoint{0.642734in}{3.094228in}}%
\pgfpathcurveto{\pgfqpoint{0.634920in}{3.086415in}}{\pgfqpoint{0.630530in}{3.075816in}}{\pgfqpoint{0.630530in}{3.064766in}}%
\pgfpathcurveto{\pgfqpoint{0.630530in}{3.053715in}}{\pgfqpoint{0.634920in}{3.043116in}}{\pgfqpoint{0.642734in}{3.035303in}}%
\pgfpathcurveto{\pgfqpoint{0.650548in}{3.027489in}}{\pgfqpoint{0.661147in}{3.023099in}}{\pgfqpoint{0.672197in}{3.023099in}}%
\pgfpathlineto{\pgfqpoint{0.672197in}{3.023099in}}%
\pgfpathclose%
\pgfusepath{stroke}%
\end{pgfscope}%
\begin{pgfscope}%
\pgfpathrectangle{\pgfqpoint{0.393053in}{0.375000in}}{\pgfqpoint{6.356833in}{5.175000in}}%
\pgfusepath{clip}%
\pgfsetbuttcap%
\pgfsetroundjoin%
\pgfsetlinewidth{1.003750pt}%
\definecolor{currentstroke}{rgb}{0.827451,0.827451,0.827451}%
\pgfsetstrokecolor{currentstroke}%
\pgfsetdash{}{0pt}%
\pgfpathmoveto{\pgfqpoint{0.508412in}{3.594376in}}%
\pgfpathcurveto{\pgfqpoint{0.519462in}{3.594376in}}{\pgfqpoint{0.530061in}{3.598766in}}{\pgfqpoint{0.537874in}{3.606580in}}%
\pgfpathcurveto{\pgfqpoint{0.545688in}{3.614393in}}{\pgfqpoint{0.550078in}{3.624992in}}{\pgfqpoint{0.550078in}{3.636042in}}%
\pgfpathcurveto{\pgfqpoint{0.550078in}{3.647092in}}{\pgfqpoint{0.545688in}{3.657692in}}{\pgfqpoint{0.537874in}{3.665505in}}%
\pgfpathcurveto{\pgfqpoint{0.530061in}{3.673319in}}{\pgfqpoint{0.519462in}{3.677709in}}{\pgfqpoint{0.508412in}{3.677709in}}%
\pgfpathcurveto{\pgfqpoint{0.497361in}{3.677709in}}{\pgfqpoint{0.486762in}{3.673319in}}{\pgfqpoint{0.478949in}{3.665505in}}%
\pgfpathcurveto{\pgfqpoint{0.471135in}{3.657692in}}{\pgfqpoint{0.466745in}{3.647092in}}{\pgfqpoint{0.466745in}{3.636042in}}%
\pgfpathcurveto{\pgfqpoint{0.466745in}{3.624992in}}{\pgfqpoint{0.471135in}{3.614393in}}{\pgfqpoint{0.478949in}{3.606580in}}%
\pgfpathcurveto{\pgfqpoint{0.486762in}{3.598766in}}{\pgfqpoint{0.497361in}{3.594376in}}{\pgfqpoint{0.508412in}{3.594376in}}%
\pgfpathlineto{\pgfqpoint{0.508412in}{3.594376in}}%
\pgfpathclose%
\pgfusepath{stroke}%
\end{pgfscope}%
\begin{pgfscope}%
\pgfpathrectangle{\pgfqpoint{0.393053in}{0.375000in}}{\pgfqpoint{6.356833in}{5.175000in}}%
\pgfusepath{clip}%
\pgfsetbuttcap%
\pgfsetroundjoin%
\pgfsetlinewidth{1.003750pt}%
\definecolor{currentstroke}{rgb}{0.827451,0.827451,0.827451}%
\pgfsetstrokecolor{currentstroke}%
\pgfsetdash{}{0pt}%
\pgfpathmoveto{\pgfqpoint{1.176673in}{2.141426in}}%
\pgfpathcurveto{\pgfqpoint{1.187723in}{2.141426in}}{\pgfqpoint{1.198322in}{2.145816in}}{\pgfqpoint{1.206136in}{2.153630in}}%
\pgfpathcurveto{\pgfqpoint{1.213949in}{2.161444in}}{\pgfqpoint{1.218340in}{2.172043in}}{\pgfqpoint{1.218340in}{2.183093in}}%
\pgfpathcurveto{\pgfqpoint{1.218340in}{2.194143in}}{\pgfqpoint{1.213949in}{2.204742in}}{\pgfqpoint{1.206136in}{2.212556in}}%
\pgfpathcurveto{\pgfqpoint{1.198322in}{2.220369in}}{\pgfqpoint{1.187723in}{2.224759in}}{\pgfqpoint{1.176673in}{2.224759in}}%
\pgfpathcurveto{\pgfqpoint{1.165623in}{2.224759in}}{\pgfqpoint{1.155024in}{2.220369in}}{\pgfqpoint{1.147210in}{2.212556in}}%
\pgfpathcurveto{\pgfqpoint{1.139396in}{2.204742in}}{\pgfqpoint{1.135006in}{2.194143in}}{\pgfqpoint{1.135006in}{2.183093in}}%
\pgfpathcurveto{\pgfqpoint{1.135006in}{2.172043in}}{\pgfqpoint{1.139396in}{2.161444in}}{\pgfqpoint{1.147210in}{2.153630in}}%
\pgfpathcurveto{\pgfqpoint{1.155024in}{2.145816in}}{\pgfqpoint{1.165623in}{2.141426in}}{\pgfqpoint{1.176673in}{2.141426in}}%
\pgfpathlineto{\pgfqpoint{1.176673in}{2.141426in}}%
\pgfpathclose%
\pgfusepath{stroke}%
\end{pgfscope}%
\begin{pgfscope}%
\pgfpathrectangle{\pgfqpoint{0.393053in}{0.375000in}}{\pgfqpoint{6.356833in}{5.175000in}}%
\pgfusepath{clip}%
\pgfsetbuttcap%
\pgfsetroundjoin%
\pgfsetlinewidth{1.003750pt}%
\definecolor{currentstroke}{rgb}{0.827451,0.827451,0.827451}%
\pgfsetstrokecolor{currentstroke}%
\pgfsetdash{}{0pt}%
\pgfpathmoveto{\pgfqpoint{5.340211in}{0.342419in}}%
\pgfpathcurveto{\pgfqpoint{5.351261in}{0.342419in}}{\pgfqpoint{5.361860in}{0.346810in}}{\pgfqpoint{5.369673in}{0.354623in}}%
\pgfpathcurveto{\pgfqpoint{5.377487in}{0.362437in}}{\pgfqpoint{5.381877in}{0.373036in}}{\pgfqpoint{5.381877in}{0.384086in}}%
\pgfpathcurveto{\pgfqpoint{5.381877in}{0.395136in}}{\pgfqpoint{5.377487in}{0.405735in}}{\pgfqpoint{5.369673in}{0.413549in}}%
\pgfpathcurveto{\pgfqpoint{5.361860in}{0.421362in}}{\pgfqpoint{5.351261in}{0.425753in}}{\pgfqpoint{5.340211in}{0.425753in}}%
\pgfpathcurveto{\pgfqpoint{5.329160in}{0.425753in}}{\pgfqpoint{5.318561in}{0.421362in}}{\pgfqpoint{5.310748in}{0.413549in}}%
\pgfpathcurveto{\pgfqpoint{5.302934in}{0.405735in}}{\pgfqpoint{5.298544in}{0.395136in}}{\pgfqpoint{5.298544in}{0.384086in}}%
\pgfpathcurveto{\pgfqpoint{5.298544in}{0.373036in}}{\pgfqpoint{5.302934in}{0.362437in}}{\pgfqpoint{5.310748in}{0.354623in}}%
\pgfpathcurveto{\pgfqpoint{5.318561in}{0.346810in}}{\pgfqpoint{5.329160in}{0.342419in}}{\pgfqpoint{5.340211in}{0.342419in}}%
\pgfusepath{stroke}%
\end{pgfscope}%
\begin{pgfscope}%
\pgfpathrectangle{\pgfqpoint{0.393053in}{0.375000in}}{\pgfqpoint{6.356833in}{5.175000in}}%
\pgfusepath{clip}%
\pgfsetbuttcap%
\pgfsetroundjoin%
\pgfsetlinewidth{1.003750pt}%
\definecolor{currentstroke}{rgb}{0.827451,0.827451,0.827451}%
\pgfsetstrokecolor{currentstroke}%
\pgfsetdash{}{0pt}%
\pgfpathmoveto{\pgfqpoint{4.310929in}{0.438129in}}%
\pgfpathcurveto{\pgfqpoint{4.321979in}{0.438129in}}{\pgfqpoint{4.332578in}{0.442519in}}{\pgfqpoint{4.340392in}{0.450333in}}%
\pgfpathcurveto{\pgfqpoint{4.348205in}{0.458147in}}{\pgfqpoint{4.352595in}{0.468746in}}{\pgfqpoint{4.352595in}{0.479796in}}%
\pgfpathcurveto{\pgfqpoint{4.352595in}{0.490846in}}{\pgfqpoint{4.348205in}{0.501445in}}{\pgfqpoint{4.340392in}{0.509258in}}%
\pgfpathcurveto{\pgfqpoint{4.332578in}{0.517072in}}{\pgfqpoint{4.321979in}{0.521462in}}{\pgfqpoint{4.310929in}{0.521462in}}%
\pgfpathcurveto{\pgfqpoint{4.299879in}{0.521462in}}{\pgfqpoint{4.289280in}{0.517072in}}{\pgfqpoint{4.281466in}{0.509258in}}%
\pgfpathcurveto{\pgfqpoint{4.273652in}{0.501445in}}{\pgfqpoint{4.269262in}{0.490846in}}{\pgfqpoint{4.269262in}{0.479796in}}%
\pgfpathcurveto{\pgfqpoint{4.269262in}{0.468746in}}{\pgfqpoint{4.273652in}{0.458147in}}{\pgfqpoint{4.281466in}{0.450333in}}%
\pgfpathcurveto{\pgfqpoint{4.289280in}{0.442519in}}{\pgfqpoint{4.299879in}{0.438129in}}{\pgfqpoint{4.310929in}{0.438129in}}%
\pgfpathlineto{\pgfqpoint{4.310929in}{0.438129in}}%
\pgfpathclose%
\pgfusepath{stroke}%
\end{pgfscope}%
\begin{pgfscope}%
\pgfpathrectangle{\pgfqpoint{0.393053in}{0.375000in}}{\pgfqpoint{6.356833in}{5.175000in}}%
\pgfusepath{clip}%
\pgfsetbuttcap%
\pgfsetroundjoin%
\pgfsetlinewidth{1.003750pt}%
\definecolor{currentstroke}{rgb}{0.827451,0.827451,0.827451}%
\pgfsetstrokecolor{currentstroke}%
\pgfsetdash{}{0pt}%
\pgfpathmoveto{\pgfqpoint{1.375643in}{1.906128in}}%
\pgfpathcurveto{\pgfqpoint{1.386694in}{1.906128in}}{\pgfqpoint{1.397293in}{1.910518in}}{\pgfqpoint{1.405106in}{1.918331in}}%
\pgfpathcurveto{\pgfqpoint{1.412920in}{1.926145in}}{\pgfqpoint{1.417310in}{1.936744in}}{\pgfqpoint{1.417310in}{1.947794in}}%
\pgfpathcurveto{\pgfqpoint{1.417310in}{1.958844in}}{\pgfqpoint{1.412920in}{1.969443in}}{\pgfqpoint{1.405106in}{1.977257in}}%
\pgfpathcurveto{\pgfqpoint{1.397293in}{1.985071in}}{\pgfqpoint{1.386694in}{1.989461in}}{\pgfqpoint{1.375643in}{1.989461in}}%
\pgfpathcurveto{\pgfqpoint{1.364593in}{1.989461in}}{\pgfqpoint{1.353994in}{1.985071in}}{\pgfqpoint{1.346181in}{1.977257in}}%
\pgfpathcurveto{\pgfqpoint{1.338367in}{1.969443in}}{\pgfqpoint{1.333977in}{1.958844in}}{\pgfqpoint{1.333977in}{1.947794in}}%
\pgfpathcurveto{\pgfqpoint{1.333977in}{1.936744in}}{\pgfqpoint{1.338367in}{1.926145in}}{\pgfqpoint{1.346181in}{1.918331in}}%
\pgfpathcurveto{\pgfqpoint{1.353994in}{1.910518in}}{\pgfqpoint{1.364593in}{1.906128in}}{\pgfqpoint{1.375643in}{1.906128in}}%
\pgfpathlineto{\pgfqpoint{1.375643in}{1.906128in}}%
\pgfpathclose%
\pgfusepath{stroke}%
\end{pgfscope}%
\begin{pgfscope}%
\pgfpathrectangle{\pgfqpoint{0.393053in}{0.375000in}}{\pgfqpoint{6.356833in}{5.175000in}}%
\pgfusepath{clip}%
\pgfsetbuttcap%
\pgfsetroundjoin%
\pgfsetlinewidth{1.003750pt}%
\definecolor{currentstroke}{rgb}{0.827451,0.827451,0.827451}%
\pgfsetstrokecolor{currentstroke}%
\pgfsetdash{}{0pt}%
\pgfpathmoveto{\pgfqpoint{2.584660in}{1.053636in}}%
\pgfpathcurveto{\pgfqpoint{2.595711in}{1.053636in}}{\pgfqpoint{2.606310in}{1.058027in}}{\pgfqpoint{2.614123in}{1.065840in}}%
\pgfpathcurveto{\pgfqpoint{2.621937in}{1.073654in}}{\pgfqpoint{2.626327in}{1.084253in}}{\pgfqpoint{2.626327in}{1.095303in}}%
\pgfpathcurveto{\pgfqpoint{2.626327in}{1.106353in}}{\pgfqpoint{2.621937in}{1.116952in}}{\pgfqpoint{2.614123in}{1.124766in}}%
\pgfpathcurveto{\pgfqpoint{2.606310in}{1.132580in}}{\pgfqpoint{2.595711in}{1.136970in}}{\pgfqpoint{2.584660in}{1.136970in}}%
\pgfpathcurveto{\pgfqpoint{2.573610in}{1.136970in}}{\pgfqpoint{2.563011in}{1.132580in}}{\pgfqpoint{2.555198in}{1.124766in}}%
\pgfpathcurveto{\pgfqpoint{2.547384in}{1.116952in}}{\pgfqpoint{2.542994in}{1.106353in}}{\pgfqpoint{2.542994in}{1.095303in}}%
\pgfpathcurveto{\pgfqpoint{2.542994in}{1.084253in}}{\pgfqpoint{2.547384in}{1.073654in}}{\pgfqpoint{2.555198in}{1.065840in}}%
\pgfpathcurveto{\pgfqpoint{2.563011in}{1.058027in}}{\pgfqpoint{2.573610in}{1.053636in}}{\pgfqpoint{2.584660in}{1.053636in}}%
\pgfpathlineto{\pgfqpoint{2.584660in}{1.053636in}}%
\pgfpathclose%
\pgfusepath{stroke}%
\end{pgfscope}%
\begin{pgfscope}%
\pgfpathrectangle{\pgfqpoint{0.393053in}{0.375000in}}{\pgfqpoint{6.356833in}{5.175000in}}%
\pgfusepath{clip}%
\pgfsetbuttcap%
\pgfsetroundjoin%
\pgfsetlinewidth{1.003750pt}%
\definecolor{currentstroke}{rgb}{0.827451,0.827451,0.827451}%
\pgfsetstrokecolor{currentstroke}%
\pgfsetdash{}{0pt}%
\pgfpathmoveto{\pgfqpoint{2.629511in}{0.982910in}}%
\pgfpathcurveto{\pgfqpoint{2.640562in}{0.982910in}}{\pgfqpoint{2.651161in}{0.987300in}}{\pgfqpoint{2.658974in}{0.995114in}}%
\pgfpathcurveto{\pgfqpoint{2.666788in}{1.002928in}}{\pgfqpoint{2.671178in}{1.013527in}}{\pgfqpoint{2.671178in}{1.024577in}}%
\pgfpathcurveto{\pgfqpoint{2.671178in}{1.035627in}}{\pgfqpoint{2.666788in}{1.046226in}}{\pgfqpoint{2.658974in}{1.054040in}}%
\pgfpathcurveto{\pgfqpoint{2.651161in}{1.061853in}}{\pgfqpoint{2.640562in}{1.066243in}}{\pgfqpoint{2.629511in}{1.066243in}}%
\pgfpathcurveto{\pgfqpoint{2.618461in}{1.066243in}}{\pgfqpoint{2.607862in}{1.061853in}}{\pgfqpoint{2.600049in}{1.054040in}}%
\pgfpathcurveto{\pgfqpoint{2.592235in}{1.046226in}}{\pgfqpoint{2.587845in}{1.035627in}}{\pgfqpoint{2.587845in}{1.024577in}}%
\pgfpathcurveto{\pgfqpoint{2.587845in}{1.013527in}}{\pgfqpoint{2.592235in}{1.002928in}}{\pgfqpoint{2.600049in}{0.995114in}}%
\pgfpathcurveto{\pgfqpoint{2.607862in}{0.987300in}}{\pgfqpoint{2.618461in}{0.982910in}}{\pgfqpoint{2.629511in}{0.982910in}}%
\pgfpathlineto{\pgfqpoint{2.629511in}{0.982910in}}%
\pgfpathclose%
\pgfusepath{stroke}%
\end{pgfscope}%
\begin{pgfscope}%
\pgfpathrectangle{\pgfqpoint{0.393053in}{0.375000in}}{\pgfqpoint{6.356833in}{5.175000in}}%
\pgfusepath{clip}%
\pgfsetbuttcap%
\pgfsetroundjoin%
\pgfsetlinewidth{1.003750pt}%
\definecolor{currentstroke}{rgb}{0.827451,0.827451,0.827451}%
\pgfsetstrokecolor{currentstroke}%
\pgfsetdash{}{0pt}%
\pgfpathmoveto{\pgfqpoint{1.431784in}{1.847586in}}%
\pgfpathcurveto{\pgfqpoint{1.442835in}{1.847586in}}{\pgfqpoint{1.453434in}{1.851976in}}{\pgfqpoint{1.461247in}{1.859790in}}%
\pgfpathcurveto{\pgfqpoint{1.469061in}{1.867603in}}{\pgfqpoint{1.473451in}{1.878202in}}{\pgfqpoint{1.473451in}{1.889253in}}%
\pgfpathcurveto{\pgfqpoint{1.473451in}{1.900303in}}{\pgfqpoint{1.469061in}{1.910902in}}{\pgfqpoint{1.461247in}{1.918715in}}%
\pgfpathcurveto{\pgfqpoint{1.453434in}{1.926529in}}{\pgfqpoint{1.442835in}{1.930919in}}{\pgfqpoint{1.431784in}{1.930919in}}%
\pgfpathcurveto{\pgfqpoint{1.420734in}{1.930919in}}{\pgfqpoint{1.410135in}{1.926529in}}{\pgfqpoint{1.402322in}{1.918715in}}%
\pgfpathcurveto{\pgfqpoint{1.394508in}{1.910902in}}{\pgfqpoint{1.390118in}{1.900303in}}{\pgfqpoint{1.390118in}{1.889253in}}%
\pgfpathcurveto{\pgfqpoint{1.390118in}{1.878202in}}{\pgfqpoint{1.394508in}{1.867603in}}{\pgfqpoint{1.402322in}{1.859790in}}%
\pgfpathcurveto{\pgfqpoint{1.410135in}{1.851976in}}{\pgfqpoint{1.420734in}{1.847586in}}{\pgfqpoint{1.431784in}{1.847586in}}%
\pgfpathlineto{\pgfqpoint{1.431784in}{1.847586in}}%
\pgfpathclose%
\pgfusepath{stroke}%
\end{pgfscope}%
\begin{pgfscope}%
\pgfpathrectangle{\pgfqpoint{0.393053in}{0.375000in}}{\pgfqpoint{6.356833in}{5.175000in}}%
\pgfusepath{clip}%
\pgfsetbuttcap%
\pgfsetroundjoin%
\pgfsetlinewidth{1.003750pt}%
\definecolor{currentstroke}{rgb}{0.827451,0.827451,0.827451}%
\pgfsetstrokecolor{currentstroke}%
\pgfsetdash{}{0pt}%
\pgfpathmoveto{\pgfqpoint{3.308571in}{0.683593in}}%
\pgfpathcurveto{\pgfqpoint{3.319621in}{0.683593in}}{\pgfqpoint{3.330220in}{0.687984in}}{\pgfqpoint{3.338034in}{0.695797in}}%
\pgfpathcurveto{\pgfqpoint{3.345848in}{0.703611in}}{\pgfqpoint{3.350238in}{0.714210in}}{\pgfqpoint{3.350238in}{0.725260in}}%
\pgfpathcurveto{\pgfqpoint{3.350238in}{0.736310in}}{\pgfqpoint{3.345848in}{0.746909in}}{\pgfqpoint{3.338034in}{0.754723in}}%
\pgfpathcurveto{\pgfqpoint{3.330220in}{0.762537in}}{\pgfqpoint{3.319621in}{0.766927in}}{\pgfqpoint{3.308571in}{0.766927in}}%
\pgfpathcurveto{\pgfqpoint{3.297521in}{0.766927in}}{\pgfqpoint{3.286922in}{0.762537in}}{\pgfqpoint{3.279108in}{0.754723in}}%
\pgfpathcurveto{\pgfqpoint{3.271295in}{0.746909in}}{\pgfqpoint{3.266904in}{0.736310in}}{\pgfqpoint{3.266904in}{0.725260in}}%
\pgfpathcurveto{\pgfqpoint{3.266904in}{0.714210in}}{\pgfqpoint{3.271295in}{0.703611in}}{\pgfqpoint{3.279108in}{0.695797in}}%
\pgfpathcurveto{\pgfqpoint{3.286922in}{0.687984in}}{\pgfqpoint{3.297521in}{0.683593in}}{\pgfqpoint{3.308571in}{0.683593in}}%
\pgfpathlineto{\pgfqpoint{3.308571in}{0.683593in}}%
\pgfpathclose%
\pgfusepath{stroke}%
\end{pgfscope}%
\begin{pgfscope}%
\pgfpathrectangle{\pgfqpoint{0.393053in}{0.375000in}}{\pgfqpoint{6.356833in}{5.175000in}}%
\pgfusepath{clip}%
\pgfsetbuttcap%
\pgfsetroundjoin%
\pgfsetlinewidth{1.003750pt}%
\definecolor{currentstroke}{rgb}{0.827451,0.827451,0.827451}%
\pgfsetstrokecolor{currentstroke}%
\pgfsetdash{}{0pt}%
\pgfpathmoveto{\pgfqpoint{0.511163in}{3.526410in}}%
\pgfpathcurveto{\pgfqpoint{0.522213in}{3.526410in}}{\pgfqpoint{0.532812in}{3.530801in}}{\pgfqpoint{0.540626in}{3.538614in}}%
\pgfpathcurveto{\pgfqpoint{0.548440in}{3.546428in}}{\pgfqpoint{0.552830in}{3.557027in}}{\pgfqpoint{0.552830in}{3.568077in}}%
\pgfpathcurveto{\pgfqpoint{0.552830in}{3.579127in}}{\pgfqpoint{0.548440in}{3.589726in}}{\pgfqpoint{0.540626in}{3.597540in}}%
\pgfpathcurveto{\pgfqpoint{0.532812in}{3.605353in}}{\pgfqpoint{0.522213in}{3.609744in}}{\pgfqpoint{0.511163in}{3.609744in}}%
\pgfpathcurveto{\pgfqpoint{0.500113in}{3.609744in}}{\pgfqpoint{0.489514in}{3.605353in}}{\pgfqpoint{0.481700in}{3.597540in}}%
\pgfpathcurveto{\pgfqpoint{0.473887in}{3.589726in}}{\pgfqpoint{0.469496in}{3.579127in}}{\pgfqpoint{0.469496in}{3.568077in}}%
\pgfpathcurveto{\pgfqpoint{0.469496in}{3.557027in}}{\pgfqpoint{0.473887in}{3.546428in}}{\pgfqpoint{0.481700in}{3.538614in}}%
\pgfpathcurveto{\pgfqpoint{0.489514in}{3.530801in}}{\pgfqpoint{0.500113in}{3.526410in}}{\pgfqpoint{0.511163in}{3.526410in}}%
\pgfpathlineto{\pgfqpoint{0.511163in}{3.526410in}}%
\pgfpathclose%
\pgfusepath{stroke}%
\end{pgfscope}%
\begin{pgfscope}%
\pgfpathrectangle{\pgfqpoint{0.393053in}{0.375000in}}{\pgfqpoint{6.356833in}{5.175000in}}%
\pgfusepath{clip}%
\pgfsetbuttcap%
\pgfsetroundjoin%
\pgfsetlinewidth{1.003750pt}%
\definecolor{currentstroke}{rgb}{0.827451,0.827451,0.827451}%
\pgfsetstrokecolor{currentstroke}%
\pgfsetdash{}{0pt}%
\pgfpathmoveto{\pgfqpoint{1.019447in}{2.455876in}}%
\pgfpathcurveto{\pgfqpoint{1.030497in}{2.455876in}}{\pgfqpoint{1.041096in}{2.460267in}}{\pgfqpoint{1.048910in}{2.468080in}}%
\pgfpathcurveto{\pgfqpoint{1.056723in}{2.475894in}}{\pgfqpoint{1.061113in}{2.486493in}}{\pgfqpoint{1.061113in}{2.497543in}}%
\pgfpathcurveto{\pgfqpoint{1.061113in}{2.508593in}}{\pgfqpoint{1.056723in}{2.519192in}}{\pgfqpoint{1.048910in}{2.527006in}}%
\pgfpathcurveto{\pgfqpoint{1.041096in}{2.534819in}}{\pgfqpoint{1.030497in}{2.539210in}}{\pgfqpoint{1.019447in}{2.539210in}}%
\pgfpathcurveto{\pgfqpoint{1.008397in}{2.539210in}}{\pgfqpoint{0.997798in}{2.534819in}}{\pgfqpoint{0.989984in}{2.527006in}}%
\pgfpathcurveto{\pgfqpoint{0.982170in}{2.519192in}}{\pgfqpoint{0.977780in}{2.508593in}}{\pgfqpoint{0.977780in}{2.497543in}}%
\pgfpathcurveto{\pgfqpoint{0.977780in}{2.486493in}}{\pgfqpoint{0.982170in}{2.475894in}}{\pgfqpoint{0.989984in}{2.468080in}}%
\pgfpathcurveto{\pgfqpoint{0.997798in}{2.460267in}}{\pgfqpoint{1.008397in}{2.455876in}}{\pgfqpoint{1.019447in}{2.455876in}}%
\pgfpathlineto{\pgfqpoint{1.019447in}{2.455876in}}%
\pgfpathclose%
\pgfusepath{stroke}%
\end{pgfscope}%
\begin{pgfscope}%
\pgfpathrectangle{\pgfqpoint{0.393053in}{0.375000in}}{\pgfqpoint{6.356833in}{5.175000in}}%
\pgfusepath{clip}%
\pgfsetbuttcap%
\pgfsetroundjoin%
\pgfsetlinewidth{1.003750pt}%
\definecolor{currentstroke}{rgb}{0.827451,0.827451,0.827451}%
\pgfsetstrokecolor{currentstroke}%
\pgfsetdash{}{0pt}%
\pgfpathmoveto{\pgfqpoint{0.911257in}{2.570368in}}%
\pgfpathcurveto{\pgfqpoint{0.922307in}{2.570368in}}{\pgfqpoint{0.932906in}{2.574759in}}{\pgfqpoint{0.940720in}{2.582572in}}%
\pgfpathcurveto{\pgfqpoint{0.948533in}{2.590386in}}{\pgfqpoint{0.952923in}{2.600985in}}{\pgfqpoint{0.952923in}{2.612035in}}%
\pgfpathcurveto{\pgfqpoint{0.952923in}{2.623085in}}{\pgfqpoint{0.948533in}{2.633684in}}{\pgfqpoint{0.940720in}{2.641498in}}%
\pgfpathcurveto{\pgfqpoint{0.932906in}{2.649311in}}{\pgfqpoint{0.922307in}{2.653702in}}{\pgfqpoint{0.911257in}{2.653702in}}%
\pgfpathcurveto{\pgfqpoint{0.900207in}{2.653702in}}{\pgfqpoint{0.889608in}{2.649311in}}{\pgfqpoint{0.881794in}{2.641498in}}%
\pgfpathcurveto{\pgfqpoint{0.873980in}{2.633684in}}{\pgfqpoint{0.869590in}{2.623085in}}{\pgfqpoint{0.869590in}{2.612035in}}%
\pgfpathcurveto{\pgfqpoint{0.869590in}{2.600985in}}{\pgfqpoint{0.873980in}{2.590386in}}{\pgfqpoint{0.881794in}{2.582572in}}%
\pgfpathcurveto{\pgfqpoint{0.889608in}{2.574759in}}{\pgfqpoint{0.900207in}{2.570368in}}{\pgfqpoint{0.911257in}{2.570368in}}%
\pgfpathlineto{\pgfqpoint{0.911257in}{2.570368in}}%
\pgfpathclose%
\pgfusepath{stroke}%
\end{pgfscope}%
\begin{pgfscope}%
\pgfpathrectangle{\pgfqpoint{0.393053in}{0.375000in}}{\pgfqpoint{6.356833in}{5.175000in}}%
\pgfusepath{clip}%
\pgfsetbuttcap%
\pgfsetroundjoin%
\pgfsetlinewidth{1.003750pt}%
\definecolor{currentstroke}{rgb}{0.827451,0.827451,0.827451}%
\pgfsetstrokecolor{currentstroke}%
\pgfsetdash{}{0pt}%
\pgfpathmoveto{\pgfqpoint{0.412548in}{4.185071in}}%
\pgfpathcurveto{\pgfqpoint{0.423598in}{4.185071in}}{\pgfqpoint{0.434197in}{4.189462in}}{\pgfqpoint{0.442011in}{4.197275in}}%
\pgfpathcurveto{\pgfqpoint{0.449824in}{4.205089in}}{\pgfqpoint{0.454215in}{4.215688in}}{\pgfqpoint{0.454215in}{4.226738in}}%
\pgfpathcurveto{\pgfqpoint{0.454215in}{4.237788in}}{\pgfqpoint{0.449824in}{4.248387in}}{\pgfqpoint{0.442011in}{4.256201in}}%
\pgfpathcurveto{\pgfqpoint{0.434197in}{4.264015in}}{\pgfqpoint{0.423598in}{4.268405in}}{\pgfqpoint{0.412548in}{4.268405in}}%
\pgfpathcurveto{\pgfqpoint{0.401498in}{4.268405in}}{\pgfqpoint{0.390899in}{4.264015in}}{\pgfqpoint{0.383085in}{4.256201in}}%
\pgfpathcurveto{\pgfqpoint{0.375271in}{4.248387in}}{\pgfqpoint{0.370881in}{4.237788in}}{\pgfqpoint{0.370881in}{4.226738in}}%
\pgfpathcurveto{\pgfqpoint{0.370881in}{4.215688in}}{\pgfqpoint{0.375271in}{4.205089in}}{\pgfqpoint{0.383085in}{4.197275in}}%
\pgfpathcurveto{\pgfqpoint{0.390899in}{4.189462in}}{\pgfqpoint{0.401498in}{4.185071in}}{\pgfqpoint{0.412548in}{4.185071in}}%
\pgfpathlineto{\pgfqpoint{0.412548in}{4.185071in}}%
\pgfpathclose%
\pgfusepath{stroke}%
\end{pgfscope}%
\begin{pgfscope}%
\pgfpathrectangle{\pgfqpoint{0.393053in}{0.375000in}}{\pgfqpoint{6.356833in}{5.175000in}}%
\pgfusepath{clip}%
\pgfsetbuttcap%
\pgfsetroundjoin%
\pgfsetlinewidth{1.003750pt}%
\definecolor{currentstroke}{rgb}{0.827451,0.827451,0.827451}%
\pgfsetstrokecolor{currentstroke}%
\pgfsetdash{}{0pt}%
\pgfpathmoveto{\pgfqpoint{1.040104in}{2.359499in}}%
\pgfpathcurveto{\pgfqpoint{1.051154in}{2.359499in}}{\pgfqpoint{1.061753in}{2.363889in}}{\pgfqpoint{1.069567in}{2.371703in}}%
\pgfpathcurveto{\pgfqpoint{1.077381in}{2.379516in}}{\pgfqpoint{1.081771in}{2.390115in}}{\pgfqpoint{1.081771in}{2.401165in}}%
\pgfpathcurveto{\pgfqpoint{1.081771in}{2.412215in}}{\pgfqpoint{1.077381in}{2.422814in}}{\pgfqpoint{1.069567in}{2.430628in}}%
\pgfpathcurveto{\pgfqpoint{1.061753in}{2.438442in}}{\pgfqpoint{1.051154in}{2.442832in}}{\pgfqpoint{1.040104in}{2.442832in}}%
\pgfpathcurveto{\pgfqpoint{1.029054in}{2.442832in}}{\pgfqpoint{1.018455in}{2.438442in}}{\pgfqpoint{1.010641in}{2.430628in}}%
\pgfpathcurveto{\pgfqpoint{1.002828in}{2.422814in}}{\pgfqpoint{0.998437in}{2.412215in}}{\pgfqpoint{0.998437in}{2.401165in}}%
\pgfpathcurveto{\pgfqpoint{0.998437in}{2.390115in}}{\pgfqpoint{1.002828in}{2.379516in}}{\pgfqpoint{1.010641in}{2.371703in}}%
\pgfpathcurveto{\pgfqpoint{1.018455in}{2.363889in}}{\pgfqpoint{1.029054in}{2.359499in}}{\pgfqpoint{1.040104in}{2.359499in}}%
\pgfpathlineto{\pgfqpoint{1.040104in}{2.359499in}}%
\pgfpathclose%
\pgfusepath{stroke}%
\end{pgfscope}%
\begin{pgfscope}%
\pgfpathrectangle{\pgfqpoint{0.393053in}{0.375000in}}{\pgfqpoint{6.356833in}{5.175000in}}%
\pgfusepath{clip}%
\pgfsetbuttcap%
\pgfsetroundjoin%
\pgfsetlinewidth{1.003750pt}%
\definecolor{currentstroke}{rgb}{0.827451,0.827451,0.827451}%
\pgfsetstrokecolor{currentstroke}%
\pgfsetdash{}{0pt}%
\pgfpathmoveto{\pgfqpoint{2.000452in}{1.370955in}}%
\pgfpathcurveto{\pgfqpoint{2.011502in}{1.370955in}}{\pgfqpoint{2.022101in}{1.375345in}}{\pgfqpoint{2.029914in}{1.383159in}}%
\pgfpathcurveto{\pgfqpoint{2.037728in}{1.390972in}}{\pgfqpoint{2.042118in}{1.401571in}}{\pgfqpoint{2.042118in}{1.412622in}}%
\pgfpathcurveto{\pgfqpoint{2.042118in}{1.423672in}}{\pgfqpoint{2.037728in}{1.434271in}}{\pgfqpoint{2.029914in}{1.442084in}}%
\pgfpathcurveto{\pgfqpoint{2.022101in}{1.449898in}}{\pgfqpoint{2.011502in}{1.454288in}}{\pgfqpoint{2.000452in}{1.454288in}}%
\pgfpathcurveto{\pgfqpoint{1.989401in}{1.454288in}}{\pgfqpoint{1.978802in}{1.449898in}}{\pgfqpoint{1.970989in}{1.442084in}}%
\pgfpathcurveto{\pgfqpoint{1.963175in}{1.434271in}}{\pgfqpoint{1.958785in}{1.423672in}}{\pgfqpoint{1.958785in}{1.412622in}}%
\pgfpathcurveto{\pgfqpoint{1.958785in}{1.401571in}}{\pgfqpoint{1.963175in}{1.390972in}}{\pgfqpoint{1.970989in}{1.383159in}}%
\pgfpathcurveto{\pgfqpoint{1.978802in}{1.375345in}}{\pgfqpoint{1.989401in}{1.370955in}}{\pgfqpoint{2.000452in}{1.370955in}}%
\pgfpathlineto{\pgfqpoint{2.000452in}{1.370955in}}%
\pgfpathclose%
\pgfusepath{stroke}%
\end{pgfscope}%
\begin{pgfscope}%
\pgfpathrectangle{\pgfqpoint{0.393053in}{0.375000in}}{\pgfqpoint{6.356833in}{5.175000in}}%
\pgfusepath{clip}%
\pgfsetbuttcap%
\pgfsetroundjoin%
\pgfsetlinewidth{1.003750pt}%
\definecolor{currentstroke}{rgb}{0.827451,0.827451,0.827451}%
\pgfsetstrokecolor{currentstroke}%
\pgfsetdash{}{0pt}%
\pgfpathmoveto{\pgfqpoint{2.230001in}{1.209659in}}%
\pgfpathcurveto{\pgfqpoint{2.241051in}{1.209659in}}{\pgfqpoint{2.251650in}{1.214049in}}{\pgfqpoint{2.259463in}{1.221862in}}%
\pgfpathcurveto{\pgfqpoint{2.267277in}{1.229676in}}{\pgfqpoint{2.271667in}{1.240275in}}{\pgfqpoint{2.271667in}{1.251325in}}%
\pgfpathcurveto{\pgfqpoint{2.271667in}{1.262375in}}{\pgfqpoint{2.267277in}{1.272974in}}{\pgfqpoint{2.259463in}{1.280788in}}%
\pgfpathcurveto{\pgfqpoint{2.251650in}{1.288602in}}{\pgfqpoint{2.241051in}{1.292992in}}{\pgfqpoint{2.230001in}{1.292992in}}%
\pgfpathcurveto{\pgfqpoint{2.218950in}{1.292992in}}{\pgfqpoint{2.208351in}{1.288602in}}{\pgfqpoint{2.200538in}{1.280788in}}%
\pgfpathcurveto{\pgfqpoint{2.192724in}{1.272974in}}{\pgfqpoint{2.188334in}{1.262375in}}{\pgfqpoint{2.188334in}{1.251325in}}%
\pgfpathcurveto{\pgfqpoint{2.188334in}{1.240275in}}{\pgfqpoint{2.192724in}{1.229676in}}{\pgfqpoint{2.200538in}{1.221862in}}%
\pgfpathcurveto{\pgfqpoint{2.208351in}{1.214049in}}{\pgfqpoint{2.218950in}{1.209659in}}{\pgfqpoint{2.230001in}{1.209659in}}%
\pgfpathlineto{\pgfqpoint{2.230001in}{1.209659in}}%
\pgfpathclose%
\pgfusepath{stroke}%
\end{pgfscope}%
\begin{pgfscope}%
\pgfpathrectangle{\pgfqpoint{0.393053in}{0.375000in}}{\pgfqpoint{6.356833in}{5.175000in}}%
\pgfusepath{clip}%
\pgfsetbuttcap%
\pgfsetroundjoin%
\pgfsetlinewidth{1.003750pt}%
\definecolor{currentstroke}{rgb}{0.827451,0.827451,0.827451}%
\pgfsetstrokecolor{currentstroke}%
\pgfsetdash{}{0pt}%
\pgfpathmoveto{\pgfqpoint{1.658477in}{1.641127in}}%
\pgfpathcurveto{\pgfqpoint{1.669527in}{1.641127in}}{\pgfqpoint{1.680126in}{1.645517in}}{\pgfqpoint{1.687939in}{1.653331in}}%
\pgfpathcurveto{\pgfqpoint{1.695753in}{1.661144in}}{\pgfqpoint{1.700143in}{1.671743in}}{\pgfqpoint{1.700143in}{1.682794in}}%
\pgfpathcurveto{\pgfqpoint{1.700143in}{1.693844in}}{\pgfqpoint{1.695753in}{1.704443in}}{\pgfqpoint{1.687939in}{1.712256in}}%
\pgfpathcurveto{\pgfqpoint{1.680126in}{1.720070in}}{\pgfqpoint{1.669527in}{1.724460in}}{\pgfqpoint{1.658477in}{1.724460in}}%
\pgfpathcurveto{\pgfqpoint{1.647427in}{1.724460in}}{\pgfqpoint{1.636828in}{1.720070in}}{\pgfqpoint{1.629014in}{1.712256in}}%
\pgfpathcurveto{\pgfqpoint{1.621200in}{1.704443in}}{\pgfqpoint{1.616810in}{1.693844in}}{\pgfqpoint{1.616810in}{1.682794in}}%
\pgfpathcurveto{\pgfqpoint{1.616810in}{1.671743in}}{\pgfqpoint{1.621200in}{1.661144in}}{\pgfqpoint{1.629014in}{1.653331in}}%
\pgfpathcurveto{\pgfqpoint{1.636828in}{1.645517in}}{\pgfqpoint{1.647427in}{1.641127in}}{\pgfqpoint{1.658477in}{1.641127in}}%
\pgfpathlineto{\pgfqpoint{1.658477in}{1.641127in}}%
\pgfpathclose%
\pgfusepath{stroke}%
\end{pgfscope}%
\begin{pgfscope}%
\pgfpathrectangle{\pgfqpoint{0.393053in}{0.375000in}}{\pgfqpoint{6.356833in}{5.175000in}}%
\pgfusepath{clip}%
\pgfsetbuttcap%
\pgfsetroundjoin%
\pgfsetlinewidth{1.003750pt}%
\definecolor{currentstroke}{rgb}{0.827451,0.827451,0.827451}%
\pgfsetstrokecolor{currentstroke}%
\pgfsetdash{}{0pt}%
\pgfpathmoveto{\pgfqpoint{3.156562in}{0.749138in}}%
\pgfpathcurveto{\pgfqpoint{3.167612in}{0.749138in}}{\pgfqpoint{3.178211in}{0.753528in}}{\pgfqpoint{3.186024in}{0.761342in}}%
\pgfpathcurveto{\pgfqpoint{3.193838in}{0.769155in}}{\pgfqpoint{3.198228in}{0.779754in}}{\pgfqpoint{3.198228in}{0.790804in}}%
\pgfpathcurveto{\pgfqpoint{3.198228in}{0.801855in}}{\pgfqpoint{3.193838in}{0.812454in}}{\pgfqpoint{3.186024in}{0.820267in}}%
\pgfpathcurveto{\pgfqpoint{3.178211in}{0.828081in}}{\pgfqpoint{3.167612in}{0.832471in}}{\pgfqpoint{3.156562in}{0.832471in}}%
\pgfpathcurveto{\pgfqpoint{3.145512in}{0.832471in}}{\pgfqpoint{3.134913in}{0.828081in}}{\pgfqpoint{3.127099in}{0.820267in}}%
\pgfpathcurveto{\pgfqpoint{3.119285in}{0.812454in}}{\pgfqpoint{3.114895in}{0.801855in}}{\pgfqpoint{3.114895in}{0.790804in}}%
\pgfpathcurveto{\pgfqpoint{3.114895in}{0.779754in}}{\pgfqpoint{3.119285in}{0.769155in}}{\pgfqpoint{3.127099in}{0.761342in}}%
\pgfpathcurveto{\pgfqpoint{3.134913in}{0.753528in}}{\pgfqpoint{3.145512in}{0.749138in}}{\pgfqpoint{3.156562in}{0.749138in}}%
\pgfpathlineto{\pgfqpoint{3.156562in}{0.749138in}}%
\pgfpathclose%
\pgfusepath{stroke}%
\end{pgfscope}%
\begin{pgfscope}%
\pgfpathrectangle{\pgfqpoint{0.393053in}{0.375000in}}{\pgfqpoint{6.356833in}{5.175000in}}%
\pgfusepath{clip}%
\pgfsetbuttcap%
\pgfsetroundjoin%
\pgfsetlinewidth{1.003750pt}%
\definecolor{currentstroke}{rgb}{0.827451,0.827451,0.827451}%
\pgfsetstrokecolor{currentstroke}%
\pgfsetdash{}{0pt}%
\pgfpathmoveto{\pgfqpoint{0.707419in}{2.963452in}}%
\pgfpathcurveto{\pgfqpoint{0.718469in}{2.963452in}}{\pgfqpoint{0.729068in}{2.967843in}}{\pgfqpoint{0.736882in}{2.975656in}}%
\pgfpathcurveto{\pgfqpoint{0.744695in}{2.983470in}}{\pgfqpoint{0.749086in}{2.994069in}}{\pgfqpoint{0.749086in}{3.005119in}}%
\pgfpathcurveto{\pgfqpoint{0.749086in}{3.016169in}}{\pgfqpoint{0.744695in}{3.026768in}}{\pgfqpoint{0.736882in}{3.034582in}}%
\pgfpathcurveto{\pgfqpoint{0.729068in}{3.042395in}}{\pgfqpoint{0.718469in}{3.046786in}}{\pgfqpoint{0.707419in}{3.046786in}}%
\pgfpathcurveto{\pgfqpoint{0.696369in}{3.046786in}}{\pgfqpoint{0.685770in}{3.042395in}}{\pgfqpoint{0.677956in}{3.034582in}}%
\pgfpathcurveto{\pgfqpoint{0.670142in}{3.026768in}}{\pgfqpoint{0.665752in}{3.016169in}}{\pgfqpoint{0.665752in}{3.005119in}}%
\pgfpathcurveto{\pgfqpoint{0.665752in}{2.994069in}}{\pgfqpoint{0.670142in}{2.983470in}}{\pgfqpoint{0.677956in}{2.975656in}}%
\pgfpathcurveto{\pgfqpoint{0.685770in}{2.967843in}}{\pgfqpoint{0.696369in}{2.963452in}}{\pgfqpoint{0.707419in}{2.963452in}}%
\pgfpathlineto{\pgfqpoint{0.707419in}{2.963452in}}%
\pgfpathclose%
\pgfusepath{stroke}%
\end{pgfscope}%
\begin{pgfscope}%
\pgfpathrectangle{\pgfqpoint{0.393053in}{0.375000in}}{\pgfqpoint{6.356833in}{5.175000in}}%
\pgfusepath{clip}%
\pgfsetbuttcap%
\pgfsetroundjoin%
\pgfsetlinewidth{1.003750pt}%
\definecolor{currentstroke}{rgb}{0.827451,0.827451,0.827451}%
\pgfsetstrokecolor{currentstroke}%
\pgfsetdash{}{0pt}%
\pgfpathmoveto{\pgfqpoint{1.485171in}{1.799332in}}%
\pgfpathcurveto{\pgfqpoint{1.496221in}{1.799332in}}{\pgfqpoint{1.506820in}{1.803723in}}{\pgfqpoint{1.514634in}{1.811536in}}%
\pgfpathcurveto{\pgfqpoint{1.522447in}{1.819350in}}{\pgfqpoint{1.526838in}{1.829949in}}{\pgfqpoint{1.526838in}{1.840999in}}%
\pgfpathcurveto{\pgfqpoint{1.526838in}{1.852049in}}{\pgfqpoint{1.522447in}{1.862648in}}{\pgfqpoint{1.514634in}{1.870462in}}%
\pgfpathcurveto{\pgfqpoint{1.506820in}{1.878275in}}{\pgfqpoint{1.496221in}{1.882666in}}{\pgfqpoint{1.485171in}{1.882666in}}%
\pgfpathcurveto{\pgfqpoint{1.474121in}{1.882666in}}{\pgfqpoint{1.463522in}{1.878275in}}{\pgfqpoint{1.455708in}{1.870462in}}%
\pgfpathcurveto{\pgfqpoint{1.447895in}{1.862648in}}{\pgfqpoint{1.443504in}{1.852049in}}{\pgfqpoint{1.443504in}{1.840999in}}%
\pgfpathcurveto{\pgfqpoint{1.443504in}{1.829949in}}{\pgfqpoint{1.447895in}{1.819350in}}{\pgfqpoint{1.455708in}{1.811536in}}%
\pgfpathcurveto{\pgfqpoint{1.463522in}{1.803723in}}{\pgfqpoint{1.474121in}{1.799332in}}{\pgfqpoint{1.485171in}{1.799332in}}%
\pgfpathlineto{\pgfqpoint{1.485171in}{1.799332in}}%
\pgfpathclose%
\pgfusepath{stroke}%
\end{pgfscope}%
\begin{pgfscope}%
\pgfpathrectangle{\pgfqpoint{0.393053in}{0.375000in}}{\pgfqpoint{6.356833in}{5.175000in}}%
\pgfusepath{clip}%
\pgfsetbuttcap%
\pgfsetroundjoin%
\pgfsetlinewidth{1.003750pt}%
\definecolor{currentstroke}{rgb}{0.827451,0.827451,0.827451}%
\pgfsetstrokecolor{currentstroke}%
\pgfsetdash{}{0pt}%
\pgfpathmoveto{\pgfqpoint{0.483847in}{3.664437in}}%
\pgfpathcurveto{\pgfqpoint{0.494897in}{3.664437in}}{\pgfqpoint{0.505496in}{3.668827in}}{\pgfqpoint{0.513309in}{3.676640in}}%
\pgfpathcurveto{\pgfqpoint{0.521123in}{3.684454in}}{\pgfqpoint{0.525513in}{3.695053in}}{\pgfqpoint{0.525513in}{3.706103in}}%
\pgfpathcurveto{\pgfqpoint{0.525513in}{3.717153in}}{\pgfqpoint{0.521123in}{3.727752in}}{\pgfqpoint{0.513309in}{3.735566in}}%
\pgfpathcurveto{\pgfqpoint{0.505496in}{3.743380in}}{\pgfqpoint{0.494897in}{3.747770in}}{\pgfqpoint{0.483847in}{3.747770in}}%
\pgfpathcurveto{\pgfqpoint{0.472797in}{3.747770in}}{\pgfqpoint{0.462198in}{3.743380in}}{\pgfqpoint{0.454384in}{3.735566in}}%
\pgfpathcurveto{\pgfqpoint{0.446570in}{3.727752in}}{\pgfqpoint{0.442180in}{3.717153in}}{\pgfqpoint{0.442180in}{3.706103in}}%
\pgfpathcurveto{\pgfqpoint{0.442180in}{3.695053in}}{\pgfqpoint{0.446570in}{3.684454in}}{\pgfqpoint{0.454384in}{3.676640in}}%
\pgfpathcurveto{\pgfqpoint{0.462198in}{3.668827in}}{\pgfqpoint{0.472797in}{3.664437in}}{\pgfqpoint{0.483847in}{3.664437in}}%
\pgfpathlineto{\pgfqpoint{0.483847in}{3.664437in}}%
\pgfpathclose%
\pgfusepath{stroke}%
\end{pgfscope}%
\begin{pgfscope}%
\pgfpathrectangle{\pgfqpoint{0.393053in}{0.375000in}}{\pgfqpoint{6.356833in}{5.175000in}}%
\pgfusepath{clip}%
\pgfsetbuttcap%
\pgfsetroundjoin%
\pgfsetlinewidth{1.003750pt}%
\definecolor{currentstroke}{rgb}{0.827451,0.827451,0.827451}%
\pgfsetstrokecolor{currentstroke}%
\pgfsetdash{}{0pt}%
\pgfpathmoveto{\pgfqpoint{0.612319in}{3.180269in}}%
\pgfpathcurveto{\pgfqpoint{0.623369in}{3.180269in}}{\pgfqpoint{0.633968in}{3.184660in}}{\pgfqpoint{0.641781in}{3.192473in}}%
\pgfpathcurveto{\pgfqpoint{0.649595in}{3.200287in}}{\pgfqpoint{0.653985in}{3.210886in}}{\pgfqpoint{0.653985in}{3.221936in}}%
\pgfpathcurveto{\pgfqpoint{0.653985in}{3.232986in}}{\pgfqpoint{0.649595in}{3.243585in}}{\pgfqpoint{0.641781in}{3.251399in}}%
\pgfpathcurveto{\pgfqpoint{0.633968in}{3.259212in}}{\pgfqpoint{0.623369in}{3.263603in}}{\pgfqpoint{0.612319in}{3.263603in}}%
\pgfpathcurveto{\pgfqpoint{0.601269in}{3.263603in}}{\pgfqpoint{0.590669in}{3.259212in}}{\pgfqpoint{0.582856in}{3.251399in}}%
\pgfpathcurveto{\pgfqpoint{0.575042in}{3.243585in}}{\pgfqpoint{0.570652in}{3.232986in}}{\pgfqpoint{0.570652in}{3.221936in}}%
\pgfpathcurveto{\pgfqpoint{0.570652in}{3.210886in}}{\pgfqpoint{0.575042in}{3.200287in}}{\pgfqpoint{0.582856in}{3.192473in}}%
\pgfpathcurveto{\pgfqpoint{0.590669in}{3.184660in}}{\pgfqpoint{0.601269in}{3.180269in}}{\pgfqpoint{0.612319in}{3.180269in}}%
\pgfpathlineto{\pgfqpoint{0.612319in}{3.180269in}}%
\pgfpathclose%
\pgfusepath{stroke}%
\end{pgfscope}%
\begin{pgfscope}%
\pgfpathrectangle{\pgfqpoint{0.393053in}{0.375000in}}{\pgfqpoint{6.356833in}{5.175000in}}%
\pgfusepath{clip}%
\pgfsetbuttcap%
\pgfsetroundjoin%
\pgfsetlinewidth{1.003750pt}%
\definecolor{currentstroke}{rgb}{0.827451,0.827451,0.827451}%
\pgfsetstrokecolor{currentstroke}%
\pgfsetdash{}{0pt}%
\pgfpathmoveto{\pgfqpoint{4.655370in}{0.409439in}}%
\pgfpathcurveto{\pgfqpoint{4.666420in}{0.409439in}}{\pgfqpoint{4.677019in}{0.413829in}}{\pgfqpoint{4.684833in}{0.421642in}}%
\pgfpathcurveto{\pgfqpoint{4.692646in}{0.429456in}}{\pgfqpoint{4.697036in}{0.440055in}}{\pgfqpoint{4.697036in}{0.451105in}}%
\pgfpathcurveto{\pgfqpoint{4.697036in}{0.462155in}}{\pgfqpoint{4.692646in}{0.472754in}}{\pgfqpoint{4.684833in}{0.480568in}}%
\pgfpathcurveto{\pgfqpoint{4.677019in}{0.488382in}}{\pgfqpoint{4.666420in}{0.492772in}}{\pgfqpoint{4.655370in}{0.492772in}}%
\pgfpathcurveto{\pgfqpoint{4.644320in}{0.492772in}}{\pgfqpoint{4.633721in}{0.488382in}}{\pgfqpoint{4.625907in}{0.480568in}}%
\pgfpathcurveto{\pgfqpoint{4.618093in}{0.472754in}}{\pgfqpoint{4.613703in}{0.462155in}}{\pgfqpoint{4.613703in}{0.451105in}}%
\pgfpathcurveto{\pgfqpoint{4.613703in}{0.440055in}}{\pgfqpoint{4.618093in}{0.429456in}}{\pgfqpoint{4.625907in}{0.421642in}}%
\pgfpathcurveto{\pgfqpoint{4.633721in}{0.413829in}}{\pgfqpoint{4.644320in}{0.409439in}}{\pgfqpoint{4.655370in}{0.409439in}}%
\pgfpathlineto{\pgfqpoint{4.655370in}{0.409439in}}%
\pgfpathclose%
\pgfusepath{stroke}%
\end{pgfscope}%
\begin{pgfscope}%
\pgfpathrectangle{\pgfqpoint{0.393053in}{0.375000in}}{\pgfqpoint{6.356833in}{5.175000in}}%
\pgfusepath{clip}%
\pgfsetbuttcap%
\pgfsetroundjoin%
\pgfsetlinewidth{1.003750pt}%
\definecolor{currentstroke}{rgb}{0.827451,0.827451,0.827451}%
\pgfsetstrokecolor{currentstroke}%
\pgfsetdash{}{0pt}%
\pgfpathmoveto{\pgfqpoint{0.411258in}{4.233714in}}%
\pgfpathcurveto{\pgfqpoint{0.422308in}{4.233714in}}{\pgfqpoint{0.432907in}{4.238104in}}{\pgfqpoint{0.440721in}{4.245918in}}%
\pgfpathcurveto{\pgfqpoint{0.448534in}{4.253732in}}{\pgfqpoint{0.452925in}{4.264331in}}{\pgfqpoint{0.452925in}{4.275381in}}%
\pgfpathcurveto{\pgfqpoint{0.452925in}{4.286431in}}{\pgfqpoint{0.448534in}{4.297030in}}{\pgfqpoint{0.440721in}{4.304844in}}%
\pgfpathcurveto{\pgfqpoint{0.432907in}{4.312657in}}{\pgfqpoint{0.422308in}{4.317048in}}{\pgfqpoint{0.411258in}{4.317048in}}%
\pgfpathcurveto{\pgfqpoint{0.400208in}{4.317048in}}{\pgfqpoint{0.389609in}{4.312657in}}{\pgfqpoint{0.381795in}{4.304844in}}%
\pgfpathcurveto{\pgfqpoint{0.373982in}{4.297030in}}{\pgfqpoint{0.369591in}{4.286431in}}{\pgfqpoint{0.369591in}{4.275381in}}%
\pgfpathcurveto{\pgfqpoint{0.369591in}{4.264331in}}{\pgfqpoint{0.373982in}{4.253732in}}{\pgfqpoint{0.381795in}{4.245918in}}%
\pgfpathcurveto{\pgfqpoint{0.389609in}{4.238104in}}{\pgfqpoint{0.400208in}{4.233714in}}{\pgfqpoint{0.411258in}{4.233714in}}%
\pgfpathlineto{\pgfqpoint{0.411258in}{4.233714in}}%
\pgfpathclose%
\pgfusepath{stroke}%
\end{pgfscope}%
\begin{pgfscope}%
\pgfpathrectangle{\pgfqpoint{0.393053in}{0.375000in}}{\pgfqpoint{6.356833in}{5.175000in}}%
\pgfusepath{clip}%
\pgfsetbuttcap%
\pgfsetroundjoin%
\pgfsetlinewidth{1.003750pt}%
\definecolor{currentstroke}{rgb}{0.827451,0.827451,0.827451}%
\pgfsetstrokecolor{currentstroke}%
\pgfsetdash{}{0pt}%
\pgfpathmoveto{\pgfqpoint{0.459942in}{3.789224in}}%
\pgfpathcurveto{\pgfqpoint{0.470992in}{3.789224in}}{\pgfqpoint{0.481591in}{3.793614in}}{\pgfqpoint{0.489404in}{3.801428in}}%
\pgfpathcurveto{\pgfqpoint{0.497218in}{3.809242in}}{\pgfqpoint{0.501608in}{3.819841in}}{\pgfqpoint{0.501608in}{3.830891in}}%
\pgfpathcurveto{\pgfqpoint{0.501608in}{3.841941in}}{\pgfqpoint{0.497218in}{3.852540in}}{\pgfqpoint{0.489404in}{3.860354in}}%
\pgfpathcurveto{\pgfqpoint{0.481591in}{3.868167in}}{\pgfqpoint{0.470992in}{3.872557in}}{\pgfqpoint{0.459942in}{3.872557in}}%
\pgfpathcurveto{\pgfqpoint{0.448891in}{3.872557in}}{\pgfqpoint{0.438292in}{3.868167in}}{\pgfqpoint{0.430479in}{3.860354in}}%
\pgfpathcurveto{\pgfqpoint{0.422665in}{3.852540in}}{\pgfqpoint{0.418275in}{3.841941in}}{\pgfqpoint{0.418275in}{3.830891in}}%
\pgfpathcurveto{\pgfqpoint{0.418275in}{3.819841in}}{\pgfqpoint{0.422665in}{3.809242in}}{\pgfqpoint{0.430479in}{3.801428in}}%
\pgfpathcurveto{\pgfqpoint{0.438292in}{3.793614in}}{\pgfqpoint{0.448891in}{3.789224in}}{\pgfqpoint{0.459942in}{3.789224in}}%
\pgfpathlineto{\pgfqpoint{0.459942in}{3.789224in}}%
\pgfpathclose%
\pgfusepath{stroke}%
\end{pgfscope}%
\begin{pgfscope}%
\pgfpathrectangle{\pgfqpoint{0.393053in}{0.375000in}}{\pgfqpoint{6.356833in}{5.175000in}}%
\pgfusepath{clip}%
\pgfsetbuttcap%
\pgfsetroundjoin%
\pgfsetlinewidth{1.003750pt}%
\definecolor{currentstroke}{rgb}{0.827451,0.827451,0.827451}%
\pgfsetstrokecolor{currentstroke}%
\pgfsetdash{}{0pt}%
\pgfpathmoveto{\pgfqpoint{1.354968in}{1.928632in}}%
\pgfpathcurveto{\pgfqpoint{1.366018in}{1.928632in}}{\pgfqpoint{1.376617in}{1.933022in}}{\pgfqpoint{1.384431in}{1.940836in}}%
\pgfpathcurveto{\pgfqpoint{1.392245in}{1.948650in}}{\pgfqpoint{1.396635in}{1.959249in}}{\pgfqpoint{1.396635in}{1.970299in}}%
\pgfpathcurveto{\pgfqpoint{1.396635in}{1.981349in}}{\pgfqpoint{1.392245in}{1.991948in}}{\pgfqpoint{1.384431in}{1.999762in}}%
\pgfpathcurveto{\pgfqpoint{1.376617in}{2.007575in}}{\pgfqpoint{1.366018in}{2.011966in}}{\pgfqpoint{1.354968in}{2.011966in}}%
\pgfpathcurveto{\pgfqpoint{1.343918in}{2.011966in}}{\pgfqpoint{1.333319in}{2.007575in}}{\pgfqpoint{1.325505in}{1.999762in}}%
\pgfpathcurveto{\pgfqpoint{1.317692in}{1.991948in}}{\pgfqpoint{1.313302in}{1.981349in}}{\pgfqpoint{1.313302in}{1.970299in}}%
\pgfpathcurveto{\pgfqpoint{1.313302in}{1.959249in}}{\pgfqpoint{1.317692in}{1.948650in}}{\pgfqpoint{1.325505in}{1.940836in}}%
\pgfpathcurveto{\pgfqpoint{1.333319in}{1.933022in}}{\pgfqpoint{1.343918in}{1.928632in}}{\pgfqpoint{1.354968in}{1.928632in}}%
\pgfpathlineto{\pgfqpoint{1.354968in}{1.928632in}}%
\pgfpathclose%
\pgfusepath{stroke}%
\end{pgfscope}%
\begin{pgfscope}%
\pgfpathrectangle{\pgfqpoint{0.393053in}{0.375000in}}{\pgfqpoint{6.356833in}{5.175000in}}%
\pgfusepath{clip}%
\pgfsetbuttcap%
\pgfsetroundjoin%
\pgfsetlinewidth{1.003750pt}%
\definecolor{currentstroke}{rgb}{0.827451,0.827451,0.827451}%
\pgfsetstrokecolor{currentstroke}%
\pgfsetdash{}{0pt}%
\pgfpathmoveto{\pgfqpoint{0.452170in}{3.855614in}}%
\pgfpathcurveto{\pgfqpoint{0.463220in}{3.855614in}}{\pgfqpoint{0.473819in}{3.860005in}}{\pgfqpoint{0.481633in}{3.867818in}}%
\pgfpathcurveto{\pgfqpoint{0.489447in}{3.875632in}}{\pgfqpoint{0.493837in}{3.886231in}}{\pgfqpoint{0.493837in}{3.897281in}}%
\pgfpathcurveto{\pgfqpoint{0.493837in}{3.908331in}}{\pgfqpoint{0.489447in}{3.918930in}}{\pgfqpoint{0.481633in}{3.926744in}}%
\pgfpathcurveto{\pgfqpoint{0.473819in}{3.934557in}}{\pgfqpoint{0.463220in}{3.938948in}}{\pgfqpoint{0.452170in}{3.938948in}}%
\pgfpathcurveto{\pgfqpoint{0.441120in}{3.938948in}}{\pgfqpoint{0.430521in}{3.934557in}}{\pgfqpoint{0.422707in}{3.926744in}}%
\pgfpathcurveto{\pgfqpoint{0.414894in}{3.918930in}}{\pgfqpoint{0.410503in}{3.908331in}}{\pgfqpoint{0.410503in}{3.897281in}}%
\pgfpathcurveto{\pgfqpoint{0.410503in}{3.886231in}}{\pgfqpoint{0.414894in}{3.875632in}}{\pgfqpoint{0.422707in}{3.867818in}}%
\pgfpathcurveto{\pgfqpoint{0.430521in}{3.860005in}}{\pgfqpoint{0.441120in}{3.855614in}}{\pgfqpoint{0.452170in}{3.855614in}}%
\pgfpathlineto{\pgfqpoint{0.452170in}{3.855614in}}%
\pgfpathclose%
\pgfusepath{stroke}%
\end{pgfscope}%
\begin{pgfscope}%
\pgfpathrectangle{\pgfqpoint{0.393053in}{0.375000in}}{\pgfqpoint{6.356833in}{5.175000in}}%
\pgfusepath{clip}%
\pgfsetbuttcap%
\pgfsetroundjoin%
\pgfsetlinewidth{1.003750pt}%
\definecolor{currentstroke}{rgb}{0.827451,0.827451,0.827451}%
\pgfsetstrokecolor{currentstroke}%
\pgfsetdash{}{0pt}%
\pgfpathmoveto{\pgfqpoint{2.479677in}{1.062927in}}%
\pgfpathcurveto{\pgfqpoint{2.490727in}{1.062927in}}{\pgfqpoint{2.501326in}{1.067318in}}{\pgfqpoint{2.509140in}{1.075131in}}%
\pgfpathcurveto{\pgfqpoint{2.516954in}{1.082945in}}{\pgfqpoint{2.521344in}{1.093544in}}{\pgfqpoint{2.521344in}{1.104594in}}%
\pgfpathcurveto{\pgfqpoint{2.521344in}{1.115644in}}{\pgfqpoint{2.516954in}{1.126243in}}{\pgfqpoint{2.509140in}{1.134057in}}%
\pgfpathcurveto{\pgfqpoint{2.501326in}{1.141870in}}{\pgfqpoint{2.490727in}{1.146261in}}{\pgfqpoint{2.479677in}{1.146261in}}%
\pgfpathcurveto{\pgfqpoint{2.468627in}{1.146261in}}{\pgfqpoint{2.458028in}{1.141870in}}{\pgfqpoint{2.450214in}{1.134057in}}%
\pgfpathcurveto{\pgfqpoint{2.442401in}{1.126243in}}{\pgfqpoint{2.438011in}{1.115644in}}{\pgfqpoint{2.438011in}{1.104594in}}%
\pgfpathcurveto{\pgfqpoint{2.438011in}{1.093544in}}{\pgfqpoint{2.442401in}{1.082945in}}{\pgfqpoint{2.450214in}{1.075131in}}%
\pgfpathcurveto{\pgfqpoint{2.458028in}{1.067318in}}{\pgfqpoint{2.468627in}{1.062927in}}{\pgfqpoint{2.479677in}{1.062927in}}%
\pgfpathlineto{\pgfqpoint{2.479677in}{1.062927in}}%
\pgfpathclose%
\pgfusepath{stroke}%
\end{pgfscope}%
\begin{pgfscope}%
\pgfpathrectangle{\pgfqpoint{0.393053in}{0.375000in}}{\pgfqpoint{6.356833in}{5.175000in}}%
\pgfusepath{clip}%
\pgfsetbuttcap%
\pgfsetroundjoin%
\pgfsetlinewidth{1.003750pt}%
\definecolor{currentstroke}{rgb}{0.827451,0.827451,0.827451}%
\pgfsetstrokecolor{currentstroke}%
\pgfsetdash{}{0pt}%
\pgfpathmoveto{\pgfqpoint{2.713290in}{0.943005in}}%
\pgfpathcurveto{\pgfqpoint{2.724341in}{0.943005in}}{\pgfqpoint{2.734940in}{0.947396in}}{\pgfqpoint{2.742753in}{0.955209in}}%
\pgfpathcurveto{\pgfqpoint{2.750567in}{0.963023in}}{\pgfqpoint{2.754957in}{0.973622in}}{\pgfqpoint{2.754957in}{0.984672in}}%
\pgfpathcurveto{\pgfqpoint{2.754957in}{0.995722in}}{\pgfqpoint{2.750567in}{1.006321in}}{\pgfqpoint{2.742753in}{1.014135in}}%
\pgfpathcurveto{\pgfqpoint{2.734940in}{1.021949in}}{\pgfqpoint{2.724341in}{1.026339in}}{\pgfqpoint{2.713290in}{1.026339in}}%
\pgfpathcurveto{\pgfqpoint{2.702240in}{1.026339in}}{\pgfqpoint{2.691641in}{1.021949in}}{\pgfqpoint{2.683828in}{1.014135in}}%
\pgfpathcurveto{\pgfqpoint{2.676014in}{1.006321in}}{\pgfqpoint{2.671624in}{0.995722in}}{\pgfqpoint{2.671624in}{0.984672in}}%
\pgfpathcurveto{\pgfqpoint{2.671624in}{0.973622in}}{\pgfqpoint{2.676014in}{0.963023in}}{\pgfqpoint{2.683828in}{0.955209in}}%
\pgfpathcurveto{\pgfqpoint{2.691641in}{0.947396in}}{\pgfqpoint{2.702240in}{0.943005in}}{\pgfqpoint{2.713290in}{0.943005in}}%
\pgfpathlineto{\pgfqpoint{2.713290in}{0.943005in}}%
\pgfpathclose%
\pgfusepath{stroke}%
\end{pgfscope}%
\begin{pgfscope}%
\pgfpathrectangle{\pgfqpoint{0.393053in}{0.375000in}}{\pgfqpoint{6.356833in}{5.175000in}}%
\pgfusepath{clip}%
\pgfsetbuttcap%
\pgfsetroundjoin%
\pgfsetlinewidth{1.003750pt}%
\definecolor{currentstroke}{rgb}{0.827451,0.827451,0.827451}%
\pgfsetstrokecolor{currentstroke}%
\pgfsetdash{}{0pt}%
\pgfpathmoveto{\pgfqpoint{0.399634in}{4.323598in}}%
\pgfpathcurveto{\pgfqpoint{0.410684in}{4.323598in}}{\pgfqpoint{0.421283in}{4.327988in}}{\pgfqpoint{0.429097in}{4.335802in}}%
\pgfpathcurveto{\pgfqpoint{0.436910in}{4.343616in}}{\pgfqpoint{0.441301in}{4.354215in}}{\pgfqpoint{0.441301in}{4.365265in}}%
\pgfpathcurveto{\pgfqpoint{0.441301in}{4.376315in}}{\pgfqpoint{0.436910in}{4.386914in}}{\pgfqpoint{0.429097in}{4.394727in}}%
\pgfpathcurveto{\pgfqpoint{0.421283in}{4.402541in}}{\pgfqpoint{0.410684in}{4.406931in}}{\pgfqpoint{0.399634in}{4.406931in}}%
\pgfpathcurveto{\pgfqpoint{0.388584in}{4.406931in}}{\pgfqpoint{0.377985in}{4.402541in}}{\pgfqpoint{0.370171in}{4.394727in}}%
\pgfpathcurveto{\pgfqpoint{0.362357in}{4.386914in}}{\pgfqpoint{0.357967in}{4.376315in}}{\pgfqpoint{0.357967in}{4.365265in}}%
\pgfpathcurveto{\pgfqpoint{0.357967in}{4.354215in}}{\pgfqpoint{0.362357in}{4.343616in}}{\pgfqpoint{0.370171in}{4.335802in}}%
\pgfpathcurveto{\pgfqpoint{0.377985in}{4.327988in}}{\pgfqpoint{0.388584in}{4.323598in}}{\pgfqpoint{0.399634in}{4.323598in}}%
\pgfpathlineto{\pgfqpoint{0.399634in}{4.323598in}}%
\pgfpathclose%
\pgfusepath{stroke}%
\end{pgfscope}%
\begin{pgfscope}%
\pgfpathrectangle{\pgfqpoint{0.393053in}{0.375000in}}{\pgfqpoint{6.356833in}{5.175000in}}%
\pgfusepath{clip}%
\pgfsetbuttcap%
\pgfsetroundjoin%
\pgfsetlinewidth{1.003750pt}%
\definecolor{currentstroke}{rgb}{0.827451,0.827451,0.827451}%
\pgfsetstrokecolor{currentstroke}%
\pgfsetdash{}{0pt}%
\pgfpathmoveto{\pgfqpoint{1.068566in}{2.310306in}}%
\pgfpathcurveto{\pgfqpoint{1.079616in}{2.310306in}}{\pgfqpoint{1.090215in}{2.314696in}}{\pgfqpoint{1.098029in}{2.322510in}}%
\pgfpathcurveto{\pgfqpoint{1.105843in}{2.330323in}}{\pgfqpoint{1.110233in}{2.340922in}}{\pgfqpoint{1.110233in}{2.351973in}}%
\pgfpathcurveto{\pgfqpoint{1.110233in}{2.363023in}}{\pgfqpoint{1.105843in}{2.373622in}}{\pgfqpoint{1.098029in}{2.381435in}}%
\pgfpathcurveto{\pgfqpoint{1.090215in}{2.389249in}}{\pgfqpoint{1.079616in}{2.393639in}}{\pgfqpoint{1.068566in}{2.393639in}}%
\pgfpathcurveto{\pgfqpoint{1.057516in}{2.393639in}}{\pgfqpoint{1.046917in}{2.389249in}}{\pgfqpoint{1.039103in}{2.381435in}}%
\pgfpathcurveto{\pgfqpoint{1.031290in}{2.373622in}}{\pgfqpoint{1.026900in}{2.363023in}}{\pgfqpoint{1.026900in}{2.351973in}}%
\pgfpathcurveto{\pgfqpoint{1.026900in}{2.340922in}}{\pgfqpoint{1.031290in}{2.330323in}}{\pgfqpoint{1.039103in}{2.322510in}}%
\pgfpathcurveto{\pgfqpoint{1.046917in}{2.314696in}}{\pgfqpoint{1.057516in}{2.310306in}}{\pgfqpoint{1.068566in}{2.310306in}}%
\pgfpathlineto{\pgfqpoint{1.068566in}{2.310306in}}%
\pgfpathclose%
\pgfusepath{stroke}%
\end{pgfscope}%
\begin{pgfscope}%
\pgfpathrectangle{\pgfqpoint{0.393053in}{0.375000in}}{\pgfqpoint{6.356833in}{5.175000in}}%
\pgfusepath{clip}%
\pgfsetbuttcap%
\pgfsetroundjoin%
\pgfsetlinewidth{1.003750pt}%
\definecolor{currentstroke}{rgb}{0.827451,0.827451,0.827451}%
\pgfsetstrokecolor{currentstroke}%
\pgfsetdash{}{0pt}%
\pgfpathmoveto{\pgfqpoint{1.536415in}{1.768054in}}%
\pgfpathcurveto{\pgfqpoint{1.547466in}{1.768054in}}{\pgfqpoint{1.558065in}{1.772444in}}{\pgfqpoint{1.565878in}{1.780258in}}%
\pgfpathcurveto{\pgfqpoint{1.573692in}{1.788071in}}{\pgfqpoint{1.578082in}{1.798670in}}{\pgfqpoint{1.578082in}{1.809720in}}%
\pgfpathcurveto{\pgfqpoint{1.578082in}{1.820770in}}{\pgfqpoint{1.573692in}{1.831370in}}{\pgfqpoint{1.565878in}{1.839183in}}%
\pgfpathcurveto{\pgfqpoint{1.558065in}{1.846997in}}{\pgfqpoint{1.547466in}{1.851387in}}{\pgfqpoint{1.536415in}{1.851387in}}%
\pgfpathcurveto{\pgfqpoint{1.525365in}{1.851387in}}{\pgfqpoint{1.514766in}{1.846997in}}{\pgfqpoint{1.506953in}{1.839183in}}%
\pgfpathcurveto{\pgfqpoint{1.499139in}{1.831370in}}{\pgfqpoint{1.494749in}{1.820770in}}{\pgfqpoint{1.494749in}{1.809720in}}%
\pgfpathcurveto{\pgfqpoint{1.494749in}{1.798670in}}{\pgfqpoint{1.499139in}{1.788071in}}{\pgfqpoint{1.506953in}{1.780258in}}%
\pgfpathcurveto{\pgfqpoint{1.514766in}{1.772444in}}{\pgfqpoint{1.525365in}{1.768054in}}{\pgfqpoint{1.536415in}{1.768054in}}%
\pgfpathlineto{\pgfqpoint{1.536415in}{1.768054in}}%
\pgfpathclose%
\pgfusepath{stroke}%
\end{pgfscope}%
\begin{pgfscope}%
\pgfpathrectangle{\pgfqpoint{0.393053in}{0.375000in}}{\pgfqpoint{6.356833in}{5.175000in}}%
\pgfusepath{clip}%
\pgfsetbuttcap%
\pgfsetroundjoin%
\pgfsetlinewidth{1.003750pt}%
\definecolor{currentstroke}{rgb}{0.827451,0.827451,0.827451}%
\pgfsetstrokecolor{currentstroke}%
\pgfsetdash{}{0pt}%
\pgfpathmoveto{\pgfqpoint{2.036490in}{1.340063in}}%
\pgfpathcurveto{\pgfqpoint{2.047540in}{1.340063in}}{\pgfqpoint{2.058139in}{1.344454in}}{\pgfqpoint{2.065953in}{1.352267in}}%
\pgfpathcurveto{\pgfqpoint{2.073767in}{1.360081in}}{\pgfqpoint{2.078157in}{1.370680in}}{\pgfqpoint{2.078157in}{1.381730in}}%
\pgfpathcurveto{\pgfqpoint{2.078157in}{1.392780in}}{\pgfqpoint{2.073767in}{1.403379in}}{\pgfqpoint{2.065953in}{1.411193in}}%
\pgfpathcurveto{\pgfqpoint{2.058139in}{1.419007in}}{\pgfqpoint{2.047540in}{1.423397in}}{\pgfqpoint{2.036490in}{1.423397in}}%
\pgfpathcurveto{\pgfqpoint{2.025440in}{1.423397in}}{\pgfqpoint{2.014841in}{1.419007in}}{\pgfqpoint{2.007027in}{1.411193in}}%
\pgfpathcurveto{\pgfqpoint{1.999214in}{1.403379in}}{\pgfqpoint{1.994824in}{1.392780in}}{\pgfqpoint{1.994824in}{1.381730in}}%
\pgfpathcurveto{\pgfqpoint{1.994824in}{1.370680in}}{\pgfqpoint{1.999214in}{1.360081in}}{\pgfqpoint{2.007027in}{1.352267in}}%
\pgfpathcurveto{\pgfqpoint{2.014841in}{1.344454in}}{\pgfqpoint{2.025440in}{1.340063in}}{\pgfqpoint{2.036490in}{1.340063in}}%
\pgfpathlineto{\pgfqpoint{2.036490in}{1.340063in}}%
\pgfpathclose%
\pgfusepath{stroke}%
\end{pgfscope}%
\begin{pgfscope}%
\pgfpathrectangle{\pgfqpoint{0.393053in}{0.375000in}}{\pgfqpoint{6.356833in}{5.175000in}}%
\pgfusepath{clip}%
\pgfsetbuttcap%
\pgfsetroundjoin%
\pgfsetlinewidth{1.003750pt}%
\definecolor{currentstroke}{rgb}{0.827451,0.827451,0.827451}%
\pgfsetstrokecolor{currentstroke}%
\pgfsetdash{}{0pt}%
\pgfpathmoveto{\pgfqpoint{0.601770in}{3.214331in}}%
\pgfpathcurveto{\pgfqpoint{0.612820in}{3.214331in}}{\pgfqpoint{0.623419in}{3.218722in}}{\pgfqpoint{0.631233in}{3.226535in}}%
\pgfpathcurveto{\pgfqpoint{0.639046in}{3.234349in}}{\pgfqpoint{0.643437in}{3.244948in}}{\pgfqpoint{0.643437in}{3.255998in}}%
\pgfpathcurveto{\pgfqpoint{0.643437in}{3.267048in}}{\pgfqpoint{0.639046in}{3.277647in}}{\pgfqpoint{0.631233in}{3.285461in}}%
\pgfpathcurveto{\pgfqpoint{0.623419in}{3.293275in}}{\pgfqpoint{0.612820in}{3.297665in}}{\pgfqpoint{0.601770in}{3.297665in}}%
\pgfpathcurveto{\pgfqpoint{0.590720in}{3.297665in}}{\pgfqpoint{0.580121in}{3.293275in}}{\pgfqpoint{0.572307in}{3.285461in}}%
\pgfpathcurveto{\pgfqpoint{0.564494in}{3.277647in}}{\pgfqpoint{0.560103in}{3.267048in}}{\pgfqpoint{0.560103in}{3.255998in}}%
\pgfpathcurveto{\pgfqpoint{0.560103in}{3.244948in}}{\pgfqpoint{0.564494in}{3.234349in}}{\pgfqpoint{0.572307in}{3.226535in}}%
\pgfpathcurveto{\pgfqpoint{0.580121in}{3.218722in}}{\pgfqpoint{0.590720in}{3.214331in}}{\pgfqpoint{0.601770in}{3.214331in}}%
\pgfpathlineto{\pgfqpoint{0.601770in}{3.214331in}}%
\pgfpathclose%
\pgfusepath{stroke}%
\end{pgfscope}%
\begin{pgfscope}%
\pgfpathrectangle{\pgfqpoint{0.393053in}{0.375000in}}{\pgfqpoint{6.356833in}{5.175000in}}%
\pgfusepath{clip}%
\pgfsetbuttcap%
\pgfsetroundjoin%
\pgfsetlinewidth{1.003750pt}%
\definecolor{currentstroke}{rgb}{0.827451,0.827451,0.827451}%
\pgfsetstrokecolor{currentstroke}%
\pgfsetdash{}{0pt}%
\pgfpathmoveto{\pgfqpoint{0.393825in}{4.495013in}}%
\pgfpathcurveto{\pgfqpoint{0.404875in}{4.495013in}}{\pgfqpoint{0.415474in}{4.499403in}}{\pgfqpoint{0.423288in}{4.507217in}}%
\pgfpathcurveto{\pgfqpoint{0.431101in}{4.515030in}}{\pgfqpoint{0.435492in}{4.525629in}}{\pgfqpoint{0.435492in}{4.536679in}}%
\pgfpathcurveto{\pgfqpoint{0.435492in}{4.547730in}}{\pgfqpoint{0.431101in}{4.558329in}}{\pgfqpoint{0.423288in}{4.566142in}}%
\pgfpathcurveto{\pgfqpoint{0.415474in}{4.573956in}}{\pgfqpoint{0.404875in}{4.578346in}}{\pgfqpoint{0.393825in}{4.578346in}}%
\pgfpathcurveto{\pgfqpoint{0.382775in}{4.578346in}}{\pgfqpoint{0.372176in}{4.573956in}}{\pgfqpoint{0.364362in}{4.566142in}}%
\pgfpathcurveto{\pgfqpoint{0.356548in}{4.558329in}}{\pgfqpoint{0.352158in}{4.547730in}}{\pgfqpoint{0.352158in}{4.536679in}}%
\pgfpathcurveto{\pgfqpoint{0.352158in}{4.525629in}}{\pgfqpoint{0.356548in}{4.515030in}}{\pgfqpoint{0.364362in}{4.507217in}}%
\pgfpathcurveto{\pgfqpoint{0.372176in}{4.499403in}}{\pgfqpoint{0.382775in}{4.495013in}}{\pgfqpoint{0.393825in}{4.495013in}}%
\pgfpathlineto{\pgfqpoint{0.393825in}{4.495013in}}%
\pgfpathclose%
\pgfusepath{stroke}%
\end{pgfscope}%
\begin{pgfscope}%
\pgfpathrectangle{\pgfqpoint{0.393053in}{0.375000in}}{\pgfqpoint{6.356833in}{5.175000in}}%
\pgfusepath{clip}%
\pgfsetbuttcap%
\pgfsetroundjoin%
\pgfsetlinewidth{1.003750pt}%
\definecolor{currentstroke}{rgb}{0.827451,0.827451,0.827451}%
\pgfsetstrokecolor{currentstroke}%
\pgfsetdash{}{0pt}%
\pgfpathmoveto{\pgfqpoint{1.256856in}{2.039646in}}%
\pgfpathcurveto{\pgfqpoint{1.267906in}{2.039646in}}{\pgfqpoint{1.278505in}{2.044036in}}{\pgfqpoint{1.286319in}{2.051849in}}%
\pgfpathcurveto{\pgfqpoint{1.294132in}{2.059663in}}{\pgfqpoint{1.298523in}{2.070262in}}{\pgfqpoint{1.298523in}{2.081312in}}%
\pgfpathcurveto{\pgfqpoint{1.298523in}{2.092362in}}{\pgfqpoint{1.294132in}{2.102961in}}{\pgfqpoint{1.286319in}{2.110775in}}%
\pgfpathcurveto{\pgfqpoint{1.278505in}{2.118589in}}{\pgfqpoint{1.267906in}{2.122979in}}{\pgfqpoint{1.256856in}{2.122979in}}%
\pgfpathcurveto{\pgfqpoint{1.245806in}{2.122979in}}{\pgfqpoint{1.235207in}{2.118589in}}{\pgfqpoint{1.227393in}{2.110775in}}%
\pgfpathcurveto{\pgfqpoint{1.219580in}{2.102961in}}{\pgfqpoint{1.215189in}{2.092362in}}{\pgfqpoint{1.215189in}{2.081312in}}%
\pgfpathcurveto{\pgfqpoint{1.215189in}{2.070262in}}{\pgfqpoint{1.219580in}{2.059663in}}{\pgfqpoint{1.227393in}{2.051849in}}%
\pgfpathcurveto{\pgfqpoint{1.235207in}{2.044036in}}{\pgfqpoint{1.245806in}{2.039646in}}{\pgfqpoint{1.256856in}{2.039646in}}%
\pgfpathlineto{\pgfqpoint{1.256856in}{2.039646in}}%
\pgfpathclose%
\pgfusepath{stroke}%
\end{pgfscope}%
\begin{pgfscope}%
\pgfpathrectangle{\pgfqpoint{0.393053in}{0.375000in}}{\pgfqpoint{6.356833in}{5.175000in}}%
\pgfusepath{clip}%
\pgfsetbuttcap%
\pgfsetroundjoin%
\pgfsetlinewidth{1.003750pt}%
\definecolor{currentstroke}{rgb}{0.827451,0.827451,0.827451}%
\pgfsetstrokecolor{currentstroke}%
\pgfsetdash{}{0pt}%
\pgfpathmoveto{\pgfqpoint{2.416040in}{1.117802in}}%
\pgfpathcurveto{\pgfqpoint{2.427090in}{1.117802in}}{\pgfqpoint{2.437689in}{1.122193in}}{\pgfqpoint{2.445503in}{1.130006in}}%
\pgfpathcurveto{\pgfqpoint{2.453317in}{1.137820in}}{\pgfqpoint{2.457707in}{1.148419in}}{\pgfqpoint{2.457707in}{1.159469in}}%
\pgfpathcurveto{\pgfqpoint{2.457707in}{1.170519in}}{\pgfqpoint{2.453317in}{1.181118in}}{\pgfqpoint{2.445503in}{1.188932in}}%
\pgfpathcurveto{\pgfqpoint{2.437689in}{1.196745in}}{\pgfqpoint{2.427090in}{1.201136in}}{\pgfqpoint{2.416040in}{1.201136in}}%
\pgfpathcurveto{\pgfqpoint{2.404990in}{1.201136in}}{\pgfqpoint{2.394391in}{1.196745in}}{\pgfqpoint{2.386577in}{1.188932in}}%
\pgfpathcurveto{\pgfqpoint{2.378764in}{1.181118in}}{\pgfqpoint{2.374373in}{1.170519in}}{\pgfqpoint{2.374373in}{1.159469in}}%
\pgfpathcurveto{\pgfqpoint{2.374373in}{1.148419in}}{\pgfqpoint{2.378764in}{1.137820in}}{\pgfqpoint{2.386577in}{1.130006in}}%
\pgfpathcurveto{\pgfqpoint{2.394391in}{1.122193in}}{\pgfqpoint{2.404990in}{1.117802in}}{\pgfqpoint{2.416040in}{1.117802in}}%
\pgfpathlineto{\pgfqpoint{2.416040in}{1.117802in}}%
\pgfpathclose%
\pgfusepath{stroke}%
\end{pgfscope}%
\begin{pgfscope}%
\pgfpathrectangle{\pgfqpoint{0.393053in}{0.375000in}}{\pgfqpoint{6.356833in}{5.175000in}}%
\pgfusepath{clip}%
\pgfsetbuttcap%
\pgfsetroundjoin%
\pgfsetlinewidth{1.003750pt}%
\definecolor{currentstroke}{rgb}{0.827451,0.827451,0.827451}%
\pgfsetstrokecolor{currentstroke}%
\pgfsetdash{}{0pt}%
\pgfpathmoveto{\pgfqpoint{3.885340in}{0.525311in}}%
\pgfpathcurveto{\pgfqpoint{3.896391in}{0.525311in}}{\pgfqpoint{3.906990in}{0.529701in}}{\pgfqpoint{3.914803in}{0.537515in}}%
\pgfpathcurveto{\pgfqpoint{3.922617in}{0.545328in}}{\pgfqpoint{3.927007in}{0.555927in}}{\pgfqpoint{3.927007in}{0.566978in}}%
\pgfpathcurveto{\pgfqpoint{3.927007in}{0.578028in}}{\pgfqpoint{3.922617in}{0.588627in}}{\pgfqpoint{3.914803in}{0.596440in}}%
\pgfpathcurveto{\pgfqpoint{3.906990in}{0.604254in}}{\pgfqpoint{3.896391in}{0.608644in}}{\pgfqpoint{3.885340in}{0.608644in}}%
\pgfpathcurveto{\pgfqpoint{3.874290in}{0.608644in}}{\pgfqpoint{3.863691in}{0.604254in}}{\pgfqpoint{3.855878in}{0.596440in}}%
\pgfpathcurveto{\pgfqpoint{3.848064in}{0.588627in}}{\pgfqpoint{3.843674in}{0.578028in}}{\pgfqpoint{3.843674in}{0.566978in}}%
\pgfpathcurveto{\pgfqpoint{3.843674in}{0.555927in}}{\pgfqpoint{3.848064in}{0.545328in}}{\pgfqpoint{3.855878in}{0.537515in}}%
\pgfpathcurveto{\pgfqpoint{3.863691in}{0.529701in}}{\pgfqpoint{3.874290in}{0.525311in}}{\pgfqpoint{3.885340in}{0.525311in}}%
\pgfpathlineto{\pgfqpoint{3.885340in}{0.525311in}}%
\pgfpathclose%
\pgfusepath{stroke}%
\end{pgfscope}%
\begin{pgfscope}%
\pgfpathrectangle{\pgfqpoint{0.393053in}{0.375000in}}{\pgfqpoint{6.356833in}{5.175000in}}%
\pgfusepath{clip}%
\pgfsetbuttcap%
\pgfsetroundjoin%
\pgfsetlinewidth{1.003750pt}%
\definecolor{currentstroke}{rgb}{0.827451,0.827451,0.827451}%
\pgfsetstrokecolor{currentstroke}%
\pgfsetdash{}{0pt}%
\pgfpathmoveto{\pgfqpoint{4.012103in}{0.498786in}}%
\pgfpathcurveto{\pgfqpoint{4.023153in}{0.498786in}}{\pgfqpoint{4.033752in}{0.503177in}}{\pgfqpoint{4.041566in}{0.510990in}}%
\pgfpathcurveto{\pgfqpoint{4.049379in}{0.518804in}}{\pgfqpoint{4.053769in}{0.529403in}}{\pgfqpoint{4.053769in}{0.540453in}}%
\pgfpathcurveto{\pgfqpoint{4.053769in}{0.551503in}}{\pgfqpoint{4.049379in}{0.562102in}}{\pgfqpoint{4.041566in}{0.569916in}}%
\pgfpathcurveto{\pgfqpoint{4.033752in}{0.577729in}}{\pgfqpoint{4.023153in}{0.582120in}}{\pgfqpoint{4.012103in}{0.582120in}}%
\pgfpathcurveto{\pgfqpoint{4.001053in}{0.582120in}}{\pgfqpoint{3.990454in}{0.577729in}}{\pgfqpoint{3.982640in}{0.569916in}}%
\pgfpathcurveto{\pgfqpoint{3.974826in}{0.562102in}}{\pgfqpoint{3.970436in}{0.551503in}}{\pgfqpoint{3.970436in}{0.540453in}}%
\pgfpathcurveto{\pgfqpoint{3.970436in}{0.529403in}}{\pgfqpoint{3.974826in}{0.518804in}}{\pgfqpoint{3.982640in}{0.510990in}}%
\pgfpathcurveto{\pgfqpoint{3.990454in}{0.503177in}}{\pgfqpoint{4.001053in}{0.498786in}}{\pgfqpoint{4.012103in}{0.498786in}}%
\pgfpathlineto{\pgfqpoint{4.012103in}{0.498786in}}%
\pgfpathclose%
\pgfusepath{stroke}%
\end{pgfscope}%
\begin{pgfscope}%
\pgfpathrectangle{\pgfqpoint{0.393053in}{0.375000in}}{\pgfqpoint{6.356833in}{5.175000in}}%
\pgfusepath{clip}%
\pgfsetbuttcap%
\pgfsetroundjoin%
\pgfsetlinewidth{1.003750pt}%
\definecolor{currentstroke}{rgb}{0.827451,0.827451,0.827451}%
\pgfsetstrokecolor{currentstroke}%
\pgfsetdash{}{0pt}%
\pgfpathmoveto{\pgfqpoint{3.844148in}{0.542510in}}%
\pgfpathcurveto{\pgfqpoint{3.855198in}{0.542510in}}{\pgfqpoint{3.865797in}{0.546901in}}{\pgfqpoint{3.873610in}{0.554714in}}%
\pgfpathcurveto{\pgfqpoint{3.881424in}{0.562528in}}{\pgfqpoint{3.885814in}{0.573127in}}{\pgfqpoint{3.885814in}{0.584177in}}%
\pgfpathcurveto{\pgfqpoint{3.885814in}{0.595227in}}{\pgfqpoint{3.881424in}{0.605826in}}{\pgfqpoint{3.873610in}{0.613640in}}%
\pgfpathcurveto{\pgfqpoint{3.865797in}{0.621453in}}{\pgfqpoint{3.855198in}{0.625844in}}{\pgfqpoint{3.844148in}{0.625844in}}%
\pgfpathcurveto{\pgfqpoint{3.833097in}{0.625844in}}{\pgfqpoint{3.822498in}{0.621453in}}{\pgfqpoint{3.814685in}{0.613640in}}%
\pgfpathcurveto{\pgfqpoint{3.806871in}{0.605826in}}{\pgfqpoint{3.802481in}{0.595227in}}{\pgfqpoint{3.802481in}{0.584177in}}%
\pgfpathcurveto{\pgfqpoint{3.802481in}{0.573127in}}{\pgfqpoint{3.806871in}{0.562528in}}{\pgfqpoint{3.814685in}{0.554714in}}%
\pgfpathcurveto{\pgfqpoint{3.822498in}{0.546901in}}{\pgfqpoint{3.833097in}{0.542510in}}{\pgfqpoint{3.844148in}{0.542510in}}%
\pgfpathlineto{\pgfqpoint{3.844148in}{0.542510in}}%
\pgfpathclose%
\pgfusepath{stroke}%
\end{pgfscope}%
\begin{pgfscope}%
\pgfpathrectangle{\pgfqpoint{0.393053in}{0.375000in}}{\pgfqpoint{6.356833in}{5.175000in}}%
\pgfusepath{clip}%
\pgfsetbuttcap%
\pgfsetroundjoin%
\pgfsetlinewidth{1.003750pt}%
\definecolor{currentstroke}{rgb}{0.827451,0.827451,0.827451}%
\pgfsetstrokecolor{currentstroke}%
\pgfsetdash{}{0pt}%
\pgfpathmoveto{\pgfqpoint{0.474470in}{3.727925in}}%
\pgfpathcurveto{\pgfqpoint{0.485520in}{3.727925in}}{\pgfqpoint{0.496119in}{3.732315in}}{\pgfqpoint{0.503933in}{3.740129in}}%
\pgfpathcurveto{\pgfqpoint{0.511746in}{3.747943in}}{\pgfqpoint{0.516137in}{3.758542in}}{\pgfqpoint{0.516137in}{3.769592in}}%
\pgfpathcurveto{\pgfqpoint{0.516137in}{3.780642in}}{\pgfqpoint{0.511746in}{3.791241in}}{\pgfqpoint{0.503933in}{3.799055in}}%
\pgfpathcurveto{\pgfqpoint{0.496119in}{3.806868in}}{\pgfqpoint{0.485520in}{3.811259in}}{\pgfqpoint{0.474470in}{3.811259in}}%
\pgfpathcurveto{\pgfqpoint{0.463420in}{3.811259in}}{\pgfqpoint{0.452821in}{3.806868in}}{\pgfqpoint{0.445007in}{3.799055in}}%
\pgfpathcurveto{\pgfqpoint{0.437194in}{3.791241in}}{\pgfqpoint{0.432803in}{3.780642in}}{\pgfqpoint{0.432803in}{3.769592in}}%
\pgfpathcurveto{\pgfqpoint{0.432803in}{3.758542in}}{\pgfqpoint{0.437194in}{3.747943in}}{\pgfqpoint{0.445007in}{3.740129in}}%
\pgfpathcurveto{\pgfqpoint{0.452821in}{3.732315in}}{\pgfqpoint{0.463420in}{3.727925in}}{\pgfqpoint{0.474470in}{3.727925in}}%
\pgfpathlineto{\pgfqpoint{0.474470in}{3.727925in}}%
\pgfpathclose%
\pgfusepath{stroke}%
\end{pgfscope}%
\begin{pgfscope}%
\pgfpathrectangle{\pgfqpoint{0.393053in}{0.375000in}}{\pgfqpoint{6.356833in}{5.175000in}}%
\pgfusepath{clip}%
\pgfsetbuttcap%
\pgfsetroundjoin%
\pgfsetlinewidth{1.003750pt}%
\definecolor{currentstroke}{rgb}{0.827451,0.827451,0.827451}%
\pgfsetstrokecolor{currentstroke}%
\pgfsetdash{}{0pt}%
\pgfpathmoveto{\pgfqpoint{2.179603in}{1.241266in}}%
\pgfpathcurveto{\pgfqpoint{2.190653in}{1.241266in}}{\pgfqpoint{2.201252in}{1.245657in}}{\pgfqpoint{2.209066in}{1.253470in}}%
\pgfpathcurveto{\pgfqpoint{2.216880in}{1.261284in}}{\pgfqpoint{2.221270in}{1.271883in}}{\pgfqpoint{2.221270in}{1.282933in}}%
\pgfpathcurveto{\pgfqpoint{2.221270in}{1.293983in}}{\pgfqpoint{2.216880in}{1.304582in}}{\pgfqpoint{2.209066in}{1.312396in}}%
\pgfpathcurveto{\pgfqpoint{2.201252in}{1.320209in}}{\pgfqpoint{2.190653in}{1.324600in}}{\pgfqpoint{2.179603in}{1.324600in}}%
\pgfpathcurveto{\pgfqpoint{2.168553in}{1.324600in}}{\pgfqpoint{2.157954in}{1.320209in}}{\pgfqpoint{2.150140in}{1.312396in}}%
\pgfpathcurveto{\pgfqpoint{2.142327in}{1.304582in}}{\pgfqpoint{2.137936in}{1.293983in}}{\pgfqpoint{2.137936in}{1.282933in}}%
\pgfpathcurveto{\pgfqpoint{2.137936in}{1.271883in}}{\pgfqpoint{2.142327in}{1.261284in}}{\pgfqpoint{2.150140in}{1.253470in}}%
\pgfpathcurveto{\pgfqpoint{2.157954in}{1.245657in}}{\pgfqpoint{2.168553in}{1.241266in}}{\pgfqpoint{2.179603in}{1.241266in}}%
\pgfpathlineto{\pgfqpoint{2.179603in}{1.241266in}}%
\pgfpathclose%
\pgfusepath{stroke}%
\end{pgfscope}%
\begin{pgfscope}%
\pgfpathrectangle{\pgfqpoint{0.393053in}{0.375000in}}{\pgfqpoint{6.356833in}{5.175000in}}%
\pgfusepath{clip}%
\pgfsetbuttcap%
\pgfsetroundjoin%
\pgfsetlinewidth{1.003750pt}%
\definecolor{currentstroke}{rgb}{0.827451,0.827451,0.827451}%
\pgfsetstrokecolor{currentstroke}%
\pgfsetdash{}{0pt}%
\pgfpathmoveto{\pgfqpoint{1.614675in}{1.708400in}}%
\pgfpathcurveto{\pgfqpoint{1.625725in}{1.708400in}}{\pgfqpoint{1.636324in}{1.712790in}}{\pgfqpoint{1.644137in}{1.720604in}}%
\pgfpathcurveto{\pgfqpoint{1.651951in}{1.728417in}}{\pgfqpoint{1.656341in}{1.739016in}}{\pgfqpoint{1.656341in}{1.750066in}}%
\pgfpathcurveto{\pgfqpoint{1.656341in}{1.761117in}}{\pgfqpoint{1.651951in}{1.771716in}}{\pgfqpoint{1.644137in}{1.779529in}}%
\pgfpathcurveto{\pgfqpoint{1.636324in}{1.787343in}}{\pgfqpoint{1.625725in}{1.791733in}}{\pgfqpoint{1.614675in}{1.791733in}}%
\pgfpathcurveto{\pgfqpoint{1.603625in}{1.791733in}}{\pgfqpoint{1.593025in}{1.787343in}}{\pgfqpoint{1.585212in}{1.779529in}}%
\pgfpathcurveto{\pgfqpoint{1.577398in}{1.771716in}}{\pgfqpoint{1.573008in}{1.761117in}}{\pgfqpoint{1.573008in}{1.750066in}}%
\pgfpathcurveto{\pgfqpoint{1.573008in}{1.739016in}}{\pgfqpoint{1.577398in}{1.728417in}}{\pgfqpoint{1.585212in}{1.720604in}}%
\pgfpathcurveto{\pgfqpoint{1.593025in}{1.712790in}}{\pgfqpoint{1.603625in}{1.708400in}}{\pgfqpoint{1.614675in}{1.708400in}}%
\pgfpathlineto{\pgfqpoint{1.614675in}{1.708400in}}%
\pgfpathclose%
\pgfusepath{stroke}%
\end{pgfscope}%
\begin{pgfscope}%
\pgfpathrectangle{\pgfqpoint{0.393053in}{0.375000in}}{\pgfqpoint{6.356833in}{5.175000in}}%
\pgfusepath{clip}%
\pgfsetbuttcap%
\pgfsetroundjoin%
\pgfsetlinewidth{1.003750pt}%
\definecolor{currentstroke}{rgb}{0.827451,0.827451,0.827451}%
\pgfsetstrokecolor{currentstroke}%
\pgfsetdash{}{0pt}%
\pgfpathmoveto{\pgfqpoint{0.580834in}{3.297265in}}%
\pgfpathcurveto{\pgfqpoint{0.591884in}{3.297265in}}{\pgfqpoint{0.602483in}{3.301655in}}{\pgfqpoint{0.610297in}{3.309469in}}%
\pgfpathcurveto{\pgfqpoint{0.618111in}{3.317282in}}{\pgfqpoint{0.622501in}{3.327881in}}{\pgfqpoint{0.622501in}{3.338932in}}%
\pgfpathcurveto{\pgfqpoint{0.622501in}{3.349982in}}{\pgfqpoint{0.618111in}{3.360581in}}{\pgfqpoint{0.610297in}{3.368394in}}%
\pgfpathcurveto{\pgfqpoint{0.602483in}{3.376208in}}{\pgfqpoint{0.591884in}{3.380598in}}{\pgfqpoint{0.580834in}{3.380598in}}%
\pgfpathcurveto{\pgfqpoint{0.569784in}{3.380598in}}{\pgfqpoint{0.559185in}{3.376208in}}{\pgfqpoint{0.551371in}{3.368394in}}%
\pgfpathcurveto{\pgfqpoint{0.543558in}{3.360581in}}{\pgfqpoint{0.539167in}{3.349982in}}{\pgfqpoint{0.539167in}{3.338932in}}%
\pgfpathcurveto{\pgfqpoint{0.539167in}{3.327881in}}{\pgfqpoint{0.543558in}{3.317282in}}{\pgfqpoint{0.551371in}{3.309469in}}%
\pgfpathcurveto{\pgfqpoint{0.559185in}{3.301655in}}{\pgfqpoint{0.569784in}{3.297265in}}{\pgfqpoint{0.580834in}{3.297265in}}%
\pgfpathlineto{\pgfqpoint{0.580834in}{3.297265in}}%
\pgfpathclose%
\pgfusepath{stroke}%
\end{pgfscope}%
\begin{pgfscope}%
\pgfpathrectangle{\pgfqpoint{0.393053in}{0.375000in}}{\pgfqpoint{6.356833in}{5.175000in}}%
\pgfusepath{clip}%
\pgfsetbuttcap%
\pgfsetroundjoin%
\pgfsetlinewidth{1.003750pt}%
\definecolor{currentstroke}{rgb}{0.827451,0.827451,0.827451}%
\pgfsetstrokecolor{currentstroke}%
\pgfsetdash{}{0pt}%
\pgfpathmoveto{\pgfqpoint{4.122151in}{0.465889in}}%
\pgfpathcurveto{\pgfqpoint{4.133202in}{0.465889in}}{\pgfqpoint{4.143801in}{0.470279in}}{\pgfqpoint{4.151614in}{0.478093in}}%
\pgfpathcurveto{\pgfqpoint{4.159428in}{0.485907in}}{\pgfqpoint{4.163818in}{0.496506in}}{\pgfqpoint{4.163818in}{0.507556in}}%
\pgfpathcurveto{\pgfqpoint{4.163818in}{0.518606in}}{\pgfqpoint{4.159428in}{0.529205in}}{\pgfqpoint{4.151614in}{0.537019in}}%
\pgfpathcurveto{\pgfqpoint{4.143801in}{0.544832in}}{\pgfqpoint{4.133202in}{0.549222in}}{\pgfqpoint{4.122151in}{0.549222in}}%
\pgfpathcurveto{\pgfqpoint{4.111101in}{0.549222in}}{\pgfqpoint{4.100502in}{0.544832in}}{\pgfqpoint{4.092689in}{0.537019in}}%
\pgfpathcurveto{\pgfqpoint{4.084875in}{0.529205in}}{\pgfqpoint{4.080485in}{0.518606in}}{\pgfqpoint{4.080485in}{0.507556in}}%
\pgfpathcurveto{\pgfqpoint{4.080485in}{0.496506in}}{\pgfqpoint{4.084875in}{0.485907in}}{\pgfqpoint{4.092689in}{0.478093in}}%
\pgfpathcurveto{\pgfqpoint{4.100502in}{0.470279in}}{\pgfqpoint{4.111101in}{0.465889in}}{\pgfqpoint{4.122151in}{0.465889in}}%
\pgfpathlineto{\pgfqpoint{4.122151in}{0.465889in}}%
\pgfpathclose%
\pgfusepath{stroke}%
\end{pgfscope}%
\begin{pgfscope}%
\pgfpathrectangle{\pgfqpoint{0.393053in}{0.375000in}}{\pgfqpoint{6.356833in}{5.175000in}}%
\pgfusepath{clip}%
\pgfsetbuttcap%
\pgfsetroundjoin%
\pgfsetlinewidth{1.003750pt}%
\definecolor{currentstroke}{rgb}{0.827451,0.827451,0.827451}%
\pgfsetstrokecolor{currentstroke}%
\pgfsetdash{}{0pt}%
\pgfpathmoveto{\pgfqpoint{2.585362in}{1.005073in}}%
\pgfpathcurveto{\pgfqpoint{2.596412in}{1.005073in}}{\pgfqpoint{2.607011in}{1.009463in}}{\pgfqpoint{2.614824in}{1.017277in}}%
\pgfpathcurveto{\pgfqpoint{2.622638in}{1.025090in}}{\pgfqpoint{2.627028in}{1.035689in}}{\pgfqpoint{2.627028in}{1.046740in}}%
\pgfpathcurveto{\pgfqpoint{2.627028in}{1.057790in}}{\pgfqpoint{2.622638in}{1.068389in}}{\pgfqpoint{2.614824in}{1.076202in}}%
\pgfpathcurveto{\pgfqpoint{2.607011in}{1.084016in}}{\pgfqpoint{2.596412in}{1.088406in}}{\pgfqpoint{2.585362in}{1.088406in}}%
\pgfpathcurveto{\pgfqpoint{2.574311in}{1.088406in}}{\pgfqpoint{2.563712in}{1.084016in}}{\pgfqpoint{2.555899in}{1.076202in}}%
\pgfpathcurveto{\pgfqpoint{2.548085in}{1.068389in}}{\pgfqpoint{2.543695in}{1.057790in}}{\pgfqpoint{2.543695in}{1.046740in}}%
\pgfpathcurveto{\pgfqpoint{2.543695in}{1.035689in}}{\pgfqpoint{2.548085in}{1.025090in}}{\pgfqpoint{2.555899in}{1.017277in}}%
\pgfpathcurveto{\pgfqpoint{2.563712in}{1.009463in}}{\pgfqpoint{2.574311in}{1.005073in}}{\pgfqpoint{2.585362in}{1.005073in}}%
\pgfpathlineto{\pgfqpoint{2.585362in}{1.005073in}}%
\pgfpathclose%
\pgfusepath{stroke}%
\end{pgfscope}%
\begin{pgfscope}%
\pgfpathrectangle{\pgfqpoint{0.393053in}{0.375000in}}{\pgfqpoint{6.356833in}{5.175000in}}%
\pgfusepath{clip}%
\pgfsetbuttcap%
\pgfsetroundjoin%
\pgfsetlinewidth{1.003750pt}%
\definecolor{currentstroke}{rgb}{0.827451,0.827451,0.827451}%
\pgfsetstrokecolor{currentstroke}%
\pgfsetdash{}{0pt}%
\pgfpathmoveto{\pgfqpoint{2.693530in}{0.950710in}}%
\pgfpathcurveto{\pgfqpoint{2.704580in}{0.950710in}}{\pgfqpoint{2.715179in}{0.955100in}}{\pgfqpoint{2.722993in}{0.962914in}}%
\pgfpathcurveto{\pgfqpoint{2.730806in}{0.970727in}}{\pgfqpoint{2.735197in}{0.981326in}}{\pgfqpoint{2.735197in}{0.992376in}}%
\pgfpathcurveto{\pgfqpoint{2.735197in}{1.003427in}}{\pgfqpoint{2.730806in}{1.014026in}}{\pgfqpoint{2.722993in}{1.021839in}}%
\pgfpathcurveto{\pgfqpoint{2.715179in}{1.029653in}}{\pgfqpoint{2.704580in}{1.034043in}}{\pgfqpoint{2.693530in}{1.034043in}}%
\pgfpathcurveto{\pgfqpoint{2.682480in}{1.034043in}}{\pgfqpoint{2.671881in}{1.029653in}}{\pgfqpoint{2.664067in}{1.021839in}}%
\pgfpathcurveto{\pgfqpoint{2.656254in}{1.014026in}}{\pgfqpoint{2.651863in}{1.003427in}}{\pgfqpoint{2.651863in}{0.992376in}}%
\pgfpathcurveto{\pgfqpoint{2.651863in}{0.981326in}}{\pgfqpoint{2.656254in}{0.970727in}}{\pgfqpoint{2.664067in}{0.962914in}}%
\pgfpathcurveto{\pgfqpoint{2.671881in}{0.955100in}}{\pgfqpoint{2.682480in}{0.950710in}}{\pgfqpoint{2.693530in}{0.950710in}}%
\pgfpathlineto{\pgfqpoint{2.693530in}{0.950710in}}%
\pgfpathclose%
\pgfusepath{stroke}%
\end{pgfscope}%
\begin{pgfscope}%
\pgfpathrectangle{\pgfqpoint{0.393053in}{0.375000in}}{\pgfqpoint{6.356833in}{5.175000in}}%
\pgfusepath{clip}%
\pgfsetbuttcap%
\pgfsetroundjoin%
\pgfsetlinewidth{1.003750pt}%
\definecolor{currentstroke}{rgb}{0.827451,0.827451,0.827451}%
\pgfsetstrokecolor{currentstroke}%
\pgfsetdash{}{0pt}%
\pgfpathmoveto{\pgfqpoint{0.651937in}{3.079342in}}%
\pgfpathcurveto{\pgfqpoint{0.662987in}{3.079342in}}{\pgfqpoint{0.673586in}{3.083732in}}{\pgfqpoint{0.681400in}{3.091546in}}%
\pgfpathcurveto{\pgfqpoint{0.689213in}{3.099359in}}{\pgfqpoint{0.693604in}{3.109958in}}{\pgfqpoint{0.693604in}{3.121009in}}%
\pgfpathcurveto{\pgfqpoint{0.693604in}{3.132059in}}{\pgfqpoint{0.689213in}{3.142658in}}{\pgfqpoint{0.681400in}{3.150471in}}%
\pgfpathcurveto{\pgfqpoint{0.673586in}{3.158285in}}{\pgfqpoint{0.662987in}{3.162675in}}{\pgfqpoint{0.651937in}{3.162675in}}%
\pgfpathcurveto{\pgfqpoint{0.640887in}{3.162675in}}{\pgfqpoint{0.630288in}{3.158285in}}{\pgfqpoint{0.622474in}{3.150471in}}%
\pgfpathcurveto{\pgfqpoint{0.614660in}{3.142658in}}{\pgfqpoint{0.610270in}{3.132059in}}{\pgfqpoint{0.610270in}{3.121009in}}%
\pgfpathcurveto{\pgfqpoint{0.610270in}{3.109958in}}{\pgfqpoint{0.614660in}{3.099359in}}{\pgfqpoint{0.622474in}{3.091546in}}%
\pgfpathcurveto{\pgfqpoint{0.630288in}{3.083732in}}{\pgfqpoint{0.640887in}{3.079342in}}{\pgfqpoint{0.651937in}{3.079342in}}%
\pgfpathlineto{\pgfqpoint{0.651937in}{3.079342in}}%
\pgfpathclose%
\pgfusepath{stroke}%
\end{pgfscope}%
\begin{pgfscope}%
\pgfpathrectangle{\pgfqpoint{0.393053in}{0.375000in}}{\pgfqpoint{6.356833in}{5.175000in}}%
\pgfusepath{clip}%
\pgfsetbuttcap%
\pgfsetroundjoin%
\pgfsetlinewidth{1.003750pt}%
\definecolor{currentstroke}{rgb}{0.827451,0.827451,0.827451}%
\pgfsetstrokecolor{currentstroke}%
\pgfsetdash{}{0pt}%
\pgfpathmoveto{\pgfqpoint{0.961220in}{2.522545in}}%
\pgfpathcurveto{\pgfqpoint{0.972270in}{2.522545in}}{\pgfqpoint{0.982869in}{2.526936in}}{\pgfqpoint{0.990683in}{2.534749in}}%
\pgfpathcurveto{\pgfqpoint{0.998497in}{2.542563in}}{\pgfqpoint{1.002887in}{2.553162in}}{\pgfqpoint{1.002887in}{2.564212in}}%
\pgfpathcurveto{\pgfqpoint{1.002887in}{2.575262in}}{\pgfqpoint{0.998497in}{2.585861in}}{\pgfqpoint{0.990683in}{2.593675in}}%
\pgfpathcurveto{\pgfqpoint{0.982869in}{2.601488in}}{\pgfqpoint{0.972270in}{2.605879in}}{\pgfqpoint{0.961220in}{2.605879in}}%
\pgfpathcurveto{\pgfqpoint{0.950170in}{2.605879in}}{\pgfqpoint{0.939571in}{2.601488in}}{\pgfqpoint{0.931757in}{2.593675in}}%
\pgfpathcurveto{\pgfqpoint{0.923944in}{2.585861in}}{\pgfqpoint{0.919554in}{2.575262in}}{\pgfqpoint{0.919554in}{2.564212in}}%
\pgfpathcurveto{\pgfqpoint{0.919554in}{2.553162in}}{\pgfqpoint{0.923944in}{2.542563in}}{\pgfqpoint{0.931757in}{2.534749in}}%
\pgfpathcurveto{\pgfqpoint{0.939571in}{2.526936in}}{\pgfqpoint{0.950170in}{2.522545in}}{\pgfqpoint{0.961220in}{2.522545in}}%
\pgfpathlineto{\pgfqpoint{0.961220in}{2.522545in}}%
\pgfpathclose%
\pgfusepath{stroke}%
\end{pgfscope}%
\begin{pgfscope}%
\pgfpathrectangle{\pgfqpoint{0.393053in}{0.375000in}}{\pgfqpoint{6.356833in}{5.175000in}}%
\pgfusepath{clip}%
\pgfsetbuttcap%
\pgfsetroundjoin%
\pgfsetlinewidth{1.003750pt}%
\definecolor{currentstroke}{rgb}{0.827451,0.827451,0.827451}%
\pgfsetstrokecolor{currentstroke}%
\pgfsetdash{}{0pt}%
\pgfpathmoveto{\pgfqpoint{2.785690in}{0.904937in}}%
\pgfpathcurveto{\pgfqpoint{2.796740in}{0.904937in}}{\pgfqpoint{2.807339in}{0.909327in}}{\pgfqpoint{2.815153in}{0.917140in}}%
\pgfpathcurveto{\pgfqpoint{2.822966in}{0.924954in}}{\pgfqpoint{2.827357in}{0.935553in}}{\pgfqpoint{2.827357in}{0.946603in}}%
\pgfpathcurveto{\pgfqpoint{2.827357in}{0.957653in}}{\pgfqpoint{2.822966in}{0.968252in}}{\pgfqpoint{2.815153in}{0.976066in}}%
\pgfpathcurveto{\pgfqpoint{2.807339in}{0.983880in}}{\pgfqpoint{2.796740in}{0.988270in}}{\pgfqpoint{2.785690in}{0.988270in}}%
\pgfpathcurveto{\pgfqpoint{2.774640in}{0.988270in}}{\pgfqpoint{2.764041in}{0.983880in}}{\pgfqpoint{2.756227in}{0.976066in}}%
\pgfpathcurveto{\pgfqpoint{2.748414in}{0.968252in}}{\pgfqpoint{2.744023in}{0.957653in}}{\pgfqpoint{2.744023in}{0.946603in}}%
\pgfpathcurveto{\pgfqpoint{2.744023in}{0.935553in}}{\pgfqpoint{2.748414in}{0.924954in}}{\pgfqpoint{2.756227in}{0.917140in}}%
\pgfpathcurveto{\pgfqpoint{2.764041in}{0.909327in}}{\pgfqpoint{2.774640in}{0.904937in}}{\pgfqpoint{2.785690in}{0.904937in}}%
\pgfpathlineto{\pgfqpoint{2.785690in}{0.904937in}}%
\pgfpathclose%
\pgfusepath{stroke}%
\end{pgfscope}%
\begin{pgfscope}%
\pgfpathrectangle{\pgfqpoint{0.393053in}{0.375000in}}{\pgfqpoint{6.356833in}{5.175000in}}%
\pgfusepath{clip}%
\pgfsetbuttcap%
\pgfsetroundjoin%
\pgfsetlinewidth{1.003750pt}%
\definecolor{currentstroke}{rgb}{0.827451,0.827451,0.827451}%
\pgfsetstrokecolor{currentstroke}%
\pgfsetdash{}{0pt}%
\pgfpathmoveto{\pgfqpoint{2.477308in}{1.087549in}}%
\pgfpathcurveto{\pgfqpoint{2.488358in}{1.087549in}}{\pgfqpoint{2.498957in}{1.091940in}}{\pgfqpoint{2.506771in}{1.099753in}}%
\pgfpathcurveto{\pgfqpoint{2.514584in}{1.107567in}}{\pgfqpoint{2.518974in}{1.118166in}}{\pgfqpoint{2.518974in}{1.129216in}}%
\pgfpathcurveto{\pgfqpoint{2.518974in}{1.140266in}}{\pgfqpoint{2.514584in}{1.150865in}}{\pgfqpoint{2.506771in}{1.158679in}}%
\pgfpathcurveto{\pgfqpoint{2.498957in}{1.166492in}}{\pgfqpoint{2.488358in}{1.170883in}}{\pgfqpoint{2.477308in}{1.170883in}}%
\pgfpathcurveto{\pgfqpoint{2.466258in}{1.170883in}}{\pgfqpoint{2.455659in}{1.166492in}}{\pgfqpoint{2.447845in}{1.158679in}}%
\pgfpathcurveto{\pgfqpoint{2.440031in}{1.150865in}}{\pgfqpoint{2.435641in}{1.140266in}}{\pgfqpoint{2.435641in}{1.129216in}}%
\pgfpathcurveto{\pgfqpoint{2.435641in}{1.118166in}}{\pgfqpoint{2.440031in}{1.107567in}}{\pgfqpoint{2.447845in}{1.099753in}}%
\pgfpathcurveto{\pgfqpoint{2.455659in}{1.091940in}}{\pgfqpoint{2.466258in}{1.087549in}}{\pgfqpoint{2.477308in}{1.087549in}}%
\pgfpathlineto{\pgfqpoint{2.477308in}{1.087549in}}%
\pgfpathclose%
\pgfusepath{stroke}%
\end{pgfscope}%
\begin{pgfscope}%
\pgfpathrectangle{\pgfqpoint{0.393053in}{0.375000in}}{\pgfqpoint{6.356833in}{5.175000in}}%
\pgfusepath{clip}%
\pgfsetbuttcap%
\pgfsetroundjoin%
\pgfsetlinewidth{1.003750pt}%
\definecolor{currentstroke}{rgb}{0.827451,0.827451,0.827451}%
\pgfsetstrokecolor{currentstroke}%
\pgfsetdash{}{0pt}%
\pgfpathmoveto{\pgfqpoint{1.626356in}{1.668438in}}%
\pgfpathcurveto{\pgfqpoint{1.637406in}{1.668438in}}{\pgfqpoint{1.648005in}{1.672828in}}{\pgfqpoint{1.655818in}{1.680641in}}%
\pgfpathcurveto{\pgfqpoint{1.663632in}{1.688455in}}{\pgfqpoint{1.668022in}{1.699054in}}{\pgfqpoint{1.668022in}{1.710104in}}%
\pgfpathcurveto{\pgfqpoint{1.668022in}{1.721154in}}{\pgfqpoint{1.663632in}{1.731753in}}{\pgfqpoint{1.655818in}{1.739567in}}%
\pgfpathcurveto{\pgfqpoint{1.648005in}{1.747381in}}{\pgfqpoint{1.637406in}{1.751771in}}{\pgfqpoint{1.626356in}{1.751771in}}%
\pgfpathcurveto{\pgfqpoint{1.615305in}{1.751771in}}{\pgfqpoint{1.604706in}{1.747381in}}{\pgfqpoint{1.596893in}{1.739567in}}%
\pgfpathcurveto{\pgfqpoint{1.589079in}{1.731753in}}{\pgfqpoint{1.584689in}{1.721154in}}{\pgfqpoint{1.584689in}{1.710104in}}%
\pgfpathcurveto{\pgfqpoint{1.584689in}{1.699054in}}{\pgfqpoint{1.589079in}{1.688455in}}{\pgfqpoint{1.596893in}{1.680641in}}%
\pgfpathcurveto{\pgfqpoint{1.604706in}{1.672828in}}{\pgfqpoint{1.615305in}{1.668438in}}{\pgfqpoint{1.626356in}{1.668438in}}%
\pgfpathlineto{\pgfqpoint{1.626356in}{1.668438in}}%
\pgfpathclose%
\pgfusepath{stroke}%
\end{pgfscope}%
\begin{pgfscope}%
\pgfpathrectangle{\pgfqpoint{0.393053in}{0.375000in}}{\pgfqpoint{6.356833in}{5.175000in}}%
\pgfusepath{clip}%
\pgfsetbuttcap%
\pgfsetroundjoin%
\pgfsetlinewidth{1.003750pt}%
\definecolor{currentstroke}{rgb}{0.827451,0.827451,0.827451}%
\pgfsetstrokecolor{currentstroke}%
\pgfsetdash{}{0pt}%
\pgfpathmoveto{\pgfqpoint{3.256902in}{0.730828in}}%
\pgfpathcurveto{\pgfqpoint{3.267953in}{0.730828in}}{\pgfqpoint{3.278552in}{0.735218in}}{\pgfqpoint{3.286365in}{0.743032in}}%
\pgfpathcurveto{\pgfqpoint{3.294179in}{0.750846in}}{\pgfqpoint{3.298569in}{0.761445in}}{\pgfqpoint{3.298569in}{0.772495in}}%
\pgfpathcurveto{\pgfqpoint{3.298569in}{0.783545in}}{\pgfqpoint{3.294179in}{0.794144in}}{\pgfqpoint{3.286365in}{0.801958in}}%
\pgfpathcurveto{\pgfqpoint{3.278552in}{0.809771in}}{\pgfqpoint{3.267953in}{0.814162in}}{\pgfqpoint{3.256902in}{0.814162in}}%
\pgfpathcurveto{\pgfqpoint{3.245852in}{0.814162in}}{\pgfqpoint{3.235253in}{0.809771in}}{\pgfqpoint{3.227440in}{0.801958in}}%
\pgfpathcurveto{\pgfqpoint{3.219626in}{0.794144in}}{\pgfqpoint{3.215236in}{0.783545in}}{\pgfqpoint{3.215236in}{0.772495in}}%
\pgfpathcurveto{\pgfqpoint{3.215236in}{0.761445in}}{\pgfqpoint{3.219626in}{0.750846in}}{\pgfqpoint{3.227440in}{0.743032in}}%
\pgfpathcurveto{\pgfqpoint{3.235253in}{0.735218in}}{\pgfqpoint{3.245852in}{0.730828in}}{\pgfqpoint{3.256902in}{0.730828in}}%
\pgfpathlineto{\pgfqpoint{3.256902in}{0.730828in}}%
\pgfpathclose%
\pgfusepath{stroke}%
\end{pgfscope}%
\begin{pgfscope}%
\pgfpathrectangle{\pgfqpoint{0.393053in}{0.375000in}}{\pgfqpoint{6.356833in}{5.175000in}}%
\pgfusepath{clip}%
\pgfsetbuttcap%
\pgfsetroundjoin%
\pgfsetlinewidth{1.003750pt}%
\definecolor{currentstroke}{rgb}{0.827451,0.827451,0.827451}%
\pgfsetstrokecolor{currentstroke}%
\pgfsetdash{}{0pt}%
\pgfpathmoveto{\pgfqpoint{3.677938in}{0.575928in}}%
\pgfpathcurveto{\pgfqpoint{3.688988in}{0.575928in}}{\pgfqpoint{3.699587in}{0.580318in}}{\pgfqpoint{3.707401in}{0.588132in}}%
\pgfpathcurveto{\pgfqpoint{3.715214in}{0.595945in}}{\pgfqpoint{3.719604in}{0.606544in}}{\pgfqpoint{3.719604in}{0.617595in}}%
\pgfpathcurveto{\pgfqpoint{3.719604in}{0.628645in}}{\pgfqpoint{3.715214in}{0.639244in}}{\pgfqpoint{3.707401in}{0.647057in}}%
\pgfpathcurveto{\pgfqpoint{3.699587in}{0.654871in}}{\pgfqpoint{3.688988in}{0.659261in}}{\pgfqpoint{3.677938in}{0.659261in}}%
\pgfpathcurveto{\pgfqpoint{3.666888in}{0.659261in}}{\pgfqpoint{3.656289in}{0.654871in}}{\pgfqpoint{3.648475in}{0.647057in}}%
\pgfpathcurveto{\pgfqpoint{3.640661in}{0.639244in}}{\pgfqpoint{3.636271in}{0.628645in}}{\pgfqpoint{3.636271in}{0.617595in}}%
\pgfpathcurveto{\pgfqpoint{3.636271in}{0.606544in}}{\pgfqpoint{3.640661in}{0.595945in}}{\pgfqpoint{3.648475in}{0.588132in}}%
\pgfpathcurveto{\pgfqpoint{3.656289in}{0.580318in}}{\pgfqpoint{3.666888in}{0.575928in}}{\pgfqpoint{3.677938in}{0.575928in}}%
\pgfpathlineto{\pgfqpoint{3.677938in}{0.575928in}}%
\pgfpathclose%
\pgfusepath{stroke}%
\end{pgfscope}%
\begin{pgfscope}%
\pgfpathrectangle{\pgfqpoint{0.393053in}{0.375000in}}{\pgfqpoint{6.356833in}{5.175000in}}%
\pgfusepath{clip}%
\pgfsetbuttcap%
\pgfsetroundjoin%
\pgfsetlinewidth{1.003750pt}%
\definecolor{currentstroke}{rgb}{0.827451,0.827451,0.827451}%
\pgfsetstrokecolor{currentstroke}%
\pgfsetdash{}{0pt}%
\pgfpathmoveto{\pgfqpoint{0.583953in}{3.273995in}}%
\pgfpathcurveto{\pgfqpoint{0.595003in}{3.273995in}}{\pgfqpoint{0.605602in}{3.278385in}}{\pgfqpoint{0.613416in}{3.286198in}}%
\pgfpathcurveto{\pgfqpoint{0.621229in}{3.294012in}}{\pgfqpoint{0.625620in}{3.304611in}}{\pgfqpoint{0.625620in}{3.315661in}}%
\pgfpathcurveto{\pgfqpoint{0.625620in}{3.326711in}}{\pgfqpoint{0.621229in}{3.337310in}}{\pgfqpoint{0.613416in}{3.345124in}}%
\pgfpathcurveto{\pgfqpoint{0.605602in}{3.352938in}}{\pgfqpoint{0.595003in}{3.357328in}}{\pgfqpoint{0.583953in}{3.357328in}}%
\pgfpathcurveto{\pgfqpoint{0.572903in}{3.357328in}}{\pgfqpoint{0.562304in}{3.352938in}}{\pgfqpoint{0.554490in}{3.345124in}}%
\pgfpathcurveto{\pgfqpoint{0.546677in}{3.337310in}}{\pgfqpoint{0.542286in}{3.326711in}}{\pgfqpoint{0.542286in}{3.315661in}}%
\pgfpathcurveto{\pgfqpoint{0.542286in}{3.304611in}}{\pgfqpoint{0.546677in}{3.294012in}}{\pgfqpoint{0.554490in}{3.286198in}}%
\pgfpathcurveto{\pgfqpoint{0.562304in}{3.278385in}}{\pgfqpoint{0.572903in}{3.273995in}}{\pgfqpoint{0.583953in}{3.273995in}}%
\pgfpathlineto{\pgfqpoint{0.583953in}{3.273995in}}%
\pgfpathclose%
\pgfusepath{stroke}%
\end{pgfscope}%
\begin{pgfscope}%
\pgfpathrectangle{\pgfqpoint{0.393053in}{0.375000in}}{\pgfqpoint{6.356833in}{5.175000in}}%
\pgfusepath{clip}%
\pgfsetbuttcap%
\pgfsetroundjoin%
\pgfsetlinewidth{1.003750pt}%
\definecolor{currentstroke}{rgb}{0.827451,0.827451,0.827451}%
\pgfsetstrokecolor{currentstroke}%
\pgfsetdash{}{0pt}%
\pgfpathmoveto{\pgfqpoint{0.401273in}{4.297038in}}%
\pgfpathcurveto{\pgfqpoint{0.412323in}{4.297038in}}{\pgfqpoint{0.422922in}{4.301428in}}{\pgfqpoint{0.430735in}{4.309242in}}%
\pgfpathcurveto{\pgfqpoint{0.438549in}{4.317056in}}{\pgfqpoint{0.442939in}{4.327655in}}{\pgfqpoint{0.442939in}{4.338705in}}%
\pgfpathcurveto{\pgfqpoint{0.442939in}{4.349755in}}{\pgfqpoint{0.438549in}{4.360354in}}{\pgfqpoint{0.430735in}{4.368168in}}%
\pgfpathcurveto{\pgfqpoint{0.422922in}{4.375981in}}{\pgfqpoint{0.412323in}{4.380371in}}{\pgfqpoint{0.401273in}{4.380371in}}%
\pgfpathcurveto{\pgfqpoint{0.390222in}{4.380371in}}{\pgfqpoint{0.379623in}{4.375981in}}{\pgfqpoint{0.371810in}{4.368168in}}%
\pgfpathcurveto{\pgfqpoint{0.363996in}{4.360354in}}{\pgfqpoint{0.359606in}{4.349755in}}{\pgfqpoint{0.359606in}{4.338705in}}%
\pgfpathcurveto{\pgfqpoint{0.359606in}{4.327655in}}{\pgfqpoint{0.363996in}{4.317056in}}{\pgfqpoint{0.371810in}{4.309242in}}%
\pgfpathcurveto{\pgfqpoint{0.379623in}{4.301428in}}{\pgfqpoint{0.390222in}{4.297038in}}{\pgfqpoint{0.401273in}{4.297038in}}%
\pgfpathlineto{\pgfqpoint{0.401273in}{4.297038in}}%
\pgfpathclose%
\pgfusepath{stroke}%
\end{pgfscope}%
\begin{pgfscope}%
\pgfpathrectangle{\pgfqpoint{0.393053in}{0.375000in}}{\pgfqpoint{6.356833in}{5.175000in}}%
\pgfusepath{clip}%
\pgfsetbuttcap%
\pgfsetroundjoin%
\pgfsetlinewidth{1.003750pt}%
\definecolor{currentstroke}{rgb}{0.827451,0.827451,0.827451}%
\pgfsetstrokecolor{currentstroke}%
\pgfsetdash{}{0pt}%
\pgfpathmoveto{\pgfqpoint{0.635911in}{3.119895in}}%
\pgfpathcurveto{\pgfqpoint{0.646961in}{3.119895in}}{\pgfqpoint{0.657560in}{3.124285in}}{\pgfqpoint{0.665374in}{3.132098in}}%
\pgfpathcurveto{\pgfqpoint{0.673187in}{3.139912in}}{\pgfqpoint{0.677577in}{3.150511in}}{\pgfqpoint{0.677577in}{3.161561in}}%
\pgfpathcurveto{\pgfqpoint{0.677577in}{3.172611in}}{\pgfqpoint{0.673187in}{3.183210in}}{\pgfqpoint{0.665374in}{3.191024in}}%
\pgfpathcurveto{\pgfqpoint{0.657560in}{3.198838in}}{\pgfqpoint{0.646961in}{3.203228in}}{\pgfqpoint{0.635911in}{3.203228in}}%
\pgfpathcurveto{\pgfqpoint{0.624861in}{3.203228in}}{\pgfqpoint{0.614262in}{3.198838in}}{\pgfqpoint{0.606448in}{3.191024in}}%
\pgfpathcurveto{\pgfqpoint{0.598634in}{3.183210in}}{\pgfqpoint{0.594244in}{3.172611in}}{\pgfqpoint{0.594244in}{3.161561in}}%
\pgfpathcurveto{\pgfqpoint{0.594244in}{3.150511in}}{\pgfqpoint{0.598634in}{3.139912in}}{\pgfqpoint{0.606448in}{3.132098in}}%
\pgfpathcurveto{\pgfqpoint{0.614262in}{3.124285in}}{\pgfqpoint{0.624861in}{3.119895in}}{\pgfqpoint{0.635911in}{3.119895in}}%
\pgfpathlineto{\pgfqpoint{0.635911in}{3.119895in}}%
\pgfpathclose%
\pgfusepath{stroke}%
\end{pgfscope}%
\begin{pgfscope}%
\pgfpathrectangle{\pgfqpoint{0.393053in}{0.375000in}}{\pgfqpoint{6.356833in}{5.175000in}}%
\pgfusepath{clip}%
\pgfsetbuttcap%
\pgfsetroundjoin%
\pgfsetlinewidth{1.003750pt}%
\definecolor{currentstroke}{rgb}{0.827451,0.827451,0.827451}%
\pgfsetstrokecolor{currentstroke}%
\pgfsetdash{}{0pt}%
\pgfpathmoveto{\pgfqpoint{2.920897in}{0.851071in}}%
\pgfpathcurveto{\pgfqpoint{2.931947in}{0.851071in}}{\pgfqpoint{2.942546in}{0.855462in}}{\pgfqpoint{2.950360in}{0.863275in}}%
\pgfpathcurveto{\pgfqpoint{2.958173in}{0.871089in}}{\pgfqpoint{2.962563in}{0.881688in}}{\pgfqpoint{2.962563in}{0.892738in}}%
\pgfpathcurveto{\pgfqpoint{2.962563in}{0.903788in}}{\pgfqpoint{2.958173in}{0.914387in}}{\pgfqpoint{2.950360in}{0.922201in}}%
\pgfpathcurveto{\pgfqpoint{2.942546in}{0.930014in}}{\pgfqpoint{2.931947in}{0.934405in}}{\pgfqpoint{2.920897in}{0.934405in}}%
\pgfpathcurveto{\pgfqpoint{2.909847in}{0.934405in}}{\pgfqpoint{2.899248in}{0.930014in}}{\pgfqpoint{2.891434in}{0.922201in}}%
\pgfpathcurveto{\pgfqpoint{2.883620in}{0.914387in}}{\pgfqpoint{2.879230in}{0.903788in}}{\pgfqpoint{2.879230in}{0.892738in}}%
\pgfpathcurveto{\pgfqpoint{2.879230in}{0.881688in}}{\pgfqpoint{2.883620in}{0.871089in}}{\pgfqpoint{2.891434in}{0.863275in}}%
\pgfpathcurveto{\pgfqpoint{2.899248in}{0.855462in}}{\pgfqpoint{2.909847in}{0.851071in}}{\pgfqpoint{2.920897in}{0.851071in}}%
\pgfpathlineto{\pgfqpoint{2.920897in}{0.851071in}}%
\pgfpathclose%
\pgfusepath{stroke}%
\end{pgfscope}%
\begin{pgfscope}%
\pgfpathrectangle{\pgfqpoint{0.393053in}{0.375000in}}{\pgfqpoint{6.356833in}{5.175000in}}%
\pgfusepath{clip}%
\pgfsetbuttcap%
\pgfsetroundjoin%
\pgfsetlinewidth{1.003750pt}%
\definecolor{currentstroke}{rgb}{0.827451,0.827451,0.827451}%
\pgfsetstrokecolor{currentstroke}%
\pgfsetdash{}{0pt}%
\pgfpathmoveto{\pgfqpoint{3.077579in}{0.783830in}}%
\pgfpathcurveto{\pgfqpoint{3.088629in}{0.783830in}}{\pgfqpoint{3.099228in}{0.788220in}}{\pgfqpoint{3.107042in}{0.796034in}}%
\pgfpathcurveto{\pgfqpoint{3.114856in}{0.803847in}}{\pgfqpoint{3.119246in}{0.814446in}}{\pgfqpoint{3.119246in}{0.825497in}}%
\pgfpathcurveto{\pgfqpoint{3.119246in}{0.836547in}}{\pgfqpoint{3.114856in}{0.847146in}}{\pgfqpoint{3.107042in}{0.854959in}}%
\pgfpathcurveto{\pgfqpoint{3.099228in}{0.862773in}}{\pgfqpoint{3.088629in}{0.867163in}}{\pgfqpoint{3.077579in}{0.867163in}}%
\pgfpathcurveto{\pgfqpoint{3.066529in}{0.867163in}}{\pgfqpoint{3.055930in}{0.862773in}}{\pgfqpoint{3.048117in}{0.854959in}}%
\pgfpathcurveto{\pgfqpoint{3.040303in}{0.847146in}}{\pgfqpoint{3.035913in}{0.836547in}}{\pgfqpoint{3.035913in}{0.825497in}}%
\pgfpathcurveto{\pgfqpoint{3.035913in}{0.814446in}}{\pgfqpoint{3.040303in}{0.803847in}}{\pgfqpoint{3.048117in}{0.796034in}}%
\pgfpathcurveto{\pgfqpoint{3.055930in}{0.788220in}}{\pgfqpoint{3.066529in}{0.783830in}}{\pgfqpoint{3.077579in}{0.783830in}}%
\pgfpathlineto{\pgfqpoint{3.077579in}{0.783830in}}%
\pgfpathclose%
\pgfusepath{stroke}%
\end{pgfscope}%
\begin{pgfscope}%
\pgfpathrectangle{\pgfqpoint{0.393053in}{0.375000in}}{\pgfqpoint{6.356833in}{5.175000in}}%
\pgfusepath{clip}%
\pgfsetbuttcap%
\pgfsetroundjoin%
\pgfsetlinewidth{1.003750pt}%
\definecolor{currentstroke}{rgb}{0.827451,0.827451,0.827451}%
\pgfsetstrokecolor{currentstroke}%
\pgfsetdash{}{0pt}%
\pgfpathmoveto{\pgfqpoint{1.312906in}{1.976052in}}%
\pgfpathcurveto{\pgfqpoint{1.323956in}{1.976052in}}{\pgfqpoint{1.334555in}{1.980442in}}{\pgfqpoint{1.342369in}{1.988256in}}%
\pgfpathcurveto{\pgfqpoint{1.350183in}{1.996070in}}{\pgfqpoint{1.354573in}{2.006669in}}{\pgfqpoint{1.354573in}{2.017719in}}%
\pgfpathcurveto{\pgfqpoint{1.354573in}{2.028769in}}{\pgfqpoint{1.350183in}{2.039368in}}{\pgfqpoint{1.342369in}{2.047182in}}%
\pgfpathcurveto{\pgfqpoint{1.334555in}{2.054995in}}{\pgfqpoint{1.323956in}{2.059386in}}{\pgfqpoint{1.312906in}{2.059386in}}%
\pgfpathcurveto{\pgfqpoint{1.301856in}{2.059386in}}{\pgfqpoint{1.291257in}{2.054995in}}{\pgfqpoint{1.283443in}{2.047182in}}%
\pgfpathcurveto{\pgfqpoint{1.275630in}{2.039368in}}{\pgfqpoint{1.271240in}{2.028769in}}{\pgfqpoint{1.271240in}{2.017719in}}%
\pgfpathcurveto{\pgfqpoint{1.271240in}{2.006669in}}{\pgfqpoint{1.275630in}{1.996070in}}{\pgfqpoint{1.283443in}{1.988256in}}%
\pgfpathcurveto{\pgfqpoint{1.291257in}{1.980442in}}{\pgfqpoint{1.301856in}{1.976052in}}{\pgfqpoint{1.312906in}{1.976052in}}%
\pgfpathlineto{\pgfqpoint{1.312906in}{1.976052in}}%
\pgfpathclose%
\pgfusepath{stroke}%
\end{pgfscope}%
\begin{pgfscope}%
\pgfpathrectangle{\pgfqpoint{0.393053in}{0.375000in}}{\pgfqpoint{6.356833in}{5.175000in}}%
\pgfusepath{clip}%
\pgfsetbuttcap%
\pgfsetroundjoin%
\pgfsetlinewidth{1.003750pt}%
\definecolor{currentstroke}{rgb}{0.827451,0.827451,0.827451}%
\pgfsetstrokecolor{currentstroke}%
\pgfsetdash{}{0pt}%
\pgfpathmoveto{\pgfqpoint{1.291595in}{1.999205in}}%
\pgfpathcurveto{\pgfqpoint{1.302646in}{1.999205in}}{\pgfqpoint{1.313245in}{2.003595in}}{\pgfqpoint{1.321058in}{2.011409in}}%
\pgfpathcurveto{\pgfqpoint{1.328872in}{2.019222in}}{\pgfqpoint{1.333262in}{2.029821in}}{\pgfqpoint{1.333262in}{2.040872in}}%
\pgfpathcurveto{\pgfqpoint{1.333262in}{2.051922in}}{\pgfqpoint{1.328872in}{2.062521in}}{\pgfqpoint{1.321058in}{2.070334in}}%
\pgfpathcurveto{\pgfqpoint{1.313245in}{2.078148in}}{\pgfqpoint{1.302646in}{2.082538in}}{\pgfqpoint{1.291595in}{2.082538in}}%
\pgfpathcurveto{\pgfqpoint{1.280545in}{2.082538in}}{\pgfqpoint{1.269946in}{2.078148in}}{\pgfqpoint{1.262133in}{2.070334in}}%
\pgfpathcurveto{\pgfqpoint{1.254319in}{2.062521in}}{\pgfqpoint{1.249929in}{2.051922in}}{\pgfqpoint{1.249929in}{2.040872in}}%
\pgfpathcurveto{\pgfqpoint{1.249929in}{2.029821in}}{\pgfqpoint{1.254319in}{2.019222in}}{\pgfqpoint{1.262133in}{2.011409in}}%
\pgfpathcurveto{\pgfqpoint{1.269946in}{2.003595in}}{\pgfqpoint{1.280545in}{1.999205in}}{\pgfqpoint{1.291595in}{1.999205in}}%
\pgfpathlineto{\pgfqpoint{1.291595in}{1.999205in}}%
\pgfpathclose%
\pgfusepath{stroke}%
\end{pgfscope}%
\begin{pgfscope}%
\pgfpathrectangle{\pgfqpoint{0.393053in}{0.375000in}}{\pgfqpoint{6.356833in}{5.175000in}}%
\pgfusepath{clip}%
\pgfsetbuttcap%
\pgfsetroundjoin%
\pgfsetlinewidth{1.003750pt}%
\definecolor{currentstroke}{rgb}{0.827451,0.827451,0.827451}%
\pgfsetstrokecolor{currentstroke}%
\pgfsetdash{}{0pt}%
\pgfpathmoveto{\pgfqpoint{4.217280in}{0.449238in}}%
\pgfpathcurveto{\pgfqpoint{4.228330in}{0.449238in}}{\pgfqpoint{4.238929in}{0.453628in}}{\pgfqpoint{4.246743in}{0.461441in}}%
\pgfpathcurveto{\pgfqpoint{4.254556in}{0.469255in}}{\pgfqpoint{4.258947in}{0.479854in}}{\pgfqpoint{4.258947in}{0.490904in}}%
\pgfpathcurveto{\pgfqpoint{4.258947in}{0.501954in}}{\pgfqpoint{4.254556in}{0.512553in}}{\pgfqpoint{4.246743in}{0.520367in}}%
\pgfpathcurveto{\pgfqpoint{4.238929in}{0.528181in}}{\pgfqpoint{4.228330in}{0.532571in}}{\pgfqpoint{4.217280in}{0.532571in}}%
\pgfpathcurveto{\pgfqpoint{4.206230in}{0.532571in}}{\pgfqpoint{4.195631in}{0.528181in}}{\pgfqpoint{4.187817in}{0.520367in}}%
\pgfpathcurveto{\pgfqpoint{4.180004in}{0.512553in}}{\pgfqpoint{4.175613in}{0.501954in}}{\pgfqpoint{4.175613in}{0.490904in}}%
\pgfpathcurveto{\pgfqpoint{4.175613in}{0.479854in}}{\pgfqpoint{4.180004in}{0.469255in}}{\pgfqpoint{4.187817in}{0.461441in}}%
\pgfpathcurveto{\pgfqpoint{4.195631in}{0.453628in}}{\pgfqpoint{4.206230in}{0.449238in}}{\pgfqpoint{4.217280in}{0.449238in}}%
\pgfpathlineto{\pgfqpoint{4.217280in}{0.449238in}}%
\pgfpathclose%
\pgfusepath{stroke}%
\end{pgfscope}%
\begin{pgfscope}%
\pgfpathrectangle{\pgfqpoint{0.393053in}{0.375000in}}{\pgfqpoint{6.356833in}{5.175000in}}%
\pgfusepath{clip}%
\pgfsetbuttcap%
\pgfsetroundjoin%
\pgfsetlinewidth{1.003750pt}%
\definecolor{currentstroke}{rgb}{0.827451,0.827451,0.827451}%
\pgfsetstrokecolor{currentstroke}%
\pgfsetdash{}{0pt}%
\pgfpathmoveto{\pgfqpoint{2.145018in}{1.265722in}}%
\pgfpathcurveto{\pgfqpoint{2.156069in}{1.265722in}}{\pgfqpoint{2.166668in}{1.270112in}}{\pgfqpoint{2.174481in}{1.277926in}}%
\pgfpathcurveto{\pgfqpoint{2.182295in}{1.285739in}}{\pgfqpoint{2.186685in}{1.296338in}}{\pgfqpoint{2.186685in}{1.307389in}}%
\pgfpathcurveto{\pgfqpoint{2.186685in}{1.318439in}}{\pgfqpoint{2.182295in}{1.329038in}}{\pgfqpoint{2.174481in}{1.336851in}}%
\pgfpathcurveto{\pgfqpoint{2.166668in}{1.344665in}}{\pgfqpoint{2.156069in}{1.349055in}}{\pgfqpoint{2.145018in}{1.349055in}}%
\pgfpathcurveto{\pgfqpoint{2.133968in}{1.349055in}}{\pgfqpoint{2.123369in}{1.344665in}}{\pgfqpoint{2.115556in}{1.336851in}}%
\pgfpathcurveto{\pgfqpoint{2.107742in}{1.329038in}}{\pgfqpoint{2.103352in}{1.318439in}}{\pgfqpoint{2.103352in}{1.307389in}}%
\pgfpathcurveto{\pgfqpoint{2.103352in}{1.296338in}}{\pgfqpoint{2.107742in}{1.285739in}}{\pgfqpoint{2.115556in}{1.277926in}}%
\pgfpathcurveto{\pgfqpoint{2.123369in}{1.270112in}}{\pgfqpoint{2.133968in}{1.265722in}}{\pgfqpoint{2.145018in}{1.265722in}}%
\pgfpathlineto{\pgfqpoint{2.145018in}{1.265722in}}%
\pgfpathclose%
\pgfusepath{stroke}%
\end{pgfscope}%
\begin{pgfscope}%
\pgfpathrectangle{\pgfqpoint{0.393053in}{0.375000in}}{\pgfqpoint{6.356833in}{5.175000in}}%
\pgfusepath{clip}%
\pgfsetbuttcap%
\pgfsetroundjoin%
\pgfsetlinewidth{1.003750pt}%
\definecolor{currentstroke}{rgb}{0.827451,0.827451,0.827451}%
\pgfsetstrokecolor{currentstroke}%
\pgfsetdash{}{0pt}%
\pgfpathmoveto{\pgfqpoint{4.176160in}{0.456257in}}%
\pgfpathcurveto{\pgfqpoint{4.187210in}{0.456257in}}{\pgfqpoint{4.197809in}{0.460647in}}{\pgfqpoint{4.205623in}{0.468461in}}%
\pgfpathcurveto{\pgfqpoint{4.213436in}{0.476274in}}{\pgfqpoint{4.217827in}{0.486873in}}{\pgfqpoint{4.217827in}{0.497923in}}%
\pgfpathcurveto{\pgfqpoint{4.217827in}{0.508973in}}{\pgfqpoint{4.213436in}{0.519572in}}{\pgfqpoint{4.205623in}{0.527386in}}%
\pgfpathcurveto{\pgfqpoint{4.197809in}{0.535200in}}{\pgfqpoint{4.187210in}{0.539590in}}{\pgfqpoint{4.176160in}{0.539590in}}%
\pgfpathcurveto{\pgfqpoint{4.165110in}{0.539590in}}{\pgfqpoint{4.154511in}{0.535200in}}{\pgfqpoint{4.146697in}{0.527386in}}%
\pgfpathcurveto{\pgfqpoint{4.138883in}{0.519572in}}{\pgfqpoint{4.134493in}{0.508973in}}{\pgfqpoint{4.134493in}{0.497923in}}%
\pgfpathcurveto{\pgfqpoint{4.134493in}{0.486873in}}{\pgfqpoint{4.138883in}{0.476274in}}{\pgfqpoint{4.146697in}{0.468461in}}%
\pgfpathcurveto{\pgfqpoint{4.154511in}{0.460647in}}{\pgfqpoint{4.165110in}{0.456257in}}{\pgfqpoint{4.176160in}{0.456257in}}%
\pgfpathlineto{\pgfqpoint{4.176160in}{0.456257in}}%
\pgfpathclose%
\pgfusepath{stroke}%
\end{pgfscope}%
\begin{pgfscope}%
\pgfpathrectangle{\pgfqpoint{0.393053in}{0.375000in}}{\pgfqpoint{6.356833in}{5.175000in}}%
\pgfusepath{clip}%
\pgfsetbuttcap%
\pgfsetroundjoin%
\pgfsetlinewidth{1.003750pt}%
\definecolor{currentstroke}{rgb}{0.827451,0.827451,0.827451}%
\pgfsetstrokecolor{currentstroke}%
\pgfsetdash{}{0pt}%
\pgfpathmoveto{\pgfqpoint{1.457810in}{1.831739in}}%
\pgfpathcurveto{\pgfqpoint{1.468860in}{1.831739in}}{\pgfqpoint{1.479459in}{1.836129in}}{\pgfqpoint{1.487273in}{1.843943in}}%
\pgfpathcurveto{\pgfqpoint{1.495086in}{1.851756in}}{\pgfqpoint{1.499477in}{1.862355in}}{\pgfqpoint{1.499477in}{1.873406in}}%
\pgfpathcurveto{\pgfqpoint{1.499477in}{1.884456in}}{\pgfqpoint{1.495086in}{1.895055in}}{\pgfqpoint{1.487273in}{1.902868in}}%
\pgfpathcurveto{\pgfqpoint{1.479459in}{1.910682in}}{\pgfqpoint{1.468860in}{1.915072in}}{\pgfqpoint{1.457810in}{1.915072in}}%
\pgfpathcurveto{\pgfqpoint{1.446760in}{1.915072in}}{\pgfqpoint{1.436161in}{1.910682in}}{\pgfqpoint{1.428347in}{1.902868in}}%
\pgfpathcurveto{\pgfqpoint{1.420533in}{1.895055in}}{\pgfqpoint{1.416143in}{1.884456in}}{\pgfqpoint{1.416143in}{1.873406in}}%
\pgfpathcurveto{\pgfqpoint{1.416143in}{1.862355in}}{\pgfqpoint{1.420533in}{1.851756in}}{\pgfqpoint{1.428347in}{1.843943in}}%
\pgfpathcurveto{\pgfqpoint{1.436161in}{1.836129in}}{\pgfqpoint{1.446760in}{1.831739in}}{\pgfqpoint{1.457810in}{1.831739in}}%
\pgfpathlineto{\pgfqpoint{1.457810in}{1.831739in}}%
\pgfpathclose%
\pgfusepath{stroke}%
\end{pgfscope}%
\begin{pgfscope}%
\pgfpathrectangle{\pgfqpoint{0.393053in}{0.375000in}}{\pgfqpoint{6.356833in}{5.175000in}}%
\pgfusepath{clip}%
\pgfsetbuttcap%
\pgfsetroundjoin%
\pgfsetlinewidth{1.003750pt}%
\definecolor{currentstroke}{rgb}{0.827451,0.827451,0.827451}%
\pgfsetstrokecolor{currentstroke}%
\pgfsetdash{}{0pt}%
\pgfpathmoveto{\pgfqpoint{0.744021in}{2.849644in}}%
\pgfpathcurveto{\pgfqpoint{0.755071in}{2.849644in}}{\pgfqpoint{0.765670in}{2.854034in}}{\pgfqpoint{0.773484in}{2.861848in}}%
\pgfpathcurveto{\pgfqpoint{0.781298in}{2.869661in}}{\pgfqpoint{0.785688in}{2.880260in}}{\pgfqpoint{0.785688in}{2.891311in}}%
\pgfpathcurveto{\pgfqpoint{0.785688in}{2.902361in}}{\pgfqpoint{0.781298in}{2.912960in}}{\pgfqpoint{0.773484in}{2.920773in}}%
\pgfpathcurveto{\pgfqpoint{0.765670in}{2.928587in}}{\pgfqpoint{0.755071in}{2.932977in}}{\pgfqpoint{0.744021in}{2.932977in}}%
\pgfpathcurveto{\pgfqpoint{0.732971in}{2.932977in}}{\pgfqpoint{0.722372in}{2.928587in}}{\pgfqpoint{0.714558in}{2.920773in}}%
\pgfpathcurveto{\pgfqpoint{0.706745in}{2.912960in}}{\pgfqpoint{0.702355in}{2.902361in}}{\pgfqpoint{0.702355in}{2.891311in}}%
\pgfpathcurveto{\pgfqpoint{0.702355in}{2.880260in}}{\pgfqpoint{0.706745in}{2.869661in}}{\pgfqpoint{0.714558in}{2.861848in}}%
\pgfpathcurveto{\pgfqpoint{0.722372in}{2.854034in}}{\pgfqpoint{0.732971in}{2.849644in}}{\pgfqpoint{0.744021in}{2.849644in}}%
\pgfpathlineto{\pgfqpoint{0.744021in}{2.849644in}}%
\pgfpathclose%
\pgfusepath{stroke}%
\end{pgfscope}%
\begin{pgfscope}%
\pgfpathrectangle{\pgfqpoint{0.393053in}{0.375000in}}{\pgfqpoint{6.356833in}{5.175000in}}%
\pgfusepath{clip}%
\pgfsetbuttcap%
\pgfsetroundjoin%
\pgfsetlinewidth{1.003750pt}%
\definecolor{currentstroke}{rgb}{0.827451,0.827451,0.827451}%
\pgfsetstrokecolor{currentstroke}%
\pgfsetdash{}{0pt}%
\pgfpathmoveto{\pgfqpoint{5.724126in}{0.333770in}}%
\pgfpathcurveto{\pgfqpoint{5.735176in}{0.333770in}}{\pgfqpoint{5.745775in}{0.338161in}}{\pgfqpoint{5.753589in}{0.345974in}}%
\pgfpathcurveto{\pgfqpoint{5.761402in}{0.353788in}}{\pgfqpoint{5.765793in}{0.364387in}}{\pgfqpoint{5.765793in}{0.375437in}}%
\pgfpathcurveto{\pgfqpoint{5.765793in}{0.386487in}}{\pgfqpoint{5.761402in}{0.397086in}}{\pgfqpoint{5.753589in}{0.404900in}}%
\pgfpathcurveto{\pgfqpoint{5.745775in}{0.412713in}}{\pgfqpoint{5.735176in}{0.417104in}}{\pgfqpoint{5.724126in}{0.417104in}}%
\pgfpathcurveto{\pgfqpoint{5.713076in}{0.417104in}}{\pgfqpoint{5.702477in}{0.412713in}}{\pgfqpoint{5.694663in}{0.404900in}}%
\pgfpathcurveto{\pgfqpoint{5.686850in}{0.397086in}}{\pgfqpoint{5.682459in}{0.386487in}}{\pgfqpoint{5.682459in}{0.375437in}}%
\pgfpathcurveto{\pgfqpoint{5.682459in}{0.364387in}}{\pgfqpoint{5.686850in}{0.353788in}}{\pgfqpoint{5.694663in}{0.345974in}}%
\pgfpathcurveto{\pgfqpoint{5.702477in}{0.338161in}}{\pgfqpoint{5.713076in}{0.333770in}}{\pgfqpoint{5.724126in}{0.333770in}}%
\pgfusepath{stroke}%
\end{pgfscope}%
\begin{pgfscope}%
\pgfpathrectangle{\pgfqpoint{0.393053in}{0.375000in}}{\pgfqpoint{6.356833in}{5.175000in}}%
\pgfusepath{clip}%
\pgfsetbuttcap%
\pgfsetroundjoin%
\pgfsetlinewidth{1.003750pt}%
\definecolor{currentstroke}{rgb}{0.827451,0.827451,0.827451}%
\pgfsetstrokecolor{currentstroke}%
\pgfsetdash{}{0pt}%
\pgfpathmoveto{\pgfqpoint{0.393415in}{4.523360in}}%
\pgfpathcurveto{\pgfqpoint{0.404465in}{4.523360in}}{\pgfqpoint{0.415064in}{4.527750in}}{\pgfqpoint{0.422878in}{4.535564in}}%
\pgfpathcurveto{\pgfqpoint{0.430691in}{4.543377in}}{\pgfqpoint{0.435082in}{4.553976in}}{\pgfqpoint{0.435082in}{4.565026in}}%
\pgfpathcurveto{\pgfqpoint{0.435082in}{4.576077in}}{\pgfqpoint{0.430691in}{4.586676in}}{\pgfqpoint{0.422878in}{4.594489in}}%
\pgfpathcurveto{\pgfqpoint{0.415064in}{4.602303in}}{\pgfqpoint{0.404465in}{4.606693in}}{\pgfqpoint{0.393415in}{4.606693in}}%
\pgfpathcurveto{\pgfqpoint{0.382365in}{4.606693in}}{\pgfqpoint{0.371766in}{4.602303in}}{\pgfqpoint{0.363952in}{4.594489in}}%
\pgfpathcurveto{\pgfqpoint{0.356139in}{4.586676in}}{\pgfqpoint{0.351748in}{4.576077in}}{\pgfqpoint{0.351748in}{4.565026in}}%
\pgfpathcurveto{\pgfqpoint{0.351748in}{4.553976in}}{\pgfqpoint{0.356139in}{4.543377in}}{\pgfqpoint{0.363952in}{4.535564in}}%
\pgfpathcurveto{\pgfqpoint{0.371766in}{4.527750in}}{\pgfqpoint{0.382365in}{4.523360in}}{\pgfqpoint{0.393415in}{4.523360in}}%
\pgfpathlineto{\pgfqpoint{0.393415in}{4.523360in}}%
\pgfpathclose%
\pgfusepath{stroke}%
\end{pgfscope}%
\begin{pgfscope}%
\pgfpathrectangle{\pgfqpoint{0.393053in}{0.375000in}}{\pgfqpoint{6.356833in}{5.175000in}}%
\pgfusepath{clip}%
\pgfsetbuttcap%
\pgfsetroundjoin%
\pgfsetlinewidth{1.003750pt}%
\definecolor{currentstroke}{rgb}{0.827451,0.827451,0.827451}%
\pgfsetstrokecolor{currentstroke}%
\pgfsetdash{}{0pt}%
\pgfpathmoveto{\pgfqpoint{2.920897in}{0.851071in}}%
\pgfpathcurveto{\pgfqpoint{2.931947in}{0.851071in}}{\pgfqpoint{2.942546in}{0.855462in}}{\pgfqpoint{2.950360in}{0.863275in}}%
\pgfpathcurveto{\pgfqpoint{2.958173in}{0.871089in}}{\pgfqpoint{2.962563in}{0.881688in}}{\pgfqpoint{2.962563in}{0.892738in}}%
\pgfpathcurveto{\pgfqpoint{2.962563in}{0.903788in}}{\pgfqpoint{2.958173in}{0.914387in}}{\pgfqpoint{2.950360in}{0.922201in}}%
\pgfpathcurveto{\pgfqpoint{2.942546in}{0.930014in}}{\pgfqpoint{2.931947in}{0.934405in}}{\pgfqpoint{2.920897in}{0.934405in}}%
\pgfpathcurveto{\pgfqpoint{2.909847in}{0.934405in}}{\pgfqpoint{2.899248in}{0.930014in}}{\pgfqpoint{2.891434in}{0.922201in}}%
\pgfpathcurveto{\pgfqpoint{2.883620in}{0.914387in}}{\pgfqpoint{2.879230in}{0.903788in}}{\pgfqpoint{2.879230in}{0.892738in}}%
\pgfpathcurveto{\pgfqpoint{2.879230in}{0.881688in}}{\pgfqpoint{2.883620in}{0.871089in}}{\pgfqpoint{2.891434in}{0.863275in}}%
\pgfpathcurveto{\pgfqpoint{2.899248in}{0.855462in}}{\pgfqpoint{2.909847in}{0.851071in}}{\pgfqpoint{2.920897in}{0.851071in}}%
\pgfpathlineto{\pgfqpoint{2.920897in}{0.851071in}}%
\pgfpathclose%
\pgfusepath{stroke}%
\end{pgfscope}%
\begin{pgfscope}%
\pgfpathrectangle{\pgfqpoint{0.393053in}{0.375000in}}{\pgfqpoint{6.356833in}{5.175000in}}%
\pgfusepath{clip}%
\pgfsetbuttcap%
\pgfsetroundjoin%
\pgfsetlinewidth{1.003750pt}%
\definecolor{currentstroke}{rgb}{0.827451,0.827451,0.827451}%
\pgfsetstrokecolor{currentstroke}%
\pgfsetdash{}{0pt}%
\pgfpathmoveto{\pgfqpoint{4.802412in}{0.377777in}}%
\pgfpathcurveto{\pgfqpoint{4.813462in}{0.377777in}}{\pgfqpoint{4.824061in}{0.382167in}}{\pgfqpoint{4.831874in}{0.389981in}}%
\pgfpathcurveto{\pgfqpoint{4.839688in}{0.397794in}}{\pgfqpoint{4.844078in}{0.408394in}}{\pgfqpoint{4.844078in}{0.419444in}}%
\pgfpathcurveto{\pgfqpoint{4.844078in}{0.430494in}}{\pgfqpoint{4.839688in}{0.441093in}}{\pgfqpoint{4.831874in}{0.448906in}}%
\pgfpathcurveto{\pgfqpoint{4.824061in}{0.456720in}}{\pgfqpoint{4.813462in}{0.461110in}}{\pgfqpoint{4.802412in}{0.461110in}}%
\pgfpathcurveto{\pgfqpoint{4.791362in}{0.461110in}}{\pgfqpoint{4.780762in}{0.456720in}}{\pgfqpoint{4.772949in}{0.448906in}}%
\pgfpathcurveto{\pgfqpoint{4.765135in}{0.441093in}}{\pgfqpoint{4.760745in}{0.430494in}}{\pgfqpoint{4.760745in}{0.419444in}}%
\pgfpathcurveto{\pgfqpoint{4.760745in}{0.408394in}}{\pgfqpoint{4.765135in}{0.397794in}}{\pgfqpoint{4.772949in}{0.389981in}}%
\pgfpathcurveto{\pgfqpoint{4.780762in}{0.382167in}}{\pgfqpoint{4.791362in}{0.377777in}}{\pgfqpoint{4.802412in}{0.377777in}}%
\pgfpathlineto{\pgfqpoint{4.802412in}{0.377777in}}%
\pgfpathclose%
\pgfusepath{stroke}%
\end{pgfscope}%
\begin{pgfscope}%
\pgfpathrectangle{\pgfqpoint{0.393053in}{0.375000in}}{\pgfqpoint{6.356833in}{5.175000in}}%
\pgfusepath{clip}%
\pgfsetbuttcap%
\pgfsetroundjoin%
\pgfsetlinewidth{1.003750pt}%
\definecolor{currentstroke}{rgb}{0.827451,0.827451,0.827451}%
\pgfsetstrokecolor{currentstroke}%
\pgfsetdash{}{0pt}%
\pgfpathmoveto{\pgfqpoint{5.017789in}{0.367635in}}%
\pgfpathcurveto{\pgfqpoint{5.028839in}{0.367635in}}{\pgfqpoint{5.039438in}{0.372025in}}{\pgfqpoint{5.047252in}{0.379839in}}%
\pgfpathcurveto{\pgfqpoint{5.055065in}{0.387652in}}{\pgfqpoint{5.059456in}{0.398251in}}{\pgfqpoint{5.059456in}{0.409302in}}%
\pgfpathcurveto{\pgfqpoint{5.059456in}{0.420352in}}{\pgfqpoint{5.055065in}{0.430951in}}{\pgfqpoint{5.047252in}{0.438764in}}%
\pgfpathcurveto{\pgfqpoint{5.039438in}{0.446578in}}{\pgfqpoint{5.028839in}{0.450968in}}{\pgfqpoint{5.017789in}{0.450968in}}%
\pgfpathcurveto{\pgfqpoint{5.006739in}{0.450968in}}{\pgfqpoint{4.996140in}{0.446578in}}{\pgfqpoint{4.988326in}{0.438764in}}%
\pgfpathcurveto{\pgfqpoint{4.980512in}{0.430951in}}{\pgfqpoint{4.976122in}{0.420352in}}{\pgfqpoint{4.976122in}{0.409302in}}%
\pgfpathcurveto{\pgfqpoint{4.976122in}{0.398251in}}{\pgfqpoint{4.980512in}{0.387652in}}{\pgfqpoint{4.988326in}{0.379839in}}%
\pgfpathcurveto{\pgfqpoint{4.996140in}{0.372025in}}{\pgfqpoint{5.006739in}{0.367635in}}{\pgfqpoint{5.017789in}{0.367635in}}%
\pgfusepath{stroke}%
\end{pgfscope}%
\begin{pgfscope}%
\pgfpathrectangle{\pgfqpoint{0.393053in}{0.375000in}}{\pgfqpoint{6.356833in}{5.175000in}}%
\pgfusepath{clip}%
\pgfsetbuttcap%
\pgfsetroundjoin%
\pgfsetlinewidth{1.003750pt}%
\definecolor{currentstroke}{rgb}{0.827451,0.827451,0.827451}%
\pgfsetstrokecolor{currentstroke}%
\pgfsetdash{}{0pt}%
\pgfpathmoveto{\pgfqpoint{3.464896in}{0.638283in}}%
\pgfpathcurveto{\pgfqpoint{3.475946in}{0.638283in}}{\pgfqpoint{3.486545in}{0.642673in}}{\pgfqpoint{3.494359in}{0.650486in}}%
\pgfpathcurveto{\pgfqpoint{3.502173in}{0.658300in}}{\pgfqpoint{3.506563in}{0.668899in}}{\pgfqpoint{3.506563in}{0.679949in}}%
\pgfpathcurveto{\pgfqpoint{3.506563in}{0.690999in}}{\pgfqpoint{3.502173in}{0.701598in}}{\pgfqpoint{3.494359in}{0.709412in}}%
\pgfpathcurveto{\pgfqpoint{3.486545in}{0.717226in}}{\pgfqpoint{3.475946in}{0.721616in}}{\pgfqpoint{3.464896in}{0.721616in}}%
\pgfpathcurveto{\pgfqpoint{3.453846in}{0.721616in}}{\pgfqpoint{3.443247in}{0.717226in}}{\pgfqpoint{3.435433in}{0.709412in}}%
\pgfpathcurveto{\pgfqpoint{3.427620in}{0.701598in}}{\pgfqpoint{3.423230in}{0.690999in}}{\pgfqpoint{3.423230in}{0.679949in}}%
\pgfpathcurveto{\pgfqpoint{3.423230in}{0.668899in}}{\pgfqpoint{3.427620in}{0.658300in}}{\pgfqpoint{3.435433in}{0.650486in}}%
\pgfpathcurveto{\pgfqpoint{3.443247in}{0.642673in}}{\pgfqpoint{3.453846in}{0.638283in}}{\pgfqpoint{3.464896in}{0.638283in}}%
\pgfpathlineto{\pgfqpoint{3.464896in}{0.638283in}}%
\pgfpathclose%
\pgfusepath{stroke}%
\end{pgfscope}%
\begin{pgfscope}%
\pgfpathrectangle{\pgfqpoint{0.393053in}{0.375000in}}{\pgfqpoint{6.356833in}{5.175000in}}%
\pgfusepath{clip}%
\pgfsetbuttcap%
\pgfsetroundjoin%
\pgfsetlinewidth{1.003750pt}%
\definecolor{currentstroke}{rgb}{0.827451,0.827451,0.827451}%
\pgfsetstrokecolor{currentstroke}%
\pgfsetdash{}{0pt}%
\pgfpathmoveto{\pgfqpoint{0.424181in}{4.027299in}}%
\pgfpathcurveto{\pgfqpoint{0.435231in}{4.027299in}}{\pgfqpoint{0.445830in}{4.031689in}}{\pgfqpoint{0.453644in}{4.039503in}}%
\pgfpathcurveto{\pgfqpoint{0.461458in}{4.047316in}}{\pgfqpoint{0.465848in}{4.057915in}}{\pgfqpoint{0.465848in}{4.068966in}}%
\pgfpathcurveto{\pgfqpoint{0.465848in}{4.080016in}}{\pgfqpoint{0.461458in}{4.090615in}}{\pgfqpoint{0.453644in}{4.098428in}}%
\pgfpathcurveto{\pgfqpoint{0.445830in}{4.106242in}}{\pgfqpoint{0.435231in}{4.110632in}}{\pgfqpoint{0.424181in}{4.110632in}}%
\pgfpathcurveto{\pgfqpoint{0.413131in}{4.110632in}}{\pgfqpoint{0.402532in}{4.106242in}}{\pgfqpoint{0.394719in}{4.098428in}}%
\pgfpathcurveto{\pgfqpoint{0.386905in}{4.090615in}}{\pgfqpoint{0.382515in}{4.080016in}}{\pgfqpoint{0.382515in}{4.068966in}}%
\pgfpathcurveto{\pgfqpoint{0.382515in}{4.057915in}}{\pgfqpoint{0.386905in}{4.047316in}}{\pgfqpoint{0.394719in}{4.039503in}}%
\pgfpathcurveto{\pgfqpoint{0.402532in}{4.031689in}}{\pgfqpoint{0.413131in}{4.027299in}}{\pgfqpoint{0.424181in}{4.027299in}}%
\pgfpathlineto{\pgfqpoint{0.424181in}{4.027299in}}%
\pgfpathclose%
\pgfusepath{stroke}%
\end{pgfscope}%
\begin{pgfscope}%
\pgfpathrectangle{\pgfqpoint{0.393053in}{0.375000in}}{\pgfqpoint{6.356833in}{5.175000in}}%
\pgfusepath{clip}%
\pgfsetbuttcap%
\pgfsetroundjoin%
\pgfsetlinewidth{1.003750pt}%
\definecolor{currentstroke}{rgb}{0.827451,0.827451,0.827451}%
\pgfsetstrokecolor{currentstroke}%
\pgfsetdash{}{0pt}%
\pgfpathmoveto{\pgfqpoint{2.785690in}{0.904937in}}%
\pgfpathcurveto{\pgfqpoint{2.796740in}{0.904937in}}{\pgfqpoint{2.807339in}{0.909327in}}{\pgfqpoint{2.815153in}{0.917140in}}%
\pgfpathcurveto{\pgfqpoint{2.822966in}{0.924954in}}{\pgfqpoint{2.827357in}{0.935553in}}{\pgfqpoint{2.827357in}{0.946603in}}%
\pgfpathcurveto{\pgfqpoint{2.827357in}{0.957653in}}{\pgfqpoint{2.822966in}{0.968252in}}{\pgfqpoint{2.815153in}{0.976066in}}%
\pgfpathcurveto{\pgfqpoint{2.807339in}{0.983880in}}{\pgfqpoint{2.796740in}{0.988270in}}{\pgfqpoint{2.785690in}{0.988270in}}%
\pgfpathcurveto{\pgfqpoint{2.774640in}{0.988270in}}{\pgfqpoint{2.764041in}{0.983880in}}{\pgfqpoint{2.756227in}{0.976066in}}%
\pgfpathcurveto{\pgfqpoint{2.748414in}{0.968252in}}{\pgfqpoint{2.744023in}{0.957653in}}{\pgfqpoint{2.744023in}{0.946603in}}%
\pgfpathcurveto{\pgfqpoint{2.744023in}{0.935553in}}{\pgfqpoint{2.748414in}{0.924954in}}{\pgfqpoint{2.756227in}{0.917140in}}%
\pgfpathcurveto{\pgfqpoint{2.764041in}{0.909327in}}{\pgfqpoint{2.774640in}{0.904937in}}{\pgfqpoint{2.785690in}{0.904937in}}%
\pgfpathlineto{\pgfqpoint{2.785690in}{0.904937in}}%
\pgfpathclose%
\pgfusepath{stroke}%
\end{pgfscope}%
\begin{pgfscope}%
\pgfpathrectangle{\pgfqpoint{0.393053in}{0.375000in}}{\pgfqpoint{6.356833in}{5.175000in}}%
\pgfusepath{clip}%
\pgfsetbuttcap%
\pgfsetroundjoin%
\pgfsetlinewidth{1.003750pt}%
\definecolor{currentstroke}{rgb}{0.827451,0.827451,0.827451}%
\pgfsetstrokecolor{currentstroke}%
\pgfsetdash{}{0pt}%
\pgfpathmoveto{\pgfqpoint{4.012103in}{0.498786in}}%
\pgfpathcurveto{\pgfqpoint{4.023153in}{0.498786in}}{\pgfqpoint{4.033752in}{0.503177in}}{\pgfqpoint{4.041566in}{0.510990in}}%
\pgfpathcurveto{\pgfqpoint{4.049379in}{0.518804in}}{\pgfqpoint{4.053769in}{0.529403in}}{\pgfqpoint{4.053769in}{0.540453in}}%
\pgfpathcurveto{\pgfqpoint{4.053769in}{0.551503in}}{\pgfqpoint{4.049379in}{0.562102in}}{\pgfqpoint{4.041566in}{0.569916in}}%
\pgfpathcurveto{\pgfqpoint{4.033752in}{0.577729in}}{\pgfqpoint{4.023153in}{0.582120in}}{\pgfqpoint{4.012103in}{0.582120in}}%
\pgfpathcurveto{\pgfqpoint{4.001053in}{0.582120in}}{\pgfqpoint{3.990454in}{0.577729in}}{\pgfqpoint{3.982640in}{0.569916in}}%
\pgfpathcurveto{\pgfqpoint{3.974826in}{0.562102in}}{\pgfqpoint{3.970436in}{0.551503in}}{\pgfqpoint{3.970436in}{0.540453in}}%
\pgfpathcurveto{\pgfqpoint{3.970436in}{0.529403in}}{\pgfqpoint{3.974826in}{0.518804in}}{\pgfqpoint{3.982640in}{0.510990in}}%
\pgfpathcurveto{\pgfqpoint{3.990454in}{0.503177in}}{\pgfqpoint{4.001053in}{0.498786in}}{\pgfqpoint{4.012103in}{0.498786in}}%
\pgfpathlineto{\pgfqpoint{4.012103in}{0.498786in}}%
\pgfpathclose%
\pgfusepath{stroke}%
\end{pgfscope}%
\begin{pgfscope}%
\pgfpathrectangle{\pgfqpoint{0.393053in}{0.375000in}}{\pgfqpoint{6.356833in}{5.175000in}}%
\pgfusepath{clip}%
\pgfsetbuttcap%
\pgfsetroundjoin%
\pgfsetlinewidth{1.003750pt}%
\definecolor{currentstroke}{rgb}{0.827451,0.827451,0.827451}%
\pgfsetstrokecolor{currentstroke}%
\pgfsetdash{}{0pt}%
\pgfpathmoveto{\pgfqpoint{3.637654in}{0.581771in}}%
\pgfpathcurveto{\pgfqpoint{3.648705in}{0.581771in}}{\pgfqpoint{3.659304in}{0.586161in}}{\pgfqpoint{3.667117in}{0.593974in}}%
\pgfpathcurveto{\pgfqpoint{3.674931in}{0.601788in}}{\pgfqpoint{3.679321in}{0.612387in}}{\pgfqpoint{3.679321in}{0.623437in}}%
\pgfpathcurveto{\pgfqpoint{3.679321in}{0.634487in}}{\pgfqpoint{3.674931in}{0.645086in}}{\pgfqpoint{3.667117in}{0.652900in}}%
\pgfpathcurveto{\pgfqpoint{3.659304in}{0.660714in}}{\pgfqpoint{3.648705in}{0.665104in}}{\pgfqpoint{3.637654in}{0.665104in}}%
\pgfpathcurveto{\pgfqpoint{3.626604in}{0.665104in}}{\pgfqpoint{3.616005in}{0.660714in}}{\pgfqpoint{3.608192in}{0.652900in}}%
\pgfpathcurveto{\pgfqpoint{3.600378in}{0.645086in}}{\pgfqpoint{3.595988in}{0.634487in}}{\pgfqpoint{3.595988in}{0.623437in}}%
\pgfpathcurveto{\pgfqpoint{3.595988in}{0.612387in}}{\pgfqpoint{3.600378in}{0.601788in}}{\pgfqpoint{3.608192in}{0.593974in}}%
\pgfpathcurveto{\pgfqpoint{3.616005in}{0.586161in}}{\pgfqpoint{3.626604in}{0.581771in}}{\pgfqpoint{3.637654in}{0.581771in}}%
\pgfpathlineto{\pgfqpoint{3.637654in}{0.581771in}}%
\pgfpathclose%
\pgfusepath{stroke}%
\end{pgfscope}%
\begin{pgfscope}%
\pgfpathrectangle{\pgfqpoint{0.393053in}{0.375000in}}{\pgfqpoint{6.356833in}{5.175000in}}%
\pgfusepath{clip}%
\pgfsetbuttcap%
\pgfsetroundjoin%
\pgfsetlinewidth{1.003750pt}%
\definecolor{currentstroke}{rgb}{0.827451,0.827451,0.827451}%
\pgfsetstrokecolor{currentstroke}%
\pgfsetdash{}{0pt}%
\pgfpathmoveto{\pgfqpoint{1.019447in}{2.455876in}}%
\pgfpathcurveto{\pgfqpoint{1.030497in}{2.455876in}}{\pgfqpoint{1.041096in}{2.460267in}}{\pgfqpoint{1.048910in}{2.468080in}}%
\pgfpathcurveto{\pgfqpoint{1.056723in}{2.475894in}}{\pgfqpoint{1.061113in}{2.486493in}}{\pgfqpoint{1.061113in}{2.497543in}}%
\pgfpathcurveto{\pgfqpoint{1.061113in}{2.508593in}}{\pgfqpoint{1.056723in}{2.519192in}}{\pgfqpoint{1.048910in}{2.527006in}}%
\pgfpathcurveto{\pgfqpoint{1.041096in}{2.534819in}}{\pgfqpoint{1.030497in}{2.539210in}}{\pgfqpoint{1.019447in}{2.539210in}}%
\pgfpathcurveto{\pgfqpoint{1.008397in}{2.539210in}}{\pgfqpoint{0.997798in}{2.534819in}}{\pgfqpoint{0.989984in}{2.527006in}}%
\pgfpathcurveto{\pgfqpoint{0.982170in}{2.519192in}}{\pgfqpoint{0.977780in}{2.508593in}}{\pgfqpoint{0.977780in}{2.497543in}}%
\pgfpathcurveto{\pgfqpoint{0.977780in}{2.486493in}}{\pgfqpoint{0.982170in}{2.475894in}}{\pgfqpoint{0.989984in}{2.468080in}}%
\pgfpathcurveto{\pgfqpoint{0.997798in}{2.460267in}}{\pgfqpoint{1.008397in}{2.455876in}}{\pgfqpoint{1.019447in}{2.455876in}}%
\pgfpathlineto{\pgfqpoint{1.019447in}{2.455876in}}%
\pgfpathclose%
\pgfusepath{stroke}%
\end{pgfscope}%
\begin{pgfscope}%
\pgfpathrectangle{\pgfqpoint{0.393053in}{0.375000in}}{\pgfqpoint{6.356833in}{5.175000in}}%
\pgfusepath{clip}%
\pgfsetbuttcap%
\pgfsetroundjoin%
\pgfsetlinewidth{1.003750pt}%
\definecolor{currentstroke}{rgb}{0.827451,0.827451,0.827451}%
\pgfsetstrokecolor{currentstroke}%
\pgfsetdash{}{0pt}%
\pgfpathmoveto{\pgfqpoint{0.412548in}{4.185071in}}%
\pgfpathcurveto{\pgfqpoint{0.423598in}{4.185071in}}{\pgfqpoint{0.434197in}{4.189462in}}{\pgfqpoint{0.442011in}{4.197275in}}%
\pgfpathcurveto{\pgfqpoint{0.449824in}{4.205089in}}{\pgfqpoint{0.454215in}{4.215688in}}{\pgfqpoint{0.454215in}{4.226738in}}%
\pgfpathcurveto{\pgfqpoint{0.454215in}{4.237788in}}{\pgfqpoint{0.449824in}{4.248387in}}{\pgfqpoint{0.442011in}{4.256201in}}%
\pgfpathcurveto{\pgfqpoint{0.434197in}{4.264015in}}{\pgfqpoint{0.423598in}{4.268405in}}{\pgfqpoint{0.412548in}{4.268405in}}%
\pgfpathcurveto{\pgfqpoint{0.401498in}{4.268405in}}{\pgfqpoint{0.390899in}{4.264015in}}{\pgfqpoint{0.383085in}{4.256201in}}%
\pgfpathcurveto{\pgfqpoint{0.375271in}{4.248387in}}{\pgfqpoint{0.370881in}{4.237788in}}{\pgfqpoint{0.370881in}{4.226738in}}%
\pgfpathcurveto{\pgfqpoint{0.370881in}{4.215688in}}{\pgfqpoint{0.375271in}{4.205089in}}{\pgfqpoint{0.383085in}{4.197275in}}%
\pgfpathcurveto{\pgfqpoint{0.390899in}{4.189462in}}{\pgfqpoint{0.401498in}{4.185071in}}{\pgfqpoint{0.412548in}{4.185071in}}%
\pgfpathlineto{\pgfqpoint{0.412548in}{4.185071in}}%
\pgfpathclose%
\pgfusepath{stroke}%
\end{pgfscope}%
\begin{pgfscope}%
\pgfpathrectangle{\pgfqpoint{0.393053in}{0.375000in}}{\pgfqpoint{6.356833in}{5.175000in}}%
\pgfusepath{clip}%
\pgfsetbuttcap%
\pgfsetroundjoin%
\pgfsetlinewidth{1.003750pt}%
\definecolor{currentstroke}{rgb}{0.827451,0.827451,0.827451}%
\pgfsetstrokecolor{currentstroke}%
\pgfsetdash{}{0pt}%
\pgfpathmoveto{\pgfqpoint{4.310929in}{0.438129in}}%
\pgfpathcurveto{\pgfqpoint{4.321979in}{0.438129in}}{\pgfqpoint{4.332578in}{0.442519in}}{\pgfqpoint{4.340392in}{0.450333in}}%
\pgfpathcurveto{\pgfqpoint{4.348205in}{0.458147in}}{\pgfqpoint{4.352595in}{0.468746in}}{\pgfqpoint{4.352595in}{0.479796in}}%
\pgfpathcurveto{\pgfqpoint{4.352595in}{0.490846in}}{\pgfqpoint{4.348205in}{0.501445in}}{\pgfqpoint{4.340392in}{0.509258in}}%
\pgfpathcurveto{\pgfqpoint{4.332578in}{0.517072in}}{\pgfqpoint{4.321979in}{0.521462in}}{\pgfqpoint{4.310929in}{0.521462in}}%
\pgfpathcurveto{\pgfqpoint{4.299879in}{0.521462in}}{\pgfqpoint{4.289280in}{0.517072in}}{\pgfqpoint{4.281466in}{0.509258in}}%
\pgfpathcurveto{\pgfqpoint{4.273652in}{0.501445in}}{\pgfqpoint{4.269262in}{0.490846in}}{\pgfqpoint{4.269262in}{0.479796in}}%
\pgfpathcurveto{\pgfqpoint{4.269262in}{0.468746in}}{\pgfqpoint{4.273652in}{0.458147in}}{\pgfqpoint{4.281466in}{0.450333in}}%
\pgfpathcurveto{\pgfqpoint{4.289280in}{0.442519in}}{\pgfqpoint{4.299879in}{0.438129in}}{\pgfqpoint{4.310929in}{0.438129in}}%
\pgfpathlineto{\pgfqpoint{4.310929in}{0.438129in}}%
\pgfpathclose%
\pgfusepath{stroke}%
\end{pgfscope}%
\begin{pgfscope}%
\pgfpathrectangle{\pgfqpoint{0.393053in}{0.375000in}}{\pgfqpoint{6.356833in}{5.175000in}}%
\pgfusepath{clip}%
\pgfsetbuttcap%
\pgfsetroundjoin%
\pgfsetlinewidth{1.003750pt}%
\definecolor{currentstroke}{rgb}{0.827451,0.827451,0.827451}%
\pgfsetstrokecolor{currentstroke}%
\pgfsetdash{}{0pt}%
\pgfpathmoveto{\pgfqpoint{1.768437in}{1.558678in}}%
\pgfpathcurveto{\pgfqpoint{1.779488in}{1.558678in}}{\pgfqpoint{1.790087in}{1.563068in}}{\pgfqpoint{1.797900in}{1.570882in}}%
\pgfpathcurveto{\pgfqpoint{1.805714in}{1.578696in}}{\pgfqpoint{1.810104in}{1.589295in}}{\pgfqpoint{1.810104in}{1.600345in}}%
\pgfpathcurveto{\pgfqpoint{1.810104in}{1.611395in}}{\pgfqpoint{1.805714in}{1.621994in}}{\pgfqpoint{1.797900in}{1.629807in}}%
\pgfpathcurveto{\pgfqpoint{1.790087in}{1.637621in}}{\pgfqpoint{1.779488in}{1.642011in}}{\pgfqpoint{1.768437in}{1.642011in}}%
\pgfpathcurveto{\pgfqpoint{1.757387in}{1.642011in}}{\pgfqpoint{1.746788in}{1.637621in}}{\pgfqpoint{1.738975in}{1.629807in}}%
\pgfpathcurveto{\pgfqpoint{1.731161in}{1.621994in}}{\pgfqpoint{1.726771in}{1.611395in}}{\pgfqpoint{1.726771in}{1.600345in}}%
\pgfpathcurveto{\pgfqpoint{1.726771in}{1.589295in}}{\pgfqpoint{1.731161in}{1.578696in}}{\pgfqpoint{1.738975in}{1.570882in}}%
\pgfpathcurveto{\pgfqpoint{1.746788in}{1.563068in}}{\pgfqpoint{1.757387in}{1.558678in}}{\pgfqpoint{1.768437in}{1.558678in}}%
\pgfpathlineto{\pgfqpoint{1.768437in}{1.558678in}}%
\pgfpathclose%
\pgfusepath{stroke}%
\end{pgfscope}%
\begin{pgfscope}%
\pgfpathrectangle{\pgfqpoint{0.393053in}{0.375000in}}{\pgfqpoint{6.356833in}{5.175000in}}%
\pgfusepath{clip}%
\pgfsetbuttcap%
\pgfsetroundjoin%
\pgfsetlinewidth{1.003750pt}%
\definecolor{currentstroke}{rgb}{0.827451,0.827451,0.827451}%
\pgfsetstrokecolor{currentstroke}%
\pgfsetdash{}{0pt}%
\pgfpathmoveto{\pgfqpoint{0.435142in}{3.936498in}}%
\pgfpathcurveto{\pgfqpoint{0.446192in}{3.936498in}}{\pgfqpoint{0.456791in}{3.940888in}}{\pgfqpoint{0.464605in}{3.948702in}}%
\pgfpathcurveto{\pgfqpoint{0.472419in}{3.956516in}}{\pgfqpoint{0.476809in}{3.967115in}}{\pgfqpoint{0.476809in}{3.978165in}}%
\pgfpathcurveto{\pgfqpoint{0.476809in}{3.989215in}}{\pgfqpoint{0.472419in}{3.999814in}}{\pgfqpoint{0.464605in}{4.007627in}}%
\pgfpathcurveto{\pgfqpoint{0.456791in}{4.015441in}}{\pgfqpoint{0.446192in}{4.019831in}}{\pgfqpoint{0.435142in}{4.019831in}}%
\pgfpathcurveto{\pgfqpoint{0.424092in}{4.019831in}}{\pgfqpoint{0.413493in}{4.015441in}}{\pgfqpoint{0.405679in}{4.007627in}}%
\pgfpathcurveto{\pgfqpoint{0.397866in}{3.999814in}}{\pgfqpoint{0.393475in}{3.989215in}}{\pgfqpoint{0.393475in}{3.978165in}}%
\pgfpathcurveto{\pgfqpoint{0.393475in}{3.967115in}}{\pgfqpoint{0.397866in}{3.956516in}}{\pgfqpoint{0.405679in}{3.948702in}}%
\pgfpathcurveto{\pgfqpoint{0.413493in}{3.940888in}}{\pgfqpoint{0.424092in}{3.936498in}}{\pgfqpoint{0.435142in}{3.936498in}}%
\pgfpathlineto{\pgfqpoint{0.435142in}{3.936498in}}%
\pgfpathclose%
\pgfusepath{stroke}%
\end{pgfscope}%
\begin{pgfscope}%
\pgfpathrectangle{\pgfqpoint{0.393053in}{0.375000in}}{\pgfqpoint{6.356833in}{5.175000in}}%
\pgfusepath{clip}%
\pgfsetbuttcap%
\pgfsetroundjoin%
\pgfsetlinewidth{1.003750pt}%
\definecolor{currentstroke}{rgb}{0.827451,0.827451,0.827451}%
\pgfsetstrokecolor{currentstroke}%
\pgfsetdash{}{0pt}%
\pgfpathmoveto{\pgfqpoint{0.744021in}{2.849644in}}%
\pgfpathcurveto{\pgfqpoint{0.755071in}{2.849644in}}{\pgfqpoint{0.765670in}{2.854034in}}{\pgfqpoint{0.773484in}{2.861848in}}%
\pgfpathcurveto{\pgfqpoint{0.781298in}{2.869661in}}{\pgfqpoint{0.785688in}{2.880260in}}{\pgfqpoint{0.785688in}{2.891311in}}%
\pgfpathcurveto{\pgfqpoint{0.785688in}{2.902361in}}{\pgfqpoint{0.781298in}{2.912960in}}{\pgfqpoint{0.773484in}{2.920773in}}%
\pgfpathcurveto{\pgfqpoint{0.765670in}{2.928587in}}{\pgfqpoint{0.755071in}{2.932977in}}{\pgfqpoint{0.744021in}{2.932977in}}%
\pgfpathcurveto{\pgfqpoint{0.732971in}{2.932977in}}{\pgfqpoint{0.722372in}{2.928587in}}{\pgfqpoint{0.714558in}{2.920773in}}%
\pgfpathcurveto{\pgfqpoint{0.706745in}{2.912960in}}{\pgfqpoint{0.702355in}{2.902361in}}{\pgfqpoint{0.702355in}{2.891311in}}%
\pgfpathcurveto{\pgfqpoint{0.702355in}{2.880260in}}{\pgfqpoint{0.706745in}{2.869661in}}{\pgfqpoint{0.714558in}{2.861848in}}%
\pgfpathcurveto{\pgfqpoint{0.722372in}{2.854034in}}{\pgfqpoint{0.732971in}{2.849644in}}{\pgfqpoint{0.744021in}{2.849644in}}%
\pgfpathlineto{\pgfqpoint{0.744021in}{2.849644in}}%
\pgfpathclose%
\pgfusepath{stroke}%
\end{pgfscope}%
\begin{pgfscope}%
\pgfpathrectangle{\pgfqpoint{0.393053in}{0.375000in}}{\pgfqpoint{6.356833in}{5.175000in}}%
\pgfusepath{clip}%
\pgfsetbuttcap%
\pgfsetroundjoin%
\pgfsetlinewidth{1.003750pt}%
\definecolor{currentstroke}{rgb}{0.827451,0.827451,0.827451}%
\pgfsetstrokecolor{currentstroke}%
\pgfsetdash{}{0pt}%
\pgfpathmoveto{\pgfqpoint{1.843211in}{1.515821in}}%
\pgfpathcurveto{\pgfqpoint{1.854261in}{1.515821in}}{\pgfqpoint{1.864860in}{1.520211in}}{\pgfqpoint{1.872674in}{1.528025in}}%
\pgfpathcurveto{\pgfqpoint{1.880487in}{1.535838in}}{\pgfqpoint{1.884878in}{1.546437in}}{\pgfqpoint{1.884878in}{1.557487in}}%
\pgfpathcurveto{\pgfqpoint{1.884878in}{1.568537in}}{\pgfqpoint{1.880487in}{1.579136in}}{\pgfqpoint{1.872674in}{1.586950in}}%
\pgfpathcurveto{\pgfqpoint{1.864860in}{1.594764in}}{\pgfqpoint{1.854261in}{1.599154in}}{\pgfqpoint{1.843211in}{1.599154in}}%
\pgfpathcurveto{\pgfqpoint{1.832161in}{1.599154in}}{\pgfqpoint{1.821562in}{1.594764in}}{\pgfqpoint{1.813748in}{1.586950in}}%
\pgfpathcurveto{\pgfqpoint{1.805934in}{1.579136in}}{\pgfqpoint{1.801544in}{1.568537in}}{\pgfqpoint{1.801544in}{1.557487in}}%
\pgfpathcurveto{\pgfqpoint{1.801544in}{1.546437in}}{\pgfqpoint{1.805934in}{1.535838in}}{\pgfqpoint{1.813748in}{1.528025in}}%
\pgfpathcurveto{\pgfqpoint{1.821562in}{1.520211in}}{\pgfqpoint{1.832161in}{1.515821in}}{\pgfqpoint{1.843211in}{1.515821in}}%
\pgfpathlineto{\pgfqpoint{1.843211in}{1.515821in}}%
\pgfpathclose%
\pgfusepath{stroke}%
\end{pgfscope}%
\begin{pgfscope}%
\pgfpathrectangle{\pgfqpoint{0.393053in}{0.375000in}}{\pgfqpoint{6.356833in}{5.175000in}}%
\pgfusepath{clip}%
\pgfsetbuttcap%
\pgfsetroundjoin%
\pgfsetlinewidth{1.003750pt}%
\definecolor{currentstroke}{rgb}{0.827451,0.827451,0.827451}%
\pgfsetstrokecolor{currentstroke}%
\pgfsetdash{}{0pt}%
\pgfpathmoveto{\pgfqpoint{3.230235in}{0.725803in}}%
\pgfpathcurveto{\pgfqpoint{3.241285in}{0.725803in}}{\pgfqpoint{3.251884in}{0.730193in}}{\pgfqpoint{3.259697in}{0.738007in}}%
\pgfpathcurveto{\pgfqpoint{3.267511in}{0.745821in}}{\pgfqpoint{3.271901in}{0.756420in}}{\pgfqpoint{3.271901in}{0.767470in}}%
\pgfpathcurveto{\pgfqpoint{3.271901in}{0.778520in}}{\pgfqpoint{3.267511in}{0.789119in}}{\pgfqpoint{3.259697in}{0.796933in}}%
\pgfpathcurveto{\pgfqpoint{3.251884in}{0.804746in}}{\pgfqpoint{3.241285in}{0.809137in}}{\pgfqpoint{3.230235in}{0.809137in}}%
\pgfpathcurveto{\pgfqpoint{3.219184in}{0.809137in}}{\pgfqpoint{3.208585in}{0.804746in}}{\pgfqpoint{3.200772in}{0.796933in}}%
\pgfpathcurveto{\pgfqpoint{3.192958in}{0.789119in}}{\pgfqpoint{3.188568in}{0.778520in}}{\pgfqpoint{3.188568in}{0.767470in}}%
\pgfpathcurveto{\pgfqpoint{3.188568in}{0.756420in}}{\pgfqpoint{3.192958in}{0.745821in}}{\pgfqpoint{3.200772in}{0.738007in}}%
\pgfpathcurveto{\pgfqpoint{3.208585in}{0.730193in}}{\pgfqpoint{3.219184in}{0.725803in}}{\pgfqpoint{3.230235in}{0.725803in}}%
\pgfpathlineto{\pgfqpoint{3.230235in}{0.725803in}}%
\pgfpathclose%
\pgfusepath{stroke}%
\end{pgfscope}%
\begin{pgfscope}%
\pgfpathrectangle{\pgfqpoint{0.393053in}{0.375000in}}{\pgfqpoint{6.356833in}{5.175000in}}%
\pgfusepath{clip}%
\pgfsetbuttcap%
\pgfsetroundjoin%
\pgfsetlinewidth{1.003750pt}%
\definecolor{currentstroke}{rgb}{0.827451,0.827451,0.827451}%
\pgfsetstrokecolor{currentstroke}%
\pgfsetdash{}{0pt}%
\pgfpathmoveto{\pgfqpoint{3.055426in}{0.786859in}}%
\pgfpathcurveto{\pgfqpoint{3.066476in}{0.786859in}}{\pgfqpoint{3.077075in}{0.791249in}}{\pgfqpoint{3.084888in}{0.799063in}}%
\pgfpathcurveto{\pgfqpoint{3.092702in}{0.806876in}}{\pgfqpoint{3.097092in}{0.817476in}}{\pgfqpoint{3.097092in}{0.828526in}}%
\pgfpathcurveto{\pgfqpoint{3.097092in}{0.839576in}}{\pgfqpoint{3.092702in}{0.850175in}}{\pgfqpoint{3.084888in}{0.857988in}}%
\pgfpathcurveto{\pgfqpoint{3.077075in}{0.865802in}}{\pgfqpoint{3.066476in}{0.870192in}}{\pgfqpoint{3.055426in}{0.870192in}}%
\pgfpathcurveto{\pgfqpoint{3.044376in}{0.870192in}}{\pgfqpoint{3.033777in}{0.865802in}}{\pgfqpoint{3.025963in}{0.857988in}}%
\pgfpathcurveto{\pgfqpoint{3.018149in}{0.850175in}}{\pgfqpoint{3.013759in}{0.839576in}}{\pgfqpoint{3.013759in}{0.828526in}}%
\pgfpathcurveto{\pgfqpoint{3.013759in}{0.817476in}}{\pgfqpoint{3.018149in}{0.806876in}}{\pgfqpoint{3.025963in}{0.799063in}}%
\pgfpathcurveto{\pgfqpoint{3.033777in}{0.791249in}}{\pgfqpoint{3.044376in}{0.786859in}}{\pgfqpoint{3.055426in}{0.786859in}}%
\pgfpathlineto{\pgfqpoint{3.055426in}{0.786859in}}%
\pgfpathclose%
\pgfusepath{stroke}%
\end{pgfscope}%
\begin{pgfscope}%
\pgfpathrectangle{\pgfqpoint{0.393053in}{0.375000in}}{\pgfqpoint{6.356833in}{5.175000in}}%
\pgfusepath{clip}%
\pgfsetbuttcap%
\pgfsetroundjoin%
\pgfsetlinewidth{1.003750pt}%
\definecolor{currentstroke}{rgb}{0.827451,0.827451,0.827451}%
\pgfsetstrokecolor{currentstroke}%
\pgfsetdash{}{0pt}%
\pgfpathmoveto{\pgfqpoint{3.156562in}{0.749138in}}%
\pgfpathcurveto{\pgfqpoint{3.167612in}{0.749138in}}{\pgfqpoint{3.178211in}{0.753528in}}{\pgfqpoint{3.186024in}{0.761342in}}%
\pgfpathcurveto{\pgfqpoint{3.193838in}{0.769155in}}{\pgfqpoint{3.198228in}{0.779754in}}{\pgfqpoint{3.198228in}{0.790804in}}%
\pgfpathcurveto{\pgfqpoint{3.198228in}{0.801855in}}{\pgfqpoint{3.193838in}{0.812454in}}{\pgfqpoint{3.186024in}{0.820267in}}%
\pgfpathcurveto{\pgfqpoint{3.178211in}{0.828081in}}{\pgfqpoint{3.167612in}{0.832471in}}{\pgfqpoint{3.156562in}{0.832471in}}%
\pgfpathcurveto{\pgfqpoint{3.145512in}{0.832471in}}{\pgfqpoint{3.134913in}{0.828081in}}{\pgfqpoint{3.127099in}{0.820267in}}%
\pgfpathcurveto{\pgfqpoint{3.119285in}{0.812454in}}{\pgfqpoint{3.114895in}{0.801855in}}{\pgfqpoint{3.114895in}{0.790804in}}%
\pgfpathcurveto{\pgfqpoint{3.114895in}{0.779754in}}{\pgfqpoint{3.119285in}{0.769155in}}{\pgfqpoint{3.127099in}{0.761342in}}%
\pgfpathcurveto{\pgfqpoint{3.134913in}{0.753528in}}{\pgfqpoint{3.145512in}{0.749138in}}{\pgfqpoint{3.156562in}{0.749138in}}%
\pgfpathlineto{\pgfqpoint{3.156562in}{0.749138in}}%
\pgfpathclose%
\pgfusepath{stroke}%
\end{pgfscope}%
\begin{pgfscope}%
\pgfpathrectangle{\pgfqpoint{0.393053in}{0.375000in}}{\pgfqpoint{6.356833in}{5.175000in}}%
\pgfusepath{clip}%
\pgfsetbuttcap%
\pgfsetroundjoin%
\pgfsetlinewidth{1.003750pt}%
\definecolor{currentstroke}{rgb}{0.827451,0.827451,0.827451}%
\pgfsetstrokecolor{currentstroke}%
\pgfsetdash{}{0pt}%
\pgfpathmoveto{\pgfqpoint{0.848540in}{2.695419in}}%
\pgfpathcurveto{\pgfqpoint{0.859590in}{2.695419in}}{\pgfqpoint{0.870189in}{2.699809in}}{\pgfqpoint{0.878002in}{2.707623in}}%
\pgfpathcurveto{\pgfqpoint{0.885816in}{2.715436in}}{\pgfqpoint{0.890206in}{2.726035in}}{\pgfqpoint{0.890206in}{2.737085in}}%
\pgfpathcurveto{\pgfqpoint{0.890206in}{2.748135in}}{\pgfqpoint{0.885816in}{2.758735in}}{\pgfqpoint{0.878002in}{2.766548in}}%
\pgfpathcurveto{\pgfqpoint{0.870189in}{2.774362in}}{\pgfqpoint{0.859590in}{2.778752in}}{\pgfqpoint{0.848540in}{2.778752in}}%
\pgfpathcurveto{\pgfqpoint{0.837489in}{2.778752in}}{\pgfqpoint{0.826890in}{2.774362in}}{\pgfqpoint{0.819077in}{2.766548in}}%
\pgfpathcurveto{\pgfqpoint{0.811263in}{2.758735in}}{\pgfqpoint{0.806873in}{2.748135in}}{\pgfqpoint{0.806873in}{2.737085in}}%
\pgfpathcurveto{\pgfqpoint{0.806873in}{2.726035in}}{\pgfqpoint{0.811263in}{2.715436in}}{\pgfqpoint{0.819077in}{2.707623in}}%
\pgfpathcurveto{\pgfqpoint{0.826890in}{2.699809in}}{\pgfqpoint{0.837489in}{2.695419in}}{\pgfqpoint{0.848540in}{2.695419in}}%
\pgfpathlineto{\pgfqpoint{0.848540in}{2.695419in}}%
\pgfpathclose%
\pgfusepath{stroke}%
\end{pgfscope}%
\begin{pgfscope}%
\pgfpathrectangle{\pgfqpoint{0.393053in}{0.375000in}}{\pgfqpoint{6.356833in}{5.175000in}}%
\pgfusepath{clip}%
\pgfsetbuttcap%
\pgfsetroundjoin%
\pgfsetlinewidth{1.003750pt}%
\definecolor{currentstroke}{rgb}{0.827451,0.827451,0.827451}%
\pgfsetstrokecolor{currentstroke}%
\pgfsetdash{}{0pt}%
\pgfpathmoveto{\pgfqpoint{0.961220in}{2.522545in}}%
\pgfpathcurveto{\pgfqpoint{0.972270in}{2.522545in}}{\pgfqpoint{0.982869in}{2.526936in}}{\pgfqpoint{0.990683in}{2.534749in}}%
\pgfpathcurveto{\pgfqpoint{0.998497in}{2.542563in}}{\pgfqpoint{1.002887in}{2.553162in}}{\pgfqpoint{1.002887in}{2.564212in}}%
\pgfpathcurveto{\pgfqpoint{1.002887in}{2.575262in}}{\pgfqpoint{0.998497in}{2.585861in}}{\pgfqpoint{0.990683in}{2.593675in}}%
\pgfpathcurveto{\pgfqpoint{0.982869in}{2.601488in}}{\pgfqpoint{0.972270in}{2.605879in}}{\pgfqpoint{0.961220in}{2.605879in}}%
\pgfpathcurveto{\pgfqpoint{0.950170in}{2.605879in}}{\pgfqpoint{0.939571in}{2.601488in}}{\pgfqpoint{0.931757in}{2.593675in}}%
\pgfpathcurveto{\pgfqpoint{0.923944in}{2.585861in}}{\pgfqpoint{0.919554in}{2.575262in}}{\pgfqpoint{0.919554in}{2.564212in}}%
\pgfpathcurveto{\pgfqpoint{0.919554in}{2.553162in}}{\pgfqpoint{0.923944in}{2.542563in}}{\pgfqpoint{0.931757in}{2.534749in}}%
\pgfpathcurveto{\pgfqpoint{0.939571in}{2.526936in}}{\pgfqpoint{0.950170in}{2.522545in}}{\pgfqpoint{0.961220in}{2.522545in}}%
\pgfpathlineto{\pgfqpoint{0.961220in}{2.522545in}}%
\pgfpathclose%
\pgfusepath{stroke}%
\end{pgfscope}%
\begin{pgfscope}%
\pgfpathrectangle{\pgfqpoint{0.393053in}{0.375000in}}{\pgfqpoint{6.356833in}{5.175000in}}%
\pgfusepath{clip}%
\pgfsetbuttcap%
\pgfsetroundjoin%
\pgfsetlinewidth{1.003750pt}%
\definecolor{currentstroke}{rgb}{0.827451,0.827451,0.827451}%
\pgfsetstrokecolor{currentstroke}%
\pgfsetdash{}{0pt}%
\pgfpathmoveto{\pgfqpoint{0.707419in}{2.963452in}}%
\pgfpathcurveto{\pgfqpoint{0.718469in}{2.963452in}}{\pgfqpoint{0.729068in}{2.967843in}}{\pgfqpoint{0.736882in}{2.975656in}}%
\pgfpathcurveto{\pgfqpoint{0.744695in}{2.983470in}}{\pgfqpoint{0.749086in}{2.994069in}}{\pgfqpoint{0.749086in}{3.005119in}}%
\pgfpathcurveto{\pgfqpoint{0.749086in}{3.016169in}}{\pgfqpoint{0.744695in}{3.026768in}}{\pgfqpoint{0.736882in}{3.034582in}}%
\pgfpathcurveto{\pgfqpoint{0.729068in}{3.042395in}}{\pgfqpoint{0.718469in}{3.046786in}}{\pgfqpoint{0.707419in}{3.046786in}}%
\pgfpathcurveto{\pgfqpoint{0.696369in}{3.046786in}}{\pgfqpoint{0.685770in}{3.042395in}}{\pgfqpoint{0.677956in}{3.034582in}}%
\pgfpathcurveto{\pgfqpoint{0.670142in}{3.026768in}}{\pgfqpoint{0.665752in}{3.016169in}}{\pgfqpoint{0.665752in}{3.005119in}}%
\pgfpathcurveto{\pgfqpoint{0.665752in}{2.994069in}}{\pgfqpoint{0.670142in}{2.983470in}}{\pgfqpoint{0.677956in}{2.975656in}}%
\pgfpathcurveto{\pgfqpoint{0.685770in}{2.967843in}}{\pgfqpoint{0.696369in}{2.963452in}}{\pgfqpoint{0.707419in}{2.963452in}}%
\pgfpathlineto{\pgfqpoint{0.707419in}{2.963452in}}%
\pgfpathclose%
\pgfusepath{stroke}%
\end{pgfscope}%
\begin{pgfscope}%
\pgfpathrectangle{\pgfqpoint{0.393053in}{0.375000in}}{\pgfqpoint{6.356833in}{5.175000in}}%
\pgfusepath{clip}%
\pgfsetbuttcap%
\pgfsetroundjoin%
\pgfsetlinewidth{1.003750pt}%
\definecolor{currentstroke}{rgb}{0.827451,0.827451,0.827451}%
\pgfsetstrokecolor{currentstroke}%
\pgfsetdash{}{0pt}%
\pgfpathmoveto{\pgfqpoint{0.396398in}{4.410130in}}%
\pgfpathcurveto{\pgfqpoint{0.407448in}{4.410130in}}{\pgfqpoint{0.418047in}{4.414521in}}{\pgfqpoint{0.425861in}{4.422334in}}%
\pgfpathcurveto{\pgfqpoint{0.433674in}{4.430148in}}{\pgfqpoint{0.438064in}{4.440747in}}{\pgfqpoint{0.438064in}{4.451797in}}%
\pgfpathcurveto{\pgfqpoint{0.438064in}{4.462847in}}{\pgfqpoint{0.433674in}{4.473446in}}{\pgfqpoint{0.425861in}{4.481260in}}%
\pgfpathcurveto{\pgfqpoint{0.418047in}{4.489073in}}{\pgfqpoint{0.407448in}{4.493464in}}{\pgfqpoint{0.396398in}{4.493464in}}%
\pgfpathcurveto{\pgfqpoint{0.385348in}{4.493464in}}{\pgfqpoint{0.374749in}{4.489073in}}{\pgfqpoint{0.366935in}{4.481260in}}%
\pgfpathcurveto{\pgfqpoint{0.359121in}{4.473446in}}{\pgfqpoint{0.354731in}{4.462847in}}{\pgfqpoint{0.354731in}{4.451797in}}%
\pgfpathcurveto{\pgfqpoint{0.354731in}{4.440747in}}{\pgfqpoint{0.359121in}{4.430148in}}{\pgfqpoint{0.366935in}{4.422334in}}%
\pgfpathcurveto{\pgfqpoint{0.374749in}{4.414521in}}{\pgfqpoint{0.385348in}{4.410130in}}{\pgfqpoint{0.396398in}{4.410130in}}%
\pgfpathlineto{\pgfqpoint{0.396398in}{4.410130in}}%
\pgfpathclose%
\pgfusepath{stroke}%
\end{pgfscope}%
\begin{pgfscope}%
\pgfpathrectangle{\pgfqpoint{0.393053in}{0.375000in}}{\pgfqpoint{6.356833in}{5.175000in}}%
\pgfusepath{clip}%
\pgfsetbuttcap%
\pgfsetroundjoin%
\pgfsetlinewidth{1.003750pt}%
\definecolor{currentstroke}{rgb}{0.827451,0.827451,0.827451}%
\pgfsetstrokecolor{currentstroke}%
\pgfsetdash{}{0pt}%
\pgfpathmoveto{\pgfqpoint{0.875375in}{2.588406in}}%
\pgfpathcurveto{\pgfqpoint{0.886425in}{2.588406in}}{\pgfqpoint{0.897024in}{2.592796in}}{\pgfqpoint{0.904838in}{2.600609in}}%
\pgfpathcurveto{\pgfqpoint{0.912652in}{2.608423in}}{\pgfqpoint{0.917042in}{2.619022in}}{\pgfqpoint{0.917042in}{2.630072in}}%
\pgfpathcurveto{\pgfqpoint{0.917042in}{2.641122in}}{\pgfqpoint{0.912652in}{2.651721in}}{\pgfqpoint{0.904838in}{2.659535in}}%
\pgfpathcurveto{\pgfqpoint{0.897024in}{2.667349in}}{\pgfqpoint{0.886425in}{2.671739in}}{\pgfqpoint{0.875375in}{2.671739in}}%
\pgfpathcurveto{\pgfqpoint{0.864325in}{2.671739in}}{\pgfqpoint{0.853726in}{2.667349in}}{\pgfqpoint{0.845912in}{2.659535in}}%
\pgfpathcurveto{\pgfqpoint{0.838099in}{2.651721in}}{\pgfqpoint{0.833708in}{2.641122in}}{\pgfqpoint{0.833708in}{2.630072in}}%
\pgfpathcurveto{\pgfqpoint{0.833708in}{2.619022in}}{\pgfqpoint{0.838099in}{2.608423in}}{\pgfqpoint{0.845912in}{2.600609in}}%
\pgfpathcurveto{\pgfqpoint{0.853726in}{2.592796in}}{\pgfqpoint{0.864325in}{2.588406in}}{\pgfqpoint{0.875375in}{2.588406in}}%
\pgfpathlineto{\pgfqpoint{0.875375in}{2.588406in}}%
\pgfpathclose%
\pgfusepath{stroke}%
\end{pgfscope}%
\begin{pgfscope}%
\pgfpathrectangle{\pgfqpoint{0.393053in}{0.375000in}}{\pgfqpoint{6.356833in}{5.175000in}}%
\pgfusepath{clip}%
\pgfsetbuttcap%
\pgfsetroundjoin%
\pgfsetlinewidth{1.003750pt}%
\definecolor{currentstroke}{rgb}{0.827451,0.827451,0.827451}%
\pgfsetstrokecolor{currentstroke}%
\pgfsetdash{}{0pt}%
\pgfpathmoveto{\pgfqpoint{4.537650in}{0.413917in}}%
\pgfpathcurveto{\pgfqpoint{4.548700in}{0.413917in}}{\pgfqpoint{4.559300in}{0.418307in}}{\pgfqpoint{4.567113in}{0.426121in}}%
\pgfpathcurveto{\pgfqpoint{4.574927in}{0.433935in}}{\pgfqpoint{4.579317in}{0.444534in}}{\pgfqpoint{4.579317in}{0.455584in}}%
\pgfpathcurveto{\pgfqpoint{4.579317in}{0.466634in}}{\pgfqpoint{4.574927in}{0.477233in}}{\pgfqpoint{4.567113in}{0.485047in}}%
\pgfpathcurveto{\pgfqpoint{4.559300in}{0.492860in}}{\pgfqpoint{4.548700in}{0.497250in}}{\pgfqpoint{4.537650in}{0.497250in}}%
\pgfpathcurveto{\pgfqpoint{4.526600in}{0.497250in}}{\pgfqpoint{4.516001in}{0.492860in}}{\pgfqpoint{4.508188in}{0.485047in}}%
\pgfpathcurveto{\pgfqpoint{4.500374in}{0.477233in}}{\pgfqpoint{4.495984in}{0.466634in}}{\pgfqpoint{4.495984in}{0.455584in}}%
\pgfpathcurveto{\pgfqpoint{4.495984in}{0.444534in}}{\pgfqpoint{4.500374in}{0.433935in}}{\pgfqpoint{4.508188in}{0.426121in}}%
\pgfpathcurveto{\pgfqpoint{4.516001in}{0.418307in}}{\pgfqpoint{4.526600in}{0.413917in}}{\pgfqpoint{4.537650in}{0.413917in}}%
\pgfpathlineto{\pgfqpoint{4.537650in}{0.413917in}}%
\pgfpathclose%
\pgfusepath{stroke}%
\end{pgfscope}%
\begin{pgfscope}%
\pgfpathrectangle{\pgfqpoint{0.393053in}{0.375000in}}{\pgfqpoint{6.356833in}{5.175000in}}%
\pgfusepath{clip}%
\pgfsetbuttcap%
\pgfsetroundjoin%
\pgfsetlinewidth{1.003750pt}%
\definecolor{currentstroke}{rgb}{0.827451,0.827451,0.827451}%
\pgfsetstrokecolor{currentstroke}%
\pgfsetdash{}{0pt}%
\pgfpathmoveto{\pgfqpoint{4.122151in}{0.465889in}}%
\pgfpathcurveto{\pgfqpoint{4.133202in}{0.465889in}}{\pgfqpoint{4.143801in}{0.470279in}}{\pgfqpoint{4.151614in}{0.478093in}}%
\pgfpathcurveto{\pgfqpoint{4.159428in}{0.485907in}}{\pgfqpoint{4.163818in}{0.496506in}}{\pgfqpoint{4.163818in}{0.507556in}}%
\pgfpathcurveto{\pgfqpoint{4.163818in}{0.518606in}}{\pgfqpoint{4.159428in}{0.529205in}}{\pgfqpoint{4.151614in}{0.537019in}}%
\pgfpathcurveto{\pgfqpoint{4.143801in}{0.544832in}}{\pgfqpoint{4.133202in}{0.549222in}}{\pgfqpoint{4.122151in}{0.549222in}}%
\pgfpathcurveto{\pgfqpoint{4.111101in}{0.549222in}}{\pgfqpoint{4.100502in}{0.544832in}}{\pgfqpoint{4.092689in}{0.537019in}}%
\pgfpathcurveto{\pgfqpoint{4.084875in}{0.529205in}}{\pgfqpoint{4.080485in}{0.518606in}}{\pgfqpoint{4.080485in}{0.507556in}}%
\pgfpathcurveto{\pgfqpoint{4.080485in}{0.496506in}}{\pgfqpoint{4.084875in}{0.485907in}}{\pgfqpoint{4.092689in}{0.478093in}}%
\pgfpathcurveto{\pgfqpoint{4.100502in}{0.470279in}}{\pgfqpoint{4.111101in}{0.465889in}}{\pgfqpoint{4.122151in}{0.465889in}}%
\pgfpathlineto{\pgfqpoint{4.122151in}{0.465889in}}%
\pgfpathclose%
\pgfusepath{stroke}%
\end{pgfscope}%
\begin{pgfscope}%
\pgfpathrectangle{\pgfqpoint{0.393053in}{0.375000in}}{\pgfqpoint{6.356833in}{5.175000in}}%
\pgfusepath{clip}%
\pgfsetbuttcap%
\pgfsetroundjoin%
\pgfsetlinewidth{1.003750pt}%
\definecolor{currentstroke}{rgb}{0.827451,0.827451,0.827451}%
\pgfsetstrokecolor{currentstroke}%
\pgfsetdash{}{0pt}%
\pgfpathmoveto{\pgfqpoint{1.585081in}{1.702736in}}%
\pgfpathcurveto{\pgfqpoint{1.596132in}{1.702736in}}{\pgfqpoint{1.606731in}{1.707127in}}{\pgfqpoint{1.614544in}{1.714940in}}%
\pgfpathcurveto{\pgfqpoint{1.622358in}{1.722754in}}{\pgfqpoint{1.626748in}{1.733353in}}{\pgfqpoint{1.626748in}{1.744403in}}%
\pgfpathcurveto{\pgfqpoint{1.626748in}{1.755453in}}{\pgfqpoint{1.622358in}{1.766052in}}{\pgfqpoint{1.614544in}{1.773866in}}%
\pgfpathcurveto{\pgfqpoint{1.606731in}{1.781679in}}{\pgfqpoint{1.596132in}{1.786070in}}{\pgfqpoint{1.585081in}{1.786070in}}%
\pgfpathcurveto{\pgfqpoint{1.574031in}{1.786070in}}{\pgfqpoint{1.563432in}{1.781679in}}{\pgfqpoint{1.555619in}{1.773866in}}%
\pgfpathcurveto{\pgfqpoint{1.547805in}{1.766052in}}{\pgfqpoint{1.543415in}{1.755453in}}{\pgfqpoint{1.543415in}{1.744403in}}%
\pgfpathcurveto{\pgfqpoint{1.543415in}{1.733353in}}{\pgfqpoint{1.547805in}{1.722754in}}{\pgfqpoint{1.555619in}{1.714940in}}%
\pgfpathcurveto{\pgfqpoint{1.563432in}{1.707127in}}{\pgfqpoint{1.574031in}{1.702736in}}{\pgfqpoint{1.585081in}{1.702736in}}%
\pgfpathlineto{\pgfqpoint{1.585081in}{1.702736in}}%
\pgfpathclose%
\pgfusepath{stroke}%
\end{pgfscope}%
\begin{pgfscope}%
\pgfpathrectangle{\pgfqpoint{0.393053in}{0.375000in}}{\pgfqpoint{6.356833in}{5.175000in}}%
\pgfusepath{clip}%
\pgfsetbuttcap%
\pgfsetroundjoin%
\pgfsetlinewidth{1.003750pt}%
\definecolor{currentstroke}{rgb}{0.827451,0.827451,0.827451}%
\pgfsetstrokecolor{currentstroke}%
\pgfsetdash{}{0pt}%
\pgfpathmoveto{\pgfqpoint{3.844148in}{0.542510in}}%
\pgfpathcurveto{\pgfqpoint{3.855198in}{0.542510in}}{\pgfqpoint{3.865797in}{0.546901in}}{\pgfqpoint{3.873610in}{0.554714in}}%
\pgfpathcurveto{\pgfqpoint{3.881424in}{0.562528in}}{\pgfqpoint{3.885814in}{0.573127in}}{\pgfqpoint{3.885814in}{0.584177in}}%
\pgfpathcurveto{\pgfqpoint{3.885814in}{0.595227in}}{\pgfqpoint{3.881424in}{0.605826in}}{\pgfqpoint{3.873610in}{0.613640in}}%
\pgfpathcurveto{\pgfqpoint{3.865797in}{0.621453in}}{\pgfqpoint{3.855198in}{0.625844in}}{\pgfqpoint{3.844148in}{0.625844in}}%
\pgfpathcurveto{\pgfqpoint{3.833097in}{0.625844in}}{\pgfqpoint{3.822498in}{0.621453in}}{\pgfqpoint{3.814685in}{0.613640in}}%
\pgfpathcurveto{\pgfqpoint{3.806871in}{0.605826in}}{\pgfqpoint{3.802481in}{0.595227in}}{\pgfqpoint{3.802481in}{0.584177in}}%
\pgfpathcurveto{\pgfqpoint{3.802481in}{0.573127in}}{\pgfqpoint{3.806871in}{0.562528in}}{\pgfqpoint{3.814685in}{0.554714in}}%
\pgfpathcurveto{\pgfqpoint{3.822498in}{0.546901in}}{\pgfqpoint{3.833097in}{0.542510in}}{\pgfqpoint{3.844148in}{0.542510in}}%
\pgfpathlineto{\pgfqpoint{3.844148in}{0.542510in}}%
\pgfpathclose%
\pgfusepath{stroke}%
\end{pgfscope}%
\begin{pgfscope}%
\pgfpathrectangle{\pgfqpoint{0.393053in}{0.375000in}}{\pgfqpoint{6.356833in}{5.175000in}}%
\pgfusepath{clip}%
\pgfsetbuttcap%
\pgfsetroundjoin%
\pgfsetlinewidth{1.003750pt}%
\definecolor{currentstroke}{rgb}{0.827451,0.827451,0.827451}%
\pgfsetstrokecolor{currentstroke}%
\pgfsetdash{}{0pt}%
\pgfpathmoveto{\pgfqpoint{3.679888in}{0.565041in}}%
\pgfpathcurveto{\pgfqpoint{3.690938in}{0.565041in}}{\pgfqpoint{3.701537in}{0.569432in}}{\pgfqpoint{3.709351in}{0.577245in}}%
\pgfpathcurveto{\pgfqpoint{3.717164in}{0.585059in}}{\pgfqpoint{3.721555in}{0.595658in}}{\pgfqpoint{3.721555in}{0.606708in}}%
\pgfpathcurveto{\pgfqpoint{3.721555in}{0.617758in}}{\pgfqpoint{3.717164in}{0.628357in}}{\pgfqpoint{3.709351in}{0.636171in}}%
\pgfpathcurveto{\pgfqpoint{3.701537in}{0.643985in}}{\pgfqpoint{3.690938in}{0.648375in}}{\pgfqpoint{3.679888in}{0.648375in}}%
\pgfpathcurveto{\pgfqpoint{3.668838in}{0.648375in}}{\pgfqpoint{3.658239in}{0.643985in}}{\pgfqpoint{3.650425in}{0.636171in}}%
\pgfpathcurveto{\pgfqpoint{3.642612in}{0.628357in}}{\pgfqpoint{3.638221in}{0.617758in}}{\pgfqpoint{3.638221in}{0.606708in}}%
\pgfpathcurveto{\pgfqpoint{3.638221in}{0.595658in}}{\pgfqpoint{3.642612in}{0.585059in}}{\pgfqpoint{3.650425in}{0.577245in}}%
\pgfpathcurveto{\pgfqpoint{3.658239in}{0.569432in}}{\pgfqpoint{3.668838in}{0.565041in}}{\pgfqpoint{3.679888in}{0.565041in}}%
\pgfpathlineto{\pgfqpoint{3.679888in}{0.565041in}}%
\pgfpathclose%
\pgfusepath{stroke}%
\end{pgfscope}%
\begin{pgfscope}%
\pgfpathrectangle{\pgfqpoint{0.393053in}{0.375000in}}{\pgfqpoint{6.356833in}{5.175000in}}%
\pgfusepath{clip}%
\pgfsetbuttcap%
\pgfsetroundjoin%
\pgfsetlinewidth{1.003750pt}%
\definecolor{currentstroke}{rgb}{0.827451,0.827451,0.827451}%
\pgfsetstrokecolor{currentstroke}%
\pgfsetdash{}{0pt}%
\pgfpathmoveto{\pgfqpoint{5.218388in}{0.351539in}}%
\pgfpathcurveto{\pgfqpoint{5.229438in}{0.351539in}}{\pgfqpoint{5.240037in}{0.355930in}}{\pgfqpoint{5.247851in}{0.363743in}}%
\pgfpathcurveto{\pgfqpoint{5.255664in}{0.371557in}}{\pgfqpoint{5.260055in}{0.382156in}}{\pgfqpoint{5.260055in}{0.393206in}}%
\pgfpathcurveto{\pgfqpoint{5.260055in}{0.404256in}}{\pgfqpoint{5.255664in}{0.414855in}}{\pgfqpoint{5.247851in}{0.422669in}}%
\pgfpathcurveto{\pgfqpoint{5.240037in}{0.430483in}}{\pgfqpoint{5.229438in}{0.434873in}}{\pgfqpoint{5.218388in}{0.434873in}}%
\pgfpathcurveto{\pgfqpoint{5.207338in}{0.434873in}}{\pgfqpoint{5.196739in}{0.430483in}}{\pgfqpoint{5.188925in}{0.422669in}}%
\pgfpathcurveto{\pgfqpoint{5.181112in}{0.414855in}}{\pgfqpoint{5.176721in}{0.404256in}}{\pgfqpoint{5.176721in}{0.393206in}}%
\pgfpathcurveto{\pgfqpoint{5.176721in}{0.382156in}}{\pgfqpoint{5.181112in}{0.371557in}}{\pgfqpoint{5.188925in}{0.363743in}}%
\pgfpathcurveto{\pgfqpoint{5.196739in}{0.355930in}}{\pgfqpoint{5.207338in}{0.351539in}}{\pgfqpoint{5.218388in}{0.351539in}}%
\pgfusepath{stroke}%
\end{pgfscope}%
\begin{pgfscope}%
\pgfpathrectangle{\pgfqpoint{0.393053in}{0.375000in}}{\pgfqpoint{6.356833in}{5.175000in}}%
\pgfusepath{clip}%
\pgfsetbuttcap%
\pgfsetroundjoin%
\pgfsetlinewidth{1.003750pt}%
\definecolor{currentstroke}{rgb}{0.827451,0.827451,0.827451}%
\pgfsetstrokecolor{currentstroke}%
\pgfsetdash{}{0pt}%
\pgfpathmoveto{\pgfqpoint{0.525326in}{3.470591in}}%
\pgfpathcurveto{\pgfqpoint{0.536376in}{3.470591in}}{\pgfqpoint{0.546975in}{3.474982in}}{\pgfqpoint{0.554788in}{3.482795in}}%
\pgfpathcurveto{\pgfqpoint{0.562602in}{3.490609in}}{\pgfqpoint{0.566992in}{3.501208in}}{\pgfqpoint{0.566992in}{3.512258in}}%
\pgfpathcurveto{\pgfqpoint{0.566992in}{3.523308in}}{\pgfqpoint{0.562602in}{3.533907in}}{\pgfqpoint{0.554788in}{3.541721in}}%
\pgfpathcurveto{\pgfqpoint{0.546975in}{3.549534in}}{\pgfqpoint{0.536376in}{3.553925in}}{\pgfqpoint{0.525326in}{3.553925in}}%
\pgfpathcurveto{\pgfqpoint{0.514275in}{3.553925in}}{\pgfqpoint{0.503676in}{3.549534in}}{\pgfqpoint{0.495863in}{3.541721in}}%
\pgfpathcurveto{\pgfqpoint{0.488049in}{3.533907in}}{\pgfqpoint{0.483659in}{3.523308in}}{\pgfqpoint{0.483659in}{3.512258in}}%
\pgfpathcurveto{\pgfqpoint{0.483659in}{3.501208in}}{\pgfqpoint{0.488049in}{3.490609in}}{\pgfqpoint{0.495863in}{3.482795in}}%
\pgfpathcurveto{\pgfqpoint{0.503676in}{3.474982in}}{\pgfqpoint{0.514275in}{3.470591in}}{\pgfqpoint{0.525326in}{3.470591in}}%
\pgfpathlineto{\pgfqpoint{0.525326in}{3.470591in}}%
\pgfpathclose%
\pgfusepath{stroke}%
\end{pgfscope}%
\begin{pgfscope}%
\pgfpathrectangle{\pgfqpoint{0.393053in}{0.375000in}}{\pgfqpoint{6.356833in}{5.175000in}}%
\pgfusepath{clip}%
\pgfsetbuttcap%
\pgfsetroundjoin%
\pgfsetlinewidth{1.003750pt}%
\definecolor{currentstroke}{rgb}{0.827451,0.827451,0.827451}%
\pgfsetstrokecolor{currentstroke}%
\pgfsetdash{}{0pt}%
\pgfpathmoveto{\pgfqpoint{0.483847in}{3.664437in}}%
\pgfpathcurveto{\pgfqpoint{0.494897in}{3.664437in}}{\pgfqpoint{0.505496in}{3.668827in}}{\pgfqpoint{0.513309in}{3.676640in}}%
\pgfpathcurveto{\pgfqpoint{0.521123in}{3.684454in}}{\pgfqpoint{0.525513in}{3.695053in}}{\pgfqpoint{0.525513in}{3.706103in}}%
\pgfpathcurveto{\pgfqpoint{0.525513in}{3.717153in}}{\pgfqpoint{0.521123in}{3.727752in}}{\pgfqpoint{0.513309in}{3.735566in}}%
\pgfpathcurveto{\pgfqpoint{0.505496in}{3.743380in}}{\pgfqpoint{0.494897in}{3.747770in}}{\pgfqpoint{0.483847in}{3.747770in}}%
\pgfpathcurveto{\pgfqpoint{0.472797in}{3.747770in}}{\pgfqpoint{0.462198in}{3.743380in}}{\pgfqpoint{0.454384in}{3.735566in}}%
\pgfpathcurveto{\pgfqpoint{0.446570in}{3.727752in}}{\pgfqpoint{0.442180in}{3.717153in}}{\pgfqpoint{0.442180in}{3.706103in}}%
\pgfpathcurveto{\pgfqpoint{0.442180in}{3.695053in}}{\pgfqpoint{0.446570in}{3.684454in}}{\pgfqpoint{0.454384in}{3.676640in}}%
\pgfpathcurveto{\pgfqpoint{0.462198in}{3.668827in}}{\pgfqpoint{0.472797in}{3.664437in}}{\pgfqpoint{0.483847in}{3.664437in}}%
\pgfpathlineto{\pgfqpoint{0.483847in}{3.664437in}}%
\pgfpathclose%
\pgfusepath{stroke}%
\end{pgfscope}%
\begin{pgfscope}%
\pgfpathrectangle{\pgfqpoint{0.393053in}{0.375000in}}{\pgfqpoint{6.356833in}{5.175000in}}%
\pgfusepath{clip}%
\pgfsetbuttcap%
\pgfsetroundjoin%
\pgfsetlinewidth{1.003750pt}%
\definecolor{currentstroke}{rgb}{0.827451,0.827451,0.827451}%
\pgfsetstrokecolor{currentstroke}%
\pgfsetdash{}{0pt}%
\pgfpathmoveto{\pgfqpoint{0.508412in}{3.594376in}}%
\pgfpathcurveto{\pgfqpoint{0.519462in}{3.594376in}}{\pgfqpoint{0.530061in}{3.598766in}}{\pgfqpoint{0.537874in}{3.606580in}}%
\pgfpathcurveto{\pgfqpoint{0.545688in}{3.614393in}}{\pgfqpoint{0.550078in}{3.624992in}}{\pgfqpoint{0.550078in}{3.636042in}}%
\pgfpathcurveto{\pgfqpoint{0.550078in}{3.647092in}}{\pgfqpoint{0.545688in}{3.657692in}}{\pgfqpoint{0.537874in}{3.665505in}}%
\pgfpathcurveto{\pgfqpoint{0.530061in}{3.673319in}}{\pgfqpoint{0.519462in}{3.677709in}}{\pgfqpoint{0.508412in}{3.677709in}}%
\pgfpathcurveto{\pgfqpoint{0.497361in}{3.677709in}}{\pgfqpoint{0.486762in}{3.673319in}}{\pgfqpoint{0.478949in}{3.665505in}}%
\pgfpathcurveto{\pgfqpoint{0.471135in}{3.657692in}}{\pgfqpoint{0.466745in}{3.647092in}}{\pgfqpoint{0.466745in}{3.636042in}}%
\pgfpathcurveto{\pgfqpoint{0.466745in}{3.624992in}}{\pgfqpoint{0.471135in}{3.614393in}}{\pgfqpoint{0.478949in}{3.606580in}}%
\pgfpathcurveto{\pgfqpoint{0.486762in}{3.598766in}}{\pgfqpoint{0.497361in}{3.594376in}}{\pgfqpoint{0.508412in}{3.594376in}}%
\pgfpathlineto{\pgfqpoint{0.508412in}{3.594376in}}%
\pgfpathclose%
\pgfusepath{stroke}%
\end{pgfscope}%
\begin{pgfscope}%
\pgfpathrectangle{\pgfqpoint{0.393053in}{0.375000in}}{\pgfqpoint{6.356833in}{5.175000in}}%
\pgfusepath{clip}%
\pgfsetbuttcap%
\pgfsetroundjoin%
\pgfsetlinewidth{1.003750pt}%
\definecolor{currentstroke}{rgb}{0.827451,0.827451,0.827451}%
\pgfsetstrokecolor{currentstroke}%
\pgfsetdash{}{0pt}%
\pgfpathmoveto{\pgfqpoint{3.885340in}{0.525311in}}%
\pgfpathcurveto{\pgfqpoint{3.896391in}{0.525311in}}{\pgfqpoint{3.906990in}{0.529701in}}{\pgfqpoint{3.914803in}{0.537515in}}%
\pgfpathcurveto{\pgfqpoint{3.922617in}{0.545328in}}{\pgfqpoint{3.927007in}{0.555927in}}{\pgfqpoint{3.927007in}{0.566978in}}%
\pgfpathcurveto{\pgfqpoint{3.927007in}{0.578028in}}{\pgfqpoint{3.922617in}{0.588627in}}{\pgfqpoint{3.914803in}{0.596440in}}%
\pgfpathcurveto{\pgfqpoint{3.906990in}{0.604254in}}{\pgfqpoint{3.896391in}{0.608644in}}{\pgfqpoint{3.885340in}{0.608644in}}%
\pgfpathcurveto{\pgfqpoint{3.874290in}{0.608644in}}{\pgfqpoint{3.863691in}{0.604254in}}{\pgfqpoint{3.855878in}{0.596440in}}%
\pgfpathcurveto{\pgfqpoint{3.848064in}{0.588627in}}{\pgfqpoint{3.843674in}{0.578028in}}{\pgfqpoint{3.843674in}{0.566978in}}%
\pgfpathcurveto{\pgfqpoint{3.843674in}{0.555927in}}{\pgfqpoint{3.848064in}{0.545328in}}{\pgfqpoint{3.855878in}{0.537515in}}%
\pgfpathcurveto{\pgfqpoint{3.863691in}{0.529701in}}{\pgfqpoint{3.874290in}{0.525311in}}{\pgfqpoint{3.885340in}{0.525311in}}%
\pgfpathlineto{\pgfqpoint{3.885340in}{0.525311in}}%
\pgfpathclose%
\pgfusepath{stroke}%
\end{pgfscope}%
\begin{pgfscope}%
\pgfpathrectangle{\pgfqpoint{0.393053in}{0.375000in}}{\pgfqpoint{6.356833in}{5.175000in}}%
\pgfusepath{clip}%
\pgfsetbuttcap%
\pgfsetroundjoin%
\pgfsetlinewidth{1.003750pt}%
\definecolor{currentstroke}{rgb}{0.827451,0.827451,0.827451}%
\pgfsetstrokecolor{currentstroke}%
\pgfsetdash{}{0pt}%
\pgfpathmoveto{\pgfqpoint{3.397541in}{0.662535in}}%
\pgfpathcurveto{\pgfqpoint{3.408591in}{0.662535in}}{\pgfqpoint{3.419190in}{0.666925in}}{\pgfqpoint{3.427004in}{0.674738in}}%
\pgfpathcurveto{\pgfqpoint{3.434817in}{0.682552in}}{\pgfqpoint{3.439208in}{0.693151in}}{\pgfqpoint{3.439208in}{0.704201in}}%
\pgfpathcurveto{\pgfqpoint{3.439208in}{0.715251in}}{\pgfqpoint{3.434817in}{0.725850in}}{\pgfqpoint{3.427004in}{0.733664in}}%
\pgfpathcurveto{\pgfqpoint{3.419190in}{0.741478in}}{\pgfqpoint{3.408591in}{0.745868in}}{\pgfqpoint{3.397541in}{0.745868in}}%
\pgfpathcurveto{\pgfqpoint{3.386491in}{0.745868in}}{\pgfqpoint{3.375892in}{0.741478in}}{\pgfqpoint{3.368078in}{0.733664in}}%
\pgfpathcurveto{\pgfqpoint{3.360265in}{0.725850in}}{\pgfqpoint{3.355874in}{0.715251in}}{\pgfqpoint{3.355874in}{0.704201in}}%
\pgfpathcurveto{\pgfqpoint{3.355874in}{0.693151in}}{\pgfqpoint{3.360265in}{0.682552in}}{\pgfqpoint{3.368078in}{0.674738in}}%
\pgfpathcurveto{\pgfqpoint{3.375892in}{0.666925in}}{\pgfqpoint{3.386491in}{0.662535in}}{\pgfqpoint{3.397541in}{0.662535in}}%
\pgfpathlineto{\pgfqpoint{3.397541in}{0.662535in}}%
\pgfpathclose%
\pgfusepath{stroke}%
\end{pgfscope}%
\begin{pgfscope}%
\pgfpathrectangle{\pgfqpoint{0.393053in}{0.375000in}}{\pgfqpoint{6.356833in}{5.175000in}}%
\pgfusepath{clip}%
\pgfsetbuttcap%
\pgfsetroundjoin%
\pgfsetlinewidth{1.003750pt}%
\definecolor{currentstroke}{rgb}{0.827451,0.827451,0.827451}%
\pgfsetstrokecolor{currentstroke}%
\pgfsetdash{}{0pt}%
\pgfpathmoveto{\pgfqpoint{1.040104in}{2.359499in}}%
\pgfpathcurveto{\pgfqpoint{1.051154in}{2.359499in}}{\pgfqpoint{1.061753in}{2.363889in}}{\pgfqpoint{1.069567in}{2.371703in}}%
\pgfpathcurveto{\pgfqpoint{1.077381in}{2.379516in}}{\pgfqpoint{1.081771in}{2.390115in}}{\pgfqpoint{1.081771in}{2.401165in}}%
\pgfpathcurveto{\pgfqpoint{1.081771in}{2.412215in}}{\pgfqpoint{1.077381in}{2.422814in}}{\pgfqpoint{1.069567in}{2.430628in}}%
\pgfpathcurveto{\pgfqpoint{1.061753in}{2.438442in}}{\pgfqpoint{1.051154in}{2.442832in}}{\pgfqpoint{1.040104in}{2.442832in}}%
\pgfpathcurveto{\pgfqpoint{1.029054in}{2.442832in}}{\pgfqpoint{1.018455in}{2.438442in}}{\pgfqpoint{1.010641in}{2.430628in}}%
\pgfpathcurveto{\pgfqpoint{1.002828in}{2.422814in}}{\pgfqpoint{0.998437in}{2.412215in}}{\pgfqpoint{0.998437in}{2.401165in}}%
\pgfpathcurveto{\pgfqpoint{0.998437in}{2.390115in}}{\pgfqpoint{1.002828in}{2.379516in}}{\pgfqpoint{1.010641in}{2.371703in}}%
\pgfpathcurveto{\pgfqpoint{1.018455in}{2.363889in}}{\pgfqpoint{1.029054in}{2.359499in}}{\pgfqpoint{1.040104in}{2.359499in}}%
\pgfpathlineto{\pgfqpoint{1.040104in}{2.359499in}}%
\pgfpathclose%
\pgfusepath{stroke}%
\end{pgfscope}%
\begin{pgfscope}%
\pgfpathrectangle{\pgfqpoint{0.393053in}{0.375000in}}{\pgfqpoint{6.356833in}{5.175000in}}%
\pgfusepath{clip}%
\pgfsetbuttcap%
\pgfsetroundjoin%
\pgfsetlinewidth{1.003750pt}%
\definecolor{currentstroke}{rgb}{0.827451,0.827451,0.827451}%
\pgfsetstrokecolor{currentstroke}%
\pgfsetdash{}{0pt}%
\pgfpathmoveto{\pgfqpoint{5.524477in}{0.341503in}}%
\pgfpathcurveto{\pgfqpoint{5.535528in}{0.341503in}}{\pgfqpoint{5.546127in}{0.345894in}}{\pgfqpoint{5.553940in}{0.353707in}}%
\pgfpathcurveto{\pgfqpoint{5.561754in}{0.361521in}}{\pgfqpoint{5.566144in}{0.372120in}}{\pgfqpoint{5.566144in}{0.383170in}}%
\pgfpathcurveto{\pgfqpoint{5.566144in}{0.394220in}}{\pgfqpoint{5.561754in}{0.404819in}}{\pgfqpoint{5.553940in}{0.412633in}}%
\pgfpathcurveto{\pgfqpoint{5.546127in}{0.420446in}}{\pgfqpoint{5.535528in}{0.424837in}}{\pgfqpoint{5.524477in}{0.424837in}}%
\pgfpathcurveto{\pgfqpoint{5.513427in}{0.424837in}}{\pgfqpoint{5.502828in}{0.420446in}}{\pgfqpoint{5.495015in}{0.412633in}}%
\pgfpathcurveto{\pgfqpoint{5.487201in}{0.404819in}}{\pgfqpoint{5.482811in}{0.394220in}}{\pgfqpoint{5.482811in}{0.383170in}}%
\pgfpathcurveto{\pgfqpoint{5.482811in}{0.372120in}}{\pgfqpoint{5.487201in}{0.361521in}}{\pgfqpoint{5.495015in}{0.353707in}}%
\pgfpathcurveto{\pgfqpoint{5.502828in}{0.345894in}}{\pgfqpoint{5.513427in}{0.341503in}}{\pgfqpoint{5.524477in}{0.341503in}}%
\pgfusepath{stroke}%
\end{pgfscope}%
\begin{pgfscope}%
\pgfpathrectangle{\pgfqpoint{0.393053in}{0.375000in}}{\pgfqpoint{6.356833in}{5.175000in}}%
\pgfusepath{clip}%
\pgfsetbuttcap%
\pgfsetroundjoin%
\pgfsetlinewidth{1.003750pt}%
\definecolor{currentstroke}{rgb}{0.827451,0.827451,0.827451}%
\pgfsetstrokecolor{currentstroke}%
\pgfsetdash{}{0pt}%
\pgfpathmoveto{\pgfqpoint{4.659727in}{0.389065in}}%
\pgfpathcurveto{\pgfqpoint{4.670777in}{0.389065in}}{\pgfqpoint{4.681376in}{0.393455in}}{\pgfqpoint{4.689190in}{0.401269in}}%
\pgfpathcurveto{\pgfqpoint{4.697004in}{0.409082in}}{\pgfqpoint{4.701394in}{0.419682in}}{\pgfqpoint{4.701394in}{0.430732in}}%
\pgfpathcurveto{\pgfqpoint{4.701394in}{0.441782in}}{\pgfqpoint{4.697004in}{0.452381in}}{\pgfqpoint{4.689190in}{0.460194in}}%
\pgfpathcurveto{\pgfqpoint{4.681376in}{0.468008in}}{\pgfqpoint{4.670777in}{0.472398in}}{\pgfqpoint{4.659727in}{0.472398in}}%
\pgfpathcurveto{\pgfqpoint{4.648677in}{0.472398in}}{\pgfqpoint{4.638078in}{0.468008in}}{\pgfqpoint{4.630264in}{0.460194in}}%
\pgfpathcurveto{\pgfqpoint{4.622451in}{0.452381in}}{\pgfqpoint{4.618060in}{0.441782in}}{\pgfqpoint{4.618060in}{0.430732in}}%
\pgfpathcurveto{\pgfqpoint{4.618060in}{0.419682in}}{\pgfqpoint{4.622451in}{0.409082in}}{\pgfqpoint{4.630264in}{0.401269in}}%
\pgfpathcurveto{\pgfqpoint{4.638078in}{0.393455in}}{\pgfqpoint{4.648677in}{0.389065in}}{\pgfqpoint{4.659727in}{0.389065in}}%
\pgfpathlineto{\pgfqpoint{4.659727in}{0.389065in}}%
\pgfpathclose%
\pgfusepath{stroke}%
\end{pgfscope}%
\begin{pgfscope}%
\pgfpathrectangle{\pgfqpoint{0.393053in}{0.375000in}}{\pgfqpoint{6.356833in}{5.175000in}}%
\pgfusepath{clip}%
\pgfsetbuttcap%
\pgfsetroundjoin%
\pgfsetlinewidth{1.003750pt}%
\definecolor{currentstroke}{rgb}{0.827451,0.827451,0.827451}%
\pgfsetstrokecolor{currentstroke}%
\pgfsetdash{}{0pt}%
\pgfpathmoveto{\pgfqpoint{0.806869in}{2.717160in}}%
\pgfpathcurveto{\pgfqpoint{0.817919in}{2.717160in}}{\pgfqpoint{0.828518in}{2.721550in}}{\pgfqpoint{0.836332in}{2.729363in}}%
\pgfpathcurveto{\pgfqpoint{0.844145in}{2.737177in}}{\pgfqpoint{0.848536in}{2.747776in}}{\pgfqpoint{0.848536in}{2.758826in}}%
\pgfpathcurveto{\pgfqpoint{0.848536in}{2.769876in}}{\pgfqpoint{0.844145in}{2.780475in}}{\pgfqpoint{0.836332in}{2.788289in}}%
\pgfpathcurveto{\pgfqpoint{0.828518in}{2.796103in}}{\pgfqpoint{0.817919in}{2.800493in}}{\pgfqpoint{0.806869in}{2.800493in}}%
\pgfpathcurveto{\pgfqpoint{0.795819in}{2.800493in}}{\pgfqpoint{0.785220in}{2.796103in}}{\pgfqpoint{0.777406in}{2.788289in}}%
\pgfpathcurveto{\pgfqpoint{0.769593in}{2.780475in}}{\pgfqpoint{0.765202in}{2.769876in}}{\pgfqpoint{0.765202in}{2.758826in}}%
\pgfpathcurveto{\pgfqpoint{0.765202in}{2.747776in}}{\pgfqpoint{0.769593in}{2.737177in}}{\pgfqpoint{0.777406in}{2.729363in}}%
\pgfpathcurveto{\pgfqpoint{0.785220in}{2.721550in}}{\pgfqpoint{0.795819in}{2.717160in}}{\pgfqpoint{0.806869in}{2.717160in}}%
\pgfpathlineto{\pgfqpoint{0.806869in}{2.717160in}}%
\pgfpathclose%
\pgfusepath{stroke}%
\end{pgfscope}%
\begin{pgfscope}%
\pgfpathrectangle{\pgfqpoint{0.393053in}{0.375000in}}{\pgfqpoint{6.356833in}{5.175000in}}%
\pgfusepath{clip}%
\pgfsetbuttcap%
\pgfsetroundjoin%
\pgfsetlinewidth{1.003750pt}%
\definecolor{currentstroke}{rgb}{0.827451,0.827451,0.827451}%
\pgfsetstrokecolor{currentstroke}%
\pgfsetdash{}{0pt}%
\pgfpathmoveto{\pgfqpoint{5.340211in}{0.342419in}}%
\pgfpathcurveto{\pgfqpoint{5.351261in}{0.342419in}}{\pgfqpoint{5.361860in}{0.346810in}}{\pgfqpoint{5.369673in}{0.354623in}}%
\pgfpathcurveto{\pgfqpoint{5.377487in}{0.362437in}}{\pgfqpoint{5.381877in}{0.373036in}}{\pgfqpoint{5.381877in}{0.384086in}}%
\pgfpathcurveto{\pgfqpoint{5.381877in}{0.395136in}}{\pgfqpoint{5.377487in}{0.405735in}}{\pgfqpoint{5.369673in}{0.413549in}}%
\pgfpathcurveto{\pgfqpoint{5.361860in}{0.421362in}}{\pgfqpoint{5.351261in}{0.425753in}}{\pgfqpoint{5.340211in}{0.425753in}}%
\pgfpathcurveto{\pgfqpoint{5.329160in}{0.425753in}}{\pgfqpoint{5.318561in}{0.421362in}}{\pgfqpoint{5.310748in}{0.413549in}}%
\pgfpathcurveto{\pgfqpoint{5.302934in}{0.405735in}}{\pgfqpoint{5.298544in}{0.395136in}}{\pgfqpoint{5.298544in}{0.384086in}}%
\pgfpathcurveto{\pgfqpoint{5.298544in}{0.373036in}}{\pgfqpoint{5.302934in}{0.362437in}}{\pgfqpoint{5.310748in}{0.354623in}}%
\pgfpathcurveto{\pgfqpoint{5.318561in}{0.346810in}}{\pgfqpoint{5.329160in}{0.342419in}}{\pgfqpoint{5.340211in}{0.342419in}}%
\pgfusepath{stroke}%
\end{pgfscope}%
\begin{pgfscope}%
\pgfpathrectangle{\pgfqpoint{0.393053in}{0.375000in}}{\pgfqpoint{6.356833in}{5.175000in}}%
\pgfusepath{clip}%
\pgfsetbuttcap%
\pgfsetroundjoin%
\pgfsetlinewidth{1.003750pt}%
\definecolor{currentstroke}{rgb}{0.827451,0.827451,0.827451}%
\pgfsetstrokecolor{currentstroke}%
\pgfsetdash{}{0pt}%
\pgfpathmoveto{\pgfqpoint{2.310226in}{1.162048in}}%
\pgfpathcurveto{\pgfqpoint{2.321276in}{1.162048in}}{\pgfqpoint{2.331875in}{1.166439in}}{\pgfqpoint{2.339688in}{1.174252in}}%
\pgfpathcurveto{\pgfqpoint{2.347502in}{1.182066in}}{\pgfqpoint{2.351892in}{1.192665in}}{\pgfqpoint{2.351892in}{1.203715in}}%
\pgfpathcurveto{\pgfqpoint{2.351892in}{1.214765in}}{\pgfqpoint{2.347502in}{1.225364in}}{\pgfqpoint{2.339688in}{1.233178in}}%
\pgfpathcurveto{\pgfqpoint{2.331875in}{1.240992in}}{\pgfqpoint{2.321276in}{1.245382in}}{\pgfqpoint{2.310226in}{1.245382in}}%
\pgfpathcurveto{\pgfqpoint{2.299175in}{1.245382in}}{\pgfqpoint{2.288576in}{1.240992in}}{\pgfqpoint{2.280763in}{1.233178in}}%
\pgfpathcurveto{\pgfqpoint{2.272949in}{1.225364in}}{\pgfqpoint{2.268559in}{1.214765in}}{\pgfqpoint{2.268559in}{1.203715in}}%
\pgfpathcurveto{\pgfqpoint{2.268559in}{1.192665in}}{\pgfqpoint{2.272949in}{1.182066in}}{\pgfqpoint{2.280763in}{1.174252in}}%
\pgfpathcurveto{\pgfqpoint{2.288576in}{1.166439in}}{\pgfqpoint{2.299175in}{1.162048in}}{\pgfqpoint{2.310226in}{1.162048in}}%
\pgfpathlineto{\pgfqpoint{2.310226in}{1.162048in}}%
\pgfpathclose%
\pgfusepath{stroke}%
\end{pgfscope}%
\begin{pgfscope}%
\pgfpathrectangle{\pgfqpoint{0.393053in}{0.375000in}}{\pgfqpoint{6.356833in}{5.175000in}}%
\pgfusepath{clip}%
\pgfsetbuttcap%
\pgfsetroundjoin%
\pgfsetlinewidth{1.003750pt}%
\definecolor{currentstroke}{rgb}{0.827451,0.827451,0.827451}%
\pgfsetstrokecolor{currentstroke}%
\pgfsetdash{}{0pt}%
\pgfpathmoveto{\pgfqpoint{1.702004in}{1.599430in}}%
\pgfpathcurveto{\pgfqpoint{1.713054in}{1.599430in}}{\pgfqpoint{1.723653in}{1.603820in}}{\pgfqpoint{1.731467in}{1.611633in}}%
\pgfpathcurveto{\pgfqpoint{1.739281in}{1.619447in}}{\pgfqpoint{1.743671in}{1.630046in}}{\pgfqpoint{1.743671in}{1.641096in}}%
\pgfpathcurveto{\pgfqpoint{1.743671in}{1.652146in}}{\pgfqpoint{1.739281in}{1.662745in}}{\pgfqpoint{1.731467in}{1.670559in}}%
\pgfpathcurveto{\pgfqpoint{1.723653in}{1.678373in}}{\pgfqpoint{1.713054in}{1.682763in}}{\pgfqpoint{1.702004in}{1.682763in}}%
\pgfpathcurveto{\pgfqpoint{1.690954in}{1.682763in}}{\pgfqpoint{1.680355in}{1.678373in}}{\pgfqpoint{1.672541in}{1.670559in}}%
\pgfpathcurveto{\pgfqpoint{1.664728in}{1.662745in}}{\pgfqpoint{1.660338in}{1.652146in}}{\pgfqpoint{1.660338in}{1.641096in}}%
\pgfpathcurveto{\pgfqpoint{1.660338in}{1.630046in}}{\pgfqpoint{1.664728in}{1.619447in}}{\pgfqpoint{1.672541in}{1.611633in}}%
\pgfpathcurveto{\pgfqpoint{1.680355in}{1.603820in}}{\pgfqpoint{1.690954in}{1.599430in}}{\pgfqpoint{1.702004in}{1.599430in}}%
\pgfpathlineto{\pgfqpoint{1.702004in}{1.599430in}}%
\pgfpathclose%
\pgfusepath{stroke}%
\end{pgfscope}%
\begin{pgfscope}%
\pgfpathrectangle{\pgfqpoint{0.393053in}{0.375000in}}{\pgfqpoint{6.356833in}{5.175000in}}%
\pgfusepath{clip}%
\pgfsetbuttcap%
\pgfsetroundjoin%
\pgfsetlinewidth{1.003750pt}%
\definecolor{currentstroke}{rgb}{0.827451,0.827451,0.827451}%
\pgfsetstrokecolor{currentstroke}%
\pgfsetdash{}{0pt}%
\pgfpathmoveto{\pgfqpoint{0.474470in}{3.727925in}}%
\pgfpathcurveto{\pgfqpoint{0.485520in}{3.727925in}}{\pgfqpoint{0.496119in}{3.732315in}}{\pgfqpoint{0.503933in}{3.740129in}}%
\pgfpathcurveto{\pgfqpoint{0.511746in}{3.747943in}}{\pgfqpoint{0.516137in}{3.758542in}}{\pgfqpoint{0.516137in}{3.769592in}}%
\pgfpathcurveto{\pgfqpoint{0.516137in}{3.780642in}}{\pgfqpoint{0.511746in}{3.791241in}}{\pgfqpoint{0.503933in}{3.799055in}}%
\pgfpathcurveto{\pgfqpoint{0.496119in}{3.806868in}}{\pgfqpoint{0.485520in}{3.811259in}}{\pgfqpoint{0.474470in}{3.811259in}}%
\pgfpathcurveto{\pgfqpoint{0.463420in}{3.811259in}}{\pgfqpoint{0.452821in}{3.806868in}}{\pgfqpoint{0.445007in}{3.799055in}}%
\pgfpathcurveto{\pgfqpoint{0.437194in}{3.791241in}}{\pgfqpoint{0.432803in}{3.780642in}}{\pgfqpoint{0.432803in}{3.769592in}}%
\pgfpathcurveto{\pgfqpoint{0.432803in}{3.758542in}}{\pgfqpoint{0.437194in}{3.747943in}}{\pgfqpoint{0.445007in}{3.740129in}}%
\pgfpathcurveto{\pgfqpoint{0.452821in}{3.732315in}}{\pgfqpoint{0.463420in}{3.727925in}}{\pgfqpoint{0.474470in}{3.727925in}}%
\pgfpathlineto{\pgfqpoint{0.474470in}{3.727925in}}%
\pgfpathclose%
\pgfusepath{stroke}%
\end{pgfscope}%
\begin{pgfscope}%
\pgfpathrectangle{\pgfqpoint{0.393053in}{0.375000in}}{\pgfqpoint{6.356833in}{5.175000in}}%
\pgfusepath{clip}%
\pgfsetbuttcap%
\pgfsetroundjoin%
\pgfsetlinewidth{1.003750pt}%
\definecolor{currentstroke}{rgb}{0.827451,0.827451,0.827451}%
\pgfsetstrokecolor{currentstroke}%
\pgfsetdash{}{0pt}%
\pgfpathmoveto{\pgfqpoint{0.459942in}{3.789224in}}%
\pgfpathcurveto{\pgfqpoint{0.470992in}{3.789224in}}{\pgfqpoint{0.481591in}{3.793614in}}{\pgfqpoint{0.489404in}{3.801428in}}%
\pgfpathcurveto{\pgfqpoint{0.497218in}{3.809242in}}{\pgfqpoint{0.501608in}{3.819841in}}{\pgfqpoint{0.501608in}{3.830891in}}%
\pgfpathcurveto{\pgfqpoint{0.501608in}{3.841941in}}{\pgfqpoint{0.497218in}{3.852540in}}{\pgfqpoint{0.489404in}{3.860354in}}%
\pgfpathcurveto{\pgfqpoint{0.481591in}{3.868167in}}{\pgfqpoint{0.470992in}{3.872557in}}{\pgfqpoint{0.459942in}{3.872557in}}%
\pgfpathcurveto{\pgfqpoint{0.448891in}{3.872557in}}{\pgfqpoint{0.438292in}{3.868167in}}{\pgfqpoint{0.430479in}{3.860354in}}%
\pgfpathcurveto{\pgfqpoint{0.422665in}{3.852540in}}{\pgfqpoint{0.418275in}{3.841941in}}{\pgfqpoint{0.418275in}{3.830891in}}%
\pgfpathcurveto{\pgfqpoint{0.418275in}{3.819841in}}{\pgfqpoint{0.422665in}{3.809242in}}{\pgfqpoint{0.430479in}{3.801428in}}%
\pgfpathcurveto{\pgfqpoint{0.438292in}{3.793614in}}{\pgfqpoint{0.448891in}{3.789224in}}{\pgfqpoint{0.459942in}{3.789224in}}%
\pgfpathlineto{\pgfqpoint{0.459942in}{3.789224in}}%
\pgfpathclose%
\pgfusepath{stroke}%
\end{pgfscope}%
\begin{pgfscope}%
\pgfpathrectangle{\pgfqpoint{0.393053in}{0.375000in}}{\pgfqpoint{6.356833in}{5.175000in}}%
\pgfusepath{clip}%
\pgfsetbuttcap%
\pgfsetroundjoin%
\pgfsetlinewidth{1.003750pt}%
\definecolor{currentstroke}{rgb}{0.827451,0.827451,0.827451}%
\pgfsetstrokecolor{currentstroke}%
\pgfsetdash{}{0pt}%
\pgfpathmoveto{\pgfqpoint{5.642998in}{0.337452in}}%
\pgfpathcurveto{\pgfqpoint{5.654048in}{0.337452in}}{\pgfqpoint{5.664647in}{0.341843in}}{\pgfqpoint{5.672461in}{0.349656in}}%
\pgfpathcurveto{\pgfqpoint{5.680274in}{0.357470in}}{\pgfqpoint{5.684665in}{0.368069in}}{\pgfqpoint{5.684665in}{0.379119in}}%
\pgfpathcurveto{\pgfqpoint{5.684665in}{0.390169in}}{\pgfqpoint{5.680274in}{0.400768in}}{\pgfqpoint{5.672461in}{0.408582in}}%
\pgfpathcurveto{\pgfqpoint{5.664647in}{0.416395in}}{\pgfqpoint{5.654048in}{0.420786in}}{\pgfqpoint{5.642998in}{0.420786in}}%
\pgfpathcurveto{\pgfqpoint{5.631948in}{0.420786in}}{\pgfqpoint{5.621349in}{0.416395in}}{\pgfqpoint{5.613535in}{0.408582in}}%
\pgfpathcurveto{\pgfqpoint{5.605721in}{0.400768in}}{\pgfqpoint{5.601331in}{0.390169in}}{\pgfqpoint{5.601331in}{0.379119in}}%
\pgfpathcurveto{\pgfqpoint{5.601331in}{0.368069in}}{\pgfqpoint{5.605721in}{0.357470in}}{\pgfqpoint{5.613535in}{0.349656in}}%
\pgfpathcurveto{\pgfqpoint{5.621349in}{0.341843in}}{\pgfqpoint{5.631948in}{0.337452in}}{\pgfqpoint{5.642998in}{0.337452in}}%
\pgfusepath{stroke}%
\end{pgfscope}%
\begin{pgfscope}%
\pgfpathrectangle{\pgfqpoint{0.393053in}{0.375000in}}{\pgfqpoint{6.356833in}{5.175000in}}%
\pgfusepath{clip}%
\pgfsetbuttcap%
\pgfsetroundjoin%
\pgfsetlinewidth{1.003750pt}%
\definecolor{currentstroke}{rgb}{0.827451,0.827451,0.827451}%
\pgfsetstrokecolor{currentstroke}%
\pgfsetdash{}{0pt}%
\pgfpathmoveto{\pgfqpoint{1.375643in}{1.906128in}}%
\pgfpathcurveto{\pgfqpoint{1.386694in}{1.906128in}}{\pgfqpoint{1.397293in}{1.910518in}}{\pgfqpoint{1.405106in}{1.918331in}}%
\pgfpathcurveto{\pgfqpoint{1.412920in}{1.926145in}}{\pgfqpoint{1.417310in}{1.936744in}}{\pgfqpoint{1.417310in}{1.947794in}}%
\pgfpathcurveto{\pgfqpoint{1.417310in}{1.958844in}}{\pgfqpoint{1.412920in}{1.969443in}}{\pgfqpoint{1.405106in}{1.977257in}}%
\pgfpathcurveto{\pgfqpoint{1.397293in}{1.985071in}}{\pgfqpoint{1.386694in}{1.989461in}}{\pgfqpoint{1.375643in}{1.989461in}}%
\pgfpathcurveto{\pgfqpoint{1.364593in}{1.989461in}}{\pgfqpoint{1.353994in}{1.985071in}}{\pgfqpoint{1.346181in}{1.977257in}}%
\pgfpathcurveto{\pgfqpoint{1.338367in}{1.969443in}}{\pgfqpoint{1.333977in}{1.958844in}}{\pgfqpoint{1.333977in}{1.947794in}}%
\pgfpathcurveto{\pgfqpoint{1.333977in}{1.936744in}}{\pgfqpoint{1.338367in}{1.926145in}}{\pgfqpoint{1.346181in}{1.918331in}}%
\pgfpathcurveto{\pgfqpoint{1.353994in}{1.910518in}}{\pgfqpoint{1.364593in}{1.906128in}}{\pgfqpoint{1.375643in}{1.906128in}}%
\pgfpathlineto{\pgfqpoint{1.375643in}{1.906128in}}%
\pgfpathclose%
\pgfusepath{stroke}%
\end{pgfscope}%
\begin{pgfscope}%
\pgfpathrectangle{\pgfqpoint{0.393053in}{0.375000in}}{\pgfqpoint{6.356833in}{5.175000in}}%
\pgfusepath{clip}%
\pgfsetbuttcap%
\pgfsetroundjoin%
\pgfsetlinewidth{1.003750pt}%
\definecolor{currentstroke}{rgb}{0.827451,0.827451,0.827451}%
\pgfsetstrokecolor{currentstroke}%
\pgfsetdash{}{0pt}%
\pgfpathmoveto{\pgfqpoint{2.036490in}{1.340063in}}%
\pgfpathcurveto{\pgfqpoint{2.047540in}{1.340063in}}{\pgfqpoint{2.058139in}{1.344454in}}{\pgfqpoint{2.065953in}{1.352267in}}%
\pgfpathcurveto{\pgfqpoint{2.073767in}{1.360081in}}{\pgfqpoint{2.078157in}{1.370680in}}{\pgfqpoint{2.078157in}{1.381730in}}%
\pgfpathcurveto{\pgfqpoint{2.078157in}{1.392780in}}{\pgfqpoint{2.073767in}{1.403379in}}{\pgfqpoint{2.065953in}{1.411193in}}%
\pgfpathcurveto{\pgfqpoint{2.058139in}{1.419007in}}{\pgfqpoint{2.047540in}{1.423397in}}{\pgfqpoint{2.036490in}{1.423397in}}%
\pgfpathcurveto{\pgfqpoint{2.025440in}{1.423397in}}{\pgfqpoint{2.014841in}{1.419007in}}{\pgfqpoint{2.007027in}{1.411193in}}%
\pgfpathcurveto{\pgfqpoint{1.999214in}{1.403379in}}{\pgfqpoint{1.994824in}{1.392780in}}{\pgfqpoint{1.994824in}{1.381730in}}%
\pgfpathcurveto{\pgfqpoint{1.994824in}{1.370680in}}{\pgfqpoint{1.999214in}{1.360081in}}{\pgfqpoint{2.007027in}{1.352267in}}%
\pgfpathcurveto{\pgfqpoint{2.014841in}{1.344454in}}{\pgfqpoint{2.025440in}{1.340063in}}{\pgfqpoint{2.036490in}{1.340063in}}%
\pgfpathlineto{\pgfqpoint{2.036490in}{1.340063in}}%
\pgfpathclose%
\pgfusepath{stroke}%
\end{pgfscope}%
\begin{pgfscope}%
\pgfpathrectangle{\pgfqpoint{0.393053in}{0.375000in}}{\pgfqpoint{6.356833in}{5.175000in}}%
\pgfusepath{clip}%
\pgfsetbuttcap%
\pgfsetroundjoin%
\pgfsetlinewidth{1.003750pt}%
\definecolor{currentstroke}{rgb}{0.827451,0.827451,0.827451}%
\pgfsetstrokecolor{currentstroke}%
\pgfsetdash{}{0pt}%
\pgfpathmoveto{\pgfqpoint{1.431784in}{1.847586in}}%
\pgfpathcurveto{\pgfqpoint{1.442835in}{1.847586in}}{\pgfqpoint{1.453434in}{1.851976in}}{\pgfqpoint{1.461247in}{1.859790in}}%
\pgfpathcurveto{\pgfqpoint{1.469061in}{1.867603in}}{\pgfqpoint{1.473451in}{1.878202in}}{\pgfqpoint{1.473451in}{1.889253in}}%
\pgfpathcurveto{\pgfqpoint{1.473451in}{1.900303in}}{\pgfqpoint{1.469061in}{1.910902in}}{\pgfqpoint{1.461247in}{1.918715in}}%
\pgfpathcurveto{\pgfqpoint{1.453434in}{1.926529in}}{\pgfqpoint{1.442835in}{1.930919in}}{\pgfqpoint{1.431784in}{1.930919in}}%
\pgfpathcurveto{\pgfqpoint{1.420734in}{1.930919in}}{\pgfqpoint{1.410135in}{1.926529in}}{\pgfqpoint{1.402322in}{1.918715in}}%
\pgfpathcurveto{\pgfqpoint{1.394508in}{1.910902in}}{\pgfqpoint{1.390118in}{1.900303in}}{\pgfqpoint{1.390118in}{1.889253in}}%
\pgfpathcurveto{\pgfqpoint{1.390118in}{1.878202in}}{\pgfqpoint{1.394508in}{1.867603in}}{\pgfqpoint{1.402322in}{1.859790in}}%
\pgfpathcurveto{\pgfqpoint{1.410135in}{1.851976in}}{\pgfqpoint{1.420734in}{1.847586in}}{\pgfqpoint{1.431784in}{1.847586in}}%
\pgfpathlineto{\pgfqpoint{1.431784in}{1.847586in}}%
\pgfpathclose%
\pgfusepath{stroke}%
\end{pgfscope}%
\begin{pgfscope}%
\pgfpathrectangle{\pgfqpoint{0.393053in}{0.375000in}}{\pgfqpoint{6.356833in}{5.175000in}}%
\pgfusepath{clip}%
\pgfsetbuttcap%
\pgfsetroundjoin%
\pgfsetlinewidth{1.003750pt}%
\definecolor{currentstroke}{rgb}{0.827451,0.827451,0.827451}%
\pgfsetstrokecolor{currentstroke}%
\pgfsetdash{}{0pt}%
\pgfpathmoveto{\pgfqpoint{0.511163in}{3.526410in}}%
\pgfpathcurveto{\pgfqpoint{0.522213in}{3.526410in}}{\pgfqpoint{0.532812in}{3.530801in}}{\pgfqpoint{0.540626in}{3.538614in}}%
\pgfpathcurveto{\pgfqpoint{0.548440in}{3.546428in}}{\pgfqpoint{0.552830in}{3.557027in}}{\pgfqpoint{0.552830in}{3.568077in}}%
\pgfpathcurveto{\pgfqpoint{0.552830in}{3.579127in}}{\pgfqpoint{0.548440in}{3.589726in}}{\pgfqpoint{0.540626in}{3.597540in}}%
\pgfpathcurveto{\pgfqpoint{0.532812in}{3.605353in}}{\pgfqpoint{0.522213in}{3.609744in}}{\pgfqpoint{0.511163in}{3.609744in}}%
\pgfpathcurveto{\pgfqpoint{0.500113in}{3.609744in}}{\pgfqpoint{0.489514in}{3.605353in}}{\pgfqpoint{0.481700in}{3.597540in}}%
\pgfpathcurveto{\pgfqpoint{0.473887in}{3.589726in}}{\pgfqpoint{0.469496in}{3.579127in}}{\pgfqpoint{0.469496in}{3.568077in}}%
\pgfpathcurveto{\pgfqpoint{0.469496in}{3.557027in}}{\pgfqpoint{0.473887in}{3.546428in}}{\pgfqpoint{0.481700in}{3.538614in}}%
\pgfpathcurveto{\pgfqpoint{0.489514in}{3.530801in}}{\pgfqpoint{0.500113in}{3.526410in}}{\pgfqpoint{0.511163in}{3.526410in}}%
\pgfpathlineto{\pgfqpoint{0.511163in}{3.526410in}}%
\pgfpathclose%
\pgfusepath{stroke}%
\end{pgfscope}%
\begin{pgfscope}%
\pgfpathrectangle{\pgfqpoint{0.393053in}{0.375000in}}{\pgfqpoint{6.356833in}{5.175000in}}%
\pgfusepath{clip}%
\pgfsetbuttcap%
\pgfsetroundjoin%
\pgfsetlinewidth{1.003750pt}%
\definecolor{currentstroke}{rgb}{0.827451,0.827451,0.827451}%
\pgfsetstrokecolor{currentstroke}%
\pgfsetdash{}{0pt}%
\pgfpathmoveto{\pgfqpoint{0.546206in}{3.392492in}}%
\pgfpathcurveto{\pgfqpoint{0.557256in}{3.392492in}}{\pgfqpoint{0.567855in}{3.396882in}}{\pgfqpoint{0.575669in}{3.404696in}}%
\pgfpathcurveto{\pgfqpoint{0.583483in}{3.412509in}}{\pgfqpoint{0.587873in}{3.423108in}}{\pgfqpoint{0.587873in}{3.434159in}}%
\pgfpathcurveto{\pgfqpoint{0.587873in}{3.445209in}}{\pgfqpoint{0.583483in}{3.455808in}}{\pgfqpoint{0.575669in}{3.463621in}}%
\pgfpathcurveto{\pgfqpoint{0.567855in}{3.471435in}}{\pgfqpoint{0.557256in}{3.475825in}}{\pgfqpoint{0.546206in}{3.475825in}}%
\pgfpathcurveto{\pgfqpoint{0.535156in}{3.475825in}}{\pgfqpoint{0.524557in}{3.471435in}}{\pgfqpoint{0.516744in}{3.463621in}}%
\pgfpathcurveto{\pgfqpoint{0.508930in}{3.455808in}}{\pgfqpoint{0.504540in}{3.445209in}}{\pgfqpoint{0.504540in}{3.434159in}}%
\pgfpathcurveto{\pgfqpoint{0.504540in}{3.423108in}}{\pgfqpoint{0.508930in}{3.412509in}}{\pgfqpoint{0.516744in}{3.404696in}}%
\pgfpathcurveto{\pgfqpoint{0.524557in}{3.396882in}}{\pgfqpoint{0.535156in}{3.392492in}}{\pgfqpoint{0.546206in}{3.392492in}}%
\pgfpathlineto{\pgfqpoint{0.546206in}{3.392492in}}%
\pgfpathclose%
\pgfusepath{stroke}%
\end{pgfscope}%
\begin{pgfscope}%
\pgfpathrectangle{\pgfqpoint{0.393053in}{0.375000in}}{\pgfqpoint{6.356833in}{5.175000in}}%
\pgfusepath{clip}%
\pgfsetbuttcap%
\pgfsetroundjoin%
\pgfsetlinewidth{1.003750pt}%
\definecolor{currentstroke}{rgb}{0.827451,0.827451,0.827451}%
\pgfsetstrokecolor{currentstroke}%
\pgfsetdash{}{0pt}%
\pgfpathmoveto{\pgfqpoint{2.376136in}{1.121818in}}%
\pgfpathcurveto{\pgfqpoint{2.387186in}{1.121818in}}{\pgfqpoint{2.397785in}{1.126208in}}{\pgfqpoint{2.405598in}{1.134022in}}%
\pgfpathcurveto{\pgfqpoint{2.413412in}{1.141835in}}{\pgfqpoint{2.417802in}{1.152434in}}{\pgfqpoint{2.417802in}{1.163485in}}%
\pgfpathcurveto{\pgfqpoint{2.417802in}{1.174535in}}{\pgfqpoint{2.413412in}{1.185134in}}{\pgfqpoint{2.405598in}{1.192947in}}%
\pgfpathcurveto{\pgfqpoint{2.397785in}{1.200761in}}{\pgfqpoint{2.387186in}{1.205151in}}{\pgfqpoint{2.376136in}{1.205151in}}%
\pgfpathcurveto{\pgfqpoint{2.365086in}{1.205151in}}{\pgfqpoint{2.354487in}{1.200761in}}{\pgfqpoint{2.346673in}{1.192947in}}%
\pgfpathcurveto{\pgfqpoint{2.338859in}{1.185134in}}{\pgfqpoint{2.334469in}{1.174535in}}{\pgfqpoint{2.334469in}{1.163485in}}%
\pgfpathcurveto{\pgfqpoint{2.334469in}{1.152434in}}{\pgfqpoint{2.338859in}{1.141835in}}{\pgfqpoint{2.346673in}{1.134022in}}%
\pgfpathcurveto{\pgfqpoint{2.354487in}{1.126208in}}{\pgfqpoint{2.365086in}{1.121818in}}{\pgfqpoint{2.376136in}{1.121818in}}%
\pgfpathlineto{\pgfqpoint{2.376136in}{1.121818in}}%
\pgfpathclose%
\pgfusepath{stroke}%
\end{pgfscope}%
\begin{pgfscope}%
\pgfpathrectangle{\pgfqpoint{0.393053in}{0.375000in}}{\pgfqpoint{6.356833in}{5.175000in}}%
\pgfusepath{clip}%
\pgfsetbuttcap%
\pgfsetroundjoin%
\pgfsetlinewidth{1.003750pt}%
\definecolor{currentstroke}{rgb}{0.827451,0.827451,0.827451}%
\pgfsetstrokecolor{currentstroke}%
\pgfsetdash{}{0pt}%
\pgfpathmoveto{\pgfqpoint{0.635911in}{3.119895in}}%
\pgfpathcurveto{\pgfqpoint{0.646961in}{3.119895in}}{\pgfqpoint{0.657560in}{3.124285in}}{\pgfqpoint{0.665374in}{3.132098in}}%
\pgfpathcurveto{\pgfqpoint{0.673187in}{3.139912in}}{\pgfqpoint{0.677577in}{3.150511in}}{\pgfqpoint{0.677577in}{3.161561in}}%
\pgfpathcurveto{\pgfqpoint{0.677577in}{3.172611in}}{\pgfqpoint{0.673187in}{3.183210in}}{\pgfqpoint{0.665374in}{3.191024in}}%
\pgfpathcurveto{\pgfqpoint{0.657560in}{3.198838in}}{\pgfqpoint{0.646961in}{3.203228in}}{\pgfqpoint{0.635911in}{3.203228in}}%
\pgfpathcurveto{\pgfqpoint{0.624861in}{3.203228in}}{\pgfqpoint{0.614262in}{3.198838in}}{\pgfqpoint{0.606448in}{3.191024in}}%
\pgfpathcurveto{\pgfqpoint{0.598634in}{3.183210in}}{\pgfqpoint{0.594244in}{3.172611in}}{\pgfqpoint{0.594244in}{3.161561in}}%
\pgfpathcurveto{\pgfqpoint{0.594244in}{3.150511in}}{\pgfqpoint{0.598634in}{3.139912in}}{\pgfqpoint{0.606448in}{3.132098in}}%
\pgfpathcurveto{\pgfqpoint{0.614262in}{3.124285in}}{\pgfqpoint{0.624861in}{3.119895in}}{\pgfqpoint{0.635911in}{3.119895in}}%
\pgfpathlineto{\pgfqpoint{0.635911in}{3.119895in}}%
\pgfpathclose%
\pgfusepath{stroke}%
\end{pgfscope}%
\begin{pgfscope}%
\pgfpathrectangle{\pgfqpoint{0.393053in}{0.375000in}}{\pgfqpoint{6.356833in}{5.175000in}}%
\pgfusepath{clip}%
\pgfsetbuttcap%
\pgfsetroundjoin%
\pgfsetlinewidth{1.003750pt}%
\definecolor{currentstroke}{rgb}{0.827451,0.827451,0.827451}%
\pgfsetstrokecolor{currentstroke}%
\pgfsetdash{}{0pt}%
\pgfpathmoveto{\pgfqpoint{0.775585in}{2.785130in}}%
\pgfpathcurveto{\pgfqpoint{0.786635in}{2.785130in}}{\pgfqpoint{0.797234in}{2.789521in}}{\pgfqpoint{0.805048in}{2.797334in}}%
\pgfpathcurveto{\pgfqpoint{0.812861in}{2.805148in}}{\pgfqpoint{0.817252in}{2.815747in}}{\pgfqpoint{0.817252in}{2.826797in}}%
\pgfpathcurveto{\pgfqpoint{0.817252in}{2.837847in}}{\pgfqpoint{0.812861in}{2.848446in}}{\pgfqpoint{0.805048in}{2.856260in}}%
\pgfpathcurveto{\pgfqpoint{0.797234in}{2.864073in}}{\pgfqpoint{0.786635in}{2.868464in}}{\pgfqpoint{0.775585in}{2.868464in}}%
\pgfpathcurveto{\pgfqpoint{0.764535in}{2.868464in}}{\pgfqpoint{0.753936in}{2.864073in}}{\pgfqpoint{0.746122in}{2.856260in}}%
\pgfpathcurveto{\pgfqpoint{0.738309in}{2.848446in}}{\pgfqpoint{0.733918in}{2.837847in}}{\pgfqpoint{0.733918in}{2.826797in}}%
\pgfpathcurveto{\pgfqpoint{0.733918in}{2.815747in}}{\pgfqpoint{0.738309in}{2.805148in}}{\pgfqpoint{0.746122in}{2.797334in}}%
\pgfpathcurveto{\pgfqpoint{0.753936in}{2.789521in}}{\pgfqpoint{0.764535in}{2.785130in}}{\pgfqpoint{0.775585in}{2.785130in}}%
\pgfpathlineto{\pgfqpoint{0.775585in}{2.785130in}}%
\pgfpathclose%
\pgfusepath{stroke}%
\end{pgfscope}%
\begin{pgfscope}%
\pgfpathrectangle{\pgfqpoint{0.393053in}{0.375000in}}{\pgfqpoint{6.356833in}{5.175000in}}%
\pgfusepath{clip}%
\pgfsetbuttcap%
\pgfsetroundjoin%
\pgfsetlinewidth{1.003750pt}%
\definecolor{currentstroke}{rgb}{0.827451,0.827451,0.827451}%
\pgfsetstrokecolor{currentstroke}%
\pgfsetdash{}{0pt}%
\pgfpathmoveto{\pgfqpoint{1.882270in}{1.458195in}}%
\pgfpathcurveto{\pgfqpoint{1.893321in}{1.458195in}}{\pgfqpoint{1.903920in}{1.462585in}}{\pgfqpoint{1.911733in}{1.470399in}}%
\pgfpathcurveto{\pgfqpoint{1.919547in}{1.478213in}}{\pgfqpoint{1.923937in}{1.488812in}}{\pgfqpoint{1.923937in}{1.499862in}}%
\pgfpathcurveto{\pgfqpoint{1.923937in}{1.510912in}}{\pgfqpoint{1.919547in}{1.521511in}}{\pgfqpoint{1.911733in}{1.529325in}}%
\pgfpathcurveto{\pgfqpoint{1.903920in}{1.537138in}}{\pgfqpoint{1.893321in}{1.541529in}}{\pgfqpoint{1.882270in}{1.541529in}}%
\pgfpathcurveto{\pgfqpoint{1.871220in}{1.541529in}}{\pgfqpoint{1.860621in}{1.537138in}}{\pgfqpoint{1.852808in}{1.529325in}}%
\pgfpathcurveto{\pgfqpoint{1.844994in}{1.521511in}}{\pgfqpoint{1.840604in}{1.510912in}}{\pgfqpoint{1.840604in}{1.499862in}}%
\pgfpathcurveto{\pgfqpoint{1.840604in}{1.488812in}}{\pgfqpoint{1.844994in}{1.478213in}}{\pgfqpoint{1.852808in}{1.470399in}}%
\pgfpathcurveto{\pgfqpoint{1.860621in}{1.462585in}}{\pgfqpoint{1.871220in}{1.458195in}}{\pgfqpoint{1.882270in}{1.458195in}}%
\pgfpathlineto{\pgfqpoint{1.882270in}{1.458195in}}%
\pgfpathclose%
\pgfusepath{stroke}%
\end{pgfscope}%
\begin{pgfscope}%
\pgfpathrectangle{\pgfqpoint{0.393053in}{0.375000in}}{\pgfqpoint{6.356833in}{5.175000in}}%
\pgfusepath{clip}%
\pgfsetbuttcap%
\pgfsetroundjoin%
\pgfsetlinewidth{1.003750pt}%
\definecolor{currentstroke}{rgb}{0.827451,0.827451,0.827451}%
\pgfsetstrokecolor{currentstroke}%
\pgfsetdash{}{0pt}%
\pgfpathmoveto{\pgfqpoint{1.536415in}{1.768054in}}%
\pgfpathcurveto{\pgfqpoint{1.547466in}{1.768054in}}{\pgfqpoint{1.558065in}{1.772444in}}{\pgfqpoint{1.565878in}{1.780258in}}%
\pgfpathcurveto{\pgfqpoint{1.573692in}{1.788071in}}{\pgfqpoint{1.578082in}{1.798670in}}{\pgfqpoint{1.578082in}{1.809720in}}%
\pgfpathcurveto{\pgfqpoint{1.578082in}{1.820770in}}{\pgfqpoint{1.573692in}{1.831370in}}{\pgfqpoint{1.565878in}{1.839183in}}%
\pgfpathcurveto{\pgfqpoint{1.558065in}{1.846997in}}{\pgfqpoint{1.547466in}{1.851387in}}{\pgfqpoint{1.536415in}{1.851387in}}%
\pgfpathcurveto{\pgfqpoint{1.525365in}{1.851387in}}{\pgfqpoint{1.514766in}{1.846997in}}{\pgfqpoint{1.506953in}{1.839183in}}%
\pgfpathcurveto{\pgfqpoint{1.499139in}{1.831370in}}{\pgfqpoint{1.494749in}{1.820770in}}{\pgfqpoint{1.494749in}{1.809720in}}%
\pgfpathcurveto{\pgfqpoint{1.494749in}{1.798670in}}{\pgfqpoint{1.499139in}{1.788071in}}{\pgfqpoint{1.506953in}{1.780258in}}%
\pgfpathcurveto{\pgfqpoint{1.514766in}{1.772444in}}{\pgfqpoint{1.525365in}{1.768054in}}{\pgfqpoint{1.536415in}{1.768054in}}%
\pgfpathlineto{\pgfqpoint{1.536415in}{1.768054in}}%
\pgfpathclose%
\pgfusepath{stroke}%
\end{pgfscope}%
\begin{pgfscope}%
\pgfpathrectangle{\pgfqpoint{0.393053in}{0.375000in}}{\pgfqpoint{6.356833in}{5.175000in}}%
\pgfusepath{clip}%
\pgfsetbuttcap%
\pgfsetroundjoin%
\pgfsetlinewidth{1.003750pt}%
\definecolor{currentstroke}{rgb}{0.827451,0.827451,0.827451}%
\pgfsetstrokecolor{currentstroke}%
\pgfsetdash{}{0pt}%
\pgfpathmoveto{\pgfqpoint{1.105942in}{2.248010in}}%
\pgfpathcurveto{\pgfqpoint{1.116993in}{2.248010in}}{\pgfqpoint{1.127592in}{2.252400in}}{\pgfqpoint{1.135405in}{2.260214in}}%
\pgfpathcurveto{\pgfqpoint{1.143219in}{2.268028in}}{\pgfqpoint{1.147609in}{2.278627in}}{\pgfqpoint{1.147609in}{2.289677in}}%
\pgfpathcurveto{\pgfqpoint{1.147609in}{2.300727in}}{\pgfqpoint{1.143219in}{2.311326in}}{\pgfqpoint{1.135405in}{2.319140in}}%
\pgfpathcurveto{\pgfqpoint{1.127592in}{2.326953in}}{\pgfqpoint{1.116993in}{2.331343in}}{\pgfqpoint{1.105942in}{2.331343in}}%
\pgfpathcurveto{\pgfqpoint{1.094892in}{2.331343in}}{\pgfqpoint{1.084293in}{2.326953in}}{\pgfqpoint{1.076480in}{2.319140in}}%
\pgfpathcurveto{\pgfqpoint{1.068666in}{2.311326in}}{\pgfqpoint{1.064276in}{2.300727in}}{\pgfqpoint{1.064276in}{2.289677in}}%
\pgfpathcurveto{\pgfqpoint{1.064276in}{2.278627in}}{\pgfqpoint{1.068666in}{2.268028in}}{\pgfqpoint{1.076480in}{2.260214in}}%
\pgfpathcurveto{\pgfqpoint{1.084293in}{2.252400in}}{\pgfqpoint{1.094892in}{2.248010in}}{\pgfqpoint{1.105942in}{2.248010in}}%
\pgfpathlineto{\pgfqpoint{1.105942in}{2.248010in}}%
\pgfpathclose%
\pgfusepath{stroke}%
\end{pgfscope}%
\begin{pgfscope}%
\pgfpathrectangle{\pgfqpoint{0.393053in}{0.375000in}}{\pgfqpoint{6.356833in}{5.175000in}}%
\pgfusepath{clip}%
\pgfsetbuttcap%
\pgfsetroundjoin%
\pgfsetlinewidth{1.003750pt}%
\definecolor{currentstroke}{rgb}{0.827451,0.827451,0.827451}%
\pgfsetstrokecolor{currentstroke}%
\pgfsetdash{}{0pt}%
\pgfpathmoveto{\pgfqpoint{4.445943in}{0.425176in}}%
\pgfpathcurveto{\pgfqpoint{4.456993in}{0.425176in}}{\pgfqpoint{4.467592in}{0.429567in}}{\pgfqpoint{4.475406in}{0.437380in}}%
\pgfpathcurveto{\pgfqpoint{4.483219in}{0.445194in}}{\pgfqpoint{4.487610in}{0.455793in}}{\pgfqpoint{4.487610in}{0.466843in}}%
\pgfpathcurveto{\pgfqpoint{4.487610in}{0.477893in}}{\pgfqpoint{4.483219in}{0.488492in}}{\pgfqpoint{4.475406in}{0.496306in}}%
\pgfpathcurveto{\pgfqpoint{4.467592in}{0.504119in}}{\pgfqpoint{4.456993in}{0.508510in}}{\pgfqpoint{4.445943in}{0.508510in}}%
\pgfpathcurveto{\pgfqpoint{4.434893in}{0.508510in}}{\pgfqpoint{4.424294in}{0.504119in}}{\pgfqpoint{4.416480in}{0.496306in}}%
\pgfpathcurveto{\pgfqpoint{4.408667in}{0.488492in}}{\pgfqpoint{4.404276in}{0.477893in}}{\pgfqpoint{4.404276in}{0.466843in}}%
\pgfpathcurveto{\pgfqpoint{4.404276in}{0.455793in}}{\pgfqpoint{4.408667in}{0.445194in}}{\pgfqpoint{4.416480in}{0.437380in}}%
\pgfpathcurveto{\pgfqpoint{4.424294in}{0.429567in}}{\pgfqpoint{4.434893in}{0.425176in}}{\pgfqpoint{4.445943in}{0.425176in}}%
\pgfpathlineto{\pgfqpoint{4.445943in}{0.425176in}}%
\pgfpathclose%
\pgfusepath{stroke}%
\end{pgfscope}%
\begin{pgfscope}%
\pgfpathrectangle{\pgfqpoint{0.393053in}{0.375000in}}{\pgfqpoint{6.356833in}{5.175000in}}%
\pgfusepath{clip}%
\pgfsetbuttcap%
\pgfsetroundjoin%
\pgfsetlinewidth{1.003750pt}%
\definecolor{currentstroke}{rgb}{0.827451,0.827451,0.827451}%
\pgfsetstrokecolor{currentstroke}%
\pgfsetdash{}{0pt}%
\pgfpathmoveto{\pgfqpoint{0.651937in}{3.079342in}}%
\pgfpathcurveto{\pgfqpoint{0.662987in}{3.079342in}}{\pgfqpoint{0.673586in}{3.083732in}}{\pgfqpoint{0.681400in}{3.091546in}}%
\pgfpathcurveto{\pgfqpoint{0.689213in}{3.099359in}}{\pgfqpoint{0.693604in}{3.109958in}}{\pgfqpoint{0.693604in}{3.121009in}}%
\pgfpathcurveto{\pgfqpoint{0.693604in}{3.132059in}}{\pgfqpoint{0.689213in}{3.142658in}}{\pgfqpoint{0.681400in}{3.150471in}}%
\pgfpathcurveto{\pgfqpoint{0.673586in}{3.158285in}}{\pgfqpoint{0.662987in}{3.162675in}}{\pgfqpoint{0.651937in}{3.162675in}}%
\pgfpathcurveto{\pgfqpoint{0.640887in}{3.162675in}}{\pgfqpoint{0.630288in}{3.158285in}}{\pgfqpoint{0.622474in}{3.150471in}}%
\pgfpathcurveto{\pgfqpoint{0.614660in}{3.142658in}}{\pgfqpoint{0.610270in}{3.132059in}}{\pgfqpoint{0.610270in}{3.121009in}}%
\pgfpathcurveto{\pgfqpoint{0.610270in}{3.109958in}}{\pgfqpoint{0.614660in}{3.099359in}}{\pgfqpoint{0.622474in}{3.091546in}}%
\pgfpathcurveto{\pgfqpoint{0.630288in}{3.083732in}}{\pgfqpoint{0.640887in}{3.079342in}}{\pgfqpoint{0.651937in}{3.079342in}}%
\pgfpathlineto{\pgfqpoint{0.651937in}{3.079342in}}%
\pgfpathclose%
\pgfusepath{stroke}%
\end{pgfscope}%
\begin{pgfscope}%
\pgfpathrectangle{\pgfqpoint{0.393053in}{0.375000in}}{\pgfqpoint{6.356833in}{5.175000in}}%
\pgfusepath{clip}%
\pgfsetbuttcap%
\pgfsetroundjoin%
\pgfsetlinewidth{1.003750pt}%
\definecolor{currentstroke}{rgb}{0.827451,0.827451,0.827451}%
\pgfsetstrokecolor{currentstroke}%
\pgfsetdash{}{0pt}%
\pgfpathmoveto{\pgfqpoint{0.672197in}{3.023099in}}%
\pgfpathcurveto{\pgfqpoint{0.683247in}{3.023099in}}{\pgfqpoint{0.693846in}{3.027489in}}{\pgfqpoint{0.701660in}{3.035303in}}%
\pgfpathcurveto{\pgfqpoint{0.709473in}{3.043116in}}{\pgfqpoint{0.713863in}{3.053715in}}{\pgfqpoint{0.713863in}{3.064766in}}%
\pgfpathcurveto{\pgfqpoint{0.713863in}{3.075816in}}{\pgfqpoint{0.709473in}{3.086415in}}{\pgfqpoint{0.701660in}{3.094228in}}%
\pgfpathcurveto{\pgfqpoint{0.693846in}{3.102042in}}{\pgfqpoint{0.683247in}{3.106432in}}{\pgfqpoint{0.672197in}{3.106432in}}%
\pgfpathcurveto{\pgfqpoint{0.661147in}{3.106432in}}{\pgfqpoint{0.650548in}{3.102042in}}{\pgfqpoint{0.642734in}{3.094228in}}%
\pgfpathcurveto{\pgfqpoint{0.634920in}{3.086415in}}{\pgfqpoint{0.630530in}{3.075816in}}{\pgfqpoint{0.630530in}{3.064766in}}%
\pgfpathcurveto{\pgfqpoint{0.630530in}{3.053715in}}{\pgfqpoint{0.634920in}{3.043116in}}{\pgfqpoint{0.642734in}{3.035303in}}%
\pgfpathcurveto{\pgfqpoint{0.650548in}{3.027489in}}{\pgfqpoint{0.661147in}{3.023099in}}{\pgfqpoint{0.672197in}{3.023099in}}%
\pgfpathlineto{\pgfqpoint{0.672197in}{3.023099in}}%
\pgfpathclose%
\pgfusepath{stroke}%
\end{pgfscope}%
\begin{pgfscope}%
\pgfpathrectangle{\pgfqpoint{0.393053in}{0.375000in}}{\pgfqpoint{6.356833in}{5.175000in}}%
\pgfusepath{clip}%
\pgfsetbuttcap%
\pgfsetroundjoin%
\pgfsetlinewidth{1.003750pt}%
\definecolor{currentstroke}{rgb}{0.827451,0.827451,0.827451}%
\pgfsetstrokecolor{currentstroke}%
\pgfsetdash{}{0pt}%
\pgfpathmoveto{\pgfqpoint{4.217280in}{0.449238in}}%
\pgfpathcurveto{\pgfqpoint{4.228330in}{0.449238in}}{\pgfqpoint{4.238929in}{0.453628in}}{\pgfqpoint{4.246743in}{0.461441in}}%
\pgfpathcurveto{\pgfqpoint{4.254556in}{0.469255in}}{\pgfqpoint{4.258947in}{0.479854in}}{\pgfqpoint{4.258947in}{0.490904in}}%
\pgfpathcurveto{\pgfqpoint{4.258947in}{0.501954in}}{\pgfqpoint{4.254556in}{0.512553in}}{\pgfqpoint{4.246743in}{0.520367in}}%
\pgfpathcurveto{\pgfqpoint{4.238929in}{0.528181in}}{\pgfqpoint{4.228330in}{0.532571in}}{\pgfqpoint{4.217280in}{0.532571in}}%
\pgfpathcurveto{\pgfqpoint{4.206230in}{0.532571in}}{\pgfqpoint{4.195631in}{0.528181in}}{\pgfqpoint{4.187817in}{0.520367in}}%
\pgfpathcurveto{\pgfqpoint{4.180004in}{0.512553in}}{\pgfqpoint{4.175613in}{0.501954in}}{\pgfqpoint{4.175613in}{0.490904in}}%
\pgfpathcurveto{\pgfqpoint{4.175613in}{0.479854in}}{\pgfqpoint{4.180004in}{0.469255in}}{\pgfqpoint{4.187817in}{0.461441in}}%
\pgfpathcurveto{\pgfqpoint{4.195631in}{0.453628in}}{\pgfqpoint{4.206230in}{0.449238in}}{\pgfqpoint{4.217280in}{0.449238in}}%
\pgfpathlineto{\pgfqpoint{4.217280in}{0.449238in}}%
\pgfpathclose%
\pgfusepath{stroke}%
\end{pgfscope}%
\begin{pgfscope}%
\pgfpathrectangle{\pgfqpoint{0.393053in}{0.375000in}}{\pgfqpoint{6.356833in}{5.175000in}}%
\pgfusepath{clip}%
\pgfsetbuttcap%
\pgfsetroundjoin%
\pgfsetlinewidth{1.003750pt}%
\definecolor{currentstroke}{rgb}{0.827451,0.827451,0.827451}%
\pgfsetstrokecolor{currentstroke}%
\pgfsetdash{}{0pt}%
\pgfpathmoveto{\pgfqpoint{2.179603in}{1.241266in}}%
\pgfpathcurveto{\pgfqpoint{2.190653in}{1.241266in}}{\pgfqpoint{2.201252in}{1.245657in}}{\pgfqpoint{2.209066in}{1.253470in}}%
\pgfpathcurveto{\pgfqpoint{2.216880in}{1.261284in}}{\pgfqpoint{2.221270in}{1.271883in}}{\pgfqpoint{2.221270in}{1.282933in}}%
\pgfpathcurveto{\pgfqpoint{2.221270in}{1.293983in}}{\pgfqpoint{2.216880in}{1.304582in}}{\pgfqpoint{2.209066in}{1.312396in}}%
\pgfpathcurveto{\pgfqpoint{2.201252in}{1.320209in}}{\pgfqpoint{2.190653in}{1.324600in}}{\pgfqpoint{2.179603in}{1.324600in}}%
\pgfpathcurveto{\pgfqpoint{2.168553in}{1.324600in}}{\pgfqpoint{2.157954in}{1.320209in}}{\pgfqpoint{2.150140in}{1.312396in}}%
\pgfpathcurveto{\pgfqpoint{2.142327in}{1.304582in}}{\pgfqpoint{2.137936in}{1.293983in}}{\pgfqpoint{2.137936in}{1.282933in}}%
\pgfpathcurveto{\pgfqpoint{2.137936in}{1.271883in}}{\pgfqpoint{2.142327in}{1.261284in}}{\pgfqpoint{2.150140in}{1.253470in}}%
\pgfpathcurveto{\pgfqpoint{2.157954in}{1.245657in}}{\pgfqpoint{2.168553in}{1.241266in}}{\pgfqpoint{2.179603in}{1.241266in}}%
\pgfpathlineto{\pgfqpoint{2.179603in}{1.241266in}}%
\pgfpathclose%
\pgfusepath{stroke}%
\end{pgfscope}%
\begin{pgfscope}%
\pgfpathrectangle{\pgfqpoint{0.393053in}{0.375000in}}{\pgfqpoint{6.356833in}{5.175000in}}%
\pgfusepath{clip}%
\pgfsetbuttcap%
\pgfsetroundjoin%
\pgfsetlinewidth{1.003750pt}%
\definecolor{currentstroke}{rgb}{0.827451,0.827451,0.827451}%
\pgfsetstrokecolor{currentstroke}%
\pgfsetdash{}{0pt}%
\pgfpathmoveto{\pgfqpoint{0.580211in}{3.378166in}}%
\pgfpathcurveto{\pgfqpoint{0.591261in}{3.378166in}}{\pgfqpoint{0.601860in}{3.382556in}}{\pgfqpoint{0.609674in}{3.390370in}}%
\pgfpathcurveto{\pgfqpoint{0.617487in}{3.398183in}}{\pgfqpoint{0.621878in}{3.408782in}}{\pgfqpoint{0.621878in}{3.419833in}}%
\pgfpathcurveto{\pgfqpoint{0.621878in}{3.430883in}}{\pgfqpoint{0.617487in}{3.441482in}}{\pgfqpoint{0.609674in}{3.449295in}}%
\pgfpathcurveto{\pgfqpoint{0.601860in}{3.457109in}}{\pgfqpoint{0.591261in}{3.461499in}}{\pgfqpoint{0.580211in}{3.461499in}}%
\pgfpathcurveto{\pgfqpoint{0.569161in}{3.461499in}}{\pgfqpoint{0.558562in}{3.457109in}}{\pgfqpoint{0.550748in}{3.449295in}}%
\pgfpathcurveto{\pgfqpoint{0.542935in}{3.441482in}}{\pgfqpoint{0.538544in}{3.430883in}}{\pgfqpoint{0.538544in}{3.419833in}}%
\pgfpathcurveto{\pgfqpoint{0.538544in}{3.408782in}}{\pgfqpoint{0.542935in}{3.398183in}}{\pgfqpoint{0.550748in}{3.390370in}}%
\pgfpathcurveto{\pgfqpoint{0.558562in}{3.382556in}}{\pgfqpoint{0.569161in}{3.378166in}}{\pgfqpoint{0.580211in}{3.378166in}}%
\pgfpathlineto{\pgfqpoint{0.580211in}{3.378166in}}%
\pgfpathclose%
\pgfusepath{stroke}%
\end{pgfscope}%
\begin{pgfscope}%
\pgfpathrectangle{\pgfqpoint{0.393053in}{0.375000in}}{\pgfqpoint{6.356833in}{5.175000in}}%
\pgfusepath{clip}%
\pgfsetbuttcap%
\pgfsetroundjoin%
\pgfsetlinewidth{1.003750pt}%
\definecolor{currentstroke}{rgb}{0.827451,0.827451,0.827451}%
\pgfsetstrokecolor{currentstroke}%
\pgfsetdash{}{0pt}%
\pgfpathmoveto{\pgfqpoint{0.612319in}{3.180269in}}%
\pgfpathcurveto{\pgfqpoint{0.623369in}{3.180269in}}{\pgfqpoint{0.633968in}{3.184660in}}{\pgfqpoint{0.641781in}{3.192473in}}%
\pgfpathcurveto{\pgfqpoint{0.649595in}{3.200287in}}{\pgfqpoint{0.653985in}{3.210886in}}{\pgfqpoint{0.653985in}{3.221936in}}%
\pgfpathcurveto{\pgfqpoint{0.653985in}{3.232986in}}{\pgfqpoint{0.649595in}{3.243585in}}{\pgfqpoint{0.641781in}{3.251399in}}%
\pgfpathcurveto{\pgfqpoint{0.633968in}{3.259212in}}{\pgfqpoint{0.623369in}{3.263603in}}{\pgfqpoint{0.612319in}{3.263603in}}%
\pgfpathcurveto{\pgfqpoint{0.601269in}{3.263603in}}{\pgfqpoint{0.590669in}{3.259212in}}{\pgfqpoint{0.582856in}{3.251399in}}%
\pgfpathcurveto{\pgfqpoint{0.575042in}{3.243585in}}{\pgfqpoint{0.570652in}{3.232986in}}{\pgfqpoint{0.570652in}{3.221936in}}%
\pgfpathcurveto{\pgfqpoint{0.570652in}{3.210886in}}{\pgfqpoint{0.575042in}{3.200287in}}{\pgfqpoint{0.582856in}{3.192473in}}%
\pgfpathcurveto{\pgfqpoint{0.590669in}{3.184660in}}{\pgfqpoint{0.601269in}{3.180269in}}{\pgfqpoint{0.612319in}{3.180269in}}%
\pgfpathlineto{\pgfqpoint{0.612319in}{3.180269in}}%
\pgfpathclose%
\pgfusepath{stroke}%
\end{pgfscope}%
\begin{pgfscope}%
\pgfpathrectangle{\pgfqpoint{0.393053in}{0.375000in}}{\pgfqpoint{6.356833in}{5.175000in}}%
\pgfusepath{clip}%
\pgfsetbuttcap%
\pgfsetroundjoin%
\pgfsetlinewidth{1.003750pt}%
\definecolor{currentstroke}{rgb}{0.827451,0.827451,0.827451}%
\pgfsetstrokecolor{currentstroke}%
\pgfsetdash{}{0pt}%
\pgfpathmoveto{\pgfqpoint{4.655370in}{0.409439in}}%
\pgfpathcurveto{\pgfqpoint{4.666420in}{0.409439in}}{\pgfqpoint{4.677019in}{0.413829in}}{\pgfqpoint{4.684833in}{0.421642in}}%
\pgfpathcurveto{\pgfqpoint{4.692646in}{0.429456in}}{\pgfqpoint{4.697036in}{0.440055in}}{\pgfqpoint{4.697036in}{0.451105in}}%
\pgfpathcurveto{\pgfqpoint{4.697036in}{0.462155in}}{\pgfqpoint{4.692646in}{0.472754in}}{\pgfqpoint{4.684833in}{0.480568in}}%
\pgfpathcurveto{\pgfqpoint{4.677019in}{0.488382in}}{\pgfqpoint{4.666420in}{0.492772in}}{\pgfqpoint{4.655370in}{0.492772in}}%
\pgfpathcurveto{\pgfqpoint{4.644320in}{0.492772in}}{\pgfqpoint{4.633721in}{0.488382in}}{\pgfqpoint{4.625907in}{0.480568in}}%
\pgfpathcurveto{\pgfqpoint{4.618093in}{0.472754in}}{\pgfqpoint{4.613703in}{0.462155in}}{\pgfqpoint{4.613703in}{0.451105in}}%
\pgfpathcurveto{\pgfqpoint{4.613703in}{0.440055in}}{\pgfqpoint{4.618093in}{0.429456in}}{\pgfqpoint{4.625907in}{0.421642in}}%
\pgfpathcurveto{\pgfqpoint{4.633721in}{0.413829in}}{\pgfqpoint{4.644320in}{0.409439in}}{\pgfqpoint{4.655370in}{0.409439in}}%
\pgfpathlineto{\pgfqpoint{4.655370in}{0.409439in}}%
\pgfpathclose%
\pgfusepath{stroke}%
\end{pgfscope}%
\begin{pgfscope}%
\pgfpathrectangle{\pgfqpoint{0.393053in}{0.375000in}}{\pgfqpoint{6.356833in}{5.175000in}}%
\pgfusepath{clip}%
\pgfsetbuttcap%
\pgfsetroundjoin%
\pgfsetlinewidth{1.003750pt}%
\definecolor{currentstroke}{rgb}{0.827451,0.827451,0.827451}%
\pgfsetstrokecolor{currentstroke}%
\pgfsetdash{}{0pt}%
\pgfpathmoveto{\pgfqpoint{1.312906in}{1.976052in}}%
\pgfpathcurveto{\pgfqpoint{1.323956in}{1.976052in}}{\pgfqpoint{1.334555in}{1.980442in}}{\pgfqpoint{1.342369in}{1.988256in}}%
\pgfpathcurveto{\pgfqpoint{1.350183in}{1.996070in}}{\pgfqpoint{1.354573in}{2.006669in}}{\pgfqpoint{1.354573in}{2.017719in}}%
\pgfpathcurveto{\pgfqpoint{1.354573in}{2.028769in}}{\pgfqpoint{1.350183in}{2.039368in}}{\pgfqpoint{1.342369in}{2.047182in}}%
\pgfpathcurveto{\pgfqpoint{1.334555in}{2.054995in}}{\pgfqpoint{1.323956in}{2.059386in}}{\pgfqpoint{1.312906in}{2.059386in}}%
\pgfpathcurveto{\pgfqpoint{1.301856in}{2.059386in}}{\pgfqpoint{1.291257in}{2.054995in}}{\pgfqpoint{1.283443in}{2.047182in}}%
\pgfpathcurveto{\pgfqpoint{1.275630in}{2.039368in}}{\pgfqpoint{1.271240in}{2.028769in}}{\pgfqpoint{1.271240in}{2.017719in}}%
\pgfpathcurveto{\pgfqpoint{1.271240in}{2.006669in}}{\pgfqpoint{1.275630in}{1.996070in}}{\pgfqpoint{1.283443in}{1.988256in}}%
\pgfpathcurveto{\pgfqpoint{1.291257in}{1.980442in}}{\pgfqpoint{1.301856in}{1.976052in}}{\pgfqpoint{1.312906in}{1.976052in}}%
\pgfpathlineto{\pgfqpoint{1.312906in}{1.976052in}}%
\pgfpathclose%
\pgfusepath{stroke}%
\end{pgfscope}%
\begin{pgfscope}%
\pgfpathrectangle{\pgfqpoint{0.393053in}{0.375000in}}{\pgfqpoint{6.356833in}{5.175000in}}%
\pgfusepath{clip}%
\pgfsetbuttcap%
\pgfsetroundjoin%
\pgfsetlinewidth{1.003750pt}%
\definecolor{currentstroke}{rgb}{0.827451,0.827451,0.827451}%
\pgfsetstrokecolor{currentstroke}%
\pgfsetdash{}{0pt}%
\pgfpathmoveto{\pgfqpoint{1.626356in}{1.668438in}}%
\pgfpathcurveto{\pgfqpoint{1.637406in}{1.668438in}}{\pgfqpoint{1.648005in}{1.672828in}}{\pgfqpoint{1.655818in}{1.680641in}}%
\pgfpathcurveto{\pgfqpoint{1.663632in}{1.688455in}}{\pgfqpoint{1.668022in}{1.699054in}}{\pgfqpoint{1.668022in}{1.710104in}}%
\pgfpathcurveto{\pgfqpoint{1.668022in}{1.721154in}}{\pgfqpoint{1.663632in}{1.731753in}}{\pgfqpoint{1.655818in}{1.739567in}}%
\pgfpathcurveto{\pgfqpoint{1.648005in}{1.747381in}}{\pgfqpoint{1.637406in}{1.751771in}}{\pgfqpoint{1.626356in}{1.751771in}}%
\pgfpathcurveto{\pgfqpoint{1.615305in}{1.751771in}}{\pgfqpoint{1.604706in}{1.747381in}}{\pgfqpoint{1.596893in}{1.739567in}}%
\pgfpathcurveto{\pgfqpoint{1.589079in}{1.731753in}}{\pgfqpoint{1.584689in}{1.721154in}}{\pgfqpoint{1.584689in}{1.710104in}}%
\pgfpathcurveto{\pgfqpoint{1.584689in}{1.699054in}}{\pgfqpoint{1.589079in}{1.688455in}}{\pgfqpoint{1.596893in}{1.680641in}}%
\pgfpathcurveto{\pgfqpoint{1.604706in}{1.672828in}}{\pgfqpoint{1.615305in}{1.668438in}}{\pgfqpoint{1.626356in}{1.668438in}}%
\pgfpathlineto{\pgfqpoint{1.626356in}{1.668438in}}%
\pgfpathclose%
\pgfusepath{stroke}%
\end{pgfscope}%
\begin{pgfscope}%
\pgfpathrectangle{\pgfqpoint{0.393053in}{0.375000in}}{\pgfqpoint{6.356833in}{5.175000in}}%
\pgfusepath{clip}%
\pgfsetbuttcap%
\pgfsetroundjoin%
\pgfsetlinewidth{1.003750pt}%
\definecolor{currentstroke}{rgb}{0.827451,0.827451,0.827451}%
\pgfsetstrokecolor{currentstroke}%
\pgfsetdash{}{0pt}%
\pgfpathmoveto{\pgfqpoint{1.354968in}{1.928632in}}%
\pgfpathcurveto{\pgfqpoint{1.366018in}{1.928632in}}{\pgfqpoint{1.376617in}{1.933022in}}{\pgfqpoint{1.384431in}{1.940836in}}%
\pgfpathcurveto{\pgfqpoint{1.392245in}{1.948650in}}{\pgfqpoint{1.396635in}{1.959249in}}{\pgfqpoint{1.396635in}{1.970299in}}%
\pgfpathcurveto{\pgfqpoint{1.396635in}{1.981349in}}{\pgfqpoint{1.392245in}{1.991948in}}{\pgfqpoint{1.384431in}{1.999762in}}%
\pgfpathcurveto{\pgfqpoint{1.376617in}{2.007575in}}{\pgfqpoint{1.366018in}{2.011966in}}{\pgfqpoint{1.354968in}{2.011966in}}%
\pgfpathcurveto{\pgfqpoint{1.343918in}{2.011966in}}{\pgfqpoint{1.333319in}{2.007575in}}{\pgfqpoint{1.325505in}{1.999762in}}%
\pgfpathcurveto{\pgfqpoint{1.317692in}{1.991948in}}{\pgfqpoint{1.313302in}{1.981349in}}{\pgfqpoint{1.313302in}{1.970299in}}%
\pgfpathcurveto{\pgfqpoint{1.313302in}{1.959249in}}{\pgfqpoint{1.317692in}{1.948650in}}{\pgfqpoint{1.325505in}{1.940836in}}%
\pgfpathcurveto{\pgfqpoint{1.333319in}{1.933022in}}{\pgfqpoint{1.343918in}{1.928632in}}{\pgfqpoint{1.354968in}{1.928632in}}%
\pgfpathlineto{\pgfqpoint{1.354968in}{1.928632in}}%
\pgfpathclose%
\pgfusepath{stroke}%
\end{pgfscope}%
\begin{pgfscope}%
\pgfpathrectangle{\pgfqpoint{0.393053in}{0.375000in}}{\pgfqpoint{6.356833in}{5.175000in}}%
\pgfusepath{clip}%
\pgfsetbuttcap%
\pgfsetroundjoin%
\pgfsetlinewidth{1.003750pt}%
\definecolor{currentstroke}{rgb}{0.827451,0.827451,0.827451}%
\pgfsetstrokecolor{currentstroke}%
\pgfsetdash{}{0pt}%
\pgfpathmoveto{\pgfqpoint{1.068566in}{2.310306in}}%
\pgfpathcurveto{\pgfqpoint{1.079616in}{2.310306in}}{\pgfqpoint{1.090215in}{2.314696in}}{\pgfqpoint{1.098029in}{2.322510in}}%
\pgfpathcurveto{\pgfqpoint{1.105843in}{2.330323in}}{\pgfqpoint{1.110233in}{2.340922in}}{\pgfqpoint{1.110233in}{2.351973in}}%
\pgfpathcurveto{\pgfqpoint{1.110233in}{2.363023in}}{\pgfqpoint{1.105843in}{2.373622in}}{\pgfqpoint{1.098029in}{2.381435in}}%
\pgfpathcurveto{\pgfqpoint{1.090215in}{2.389249in}}{\pgfqpoint{1.079616in}{2.393639in}}{\pgfqpoint{1.068566in}{2.393639in}}%
\pgfpathcurveto{\pgfqpoint{1.057516in}{2.393639in}}{\pgfqpoint{1.046917in}{2.389249in}}{\pgfqpoint{1.039103in}{2.381435in}}%
\pgfpathcurveto{\pgfqpoint{1.031290in}{2.373622in}}{\pgfqpoint{1.026900in}{2.363023in}}{\pgfqpoint{1.026900in}{2.351973in}}%
\pgfpathcurveto{\pgfqpoint{1.026900in}{2.340922in}}{\pgfqpoint{1.031290in}{2.330323in}}{\pgfqpoint{1.039103in}{2.322510in}}%
\pgfpathcurveto{\pgfqpoint{1.046917in}{2.314696in}}{\pgfqpoint{1.057516in}{2.310306in}}{\pgfqpoint{1.068566in}{2.310306in}}%
\pgfpathlineto{\pgfqpoint{1.068566in}{2.310306in}}%
\pgfpathclose%
\pgfusepath{stroke}%
\end{pgfscope}%
\begin{pgfscope}%
\pgfpathrectangle{\pgfqpoint{0.393053in}{0.375000in}}{\pgfqpoint{6.356833in}{5.175000in}}%
\pgfusepath{clip}%
\pgfsetbuttcap%
\pgfsetroundjoin%
\pgfsetlinewidth{1.003750pt}%
\definecolor{currentstroke}{rgb}{0.827451,0.827451,0.827451}%
\pgfsetstrokecolor{currentstroke}%
\pgfsetdash{}{0pt}%
\pgfpathmoveto{\pgfqpoint{2.713290in}{0.943005in}}%
\pgfpathcurveto{\pgfqpoint{2.724341in}{0.943005in}}{\pgfqpoint{2.734940in}{0.947396in}}{\pgfqpoint{2.742753in}{0.955209in}}%
\pgfpathcurveto{\pgfqpoint{2.750567in}{0.963023in}}{\pgfqpoint{2.754957in}{0.973622in}}{\pgfqpoint{2.754957in}{0.984672in}}%
\pgfpathcurveto{\pgfqpoint{2.754957in}{0.995722in}}{\pgfqpoint{2.750567in}{1.006321in}}{\pgfqpoint{2.742753in}{1.014135in}}%
\pgfpathcurveto{\pgfqpoint{2.734940in}{1.021949in}}{\pgfqpoint{2.724341in}{1.026339in}}{\pgfqpoint{2.713290in}{1.026339in}}%
\pgfpathcurveto{\pgfqpoint{2.702240in}{1.026339in}}{\pgfqpoint{2.691641in}{1.021949in}}{\pgfqpoint{2.683828in}{1.014135in}}%
\pgfpathcurveto{\pgfqpoint{2.676014in}{1.006321in}}{\pgfqpoint{2.671624in}{0.995722in}}{\pgfqpoint{2.671624in}{0.984672in}}%
\pgfpathcurveto{\pgfqpoint{2.671624in}{0.973622in}}{\pgfqpoint{2.676014in}{0.963023in}}{\pgfqpoint{2.683828in}{0.955209in}}%
\pgfpathcurveto{\pgfqpoint{2.691641in}{0.947396in}}{\pgfqpoint{2.702240in}{0.943005in}}{\pgfqpoint{2.713290in}{0.943005in}}%
\pgfpathlineto{\pgfqpoint{2.713290in}{0.943005in}}%
\pgfpathclose%
\pgfusepath{stroke}%
\end{pgfscope}%
\begin{pgfscope}%
\pgfpathrectangle{\pgfqpoint{0.393053in}{0.375000in}}{\pgfqpoint{6.356833in}{5.175000in}}%
\pgfusepath{clip}%
\pgfsetbuttcap%
\pgfsetroundjoin%
\pgfsetlinewidth{1.003750pt}%
\definecolor{currentstroke}{rgb}{0.827451,0.827451,0.827451}%
\pgfsetstrokecolor{currentstroke}%
\pgfsetdash{}{0pt}%
\pgfpathmoveto{\pgfqpoint{2.000452in}{1.370955in}}%
\pgfpathcurveto{\pgfqpoint{2.011502in}{1.370955in}}{\pgfqpoint{2.022101in}{1.375345in}}{\pgfqpoint{2.029914in}{1.383159in}}%
\pgfpathcurveto{\pgfqpoint{2.037728in}{1.390972in}}{\pgfqpoint{2.042118in}{1.401571in}}{\pgfqpoint{2.042118in}{1.412622in}}%
\pgfpathcurveto{\pgfqpoint{2.042118in}{1.423672in}}{\pgfqpoint{2.037728in}{1.434271in}}{\pgfqpoint{2.029914in}{1.442084in}}%
\pgfpathcurveto{\pgfqpoint{2.022101in}{1.449898in}}{\pgfqpoint{2.011502in}{1.454288in}}{\pgfqpoint{2.000452in}{1.454288in}}%
\pgfpathcurveto{\pgfqpoint{1.989401in}{1.454288in}}{\pgfqpoint{1.978802in}{1.449898in}}{\pgfqpoint{1.970989in}{1.442084in}}%
\pgfpathcurveto{\pgfqpoint{1.963175in}{1.434271in}}{\pgfqpoint{1.958785in}{1.423672in}}{\pgfqpoint{1.958785in}{1.412622in}}%
\pgfpathcurveto{\pgfqpoint{1.958785in}{1.401571in}}{\pgfqpoint{1.963175in}{1.390972in}}{\pgfqpoint{1.970989in}{1.383159in}}%
\pgfpathcurveto{\pgfqpoint{1.978802in}{1.375345in}}{\pgfqpoint{1.989401in}{1.370955in}}{\pgfqpoint{2.000452in}{1.370955in}}%
\pgfpathlineto{\pgfqpoint{2.000452in}{1.370955in}}%
\pgfpathclose%
\pgfusepath{stroke}%
\end{pgfscope}%
\begin{pgfscope}%
\pgfpathrectangle{\pgfqpoint{0.393053in}{0.375000in}}{\pgfqpoint{6.356833in}{5.175000in}}%
\pgfusepath{clip}%
\pgfsetbuttcap%
\pgfsetroundjoin%
\pgfsetlinewidth{1.003750pt}%
\definecolor{currentstroke}{rgb}{0.827451,0.827451,0.827451}%
\pgfsetstrokecolor{currentstroke}%
\pgfsetdash{}{0pt}%
\pgfpathmoveto{\pgfqpoint{3.065603in}{0.780037in}}%
\pgfpathcurveto{\pgfqpoint{3.076653in}{0.780037in}}{\pgfqpoint{3.087252in}{0.784427in}}{\pgfqpoint{3.095066in}{0.792240in}}%
\pgfpathcurveto{\pgfqpoint{3.102880in}{0.800054in}}{\pgfqpoint{3.107270in}{0.810653in}}{\pgfqpoint{3.107270in}{0.821703in}}%
\pgfpathcurveto{\pgfqpoint{3.107270in}{0.832753in}}{\pgfqpoint{3.102880in}{0.843352in}}{\pgfqpoint{3.095066in}{0.851166in}}%
\pgfpathcurveto{\pgfqpoint{3.087252in}{0.858980in}}{\pgfqpoint{3.076653in}{0.863370in}}{\pgfqpoint{3.065603in}{0.863370in}}%
\pgfpathcurveto{\pgfqpoint{3.054553in}{0.863370in}}{\pgfqpoint{3.043954in}{0.858980in}}{\pgfqpoint{3.036140in}{0.851166in}}%
\pgfpathcurveto{\pgfqpoint{3.028327in}{0.843352in}}{\pgfqpoint{3.023937in}{0.832753in}}{\pgfqpoint{3.023937in}{0.821703in}}%
\pgfpathcurveto{\pgfqpoint{3.023937in}{0.810653in}}{\pgfqpoint{3.028327in}{0.800054in}}{\pgfqpoint{3.036140in}{0.792240in}}%
\pgfpathcurveto{\pgfqpoint{3.043954in}{0.784427in}}{\pgfqpoint{3.054553in}{0.780037in}}{\pgfqpoint{3.065603in}{0.780037in}}%
\pgfpathlineto{\pgfqpoint{3.065603in}{0.780037in}}%
\pgfpathclose%
\pgfusepath{stroke}%
\end{pgfscope}%
\begin{pgfscope}%
\pgfpathrectangle{\pgfqpoint{0.393053in}{0.375000in}}{\pgfqpoint{6.356833in}{5.175000in}}%
\pgfusepath{clip}%
\pgfsetbuttcap%
\pgfsetroundjoin%
\pgfsetlinewidth{1.003750pt}%
\definecolor{currentstroke}{rgb}{0.827451,0.827451,0.827451}%
\pgfsetstrokecolor{currentstroke}%
\pgfsetdash{}{0pt}%
\pgfpathmoveto{\pgfqpoint{0.399634in}{4.323598in}}%
\pgfpathcurveto{\pgfqpoint{0.410684in}{4.323598in}}{\pgfqpoint{0.421283in}{4.327988in}}{\pgfqpoint{0.429097in}{4.335802in}}%
\pgfpathcurveto{\pgfqpoint{0.436910in}{4.343616in}}{\pgfqpoint{0.441301in}{4.354215in}}{\pgfqpoint{0.441301in}{4.365265in}}%
\pgfpathcurveto{\pgfqpoint{0.441301in}{4.376315in}}{\pgfqpoint{0.436910in}{4.386914in}}{\pgfqpoint{0.429097in}{4.394727in}}%
\pgfpathcurveto{\pgfqpoint{0.421283in}{4.402541in}}{\pgfqpoint{0.410684in}{4.406931in}}{\pgfqpoint{0.399634in}{4.406931in}}%
\pgfpathcurveto{\pgfqpoint{0.388584in}{4.406931in}}{\pgfqpoint{0.377985in}{4.402541in}}{\pgfqpoint{0.370171in}{4.394727in}}%
\pgfpathcurveto{\pgfqpoint{0.362357in}{4.386914in}}{\pgfqpoint{0.357967in}{4.376315in}}{\pgfqpoint{0.357967in}{4.365265in}}%
\pgfpathcurveto{\pgfqpoint{0.357967in}{4.354215in}}{\pgfqpoint{0.362357in}{4.343616in}}{\pgfqpoint{0.370171in}{4.335802in}}%
\pgfpathcurveto{\pgfqpoint{0.377985in}{4.327988in}}{\pgfqpoint{0.388584in}{4.323598in}}{\pgfqpoint{0.399634in}{4.323598in}}%
\pgfpathlineto{\pgfqpoint{0.399634in}{4.323598in}}%
\pgfpathclose%
\pgfusepath{stroke}%
\end{pgfscope}%
\begin{pgfscope}%
\pgfpathrectangle{\pgfqpoint{0.393053in}{0.375000in}}{\pgfqpoint{6.356833in}{5.175000in}}%
\pgfusepath{clip}%
\pgfsetbuttcap%
\pgfsetroundjoin%
\pgfsetlinewidth{1.003750pt}%
\definecolor{currentstroke}{rgb}{0.827451,0.827451,0.827451}%
\pgfsetstrokecolor{currentstroke}%
\pgfsetdash{}{0pt}%
\pgfpathmoveto{\pgfqpoint{0.393825in}{4.495013in}}%
\pgfpathcurveto{\pgfqpoint{0.404875in}{4.495013in}}{\pgfqpoint{0.415474in}{4.499403in}}{\pgfqpoint{0.423288in}{4.507217in}}%
\pgfpathcurveto{\pgfqpoint{0.431101in}{4.515030in}}{\pgfqpoint{0.435492in}{4.525629in}}{\pgfqpoint{0.435492in}{4.536679in}}%
\pgfpathcurveto{\pgfqpoint{0.435492in}{4.547730in}}{\pgfqpoint{0.431101in}{4.558329in}}{\pgfqpoint{0.423288in}{4.566142in}}%
\pgfpathcurveto{\pgfqpoint{0.415474in}{4.573956in}}{\pgfqpoint{0.404875in}{4.578346in}}{\pgfqpoint{0.393825in}{4.578346in}}%
\pgfpathcurveto{\pgfqpoint{0.382775in}{4.578346in}}{\pgfqpoint{0.372176in}{4.573956in}}{\pgfqpoint{0.364362in}{4.566142in}}%
\pgfpathcurveto{\pgfqpoint{0.356548in}{4.558329in}}{\pgfqpoint{0.352158in}{4.547730in}}{\pgfqpoint{0.352158in}{4.536679in}}%
\pgfpathcurveto{\pgfqpoint{0.352158in}{4.525629in}}{\pgfqpoint{0.356548in}{4.515030in}}{\pgfqpoint{0.364362in}{4.507217in}}%
\pgfpathcurveto{\pgfqpoint{0.372176in}{4.499403in}}{\pgfqpoint{0.382775in}{4.495013in}}{\pgfqpoint{0.393825in}{4.495013in}}%
\pgfpathlineto{\pgfqpoint{0.393825in}{4.495013in}}%
\pgfpathclose%
\pgfusepath{stroke}%
\end{pgfscope}%
\begin{pgfscope}%
\pgfpathrectangle{\pgfqpoint{0.393053in}{0.375000in}}{\pgfqpoint{6.356833in}{5.175000in}}%
\pgfusepath{clip}%
\pgfsetbuttcap%
\pgfsetroundjoin%
\pgfsetlinewidth{1.003750pt}%
\definecolor{currentstroke}{rgb}{0.827451,0.827451,0.827451}%
\pgfsetstrokecolor{currentstroke}%
\pgfsetdash{}{0pt}%
\pgfpathmoveto{\pgfqpoint{3.308571in}{0.683593in}}%
\pgfpathcurveto{\pgfqpoint{3.319621in}{0.683593in}}{\pgfqpoint{3.330220in}{0.687984in}}{\pgfqpoint{3.338034in}{0.695797in}}%
\pgfpathcurveto{\pgfqpoint{3.345848in}{0.703611in}}{\pgfqpoint{3.350238in}{0.714210in}}{\pgfqpoint{3.350238in}{0.725260in}}%
\pgfpathcurveto{\pgfqpoint{3.350238in}{0.736310in}}{\pgfqpoint{3.345848in}{0.746909in}}{\pgfqpoint{3.338034in}{0.754723in}}%
\pgfpathcurveto{\pgfqpoint{3.330220in}{0.762537in}}{\pgfqpoint{3.319621in}{0.766927in}}{\pgfqpoint{3.308571in}{0.766927in}}%
\pgfpathcurveto{\pgfqpoint{3.297521in}{0.766927in}}{\pgfqpoint{3.286922in}{0.762537in}}{\pgfqpoint{3.279108in}{0.754723in}}%
\pgfpathcurveto{\pgfqpoint{3.271295in}{0.746909in}}{\pgfqpoint{3.266904in}{0.736310in}}{\pgfqpoint{3.266904in}{0.725260in}}%
\pgfpathcurveto{\pgfqpoint{3.266904in}{0.714210in}}{\pgfqpoint{3.271295in}{0.703611in}}{\pgfqpoint{3.279108in}{0.695797in}}%
\pgfpathcurveto{\pgfqpoint{3.286922in}{0.687984in}}{\pgfqpoint{3.297521in}{0.683593in}}{\pgfqpoint{3.308571in}{0.683593in}}%
\pgfpathlineto{\pgfqpoint{3.308571in}{0.683593in}}%
\pgfpathclose%
\pgfusepath{stroke}%
\end{pgfscope}%
\begin{pgfscope}%
\pgfpathrectangle{\pgfqpoint{0.393053in}{0.375000in}}{\pgfqpoint{6.356833in}{5.175000in}}%
\pgfusepath{clip}%
\pgfsetbuttcap%
\pgfsetroundjoin%
\pgfsetlinewidth{1.003750pt}%
\definecolor{currentstroke}{rgb}{0.827451,0.827451,0.827451}%
\pgfsetstrokecolor{currentstroke}%
\pgfsetdash{}{0pt}%
\pgfpathmoveto{\pgfqpoint{2.414235in}{1.111502in}}%
\pgfpathcurveto{\pgfqpoint{2.425285in}{1.111502in}}{\pgfqpoint{2.435884in}{1.115892in}}{\pgfqpoint{2.443698in}{1.123706in}}%
\pgfpathcurveto{\pgfqpoint{2.451511in}{1.131519in}}{\pgfqpoint{2.455902in}{1.142118in}}{\pgfqpoint{2.455902in}{1.153169in}}%
\pgfpathcurveto{\pgfqpoint{2.455902in}{1.164219in}}{\pgfqpoint{2.451511in}{1.174818in}}{\pgfqpoint{2.443698in}{1.182631in}}%
\pgfpathcurveto{\pgfqpoint{2.435884in}{1.190445in}}{\pgfqpoint{2.425285in}{1.194835in}}{\pgfqpoint{2.414235in}{1.194835in}}%
\pgfpathcurveto{\pgfqpoint{2.403185in}{1.194835in}}{\pgfqpoint{2.392586in}{1.190445in}}{\pgfqpoint{2.384772in}{1.182631in}}%
\pgfpathcurveto{\pgfqpoint{2.376959in}{1.174818in}}{\pgfqpoint{2.372568in}{1.164219in}}{\pgfqpoint{2.372568in}{1.153169in}}%
\pgfpathcurveto{\pgfqpoint{2.372568in}{1.142118in}}{\pgfqpoint{2.376959in}{1.131519in}}{\pgfqpoint{2.384772in}{1.123706in}}%
\pgfpathcurveto{\pgfqpoint{2.392586in}{1.115892in}}{\pgfqpoint{2.403185in}{1.111502in}}{\pgfqpoint{2.414235in}{1.111502in}}%
\pgfpathlineto{\pgfqpoint{2.414235in}{1.111502in}}%
\pgfpathclose%
\pgfusepath{stroke}%
\end{pgfscope}%
\begin{pgfscope}%
\pgfpathrectangle{\pgfqpoint{0.393053in}{0.375000in}}{\pgfqpoint{6.356833in}{5.175000in}}%
\pgfusepath{clip}%
\pgfsetbuttcap%
\pgfsetroundjoin%
\pgfsetlinewidth{1.003750pt}%
\definecolor{currentstroke}{rgb}{0.827451,0.827451,0.827451}%
\pgfsetstrokecolor{currentstroke}%
\pgfsetdash{}{0pt}%
\pgfpathmoveto{\pgfqpoint{0.601770in}{3.214331in}}%
\pgfpathcurveto{\pgfqpoint{0.612820in}{3.214331in}}{\pgfqpoint{0.623419in}{3.218722in}}{\pgfqpoint{0.631233in}{3.226535in}}%
\pgfpathcurveto{\pgfqpoint{0.639046in}{3.234349in}}{\pgfqpoint{0.643437in}{3.244948in}}{\pgfqpoint{0.643437in}{3.255998in}}%
\pgfpathcurveto{\pgfqpoint{0.643437in}{3.267048in}}{\pgfqpoint{0.639046in}{3.277647in}}{\pgfqpoint{0.631233in}{3.285461in}}%
\pgfpathcurveto{\pgfqpoint{0.623419in}{3.293275in}}{\pgfqpoint{0.612820in}{3.297665in}}{\pgfqpoint{0.601770in}{3.297665in}}%
\pgfpathcurveto{\pgfqpoint{0.590720in}{3.297665in}}{\pgfqpoint{0.580121in}{3.293275in}}{\pgfqpoint{0.572307in}{3.285461in}}%
\pgfpathcurveto{\pgfqpoint{0.564494in}{3.277647in}}{\pgfqpoint{0.560103in}{3.267048in}}{\pgfqpoint{0.560103in}{3.255998in}}%
\pgfpathcurveto{\pgfqpoint{0.560103in}{3.244948in}}{\pgfqpoint{0.564494in}{3.234349in}}{\pgfqpoint{0.572307in}{3.226535in}}%
\pgfpathcurveto{\pgfqpoint{0.580121in}{3.218722in}}{\pgfqpoint{0.590720in}{3.214331in}}{\pgfqpoint{0.601770in}{3.214331in}}%
\pgfpathlineto{\pgfqpoint{0.601770in}{3.214331in}}%
\pgfpathclose%
\pgfusepath{stroke}%
\end{pgfscope}%
\begin{pgfscope}%
\pgfpathrectangle{\pgfqpoint{0.393053in}{0.375000in}}{\pgfqpoint{6.356833in}{5.175000in}}%
\pgfusepath{clip}%
\pgfsetbuttcap%
\pgfsetroundjoin%
\pgfsetlinewidth{1.003750pt}%
\definecolor{currentstroke}{rgb}{0.827451,0.827451,0.827451}%
\pgfsetstrokecolor{currentstroke}%
\pgfsetdash{}{0pt}%
\pgfpathmoveto{\pgfqpoint{2.584660in}{1.053636in}}%
\pgfpathcurveto{\pgfqpoint{2.595711in}{1.053636in}}{\pgfqpoint{2.606310in}{1.058027in}}{\pgfqpoint{2.614123in}{1.065840in}}%
\pgfpathcurveto{\pgfqpoint{2.621937in}{1.073654in}}{\pgfqpoint{2.626327in}{1.084253in}}{\pgfqpoint{2.626327in}{1.095303in}}%
\pgfpathcurveto{\pgfqpoint{2.626327in}{1.106353in}}{\pgfqpoint{2.621937in}{1.116952in}}{\pgfqpoint{2.614123in}{1.124766in}}%
\pgfpathcurveto{\pgfqpoint{2.606310in}{1.132580in}}{\pgfqpoint{2.595711in}{1.136970in}}{\pgfqpoint{2.584660in}{1.136970in}}%
\pgfpathcurveto{\pgfqpoint{2.573610in}{1.136970in}}{\pgfqpoint{2.563011in}{1.132580in}}{\pgfqpoint{2.555198in}{1.124766in}}%
\pgfpathcurveto{\pgfqpoint{2.547384in}{1.116952in}}{\pgfqpoint{2.542994in}{1.106353in}}{\pgfqpoint{2.542994in}{1.095303in}}%
\pgfpathcurveto{\pgfqpoint{2.542994in}{1.084253in}}{\pgfqpoint{2.547384in}{1.073654in}}{\pgfqpoint{2.555198in}{1.065840in}}%
\pgfpathcurveto{\pgfqpoint{2.563011in}{1.058027in}}{\pgfqpoint{2.573610in}{1.053636in}}{\pgfqpoint{2.584660in}{1.053636in}}%
\pgfpathlineto{\pgfqpoint{2.584660in}{1.053636in}}%
\pgfpathclose%
\pgfusepath{stroke}%
\end{pgfscope}%
\begin{pgfscope}%
\pgfpathrectangle{\pgfqpoint{0.393053in}{0.375000in}}{\pgfqpoint{6.356833in}{5.175000in}}%
\pgfusepath{clip}%
\pgfsetbuttcap%
\pgfsetroundjoin%
\pgfsetlinewidth{1.003750pt}%
\definecolor{currentstroke}{rgb}{0.827451,0.827451,0.827451}%
\pgfsetstrokecolor{currentstroke}%
\pgfsetdash{}{0pt}%
\pgfpathmoveto{\pgfqpoint{0.580834in}{3.297265in}}%
\pgfpathcurveto{\pgfqpoint{0.591884in}{3.297265in}}{\pgfqpoint{0.602483in}{3.301655in}}{\pgfqpoint{0.610297in}{3.309469in}}%
\pgfpathcurveto{\pgfqpoint{0.618111in}{3.317282in}}{\pgfqpoint{0.622501in}{3.327881in}}{\pgfqpoint{0.622501in}{3.338932in}}%
\pgfpathcurveto{\pgfqpoint{0.622501in}{3.349982in}}{\pgfqpoint{0.618111in}{3.360581in}}{\pgfqpoint{0.610297in}{3.368394in}}%
\pgfpathcurveto{\pgfqpoint{0.602483in}{3.376208in}}{\pgfqpoint{0.591884in}{3.380598in}}{\pgfqpoint{0.580834in}{3.380598in}}%
\pgfpathcurveto{\pgfqpoint{0.569784in}{3.380598in}}{\pgfqpoint{0.559185in}{3.376208in}}{\pgfqpoint{0.551371in}{3.368394in}}%
\pgfpathcurveto{\pgfqpoint{0.543558in}{3.360581in}}{\pgfqpoint{0.539167in}{3.349982in}}{\pgfqpoint{0.539167in}{3.338932in}}%
\pgfpathcurveto{\pgfqpoint{0.539167in}{3.327881in}}{\pgfqpoint{0.543558in}{3.317282in}}{\pgfqpoint{0.551371in}{3.309469in}}%
\pgfpathcurveto{\pgfqpoint{0.559185in}{3.301655in}}{\pgfqpoint{0.569784in}{3.297265in}}{\pgfqpoint{0.580834in}{3.297265in}}%
\pgfpathlineto{\pgfqpoint{0.580834in}{3.297265in}}%
\pgfpathclose%
\pgfusepath{stroke}%
\end{pgfscope}%
\begin{pgfscope}%
\pgfpathrectangle{\pgfqpoint{0.393053in}{0.375000in}}{\pgfqpoint{6.356833in}{5.175000in}}%
\pgfusepath{clip}%
\pgfsetbuttcap%
\pgfsetroundjoin%
\pgfsetlinewidth{1.003750pt}%
\definecolor{currentstroke}{rgb}{0.827451,0.827451,0.827451}%
\pgfsetstrokecolor{currentstroke}%
\pgfsetdash{}{0pt}%
\pgfpathmoveto{\pgfqpoint{5.142473in}{0.360415in}}%
\pgfpathcurveto{\pgfqpoint{5.153523in}{0.360415in}}{\pgfqpoint{5.164122in}{0.364805in}}{\pgfqpoint{5.171936in}{0.372618in}}%
\pgfpathcurveto{\pgfqpoint{5.179749in}{0.380432in}}{\pgfqpoint{5.184140in}{0.391031in}}{\pgfqpoint{5.184140in}{0.402081in}}%
\pgfpathcurveto{\pgfqpoint{5.184140in}{0.413131in}}{\pgfqpoint{5.179749in}{0.423730in}}{\pgfqpoint{5.171936in}{0.431544in}}%
\pgfpathcurveto{\pgfqpoint{5.164122in}{0.439358in}}{\pgfqpoint{5.153523in}{0.443748in}}{\pgfqpoint{5.142473in}{0.443748in}}%
\pgfpathcurveto{\pgfqpoint{5.131423in}{0.443748in}}{\pgfqpoint{5.120824in}{0.439358in}}{\pgfqpoint{5.113010in}{0.431544in}}%
\pgfpathcurveto{\pgfqpoint{5.105196in}{0.423730in}}{\pgfqpoint{5.100806in}{0.413131in}}{\pgfqpoint{5.100806in}{0.402081in}}%
\pgfpathcurveto{\pgfqpoint{5.100806in}{0.391031in}}{\pgfqpoint{5.105196in}{0.380432in}}{\pgfqpoint{5.113010in}{0.372618in}}%
\pgfpathcurveto{\pgfqpoint{5.120824in}{0.364805in}}{\pgfqpoint{5.131423in}{0.360415in}}{\pgfqpoint{5.142473in}{0.360415in}}%
\pgfusepath{stroke}%
\end{pgfscope}%
\begin{pgfscope}%
\pgfpathrectangle{\pgfqpoint{0.393053in}{0.375000in}}{\pgfqpoint{6.356833in}{5.175000in}}%
\pgfusepath{clip}%
\pgfsetbuttcap%
\pgfsetroundjoin%
\pgfsetlinewidth{1.003750pt}%
\definecolor{currentstroke}{rgb}{0.827451,0.827451,0.827451}%
\pgfsetstrokecolor{currentstroke}%
\pgfsetdash{}{0pt}%
\pgfpathmoveto{\pgfqpoint{1.291595in}{1.999205in}}%
\pgfpathcurveto{\pgfqpoint{1.302646in}{1.999205in}}{\pgfqpoint{1.313245in}{2.003595in}}{\pgfqpoint{1.321058in}{2.011409in}}%
\pgfpathcurveto{\pgfqpoint{1.328872in}{2.019222in}}{\pgfqpoint{1.333262in}{2.029821in}}{\pgfqpoint{1.333262in}{2.040872in}}%
\pgfpathcurveto{\pgfqpoint{1.333262in}{2.051922in}}{\pgfqpoint{1.328872in}{2.062521in}}{\pgfqpoint{1.321058in}{2.070334in}}%
\pgfpathcurveto{\pgfqpoint{1.313245in}{2.078148in}}{\pgfqpoint{1.302646in}{2.082538in}}{\pgfqpoint{1.291595in}{2.082538in}}%
\pgfpathcurveto{\pgfqpoint{1.280545in}{2.082538in}}{\pgfqpoint{1.269946in}{2.078148in}}{\pgfqpoint{1.262133in}{2.070334in}}%
\pgfpathcurveto{\pgfqpoint{1.254319in}{2.062521in}}{\pgfqpoint{1.249929in}{2.051922in}}{\pgfqpoint{1.249929in}{2.040872in}}%
\pgfpathcurveto{\pgfqpoint{1.249929in}{2.029821in}}{\pgfqpoint{1.254319in}{2.019222in}}{\pgfqpoint{1.262133in}{2.011409in}}%
\pgfpathcurveto{\pgfqpoint{1.269946in}{2.003595in}}{\pgfqpoint{1.280545in}{1.999205in}}{\pgfqpoint{1.291595in}{1.999205in}}%
\pgfpathlineto{\pgfqpoint{1.291595in}{1.999205in}}%
\pgfpathclose%
\pgfusepath{stroke}%
\end{pgfscope}%
\begin{pgfscope}%
\pgfpathrectangle{\pgfqpoint{0.393053in}{0.375000in}}{\pgfqpoint{6.356833in}{5.175000in}}%
\pgfusepath{clip}%
\pgfsetbuttcap%
\pgfsetroundjoin%
\pgfsetlinewidth{1.003750pt}%
\definecolor{currentstroke}{rgb}{0.827451,0.827451,0.827451}%
\pgfsetstrokecolor{currentstroke}%
\pgfsetdash{}{0pt}%
\pgfpathmoveto{\pgfqpoint{2.230001in}{1.209659in}}%
\pgfpathcurveto{\pgfqpoint{2.241051in}{1.209659in}}{\pgfqpoint{2.251650in}{1.214049in}}{\pgfqpoint{2.259463in}{1.221862in}}%
\pgfpathcurveto{\pgfqpoint{2.267277in}{1.229676in}}{\pgfqpoint{2.271667in}{1.240275in}}{\pgfqpoint{2.271667in}{1.251325in}}%
\pgfpathcurveto{\pgfqpoint{2.271667in}{1.262375in}}{\pgfqpoint{2.267277in}{1.272974in}}{\pgfqpoint{2.259463in}{1.280788in}}%
\pgfpathcurveto{\pgfqpoint{2.251650in}{1.288602in}}{\pgfqpoint{2.241051in}{1.292992in}}{\pgfqpoint{2.230001in}{1.292992in}}%
\pgfpathcurveto{\pgfqpoint{2.218950in}{1.292992in}}{\pgfqpoint{2.208351in}{1.288602in}}{\pgfqpoint{2.200538in}{1.280788in}}%
\pgfpathcurveto{\pgfqpoint{2.192724in}{1.272974in}}{\pgfqpoint{2.188334in}{1.262375in}}{\pgfqpoint{2.188334in}{1.251325in}}%
\pgfpathcurveto{\pgfqpoint{2.188334in}{1.240275in}}{\pgfqpoint{2.192724in}{1.229676in}}{\pgfqpoint{2.200538in}{1.221862in}}%
\pgfpathcurveto{\pgfqpoint{2.208351in}{1.214049in}}{\pgfqpoint{2.218950in}{1.209659in}}{\pgfqpoint{2.230001in}{1.209659in}}%
\pgfpathlineto{\pgfqpoint{2.230001in}{1.209659in}}%
\pgfpathclose%
\pgfusepath{stroke}%
\end{pgfscope}%
\begin{pgfscope}%
\pgfpathrectangle{\pgfqpoint{0.393053in}{0.375000in}}{\pgfqpoint{6.356833in}{5.175000in}}%
\pgfusepath{clip}%
\pgfsetbuttcap%
\pgfsetroundjoin%
\pgfsetlinewidth{1.003750pt}%
\definecolor{currentstroke}{rgb}{0.827451,0.827451,0.827451}%
\pgfsetstrokecolor{currentstroke}%
\pgfsetdash{}{0pt}%
\pgfpathmoveto{\pgfqpoint{5.335363in}{0.346733in}}%
\pgfpathcurveto{\pgfqpoint{5.346413in}{0.346733in}}{\pgfqpoint{5.357012in}{0.351123in}}{\pgfqpoint{5.364826in}{0.358937in}}%
\pgfpathcurveto{\pgfqpoint{5.372639in}{0.366750in}}{\pgfqpoint{5.377030in}{0.377349in}}{\pgfqpoint{5.377030in}{0.388400in}}%
\pgfpathcurveto{\pgfqpoint{5.377030in}{0.399450in}}{\pgfqpoint{5.372639in}{0.410049in}}{\pgfqpoint{5.364826in}{0.417862in}}%
\pgfpathcurveto{\pgfqpoint{5.357012in}{0.425676in}}{\pgfqpoint{5.346413in}{0.430066in}}{\pgfqpoint{5.335363in}{0.430066in}}%
\pgfpathcurveto{\pgfqpoint{5.324313in}{0.430066in}}{\pgfqpoint{5.313714in}{0.425676in}}{\pgfqpoint{5.305900in}{0.417862in}}%
\pgfpathcurveto{\pgfqpoint{5.298086in}{0.410049in}}{\pgfqpoint{5.293696in}{0.399450in}}{\pgfqpoint{5.293696in}{0.388400in}}%
\pgfpathcurveto{\pgfqpoint{5.293696in}{0.377349in}}{\pgfqpoint{5.298086in}{0.366750in}}{\pgfqpoint{5.305900in}{0.358937in}}%
\pgfpathcurveto{\pgfqpoint{5.313714in}{0.351123in}}{\pgfqpoint{5.324313in}{0.346733in}}{\pgfqpoint{5.335363in}{0.346733in}}%
\pgfusepath{stroke}%
\end{pgfscope}%
\begin{pgfscope}%
\pgfpathrectangle{\pgfqpoint{0.393053in}{0.375000in}}{\pgfqpoint{6.356833in}{5.175000in}}%
\pgfusepath{clip}%
\pgfsetbuttcap%
\pgfsetroundjoin%
\pgfsetlinewidth{1.003750pt}%
\definecolor{currentstroke}{rgb}{0.827451,0.827451,0.827451}%
\pgfsetstrokecolor{currentstroke}%
\pgfsetdash{}{0pt}%
\pgfpathmoveto{\pgfqpoint{2.477308in}{1.087549in}}%
\pgfpathcurveto{\pgfqpoint{2.488358in}{1.087549in}}{\pgfqpoint{2.498957in}{1.091940in}}{\pgfqpoint{2.506771in}{1.099753in}}%
\pgfpathcurveto{\pgfqpoint{2.514584in}{1.107567in}}{\pgfqpoint{2.518974in}{1.118166in}}{\pgfqpoint{2.518974in}{1.129216in}}%
\pgfpathcurveto{\pgfqpoint{2.518974in}{1.140266in}}{\pgfqpoint{2.514584in}{1.150865in}}{\pgfqpoint{2.506771in}{1.158679in}}%
\pgfpathcurveto{\pgfqpoint{2.498957in}{1.166492in}}{\pgfqpoint{2.488358in}{1.170883in}}{\pgfqpoint{2.477308in}{1.170883in}}%
\pgfpathcurveto{\pgfqpoint{2.466258in}{1.170883in}}{\pgfqpoint{2.455659in}{1.166492in}}{\pgfqpoint{2.447845in}{1.158679in}}%
\pgfpathcurveto{\pgfqpoint{2.440031in}{1.150865in}}{\pgfqpoint{2.435641in}{1.140266in}}{\pgfqpoint{2.435641in}{1.129216in}}%
\pgfpathcurveto{\pgfqpoint{2.435641in}{1.118166in}}{\pgfqpoint{2.440031in}{1.107567in}}{\pgfqpoint{2.447845in}{1.099753in}}%
\pgfpathcurveto{\pgfqpoint{2.455659in}{1.091940in}}{\pgfqpoint{2.466258in}{1.087549in}}{\pgfqpoint{2.477308in}{1.087549in}}%
\pgfpathlineto{\pgfqpoint{2.477308in}{1.087549in}}%
\pgfpathclose%
\pgfusepath{stroke}%
\end{pgfscope}%
\begin{pgfscope}%
\pgfpathrectangle{\pgfqpoint{0.393053in}{0.375000in}}{\pgfqpoint{6.356833in}{5.175000in}}%
\pgfusepath{clip}%
\pgfsetbuttcap%
\pgfsetroundjoin%
\pgfsetlinewidth{1.003750pt}%
\definecolor{currentstroke}{rgb}{0.827451,0.827451,0.827451}%
\pgfsetstrokecolor{currentstroke}%
\pgfsetdash{}{0pt}%
\pgfpathmoveto{\pgfqpoint{0.583448in}{3.270660in}}%
\pgfpathcurveto{\pgfqpoint{0.594498in}{3.270660in}}{\pgfqpoint{0.605097in}{3.275051in}}{\pgfqpoint{0.612911in}{3.282864in}}%
\pgfpathcurveto{\pgfqpoint{0.620724in}{3.290678in}}{\pgfqpoint{0.625115in}{3.301277in}}{\pgfqpoint{0.625115in}{3.312327in}}%
\pgfpathcurveto{\pgfqpoint{0.625115in}{3.323377in}}{\pgfqpoint{0.620724in}{3.333976in}}{\pgfqpoint{0.612911in}{3.341790in}}%
\pgfpathcurveto{\pgfqpoint{0.605097in}{3.349604in}}{\pgfqpoint{0.594498in}{3.353994in}}{\pgfqpoint{0.583448in}{3.353994in}}%
\pgfpathcurveto{\pgfqpoint{0.572398in}{3.353994in}}{\pgfqpoint{0.561799in}{3.349604in}}{\pgfqpoint{0.553985in}{3.341790in}}%
\pgfpathcurveto{\pgfqpoint{0.546171in}{3.333976in}}{\pgfqpoint{0.541781in}{3.323377in}}{\pgfqpoint{0.541781in}{3.312327in}}%
\pgfpathcurveto{\pgfqpoint{0.541781in}{3.301277in}}{\pgfqpoint{0.546171in}{3.290678in}}{\pgfqpoint{0.553985in}{3.282864in}}%
\pgfpathcurveto{\pgfqpoint{0.561799in}{3.275051in}}{\pgfqpoint{0.572398in}{3.270660in}}{\pgfqpoint{0.583448in}{3.270660in}}%
\pgfpathlineto{\pgfqpoint{0.583448in}{3.270660in}}%
\pgfpathclose%
\pgfusepath{stroke}%
\end{pgfscope}%
\begin{pgfscope}%
\pgfpathrectangle{\pgfqpoint{0.393053in}{0.375000in}}{\pgfqpoint{6.356833in}{5.175000in}}%
\pgfusepath{clip}%
\pgfsetbuttcap%
\pgfsetroundjoin%
\pgfsetlinewidth{1.003750pt}%
\definecolor{currentstroke}{rgb}{0.827451,0.827451,0.827451}%
\pgfsetstrokecolor{currentstroke}%
\pgfsetdash{}{0pt}%
\pgfpathmoveto{\pgfqpoint{5.528065in}{0.337504in}}%
\pgfpathcurveto{\pgfqpoint{5.539115in}{0.337504in}}{\pgfqpoint{5.549714in}{0.341895in}}{\pgfqpoint{5.557528in}{0.349708in}}%
\pgfpathcurveto{\pgfqpoint{5.565341in}{0.357522in}}{\pgfqpoint{5.569732in}{0.368121in}}{\pgfqpoint{5.569732in}{0.379171in}}%
\pgfpathcurveto{\pgfqpoint{5.569732in}{0.390221in}}{\pgfqpoint{5.565341in}{0.400820in}}{\pgfqpoint{5.557528in}{0.408634in}}%
\pgfpathcurveto{\pgfqpoint{5.549714in}{0.416447in}}{\pgfqpoint{5.539115in}{0.420838in}}{\pgfqpoint{5.528065in}{0.420838in}}%
\pgfpathcurveto{\pgfqpoint{5.517015in}{0.420838in}}{\pgfqpoint{5.506416in}{0.416447in}}{\pgfqpoint{5.498602in}{0.408634in}}%
\pgfpathcurveto{\pgfqpoint{5.490789in}{0.400820in}}{\pgfqpoint{5.486398in}{0.390221in}}{\pgfqpoint{5.486398in}{0.379171in}}%
\pgfpathcurveto{\pgfqpoint{5.486398in}{0.368121in}}{\pgfqpoint{5.490789in}{0.357522in}}{\pgfqpoint{5.498602in}{0.349708in}}%
\pgfpathcurveto{\pgfqpoint{5.506416in}{0.341895in}}{\pgfqpoint{5.517015in}{0.337504in}}{\pgfqpoint{5.528065in}{0.337504in}}%
\pgfusepath{stroke}%
\end{pgfscope}%
\begin{pgfscope}%
\pgfpathrectangle{\pgfqpoint{0.393053in}{0.375000in}}{\pgfqpoint{6.356833in}{5.175000in}}%
\pgfusepath{clip}%
\pgfsetbuttcap%
\pgfsetroundjoin%
\pgfsetlinewidth{1.003750pt}%
\definecolor{currentstroke}{rgb}{0.827451,0.827451,0.827451}%
\pgfsetstrokecolor{currentstroke}%
\pgfsetdash{}{0pt}%
\pgfpathmoveto{\pgfqpoint{1.217500in}{2.119012in}}%
\pgfpathcurveto{\pgfqpoint{1.228550in}{2.119012in}}{\pgfqpoint{1.239149in}{2.123403in}}{\pgfqpoint{1.246963in}{2.131216in}}%
\pgfpathcurveto{\pgfqpoint{1.254776in}{2.139030in}}{\pgfqpoint{1.259167in}{2.149629in}}{\pgfqpoint{1.259167in}{2.160679in}}%
\pgfpathcurveto{\pgfqpoint{1.259167in}{2.171729in}}{\pgfqpoint{1.254776in}{2.182328in}}{\pgfqpoint{1.246963in}{2.190142in}}%
\pgfpathcurveto{\pgfqpoint{1.239149in}{2.197955in}}{\pgfqpoint{1.228550in}{2.202346in}}{\pgfqpoint{1.217500in}{2.202346in}}%
\pgfpathcurveto{\pgfqpoint{1.206450in}{2.202346in}}{\pgfqpoint{1.195851in}{2.197955in}}{\pgfqpoint{1.188037in}{2.190142in}}%
\pgfpathcurveto{\pgfqpoint{1.180224in}{2.182328in}}{\pgfqpoint{1.175833in}{2.171729in}}{\pgfqpoint{1.175833in}{2.160679in}}%
\pgfpathcurveto{\pgfqpoint{1.175833in}{2.149629in}}{\pgfqpoint{1.180224in}{2.139030in}}{\pgfqpoint{1.188037in}{2.131216in}}%
\pgfpathcurveto{\pgfqpoint{1.195851in}{2.123403in}}{\pgfqpoint{1.206450in}{2.119012in}}{\pgfqpoint{1.217500in}{2.119012in}}%
\pgfpathlineto{\pgfqpoint{1.217500in}{2.119012in}}%
\pgfpathclose%
\pgfusepath{stroke}%
\end{pgfscope}%
\begin{pgfscope}%
\pgfpathrectangle{\pgfqpoint{0.393053in}{0.375000in}}{\pgfqpoint{6.356833in}{5.175000in}}%
\pgfusepath{clip}%
\pgfsetbuttcap%
\pgfsetroundjoin%
\pgfsetlinewidth{1.003750pt}%
\definecolor{currentstroke}{rgb}{0.827451,0.827451,0.827451}%
\pgfsetstrokecolor{currentstroke}%
\pgfsetdash{}{0pt}%
\pgfpathmoveto{\pgfqpoint{0.452170in}{3.855614in}}%
\pgfpathcurveto{\pgfqpoint{0.463220in}{3.855614in}}{\pgfqpoint{0.473819in}{3.860005in}}{\pgfqpoint{0.481633in}{3.867818in}}%
\pgfpathcurveto{\pgfqpoint{0.489447in}{3.875632in}}{\pgfqpoint{0.493837in}{3.886231in}}{\pgfqpoint{0.493837in}{3.897281in}}%
\pgfpathcurveto{\pgfqpoint{0.493837in}{3.908331in}}{\pgfqpoint{0.489447in}{3.918930in}}{\pgfqpoint{0.481633in}{3.926744in}}%
\pgfpathcurveto{\pgfqpoint{0.473819in}{3.934557in}}{\pgfqpoint{0.463220in}{3.938948in}}{\pgfqpoint{0.452170in}{3.938948in}}%
\pgfpathcurveto{\pgfqpoint{0.441120in}{3.938948in}}{\pgfqpoint{0.430521in}{3.934557in}}{\pgfqpoint{0.422707in}{3.926744in}}%
\pgfpathcurveto{\pgfqpoint{0.414894in}{3.918930in}}{\pgfqpoint{0.410503in}{3.908331in}}{\pgfqpoint{0.410503in}{3.897281in}}%
\pgfpathcurveto{\pgfqpoint{0.410503in}{3.886231in}}{\pgfqpoint{0.414894in}{3.875632in}}{\pgfqpoint{0.422707in}{3.867818in}}%
\pgfpathcurveto{\pgfqpoint{0.430521in}{3.860005in}}{\pgfqpoint{0.441120in}{3.855614in}}{\pgfqpoint{0.452170in}{3.855614in}}%
\pgfpathlineto{\pgfqpoint{0.452170in}{3.855614in}}%
\pgfpathclose%
\pgfusepath{stroke}%
\end{pgfscope}%
\begin{pgfscope}%
\pgfpathrectangle{\pgfqpoint{0.393053in}{0.375000in}}{\pgfqpoint{6.356833in}{5.175000in}}%
\pgfusepath{clip}%
\pgfsetbuttcap%
\pgfsetroundjoin%
\pgfsetlinewidth{1.003750pt}%
\definecolor{currentstroke}{rgb}{0.827451,0.827451,0.827451}%
\pgfsetstrokecolor{currentstroke}%
\pgfsetdash{}{0pt}%
\pgfpathmoveto{\pgfqpoint{2.145018in}{1.265722in}}%
\pgfpathcurveto{\pgfqpoint{2.156069in}{1.265722in}}{\pgfqpoint{2.166668in}{1.270112in}}{\pgfqpoint{2.174481in}{1.277926in}}%
\pgfpathcurveto{\pgfqpoint{2.182295in}{1.285739in}}{\pgfqpoint{2.186685in}{1.296338in}}{\pgfqpoint{2.186685in}{1.307389in}}%
\pgfpathcurveto{\pgfqpoint{2.186685in}{1.318439in}}{\pgfqpoint{2.182295in}{1.329038in}}{\pgfqpoint{2.174481in}{1.336851in}}%
\pgfpathcurveto{\pgfqpoint{2.166668in}{1.344665in}}{\pgfqpoint{2.156069in}{1.349055in}}{\pgfqpoint{2.145018in}{1.349055in}}%
\pgfpathcurveto{\pgfqpoint{2.133968in}{1.349055in}}{\pgfqpoint{2.123369in}{1.344665in}}{\pgfqpoint{2.115556in}{1.336851in}}%
\pgfpathcurveto{\pgfqpoint{2.107742in}{1.329038in}}{\pgfqpoint{2.103352in}{1.318439in}}{\pgfqpoint{2.103352in}{1.307389in}}%
\pgfpathcurveto{\pgfqpoint{2.103352in}{1.296338in}}{\pgfqpoint{2.107742in}{1.285739in}}{\pgfqpoint{2.115556in}{1.277926in}}%
\pgfpathcurveto{\pgfqpoint{2.123369in}{1.270112in}}{\pgfqpoint{2.133968in}{1.265722in}}{\pgfqpoint{2.145018in}{1.265722in}}%
\pgfpathlineto{\pgfqpoint{2.145018in}{1.265722in}}%
\pgfpathclose%
\pgfusepath{stroke}%
\end{pgfscope}%
\begin{pgfscope}%
\pgfpathrectangle{\pgfqpoint{0.393053in}{0.375000in}}{\pgfqpoint{6.356833in}{5.175000in}}%
\pgfusepath{clip}%
\pgfsetbuttcap%
\pgfsetroundjoin%
\pgfsetlinewidth{1.003750pt}%
\definecolor{currentstroke}{rgb}{0.827451,0.827451,0.827451}%
\pgfsetstrokecolor{currentstroke}%
\pgfsetdash{}{0pt}%
\pgfpathmoveto{\pgfqpoint{1.234935in}{2.069821in}}%
\pgfpathcurveto{\pgfqpoint{1.245986in}{2.069821in}}{\pgfqpoint{1.256585in}{2.074212in}}{\pgfqpoint{1.264398in}{2.082025in}}%
\pgfpathcurveto{\pgfqpoint{1.272212in}{2.089839in}}{\pgfqpoint{1.276602in}{2.100438in}}{\pgfqpoint{1.276602in}{2.111488in}}%
\pgfpathcurveto{\pgfqpoint{1.276602in}{2.122538in}}{\pgfqpoint{1.272212in}{2.133137in}}{\pgfqpoint{1.264398in}{2.140951in}}%
\pgfpathcurveto{\pgfqpoint{1.256585in}{2.148764in}}{\pgfqpoint{1.245986in}{2.153155in}}{\pgfqpoint{1.234935in}{2.153155in}}%
\pgfpathcurveto{\pgfqpoint{1.223885in}{2.153155in}}{\pgfqpoint{1.213286in}{2.148764in}}{\pgfqpoint{1.205473in}{2.140951in}}%
\pgfpathcurveto{\pgfqpoint{1.197659in}{2.133137in}}{\pgfqpoint{1.193269in}{2.122538in}}{\pgfqpoint{1.193269in}{2.111488in}}%
\pgfpathcurveto{\pgfqpoint{1.193269in}{2.100438in}}{\pgfqpoint{1.197659in}{2.089839in}}{\pgfqpoint{1.205473in}{2.082025in}}%
\pgfpathcurveto{\pgfqpoint{1.213286in}{2.074212in}}{\pgfqpoint{1.223885in}{2.069821in}}{\pgfqpoint{1.234935in}{2.069821in}}%
\pgfpathlineto{\pgfqpoint{1.234935in}{2.069821in}}%
\pgfpathclose%
\pgfusepath{stroke}%
\end{pgfscope}%
\begin{pgfscope}%
\pgfpathrectangle{\pgfqpoint{0.393053in}{0.375000in}}{\pgfqpoint{6.356833in}{5.175000in}}%
\pgfusepath{clip}%
\pgfsetbuttcap%
\pgfsetroundjoin%
\pgfsetlinewidth{1.003750pt}%
\definecolor{currentstroke}{rgb}{0.827451,0.827451,0.827451}%
\pgfsetstrokecolor{currentstroke}%
\pgfsetdash{}{0pt}%
\pgfpathmoveto{\pgfqpoint{2.504896in}{1.054151in}}%
\pgfpathcurveto{\pgfqpoint{2.515946in}{1.054151in}}{\pgfqpoint{2.526545in}{1.058541in}}{\pgfqpoint{2.534359in}{1.066355in}}%
\pgfpathcurveto{\pgfqpoint{2.542172in}{1.074169in}}{\pgfqpoint{2.546562in}{1.084768in}}{\pgfqpoint{2.546562in}{1.095818in}}%
\pgfpathcurveto{\pgfqpoint{2.546562in}{1.106868in}}{\pgfqpoint{2.542172in}{1.117467in}}{\pgfqpoint{2.534359in}{1.125281in}}%
\pgfpathcurveto{\pgfqpoint{2.526545in}{1.133094in}}{\pgfqpoint{2.515946in}{1.137484in}}{\pgfqpoint{2.504896in}{1.137484in}}%
\pgfpathcurveto{\pgfqpoint{2.493846in}{1.137484in}}{\pgfqpoint{2.483247in}{1.133094in}}{\pgfqpoint{2.475433in}{1.125281in}}%
\pgfpathcurveto{\pgfqpoint{2.467619in}{1.117467in}}{\pgfqpoint{2.463229in}{1.106868in}}{\pgfqpoint{2.463229in}{1.095818in}}%
\pgfpathcurveto{\pgfqpoint{2.463229in}{1.084768in}}{\pgfqpoint{2.467619in}{1.074169in}}{\pgfqpoint{2.475433in}{1.066355in}}%
\pgfpathcurveto{\pgfqpoint{2.483247in}{1.058541in}}{\pgfqpoint{2.493846in}{1.054151in}}{\pgfqpoint{2.504896in}{1.054151in}}%
\pgfpathlineto{\pgfqpoint{2.504896in}{1.054151in}}%
\pgfpathclose%
\pgfusepath{stroke}%
\end{pgfscope}%
\begin{pgfscope}%
\pgfpathrectangle{\pgfqpoint{0.393053in}{0.375000in}}{\pgfqpoint{6.356833in}{5.175000in}}%
\pgfusepath{clip}%
\pgfsetbuttcap%
\pgfsetroundjoin%
\pgfsetlinewidth{1.003750pt}%
\definecolor{currentstroke}{rgb}{0.827451,0.827451,0.827451}%
\pgfsetstrokecolor{currentstroke}%
\pgfsetdash{}{0pt}%
\pgfpathmoveto{\pgfqpoint{4.176160in}{0.456257in}}%
\pgfpathcurveto{\pgfqpoint{4.187210in}{0.456257in}}{\pgfqpoint{4.197809in}{0.460647in}}{\pgfqpoint{4.205623in}{0.468461in}}%
\pgfpathcurveto{\pgfqpoint{4.213436in}{0.476274in}}{\pgfqpoint{4.217827in}{0.486873in}}{\pgfqpoint{4.217827in}{0.497923in}}%
\pgfpathcurveto{\pgfqpoint{4.217827in}{0.508973in}}{\pgfqpoint{4.213436in}{0.519572in}}{\pgfqpoint{4.205623in}{0.527386in}}%
\pgfpathcurveto{\pgfqpoint{4.197809in}{0.535200in}}{\pgfqpoint{4.187210in}{0.539590in}}{\pgfqpoint{4.176160in}{0.539590in}}%
\pgfpathcurveto{\pgfqpoint{4.165110in}{0.539590in}}{\pgfqpoint{4.154511in}{0.535200in}}{\pgfqpoint{4.146697in}{0.527386in}}%
\pgfpathcurveto{\pgfqpoint{4.138883in}{0.519572in}}{\pgfqpoint{4.134493in}{0.508973in}}{\pgfqpoint{4.134493in}{0.497923in}}%
\pgfpathcurveto{\pgfqpoint{4.134493in}{0.486873in}}{\pgfqpoint{4.138883in}{0.476274in}}{\pgfqpoint{4.146697in}{0.468461in}}%
\pgfpathcurveto{\pgfqpoint{4.154511in}{0.460647in}}{\pgfqpoint{4.165110in}{0.456257in}}{\pgfqpoint{4.176160in}{0.456257in}}%
\pgfpathlineto{\pgfqpoint{4.176160in}{0.456257in}}%
\pgfpathclose%
\pgfusepath{stroke}%
\end{pgfscope}%
\begin{pgfscope}%
\pgfpathrectangle{\pgfqpoint{0.393053in}{0.375000in}}{\pgfqpoint{6.356833in}{5.175000in}}%
\pgfusepath{clip}%
\pgfsetbuttcap%
\pgfsetroundjoin%
\pgfsetlinewidth{1.003750pt}%
\definecolor{currentstroke}{rgb}{0.827451,0.827451,0.827451}%
\pgfsetstrokecolor{currentstroke}%
\pgfsetdash{}{0pt}%
\pgfpathmoveto{\pgfqpoint{3.317358in}{0.680478in}}%
\pgfpathcurveto{\pgfqpoint{3.328408in}{0.680478in}}{\pgfqpoint{3.339007in}{0.684868in}}{\pgfqpoint{3.346821in}{0.692682in}}%
\pgfpathcurveto{\pgfqpoint{3.354635in}{0.700495in}}{\pgfqpoint{3.359025in}{0.711094in}}{\pgfqpoint{3.359025in}{0.722144in}}%
\pgfpathcurveto{\pgfqpoint{3.359025in}{0.733195in}}{\pgfqpoint{3.354635in}{0.743794in}}{\pgfqpoint{3.346821in}{0.751607in}}%
\pgfpathcurveto{\pgfqpoint{3.339007in}{0.759421in}}{\pgfqpoint{3.328408in}{0.763811in}}{\pgfqpoint{3.317358in}{0.763811in}}%
\pgfpathcurveto{\pgfqpoint{3.306308in}{0.763811in}}{\pgfqpoint{3.295709in}{0.759421in}}{\pgfqpoint{3.287895in}{0.751607in}}%
\pgfpathcurveto{\pgfqpoint{3.280082in}{0.743794in}}{\pgfqpoint{3.275692in}{0.733195in}}{\pgfqpoint{3.275692in}{0.722144in}}%
\pgfpathcurveto{\pgfqpoint{3.275692in}{0.711094in}}{\pgfqpoint{3.280082in}{0.700495in}}{\pgfqpoint{3.287895in}{0.692682in}}%
\pgfpathcurveto{\pgfqpoint{3.295709in}{0.684868in}}{\pgfqpoint{3.306308in}{0.680478in}}{\pgfqpoint{3.317358in}{0.680478in}}%
\pgfpathlineto{\pgfqpoint{3.317358in}{0.680478in}}%
\pgfpathclose%
\pgfusepath{stroke}%
\end{pgfscope}%
\begin{pgfscope}%
\pgfpathrectangle{\pgfqpoint{0.393053in}{0.375000in}}{\pgfqpoint{6.356833in}{5.175000in}}%
\pgfusepath{clip}%
\pgfsetbuttcap%
\pgfsetroundjoin%
\pgfsetlinewidth{1.003750pt}%
\definecolor{currentstroke}{rgb}{0.827451,0.827451,0.827451}%
\pgfsetstrokecolor{currentstroke}%
\pgfsetdash{}{0pt}%
\pgfpathmoveto{\pgfqpoint{2.103640in}{1.312371in}}%
\pgfpathcurveto{\pgfqpoint{2.114690in}{1.312371in}}{\pgfqpoint{2.125289in}{1.316762in}}{\pgfqpoint{2.133103in}{1.324575in}}%
\pgfpathcurveto{\pgfqpoint{2.140916in}{1.332389in}}{\pgfqpoint{2.145307in}{1.342988in}}{\pgfqpoint{2.145307in}{1.354038in}}%
\pgfpathcurveto{\pgfqpoint{2.145307in}{1.365088in}}{\pgfqpoint{2.140916in}{1.375687in}}{\pgfqpoint{2.133103in}{1.383501in}}%
\pgfpathcurveto{\pgfqpoint{2.125289in}{1.391314in}}{\pgfqpoint{2.114690in}{1.395705in}}{\pgfqpoint{2.103640in}{1.395705in}}%
\pgfpathcurveto{\pgfqpoint{2.092590in}{1.395705in}}{\pgfqpoint{2.081991in}{1.391314in}}{\pgfqpoint{2.074177in}{1.383501in}}%
\pgfpathcurveto{\pgfqpoint{2.066364in}{1.375687in}}{\pgfqpoint{2.061973in}{1.365088in}}{\pgfqpoint{2.061973in}{1.354038in}}%
\pgfpathcurveto{\pgfqpoint{2.061973in}{1.342988in}}{\pgfqpoint{2.066364in}{1.332389in}}{\pgfqpoint{2.074177in}{1.324575in}}%
\pgfpathcurveto{\pgfqpoint{2.081991in}{1.316762in}}{\pgfqpoint{2.092590in}{1.312371in}}{\pgfqpoint{2.103640in}{1.312371in}}%
\pgfpathlineto{\pgfqpoint{2.103640in}{1.312371in}}%
\pgfpathclose%
\pgfusepath{stroke}%
\end{pgfscope}%
\begin{pgfscope}%
\pgfpathrectangle{\pgfqpoint{0.393053in}{0.375000in}}{\pgfqpoint{6.356833in}{5.175000in}}%
\pgfusepath{clip}%
\pgfsetbuttcap%
\pgfsetroundjoin%
\pgfsetlinewidth{1.003750pt}%
\definecolor{currentstroke}{rgb}{0.827451,0.827451,0.827451}%
\pgfsetstrokecolor{currentstroke}%
\pgfsetdash{}{0pt}%
\pgfpathmoveto{\pgfqpoint{0.773700in}{2.823025in}}%
\pgfpathcurveto{\pgfqpoint{0.784750in}{2.823025in}}{\pgfqpoint{0.795350in}{2.827415in}}{\pgfqpoint{0.803163in}{2.835229in}}%
\pgfpathcurveto{\pgfqpoint{0.810977in}{2.843042in}}{\pgfqpoint{0.815367in}{2.853641in}}{\pgfqpoint{0.815367in}{2.864691in}}%
\pgfpathcurveto{\pgfqpoint{0.815367in}{2.875742in}}{\pgfqpoint{0.810977in}{2.886341in}}{\pgfqpoint{0.803163in}{2.894154in}}%
\pgfpathcurveto{\pgfqpoint{0.795350in}{2.901968in}}{\pgfqpoint{0.784750in}{2.906358in}}{\pgfqpoint{0.773700in}{2.906358in}}%
\pgfpathcurveto{\pgfqpoint{0.762650in}{2.906358in}}{\pgfqpoint{0.752051in}{2.901968in}}{\pgfqpoint{0.744238in}{2.894154in}}%
\pgfpathcurveto{\pgfqpoint{0.736424in}{2.886341in}}{\pgfqpoint{0.732034in}{2.875742in}}{\pgfqpoint{0.732034in}{2.864691in}}%
\pgfpathcurveto{\pgfqpoint{0.732034in}{2.853641in}}{\pgfqpoint{0.736424in}{2.843042in}}{\pgfqpoint{0.744238in}{2.835229in}}%
\pgfpathcurveto{\pgfqpoint{0.752051in}{2.827415in}}{\pgfqpoint{0.762650in}{2.823025in}}{\pgfqpoint{0.773700in}{2.823025in}}%
\pgfpathlineto{\pgfqpoint{0.773700in}{2.823025in}}%
\pgfpathclose%
\pgfusepath{stroke}%
\end{pgfscope}%
\begin{pgfscope}%
\pgfpathrectangle{\pgfqpoint{0.393053in}{0.375000in}}{\pgfqpoint{6.356833in}{5.175000in}}%
\pgfusepath{clip}%
\pgfsetbuttcap%
\pgfsetroundjoin%
\pgfsetlinewidth{1.003750pt}%
\definecolor{currentstroke}{rgb}{0.827451,0.827451,0.827451}%
\pgfsetstrokecolor{currentstroke}%
\pgfsetdash{}{0pt}%
\pgfpathmoveto{\pgfqpoint{0.686941in}{2.982665in}}%
\pgfpathcurveto{\pgfqpoint{0.697991in}{2.982665in}}{\pgfqpoint{0.708590in}{2.987055in}}{\pgfqpoint{0.716403in}{2.994869in}}%
\pgfpathcurveto{\pgfqpoint{0.724217in}{3.002682in}}{\pgfqpoint{0.728607in}{3.013281in}}{\pgfqpoint{0.728607in}{3.024332in}}%
\pgfpathcurveto{\pgfqpoint{0.728607in}{3.035382in}}{\pgfqpoint{0.724217in}{3.045981in}}{\pgfqpoint{0.716403in}{3.053794in}}%
\pgfpathcurveto{\pgfqpoint{0.708590in}{3.061608in}}{\pgfqpoint{0.697991in}{3.065998in}}{\pgfqpoint{0.686941in}{3.065998in}}%
\pgfpathcurveto{\pgfqpoint{0.675890in}{3.065998in}}{\pgfqpoint{0.665291in}{3.061608in}}{\pgfqpoint{0.657478in}{3.053794in}}%
\pgfpathcurveto{\pgfqpoint{0.649664in}{3.045981in}}{\pgfqpoint{0.645274in}{3.035382in}}{\pgfqpoint{0.645274in}{3.024332in}}%
\pgfpathcurveto{\pgfqpoint{0.645274in}{3.013281in}}{\pgfqpoint{0.649664in}{3.002682in}}{\pgfqpoint{0.657478in}{2.994869in}}%
\pgfpathcurveto{\pgfqpoint{0.665291in}{2.987055in}}{\pgfqpoint{0.675890in}{2.982665in}}{\pgfqpoint{0.686941in}{2.982665in}}%
\pgfpathlineto{\pgfqpoint{0.686941in}{2.982665in}}%
\pgfpathclose%
\pgfusepath{stroke}%
\end{pgfscope}%
\begin{pgfscope}%
\pgfpathrectangle{\pgfqpoint{0.393053in}{0.375000in}}{\pgfqpoint{6.356833in}{5.175000in}}%
\pgfusepath{clip}%
\pgfsetbuttcap%
\pgfsetroundjoin%
\pgfsetlinewidth{1.003750pt}%
\definecolor{currentstroke}{rgb}{0.827451,0.827451,0.827451}%
\pgfsetstrokecolor{currentstroke}%
\pgfsetdash{}{0pt}%
\pgfpathmoveto{\pgfqpoint{0.918047in}{2.554880in}}%
\pgfpathcurveto{\pgfqpoint{0.929097in}{2.554880in}}{\pgfqpoint{0.939696in}{2.559271in}}{\pgfqpoint{0.947510in}{2.567084in}}%
\pgfpathcurveto{\pgfqpoint{0.955323in}{2.574898in}}{\pgfqpoint{0.959714in}{2.585497in}}{\pgfqpoint{0.959714in}{2.596547in}}%
\pgfpathcurveto{\pgfqpoint{0.959714in}{2.607597in}}{\pgfqpoint{0.955323in}{2.618196in}}{\pgfqpoint{0.947510in}{2.626010in}}%
\pgfpathcurveto{\pgfqpoint{0.939696in}{2.633823in}}{\pgfqpoint{0.929097in}{2.638214in}}{\pgfqpoint{0.918047in}{2.638214in}}%
\pgfpathcurveto{\pgfqpoint{0.906997in}{2.638214in}}{\pgfqpoint{0.896398in}{2.633823in}}{\pgfqpoint{0.888584in}{2.626010in}}%
\pgfpathcurveto{\pgfqpoint{0.880771in}{2.618196in}}{\pgfqpoint{0.876380in}{2.607597in}}{\pgfqpoint{0.876380in}{2.596547in}}%
\pgfpathcurveto{\pgfqpoint{0.876380in}{2.585497in}}{\pgfqpoint{0.880771in}{2.574898in}}{\pgfqpoint{0.888584in}{2.567084in}}%
\pgfpathcurveto{\pgfqpoint{0.896398in}{2.559271in}}{\pgfqpoint{0.906997in}{2.554880in}}{\pgfqpoint{0.918047in}{2.554880in}}%
\pgfpathlineto{\pgfqpoint{0.918047in}{2.554880in}}%
\pgfpathclose%
\pgfusepath{stroke}%
\end{pgfscope}%
\begin{pgfscope}%
\pgfpathrectangle{\pgfqpoint{0.393053in}{0.375000in}}{\pgfqpoint{6.356833in}{5.175000in}}%
\pgfusepath{clip}%
\pgfsetbuttcap%
\pgfsetroundjoin%
\pgfsetlinewidth{1.003750pt}%
\definecolor{currentstroke}{rgb}{0.827451,0.827451,0.827451}%
\pgfsetstrokecolor{currentstroke}%
\pgfsetdash{}{0pt}%
\pgfpathmoveto{\pgfqpoint{1.256856in}{2.039646in}}%
\pgfpathcurveto{\pgfqpoint{1.267906in}{2.039646in}}{\pgfqpoint{1.278505in}{2.044036in}}{\pgfqpoint{1.286319in}{2.051849in}}%
\pgfpathcurveto{\pgfqpoint{1.294132in}{2.059663in}}{\pgfqpoint{1.298523in}{2.070262in}}{\pgfqpoint{1.298523in}{2.081312in}}%
\pgfpathcurveto{\pgfqpoint{1.298523in}{2.092362in}}{\pgfqpoint{1.294132in}{2.102961in}}{\pgfqpoint{1.286319in}{2.110775in}}%
\pgfpathcurveto{\pgfqpoint{1.278505in}{2.118589in}}{\pgfqpoint{1.267906in}{2.122979in}}{\pgfqpoint{1.256856in}{2.122979in}}%
\pgfpathcurveto{\pgfqpoint{1.245806in}{2.122979in}}{\pgfqpoint{1.235207in}{2.118589in}}{\pgfqpoint{1.227393in}{2.110775in}}%
\pgfpathcurveto{\pgfqpoint{1.219580in}{2.102961in}}{\pgfqpoint{1.215189in}{2.092362in}}{\pgfqpoint{1.215189in}{2.081312in}}%
\pgfpathcurveto{\pgfqpoint{1.215189in}{2.070262in}}{\pgfqpoint{1.219580in}{2.059663in}}{\pgfqpoint{1.227393in}{2.051849in}}%
\pgfpathcurveto{\pgfqpoint{1.235207in}{2.044036in}}{\pgfqpoint{1.245806in}{2.039646in}}{\pgfqpoint{1.256856in}{2.039646in}}%
\pgfpathlineto{\pgfqpoint{1.256856in}{2.039646in}}%
\pgfpathclose%
\pgfusepath{stroke}%
\end{pgfscope}%
\begin{pgfscope}%
\pgfpathrectangle{\pgfqpoint{0.393053in}{0.375000in}}{\pgfqpoint{6.356833in}{5.175000in}}%
\pgfusepath{clip}%
\pgfsetbuttcap%
\pgfsetroundjoin%
\pgfsetlinewidth{1.003750pt}%
\definecolor{currentstroke}{rgb}{0.827451,0.827451,0.827451}%
\pgfsetstrokecolor{currentstroke}%
\pgfsetdash{}{0pt}%
\pgfpathmoveto{\pgfqpoint{1.135663in}{2.200937in}}%
\pgfpathcurveto{\pgfqpoint{1.146713in}{2.200937in}}{\pgfqpoint{1.157312in}{2.205328in}}{\pgfqpoint{1.165126in}{2.213141in}}%
\pgfpathcurveto{\pgfqpoint{1.172939in}{2.220955in}}{\pgfqpoint{1.177330in}{2.231554in}}{\pgfqpoint{1.177330in}{2.242604in}}%
\pgfpathcurveto{\pgfqpoint{1.177330in}{2.253654in}}{\pgfqpoint{1.172939in}{2.264253in}}{\pgfqpoint{1.165126in}{2.272067in}}%
\pgfpathcurveto{\pgfqpoint{1.157312in}{2.279880in}}{\pgfqpoint{1.146713in}{2.284271in}}{\pgfqpoint{1.135663in}{2.284271in}}%
\pgfpathcurveto{\pgfqpoint{1.124613in}{2.284271in}}{\pgfqpoint{1.114014in}{2.279880in}}{\pgfqpoint{1.106200in}{2.272067in}}%
\pgfpathcurveto{\pgfqpoint{1.098387in}{2.264253in}}{\pgfqpoint{1.093996in}{2.253654in}}{\pgfqpoint{1.093996in}{2.242604in}}%
\pgfpathcurveto{\pgfqpoint{1.093996in}{2.231554in}}{\pgfqpoint{1.098387in}{2.220955in}}{\pgfqpoint{1.106200in}{2.213141in}}%
\pgfpathcurveto{\pgfqpoint{1.114014in}{2.205328in}}{\pgfqpoint{1.124613in}{2.200937in}}{\pgfqpoint{1.135663in}{2.200937in}}%
\pgfpathlineto{\pgfqpoint{1.135663in}{2.200937in}}%
\pgfpathclose%
\pgfusepath{stroke}%
\end{pgfscope}%
\begin{pgfscope}%
\pgfpathrectangle{\pgfqpoint{0.393053in}{0.375000in}}{\pgfqpoint{6.356833in}{5.175000in}}%
\pgfusepath{clip}%
\pgfsetbuttcap%
\pgfsetroundjoin%
\pgfsetlinewidth{1.003750pt}%
\definecolor{currentstroke}{rgb}{0.827451,0.827451,0.827451}%
\pgfsetstrokecolor{currentstroke}%
\pgfsetdash{}{0pt}%
\pgfpathmoveto{\pgfqpoint{1.457810in}{1.831739in}}%
\pgfpathcurveto{\pgfqpoint{1.468860in}{1.831739in}}{\pgfqpoint{1.479459in}{1.836129in}}{\pgfqpoint{1.487273in}{1.843943in}}%
\pgfpathcurveto{\pgfqpoint{1.495086in}{1.851756in}}{\pgfqpoint{1.499477in}{1.862355in}}{\pgfqpoint{1.499477in}{1.873406in}}%
\pgfpathcurveto{\pgfqpoint{1.499477in}{1.884456in}}{\pgfqpoint{1.495086in}{1.895055in}}{\pgfqpoint{1.487273in}{1.902868in}}%
\pgfpathcurveto{\pgfqpoint{1.479459in}{1.910682in}}{\pgfqpoint{1.468860in}{1.915072in}}{\pgfqpoint{1.457810in}{1.915072in}}%
\pgfpathcurveto{\pgfqpoint{1.446760in}{1.915072in}}{\pgfqpoint{1.436161in}{1.910682in}}{\pgfqpoint{1.428347in}{1.902868in}}%
\pgfpathcurveto{\pgfqpoint{1.420533in}{1.895055in}}{\pgfqpoint{1.416143in}{1.884456in}}{\pgfqpoint{1.416143in}{1.873406in}}%
\pgfpathcurveto{\pgfqpoint{1.416143in}{1.862355in}}{\pgfqpoint{1.420533in}{1.851756in}}{\pgfqpoint{1.428347in}{1.843943in}}%
\pgfpathcurveto{\pgfqpoint{1.436161in}{1.836129in}}{\pgfqpoint{1.446760in}{1.831739in}}{\pgfqpoint{1.457810in}{1.831739in}}%
\pgfpathlineto{\pgfqpoint{1.457810in}{1.831739in}}%
\pgfpathclose%
\pgfusepath{stroke}%
\end{pgfscope}%
\begin{pgfscope}%
\pgfpathrectangle{\pgfqpoint{0.393053in}{0.375000in}}{\pgfqpoint{6.356833in}{5.175000in}}%
\pgfusepath{clip}%
\pgfsetbuttcap%
\pgfsetroundjoin%
\pgfsetlinewidth{1.003750pt}%
\definecolor{currentstroke}{rgb}{0.827451,0.827451,0.827451}%
\pgfsetstrokecolor{currentstroke}%
\pgfsetdash{}{0pt}%
\pgfpathmoveto{\pgfqpoint{0.393415in}{4.523360in}}%
\pgfpathcurveto{\pgfqpoint{0.404465in}{4.523360in}}{\pgfqpoint{0.415064in}{4.527750in}}{\pgfqpoint{0.422878in}{4.535564in}}%
\pgfpathcurveto{\pgfqpoint{0.430691in}{4.543377in}}{\pgfqpoint{0.435082in}{4.553976in}}{\pgfqpoint{0.435082in}{4.565026in}}%
\pgfpathcurveto{\pgfqpoint{0.435082in}{4.576077in}}{\pgfqpoint{0.430691in}{4.586676in}}{\pgfqpoint{0.422878in}{4.594489in}}%
\pgfpathcurveto{\pgfqpoint{0.415064in}{4.602303in}}{\pgfqpoint{0.404465in}{4.606693in}}{\pgfqpoint{0.393415in}{4.606693in}}%
\pgfpathcurveto{\pgfqpoint{0.382365in}{4.606693in}}{\pgfqpoint{0.371766in}{4.602303in}}{\pgfqpoint{0.363952in}{4.594489in}}%
\pgfpathcurveto{\pgfqpoint{0.356139in}{4.586676in}}{\pgfqpoint{0.351748in}{4.576077in}}{\pgfqpoint{0.351748in}{4.565026in}}%
\pgfpathcurveto{\pgfqpoint{0.351748in}{4.553976in}}{\pgfqpoint{0.356139in}{4.543377in}}{\pgfqpoint{0.363952in}{4.535564in}}%
\pgfpathcurveto{\pgfqpoint{0.371766in}{4.527750in}}{\pgfqpoint{0.382365in}{4.523360in}}{\pgfqpoint{0.393415in}{4.523360in}}%
\pgfpathlineto{\pgfqpoint{0.393415in}{4.523360in}}%
\pgfpathclose%
\pgfusepath{stroke}%
\end{pgfscope}%
\begin{pgfscope}%
\pgfpathrectangle{\pgfqpoint{0.393053in}{0.375000in}}{\pgfqpoint{6.356833in}{5.175000in}}%
\pgfusepath{clip}%
\pgfsetbuttcap%
\pgfsetroundjoin%
\pgfsetlinewidth{1.003750pt}%
\definecolor{currentstroke}{rgb}{0.827451,0.827451,0.827451}%
\pgfsetstrokecolor{currentstroke}%
\pgfsetdash{}{0pt}%
\pgfpathmoveto{\pgfqpoint{5.724126in}{0.333770in}}%
\pgfpathcurveto{\pgfqpoint{5.735176in}{0.333770in}}{\pgfqpoint{5.745775in}{0.338161in}}{\pgfqpoint{5.753589in}{0.345974in}}%
\pgfpathcurveto{\pgfqpoint{5.761402in}{0.353788in}}{\pgfqpoint{5.765793in}{0.364387in}}{\pgfqpoint{5.765793in}{0.375437in}}%
\pgfpathcurveto{\pgfqpoint{5.765793in}{0.386487in}}{\pgfqpoint{5.761402in}{0.397086in}}{\pgfqpoint{5.753589in}{0.404900in}}%
\pgfpathcurveto{\pgfqpoint{5.745775in}{0.412713in}}{\pgfqpoint{5.735176in}{0.417104in}}{\pgfqpoint{5.724126in}{0.417104in}}%
\pgfpathcurveto{\pgfqpoint{5.713076in}{0.417104in}}{\pgfqpoint{5.702477in}{0.412713in}}{\pgfqpoint{5.694663in}{0.404900in}}%
\pgfpathcurveto{\pgfqpoint{5.686850in}{0.397086in}}{\pgfqpoint{5.682459in}{0.386487in}}{\pgfqpoint{5.682459in}{0.375437in}}%
\pgfpathcurveto{\pgfqpoint{5.682459in}{0.364387in}}{\pgfqpoint{5.686850in}{0.353788in}}{\pgfqpoint{5.694663in}{0.345974in}}%
\pgfpathcurveto{\pgfqpoint{5.702477in}{0.338161in}}{\pgfqpoint{5.713076in}{0.333770in}}{\pgfqpoint{5.724126in}{0.333770in}}%
\pgfusepath{stroke}%
\end{pgfscope}%
\begin{pgfscope}%
\pgfpathrectangle{\pgfqpoint{0.393053in}{0.375000in}}{\pgfqpoint{6.356833in}{5.175000in}}%
\pgfusepath{clip}%
\pgfsetbuttcap%
\pgfsetroundjoin%
\pgfsetlinewidth{1.003750pt}%
\definecolor{currentstroke}{rgb}{0.827451,0.827451,0.827451}%
\pgfsetstrokecolor{currentstroke}%
\pgfsetdash{}{0pt}%
\pgfpathmoveto{\pgfqpoint{0.399634in}{4.323598in}}%
\pgfpathcurveto{\pgfqpoint{0.410684in}{4.323598in}}{\pgfqpoint{0.421283in}{4.327988in}}{\pgfqpoint{0.429097in}{4.335802in}}%
\pgfpathcurveto{\pgfqpoint{0.436910in}{4.343616in}}{\pgfqpoint{0.441301in}{4.354215in}}{\pgfqpoint{0.441301in}{4.365265in}}%
\pgfpathcurveto{\pgfqpoint{0.441301in}{4.376315in}}{\pgfqpoint{0.436910in}{4.386914in}}{\pgfqpoint{0.429097in}{4.394727in}}%
\pgfpathcurveto{\pgfqpoint{0.421283in}{4.402541in}}{\pgfqpoint{0.410684in}{4.406931in}}{\pgfqpoint{0.399634in}{4.406931in}}%
\pgfpathcurveto{\pgfqpoint{0.388584in}{4.406931in}}{\pgfqpoint{0.377985in}{4.402541in}}{\pgfqpoint{0.370171in}{4.394727in}}%
\pgfpathcurveto{\pgfqpoint{0.362357in}{4.386914in}}{\pgfqpoint{0.357967in}{4.376315in}}{\pgfqpoint{0.357967in}{4.365265in}}%
\pgfpathcurveto{\pgfqpoint{0.357967in}{4.354215in}}{\pgfqpoint{0.362357in}{4.343616in}}{\pgfqpoint{0.370171in}{4.335802in}}%
\pgfpathcurveto{\pgfqpoint{0.377985in}{4.327988in}}{\pgfqpoint{0.388584in}{4.323598in}}{\pgfqpoint{0.399634in}{4.323598in}}%
\pgfpathlineto{\pgfqpoint{0.399634in}{4.323598in}}%
\pgfpathclose%
\pgfusepath{stroke}%
\end{pgfscope}%
\begin{pgfscope}%
\pgfpathrectangle{\pgfqpoint{0.393053in}{0.375000in}}{\pgfqpoint{6.356833in}{5.175000in}}%
\pgfusepath{clip}%
\pgfsetbuttcap%
\pgfsetroundjoin%
\pgfsetlinewidth{1.003750pt}%
\definecolor{currentstroke}{rgb}{0.827451,0.827451,0.827451}%
\pgfsetstrokecolor{currentstroke}%
\pgfsetdash{}{0pt}%
\pgfpathmoveto{\pgfqpoint{2.920897in}{0.851071in}}%
\pgfpathcurveto{\pgfqpoint{2.931947in}{0.851071in}}{\pgfqpoint{2.942546in}{0.855462in}}{\pgfqpoint{2.950360in}{0.863275in}}%
\pgfpathcurveto{\pgfqpoint{2.958173in}{0.871089in}}{\pgfqpoint{2.962563in}{0.881688in}}{\pgfqpoint{2.962563in}{0.892738in}}%
\pgfpathcurveto{\pgfqpoint{2.962563in}{0.903788in}}{\pgfqpoint{2.958173in}{0.914387in}}{\pgfqpoint{2.950360in}{0.922201in}}%
\pgfpathcurveto{\pgfqpoint{2.942546in}{0.930014in}}{\pgfqpoint{2.931947in}{0.934405in}}{\pgfqpoint{2.920897in}{0.934405in}}%
\pgfpathcurveto{\pgfqpoint{2.909847in}{0.934405in}}{\pgfqpoint{2.899248in}{0.930014in}}{\pgfqpoint{2.891434in}{0.922201in}}%
\pgfpathcurveto{\pgfqpoint{2.883620in}{0.914387in}}{\pgfqpoint{2.879230in}{0.903788in}}{\pgfqpoint{2.879230in}{0.892738in}}%
\pgfpathcurveto{\pgfqpoint{2.879230in}{0.881688in}}{\pgfqpoint{2.883620in}{0.871089in}}{\pgfqpoint{2.891434in}{0.863275in}}%
\pgfpathcurveto{\pgfqpoint{2.899248in}{0.855462in}}{\pgfqpoint{2.909847in}{0.851071in}}{\pgfqpoint{2.920897in}{0.851071in}}%
\pgfpathlineto{\pgfqpoint{2.920897in}{0.851071in}}%
\pgfpathclose%
\pgfusepath{stroke}%
\end{pgfscope}%
\begin{pgfscope}%
\pgfpathrectangle{\pgfqpoint{0.393053in}{0.375000in}}{\pgfqpoint{6.356833in}{5.175000in}}%
\pgfusepath{clip}%
\pgfsetbuttcap%
\pgfsetroundjoin%
\pgfsetlinewidth{1.003750pt}%
\definecolor{currentstroke}{rgb}{0.827451,0.827451,0.827451}%
\pgfsetstrokecolor{currentstroke}%
\pgfsetdash{}{0pt}%
\pgfpathmoveto{\pgfqpoint{1.881292in}{1.456529in}}%
\pgfpathcurveto{\pgfqpoint{1.892342in}{1.456529in}}{\pgfqpoint{1.902941in}{1.460919in}}{\pgfqpoint{1.910755in}{1.468732in}}%
\pgfpathcurveto{\pgfqpoint{1.918569in}{1.476546in}}{\pgfqpoint{1.922959in}{1.487145in}}{\pgfqpoint{1.922959in}{1.498195in}}%
\pgfpathcurveto{\pgfqpoint{1.922959in}{1.509245in}}{\pgfqpoint{1.918569in}{1.519844in}}{\pgfqpoint{1.910755in}{1.527658in}}%
\pgfpathcurveto{\pgfqpoint{1.902941in}{1.535472in}}{\pgfqpoint{1.892342in}{1.539862in}}{\pgfqpoint{1.881292in}{1.539862in}}%
\pgfpathcurveto{\pgfqpoint{1.870242in}{1.539862in}}{\pgfqpoint{1.859643in}{1.535472in}}{\pgfqpoint{1.851829in}{1.527658in}}%
\pgfpathcurveto{\pgfqpoint{1.844016in}{1.519844in}}{\pgfqpoint{1.839626in}{1.509245in}}{\pgfqpoint{1.839626in}{1.498195in}}%
\pgfpathcurveto{\pgfqpoint{1.839626in}{1.487145in}}{\pgfqpoint{1.844016in}{1.476546in}}{\pgfqpoint{1.851829in}{1.468732in}}%
\pgfpathcurveto{\pgfqpoint{1.859643in}{1.460919in}}{\pgfqpoint{1.870242in}{1.456529in}}{\pgfqpoint{1.881292in}{1.456529in}}%
\pgfpathlineto{\pgfqpoint{1.881292in}{1.456529in}}%
\pgfpathclose%
\pgfusepath{stroke}%
\end{pgfscope}%
\begin{pgfscope}%
\pgfpathrectangle{\pgfqpoint{0.393053in}{0.375000in}}{\pgfqpoint{6.356833in}{5.175000in}}%
\pgfusepath{clip}%
\pgfsetbuttcap%
\pgfsetroundjoin%
\pgfsetlinewidth{1.003750pt}%
\definecolor{currentstroke}{rgb}{0.827451,0.827451,0.827451}%
\pgfsetstrokecolor{currentstroke}%
\pgfsetdash{}{0pt}%
\pgfpathmoveto{\pgfqpoint{4.012103in}{0.498786in}}%
\pgfpathcurveto{\pgfqpoint{4.023153in}{0.498786in}}{\pgfqpoint{4.033752in}{0.503177in}}{\pgfqpoint{4.041566in}{0.510990in}}%
\pgfpathcurveto{\pgfqpoint{4.049379in}{0.518804in}}{\pgfqpoint{4.053769in}{0.529403in}}{\pgfqpoint{4.053769in}{0.540453in}}%
\pgfpathcurveto{\pgfqpoint{4.053769in}{0.551503in}}{\pgfqpoint{4.049379in}{0.562102in}}{\pgfqpoint{4.041566in}{0.569916in}}%
\pgfpathcurveto{\pgfqpoint{4.033752in}{0.577729in}}{\pgfqpoint{4.023153in}{0.582120in}}{\pgfqpoint{4.012103in}{0.582120in}}%
\pgfpathcurveto{\pgfqpoint{4.001053in}{0.582120in}}{\pgfqpoint{3.990454in}{0.577729in}}{\pgfqpoint{3.982640in}{0.569916in}}%
\pgfpathcurveto{\pgfqpoint{3.974826in}{0.562102in}}{\pgfqpoint{3.970436in}{0.551503in}}{\pgfqpoint{3.970436in}{0.540453in}}%
\pgfpathcurveto{\pgfqpoint{3.970436in}{0.529403in}}{\pgfqpoint{3.974826in}{0.518804in}}{\pgfqpoint{3.982640in}{0.510990in}}%
\pgfpathcurveto{\pgfqpoint{3.990454in}{0.503177in}}{\pgfqpoint{4.001053in}{0.498786in}}{\pgfqpoint{4.012103in}{0.498786in}}%
\pgfpathlineto{\pgfqpoint{4.012103in}{0.498786in}}%
\pgfpathclose%
\pgfusepath{stroke}%
\end{pgfscope}%
\begin{pgfscope}%
\pgfpathrectangle{\pgfqpoint{0.393053in}{0.375000in}}{\pgfqpoint{6.356833in}{5.175000in}}%
\pgfusepath{clip}%
\pgfsetbuttcap%
\pgfsetroundjoin%
\pgfsetlinewidth{1.003750pt}%
\definecolor{currentstroke}{rgb}{0.827451,0.827451,0.827451}%
\pgfsetstrokecolor{currentstroke}%
\pgfsetdash{}{0pt}%
\pgfpathmoveto{\pgfqpoint{1.702004in}{1.599430in}}%
\pgfpathcurveto{\pgfqpoint{1.713054in}{1.599430in}}{\pgfqpoint{1.723653in}{1.603820in}}{\pgfqpoint{1.731467in}{1.611633in}}%
\pgfpathcurveto{\pgfqpoint{1.739281in}{1.619447in}}{\pgfqpoint{1.743671in}{1.630046in}}{\pgfqpoint{1.743671in}{1.641096in}}%
\pgfpathcurveto{\pgfqpoint{1.743671in}{1.652146in}}{\pgfqpoint{1.739281in}{1.662745in}}{\pgfqpoint{1.731467in}{1.670559in}}%
\pgfpathcurveto{\pgfqpoint{1.723653in}{1.678373in}}{\pgfqpoint{1.713054in}{1.682763in}}{\pgfqpoint{1.702004in}{1.682763in}}%
\pgfpathcurveto{\pgfqpoint{1.690954in}{1.682763in}}{\pgfqpoint{1.680355in}{1.678373in}}{\pgfqpoint{1.672541in}{1.670559in}}%
\pgfpathcurveto{\pgfqpoint{1.664728in}{1.662745in}}{\pgfqpoint{1.660338in}{1.652146in}}{\pgfqpoint{1.660338in}{1.641096in}}%
\pgfpathcurveto{\pgfqpoint{1.660338in}{1.630046in}}{\pgfqpoint{1.664728in}{1.619447in}}{\pgfqpoint{1.672541in}{1.611633in}}%
\pgfpathcurveto{\pgfqpoint{1.680355in}{1.603820in}}{\pgfqpoint{1.690954in}{1.599430in}}{\pgfqpoint{1.702004in}{1.599430in}}%
\pgfpathlineto{\pgfqpoint{1.702004in}{1.599430in}}%
\pgfpathclose%
\pgfusepath{stroke}%
\end{pgfscope}%
\begin{pgfscope}%
\pgfpathrectangle{\pgfqpoint{0.393053in}{0.375000in}}{\pgfqpoint{6.356833in}{5.175000in}}%
\pgfusepath{clip}%
\pgfsetbuttcap%
\pgfsetroundjoin%
\pgfsetlinewidth{1.003750pt}%
\definecolor{currentstroke}{rgb}{0.827451,0.827451,0.827451}%
\pgfsetstrokecolor{currentstroke}%
\pgfsetdash{}{0pt}%
\pgfpathmoveto{\pgfqpoint{5.144050in}{0.350542in}}%
\pgfpathcurveto{\pgfqpoint{5.155100in}{0.350542in}}{\pgfqpoint{5.165699in}{0.354932in}}{\pgfqpoint{5.173513in}{0.362746in}}%
\pgfpathcurveto{\pgfqpoint{5.181327in}{0.370560in}}{\pgfqpoint{5.185717in}{0.381159in}}{\pgfqpoint{5.185717in}{0.392209in}}%
\pgfpathcurveto{\pgfqpoint{5.185717in}{0.403259in}}{\pgfqpoint{5.181327in}{0.413858in}}{\pgfqpoint{5.173513in}{0.421672in}}%
\pgfpathcurveto{\pgfqpoint{5.165699in}{0.429485in}}{\pgfqpoint{5.155100in}{0.433875in}}{\pgfqpoint{5.144050in}{0.433875in}}%
\pgfpathcurveto{\pgfqpoint{5.133000in}{0.433875in}}{\pgfqpoint{5.122401in}{0.429485in}}{\pgfqpoint{5.114587in}{0.421672in}}%
\pgfpathcurveto{\pgfqpoint{5.106774in}{0.413858in}}{\pgfqpoint{5.102383in}{0.403259in}}{\pgfqpoint{5.102383in}{0.392209in}}%
\pgfpathcurveto{\pgfqpoint{5.102383in}{0.381159in}}{\pgfqpoint{5.106774in}{0.370560in}}{\pgfqpoint{5.114587in}{0.362746in}}%
\pgfpathcurveto{\pgfqpoint{5.122401in}{0.354932in}}{\pgfqpoint{5.133000in}{0.350542in}}{\pgfqpoint{5.144050in}{0.350542in}}%
\pgfusepath{stroke}%
\end{pgfscope}%
\begin{pgfscope}%
\pgfpathrectangle{\pgfqpoint{0.393053in}{0.375000in}}{\pgfqpoint{6.356833in}{5.175000in}}%
\pgfusepath{clip}%
\pgfsetbuttcap%
\pgfsetroundjoin%
\pgfsetlinewidth{1.003750pt}%
\definecolor{currentstroke}{rgb}{0.827451,0.827451,0.827451}%
\pgfsetstrokecolor{currentstroke}%
\pgfsetdash{}{0pt}%
\pgfpathmoveto{\pgfqpoint{4.802412in}{0.377777in}}%
\pgfpathcurveto{\pgfqpoint{4.813462in}{0.377777in}}{\pgfqpoint{4.824061in}{0.382167in}}{\pgfqpoint{4.831874in}{0.389981in}}%
\pgfpathcurveto{\pgfqpoint{4.839688in}{0.397794in}}{\pgfqpoint{4.844078in}{0.408394in}}{\pgfqpoint{4.844078in}{0.419444in}}%
\pgfpathcurveto{\pgfqpoint{4.844078in}{0.430494in}}{\pgfqpoint{4.839688in}{0.441093in}}{\pgfqpoint{4.831874in}{0.448906in}}%
\pgfpathcurveto{\pgfqpoint{4.824061in}{0.456720in}}{\pgfqpoint{4.813462in}{0.461110in}}{\pgfqpoint{4.802412in}{0.461110in}}%
\pgfpathcurveto{\pgfqpoint{4.791362in}{0.461110in}}{\pgfqpoint{4.780762in}{0.456720in}}{\pgfqpoint{4.772949in}{0.448906in}}%
\pgfpathcurveto{\pgfqpoint{4.765135in}{0.441093in}}{\pgfqpoint{4.760745in}{0.430494in}}{\pgfqpoint{4.760745in}{0.419444in}}%
\pgfpathcurveto{\pgfqpoint{4.760745in}{0.408394in}}{\pgfqpoint{4.765135in}{0.397794in}}{\pgfqpoint{4.772949in}{0.389981in}}%
\pgfpathcurveto{\pgfqpoint{4.780762in}{0.382167in}}{\pgfqpoint{4.791362in}{0.377777in}}{\pgfqpoint{4.802412in}{0.377777in}}%
\pgfpathlineto{\pgfqpoint{4.802412in}{0.377777in}}%
\pgfpathclose%
\pgfusepath{stroke}%
\end{pgfscope}%
\begin{pgfscope}%
\pgfpathrectangle{\pgfqpoint{0.393053in}{0.375000in}}{\pgfqpoint{6.356833in}{5.175000in}}%
\pgfusepath{clip}%
\pgfsetbuttcap%
\pgfsetroundjoin%
\pgfsetlinewidth{1.003750pt}%
\definecolor{currentstroke}{rgb}{0.827451,0.827451,0.827451}%
\pgfsetstrokecolor{currentstroke}%
\pgfsetdash{}{0pt}%
\pgfpathmoveto{\pgfqpoint{2.675195in}{0.994248in}}%
\pgfpathcurveto{\pgfqpoint{2.686245in}{0.994248in}}{\pgfqpoint{2.696845in}{0.998639in}}{\pgfqpoint{2.704658in}{1.006452in}}%
\pgfpathcurveto{\pgfqpoint{2.712472in}{1.014266in}}{\pgfqpoint{2.716862in}{1.024865in}}{\pgfqpoint{2.716862in}{1.035915in}}%
\pgfpathcurveto{\pgfqpoint{2.716862in}{1.046965in}}{\pgfqpoint{2.712472in}{1.057564in}}{\pgfqpoint{2.704658in}{1.065378in}}%
\pgfpathcurveto{\pgfqpoint{2.696845in}{1.073191in}}{\pgfqpoint{2.686245in}{1.077582in}}{\pgfqpoint{2.675195in}{1.077582in}}%
\pgfpathcurveto{\pgfqpoint{2.664145in}{1.077582in}}{\pgfqpoint{2.653546in}{1.073191in}}{\pgfqpoint{2.645733in}{1.065378in}}%
\pgfpathcurveto{\pgfqpoint{2.637919in}{1.057564in}}{\pgfqpoint{2.633529in}{1.046965in}}{\pgfqpoint{2.633529in}{1.035915in}}%
\pgfpathcurveto{\pgfqpoint{2.633529in}{1.024865in}}{\pgfqpoint{2.637919in}{1.014266in}}{\pgfqpoint{2.645733in}{1.006452in}}%
\pgfpathcurveto{\pgfqpoint{2.653546in}{0.998639in}}{\pgfqpoint{2.664145in}{0.994248in}}{\pgfqpoint{2.675195in}{0.994248in}}%
\pgfpathlineto{\pgfqpoint{2.675195in}{0.994248in}}%
\pgfpathclose%
\pgfusepath{stroke}%
\end{pgfscope}%
\begin{pgfscope}%
\pgfpathrectangle{\pgfqpoint{0.393053in}{0.375000in}}{\pgfqpoint{6.356833in}{5.175000in}}%
\pgfusepath{clip}%
\pgfsetbuttcap%
\pgfsetroundjoin%
\pgfsetlinewidth{1.003750pt}%
\definecolor{currentstroke}{rgb}{0.827451,0.827451,0.827451}%
\pgfsetstrokecolor{currentstroke}%
\pgfsetdash{}{0pt}%
\pgfpathmoveto{\pgfqpoint{3.637654in}{0.581771in}}%
\pgfpathcurveto{\pgfqpoint{3.648705in}{0.581771in}}{\pgfqpoint{3.659304in}{0.586161in}}{\pgfqpoint{3.667117in}{0.593974in}}%
\pgfpathcurveto{\pgfqpoint{3.674931in}{0.601788in}}{\pgfqpoint{3.679321in}{0.612387in}}{\pgfqpoint{3.679321in}{0.623437in}}%
\pgfpathcurveto{\pgfqpoint{3.679321in}{0.634487in}}{\pgfqpoint{3.674931in}{0.645086in}}{\pgfqpoint{3.667117in}{0.652900in}}%
\pgfpathcurveto{\pgfqpoint{3.659304in}{0.660714in}}{\pgfqpoint{3.648705in}{0.665104in}}{\pgfqpoint{3.637654in}{0.665104in}}%
\pgfpathcurveto{\pgfqpoint{3.626604in}{0.665104in}}{\pgfqpoint{3.616005in}{0.660714in}}{\pgfqpoint{3.608192in}{0.652900in}}%
\pgfpathcurveto{\pgfqpoint{3.600378in}{0.645086in}}{\pgfqpoint{3.595988in}{0.634487in}}{\pgfqpoint{3.595988in}{0.623437in}}%
\pgfpathcurveto{\pgfqpoint{3.595988in}{0.612387in}}{\pgfqpoint{3.600378in}{0.601788in}}{\pgfqpoint{3.608192in}{0.593974in}}%
\pgfpathcurveto{\pgfqpoint{3.616005in}{0.586161in}}{\pgfqpoint{3.626604in}{0.581771in}}{\pgfqpoint{3.637654in}{0.581771in}}%
\pgfpathlineto{\pgfqpoint{3.637654in}{0.581771in}}%
\pgfpathclose%
\pgfusepath{stroke}%
\end{pgfscope}%
\begin{pgfscope}%
\pgfpathrectangle{\pgfqpoint{0.393053in}{0.375000in}}{\pgfqpoint{6.356833in}{5.175000in}}%
\pgfusepath{clip}%
\pgfsetbuttcap%
\pgfsetroundjoin%
\pgfsetlinewidth{1.003750pt}%
\definecolor{currentstroke}{rgb}{0.827451,0.827451,0.827451}%
\pgfsetstrokecolor{currentstroke}%
\pgfsetdash{}{0pt}%
\pgfpathmoveto{\pgfqpoint{4.512237in}{0.406694in}}%
\pgfpathcurveto{\pgfqpoint{4.523287in}{0.406694in}}{\pgfqpoint{4.533886in}{0.411084in}}{\pgfqpoint{4.541700in}{0.418898in}}%
\pgfpathcurveto{\pgfqpoint{4.549514in}{0.426711in}}{\pgfqpoint{4.553904in}{0.437310in}}{\pgfqpoint{4.553904in}{0.448360in}}%
\pgfpathcurveto{\pgfqpoint{4.553904in}{0.459410in}}{\pgfqpoint{4.549514in}{0.470010in}}{\pgfqpoint{4.541700in}{0.477823in}}%
\pgfpathcurveto{\pgfqpoint{4.533886in}{0.485637in}}{\pgfqpoint{4.523287in}{0.490027in}}{\pgfqpoint{4.512237in}{0.490027in}}%
\pgfpathcurveto{\pgfqpoint{4.501187in}{0.490027in}}{\pgfqpoint{4.490588in}{0.485637in}}{\pgfqpoint{4.482774in}{0.477823in}}%
\pgfpathcurveto{\pgfqpoint{4.474961in}{0.470010in}}{\pgfqpoint{4.470570in}{0.459410in}}{\pgfqpoint{4.470570in}{0.448360in}}%
\pgfpathcurveto{\pgfqpoint{4.470570in}{0.437310in}}{\pgfqpoint{4.474961in}{0.426711in}}{\pgfqpoint{4.482774in}{0.418898in}}%
\pgfpathcurveto{\pgfqpoint{4.490588in}{0.411084in}}{\pgfqpoint{4.501187in}{0.406694in}}{\pgfqpoint{4.512237in}{0.406694in}}%
\pgfpathlineto{\pgfqpoint{4.512237in}{0.406694in}}%
\pgfpathclose%
\pgfusepath{stroke}%
\end{pgfscope}%
\begin{pgfscope}%
\pgfpathrectangle{\pgfqpoint{0.393053in}{0.375000in}}{\pgfqpoint{6.356833in}{5.175000in}}%
\pgfusepath{clip}%
\pgfsetbuttcap%
\pgfsetroundjoin%
\pgfsetlinewidth{1.003750pt}%
\definecolor{currentstroke}{rgb}{0.827451,0.827451,0.827451}%
\pgfsetstrokecolor{currentstroke}%
\pgfsetdash{}{0pt}%
\pgfpathmoveto{\pgfqpoint{2.785690in}{0.904937in}}%
\pgfpathcurveto{\pgfqpoint{2.796740in}{0.904937in}}{\pgfqpoint{2.807339in}{0.909327in}}{\pgfqpoint{2.815153in}{0.917140in}}%
\pgfpathcurveto{\pgfqpoint{2.822966in}{0.924954in}}{\pgfqpoint{2.827357in}{0.935553in}}{\pgfqpoint{2.827357in}{0.946603in}}%
\pgfpathcurveto{\pgfqpoint{2.827357in}{0.957653in}}{\pgfqpoint{2.822966in}{0.968252in}}{\pgfqpoint{2.815153in}{0.976066in}}%
\pgfpathcurveto{\pgfqpoint{2.807339in}{0.983880in}}{\pgfqpoint{2.796740in}{0.988270in}}{\pgfqpoint{2.785690in}{0.988270in}}%
\pgfpathcurveto{\pgfqpoint{2.774640in}{0.988270in}}{\pgfqpoint{2.764041in}{0.983880in}}{\pgfqpoint{2.756227in}{0.976066in}}%
\pgfpathcurveto{\pgfqpoint{2.748414in}{0.968252in}}{\pgfqpoint{2.744023in}{0.957653in}}{\pgfqpoint{2.744023in}{0.946603in}}%
\pgfpathcurveto{\pgfqpoint{2.744023in}{0.935553in}}{\pgfqpoint{2.748414in}{0.924954in}}{\pgfqpoint{2.756227in}{0.917140in}}%
\pgfpathcurveto{\pgfqpoint{2.764041in}{0.909327in}}{\pgfqpoint{2.774640in}{0.904937in}}{\pgfqpoint{2.785690in}{0.904937in}}%
\pgfpathlineto{\pgfqpoint{2.785690in}{0.904937in}}%
\pgfpathclose%
\pgfusepath{stroke}%
\end{pgfscope}%
\begin{pgfscope}%
\pgfpathrectangle{\pgfqpoint{0.393053in}{0.375000in}}{\pgfqpoint{6.356833in}{5.175000in}}%
\pgfusepath{clip}%
\pgfsetbuttcap%
\pgfsetroundjoin%
\pgfsetlinewidth{1.003750pt}%
\definecolor{currentstroke}{rgb}{0.827451,0.827451,0.827451}%
\pgfsetstrokecolor{currentstroke}%
\pgfsetdash{}{0pt}%
\pgfpathmoveto{\pgfqpoint{2.310226in}{1.162048in}}%
\pgfpathcurveto{\pgfqpoint{2.321276in}{1.162048in}}{\pgfqpoint{2.331875in}{1.166439in}}{\pgfqpoint{2.339688in}{1.174252in}}%
\pgfpathcurveto{\pgfqpoint{2.347502in}{1.182066in}}{\pgfqpoint{2.351892in}{1.192665in}}{\pgfqpoint{2.351892in}{1.203715in}}%
\pgfpathcurveto{\pgfqpoint{2.351892in}{1.214765in}}{\pgfqpoint{2.347502in}{1.225364in}}{\pgfqpoint{2.339688in}{1.233178in}}%
\pgfpathcurveto{\pgfqpoint{2.331875in}{1.240992in}}{\pgfqpoint{2.321276in}{1.245382in}}{\pgfqpoint{2.310226in}{1.245382in}}%
\pgfpathcurveto{\pgfqpoint{2.299175in}{1.245382in}}{\pgfqpoint{2.288576in}{1.240992in}}{\pgfqpoint{2.280763in}{1.233178in}}%
\pgfpathcurveto{\pgfqpoint{2.272949in}{1.225364in}}{\pgfqpoint{2.268559in}{1.214765in}}{\pgfqpoint{2.268559in}{1.203715in}}%
\pgfpathcurveto{\pgfqpoint{2.268559in}{1.192665in}}{\pgfqpoint{2.272949in}{1.182066in}}{\pgfqpoint{2.280763in}{1.174252in}}%
\pgfpathcurveto{\pgfqpoint{2.288576in}{1.166439in}}{\pgfqpoint{2.299175in}{1.162048in}}{\pgfqpoint{2.310226in}{1.162048in}}%
\pgfpathlineto{\pgfqpoint{2.310226in}{1.162048in}}%
\pgfpathclose%
\pgfusepath{stroke}%
\end{pgfscope}%
\begin{pgfscope}%
\pgfpathrectangle{\pgfqpoint{0.393053in}{0.375000in}}{\pgfqpoint{6.356833in}{5.175000in}}%
\pgfusepath{clip}%
\pgfsetbuttcap%
\pgfsetroundjoin%
\pgfsetlinewidth{1.003750pt}%
\definecolor{currentstroke}{rgb}{0.827451,0.827451,0.827451}%
\pgfsetstrokecolor{currentstroke}%
\pgfsetdash{}{0pt}%
\pgfpathmoveto{\pgfqpoint{1.626356in}{1.668438in}}%
\pgfpathcurveto{\pgfqpoint{1.637406in}{1.668438in}}{\pgfqpoint{1.648005in}{1.672828in}}{\pgfqpoint{1.655818in}{1.680641in}}%
\pgfpathcurveto{\pgfqpoint{1.663632in}{1.688455in}}{\pgfqpoint{1.668022in}{1.699054in}}{\pgfqpoint{1.668022in}{1.710104in}}%
\pgfpathcurveto{\pgfqpoint{1.668022in}{1.721154in}}{\pgfqpoint{1.663632in}{1.731753in}}{\pgfqpoint{1.655818in}{1.739567in}}%
\pgfpathcurveto{\pgfqpoint{1.648005in}{1.747381in}}{\pgfqpoint{1.637406in}{1.751771in}}{\pgfqpoint{1.626356in}{1.751771in}}%
\pgfpathcurveto{\pgfqpoint{1.615305in}{1.751771in}}{\pgfqpoint{1.604706in}{1.747381in}}{\pgfqpoint{1.596893in}{1.739567in}}%
\pgfpathcurveto{\pgfqpoint{1.589079in}{1.731753in}}{\pgfqpoint{1.584689in}{1.721154in}}{\pgfqpoint{1.584689in}{1.710104in}}%
\pgfpathcurveto{\pgfqpoint{1.584689in}{1.699054in}}{\pgfqpoint{1.589079in}{1.688455in}}{\pgfqpoint{1.596893in}{1.680641in}}%
\pgfpathcurveto{\pgfqpoint{1.604706in}{1.672828in}}{\pgfqpoint{1.615305in}{1.668438in}}{\pgfqpoint{1.626356in}{1.668438in}}%
\pgfpathlineto{\pgfqpoint{1.626356in}{1.668438in}}%
\pgfpathclose%
\pgfusepath{stroke}%
\end{pgfscope}%
\begin{pgfscope}%
\pgfpathrectangle{\pgfqpoint{0.393053in}{0.375000in}}{\pgfqpoint{6.356833in}{5.175000in}}%
\pgfusepath{clip}%
\pgfsetbuttcap%
\pgfsetroundjoin%
\pgfsetlinewidth{1.003750pt}%
\definecolor{currentstroke}{rgb}{0.827451,0.827451,0.827451}%
\pgfsetstrokecolor{currentstroke}%
\pgfsetdash{}{0pt}%
\pgfpathmoveto{\pgfqpoint{0.875375in}{2.588406in}}%
\pgfpathcurveto{\pgfqpoint{0.886425in}{2.588406in}}{\pgfqpoint{0.897024in}{2.592796in}}{\pgfqpoint{0.904838in}{2.600609in}}%
\pgfpathcurveto{\pgfqpoint{0.912652in}{2.608423in}}{\pgfqpoint{0.917042in}{2.619022in}}{\pgfqpoint{0.917042in}{2.630072in}}%
\pgfpathcurveto{\pgfqpoint{0.917042in}{2.641122in}}{\pgfqpoint{0.912652in}{2.651721in}}{\pgfqpoint{0.904838in}{2.659535in}}%
\pgfpathcurveto{\pgfqpoint{0.897024in}{2.667349in}}{\pgfqpoint{0.886425in}{2.671739in}}{\pgfqpoint{0.875375in}{2.671739in}}%
\pgfpathcurveto{\pgfqpoint{0.864325in}{2.671739in}}{\pgfqpoint{0.853726in}{2.667349in}}{\pgfqpoint{0.845912in}{2.659535in}}%
\pgfpathcurveto{\pgfqpoint{0.838099in}{2.651721in}}{\pgfqpoint{0.833708in}{2.641122in}}{\pgfqpoint{0.833708in}{2.630072in}}%
\pgfpathcurveto{\pgfqpoint{0.833708in}{2.619022in}}{\pgfqpoint{0.838099in}{2.608423in}}{\pgfqpoint{0.845912in}{2.600609in}}%
\pgfpathcurveto{\pgfqpoint{0.853726in}{2.592796in}}{\pgfqpoint{0.864325in}{2.588406in}}{\pgfqpoint{0.875375in}{2.588406in}}%
\pgfpathlineto{\pgfqpoint{0.875375in}{2.588406in}}%
\pgfpathclose%
\pgfusepath{stroke}%
\end{pgfscope}%
\begin{pgfscope}%
\pgfpathrectangle{\pgfqpoint{0.393053in}{0.375000in}}{\pgfqpoint{6.356833in}{5.175000in}}%
\pgfusepath{clip}%
\pgfsetbuttcap%
\pgfsetroundjoin%
\pgfsetlinewidth{1.003750pt}%
\definecolor{currentstroke}{rgb}{0.827451,0.827451,0.827451}%
\pgfsetstrokecolor{currentstroke}%
\pgfsetdash{}{0pt}%
\pgfpathmoveto{\pgfqpoint{1.768437in}{1.558678in}}%
\pgfpathcurveto{\pgfqpoint{1.779488in}{1.558678in}}{\pgfqpoint{1.790087in}{1.563068in}}{\pgfqpoint{1.797900in}{1.570882in}}%
\pgfpathcurveto{\pgfqpoint{1.805714in}{1.578696in}}{\pgfqpoint{1.810104in}{1.589295in}}{\pgfqpoint{1.810104in}{1.600345in}}%
\pgfpathcurveto{\pgfqpoint{1.810104in}{1.611395in}}{\pgfqpoint{1.805714in}{1.621994in}}{\pgfqpoint{1.797900in}{1.629807in}}%
\pgfpathcurveto{\pgfqpoint{1.790087in}{1.637621in}}{\pgfqpoint{1.779488in}{1.642011in}}{\pgfqpoint{1.768437in}{1.642011in}}%
\pgfpathcurveto{\pgfqpoint{1.757387in}{1.642011in}}{\pgfqpoint{1.746788in}{1.637621in}}{\pgfqpoint{1.738975in}{1.629807in}}%
\pgfpathcurveto{\pgfqpoint{1.731161in}{1.621994in}}{\pgfqpoint{1.726771in}{1.611395in}}{\pgfqpoint{1.726771in}{1.600345in}}%
\pgfpathcurveto{\pgfqpoint{1.726771in}{1.589295in}}{\pgfqpoint{1.731161in}{1.578696in}}{\pgfqpoint{1.738975in}{1.570882in}}%
\pgfpathcurveto{\pgfqpoint{1.746788in}{1.563068in}}{\pgfqpoint{1.757387in}{1.558678in}}{\pgfqpoint{1.768437in}{1.558678in}}%
\pgfpathlineto{\pgfqpoint{1.768437in}{1.558678in}}%
\pgfpathclose%
\pgfusepath{stroke}%
\end{pgfscope}%
\begin{pgfscope}%
\pgfpathrectangle{\pgfqpoint{0.393053in}{0.375000in}}{\pgfqpoint{6.356833in}{5.175000in}}%
\pgfusepath{clip}%
\pgfsetbuttcap%
\pgfsetroundjoin%
\pgfsetlinewidth{1.003750pt}%
\definecolor{currentstroke}{rgb}{0.827451,0.827451,0.827451}%
\pgfsetstrokecolor{currentstroke}%
\pgfsetdash{}{0pt}%
\pgfpathmoveto{\pgfqpoint{4.659727in}{0.389065in}}%
\pgfpathcurveto{\pgfqpoint{4.670777in}{0.389065in}}{\pgfqpoint{4.681376in}{0.393455in}}{\pgfqpoint{4.689190in}{0.401269in}}%
\pgfpathcurveto{\pgfqpoint{4.697004in}{0.409082in}}{\pgfqpoint{4.701394in}{0.419682in}}{\pgfqpoint{4.701394in}{0.430732in}}%
\pgfpathcurveto{\pgfqpoint{4.701394in}{0.441782in}}{\pgfqpoint{4.697004in}{0.452381in}}{\pgfqpoint{4.689190in}{0.460194in}}%
\pgfpathcurveto{\pgfqpoint{4.681376in}{0.468008in}}{\pgfqpoint{4.670777in}{0.472398in}}{\pgfqpoint{4.659727in}{0.472398in}}%
\pgfpathcurveto{\pgfqpoint{4.648677in}{0.472398in}}{\pgfqpoint{4.638078in}{0.468008in}}{\pgfqpoint{4.630264in}{0.460194in}}%
\pgfpathcurveto{\pgfqpoint{4.622451in}{0.452381in}}{\pgfqpoint{4.618060in}{0.441782in}}{\pgfqpoint{4.618060in}{0.430732in}}%
\pgfpathcurveto{\pgfqpoint{4.618060in}{0.419682in}}{\pgfqpoint{4.622451in}{0.409082in}}{\pgfqpoint{4.630264in}{0.401269in}}%
\pgfpathcurveto{\pgfqpoint{4.638078in}{0.393455in}}{\pgfqpoint{4.648677in}{0.389065in}}{\pgfqpoint{4.659727in}{0.389065in}}%
\pgfpathlineto{\pgfqpoint{4.659727in}{0.389065in}}%
\pgfpathclose%
\pgfusepath{stroke}%
\end{pgfscope}%
\begin{pgfscope}%
\pgfpathrectangle{\pgfqpoint{0.393053in}{0.375000in}}{\pgfqpoint{6.356833in}{5.175000in}}%
\pgfusepath{clip}%
\pgfsetbuttcap%
\pgfsetroundjoin%
\pgfsetlinewidth{1.003750pt}%
\definecolor{currentstroke}{rgb}{0.827451,0.827451,0.827451}%
\pgfsetstrokecolor{currentstroke}%
\pgfsetdash{}{0pt}%
\pgfpathmoveto{\pgfqpoint{2.584660in}{1.053636in}}%
\pgfpathcurveto{\pgfqpoint{2.595711in}{1.053636in}}{\pgfqpoint{2.606310in}{1.058027in}}{\pgfqpoint{2.614123in}{1.065840in}}%
\pgfpathcurveto{\pgfqpoint{2.621937in}{1.073654in}}{\pgfqpoint{2.626327in}{1.084253in}}{\pgfqpoint{2.626327in}{1.095303in}}%
\pgfpathcurveto{\pgfqpoint{2.626327in}{1.106353in}}{\pgfqpoint{2.621937in}{1.116952in}}{\pgfqpoint{2.614123in}{1.124766in}}%
\pgfpathcurveto{\pgfqpoint{2.606310in}{1.132580in}}{\pgfqpoint{2.595711in}{1.136970in}}{\pgfqpoint{2.584660in}{1.136970in}}%
\pgfpathcurveto{\pgfqpoint{2.573610in}{1.136970in}}{\pgfqpoint{2.563011in}{1.132580in}}{\pgfqpoint{2.555198in}{1.124766in}}%
\pgfpathcurveto{\pgfqpoint{2.547384in}{1.116952in}}{\pgfqpoint{2.542994in}{1.106353in}}{\pgfqpoint{2.542994in}{1.095303in}}%
\pgfpathcurveto{\pgfqpoint{2.542994in}{1.084253in}}{\pgfqpoint{2.547384in}{1.073654in}}{\pgfqpoint{2.555198in}{1.065840in}}%
\pgfpathcurveto{\pgfqpoint{2.563011in}{1.058027in}}{\pgfqpoint{2.573610in}{1.053636in}}{\pgfqpoint{2.584660in}{1.053636in}}%
\pgfpathlineto{\pgfqpoint{2.584660in}{1.053636in}}%
\pgfpathclose%
\pgfusepath{stroke}%
\end{pgfscope}%
\begin{pgfscope}%
\pgfpathrectangle{\pgfqpoint{0.393053in}{0.375000in}}{\pgfqpoint{6.356833in}{5.175000in}}%
\pgfusepath{clip}%
\pgfsetbuttcap%
\pgfsetroundjoin%
\pgfsetlinewidth{1.003750pt}%
\definecolor{currentstroke}{rgb}{0.827451,0.827451,0.827451}%
\pgfsetstrokecolor{currentstroke}%
\pgfsetdash{}{0pt}%
\pgfpathmoveto{\pgfqpoint{3.496019in}{0.629204in}}%
\pgfpathcurveto{\pgfqpoint{3.507069in}{0.629204in}}{\pgfqpoint{3.517668in}{0.633594in}}{\pgfqpoint{3.525482in}{0.641407in}}%
\pgfpathcurveto{\pgfqpoint{3.533295in}{0.649221in}}{\pgfqpoint{3.537686in}{0.659820in}}{\pgfqpoint{3.537686in}{0.670870in}}%
\pgfpathcurveto{\pgfqpoint{3.537686in}{0.681920in}}{\pgfqpoint{3.533295in}{0.692519in}}{\pgfqpoint{3.525482in}{0.700333in}}%
\pgfpathcurveto{\pgfqpoint{3.517668in}{0.708147in}}{\pgfqpoint{3.507069in}{0.712537in}}{\pgfqpoint{3.496019in}{0.712537in}}%
\pgfpathcurveto{\pgfqpoint{3.484969in}{0.712537in}}{\pgfqpoint{3.474370in}{0.708147in}}{\pgfqpoint{3.466556in}{0.700333in}}%
\pgfpathcurveto{\pgfqpoint{3.458743in}{0.692519in}}{\pgfqpoint{3.454352in}{0.681920in}}{\pgfqpoint{3.454352in}{0.670870in}}%
\pgfpathcurveto{\pgfqpoint{3.454352in}{0.659820in}}{\pgfqpoint{3.458743in}{0.649221in}}{\pgfqpoint{3.466556in}{0.641407in}}%
\pgfpathcurveto{\pgfqpoint{3.474370in}{0.633594in}}{\pgfqpoint{3.484969in}{0.629204in}}{\pgfqpoint{3.496019in}{0.629204in}}%
\pgfpathlineto{\pgfqpoint{3.496019in}{0.629204in}}%
\pgfpathclose%
\pgfusepath{stroke}%
\end{pgfscope}%
\begin{pgfscope}%
\pgfpathrectangle{\pgfqpoint{0.393053in}{0.375000in}}{\pgfqpoint{6.356833in}{5.175000in}}%
\pgfusepath{clip}%
\pgfsetbuttcap%
\pgfsetroundjoin%
\pgfsetlinewidth{1.003750pt}%
\definecolor{currentstroke}{rgb}{0.827451,0.827451,0.827451}%
\pgfsetstrokecolor{currentstroke}%
\pgfsetdash{}{0pt}%
\pgfpathmoveto{\pgfqpoint{1.843211in}{1.515821in}}%
\pgfpathcurveto{\pgfqpoint{1.854261in}{1.515821in}}{\pgfqpoint{1.864860in}{1.520211in}}{\pgfqpoint{1.872674in}{1.528025in}}%
\pgfpathcurveto{\pgfqpoint{1.880487in}{1.535838in}}{\pgfqpoint{1.884878in}{1.546437in}}{\pgfqpoint{1.884878in}{1.557487in}}%
\pgfpathcurveto{\pgfqpoint{1.884878in}{1.568537in}}{\pgfqpoint{1.880487in}{1.579136in}}{\pgfqpoint{1.872674in}{1.586950in}}%
\pgfpathcurveto{\pgfqpoint{1.864860in}{1.594764in}}{\pgfqpoint{1.854261in}{1.599154in}}{\pgfqpoint{1.843211in}{1.599154in}}%
\pgfpathcurveto{\pgfqpoint{1.832161in}{1.599154in}}{\pgfqpoint{1.821562in}{1.594764in}}{\pgfqpoint{1.813748in}{1.586950in}}%
\pgfpathcurveto{\pgfqpoint{1.805934in}{1.579136in}}{\pgfqpoint{1.801544in}{1.568537in}}{\pgfqpoint{1.801544in}{1.557487in}}%
\pgfpathcurveto{\pgfqpoint{1.801544in}{1.546437in}}{\pgfqpoint{1.805934in}{1.535838in}}{\pgfqpoint{1.813748in}{1.528025in}}%
\pgfpathcurveto{\pgfqpoint{1.821562in}{1.520211in}}{\pgfqpoint{1.832161in}{1.515821in}}{\pgfqpoint{1.843211in}{1.515821in}}%
\pgfpathlineto{\pgfqpoint{1.843211in}{1.515821in}}%
\pgfpathclose%
\pgfusepath{stroke}%
\end{pgfscope}%
\begin{pgfscope}%
\pgfpathrectangle{\pgfqpoint{0.393053in}{0.375000in}}{\pgfqpoint{6.356833in}{5.175000in}}%
\pgfusepath{clip}%
\pgfsetbuttcap%
\pgfsetroundjoin%
\pgfsetlinewidth{1.003750pt}%
\definecolor{currentstroke}{rgb}{0.827451,0.827451,0.827451}%
\pgfsetstrokecolor{currentstroke}%
\pgfsetdash{}{0pt}%
\pgfpathmoveto{\pgfqpoint{0.411533in}{4.155351in}}%
\pgfpathcurveto{\pgfqpoint{0.422583in}{4.155351in}}{\pgfqpoint{0.433182in}{4.159741in}}{\pgfqpoint{0.440996in}{4.167555in}}%
\pgfpathcurveto{\pgfqpoint{0.448810in}{4.175369in}}{\pgfqpoint{0.453200in}{4.185968in}}{\pgfqpoint{0.453200in}{4.197018in}}%
\pgfpathcurveto{\pgfqpoint{0.453200in}{4.208068in}}{\pgfqpoint{0.448810in}{4.218667in}}{\pgfqpoint{0.440996in}{4.226481in}}%
\pgfpathcurveto{\pgfqpoint{0.433182in}{4.234294in}}{\pgfqpoint{0.422583in}{4.238685in}}{\pgfqpoint{0.411533in}{4.238685in}}%
\pgfpathcurveto{\pgfqpoint{0.400483in}{4.238685in}}{\pgfqpoint{0.389884in}{4.234294in}}{\pgfqpoint{0.382070in}{4.226481in}}%
\pgfpathcurveto{\pgfqpoint{0.374257in}{4.218667in}}{\pgfqpoint{0.369866in}{4.208068in}}{\pgfqpoint{0.369866in}{4.197018in}}%
\pgfpathcurveto{\pgfqpoint{0.369866in}{4.185968in}}{\pgfqpoint{0.374257in}{4.175369in}}{\pgfqpoint{0.382070in}{4.167555in}}%
\pgfpathcurveto{\pgfqpoint{0.389884in}{4.159741in}}{\pgfqpoint{0.400483in}{4.155351in}}{\pgfqpoint{0.411533in}{4.155351in}}%
\pgfpathlineto{\pgfqpoint{0.411533in}{4.155351in}}%
\pgfpathclose%
\pgfusepath{stroke}%
\end{pgfscope}%
\begin{pgfscope}%
\pgfpathrectangle{\pgfqpoint{0.393053in}{0.375000in}}{\pgfqpoint{6.356833in}{5.175000in}}%
\pgfusepath{clip}%
\pgfsetbuttcap%
\pgfsetroundjoin%
\pgfsetlinewidth{1.003750pt}%
\definecolor{currentstroke}{rgb}{0.827451,0.827451,0.827451}%
\pgfsetstrokecolor{currentstroke}%
\pgfsetdash{}{0pt}%
\pgfpathmoveto{\pgfqpoint{1.040104in}{2.359499in}}%
\pgfpathcurveto{\pgfqpoint{1.051154in}{2.359499in}}{\pgfqpoint{1.061753in}{2.363889in}}{\pgfqpoint{1.069567in}{2.371703in}}%
\pgfpathcurveto{\pgfqpoint{1.077381in}{2.379516in}}{\pgfqpoint{1.081771in}{2.390115in}}{\pgfqpoint{1.081771in}{2.401165in}}%
\pgfpathcurveto{\pgfqpoint{1.081771in}{2.412215in}}{\pgfqpoint{1.077381in}{2.422814in}}{\pgfqpoint{1.069567in}{2.430628in}}%
\pgfpathcurveto{\pgfqpoint{1.061753in}{2.438442in}}{\pgfqpoint{1.051154in}{2.442832in}}{\pgfqpoint{1.040104in}{2.442832in}}%
\pgfpathcurveto{\pgfqpoint{1.029054in}{2.442832in}}{\pgfqpoint{1.018455in}{2.438442in}}{\pgfqpoint{1.010641in}{2.430628in}}%
\pgfpathcurveto{\pgfqpoint{1.002828in}{2.422814in}}{\pgfqpoint{0.998437in}{2.412215in}}{\pgfqpoint{0.998437in}{2.401165in}}%
\pgfpathcurveto{\pgfqpoint{0.998437in}{2.390115in}}{\pgfqpoint{1.002828in}{2.379516in}}{\pgfqpoint{1.010641in}{2.371703in}}%
\pgfpathcurveto{\pgfqpoint{1.018455in}{2.363889in}}{\pgfqpoint{1.029054in}{2.359499in}}{\pgfqpoint{1.040104in}{2.359499in}}%
\pgfpathlineto{\pgfqpoint{1.040104in}{2.359499in}}%
\pgfpathclose%
\pgfusepath{stroke}%
\end{pgfscope}%
\begin{pgfscope}%
\pgfpathrectangle{\pgfqpoint{0.393053in}{0.375000in}}{\pgfqpoint{6.356833in}{5.175000in}}%
\pgfusepath{clip}%
\pgfsetbuttcap%
\pgfsetroundjoin%
\pgfsetlinewidth{1.003750pt}%
\definecolor{currentstroke}{rgb}{0.827451,0.827451,0.827451}%
\pgfsetstrokecolor{currentstroke}%
\pgfsetdash{}{0pt}%
\pgfpathmoveto{\pgfqpoint{0.848540in}{2.695419in}}%
\pgfpathcurveto{\pgfqpoint{0.859590in}{2.695419in}}{\pgfqpoint{0.870189in}{2.699809in}}{\pgfqpoint{0.878002in}{2.707623in}}%
\pgfpathcurveto{\pgfqpoint{0.885816in}{2.715436in}}{\pgfqpoint{0.890206in}{2.726035in}}{\pgfqpoint{0.890206in}{2.737085in}}%
\pgfpathcurveto{\pgfqpoint{0.890206in}{2.748135in}}{\pgfqpoint{0.885816in}{2.758735in}}{\pgfqpoint{0.878002in}{2.766548in}}%
\pgfpathcurveto{\pgfqpoint{0.870189in}{2.774362in}}{\pgfqpoint{0.859590in}{2.778752in}}{\pgfqpoint{0.848540in}{2.778752in}}%
\pgfpathcurveto{\pgfqpoint{0.837489in}{2.778752in}}{\pgfqpoint{0.826890in}{2.774362in}}{\pgfqpoint{0.819077in}{2.766548in}}%
\pgfpathcurveto{\pgfqpoint{0.811263in}{2.758735in}}{\pgfqpoint{0.806873in}{2.748135in}}{\pgfqpoint{0.806873in}{2.737085in}}%
\pgfpathcurveto{\pgfqpoint{0.806873in}{2.726035in}}{\pgfqpoint{0.811263in}{2.715436in}}{\pgfqpoint{0.819077in}{2.707623in}}%
\pgfpathcurveto{\pgfqpoint{0.826890in}{2.699809in}}{\pgfqpoint{0.837489in}{2.695419in}}{\pgfqpoint{0.848540in}{2.695419in}}%
\pgfpathlineto{\pgfqpoint{0.848540in}{2.695419in}}%
\pgfpathclose%
\pgfusepath{stroke}%
\end{pgfscope}%
\begin{pgfscope}%
\pgfpathrectangle{\pgfqpoint{0.393053in}{0.375000in}}{\pgfqpoint{6.356833in}{5.175000in}}%
\pgfusepath{clip}%
\pgfsetbuttcap%
\pgfsetroundjoin%
\pgfsetlinewidth{1.003750pt}%
\definecolor{currentstroke}{rgb}{0.827451,0.827451,0.827451}%
\pgfsetstrokecolor{currentstroke}%
\pgfsetdash{}{0pt}%
\pgfpathmoveto{\pgfqpoint{2.230001in}{1.209659in}}%
\pgfpathcurveto{\pgfqpoint{2.241051in}{1.209659in}}{\pgfqpoint{2.251650in}{1.214049in}}{\pgfqpoint{2.259463in}{1.221862in}}%
\pgfpathcurveto{\pgfqpoint{2.267277in}{1.229676in}}{\pgfqpoint{2.271667in}{1.240275in}}{\pgfqpoint{2.271667in}{1.251325in}}%
\pgfpathcurveto{\pgfqpoint{2.271667in}{1.262375in}}{\pgfqpoint{2.267277in}{1.272974in}}{\pgfqpoint{2.259463in}{1.280788in}}%
\pgfpathcurveto{\pgfqpoint{2.251650in}{1.288602in}}{\pgfqpoint{2.241051in}{1.292992in}}{\pgfqpoint{2.230001in}{1.292992in}}%
\pgfpathcurveto{\pgfqpoint{2.218950in}{1.292992in}}{\pgfqpoint{2.208351in}{1.288602in}}{\pgfqpoint{2.200538in}{1.280788in}}%
\pgfpathcurveto{\pgfqpoint{2.192724in}{1.272974in}}{\pgfqpoint{2.188334in}{1.262375in}}{\pgfqpoint{2.188334in}{1.251325in}}%
\pgfpathcurveto{\pgfqpoint{2.188334in}{1.240275in}}{\pgfqpoint{2.192724in}{1.229676in}}{\pgfqpoint{2.200538in}{1.221862in}}%
\pgfpathcurveto{\pgfqpoint{2.208351in}{1.214049in}}{\pgfqpoint{2.218950in}{1.209659in}}{\pgfqpoint{2.230001in}{1.209659in}}%
\pgfpathlineto{\pgfqpoint{2.230001in}{1.209659in}}%
\pgfpathclose%
\pgfusepath{stroke}%
\end{pgfscope}%
\begin{pgfscope}%
\pgfpathrectangle{\pgfqpoint{0.393053in}{0.375000in}}{\pgfqpoint{6.356833in}{5.175000in}}%
\pgfusepath{clip}%
\pgfsetbuttcap%
\pgfsetroundjoin%
\pgfsetlinewidth{1.003750pt}%
\definecolor{currentstroke}{rgb}{0.827451,0.827451,0.827451}%
\pgfsetstrokecolor{currentstroke}%
\pgfsetdash{}{0pt}%
\pgfpathmoveto{\pgfqpoint{4.989399in}{0.367641in}}%
\pgfpathcurveto{\pgfqpoint{5.000449in}{0.367641in}}{\pgfqpoint{5.011048in}{0.372031in}}{\pgfqpoint{5.018862in}{0.379845in}}%
\pgfpathcurveto{\pgfqpoint{5.026675in}{0.387659in}}{\pgfqpoint{5.031065in}{0.398258in}}{\pgfqpoint{5.031065in}{0.409308in}}%
\pgfpathcurveto{\pgfqpoint{5.031065in}{0.420358in}}{\pgfqpoint{5.026675in}{0.430957in}}{\pgfqpoint{5.018862in}{0.438771in}}%
\pgfpathcurveto{\pgfqpoint{5.011048in}{0.446584in}}{\pgfqpoint{5.000449in}{0.450974in}}{\pgfqpoint{4.989399in}{0.450974in}}%
\pgfpathcurveto{\pgfqpoint{4.978349in}{0.450974in}}{\pgfqpoint{4.967750in}{0.446584in}}{\pgfqpoint{4.959936in}{0.438771in}}%
\pgfpathcurveto{\pgfqpoint{4.952122in}{0.430957in}}{\pgfqpoint{4.947732in}{0.420358in}}{\pgfqpoint{4.947732in}{0.409308in}}%
\pgfpathcurveto{\pgfqpoint{4.947732in}{0.398258in}}{\pgfqpoint{4.952122in}{0.387659in}}{\pgfqpoint{4.959936in}{0.379845in}}%
\pgfpathcurveto{\pgfqpoint{4.967750in}{0.372031in}}{\pgfqpoint{4.978349in}{0.367641in}}{\pgfqpoint{4.989399in}{0.367641in}}%
\pgfusepath{stroke}%
\end{pgfscope}%
\begin{pgfscope}%
\pgfpathrectangle{\pgfqpoint{0.393053in}{0.375000in}}{\pgfqpoint{6.356833in}{5.175000in}}%
\pgfusepath{clip}%
\pgfsetbuttcap%
\pgfsetroundjoin%
\pgfsetlinewidth{1.003750pt}%
\definecolor{currentstroke}{rgb}{0.827451,0.827451,0.827451}%
\pgfsetstrokecolor{currentstroke}%
\pgfsetdash{}{0pt}%
\pgfpathmoveto{\pgfqpoint{1.978093in}{1.393161in}}%
\pgfpathcurveto{\pgfqpoint{1.989143in}{1.393161in}}{\pgfqpoint{1.999742in}{1.397552in}}{\pgfqpoint{2.007555in}{1.405365in}}%
\pgfpathcurveto{\pgfqpoint{2.015369in}{1.413179in}}{\pgfqpoint{2.019759in}{1.423778in}}{\pgfqpoint{2.019759in}{1.434828in}}%
\pgfpathcurveto{\pgfqpoint{2.019759in}{1.445878in}}{\pgfqpoint{2.015369in}{1.456477in}}{\pgfqpoint{2.007555in}{1.464291in}}%
\pgfpathcurveto{\pgfqpoint{1.999742in}{1.472104in}}{\pgfqpoint{1.989143in}{1.476495in}}{\pgfqpoint{1.978093in}{1.476495in}}%
\pgfpathcurveto{\pgfqpoint{1.967042in}{1.476495in}}{\pgfqpoint{1.956443in}{1.472104in}}{\pgfqpoint{1.948630in}{1.464291in}}%
\pgfpathcurveto{\pgfqpoint{1.940816in}{1.456477in}}{\pgfqpoint{1.936426in}{1.445878in}}{\pgfqpoint{1.936426in}{1.434828in}}%
\pgfpathcurveto{\pgfqpoint{1.936426in}{1.423778in}}{\pgfqpoint{1.940816in}{1.413179in}}{\pgfqpoint{1.948630in}{1.405365in}}%
\pgfpathcurveto{\pgfqpoint{1.956443in}{1.397552in}}{\pgfqpoint{1.967042in}{1.393161in}}{\pgfqpoint{1.978093in}{1.393161in}}%
\pgfpathlineto{\pgfqpoint{1.978093in}{1.393161in}}%
\pgfpathclose%
\pgfusepath{stroke}%
\end{pgfscope}%
\begin{pgfscope}%
\pgfpathrectangle{\pgfqpoint{0.393053in}{0.375000in}}{\pgfqpoint{6.356833in}{5.175000in}}%
\pgfusepath{clip}%
\pgfsetbuttcap%
\pgfsetroundjoin%
\pgfsetlinewidth{1.003750pt}%
\definecolor{currentstroke}{rgb}{0.827451,0.827451,0.827451}%
\pgfsetstrokecolor{currentstroke}%
\pgfsetdash{}{0pt}%
\pgfpathmoveto{\pgfqpoint{1.217500in}{2.119012in}}%
\pgfpathcurveto{\pgfqpoint{1.228550in}{2.119012in}}{\pgfqpoint{1.239149in}{2.123403in}}{\pgfqpoint{1.246963in}{2.131216in}}%
\pgfpathcurveto{\pgfqpoint{1.254776in}{2.139030in}}{\pgfqpoint{1.259167in}{2.149629in}}{\pgfqpoint{1.259167in}{2.160679in}}%
\pgfpathcurveto{\pgfqpoint{1.259167in}{2.171729in}}{\pgfqpoint{1.254776in}{2.182328in}}{\pgfqpoint{1.246963in}{2.190142in}}%
\pgfpathcurveto{\pgfqpoint{1.239149in}{2.197955in}}{\pgfqpoint{1.228550in}{2.202346in}}{\pgfqpoint{1.217500in}{2.202346in}}%
\pgfpathcurveto{\pgfqpoint{1.206450in}{2.202346in}}{\pgfqpoint{1.195851in}{2.197955in}}{\pgfqpoint{1.188037in}{2.190142in}}%
\pgfpathcurveto{\pgfqpoint{1.180224in}{2.182328in}}{\pgfqpoint{1.175833in}{2.171729in}}{\pgfqpoint{1.175833in}{2.160679in}}%
\pgfpathcurveto{\pgfqpoint{1.175833in}{2.149629in}}{\pgfqpoint{1.180224in}{2.139030in}}{\pgfqpoint{1.188037in}{2.131216in}}%
\pgfpathcurveto{\pgfqpoint{1.195851in}{2.123403in}}{\pgfqpoint{1.206450in}{2.119012in}}{\pgfqpoint{1.217500in}{2.119012in}}%
\pgfpathlineto{\pgfqpoint{1.217500in}{2.119012in}}%
\pgfpathclose%
\pgfusepath{stroke}%
\end{pgfscope}%
\begin{pgfscope}%
\pgfpathrectangle{\pgfqpoint{0.393053in}{0.375000in}}{\pgfqpoint{6.356833in}{5.175000in}}%
\pgfusepath{clip}%
\pgfsetbuttcap%
\pgfsetroundjoin%
\pgfsetlinewidth{1.003750pt}%
\definecolor{currentstroke}{rgb}{0.827451,0.827451,0.827451}%
\pgfsetstrokecolor{currentstroke}%
\pgfsetdash{}{0pt}%
\pgfpathmoveto{\pgfqpoint{0.944104in}{2.490312in}}%
\pgfpathcurveto{\pgfqpoint{0.955154in}{2.490312in}}{\pgfqpoint{0.965753in}{2.494702in}}{\pgfqpoint{0.973567in}{2.502516in}}%
\pgfpathcurveto{\pgfqpoint{0.981380in}{2.510329in}}{\pgfqpoint{0.985770in}{2.520928in}}{\pgfqpoint{0.985770in}{2.531978in}}%
\pgfpathcurveto{\pgfqpoint{0.985770in}{2.543029in}}{\pgfqpoint{0.981380in}{2.553628in}}{\pgfqpoint{0.973567in}{2.561441in}}%
\pgfpathcurveto{\pgfqpoint{0.965753in}{2.569255in}}{\pgfqpoint{0.955154in}{2.573645in}}{\pgfqpoint{0.944104in}{2.573645in}}%
\pgfpathcurveto{\pgfqpoint{0.933054in}{2.573645in}}{\pgfqpoint{0.922455in}{2.569255in}}{\pgfqpoint{0.914641in}{2.561441in}}%
\pgfpathcurveto{\pgfqpoint{0.906827in}{2.553628in}}{\pgfqpoint{0.902437in}{2.543029in}}{\pgfqpoint{0.902437in}{2.531978in}}%
\pgfpathcurveto{\pgfqpoint{0.902437in}{2.520928in}}{\pgfqpoint{0.906827in}{2.510329in}}{\pgfqpoint{0.914641in}{2.502516in}}%
\pgfpathcurveto{\pgfqpoint{0.922455in}{2.494702in}}{\pgfqpoint{0.933054in}{2.490312in}}{\pgfqpoint{0.944104in}{2.490312in}}%
\pgfpathlineto{\pgfqpoint{0.944104in}{2.490312in}}%
\pgfpathclose%
\pgfusepath{stroke}%
\end{pgfscope}%
\begin{pgfscope}%
\pgfpathrectangle{\pgfqpoint{0.393053in}{0.375000in}}{\pgfqpoint{6.356833in}{5.175000in}}%
\pgfusepath{clip}%
\pgfsetbuttcap%
\pgfsetroundjoin%
\pgfsetlinewidth{1.003750pt}%
\definecolor{currentstroke}{rgb}{0.827451,0.827451,0.827451}%
\pgfsetstrokecolor{currentstroke}%
\pgfsetdash{}{0pt}%
\pgfpathmoveto{\pgfqpoint{4.122151in}{0.465889in}}%
\pgfpathcurveto{\pgfqpoint{4.133202in}{0.465889in}}{\pgfqpoint{4.143801in}{0.470279in}}{\pgfqpoint{4.151614in}{0.478093in}}%
\pgfpathcurveto{\pgfqpoint{4.159428in}{0.485907in}}{\pgfqpoint{4.163818in}{0.496506in}}{\pgfqpoint{4.163818in}{0.507556in}}%
\pgfpathcurveto{\pgfqpoint{4.163818in}{0.518606in}}{\pgfqpoint{4.159428in}{0.529205in}}{\pgfqpoint{4.151614in}{0.537019in}}%
\pgfpathcurveto{\pgfqpoint{4.143801in}{0.544832in}}{\pgfqpoint{4.133202in}{0.549222in}}{\pgfqpoint{4.122151in}{0.549222in}}%
\pgfpathcurveto{\pgfqpoint{4.111101in}{0.549222in}}{\pgfqpoint{4.100502in}{0.544832in}}{\pgfqpoint{4.092689in}{0.537019in}}%
\pgfpathcurveto{\pgfqpoint{4.084875in}{0.529205in}}{\pgfqpoint{4.080485in}{0.518606in}}{\pgfqpoint{4.080485in}{0.507556in}}%
\pgfpathcurveto{\pgfqpoint{4.080485in}{0.496506in}}{\pgfqpoint{4.084875in}{0.485907in}}{\pgfqpoint{4.092689in}{0.478093in}}%
\pgfpathcurveto{\pgfqpoint{4.100502in}{0.470279in}}{\pgfqpoint{4.111101in}{0.465889in}}{\pgfqpoint{4.122151in}{0.465889in}}%
\pgfpathlineto{\pgfqpoint{4.122151in}{0.465889in}}%
\pgfpathclose%
\pgfusepath{stroke}%
\end{pgfscope}%
\begin{pgfscope}%
\pgfpathrectangle{\pgfqpoint{0.393053in}{0.375000in}}{\pgfqpoint{6.356833in}{5.175000in}}%
\pgfusepath{clip}%
\pgfsetbuttcap%
\pgfsetroundjoin%
\pgfsetlinewidth{1.003750pt}%
\definecolor{currentstroke}{rgb}{0.827451,0.827451,0.827451}%
\pgfsetstrokecolor{currentstroke}%
\pgfsetdash{}{0pt}%
\pgfpathmoveto{\pgfqpoint{1.585081in}{1.702736in}}%
\pgfpathcurveto{\pgfqpoint{1.596132in}{1.702736in}}{\pgfqpoint{1.606731in}{1.707127in}}{\pgfqpoint{1.614544in}{1.714940in}}%
\pgfpathcurveto{\pgfqpoint{1.622358in}{1.722754in}}{\pgfqpoint{1.626748in}{1.733353in}}{\pgfqpoint{1.626748in}{1.744403in}}%
\pgfpathcurveto{\pgfqpoint{1.626748in}{1.755453in}}{\pgfqpoint{1.622358in}{1.766052in}}{\pgfqpoint{1.614544in}{1.773866in}}%
\pgfpathcurveto{\pgfqpoint{1.606731in}{1.781679in}}{\pgfqpoint{1.596132in}{1.786070in}}{\pgfqpoint{1.585081in}{1.786070in}}%
\pgfpathcurveto{\pgfqpoint{1.574031in}{1.786070in}}{\pgfqpoint{1.563432in}{1.781679in}}{\pgfqpoint{1.555619in}{1.773866in}}%
\pgfpathcurveto{\pgfqpoint{1.547805in}{1.766052in}}{\pgfqpoint{1.543415in}{1.755453in}}{\pgfqpoint{1.543415in}{1.744403in}}%
\pgfpathcurveto{\pgfqpoint{1.543415in}{1.733353in}}{\pgfqpoint{1.547805in}{1.722754in}}{\pgfqpoint{1.555619in}{1.714940in}}%
\pgfpathcurveto{\pgfqpoint{1.563432in}{1.707127in}}{\pgfqpoint{1.574031in}{1.702736in}}{\pgfqpoint{1.585081in}{1.702736in}}%
\pgfpathlineto{\pgfqpoint{1.585081in}{1.702736in}}%
\pgfpathclose%
\pgfusepath{stroke}%
\end{pgfscope}%
\begin{pgfscope}%
\pgfpathrectangle{\pgfqpoint{0.393053in}{0.375000in}}{\pgfqpoint{6.356833in}{5.175000in}}%
\pgfusepath{clip}%
\pgfsetbuttcap%
\pgfsetroundjoin%
\pgfsetlinewidth{1.003750pt}%
\definecolor{currentstroke}{rgb}{0.827451,0.827451,0.827451}%
\pgfsetstrokecolor{currentstroke}%
\pgfsetdash{}{0pt}%
\pgfpathmoveto{\pgfqpoint{2.986480in}{0.818757in}}%
\pgfpathcurveto{\pgfqpoint{2.997531in}{0.818757in}}{\pgfqpoint{3.008130in}{0.823147in}}{\pgfqpoint{3.015943in}{0.830961in}}%
\pgfpathcurveto{\pgfqpoint{3.023757in}{0.838775in}}{\pgfqpoint{3.028147in}{0.849374in}}{\pgfqpoint{3.028147in}{0.860424in}}%
\pgfpathcurveto{\pgfqpoint{3.028147in}{0.871474in}}{\pgfqpoint{3.023757in}{0.882073in}}{\pgfqpoint{3.015943in}{0.889887in}}%
\pgfpathcurveto{\pgfqpoint{3.008130in}{0.897700in}}{\pgfqpoint{2.997531in}{0.902090in}}{\pgfqpoint{2.986480in}{0.902090in}}%
\pgfpathcurveto{\pgfqpoint{2.975430in}{0.902090in}}{\pgfqpoint{2.964831in}{0.897700in}}{\pgfqpoint{2.957018in}{0.889887in}}%
\pgfpathcurveto{\pgfqpoint{2.949204in}{0.882073in}}{\pgfqpoint{2.944814in}{0.871474in}}{\pgfqpoint{2.944814in}{0.860424in}}%
\pgfpathcurveto{\pgfqpoint{2.944814in}{0.849374in}}{\pgfqpoint{2.949204in}{0.838775in}}{\pgfqpoint{2.957018in}{0.830961in}}%
\pgfpathcurveto{\pgfqpoint{2.964831in}{0.823147in}}{\pgfqpoint{2.975430in}{0.818757in}}{\pgfqpoint{2.986480in}{0.818757in}}%
\pgfpathlineto{\pgfqpoint{2.986480in}{0.818757in}}%
\pgfpathclose%
\pgfusepath{stroke}%
\end{pgfscope}%
\begin{pgfscope}%
\pgfpathrectangle{\pgfqpoint{0.393053in}{0.375000in}}{\pgfqpoint{6.356833in}{5.175000in}}%
\pgfusepath{clip}%
\pgfsetbuttcap%
\pgfsetroundjoin%
\pgfsetlinewidth{1.003750pt}%
\definecolor{currentstroke}{rgb}{0.827451,0.827451,0.827451}%
\pgfsetstrokecolor{currentstroke}%
\pgfsetdash{}{0pt}%
\pgfpathmoveto{\pgfqpoint{1.068566in}{2.310306in}}%
\pgfpathcurveto{\pgfqpoint{1.079616in}{2.310306in}}{\pgfqpoint{1.090215in}{2.314696in}}{\pgfqpoint{1.098029in}{2.322510in}}%
\pgfpathcurveto{\pgfqpoint{1.105843in}{2.330323in}}{\pgfqpoint{1.110233in}{2.340922in}}{\pgfqpoint{1.110233in}{2.351973in}}%
\pgfpathcurveto{\pgfqpoint{1.110233in}{2.363023in}}{\pgfqpoint{1.105843in}{2.373622in}}{\pgfqpoint{1.098029in}{2.381435in}}%
\pgfpathcurveto{\pgfqpoint{1.090215in}{2.389249in}}{\pgfqpoint{1.079616in}{2.393639in}}{\pgfqpoint{1.068566in}{2.393639in}}%
\pgfpathcurveto{\pgfqpoint{1.057516in}{2.393639in}}{\pgfqpoint{1.046917in}{2.389249in}}{\pgfqpoint{1.039103in}{2.381435in}}%
\pgfpathcurveto{\pgfqpoint{1.031290in}{2.373622in}}{\pgfqpoint{1.026900in}{2.363023in}}{\pgfqpoint{1.026900in}{2.351973in}}%
\pgfpathcurveto{\pgfqpoint{1.026900in}{2.340922in}}{\pgfqpoint{1.031290in}{2.330323in}}{\pgfqpoint{1.039103in}{2.322510in}}%
\pgfpathcurveto{\pgfqpoint{1.046917in}{2.314696in}}{\pgfqpoint{1.057516in}{2.310306in}}{\pgfqpoint{1.068566in}{2.310306in}}%
\pgfpathlineto{\pgfqpoint{1.068566in}{2.310306in}}%
\pgfpathclose%
\pgfusepath{stroke}%
\end{pgfscope}%
\begin{pgfscope}%
\pgfpathrectangle{\pgfqpoint{0.393053in}{0.375000in}}{\pgfqpoint{6.356833in}{5.175000in}}%
\pgfusepath{clip}%
\pgfsetbuttcap%
\pgfsetroundjoin%
\pgfsetlinewidth{1.003750pt}%
\definecolor{currentstroke}{rgb}{0.827451,0.827451,0.827451}%
\pgfsetstrokecolor{currentstroke}%
\pgfsetdash{}{0pt}%
\pgfpathmoveto{\pgfqpoint{5.335363in}{0.346733in}}%
\pgfpathcurveto{\pgfqpoint{5.346413in}{0.346733in}}{\pgfqpoint{5.357012in}{0.351123in}}{\pgfqpoint{5.364826in}{0.358937in}}%
\pgfpathcurveto{\pgfqpoint{5.372639in}{0.366750in}}{\pgfqpoint{5.377030in}{0.377349in}}{\pgfqpoint{5.377030in}{0.388400in}}%
\pgfpathcurveto{\pgfqpoint{5.377030in}{0.399450in}}{\pgfqpoint{5.372639in}{0.410049in}}{\pgfqpoint{5.364826in}{0.417862in}}%
\pgfpathcurveto{\pgfqpoint{5.357012in}{0.425676in}}{\pgfqpoint{5.346413in}{0.430066in}}{\pgfqpoint{5.335363in}{0.430066in}}%
\pgfpathcurveto{\pgfqpoint{5.324313in}{0.430066in}}{\pgfqpoint{5.313714in}{0.425676in}}{\pgfqpoint{5.305900in}{0.417862in}}%
\pgfpathcurveto{\pgfqpoint{5.298086in}{0.410049in}}{\pgfqpoint{5.293696in}{0.399450in}}{\pgfqpoint{5.293696in}{0.388400in}}%
\pgfpathcurveto{\pgfqpoint{5.293696in}{0.377349in}}{\pgfqpoint{5.298086in}{0.366750in}}{\pgfqpoint{5.305900in}{0.358937in}}%
\pgfpathcurveto{\pgfqpoint{5.313714in}{0.351123in}}{\pgfqpoint{5.324313in}{0.346733in}}{\pgfqpoint{5.335363in}{0.346733in}}%
\pgfusepath{stroke}%
\end{pgfscope}%
\begin{pgfscope}%
\pgfpathrectangle{\pgfqpoint{0.393053in}{0.375000in}}{\pgfqpoint{6.356833in}{5.175000in}}%
\pgfusepath{clip}%
\pgfsetbuttcap%
\pgfsetroundjoin%
\pgfsetlinewidth{1.003750pt}%
\definecolor{currentstroke}{rgb}{0.827451,0.827451,0.827451}%
\pgfsetstrokecolor{currentstroke}%
\pgfsetdash{}{0pt}%
\pgfpathmoveto{\pgfqpoint{1.105942in}{2.248010in}}%
\pgfpathcurveto{\pgfqpoint{1.116993in}{2.248010in}}{\pgfqpoint{1.127592in}{2.252400in}}{\pgfqpoint{1.135405in}{2.260214in}}%
\pgfpathcurveto{\pgfqpoint{1.143219in}{2.268028in}}{\pgfqpoint{1.147609in}{2.278627in}}{\pgfqpoint{1.147609in}{2.289677in}}%
\pgfpathcurveto{\pgfqpoint{1.147609in}{2.300727in}}{\pgfqpoint{1.143219in}{2.311326in}}{\pgfqpoint{1.135405in}{2.319140in}}%
\pgfpathcurveto{\pgfqpoint{1.127592in}{2.326953in}}{\pgfqpoint{1.116993in}{2.331343in}}{\pgfqpoint{1.105942in}{2.331343in}}%
\pgfpathcurveto{\pgfqpoint{1.094892in}{2.331343in}}{\pgfqpoint{1.084293in}{2.326953in}}{\pgfqpoint{1.076480in}{2.319140in}}%
\pgfpathcurveto{\pgfqpoint{1.068666in}{2.311326in}}{\pgfqpoint{1.064276in}{2.300727in}}{\pgfqpoint{1.064276in}{2.289677in}}%
\pgfpathcurveto{\pgfqpoint{1.064276in}{2.278627in}}{\pgfqpoint{1.068666in}{2.268028in}}{\pgfqpoint{1.076480in}{2.260214in}}%
\pgfpathcurveto{\pgfqpoint{1.084293in}{2.252400in}}{\pgfqpoint{1.094892in}{2.248010in}}{\pgfqpoint{1.105942in}{2.248010in}}%
\pgfpathlineto{\pgfqpoint{1.105942in}{2.248010in}}%
\pgfpathclose%
\pgfusepath{stroke}%
\end{pgfscope}%
\begin{pgfscope}%
\pgfpathrectangle{\pgfqpoint{0.393053in}{0.375000in}}{\pgfqpoint{6.356833in}{5.175000in}}%
\pgfusepath{clip}%
\pgfsetbuttcap%
\pgfsetroundjoin%
\pgfsetlinewidth{1.003750pt}%
\definecolor{currentstroke}{rgb}{0.827451,0.827451,0.827451}%
\pgfsetstrokecolor{currentstroke}%
\pgfsetdash{}{0pt}%
\pgfpathmoveto{\pgfqpoint{0.525326in}{3.470591in}}%
\pgfpathcurveto{\pgfqpoint{0.536376in}{3.470591in}}{\pgfqpoint{0.546975in}{3.474982in}}{\pgfqpoint{0.554788in}{3.482795in}}%
\pgfpathcurveto{\pgfqpoint{0.562602in}{3.490609in}}{\pgfqpoint{0.566992in}{3.501208in}}{\pgfqpoint{0.566992in}{3.512258in}}%
\pgfpathcurveto{\pgfqpoint{0.566992in}{3.523308in}}{\pgfqpoint{0.562602in}{3.533907in}}{\pgfqpoint{0.554788in}{3.541721in}}%
\pgfpathcurveto{\pgfqpoint{0.546975in}{3.549534in}}{\pgfqpoint{0.536376in}{3.553925in}}{\pgfqpoint{0.525326in}{3.553925in}}%
\pgfpathcurveto{\pgfqpoint{0.514275in}{3.553925in}}{\pgfqpoint{0.503676in}{3.549534in}}{\pgfqpoint{0.495863in}{3.541721in}}%
\pgfpathcurveto{\pgfqpoint{0.488049in}{3.533907in}}{\pgfqpoint{0.483659in}{3.523308in}}{\pgfqpoint{0.483659in}{3.512258in}}%
\pgfpathcurveto{\pgfqpoint{0.483659in}{3.501208in}}{\pgfqpoint{0.488049in}{3.490609in}}{\pgfqpoint{0.495863in}{3.482795in}}%
\pgfpathcurveto{\pgfqpoint{0.503676in}{3.474982in}}{\pgfqpoint{0.514275in}{3.470591in}}{\pgfqpoint{0.525326in}{3.470591in}}%
\pgfpathlineto{\pgfqpoint{0.525326in}{3.470591in}}%
\pgfpathclose%
\pgfusepath{stroke}%
\end{pgfscope}%
\begin{pgfscope}%
\pgfpathrectangle{\pgfqpoint{0.393053in}{0.375000in}}{\pgfqpoint{6.356833in}{5.175000in}}%
\pgfusepath{clip}%
\pgfsetbuttcap%
\pgfsetroundjoin%
\pgfsetlinewidth{1.003750pt}%
\definecolor{currentstroke}{rgb}{0.827451,0.827451,0.827451}%
\pgfsetstrokecolor{currentstroke}%
\pgfsetdash{}{0pt}%
\pgfpathmoveto{\pgfqpoint{2.103640in}{1.312371in}}%
\pgfpathcurveto{\pgfqpoint{2.114690in}{1.312371in}}{\pgfqpoint{2.125289in}{1.316762in}}{\pgfqpoint{2.133103in}{1.324575in}}%
\pgfpathcurveto{\pgfqpoint{2.140916in}{1.332389in}}{\pgfqpoint{2.145307in}{1.342988in}}{\pgfqpoint{2.145307in}{1.354038in}}%
\pgfpathcurveto{\pgfqpoint{2.145307in}{1.365088in}}{\pgfqpoint{2.140916in}{1.375687in}}{\pgfqpoint{2.133103in}{1.383501in}}%
\pgfpathcurveto{\pgfqpoint{2.125289in}{1.391314in}}{\pgfqpoint{2.114690in}{1.395705in}}{\pgfqpoint{2.103640in}{1.395705in}}%
\pgfpathcurveto{\pgfqpoint{2.092590in}{1.395705in}}{\pgfqpoint{2.081991in}{1.391314in}}{\pgfqpoint{2.074177in}{1.383501in}}%
\pgfpathcurveto{\pgfqpoint{2.066364in}{1.375687in}}{\pgfqpoint{2.061973in}{1.365088in}}{\pgfqpoint{2.061973in}{1.354038in}}%
\pgfpathcurveto{\pgfqpoint{2.061973in}{1.342988in}}{\pgfqpoint{2.066364in}{1.332389in}}{\pgfqpoint{2.074177in}{1.324575in}}%
\pgfpathcurveto{\pgfqpoint{2.081991in}{1.316762in}}{\pgfqpoint{2.092590in}{1.312371in}}{\pgfqpoint{2.103640in}{1.312371in}}%
\pgfpathlineto{\pgfqpoint{2.103640in}{1.312371in}}%
\pgfpathclose%
\pgfusepath{stroke}%
\end{pgfscope}%
\begin{pgfscope}%
\pgfpathrectangle{\pgfqpoint{0.393053in}{0.375000in}}{\pgfqpoint{6.356833in}{5.175000in}}%
\pgfusepath{clip}%
\pgfsetbuttcap%
\pgfsetroundjoin%
\pgfsetlinewidth{1.003750pt}%
\definecolor{currentstroke}{rgb}{0.827451,0.827451,0.827451}%
\pgfsetstrokecolor{currentstroke}%
\pgfsetdash{}{0pt}%
\pgfpathmoveto{\pgfqpoint{5.642998in}{0.337452in}}%
\pgfpathcurveto{\pgfqpoint{5.654048in}{0.337452in}}{\pgfqpoint{5.664647in}{0.341843in}}{\pgfqpoint{5.672461in}{0.349656in}}%
\pgfpathcurveto{\pgfqpoint{5.680274in}{0.357470in}}{\pgfqpoint{5.684665in}{0.368069in}}{\pgfqpoint{5.684665in}{0.379119in}}%
\pgfpathcurveto{\pgfqpoint{5.684665in}{0.390169in}}{\pgfqpoint{5.680274in}{0.400768in}}{\pgfqpoint{5.672461in}{0.408582in}}%
\pgfpathcurveto{\pgfqpoint{5.664647in}{0.416395in}}{\pgfqpoint{5.654048in}{0.420786in}}{\pgfqpoint{5.642998in}{0.420786in}}%
\pgfpathcurveto{\pgfqpoint{5.631948in}{0.420786in}}{\pgfqpoint{5.621349in}{0.416395in}}{\pgfqpoint{5.613535in}{0.408582in}}%
\pgfpathcurveto{\pgfqpoint{5.605721in}{0.400768in}}{\pgfqpoint{5.601331in}{0.390169in}}{\pgfqpoint{5.601331in}{0.379119in}}%
\pgfpathcurveto{\pgfqpoint{5.601331in}{0.368069in}}{\pgfqpoint{5.605721in}{0.357470in}}{\pgfqpoint{5.613535in}{0.349656in}}%
\pgfpathcurveto{\pgfqpoint{5.621349in}{0.341843in}}{\pgfqpoint{5.631948in}{0.337452in}}{\pgfqpoint{5.642998in}{0.337452in}}%
\pgfusepath{stroke}%
\end{pgfscope}%
\begin{pgfscope}%
\pgfpathrectangle{\pgfqpoint{0.393053in}{0.375000in}}{\pgfqpoint{6.356833in}{5.175000in}}%
\pgfusepath{clip}%
\pgfsetbuttcap%
\pgfsetroundjoin%
\pgfsetlinewidth{1.003750pt}%
\definecolor{currentstroke}{rgb}{0.827451,0.827451,0.827451}%
\pgfsetstrokecolor{currentstroke}%
\pgfsetdash{}{0pt}%
\pgfpathmoveto{\pgfqpoint{3.762110in}{0.564100in}}%
\pgfpathcurveto{\pgfqpoint{3.773161in}{0.564100in}}{\pgfqpoint{3.783760in}{0.568490in}}{\pgfqpoint{3.791573in}{0.576304in}}%
\pgfpathcurveto{\pgfqpoint{3.799387in}{0.584117in}}{\pgfqpoint{3.803777in}{0.594716in}}{\pgfqpoint{3.803777in}{0.605766in}}%
\pgfpathcurveto{\pgfqpoint{3.803777in}{0.616816in}}{\pgfqpoint{3.799387in}{0.627416in}}{\pgfqpoint{3.791573in}{0.635229in}}%
\pgfpathcurveto{\pgfqpoint{3.783760in}{0.643043in}}{\pgfqpoint{3.773161in}{0.647433in}}{\pgfqpoint{3.762110in}{0.647433in}}%
\pgfpathcurveto{\pgfqpoint{3.751060in}{0.647433in}}{\pgfqpoint{3.740461in}{0.643043in}}{\pgfqpoint{3.732648in}{0.635229in}}%
\pgfpathcurveto{\pgfqpoint{3.724834in}{0.627416in}}{\pgfqpoint{3.720444in}{0.616816in}}{\pgfqpoint{3.720444in}{0.605766in}}%
\pgfpathcurveto{\pgfqpoint{3.720444in}{0.594716in}}{\pgfqpoint{3.724834in}{0.584117in}}{\pgfqpoint{3.732648in}{0.576304in}}%
\pgfpathcurveto{\pgfqpoint{3.740461in}{0.568490in}}{\pgfqpoint{3.751060in}{0.564100in}}{\pgfqpoint{3.762110in}{0.564100in}}%
\pgfpathlineto{\pgfqpoint{3.762110in}{0.564100in}}%
\pgfpathclose%
\pgfusepath{stroke}%
\end{pgfscope}%
\begin{pgfscope}%
\pgfpathrectangle{\pgfqpoint{0.393053in}{0.375000in}}{\pgfqpoint{6.356833in}{5.175000in}}%
\pgfusepath{clip}%
\pgfsetbuttcap%
\pgfsetroundjoin%
\pgfsetlinewidth{1.003750pt}%
\definecolor{currentstroke}{rgb}{0.827451,0.827451,0.827451}%
\pgfsetstrokecolor{currentstroke}%
\pgfsetdash{}{0pt}%
\pgfpathmoveto{\pgfqpoint{5.508231in}{0.339636in}}%
\pgfpathcurveto{\pgfqpoint{5.519281in}{0.339636in}}{\pgfqpoint{5.529880in}{0.344026in}}{\pgfqpoint{5.537694in}{0.351840in}}%
\pgfpathcurveto{\pgfqpoint{5.545508in}{0.359653in}}{\pgfqpoint{5.549898in}{0.370252in}}{\pgfqpoint{5.549898in}{0.381302in}}%
\pgfpathcurveto{\pgfqpoint{5.549898in}{0.392352in}}{\pgfqpoint{5.545508in}{0.402951in}}{\pgfqpoint{5.537694in}{0.410765in}}%
\pgfpathcurveto{\pgfqpoint{5.529880in}{0.418579in}}{\pgfqpoint{5.519281in}{0.422969in}}{\pgfqpoint{5.508231in}{0.422969in}}%
\pgfpathcurveto{\pgfqpoint{5.497181in}{0.422969in}}{\pgfqpoint{5.486582in}{0.418579in}}{\pgfqpoint{5.478769in}{0.410765in}}%
\pgfpathcurveto{\pgfqpoint{5.470955in}{0.402951in}}{\pgfqpoint{5.466565in}{0.392352in}}{\pgfqpoint{5.466565in}{0.381302in}}%
\pgfpathcurveto{\pgfqpoint{5.466565in}{0.370252in}}{\pgfqpoint{5.470955in}{0.359653in}}{\pgfqpoint{5.478769in}{0.351840in}}%
\pgfpathcurveto{\pgfqpoint{5.486582in}{0.344026in}}{\pgfqpoint{5.497181in}{0.339636in}}{\pgfqpoint{5.508231in}{0.339636in}}%
\pgfusepath{stroke}%
\end{pgfscope}%
\begin{pgfscope}%
\pgfpathrectangle{\pgfqpoint{0.393053in}{0.375000in}}{\pgfqpoint{6.356833in}{5.175000in}}%
\pgfusepath{clip}%
\pgfsetbuttcap%
\pgfsetroundjoin%
\pgfsetlinewidth{1.003750pt}%
\definecolor{currentstroke}{rgb}{0.827451,0.827451,0.827451}%
\pgfsetstrokecolor{currentstroke}%
\pgfsetdash{}{0pt}%
\pgfpathmoveto{\pgfqpoint{0.918047in}{2.554880in}}%
\pgfpathcurveto{\pgfqpoint{0.929097in}{2.554880in}}{\pgfqpoint{0.939696in}{2.559271in}}{\pgfqpoint{0.947510in}{2.567084in}}%
\pgfpathcurveto{\pgfqpoint{0.955323in}{2.574898in}}{\pgfqpoint{0.959714in}{2.585497in}}{\pgfqpoint{0.959714in}{2.596547in}}%
\pgfpathcurveto{\pgfqpoint{0.959714in}{2.607597in}}{\pgfqpoint{0.955323in}{2.618196in}}{\pgfqpoint{0.947510in}{2.626010in}}%
\pgfpathcurveto{\pgfqpoint{0.939696in}{2.633823in}}{\pgfqpoint{0.929097in}{2.638214in}}{\pgfqpoint{0.918047in}{2.638214in}}%
\pgfpathcurveto{\pgfqpoint{0.906997in}{2.638214in}}{\pgfqpoint{0.896398in}{2.633823in}}{\pgfqpoint{0.888584in}{2.626010in}}%
\pgfpathcurveto{\pgfqpoint{0.880771in}{2.618196in}}{\pgfqpoint{0.876380in}{2.607597in}}{\pgfqpoint{0.876380in}{2.596547in}}%
\pgfpathcurveto{\pgfqpoint{0.876380in}{2.585497in}}{\pgfqpoint{0.880771in}{2.574898in}}{\pgfqpoint{0.888584in}{2.567084in}}%
\pgfpathcurveto{\pgfqpoint{0.896398in}{2.559271in}}{\pgfqpoint{0.906997in}{2.554880in}}{\pgfqpoint{0.918047in}{2.554880in}}%
\pgfpathlineto{\pgfqpoint{0.918047in}{2.554880in}}%
\pgfpathclose%
\pgfusepath{stroke}%
\end{pgfscope}%
\begin{pgfscope}%
\pgfpathrectangle{\pgfqpoint{0.393053in}{0.375000in}}{\pgfqpoint{6.356833in}{5.175000in}}%
\pgfusepath{clip}%
\pgfsetbuttcap%
\pgfsetroundjoin%
\pgfsetlinewidth{1.003750pt}%
\definecolor{currentstroke}{rgb}{0.827451,0.827451,0.827451}%
\pgfsetstrokecolor{currentstroke}%
\pgfsetdash{}{0pt}%
\pgfpathmoveto{\pgfqpoint{4.445943in}{0.425176in}}%
\pgfpathcurveto{\pgfqpoint{4.456993in}{0.425176in}}{\pgfqpoint{4.467592in}{0.429567in}}{\pgfqpoint{4.475406in}{0.437380in}}%
\pgfpathcurveto{\pgfqpoint{4.483219in}{0.445194in}}{\pgfqpoint{4.487610in}{0.455793in}}{\pgfqpoint{4.487610in}{0.466843in}}%
\pgfpathcurveto{\pgfqpoint{4.487610in}{0.477893in}}{\pgfqpoint{4.483219in}{0.488492in}}{\pgfqpoint{4.475406in}{0.496306in}}%
\pgfpathcurveto{\pgfqpoint{4.467592in}{0.504119in}}{\pgfqpoint{4.456993in}{0.508510in}}{\pgfqpoint{4.445943in}{0.508510in}}%
\pgfpathcurveto{\pgfqpoint{4.434893in}{0.508510in}}{\pgfqpoint{4.424294in}{0.504119in}}{\pgfqpoint{4.416480in}{0.496306in}}%
\pgfpathcurveto{\pgfqpoint{4.408667in}{0.488492in}}{\pgfqpoint{4.404276in}{0.477893in}}{\pgfqpoint{4.404276in}{0.466843in}}%
\pgfpathcurveto{\pgfqpoint{4.404276in}{0.455793in}}{\pgfqpoint{4.408667in}{0.445194in}}{\pgfqpoint{4.416480in}{0.437380in}}%
\pgfpathcurveto{\pgfqpoint{4.424294in}{0.429567in}}{\pgfqpoint{4.434893in}{0.425176in}}{\pgfqpoint{4.445943in}{0.425176in}}%
\pgfpathlineto{\pgfqpoint{4.445943in}{0.425176in}}%
\pgfpathclose%
\pgfusepath{stroke}%
\end{pgfscope}%
\begin{pgfscope}%
\pgfpathrectangle{\pgfqpoint{0.393053in}{0.375000in}}{\pgfqpoint{6.356833in}{5.175000in}}%
\pgfusepath{clip}%
\pgfsetbuttcap%
\pgfsetroundjoin%
\pgfsetlinewidth{1.003750pt}%
\definecolor{currentstroke}{rgb}{0.827451,0.827451,0.827451}%
\pgfsetstrokecolor{currentstroke}%
\pgfsetdash{}{0pt}%
\pgfpathmoveto{\pgfqpoint{1.150964in}{2.182718in}}%
\pgfpathcurveto{\pgfqpoint{1.162014in}{2.182718in}}{\pgfqpoint{1.172614in}{2.187108in}}{\pgfqpoint{1.180427in}{2.194921in}}%
\pgfpathcurveto{\pgfqpoint{1.188241in}{2.202735in}}{\pgfqpoint{1.192631in}{2.213334in}}{\pgfqpoint{1.192631in}{2.224384in}}%
\pgfpathcurveto{\pgfqpoint{1.192631in}{2.235434in}}{\pgfqpoint{1.188241in}{2.246033in}}{\pgfqpoint{1.180427in}{2.253847in}}%
\pgfpathcurveto{\pgfqpoint{1.172614in}{2.261661in}}{\pgfqpoint{1.162014in}{2.266051in}}{\pgfqpoint{1.150964in}{2.266051in}}%
\pgfpathcurveto{\pgfqpoint{1.139914in}{2.266051in}}{\pgfqpoint{1.129315in}{2.261661in}}{\pgfqpoint{1.121502in}{2.253847in}}%
\pgfpathcurveto{\pgfqpoint{1.113688in}{2.246033in}}{\pgfqpoint{1.109298in}{2.235434in}}{\pgfqpoint{1.109298in}{2.224384in}}%
\pgfpathcurveto{\pgfqpoint{1.109298in}{2.213334in}}{\pgfqpoint{1.113688in}{2.202735in}}{\pgfqpoint{1.121502in}{2.194921in}}%
\pgfpathcurveto{\pgfqpoint{1.129315in}{2.187108in}}{\pgfqpoint{1.139914in}{2.182718in}}{\pgfqpoint{1.150964in}{2.182718in}}%
\pgfpathlineto{\pgfqpoint{1.150964in}{2.182718in}}%
\pgfpathclose%
\pgfusepath{stroke}%
\end{pgfscope}%
\begin{pgfscope}%
\pgfpathrectangle{\pgfqpoint{0.393053in}{0.375000in}}{\pgfqpoint{6.356833in}{5.175000in}}%
\pgfusepath{clip}%
\pgfsetbuttcap%
\pgfsetroundjoin%
\pgfsetlinewidth{1.003750pt}%
\definecolor{currentstroke}{rgb}{0.827451,0.827451,0.827451}%
\pgfsetstrokecolor{currentstroke}%
\pgfsetdash{}{0pt}%
\pgfpathmoveto{\pgfqpoint{0.483847in}{3.664437in}}%
\pgfpathcurveto{\pgfqpoint{0.494897in}{3.664437in}}{\pgfqpoint{0.505496in}{3.668827in}}{\pgfqpoint{0.513309in}{3.676640in}}%
\pgfpathcurveto{\pgfqpoint{0.521123in}{3.684454in}}{\pgfqpoint{0.525513in}{3.695053in}}{\pgfqpoint{0.525513in}{3.706103in}}%
\pgfpathcurveto{\pgfqpoint{0.525513in}{3.717153in}}{\pgfqpoint{0.521123in}{3.727752in}}{\pgfqpoint{0.513309in}{3.735566in}}%
\pgfpathcurveto{\pgfqpoint{0.505496in}{3.743380in}}{\pgfqpoint{0.494897in}{3.747770in}}{\pgfqpoint{0.483847in}{3.747770in}}%
\pgfpathcurveto{\pgfqpoint{0.472797in}{3.747770in}}{\pgfqpoint{0.462198in}{3.743380in}}{\pgfqpoint{0.454384in}{3.735566in}}%
\pgfpathcurveto{\pgfqpoint{0.446570in}{3.727752in}}{\pgfqpoint{0.442180in}{3.717153in}}{\pgfqpoint{0.442180in}{3.706103in}}%
\pgfpathcurveto{\pgfqpoint{0.442180in}{3.695053in}}{\pgfqpoint{0.446570in}{3.684454in}}{\pgfqpoint{0.454384in}{3.676640in}}%
\pgfpathcurveto{\pgfqpoint{0.462198in}{3.668827in}}{\pgfqpoint{0.472797in}{3.664437in}}{\pgfqpoint{0.483847in}{3.664437in}}%
\pgfpathlineto{\pgfqpoint{0.483847in}{3.664437in}}%
\pgfpathclose%
\pgfusepath{stroke}%
\end{pgfscope}%
\begin{pgfscope}%
\pgfpathrectangle{\pgfqpoint{0.393053in}{0.375000in}}{\pgfqpoint{6.356833in}{5.175000in}}%
\pgfusepath{clip}%
\pgfsetbuttcap%
\pgfsetroundjoin%
\pgfsetlinewidth{1.003750pt}%
\definecolor{currentstroke}{rgb}{0.827451,0.827451,0.827451}%
\pgfsetstrokecolor{currentstroke}%
\pgfsetdash{}{0pt}%
\pgfpathmoveto{\pgfqpoint{5.340211in}{0.342419in}}%
\pgfpathcurveto{\pgfqpoint{5.351261in}{0.342419in}}{\pgfqpoint{5.361860in}{0.346810in}}{\pgfqpoint{5.369673in}{0.354623in}}%
\pgfpathcurveto{\pgfqpoint{5.377487in}{0.362437in}}{\pgfqpoint{5.381877in}{0.373036in}}{\pgfqpoint{5.381877in}{0.384086in}}%
\pgfpathcurveto{\pgfqpoint{5.381877in}{0.395136in}}{\pgfqpoint{5.377487in}{0.405735in}}{\pgfqpoint{5.369673in}{0.413549in}}%
\pgfpathcurveto{\pgfqpoint{5.361860in}{0.421362in}}{\pgfqpoint{5.351261in}{0.425753in}}{\pgfqpoint{5.340211in}{0.425753in}}%
\pgfpathcurveto{\pgfqpoint{5.329160in}{0.425753in}}{\pgfqpoint{5.318561in}{0.421362in}}{\pgfqpoint{5.310748in}{0.413549in}}%
\pgfpathcurveto{\pgfqpoint{5.302934in}{0.405735in}}{\pgfqpoint{5.298544in}{0.395136in}}{\pgfqpoint{5.298544in}{0.384086in}}%
\pgfpathcurveto{\pgfqpoint{5.298544in}{0.373036in}}{\pgfqpoint{5.302934in}{0.362437in}}{\pgfqpoint{5.310748in}{0.354623in}}%
\pgfpathcurveto{\pgfqpoint{5.318561in}{0.346810in}}{\pgfqpoint{5.329160in}{0.342419in}}{\pgfqpoint{5.340211in}{0.342419in}}%
\pgfusepath{stroke}%
\end{pgfscope}%
\begin{pgfscope}%
\pgfpathrectangle{\pgfqpoint{0.393053in}{0.375000in}}{\pgfqpoint{6.356833in}{5.175000in}}%
\pgfusepath{clip}%
\pgfsetbuttcap%
\pgfsetroundjoin%
\pgfsetlinewidth{1.003750pt}%
\definecolor{currentstroke}{rgb}{0.827451,0.827451,0.827451}%
\pgfsetstrokecolor{currentstroke}%
\pgfsetdash{}{0pt}%
\pgfpathmoveto{\pgfqpoint{2.036490in}{1.340063in}}%
\pgfpathcurveto{\pgfqpoint{2.047540in}{1.340063in}}{\pgfqpoint{2.058139in}{1.344454in}}{\pgfqpoint{2.065953in}{1.352267in}}%
\pgfpathcurveto{\pgfqpoint{2.073767in}{1.360081in}}{\pgfqpoint{2.078157in}{1.370680in}}{\pgfqpoint{2.078157in}{1.381730in}}%
\pgfpathcurveto{\pgfqpoint{2.078157in}{1.392780in}}{\pgfqpoint{2.073767in}{1.403379in}}{\pgfqpoint{2.065953in}{1.411193in}}%
\pgfpathcurveto{\pgfqpoint{2.058139in}{1.419007in}}{\pgfqpoint{2.047540in}{1.423397in}}{\pgfqpoint{2.036490in}{1.423397in}}%
\pgfpathcurveto{\pgfqpoint{2.025440in}{1.423397in}}{\pgfqpoint{2.014841in}{1.419007in}}{\pgfqpoint{2.007027in}{1.411193in}}%
\pgfpathcurveto{\pgfqpoint{1.999214in}{1.403379in}}{\pgfqpoint{1.994824in}{1.392780in}}{\pgfqpoint{1.994824in}{1.381730in}}%
\pgfpathcurveto{\pgfqpoint{1.994824in}{1.370680in}}{\pgfqpoint{1.999214in}{1.360081in}}{\pgfqpoint{2.007027in}{1.352267in}}%
\pgfpathcurveto{\pgfqpoint{2.014841in}{1.344454in}}{\pgfqpoint{2.025440in}{1.340063in}}{\pgfqpoint{2.036490in}{1.340063in}}%
\pgfpathlineto{\pgfqpoint{2.036490in}{1.340063in}}%
\pgfpathclose%
\pgfusepath{stroke}%
\end{pgfscope}%
\begin{pgfscope}%
\pgfpathrectangle{\pgfqpoint{0.393053in}{0.375000in}}{\pgfqpoint{6.356833in}{5.175000in}}%
\pgfusepath{clip}%
\pgfsetbuttcap%
\pgfsetroundjoin%
\pgfsetlinewidth{1.003750pt}%
\definecolor{currentstroke}{rgb}{0.827451,0.827451,0.827451}%
\pgfsetstrokecolor{currentstroke}%
\pgfsetdash{}{0pt}%
\pgfpathmoveto{\pgfqpoint{0.474470in}{3.727925in}}%
\pgfpathcurveto{\pgfqpoint{0.485520in}{3.727925in}}{\pgfqpoint{0.496119in}{3.732315in}}{\pgfqpoint{0.503933in}{3.740129in}}%
\pgfpathcurveto{\pgfqpoint{0.511746in}{3.747943in}}{\pgfqpoint{0.516137in}{3.758542in}}{\pgfqpoint{0.516137in}{3.769592in}}%
\pgfpathcurveto{\pgfqpoint{0.516137in}{3.780642in}}{\pgfqpoint{0.511746in}{3.791241in}}{\pgfqpoint{0.503933in}{3.799055in}}%
\pgfpathcurveto{\pgfqpoint{0.496119in}{3.806868in}}{\pgfqpoint{0.485520in}{3.811259in}}{\pgfqpoint{0.474470in}{3.811259in}}%
\pgfpathcurveto{\pgfqpoint{0.463420in}{3.811259in}}{\pgfqpoint{0.452821in}{3.806868in}}{\pgfqpoint{0.445007in}{3.799055in}}%
\pgfpathcurveto{\pgfqpoint{0.437194in}{3.791241in}}{\pgfqpoint{0.432803in}{3.780642in}}{\pgfqpoint{0.432803in}{3.769592in}}%
\pgfpathcurveto{\pgfqpoint{0.432803in}{3.758542in}}{\pgfqpoint{0.437194in}{3.747943in}}{\pgfqpoint{0.445007in}{3.740129in}}%
\pgfpathcurveto{\pgfqpoint{0.452821in}{3.732315in}}{\pgfqpoint{0.463420in}{3.727925in}}{\pgfqpoint{0.474470in}{3.727925in}}%
\pgfpathlineto{\pgfqpoint{0.474470in}{3.727925in}}%
\pgfpathclose%
\pgfusepath{stroke}%
\end{pgfscope}%
\begin{pgfscope}%
\pgfpathrectangle{\pgfqpoint{0.393053in}{0.375000in}}{\pgfqpoint{6.356833in}{5.175000in}}%
\pgfusepath{clip}%
\pgfsetbuttcap%
\pgfsetroundjoin%
\pgfsetlinewidth{1.003750pt}%
\definecolor{currentstroke}{rgb}{0.827451,0.827451,0.827451}%
\pgfsetstrokecolor{currentstroke}%
\pgfsetdash{}{0pt}%
\pgfpathmoveto{\pgfqpoint{0.511163in}{3.526410in}}%
\pgfpathcurveto{\pgfqpoint{0.522213in}{3.526410in}}{\pgfqpoint{0.532812in}{3.530801in}}{\pgfqpoint{0.540626in}{3.538614in}}%
\pgfpathcurveto{\pgfqpoint{0.548440in}{3.546428in}}{\pgfqpoint{0.552830in}{3.557027in}}{\pgfqpoint{0.552830in}{3.568077in}}%
\pgfpathcurveto{\pgfqpoint{0.552830in}{3.579127in}}{\pgfqpoint{0.548440in}{3.589726in}}{\pgfqpoint{0.540626in}{3.597540in}}%
\pgfpathcurveto{\pgfqpoint{0.532812in}{3.605353in}}{\pgfqpoint{0.522213in}{3.609744in}}{\pgfqpoint{0.511163in}{3.609744in}}%
\pgfpathcurveto{\pgfqpoint{0.500113in}{3.609744in}}{\pgfqpoint{0.489514in}{3.605353in}}{\pgfqpoint{0.481700in}{3.597540in}}%
\pgfpathcurveto{\pgfqpoint{0.473887in}{3.589726in}}{\pgfqpoint{0.469496in}{3.579127in}}{\pgfqpoint{0.469496in}{3.568077in}}%
\pgfpathcurveto{\pgfqpoint{0.469496in}{3.557027in}}{\pgfqpoint{0.473887in}{3.546428in}}{\pgfqpoint{0.481700in}{3.538614in}}%
\pgfpathcurveto{\pgfqpoint{0.489514in}{3.530801in}}{\pgfqpoint{0.500113in}{3.526410in}}{\pgfqpoint{0.511163in}{3.526410in}}%
\pgfpathlineto{\pgfqpoint{0.511163in}{3.526410in}}%
\pgfpathclose%
\pgfusepath{stroke}%
\end{pgfscope}%
\begin{pgfscope}%
\pgfpathrectangle{\pgfqpoint{0.393053in}{0.375000in}}{\pgfqpoint{6.356833in}{5.175000in}}%
\pgfusepath{clip}%
\pgfsetbuttcap%
\pgfsetroundjoin%
\pgfsetlinewidth{1.003750pt}%
\definecolor{currentstroke}{rgb}{0.827451,0.827451,0.827451}%
\pgfsetstrokecolor{currentstroke}%
\pgfsetdash{}{0pt}%
\pgfpathmoveto{\pgfqpoint{3.230235in}{0.725803in}}%
\pgfpathcurveto{\pgfqpoint{3.241285in}{0.725803in}}{\pgfqpoint{3.251884in}{0.730193in}}{\pgfqpoint{3.259697in}{0.738007in}}%
\pgfpathcurveto{\pgfqpoint{3.267511in}{0.745821in}}{\pgfqpoint{3.271901in}{0.756420in}}{\pgfqpoint{3.271901in}{0.767470in}}%
\pgfpathcurveto{\pgfqpoint{3.271901in}{0.778520in}}{\pgfqpoint{3.267511in}{0.789119in}}{\pgfqpoint{3.259697in}{0.796933in}}%
\pgfpathcurveto{\pgfqpoint{3.251884in}{0.804746in}}{\pgfqpoint{3.241285in}{0.809137in}}{\pgfqpoint{3.230235in}{0.809137in}}%
\pgfpathcurveto{\pgfqpoint{3.219184in}{0.809137in}}{\pgfqpoint{3.208585in}{0.804746in}}{\pgfqpoint{3.200772in}{0.796933in}}%
\pgfpathcurveto{\pgfqpoint{3.192958in}{0.789119in}}{\pgfqpoint{3.188568in}{0.778520in}}{\pgfqpoint{3.188568in}{0.767470in}}%
\pgfpathcurveto{\pgfqpoint{3.188568in}{0.756420in}}{\pgfqpoint{3.192958in}{0.745821in}}{\pgfqpoint{3.200772in}{0.738007in}}%
\pgfpathcurveto{\pgfqpoint{3.208585in}{0.730193in}}{\pgfqpoint{3.219184in}{0.725803in}}{\pgfqpoint{3.230235in}{0.725803in}}%
\pgfpathlineto{\pgfqpoint{3.230235in}{0.725803in}}%
\pgfpathclose%
\pgfusepath{stroke}%
\end{pgfscope}%
\begin{pgfscope}%
\pgfpathrectangle{\pgfqpoint{0.393053in}{0.375000in}}{\pgfqpoint{6.356833in}{5.175000in}}%
\pgfusepath{clip}%
\pgfsetbuttcap%
\pgfsetroundjoin%
\pgfsetlinewidth{1.003750pt}%
\definecolor{currentstroke}{rgb}{0.827451,0.827451,0.827451}%
\pgfsetstrokecolor{currentstroke}%
\pgfsetdash{}{0pt}%
\pgfpathmoveto{\pgfqpoint{0.546206in}{3.392492in}}%
\pgfpathcurveto{\pgfqpoint{0.557256in}{3.392492in}}{\pgfqpoint{0.567855in}{3.396882in}}{\pgfqpoint{0.575669in}{3.404696in}}%
\pgfpathcurveto{\pgfqpoint{0.583483in}{3.412509in}}{\pgfqpoint{0.587873in}{3.423108in}}{\pgfqpoint{0.587873in}{3.434159in}}%
\pgfpathcurveto{\pgfqpoint{0.587873in}{3.445209in}}{\pgfqpoint{0.583483in}{3.455808in}}{\pgfqpoint{0.575669in}{3.463621in}}%
\pgfpathcurveto{\pgfqpoint{0.567855in}{3.471435in}}{\pgfqpoint{0.557256in}{3.475825in}}{\pgfqpoint{0.546206in}{3.475825in}}%
\pgfpathcurveto{\pgfqpoint{0.535156in}{3.475825in}}{\pgfqpoint{0.524557in}{3.471435in}}{\pgfqpoint{0.516744in}{3.463621in}}%
\pgfpathcurveto{\pgfqpoint{0.508930in}{3.455808in}}{\pgfqpoint{0.504540in}{3.445209in}}{\pgfqpoint{0.504540in}{3.434159in}}%
\pgfpathcurveto{\pgfqpoint{0.504540in}{3.423108in}}{\pgfqpoint{0.508930in}{3.412509in}}{\pgfqpoint{0.516744in}{3.404696in}}%
\pgfpathcurveto{\pgfqpoint{0.524557in}{3.396882in}}{\pgfqpoint{0.535156in}{3.392492in}}{\pgfqpoint{0.546206in}{3.392492in}}%
\pgfpathlineto{\pgfqpoint{0.546206in}{3.392492in}}%
\pgfpathclose%
\pgfusepath{stroke}%
\end{pgfscope}%
\begin{pgfscope}%
\pgfpathrectangle{\pgfqpoint{0.393053in}{0.375000in}}{\pgfqpoint{6.356833in}{5.175000in}}%
\pgfusepath{clip}%
\pgfsetbuttcap%
\pgfsetroundjoin%
\pgfsetlinewidth{1.003750pt}%
\definecolor{currentstroke}{rgb}{0.827451,0.827451,0.827451}%
\pgfsetstrokecolor{currentstroke}%
\pgfsetdash{}{0pt}%
\pgfpathmoveto{\pgfqpoint{2.376136in}{1.121818in}}%
\pgfpathcurveto{\pgfqpoint{2.387186in}{1.121818in}}{\pgfqpoint{2.397785in}{1.126208in}}{\pgfqpoint{2.405598in}{1.134022in}}%
\pgfpathcurveto{\pgfqpoint{2.413412in}{1.141835in}}{\pgfqpoint{2.417802in}{1.152434in}}{\pgfqpoint{2.417802in}{1.163485in}}%
\pgfpathcurveto{\pgfqpoint{2.417802in}{1.174535in}}{\pgfqpoint{2.413412in}{1.185134in}}{\pgfqpoint{2.405598in}{1.192947in}}%
\pgfpathcurveto{\pgfqpoint{2.397785in}{1.200761in}}{\pgfqpoint{2.387186in}{1.205151in}}{\pgfqpoint{2.376136in}{1.205151in}}%
\pgfpathcurveto{\pgfqpoint{2.365086in}{1.205151in}}{\pgfqpoint{2.354487in}{1.200761in}}{\pgfqpoint{2.346673in}{1.192947in}}%
\pgfpathcurveto{\pgfqpoint{2.338859in}{1.185134in}}{\pgfqpoint{2.334469in}{1.174535in}}{\pgfqpoint{2.334469in}{1.163485in}}%
\pgfpathcurveto{\pgfqpoint{2.334469in}{1.152434in}}{\pgfqpoint{2.338859in}{1.141835in}}{\pgfqpoint{2.346673in}{1.134022in}}%
\pgfpathcurveto{\pgfqpoint{2.354487in}{1.126208in}}{\pgfqpoint{2.365086in}{1.121818in}}{\pgfqpoint{2.376136in}{1.121818in}}%
\pgfpathlineto{\pgfqpoint{2.376136in}{1.121818in}}%
\pgfpathclose%
\pgfusepath{stroke}%
\end{pgfscope}%
\begin{pgfscope}%
\pgfpathrectangle{\pgfqpoint{0.393053in}{0.375000in}}{\pgfqpoint{6.356833in}{5.175000in}}%
\pgfusepath{clip}%
\pgfsetbuttcap%
\pgfsetroundjoin%
\pgfsetlinewidth{1.003750pt}%
\definecolor{currentstroke}{rgb}{0.827451,0.827451,0.827451}%
\pgfsetstrokecolor{currentstroke}%
\pgfsetdash{}{0pt}%
\pgfpathmoveto{\pgfqpoint{1.536415in}{1.768054in}}%
\pgfpathcurveto{\pgfqpoint{1.547466in}{1.768054in}}{\pgfqpoint{1.558065in}{1.772444in}}{\pgfqpoint{1.565878in}{1.780258in}}%
\pgfpathcurveto{\pgfqpoint{1.573692in}{1.788071in}}{\pgfqpoint{1.578082in}{1.798670in}}{\pgfqpoint{1.578082in}{1.809720in}}%
\pgfpathcurveto{\pgfqpoint{1.578082in}{1.820770in}}{\pgfqpoint{1.573692in}{1.831370in}}{\pgfqpoint{1.565878in}{1.839183in}}%
\pgfpathcurveto{\pgfqpoint{1.558065in}{1.846997in}}{\pgfqpoint{1.547466in}{1.851387in}}{\pgfqpoint{1.536415in}{1.851387in}}%
\pgfpathcurveto{\pgfqpoint{1.525365in}{1.851387in}}{\pgfqpoint{1.514766in}{1.846997in}}{\pgfqpoint{1.506953in}{1.839183in}}%
\pgfpathcurveto{\pgfqpoint{1.499139in}{1.831370in}}{\pgfqpoint{1.494749in}{1.820770in}}{\pgfqpoint{1.494749in}{1.809720in}}%
\pgfpathcurveto{\pgfqpoint{1.494749in}{1.798670in}}{\pgfqpoint{1.499139in}{1.788071in}}{\pgfqpoint{1.506953in}{1.780258in}}%
\pgfpathcurveto{\pgfqpoint{1.514766in}{1.772444in}}{\pgfqpoint{1.525365in}{1.768054in}}{\pgfqpoint{1.536415in}{1.768054in}}%
\pgfpathlineto{\pgfqpoint{1.536415in}{1.768054in}}%
\pgfpathclose%
\pgfusepath{stroke}%
\end{pgfscope}%
\begin{pgfscope}%
\pgfpathrectangle{\pgfqpoint{0.393053in}{0.375000in}}{\pgfqpoint{6.356833in}{5.175000in}}%
\pgfusepath{clip}%
\pgfsetbuttcap%
\pgfsetroundjoin%
\pgfsetlinewidth{1.003750pt}%
\definecolor{currentstroke}{rgb}{0.827451,0.827451,0.827451}%
\pgfsetstrokecolor{currentstroke}%
\pgfsetdash{}{0pt}%
\pgfpathmoveto{\pgfqpoint{2.145018in}{1.265722in}}%
\pgfpathcurveto{\pgfqpoint{2.156069in}{1.265722in}}{\pgfqpoint{2.166668in}{1.270112in}}{\pgfqpoint{2.174481in}{1.277926in}}%
\pgfpathcurveto{\pgfqpoint{2.182295in}{1.285739in}}{\pgfqpoint{2.186685in}{1.296338in}}{\pgfqpoint{2.186685in}{1.307389in}}%
\pgfpathcurveto{\pgfqpoint{2.186685in}{1.318439in}}{\pgfqpoint{2.182295in}{1.329038in}}{\pgfqpoint{2.174481in}{1.336851in}}%
\pgfpathcurveto{\pgfqpoint{2.166668in}{1.344665in}}{\pgfqpoint{2.156069in}{1.349055in}}{\pgfqpoint{2.145018in}{1.349055in}}%
\pgfpathcurveto{\pgfqpoint{2.133968in}{1.349055in}}{\pgfqpoint{2.123369in}{1.344665in}}{\pgfqpoint{2.115556in}{1.336851in}}%
\pgfpathcurveto{\pgfqpoint{2.107742in}{1.329038in}}{\pgfqpoint{2.103352in}{1.318439in}}{\pgfqpoint{2.103352in}{1.307389in}}%
\pgfpathcurveto{\pgfqpoint{2.103352in}{1.296338in}}{\pgfqpoint{2.107742in}{1.285739in}}{\pgfqpoint{2.115556in}{1.277926in}}%
\pgfpathcurveto{\pgfqpoint{2.123369in}{1.270112in}}{\pgfqpoint{2.133968in}{1.265722in}}{\pgfqpoint{2.145018in}{1.265722in}}%
\pgfpathlineto{\pgfqpoint{2.145018in}{1.265722in}}%
\pgfpathclose%
\pgfusepath{stroke}%
\end{pgfscope}%
\begin{pgfscope}%
\pgfpathrectangle{\pgfqpoint{0.393053in}{0.375000in}}{\pgfqpoint{6.356833in}{5.175000in}}%
\pgfusepath{clip}%
\pgfsetbuttcap%
\pgfsetroundjoin%
\pgfsetlinewidth{1.003750pt}%
\definecolor{currentstroke}{rgb}{0.827451,0.827451,0.827451}%
\pgfsetstrokecolor{currentstroke}%
\pgfsetdash{}{0pt}%
\pgfpathmoveto{\pgfqpoint{0.452170in}{3.855614in}}%
\pgfpathcurveto{\pgfqpoint{0.463220in}{3.855614in}}{\pgfqpoint{0.473819in}{3.860005in}}{\pgfqpoint{0.481633in}{3.867818in}}%
\pgfpathcurveto{\pgfqpoint{0.489447in}{3.875632in}}{\pgfqpoint{0.493837in}{3.886231in}}{\pgfqpoint{0.493837in}{3.897281in}}%
\pgfpathcurveto{\pgfqpoint{0.493837in}{3.908331in}}{\pgfqpoint{0.489447in}{3.918930in}}{\pgfqpoint{0.481633in}{3.926744in}}%
\pgfpathcurveto{\pgfqpoint{0.473819in}{3.934557in}}{\pgfqpoint{0.463220in}{3.938948in}}{\pgfqpoint{0.452170in}{3.938948in}}%
\pgfpathcurveto{\pgfqpoint{0.441120in}{3.938948in}}{\pgfqpoint{0.430521in}{3.934557in}}{\pgfqpoint{0.422707in}{3.926744in}}%
\pgfpathcurveto{\pgfqpoint{0.414894in}{3.918930in}}{\pgfqpoint{0.410503in}{3.908331in}}{\pgfqpoint{0.410503in}{3.897281in}}%
\pgfpathcurveto{\pgfqpoint{0.410503in}{3.886231in}}{\pgfqpoint{0.414894in}{3.875632in}}{\pgfqpoint{0.422707in}{3.867818in}}%
\pgfpathcurveto{\pgfqpoint{0.430521in}{3.860005in}}{\pgfqpoint{0.441120in}{3.855614in}}{\pgfqpoint{0.452170in}{3.855614in}}%
\pgfpathlineto{\pgfqpoint{0.452170in}{3.855614in}}%
\pgfpathclose%
\pgfusepath{stroke}%
\end{pgfscope}%
\begin{pgfscope}%
\pgfpathrectangle{\pgfqpoint{0.393053in}{0.375000in}}{\pgfqpoint{6.356833in}{5.175000in}}%
\pgfusepath{clip}%
\pgfsetbuttcap%
\pgfsetroundjoin%
\pgfsetlinewidth{1.003750pt}%
\definecolor{currentstroke}{rgb}{0.827451,0.827451,0.827451}%
\pgfsetstrokecolor{currentstroke}%
\pgfsetdash{}{0pt}%
\pgfpathmoveto{\pgfqpoint{3.397541in}{0.662535in}}%
\pgfpathcurveto{\pgfqpoint{3.408591in}{0.662535in}}{\pgfqpoint{3.419190in}{0.666925in}}{\pgfqpoint{3.427004in}{0.674738in}}%
\pgfpathcurveto{\pgfqpoint{3.434817in}{0.682552in}}{\pgfqpoint{3.439208in}{0.693151in}}{\pgfqpoint{3.439208in}{0.704201in}}%
\pgfpathcurveto{\pgfqpoint{3.439208in}{0.715251in}}{\pgfqpoint{3.434817in}{0.725850in}}{\pgfqpoint{3.427004in}{0.733664in}}%
\pgfpathcurveto{\pgfqpoint{3.419190in}{0.741478in}}{\pgfqpoint{3.408591in}{0.745868in}}{\pgfqpoint{3.397541in}{0.745868in}}%
\pgfpathcurveto{\pgfqpoint{3.386491in}{0.745868in}}{\pgfqpoint{3.375892in}{0.741478in}}{\pgfqpoint{3.368078in}{0.733664in}}%
\pgfpathcurveto{\pgfqpoint{3.360265in}{0.725850in}}{\pgfqpoint{3.355874in}{0.715251in}}{\pgfqpoint{3.355874in}{0.704201in}}%
\pgfpathcurveto{\pgfqpoint{3.355874in}{0.693151in}}{\pgfqpoint{3.360265in}{0.682552in}}{\pgfqpoint{3.368078in}{0.674738in}}%
\pgfpathcurveto{\pgfqpoint{3.375892in}{0.666925in}}{\pgfqpoint{3.386491in}{0.662535in}}{\pgfqpoint{3.397541in}{0.662535in}}%
\pgfpathlineto{\pgfqpoint{3.397541in}{0.662535in}}%
\pgfpathclose%
\pgfusepath{stroke}%
\end{pgfscope}%
\begin{pgfscope}%
\pgfpathrectangle{\pgfqpoint{0.393053in}{0.375000in}}{\pgfqpoint{6.356833in}{5.175000in}}%
\pgfusepath{clip}%
\pgfsetbuttcap%
\pgfsetroundjoin%
\pgfsetlinewidth{1.003750pt}%
\definecolor{currentstroke}{rgb}{0.827451,0.827451,0.827451}%
\pgfsetstrokecolor{currentstroke}%
\pgfsetdash{}{0pt}%
\pgfpathmoveto{\pgfqpoint{2.477308in}{1.087549in}}%
\pgfpathcurveto{\pgfqpoint{2.488358in}{1.087549in}}{\pgfqpoint{2.498957in}{1.091940in}}{\pgfqpoint{2.506771in}{1.099753in}}%
\pgfpathcurveto{\pgfqpoint{2.514584in}{1.107567in}}{\pgfqpoint{2.518974in}{1.118166in}}{\pgfqpoint{2.518974in}{1.129216in}}%
\pgfpathcurveto{\pgfqpoint{2.518974in}{1.140266in}}{\pgfqpoint{2.514584in}{1.150865in}}{\pgfqpoint{2.506771in}{1.158679in}}%
\pgfpathcurveto{\pgfqpoint{2.498957in}{1.166492in}}{\pgfqpoint{2.488358in}{1.170883in}}{\pgfqpoint{2.477308in}{1.170883in}}%
\pgfpathcurveto{\pgfqpoint{2.466258in}{1.170883in}}{\pgfqpoint{2.455659in}{1.166492in}}{\pgfqpoint{2.447845in}{1.158679in}}%
\pgfpathcurveto{\pgfqpoint{2.440031in}{1.150865in}}{\pgfqpoint{2.435641in}{1.140266in}}{\pgfqpoint{2.435641in}{1.129216in}}%
\pgfpathcurveto{\pgfqpoint{2.435641in}{1.118166in}}{\pgfqpoint{2.440031in}{1.107567in}}{\pgfqpoint{2.447845in}{1.099753in}}%
\pgfpathcurveto{\pgfqpoint{2.455659in}{1.091940in}}{\pgfqpoint{2.466258in}{1.087549in}}{\pgfqpoint{2.477308in}{1.087549in}}%
\pgfpathlineto{\pgfqpoint{2.477308in}{1.087549in}}%
\pgfpathclose%
\pgfusepath{stroke}%
\end{pgfscope}%
\begin{pgfscope}%
\pgfpathrectangle{\pgfqpoint{0.393053in}{0.375000in}}{\pgfqpoint{6.356833in}{5.175000in}}%
\pgfusepath{clip}%
\pgfsetbuttcap%
\pgfsetroundjoin%
\pgfsetlinewidth{1.003750pt}%
\definecolor{currentstroke}{rgb}{0.827451,0.827451,0.827451}%
\pgfsetstrokecolor{currentstroke}%
\pgfsetdash{}{0pt}%
\pgfpathmoveto{\pgfqpoint{3.831790in}{0.529833in}}%
\pgfpathcurveto{\pgfqpoint{3.842841in}{0.529833in}}{\pgfqpoint{3.853440in}{0.534224in}}{\pgfqpoint{3.861253in}{0.542037in}}%
\pgfpathcurveto{\pgfqpoint{3.869067in}{0.549851in}}{\pgfqpoint{3.873457in}{0.560450in}}{\pgfqpoint{3.873457in}{0.571500in}}%
\pgfpathcurveto{\pgfqpoint{3.873457in}{0.582550in}}{\pgfqpoint{3.869067in}{0.593149in}}{\pgfqpoint{3.861253in}{0.600963in}}%
\pgfpathcurveto{\pgfqpoint{3.853440in}{0.608776in}}{\pgfqpoint{3.842841in}{0.613167in}}{\pgfqpoint{3.831790in}{0.613167in}}%
\pgfpathcurveto{\pgfqpoint{3.820740in}{0.613167in}}{\pgfqpoint{3.810141in}{0.608776in}}{\pgfqpoint{3.802328in}{0.600963in}}%
\pgfpathcurveto{\pgfqpoint{3.794514in}{0.593149in}}{\pgfqpoint{3.790124in}{0.582550in}}{\pgfqpoint{3.790124in}{0.571500in}}%
\pgfpathcurveto{\pgfqpoint{3.790124in}{0.560450in}}{\pgfqpoint{3.794514in}{0.549851in}}{\pgfqpoint{3.802328in}{0.542037in}}%
\pgfpathcurveto{\pgfqpoint{3.810141in}{0.534224in}}{\pgfqpoint{3.820740in}{0.529833in}}{\pgfqpoint{3.831790in}{0.529833in}}%
\pgfpathlineto{\pgfqpoint{3.831790in}{0.529833in}}%
\pgfpathclose%
\pgfusepath{stroke}%
\end{pgfscope}%
\begin{pgfscope}%
\pgfpathrectangle{\pgfqpoint{0.393053in}{0.375000in}}{\pgfqpoint{6.356833in}{5.175000in}}%
\pgfusepath{clip}%
\pgfsetbuttcap%
\pgfsetroundjoin%
\pgfsetlinewidth{1.003750pt}%
\definecolor{currentstroke}{rgb}{0.827451,0.827451,0.827451}%
\pgfsetstrokecolor{currentstroke}%
\pgfsetdash{}{0pt}%
\pgfpathmoveto{\pgfqpoint{0.635911in}{3.119895in}}%
\pgfpathcurveto{\pgfqpoint{0.646961in}{3.119895in}}{\pgfqpoint{0.657560in}{3.124285in}}{\pgfqpoint{0.665374in}{3.132098in}}%
\pgfpathcurveto{\pgfqpoint{0.673187in}{3.139912in}}{\pgfqpoint{0.677577in}{3.150511in}}{\pgfqpoint{0.677577in}{3.161561in}}%
\pgfpathcurveto{\pgfqpoint{0.677577in}{3.172611in}}{\pgfqpoint{0.673187in}{3.183210in}}{\pgfqpoint{0.665374in}{3.191024in}}%
\pgfpathcurveto{\pgfqpoint{0.657560in}{3.198838in}}{\pgfqpoint{0.646961in}{3.203228in}}{\pgfqpoint{0.635911in}{3.203228in}}%
\pgfpathcurveto{\pgfqpoint{0.624861in}{3.203228in}}{\pgfqpoint{0.614262in}{3.198838in}}{\pgfqpoint{0.606448in}{3.191024in}}%
\pgfpathcurveto{\pgfqpoint{0.598634in}{3.183210in}}{\pgfqpoint{0.594244in}{3.172611in}}{\pgfqpoint{0.594244in}{3.161561in}}%
\pgfpathcurveto{\pgfqpoint{0.594244in}{3.150511in}}{\pgfqpoint{0.598634in}{3.139912in}}{\pgfqpoint{0.606448in}{3.132098in}}%
\pgfpathcurveto{\pgfqpoint{0.614262in}{3.124285in}}{\pgfqpoint{0.624861in}{3.119895in}}{\pgfqpoint{0.635911in}{3.119895in}}%
\pgfpathlineto{\pgfqpoint{0.635911in}{3.119895in}}%
\pgfpathclose%
\pgfusepath{stroke}%
\end{pgfscope}%
\begin{pgfscope}%
\pgfpathrectangle{\pgfqpoint{0.393053in}{0.375000in}}{\pgfqpoint{6.356833in}{5.175000in}}%
\pgfusepath{clip}%
\pgfsetbuttcap%
\pgfsetroundjoin%
\pgfsetlinewidth{1.003750pt}%
\definecolor{currentstroke}{rgb}{0.827451,0.827451,0.827451}%
\pgfsetstrokecolor{currentstroke}%
\pgfsetdash{}{0pt}%
\pgfpathmoveto{\pgfqpoint{4.217280in}{0.449238in}}%
\pgfpathcurveto{\pgfqpoint{4.228330in}{0.449238in}}{\pgfqpoint{4.238929in}{0.453628in}}{\pgfqpoint{4.246743in}{0.461441in}}%
\pgfpathcurveto{\pgfqpoint{4.254556in}{0.469255in}}{\pgfqpoint{4.258947in}{0.479854in}}{\pgfqpoint{4.258947in}{0.490904in}}%
\pgfpathcurveto{\pgfqpoint{4.258947in}{0.501954in}}{\pgfqpoint{4.254556in}{0.512553in}}{\pgfqpoint{4.246743in}{0.520367in}}%
\pgfpathcurveto{\pgfqpoint{4.238929in}{0.528181in}}{\pgfqpoint{4.228330in}{0.532571in}}{\pgfqpoint{4.217280in}{0.532571in}}%
\pgfpathcurveto{\pgfqpoint{4.206230in}{0.532571in}}{\pgfqpoint{4.195631in}{0.528181in}}{\pgfqpoint{4.187817in}{0.520367in}}%
\pgfpathcurveto{\pgfqpoint{4.180004in}{0.512553in}}{\pgfqpoint{4.175613in}{0.501954in}}{\pgfqpoint{4.175613in}{0.490904in}}%
\pgfpathcurveto{\pgfqpoint{4.175613in}{0.479854in}}{\pgfqpoint{4.180004in}{0.469255in}}{\pgfqpoint{4.187817in}{0.461441in}}%
\pgfpathcurveto{\pgfqpoint{4.195631in}{0.453628in}}{\pgfqpoint{4.206230in}{0.449238in}}{\pgfqpoint{4.217280in}{0.449238in}}%
\pgfpathlineto{\pgfqpoint{4.217280in}{0.449238in}}%
\pgfpathclose%
\pgfusepath{stroke}%
\end{pgfscope}%
\begin{pgfscope}%
\pgfpathrectangle{\pgfqpoint{0.393053in}{0.375000in}}{\pgfqpoint{6.356833in}{5.175000in}}%
\pgfusepath{clip}%
\pgfsetbuttcap%
\pgfsetroundjoin%
\pgfsetlinewidth{1.003750pt}%
\definecolor{currentstroke}{rgb}{0.827451,0.827451,0.827451}%
\pgfsetstrokecolor{currentstroke}%
\pgfsetdash{}{0pt}%
\pgfpathmoveto{\pgfqpoint{0.711033in}{2.928891in}}%
\pgfpathcurveto{\pgfqpoint{0.722083in}{2.928891in}}{\pgfqpoint{0.732682in}{2.933281in}}{\pgfqpoint{0.740496in}{2.941095in}}%
\pgfpathcurveto{\pgfqpoint{0.748310in}{2.948909in}}{\pgfqpoint{0.752700in}{2.959508in}}{\pgfqpoint{0.752700in}{2.970558in}}%
\pgfpathcurveto{\pgfqpoint{0.752700in}{2.981608in}}{\pgfqpoint{0.748310in}{2.992207in}}{\pgfqpoint{0.740496in}{3.000021in}}%
\pgfpathcurveto{\pgfqpoint{0.732682in}{3.007834in}}{\pgfqpoint{0.722083in}{3.012224in}}{\pgfqpoint{0.711033in}{3.012224in}}%
\pgfpathcurveto{\pgfqpoint{0.699983in}{3.012224in}}{\pgfqpoint{0.689384in}{3.007834in}}{\pgfqpoint{0.681570in}{3.000021in}}%
\pgfpathcurveto{\pgfqpoint{0.673757in}{2.992207in}}{\pgfqpoint{0.669366in}{2.981608in}}{\pgfqpoint{0.669366in}{2.970558in}}%
\pgfpathcurveto{\pgfqpoint{0.669366in}{2.959508in}}{\pgfqpoint{0.673757in}{2.948909in}}{\pgfqpoint{0.681570in}{2.941095in}}%
\pgfpathcurveto{\pgfqpoint{0.689384in}{2.933281in}}{\pgfqpoint{0.699983in}{2.928891in}}{\pgfqpoint{0.711033in}{2.928891in}}%
\pgfpathlineto{\pgfqpoint{0.711033in}{2.928891in}}%
\pgfpathclose%
\pgfusepath{stroke}%
\end{pgfscope}%
\begin{pgfscope}%
\pgfpathrectangle{\pgfqpoint{0.393053in}{0.375000in}}{\pgfqpoint{6.356833in}{5.175000in}}%
\pgfusepath{clip}%
\pgfsetbuttcap%
\pgfsetroundjoin%
\pgfsetlinewidth{1.003750pt}%
\definecolor{currentstroke}{rgb}{0.827451,0.827451,0.827451}%
\pgfsetstrokecolor{currentstroke}%
\pgfsetdash{}{0pt}%
\pgfpathmoveto{\pgfqpoint{4.310929in}{0.438129in}}%
\pgfpathcurveto{\pgfqpoint{4.321979in}{0.438129in}}{\pgfqpoint{4.332578in}{0.442519in}}{\pgfqpoint{4.340392in}{0.450333in}}%
\pgfpathcurveto{\pgfqpoint{4.348205in}{0.458147in}}{\pgfqpoint{4.352595in}{0.468746in}}{\pgfqpoint{4.352595in}{0.479796in}}%
\pgfpathcurveto{\pgfqpoint{4.352595in}{0.490846in}}{\pgfqpoint{4.348205in}{0.501445in}}{\pgfqpoint{4.340392in}{0.509258in}}%
\pgfpathcurveto{\pgfqpoint{4.332578in}{0.517072in}}{\pgfqpoint{4.321979in}{0.521462in}}{\pgfqpoint{4.310929in}{0.521462in}}%
\pgfpathcurveto{\pgfqpoint{4.299879in}{0.521462in}}{\pgfqpoint{4.289280in}{0.517072in}}{\pgfqpoint{4.281466in}{0.509258in}}%
\pgfpathcurveto{\pgfqpoint{4.273652in}{0.501445in}}{\pgfqpoint{4.269262in}{0.490846in}}{\pgfqpoint{4.269262in}{0.479796in}}%
\pgfpathcurveto{\pgfqpoint{4.269262in}{0.468746in}}{\pgfqpoint{4.273652in}{0.458147in}}{\pgfqpoint{4.281466in}{0.450333in}}%
\pgfpathcurveto{\pgfqpoint{4.289280in}{0.442519in}}{\pgfqpoint{4.299879in}{0.438129in}}{\pgfqpoint{4.310929in}{0.438129in}}%
\pgfpathlineto{\pgfqpoint{4.310929in}{0.438129in}}%
\pgfpathclose%
\pgfusepath{stroke}%
\end{pgfscope}%
\begin{pgfscope}%
\pgfpathrectangle{\pgfqpoint{0.393053in}{0.375000in}}{\pgfqpoint{6.356833in}{5.175000in}}%
\pgfusepath{clip}%
\pgfsetbuttcap%
\pgfsetroundjoin%
\pgfsetlinewidth{1.003750pt}%
\definecolor{currentstroke}{rgb}{0.827451,0.827451,0.827451}%
\pgfsetstrokecolor{currentstroke}%
\pgfsetdash{}{0pt}%
\pgfpathmoveto{\pgfqpoint{4.727978in}{0.381934in}}%
\pgfpathcurveto{\pgfqpoint{4.739028in}{0.381934in}}{\pgfqpoint{4.749627in}{0.386324in}}{\pgfqpoint{4.757441in}{0.394137in}}%
\pgfpathcurveto{\pgfqpoint{4.765255in}{0.401951in}}{\pgfqpoint{4.769645in}{0.412550in}}{\pgfqpoint{4.769645in}{0.423600in}}%
\pgfpathcurveto{\pgfqpoint{4.769645in}{0.434650in}}{\pgfqpoint{4.765255in}{0.445249in}}{\pgfqpoint{4.757441in}{0.453063in}}%
\pgfpathcurveto{\pgfqpoint{4.749627in}{0.460877in}}{\pgfqpoint{4.739028in}{0.465267in}}{\pgfqpoint{4.727978in}{0.465267in}}%
\pgfpathcurveto{\pgfqpoint{4.716928in}{0.465267in}}{\pgfqpoint{4.706329in}{0.460877in}}{\pgfqpoint{4.698516in}{0.453063in}}%
\pgfpathcurveto{\pgfqpoint{4.690702in}{0.445249in}}{\pgfqpoint{4.686312in}{0.434650in}}{\pgfqpoint{4.686312in}{0.423600in}}%
\pgfpathcurveto{\pgfqpoint{4.686312in}{0.412550in}}{\pgfqpoint{4.690702in}{0.401951in}}{\pgfqpoint{4.698516in}{0.394137in}}%
\pgfpathcurveto{\pgfqpoint{4.706329in}{0.386324in}}{\pgfqpoint{4.716928in}{0.381934in}}{\pgfqpoint{4.727978in}{0.381934in}}%
\pgfpathlineto{\pgfqpoint{4.727978in}{0.381934in}}%
\pgfpathclose%
\pgfusepath{stroke}%
\end{pgfscope}%
\begin{pgfscope}%
\pgfpathrectangle{\pgfqpoint{0.393053in}{0.375000in}}{\pgfqpoint{6.356833in}{5.175000in}}%
\pgfusepath{clip}%
\pgfsetbuttcap%
\pgfsetroundjoin%
\pgfsetlinewidth{1.003750pt}%
\definecolor{currentstroke}{rgb}{0.827451,0.827451,0.827451}%
\pgfsetstrokecolor{currentstroke}%
\pgfsetdash{}{0pt}%
\pgfpathmoveto{\pgfqpoint{2.179603in}{1.241266in}}%
\pgfpathcurveto{\pgfqpoint{2.190653in}{1.241266in}}{\pgfqpoint{2.201252in}{1.245657in}}{\pgfqpoint{2.209066in}{1.253470in}}%
\pgfpathcurveto{\pgfqpoint{2.216880in}{1.261284in}}{\pgfqpoint{2.221270in}{1.271883in}}{\pgfqpoint{2.221270in}{1.282933in}}%
\pgfpathcurveto{\pgfqpoint{2.221270in}{1.293983in}}{\pgfqpoint{2.216880in}{1.304582in}}{\pgfqpoint{2.209066in}{1.312396in}}%
\pgfpathcurveto{\pgfqpoint{2.201252in}{1.320209in}}{\pgfqpoint{2.190653in}{1.324600in}}{\pgfqpoint{2.179603in}{1.324600in}}%
\pgfpathcurveto{\pgfqpoint{2.168553in}{1.324600in}}{\pgfqpoint{2.157954in}{1.320209in}}{\pgfqpoint{2.150140in}{1.312396in}}%
\pgfpathcurveto{\pgfqpoint{2.142327in}{1.304582in}}{\pgfqpoint{2.137936in}{1.293983in}}{\pgfqpoint{2.137936in}{1.282933in}}%
\pgfpathcurveto{\pgfqpoint{2.137936in}{1.271883in}}{\pgfqpoint{2.142327in}{1.261284in}}{\pgfqpoint{2.150140in}{1.253470in}}%
\pgfpathcurveto{\pgfqpoint{2.157954in}{1.245657in}}{\pgfqpoint{2.168553in}{1.241266in}}{\pgfqpoint{2.179603in}{1.241266in}}%
\pgfpathlineto{\pgfqpoint{2.179603in}{1.241266in}}%
\pgfpathclose%
\pgfusepath{stroke}%
\end{pgfscope}%
\begin{pgfscope}%
\pgfpathrectangle{\pgfqpoint{0.393053in}{0.375000in}}{\pgfqpoint{6.356833in}{5.175000in}}%
\pgfusepath{clip}%
\pgfsetbuttcap%
\pgfsetroundjoin%
\pgfsetlinewidth{1.003750pt}%
\definecolor{currentstroke}{rgb}{0.827451,0.827451,0.827451}%
\pgfsetstrokecolor{currentstroke}%
\pgfsetdash{}{0pt}%
\pgfpathmoveto{\pgfqpoint{0.556022in}{3.362195in}}%
\pgfpathcurveto{\pgfqpoint{0.567072in}{3.362195in}}{\pgfqpoint{0.577671in}{3.366585in}}{\pgfqpoint{0.585485in}{3.374399in}}%
\pgfpathcurveto{\pgfqpoint{0.593298in}{3.382212in}}{\pgfqpoint{0.597688in}{3.392811in}}{\pgfqpoint{0.597688in}{3.403862in}}%
\pgfpathcurveto{\pgfqpoint{0.597688in}{3.414912in}}{\pgfqpoint{0.593298in}{3.425511in}}{\pgfqpoint{0.585485in}{3.433324in}}%
\pgfpathcurveto{\pgfqpoint{0.577671in}{3.441138in}}{\pgfqpoint{0.567072in}{3.445528in}}{\pgfqpoint{0.556022in}{3.445528in}}%
\pgfpathcurveto{\pgfqpoint{0.544972in}{3.445528in}}{\pgfqpoint{0.534373in}{3.441138in}}{\pgfqpoint{0.526559in}{3.433324in}}%
\pgfpathcurveto{\pgfqpoint{0.518745in}{3.425511in}}{\pgfqpoint{0.514355in}{3.414912in}}{\pgfqpoint{0.514355in}{3.403862in}}%
\pgfpathcurveto{\pgfqpoint{0.514355in}{3.392811in}}{\pgfqpoint{0.518745in}{3.382212in}}{\pgfqpoint{0.526559in}{3.374399in}}%
\pgfpathcurveto{\pgfqpoint{0.534373in}{3.366585in}}{\pgfqpoint{0.544972in}{3.362195in}}{\pgfqpoint{0.556022in}{3.362195in}}%
\pgfpathlineto{\pgfqpoint{0.556022in}{3.362195in}}%
\pgfpathclose%
\pgfusepath{stroke}%
\end{pgfscope}%
\begin{pgfscope}%
\pgfpathrectangle{\pgfqpoint{0.393053in}{0.375000in}}{\pgfqpoint{6.356833in}{5.175000in}}%
\pgfusepath{clip}%
\pgfsetbuttcap%
\pgfsetroundjoin%
\pgfsetlinewidth{1.003750pt}%
\definecolor{currentstroke}{rgb}{0.827451,0.827451,0.827451}%
\pgfsetstrokecolor{currentstroke}%
\pgfsetdash{}{0pt}%
\pgfpathmoveto{\pgfqpoint{0.435142in}{3.936498in}}%
\pgfpathcurveto{\pgfqpoint{0.446192in}{3.936498in}}{\pgfqpoint{0.456791in}{3.940888in}}{\pgfqpoint{0.464605in}{3.948702in}}%
\pgfpathcurveto{\pgfqpoint{0.472419in}{3.956516in}}{\pgfqpoint{0.476809in}{3.967115in}}{\pgfqpoint{0.476809in}{3.978165in}}%
\pgfpathcurveto{\pgfqpoint{0.476809in}{3.989215in}}{\pgfqpoint{0.472419in}{3.999814in}}{\pgfqpoint{0.464605in}{4.007627in}}%
\pgfpathcurveto{\pgfqpoint{0.456791in}{4.015441in}}{\pgfqpoint{0.446192in}{4.019831in}}{\pgfqpoint{0.435142in}{4.019831in}}%
\pgfpathcurveto{\pgfqpoint{0.424092in}{4.019831in}}{\pgfqpoint{0.413493in}{4.015441in}}{\pgfqpoint{0.405679in}{4.007627in}}%
\pgfpathcurveto{\pgfqpoint{0.397866in}{3.999814in}}{\pgfqpoint{0.393475in}{3.989215in}}{\pgfqpoint{0.393475in}{3.978165in}}%
\pgfpathcurveto{\pgfqpoint{0.393475in}{3.967115in}}{\pgfqpoint{0.397866in}{3.956516in}}{\pgfqpoint{0.405679in}{3.948702in}}%
\pgfpathcurveto{\pgfqpoint{0.413493in}{3.940888in}}{\pgfqpoint{0.424092in}{3.936498in}}{\pgfqpoint{0.435142in}{3.936498in}}%
\pgfpathlineto{\pgfqpoint{0.435142in}{3.936498in}}%
\pgfpathclose%
\pgfusepath{stroke}%
\end{pgfscope}%
\begin{pgfscope}%
\pgfpathrectangle{\pgfqpoint{0.393053in}{0.375000in}}{\pgfqpoint{6.356833in}{5.175000in}}%
\pgfusepath{clip}%
\pgfsetbuttcap%
\pgfsetroundjoin%
\pgfsetlinewidth{1.003750pt}%
\definecolor{currentstroke}{rgb}{0.827451,0.827451,0.827451}%
\pgfsetstrokecolor{currentstroke}%
\pgfsetdash{}{0pt}%
\pgfpathmoveto{\pgfqpoint{4.348472in}{0.428439in}}%
\pgfpathcurveto{\pgfqpoint{4.359523in}{0.428439in}}{\pgfqpoint{4.370122in}{0.432829in}}{\pgfqpoint{4.377935in}{0.440643in}}%
\pgfpathcurveto{\pgfqpoint{4.385749in}{0.448456in}}{\pgfqpoint{4.390139in}{0.459056in}}{\pgfqpoint{4.390139in}{0.470106in}}%
\pgfpathcurveto{\pgfqpoint{4.390139in}{0.481156in}}{\pgfqpoint{4.385749in}{0.491755in}}{\pgfqpoint{4.377935in}{0.499568in}}%
\pgfpathcurveto{\pgfqpoint{4.370122in}{0.507382in}}{\pgfqpoint{4.359523in}{0.511772in}}{\pgfqpoint{4.348472in}{0.511772in}}%
\pgfpathcurveto{\pgfqpoint{4.337422in}{0.511772in}}{\pgfqpoint{4.326823in}{0.507382in}}{\pgfqpoint{4.319010in}{0.499568in}}%
\pgfpathcurveto{\pgfqpoint{4.311196in}{0.491755in}}{\pgfqpoint{4.306806in}{0.481156in}}{\pgfqpoint{4.306806in}{0.470106in}}%
\pgfpathcurveto{\pgfqpoint{4.306806in}{0.459056in}}{\pgfqpoint{4.311196in}{0.448456in}}{\pgfqpoint{4.319010in}{0.440643in}}%
\pgfpathcurveto{\pgfqpoint{4.326823in}{0.432829in}}{\pgfqpoint{4.337422in}{0.428439in}}{\pgfqpoint{4.348472in}{0.428439in}}%
\pgfpathlineto{\pgfqpoint{4.348472in}{0.428439in}}%
\pgfpathclose%
\pgfusepath{stroke}%
\end{pgfscope}%
\begin{pgfscope}%
\pgfpathrectangle{\pgfqpoint{0.393053in}{0.375000in}}{\pgfqpoint{6.356833in}{5.175000in}}%
\pgfusepath{clip}%
\pgfsetbuttcap%
\pgfsetroundjoin%
\pgfsetlinewidth{1.003750pt}%
\definecolor{currentstroke}{rgb}{0.827451,0.827451,0.827451}%
\pgfsetstrokecolor{currentstroke}%
\pgfsetdash{}{0pt}%
\pgfpathmoveto{\pgfqpoint{3.892407in}{0.513827in}}%
\pgfpathcurveto{\pgfqpoint{3.903457in}{0.513827in}}{\pgfqpoint{3.914056in}{0.518217in}}{\pgfqpoint{3.921869in}{0.526031in}}%
\pgfpathcurveto{\pgfqpoint{3.929683in}{0.533845in}}{\pgfqpoint{3.934073in}{0.544444in}}{\pgfqpoint{3.934073in}{0.555494in}}%
\pgfpathcurveto{\pgfqpoint{3.934073in}{0.566544in}}{\pgfqpoint{3.929683in}{0.577143in}}{\pgfqpoint{3.921869in}{0.584957in}}%
\pgfpathcurveto{\pgfqpoint{3.914056in}{0.592770in}}{\pgfqpoint{3.903457in}{0.597160in}}{\pgfqpoint{3.892407in}{0.597160in}}%
\pgfpathcurveto{\pgfqpoint{3.881356in}{0.597160in}}{\pgfqpoint{3.870757in}{0.592770in}}{\pgfqpoint{3.862944in}{0.584957in}}%
\pgfpathcurveto{\pgfqpoint{3.855130in}{0.577143in}}{\pgfqpoint{3.850740in}{0.566544in}}{\pgfqpoint{3.850740in}{0.555494in}}%
\pgfpathcurveto{\pgfqpoint{3.850740in}{0.544444in}}{\pgfqpoint{3.855130in}{0.533845in}}{\pgfqpoint{3.862944in}{0.526031in}}%
\pgfpathcurveto{\pgfqpoint{3.870757in}{0.518217in}}{\pgfqpoint{3.881356in}{0.513827in}}{\pgfqpoint{3.892407in}{0.513827in}}%
\pgfpathlineto{\pgfqpoint{3.892407in}{0.513827in}}%
\pgfpathclose%
\pgfusepath{stroke}%
\end{pgfscope}%
\begin{pgfscope}%
\pgfpathrectangle{\pgfqpoint{0.393053in}{0.375000in}}{\pgfqpoint{6.356833in}{5.175000in}}%
\pgfusepath{clip}%
\pgfsetbuttcap%
\pgfsetroundjoin%
\pgfsetlinewidth{1.003750pt}%
\definecolor{currentstroke}{rgb}{0.827451,0.827451,0.827451}%
\pgfsetstrokecolor{currentstroke}%
\pgfsetdash{}{0pt}%
\pgfpathmoveto{\pgfqpoint{2.713290in}{0.943005in}}%
\pgfpathcurveto{\pgfqpoint{2.724341in}{0.943005in}}{\pgfqpoint{2.734940in}{0.947396in}}{\pgfqpoint{2.742753in}{0.955209in}}%
\pgfpathcurveto{\pgfqpoint{2.750567in}{0.963023in}}{\pgfqpoint{2.754957in}{0.973622in}}{\pgfqpoint{2.754957in}{0.984672in}}%
\pgfpathcurveto{\pgfqpoint{2.754957in}{0.995722in}}{\pgfqpoint{2.750567in}{1.006321in}}{\pgfqpoint{2.742753in}{1.014135in}}%
\pgfpathcurveto{\pgfqpoint{2.734940in}{1.021949in}}{\pgfqpoint{2.724341in}{1.026339in}}{\pgfqpoint{2.713290in}{1.026339in}}%
\pgfpathcurveto{\pgfqpoint{2.702240in}{1.026339in}}{\pgfqpoint{2.691641in}{1.021949in}}{\pgfqpoint{2.683828in}{1.014135in}}%
\pgfpathcurveto{\pgfqpoint{2.676014in}{1.006321in}}{\pgfqpoint{2.671624in}{0.995722in}}{\pgfqpoint{2.671624in}{0.984672in}}%
\pgfpathcurveto{\pgfqpoint{2.671624in}{0.973622in}}{\pgfqpoint{2.676014in}{0.963023in}}{\pgfqpoint{2.683828in}{0.955209in}}%
\pgfpathcurveto{\pgfqpoint{2.691641in}{0.947396in}}{\pgfqpoint{2.702240in}{0.943005in}}{\pgfqpoint{2.713290in}{0.943005in}}%
\pgfpathlineto{\pgfqpoint{2.713290in}{0.943005in}}%
\pgfpathclose%
\pgfusepath{stroke}%
\end{pgfscope}%
\begin{pgfscope}%
\pgfpathrectangle{\pgfqpoint{0.393053in}{0.375000in}}{\pgfqpoint{6.356833in}{5.175000in}}%
\pgfusepath{clip}%
\pgfsetbuttcap%
\pgfsetroundjoin%
\pgfsetlinewidth{1.003750pt}%
\definecolor{currentstroke}{rgb}{0.827451,0.827451,0.827451}%
\pgfsetstrokecolor{currentstroke}%
\pgfsetdash{}{0pt}%
\pgfpathmoveto{\pgfqpoint{2.504896in}{1.054151in}}%
\pgfpathcurveto{\pgfqpoint{2.515946in}{1.054151in}}{\pgfqpoint{2.526545in}{1.058541in}}{\pgfqpoint{2.534359in}{1.066355in}}%
\pgfpathcurveto{\pgfqpoint{2.542172in}{1.074169in}}{\pgfqpoint{2.546562in}{1.084768in}}{\pgfqpoint{2.546562in}{1.095818in}}%
\pgfpathcurveto{\pgfqpoint{2.546562in}{1.106868in}}{\pgfqpoint{2.542172in}{1.117467in}}{\pgfqpoint{2.534359in}{1.125281in}}%
\pgfpathcurveto{\pgfqpoint{2.526545in}{1.133094in}}{\pgfqpoint{2.515946in}{1.137484in}}{\pgfqpoint{2.504896in}{1.137484in}}%
\pgfpathcurveto{\pgfqpoint{2.493846in}{1.137484in}}{\pgfqpoint{2.483247in}{1.133094in}}{\pgfqpoint{2.475433in}{1.125281in}}%
\pgfpathcurveto{\pgfqpoint{2.467619in}{1.117467in}}{\pgfqpoint{2.463229in}{1.106868in}}{\pgfqpoint{2.463229in}{1.095818in}}%
\pgfpathcurveto{\pgfqpoint{2.463229in}{1.084768in}}{\pgfqpoint{2.467619in}{1.074169in}}{\pgfqpoint{2.475433in}{1.066355in}}%
\pgfpathcurveto{\pgfqpoint{2.483247in}{1.058541in}}{\pgfqpoint{2.493846in}{1.054151in}}{\pgfqpoint{2.504896in}{1.054151in}}%
\pgfpathlineto{\pgfqpoint{2.504896in}{1.054151in}}%
\pgfpathclose%
\pgfusepath{stroke}%
\end{pgfscope}%
\begin{pgfscope}%
\pgfpathrectangle{\pgfqpoint{0.393053in}{0.375000in}}{\pgfqpoint{6.356833in}{5.175000in}}%
\pgfusepath{clip}%
\pgfsetbuttcap%
\pgfsetroundjoin%
\pgfsetlinewidth{1.003750pt}%
\definecolor{currentstroke}{rgb}{0.827451,0.827451,0.827451}%
\pgfsetstrokecolor{currentstroke}%
\pgfsetdash{}{0pt}%
\pgfpathmoveto{\pgfqpoint{5.069843in}{0.359422in}}%
\pgfpathcurveto{\pgfqpoint{5.080893in}{0.359422in}}{\pgfqpoint{5.091492in}{0.363812in}}{\pgfqpoint{5.099306in}{0.371626in}}%
\pgfpathcurveto{\pgfqpoint{5.107119in}{0.379440in}}{\pgfqpoint{5.111510in}{0.390039in}}{\pgfqpoint{5.111510in}{0.401089in}}%
\pgfpathcurveto{\pgfqpoint{5.111510in}{0.412139in}}{\pgfqpoint{5.107119in}{0.422738in}}{\pgfqpoint{5.099306in}{0.430551in}}%
\pgfpathcurveto{\pgfqpoint{5.091492in}{0.438365in}}{\pgfqpoint{5.080893in}{0.442755in}}{\pgfqpoint{5.069843in}{0.442755in}}%
\pgfpathcurveto{\pgfqpoint{5.058793in}{0.442755in}}{\pgfqpoint{5.048194in}{0.438365in}}{\pgfqpoint{5.040380in}{0.430551in}}%
\pgfpathcurveto{\pgfqpoint{5.032567in}{0.422738in}}{\pgfqpoint{5.028176in}{0.412139in}}{\pgfqpoint{5.028176in}{0.401089in}}%
\pgfpathcurveto{\pgfqpoint{5.028176in}{0.390039in}}{\pgfqpoint{5.032567in}{0.379440in}}{\pgfqpoint{5.040380in}{0.371626in}}%
\pgfpathcurveto{\pgfqpoint{5.048194in}{0.363812in}}{\pgfqpoint{5.058793in}{0.359422in}}{\pgfqpoint{5.069843in}{0.359422in}}%
\pgfusepath{stroke}%
\end{pgfscope}%
\begin{pgfscope}%
\pgfpathrectangle{\pgfqpoint{0.393053in}{0.375000in}}{\pgfqpoint{6.356833in}{5.175000in}}%
\pgfusepath{clip}%
\pgfsetbuttcap%
\pgfsetroundjoin%
\pgfsetlinewidth{1.003750pt}%
\definecolor{currentstroke}{rgb}{0.827451,0.827451,0.827451}%
\pgfsetstrokecolor{currentstroke}%
\pgfsetdash{}{0pt}%
\pgfpathmoveto{\pgfqpoint{0.421923in}{4.061978in}}%
\pgfpathcurveto{\pgfqpoint{0.432973in}{4.061978in}}{\pgfqpoint{0.443572in}{4.066368in}}{\pgfqpoint{0.451385in}{4.074182in}}%
\pgfpathcurveto{\pgfqpoint{0.459199in}{4.081995in}}{\pgfqpoint{0.463589in}{4.092594in}}{\pgfqpoint{0.463589in}{4.103644in}}%
\pgfpathcurveto{\pgfqpoint{0.463589in}{4.114694in}}{\pgfqpoint{0.459199in}{4.125293in}}{\pgfqpoint{0.451385in}{4.133107in}}%
\pgfpathcurveto{\pgfqpoint{0.443572in}{4.140921in}}{\pgfqpoint{0.432973in}{4.145311in}}{\pgfqpoint{0.421923in}{4.145311in}}%
\pgfpathcurveto{\pgfqpoint{0.410872in}{4.145311in}}{\pgfqpoint{0.400273in}{4.140921in}}{\pgfqpoint{0.392460in}{4.133107in}}%
\pgfpathcurveto{\pgfqpoint{0.384646in}{4.125293in}}{\pgfqpoint{0.380256in}{4.114694in}}{\pgfqpoint{0.380256in}{4.103644in}}%
\pgfpathcurveto{\pgfqpoint{0.380256in}{4.092594in}}{\pgfqpoint{0.384646in}{4.081995in}}{\pgfqpoint{0.392460in}{4.074182in}}%
\pgfpathcurveto{\pgfqpoint{0.400273in}{4.066368in}}{\pgfqpoint{0.410872in}{4.061978in}}{\pgfqpoint{0.421923in}{4.061978in}}%
\pgfpathlineto{\pgfqpoint{0.421923in}{4.061978in}}%
\pgfpathclose%
\pgfusepath{stroke}%
\end{pgfscope}%
\begin{pgfscope}%
\pgfpathrectangle{\pgfqpoint{0.393053in}{0.375000in}}{\pgfqpoint{6.356833in}{5.175000in}}%
\pgfusepath{clip}%
\pgfsetbuttcap%
\pgfsetroundjoin%
\pgfsetlinewidth{1.003750pt}%
\definecolor{currentstroke}{rgb}{0.827451,0.827451,0.827451}%
\pgfsetstrokecolor{currentstroke}%
\pgfsetdash{}{0pt}%
\pgfpathmoveto{\pgfqpoint{0.393825in}{4.495013in}}%
\pgfpathcurveto{\pgfqpoint{0.404875in}{4.495013in}}{\pgfqpoint{0.415474in}{4.499403in}}{\pgfqpoint{0.423288in}{4.507217in}}%
\pgfpathcurveto{\pgfqpoint{0.431101in}{4.515030in}}{\pgfqpoint{0.435492in}{4.525629in}}{\pgfqpoint{0.435492in}{4.536679in}}%
\pgfpathcurveto{\pgfqpoint{0.435492in}{4.547730in}}{\pgfqpoint{0.431101in}{4.558329in}}{\pgfqpoint{0.423288in}{4.566142in}}%
\pgfpathcurveto{\pgfqpoint{0.415474in}{4.573956in}}{\pgfqpoint{0.404875in}{4.578346in}}{\pgfqpoint{0.393825in}{4.578346in}}%
\pgfpathcurveto{\pgfqpoint{0.382775in}{4.578346in}}{\pgfqpoint{0.372176in}{4.573956in}}{\pgfqpoint{0.364362in}{4.566142in}}%
\pgfpathcurveto{\pgfqpoint{0.356548in}{4.558329in}}{\pgfqpoint{0.352158in}{4.547730in}}{\pgfqpoint{0.352158in}{4.536679in}}%
\pgfpathcurveto{\pgfqpoint{0.352158in}{4.525629in}}{\pgfqpoint{0.356548in}{4.515030in}}{\pgfqpoint{0.364362in}{4.507217in}}%
\pgfpathcurveto{\pgfqpoint{0.372176in}{4.499403in}}{\pgfqpoint{0.382775in}{4.495013in}}{\pgfqpoint{0.393825in}{4.495013in}}%
\pgfpathlineto{\pgfqpoint{0.393825in}{4.495013in}}%
\pgfpathclose%
\pgfusepath{stroke}%
\end{pgfscope}%
\begin{pgfscope}%
\pgfpathrectangle{\pgfqpoint{0.393053in}{0.375000in}}{\pgfqpoint{6.356833in}{5.175000in}}%
\pgfusepath{clip}%
\pgfsetbuttcap%
\pgfsetroundjoin%
\pgfsetlinewidth{1.003750pt}%
\definecolor{currentstroke}{rgb}{0.827451,0.827451,0.827451}%
\pgfsetstrokecolor{currentstroke}%
\pgfsetdash{}{0pt}%
\pgfpathmoveto{\pgfqpoint{3.679888in}{0.565041in}}%
\pgfpathcurveto{\pgfqpoint{3.690938in}{0.565041in}}{\pgfqpoint{3.701537in}{0.569432in}}{\pgfqpoint{3.709351in}{0.577245in}}%
\pgfpathcurveto{\pgfqpoint{3.717164in}{0.585059in}}{\pgfqpoint{3.721555in}{0.595658in}}{\pgfqpoint{3.721555in}{0.606708in}}%
\pgfpathcurveto{\pgfqpoint{3.721555in}{0.617758in}}{\pgfqpoint{3.717164in}{0.628357in}}{\pgfqpoint{3.709351in}{0.636171in}}%
\pgfpathcurveto{\pgfqpoint{3.701537in}{0.643985in}}{\pgfqpoint{3.690938in}{0.648375in}}{\pgfqpoint{3.679888in}{0.648375in}}%
\pgfpathcurveto{\pgfqpoint{3.668838in}{0.648375in}}{\pgfqpoint{3.658239in}{0.643985in}}{\pgfqpoint{3.650425in}{0.636171in}}%
\pgfpathcurveto{\pgfqpoint{3.642612in}{0.628357in}}{\pgfqpoint{3.638221in}{0.617758in}}{\pgfqpoint{3.638221in}{0.606708in}}%
\pgfpathcurveto{\pgfqpoint{3.638221in}{0.595658in}}{\pgfqpoint{3.642612in}{0.585059in}}{\pgfqpoint{3.650425in}{0.577245in}}%
\pgfpathcurveto{\pgfqpoint{3.658239in}{0.569432in}}{\pgfqpoint{3.668838in}{0.565041in}}{\pgfqpoint{3.679888in}{0.565041in}}%
\pgfpathlineto{\pgfqpoint{3.679888in}{0.565041in}}%
\pgfpathclose%
\pgfusepath{stroke}%
\end{pgfscope}%
\begin{pgfscope}%
\pgfpathrectangle{\pgfqpoint{0.393053in}{0.375000in}}{\pgfqpoint{6.356833in}{5.175000in}}%
\pgfusepath{clip}%
\pgfsetbuttcap%
\pgfsetroundjoin%
\pgfsetlinewidth{1.003750pt}%
\definecolor{currentstroke}{rgb}{0.827451,0.827451,0.827451}%
\pgfsetstrokecolor{currentstroke}%
\pgfsetdash{}{0pt}%
\pgfpathmoveto{\pgfqpoint{1.256856in}{2.039646in}}%
\pgfpathcurveto{\pgfqpoint{1.267906in}{2.039646in}}{\pgfqpoint{1.278505in}{2.044036in}}{\pgfqpoint{1.286319in}{2.051849in}}%
\pgfpathcurveto{\pgfqpoint{1.294132in}{2.059663in}}{\pgfqpoint{1.298523in}{2.070262in}}{\pgfqpoint{1.298523in}{2.081312in}}%
\pgfpathcurveto{\pgfqpoint{1.298523in}{2.092362in}}{\pgfqpoint{1.294132in}{2.102961in}}{\pgfqpoint{1.286319in}{2.110775in}}%
\pgfpathcurveto{\pgfqpoint{1.278505in}{2.118589in}}{\pgfqpoint{1.267906in}{2.122979in}}{\pgfqpoint{1.256856in}{2.122979in}}%
\pgfpathcurveto{\pgfqpoint{1.245806in}{2.122979in}}{\pgfqpoint{1.235207in}{2.118589in}}{\pgfqpoint{1.227393in}{2.110775in}}%
\pgfpathcurveto{\pgfqpoint{1.219580in}{2.102961in}}{\pgfqpoint{1.215189in}{2.092362in}}{\pgfqpoint{1.215189in}{2.081312in}}%
\pgfpathcurveto{\pgfqpoint{1.215189in}{2.070262in}}{\pgfqpoint{1.219580in}{2.059663in}}{\pgfqpoint{1.227393in}{2.051849in}}%
\pgfpathcurveto{\pgfqpoint{1.235207in}{2.044036in}}{\pgfqpoint{1.245806in}{2.039646in}}{\pgfqpoint{1.256856in}{2.039646in}}%
\pgfpathlineto{\pgfqpoint{1.256856in}{2.039646in}}%
\pgfpathclose%
\pgfusepath{stroke}%
\end{pgfscope}%
\begin{pgfscope}%
\pgfpathrectangle{\pgfqpoint{0.393053in}{0.375000in}}{\pgfqpoint{6.356833in}{5.175000in}}%
\pgfusepath{clip}%
\pgfsetbuttcap%
\pgfsetroundjoin%
\pgfsetlinewidth{1.003750pt}%
\definecolor{currentstroke}{rgb}{0.827451,0.827451,0.827451}%
\pgfsetstrokecolor{currentstroke}%
\pgfsetdash{}{0pt}%
\pgfpathmoveto{\pgfqpoint{1.019447in}{2.455876in}}%
\pgfpathcurveto{\pgfqpoint{1.030497in}{2.455876in}}{\pgfqpoint{1.041096in}{2.460267in}}{\pgfqpoint{1.048910in}{2.468080in}}%
\pgfpathcurveto{\pgfqpoint{1.056723in}{2.475894in}}{\pgfqpoint{1.061113in}{2.486493in}}{\pgfqpoint{1.061113in}{2.497543in}}%
\pgfpathcurveto{\pgfqpoint{1.061113in}{2.508593in}}{\pgfqpoint{1.056723in}{2.519192in}}{\pgfqpoint{1.048910in}{2.527006in}}%
\pgfpathcurveto{\pgfqpoint{1.041096in}{2.534819in}}{\pgfqpoint{1.030497in}{2.539210in}}{\pgfqpoint{1.019447in}{2.539210in}}%
\pgfpathcurveto{\pgfqpoint{1.008397in}{2.539210in}}{\pgfqpoint{0.997798in}{2.534819in}}{\pgfqpoint{0.989984in}{2.527006in}}%
\pgfpathcurveto{\pgfqpoint{0.982170in}{2.519192in}}{\pgfqpoint{0.977780in}{2.508593in}}{\pgfqpoint{0.977780in}{2.497543in}}%
\pgfpathcurveto{\pgfqpoint{0.977780in}{2.486493in}}{\pgfqpoint{0.982170in}{2.475894in}}{\pgfqpoint{0.989984in}{2.468080in}}%
\pgfpathcurveto{\pgfqpoint{0.997798in}{2.460267in}}{\pgfqpoint{1.008397in}{2.455876in}}{\pgfqpoint{1.019447in}{2.455876in}}%
\pgfpathlineto{\pgfqpoint{1.019447in}{2.455876in}}%
\pgfpathclose%
\pgfusepath{stroke}%
\end{pgfscope}%
\begin{pgfscope}%
\pgfpathrectangle{\pgfqpoint{0.393053in}{0.375000in}}{\pgfqpoint{6.356833in}{5.175000in}}%
\pgfusepath{clip}%
\pgfsetbuttcap%
\pgfsetroundjoin%
\pgfsetlinewidth{1.003750pt}%
\definecolor{currentstroke}{rgb}{0.827451,0.827451,0.827451}%
\pgfsetstrokecolor{currentstroke}%
\pgfsetdash{}{0pt}%
\pgfpathmoveto{\pgfqpoint{2.414235in}{1.111502in}}%
\pgfpathcurveto{\pgfqpoint{2.425285in}{1.111502in}}{\pgfqpoint{2.435884in}{1.115892in}}{\pgfqpoint{2.443698in}{1.123706in}}%
\pgfpathcurveto{\pgfqpoint{2.451511in}{1.131519in}}{\pgfqpoint{2.455902in}{1.142118in}}{\pgfqpoint{2.455902in}{1.153169in}}%
\pgfpathcurveto{\pgfqpoint{2.455902in}{1.164219in}}{\pgfqpoint{2.451511in}{1.174818in}}{\pgfqpoint{2.443698in}{1.182631in}}%
\pgfpathcurveto{\pgfqpoint{2.435884in}{1.190445in}}{\pgfqpoint{2.425285in}{1.194835in}}{\pgfqpoint{2.414235in}{1.194835in}}%
\pgfpathcurveto{\pgfqpoint{2.403185in}{1.194835in}}{\pgfqpoint{2.392586in}{1.190445in}}{\pgfqpoint{2.384772in}{1.182631in}}%
\pgfpathcurveto{\pgfqpoint{2.376959in}{1.174818in}}{\pgfqpoint{2.372568in}{1.164219in}}{\pgfqpoint{2.372568in}{1.153169in}}%
\pgfpathcurveto{\pgfqpoint{2.372568in}{1.142118in}}{\pgfqpoint{2.376959in}{1.131519in}}{\pgfqpoint{2.384772in}{1.123706in}}%
\pgfpathcurveto{\pgfqpoint{2.392586in}{1.115892in}}{\pgfqpoint{2.403185in}{1.111502in}}{\pgfqpoint{2.414235in}{1.111502in}}%
\pgfpathlineto{\pgfqpoint{2.414235in}{1.111502in}}%
\pgfpathclose%
\pgfusepath{stroke}%
\end{pgfscope}%
\begin{pgfscope}%
\pgfpathrectangle{\pgfqpoint{0.393053in}{0.375000in}}{\pgfqpoint{6.356833in}{5.175000in}}%
\pgfusepath{clip}%
\pgfsetbuttcap%
\pgfsetroundjoin%
\pgfsetlinewidth{1.003750pt}%
\definecolor{currentstroke}{rgb}{0.827451,0.827451,0.827451}%
\pgfsetstrokecolor{currentstroke}%
\pgfsetdash{}{0pt}%
\pgfpathmoveto{\pgfqpoint{0.580834in}{3.297265in}}%
\pgfpathcurveto{\pgfqpoint{0.591884in}{3.297265in}}{\pgfqpoint{0.602483in}{3.301655in}}{\pgfqpoint{0.610297in}{3.309469in}}%
\pgfpathcurveto{\pgfqpoint{0.618111in}{3.317282in}}{\pgfqpoint{0.622501in}{3.327881in}}{\pgfqpoint{0.622501in}{3.338932in}}%
\pgfpathcurveto{\pgfqpoint{0.622501in}{3.349982in}}{\pgfqpoint{0.618111in}{3.360581in}}{\pgfqpoint{0.610297in}{3.368394in}}%
\pgfpathcurveto{\pgfqpoint{0.602483in}{3.376208in}}{\pgfqpoint{0.591884in}{3.380598in}}{\pgfqpoint{0.580834in}{3.380598in}}%
\pgfpathcurveto{\pgfqpoint{0.569784in}{3.380598in}}{\pgfqpoint{0.559185in}{3.376208in}}{\pgfqpoint{0.551371in}{3.368394in}}%
\pgfpathcurveto{\pgfqpoint{0.543558in}{3.360581in}}{\pgfqpoint{0.539167in}{3.349982in}}{\pgfqpoint{0.539167in}{3.338932in}}%
\pgfpathcurveto{\pgfqpoint{0.539167in}{3.327881in}}{\pgfqpoint{0.543558in}{3.317282in}}{\pgfqpoint{0.551371in}{3.309469in}}%
\pgfpathcurveto{\pgfqpoint{0.559185in}{3.301655in}}{\pgfqpoint{0.569784in}{3.297265in}}{\pgfqpoint{0.580834in}{3.297265in}}%
\pgfpathlineto{\pgfqpoint{0.580834in}{3.297265in}}%
\pgfpathclose%
\pgfusepath{stroke}%
\end{pgfscope}%
\begin{pgfscope}%
\pgfpathrectangle{\pgfqpoint{0.393053in}{0.375000in}}{\pgfqpoint{6.356833in}{5.175000in}}%
\pgfusepath{clip}%
\pgfsetbuttcap%
\pgfsetroundjoin%
\pgfsetlinewidth{1.003750pt}%
\definecolor{currentstroke}{rgb}{0.827451,0.827451,0.827451}%
\pgfsetstrokecolor{currentstroke}%
\pgfsetdash{}{0pt}%
\pgfpathmoveto{\pgfqpoint{3.464896in}{0.638283in}}%
\pgfpathcurveto{\pgfqpoint{3.475946in}{0.638283in}}{\pgfqpoint{3.486545in}{0.642673in}}{\pgfqpoint{3.494359in}{0.650486in}}%
\pgfpathcurveto{\pgfqpoint{3.502173in}{0.658300in}}{\pgfqpoint{3.506563in}{0.668899in}}{\pgfqpoint{3.506563in}{0.679949in}}%
\pgfpathcurveto{\pgfqpoint{3.506563in}{0.690999in}}{\pgfqpoint{3.502173in}{0.701598in}}{\pgfqpoint{3.494359in}{0.709412in}}%
\pgfpathcurveto{\pgfqpoint{3.486545in}{0.717226in}}{\pgfqpoint{3.475946in}{0.721616in}}{\pgfqpoint{3.464896in}{0.721616in}}%
\pgfpathcurveto{\pgfqpoint{3.453846in}{0.721616in}}{\pgfqpoint{3.443247in}{0.717226in}}{\pgfqpoint{3.435433in}{0.709412in}}%
\pgfpathcurveto{\pgfqpoint{3.427620in}{0.701598in}}{\pgfqpoint{3.423230in}{0.690999in}}{\pgfqpoint{3.423230in}{0.679949in}}%
\pgfpathcurveto{\pgfqpoint{3.423230in}{0.668899in}}{\pgfqpoint{3.427620in}{0.658300in}}{\pgfqpoint{3.435433in}{0.650486in}}%
\pgfpathcurveto{\pgfqpoint{3.443247in}{0.642673in}}{\pgfqpoint{3.453846in}{0.638283in}}{\pgfqpoint{3.464896in}{0.638283in}}%
\pgfpathlineto{\pgfqpoint{3.464896in}{0.638283in}}%
\pgfpathclose%
\pgfusepath{stroke}%
\end{pgfscope}%
\begin{pgfscope}%
\pgfpathrectangle{\pgfqpoint{0.393053in}{0.375000in}}{\pgfqpoint{6.356833in}{5.175000in}}%
\pgfusepath{clip}%
\pgfsetbuttcap%
\pgfsetroundjoin%
\pgfsetlinewidth{1.003750pt}%
\definecolor{currentstroke}{rgb}{0.827451,0.827451,0.827451}%
\pgfsetstrokecolor{currentstroke}%
\pgfsetdash{}{0pt}%
\pgfpathmoveto{\pgfqpoint{1.234935in}{2.069821in}}%
\pgfpathcurveto{\pgfqpoint{1.245986in}{2.069821in}}{\pgfqpoint{1.256585in}{2.074212in}}{\pgfqpoint{1.264398in}{2.082025in}}%
\pgfpathcurveto{\pgfqpoint{1.272212in}{2.089839in}}{\pgfqpoint{1.276602in}{2.100438in}}{\pgfqpoint{1.276602in}{2.111488in}}%
\pgfpathcurveto{\pgfqpoint{1.276602in}{2.122538in}}{\pgfqpoint{1.272212in}{2.133137in}}{\pgfqpoint{1.264398in}{2.140951in}}%
\pgfpathcurveto{\pgfqpoint{1.256585in}{2.148764in}}{\pgfqpoint{1.245986in}{2.153155in}}{\pgfqpoint{1.234935in}{2.153155in}}%
\pgfpathcurveto{\pgfqpoint{1.223885in}{2.153155in}}{\pgfqpoint{1.213286in}{2.148764in}}{\pgfqpoint{1.205473in}{2.140951in}}%
\pgfpathcurveto{\pgfqpoint{1.197659in}{2.133137in}}{\pgfqpoint{1.193269in}{2.122538in}}{\pgfqpoint{1.193269in}{2.111488in}}%
\pgfpathcurveto{\pgfqpoint{1.193269in}{2.100438in}}{\pgfqpoint{1.197659in}{2.089839in}}{\pgfqpoint{1.205473in}{2.082025in}}%
\pgfpathcurveto{\pgfqpoint{1.213286in}{2.074212in}}{\pgfqpoint{1.223885in}{2.069821in}}{\pgfqpoint{1.234935in}{2.069821in}}%
\pgfpathlineto{\pgfqpoint{1.234935in}{2.069821in}}%
\pgfpathclose%
\pgfusepath{stroke}%
\end{pgfscope}%
\begin{pgfscope}%
\pgfpathrectangle{\pgfqpoint{0.393053in}{0.375000in}}{\pgfqpoint{6.356833in}{5.175000in}}%
\pgfusepath{clip}%
\pgfsetbuttcap%
\pgfsetroundjoin%
\pgfsetlinewidth{1.003750pt}%
\definecolor{currentstroke}{rgb}{0.827451,0.827451,0.827451}%
\pgfsetstrokecolor{currentstroke}%
\pgfsetdash{}{0pt}%
\pgfpathmoveto{\pgfqpoint{5.528065in}{0.337504in}}%
\pgfpathcurveto{\pgfqpoint{5.539115in}{0.337504in}}{\pgfqpoint{5.549714in}{0.341895in}}{\pgfqpoint{5.557528in}{0.349708in}}%
\pgfpathcurveto{\pgfqpoint{5.565341in}{0.357522in}}{\pgfqpoint{5.569732in}{0.368121in}}{\pgfqpoint{5.569732in}{0.379171in}}%
\pgfpathcurveto{\pgfqpoint{5.569732in}{0.390221in}}{\pgfqpoint{5.565341in}{0.400820in}}{\pgfqpoint{5.557528in}{0.408634in}}%
\pgfpathcurveto{\pgfqpoint{5.549714in}{0.416447in}}{\pgfqpoint{5.539115in}{0.420838in}}{\pgfqpoint{5.528065in}{0.420838in}}%
\pgfpathcurveto{\pgfqpoint{5.517015in}{0.420838in}}{\pgfqpoint{5.506416in}{0.416447in}}{\pgfqpoint{5.498602in}{0.408634in}}%
\pgfpathcurveto{\pgfqpoint{5.490789in}{0.400820in}}{\pgfqpoint{5.486398in}{0.390221in}}{\pgfqpoint{5.486398in}{0.379171in}}%
\pgfpathcurveto{\pgfqpoint{5.486398in}{0.368121in}}{\pgfqpoint{5.490789in}{0.357522in}}{\pgfqpoint{5.498602in}{0.349708in}}%
\pgfpathcurveto{\pgfqpoint{5.506416in}{0.341895in}}{\pgfqpoint{5.517015in}{0.337504in}}{\pgfqpoint{5.528065in}{0.337504in}}%
\pgfusepath{stroke}%
\end{pgfscope}%
\begin{pgfscope}%
\pgfpathrectangle{\pgfqpoint{0.393053in}{0.375000in}}{\pgfqpoint{6.356833in}{5.175000in}}%
\pgfusepath{clip}%
\pgfsetbuttcap%
\pgfsetroundjoin%
\pgfsetlinewidth{1.003750pt}%
\definecolor{currentstroke}{rgb}{0.827451,0.827451,0.827451}%
\pgfsetstrokecolor{currentstroke}%
\pgfsetdash{}{0pt}%
\pgfpathmoveto{\pgfqpoint{1.291595in}{1.999205in}}%
\pgfpathcurveto{\pgfqpoint{1.302646in}{1.999205in}}{\pgfqpoint{1.313245in}{2.003595in}}{\pgfqpoint{1.321058in}{2.011409in}}%
\pgfpathcurveto{\pgfqpoint{1.328872in}{2.019222in}}{\pgfqpoint{1.333262in}{2.029821in}}{\pgfqpoint{1.333262in}{2.040872in}}%
\pgfpathcurveto{\pgfqpoint{1.333262in}{2.051922in}}{\pgfqpoint{1.328872in}{2.062521in}}{\pgfqpoint{1.321058in}{2.070334in}}%
\pgfpathcurveto{\pgfqpoint{1.313245in}{2.078148in}}{\pgfqpoint{1.302646in}{2.082538in}}{\pgfqpoint{1.291595in}{2.082538in}}%
\pgfpathcurveto{\pgfqpoint{1.280545in}{2.082538in}}{\pgfqpoint{1.269946in}{2.078148in}}{\pgfqpoint{1.262133in}{2.070334in}}%
\pgfpathcurveto{\pgfqpoint{1.254319in}{2.062521in}}{\pgfqpoint{1.249929in}{2.051922in}}{\pgfqpoint{1.249929in}{2.040872in}}%
\pgfpathcurveto{\pgfqpoint{1.249929in}{2.029821in}}{\pgfqpoint{1.254319in}{2.019222in}}{\pgfqpoint{1.262133in}{2.011409in}}%
\pgfpathcurveto{\pgfqpoint{1.269946in}{2.003595in}}{\pgfqpoint{1.280545in}{1.999205in}}{\pgfqpoint{1.291595in}{1.999205in}}%
\pgfpathlineto{\pgfqpoint{1.291595in}{1.999205in}}%
\pgfpathclose%
\pgfusepath{stroke}%
\end{pgfscope}%
\begin{pgfscope}%
\pgfpathrectangle{\pgfqpoint{0.393053in}{0.375000in}}{\pgfqpoint{6.356833in}{5.175000in}}%
\pgfusepath{clip}%
\pgfsetbuttcap%
\pgfsetroundjoin%
\pgfsetlinewidth{1.003750pt}%
\definecolor{currentstroke}{rgb}{0.827451,0.827451,0.827451}%
\pgfsetstrokecolor{currentstroke}%
\pgfsetdash{}{0pt}%
\pgfpathmoveto{\pgfqpoint{0.737172in}{2.862852in}}%
\pgfpathcurveto{\pgfqpoint{0.748222in}{2.862852in}}{\pgfqpoint{0.758821in}{2.867242in}}{\pgfqpoint{0.766635in}{2.875055in}}%
\pgfpathcurveto{\pgfqpoint{0.774449in}{2.882869in}}{\pgfqpoint{0.778839in}{2.893468in}}{\pgfqpoint{0.778839in}{2.904518in}}%
\pgfpathcurveto{\pgfqpoint{0.778839in}{2.915568in}}{\pgfqpoint{0.774449in}{2.926167in}}{\pgfqpoint{0.766635in}{2.933981in}}%
\pgfpathcurveto{\pgfqpoint{0.758821in}{2.941795in}}{\pgfqpoint{0.748222in}{2.946185in}}{\pgfqpoint{0.737172in}{2.946185in}}%
\pgfpathcurveto{\pgfqpoint{0.726122in}{2.946185in}}{\pgfqpoint{0.715523in}{2.941795in}}{\pgfqpoint{0.707710in}{2.933981in}}%
\pgfpathcurveto{\pgfqpoint{0.699896in}{2.926167in}}{\pgfqpoint{0.695506in}{2.915568in}}{\pgfqpoint{0.695506in}{2.904518in}}%
\pgfpathcurveto{\pgfqpoint{0.695506in}{2.893468in}}{\pgfqpoint{0.699896in}{2.882869in}}{\pgfqpoint{0.707710in}{2.875055in}}%
\pgfpathcurveto{\pgfqpoint{0.715523in}{2.867242in}}{\pgfqpoint{0.726122in}{2.862852in}}{\pgfqpoint{0.737172in}{2.862852in}}%
\pgfpathlineto{\pgfqpoint{0.737172in}{2.862852in}}%
\pgfpathclose%
\pgfusepath{stroke}%
\end{pgfscope}%
\begin{pgfscope}%
\pgfpathrectangle{\pgfqpoint{0.393053in}{0.375000in}}{\pgfqpoint{6.356833in}{5.175000in}}%
\pgfusepath{clip}%
\pgfsetbuttcap%
\pgfsetroundjoin%
\pgfsetlinewidth{1.003750pt}%
\definecolor{currentstroke}{rgb}{0.827451,0.827451,0.827451}%
\pgfsetstrokecolor{currentstroke}%
\pgfsetdash{}{0pt}%
\pgfpathmoveto{\pgfqpoint{3.305179in}{0.685348in}}%
\pgfpathcurveto{\pgfqpoint{3.316229in}{0.685348in}}{\pgfqpoint{3.326828in}{0.689738in}}{\pgfqpoint{3.334642in}{0.697552in}}%
\pgfpathcurveto{\pgfqpoint{3.342455in}{0.705365in}}{\pgfqpoint{3.346846in}{0.715964in}}{\pgfqpoint{3.346846in}{0.727015in}}%
\pgfpathcurveto{\pgfqpoint{3.346846in}{0.738065in}}{\pgfqpoint{3.342455in}{0.748664in}}{\pgfqpoint{3.334642in}{0.756477in}}%
\pgfpathcurveto{\pgfqpoint{3.326828in}{0.764291in}}{\pgfqpoint{3.316229in}{0.768681in}}{\pgfqpoint{3.305179in}{0.768681in}}%
\pgfpathcurveto{\pgfqpoint{3.294129in}{0.768681in}}{\pgfqpoint{3.283530in}{0.764291in}}{\pgfqpoint{3.275716in}{0.756477in}}%
\pgfpathcurveto{\pgfqpoint{3.267903in}{0.748664in}}{\pgfqpoint{3.263512in}{0.738065in}}{\pgfqpoint{3.263512in}{0.727015in}}%
\pgfpathcurveto{\pgfqpoint{3.263512in}{0.715964in}}{\pgfqpoint{3.267903in}{0.705365in}}{\pgfqpoint{3.275716in}{0.697552in}}%
\pgfpathcurveto{\pgfqpoint{3.283530in}{0.689738in}}{\pgfqpoint{3.294129in}{0.685348in}}{\pgfqpoint{3.305179in}{0.685348in}}%
\pgfpathlineto{\pgfqpoint{3.305179in}{0.685348in}}%
\pgfpathclose%
\pgfusepath{stroke}%
\end{pgfscope}%
\begin{pgfscope}%
\pgfpathrectangle{\pgfqpoint{0.393053in}{0.375000in}}{\pgfqpoint{6.356833in}{5.175000in}}%
\pgfusepath{clip}%
\pgfsetbuttcap%
\pgfsetroundjoin%
\pgfsetlinewidth{1.003750pt}%
\definecolor{currentstroke}{rgb}{0.827451,0.827451,0.827451}%
\pgfsetstrokecolor{currentstroke}%
\pgfsetdash{}{0pt}%
\pgfpathmoveto{\pgfqpoint{0.416019in}{4.146952in}}%
\pgfpathcurveto{\pgfqpoint{0.427069in}{4.146952in}}{\pgfqpoint{0.437668in}{4.151342in}}{\pgfqpoint{0.445482in}{4.159156in}}%
\pgfpathcurveto{\pgfqpoint{0.453296in}{4.166969in}}{\pgfqpoint{0.457686in}{4.177568in}}{\pgfqpoint{0.457686in}{4.188619in}}%
\pgfpathcurveto{\pgfqpoint{0.457686in}{4.199669in}}{\pgfqpoint{0.453296in}{4.210268in}}{\pgfqpoint{0.445482in}{4.218081in}}%
\pgfpathcurveto{\pgfqpoint{0.437668in}{4.225895in}}{\pgfqpoint{0.427069in}{4.230285in}}{\pgfqpoint{0.416019in}{4.230285in}}%
\pgfpathcurveto{\pgfqpoint{0.404969in}{4.230285in}}{\pgfqpoint{0.394370in}{4.225895in}}{\pgfqpoint{0.386556in}{4.218081in}}%
\pgfpathcurveto{\pgfqpoint{0.378743in}{4.210268in}}{\pgfqpoint{0.374352in}{4.199669in}}{\pgfqpoint{0.374352in}{4.188619in}}%
\pgfpathcurveto{\pgfqpoint{0.374352in}{4.177568in}}{\pgfqpoint{0.378743in}{4.166969in}}{\pgfqpoint{0.386556in}{4.159156in}}%
\pgfpathcurveto{\pgfqpoint{0.394370in}{4.151342in}}{\pgfqpoint{0.404969in}{4.146952in}}{\pgfqpoint{0.416019in}{4.146952in}}%
\pgfpathlineto{\pgfqpoint{0.416019in}{4.146952in}}%
\pgfpathclose%
\pgfusepath{stroke}%
\end{pgfscope}%
\begin{pgfscope}%
\pgfpathrectangle{\pgfqpoint{0.393053in}{0.375000in}}{\pgfqpoint{6.356833in}{5.175000in}}%
\pgfusepath{clip}%
\pgfsetbuttcap%
\pgfsetroundjoin%
\pgfsetlinewidth{1.003750pt}%
\definecolor{currentstroke}{rgb}{0.827451,0.827451,0.827451}%
\pgfsetstrokecolor{currentstroke}%
\pgfsetdash{}{0pt}%
\pgfpathmoveto{\pgfqpoint{3.055426in}{0.786859in}}%
\pgfpathcurveto{\pgfqpoint{3.066476in}{0.786859in}}{\pgfqpoint{3.077075in}{0.791249in}}{\pgfqpoint{3.084888in}{0.799063in}}%
\pgfpathcurveto{\pgfqpoint{3.092702in}{0.806876in}}{\pgfqpoint{3.097092in}{0.817476in}}{\pgfqpoint{3.097092in}{0.828526in}}%
\pgfpathcurveto{\pgfqpoint{3.097092in}{0.839576in}}{\pgfqpoint{3.092702in}{0.850175in}}{\pgfqpoint{3.084888in}{0.857988in}}%
\pgfpathcurveto{\pgfqpoint{3.077075in}{0.865802in}}{\pgfqpoint{3.066476in}{0.870192in}}{\pgfqpoint{3.055426in}{0.870192in}}%
\pgfpathcurveto{\pgfqpoint{3.044376in}{0.870192in}}{\pgfqpoint{3.033777in}{0.865802in}}{\pgfqpoint{3.025963in}{0.857988in}}%
\pgfpathcurveto{\pgfqpoint{3.018149in}{0.850175in}}{\pgfqpoint{3.013759in}{0.839576in}}{\pgfqpoint{3.013759in}{0.828526in}}%
\pgfpathcurveto{\pgfqpoint{3.013759in}{0.817476in}}{\pgfqpoint{3.018149in}{0.806876in}}{\pgfqpoint{3.025963in}{0.799063in}}%
\pgfpathcurveto{\pgfqpoint{3.033777in}{0.791249in}}{\pgfqpoint{3.044376in}{0.786859in}}{\pgfqpoint{3.055426in}{0.786859in}}%
\pgfpathlineto{\pgfqpoint{3.055426in}{0.786859in}}%
\pgfpathclose%
\pgfusepath{stroke}%
\end{pgfscope}%
\begin{pgfscope}%
\pgfpathrectangle{\pgfqpoint{0.393053in}{0.375000in}}{\pgfqpoint{6.356833in}{5.175000in}}%
\pgfusepath{clip}%
\pgfsetbuttcap%
\pgfsetroundjoin%
\pgfsetlinewidth{1.003750pt}%
\definecolor{currentstroke}{rgb}{0.827451,0.827451,0.827451}%
\pgfsetstrokecolor{currentstroke}%
\pgfsetdash{}{0pt}%
\pgfpathmoveto{\pgfqpoint{1.135663in}{2.200937in}}%
\pgfpathcurveto{\pgfqpoint{1.146713in}{2.200937in}}{\pgfqpoint{1.157312in}{2.205328in}}{\pgfqpoint{1.165126in}{2.213141in}}%
\pgfpathcurveto{\pgfqpoint{1.172939in}{2.220955in}}{\pgfqpoint{1.177330in}{2.231554in}}{\pgfqpoint{1.177330in}{2.242604in}}%
\pgfpathcurveto{\pgfqpoint{1.177330in}{2.253654in}}{\pgfqpoint{1.172939in}{2.264253in}}{\pgfqpoint{1.165126in}{2.272067in}}%
\pgfpathcurveto{\pgfqpoint{1.157312in}{2.279880in}}{\pgfqpoint{1.146713in}{2.284271in}}{\pgfqpoint{1.135663in}{2.284271in}}%
\pgfpathcurveto{\pgfqpoint{1.124613in}{2.284271in}}{\pgfqpoint{1.114014in}{2.279880in}}{\pgfqpoint{1.106200in}{2.272067in}}%
\pgfpathcurveto{\pgfqpoint{1.098387in}{2.264253in}}{\pgfqpoint{1.093996in}{2.253654in}}{\pgfqpoint{1.093996in}{2.242604in}}%
\pgfpathcurveto{\pgfqpoint{1.093996in}{2.231554in}}{\pgfqpoint{1.098387in}{2.220955in}}{\pgfqpoint{1.106200in}{2.213141in}}%
\pgfpathcurveto{\pgfqpoint{1.114014in}{2.205328in}}{\pgfqpoint{1.124613in}{2.200937in}}{\pgfqpoint{1.135663in}{2.200937in}}%
\pgfpathlineto{\pgfqpoint{1.135663in}{2.200937in}}%
\pgfpathclose%
\pgfusepath{stroke}%
\end{pgfscope}%
\begin{pgfscope}%
\pgfpathrectangle{\pgfqpoint{0.393053in}{0.375000in}}{\pgfqpoint{6.356833in}{5.175000in}}%
\pgfusepath{clip}%
\pgfsetbuttcap%
\pgfsetroundjoin%
\pgfsetlinewidth{1.003750pt}%
\definecolor{currentstroke}{rgb}{0.827451,0.827451,0.827451}%
\pgfsetstrokecolor{currentstroke}%
\pgfsetdash{}{0pt}%
\pgfpathmoveto{\pgfqpoint{1.509166in}{1.774419in}}%
\pgfpathcurveto{\pgfqpoint{1.520216in}{1.774419in}}{\pgfqpoint{1.530815in}{1.778809in}}{\pgfqpoint{1.538629in}{1.786623in}}%
\pgfpathcurveto{\pgfqpoint{1.546443in}{1.794436in}}{\pgfqpoint{1.550833in}{1.805036in}}{\pgfqpoint{1.550833in}{1.816086in}}%
\pgfpathcurveto{\pgfqpoint{1.550833in}{1.827136in}}{\pgfqpoint{1.546443in}{1.837735in}}{\pgfqpoint{1.538629in}{1.845548in}}%
\pgfpathcurveto{\pgfqpoint{1.530815in}{1.853362in}}{\pgfqpoint{1.520216in}{1.857752in}}{\pgfqpoint{1.509166in}{1.857752in}}%
\pgfpathcurveto{\pgfqpoint{1.498116in}{1.857752in}}{\pgfqpoint{1.487517in}{1.853362in}}{\pgfqpoint{1.479703in}{1.845548in}}%
\pgfpathcurveto{\pgfqpoint{1.471890in}{1.837735in}}{\pgfqpoint{1.467500in}{1.827136in}}{\pgfqpoint{1.467500in}{1.816086in}}%
\pgfpathcurveto{\pgfqpoint{1.467500in}{1.805036in}}{\pgfqpoint{1.471890in}{1.794436in}}{\pgfqpoint{1.479703in}{1.786623in}}%
\pgfpathcurveto{\pgfqpoint{1.487517in}{1.778809in}}{\pgfqpoint{1.498116in}{1.774419in}}{\pgfqpoint{1.509166in}{1.774419in}}%
\pgfpathlineto{\pgfqpoint{1.509166in}{1.774419in}}%
\pgfpathclose%
\pgfusepath{stroke}%
\end{pgfscope}%
\begin{pgfscope}%
\pgfpathrectangle{\pgfqpoint{0.393053in}{0.375000in}}{\pgfqpoint{6.356833in}{5.175000in}}%
\pgfusepath{clip}%
\pgfsetbuttcap%
\pgfsetroundjoin%
\pgfsetlinewidth{1.003750pt}%
\definecolor{currentstroke}{rgb}{0.827451,0.827451,0.827451}%
\pgfsetstrokecolor{currentstroke}%
\pgfsetdash{}{0pt}%
\pgfpathmoveto{\pgfqpoint{0.426530in}{4.012842in}}%
\pgfpathcurveto{\pgfqpoint{0.437580in}{4.012842in}}{\pgfqpoint{0.448179in}{4.017232in}}{\pgfqpoint{0.455992in}{4.025046in}}%
\pgfpathcurveto{\pgfqpoint{0.463806in}{4.032860in}}{\pgfqpoint{0.468196in}{4.043459in}}{\pgfqpoint{0.468196in}{4.054509in}}%
\pgfpathcurveto{\pgfqpoint{0.468196in}{4.065559in}}{\pgfqpoint{0.463806in}{4.076158in}}{\pgfqpoint{0.455992in}{4.083972in}}%
\pgfpathcurveto{\pgfqpoint{0.448179in}{4.091785in}}{\pgfqpoint{0.437580in}{4.096175in}}{\pgfqpoint{0.426530in}{4.096175in}}%
\pgfpathcurveto{\pgfqpoint{0.415479in}{4.096175in}}{\pgfqpoint{0.404880in}{4.091785in}}{\pgfqpoint{0.397067in}{4.083972in}}%
\pgfpathcurveto{\pgfqpoint{0.389253in}{4.076158in}}{\pgfqpoint{0.384863in}{4.065559in}}{\pgfqpoint{0.384863in}{4.054509in}}%
\pgfpathcurveto{\pgfqpoint{0.384863in}{4.043459in}}{\pgfqpoint{0.389253in}{4.032860in}}{\pgfqpoint{0.397067in}{4.025046in}}%
\pgfpathcurveto{\pgfqpoint{0.404880in}{4.017232in}}{\pgfqpoint{0.415479in}{4.012842in}}{\pgfqpoint{0.426530in}{4.012842in}}%
\pgfpathlineto{\pgfqpoint{0.426530in}{4.012842in}}%
\pgfpathclose%
\pgfusepath{stroke}%
\end{pgfscope}%
\begin{pgfscope}%
\pgfpathrectangle{\pgfqpoint{0.393053in}{0.375000in}}{\pgfqpoint{6.356833in}{5.175000in}}%
\pgfusepath{clip}%
\pgfsetbuttcap%
\pgfsetroundjoin%
\pgfsetlinewidth{1.003750pt}%
\definecolor{currentstroke}{rgb}{0.827451,0.827451,0.827451}%
\pgfsetstrokecolor{currentstroke}%
\pgfsetdash{}{0pt}%
\pgfpathmoveto{\pgfqpoint{2.000452in}{1.370955in}}%
\pgfpathcurveto{\pgfqpoint{2.011502in}{1.370955in}}{\pgfqpoint{2.022101in}{1.375345in}}{\pgfqpoint{2.029914in}{1.383159in}}%
\pgfpathcurveto{\pgfqpoint{2.037728in}{1.390972in}}{\pgfqpoint{2.042118in}{1.401571in}}{\pgfqpoint{2.042118in}{1.412622in}}%
\pgfpathcurveto{\pgfqpoint{2.042118in}{1.423672in}}{\pgfqpoint{2.037728in}{1.434271in}}{\pgfqpoint{2.029914in}{1.442084in}}%
\pgfpathcurveto{\pgfqpoint{2.022101in}{1.449898in}}{\pgfqpoint{2.011502in}{1.454288in}}{\pgfqpoint{2.000452in}{1.454288in}}%
\pgfpathcurveto{\pgfqpoint{1.989401in}{1.454288in}}{\pgfqpoint{1.978802in}{1.449898in}}{\pgfqpoint{1.970989in}{1.442084in}}%
\pgfpathcurveto{\pgfqpoint{1.963175in}{1.434271in}}{\pgfqpoint{1.958785in}{1.423672in}}{\pgfqpoint{1.958785in}{1.412622in}}%
\pgfpathcurveto{\pgfqpoint{1.958785in}{1.401571in}}{\pgfqpoint{1.963175in}{1.390972in}}{\pgfqpoint{1.970989in}{1.383159in}}%
\pgfpathcurveto{\pgfqpoint{1.978802in}{1.375345in}}{\pgfqpoint{1.989401in}{1.370955in}}{\pgfqpoint{2.000452in}{1.370955in}}%
\pgfpathlineto{\pgfqpoint{2.000452in}{1.370955in}}%
\pgfpathclose%
\pgfusepath{stroke}%
\end{pgfscope}%
\begin{pgfscope}%
\pgfpathrectangle{\pgfqpoint{0.393053in}{0.375000in}}{\pgfqpoint{6.356833in}{5.175000in}}%
\pgfusepath{clip}%
\pgfsetbuttcap%
\pgfsetroundjoin%
\pgfsetlinewidth{1.003750pt}%
\definecolor{currentstroke}{rgb}{0.827451,0.827451,0.827451}%
\pgfsetstrokecolor{currentstroke}%
\pgfsetdash{}{0pt}%
\pgfpathmoveto{\pgfqpoint{0.612319in}{3.180269in}}%
\pgfpathcurveto{\pgfqpoint{0.623369in}{3.180269in}}{\pgfqpoint{0.633968in}{3.184660in}}{\pgfqpoint{0.641781in}{3.192473in}}%
\pgfpathcurveto{\pgfqpoint{0.649595in}{3.200287in}}{\pgfqpoint{0.653985in}{3.210886in}}{\pgfqpoint{0.653985in}{3.221936in}}%
\pgfpathcurveto{\pgfqpoint{0.653985in}{3.232986in}}{\pgfqpoint{0.649595in}{3.243585in}}{\pgfqpoint{0.641781in}{3.251399in}}%
\pgfpathcurveto{\pgfqpoint{0.633968in}{3.259212in}}{\pgfqpoint{0.623369in}{3.263603in}}{\pgfqpoint{0.612319in}{3.263603in}}%
\pgfpathcurveto{\pgfqpoint{0.601269in}{3.263603in}}{\pgfqpoint{0.590669in}{3.259212in}}{\pgfqpoint{0.582856in}{3.251399in}}%
\pgfpathcurveto{\pgfqpoint{0.575042in}{3.243585in}}{\pgfqpoint{0.570652in}{3.232986in}}{\pgfqpoint{0.570652in}{3.221936in}}%
\pgfpathcurveto{\pgfqpoint{0.570652in}{3.210886in}}{\pgfqpoint{0.575042in}{3.200287in}}{\pgfqpoint{0.582856in}{3.192473in}}%
\pgfpathcurveto{\pgfqpoint{0.590669in}{3.184660in}}{\pgfqpoint{0.601269in}{3.180269in}}{\pgfqpoint{0.612319in}{3.180269in}}%
\pgfpathlineto{\pgfqpoint{0.612319in}{3.180269in}}%
\pgfpathclose%
\pgfusepath{stroke}%
\end{pgfscope}%
\begin{pgfscope}%
\pgfpathrectangle{\pgfqpoint{0.393053in}{0.375000in}}{\pgfqpoint{6.356833in}{5.175000in}}%
\pgfusepath{clip}%
\pgfsetbuttcap%
\pgfsetroundjoin%
\pgfsetlinewidth{1.003750pt}%
\definecolor{currentstroke}{rgb}{0.827451,0.827451,0.827451}%
\pgfsetstrokecolor{currentstroke}%
\pgfsetdash{}{0pt}%
\pgfpathmoveto{\pgfqpoint{1.354968in}{1.928632in}}%
\pgfpathcurveto{\pgfqpoint{1.366018in}{1.928632in}}{\pgfqpoint{1.376617in}{1.933022in}}{\pgfqpoint{1.384431in}{1.940836in}}%
\pgfpathcurveto{\pgfqpoint{1.392245in}{1.948650in}}{\pgfqpoint{1.396635in}{1.959249in}}{\pgfqpoint{1.396635in}{1.970299in}}%
\pgfpathcurveto{\pgfqpoint{1.396635in}{1.981349in}}{\pgfqpoint{1.392245in}{1.991948in}}{\pgfqpoint{1.384431in}{1.999762in}}%
\pgfpathcurveto{\pgfqpoint{1.376617in}{2.007575in}}{\pgfqpoint{1.366018in}{2.011966in}}{\pgfqpoint{1.354968in}{2.011966in}}%
\pgfpathcurveto{\pgfqpoint{1.343918in}{2.011966in}}{\pgfqpoint{1.333319in}{2.007575in}}{\pgfqpoint{1.325505in}{1.999762in}}%
\pgfpathcurveto{\pgfqpoint{1.317692in}{1.991948in}}{\pgfqpoint{1.313302in}{1.981349in}}{\pgfqpoint{1.313302in}{1.970299in}}%
\pgfpathcurveto{\pgfqpoint{1.313302in}{1.959249in}}{\pgfqpoint{1.317692in}{1.948650in}}{\pgfqpoint{1.325505in}{1.940836in}}%
\pgfpathcurveto{\pgfqpoint{1.333319in}{1.933022in}}{\pgfqpoint{1.343918in}{1.928632in}}{\pgfqpoint{1.354968in}{1.928632in}}%
\pgfpathlineto{\pgfqpoint{1.354968in}{1.928632in}}%
\pgfpathclose%
\pgfusepath{stroke}%
\end{pgfscope}%
\begin{pgfscope}%
\pgfpathrectangle{\pgfqpoint{0.393053in}{0.375000in}}{\pgfqpoint{6.356833in}{5.175000in}}%
\pgfusepath{clip}%
\pgfsetbuttcap%
\pgfsetroundjoin%
\pgfsetlinewidth{1.003750pt}%
\definecolor{currentstroke}{rgb}{0.827451,0.827451,0.827451}%
\pgfsetstrokecolor{currentstroke}%
\pgfsetdash{}{0pt}%
\pgfpathmoveto{\pgfqpoint{0.451470in}{3.913124in}}%
\pgfpathcurveto{\pgfqpoint{0.462520in}{3.913124in}}{\pgfqpoint{0.473119in}{3.917514in}}{\pgfqpoint{0.480933in}{3.925328in}}%
\pgfpathcurveto{\pgfqpoint{0.488746in}{3.933141in}}{\pgfqpoint{0.493136in}{3.943740in}}{\pgfqpoint{0.493136in}{3.954790in}}%
\pgfpathcurveto{\pgfqpoint{0.493136in}{3.965840in}}{\pgfqpoint{0.488746in}{3.976439in}}{\pgfqpoint{0.480933in}{3.984253in}}%
\pgfpathcurveto{\pgfqpoint{0.473119in}{3.992067in}}{\pgfqpoint{0.462520in}{3.996457in}}{\pgfqpoint{0.451470in}{3.996457in}}%
\pgfpathcurveto{\pgfqpoint{0.440420in}{3.996457in}}{\pgfqpoint{0.429821in}{3.992067in}}{\pgfqpoint{0.422007in}{3.984253in}}%
\pgfpathcurveto{\pgfqpoint{0.414193in}{3.976439in}}{\pgfqpoint{0.409803in}{3.965840in}}{\pgfqpoint{0.409803in}{3.954790in}}%
\pgfpathcurveto{\pgfqpoint{0.409803in}{3.943740in}}{\pgfqpoint{0.414193in}{3.933141in}}{\pgfqpoint{0.422007in}{3.925328in}}%
\pgfpathcurveto{\pgfqpoint{0.429821in}{3.917514in}}{\pgfqpoint{0.440420in}{3.913124in}}{\pgfqpoint{0.451470in}{3.913124in}}%
\pgfpathlineto{\pgfqpoint{0.451470in}{3.913124in}}%
\pgfpathclose%
\pgfusepath{stroke}%
\end{pgfscope}%
\begin{pgfscope}%
\pgfpathrectangle{\pgfqpoint{0.393053in}{0.375000in}}{\pgfqpoint{6.356833in}{5.175000in}}%
\pgfusepath{clip}%
\pgfsetbuttcap%
\pgfsetroundjoin%
\pgfsetlinewidth{1.003750pt}%
\definecolor{currentstroke}{rgb}{0.827451,0.827451,0.827451}%
\pgfsetstrokecolor{currentstroke}%
\pgfsetdash{}{0pt}%
\pgfpathmoveto{\pgfqpoint{1.018989in}{2.457000in}}%
\pgfpathcurveto{\pgfqpoint{1.030039in}{2.457000in}}{\pgfqpoint{1.040638in}{2.461390in}}{\pgfqpoint{1.048452in}{2.469204in}}%
\pgfpathcurveto{\pgfqpoint{1.056265in}{2.477017in}}{\pgfqpoint{1.060655in}{2.487616in}}{\pgfqpoint{1.060655in}{2.498666in}}%
\pgfpathcurveto{\pgfqpoint{1.060655in}{2.509717in}}{\pgfqpoint{1.056265in}{2.520316in}}{\pgfqpoint{1.048452in}{2.528129in}}%
\pgfpathcurveto{\pgfqpoint{1.040638in}{2.535943in}}{\pgfqpoint{1.030039in}{2.540333in}}{\pgfqpoint{1.018989in}{2.540333in}}%
\pgfpathcurveto{\pgfqpoint{1.007939in}{2.540333in}}{\pgfqpoint{0.997340in}{2.535943in}}{\pgfqpoint{0.989526in}{2.528129in}}%
\pgfpathcurveto{\pgfqpoint{0.981712in}{2.520316in}}{\pgfqpoint{0.977322in}{2.509717in}}{\pgfqpoint{0.977322in}{2.498666in}}%
\pgfpathcurveto{\pgfqpoint{0.977322in}{2.487616in}}{\pgfqpoint{0.981712in}{2.477017in}}{\pgfqpoint{0.989526in}{2.469204in}}%
\pgfpathcurveto{\pgfqpoint{0.997340in}{2.461390in}}{\pgfqpoint{1.007939in}{2.457000in}}{\pgfqpoint{1.018989in}{2.457000in}}%
\pgfpathlineto{\pgfqpoint{1.018989in}{2.457000in}}%
\pgfpathclose%
\pgfusepath{stroke}%
\end{pgfscope}%
\begin{pgfscope}%
\pgfpathrectangle{\pgfqpoint{0.393053in}{0.375000in}}{\pgfqpoint{6.356833in}{5.175000in}}%
\pgfusepath{clip}%
\pgfsetbuttcap%
\pgfsetroundjoin%
\pgfsetlinewidth{1.003750pt}%
\definecolor{currentstroke}{rgb}{0.827451,0.827451,0.827451}%
\pgfsetstrokecolor{currentstroke}%
\pgfsetdash{}{0pt}%
\pgfpathmoveto{\pgfqpoint{1.466180in}{1.813354in}}%
\pgfpathcurveto{\pgfqpoint{1.477230in}{1.813354in}}{\pgfqpoint{1.487829in}{1.817745in}}{\pgfqpoint{1.495643in}{1.825558in}}%
\pgfpathcurveto{\pgfqpoint{1.503457in}{1.833372in}}{\pgfqpoint{1.507847in}{1.843971in}}{\pgfqpoint{1.507847in}{1.855021in}}%
\pgfpathcurveto{\pgfqpoint{1.507847in}{1.866071in}}{\pgfqpoint{1.503457in}{1.876670in}}{\pgfqpoint{1.495643in}{1.884484in}}%
\pgfpathcurveto{\pgfqpoint{1.487829in}{1.892298in}}{\pgfqpoint{1.477230in}{1.896688in}}{\pgfqpoint{1.466180in}{1.896688in}}%
\pgfpathcurveto{\pgfqpoint{1.455130in}{1.896688in}}{\pgfqpoint{1.444531in}{1.892298in}}{\pgfqpoint{1.436717in}{1.884484in}}%
\pgfpathcurveto{\pgfqpoint{1.428904in}{1.876670in}}{\pgfqpoint{1.424514in}{1.866071in}}{\pgfqpoint{1.424514in}{1.855021in}}%
\pgfpathcurveto{\pgfqpoint{1.424514in}{1.843971in}}{\pgfqpoint{1.428904in}{1.833372in}}{\pgfqpoint{1.436717in}{1.825558in}}%
\pgfpathcurveto{\pgfqpoint{1.444531in}{1.817745in}}{\pgfqpoint{1.455130in}{1.813354in}}{\pgfqpoint{1.466180in}{1.813354in}}%
\pgfpathlineto{\pgfqpoint{1.466180in}{1.813354in}}%
\pgfpathclose%
\pgfusepath{stroke}%
\end{pgfscope}%
\begin{pgfscope}%
\pgfpathrectangle{\pgfqpoint{0.393053in}{0.375000in}}{\pgfqpoint{6.356833in}{5.175000in}}%
\pgfusepath{clip}%
\pgfsetbuttcap%
\pgfsetroundjoin%
\pgfsetlinewidth{1.003750pt}%
\definecolor{currentstroke}{rgb}{0.827451,0.827451,0.827451}%
\pgfsetstrokecolor{currentstroke}%
\pgfsetdash{}{0pt}%
\pgfpathmoveto{\pgfqpoint{4.176160in}{0.456257in}}%
\pgfpathcurveto{\pgfqpoint{4.187210in}{0.456257in}}{\pgfqpoint{4.197809in}{0.460647in}}{\pgfqpoint{4.205623in}{0.468461in}}%
\pgfpathcurveto{\pgfqpoint{4.213436in}{0.476274in}}{\pgfqpoint{4.217827in}{0.486873in}}{\pgfqpoint{4.217827in}{0.497923in}}%
\pgfpathcurveto{\pgfqpoint{4.217827in}{0.508973in}}{\pgfqpoint{4.213436in}{0.519572in}}{\pgfqpoint{4.205623in}{0.527386in}}%
\pgfpathcurveto{\pgfqpoint{4.197809in}{0.535200in}}{\pgfqpoint{4.187210in}{0.539590in}}{\pgfqpoint{4.176160in}{0.539590in}}%
\pgfpathcurveto{\pgfqpoint{4.165110in}{0.539590in}}{\pgfqpoint{4.154511in}{0.535200in}}{\pgfqpoint{4.146697in}{0.527386in}}%
\pgfpathcurveto{\pgfqpoint{4.138883in}{0.519572in}}{\pgfqpoint{4.134493in}{0.508973in}}{\pgfqpoint{4.134493in}{0.497923in}}%
\pgfpathcurveto{\pgfqpoint{4.134493in}{0.486873in}}{\pgfqpoint{4.138883in}{0.476274in}}{\pgfqpoint{4.146697in}{0.468461in}}%
\pgfpathcurveto{\pgfqpoint{4.154511in}{0.460647in}}{\pgfqpoint{4.165110in}{0.456257in}}{\pgfqpoint{4.176160in}{0.456257in}}%
\pgfpathlineto{\pgfqpoint{4.176160in}{0.456257in}}%
\pgfpathclose%
\pgfusepath{stroke}%
\end{pgfscope}%
\begin{pgfscope}%
\pgfpathrectangle{\pgfqpoint{0.393053in}{0.375000in}}{\pgfqpoint{6.356833in}{5.175000in}}%
\pgfusepath{clip}%
\pgfsetbuttcap%
\pgfsetroundjoin%
\pgfsetlinewidth{1.003750pt}%
\definecolor{currentstroke}{rgb}{0.827451,0.827451,0.827451}%
\pgfsetstrokecolor{currentstroke}%
\pgfsetdash{}{0pt}%
\pgfpathmoveto{\pgfqpoint{1.375643in}{1.906128in}}%
\pgfpathcurveto{\pgfqpoint{1.386694in}{1.906128in}}{\pgfqpoint{1.397293in}{1.910518in}}{\pgfqpoint{1.405106in}{1.918331in}}%
\pgfpathcurveto{\pgfqpoint{1.412920in}{1.926145in}}{\pgfqpoint{1.417310in}{1.936744in}}{\pgfqpoint{1.417310in}{1.947794in}}%
\pgfpathcurveto{\pgfqpoint{1.417310in}{1.958844in}}{\pgfqpoint{1.412920in}{1.969443in}}{\pgfqpoint{1.405106in}{1.977257in}}%
\pgfpathcurveto{\pgfqpoint{1.397293in}{1.985071in}}{\pgfqpoint{1.386694in}{1.989461in}}{\pgfqpoint{1.375643in}{1.989461in}}%
\pgfpathcurveto{\pgfqpoint{1.364593in}{1.989461in}}{\pgfqpoint{1.353994in}{1.985071in}}{\pgfqpoint{1.346181in}{1.977257in}}%
\pgfpathcurveto{\pgfqpoint{1.338367in}{1.969443in}}{\pgfqpoint{1.333977in}{1.958844in}}{\pgfqpoint{1.333977in}{1.947794in}}%
\pgfpathcurveto{\pgfqpoint{1.333977in}{1.936744in}}{\pgfqpoint{1.338367in}{1.926145in}}{\pgfqpoint{1.346181in}{1.918331in}}%
\pgfpathcurveto{\pgfqpoint{1.353994in}{1.910518in}}{\pgfqpoint{1.364593in}{1.906128in}}{\pgfqpoint{1.375643in}{1.906128in}}%
\pgfpathlineto{\pgfqpoint{1.375643in}{1.906128in}}%
\pgfpathclose%
\pgfusepath{stroke}%
\end{pgfscope}%
\begin{pgfscope}%
\pgfpathrectangle{\pgfqpoint{0.393053in}{0.375000in}}{\pgfqpoint{6.356833in}{5.175000in}}%
\pgfusepath{clip}%
\pgfsetbuttcap%
\pgfsetroundjoin%
\pgfsetlinewidth{1.003750pt}%
\definecolor{currentstroke}{rgb}{0.827451,0.827451,0.827451}%
\pgfsetstrokecolor{currentstroke}%
\pgfsetdash{}{0pt}%
\pgfpathmoveto{\pgfqpoint{0.494809in}{3.599260in}}%
\pgfpathcurveto{\pgfqpoint{0.505859in}{3.599260in}}{\pgfqpoint{0.516458in}{3.603650in}}{\pgfqpoint{0.524272in}{3.611464in}}%
\pgfpathcurveto{\pgfqpoint{0.532085in}{3.619278in}}{\pgfqpoint{0.536476in}{3.629877in}}{\pgfqpoint{0.536476in}{3.640927in}}%
\pgfpathcurveto{\pgfqpoint{0.536476in}{3.651977in}}{\pgfqpoint{0.532085in}{3.662576in}}{\pgfqpoint{0.524272in}{3.670390in}}%
\pgfpathcurveto{\pgfqpoint{0.516458in}{3.678203in}}{\pgfqpoint{0.505859in}{3.682594in}}{\pgfqpoint{0.494809in}{3.682594in}}%
\pgfpathcurveto{\pgfqpoint{0.483759in}{3.682594in}}{\pgfqpoint{0.473160in}{3.678203in}}{\pgfqpoint{0.465346in}{3.670390in}}%
\pgfpathcurveto{\pgfqpoint{0.457533in}{3.662576in}}{\pgfqpoint{0.453142in}{3.651977in}}{\pgfqpoint{0.453142in}{3.640927in}}%
\pgfpathcurveto{\pgfqpoint{0.453142in}{3.629877in}}{\pgfqpoint{0.457533in}{3.619278in}}{\pgfqpoint{0.465346in}{3.611464in}}%
\pgfpathcurveto{\pgfqpoint{0.473160in}{3.603650in}}{\pgfqpoint{0.483759in}{3.599260in}}{\pgfqpoint{0.494809in}{3.599260in}}%
\pgfpathlineto{\pgfqpoint{0.494809in}{3.599260in}}%
\pgfpathclose%
\pgfusepath{stroke}%
\end{pgfscope}%
\begin{pgfscope}%
\pgfpathrectangle{\pgfqpoint{0.393053in}{0.375000in}}{\pgfqpoint{6.356833in}{5.175000in}}%
\pgfusepath{clip}%
\pgfsetbuttcap%
\pgfsetroundjoin%
\pgfsetlinewidth{1.003750pt}%
\definecolor{currentstroke}{rgb}{0.827451,0.827451,0.827451}%
\pgfsetstrokecolor{currentstroke}%
\pgfsetdash{}{0pt}%
\pgfpathmoveto{\pgfqpoint{0.773700in}{2.823025in}}%
\pgfpathcurveto{\pgfqpoint{0.784750in}{2.823025in}}{\pgfqpoint{0.795350in}{2.827415in}}{\pgfqpoint{0.803163in}{2.835229in}}%
\pgfpathcurveto{\pgfqpoint{0.810977in}{2.843042in}}{\pgfqpoint{0.815367in}{2.853641in}}{\pgfqpoint{0.815367in}{2.864691in}}%
\pgfpathcurveto{\pgfqpoint{0.815367in}{2.875742in}}{\pgfqpoint{0.810977in}{2.886341in}}{\pgfqpoint{0.803163in}{2.894154in}}%
\pgfpathcurveto{\pgfqpoint{0.795350in}{2.901968in}}{\pgfqpoint{0.784750in}{2.906358in}}{\pgfqpoint{0.773700in}{2.906358in}}%
\pgfpathcurveto{\pgfqpoint{0.762650in}{2.906358in}}{\pgfqpoint{0.752051in}{2.901968in}}{\pgfqpoint{0.744238in}{2.894154in}}%
\pgfpathcurveto{\pgfqpoint{0.736424in}{2.886341in}}{\pgfqpoint{0.732034in}{2.875742in}}{\pgfqpoint{0.732034in}{2.864691in}}%
\pgfpathcurveto{\pgfqpoint{0.732034in}{2.853641in}}{\pgfqpoint{0.736424in}{2.843042in}}{\pgfqpoint{0.744238in}{2.835229in}}%
\pgfpathcurveto{\pgfqpoint{0.752051in}{2.827415in}}{\pgfqpoint{0.762650in}{2.823025in}}{\pgfqpoint{0.773700in}{2.823025in}}%
\pgfpathlineto{\pgfqpoint{0.773700in}{2.823025in}}%
\pgfpathclose%
\pgfusepath{stroke}%
\end{pgfscope}%
\begin{pgfscope}%
\pgfpathrectangle{\pgfqpoint{0.393053in}{0.375000in}}{\pgfqpoint{6.356833in}{5.175000in}}%
\pgfusepath{clip}%
\pgfsetbuttcap%
\pgfsetroundjoin%
\pgfsetlinewidth{1.003750pt}%
\definecolor{currentstroke}{rgb}{0.827451,0.827451,0.827451}%
\pgfsetstrokecolor{currentstroke}%
\pgfsetdash{}{0pt}%
\pgfpathmoveto{\pgfqpoint{0.397222in}{4.408616in}}%
\pgfpathcurveto{\pgfqpoint{0.408272in}{4.408616in}}{\pgfqpoint{0.418871in}{4.413006in}}{\pgfqpoint{0.426685in}{4.420819in}}%
\pgfpathcurveto{\pgfqpoint{0.434499in}{4.428633in}}{\pgfqpoint{0.438889in}{4.439232in}}{\pgfqpoint{0.438889in}{4.450282in}}%
\pgfpathcurveto{\pgfqpoint{0.438889in}{4.461332in}}{\pgfqpoint{0.434499in}{4.471931in}}{\pgfqpoint{0.426685in}{4.479745in}}%
\pgfpathcurveto{\pgfqpoint{0.418871in}{4.487559in}}{\pgfqpoint{0.408272in}{4.491949in}}{\pgfqpoint{0.397222in}{4.491949in}}%
\pgfpathcurveto{\pgfqpoint{0.386172in}{4.491949in}}{\pgfqpoint{0.375573in}{4.487559in}}{\pgfqpoint{0.367760in}{4.479745in}}%
\pgfpathcurveto{\pgfqpoint{0.359946in}{4.471931in}}{\pgfqpoint{0.355556in}{4.461332in}}{\pgfqpoint{0.355556in}{4.450282in}}%
\pgfpathcurveto{\pgfqpoint{0.355556in}{4.439232in}}{\pgfqpoint{0.359946in}{4.428633in}}{\pgfqpoint{0.367760in}{4.420819in}}%
\pgfpathcurveto{\pgfqpoint{0.375573in}{4.413006in}}{\pgfqpoint{0.386172in}{4.408616in}}{\pgfqpoint{0.397222in}{4.408616in}}%
\pgfpathlineto{\pgfqpoint{0.397222in}{4.408616in}}%
\pgfpathclose%
\pgfusepath{stroke}%
\end{pgfscope}%
\begin{pgfscope}%
\pgfpathrectangle{\pgfqpoint{0.393053in}{0.375000in}}{\pgfqpoint{6.356833in}{5.175000in}}%
\pgfusepath{clip}%
\pgfsetbuttcap%
\pgfsetroundjoin%
\pgfsetlinewidth{1.003750pt}%
\definecolor{currentstroke}{rgb}{0.827451,0.827451,0.827451}%
\pgfsetstrokecolor{currentstroke}%
\pgfsetdash{}{0pt}%
\pgfpathmoveto{\pgfqpoint{0.396398in}{4.410130in}}%
\pgfpathcurveto{\pgfqpoint{0.407448in}{4.410130in}}{\pgfqpoint{0.418047in}{4.414521in}}{\pgfqpoint{0.425861in}{4.422334in}}%
\pgfpathcurveto{\pgfqpoint{0.433674in}{4.430148in}}{\pgfqpoint{0.438064in}{4.440747in}}{\pgfqpoint{0.438064in}{4.451797in}}%
\pgfpathcurveto{\pgfqpoint{0.438064in}{4.462847in}}{\pgfqpoint{0.433674in}{4.473446in}}{\pgfqpoint{0.425861in}{4.481260in}}%
\pgfpathcurveto{\pgfqpoint{0.418047in}{4.489073in}}{\pgfqpoint{0.407448in}{4.493464in}}{\pgfqpoint{0.396398in}{4.493464in}}%
\pgfpathcurveto{\pgfqpoint{0.385348in}{4.493464in}}{\pgfqpoint{0.374749in}{4.489073in}}{\pgfqpoint{0.366935in}{4.481260in}}%
\pgfpathcurveto{\pgfqpoint{0.359121in}{4.473446in}}{\pgfqpoint{0.354731in}{4.462847in}}{\pgfqpoint{0.354731in}{4.451797in}}%
\pgfpathcurveto{\pgfqpoint{0.354731in}{4.440747in}}{\pgfqpoint{0.359121in}{4.430148in}}{\pgfqpoint{0.366935in}{4.422334in}}%
\pgfpathcurveto{\pgfqpoint{0.374749in}{4.414521in}}{\pgfqpoint{0.385348in}{4.410130in}}{\pgfqpoint{0.396398in}{4.410130in}}%
\pgfpathlineto{\pgfqpoint{0.396398in}{4.410130in}}%
\pgfpathclose%
\pgfusepath{stroke}%
\end{pgfscope}%
\begin{pgfscope}%
\pgfpathrectangle{\pgfqpoint{0.393053in}{0.375000in}}{\pgfqpoint{6.356833in}{5.175000in}}%
\pgfusepath{clip}%
\pgfsetbuttcap%
\pgfsetroundjoin%
\pgfsetlinewidth{1.003750pt}%
\definecolor{currentstroke}{rgb}{0.827451,0.827451,0.827451}%
\pgfsetstrokecolor{currentstroke}%
\pgfsetdash{}{0pt}%
\pgfpathmoveto{\pgfqpoint{5.724126in}{0.333770in}}%
\pgfpathcurveto{\pgfqpoint{5.735176in}{0.333770in}}{\pgfqpoint{5.745775in}{0.338161in}}{\pgfqpoint{5.753589in}{0.345974in}}%
\pgfpathcurveto{\pgfqpoint{5.761402in}{0.353788in}}{\pgfqpoint{5.765793in}{0.364387in}}{\pgfqpoint{5.765793in}{0.375437in}}%
\pgfpathcurveto{\pgfqpoint{5.765793in}{0.386487in}}{\pgfqpoint{5.761402in}{0.397086in}}{\pgfqpoint{5.753589in}{0.404900in}}%
\pgfpathcurveto{\pgfqpoint{5.745775in}{0.412713in}}{\pgfqpoint{5.735176in}{0.417104in}}{\pgfqpoint{5.724126in}{0.417104in}}%
\pgfpathcurveto{\pgfqpoint{5.713076in}{0.417104in}}{\pgfqpoint{5.702477in}{0.412713in}}{\pgfqpoint{5.694663in}{0.404900in}}%
\pgfpathcurveto{\pgfqpoint{5.686850in}{0.397086in}}{\pgfqpoint{5.682459in}{0.386487in}}{\pgfqpoint{5.682459in}{0.375437in}}%
\pgfpathcurveto{\pgfqpoint{5.682459in}{0.364387in}}{\pgfqpoint{5.686850in}{0.353788in}}{\pgfqpoint{5.694663in}{0.345974in}}%
\pgfpathcurveto{\pgfqpoint{5.702477in}{0.338161in}}{\pgfqpoint{5.713076in}{0.333770in}}{\pgfqpoint{5.724126in}{0.333770in}}%
\pgfusepath{stroke}%
\end{pgfscope}%
\begin{pgfscope}%
\pgfpathrectangle{\pgfqpoint{0.393053in}{0.375000in}}{\pgfqpoint{6.356833in}{5.175000in}}%
\pgfusepath{clip}%
\pgfsetbuttcap%
\pgfsetroundjoin%
\pgfsetlinewidth{1.003750pt}%
\definecolor{currentstroke}{rgb}{0.827451,0.827451,0.827451}%
\pgfsetstrokecolor{currentstroke}%
\pgfsetdash{}{0pt}%
\pgfpathmoveto{\pgfqpoint{0.393276in}{4.535432in}}%
\pgfpathcurveto{\pgfqpoint{0.404326in}{4.535432in}}{\pgfqpoint{0.414925in}{4.539822in}}{\pgfqpoint{0.422739in}{4.547635in}}%
\pgfpathcurveto{\pgfqpoint{0.430552in}{4.555449in}}{\pgfqpoint{0.434942in}{4.566048in}}{\pgfqpoint{0.434942in}{4.577098in}}%
\pgfpathcurveto{\pgfqpoint{0.434942in}{4.588148in}}{\pgfqpoint{0.430552in}{4.598747in}}{\pgfqpoint{0.422739in}{4.606561in}}%
\pgfpathcurveto{\pgfqpoint{0.414925in}{4.614375in}}{\pgfqpoint{0.404326in}{4.618765in}}{\pgfqpoint{0.393276in}{4.618765in}}%
\pgfpathcurveto{\pgfqpoint{0.382226in}{4.618765in}}{\pgfqpoint{0.371627in}{4.614375in}}{\pgfqpoint{0.363813in}{4.606561in}}%
\pgfpathcurveto{\pgfqpoint{0.355999in}{4.598747in}}{\pgfqpoint{0.351609in}{4.588148in}}{\pgfqpoint{0.351609in}{4.577098in}}%
\pgfpathcurveto{\pgfqpoint{0.351609in}{4.566048in}}{\pgfqpoint{0.355999in}{4.555449in}}{\pgfqpoint{0.363813in}{4.547635in}}%
\pgfpathcurveto{\pgfqpoint{0.371627in}{4.539822in}}{\pgfqpoint{0.382226in}{4.535432in}}{\pgfqpoint{0.393276in}{4.535432in}}%
\pgfpathlineto{\pgfqpoint{0.393276in}{4.535432in}}%
\pgfpathclose%
\pgfusepath{stroke}%
\end{pgfscope}%
\begin{pgfscope}%
\pgfpathrectangle{\pgfqpoint{0.393053in}{0.375000in}}{\pgfqpoint{6.356833in}{5.175000in}}%
\pgfusepath{clip}%
\pgfsetbuttcap%
\pgfsetroundjoin%
\pgfsetlinewidth{1.003750pt}%
\definecolor{currentstroke}{rgb}{0.827451,0.827451,0.827451}%
\pgfsetstrokecolor{currentstroke}%
\pgfsetdash{}{0pt}%
\pgfpathmoveto{\pgfqpoint{0.635911in}{3.119895in}}%
\pgfpathcurveto{\pgfqpoint{0.646961in}{3.119895in}}{\pgfqpoint{0.657560in}{3.124285in}}{\pgfqpoint{0.665374in}{3.132098in}}%
\pgfpathcurveto{\pgfqpoint{0.673187in}{3.139912in}}{\pgfqpoint{0.677577in}{3.150511in}}{\pgfqpoint{0.677577in}{3.161561in}}%
\pgfpathcurveto{\pgfqpoint{0.677577in}{3.172611in}}{\pgfqpoint{0.673187in}{3.183210in}}{\pgfqpoint{0.665374in}{3.191024in}}%
\pgfpathcurveto{\pgfqpoint{0.657560in}{3.198838in}}{\pgfqpoint{0.646961in}{3.203228in}}{\pgfqpoint{0.635911in}{3.203228in}}%
\pgfpathcurveto{\pgfqpoint{0.624861in}{3.203228in}}{\pgfqpoint{0.614262in}{3.198838in}}{\pgfqpoint{0.606448in}{3.191024in}}%
\pgfpathcurveto{\pgfqpoint{0.598634in}{3.183210in}}{\pgfqpoint{0.594244in}{3.172611in}}{\pgfqpoint{0.594244in}{3.161561in}}%
\pgfpathcurveto{\pgfqpoint{0.594244in}{3.150511in}}{\pgfqpoint{0.598634in}{3.139912in}}{\pgfqpoint{0.606448in}{3.132098in}}%
\pgfpathcurveto{\pgfqpoint{0.614262in}{3.124285in}}{\pgfqpoint{0.624861in}{3.119895in}}{\pgfqpoint{0.635911in}{3.119895in}}%
\pgfpathlineto{\pgfqpoint{0.635911in}{3.119895in}}%
\pgfpathclose%
\pgfusepath{stroke}%
\end{pgfscope}%
\begin{pgfscope}%
\pgfpathrectangle{\pgfqpoint{0.393053in}{0.375000in}}{\pgfqpoint{6.356833in}{5.175000in}}%
\pgfusepath{clip}%
\pgfsetbuttcap%
\pgfsetroundjoin%
\pgfsetlinewidth{1.003750pt}%
\definecolor{currentstroke}{rgb}{0.827451,0.827451,0.827451}%
\pgfsetstrokecolor{currentstroke}%
\pgfsetdash{}{0pt}%
\pgfpathmoveto{\pgfqpoint{0.703396in}{2.946651in}}%
\pgfpathcurveto{\pgfqpoint{0.714446in}{2.946651in}}{\pgfqpoint{0.725045in}{2.951041in}}{\pgfqpoint{0.732858in}{2.958855in}}%
\pgfpathcurveto{\pgfqpoint{0.740672in}{2.966668in}}{\pgfqpoint{0.745062in}{2.977267in}}{\pgfqpoint{0.745062in}{2.988318in}}%
\pgfpathcurveto{\pgfqpoint{0.745062in}{2.999368in}}{\pgfqpoint{0.740672in}{3.009967in}}{\pgfqpoint{0.732858in}{3.017780in}}%
\pgfpathcurveto{\pgfqpoint{0.725045in}{3.025594in}}{\pgfqpoint{0.714446in}{3.029984in}}{\pgfqpoint{0.703396in}{3.029984in}}%
\pgfpathcurveto{\pgfqpoint{0.692345in}{3.029984in}}{\pgfqpoint{0.681746in}{3.025594in}}{\pgfqpoint{0.673933in}{3.017780in}}%
\pgfpathcurveto{\pgfqpoint{0.666119in}{3.009967in}}{\pgfqpoint{0.661729in}{2.999368in}}{\pgfqpoint{0.661729in}{2.988318in}}%
\pgfpathcurveto{\pgfqpoint{0.661729in}{2.977267in}}{\pgfqpoint{0.666119in}{2.966668in}}{\pgfqpoint{0.673933in}{2.958855in}}%
\pgfpathcurveto{\pgfqpoint{0.681746in}{2.951041in}}{\pgfqpoint{0.692345in}{2.946651in}}{\pgfqpoint{0.703396in}{2.946651in}}%
\pgfpathlineto{\pgfqpoint{0.703396in}{2.946651in}}%
\pgfpathclose%
\pgfusepath{stroke}%
\end{pgfscope}%
\begin{pgfscope}%
\pgfpathrectangle{\pgfqpoint{0.393053in}{0.375000in}}{\pgfqpoint{6.356833in}{5.175000in}}%
\pgfusepath{clip}%
\pgfsetbuttcap%
\pgfsetroundjoin%
\pgfsetlinewidth{1.003750pt}%
\definecolor{currentstroke}{rgb}{0.827451,0.827451,0.827451}%
\pgfsetstrokecolor{currentstroke}%
\pgfsetdash{}{0pt}%
\pgfpathmoveto{\pgfqpoint{3.055426in}{0.786859in}}%
\pgfpathcurveto{\pgfqpoint{3.066476in}{0.786859in}}{\pgfqpoint{3.077075in}{0.791249in}}{\pgfqpoint{3.084888in}{0.799063in}}%
\pgfpathcurveto{\pgfqpoint{3.092702in}{0.806876in}}{\pgfqpoint{3.097092in}{0.817476in}}{\pgfqpoint{3.097092in}{0.828526in}}%
\pgfpathcurveto{\pgfqpoint{3.097092in}{0.839576in}}{\pgfqpoint{3.092702in}{0.850175in}}{\pgfqpoint{3.084888in}{0.857988in}}%
\pgfpathcurveto{\pgfqpoint{3.077075in}{0.865802in}}{\pgfqpoint{3.066476in}{0.870192in}}{\pgfqpoint{3.055426in}{0.870192in}}%
\pgfpathcurveto{\pgfqpoint{3.044376in}{0.870192in}}{\pgfqpoint{3.033777in}{0.865802in}}{\pgfqpoint{3.025963in}{0.857988in}}%
\pgfpathcurveto{\pgfqpoint{3.018149in}{0.850175in}}{\pgfqpoint{3.013759in}{0.839576in}}{\pgfqpoint{3.013759in}{0.828526in}}%
\pgfpathcurveto{\pgfqpoint{3.013759in}{0.817476in}}{\pgfqpoint{3.018149in}{0.806876in}}{\pgfqpoint{3.025963in}{0.799063in}}%
\pgfpathcurveto{\pgfqpoint{3.033777in}{0.791249in}}{\pgfqpoint{3.044376in}{0.786859in}}{\pgfqpoint{3.055426in}{0.786859in}}%
\pgfpathlineto{\pgfqpoint{3.055426in}{0.786859in}}%
\pgfpathclose%
\pgfusepath{stroke}%
\end{pgfscope}%
\begin{pgfscope}%
\pgfpathrectangle{\pgfqpoint{0.393053in}{0.375000in}}{\pgfqpoint{6.356833in}{5.175000in}}%
\pgfusepath{clip}%
\pgfsetbuttcap%
\pgfsetroundjoin%
\pgfsetlinewidth{1.003750pt}%
\definecolor{currentstroke}{rgb}{0.827451,0.827451,0.827451}%
\pgfsetstrokecolor{currentstroke}%
\pgfsetdash{}{0pt}%
\pgfpathmoveto{\pgfqpoint{0.580834in}{3.297265in}}%
\pgfpathcurveto{\pgfqpoint{0.591884in}{3.297265in}}{\pgfqpoint{0.602483in}{3.301655in}}{\pgfqpoint{0.610297in}{3.309469in}}%
\pgfpathcurveto{\pgfqpoint{0.618111in}{3.317282in}}{\pgfqpoint{0.622501in}{3.327881in}}{\pgfqpoint{0.622501in}{3.338932in}}%
\pgfpathcurveto{\pgfqpoint{0.622501in}{3.349982in}}{\pgfqpoint{0.618111in}{3.360581in}}{\pgfqpoint{0.610297in}{3.368394in}}%
\pgfpathcurveto{\pgfqpoint{0.602483in}{3.376208in}}{\pgfqpoint{0.591884in}{3.380598in}}{\pgfqpoint{0.580834in}{3.380598in}}%
\pgfpathcurveto{\pgfqpoint{0.569784in}{3.380598in}}{\pgfqpoint{0.559185in}{3.376208in}}{\pgfqpoint{0.551371in}{3.368394in}}%
\pgfpathcurveto{\pgfqpoint{0.543558in}{3.360581in}}{\pgfqpoint{0.539167in}{3.349982in}}{\pgfqpoint{0.539167in}{3.338932in}}%
\pgfpathcurveto{\pgfqpoint{0.539167in}{3.327881in}}{\pgfqpoint{0.543558in}{3.317282in}}{\pgfqpoint{0.551371in}{3.309469in}}%
\pgfpathcurveto{\pgfqpoint{0.559185in}{3.301655in}}{\pgfqpoint{0.569784in}{3.297265in}}{\pgfqpoint{0.580834in}{3.297265in}}%
\pgfpathlineto{\pgfqpoint{0.580834in}{3.297265in}}%
\pgfpathclose%
\pgfusepath{stroke}%
\end{pgfscope}%
\begin{pgfscope}%
\pgfpathrectangle{\pgfqpoint{0.393053in}{0.375000in}}{\pgfqpoint{6.356833in}{5.175000in}}%
\pgfusepath{clip}%
\pgfsetbuttcap%
\pgfsetroundjoin%
\pgfsetlinewidth{1.003750pt}%
\definecolor{currentstroke}{rgb}{0.827451,0.827451,0.827451}%
\pgfsetstrokecolor{currentstroke}%
\pgfsetdash{}{0pt}%
\pgfpathmoveto{\pgfqpoint{0.612319in}{3.180269in}}%
\pgfpathcurveto{\pgfqpoint{0.623369in}{3.180269in}}{\pgfqpoint{0.633968in}{3.184660in}}{\pgfqpoint{0.641781in}{3.192473in}}%
\pgfpathcurveto{\pgfqpoint{0.649595in}{3.200287in}}{\pgfqpoint{0.653985in}{3.210886in}}{\pgfqpoint{0.653985in}{3.221936in}}%
\pgfpathcurveto{\pgfqpoint{0.653985in}{3.232986in}}{\pgfqpoint{0.649595in}{3.243585in}}{\pgfqpoint{0.641781in}{3.251399in}}%
\pgfpathcurveto{\pgfqpoint{0.633968in}{3.259212in}}{\pgfqpoint{0.623369in}{3.263603in}}{\pgfqpoint{0.612319in}{3.263603in}}%
\pgfpathcurveto{\pgfqpoint{0.601269in}{3.263603in}}{\pgfqpoint{0.590669in}{3.259212in}}{\pgfqpoint{0.582856in}{3.251399in}}%
\pgfpathcurveto{\pgfqpoint{0.575042in}{3.243585in}}{\pgfqpoint{0.570652in}{3.232986in}}{\pgfqpoint{0.570652in}{3.221936in}}%
\pgfpathcurveto{\pgfqpoint{0.570652in}{3.210886in}}{\pgfqpoint{0.575042in}{3.200287in}}{\pgfqpoint{0.582856in}{3.192473in}}%
\pgfpathcurveto{\pgfqpoint{0.590669in}{3.184660in}}{\pgfqpoint{0.601269in}{3.180269in}}{\pgfqpoint{0.612319in}{3.180269in}}%
\pgfpathlineto{\pgfqpoint{0.612319in}{3.180269in}}%
\pgfpathclose%
\pgfusepath{stroke}%
\end{pgfscope}%
\begin{pgfscope}%
\pgfpathrectangle{\pgfqpoint{0.393053in}{0.375000in}}{\pgfqpoint{6.356833in}{5.175000in}}%
\pgfusepath{clip}%
\pgfsetbuttcap%
\pgfsetroundjoin%
\pgfsetlinewidth{1.003750pt}%
\definecolor{currentstroke}{rgb}{0.827451,0.827451,0.827451}%
\pgfsetstrokecolor{currentstroke}%
\pgfsetdash{}{0pt}%
\pgfpathmoveto{\pgfqpoint{4.802412in}{0.377777in}}%
\pgfpathcurveto{\pgfqpoint{4.813462in}{0.377777in}}{\pgfqpoint{4.824061in}{0.382167in}}{\pgfqpoint{4.831874in}{0.389981in}}%
\pgfpathcurveto{\pgfqpoint{4.839688in}{0.397794in}}{\pgfqpoint{4.844078in}{0.408394in}}{\pgfqpoint{4.844078in}{0.419444in}}%
\pgfpathcurveto{\pgfqpoint{4.844078in}{0.430494in}}{\pgfqpoint{4.839688in}{0.441093in}}{\pgfqpoint{4.831874in}{0.448906in}}%
\pgfpathcurveto{\pgfqpoint{4.824061in}{0.456720in}}{\pgfqpoint{4.813462in}{0.461110in}}{\pgfqpoint{4.802412in}{0.461110in}}%
\pgfpathcurveto{\pgfqpoint{4.791362in}{0.461110in}}{\pgfqpoint{4.780762in}{0.456720in}}{\pgfqpoint{4.772949in}{0.448906in}}%
\pgfpathcurveto{\pgfqpoint{4.765135in}{0.441093in}}{\pgfqpoint{4.760745in}{0.430494in}}{\pgfqpoint{4.760745in}{0.419444in}}%
\pgfpathcurveto{\pgfqpoint{4.760745in}{0.408394in}}{\pgfqpoint{4.765135in}{0.397794in}}{\pgfqpoint{4.772949in}{0.389981in}}%
\pgfpathcurveto{\pgfqpoint{4.780762in}{0.382167in}}{\pgfqpoint{4.791362in}{0.377777in}}{\pgfqpoint{4.802412in}{0.377777in}}%
\pgfpathlineto{\pgfqpoint{4.802412in}{0.377777in}}%
\pgfpathclose%
\pgfusepath{stroke}%
\end{pgfscope}%
\begin{pgfscope}%
\pgfpathrectangle{\pgfqpoint{0.393053in}{0.375000in}}{\pgfqpoint{6.356833in}{5.175000in}}%
\pgfusepath{clip}%
\pgfsetbuttcap%
\pgfsetroundjoin%
\pgfsetlinewidth{1.003750pt}%
\definecolor{currentstroke}{rgb}{0.827451,0.827451,0.827451}%
\pgfsetstrokecolor{currentstroke}%
\pgfsetdash{}{0pt}%
\pgfpathmoveto{\pgfqpoint{0.474470in}{3.727925in}}%
\pgfpathcurveto{\pgfqpoint{0.485520in}{3.727925in}}{\pgfqpoint{0.496119in}{3.732315in}}{\pgfqpoint{0.503933in}{3.740129in}}%
\pgfpathcurveto{\pgfqpoint{0.511746in}{3.747943in}}{\pgfqpoint{0.516137in}{3.758542in}}{\pgfqpoint{0.516137in}{3.769592in}}%
\pgfpathcurveto{\pgfqpoint{0.516137in}{3.780642in}}{\pgfqpoint{0.511746in}{3.791241in}}{\pgfqpoint{0.503933in}{3.799055in}}%
\pgfpathcurveto{\pgfqpoint{0.496119in}{3.806868in}}{\pgfqpoint{0.485520in}{3.811259in}}{\pgfqpoint{0.474470in}{3.811259in}}%
\pgfpathcurveto{\pgfqpoint{0.463420in}{3.811259in}}{\pgfqpoint{0.452821in}{3.806868in}}{\pgfqpoint{0.445007in}{3.799055in}}%
\pgfpathcurveto{\pgfqpoint{0.437194in}{3.791241in}}{\pgfqpoint{0.432803in}{3.780642in}}{\pgfqpoint{0.432803in}{3.769592in}}%
\pgfpathcurveto{\pgfqpoint{0.432803in}{3.758542in}}{\pgfqpoint{0.437194in}{3.747943in}}{\pgfqpoint{0.445007in}{3.740129in}}%
\pgfpathcurveto{\pgfqpoint{0.452821in}{3.732315in}}{\pgfqpoint{0.463420in}{3.727925in}}{\pgfqpoint{0.474470in}{3.727925in}}%
\pgfpathlineto{\pgfqpoint{0.474470in}{3.727925in}}%
\pgfpathclose%
\pgfusepath{stroke}%
\end{pgfscope}%
\begin{pgfscope}%
\pgfpathrectangle{\pgfqpoint{0.393053in}{0.375000in}}{\pgfqpoint{6.356833in}{5.175000in}}%
\pgfusepath{clip}%
\pgfsetbuttcap%
\pgfsetroundjoin%
\pgfsetlinewidth{1.003750pt}%
\definecolor{currentstroke}{rgb}{0.827451,0.827451,0.827451}%
\pgfsetstrokecolor{currentstroke}%
\pgfsetdash{}{0pt}%
\pgfpathmoveto{\pgfqpoint{2.785690in}{0.904937in}}%
\pgfpathcurveto{\pgfqpoint{2.796740in}{0.904937in}}{\pgfqpoint{2.807339in}{0.909327in}}{\pgfqpoint{2.815153in}{0.917140in}}%
\pgfpathcurveto{\pgfqpoint{2.822966in}{0.924954in}}{\pgfqpoint{2.827357in}{0.935553in}}{\pgfqpoint{2.827357in}{0.946603in}}%
\pgfpathcurveto{\pgfqpoint{2.827357in}{0.957653in}}{\pgfqpoint{2.822966in}{0.968252in}}{\pgfqpoint{2.815153in}{0.976066in}}%
\pgfpathcurveto{\pgfqpoint{2.807339in}{0.983880in}}{\pgfqpoint{2.796740in}{0.988270in}}{\pgfqpoint{2.785690in}{0.988270in}}%
\pgfpathcurveto{\pgfqpoint{2.774640in}{0.988270in}}{\pgfqpoint{2.764041in}{0.983880in}}{\pgfqpoint{2.756227in}{0.976066in}}%
\pgfpathcurveto{\pgfqpoint{2.748414in}{0.968252in}}{\pgfqpoint{2.744023in}{0.957653in}}{\pgfqpoint{2.744023in}{0.946603in}}%
\pgfpathcurveto{\pgfqpoint{2.744023in}{0.935553in}}{\pgfqpoint{2.748414in}{0.924954in}}{\pgfqpoint{2.756227in}{0.917140in}}%
\pgfpathcurveto{\pgfqpoint{2.764041in}{0.909327in}}{\pgfqpoint{2.774640in}{0.904937in}}{\pgfqpoint{2.785690in}{0.904937in}}%
\pgfpathlineto{\pgfqpoint{2.785690in}{0.904937in}}%
\pgfpathclose%
\pgfusepath{stroke}%
\end{pgfscope}%
\begin{pgfscope}%
\pgfpathrectangle{\pgfqpoint{0.393053in}{0.375000in}}{\pgfqpoint{6.356833in}{5.175000in}}%
\pgfusepath{clip}%
\pgfsetbuttcap%
\pgfsetroundjoin%
\pgfsetlinewidth{1.003750pt}%
\definecolor{currentstroke}{rgb}{0.827451,0.827451,0.827451}%
\pgfsetstrokecolor{currentstroke}%
\pgfsetdash{}{0pt}%
\pgfpathmoveto{\pgfqpoint{3.496019in}{0.629204in}}%
\pgfpathcurveto{\pgfqpoint{3.507069in}{0.629204in}}{\pgfqpoint{3.517668in}{0.633594in}}{\pgfqpoint{3.525482in}{0.641407in}}%
\pgfpathcurveto{\pgfqpoint{3.533295in}{0.649221in}}{\pgfqpoint{3.537686in}{0.659820in}}{\pgfqpoint{3.537686in}{0.670870in}}%
\pgfpathcurveto{\pgfqpoint{3.537686in}{0.681920in}}{\pgfqpoint{3.533295in}{0.692519in}}{\pgfqpoint{3.525482in}{0.700333in}}%
\pgfpathcurveto{\pgfqpoint{3.517668in}{0.708147in}}{\pgfqpoint{3.507069in}{0.712537in}}{\pgfqpoint{3.496019in}{0.712537in}}%
\pgfpathcurveto{\pgfqpoint{3.484969in}{0.712537in}}{\pgfqpoint{3.474370in}{0.708147in}}{\pgfqpoint{3.466556in}{0.700333in}}%
\pgfpathcurveto{\pgfqpoint{3.458743in}{0.692519in}}{\pgfqpoint{3.454352in}{0.681920in}}{\pgfqpoint{3.454352in}{0.670870in}}%
\pgfpathcurveto{\pgfqpoint{3.454352in}{0.659820in}}{\pgfqpoint{3.458743in}{0.649221in}}{\pgfqpoint{3.466556in}{0.641407in}}%
\pgfpathcurveto{\pgfqpoint{3.474370in}{0.633594in}}{\pgfqpoint{3.484969in}{0.629204in}}{\pgfqpoint{3.496019in}{0.629204in}}%
\pgfpathlineto{\pgfqpoint{3.496019in}{0.629204in}}%
\pgfpathclose%
\pgfusepath{stroke}%
\end{pgfscope}%
\begin{pgfscope}%
\pgfpathrectangle{\pgfqpoint{0.393053in}{0.375000in}}{\pgfqpoint{6.356833in}{5.175000in}}%
\pgfusepath{clip}%
\pgfsetbuttcap%
\pgfsetroundjoin%
\pgfsetlinewidth{1.003750pt}%
\definecolor{currentstroke}{rgb}{0.827451,0.827451,0.827451}%
\pgfsetstrokecolor{currentstroke}%
\pgfsetdash{}{0pt}%
\pgfpathmoveto{\pgfqpoint{3.637654in}{0.581771in}}%
\pgfpathcurveto{\pgfqpoint{3.648705in}{0.581771in}}{\pgfqpoint{3.659304in}{0.586161in}}{\pgfqpoint{3.667117in}{0.593974in}}%
\pgfpathcurveto{\pgfqpoint{3.674931in}{0.601788in}}{\pgfqpoint{3.679321in}{0.612387in}}{\pgfqpoint{3.679321in}{0.623437in}}%
\pgfpathcurveto{\pgfqpoint{3.679321in}{0.634487in}}{\pgfqpoint{3.674931in}{0.645086in}}{\pgfqpoint{3.667117in}{0.652900in}}%
\pgfpathcurveto{\pgfqpoint{3.659304in}{0.660714in}}{\pgfqpoint{3.648705in}{0.665104in}}{\pgfqpoint{3.637654in}{0.665104in}}%
\pgfpathcurveto{\pgfqpoint{3.626604in}{0.665104in}}{\pgfqpoint{3.616005in}{0.660714in}}{\pgfqpoint{3.608192in}{0.652900in}}%
\pgfpathcurveto{\pgfqpoint{3.600378in}{0.645086in}}{\pgfqpoint{3.595988in}{0.634487in}}{\pgfqpoint{3.595988in}{0.623437in}}%
\pgfpathcurveto{\pgfqpoint{3.595988in}{0.612387in}}{\pgfqpoint{3.600378in}{0.601788in}}{\pgfqpoint{3.608192in}{0.593974in}}%
\pgfpathcurveto{\pgfqpoint{3.616005in}{0.586161in}}{\pgfqpoint{3.626604in}{0.581771in}}{\pgfqpoint{3.637654in}{0.581771in}}%
\pgfpathlineto{\pgfqpoint{3.637654in}{0.581771in}}%
\pgfpathclose%
\pgfusepath{stroke}%
\end{pgfscope}%
\begin{pgfscope}%
\pgfpathrectangle{\pgfqpoint{0.393053in}{0.375000in}}{\pgfqpoint{6.356833in}{5.175000in}}%
\pgfusepath{clip}%
\pgfsetbuttcap%
\pgfsetroundjoin%
\pgfsetlinewidth{1.003750pt}%
\definecolor{currentstroke}{rgb}{0.827451,0.827451,0.827451}%
\pgfsetstrokecolor{currentstroke}%
\pgfsetdash{}{0pt}%
\pgfpathmoveto{\pgfqpoint{1.466180in}{1.813354in}}%
\pgfpathcurveto{\pgfqpoint{1.477230in}{1.813354in}}{\pgfqpoint{1.487829in}{1.817745in}}{\pgfqpoint{1.495643in}{1.825558in}}%
\pgfpathcurveto{\pgfqpoint{1.503457in}{1.833372in}}{\pgfqpoint{1.507847in}{1.843971in}}{\pgfqpoint{1.507847in}{1.855021in}}%
\pgfpathcurveto{\pgfqpoint{1.507847in}{1.866071in}}{\pgfqpoint{1.503457in}{1.876670in}}{\pgfqpoint{1.495643in}{1.884484in}}%
\pgfpathcurveto{\pgfqpoint{1.487829in}{1.892298in}}{\pgfqpoint{1.477230in}{1.896688in}}{\pgfqpoint{1.466180in}{1.896688in}}%
\pgfpathcurveto{\pgfqpoint{1.455130in}{1.896688in}}{\pgfqpoint{1.444531in}{1.892298in}}{\pgfqpoint{1.436717in}{1.884484in}}%
\pgfpathcurveto{\pgfqpoint{1.428904in}{1.876670in}}{\pgfqpoint{1.424514in}{1.866071in}}{\pgfqpoint{1.424514in}{1.855021in}}%
\pgfpathcurveto{\pgfqpoint{1.424514in}{1.843971in}}{\pgfqpoint{1.428904in}{1.833372in}}{\pgfqpoint{1.436717in}{1.825558in}}%
\pgfpathcurveto{\pgfqpoint{1.444531in}{1.817745in}}{\pgfqpoint{1.455130in}{1.813354in}}{\pgfqpoint{1.466180in}{1.813354in}}%
\pgfpathlineto{\pgfqpoint{1.466180in}{1.813354in}}%
\pgfpathclose%
\pgfusepath{stroke}%
\end{pgfscope}%
\begin{pgfscope}%
\pgfpathrectangle{\pgfqpoint{0.393053in}{0.375000in}}{\pgfqpoint{6.356833in}{5.175000in}}%
\pgfusepath{clip}%
\pgfsetbuttcap%
\pgfsetroundjoin%
\pgfsetlinewidth{1.003750pt}%
\definecolor{currentstroke}{rgb}{0.827451,0.827451,0.827451}%
\pgfsetstrokecolor{currentstroke}%
\pgfsetdash{}{0pt}%
\pgfpathmoveto{\pgfqpoint{3.831790in}{0.529833in}}%
\pgfpathcurveto{\pgfqpoint{3.842841in}{0.529833in}}{\pgfqpoint{3.853440in}{0.534224in}}{\pgfqpoint{3.861253in}{0.542037in}}%
\pgfpathcurveto{\pgfqpoint{3.869067in}{0.549851in}}{\pgfqpoint{3.873457in}{0.560450in}}{\pgfqpoint{3.873457in}{0.571500in}}%
\pgfpathcurveto{\pgfqpoint{3.873457in}{0.582550in}}{\pgfqpoint{3.869067in}{0.593149in}}{\pgfqpoint{3.861253in}{0.600963in}}%
\pgfpathcurveto{\pgfqpoint{3.853440in}{0.608776in}}{\pgfqpoint{3.842841in}{0.613167in}}{\pgfqpoint{3.831790in}{0.613167in}}%
\pgfpathcurveto{\pgfqpoint{3.820740in}{0.613167in}}{\pgfqpoint{3.810141in}{0.608776in}}{\pgfqpoint{3.802328in}{0.600963in}}%
\pgfpathcurveto{\pgfqpoint{3.794514in}{0.593149in}}{\pgfqpoint{3.790124in}{0.582550in}}{\pgfqpoint{3.790124in}{0.571500in}}%
\pgfpathcurveto{\pgfqpoint{3.790124in}{0.560450in}}{\pgfqpoint{3.794514in}{0.549851in}}{\pgfqpoint{3.802328in}{0.542037in}}%
\pgfpathcurveto{\pgfqpoint{3.810141in}{0.534224in}}{\pgfqpoint{3.820740in}{0.529833in}}{\pgfqpoint{3.831790in}{0.529833in}}%
\pgfpathlineto{\pgfqpoint{3.831790in}{0.529833in}}%
\pgfpathclose%
\pgfusepath{stroke}%
\end{pgfscope}%
\begin{pgfscope}%
\pgfpathrectangle{\pgfqpoint{0.393053in}{0.375000in}}{\pgfqpoint{6.356833in}{5.175000in}}%
\pgfusepath{clip}%
\pgfsetbuttcap%
\pgfsetroundjoin%
\pgfsetlinewidth{1.003750pt}%
\definecolor{currentstroke}{rgb}{0.827451,0.827451,0.827451}%
\pgfsetstrokecolor{currentstroke}%
\pgfsetdash{}{0pt}%
\pgfpathmoveto{\pgfqpoint{1.702004in}{1.599430in}}%
\pgfpathcurveto{\pgfqpoint{1.713054in}{1.599430in}}{\pgfqpoint{1.723653in}{1.603820in}}{\pgfqpoint{1.731467in}{1.611633in}}%
\pgfpathcurveto{\pgfqpoint{1.739281in}{1.619447in}}{\pgfqpoint{1.743671in}{1.630046in}}{\pgfqpoint{1.743671in}{1.641096in}}%
\pgfpathcurveto{\pgfqpoint{1.743671in}{1.652146in}}{\pgfqpoint{1.739281in}{1.662745in}}{\pgfqpoint{1.731467in}{1.670559in}}%
\pgfpathcurveto{\pgfqpoint{1.723653in}{1.678373in}}{\pgfqpoint{1.713054in}{1.682763in}}{\pgfqpoint{1.702004in}{1.682763in}}%
\pgfpathcurveto{\pgfqpoint{1.690954in}{1.682763in}}{\pgfqpoint{1.680355in}{1.678373in}}{\pgfqpoint{1.672541in}{1.670559in}}%
\pgfpathcurveto{\pgfqpoint{1.664728in}{1.662745in}}{\pgfqpoint{1.660338in}{1.652146in}}{\pgfqpoint{1.660338in}{1.641096in}}%
\pgfpathcurveto{\pgfqpoint{1.660338in}{1.630046in}}{\pgfqpoint{1.664728in}{1.619447in}}{\pgfqpoint{1.672541in}{1.611633in}}%
\pgfpathcurveto{\pgfqpoint{1.680355in}{1.603820in}}{\pgfqpoint{1.690954in}{1.599430in}}{\pgfqpoint{1.702004in}{1.599430in}}%
\pgfpathlineto{\pgfqpoint{1.702004in}{1.599430in}}%
\pgfpathclose%
\pgfusepath{stroke}%
\end{pgfscope}%
\begin{pgfscope}%
\pgfpathrectangle{\pgfqpoint{0.393053in}{0.375000in}}{\pgfqpoint{6.356833in}{5.175000in}}%
\pgfusepath{clip}%
\pgfsetbuttcap%
\pgfsetroundjoin%
\pgfsetlinewidth{1.003750pt}%
\definecolor{currentstroke}{rgb}{0.827451,0.827451,0.827451}%
\pgfsetstrokecolor{currentstroke}%
\pgfsetdash{}{0pt}%
\pgfpathmoveto{\pgfqpoint{3.184544in}{0.747653in}}%
\pgfpathcurveto{\pgfqpoint{3.195594in}{0.747653in}}{\pgfqpoint{3.206193in}{0.752043in}}{\pgfqpoint{3.214007in}{0.759857in}}%
\pgfpathcurveto{\pgfqpoint{3.221820in}{0.767670in}}{\pgfqpoint{3.226210in}{0.778269in}}{\pgfqpoint{3.226210in}{0.789319in}}%
\pgfpathcurveto{\pgfqpoint{3.226210in}{0.800369in}}{\pgfqpoint{3.221820in}{0.810968in}}{\pgfqpoint{3.214007in}{0.818782in}}%
\pgfpathcurveto{\pgfqpoint{3.206193in}{0.826596in}}{\pgfqpoint{3.195594in}{0.830986in}}{\pgfqpoint{3.184544in}{0.830986in}}%
\pgfpathcurveto{\pgfqpoint{3.173494in}{0.830986in}}{\pgfqpoint{3.162895in}{0.826596in}}{\pgfqpoint{3.155081in}{0.818782in}}%
\pgfpathcurveto{\pgfqpoint{3.147267in}{0.810968in}}{\pgfqpoint{3.142877in}{0.800369in}}{\pgfqpoint{3.142877in}{0.789319in}}%
\pgfpathcurveto{\pgfqpoint{3.142877in}{0.778269in}}{\pgfqpoint{3.147267in}{0.767670in}}{\pgfqpoint{3.155081in}{0.759857in}}%
\pgfpathcurveto{\pgfqpoint{3.162895in}{0.752043in}}{\pgfqpoint{3.173494in}{0.747653in}}{\pgfqpoint{3.184544in}{0.747653in}}%
\pgfpathlineto{\pgfqpoint{3.184544in}{0.747653in}}%
\pgfpathclose%
\pgfusepath{stroke}%
\end{pgfscope}%
\begin{pgfscope}%
\pgfpathrectangle{\pgfqpoint{0.393053in}{0.375000in}}{\pgfqpoint{6.356833in}{5.175000in}}%
\pgfusepath{clip}%
\pgfsetbuttcap%
\pgfsetroundjoin%
\pgfsetlinewidth{1.003750pt}%
\definecolor{currentstroke}{rgb}{0.827451,0.827451,0.827451}%
\pgfsetstrokecolor{currentstroke}%
\pgfsetdash{}{0pt}%
\pgfpathmoveto{\pgfqpoint{4.989399in}{0.367641in}}%
\pgfpathcurveto{\pgfqpoint{5.000449in}{0.367641in}}{\pgfqpoint{5.011048in}{0.372031in}}{\pgfqpoint{5.018862in}{0.379845in}}%
\pgfpathcurveto{\pgfqpoint{5.026675in}{0.387659in}}{\pgfqpoint{5.031065in}{0.398258in}}{\pgfqpoint{5.031065in}{0.409308in}}%
\pgfpathcurveto{\pgfqpoint{5.031065in}{0.420358in}}{\pgfqpoint{5.026675in}{0.430957in}}{\pgfqpoint{5.018862in}{0.438771in}}%
\pgfpathcurveto{\pgfqpoint{5.011048in}{0.446584in}}{\pgfqpoint{5.000449in}{0.450974in}}{\pgfqpoint{4.989399in}{0.450974in}}%
\pgfpathcurveto{\pgfqpoint{4.978349in}{0.450974in}}{\pgfqpoint{4.967750in}{0.446584in}}{\pgfqpoint{4.959936in}{0.438771in}}%
\pgfpathcurveto{\pgfqpoint{4.952122in}{0.430957in}}{\pgfqpoint{4.947732in}{0.420358in}}{\pgfqpoint{4.947732in}{0.409308in}}%
\pgfpathcurveto{\pgfqpoint{4.947732in}{0.398258in}}{\pgfqpoint{4.952122in}{0.387659in}}{\pgfqpoint{4.959936in}{0.379845in}}%
\pgfpathcurveto{\pgfqpoint{4.967750in}{0.372031in}}{\pgfqpoint{4.978349in}{0.367641in}}{\pgfqpoint{4.989399in}{0.367641in}}%
\pgfusepath{stroke}%
\end{pgfscope}%
\begin{pgfscope}%
\pgfpathrectangle{\pgfqpoint{0.393053in}{0.375000in}}{\pgfqpoint{6.356833in}{5.175000in}}%
\pgfusepath{clip}%
\pgfsetbuttcap%
\pgfsetroundjoin%
\pgfsetlinewidth{1.003750pt}%
\definecolor{currentstroke}{rgb}{0.827451,0.827451,0.827451}%
\pgfsetstrokecolor{currentstroke}%
\pgfsetdash{}{0pt}%
\pgfpathmoveto{\pgfqpoint{1.375643in}{1.906128in}}%
\pgfpathcurveto{\pgfqpoint{1.386694in}{1.906128in}}{\pgfqpoint{1.397293in}{1.910518in}}{\pgfqpoint{1.405106in}{1.918331in}}%
\pgfpathcurveto{\pgfqpoint{1.412920in}{1.926145in}}{\pgfqpoint{1.417310in}{1.936744in}}{\pgfqpoint{1.417310in}{1.947794in}}%
\pgfpathcurveto{\pgfqpoint{1.417310in}{1.958844in}}{\pgfqpoint{1.412920in}{1.969443in}}{\pgfqpoint{1.405106in}{1.977257in}}%
\pgfpathcurveto{\pgfqpoint{1.397293in}{1.985071in}}{\pgfqpoint{1.386694in}{1.989461in}}{\pgfqpoint{1.375643in}{1.989461in}}%
\pgfpathcurveto{\pgfqpoint{1.364593in}{1.989461in}}{\pgfqpoint{1.353994in}{1.985071in}}{\pgfqpoint{1.346181in}{1.977257in}}%
\pgfpathcurveto{\pgfqpoint{1.338367in}{1.969443in}}{\pgfqpoint{1.333977in}{1.958844in}}{\pgfqpoint{1.333977in}{1.947794in}}%
\pgfpathcurveto{\pgfqpoint{1.333977in}{1.936744in}}{\pgfqpoint{1.338367in}{1.926145in}}{\pgfqpoint{1.346181in}{1.918331in}}%
\pgfpathcurveto{\pgfqpoint{1.353994in}{1.910518in}}{\pgfqpoint{1.364593in}{1.906128in}}{\pgfqpoint{1.375643in}{1.906128in}}%
\pgfpathlineto{\pgfqpoint{1.375643in}{1.906128in}}%
\pgfpathclose%
\pgfusepath{stroke}%
\end{pgfscope}%
\begin{pgfscope}%
\pgfpathrectangle{\pgfqpoint{0.393053in}{0.375000in}}{\pgfqpoint{6.356833in}{5.175000in}}%
\pgfusepath{clip}%
\pgfsetbuttcap%
\pgfsetroundjoin%
\pgfsetlinewidth{1.003750pt}%
\definecolor{currentstroke}{rgb}{0.827451,0.827451,0.827451}%
\pgfsetstrokecolor{currentstroke}%
\pgfsetdash{}{0pt}%
\pgfpathmoveto{\pgfqpoint{0.875375in}{2.588406in}}%
\pgfpathcurveto{\pgfqpoint{0.886425in}{2.588406in}}{\pgfqpoint{0.897024in}{2.592796in}}{\pgfqpoint{0.904838in}{2.600609in}}%
\pgfpathcurveto{\pgfqpoint{0.912652in}{2.608423in}}{\pgfqpoint{0.917042in}{2.619022in}}{\pgfqpoint{0.917042in}{2.630072in}}%
\pgfpathcurveto{\pgfqpoint{0.917042in}{2.641122in}}{\pgfqpoint{0.912652in}{2.651721in}}{\pgfqpoint{0.904838in}{2.659535in}}%
\pgfpathcurveto{\pgfqpoint{0.897024in}{2.667349in}}{\pgfqpoint{0.886425in}{2.671739in}}{\pgfqpoint{0.875375in}{2.671739in}}%
\pgfpathcurveto{\pgfqpoint{0.864325in}{2.671739in}}{\pgfqpoint{0.853726in}{2.667349in}}{\pgfqpoint{0.845912in}{2.659535in}}%
\pgfpathcurveto{\pgfqpoint{0.838099in}{2.651721in}}{\pgfqpoint{0.833708in}{2.641122in}}{\pgfqpoint{0.833708in}{2.630072in}}%
\pgfpathcurveto{\pgfqpoint{0.833708in}{2.619022in}}{\pgfqpoint{0.838099in}{2.608423in}}{\pgfqpoint{0.845912in}{2.600609in}}%
\pgfpathcurveto{\pgfqpoint{0.853726in}{2.592796in}}{\pgfqpoint{0.864325in}{2.588406in}}{\pgfqpoint{0.875375in}{2.588406in}}%
\pgfpathlineto{\pgfqpoint{0.875375in}{2.588406in}}%
\pgfpathclose%
\pgfusepath{stroke}%
\end{pgfscope}%
\begin{pgfscope}%
\pgfpathrectangle{\pgfqpoint{0.393053in}{0.375000in}}{\pgfqpoint{6.356833in}{5.175000in}}%
\pgfusepath{clip}%
\pgfsetbuttcap%
\pgfsetroundjoin%
\pgfsetlinewidth{1.003750pt}%
\definecolor{currentstroke}{rgb}{0.827451,0.827451,0.827451}%
\pgfsetstrokecolor{currentstroke}%
\pgfsetdash{}{0pt}%
\pgfpathmoveto{\pgfqpoint{0.411533in}{4.155351in}}%
\pgfpathcurveto{\pgfqpoint{0.422583in}{4.155351in}}{\pgfqpoint{0.433182in}{4.159741in}}{\pgfqpoint{0.440996in}{4.167555in}}%
\pgfpathcurveto{\pgfqpoint{0.448810in}{4.175369in}}{\pgfqpoint{0.453200in}{4.185968in}}{\pgfqpoint{0.453200in}{4.197018in}}%
\pgfpathcurveto{\pgfqpoint{0.453200in}{4.208068in}}{\pgfqpoint{0.448810in}{4.218667in}}{\pgfqpoint{0.440996in}{4.226481in}}%
\pgfpathcurveto{\pgfqpoint{0.433182in}{4.234294in}}{\pgfqpoint{0.422583in}{4.238685in}}{\pgfqpoint{0.411533in}{4.238685in}}%
\pgfpathcurveto{\pgfqpoint{0.400483in}{4.238685in}}{\pgfqpoint{0.389884in}{4.234294in}}{\pgfqpoint{0.382070in}{4.226481in}}%
\pgfpathcurveto{\pgfqpoint{0.374257in}{4.218667in}}{\pgfqpoint{0.369866in}{4.208068in}}{\pgfqpoint{0.369866in}{4.197018in}}%
\pgfpathcurveto{\pgfqpoint{0.369866in}{4.185968in}}{\pgfqpoint{0.374257in}{4.175369in}}{\pgfqpoint{0.382070in}{4.167555in}}%
\pgfpathcurveto{\pgfqpoint{0.389884in}{4.159741in}}{\pgfqpoint{0.400483in}{4.155351in}}{\pgfqpoint{0.411533in}{4.155351in}}%
\pgfpathlineto{\pgfqpoint{0.411533in}{4.155351in}}%
\pgfpathclose%
\pgfusepath{stroke}%
\end{pgfscope}%
\begin{pgfscope}%
\pgfpathrectangle{\pgfqpoint{0.393053in}{0.375000in}}{\pgfqpoint{6.356833in}{5.175000in}}%
\pgfusepath{clip}%
\pgfsetbuttcap%
\pgfsetroundjoin%
\pgfsetlinewidth{1.003750pt}%
\definecolor{currentstroke}{rgb}{0.827451,0.827451,0.827451}%
\pgfsetstrokecolor{currentstroke}%
\pgfsetdash{}{0pt}%
\pgfpathmoveto{\pgfqpoint{0.944104in}{2.490312in}}%
\pgfpathcurveto{\pgfqpoint{0.955154in}{2.490312in}}{\pgfqpoint{0.965753in}{2.494702in}}{\pgfqpoint{0.973567in}{2.502516in}}%
\pgfpathcurveto{\pgfqpoint{0.981380in}{2.510329in}}{\pgfqpoint{0.985770in}{2.520928in}}{\pgfqpoint{0.985770in}{2.531978in}}%
\pgfpathcurveto{\pgfqpoint{0.985770in}{2.543029in}}{\pgfqpoint{0.981380in}{2.553628in}}{\pgfqpoint{0.973567in}{2.561441in}}%
\pgfpathcurveto{\pgfqpoint{0.965753in}{2.569255in}}{\pgfqpoint{0.955154in}{2.573645in}}{\pgfqpoint{0.944104in}{2.573645in}}%
\pgfpathcurveto{\pgfqpoint{0.933054in}{2.573645in}}{\pgfqpoint{0.922455in}{2.569255in}}{\pgfqpoint{0.914641in}{2.561441in}}%
\pgfpathcurveto{\pgfqpoint{0.906827in}{2.553628in}}{\pgfqpoint{0.902437in}{2.543029in}}{\pgfqpoint{0.902437in}{2.531978in}}%
\pgfpathcurveto{\pgfqpoint{0.902437in}{2.520928in}}{\pgfqpoint{0.906827in}{2.510329in}}{\pgfqpoint{0.914641in}{2.502516in}}%
\pgfpathcurveto{\pgfqpoint{0.922455in}{2.494702in}}{\pgfqpoint{0.933054in}{2.490312in}}{\pgfqpoint{0.944104in}{2.490312in}}%
\pgfpathlineto{\pgfqpoint{0.944104in}{2.490312in}}%
\pgfpathclose%
\pgfusepath{stroke}%
\end{pgfscope}%
\begin{pgfscope}%
\pgfpathrectangle{\pgfqpoint{0.393053in}{0.375000in}}{\pgfqpoint{6.356833in}{5.175000in}}%
\pgfusepath{clip}%
\pgfsetbuttcap%
\pgfsetroundjoin%
\pgfsetlinewidth{1.003750pt}%
\definecolor{currentstroke}{rgb}{0.827451,0.827451,0.827451}%
\pgfsetstrokecolor{currentstroke}%
\pgfsetdash{}{0pt}%
\pgfpathmoveto{\pgfqpoint{2.555601in}{1.049270in}}%
\pgfpathcurveto{\pgfqpoint{2.566651in}{1.049270in}}{\pgfqpoint{2.577250in}{1.053660in}}{\pgfqpoint{2.585063in}{1.061474in}}%
\pgfpathcurveto{\pgfqpoint{2.592877in}{1.069288in}}{\pgfqpoint{2.597267in}{1.079887in}}{\pgfqpoint{2.597267in}{1.090937in}}%
\pgfpathcurveto{\pgfqpoint{2.597267in}{1.101987in}}{\pgfqpoint{2.592877in}{1.112586in}}{\pgfqpoint{2.585063in}{1.120400in}}%
\pgfpathcurveto{\pgfqpoint{2.577250in}{1.128213in}}{\pgfqpoint{2.566651in}{1.132603in}}{\pgfqpoint{2.555601in}{1.132603in}}%
\pgfpathcurveto{\pgfqpoint{2.544551in}{1.132603in}}{\pgfqpoint{2.533952in}{1.128213in}}{\pgfqpoint{2.526138in}{1.120400in}}%
\pgfpathcurveto{\pgfqpoint{2.518324in}{1.112586in}}{\pgfqpoint{2.513934in}{1.101987in}}{\pgfqpoint{2.513934in}{1.090937in}}%
\pgfpathcurveto{\pgfqpoint{2.513934in}{1.079887in}}{\pgfqpoint{2.518324in}{1.069288in}}{\pgfqpoint{2.526138in}{1.061474in}}%
\pgfpathcurveto{\pgfqpoint{2.533952in}{1.053660in}}{\pgfqpoint{2.544551in}{1.049270in}}{\pgfqpoint{2.555601in}{1.049270in}}%
\pgfpathlineto{\pgfqpoint{2.555601in}{1.049270in}}%
\pgfpathclose%
\pgfusepath{stroke}%
\end{pgfscope}%
\begin{pgfscope}%
\pgfpathrectangle{\pgfqpoint{0.393053in}{0.375000in}}{\pgfqpoint{6.356833in}{5.175000in}}%
\pgfusepath{clip}%
\pgfsetbuttcap%
\pgfsetroundjoin%
\pgfsetlinewidth{1.003750pt}%
\definecolor{currentstroke}{rgb}{0.827451,0.827451,0.827451}%
\pgfsetstrokecolor{currentstroke}%
\pgfsetdash{}{0pt}%
\pgfpathmoveto{\pgfqpoint{0.785854in}{2.762093in}}%
\pgfpathcurveto{\pgfqpoint{0.796904in}{2.762093in}}{\pgfqpoint{0.807503in}{2.766483in}}{\pgfqpoint{0.815317in}{2.774297in}}%
\pgfpathcurveto{\pgfqpoint{0.823130in}{2.782111in}}{\pgfqpoint{0.827521in}{2.792710in}}{\pgfqpoint{0.827521in}{2.803760in}}%
\pgfpathcurveto{\pgfqpoint{0.827521in}{2.814810in}}{\pgfqpoint{0.823130in}{2.825409in}}{\pgfqpoint{0.815317in}{2.833223in}}%
\pgfpathcurveto{\pgfqpoint{0.807503in}{2.841036in}}{\pgfqpoint{0.796904in}{2.845427in}}{\pgfqpoint{0.785854in}{2.845427in}}%
\pgfpathcurveto{\pgfqpoint{0.774804in}{2.845427in}}{\pgfqpoint{0.764205in}{2.841036in}}{\pgfqpoint{0.756391in}{2.833223in}}%
\pgfpathcurveto{\pgfqpoint{0.748577in}{2.825409in}}{\pgfqpoint{0.744187in}{2.814810in}}{\pgfqpoint{0.744187in}{2.803760in}}%
\pgfpathcurveto{\pgfqpoint{0.744187in}{2.792710in}}{\pgfqpoint{0.748577in}{2.782111in}}{\pgfqpoint{0.756391in}{2.774297in}}%
\pgfpathcurveto{\pgfqpoint{0.764205in}{2.766483in}}{\pgfqpoint{0.774804in}{2.762093in}}{\pgfqpoint{0.785854in}{2.762093in}}%
\pgfpathlineto{\pgfqpoint{0.785854in}{2.762093in}}%
\pgfpathclose%
\pgfusepath{stroke}%
\end{pgfscope}%
\begin{pgfscope}%
\pgfpathrectangle{\pgfqpoint{0.393053in}{0.375000in}}{\pgfqpoint{6.356833in}{5.175000in}}%
\pgfusepath{clip}%
\pgfsetbuttcap%
\pgfsetroundjoin%
\pgfsetlinewidth{1.003750pt}%
\definecolor{currentstroke}{rgb}{0.827451,0.827451,0.827451}%
\pgfsetstrokecolor{currentstroke}%
\pgfsetdash{}{0pt}%
\pgfpathmoveto{\pgfqpoint{0.399634in}{4.323598in}}%
\pgfpathcurveto{\pgfqpoint{0.410684in}{4.323598in}}{\pgfqpoint{0.421283in}{4.327988in}}{\pgfqpoint{0.429097in}{4.335802in}}%
\pgfpathcurveto{\pgfqpoint{0.436910in}{4.343616in}}{\pgfqpoint{0.441301in}{4.354215in}}{\pgfqpoint{0.441301in}{4.365265in}}%
\pgfpathcurveto{\pgfqpoint{0.441301in}{4.376315in}}{\pgfqpoint{0.436910in}{4.386914in}}{\pgfqpoint{0.429097in}{4.394727in}}%
\pgfpathcurveto{\pgfqpoint{0.421283in}{4.402541in}}{\pgfqpoint{0.410684in}{4.406931in}}{\pgfqpoint{0.399634in}{4.406931in}}%
\pgfpathcurveto{\pgfqpoint{0.388584in}{4.406931in}}{\pgfqpoint{0.377985in}{4.402541in}}{\pgfqpoint{0.370171in}{4.394727in}}%
\pgfpathcurveto{\pgfqpoint{0.362357in}{4.386914in}}{\pgfqpoint{0.357967in}{4.376315in}}{\pgfqpoint{0.357967in}{4.365265in}}%
\pgfpathcurveto{\pgfqpoint{0.357967in}{4.354215in}}{\pgfqpoint{0.362357in}{4.343616in}}{\pgfqpoint{0.370171in}{4.335802in}}%
\pgfpathcurveto{\pgfqpoint{0.377985in}{4.327988in}}{\pgfqpoint{0.388584in}{4.323598in}}{\pgfqpoint{0.399634in}{4.323598in}}%
\pgfpathlineto{\pgfqpoint{0.399634in}{4.323598in}}%
\pgfpathclose%
\pgfusepath{stroke}%
\end{pgfscope}%
\begin{pgfscope}%
\pgfpathrectangle{\pgfqpoint{0.393053in}{0.375000in}}{\pgfqpoint{6.356833in}{5.175000in}}%
\pgfusepath{clip}%
\pgfsetbuttcap%
\pgfsetroundjoin%
\pgfsetlinewidth{1.003750pt}%
\definecolor{currentstroke}{rgb}{0.827451,0.827451,0.827451}%
\pgfsetstrokecolor{currentstroke}%
\pgfsetdash{}{0pt}%
\pgfpathmoveto{\pgfqpoint{3.892407in}{0.513827in}}%
\pgfpathcurveto{\pgfqpoint{3.903457in}{0.513827in}}{\pgfqpoint{3.914056in}{0.518217in}}{\pgfqpoint{3.921869in}{0.526031in}}%
\pgfpathcurveto{\pgfqpoint{3.929683in}{0.533845in}}{\pgfqpoint{3.934073in}{0.544444in}}{\pgfqpoint{3.934073in}{0.555494in}}%
\pgfpathcurveto{\pgfqpoint{3.934073in}{0.566544in}}{\pgfqpoint{3.929683in}{0.577143in}}{\pgfqpoint{3.921869in}{0.584957in}}%
\pgfpathcurveto{\pgfqpoint{3.914056in}{0.592770in}}{\pgfqpoint{3.903457in}{0.597160in}}{\pgfqpoint{3.892407in}{0.597160in}}%
\pgfpathcurveto{\pgfqpoint{3.881356in}{0.597160in}}{\pgfqpoint{3.870757in}{0.592770in}}{\pgfqpoint{3.862944in}{0.584957in}}%
\pgfpathcurveto{\pgfqpoint{3.855130in}{0.577143in}}{\pgfqpoint{3.850740in}{0.566544in}}{\pgfqpoint{3.850740in}{0.555494in}}%
\pgfpathcurveto{\pgfqpoint{3.850740in}{0.544444in}}{\pgfqpoint{3.855130in}{0.533845in}}{\pgfqpoint{3.862944in}{0.526031in}}%
\pgfpathcurveto{\pgfqpoint{3.870757in}{0.518217in}}{\pgfqpoint{3.881356in}{0.513827in}}{\pgfqpoint{3.892407in}{0.513827in}}%
\pgfpathlineto{\pgfqpoint{3.892407in}{0.513827in}}%
\pgfpathclose%
\pgfusepath{stroke}%
\end{pgfscope}%
\begin{pgfscope}%
\pgfpathrectangle{\pgfqpoint{0.393053in}{0.375000in}}{\pgfqpoint{6.356833in}{5.175000in}}%
\pgfusepath{clip}%
\pgfsetbuttcap%
\pgfsetroundjoin%
\pgfsetlinewidth{1.003750pt}%
\definecolor{currentstroke}{rgb}{0.827451,0.827451,0.827451}%
\pgfsetstrokecolor{currentstroke}%
\pgfsetdash{}{0pt}%
\pgfpathmoveto{\pgfqpoint{4.512237in}{0.406694in}}%
\pgfpathcurveto{\pgfqpoint{4.523287in}{0.406694in}}{\pgfqpoint{4.533886in}{0.411084in}}{\pgfqpoint{4.541700in}{0.418898in}}%
\pgfpathcurveto{\pgfqpoint{4.549514in}{0.426711in}}{\pgfqpoint{4.553904in}{0.437310in}}{\pgfqpoint{4.553904in}{0.448360in}}%
\pgfpathcurveto{\pgfqpoint{4.553904in}{0.459410in}}{\pgfqpoint{4.549514in}{0.470010in}}{\pgfqpoint{4.541700in}{0.477823in}}%
\pgfpathcurveto{\pgfqpoint{4.533886in}{0.485637in}}{\pgfqpoint{4.523287in}{0.490027in}}{\pgfqpoint{4.512237in}{0.490027in}}%
\pgfpathcurveto{\pgfqpoint{4.501187in}{0.490027in}}{\pgfqpoint{4.490588in}{0.485637in}}{\pgfqpoint{4.482774in}{0.477823in}}%
\pgfpathcurveto{\pgfqpoint{4.474961in}{0.470010in}}{\pgfqpoint{4.470570in}{0.459410in}}{\pgfqpoint{4.470570in}{0.448360in}}%
\pgfpathcurveto{\pgfqpoint{4.470570in}{0.437310in}}{\pgfqpoint{4.474961in}{0.426711in}}{\pgfqpoint{4.482774in}{0.418898in}}%
\pgfpathcurveto{\pgfqpoint{4.490588in}{0.411084in}}{\pgfqpoint{4.501187in}{0.406694in}}{\pgfqpoint{4.512237in}{0.406694in}}%
\pgfpathlineto{\pgfqpoint{4.512237in}{0.406694in}}%
\pgfpathclose%
\pgfusepath{stroke}%
\end{pgfscope}%
\begin{pgfscope}%
\pgfpathrectangle{\pgfqpoint{0.393053in}{0.375000in}}{\pgfqpoint{6.356833in}{5.175000in}}%
\pgfusepath{clip}%
\pgfsetbuttcap%
\pgfsetroundjoin%
\pgfsetlinewidth{1.003750pt}%
\definecolor{currentstroke}{rgb}{0.827451,0.827451,0.827451}%
\pgfsetstrokecolor{currentstroke}%
\pgfsetdash{}{0pt}%
\pgfpathmoveto{\pgfqpoint{2.675195in}{0.994248in}}%
\pgfpathcurveto{\pgfqpoint{2.686245in}{0.994248in}}{\pgfqpoint{2.696845in}{0.998639in}}{\pgfqpoint{2.704658in}{1.006452in}}%
\pgfpathcurveto{\pgfqpoint{2.712472in}{1.014266in}}{\pgfqpoint{2.716862in}{1.024865in}}{\pgfqpoint{2.716862in}{1.035915in}}%
\pgfpathcurveto{\pgfqpoint{2.716862in}{1.046965in}}{\pgfqpoint{2.712472in}{1.057564in}}{\pgfqpoint{2.704658in}{1.065378in}}%
\pgfpathcurveto{\pgfqpoint{2.696845in}{1.073191in}}{\pgfqpoint{2.686245in}{1.077582in}}{\pgfqpoint{2.675195in}{1.077582in}}%
\pgfpathcurveto{\pgfqpoint{2.664145in}{1.077582in}}{\pgfqpoint{2.653546in}{1.073191in}}{\pgfqpoint{2.645733in}{1.065378in}}%
\pgfpathcurveto{\pgfqpoint{2.637919in}{1.057564in}}{\pgfqpoint{2.633529in}{1.046965in}}{\pgfqpoint{2.633529in}{1.035915in}}%
\pgfpathcurveto{\pgfqpoint{2.633529in}{1.024865in}}{\pgfqpoint{2.637919in}{1.014266in}}{\pgfqpoint{2.645733in}{1.006452in}}%
\pgfpathcurveto{\pgfqpoint{2.653546in}{0.998639in}}{\pgfqpoint{2.664145in}{0.994248in}}{\pgfqpoint{2.675195in}{0.994248in}}%
\pgfpathlineto{\pgfqpoint{2.675195in}{0.994248in}}%
\pgfpathclose%
\pgfusepath{stroke}%
\end{pgfscope}%
\begin{pgfscope}%
\pgfpathrectangle{\pgfqpoint{0.393053in}{0.375000in}}{\pgfqpoint{6.356833in}{5.175000in}}%
\pgfusepath{clip}%
\pgfsetbuttcap%
\pgfsetroundjoin%
\pgfsetlinewidth{1.003750pt}%
\definecolor{currentstroke}{rgb}{0.827451,0.827451,0.827451}%
\pgfsetstrokecolor{currentstroke}%
\pgfsetdash{}{0pt}%
\pgfpathmoveto{\pgfqpoint{1.105942in}{2.248010in}}%
\pgfpathcurveto{\pgfqpoint{1.116993in}{2.248010in}}{\pgfqpoint{1.127592in}{2.252400in}}{\pgfqpoint{1.135405in}{2.260214in}}%
\pgfpathcurveto{\pgfqpoint{1.143219in}{2.268028in}}{\pgfqpoint{1.147609in}{2.278627in}}{\pgfqpoint{1.147609in}{2.289677in}}%
\pgfpathcurveto{\pgfqpoint{1.147609in}{2.300727in}}{\pgfqpoint{1.143219in}{2.311326in}}{\pgfqpoint{1.135405in}{2.319140in}}%
\pgfpathcurveto{\pgfqpoint{1.127592in}{2.326953in}}{\pgfqpoint{1.116993in}{2.331343in}}{\pgfqpoint{1.105942in}{2.331343in}}%
\pgfpathcurveto{\pgfqpoint{1.094892in}{2.331343in}}{\pgfqpoint{1.084293in}{2.326953in}}{\pgfqpoint{1.076480in}{2.319140in}}%
\pgfpathcurveto{\pgfqpoint{1.068666in}{2.311326in}}{\pgfqpoint{1.064276in}{2.300727in}}{\pgfqpoint{1.064276in}{2.289677in}}%
\pgfpathcurveto{\pgfqpoint{1.064276in}{2.278627in}}{\pgfqpoint{1.068666in}{2.268028in}}{\pgfqpoint{1.076480in}{2.260214in}}%
\pgfpathcurveto{\pgfqpoint{1.084293in}{2.252400in}}{\pgfqpoint{1.094892in}{2.248010in}}{\pgfqpoint{1.105942in}{2.248010in}}%
\pgfpathlineto{\pgfqpoint{1.105942in}{2.248010in}}%
\pgfpathclose%
\pgfusepath{stroke}%
\end{pgfscope}%
\begin{pgfscope}%
\pgfpathrectangle{\pgfqpoint{0.393053in}{0.375000in}}{\pgfqpoint{6.356833in}{5.175000in}}%
\pgfusepath{clip}%
\pgfsetbuttcap%
\pgfsetroundjoin%
\pgfsetlinewidth{1.003750pt}%
\definecolor{currentstroke}{rgb}{0.827451,0.827451,0.827451}%
\pgfsetstrokecolor{currentstroke}%
\pgfsetdash{}{0pt}%
\pgfpathmoveto{\pgfqpoint{2.713290in}{0.943005in}}%
\pgfpathcurveto{\pgfqpoint{2.724341in}{0.943005in}}{\pgfqpoint{2.734940in}{0.947396in}}{\pgfqpoint{2.742753in}{0.955209in}}%
\pgfpathcurveto{\pgfqpoint{2.750567in}{0.963023in}}{\pgfqpoint{2.754957in}{0.973622in}}{\pgfqpoint{2.754957in}{0.984672in}}%
\pgfpathcurveto{\pgfqpoint{2.754957in}{0.995722in}}{\pgfqpoint{2.750567in}{1.006321in}}{\pgfqpoint{2.742753in}{1.014135in}}%
\pgfpathcurveto{\pgfqpoint{2.734940in}{1.021949in}}{\pgfqpoint{2.724341in}{1.026339in}}{\pgfqpoint{2.713290in}{1.026339in}}%
\pgfpathcurveto{\pgfqpoint{2.702240in}{1.026339in}}{\pgfqpoint{2.691641in}{1.021949in}}{\pgfqpoint{2.683828in}{1.014135in}}%
\pgfpathcurveto{\pgfqpoint{2.676014in}{1.006321in}}{\pgfqpoint{2.671624in}{0.995722in}}{\pgfqpoint{2.671624in}{0.984672in}}%
\pgfpathcurveto{\pgfqpoint{2.671624in}{0.973622in}}{\pgfqpoint{2.676014in}{0.963023in}}{\pgfqpoint{2.683828in}{0.955209in}}%
\pgfpathcurveto{\pgfqpoint{2.691641in}{0.947396in}}{\pgfqpoint{2.702240in}{0.943005in}}{\pgfqpoint{2.713290in}{0.943005in}}%
\pgfpathlineto{\pgfqpoint{2.713290in}{0.943005in}}%
\pgfpathclose%
\pgfusepath{stroke}%
\end{pgfscope}%
\begin{pgfscope}%
\pgfpathrectangle{\pgfqpoint{0.393053in}{0.375000in}}{\pgfqpoint{6.356833in}{5.175000in}}%
\pgfusepath{clip}%
\pgfsetbuttcap%
\pgfsetroundjoin%
\pgfsetlinewidth{1.003750pt}%
\definecolor{currentstroke}{rgb}{0.827451,0.827451,0.827451}%
\pgfsetstrokecolor{currentstroke}%
\pgfsetdash{}{0pt}%
\pgfpathmoveto{\pgfqpoint{0.452170in}{3.855614in}}%
\pgfpathcurveto{\pgfqpoint{0.463220in}{3.855614in}}{\pgfqpoint{0.473819in}{3.860005in}}{\pgfqpoint{0.481633in}{3.867818in}}%
\pgfpathcurveto{\pgfqpoint{0.489447in}{3.875632in}}{\pgfqpoint{0.493837in}{3.886231in}}{\pgfqpoint{0.493837in}{3.897281in}}%
\pgfpathcurveto{\pgfqpoint{0.493837in}{3.908331in}}{\pgfqpoint{0.489447in}{3.918930in}}{\pgfqpoint{0.481633in}{3.926744in}}%
\pgfpathcurveto{\pgfqpoint{0.473819in}{3.934557in}}{\pgfqpoint{0.463220in}{3.938948in}}{\pgfqpoint{0.452170in}{3.938948in}}%
\pgfpathcurveto{\pgfqpoint{0.441120in}{3.938948in}}{\pgfqpoint{0.430521in}{3.934557in}}{\pgfqpoint{0.422707in}{3.926744in}}%
\pgfpathcurveto{\pgfqpoint{0.414894in}{3.918930in}}{\pgfqpoint{0.410503in}{3.908331in}}{\pgfqpoint{0.410503in}{3.897281in}}%
\pgfpathcurveto{\pgfqpoint{0.410503in}{3.886231in}}{\pgfqpoint{0.414894in}{3.875632in}}{\pgfqpoint{0.422707in}{3.867818in}}%
\pgfpathcurveto{\pgfqpoint{0.430521in}{3.860005in}}{\pgfqpoint{0.441120in}{3.855614in}}{\pgfqpoint{0.452170in}{3.855614in}}%
\pgfpathlineto{\pgfqpoint{0.452170in}{3.855614in}}%
\pgfpathclose%
\pgfusepath{stroke}%
\end{pgfscope}%
\begin{pgfscope}%
\pgfpathrectangle{\pgfqpoint{0.393053in}{0.375000in}}{\pgfqpoint{6.356833in}{5.175000in}}%
\pgfusepath{clip}%
\pgfsetbuttcap%
\pgfsetroundjoin%
\pgfsetlinewidth{1.003750pt}%
\definecolor{currentstroke}{rgb}{0.827451,0.827451,0.827451}%
\pgfsetstrokecolor{currentstroke}%
\pgfsetdash{}{0pt}%
\pgfpathmoveto{\pgfqpoint{2.876504in}{0.862226in}}%
\pgfpathcurveto{\pgfqpoint{2.887555in}{0.862226in}}{\pgfqpoint{2.898154in}{0.866616in}}{\pgfqpoint{2.905967in}{0.874430in}}%
\pgfpathcurveto{\pgfqpoint{2.913781in}{0.882244in}}{\pgfqpoint{2.918171in}{0.892843in}}{\pgfqpoint{2.918171in}{0.903893in}}%
\pgfpathcurveto{\pgfqpoint{2.918171in}{0.914943in}}{\pgfqpoint{2.913781in}{0.925542in}}{\pgfqpoint{2.905967in}{0.933355in}}%
\pgfpathcurveto{\pgfqpoint{2.898154in}{0.941169in}}{\pgfqpoint{2.887555in}{0.945559in}}{\pgfqpoint{2.876504in}{0.945559in}}%
\pgfpathcurveto{\pgfqpoint{2.865454in}{0.945559in}}{\pgfqpoint{2.854855in}{0.941169in}}{\pgfqpoint{2.847042in}{0.933355in}}%
\pgfpathcurveto{\pgfqpoint{2.839228in}{0.925542in}}{\pgfqpoint{2.834838in}{0.914943in}}{\pgfqpoint{2.834838in}{0.903893in}}%
\pgfpathcurveto{\pgfqpoint{2.834838in}{0.892843in}}{\pgfqpoint{2.839228in}{0.882244in}}{\pgfqpoint{2.847042in}{0.874430in}}%
\pgfpathcurveto{\pgfqpoint{2.854855in}{0.866616in}}{\pgfqpoint{2.865454in}{0.862226in}}{\pgfqpoint{2.876504in}{0.862226in}}%
\pgfpathlineto{\pgfqpoint{2.876504in}{0.862226in}}%
\pgfpathclose%
\pgfusepath{stroke}%
\end{pgfscope}%
\begin{pgfscope}%
\pgfpathrectangle{\pgfqpoint{0.393053in}{0.375000in}}{\pgfqpoint{6.356833in}{5.175000in}}%
\pgfusepath{clip}%
\pgfsetbuttcap%
\pgfsetroundjoin%
\pgfsetlinewidth{1.003750pt}%
\definecolor{currentstroke}{rgb}{0.827451,0.827451,0.827451}%
\pgfsetstrokecolor{currentstroke}%
\pgfsetdash{}{0pt}%
\pgfpathmoveto{\pgfqpoint{3.230235in}{0.725803in}}%
\pgfpathcurveto{\pgfqpoint{3.241285in}{0.725803in}}{\pgfqpoint{3.251884in}{0.730193in}}{\pgfqpoint{3.259697in}{0.738007in}}%
\pgfpathcurveto{\pgfqpoint{3.267511in}{0.745821in}}{\pgfqpoint{3.271901in}{0.756420in}}{\pgfqpoint{3.271901in}{0.767470in}}%
\pgfpathcurveto{\pgfqpoint{3.271901in}{0.778520in}}{\pgfqpoint{3.267511in}{0.789119in}}{\pgfqpoint{3.259697in}{0.796933in}}%
\pgfpathcurveto{\pgfqpoint{3.251884in}{0.804746in}}{\pgfqpoint{3.241285in}{0.809137in}}{\pgfqpoint{3.230235in}{0.809137in}}%
\pgfpathcurveto{\pgfqpoint{3.219184in}{0.809137in}}{\pgfqpoint{3.208585in}{0.804746in}}{\pgfqpoint{3.200772in}{0.796933in}}%
\pgfpathcurveto{\pgfqpoint{3.192958in}{0.789119in}}{\pgfqpoint{3.188568in}{0.778520in}}{\pgfqpoint{3.188568in}{0.767470in}}%
\pgfpathcurveto{\pgfqpoint{3.188568in}{0.756420in}}{\pgfqpoint{3.192958in}{0.745821in}}{\pgfqpoint{3.200772in}{0.738007in}}%
\pgfpathcurveto{\pgfqpoint{3.208585in}{0.730193in}}{\pgfqpoint{3.219184in}{0.725803in}}{\pgfqpoint{3.230235in}{0.725803in}}%
\pgfpathlineto{\pgfqpoint{3.230235in}{0.725803in}}%
\pgfpathclose%
\pgfusepath{stroke}%
\end{pgfscope}%
\begin{pgfscope}%
\pgfpathrectangle{\pgfqpoint{0.393053in}{0.375000in}}{\pgfqpoint{6.356833in}{5.175000in}}%
\pgfusepath{clip}%
\pgfsetbuttcap%
\pgfsetroundjoin%
\pgfsetlinewidth{1.003750pt}%
\definecolor{currentstroke}{rgb}{0.827451,0.827451,0.827451}%
\pgfsetstrokecolor{currentstroke}%
\pgfsetdash{}{0pt}%
\pgfpathmoveto{\pgfqpoint{1.585081in}{1.702736in}}%
\pgfpathcurveto{\pgfqpoint{1.596132in}{1.702736in}}{\pgfqpoint{1.606731in}{1.707127in}}{\pgfqpoint{1.614544in}{1.714940in}}%
\pgfpathcurveto{\pgfqpoint{1.622358in}{1.722754in}}{\pgfqpoint{1.626748in}{1.733353in}}{\pgfqpoint{1.626748in}{1.744403in}}%
\pgfpathcurveto{\pgfqpoint{1.626748in}{1.755453in}}{\pgfqpoint{1.622358in}{1.766052in}}{\pgfqpoint{1.614544in}{1.773866in}}%
\pgfpathcurveto{\pgfqpoint{1.606731in}{1.781679in}}{\pgfqpoint{1.596132in}{1.786070in}}{\pgfqpoint{1.585081in}{1.786070in}}%
\pgfpathcurveto{\pgfqpoint{1.574031in}{1.786070in}}{\pgfqpoint{1.563432in}{1.781679in}}{\pgfqpoint{1.555619in}{1.773866in}}%
\pgfpathcurveto{\pgfqpoint{1.547805in}{1.766052in}}{\pgfqpoint{1.543415in}{1.755453in}}{\pgfqpoint{1.543415in}{1.744403in}}%
\pgfpathcurveto{\pgfqpoint{1.543415in}{1.733353in}}{\pgfqpoint{1.547805in}{1.722754in}}{\pgfqpoint{1.555619in}{1.714940in}}%
\pgfpathcurveto{\pgfqpoint{1.563432in}{1.707127in}}{\pgfqpoint{1.574031in}{1.702736in}}{\pgfqpoint{1.585081in}{1.702736in}}%
\pgfpathlineto{\pgfqpoint{1.585081in}{1.702736in}}%
\pgfpathclose%
\pgfusepath{stroke}%
\end{pgfscope}%
\begin{pgfscope}%
\pgfpathrectangle{\pgfqpoint{0.393053in}{0.375000in}}{\pgfqpoint{6.356833in}{5.175000in}}%
\pgfusepath{clip}%
\pgfsetbuttcap%
\pgfsetroundjoin%
\pgfsetlinewidth{1.003750pt}%
\definecolor{currentstroke}{rgb}{0.827451,0.827451,0.827451}%
\pgfsetstrokecolor{currentstroke}%
\pgfsetdash{}{0pt}%
\pgfpathmoveto{\pgfqpoint{0.737172in}{2.862852in}}%
\pgfpathcurveto{\pgfqpoint{0.748222in}{2.862852in}}{\pgfqpoint{0.758821in}{2.867242in}}{\pgfqpoint{0.766635in}{2.875055in}}%
\pgfpathcurveto{\pgfqpoint{0.774449in}{2.882869in}}{\pgfqpoint{0.778839in}{2.893468in}}{\pgfqpoint{0.778839in}{2.904518in}}%
\pgfpathcurveto{\pgfqpoint{0.778839in}{2.915568in}}{\pgfqpoint{0.774449in}{2.926167in}}{\pgfqpoint{0.766635in}{2.933981in}}%
\pgfpathcurveto{\pgfqpoint{0.758821in}{2.941795in}}{\pgfqpoint{0.748222in}{2.946185in}}{\pgfqpoint{0.737172in}{2.946185in}}%
\pgfpathcurveto{\pgfqpoint{0.726122in}{2.946185in}}{\pgfqpoint{0.715523in}{2.941795in}}{\pgfqpoint{0.707710in}{2.933981in}}%
\pgfpathcurveto{\pgfqpoint{0.699896in}{2.926167in}}{\pgfqpoint{0.695506in}{2.915568in}}{\pgfqpoint{0.695506in}{2.904518in}}%
\pgfpathcurveto{\pgfqpoint{0.695506in}{2.893468in}}{\pgfqpoint{0.699896in}{2.882869in}}{\pgfqpoint{0.707710in}{2.875055in}}%
\pgfpathcurveto{\pgfqpoint{0.715523in}{2.867242in}}{\pgfqpoint{0.726122in}{2.862852in}}{\pgfqpoint{0.737172in}{2.862852in}}%
\pgfpathlineto{\pgfqpoint{0.737172in}{2.862852in}}%
\pgfpathclose%
\pgfusepath{stroke}%
\end{pgfscope}%
\begin{pgfscope}%
\pgfpathrectangle{\pgfqpoint{0.393053in}{0.375000in}}{\pgfqpoint{6.356833in}{5.175000in}}%
\pgfusepath{clip}%
\pgfsetbuttcap%
\pgfsetroundjoin%
\pgfsetlinewidth{1.003750pt}%
\definecolor{currentstroke}{rgb}{0.827451,0.827451,0.827451}%
\pgfsetstrokecolor{currentstroke}%
\pgfsetdash{}{0pt}%
\pgfpathmoveto{\pgfqpoint{3.693960in}{0.561287in}}%
\pgfpathcurveto{\pgfqpoint{3.705010in}{0.561287in}}{\pgfqpoint{3.715609in}{0.565678in}}{\pgfqpoint{3.723423in}{0.573491in}}%
\pgfpathcurveto{\pgfqpoint{3.731237in}{0.581305in}}{\pgfqpoint{3.735627in}{0.591904in}}{\pgfqpoint{3.735627in}{0.602954in}}%
\pgfpathcurveto{\pgfqpoint{3.735627in}{0.614004in}}{\pgfqpoint{3.731237in}{0.624603in}}{\pgfqpoint{3.723423in}{0.632417in}}%
\pgfpathcurveto{\pgfqpoint{3.715609in}{0.640230in}}{\pgfqpoint{3.705010in}{0.644621in}}{\pgfqpoint{3.693960in}{0.644621in}}%
\pgfpathcurveto{\pgfqpoint{3.682910in}{0.644621in}}{\pgfqpoint{3.672311in}{0.640230in}}{\pgfqpoint{3.664498in}{0.632417in}}%
\pgfpathcurveto{\pgfqpoint{3.656684in}{0.624603in}}{\pgfqpoint{3.652294in}{0.614004in}}{\pgfqpoint{3.652294in}{0.602954in}}%
\pgfpathcurveto{\pgfqpoint{3.652294in}{0.591904in}}{\pgfqpoint{3.656684in}{0.581305in}}{\pgfqpoint{3.664498in}{0.573491in}}%
\pgfpathcurveto{\pgfqpoint{3.672311in}{0.565678in}}{\pgfqpoint{3.682910in}{0.561287in}}{\pgfqpoint{3.693960in}{0.561287in}}%
\pgfpathlineto{\pgfqpoint{3.693960in}{0.561287in}}%
\pgfpathclose%
\pgfusepath{stroke}%
\end{pgfscope}%
\begin{pgfscope}%
\pgfpathrectangle{\pgfqpoint{0.393053in}{0.375000in}}{\pgfqpoint{6.356833in}{5.175000in}}%
\pgfusepath{clip}%
\pgfsetbuttcap%
\pgfsetroundjoin%
\pgfsetlinewidth{1.003750pt}%
\definecolor{currentstroke}{rgb}{0.827451,0.827451,0.827451}%
\pgfsetstrokecolor{currentstroke}%
\pgfsetdash{}{0pt}%
\pgfpathmoveto{\pgfqpoint{5.508231in}{0.339636in}}%
\pgfpathcurveto{\pgfqpoint{5.519281in}{0.339636in}}{\pgfqpoint{5.529880in}{0.344026in}}{\pgfqpoint{5.537694in}{0.351840in}}%
\pgfpathcurveto{\pgfqpoint{5.545508in}{0.359653in}}{\pgfqpoint{5.549898in}{0.370252in}}{\pgfqpoint{5.549898in}{0.381302in}}%
\pgfpathcurveto{\pgfqpoint{5.549898in}{0.392352in}}{\pgfqpoint{5.545508in}{0.402951in}}{\pgfqpoint{5.537694in}{0.410765in}}%
\pgfpathcurveto{\pgfqpoint{5.529880in}{0.418579in}}{\pgfqpoint{5.519281in}{0.422969in}}{\pgfqpoint{5.508231in}{0.422969in}}%
\pgfpathcurveto{\pgfqpoint{5.497181in}{0.422969in}}{\pgfqpoint{5.486582in}{0.418579in}}{\pgfqpoint{5.478769in}{0.410765in}}%
\pgfpathcurveto{\pgfqpoint{5.470955in}{0.402951in}}{\pgfqpoint{5.466565in}{0.392352in}}{\pgfqpoint{5.466565in}{0.381302in}}%
\pgfpathcurveto{\pgfqpoint{5.466565in}{0.370252in}}{\pgfqpoint{5.470955in}{0.359653in}}{\pgfqpoint{5.478769in}{0.351840in}}%
\pgfpathcurveto{\pgfqpoint{5.486582in}{0.344026in}}{\pgfqpoint{5.497181in}{0.339636in}}{\pgfqpoint{5.508231in}{0.339636in}}%
\pgfusepath{stroke}%
\end{pgfscope}%
\begin{pgfscope}%
\pgfpathrectangle{\pgfqpoint{0.393053in}{0.375000in}}{\pgfqpoint{6.356833in}{5.175000in}}%
\pgfusepath{clip}%
\pgfsetbuttcap%
\pgfsetroundjoin%
\pgfsetlinewidth{1.003750pt}%
\definecolor{currentstroke}{rgb}{0.827451,0.827451,0.827451}%
\pgfsetstrokecolor{currentstroke}%
\pgfsetdash{}{0pt}%
\pgfpathmoveto{\pgfqpoint{0.899669in}{2.545459in}}%
\pgfpathcurveto{\pgfqpoint{0.910719in}{2.545459in}}{\pgfqpoint{0.921318in}{2.549849in}}{\pgfqpoint{0.929132in}{2.557663in}}%
\pgfpathcurveto{\pgfqpoint{0.936945in}{2.565477in}}{\pgfqpoint{0.941336in}{2.576076in}}{\pgfqpoint{0.941336in}{2.587126in}}%
\pgfpathcurveto{\pgfqpoint{0.941336in}{2.598176in}}{\pgfqpoint{0.936945in}{2.608775in}}{\pgfqpoint{0.929132in}{2.616589in}}%
\pgfpathcurveto{\pgfqpoint{0.921318in}{2.624402in}}{\pgfqpoint{0.910719in}{2.628792in}}{\pgfqpoint{0.899669in}{2.628792in}}%
\pgfpathcurveto{\pgfqpoint{0.888619in}{2.628792in}}{\pgfqpoint{0.878020in}{2.624402in}}{\pgfqpoint{0.870206in}{2.616589in}}%
\pgfpathcurveto{\pgfqpoint{0.862393in}{2.608775in}}{\pgfqpoint{0.858002in}{2.598176in}}{\pgfqpoint{0.858002in}{2.587126in}}%
\pgfpathcurveto{\pgfqpoint{0.858002in}{2.576076in}}{\pgfqpoint{0.862393in}{2.565477in}}{\pgfqpoint{0.870206in}{2.557663in}}%
\pgfpathcurveto{\pgfqpoint{0.878020in}{2.549849in}}{\pgfqpoint{0.888619in}{2.545459in}}{\pgfqpoint{0.899669in}{2.545459in}}%
\pgfpathlineto{\pgfqpoint{0.899669in}{2.545459in}}%
\pgfpathclose%
\pgfusepath{stroke}%
\end{pgfscope}%
\begin{pgfscope}%
\pgfpathrectangle{\pgfqpoint{0.393053in}{0.375000in}}{\pgfqpoint{6.356833in}{5.175000in}}%
\pgfusepath{clip}%
\pgfsetbuttcap%
\pgfsetroundjoin%
\pgfsetlinewidth{1.003750pt}%
\definecolor{currentstroke}{rgb}{0.827451,0.827451,0.827451}%
\pgfsetstrokecolor{currentstroke}%
\pgfsetdash{}{0pt}%
\pgfpathmoveto{\pgfqpoint{1.291595in}{1.999205in}}%
\pgfpathcurveto{\pgfqpoint{1.302646in}{1.999205in}}{\pgfqpoint{1.313245in}{2.003595in}}{\pgfqpoint{1.321058in}{2.011409in}}%
\pgfpathcurveto{\pgfqpoint{1.328872in}{2.019222in}}{\pgfqpoint{1.333262in}{2.029821in}}{\pgfqpoint{1.333262in}{2.040872in}}%
\pgfpathcurveto{\pgfqpoint{1.333262in}{2.051922in}}{\pgfqpoint{1.328872in}{2.062521in}}{\pgfqpoint{1.321058in}{2.070334in}}%
\pgfpathcurveto{\pgfqpoint{1.313245in}{2.078148in}}{\pgfqpoint{1.302646in}{2.082538in}}{\pgfqpoint{1.291595in}{2.082538in}}%
\pgfpathcurveto{\pgfqpoint{1.280545in}{2.082538in}}{\pgfqpoint{1.269946in}{2.078148in}}{\pgfqpoint{1.262133in}{2.070334in}}%
\pgfpathcurveto{\pgfqpoint{1.254319in}{2.062521in}}{\pgfqpoint{1.249929in}{2.051922in}}{\pgfqpoint{1.249929in}{2.040872in}}%
\pgfpathcurveto{\pgfqpoint{1.249929in}{2.029821in}}{\pgfqpoint{1.254319in}{2.019222in}}{\pgfqpoint{1.262133in}{2.011409in}}%
\pgfpathcurveto{\pgfqpoint{1.269946in}{2.003595in}}{\pgfqpoint{1.280545in}{1.999205in}}{\pgfqpoint{1.291595in}{1.999205in}}%
\pgfpathlineto{\pgfqpoint{1.291595in}{1.999205in}}%
\pgfpathclose%
\pgfusepath{stroke}%
\end{pgfscope}%
\begin{pgfscope}%
\pgfpathrectangle{\pgfqpoint{0.393053in}{0.375000in}}{\pgfqpoint{6.356833in}{5.175000in}}%
\pgfusepath{clip}%
\pgfsetbuttcap%
\pgfsetroundjoin%
\pgfsetlinewidth{1.003750pt}%
\definecolor{currentstroke}{rgb}{0.827451,0.827451,0.827451}%
\pgfsetstrokecolor{currentstroke}%
\pgfsetdash{}{0pt}%
\pgfpathmoveto{\pgfqpoint{1.068566in}{2.310306in}}%
\pgfpathcurveto{\pgfqpoint{1.079616in}{2.310306in}}{\pgfqpoint{1.090215in}{2.314696in}}{\pgfqpoint{1.098029in}{2.322510in}}%
\pgfpathcurveto{\pgfqpoint{1.105843in}{2.330323in}}{\pgfqpoint{1.110233in}{2.340922in}}{\pgfqpoint{1.110233in}{2.351973in}}%
\pgfpathcurveto{\pgfqpoint{1.110233in}{2.363023in}}{\pgfqpoint{1.105843in}{2.373622in}}{\pgfqpoint{1.098029in}{2.381435in}}%
\pgfpathcurveto{\pgfqpoint{1.090215in}{2.389249in}}{\pgfqpoint{1.079616in}{2.393639in}}{\pgfqpoint{1.068566in}{2.393639in}}%
\pgfpathcurveto{\pgfqpoint{1.057516in}{2.393639in}}{\pgfqpoint{1.046917in}{2.389249in}}{\pgfqpoint{1.039103in}{2.381435in}}%
\pgfpathcurveto{\pgfqpoint{1.031290in}{2.373622in}}{\pgfqpoint{1.026900in}{2.363023in}}{\pgfqpoint{1.026900in}{2.351973in}}%
\pgfpathcurveto{\pgfqpoint{1.026900in}{2.340922in}}{\pgfqpoint{1.031290in}{2.330323in}}{\pgfqpoint{1.039103in}{2.322510in}}%
\pgfpathcurveto{\pgfqpoint{1.046917in}{2.314696in}}{\pgfqpoint{1.057516in}{2.310306in}}{\pgfqpoint{1.068566in}{2.310306in}}%
\pgfpathlineto{\pgfqpoint{1.068566in}{2.310306in}}%
\pgfpathclose%
\pgfusepath{stroke}%
\end{pgfscope}%
\begin{pgfscope}%
\pgfpathrectangle{\pgfqpoint{0.393053in}{0.375000in}}{\pgfqpoint{6.356833in}{5.175000in}}%
\pgfusepath{clip}%
\pgfsetbuttcap%
\pgfsetroundjoin%
\pgfsetlinewidth{1.003750pt}%
\definecolor{currentstroke}{rgb}{0.827451,0.827451,0.827451}%
\pgfsetstrokecolor{currentstroke}%
\pgfsetdash{}{0pt}%
\pgfpathmoveto{\pgfqpoint{2.096051in}{1.300749in}}%
\pgfpathcurveto{\pgfqpoint{2.107101in}{1.300749in}}{\pgfqpoint{2.117700in}{1.305139in}}{\pgfqpoint{2.125513in}{1.312953in}}%
\pgfpathcurveto{\pgfqpoint{2.133327in}{1.320766in}}{\pgfqpoint{2.137717in}{1.331365in}}{\pgfqpoint{2.137717in}{1.342415in}}%
\pgfpathcurveto{\pgfqpoint{2.137717in}{1.353465in}}{\pgfqpoint{2.133327in}{1.364065in}}{\pgfqpoint{2.125513in}{1.371878in}}%
\pgfpathcurveto{\pgfqpoint{2.117700in}{1.379692in}}{\pgfqpoint{2.107101in}{1.384082in}}{\pgfqpoint{2.096051in}{1.384082in}}%
\pgfpathcurveto{\pgfqpoint{2.085000in}{1.384082in}}{\pgfqpoint{2.074401in}{1.379692in}}{\pgfqpoint{2.066588in}{1.371878in}}%
\pgfpathcurveto{\pgfqpoint{2.058774in}{1.364065in}}{\pgfqpoint{2.054384in}{1.353465in}}{\pgfqpoint{2.054384in}{1.342415in}}%
\pgfpathcurveto{\pgfqpoint{2.054384in}{1.331365in}}{\pgfqpoint{2.058774in}{1.320766in}}{\pgfqpoint{2.066588in}{1.312953in}}%
\pgfpathcurveto{\pgfqpoint{2.074401in}{1.305139in}}{\pgfqpoint{2.085000in}{1.300749in}}{\pgfqpoint{2.096051in}{1.300749in}}%
\pgfpathlineto{\pgfqpoint{2.096051in}{1.300749in}}%
\pgfpathclose%
\pgfusepath{stroke}%
\end{pgfscope}%
\begin{pgfscope}%
\pgfpathrectangle{\pgfqpoint{0.393053in}{0.375000in}}{\pgfqpoint{6.356833in}{5.175000in}}%
\pgfusepath{clip}%
\pgfsetbuttcap%
\pgfsetroundjoin%
\pgfsetlinewidth{1.003750pt}%
\definecolor{currentstroke}{rgb}{0.827451,0.827451,0.827451}%
\pgfsetstrokecolor{currentstroke}%
\pgfsetdash{}{0pt}%
\pgfpathmoveto{\pgfqpoint{0.483847in}{3.664437in}}%
\pgfpathcurveto{\pgfqpoint{0.494897in}{3.664437in}}{\pgfqpoint{0.505496in}{3.668827in}}{\pgfqpoint{0.513309in}{3.676640in}}%
\pgfpathcurveto{\pgfqpoint{0.521123in}{3.684454in}}{\pgfqpoint{0.525513in}{3.695053in}}{\pgfqpoint{0.525513in}{3.706103in}}%
\pgfpathcurveto{\pgfqpoint{0.525513in}{3.717153in}}{\pgfqpoint{0.521123in}{3.727752in}}{\pgfqpoint{0.513309in}{3.735566in}}%
\pgfpathcurveto{\pgfqpoint{0.505496in}{3.743380in}}{\pgfqpoint{0.494897in}{3.747770in}}{\pgfqpoint{0.483847in}{3.747770in}}%
\pgfpathcurveto{\pgfqpoint{0.472797in}{3.747770in}}{\pgfqpoint{0.462198in}{3.743380in}}{\pgfqpoint{0.454384in}{3.735566in}}%
\pgfpathcurveto{\pgfqpoint{0.446570in}{3.727752in}}{\pgfqpoint{0.442180in}{3.717153in}}{\pgfqpoint{0.442180in}{3.706103in}}%
\pgfpathcurveto{\pgfqpoint{0.442180in}{3.695053in}}{\pgfqpoint{0.446570in}{3.684454in}}{\pgfqpoint{0.454384in}{3.676640in}}%
\pgfpathcurveto{\pgfqpoint{0.462198in}{3.668827in}}{\pgfqpoint{0.472797in}{3.664437in}}{\pgfqpoint{0.483847in}{3.664437in}}%
\pgfpathlineto{\pgfqpoint{0.483847in}{3.664437in}}%
\pgfpathclose%
\pgfusepath{stroke}%
\end{pgfscope}%
\begin{pgfscope}%
\pgfpathrectangle{\pgfqpoint{0.393053in}{0.375000in}}{\pgfqpoint{6.356833in}{5.175000in}}%
\pgfusepath{clip}%
\pgfsetbuttcap%
\pgfsetroundjoin%
\pgfsetlinewidth{1.003750pt}%
\definecolor{currentstroke}{rgb}{0.827451,0.827451,0.827451}%
\pgfsetstrokecolor{currentstroke}%
\pgfsetdash{}{0pt}%
\pgfpathmoveto{\pgfqpoint{5.177456in}{0.348775in}}%
\pgfpathcurveto{\pgfqpoint{5.188506in}{0.348775in}}{\pgfqpoint{5.199105in}{0.353165in}}{\pgfqpoint{5.206919in}{0.360979in}}%
\pgfpathcurveto{\pgfqpoint{5.214732in}{0.368792in}}{\pgfqpoint{5.219122in}{0.379391in}}{\pgfqpoint{5.219122in}{0.390442in}}%
\pgfpathcurveto{\pgfqpoint{5.219122in}{0.401492in}}{\pgfqpoint{5.214732in}{0.412091in}}{\pgfqpoint{5.206919in}{0.419904in}}%
\pgfpathcurveto{\pgfqpoint{5.199105in}{0.427718in}}{\pgfqpoint{5.188506in}{0.432108in}}{\pgfqpoint{5.177456in}{0.432108in}}%
\pgfpathcurveto{\pgfqpoint{5.166406in}{0.432108in}}{\pgfqpoint{5.155807in}{0.427718in}}{\pgfqpoint{5.147993in}{0.419904in}}%
\pgfpathcurveto{\pgfqpoint{5.140179in}{0.412091in}}{\pgfqpoint{5.135789in}{0.401492in}}{\pgfqpoint{5.135789in}{0.390442in}}%
\pgfpathcurveto{\pgfqpoint{5.135789in}{0.379391in}}{\pgfqpoint{5.140179in}{0.368792in}}{\pgfqpoint{5.147993in}{0.360979in}}%
\pgfpathcurveto{\pgfqpoint{5.155807in}{0.353165in}}{\pgfqpoint{5.166406in}{0.348775in}}{\pgfqpoint{5.177456in}{0.348775in}}%
\pgfusepath{stroke}%
\end{pgfscope}%
\begin{pgfscope}%
\pgfpathrectangle{\pgfqpoint{0.393053in}{0.375000in}}{\pgfqpoint{6.356833in}{5.175000in}}%
\pgfusepath{clip}%
\pgfsetbuttcap%
\pgfsetroundjoin%
\pgfsetlinewidth{1.003750pt}%
\definecolor{currentstroke}{rgb}{0.827451,0.827451,0.827451}%
\pgfsetstrokecolor{currentstroke}%
\pgfsetdash{}{0pt}%
\pgfpathmoveto{\pgfqpoint{2.230001in}{1.209659in}}%
\pgfpathcurveto{\pgfqpoint{2.241051in}{1.209659in}}{\pgfqpoint{2.251650in}{1.214049in}}{\pgfqpoint{2.259463in}{1.221862in}}%
\pgfpathcurveto{\pgfqpoint{2.267277in}{1.229676in}}{\pgfqpoint{2.271667in}{1.240275in}}{\pgfqpoint{2.271667in}{1.251325in}}%
\pgfpathcurveto{\pgfqpoint{2.271667in}{1.262375in}}{\pgfqpoint{2.267277in}{1.272974in}}{\pgfqpoint{2.259463in}{1.280788in}}%
\pgfpathcurveto{\pgfqpoint{2.251650in}{1.288602in}}{\pgfqpoint{2.241051in}{1.292992in}}{\pgfqpoint{2.230001in}{1.292992in}}%
\pgfpathcurveto{\pgfqpoint{2.218950in}{1.292992in}}{\pgfqpoint{2.208351in}{1.288602in}}{\pgfqpoint{2.200538in}{1.280788in}}%
\pgfpathcurveto{\pgfqpoint{2.192724in}{1.272974in}}{\pgfqpoint{2.188334in}{1.262375in}}{\pgfqpoint{2.188334in}{1.251325in}}%
\pgfpathcurveto{\pgfqpoint{2.188334in}{1.240275in}}{\pgfqpoint{2.192724in}{1.229676in}}{\pgfqpoint{2.200538in}{1.221862in}}%
\pgfpathcurveto{\pgfqpoint{2.208351in}{1.214049in}}{\pgfqpoint{2.218950in}{1.209659in}}{\pgfqpoint{2.230001in}{1.209659in}}%
\pgfpathlineto{\pgfqpoint{2.230001in}{1.209659in}}%
\pgfpathclose%
\pgfusepath{stroke}%
\end{pgfscope}%
\begin{pgfscope}%
\pgfpathrectangle{\pgfqpoint{0.393053in}{0.375000in}}{\pgfqpoint{6.356833in}{5.175000in}}%
\pgfusepath{clip}%
\pgfsetbuttcap%
\pgfsetroundjoin%
\pgfsetlinewidth{1.003750pt}%
\definecolor{currentstroke}{rgb}{0.827451,0.827451,0.827451}%
\pgfsetstrokecolor{currentstroke}%
\pgfsetdash{}{0pt}%
\pgfpathmoveto{\pgfqpoint{5.340211in}{0.342419in}}%
\pgfpathcurveto{\pgfqpoint{5.351261in}{0.342419in}}{\pgfqpoint{5.361860in}{0.346810in}}{\pgfqpoint{5.369673in}{0.354623in}}%
\pgfpathcurveto{\pgfqpoint{5.377487in}{0.362437in}}{\pgfqpoint{5.381877in}{0.373036in}}{\pgfqpoint{5.381877in}{0.384086in}}%
\pgfpathcurveto{\pgfqpoint{5.381877in}{0.395136in}}{\pgfqpoint{5.377487in}{0.405735in}}{\pgfqpoint{5.369673in}{0.413549in}}%
\pgfpathcurveto{\pgfqpoint{5.361860in}{0.421362in}}{\pgfqpoint{5.351261in}{0.425753in}}{\pgfqpoint{5.340211in}{0.425753in}}%
\pgfpathcurveto{\pgfqpoint{5.329160in}{0.425753in}}{\pgfqpoint{5.318561in}{0.421362in}}{\pgfqpoint{5.310748in}{0.413549in}}%
\pgfpathcurveto{\pgfqpoint{5.302934in}{0.405735in}}{\pgfqpoint{5.298544in}{0.395136in}}{\pgfqpoint{5.298544in}{0.384086in}}%
\pgfpathcurveto{\pgfqpoint{5.298544in}{0.373036in}}{\pgfqpoint{5.302934in}{0.362437in}}{\pgfqpoint{5.310748in}{0.354623in}}%
\pgfpathcurveto{\pgfqpoint{5.318561in}{0.346810in}}{\pgfqpoint{5.329160in}{0.342419in}}{\pgfqpoint{5.340211in}{0.342419in}}%
\pgfusepath{stroke}%
\end{pgfscope}%
\begin{pgfscope}%
\pgfpathrectangle{\pgfqpoint{0.393053in}{0.375000in}}{\pgfqpoint{6.356833in}{5.175000in}}%
\pgfusepath{clip}%
\pgfsetbuttcap%
\pgfsetroundjoin%
\pgfsetlinewidth{1.003750pt}%
\definecolor{currentstroke}{rgb}{0.827451,0.827451,0.827451}%
\pgfsetstrokecolor{currentstroke}%
\pgfsetdash{}{0pt}%
\pgfpathmoveto{\pgfqpoint{1.881292in}{1.456529in}}%
\pgfpathcurveto{\pgfqpoint{1.892342in}{1.456529in}}{\pgfqpoint{1.902941in}{1.460919in}}{\pgfqpoint{1.910755in}{1.468732in}}%
\pgfpathcurveto{\pgfqpoint{1.918569in}{1.476546in}}{\pgfqpoint{1.922959in}{1.487145in}}{\pgfqpoint{1.922959in}{1.498195in}}%
\pgfpathcurveto{\pgfqpoint{1.922959in}{1.509245in}}{\pgfqpoint{1.918569in}{1.519844in}}{\pgfqpoint{1.910755in}{1.527658in}}%
\pgfpathcurveto{\pgfqpoint{1.902941in}{1.535472in}}{\pgfqpoint{1.892342in}{1.539862in}}{\pgfqpoint{1.881292in}{1.539862in}}%
\pgfpathcurveto{\pgfqpoint{1.870242in}{1.539862in}}{\pgfqpoint{1.859643in}{1.535472in}}{\pgfqpoint{1.851829in}{1.527658in}}%
\pgfpathcurveto{\pgfqpoint{1.844016in}{1.519844in}}{\pgfqpoint{1.839626in}{1.509245in}}{\pgfqpoint{1.839626in}{1.498195in}}%
\pgfpathcurveto{\pgfqpoint{1.839626in}{1.487145in}}{\pgfqpoint{1.844016in}{1.476546in}}{\pgfqpoint{1.851829in}{1.468732in}}%
\pgfpathcurveto{\pgfqpoint{1.859643in}{1.460919in}}{\pgfqpoint{1.870242in}{1.456529in}}{\pgfqpoint{1.881292in}{1.456529in}}%
\pgfpathlineto{\pgfqpoint{1.881292in}{1.456529in}}%
\pgfpathclose%
\pgfusepath{stroke}%
\end{pgfscope}%
\begin{pgfscope}%
\pgfpathrectangle{\pgfqpoint{0.393053in}{0.375000in}}{\pgfqpoint{6.356833in}{5.175000in}}%
\pgfusepath{clip}%
\pgfsetbuttcap%
\pgfsetroundjoin%
\pgfsetlinewidth{1.003750pt}%
\definecolor{currentstroke}{rgb}{0.827451,0.827451,0.827451}%
\pgfsetstrokecolor{currentstroke}%
\pgfsetdash{}{0pt}%
\pgfpathmoveto{\pgfqpoint{4.012103in}{0.498786in}}%
\pgfpathcurveto{\pgfqpoint{4.023153in}{0.498786in}}{\pgfqpoint{4.033752in}{0.503177in}}{\pgfqpoint{4.041566in}{0.510990in}}%
\pgfpathcurveto{\pgfqpoint{4.049379in}{0.518804in}}{\pgfqpoint{4.053769in}{0.529403in}}{\pgfqpoint{4.053769in}{0.540453in}}%
\pgfpathcurveto{\pgfqpoint{4.053769in}{0.551503in}}{\pgfqpoint{4.049379in}{0.562102in}}{\pgfqpoint{4.041566in}{0.569916in}}%
\pgfpathcurveto{\pgfqpoint{4.033752in}{0.577729in}}{\pgfqpoint{4.023153in}{0.582120in}}{\pgfqpoint{4.012103in}{0.582120in}}%
\pgfpathcurveto{\pgfqpoint{4.001053in}{0.582120in}}{\pgfqpoint{3.990454in}{0.577729in}}{\pgfqpoint{3.982640in}{0.569916in}}%
\pgfpathcurveto{\pgfqpoint{3.974826in}{0.562102in}}{\pgfqpoint{3.970436in}{0.551503in}}{\pgfqpoint{3.970436in}{0.540453in}}%
\pgfpathcurveto{\pgfqpoint{3.970436in}{0.529403in}}{\pgfqpoint{3.974826in}{0.518804in}}{\pgfqpoint{3.982640in}{0.510990in}}%
\pgfpathcurveto{\pgfqpoint{3.990454in}{0.503177in}}{\pgfqpoint{4.001053in}{0.498786in}}{\pgfqpoint{4.012103in}{0.498786in}}%
\pgfpathlineto{\pgfqpoint{4.012103in}{0.498786in}}%
\pgfpathclose%
\pgfusepath{stroke}%
\end{pgfscope}%
\begin{pgfscope}%
\pgfpathrectangle{\pgfqpoint{0.393053in}{0.375000in}}{\pgfqpoint{6.356833in}{5.175000in}}%
\pgfusepath{clip}%
\pgfsetbuttcap%
\pgfsetroundjoin%
\pgfsetlinewidth{1.003750pt}%
\definecolor{currentstroke}{rgb}{0.827451,0.827451,0.827451}%
\pgfsetstrokecolor{currentstroke}%
\pgfsetdash{}{0pt}%
\pgfpathmoveto{\pgfqpoint{0.773700in}{2.823025in}}%
\pgfpathcurveto{\pgfqpoint{0.784750in}{2.823025in}}{\pgfqpoint{0.795350in}{2.827415in}}{\pgfqpoint{0.803163in}{2.835229in}}%
\pgfpathcurveto{\pgfqpoint{0.810977in}{2.843042in}}{\pgfqpoint{0.815367in}{2.853641in}}{\pgfqpoint{0.815367in}{2.864691in}}%
\pgfpathcurveto{\pgfqpoint{0.815367in}{2.875742in}}{\pgfqpoint{0.810977in}{2.886341in}}{\pgfqpoint{0.803163in}{2.894154in}}%
\pgfpathcurveto{\pgfqpoint{0.795350in}{2.901968in}}{\pgfqpoint{0.784750in}{2.906358in}}{\pgfqpoint{0.773700in}{2.906358in}}%
\pgfpathcurveto{\pgfqpoint{0.762650in}{2.906358in}}{\pgfqpoint{0.752051in}{2.901968in}}{\pgfqpoint{0.744238in}{2.894154in}}%
\pgfpathcurveto{\pgfqpoint{0.736424in}{2.886341in}}{\pgfqpoint{0.732034in}{2.875742in}}{\pgfqpoint{0.732034in}{2.864691in}}%
\pgfpathcurveto{\pgfqpoint{0.732034in}{2.853641in}}{\pgfqpoint{0.736424in}{2.843042in}}{\pgfqpoint{0.744238in}{2.835229in}}%
\pgfpathcurveto{\pgfqpoint{0.752051in}{2.827415in}}{\pgfqpoint{0.762650in}{2.823025in}}{\pgfqpoint{0.773700in}{2.823025in}}%
\pgfpathlineto{\pgfqpoint{0.773700in}{2.823025in}}%
\pgfpathclose%
\pgfusepath{stroke}%
\end{pgfscope}%
\begin{pgfscope}%
\pgfpathrectangle{\pgfqpoint{0.393053in}{0.375000in}}{\pgfqpoint{6.356833in}{5.175000in}}%
\pgfusepath{clip}%
\pgfsetbuttcap%
\pgfsetroundjoin%
\pgfsetlinewidth{1.003750pt}%
\definecolor{currentstroke}{rgb}{0.827451,0.827451,0.827451}%
\pgfsetstrokecolor{currentstroke}%
\pgfsetdash{}{0pt}%
\pgfpathmoveto{\pgfqpoint{5.028115in}{0.358227in}}%
\pgfpathcurveto{\pgfqpoint{5.039165in}{0.358227in}}{\pgfqpoint{5.049764in}{0.362617in}}{\pgfqpoint{5.057577in}{0.370431in}}%
\pgfpathcurveto{\pgfqpoint{5.065391in}{0.378244in}}{\pgfqpoint{5.069781in}{0.388843in}}{\pgfqpoint{5.069781in}{0.399893in}}%
\pgfpathcurveto{\pgfqpoint{5.069781in}{0.410944in}}{\pgfqpoint{5.065391in}{0.421543in}}{\pgfqpoint{5.057577in}{0.429356in}}%
\pgfpathcurveto{\pgfqpoint{5.049764in}{0.437170in}}{\pgfqpoint{5.039165in}{0.441560in}}{\pgfqpoint{5.028115in}{0.441560in}}%
\pgfpathcurveto{\pgfqpoint{5.017064in}{0.441560in}}{\pgfqpoint{5.006465in}{0.437170in}}{\pgfqpoint{4.998652in}{0.429356in}}%
\pgfpathcurveto{\pgfqpoint{4.990838in}{0.421543in}}{\pgfqpoint{4.986448in}{0.410944in}}{\pgfqpoint{4.986448in}{0.399893in}}%
\pgfpathcurveto{\pgfqpoint{4.986448in}{0.388843in}}{\pgfqpoint{4.990838in}{0.378244in}}{\pgfqpoint{4.998652in}{0.370431in}}%
\pgfpathcurveto{\pgfqpoint{5.006465in}{0.362617in}}{\pgfqpoint{5.017064in}{0.358227in}}{\pgfqpoint{5.028115in}{0.358227in}}%
\pgfusepath{stroke}%
\end{pgfscope}%
\begin{pgfscope}%
\pgfpathrectangle{\pgfqpoint{0.393053in}{0.375000in}}{\pgfqpoint{6.356833in}{5.175000in}}%
\pgfusepath{clip}%
\pgfsetbuttcap%
\pgfsetroundjoin%
\pgfsetlinewidth{1.003750pt}%
\definecolor{currentstroke}{rgb}{0.827451,0.827451,0.827451}%
\pgfsetstrokecolor{currentstroke}%
\pgfsetdash{}{0pt}%
\pgfpathmoveto{\pgfqpoint{1.843211in}{1.515821in}}%
\pgfpathcurveto{\pgfqpoint{1.854261in}{1.515821in}}{\pgfqpoint{1.864860in}{1.520211in}}{\pgfqpoint{1.872674in}{1.528025in}}%
\pgfpathcurveto{\pgfqpoint{1.880487in}{1.535838in}}{\pgfqpoint{1.884878in}{1.546437in}}{\pgfqpoint{1.884878in}{1.557487in}}%
\pgfpathcurveto{\pgfqpoint{1.884878in}{1.568537in}}{\pgfqpoint{1.880487in}{1.579136in}}{\pgfqpoint{1.872674in}{1.586950in}}%
\pgfpathcurveto{\pgfqpoint{1.864860in}{1.594764in}}{\pgfqpoint{1.854261in}{1.599154in}}{\pgfqpoint{1.843211in}{1.599154in}}%
\pgfpathcurveto{\pgfqpoint{1.832161in}{1.599154in}}{\pgfqpoint{1.821562in}{1.594764in}}{\pgfqpoint{1.813748in}{1.586950in}}%
\pgfpathcurveto{\pgfqpoint{1.805934in}{1.579136in}}{\pgfqpoint{1.801544in}{1.568537in}}{\pgfqpoint{1.801544in}{1.557487in}}%
\pgfpathcurveto{\pgfqpoint{1.801544in}{1.546437in}}{\pgfqpoint{1.805934in}{1.535838in}}{\pgfqpoint{1.813748in}{1.528025in}}%
\pgfpathcurveto{\pgfqpoint{1.821562in}{1.520211in}}{\pgfqpoint{1.832161in}{1.515821in}}{\pgfqpoint{1.843211in}{1.515821in}}%
\pgfpathlineto{\pgfqpoint{1.843211in}{1.515821in}}%
\pgfpathclose%
\pgfusepath{stroke}%
\end{pgfscope}%
\begin{pgfscope}%
\pgfpathrectangle{\pgfqpoint{0.393053in}{0.375000in}}{\pgfqpoint{6.356833in}{5.175000in}}%
\pgfusepath{clip}%
\pgfsetbuttcap%
\pgfsetroundjoin%
\pgfsetlinewidth{1.003750pt}%
\definecolor{currentstroke}{rgb}{0.827451,0.827451,0.827451}%
\pgfsetstrokecolor{currentstroke}%
\pgfsetdash{}{0pt}%
\pgfpathmoveto{\pgfqpoint{5.335363in}{0.346733in}}%
\pgfpathcurveto{\pgfqpoint{5.346413in}{0.346733in}}{\pgfqpoint{5.357012in}{0.351123in}}{\pgfqpoint{5.364826in}{0.358937in}}%
\pgfpathcurveto{\pgfqpoint{5.372639in}{0.366750in}}{\pgfqpoint{5.377030in}{0.377349in}}{\pgfqpoint{5.377030in}{0.388400in}}%
\pgfpathcurveto{\pgfqpoint{5.377030in}{0.399450in}}{\pgfqpoint{5.372639in}{0.410049in}}{\pgfqpoint{5.364826in}{0.417862in}}%
\pgfpathcurveto{\pgfqpoint{5.357012in}{0.425676in}}{\pgfqpoint{5.346413in}{0.430066in}}{\pgfqpoint{5.335363in}{0.430066in}}%
\pgfpathcurveto{\pgfqpoint{5.324313in}{0.430066in}}{\pgfqpoint{5.313714in}{0.425676in}}{\pgfqpoint{5.305900in}{0.417862in}}%
\pgfpathcurveto{\pgfqpoint{5.298086in}{0.410049in}}{\pgfqpoint{5.293696in}{0.399450in}}{\pgfqpoint{5.293696in}{0.388400in}}%
\pgfpathcurveto{\pgfqpoint{5.293696in}{0.377349in}}{\pgfqpoint{5.298086in}{0.366750in}}{\pgfqpoint{5.305900in}{0.358937in}}%
\pgfpathcurveto{\pgfqpoint{5.313714in}{0.351123in}}{\pgfqpoint{5.324313in}{0.346733in}}{\pgfqpoint{5.335363in}{0.346733in}}%
\pgfusepath{stroke}%
\end{pgfscope}%
\begin{pgfscope}%
\pgfpathrectangle{\pgfqpoint{0.393053in}{0.375000in}}{\pgfqpoint{6.356833in}{5.175000in}}%
\pgfusepath{clip}%
\pgfsetbuttcap%
\pgfsetroundjoin%
\pgfsetlinewidth{1.003750pt}%
\definecolor{currentstroke}{rgb}{0.827451,0.827451,0.827451}%
\pgfsetstrokecolor{currentstroke}%
\pgfsetdash{}{0pt}%
\pgfpathmoveto{\pgfqpoint{4.638900in}{0.397904in}}%
\pgfpathcurveto{\pgfqpoint{4.649950in}{0.397904in}}{\pgfqpoint{4.660549in}{0.402294in}}{\pgfqpoint{4.668363in}{0.410108in}}%
\pgfpathcurveto{\pgfqpoint{4.676176in}{0.417922in}}{\pgfqpoint{4.680567in}{0.428521in}}{\pgfqpoint{4.680567in}{0.439571in}}%
\pgfpathcurveto{\pgfqpoint{4.680567in}{0.450621in}}{\pgfqpoint{4.676176in}{0.461220in}}{\pgfqpoint{4.668363in}{0.469034in}}%
\pgfpathcurveto{\pgfqpoint{4.660549in}{0.476847in}}{\pgfqpoint{4.649950in}{0.481237in}}{\pgfqpoint{4.638900in}{0.481237in}}%
\pgfpathcurveto{\pgfqpoint{4.627850in}{0.481237in}}{\pgfqpoint{4.617251in}{0.476847in}}{\pgfqpoint{4.609437in}{0.469034in}}%
\pgfpathcurveto{\pgfqpoint{4.601624in}{0.461220in}}{\pgfqpoint{4.597233in}{0.450621in}}{\pgfqpoint{4.597233in}{0.439571in}}%
\pgfpathcurveto{\pgfqpoint{4.597233in}{0.428521in}}{\pgfqpoint{4.601624in}{0.417922in}}{\pgfqpoint{4.609437in}{0.410108in}}%
\pgfpathcurveto{\pgfqpoint{4.617251in}{0.402294in}}{\pgfqpoint{4.627850in}{0.397904in}}{\pgfqpoint{4.638900in}{0.397904in}}%
\pgfpathlineto{\pgfqpoint{4.638900in}{0.397904in}}%
\pgfpathclose%
\pgfusepath{stroke}%
\end{pgfscope}%
\begin{pgfscope}%
\pgfpathrectangle{\pgfqpoint{0.393053in}{0.375000in}}{\pgfqpoint{6.356833in}{5.175000in}}%
\pgfusepath{clip}%
\pgfsetbuttcap%
\pgfsetroundjoin%
\pgfsetlinewidth{1.003750pt}%
\definecolor{currentstroke}{rgb}{0.827451,0.827451,0.827451}%
\pgfsetstrokecolor{currentstroke}%
\pgfsetdash{}{0pt}%
\pgfpathmoveto{\pgfqpoint{0.848540in}{2.695419in}}%
\pgfpathcurveto{\pgfqpoint{0.859590in}{2.695419in}}{\pgfqpoint{0.870189in}{2.699809in}}{\pgfqpoint{0.878002in}{2.707623in}}%
\pgfpathcurveto{\pgfqpoint{0.885816in}{2.715436in}}{\pgfqpoint{0.890206in}{2.726035in}}{\pgfqpoint{0.890206in}{2.737085in}}%
\pgfpathcurveto{\pgfqpoint{0.890206in}{2.748135in}}{\pgfqpoint{0.885816in}{2.758735in}}{\pgfqpoint{0.878002in}{2.766548in}}%
\pgfpathcurveto{\pgfqpoint{0.870189in}{2.774362in}}{\pgfqpoint{0.859590in}{2.778752in}}{\pgfqpoint{0.848540in}{2.778752in}}%
\pgfpathcurveto{\pgfqpoint{0.837489in}{2.778752in}}{\pgfqpoint{0.826890in}{2.774362in}}{\pgfqpoint{0.819077in}{2.766548in}}%
\pgfpathcurveto{\pgfqpoint{0.811263in}{2.758735in}}{\pgfqpoint{0.806873in}{2.748135in}}{\pgfqpoint{0.806873in}{2.737085in}}%
\pgfpathcurveto{\pgfqpoint{0.806873in}{2.726035in}}{\pgfqpoint{0.811263in}{2.715436in}}{\pgfqpoint{0.819077in}{2.707623in}}%
\pgfpathcurveto{\pgfqpoint{0.826890in}{2.699809in}}{\pgfqpoint{0.837489in}{2.695419in}}{\pgfqpoint{0.848540in}{2.695419in}}%
\pgfpathlineto{\pgfqpoint{0.848540in}{2.695419in}}%
\pgfpathclose%
\pgfusepath{stroke}%
\end{pgfscope}%
\begin{pgfscope}%
\pgfpathrectangle{\pgfqpoint{0.393053in}{0.375000in}}{\pgfqpoint{6.356833in}{5.175000in}}%
\pgfusepath{clip}%
\pgfsetbuttcap%
\pgfsetroundjoin%
\pgfsetlinewidth{1.003750pt}%
\definecolor{currentstroke}{rgb}{0.827451,0.827451,0.827451}%
\pgfsetstrokecolor{currentstroke}%
\pgfsetdash{}{0pt}%
\pgfpathmoveto{\pgfqpoint{3.397541in}{0.662535in}}%
\pgfpathcurveto{\pgfqpoint{3.408591in}{0.662535in}}{\pgfqpoint{3.419190in}{0.666925in}}{\pgfqpoint{3.427004in}{0.674738in}}%
\pgfpathcurveto{\pgfqpoint{3.434817in}{0.682552in}}{\pgfqpoint{3.439208in}{0.693151in}}{\pgfqpoint{3.439208in}{0.704201in}}%
\pgfpathcurveto{\pgfqpoint{3.439208in}{0.715251in}}{\pgfqpoint{3.434817in}{0.725850in}}{\pgfqpoint{3.427004in}{0.733664in}}%
\pgfpathcurveto{\pgfqpoint{3.419190in}{0.741478in}}{\pgfqpoint{3.408591in}{0.745868in}}{\pgfqpoint{3.397541in}{0.745868in}}%
\pgfpathcurveto{\pgfqpoint{3.386491in}{0.745868in}}{\pgfqpoint{3.375892in}{0.741478in}}{\pgfqpoint{3.368078in}{0.733664in}}%
\pgfpathcurveto{\pgfqpoint{3.360265in}{0.725850in}}{\pgfqpoint{3.355874in}{0.715251in}}{\pgfqpoint{3.355874in}{0.704201in}}%
\pgfpathcurveto{\pgfqpoint{3.355874in}{0.693151in}}{\pgfqpoint{3.360265in}{0.682552in}}{\pgfqpoint{3.368078in}{0.674738in}}%
\pgfpathcurveto{\pgfqpoint{3.375892in}{0.666925in}}{\pgfqpoint{3.386491in}{0.662535in}}{\pgfqpoint{3.397541in}{0.662535in}}%
\pgfpathlineto{\pgfqpoint{3.397541in}{0.662535in}}%
\pgfpathclose%
\pgfusepath{stroke}%
\end{pgfscope}%
\begin{pgfscope}%
\pgfpathrectangle{\pgfqpoint{0.393053in}{0.375000in}}{\pgfqpoint{6.356833in}{5.175000in}}%
\pgfusepath{clip}%
\pgfsetbuttcap%
\pgfsetroundjoin%
\pgfsetlinewidth{1.003750pt}%
\definecolor{currentstroke}{rgb}{0.827451,0.827451,0.827451}%
\pgfsetstrokecolor{currentstroke}%
\pgfsetdash{}{0pt}%
\pgfpathmoveto{\pgfqpoint{1.354968in}{1.928632in}}%
\pgfpathcurveto{\pgfqpoint{1.366018in}{1.928632in}}{\pgfqpoint{1.376617in}{1.933022in}}{\pgfqpoint{1.384431in}{1.940836in}}%
\pgfpathcurveto{\pgfqpoint{1.392245in}{1.948650in}}{\pgfqpoint{1.396635in}{1.959249in}}{\pgfqpoint{1.396635in}{1.970299in}}%
\pgfpathcurveto{\pgfqpoint{1.396635in}{1.981349in}}{\pgfqpoint{1.392245in}{1.991948in}}{\pgfqpoint{1.384431in}{1.999762in}}%
\pgfpathcurveto{\pgfqpoint{1.376617in}{2.007575in}}{\pgfqpoint{1.366018in}{2.011966in}}{\pgfqpoint{1.354968in}{2.011966in}}%
\pgfpathcurveto{\pgfqpoint{1.343918in}{2.011966in}}{\pgfqpoint{1.333319in}{2.007575in}}{\pgfqpoint{1.325505in}{1.999762in}}%
\pgfpathcurveto{\pgfqpoint{1.317692in}{1.991948in}}{\pgfqpoint{1.313302in}{1.981349in}}{\pgfqpoint{1.313302in}{1.970299in}}%
\pgfpathcurveto{\pgfqpoint{1.313302in}{1.959249in}}{\pgfqpoint{1.317692in}{1.948650in}}{\pgfqpoint{1.325505in}{1.940836in}}%
\pgfpathcurveto{\pgfqpoint{1.333319in}{1.933022in}}{\pgfqpoint{1.343918in}{1.928632in}}{\pgfqpoint{1.354968in}{1.928632in}}%
\pgfpathlineto{\pgfqpoint{1.354968in}{1.928632in}}%
\pgfpathclose%
\pgfusepath{stroke}%
\end{pgfscope}%
\begin{pgfscope}%
\pgfpathrectangle{\pgfqpoint{0.393053in}{0.375000in}}{\pgfqpoint{6.356833in}{5.175000in}}%
\pgfusepath{clip}%
\pgfsetbuttcap%
\pgfsetroundjoin%
\pgfsetlinewidth{1.003750pt}%
\definecolor{currentstroke}{rgb}{0.827451,0.827451,0.827451}%
\pgfsetstrokecolor{currentstroke}%
\pgfsetdash{}{0pt}%
\pgfpathmoveto{\pgfqpoint{0.421923in}{4.061978in}}%
\pgfpathcurveto{\pgfqpoint{0.432973in}{4.061978in}}{\pgfqpoint{0.443572in}{4.066368in}}{\pgfqpoint{0.451385in}{4.074182in}}%
\pgfpathcurveto{\pgfqpoint{0.459199in}{4.081995in}}{\pgfqpoint{0.463589in}{4.092594in}}{\pgfqpoint{0.463589in}{4.103644in}}%
\pgfpathcurveto{\pgfqpoint{0.463589in}{4.114694in}}{\pgfqpoint{0.459199in}{4.125293in}}{\pgfqpoint{0.451385in}{4.133107in}}%
\pgfpathcurveto{\pgfqpoint{0.443572in}{4.140921in}}{\pgfqpoint{0.432973in}{4.145311in}}{\pgfqpoint{0.421923in}{4.145311in}}%
\pgfpathcurveto{\pgfqpoint{0.410872in}{4.145311in}}{\pgfqpoint{0.400273in}{4.140921in}}{\pgfqpoint{0.392460in}{4.133107in}}%
\pgfpathcurveto{\pgfqpoint{0.384646in}{4.125293in}}{\pgfqpoint{0.380256in}{4.114694in}}{\pgfqpoint{0.380256in}{4.103644in}}%
\pgfpathcurveto{\pgfqpoint{0.380256in}{4.092594in}}{\pgfqpoint{0.384646in}{4.081995in}}{\pgfqpoint{0.392460in}{4.074182in}}%
\pgfpathcurveto{\pgfqpoint{0.400273in}{4.066368in}}{\pgfqpoint{0.410872in}{4.061978in}}{\pgfqpoint{0.421923in}{4.061978in}}%
\pgfpathlineto{\pgfqpoint{0.421923in}{4.061978in}}%
\pgfpathclose%
\pgfusepath{stroke}%
\end{pgfscope}%
\begin{pgfscope}%
\pgfpathrectangle{\pgfqpoint{0.393053in}{0.375000in}}{\pgfqpoint{6.356833in}{5.175000in}}%
\pgfusepath{clip}%
\pgfsetbuttcap%
\pgfsetroundjoin%
\pgfsetlinewidth{1.003750pt}%
\definecolor{currentstroke}{rgb}{0.827451,0.827451,0.827451}%
\pgfsetstrokecolor{currentstroke}%
\pgfsetdash{}{0pt}%
\pgfpathmoveto{\pgfqpoint{1.040104in}{2.359499in}}%
\pgfpathcurveto{\pgfqpoint{1.051154in}{2.359499in}}{\pgfqpoint{1.061753in}{2.363889in}}{\pgfqpoint{1.069567in}{2.371703in}}%
\pgfpathcurveto{\pgfqpoint{1.077381in}{2.379516in}}{\pgfqpoint{1.081771in}{2.390115in}}{\pgfqpoint{1.081771in}{2.401165in}}%
\pgfpathcurveto{\pgfqpoint{1.081771in}{2.412215in}}{\pgfqpoint{1.077381in}{2.422814in}}{\pgfqpoint{1.069567in}{2.430628in}}%
\pgfpathcurveto{\pgfqpoint{1.061753in}{2.438442in}}{\pgfqpoint{1.051154in}{2.442832in}}{\pgfqpoint{1.040104in}{2.442832in}}%
\pgfpathcurveto{\pgfqpoint{1.029054in}{2.442832in}}{\pgfqpoint{1.018455in}{2.438442in}}{\pgfqpoint{1.010641in}{2.430628in}}%
\pgfpathcurveto{\pgfqpoint{1.002828in}{2.422814in}}{\pgfqpoint{0.998437in}{2.412215in}}{\pgfqpoint{0.998437in}{2.401165in}}%
\pgfpathcurveto{\pgfqpoint{0.998437in}{2.390115in}}{\pgfqpoint{1.002828in}{2.379516in}}{\pgfqpoint{1.010641in}{2.371703in}}%
\pgfpathcurveto{\pgfqpoint{1.018455in}{2.363889in}}{\pgfqpoint{1.029054in}{2.359499in}}{\pgfqpoint{1.040104in}{2.359499in}}%
\pgfpathlineto{\pgfqpoint{1.040104in}{2.359499in}}%
\pgfpathclose%
\pgfusepath{stroke}%
\end{pgfscope}%
\begin{pgfscope}%
\pgfpathrectangle{\pgfqpoint{0.393053in}{0.375000in}}{\pgfqpoint{6.356833in}{5.175000in}}%
\pgfusepath{clip}%
\pgfsetbuttcap%
\pgfsetroundjoin%
\pgfsetlinewidth{1.003750pt}%
\definecolor{currentstroke}{rgb}{0.827451,0.827451,0.827451}%
\pgfsetstrokecolor{currentstroke}%
\pgfsetdash{}{0pt}%
\pgfpathmoveto{\pgfqpoint{5.582662in}{0.336418in}}%
\pgfpathcurveto{\pgfqpoint{5.593712in}{0.336418in}}{\pgfqpoint{5.604311in}{0.340808in}}{\pgfqpoint{5.612124in}{0.348622in}}%
\pgfpathcurveto{\pgfqpoint{5.619938in}{0.356435in}}{\pgfqpoint{5.624328in}{0.367034in}}{\pgfqpoint{5.624328in}{0.378084in}}%
\pgfpathcurveto{\pgfqpoint{5.624328in}{0.389135in}}{\pgfqpoint{5.619938in}{0.399734in}}{\pgfqpoint{5.612124in}{0.407547in}}%
\pgfpathcurveto{\pgfqpoint{5.604311in}{0.415361in}}{\pgfqpoint{5.593712in}{0.419751in}}{\pgfqpoint{5.582662in}{0.419751in}}%
\pgfpathcurveto{\pgfqpoint{5.571611in}{0.419751in}}{\pgfqpoint{5.561012in}{0.415361in}}{\pgfqpoint{5.553199in}{0.407547in}}%
\pgfpathcurveto{\pgfqpoint{5.545385in}{0.399734in}}{\pgfqpoint{5.540995in}{0.389135in}}{\pgfqpoint{5.540995in}{0.378084in}}%
\pgfpathcurveto{\pgfqpoint{5.540995in}{0.367034in}}{\pgfqpoint{5.545385in}{0.356435in}}{\pgfqpoint{5.553199in}{0.348622in}}%
\pgfpathcurveto{\pgfqpoint{5.561012in}{0.340808in}}{\pgfqpoint{5.571611in}{0.336418in}}{\pgfqpoint{5.582662in}{0.336418in}}%
\pgfusepath{stroke}%
\end{pgfscope}%
\begin{pgfscope}%
\pgfpathrectangle{\pgfqpoint{0.393053in}{0.375000in}}{\pgfqpoint{6.356833in}{5.175000in}}%
\pgfusepath{clip}%
\pgfsetbuttcap%
\pgfsetroundjoin%
\pgfsetlinewidth{1.003750pt}%
\definecolor{currentstroke}{rgb}{0.827451,0.827451,0.827451}%
\pgfsetstrokecolor{currentstroke}%
\pgfsetdash{}{0pt}%
\pgfpathmoveto{\pgfqpoint{4.217280in}{0.449238in}}%
\pgfpathcurveto{\pgfqpoint{4.228330in}{0.449238in}}{\pgfqpoint{4.238929in}{0.453628in}}{\pgfqpoint{4.246743in}{0.461441in}}%
\pgfpathcurveto{\pgfqpoint{4.254556in}{0.469255in}}{\pgfqpoint{4.258947in}{0.479854in}}{\pgfqpoint{4.258947in}{0.490904in}}%
\pgfpathcurveto{\pgfqpoint{4.258947in}{0.501954in}}{\pgfqpoint{4.254556in}{0.512553in}}{\pgfqpoint{4.246743in}{0.520367in}}%
\pgfpathcurveto{\pgfqpoint{4.238929in}{0.528181in}}{\pgfqpoint{4.228330in}{0.532571in}}{\pgfqpoint{4.217280in}{0.532571in}}%
\pgfpathcurveto{\pgfqpoint{4.206230in}{0.532571in}}{\pgfqpoint{4.195631in}{0.528181in}}{\pgfqpoint{4.187817in}{0.520367in}}%
\pgfpathcurveto{\pgfqpoint{4.180004in}{0.512553in}}{\pgfqpoint{4.175613in}{0.501954in}}{\pgfqpoint{4.175613in}{0.490904in}}%
\pgfpathcurveto{\pgfqpoint{4.175613in}{0.479854in}}{\pgfqpoint{4.180004in}{0.469255in}}{\pgfqpoint{4.187817in}{0.461441in}}%
\pgfpathcurveto{\pgfqpoint{4.195631in}{0.453628in}}{\pgfqpoint{4.206230in}{0.449238in}}{\pgfqpoint{4.217280in}{0.449238in}}%
\pgfpathlineto{\pgfqpoint{4.217280in}{0.449238in}}%
\pgfpathclose%
\pgfusepath{stroke}%
\end{pgfscope}%
\begin{pgfscope}%
\pgfpathrectangle{\pgfqpoint{0.393053in}{0.375000in}}{\pgfqpoint{6.356833in}{5.175000in}}%
\pgfusepath{clip}%
\pgfsetbuttcap%
\pgfsetroundjoin%
\pgfsetlinewidth{1.003750pt}%
\definecolor{currentstroke}{rgb}{0.827451,0.827451,0.827451}%
\pgfsetstrokecolor{currentstroke}%
\pgfsetdash{}{0pt}%
\pgfpathmoveto{\pgfqpoint{3.305179in}{0.685348in}}%
\pgfpathcurveto{\pgfqpoint{3.316229in}{0.685348in}}{\pgfqpoint{3.326828in}{0.689738in}}{\pgfqpoint{3.334642in}{0.697552in}}%
\pgfpathcurveto{\pgfqpoint{3.342455in}{0.705365in}}{\pgfqpoint{3.346846in}{0.715964in}}{\pgfqpoint{3.346846in}{0.727015in}}%
\pgfpathcurveto{\pgfqpoint{3.346846in}{0.738065in}}{\pgfqpoint{3.342455in}{0.748664in}}{\pgfqpoint{3.334642in}{0.756477in}}%
\pgfpathcurveto{\pgfqpoint{3.326828in}{0.764291in}}{\pgfqpoint{3.316229in}{0.768681in}}{\pgfqpoint{3.305179in}{0.768681in}}%
\pgfpathcurveto{\pgfqpoint{3.294129in}{0.768681in}}{\pgfqpoint{3.283530in}{0.764291in}}{\pgfqpoint{3.275716in}{0.756477in}}%
\pgfpathcurveto{\pgfqpoint{3.267903in}{0.748664in}}{\pgfqpoint{3.263512in}{0.738065in}}{\pgfqpoint{3.263512in}{0.727015in}}%
\pgfpathcurveto{\pgfqpoint{3.263512in}{0.715964in}}{\pgfqpoint{3.267903in}{0.705365in}}{\pgfqpoint{3.275716in}{0.697552in}}%
\pgfpathcurveto{\pgfqpoint{3.283530in}{0.689738in}}{\pgfqpoint{3.294129in}{0.685348in}}{\pgfqpoint{3.305179in}{0.685348in}}%
\pgfpathlineto{\pgfqpoint{3.305179in}{0.685348in}}%
\pgfpathclose%
\pgfusepath{stroke}%
\end{pgfscope}%
\begin{pgfscope}%
\pgfpathrectangle{\pgfqpoint{0.393053in}{0.375000in}}{\pgfqpoint{6.356833in}{5.175000in}}%
\pgfusepath{clip}%
\pgfsetbuttcap%
\pgfsetroundjoin%
\pgfsetlinewidth{1.003750pt}%
\definecolor{currentstroke}{rgb}{0.827451,0.827451,0.827451}%
\pgfsetstrokecolor{currentstroke}%
\pgfsetdash{}{0pt}%
\pgfpathmoveto{\pgfqpoint{0.511163in}{3.526410in}}%
\pgfpathcurveto{\pgfqpoint{0.522213in}{3.526410in}}{\pgfqpoint{0.532812in}{3.530801in}}{\pgfqpoint{0.540626in}{3.538614in}}%
\pgfpathcurveto{\pgfqpoint{0.548440in}{3.546428in}}{\pgfqpoint{0.552830in}{3.557027in}}{\pgfqpoint{0.552830in}{3.568077in}}%
\pgfpathcurveto{\pgfqpoint{0.552830in}{3.579127in}}{\pgfqpoint{0.548440in}{3.589726in}}{\pgfqpoint{0.540626in}{3.597540in}}%
\pgfpathcurveto{\pgfqpoint{0.532812in}{3.605353in}}{\pgfqpoint{0.522213in}{3.609744in}}{\pgfqpoint{0.511163in}{3.609744in}}%
\pgfpathcurveto{\pgfqpoint{0.500113in}{3.609744in}}{\pgfqpoint{0.489514in}{3.605353in}}{\pgfqpoint{0.481700in}{3.597540in}}%
\pgfpathcurveto{\pgfqpoint{0.473887in}{3.589726in}}{\pgfqpoint{0.469496in}{3.579127in}}{\pgfqpoint{0.469496in}{3.568077in}}%
\pgfpathcurveto{\pgfqpoint{0.469496in}{3.557027in}}{\pgfqpoint{0.473887in}{3.546428in}}{\pgfqpoint{0.481700in}{3.538614in}}%
\pgfpathcurveto{\pgfqpoint{0.489514in}{3.530801in}}{\pgfqpoint{0.500113in}{3.526410in}}{\pgfqpoint{0.511163in}{3.526410in}}%
\pgfpathlineto{\pgfqpoint{0.511163in}{3.526410in}}%
\pgfpathclose%
\pgfusepath{stroke}%
\end{pgfscope}%
\begin{pgfscope}%
\pgfpathrectangle{\pgfqpoint{0.393053in}{0.375000in}}{\pgfqpoint{6.356833in}{5.175000in}}%
\pgfusepath{clip}%
\pgfsetbuttcap%
\pgfsetroundjoin%
\pgfsetlinewidth{1.003750pt}%
\definecolor{currentstroke}{rgb}{0.827451,0.827451,0.827451}%
\pgfsetstrokecolor{currentstroke}%
\pgfsetdash{}{0pt}%
\pgfpathmoveto{\pgfqpoint{4.727978in}{0.381934in}}%
\pgfpathcurveto{\pgfqpoint{4.739028in}{0.381934in}}{\pgfqpoint{4.749627in}{0.386324in}}{\pgfqpoint{4.757441in}{0.394137in}}%
\pgfpathcurveto{\pgfqpoint{4.765255in}{0.401951in}}{\pgfqpoint{4.769645in}{0.412550in}}{\pgfqpoint{4.769645in}{0.423600in}}%
\pgfpathcurveto{\pgfqpoint{4.769645in}{0.434650in}}{\pgfqpoint{4.765255in}{0.445249in}}{\pgfqpoint{4.757441in}{0.453063in}}%
\pgfpathcurveto{\pgfqpoint{4.749627in}{0.460877in}}{\pgfqpoint{4.739028in}{0.465267in}}{\pgfqpoint{4.727978in}{0.465267in}}%
\pgfpathcurveto{\pgfqpoint{4.716928in}{0.465267in}}{\pgfqpoint{4.706329in}{0.460877in}}{\pgfqpoint{4.698516in}{0.453063in}}%
\pgfpathcurveto{\pgfqpoint{4.690702in}{0.445249in}}{\pgfqpoint{4.686312in}{0.434650in}}{\pgfqpoint{4.686312in}{0.423600in}}%
\pgfpathcurveto{\pgfqpoint{4.686312in}{0.412550in}}{\pgfqpoint{4.690702in}{0.401951in}}{\pgfqpoint{4.698516in}{0.394137in}}%
\pgfpathcurveto{\pgfqpoint{4.706329in}{0.386324in}}{\pgfqpoint{4.716928in}{0.381934in}}{\pgfqpoint{4.727978in}{0.381934in}}%
\pgfpathlineto{\pgfqpoint{4.727978in}{0.381934in}}%
\pgfpathclose%
\pgfusepath{stroke}%
\end{pgfscope}%
\begin{pgfscope}%
\pgfpathrectangle{\pgfqpoint{0.393053in}{0.375000in}}{\pgfqpoint{6.356833in}{5.175000in}}%
\pgfusepath{clip}%
\pgfsetbuttcap%
\pgfsetroundjoin%
\pgfsetlinewidth{1.003750pt}%
\definecolor{currentstroke}{rgb}{0.827451,0.827451,0.827451}%
\pgfsetstrokecolor{currentstroke}%
\pgfsetdash{}{0pt}%
\pgfpathmoveto{\pgfqpoint{1.635957in}{1.655850in}}%
\pgfpathcurveto{\pgfqpoint{1.647007in}{1.655850in}}{\pgfqpoint{1.657606in}{1.660240in}}{\pgfqpoint{1.665420in}{1.668054in}}%
\pgfpathcurveto{\pgfqpoint{1.673233in}{1.675867in}}{\pgfqpoint{1.677624in}{1.686466in}}{\pgfqpoint{1.677624in}{1.697516in}}%
\pgfpathcurveto{\pgfqpoint{1.677624in}{1.708566in}}{\pgfqpoint{1.673233in}{1.719166in}}{\pgfqpoint{1.665420in}{1.726979in}}%
\pgfpathcurveto{\pgfqpoint{1.657606in}{1.734793in}}{\pgfqpoint{1.647007in}{1.739183in}}{\pgfqpoint{1.635957in}{1.739183in}}%
\pgfpathcurveto{\pgfqpoint{1.624907in}{1.739183in}}{\pgfqpoint{1.614308in}{1.734793in}}{\pgfqpoint{1.606494in}{1.726979in}}%
\pgfpathcurveto{\pgfqpoint{1.598680in}{1.719166in}}{\pgfqpoint{1.594290in}{1.708566in}}{\pgfqpoint{1.594290in}{1.697516in}}%
\pgfpathcurveto{\pgfqpoint{1.594290in}{1.686466in}}{\pgfqpoint{1.598680in}{1.675867in}}{\pgfqpoint{1.606494in}{1.668054in}}%
\pgfpathcurveto{\pgfqpoint{1.614308in}{1.660240in}}{\pgfqpoint{1.624907in}{1.655850in}}{\pgfqpoint{1.635957in}{1.655850in}}%
\pgfpathlineto{\pgfqpoint{1.635957in}{1.655850in}}%
\pgfpathclose%
\pgfusepath{stroke}%
\end{pgfscope}%
\begin{pgfscope}%
\pgfpathrectangle{\pgfqpoint{0.393053in}{0.375000in}}{\pgfqpoint{6.356833in}{5.175000in}}%
\pgfusepath{clip}%
\pgfsetbuttcap%
\pgfsetroundjoin%
\pgfsetlinewidth{1.003750pt}%
\definecolor{currentstroke}{rgb}{0.827451,0.827451,0.827451}%
\pgfsetstrokecolor{currentstroke}%
\pgfsetdash{}{0pt}%
\pgfpathmoveto{\pgfqpoint{0.556022in}{3.362195in}}%
\pgfpathcurveto{\pgfqpoint{0.567072in}{3.362195in}}{\pgfqpoint{0.577671in}{3.366585in}}{\pgfqpoint{0.585485in}{3.374399in}}%
\pgfpathcurveto{\pgfqpoint{0.593298in}{3.382212in}}{\pgfqpoint{0.597688in}{3.392811in}}{\pgfqpoint{0.597688in}{3.403862in}}%
\pgfpathcurveto{\pgfqpoint{0.597688in}{3.414912in}}{\pgfqpoint{0.593298in}{3.425511in}}{\pgfqpoint{0.585485in}{3.433324in}}%
\pgfpathcurveto{\pgfqpoint{0.577671in}{3.441138in}}{\pgfqpoint{0.567072in}{3.445528in}}{\pgfqpoint{0.556022in}{3.445528in}}%
\pgfpathcurveto{\pgfqpoint{0.544972in}{3.445528in}}{\pgfqpoint{0.534373in}{3.441138in}}{\pgfqpoint{0.526559in}{3.433324in}}%
\pgfpathcurveto{\pgfqpoint{0.518745in}{3.425511in}}{\pgfqpoint{0.514355in}{3.414912in}}{\pgfqpoint{0.514355in}{3.403862in}}%
\pgfpathcurveto{\pgfqpoint{0.514355in}{3.392811in}}{\pgfqpoint{0.518745in}{3.382212in}}{\pgfqpoint{0.526559in}{3.374399in}}%
\pgfpathcurveto{\pgfqpoint{0.534373in}{3.366585in}}{\pgfqpoint{0.544972in}{3.362195in}}{\pgfqpoint{0.556022in}{3.362195in}}%
\pgfpathlineto{\pgfqpoint{0.556022in}{3.362195in}}%
\pgfpathclose%
\pgfusepath{stroke}%
\end{pgfscope}%
\begin{pgfscope}%
\pgfpathrectangle{\pgfqpoint{0.393053in}{0.375000in}}{\pgfqpoint{6.356833in}{5.175000in}}%
\pgfusepath{clip}%
\pgfsetbuttcap%
\pgfsetroundjoin%
\pgfsetlinewidth{1.003750pt}%
\definecolor{currentstroke}{rgb}{0.827451,0.827451,0.827451}%
\pgfsetstrokecolor{currentstroke}%
\pgfsetdash{}{0pt}%
\pgfpathmoveto{\pgfqpoint{2.914468in}{0.847035in}}%
\pgfpathcurveto{\pgfqpoint{2.925518in}{0.847035in}}{\pgfqpoint{2.936117in}{0.851426in}}{\pgfqpoint{2.943931in}{0.859239in}}%
\pgfpathcurveto{\pgfqpoint{2.951744in}{0.867053in}}{\pgfqpoint{2.956135in}{0.877652in}}{\pgfqpoint{2.956135in}{0.888702in}}%
\pgfpathcurveto{\pgfqpoint{2.956135in}{0.899752in}}{\pgfqpoint{2.951744in}{0.910351in}}{\pgfqpoint{2.943931in}{0.918165in}}%
\pgfpathcurveto{\pgfqpoint{2.936117in}{0.925978in}}{\pgfqpoint{2.925518in}{0.930369in}}{\pgfqpoint{2.914468in}{0.930369in}}%
\pgfpathcurveto{\pgfqpoint{2.903418in}{0.930369in}}{\pgfqpoint{2.892819in}{0.925978in}}{\pgfqpoint{2.885005in}{0.918165in}}%
\pgfpathcurveto{\pgfqpoint{2.877192in}{0.910351in}}{\pgfqpoint{2.872801in}{0.899752in}}{\pgfqpoint{2.872801in}{0.888702in}}%
\pgfpathcurveto{\pgfqpoint{2.872801in}{0.877652in}}{\pgfqpoint{2.877192in}{0.867053in}}{\pgfqpoint{2.885005in}{0.859239in}}%
\pgfpathcurveto{\pgfqpoint{2.892819in}{0.851426in}}{\pgfqpoint{2.903418in}{0.847035in}}{\pgfqpoint{2.914468in}{0.847035in}}%
\pgfpathlineto{\pgfqpoint{2.914468in}{0.847035in}}%
\pgfpathclose%
\pgfusepath{stroke}%
\end{pgfscope}%
\begin{pgfscope}%
\pgfpathrectangle{\pgfqpoint{0.393053in}{0.375000in}}{\pgfqpoint{6.356833in}{5.175000in}}%
\pgfusepath{clip}%
\pgfsetbuttcap%
\pgfsetroundjoin%
\pgfsetlinewidth{1.003750pt}%
\definecolor{currentstroke}{rgb}{0.827451,0.827451,0.827451}%
\pgfsetstrokecolor{currentstroke}%
\pgfsetdash{}{0pt}%
\pgfpathmoveto{\pgfqpoint{1.943500in}{1.413335in}}%
\pgfpathcurveto{\pgfqpoint{1.954550in}{1.413335in}}{\pgfqpoint{1.965149in}{1.417726in}}{\pgfqpoint{1.972963in}{1.425539in}}%
\pgfpathcurveto{\pgfqpoint{1.980776in}{1.433353in}}{\pgfqpoint{1.985166in}{1.443952in}}{\pgfqpoint{1.985166in}{1.455002in}}%
\pgfpathcurveto{\pgfqpoint{1.985166in}{1.466052in}}{\pgfqpoint{1.980776in}{1.476651in}}{\pgfqpoint{1.972963in}{1.484465in}}%
\pgfpathcurveto{\pgfqpoint{1.965149in}{1.492278in}}{\pgfqpoint{1.954550in}{1.496669in}}{\pgfqpoint{1.943500in}{1.496669in}}%
\pgfpathcurveto{\pgfqpoint{1.932450in}{1.496669in}}{\pgfqpoint{1.921851in}{1.492278in}}{\pgfqpoint{1.914037in}{1.484465in}}%
\pgfpathcurveto{\pgfqpoint{1.906223in}{1.476651in}}{\pgfqpoint{1.901833in}{1.466052in}}{\pgfqpoint{1.901833in}{1.455002in}}%
\pgfpathcurveto{\pgfqpoint{1.901833in}{1.443952in}}{\pgfqpoint{1.906223in}{1.433353in}}{\pgfqpoint{1.914037in}{1.425539in}}%
\pgfpathcurveto{\pgfqpoint{1.921851in}{1.417726in}}{\pgfqpoint{1.932450in}{1.413335in}}{\pgfqpoint{1.943500in}{1.413335in}}%
\pgfpathlineto{\pgfqpoint{1.943500in}{1.413335in}}%
\pgfpathclose%
\pgfusepath{stroke}%
\end{pgfscope}%
\begin{pgfscope}%
\pgfpathrectangle{\pgfqpoint{0.393053in}{0.375000in}}{\pgfqpoint{6.356833in}{5.175000in}}%
\pgfusepath{clip}%
\pgfsetbuttcap%
\pgfsetroundjoin%
\pgfsetlinewidth{1.003750pt}%
\definecolor{currentstroke}{rgb}{0.827451,0.827451,0.827451}%
\pgfsetstrokecolor{currentstroke}%
\pgfsetdash{}{0pt}%
\pgfpathmoveto{\pgfqpoint{0.846344in}{2.698062in}}%
\pgfpathcurveto{\pgfqpoint{0.857394in}{2.698062in}}{\pgfqpoint{0.867993in}{2.702452in}}{\pgfqpoint{0.875807in}{2.710266in}}%
\pgfpathcurveto{\pgfqpoint{0.883621in}{2.718079in}}{\pgfqpoint{0.888011in}{2.728678in}}{\pgfqpoint{0.888011in}{2.739729in}}%
\pgfpathcurveto{\pgfqpoint{0.888011in}{2.750779in}}{\pgfqpoint{0.883621in}{2.761378in}}{\pgfqpoint{0.875807in}{2.769191in}}%
\pgfpathcurveto{\pgfqpoint{0.867993in}{2.777005in}}{\pgfqpoint{0.857394in}{2.781395in}}{\pgfqpoint{0.846344in}{2.781395in}}%
\pgfpathcurveto{\pgfqpoint{0.835294in}{2.781395in}}{\pgfqpoint{0.824695in}{2.777005in}}{\pgfqpoint{0.816881in}{2.769191in}}%
\pgfpathcurveto{\pgfqpoint{0.809068in}{2.761378in}}{\pgfqpoint{0.804678in}{2.750779in}}{\pgfqpoint{0.804678in}{2.739729in}}%
\pgfpathcurveto{\pgfqpoint{0.804678in}{2.728678in}}{\pgfqpoint{0.809068in}{2.718079in}}{\pgfqpoint{0.816881in}{2.710266in}}%
\pgfpathcurveto{\pgfqpoint{0.824695in}{2.702452in}}{\pgfqpoint{0.835294in}{2.698062in}}{\pgfqpoint{0.846344in}{2.698062in}}%
\pgfpathlineto{\pgfqpoint{0.846344in}{2.698062in}}%
\pgfpathclose%
\pgfusepath{stroke}%
\end{pgfscope}%
\begin{pgfscope}%
\pgfpathrectangle{\pgfqpoint{0.393053in}{0.375000in}}{\pgfqpoint{6.356833in}{5.175000in}}%
\pgfusepath{clip}%
\pgfsetbuttcap%
\pgfsetroundjoin%
\pgfsetlinewidth{1.003750pt}%
\definecolor{currentstroke}{rgb}{0.827451,0.827451,0.827451}%
\pgfsetstrokecolor{currentstroke}%
\pgfsetdash{}{0pt}%
\pgfpathmoveto{\pgfqpoint{0.393825in}{4.495013in}}%
\pgfpathcurveto{\pgfqpoint{0.404875in}{4.495013in}}{\pgfqpoint{0.415474in}{4.499403in}}{\pgfqpoint{0.423288in}{4.507217in}}%
\pgfpathcurveto{\pgfqpoint{0.431101in}{4.515030in}}{\pgfqpoint{0.435492in}{4.525629in}}{\pgfqpoint{0.435492in}{4.536679in}}%
\pgfpathcurveto{\pgfqpoint{0.435492in}{4.547730in}}{\pgfqpoint{0.431101in}{4.558329in}}{\pgfqpoint{0.423288in}{4.566142in}}%
\pgfpathcurveto{\pgfqpoint{0.415474in}{4.573956in}}{\pgfqpoint{0.404875in}{4.578346in}}{\pgfqpoint{0.393825in}{4.578346in}}%
\pgfpathcurveto{\pgfqpoint{0.382775in}{4.578346in}}{\pgfqpoint{0.372176in}{4.573956in}}{\pgfqpoint{0.364362in}{4.566142in}}%
\pgfpathcurveto{\pgfqpoint{0.356548in}{4.558329in}}{\pgfqpoint{0.352158in}{4.547730in}}{\pgfqpoint{0.352158in}{4.536679in}}%
\pgfpathcurveto{\pgfqpoint{0.352158in}{4.525629in}}{\pgfqpoint{0.356548in}{4.515030in}}{\pgfqpoint{0.364362in}{4.507217in}}%
\pgfpathcurveto{\pgfqpoint{0.372176in}{4.499403in}}{\pgfqpoint{0.382775in}{4.495013in}}{\pgfqpoint{0.393825in}{4.495013in}}%
\pgfpathlineto{\pgfqpoint{0.393825in}{4.495013in}}%
\pgfpathclose%
\pgfusepath{stroke}%
\end{pgfscope}%
\begin{pgfscope}%
\pgfpathrectangle{\pgfqpoint{0.393053in}{0.375000in}}{\pgfqpoint{6.356833in}{5.175000in}}%
\pgfusepath{clip}%
\pgfsetbuttcap%
\pgfsetroundjoin%
\pgfsetlinewidth{1.003750pt}%
\definecolor{currentstroke}{rgb}{0.827451,0.827451,0.827451}%
\pgfsetstrokecolor{currentstroke}%
\pgfsetdash{}{0pt}%
\pgfpathmoveto{\pgfqpoint{1.150964in}{2.182718in}}%
\pgfpathcurveto{\pgfqpoint{1.162014in}{2.182718in}}{\pgfqpoint{1.172614in}{2.187108in}}{\pgfqpoint{1.180427in}{2.194921in}}%
\pgfpathcurveto{\pgfqpoint{1.188241in}{2.202735in}}{\pgfqpoint{1.192631in}{2.213334in}}{\pgfqpoint{1.192631in}{2.224384in}}%
\pgfpathcurveto{\pgfqpoint{1.192631in}{2.235434in}}{\pgfqpoint{1.188241in}{2.246033in}}{\pgfqpoint{1.180427in}{2.253847in}}%
\pgfpathcurveto{\pgfqpoint{1.172614in}{2.261661in}}{\pgfqpoint{1.162014in}{2.266051in}}{\pgfqpoint{1.150964in}{2.266051in}}%
\pgfpathcurveto{\pgfqpoint{1.139914in}{2.266051in}}{\pgfqpoint{1.129315in}{2.261661in}}{\pgfqpoint{1.121502in}{2.253847in}}%
\pgfpathcurveto{\pgfqpoint{1.113688in}{2.246033in}}{\pgfqpoint{1.109298in}{2.235434in}}{\pgfqpoint{1.109298in}{2.224384in}}%
\pgfpathcurveto{\pgfqpoint{1.109298in}{2.213334in}}{\pgfqpoint{1.113688in}{2.202735in}}{\pgfqpoint{1.121502in}{2.194921in}}%
\pgfpathcurveto{\pgfqpoint{1.129315in}{2.187108in}}{\pgfqpoint{1.139914in}{2.182718in}}{\pgfqpoint{1.150964in}{2.182718in}}%
\pgfpathlineto{\pgfqpoint{1.150964in}{2.182718in}}%
\pgfpathclose%
\pgfusepath{stroke}%
\end{pgfscope}%
\begin{pgfscope}%
\pgfpathrectangle{\pgfqpoint{0.393053in}{0.375000in}}{\pgfqpoint{6.356833in}{5.175000in}}%
\pgfusepath{clip}%
\pgfsetbuttcap%
\pgfsetroundjoin%
\pgfsetlinewidth{1.003750pt}%
\definecolor{currentstroke}{rgb}{0.827451,0.827451,0.827451}%
\pgfsetstrokecolor{currentstroke}%
\pgfsetdash{}{0pt}%
\pgfpathmoveto{\pgfqpoint{1.019447in}{2.455876in}}%
\pgfpathcurveto{\pgfqpoint{1.030497in}{2.455876in}}{\pgfqpoint{1.041096in}{2.460267in}}{\pgfqpoint{1.048910in}{2.468080in}}%
\pgfpathcurveto{\pgfqpoint{1.056723in}{2.475894in}}{\pgfqpoint{1.061113in}{2.486493in}}{\pgfqpoint{1.061113in}{2.497543in}}%
\pgfpathcurveto{\pgfqpoint{1.061113in}{2.508593in}}{\pgfqpoint{1.056723in}{2.519192in}}{\pgfqpoint{1.048910in}{2.527006in}}%
\pgfpathcurveto{\pgfqpoint{1.041096in}{2.534819in}}{\pgfqpoint{1.030497in}{2.539210in}}{\pgfqpoint{1.019447in}{2.539210in}}%
\pgfpathcurveto{\pgfqpoint{1.008397in}{2.539210in}}{\pgfqpoint{0.997798in}{2.534819in}}{\pgfqpoint{0.989984in}{2.527006in}}%
\pgfpathcurveto{\pgfqpoint{0.982170in}{2.519192in}}{\pgfqpoint{0.977780in}{2.508593in}}{\pgfqpoint{0.977780in}{2.497543in}}%
\pgfpathcurveto{\pgfqpoint{0.977780in}{2.486493in}}{\pgfqpoint{0.982170in}{2.475894in}}{\pgfqpoint{0.989984in}{2.468080in}}%
\pgfpathcurveto{\pgfqpoint{0.997798in}{2.460267in}}{\pgfqpoint{1.008397in}{2.455876in}}{\pgfqpoint{1.019447in}{2.455876in}}%
\pgfpathlineto{\pgfqpoint{1.019447in}{2.455876in}}%
\pgfpathclose%
\pgfusepath{stroke}%
\end{pgfscope}%
\begin{pgfscope}%
\pgfpathrectangle{\pgfqpoint{0.393053in}{0.375000in}}{\pgfqpoint{6.356833in}{5.175000in}}%
\pgfusepath{clip}%
\pgfsetbuttcap%
\pgfsetroundjoin%
\pgfsetlinewidth{1.003750pt}%
\definecolor{currentstroke}{rgb}{0.827451,0.827451,0.827451}%
\pgfsetstrokecolor{currentstroke}%
\pgfsetdash{}{0pt}%
\pgfpathmoveto{\pgfqpoint{0.435142in}{3.936498in}}%
\pgfpathcurveto{\pgfqpoint{0.446192in}{3.936498in}}{\pgfqpoint{0.456791in}{3.940888in}}{\pgfqpoint{0.464605in}{3.948702in}}%
\pgfpathcurveto{\pgfqpoint{0.472419in}{3.956516in}}{\pgfqpoint{0.476809in}{3.967115in}}{\pgfqpoint{0.476809in}{3.978165in}}%
\pgfpathcurveto{\pgfqpoint{0.476809in}{3.989215in}}{\pgfqpoint{0.472419in}{3.999814in}}{\pgfqpoint{0.464605in}{4.007627in}}%
\pgfpathcurveto{\pgfqpoint{0.456791in}{4.015441in}}{\pgfqpoint{0.446192in}{4.019831in}}{\pgfqpoint{0.435142in}{4.019831in}}%
\pgfpathcurveto{\pgfqpoint{0.424092in}{4.019831in}}{\pgfqpoint{0.413493in}{4.015441in}}{\pgfqpoint{0.405679in}{4.007627in}}%
\pgfpathcurveto{\pgfqpoint{0.397866in}{3.999814in}}{\pgfqpoint{0.393475in}{3.989215in}}{\pgfqpoint{0.393475in}{3.978165in}}%
\pgfpathcurveto{\pgfqpoint{0.393475in}{3.967115in}}{\pgfqpoint{0.397866in}{3.956516in}}{\pgfqpoint{0.405679in}{3.948702in}}%
\pgfpathcurveto{\pgfqpoint{0.413493in}{3.940888in}}{\pgfqpoint{0.424092in}{3.936498in}}{\pgfqpoint{0.435142in}{3.936498in}}%
\pgfpathlineto{\pgfqpoint{0.435142in}{3.936498in}}%
\pgfpathclose%
\pgfusepath{stroke}%
\end{pgfscope}%
\begin{pgfscope}%
\pgfpathrectangle{\pgfqpoint{0.393053in}{0.375000in}}{\pgfqpoint{6.356833in}{5.175000in}}%
\pgfusepath{clip}%
\pgfsetbuttcap%
\pgfsetroundjoin%
\pgfsetlinewidth{1.003750pt}%
\definecolor{currentstroke}{rgb}{0.827451,0.827451,0.827451}%
\pgfsetstrokecolor{currentstroke}%
\pgfsetdash{}{0pt}%
\pgfpathmoveto{\pgfqpoint{2.414235in}{1.111502in}}%
\pgfpathcurveto{\pgfqpoint{2.425285in}{1.111502in}}{\pgfqpoint{2.435884in}{1.115892in}}{\pgfqpoint{2.443698in}{1.123706in}}%
\pgfpathcurveto{\pgfqpoint{2.451511in}{1.131519in}}{\pgfqpoint{2.455902in}{1.142118in}}{\pgfqpoint{2.455902in}{1.153169in}}%
\pgfpathcurveto{\pgfqpoint{2.455902in}{1.164219in}}{\pgfqpoint{2.451511in}{1.174818in}}{\pgfqpoint{2.443698in}{1.182631in}}%
\pgfpathcurveto{\pgfqpoint{2.435884in}{1.190445in}}{\pgfqpoint{2.425285in}{1.194835in}}{\pgfqpoint{2.414235in}{1.194835in}}%
\pgfpathcurveto{\pgfqpoint{2.403185in}{1.194835in}}{\pgfqpoint{2.392586in}{1.190445in}}{\pgfqpoint{2.384772in}{1.182631in}}%
\pgfpathcurveto{\pgfqpoint{2.376959in}{1.174818in}}{\pgfqpoint{2.372568in}{1.164219in}}{\pgfqpoint{2.372568in}{1.153169in}}%
\pgfpathcurveto{\pgfqpoint{2.372568in}{1.142118in}}{\pgfqpoint{2.376959in}{1.131519in}}{\pgfqpoint{2.384772in}{1.123706in}}%
\pgfpathcurveto{\pgfqpoint{2.392586in}{1.115892in}}{\pgfqpoint{2.403185in}{1.111502in}}{\pgfqpoint{2.414235in}{1.111502in}}%
\pgfpathlineto{\pgfqpoint{2.414235in}{1.111502in}}%
\pgfpathclose%
\pgfusepath{stroke}%
\end{pgfscope}%
\begin{pgfscope}%
\pgfpathrectangle{\pgfqpoint{0.393053in}{0.375000in}}{\pgfqpoint{6.356833in}{5.175000in}}%
\pgfusepath{clip}%
\pgfsetbuttcap%
\pgfsetroundjoin%
\pgfsetlinewidth{1.003750pt}%
\definecolor{currentstroke}{rgb}{0.827451,0.827451,0.827451}%
\pgfsetstrokecolor{currentstroke}%
\pgfsetdash{}{0pt}%
\pgfpathmoveto{\pgfqpoint{4.122151in}{0.465889in}}%
\pgfpathcurveto{\pgfqpoint{4.133202in}{0.465889in}}{\pgfqpoint{4.143801in}{0.470279in}}{\pgfqpoint{4.151614in}{0.478093in}}%
\pgfpathcurveto{\pgfqpoint{4.159428in}{0.485907in}}{\pgfqpoint{4.163818in}{0.496506in}}{\pgfqpoint{4.163818in}{0.507556in}}%
\pgfpathcurveto{\pgfqpoint{4.163818in}{0.518606in}}{\pgfqpoint{4.159428in}{0.529205in}}{\pgfqpoint{4.151614in}{0.537019in}}%
\pgfpathcurveto{\pgfqpoint{4.143801in}{0.544832in}}{\pgfqpoint{4.133202in}{0.549222in}}{\pgfqpoint{4.122151in}{0.549222in}}%
\pgfpathcurveto{\pgfqpoint{4.111101in}{0.549222in}}{\pgfqpoint{4.100502in}{0.544832in}}{\pgfqpoint{4.092689in}{0.537019in}}%
\pgfpathcurveto{\pgfqpoint{4.084875in}{0.529205in}}{\pgfqpoint{4.080485in}{0.518606in}}{\pgfqpoint{4.080485in}{0.507556in}}%
\pgfpathcurveto{\pgfqpoint{4.080485in}{0.496506in}}{\pgfqpoint{4.084875in}{0.485907in}}{\pgfqpoint{4.092689in}{0.478093in}}%
\pgfpathcurveto{\pgfqpoint{4.100502in}{0.470279in}}{\pgfqpoint{4.111101in}{0.465889in}}{\pgfqpoint{4.122151in}{0.465889in}}%
\pgfpathlineto{\pgfqpoint{4.122151in}{0.465889in}}%
\pgfpathclose%
\pgfusepath{stroke}%
\end{pgfscope}%
\begin{pgfscope}%
\pgfpathrectangle{\pgfqpoint{0.393053in}{0.375000in}}{\pgfqpoint{6.356833in}{5.175000in}}%
\pgfusepath{clip}%
\pgfsetbuttcap%
\pgfsetroundjoin%
\pgfsetlinewidth{1.003750pt}%
\definecolor{currentstroke}{rgb}{0.827451,0.827451,0.827451}%
\pgfsetstrokecolor{currentstroke}%
\pgfsetdash{}{0pt}%
\pgfpathmoveto{\pgfqpoint{3.440925in}{0.635984in}}%
\pgfpathcurveto{\pgfqpoint{3.451976in}{0.635984in}}{\pgfqpoint{3.462575in}{0.640375in}}{\pgfqpoint{3.470388in}{0.648188in}}%
\pgfpathcurveto{\pgfqpoint{3.478202in}{0.656002in}}{\pgfqpoint{3.482592in}{0.666601in}}{\pgfqpoint{3.482592in}{0.677651in}}%
\pgfpathcurveto{\pgfqpoint{3.482592in}{0.688701in}}{\pgfqpoint{3.478202in}{0.699300in}}{\pgfqpoint{3.470388in}{0.707114in}}%
\pgfpathcurveto{\pgfqpoint{3.462575in}{0.714927in}}{\pgfqpoint{3.451976in}{0.719318in}}{\pgfqpoint{3.440925in}{0.719318in}}%
\pgfpathcurveto{\pgfqpoint{3.429875in}{0.719318in}}{\pgfqpoint{3.419276in}{0.714927in}}{\pgfqpoint{3.411463in}{0.707114in}}%
\pgfpathcurveto{\pgfqpoint{3.403649in}{0.699300in}}{\pgfqpoint{3.399259in}{0.688701in}}{\pgfqpoint{3.399259in}{0.677651in}}%
\pgfpathcurveto{\pgfqpoint{3.399259in}{0.666601in}}{\pgfqpoint{3.403649in}{0.656002in}}{\pgfqpoint{3.411463in}{0.648188in}}%
\pgfpathcurveto{\pgfqpoint{3.419276in}{0.640375in}}{\pgfqpoint{3.429875in}{0.635984in}}{\pgfqpoint{3.440925in}{0.635984in}}%
\pgfpathlineto{\pgfqpoint{3.440925in}{0.635984in}}%
\pgfpathclose%
\pgfusepath{stroke}%
\end{pgfscope}%
\begin{pgfscope}%
\pgfpathrectangle{\pgfqpoint{0.393053in}{0.375000in}}{\pgfqpoint{6.356833in}{5.175000in}}%
\pgfusepath{clip}%
\pgfsetbuttcap%
\pgfsetroundjoin%
\pgfsetlinewidth{1.003750pt}%
\definecolor{currentstroke}{rgb}{0.827451,0.827451,0.827451}%
\pgfsetstrokecolor{currentstroke}%
\pgfsetdash{}{0pt}%
\pgfpathmoveto{\pgfqpoint{2.288220in}{1.182911in}}%
\pgfpathcurveto{\pgfqpoint{2.299270in}{1.182911in}}{\pgfqpoint{2.309870in}{1.187302in}}{\pgfqpoint{2.317683in}{1.195115in}}%
\pgfpathcurveto{\pgfqpoint{2.325497in}{1.202929in}}{\pgfqpoint{2.329887in}{1.213528in}}{\pgfqpoint{2.329887in}{1.224578in}}%
\pgfpathcurveto{\pgfqpoint{2.329887in}{1.235628in}}{\pgfqpoint{2.325497in}{1.246227in}}{\pgfqpoint{2.317683in}{1.254041in}}%
\pgfpathcurveto{\pgfqpoint{2.309870in}{1.261854in}}{\pgfqpoint{2.299270in}{1.266245in}}{\pgfqpoint{2.288220in}{1.266245in}}%
\pgfpathcurveto{\pgfqpoint{2.277170in}{1.266245in}}{\pgfqpoint{2.266571in}{1.261854in}}{\pgfqpoint{2.258758in}{1.254041in}}%
\pgfpathcurveto{\pgfqpoint{2.250944in}{1.246227in}}{\pgfqpoint{2.246554in}{1.235628in}}{\pgfqpoint{2.246554in}{1.224578in}}%
\pgfpathcurveto{\pgfqpoint{2.246554in}{1.213528in}}{\pgfqpoint{2.250944in}{1.202929in}}{\pgfqpoint{2.258758in}{1.195115in}}%
\pgfpathcurveto{\pgfqpoint{2.266571in}{1.187302in}}{\pgfqpoint{2.277170in}{1.182911in}}{\pgfqpoint{2.288220in}{1.182911in}}%
\pgfpathlineto{\pgfqpoint{2.288220in}{1.182911in}}%
\pgfpathclose%
\pgfusepath{stroke}%
\end{pgfscope}%
\begin{pgfscope}%
\pgfpathrectangle{\pgfqpoint{0.393053in}{0.375000in}}{\pgfqpoint{6.356833in}{5.175000in}}%
\pgfusepath{clip}%
\pgfsetbuttcap%
\pgfsetroundjoin%
\pgfsetlinewidth{1.003750pt}%
\definecolor{currentstroke}{rgb}{0.827451,0.827451,0.827451}%
\pgfsetstrokecolor{currentstroke}%
\pgfsetdash{}{0pt}%
\pgfpathmoveto{\pgfqpoint{2.145018in}{1.265722in}}%
\pgfpathcurveto{\pgfqpoint{2.156069in}{1.265722in}}{\pgfqpoint{2.166668in}{1.270112in}}{\pgfqpoint{2.174481in}{1.277926in}}%
\pgfpathcurveto{\pgfqpoint{2.182295in}{1.285739in}}{\pgfqpoint{2.186685in}{1.296338in}}{\pgfqpoint{2.186685in}{1.307389in}}%
\pgfpathcurveto{\pgfqpoint{2.186685in}{1.318439in}}{\pgfqpoint{2.182295in}{1.329038in}}{\pgfqpoint{2.174481in}{1.336851in}}%
\pgfpathcurveto{\pgfqpoint{2.166668in}{1.344665in}}{\pgfqpoint{2.156069in}{1.349055in}}{\pgfqpoint{2.145018in}{1.349055in}}%
\pgfpathcurveto{\pgfqpoint{2.133968in}{1.349055in}}{\pgfqpoint{2.123369in}{1.344665in}}{\pgfqpoint{2.115556in}{1.336851in}}%
\pgfpathcurveto{\pgfqpoint{2.107742in}{1.329038in}}{\pgfqpoint{2.103352in}{1.318439in}}{\pgfqpoint{2.103352in}{1.307389in}}%
\pgfpathcurveto{\pgfqpoint{2.103352in}{1.296338in}}{\pgfqpoint{2.107742in}{1.285739in}}{\pgfqpoint{2.115556in}{1.277926in}}%
\pgfpathcurveto{\pgfqpoint{2.123369in}{1.270112in}}{\pgfqpoint{2.133968in}{1.265722in}}{\pgfqpoint{2.145018in}{1.265722in}}%
\pgfpathlineto{\pgfqpoint{2.145018in}{1.265722in}}%
\pgfpathclose%
\pgfusepath{stroke}%
\end{pgfscope}%
\begin{pgfscope}%
\pgfpathrectangle{\pgfqpoint{0.393053in}{0.375000in}}{\pgfqpoint{6.356833in}{5.175000in}}%
\pgfusepath{clip}%
\pgfsetbuttcap%
\pgfsetroundjoin%
\pgfsetlinewidth{1.003750pt}%
\definecolor{currentstroke}{rgb}{0.827451,0.827451,0.827451}%
\pgfsetstrokecolor{currentstroke}%
\pgfsetdash{}{0pt}%
\pgfpathmoveto{\pgfqpoint{0.711033in}{2.928891in}}%
\pgfpathcurveto{\pgfqpoint{0.722083in}{2.928891in}}{\pgfqpoint{0.732682in}{2.933281in}}{\pgfqpoint{0.740496in}{2.941095in}}%
\pgfpathcurveto{\pgfqpoint{0.748310in}{2.948909in}}{\pgfqpoint{0.752700in}{2.959508in}}{\pgfqpoint{0.752700in}{2.970558in}}%
\pgfpathcurveto{\pgfqpoint{0.752700in}{2.981608in}}{\pgfqpoint{0.748310in}{2.992207in}}{\pgfqpoint{0.740496in}{3.000021in}}%
\pgfpathcurveto{\pgfqpoint{0.732682in}{3.007834in}}{\pgfqpoint{0.722083in}{3.012224in}}{\pgfqpoint{0.711033in}{3.012224in}}%
\pgfpathcurveto{\pgfqpoint{0.699983in}{3.012224in}}{\pgfqpoint{0.689384in}{3.007834in}}{\pgfqpoint{0.681570in}{3.000021in}}%
\pgfpathcurveto{\pgfqpoint{0.673757in}{2.992207in}}{\pgfqpoint{0.669366in}{2.981608in}}{\pgfqpoint{0.669366in}{2.970558in}}%
\pgfpathcurveto{\pgfqpoint{0.669366in}{2.959508in}}{\pgfqpoint{0.673757in}{2.948909in}}{\pgfqpoint{0.681570in}{2.941095in}}%
\pgfpathcurveto{\pgfqpoint{0.689384in}{2.933281in}}{\pgfqpoint{0.699983in}{2.928891in}}{\pgfqpoint{0.711033in}{2.928891in}}%
\pgfpathlineto{\pgfqpoint{0.711033in}{2.928891in}}%
\pgfpathclose%
\pgfusepath{stroke}%
\end{pgfscope}%
\begin{pgfscope}%
\pgfpathrectangle{\pgfqpoint{0.393053in}{0.375000in}}{\pgfqpoint{6.356833in}{5.175000in}}%
\pgfusepath{clip}%
\pgfsetbuttcap%
\pgfsetroundjoin%
\pgfsetlinewidth{1.003750pt}%
\definecolor{currentstroke}{rgb}{0.827451,0.827451,0.827451}%
\pgfsetstrokecolor{currentstroke}%
\pgfsetdash{}{0pt}%
\pgfpathmoveto{\pgfqpoint{1.234935in}{2.069821in}}%
\pgfpathcurveto{\pgfqpoint{1.245986in}{2.069821in}}{\pgfqpoint{1.256585in}{2.074212in}}{\pgfqpoint{1.264398in}{2.082025in}}%
\pgfpathcurveto{\pgfqpoint{1.272212in}{2.089839in}}{\pgfqpoint{1.276602in}{2.100438in}}{\pgfqpoint{1.276602in}{2.111488in}}%
\pgfpathcurveto{\pgfqpoint{1.276602in}{2.122538in}}{\pgfqpoint{1.272212in}{2.133137in}}{\pgfqpoint{1.264398in}{2.140951in}}%
\pgfpathcurveto{\pgfqpoint{1.256585in}{2.148764in}}{\pgfqpoint{1.245986in}{2.153155in}}{\pgfqpoint{1.234935in}{2.153155in}}%
\pgfpathcurveto{\pgfqpoint{1.223885in}{2.153155in}}{\pgfqpoint{1.213286in}{2.148764in}}{\pgfqpoint{1.205473in}{2.140951in}}%
\pgfpathcurveto{\pgfqpoint{1.197659in}{2.133137in}}{\pgfqpoint{1.193269in}{2.122538in}}{\pgfqpoint{1.193269in}{2.111488in}}%
\pgfpathcurveto{\pgfqpoint{1.193269in}{2.100438in}}{\pgfqpoint{1.197659in}{2.089839in}}{\pgfqpoint{1.205473in}{2.082025in}}%
\pgfpathcurveto{\pgfqpoint{1.213286in}{2.074212in}}{\pgfqpoint{1.223885in}{2.069821in}}{\pgfqpoint{1.234935in}{2.069821in}}%
\pgfpathlineto{\pgfqpoint{1.234935in}{2.069821in}}%
\pgfpathclose%
\pgfusepath{stroke}%
\end{pgfscope}%
\begin{pgfscope}%
\pgfpathrectangle{\pgfqpoint{0.393053in}{0.375000in}}{\pgfqpoint{6.356833in}{5.175000in}}%
\pgfusepath{clip}%
\pgfsetbuttcap%
\pgfsetroundjoin%
\pgfsetlinewidth{1.003750pt}%
\definecolor{currentstroke}{rgb}{0.827451,0.827451,0.827451}%
\pgfsetstrokecolor{currentstroke}%
\pgfsetdash{}{0pt}%
\pgfpathmoveto{\pgfqpoint{5.144050in}{0.350542in}}%
\pgfpathcurveto{\pgfqpoint{5.155100in}{0.350542in}}{\pgfqpoint{5.165699in}{0.354932in}}{\pgfqpoint{5.173513in}{0.362746in}}%
\pgfpathcurveto{\pgfqpoint{5.181327in}{0.370560in}}{\pgfqpoint{5.185717in}{0.381159in}}{\pgfqpoint{5.185717in}{0.392209in}}%
\pgfpathcurveto{\pgfqpoint{5.185717in}{0.403259in}}{\pgfqpoint{5.181327in}{0.413858in}}{\pgfqpoint{5.173513in}{0.421672in}}%
\pgfpathcurveto{\pgfqpoint{5.165699in}{0.429485in}}{\pgfqpoint{5.155100in}{0.433875in}}{\pgfqpoint{5.144050in}{0.433875in}}%
\pgfpathcurveto{\pgfqpoint{5.133000in}{0.433875in}}{\pgfqpoint{5.122401in}{0.429485in}}{\pgfqpoint{5.114587in}{0.421672in}}%
\pgfpathcurveto{\pgfqpoint{5.106774in}{0.413858in}}{\pgfqpoint{5.102383in}{0.403259in}}{\pgfqpoint{5.102383in}{0.392209in}}%
\pgfpathcurveto{\pgfqpoint{5.102383in}{0.381159in}}{\pgfqpoint{5.106774in}{0.370560in}}{\pgfqpoint{5.114587in}{0.362746in}}%
\pgfpathcurveto{\pgfqpoint{5.122401in}{0.354932in}}{\pgfqpoint{5.133000in}{0.350542in}}{\pgfqpoint{5.144050in}{0.350542in}}%
\pgfusepath{stroke}%
\end{pgfscope}%
\begin{pgfscope}%
\pgfpathrectangle{\pgfqpoint{0.393053in}{0.375000in}}{\pgfqpoint{6.356833in}{5.175000in}}%
\pgfusepath{clip}%
\pgfsetbuttcap%
\pgfsetroundjoin%
\pgfsetlinewidth{1.003750pt}%
\definecolor{currentstroke}{rgb}{0.827451,0.827451,0.827451}%
\pgfsetstrokecolor{currentstroke}%
\pgfsetdash{}{0pt}%
\pgfpathmoveto{\pgfqpoint{0.546206in}{3.392492in}}%
\pgfpathcurveto{\pgfqpoint{0.557256in}{3.392492in}}{\pgfqpoint{0.567855in}{3.396882in}}{\pgfqpoint{0.575669in}{3.404696in}}%
\pgfpathcurveto{\pgfqpoint{0.583483in}{3.412509in}}{\pgfqpoint{0.587873in}{3.423108in}}{\pgfqpoint{0.587873in}{3.434159in}}%
\pgfpathcurveto{\pgfqpoint{0.587873in}{3.445209in}}{\pgfqpoint{0.583483in}{3.455808in}}{\pgfqpoint{0.575669in}{3.463621in}}%
\pgfpathcurveto{\pgfqpoint{0.567855in}{3.471435in}}{\pgfqpoint{0.557256in}{3.475825in}}{\pgfqpoint{0.546206in}{3.475825in}}%
\pgfpathcurveto{\pgfqpoint{0.535156in}{3.475825in}}{\pgfqpoint{0.524557in}{3.471435in}}{\pgfqpoint{0.516744in}{3.463621in}}%
\pgfpathcurveto{\pgfqpoint{0.508930in}{3.455808in}}{\pgfqpoint{0.504540in}{3.445209in}}{\pgfqpoint{0.504540in}{3.434159in}}%
\pgfpathcurveto{\pgfqpoint{0.504540in}{3.423108in}}{\pgfqpoint{0.508930in}{3.412509in}}{\pgfqpoint{0.516744in}{3.404696in}}%
\pgfpathcurveto{\pgfqpoint{0.524557in}{3.396882in}}{\pgfqpoint{0.535156in}{3.392492in}}{\pgfqpoint{0.546206in}{3.392492in}}%
\pgfpathlineto{\pgfqpoint{0.546206in}{3.392492in}}%
\pgfpathclose%
\pgfusepath{stroke}%
\end{pgfscope}%
\begin{pgfscope}%
\pgfpathrectangle{\pgfqpoint{0.393053in}{0.375000in}}{\pgfqpoint{6.356833in}{5.175000in}}%
\pgfusepath{clip}%
\pgfsetbuttcap%
\pgfsetroundjoin%
\pgfsetlinewidth{1.003750pt}%
\definecolor{currentstroke}{rgb}{0.827451,0.827451,0.827451}%
\pgfsetstrokecolor{currentstroke}%
\pgfsetdash{}{0pt}%
\pgfpathmoveto{\pgfqpoint{4.310929in}{0.438129in}}%
\pgfpathcurveto{\pgfqpoint{4.321979in}{0.438129in}}{\pgfqpoint{4.332578in}{0.442519in}}{\pgfqpoint{4.340392in}{0.450333in}}%
\pgfpathcurveto{\pgfqpoint{4.348205in}{0.458147in}}{\pgfqpoint{4.352595in}{0.468746in}}{\pgfqpoint{4.352595in}{0.479796in}}%
\pgfpathcurveto{\pgfqpoint{4.352595in}{0.490846in}}{\pgfqpoint{4.348205in}{0.501445in}}{\pgfqpoint{4.340392in}{0.509258in}}%
\pgfpathcurveto{\pgfqpoint{4.332578in}{0.517072in}}{\pgfqpoint{4.321979in}{0.521462in}}{\pgfqpoint{4.310929in}{0.521462in}}%
\pgfpathcurveto{\pgfqpoint{4.299879in}{0.521462in}}{\pgfqpoint{4.289280in}{0.517072in}}{\pgfqpoint{4.281466in}{0.509258in}}%
\pgfpathcurveto{\pgfqpoint{4.273652in}{0.501445in}}{\pgfqpoint{4.269262in}{0.490846in}}{\pgfqpoint{4.269262in}{0.479796in}}%
\pgfpathcurveto{\pgfqpoint{4.269262in}{0.468746in}}{\pgfqpoint{4.273652in}{0.458147in}}{\pgfqpoint{4.281466in}{0.450333in}}%
\pgfpathcurveto{\pgfqpoint{4.289280in}{0.442519in}}{\pgfqpoint{4.299879in}{0.438129in}}{\pgfqpoint{4.310929in}{0.438129in}}%
\pgfpathlineto{\pgfqpoint{4.310929in}{0.438129in}}%
\pgfpathclose%
\pgfusepath{stroke}%
\end{pgfscope}%
\begin{pgfscope}%
\pgfpathrectangle{\pgfqpoint{0.393053in}{0.375000in}}{\pgfqpoint{6.356833in}{5.175000in}}%
\pgfusepath{clip}%
\pgfsetbuttcap%
\pgfsetroundjoin%
\pgfsetlinewidth{1.003750pt}%
\definecolor{currentstroke}{rgb}{0.827451,0.827451,0.827451}%
\pgfsetstrokecolor{currentstroke}%
\pgfsetdash{}{0pt}%
\pgfpathmoveto{\pgfqpoint{1.545409in}{1.760911in}}%
\pgfpathcurveto{\pgfqpoint{1.556460in}{1.760911in}}{\pgfqpoint{1.567059in}{1.765301in}}{\pgfqpoint{1.574872in}{1.773115in}}%
\pgfpathcurveto{\pgfqpoint{1.582686in}{1.780928in}}{\pgfqpoint{1.587076in}{1.791527in}}{\pgfqpoint{1.587076in}{1.802578in}}%
\pgfpathcurveto{\pgfqpoint{1.587076in}{1.813628in}}{\pgfqpoint{1.582686in}{1.824227in}}{\pgfqpoint{1.574872in}{1.832040in}}%
\pgfpathcurveto{\pgfqpoint{1.567059in}{1.839854in}}{\pgfqpoint{1.556460in}{1.844244in}}{\pgfqpoint{1.545409in}{1.844244in}}%
\pgfpathcurveto{\pgfqpoint{1.534359in}{1.844244in}}{\pgfqpoint{1.523760in}{1.839854in}}{\pgfqpoint{1.515947in}{1.832040in}}%
\pgfpathcurveto{\pgfqpoint{1.508133in}{1.824227in}}{\pgfqpoint{1.503743in}{1.813628in}}{\pgfqpoint{1.503743in}{1.802578in}}%
\pgfpathcurveto{\pgfqpoint{1.503743in}{1.791527in}}{\pgfqpoint{1.508133in}{1.780928in}}{\pgfqpoint{1.515947in}{1.773115in}}%
\pgfpathcurveto{\pgfqpoint{1.523760in}{1.765301in}}{\pgfqpoint{1.534359in}{1.760911in}}{\pgfqpoint{1.545409in}{1.760911in}}%
\pgfpathlineto{\pgfqpoint{1.545409in}{1.760911in}}%
\pgfpathclose%
\pgfusepath{stroke}%
\end{pgfscope}%
\begin{pgfscope}%
\pgfpathrectangle{\pgfqpoint{0.393053in}{0.375000in}}{\pgfqpoint{6.356833in}{5.175000in}}%
\pgfusepath{clip}%
\pgfsetbuttcap%
\pgfsetroundjoin%
\pgfsetlinewidth{1.003750pt}%
\definecolor{currentstroke}{rgb}{0.827451,0.827451,0.827451}%
\pgfsetstrokecolor{currentstroke}%
\pgfsetdash{}{0pt}%
\pgfpathmoveto{\pgfqpoint{0.494809in}{3.599260in}}%
\pgfpathcurveto{\pgfqpoint{0.505859in}{3.599260in}}{\pgfqpoint{0.516458in}{3.603650in}}{\pgfqpoint{0.524272in}{3.611464in}}%
\pgfpathcurveto{\pgfqpoint{0.532085in}{3.619278in}}{\pgfqpoint{0.536476in}{3.629877in}}{\pgfqpoint{0.536476in}{3.640927in}}%
\pgfpathcurveto{\pgfqpoint{0.536476in}{3.651977in}}{\pgfqpoint{0.532085in}{3.662576in}}{\pgfqpoint{0.524272in}{3.670390in}}%
\pgfpathcurveto{\pgfqpoint{0.516458in}{3.678203in}}{\pgfqpoint{0.505859in}{3.682594in}}{\pgfqpoint{0.494809in}{3.682594in}}%
\pgfpathcurveto{\pgfqpoint{0.483759in}{3.682594in}}{\pgfqpoint{0.473160in}{3.678203in}}{\pgfqpoint{0.465346in}{3.670390in}}%
\pgfpathcurveto{\pgfqpoint{0.457533in}{3.662576in}}{\pgfqpoint{0.453142in}{3.651977in}}{\pgfqpoint{0.453142in}{3.640927in}}%
\pgfpathcurveto{\pgfqpoint{0.453142in}{3.629877in}}{\pgfqpoint{0.457533in}{3.619278in}}{\pgfqpoint{0.465346in}{3.611464in}}%
\pgfpathcurveto{\pgfqpoint{0.473160in}{3.603650in}}{\pgfqpoint{0.483759in}{3.599260in}}{\pgfqpoint{0.494809in}{3.599260in}}%
\pgfpathlineto{\pgfqpoint{0.494809in}{3.599260in}}%
\pgfpathclose%
\pgfusepath{stroke}%
\end{pgfscope}%
\begin{pgfscope}%
\pgfpathrectangle{\pgfqpoint{0.393053in}{0.375000in}}{\pgfqpoint{6.356833in}{5.175000in}}%
\pgfusepath{clip}%
\pgfsetbuttcap%
\pgfsetroundjoin%
\pgfsetlinewidth{1.003750pt}%
\definecolor{currentstroke}{rgb}{0.827451,0.827451,0.827451}%
\pgfsetstrokecolor{currentstroke}%
\pgfsetdash{}{0pt}%
\pgfpathmoveto{\pgfqpoint{1.770754in}{1.556106in}}%
\pgfpathcurveto{\pgfqpoint{1.781805in}{1.556106in}}{\pgfqpoint{1.792404in}{1.560496in}}{\pgfqpoint{1.800217in}{1.568310in}}%
\pgfpathcurveto{\pgfqpoint{1.808031in}{1.576123in}}{\pgfqpoint{1.812421in}{1.586722in}}{\pgfqpoint{1.812421in}{1.597772in}}%
\pgfpathcurveto{\pgfqpoint{1.812421in}{1.608823in}}{\pgfqpoint{1.808031in}{1.619422in}}{\pgfqpoint{1.800217in}{1.627235in}}%
\pgfpathcurveto{\pgfqpoint{1.792404in}{1.635049in}}{\pgfqpoint{1.781805in}{1.639439in}}{\pgfqpoint{1.770754in}{1.639439in}}%
\pgfpathcurveto{\pgfqpoint{1.759704in}{1.639439in}}{\pgfqpoint{1.749105in}{1.635049in}}{\pgfqpoint{1.741292in}{1.627235in}}%
\pgfpathcurveto{\pgfqpoint{1.733478in}{1.619422in}}{\pgfqpoint{1.729088in}{1.608823in}}{\pgfqpoint{1.729088in}{1.597772in}}%
\pgfpathcurveto{\pgfqpoint{1.729088in}{1.586722in}}{\pgfqpoint{1.733478in}{1.576123in}}{\pgfqpoint{1.741292in}{1.568310in}}%
\pgfpathcurveto{\pgfqpoint{1.749105in}{1.560496in}}{\pgfqpoint{1.759704in}{1.556106in}}{\pgfqpoint{1.770754in}{1.556106in}}%
\pgfpathlineto{\pgfqpoint{1.770754in}{1.556106in}}%
\pgfpathclose%
\pgfusepath{stroke}%
\end{pgfscope}%
\begin{pgfscope}%
\pgfpathrectangle{\pgfqpoint{0.393053in}{0.375000in}}{\pgfqpoint{6.356833in}{5.175000in}}%
\pgfusepath{clip}%
\pgfsetbuttcap%
\pgfsetroundjoin%
\pgfsetlinewidth{1.003750pt}%
\definecolor{currentstroke}{rgb}{0.827451,0.827451,0.827451}%
\pgfsetstrokecolor{currentstroke}%
\pgfsetdash{}{0pt}%
\pgfpathmoveto{\pgfqpoint{0.416019in}{4.146952in}}%
\pgfpathcurveto{\pgfqpoint{0.427069in}{4.146952in}}{\pgfqpoint{0.437668in}{4.151342in}}{\pgfqpoint{0.445482in}{4.159156in}}%
\pgfpathcurveto{\pgfqpoint{0.453296in}{4.166969in}}{\pgfqpoint{0.457686in}{4.177568in}}{\pgfqpoint{0.457686in}{4.188619in}}%
\pgfpathcurveto{\pgfqpoint{0.457686in}{4.199669in}}{\pgfqpoint{0.453296in}{4.210268in}}{\pgfqpoint{0.445482in}{4.218081in}}%
\pgfpathcurveto{\pgfqpoint{0.437668in}{4.225895in}}{\pgfqpoint{0.427069in}{4.230285in}}{\pgfqpoint{0.416019in}{4.230285in}}%
\pgfpathcurveto{\pgfqpoint{0.404969in}{4.230285in}}{\pgfqpoint{0.394370in}{4.225895in}}{\pgfqpoint{0.386556in}{4.218081in}}%
\pgfpathcurveto{\pgfqpoint{0.378743in}{4.210268in}}{\pgfqpoint{0.374352in}{4.199669in}}{\pgfqpoint{0.374352in}{4.188619in}}%
\pgfpathcurveto{\pgfqpoint{0.374352in}{4.177568in}}{\pgfqpoint{0.378743in}{4.166969in}}{\pgfqpoint{0.386556in}{4.159156in}}%
\pgfpathcurveto{\pgfqpoint{0.394370in}{4.151342in}}{\pgfqpoint{0.404969in}{4.146952in}}{\pgfqpoint{0.416019in}{4.146952in}}%
\pgfpathlineto{\pgfqpoint{0.416019in}{4.146952in}}%
\pgfpathclose%
\pgfusepath{stroke}%
\end{pgfscope}%
\begin{pgfscope}%
\pgfpathrectangle{\pgfqpoint{0.393053in}{0.375000in}}{\pgfqpoint{6.356833in}{5.175000in}}%
\pgfusepath{clip}%
\pgfsetbuttcap%
\pgfsetroundjoin%
\pgfsetlinewidth{1.003750pt}%
\definecolor{currentstroke}{rgb}{0.827451,0.827451,0.827451}%
\pgfsetstrokecolor{currentstroke}%
\pgfsetdash{}{0pt}%
\pgfpathmoveto{\pgfqpoint{1.135663in}{2.200937in}}%
\pgfpathcurveto{\pgfqpoint{1.146713in}{2.200937in}}{\pgfqpoint{1.157312in}{2.205328in}}{\pgfqpoint{1.165126in}{2.213141in}}%
\pgfpathcurveto{\pgfqpoint{1.172939in}{2.220955in}}{\pgfqpoint{1.177330in}{2.231554in}}{\pgfqpoint{1.177330in}{2.242604in}}%
\pgfpathcurveto{\pgfqpoint{1.177330in}{2.253654in}}{\pgfqpoint{1.172939in}{2.264253in}}{\pgfqpoint{1.165126in}{2.272067in}}%
\pgfpathcurveto{\pgfqpoint{1.157312in}{2.279880in}}{\pgfqpoint{1.146713in}{2.284271in}}{\pgfqpoint{1.135663in}{2.284271in}}%
\pgfpathcurveto{\pgfqpoint{1.124613in}{2.284271in}}{\pgfqpoint{1.114014in}{2.279880in}}{\pgfqpoint{1.106200in}{2.272067in}}%
\pgfpathcurveto{\pgfqpoint{1.098387in}{2.264253in}}{\pgfqpoint{1.093996in}{2.253654in}}{\pgfqpoint{1.093996in}{2.242604in}}%
\pgfpathcurveto{\pgfqpoint{1.093996in}{2.231554in}}{\pgfqpoint{1.098387in}{2.220955in}}{\pgfqpoint{1.106200in}{2.213141in}}%
\pgfpathcurveto{\pgfqpoint{1.114014in}{2.205328in}}{\pgfqpoint{1.124613in}{2.200937in}}{\pgfqpoint{1.135663in}{2.200937in}}%
\pgfpathlineto{\pgfqpoint{1.135663in}{2.200937in}}%
\pgfpathclose%
\pgfusepath{stroke}%
\end{pgfscope}%
\begin{pgfscope}%
\pgfpathrectangle{\pgfqpoint{0.393053in}{0.375000in}}{\pgfqpoint{6.356833in}{5.175000in}}%
\pgfusepath{clip}%
\pgfsetbuttcap%
\pgfsetroundjoin%
\pgfsetlinewidth{1.003750pt}%
\definecolor{currentstroke}{rgb}{0.827451,0.827451,0.827451}%
\pgfsetstrokecolor{currentstroke}%
\pgfsetdash{}{0pt}%
\pgfpathmoveto{\pgfqpoint{4.348472in}{0.428439in}}%
\pgfpathcurveto{\pgfqpoint{4.359523in}{0.428439in}}{\pgfqpoint{4.370122in}{0.432829in}}{\pgfqpoint{4.377935in}{0.440643in}}%
\pgfpathcurveto{\pgfqpoint{4.385749in}{0.448456in}}{\pgfqpoint{4.390139in}{0.459056in}}{\pgfqpoint{4.390139in}{0.470106in}}%
\pgfpathcurveto{\pgfqpoint{4.390139in}{0.481156in}}{\pgfqpoint{4.385749in}{0.491755in}}{\pgfqpoint{4.377935in}{0.499568in}}%
\pgfpathcurveto{\pgfqpoint{4.370122in}{0.507382in}}{\pgfqpoint{4.359523in}{0.511772in}}{\pgfqpoint{4.348472in}{0.511772in}}%
\pgfpathcurveto{\pgfqpoint{4.337422in}{0.511772in}}{\pgfqpoint{4.326823in}{0.507382in}}{\pgfqpoint{4.319010in}{0.499568in}}%
\pgfpathcurveto{\pgfqpoint{4.311196in}{0.491755in}}{\pgfqpoint{4.306806in}{0.481156in}}{\pgfqpoint{4.306806in}{0.470106in}}%
\pgfpathcurveto{\pgfqpoint{4.306806in}{0.459056in}}{\pgfqpoint{4.311196in}{0.448456in}}{\pgfqpoint{4.319010in}{0.440643in}}%
\pgfpathcurveto{\pgfqpoint{4.326823in}{0.432829in}}{\pgfqpoint{4.337422in}{0.428439in}}{\pgfqpoint{4.348472in}{0.428439in}}%
\pgfpathlineto{\pgfqpoint{4.348472in}{0.428439in}}%
\pgfpathclose%
\pgfusepath{stroke}%
\end{pgfscope}%
\begin{pgfscope}%
\pgfpathrectangle{\pgfqpoint{0.393053in}{0.375000in}}{\pgfqpoint{6.356833in}{5.175000in}}%
\pgfusepath{clip}%
\pgfsetbuttcap%
\pgfsetroundjoin%
\pgfsetlinewidth{1.003750pt}%
\definecolor{currentstroke}{rgb}{0.827451,0.827451,0.827451}%
\pgfsetstrokecolor{currentstroke}%
\pgfsetdash{}{0pt}%
\pgfpathmoveto{\pgfqpoint{0.426530in}{4.012842in}}%
\pgfpathcurveto{\pgfqpoint{0.437580in}{4.012842in}}{\pgfqpoint{0.448179in}{4.017232in}}{\pgfqpoint{0.455992in}{4.025046in}}%
\pgfpathcurveto{\pgfqpoint{0.463806in}{4.032860in}}{\pgfqpoint{0.468196in}{4.043459in}}{\pgfqpoint{0.468196in}{4.054509in}}%
\pgfpathcurveto{\pgfqpoint{0.468196in}{4.065559in}}{\pgfqpoint{0.463806in}{4.076158in}}{\pgfqpoint{0.455992in}{4.083972in}}%
\pgfpathcurveto{\pgfqpoint{0.448179in}{4.091785in}}{\pgfqpoint{0.437580in}{4.096175in}}{\pgfqpoint{0.426530in}{4.096175in}}%
\pgfpathcurveto{\pgfqpoint{0.415479in}{4.096175in}}{\pgfqpoint{0.404880in}{4.091785in}}{\pgfqpoint{0.397067in}{4.083972in}}%
\pgfpathcurveto{\pgfqpoint{0.389253in}{4.076158in}}{\pgfqpoint{0.384863in}{4.065559in}}{\pgfqpoint{0.384863in}{4.054509in}}%
\pgfpathcurveto{\pgfqpoint{0.384863in}{4.043459in}}{\pgfqpoint{0.389253in}{4.032860in}}{\pgfqpoint{0.397067in}{4.025046in}}%
\pgfpathcurveto{\pgfqpoint{0.404880in}{4.017232in}}{\pgfqpoint{0.415479in}{4.012842in}}{\pgfqpoint{0.426530in}{4.012842in}}%
\pgfpathlineto{\pgfqpoint{0.426530in}{4.012842in}}%
\pgfpathclose%
\pgfusepath{stroke}%
\end{pgfscope}%
\begin{pgfscope}%
\pgfpathrectangle{\pgfqpoint{0.393053in}{0.375000in}}{\pgfqpoint{6.356833in}{5.175000in}}%
\pgfusepath{clip}%
\pgfsetbuttcap%
\pgfsetroundjoin%
\pgfsetlinewidth{1.003750pt}%
\definecolor{currentstroke}{rgb}{0.827451,0.827451,0.827451}%
\pgfsetstrokecolor{currentstroke}%
\pgfsetdash{}{0pt}%
\pgfpathmoveto{\pgfqpoint{1.768437in}{1.558678in}}%
\pgfpathcurveto{\pgfqpoint{1.779488in}{1.558678in}}{\pgfqpoint{1.790087in}{1.563068in}}{\pgfqpoint{1.797900in}{1.570882in}}%
\pgfpathcurveto{\pgfqpoint{1.805714in}{1.578696in}}{\pgfqpoint{1.810104in}{1.589295in}}{\pgfqpoint{1.810104in}{1.600345in}}%
\pgfpathcurveto{\pgfqpoint{1.810104in}{1.611395in}}{\pgfqpoint{1.805714in}{1.621994in}}{\pgfqpoint{1.797900in}{1.629807in}}%
\pgfpathcurveto{\pgfqpoint{1.790087in}{1.637621in}}{\pgfqpoint{1.779488in}{1.642011in}}{\pgfqpoint{1.768437in}{1.642011in}}%
\pgfpathcurveto{\pgfqpoint{1.757387in}{1.642011in}}{\pgfqpoint{1.746788in}{1.637621in}}{\pgfqpoint{1.738975in}{1.629807in}}%
\pgfpathcurveto{\pgfqpoint{1.731161in}{1.621994in}}{\pgfqpoint{1.726771in}{1.611395in}}{\pgfqpoint{1.726771in}{1.600345in}}%
\pgfpathcurveto{\pgfqpoint{1.726771in}{1.589295in}}{\pgfqpoint{1.731161in}{1.578696in}}{\pgfqpoint{1.738975in}{1.570882in}}%
\pgfpathcurveto{\pgfqpoint{1.746788in}{1.563068in}}{\pgfqpoint{1.757387in}{1.558678in}}{\pgfqpoint{1.768437in}{1.558678in}}%
\pgfpathlineto{\pgfqpoint{1.768437in}{1.558678in}}%
\pgfpathclose%
\pgfusepath{stroke}%
\end{pgfscope}%
\begin{pgfscope}%
\pgfpathrectangle{\pgfqpoint{0.393053in}{0.375000in}}{\pgfqpoint{6.356833in}{5.175000in}}%
\pgfusepath{clip}%
\pgfsetbuttcap%
\pgfsetroundjoin%
\pgfsetlinewidth{1.003750pt}%
\definecolor{currentstroke}{rgb}{0.827451,0.827451,0.827451}%
\pgfsetstrokecolor{currentstroke}%
\pgfsetdash{}{0pt}%
\pgfpathmoveto{\pgfqpoint{3.331975in}{0.681897in}}%
\pgfpathcurveto{\pgfqpoint{3.343025in}{0.681897in}}{\pgfqpoint{3.353624in}{0.686287in}}{\pgfqpoint{3.361437in}{0.694101in}}%
\pgfpathcurveto{\pgfqpoint{3.369251in}{0.701915in}}{\pgfqpoint{3.373641in}{0.712514in}}{\pgfqpoint{3.373641in}{0.723564in}}%
\pgfpathcurveto{\pgfqpoint{3.373641in}{0.734614in}}{\pgfqpoint{3.369251in}{0.745213in}}{\pgfqpoint{3.361437in}{0.753027in}}%
\pgfpathcurveto{\pgfqpoint{3.353624in}{0.760840in}}{\pgfqpoint{3.343025in}{0.765230in}}{\pgfqpoint{3.331975in}{0.765230in}}%
\pgfpathcurveto{\pgfqpoint{3.320924in}{0.765230in}}{\pgfqpoint{3.310325in}{0.760840in}}{\pgfqpoint{3.302512in}{0.753027in}}%
\pgfpathcurveto{\pgfqpoint{3.294698in}{0.745213in}}{\pgfqpoint{3.290308in}{0.734614in}}{\pgfqpoint{3.290308in}{0.723564in}}%
\pgfpathcurveto{\pgfqpoint{3.290308in}{0.712514in}}{\pgfqpoint{3.294698in}{0.701915in}}{\pgfqpoint{3.302512in}{0.694101in}}%
\pgfpathcurveto{\pgfqpoint{3.310325in}{0.686287in}}{\pgfqpoint{3.320924in}{0.681897in}}{\pgfqpoint{3.331975in}{0.681897in}}%
\pgfpathlineto{\pgfqpoint{3.331975in}{0.681897in}}%
\pgfpathclose%
\pgfusepath{stroke}%
\end{pgfscope}%
\begin{pgfscope}%
\pgfpathrectangle{\pgfqpoint{0.393053in}{0.375000in}}{\pgfqpoint{6.356833in}{5.175000in}}%
\pgfusepath{clip}%
\pgfsetbuttcap%
\pgfsetroundjoin%
\pgfsetlinewidth{1.003750pt}%
\definecolor{currentstroke}{rgb}{0.827451,0.827451,0.827451}%
\pgfsetstrokecolor{currentstroke}%
\pgfsetdash{}{0pt}%
\pgfpathmoveto{\pgfqpoint{2.986480in}{0.818757in}}%
\pgfpathcurveto{\pgfqpoint{2.997531in}{0.818757in}}{\pgfqpoint{3.008130in}{0.823147in}}{\pgfqpoint{3.015943in}{0.830961in}}%
\pgfpathcurveto{\pgfqpoint{3.023757in}{0.838775in}}{\pgfqpoint{3.028147in}{0.849374in}}{\pgfqpoint{3.028147in}{0.860424in}}%
\pgfpathcurveto{\pgfqpoint{3.028147in}{0.871474in}}{\pgfqpoint{3.023757in}{0.882073in}}{\pgfqpoint{3.015943in}{0.889887in}}%
\pgfpathcurveto{\pgfqpoint{3.008130in}{0.897700in}}{\pgfqpoint{2.997531in}{0.902090in}}{\pgfqpoint{2.986480in}{0.902090in}}%
\pgfpathcurveto{\pgfqpoint{2.975430in}{0.902090in}}{\pgfqpoint{2.964831in}{0.897700in}}{\pgfqpoint{2.957018in}{0.889887in}}%
\pgfpathcurveto{\pgfqpoint{2.949204in}{0.882073in}}{\pgfqpoint{2.944814in}{0.871474in}}{\pgfqpoint{2.944814in}{0.860424in}}%
\pgfpathcurveto{\pgfqpoint{2.944814in}{0.849374in}}{\pgfqpoint{2.949204in}{0.838775in}}{\pgfqpoint{2.957018in}{0.830961in}}%
\pgfpathcurveto{\pgfqpoint{2.964831in}{0.823147in}}{\pgfqpoint{2.975430in}{0.818757in}}{\pgfqpoint{2.986480in}{0.818757in}}%
\pgfpathlineto{\pgfqpoint{2.986480in}{0.818757in}}%
\pgfpathclose%
\pgfusepath{stroke}%
\end{pgfscope}%
\begin{pgfscope}%
\pgfpathrectangle{\pgfqpoint{0.393053in}{0.375000in}}{\pgfqpoint{6.356833in}{5.175000in}}%
\pgfusepath{clip}%
\pgfsetbuttcap%
\pgfsetroundjoin%
\pgfsetlinewidth{1.003750pt}%
\definecolor{currentstroke}{rgb}{0.827451,0.827451,0.827451}%
\pgfsetstrokecolor{currentstroke}%
\pgfsetdash{}{0pt}%
\pgfpathmoveto{\pgfqpoint{1.202031in}{2.138172in}}%
\pgfpathcurveto{\pgfqpoint{1.213081in}{2.138172in}}{\pgfqpoint{1.223680in}{2.142562in}}{\pgfqpoint{1.231494in}{2.150376in}}%
\pgfpathcurveto{\pgfqpoint{1.239308in}{2.158190in}}{\pgfqpoint{1.243698in}{2.168789in}}{\pgfqpoint{1.243698in}{2.179839in}}%
\pgfpathcurveto{\pgfqpoint{1.243698in}{2.190889in}}{\pgfqpoint{1.239308in}{2.201488in}}{\pgfqpoint{1.231494in}{2.209301in}}%
\pgfpathcurveto{\pgfqpoint{1.223680in}{2.217115in}}{\pgfqpoint{1.213081in}{2.221505in}}{\pgfqpoint{1.202031in}{2.221505in}}%
\pgfpathcurveto{\pgfqpoint{1.190981in}{2.221505in}}{\pgfqpoint{1.180382in}{2.217115in}}{\pgfqpoint{1.172568in}{2.209301in}}%
\pgfpathcurveto{\pgfqpoint{1.164755in}{2.201488in}}{\pgfqpoint{1.160364in}{2.190889in}}{\pgfqpoint{1.160364in}{2.179839in}}%
\pgfpathcurveto{\pgfqpoint{1.160364in}{2.168789in}}{\pgfqpoint{1.164755in}{2.158190in}}{\pgfqpoint{1.172568in}{2.150376in}}%
\pgfpathcurveto{\pgfqpoint{1.180382in}{2.142562in}}{\pgfqpoint{1.190981in}{2.138172in}}{\pgfqpoint{1.202031in}{2.138172in}}%
\pgfpathlineto{\pgfqpoint{1.202031in}{2.138172in}}%
\pgfpathclose%
\pgfusepath{stroke}%
\end{pgfscope}%
\begin{pgfscope}%
\pgfpathrectangle{\pgfqpoint{0.393053in}{0.375000in}}{\pgfqpoint{6.356833in}{5.175000in}}%
\pgfusepath{clip}%
\pgfsetbuttcap%
\pgfsetroundjoin%
\pgfsetlinewidth{1.003750pt}%
\definecolor{currentstroke}{rgb}{0.827451,0.827451,0.827451}%
\pgfsetstrokecolor{currentstroke}%
\pgfsetdash{}{0pt}%
\pgfpathmoveto{\pgfqpoint{0.532880in}{3.450148in}}%
\pgfpathcurveto{\pgfqpoint{0.543930in}{3.450148in}}{\pgfqpoint{0.554529in}{3.454539in}}{\pgfqpoint{0.562342in}{3.462352in}}%
\pgfpathcurveto{\pgfqpoint{0.570156in}{3.470166in}}{\pgfqpoint{0.574546in}{3.480765in}}{\pgfqpoint{0.574546in}{3.491815in}}%
\pgfpathcurveto{\pgfqpoint{0.574546in}{3.502865in}}{\pgfqpoint{0.570156in}{3.513464in}}{\pgfqpoint{0.562342in}{3.521278in}}%
\pgfpathcurveto{\pgfqpoint{0.554529in}{3.529091in}}{\pgfqpoint{0.543930in}{3.533482in}}{\pgfqpoint{0.532880in}{3.533482in}}%
\pgfpathcurveto{\pgfqpoint{0.521829in}{3.533482in}}{\pgfqpoint{0.511230in}{3.529091in}}{\pgfqpoint{0.503417in}{3.521278in}}%
\pgfpathcurveto{\pgfqpoint{0.495603in}{3.513464in}}{\pgfqpoint{0.491213in}{3.502865in}}{\pgfqpoint{0.491213in}{3.491815in}}%
\pgfpathcurveto{\pgfqpoint{0.491213in}{3.480765in}}{\pgfqpoint{0.495603in}{3.470166in}}{\pgfqpoint{0.503417in}{3.462352in}}%
\pgfpathcurveto{\pgfqpoint{0.511230in}{3.454539in}}{\pgfqpoint{0.521829in}{3.450148in}}{\pgfqpoint{0.532880in}{3.450148in}}%
\pgfpathlineto{\pgfqpoint{0.532880in}{3.450148in}}%
\pgfpathclose%
\pgfusepath{stroke}%
\end{pgfscope}%
\begin{pgfscope}%
\pgfpathrectangle{\pgfqpoint{0.393053in}{0.375000in}}{\pgfqpoint{6.356833in}{5.175000in}}%
\pgfusepath{clip}%
\pgfsetbuttcap%
\pgfsetroundjoin%
\pgfsetlinewidth{1.003750pt}%
\definecolor{currentstroke}{rgb}{0.827451,0.827451,0.827451}%
\pgfsetstrokecolor{currentstroke}%
\pgfsetdash{}{0pt}%
\pgfpathmoveto{\pgfqpoint{1.018989in}{2.457000in}}%
\pgfpathcurveto{\pgfqpoint{1.030039in}{2.457000in}}{\pgfqpoint{1.040638in}{2.461390in}}{\pgfqpoint{1.048452in}{2.469204in}}%
\pgfpathcurveto{\pgfqpoint{1.056265in}{2.477017in}}{\pgfqpoint{1.060655in}{2.487616in}}{\pgfqpoint{1.060655in}{2.498666in}}%
\pgfpathcurveto{\pgfqpoint{1.060655in}{2.509717in}}{\pgfqpoint{1.056265in}{2.520316in}}{\pgfqpoint{1.048452in}{2.528129in}}%
\pgfpathcurveto{\pgfqpoint{1.040638in}{2.535943in}}{\pgfqpoint{1.030039in}{2.540333in}}{\pgfqpoint{1.018989in}{2.540333in}}%
\pgfpathcurveto{\pgfqpoint{1.007939in}{2.540333in}}{\pgfqpoint{0.997340in}{2.535943in}}{\pgfqpoint{0.989526in}{2.528129in}}%
\pgfpathcurveto{\pgfqpoint{0.981712in}{2.520316in}}{\pgfqpoint{0.977322in}{2.509717in}}{\pgfqpoint{0.977322in}{2.498666in}}%
\pgfpathcurveto{\pgfqpoint{0.977322in}{2.487616in}}{\pgfqpoint{0.981712in}{2.477017in}}{\pgfqpoint{0.989526in}{2.469204in}}%
\pgfpathcurveto{\pgfqpoint{0.997340in}{2.461390in}}{\pgfqpoint{1.007939in}{2.457000in}}{\pgfqpoint{1.018989in}{2.457000in}}%
\pgfpathlineto{\pgfqpoint{1.018989in}{2.457000in}}%
\pgfpathclose%
\pgfusepath{stroke}%
\end{pgfscope}%
\begin{pgfscope}%
\pgfpathrectangle{\pgfqpoint{0.393053in}{0.375000in}}{\pgfqpoint{6.356833in}{5.175000in}}%
\pgfusepath{clip}%
\pgfsetbuttcap%
\pgfsetroundjoin%
\pgfsetlinewidth{1.003750pt}%
\definecolor{currentstroke}{rgb}{0.827451,0.827451,0.827451}%
\pgfsetstrokecolor{currentstroke}%
\pgfsetdash{}{0pt}%
\pgfpathmoveto{\pgfqpoint{0.525326in}{3.470591in}}%
\pgfpathcurveto{\pgfqpoint{0.536376in}{3.470591in}}{\pgfqpoint{0.546975in}{3.474982in}}{\pgfqpoint{0.554788in}{3.482795in}}%
\pgfpathcurveto{\pgfqpoint{0.562602in}{3.490609in}}{\pgfqpoint{0.566992in}{3.501208in}}{\pgfqpoint{0.566992in}{3.512258in}}%
\pgfpathcurveto{\pgfqpoint{0.566992in}{3.523308in}}{\pgfqpoint{0.562602in}{3.533907in}}{\pgfqpoint{0.554788in}{3.541721in}}%
\pgfpathcurveto{\pgfqpoint{0.546975in}{3.549534in}}{\pgfqpoint{0.536376in}{3.553925in}}{\pgfqpoint{0.525326in}{3.553925in}}%
\pgfpathcurveto{\pgfqpoint{0.514275in}{3.553925in}}{\pgfqpoint{0.503676in}{3.549534in}}{\pgfqpoint{0.495863in}{3.541721in}}%
\pgfpathcurveto{\pgfqpoint{0.488049in}{3.533907in}}{\pgfqpoint{0.483659in}{3.523308in}}{\pgfqpoint{0.483659in}{3.512258in}}%
\pgfpathcurveto{\pgfqpoint{0.483659in}{3.501208in}}{\pgfqpoint{0.488049in}{3.490609in}}{\pgfqpoint{0.495863in}{3.482795in}}%
\pgfpathcurveto{\pgfqpoint{0.503676in}{3.474982in}}{\pgfqpoint{0.514275in}{3.470591in}}{\pgfqpoint{0.525326in}{3.470591in}}%
\pgfpathlineto{\pgfqpoint{0.525326in}{3.470591in}}%
\pgfpathclose%
\pgfusepath{stroke}%
\end{pgfscope}%
\begin{pgfscope}%
\pgfpathrectangle{\pgfqpoint{0.393053in}{0.375000in}}{\pgfqpoint{6.356833in}{5.175000in}}%
\pgfusepath{clip}%
\pgfsetbuttcap%
\pgfsetroundjoin%
\pgfsetlinewidth{1.003750pt}%
\definecolor{currentstroke}{rgb}{0.827451,0.827451,0.827451}%
\pgfsetstrokecolor{currentstroke}%
\pgfsetdash{}{0pt}%
\pgfpathmoveto{\pgfqpoint{4.176160in}{0.456257in}}%
\pgfpathcurveto{\pgfqpoint{4.187210in}{0.456257in}}{\pgfqpoint{4.197809in}{0.460647in}}{\pgfqpoint{4.205623in}{0.468461in}}%
\pgfpathcurveto{\pgfqpoint{4.213436in}{0.476274in}}{\pgfqpoint{4.217827in}{0.486873in}}{\pgfqpoint{4.217827in}{0.497923in}}%
\pgfpathcurveto{\pgfqpoint{4.217827in}{0.508973in}}{\pgfqpoint{4.213436in}{0.519572in}}{\pgfqpoint{4.205623in}{0.527386in}}%
\pgfpathcurveto{\pgfqpoint{4.197809in}{0.535200in}}{\pgfqpoint{4.187210in}{0.539590in}}{\pgfqpoint{4.176160in}{0.539590in}}%
\pgfpathcurveto{\pgfqpoint{4.165110in}{0.539590in}}{\pgfqpoint{4.154511in}{0.535200in}}{\pgfqpoint{4.146697in}{0.527386in}}%
\pgfpathcurveto{\pgfqpoint{4.138883in}{0.519572in}}{\pgfqpoint{4.134493in}{0.508973in}}{\pgfqpoint{4.134493in}{0.497923in}}%
\pgfpathcurveto{\pgfqpoint{4.134493in}{0.486873in}}{\pgfqpoint{4.138883in}{0.476274in}}{\pgfqpoint{4.146697in}{0.468461in}}%
\pgfpathcurveto{\pgfqpoint{4.154511in}{0.460647in}}{\pgfqpoint{4.165110in}{0.456257in}}{\pgfqpoint{4.176160in}{0.456257in}}%
\pgfpathlineto{\pgfqpoint{4.176160in}{0.456257in}}%
\pgfpathclose%
\pgfusepath{stroke}%
\end{pgfscope}%
\begin{pgfscope}%
\pgfpathrectangle{\pgfqpoint{0.393053in}{0.375000in}}{\pgfqpoint{6.356833in}{5.175000in}}%
\pgfusepath{clip}%
\pgfsetbuttcap%
\pgfsetroundjoin%
\pgfsetlinewidth{1.003750pt}%
\definecolor{currentstroke}{rgb}{0.827451,0.827451,0.827451}%
\pgfsetstrokecolor{currentstroke}%
\pgfsetdash{}{0pt}%
\pgfpathmoveto{\pgfqpoint{4.445943in}{0.425176in}}%
\pgfpathcurveto{\pgfqpoint{4.456993in}{0.425176in}}{\pgfqpoint{4.467592in}{0.429567in}}{\pgfqpoint{4.475406in}{0.437380in}}%
\pgfpathcurveto{\pgfqpoint{4.483219in}{0.445194in}}{\pgfqpoint{4.487610in}{0.455793in}}{\pgfqpoint{4.487610in}{0.466843in}}%
\pgfpathcurveto{\pgfqpoint{4.487610in}{0.477893in}}{\pgfqpoint{4.483219in}{0.488492in}}{\pgfqpoint{4.475406in}{0.496306in}}%
\pgfpathcurveto{\pgfqpoint{4.467592in}{0.504119in}}{\pgfqpoint{4.456993in}{0.508510in}}{\pgfqpoint{4.445943in}{0.508510in}}%
\pgfpathcurveto{\pgfqpoint{4.434893in}{0.508510in}}{\pgfqpoint{4.424294in}{0.504119in}}{\pgfqpoint{4.416480in}{0.496306in}}%
\pgfpathcurveto{\pgfqpoint{4.408667in}{0.488492in}}{\pgfqpoint{4.404276in}{0.477893in}}{\pgfqpoint{4.404276in}{0.466843in}}%
\pgfpathcurveto{\pgfqpoint{4.404276in}{0.455793in}}{\pgfqpoint{4.408667in}{0.445194in}}{\pgfqpoint{4.416480in}{0.437380in}}%
\pgfpathcurveto{\pgfqpoint{4.424294in}{0.429567in}}{\pgfqpoint{4.434893in}{0.425176in}}{\pgfqpoint{4.445943in}{0.425176in}}%
\pgfpathlineto{\pgfqpoint{4.445943in}{0.425176in}}%
\pgfpathclose%
\pgfusepath{stroke}%
\end{pgfscope}%
\begin{pgfscope}%
\pgfpathrectangle{\pgfqpoint{0.393053in}{0.375000in}}{\pgfqpoint{6.356833in}{5.175000in}}%
\pgfusepath{clip}%
\pgfsetbuttcap%
\pgfsetroundjoin%
\pgfsetlinewidth{1.003750pt}%
\definecolor{currentstroke}{rgb}{0.827451,0.827451,0.827451}%
\pgfsetstrokecolor{currentstroke}%
\pgfsetdash{}{0pt}%
\pgfpathmoveto{\pgfqpoint{1.847609in}{1.480323in}}%
\pgfpathcurveto{\pgfqpoint{1.858659in}{1.480323in}}{\pgfqpoint{1.869258in}{1.484713in}}{\pgfqpoint{1.877071in}{1.492527in}}%
\pgfpathcurveto{\pgfqpoint{1.884885in}{1.500341in}}{\pgfqpoint{1.889275in}{1.510940in}}{\pgfqpoint{1.889275in}{1.521990in}}%
\pgfpathcurveto{\pgfqpoint{1.889275in}{1.533040in}}{\pgfqpoint{1.884885in}{1.543639in}}{\pgfqpoint{1.877071in}{1.551453in}}%
\pgfpathcurveto{\pgfqpoint{1.869258in}{1.559266in}}{\pgfqpoint{1.858659in}{1.563656in}}{\pgfqpoint{1.847609in}{1.563656in}}%
\pgfpathcurveto{\pgfqpoint{1.836559in}{1.563656in}}{\pgfqpoint{1.825960in}{1.559266in}}{\pgfqpoint{1.818146in}{1.551453in}}%
\pgfpathcurveto{\pgfqpoint{1.810332in}{1.543639in}}{\pgfqpoint{1.805942in}{1.533040in}}{\pgfqpoint{1.805942in}{1.521990in}}%
\pgfpathcurveto{\pgfqpoint{1.805942in}{1.510940in}}{\pgfqpoint{1.810332in}{1.500341in}}{\pgfqpoint{1.818146in}{1.492527in}}%
\pgfpathcurveto{\pgfqpoint{1.825960in}{1.484713in}}{\pgfqpoint{1.836559in}{1.480323in}}{\pgfqpoint{1.847609in}{1.480323in}}%
\pgfpathlineto{\pgfqpoint{1.847609in}{1.480323in}}%
\pgfpathclose%
\pgfusepath{stroke}%
\end{pgfscope}%
\begin{pgfscope}%
\pgfpathrectangle{\pgfqpoint{0.393053in}{0.375000in}}{\pgfqpoint{6.356833in}{5.175000in}}%
\pgfusepath{clip}%
\pgfsetbuttcap%
\pgfsetroundjoin%
\pgfsetlinewidth{1.003750pt}%
\definecolor{currentstroke}{rgb}{0.827451,0.827451,0.827451}%
\pgfsetstrokecolor{currentstroke}%
\pgfsetdash{}{0pt}%
\pgfpathmoveto{\pgfqpoint{0.396398in}{4.410130in}}%
\pgfpathcurveto{\pgfqpoint{0.407448in}{4.410130in}}{\pgfqpoint{0.418047in}{4.414521in}}{\pgfqpoint{0.425861in}{4.422334in}}%
\pgfpathcurveto{\pgfqpoint{0.433674in}{4.430148in}}{\pgfqpoint{0.438064in}{4.440747in}}{\pgfqpoint{0.438064in}{4.451797in}}%
\pgfpathcurveto{\pgfqpoint{0.438064in}{4.462847in}}{\pgfqpoint{0.433674in}{4.473446in}}{\pgfqpoint{0.425861in}{4.481260in}}%
\pgfpathcurveto{\pgfqpoint{0.418047in}{4.489073in}}{\pgfqpoint{0.407448in}{4.493464in}}{\pgfqpoint{0.396398in}{4.493464in}}%
\pgfpathcurveto{\pgfqpoint{0.385348in}{4.493464in}}{\pgfqpoint{0.374749in}{4.489073in}}{\pgfqpoint{0.366935in}{4.481260in}}%
\pgfpathcurveto{\pgfqpoint{0.359121in}{4.473446in}}{\pgfqpoint{0.354731in}{4.462847in}}{\pgfqpoint{0.354731in}{4.451797in}}%
\pgfpathcurveto{\pgfqpoint{0.354731in}{4.440747in}}{\pgfqpoint{0.359121in}{4.430148in}}{\pgfqpoint{0.366935in}{4.422334in}}%
\pgfpathcurveto{\pgfqpoint{0.374749in}{4.414521in}}{\pgfqpoint{0.385348in}{4.410130in}}{\pgfqpoint{0.396398in}{4.410130in}}%
\pgfpathlineto{\pgfqpoint{0.396398in}{4.410130in}}%
\pgfpathclose%
\pgfusepath{stroke}%
\end{pgfscope}%
\begin{pgfscope}%
\pgfpathrectangle{\pgfqpoint{0.393053in}{0.375000in}}{\pgfqpoint{6.356833in}{5.175000in}}%
\pgfusepath{clip}%
\pgfsetbuttcap%
\pgfsetroundjoin%
\pgfsetlinewidth{1.003750pt}%
\definecolor{currentstroke}{rgb}{0.827451,0.827451,0.827451}%
\pgfsetstrokecolor{currentstroke}%
\pgfsetdash{}{0pt}%
\pgfpathmoveto{\pgfqpoint{0.397222in}{4.408616in}}%
\pgfpathcurveto{\pgfqpoint{0.408272in}{4.408616in}}{\pgfqpoint{0.418871in}{4.413006in}}{\pgfqpoint{0.426685in}{4.420819in}}%
\pgfpathcurveto{\pgfqpoint{0.434499in}{4.428633in}}{\pgfqpoint{0.438889in}{4.439232in}}{\pgfqpoint{0.438889in}{4.450282in}}%
\pgfpathcurveto{\pgfqpoint{0.438889in}{4.461332in}}{\pgfqpoint{0.434499in}{4.471931in}}{\pgfqpoint{0.426685in}{4.479745in}}%
\pgfpathcurveto{\pgfqpoint{0.418871in}{4.487559in}}{\pgfqpoint{0.408272in}{4.491949in}}{\pgfqpoint{0.397222in}{4.491949in}}%
\pgfpathcurveto{\pgfqpoint{0.386172in}{4.491949in}}{\pgfqpoint{0.375573in}{4.487559in}}{\pgfqpoint{0.367760in}{4.479745in}}%
\pgfpathcurveto{\pgfqpoint{0.359946in}{4.471931in}}{\pgfqpoint{0.355556in}{4.461332in}}{\pgfqpoint{0.355556in}{4.450282in}}%
\pgfpathcurveto{\pgfqpoint{0.355556in}{4.439232in}}{\pgfqpoint{0.359946in}{4.428633in}}{\pgfqpoint{0.367760in}{4.420819in}}%
\pgfpathcurveto{\pgfqpoint{0.375573in}{4.413006in}}{\pgfqpoint{0.386172in}{4.408616in}}{\pgfqpoint{0.397222in}{4.408616in}}%
\pgfpathlineto{\pgfqpoint{0.397222in}{4.408616in}}%
\pgfpathclose%
\pgfusepath{stroke}%
\end{pgfscope}%
\begin{pgfscope}%
\pgfpathrectangle{\pgfqpoint{0.393053in}{0.375000in}}{\pgfqpoint{6.356833in}{5.175000in}}%
\pgfusepath{clip}%
\pgfsetbuttcap%
\pgfsetroundjoin%
\pgfsetlinewidth{1.003750pt}%
\definecolor{currentstroke}{rgb}{0.827451,0.827451,0.827451}%
\pgfsetstrokecolor{currentstroke}%
\pgfsetdash{}{0pt}%
\pgfpathmoveto{\pgfqpoint{4.064043in}{0.479739in}}%
\pgfpathcurveto{\pgfqpoint{4.075093in}{0.479739in}}{\pgfqpoint{4.085692in}{0.484129in}}{\pgfqpoint{4.093506in}{0.491943in}}%
\pgfpathcurveto{\pgfqpoint{4.101320in}{0.499756in}}{\pgfqpoint{4.105710in}{0.510355in}}{\pgfqpoint{4.105710in}{0.521405in}}%
\pgfpathcurveto{\pgfqpoint{4.105710in}{0.532456in}}{\pgfqpoint{4.101320in}{0.543055in}}{\pgfqpoint{4.093506in}{0.550868in}}%
\pgfpathcurveto{\pgfqpoint{4.085692in}{0.558682in}}{\pgfqpoint{4.075093in}{0.563072in}}{\pgfqpoint{4.064043in}{0.563072in}}%
\pgfpathcurveto{\pgfqpoint{4.052993in}{0.563072in}}{\pgfqpoint{4.042394in}{0.558682in}}{\pgfqpoint{4.034581in}{0.550868in}}%
\pgfpathcurveto{\pgfqpoint{4.026767in}{0.543055in}}{\pgfqpoint{4.022377in}{0.532456in}}{\pgfqpoint{4.022377in}{0.521405in}}%
\pgfpathcurveto{\pgfqpoint{4.022377in}{0.510355in}}{\pgfqpoint{4.026767in}{0.499756in}}{\pgfqpoint{4.034581in}{0.491943in}}%
\pgfpathcurveto{\pgfqpoint{4.042394in}{0.484129in}}{\pgfqpoint{4.052993in}{0.479739in}}{\pgfqpoint{4.064043in}{0.479739in}}%
\pgfpathlineto{\pgfqpoint{4.064043in}{0.479739in}}%
\pgfpathclose%
\pgfusepath{stroke}%
\end{pgfscope}%
\begin{pgfscope}%
\pgfpathrectangle{\pgfqpoint{0.393053in}{0.375000in}}{\pgfqpoint{6.356833in}{5.175000in}}%
\pgfusepath{clip}%
\pgfsetbuttcap%
\pgfsetroundjoin%
\pgfsetlinewidth{1.003750pt}%
\definecolor{currentstroke}{rgb}{1.000000,0.000000,0.000000}%
\pgfsetstrokecolor{currentstroke}%
\pgfsetdash{}{0pt}%
\pgfpathmoveto{\pgfqpoint{0.393053in}{4.584226in}}%
\pgfpathcurveto{\pgfqpoint{0.404103in}{4.584226in}}{\pgfqpoint{0.414702in}{4.588616in}}{\pgfqpoint{0.422516in}{4.596430in}}%
\pgfpathcurveto{\pgfqpoint{0.430329in}{4.604244in}}{\pgfqpoint{0.434720in}{4.614843in}}{\pgfqpoint{0.434720in}{4.625893in}}%
\pgfpathcurveto{\pgfqpoint{0.434720in}{4.636943in}}{\pgfqpoint{0.430329in}{4.647542in}}{\pgfqpoint{0.422516in}{4.655356in}}%
\pgfpathcurveto{\pgfqpoint{0.414702in}{4.663169in}}{\pgfqpoint{0.404103in}{4.667559in}}{\pgfqpoint{0.393053in}{4.667559in}}%
\pgfpathcurveto{\pgfqpoint{0.382003in}{4.667559in}}{\pgfqpoint{0.371404in}{4.663169in}}{\pgfqpoint{0.363590in}{4.655356in}}%
\pgfpathcurveto{\pgfqpoint{0.355777in}{4.647542in}}{\pgfqpoint{0.351386in}{4.636943in}}{\pgfqpoint{0.351386in}{4.625893in}}%
\pgfpathcurveto{\pgfqpoint{0.351386in}{4.614843in}}{\pgfqpoint{0.355777in}{4.604244in}}{\pgfqpoint{0.363590in}{4.596430in}}%
\pgfpathcurveto{\pgfqpoint{0.371404in}{4.588616in}}{\pgfqpoint{0.382003in}{4.584226in}}{\pgfqpoint{0.393053in}{4.584226in}}%
\pgfpathlineto{\pgfqpoint{0.393053in}{4.584226in}}%
\pgfpathclose%
\pgfusepath{stroke}%
\end{pgfscope}%
\begin{pgfscope}%
\pgfpathrectangle{\pgfqpoint{0.393053in}{0.375000in}}{\pgfqpoint{6.356833in}{5.175000in}}%
\pgfusepath{clip}%
\pgfsetbuttcap%
\pgfsetroundjoin%
\pgfsetlinewidth{1.003750pt}%
\definecolor{currentstroke}{rgb}{1.000000,0.000000,0.000000}%
\pgfsetstrokecolor{currentstroke}%
\pgfsetdash{}{0pt}%
\pgfpathmoveto{\pgfqpoint{5.796361in}{0.333333in}}%
\pgfpathcurveto{\pgfqpoint{5.807411in}{0.333333in}}{\pgfqpoint{5.818010in}{0.337723in}}{\pgfqpoint{5.825824in}{0.345537in}}%
\pgfpathcurveto{\pgfqpoint{5.833637in}{0.353350in}}{\pgfqpoint{5.838028in}{0.363949in}}{\pgfqpoint{5.838028in}{0.375000in}}%
\pgfpathcurveto{\pgfqpoint{5.838028in}{0.386050in}}{\pgfqpoint{5.833637in}{0.396649in}}{\pgfqpoint{5.825824in}{0.404462in}}%
\pgfpathcurveto{\pgfqpoint{5.818010in}{0.412276in}}{\pgfqpoint{5.807411in}{0.416666in}}{\pgfqpoint{5.796361in}{0.416666in}}%
\pgfpathcurveto{\pgfqpoint{5.785311in}{0.416666in}}{\pgfqpoint{5.774712in}{0.412276in}}{\pgfqpoint{5.766898in}{0.404462in}}%
\pgfpathcurveto{\pgfqpoint{5.759085in}{0.396649in}}{\pgfqpoint{5.754694in}{0.386050in}}{\pgfqpoint{5.754694in}{0.375000in}}%
\pgfpathcurveto{\pgfqpoint{5.754694in}{0.363949in}}{\pgfqpoint{5.759085in}{0.353350in}}{\pgfqpoint{5.766898in}{0.345537in}}%
\pgfpathcurveto{\pgfqpoint{5.774712in}{0.337723in}}{\pgfqpoint{5.785311in}{0.333333in}}{\pgfqpoint{5.796361in}{0.333333in}}%
\pgfusepath{stroke}%
\end{pgfscope}%
\begin{pgfscope}%
\pgfpathrectangle{\pgfqpoint{0.393053in}{0.375000in}}{\pgfqpoint{6.356833in}{5.175000in}}%
\pgfusepath{clip}%
\pgfsetbuttcap%
\pgfsetroundjoin%
\pgfsetlinewidth{1.003750pt}%
\definecolor{currentstroke}{rgb}{1.000000,0.000000,0.000000}%
\pgfsetstrokecolor{currentstroke}%
\pgfsetdash{}{0pt}%
\pgfpathmoveto{\pgfqpoint{2.726457in}{0.951004in}}%
\pgfpathcurveto{\pgfqpoint{2.737507in}{0.951004in}}{\pgfqpoint{2.748106in}{0.955394in}}{\pgfqpoint{2.755920in}{0.963208in}}%
\pgfpathcurveto{\pgfqpoint{2.763733in}{0.971021in}}{\pgfqpoint{2.768124in}{0.981620in}}{\pgfqpoint{2.768124in}{0.992670in}}%
\pgfpathcurveto{\pgfqpoint{2.768124in}{1.003721in}}{\pgfqpoint{2.763733in}{1.014320in}}{\pgfqpoint{2.755920in}{1.022133in}}%
\pgfpathcurveto{\pgfqpoint{2.748106in}{1.029947in}}{\pgfqpoint{2.737507in}{1.034337in}}{\pgfqpoint{2.726457in}{1.034337in}}%
\pgfpathcurveto{\pgfqpoint{2.715407in}{1.034337in}}{\pgfqpoint{2.704808in}{1.029947in}}{\pgfqpoint{2.696994in}{1.022133in}}%
\pgfpathcurveto{\pgfqpoint{2.689181in}{1.014320in}}{\pgfqpoint{2.684790in}{1.003721in}}{\pgfqpoint{2.684790in}{0.992670in}}%
\pgfpathcurveto{\pgfqpoint{2.684790in}{0.981620in}}{\pgfqpoint{2.689181in}{0.971021in}}{\pgfqpoint{2.696994in}{0.963208in}}%
\pgfpathcurveto{\pgfqpoint{2.704808in}{0.955394in}}{\pgfqpoint{2.715407in}{0.951004in}}{\pgfqpoint{2.726457in}{0.951004in}}%
\pgfpathlineto{\pgfqpoint{2.726457in}{0.951004in}}%
\pgfpathclose%
\pgfusepath{stroke}%
\end{pgfscope}%
\begin{pgfscope}%
\pgfpathrectangle{\pgfqpoint{0.393053in}{0.375000in}}{\pgfqpoint{6.356833in}{5.175000in}}%
\pgfusepath{clip}%
\pgfsetbuttcap%
\pgfsetroundjoin%
\pgfsetlinewidth{1.003750pt}%
\definecolor{currentstroke}{rgb}{1.000000,0.000000,0.000000}%
\pgfsetstrokecolor{currentstroke}%
\pgfsetdash{}{0pt}%
\pgfpathmoveto{\pgfqpoint{0.421306in}{4.054197in}}%
\pgfpathcurveto{\pgfqpoint{0.432357in}{4.054197in}}{\pgfqpoint{0.442956in}{4.058587in}}{\pgfqpoint{0.450769in}{4.066401in}}%
\pgfpathcurveto{\pgfqpoint{0.458583in}{4.074214in}}{\pgfqpoint{0.462973in}{4.084813in}}{\pgfqpoint{0.462973in}{4.095863in}}%
\pgfpathcurveto{\pgfqpoint{0.462973in}{4.106913in}}{\pgfqpoint{0.458583in}{4.117512in}}{\pgfqpoint{0.450769in}{4.125326in}}%
\pgfpathcurveto{\pgfqpoint{0.442956in}{4.133140in}}{\pgfqpoint{0.432357in}{4.137530in}}{\pgfqpoint{0.421306in}{4.137530in}}%
\pgfpathcurveto{\pgfqpoint{0.410256in}{4.137530in}}{\pgfqpoint{0.399657in}{4.133140in}}{\pgfqpoint{0.391844in}{4.125326in}}%
\pgfpathcurveto{\pgfqpoint{0.384030in}{4.117512in}}{\pgfqpoint{0.379640in}{4.106913in}}{\pgfqpoint{0.379640in}{4.095863in}}%
\pgfpathcurveto{\pgfqpoint{0.379640in}{4.084813in}}{\pgfqpoint{0.384030in}{4.074214in}}{\pgfqpoint{0.391844in}{4.066401in}}%
\pgfpathcurveto{\pgfqpoint{0.399657in}{4.058587in}}{\pgfqpoint{0.410256in}{4.054197in}}{\pgfqpoint{0.421306in}{4.054197in}}%
\pgfpathlineto{\pgfqpoint{0.421306in}{4.054197in}}%
\pgfpathclose%
\pgfusepath{stroke}%
\end{pgfscope}%
\begin{pgfscope}%
\pgfpathrectangle{\pgfqpoint{0.393053in}{0.375000in}}{\pgfqpoint{6.356833in}{5.175000in}}%
\pgfusepath{clip}%
\pgfsetbuttcap%
\pgfsetroundjoin%
\pgfsetlinewidth{1.003750pt}%
\definecolor{currentstroke}{rgb}{1.000000,0.000000,0.000000}%
\pgfsetstrokecolor{currentstroke}%
\pgfsetdash{}{0pt}%
\pgfpathmoveto{\pgfqpoint{4.756169in}{0.379150in}}%
\pgfpathcurveto{\pgfqpoint{4.767219in}{0.379150in}}{\pgfqpoint{4.777818in}{0.383540in}}{\pgfqpoint{4.785632in}{0.391354in}}%
\pgfpathcurveto{\pgfqpoint{4.793446in}{0.399168in}}{\pgfqpoint{4.797836in}{0.409767in}}{\pgfqpoint{4.797836in}{0.420817in}}%
\pgfpathcurveto{\pgfqpoint{4.797836in}{0.431867in}}{\pgfqpoint{4.793446in}{0.442466in}}{\pgfqpoint{4.785632in}{0.450279in}}%
\pgfpathcurveto{\pgfqpoint{4.777818in}{0.458093in}}{\pgfqpoint{4.767219in}{0.462483in}}{\pgfqpoint{4.756169in}{0.462483in}}%
\pgfpathcurveto{\pgfqpoint{4.745119in}{0.462483in}}{\pgfqpoint{4.734520in}{0.458093in}}{\pgfqpoint{4.726706in}{0.450279in}}%
\pgfpathcurveto{\pgfqpoint{4.718893in}{0.442466in}}{\pgfqpoint{4.714502in}{0.431867in}}{\pgfqpoint{4.714502in}{0.420817in}}%
\pgfpathcurveto{\pgfqpoint{4.714502in}{0.409767in}}{\pgfqpoint{4.718893in}{0.399168in}}{\pgfqpoint{4.726706in}{0.391354in}}%
\pgfpathcurveto{\pgfqpoint{4.734520in}{0.383540in}}{\pgfqpoint{4.745119in}{0.379150in}}{\pgfqpoint{4.756169in}{0.379150in}}%
\pgfpathlineto{\pgfqpoint{4.756169in}{0.379150in}}%
\pgfpathclose%
\pgfusepath{stroke}%
\end{pgfscope}%
\begin{pgfscope}%
\pgfpathrectangle{\pgfqpoint{0.393053in}{0.375000in}}{\pgfqpoint{6.356833in}{5.175000in}}%
\pgfusepath{clip}%
\pgfsetbuttcap%
\pgfsetroundjoin%
\pgfsetlinewidth{1.003750pt}%
\definecolor{currentstroke}{rgb}{1.000000,0.000000,0.000000}%
\pgfsetstrokecolor{currentstroke}%
\pgfsetdash{}{0pt}%
\pgfpathmoveto{\pgfqpoint{0.409284in}{4.182264in}}%
\pgfpathcurveto{\pgfqpoint{0.420334in}{4.182264in}}{\pgfqpoint{0.430933in}{4.186654in}}{\pgfqpoint{0.438747in}{4.194468in}}%
\pgfpathcurveto{\pgfqpoint{0.446561in}{4.202282in}}{\pgfqpoint{0.450951in}{4.212881in}}{\pgfqpoint{0.450951in}{4.223931in}}%
\pgfpathcurveto{\pgfqpoint{0.450951in}{4.234981in}}{\pgfqpoint{0.446561in}{4.245580in}}{\pgfqpoint{0.438747in}{4.253394in}}%
\pgfpathcurveto{\pgfqpoint{0.430933in}{4.261207in}}{\pgfqpoint{0.420334in}{4.265597in}}{\pgfqpoint{0.409284in}{4.265597in}}%
\pgfpathcurveto{\pgfqpoint{0.398234in}{4.265597in}}{\pgfqpoint{0.387635in}{4.261207in}}{\pgfqpoint{0.379822in}{4.253394in}}%
\pgfpathcurveto{\pgfqpoint{0.372008in}{4.245580in}}{\pgfqpoint{0.367618in}{4.234981in}}{\pgfqpoint{0.367618in}{4.223931in}}%
\pgfpathcurveto{\pgfqpoint{0.367618in}{4.212881in}}{\pgfqpoint{0.372008in}{4.202282in}}{\pgfqpoint{0.379822in}{4.194468in}}%
\pgfpathcurveto{\pgfqpoint{0.387635in}{4.186654in}}{\pgfqpoint{0.398234in}{4.182264in}}{\pgfqpoint{0.409284in}{4.182264in}}%
\pgfpathlineto{\pgfqpoint{0.409284in}{4.182264in}}%
\pgfpathclose%
\pgfusepath{stroke}%
\end{pgfscope}%
\begin{pgfscope}%
\pgfpathrectangle{\pgfqpoint{0.393053in}{0.375000in}}{\pgfqpoint{6.356833in}{5.175000in}}%
\pgfusepath{clip}%
\pgfsetbuttcap%
\pgfsetroundjoin%
\pgfsetlinewidth{1.003750pt}%
\definecolor{currentstroke}{rgb}{1.000000,0.000000,0.000000}%
\pgfsetstrokecolor{currentstroke}%
\pgfsetdash{}{0pt}%
\pgfpathmoveto{\pgfqpoint{2.619498in}{0.991979in}}%
\pgfpathcurveto{\pgfqpoint{2.630548in}{0.991979in}}{\pgfqpoint{2.641147in}{0.996369in}}{\pgfqpoint{2.648961in}{1.004183in}}%
\pgfpathcurveto{\pgfqpoint{2.656774in}{1.011996in}}{\pgfqpoint{2.661165in}{1.022595in}}{\pgfqpoint{2.661165in}{1.033645in}}%
\pgfpathcurveto{\pgfqpoint{2.661165in}{1.044696in}}{\pgfqpoint{2.656774in}{1.055295in}}{\pgfqpoint{2.648961in}{1.063108in}}%
\pgfpathcurveto{\pgfqpoint{2.641147in}{1.070922in}}{\pgfqpoint{2.630548in}{1.075312in}}{\pgfqpoint{2.619498in}{1.075312in}}%
\pgfpathcurveto{\pgfqpoint{2.608448in}{1.075312in}}{\pgfqpoint{2.597849in}{1.070922in}}{\pgfqpoint{2.590035in}{1.063108in}}%
\pgfpathcurveto{\pgfqpoint{2.582222in}{1.055295in}}{\pgfqpoint{2.577831in}{1.044696in}}{\pgfqpoint{2.577831in}{1.033645in}}%
\pgfpathcurveto{\pgfqpoint{2.577831in}{1.022595in}}{\pgfqpoint{2.582222in}{1.011996in}}{\pgfqpoint{2.590035in}{1.004183in}}%
\pgfpathcurveto{\pgfqpoint{2.597849in}{0.996369in}}{\pgfqpoint{2.608448in}{0.991979in}}{\pgfqpoint{2.619498in}{0.991979in}}%
\pgfpathlineto{\pgfqpoint{2.619498in}{0.991979in}}%
\pgfpathclose%
\pgfusepath{stroke}%
\end{pgfscope}%
\begin{pgfscope}%
\pgfpathrectangle{\pgfqpoint{0.393053in}{0.375000in}}{\pgfqpoint{6.356833in}{5.175000in}}%
\pgfusepath{clip}%
\pgfsetbuttcap%
\pgfsetroundjoin%
\pgfsetlinewidth{1.003750pt}%
\definecolor{currentstroke}{rgb}{1.000000,0.000000,0.000000}%
\pgfsetstrokecolor{currentstroke}%
\pgfsetdash{}{0pt}%
\pgfpathmoveto{\pgfqpoint{5.161763in}{0.349379in}}%
\pgfpathcurveto{\pgfqpoint{5.172813in}{0.349379in}}{\pgfqpoint{5.183412in}{0.353769in}}{\pgfqpoint{5.191226in}{0.361583in}}%
\pgfpathcurveto{\pgfqpoint{5.199040in}{0.369396in}}{\pgfqpoint{5.203430in}{0.379995in}}{\pgfqpoint{5.203430in}{0.391045in}}%
\pgfpathcurveto{\pgfqpoint{5.203430in}{0.402096in}}{\pgfqpoint{5.199040in}{0.412695in}}{\pgfqpoint{5.191226in}{0.420508in}}%
\pgfpathcurveto{\pgfqpoint{5.183412in}{0.428322in}}{\pgfqpoint{5.172813in}{0.432712in}}{\pgfqpoint{5.161763in}{0.432712in}}%
\pgfpathcurveto{\pgfqpoint{5.150713in}{0.432712in}}{\pgfqpoint{5.140114in}{0.428322in}}{\pgfqpoint{5.132300in}{0.420508in}}%
\pgfpathcurveto{\pgfqpoint{5.124487in}{0.412695in}}{\pgfqpoint{5.120096in}{0.402096in}}{\pgfqpoint{5.120096in}{0.391045in}}%
\pgfpathcurveto{\pgfqpoint{5.120096in}{0.379995in}}{\pgfqpoint{5.124487in}{0.369396in}}{\pgfqpoint{5.132300in}{0.361583in}}%
\pgfpathcurveto{\pgfqpoint{5.140114in}{0.353769in}}{\pgfqpoint{5.150713in}{0.349379in}}{\pgfqpoint{5.161763in}{0.349379in}}%
\pgfusepath{stroke}%
\end{pgfscope}%
\begin{pgfscope}%
\pgfpathrectangle{\pgfqpoint{0.393053in}{0.375000in}}{\pgfqpoint{6.356833in}{5.175000in}}%
\pgfusepath{clip}%
\pgfsetbuttcap%
\pgfsetroundjoin%
\pgfsetlinewidth{1.003750pt}%
\definecolor{currentstroke}{rgb}{1.000000,0.000000,0.000000}%
\pgfsetstrokecolor{currentstroke}%
\pgfsetdash{}{0pt}%
\pgfpathmoveto{\pgfqpoint{5.411535in}{0.344715in}}%
\pgfpathcurveto{\pgfqpoint{5.422585in}{0.344715in}}{\pgfqpoint{5.433184in}{0.349105in}}{\pgfqpoint{5.440998in}{0.356919in}}%
\pgfpathcurveto{\pgfqpoint{5.448811in}{0.364732in}}{\pgfqpoint{5.453202in}{0.375331in}}{\pgfqpoint{5.453202in}{0.386382in}}%
\pgfpathcurveto{\pgfqpoint{5.453202in}{0.397432in}}{\pgfqpoint{5.448811in}{0.408031in}}{\pgfqpoint{5.440998in}{0.415844in}}%
\pgfpathcurveto{\pgfqpoint{5.433184in}{0.423658in}}{\pgfqpoint{5.422585in}{0.428048in}}{\pgfqpoint{5.411535in}{0.428048in}}%
\pgfpathcurveto{\pgfqpoint{5.400485in}{0.428048in}}{\pgfqpoint{5.389886in}{0.423658in}}{\pgfqpoint{5.382072in}{0.415844in}}%
\pgfpathcurveto{\pgfqpoint{5.374258in}{0.408031in}}{\pgfqpoint{5.369868in}{0.397432in}}{\pgfqpoint{5.369868in}{0.386382in}}%
\pgfpathcurveto{\pgfqpoint{5.369868in}{0.375331in}}{\pgfqpoint{5.374258in}{0.364732in}}{\pgfqpoint{5.382072in}{0.356919in}}%
\pgfpathcurveto{\pgfqpoint{5.389886in}{0.349105in}}{\pgfqpoint{5.400485in}{0.344715in}}{\pgfqpoint{5.411535in}{0.344715in}}%
\pgfusepath{stroke}%
\end{pgfscope}%
\begin{pgfscope}%
\pgfpathrectangle{\pgfqpoint{0.393053in}{0.375000in}}{\pgfqpoint{6.356833in}{5.175000in}}%
\pgfusepath{clip}%
\pgfsetbuttcap%
\pgfsetroundjoin%
\pgfsetlinewidth{1.003750pt}%
\definecolor{currentstroke}{rgb}{1.000000,0.000000,0.000000}%
\pgfsetstrokecolor{currentstroke}%
\pgfsetdash{}{0pt}%
\pgfpathmoveto{\pgfqpoint{2.845716in}{0.881330in}}%
\pgfpathcurveto{\pgfqpoint{2.856766in}{0.881330in}}{\pgfqpoint{2.867365in}{0.885720in}}{\pgfqpoint{2.875179in}{0.893534in}}%
\pgfpathcurveto{\pgfqpoint{2.882992in}{0.901347in}}{\pgfqpoint{2.887382in}{0.911946in}}{\pgfqpoint{2.887382in}{0.922997in}}%
\pgfpathcurveto{\pgfqpoint{2.887382in}{0.934047in}}{\pgfqpoint{2.882992in}{0.944646in}}{\pgfqpoint{2.875179in}{0.952459in}}%
\pgfpathcurveto{\pgfqpoint{2.867365in}{0.960273in}}{\pgfqpoint{2.856766in}{0.964663in}}{\pgfqpoint{2.845716in}{0.964663in}}%
\pgfpathcurveto{\pgfqpoint{2.834666in}{0.964663in}}{\pgfqpoint{2.824067in}{0.960273in}}{\pgfqpoint{2.816253in}{0.952459in}}%
\pgfpathcurveto{\pgfqpoint{2.808439in}{0.944646in}}{\pgfqpoint{2.804049in}{0.934047in}}{\pgfqpoint{2.804049in}{0.922997in}}%
\pgfpathcurveto{\pgfqpoint{2.804049in}{0.911946in}}{\pgfqpoint{2.808439in}{0.901347in}}{\pgfqpoint{2.816253in}{0.893534in}}%
\pgfpathcurveto{\pgfqpoint{2.824067in}{0.885720in}}{\pgfqpoint{2.834666in}{0.881330in}}{\pgfqpoint{2.845716in}{0.881330in}}%
\pgfpathlineto{\pgfqpoint{2.845716in}{0.881330in}}%
\pgfpathclose%
\pgfusepath{stroke}%
\end{pgfscope}%
\begin{pgfscope}%
\pgfpathrectangle{\pgfqpoint{0.393053in}{0.375000in}}{\pgfqpoint{6.356833in}{5.175000in}}%
\pgfusepath{clip}%
\pgfsetbuttcap%
\pgfsetroundjoin%
\pgfsetlinewidth{1.003750pt}%
\definecolor{currentstroke}{rgb}{1.000000,0.000000,0.000000}%
\pgfsetstrokecolor{currentstroke}%
\pgfsetdash{}{0pt}%
\pgfpathmoveto{\pgfqpoint{0.507359in}{3.548096in}}%
\pgfpathcurveto{\pgfqpoint{0.518409in}{3.548096in}}{\pgfqpoint{0.529008in}{3.552486in}}{\pgfqpoint{0.536821in}{3.560300in}}%
\pgfpathcurveto{\pgfqpoint{0.544635in}{3.568113in}}{\pgfqpoint{0.549025in}{3.578712in}}{\pgfqpoint{0.549025in}{3.589762in}}%
\pgfpathcurveto{\pgfqpoint{0.549025in}{3.600813in}}{\pgfqpoint{0.544635in}{3.611412in}}{\pgfqpoint{0.536821in}{3.619225in}}%
\pgfpathcurveto{\pgfqpoint{0.529008in}{3.627039in}}{\pgfqpoint{0.518409in}{3.631429in}}{\pgfqpoint{0.507359in}{3.631429in}}%
\pgfpathcurveto{\pgfqpoint{0.496309in}{3.631429in}}{\pgfqpoint{0.485710in}{3.627039in}}{\pgfqpoint{0.477896in}{3.619225in}}%
\pgfpathcurveto{\pgfqpoint{0.470082in}{3.611412in}}{\pgfqpoint{0.465692in}{3.600813in}}{\pgfqpoint{0.465692in}{3.589762in}}%
\pgfpathcurveto{\pgfqpoint{0.465692in}{3.578712in}}{\pgfqpoint{0.470082in}{3.568113in}}{\pgfqpoint{0.477896in}{3.560300in}}%
\pgfpathcurveto{\pgfqpoint{0.485710in}{3.552486in}}{\pgfqpoint{0.496309in}{3.548096in}}{\pgfqpoint{0.507359in}{3.548096in}}%
\pgfpathlineto{\pgfqpoint{0.507359in}{3.548096in}}%
\pgfpathclose%
\pgfusepath{stroke}%
\end{pgfscope}%
\begin{pgfscope}%
\pgfpathrectangle{\pgfqpoint{0.393053in}{0.375000in}}{\pgfqpoint{6.356833in}{5.175000in}}%
\pgfusepath{clip}%
\pgfsetbuttcap%
\pgfsetroundjoin%
\pgfsetlinewidth{1.003750pt}%
\definecolor{currentstroke}{rgb}{1.000000,0.000000,0.000000}%
\pgfsetstrokecolor{currentstroke}%
\pgfsetdash{}{0pt}%
\pgfpathmoveto{\pgfqpoint{4.234509in}{0.446167in}}%
\pgfpathcurveto{\pgfqpoint{4.245559in}{0.446167in}}{\pgfqpoint{4.256158in}{0.450558in}}{\pgfqpoint{4.263972in}{0.458371in}}%
\pgfpathcurveto{\pgfqpoint{4.271786in}{0.466185in}}{\pgfqpoint{4.276176in}{0.476784in}}{\pgfqpoint{4.276176in}{0.487834in}}%
\pgfpathcurveto{\pgfqpoint{4.276176in}{0.498884in}}{\pgfqpoint{4.271786in}{0.509483in}}{\pgfqpoint{4.263972in}{0.517297in}}%
\pgfpathcurveto{\pgfqpoint{4.256158in}{0.525110in}}{\pgfqpoint{4.245559in}{0.529501in}}{\pgfqpoint{4.234509in}{0.529501in}}%
\pgfpathcurveto{\pgfqpoint{4.223459in}{0.529501in}}{\pgfqpoint{4.212860in}{0.525110in}}{\pgfqpoint{4.205046in}{0.517297in}}%
\pgfpathcurveto{\pgfqpoint{4.197233in}{0.509483in}}{\pgfqpoint{4.192842in}{0.498884in}}{\pgfqpoint{4.192842in}{0.487834in}}%
\pgfpathcurveto{\pgfqpoint{4.192842in}{0.476784in}}{\pgfqpoint{4.197233in}{0.466185in}}{\pgfqpoint{4.205046in}{0.458371in}}%
\pgfpathcurveto{\pgfqpoint{4.212860in}{0.450558in}}{\pgfqpoint{4.223459in}{0.446167in}}{\pgfqpoint{4.234509in}{0.446167in}}%
\pgfpathlineto{\pgfqpoint{4.234509in}{0.446167in}}%
\pgfpathclose%
\pgfusepath{stroke}%
\end{pgfscope}%
\begin{pgfscope}%
\pgfpathrectangle{\pgfqpoint{0.393053in}{0.375000in}}{\pgfqpoint{6.356833in}{5.175000in}}%
\pgfusepath{clip}%
\pgfsetbuttcap%
\pgfsetroundjoin%
\pgfsetlinewidth{1.003750pt}%
\definecolor{currentstroke}{rgb}{1.000000,0.000000,0.000000}%
\pgfsetstrokecolor{currentstroke}%
\pgfsetdash{}{0pt}%
\pgfpathmoveto{\pgfqpoint{0.819978in}{2.699360in}}%
\pgfpathcurveto{\pgfqpoint{0.831028in}{2.699360in}}{\pgfqpoint{0.841627in}{2.703750in}}{\pgfqpoint{0.849440in}{2.711564in}}%
\pgfpathcurveto{\pgfqpoint{0.857254in}{2.719377in}}{\pgfqpoint{0.861644in}{2.729976in}}{\pgfqpoint{0.861644in}{2.741026in}}%
\pgfpathcurveto{\pgfqpoint{0.861644in}{2.752077in}}{\pgfqpoint{0.857254in}{2.762676in}}{\pgfqpoint{0.849440in}{2.770489in}}%
\pgfpathcurveto{\pgfqpoint{0.841627in}{2.778303in}}{\pgfqpoint{0.831028in}{2.782693in}}{\pgfqpoint{0.819978in}{2.782693in}}%
\pgfpathcurveto{\pgfqpoint{0.808927in}{2.782693in}}{\pgfqpoint{0.798328in}{2.778303in}}{\pgfqpoint{0.790515in}{2.770489in}}%
\pgfpathcurveto{\pgfqpoint{0.782701in}{2.762676in}}{\pgfqpoint{0.778311in}{2.752077in}}{\pgfqpoint{0.778311in}{2.741026in}}%
\pgfpathcurveto{\pgfqpoint{0.778311in}{2.729976in}}{\pgfqpoint{0.782701in}{2.719377in}}{\pgfqpoint{0.790515in}{2.711564in}}%
\pgfpathcurveto{\pgfqpoint{0.798328in}{2.703750in}}{\pgfqpoint{0.808927in}{2.699360in}}{\pgfqpoint{0.819978in}{2.699360in}}%
\pgfpathlineto{\pgfqpoint{0.819978in}{2.699360in}}%
\pgfpathclose%
\pgfusepath{stroke}%
\end{pgfscope}%
\begin{pgfscope}%
\pgfpathrectangle{\pgfqpoint{0.393053in}{0.375000in}}{\pgfqpoint{6.356833in}{5.175000in}}%
\pgfusepath{clip}%
\pgfsetbuttcap%
\pgfsetroundjoin%
\pgfsetlinewidth{1.003750pt}%
\definecolor{currentstroke}{rgb}{1.000000,0.000000,0.000000}%
\pgfsetstrokecolor{currentstroke}%
\pgfsetdash{}{0pt}%
\pgfpathmoveto{\pgfqpoint{3.337184in}{0.672497in}}%
\pgfpathcurveto{\pgfqpoint{3.348234in}{0.672497in}}{\pgfqpoint{3.358833in}{0.676887in}}{\pgfqpoint{3.366647in}{0.684701in}}%
\pgfpathcurveto{\pgfqpoint{3.374460in}{0.692515in}}{\pgfqpoint{3.378850in}{0.703114in}}{\pgfqpoint{3.378850in}{0.714164in}}%
\pgfpathcurveto{\pgfqpoint{3.378850in}{0.725214in}}{\pgfqpoint{3.374460in}{0.735813in}}{\pgfqpoint{3.366647in}{0.743627in}}%
\pgfpathcurveto{\pgfqpoint{3.358833in}{0.751440in}}{\pgfqpoint{3.348234in}{0.755830in}}{\pgfqpoint{3.337184in}{0.755830in}}%
\pgfpathcurveto{\pgfqpoint{3.326134in}{0.755830in}}{\pgfqpoint{3.315535in}{0.751440in}}{\pgfqpoint{3.307721in}{0.743627in}}%
\pgfpathcurveto{\pgfqpoint{3.299907in}{0.735813in}}{\pgfqpoint{3.295517in}{0.725214in}}{\pgfqpoint{3.295517in}{0.714164in}}%
\pgfpathcurveto{\pgfqpoint{3.295517in}{0.703114in}}{\pgfqpoint{3.299907in}{0.692515in}}{\pgfqpoint{3.307721in}{0.684701in}}%
\pgfpathcurveto{\pgfqpoint{3.315535in}{0.676887in}}{\pgfqpoint{3.326134in}{0.672497in}}{\pgfqpoint{3.337184in}{0.672497in}}%
\pgfpathlineto{\pgfqpoint{3.337184in}{0.672497in}}%
\pgfpathclose%
\pgfusepath{stroke}%
\end{pgfscope}%
\begin{pgfscope}%
\pgfpathrectangle{\pgfqpoint{0.393053in}{0.375000in}}{\pgfqpoint{6.356833in}{5.175000in}}%
\pgfusepath{clip}%
\pgfsetbuttcap%
\pgfsetroundjoin%
\pgfsetlinewidth{1.003750pt}%
\definecolor{currentstroke}{rgb}{1.000000,0.000000,0.000000}%
\pgfsetstrokecolor{currentstroke}%
\pgfsetdash{}{0pt}%
\pgfpathmoveto{\pgfqpoint{1.945388in}{1.410432in}}%
\pgfpathcurveto{\pgfqpoint{1.956438in}{1.410432in}}{\pgfqpoint{1.967037in}{1.414822in}}{\pgfqpoint{1.974851in}{1.422636in}}%
\pgfpathcurveto{\pgfqpoint{1.982664in}{1.430449in}}{\pgfqpoint{1.987054in}{1.441048in}}{\pgfqpoint{1.987054in}{1.452099in}}%
\pgfpathcurveto{\pgfqpoint{1.987054in}{1.463149in}}{\pgfqpoint{1.982664in}{1.473748in}}{\pgfqpoint{1.974851in}{1.481561in}}%
\pgfpathcurveto{\pgfqpoint{1.967037in}{1.489375in}}{\pgfqpoint{1.956438in}{1.493765in}}{\pgfqpoint{1.945388in}{1.493765in}}%
\pgfpathcurveto{\pgfqpoint{1.934338in}{1.493765in}}{\pgfqpoint{1.923739in}{1.489375in}}{\pgfqpoint{1.915925in}{1.481561in}}%
\pgfpathcurveto{\pgfqpoint{1.908111in}{1.473748in}}{\pgfqpoint{1.903721in}{1.463149in}}{\pgfqpoint{1.903721in}{1.452099in}}%
\pgfpathcurveto{\pgfqpoint{1.903721in}{1.441048in}}{\pgfqpoint{1.908111in}{1.430449in}}{\pgfqpoint{1.915925in}{1.422636in}}%
\pgfpathcurveto{\pgfqpoint{1.923739in}{1.414822in}}{\pgfqpoint{1.934338in}{1.410432in}}{\pgfqpoint{1.945388in}{1.410432in}}%
\pgfpathlineto{\pgfqpoint{1.945388in}{1.410432in}}%
\pgfpathclose%
\pgfusepath{stroke}%
\end{pgfscope}%
\begin{pgfscope}%
\pgfpathrectangle{\pgfqpoint{0.393053in}{0.375000in}}{\pgfqpoint{6.356833in}{5.175000in}}%
\pgfusepath{clip}%
\pgfsetbuttcap%
\pgfsetroundjoin%
\pgfsetlinewidth{1.003750pt}%
\definecolor{currentstroke}{rgb}{1.000000,0.000000,0.000000}%
\pgfsetstrokecolor{currentstroke}%
\pgfsetdash{}{0pt}%
\pgfpathmoveto{\pgfqpoint{0.536239in}{3.427006in}}%
\pgfpathcurveto{\pgfqpoint{0.547289in}{3.427006in}}{\pgfqpoint{0.557888in}{3.431396in}}{\pgfqpoint{0.565702in}{3.439210in}}%
\pgfpathcurveto{\pgfqpoint{0.573515in}{3.447024in}}{\pgfqpoint{0.577905in}{3.457623in}}{\pgfqpoint{0.577905in}{3.468673in}}%
\pgfpathcurveto{\pgfqpoint{0.577905in}{3.479723in}}{\pgfqpoint{0.573515in}{3.490322in}}{\pgfqpoint{0.565702in}{3.498136in}}%
\pgfpathcurveto{\pgfqpoint{0.557888in}{3.505949in}}{\pgfqpoint{0.547289in}{3.510339in}}{\pgfqpoint{0.536239in}{3.510339in}}%
\pgfpathcurveto{\pgfqpoint{0.525189in}{3.510339in}}{\pgfqpoint{0.514590in}{3.505949in}}{\pgfqpoint{0.506776in}{3.498136in}}%
\pgfpathcurveto{\pgfqpoint{0.498962in}{3.490322in}}{\pgfqpoint{0.494572in}{3.479723in}}{\pgfqpoint{0.494572in}{3.468673in}}%
\pgfpathcurveto{\pgfqpoint{0.494572in}{3.457623in}}{\pgfqpoint{0.498962in}{3.447024in}}{\pgfqpoint{0.506776in}{3.439210in}}%
\pgfpathcurveto{\pgfqpoint{0.514590in}{3.431396in}}{\pgfqpoint{0.525189in}{3.427006in}}{\pgfqpoint{0.536239in}{3.427006in}}%
\pgfpathlineto{\pgfqpoint{0.536239in}{3.427006in}}%
\pgfpathclose%
\pgfusepath{stroke}%
\end{pgfscope}%
\begin{pgfscope}%
\pgfpathrectangle{\pgfqpoint{0.393053in}{0.375000in}}{\pgfqpoint{6.356833in}{5.175000in}}%
\pgfusepath{clip}%
\pgfsetbuttcap%
\pgfsetroundjoin%
\pgfsetlinewidth{1.003750pt}%
\definecolor{currentstroke}{rgb}{1.000000,0.000000,0.000000}%
\pgfsetstrokecolor{currentstroke}%
\pgfsetdash{}{0pt}%
\pgfpathmoveto{\pgfqpoint{1.416963in}{1.864366in}}%
\pgfpathcurveto{\pgfqpoint{1.428013in}{1.864366in}}{\pgfqpoint{1.438612in}{1.868756in}}{\pgfqpoint{1.446426in}{1.876570in}}%
\pgfpathcurveto{\pgfqpoint{1.454240in}{1.884384in}}{\pgfqpoint{1.458630in}{1.894983in}}{\pgfqpoint{1.458630in}{1.906033in}}%
\pgfpathcurveto{\pgfqpoint{1.458630in}{1.917083in}}{\pgfqpoint{1.454240in}{1.927682in}}{\pgfqpoint{1.446426in}{1.935496in}}%
\pgfpathcurveto{\pgfqpoint{1.438612in}{1.943309in}}{\pgfqpoint{1.428013in}{1.947699in}}{\pgfqpoint{1.416963in}{1.947699in}}%
\pgfpathcurveto{\pgfqpoint{1.405913in}{1.947699in}}{\pgfqpoint{1.395314in}{1.943309in}}{\pgfqpoint{1.387501in}{1.935496in}}%
\pgfpathcurveto{\pgfqpoint{1.379687in}{1.927682in}}{\pgfqpoint{1.375297in}{1.917083in}}{\pgfqpoint{1.375297in}{1.906033in}}%
\pgfpathcurveto{\pgfqpoint{1.375297in}{1.894983in}}{\pgfqpoint{1.379687in}{1.884384in}}{\pgfqpoint{1.387501in}{1.876570in}}%
\pgfpathcurveto{\pgfqpoint{1.395314in}{1.868756in}}{\pgfqpoint{1.405913in}{1.864366in}}{\pgfqpoint{1.416963in}{1.864366in}}%
\pgfpathlineto{\pgfqpoint{1.416963in}{1.864366in}}%
\pgfpathclose%
\pgfusepath{stroke}%
\end{pgfscope}%
\begin{pgfscope}%
\pgfpathrectangle{\pgfqpoint{0.393053in}{0.375000in}}{\pgfqpoint{6.356833in}{5.175000in}}%
\pgfusepath{clip}%
\pgfsetbuttcap%
\pgfsetroundjoin%
\pgfsetlinewidth{1.003750pt}%
\definecolor{currentstroke}{rgb}{1.000000,0.000000,0.000000}%
\pgfsetstrokecolor{currentstroke}%
\pgfsetdash{}{0pt}%
\pgfpathmoveto{\pgfqpoint{4.660728in}{0.388931in}}%
\pgfpathcurveto{\pgfqpoint{4.671778in}{0.388931in}}{\pgfqpoint{4.682377in}{0.393321in}}{\pgfqpoint{4.690190in}{0.401135in}}%
\pgfpathcurveto{\pgfqpoint{4.698004in}{0.408948in}}{\pgfqpoint{4.702394in}{0.419547in}}{\pgfqpoint{4.702394in}{0.430598in}}%
\pgfpathcurveto{\pgfqpoint{4.702394in}{0.441648in}}{\pgfqpoint{4.698004in}{0.452247in}}{\pgfqpoint{4.690190in}{0.460060in}}%
\pgfpathcurveto{\pgfqpoint{4.682377in}{0.467874in}}{\pgfqpoint{4.671778in}{0.472264in}}{\pgfqpoint{4.660728in}{0.472264in}}%
\pgfpathcurveto{\pgfqpoint{4.649678in}{0.472264in}}{\pgfqpoint{4.639079in}{0.467874in}}{\pgfqpoint{4.631265in}{0.460060in}}%
\pgfpathcurveto{\pgfqpoint{4.623451in}{0.452247in}}{\pgfqpoint{4.619061in}{0.441648in}}{\pgfqpoint{4.619061in}{0.430598in}}%
\pgfpathcurveto{\pgfqpoint{4.619061in}{0.419547in}}{\pgfqpoint{4.623451in}{0.408948in}}{\pgfqpoint{4.631265in}{0.401135in}}%
\pgfpathcurveto{\pgfqpoint{4.639079in}{0.393321in}}{\pgfqpoint{4.649678in}{0.388931in}}{\pgfqpoint{4.660728in}{0.388931in}}%
\pgfpathlineto{\pgfqpoint{4.660728in}{0.388931in}}%
\pgfpathclose%
\pgfusepath{stroke}%
\end{pgfscope}%
\begin{pgfscope}%
\pgfpathrectangle{\pgfqpoint{0.393053in}{0.375000in}}{\pgfqpoint{6.356833in}{5.175000in}}%
\pgfusepath{clip}%
\pgfsetbuttcap%
\pgfsetroundjoin%
\pgfsetlinewidth{1.003750pt}%
\definecolor{currentstroke}{rgb}{1.000000,0.000000,0.000000}%
\pgfsetstrokecolor{currentstroke}%
\pgfsetdash{}{0pt}%
\pgfpathmoveto{\pgfqpoint{2.322980in}{1.153725in}}%
\pgfpathcurveto{\pgfqpoint{2.334030in}{1.153725in}}{\pgfqpoint{2.344629in}{1.158116in}}{\pgfqpoint{2.352443in}{1.165929in}}%
\pgfpathcurveto{\pgfqpoint{2.360256in}{1.173743in}}{\pgfqpoint{2.364646in}{1.184342in}}{\pgfqpoint{2.364646in}{1.195392in}}%
\pgfpathcurveto{\pgfqpoint{2.364646in}{1.206442in}}{\pgfqpoint{2.360256in}{1.217041in}}{\pgfqpoint{2.352443in}{1.224855in}}%
\pgfpathcurveto{\pgfqpoint{2.344629in}{1.232669in}}{\pgfqpoint{2.334030in}{1.237059in}}{\pgfqpoint{2.322980in}{1.237059in}}%
\pgfpathcurveto{\pgfqpoint{2.311930in}{1.237059in}}{\pgfqpoint{2.301331in}{1.232669in}}{\pgfqpoint{2.293517in}{1.224855in}}%
\pgfpathcurveto{\pgfqpoint{2.285703in}{1.217041in}}{\pgfqpoint{2.281313in}{1.206442in}}{\pgfqpoint{2.281313in}{1.195392in}}%
\pgfpathcurveto{\pgfqpoint{2.281313in}{1.184342in}}{\pgfqpoint{2.285703in}{1.173743in}}{\pgfqpoint{2.293517in}{1.165929in}}%
\pgfpathcurveto{\pgfqpoint{2.301331in}{1.158116in}}{\pgfqpoint{2.311930in}{1.153725in}}{\pgfqpoint{2.322980in}{1.153725in}}%
\pgfpathlineto{\pgfqpoint{2.322980in}{1.153725in}}%
\pgfpathclose%
\pgfusepath{stroke}%
\end{pgfscope}%
\begin{pgfscope}%
\pgfpathrectangle{\pgfqpoint{0.393053in}{0.375000in}}{\pgfqpoint{6.356833in}{5.175000in}}%
\pgfusepath{clip}%
\pgfsetbuttcap%
\pgfsetroundjoin%
\pgfsetlinewidth{1.003750pt}%
\definecolor{currentstroke}{rgb}{1.000000,0.000000,0.000000}%
\pgfsetstrokecolor{currentstroke}%
\pgfsetdash{}{0pt}%
\pgfpathmoveto{\pgfqpoint{5.569525in}{0.335319in}}%
\pgfpathcurveto{\pgfqpoint{5.580575in}{0.335319in}}{\pgfqpoint{5.591174in}{0.339709in}}{\pgfqpoint{5.598988in}{0.347523in}}%
\pgfpathcurveto{\pgfqpoint{5.606801in}{0.355337in}}{\pgfqpoint{5.611192in}{0.365936in}}{\pgfqpoint{5.611192in}{0.376986in}}%
\pgfpathcurveto{\pgfqpoint{5.611192in}{0.388036in}}{\pgfqpoint{5.606801in}{0.398635in}}{\pgfqpoint{5.598988in}{0.406448in}}%
\pgfpathcurveto{\pgfqpoint{5.591174in}{0.414262in}}{\pgfqpoint{5.580575in}{0.418652in}}{\pgfqpoint{5.569525in}{0.418652in}}%
\pgfpathcurveto{\pgfqpoint{5.558475in}{0.418652in}}{\pgfqpoint{5.547876in}{0.414262in}}{\pgfqpoint{5.540062in}{0.406448in}}%
\pgfpathcurveto{\pgfqpoint{5.532249in}{0.398635in}}{\pgfqpoint{5.527858in}{0.388036in}}{\pgfqpoint{5.527858in}{0.376986in}}%
\pgfpathcurveto{\pgfqpoint{5.527858in}{0.365936in}}{\pgfqpoint{5.532249in}{0.355337in}}{\pgfqpoint{5.540062in}{0.347523in}}%
\pgfpathcurveto{\pgfqpoint{5.547876in}{0.339709in}}{\pgfqpoint{5.558475in}{0.335319in}}{\pgfqpoint{5.569525in}{0.335319in}}%
\pgfusepath{stroke}%
\end{pgfscope}%
\begin{pgfscope}%
\pgfpathrectangle{\pgfqpoint{0.393053in}{0.375000in}}{\pgfqpoint{6.356833in}{5.175000in}}%
\pgfusepath{clip}%
\pgfsetbuttcap%
\pgfsetroundjoin%
\pgfsetlinewidth{1.003750pt}%
\definecolor{currentstroke}{rgb}{1.000000,0.000000,0.000000}%
\pgfsetstrokecolor{currentstroke}%
\pgfsetdash{}{0pt}%
\pgfpathmoveto{\pgfqpoint{0.431863in}{3.962367in}}%
\pgfpathcurveto{\pgfqpoint{0.442913in}{3.962367in}}{\pgfqpoint{0.453512in}{3.966758in}}{\pgfqpoint{0.461325in}{3.974571in}}%
\pgfpathcurveto{\pgfqpoint{0.469139in}{3.982385in}}{\pgfqpoint{0.473529in}{3.992984in}}{\pgfqpoint{0.473529in}{4.004034in}}%
\pgfpathcurveto{\pgfqpoint{0.473529in}{4.015084in}}{\pgfqpoint{0.469139in}{4.025683in}}{\pgfqpoint{0.461325in}{4.033497in}}%
\pgfpathcurveto{\pgfqpoint{0.453512in}{4.041310in}}{\pgfqpoint{0.442913in}{4.045701in}}{\pgfqpoint{0.431863in}{4.045701in}}%
\pgfpathcurveto{\pgfqpoint{0.420813in}{4.045701in}}{\pgfqpoint{0.410213in}{4.041310in}}{\pgfqpoint{0.402400in}{4.033497in}}%
\pgfpathcurveto{\pgfqpoint{0.394586in}{4.025683in}}{\pgfqpoint{0.390196in}{4.015084in}}{\pgfqpoint{0.390196in}{4.004034in}}%
\pgfpathcurveto{\pgfqpoint{0.390196in}{3.992984in}}{\pgfqpoint{0.394586in}{3.982385in}}{\pgfqpoint{0.402400in}{3.974571in}}%
\pgfpathcurveto{\pgfqpoint{0.410213in}{3.966758in}}{\pgfqpoint{0.420813in}{3.962367in}}{\pgfqpoint{0.431863in}{3.962367in}}%
\pgfpathlineto{\pgfqpoint{0.431863in}{3.962367in}}%
\pgfpathclose%
\pgfusepath{stroke}%
\end{pgfscope}%
\begin{pgfscope}%
\pgfpathrectangle{\pgfqpoint{0.393053in}{0.375000in}}{\pgfqpoint{6.356833in}{5.175000in}}%
\pgfusepath{clip}%
\pgfsetbuttcap%
\pgfsetroundjoin%
\pgfsetlinewidth{1.003750pt}%
\definecolor{currentstroke}{rgb}{1.000000,0.000000,0.000000}%
\pgfsetstrokecolor{currentstroke}%
\pgfsetdash{}{0pt}%
\pgfpathmoveto{\pgfqpoint{1.994358in}{1.371224in}}%
\pgfpathcurveto{\pgfqpoint{2.005408in}{1.371224in}}{\pgfqpoint{2.016007in}{1.375615in}}{\pgfqpoint{2.023820in}{1.383428in}}%
\pgfpathcurveto{\pgfqpoint{2.031634in}{1.391242in}}{\pgfqpoint{2.036024in}{1.401841in}}{\pgfqpoint{2.036024in}{1.412891in}}%
\pgfpathcurveto{\pgfqpoint{2.036024in}{1.423941in}}{\pgfqpoint{2.031634in}{1.434540in}}{\pgfqpoint{2.023820in}{1.442354in}}%
\pgfpathcurveto{\pgfqpoint{2.016007in}{1.450167in}}{\pgfqpoint{2.005408in}{1.454558in}}{\pgfqpoint{1.994358in}{1.454558in}}%
\pgfpathcurveto{\pgfqpoint{1.983307in}{1.454558in}}{\pgfqpoint{1.972708in}{1.450167in}}{\pgfqpoint{1.964895in}{1.442354in}}%
\pgfpathcurveto{\pgfqpoint{1.957081in}{1.434540in}}{\pgfqpoint{1.952691in}{1.423941in}}{\pgfqpoint{1.952691in}{1.412891in}}%
\pgfpathcurveto{\pgfqpoint{1.952691in}{1.401841in}}{\pgfqpoint{1.957081in}{1.391242in}}{\pgfqpoint{1.964895in}{1.383428in}}%
\pgfpathcurveto{\pgfqpoint{1.972708in}{1.375615in}}{\pgfqpoint{1.983307in}{1.371224in}}{\pgfqpoint{1.994358in}{1.371224in}}%
\pgfpathlineto{\pgfqpoint{1.994358in}{1.371224in}}%
\pgfpathclose%
\pgfusepath{stroke}%
\end{pgfscope}%
\begin{pgfscope}%
\pgfpathrectangle{\pgfqpoint{0.393053in}{0.375000in}}{\pgfqpoint{6.356833in}{5.175000in}}%
\pgfusepath{clip}%
\pgfsetbuttcap%
\pgfsetroundjoin%
\pgfsetlinewidth{1.003750pt}%
\definecolor{currentstroke}{rgb}{1.000000,0.000000,0.000000}%
\pgfsetstrokecolor{currentstroke}%
\pgfsetdash{}{0pt}%
\pgfpathmoveto{\pgfqpoint{2.424610in}{1.139652in}}%
\pgfpathcurveto{\pgfqpoint{2.435660in}{1.139652in}}{\pgfqpoint{2.446259in}{1.144042in}}{\pgfqpoint{2.454073in}{1.151856in}}%
\pgfpathcurveto{\pgfqpoint{2.461887in}{1.159669in}}{\pgfqpoint{2.466277in}{1.170268in}}{\pgfqpoint{2.466277in}{1.181319in}}%
\pgfpathcurveto{\pgfqpoint{2.466277in}{1.192369in}}{\pgfqpoint{2.461887in}{1.202968in}}{\pgfqpoint{2.454073in}{1.210781in}}%
\pgfpathcurveto{\pgfqpoint{2.446259in}{1.218595in}}{\pgfqpoint{2.435660in}{1.222985in}}{\pgfqpoint{2.424610in}{1.222985in}}%
\pgfpathcurveto{\pgfqpoint{2.413560in}{1.222985in}}{\pgfqpoint{2.402961in}{1.218595in}}{\pgfqpoint{2.395148in}{1.210781in}}%
\pgfpathcurveto{\pgfqpoint{2.387334in}{1.202968in}}{\pgfqpoint{2.382944in}{1.192369in}}{\pgfqpoint{2.382944in}{1.181319in}}%
\pgfpathcurveto{\pgfqpoint{2.382944in}{1.170268in}}{\pgfqpoint{2.387334in}{1.159669in}}{\pgfqpoint{2.395148in}{1.151856in}}%
\pgfpathcurveto{\pgfqpoint{2.402961in}{1.144042in}}{\pgfqpoint{2.413560in}{1.139652in}}{\pgfqpoint{2.424610in}{1.139652in}}%
\pgfpathlineto{\pgfqpoint{2.424610in}{1.139652in}}%
\pgfpathclose%
\pgfusepath{stroke}%
\end{pgfscope}%
\begin{pgfscope}%
\pgfpathrectangle{\pgfqpoint{0.393053in}{0.375000in}}{\pgfqpoint{6.356833in}{5.175000in}}%
\pgfusepath{clip}%
\pgfsetbuttcap%
\pgfsetroundjoin%
\pgfsetlinewidth{1.003750pt}%
\definecolor{currentstroke}{rgb}{1.000000,0.000000,0.000000}%
\pgfsetstrokecolor{currentstroke}%
\pgfsetdash{}{0pt}%
\pgfpathmoveto{\pgfqpoint{3.605552in}{0.596674in}}%
\pgfpathcurveto{\pgfqpoint{3.616602in}{0.596674in}}{\pgfqpoint{3.627201in}{0.601064in}}{\pgfqpoint{3.635015in}{0.608877in}}%
\pgfpathcurveto{\pgfqpoint{3.642829in}{0.616691in}}{\pgfqpoint{3.647219in}{0.627290in}}{\pgfqpoint{3.647219in}{0.638340in}}%
\pgfpathcurveto{\pgfqpoint{3.647219in}{0.649390in}}{\pgfqpoint{3.642829in}{0.659989in}}{\pgfqpoint{3.635015in}{0.667803in}}%
\pgfpathcurveto{\pgfqpoint{3.627201in}{0.675617in}}{\pgfqpoint{3.616602in}{0.680007in}}{\pgfqpoint{3.605552in}{0.680007in}}%
\pgfpathcurveto{\pgfqpoint{3.594502in}{0.680007in}}{\pgfqpoint{3.583903in}{0.675617in}}{\pgfqpoint{3.576090in}{0.667803in}}%
\pgfpathcurveto{\pgfqpoint{3.568276in}{0.659989in}}{\pgfqpoint{3.563886in}{0.649390in}}{\pgfqpoint{3.563886in}{0.638340in}}%
\pgfpathcurveto{\pgfqpoint{3.563886in}{0.627290in}}{\pgfqpoint{3.568276in}{0.616691in}}{\pgfqpoint{3.576090in}{0.608877in}}%
\pgfpathcurveto{\pgfqpoint{3.583903in}{0.601064in}}{\pgfqpoint{3.594502in}{0.596674in}}{\pgfqpoint{3.605552in}{0.596674in}}%
\pgfpathlineto{\pgfqpoint{3.605552in}{0.596674in}}%
\pgfpathclose%
\pgfusepath{stroke}%
\end{pgfscope}%
\begin{pgfscope}%
\pgfpathrectangle{\pgfqpoint{0.393053in}{0.375000in}}{\pgfqpoint{6.356833in}{5.175000in}}%
\pgfusepath{clip}%
\pgfsetbuttcap%
\pgfsetroundjoin%
\pgfsetlinewidth{1.003750pt}%
\definecolor{currentstroke}{rgb}{1.000000,0.000000,0.000000}%
\pgfsetstrokecolor{currentstroke}%
\pgfsetdash{}{0pt}%
\pgfpathmoveto{\pgfqpoint{0.402974in}{4.269825in}}%
\pgfpathcurveto{\pgfqpoint{0.414024in}{4.269825in}}{\pgfqpoint{0.424623in}{4.274215in}}{\pgfqpoint{0.432437in}{4.282029in}}%
\pgfpathcurveto{\pgfqpoint{0.440251in}{4.289842in}}{\pgfqpoint{0.444641in}{4.300441in}}{\pgfqpoint{0.444641in}{4.311491in}}%
\pgfpathcurveto{\pgfqpoint{0.444641in}{4.322541in}}{\pgfqpoint{0.440251in}{4.333141in}}{\pgfqpoint{0.432437in}{4.340954in}}%
\pgfpathcurveto{\pgfqpoint{0.424623in}{4.348768in}}{\pgfqpoint{0.414024in}{4.353158in}}{\pgfqpoint{0.402974in}{4.353158in}}%
\pgfpathcurveto{\pgfqpoint{0.391924in}{4.353158in}}{\pgfqpoint{0.381325in}{4.348768in}}{\pgfqpoint{0.373511in}{4.340954in}}%
\pgfpathcurveto{\pgfqpoint{0.365698in}{4.333141in}}{\pgfqpoint{0.361308in}{4.322541in}}{\pgfqpoint{0.361308in}{4.311491in}}%
\pgfpathcurveto{\pgfqpoint{0.361308in}{4.300441in}}{\pgfqpoint{0.365698in}{4.289842in}}{\pgfqpoint{0.373511in}{4.282029in}}%
\pgfpathcurveto{\pgfqpoint{0.381325in}{4.274215in}}{\pgfqpoint{0.391924in}{4.269825in}}{\pgfqpoint{0.402974in}{4.269825in}}%
\pgfpathlineto{\pgfqpoint{0.402974in}{4.269825in}}%
\pgfpathclose%
\pgfusepath{stroke}%
\end{pgfscope}%
\begin{pgfscope}%
\pgfpathrectangle{\pgfqpoint{0.393053in}{0.375000in}}{\pgfqpoint{6.356833in}{5.175000in}}%
\pgfusepath{clip}%
\pgfsetbuttcap%
\pgfsetroundjoin%
\pgfsetlinewidth{1.003750pt}%
\definecolor{currentstroke}{rgb}{1.000000,0.000000,0.000000}%
\pgfsetstrokecolor{currentstroke}%
\pgfsetdash{}{0pt}%
\pgfpathmoveto{\pgfqpoint{1.869103in}{1.460707in}}%
\pgfpathcurveto{\pgfqpoint{1.880153in}{1.460707in}}{\pgfqpoint{1.890752in}{1.465097in}}{\pgfqpoint{1.898566in}{1.472911in}}%
\pgfpathcurveto{\pgfqpoint{1.906379in}{1.480725in}}{\pgfqpoint{1.910769in}{1.491324in}}{\pgfqpoint{1.910769in}{1.502374in}}%
\pgfpathcurveto{\pgfqpoint{1.910769in}{1.513424in}}{\pgfqpoint{1.906379in}{1.524023in}}{\pgfqpoint{1.898566in}{1.531837in}}%
\pgfpathcurveto{\pgfqpoint{1.890752in}{1.539650in}}{\pgfqpoint{1.880153in}{1.544040in}}{\pgfqpoint{1.869103in}{1.544040in}}%
\pgfpathcurveto{\pgfqpoint{1.858053in}{1.544040in}}{\pgfqpoint{1.847454in}{1.539650in}}{\pgfqpoint{1.839640in}{1.531837in}}%
\pgfpathcurveto{\pgfqpoint{1.831826in}{1.524023in}}{\pgfqpoint{1.827436in}{1.513424in}}{\pgfqpoint{1.827436in}{1.502374in}}%
\pgfpathcurveto{\pgfqpoint{1.827436in}{1.491324in}}{\pgfqpoint{1.831826in}{1.480725in}}{\pgfqpoint{1.839640in}{1.472911in}}%
\pgfpathcurveto{\pgfqpoint{1.847454in}{1.465097in}}{\pgfqpoint{1.858053in}{1.460707in}}{\pgfqpoint{1.869103in}{1.460707in}}%
\pgfpathlineto{\pgfqpoint{1.869103in}{1.460707in}}%
\pgfpathclose%
\pgfusepath{stroke}%
\end{pgfscope}%
\begin{pgfscope}%
\pgfpathrectangle{\pgfqpoint{0.393053in}{0.375000in}}{\pgfqpoint{6.356833in}{5.175000in}}%
\pgfusepath{clip}%
\pgfsetbuttcap%
\pgfsetroundjoin%
\pgfsetlinewidth{1.003750pt}%
\definecolor{currentstroke}{rgb}{1.000000,0.000000,0.000000}%
\pgfsetstrokecolor{currentstroke}%
\pgfsetdash{}{0pt}%
\pgfpathmoveto{\pgfqpoint{0.867345in}{2.604654in}}%
\pgfpathcurveto{\pgfqpoint{0.878395in}{2.604654in}}{\pgfqpoint{0.888994in}{2.609045in}}{\pgfqpoint{0.896808in}{2.616858in}}%
\pgfpathcurveto{\pgfqpoint{0.904622in}{2.624672in}}{\pgfqpoint{0.909012in}{2.635271in}}{\pgfqpoint{0.909012in}{2.646321in}}%
\pgfpathcurveto{\pgfqpoint{0.909012in}{2.657371in}}{\pgfqpoint{0.904622in}{2.667970in}}{\pgfqpoint{0.896808in}{2.675784in}}%
\pgfpathcurveto{\pgfqpoint{0.888994in}{2.683597in}}{\pgfqpoint{0.878395in}{2.687988in}}{\pgfqpoint{0.867345in}{2.687988in}}%
\pgfpathcurveto{\pgfqpoint{0.856295in}{2.687988in}}{\pgfqpoint{0.845696in}{2.683597in}}{\pgfqpoint{0.837882in}{2.675784in}}%
\pgfpathcurveto{\pgfqpoint{0.830069in}{2.667970in}}{\pgfqpoint{0.825679in}{2.657371in}}{\pgfqpoint{0.825679in}{2.646321in}}%
\pgfpathcurveto{\pgfqpoint{0.825679in}{2.635271in}}{\pgfqpoint{0.830069in}{2.624672in}}{\pgfqpoint{0.837882in}{2.616858in}}%
\pgfpathcurveto{\pgfqpoint{0.845696in}{2.609045in}}{\pgfqpoint{0.856295in}{2.604654in}}{\pgfqpoint{0.867345in}{2.604654in}}%
\pgfpathlineto{\pgfqpoint{0.867345in}{2.604654in}}%
\pgfpathclose%
\pgfusepath{stroke}%
\end{pgfscope}%
\begin{pgfscope}%
\pgfpathrectangle{\pgfqpoint{0.393053in}{0.375000in}}{\pgfqpoint{6.356833in}{5.175000in}}%
\pgfusepath{clip}%
\pgfsetbuttcap%
\pgfsetroundjoin%
\pgfsetlinewidth{1.003750pt}%
\definecolor{currentstroke}{rgb}{1.000000,0.000000,0.000000}%
\pgfsetstrokecolor{currentstroke}%
\pgfsetdash{}{0pt}%
\pgfpathmoveto{\pgfqpoint{4.088192in}{0.471984in}}%
\pgfpathcurveto{\pgfqpoint{4.099242in}{0.471984in}}{\pgfqpoint{4.109841in}{0.476374in}}{\pgfqpoint{4.117655in}{0.484188in}}%
\pgfpathcurveto{\pgfqpoint{4.125468in}{0.492001in}}{\pgfqpoint{4.129858in}{0.502600in}}{\pgfqpoint{4.129858in}{0.513650in}}%
\pgfpathcurveto{\pgfqpoint{4.129858in}{0.524701in}}{\pgfqpoint{4.125468in}{0.535300in}}{\pgfqpoint{4.117655in}{0.543113in}}%
\pgfpathcurveto{\pgfqpoint{4.109841in}{0.550927in}}{\pgfqpoint{4.099242in}{0.555317in}}{\pgfqpoint{4.088192in}{0.555317in}}%
\pgfpathcurveto{\pgfqpoint{4.077142in}{0.555317in}}{\pgfqpoint{4.066543in}{0.550927in}}{\pgfqpoint{4.058729in}{0.543113in}}%
\pgfpathcurveto{\pgfqpoint{4.050915in}{0.535300in}}{\pgfqpoint{4.046525in}{0.524701in}}{\pgfqpoint{4.046525in}{0.513650in}}%
\pgfpathcurveto{\pgfqpoint{4.046525in}{0.502600in}}{\pgfqpoint{4.050915in}{0.492001in}}{\pgfqpoint{4.058729in}{0.484188in}}%
\pgfpathcurveto{\pgfqpoint{4.066543in}{0.476374in}}{\pgfqpoint{4.077142in}{0.471984in}}{\pgfqpoint{4.088192in}{0.471984in}}%
\pgfpathlineto{\pgfqpoint{4.088192in}{0.471984in}}%
\pgfpathclose%
\pgfusepath{stroke}%
\end{pgfscope}%
\begin{pgfscope}%
\pgfpathrectangle{\pgfqpoint{0.393053in}{0.375000in}}{\pgfqpoint{6.356833in}{5.175000in}}%
\pgfusepath{clip}%
\pgfsetbuttcap%
\pgfsetroundjoin%
\pgfsetlinewidth{1.003750pt}%
\definecolor{currentstroke}{rgb}{1.000000,0.000000,0.000000}%
\pgfsetstrokecolor{currentstroke}%
\pgfsetdash{}{0pt}%
\pgfpathmoveto{\pgfqpoint{2.547235in}{1.029858in}}%
\pgfpathcurveto{\pgfqpoint{2.558285in}{1.029858in}}{\pgfqpoint{2.568885in}{1.034248in}}{\pgfqpoint{2.576698in}{1.042062in}}%
\pgfpathcurveto{\pgfqpoint{2.584512in}{1.049876in}}{\pgfqpoint{2.588902in}{1.060475in}}{\pgfqpoint{2.588902in}{1.071525in}}%
\pgfpathcurveto{\pgfqpoint{2.588902in}{1.082575in}}{\pgfqpoint{2.584512in}{1.093174in}}{\pgfqpoint{2.576698in}{1.100988in}}%
\pgfpathcurveto{\pgfqpoint{2.568885in}{1.108801in}}{\pgfqpoint{2.558285in}{1.113191in}}{\pgfqpoint{2.547235in}{1.113191in}}%
\pgfpathcurveto{\pgfqpoint{2.536185in}{1.113191in}}{\pgfqpoint{2.525586in}{1.108801in}}{\pgfqpoint{2.517773in}{1.100988in}}%
\pgfpathcurveto{\pgfqpoint{2.509959in}{1.093174in}}{\pgfqpoint{2.505569in}{1.082575in}}{\pgfqpoint{2.505569in}{1.071525in}}%
\pgfpathcurveto{\pgfqpoint{2.505569in}{1.060475in}}{\pgfqpoint{2.509959in}{1.049876in}}{\pgfqpoint{2.517773in}{1.042062in}}%
\pgfpathcurveto{\pgfqpoint{2.525586in}{1.034248in}}{\pgfqpoint{2.536185in}{1.029858in}}{\pgfqpoint{2.547235in}{1.029858in}}%
\pgfpathlineto{\pgfqpoint{2.547235in}{1.029858in}}%
\pgfpathclose%
\pgfusepath{stroke}%
\end{pgfscope}%
\begin{pgfscope}%
\pgfpathrectangle{\pgfqpoint{0.393053in}{0.375000in}}{\pgfqpoint{6.356833in}{5.175000in}}%
\pgfusepath{clip}%
\pgfsetbuttcap%
\pgfsetroundjoin%
\pgfsetlinewidth{1.003750pt}%
\definecolor{currentstroke}{rgb}{1.000000,0.000000,0.000000}%
\pgfsetstrokecolor{currentstroke}%
\pgfsetdash{}{0pt}%
\pgfpathmoveto{\pgfqpoint{0.626407in}{3.136794in}}%
\pgfpathcurveto{\pgfqpoint{0.637457in}{3.136794in}}{\pgfqpoint{0.648057in}{3.141184in}}{\pgfqpoint{0.655870in}{3.148998in}}%
\pgfpathcurveto{\pgfqpoint{0.663684in}{3.156812in}}{\pgfqpoint{0.668074in}{3.167411in}}{\pgfqpoint{0.668074in}{3.178461in}}%
\pgfpathcurveto{\pgfqpoint{0.668074in}{3.189511in}}{\pgfqpoint{0.663684in}{3.200110in}}{\pgfqpoint{0.655870in}{3.207924in}}%
\pgfpathcurveto{\pgfqpoint{0.648057in}{3.215737in}}{\pgfqpoint{0.637457in}{3.220127in}}{\pgfqpoint{0.626407in}{3.220127in}}%
\pgfpathcurveto{\pgfqpoint{0.615357in}{3.220127in}}{\pgfqpoint{0.604758in}{3.215737in}}{\pgfqpoint{0.596945in}{3.207924in}}%
\pgfpathcurveto{\pgfqpoint{0.589131in}{3.200110in}}{\pgfqpoint{0.584741in}{3.189511in}}{\pgfqpoint{0.584741in}{3.178461in}}%
\pgfpathcurveto{\pgfqpoint{0.584741in}{3.167411in}}{\pgfqpoint{0.589131in}{3.156812in}}{\pgfqpoint{0.596945in}{3.148998in}}%
\pgfpathcurveto{\pgfqpoint{0.604758in}{3.141184in}}{\pgfqpoint{0.615357in}{3.136794in}}{\pgfqpoint{0.626407in}{3.136794in}}%
\pgfpathlineto{\pgfqpoint{0.626407in}{3.136794in}}%
\pgfpathclose%
\pgfusepath{stroke}%
\end{pgfscope}%
\begin{pgfscope}%
\pgfpathrectangle{\pgfqpoint{0.393053in}{0.375000in}}{\pgfqpoint{6.356833in}{5.175000in}}%
\pgfusepath{clip}%
\pgfsetbuttcap%
\pgfsetroundjoin%
\pgfsetlinewidth{1.003750pt}%
\definecolor{currentstroke}{rgb}{1.000000,0.000000,0.000000}%
\pgfsetstrokecolor{currentstroke}%
\pgfsetdash{}{0pt}%
\pgfpathmoveto{\pgfqpoint{1.833412in}{1.488977in}}%
\pgfpathcurveto{\pgfqpoint{1.844463in}{1.488977in}}{\pgfqpoint{1.855062in}{1.493367in}}{\pgfqpoint{1.862875in}{1.501181in}}%
\pgfpathcurveto{\pgfqpoint{1.870689in}{1.508994in}}{\pgfqpoint{1.875079in}{1.519593in}}{\pgfqpoint{1.875079in}{1.530644in}}%
\pgfpathcurveto{\pgfqpoint{1.875079in}{1.541694in}}{\pgfqpoint{1.870689in}{1.552293in}}{\pgfqpoint{1.862875in}{1.560106in}}%
\pgfpathcurveto{\pgfqpoint{1.855062in}{1.567920in}}{\pgfqpoint{1.844463in}{1.572310in}}{\pgfqpoint{1.833412in}{1.572310in}}%
\pgfpathcurveto{\pgfqpoint{1.822362in}{1.572310in}}{\pgfqpoint{1.811763in}{1.567920in}}{\pgfqpoint{1.803950in}{1.560106in}}%
\pgfpathcurveto{\pgfqpoint{1.796136in}{1.552293in}}{\pgfqpoint{1.791746in}{1.541694in}}{\pgfqpoint{1.791746in}{1.530644in}}%
\pgfpathcurveto{\pgfqpoint{1.791746in}{1.519593in}}{\pgfqpoint{1.796136in}{1.508994in}}{\pgfqpoint{1.803950in}{1.501181in}}%
\pgfpathcurveto{\pgfqpoint{1.811763in}{1.493367in}}{\pgfqpoint{1.822362in}{1.488977in}}{\pgfqpoint{1.833412in}{1.488977in}}%
\pgfpathlineto{\pgfqpoint{1.833412in}{1.488977in}}%
\pgfpathclose%
\pgfusepath{stroke}%
\end{pgfscope}%
\begin{pgfscope}%
\pgfpathrectangle{\pgfqpoint{0.393053in}{0.375000in}}{\pgfqpoint{6.356833in}{5.175000in}}%
\pgfusepath{clip}%
\pgfsetbuttcap%
\pgfsetroundjoin%
\pgfsetlinewidth{1.003750pt}%
\definecolor{currentstroke}{rgb}{1.000000,0.000000,0.000000}%
\pgfsetstrokecolor{currentstroke}%
\pgfsetdash{}{0pt}%
\pgfpathmoveto{\pgfqpoint{3.955488in}{0.501844in}}%
\pgfpathcurveto{\pgfqpoint{3.966538in}{0.501844in}}{\pgfqpoint{3.977137in}{0.506234in}}{\pgfqpoint{3.984951in}{0.514048in}}%
\pgfpathcurveto{\pgfqpoint{3.992764in}{0.521861in}}{\pgfqpoint{3.997155in}{0.532460in}}{\pgfqpoint{3.997155in}{0.543510in}}%
\pgfpathcurveto{\pgfqpoint{3.997155in}{0.554560in}}{\pgfqpoint{3.992764in}{0.565160in}}{\pgfqpoint{3.984951in}{0.572973in}}%
\pgfpathcurveto{\pgfqpoint{3.977137in}{0.580787in}}{\pgfqpoint{3.966538in}{0.585177in}}{\pgfqpoint{3.955488in}{0.585177in}}%
\pgfpathcurveto{\pgfqpoint{3.944438in}{0.585177in}}{\pgfqpoint{3.933839in}{0.580787in}}{\pgfqpoint{3.926025in}{0.572973in}}%
\pgfpathcurveto{\pgfqpoint{3.918211in}{0.565160in}}{\pgfqpoint{3.913821in}{0.554560in}}{\pgfqpoint{3.913821in}{0.543510in}}%
\pgfpathcurveto{\pgfqpoint{3.913821in}{0.532460in}}{\pgfqpoint{3.918211in}{0.521861in}}{\pgfqpoint{3.926025in}{0.514048in}}%
\pgfpathcurveto{\pgfqpoint{3.933839in}{0.506234in}}{\pgfqpoint{3.944438in}{0.501844in}}{\pgfqpoint{3.955488in}{0.501844in}}%
\pgfpathlineto{\pgfqpoint{3.955488in}{0.501844in}}%
\pgfpathclose%
\pgfusepath{stroke}%
\end{pgfscope}%
\begin{pgfscope}%
\pgfpathrectangle{\pgfqpoint{0.393053in}{0.375000in}}{\pgfqpoint{6.356833in}{5.175000in}}%
\pgfusepath{clip}%
\pgfsetbuttcap%
\pgfsetroundjoin%
\pgfsetlinewidth{1.003750pt}%
\definecolor{currentstroke}{rgb}{1.000000,0.000000,0.000000}%
\pgfsetstrokecolor{currentstroke}%
\pgfsetdash{}{0pt}%
\pgfpathmoveto{\pgfqpoint{4.925395in}{0.364627in}}%
\pgfpathcurveto{\pgfqpoint{4.936445in}{0.364627in}}{\pgfqpoint{4.947045in}{0.369017in}}{\pgfqpoint{4.954858in}{0.376830in}}%
\pgfpathcurveto{\pgfqpoint{4.962672in}{0.384644in}}{\pgfqpoint{4.967062in}{0.395243in}}{\pgfqpoint{4.967062in}{0.406293in}}%
\pgfpathcurveto{\pgfqpoint{4.967062in}{0.417343in}}{\pgfqpoint{4.962672in}{0.427942in}}{\pgfqpoint{4.954858in}{0.435756in}}%
\pgfpathcurveto{\pgfqpoint{4.947045in}{0.443570in}}{\pgfqpoint{4.936445in}{0.447960in}}{\pgfqpoint{4.925395in}{0.447960in}}%
\pgfpathcurveto{\pgfqpoint{4.914345in}{0.447960in}}{\pgfqpoint{4.903746in}{0.443570in}}{\pgfqpoint{4.895933in}{0.435756in}}%
\pgfpathcurveto{\pgfqpoint{4.888119in}{0.427942in}}{\pgfqpoint{4.883729in}{0.417343in}}{\pgfqpoint{4.883729in}{0.406293in}}%
\pgfpathcurveto{\pgfqpoint{4.883729in}{0.395243in}}{\pgfqpoint{4.888119in}{0.384644in}}{\pgfqpoint{4.895933in}{0.376830in}}%
\pgfpathcurveto{\pgfqpoint{4.903746in}{0.369017in}}{\pgfqpoint{4.914345in}{0.364627in}}{\pgfqpoint{4.925395in}{0.364627in}}%
\pgfusepath{stroke}%
\end{pgfscope}%
\begin{pgfscope}%
\pgfpathrectangle{\pgfqpoint{0.393053in}{0.375000in}}{\pgfqpoint{6.356833in}{5.175000in}}%
\pgfusepath{clip}%
\pgfsetbuttcap%
\pgfsetroundjoin%
\pgfsetlinewidth{1.003750pt}%
\definecolor{currentstroke}{rgb}{1.000000,0.000000,0.000000}%
\pgfsetstrokecolor{currentstroke}%
\pgfsetdash{}{0pt}%
\pgfpathmoveto{\pgfqpoint{1.351543in}{1.932427in}}%
\pgfpathcurveto{\pgfqpoint{1.362593in}{1.932427in}}{\pgfqpoint{1.373192in}{1.936817in}}{\pgfqpoint{1.381005in}{1.944631in}}%
\pgfpathcurveto{\pgfqpoint{1.388819in}{1.952444in}}{\pgfqpoint{1.393209in}{1.963043in}}{\pgfqpoint{1.393209in}{1.974093in}}%
\pgfpathcurveto{\pgfqpoint{1.393209in}{1.985143in}}{\pgfqpoint{1.388819in}{1.995742in}}{\pgfqpoint{1.381005in}{2.003556in}}%
\pgfpathcurveto{\pgfqpoint{1.373192in}{2.011370in}}{\pgfqpoint{1.362593in}{2.015760in}}{\pgfqpoint{1.351543in}{2.015760in}}%
\pgfpathcurveto{\pgfqpoint{1.340492in}{2.015760in}}{\pgfqpoint{1.329893in}{2.011370in}}{\pgfqpoint{1.322080in}{2.003556in}}%
\pgfpathcurveto{\pgfqpoint{1.314266in}{1.995742in}}{\pgfqpoint{1.309876in}{1.985143in}}{\pgfqpoint{1.309876in}{1.974093in}}%
\pgfpathcurveto{\pgfqpoint{1.309876in}{1.963043in}}{\pgfqpoint{1.314266in}{1.952444in}}{\pgfqpoint{1.322080in}{1.944631in}}%
\pgfpathcurveto{\pgfqpoint{1.329893in}{1.936817in}}{\pgfqpoint{1.340492in}{1.932427in}}{\pgfqpoint{1.351543in}{1.932427in}}%
\pgfpathlineto{\pgfqpoint{1.351543in}{1.932427in}}%
\pgfpathclose%
\pgfusepath{stroke}%
\end{pgfscope}%
\begin{pgfscope}%
\pgfpathrectangle{\pgfqpoint{0.393053in}{0.375000in}}{\pgfqpoint{6.356833in}{5.175000in}}%
\pgfusepath{clip}%
\pgfsetbuttcap%
\pgfsetroundjoin%
\pgfsetlinewidth{1.003750pt}%
\definecolor{currentstroke}{rgb}{1.000000,0.000000,0.000000}%
\pgfsetstrokecolor{currentstroke}%
\pgfsetdash{}{0pt}%
\pgfpathmoveto{\pgfqpoint{1.168538in}{2.152416in}}%
\pgfpathcurveto{\pgfqpoint{1.179589in}{2.152416in}}{\pgfqpoint{1.190188in}{2.156807in}}{\pgfqpoint{1.198001in}{2.164620in}}%
\pgfpathcurveto{\pgfqpoint{1.205815in}{2.172434in}}{\pgfqpoint{1.210205in}{2.183033in}}{\pgfqpoint{1.210205in}{2.194083in}}%
\pgfpathcurveto{\pgfqpoint{1.210205in}{2.205133in}}{\pgfqpoint{1.205815in}{2.215732in}}{\pgfqpoint{1.198001in}{2.223546in}}%
\pgfpathcurveto{\pgfqpoint{1.190188in}{2.231360in}}{\pgfqpoint{1.179589in}{2.235750in}}{\pgfqpoint{1.168538in}{2.235750in}}%
\pgfpathcurveto{\pgfqpoint{1.157488in}{2.235750in}}{\pgfqpoint{1.146889in}{2.231360in}}{\pgfqpoint{1.139076in}{2.223546in}}%
\pgfpathcurveto{\pgfqpoint{1.131262in}{2.215732in}}{\pgfqpoint{1.126872in}{2.205133in}}{\pgfqpoint{1.126872in}{2.194083in}}%
\pgfpathcurveto{\pgfqpoint{1.126872in}{2.183033in}}{\pgfqpoint{1.131262in}{2.172434in}}{\pgfqpoint{1.139076in}{2.164620in}}%
\pgfpathcurveto{\pgfqpoint{1.146889in}{2.156807in}}{\pgfqpoint{1.157488in}{2.152416in}}{\pgfqpoint{1.168538in}{2.152416in}}%
\pgfpathlineto{\pgfqpoint{1.168538in}{2.152416in}}%
\pgfpathclose%
\pgfusepath{stroke}%
\end{pgfscope}%
\begin{pgfscope}%
\pgfpathrectangle{\pgfqpoint{0.393053in}{0.375000in}}{\pgfqpoint{6.356833in}{5.175000in}}%
\pgfusepath{clip}%
\pgfsetbuttcap%
\pgfsetroundjoin%
\pgfsetlinewidth{1.003750pt}%
\definecolor{currentstroke}{rgb}{1.000000,0.000000,0.000000}%
\pgfsetstrokecolor{currentstroke}%
\pgfsetdash{}{0pt}%
\pgfpathmoveto{\pgfqpoint{2.055365in}{1.324173in}}%
\pgfpathcurveto{\pgfqpoint{2.066415in}{1.324173in}}{\pgfqpoint{2.077014in}{1.328563in}}{\pgfqpoint{2.084827in}{1.336376in}}%
\pgfpathcurveto{\pgfqpoint{2.092641in}{1.344190in}}{\pgfqpoint{2.097031in}{1.354789in}}{\pgfqpoint{2.097031in}{1.365839in}}%
\pgfpathcurveto{\pgfqpoint{2.097031in}{1.376889in}}{\pgfqpoint{2.092641in}{1.387488in}}{\pgfqpoint{2.084827in}{1.395302in}}%
\pgfpathcurveto{\pgfqpoint{2.077014in}{1.403116in}}{\pgfqpoint{2.066415in}{1.407506in}}{\pgfqpoint{2.055365in}{1.407506in}}%
\pgfpathcurveto{\pgfqpoint{2.044315in}{1.407506in}}{\pgfqpoint{2.033716in}{1.403116in}}{\pgfqpoint{2.025902in}{1.395302in}}%
\pgfpathcurveto{\pgfqpoint{2.018088in}{1.387488in}}{\pgfqpoint{2.013698in}{1.376889in}}{\pgfqpoint{2.013698in}{1.365839in}}%
\pgfpathcurveto{\pgfqpoint{2.013698in}{1.354789in}}{\pgfqpoint{2.018088in}{1.344190in}}{\pgfqpoint{2.025902in}{1.336376in}}%
\pgfpathcurveto{\pgfqpoint{2.033716in}{1.328563in}}{\pgfqpoint{2.044315in}{1.324173in}}{\pgfqpoint{2.055365in}{1.324173in}}%
\pgfpathlineto{\pgfqpoint{2.055365in}{1.324173in}}%
\pgfpathclose%
\pgfusepath{stroke}%
\end{pgfscope}%
\begin{pgfscope}%
\pgfpathrectangle{\pgfqpoint{0.393053in}{0.375000in}}{\pgfqpoint{6.356833in}{5.175000in}}%
\pgfusepath{clip}%
\pgfsetbuttcap%
\pgfsetroundjoin%
\pgfsetlinewidth{1.003750pt}%
\definecolor{currentstroke}{rgb}{1.000000,0.000000,0.000000}%
\pgfsetstrokecolor{currentstroke}%
\pgfsetdash{}{0pt}%
\pgfpathmoveto{\pgfqpoint{0.604931in}{3.199811in}}%
\pgfpathcurveto{\pgfqpoint{0.615981in}{3.199811in}}{\pgfqpoint{0.626580in}{3.204201in}}{\pgfqpoint{0.634394in}{3.212015in}}%
\pgfpathcurveto{\pgfqpoint{0.642207in}{3.219828in}}{\pgfqpoint{0.646597in}{3.230428in}}{\pgfqpoint{0.646597in}{3.241478in}}%
\pgfpathcurveto{\pgfqpoint{0.646597in}{3.252528in}}{\pgfqpoint{0.642207in}{3.263127in}}{\pgfqpoint{0.634394in}{3.270940in}}%
\pgfpathcurveto{\pgfqpoint{0.626580in}{3.278754in}}{\pgfqpoint{0.615981in}{3.283144in}}{\pgfqpoint{0.604931in}{3.283144in}}%
\pgfpathcurveto{\pgfqpoint{0.593881in}{3.283144in}}{\pgfqpoint{0.583282in}{3.278754in}}{\pgfqpoint{0.575468in}{3.270940in}}%
\pgfpathcurveto{\pgfqpoint{0.567654in}{3.263127in}}{\pgfqpoint{0.563264in}{3.252528in}}{\pgfqpoint{0.563264in}{3.241478in}}%
\pgfpathcurveto{\pgfqpoint{0.563264in}{3.230428in}}{\pgfqpoint{0.567654in}{3.219828in}}{\pgfqpoint{0.575468in}{3.212015in}}%
\pgfpathcurveto{\pgfqpoint{0.583282in}{3.204201in}}{\pgfqpoint{0.593881in}{3.199811in}}{\pgfqpoint{0.604931in}{3.199811in}}%
\pgfpathlineto{\pgfqpoint{0.604931in}{3.199811in}}%
\pgfpathclose%
\pgfusepath{stroke}%
\end{pgfscope}%
\begin{pgfscope}%
\pgfpathrectangle{\pgfqpoint{0.393053in}{0.375000in}}{\pgfqpoint{6.356833in}{5.175000in}}%
\pgfusepath{clip}%
\pgfsetbuttcap%
\pgfsetroundjoin%
\pgfsetlinewidth{1.003750pt}%
\definecolor{currentstroke}{rgb}{1.000000,0.000000,0.000000}%
\pgfsetstrokecolor{currentstroke}%
\pgfsetdash{}{0pt}%
\pgfpathmoveto{\pgfqpoint{1.195625in}{2.127045in}}%
\pgfpathcurveto{\pgfqpoint{1.206675in}{2.127045in}}{\pgfqpoint{1.217274in}{2.131435in}}{\pgfqpoint{1.225088in}{2.139249in}}%
\pgfpathcurveto{\pgfqpoint{1.232902in}{2.147062in}}{\pgfqpoint{1.237292in}{2.157661in}}{\pgfqpoint{1.237292in}{2.168711in}}%
\pgfpathcurveto{\pgfqpoint{1.237292in}{2.179762in}}{\pgfqpoint{1.232902in}{2.190361in}}{\pgfqpoint{1.225088in}{2.198174in}}%
\pgfpathcurveto{\pgfqpoint{1.217274in}{2.205988in}}{\pgfqpoint{1.206675in}{2.210378in}}{\pgfqpoint{1.195625in}{2.210378in}}%
\pgfpathcurveto{\pgfqpoint{1.184575in}{2.210378in}}{\pgfqpoint{1.173976in}{2.205988in}}{\pgfqpoint{1.166163in}{2.198174in}}%
\pgfpathcurveto{\pgfqpoint{1.158349in}{2.190361in}}{\pgfqpoint{1.153959in}{2.179762in}}{\pgfqpoint{1.153959in}{2.168711in}}%
\pgfpathcurveto{\pgfqpoint{1.153959in}{2.157661in}}{\pgfqpoint{1.158349in}{2.147062in}}{\pgfqpoint{1.166163in}{2.139249in}}%
\pgfpathcurveto{\pgfqpoint{1.173976in}{2.131435in}}{\pgfqpoint{1.184575in}{2.127045in}}{\pgfqpoint{1.195625in}{2.127045in}}%
\pgfpathlineto{\pgfqpoint{1.195625in}{2.127045in}}%
\pgfpathclose%
\pgfusepath{stroke}%
\end{pgfscope}%
\begin{pgfscope}%
\pgfpathrectangle{\pgfqpoint{0.393053in}{0.375000in}}{\pgfqpoint{6.356833in}{5.175000in}}%
\pgfusepath{clip}%
\pgfsetbuttcap%
\pgfsetroundjoin%
\pgfsetlinewidth{1.003750pt}%
\definecolor{currentstroke}{rgb}{1.000000,0.000000,0.000000}%
\pgfsetstrokecolor{currentstroke}%
\pgfsetdash{}{0pt}%
\pgfpathmoveto{\pgfqpoint{4.119304in}{0.466194in}}%
\pgfpathcurveto{\pgfqpoint{4.130354in}{0.466194in}}{\pgfqpoint{4.140953in}{0.470584in}}{\pgfqpoint{4.148767in}{0.478398in}}%
\pgfpathcurveto{\pgfqpoint{4.156580in}{0.486212in}}{\pgfqpoint{4.160970in}{0.496811in}}{\pgfqpoint{4.160970in}{0.507861in}}%
\pgfpathcurveto{\pgfqpoint{4.160970in}{0.518911in}}{\pgfqpoint{4.156580in}{0.529510in}}{\pgfqpoint{4.148767in}{0.537323in}}%
\pgfpathcurveto{\pgfqpoint{4.140953in}{0.545137in}}{\pgfqpoint{4.130354in}{0.549527in}}{\pgfqpoint{4.119304in}{0.549527in}}%
\pgfpathcurveto{\pgfqpoint{4.108254in}{0.549527in}}{\pgfqpoint{4.097655in}{0.545137in}}{\pgfqpoint{4.089841in}{0.537323in}}%
\pgfpathcurveto{\pgfqpoint{4.082027in}{0.529510in}}{\pgfqpoint{4.077637in}{0.518911in}}{\pgfqpoint{4.077637in}{0.507861in}}%
\pgfpathcurveto{\pgfqpoint{4.077637in}{0.496811in}}{\pgfqpoint{4.082027in}{0.486212in}}{\pgfqpoint{4.089841in}{0.478398in}}%
\pgfpathcurveto{\pgfqpoint{4.097655in}{0.470584in}}{\pgfqpoint{4.108254in}{0.466194in}}{\pgfqpoint{4.119304in}{0.466194in}}%
\pgfpathlineto{\pgfqpoint{4.119304in}{0.466194in}}%
\pgfpathclose%
\pgfusepath{stroke}%
\end{pgfscope}%
\begin{pgfscope}%
\pgfpathrectangle{\pgfqpoint{0.393053in}{0.375000in}}{\pgfqpoint{6.356833in}{5.175000in}}%
\pgfusepath{clip}%
\pgfsetbuttcap%
\pgfsetroundjoin%
\pgfsetlinewidth{1.003750pt}%
\definecolor{currentstroke}{rgb}{1.000000,0.000000,0.000000}%
\pgfsetstrokecolor{currentstroke}%
\pgfsetdash{}{0pt}%
\pgfpathmoveto{\pgfqpoint{3.126428in}{0.755125in}}%
\pgfpathcurveto{\pgfqpoint{3.137478in}{0.755125in}}{\pgfqpoint{3.148077in}{0.759515in}}{\pgfqpoint{3.155891in}{0.767329in}}%
\pgfpathcurveto{\pgfqpoint{3.163705in}{0.775143in}}{\pgfqpoint{3.168095in}{0.785742in}}{\pgfqpoint{3.168095in}{0.796792in}}%
\pgfpathcurveto{\pgfqpoint{3.168095in}{0.807842in}}{\pgfqpoint{3.163705in}{0.818441in}}{\pgfqpoint{3.155891in}{0.826254in}}%
\pgfpathcurveto{\pgfqpoint{3.148077in}{0.834068in}}{\pgfqpoint{3.137478in}{0.838458in}}{\pgfqpoint{3.126428in}{0.838458in}}%
\pgfpathcurveto{\pgfqpoint{3.115378in}{0.838458in}}{\pgfqpoint{3.104779in}{0.834068in}}{\pgfqpoint{3.096965in}{0.826254in}}%
\pgfpathcurveto{\pgfqpoint{3.089152in}{0.818441in}}{\pgfqpoint{3.084762in}{0.807842in}}{\pgfqpoint{3.084762in}{0.796792in}}%
\pgfpathcurveto{\pgfqpoint{3.084762in}{0.785742in}}{\pgfqpoint{3.089152in}{0.775143in}}{\pgfqpoint{3.096965in}{0.767329in}}%
\pgfpathcurveto{\pgfqpoint{3.104779in}{0.759515in}}{\pgfqpoint{3.115378in}{0.755125in}}{\pgfqpoint{3.126428in}{0.755125in}}%
\pgfpathlineto{\pgfqpoint{3.126428in}{0.755125in}}%
\pgfpathclose%
\pgfusepath{stroke}%
\end{pgfscope}%
\begin{pgfscope}%
\pgfpathrectangle{\pgfqpoint{0.393053in}{0.375000in}}{\pgfqpoint{6.356833in}{5.175000in}}%
\pgfusepath{clip}%
\pgfsetbuttcap%
\pgfsetroundjoin%
\pgfsetlinewidth{1.003750pt}%
\definecolor{currentstroke}{rgb}{1.000000,0.000000,0.000000}%
\pgfsetstrokecolor{currentstroke}%
\pgfsetdash{}{0pt}%
\pgfpathmoveto{\pgfqpoint{3.726296in}{0.552386in}}%
\pgfpathcurveto{\pgfqpoint{3.737346in}{0.552386in}}{\pgfqpoint{3.747945in}{0.556776in}}{\pgfqpoint{3.755758in}{0.564590in}}%
\pgfpathcurveto{\pgfqpoint{3.763572in}{0.572404in}}{\pgfqpoint{3.767962in}{0.583003in}}{\pgfqpoint{3.767962in}{0.594053in}}%
\pgfpathcurveto{\pgfqpoint{3.767962in}{0.605103in}}{\pgfqpoint{3.763572in}{0.615702in}}{\pgfqpoint{3.755758in}{0.623516in}}%
\pgfpathcurveto{\pgfqpoint{3.747945in}{0.631329in}}{\pgfqpoint{3.737346in}{0.635719in}}{\pgfqpoint{3.726296in}{0.635719in}}%
\pgfpathcurveto{\pgfqpoint{3.715246in}{0.635719in}}{\pgfqpoint{3.704647in}{0.631329in}}{\pgfqpoint{3.696833in}{0.623516in}}%
\pgfpathcurveto{\pgfqpoint{3.689019in}{0.615702in}}{\pgfqpoint{3.684629in}{0.605103in}}{\pgfqpoint{3.684629in}{0.594053in}}%
\pgfpathcurveto{\pgfqpoint{3.684629in}{0.583003in}}{\pgfqpoint{3.689019in}{0.572404in}}{\pgfqpoint{3.696833in}{0.564590in}}%
\pgfpathcurveto{\pgfqpoint{3.704647in}{0.556776in}}{\pgfqpoint{3.715246in}{0.552386in}}{\pgfqpoint{3.726296in}{0.552386in}}%
\pgfpathlineto{\pgfqpoint{3.726296in}{0.552386in}}%
\pgfpathclose%
\pgfusepath{stroke}%
\end{pgfscope}%
\begin{pgfscope}%
\pgfpathrectangle{\pgfqpoint{0.393053in}{0.375000in}}{\pgfqpoint{6.356833in}{5.175000in}}%
\pgfusepath{clip}%
\pgfsetbuttcap%
\pgfsetroundjoin%
\pgfsetlinewidth{1.003750pt}%
\definecolor{currentstroke}{rgb}{1.000000,0.000000,0.000000}%
\pgfsetstrokecolor{currentstroke}%
\pgfsetdash{}{0pt}%
\pgfpathmoveto{\pgfqpoint{3.230686in}{0.711915in}}%
\pgfpathcurveto{\pgfqpoint{3.241736in}{0.711915in}}{\pgfqpoint{3.252335in}{0.716305in}}{\pgfqpoint{3.260149in}{0.724119in}}%
\pgfpathcurveto{\pgfqpoint{3.267962in}{0.731933in}}{\pgfqpoint{3.272353in}{0.742532in}}{\pgfqpoint{3.272353in}{0.753582in}}%
\pgfpathcurveto{\pgfqpoint{3.272353in}{0.764632in}}{\pgfqpoint{3.267962in}{0.775231in}}{\pgfqpoint{3.260149in}{0.783045in}}%
\pgfpathcurveto{\pgfqpoint{3.252335in}{0.790858in}}{\pgfqpoint{3.241736in}{0.795248in}}{\pgfqpoint{3.230686in}{0.795248in}}%
\pgfpathcurveto{\pgfqpoint{3.219636in}{0.795248in}}{\pgfqpoint{3.209037in}{0.790858in}}{\pgfqpoint{3.201223in}{0.783045in}}%
\pgfpathcurveto{\pgfqpoint{3.193410in}{0.775231in}}{\pgfqpoint{3.189019in}{0.764632in}}{\pgfqpoint{3.189019in}{0.753582in}}%
\pgfpathcurveto{\pgfqpoint{3.189019in}{0.742532in}}{\pgfqpoint{3.193410in}{0.731933in}}{\pgfqpoint{3.201223in}{0.724119in}}%
\pgfpathcurveto{\pgfqpoint{3.209037in}{0.716305in}}{\pgfqpoint{3.219636in}{0.711915in}}{\pgfqpoint{3.230686in}{0.711915in}}%
\pgfpathlineto{\pgfqpoint{3.230686in}{0.711915in}}%
\pgfpathclose%
\pgfusepath{stroke}%
\end{pgfscope}%
\begin{pgfscope}%
\pgfpathrectangle{\pgfqpoint{0.393053in}{0.375000in}}{\pgfqpoint{6.356833in}{5.175000in}}%
\pgfusepath{clip}%
\pgfsetbuttcap%
\pgfsetroundjoin%
\pgfsetlinewidth{1.003750pt}%
\definecolor{currentstroke}{rgb}{1.000000,0.000000,0.000000}%
\pgfsetstrokecolor{currentstroke}%
\pgfsetdash{}{0pt}%
\pgfpathmoveto{\pgfqpoint{1.699353in}{1.610795in}}%
\pgfpathcurveto{\pgfqpoint{1.710403in}{1.610795in}}{\pgfqpoint{1.721002in}{1.615186in}}{\pgfqpoint{1.728816in}{1.622999in}}%
\pgfpathcurveto{\pgfqpoint{1.736630in}{1.630813in}}{\pgfqpoint{1.741020in}{1.641412in}}{\pgfqpoint{1.741020in}{1.652462in}}%
\pgfpathcurveto{\pgfqpoint{1.741020in}{1.663512in}}{\pgfqpoint{1.736630in}{1.674111in}}{\pgfqpoint{1.728816in}{1.681925in}}%
\pgfpathcurveto{\pgfqpoint{1.721002in}{1.689738in}}{\pgfqpoint{1.710403in}{1.694129in}}{\pgfqpoint{1.699353in}{1.694129in}}%
\pgfpathcurveto{\pgfqpoint{1.688303in}{1.694129in}}{\pgfqpoint{1.677704in}{1.689738in}}{\pgfqpoint{1.669891in}{1.681925in}}%
\pgfpathcurveto{\pgfqpoint{1.662077in}{1.674111in}}{\pgfqpoint{1.657687in}{1.663512in}}{\pgfqpoint{1.657687in}{1.652462in}}%
\pgfpathcurveto{\pgfqpoint{1.657687in}{1.641412in}}{\pgfqpoint{1.662077in}{1.630813in}}{\pgfqpoint{1.669891in}{1.622999in}}%
\pgfpathcurveto{\pgfqpoint{1.677704in}{1.615186in}}{\pgfqpoint{1.688303in}{1.610795in}}{\pgfqpoint{1.699353in}{1.610795in}}%
\pgfpathlineto{\pgfqpoint{1.699353in}{1.610795in}}%
\pgfpathclose%
\pgfusepath{stroke}%
\end{pgfscope}%
\begin{pgfscope}%
\pgfpathrectangle{\pgfqpoint{0.393053in}{0.375000in}}{\pgfqpoint{6.356833in}{5.175000in}}%
\pgfusepath{clip}%
\pgfsetbuttcap%
\pgfsetroundjoin%
\pgfsetlinewidth{1.003750pt}%
\definecolor{currentstroke}{rgb}{1.000000,0.000000,0.000000}%
\pgfsetstrokecolor{currentstroke}%
\pgfsetdash{}{0pt}%
\pgfpathmoveto{\pgfqpoint{1.761091in}{1.545819in}}%
\pgfpathcurveto{\pgfqpoint{1.772141in}{1.545819in}}{\pgfqpoint{1.782740in}{1.550210in}}{\pgfqpoint{1.790554in}{1.558023in}}%
\pgfpathcurveto{\pgfqpoint{1.798367in}{1.565837in}}{\pgfqpoint{1.802758in}{1.576436in}}{\pgfqpoint{1.802758in}{1.587486in}}%
\pgfpathcurveto{\pgfqpoint{1.802758in}{1.598536in}}{\pgfqpoint{1.798367in}{1.609135in}}{\pgfqpoint{1.790554in}{1.616949in}}%
\pgfpathcurveto{\pgfqpoint{1.782740in}{1.624763in}}{\pgfqpoint{1.772141in}{1.629153in}}{\pgfqpoint{1.761091in}{1.629153in}}%
\pgfpathcurveto{\pgfqpoint{1.750041in}{1.629153in}}{\pgfqpoint{1.739442in}{1.624763in}}{\pgfqpoint{1.731628in}{1.616949in}}%
\pgfpathcurveto{\pgfqpoint{1.723814in}{1.609135in}}{\pgfqpoint{1.719424in}{1.598536in}}{\pgfqpoint{1.719424in}{1.587486in}}%
\pgfpathcurveto{\pgfqpoint{1.719424in}{1.576436in}}{\pgfqpoint{1.723814in}{1.565837in}}{\pgfqpoint{1.731628in}{1.558023in}}%
\pgfpathcurveto{\pgfqpoint{1.739442in}{1.550210in}}{\pgfqpoint{1.750041in}{1.545819in}}{\pgfqpoint{1.761091in}{1.545819in}}%
\pgfpathlineto{\pgfqpoint{1.761091in}{1.545819in}}%
\pgfpathclose%
\pgfusepath{stroke}%
\end{pgfscope}%
\begin{pgfscope}%
\pgfpathrectangle{\pgfqpoint{0.393053in}{0.375000in}}{\pgfqpoint{6.356833in}{5.175000in}}%
\pgfusepath{clip}%
\pgfsetbuttcap%
\pgfsetroundjoin%
\pgfsetlinewidth{1.003750pt}%
\definecolor{currentstroke}{rgb}{1.000000,0.000000,0.000000}%
\pgfsetstrokecolor{currentstroke}%
\pgfsetdash{}{0pt}%
\pgfpathmoveto{\pgfqpoint{5.426982in}{0.338684in}}%
\pgfpathcurveto{\pgfqpoint{5.438032in}{0.338684in}}{\pgfqpoint{5.448631in}{0.343075in}}{\pgfqpoint{5.456445in}{0.350888in}}%
\pgfpathcurveto{\pgfqpoint{5.464259in}{0.358702in}}{\pgfqpoint{5.468649in}{0.369301in}}{\pgfqpoint{5.468649in}{0.380351in}}%
\pgfpathcurveto{\pgfqpoint{5.468649in}{0.391401in}}{\pgfqpoint{5.464259in}{0.402000in}}{\pgfqpoint{5.456445in}{0.409814in}}%
\pgfpathcurveto{\pgfqpoint{5.448631in}{0.417627in}}{\pgfqpoint{5.438032in}{0.422018in}}{\pgfqpoint{5.426982in}{0.422018in}}%
\pgfpathcurveto{\pgfqpoint{5.415932in}{0.422018in}}{\pgfqpoint{5.405333in}{0.417627in}}{\pgfqpoint{5.397519in}{0.409814in}}%
\pgfpathcurveto{\pgfqpoint{5.389706in}{0.402000in}}{\pgfqpoint{5.385315in}{0.391401in}}{\pgfqpoint{5.385315in}{0.380351in}}%
\pgfpathcurveto{\pgfqpoint{5.385315in}{0.369301in}}{\pgfqpoint{5.389706in}{0.358702in}}{\pgfqpoint{5.397519in}{0.350888in}}%
\pgfpathcurveto{\pgfqpoint{5.405333in}{0.343075in}}{\pgfqpoint{5.415932in}{0.338684in}}{\pgfqpoint{5.426982in}{0.338684in}}%
\pgfusepath{stroke}%
\end{pgfscope}%
\begin{pgfscope}%
\pgfpathrectangle{\pgfqpoint{0.393053in}{0.375000in}}{\pgfqpoint{6.356833in}{5.175000in}}%
\pgfusepath{clip}%
\pgfsetbuttcap%
\pgfsetroundjoin%
\pgfsetlinewidth{1.003750pt}%
\definecolor{currentstroke}{rgb}{1.000000,0.000000,0.000000}%
\pgfsetstrokecolor{currentstroke}%
\pgfsetdash{}{0pt}%
\pgfpathmoveto{\pgfqpoint{3.846323in}{0.524404in}}%
\pgfpathcurveto{\pgfqpoint{3.857373in}{0.524404in}}{\pgfqpoint{3.867973in}{0.528795in}}{\pgfqpoint{3.875786in}{0.536608in}}%
\pgfpathcurveto{\pgfqpoint{3.883600in}{0.544422in}}{\pgfqpoint{3.887990in}{0.555021in}}{\pgfqpoint{3.887990in}{0.566071in}}%
\pgfpathcurveto{\pgfqpoint{3.887990in}{0.577121in}}{\pgfqpoint{3.883600in}{0.587720in}}{\pgfqpoint{3.875786in}{0.595534in}}%
\pgfpathcurveto{\pgfqpoint{3.867973in}{0.603348in}}{\pgfqpoint{3.857373in}{0.607738in}}{\pgfqpoint{3.846323in}{0.607738in}}%
\pgfpathcurveto{\pgfqpoint{3.835273in}{0.607738in}}{\pgfqpoint{3.824674in}{0.603348in}}{\pgfqpoint{3.816861in}{0.595534in}}%
\pgfpathcurveto{\pgfqpoint{3.809047in}{0.587720in}}{\pgfqpoint{3.804657in}{0.577121in}}{\pgfqpoint{3.804657in}{0.566071in}}%
\pgfpathcurveto{\pgfqpoint{3.804657in}{0.555021in}}{\pgfqpoint{3.809047in}{0.544422in}}{\pgfqpoint{3.816861in}{0.536608in}}%
\pgfpathcurveto{\pgfqpoint{3.824674in}{0.528795in}}{\pgfqpoint{3.835273in}{0.524404in}}{\pgfqpoint{3.846323in}{0.524404in}}%
\pgfpathlineto{\pgfqpoint{3.846323in}{0.524404in}}%
\pgfpathclose%
\pgfusepath{stroke}%
\end{pgfscope}%
\begin{pgfscope}%
\pgfpathrectangle{\pgfqpoint{0.393053in}{0.375000in}}{\pgfqpoint{6.356833in}{5.175000in}}%
\pgfusepath{clip}%
\pgfsetbuttcap%
\pgfsetroundjoin%
\pgfsetlinewidth{1.003750pt}%
\definecolor{currentstroke}{rgb}{1.000000,0.000000,0.000000}%
\pgfsetstrokecolor{currentstroke}%
\pgfsetdash{}{0pt}%
\pgfpathmoveto{\pgfqpoint{4.563708in}{0.403232in}}%
\pgfpathcurveto{\pgfqpoint{4.574758in}{0.403232in}}{\pgfqpoint{4.585357in}{0.407622in}}{\pgfqpoint{4.593170in}{0.415436in}}%
\pgfpathcurveto{\pgfqpoint{4.600984in}{0.423250in}}{\pgfqpoint{4.605374in}{0.433849in}}{\pgfqpoint{4.605374in}{0.444899in}}%
\pgfpathcurveto{\pgfqpoint{4.605374in}{0.455949in}}{\pgfqpoint{4.600984in}{0.466548in}}{\pgfqpoint{4.593170in}{0.474362in}}%
\pgfpathcurveto{\pgfqpoint{4.585357in}{0.482175in}}{\pgfqpoint{4.574758in}{0.486565in}}{\pgfqpoint{4.563708in}{0.486565in}}%
\pgfpathcurveto{\pgfqpoint{4.552658in}{0.486565in}}{\pgfqpoint{4.542059in}{0.482175in}}{\pgfqpoint{4.534245in}{0.474362in}}%
\pgfpathcurveto{\pgfqpoint{4.526431in}{0.466548in}}{\pgfqpoint{4.522041in}{0.455949in}}{\pgfqpoint{4.522041in}{0.444899in}}%
\pgfpathcurveto{\pgfqpoint{4.522041in}{0.433849in}}{\pgfqpoint{4.526431in}{0.423250in}}{\pgfqpoint{4.534245in}{0.415436in}}%
\pgfpathcurveto{\pgfqpoint{4.542059in}{0.407622in}}{\pgfqpoint{4.552658in}{0.403232in}}{\pgfqpoint{4.563708in}{0.403232in}}%
\pgfpathlineto{\pgfqpoint{4.563708in}{0.403232in}}%
\pgfpathclose%
\pgfusepath{stroke}%
\end{pgfscope}%
\begin{pgfscope}%
\pgfpathrectangle{\pgfqpoint{0.393053in}{0.375000in}}{\pgfqpoint{6.356833in}{5.175000in}}%
\pgfusepath{clip}%
\pgfsetbuttcap%
\pgfsetroundjoin%
\pgfsetlinewidth{1.003750pt}%
\definecolor{currentstroke}{rgb}{1.000000,0.000000,0.000000}%
\pgfsetstrokecolor{currentstroke}%
\pgfsetdash{}{0pt}%
\pgfpathmoveto{\pgfqpoint{0.648566in}{3.080738in}}%
\pgfpathcurveto{\pgfqpoint{0.659616in}{3.080738in}}{\pgfqpoint{0.670215in}{3.085128in}}{\pgfqpoint{0.678029in}{3.092942in}}%
\pgfpathcurveto{\pgfqpoint{0.685842in}{3.100756in}}{\pgfqpoint{0.690233in}{3.111355in}}{\pgfqpoint{0.690233in}{3.122405in}}%
\pgfpathcurveto{\pgfqpoint{0.690233in}{3.133455in}}{\pgfqpoint{0.685842in}{3.144054in}}{\pgfqpoint{0.678029in}{3.151868in}}%
\pgfpathcurveto{\pgfqpoint{0.670215in}{3.159681in}}{\pgfqpoint{0.659616in}{3.164071in}}{\pgfqpoint{0.648566in}{3.164071in}}%
\pgfpathcurveto{\pgfqpoint{0.637516in}{3.164071in}}{\pgfqpoint{0.626917in}{3.159681in}}{\pgfqpoint{0.619103in}{3.151868in}}%
\pgfpathcurveto{\pgfqpoint{0.611290in}{3.144054in}}{\pgfqpoint{0.606899in}{3.133455in}}{\pgfqpoint{0.606899in}{3.122405in}}%
\pgfpathcurveto{\pgfqpoint{0.606899in}{3.111355in}}{\pgfqpoint{0.611290in}{3.100756in}}{\pgfqpoint{0.619103in}{3.092942in}}%
\pgfpathcurveto{\pgfqpoint{0.626917in}{3.085128in}}{\pgfqpoint{0.637516in}{3.080738in}}{\pgfqpoint{0.648566in}{3.080738in}}%
\pgfpathlineto{\pgfqpoint{0.648566in}{3.080738in}}%
\pgfpathclose%
\pgfusepath{stroke}%
\end{pgfscope}%
\begin{pgfscope}%
\pgfpathrectangle{\pgfqpoint{0.393053in}{0.375000in}}{\pgfqpoint{6.356833in}{5.175000in}}%
\pgfusepath{clip}%
\pgfsetbuttcap%
\pgfsetroundjoin%
\pgfsetlinewidth{1.003750pt}%
\definecolor{currentstroke}{rgb}{1.000000,0.000000,0.000000}%
\pgfsetstrokecolor{currentstroke}%
\pgfsetdash{}{0pt}%
\pgfpathmoveto{\pgfqpoint{1.039854in}{2.323868in}}%
\pgfpathcurveto{\pgfqpoint{1.050904in}{2.323868in}}{\pgfqpoint{1.061503in}{2.328258in}}{\pgfqpoint{1.069317in}{2.336072in}}%
\pgfpathcurveto{\pgfqpoint{1.077131in}{2.343886in}}{\pgfqpoint{1.081521in}{2.354485in}}{\pgfqpoint{1.081521in}{2.365535in}}%
\pgfpathcurveto{\pgfqpoint{1.081521in}{2.376585in}}{\pgfqpoint{1.077131in}{2.387184in}}{\pgfqpoint{1.069317in}{2.394998in}}%
\pgfpathcurveto{\pgfqpoint{1.061503in}{2.402811in}}{\pgfqpoint{1.050904in}{2.407201in}}{\pgfqpoint{1.039854in}{2.407201in}}%
\pgfpathcurveto{\pgfqpoint{1.028804in}{2.407201in}}{\pgfqpoint{1.018205in}{2.402811in}}{\pgfqpoint{1.010391in}{2.394998in}}%
\pgfpathcurveto{\pgfqpoint{1.002578in}{2.387184in}}{\pgfqpoint{0.998187in}{2.376585in}}{\pgfqpoint{0.998187in}{2.365535in}}%
\pgfpathcurveto{\pgfqpoint{0.998187in}{2.354485in}}{\pgfqpoint{1.002578in}{2.343886in}}{\pgfqpoint{1.010391in}{2.336072in}}%
\pgfpathcurveto{\pgfqpoint{1.018205in}{2.328258in}}{\pgfqpoint{1.028804in}{2.323868in}}{\pgfqpoint{1.039854in}{2.323868in}}%
\pgfpathlineto{\pgfqpoint{1.039854in}{2.323868in}}%
\pgfpathclose%
\pgfusepath{stroke}%
\end{pgfscope}%
\begin{pgfscope}%
\pgfpathrectangle{\pgfqpoint{0.393053in}{0.375000in}}{\pgfqpoint{6.356833in}{5.175000in}}%
\pgfusepath{clip}%
\pgfsetbuttcap%
\pgfsetroundjoin%
\pgfsetlinewidth{1.003750pt}%
\definecolor{currentstroke}{rgb}{1.000000,0.000000,0.000000}%
\pgfsetstrokecolor{currentstroke}%
\pgfsetdash{}{0pt}%
\pgfpathmoveto{\pgfqpoint{1.336567in}{1.948608in}}%
\pgfpathcurveto{\pgfqpoint{1.347617in}{1.948608in}}{\pgfqpoint{1.358216in}{1.952998in}}{\pgfqpoint{1.366029in}{1.960812in}}%
\pgfpathcurveto{\pgfqpoint{1.373843in}{1.968626in}}{\pgfqpoint{1.378233in}{1.979225in}}{\pgfqpoint{1.378233in}{1.990275in}}%
\pgfpathcurveto{\pgfqpoint{1.378233in}{2.001325in}}{\pgfqpoint{1.373843in}{2.011924in}}{\pgfqpoint{1.366029in}{2.019738in}}%
\pgfpathcurveto{\pgfqpoint{1.358216in}{2.027551in}}{\pgfqpoint{1.347617in}{2.031942in}}{\pgfqpoint{1.336567in}{2.031942in}}%
\pgfpathcurveto{\pgfqpoint{1.325516in}{2.031942in}}{\pgfqpoint{1.314917in}{2.027551in}}{\pgfqpoint{1.307104in}{2.019738in}}%
\pgfpathcurveto{\pgfqpoint{1.299290in}{2.011924in}}{\pgfqpoint{1.294900in}{2.001325in}}{\pgfqpoint{1.294900in}{1.990275in}}%
\pgfpathcurveto{\pgfqpoint{1.294900in}{1.979225in}}{\pgfqpoint{1.299290in}{1.968626in}}{\pgfqpoint{1.307104in}{1.960812in}}%
\pgfpathcurveto{\pgfqpoint{1.314917in}{1.952998in}}{\pgfqpoint{1.325516in}{1.948608in}}{\pgfqpoint{1.336567in}{1.948608in}}%
\pgfpathlineto{\pgfqpoint{1.336567in}{1.948608in}}%
\pgfpathclose%
\pgfusepath{stroke}%
\end{pgfscope}%
\begin{pgfscope}%
\pgfpathrectangle{\pgfqpoint{0.393053in}{0.375000in}}{\pgfqpoint{6.356833in}{5.175000in}}%
\pgfusepath{clip}%
\pgfsetbuttcap%
\pgfsetroundjoin%
\pgfsetlinewidth{1.003750pt}%
\definecolor{currentstroke}{rgb}{1.000000,0.000000,0.000000}%
\pgfsetstrokecolor{currentstroke}%
\pgfsetdash{}{0pt}%
\pgfpathmoveto{\pgfqpoint{0.575909in}{3.290319in}}%
\pgfpathcurveto{\pgfqpoint{0.586959in}{3.290319in}}{\pgfqpoint{0.597558in}{3.294709in}}{\pgfqpoint{0.605372in}{3.302523in}}%
\pgfpathcurveto{\pgfqpoint{0.613185in}{3.310336in}}{\pgfqpoint{0.617575in}{3.320935in}}{\pgfqpoint{0.617575in}{3.331986in}}%
\pgfpathcurveto{\pgfqpoint{0.617575in}{3.343036in}}{\pgfqpoint{0.613185in}{3.353635in}}{\pgfqpoint{0.605372in}{3.361448in}}%
\pgfpathcurveto{\pgfqpoint{0.597558in}{3.369262in}}{\pgfqpoint{0.586959in}{3.373652in}}{\pgfqpoint{0.575909in}{3.373652in}}%
\pgfpathcurveto{\pgfqpoint{0.564859in}{3.373652in}}{\pgfqpoint{0.554260in}{3.369262in}}{\pgfqpoint{0.546446in}{3.361448in}}%
\pgfpathcurveto{\pgfqpoint{0.538632in}{3.353635in}}{\pgfqpoint{0.534242in}{3.343036in}}{\pgfqpoint{0.534242in}{3.331986in}}%
\pgfpathcurveto{\pgfqpoint{0.534242in}{3.320935in}}{\pgfqpoint{0.538632in}{3.310336in}}{\pgfqpoint{0.546446in}{3.302523in}}%
\pgfpathcurveto{\pgfqpoint{0.554260in}{3.294709in}}{\pgfqpoint{0.564859in}{3.290319in}}{\pgfqpoint{0.575909in}{3.290319in}}%
\pgfpathlineto{\pgfqpoint{0.575909in}{3.290319in}}%
\pgfpathclose%
\pgfusepath{stroke}%
\end{pgfscope}%
\begin{pgfscope}%
\pgfpathrectangle{\pgfqpoint{0.393053in}{0.375000in}}{\pgfqpoint{6.356833in}{5.175000in}}%
\pgfusepath{clip}%
\pgfsetbuttcap%
\pgfsetroundjoin%
\pgfsetlinewidth{1.003750pt}%
\definecolor{currentstroke}{rgb}{1.000000,0.000000,0.000000}%
\pgfsetstrokecolor{currentstroke}%
\pgfsetdash{}{0pt}%
\pgfpathmoveto{\pgfqpoint{3.242649in}{0.707392in}}%
\pgfpathcurveto{\pgfqpoint{3.253699in}{0.707392in}}{\pgfqpoint{3.264298in}{0.711782in}}{\pgfqpoint{3.272112in}{0.719596in}}%
\pgfpathcurveto{\pgfqpoint{3.279925in}{0.727409in}}{\pgfqpoint{3.284315in}{0.738008in}}{\pgfqpoint{3.284315in}{0.749058in}}%
\pgfpathcurveto{\pgfqpoint{3.284315in}{0.760108in}}{\pgfqpoint{3.279925in}{0.770707in}}{\pgfqpoint{3.272112in}{0.778521in}}%
\pgfpathcurveto{\pgfqpoint{3.264298in}{0.786335in}}{\pgfqpoint{3.253699in}{0.790725in}}{\pgfqpoint{3.242649in}{0.790725in}}%
\pgfpathcurveto{\pgfqpoint{3.231599in}{0.790725in}}{\pgfqpoint{3.221000in}{0.786335in}}{\pgfqpoint{3.213186in}{0.778521in}}%
\pgfpathcurveto{\pgfqpoint{3.205372in}{0.770707in}}{\pgfqpoint{3.200982in}{0.760108in}}{\pgfqpoint{3.200982in}{0.749058in}}%
\pgfpathcurveto{\pgfqpoint{3.200982in}{0.738008in}}{\pgfqpoint{3.205372in}{0.727409in}}{\pgfqpoint{3.213186in}{0.719596in}}%
\pgfpathcurveto{\pgfqpoint{3.221000in}{0.711782in}}{\pgfqpoint{3.231599in}{0.707392in}}{\pgfqpoint{3.242649in}{0.707392in}}%
\pgfpathlineto{\pgfqpoint{3.242649in}{0.707392in}}%
\pgfpathclose%
\pgfusepath{stroke}%
\end{pgfscope}%
\begin{pgfscope}%
\pgfpathrectangle{\pgfqpoint{0.393053in}{0.375000in}}{\pgfqpoint{6.356833in}{5.175000in}}%
\pgfusepath{clip}%
\pgfsetbuttcap%
\pgfsetroundjoin%
\pgfsetlinewidth{1.003750pt}%
\definecolor{currentstroke}{rgb}{1.000000,0.000000,0.000000}%
\pgfsetstrokecolor{currentstroke}%
\pgfsetdash{}{0pt}%
\pgfpathmoveto{\pgfqpoint{3.726296in}{0.552386in}}%
\pgfpathcurveto{\pgfqpoint{3.737346in}{0.552386in}}{\pgfqpoint{3.747945in}{0.556776in}}{\pgfqpoint{3.755759in}{0.564590in}}%
\pgfpathcurveto{\pgfqpoint{3.763572in}{0.572404in}}{\pgfqpoint{3.767962in}{0.583003in}}{\pgfqpoint{3.767962in}{0.594053in}}%
\pgfpathcurveto{\pgfqpoint{3.767962in}{0.605103in}}{\pgfqpoint{3.763572in}{0.615702in}}{\pgfqpoint{3.755759in}{0.623515in}}%
\pgfpathcurveto{\pgfqpoint{3.747945in}{0.631329in}}{\pgfqpoint{3.737346in}{0.635719in}}{\pgfqpoint{3.726296in}{0.635719in}}%
\pgfpathcurveto{\pgfqpoint{3.715246in}{0.635719in}}{\pgfqpoint{3.704647in}{0.631329in}}{\pgfqpoint{3.696833in}{0.623515in}}%
\pgfpathcurveto{\pgfqpoint{3.689019in}{0.615702in}}{\pgfqpoint{3.684629in}{0.605103in}}{\pgfqpoint{3.684629in}{0.594053in}}%
\pgfpathcurveto{\pgfqpoint{3.684629in}{0.583003in}}{\pgfqpoint{3.689019in}{0.572404in}}{\pgfqpoint{3.696833in}{0.564590in}}%
\pgfpathcurveto{\pgfqpoint{3.704647in}{0.556776in}}{\pgfqpoint{3.715246in}{0.552386in}}{\pgfqpoint{3.726296in}{0.552386in}}%
\pgfpathlineto{\pgfqpoint{3.726296in}{0.552386in}}%
\pgfpathclose%
\pgfusepath{stroke}%
\end{pgfscope}%
\begin{pgfscope}%
\pgfpathrectangle{\pgfqpoint{0.393053in}{0.375000in}}{\pgfqpoint{6.356833in}{5.175000in}}%
\pgfusepath{clip}%
\pgfsetbuttcap%
\pgfsetroundjoin%
\pgfsetlinewidth{1.003750pt}%
\definecolor{currentstroke}{rgb}{1.000000,0.000000,0.000000}%
\pgfsetstrokecolor{currentstroke}%
\pgfsetdash{}{0pt}%
\pgfpathmoveto{\pgfqpoint{0.663278in}{3.039036in}}%
\pgfpathcurveto{\pgfqpoint{0.674328in}{3.039036in}}{\pgfqpoint{0.684927in}{3.043426in}}{\pgfqpoint{0.692741in}{3.051240in}}%
\pgfpathcurveto{\pgfqpoint{0.700554in}{3.059054in}}{\pgfqpoint{0.704945in}{3.069653in}}{\pgfqpoint{0.704945in}{3.080703in}}%
\pgfpathcurveto{\pgfqpoint{0.704945in}{3.091753in}}{\pgfqpoint{0.700554in}{3.102352in}}{\pgfqpoint{0.692741in}{3.110166in}}%
\pgfpathcurveto{\pgfqpoint{0.684927in}{3.117979in}}{\pgfqpoint{0.674328in}{3.122369in}}{\pgfqpoint{0.663278in}{3.122369in}}%
\pgfpathcurveto{\pgfqpoint{0.652228in}{3.122369in}}{\pgfqpoint{0.641629in}{3.117979in}}{\pgfqpoint{0.633815in}{3.110166in}}%
\pgfpathcurveto{\pgfqpoint{0.626001in}{3.102352in}}{\pgfqpoint{0.621611in}{3.091753in}}{\pgfqpoint{0.621611in}{3.080703in}}%
\pgfpathcurveto{\pgfqpoint{0.621611in}{3.069653in}}{\pgfqpoint{0.626001in}{3.059054in}}{\pgfqpoint{0.633815in}{3.051240in}}%
\pgfpathcurveto{\pgfqpoint{0.641629in}{3.043426in}}{\pgfqpoint{0.652228in}{3.039036in}}{\pgfqpoint{0.663278in}{3.039036in}}%
\pgfpathlineto{\pgfqpoint{0.663278in}{3.039036in}}%
\pgfpathclose%
\pgfusepath{stroke}%
\end{pgfscope}%
\begin{pgfscope}%
\pgfpathrectangle{\pgfqpoint{0.393053in}{0.375000in}}{\pgfqpoint{6.356833in}{5.175000in}}%
\pgfusepath{clip}%
\pgfsetbuttcap%
\pgfsetroundjoin%
\pgfsetlinewidth{1.003750pt}%
\definecolor{currentstroke}{rgb}{1.000000,0.000000,0.000000}%
\pgfsetstrokecolor{currentstroke}%
\pgfsetdash{}{0pt}%
\pgfpathmoveto{\pgfqpoint{0.393173in}{4.558864in}}%
\pgfpathcurveto{\pgfqpoint{0.404224in}{4.558864in}}{\pgfqpoint{0.414823in}{4.563255in}}{\pgfqpoint{0.422636in}{4.571068in}}%
\pgfpathcurveto{\pgfqpoint{0.430450in}{4.578882in}}{\pgfqpoint{0.434840in}{4.589481in}}{\pgfqpoint{0.434840in}{4.600531in}}%
\pgfpathcurveto{\pgfqpoint{0.434840in}{4.611581in}}{\pgfqpoint{0.430450in}{4.622180in}}{\pgfqpoint{0.422636in}{4.629994in}}%
\pgfpathcurveto{\pgfqpoint{0.414823in}{4.637807in}}{\pgfqpoint{0.404224in}{4.642198in}}{\pgfqpoint{0.393173in}{4.642198in}}%
\pgfpathcurveto{\pgfqpoint{0.382123in}{4.642198in}}{\pgfqpoint{0.371524in}{4.637807in}}{\pgfqpoint{0.363711in}{4.629994in}}%
\pgfpathcurveto{\pgfqpoint{0.355897in}{4.622180in}}{\pgfqpoint{0.351507in}{4.611581in}}{\pgfqpoint{0.351507in}{4.600531in}}%
\pgfpathcurveto{\pgfqpoint{0.351507in}{4.589481in}}{\pgfqpoint{0.355897in}{4.578882in}}{\pgfqpoint{0.363711in}{4.571068in}}%
\pgfpathcurveto{\pgfqpoint{0.371524in}{4.563255in}}{\pgfqpoint{0.382123in}{4.558864in}}{\pgfqpoint{0.393173in}{4.558864in}}%
\pgfpathlineto{\pgfqpoint{0.393173in}{4.558864in}}%
\pgfpathclose%
\pgfusepath{stroke}%
\end{pgfscope}%
\begin{pgfscope}%
\pgfpathrectangle{\pgfqpoint{0.393053in}{0.375000in}}{\pgfqpoint{6.356833in}{5.175000in}}%
\pgfusepath{clip}%
\pgfsetbuttcap%
\pgfsetroundjoin%
\pgfsetlinewidth{1.003750pt}%
\definecolor{currentstroke}{rgb}{1.000000,0.000000,0.000000}%
\pgfsetstrokecolor{currentstroke}%
\pgfsetdash{}{0pt}%
\pgfpathmoveto{\pgfqpoint{1.628470in}{1.666295in}}%
\pgfpathcurveto{\pgfqpoint{1.639520in}{1.666295in}}{\pgfqpoint{1.650119in}{1.670685in}}{\pgfqpoint{1.657933in}{1.678498in}}%
\pgfpathcurveto{\pgfqpoint{1.665746in}{1.686312in}}{\pgfqpoint{1.670137in}{1.696911in}}{\pgfqpoint{1.670137in}{1.707961in}}%
\pgfpathcurveto{\pgfqpoint{1.670137in}{1.719011in}}{\pgfqpoint{1.665746in}{1.729610in}}{\pgfqpoint{1.657933in}{1.737424in}}%
\pgfpathcurveto{\pgfqpoint{1.650119in}{1.745238in}}{\pgfqpoint{1.639520in}{1.749628in}}{\pgfqpoint{1.628470in}{1.749628in}}%
\pgfpathcurveto{\pgfqpoint{1.617420in}{1.749628in}}{\pgfqpoint{1.606821in}{1.745238in}}{\pgfqpoint{1.599007in}{1.737424in}}%
\pgfpathcurveto{\pgfqpoint{1.591194in}{1.729610in}}{\pgfqpoint{1.586803in}{1.719011in}}{\pgfqpoint{1.586803in}{1.707961in}}%
\pgfpathcurveto{\pgfqpoint{1.586803in}{1.696911in}}{\pgfqpoint{1.591194in}{1.686312in}}{\pgfqpoint{1.599007in}{1.678498in}}%
\pgfpathcurveto{\pgfqpoint{1.606821in}{1.670685in}}{\pgfqpoint{1.617420in}{1.666295in}}{\pgfqpoint{1.628470in}{1.666295in}}%
\pgfpathlineto{\pgfqpoint{1.628470in}{1.666295in}}%
\pgfpathclose%
\pgfusepath{stroke}%
\end{pgfscope}%
\begin{pgfscope}%
\pgfpathrectangle{\pgfqpoint{0.393053in}{0.375000in}}{\pgfqpoint{6.356833in}{5.175000in}}%
\pgfusepath{clip}%
\pgfsetbuttcap%
\pgfsetroundjoin%
\pgfsetlinewidth{1.003750pt}%
\definecolor{currentstroke}{rgb}{1.000000,0.000000,0.000000}%
\pgfsetstrokecolor{currentstroke}%
\pgfsetdash{}{0pt}%
\pgfpathmoveto{\pgfqpoint{0.467360in}{3.757867in}}%
\pgfpathcurveto{\pgfqpoint{0.478410in}{3.757867in}}{\pgfqpoint{0.489009in}{3.762257in}}{\pgfqpoint{0.496823in}{3.770071in}}%
\pgfpathcurveto{\pgfqpoint{0.504637in}{3.777885in}}{\pgfqpoint{0.509027in}{3.788484in}}{\pgfqpoint{0.509027in}{3.799534in}}%
\pgfpathcurveto{\pgfqpoint{0.509027in}{3.810584in}}{\pgfqpoint{0.504637in}{3.821183in}}{\pgfqpoint{0.496823in}{3.828997in}}%
\pgfpathcurveto{\pgfqpoint{0.489009in}{3.836810in}}{\pgfqpoint{0.478410in}{3.841200in}}{\pgfqpoint{0.467360in}{3.841200in}}%
\pgfpathcurveto{\pgfqpoint{0.456310in}{3.841200in}}{\pgfqpoint{0.445711in}{3.836810in}}{\pgfqpoint{0.437897in}{3.828997in}}%
\pgfpathcurveto{\pgfqpoint{0.430084in}{3.821183in}}{\pgfqpoint{0.425694in}{3.810584in}}{\pgfqpoint{0.425694in}{3.799534in}}%
\pgfpathcurveto{\pgfqpoint{0.425694in}{3.788484in}}{\pgfqpoint{0.430084in}{3.777885in}}{\pgfqpoint{0.437897in}{3.770071in}}%
\pgfpathcurveto{\pgfqpoint{0.445711in}{3.762257in}}{\pgfqpoint{0.456310in}{3.757867in}}{\pgfqpoint{0.467360in}{3.757867in}}%
\pgfpathlineto{\pgfqpoint{0.467360in}{3.757867in}}%
\pgfpathclose%
\pgfusepath{stroke}%
\end{pgfscope}%
\begin{pgfscope}%
\pgfpathrectangle{\pgfqpoint{0.393053in}{0.375000in}}{\pgfqpoint{6.356833in}{5.175000in}}%
\pgfusepath{clip}%
\pgfsetbuttcap%
\pgfsetroundjoin%
\pgfsetlinewidth{1.003750pt}%
\definecolor{currentstroke}{rgb}{1.000000,0.000000,0.000000}%
\pgfsetstrokecolor{currentstroke}%
\pgfsetdash{}{0pt}%
\pgfpathmoveto{\pgfqpoint{1.118909in}{2.213321in}}%
\pgfpathcurveto{\pgfqpoint{1.129959in}{2.213321in}}{\pgfqpoint{1.140558in}{2.217711in}}{\pgfqpoint{1.148372in}{2.225525in}}%
\pgfpathcurveto{\pgfqpoint{1.156186in}{2.233339in}}{\pgfqpoint{1.160576in}{2.243938in}}{\pgfqpoint{1.160576in}{2.254988in}}%
\pgfpathcurveto{\pgfqpoint{1.160576in}{2.266038in}}{\pgfqpoint{1.156186in}{2.276637in}}{\pgfqpoint{1.148372in}{2.284450in}}%
\pgfpathcurveto{\pgfqpoint{1.140558in}{2.292264in}}{\pgfqpoint{1.129959in}{2.296654in}}{\pgfqpoint{1.118909in}{2.296654in}}%
\pgfpathcurveto{\pgfqpoint{1.107859in}{2.296654in}}{\pgfqpoint{1.097260in}{2.292264in}}{\pgfqpoint{1.089446in}{2.284450in}}%
\pgfpathcurveto{\pgfqpoint{1.081633in}{2.276637in}}{\pgfqpoint{1.077242in}{2.266038in}}{\pgfqpoint{1.077242in}{2.254988in}}%
\pgfpathcurveto{\pgfqpoint{1.077242in}{2.243938in}}{\pgfqpoint{1.081633in}{2.233339in}}{\pgfqpoint{1.089446in}{2.225525in}}%
\pgfpathcurveto{\pgfqpoint{1.097260in}{2.217711in}}{\pgfqpoint{1.107859in}{2.213321in}}{\pgfqpoint{1.118909in}{2.213321in}}%
\pgfpathlineto{\pgfqpoint{1.118909in}{2.213321in}}%
\pgfpathclose%
\pgfusepath{stroke}%
\end{pgfscope}%
\begin{pgfscope}%
\pgfpathrectangle{\pgfqpoint{0.393053in}{0.375000in}}{\pgfqpoint{6.356833in}{5.175000in}}%
\pgfusepath{clip}%
\pgfsetbuttcap%
\pgfsetroundjoin%
\pgfsetlinewidth{1.003750pt}%
\definecolor{currentstroke}{rgb}{1.000000,0.000000,0.000000}%
\pgfsetstrokecolor{currentstroke}%
\pgfsetdash{}{0pt}%
\pgfpathmoveto{\pgfqpoint{1.294062in}{2.011442in}}%
\pgfpathcurveto{\pgfqpoint{1.305112in}{2.011442in}}{\pgfqpoint{1.315711in}{2.015832in}}{\pgfqpoint{1.323524in}{2.023646in}}%
\pgfpathcurveto{\pgfqpoint{1.331338in}{2.031459in}}{\pgfqpoint{1.335728in}{2.042058in}}{\pgfqpoint{1.335728in}{2.053108in}}%
\pgfpathcurveto{\pgfqpoint{1.335728in}{2.064159in}}{\pgfqpoint{1.331338in}{2.074758in}}{\pgfqpoint{1.323524in}{2.082571in}}%
\pgfpathcurveto{\pgfqpoint{1.315711in}{2.090385in}}{\pgfqpoint{1.305112in}{2.094775in}}{\pgfqpoint{1.294062in}{2.094775in}}%
\pgfpathcurveto{\pgfqpoint{1.283012in}{2.094775in}}{\pgfqpoint{1.272413in}{2.090385in}}{\pgfqpoint{1.264599in}{2.082571in}}%
\pgfpathcurveto{\pgfqpoint{1.256785in}{2.074758in}}{\pgfqpoint{1.252395in}{2.064159in}}{\pgfqpoint{1.252395in}{2.053108in}}%
\pgfpathcurveto{\pgfqpoint{1.252395in}{2.042058in}}{\pgfqpoint{1.256785in}{2.031459in}}{\pgfqpoint{1.264599in}{2.023646in}}%
\pgfpathcurveto{\pgfqpoint{1.272413in}{2.015832in}}{\pgfqpoint{1.283012in}{2.011442in}}{\pgfqpoint{1.294062in}{2.011442in}}%
\pgfpathlineto{\pgfqpoint{1.294062in}{2.011442in}}%
\pgfpathclose%
\pgfusepath{stroke}%
\end{pgfscope}%
\begin{pgfscope}%
\pgfpathrectangle{\pgfqpoint{0.393053in}{0.375000in}}{\pgfqpoint{6.356833in}{5.175000in}}%
\pgfusepath{clip}%
\pgfsetbuttcap%
\pgfsetroundjoin%
\pgfsetlinewidth{1.003750pt}%
\definecolor{currentstroke}{rgb}{1.000000,0.000000,0.000000}%
\pgfsetstrokecolor{currentstroke}%
\pgfsetdash{}{0pt}%
\pgfpathmoveto{\pgfqpoint{1.085182in}{2.276360in}}%
\pgfpathcurveto{\pgfqpoint{1.096232in}{2.276360in}}{\pgfqpoint{1.106831in}{2.280751in}}{\pgfqpoint{1.114645in}{2.288564in}}%
\pgfpathcurveto{\pgfqpoint{1.122458in}{2.296378in}}{\pgfqpoint{1.126849in}{2.306977in}}{\pgfqpoint{1.126849in}{2.318027in}}%
\pgfpathcurveto{\pgfqpoint{1.126849in}{2.329077in}}{\pgfqpoint{1.122458in}{2.339676in}}{\pgfqpoint{1.114645in}{2.347490in}}%
\pgfpathcurveto{\pgfqpoint{1.106831in}{2.355303in}}{\pgfqpoint{1.096232in}{2.359694in}}{\pgfqpoint{1.085182in}{2.359694in}}%
\pgfpathcurveto{\pgfqpoint{1.074132in}{2.359694in}}{\pgfqpoint{1.063533in}{2.355303in}}{\pgfqpoint{1.055719in}{2.347490in}}%
\pgfpathcurveto{\pgfqpoint{1.047905in}{2.339676in}}{\pgfqpoint{1.043515in}{2.329077in}}{\pgfqpoint{1.043515in}{2.318027in}}%
\pgfpathcurveto{\pgfqpoint{1.043515in}{2.306977in}}{\pgfqpoint{1.047905in}{2.296378in}}{\pgfqpoint{1.055719in}{2.288564in}}%
\pgfpathcurveto{\pgfqpoint{1.063533in}{2.280751in}}{\pgfqpoint{1.074132in}{2.276360in}}{\pgfqpoint{1.085182in}{2.276360in}}%
\pgfpathlineto{\pgfqpoint{1.085182in}{2.276360in}}%
\pgfpathclose%
\pgfusepath{stroke}%
\end{pgfscope}%
\begin{pgfscope}%
\pgfpathrectangle{\pgfqpoint{0.393053in}{0.375000in}}{\pgfqpoint{6.356833in}{5.175000in}}%
\pgfusepath{clip}%
\pgfsetbuttcap%
\pgfsetroundjoin%
\pgfsetlinewidth{1.003750pt}%
\definecolor{currentstroke}{rgb}{1.000000,0.000000,0.000000}%
\pgfsetstrokecolor{currentstroke}%
\pgfsetdash{}{0pt}%
\pgfpathmoveto{\pgfqpoint{1.242191in}{2.067236in}}%
\pgfpathcurveto{\pgfqpoint{1.253241in}{2.067236in}}{\pgfqpoint{1.263840in}{2.071627in}}{\pgfqpoint{1.271654in}{2.079440in}}%
\pgfpathcurveto{\pgfqpoint{1.279467in}{2.087254in}}{\pgfqpoint{1.283858in}{2.097853in}}{\pgfqpoint{1.283858in}{2.108903in}}%
\pgfpathcurveto{\pgfqpoint{1.283858in}{2.119953in}}{\pgfqpoint{1.279467in}{2.130552in}}{\pgfqpoint{1.271654in}{2.138366in}}%
\pgfpathcurveto{\pgfqpoint{1.263840in}{2.146180in}}{\pgfqpoint{1.253241in}{2.150570in}}{\pgfqpoint{1.242191in}{2.150570in}}%
\pgfpathcurveto{\pgfqpoint{1.231141in}{2.150570in}}{\pgfqpoint{1.220542in}{2.146180in}}{\pgfqpoint{1.212728in}{2.138366in}}%
\pgfpathcurveto{\pgfqpoint{1.204915in}{2.130552in}}{\pgfqpoint{1.200524in}{2.119953in}}{\pgfqpoint{1.200524in}{2.108903in}}%
\pgfpathcurveto{\pgfqpoint{1.200524in}{2.097853in}}{\pgfqpoint{1.204915in}{2.087254in}}{\pgfqpoint{1.212728in}{2.079440in}}%
\pgfpathcurveto{\pgfqpoint{1.220542in}{2.071627in}}{\pgfqpoint{1.231141in}{2.067236in}}{\pgfqpoint{1.242191in}{2.067236in}}%
\pgfpathlineto{\pgfqpoint{1.242191in}{2.067236in}}%
\pgfpathclose%
\pgfusepath{stroke}%
\end{pgfscope}%
\begin{pgfscope}%
\pgfpathrectangle{\pgfqpoint{0.393053in}{0.375000in}}{\pgfqpoint{6.356833in}{5.175000in}}%
\pgfusepath{clip}%
\pgfsetbuttcap%
\pgfsetroundjoin%
\pgfsetlinewidth{1.003750pt}%
\definecolor{currentstroke}{rgb}{1.000000,0.000000,0.000000}%
\pgfsetstrokecolor{currentstroke}%
\pgfsetdash{}{0pt}%
\pgfpathmoveto{\pgfqpoint{1.784024in}{1.543318in}}%
\pgfpathcurveto{\pgfqpoint{1.795074in}{1.543318in}}{\pgfqpoint{1.805673in}{1.547708in}}{\pgfqpoint{1.813487in}{1.555522in}}%
\pgfpathcurveto{\pgfqpoint{1.821300in}{1.563335in}}{\pgfqpoint{1.825691in}{1.573934in}}{\pgfqpoint{1.825691in}{1.584984in}}%
\pgfpathcurveto{\pgfqpoint{1.825691in}{1.596035in}}{\pgfqpoint{1.821300in}{1.606634in}}{\pgfqpoint{1.813487in}{1.614447in}}%
\pgfpathcurveto{\pgfqpoint{1.805673in}{1.622261in}}{\pgfqpoint{1.795074in}{1.626651in}}{\pgfqpoint{1.784024in}{1.626651in}}%
\pgfpathcurveto{\pgfqpoint{1.772974in}{1.626651in}}{\pgfqpoint{1.762375in}{1.622261in}}{\pgfqpoint{1.754561in}{1.614447in}}%
\pgfpathcurveto{\pgfqpoint{1.746748in}{1.606634in}}{\pgfqpoint{1.742357in}{1.596035in}}{\pgfqpoint{1.742357in}{1.584984in}}%
\pgfpathcurveto{\pgfqpoint{1.742357in}{1.573934in}}{\pgfqpoint{1.746748in}{1.563335in}}{\pgfqpoint{1.754561in}{1.555522in}}%
\pgfpathcurveto{\pgfqpoint{1.762375in}{1.547708in}}{\pgfqpoint{1.772974in}{1.543318in}}{\pgfqpoint{1.784024in}{1.543318in}}%
\pgfpathlineto{\pgfqpoint{1.784024in}{1.543318in}}%
\pgfpathclose%
\pgfusepath{stroke}%
\end{pgfscope}%
\begin{pgfscope}%
\pgfpathrectangle{\pgfqpoint{0.393053in}{0.375000in}}{\pgfqpoint{6.356833in}{5.175000in}}%
\pgfusepath{clip}%
\pgfsetbuttcap%
\pgfsetroundjoin%
\pgfsetlinewidth{1.003750pt}%
\definecolor{currentstroke}{rgb}{1.000000,0.000000,0.000000}%
\pgfsetstrokecolor{currentstroke}%
\pgfsetdash{}{0pt}%
\pgfpathmoveto{\pgfqpoint{0.590694in}{3.252942in}}%
\pgfpathcurveto{\pgfqpoint{0.601744in}{3.252942in}}{\pgfqpoint{0.612343in}{3.257332in}}{\pgfqpoint{0.620156in}{3.265146in}}%
\pgfpathcurveto{\pgfqpoint{0.627970in}{3.272959in}}{\pgfqpoint{0.632360in}{3.283558in}}{\pgfqpoint{0.632360in}{3.294608in}}%
\pgfpathcurveto{\pgfqpoint{0.632360in}{3.305658in}}{\pgfqpoint{0.627970in}{3.316257in}}{\pgfqpoint{0.620156in}{3.324071in}}%
\pgfpathcurveto{\pgfqpoint{0.612343in}{3.331885in}}{\pgfqpoint{0.601744in}{3.336275in}}{\pgfqpoint{0.590694in}{3.336275in}}%
\pgfpathcurveto{\pgfqpoint{0.579643in}{3.336275in}}{\pgfqpoint{0.569044in}{3.331885in}}{\pgfqpoint{0.561231in}{3.324071in}}%
\pgfpathcurveto{\pgfqpoint{0.553417in}{3.316257in}}{\pgfqpoint{0.549027in}{3.305658in}}{\pgfqpoint{0.549027in}{3.294608in}}%
\pgfpathcurveto{\pgfqpoint{0.549027in}{3.283558in}}{\pgfqpoint{0.553417in}{3.272959in}}{\pgfqpoint{0.561231in}{3.265146in}}%
\pgfpathcurveto{\pgfqpoint{0.569044in}{3.257332in}}{\pgfqpoint{0.579643in}{3.252942in}}{\pgfqpoint{0.590694in}{3.252942in}}%
\pgfpathlineto{\pgfqpoint{0.590694in}{3.252942in}}%
\pgfpathclose%
\pgfusepath{stroke}%
\end{pgfscope}%
\begin{pgfscope}%
\pgfpathrectangle{\pgfqpoint{0.393053in}{0.375000in}}{\pgfqpoint{6.356833in}{5.175000in}}%
\pgfusepath{clip}%
\pgfsetbuttcap%
\pgfsetroundjoin%
\pgfsetlinewidth{1.003750pt}%
\definecolor{currentstroke}{rgb}{1.000000,0.000000,0.000000}%
\pgfsetstrokecolor{currentstroke}%
\pgfsetdash{}{0pt}%
\pgfpathmoveto{\pgfqpoint{0.448386in}{3.849208in}}%
\pgfpathcurveto{\pgfqpoint{0.459437in}{3.849208in}}{\pgfqpoint{0.470036in}{3.853598in}}{\pgfqpoint{0.477849in}{3.861412in}}%
\pgfpathcurveto{\pgfqpoint{0.485663in}{3.869225in}}{\pgfqpoint{0.490053in}{3.879824in}}{\pgfqpoint{0.490053in}{3.890875in}}%
\pgfpathcurveto{\pgfqpoint{0.490053in}{3.901925in}}{\pgfqpoint{0.485663in}{3.912524in}}{\pgfqpoint{0.477849in}{3.920337in}}%
\pgfpathcurveto{\pgfqpoint{0.470036in}{3.928151in}}{\pgfqpoint{0.459437in}{3.932541in}}{\pgfqpoint{0.448386in}{3.932541in}}%
\pgfpathcurveto{\pgfqpoint{0.437336in}{3.932541in}}{\pgfqpoint{0.426737in}{3.928151in}}{\pgfqpoint{0.418924in}{3.920337in}}%
\pgfpathcurveto{\pgfqpoint{0.411110in}{3.912524in}}{\pgfqpoint{0.406720in}{3.901925in}}{\pgfqpoint{0.406720in}{3.890875in}}%
\pgfpathcurveto{\pgfqpoint{0.406720in}{3.879824in}}{\pgfqpoint{0.411110in}{3.869225in}}{\pgfqpoint{0.418924in}{3.861412in}}%
\pgfpathcurveto{\pgfqpoint{0.426737in}{3.853598in}}{\pgfqpoint{0.437336in}{3.849208in}}{\pgfqpoint{0.448386in}{3.849208in}}%
\pgfpathlineto{\pgfqpoint{0.448386in}{3.849208in}}%
\pgfpathclose%
\pgfusepath{stroke}%
\end{pgfscope}%
\begin{pgfscope}%
\pgfpathrectangle{\pgfqpoint{0.393053in}{0.375000in}}{\pgfqpoint{6.356833in}{5.175000in}}%
\pgfusepath{clip}%
\pgfsetbuttcap%
\pgfsetroundjoin%
\pgfsetlinewidth{1.003750pt}%
\definecolor{currentstroke}{rgb}{1.000000,0.000000,0.000000}%
\pgfsetstrokecolor{currentstroke}%
\pgfsetdash{}{0pt}%
\pgfpathmoveto{\pgfqpoint{1.686434in}{1.622675in}}%
\pgfpathcurveto{\pgfqpoint{1.697484in}{1.622675in}}{\pgfqpoint{1.708083in}{1.627065in}}{\pgfqpoint{1.715897in}{1.634879in}}%
\pgfpathcurveto{\pgfqpoint{1.723710in}{1.642693in}}{\pgfqpoint{1.728100in}{1.653292in}}{\pgfqpoint{1.728100in}{1.664342in}}%
\pgfpathcurveto{\pgfqpoint{1.728100in}{1.675392in}}{\pgfqpoint{1.723710in}{1.685991in}}{\pgfqpoint{1.715897in}{1.693805in}}%
\pgfpathcurveto{\pgfqpoint{1.708083in}{1.701618in}}{\pgfqpoint{1.697484in}{1.706008in}}{\pgfqpoint{1.686434in}{1.706008in}}%
\pgfpathcurveto{\pgfqpoint{1.675384in}{1.706008in}}{\pgfqpoint{1.664785in}{1.701618in}}{\pgfqpoint{1.656971in}{1.693805in}}%
\pgfpathcurveto{\pgfqpoint{1.649157in}{1.685991in}}{\pgfqpoint{1.644767in}{1.675392in}}{\pgfqpoint{1.644767in}{1.664342in}}%
\pgfpathcurveto{\pgfqpoint{1.644767in}{1.653292in}}{\pgfqpoint{1.649157in}{1.642693in}}{\pgfqpoint{1.656971in}{1.634879in}}%
\pgfpathcurveto{\pgfqpoint{1.664785in}{1.627065in}}{\pgfqpoint{1.675384in}{1.622675in}}{\pgfqpoint{1.686434in}{1.622675in}}%
\pgfpathlineto{\pgfqpoint{1.686434in}{1.622675in}}%
\pgfpathclose%
\pgfusepath{stroke}%
\end{pgfscope}%
\begin{pgfscope}%
\pgfpathrectangle{\pgfqpoint{0.393053in}{0.375000in}}{\pgfqpoint{6.356833in}{5.175000in}}%
\pgfusepath{clip}%
\pgfsetbuttcap%
\pgfsetroundjoin%
\pgfsetlinewidth{1.003750pt}%
\definecolor{currentstroke}{rgb}{1.000000,0.000000,0.000000}%
\pgfsetstrokecolor{currentstroke}%
\pgfsetdash{}{0pt}%
\pgfpathmoveto{\pgfqpoint{0.713866in}{2.914032in}}%
\pgfpathcurveto{\pgfqpoint{0.724916in}{2.914032in}}{\pgfqpoint{0.735515in}{2.918423in}}{\pgfqpoint{0.743328in}{2.926236in}}%
\pgfpathcurveto{\pgfqpoint{0.751142in}{2.934050in}}{\pgfqpoint{0.755532in}{2.944649in}}{\pgfqpoint{0.755532in}{2.955699in}}%
\pgfpathcurveto{\pgfqpoint{0.755532in}{2.966749in}}{\pgfqpoint{0.751142in}{2.977348in}}{\pgfqpoint{0.743328in}{2.985162in}}%
\pgfpathcurveto{\pgfqpoint{0.735515in}{2.992975in}}{\pgfqpoint{0.724916in}{2.997366in}}{\pgfqpoint{0.713866in}{2.997366in}}%
\pgfpathcurveto{\pgfqpoint{0.702816in}{2.997366in}}{\pgfqpoint{0.692216in}{2.992975in}}{\pgfqpoint{0.684403in}{2.985162in}}%
\pgfpathcurveto{\pgfqpoint{0.676589in}{2.977348in}}{\pgfqpoint{0.672199in}{2.966749in}}{\pgfqpoint{0.672199in}{2.955699in}}%
\pgfpathcurveto{\pgfqpoint{0.672199in}{2.944649in}}{\pgfqpoint{0.676589in}{2.934050in}}{\pgfqpoint{0.684403in}{2.926236in}}%
\pgfpathcurveto{\pgfqpoint{0.692216in}{2.918423in}}{\pgfqpoint{0.702816in}{2.914032in}}{\pgfqpoint{0.713866in}{2.914032in}}%
\pgfpathlineto{\pgfqpoint{0.713866in}{2.914032in}}%
\pgfpathclose%
\pgfusepath{stroke}%
\end{pgfscope}%
\begin{pgfscope}%
\pgfpathrectangle{\pgfqpoint{0.393053in}{0.375000in}}{\pgfqpoint{6.356833in}{5.175000in}}%
\pgfusepath{clip}%
\pgfsetbuttcap%
\pgfsetroundjoin%
\pgfsetlinewidth{1.003750pt}%
\definecolor{currentstroke}{rgb}{1.000000,0.000000,0.000000}%
\pgfsetstrokecolor{currentstroke}%
\pgfsetdash{}{0pt}%
\pgfpathmoveto{\pgfqpoint{0.922687in}{2.506553in}}%
\pgfpathcurveto{\pgfqpoint{0.933737in}{2.506553in}}{\pgfqpoint{0.944337in}{2.510943in}}{\pgfqpoint{0.952150in}{2.518757in}}%
\pgfpathcurveto{\pgfqpoint{0.959964in}{2.526571in}}{\pgfqpoint{0.964354in}{2.537170in}}{\pgfqpoint{0.964354in}{2.548220in}}%
\pgfpathcurveto{\pgfqpoint{0.964354in}{2.559270in}}{\pgfqpoint{0.959964in}{2.569869in}}{\pgfqpoint{0.952150in}{2.577683in}}%
\pgfpathcurveto{\pgfqpoint{0.944337in}{2.585496in}}{\pgfqpoint{0.933737in}{2.589886in}}{\pgfqpoint{0.922687in}{2.589886in}}%
\pgfpathcurveto{\pgfqpoint{0.911637in}{2.589886in}}{\pgfqpoint{0.901038in}{2.585496in}}{\pgfqpoint{0.893225in}{2.577683in}}%
\pgfpathcurveto{\pgfqpoint{0.885411in}{2.569869in}}{\pgfqpoint{0.881021in}{2.559270in}}{\pgfqpoint{0.881021in}{2.548220in}}%
\pgfpathcurveto{\pgfqpoint{0.881021in}{2.537170in}}{\pgfqpoint{0.885411in}{2.526571in}}{\pgfqpoint{0.893225in}{2.518757in}}%
\pgfpathcurveto{\pgfqpoint{0.901038in}{2.510943in}}{\pgfqpoint{0.911637in}{2.506553in}}{\pgfqpoint{0.922687in}{2.506553in}}%
\pgfpathlineto{\pgfqpoint{0.922687in}{2.506553in}}%
\pgfpathclose%
\pgfusepath{stroke}%
\end{pgfscope}%
\begin{pgfscope}%
\pgfpathrectangle{\pgfqpoint{0.393053in}{0.375000in}}{\pgfqpoint{6.356833in}{5.175000in}}%
\pgfusepath{clip}%
\pgfsetbuttcap%
\pgfsetroundjoin%
\pgfsetlinewidth{1.003750pt}%
\definecolor{currentstroke}{rgb}{1.000000,0.000000,0.000000}%
\pgfsetstrokecolor{currentstroke}%
\pgfsetdash{}{0pt}%
\pgfpathmoveto{\pgfqpoint{3.575009in}{0.604895in}}%
\pgfpathcurveto{\pgfqpoint{3.586059in}{0.604895in}}{\pgfqpoint{3.596658in}{0.609286in}}{\pgfqpoint{3.604472in}{0.617099in}}%
\pgfpathcurveto{\pgfqpoint{3.612286in}{0.624913in}}{\pgfqpoint{3.616676in}{0.635512in}}{\pgfqpoint{3.616676in}{0.646562in}}%
\pgfpathcurveto{\pgfqpoint{3.616676in}{0.657612in}}{\pgfqpoint{3.612286in}{0.668211in}}{\pgfqpoint{3.604472in}{0.676025in}}%
\pgfpathcurveto{\pgfqpoint{3.596658in}{0.683839in}}{\pgfqpoint{3.586059in}{0.688229in}}{\pgfqpoint{3.575009in}{0.688229in}}%
\pgfpathcurveto{\pgfqpoint{3.563959in}{0.688229in}}{\pgfqpoint{3.553360in}{0.683839in}}{\pgfqpoint{3.545546in}{0.676025in}}%
\pgfpathcurveto{\pgfqpoint{3.537733in}{0.668211in}}{\pgfqpoint{3.533343in}{0.657612in}}{\pgfqpoint{3.533343in}{0.646562in}}%
\pgfpathcurveto{\pgfqpoint{3.533343in}{0.635512in}}{\pgfqpoint{3.537733in}{0.624913in}}{\pgfqpoint{3.545546in}{0.617099in}}%
\pgfpathcurveto{\pgfqpoint{3.553360in}{0.609286in}}{\pgfqpoint{3.563959in}{0.604895in}}{\pgfqpoint{3.575009in}{0.604895in}}%
\pgfpathlineto{\pgfqpoint{3.575009in}{0.604895in}}%
\pgfpathclose%
\pgfusepath{stroke}%
\end{pgfscope}%
\begin{pgfscope}%
\pgfpathrectangle{\pgfqpoint{0.393053in}{0.375000in}}{\pgfqpoint{6.356833in}{5.175000in}}%
\pgfusepath{clip}%
\pgfsetbuttcap%
\pgfsetroundjoin%
\pgfsetlinewidth{1.003750pt}%
\definecolor{currentstroke}{rgb}{1.000000,0.000000,0.000000}%
\pgfsetstrokecolor{currentstroke}%
\pgfsetdash{}{0pt}%
\pgfpathmoveto{\pgfqpoint{0.394371in}{4.468971in}}%
\pgfpathcurveto{\pgfqpoint{0.405421in}{4.468971in}}{\pgfqpoint{0.416020in}{4.473361in}}{\pgfqpoint{0.423834in}{4.481175in}}%
\pgfpathcurveto{\pgfqpoint{0.431648in}{4.488989in}}{\pgfqpoint{0.436038in}{4.499588in}}{\pgfqpoint{0.436038in}{4.510638in}}%
\pgfpathcurveto{\pgfqpoint{0.436038in}{4.521688in}}{\pgfqpoint{0.431648in}{4.532287in}}{\pgfqpoint{0.423834in}{4.540101in}}%
\pgfpathcurveto{\pgfqpoint{0.416020in}{4.547914in}}{\pgfqpoint{0.405421in}{4.552305in}}{\pgfqpoint{0.394371in}{4.552305in}}%
\pgfpathcurveto{\pgfqpoint{0.383321in}{4.552305in}}{\pgfqpoint{0.372722in}{4.547914in}}{\pgfqpoint{0.364909in}{4.540101in}}%
\pgfpathcurveto{\pgfqpoint{0.357095in}{4.532287in}}{\pgfqpoint{0.352705in}{4.521688in}}{\pgfqpoint{0.352705in}{4.510638in}}%
\pgfpathcurveto{\pgfqpoint{0.352705in}{4.499588in}}{\pgfqpoint{0.357095in}{4.488989in}}{\pgfqpoint{0.364909in}{4.481175in}}%
\pgfpathcurveto{\pgfqpoint{0.372722in}{4.473361in}}{\pgfqpoint{0.383321in}{4.468971in}}{\pgfqpoint{0.394371in}{4.468971in}}%
\pgfpathlineto{\pgfqpoint{0.394371in}{4.468971in}}%
\pgfpathclose%
\pgfusepath{stroke}%
\end{pgfscope}%
\begin{pgfscope}%
\pgfpathrectangle{\pgfqpoint{0.393053in}{0.375000in}}{\pgfqpoint{6.356833in}{5.175000in}}%
\pgfusepath{clip}%
\pgfsetbuttcap%
\pgfsetroundjoin%
\pgfsetlinewidth{1.003750pt}%
\definecolor{currentstroke}{rgb}{1.000000,0.000000,0.000000}%
\pgfsetstrokecolor{currentstroke}%
\pgfsetdash{}{0pt}%
\pgfpathmoveto{\pgfqpoint{3.495589in}{0.618345in}}%
\pgfpathcurveto{\pgfqpoint{3.506639in}{0.618345in}}{\pgfqpoint{3.517238in}{0.622735in}}{\pgfqpoint{3.525052in}{0.630548in}}%
\pgfpathcurveto{\pgfqpoint{3.532866in}{0.638362in}}{\pgfqpoint{3.537256in}{0.648961in}}{\pgfqpoint{3.537256in}{0.660011in}}%
\pgfpathcurveto{\pgfqpoint{3.537256in}{0.671061in}}{\pgfqpoint{3.532866in}{0.681660in}}{\pgfqpoint{3.525052in}{0.689474in}}%
\pgfpathcurveto{\pgfqpoint{3.517238in}{0.697288in}}{\pgfqpoint{3.506639in}{0.701678in}}{\pgfqpoint{3.495589in}{0.701678in}}%
\pgfpathcurveto{\pgfqpoint{3.484539in}{0.701678in}}{\pgfqpoint{3.473940in}{0.697288in}}{\pgfqpoint{3.466126in}{0.689474in}}%
\pgfpathcurveto{\pgfqpoint{3.458313in}{0.681660in}}{\pgfqpoint{3.453923in}{0.671061in}}{\pgfqpoint{3.453923in}{0.660011in}}%
\pgfpathcurveto{\pgfqpoint{3.453923in}{0.648961in}}{\pgfqpoint{3.458313in}{0.638362in}}{\pgfqpoint{3.466126in}{0.630548in}}%
\pgfpathcurveto{\pgfqpoint{3.473940in}{0.622735in}}{\pgfqpoint{3.484539in}{0.618345in}}{\pgfqpoint{3.495589in}{0.618345in}}%
\pgfpathlineto{\pgfqpoint{3.495589in}{0.618345in}}%
\pgfpathclose%
\pgfusepath{stroke}%
\end{pgfscope}%
\begin{pgfscope}%
\pgfpathrectangle{\pgfqpoint{0.393053in}{0.375000in}}{\pgfqpoint{6.356833in}{5.175000in}}%
\pgfusepath{clip}%
\pgfsetbuttcap%
\pgfsetroundjoin%
\pgfsetlinewidth{1.003750pt}%
\definecolor{currentstroke}{rgb}{1.000000,0.000000,0.000000}%
\pgfsetstrokecolor{currentstroke}%
\pgfsetdash{}{0pt}%
\pgfpathmoveto{\pgfqpoint{0.463261in}{3.789657in}}%
\pgfpathcurveto{\pgfqpoint{0.474311in}{3.789657in}}{\pgfqpoint{0.484910in}{3.794048in}}{\pgfqpoint{0.492724in}{3.801861in}}%
\pgfpathcurveto{\pgfqpoint{0.500538in}{3.809675in}}{\pgfqpoint{0.504928in}{3.820274in}}{\pgfqpoint{0.504928in}{3.831324in}}%
\pgfpathcurveto{\pgfqpoint{0.504928in}{3.842374in}}{\pgfqpoint{0.500538in}{3.852973in}}{\pgfqpoint{0.492724in}{3.860787in}}%
\pgfpathcurveto{\pgfqpoint{0.484910in}{3.868600in}}{\pgfqpoint{0.474311in}{3.872991in}}{\pgfqpoint{0.463261in}{3.872991in}}%
\pgfpathcurveto{\pgfqpoint{0.452211in}{3.872991in}}{\pgfqpoint{0.441612in}{3.868600in}}{\pgfqpoint{0.433798in}{3.860787in}}%
\pgfpathcurveto{\pgfqpoint{0.425985in}{3.852973in}}{\pgfqpoint{0.421595in}{3.842374in}}{\pgfqpoint{0.421595in}{3.831324in}}%
\pgfpathcurveto{\pgfqpoint{0.421595in}{3.820274in}}{\pgfqpoint{0.425985in}{3.809675in}}{\pgfqpoint{0.433798in}{3.801861in}}%
\pgfpathcurveto{\pgfqpoint{0.441612in}{3.794048in}}{\pgfqpoint{0.452211in}{3.789657in}}{\pgfqpoint{0.463261in}{3.789657in}}%
\pgfpathlineto{\pgfqpoint{0.463261in}{3.789657in}}%
\pgfpathclose%
\pgfusepath{stroke}%
\end{pgfscope}%
\begin{pgfscope}%
\pgfpathrectangle{\pgfqpoint{0.393053in}{0.375000in}}{\pgfqpoint{6.356833in}{5.175000in}}%
\pgfusepath{clip}%
\pgfsetbuttcap%
\pgfsetroundjoin%
\pgfsetlinewidth{1.003750pt}%
\definecolor{currentstroke}{rgb}{1.000000,0.000000,0.000000}%
\pgfsetstrokecolor{currentstroke}%
\pgfsetdash{}{0pt}%
\pgfpathmoveto{\pgfqpoint{2.264090in}{1.188623in}}%
\pgfpathcurveto{\pgfqpoint{2.275140in}{1.188623in}}{\pgfqpoint{2.285739in}{1.193014in}}{\pgfqpoint{2.293553in}{1.200827in}}%
\pgfpathcurveto{\pgfqpoint{2.301366in}{1.208641in}}{\pgfqpoint{2.305757in}{1.219240in}}{\pgfqpoint{2.305757in}{1.230290in}}%
\pgfpathcurveto{\pgfqpoint{2.305757in}{1.241340in}}{\pgfqpoint{2.301366in}{1.251939in}}{\pgfqpoint{2.293553in}{1.259753in}}%
\pgfpathcurveto{\pgfqpoint{2.285739in}{1.267566in}}{\pgfqpoint{2.275140in}{1.271957in}}{\pgfqpoint{2.264090in}{1.271957in}}%
\pgfpathcurveto{\pgfqpoint{2.253040in}{1.271957in}}{\pgfqpoint{2.242441in}{1.267566in}}{\pgfqpoint{2.234627in}{1.259753in}}%
\pgfpathcurveto{\pgfqpoint{2.226814in}{1.251939in}}{\pgfqpoint{2.222423in}{1.241340in}}{\pgfqpoint{2.222423in}{1.230290in}}%
\pgfpathcurveto{\pgfqpoint{2.222423in}{1.219240in}}{\pgfqpoint{2.226814in}{1.208641in}}{\pgfqpoint{2.234627in}{1.200827in}}%
\pgfpathcurveto{\pgfqpoint{2.242441in}{1.193014in}}{\pgfqpoint{2.253040in}{1.188623in}}{\pgfqpoint{2.264090in}{1.188623in}}%
\pgfpathlineto{\pgfqpoint{2.264090in}{1.188623in}}%
\pgfpathclose%
\pgfusepath{stroke}%
\end{pgfscope}%
\begin{pgfscope}%
\pgfpathrectangle{\pgfqpoint{0.393053in}{0.375000in}}{\pgfqpoint{6.356833in}{5.175000in}}%
\pgfusepath{clip}%
\pgfsetbuttcap%
\pgfsetroundjoin%
\pgfsetlinewidth{1.003750pt}%
\definecolor{currentstroke}{rgb}{1.000000,0.000000,0.000000}%
\pgfsetstrokecolor{currentstroke}%
\pgfsetdash{}{0pt}%
\pgfpathmoveto{\pgfqpoint{4.330993in}{0.434433in}}%
\pgfpathcurveto{\pgfqpoint{4.342043in}{0.434433in}}{\pgfqpoint{4.352642in}{0.438824in}}{\pgfqpoint{4.360456in}{0.446637in}}%
\pgfpathcurveto{\pgfqpoint{4.368269in}{0.454451in}}{\pgfqpoint{4.372659in}{0.465050in}}{\pgfqpoint{4.372659in}{0.476100in}}%
\pgfpathcurveto{\pgfqpoint{4.372659in}{0.487150in}}{\pgfqpoint{4.368269in}{0.497749in}}{\pgfqpoint{4.360456in}{0.505563in}}%
\pgfpathcurveto{\pgfqpoint{4.352642in}{0.513377in}}{\pgfqpoint{4.342043in}{0.517767in}}{\pgfqpoint{4.330993in}{0.517767in}}%
\pgfpathcurveto{\pgfqpoint{4.319943in}{0.517767in}}{\pgfqpoint{4.309344in}{0.513377in}}{\pgfqpoint{4.301530in}{0.505563in}}%
\pgfpathcurveto{\pgfqpoint{4.293716in}{0.497749in}}{\pgfqpoint{4.289326in}{0.487150in}}{\pgfqpoint{4.289326in}{0.476100in}}%
\pgfpathcurveto{\pgfqpoint{4.289326in}{0.465050in}}{\pgfqpoint{4.293716in}{0.454451in}}{\pgfqpoint{4.301530in}{0.446637in}}%
\pgfpathcurveto{\pgfqpoint{4.309344in}{0.438824in}}{\pgfqpoint{4.319943in}{0.434433in}}{\pgfqpoint{4.330993in}{0.434433in}}%
\pgfpathlineto{\pgfqpoint{4.330993in}{0.434433in}}%
\pgfpathclose%
\pgfusepath{stroke}%
\end{pgfscope}%
\begin{pgfscope}%
\pgfpathrectangle{\pgfqpoint{0.393053in}{0.375000in}}{\pgfqpoint{6.356833in}{5.175000in}}%
\pgfusepath{clip}%
\pgfsetbuttcap%
\pgfsetroundjoin%
\pgfsetlinewidth{1.003750pt}%
\definecolor{currentstroke}{rgb}{1.000000,0.000000,0.000000}%
\pgfsetstrokecolor{currentstroke}%
\pgfsetdash{}{0pt}%
\pgfpathmoveto{\pgfqpoint{4.417891in}{0.424064in}}%
\pgfpathcurveto{\pgfqpoint{4.428941in}{0.424064in}}{\pgfqpoint{4.439540in}{0.428454in}}{\pgfqpoint{4.447354in}{0.436268in}}%
\pgfpathcurveto{\pgfqpoint{4.455167in}{0.444082in}}{\pgfqpoint{4.459557in}{0.454681in}}{\pgfqpoint{4.459557in}{0.465731in}}%
\pgfpathcurveto{\pgfqpoint{4.459557in}{0.476781in}}{\pgfqpoint{4.455167in}{0.487380in}}{\pgfqpoint{4.447354in}{0.495194in}}%
\pgfpathcurveto{\pgfqpoint{4.439540in}{0.503007in}}{\pgfqpoint{4.428941in}{0.507397in}}{\pgfqpoint{4.417891in}{0.507397in}}%
\pgfpathcurveto{\pgfqpoint{4.406841in}{0.507397in}}{\pgfqpoint{4.396242in}{0.503007in}}{\pgfqpoint{4.388428in}{0.495194in}}%
\pgfpathcurveto{\pgfqpoint{4.380614in}{0.487380in}}{\pgfqpoint{4.376224in}{0.476781in}}{\pgfqpoint{4.376224in}{0.465731in}}%
\pgfpathcurveto{\pgfqpoint{4.376224in}{0.454681in}}{\pgfqpoint{4.380614in}{0.444082in}}{\pgfqpoint{4.388428in}{0.436268in}}%
\pgfpathcurveto{\pgfqpoint{4.396242in}{0.428454in}}{\pgfqpoint{4.406841in}{0.424064in}}{\pgfqpoint{4.417891in}{0.424064in}}%
\pgfpathlineto{\pgfqpoint{4.417891in}{0.424064in}}%
\pgfpathclose%
\pgfusepath{stroke}%
\end{pgfscope}%
\begin{pgfscope}%
\pgfpathrectangle{\pgfqpoint{0.393053in}{0.375000in}}{\pgfqpoint{6.356833in}{5.175000in}}%
\pgfusepath{clip}%
\pgfsetbuttcap%
\pgfsetroundjoin%
\pgfsetlinewidth{1.003750pt}%
\definecolor{currentstroke}{rgb}{1.000000,0.000000,0.000000}%
\pgfsetstrokecolor{currentstroke}%
\pgfsetdash{}{0pt}%
\pgfpathmoveto{\pgfqpoint{2.893564in}{0.861159in}}%
\pgfpathcurveto{\pgfqpoint{2.904614in}{0.861159in}}{\pgfqpoint{2.915213in}{0.865550in}}{\pgfqpoint{2.923027in}{0.873363in}}%
\pgfpathcurveto{\pgfqpoint{2.930840in}{0.881177in}}{\pgfqpoint{2.935231in}{0.891776in}}{\pgfqpoint{2.935231in}{0.902826in}}%
\pgfpathcurveto{\pgfqpoint{2.935231in}{0.913876in}}{\pgfqpoint{2.930840in}{0.924475in}}{\pgfqpoint{2.923027in}{0.932289in}}%
\pgfpathcurveto{\pgfqpoint{2.915213in}{0.940103in}}{\pgfqpoint{2.904614in}{0.944493in}}{\pgfqpoint{2.893564in}{0.944493in}}%
\pgfpathcurveto{\pgfqpoint{2.882514in}{0.944493in}}{\pgfqpoint{2.871915in}{0.940103in}}{\pgfqpoint{2.864101in}{0.932289in}}%
\pgfpathcurveto{\pgfqpoint{2.856287in}{0.924475in}}{\pgfqpoint{2.851897in}{0.913876in}}{\pgfqpoint{2.851897in}{0.902826in}}%
\pgfpathcurveto{\pgfqpoint{2.851897in}{0.891776in}}{\pgfqpoint{2.856287in}{0.881177in}}{\pgfqpoint{2.864101in}{0.873363in}}%
\pgfpathcurveto{\pgfqpoint{2.871915in}{0.865550in}}{\pgfqpoint{2.882514in}{0.861159in}}{\pgfqpoint{2.893564in}{0.861159in}}%
\pgfpathlineto{\pgfqpoint{2.893564in}{0.861159in}}%
\pgfpathclose%
\pgfusepath{stroke}%
\end{pgfscope}%
\begin{pgfscope}%
\pgfpathrectangle{\pgfqpoint{0.393053in}{0.375000in}}{\pgfqpoint{6.356833in}{5.175000in}}%
\pgfusepath{clip}%
\pgfsetbuttcap%
\pgfsetroundjoin%
\pgfsetlinewidth{1.003750pt}%
\definecolor{currentstroke}{rgb}{1.000000,0.000000,0.000000}%
\pgfsetstrokecolor{currentstroke}%
\pgfsetdash{}{0pt}%
\pgfpathmoveto{\pgfqpoint{0.496834in}{3.594450in}}%
\pgfpathcurveto{\pgfqpoint{0.507884in}{3.594450in}}{\pgfqpoint{0.518483in}{3.598840in}}{\pgfqpoint{0.526297in}{3.606654in}}%
\pgfpathcurveto{\pgfqpoint{0.534110in}{3.614468in}}{\pgfqpoint{0.538500in}{3.625067in}}{\pgfqpoint{0.538500in}{3.636117in}}%
\pgfpathcurveto{\pgfqpoint{0.538500in}{3.647167in}}{\pgfqpoint{0.534110in}{3.657766in}}{\pgfqpoint{0.526297in}{3.665580in}}%
\pgfpathcurveto{\pgfqpoint{0.518483in}{3.673393in}}{\pgfqpoint{0.507884in}{3.677783in}}{\pgfqpoint{0.496834in}{3.677783in}}%
\pgfpathcurveto{\pgfqpoint{0.485784in}{3.677783in}}{\pgfqpoint{0.475185in}{3.673393in}}{\pgfqpoint{0.467371in}{3.665580in}}%
\pgfpathcurveto{\pgfqpoint{0.459557in}{3.657766in}}{\pgfqpoint{0.455167in}{3.647167in}}{\pgfqpoint{0.455167in}{3.636117in}}%
\pgfpathcurveto{\pgfqpoint{0.455167in}{3.625067in}}{\pgfqpoint{0.459557in}{3.614468in}}{\pgfqpoint{0.467371in}{3.606654in}}%
\pgfpathcurveto{\pgfqpoint{0.475185in}{3.598840in}}{\pgfqpoint{0.485784in}{3.594450in}}{\pgfqpoint{0.496834in}{3.594450in}}%
\pgfpathlineto{\pgfqpoint{0.496834in}{3.594450in}}%
\pgfpathclose%
\pgfusepath{stroke}%
\end{pgfscope}%
\begin{pgfscope}%
\pgfpathrectangle{\pgfqpoint{0.393053in}{0.375000in}}{\pgfqpoint{6.356833in}{5.175000in}}%
\pgfusepath{clip}%
\pgfsetbuttcap%
\pgfsetroundjoin%
\pgfsetlinewidth{1.003750pt}%
\definecolor{currentstroke}{rgb}{1.000000,0.000000,0.000000}%
\pgfsetstrokecolor{currentstroke}%
\pgfsetdash{}{0pt}%
\pgfpathmoveto{\pgfqpoint{0.811365in}{2.755186in}}%
\pgfpathcurveto{\pgfqpoint{0.822415in}{2.755186in}}{\pgfqpoint{0.833014in}{2.759576in}}{\pgfqpoint{0.840828in}{2.767390in}}%
\pgfpathcurveto{\pgfqpoint{0.848641in}{2.775204in}}{\pgfqpoint{0.853031in}{2.785803in}}{\pgfqpoint{0.853031in}{2.796853in}}%
\pgfpathcurveto{\pgfqpoint{0.853031in}{2.807903in}}{\pgfqpoint{0.848641in}{2.818502in}}{\pgfqpoint{0.840828in}{2.826316in}}%
\pgfpathcurveto{\pgfqpoint{0.833014in}{2.834129in}}{\pgfqpoint{0.822415in}{2.838519in}}{\pgfqpoint{0.811365in}{2.838519in}}%
\pgfpathcurveto{\pgfqpoint{0.800315in}{2.838519in}}{\pgfqpoint{0.789716in}{2.834129in}}{\pgfqpoint{0.781902in}{2.826316in}}%
\pgfpathcurveto{\pgfqpoint{0.774088in}{2.818502in}}{\pgfqpoint{0.769698in}{2.807903in}}{\pgfqpoint{0.769698in}{2.796853in}}%
\pgfpathcurveto{\pgfqpoint{0.769698in}{2.785803in}}{\pgfqpoint{0.774088in}{2.775204in}}{\pgfqpoint{0.781902in}{2.767390in}}%
\pgfpathcurveto{\pgfqpoint{0.789716in}{2.759576in}}{\pgfqpoint{0.800315in}{2.755186in}}{\pgfqpoint{0.811365in}{2.755186in}}%
\pgfpathlineto{\pgfqpoint{0.811365in}{2.755186in}}%
\pgfpathclose%
\pgfusepath{stroke}%
\end{pgfscope}%
\begin{pgfscope}%
\pgfpathrectangle{\pgfqpoint{0.393053in}{0.375000in}}{\pgfqpoint{6.356833in}{5.175000in}}%
\pgfusepath{clip}%
\pgfsetbuttcap%
\pgfsetroundjoin%
\pgfsetlinewidth{1.003750pt}%
\definecolor{currentstroke}{rgb}{1.000000,0.000000,0.000000}%
\pgfsetstrokecolor{currentstroke}%
\pgfsetdash{}{0pt}%
\pgfpathmoveto{\pgfqpoint{2.448263in}{1.079636in}}%
\pgfpathcurveto{\pgfqpoint{2.459314in}{1.079636in}}{\pgfqpoint{2.469913in}{1.084026in}}{\pgfqpoint{2.477726in}{1.091840in}}%
\pgfpathcurveto{\pgfqpoint{2.485540in}{1.099654in}}{\pgfqpoint{2.489930in}{1.110253in}}{\pgfqpoint{2.489930in}{1.121303in}}%
\pgfpathcurveto{\pgfqpoint{2.489930in}{1.132353in}}{\pgfqpoint{2.485540in}{1.142952in}}{\pgfqpoint{2.477726in}{1.150766in}}%
\pgfpathcurveto{\pgfqpoint{2.469913in}{1.158579in}}{\pgfqpoint{2.459314in}{1.162969in}}{\pgfqpoint{2.448263in}{1.162969in}}%
\pgfpathcurveto{\pgfqpoint{2.437213in}{1.162969in}}{\pgfqpoint{2.426614in}{1.158579in}}{\pgfqpoint{2.418801in}{1.150766in}}%
\pgfpathcurveto{\pgfqpoint{2.410987in}{1.142952in}}{\pgfqpoint{2.406597in}{1.132353in}}{\pgfqpoint{2.406597in}{1.121303in}}%
\pgfpathcurveto{\pgfqpoint{2.406597in}{1.110253in}}{\pgfqpoint{2.410987in}{1.099654in}}{\pgfqpoint{2.418801in}{1.091840in}}%
\pgfpathcurveto{\pgfqpoint{2.426614in}{1.084026in}}{\pgfqpoint{2.437213in}{1.079636in}}{\pgfqpoint{2.448263in}{1.079636in}}%
\pgfpathlineto{\pgfqpoint{2.448263in}{1.079636in}}%
\pgfpathclose%
\pgfusepath{stroke}%
\end{pgfscope}%
\begin{pgfscope}%
\pgfpathrectangle{\pgfqpoint{0.393053in}{0.375000in}}{\pgfqpoint{6.356833in}{5.175000in}}%
\pgfusepath{clip}%
\pgfsetbuttcap%
\pgfsetroundjoin%
\pgfsetlinewidth{1.003750pt}%
\definecolor{currentstroke}{rgb}{1.000000,0.000000,0.000000}%
\pgfsetstrokecolor{currentstroke}%
\pgfsetdash{}{0pt}%
\pgfpathmoveto{\pgfqpoint{0.394714in}{4.452967in}}%
\pgfpathcurveto{\pgfqpoint{0.405764in}{4.452967in}}{\pgfqpoint{0.416363in}{4.457357in}}{\pgfqpoint{0.424176in}{4.465170in}}%
\pgfpathcurveto{\pgfqpoint{0.431990in}{4.472984in}}{\pgfqpoint{0.436380in}{4.483583in}}{\pgfqpoint{0.436380in}{4.494633in}}%
\pgfpathcurveto{\pgfqpoint{0.436380in}{4.505683in}}{\pgfqpoint{0.431990in}{4.516282in}}{\pgfqpoint{0.424176in}{4.524096in}}%
\pgfpathcurveto{\pgfqpoint{0.416363in}{4.531910in}}{\pgfqpoint{0.405764in}{4.536300in}}{\pgfqpoint{0.394714in}{4.536300in}}%
\pgfpathcurveto{\pgfqpoint{0.383663in}{4.536300in}}{\pgfqpoint{0.373064in}{4.531910in}}{\pgfqpoint{0.365251in}{4.524096in}}%
\pgfpathcurveto{\pgfqpoint{0.357437in}{4.516282in}}{\pgfqpoint{0.353047in}{4.505683in}}{\pgfqpoint{0.353047in}{4.494633in}}%
\pgfpathcurveto{\pgfqpoint{0.353047in}{4.483583in}}{\pgfqpoint{0.357437in}{4.472984in}}{\pgfqpoint{0.365251in}{4.465170in}}%
\pgfpathcurveto{\pgfqpoint{0.373064in}{4.457357in}}{\pgfqpoint{0.383663in}{4.452967in}}{\pgfqpoint{0.394714in}{4.452967in}}%
\pgfpathlineto{\pgfqpoint{0.394714in}{4.452967in}}%
\pgfpathclose%
\pgfusepath{stroke}%
\end{pgfscope}%
\begin{pgfscope}%
\pgfpathrectangle{\pgfqpoint{0.393053in}{0.375000in}}{\pgfqpoint{6.356833in}{5.175000in}}%
\pgfusepath{clip}%
\pgfsetbuttcap%
\pgfsetroundjoin%
\pgfsetlinewidth{1.003750pt}%
\definecolor{currentstroke}{rgb}{1.000000,0.000000,0.000000}%
\pgfsetstrokecolor{currentstroke}%
\pgfsetdash{}{0pt}%
\pgfpathmoveto{\pgfqpoint{1.455457in}{1.828177in}}%
\pgfpathcurveto{\pgfqpoint{1.466507in}{1.828177in}}{\pgfqpoint{1.477106in}{1.832567in}}{\pgfqpoint{1.484920in}{1.840381in}}%
\pgfpathcurveto{\pgfqpoint{1.492733in}{1.848194in}}{\pgfqpoint{1.497123in}{1.858793in}}{\pgfqpoint{1.497123in}{1.869844in}}%
\pgfpathcurveto{\pgfqpoint{1.497123in}{1.880894in}}{\pgfqpoint{1.492733in}{1.891493in}}{\pgfqpoint{1.484920in}{1.899306in}}%
\pgfpathcurveto{\pgfqpoint{1.477106in}{1.907120in}}{\pgfqpoint{1.466507in}{1.911510in}}{\pgfqpoint{1.455457in}{1.911510in}}%
\pgfpathcurveto{\pgfqpoint{1.444407in}{1.911510in}}{\pgfqpoint{1.433808in}{1.907120in}}{\pgfqpoint{1.425994in}{1.899306in}}%
\pgfpathcurveto{\pgfqpoint{1.418180in}{1.891493in}}{\pgfqpoint{1.413790in}{1.880894in}}{\pgfqpoint{1.413790in}{1.869844in}}%
\pgfpathcurveto{\pgfqpoint{1.413790in}{1.858793in}}{\pgfqpoint{1.418180in}{1.848194in}}{\pgfqpoint{1.425994in}{1.840381in}}%
\pgfpathcurveto{\pgfqpoint{1.433808in}{1.832567in}}{\pgfqpoint{1.444407in}{1.828177in}}{\pgfqpoint{1.455457in}{1.828177in}}%
\pgfpathlineto{\pgfqpoint{1.455457in}{1.828177in}}%
\pgfpathclose%
\pgfusepath{stroke}%
\end{pgfscope}%
\begin{pgfscope}%
\pgfpathrectangle{\pgfqpoint{0.393053in}{0.375000in}}{\pgfqpoint{6.356833in}{5.175000in}}%
\pgfusepath{clip}%
\pgfsetbuttcap%
\pgfsetroundjoin%
\pgfsetlinewidth{1.003750pt}%
\definecolor{currentstroke}{rgb}{1.000000,0.000000,0.000000}%
\pgfsetstrokecolor{currentstroke}%
\pgfsetdash{}{0pt}%
\pgfpathmoveto{\pgfqpoint{0.400473in}{4.344693in}}%
\pgfpathcurveto{\pgfqpoint{0.411523in}{4.344693in}}{\pgfqpoint{0.422122in}{4.349083in}}{\pgfqpoint{0.429935in}{4.356897in}}%
\pgfpathcurveto{\pgfqpoint{0.437749in}{4.364711in}}{\pgfqpoint{0.442139in}{4.375310in}}{\pgfqpoint{0.442139in}{4.386360in}}%
\pgfpathcurveto{\pgfqpoint{0.442139in}{4.397410in}}{\pgfqpoint{0.437749in}{4.408009in}}{\pgfqpoint{0.429935in}{4.415823in}}%
\pgfpathcurveto{\pgfqpoint{0.422122in}{4.423636in}}{\pgfqpoint{0.411523in}{4.428027in}}{\pgfqpoint{0.400473in}{4.428027in}}%
\pgfpathcurveto{\pgfqpoint{0.389422in}{4.428027in}}{\pgfqpoint{0.378823in}{4.423636in}}{\pgfqpoint{0.371010in}{4.415823in}}%
\pgfpathcurveto{\pgfqpoint{0.363196in}{4.408009in}}{\pgfqpoint{0.358806in}{4.397410in}}{\pgfqpoint{0.358806in}{4.386360in}}%
\pgfpathcurveto{\pgfqpoint{0.358806in}{4.375310in}}{\pgfqpoint{0.363196in}{4.364711in}}{\pgfqpoint{0.371010in}{4.356897in}}%
\pgfpathcurveto{\pgfqpoint{0.378823in}{4.349083in}}{\pgfqpoint{0.389422in}{4.344693in}}{\pgfqpoint{0.400473in}{4.344693in}}%
\pgfpathlineto{\pgfqpoint{0.400473in}{4.344693in}}%
\pgfpathclose%
\pgfusepath{stroke}%
\end{pgfscope}%
\begin{pgfscope}%
\pgfpathrectangle{\pgfqpoint{0.393053in}{0.375000in}}{\pgfqpoint{6.356833in}{5.175000in}}%
\pgfusepath{clip}%
\pgfsetbuttcap%
\pgfsetroundjoin%
\pgfsetlinewidth{1.003750pt}%
\definecolor{currentstroke}{rgb}{1.000000,0.000000,0.000000}%
\pgfsetstrokecolor{currentstroke}%
\pgfsetdash{}{0pt}%
\pgfpathmoveto{\pgfqpoint{2.128275in}{1.300730in}}%
\pgfpathcurveto{\pgfqpoint{2.139325in}{1.300730in}}{\pgfqpoint{2.149924in}{1.305120in}}{\pgfqpoint{2.157738in}{1.312934in}}%
\pgfpathcurveto{\pgfqpoint{2.165551in}{1.320747in}}{\pgfqpoint{2.169942in}{1.331346in}}{\pgfqpoint{2.169942in}{1.342396in}}%
\pgfpathcurveto{\pgfqpoint{2.169942in}{1.353446in}}{\pgfqpoint{2.165551in}{1.364045in}}{\pgfqpoint{2.157738in}{1.371859in}}%
\pgfpathcurveto{\pgfqpoint{2.149924in}{1.379673in}}{\pgfqpoint{2.139325in}{1.384063in}}{\pgfqpoint{2.128275in}{1.384063in}}%
\pgfpathcurveto{\pgfqpoint{2.117225in}{1.384063in}}{\pgfqpoint{2.106626in}{1.379673in}}{\pgfqpoint{2.098812in}{1.371859in}}%
\pgfpathcurveto{\pgfqpoint{2.090998in}{1.364045in}}{\pgfqpoint{2.086608in}{1.353446in}}{\pgfqpoint{2.086608in}{1.342396in}}%
\pgfpathcurveto{\pgfqpoint{2.086608in}{1.331346in}}{\pgfqpoint{2.090998in}{1.320747in}}{\pgfqpoint{2.098812in}{1.312934in}}%
\pgfpathcurveto{\pgfqpoint{2.106626in}{1.305120in}}{\pgfqpoint{2.117225in}{1.300730in}}{\pgfqpoint{2.128275in}{1.300730in}}%
\pgfpathlineto{\pgfqpoint{2.128275in}{1.300730in}}%
\pgfpathclose%
\pgfusepath{stroke}%
\end{pgfscope}%
\begin{pgfscope}%
\pgfpathrectangle{\pgfqpoint{0.393053in}{0.375000in}}{\pgfqpoint{6.356833in}{5.175000in}}%
\pgfusepath{clip}%
\pgfsetbuttcap%
\pgfsetroundjoin%
\pgfsetlinewidth{1.003750pt}%
\definecolor{currentstroke}{rgb}{1.000000,0.000000,0.000000}%
\pgfsetstrokecolor{currentstroke}%
\pgfsetdash{}{0pt}%
\pgfpathmoveto{\pgfqpoint{0.781200in}{2.770047in}}%
\pgfpathcurveto{\pgfqpoint{0.792250in}{2.770047in}}{\pgfqpoint{0.802849in}{2.774437in}}{\pgfqpoint{0.810663in}{2.782251in}}%
\pgfpathcurveto{\pgfqpoint{0.818476in}{2.790065in}}{\pgfqpoint{0.822867in}{2.800664in}}{\pgfqpoint{0.822867in}{2.811714in}}%
\pgfpathcurveto{\pgfqpoint{0.822867in}{2.822764in}}{\pgfqpoint{0.818476in}{2.833363in}}{\pgfqpoint{0.810663in}{2.841177in}}%
\pgfpathcurveto{\pgfqpoint{0.802849in}{2.848990in}}{\pgfqpoint{0.792250in}{2.853381in}}{\pgfqpoint{0.781200in}{2.853381in}}%
\pgfpathcurveto{\pgfqpoint{0.770150in}{2.853381in}}{\pgfqpoint{0.759551in}{2.848990in}}{\pgfqpoint{0.751737in}{2.841177in}}%
\pgfpathcurveto{\pgfqpoint{0.743924in}{2.833363in}}{\pgfqpoint{0.739533in}{2.822764in}}{\pgfqpoint{0.739533in}{2.811714in}}%
\pgfpathcurveto{\pgfqpoint{0.739533in}{2.800664in}}{\pgfqpoint{0.743924in}{2.790065in}}{\pgfqpoint{0.751737in}{2.782251in}}%
\pgfpathcurveto{\pgfqpoint{0.759551in}{2.774437in}}{\pgfqpoint{0.770150in}{2.770047in}}{\pgfqpoint{0.781200in}{2.770047in}}%
\pgfpathlineto{\pgfqpoint{0.781200in}{2.770047in}}%
\pgfpathclose%
\pgfusepath{stroke}%
\end{pgfscope}%
\begin{pgfscope}%
\pgfpathrectangle{\pgfqpoint{0.393053in}{0.375000in}}{\pgfqpoint{6.356833in}{5.175000in}}%
\pgfusepath{clip}%
\pgfsetbuttcap%
\pgfsetroundjoin%
\pgfsetlinewidth{1.003750pt}%
\definecolor{currentstroke}{rgb}{1.000000,0.000000,0.000000}%
\pgfsetstrokecolor{currentstroke}%
\pgfsetdash{}{0pt}%
\pgfpathmoveto{\pgfqpoint{3.938392in}{0.502273in}}%
\pgfpathcurveto{\pgfqpoint{3.949443in}{0.502273in}}{\pgfqpoint{3.960042in}{0.506663in}}{\pgfqpoint{3.967855in}{0.514477in}}%
\pgfpathcurveto{\pgfqpoint{3.975669in}{0.522290in}}{\pgfqpoint{3.980059in}{0.532889in}}{\pgfqpoint{3.980059in}{0.543939in}}%
\pgfpathcurveto{\pgfqpoint{3.980059in}{0.554990in}}{\pgfqpoint{3.975669in}{0.565589in}}{\pgfqpoint{3.967855in}{0.573402in}}%
\pgfpathcurveto{\pgfqpoint{3.960042in}{0.581216in}}{\pgfqpoint{3.949443in}{0.585606in}}{\pgfqpoint{3.938392in}{0.585606in}}%
\pgfpathcurveto{\pgfqpoint{3.927342in}{0.585606in}}{\pgfqpoint{3.916743in}{0.581216in}}{\pgfqpoint{3.908930in}{0.573402in}}%
\pgfpathcurveto{\pgfqpoint{3.901116in}{0.565589in}}{\pgfqpoint{3.896726in}{0.554990in}}{\pgfqpoint{3.896726in}{0.543939in}}%
\pgfpathcurveto{\pgfqpoint{3.896726in}{0.532889in}}{\pgfqpoint{3.901116in}{0.522290in}}{\pgfqpoint{3.908930in}{0.514477in}}%
\pgfpathcurveto{\pgfqpoint{3.916743in}{0.506663in}}{\pgfqpoint{3.927342in}{0.502273in}}{\pgfqpoint{3.938392in}{0.502273in}}%
\pgfpathlineto{\pgfqpoint{3.938392in}{0.502273in}}%
\pgfpathclose%
\pgfusepath{stroke}%
\end{pgfscope}%
\begin{pgfscope}%
\pgfpathrectangle{\pgfqpoint{0.393053in}{0.375000in}}{\pgfqpoint{6.356833in}{5.175000in}}%
\pgfusepath{clip}%
\pgfsetbuttcap%
\pgfsetroundjoin%
\pgfsetlinewidth{1.003750pt}%
\definecolor{currentstroke}{rgb}{1.000000,0.000000,0.000000}%
\pgfsetstrokecolor{currentstroke}%
\pgfsetdash{}{0pt}%
\pgfpathmoveto{\pgfqpoint{1.594887in}{1.701429in}}%
\pgfpathcurveto{\pgfqpoint{1.605937in}{1.701429in}}{\pgfqpoint{1.616536in}{1.705819in}}{\pgfqpoint{1.624350in}{1.713633in}}%
\pgfpathcurveto{\pgfqpoint{1.632164in}{1.721446in}}{\pgfqpoint{1.636554in}{1.732045in}}{\pgfqpoint{1.636554in}{1.743095in}}%
\pgfpathcurveto{\pgfqpoint{1.636554in}{1.754145in}}{\pgfqpoint{1.632164in}{1.764744in}}{\pgfqpoint{1.624350in}{1.772558in}}%
\pgfpathcurveto{\pgfqpoint{1.616536in}{1.780372in}}{\pgfqpoint{1.605937in}{1.784762in}}{\pgfqpoint{1.594887in}{1.784762in}}%
\pgfpathcurveto{\pgfqpoint{1.583837in}{1.784762in}}{\pgfqpoint{1.573238in}{1.780372in}}{\pgfqpoint{1.565425in}{1.772558in}}%
\pgfpathcurveto{\pgfqpoint{1.557611in}{1.764744in}}{\pgfqpoint{1.553221in}{1.754145in}}{\pgfqpoint{1.553221in}{1.743095in}}%
\pgfpathcurveto{\pgfqpoint{1.553221in}{1.732045in}}{\pgfqpoint{1.557611in}{1.721446in}}{\pgfqpoint{1.565425in}{1.713633in}}%
\pgfpathcurveto{\pgfqpoint{1.573238in}{1.705819in}}{\pgfqpoint{1.583837in}{1.701429in}}{\pgfqpoint{1.594887in}{1.701429in}}%
\pgfpathlineto{\pgfqpoint{1.594887in}{1.701429in}}%
\pgfpathclose%
\pgfusepath{stroke}%
\end{pgfscope}%
\begin{pgfscope}%
\pgfpathrectangle{\pgfqpoint{0.393053in}{0.375000in}}{\pgfqpoint{6.356833in}{5.175000in}}%
\pgfusepath{clip}%
\pgfsetbuttcap%
\pgfsetroundjoin%
\pgfsetlinewidth{1.003750pt}%
\definecolor{currentstroke}{rgb}{1.000000,0.000000,0.000000}%
\pgfsetstrokecolor{currentstroke}%
\pgfsetdash{}{0pt}%
\pgfpathmoveto{\pgfqpoint{0.968245in}{2.452909in}}%
\pgfpathcurveto{\pgfqpoint{0.979296in}{2.452909in}}{\pgfqpoint{0.989895in}{2.457299in}}{\pgfqpoint{0.997708in}{2.465113in}}%
\pgfpathcurveto{\pgfqpoint{1.005522in}{2.472926in}}{\pgfqpoint{1.009912in}{2.483525in}}{\pgfqpoint{1.009912in}{2.494576in}}%
\pgfpathcurveto{\pgfqpoint{1.009912in}{2.505626in}}{\pgfqpoint{1.005522in}{2.516225in}}{\pgfqpoint{0.997708in}{2.524038in}}%
\pgfpathcurveto{\pgfqpoint{0.989895in}{2.531852in}}{\pgfqpoint{0.979296in}{2.536242in}}{\pgfqpoint{0.968245in}{2.536242in}}%
\pgfpathcurveto{\pgfqpoint{0.957195in}{2.536242in}}{\pgfqpoint{0.946596in}{2.531852in}}{\pgfqpoint{0.938783in}{2.524038in}}%
\pgfpathcurveto{\pgfqpoint{0.930969in}{2.516225in}}{\pgfqpoint{0.926579in}{2.505626in}}{\pgfqpoint{0.926579in}{2.494576in}}%
\pgfpathcurveto{\pgfqpoint{0.926579in}{2.483525in}}{\pgfqpoint{0.930969in}{2.472926in}}{\pgfqpoint{0.938783in}{2.465113in}}%
\pgfpathcurveto{\pgfqpoint{0.946596in}{2.457299in}}{\pgfqpoint{0.957195in}{2.452909in}}{\pgfqpoint{0.968245in}{2.452909in}}%
\pgfpathlineto{\pgfqpoint{0.968245in}{2.452909in}}%
\pgfpathclose%
\pgfusepath{stroke}%
\end{pgfscope}%
\begin{pgfscope}%
\pgfpathrectangle{\pgfqpoint{0.393053in}{0.375000in}}{\pgfqpoint{6.356833in}{5.175000in}}%
\pgfusepath{clip}%
\pgfsetbuttcap%
\pgfsetroundjoin%
\pgfsetlinewidth{1.003750pt}%
\definecolor{currentstroke}{rgb}{1.000000,0.000000,0.000000}%
\pgfsetstrokecolor{currentstroke}%
\pgfsetdash{}{0pt}%
\pgfpathmoveto{\pgfqpoint{5.122378in}{0.351713in}}%
\pgfpathcurveto{\pgfqpoint{5.133428in}{0.351713in}}{\pgfqpoint{5.144027in}{0.356104in}}{\pgfqpoint{5.151841in}{0.363917in}}%
\pgfpathcurveto{\pgfqpoint{5.159654in}{0.371731in}}{\pgfqpoint{5.164045in}{0.382330in}}{\pgfqpoint{5.164045in}{0.393380in}}%
\pgfpathcurveto{\pgfqpoint{5.164045in}{0.404430in}}{\pgfqpoint{5.159654in}{0.415029in}}{\pgfqpoint{5.151841in}{0.422843in}}%
\pgfpathcurveto{\pgfqpoint{5.144027in}{0.430656in}}{\pgfqpoint{5.133428in}{0.435047in}}{\pgfqpoint{5.122378in}{0.435047in}}%
\pgfpathcurveto{\pgfqpoint{5.111328in}{0.435047in}}{\pgfqpoint{5.100729in}{0.430656in}}{\pgfqpoint{5.092915in}{0.422843in}}%
\pgfpathcurveto{\pgfqpoint{5.085102in}{0.415029in}}{\pgfqpoint{5.080711in}{0.404430in}}{\pgfqpoint{5.080711in}{0.393380in}}%
\pgfpathcurveto{\pgfqpoint{5.080711in}{0.382330in}}{\pgfqpoint{5.085102in}{0.371731in}}{\pgfqpoint{5.092915in}{0.363917in}}%
\pgfpathcurveto{\pgfqpoint{5.100729in}{0.356104in}}{\pgfqpoint{5.111328in}{0.351713in}}{\pgfqpoint{5.122378in}{0.351713in}}%
\pgfusepath{stroke}%
\end{pgfscope}%
\begin{pgfscope}%
\pgfpathrectangle{\pgfqpoint{0.393053in}{0.375000in}}{\pgfqpoint{6.356833in}{5.175000in}}%
\pgfusepath{clip}%
\pgfsetbuttcap%
\pgfsetroundjoin%
\pgfsetlinewidth{1.003750pt}%
\definecolor{currentstroke}{rgb}{1.000000,0.000000,0.000000}%
\pgfsetstrokecolor{currentstroke}%
\pgfsetdash{}{0pt}%
\pgfpathmoveto{\pgfqpoint{0.760157in}{2.812990in}}%
\pgfpathcurveto{\pgfqpoint{0.771207in}{2.812990in}}{\pgfqpoint{0.781806in}{2.817380in}}{\pgfqpoint{0.789620in}{2.825194in}}%
\pgfpathcurveto{\pgfqpoint{0.797434in}{2.833007in}}{\pgfqpoint{0.801824in}{2.843607in}}{\pgfqpoint{0.801824in}{2.854657in}}%
\pgfpathcurveto{\pgfqpoint{0.801824in}{2.865707in}}{\pgfqpoint{0.797434in}{2.876306in}}{\pgfqpoint{0.789620in}{2.884119in}}%
\pgfpathcurveto{\pgfqpoint{0.781806in}{2.891933in}}{\pgfqpoint{0.771207in}{2.896323in}}{\pgfqpoint{0.760157in}{2.896323in}}%
\pgfpathcurveto{\pgfqpoint{0.749107in}{2.896323in}}{\pgfqpoint{0.738508in}{2.891933in}}{\pgfqpoint{0.730694in}{2.884119in}}%
\pgfpathcurveto{\pgfqpoint{0.722881in}{2.876306in}}{\pgfqpoint{0.718490in}{2.865707in}}{\pgfqpoint{0.718490in}{2.854657in}}%
\pgfpathcurveto{\pgfqpoint{0.718490in}{2.843607in}}{\pgfqpoint{0.722881in}{2.833007in}}{\pgfqpoint{0.730694in}{2.825194in}}%
\pgfpathcurveto{\pgfqpoint{0.738508in}{2.817380in}}{\pgfqpoint{0.749107in}{2.812990in}}{\pgfqpoint{0.760157in}{2.812990in}}%
\pgfpathlineto{\pgfqpoint{0.760157in}{2.812990in}}%
\pgfpathclose%
\pgfusepath{stroke}%
\end{pgfscope}%
\begin{pgfscope}%
\pgfpathrectangle{\pgfqpoint{0.393053in}{0.375000in}}{\pgfqpoint{6.356833in}{5.175000in}}%
\pgfusepath{clip}%
\pgfsetbuttcap%
\pgfsetroundjoin%
\pgfsetlinewidth{1.003750pt}%
\definecolor{currentstroke}{rgb}{1.000000,0.000000,0.000000}%
\pgfsetstrokecolor{currentstroke}%
\pgfsetdash{}{0pt}%
\pgfpathmoveto{\pgfqpoint{0.489764in}{3.636170in}}%
\pgfpathcurveto{\pgfqpoint{0.500814in}{3.636170in}}{\pgfqpoint{0.511413in}{3.640560in}}{\pgfqpoint{0.519227in}{3.648374in}}%
\pgfpathcurveto{\pgfqpoint{0.527041in}{3.656188in}}{\pgfqpoint{0.531431in}{3.666787in}}{\pgfqpoint{0.531431in}{3.677837in}}%
\pgfpathcurveto{\pgfqpoint{0.531431in}{3.688887in}}{\pgfqpoint{0.527041in}{3.699486in}}{\pgfqpoint{0.519227in}{3.707300in}}%
\pgfpathcurveto{\pgfqpoint{0.511413in}{3.715113in}}{\pgfqpoint{0.500814in}{3.719504in}}{\pgfqpoint{0.489764in}{3.719504in}}%
\pgfpathcurveto{\pgfqpoint{0.478714in}{3.719504in}}{\pgfqpoint{0.468115in}{3.715113in}}{\pgfqpoint{0.460301in}{3.707300in}}%
\pgfpathcurveto{\pgfqpoint{0.452488in}{3.699486in}}{\pgfqpoint{0.448098in}{3.688887in}}{\pgfqpoint{0.448098in}{3.677837in}}%
\pgfpathcurveto{\pgfqpoint{0.448098in}{3.666787in}}{\pgfqpoint{0.452488in}{3.656188in}}{\pgfqpoint{0.460301in}{3.648374in}}%
\pgfpathcurveto{\pgfqpoint{0.468115in}{3.640560in}}{\pgfqpoint{0.478714in}{3.636170in}}{\pgfqpoint{0.489764in}{3.636170in}}%
\pgfpathlineto{\pgfqpoint{0.489764in}{3.636170in}}%
\pgfpathclose%
\pgfusepath{stroke}%
\end{pgfscope}%
\begin{pgfscope}%
\pgfpathrectangle{\pgfqpoint{0.393053in}{0.375000in}}{\pgfqpoint{6.356833in}{5.175000in}}%
\pgfusepath{clip}%
\pgfsetbuttcap%
\pgfsetroundjoin%
\pgfsetlinewidth{1.003750pt}%
\definecolor{currentstroke}{rgb}{1.000000,0.000000,0.000000}%
\pgfsetstrokecolor{currentstroke}%
\pgfsetdash{}{0pt}%
\pgfpathmoveto{\pgfqpoint{2.960491in}{0.831278in}}%
\pgfpathcurveto{\pgfqpoint{2.971541in}{0.831278in}}{\pgfqpoint{2.982140in}{0.835669in}}{\pgfqpoint{2.989954in}{0.843482in}}%
\pgfpathcurveto{\pgfqpoint{2.997767in}{0.851296in}}{\pgfqpoint{3.002158in}{0.861895in}}{\pgfqpoint{3.002158in}{0.872945in}}%
\pgfpathcurveto{\pgfqpoint{3.002158in}{0.883995in}}{\pgfqpoint{2.997767in}{0.894594in}}{\pgfqpoint{2.989954in}{0.902408in}}%
\pgfpathcurveto{\pgfqpoint{2.982140in}{0.910222in}}{\pgfqpoint{2.971541in}{0.914612in}}{\pgfqpoint{2.960491in}{0.914612in}}%
\pgfpathcurveto{\pgfqpoint{2.949441in}{0.914612in}}{\pgfqpoint{2.938842in}{0.910222in}}{\pgfqpoint{2.931028in}{0.902408in}}%
\pgfpathcurveto{\pgfqpoint{2.923215in}{0.894594in}}{\pgfqpoint{2.918824in}{0.883995in}}{\pgfqpoint{2.918824in}{0.872945in}}%
\pgfpathcurveto{\pgfqpoint{2.918824in}{0.861895in}}{\pgfqpoint{2.923215in}{0.851296in}}{\pgfqpoint{2.931028in}{0.843482in}}%
\pgfpathcurveto{\pgfqpoint{2.938842in}{0.835669in}}{\pgfqpoint{2.949441in}{0.831278in}}{\pgfqpoint{2.960491in}{0.831278in}}%
\pgfpathlineto{\pgfqpoint{2.960491in}{0.831278in}}%
\pgfpathclose%
\pgfusepath{stroke}%
\end{pgfscope}%
\begin{pgfscope}%
\pgfpathrectangle{\pgfqpoint{0.393053in}{0.375000in}}{\pgfqpoint{6.356833in}{5.175000in}}%
\pgfusepath{clip}%
\pgfsetbuttcap%
\pgfsetroundjoin%
\pgfsetlinewidth{1.003750pt}%
\definecolor{currentstroke}{rgb}{1.000000,0.000000,0.000000}%
\pgfsetstrokecolor{currentstroke}%
\pgfsetdash{}{0pt}%
\pgfpathmoveto{\pgfqpoint{3.411603in}{0.646276in}}%
\pgfpathcurveto{\pgfqpoint{3.422653in}{0.646276in}}{\pgfqpoint{3.433252in}{0.650667in}}{\pgfqpoint{3.441066in}{0.658480in}}%
\pgfpathcurveto{\pgfqpoint{3.448880in}{0.666294in}}{\pgfqpoint{3.453270in}{0.676893in}}{\pgfqpoint{3.453270in}{0.687943in}}%
\pgfpathcurveto{\pgfqpoint{3.453270in}{0.698993in}}{\pgfqpoint{3.448880in}{0.709592in}}{\pgfqpoint{3.441066in}{0.717406in}}%
\pgfpathcurveto{\pgfqpoint{3.433252in}{0.725219in}}{\pgfqpoint{3.422653in}{0.729610in}}{\pgfqpoint{3.411603in}{0.729610in}}%
\pgfpathcurveto{\pgfqpoint{3.400553in}{0.729610in}}{\pgfqpoint{3.389954in}{0.725219in}}{\pgfqpoint{3.382140in}{0.717406in}}%
\pgfpathcurveto{\pgfqpoint{3.374327in}{0.709592in}}{\pgfqpoint{3.369937in}{0.698993in}}{\pgfqpoint{3.369937in}{0.687943in}}%
\pgfpathcurveto{\pgfqpoint{3.369937in}{0.676893in}}{\pgfqpoint{3.374327in}{0.666294in}}{\pgfqpoint{3.382140in}{0.658480in}}%
\pgfpathcurveto{\pgfqpoint{3.389954in}{0.650667in}}{\pgfqpoint{3.400553in}{0.646276in}}{\pgfqpoint{3.411603in}{0.646276in}}%
\pgfpathlineto{\pgfqpoint{3.411603in}{0.646276in}}%
\pgfpathclose%
\pgfusepath{stroke}%
\end{pgfscope}%
\begin{pgfscope}%
\pgfpathrectangle{\pgfqpoint{0.393053in}{0.375000in}}{\pgfqpoint{6.356833in}{5.175000in}}%
\pgfusepath{clip}%
\pgfsetbuttcap%
\pgfsetroundjoin%
\pgfsetlinewidth{1.003750pt}%
\definecolor{currentstroke}{rgb}{1.000000,0.000000,0.000000}%
\pgfsetstrokecolor{currentstroke}%
\pgfsetdash{}{0pt}%
\pgfpathmoveto{\pgfqpoint{2.972703in}{0.826510in}}%
\pgfpathcurveto{\pgfqpoint{2.983753in}{0.826510in}}{\pgfqpoint{2.994352in}{0.830901in}}{\pgfqpoint{3.002165in}{0.838714in}}%
\pgfpathcurveto{\pgfqpoint{3.009979in}{0.846528in}}{\pgfqpoint{3.014369in}{0.857127in}}{\pgfqpoint{3.014369in}{0.868177in}}%
\pgfpathcurveto{\pgfqpoint{3.014369in}{0.879227in}}{\pgfqpoint{3.009979in}{0.889826in}}{\pgfqpoint{3.002165in}{0.897640in}}%
\pgfpathcurveto{\pgfqpoint{2.994352in}{0.905453in}}{\pgfqpoint{2.983753in}{0.909844in}}{\pgfqpoint{2.972703in}{0.909844in}}%
\pgfpathcurveto{\pgfqpoint{2.961652in}{0.909844in}}{\pgfqpoint{2.951053in}{0.905453in}}{\pgfqpoint{2.943240in}{0.897640in}}%
\pgfpathcurveto{\pgfqpoint{2.935426in}{0.889826in}}{\pgfqpoint{2.931036in}{0.879227in}}{\pgfqpoint{2.931036in}{0.868177in}}%
\pgfpathcurveto{\pgfqpoint{2.931036in}{0.857127in}}{\pgfqpoint{2.935426in}{0.846528in}}{\pgfqpoint{2.943240in}{0.838714in}}%
\pgfpathcurveto{\pgfqpoint{2.951053in}{0.830901in}}{\pgfqpoint{2.961652in}{0.826510in}}{\pgfqpoint{2.972703in}{0.826510in}}%
\pgfpathlineto{\pgfqpoint{2.972703in}{0.826510in}}%
\pgfpathclose%
\pgfusepath{stroke}%
\end{pgfscope}%
\begin{pgfscope}%
\pgfpathrectangle{\pgfqpoint{0.393053in}{0.375000in}}{\pgfqpoint{6.356833in}{5.175000in}}%
\pgfusepath{clip}%
\pgfsetbuttcap%
\pgfsetroundjoin%
\pgfsetlinewidth{1.003750pt}%
\definecolor{currentstroke}{rgb}{1.000000,0.000000,0.000000}%
\pgfsetstrokecolor{currentstroke}%
\pgfsetdash{}{0pt}%
\pgfpathmoveto{\pgfqpoint{0.988989in}{2.400406in}}%
\pgfpathcurveto{\pgfqpoint{1.000039in}{2.400406in}}{\pgfqpoint{1.010638in}{2.404797in}}{\pgfqpoint{1.018452in}{2.412610in}}%
\pgfpathcurveto{\pgfqpoint{1.026265in}{2.420424in}}{\pgfqpoint{1.030656in}{2.431023in}}{\pgfqpoint{1.030656in}{2.442073in}}%
\pgfpathcurveto{\pgfqpoint{1.030656in}{2.453123in}}{\pgfqpoint{1.026265in}{2.463722in}}{\pgfqpoint{1.018452in}{2.471536in}}%
\pgfpathcurveto{\pgfqpoint{1.010638in}{2.479349in}}{\pgfqpoint{1.000039in}{2.483740in}}{\pgfqpoint{0.988989in}{2.483740in}}%
\pgfpathcurveto{\pgfqpoint{0.977939in}{2.483740in}}{\pgfqpoint{0.967340in}{2.479349in}}{\pgfqpoint{0.959526in}{2.471536in}}%
\pgfpathcurveto{\pgfqpoint{0.951713in}{2.463722in}}{\pgfqpoint{0.947322in}{2.453123in}}{\pgfqpoint{0.947322in}{2.442073in}}%
\pgfpathcurveto{\pgfqpoint{0.947322in}{2.431023in}}{\pgfqpoint{0.951713in}{2.420424in}}{\pgfqpoint{0.959526in}{2.412610in}}%
\pgfpathcurveto{\pgfqpoint{0.967340in}{2.404797in}}{\pgfqpoint{0.977939in}{2.400406in}}{\pgfqpoint{0.988989in}{2.400406in}}%
\pgfpathlineto{\pgfqpoint{0.988989in}{2.400406in}}%
\pgfpathclose%
\pgfusepath{stroke}%
\end{pgfscope}%
\begin{pgfscope}%
\pgfpathrectangle{\pgfqpoint{0.393053in}{0.375000in}}{\pgfqpoint{6.356833in}{5.175000in}}%
\pgfusepath{clip}%
\pgfsetbuttcap%
\pgfsetroundjoin%
\pgfsetlinewidth{1.003750pt}%
\definecolor{currentstroke}{rgb}{1.000000,0.000000,0.000000}%
\pgfsetstrokecolor{currentstroke}%
\pgfsetdash{}{0pt}%
\pgfpathmoveto{\pgfqpoint{3.855417in}{0.520863in}}%
\pgfpathcurveto{\pgfqpoint{3.866467in}{0.520863in}}{\pgfqpoint{3.877066in}{0.525253in}}{\pgfqpoint{3.884880in}{0.533067in}}%
\pgfpathcurveto{\pgfqpoint{3.892693in}{0.540881in}}{\pgfqpoint{3.897084in}{0.551480in}}{\pgfqpoint{3.897084in}{0.562530in}}%
\pgfpathcurveto{\pgfqpoint{3.897084in}{0.573580in}}{\pgfqpoint{3.892693in}{0.584179in}}{\pgfqpoint{3.884880in}{0.591993in}}%
\pgfpathcurveto{\pgfqpoint{3.877066in}{0.599806in}}{\pgfqpoint{3.866467in}{0.604196in}}{\pgfqpoint{3.855417in}{0.604196in}}%
\pgfpathcurveto{\pgfqpoint{3.844367in}{0.604196in}}{\pgfqpoint{3.833768in}{0.599806in}}{\pgfqpoint{3.825954in}{0.591993in}}%
\pgfpathcurveto{\pgfqpoint{3.818141in}{0.584179in}}{\pgfqpoint{3.813750in}{0.573580in}}{\pgfqpoint{3.813750in}{0.562530in}}%
\pgfpathcurveto{\pgfqpoint{3.813750in}{0.551480in}}{\pgfqpoint{3.818141in}{0.540881in}}{\pgfqpoint{3.825954in}{0.533067in}}%
\pgfpathcurveto{\pgfqpoint{3.833768in}{0.525253in}}{\pgfqpoint{3.844367in}{0.520863in}}{\pgfqpoint{3.855417in}{0.520863in}}%
\pgfpathlineto{\pgfqpoint{3.855417in}{0.520863in}}%
\pgfpathclose%
\pgfusepath{stroke}%
\end{pgfscope}%
\begin{pgfscope}%
\pgfpathrectangle{\pgfqpoint{0.393053in}{0.375000in}}{\pgfqpoint{6.356833in}{5.175000in}}%
\pgfusepath{clip}%
\pgfsetbuttcap%
\pgfsetroundjoin%
\pgfsetlinewidth{1.003750pt}%
\definecolor{currentstroke}{rgb}{1.000000,0.000000,0.000000}%
\pgfsetstrokecolor{currentstroke}%
\pgfsetdash{}{0pt}%
\pgfpathmoveto{\pgfqpoint{3.113675in}{0.760282in}}%
\pgfpathcurveto{\pgfqpoint{3.124725in}{0.760282in}}{\pgfqpoint{3.135324in}{0.764672in}}{\pgfqpoint{3.143138in}{0.772486in}}%
\pgfpathcurveto{\pgfqpoint{3.150952in}{0.780299in}}{\pgfqpoint{3.155342in}{0.790898in}}{\pgfqpoint{3.155342in}{0.801949in}}%
\pgfpathcurveto{\pgfqpoint{3.155342in}{0.812999in}}{\pgfqpoint{3.150952in}{0.823598in}}{\pgfqpoint{3.143138in}{0.831411in}}%
\pgfpathcurveto{\pgfqpoint{3.135324in}{0.839225in}}{\pgfqpoint{3.124725in}{0.843615in}}{\pgfqpoint{3.113675in}{0.843615in}}%
\pgfpathcurveto{\pgfqpoint{3.102625in}{0.843615in}}{\pgfqpoint{3.092026in}{0.839225in}}{\pgfqpoint{3.084212in}{0.831411in}}%
\pgfpathcurveto{\pgfqpoint{3.076399in}{0.823598in}}{\pgfqpoint{3.072009in}{0.812999in}}{\pgfqpoint{3.072009in}{0.801949in}}%
\pgfpathcurveto{\pgfqpoint{3.072009in}{0.790898in}}{\pgfqpoint{3.076399in}{0.780299in}}{\pgfqpoint{3.084212in}{0.772486in}}%
\pgfpathcurveto{\pgfqpoint{3.092026in}{0.764672in}}{\pgfqpoint{3.102625in}{0.760282in}}{\pgfqpoint{3.113675in}{0.760282in}}%
\pgfpathlineto{\pgfqpoint{3.113675in}{0.760282in}}%
\pgfpathclose%
\pgfusepath{stroke}%
\end{pgfscope}%
\begin{pgfscope}%
\pgfpathrectangle{\pgfqpoint{0.393053in}{0.375000in}}{\pgfqpoint{6.356833in}{5.175000in}}%
\pgfusepath{clip}%
\pgfsetbuttcap%
\pgfsetroundjoin%
\pgfsetlinewidth{1.003750pt}%
\definecolor{currentstroke}{rgb}{1.000000,0.000000,0.000000}%
\pgfsetstrokecolor{currentstroke}%
\pgfsetdash{}{0pt}%
\pgfpathmoveto{\pgfqpoint{0.687786in}{2.976853in}}%
\pgfpathcurveto{\pgfqpoint{0.698836in}{2.976853in}}{\pgfqpoint{0.709435in}{2.981244in}}{\pgfqpoint{0.717249in}{2.989057in}}%
\pgfpathcurveto{\pgfqpoint{0.725063in}{2.996871in}}{\pgfqpoint{0.729453in}{3.007470in}}{\pgfqpoint{0.729453in}{3.018520in}}%
\pgfpathcurveto{\pgfqpoint{0.729453in}{3.029570in}}{\pgfqpoint{0.725063in}{3.040169in}}{\pgfqpoint{0.717249in}{3.047983in}}%
\pgfpathcurveto{\pgfqpoint{0.709435in}{3.055796in}}{\pgfqpoint{0.698836in}{3.060187in}}{\pgfqpoint{0.687786in}{3.060187in}}%
\pgfpathcurveto{\pgfqpoint{0.676736in}{3.060187in}}{\pgfqpoint{0.666137in}{3.055796in}}{\pgfqpoint{0.658324in}{3.047983in}}%
\pgfpathcurveto{\pgfqpoint{0.650510in}{3.040169in}}{\pgfqpoint{0.646120in}{3.029570in}}{\pgfqpoint{0.646120in}{3.018520in}}%
\pgfpathcurveto{\pgfqpoint{0.646120in}{3.007470in}}{\pgfqpoint{0.650510in}{2.996871in}}{\pgfqpoint{0.658324in}{2.989057in}}%
\pgfpathcurveto{\pgfqpoint{0.666137in}{2.981244in}}{\pgfqpoint{0.676736in}{2.976853in}}{\pgfqpoint{0.687786in}{2.976853in}}%
\pgfpathlineto{\pgfqpoint{0.687786in}{2.976853in}}%
\pgfpathclose%
\pgfusepath{stroke}%
\end{pgfscope}%
\begin{pgfscope}%
\pgfpathrectangle{\pgfqpoint{0.393053in}{0.375000in}}{\pgfqpoint{6.356833in}{5.175000in}}%
\pgfusepath{clip}%
\pgfsetbuttcap%
\pgfsetroundjoin%
\pgfsetlinewidth{1.003750pt}%
\definecolor{currentstroke}{rgb}{1.000000,0.000000,0.000000}%
\pgfsetstrokecolor{currentstroke}%
\pgfsetdash{}{0pt}%
\pgfpathmoveto{\pgfqpoint{0.722783in}{2.894569in}}%
\pgfpathcurveto{\pgfqpoint{0.733833in}{2.894569in}}{\pgfqpoint{0.744432in}{2.898960in}}{\pgfqpoint{0.752246in}{2.906773in}}%
\pgfpathcurveto{\pgfqpoint{0.760059in}{2.914587in}}{\pgfqpoint{0.764449in}{2.925186in}}{\pgfqpoint{0.764449in}{2.936236in}}%
\pgfpathcurveto{\pgfqpoint{0.764449in}{2.947286in}}{\pgfqpoint{0.760059in}{2.957885in}}{\pgfqpoint{0.752246in}{2.965699in}}%
\pgfpathcurveto{\pgfqpoint{0.744432in}{2.973512in}}{\pgfqpoint{0.733833in}{2.977903in}}{\pgfqpoint{0.722783in}{2.977903in}}%
\pgfpathcurveto{\pgfqpoint{0.711733in}{2.977903in}}{\pgfqpoint{0.701134in}{2.973512in}}{\pgfqpoint{0.693320in}{2.965699in}}%
\pgfpathcurveto{\pgfqpoint{0.685506in}{2.957885in}}{\pgfqpoint{0.681116in}{2.947286in}}{\pgfqpoint{0.681116in}{2.936236in}}%
\pgfpathcurveto{\pgfqpoint{0.681116in}{2.925186in}}{\pgfqpoint{0.685506in}{2.914587in}}{\pgfqpoint{0.693320in}{2.906773in}}%
\pgfpathcurveto{\pgfqpoint{0.701134in}{2.898960in}}{\pgfqpoint{0.711733in}{2.894569in}}{\pgfqpoint{0.722783in}{2.894569in}}%
\pgfpathlineto{\pgfqpoint{0.722783in}{2.894569in}}%
\pgfpathclose%
\pgfusepath{stroke}%
\end{pgfscope}%
\begin{pgfscope}%
\pgfpathrectangle{\pgfqpoint{0.393053in}{0.375000in}}{\pgfqpoint{6.356833in}{5.175000in}}%
\pgfusepath{clip}%
\pgfsetbuttcap%
\pgfsetroundjoin%
\pgfsetlinewidth{1.003750pt}%
\definecolor{currentstroke}{rgb}{1.000000,0.000000,0.000000}%
\pgfsetstrokecolor{currentstroke}%
\pgfsetdash{}{0pt}%
\pgfpathmoveto{\pgfqpoint{3.047147in}{0.786875in}}%
\pgfpathcurveto{\pgfqpoint{3.058197in}{0.786875in}}{\pgfqpoint{3.068796in}{0.791266in}}{\pgfqpoint{3.076610in}{0.799079in}}%
\pgfpathcurveto{\pgfqpoint{3.084424in}{0.806893in}}{\pgfqpoint{3.088814in}{0.817492in}}{\pgfqpoint{3.088814in}{0.828542in}}%
\pgfpathcurveto{\pgfqpoint{3.088814in}{0.839592in}}{\pgfqpoint{3.084424in}{0.850191in}}{\pgfqpoint{3.076610in}{0.858005in}}%
\pgfpathcurveto{\pgfqpoint{3.068796in}{0.865818in}}{\pgfqpoint{3.058197in}{0.870209in}}{\pgfqpoint{3.047147in}{0.870209in}}%
\pgfpathcurveto{\pgfqpoint{3.036097in}{0.870209in}}{\pgfqpoint{3.025498in}{0.865818in}}{\pgfqpoint{3.017684in}{0.858005in}}%
\pgfpathcurveto{\pgfqpoint{3.009871in}{0.850191in}}{\pgfqpoint{3.005481in}{0.839592in}}{\pgfqpoint{3.005481in}{0.828542in}}%
\pgfpathcurveto{\pgfqpoint{3.005481in}{0.817492in}}{\pgfqpoint{3.009871in}{0.806893in}}{\pgfqpoint{3.017684in}{0.799079in}}%
\pgfpathcurveto{\pgfqpoint{3.025498in}{0.791266in}}{\pgfqpoint{3.036097in}{0.786875in}}{\pgfqpoint{3.047147in}{0.786875in}}%
\pgfpathlineto{\pgfqpoint{3.047147in}{0.786875in}}%
\pgfpathclose%
\pgfusepath{stroke}%
\end{pgfscope}%
\begin{pgfscope}%
\pgfpathrectangle{\pgfqpoint{0.393053in}{0.375000in}}{\pgfqpoint{6.356833in}{5.175000in}}%
\pgfusepath{clip}%
\pgfsetbuttcap%
\pgfsetroundjoin%
\pgfsetlinewidth{1.003750pt}%
\definecolor{currentstroke}{rgb}{1.000000,0.000000,0.000000}%
\pgfsetstrokecolor{currentstroke}%
\pgfsetdash{}{0pt}%
\pgfpathmoveto{\pgfqpoint{0.559943in}{3.351957in}}%
\pgfpathcurveto{\pgfqpoint{0.570993in}{3.351957in}}{\pgfqpoint{0.581592in}{3.356347in}}{\pgfqpoint{0.589406in}{3.364161in}}%
\pgfpathcurveto{\pgfqpoint{0.597220in}{3.371974in}}{\pgfqpoint{0.601610in}{3.382573in}}{\pgfqpoint{0.601610in}{3.393623in}}%
\pgfpathcurveto{\pgfqpoint{0.601610in}{3.404673in}}{\pgfqpoint{0.597220in}{3.415273in}}{\pgfqpoint{0.589406in}{3.423086in}}%
\pgfpathcurveto{\pgfqpoint{0.581592in}{3.430900in}}{\pgfqpoint{0.570993in}{3.435290in}}{\pgfqpoint{0.559943in}{3.435290in}}%
\pgfpathcurveto{\pgfqpoint{0.548893in}{3.435290in}}{\pgfqpoint{0.538294in}{3.430900in}}{\pgfqpoint{0.530480in}{3.423086in}}%
\pgfpathcurveto{\pgfqpoint{0.522667in}{3.415273in}}{\pgfqpoint{0.518276in}{3.404673in}}{\pgfqpoint{0.518276in}{3.393623in}}%
\pgfpathcurveto{\pgfqpoint{0.518276in}{3.382573in}}{\pgfqpoint{0.522667in}{3.371974in}}{\pgfqpoint{0.530480in}{3.364161in}}%
\pgfpathcurveto{\pgfqpoint{0.538294in}{3.356347in}}{\pgfqpoint{0.548893in}{3.351957in}}{\pgfqpoint{0.559943in}{3.351957in}}%
\pgfpathlineto{\pgfqpoint{0.559943in}{3.351957in}}%
\pgfpathclose%
\pgfusepath{stroke}%
\end{pgfscope}%
\begin{pgfscope}%
\pgfpathrectangle{\pgfqpoint{0.393053in}{0.375000in}}{\pgfqpoint{6.356833in}{5.175000in}}%
\pgfusepath{clip}%
\pgfsetbuttcap%
\pgfsetroundjoin%
\pgfsetlinewidth{1.003750pt}%
\definecolor{currentstroke}{rgb}{1.000000,0.000000,0.000000}%
\pgfsetstrokecolor{currentstroke}%
\pgfsetdash{}{0pt}%
\pgfpathmoveto{\pgfqpoint{0.398120in}{4.371940in}}%
\pgfpathcurveto{\pgfqpoint{0.409170in}{4.371940in}}{\pgfqpoint{0.419769in}{4.376331in}}{\pgfqpoint{0.427583in}{4.384144in}}%
\pgfpathcurveto{\pgfqpoint{0.435396in}{4.391958in}}{\pgfqpoint{0.439787in}{4.402557in}}{\pgfqpoint{0.439787in}{4.413607in}}%
\pgfpathcurveto{\pgfqpoint{0.439787in}{4.424657in}}{\pgfqpoint{0.435396in}{4.435256in}}{\pgfqpoint{0.427583in}{4.443070in}}%
\pgfpathcurveto{\pgfqpoint{0.419769in}{4.450883in}}{\pgfqpoint{0.409170in}{4.455274in}}{\pgfqpoint{0.398120in}{4.455274in}}%
\pgfpathcurveto{\pgfqpoint{0.387070in}{4.455274in}}{\pgfqpoint{0.376471in}{4.450883in}}{\pgfqpoint{0.368657in}{4.443070in}}%
\pgfpathcurveto{\pgfqpoint{0.360844in}{4.435256in}}{\pgfqpoint{0.356453in}{4.424657in}}{\pgfqpoint{0.356453in}{4.413607in}}%
\pgfpathcurveto{\pgfqpoint{0.356453in}{4.402557in}}{\pgfqpoint{0.360844in}{4.391958in}}{\pgfqpoint{0.368657in}{4.384144in}}%
\pgfpathcurveto{\pgfqpoint{0.376471in}{4.376331in}}{\pgfqpoint{0.387070in}{4.371940in}}{\pgfqpoint{0.398120in}{4.371940in}}%
\pgfpathlineto{\pgfqpoint{0.398120in}{4.371940in}}%
\pgfpathclose%
\pgfusepath{stroke}%
\end{pgfscope}%
\begin{pgfscope}%
\pgfpathrectangle{\pgfqpoint{0.393053in}{0.375000in}}{\pgfqpoint{6.356833in}{5.175000in}}%
\pgfusepath{clip}%
\pgfsetbuttcap%
\pgfsetroundjoin%
\definecolor{currentfill}{rgb}{0.121569,0.466667,0.705882}%
\pgfsetfillcolor{currentfill}%
\pgfsetlinewidth{1.003750pt}%
\definecolor{currentstroke}{rgb}{0.121569,0.466667,0.705882}%
\pgfsetstrokecolor{currentstroke}%
\pgfsetdash{}{0pt}%
\pgfpathmoveto{\pgfqpoint{0.630960in}{3.124034in}}%
\pgfpathcurveto{\pgfqpoint{0.647432in}{3.124034in}}{\pgfqpoint{0.663232in}{3.130578in}}{\pgfqpoint{0.674880in}{3.142226in}}%
\pgfpathcurveto{\pgfqpoint{0.686528in}{3.153874in}}{\pgfqpoint{0.693073in}{3.169674in}}{\pgfqpoint{0.693073in}{3.186147in}}%
\pgfpathcurveto{\pgfqpoint{0.693073in}{3.202619in}}{\pgfqpoint{0.686528in}{3.218419in}}{\pgfqpoint{0.674880in}{3.230067in}}%
\pgfpathcurveto{\pgfqpoint{0.663232in}{3.241715in}}{\pgfqpoint{0.647432in}{3.248260in}}{\pgfqpoint{0.630960in}{3.248260in}}%
\pgfpathcurveto{\pgfqpoint{0.614487in}{3.248260in}}{\pgfqpoint{0.598687in}{3.241715in}}{\pgfqpoint{0.587039in}{3.230067in}}%
\pgfpathcurveto{\pgfqpoint{0.575391in}{3.218419in}}{\pgfqpoint{0.568847in}{3.202619in}}{\pgfqpoint{0.568847in}{3.186147in}}%
\pgfpathcurveto{\pgfqpoint{0.568847in}{3.169674in}}{\pgfqpoint{0.575391in}{3.153874in}}{\pgfqpoint{0.587039in}{3.142226in}}%
\pgfpathcurveto{\pgfqpoint{0.598687in}{3.130578in}}{\pgfqpoint{0.614487in}{3.124034in}}{\pgfqpoint{0.630960in}{3.124034in}}%
\pgfpathlineto{\pgfqpoint{0.630960in}{3.124034in}}%
\pgfpathclose%
\pgfusepath{stroke,fill}%
\end{pgfscope}%
\begin{pgfscope}%
\pgfpathrectangle{\pgfqpoint{0.393053in}{0.375000in}}{\pgfqpoint{6.356833in}{5.175000in}}%
\pgfusepath{clip}%
\pgfsetbuttcap%
\pgfsetroundjoin%
\definecolor{currentfill}{rgb}{0.172549,0.627451,0.172549}%
\pgfsetfillcolor{currentfill}%
\pgfsetlinewidth{1.003750pt}%
\definecolor{currentstroke}{rgb}{0.172549,0.627451,0.172549}%
\pgfsetstrokecolor{currentstroke}%
\pgfsetdash{}{0pt}%
\pgfpathmoveto{\pgfqpoint{5.710861in}{0.318738in}}%
\pgfpathcurveto{\pgfqpoint{5.727333in}{0.318738in}}{\pgfqpoint{5.743133in}{0.325282in}}{\pgfqpoint{5.754781in}{0.336930in}}%
\pgfpathcurveto{\pgfqpoint{5.766429in}{0.348578in}}{\pgfqpoint{5.772974in}{0.364378in}}{\pgfqpoint{5.772974in}{0.380851in}}%
\pgfpathcurveto{\pgfqpoint{5.772974in}{0.397323in}}{\pgfqpoint{5.766429in}{0.413123in}}{\pgfqpoint{5.754781in}{0.424771in}}%
\pgfpathcurveto{\pgfqpoint{5.743133in}{0.436419in}}{\pgfqpoint{5.727333in}{0.442964in}}{\pgfqpoint{5.710861in}{0.442964in}}%
\pgfpathcurveto{\pgfqpoint{5.694388in}{0.442964in}}{\pgfqpoint{5.678588in}{0.436419in}}{\pgfqpoint{5.666940in}{0.424771in}}%
\pgfpathcurveto{\pgfqpoint{5.655292in}{0.413123in}}{\pgfqpoint{5.648748in}{0.397323in}}{\pgfqpoint{5.648748in}{0.380851in}}%
\pgfpathcurveto{\pgfqpoint{5.648748in}{0.364378in}}{\pgfqpoint{5.655292in}{0.348578in}}{\pgfqpoint{5.666940in}{0.336930in}}%
\pgfpathcurveto{\pgfqpoint{5.678588in}{0.325282in}}{\pgfqpoint{5.694388in}{0.318738in}}{\pgfqpoint{5.710861in}{0.318738in}}%
\pgfpathlineto{\pgfqpoint{5.710861in}{0.318738in}}%
\pgfpathclose%
\pgfusepath{stroke,fill}%
\end{pgfscope}%
\begin{pgfscope}%
\pgfpathrectangle{\pgfqpoint{0.393053in}{0.375000in}}{\pgfqpoint{6.356833in}{5.175000in}}%
\pgfusepath{clip}%
\pgfsetbuttcap%
\pgfsetroundjoin%
\definecolor{currentfill}{rgb}{0.580392,0.403922,0.741176}%
\pgfsetfillcolor{currentfill}%
\pgfsetlinewidth{1.003750pt}%
\definecolor{currentstroke}{rgb}{0.580392,0.403922,0.741176}%
\pgfsetstrokecolor{currentstroke}%
\pgfsetdash{}{0pt}%
\pgfpathmoveto{\pgfqpoint{0.394494in}{4.470565in}}%
\pgfpathcurveto{\pgfqpoint{0.410966in}{4.470565in}}{\pgfqpoint{0.426766in}{4.477110in}}{\pgfqpoint{0.438414in}{4.488757in}}%
\pgfpathcurveto{\pgfqpoint{0.450062in}{4.500405in}}{\pgfqpoint{0.456607in}{4.516205in}}{\pgfqpoint{0.456607in}{4.532678in}}%
\pgfpathcurveto{\pgfqpoint{0.456607in}{4.549150in}}{\pgfqpoint{0.450062in}{4.564951in}}{\pgfqpoint{0.438414in}{4.576598in}}%
\pgfpathcurveto{\pgfqpoint{0.426766in}{4.588246in}}{\pgfqpoint{0.410966in}{4.594791in}}{\pgfqpoint{0.394494in}{4.594791in}}%
\pgfpathcurveto{\pgfqpoint{0.378021in}{4.594791in}}{\pgfqpoint{0.362221in}{4.588246in}}{\pgfqpoint{0.350573in}{4.576598in}}%
\pgfpathcurveto{\pgfqpoint{0.338925in}{4.564951in}}{\pgfqpoint{0.332381in}{4.549150in}}{\pgfqpoint{0.332381in}{4.532678in}}%
\pgfpathcurveto{\pgfqpoint{0.332381in}{4.516205in}}{\pgfqpoint{0.338925in}{4.500405in}}{\pgfqpoint{0.350573in}{4.488757in}}%
\pgfpathcurveto{\pgfqpoint{0.362221in}{4.477110in}}{\pgfqpoint{0.378021in}{4.470565in}}{\pgfqpoint{0.394494in}{4.470565in}}%
\pgfpathlineto{\pgfqpoint{0.394494in}{4.470565in}}%
\pgfpathclose%
\pgfusepath{stroke,fill}%
\end{pgfscope}%
\begin{pgfscope}%
\pgfpathrectangle{\pgfqpoint{0.393053in}{0.375000in}}{\pgfqpoint{6.356833in}{5.175000in}}%
\pgfusepath{clip}%
\pgfsetbuttcap%
\pgfsetroundjoin%
\definecolor{currentfill}{rgb}{0.890196,0.466667,0.760784}%
\pgfsetfillcolor{currentfill}%
\pgfsetlinewidth{1.003750pt}%
\definecolor{currentstroke}{rgb}{0.890196,0.466667,0.760784}%
\pgfsetstrokecolor{currentstroke}%
\pgfsetdash{}{0pt}%
\pgfpathmoveto{\pgfqpoint{5.564780in}{0.411879in}}%
\pgfpathcurveto{\pgfqpoint{5.581253in}{0.411879in}}{\pgfqpoint{5.597053in}{0.418424in}}{\pgfqpoint{5.608701in}{0.430072in}}%
\pgfpathcurveto{\pgfqpoint{5.620349in}{0.441720in}}{\pgfqpoint{5.626893in}{0.457520in}}{\pgfqpoint{5.626893in}{0.473992in}}%
\pgfpathcurveto{\pgfqpoint{5.626893in}{0.490465in}}{\pgfqpoint{5.620349in}{0.506265in}}{\pgfqpoint{5.608701in}{0.517913in}}%
\pgfpathcurveto{\pgfqpoint{5.597053in}{0.529561in}}{\pgfqpoint{5.581253in}{0.536105in}}{\pgfqpoint{5.564780in}{0.536105in}}%
\pgfpathcurveto{\pgfqpoint{5.548308in}{0.536105in}}{\pgfqpoint{5.532508in}{0.529561in}}{\pgfqpoint{5.520860in}{0.517913in}}%
\pgfpathcurveto{\pgfqpoint{5.509212in}{0.506265in}}{\pgfqpoint{5.502667in}{0.490465in}}{\pgfqpoint{5.502667in}{0.473992in}}%
\pgfpathcurveto{\pgfqpoint{5.502667in}{0.457520in}}{\pgfqpoint{5.509212in}{0.441720in}}{\pgfqpoint{5.520860in}{0.430072in}}%
\pgfpathcurveto{\pgfqpoint{5.532508in}{0.418424in}}{\pgfqpoint{5.548308in}{0.411879in}}{\pgfqpoint{5.564780in}{0.411879in}}%
\pgfpathlineto{\pgfqpoint{5.564780in}{0.411879in}}%
\pgfpathclose%
\pgfusepath{stroke,fill}%
\end{pgfscope}%
\begin{pgfscope}%
\pgfpathrectangle{\pgfqpoint{0.393053in}{0.375000in}}{\pgfqpoint{6.356833in}{5.175000in}}%
\pgfusepath{clip}%
\pgfsetbuttcap%
\pgfsetroundjoin%
\definecolor{currentfill}{rgb}{0.737255,0.741176,0.133333}%
\pgfsetfillcolor{currentfill}%
\pgfsetlinewidth{1.003750pt}%
\definecolor{currentstroke}{rgb}{0.737255,0.741176,0.133333}%
\pgfsetstrokecolor{currentstroke}%
\pgfsetdash{}{0pt}%
\pgfpathmoveto{\pgfqpoint{1.053407in}{2.857975in}}%
\pgfpathcurveto{\pgfqpoint{1.069880in}{2.857975in}}{\pgfqpoint{1.085680in}{2.864519in}}{\pgfqpoint{1.097328in}{2.876167in}}%
\pgfpathcurveto{\pgfqpoint{1.108976in}{2.887815in}}{\pgfqpoint{1.115520in}{2.903615in}}{\pgfqpoint{1.115520in}{2.920088in}}%
\pgfpathcurveto{\pgfqpoint{1.115520in}{2.936560in}}{\pgfqpoint{1.108976in}{2.952361in}}{\pgfqpoint{1.097328in}{2.964008in}}%
\pgfpathcurveto{\pgfqpoint{1.085680in}{2.975656in}}{\pgfqpoint{1.069880in}{2.982201in}}{\pgfqpoint{1.053407in}{2.982201in}}%
\pgfpathcurveto{\pgfqpoint{1.036935in}{2.982201in}}{\pgfqpoint{1.021135in}{2.975656in}}{\pgfqpoint{1.009487in}{2.964008in}}%
\pgfpathcurveto{\pgfqpoint{0.997839in}{2.952361in}}{\pgfqpoint{0.991294in}{2.936560in}}{\pgfqpoint{0.991294in}{2.920088in}}%
\pgfpathcurveto{\pgfqpoint{0.991294in}{2.903615in}}{\pgfqpoint{0.997839in}{2.887815in}}{\pgfqpoint{1.009487in}{2.876167in}}%
\pgfpathcurveto{\pgfqpoint{1.021135in}{2.864519in}}{\pgfqpoint{1.036935in}{2.857975in}}{\pgfqpoint{1.053407in}{2.857975in}}%
\pgfpathlineto{\pgfqpoint{1.053407in}{2.857975in}}%
\pgfpathclose%
\pgfusepath{stroke,fill}%
\end{pgfscope}%
\begin{pgfscope}%
\pgfpathrectangle{\pgfqpoint{0.393053in}{0.375000in}}{\pgfqpoint{6.356833in}{5.175000in}}%
\pgfusepath{clip}%
\pgfsetbuttcap%
\pgfsetroundjoin%
\definecolor{currentfill}{rgb}{0.090196,0.745098,0.811765}%
\pgfsetfillcolor{currentfill}%
\pgfsetlinewidth{1.003750pt}%
\definecolor{currentstroke}{rgb}{0.090196,0.745098,0.811765}%
\pgfsetstrokecolor{currentstroke}%
\pgfsetdash{}{0pt}%
\pgfpathmoveto{\pgfqpoint{5.707928in}{0.316899in}}%
\pgfpathcurveto{\pgfqpoint{5.724401in}{0.316899in}}{\pgfqpoint{5.740201in}{0.323444in}}{\pgfqpoint{5.751849in}{0.335092in}}%
\pgfpathcurveto{\pgfqpoint{5.763497in}{0.346740in}}{\pgfqpoint{5.770041in}{0.362540in}}{\pgfqpoint{5.770041in}{0.379012in}}%
\pgfpathcurveto{\pgfqpoint{5.770041in}{0.395485in}}{\pgfqpoint{5.763497in}{0.411285in}}{\pgfqpoint{5.751849in}{0.422933in}}%
\pgfpathcurveto{\pgfqpoint{5.740201in}{0.434581in}}{\pgfqpoint{5.724401in}{0.441125in}}{\pgfqpoint{5.707928in}{0.441125in}}%
\pgfpathcurveto{\pgfqpoint{5.691456in}{0.441125in}}{\pgfqpoint{5.675656in}{0.434581in}}{\pgfqpoint{5.664008in}{0.422933in}}%
\pgfpathcurveto{\pgfqpoint{5.652360in}{0.411285in}}{\pgfqpoint{5.645815in}{0.395485in}}{\pgfqpoint{5.645815in}{0.379012in}}%
\pgfpathcurveto{\pgfqpoint{5.645815in}{0.362540in}}{\pgfqpoint{5.652360in}{0.346740in}}{\pgfqpoint{5.664008in}{0.335092in}}%
\pgfpathcurveto{\pgfqpoint{5.675656in}{0.323444in}}{\pgfqpoint{5.691456in}{0.316899in}}{\pgfqpoint{5.707928in}{0.316899in}}%
\pgfpathlineto{\pgfqpoint{5.707928in}{0.316899in}}%
\pgfpathclose%
\pgfusepath{stroke,fill}%
\end{pgfscope}%
\begin{pgfscope}%
\pgfpathrectangle{\pgfqpoint{0.393053in}{0.375000in}}{\pgfqpoint{6.356833in}{5.175000in}}%
\pgfusepath{clip}%
\pgfsetbuttcap%
\pgfsetmiterjoin%
\definecolor{currentfill}{rgb}{0.827451,0.827451,0.827451}%
\pgfsetfillcolor{currentfill}%
\pgfsetfillopacity{0.300000}%
\pgfsetlinewidth{0.000000pt}%
\definecolor{currentstroke}{rgb}{0.000000,0.000000,0.000000}%
\pgfsetstrokecolor{currentstroke}%
\pgfsetstrokeopacity{0.300000}%
\pgfsetdash{}{0pt}%
\pgfpathmoveto{\pgfqpoint{0.393053in}{4.625893in}}%
\pgfpathlineto{\pgfqpoint{0.393864in}{4.533020in}}%
\pgfpathlineto{\pgfqpoint{0.396296in}{4.441090in}}%
\pgfpathlineto{\pgfqpoint{0.400350in}{4.350103in}}%
\pgfpathlineto{\pgfqpoint{0.406025in}{4.260059in}}%
\pgfpathlineto{\pgfqpoint{0.413322in}{4.170958in}}%
\pgfpathlineto{\pgfqpoint{0.422240in}{4.082799in}}%
\pgfpathlineto{\pgfqpoint{0.432779in}{3.995584in}}%
\pgfpathlineto{\pgfqpoint{0.444940in}{3.909311in}}%
\pgfpathlineto{\pgfqpoint{0.458723in}{3.823982in}}%
\pgfpathlineto{\pgfqpoint{0.474127in}{3.739595in}}%
\pgfpathlineto{\pgfqpoint{0.491152in}{3.656151in}}%
\pgfpathlineto{\pgfqpoint{0.509799in}{3.573649in}}%
\pgfpathlineto{\pgfqpoint{0.530068in}{3.492091in}}%
\pgfpathlineto{\pgfqpoint{0.551958in}{3.411476in}}%
\pgfpathlineto{\pgfqpoint{0.575469in}{3.331803in}}%
\pgfpathlineto{\pgfqpoint{0.600602in}{3.253074in}}%
\pgfpathlineto{\pgfqpoint{0.627356in}{3.175287in}}%
\pgfpathlineto{\pgfqpoint{0.655732in}{3.098443in}}%
\pgfpathlineto{\pgfqpoint{0.685729in}{3.022542in}}%
\pgfpathlineto{\pgfqpoint{0.717348in}{2.947584in}}%
\pgfpathlineto{\pgfqpoint{0.750588in}{2.873568in}}%
\pgfpathlineto{\pgfqpoint{0.785450in}{2.800496in}}%
\pgfpathlineto{\pgfqpoint{0.821933in}{2.728366in}}%
\pgfpathlineto{\pgfqpoint{0.860038in}{2.657180in}}%
\pgfpathlineto{\pgfqpoint{0.899764in}{2.586936in}}%
\pgfpathlineto{\pgfqpoint{0.941112in}{2.517635in}}%
\pgfpathlineto{\pgfqpoint{0.984081in}{2.449277in}}%
\pgfpathlineto{\pgfqpoint{1.028671in}{2.381861in}}%
\pgfpathlineto{\pgfqpoint{1.074884in}{2.315389in}}%
\pgfpathlineto{\pgfqpoint{1.122717in}{2.249860in}}%
\pgfpathlineto{\pgfqpoint{1.172172in}{2.185273in}}%
\pgfpathlineto{\pgfqpoint{1.223249in}{2.121629in}}%
\pgfpathlineto{\pgfqpoint{1.275947in}{2.058928in}}%
\pgfpathlineto{\pgfqpoint{1.330266in}{1.997170in}}%
\pgfpathlineto{\pgfqpoint{1.386207in}{1.936355in}}%
\pgfpathlineto{\pgfqpoint{1.443769in}{1.876483in}}%
\pgfpathlineto{\pgfqpoint{1.502953in}{1.817553in}}%
\pgfpathlineto{\pgfqpoint{1.563758in}{1.759567in}}%
\pgfpathlineto{\pgfqpoint{1.626185in}{1.702523in}}%
\pgfpathlineto{\pgfqpoint{1.690234in}{1.646423in}}%
\pgfpathlineto{\pgfqpoint{1.755903in}{1.591265in}}%
\pgfpathlineto{\pgfqpoint{1.823195in}{1.537050in}}%
\pgfpathlineto{\pgfqpoint{1.892107in}{1.483777in}}%
\pgfpathlineto{\pgfqpoint{1.962641in}{1.431448in}}%
\pgfpathlineto{\pgfqpoint{2.034797in}{1.380062in}}%
\pgfpathlineto{\pgfqpoint{2.108574in}{1.329618in}}%
\pgfpathlineto{\pgfqpoint{2.183973in}{1.280117in}}%
\pgfpathlineto{\pgfqpoint{2.260993in}{1.231560in}}%
\pgfpathlineto{\pgfqpoint{2.339634in}{1.183945in}}%
\pgfpathlineto{\pgfqpoint{2.419898in}{1.137273in}}%
\pgfpathlineto{\pgfqpoint{2.501782in}{1.091543in}}%
\pgfpathlineto{\pgfqpoint{2.585288in}{1.046757in}}%
\pgfpathlineto{\pgfqpoint{2.670416in}{1.002914in}}%
\pgfpathlineto{\pgfqpoint{2.757164in}{0.960013in}}%
\pgfpathlineto{\pgfqpoint{2.845535in}{0.918055in}}%
\pgfpathlineto{\pgfqpoint{2.935527in}{0.877040in}}%
\pgfpathlineto{\pgfqpoint{3.027140in}{0.836968in}}%
\pgfpathlineto{\pgfqpoint{3.120375in}{0.797839in}}%
\pgfpathlineto{\pgfqpoint{3.215231in}{0.759653in}}%
\pgfpathlineto{\pgfqpoint{3.282668in}{0.733526in}}%
\pgfpathlineto{\pgfqpoint{3.331718in}{0.715376in}}%
\pgfpathlineto{\pgfqpoint{3.381578in}{0.697697in}}%
\pgfpathlineto{\pgfqpoint{3.432250in}{0.680490in}}%
\pgfpathlineto{\pgfqpoint{3.483731in}{0.663754in}}%
\pgfpathlineto{\pgfqpoint{3.536024in}{0.647489in}}%
\pgfpathlineto{\pgfqpoint{3.589127in}{0.631696in}}%
\pgfpathlineto{\pgfqpoint{3.643041in}{0.616374in}}%
\pgfpathlineto{\pgfqpoint{3.697766in}{0.601524in}}%
\pgfpathlineto{\pgfqpoint{3.753302in}{0.587145in}}%
\pgfpathlineto{\pgfqpoint{3.809648in}{0.573238in}}%
\pgfpathlineto{\pgfqpoint{3.866805in}{0.559802in}}%
\pgfpathlineto{\pgfqpoint{3.924773in}{0.546838in}}%
\pgfpathlineto{\pgfqpoint{3.983551in}{0.534345in}}%
\pgfpathlineto{\pgfqpoint{4.043141in}{0.522323in}}%
\pgfpathlineto{\pgfqpoint{4.103540in}{0.510773in}}%
\pgfpathlineto{\pgfqpoint{4.164751in}{0.499694in}}%
\pgfpathlineto{\pgfqpoint{4.226773in}{0.489087in}}%
\pgfpathlineto{\pgfqpoint{4.289605in}{0.478951in}}%
\pgfpathlineto{\pgfqpoint{4.353248in}{0.469287in}}%
\pgfpathlineto{\pgfqpoint{4.417701in}{0.460094in}}%
\pgfpathlineto{\pgfqpoint{4.482966in}{0.451372in}}%
\pgfpathlineto{\pgfqpoint{4.549041in}{0.443122in}}%
\pgfpathlineto{\pgfqpoint{4.615927in}{0.435343in}}%
\pgfpathlineto{\pgfqpoint{4.683623in}{0.428036in}}%
\pgfpathlineto{\pgfqpoint{4.752131in}{0.421200in}}%
\pgfpathlineto{\pgfqpoint{4.821449in}{0.414836in}}%
\pgfpathlineto{\pgfqpoint{4.891578in}{0.408943in}}%
\pgfpathlineto{\pgfqpoint{4.962517in}{0.403521in}}%
\pgfpathlineto{\pgfqpoint{5.034267in}{0.398571in}}%
\pgfpathlineto{\pgfqpoint{5.106828in}{0.394093in}}%
\pgfpathlineto{\pgfqpoint{5.180200in}{0.390086in}}%
\pgfpathlineto{\pgfqpoint{5.254383in}{0.386550in}}%
\pgfpathlineto{\pgfqpoint{5.329376in}{0.383485in}}%
\pgfpathlineto{\pgfqpoint{5.405180in}{0.380893in}}%
\pgfpathlineto{\pgfqpoint{5.481795in}{0.378771in}}%
\pgfpathlineto{\pgfqpoint{5.559220in}{0.377121in}}%
\pgfpathlineto{\pgfqpoint{5.637456in}{0.375942in}}%
\pgfpathlineto{\pgfqpoint{5.716503in}{0.375235in}}%
\pgfpathlineto{\pgfqpoint{5.796361in}{0.375000in}}%
\pgfpathlineto{\pgfqpoint{6.877023in}{0.448928in}}%
\pgfpathlineto{\pgfqpoint{6.781193in}{0.449211in}}%
\pgfpathlineto{\pgfqpoint{6.686337in}{0.450060in}}%
\pgfpathlineto{\pgfqpoint{6.592454in}{0.451474in}}%
\pgfpathlineto{\pgfqpoint{6.499543in}{0.453454in}}%
\pgfpathlineto{\pgfqpoint{6.407605in}{0.456000in}}%
\pgfpathlineto{\pgfqpoint{6.316641in}{0.459111in}}%
\pgfpathlineto{\pgfqpoint{6.226649in}{0.462788in}}%
\pgfpathlineto{\pgfqpoint{6.137630in}{0.467031in}}%
\pgfpathlineto{\pgfqpoint{6.049584in}{0.471840in}}%
\pgfpathlineto{\pgfqpoint{5.962510in}{0.477214in}}%
\pgfpathlineto{\pgfqpoint{5.876410in}{0.483154in}}%
\pgfpathlineto{\pgfqpoint{5.791283in}{0.489660in}}%
\pgfpathlineto{\pgfqpoint{5.707128in}{0.496732in}}%
\pgfpathlineto{\pgfqpoint{5.623946in}{0.504369in}}%
\pgfpathlineto{\pgfqpoint{5.541737in}{0.512572in}}%
\pgfpathlineto{\pgfqpoint{5.460501in}{0.521341in}}%
\pgfpathlineto{\pgfqpoint{5.380238in}{0.530675in}}%
\pgfpathlineto{\pgfqpoint{5.300948in}{0.540575in}}%
\pgfpathlineto{\pgfqpoint{5.222631in}{0.551041in}}%
\pgfpathlineto{\pgfqpoint{5.145287in}{0.562073in}}%
\pgfpathlineto{\pgfqpoint{5.068915in}{0.573670in}}%
\pgfpathlineto{\pgfqpoint{4.993517in}{0.585833in}}%
\pgfpathlineto{\pgfqpoint{4.919091in}{0.598562in}}%
\pgfpathlineto{\pgfqpoint{4.845638in}{0.611856in}}%
\pgfpathlineto{\pgfqpoint{4.773158in}{0.625716in}}%
\pgfpathlineto{\pgfqpoint{4.701651in}{0.640142in}}%
\pgfpathlineto{\pgfqpoint{4.631117in}{0.655134in}}%
\pgfpathlineto{\pgfqpoint{4.561555in}{0.670691in}}%
\pgfpathlineto{\pgfqpoint{4.492967in}{0.686814in}}%
\pgfpathlineto{\pgfqpoint{4.425351in}{0.703503in}}%
\pgfpathlineto{\pgfqpoint{4.358709in}{0.720758in}}%
\pgfpathlineto{\pgfqpoint{4.293039in}{0.738578in}}%
\pgfpathlineto{\pgfqpoint{4.228342in}{0.756964in}}%
\pgfpathlineto{\pgfqpoint{4.164618in}{0.775916in}}%
\pgfpathlineto{\pgfqpoint{4.101867in}{0.795433in}}%
\pgfpathlineto{\pgfqpoint{4.040089in}{0.815516in}}%
\pgfpathlineto{\pgfqpoint{3.979284in}{0.836165in}}%
\pgfpathlineto{\pgfqpoint{3.919451in}{0.857380in}}%
\pgfpathlineto{\pgfqpoint{3.860592in}{0.879160in}}%
\pgfpathlineto{\pgfqpoint{3.779667in}{0.910512in}}%
\pgfpathlineto{\pgfqpoint{3.665839in}{0.956336in}}%
\pgfpathlineto{\pgfqpoint{3.553958in}{1.003291in}}%
\pgfpathlineto{\pgfqpoint{3.444021in}{1.051377in}}%
\pgfpathlineto{\pgfqpoint{3.336031in}{1.100595in}}%
\pgfpathlineto{\pgfqpoint{3.229987in}{1.150944in}}%
\pgfpathlineto{\pgfqpoint{3.125888in}{1.202425in}}%
\pgfpathlineto{\pgfqpoint{3.023735in}{1.255037in}}%
\pgfpathlineto{\pgfqpoint{2.923528in}{1.308781in}}%
\pgfpathlineto{\pgfqpoint{2.825266in}{1.363656in}}%
\pgfpathlineto{\pgfqpoint{2.728951in}{1.419662in}}%
\pgfpathlineto{\pgfqpoint{2.634581in}{1.476800in}}%
\pgfpathlineto{\pgfqpoint{2.542157in}{1.535070in}}%
\pgfpathlineto{\pgfqpoint{2.451678in}{1.594470in}}%
\pgfpathlineto{\pgfqpoint{2.363146in}{1.655003in}}%
\pgfpathlineto{\pgfqpoint{2.276559in}{1.716666in}}%
\pgfpathlineto{\pgfqpoint{2.191918in}{1.779462in}}%
\pgfpathlineto{\pgfqpoint{2.109223in}{1.843388in}}%
\pgfpathlineto{\pgfqpoint{2.028473in}{1.908446in}}%
\pgfpathlineto{\pgfqpoint{1.949670in}{1.974636in}}%
\pgfpathlineto{\pgfqpoint{1.872812in}{2.041957in}}%
\pgfpathlineto{\pgfqpoint{1.797899in}{2.110409in}}%
\pgfpathlineto{\pgfqpoint{1.724933in}{2.179993in}}%
\pgfpathlineto{\pgfqpoint{1.653912in}{2.250708in}}%
\pgfpathlineto{\pgfqpoint{1.584838in}{2.322555in}}%
\pgfpathlineto{\pgfqpoint{1.517709in}{2.395533in}}%
\pgfpathlineto{\pgfqpoint{1.452525in}{2.469643in}}%
\pgfpathlineto{\pgfqpoint{1.389288in}{2.544884in}}%
\pgfpathlineto{\pgfqpoint{1.327996in}{2.621256in}}%
\pgfpathlineto{\pgfqpoint{1.268650in}{2.698760in}}%
\pgfpathlineto{\pgfqpoint{1.211250in}{2.777395in}}%
\pgfpathlineto{\pgfqpoint{1.155795in}{2.857162in}}%
\pgfpathlineto{\pgfqpoint{1.102286in}{2.938061in}}%
\pgfpathlineto{\pgfqpoint{1.050724in}{3.020090in}}%
\pgfpathlineto{\pgfqpoint{1.001106in}{3.103251in}}%
\pgfpathlineto{\pgfqpoint{0.953435in}{3.187544in}}%
\pgfpathlineto{\pgfqpoint{0.907709in}{3.272968in}}%
\pgfpathlineto{\pgfqpoint{0.863930in}{3.359524in}}%
\pgfpathlineto{\pgfqpoint{0.822095in}{3.447211in}}%
\pgfpathlineto{\pgfqpoint{0.782207in}{3.536029in}}%
\pgfpathlineto{\pgfqpoint{0.744265in}{3.625979in}}%
\pgfpathlineto{\pgfqpoint{0.708268in}{3.717060in}}%
\pgfpathlineto{\pgfqpoint{0.674217in}{3.809273in}}%
\pgfpathlineto{\pgfqpoint{0.642112in}{3.902617in}}%
\pgfpathlineto{\pgfqpoint{0.611952in}{3.997093in}}%
\pgfpathlineto{\pgfqpoint{0.583739in}{4.092700in}}%
\pgfpathlineto{\pgfqpoint{0.557471in}{4.189438in}}%
\pgfpathlineto{\pgfqpoint{0.533149in}{4.287308in}}%
\pgfpathlineto{\pgfqpoint{0.510772in}{4.386309in}}%
\pgfpathlineto{\pgfqpoint{0.490342in}{4.486442in}}%
\pgfpathlineto{\pgfqpoint{0.471857in}{4.587707in}}%
\pgfpathlineto{\pgfqpoint{0.455318in}{4.690102in}}%
\pgfpathlineto{\pgfqpoint{0.440724in}{4.793629in}}%
\pgfpathlineto{\pgfqpoint{0.428077in}{4.898288in}}%
\pgfpathlineto{\pgfqpoint{0.417375in}{5.004078in}}%
\pgfpathlineto{\pgfqpoint{0.408619in}{5.111000in}}%
\pgfpathlineto{\pgfqpoint{0.401809in}{5.219053in}}%
\pgfpathlineto{\pgfqpoint{0.396945in}{5.328237in}}%
\pgfpathlineto{\pgfqpoint{0.394026in}{5.438553in}}%
\pgfpathlineto{\pgfqpoint{0.393053in}{5.550000in}}%
\pgfpathlineto{\pgfqpoint{0.393053in}{4.625893in}}%
\pgfpathclose%
\pgfusepath{fill}%
\end{pgfscope}%
\begin{pgfscope}%
\pgfsetbuttcap%
\pgfsetroundjoin%
\definecolor{currentfill}{rgb}{0.000000,0.000000,0.000000}%
\pgfsetfillcolor{currentfill}%
\pgfsetlinewidth{0.803000pt}%
\definecolor{currentstroke}{rgb}{0.000000,0.000000,0.000000}%
\pgfsetstrokecolor{currentstroke}%
\pgfsetdash{}{0pt}%
\pgfsys@defobject{currentmarker}{\pgfqpoint{0.000000in}{-0.048611in}}{\pgfqpoint{0.000000in}{0.000000in}}{%
\pgfpathmoveto{\pgfqpoint{0.000000in}{0.000000in}}%
\pgfpathlineto{\pgfqpoint{0.000000in}{-0.048611in}}%
\pgfusepath{stroke,fill}%
}%
\begin{pgfscope}%
\pgfsys@transformshift{0.393053in}{0.375000in}%
\pgfsys@useobject{currentmarker}{}%
\end{pgfscope}%
\end{pgfscope}%
\begin{pgfscope}%
\definecolor{textcolor}{rgb}{0.000000,0.000000,0.000000}%
\pgfsetstrokecolor{textcolor}%
\pgfsetfillcolor{textcolor}%
\pgftext[x=0.393053in,y=0.277777in,,top]{\color{textcolor}{\rmfamily\fontsize{14.000000}{16.800000}\selectfont\catcode`\^=\active\def^{\ifmmode\sp\else\^{}\fi}\catcode`\%=\active\def%{\%}$\mathdefault{0}$}}%
\end{pgfscope}%
\begin{pgfscope}%
\pgfsetbuttcap%
\pgfsetroundjoin%
\definecolor{currentfill}{rgb}{0.000000,0.000000,0.000000}%
\pgfsetfillcolor{currentfill}%
\pgfsetlinewidth{0.803000pt}%
\definecolor{currentstroke}{rgb}{0.000000,0.000000,0.000000}%
\pgfsetstrokecolor{currentstroke}%
\pgfsetdash{}{0pt}%
\pgfsys@defobject{currentmarker}{\pgfqpoint{0.000000in}{-0.048611in}}{\pgfqpoint{0.000000in}{0.000000in}}{%
\pgfpathmoveto{\pgfqpoint{0.000000in}{0.000000in}}%
\pgfpathlineto{\pgfqpoint{0.000000in}{-0.048611in}}%
\pgfusepath{stroke,fill}%
}%
\begin{pgfscope}%
\pgfsys@transformshift{1.187657in}{0.375000in}%
\pgfsys@useobject{currentmarker}{}%
\end{pgfscope}%
\end{pgfscope}%
\begin{pgfscope}%
\definecolor{textcolor}{rgb}{0.000000,0.000000,0.000000}%
\pgfsetstrokecolor{textcolor}%
\pgfsetfillcolor{textcolor}%
\pgftext[x=1.187657in,y=0.277777in,,top]{\color{textcolor}{\rmfamily\fontsize{14.000000}{16.800000}\selectfont\catcode`\^=\active\def^{\ifmmode\sp\else\^{}\fi}\catcode`\%=\active\def%{\%}$\mathdefault{20}$}}%
\end{pgfscope}%
\begin{pgfscope}%
\pgfsetbuttcap%
\pgfsetroundjoin%
\definecolor{currentfill}{rgb}{0.000000,0.000000,0.000000}%
\pgfsetfillcolor{currentfill}%
\pgfsetlinewidth{0.803000pt}%
\definecolor{currentstroke}{rgb}{0.000000,0.000000,0.000000}%
\pgfsetstrokecolor{currentstroke}%
\pgfsetdash{}{0pt}%
\pgfsys@defobject{currentmarker}{\pgfqpoint{0.000000in}{-0.048611in}}{\pgfqpoint{0.000000in}{0.000000in}}{%
\pgfpathmoveto{\pgfqpoint{0.000000in}{0.000000in}}%
\pgfpathlineto{\pgfqpoint{0.000000in}{-0.048611in}}%
\pgfusepath{stroke,fill}%
}%
\begin{pgfscope}%
\pgfsys@transformshift{1.982261in}{0.375000in}%
\pgfsys@useobject{currentmarker}{}%
\end{pgfscope}%
\end{pgfscope}%
\begin{pgfscope}%
\definecolor{textcolor}{rgb}{0.000000,0.000000,0.000000}%
\pgfsetstrokecolor{textcolor}%
\pgfsetfillcolor{textcolor}%
\pgftext[x=1.982261in,y=0.277777in,,top]{\color{textcolor}{\rmfamily\fontsize{14.000000}{16.800000}\selectfont\catcode`\^=\active\def^{\ifmmode\sp\else\^{}\fi}\catcode`\%=\active\def%{\%}$\mathdefault{40}$}}%
\end{pgfscope}%
\begin{pgfscope}%
\pgfsetbuttcap%
\pgfsetroundjoin%
\definecolor{currentfill}{rgb}{0.000000,0.000000,0.000000}%
\pgfsetfillcolor{currentfill}%
\pgfsetlinewidth{0.803000pt}%
\definecolor{currentstroke}{rgb}{0.000000,0.000000,0.000000}%
\pgfsetstrokecolor{currentstroke}%
\pgfsetdash{}{0pt}%
\pgfsys@defobject{currentmarker}{\pgfqpoint{0.000000in}{-0.048611in}}{\pgfqpoint{0.000000in}{0.000000in}}{%
\pgfpathmoveto{\pgfqpoint{0.000000in}{0.000000in}}%
\pgfpathlineto{\pgfqpoint{0.000000in}{-0.048611in}}%
\pgfusepath{stroke,fill}%
}%
\begin{pgfscope}%
\pgfsys@transformshift{2.776865in}{0.375000in}%
\pgfsys@useobject{currentmarker}{}%
\end{pgfscope}%
\end{pgfscope}%
\begin{pgfscope}%
\definecolor{textcolor}{rgb}{0.000000,0.000000,0.000000}%
\pgfsetstrokecolor{textcolor}%
\pgfsetfillcolor{textcolor}%
\pgftext[x=2.776865in,y=0.277777in,,top]{\color{textcolor}{\rmfamily\fontsize{14.000000}{16.800000}\selectfont\catcode`\^=\active\def^{\ifmmode\sp\else\^{}\fi}\catcode`\%=\active\def%{\%}$\mathdefault{60}$}}%
\end{pgfscope}%
\begin{pgfscope}%
\pgfsetbuttcap%
\pgfsetroundjoin%
\definecolor{currentfill}{rgb}{0.000000,0.000000,0.000000}%
\pgfsetfillcolor{currentfill}%
\pgfsetlinewidth{0.803000pt}%
\definecolor{currentstroke}{rgb}{0.000000,0.000000,0.000000}%
\pgfsetstrokecolor{currentstroke}%
\pgfsetdash{}{0pt}%
\pgfsys@defobject{currentmarker}{\pgfqpoint{0.000000in}{-0.048611in}}{\pgfqpoint{0.000000in}{0.000000in}}{%
\pgfpathmoveto{\pgfqpoint{0.000000in}{0.000000in}}%
\pgfpathlineto{\pgfqpoint{0.000000in}{-0.048611in}}%
\pgfusepath{stroke,fill}%
}%
\begin{pgfscope}%
\pgfsys@transformshift{3.571469in}{0.375000in}%
\pgfsys@useobject{currentmarker}{}%
\end{pgfscope}%
\end{pgfscope}%
\begin{pgfscope}%
\definecolor{textcolor}{rgb}{0.000000,0.000000,0.000000}%
\pgfsetstrokecolor{textcolor}%
\pgfsetfillcolor{textcolor}%
\pgftext[x=3.571469in,y=0.277777in,,top]{\color{textcolor}{\rmfamily\fontsize{14.000000}{16.800000}\selectfont\catcode`\^=\active\def^{\ifmmode\sp\else\^{}\fi}\catcode`\%=\active\def%{\%}$\mathdefault{80}$}}%
\end{pgfscope}%
\begin{pgfscope}%
\pgfsetbuttcap%
\pgfsetroundjoin%
\definecolor{currentfill}{rgb}{0.000000,0.000000,0.000000}%
\pgfsetfillcolor{currentfill}%
\pgfsetlinewidth{0.803000pt}%
\definecolor{currentstroke}{rgb}{0.000000,0.000000,0.000000}%
\pgfsetstrokecolor{currentstroke}%
\pgfsetdash{}{0pt}%
\pgfsys@defobject{currentmarker}{\pgfqpoint{0.000000in}{-0.048611in}}{\pgfqpoint{0.000000in}{0.000000in}}{%
\pgfpathmoveto{\pgfqpoint{0.000000in}{0.000000in}}%
\pgfpathlineto{\pgfqpoint{0.000000in}{-0.048611in}}%
\pgfusepath{stroke,fill}%
}%
\begin{pgfscope}%
\pgfsys@transformshift{4.366074in}{0.375000in}%
\pgfsys@useobject{currentmarker}{}%
\end{pgfscope}%
\end{pgfscope}%
\begin{pgfscope}%
\definecolor{textcolor}{rgb}{0.000000,0.000000,0.000000}%
\pgfsetstrokecolor{textcolor}%
\pgfsetfillcolor{textcolor}%
\pgftext[x=4.366074in,y=0.277777in,,top]{\color{textcolor}{\rmfamily\fontsize{14.000000}{16.800000}\selectfont\catcode`\^=\active\def^{\ifmmode\sp\else\^{}\fi}\catcode`\%=\active\def%{\%}$\mathdefault{100}$}}%
\end{pgfscope}%
\begin{pgfscope}%
\pgfsetbuttcap%
\pgfsetroundjoin%
\definecolor{currentfill}{rgb}{0.000000,0.000000,0.000000}%
\pgfsetfillcolor{currentfill}%
\pgfsetlinewidth{0.803000pt}%
\definecolor{currentstroke}{rgb}{0.000000,0.000000,0.000000}%
\pgfsetstrokecolor{currentstroke}%
\pgfsetdash{}{0pt}%
\pgfsys@defobject{currentmarker}{\pgfqpoint{0.000000in}{-0.048611in}}{\pgfqpoint{0.000000in}{0.000000in}}{%
\pgfpathmoveto{\pgfqpoint{0.000000in}{0.000000in}}%
\pgfpathlineto{\pgfqpoint{0.000000in}{-0.048611in}}%
\pgfusepath{stroke,fill}%
}%
\begin{pgfscope}%
\pgfsys@transformshift{5.160678in}{0.375000in}%
\pgfsys@useobject{currentmarker}{}%
\end{pgfscope}%
\end{pgfscope}%
\begin{pgfscope}%
\definecolor{textcolor}{rgb}{0.000000,0.000000,0.000000}%
\pgfsetstrokecolor{textcolor}%
\pgfsetfillcolor{textcolor}%
\pgftext[x=5.160678in,y=0.277777in,,top]{\color{textcolor}{\rmfamily\fontsize{14.000000}{16.800000}\selectfont\catcode`\^=\active\def^{\ifmmode\sp\else\^{}\fi}\catcode`\%=\active\def%{\%}$\mathdefault{120}$}}%
\end{pgfscope}%
\begin{pgfscope}%
\pgfsetbuttcap%
\pgfsetroundjoin%
\definecolor{currentfill}{rgb}{0.000000,0.000000,0.000000}%
\pgfsetfillcolor{currentfill}%
\pgfsetlinewidth{0.803000pt}%
\definecolor{currentstroke}{rgb}{0.000000,0.000000,0.000000}%
\pgfsetstrokecolor{currentstroke}%
\pgfsetdash{}{0pt}%
\pgfsys@defobject{currentmarker}{\pgfqpoint{0.000000in}{-0.048611in}}{\pgfqpoint{0.000000in}{0.000000in}}{%
\pgfpathmoveto{\pgfqpoint{0.000000in}{0.000000in}}%
\pgfpathlineto{\pgfqpoint{0.000000in}{-0.048611in}}%
\pgfusepath{stroke,fill}%
}%
\begin{pgfscope}%
\pgfsys@transformshift{5.955282in}{0.375000in}%
\pgfsys@useobject{currentmarker}{}%
\end{pgfscope}%
\end{pgfscope}%
\begin{pgfscope}%
\definecolor{textcolor}{rgb}{0.000000,0.000000,0.000000}%
\pgfsetstrokecolor{textcolor}%
\pgfsetfillcolor{textcolor}%
\pgftext[x=5.955282in,y=0.277777in,,top]{\color{textcolor}{\rmfamily\fontsize{14.000000}{16.800000}\selectfont\catcode`\^=\active\def^{\ifmmode\sp\else\^{}\fi}\catcode`\%=\active\def%{\%}$\mathdefault{140}$}}%
\end{pgfscope}%
\begin{pgfscope}%
\pgfsetbuttcap%
\pgfsetroundjoin%
\definecolor{currentfill}{rgb}{0.000000,0.000000,0.000000}%
\pgfsetfillcolor{currentfill}%
\pgfsetlinewidth{0.803000pt}%
\definecolor{currentstroke}{rgb}{0.000000,0.000000,0.000000}%
\pgfsetstrokecolor{currentstroke}%
\pgfsetdash{}{0pt}%
\pgfsys@defobject{currentmarker}{\pgfqpoint{0.000000in}{-0.048611in}}{\pgfqpoint{0.000000in}{0.000000in}}{%
\pgfpathmoveto{\pgfqpoint{0.000000in}{0.000000in}}%
\pgfpathlineto{\pgfqpoint{0.000000in}{-0.048611in}}%
\pgfusepath{stroke,fill}%
}%
\begin{pgfscope}%
\pgfsys@transformshift{6.749886in}{0.375000in}%
\pgfsys@useobject{currentmarker}{}%
\end{pgfscope}%
\end{pgfscope}%
\begin{pgfscope}%
\definecolor{textcolor}{rgb}{0.000000,0.000000,0.000000}%
\pgfsetstrokecolor{textcolor}%
\pgfsetfillcolor{textcolor}%
\pgftext[x=6.749886in,y=0.277777in,,top]{\color{textcolor}{\rmfamily\fontsize{14.000000}{16.800000}\selectfont\catcode`\^=\active\def^{\ifmmode\sp\else\^{}\fi}\catcode`\%=\active\def%{\%}$\mathdefault{160}$}}%
\end{pgfscope}%
\begin{pgfscope}%
\pgfsetbuttcap%
\pgfsetroundjoin%
\definecolor{currentfill}{rgb}{0.000000,0.000000,0.000000}%
\pgfsetfillcolor{currentfill}%
\pgfsetlinewidth{0.803000pt}%
\definecolor{currentstroke}{rgb}{0.000000,0.000000,0.000000}%
\pgfsetstrokecolor{currentstroke}%
\pgfsetdash{}{0pt}%
\pgfsys@defobject{currentmarker}{\pgfqpoint{-0.048611in}{0.000000in}}{\pgfqpoint{-0.000000in}{0.000000in}}{%
\pgfpathmoveto{\pgfqpoint{-0.000000in}{0.000000in}}%
\pgfpathlineto{\pgfqpoint{-0.048611in}{0.000000in}}%
\pgfusepath{stroke,fill}%
}%
\begin{pgfscope}%
\pgfsys@transformshift{0.393053in}{0.929464in}%
\pgfsys@useobject{currentmarker}{}%
\end{pgfscope}%
\end{pgfscope}%
\begin{pgfscope}%
\definecolor{textcolor}{rgb}{0.000000,0.000000,0.000000}%
\pgfsetstrokecolor{textcolor}%
\pgfsetfillcolor{textcolor}%
\pgftext[x=0.100000in, y=0.860020in, left, base]{\color{textcolor}{\rmfamily\fontsize{14.000000}{16.800000}\selectfont\catcode`\^=\active\def^{\ifmmode\sp\else\^{}\fi}\catcode`\%=\active\def%{\%}$\mathdefault{10}$}}%
\end{pgfscope}%
\begin{pgfscope}%
\pgfsetbuttcap%
\pgfsetroundjoin%
\definecolor{currentfill}{rgb}{0.000000,0.000000,0.000000}%
\pgfsetfillcolor{currentfill}%
\pgfsetlinewidth{0.803000pt}%
\definecolor{currentstroke}{rgb}{0.000000,0.000000,0.000000}%
\pgfsetstrokecolor{currentstroke}%
\pgfsetdash{}{0pt}%
\pgfsys@defobject{currentmarker}{\pgfqpoint{-0.048611in}{0.000000in}}{\pgfqpoint{-0.000000in}{0.000000in}}{%
\pgfpathmoveto{\pgfqpoint{-0.000000in}{0.000000in}}%
\pgfpathlineto{\pgfqpoint{-0.048611in}{0.000000in}}%
\pgfusepath{stroke,fill}%
}%
\begin{pgfscope}%
\pgfsys@transformshift{0.393053in}{1.853571in}%
\pgfsys@useobject{currentmarker}{}%
\end{pgfscope}%
\end{pgfscope}%
\begin{pgfscope}%
\definecolor{textcolor}{rgb}{0.000000,0.000000,0.000000}%
\pgfsetstrokecolor{textcolor}%
\pgfsetfillcolor{textcolor}%
\pgftext[x=0.100000in, y=1.784127in, left, base]{\color{textcolor}{\rmfamily\fontsize{14.000000}{16.800000}\selectfont\catcode`\^=\active\def^{\ifmmode\sp\else\^{}\fi}\catcode`\%=\active\def%{\%}$\mathdefault{20}$}}%
\end{pgfscope}%
\begin{pgfscope}%
\pgfsetbuttcap%
\pgfsetroundjoin%
\definecolor{currentfill}{rgb}{0.000000,0.000000,0.000000}%
\pgfsetfillcolor{currentfill}%
\pgfsetlinewidth{0.803000pt}%
\definecolor{currentstroke}{rgb}{0.000000,0.000000,0.000000}%
\pgfsetstrokecolor{currentstroke}%
\pgfsetdash{}{0pt}%
\pgfsys@defobject{currentmarker}{\pgfqpoint{-0.048611in}{0.000000in}}{\pgfqpoint{-0.000000in}{0.000000in}}{%
\pgfpathmoveto{\pgfqpoint{-0.000000in}{0.000000in}}%
\pgfpathlineto{\pgfqpoint{-0.048611in}{0.000000in}}%
\pgfusepath{stroke,fill}%
}%
\begin{pgfscope}%
\pgfsys@transformshift{0.393053in}{2.777678in}%
\pgfsys@useobject{currentmarker}{}%
\end{pgfscope}%
\end{pgfscope}%
\begin{pgfscope}%
\definecolor{textcolor}{rgb}{0.000000,0.000000,0.000000}%
\pgfsetstrokecolor{textcolor}%
\pgfsetfillcolor{textcolor}%
\pgftext[x=0.100000in, y=2.708234in, left, base]{\color{textcolor}{\rmfamily\fontsize{14.000000}{16.800000}\selectfont\catcode`\^=\active\def^{\ifmmode\sp\else\^{}\fi}\catcode`\%=\active\def%{\%}$\mathdefault{30}$}}%
\end{pgfscope}%
\begin{pgfscope}%
\pgfsetbuttcap%
\pgfsetroundjoin%
\definecolor{currentfill}{rgb}{0.000000,0.000000,0.000000}%
\pgfsetfillcolor{currentfill}%
\pgfsetlinewidth{0.803000pt}%
\definecolor{currentstroke}{rgb}{0.000000,0.000000,0.000000}%
\pgfsetstrokecolor{currentstroke}%
\pgfsetdash{}{0pt}%
\pgfsys@defobject{currentmarker}{\pgfqpoint{-0.048611in}{0.000000in}}{\pgfqpoint{-0.000000in}{0.000000in}}{%
\pgfpathmoveto{\pgfqpoint{-0.000000in}{0.000000in}}%
\pgfpathlineto{\pgfqpoint{-0.048611in}{0.000000in}}%
\pgfusepath{stroke,fill}%
}%
\begin{pgfscope}%
\pgfsys@transformshift{0.393053in}{3.701786in}%
\pgfsys@useobject{currentmarker}{}%
\end{pgfscope}%
\end{pgfscope}%
\begin{pgfscope}%
\definecolor{textcolor}{rgb}{0.000000,0.000000,0.000000}%
\pgfsetstrokecolor{textcolor}%
\pgfsetfillcolor{textcolor}%
\pgftext[x=0.100000in, y=3.632341in, left, base]{\color{textcolor}{\rmfamily\fontsize{14.000000}{16.800000}\selectfont\catcode`\^=\active\def^{\ifmmode\sp\else\^{}\fi}\catcode`\%=\active\def%{\%}$\mathdefault{40}$}}%
\end{pgfscope}%
\begin{pgfscope}%
\pgfsetbuttcap%
\pgfsetroundjoin%
\definecolor{currentfill}{rgb}{0.000000,0.000000,0.000000}%
\pgfsetfillcolor{currentfill}%
\pgfsetlinewidth{0.803000pt}%
\definecolor{currentstroke}{rgb}{0.000000,0.000000,0.000000}%
\pgfsetstrokecolor{currentstroke}%
\pgfsetdash{}{0pt}%
\pgfsys@defobject{currentmarker}{\pgfqpoint{-0.048611in}{0.000000in}}{\pgfqpoint{-0.000000in}{0.000000in}}{%
\pgfpathmoveto{\pgfqpoint{-0.000000in}{0.000000in}}%
\pgfpathlineto{\pgfqpoint{-0.048611in}{0.000000in}}%
\pgfusepath{stroke,fill}%
}%
\begin{pgfscope}%
\pgfsys@transformshift{0.393053in}{4.625893in}%
\pgfsys@useobject{currentmarker}{}%
\end{pgfscope}%
\end{pgfscope}%
\begin{pgfscope}%
\definecolor{textcolor}{rgb}{0.000000,0.000000,0.000000}%
\pgfsetstrokecolor{textcolor}%
\pgfsetfillcolor{textcolor}%
\pgftext[x=0.100000in, y=4.556448in, left, base]{\color{textcolor}{\rmfamily\fontsize{14.000000}{16.800000}\selectfont\catcode`\^=\active\def^{\ifmmode\sp\else\^{}\fi}\catcode`\%=\active\def%{\%}$\mathdefault{50}$}}%
\end{pgfscope}%
\begin{pgfscope}%
\pgfsetbuttcap%
\pgfsetroundjoin%
\definecolor{currentfill}{rgb}{0.000000,0.000000,0.000000}%
\pgfsetfillcolor{currentfill}%
\pgfsetlinewidth{0.803000pt}%
\definecolor{currentstroke}{rgb}{0.000000,0.000000,0.000000}%
\pgfsetstrokecolor{currentstroke}%
\pgfsetdash{}{0pt}%
\pgfsys@defobject{currentmarker}{\pgfqpoint{-0.048611in}{0.000000in}}{\pgfqpoint{-0.000000in}{0.000000in}}{%
\pgfpathmoveto{\pgfqpoint{-0.000000in}{0.000000in}}%
\pgfpathlineto{\pgfqpoint{-0.048611in}{0.000000in}}%
\pgfusepath{stroke,fill}%
}%
\begin{pgfscope}%
\pgfsys@transformshift{0.393053in}{5.550000in}%
\pgfsys@useobject{currentmarker}{}%
\end{pgfscope}%
\end{pgfscope}%
\begin{pgfscope}%
\definecolor{textcolor}{rgb}{0.000000,0.000000,0.000000}%
\pgfsetstrokecolor{textcolor}%
\pgfsetfillcolor{textcolor}%
\pgftext[x=0.100000in, y=5.480556in, left, base]{\color{textcolor}{\rmfamily\fontsize{14.000000}{16.800000}\selectfont\catcode`\^=\active\def^{\ifmmode\sp\else\^{}\fi}\catcode`\%=\active\def%{\%}$\mathdefault{60}$}}%
\end{pgfscope}%
\begin{pgfscope}%
\pgfsetrectcap%
\pgfsetmiterjoin%
\pgfsetlinewidth{0.803000pt}%
\definecolor{currentstroke}{rgb}{0.000000,0.000000,0.000000}%
\pgfsetstrokecolor{currentstroke}%
\pgfsetdash{}{0pt}%
\pgfpathmoveto{\pgfqpoint{0.393053in}{0.375000in}}%
\pgfpathlineto{\pgfqpoint{0.393053in}{5.550000in}}%
\pgfusepath{stroke}%
\end{pgfscope}%
\begin{pgfscope}%
\pgfsetrectcap%
\pgfsetmiterjoin%
\pgfsetlinewidth{0.803000pt}%
\definecolor{currentstroke}{rgb}{0.000000,0.000000,0.000000}%
\pgfsetstrokecolor{currentstroke}%
\pgfsetdash{}{0pt}%
\pgfpathmoveto{\pgfqpoint{6.749886in}{0.375000in}}%
\pgfpathlineto{\pgfqpoint{6.749886in}{5.550000in}}%
\pgfusepath{stroke}%
\end{pgfscope}%
\begin{pgfscope}%
\pgfsetrectcap%
\pgfsetmiterjoin%
\pgfsetlinewidth{0.803000pt}%
\definecolor{currentstroke}{rgb}{0.000000,0.000000,0.000000}%
\pgfsetstrokecolor{currentstroke}%
\pgfsetdash{}{0pt}%
\pgfpathmoveto{\pgfqpoint{0.393053in}{0.375000in}}%
\pgfpathlineto{\pgfqpoint{6.749886in}{0.375000in}}%
\pgfusepath{stroke}%
\end{pgfscope}%
\begin{pgfscope}%
\pgfsetrectcap%
\pgfsetmiterjoin%
\pgfsetlinewidth{0.803000pt}%
\definecolor{currentstroke}{rgb}{0.000000,0.000000,0.000000}%
\pgfsetstrokecolor{currentstroke}%
\pgfsetdash{}{0pt}%
\pgfpathmoveto{\pgfqpoint{0.393053in}{5.550000in}}%
\pgfpathlineto{\pgfqpoint{6.749886in}{5.550000in}}%
\pgfusepath{stroke}%
\end{pgfscope}%
\begin{pgfscope}%
\definecolor{textcolor}{rgb}{0.000000,0.000000,0.000000}%
\pgfsetstrokecolor{textcolor}%
\pgfsetfillcolor{textcolor}%
\pgftext[x=3.571469in,y=5.633333in,,base]{\color{textcolor}{\rmfamily\fontsize{16.000000}{19.200000}\selectfont\catcode`\^=\active\def^{\ifmmode\sp\else\^{}\fi}\catcode`\%=\active\def%{\%}Objective Space}}%
\end{pgfscope}%
\begin{pgfscope}%
\pgfsetbuttcap%
\pgfsetmiterjoin%
\definecolor{currentfill}{rgb}{1.000000,1.000000,1.000000}%
\pgfsetfillcolor{currentfill}%
\pgfsetfillopacity{0.800000}%
\pgfsetlinewidth{1.003750pt}%
\definecolor{currentstroke}{rgb}{0.800000,0.800000,0.800000}%
\pgfsetstrokecolor{currentstroke}%
\pgfsetstrokeopacity{0.800000}%
\pgfsetdash{}{0pt}%
\pgfpathmoveto{\pgfqpoint{4.342291in}{4.294446in}}%
\pgfpathlineto{\pgfqpoint{6.613775in}{4.294446in}}%
\pgfpathquadraticcurveto{\pgfqpoint{6.652664in}{4.294446in}}{\pgfqpoint{6.652664in}{4.333335in}}%
\pgfpathlineto{\pgfqpoint{6.652664in}{5.413889in}}%
\pgfpathquadraticcurveto{\pgfqpoint{6.652664in}{5.452778in}}{\pgfqpoint{6.613775in}{5.452778in}}%
\pgfpathlineto{\pgfqpoint{4.342291in}{5.452778in}}%
\pgfpathquadraticcurveto{\pgfqpoint{4.303402in}{5.452778in}}{\pgfqpoint{4.303402in}{5.413889in}}%
\pgfpathlineto{\pgfqpoint{4.303402in}{4.333335in}}%
\pgfpathquadraticcurveto{\pgfqpoint{4.303402in}{4.294446in}}{\pgfqpoint{4.342291in}{4.294446in}}%
\pgfpathlineto{\pgfqpoint{4.342291in}{4.294446in}}%
\pgfpathclose%
\pgfusepath{stroke,fill}%
\end{pgfscope}%
\begin{pgfscope}%
\pgfsetbuttcap%
\pgfsetroundjoin%
\pgfsetlinewidth{1.003750pt}%
\definecolor{currentstroke}{rgb}{0.827451,0.827451,0.827451}%
\pgfsetstrokecolor{currentstroke}%
\pgfsetdash{}{0pt}%
\pgfpathmoveto{\pgfqpoint{4.575624in}{5.245486in}}%
\pgfpathcurveto{\pgfqpoint{4.586674in}{5.245486in}}{\pgfqpoint{4.597273in}{5.249877in}}{\pgfqpoint{4.605087in}{5.257690in}}%
\pgfpathcurveto{\pgfqpoint{4.612900in}{5.265504in}}{\pgfqpoint{4.617291in}{5.276103in}}{\pgfqpoint{4.617291in}{5.287153in}}%
\pgfpathcurveto{\pgfqpoint{4.617291in}{5.298203in}}{\pgfqpoint{4.612900in}{5.308802in}}{\pgfqpoint{4.605087in}{5.316616in}}%
\pgfpathcurveto{\pgfqpoint{4.597273in}{5.324429in}}{\pgfqpoint{4.586674in}{5.328820in}}{\pgfqpoint{4.575624in}{5.328820in}}%
\pgfpathcurveto{\pgfqpoint{4.564574in}{5.328820in}}{\pgfqpoint{4.553975in}{5.324429in}}{\pgfqpoint{4.546161in}{5.316616in}}%
\pgfpathcurveto{\pgfqpoint{4.538348in}{5.308802in}}{\pgfqpoint{4.533957in}{5.298203in}}{\pgfqpoint{4.533957in}{5.287153in}}%
\pgfpathcurveto{\pgfqpoint{4.533957in}{5.276103in}}{\pgfqpoint{4.538348in}{5.265504in}}{\pgfqpoint{4.546161in}{5.257690in}}%
\pgfpathcurveto{\pgfqpoint{4.553975in}{5.249877in}}{\pgfqpoint{4.564574in}{5.245486in}}{\pgfqpoint{4.575624in}{5.245486in}}%
\pgfpathlineto{\pgfqpoint{4.575624in}{5.245486in}}%
\pgfpathclose%
\pgfusepath{stroke}%
\end{pgfscope}%
\begin{pgfscope}%
\definecolor{textcolor}{rgb}{0.000000,0.000000,0.000000}%
\pgfsetstrokecolor{textcolor}%
\pgfsetfillcolor{textcolor}%
\pgftext[x=4.925624in,y=5.236111in,left,base]{\color{textcolor}{\rmfamily\fontsize{14.000000}{16.800000}\selectfont\catcode`\^=\active\def^{\ifmmode\sp\else\^{}\fi}\catcode`\%=\active\def%{\%}tested points}}%
\end{pgfscope}%
\begin{pgfscope}%
\pgfsetbuttcap%
\pgfsetroundjoin%
\pgfsetlinewidth{1.003750pt}%
\definecolor{currentstroke}{rgb}{1.000000,0.000000,0.000000}%
\pgfsetstrokecolor{currentstroke}%
\pgfsetdash{}{0pt}%
\pgfpathmoveto{\pgfqpoint{4.575624in}{4.970487in}}%
\pgfpathcurveto{\pgfqpoint{4.586674in}{4.970487in}}{\pgfqpoint{4.597273in}{4.974877in}}{\pgfqpoint{4.605087in}{4.982691in}}%
\pgfpathcurveto{\pgfqpoint{4.612900in}{4.990504in}}{\pgfqpoint{4.617291in}{5.001103in}}{\pgfqpoint{4.617291in}{5.012153in}}%
\pgfpathcurveto{\pgfqpoint{4.617291in}{5.023204in}}{\pgfqpoint{4.612900in}{5.033803in}}{\pgfqpoint{4.605087in}{5.041616in}}%
\pgfpathcurveto{\pgfqpoint{4.597273in}{5.049430in}}{\pgfqpoint{4.586674in}{5.053820in}}{\pgfqpoint{4.575624in}{5.053820in}}%
\pgfpathcurveto{\pgfqpoint{4.564574in}{5.053820in}}{\pgfqpoint{4.553975in}{5.049430in}}{\pgfqpoint{4.546161in}{5.041616in}}%
\pgfpathcurveto{\pgfqpoint{4.538348in}{5.033803in}}{\pgfqpoint{4.533957in}{5.023204in}}{\pgfqpoint{4.533957in}{5.012153in}}%
\pgfpathcurveto{\pgfqpoint{4.533957in}{5.001103in}}{\pgfqpoint{4.538348in}{4.990504in}}{\pgfqpoint{4.546161in}{4.982691in}}%
\pgfpathcurveto{\pgfqpoint{4.553975in}{4.974877in}}{\pgfqpoint{4.564574in}{4.970487in}}{\pgfqpoint{4.575624in}{4.970487in}}%
\pgfpathlineto{\pgfqpoint{4.575624in}{4.970487in}}%
\pgfpathclose%
\pgfusepath{stroke}%
\end{pgfscope}%
\begin{pgfscope}%
\definecolor{textcolor}{rgb}{0.000000,0.000000,0.000000}%
\pgfsetstrokecolor{textcolor}%
\pgfsetfillcolor{textcolor}%
\pgftext[x=4.925624in,y=4.961112in,left,base]{\color{textcolor}{\rmfamily\fontsize{14.000000}{16.800000}\selectfont\catcode`\^=\active\def^{\ifmmode\sp\else\^{}\fi}\catcode`\%=\active\def%{\%}optimal points}}%
\end{pgfscope}%
\begin{pgfscope}%
\pgfsetbuttcap%
\pgfsetroundjoin%
\definecolor{currentfill}{rgb}{0.121569,0.466667,0.705882}%
\pgfsetfillcolor{currentfill}%
\pgfsetlinewidth{1.003750pt}%
\definecolor{currentstroke}{rgb}{0.121569,0.466667,0.705882}%
\pgfsetstrokecolor{currentstroke}%
\pgfsetdash{}{0pt}%
\pgfpathmoveto{\pgfqpoint{4.575624in}{4.675041in}}%
\pgfpathcurveto{\pgfqpoint{4.592097in}{4.675041in}}{\pgfqpoint{4.607897in}{4.681586in}}{\pgfqpoint{4.619544in}{4.693233in}}%
\pgfpathcurveto{\pgfqpoint{4.631192in}{4.704881in}}{\pgfqpoint{4.637737in}{4.720681in}}{\pgfqpoint{4.637737in}{4.737154in}}%
\pgfpathcurveto{\pgfqpoint{4.637737in}{4.753626in}}{\pgfqpoint{4.631192in}{4.769427in}}{\pgfqpoint{4.619544in}{4.781074in}}%
\pgfpathcurveto{\pgfqpoint{4.607897in}{4.792722in}}{\pgfqpoint{4.592097in}{4.799267in}}{\pgfqpoint{4.575624in}{4.799267in}}%
\pgfpathcurveto{\pgfqpoint{4.559151in}{4.799267in}}{\pgfqpoint{4.543351in}{4.792722in}}{\pgfqpoint{4.531703in}{4.781074in}}%
\pgfpathcurveto{\pgfqpoint{4.520056in}{4.769427in}}{\pgfqpoint{4.513511in}{4.753626in}}{\pgfqpoint{4.513511in}{4.737154in}}%
\pgfpathcurveto{\pgfqpoint{4.513511in}{4.720681in}}{\pgfqpoint{4.520056in}{4.704881in}}{\pgfqpoint{4.531703in}{4.693233in}}%
\pgfpathcurveto{\pgfqpoint{4.543351in}{4.681586in}}{\pgfqpoint{4.559151in}{4.675041in}}{\pgfqpoint{4.575624in}{4.675041in}}%
\pgfpathlineto{\pgfqpoint{4.575624in}{4.675041in}}%
\pgfpathclose%
\pgfusepath{stroke,fill}%
\end{pgfscope}%
\begin{pgfscope}%
\definecolor{textcolor}{rgb}{0.000000,0.000000,0.000000}%
\pgfsetstrokecolor{textcolor}%
\pgfsetfillcolor{textcolor}%
\pgftext[x=4.925624in,y=4.686112in,left,base]{\color{textcolor}{\rmfamily\fontsize{14.000000}{16.800000}\selectfont\catcode`\^=\active\def^{\ifmmode\sp\else\^{}\fi}\catcode`\%=\active\def%{\%}selected points}}%
\end{pgfscope}%
\begin{pgfscope}%
\pgfsetbuttcap%
\pgfsetmiterjoin%
\definecolor{currentfill}{rgb}{0.827451,0.827451,0.827451}%
\pgfsetfillcolor{currentfill}%
\pgfsetfillopacity{0.300000}%
\pgfsetlinewidth{0.000000pt}%
\definecolor{currentstroke}{rgb}{0.000000,0.000000,0.000000}%
\pgfsetstrokecolor{currentstroke}%
\pgfsetstrokeopacity{0.300000}%
\pgfsetdash{}{0pt}%
\pgfpathmoveto{\pgfqpoint{4.381180in}{4.411113in}}%
\pgfpathlineto{\pgfqpoint{4.770068in}{4.411113in}}%
\pgfpathlineto{\pgfqpoint{4.770068in}{4.547224in}}%
\pgfpathlineto{\pgfqpoint{4.381180in}{4.547224in}}%
\pgfpathlineto{\pgfqpoint{4.381180in}{4.411113in}}%
\pgfpathclose%
\pgfusepath{fill}%
\end{pgfscope}%
\begin{pgfscope}%
\definecolor{textcolor}{rgb}{0.000000,0.000000,0.000000}%
\pgfsetstrokecolor{textcolor}%
\pgfsetfillcolor{textcolor}%
\pgftext[x=4.925624in,y=4.411113in,left,base]{\color{textcolor}{\rmfamily\fontsize{14.000000}{16.800000}\selectfont\catcode`\^=\active\def^{\ifmmode\sp\else\^{}\fi}\catcode`\%=\active\def%{\%}Near-optimal space}}%
\end{pgfscope}%
\begin{pgfscope}%
\pgfsetbuttcap%
\pgfsetmiterjoin%
\definecolor{currentfill}{rgb}{1.000000,1.000000,1.000000}%
\pgfsetfillcolor{currentfill}%
\pgfsetlinewidth{0.000000pt}%
\definecolor{currentstroke}{rgb}{0.000000,0.000000,0.000000}%
\pgfsetstrokecolor{currentstroke}%
\pgfsetstrokeopacity{0.000000}%
\pgfsetdash{}{0pt}%
\pgfpathmoveto{\pgfqpoint{7.394209in}{0.375000in}}%
\pgfpathlineto{\pgfqpoint{13.751042in}{0.375000in}}%
\pgfpathlineto{\pgfqpoint{13.751042in}{5.550000in}}%
\pgfpathlineto{\pgfqpoint{7.394209in}{5.550000in}}%
\pgfpathlineto{\pgfqpoint{7.394209in}{0.375000in}}%
\pgfpathclose%
\pgfusepath{fill}%
\end{pgfscope}%
\begin{pgfscope}%
\pgfpathrectangle{\pgfqpoint{7.394209in}{0.375000in}}{\pgfqpoint{6.356833in}{5.175000in}}%
\pgfusepath{clip}%
\pgfsetbuttcap%
\pgfsetroundjoin%
\pgfsetlinewidth{1.003750pt}%
\definecolor{currentstroke}{rgb}{0.827451,0.827451,0.827451}%
\pgfsetstrokecolor{currentstroke}%
\pgfsetdash{}{0pt}%
\pgfpathmoveto{\pgfqpoint{8.611667in}{3.552746in}}%
\pgfpathcurveto{\pgfqpoint{8.622717in}{3.552746in}}{\pgfqpoint{8.633316in}{3.557136in}}{\pgfqpoint{8.641129in}{3.564950in}}%
\pgfpathcurveto{\pgfqpoint{8.648943in}{3.572764in}}{\pgfqpoint{8.653333in}{3.583363in}}{\pgfqpoint{8.653333in}{3.594413in}}%
\pgfpathcurveto{\pgfqpoint{8.653333in}{3.605463in}}{\pgfqpoint{8.648943in}{3.616062in}}{\pgfqpoint{8.641129in}{3.623876in}}%
\pgfpathcurveto{\pgfqpoint{8.633316in}{3.631689in}}{\pgfqpoint{8.622717in}{3.636079in}}{\pgfqpoint{8.611667in}{3.636079in}}%
\pgfpathcurveto{\pgfqpoint{8.600616in}{3.636079in}}{\pgfqpoint{8.590017in}{3.631689in}}{\pgfqpoint{8.582204in}{3.623876in}}%
\pgfpathcurveto{\pgfqpoint{8.574390in}{3.616062in}}{\pgfqpoint{8.570000in}{3.605463in}}{\pgfqpoint{8.570000in}{3.594413in}}%
\pgfpathcurveto{\pgfqpoint{8.570000in}{3.583363in}}{\pgfqpoint{8.574390in}{3.572764in}}{\pgfqpoint{8.582204in}{3.564950in}}%
\pgfpathcurveto{\pgfqpoint{8.590017in}{3.557136in}}{\pgfqpoint{8.600616in}{3.552746in}}{\pgfqpoint{8.611667in}{3.552746in}}%
\pgfpathlineto{\pgfqpoint{8.611667in}{3.552746in}}%
\pgfpathclose%
\pgfusepath{stroke}%
\end{pgfscope}%
\begin{pgfscope}%
\pgfpathrectangle{\pgfqpoint{7.394209in}{0.375000in}}{\pgfqpoint{6.356833in}{5.175000in}}%
\pgfusepath{clip}%
\pgfsetbuttcap%
\pgfsetroundjoin%
\pgfsetlinewidth{1.003750pt}%
\definecolor{currentstroke}{rgb}{0.827451,0.827451,0.827451}%
\pgfsetstrokecolor{currentstroke}%
\pgfsetdash{}{0pt}%
\pgfpathmoveto{\pgfqpoint{9.151647in}{4.483022in}}%
\pgfpathcurveto{\pgfqpoint{9.162697in}{4.483022in}}{\pgfqpoint{9.173296in}{4.487412in}}{\pgfqpoint{9.181109in}{4.495226in}}%
\pgfpathcurveto{\pgfqpoint{9.188923in}{4.503039in}}{\pgfqpoint{9.193313in}{4.513638in}}{\pgfqpoint{9.193313in}{4.524688in}}%
\pgfpathcurveto{\pgfqpoint{9.193313in}{4.535739in}}{\pgfqpoint{9.188923in}{4.546338in}}{\pgfqpoint{9.181109in}{4.554151in}}%
\pgfpathcurveto{\pgfqpoint{9.173296in}{4.561965in}}{\pgfqpoint{9.162697in}{4.566355in}}{\pgfqpoint{9.151647in}{4.566355in}}%
\pgfpathcurveto{\pgfqpoint{9.140596in}{4.566355in}}{\pgfqpoint{9.129997in}{4.561965in}}{\pgfqpoint{9.122184in}{4.554151in}}%
\pgfpathcurveto{\pgfqpoint{9.114370in}{4.546338in}}{\pgfqpoint{9.109980in}{4.535739in}}{\pgfqpoint{9.109980in}{4.524688in}}%
\pgfpathcurveto{\pgfqpoint{9.109980in}{4.513638in}}{\pgfqpoint{9.114370in}{4.503039in}}{\pgfqpoint{9.122184in}{4.495226in}}%
\pgfpathcurveto{\pgfqpoint{9.129997in}{4.487412in}}{\pgfqpoint{9.140596in}{4.483022in}}{\pgfqpoint{9.151647in}{4.483022in}}%
\pgfpathlineto{\pgfqpoint{9.151647in}{4.483022in}}%
\pgfpathclose%
\pgfusepath{stroke}%
\end{pgfscope}%
\begin{pgfscope}%
\pgfpathrectangle{\pgfqpoint{7.394209in}{0.375000in}}{\pgfqpoint{6.356833in}{5.175000in}}%
\pgfusepath{clip}%
\pgfsetbuttcap%
\pgfsetroundjoin%
\pgfsetlinewidth{1.003750pt}%
\definecolor{currentstroke}{rgb}{0.827451,0.827451,0.827451}%
\pgfsetstrokecolor{currentstroke}%
\pgfsetdash{}{0pt}%
\pgfpathmoveto{\pgfqpoint{12.306939in}{4.901001in}}%
\pgfpathcurveto{\pgfqpoint{12.317989in}{4.901001in}}{\pgfqpoint{12.328588in}{4.905392in}}{\pgfqpoint{12.336402in}{4.913205in}}%
\pgfpathcurveto{\pgfqpoint{12.344216in}{4.921019in}}{\pgfqpoint{12.348606in}{4.931618in}}{\pgfqpoint{12.348606in}{4.942668in}}%
\pgfpathcurveto{\pgfqpoint{12.348606in}{4.953718in}}{\pgfqpoint{12.344216in}{4.964317in}}{\pgfqpoint{12.336402in}{4.972131in}}%
\pgfpathcurveto{\pgfqpoint{12.328588in}{4.979945in}}{\pgfqpoint{12.317989in}{4.984335in}}{\pgfqpoint{12.306939in}{4.984335in}}%
\pgfpathcurveto{\pgfqpoint{12.295889in}{4.984335in}}{\pgfqpoint{12.285290in}{4.979945in}}{\pgfqpoint{12.277476in}{4.972131in}}%
\pgfpathcurveto{\pgfqpoint{12.269663in}{4.964317in}}{\pgfqpoint{12.265272in}{4.953718in}}{\pgfqpoint{12.265272in}{4.942668in}}%
\pgfpathcurveto{\pgfqpoint{12.265272in}{4.931618in}}{\pgfqpoint{12.269663in}{4.921019in}}{\pgfqpoint{12.277476in}{4.913205in}}%
\pgfpathcurveto{\pgfqpoint{12.285290in}{4.905392in}}{\pgfqpoint{12.295889in}{4.901001in}}{\pgfqpoint{12.306939in}{4.901001in}}%
\pgfpathlineto{\pgfqpoint{12.306939in}{4.901001in}}%
\pgfpathclose%
\pgfusepath{stroke}%
\end{pgfscope}%
\begin{pgfscope}%
\pgfpathrectangle{\pgfqpoint{7.394209in}{0.375000in}}{\pgfqpoint{6.356833in}{5.175000in}}%
\pgfusepath{clip}%
\pgfsetbuttcap%
\pgfsetroundjoin%
\pgfsetlinewidth{1.003750pt}%
\definecolor{currentstroke}{rgb}{0.827451,0.827451,0.827451}%
\pgfsetstrokecolor{currentstroke}%
\pgfsetdash{}{0pt}%
\pgfpathmoveto{\pgfqpoint{7.873395in}{2.241997in}}%
\pgfpathcurveto{\pgfqpoint{7.884446in}{2.241997in}}{\pgfqpoint{7.895045in}{2.246388in}}{\pgfqpoint{7.902858in}{2.254201in}}%
\pgfpathcurveto{\pgfqpoint{7.910672in}{2.262015in}}{\pgfqpoint{7.915062in}{2.272614in}}{\pgfqpoint{7.915062in}{2.283664in}}%
\pgfpathcurveto{\pgfqpoint{7.915062in}{2.294714in}}{\pgfqpoint{7.910672in}{2.305313in}}{\pgfqpoint{7.902858in}{2.313127in}}%
\pgfpathcurveto{\pgfqpoint{7.895045in}{2.320940in}}{\pgfqpoint{7.884446in}{2.325331in}}{\pgfqpoint{7.873395in}{2.325331in}}%
\pgfpathcurveto{\pgfqpoint{7.862345in}{2.325331in}}{\pgfqpoint{7.851746in}{2.320940in}}{\pgfqpoint{7.843933in}{2.313127in}}%
\pgfpathcurveto{\pgfqpoint{7.836119in}{2.305313in}}{\pgfqpoint{7.831729in}{2.294714in}}{\pgfqpoint{7.831729in}{2.283664in}}%
\pgfpathcurveto{\pgfqpoint{7.831729in}{2.272614in}}{\pgfqpoint{7.836119in}{2.262015in}}{\pgfqpoint{7.843933in}{2.254201in}}%
\pgfpathcurveto{\pgfqpoint{7.851746in}{2.246388in}}{\pgfqpoint{7.862345in}{2.241997in}}{\pgfqpoint{7.873395in}{2.241997in}}%
\pgfpathlineto{\pgfqpoint{7.873395in}{2.241997in}}%
\pgfpathclose%
\pgfusepath{stroke}%
\end{pgfscope}%
\begin{pgfscope}%
\pgfpathrectangle{\pgfqpoint{7.394209in}{0.375000in}}{\pgfqpoint{6.356833in}{5.175000in}}%
\pgfusepath{clip}%
\pgfsetbuttcap%
\pgfsetroundjoin%
\pgfsetlinewidth{1.003750pt}%
\definecolor{currentstroke}{rgb}{0.827451,0.827451,0.827451}%
\pgfsetstrokecolor{currentstroke}%
\pgfsetdash{}{0pt}%
\pgfpathmoveto{\pgfqpoint{10.204824in}{5.039043in}}%
\pgfpathcurveto{\pgfqpoint{10.215874in}{5.039043in}}{\pgfqpoint{10.226473in}{5.043434in}}{\pgfqpoint{10.234287in}{5.051247in}}%
\pgfpathcurveto{\pgfqpoint{10.242101in}{5.059061in}}{\pgfqpoint{10.246491in}{5.069660in}}{\pgfqpoint{10.246491in}{5.080710in}}%
\pgfpathcurveto{\pgfqpoint{10.246491in}{5.091760in}}{\pgfqpoint{10.242101in}{5.102359in}}{\pgfqpoint{10.234287in}{5.110173in}}%
\pgfpathcurveto{\pgfqpoint{10.226473in}{5.117986in}}{\pgfqpoint{10.215874in}{5.122377in}}{\pgfqpoint{10.204824in}{5.122377in}}%
\pgfpathcurveto{\pgfqpoint{10.193774in}{5.122377in}}{\pgfqpoint{10.183175in}{5.117986in}}{\pgfqpoint{10.175362in}{5.110173in}}%
\pgfpathcurveto{\pgfqpoint{10.167548in}{5.102359in}}{\pgfqpoint{10.163158in}{5.091760in}}{\pgfqpoint{10.163158in}{5.080710in}}%
\pgfpathcurveto{\pgfqpoint{10.163158in}{5.069660in}}{\pgfqpoint{10.167548in}{5.059061in}}{\pgfqpoint{10.175362in}{5.051247in}}%
\pgfpathcurveto{\pgfqpoint{10.183175in}{5.043434in}}{\pgfqpoint{10.193774in}{5.039043in}}{\pgfqpoint{10.204824in}{5.039043in}}%
\pgfpathlineto{\pgfqpoint{10.204824in}{5.039043in}}%
\pgfpathclose%
\pgfusepath{stroke}%
\end{pgfscope}%
\begin{pgfscope}%
\pgfpathrectangle{\pgfqpoint{7.394209in}{0.375000in}}{\pgfqpoint{6.356833in}{5.175000in}}%
\pgfusepath{clip}%
\pgfsetbuttcap%
\pgfsetroundjoin%
\pgfsetlinewidth{1.003750pt}%
\definecolor{currentstroke}{rgb}{0.827451,0.827451,0.827451}%
\pgfsetstrokecolor{currentstroke}%
\pgfsetdash{}{0pt}%
\pgfpathmoveto{\pgfqpoint{10.265263in}{5.415208in}}%
\pgfpathcurveto{\pgfqpoint{10.276313in}{5.415208in}}{\pgfqpoint{10.286912in}{5.419598in}}{\pgfqpoint{10.294726in}{5.427412in}}%
\pgfpathcurveto{\pgfqpoint{10.302539in}{5.435225in}}{\pgfqpoint{10.306930in}{5.445824in}}{\pgfqpoint{10.306930in}{5.456875in}}%
\pgfpathcurveto{\pgfqpoint{10.306930in}{5.467925in}}{\pgfqpoint{10.302539in}{5.478524in}}{\pgfqpoint{10.294726in}{5.486337in}}%
\pgfpathcurveto{\pgfqpoint{10.286912in}{5.494151in}}{\pgfqpoint{10.276313in}{5.498541in}}{\pgfqpoint{10.265263in}{5.498541in}}%
\pgfpathcurveto{\pgfqpoint{10.254213in}{5.498541in}}{\pgfqpoint{10.243614in}{5.494151in}}{\pgfqpoint{10.235800in}{5.486337in}}%
\pgfpathcurveto{\pgfqpoint{10.227986in}{5.478524in}}{\pgfqpoint{10.223596in}{5.467925in}}{\pgfqpoint{10.223596in}{5.456875in}}%
\pgfpathcurveto{\pgfqpoint{10.223596in}{5.445824in}}{\pgfqpoint{10.227986in}{5.435225in}}{\pgfqpoint{10.235800in}{5.427412in}}%
\pgfpathcurveto{\pgfqpoint{10.243614in}{5.419598in}}{\pgfqpoint{10.254213in}{5.415208in}}{\pgfqpoint{10.265263in}{5.415208in}}%
\pgfpathlineto{\pgfqpoint{10.265263in}{5.415208in}}%
\pgfpathclose%
\pgfusepath{stroke}%
\end{pgfscope}%
\begin{pgfscope}%
\pgfpathrectangle{\pgfqpoint{7.394209in}{0.375000in}}{\pgfqpoint{6.356833in}{5.175000in}}%
\pgfusepath{clip}%
\pgfsetbuttcap%
\pgfsetroundjoin%
\pgfsetlinewidth{1.003750pt}%
\definecolor{currentstroke}{rgb}{0.827451,0.827451,0.827451}%
\pgfsetstrokecolor{currentstroke}%
\pgfsetdash{}{0pt}%
\pgfpathmoveto{\pgfqpoint{12.136810in}{4.633794in}}%
\pgfpathcurveto{\pgfqpoint{12.147860in}{4.633794in}}{\pgfqpoint{12.158459in}{4.638185in}}{\pgfqpoint{12.166273in}{4.645998in}}%
\pgfpathcurveto{\pgfqpoint{12.174086in}{4.653812in}}{\pgfqpoint{12.178476in}{4.664411in}}{\pgfqpoint{12.178476in}{4.675461in}}%
\pgfpathcurveto{\pgfqpoint{12.178476in}{4.686511in}}{\pgfqpoint{12.174086in}{4.697110in}}{\pgfqpoint{12.166273in}{4.704924in}}%
\pgfpathcurveto{\pgfqpoint{12.158459in}{4.712738in}}{\pgfqpoint{12.147860in}{4.717128in}}{\pgfqpoint{12.136810in}{4.717128in}}%
\pgfpathcurveto{\pgfqpoint{12.125760in}{4.717128in}}{\pgfqpoint{12.115161in}{4.712738in}}{\pgfqpoint{12.107347in}{4.704924in}}%
\pgfpathcurveto{\pgfqpoint{12.099533in}{4.697110in}}{\pgfqpoint{12.095143in}{4.686511in}}{\pgfqpoint{12.095143in}{4.675461in}}%
\pgfpathcurveto{\pgfqpoint{12.095143in}{4.664411in}}{\pgfqpoint{12.099533in}{4.653812in}}{\pgfqpoint{12.107347in}{4.645998in}}%
\pgfpathcurveto{\pgfqpoint{12.115161in}{4.638185in}}{\pgfqpoint{12.125760in}{4.633794in}}{\pgfqpoint{12.136810in}{4.633794in}}%
\pgfpathlineto{\pgfqpoint{12.136810in}{4.633794in}}%
\pgfpathclose%
\pgfusepath{stroke}%
\end{pgfscope}%
\begin{pgfscope}%
\pgfpathrectangle{\pgfqpoint{7.394209in}{0.375000in}}{\pgfqpoint{6.356833in}{5.175000in}}%
\pgfusepath{clip}%
\pgfsetbuttcap%
\pgfsetroundjoin%
\pgfsetlinewidth{1.003750pt}%
\definecolor{currentstroke}{rgb}{0.827451,0.827451,0.827451}%
\pgfsetstrokecolor{currentstroke}%
\pgfsetdash{}{0pt}%
\pgfpathmoveto{\pgfqpoint{8.364089in}{3.274853in}}%
\pgfpathcurveto{\pgfqpoint{8.375139in}{3.274853in}}{\pgfqpoint{8.385738in}{3.279243in}}{\pgfqpoint{8.393552in}{3.287057in}}%
\pgfpathcurveto{\pgfqpoint{8.401365in}{3.294870in}}{\pgfqpoint{8.405756in}{3.305469in}}{\pgfqpoint{8.405756in}{3.316520in}}%
\pgfpathcurveto{\pgfqpoint{8.405756in}{3.327570in}}{\pgfqpoint{8.401365in}{3.338169in}}{\pgfqpoint{8.393552in}{3.345982in}}%
\pgfpathcurveto{\pgfqpoint{8.385738in}{3.353796in}}{\pgfqpoint{8.375139in}{3.358186in}}{\pgfqpoint{8.364089in}{3.358186in}}%
\pgfpathcurveto{\pgfqpoint{8.353039in}{3.358186in}}{\pgfqpoint{8.342440in}{3.353796in}}{\pgfqpoint{8.334626in}{3.345982in}}%
\pgfpathcurveto{\pgfqpoint{8.326813in}{3.338169in}}{\pgfqpoint{8.322422in}{3.327570in}}{\pgfqpoint{8.322422in}{3.316520in}}%
\pgfpathcurveto{\pgfqpoint{8.322422in}{3.305469in}}{\pgfqpoint{8.326813in}{3.294870in}}{\pgfqpoint{8.334626in}{3.287057in}}%
\pgfpathcurveto{\pgfqpoint{8.342440in}{3.279243in}}{\pgfqpoint{8.353039in}{3.274853in}}{\pgfqpoint{8.364089in}{3.274853in}}%
\pgfpathlineto{\pgfqpoint{8.364089in}{3.274853in}}%
\pgfpathclose%
\pgfusepath{stroke}%
\end{pgfscope}%
\begin{pgfscope}%
\pgfpathrectangle{\pgfqpoint{7.394209in}{0.375000in}}{\pgfqpoint{6.356833in}{5.175000in}}%
\pgfusepath{clip}%
\pgfsetbuttcap%
\pgfsetroundjoin%
\pgfsetlinewidth{1.003750pt}%
\definecolor{currentstroke}{rgb}{0.827451,0.827451,0.827451}%
\pgfsetstrokecolor{currentstroke}%
\pgfsetdash{}{0pt}%
\pgfpathmoveto{\pgfqpoint{10.447772in}{2.934079in}}%
\pgfpathcurveto{\pgfqpoint{10.458823in}{2.934079in}}{\pgfqpoint{10.469422in}{2.938469in}}{\pgfqpoint{10.477235in}{2.946283in}}%
\pgfpathcurveto{\pgfqpoint{10.485049in}{2.954096in}}{\pgfqpoint{10.489439in}{2.964695in}}{\pgfqpoint{10.489439in}{2.975746in}}%
\pgfpathcurveto{\pgfqpoint{10.489439in}{2.986796in}}{\pgfqpoint{10.485049in}{2.997395in}}{\pgfqpoint{10.477235in}{3.005208in}}%
\pgfpathcurveto{\pgfqpoint{10.469422in}{3.013022in}}{\pgfqpoint{10.458823in}{3.017412in}}{\pgfqpoint{10.447772in}{3.017412in}}%
\pgfpathcurveto{\pgfqpoint{10.436722in}{3.017412in}}{\pgfqpoint{10.426123in}{3.013022in}}{\pgfqpoint{10.418310in}{3.005208in}}%
\pgfpathcurveto{\pgfqpoint{10.410496in}{2.997395in}}{\pgfqpoint{10.406106in}{2.986796in}}{\pgfqpoint{10.406106in}{2.975746in}}%
\pgfpathcurveto{\pgfqpoint{10.406106in}{2.964695in}}{\pgfqpoint{10.410496in}{2.954096in}}{\pgfqpoint{10.418310in}{2.946283in}}%
\pgfpathcurveto{\pgfqpoint{10.426123in}{2.938469in}}{\pgfqpoint{10.436722in}{2.934079in}}{\pgfqpoint{10.447772in}{2.934079in}}%
\pgfpathlineto{\pgfqpoint{10.447772in}{2.934079in}}%
\pgfpathclose%
\pgfusepath{stroke}%
\end{pgfscope}%
\begin{pgfscope}%
\pgfpathrectangle{\pgfqpoint{7.394209in}{0.375000in}}{\pgfqpoint{6.356833in}{5.175000in}}%
\pgfusepath{clip}%
\pgfsetbuttcap%
\pgfsetroundjoin%
\pgfsetlinewidth{1.003750pt}%
\definecolor{currentstroke}{rgb}{0.827451,0.827451,0.827451}%
\pgfsetstrokecolor{currentstroke}%
\pgfsetdash{}{0pt}%
\pgfpathmoveto{\pgfqpoint{9.464677in}{1.335310in}}%
\pgfpathcurveto{\pgfqpoint{9.475727in}{1.335310in}}{\pgfqpoint{9.486326in}{1.339700in}}{\pgfqpoint{9.494140in}{1.347514in}}%
\pgfpathcurveto{\pgfqpoint{9.501953in}{1.355327in}}{\pgfqpoint{9.506344in}{1.365926in}}{\pgfqpoint{9.506344in}{1.376976in}}%
\pgfpathcurveto{\pgfqpoint{9.506344in}{1.388027in}}{\pgfqpoint{9.501953in}{1.398626in}}{\pgfqpoint{9.494140in}{1.406439in}}%
\pgfpathcurveto{\pgfqpoint{9.486326in}{1.414253in}}{\pgfqpoint{9.475727in}{1.418643in}}{\pgfqpoint{9.464677in}{1.418643in}}%
\pgfpathcurveto{\pgfqpoint{9.453627in}{1.418643in}}{\pgfqpoint{9.443028in}{1.414253in}}{\pgfqpoint{9.435214in}{1.406439in}}%
\pgfpathcurveto{\pgfqpoint{9.427401in}{1.398626in}}{\pgfqpoint{9.423010in}{1.388027in}}{\pgfqpoint{9.423010in}{1.376976in}}%
\pgfpathcurveto{\pgfqpoint{9.423010in}{1.365926in}}{\pgfqpoint{9.427401in}{1.355327in}}{\pgfqpoint{9.435214in}{1.347514in}}%
\pgfpathcurveto{\pgfqpoint{9.443028in}{1.339700in}}{\pgfqpoint{9.453627in}{1.335310in}}{\pgfqpoint{9.464677in}{1.335310in}}%
\pgfpathlineto{\pgfqpoint{9.464677in}{1.335310in}}%
\pgfpathclose%
\pgfusepath{stroke}%
\end{pgfscope}%
\begin{pgfscope}%
\pgfpathrectangle{\pgfqpoint{7.394209in}{0.375000in}}{\pgfqpoint{6.356833in}{5.175000in}}%
\pgfusepath{clip}%
\pgfsetbuttcap%
\pgfsetroundjoin%
\pgfsetlinewidth{1.003750pt}%
\definecolor{currentstroke}{rgb}{0.827451,0.827451,0.827451}%
\pgfsetstrokecolor{currentstroke}%
\pgfsetdash{}{0pt}%
\pgfpathmoveto{\pgfqpoint{9.606443in}{1.278230in}}%
\pgfpathcurveto{\pgfqpoint{9.617493in}{1.278230in}}{\pgfqpoint{9.628092in}{1.282620in}}{\pgfqpoint{9.635906in}{1.290434in}}%
\pgfpathcurveto{\pgfqpoint{9.643720in}{1.298247in}}{\pgfqpoint{9.648110in}{1.308846in}}{\pgfqpoint{9.648110in}{1.319896in}}%
\pgfpathcurveto{\pgfqpoint{9.648110in}{1.330946in}}{\pgfqpoint{9.643720in}{1.341546in}}{\pgfqpoint{9.635906in}{1.349359in}}%
\pgfpathcurveto{\pgfqpoint{9.628092in}{1.357173in}}{\pgfqpoint{9.617493in}{1.361563in}}{\pgfqpoint{9.606443in}{1.361563in}}%
\pgfpathcurveto{\pgfqpoint{9.595393in}{1.361563in}}{\pgfqpoint{9.584794in}{1.357173in}}{\pgfqpoint{9.576980in}{1.349359in}}%
\pgfpathcurveto{\pgfqpoint{9.569167in}{1.341546in}}{\pgfqpoint{9.564777in}{1.330946in}}{\pgfqpoint{9.564777in}{1.319896in}}%
\pgfpathcurveto{\pgfqpoint{9.564777in}{1.308846in}}{\pgfqpoint{9.569167in}{1.298247in}}{\pgfqpoint{9.576980in}{1.290434in}}%
\pgfpathcurveto{\pgfqpoint{9.584794in}{1.282620in}}{\pgfqpoint{9.595393in}{1.278230in}}{\pgfqpoint{9.606443in}{1.278230in}}%
\pgfpathlineto{\pgfqpoint{9.606443in}{1.278230in}}%
\pgfpathclose%
\pgfusepath{stroke}%
\end{pgfscope}%
\begin{pgfscope}%
\pgfpathrectangle{\pgfqpoint{7.394209in}{0.375000in}}{\pgfqpoint{6.356833in}{5.175000in}}%
\pgfusepath{clip}%
\pgfsetbuttcap%
\pgfsetroundjoin%
\pgfsetlinewidth{1.003750pt}%
\definecolor{currentstroke}{rgb}{0.827451,0.827451,0.827451}%
\pgfsetstrokecolor{currentstroke}%
\pgfsetdash{}{0pt}%
\pgfpathmoveto{\pgfqpoint{12.013456in}{4.991288in}}%
\pgfpathcurveto{\pgfqpoint{12.024506in}{4.991288in}}{\pgfqpoint{12.035105in}{4.995678in}}{\pgfqpoint{12.042919in}{5.003492in}}%
\pgfpathcurveto{\pgfqpoint{12.050732in}{5.011305in}}{\pgfqpoint{12.055122in}{5.021904in}}{\pgfqpoint{12.055122in}{5.032955in}}%
\pgfpathcurveto{\pgfqpoint{12.055122in}{5.044005in}}{\pgfqpoint{12.050732in}{5.054604in}}{\pgfqpoint{12.042919in}{5.062417in}}%
\pgfpathcurveto{\pgfqpoint{12.035105in}{5.070231in}}{\pgfqpoint{12.024506in}{5.074621in}}{\pgfqpoint{12.013456in}{5.074621in}}%
\pgfpathcurveto{\pgfqpoint{12.002406in}{5.074621in}}{\pgfqpoint{11.991807in}{5.070231in}}{\pgfqpoint{11.983993in}{5.062417in}}%
\pgfpathcurveto{\pgfqpoint{11.976179in}{5.054604in}}{\pgfqpoint{11.971789in}{5.044005in}}{\pgfqpoint{11.971789in}{5.032955in}}%
\pgfpathcurveto{\pgfqpoint{11.971789in}{5.021904in}}{\pgfqpoint{11.976179in}{5.011305in}}{\pgfqpoint{11.983993in}{5.003492in}}%
\pgfpathcurveto{\pgfqpoint{11.991807in}{4.995678in}}{\pgfqpoint{12.002406in}{4.991288in}}{\pgfqpoint{12.013456in}{4.991288in}}%
\pgfpathlineto{\pgfqpoint{12.013456in}{4.991288in}}%
\pgfpathclose%
\pgfusepath{stroke}%
\end{pgfscope}%
\begin{pgfscope}%
\pgfpathrectangle{\pgfqpoint{7.394209in}{0.375000in}}{\pgfqpoint{6.356833in}{5.175000in}}%
\pgfusepath{clip}%
\pgfsetbuttcap%
\pgfsetroundjoin%
\pgfsetlinewidth{1.003750pt}%
\definecolor{currentstroke}{rgb}{0.827451,0.827451,0.827451}%
\pgfsetstrokecolor{currentstroke}%
\pgfsetdash{}{0pt}%
\pgfpathmoveto{\pgfqpoint{9.244844in}{1.116803in}}%
\pgfpathcurveto{\pgfqpoint{9.255894in}{1.116803in}}{\pgfqpoint{9.266493in}{1.121194in}}{\pgfqpoint{9.274307in}{1.129007in}}%
\pgfpathcurveto{\pgfqpoint{9.282120in}{1.136821in}}{\pgfqpoint{9.286511in}{1.147420in}}{\pgfqpoint{9.286511in}{1.158470in}}%
\pgfpathcurveto{\pgfqpoint{9.286511in}{1.169520in}}{\pgfqpoint{9.282120in}{1.180119in}}{\pgfqpoint{9.274307in}{1.187933in}}%
\pgfpathcurveto{\pgfqpoint{9.266493in}{1.195747in}}{\pgfqpoint{9.255894in}{1.200137in}}{\pgfqpoint{9.244844in}{1.200137in}}%
\pgfpathcurveto{\pgfqpoint{9.233794in}{1.200137in}}{\pgfqpoint{9.223195in}{1.195747in}}{\pgfqpoint{9.215381in}{1.187933in}}%
\pgfpathcurveto{\pgfqpoint{9.207568in}{1.180119in}}{\pgfqpoint{9.203177in}{1.169520in}}{\pgfqpoint{9.203177in}{1.158470in}}%
\pgfpathcurveto{\pgfqpoint{9.203177in}{1.147420in}}{\pgfqpoint{9.207568in}{1.136821in}}{\pgfqpoint{9.215381in}{1.129007in}}%
\pgfpathcurveto{\pgfqpoint{9.223195in}{1.121194in}}{\pgfqpoint{9.233794in}{1.116803in}}{\pgfqpoint{9.244844in}{1.116803in}}%
\pgfpathlineto{\pgfqpoint{9.244844in}{1.116803in}}%
\pgfpathclose%
\pgfusepath{stroke}%
\end{pgfscope}%
\begin{pgfscope}%
\pgfpathrectangle{\pgfqpoint{7.394209in}{0.375000in}}{\pgfqpoint{6.356833in}{5.175000in}}%
\pgfusepath{clip}%
\pgfsetbuttcap%
\pgfsetroundjoin%
\pgfsetlinewidth{1.003750pt}%
\definecolor{currentstroke}{rgb}{0.827451,0.827451,0.827451}%
\pgfsetstrokecolor{currentstroke}%
\pgfsetdash{}{0pt}%
\pgfpathmoveto{\pgfqpoint{9.524859in}{3.736164in}}%
\pgfpathcurveto{\pgfqpoint{9.535909in}{3.736164in}}{\pgfqpoint{9.546508in}{3.740554in}}{\pgfqpoint{9.554322in}{3.748368in}}%
\pgfpathcurveto{\pgfqpoint{9.562135in}{3.756181in}}{\pgfqpoint{9.566525in}{3.766780in}}{\pgfqpoint{9.566525in}{3.777830in}}%
\pgfpathcurveto{\pgfqpoint{9.566525in}{3.788880in}}{\pgfqpoint{9.562135in}{3.799479in}}{\pgfqpoint{9.554322in}{3.807293in}}%
\pgfpathcurveto{\pgfqpoint{9.546508in}{3.815107in}}{\pgfqpoint{9.535909in}{3.819497in}}{\pgfqpoint{9.524859in}{3.819497in}}%
\pgfpathcurveto{\pgfqpoint{9.513809in}{3.819497in}}{\pgfqpoint{9.503210in}{3.815107in}}{\pgfqpoint{9.495396in}{3.807293in}}%
\pgfpathcurveto{\pgfqpoint{9.487582in}{3.799479in}}{\pgfqpoint{9.483192in}{3.788880in}}{\pgfqpoint{9.483192in}{3.777830in}}%
\pgfpathcurveto{\pgfqpoint{9.483192in}{3.766780in}}{\pgfqpoint{9.487582in}{3.756181in}}{\pgfqpoint{9.495396in}{3.748368in}}%
\pgfpathcurveto{\pgfqpoint{9.503210in}{3.740554in}}{\pgfqpoint{9.513809in}{3.736164in}}{\pgfqpoint{9.524859in}{3.736164in}}%
\pgfpathlineto{\pgfqpoint{9.524859in}{3.736164in}}%
\pgfpathclose%
\pgfusepath{stroke}%
\end{pgfscope}%
\begin{pgfscope}%
\pgfpathrectangle{\pgfqpoint{7.394209in}{0.375000in}}{\pgfqpoint{6.356833in}{5.175000in}}%
\pgfusepath{clip}%
\pgfsetbuttcap%
\pgfsetroundjoin%
\pgfsetlinewidth{1.003750pt}%
\definecolor{currentstroke}{rgb}{0.827451,0.827451,0.827451}%
\pgfsetstrokecolor{currentstroke}%
\pgfsetdash{}{0pt}%
\pgfpathmoveto{\pgfqpoint{8.494369in}{1.026893in}}%
\pgfpathcurveto{\pgfqpoint{8.505419in}{1.026893in}}{\pgfqpoint{8.516018in}{1.031283in}}{\pgfqpoint{8.523832in}{1.039097in}}%
\pgfpathcurveto{\pgfqpoint{8.531645in}{1.046910in}}{\pgfqpoint{8.536036in}{1.057509in}}{\pgfqpoint{8.536036in}{1.068559in}}%
\pgfpathcurveto{\pgfqpoint{8.536036in}{1.079610in}}{\pgfqpoint{8.531645in}{1.090209in}}{\pgfqpoint{8.523832in}{1.098022in}}%
\pgfpathcurveto{\pgfqpoint{8.516018in}{1.105836in}}{\pgfqpoint{8.505419in}{1.110226in}}{\pgfqpoint{8.494369in}{1.110226in}}%
\pgfpathcurveto{\pgfqpoint{8.483319in}{1.110226in}}{\pgfqpoint{8.472720in}{1.105836in}}{\pgfqpoint{8.464906in}{1.098022in}}%
\pgfpathcurveto{\pgfqpoint{8.457092in}{1.090209in}}{\pgfqpoint{8.452702in}{1.079610in}}{\pgfqpoint{8.452702in}{1.068559in}}%
\pgfpathcurveto{\pgfqpoint{8.452702in}{1.057509in}}{\pgfqpoint{8.457092in}{1.046910in}}{\pgfqpoint{8.464906in}{1.039097in}}%
\pgfpathcurveto{\pgfqpoint{8.472720in}{1.031283in}}{\pgfqpoint{8.483319in}{1.026893in}}{\pgfqpoint{8.494369in}{1.026893in}}%
\pgfpathlineto{\pgfqpoint{8.494369in}{1.026893in}}%
\pgfpathclose%
\pgfusepath{stroke}%
\end{pgfscope}%
\begin{pgfscope}%
\pgfpathrectangle{\pgfqpoint{7.394209in}{0.375000in}}{\pgfqpoint{6.356833in}{5.175000in}}%
\pgfusepath{clip}%
\pgfsetbuttcap%
\pgfsetroundjoin%
\pgfsetlinewidth{1.003750pt}%
\definecolor{currentstroke}{rgb}{0.827451,0.827451,0.827451}%
\pgfsetstrokecolor{currentstroke}%
\pgfsetdash{}{0pt}%
\pgfpathmoveto{\pgfqpoint{7.933103in}{2.056621in}}%
\pgfpathcurveto{\pgfqpoint{7.944153in}{2.056621in}}{\pgfqpoint{7.954752in}{2.061011in}}{\pgfqpoint{7.962565in}{2.068825in}}%
\pgfpathcurveto{\pgfqpoint{7.970379in}{2.076638in}}{\pgfqpoint{7.974769in}{2.087237in}}{\pgfqpoint{7.974769in}{2.098287in}}%
\pgfpathcurveto{\pgfqpoint{7.974769in}{2.109338in}}{\pgfqpoint{7.970379in}{2.119937in}}{\pgfqpoint{7.962565in}{2.127750in}}%
\pgfpathcurveto{\pgfqpoint{7.954752in}{2.135564in}}{\pgfqpoint{7.944153in}{2.139954in}}{\pgfqpoint{7.933103in}{2.139954in}}%
\pgfpathcurveto{\pgfqpoint{7.922052in}{2.139954in}}{\pgfqpoint{7.911453in}{2.135564in}}{\pgfqpoint{7.903640in}{2.127750in}}%
\pgfpathcurveto{\pgfqpoint{7.895826in}{2.119937in}}{\pgfqpoint{7.891436in}{2.109338in}}{\pgfqpoint{7.891436in}{2.098287in}}%
\pgfpathcurveto{\pgfqpoint{7.891436in}{2.087237in}}{\pgfqpoint{7.895826in}{2.076638in}}{\pgfqpoint{7.903640in}{2.068825in}}%
\pgfpathcurveto{\pgfqpoint{7.911453in}{2.061011in}}{\pgfqpoint{7.922052in}{2.056621in}}{\pgfqpoint{7.933103in}{2.056621in}}%
\pgfpathlineto{\pgfqpoint{7.933103in}{2.056621in}}%
\pgfpathclose%
\pgfusepath{stroke}%
\end{pgfscope}%
\begin{pgfscope}%
\pgfpathrectangle{\pgfqpoint{7.394209in}{0.375000in}}{\pgfqpoint{6.356833in}{5.175000in}}%
\pgfusepath{clip}%
\pgfsetbuttcap%
\pgfsetroundjoin%
\pgfsetlinewidth{1.003750pt}%
\definecolor{currentstroke}{rgb}{0.827451,0.827451,0.827451}%
\pgfsetstrokecolor{currentstroke}%
\pgfsetdash{}{0pt}%
\pgfpathmoveto{\pgfqpoint{12.235598in}{5.065369in}}%
\pgfpathcurveto{\pgfqpoint{12.246648in}{5.065369in}}{\pgfqpoint{12.257247in}{5.069760in}}{\pgfqpoint{12.265061in}{5.077573in}}%
\pgfpathcurveto{\pgfqpoint{12.272875in}{5.085387in}}{\pgfqpoint{12.277265in}{5.095986in}}{\pgfqpoint{12.277265in}{5.107036in}}%
\pgfpathcurveto{\pgfqpoint{12.277265in}{5.118086in}}{\pgfqpoint{12.272875in}{5.128685in}}{\pgfqpoint{12.265061in}{5.136499in}}%
\pgfpathcurveto{\pgfqpoint{12.257247in}{5.144312in}}{\pgfqpoint{12.246648in}{5.148703in}}{\pgfqpoint{12.235598in}{5.148703in}}%
\pgfpathcurveto{\pgfqpoint{12.224548in}{5.148703in}}{\pgfqpoint{12.213949in}{5.144312in}}{\pgfqpoint{12.206135in}{5.136499in}}%
\pgfpathcurveto{\pgfqpoint{12.198322in}{5.128685in}}{\pgfqpoint{12.193932in}{5.118086in}}{\pgfqpoint{12.193932in}{5.107036in}}%
\pgfpathcurveto{\pgfqpoint{12.193932in}{5.095986in}}{\pgfqpoint{12.198322in}{5.085387in}}{\pgfqpoint{12.206135in}{5.077573in}}%
\pgfpathcurveto{\pgfqpoint{12.213949in}{5.069760in}}{\pgfqpoint{12.224548in}{5.065369in}}{\pgfqpoint{12.235598in}{5.065369in}}%
\pgfpathlineto{\pgfqpoint{12.235598in}{5.065369in}}%
\pgfpathclose%
\pgfusepath{stroke}%
\end{pgfscope}%
\begin{pgfscope}%
\pgfpathrectangle{\pgfqpoint{7.394209in}{0.375000in}}{\pgfqpoint{6.356833in}{5.175000in}}%
\pgfusepath{clip}%
\pgfsetbuttcap%
\pgfsetroundjoin%
\pgfsetlinewidth{1.003750pt}%
\definecolor{currentstroke}{rgb}{0.827451,0.827451,0.827451}%
\pgfsetstrokecolor{currentstroke}%
\pgfsetdash{}{0pt}%
\pgfpathmoveto{\pgfqpoint{13.055333in}{5.369979in}}%
\pgfpathcurveto{\pgfqpoint{13.066383in}{5.369979in}}{\pgfqpoint{13.076982in}{5.374369in}}{\pgfqpoint{13.084795in}{5.382183in}}%
\pgfpathcurveto{\pgfqpoint{13.092609in}{5.389996in}}{\pgfqpoint{13.096999in}{5.400595in}}{\pgfqpoint{13.096999in}{5.411645in}}%
\pgfpathcurveto{\pgfqpoint{13.096999in}{5.422695in}}{\pgfqpoint{13.092609in}{5.433294in}}{\pgfqpoint{13.084795in}{5.441108in}}%
\pgfpathcurveto{\pgfqpoint{13.076982in}{5.448922in}}{\pgfqpoint{13.066383in}{5.453312in}}{\pgfqpoint{13.055333in}{5.453312in}}%
\pgfpathcurveto{\pgfqpoint{13.044282in}{5.453312in}}{\pgfqpoint{13.033683in}{5.448922in}}{\pgfqpoint{13.025870in}{5.441108in}}%
\pgfpathcurveto{\pgfqpoint{13.018056in}{5.433294in}}{\pgfqpoint{13.013666in}{5.422695in}}{\pgfqpoint{13.013666in}{5.411645in}}%
\pgfpathcurveto{\pgfqpoint{13.013666in}{5.400595in}}{\pgfqpoint{13.018056in}{5.389996in}}{\pgfqpoint{13.025870in}{5.382183in}}%
\pgfpathcurveto{\pgfqpoint{13.033683in}{5.374369in}}{\pgfqpoint{13.044282in}{5.369979in}}{\pgfqpoint{13.055333in}{5.369979in}}%
\pgfpathlineto{\pgfqpoint{13.055333in}{5.369979in}}%
\pgfpathclose%
\pgfusepath{stroke}%
\end{pgfscope}%
\begin{pgfscope}%
\pgfpathrectangle{\pgfqpoint{7.394209in}{0.375000in}}{\pgfqpoint{6.356833in}{5.175000in}}%
\pgfusepath{clip}%
\pgfsetbuttcap%
\pgfsetroundjoin%
\pgfsetlinewidth{1.003750pt}%
\definecolor{currentstroke}{rgb}{0.827451,0.827451,0.827451}%
\pgfsetstrokecolor{currentstroke}%
\pgfsetdash{}{0pt}%
\pgfpathmoveto{\pgfqpoint{9.718389in}{4.481627in}}%
\pgfpathcurveto{\pgfqpoint{9.729439in}{4.481627in}}{\pgfqpoint{9.740038in}{4.486017in}}{\pgfqpoint{9.747852in}{4.493831in}}%
\pgfpathcurveto{\pgfqpoint{9.755665in}{4.501644in}}{\pgfqpoint{9.760056in}{4.512243in}}{\pgfqpoint{9.760056in}{4.523293in}}%
\pgfpathcurveto{\pgfqpoint{9.760056in}{4.534343in}}{\pgfqpoint{9.755665in}{4.544943in}}{\pgfqpoint{9.747852in}{4.552756in}}%
\pgfpathcurveto{\pgfqpoint{9.740038in}{4.560570in}}{\pgfqpoint{9.729439in}{4.564960in}}{\pgfqpoint{9.718389in}{4.564960in}}%
\pgfpathcurveto{\pgfqpoint{9.707339in}{4.564960in}}{\pgfqpoint{9.696740in}{4.560570in}}{\pgfqpoint{9.688926in}{4.552756in}}%
\pgfpathcurveto{\pgfqpoint{9.681112in}{4.544943in}}{\pgfqpoint{9.676722in}{4.534343in}}{\pgfqpoint{9.676722in}{4.523293in}}%
\pgfpathcurveto{\pgfqpoint{9.676722in}{4.512243in}}{\pgfqpoint{9.681112in}{4.501644in}}{\pgfqpoint{9.688926in}{4.493831in}}%
\pgfpathcurveto{\pgfqpoint{9.696740in}{4.486017in}}{\pgfqpoint{9.707339in}{4.481627in}}{\pgfqpoint{9.718389in}{4.481627in}}%
\pgfpathlineto{\pgfqpoint{9.718389in}{4.481627in}}%
\pgfpathclose%
\pgfusepath{stroke}%
\end{pgfscope}%
\begin{pgfscope}%
\pgfpathrectangle{\pgfqpoint{7.394209in}{0.375000in}}{\pgfqpoint{6.356833in}{5.175000in}}%
\pgfusepath{clip}%
\pgfsetbuttcap%
\pgfsetroundjoin%
\pgfsetlinewidth{1.003750pt}%
\definecolor{currentstroke}{rgb}{0.827451,0.827451,0.827451}%
\pgfsetstrokecolor{currentstroke}%
\pgfsetdash{}{0pt}%
\pgfpathmoveto{\pgfqpoint{8.201714in}{2.246271in}}%
\pgfpathcurveto{\pgfqpoint{8.212764in}{2.246271in}}{\pgfqpoint{8.223363in}{2.250661in}}{\pgfqpoint{8.231177in}{2.258475in}}%
\pgfpathcurveto{\pgfqpoint{8.238990in}{2.266289in}}{\pgfqpoint{8.243381in}{2.276888in}}{\pgfqpoint{8.243381in}{2.287938in}}%
\pgfpathcurveto{\pgfqpoint{8.243381in}{2.298988in}}{\pgfqpoint{8.238990in}{2.309587in}}{\pgfqpoint{8.231177in}{2.317401in}}%
\pgfpathcurveto{\pgfqpoint{8.223363in}{2.325214in}}{\pgfqpoint{8.212764in}{2.329605in}}{\pgfqpoint{8.201714in}{2.329605in}}%
\pgfpathcurveto{\pgfqpoint{8.190664in}{2.329605in}}{\pgfqpoint{8.180065in}{2.325214in}}{\pgfqpoint{8.172251in}{2.317401in}}%
\pgfpathcurveto{\pgfqpoint{8.164438in}{2.309587in}}{\pgfqpoint{8.160047in}{2.298988in}}{\pgfqpoint{8.160047in}{2.287938in}}%
\pgfpathcurveto{\pgfqpoint{8.160047in}{2.276888in}}{\pgfqpoint{8.164438in}{2.266289in}}{\pgfqpoint{8.172251in}{2.258475in}}%
\pgfpathcurveto{\pgfqpoint{8.180065in}{2.250661in}}{\pgfqpoint{8.190664in}{2.246271in}}{\pgfqpoint{8.201714in}{2.246271in}}%
\pgfpathlineto{\pgfqpoint{8.201714in}{2.246271in}}%
\pgfpathclose%
\pgfusepath{stroke}%
\end{pgfscope}%
\begin{pgfscope}%
\pgfpathrectangle{\pgfqpoint{7.394209in}{0.375000in}}{\pgfqpoint{6.356833in}{5.175000in}}%
\pgfusepath{clip}%
\pgfsetbuttcap%
\pgfsetroundjoin%
\pgfsetlinewidth{1.003750pt}%
\definecolor{currentstroke}{rgb}{0.827451,0.827451,0.827451}%
\pgfsetstrokecolor{currentstroke}%
\pgfsetdash{}{0pt}%
\pgfpathmoveto{\pgfqpoint{12.494768in}{5.226571in}}%
\pgfpathcurveto{\pgfqpoint{12.505818in}{5.226571in}}{\pgfqpoint{12.516417in}{5.230962in}}{\pgfqpoint{12.524231in}{5.238775in}}%
\pgfpathcurveto{\pgfqpoint{12.532044in}{5.246589in}}{\pgfqpoint{12.536435in}{5.257188in}}{\pgfqpoint{12.536435in}{5.268238in}}%
\pgfpathcurveto{\pgfqpoint{12.536435in}{5.279288in}}{\pgfqpoint{12.532044in}{5.289887in}}{\pgfqpoint{12.524231in}{5.297701in}}%
\pgfpathcurveto{\pgfqpoint{12.516417in}{5.305514in}}{\pgfqpoint{12.505818in}{5.309905in}}{\pgfqpoint{12.494768in}{5.309905in}}%
\pgfpathcurveto{\pgfqpoint{12.483718in}{5.309905in}}{\pgfqpoint{12.473119in}{5.305514in}}{\pgfqpoint{12.465305in}{5.297701in}}%
\pgfpathcurveto{\pgfqpoint{12.457492in}{5.289887in}}{\pgfqpoint{12.453101in}{5.279288in}}{\pgfqpoint{12.453101in}{5.268238in}}%
\pgfpathcurveto{\pgfqpoint{12.453101in}{5.257188in}}{\pgfqpoint{12.457492in}{5.246589in}}{\pgfqpoint{12.465305in}{5.238775in}}%
\pgfpathcurveto{\pgfqpoint{12.473119in}{5.230962in}}{\pgfqpoint{12.483718in}{5.226571in}}{\pgfqpoint{12.494768in}{5.226571in}}%
\pgfpathlineto{\pgfqpoint{12.494768in}{5.226571in}}%
\pgfpathclose%
\pgfusepath{stroke}%
\end{pgfscope}%
\begin{pgfscope}%
\pgfpathrectangle{\pgfqpoint{7.394209in}{0.375000in}}{\pgfqpoint{6.356833in}{5.175000in}}%
\pgfusepath{clip}%
\pgfsetbuttcap%
\pgfsetroundjoin%
\pgfsetlinewidth{1.003750pt}%
\definecolor{currentstroke}{rgb}{0.827451,0.827451,0.827451}%
\pgfsetstrokecolor{currentstroke}%
\pgfsetdash{}{0pt}%
\pgfpathmoveto{\pgfqpoint{11.174140in}{3.093213in}}%
\pgfpathcurveto{\pgfqpoint{11.185190in}{3.093213in}}{\pgfqpoint{11.195789in}{3.097603in}}{\pgfqpoint{11.203603in}{3.105417in}}%
\pgfpathcurveto{\pgfqpoint{11.211416in}{3.113231in}}{\pgfqpoint{11.215806in}{3.123830in}}{\pgfqpoint{11.215806in}{3.134880in}}%
\pgfpathcurveto{\pgfqpoint{11.215806in}{3.145930in}}{\pgfqpoint{11.211416in}{3.156529in}}{\pgfqpoint{11.203603in}{3.164343in}}%
\pgfpathcurveto{\pgfqpoint{11.195789in}{3.172156in}}{\pgfqpoint{11.185190in}{3.176547in}}{\pgfqpoint{11.174140in}{3.176547in}}%
\pgfpathcurveto{\pgfqpoint{11.163090in}{3.176547in}}{\pgfqpoint{11.152491in}{3.172156in}}{\pgfqpoint{11.144677in}{3.164343in}}%
\pgfpathcurveto{\pgfqpoint{11.136863in}{3.156529in}}{\pgfqpoint{11.132473in}{3.145930in}}{\pgfqpoint{11.132473in}{3.134880in}}%
\pgfpathcurveto{\pgfqpoint{11.132473in}{3.123830in}}{\pgfqpoint{11.136863in}{3.113231in}}{\pgfqpoint{11.144677in}{3.105417in}}%
\pgfpathcurveto{\pgfqpoint{11.152491in}{3.097603in}}{\pgfqpoint{11.163090in}{3.093213in}}{\pgfqpoint{11.174140in}{3.093213in}}%
\pgfpathlineto{\pgfqpoint{11.174140in}{3.093213in}}%
\pgfpathclose%
\pgfusepath{stroke}%
\end{pgfscope}%
\begin{pgfscope}%
\pgfpathrectangle{\pgfqpoint{7.394209in}{0.375000in}}{\pgfqpoint{6.356833in}{5.175000in}}%
\pgfusepath{clip}%
\pgfsetbuttcap%
\pgfsetroundjoin%
\pgfsetlinewidth{1.003750pt}%
\definecolor{currentstroke}{rgb}{0.827451,0.827451,0.827451}%
\pgfsetstrokecolor{currentstroke}%
\pgfsetdash{}{0pt}%
\pgfpathmoveto{\pgfqpoint{8.676737in}{3.947017in}}%
\pgfpathcurveto{\pgfqpoint{8.687787in}{3.947017in}}{\pgfqpoint{8.698386in}{3.951407in}}{\pgfqpoint{8.706199in}{3.959221in}}%
\pgfpathcurveto{\pgfqpoint{8.714013in}{3.967034in}}{\pgfqpoint{8.718403in}{3.977633in}}{\pgfqpoint{8.718403in}{3.988684in}}%
\pgfpathcurveto{\pgfqpoint{8.718403in}{3.999734in}}{\pgfqpoint{8.714013in}{4.010333in}}{\pgfqpoint{8.706199in}{4.018146in}}%
\pgfpathcurveto{\pgfqpoint{8.698386in}{4.025960in}}{\pgfqpoint{8.687787in}{4.030350in}}{\pgfqpoint{8.676737in}{4.030350in}}%
\pgfpathcurveto{\pgfqpoint{8.665687in}{4.030350in}}{\pgfqpoint{8.655087in}{4.025960in}}{\pgfqpoint{8.647274in}{4.018146in}}%
\pgfpathcurveto{\pgfqpoint{8.639460in}{4.010333in}}{\pgfqpoint{8.635070in}{3.999734in}}{\pgfqpoint{8.635070in}{3.988684in}}%
\pgfpathcurveto{\pgfqpoint{8.635070in}{3.977633in}}{\pgfqpoint{8.639460in}{3.967034in}}{\pgfqpoint{8.647274in}{3.959221in}}%
\pgfpathcurveto{\pgfqpoint{8.655087in}{3.951407in}}{\pgfqpoint{8.665687in}{3.947017in}}{\pgfqpoint{8.676737in}{3.947017in}}%
\pgfpathlineto{\pgfqpoint{8.676737in}{3.947017in}}%
\pgfpathclose%
\pgfusepath{stroke}%
\end{pgfscope}%
\begin{pgfscope}%
\pgfpathrectangle{\pgfqpoint{7.394209in}{0.375000in}}{\pgfqpoint{6.356833in}{5.175000in}}%
\pgfusepath{clip}%
\pgfsetbuttcap%
\pgfsetroundjoin%
\pgfsetlinewidth{1.003750pt}%
\definecolor{currentstroke}{rgb}{0.827451,0.827451,0.827451}%
\pgfsetstrokecolor{currentstroke}%
\pgfsetdash{}{0pt}%
\pgfpathmoveto{\pgfqpoint{12.270642in}{3.997830in}}%
\pgfpathcurveto{\pgfqpoint{12.281692in}{3.997830in}}{\pgfqpoint{12.292291in}{4.002220in}}{\pgfqpoint{12.300104in}{4.010034in}}%
\pgfpathcurveto{\pgfqpoint{12.307918in}{4.017848in}}{\pgfqpoint{12.312308in}{4.028447in}}{\pgfqpoint{12.312308in}{4.039497in}}%
\pgfpathcurveto{\pgfqpoint{12.312308in}{4.050547in}}{\pgfqpoint{12.307918in}{4.061146in}}{\pgfqpoint{12.300104in}{4.068960in}}%
\pgfpathcurveto{\pgfqpoint{12.292291in}{4.076773in}}{\pgfqpoint{12.281692in}{4.081164in}}{\pgfqpoint{12.270642in}{4.081164in}}%
\pgfpathcurveto{\pgfqpoint{12.259591in}{4.081164in}}{\pgfqpoint{12.248992in}{4.076773in}}{\pgfqpoint{12.241179in}{4.068960in}}%
\pgfpathcurveto{\pgfqpoint{12.233365in}{4.061146in}}{\pgfqpoint{12.228975in}{4.050547in}}{\pgfqpoint{12.228975in}{4.039497in}}%
\pgfpathcurveto{\pgfqpoint{12.228975in}{4.028447in}}{\pgfqpoint{12.233365in}{4.017848in}}{\pgfqpoint{12.241179in}{4.010034in}}%
\pgfpathcurveto{\pgfqpoint{12.248992in}{4.002220in}}{\pgfqpoint{12.259591in}{3.997830in}}{\pgfqpoint{12.270642in}{3.997830in}}%
\pgfpathlineto{\pgfqpoint{12.270642in}{3.997830in}}%
\pgfpathclose%
\pgfusepath{stroke}%
\end{pgfscope}%
\begin{pgfscope}%
\pgfpathrectangle{\pgfqpoint{7.394209in}{0.375000in}}{\pgfqpoint{6.356833in}{5.175000in}}%
\pgfusepath{clip}%
\pgfsetbuttcap%
\pgfsetroundjoin%
\pgfsetlinewidth{1.003750pt}%
\definecolor{currentstroke}{rgb}{0.827451,0.827451,0.827451}%
\pgfsetstrokecolor{currentstroke}%
\pgfsetdash{}{0pt}%
\pgfpathmoveto{\pgfqpoint{12.369929in}{3.962847in}}%
\pgfpathcurveto{\pgfqpoint{12.380979in}{3.962847in}}{\pgfqpoint{12.391578in}{3.967238in}}{\pgfqpoint{12.399392in}{3.975051in}}%
\pgfpathcurveto{\pgfqpoint{12.407205in}{3.982865in}}{\pgfqpoint{12.411595in}{3.993464in}}{\pgfqpoint{12.411595in}{4.004514in}}%
\pgfpathcurveto{\pgfqpoint{12.411595in}{4.015564in}}{\pgfqpoint{12.407205in}{4.026163in}}{\pgfqpoint{12.399392in}{4.033977in}}%
\pgfpathcurveto{\pgfqpoint{12.391578in}{4.041790in}}{\pgfqpoint{12.380979in}{4.046181in}}{\pgfqpoint{12.369929in}{4.046181in}}%
\pgfpathcurveto{\pgfqpoint{12.358879in}{4.046181in}}{\pgfqpoint{12.348280in}{4.041790in}}{\pgfqpoint{12.340466in}{4.033977in}}%
\pgfpathcurveto{\pgfqpoint{12.332652in}{4.026163in}}{\pgfqpoint{12.328262in}{4.015564in}}{\pgfqpoint{12.328262in}{4.004514in}}%
\pgfpathcurveto{\pgfqpoint{12.328262in}{3.993464in}}{\pgfqpoint{12.332652in}{3.982865in}}{\pgfqpoint{12.340466in}{3.975051in}}%
\pgfpathcurveto{\pgfqpoint{12.348280in}{3.967238in}}{\pgfqpoint{12.358879in}{3.962847in}}{\pgfqpoint{12.369929in}{3.962847in}}%
\pgfpathlineto{\pgfqpoint{12.369929in}{3.962847in}}%
\pgfpathclose%
\pgfusepath{stroke}%
\end{pgfscope}%
\begin{pgfscope}%
\pgfpathrectangle{\pgfqpoint{7.394209in}{0.375000in}}{\pgfqpoint{6.356833in}{5.175000in}}%
\pgfusepath{clip}%
\pgfsetbuttcap%
\pgfsetroundjoin%
\pgfsetlinewidth{1.003750pt}%
\definecolor{currentstroke}{rgb}{0.827451,0.827451,0.827451}%
\pgfsetstrokecolor{currentstroke}%
\pgfsetdash{}{0pt}%
\pgfpathmoveto{\pgfqpoint{8.105503in}{3.475561in}}%
\pgfpathcurveto{\pgfqpoint{8.116553in}{3.475561in}}{\pgfqpoint{8.127152in}{3.479951in}}{\pgfqpoint{8.134966in}{3.487764in}}%
\pgfpathcurveto{\pgfqpoint{8.142779in}{3.495578in}}{\pgfqpoint{8.147170in}{3.506177in}}{\pgfqpoint{8.147170in}{3.517227in}}%
\pgfpathcurveto{\pgfqpoint{8.147170in}{3.528277in}}{\pgfqpoint{8.142779in}{3.538876in}}{\pgfqpoint{8.134966in}{3.546690in}}%
\pgfpathcurveto{\pgfqpoint{8.127152in}{3.554504in}}{\pgfqpoint{8.116553in}{3.558894in}}{\pgfqpoint{8.105503in}{3.558894in}}%
\pgfpathcurveto{\pgfqpoint{8.094453in}{3.558894in}}{\pgfqpoint{8.083854in}{3.554504in}}{\pgfqpoint{8.076040in}{3.546690in}}%
\pgfpathcurveto{\pgfqpoint{8.068227in}{3.538876in}}{\pgfqpoint{8.063836in}{3.528277in}}{\pgfqpoint{8.063836in}{3.517227in}}%
\pgfpathcurveto{\pgfqpoint{8.063836in}{3.506177in}}{\pgfqpoint{8.068227in}{3.495578in}}{\pgfqpoint{8.076040in}{3.487764in}}%
\pgfpathcurveto{\pgfqpoint{8.083854in}{3.479951in}}{\pgfqpoint{8.094453in}{3.475561in}}{\pgfqpoint{8.105503in}{3.475561in}}%
\pgfpathlineto{\pgfqpoint{8.105503in}{3.475561in}}%
\pgfpathclose%
\pgfusepath{stroke}%
\end{pgfscope}%
\begin{pgfscope}%
\pgfpathrectangle{\pgfqpoint{7.394209in}{0.375000in}}{\pgfqpoint{6.356833in}{5.175000in}}%
\pgfusepath{clip}%
\pgfsetbuttcap%
\pgfsetroundjoin%
\pgfsetlinewidth{1.003750pt}%
\definecolor{currentstroke}{rgb}{0.827451,0.827451,0.827451}%
\pgfsetstrokecolor{currentstroke}%
\pgfsetdash{}{0pt}%
\pgfpathmoveto{\pgfqpoint{11.738869in}{4.021566in}}%
\pgfpathcurveto{\pgfqpoint{11.749919in}{4.021566in}}{\pgfqpoint{11.760518in}{4.025956in}}{\pgfqpoint{11.768332in}{4.033770in}}%
\pgfpathcurveto{\pgfqpoint{11.776145in}{4.041584in}}{\pgfqpoint{11.780536in}{4.052183in}}{\pgfqpoint{11.780536in}{4.063233in}}%
\pgfpathcurveto{\pgfqpoint{11.780536in}{4.074283in}}{\pgfqpoint{11.776145in}{4.084882in}}{\pgfqpoint{11.768332in}{4.092696in}}%
\pgfpathcurveto{\pgfqpoint{11.760518in}{4.100509in}}{\pgfqpoint{11.749919in}{4.104900in}}{\pgfqpoint{11.738869in}{4.104900in}}%
\pgfpathcurveto{\pgfqpoint{11.727819in}{4.104900in}}{\pgfqpoint{11.717220in}{4.100509in}}{\pgfqpoint{11.709406in}{4.092696in}}%
\pgfpathcurveto{\pgfqpoint{11.701593in}{4.084882in}}{\pgfqpoint{11.697202in}{4.074283in}}{\pgfqpoint{11.697202in}{4.063233in}}%
\pgfpathcurveto{\pgfqpoint{11.697202in}{4.052183in}}{\pgfqpoint{11.701593in}{4.041584in}}{\pgfqpoint{11.709406in}{4.033770in}}%
\pgfpathcurveto{\pgfqpoint{11.717220in}{4.025956in}}{\pgfqpoint{11.727819in}{4.021566in}}{\pgfqpoint{11.738869in}{4.021566in}}%
\pgfpathlineto{\pgfqpoint{11.738869in}{4.021566in}}%
\pgfpathclose%
\pgfusepath{stroke}%
\end{pgfscope}%
\begin{pgfscope}%
\pgfpathrectangle{\pgfqpoint{7.394209in}{0.375000in}}{\pgfqpoint{6.356833in}{5.175000in}}%
\pgfusepath{clip}%
\pgfsetbuttcap%
\pgfsetroundjoin%
\pgfsetlinewidth{1.003750pt}%
\definecolor{currentstroke}{rgb}{0.827451,0.827451,0.827451}%
\pgfsetstrokecolor{currentstroke}%
\pgfsetdash{}{0pt}%
\pgfpathmoveto{\pgfqpoint{11.871079in}{3.979541in}}%
\pgfpathcurveto{\pgfqpoint{11.882129in}{3.979541in}}{\pgfqpoint{11.892728in}{3.983932in}}{\pgfqpoint{11.900542in}{3.991745in}}%
\pgfpathcurveto{\pgfqpoint{11.908355in}{3.999559in}}{\pgfqpoint{11.912745in}{4.010158in}}{\pgfqpoint{11.912745in}{4.021208in}}%
\pgfpathcurveto{\pgfqpoint{11.912745in}{4.032258in}}{\pgfqpoint{11.908355in}{4.042857in}}{\pgfqpoint{11.900542in}{4.050671in}}%
\pgfpathcurveto{\pgfqpoint{11.892728in}{4.058485in}}{\pgfqpoint{11.882129in}{4.062875in}}{\pgfqpoint{11.871079in}{4.062875in}}%
\pgfpathcurveto{\pgfqpoint{11.860029in}{4.062875in}}{\pgfqpoint{11.849430in}{4.058485in}}{\pgfqpoint{11.841616in}{4.050671in}}%
\pgfpathcurveto{\pgfqpoint{11.833802in}{4.042857in}}{\pgfqpoint{11.829412in}{4.032258in}}{\pgfqpoint{11.829412in}{4.021208in}}%
\pgfpathcurveto{\pgfqpoint{11.829412in}{4.010158in}}{\pgfqpoint{11.833802in}{3.999559in}}{\pgfqpoint{11.841616in}{3.991745in}}%
\pgfpathcurveto{\pgfqpoint{11.849430in}{3.983932in}}{\pgfqpoint{11.860029in}{3.979541in}}{\pgfqpoint{11.871079in}{3.979541in}}%
\pgfpathlineto{\pgfqpoint{11.871079in}{3.979541in}}%
\pgfpathclose%
\pgfusepath{stroke}%
\end{pgfscope}%
\begin{pgfscope}%
\pgfpathrectangle{\pgfqpoint{7.394209in}{0.375000in}}{\pgfqpoint{6.356833in}{5.175000in}}%
\pgfusepath{clip}%
\pgfsetbuttcap%
\pgfsetroundjoin%
\pgfsetlinewidth{1.003750pt}%
\definecolor{currentstroke}{rgb}{0.827451,0.827451,0.827451}%
\pgfsetstrokecolor{currentstroke}%
\pgfsetdash{}{0pt}%
\pgfpathmoveto{\pgfqpoint{11.580964in}{3.274635in}}%
\pgfpathcurveto{\pgfqpoint{11.592015in}{3.274635in}}{\pgfqpoint{11.602614in}{3.279026in}}{\pgfqpoint{11.610427in}{3.286839in}}%
\pgfpathcurveto{\pgfqpoint{11.618241in}{3.294653in}}{\pgfqpoint{11.622631in}{3.305252in}}{\pgfqpoint{11.622631in}{3.316302in}}%
\pgfpathcurveto{\pgfqpoint{11.622631in}{3.327352in}}{\pgfqpoint{11.618241in}{3.337951in}}{\pgfqpoint{11.610427in}{3.345765in}}%
\pgfpathcurveto{\pgfqpoint{11.602614in}{3.353578in}}{\pgfqpoint{11.592015in}{3.357969in}}{\pgfqpoint{11.580964in}{3.357969in}}%
\pgfpathcurveto{\pgfqpoint{11.569914in}{3.357969in}}{\pgfqpoint{11.559315in}{3.353578in}}{\pgfqpoint{11.551502in}{3.345765in}}%
\pgfpathcurveto{\pgfqpoint{11.543688in}{3.337951in}}{\pgfqpoint{11.539298in}{3.327352in}}{\pgfqpoint{11.539298in}{3.316302in}}%
\pgfpathcurveto{\pgfqpoint{11.539298in}{3.305252in}}{\pgfqpoint{11.543688in}{3.294653in}}{\pgfqpoint{11.551502in}{3.286839in}}%
\pgfpathcurveto{\pgfqpoint{11.559315in}{3.279026in}}{\pgfqpoint{11.569914in}{3.274635in}}{\pgfqpoint{11.580964in}{3.274635in}}%
\pgfpathlineto{\pgfqpoint{11.580964in}{3.274635in}}%
\pgfpathclose%
\pgfusepath{stroke}%
\end{pgfscope}%
\begin{pgfscope}%
\pgfpathrectangle{\pgfqpoint{7.394209in}{0.375000in}}{\pgfqpoint{6.356833in}{5.175000in}}%
\pgfusepath{clip}%
\pgfsetbuttcap%
\pgfsetroundjoin%
\pgfsetlinewidth{1.003750pt}%
\definecolor{currentstroke}{rgb}{0.827451,0.827451,0.827451}%
\pgfsetstrokecolor{currentstroke}%
\pgfsetdash{}{0pt}%
\pgfpathmoveto{\pgfqpoint{12.328261in}{4.382657in}}%
\pgfpathcurveto{\pgfqpoint{12.339311in}{4.382657in}}{\pgfqpoint{12.349910in}{4.387047in}}{\pgfqpoint{12.357723in}{4.394861in}}%
\pgfpathcurveto{\pgfqpoint{12.365537in}{4.402674in}}{\pgfqpoint{12.369927in}{4.413273in}}{\pgfqpoint{12.369927in}{4.424324in}}%
\pgfpathcurveto{\pgfqpoint{12.369927in}{4.435374in}}{\pgfqpoint{12.365537in}{4.445973in}}{\pgfqpoint{12.357723in}{4.453786in}}%
\pgfpathcurveto{\pgfqpoint{12.349910in}{4.461600in}}{\pgfqpoint{12.339311in}{4.465990in}}{\pgfqpoint{12.328261in}{4.465990in}}%
\pgfpathcurveto{\pgfqpoint{12.317210in}{4.465990in}}{\pgfqpoint{12.306611in}{4.461600in}}{\pgfqpoint{12.298798in}{4.453786in}}%
\pgfpathcurveto{\pgfqpoint{12.290984in}{4.445973in}}{\pgfqpoint{12.286594in}{4.435374in}}{\pgfqpoint{12.286594in}{4.424324in}}%
\pgfpathcurveto{\pgfqpoint{12.286594in}{4.413273in}}{\pgfqpoint{12.290984in}{4.402674in}}{\pgfqpoint{12.298798in}{4.394861in}}%
\pgfpathcurveto{\pgfqpoint{12.306611in}{4.387047in}}{\pgfqpoint{12.317210in}{4.382657in}}{\pgfqpoint{12.328261in}{4.382657in}}%
\pgfpathlineto{\pgfqpoint{12.328261in}{4.382657in}}%
\pgfpathclose%
\pgfusepath{stroke}%
\end{pgfscope}%
\begin{pgfscope}%
\pgfpathrectangle{\pgfqpoint{7.394209in}{0.375000in}}{\pgfqpoint{6.356833in}{5.175000in}}%
\pgfusepath{clip}%
\pgfsetbuttcap%
\pgfsetroundjoin%
\pgfsetlinewidth{1.003750pt}%
\definecolor{currentstroke}{rgb}{0.827451,0.827451,0.827451}%
\pgfsetstrokecolor{currentstroke}%
\pgfsetdash{}{0pt}%
\pgfpathmoveto{\pgfqpoint{13.447157in}{4.938893in}}%
\pgfpathcurveto{\pgfqpoint{13.458207in}{4.938893in}}{\pgfqpoint{13.468806in}{4.943284in}}{\pgfqpoint{13.476619in}{4.951097in}}%
\pgfpathcurveto{\pgfqpoint{13.484433in}{4.958911in}}{\pgfqpoint{13.488823in}{4.969510in}}{\pgfqpoint{13.488823in}{4.980560in}}%
\pgfpathcurveto{\pgfqpoint{13.488823in}{4.991610in}}{\pgfqpoint{13.484433in}{5.002209in}}{\pgfqpoint{13.476619in}{5.010023in}}%
\pgfpathcurveto{\pgfqpoint{13.468806in}{5.017836in}}{\pgfqpoint{13.458207in}{5.022227in}}{\pgfqpoint{13.447157in}{5.022227in}}%
\pgfpathcurveto{\pgfqpoint{13.436106in}{5.022227in}}{\pgfqpoint{13.425507in}{5.017836in}}{\pgfqpoint{13.417694in}{5.010023in}}%
\pgfpathcurveto{\pgfqpoint{13.409880in}{5.002209in}}{\pgfqpoint{13.405490in}{4.991610in}}{\pgfqpoint{13.405490in}{4.980560in}}%
\pgfpathcurveto{\pgfqpoint{13.405490in}{4.969510in}}{\pgfqpoint{13.409880in}{4.958911in}}{\pgfqpoint{13.417694in}{4.951097in}}%
\pgfpathcurveto{\pgfqpoint{13.425507in}{4.943284in}}{\pgfqpoint{13.436106in}{4.938893in}}{\pgfqpoint{13.447157in}{4.938893in}}%
\pgfpathlineto{\pgfqpoint{13.447157in}{4.938893in}}%
\pgfpathclose%
\pgfusepath{stroke}%
\end{pgfscope}%
\begin{pgfscope}%
\pgfpathrectangle{\pgfqpoint{7.394209in}{0.375000in}}{\pgfqpoint{6.356833in}{5.175000in}}%
\pgfusepath{clip}%
\pgfsetbuttcap%
\pgfsetroundjoin%
\pgfsetlinewidth{1.003750pt}%
\definecolor{currentstroke}{rgb}{0.827451,0.827451,0.827451}%
\pgfsetstrokecolor{currentstroke}%
\pgfsetdash{}{0pt}%
\pgfpathmoveto{\pgfqpoint{9.448068in}{4.842949in}}%
\pgfpathcurveto{\pgfqpoint{9.459119in}{4.842949in}}{\pgfqpoint{9.469718in}{4.847339in}}{\pgfqpoint{9.477531in}{4.855153in}}%
\pgfpathcurveto{\pgfqpoint{9.485345in}{4.862966in}}{\pgfqpoint{9.489735in}{4.873565in}}{\pgfqpoint{9.489735in}{4.884615in}}%
\pgfpathcurveto{\pgfqpoint{9.489735in}{4.895665in}}{\pgfqpoint{9.485345in}{4.906264in}}{\pgfqpoint{9.477531in}{4.914078in}}%
\pgfpathcurveto{\pgfqpoint{9.469718in}{4.921892in}}{\pgfqpoint{9.459119in}{4.926282in}}{\pgfqpoint{9.448068in}{4.926282in}}%
\pgfpathcurveto{\pgfqpoint{9.437018in}{4.926282in}}{\pgfqpoint{9.426419in}{4.921892in}}{\pgfqpoint{9.418606in}{4.914078in}}%
\pgfpathcurveto{\pgfqpoint{9.410792in}{4.906264in}}{\pgfqpoint{9.406402in}{4.895665in}}{\pgfqpoint{9.406402in}{4.884615in}}%
\pgfpathcurveto{\pgfqpoint{9.406402in}{4.873565in}}{\pgfqpoint{9.410792in}{4.862966in}}{\pgfqpoint{9.418606in}{4.855153in}}%
\pgfpathcurveto{\pgfqpoint{9.426419in}{4.847339in}}{\pgfqpoint{9.437018in}{4.842949in}}{\pgfqpoint{9.448068in}{4.842949in}}%
\pgfpathlineto{\pgfqpoint{9.448068in}{4.842949in}}%
\pgfpathclose%
\pgfusepath{stroke}%
\end{pgfscope}%
\begin{pgfscope}%
\pgfpathrectangle{\pgfqpoint{7.394209in}{0.375000in}}{\pgfqpoint{6.356833in}{5.175000in}}%
\pgfusepath{clip}%
\pgfsetbuttcap%
\pgfsetroundjoin%
\pgfsetlinewidth{1.003750pt}%
\definecolor{currentstroke}{rgb}{0.827451,0.827451,0.827451}%
\pgfsetstrokecolor{currentstroke}%
\pgfsetdash{}{0pt}%
\pgfpathmoveto{\pgfqpoint{8.851314in}{4.990653in}}%
\pgfpathcurveto{\pgfqpoint{8.862364in}{4.990653in}}{\pgfqpoint{8.872963in}{4.995043in}}{\pgfqpoint{8.880776in}{5.002857in}}%
\pgfpathcurveto{\pgfqpoint{8.888590in}{5.010671in}}{\pgfqpoint{8.892980in}{5.021270in}}{\pgfqpoint{8.892980in}{5.032320in}}%
\pgfpathcurveto{\pgfqpoint{8.892980in}{5.043370in}}{\pgfqpoint{8.888590in}{5.053969in}}{\pgfqpoint{8.880776in}{5.061783in}}%
\pgfpathcurveto{\pgfqpoint{8.872963in}{5.069596in}}{\pgfqpoint{8.862364in}{5.073987in}}{\pgfqpoint{8.851314in}{5.073987in}}%
\pgfpathcurveto{\pgfqpoint{8.840263in}{5.073987in}}{\pgfqpoint{8.829664in}{5.069596in}}{\pgfqpoint{8.821851in}{5.061783in}}%
\pgfpathcurveto{\pgfqpoint{8.814037in}{5.053969in}}{\pgfqpoint{8.809647in}{5.043370in}}{\pgfqpoint{8.809647in}{5.032320in}}%
\pgfpathcurveto{\pgfqpoint{8.809647in}{5.021270in}}{\pgfqpoint{8.814037in}{5.010671in}}{\pgfqpoint{8.821851in}{5.002857in}}%
\pgfpathcurveto{\pgfqpoint{8.829664in}{4.995043in}}{\pgfqpoint{8.840263in}{4.990653in}}{\pgfqpoint{8.851314in}{4.990653in}}%
\pgfpathlineto{\pgfqpoint{8.851314in}{4.990653in}}%
\pgfpathclose%
\pgfusepath{stroke}%
\end{pgfscope}%
\begin{pgfscope}%
\pgfpathrectangle{\pgfqpoint{7.394209in}{0.375000in}}{\pgfqpoint{6.356833in}{5.175000in}}%
\pgfusepath{clip}%
\pgfsetbuttcap%
\pgfsetroundjoin%
\pgfsetlinewidth{1.003750pt}%
\definecolor{currentstroke}{rgb}{0.827451,0.827451,0.827451}%
\pgfsetstrokecolor{currentstroke}%
\pgfsetdash{}{0pt}%
\pgfpathmoveto{\pgfqpoint{10.906162in}{1.746332in}}%
\pgfpathcurveto{\pgfqpoint{10.917212in}{1.746332in}}{\pgfqpoint{10.927811in}{1.750722in}}{\pgfqpoint{10.935625in}{1.758536in}}%
\pgfpathcurveto{\pgfqpoint{10.943439in}{1.766349in}}{\pgfqpoint{10.947829in}{1.776948in}}{\pgfqpoint{10.947829in}{1.787999in}}%
\pgfpathcurveto{\pgfqpoint{10.947829in}{1.799049in}}{\pgfqpoint{10.943439in}{1.809648in}}{\pgfqpoint{10.935625in}{1.817461in}}%
\pgfpathcurveto{\pgfqpoint{10.927811in}{1.825275in}}{\pgfqpoint{10.917212in}{1.829665in}}{\pgfqpoint{10.906162in}{1.829665in}}%
\pgfpathcurveto{\pgfqpoint{10.895112in}{1.829665in}}{\pgfqpoint{10.884513in}{1.825275in}}{\pgfqpoint{10.876699in}{1.817461in}}%
\pgfpathcurveto{\pgfqpoint{10.868886in}{1.809648in}}{\pgfqpoint{10.864495in}{1.799049in}}{\pgfqpoint{10.864495in}{1.787999in}}%
\pgfpathcurveto{\pgfqpoint{10.864495in}{1.776948in}}{\pgfqpoint{10.868886in}{1.766349in}}{\pgfqpoint{10.876699in}{1.758536in}}%
\pgfpathcurveto{\pgfqpoint{10.884513in}{1.750722in}}{\pgfqpoint{10.895112in}{1.746332in}}{\pgfqpoint{10.906162in}{1.746332in}}%
\pgfpathlineto{\pgfqpoint{10.906162in}{1.746332in}}%
\pgfpathclose%
\pgfusepath{stroke}%
\end{pgfscope}%
\begin{pgfscope}%
\pgfpathrectangle{\pgfqpoint{7.394209in}{0.375000in}}{\pgfqpoint{6.356833in}{5.175000in}}%
\pgfusepath{clip}%
\pgfsetbuttcap%
\pgfsetroundjoin%
\pgfsetlinewidth{1.003750pt}%
\definecolor{currentstroke}{rgb}{0.827451,0.827451,0.827451}%
\pgfsetstrokecolor{currentstroke}%
\pgfsetdash{}{0pt}%
\pgfpathmoveto{\pgfqpoint{10.859460in}{2.668573in}}%
\pgfpathcurveto{\pgfqpoint{10.870510in}{2.668573in}}{\pgfqpoint{10.881109in}{2.672963in}}{\pgfqpoint{10.888923in}{2.680777in}}%
\pgfpathcurveto{\pgfqpoint{10.896736in}{2.688590in}}{\pgfqpoint{10.901127in}{2.699189in}}{\pgfqpoint{10.901127in}{2.710240in}}%
\pgfpathcurveto{\pgfqpoint{10.901127in}{2.721290in}}{\pgfqpoint{10.896736in}{2.731889in}}{\pgfqpoint{10.888923in}{2.739702in}}%
\pgfpathcurveto{\pgfqpoint{10.881109in}{2.747516in}}{\pgfqpoint{10.870510in}{2.751906in}}{\pgfqpoint{10.859460in}{2.751906in}}%
\pgfpathcurveto{\pgfqpoint{10.848410in}{2.751906in}}{\pgfqpoint{10.837811in}{2.747516in}}{\pgfqpoint{10.829997in}{2.739702in}}%
\pgfpathcurveto{\pgfqpoint{10.822184in}{2.731889in}}{\pgfqpoint{10.817793in}{2.721290in}}{\pgfqpoint{10.817793in}{2.710240in}}%
\pgfpathcurveto{\pgfqpoint{10.817793in}{2.699189in}}{\pgfqpoint{10.822184in}{2.688590in}}{\pgfqpoint{10.829997in}{2.680777in}}%
\pgfpathcurveto{\pgfqpoint{10.837811in}{2.672963in}}{\pgfqpoint{10.848410in}{2.668573in}}{\pgfqpoint{10.859460in}{2.668573in}}%
\pgfpathlineto{\pgfqpoint{10.859460in}{2.668573in}}%
\pgfpathclose%
\pgfusepath{stroke}%
\end{pgfscope}%
\begin{pgfscope}%
\pgfpathrectangle{\pgfqpoint{7.394209in}{0.375000in}}{\pgfqpoint{6.356833in}{5.175000in}}%
\pgfusepath{clip}%
\pgfsetbuttcap%
\pgfsetroundjoin%
\pgfsetlinewidth{1.003750pt}%
\definecolor{currentstroke}{rgb}{0.827451,0.827451,0.827451}%
\pgfsetstrokecolor{currentstroke}%
\pgfsetdash{}{0pt}%
\pgfpathmoveto{\pgfqpoint{9.019355in}{0.893845in}}%
\pgfpathcurveto{\pgfqpoint{9.030405in}{0.893845in}}{\pgfqpoint{9.041004in}{0.898235in}}{\pgfqpoint{9.048817in}{0.906049in}}%
\pgfpathcurveto{\pgfqpoint{9.056631in}{0.913862in}}{\pgfqpoint{9.061021in}{0.924461in}}{\pgfqpoint{9.061021in}{0.935512in}}%
\pgfpathcurveto{\pgfqpoint{9.061021in}{0.946562in}}{\pgfqpoint{9.056631in}{0.957161in}}{\pgfqpoint{9.048817in}{0.964974in}}%
\pgfpathcurveto{\pgfqpoint{9.041004in}{0.972788in}}{\pgfqpoint{9.030405in}{0.977178in}}{\pgfqpoint{9.019355in}{0.977178in}}%
\pgfpathcurveto{\pgfqpoint{9.008305in}{0.977178in}}{\pgfqpoint{8.997705in}{0.972788in}}{\pgfqpoint{8.989892in}{0.964974in}}%
\pgfpathcurveto{\pgfqpoint{8.982078in}{0.957161in}}{\pgfqpoint{8.977688in}{0.946562in}}{\pgfqpoint{8.977688in}{0.935512in}}%
\pgfpathcurveto{\pgfqpoint{8.977688in}{0.924461in}}{\pgfqpoint{8.982078in}{0.913862in}}{\pgfqpoint{8.989892in}{0.906049in}}%
\pgfpathcurveto{\pgfqpoint{8.997705in}{0.898235in}}{\pgfqpoint{9.008305in}{0.893845in}}{\pgfqpoint{9.019355in}{0.893845in}}%
\pgfpathlineto{\pgfqpoint{9.019355in}{0.893845in}}%
\pgfpathclose%
\pgfusepath{stroke}%
\end{pgfscope}%
\begin{pgfscope}%
\pgfpathrectangle{\pgfqpoint{7.394209in}{0.375000in}}{\pgfqpoint{6.356833in}{5.175000in}}%
\pgfusepath{clip}%
\pgfsetbuttcap%
\pgfsetroundjoin%
\pgfsetlinewidth{1.003750pt}%
\definecolor{currentstroke}{rgb}{0.827451,0.827451,0.827451}%
\pgfsetstrokecolor{currentstroke}%
\pgfsetdash{}{0pt}%
\pgfpathmoveto{\pgfqpoint{11.422698in}{2.601587in}}%
\pgfpathcurveto{\pgfqpoint{11.433748in}{2.601587in}}{\pgfqpoint{11.444347in}{2.605977in}}{\pgfqpoint{11.452161in}{2.613791in}}%
\pgfpathcurveto{\pgfqpoint{11.459975in}{2.621604in}}{\pgfqpoint{11.464365in}{2.632203in}}{\pgfqpoint{11.464365in}{2.643253in}}%
\pgfpathcurveto{\pgfqpoint{11.464365in}{2.654304in}}{\pgfqpoint{11.459975in}{2.664903in}}{\pgfqpoint{11.452161in}{2.672716in}}%
\pgfpathcurveto{\pgfqpoint{11.444347in}{2.680530in}}{\pgfqpoint{11.433748in}{2.684920in}}{\pgfqpoint{11.422698in}{2.684920in}}%
\pgfpathcurveto{\pgfqpoint{11.411648in}{2.684920in}}{\pgfqpoint{11.401049in}{2.680530in}}{\pgfqpoint{11.393235in}{2.672716in}}%
\pgfpathcurveto{\pgfqpoint{11.385422in}{2.664903in}}{\pgfqpoint{11.381032in}{2.654304in}}{\pgfqpoint{11.381032in}{2.643253in}}%
\pgfpathcurveto{\pgfqpoint{11.381032in}{2.632203in}}{\pgfqpoint{11.385422in}{2.621604in}}{\pgfqpoint{11.393235in}{2.613791in}}%
\pgfpathcurveto{\pgfqpoint{11.401049in}{2.605977in}}{\pgfqpoint{11.411648in}{2.601587in}}{\pgfqpoint{11.422698in}{2.601587in}}%
\pgfpathlineto{\pgfqpoint{11.422698in}{2.601587in}}%
\pgfpathclose%
\pgfusepath{stroke}%
\end{pgfscope}%
\begin{pgfscope}%
\pgfpathrectangle{\pgfqpoint{7.394209in}{0.375000in}}{\pgfqpoint{6.356833in}{5.175000in}}%
\pgfusepath{clip}%
\pgfsetbuttcap%
\pgfsetroundjoin%
\pgfsetlinewidth{1.003750pt}%
\definecolor{currentstroke}{rgb}{0.827451,0.827451,0.827451}%
\pgfsetstrokecolor{currentstroke}%
\pgfsetdash{}{0pt}%
\pgfpathmoveto{\pgfqpoint{11.353096in}{2.888141in}}%
\pgfpathcurveto{\pgfqpoint{11.364146in}{2.888141in}}{\pgfqpoint{11.374745in}{2.892531in}}{\pgfqpoint{11.382559in}{2.900345in}}%
\pgfpathcurveto{\pgfqpoint{11.390373in}{2.908158in}}{\pgfqpoint{11.394763in}{2.918757in}}{\pgfqpoint{11.394763in}{2.929807in}}%
\pgfpathcurveto{\pgfqpoint{11.394763in}{2.940858in}}{\pgfqpoint{11.390373in}{2.951457in}}{\pgfqpoint{11.382559in}{2.959270in}}%
\pgfpathcurveto{\pgfqpoint{11.374745in}{2.967084in}}{\pgfqpoint{11.364146in}{2.971474in}}{\pgfqpoint{11.353096in}{2.971474in}}%
\pgfpathcurveto{\pgfqpoint{11.342046in}{2.971474in}}{\pgfqpoint{11.331447in}{2.967084in}}{\pgfqpoint{11.323633in}{2.959270in}}%
\pgfpathcurveto{\pgfqpoint{11.315820in}{2.951457in}}{\pgfqpoint{11.311429in}{2.940858in}}{\pgfqpoint{11.311429in}{2.929807in}}%
\pgfpathcurveto{\pgfqpoint{11.311429in}{2.918757in}}{\pgfqpoint{11.315820in}{2.908158in}}{\pgfqpoint{11.323633in}{2.900345in}}%
\pgfpathcurveto{\pgfqpoint{11.331447in}{2.892531in}}{\pgfqpoint{11.342046in}{2.888141in}}{\pgfqpoint{11.353096in}{2.888141in}}%
\pgfpathlineto{\pgfqpoint{11.353096in}{2.888141in}}%
\pgfpathclose%
\pgfusepath{stroke}%
\end{pgfscope}%
\begin{pgfscope}%
\pgfpathrectangle{\pgfqpoint{7.394209in}{0.375000in}}{\pgfqpoint{6.356833in}{5.175000in}}%
\pgfusepath{clip}%
\pgfsetbuttcap%
\pgfsetroundjoin%
\pgfsetlinewidth{1.003750pt}%
\definecolor{currentstroke}{rgb}{0.827451,0.827451,0.827451}%
\pgfsetstrokecolor{currentstroke}%
\pgfsetdash{}{0pt}%
\pgfpathmoveto{\pgfqpoint{9.824538in}{0.612129in}}%
\pgfpathcurveto{\pgfqpoint{9.835588in}{0.612129in}}{\pgfqpoint{9.846187in}{0.616520in}}{\pgfqpoint{9.854000in}{0.624333in}}%
\pgfpathcurveto{\pgfqpoint{9.861814in}{0.632147in}}{\pgfqpoint{9.866204in}{0.642746in}}{\pgfqpoint{9.866204in}{0.653796in}}%
\pgfpathcurveto{\pgfqpoint{9.866204in}{0.664846in}}{\pgfqpoint{9.861814in}{0.675445in}}{\pgfqpoint{9.854000in}{0.683259in}}%
\pgfpathcurveto{\pgfqpoint{9.846187in}{0.691072in}}{\pgfqpoint{9.835588in}{0.695463in}}{\pgfqpoint{9.824538in}{0.695463in}}%
\pgfpathcurveto{\pgfqpoint{9.813487in}{0.695463in}}{\pgfqpoint{9.802888in}{0.691072in}}{\pgfqpoint{9.795075in}{0.683259in}}%
\pgfpathcurveto{\pgfqpoint{9.787261in}{0.675445in}}{\pgfqpoint{9.782871in}{0.664846in}}{\pgfqpoint{9.782871in}{0.653796in}}%
\pgfpathcurveto{\pgfqpoint{9.782871in}{0.642746in}}{\pgfqpoint{9.787261in}{0.632147in}}{\pgfqpoint{9.795075in}{0.624333in}}%
\pgfpathcurveto{\pgfqpoint{9.802888in}{0.616520in}}{\pgfqpoint{9.813487in}{0.612129in}}{\pgfqpoint{9.824538in}{0.612129in}}%
\pgfpathlineto{\pgfqpoint{9.824538in}{0.612129in}}%
\pgfpathclose%
\pgfusepath{stroke}%
\end{pgfscope}%
\begin{pgfscope}%
\pgfpathrectangle{\pgfqpoint{7.394209in}{0.375000in}}{\pgfqpoint{6.356833in}{5.175000in}}%
\pgfusepath{clip}%
\pgfsetbuttcap%
\pgfsetroundjoin%
\pgfsetlinewidth{1.003750pt}%
\definecolor{currentstroke}{rgb}{0.827451,0.827451,0.827451}%
\pgfsetstrokecolor{currentstroke}%
\pgfsetdash{}{0pt}%
\pgfpathmoveto{\pgfqpoint{12.088877in}{3.372629in}}%
\pgfpathcurveto{\pgfqpoint{12.099927in}{3.372629in}}{\pgfqpoint{12.110526in}{3.377020in}}{\pgfqpoint{12.118340in}{3.384833in}}%
\pgfpathcurveto{\pgfqpoint{12.126153in}{3.392647in}}{\pgfqpoint{12.130544in}{3.403246in}}{\pgfqpoint{12.130544in}{3.414296in}}%
\pgfpathcurveto{\pgfqpoint{12.130544in}{3.425346in}}{\pgfqpoint{12.126153in}{3.435945in}}{\pgfqpoint{12.118340in}{3.443759in}}%
\pgfpathcurveto{\pgfqpoint{12.110526in}{3.451573in}}{\pgfqpoint{12.099927in}{3.455963in}}{\pgfqpoint{12.088877in}{3.455963in}}%
\pgfpathcurveto{\pgfqpoint{12.077827in}{3.455963in}}{\pgfqpoint{12.067228in}{3.451573in}}{\pgfqpoint{12.059414in}{3.443759in}}%
\pgfpathcurveto{\pgfqpoint{12.051601in}{3.435945in}}{\pgfqpoint{12.047210in}{3.425346in}}{\pgfqpoint{12.047210in}{3.414296in}}%
\pgfpathcurveto{\pgfqpoint{12.047210in}{3.403246in}}{\pgfqpoint{12.051601in}{3.392647in}}{\pgfqpoint{12.059414in}{3.384833in}}%
\pgfpathcurveto{\pgfqpoint{12.067228in}{3.377020in}}{\pgfqpoint{12.077827in}{3.372629in}}{\pgfqpoint{12.088877in}{3.372629in}}%
\pgfpathlineto{\pgfqpoint{12.088877in}{3.372629in}}%
\pgfpathclose%
\pgfusepath{stroke}%
\end{pgfscope}%
\begin{pgfscope}%
\pgfpathrectangle{\pgfqpoint{7.394209in}{0.375000in}}{\pgfqpoint{6.356833in}{5.175000in}}%
\pgfusepath{clip}%
\pgfsetbuttcap%
\pgfsetroundjoin%
\pgfsetlinewidth{1.003750pt}%
\definecolor{currentstroke}{rgb}{0.827451,0.827451,0.827451}%
\pgfsetstrokecolor{currentstroke}%
\pgfsetdash{}{0pt}%
\pgfpathmoveto{\pgfqpoint{12.347223in}{3.433959in}}%
\pgfpathcurveto{\pgfqpoint{12.358274in}{3.433959in}}{\pgfqpoint{12.368873in}{3.438349in}}{\pgfqpoint{12.376686in}{3.446163in}}%
\pgfpathcurveto{\pgfqpoint{12.384500in}{3.453977in}}{\pgfqpoint{12.388890in}{3.464576in}}{\pgfqpoint{12.388890in}{3.475626in}}%
\pgfpathcurveto{\pgfqpoint{12.388890in}{3.486676in}}{\pgfqpoint{12.384500in}{3.497275in}}{\pgfqpoint{12.376686in}{3.505089in}}%
\pgfpathcurveto{\pgfqpoint{12.368873in}{3.512902in}}{\pgfqpoint{12.358274in}{3.517292in}}{\pgfqpoint{12.347223in}{3.517292in}}%
\pgfpathcurveto{\pgfqpoint{12.336173in}{3.517292in}}{\pgfqpoint{12.325574in}{3.512902in}}{\pgfqpoint{12.317761in}{3.505089in}}%
\pgfpathcurveto{\pgfqpoint{12.309947in}{3.497275in}}{\pgfqpoint{12.305557in}{3.486676in}}{\pgfqpoint{12.305557in}{3.475626in}}%
\pgfpathcurveto{\pgfqpoint{12.305557in}{3.464576in}}{\pgfqpoint{12.309947in}{3.453977in}}{\pgfqpoint{12.317761in}{3.446163in}}%
\pgfpathcurveto{\pgfqpoint{12.325574in}{3.438349in}}{\pgfqpoint{12.336173in}{3.433959in}}{\pgfqpoint{12.347223in}{3.433959in}}%
\pgfpathlineto{\pgfqpoint{12.347223in}{3.433959in}}%
\pgfpathclose%
\pgfusepath{stroke}%
\end{pgfscope}%
\begin{pgfscope}%
\pgfpathrectangle{\pgfqpoint{7.394209in}{0.375000in}}{\pgfqpoint{6.356833in}{5.175000in}}%
\pgfusepath{clip}%
\pgfsetbuttcap%
\pgfsetroundjoin%
\pgfsetlinewidth{1.003750pt}%
\definecolor{currentstroke}{rgb}{0.827451,0.827451,0.827451}%
\pgfsetstrokecolor{currentstroke}%
\pgfsetdash{}{0pt}%
\pgfpathmoveto{\pgfqpoint{12.737367in}{4.018435in}}%
\pgfpathcurveto{\pgfqpoint{12.748417in}{4.018435in}}{\pgfqpoint{12.759016in}{4.022825in}}{\pgfqpoint{12.766830in}{4.030639in}}%
\pgfpathcurveto{\pgfqpoint{12.774644in}{4.038453in}}{\pgfqpoint{12.779034in}{4.049052in}}{\pgfqpoint{12.779034in}{4.060102in}}%
\pgfpathcurveto{\pgfqpoint{12.779034in}{4.071152in}}{\pgfqpoint{12.774644in}{4.081751in}}{\pgfqpoint{12.766830in}{4.089565in}}%
\pgfpathcurveto{\pgfqpoint{12.759016in}{4.097378in}}{\pgfqpoint{12.748417in}{4.101768in}}{\pgfqpoint{12.737367in}{4.101768in}}%
\pgfpathcurveto{\pgfqpoint{12.726317in}{4.101768in}}{\pgfqpoint{12.715718in}{4.097378in}}{\pgfqpoint{12.707904in}{4.089565in}}%
\pgfpathcurveto{\pgfqpoint{12.700091in}{4.081751in}}{\pgfqpoint{12.695700in}{4.071152in}}{\pgfqpoint{12.695700in}{4.060102in}}%
\pgfpathcurveto{\pgfqpoint{12.695700in}{4.049052in}}{\pgfqpoint{12.700091in}{4.038453in}}{\pgfqpoint{12.707904in}{4.030639in}}%
\pgfpathcurveto{\pgfqpoint{12.715718in}{4.022825in}}{\pgfqpoint{12.726317in}{4.018435in}}{\pgfqpoint{12.737367in}{4.018435in}}%
\pgfpathlineto{\pgfqpoint{12.737367in}{4.018435in}}%
\pgfpathclose%
\pgfusepath{stroke}%
\end{pgfscope}%
\begin{pgfscope}%
\pgfpathrectangle{\pgfqpoint{7.394209in}{0.375000in}}{\pgfqpoint{6.356833in}{5.175000in}}%
\pgfusepath{clip}%
\pgfsetbuttcap%
\pgfsetroundjoin%
\pgfsetlinewidth{1.003750pt}%
\definecolor{currentstroke}{rgb}{0.827451,0.827451,0.827451}%
\pgfsetstrokecolor{currentstroke}%
\pgfsetdash{}{0pt}%
\pgfpathmoveto{\pgfqpoint{8.764959in}{4.136675in}}%
\pgfpathcurveto{\pgfqpoint{8.776009in}{4.136675in}}{\pgfqpoint{8.786608in}{4.141065in}}{\pgfqpoint{8.794422in}{4.148878in}}%
\pgfpathcurveto{\pgfqpoint{8.802236in}{4.156692in}}{\pgfqpoint{8.806626in}{4.167291in}}{\pgfqpoint{8.806626in}{4.178341in}}%
\pgfpathcurveto{\pgfqpoint{8.806626in}{4.189391in}}{\pgfqpoint{8.802236in}{4.199990in}}{\pgfqpoint{8.794422in}{4.207804in}}%
\pgfpathcurveto{\pgfqpoint{8.786608in}{4.215618in}}{\pgfqpoint{8.776009in}{4.220008in}}{\pgfqpoint{8.764959in}{4.220008in}}%
\pgfpathcurveto{\pgfqpoint{8.753909in}{4.220008in}}{\pgfqpoint{8.743310in}{4.215618in}}{\pgfqpoint{8.735496in}{4.207804in}}%
\pgfpathcurveto{\pgfqpoint{8.727683in}{4.199990in}}{\pgfqpoint{8.723292in}{4.189391in}}{\pgfqpoint{8.723292in}{4.178341in}}%
\pgfpathcurveto{\pgfqpoint{8.723292in}{4.167291in}}{\pgfqpoint{8.727683in}{4.156692in}}{\pgfqpoint{8.735496in}{4.148878in}}%
\pgfpathcurveto{\pgfqpoint{8.743310in}{4.141065in}}{\pgfqpoint{8.753909in}{4.136675in}}{\pgfqpoint{8.764959in}{4.136675in}}%
\pgfpathlineto{\pgfqpoint{8.764959in}{4.136675in}}%
\pgfpathclose%
\pgfusepath{stroke}%
\end{pgfscope}%
\begin{pgfscope}%
\pgfpathrectangle{\pgfqpoint{7.394209in}{0.375000in}}{\pgfqpoint{6.356833in}{5.175000in}}%
\pgfusepath{clip}%
\pgfsetbuttcap%
\pgfsetroundjoin%
\pgfsetlinewidth{1.003750pt}%
\definecolor{currentstroke}{rgb}{0.827451,0.827451,0.827451}%
\pgfsetstrokecolor{currentstroke}%
\pgfsetdash{}{0pt}%
\pgfpathmoveto{\pgfqpoint{10.417307in}{2.115691in}}%
\pgfpathcurveto{\pgfqpoint{10.428357in}{2.115691in}}{\pgfqpoint{10.438956in}{2.120081in}}{\pgfqpoint{10.446769in}{2.127895in}}%
\pgfpathcurveto{\pgfqpoint{10.454583in}{2.135708in}}{\pgfqpoint{10.458973in}{2.146307in}}{\pgfqpoint{10.458973in}{2.157358in}}%
\pgfpathcurveto{\pgfqpoint{10.458973in}{2.168408in}}{\pgfqpoint{10.454583in}{2.179007in}}{\pgfqpoint{10.446769in}{2.186820in}}%
\pgfpathcurveto{\pgfqpoint{10.438956in}{2.194634in}}{\pgfqpoint{10.428357in}{2.199024in}}{\pgfqpoint{10.417307in}{2.199024in}}%
\pgfpathcurveto{\pgfqpoint{10.406257in}{2.199024in}}{\pgfqpoint{10.395657in}{2.194634in}}{\pgfqpoint{10.387844in}{2.186820in}}%
\pgfpathcurveto{\pgfqpoint{10.380030in}{2.179007in}}{\pgfqpoint{10.375640in}{2.168408in}}{\pgfqpoint{10.375640in}{2.157358in}}%
\pgfpathcurveto{\pgfqpoint{10.375640in}{2.146307in}}{\pgfqpoint{10.380030in}{2.135708in}}{\pgfqpoint{10.387844in}{2.127895in}}%
\pgfpathcurveto{\pgfqpoint{10.395657in}{2.120081in}}{\pgfqpoint{10.406257in}{2.115691in}}{\pgfqpoint{10.417307in}{2.115691in}}%
\pgfpathlineto{\pgfqpoint{10.417307in}{2.115691in}}%
\pgfpathclose%
\pgfusepath{stroke}%
\end{pgfscope}%
\begin{pgfscope}%
\pgfpathrectangle{\pgfqpoint{7.394209in}{0.375000in}}{\pgfqpoint{6.356833in}{5.175000in}}%
\pgfusepath{clip}%
\pgfsetbuttcap%
\pgfsetroundjoin%
\pgfsetlinewidth{1.003750pt}%
\definecolor{currentstroke}{rgb}{0.827451,0.827451,0.827451}%
\pgfsetstrokecolor{currentstroke}%
\pgfsetdash{}{0pt}%
\pgfpathmoveto{\pgfqpoint{8.484185in}{4.147756in}}%
\pgfpathcurveto{\pgfqpoint{8.495236in}{4.147756in}}{\pgfqpoint{8.505835in}{4.152146in}}{\pgfqpoint{8.513648in}{4.159960in}}%
\pgfpathcurveto{\pgfqpoint{8.521462in}{4.167773in}}{\pgfqpoint{8.525852in}{4.178372in}}{\pgfqpoint{8.525852in}{4.189422in}}%
\pgfpathcurveto{\pgfqpoint{8.525852in}{4.200473in}}{\pgfqpoint{8.521462in}{4.211072in}}{\pgfqpoint{8.513648in}{4.218885in}}%
\pgfpathcurveto{\pgfqpoint{8.505835in}{4.226699in}}{\pgfqpoint{8.495236in}{4.231089in}}{\pgfqpoint{8.484185in}{4.231089in}}%
\pgfpathcurveto{\pgfqpoint{8.473135in}{4.231089in}}{\pgfqpoint{8.462536in}{4.226699in}}{\pgfqpoint{8.454723in}{4.218885in}}%
\pgfpathcurveto{\pgfqpoint{8.446909in}{4.211072in}}{\pgfqpoint{8.442519in}{4.200473in}}{\pgfqpoint{8.442519in}{4.189422in}}%
\pgfpathcurveto{\pgfqpoint{8.442519in}{4.178372in}}{\pgfqpoint{8.446909in}{4.167773in}}{\pgfqpoint{8.454723in}{4.159960in}}%
\pgfpathcurveto{\pgfqpoint{8.462536in}{4.152146in}}{\pgfqpoint{8.473135in}{4.147756in}}{\pgfqpoint{8.484185in}{4.147756in}}%
\pgfpathlineto{\pgfqpoint{8.484185in}{4.147756in}}%
\pgfpathclose%
\pgfusepath{stroke}%
\end{pgfscope}%
\begin{pgfscope}%
\pgfpathrectangle{\pgfqpoint{7.394209in}{0.375000in}}{\pgfqpoint{6.356833in}{5.175000in}}%
\pgfusepath{clip}%
\pgfsetbuttcap%
\pgfsetroundjoin%
\pgfsetlinewidth{1.003750pt}%
\definecolor{currentstroke}{rgb}{0.827451,0.827451,0.827451}%
\pgfsetstrokecolor{currentstroke}%
\pgfsetdash{}{0pt}%
\pgfpathmoveto{\pgfqpoint{13.484941in}{4.866285in}}%
\pgfpathcurveto{\pgfqpoint{13.495991in}{4.866285in}}{\pgfqpoint{13.506590in}{4.870675in}}{\pgfqpoint{13.514404in}{4.878489in}}%
\pgfpathcurveto{\pgfqpoint{13.522218in}{4.886302in}}{\pgfqpoint{13.526608in}{4.896901in}}{\pgfqpoint{13.526608in}{4.907951in}}%
\pgfpathcurveto{\pgfqpoint{13.526608in}{4.919001in}}{\pgfqpoint{13.522218in}{4.929600in}}{\pgfqpoint{13.514404in}{4.937414in}}%
\pgfpathcurveto{\pgfqpoint{13.506590in}{4.945228in}}{\pgfqpoint{13.495991in}{4.949618in}}{\pgfqpoint{13.484941in}{4.949618in}}%
\pgfpathcurveto{\pgfqpoint{13.473891in}{4.949618in}}{\pgfqpoint{13.463292in}{4.945228in}}{\pgfqpoint{13.455478in}{4.937414in}}%
\pgfpathcurveto{\pgfqpoint{13.447665in}{4.929600in}}{\pgfqpoint{13.443274in}{4.919001in}}{\pgfqpoint{13.443274in}{4.907951in}}%
\pgfpathcurveto{\pgfqpoint{13.443274in}{4.896901in}}{\pgfqpoint{13.447665in}{4.886302in}}{\pgfqpoint{13.455478in}{4.878489in}}%
\pgfpathcurveto{\pgfqpoint{13.463292in}{4.870675in}}{\pgfqpoint{13.473891in}{4.866285in}}{\pgfqpoint{13.484941in}{4.866285in}}%
\pgfpathlineto{\pgfqpoint{13.484941in}{4.866285in}}%
\pgfpathclose%
\pgfusepath{stroke}%
\end{pgfscope}%
\begin{pgfscope}%
\pgfpathrectangle{\pgfqpoint{7.394209in}{0.375000in}}{\pgfqpoint{6.356833in}{5.175000in}}%
\pgfusepath{clip}%
\pgfsetbuttcap%
\pgfsetroundjoin%
\pgfsetlinewidth{1.003750pt}%
\definecolor{currentstroke}{rgb}{0.827451,0.827451,0.827451}%
\pgfsetstrokecolor{currentstroke}%
\pgfsetdash{}{0pt}%
\pgfpathmoveto{\pgfqpoint{8.785034in}{5.119523in}}%
\pgfpathcurveto{\pgfqpoint{8.796084in}{5.119523in}}{\pgfqpoint{8.806683in}{5.123914in}}{\pgfqpoint{8.814497in}{5.131727in}}%
\pgfpathcurveto{\pgfqpoint{8.822311in}{5.139541in}}{\pgfqpoint{8.826701in}{5.150140in}}{\pgfqpoint{8.826701in}{5.161190in}}%
\pgfpathcurveto{\pgfqpoint{8.826701in}{5.172240in}}{\pgfqpoint{8.822311in}{5.182839in}}{\pgfqpoint{8.814497in}{5.190653in}}%
\pgfpathcurveto{\pgfqpoint{8.806683in}{5.198466in}}{\pgfqpoint{8.796084in}{5.202857in}}{\pgfqpoint{8.785034in}{5.202857in}}%
\pgfpathcurveto{\pgfqpoint{8.773984in}{5.202857in}}{\pgfqpoint{8.763385in}{5.198466in}}{\pgfqpoint{8.755572in}{5.190653in}}%
\pgfpathcurveto{\pgfqpoint{8.747758in}{5.182839in}}{\pgfqpoint{8.743368in}{5.172240in}}{\pgfqpoint{8.743368in}{5.161190in}}%
\pgfpathcurveto{\pgfqpoint{8.743368in}{5.150140in}}{\pgfqpoint{8.747758in}{5.139541in}}{\pgfqpoint{8.755572in}{5.131727in}}%
\pgfpathcurveto{\pgfqpoint{8.763385in}{5.123914in}}{\pgfqpoint{8.773984in}{5.119523in}}{\pgfqpoint{8.785034in}{5.119523in}}%
\pgfpathlineto{\pgfqpoint{8.785034in}{5.119523in}}%
\pgfpathclose%
\pgfusepath{stroke}%
\end{pgfscope}%
\begin{pgfscope}%
\pgfpathrectangle{\pgfqpoint{7.394209in}{0.375000in}}{\pgfqpoint{6.356833in}{5.175000in}}%
\pgfusepath{clip}%
\pgfsetbuttcap%
\pgfsetroundjoin%
\pgfsetlinewidth{1.003750pt}%
\definecolor{currentstroke}{rgb}{0.827451,0.827451,0.827451}%
\pgfsetstrokecolor{currentstroke}%
\pgfsetdash{}{0pt}%
\pgfpathmoveto{\pgfqpoint{11.166426in}{2.227769in}}%
\pgfpathcurveto{\pgfqpoint{11.177476in}{2.227769in}}{\pgfqpoint{11.188075in}{2.232159in}}{\pgfqpoint{11.195889in}{2.239972in}}%
\pgfpathcurveto{\pgfqpoint{11.203702in}{2.247786in}}{\pgfqpoint{11.208093in}{2.258385in}}{\pgfqpoint{11.208093in}{2.269435in}}%
\pgfpathcurveto{\pgfqpoint{11.208093in}{2.280485in}}{\pgfqpoint{11.203702in}{2.291084in}}{\pgfqpoint{11.195889in}{2.298898in}}%
\pgfpathcurveto{\pgfqpoint{11.188075in}{2.306712in}}{\pgfqpoint{11.177476in}{2.311102in}}{\pgfqpoint{11.166426in}{2.311102in}}%
\pgfpathcurveto{\pgfqpoint{11.155376in}{2.311102in}}{\pgfqpoint{11.144777in}{2.306712in}}{\pgfqpoint{11.136963in}{2.298898in}}%
\pgfpathcurveto{\pgfqpoint{11.129150in}{2.291084in}}{\pgfqpoint{11.124759in}{2.280485in}}{\pgfqpoint{11.124759in}{2.269435in}}%
\pgfpathcurveto{\pgfqpoint{11.124759in}{2.258385in}}{\pgfqpoint{11.129150in}{2.247786in}}{\pgfqpoint{11.136963in}{2.239972in}}%
\pgfpathcurveto{\pgfqpoint{11.144777in}{2.232159in}}{\pgfqpoint{11.155376in}{2.227769in}}{\pgfqpoint{11.166426in}{2.227769in}}%
\pgfpathlineto{\pgfqpoint{11.166426in}{2.227769in}}%
\pgfpathclose%
\pgfusepath{stroke}%
\end{pgfscope}%
\begin{pgfscope}%
\pgfpathrectangle{\pgfqpoint{7.394209in}{0.375000in}}{\pgfqpoint{6.356833in}{5.175000in}}%
\pgfusepath{clip}%
\pgfsetbuttcap%
\pgfsetroundjoin%
\pgfsetlinewidth{1.003750pt}%
\definecolor{currentstroke}{rgb}{0.827451,0.827451,0.827451}%
\pgfsetstrokecolor{currentstroke}%
\pgfsetdash{}{0pt}%
\pgfpathmoveto{\pgfqpoint{10.433372in}{1.345952in}}%
\pgfpathcurveto{\pgfqpoint{10.444422in}{1.345952in}}{\pgfqpoint{10.455021in}{1.350342in}}{\pgfqpoint{10.462835in}{1.358156in}}%
\pgfpathcurveto{\pgfqpoint{10.470648in}{1.365970in}}{\pgfqpoint{10.475038in}{1.376569in}}{\pgfqpoint{10.475038in}{1.387619in}}%
\pgfpathcurveto{\pgfqpoint{10.475038in}{1.398669in}}{\pgfqpoint{10.470648in}{1.409268in}}{\pgfqpoint{10.462835in}{1.417081in}}%
\pgfpathcurveto{\pgfqpoint{10.455021in}{1.424895in}}{\pgfqpoint{10.444422in}{1.429285in}}{\pgfqpoint{10.433372in}{1.429285in}}%
\pgfpathcurveto{\pgfqpoint{10.422322in}{1.429285in}}{\pgfqpoint{10.411723in}{1.424895in}}{\pgfqpoint{10.403909in}{1.417081in}}%
\pgfpathcurveto{\pgfqpoint{10.396095in}{1.409268in}}{\pgfqpoint{10.391705in}{1.398669in}}{\pgfqpoint{10.391705in}{1.387619in}}%
\pgfpathcurveto{\pgfqpoint{10.391705in}{1.376569in}}{\pgfqpoint{10.396095in}{1.365970in}}{\pgfqpoint{10.403909in}{1.358156in}}%
\pgfpathcurveto{\pgfqpoint{10.411723in}{1.350342in}}{\pgfqpoint{10.422322in}{1.345952in}}{\pgfqpoint{10.433372in}{1.345952in}}%
\pgfpathlineto{\pgfqpoint{10.433372in}{1.345952in}}%
\pgfpathclose%
\pgfusepath{stroke}%
\end{pgfscope}%
\begin{pgfscope}%
\pgfpathrectangle{\pgfqpoint{7.394209in}{0.375000in}}{\pgfqpoint{6.356833in}{5.175000in}}%
\pgfusepath{clip}%
\pgfsetbuttcap%
\pgfsetroundjoin%
\pgfsetlinewidth{1.003750pt}%
\definecolor{currentstroke}{rgb}{0.827451,0.827451,0.827451}%
\pgfsetstrokecolor{currentstroke}%
\pgfsetdash{}{0pt}%
\pgfpathmoveto{\pgfqpoint{7.433676in}{1.889154in}}%
\pgfpathcurveto{\pgfqpoint{7.444726in}{1.889154in}}{\pgfqpoint{7.455325in}{1.893544in}}{\pgfqpoint{7.463139in}{1.901358in}}%
\pgfpathcurveto{\pgfqpoint{7.470952in}{1.909171in}}{\pgfqpoint{7.475343in}{1.919770in}}{\pgfqpoint{7.475343in}{1.930821in}}%
\pgfpathcurveto{\pgfqpoint{7.475343in}{1.941871in}}{\pgfqpoint{7.470952in}{1.952470in}}{\pgfqpoint{7.463139in}{1.960283in}}%
\pgfpathcurveto{\pgfqpoint{7.455325in}{1.968097in}}{\pgfqpoint{7.444726in}{1.972487in}}{\pgfqpoint{7.433676in}{1.972487in}}%
\pgfpathcurveto{\pgfqpoint{7.422626in}{1.972487in}}{\pgfqpoint{7.412027in}{1.968097in}}{\pgfqpoint{7.404213in}{1.960283in}}%
\pgfpathcurveto{\pgfqpoint{7.396400in}{1.952470in}}{\pgfqpoint{7.392009in}{1.941871in}}{\pgfqpoint{7.392009in}{1.930821in}}%
\pgfpathcurveto{\pgfqpoint{7.392009in}{1.919770in}}{\pgfqpoint{7.396400in}{1.909171in}}{\pgfqpoint{7.404213in}{1.901358in}}%
\pgfpathcurveto{\pgfqpoint{7.412027in}{1.893544in}}{\pgfqpoint{7.422626in}{1.889154in}}{\pgfqpoint{7.433676in}{1.889154in}}%
\pgfpathlineto{\pgfqpoint{7.433676in}{1.889154in}}%
\pgfpathclose%
\pgfusepath{stroke}%
\end{pgfscope}%
\begin{pgfscope}%
\pgfpathrectangle{\pgfqpoint{7.394209in}{0.375000in}}{\pgfqpoint{6.356833in}{5.175000in}}%
\pgfusepath{clip}%
\pgfsetbuttcap%
\pgfsetroundjoin%
\pgfsetlinewidth{1.003750pt}%
\definecolor{currentstroke}{rgb}{0.827451,0.827451,0.827451}%
\pgfsetstrokecolor{currentstroke}%
\pgfsetdash{}{0pt}%
\pgfpathmoveto{\pgfqpoint{7.669613in}{3.238749in}}%
\pgfpathcurveto{\pgfqpoint{7.680663in}{3.238749in}}{\pgfqpoint{7.691262in}{3.243140in}}{\pgfqpoint{7.699076in}{3.250953in}}%
\pgfpathcurveto{\pgfqpoint{7.706890in}{3.258767in}}{\pgfqpoint{7.711280in}{3.269366in}}{\pgfqpoint{7.711280in}{3.280416in}}%
\pgfpathcurveto{\pgfqpoint{7.711280in}{3.291466in}}{\pgfqpoint{7.706890in}{3.302065in}}{\pgfqpoint{7.699076in}{3.309879in}}%
\pgfpathcurveto{\pgfqpoint{7.691262in}{3.317692in}}{\pgfqpoint{7.680663in}{3.322083in}}{\pgfqpoint{7.669613in}{3.322083in}}%
\pgfpathcurveto{\pgfqpoint{7.658563in}{3.322083in}}{\pgfqpoint{7.647964in}{3.317692in}}{\pgfqpoint{7.640150in}{3.309879in}}%
\pgfpathcurveto{\pgfqpoint{7.632337in}{3.302065in}}{\pgfqpoint{7.627947in}{3.291466in}}{\pgfqpoint{7.627947in}{3.280416in}}%
\pgfpathcurveto{\pgfqpoint{7.627947in}{3.269366in}}{\pgfqpoint{7.632337in}{3.258767in}}{\pgfqpoint{7.640150in}{3.250953in}}%
\pgfpathcurveto{\pgfqpoint{7.647964in}{3.243140in}}{\pgfqpoint{7.658563in}{3.238749in}}{\pgfqpoint{7.669613in}{3.238749in}}%
\pgfpathlineto{\pgfqpoint{7.669613in}{3.238749in}}%
\pgfpathclose%
\pgfusepath{stroke}%
\end{pgfscope}%
\begin{pgfscope}%
\pgfpathrectangle{\pgfqpoint{7.394209in}{0.375000in}}{\pgfqpoint{6.356833in}{5.175000in}}%
\pgfusepath{clip}%
\pgfsetbuttcap%
\pgfsetroundjoin%
\pgfsetlinewidth{1.003750pt}%
\definecolor{currentstroke}{rgb}{0.827451,0.827451,0.827451}%
\pgfsetstrokecolor{currentstroke}%
\pgfsetdash{}{0pt}%
\pgfpathmoveto{\pgfqpoint{7.757284in}{3.797591in}}%
\pgfpathcurveto{\pgfqpoint{7.768334in}{3.797591in}}{\pgfqpoint{7.778933in}{3.801981in}}{\pgfqpoint{7.786747in}{3.809795in}}%
\pgfpathcurveto{\pgfqpoint{7.794560in}{3.817608in}}{\pgfqpoint{7.798951in}{3.828207in}}{\pgfqpoint{7.798951in}{3.839257in}}%
\pgfpathcurveto{\pgfqpoint{7.798951in}{3.850308in}}{\pgfqpoint{7.794560in}{3.860907in}}{\pgfqpoint{7.786747in}{3.868720in}}%
\pgfpathcurveto{\pgfqpoint{7.778933in}{3.876534in}}{\pgfqpoint{7.768334in}{3.880924in}}{\pgfqpoint{7.757284in}{3.880924in}}%
\pgfpathcurveto{\pgfqpoint{7.746234in}{3.880924in}}{\pgfqpoint{7.735635in}{3.876534in}}{\pgfqpoint{7.727821in}{3.868720in}}%
\pgfpathcurveto{\pgfqpoint{7.720008in}{3.860907in}}{\pgfqpoint{7.715617in}{3.850308in}}{\pgfqpoint{7.715617in}{3.839257in}}%
\pgfpathcurveto{\pgfqpoint{7.715617in}{3.828207in}}{\pgfqpoint{7.720008in}{3.817608in}}{\pgfqpoint{7.727821in}{3.809795in}}%
\pgfpathcurveto{\pgfqpoint{7.735635in}{3.801981in}}{\pgfqpoint{7.746234in}{3.797591in}}{\pgfqpoint{7.757284in}{3.797591in}}%
\pgfpathlineto{\pgfqpoint{7.757284in}{3.797591in}}%
\pgfpathclose%
\pgfusepath{stroke}%
\end{pgfscope}%
\begin{pgfscope}%
\pgfpathrectangle{\pgfqpoint{7.394209in}{0.375000in}}{\pgfqpoint{6.356833in}{5.175000in}}%
\pgfusepath{clip}%
\pgfsetbuttcap%
\pgfsetroundjoin%
\pgfsetlinewidth{1.003750pt}%
\definecolor{currentstroke}{rgb}{0.827451,0.827451,0.827451}%
\pgfsetstrokecolor{currentstroke}%
\pgfsetdash{}{0pt}%
\pgfpathmoveto{\pgfqpoint{7.582670in}{3.406732in}}%
\pgfpathcurveto{\pgfqpoint{7.593721in}{3.406732in}}{\pgfqpoint{7.604320in}{3.411122in}}{\pgfqpoint{7.612133in}{3.418936in}}%
\pgfpathcurveto{\pgfqpoint{7.619947in}{3.426749in}}{\pgfqpoint{7.624337in}{3.437349in}}{\pgfqpoint{7.624337in}{3.448399in}}%
\pgfpathcurveto{\pgfqpoint{7.624337in}{3.459449in}}{\pgfqpoint{7.619947in}{3.470048in}}{\pgfqpoint{7.612133in}{3.477861in}}%
\pgfpathcurveto{\pgfqpoint{7.604320in}{3.485675in}}{\pgfqpoint{7.593721in}{3.490065in}}{\pgfqpoint{7.582670in}{3.490065in}}%
\pgfpathcurveto{\pgfqpoint{7.571620in}{3.490065in}}{\pgfqpoint{7.561021in}{3.485675in}}{\pgfqpoint{7.553208in}{3.477861in}}%
\pgfpathcurveto{\pgfqpoint{7.545394in}{3.470048in}}{\pgfqpoint{7.541004in}{3.459449in}}{\pgfqpoint{7.541004in}{3.448399in}}%
\pgfpathcurveto{\pgfqpoint{7.541004in}{3.437349in}}{\pgfqpoint{7.545394in}{3.426749in}}{\pgfqpoint{7.553208in}{3.418936in}}%
\pgfpathcurveto{\pgfqpoint{7.561021in}{3.411122in}}{\pgfqpoint{7.571620in}{3.406732in}}{\pgfqpoint{7.582670in}{3.406732in}}%
\pgfpathlineto{\pgfqpoint{7.582670in}{3.406732in}}%
\pgfpathclose%
\pgfusepath{stroke}%
\end{pgfscope}%
\begin{pgfscope}%
\pgfpathrectangle{\pgfqpoint{7.394209in}{0.375000in}}{\pgfqpoint{6.356833in}{5.175000in}}%
\pgfusepath{clip}%
\pgfsetbuttcap%
\pgfsetroundjoin%
\pgfsetlinewidth{1.003750pt}%
\definecolor{currentstroke}{rgb}{0.827451,0.827451,0.827451}%
\pgfsetstrokecolor{currentstroke}%
\pgfsetdash{}{0pt}%
\pgfpathmoveto{\pgfqpoint{7.695239in}{3.825842in}}%
\pgfpathcurveto{\pgfqpoint{7.706289in}{3.825842in}}{\pgfqpoint{7.716888in}{3.830232in}}{\pgfqpoint{7.724702in}{3.838046in}}%
\pgfpathcurveto{\pgfqpoint{7.732515in}{3.845860in}}{\pgfqpoint{7.736906in}{3.856459in}}{\pgfqpoint{7.736906in}{3.867509in}}%
\pgfpathcurveto{\pgfqpoint{7.736906in}{3.878559in}}{\pgfqpoint{7.732515in}{3.889158in}}{\pgfqpoint{7.724702in}{3.896972in}}%
\pgfpathcurveto{\pgfqpoint{7.716888in}{3.904785in}}{\pgfqpoint{7.706289in}{3.909175in}}{\pgfqpoint{7.695239in}{3.909175in}}%
\pgfpathcurveto{\pgfqpoint{7.684189in}{3.909175in}}{\pgfqpoint{7.673590in}{3.904785in}}{\pgfqpoint{7.665776in}{3.896972in}}%
\pgfpathcurveto{\pgfqpoint{7.657963in}{3.889158in}}{\pgfqpoint{7.653572in}{3.878559in}}{\pgfqpoint{7.653572in}{3.867509in}}%
\pgfpathcurveto{\pgfqpoint{7.653572in}{3.856459in}}{\pgfqpoint{7.657963in}{3.845860in}}{\pgfqpoint{7.665776in}{3.838046in}}%
\pgfpathcurveto{\pgfqpoint{7.673590in}{3.830232in}}{\pgfqpoint{7.684189in}{3.825842in}}{\pgfqpoint{7.695239in}{3.825842in}}%
\pgfpathlineto{\pgfqpoint{7.695239in}{3.825842in}}%
\pgfpathclose%
\pgfusepath{stroke}%
\end{pgfscope}%
\begin{pgfscope}%
\pgfpathrectangle{\pgfqpoint{7.394209in}{0.375000in}}{\pgfqpoint{6.356833in}{5.175000in}}%
\pgfusepath{clip}%
\pgfsetbuttcap%
\pgfsetroundjoin%
\pgfsetlinewidth{1.003750pt}%
\definecolor{currentstroke}{rgb}{0.827451,0.827451,0.827451}%
\pgfsetstrokecolor{currentstroke}%
\pgfsetdash{}{0pt}%
\pgfpathmoveto{\pgfqpoint{9.489857in}{1.337136in}}%
\pgfpathcurveto{\pgfqpoint{9.500907in}{1.337136in}}{\pgfqpoint{9.511506in}{1.341526in}}{\pgfqpoint{9.519319in}{1.349340in}}%
\pgfpathcurveto{\pgfqpoint{9.527133in}{1.357153in}}{\pgfqpoint{9.531523in}{1.367752in}}{\pgfqpoint{9.531523in}{1.378802in}}%
\pgfpathcurveto{\pgfqpoint{9.531523in}{1.389853in}}{\pgfqpoint{9.527133in}{1.400452in}}{\pgfqpoint{9.519319in}{1.408265in}}%
\pgfpathcurveto{\pgfqpoint{9.511506in}{1.416079in}}{\pgfqpoint{9.500907in}{1.420469in}}{\pgfqpoint{9.489857in}{1.420469in}}%
\pgfpathcurveto{\pgfqpoint{9.478807in}{1.420469in}}{\pgfqpoint{9.468207in}{1.416079in}}{\pgfqpoint{9.460394in}{1.408265in}}%
\pgfpathcurveto{\pgfqpoint{9.452580in}{1.400452in}}{\pgfqpoint{9.448190in}{1.389853in}}{\pgfqpoint{9.448190in}{1.378802in}}%
\pgfpathcurveto{\pgfqpoint{9.448190in}{1.367752in}}{\pgfqpoint{9.452580in}{1.357153in}}{\pgfqpoint{9.460394in}{1.349340in}}%
\pgfpathcurveto{\pgfqpoint{9.468207in}{1.341526in}}{\pgfqpoint{9.478807in}{1.337136in}}{\pgfqpoint{9.489857in}{1.337136in}}%
\pgfpathlineto{\pgfqpoint{9.489857in}{1.337136in}}%
\pgfpathclose%
\pgfusepath{stroke}%
\end{pgfscope}%
\begin{pgfscope}%
\pgfpathrectangle{\pgfqpoint{7.394209in}{0.375000in}}{\pgfqpoint{6.356833in}{5.175000in}}%
\pgfusepath{clip}%
\pgfsetbuttcap%
\pgfsetroundjoin%
\pgfsetlinewidth{1.003750pt}%
\definecolor{currentstroke}{rgb}{0.827451,0.827451,0.827451}%
\pgfsetstrokecolor{currentstroke}%
\pgfsetdash{}{0pt}%
\pgfpathmoveto{\pgfqpoint{8.328252in}{3.475561in}}%
\pgfpathcurveto{\pgfqpoint{8.339302in}{3.475561in}}{\pgfqpoint{8.349901in}{3.479951in}}{\pgfqpoint{8.357715in}{3.487764in}}%
\pgfpathcurveto{\pgfqpoint{8.365528in}{3.495578in}}{\pgfqpoint{8.369919in}{3.506177in}}{\pgfqpoint{8.369919in}{3.517227in}}%
\pgfpathcurveto{\pgfqpoint{8.369919in}{3.528277in}}{\pgfqpoint{8.365528in}{3.538876in}}{\pgfqpoint{8.357715in}{3.546690in}}%
\pgfpathcurveto{\pgfqpoint{8.349901in}{3.554504in}}{\pgfqpoint{8.339302in}{3.558894in}}{\pgfqpoint{8.328252in}{3.558894in}}%
\pgfpathcurveto{\pgfqpoint{8.317202in}{3.558894in}}{\pgfqpoint{8.306603in}{3.554504in}}{\pgfqpoint{8.298789in}{3.546690in}}%
\pgfpathcurveto{\pgfqpoint{8.290975in}{3.538876in}}{\pgfqpoint{8.286585in}{3.528277in}}{\pgfqpoint{8.286585in}{3.517227in}}%
\pgfpathcurveto{\pgfqpoint{8.286585in}{3.506177in}}{\pgfqpoint{8.290975in}{3.495578in}}{\pgfqpoint{8.298789in}{3.487764in}}%
\pgfpathcurveto{\pgfqpoint{8.306603in}{3.479951in}}{\pgfqpoint{8.317202in}{3.475561in}}{\pgfqpoint{8.328252in}{3.475561in}}%
\pgfpathlineto{\pgfqpoint{8.328252in}{3.475561in}}%
\pgfpathclose%
\pgfusepath{stroke}%
\end{pgfscope}%
\begin{pgfscope}%
\pgfpathrectangle{\pgfqpoint{7.394209in}{0.375000in}}{\pgfqpoint{6.356833in}{5.175000in}}%
\pgfusepath{clip}%
\pgfsetbuttcap%
\pgfsetroundjoin%
\pgfsetlinewidth{1.003750pt}%
\definecolor{currentstroke}{rgb}{0.827451,0.827451,0.827451}%
\pgfsetstrokecolor{currentstroke}%
\pgfsetdash{}{0pt}%
\pgfpathmoveto{\pgfqpoint{13.055333in}{5.422463in}}%
\pgfpathcurveto{\pgfqpoint{13.066383in}{5.422463in}}{\pgfqpoint{13.076982in}{5.426853in}}{\pgfqpoint{13.084795in}{5.434667in}}%
\pgfpathcurveto{\pgfqpoint{13.092609in}{5.442481in}}{\pgfqpoint{13.096999in}{5.453080in}}{\pgfqpoint{13.096999in}{5.464130in}}%
\pgfpathcurveto{\pgfqpoint{13.096999in}{5.475180in}}{\pgfqpoint{13.092609in}{5.485779in}}{\pgfqpoint{13.084795in}{5.493592in}}%
\pgfpathcurveto{\pgfqpoint{13.076982in}{5.501406in}}{\pgfqpoint{13.066383in}{5.505796in}}{\pgfqpoint{13.055333in}{5.505796in}}%
\pgfpathcurveto{\pgfqpoint{13.044282in}{5.505796in}}{\pgfqpoint{13.033683in}{5.501406in}}{\pgfqpoint{13.025870in}{5.493592in}}%
\pgfpathcurveto{\pgfqpoint{13.018056in}{5.485779in}}{\pgfqpoint{13.013666in}{5.475180in}}{\pgfqpoint{13.013666in}{5.464130in}}%
\pgfpathcurveto{\pgfqpoint{13.013666in}{5.453080in}}{\pgfqpoint{13.018056in}{5.442481in}}{\pgfqpoint{13.025870in}{5.434667in}}%
\pgfpathcurveto{\pgfqpoint{13.033683in}{5.426853in}}{\pgfqpoint{13.044282in}{5.422463in}}{\pgfqpoint{13.055333in}{5.422463in}}%
\pgfpathlineto{\pgfqpoint{13.055333in}{5.422463in}}%
\pgfpathclose%
\pgfusepath{stroke}%
\end{pgfscope}%
\begin{pgfscope}%
\pgfpathrectangle{\pgfqpoint{7.394209in}{0.375000in}}{\pgfqpoint{6.356833in}{5.175000in}}%
\pgfusepath{clip}%
\pgfsetbuttcap%
\pgfsetroundjoin%
\pgfsetlinewidth{1.003750pt}%
\definecolor{currentstroke}{rgb}{0.827451,0.827451,0.827451}%
\pgfsetstrokecolor{currentstroke}%
\pgfsetdash{}{0pt}%
\pgfpathmoveto{\pgfqpoint{8.484185in}{3.390342in}}%
\pgfpathcurveto{\pgfqpoint{8.495236in}{3.390342in}}{\pgfqpoint{8.505835in}{3.394732in}}{\pgfqpoint{8.513648in}{3.402546in}}%
\pgfpathcurveto{\pgfqpoint{8.521462in}{3.410359in}}{\pgfqpoint{8.525852in}{3.420958in}}{\pgfqpoint{8.525852in}{3.432008in}}%
\pgfpathcurveto{\pgfqpoint{8.525852in}{3.443059in}}{\pgfqpoint{8.521462in}{3.453658in}}{\pgfqpoint{8.513648in}{3.461471in}}%
\pgfpathcurveto{\pgfqpoint{8.505835in}{3.469285in}}{\pgfqpoint{8.495236in}{3.473675in}}{\pgfqpoint{8.484185in}{3.473675in}}%
\pgfpathcurveto{\pgfqpoint{8.473135in}{3.473675in}}{\pgfqpoint{8.462536in}{3.469285in}}{\pgfqpoint{8.454723in}{3.461471in}}%
\pgfpathcurveto{\pgfqpoint{8.446909in}{3.453658in}}{\pgfqpoint{8.442519in}{3.443059in}}{\pgfqpoint{8.442519in}{3.432008in}}%
\pgfpathcurveto{\pgfqpoint{8.442519in}{3.420958in}}{\pgfqpoint{8.446909in}{3.410359in}}{\pgfqpoint{8.454723in}{3.402546in}}%
\pgfpathcurveto{\pgfqpoint{8.462536in}{3.394732in}}{\pgfqpoint{8.473135in}{3.390342in}}{\pgfqpoint{8.484185in}{3.390342in}}%
\pgfpathlineto{\pgfqpoint{8.484185in}{3.390342in}}%
\pgfpathclose%
\pgfusepath{stroke}%
\end{pgfscope}%
\begin{pgfscope}%
\pgfpathrectangle{\pgfqpoint{7.394209in}{0.375000in}}{\pgfqpoint{6.356833in}{5.175000in}}%
\pgfusepath{clip}%
\pgfsetbuttcap%
\pgfsetroundjoin%
\pgfsetlinewidth{1.003750pt}%
\definecolor{currentstroke}{rgb}{0.827451,0.827451,0.827451}%
\pgfsetstrokecolor{currentstroke}%
\pgfsetdash{}{0pt}%
\pgfpathmoveto{\pgfqpoint{12.391657in}{5.306552in}}%
\pgfpathcurveto{\pgfqpoint{12.402707in}{5.306552in}}{\pgfqpoint{12.413306in}{5.310942in}}{\pgfqpoint{12.421119in}{5.318756in}}%
\pgfpathcurveto{\pgfqpoint{12.428933in}{5.326569in}}{\pgfqpoint{12.433323in}{5.337168in}}{\pgfqpoint{12.433323in}{5.348219in}}%
\pgfpathcurveto{\pgfqpoint{12.433323in}{5.359269in}}{\pgfqpoint{12.428933in}{5.369868in}}{\pgfqpoint{12.421119in}{5.377681in}}%
\pgfpathcurveto{\pgfqpoint{12.413306in}{5.385495in}}{\pgfqpoint{12.402707in}{5.389885in}}{\pgfqpoint{12.391657in}{5.389885in}}%
\pgfpathcurveto{\pgfqpoint{12.380607in}{5.389885in}}{\pgfqpoint{12.370007in}{5.385495in}}{\pgfqpoint{12.362194in}{5.377681in}}%
\pgfpathcurveto{\pgfqpoint{12.354380in}{5.369868in}}{\pgfqpoint{12.349990in}{5.359269in}}{\pgfqpoint{12.349990in}{5.348219in}}%
\pgfpathcurveto{\pgfqpoint{12.349990in}{5.337168in}}{\pgfqpoint{12.354380in}{5.326569in}}{\pgfqpoint{12.362194in}{5.318756in}}%
\pgfpathcurveto{\pgfqpoint{12.370007in}{5.310942in}}{\pgfqpoint{12.380607in}{5.306552in}}{\pgfqpoint{12.391657in}{5.306552in}}%
\pgfpathlineto{\pgfqpoint{12.391657in}{5.306552in}}%
\pgfpathclose%
\pgfusepath{stroke}%
\end{pgfscope}%
\begin{pgfscope}%
\pgfpathrectangle{\pgfqpoint{7.394209in}{0.375000in}}{\pgfqpoint{6.356833in}{5.175000in}}%
\pgfusepath{clip}%
\pgfsetbuttcap%
\pgfsetroundjoin%
\pgfsetlinewidth{1.003750pt}%
\definecolor{currentstroke}{rgb}{0.827451,0.827451,0.827451}%
\pgfsetstrokecolor{currentstroke}%
\pgfsetdash{}{0pt}%
\pgfpathmoveto{\pgfqpoint{11.580964in}{4.954275in}}%
\pgfpathcurveto{\pgfqpoint{11.592015in}{4.954275in}}{\pgfqpoint{11.602614in}{4.958665in}}{\pgfqpoint{11.610427in}{4.966479in}}%
\pgfpathcurveto{\pgfqpoint{11.618241in}{4.974293in}}{\pgfqpoint{11.622631in}{4.984892in}}{\pgfqpoint{11.622631in}{4.995942in}}%
\pgfpathcurveto{\pgfqpoint{11.622631in}{5.006992in}}{\pgfqpoint{11.618241in}{5.017591in}}{\pgfqpoint{11.610427in}{5.025405in}}%
\pgfpathcurveto{\pgfqpoint{11.602614in}{5.033218in}}{\pgfqpoint{11.592015in}{5.037608in}}{\pgfqpoint{11.580964in}{5.037608in}}%
\pgfpathcurveto{\pgfqpoint{11.569914in}{5.037608in}}{\pgfqpoint{11.559315in}{5.033218in}}{\pgfqpoint{11.551502in}{5.025405in}}%
\pgfpathcurveto{\pgfqpoint{11.543688in}{5.017591in}}{\pgfqpoint{11.539298in}{5.006992in}}{\pgfqpoint{11.539298in}{4.995942in}}%
\pgfpathcurveto{\pgfqpoint{11.539298in}{4.984892in}}{\pgfqpoint{11.543688in}{4.974293in}}{\pgfqpoint{11.551502in}{4.966479in}}%
\pgfpathcurveto{\pgfqpoint{11.559315in}{4.958665in}}{\pgfqpoint{11.569914in}{4.954275in}}{\pgfqpoint{11.580964in}{4.954275in}}%
\pgfpathlineto{\pgfqpoint{11.580964in}{4.954275in}}%
\pgfpathclose%
\pgfusepath{stroke}%
\end{pgfscope}%
\begin{pgfscope}%
\pgfpathrectangle{\pgfqpoint{7.394209in}{0.375000in}}{\pgfqpoint{6.356833in}{5.175000in}}%
\pgfusepath{clip}%
\pgfsetbuttcap%
\pgfsetroundjoin%
\pgfsetlinewidth{1.003750pt}%
\definecolor{currentstroke}{rgb}{0.827451,0.827451,0.827451}%
\pgfsetstrokecolor{currentstroke}%
\pgfsetdash{}{0pt}%
\pgfpathmoveto{\pgfqpoint{9.751504in}{1.776818in}}%
\pgfpathcurveto{\pgfqpoint{9.762554in}{1.776818in}}{\pgfqpoint{9.773153in}{1.781209in}}{\pgfqpoint{9.780967in}{1.789022in}}%
\pgfpathcurveto{\pgfqpoint{9.788781in}{1.796836in}}{\pgfqpoint{9.793171in}{1.807435in}}{\pgfqpoint{9.793171in}{1.818485in}}%
\pgfpathcurveto{\pgfqpoint{9.793171in}{1.829535in}}{\pgfqpoint{9.788781in}{1.840134in}}{\pgfqpoint{9.780967in}{1.847948in}}%
\pgfpathcurveto{\pgfqpoint{9.773153in}{1.855761in}}{\pgfqpoint{9.762554in}{1.860152in}}{\pgfqpoint{9.751504in}{1.860152in}}%
\pgfpathcurveto{\pgfqpoint{9.740454in}{1.860152in}}{\pgfqpoint{9.729855in}{1.855761in}}{\pgfqpoint{9.722042in}{1.847948in}}%
\pgfpathcurveto{\pgfqpoint{9.714228in}{1.840134in}}{\pgfqpoint{9.709838in}{1.829535in}}{\pgfqpoint{9.709838in}{1.818485in}}%
\pgfpathcurveto{\pgfqpoint{9.709838in}{1.807435in}}{\pgfqpoint{9.714228in}{1.796836in}}{\pgfqpoint{9.722042in}{1.789022in}}%
\pgfpathcurveto{\pgfqpoint{9.729855in}{1.781209in}}{\pgfqpoint{9.740454in}{1.776818in}}{\pgfqpoint{9.751504in}{1.776818in}}%
\pgfpathlineto{\pgfqpoint{9.751504in}{1.776818in}}%
\pgfpathclose%
\pgfusepath{stroke}%
\end{pgfscope}%
\begin{pgfscope}%
\pgfpathrectangle{\pgfqpoint{7.394209in}{0.375000in}}{\pgfqpoint{6.356833in}{5.175000in}}%
\pgfusepath{clip}%
\pgfsetbuttcap%
\pgfsetroundjoin%
\pgfsetlinewidth{1.003750pt}%
\definecolor{currentstroke}{rgb}{0.827451,0.827451,0.827451}%
\pgfsetstrokecolor{currentstroke}%
\pgfsetdash{}{0pt}%
\pgfpathmoveto{\pgfqpoint{7.833310in}{1.615383in}}%
\pgfpathcurveto{\pgfqpoint{7.844360in}{1.615383in}}{\pgfqpoint{7.854959in}{1.619773in}}{\pgfqpoint{7.862773in}{1.627586in}}%
\pgfpathcurveto{\pgfqpoint{7.870586in}{1.635400in}}{\pgfqpoint{7.874977in}{1.645999in}}{\pgfqpoint{7.874977in}{1.657049in}}%
\pgfpathcurveto{\pgfqpoint{7.874977in}{1.668099in}}{\pgfqpoint{7.870586in}{1.678698in}}{\pgfqpoint{7.862773in}{1.686512in}}%
\pgfpathcurveto{\pgfqpoint{7.854959in}{1.694326in}}{\pgfqpoint{7.844360in}{1.698716in}}{\pgfqpoint{7.833310in}{1.698716in}}%
\pgfpathcurveto{\pgfqpoint{7.822260in}{1.698716in}}{\pgfqpoint{7.811661in}{1.694326in}}{\pgfqpoint{7.803847in}{1.686512in}}%
\pgfpathcurveto{\pgfqpoint{7.796033in}{1.678698in}}{\pgfqpoint{7.791643in}{1.668099in}}{\pgfqpoint{7.791643in}{1.657049in}}%
\pgfpathcurveto{\pgfqpoint{7.791643in}{1.645999in}}{\pgfqpoint{7.796033in}{1.635400in}}{\pgfqpoint{7.803847in}{1.627586in}}%
\pgfpathcurveto{\pgfqpoint{7.811661in}{1.619773in}}{\pgfqpoint{7.822260in}{1.615383in}}{\pgfqpoint{7.833310in}{1.615383in}}%
\pgfpathlineto{\pgfqpoint{7.833310in}{1.615383in}}%
\pgfpathclose%
\pgfusepath{stroke}%
\end{pgfscope}%
\begin{pgfscope}%
\pgfpathrectangle{\pgfqpoint{7.394209in}{0.375000in}}{\pgfqpoint{6.356833in}{5.175000in}}%
\pgfusepath{clip}%
\pgfsetbuttcap%
\pgfsetroundjoin%
\pgfsetlinewidth{1.003750pt}%
\definecolor{currentstroke}{rgb}{0.827451,0.827451,0.827451}%
\pgfsetstrokecolor{currentstroke}%
\pgfsetdash{}{0pt}%
\pgfpathmoveto{\pgfqpoint{8.153133in}{2.252901in}}%
\pgfpathcurveto{\pgfqpoint{8.164183in}{2.252901in}}{\pgfqpoint{8.174782in}{2.257291in}}{\pgfqpoint{8.182595in}{2.265105in}}%
\pgfpathcurveto{\pgfqpoint{8.190409in}{2.272918in}}{\pgfqpoint{8.194799in}{2.283517in}}{\pgfqpoint{8.194799in}{2.294568in}}%
\pgfpathcurveto{\pgfqpoint{8.194799in}{2.305618in}}{\pgfqpoint{8.190409in}{2.316217in}}{\pgfqpoint{8.182595in}{2.324030in}}%
\pgfpathcurveto{\pgfqpoint{8.174782in}{2.331844in}}{\pgfqpoint{8.164183in}{2.336234in}}{\pgfqpoint{8.153133in}{2.336234in}}%
\pgfpathcurveto{\pgfqpoint{8.142082in}{2.336234in}}{\pgfqpoint{8.131483in}{2.331844in}}{\pgfqpoint{8.123670in}{2.324030in}}%
\pgfpathcurveto{\pgfqpoint{8.115856in}{2.316217in}}{\pgfqpoint{8.111466in}{2.305618in}}{\pgfqpoint{8.111466in}{2.294568in}}%
\pgfpathcurveto{\pgfqpoint{8.111466in}{2.283517in}}{\pgfqpoint{8.115856in}{2.272918in}}{\pgfqpoint{8.123670in}{2.265105in}}%
\pgfpathcurveto{\pgfqpoint{8.131483in}{2.257291in}}{\pgfqpoint{8.142082in}{2.252901in}}{\pgfqpoint{8.153133in}{2.252901in}}%
\pgfpathlineto{\pgfqpoint{8.153133in}{2.252901in}}%
\pgfpathclose%
\pgfusepath{stroke}%
\end{pgfscope}%
\begin{pgfscope}%
\pgfpathrectangle{\pgfqpoint{7.394209in}{0.375000in}}{\pgfqpoint{6.356833in}{5.175000in}}%
\pgfusepath{clip}%
\pgfsetbuttcap%
\pgfsetroundjoin%
\pgfsetlinewidth{1.003750pt}%
\definecolor{currentstroke}{rgb}{0.827451,0.827451,0.827451}%
\pgfsetstrokecolor{currentstroke}%
\pgfsetdash{}{0pt}%
\pgfpathmoveto{\pgfqpoint{12.022552in}{5.065369in}}%
\pgfpathcurveto{\pgfqpoint{12.033602in}{5.065369in}}{\pgfqpoint{12.044201in}{5.069760in}}{\pgfqpoint{12.052015in}{5.077573in}}%
\pgfpathcurveto{\pgfqpoint{12.059828in}{5.085387in}}{\pgfqpoint{12.064219in}{5.095986in}}{\pgfqpoint{12.064219in}{5.107036in}}%
\pgfpathcurveto{\pgfqpoint{12.064219in}{5.118086in}}{\pgfqpoint{12.059828in}{5.128685in}}{\pgfqpoint{12.052015in}{5.136499in}}%
\pgfpathcurveto{\pgfqpoint{12.044201in}{5.144312in}}{\pgfqpoint{12.033602in}{5.148703in}}{\pgfqpoint{12.022552in}{5.148703in}}%
\pgfpathcurveto{\pgfqpoint{12.011502in}{5.148703in}}{\pgfqpoint{12.000903in}{5.144312in}}{\pgfqpoint{11.993089in}{5.136499in}}%
\pgfpathcurveto{\pgfqpoint{11.985276in}{5.128685in}}{\pgfqpoint{11.980885in}{5.118086in}}{\pgfqpoint{11.980885in}{5.107036in}}%
\pgfpathcurveto{\pgfqpoint{11.980885in}{5.095986in}}{\pgfqpoint{11.985276in}{5.085387in}}{\pgfqpoint{11.993089in}{5.077573in}}%
\pgfpathcurveto{\pgfqpoint{12.000903in}{5.069760in}}{\pgfqpoint{12.011502in}{5.065369in}}{\pgfqpoint{12.022552in}{5.065369in}}%
\pgfpathlineto{\pgfqpoint{12.022552in}{5.065369in}}%
\pgfpathclose%
\pgfusepath{stroke}%
\end{pgfscope}%
\begin{pgfscope}%
\pgfpathrectangle{\pgfqpoint{7.394209in}{0.375000in}}{\pgfqpoint{6.356833in}{5.175000in}}%
\pgfusepath{clip}%
\pgfsetbuttcap%
\pgfsetroundjoin%
\pgfsetlinewidth{1.003750pt}%
\definecolor{currentstroke}{rgb}{0.827451,0.827451,0.827451}%
\pgfsetstrokecolor{currentstroke}%
\pgfsetdash{}{0pt}%
\pgfpathmoveto{\pgfqpoint{9.927656in}{2.863182in}}%
\pgfpathcurveto{\pgfqpoint{9.938706in}{2.863182in}}{\pgfqpoint{9.949306in}{2.867572in}}{\pgfqpoint{9.957119in}{2.875385in}}%
\pgfpathcurveto{\pgfqpoint{9.964933in}{2.883199in}}{\pgfqpoint{9.969323in}{2.893798in}}{\pgfqpoint{9.969323in}{2.904848in}}%
\pgfpathcurveto{\pgfqpoint{9.969323in}{2.915898in}}{\pgfqpoint{9.964933in}{2.926497in}}{\pgfqpoint{9.957119in}{2.934311in}}%
\pgfpathcurveto{\pgfqpoint{9.949306in}{2.942125in}}{\pgfqpoint{9.938706in}{2.946515in}}{\pgfqpoint{9.927656in}{2.946515in}}%
\pgfpathcurveto{\pgfqpoint{9.916606in}{2.946515in}}{\pgfqpoint{9.906007in}{2.942125in}}{\pgfqpoint{9.898194in}{2.934311in}}%
\pgfpathcurveto{\pgfqpoint{9.890380in}{2.926497in}}{\pgfqpoint{9.885990in}{2.915898in}}{\pgfqpoint{9.885990in}{2.904848in}}%
\pgfpathcurveto{\pgfqpoint{9.885990in}{2.893798in}}{\pgfqpoint{9.890380in}{2.883199in}}{\pgfqpoint{9.898194in}{2.875385in}}%
\pgfpathcurveto{\pgfqpoint{9.906007in}{2.867572in}}{\pgfqpoint{9.916606in}{2.863182in}}{\pgfqpoint{9.927656in}{2.863182in}}%
\pgfpathlineto{\pgfqpoint{9.927656in}{2.863182in}}%
\pgfpathclose%
\pgfusepath{stroke}%
\end{pgfscope}%
\begin{pgfscope}%
\pgfpathrectangle{\pgfqpoint{7.394209in}{0.375000in}}{\pgfqpoint{6.356833in}{5.175000in}}%
\pgfusepath{clip}%
\pgfsetbuttcap%
\pgfsetroundjoin%
\pgfsetlinewidth{1.003750pt}%
\definecolor{currentstroke}{rgb}{0.827451,0.827451,0.827451}%
\pgfsetstrokecolor{currentstroke}%
\pgfsetdash{}{0pt}%
\pgfpathmoveto{\pgfqpoint{10.798567in}{4.094253in}}%
\pgfpathcurveto{\pgfqpoint{10.809618in}{4.094253in}}{\pgfqpoint{10.820217in}{4.098643in}}{\pgfqpoint{10.828030in}{4.106457in}}%
\pgfpathcurveto{\pgfqpoint{10.835844in}{4.114270in}}{\pgfqpoint{10.840234in}{4.124869in}}{\pgfqpoint{10.840234in}{4.135920in}}%
\pgfpathcurveto{\pgfqpoint{10.840234in}{4.146970in}}{\pgfqpoint{10.835844in}{4.157569in}}{\pgfqpoint{10.828030in}{4.165382in}}%
\pgfpathcurveto{\pgfqpoint{10.820217in}{4.173196in}}{\pgfqpoint{10.809618in}{4.177586in}}{\pgfqpoint{10.798567in}{4.177586in}}%
\pgfpathcurveto{\pgfqpoint{10.787517in}{4.177586in}}{\pgfqpoint{10.776918in}{4.173196in}}{\pgfqpoint{10.769105in}{4.165382in}}%
\pgfpathcurveto{\pgfqpoint{10.761291in}{4.157569in}}{\pgfqpoint{10.756901in}{4.146970in}}{\pgfqpoint{10.756901in}{4.135920in}}%
\pgfpathcurveto{\pgfqpoint{10.756901in}{4.124869in}}{\pgfqpoint{10.761291in}{4.114270in}}{\pgfqpoint{10.769105in}{4.106457in}}%
\pgfpathcurveto{\pgfqpoint{10.776918in}{4.098643in}}{\pgfqpoint{10.787517in}{4.094253in}}{\pgfqpoint{10.798567in}{4.094253in}}%
\pgfpathlineto{\pgfqpoint{10.798567in}{4.094253in}}%
\pgfpathclose%
\pgfusepath{stroke}%
\end{pgfscope}%
\begin{pgfscope}%
\pgfpathrectangle{\pgfqpoint{7.394209in}{0.375000in}}{\pgfqpoint{6.356833in}{5.175000in}}%
\pgfusepath{clip}%
\pgfsetbuttcap%
\pgfsetroundjoin%
\pgfsetlinewidth{1.003750pt}%
\definecolor{currentstroke}{rgb}{0.827451,0.827451,0.827451}%
\pgfsetstrokecolor{currentstroke}%
\pgfsetdash{}{0pt}%
\pgfpathmoveto{\pgfqpoint{10.265263in}{5.431100in}}%
\pgfpathcurveto{\pgfqpoint{10.276313in}{5.431100in}}{\pgfqpoint{10.286912in}{5.435490in}}{\pgfqpoint{10.294726in}{5.443303in}}%
\pgfpathcurveto{\pgfqpoint{10.302539in}{5.451117in}}{\pgfqpoint{10.306930in}{5.461716in}}{\pgfqpoint{10.306930in}{5.472766in}}%
\pgfpathcurveto{\pgfqpoint{10.306930in}{5.483816in}}{\pgfqpoint{10.302539in}{5.494415in}}{\pgfqpoint{10.294726in}{5.502229in}}%
\pgfpathcurveto{\pgfqpoint{10.286912in}{5.510043in}}{\pgfqpoint{10.276313in}{5.514433in}}{\pgfqpoint{10.265263in}{5.514433in}}%
\pgfpathcurveto{\pgfqpoint{10.254213in}{5.514433in}}{\pgfqpoint{10.243614in}{5.510043in}}{\pgfqpoint{10.235800in}{5.502229in}}%
\pgfpathcurveto{\pgfqpoint{10.227986in}{5.494415in}}{\pgfqpoint{10.223596in}{5.483816in}}{\pgfqpoint{10.223596in}{5.472766in}}%
\pgfpathcurveto{\pgfqpoint{10.223596in}{5.461716in}}{\pgfqpoint{10.227986in}{5.451117in}}{\pgfqpoint{10.235800in}{5.443303in}}%
\pgfpathcurveto{\pgfqpoint{10.243614in}{5.435490in}}{\pgfqpoint{10.254213in}{5.431100in}}{\pgfqpoint{10.265263in}{5.431100in}}%
\pgfpathlineto{\pgfqpoint{10.265263in}{5.431100in}}%
\pgfpathclose%
\pgfusepath{stroke}%
\end{pgfscope}%
\begin{pgfscope}%
\pgfpathrectangle{\pgfqpoint{7.394209in}{0.375000in}}{\pgfqpoint{6.356833in}{5.175000in}}%
\pgfusepath{clip}%
\pgfsetbuttcap%
\pgfsetroundjoin%
\pgfsetlinewidth{1.003750pt}%
\definecolor{currentstroke}{rgb}{0.827451,0.827451,0.827451}%
\pgfsetstrokecolor{currentstroke}%
\pgfsetdash{}{0pt}%
\pgfpathmoveto{\pgfqpoint{9.775019in}{4.457496in}}%
\pgfpathcurveto{\pgfqpoint{9.786070in}{4.457496in}}{\pgfqpoint{9.796669in}{4.461886in}}{\pgfqpoint{9.804482in}{4.469700in}}%
\pgfpathcurveto{\pgfqpoint{9.812296in}{4.477514in}}{\pgfqpoint{9.816686in}{4.488113in}}{\pgfqpoint{9.816686in}{4.499163in}}%
\pgfpathcurveto{\pgfqpoint{9.816686in}{4.510213in}}{\pgfqpoint{9.812296in}{4.520812in}}{\pgfqpoint{9.804482in}{4.528626in}}%
\pgfpathcurveto{\pgfqpoint{9.796669in}{4.536439in}}{\pgfqpoint{9.786070in}{4.540829in}}{\pgfqpoint{9.775019in}{4.540829in}}%
\pgfpathcurveto{\pgfqpoint{9.763969in}{4.540829in}}{\pgfqpoint{9.753370in}{4.536439in}}{\pgfqpoint{9.745557in}{4.528626in}}%
\pgfpathcurveto{\pgfqpoint{9.737743in}{4.520812in}}{\pgfqpoint{9.733353in}{4.510213in}}{\pgfqpoint{9.733353in}{4.499163in}}%
\pgfpathcurveto{\pgfqpoint{9.733353in}{4.488113in}}{\pgfqpoint{9.737743in}{4.477514in}}{\pgfqpoint{9.745557in}{4.469700in}}%
\pgfpathcurveto{\pgfqpoint{9.753370in}{4.461886in}}{\pgfqpoint{9.763969in}{4.457496in}}{\pgfqpoint{9.775019in}{4.457496in}}%
\pgfpathlineto{\pgfqpoint{9.775019in}{4.457496in}}%
\pgfpathclose%
\pgfusepath{stroke}%
\end{pgfscope}%
\begin{pgfscope}%
\pgfpathrectangle{\pgfqpoint{7.394209in}{0.375000in}}{\pgfqpoint{6.356833in}{5.175000in}}%
\pgfusepath{clip}%
\pgfsetbuttcap%
\pgfsetroundjoin%
\pgfsetlinewidth{1.003750pt}%
\definecolor{currentstroke}{rgb}{0.827451,0.827451,0.827451}%
\pgfsetstrokecolor{currentstroke}%
\pgfsetdash{}{0pt}%
\pgfpathmoveto{\pgfqpoint{9.554865in}{4.526254in}}%
\pgfpathcurveto{\pgfqpoint{9.565915in}{4.526254in}}{\pgfqpoint{9.576514in}{4.530644in}}{\pgfqpoint{9.584327in}{4.538458in}}%
\pgfpathcurveto{\pgfqpoint{9.592141in}{4.546271in}}{\pgfqpoint{9.596531in}{4.556870in}}{\pgfqpoint{9.596531in}{4.567920in}}%
\pgfpathcurveto{\pgfqpoint{9.596531in}{4.578971in}}{\pgfqpoint{9.592141in}{4.589570in}}{\pgfqpoint{9.584327in}{4.597383in}}%
\pgfpathcurveto{\pgfqpoint{9.576514in}{4.605197in}}{\pgfqpoint{9.565915in}{4.609587in}}{\pgfqpoint{9.554865in}{4.609587in}}%
\pgfpathcurveto{\pgfqpoint{9.543815in}{4.609587in}}{\pgfqpoint{9.533216in}{4.605197in}}{\pgfqpoint{9.525402in}{4.597383in}}%
\pgfpathcurveto{\pgfqpoint{9.517588in}{4.589570in}}{\pgfqpoint{9.513198in}{4.578971in}}{\pgfqpoint{9.513198in}{4.567920in}}%
\pgfpathcurveto{\pgfqpoint{9.513198in}{4.556870in}}{\pgfqpoint{9.517588in}{4.546271in}}{\pgfqpoint{9.525402in}{4.538458in}}%
\pgfpathcurveto{\pgfqpoint{9.533216in}{4.530644in}}{\pgfqpoint{9.543815in}{4.526254in}}{\pgfqpoint{9.554865in}{4.526254in}}%
\pgfpathlineto{\pgfqpoint{9.554865in}{4.526254in}}%
\pgfpathclose%
\pgfusepath{stroke}%
\end{pgfscope}%
\begin{pgfscope}%
\pgfpathrectangle{\pgfqpoint{7.394209in}{0.375000in}}{\pgfqpoint{6.356833in}{5.175000in}}%
\pgfusepath{clip}%
\pgfsetbuttcap%
\pgfsetroundjoin%
\pgfsetlinewidth{1.003750pt}%
\definecolor{currentstroke}{rgb}{0.827451,0.827451,0.827451}%
\pgfsetstrokecolor{currentstroke}%
\pgfsetdash{}{0pt}%
\pgfpathmoveto{\pgfqpoint{11.363958in}{4.579484in}}%
\pgfpathcurveto{\pgfqpoint{11.375008in}{4.579484in}}{\pgfqpoint{11.385607in}{4.583874in}}{\pgfqpoint{11.393421in}{4.591687in}}%
\pgfpathcurveto{\pgfqpoint{11.401235in}{4.599501in}}{\pgfqpoint{11.405625in}{4.610100in}}{\pgfqpoint{11.405625in}{4.621150in}}%
\pgfpathcurveto{\pgfqpoint{11.405625in}{4.632200in}}{\pgfqpoint{11.401235in}{4.642799in}}{\pgfqpoint{11.393421in}{4.650613in}}%
\pgfpathcurveto{\pgfqpoint{11.385607in}{4.658427in}}{\pgfqpoint{11.375008in}{4.662817in}}{\pgfqpoint{11.363958in}{4.662817in}}%
\pgfpathcurveto{\pgfqpoint{11.352908in}{4.662817in}}{\pgfqpoint{11.342309in}{4.658427in}}{\pgfqpoint{11.334496in}{4.650613in}}%
\pgfpathcurveto{\pgfqpoint{11.326682in}{4.642799in}}{\pgfqpoint{11.322292in}{4.632200in}}{\pgfqpoint{11.322292in}{4.621150in}}%
\pgfpathcurveto{\pgfqpoint{11.322292in}{4.610100in}}{\pgfqpoint{11.326682in}{4.599501in}}{\pgfqpoint{11.334496in}{4.591687in}}%
\pgfpathcurveto{\pgfqpoint{11.342309in}{4.583874in}}{\pgfqpoint{11.352908in}{4.579484in}}{\pgfqpoint{11.363958in}{4.579484in}}%
\pgfpathlineto{\pgfqpoint{11.363958in}{4.579484in}}%
\pgfpathclose%
\pgfusepath{stroke}%
\end{pgfscope}%
\begin{pgfscope}%
\pgfpathrectangle{\pgfqpoint{7.394209in}{0.375000in}}{\pgfqpoint{6.356833in}{5.175000in}}%
\pgfusepath{clip}%
\pgfsetbuttcap%
\pgfsetroundjoin%
\pgfsetlinewidth{1.003750pt}%
\definecolor{currentstroke}{rgb}{0.827451,0.827451,0.827451}%
\pgfsetstrokecolor{currentstroke}%
\pgfsetdash{}{0pt}%
\pgfpathmoveto{\pgfqpoint{9.158887in}{4.483022in}}%
\pgfpathcurveto{\pgfqpoint{9.169937in}{4.483022in}}{\pgfqpoint{9.180536in}{4.487412in}}{\pgfqpoint{9.188350in}{4.495226in}}%
\pgfpathcurveto{\pgfqpoint{9.196163in}{4.503039in}}{\pgfqpoint{9.200554in}{4.513638in}}{\pgfqpoint{9.200554in}{4.524688in}}%
\pgfpathcurveto{\pgfqpoint{9.200554in}{4.535739in}}{\pgfqpoint{9.196163in}{4.546338in}}{\pgfqpoint{9.188350in}{4.554151in}}%
\pgfpathcurveto{\pgfqpoint{9.180536in}{4.561965in}}{\pgfqpoint{9.169937in}{4.566355in}}{\pgfqpoint{9.158887in}{4.566355in}}%
\pgfpathcurveto{\pgfqpoint{9.147837in}{4.566355in}}{\pgfqpoint{9.137238in}{4.561965in}}{\pgfqpoint{9.129424in}{4.554151in}}%
\pgfpathcurveto{\pgfqpoint{9.121611in}{4.546338in}}{\pgfqpoint{9.117220in}{4.535739in}}{\pgfqpoint{9.117220in}{4.524688in}}%
\pgfpathcurveto{\pgfqpoint{9.117220in}{4.513638in}}{\pgfqpoint{9.121611in}{4.503039in}}{\pgfqpoint{9.129424in}{4.495226in}}%
\pgfpathcurveto{\pgfqpoint{9.137238in}{4.487412in}}{\pgfqpoint{9.147837in}{4.483022in}}{\pgfqpoint{9.158887in}{4.483022in}}%
\pgfpathlineto{\pgfqpoint{9.158887in}{4.483022in}}%
\pgfpathclose%
\pgfusepath{stroke}%
\end{pgfscope}%
\begin{pgfscope}%
\pgfpathrectangle{\pgfqpoint{7.394209in}{0.375000in}}{\pgfqpoint{6.356833in}{5.175000in}}%
\pgfusepath{clip}%
\pgfsetbuttcap%
\pgfsetroundjoin%
\pgfsetlinewidth{1.003750pt}%
\definecolor{currentstroke}{rgb}{0.827451,0.827451,0.827451}%
\pgfsetstrokecolor{currentstroke}%
\pgfsetdash{}{0pt}%
\pgfpathmoveto{\pgfqpoint{12.022312in}{5.064443in}}%
\pgfpathcurveto{\pgfqpoint{12.033363in}{5.064443in}}{\pgfqpoint{12.043962in}{5.068833in}}{\pgfqpoint{12.051775in}{5.076647in}}%
\pgfpathcurveto{\pgfqpoint{12.059589in}{5.084461in}}{\pgfqpoint{12.063979in}{5.095060in}}{\pgfqpoint{12.063979in}{5.106110in}}%
\pgfpathcurveto{\pgfqpoint{12.063979in}{5.117160in}}{\pgfqpoint{12.059589in}{5.127759in}}{\pgfqpoint{12.051775in}{5.135572in}}%
\pgfpathcurveto{\pgfqpoint{12.043962in}{5.143386in}}{\pgfqpoint{12.033363in}{5.147776in}}{\pgfqpoint{12.022312in}{5.147776in}}%
\pgfpathcurveto{\pgfqpoint{12.011262in}{5.147776in}}{\pgfqpoint{12.000663in}{5.143386in}}{\pgfqpoint{11.992850in}{5.135572in}}%
\pgfpathcurveto{\pgfqpoint{11.985036in}{5.127759in}}{\pgfqpoint{11.980646in}{5.117160in}}{\pgfqpoint{11.980646in}{5.106110in}}%
\pgfpathcurveto{\pgfqpoint{11.980646in}{5.095060in}}{\pgfqpoint{11.985036in}{5.084461in}}{\pgfqpoint{11.992850in}{5.076647in}}%
\pgfpathcurveto{\pgfqpoint{12.000663in}{5.068833in}}{\pgfqpoint{12.011262in}{5.064443in}}{\pgfqpoint{12.022312in}{5.064443in}}%
\pgfpathlineto{\pgfqpoint{12.022312in}{5.064443in}}%
\pgfpathclose%
\pgfusepath{stroke}%
\end{pgfscope}%
\begin{pgfscope}%
\pgfpathrectangle{\pgfqpoint{7.394209in}{0.375000in}}{\pgfqpoint{6.356833in}{5.175000in}}%
\pgfusepath{clip}%
\pgfsetbuttcap%
\pgfsetroundjoin%
\pgfsetlinewidth{1.003750pt}%
\definecolor{currentstroke}{rgb}{0.827451,0.827451,0.827451}%
\pgfsetstrokecolor{currentstroke}%
\pgfsetdash{}{0pt}%
\pgfpathmoveto{\pgfqpoint{9.207502in}{3.894574in}}%
\pgfpathcurveto{\pgfqpoint{9.218552in}{3.894574in}}{\pgfqpoint{9.229151in}{3.898964in}}{\pgfqpoint{9.236965in}{3.906778in}}%
\pgfpathcurveto{\pgfqpoint{9.244779in}{3.914592in}}{\pgfqpoint{9.249169in}{3.925191in}}{\pgfqpoint{9.249169in}{3.936241in}}%
\pgfpathcurveto{\pgfqpoint{9.249169in}{3.947291in}}{\pgfqpoint{9.244779in}{3.957890in}}{\pgfqpoint{9.236965in}{3.965704in}}%
\pgfpathcurveto{\pgfqpoint{9.229151in}{3.973517in}}{\pgfqpoint{9.218552in}{3.977907in}}{\pgfqpoint{9.207502in}{3.977907in}}%
\pgfpathcurveto{\pgfqpoint{9.196452in}{3.977907in}}{\pgfqpoint{9.185853in}{3.973517in}}{\pgfqpoint{9.178039in}{3.965704in}}%
\pgfpathcurveto{\pgfqpoint{9.170226in}{3.957890in}}{\pgfqpoint{9.165835in}{3.947291in}}{\pgfqpoint{9.165835in}{3.936241in}}%
\pgfpathcurveto{\pgfqpoint{9.165835in}{3.925191in}}{\pgfqpoint{9.170226in}{3.914592in}}{\pgfqpoint{9.178039in}{3.906778in}}%
\pgfpathcurveto{\pgfqpoint{9.185853in}{3.898964in}}{\pgfqpoint{9.196452in}{3.894574in}}{\pgfqpoint{9.207502in}{3.894574in}}%
\pgfpathlineto{\pgfqpoint{9.207502in}{3.894574in}}%
\pgfpathclose%
\pgfusepath{stroke}%
\end{pgfscope}%
\begin{pgfscope}%
\pgfpathrectangle{\pgfqpoint{7.394209in}{0.375000in}}{\pgfqpoint{6.356833in}{5.175000in}}%
\pgfusepath{clip}%
\pgfsetbuttcap%
\pgfsetroundjoin%
\pgfsetlinewidth{1.003750pt}%
\definecolor{currentstroke}{rgb}{0.827451,0.827451,0.827451}%
\pgfsetstrokecolor{currentstroke}%
\pgfsetdash{}{0pt}%
\pgfpathmoveto{\pgfqpoint{11.783747in}{4.259497in}}%
\pgfpathcurveto{\pgfqpoint{11.794797in}{4.259497in}}{\pgfqpoint{11.805396in}{4.263887in}}{\pgfqpoint{11.813210in}{4.271701in}}%
\pgfpathcurveto{\pgfqpoint{11.821023in}{4.279514in}}{\pgfqpoint{11.825414in}{4.290113in}}{\pgfqpoint{11.825414in}{4.301163in}}%
\pgfpathcurveto{\pgfqpoint{11.825414in}{4.312213in}}{\pgfqpoint{11.821023in}{4.322812in}}{\pgfqpoint{11.813210in}{4.330626in}}%
\pgfpathcurveto{\pgfqpoint{11.805396in}{4.338440in}}{\pgfqpoint{11.794797in}{4.342830in}}{\pgfqpoint{11.783747in}{4.342830in}}%
\pgfpathcurveto{\pgfqpoint{11.772697in}{4.342830in}}{\pgfqpoint{11.762098in}{4.338440in}}{\pgfqpoint{11.754284in}{4.330626in}}%
\pgfpathcurveto{\pgfqpoint{11.746471in}{4.322812in}}{\pgfqpoint{11.742080in}{4.312213in}}{\pgfqpoint{11.742080in}{4.301163in}}%
\pgfpathcurveto{\pgfqpoint{11.742080in}{4.290113in}}{\pgfqpoint{11.746471in}{4.279514in}}{\pgfqpoint{11.754284in}{4.271701in}}%
\pgfpathcurveto{\pgfqpoint{11.762098in}{4.263887in}}{\pgfqpoint{11.772697in}{4.259497in}}{\pgfqpoint{11.783747in}{4.259497in}}%
\pgfpathlineto{\pgfqpoint{11.783747in}{4.259497in}}%
\pgfpathclose%
\pgfusepath{stroke}%
\end{pgfscope}%
\begin{pgfscope}%
\pgfpathrectangle{\pgfqpoint{7.394209in}{0.375000in}}{\pgfqpoint{6.356833in}{5.175000in}}%
\pgfusepath{clip}%
\pgfsetbuttcap%
\pgfsetroundjoin%
\pgfsetlinewidth{1.003750pt}%
\definecolor{currentstroke}{rgb}{0.827451,0.827451,0.827451}%
\pgfsetstrokecolor{currentstroke}%
\pgfsetdash{}{0pt}%
\pgfpathmoveto{\pgfqpoint{12.070757in}{4.922804in}}%
\pgfpathcurveto{\pgfqpoint{12.081808in}{4.922804in}}{\pgfqpoint{12.092407in}{4.927194in}}{\pgfqpoint{12.100220in}{4.935008in}}%
\pgfpathcurveto{\pgfqpoint{12.108034in}{4.942821in}}{\pgfqpoint{12.112424in}{4.953420in}}{\pgfqpoint{12.112424in}{4.964471in}}%
\pgfpathcurveto{\pgfqpoint{12.112424in}{4.975521in}}{\pgfqpoint{12.108034in}{4.986120in}}{\pgfqpoint{12.100220in}{4.993933in}}%
\pgfpathcurveto{\pgfqpoint{12.092407in}{5.001747in}}{\pgfqpoint{12.081808in}{5.006137in}}{\pgfqpoint{12.070757in}{5.006137in}}%
\pgfpathcurveto{\pgfqpoint{12.059707in}{5.006137in}}{\pgfqpoint{12.049108in}{5.001747in}}{\pgfqpoint{12.041295in}{4.993933in}}%
\pgfpathcurveto{\pgfqpoint{12.033481in}{4.986120in}}{\pgfqpoint{12.029091in}{4.975521in}}{\pgfqpoint{12.029091in}{4.964471in}}%
\pgfpathcurveto{\pgfqpoint{12.029091in}{4.953420in}}{\pgfqpoint{12.033481in}{4.942821in}}{\pgfqpoint{12.041295in}{4.935008in}}%
\pgfpathcurveto{\pgfqpoint{12.049108in}{4.927194in}}{\pgfqpoint{12.059707in}{4.922804in}}{\pgfqpoint{12.070757in}{4.922804in}}%
\pgfpathlineto{\pgfqpoint{12.070757in}{4.922804in}}%
\pgfpathclose%
\pgfusepath{stroke}%
\end{pgfscope}%
\begin{pgfscope}%
\pgfpathrectangle{\pgfqpoint{7.394209in}{0.375000in}}{\pgfqpoint{6.356833in}{5.175000in}}%
\pgfusepath{clip}%
\pgfsetbuttcap%
\pgfsetroundjoin%
\pgfsetlinewidth{1.003750pt}%
\definecolor{currentstroke}{rgb}{0.827451,0.827451,0.827451}%
\pgfsetstrokecolor{currentstroke}%
\pgfsetdash{}{0pt}%
\pgfpathmoveto{\pgfqpoint{12.306939in}{4.988834in}}%
\pgfpathcurveto{\pgfqpoint{12.317989in}{4.988834in}}{\pgfqpoint{12.328588in}{4.993225in}}{\pgfqpoint{12.336402in}{5.001038in}}%
\pgfpathcurveto{\pgfqpoint{12.344216in}{5.008852in}}{\pgfqpoint{12.348606in}{5.019451in}}{\pgfqpoint{12.348606in}{5.030501in}}%
\pgfpathcurveto{\pgfqpoint{12.348606in}{5.041551in}}{\pgfqpoint{12.344216in}{5.052150in}}{\pgfqpoint{12.336402in}{5.059964in}}%
\pgfpathcurveto{\pgfqpoint{12.328588in}{5.067777in}}{\pgfqpoint{12.317989in}{5.072168in}}{\pgfqpoint{12.306939in}{5.072168in}}%
\pgfpathcurveto{\pgfqpoint{12.295889in}{5.072168in}}{\pgfqpoint{12.285290in}{5.067777in}}{\pgfqpoint{12.277476in}{5.059964in}}%
\pgfpathcurveto{\pgfqpoint{12.269663in}{5.052150in}}{\pgfqpoint{12.265272in}{5.041551in}}{\pgfqpoint{12.265272in}{5.030501in}}%
\pgfpathcurveto{\pgfqpoint{12.265272in}{5.019451in}}{\pgfqpoint{12.269663in}{5.008852in}}{\pgfqpoint{12.277476in}{5.001038in}}%
\pgfpathcurveto{\pgfqpoint{12.285290in}{4.993225in}}{\pgfqpoint{12.295889in}{4.988834in}}{\pgfqpoint{12.306939in}{4.988834in}}%
\pgfpathlineto{\pgfqpoint{12.306939in}{4.988834in}}%
\pgfpathclose%
\pgfusepath{stroke}%
\end{pgfscope}%
\begin{pgfscope}%
\pgfpathrectangle{\pgfqpoint{7.394209in}{0.375000in}}{\pgfqpoint{6.356833in}{5.175000in}}%
\pgfusepath{clip}%
\pgfsetbuttcap%
\pgfsetroundjoin%
\pgfsetlinewidth{1.003750pt}%
\definecolor{currentstroke}{rgb}{0.827451,0.827451,0.827451}%
\pgfsetstrokecolor{currentstroke}%
\pgfsetdash{}{0pt}%
\pgfpathmoveto{\pgfqpoint{9.494878in}{0.663715in}}%
\pgfpathcurveto{\pgfqpoint{9.505928in}{0.663715in}}{\pgfqpoint{9.516527in}{0.668105in}}{\pgfqpoint{9.524341in}{0.675918in}}%
\pgfpathcurveto{\pgfqpoint{9.532154in}{0.683732in}}{\pgfqpoint{9.536544in}{0.694331in}}{\pgfqpoint{9.536544in}{0.705381in}}%
\pgfpathcurveto{\pgfqpoint{9.536544in}{0.716431in}}{\pgfqpoint{9.532154in}{0.727030in}}{\pgfqpoint{9.524341in}{0.734844in}}%
\pgfpathcurveto{\pgfqpoint{9.516527in}{0.742658in}}{\pgfqpoint{9.505928in}{0.747048in}}{\pgfqpoint{9.494878in}{0.747048in}}%
\pgfpathcurveto{\pgfqpoint{9.483828in}{0.747048in}}{\pgfqpoint{9.473229in}{0.742658in}}{\pgfqpoint{9.465415in}{0.734844in}}%
\pgfpathcurveto{\pgfqpoint{9.457601in}{0.727030in}}{\pgfqpoint{9.453211in}{0.716431in}}{\pgfqpoint{9.453211in}{0.705381in}}%
\pgfpathcurveto{\pgfqpoint{9.453211in}{0.694331in}}{\pgfqpoint{9.457601in}{0.683732in}}{\pgfqpoint{9.465415in}{0.675918in}}%
\pgfpathcurveto{\pgfqpoint{9.473229in}{0.668105in}}{\pgfqpoint{9.483828in}{0.663715in}}{\pgfqpoint{9.494878in}{0.663715in}}%
\pgfpathlineto{\pgfqpoint{9.494878in}{0.663715in}}%
\pgfpathclose%
\pgfusepath{stroke}%
\end{pgfscope}%
\begin{pgfscope}%
\pgfpathrectangle{\pgfqpoint{7.394209in}{0.375000in}}{\pgfqpoint{6.356833in}{5.175000in}}%
\pgfusepath{clip}%
\pgfsetbuttcap%
\pgfsetroundjoin%
\pgfsetlinewidth{1.003750pt}%
\definecolor{currentstroke}{rgb}{0.827451,0.827451,0.827451}%
\pgfsetstrokecolor{currentstroke}%
\pgfsetdash{}{0pt}%
\pgfpathmoveto{\pgfqpoint{13.749892in}{5.073740in}}%
\pgfpathcurveto{\pgfqpoint{13.760942in}{5.073740in}}{\pgfqpoint{13.771541in}{5.078130in}}{\pgfqpoint{13.779354in}{5.085943in}}%
\pgfpathcurveto{\pgfqpoint{13.787168in}{5.093757in}}{\pgfqpoint{13.791558in}{5.104356in}}{\pgfqpoint{13.791558in}{5.115406in}}%
\pgfpathcurveto{\pgfqpoint{13.791558in}{5.126456in}}{\pgfqpoint{13.787168in}{5.137055in}}{\pgfqpoint{13.779354in}{5.144869in}}%
\pgfpathcurveto{\pgfqpoint{13.771541in}{5.152683in}}{\pgfqpoint{13.760942in}{5.157073in}}{\pgfqpoint{13.749892in}{5.157073in}}%
\pgfpathcurveto{\pgfqpoint{13.738841in}{5.157073in}}{\pgfqpoint{13.728242in}{5.152683in}}{\pgfqpoint{13.720429in}{5.144869in}}%
\pgfpathcurveto{\pgfqpoint{13.712615in}{5.137055in}}{\pgfqpoint{13.708225in}{5.126456in}}{\pgfqpoint{13.708225in}{5.115406in}}%
\pgfpathcurveto{\pgfqpoint{13.708225in}{5.104356in}}{\pgfqpoint{13.712615in}{5.093757in}}{\pgfqpoint{13.720429in}{5.085943in}}%
\pgfpathcurveto{\pgfqpoint{13.728242in}{5.078130in}}{\pgfqpoint{13.738841in}{5.073740in}}{\pgfqpoint{13.749892in}{5.073740in}}%
\pgfpathlineto{\pgfqpoint{13.749892in}{5.073740in}}%
\pgfpathclose%
\pgfusepath{stroke}%
\end{pgfscope}%
\begin{pgfscope}%
\pgfpathrectangle{\pgfqpoint{7.394209in}{0.375000in}}{\pgfqpoint{6.356833in}{5.175000in}}%
\pgfusepath{clip}%
\pgfsetbuttcap%
\pgfsetroundjoin%
\pgfsetlinewidth{1.003750pt}%
\definecolor{currentstroke}{rgb}{0.827451,0.827451,0.827451}%
\pgfsetstrokecolor{currentstroke}%
\pgfsetdash{}{0pt}%
\pgfpathmoveto{\pgfqpoint{9.019355in}{3.899320in}}%
\pgfpathcurveto{\pgfqpoint{9.030405in}{3.899320in}}{\pgfqpoint{9.041004in}{3.903710in}}{\pgfqpoint{9.048817in}{3.911524in}}%
\pgfpathcurveto{\pgfqpoint{9.056631in}{3.919338in}}{\pgfqpoint{9.061021in}{3.929937in}}{\pgfqpoint{9.061021in}{3.940987in}}%
\pgfpathcurveto{\pgfqpoint{9.061021in}{3.952037in}}{\pgfqpoint{9.056631in}{3.962636in}}{\pgfqpoint{9.048817in}{3.970450in}}%
\pgfpathcurveto{\pgfqpoint{9.041004in}{3.978263in}}{\pgfqpoint{9.030405in}{3.982654in}}{\pgfqpoint{9.019355in}{3.982654in}}%
\pgfpathcurveto{\pgfqpoint{9.008305in}{3.982654in}}{\pgfqpoint{8.997705in}{3.978263in}}{\pgfqpoint{8.989892in}{3.970450in}}%
\pgfpathcurveto{\pgfqpoint{8.982078in}{3.962636in}}{\pgfqpoint{8.977688in}{3.952037in}}{\pgfqpoint{8.977688in}{3.940987in}}%
\pgfpathcurveto{\pgfqpoint{8.977688in}{3.929937in}}{\pgfqpoint{8.982078in}{3.919338in}}{\pgfqpoint{8.989892in}{3.911524in}}%
\pgfpathcurveto{\pgfqpoint{8.997705in}{3.903710in}}{\pgfqpoint{9.008305in}{3.899320in}}{\pgfqpoint{9.019355in}{3.899320in}}%
\pgfpathlineto{\pgfqpoint{9.019355in}{3.899320in}}%
\pgfpathclose%
\pgfusepath{stroke}%
\end{pgfscope}%
\begin{pgfscope}%
\pgfpathrectangle{\pgfqpoint{7.394209in}{0.375000in}}{\pgfqpoint{6.356833in}{5.175000in}}%
\pgfusepath{clip}%
\pgfsetbuttcap%
\pgfsetroundjoin%
\pgfsetlinewidth{1.003750pt}%
\definecolor{currentstroke}{rgb}{0.827451,0.827451,0.827451}%
\pgfsetstrokecolor{currentstroke}%
\pgfsetdash{}{0pt}%
\pgfpathmoveto{\pgfqpoint{8.665567in}{3.908891in}}%
\pgfpathcurveto{\pgfqpoint{8.676617in}{3.908891in}}{\pgfqpoint{8.687216in}{3.913281in}}{\pgfqpoint{8.695030in}{3.921095in}}%
\pgfpathcurveto{\pgfqpoint{8.702843in}{3.928909in}}{\pgfqpoint{8.707234in}{3.939508in}}{\pgfqpoint{8.707234in}{3.950558in}}%
\pgfpathcurveto{\pgfqpoint{8.707234in}{3.961608in}}{\pgfqpoint{8.702843in}{3.972207in}}{\pgfqpoint{8.695030in}{3.980021in}}%
\pgfpathcurveto{\pgfqpoint{8.687216in}{3.987834in}}{\pgfqpoint{8.676617in}{3.992224in}}{\pgfqpoint{8.665567in}{3.992224in}}%
\pgfpathcurveto{\pgfqpoint{8.654517in}{3.992224in}}{\pgfqpoint{8.643918in}{3.987834in}}{\pgfqpoint{8.636104in}{3.980021in}}%
\pgfpathcurveto{\pgfqpoint{8.628291in}{3.972207in}}{\pgfqpoint{8.623900in}{3.961608in}}{\pgfqpoint{8.623900in}{3.950558in}}%
\pgfpathcurveto{\pgfqpoint{8.623900in}{3.939508in}}{\pgfqpoint{8.628291in}{3.928909in}}{\pgfqpoint{8.636104in}{3.921095in}}%
\pgfpathcurveto{\pgfqpoint{8.643918in}{3.913281in}}{\pgfqpoint{8.654517in}{3.908891in}}{\pgfqpoint{8.665567in}{3.908891in}}%
\pgfpathlineto{\pgfqpoint{8.665567in}{3.908891in}}%
\pgfpathclose%
\pgfusepath{stroke}%
\end{pgfscope}%
\begin{pgfscope}%
\pgfpathrectangle{\pgfqpoint{7.394209in}{0.375000in}}{\pgfqpoint{6.356833in}{5.175000in}}%
\pgfusepath{clip}%
\pgfsetbuttcap%
\pgfsetroundjoin%
\pgfsetlinewidth{1.003750pt}%
\definecolor{currentstroke}{rgb}{0.827451,0.827451,0.827451}%
\pgfsetstrokecolor{currentstroke}%
\pgfsetdash{}{0pt}%
\pgfpathmoveto{\pgfqpoint{11.797996in}{4.621886in}}%
\pgfpathcurveto{\pgfqpoint{11.809046in}{4.621886in}}{\pgfqpoint{11.819645in}{4.626276in}}{\pgfqpoint{11.827459in}{4.634090in}}%
\pgfpathcurveto{\pgfqpoint{11.835273in}{4.641904in}}{\pgfqpoint{11.839663in}{4.652503in}}{\pgfqpoint{11.839663in}{4.663553in}}%
\pgfpathcurveto{\pgfqpoint{11.839663in}{4.674603in}}{\pgfqpoint{11.835273in}{4.685202in}}{\pgfqpoint{11.827459in}{4.693016in}}%
\pgfpathcurveto{\pgfqpoint{11.819645in}{4.700829in}}{\pgfqpoint{11.809046in}{4.705220in}}{\pgfqpoint{11.797996in}{4.705220in}}%
\pgfpathcurveto{\pgfqpoint{11.786946in}{4.705220in}}{\pgfqpoint{11.776347in}{4.700829in}}{\pgfqpoint{11.768534in}{4.693016in}}%
\pgfpathcurveto{\pgfqpoint{11.760720in}{4.685202in}}{\pgfqpoint{11.756330in}{4.674603in}}{\pgfqpoint{11.756330in}{4.663553in}}%
\pgfpathcurveto{\pgfqpoint{11.756330in}{4.652503in}}{\pgfqpoint{11.760720in}{4.641904in}}{\pgfqpoint{11.768534in}{4.634090in}}%
\pgfpathcurveto{\pgfqpoint{11.776347in}{4.626276in}}{\pgfqpoint{11.786946in}{4.621886in}}{\pgfqpoint{11.797996in}{4.621886in}}%
\pgfpathlineto{\pgfqpoint{11.797996in}{4.621886in}}%
\pgfpathclose%
\pgfusepath{stroke}%
\end{pgfscope}%
\begin{pgfscope}%
\pgfpathrectangle{\pgfqpoint{7.394209in}{0.375000in}}{\pgfqpoint{6.356833in}{5.175000in}}%
\pgfusepath{clip}%
\pgfsetbuttcap%
\pgfsetroundjoin%
\pgfsetlinewidth{1.003750pt}%
\definecolor{currentstroke}{rgb}{0.827451,0.827451,0.827451}%
\pgfsetstrokecolor{currentstroke}%
\pgfsetdash{}{0pt}%
\pgfpathmoveto{\pgfqpoint{12.328261in}{4.263711in}}%
\pgfpathcurveto{\pgfqpoint{12.339311in}{4.263711in}}{\pgfqpoint{12.349910in}{4.268101in}}{\pgfqpoint{12.357723in}{4.275915in}}%
\pgfpathcurveto{\pgfqpoint{12.365537in}{4.283728in}}{\pgfqpoint{12.369927in}{4.294327in}}{\pgfqpoint{12.369927in}{4.305378in}}%
\pgfpathcurveto{\pgfqpoint{12.369927in}{4.316428in}}{\pgfqpoint{12.365537in}{4.327027in}}{\pgfqpoint{12.357723in}{4.334840in}}%
\pgfpathcurveto{\pgfqpoint{12.349910in}{4.342654in}}{\pgfqpoint{12.339311in}{4.347044in}}{\pgfqpoint{12.328261in}{4.347044in}}%
\pgfpathcurveto{\pgfqpoint{12.317210in}{4.347044in}}{\pgfqpoint{12.306611in}{4.342654in}}{\pgfqpoint{12.298798in}{4.334840in}}%
\pgfpathcurveto{\pgfqpoint{12.290984in}{4.327027in}}{\pgfqpoint{12.286594in}{4.316428in}}{\pgfqpoint{12.286594in}{4.305378in}}%
\pgfpathcurveto{\pgfqpoint{12.286594in}{4.294327in}}{\pgfqpoint{12.290984in}{4.283728in}}{\pgfqpoint{12.298798in}{4.275915in}}%
\pgfpathcurveto{\pgfqpoint{12.306611in}{4.268101in}}{\pgfqpoint{12.317210in}{4.263711in}}{\pgfqpoint{12.328261in}{4.263711in}}%
\pgfpathlineto{\pgfqpoint{12.328261in}{4.263711in}}%
\pgfpathclose%
\pgfusepath{stroke}%
\end{pgfscope}%
\begin{pgfscope}%
\pgfpathrectangle{\pgfqpoint{7.394209in}{0.375000in}}{\pgfqpoint{6.356833in}{5.175000in}}%
\pgfusepath{clip}%
\pgfsetbuttcap%
\pgfsetroundjoin%
\pgfsetlinewidth{1.003750pt}%
\definecolor{currentstroke}{rgb}{0.827451,0.827451,0.827451}%
\pgfsetstrokecolor{currentstroke}%
\pgfsetdash{}{0pt}%
\pgfpathmoveto{\pgfqpoint{10.774622in}{3.477690in}}%
\pgfpathcurveto{\pgfqpoint{10.785672in}{3.477690in}}{\pgfqpoint{10.796271in}{3.482080in}}{\pgfqpoint{10.804085in}{3.489893in}}%
\pgfpathcurveto{\pgfqpoint{10.811899in}{3.497707in}}{\pgfqpoint{10.816289in}{3.508306in}}{\pgfqpoint{10.816289in}{3.519356in}}%
\pgfpathcurveto{\pgfqpoint{10.816289in}{3.530406in}}{\pgfqpoint{10.811899in}{3.541005in}}{\pgfqpoint{10.804085in}{3.548819in}}%
\pgfpathcurveto{\pgfqpoint{10.796271in}{3.556633in}}{\pgfqpoint{10.785672in}{3.561023in}}{\pgfqpoint{10.774622in}{3.561023in}}%
\pgfpathcurveto{\pgfqpoint{10.763572in}{3.561023in}}{\pgfqpoint{10.752973in}{3.556633in}}{\pgfqpoint{10.745160in}{3.548819in}}%
\pgfpathcurveto{\pgfqpoint{10.737346in}{3.541005in}}{\pgfqpoint{10.732956in}{3.530406in}}{\pgfqpoint{10.732956in}{3.519356in}}%
\pgfpathcurveto{\pgfqpoint{10.732956in}{3.508306in}}{\pgfqpoint{10.737346in}{3.497707in}}{\pgfqpoint{10.745160in}{3.489893in}}%
\pgfpathcurveto{\pgfqpoint{10.752973in}{3.482080in}}{\pgfqpoint{10.763572in}{3.477690in}}{\pgfqpoint{10.774622in}{3.477690in}}%
\pgfpathlineto{\pgfqpoint{10.774622in}{3.477690in}}%
\pgfpathclose%
\pgfusepath{stroke}%
\end{pgfscope}%
\begin{pgfscope}%
\pgfpathrectangle{\pgfqpoint{7.394209in}{0.375000in}}{\pgfqpoint{6.356833in}{5.175000in}}%
\pgfusepath{clip}%
\pgfsetbuttcap%
\pgfsetroundjoin%
\pgfsetlinewidth{1.003750pt}%
\definecolor{currentstroke}{rgb}{0.827451,0.827451,0.827451}%
\pgfsetstrokecolor{currentstroke}%
\pgfsetdash{}{0pt}%
\pgfpathmoveto{\pgfqpoint{11.557952in}{2.868097in}}%
\pgfpathcurveto{\pgfqpoint{11.569002in}{2.868097in}}{\pgfqpoint{11.579601in}{2.872487in}}{\pgfqpoint{11.587415in}{2.880301in}}%
\pgfpathcurveto{\pgfqpoint{11.595228in}{2.888115in}}{\pgfqpoint{11.599619in}{2.898714in}}{\pgfqpoint{11.599619in}{2.909764in}}%
\pgfpathcurveto{\pgfqpoint{11.599619in}{2.920814in}}{\pgfqpoint{11.595228in}{2.931413in}}{\pgfqpoint{11.587415in}{2.939227in}}%
\pgfpathcurveto{\pgfqpoint{11.579601in}{2.947040in}}{\pgfqpoint{11.569002in}{2.951430in}}{\pgfqpoint{11.557952in}{2.951430in}}%
\pgfpathcurveto{\pgfqpoint{11.546902in}{2.951430in}}{\pgfqpoint{11.536303in}{2.947040in}}{\pgfqpoint{11.528489in}{2.939227in}}%
\pgfpathcurveto{\pgfqpoint{11.520676in}{2.931413in}}{\pgfqpoint{11.516285in}{2.920814in}}{\pgfqpoint{11.516285in}{2.909764in}}%
\pgfpathcurveto{\pgfqpoint{11.516285in}{2.898714in}}{\pgfqpoint{11.520676in}{2.888115in}}{\pgfqpoint{11.528489in}{2.880301in}}%
\pgfpathcurveto{\pgfqpoint{11.536303in}{2.872487in}}{\pgfqpoint{11.546902in}{2.868097in}}{\pgfqpoint{11.557952in}{2.868097in}}%
\pgfpathlineto{\pgfqpoint{11.557952in}{2.868097in}}%
\pgfpathclose%
\pgfusepath{stroke}%
\end{pgfscope}%
\begin{pgfscope}%
\pgfpathrectangle{\pgfqpoint{7.394209in}{0.375000in}}{\pgfqpoint{6.356833in}{5.175000in}}%
\pgfusepath{clip}%
\pgfsetbuttcap%
\pgfsetroundjoin%
\pgfsetlinewidth{1.003750pt}%
\definecolor{currentstroke}{rgb}{0.827451,0.827451,0.827451}%
\pgfsetstrokecolor{currentstroke}%
\pgfsetdash{}{0pt}%
\pgfpathmoveto{\pgfqpoint{11.613995in}{3.805531in}}%
\pgfpathcurveto{\pgfqpoint{11.625045in}{3.805531in}}{\pgfqpoint{11.635644in}{3.809921in}}{\pgfqpoint{11.643458in}{3.817735in}}%
\pgfpathcurveto{\pgfqpoint{11.651271in}{3.825549in}}{\pgfqpoint{11.655662in}{3.836148in}}{\pgfqpoint{11.655662in}{3.847198in}}%
\pgfpathcurveto{\pgfqpoint{11.655662in}{3.858248in}}{\pgfqpoint{11.651271in}{3.868847in}}{\pgfqpoint{11.643458in}{3.876661in}}%
\pgfpathcurveto{\pgfqpoint{11.635644in}{3.884474in}}{\pgfqpoint{11.625045in}{3.888864in}}{\pgfqpoint{11.613995in}{3.888864in}}%
\pgfpathcurveto{\pgfqpoint{11.602945in}{3.888864in}}{\pgfqpoint{11.592346in}{3.884474in}}{\pgfqpoint{11.584532in}{3.876661in}}%
\pgfpathcurveto{\pgfqpoint{11.576719in}{3.868847in}}{\pgfqpoint{11.572328in}{3.858248in}}{\pgfqpoint{11.572328in}{3.847198in}}%
\pgfpathcurveto{\pgfqpoint{11.572328in}{3.836148in}}{\pgfqpoint{11.576719in}{3.825549in}}{\pgfqpoint{11.584532in}{3.817735in}}%
\pgfpathcurveto{\pgfqpoint{11.592346in}{3.809921in}}{\pgfqpoint{11.602945in}{3.805531in}}{\pgfqpoint{11.613995in}{3.805531in}}%
\pgfpathlineto{\pgfqpoint{11.613995in}{3.805531in}}%
\pgfpathclose%
\pgfusepath{stroke}%
\end{pgfscope}%
\begin{pgfscope}%
\pgfpathrectangle{\pgfqpoint{7.394209in}{0.375000in}}{\pgfqpoint{6.356833in}{5.175000in}}%
\pgfusepath{clip}%
\pgfsetbuttcap%
\pgfsetroundjoin%
\pgfsetlinewidth{1.003750pt}%
\definecolor{currentstroke}{rgb}{0.827451,0.827451,0.827451}%
\pgfsetstrokecolor{currentstroke}%
\pgfsetdash{}{0pt}%
\pgfpathmoveto{\pgfqpoint{9.448068in}{4.789958in}}%
\pgfpathcurveto{\pgfqpoint{9.459119in}{4.789958in}}{\pgfqpoint{9.469718in}{4.794348in}}{\pgfqpoint{9.477531in}{4.802162in}}%
\pgfpathcurveto{\pgfqpoint{9.485345in}{4.809975in}}{\pgfqpoint{9.489735in}{4.820574in}}{\pgfqpoint{9.489735in}{4.831624in}}%
\pgfpathcurveto{\pgfqpoint{9.489735in}{4.842674in}}{\pgfqpoint{9.485345in}{4.853273in}}{\pgfqpoint{9.477531in}{4.861087in}}%
\pgfpathcurveto{\pgfqpoint{9.469718in}{4.868901in}}{\pgfqpoint{9.459119in}{4.873291in}}{\pgfqpoint{9.448068in}{4.873291in}}%
\pgfpathcurveto{\pgfqpoint{9.437018in}{4.873291in}}{\pgfqpoint{9.426419in}{4.868901in}}{\pgfqpoint{9.418606in}{4.861087in}}%
\pgfpathcurveto{\pgfqpoint{9.410792in}{4.853273in}}{\pgfqpoint{9.406402in}{4.842674in}}{\pgfqpoint{9.406402in}{4.831624in}}%
\pgfpathcurveto{\pgfqpoint{9.406402in}{4.820574in}}{\pgfqpoint{9.410792in}{4.809975in}}{\pgfqpoint{9.418606in}{4.802162in}}%
\pgfpathcurveto{\pgfqpoint{9.426419in}{4.794348in}}{\pgfqpoint{9.437018in}{4.789958in}}{\pgfqpoint{9.448068in}{4.789958in}}%
\pgfpathlineto{\pgfqpoint{9.448068in}{4.789958in}}%
\pgfpathclose%
\pgfusepath{stroke}%
\end{pgfscope}%
\begin{pgfscope}%
\pgfpathrectangle{\pgfqpoint{7.394209in}{0.375000in}}{\pgfqpoint{6.356833in}{5.175000in}}%
\pgfusepath{clip}%
\pgfsetbuttcap%
\pgfsetroundjoin%
\pgfsetlinewidth{1.003750pt}%
\definecolor{currentstroke}{rgb}{0.827451,0.827451,0.827451}%
\pgfsetstrokecolor{currentstroke}%
\pgfsetdash{}{0pt}%
\pgfpathmoveto{\pgfqpoint{13.447157in}{4.954512in}}%
\pgfpathcurveto{\pgfqpoint{13.458207in}{4.954512in}}{\pgfqpoint{13.468806in}{4.958902in}}{\pgfqpoint{13.476619in}{4.966715in}}%
\pgfpathcurveto{\pgfqpoint{13.484433in}{4.974529in}}{\pgfqpoint{13.488823in}{4.985128in}}{\pgfqpoint{13.488823in}{4.996178in}}%
\pgfpathcurveto{\pgfqpoint{13.488823in}{5.007228in}}{\pgfqpoint{13.484433in}{5.017827in}}{\pgfqpoint{13.476619in}{5.025641in}}%
\pgfpathcurveto{\pgfqpoint{13.468806in}{5.033455in}}{\pgfqpoint{13.458207in}{5.037845in}}{\pgfqpoint{13.447157in}{5.037845in}}%
\pgfpathcurveto{\pgfqpoint{13.436106in}{5.037845in}}{\pgfqpoint{13.425507in}{5.033455in}}{\pgfqpoint{13.417694in}{5.025641in}}%
\pgfpathcurveto{\pgfqpoint{13.409880in}{5.017827in}}{\pgfqpoint{13.405490in}{5.007228in}}{\pgfqpoint{13.405490in}{4.996178in}}%
\pgfpathcurveto{\pgfqpoint{13.405490in}{4.985128in}}{\pgfqpoint{13.409880in}{4.974529in}}{\pgfqpoint{13.417694in}{4.966715in}}%
\pgfpathcurveto{\pgfqpoint{13.425507in}{4.958902in}}{\pgfqpoint{13.436106in}{4.954512in}}{\pgfqpoint{13.447157in}{4.954512in}}%
\pgfpathlineto{\pgfqpoint{13.447157in}{4.954512in}}%
\pgfpathclose%
\pgfusepath{stroke}%
\end{pgfscope}%
\begin{pgfscope}%
\pgfpathrectangle{\pgfqpoint{7.394209in}{0.375000in}}{\pgfqpoint{6.356833in}{5.175000in}}%
\pgfusepath{clip}%
\pgfsetbuttcap%
\pgfsetroundjoin%
\pgfsetlinewidth{1.003750pt}%
\definecolor{currentstroke}{rgb}{0.827451,0.827451,0.827451}%
\pgfsetstrokecolor{currentstroke}%
\pgfsetdash{}{0pt}%
\pgfpathmoveto{\pgfqpoint{12.471468in}{5.241144in}}%
\pgfpathcurveto{\pgfqpoint{12.482518in}{5.241144in}}{\pgfqpoint{12.493117in}{5.245535in}}{\pgfqpoint{12.500931in}{5.253348in}}%
\pgfpathcurveto{\pgfqpoint{12.508744in}{5.261162in}}{\pgfqpoint{12.513135in}{5.271761in}}{\pgfqpoint{12.513135in}{5.282811in}}%
\pgfpathcurveto{\pgfqpoint{12.513135in}{5.293861in}}{\pgfqpoint{12.508744in}{5.304460in}}{\pgfqpoint{12.500931in}{5.312274in}}%
\pgfpathcurveto{\pgfqpoint{12.493117in}{5.320087in}}{\pgfqpoint{12.482518in}{5.324478in}}{\pgfqpoint{12.471468in}{5.324478in}}%
\pgfpathcurveto{\pgfqpoint{12.460418in}{5.324478in}}{\pgfqpoint{12.449819in}{5.320087in}}{\pgfqpoint{12.442005in}{5.312274in}}%
\pgfpathcurveto{\pgfqpoint{12.434192in}{5.304460in}}{\pgfqpoint{12.429801in}{5.293861in}}{\pgfqpoint{12.429801in}{5.282811in}}%
\pgfpathcurveto{\pgfqpoint{12.429801in}{5.271761in}}{\pgfqpoint{12.434192in}{5.261162in}}{\pgfqpoint{12.442005in}{5.253348in}}%
\pgfpathcurveto{\pgfqpoint{12.449819in}{5.245535in}}{\pgfqpoint{12.460418in}{5.241144in}}{\pgfqpoint{12.471468in}{5.241144in}}%
\pgfpathlineto{\pgfqpoint{12.471468in}{5.241144in}}%
\pgfpathclose%
\pgfusepath{stroke}%
\end{pgfscope}%
\begin{pgfscope}%
\pgfpathrectangle{\pgfqpoint{7.394209in}{0.375000in}}{\pgfqpoint{6.356833in}{5.175000in}}%
\pgfusepath{clip}%
\pgfsetbuttcap%
\pgfsetroundjoin%
\pgfsetlinewidth{1.003750pt}%
\definecolor{currentstroke}{rgb}{0.827451,0.827451,0.827451}%
\pgfsetstrokecolor{currentstroke}%
\pgfsetdash{}{0pt}%
\pgfpathmoveto{\pgfqpoint{11.517976in}{3.686275in}}%
\pgfpathcurveto{\pgfqpoint{11.529026in}{3.686275in}}{\pgfqpoint{11.539625in}{3.690666in}}{\pgfqpoint{11.547439in}{3.698479in}}%
\pgfpathcurveto{\pgfqpoint{11.555253in}{3.706293in}}{\pgfqpoint{11.559643in}{3.716892in}}{\pgfqpoint{11.559643in}{3.727942in}}%
\pgfpathcurveto{\pgfqpoint{11.559643in}{3.738992in}}{\pgfqpoint{11.555253in}{3.749591in}}{\pgfqpoint{11.547439in}{3.757405in}}%
\pgfpathcurveto{\pgfqpoint{11.539625in}{3.765218in}}{\pgfqpoint{11.529026in}{3.769609in}}{\pgfqpoint{11.517976in}{3.769609in}}%
\pgfpathcurveto{\pgfqpoint{11.506926in}{3.769609in}}{\pgfqpoint{11.496327in}{3.765218in}}{\pgfqpoint{11.488514in}{3.757405in}}%
\pgfpathcurveto{\pgfqpoint{11.480700in}{3.749591in}}{\pgfqpoint{11.476310in}{3.738992in}}{\pgfqpoint{11.476310in}{3.727942in}}%
\pgfpathcurveto{\pgfqpoint{11.476310in}{3.716892in}}{\pgfqpoint{11.480700in}{3.706293in}}{\pgfqpoint{11.488514in}{3.698479in}}%
\pgfpathcurveto{\pgfqpoint{11.496327in}{3.690666in}}{\pgfqpoint{11.506926in}{3.686275in}}{\pgfqpoint{11.517976in}{3.686275in}}%
\pgfpathlineto{\pgfqpoint{11.517976in}{3.686275in}}%
\pgfpathclose%
\pgfusepath{stroke}%
\end{pgfscope}%
\begin{pgfscope}%
\pgfpathrectangle{\pgfqpoint{7.394209in}{0.375000in}}{\pgfqpoint{6.356833in}{5.175000in}}%
\pgfusepath{clip}%
\pgfsetbuttcap%
\pgfsetroundjoin%
\pgfsetlinewidth{1.003750pt}%
\definecolor{currentstroke}{rgb}{0.827451,0.827451,0.827451}%
\pgfsetstrokecolor{currentstroke}%
\pgfsetdash{}{0pt}%
\pgfpathmoveto{\pgfqpoint{12.373499in}{4.619221in}}%
\pgfpathcurveto{\pgfqpoint{12.384549in}{4.619221in}}{\pgfqpoint{12.395148in}{4.623611in}}{\pgfqpoint{12.402962in}{4.631425in}}%
\pgfpathcurveto{\pgfqpoint{12.410775in}{4.639238in}}{\pgfqpoint{12.415166in}{4.649837in}}{\pgfqpoint{12.415166in}{4.660887in}}%
\pgfpathcurveto{\pgfqpoint{12.415166in}{4.671938in}}{\pgfqpoint{12.410775in}{4.682537in}}{\pgfqpoint{12.402962in}{4.690350in}}%
\pgfpathcurveto{\pgfqpoint{12.395148in}{4.698164in}}{\pgfqpoint{12.384549in}{4.702554in}}{\pgfqpoint{12.373499in}{4.702554in}}%
\pgfpathcurveto{\pgfqpoint{12.362449in}{4.702554in}}{\pgfqpoint{12.351850in}{4.698164in}}{\pgfqpoint{12.344036in}{4.690350in}}%
\pgfpathcurveto{\pgfqpoint{12.336222in}{4.682537in}}{\pgfqpoint{12.331832in}{4.671938in}}{\pgfqpoint{12.331832in}{4.660887in}}%
\pgfpathcurveto{\pgfqpoint{12.331832in}{4.649837in}}{\pgfqpoint{12.336222in}{4.639238in}}{\pgfqpoint{12.344036in}{4.631425in}}%
\pgfpathcurveto{\pgfqpoint{12.351850in}{4.623611in}}{\pgfqpoint{12.362449in}{4.619221in}}{\pgfqpoint{12.373499in}{4.619221in}}%
\pgfpathlineto{\pgfqpoint{12.373499in}{4.619221in}}%
\pgfpathclose%
\pgfusepath{stroke}%
\end{pgfscope}%
\begin{pgfscope}%
\pgfpathrectangle{\pgfqpoint{7.394209in}{0.375000in}}{\pgfqpoint{6.356833in}{5.175000in}}%
\pgfusepath{clip}%
\pgfsetbuttcap%
\pgfsetroundjoin%
\pgfsetlinewidth{1.003750pt}%
\definecolor{currentstroke}{rgb}{0.827451,0.827451,0.827451}%
\pgfsetstrokecolor{currentstroke}%
\pgfsetdash{}{0pt}%
\pgfpathmoveto{\pgfqpoint{13.120481in}{5.454756in}}%
\pgfpathcurveto{\pgfqpoint{13.131531in}{5.454756in}}{\pgfqpoint{13.142130in}{5.459146in}}{\pgfqpoint{13.149943in}{5.466960in}}%
\pgfpathcurveto{\pgfqpoint{13.157757in}{5.474773in}}{\pgfqpoint{13.162147in}{5.485372in}}{\pgfqpoint{13.162147in}{5.496422in}}%
\pgfpathcurveto{\pgfqpoint{13.162147in}{5.507473in}}{\pgfqpoint{13.157757in}{5.518072in}}{\pgfqpoint{13.149943in}{5.525885in}}%
\pgfpathcurveto{\pgfqpoint{13.142130in}{5.533699in}}{\pgfqpoint{13.131531in}{5.538089in}}{\pgfqpoint{13.120481in}{5.538089in}}%
\pgfpathcurveto{\pgfqpoint{13.109431in}{5.538089in}}{\pgfqpoint{13.098831in}{5.533699in}}{\pgfqpoint{13.091018in}{5.525885in}}%
\pgfpathcurveto{\pgfqpoint{13.083204in}{5.518072in}}{\pgfqpoint{13.078814in}{5.507473in}}{\pgfqpoint{13.078814in}{5.496422in}}%
\pgfpathcurveto{\pgfqpoint{13.078814in}{5.485372in}}{\pgfqpoint{13.083204in}{5.474773in}}{\pgfqpoint{13.091018in}{5.466960in}}%
\pgfpathcurveto{\pgfqpoint{13.098831in}{5.459146in}}{\pgfqpoint{13.109431in}{5.454756in}}{\pgfqpoint{13.120481in}{5.454756in}}%
\pgfpathlineto{\pgfqpoint{13.120481in}{5.454756in}}%
\pgfpathclose%
\pgfusepath{stroke}%
\end{pgfscope}%
\begin{pgfscope}%
\pgfpathrectangle{\pgfqpoint{7.394209in}{0.375000in}}{\pgfqpoint{6.356833in}{5.175000in}}%
\pgfusepath{clip}%
\pgfsetbuttcap%
\pgfsetroundjoin%
\pgfsetlinewidth{1.003750pt}%
\definecolor{currentstroke}{rgb}{0.827451,0.827451,0.827451}%
\pgfsetstrokecolor{currentstroke}%
\pgfsetdash{}{0pt}%
\pgfpathmoveto{\pgfqpoint{10.217378in}{3.197050in}}%
\pgfpathcurveto{\pgfqpoint{10.228428in}{3.197050in}}{\pgfqpoint{10.239027in}{3.201440in}}{\pgfqpoint{10.246841in}{3.209254in}}%
\pgfpathcurveto{\pgfqpoint{10.254654in}{3.217067in}}{\pgfqpoint{10.259044in}{3.227666in}}{\pgfqpoint{10.259044in}{3.238717in}}%
\pgfpathcurveto{\pgfqpoint{10.259044in}{3.249767in}}{\pgfqpoint{10.254654in}{3.260366in}}{\pgfqpoint{10.246841in}{3.268179in}}%
\pgfpathcurveto{\pgfqpoint{10.239027in}{3.275993in}}{\pgfqpoint{10.228428in}{3.280383in}}{\pgfqpoint{10.217378in}{3.280383in}}%
\pgfpathcurveto{\pgfqpoint{10.206328in}{3.280383in}}{\pgfqpoint{10.195729in}{3.275993in}}{\pgfqpoint{10.187915in}{3.268179in}}%
\pgfpathcurveto{\pgfqpoint{10.180101in}{3.260366in}}{\pgfqpoint{10.175711in}{3.249767in}}{\pgfqpoint{10.175711in}{3.238717in}}%
\pgfpathcurveto{\pgfqpoint{10.175711in}{3.227666in}}{\pgfqpoint{10.180101in}{3.217067in}}{\pgfqpoint{10.187915in}{3.209254in}}%
\pgfpathcurveto{\pgfqpoint{10.195729in}{3.201440in}}{\pgfqpoint{10.206328in}{3.197050in}}{\pgfqpoint{10.217378in}{3.197050in}}%
\pgfpathlineto{\pgfqpoint{10.217378in}{3.197050in}}%
\pgfpathclose%
\pgfusepath{stroke}%
\end{pgfscope}%
\begin{pgfscope}%
\pgfpathrectangle{\pgfqpoint{7.394209in}{0.375000in}}{\pgfqpoint{6.356833in}{5.175000in}}%
\pgfusepath{clip}%
\pgfsetbuttcap%
\pgfsetroundjoin%
\pgfsetlinewidth{1.003750pt}%
\definecolor{currentstroke}{rgb}{0.827451,0.827451,0.827451}%
\pgfsetstrokecolor{currentstroke}%
\pgfsetdash{}{0pt}%
\pgfpathmoveto{\pgfqpoint{8.576907in}{3.290890in}}%
\pgfpathcurveto{\pgfqpoint{8.587957in}{3.290890in}}{\pgfqpoint{8.598556in}{3.295280in}}{\pgfqpoint{8.606370in}{3.303093in}}%
\pgfpathcurveto{\pgfqpoint{8.614183in}{3.310907in}}{\pgfqpoint{8.618574in}{3.321506in}}{\pgfqpoint{8.618574in}{3.332556in}}%
\pgfpathcurveto{\pgfqpoint{8.618574in}{3.343606in}}{\pgfqpoint{8.614183in}{3.354205in}}{\pgfqpoint{8.606370in}{3.362019in}}%
\pgfpathcurveto{\pgfqpoint{8.598556in}{3.369833in}}{\pgfqpoint{8.587957in}{3.374223in}}{\pgfqpoint{8.576907in}{3.374223in}}%
\pgfpathcurveto{\pgfqpoint{8.565857in}{3.374223in}}{\pgfqpoint{8.555258in}{3.369833in}}{\pgfqpoint{8.547444in}{3.362019in}}%
\pgfpathcurveto{\pgfqpoint{8.539631in}{3.354205in}}{\pgfqpoint{8.535240in}{3.343606in}}{\pgfqpoint{8.535240in}{3.332556in}}%
\pgfpathcurveto{\pgfqpoint{8.535240in}{3.321506in}}{\pgfqpoint{8.539631in}{3.310907in}}{\pgfqpoint{8.547444in}{3.303093in}}%
\pgfpathcurveto{\pgfqpoint{8.555258in}{3.295280in}}{\pgfqpoint{8.565857in}{3.290890in}}{\pgfqpoint{8.576907in}{3.290890in}}%
\pgfpathlineto{\pgfqpoint{8.576907in}{3.290890in}}%
\pgfpathclose%
\pgfusepath{stroke}%
\end{pgfscope}%
\begin{pgfscope}%
\pgfpathrectangle{\pgfqpoint{7.394209in}{0.375000in}}{\pgfqpoint{6.356833in}{5.175000in}}%
\pgfusepath{clip}%
\pgfsetbuttcap%
\pgfsetroundjoin%
\pgfsetlinewidth{1.003750pt}%
\definecolor{currentstroke}{rgb}{0.827451,0.827451,0.827451}%
\pgfsetstrokecolor{currentstroke}%
\pgfsetdash{}{0pt}%
\pgfpathmoveto{\pgfqpoint{7.911275in}{0.627011in}}%
\pgfpathcurveto{\pgfqpoint{7.922325in}{0.627011in}}{\pgfqpoint{7.932924in}{0.631401in}}{\pgfqpoint{7.940738in}{0.639215in}}%
\pgfpathcurveto{\pgfqpoint{7.948552in}{0.647028in}}{\pgfqpoint{7.952942in}{0.657627in}}{\pgfqpoint{7.952942in}{0.668678in}}%
\pgfpathcurveto{\pgfqpoint{7.952942in}{0.679728in}}{\pgfqpoint{7.948552in}{0.690327in}}{\pgfqpoint{7.940738in}{0.698140in}}%
\pgfpathcurveto{\pgfqpoint{7.932924in}{0.705954in}}{\pgfqpoint{7.922325in}{0.710344in}}{\pgfqpoint{7.911275in}{0.710344in}}%
\pgfpathcurveto{\pgfqpoint{7.900225in}{0.710344in}}{\pgfqpoint{7.889626in}{0.705954in}}{\pgfqpoint{7.881812in}{0.698140in}}%
\pgfpathcurveto{\pgfqpoint{7.873999in}{0.690327in}}{\pgfqpoint{7.869609in}{0.679728in}}{\pgfqpoint{7.869609in}{0.668678in}}%
\pgfpathcurveto{\pgfqpoint{7.869609in}{0.657627in}}{\pgfqpoint{7.873999in}{0.647028in}}{\pgfqpoint{7.881812in}{0.639215in}}%
\pgfpathcurveto{\pgfqpoint{7.889626in}{0.631401in}}{\pgfqpoint{7.900225in}{0.627011in}}{\pgfqpoint{7.911275in}{0.627011in}}%
\pgfpathlineto{\pgfqpoint{7.911275in}{0.627011in}}%
\pgfpathclose%
\pgfusepath{stroke}%
\end{pgfscope}%
\begin{pgfscope}%
\pgfpathrectangle{\pgfqpoint{7.394209in}{0.375000in}}{\pgfqpoint{6.356833in}{5.175000in}}%
\pgfusepath{clip}%
\pgfsetbuttcap%
\pgfsetroundjoin%
\pgfsetlinewidth{1.003750pt}%
\definecolor{currentstroke}{rgb}{0.827451,0.827451,0.827451}%
\pgfsetstrokecolor{currentstroke}%
\pgfsetdash{}{0pt}%
\pgfpathmoveto{\pgfqpoint{10.304442in}{5.174183in}}%
\pgfpathcurveto{\pgfqpoint{10.315492in}{5.174183in}}{\pgfqpoint{10.326091in}{5.178574in}}{\pgfqpoint{10.333904in}{5.186387in}}%
\pgfpathcurveto{\pgfqpoint{10.341718in}{5.194201in}}{\pgfqpoint{10.346108in}{5.204800in}}{\pgfqpoint{10.346108in}{5.215850in}}%
\pgfpathcurveto{\pgfqpoint{10.346108in}{5.226900in}}{\pgfqpoint{10.341718in}{5.237499in}}{\pgfqpoint{10.333904in}{5.245313in}}%
\pgfpathcurveto{\pgfqpoint{10.326091in}{5.253127in}}{\pgfqpoint{10.315492in}{5.257517in}}{\pgfqpoint{10.304442in}{5.257517in}}%
\pgfpathcurveto{\pgfqpoint{10.293391in}{5.257517in}}{\pgfqpoint{10.282792in}{5.253127in}}{\pgfqpoint{10.274979in}{5.245313in}}%
\pgfpathcurveto{\pgfqpoint{10.267165in}{5.237499in}}{\pgfqpoint{10.262775in}{5.226900in}}{\pgfqpoint{10.262775in}{5.215850in}}%
\pgfpathcurveto{\pgfqpoint{10.262775in}{5.204800in}}{\pgfqpoint{10.267165in}{5.194201in}}{\pgfqpoint{10.274979in}{5.186387in}}%
\pgfpathcurveto{\pgfqpoint{10.282792in}{5.178574in}}{\pgfqpoint{10.293391in}{5.174183in}}{\pgfqpoint{10.304442in}{5.174183in}}%
\pgfpathlineto{\pgfqpoint{10.304442in}{5.174183in}}%
\pgfpathclose%
\pgfusepath{stroke}%
\end{pgfscope}%
\begin{pgfscope}%
\pgfpathrectangle{\pgfqpoint{7.394209in}{0.375000in}}{\pgfqpoint{6.356833in}{5.175000in}}%
\pgfusepath{clip}%
\pgfsetbuttcap%
\pgfsetroundjoin%
\pgfsetlinewidth{1.003750pt}%
\definecolor{currentstroke}{rgb}{0.827451,0.827451,0.827451}%
\pgfsetstrokecolor{currentstroke}%
\pgfsetdash{}{0pt}%
\pgfpathmoveto{\pgfqpoint{9.726300in}{1.794715in}}%
\pgfpathcurveto{\pgfqpoint{9.737351in}{1.794715in}}{\pgfqpoint{9.747950in}{1.799105in}}{\pgfqpoint{9.755763in}{1.806919in}}%
\pgfpathcurveto{\pgfqpoint{9.763577in}{1.814732in}}{\pgfqpoint{9.767967in}{1.825331in}}{\pgfqpoint{9.767967in}{1.836381in}}%
\pgfpathcurveto{\pgfqpoint{9.767967in}{1.847431in}}{\pgfqpoint{9.763577in}{1.858030in}}{\pgfqpoint{9.755763in}{1.865844in}}%
\pgfpathcurveto{\pgfqpoint{9.747950in}{1.873658in}}{\pgfqpoint{9.737351in}{1.878048in}}{\pgfqpoint{9.726300in}{1.878048in}}%
\pgfpathcurveto{\pgfqpoint{9.715250in}{1.878048in}}{\pgfqpoint{9.704651in}{1.873658in}}{\pgfqpoint{9.696838in}{1.865844in}}%
\pgfpathcurveto{\pgfqpoint{9.689024in}{1.858030in}}{\pgfqpoint{9.684634in}{1.847431in}}{\pgfqpoint{9.684634in}{1.836381in}}%
\pgfpathcurveto{\pgfqpoint{9.684634in}{1.825331in}}{\pgfqpoint{9.689024in}{1.814732in}}{\pgfqpoint{9.696838in}{1.806919in}}%
\pgfpathcurveto{\pgfqpoint{9.704651in}{1.799105in}}{\pgfqpoint{9.715250in}{1.794715in}}{\pgfqpoint{9.726300in}{1.794715in}}%
\pgfpathlineto{\pgfqpoint{9.726300in}{1.794715in}}%
\pgfpathclose%
\pgfusepath{stroke}%
\end{pgfscope}%
\begin{pgfscope}%
\pgfpathrectangle{\pgfqpoint{7.394209in}{0.375000in}}{\pgfqpoint{6.356833in}{5.175000in}}%
\pgfusepath{clip}%
\pgfsetbuttcap%
\pgfsetroundjoin%
\pgfsetlinewidth{1.003750pt}%
\definecolor{currentstroke}{rgb}{0.827451,0.827451,0.827451}%
\pgfsetstrokecolor{currentstroke}%
\pgfsetdash{}{0pt}%
\pgfpathmoveto{\pgfqpoint{9.606474in}{1.370894in}}%
\pgfpathcurveto{\pgfqpoint{9.617524in}{1.370894in}}{\pgfqpoint{9.628124in}{1.375285in}}{\pgfqpoint{9.635937in}{1.383098in}}%
\pgfpathcurveto{\pgfqpoint{9.643751in}{1.390912in}}{\pgfqpoint{9.648141in}{1.401511in}}{\pgfqpoint{9.648141in}{1.412561in}}%
\pgfpathcurveto{\pgfqpoint{9.648141in}{1.423611in}}{\pgfqpoint{9.643751in}{1.434210in}}{\pgfqpoint{9.635937in}{1.442024in}}%
\pgfpathcurveto{\pgfqpoint{9.628124in}{1.449837in}}{\pgfqpoint{9.617524in}{1.454228in}}{\pgfqpoint{9.606474in}{1.454228in}}%
\pgfpathcurveto{\pgfqpoint{9.595424in}{1.454228in}}{\pgfqpoint{9.584825in}{1.449837in}}{\pgfqpoint{9.577012in}{1.442024in}}%
\pgfpathcurveto{\pgfqpoint{9.569198in}{1.434210in}}{\pgfqpoint{9.564808in}{1.423611in}}{\pgfqpoint{9.564808in}{1.412561in}}%
\pgfpathcurveto{\pgfqpoint{9.564808in}{1.401511in}}{\pgfqpoint{9.569198in}{1.390912in}}{\pgfqpoint{9.577012in}{1.383098in}}%
\pgfpathcurveto{\pgfqpoint{9.584825in}{1.375285in}}{\pgfqpoint{9.595424in}{1.370894in}}{\pgfqpoint{9.606474in}{1.370894in}}%
\pgfpathlineto{\pgfqpoint{9.606474in}{1.370894in}}%
\pgfpathclose%
\pgfusepath{stroke}%
\end{pgfscope}%
\begin{pgfscope}%
\pgfpathrectangle{\pgfqpoint{7.394209in}{0.375000in}}{\pgfqpoint{6.356833in}{5.175000in}}%
\pgfusepath{clip}%
\pgfsetbuttcap%
\pgfsetroundjoin%
\pgfsetlinewidth{1.003750pt}%
\definecolor{currentstroke}{rgb}{0.827451,0.827451,0.827451}%
\pgfsetstrokecolor{currentstroke}%
\pgfsetdash{}{0pt}%
\pgfpathmoveto{\pgfqpoint{12.008953in}{5.194176in}}%
\pgfpathcurveto{\pgfqpoint{12.020003in}{5.194176in}}{\pgfqpoint{12.030602in}{5.198567in}}{\pgfqpoint{12.038416in}{5.206380in}}%
\pgfpathcurveto{\pgfqpoint{12.046229in}{5.214194in}}{\pgfqpoint{12.050620in}{5.224793in}}{\pgfqpoint{12.050620in}{5.235843in}}%
\pgfpathcurveto{\pgfqpoint{12.050620in}{5.246893in}}{\pgfqpoint{12.046229in}{5.257492in}}{\pgfqpoint{12.038416in}{5.265306in}}%
\pgfpathcurveto{\pgfqpoint{12.030602in}{5.273120in}}{\pgfqpoint{12.020003in}{5.277510in}}{\pgfqpoint{12.008953in}{5.277510in}}%
\pgfpathcurveto{\pgfqpoint{11.997903in}{5.277510in}}{\pgfqpoint{11.987304in}{5.273120in}}{\pgfqpoint{11.979490in}{5.265306in}}%
\pgfpathcurveto{\pgfqpoint{11.971677in}{5.257492in}}{\pgfqpoint{11.967286in}{5.246893in}}{\pgfqpoint{11.967286in}{5.235843in}}%
\pgfpathcurveto{\pgfqpoint{11.967286in}{5.224793in}}{\pgfqpoint{11.971677in}{5.214194in}}{\pgfqpoint{11.979490in}{5.206380in}}%
\pgfpathcurveto{\pgfqpoint{11.987304in}{5.198567in}}{\pgfqpoint{11.997903in}{5.194176in}}{\pgfqpoint{12.008953in}{5.194176in}}%
\pgfpathlineto{\pgfqpoint{12.008953in}{5.194176in}}%
\pgfpathclose%
\pgfusepath{stroke}%
\end{pgfscope}%
\begin{pgfscope}%
\pgfpathrectangle{\pgfqpoint{7.394209in}{0.375000in}}{\pgfqpoint{6.356833in}{5.175000in}}%
\pgfusepath{clip}%
\pgfsetbuttcap%
\pgfsetroundjoin%
\pgfsetlinewidth{1.003750pt}%
\definecolor{currentstroke}{rgb}{0.827451,0.827451,0.827451}%
\pgfsetstrokecolor{currentstroke}%
\pgfsetdash{}{0pt}%
\pgfpathmoveto{\pgfqpoint{11.585879in}{4.579484in}}%
\pgfpathcurveto{\pgfqpoint{11.596929in}{4.579484in}}{\pgfqpoint{11.607528in}{4.583874in}}{\pgfqpoint{11.615342in}{4.591687in}}%
\pgfpathcurveto{\pgfqpoint{11.623156in}{4.599501in}}{\pgfqpoint{11.627546in}{4.610100in}}{\pgfqpoint{11.627546in}{4.621150in}}%
\pgfpathcurveto{\pgfqpoint{11.627546in}{4.632200in}}{\pgfqpoint{11.623156in}{4.642799in}}{\pgfqpoint{11.615342in}{4.650613in}}%
\pgfpathcurveto{\pgfqpoint{11.607528in}{4.658427in}}{\pgfqpoint{11.596929in}{4.662817in}}{\pgfqpoint{11.585879in}{4.662817in}}%
\pgfpathcurveto{\pgfqpoint{11.574829in}{4.662817in}}{\pgfqpoint{11.564230in}{4.658427in}}{\pgfqpoint{11.556417in}{4.650613in}}%
\pgfpathcurveto{\pgfqpoint{11.548603in}{4.642799in}}{\pgfqpoint{11.544213in}{4.632200in}}{\pgfqpoint{11.544213in}{4.621150in}}%
\pgfpathcurveto{\pgfqpoint{11.544213in}{4.610100in}}{\pgfqpoint{11.548603in}{4.599501in}}{\pgfqpoint{11.556417in}{4.591687in}}%
\pgfpathcurveto{\pgfqpoint{11.564230in}{4.583874in}}{\pgfqpoint{11.574829in}{4.579484in}}{\pgfqpoint{11.585879in}{4.579484in}}%
\pgfpathlineto{\pgfqpoint{11.585879in}{4.579484in}}%
\pgfpathclose%
\pgfusepath{stroke}%
\end{pgfscope}%
\begin{pgfscope}%
\pgfpathrectangle{\pgfqpoint{7.394209in}{0.375000in}}{\pgfqpoint{6.356833in}{5.175000in}}%
\pgfusepath{clip}%
\pgfsetbuttcap%
\pgfsetroundjoin%
\pgfsetlinewidth{1.003750pt}%
\definecolor{currentstroke}{rgb}{0.827451,0.827451,0.827451}%
\pgfsetstrokecolor{currentstroke}%
\pgfsetdash{}{0pt}%
\pgfpathmoveto{\pgfqpoint{11.376441in}{5.053569in}}%
\pgfpathcurveto{\pgfqpoint{11.387491in}{5.053569in}}{\pgfqpoint{11.398090in}{5.057959in}}{\pgfqpoint{11.405904in}{5.065773in}}%
\pgfpathcurveto{\pgfqpoint{11.413718in}{5.073587in}}{\pgfqpoint{11.418108in}{5.084186in}}{\pgfqpoint{11.418108in}{5.095236in}}%
\pgfpathcurveto{\pgfqpoint{11.418108in}{5.106286in}}{\pgfqpoint{11.413718in}{5.116885in}}{\pgfqpoint{11.405904in}{5.124699in}}%
\pgfpathcurveto{\pgfqpoint{11.398090in}{5.132512in}}{\pgfqpoint{11.387491in}{5.136903in}}{\pgfqpoint{11.376441in}{5.136903in}}%
\pgfpathcurveto{\pgfqpoint{11.365391in}{5.136903in}}{\pgfqpoint{11.354792in}{5.132512in}}{\pgfqpoint{11.346978in}{5.124699in}}%
\pgfpathcurveto{\pgfqpoint{11.339165in}{5.116885in}}{\pgfqpoint{11.334774in}{5.106286in}}{\pgfqpoint{11.334774in}{5.095236in}}%
\pgfpathcurveto{\pgfqpoint{11.334774in}{5.084186in}}{\pgfqpoint{11.339165in}{5.073587in}}{\pgfqpoint{11.346978in}{5.065773in}}%
\pgfpathcurveto{\pgfqpoint{11.354792in}{5.057959in}}{\pgfqpoint{11.365391in}{5.053569in}}{\pgfqpoint{11.376441in}{5.053569in}}%
\pgfpathlineto{\pgfqpoint{11.376441in}{5.053569in}}%
\pgfpathclose%
\pgfusepath{stroke}%
\end{pgfscope}%
\begin{pgfscope}%
\pgfpathrectangle{\pgfqpoint{7.394209in}{0.375000in}}{\pgfqpoint{6.356833in}{5.175000in}}%
\pgfusepath{clip}%
\pgfsetbuttcap%
\pgfsetroundjoin%
\pgfsetlinewidth{1.003750pt}%
\definecolor{currentstroke}{rgb}{0.827451,0.827451,0.827451}%
\pgfsetstrokecolor{currentstroke}%
\pgfsetdash{}{0pt}%
\pgfpathmoveto{\pgfqpoint{13.055333in}{5.283472in}}%
\pgfpathcurveto{\pgfqpoint{13.066383in}{5.283472in}}{\pgfqpoint{13.076982in}{5.287862in}}{\pgfqpoint{13.084795in}{5.295676in}}%
\pgfpathcurveto{\pgfqpoint{13.092609in}{5.303489in}}{\pgfqpoint{13.096999in}{5.314088in}}{\pgfqpoint{13.096999in}{5.325139in}}%
\pgfpathcurveto{\pgfqpoint{13.096999in}{5.336189in}}{\pgfqpoint{13.092609in}{5.346788in}}{\pgfqpoint{13.084795in}{5.354601in}}%
\pgfpathcurveto{\pgfqpoint{13.076982in}{5.362415in}}{\pgfqpoint{13.066383in}{5.366805in}}{\pgfqpoint{13.055333in}{5.366805in}}%
\pgfpathcurveto{\pgfqpoint{13.044282in}{5.366805in}}{\pgfqpoint{13.033683in}{5.362415in}}{\pgfqpoint{13.025870in}{5.354601in}}%
\pgfpathcurveto{\pgfqpoint{13.018056in}{5.346788in}}{\pgfqpoint{13.013666in}{5.336189in}}{\pgfqpoint{13.013666in}{5.325139in}}%
\pgfpathcurveto{\pgfqpoint{13.013666in}{5.314088in}}{\pgfqpoint{13.018056in}{5.303489in}}{\pgfqpoint{13.025870in}{5.295676in}}%
\pgfpathcurveto{\pgfqpoint{13.033683in}{5.287862in}}{\pgfqpoint{13.044282in}{5.283472in}}{\pgfqpoint{13.055333in}{5.283472in}}%
\pgfpathlineto{\pgfqpoint{13.055333in}{5.283472in}}%
\pgfpathclose%
\pgfusepath{stroke}%
\end{pgfscope}%
\begin{pgfscope}%
\pgfpathrectangle{\pgfqpoint{7.394209in}{0.375000in}}{\pgfqpoint{6.356833in}{5.175000in}}%
\pgfusepath{clip}%
\pgfsetbuttcap%
\pgfsetroundjoin%
\pgfsetlinewidth{1.003750pt}%
\definecolor{currentstroke}{rgb}{0.827451,0.827451,0.827451}%
\pgfsetstrokecolor{currentstroke}%
\pgfsetdash{}{0pt}%
\pgfpathmoveto{\pgfqpoint{8.070538in}{2.133794in}}%
\pgfpathcurveto{\pgfqpoint{8.081588in}{2.133794in}}{\pgfqpoint{8.092187in}{2.138185in}}{\pgfqpoint{8.100000in}{2.145998in}}%
\pgfpathcurveto{\pgfqpoint{8.107814in}{2.153812in}}{\pgfqpoint{8.112204in}{2.164411in}}{\pgfqpoint{8.112204in}{2.175461in}}%
\pgfpathcurveto{\pgfqpoint{8.112204in}{2.186511in}}{\pgfqpoint{8.107814in}{2.197110in}}{\pgfqpoint{8.100000in}{2.204924in}}%
\pgfpathcurveto{\pgfqpoint{8.092187in}{2.212737in}}{\pgfqpoint{8.081588in}{2.217128in}}{\pgfqpoint{8.070538in}{2.217128in}}%
\pgfpathcurveto{\pgfqpoint{8.059488in}{2.217128in}}{\pgfqpoint{8.048888in}{2.212737in}}{\pgfqpoint{8.041075in}{2.204924in}}%
\pgfpathcurveto{\pgfqpoint{8.033261in}{2.197110in}}{\pgfqpoint{8.028871in}{2.186511in}}{\pgfqpoint{8.028871in}{2.175461in}}%
\pgfpathcurveto{\pgfqpoint{8.028871in}{2.164411in}}{\pgfqpoint{8.033261in}{2.153812in}}{\pgfqpoint{8.041075in}{2.145998in}}%
\pgfpathcurveto{\pgfqpoint{8.048888in}{2.138185in}}{\pgfqpoint{8.059488in}{2.133794in}}{\pgfqpoint{8.070538in}{2.133794in}}%
\pgfpathlineto{\pgfqpoint{8.070538in}{2.133794in}}%
\pgfpathclose%
\pgfusepath{stroke}%
\end{pgfscope}%
\begin{pgfscope}%
\pgfpathrectangle{\pgfqpoint{7.394209in}{0.375000in}}{\pgfqpoint{6.356833in}{5.175000in}}%
\pgfusepath{clip}%
\pgfsetbuttcap%
\pgfsetroundjoin%
\pgfsetlinewidth{1.003750pt}%
\definecolor{currentstroke}{rgb}{0.827451,0.827451,0.827451}%
\pgfsetstrokecolor{currentstroke}%
\pgfsetdash{}{0pt}%
\pgfpathmoveto{\pgfqpoint{10.331704in}{3.474510in}}%
\pgfpathcurveto{\pgfqpoint{10.342754in}{3.474510in}}{\pgfqpoint{10.353353in}{3.478900in}}{\pgfqpoint{10.361167in}{3.486714in}}%
\pgfpathcurveto{\pgfqpoint{10.368981in}{3.494528in}}{\pgfqpoint{10.373371in}{3.505127in}}{\pgfqpoint{10.373371in}{3.516177in}}%
\pgfpathcurveto{\pgfqpoint{10.373371in}{3.527227in}}{\pgfqpoint{10.368981in}{3.537826in}}{\pgfqpoint{10.361167in}{3.545640in}}%
\pgfpathcurveto{\pgfqpoint{10.353353in}{3.553453in}}{\pgfqpoint{10.342754in}{3.557843in}}{\pgfqpoint{10.331704in}{3.557843in}}%
\pgfpathcurveto{\pgfqpoint{10.320654in}{3.557843in}}{\pgfqpoint{10.310055in}{3.553453in}}{\pgfqpoint{10.302241in}{3.545640in}}%
\pgfpathcurveto{\pgfqpoint{10.294428in}{3.537826in}}{\pgfqpoint{10.290038in}{3.527227in}}{\pgfqpoint{10.290038in}{3.516177in}}%
\pgfpathcurveto{\pgfqpoint{10.290038in}{3.505127in}}{\pgfqpoint{10.294428in}{3.494528in}}{\pgfqpoint{10.302241in}{3.486714in}}%
\pgfpathcurveto{\pgfqpoint{10.310055in}{3.478900in}}{\pgfqpoint{10.320654in}{3.474510in}}{\pgfqpoint{10.331704in}{3.474510in}}%
\pgfpathlineto{\pgfqpoint{10.331704in}{3.474510in}}%
\pgfpathclose%
\pgfusepath{stroke}%
\end{pgfscope}%
\begin{pgfscope}%
\pgfpathrectangle{\pgfqpoint{7.394209in}{0.375000in}}{\pgfqpoint{6.356833in}{5.175000in}}%
\pgfusepath{clip}%
\pgfsetbuttcap%
\pgfsetroundjoin%
\pgfsetlinewidth{1.003750pt}%
\definecolor{currentstroke}{rgb}{0.827451,0.827451,0.827451}%
\pgfsetstrokecolor{currentstroke}%
\pgfsetdash{}{0pt}%
\pgfpathmoveto{\pgfqpoint{8.502943in}{2.501948in}}%
\pgfpathcurveto{\pgfqpoint{8.513993in}{2.501948in}}{\pgfqpoint{8.524592in}{2.506338in}}{\pgfqpoint{8.532406in}{2.514152in}}%
\pgfpathcurveto{\pgfqpoint{8.540220in}{2.521966in}}{\pgfqpoint{8.544610in}{2.532565in}}{\pgfqpoint{8.544610in}{2.543615in}}%
\pgfpathcurveto{\pgfqpoint{8.544610in}{2.554665in}}{\pgfqpoint{8.540220in}{2.565264in}}{\pgfqpoint{8.532406in}{2.573077in}}%
\pgfpathcurveto{\pgfqpoint{8.524592in}{2.580891in}}{\pgfqpoint{8.513993in}{2.585281in}}{\pgfqpoint{8.502943in}{2.585281in}}%
\pgfpathcurveto{\pgfqpoint{8.491893in}{2.585281in}}{\pgfqpoint{8.481294in}{2.580891in}}{\pgfqpoint{8.473480in}{2.573077in}}%
\pgfpathcurveto{\pgfqpoint{8.465667in}{2.565264in}}{\pgfqpoint{8.461276in}{2.554665in}}{\pgfqpoint{8.461276in}{2.543615in}}%
\pgfpathcurveto{\pgfqpoint{8.461276in}{2.532565in}}{\pgfqpoint{8.465667in}{2.521966in}}{\pgfqpoint{8.473480in}{2.514152in}}%
\pgfpathcurveto{\pgfqpoint{8.481294in}{2.506338in}}{\pgfqpoint{8.491893in}{2.501948in}}{\pgfqpoint{8.502943in}{2.501948in}}%
\pgfpathlineto{\pgfqpoint{8.502943in}{2.501948in}}%
\pgfpathclose%
\pgfusepath{stroke}%
\end{pgfscope}%
\begin{pgfscope}%
\pgfpathrectangle{\pgfqpoint{7.394209in}{0.375000in}}{\pgfqpoint{6.356833in}{5.175000in}}%
\pgfusepath{clip}%
\pgfsetbuttcap%
\pgfsetroundjoin%
\pgfsetlinewidth{1.003750pt}%
\definecolor{currentstroke}{rgb}{0.827451,0.827451,0.827451}%
\pgfsetstrokecolor{currentstroke}%
\pgfsetdash{}{0pt}%
\pgfpathmoveto{\pgfqpoint{8.401262in}{1.260710in}}%
\pgfpathcurveto{\pgfqpoint{8.412312in}{1.260710in}}{\pgfqpoint{8.422911in}{1.265100in}}{\pgfqpoint{8.430724in}{1.272914in}}%
\pgfpathcurveto{\pgfqpoint{8.438538in}{1.280727in}}{\pgfqpoint{8.442928in}{1.291326in}}{\pgfqpoint{8.442928in}{1.302376in}}%
\pgfpathcurveto{\pgfqpoint{8.442928in}{1.313427in}}{\pgfqpoint{8.438538in}{1.324026in}}{\pgfqpoint{8.430724in}{1.331839in}}%
\pgfpathcurveto{\pgfqpoint{8.422911in}{1.339653in}}{\pgfqpoint{8.412312in}{1.344043in}}{\pgfqpoint{8.401262in}{1.344043in}}%
\pgfpathcurveto{\pgfqpoint{8.390211in}{1.344043in}}{\pgfqpoint{8.379612in}{1.339653in}}{\pgfqpoint{8.371799in}{1.331839in}}%
\pgfpathcurveto{\pgfqpoint{8.363985in}{1.324026in}}{\pgfqpoint{8.359595in}{1.313427in}}{\pgfqpoint{8.359595in}{1.302376in}}%
\pgfpathcurveto{\pgfqpoint{8.359595in}{1.291326in}}{\pgfqpoint{8.363985in}{1.280727in}}{\pgfqpoint{8.371799in}{1.272914in}}%
\pgfpathcurveto{\pgfqpoint{8.379612in}{1.265100in}}{\pgfqpoint{8.390211in}{1.260710in}}{\pgfqpoint{8.401262in}{1.260710in}}%
\pgfpathlineto{\pgfqpoint{8.401262in}{1.260710in}}%
\pgfpathclose%
\pgfusepath{stroke}%
\end{pgfscope}%
\begin{pgfscope}%
\pgfpathrectangle{\pgfqpoint{7.394209in}{0.375000in}}{\pgfqpoint{6.356833in}{5.175000in}}%
\pgfusepath{clip}%
\pgfsetbuttcap%
\pgfsetroundjoin%
\pgfsetlinewidth{1.003750pt}%
\definecolor{currentstroke}{rgb}{0.827451,0.827451,0.827451}%
\pgfsetstrokecolor{currentstroke}%
\pgfsetdash{}{0pt}%
\pgfpathmoveto{\pgfqpoint{13.578570in}{5.304631in}}%
\pgfpathcurveto{\pgfqpoint{13.589620in}{5.304631in}}{\pgfqpoint{13.600219in}{5.309022in}}{\pgfqpoint{13.608033in}{5.316835in}}%
\pgfpathcurveto{\pgfqpoint{13.615847in}{5.324649in}}{\pgfqpoint{13.620237in}{5.335248in}}{\pgfqpoint{13.620237in}{5.346298in}}%
\pgfpathcurveto{\pgfqpoint{13.620237in}{5.357348in}}{\pgfqpoint{13.615847in}{5.367947in}}{\pgfqpoint{13.608033in}{5.375761in}}%
\pgfpathcurveto{\pgfqpoint{13.600219in}{5.383574in}}{\pgfqpoint{13.589620in}{5.387965in}}{\pgfqpoint{13.578570in}{5.387965in}}%
\pgfpathcurveto{\pgfqpoint{13.567520in}{5.387965in}}{\pgfqpoint{13.556921in}{5.383574in}}{\pgfqpoint{13.549107in}{5.375761in}}%
\pgfpathcurveto{\pgfqpoint{13.541294in}{5.367947in}}{\pgfqpoint{13.536903in}{5.357348in}}{\pgfqpoint{13.536903in}{5.346298in}}%
\pgfpathcurveto{\pgfqpoint{13.536903in}{5.335248in}}{\pgfqpoint{13.541294in}{5.324649in}}{\pgfqpoint{13.549107in}{5.316835in}}%
\pgfpathcurveto{\pgfqpoint{13.556921in}{5.309022in}}{\pgfqpoint{13.567520in}{5.304631in}}{\pgfqpoint{13.578570in}{5.304631in}}%
\pgfpathlineto{\pgfqpoint{13.578570in}{5.304631in}}%
\pgfpathclose%
\pgfusepath{stroke}%
\end{pgfscope}%
\begin{pgfscope}%
\pgfpathrectangle{\pgfqpoint{7.394209in}{0.375000in}}{\pgfqpoint{6.356833in}{5.175000in}}%
\pgfusepath{clip}%
\pgfsetbuttcap%
\pgfsetroundjoin%
\pgfsetlinewidth{1.003750pt}%
\definecolor{currentstroke}{rgb}{0.827451,0.827451,0.827451}%
\pgfsetstrokecolor{currentstroke}%
\pgfsetdash{}{0pt}%
\pgfpathmoveto{\pgfqpoint{13.010070in}{5.138818in}}%
\pgfpathcurveto{\pgfqpoint{13.021120in}{5.138818in}}{\pgfqpoint{13.031719in}{5.143209in}}{\pgfqpoint{13.039533in}{5.151022in}}%
\pgfpathcurveto{\pgfqpoint{13.047346in}{5.158836in}}{\pgfqpoint{13.051737in}{5.169435in}}{\pgfqpoint{13.051737in}{5.180485in}}%
\pgfpathcurveto{\pgfqpoint{13.051737in}{5.191535in}}{\pgfqpoint{13.047346in}{5.202134in}}{\pgfqpoint{13.039533in}{5.209948in}}%
\pgfpathcurveto{\pgfqpoint{13.031719in}{5.217761in}}{\pgfqpoint{13.021120in}{5.222152in}}{\pgfqpoint{13.010070in}{5.222152in}}%
\pgfpathcurveto{\pgfqpoint{12.999020in}{5.222152in}}{\pgfqpoint{12.988421in}{5.217761in}}{\pgfqpoint{12.980607in}{5.209948in}}%
\pgfpathcurveto{\pgfqpoint{12.972793in}{5.202134in}}{\pgfqpoint{12.968403in}{5.191535in}}{\pgfqpoint{12.968403in}{5.180485in}}%
\pgfpathcurveto{\pgfqpoint{12.968403in}{5.169435in}}{\pgfqpoint{12.972793in}{5.158836in}}{\pgfqpoint{12.980607in}{5.151022in}}%
\pgfpathcurveto{\pgfqpoint{12.988421in}{5.143209in}}{\pgfqpoint{12.999020in}{5.138818in}}{\pgfqpoint{13.010070in}{5.138818in}}%
\pgfpathlineto{\pgfqpoint{13.010070in}{5.138818in}}%
\pgfpathclose%
\pgfusepath{stroke}%
\end{pgfscope}%
\begin{pgfscope}%
\pgfpathrectangle{\pgfqpoint{7.394209in}{0.375000in}}{\pgfqpoint{6.356833in}{5.175000in}}%
\pgfusepath{clip}%
\pgfsetbuttcap%
\pgfsetroundjoin%
\pgfsetlinewidth{1.003750pt}%
\definecolor{currentstroke}{rgb}{0.827451,0.827451,0.827451}%
\pgfsetstrokecolor{currentstroke}%
\pgfsetdash{}{0pt}%
\pgfpathmoveto{\pgfqpoint{11.348457in}{3.805531in}}%
\pgfpathcurveto{\pgfqpoint{11.359508in}{3.805531in}}{\pgfqpoint{11.370107in}{3.809921in}}{\pgfqpoint{11.377920in}{3.817735in}}%
\pgfpathcurveto{\pgfqpoint{11.385734in}{3.825549in}}{\pgfqpoint{11.390124in}{3.836148in}}{\pgfqpoint{11.390124in}{3.847198in}}%
\pgfpathcurveto{\pgfqpoint{11.390124in}{3.858248in}}{\pgfqpoint{11.385734in}{3.868847in}}{\pgfqpoint{11.377920in}{3.876661in}}%
\pgfpathcurveto{\pgfqpoint{11.370107in}{3.884474in}}{\pgfqpoint{11.359508in}{3.888864in}}{\pgfqpoint{11.348457in}{3.888864in}}%
\pgfpathcurveto{\pgfqpoint{11.337407in}{3.888864in}}{\pgfqpoint{11.326808in}{3.884474in}}{\pgfqpoint{11.318995in}{3.876661in}}%
\pgfpathcurveto{\pgfqpoint{11.311181in}{3.868847in}}{\pgfqpoint{11.306791in}{3.858248in}}{\pgfqpoint{11.306791in}{3.847198in}}%
\pgfpathcurveto{\pgfqpoint{11.306791in}{3.836148in}}{\pgfqpoint{11.311181in}{3.825549in}}{\pgfqpoint{11.318995in}{3.817735in}}%
\pgfpathcurveto{\pgfqpoint{11.326808in}{3.809921in}}{\pgfqpoint{11.337407in}{3.805531in}}{\pgfqpoint{11.348457in}{3.805531in}}%
\pgfpathlineto{\pgfqpoint{11.348457in}{3.805531in}}%
\pgfpathclose%
\pgfusepath{stroke}%
\end{pgfscope}%
\begin{pgfscope}%
\pgfpathrectangle{\pgfqpoint{7.394209in}{0.375000in}}{\pgfqpoint{6.356833in}{5.175000in}}%
\pgfusepath{clip}%
\pgfsetbuttcap%
\pgfsetroundjoin%
\pgfsetlinewidth{1.003750pt}%
\definecolor{currentstroke}{rgb}{0.827451,0.827451,0.827451}%
\pgfsetstrokecolor{currentstroke}%
\pgfsetdash{}{0pt}%
\pgfpathmoveto{\pgfqpoint{12.138952in}{5.356391in}}%
\pgfpathcurveto{\pgfqpoint{12.150002in}{5.356391in}}{\pgfqpoint{12.160601in}{5.360782in}}{\pgfqpoint{12.168414in}{5.368595in}}%
\pgfpathcurveto{\pgfqpoint{12.176228in}{5.376409in}}{\pgfqpoint{12.180618in}{5.387008in}}{\pgfqpoint{12.180618in}{5.398058in}}%
\pgfpathcurveto{\pgfqpoint{12.180618in}{5.409108in}}{\pgfqpoint{12.176228in}{5.419707in}}{\pgfqpoint{12.168414in}{5.427521in}}%
\pgfpathcurveto{\pgfqpoint{12.160601in}{5.435335in}}{\pgfqpoint{12.150002in}{5.439725in}}{\pgfqpoint{12.138952in}{5.439725in}}%
\pgfpathcurveto{\pgfqpoint{12.127901in}{5.439725in}}{\pgfqpoint{12.117302in}{5.435335in}}{\pgfqpoint{12.109489in}{5.427521in}}%
\pgfpathcurveto{\pgfqpoint{12.101675in}{5.419707in}}{\pgfqpoint{12.097285in}{5.409108in}}{\pgfqpoint{12.097285in}{5.398058in}}%
\pgfpathcurveto{\pgfqpoint{12.097285in}{5.387008in}}{\pgfqpoint{12.101675in}{5.376409in}}{\pgfqpoint{12.109489in}{5.368595in}}%
\pgfpathcurveto{\pgfqpoint{12.117302in}{5.360782in}}{\pgfqpoint{12.127901in}{5.356391in}}{\pgfqpoint{12.138952in}{5.356391in}}%
\pgfpathlineto{\pgfqpoint{12.138952in}{5.356391in}}%
\pgfpathclose%
\pgfusepath{stroke}%
\end{pgfscope}%
\begin{pgfscope}%
\pgfpathrectangle{\pgfqpoint{7.394209in}{0.375000in}}{\pgfqpoint{6.356833in}{5.175000in}}%
\pgfusepath{clip}%
\pgfsetbuttcap%
\pgfsetroundjoin%
\pgfsetlinewidth{1.003750pt}%
\definecolor{currentstroke}{rgb}{0.827451,0.827451,0.827451}%
\pgfsetstrokecolor{currentstroke}%
\pgfsetdash{}{0pt}%
\pgfpathmoveto{\pgfqpoint{13.230161in}{5.454756in}}%
\pgfpathcurveto{\pgfqpoint{13.241211in}{5.454756in}}{\pgfqpoint{13.251810in}{5.459146in}}{\pgfqpoint{13.259624in}{5.466960in}}%
\pgfpathcurveto{\pgfqpoint{13.267438in}{5.474773in}}{\pgfqpoint{13.271828in}{5.485372in}}{\pgfqpoint{13.271828in}{5.496422in}}%
\pgfpathcurveto{\pgfqpoint{13.271828in}{5.507473in}}{\pgfqpoint{13.267438in}{5.518072in}}{\pgfqpoint{13.259624in}{5.525885in}}%
\pgfpathcurveto{\pgfqpoint{13.251810in}{5.533699in}}{\pgfqpoint{13.241211in}{5.538089in}}{\pgfqpoint{13.230161in}{5.538089in}}%
\pgfpathcurveto{\pgfqpoint{13.219111in}{5.538089in}}{\pgfqpoint{13.208512in}{5.533699in}}{\pgfqpoint{13.200699in}{5.525885in}}%
\pgfpathcurveto{\pgfqpoint{13.192885in}{5.518072in}}{\pgfqpoint{13.188495in}{5.507473in}}{\pgfqpoint{13.188495in}{5.496422in}}%
\pgfpathcurveto{\pgfqpoint{13.188495in}{5.485372in}}{\pgfqpoint{13.192885in}{5.474773in}}{\pgfqpoint{13.200699in}{5.466960in}}%
\pgfpathcurveto{\pgfqpoint{13.208512in}{5.459146in}}{\pgfqpoint{13.219111in}{5.454756in}}{\pgfqpoint{13.230161in}{5.454756in}}%
\pgfpathlineto{\pgfqpoint{13.230161in}{5.454756in}}%
\pgfpathclose%
\pgfusepath{stroke}%
\end{pgfscope}%
\begin{pgfscope}%
\pgfpathrectangle{\pgfqpoint{7.394209in}{0.375000in}}{\pgfqpoint{6.356833in}{5.175000in}}%
\pgfusepath{clip}%
\pgfsetbuttcap%
\pgfsetroundjoin%
\pgfsetlinewidth{1.003750pt}%
\definecolor{currentstroke}{rgb}{0.827451,0.827451,0.827451}%
\pgfsetstrokecolor{currentstroke}%
\pgfsetdash{}{0pt}%
\pgfpathmoveto{\pgfqpoint{12.825049in}{5.384180in}}%
\pgfpathcurveto{\pgfqpoint{12.836099in}{5.384180in}}{\pgfqpoint{12.846698in}{5.388570in}}{\pgfqpoint{12.854512in}{5.396383in}}%
\pgfpathcurveto{\pgfqpoint{12.862325in}{5.404197in}}{\pgfqpoint{12.866716in}{5.414796in}}{\pgfqpoint{12.866716in}{5.425846in}}%
\pgfpathcurveto{\pgfqpoint{12.866716in}{5.436896in}}{\pgfqpoint{12.862325in}{5.447495in}}{\pgfqpoint{12.854512in}{5.455309in}}%
\pgfpathcurveto{\pgfqpoint{12.846698in}{5.463123in}}{\pgfqpoint{12.836099in}{5.467513in}}{\pgfqpoint{12.825049in}{5.467513in}}%
\pgfpathcurveto{\pgfqpoint{12.813999in}{5.467513in}}{\pgfqpoint{12.803400in}{5.463123in}}{\pgfqpoint{12.795586in}{5.455309in}}%
\pgfpathcurveto{\pgfqpoint{12.787773in}{5.447495in}}{\pgfqpoint{12.783382in}{5.436896in}}{\pgfqpoint{12.783382in}{5.425846in}}%
\pgfpathcurveto{\pgfqpoint{12.783382in}{5.414796in}}{\pgfqpoint{12.787773in}{5.404197in}}{\pgfqpoint{12.795586in}{5.396383in}}%
\pgfpathcurveto{\pgfqpoint{12.803400in}{5.388570in}}{\pgfqpoint{12.813999in}{5.384180in}}{\pgfqpoint{12.825049in}{5.384180in}}%
\pgfpathlineto{\pgfqpoint{12.825049in}{5.384180in}}%
\pgfpathclose%
\pgfusepath{stroke}%
\end{pgfscope}%
\begin{pgfscope}%
\pgfpathrectangle{\pgfqpoint{7.394209in}{0.375000in}}{\pgfqpoint{6.356833in}{5.175000in}}%
\pgfusepath{clip}%
\pgfsetbuttcap%
\pgfsetroundjoin%
\pgfsetlinewidth{1.003750pt}%
\definecolor{currentstroke}{rgb}{0.827451,0.827451,0.827451}%
\pgfsetstrokecolor{currentstroke}%
\pgfsetdash{}{0pt}%
\pgfpathmoveto{\pgfqpoint{8.501316in}{2.839956in}}%
\pgfpathcurveto{\pgfqpoint{8.512366in}{2.839956in}}{\pgfqpoint{8.522965in}{2.844346in}}{\pgfqpoint{8.530778in}{2.852160in}}%
\pgfpathcurveto{\pgfqpoint{8.538592in}{2.859973in}}{\pgfqpoint{8.542982in}{2.870572in}}{\pgfqpoint{8.542982in}{2.881622in}}%
\pgfpathcurveto{\pgfqpoint{8.542982in}{2.892673in}}{\pgfqpoint{8.538592in}{2.903272in}}{\pgfqpoint{8.530778in}{2.911085in}}%
\pgfpathcurveto{\pgfqpoint{8.522965in}{2.918899in}}{\pgfqpoint{8.512366in}{2.923289in}}{\pgfqpoint{8.501316in}{2.923289in}}%
\pgfpathcurveto{\pgfqpoint{8.490266in}{2.923289in}}{\pgfqpoint{8.479666in}{2.918899in}}{\pgfqpoint{8.471853in}{2.911085in}}%
\pgfpathcurveto{\pgfqpoint{8.464039in}{2.903272in}}{\pgfqpoint{8.459649in}{2.892673in}}{\pgfqpoint{8.459649in}{2.881622in}}%
\pgfpathcurveto{\pgfqpoint{8.459649in}{2.870572in}}{\pgfqpoint{8.464039in}{2.859973in}}{\pgfqpoint{8.471853in}{2.852160in}}%
\pgfpathcurveto{\pgfqpoint{8.479666in}{2.844346in}}{\pgfqpoint{8.490266in}{2.839956in}}{\pgfqpoint{8.501316in}{2.839956in}}%
\pgfpathlineto{\pgfqpoint{8.501316in}{2.839956in}}%
\pgfpathclose%
\pgfusepath{stroke}%
\end{pgfscope}%
\begin{pgfscope}%
\pgfpathrectangle{\pgfqpoint{7.394209in}{0.375000in}}{\pgfqpoint{6.356833in}{5.175000in}}%
\pgfusepath{clip}%
\pgfsetbuttcap%
\pgfsetroundjoin%
\pgfsetlinewidth{1.003750pt}%
\definecolor{currentstroke}{rgb}{0.827451,0.827451,0.827451}%
\pgfsetstrokecolor{currentstroke}%
\pgfsetdash{}{0pt}%
\pgfpathmoveto{\pgfqpoint{9.490704in}{1.603023in}}%
\pgfpathcurveto{\pgfqpoint{9.501754in}{1.603023in}}{\pgfqpoint{9.512353in}{1.607413in}}{\pgfqpoint{9.520166in}{1.615227in}}%
\pgfpathcurveto{\pgfqpoint{9.527980in}{1.623040in}}{\pgfqpoint{9.532370in}{1.633639in}}{\pgfqpoint{9.532370in}{1.644689in}}%
\pgfpathcurveto{\pgfqpoint{9.532370in}{1.655740in}}{\pgfqpoint{9.527980in}{1.666339in}}{\pgfqpoint{9.520166in}{1.674152in}}%
\pgfpathcurveto{\pgfqpoint{9.512353in}{1.681966in}}{\pgfqpoint{9.501754in}{1.686356in}}{\pgfqpoint{9.490704in}{1.686356in}}%
\pgfpathcurveto{\pgfqpoint{9.479654in}{1.686356in}}{\pgfqpoint{9.469055in}{1.681966in}}{\pgfqpoint{9.461241in}{1.674152in}}%
\pgfpathcurveto{\pgfqpoint{9.453427in}{1.666339in}}{\pgfqpoint{9.449037in}{1.655740in}}{\pgfqpoint{9.449037in}{1.644689in}}%
\pgfpathcurveto{\pgfqpoint{9.449037in}{1.633639in}}{\pgfqpoint{9.453427in}{1.623040in}}{\pgfqpoint{9.461241in}{1.615227in}}%
\pgfpathcurveto{\pgfqpoint{9.469055in}{1.607413in}}{\pgfqpoint{9.479654in}{1.603023in}}{\pgfqpoint{9.490704in}{1.603023in}}%
\pgfpathlineto{\pgfqpoint{9.490704in}{1.603023in}}%
\pgfpathclose%
\pgfusepath{stroke}%
\end{pgfscope}%
\begin{pgfscope}%
\pgfpathrectangle{\pgfqpoint{7.394209in}{0.375000in}}{\pgfqpoint{6.356833in}{5.175000in}}%
\pgfusepath{clip}%
\pgfsetbuttcap%
\pgfsetroundjoin%
\pgfsetlinewidth{1.003750pt}%
\definecolor{currentstroke}{rgb}{0.827451,0.827451,0.827451}%
\pgfsetstrokecolor{currentstroke}%
\pgfsetdash{}{0pt}%
\pgfpathmoveto{\pgfqpoint{8.613294in}{3.332106in}}%
\pgfpathcurveto{\pgfqpoint{8.624344in}{3.332106in}}{\pgfqpoint{8.634943in}{3.336496in}}{\pgfqpoint{8.642757in}{3.344310in}}%
\pgfpathcurveto{\pgfqpoint{8.650570in}{3.352123in}}{\pgfqpoint{8.654961in}{3.362722in}}{\pgfqpoint{8.654961in}{3.373772in}}%
\pgfpathcurveto{\pgfqpoint{8.654961in}{3.384823in}}{\pgfqpoint{8.650570in}{3.395422in}}{\pgfqpoint{8.642757in}{3.403235in}}%
\pgfpathcurveto{\pgfqpoint{8.634943in}{3.411049in}}{\pgfqpoint{8.624344in}{3.415439in}}{\pgfqpoint{8.613294in}{3.415439in}}%
\pgfpathcurveto{\pgfqpoint{8.602244in}{3.415439in}}{\pgfqpoint{8.591645in}{3.411049in}}{\pgfqpoint{8.583831in}{3.403235in}}%
\pgfpathcurveto{\pgfqpoint{8.576018in}{3.395422in}}{\pgfqpoint{8.571627in}{3.384823in}}{\pgfqpoint{8.571627in}{3.373772in}}%
\pgfpathcurveto{\pgfqpoint{8.571627in}{3.362722in}}{\pgfqpoint{8.576018in}{3.352123in}}{\pgfqpoint{8.583831in}{3.344310in}}%
\pgfpathcurveto{\pgfqpoint{8.591645in}{3.336496in}}{\pgfqpoint{8.602244in}{3.332106in}}{\pgfqpoint{8.613294in}{3.332106in}}%
\pgfpathlineto{\pgfqpoint{8.613294in}{3.332106in}}%
\pgfpathclose%
\pgfusepath{stroke}%
\end{pgfscope}%
\begin{pgfscope}%
\pgfpathrectangle{\pgfqpoint{7.394209in}{0.375000in}}{\pgfqpoint{6.356833in}{5.175000in}}%
\pgfusepath{clip}%
\pgfsetbuttcap%
\pgfsetroundjoin%
\pgfsetlinewidth{1.003750pt}%
\definecolor{currentstroke}{rgb}{0.827451,0.827451,0.827451}%
\pgfsetstrokecolor{currentstroke}%
\pgfsetdash{}{0pt}%
\pgfpathmoveto{\pgfqpoint{9.033479in}{1.809907in}}%
\pgfpathcurveto{\pgfqpoint{9.044529in}{1.809907in}}{\pgfqpoint{9.055128in}{1.814297in}}{\pgfqpoint{9.062942in}{1.822111in}}%
\pgfpathcurveto{\pgfqpoint{9.070755in}{1.829925in}}{\pgfqpoint{9.075146in}{1.840524in}}{\pgfqpoint{9.075146in}{1.851574in}}%
\pgfpathcurveto{\pgfqpoint{9.075146in}{1.862624in}}{\pgfqpoint{9.070755in}{1.873223in}}{\pgfqpoint{9.062942in}{1.881037in}}%
\pgfpathcurveto{\pgfqpoint{9.055128in}{1.888850in}}{\pgfqpoint{9.044529in}{1.893240in}}{\pgfqpoint{9.033479in}{1.893240in}}%
\pgfpathcurveto{\pgfqpoint{9.022429in}{1.893240in}}{\pgfqpoint{9.011830in}{1.888850in}}{\pgfqpoint{9.004016in}{1.881037in}}%
\pgfpathcurveto{\pgfqpoint{8.996203in}{1.873223in}}{\pgfqpoint{8.991812in}{1.862624in}}{\pgfqpoint{8.991812in}{1.851574in}}%
\pgfpathcurveto{\pgfqpoint{8.991812in}{1.840524in}}{\pgfqpoint{8.996203in}{1.829925in}}{\pgfqpoint{9.004016in}{1.822111in}}%
\pgfpathcurveto{\pgfqpoint{9.011830in}{1.814297in}}{\pgfqpoint{9.022429in}{1.809907in}}{\pgfqpoint{9.033479in}{1.809907in}}%
\pgfpathlineto{\pgfqpoint{9.033479in}{1.809907in}}%
\pgfpathclose%
\pgfusepath{stroke}%
\end{pgfscope}%
\begin{pgfscope}%
\pgfpathrectangle{\pgfqpoint{7.394209in}{0.375000in}}{\pgfqpoint{6.356833in}{5.175000in}}%
\pgfusepath{clip}%
\pgfsetbuttcap%
\pgfsetroundjoin%
\pgfsetlinewidth{1.003750pt}%
\definecolor{currentstroke}{rgb}{0.827451,0.827451,0.827451}%
\pgfsetstrokecolor{currentstroke}%
\pgfsetdash{}{0pt}%
\pgfpathmoveto{\pgfqpoint{8.900152in}{3.229377in}}%
\pgfpathcurveto{\pgfqpoint{8.911202in}{3.229377in}}{\pgfqpoint{8.921801in}{3.233767in}}{\pgfqpoint{8.929614in}{3.241581in}}%
\pgfpathcurveto{\pgfqpoint{8.937428in}{3.249395in}}{\pgfqpoint{8.941818in}{3.259994in}}{\pgfqpoint{8.941818in}{3.271044in}}%
\pgfpathcurveto{\pgfqpoint{8.941818in}{3.282094in}}{\pgfqpoint{8.937428in}{3.292693in}}{\pgfqpoint{8.929614in}{3.300507in}}%
\pgfpathcurveto{\pgfqpoint{8.921801in}{3.308320in}}{\pgfqpoint{8.911202in}{3.312711in}}{\pgfqpoint{8.900152in}{3.312711in}}%
\pgfpathcurveto{\pgfqpoint{8.889102in}{3.312711in}}{\pgfqpoint{8.878503in}{3.308320in}}{\pgfqpoint{8.870689in}{3.300507in}}%
\pgfpathcurveto{\pgfqpoint{8.862875in}{3.292693in}}{\pgfqpoint{8.858485in}{3.282094in}}{\pgfqpoint{8.858485in}{3.271044in}}%
\pgfpathcurveto{\pgfqpoint{8.858485in}{3.259994in}}{\pgfqpoint{8.862875in}{3.249395in}}{\pgfqpoint{8.870689in}{3.241581in}}%
\pgfpathcurveto{\pgfqpoint{8.878503in}{3.233767in}}{\pgfqpoint{8.889102in}{3.229377in}}{\pgfqpoint{8.900152in}{3.229377in}}%
\pgfpathlineto{\pgfqpoint{8.900152in}{3.229377in}}%
\pgfpathclose%
\pgfusepath{stroke}%
\end{pgfscope}%
\begin{pgfscope}%
\pgfpathrectangle{\pgfqpoint{7.394209in}{0.375000in}}{\pgfqpoint{6.356833in}{5.175000in}}%
\pgfusepath{clip}%
\pgfsetbuttcap%
\pgfsetroundjoin%
\pgfsetlinewidth{1.003750pt}%
\definecolor{currentstroke}{rgb}{0.827451,0.827451,0.827451}%
\pgfsetstrokecolor{currentstroke}%
\pgfsetdash{}{0pt}%
\pgfpathmoveto{\pgfqpoint{10.449791in}{5.308229in}}%
\pgfpathcurveto{\pgfqpoint{10.460841in}{5.308229in}}{\pgfqpoint{10.471440in}{5.312620in}}{\pgfqpoint{10.479254in}{5.320433in}}%
\pgfpathcurveto{\pgfqpoint{10.487067in}{5.328247in}}{\pgfqpoint{10.491458in}{5.338846in}}{\pgfqpoint{10.491458in}{5.349896in}}%
\pgfpathcurveto{\pgfqpoint{10.491458in}{5.360946in}}{\pgfqpoint{10.487067in}{5.371545in}}{\pgfqpoint{10.479254in}{5.379359in}}%
\pgfpathcurveto{\pgfqpoint{10.471440in}{5.387172in}}{\pgfqpoint{10.460841in}{5.391563in}}{\pgfqpoint{10.449791in}{5.391563in}}%
\pgfpathcurveto{\pgfqpoint{10.438741in}{5.391563in}}{\pgfqpoint{10.428142in}{5.387172in}}{\pgfqpoint{10.420328in}{5.379359in}}%
\pgfpathcurveto{\pgfqpoint{10.412515in}{5.371545in}}{\pgfqpoint{10.408124in}{5.360946in}}{\pgfqpoint{10.408124in}{5.349896in}}%
\pgfpathcurveto{\pgfqpoint{10.408124in}{5.338846in}}{\pgfqpoint{10.412515in}{5.328247in}}{\pgfqpoint{10.420328in}{5.320433in}}%
\pgfpathcurveto{\pgfqpoint{10.428142in}{5.312620in}}{\pgfqpoint{10.438741in}{5.308229in}}{\pgfqpoint{10.449791in}{5.308229in}}%
\pgfpathlineto{\pgfqpoint{10.449791in}{5.308229in}}%
\pgfpathclose%
\pgfusepath{stroke}%
\end{pgfscope}%
\begin{pgfscope}%
\pgfpathrectangle{\pgfqpoint{7.394209in}{0.375000in}}{\pgfqpoint{6.356833in}{5.175000in}}%
\pgfusepath{clip}%
\pgfsetbuttcap%
\pgfsetroundjoin%
\pgfsetlinewidth{1.003750pt}%
\definecolor{currentstroke}{rgb}{0.827451,0.827451,0.827451}%
\pgfsetstrokecolor{currentstroke}%
\pgfsetdash{}{0pt}%
\pgfpathmoveto{\pgfqpoint{8.414159in}{1.001343in}}%
\pgfpathcurveto{\pgfqpoint{8.425210in}{1.001343in}}{\pgfqpoint{8.435809in}{1.005733in}}{\pgfqpoint{8.443622in}{1.013547in}}%
\pgfpathcurveto{\pgfqpoint{8.451436in}{1.021360in}}{\pgfqpoint{8.455826in}{1.031959in}}{\pgfqpoint{8.455826in}{1.043010in}}%
\pgfpathcurveto{\pgfqpoint{8.455826in}{1.054060in}}{\pgfqpoint{8.451436in}{1.064659in}}{\pgfqpoint{8.443622in}{1.072472in}}%
\pgfpathcurveto{\pgfqpoint{8.435809in}{1.080286in}}{\pgfqpoint{8.425210in}{1.084676in}}{\pgfqpoint{8.414159in}{1.084676in}}%
\pgfpathcurveto{\pgfqpoint{8.403109in}{1.084676in}}{\pgfqpoint{8.392510in}{1.080286in}}{\pgfqpoint{8.384697in}{1.072472in}}%
\pgfpathcurveto{\pgfqpoint{8.376883in}{1.064659in}}{\pgfqpoint{8.372493in}{1.054060in}}{\pgfqpoint{8.372493in}{1.043010in}}%
\pgfpathcurveto{\pgfqpoint{8.372493in}{1.031959in}}{\pgfqpoint{8.376883in}{1.021360in}}{\pgfqpoint{8.384697in}{1.013547in}}%
\pgfpathcurveto{\pgfqpoint{8.392510in}{1.005733in}}{\pgfqpoint{8.403109in}{1.001343in}}{\pgfqpoint{8.414159in}{1.001343in}}%
\pgfpathlineto{\pgfqpoint{8.414159in}{1.001343in}}%
\pgfpathclose%
\pgfusepath{stroke}%
\end{pgfscope}%
\begin{pgfscope}%
\pgfpathrectangle{\pgfqpoint{7.394209in}{0.375000in}}{\pgfqpoint{6.356833in}{5.175000in}}%
\pgfusepath{clip}%
\pgfsetbuttcap%
\pgfsetroundjoin%
\pgfsetlinewidth{1.003750pt}%
\definecolor{currentstroke}{rgb}{0.827451,0.827451,0.827451}%
\pgfsetstrokecolor{currentstroke}%
\pgfsetdash{}{0pt}%
\pgfpathmoveto{\pgfqpoint{9.927656in}{3.051439in}}%
\pgfpathcurveto{\pgfqpoint{9.938706in}{3.051439in}}{\pgfqpoint{9.949306in}{3.055830in}}{\pgfqpoint{9.957119in}{3.063643in}}%
\pgfpathcurveto{\pgfqpoint{9.964933in}{3.071457in}}{\pgfqpoint{9.969323in}{3.082056in}}{\pgfqpoint{9.969323in}{3.093106in}}%
\pgfpathcurveto{\pgfqpoint{9.969323in}{3.104156in}}{\pgfqpoint{9.964933in}{3.114755in}}{\pgfqpoint{9.957119in}{3.122569in}}%
\pgfpathcurveto{\pgfqpoint{9.949306in}{3.130382in}}{\pgfqpoint{9.938706in}{3.134773in}}{\pgfqpoint{9.927656in}{3.134773in}}%
\pgfpathcurveto{\pgfqpoint{9.916606in}{3.134773in}}{\pgfqpoint{9.906007in}{3.130382in}}{\pgfqpoint{9.898194in}{3.122569in}}%
\pgfpathcurveto{\pgfqpoint{9.890380in}{3.114755in}}{\pgfqpoint{9.885990in}{3.104156in}}{\pgfqpoint{9.885990in}{3.093106in}}%
\pgfpathcurveto{\pgfqpoint{9.885990in}{3.082056in}}{\pgfqpoint{9.890380in}{3.071457in}}{\pgfqpoint{9.898194in}{3.063643in}}%
\pgfpathcurveto{\pgfqpoint{9.906007in}{3.055830in}}{\pgfqpoint{9.916606in}{3.051439in}}{\pgfqpoint{9.927656in}{3.051439in}}%
\pgfpathlineto{\pgfqpoint{9.927656in}{3.051439in}}%
\pgfpathclose%
\pgfusepath{stroke}%
\end{pgfscope}%
\begin{pgfscope}%
\pgfpathrectangle{\pgfqpoint{7.394209in}{0.375000in}}{\pgfqpoint{6.356833in}{5.175000in}}%
\pgfusepath{clip}%
\pgfsetbuttcap%
\pgfsetroundjoin%
\pgfsetlinewidth{1.003750pt}%
\definecolor{currentstroke}{rgb}{0.827451,0.827451,0.827451}%
\pgfsetstrokecolor{currentstroke}%
\pgfsetdash{}{0pt}%
\pgfpathmoveto{\pgfqpoint{11.320455in}{4.577130in}}%
\pgfpathcurveto{\pgfqpoint{11.331505in}{4.577130in}}{\pgfqpoint{11.342104in}{4.581521in}}{\pgfqpoint{11.349918in}{4.589334in}}%
\pgfpathcurveto{\pgfqpoint{11.357731in}{4.597148in}}{\pgfqpoint{11.362121in}{4.607747in}}{\pgfqpoint{11.362121in}{4.618797in}}%
\pgfpathcurveto{\pgfqpoint{11.362121in}{4.629847in}}{\pgfqpoint{11.357731in}{4.640446in}}{\pgfqpoint{11.349918in}{4.648260in}}%
\pgfpathcurveto{\pgfqpoint{11.342104in}{4.656074in}}{\pgfqpoint{11.331505in}{4.660464in}}{\pgfqpoint{11.320455in}{4.660464in}}%
\pgfpathcurveto{\pgfqpoint{11.309405in}{4.660464in}}{\pgfqpoint{11.298806in}{4.656074in}}{\pgfqpoint{11.290992in}{4.648260in}}%
\pgfpathcurveto{\pgfqpoint{11.283178in}{4.640446in}}{\pgfqpoint{11.278788in}{4.629847in}}{\pgfqpoint{11.278788in}{4.618797in}}%
\pgfpathcurveto{\pgfqpoint{11.278788in}{4.607747in}}{\pgfqpoint{11.283178in}{4.597148in}}{\pgfqpoint{11.290992in}{4.589334in}}%
\pgfpathcurveto{\pgfqpoint{11.298806in}{4.581521in}}{\pgfqpoint{11.309405in}{4.577130in}}{\pgfqpoint{11.320455in}{4.577130in}}%
\pgfpathlineto{\pgfqpoint{11.320455in}{4.577130in}}%
\pgfpathclose%
\pgfusepath{stroke}%
\end{pgfscope}%
\begin{pgfscope}%
\pgfpathrectangle{\pgfqpoint{7.394209in}{0.375000in}}{\pgfqpoint{6.356833in}{5.175000in}}%
\pgfusepath{clip}%
\pgfsetbuttcap%
\pgfsetroundjoin%
\pgfsetlinewidth{1.003750pt}%
\definecolor{currentstroke}{rgb}{0.827451,0.827451,0.827451}%
\pgfsetstrokecolor{currentstroke}%
\pgfsetdash{}{0pt}%
\pgfpathmoveto{\pgfqpoint{7.880326in}{1.646702in}}%
\pgfpathcurveto{\pgfqpoint{7.891376in}{1.646702in}}{\pgfqpoint{7.901975in}{1.651092in}}{\pgfqpoint{7.909789in}{1.658906in}}%
\pgfpathcurveto{\pgfqpoint{7.917603in}{1.666720in}}{\pgfqpoint{7.921993in}{1.677319in}}{\pgfqpoint{7.921993in}{1.688369in}}%
\pgfpathcurveto{\pgfqpoint{7.921993in}{1.699419in}}{\pgfqpoint{7.917603in}{1.710018in}}{\pgfqpoint{7.909789in}{1.717832in}}%
\pgfpathcurveto{\pgfqpoint{7.901975in}{1.725645in}}{\pgfqpoint{7.891376in}{1.730035in}}{\pgfqpoint{7.880326in}{1.730035in}}%
\pgfpathcurveto{\pgfqpoint{7.869276in}{1.730035in}}{\pgfqpoint{7.858677in}{1.725645in}}{\pgfqpoint{7.850863in}{1.717832in}}%
\pgfpathcurveto{\pgfqpoint{7.843050in}{1.710018in}}{\pgfqpoint{7.838660in}{1.699419in}}{\pgfqpoint{7.838660in}{1.688369in}}%
\pgfpathcurveto{\pgfqpoint{7.838660in}{1.677319in}}{\pgfqpoint{7.843050in}{1.666720in}}{\pgfqpoint{7.850863in}{1.658906in}}%
\pgfpathcurveto{\pgfqpoint{7.858677in}{1.651092in}}{\pgfqpoint{7.869276in}{1.646702in}}{\pgfqpoint{7.880326in}{1.646702in}}%
\pgfpathlineto{\pgfqpoint{7.880326in}{1.646702in}}%
\pgfpathclose%
\pgfusepath{stroke}%
\end{pgfscope}%
\begin{pgfscope}%
\pgfpathrectangle{\pgfqpoint{7.394209in}{0.375000in}}{\pgfqpoint{6.356833in}{5.175000in}}%
\pgfusepath{clip}%
\pgfsetbuttcap%
\pgfsetroundjoin%
\pgfsetlinewidth{1.003750pt}%
\definecolor{currentstroke}{rgb}{0.827451,0.827451,0.827451}%
\pgfsetstrokecolor{currentstroke}%
\pgfsetdash{}{0pt}%
\pgfpathmoveto{\pgfqpoint{12.008953in}{5.189813in}}%
\pgfpathcurveto{\pgfqpoint{12.020003in}{5.189813in}}{\pgfqpoint{12.030602in}{5.194203in}}{\pgfqpoint{12.038416in}{5.202017in}}%
\pgfpathcurveto{\pgfqpoint{12.046229in}{5.209830in}}{\pgfqpoint{12.050620in}{5.220429in}}{\pgfqpoint{12.050620in}{5.231479in}}%
\pgfpathcurveto{\pgfqpoint{12.050620in}{5.242529in}}{\pgfqpoint{12.046229in}{5.253129in}}{\pgfqpoint{12.038416in}{5.260942in}}%
\pgfpathcurveto{\pgfqpoint{12.030602in}{5.268756in}}{\pgfqpoint{12.020003in}{5.273146in}}{\pgfqpoint{12.008953in}{5.273146in}}%
\pgfpathcurveto{\pgfqpoint{11.997903in}{5.273146in}}{\pgfqpoint{11.987304in}{5.268756in}}{\pgfqpoint{11.979490in}{5.260942in}}%
\pgfpathcurveto{\pgfqpoint{11.971677in}{5.253129in}}{\pgfqpoint{11.967286in}{5.242529in}}{\pgfqpoint{11.967286in}{5.231479in}}%
\pgfpathcurveto{\pgfqpoint{11.967286in}{5.220429in}}{\pgfqpoint{11.971677in}{5.209830in}}{\pgfqpoint{11.979490in}{5.202017in}}%
\pgfpathcurveto{\pgfqpoint{11.987304in}{5.194203in}}{\pgfqpoint{11.997903in}{5.189813in}}{\pgfqpoint{12.008953in}{5.189813in}}%
\pgfpathlineto{\pgfqpoint{12.008953in}{5.189813in}}%
\pgfpathclose%
\pgfusepath{stroke}%
\end{pgfscope}%
\begin{pgfscope}%
\pgfpathrectangle{\pgfqpoint{7.394209in}{0.375000in}}{\pgfqpoint{6.356833in}{5.175000in}}%
\pgfusepath{clip}%
\pgfsetbuttcap%
\pgfsetroundjoin%
\pgfsetlinewidth{1.003750pt}%
\definecolor{currentstroke}{rgb}{0.827451,0.827451,0.827451}%
\pgfsetstrokecolor{currentstroke}%
\pgfsetdash{}{0pt}%
\pgfpathmoveto{\pgfqpoint{10.176772in}{4.925388in}}%
\pgfpathcurveto{\pgfqpoint{10.187822in}{4.925388in}}{\pgfqpoint{10.198421in}{4.929778in}}{\pgfqpoint{10.206234in}{4.937592in}}%
\pgfpathcurveto{\pgfqpoint{10.214048in}{4.945405in}}{\pgfqpoint{10.218438in}{4.956004in}}{\pgfqpoint{10.218438in}{4.967054in}}%
\pgfpathcurveto{\pgfqpoint{10.218438in}{4.978104in}}{\pgfqpoint{10.214048in}{4.988704in}}{\pgfqpoint{10.206234in}{4.996517in}}%
\pgfpathcurveto{\pgfqpoint{10.198421in}{5.004331in}}{\pgfqpoint{10.187822in}{5.008721in}}{\pgfqpoint{10.176772in}{5.008721in}}%
\pgfpathcurveto{\pgfqpoint{10.165721in}{5.008721in}}{\pgfqpoint{10.155122in}{5.004331in}}{\pgfqpoint{10.147309in}{4.996517in}}%
\pgfpathcurveto{\pgfqpoint{10.139495in}{4.988704in}}{\pgfqpoint{10.135105in}{4.978104in}}{\pgfqpoint{10.135105in}{4.967054in}}%
\pgfpathcurveto{\pgfqpoint{10.135105in}{4.956004in}}{\pgfqpoint{10.139495in}{4.945405in}}{\pgfqpoint{10.147309in}{4.937592in}}%
\pgfpathcurveto{\pgfqpoint{10.155122in}{4.929778in}}{\pgfqpoint{10.165721in}{4.925388in}}{\pgfqpoint{10.176772in}{4.925388in}}%
\pgfpathlineto{\pgfqpoint{10.176772in}{4.925388in}}%
\pgfpathclose%
\pgfusepath{stroke}%
\end{pgfscope}%
\begin{pgfscope}%
\pgfpathrectangle{\pgfqpoint{7.394209in}{0.375000in}}{\pgfqpoint{6.356833in}{5.175000in}}%
\pgfusepath{clip}%
\pgfsetbuttcap%
\pgfsetroundjoin%
\pgfsetlinewidth{1.003750pt}%
\definecolor{currentstroke}{rgb}{0.827451,0.827451,0.827451}%
\pgfsetstrokecolor{currentstroke}%
\pgfsetdash{}{0pt}%
\pgfpathmoveto{\pgfqpoint{13.694003in}{5.481997in}}%
\pgfpathcurveto{\pgfqpoint{13.705053in}{5.481997in}}{\pgfqpoint{13.715652in}{5.486387in}}{\pgfqpoint{13.723465in}{5.494201in}}%
\pgfpathcurveto{\pgfqpoint{13.731279in}{5.502015in}}{\pgfqpoint{13.735669in}{5.512614in}}{\pgfqpoint{13.735669in}{5.523664in}}%
\pgfpathcurveto{\pgfqpoint{13.735669in}{5.534714in}}{\pgfqpoint{13.731279in}{5.545313in}}{\pgfqpoint{13.723465in}{5.553126in}}%
\pgfpathcurveto{\pgfqpoint{13.715652in}{5.560940in}}{\pgfqpoint{13.705053in}{5.565330in}}{\pgfqpoint{13.694003in}{5.565330in}}%
\pgfpathcurveto{\pgfqpoint{13.682953in}{5.565330in}}{\pgfqpoint{13.672354in}{5.560940in}}{\pgfqpoint{13.664540in}{5.553126in}}%
\pgfpathcurveto{\pgfqpoint{13.656726in}{5.545313in}}{\pgfqpoint{13.652336in}{5.534714in}}{\pgfqpoint{13.652336in}{5.523664in}}%
\pgfpathcurveto{\pgfqpoint{13.652336in}{5.512614in}}{\pgfqpoint{13.656726in}{5.502015in}}{\pgfqpoint{13.664540in}{5.494201in}}%
\pgfpathcurveto{\pgfqpoint{13.672354in}{5.486387in}}{\pgfqpoint{13.682953in}{5.481997in}}{\pgfqpoint{13.694003in}{5.481997in}}%
\pgfpathlineto{\pgfqpoint{13.694003in}{5.481997in}}%
\pgfpathclose%
\pgfusepath{stroke}%
\end{pgfscope}%
\begin{pgfscope}%
\pgfpathrectangle{\pgfqpoint{7.394209in}{0.375000in}}{\pgfqpoint{6.356833in}{5.175000in}}%
\pgfusepath{clip}%
\pgfsetbuttcap%
\pgfsetroundjoin%
\pgfsetlinewidth{1.003750pt}%
\definecolor{currentstroke}{rgb}{0.827451,0.827451,0.827451}%
\pgfsetstrokecolor{currentstroke}%
\pgfsetdash{}{0pt}%
\pgfpathmoveto{\pgfqpoint{9.469749in}{1.620689in}}%
\pgfpathcurveto{\pgfqpoint{9.480799in}{1.620689in}}{\pgfqpoint{9.491398in}{1.625079in}}{\pgfqpoint{9.499212in}{1.632893in}}%
\pgfpathcurveto{\pgfqpoint{9.507026in}{1.640707in}}{\pgfqpoint{9.511416in}{1.651306in}}{\pgfqpoint{9.511416in}{1.662356in}}%
\pgfpathcurveto{\pgfqpoint{9.511416in}{1.673406in}}{\pgfqpoint{9.507026in}{1.684005in}}{\pgfqpoint{9.499212in}{1.691818in}}%
\pgfpathcurveto{\pgfqpoint{9.491398in}{1.699632in}}{\pgfqpoint{9.480799in}{1.704022in}}{\pgfqpoint{9.469749in}{1.704022in}}%
\pgfpathcurveto{\pgfqpoint{9.458699in}{1.704022in}}{\pgfqpoint{9.448100in}{1.699632in}}{\pgfqpoint{9.440286in}{1.691818in}}%
\pgfpathcurveto{\pgfqpoint{9.432473in}{1.684005in}}{\pgfqpoint{9.428083in}{1.673406in}}{\pgfqpoint{9.428083in}{1.662356in}}%
\pgfpathcurveto{\pgfqpoint{9.428083in}{1.651306in}}{\pgfqpoint{9.432473in}{1.640707in}}{\pgfqpoint{9.440286in}{1.632893in}}%
\pgfpathcurveto{\pgfqpoint{9.448100in}{1.625079in}}{\pgfqpoint{9.458699in}{1.620689in}}{\pgfqpoint{9.469749in}{1.620689in}}%
\pgfpathlineto{\pgfqpoint{9.469749in}{1.620689in}}%
\pgfpathclose%
\pgfusepath{stroke}%
\end{pgfscope}%
\begin{pgfscope}%
\pgfpathrectangle{\pgfqpoint{7.394209in}{0.375000in}}{\pgfqpoint{6.356833in}{5.175000in}}%
\pgfusepath{clip}%
\pgfsetbuttcap%
\pgfsetroundjoin%
\pgfsetlinewidth{1.003750pt}%
\definecolor{currentstroke}{rgb}{0.827451,0.827451,0.827451}%
\pgfsetstrokecolor{currentstroke}%
\pgfsetdash{}{0pt}%
\pgfpathmoveto{\pgfqpoint{8.884449in}{3.166528in}}%
\pgfpathcurveto{\pgfqpoint{8.895499in}{3.166528in}}{\pgfqpoint{8.906098in}{3.170918in}}{\pgfqpoint{8.913912in}{3.178732in}}%
\pgfpathcurveto{\pgfqpoint{8.921726in}{3.186545in}}{\pgfqpoint{8.926116in}{3.197144in}}{\pgfqpoint{8.926116in}{3.208194in}}%
\pgfpathcurveto{\pgfqpoint{8.926116in}{3.219245in}}{\pgfqpoint{8.921726in}{3.229844in}}{\pgfqpoint{8.913912in}{3.237657in}}%
\pgfpathcurveto{\pgfqpoint{8.906098in}{3.245471in}}{\pgfqpoint{8.895499in}{3.249861in}}{\pgfqpoint{8.884449in}{3.249861in}}%
\pgfpathcurveto{\pgfqpoint{8.873399in}{3.249861in}}{\pgfqpoint{8.862800in}{3.245471in}}{\pgfqpoint{8.854986in}{3.237657in}}%
\pgfpathcurveto{\pgfqpoint{8.847173in}{3.229844in}}{\pgfqpoint{8.842783in}{3.219245in}}{\pgfqpoint{8.842783in}{3.208194in}}%
\pgfpathcurveto{\pgfqpoint{8.842783in}{3.197144in}}{\pgfqpoint{8.847173in}{3.186545in}}{\pgfqpoint{8.854986in}{3.178732in}}%
\pgfpathcurveto{\pgfqpoint{8.862800in}{3.170918in}}{\pgfqpoint{8.873399in}{3.166528in}}{\pgfqpoint{8.884449in}{3.166528in}}%
\pgfpathlineto{\pgfqpoint{8.884449in}{3.166528in}}%
\pgfpathclose%
\pgfusepath{stroke}%
\end{pgfscope}%
\begin{pgfscope}%
\pgfpathrectangle{\pgfqpoint{7.394209in}{0.375000in}}{\pgfqpoint{6.356833in}{5.175000in}}%
\pgfusepath{clip}%
\pgfsetbuttcap%
\pgfsetroundjoin%
\pgfsetlinewidth{1.003750pt}%
\definecolor{currentstroke}{rgb}{0.827451,0.827451,0.827451}%
\pgfsetstrokecolor{currentstroke}%
\pgfsetdash{}{0pt}%
\pgfpathmoveto{\pgfqpoint{8.174087in}{2.235235in}}%
\pgfpathcurveto{\pgfqpoint{8.185137in}{2.235235in}}{\pgfqpoint{8.195736in}{2.239625in}}{\pgfqpoint{8.203550in}{2.247439in}}%
\pgfpathcurveto{\pgfqpoint{8.211364in}{2.255252in}}{\pgfqpoint{8.215754in}{2.265851in}}{\pgfqpoint{8.215754in}{2.276901in}}%
\pgfpathcurveto{\pgfqpoint{8.215754in}{2.287951in}}{\pgfqpoint{8.211364in}{2.298550in}}{\pgfqpoint{8.203550in}{2.306364in}}%
\pgfpathcurveto{\pgfqpoint{8.195736in}{2.314178in}}{\pgfqpoint{8.185137in}{2.318568in}}{\pgfqpoint{8.174087in}{2.318568in}}%
\pgfpathcurveto{\pgfqpoint{8.163037in}{2.318568in}}{\pgfqpoint{8.152438in}{2.314178in}}{\pgfqpoint{8.144624in}{2.306364in}}%
\pgfpathcurveto{\pgfqpoint{8.136811in}{2.298550in}}{\pgfqpoint{8.132421in}{2.287951in}}{\pgfqpoint{8.132421in}{2.276901in}}%
\pgfpathcurveto{\pgfqpoint{8.132421in}{2.265851in}}{\pgfqpoint{8.136811in}{2.255252in}}{\pgfqpoint{8.144624in}{2.247439in}}%
\pgfpathcurveto{\pgfqpoint{8.152438in}{2.239625in}}{\pgfqpoint{8.163037in}{2.235235in}}{\pgfqpoint{8.174087in}{2.235235in}}%
\pgfpathlineto{\pgfqpoint{8.174087in}{2.235235in}}%
\pgfpathclose%
\pgfusepath{stroke}%
\end{pgfscope}%
\begin{pgfscope}%
\pgfpathrectangle{\pgfqpoint{7.394209in}{0.375000in}}{\pgfqpoint{6.356833in}{5.175000in}}%
\pgfusepath{clip}%
\pgfsetbuttcap%
\pgfsetroundjoin%
\pgfsetlinewidth{1.003750pt}%
\definecolor{currentstroke}{rgb}{0.827451,0.827451,0.827451}%
\pgfsetstrokecolor{currentstroke}%
\pgfsetdash{}{0pt}%
\pgfpathmoveto{\pgfqpoint{8.126648in}{1.643934in}}%
\pgfpathcurveto{\pgfqpoint{8.137698in}{1.643934in}}{\pgfqpoint{8.148297in}{1.648324in}}{\pgfqpoint{8.156111in}{1.656138in}}%
\pgfpathcurveto{\pgfqpoint{8.163925in}{1.663952in}}{\pgfqpoint{8.168315in}{1.674551in}}{\pgfqpoint{8.168315in}{1.685601in}}%
\pgfpathcurveto{\pgfqpoint{8.168315in}{1.696651in}}{\pgfqpoint{8.163925in}{1.707250in}}{\pgfqpoint{8.156111in}{1.715064in}}%
\pgfpathcurveto{\pgfqpoint{8.148297in}{1.722877in}}{\pgfqpoint{8.137698in}{1.727267in}}{\pgfqpoint{8.126648in}{1.727267in}}%
\pgfpathcurveto{\pgfqpoint{8.115598in}{1.727267in}}{\pgfqpoint{8.104999in}{1.722877in}}{\pgfqpoint{8.097185in}{1.715064in}}%
\pgfpathcurveto{\pgfqpoint{8.089372in}{1.707250in}}{\pgfqpoint{8.084981in}{1.696651in}}{\pgfqpoint{8.084981in}{1.685601in}}%
\pgfpathcurveto{\pgfqpoint{8.084981in}{1.674551in}}{\pgfqpoint{8.089372in}{1.663952in}}{\pgfqpoint{8.097185in}{1.656138in}}%
\pgfpathcurveto{\pgfqpoint{8.104999in}{1.648324in}}{\pgfqpoint{8.115598in}{1.643934in}}{\pgfqpoint{8.126648in}{1.643934in}}%
\pgfpathlineto{\pgfqpoint{8.126648in}{1.643934in}}%
\pgfpathclose%
\pgfusepath{stroke}%
\end{pgfscope}%
\begin{pgfscope}%
\pgfpathrectangle{\pgfqpoint{7.394209in}{0.375000in}}{\pgfqpoint{6.356833in}{5.175000in}}%
\pgfusepath{clip}%
\pgfsetbuttcap%
\pgfsetroundjoin%
\pgfsetlinewidth{1.003750pt}%
\definecolor{currentstroke}{rgb}{0.827451,0.827451,0.827451}%
\pgfsetstrokecolor{currentstroke}%
\pgfsetdash{}{0pt}%
\pgfpathmoveto{\pgfqpoint{7.760412in}{1.260710in}}%
\pgfpathcurveto{\pgfqpoint{7.771462in}{1.260710in}}{\pgfqpoint{7.782061in}{1.265100in}}{\pgfqpoint{7.789875in}{1.272914in}}%
\pgfpathcurveto{\pgfqpoint{7.797689in}{1.280727in}}{\pgfqpoint{7.802079in}{1.291326in}}{\pgfqpoint{7.802079in}{1.302376in}}%
\pgfpathcurveto{\pgfqpoint{7.802079in}{1.313427in}}{\pgfqpoint{7.797689in}{1.324026in}}{\pgfqpoint{7.789875in}{1.331839in}}%
\pgfpathcurveto{\pgfqpoint{7.782061in}{1.339653in}}{\pgfqpoint{7.771462in}{1.344043in}}{\pgfqpoint{7.760412in}{1.344043in}}%
\pgfpathcurveto{\pgfqpoint{7.749362in}{1.344043in}}{\pgfqpoint{7.738763in}{1.339653in}}{\pgfqpoint{7.730949in}{1.331839in}}%
\pgfpathcurveto{\pgfqpoint{7.723136in}{1.324026in}}{\pgfqpoint{7.718746in}{1.313427in}}{\pgfqpoint{7.718746in}{1.302376in}}%
\pgfpathcurveto{\pgfqpoint{7.718746in}{1.291326in}}{\pgfqpoint{7.723136in}{1.280727in}}{\pgfqpoint{7.730949in}{1.272914in}}%
\pgfpathcurveto{\pgfqpoint{7.738763in}{1.265100in}}{\pgfqpoint{7.749362in}{1.260710in}}{\pgfqpoint{7.760412in}{1.260710in}}%
\pgfpathlineto{\pgfqpoint{7.760412in}{1.260710in}}%
\pgfpathclose%
\pgfusepath{stroke}%
\end{pgfscope}%
\begin{pgfscope}%
\pgfpathrectangle{\pgfqpoint{7.394209in}{0.375000in}}{\pgfqpoint{6.356833in}{5.175000in}}%
\pgfusepath{clip}%
\pgfsetbuttcap%
\pgfsetroundjoin%
\pgfsetlinewidth{1.003750pt}%
\definecolor{currentstroke}{rgb}{0.827451,0.827451,0.827451}%
\pgfsetstrokecolor{currentstroke}%
\pgfsetdash{}{0pt}%
\pgfpathmoveto{\pgfqpoint{12.598429in}{5.434960in}}%
\pgfpathcurveto{\pgfqpoint{12.609479in}{5.434960in}}{\pgfqpoint{12.620078in}{5.439351in}}{\pgfqpoint{12.627892in}{5.447164in}}%
\pgfpathcurveto{\pgfqpoint{12.635706in}{5.454978in}}{\pgfqpoint{12.640096in}{5.465577in}}{\pgfqpoint{12.640096in}{5.476627in}}%
\pgfpathcurveto{\pgfqpoint{12.640096in}{5.487677in}}{\pgfqpoint{12.635706in}{5.498276in}}{\pgfqpoint{12.627892in}{5.506090in}}%
\pgfpathcurveto{\pgfqpoint{12.620078in}{5.513903in}}{\pgfqpoint{12.609479in}{5.518294in}}{\pgfqpoint{12.598429in}{5.518294in}}%
\pgfpathcurveto{\pgfqpoint{12.587379in}{5.518294in}}{\pgfqpoint{12.576780in}{5.513903in}}{\pgfqpoint{12.568966in}{5.506090in}}%
\pgfpathcurveto{\pgfqpoint{12.561153in}{5.498276in}}{\pgfqpoint{12.556763in}{5.487677in}}{\pgfqpoint{12.556763in}{5.476627in}}%
\pgfpathcurveto{\pgfqpoint{12.556763in}{5.465577in}}{\pgfqpoint{12.561153in}{5.454978in}}{\pgfqpoint{12.568966in}{5.447164in}}%
\pgfpathcurveto{\pgfqpoint{12.576780in}{5.439351in}}{\pgfqpoint{12.587379in}{5.434960in}}{\pgfqpoint{12.598429in}{5.434960in}}%
\pgfpathlineto{\pgfqpoint{12.598429in}{5.434960in}}%
\pgfpathclose%
\pgfusepath{stroke}%
\end{pgfscope}%
\begin{pgfscope}%
\pgfpathrectangle{\pgfqpoint{7.394209in}{0.375000in}}{\pgfqpoint{6.356833in}{5.175000in}}%
\pgfusepath{clip}%
\pgfsetbuttcap%
\pgfsetroundjoin%
\pgfsetlinewidth{1.003750pt}%
\definecolor{currentstroke}{rgb}{0.827451,0.827451,0.827451}%
\pgfsetstrokecolor{currentstroke}%
\pgfsetdash{}{0pt}%
\pgfpathmoveto{\pgfqpoint{12.152111in}{5.357979in}}%
\pgfpathcurveto{\pgfqpoint{12.163161in}{5.357979in}}{\pgfqpoint{12.173760in}{5.362370in}}{\pgfqpoint{12.181574in}{5.370183in}}%
\pgfpathcurveto{\pgfqpoint{12.189387in}{5.377997in}}{\pgfqpoint{12.193778in}{5.388596in}}{\pgfqpoint{12.193778in}{5.399646in}}%
\pgfpathcurveto{\pgfqpoint{12.193778in}{5.410696in}}{\pgfqpoint{12.189387in}{5.421295in}}{\pgfqpoint{12.181574in}{5.429109in}}%
\pgfpathcurveto{\pgfqpoint{12.173760in}{5.436923in}}{\pgfqpoint{12.163161in}{5.441313in}}{\pgfqpoint{12.152111in}{5.441313in}}%
\pgfpathcurveto{\pgfqpoint{12.141061in}{5.441313in}}{\pgfqpoint{12.130462in}{5.436923in}}{\pgfqpoint{12.122648in}{5.429109in}}%
\pgfpathcurveto{\pgfqpoint{12.114835in}{5.421295in}}{\pgfqpoint{12.110444in}{5.410696in}}{\pgfqpoint{12.110444in}{5.399646in}}%
\pgfpathcurveto{\pgfqpoint{12.110444in}{5.388596in}}{\pgfqpoint{12.114835in}{5.377997in}}{\pgfqpoint{12.122648in}{5.370183in}}%
\pgfpathcurveto{\pgfqpoint{12.130462in}{5.362370in}}{\pgfqpoint{12.141061in}{5.357979in}}{\pgfqpoint{12.152111in}{5.357979in}}%
\pgfpathlineto{\pgfqpoint{12.152111in}{5.357979in}}%
\pgfpathclose%
\pgfusepath{stroke}%
\end{pgfscope}%
\begin{pgfscope}%
\pgfpathrectangle{\pgfqpoint{7.394209in}{0.375000in}}{\pgfqpoint{6.356833in}{5.175000in}}%
\pgfusepath{clip}%
\pgfsetbuttcap%
\pgfsetroundjoin%
\pgfsetlinewidth{1.003750pt}%
\definecolor{currentstroke}{rgb}{0.827451,0.827451,0.827451}%
\pgfsetstrokecolor{currentstroke}%
\pgfsetdash{}{0pt}%
\pgfpathmoveto{\pgfqpoint{9.760596in}{4.263196in}}%
\pgfpathcurveto{\pgfqpoint{9.771646in}{4.263196in}}{\pgfqpoint{9.782245in}{4.267587in}}{\pgfqpoint{9.790059in}{4.275400in}}%
\pgfpathcurveto{\pgfqpoint{9.797873in}{4.283214in}}{\pgfqpoint{9.802263in}{4.293813in}}{\pgfqpoint{9.802263in}{4.304863in}}%
\pgfpathcurveto{\pgfqpoint{9.802263in}{4.315913in}}{\pgfqpoint{9.797873in}{4.326512in}}{\pgfqpoint{9.790059in}{4.334326in}}%
\pgfpathcurveto{\pgfqpoint{9.782245in}{4.342139in}}{\pgfqpoint{9.771646in}{4.346530in}}{\pgfqpoint{9.760596in}{4.346530in}}%
\pgfpathcurveto{\pgfqpoint{9.749546in}{4.346530in}}{\pgfqpoint{9.738947in}{4.342139in}}{\pgfqpoint{9.731134in}{4.334326in}}%
\pgfpathcurveto{\pgfqpoint{9.723320in}{4.326512in}}{\pgfqpoint{9.718930in}{4.315913in}}{\pgfqpoint{9.718930in}{4.304863in}}%
\pgfpathcurveto{\pgfqpoint{9.718930in}{4.293813in}}{\pgfqpoint{9.723320in}{4.283214in}}{\pgfqpoint{9.731134in}{4.275400in}}%
\pgfpathcurveto{\pgfqpoint{9.738947in}{4.267587in}}{\pgfqpoint{9.749546in}{4.263196in}}{\pgfqpoint{9.760596in}{4.263196in}}%
\pgfpathlineto{\pgfqpoint{9.760596in}{4.263196in}}%
\pgfpathclose%
\pgfusepath{stroke}%
\end{pgfscope}%
\begin{pgfscope}%
\pgfpathrectangle{\pgfqpoint{7.394209in}{0.375000in}}{\pgfqpoint{6.356833in}{5.175000in}}%
\pgfusepath{clip}%
\pgfsetbuttcap%
\pgfsetroundjoin%
\pgfsetlinewidth{1.003750pt}%
\definecolor{currentstroke}{rgb}{0.827451,0.827451,0.827451}%
\pgfsetstrokecolor{currentstroke}%
\pgfsetdash{}{0pt}%
\pgfpathmoveto{\pgfqpoint{10.357993in}{5.308229in}}%
\pgfpathcurveto{\pgfqpoint{10.369043in}{5.308229in}}{\pgfqpoint{10.379642in}{5.312620in}}{\pgfqpoint{10.387456in}{5.320433in}}%
\pgfpathcurveto{\pgfqpoint{10.395269in}{5.328247in}}{\pgfqpoint{10.399660in}{5.338846in}}{\pgfqpoint{10.399660in}{5.349896in}}%
\pgfpathcurveto{\pgfqpoint{10.399660in}{5.360946in}}{\pgfqpoint{10.395269in}{5.371545in}}{\pgfqpoint{10.387456in}{5.379359in}}%
\pgfpathcurveto{\pgfqpoint{10.379642in}{5.387172in}}{\pgfqpoint{10.369043in}{5.391563in}}{\pgfqpoint{10.357993in}{5.391563in}}%
\pgfpathcurveto{\pgfqpoint{10.346943in}{5.391563in}}{\pgfqpoint{10.336344in}{5.387172in}}{\pgfqpoint{10.328530in}{5.379359in}}%
\pgfpathcurveto{\pgfqpoint{10.320717in}{5.371545in}}{\pgfqpoint{10.316326in}{5.360946in}}{\pgfqpoint{10.316326in}{5.349896in}}%
\pgfpathcurveto{\pgfqpoint{10.316326in}{5.338846in}}{\pgfqpoint{10.320717in}{5.328247in}}{\pgfqpoint{10.328530in}{5.320433in}}%
\pgfpathcurveto{\pgfqpoint{10.336344in}{5.312620in}}{\pgfqpoint{10.346943in}{5.308229in}}{\pgfqpoint{10.357993in}{5.308229in}}%
\pgfpathlineto{\pgfqpoint{10.357993in}{5.308229in}}%
\pgfpathclose%
\pgfusepath{stroke}%
\end{pgfscope}%
\begin{pgfscope}%
\pgfpathrectangle{\pgfqpoint{7.394209in}{0.375000in}}{\pgfqpoint{6.356833in}{5.175000in}}%
\pgfusepath{clip}%
\pgfsetbuttcap%
\pgfsetroundjoin%
\pgfsetlinewidth{1.003750pt}%
\definecolor{currentstroke}{rgb}{0.827451,0.827451,0.827451}%
\pgfsetstrokecolor{currentstroke}%
\pgfsetdash{}{0pt}%
\pgfpathmoveto{\pgfqpoint{13.687757in}{5.490720in}}%
\pgfpathcurveto{\pgfqpoint{13.698807in}{5.490720in}}{\pgfqpoint{13.709406in}{5.495110in}}{\pgfqpoint{13.717220in}{5.502924in}}%
\pgfpathcurveto{\pgfqpoint{13.725033in}{5.510737in}}{\pgfqpoint{13.729424in}{5.521337in}}{\pgfqpoint{13.729424in}{5.532387in}}%
\pgfpathcurveto{\pgfqpoint{13.729424in}{5.543437in}}{\pgfqpoint{13.725033in}{5.554036in}}{\pgfqpoint{13.717220in}{5.561849in}}%
\pgfpathcurveto{\pgfqpoint{13.709406in}{5.569663in}}{\pgfqpoint{13.698807in}{5.574053in}}{\pgfqpoint{13.687757in}{5.574053in}}%
\pgfpathcurveto{\pgfqpoint{13.676707in}{5.574053in}}{\pgfqpoint{13.666108in}{5.569663in}}{\pgfqpoint{13.658294in}{5.561849in}}%
\pgfpathcurveto{\pgfqpoint{13.650481in}{5.554036in}}{\pgfqpoint{13.646090in}{5.543437in}}{\pgfqpoint{13.646090in}{5.532387in}}%
\pgfpathcurveto{\pgfqpoint{13.646090in}{5.521337in}}{\pgfqpoint{13.650481in}{5.510737in}}{\pgfqpoint{13.658294in}{5.502924in}}%
\pgfpathcurveto{\pgfqpoint{13.666108in}{5.495110in}}{\pgfqpoint{13.676707in}{5.490720in}}{\pgfqpoint{13.687757in}{5.490720in}}%
\pgfpathlineto{\pgfqpoint{13.687757in}{5.490720in}}%
\pgfpathclose%
\pgfusepath{stroke}%
\end{pgfscope}%
\begin{pgfscope}%
\pgfpathrectangle{\pgfqpoint{7.394209in}{0.375000in}}{\pgfqpoint{6.356833in}{5.175000in}}%
\pgfusepath{clip}%
\pgfsetbuttcap%
\pgfsetroundjoin%
\pgfsetlinewidth{1.003750pt}%
\definecolor{currentstroke}{rgb}{0.827451,0.827451,0.827451}%
\pgfsetstrokecolor{currentstroke}%
\pgfsetdash{}{0pt}%
\pgfpathmoveto{\pgfqpoint{13.541705in}{5.481997in}}%
\pgfpathcurveto{\pgfqpoint{13.552755in}{5.481997in}}{\pgfqpoint{13.563354in}{5.486387in}}{\pgfqpoint{13.571168in}{5.494201in}}%
\pgfpathcurveto{\pgfqpoint{13.578982in}{5.502015in}}{\pgfqpoint{13.583372in}{5.512614in}}{\pgfqpoint{13.583372in}{5.523664in}}%
\pgfpathcurveto{\pgfqpoint{13.583372in}{5.534714in}}{\pgfqpoint{13.578982in}{5.545313in}}{\pgfqpoint{13.571168in}{5.553126in}}%
\pgfpathcurveto{\pgfqpoint{13.563354in}{5.560940in}}{\pgfqpoint{13.552755in}{5.565330in}}{\pgfqpoint{13.541705in}{5.565330in}}%
\pgfpathcurveto{\pgfqpoint{13.530655in}{5.565330in}}{\pgfqpoint{13.520056in}{5.560940in}}{\pgfqpoint{13.512242in}{5.553126in}}%
\pgfpathcurveto{\pgfqpoint{13.504429in}{5.545313in}}{\pgfqpoint{13.500039in}{5.534714in}}{\pgfqpoint{13.500039in}{5.523664in}}%
\pgfpathcurveto{\pgfqpoint{13.500039in}{5.512614in}}{\pgfqpoint{13.504429in}{5.502015in}}{\pgfqpoint{13.512242in}{5.494201in}}%
\pgfpathcurveto{\pgfqpoint{13.520056in}{5.486387in}}{\pgfqpoint{13.530655in}{5.481997in}}{\pgfqpoint{13.541705in}{5.481997in}}%
\pgfpathlineto{\pgfqpoint{13.541705in}{5.481997in}}%
\pgfpathclose%
\pgfusepath{stroke}%
\end{pgfscope}%
\begin{pgfscope}%
\pgfpathrectangle{\pgfqpoint{7.394209in}{0.375000in}}{\pgfqpoint{6.356833in}{5.175000in}}%
\pgfusepath{clip}%
\pgfsetbuttcap%
\pgfsetroundjoin%
\pgfsetlinewidth{1.003750pt}%
\definecolor{currentstroke}{rgb}{0.827451,0.827451,0.827451}%
\pgfsetstrokecolor{currentstroke}%
\pgfsetdash{}{0pt}%
\pgfpathmoveto{\pgfqpoint{8.174087in}{1.881343in}}%
\pgfpathcurveto{\pgfqpoint{8.185137in}{1.881343in}}{\pgfqpoint{8.195736in}{1.885733in}}{\pgfqpoint{8.203550in}{1.893546in}}%
\pgfpathcurveto{\pgfqpoint{8.211364in}{1.901360in}}{\pgfqpoint{8.215754in}{1.911959in}}{\pgfqpoint{8.215754in}{1.923009in}}%
\pgfpathcurveto{\pgfqpoint{8.215754in}{1.934059in}}{\pgfqpoint{8.211364in}{1.944658in}}{\pgfqpoint{8.203550in}{1.952472in}}%
\pgfpathcurveto{\pgfqpoint{8.195736in}{1.960286in}}{\pgfqpoint{8.185137in}{1.964676in}}{\pgfqpoint{8.174087in}{1.964676in}}%
\pgfpathcurveto{\pgfqpoint{8.163037in}{1.964676in}}{\pgfqpoint{8.152438in}{1.960286in}}{\pgfqpoint{8.144624in}{1.952472in}}%
\pgfpathcurveto{\pgfqpoint{8.136811in}{1.944658in}}{\pgfqpoint{8.132421in}{1.934059in}}{\pgfqpoint{8.132421in}{1.923009in}}%
\pgfpathcurveto{\pgfqpoint{8.132421in}{1.911959in}}{\pgfqpoint{8.136811in}{1.901360in}}{\pgfqpoint{8.144624in}{1.893546in}}%
\pgfpathcurveto{\pgfqpoint{8.152438in}{1.885733in}}{\pgfqpoint{8.163037in}{1.881343in}}{\pgfqpoint{8.174087in}{1.881343in}}%
\pgfpathlineto{\pgfqpoint{8.174087in}{1.881343in}}%
\pgfpathclose%
\pgfusepath{stroke}%
\end{pgfscope}%
\begin{pgfscope}%
\pgfpathrectangle{\pgfqpoint{7.394209in}{0.375000in}}{\pgfqpoint{6.356833in}{5.175000in}}%
\pgfusepath{clip}%
\pgfsetbuttcap%
\pgfsetroundjoin%
\pgfsetlinewidth{1.003750pt}%
\definecolor{currentstroke}{rgb}{0.827451,0.827451,0.827451}%
\pgfsetstrokecolor{currentstroke}%
\pgfsetdash{}{0pt}%
\pgfpathmoveto{\pgfqpoint{11.921904in}{5.329219in}}%
\pgfpathcurveto{\pgfqpoint{11.932954in}{5.329219in}}{\pgfqpoint{11.943553in}{5.333609in}}{\pgfqpoint{11.951367in}{5.341422in}}%
\pgfpathcurveto{\pgfqpoint{11.959180in}{5.349236in}}{\pgfqpoint{11.963570in}{5.359835in}}{\pgfqpoint{11.963570in}{5.370885in}}%
\pgfpathcurveto{\pgfqpoint{11.963570in}{5.381935in}}{\pgfqpoint{11.959180in}{5.392534in}}{\pgfqpoint{11.951367in}{5.400348in}}%
\pgfpathcurveto{\pgfqpoint{11.943553in}{5.408162in}}{\pgfqpoint{11.932954in}{5.412552in}}{\pgfqpoint{11.921904in}{5.412552in}}%
\pgfpathcurveto{\pgfqpoint{11.910854in}{5.412552in}}{\pgfqpoint{11.900255in}{5.408162in}}{\pgfqpoint{11.892441in}{5.400348in}}%
\pgfpathcurveto{\pgfqpoint{11.884627in}{5.392534in}}{\pgfqpoint{11.880237in}{5.381935in}}{\pgfqpoint{11.880237in}{5.370885in}}%
\pgfpathcurveto{\pgfqpoint{11.880237in}{5.359835in}}{\pgfqpoint{11.884627in}{5.349236in}}{\pgfqpoint{11.892441in}{5.341422in}}%
\pgfpathcurveto{\pgfqpoint{11.900255in}{5.333609in}}{\pgfqpoint{11.910854in}{5.329219in}}{\pgfqpoint{11.921904in}{5.329219in}}%
\pgfpathlineto{\pgfqpoint{11.921904in}{5.329219in}}%
\pgfpathclose%
\pgfusepath{stroke}%
\end{pgfscope}%
\begin{pgfscope}%
\pgfpathrectangle{\pgfqpoint{7.394209in}{0.375000in}}{\pgfqpoint{6.356833in}{5.175000in}}%
\pgfusepath{clip}%
\pgfsetbuttcap%
\pgfsetroundjoin%
\pgfsetlinewidth{1.003750pt}%
\definecolor{currentstroke}{rgb}{0.827451,0.827451,0.827451}%
\pgfsetstrokecolor{currentstroke}%
\pgfsetdash{}{0pt}%
\pgfpathmoveto{\pgfqpoint{8.419667in}{2.434992in}}%
\pgfpathcurveto{\pgfqpoint{8.430717in}{2.434992in}}{\pgfqpoint{8.441317in}{2.439382in}}{\pgfqpoint{8.449130in}{2.447196in}}%
\pgfpathcurveto{\pgfqpoint{8.456944in}{2.455010in}}{\pgfqpoint{8.461334in}{2.465609in}}{\pgfqpoint{8.461334in}{2.476659in}}%
\pgfpathcurveto{\pgfqpoint{8.461334in}{2.487709in}}{\pgfqpoint{8.456944in}{2.498308in}}{\pgfqpoint{8.449130in}{2.506121in}}%
\pgfpathcurveto{\pgfqpoint{8.441317in}{2.513935in}}{\pgfqpoint{8.430717in}{2.518325in}}{\pgfqpoint{8.419667in}{2.518325in}}%
\pgfpathcurveto{\pgfqpoint{8.408617in}{2.518325in}}{\pgfqpoint{8.398018in}{2.513935in}}{\pgfqpoint{8.390205in}{2.506121in}}%
\pgfpathcurveto{\pgfqpoint{8.382391in}{2.498308in}}{\pgfqpoint{8.378001in}{2.487709in}}{\pgfqpoint{8.378001in}{2.476659in}}%
\pgfpathcurveto{\pgfqpoint{8.378001in}{2.465609in}}{\pgfqpoint{8.382391in}{2.455010in}}{\pgfqpoint{8.390205in}{2.447196in}}%
\pgfpathcurveto{\pgfqpoint{8.398018in}{2.439382in}}{\pgfqpoint{8.408617in}{2.434992in}}{\pgfqpoint{8.419667in}{2.434992in}}%
\pgfpathlineto{\pgfqpoint{8.419667in}{2.434992in}}%
\pgfpathclose%
\pgfusepath{stroke}%
\end{pgfscope}%
\begin{pgfscope}%
\pgfpathrectangle{\pgfqpoint{7.394209in}{0.375000in}}{\pgfqpoint{6.356833in}{5.175000in}}%
\pgfusepath{clip}%
\pgfsetbuttcap%
\pgfsetroundjoin%
\pgfsetlinewidth{1.003750pt}%
\definecolor{currentstroke}{rgb}{0.827451,0.827451,0.827451}%
\pgfsetstrokecolor{currentstroke}%
\pgfsetdash{}{0pt}%
\pgfpathmoveto{\pgfqpoint{8.613294in}{3.144455in}}%
\pgfpathcurveto{\pgfqpoint{8.624344in}{3.144455in}}{\pgfqpoint{8.634943in}{3.148845in}}{\pgfqpoint{8.642757in}{3.156659in}}%
\pgfpathcurveto{\pgfqpoint{8.650570in}{3.164472in}}{\pgfqpoint{8.654961in}{3.175071in}}{\pgfqpoint{8.654961in}{3.186122in}}%
\pgfpathcurveto{\pgfqpoint{8.654961in}{3.197172in}}{\pgfqpoint{8.650570in}{3.207771in}}{\pgfqpoint{8.642757in}{3.215584in}}%
\pgfpathcurveto{\pgfqpoint{8.634943in}{3.223398in}}{\pgfqpoint{8.624344in}{3.227788in}}{\pgfqpoint{8.613294in}{3.227788in}}%
\pgfpathcurveto{\pgfqpoint{8.602244in}{3.227788in}}{\pgfqpoint{8.591645in}{3.223398in}}{\pgfqpoint{8.583831in}{3.215584in}}%
\pgfpathcurveto{\pgfqpoint{8.576018in}{3.207771in}}{\pgfqpoint{8.571627in}{3.197172in}}{\pgfqpoint{8.571627in}{3.186122in}}%
\pgfpathcurveto{\pgfqpoint{8.571627in}{3.175071in}}{\pgfqpoint{8.576018in}{3.164472in}}{\pgfqpoint{8.583831in}{3.156659in}}%
\pgfpathcurveto{\pgfqpoint{8.591645in}{3.148845in}}{\pgfqpoint{8.602244in}{3.144455in}}{\pgfqpoint{8.613294in}{3.144455in}}%
\pgfpathlineto{\pgfqpoint{8.613294in}{3.144455in}}%
\pgfpathclose%
\pgfusepath{stroke}%
\end{pgfscope}%
\begin{pgfscope}%
\pgfpathrectangle{\pgfqpoint{7.394209in}{0.375000in}}{\pgfqpoint{6.356833in}{5.175000in}}%
\pgfusepath{clip}%
\pgfsetbuttcap%
\pgfsetroundjoin%
\pgfsetlinewidth{1.003750pt}%
\definecolor{currentstroke}{rgb}{0.827451,0.827451,0.827451}%
\pgfsetstrokecolor{currentstroke}%
\pgfsetdash{}{0pt}%
\pgfpathmoveto{\pgfqpoint{10.230085in}{5.454900in}}%
\pgfpathcurveto{\pgfqpoint{10.241135in}{5.454900in}}{\pgfqpoint{10.251734in}{5.459291in}}{\pgfqpoint{10.259548in}{5.467104in}}%
\pgfpathcurveto{\pgfqpoint{10.267362in}{5.474918in}}{\pgfqpoint{10.271752in}{5.485517in}}{\pgfqpoint{10.271752in}{5.496567in}}%
\pgfpathcurveto{\pgfqpoint{10.271752in}{5.507617in}}{\pgfqpoint{10.267362in}{5.518216in}}{\pgfqpoint{10.259548in}{5.526030in}}%
\pgfpathcurveto{\pgfqpoint{10.251734in}{5.533844in}}{\pgfqpoint{10.241135in}{5.538234in}}{\pgfqpoint{10.230085in}{5.538234in}}%
\pgfpathcurveto{\pgfqpoint{10.219035in}{5.538234in}}{\pgfqpoint{10.208436in}{5.533844in}}{\pgfqpoint{10.200622in}{5.526030in}}%
\pgfpathcurveto{\pgfqpoint{10.192809in}{5.518216in}}{\pgfqpoint{10.188419in}{5.507617in}}{\pgfqpoint{10.188419in}{5.496567in}}%
\pgfpathcurveto{\pgfqpoint{10.188419in}{5.485517in}}{\pgfqpoint{10.192809in}{5.474918in}}{\pgfqpoint{10.200622in}{5.467104in}}%
\pgfpathcurveto{\pgfqpoint{10.208436in}{5.459291in}}{\pgfqpoint{10.219035in}{5.454900in}}{\pgfqpoint{10.230085in}{5.454900in}}%
\pgfpathlineto{\pgfqpoint{10.230085in}{5.454900in}}%
\pgfpathclose%
\pgfusepath{stroke}%
\end{pgfscope}%
\begin{pgfscope}%
\pgfpathrectangle{\pgfqpoint{7.394209in}{0.375000in}}{\pgfqpoint{6.356833in}{5.175000in}}%
\pgfusepath{clip}%
\pgfsetbuttcap%
\pgfsetroundjoin%
\pgfsetlinewidth{1.003750pt}%
\definecolor{currentstroke}{rgb}{0.827451,0.827451,0.827451}%
\pgfsetstrokecolor{currentstroke}%
\pgfsetdash{}{0pt}%
\pgfpathmoveto{\pgfqpoint{7.537275in}{0.905470in}}%
\pgfpathcurveto{\pgfqpoint{7.548325in}{0.905470in}}{\pgfqpoint{7.558924in}{0.909860in}}{\pgfqpoint{7.566738in}{0.917673in}}%
\pgfpathcurveto{\pgfqpoint{7.574552in}{0.925487in}}{\pgfqpoint{7.578942in}{0.936086in}}{\pgfqpoint{7.578942in}{0.947136in}}%
\pgfpathcurveto{\pgfqpoint{7.578942in}{0.958186in}}{\pgfqpoint{7.574552in}{0.968785in}}{\pgfqpoint{7.566738in}{0.976599in}}%
\pgfpathcurveto{\pgfqpoint{7.558924in}{0.984413in}}{\pgfqpoint{7.548325in}{0.988803in}}{\pgfqpoint{7.537275in}{0.988803in}}%
\pgfpathcurveto{\pgfqpoint{7.526225in}{0.988803in}}{\pgfqpoint{7.515626in}{0.984413in}}{\pgfqpoint{7.507812in}{0.976599in}}%
\pgfpathcurveto{\pgfqpoint{7.499999in}{0.968785in}}{\pgfqpoint{7.495608in}{0.958186in}}{\pgfqpoint{7.495608in}{0.947136in}}%
\pgfpathcurveto{\pgfqpoint{7.495608in}{0.936086in}}{\pgfqpoint{7.499999in}{0.925487in}}{\pgfqpoint{7.507812in}{0.917673in}}%
\pgfpathcurveto{\pgfqpoint{7.515626in}{0.909860in}}{\pgfqpoint{7.526225in}{0.905470in}}{\pgfqpoint{7.537275in}{0.905470in}}%
\pgfpathlineto{\pgfqpoint{7.537275in}{0.905470in}}%
\pgfpathclose%
\pgfusepath{stroke}%
\end{pgfscope}%
\begin{pgfscope}%
\pgfpathrectangle{\pgfqpoint{7.394209in}{0.375000in}}{\pgfqpoint{6.356833in}{5.175000in}}%
\pgfusepath{clip}%
\pgfsetbuttcap%
\pgfsetroundjoin%
\pgfsetlinewidth{1.003750pt}%
\definecolor{currentstroke}{rgb}{0.827451,0.827451,0.827451}%
\pgfsetstrokecolor{currentstroke}%
\pgfsetdash{}{0pt}%
\pgfpathmoveto{\pgfqpoint{7.818333in}{0.431929in}}%
\pgfpathcurveto{\pgfqpoint{7.829383in}{0.431929in}}{\pgfqpoint{7.839982in}{0.436319in}}{\pgfqpoint{7.847795in}{0.444133in}}%
\pgfpathcurveto{\pgfqpoint{7.855609in}{0.451946in}}{\pgfqpoint{7.859999in}{0.462545in}}{\pgfqpoint{7.859999in}{0.473595in}}%
\pgfpathcurveto{\pgfqpoint{7.859999in}{0.484645in}}{\pgfqpoint{7.855609in}{0.495244in}}{\pgfqpoint{7.847795in}{0.503058in}}%
\pgfpathcurveto{\pgfqpoint{7.839982in}{0.510872in}}{\pgfqpoint{7.829383in}{0.515262in}}{\pgfqpoint{7.818333in}{0.515262in}}%
\pgfpathcurveto{\pgfqpoint{7.807283in}{0.515262in}}{\pgfqpoint{7.796683in}{0.510872in}}{\pgfqpoint{7.788870in}{0.503058in}}%
\pgfpathcurveto{\pgfqpoint{7.781056in}{0.495244in}}{\pgfqpoint{7.776666in}{0.484645in}}{\pgfqpoint{7.776666in}{0.473595in}}%
\pgfpathcurveto{\pgfqpoint{7.776666in}{0.462545in}}{\pgfqpoint{7.781056in}{0.451946in}}{\pgfqpoint{7.788870in}{0.444133in}}%
\pgfpathcurveto{\pgfqpoint{7.796683in}{0.436319in}}{\pgfqpoint{7.807283in}{0.431929in}}{\pgfqpoint{7.818333in}{0.431929in}}%
\pgfpathlineto{\pgfqpoint{7.818333in}{0.431929in}}%
\pgfpathclose%
\pgfusepath{stroke}%
\end{pgfscope}%
\begin{pgfscope}%
\pgfpathrectangle{\pgfqpoint{7.394209in}{0.375000in}}{\pgfqpoint{6.356833in}{5.175000in}}%
\pgfusepath{clip}%
\pgfsetbuttcap%
\pgfsetroundjoin%
\pgfsetlinewidth{1.003750pt}%
\definecolor{currentstroke}{rgb}{0.827451,0.827451,0.827451}%
\pgfsetstrokecolor{currentstroke}%
\pgfsetdash{}{0pt}%
\pgfpathmoveto{\pgfqpoint{12.043861in}{5.391981in}}%
\pgfpathcurveto{\pgfqpoint{12.054911in}{5.391981in}}{\pgfqpoint{12.065510in}{5.396371in}}{\pgfqpoint{12.073324in}{5.404185in}}%
\pgfpathcurveto{\pgfqpoint{12.081138in}{5.411999in}}{\pgfqpoint{12.085528in}{5.422598in}}{\pgfqpoint{12.085528in}{5.433648in}}%
\pgfpathcurveto{\pgfqpoint{12.085528in}{5.444698in}}{\pgfqpoint{12.081138in}{5.455297in}}{\pgfqpoint{12.073324in}{5.463111in}}%
\pgfpathcurveto{\pgfqpoint{12.065510in}{5.470924in}}{\pgfqpoint{12.054911in}{5.475314in}}{\pgfqpoint{12.043861in}{5.475314in}}%
\pgfpathcurveto{\pgfqpoint{12.032811in}{5.475314in}}{\pgfqpoint{12.022212in}{5.470924in}}{\pgfqpoint{12.014398in}{5.463111in}}%
\pgfpathcurveto{\pgfqpoint{12.006585in}{5.455297in}}{\pgfqpoint{12.002194in}{5.444698in}}{\pgfqpoint{12.002194in}{5.433648in}}%
\pgfpathcurveto{\pgfqpoint{12.002194in}{5.422598in}}{\pgfqpoint{12.006585in}{5.411999in}}{\pgfqpoint{12.014398in}{5.404185in}}%
\pgfpathcurveto{\pgfqpoint{12.022212in}{5.396371in}}{\pgfqpoint{12.032811in}{5.391981in}}{\pgfqpoint{12.043861in}{5.391981in}}%
\pgfpathlineto{\pgfqpoint{12.043861in}{5.391981in}}%
\pgfpathclose%
\pgfusepath{stroke}%
\end{pgfscope}%
\begin{pgfscope}%
\pgfpathrectangle{\pgfqpoint{7.394209in}{0.375000in}}{\pgfqpoint{6.356833in}{5.175000in}}%
\pgfusepath{clip}%
\pgfsetbuttcap%
\pgfsetroundjoin%
\pgfsetlinewidth{1.003750pt}%
\definecolor{currentstroke}{rgb}{0.827451,0.827451,0.827451}%
\pgfsetstrokecolor{currentstroke}%
\pgfsetdash{}{0pt}%
\pgfpathmoveto{\pgfqpoint{13.283104in}{5.481891in}}%
\pgfpathcurveto{\pgfqpoint{13.294154in}{5.481891in}}{\pgfqpoint{13.304753in}{5.486281in}}{\pgfqpoint{13.312567in}{5.494095in}}%
\pgfpathcurveto{\pgfqpoint{13.320381in}{5.501909in}}{\pgfqpoint{13.324771in}{5.512508in}}{\pgfqpoint{13.324771in}{5.523558in}}%
\pgfpathcurveto{\pgfqpoint{13.324771in}{5.534608in}}{\pgfqpoint{13.320381in}{5.545207in}}{\pgfqpoint{13.312567in}{5.553021in}}%
\pgfpathcurveto{\pgfqpoint{13.304753in}{5.560834in}}{\pgfqpoint{13.294154in}{5.565224in}}{\pgfqpoint{13.283104in}{5.565224in}}%
\pgfpathcurveto{\pgfqpoint{13.272054in}{5.565224in}}{\pgfqpoint{13.261455in}{5.560834in}}{\pgfqpoint{13.253641in}{5.553021in}}%
\pgfpathcurveto{\pgfqpoint{13.245828in}{5.545207in}}{\pgfqpoint{13.241438in}{5.534608in}}{\pgfqpoint{13.241438in}{5.523558in}}%
\pgfpathcurveto{\pgfqpoint{13.241438in}{5.512508in}}{\pgfqpoint{13.245828in}{5.501909in}}{\pgfqpoint{13.253641in}{5.494095in}}%
\pgfpathcurveto{\pgfqpoint{13.261455in}{5.486281in}}{\pgfqpoint{13.272054in}{5.481891in}}{\pgfqpoint{13.283104in}{5.481891in}}%
\pgfpathlineto{\pgfqpoint{13.283104in}{5.481891in}}%
\pgfpathclose%
\pgfusepath{stroke}%
\end{pgfscope}%
\begin{pgfscope}%
\pgfpathrectangle{\pgfqpoint{7.394209in}{0.375000in}}{\pgfqpoint{6.356833in}{5.175000in}}%
\pgfusepath{clip}%
\pgfsetbuttcap%
\pgfsetroundjoin%
\pgfsetlinewidth{1.003750pt}%
\definecolor{currentstroke}{rgb}{0.827451,0.827451,0.827451}%
\pgfsetstrokecolor{currentstroke}%
\pgfsetdash{}{0pt}%
\pgfpathmoveto{\pgfqpoint{13.634140in}{5.492791in}}%
\pgfpathcurveto{\pgfqpoint{13.645190in}{5.492791in}}{\pgfqpoint{13.655789in}{5.497181in}}{\pgfqpoint{13.663602in}{5.504995in}}%
\pgfpathcurveto{\pgfqpoint{13.671416in}{5.512808in}}{\pgfqpoint{13.675806in}{5.523408in}}{\pgfqpoint{13.675806in}{5.534458in}}%
\pgfpathcurveto{\pgfqpoint{13.675806in}{5.545508in}}{\pgfqpoint{13.671416in}{5.556107in}}{\pgfqpoint{13.663602in}{5.563920in}}%
\pgfpathcurveto{\pgfqpoint{13.655789in}{5.571734in}}{\pgfqpoint{13.645190in}{5.576124in}}{\pgfqpoint{13.634140in}{5.576124in}}%
\pgfpathcurveto{\pgfqpoint{13.623089in}{5.576124in}}{\pgfqpoint{13.612490in}{5.571734in}}{\pgfqpoint{13.604677in}{5.563920in}}%
\pgfpathcurveto{\pgfqpoint{13.596863in}{5.556107in}}{\pgfqpoint{13.592473in}{5.545508in}}{\pgfqpoint{13.592473in}{5.534458in}}%
\pgfpathcurveto{\pgfqpoint{13.592473in}{5.523408in}}{\pgfqpoint{13.596863in}{5.512808in}}{\pgfqpoint{13.604677in}{5.504995in}}%
\pgfpathcurveto{\pgfqpoint{13.612490in}{5.497181in}}{\pgfqpoint{13.623089in}{5.492791in}}{\pgfqpoint{13.634140in}{5.492791in}}%
\pgfpathlineto{\pgfqpoint{13.634140in}{5.492791in}}%
\pgfpathclose%
\pgfusepath{stroke}%
\end{pgfscope}%
\begin{pgfscope}%
\pgfpathrectangle{\pgfqpoint{7.394209in}{0.375000in}}{\pgfqpoint{6.356833in}{5.175000in}}%
\pgfusepath{clip}%
\pgfsetbuttcap%
\pgfsetroundjoin%
\pgfsetlinewidth{1.003750pt}%
\definecolor{currentstroke}{rgb}{0.827451,0.827451,0.827451}%
\pgfsetstrokecolor{currentstroke}%
\pgfsetdash{}{0pt}%
\pgfpathmoveto{\pgfqpoint{10.369910in}{5.474314in}}%
\pgfpathcurveto{\pgfqpoint{10.380960in}{5.474314in}}{\pgfqpoint{10.391559in}{5.478704in}}{\pgfqpoint{10.399373in}{5.486518in}}%
\pgfpathcurveto{\pgfqpoint{10.407186in}{5.494331in}}{\pgfqpoint{10.411577in}{5.504930in}}{\pgfqpoint{10.411577in}{5.515981in}}%
\pgfpathcurveto{\pgfqpoint{10.411577in}{5.527031in}}{\pgfqpoint{10.407186in}{5.537630in}}{\pgfqpoint{10.399373in}{5.545443in}}%
\pgfpathcurveto{\pgfqpoint{10.391559in}{5.553257in}}{\pgfqpoint{10.380960in}{5.557647in}}{\pgfqpoint{10.369910in}{5.557647in}}%
\pgfpathcurveto{\pgfqpoint{10.358860in}{5.557647in}}{\pgfqpoint{10.348261in}{5.553257in}}{\pgfqpoint{10.340447in}{5.545443in}}%
\pgfpathcurveto{\pgfqpoint{10.332634in}{5.537630in}}{\pgfqpoint{10.328243in}{5.527031in}}{\pgfqpoint{10.328243in}{5.515981in}}%
\pgfpathcurveto{\pgfqpoint{10.328243in}{5.504930in}}{\pgfqpoint{10.332634in}{5.494331in}}{\pgfqpoint{10.340447in}{5.486518in}}%
\pgfpathcurveto{\pgfqpoint{10.348261in}{5.478704in}}{\pgfqpoint{10.358860in}{5.474314in}}{\pgfqpoint{10.369910in}{5.474314in}}%
\pgfpathlineto{\pgfqpoint{10.369910in}{5.474314in}}%
\pgfpathclose%
\pgfusepath{stroke}%
\end{pgfscope}%
\begin{pgfscope}%
\pgfpathrectangle{\pgfqpoint{7.394209in}{0.375000in}}{\pgfqpoint{6.356833in}{5.175000in}}%
\pgfusepath{clip}%
\pgfsetbuttcap%
\pgfsetroundjoin%
\pgfsetlinewidth{1.003750pt}%
\definecolor{currentstroke}{rgb}{0.827451,0.827451,0.827451}%
\pgfsetstrokecolor{currentstroke}%
\pgfsetdash{}{0pt}%
\pgfpathmoveto{\pgfqpoint{11.646088in}{5.315163in}}%
\pgfpathcurveto{\pgfqpoint{11.657138in}{5.315163in}}{\pgfqpoint{11.667737in}{5.319553in}}{\pgfqpoint{11.675550in}{5.327367in}}%
\pgfpathcurveto{\pgfqpoint{11.683364in}{5.335180in}}{\pgfqpoint{11.687754in}{5.345779in}}{\pgfqpoint{11.687754in}{5.356829in}}%
\pgfpathcurveto{\pgfqpoint{11.687754in}{5.367879in}}{\pgfqpoint{11.683364in}{5.378478in}}{\pgfqpoint{11.675550in}{5.386292in}}%
\pgfpathcurveto{\pgfqpoint{11.667737in}{5.394106in}}{\pgfqpoint{11.657138in}{5.398496in}}{\pgfqpoint{11.646088in}{5.398496in}}%
\pgfpathcurveto{\pgfqpoint{11.635038in}{5.398496in}}{\pgfqpoint{11.624439in}{5.394106in}}{\pgfqpoint{11.616625in}{5.386292in}}%
\pgfpathcurveto{\pgfqpoint{11.608811in}{5.378478in}}{\pgfqpoint{11.604421in}{5.367879in}}{\pgfqpoint{11.604421in}{5.356829in}}%
\pgfpathcurveto{\pgfqpoint{11.604421in}{5.345779in}}{\pgfqpoint{11.608811in}{5.335180in}}{\pgfqpoint{11.616625in}{5.327367in}}%
\pgfpathcurveto{\pgfqpoint{11.624439in}{5.319553in}}{\pgfqpoint{11.635038in}{5.315163in}}{\pgfqpoint{11.646088in}{5.315163in}}%
\pgfpathlineto{\pgfqpoint{11.646088in}{5.315163in}}%
\pgfpathclose%
\pgfusepath{stroke}%
\end{pgfscope}%
\begin{pgfscope}%
\pgfpathrectangle{\pgfqpoint{7.394209in}{0.375000in}}{\pgfqpoint{6.356833in}{5.175000in}}%
\pgfusepath{clip}%
\pgfsetbuttcap%
\pgfsetroundjoin%
\pgfsetlinewidth{1.003750pt}%
\definecolor{currentstroke}{rgb}{0.827451,0.827451,0.827451}%
\pgfsetstrokecolor{currentstroke}%
\pgfsetdash{}{0pt}%
\pgfpathmoveto{\pgfqpoint{13.090198in}{5.465016in}}%
\pgfpathcurveto{\pgfqpoint{13.101248in}{5.465016in}}{\pgfqpoint{13.111847in}{5.469407in}}{\pgfqpoint{13.119660in}{5.477220in}}%
\pgfpathcurveto{\pgfqpoint{13.127474in}{5.485034in}}{\pgfqpoint{13.131864in}{5.495633in}}{\pgfqpoint{13.131864in}{5.506683in}}%
\pgfpathcurveto{\pgfqpoint{13.131864in}{5.517733in}}{\pgfqpoint{13.127474in}{5.528332in}}{\pgfqpoint{13.119660in}{5.536146in}}%
\pgfpathcurveto{\pgfqpoint{13.111847in}{5.543960in}}{\pgfqpoint{13.101248in}{5.548350in}}{\pgfqpoint{13.090198in}{5.548350in}}%
\pgfpathcurveto{\pgfqpoint{13.079147in}{5.548350in}}{\pgfqpoint{13.068548in}{5.543960in}}{\pgfqpoint{13.060735in}{5.536146in}}%
\pgfpathcurveto{\pgfqpoint{13.052921in}{5.528332in}}{\pgfqpoint{13.048531in}{5.517733in}}{\pgfqpoint{13.048531in}{5.506683in}}%
\pgfpathcurveto{\pgfqpoint{13.048531in}{5.495633in}}{\pgfqpoint{13.052921in}{5.485034in}}{\pgfqpoint{13.060735in}{5.477220in}}%
\pgfpathcurveto{\pgfqpoint{13.068548in}{5.469407in}}{\pgfqpoint{13.079147in}{5.465016in}}{\pgfqpoint{13.090198in}{5.465016in}}%
\pgfpathlineto{\pgfqpoint{13.090198in}{5.465016in}}%
\pgfpathclose%
\pgfusepath{stroke}%
\end{pgfscope}%
\begin{pgfscope}%
\pgfpathrectangle{\pgfqpoint{7.394209in}{0.375000in}}{\pgfqpoint{6.356833in}{5.175000in}}%
\pgfusepath{clip}%
\pgfsetbuttcap%
\pgfsetroundjoin%
\pgfsetlinewidth{1.003750pt}%
\definecolor{currentstroke}{rgb}{0.827451,0.827451,0.827451}%
\pgfsetstrokecolor{currentstroke}%
\pgfsetdash{}{0pt}%
\pgfpathmoveto{\pgfqpoint{11.972272in}{5.282949in}}%
\pgfpathcurveto{\pgfqpoint{11.983322in}{5.282949in}}{\pgfqpoint{11.993921in}{5.287339in}}{\pgfqpoint{12.001734in}{5.295153in}}%
\pgfpathcurveto{\pgfqpoint{12.009548in}{5.302966in}}{\pgfqpoint{12.013938in}{5.313565in}}{\pgfqpoint{12.013938in}{5.324616in}}%
\pgfpathcurveto{\pgfqpoint{12.013938in}{5.335666in}}{\pgfqpoint{12.009548in}{5.346265in}}{\pgfqpoint{12.001734in}{5.354078in}}%
\pgfpathcurveto{\pgfqpoint{11.993921in}{5.361892in}}{\pgfqpoint{11.983322in}{5.366282in}}{\pgfqpoint{11.972272in}{5.366282in}}%
\pgfpathcurveto{\pgfqpoint{11.961222in}{5.366282in}}{\pgfqpoint{11.950623in}{5.361892in}}{\pgfqpoint{11.942809in}{5.354078in}}%
\pgfpathcurveto{\pgfqpoint{11.934995in}{5.346265in}}{\pgfqpoint{11.930605in}{5.335666in}}{\pgfqpoint{11.930605in}{5.324616in}}%
\pgfpathcurveto{\pgfqpoint{11.930605in}{5.313565in}}{\pgfqpoint{11.934995in}{5.302966in}}{\pgfqpoint{11.942809in}{5.295153in}}%
\pgfpathcurveto{\pgfqpoint{11.950623in}{5.287339in}}{\pgfqpoint{11.961222in}{5.282949in}}{\pgfqpoint{11.972272in}{5.282949in}}%
\pgfpathlineto{\pgfqpoint{11.972272in}{5.282949in}}%
\pgfpathclose%
\pgfusepath{stroke}%
\end{pgfscope}%
\begin{pgfscope}%
\pgfpathrectangle{\pgfqpoint{7.394209in}{0.375000in}}{\pgfqpoint{6.356833in}{5.175000in}}%
\pgfusepath{clip}%
\pgfsetbuttcap%
\pgfsetroundjoin%
\pgfsetlinewidth{1.003750pt}%
\definecolor{currentstroke}{rgb}{0.827451,0.827451,0.827451}%
\pgfsetstrokecolor{currentstroke}%
\pgfsetdash{}{0pt}%
\pgfpathmoveto{\pgfqpoint{13.523323in}{5.492791in}}%
\pgfpathcurveto{\pgfqpoint{13.534373in}{5.492791in}}{\pgfqpoint{13.544972in}{5.497181in}}{\pgfqpoint{13.552785in}{5.504995in}}%
\pgfpathcurveto{\pgfqpoint{13.560599in}{5.512808in}}{\pgfqpoint{13.564989in}{5.523408in}}{\pgfqpoint{13.564989in}{5.534458in}}%
\pgfpathcurveto{\pgfqpoint{13.564989in}{5.545508in}}{\pgfqpoint{13.560599in}{5.556107in}}{\pgfqpoint{13.552785in}{5.563920in}}%
\pgfpathcurveto{\pgfqpoint{13.544972in}{5.571734in}}{\pgfqpoint{13.534373in}{5.576124in}}{\pgfqpoint{13.523323in}{5.576124in}}%
\pgfpathcurveto{\pgfqpoint{13.512272in}{5.576124in}}{\pgfqpoint{13.501673in}{5.571734in}}{\pgfqpoint{13.493860in}{5.563920in}}%
\pgfpathcurveto{\pgfqpoint{13.486046in}{5.556107in}}{\pgfqpoint{13.481656in}{5.545508in}}{\pgfqpoint{13.481656in}{5.534458in}}%
\pgfpathcurveto{\pgfqpoint{13.481656in}{5.523408in}}{\pgfqpoint{13.486046in}{5.512808in}}{\pgfqpoint{13.493860in}{5.504995in}}%
\pgfpathcurveto{\pgfqpoint{13.501673in}{5.497181in}}{\pgfqpoint{13.512272in}{5.492791in}}{\pgfqpoint{13.523323in}{5.492791in}}%
\pgfpathlineto{\pgfqpoint{13.523323in}{5.492791in}}%
\pgfpathclose%
\pgfusepath{stroke}%
\end{pgfscope}%
\begin{pgfscope}%
\pgfpathrectangle{\pgfqpoint{7.394209in}{0.375000in}}{\pgfqpoint{6.356833in}{5.175000in}}%
\pgfusepath{clip}%
\pgfsetbuttcap%
\pgfsetroundjoin%
\pgfsetlinewidth{1.003750pt}%
\definecolor{currentstroke}{rgb}{0.827451,0.827451,0.827451}%
\pgfsetstrokecolor{currentstroke}%
\pgfsetdash{}{0pt}%
\pgfpathmoveto{\pgfqpoint{7.861570in}{0.705414in}}%
\pgfpathcurveto{\pgfqpoint{7.872621in}{0.705414in}}{\pgfqpoint{7.883220in}{0.709804in}}{\pgfqpoint{7.891033in}{0.717618in}}%
\pgfpathcurveto{\pgfqpoint{7.898847in}{0.725431in}}{\pgfqpoint{7.903237in}{0.736030in}}{\pgfqpoint{7.903237in}{0.747081in}}%
\pgfpathcurveto{\pgfqpoint{7.903237in}{0.758131in}}{\pgfqpoint{7.898847in}{0.768730in}}{\pgfqpoint{7.891033in}{0.776543in}}%
\pgfpathcurveto{\pgfqpoint{7.883220in}{0.784357in}}{\pgfqpoint{7.872621in}{0.788747in}}{\pgfqpoint{7.861570in}{0.788747in}}%
\pgfpathcurveto{\pgfqpoint{7.850520in}{0.788747in}}{\pgfqpoint{7.839921in}{0.784357in}}{\pgfqpoint{7.832108in}{0.776543in}}%
\pgfpathcurveto{\pgfqpoint{7.824294in}{0.768730in}}{\pgfqpoint{7.819904in}{0.758131in}}{\pgfqpoint{7.819904in}{0.747081in}}%
\pgfpathcurveto{\pgfqpoint{7.819904in}{0.736030in}}{\pgfqpoint{7.824294in}{0.725431in}}{\pgfqpoint{7.832108in}{0.717618in}}%
\pgfpathcurveto{\pgfqpoint{7.839921in}{0.709804in}}{\pgfqpoint{7.850520in}{0.705414in}}{\pgfqpoint{7.861570in}{0.705414in}}%
\pgfpathlineto{\pgfqpoint{7.861570in}{0.705414in}}%
\pgfpathclose%
\pgfusepath{stroke}%
\end{pgfscope}%
\begin{pgfscope}%
\pgfpathrectangle{\pgfqpoint{7.394209in}{0.375000in}}{\pgfqpoint{6.356833in}{5.175000in}}%
\pgfusepath{clip}%
\pgfsetbuttcap%
\pgfsetroundjoin%
\pgfsetlinewidth{1.003750pt}%
\definecolor{currentstroke}{rgb}{0.827451,0.827451,0.827451}%
\pgfsetstrokecolor{currentstroke}%
\pgfsetdash{}{0pt}%
\pgfpathmoveto{\pgfqpoint{7.630542in}{1.328808in}}%
\pgfpathcurveto{\pgfqpoint{7.641592in}{1.328808in}}{\pgfqpoint{7.652191in}{1.333198in}}{\pgfqpoint{7.660005in}{1.341012in}}%
\pgfpathcurveto{\pgfqpoint{7.667818in}{1.348826in}}{\pgfqpoint{7.672209in}{1.359425in}}{\pgfqpoint{7.672209in}{1.370475in}}%
\pgfpathcurveto{\pgfqpoint{7.672209in}{1.381525in}}{\pgfqpoint{7.667818in}{1.392124in}}{\pgfqpoint{7.660005in}{1.399937in}}%
\pgfpathcurveto{\pgfqpoint{7.652191in}{1.407751in}}{\pgfqpoint{7.641592in}{1.412141in}}{\pgfqpoint{7.630542in}{1.412141in}}%
\pgfpathcurveto{\pgfqpoint{7.619492in}{1.412141in}}{\pgfqpoint{7.608893in}{1.407751in}}{\pgfqpoint{7.601079in}{1.399937in}}%
\pgfpathcurveto{\pgfqpoint{7.593266in}{1.392124in}}{\pgfqpoint{7.588875in}{1.381525in}}{\pgfqpoint{7.588875in}{1.370475in}}%
\pgfpathcurveto{\pgfqpoint{7.588875in}{1.359425in}}{\pgfqpoint{7.593266in}{1.348826in}}{\pgfqpoint{7.601079in}{1.341012in}}%
\pgfpathcurveto{\pgfqpoint{7.608893in}{1.333198in}}{\pgfqpoint{7.619492in}{1.328808in}}{\pgfqpoint{7.630542in}{1.328808in}}%
\pgfpathlineto{\pgfqpoint{7.630542in}{1.328808in}}%
\pgfpathclose%
\pgfusepath{stroke}%
\end{pgfscope}%
\begin{pgfscope}%
\pgfpathrectangle{\pgfqpoint{7.394209in}{0.375000in}}{\pgfqpoint{6.356833in}{5.175000in}}%
\pgfusepath{clip}%
\pgfsetbuttcap%
\pgfsetroundjoin%
\pgfsetlinewidth{1.003750pt}%
\definecolor{currentstroke}{rgb}{0.827451,0.827451,0.827451}%
\pgfsetstrokecolor{currentstroke}%
\pgfsetdash{}{0pt}%
\pgfpathmoveto{\pgfqpoint{8.611382in}{3.332106in}}%
\pgfpathcurveto{\pgfqpoint{8.622432in}{3.332106in}}{\pgfqpoint{8.633031in}{3.336496in}}{\pgfqpoint{8.640845in}{3.344310in}}%
\pgfpathcurveto{\pgfqpoint{8.648658in}{3.352123in}}{\pgfqpoint{8.653049in}{3.362722in}}{\pgfqpoint{8.653049in}{3.373772in}}%
\pgfpathcurveto{\pgfqpoint{8.653049in}{3.384823in}}{\pgfqpoint{8.648658in}{3.395422in}}{\pgfqpoint{8.640845in}{3.403235in}}%
\pgfpathcurveto{\pgfqpoint{8.633031in}{3.411049in}}{\pgfqpoint{8.622432in}{3.415439in}}{\pgfqpoint{8.611382in}{3.415439in}}%
\pgfpathcurveto{\pgfqpoint{8.600332in}{3.415439in}}{\pgfqpoint{8.589733in}{3.411049in}}{\pgfqpoint{8.581919in}{3.403235in}}%
\pgfpathcurveto{\pgfqpoint{8.574106in}{3.395422in}}{\pgfqpoint{8.569715in}{3.384823in}}{\pgfqpoint{8.569715in}{3.373772in}}%
\pgfpathcurveto{\pgfqpoint{8.569715in}{3.362722in}}{\pgfqpoint{8.574106in}{3.352123in}}{\pgfqpoint{8.581919in}{3.344310in}}%
\pgfpathcurveto{\pgfqpoint{8.589733in}{3.336496in}}{\pgfqpoint{8.600332in}{3.332106in}}{\pgfqpoint{8.611382in}{3.332106in}}%
\pgfpathlineto{\pgfqpoint{8.611382in}{3.332106in}}%
\pgfpathclose%
\pgfusepath{stroke}%
\end{pgfscope}%
\begin{pgfscope}%
\pgfpathrectangle{\pgfqpoint{7.394209in}{0.375000in}}{\pgfqpoint{6.356833in}{5.175000in}}%
\pgfusepath{clip}%
\pgfsetbuttcap%
\pgfsetroundjoin%
\pgfsetlinewidth{1.003750pt}%
\definecolor{currentstroke}{rgb}{0.827451,0.827451,0.827451}%
\pgfsetstrokecolor{currentstroke}%
\pgfsetdash{}{0pt}%
\pgfpathmoveto{\pgfqpoint{10.331046in}{5.294524in}}%
\pgfpathcurveto{\pgfqpoint{10.342096in}{5.294524in}}{\pgfqpoint{10.352695in}{5.298915in}}{\pgfqpoint{10.360508in}{5.306728in}}%
\pgfpathcurveto{\pgfqpoint{10.368322in}{5.314542in}}{\pgfqpoint{10.372712in}{5.325141in}}{\pgfqpoint{10.372712in}{5.336191in}}%
\pgfpathcurveto{\pgfqpoint{10.372712in}{5.347241in}}{\pgfqpoint{10.368322in}{5.357840in}}{\pgfqpoint{10.360508in}{5.365654in}}%
\pgfpathcurveto{\pgfqpoint{10.352695in}{5.373467in}}{\pgfqpoint{10.342096in}{5.377858in}}{\pgfqpoint{10.331046in}{5.377858in}}%
\pgfpathcurveto{\pgfqpoint{10.319995in}{5.377858in}}{\pgfqpoint{10.309396in}{5.373467in}}{\pgfqpoint{10.301583in}{5.365654in}}%
\pgfpathcurveto{\pgfqpoint{10.293769in}{5.357840in}}{\pgfqpoint{10.289379in}{5.347241in}}{\pgfqpoint{10.289379in}{5.336191in}}%
\pgfpathcurveto{\pgfqpoint{10.289379in}{5.325141in}}{\pgfqpoint{10.293769in}{5.314542in}}{\pgfqpoint{10.301583in}{5.306728in}}%
\pgfpathcurveto{\pgfqpoint{10.309396in}{5.298915in}}{\pgfqpoint{10.319995in}{5.294524in}}{\pgfqpoint{10.331046in}{5.294524in}}%
\pgfpathlineto{\pgfqpoint{10.331046in}{5.294524in}}%
\pgfpathclose%
\pgfusepath{stroke}%
\end{pgfscope}%
\begin{pgfscope}%
\pgfpathrectangle{\pgfqpoint{7.394209in}{0.375000in}}{\pgfqpoint{6.356833in}{5.175000in}}%
\pgfusepath{clip}%
\pgfsetbuttcap%
\pgfsetroundjoin%
\pgfsetlinewidth{1.003750pt}%
\definecolor{currentstroke}{rgb}{0.827451,0.827451,0.827451}%
\pgfsetstrokecolor{currentstroke}%
\pgfsetdash{}{0pt}%
\pgfpathmoveto{\pgfqpoint{13.580741in}{5.501682in}}%
\pgfpathcurveto{\pgfqpoint{13.591791in}{5.501682in}}{\pgfqpoint{13.602390in}{5.506072in}}{\pgfqpoint{13.610204in}{5.513885in}}%
\pgfpathcurveto{\pgfqpoint{13.618018in}{5.521699in}}{\pgfqpoint{13.622408in}{5.532298in}}{\pgfqpoint{13.622408in}{5.543348in}}%
\pgfpathcurveto{\pgfqpoint{13.622408in}{5.554398in}}{\pgfqpoint{13.618018in}{5.564997in}}{\pgfqpoint{13.610204in}{5.572811in}}%
\pgfpathcurveto{\pgfqpoint{13.602390in}{5.580625in}}{\pgfqpoint{13.591791in}{5.585015in}}{\pgfqpoint{13.580741in}{5.585015in}}%
\pgfpathcurveto{\pgfqpoint{13.569691in}{5.585015in}}{\pgfqpoint{13.559092in}{5.580625in}}{\pgfqpoint{13.551278in}{5.572811in}}%
\pgfpathcurveto{\pgfqpoint{13.543465in}{5.564997in}}{\pgfqpoint{13.539075in}{5.554398in}}{\pgfqpoint{13.539075in}{5.543348in}}%
\pgfpathcurveto{\pgfqpoint{13.539075in}{5.532298in}}{\pgfqpoint{13.543465in}{5.521699in}}{\pgfqpoint{13.551278in}{5.513885in}}%
\pgfpathcurveto{\pgfqpoint{13.559092in}{5.506072in}}{\pgfqpoint{13.569691in}{5.501682in}}{\pgfqpoint{13.580741in}{5.501682in}}%
\pgfpathlineto{\pgfqpoint{13.580741in}{5.501682in}}%
\pgfpathclose%
\pgfusepath{stroke}%
\end{pgfscope}%
\begin{pgfscope}%
\pgfpathrectangle{\pgfqpoint{7.394209in}{0.375000in}}{\pgfqpoint{6.356833in}{5.175000in}}%
\pgfusepath{clip}%
\pgfsetbuttcap%
\pgfsetroundjoin%
\pgfsetlinewidth{1.003750pt}%
\definecolor{currentstroke}{rgb}{0.827451,0.827451,0.827451}%
\pgfsetstrokecolor{currentstroke}%
\pgfsetdash{}{0pt}%
\pgfpathmoveto{\pgfqpoint{9.767887in}{4.260621in}}%
\pgfpathcurveto{\pgfqpoint{9.778937in}{4.260621in}}{\pgfqpoint{9.789536in}{4.265011in}}{\pgfqpoint{9.797350in}{4.272825in}}%
\pgfpathcurveto{\pgfqpoint{9.805163in}{4.280639in}}{\pgfqpoint{9.809553in}{4.291238in}}{\pgfqpoint{9.809553in}{4.302288in}}%
\pgfpathcurveto{\pgfqpoint{9.809553in}{4.313338in}}{\pgfqpoint{9.805163in}{4.323937in}}{\pgfqpoint{9.797350in}{4.331751in}}%
\pgfpathcurveto{\pgfqpoint{9.789536in}{4.339564in}}{\pgfqpoint{9.778937in}{4.343954in}}{\pgfqpoint{9.767887in}{4.343954in}}%
\pgfpathcurveto{\pgfqpoint{9.756837in}{4.343954in}}{\pgfqpoint{9.746238in}{4.339564in}}{\pgfqpoint{9.738424in}{4.331751in}}%
\pgfpathcurveto{\pgfqpoint{9.730610in}{4.323937in}}{\pgfqpoint{9.726220in}{4.313338in}}{\pgfqpoint{9.726220in}{4.302288in}}%
\pgfpathcurveto{\pgfqpoint{9.726220in}{4.291238in}}{\pgfqpoint{9.730610in}{4.280639in}}{\pgfqpoint{9.738424in}{4.272825in}}%
\pgfpathcurveto{\pgfqpoint{9.746238in}{4.265011in}}{\pgfqpoint{9.756837in}{4.260621in}}{\pgfqpoint{9.767887in}{4.260621in}}%
\pgfpathlineto{\pgfqpoint{9.767887in}{4.260621in}}%
\pgfpathclose%
\pgfusepath{stroke}%
\end{pgfscope}%
\begin{pgfscope}%
\pgfpathrectangle{\pgfqpoint{7.394209in}{0.375000in}}{\pgfqpoint{6.356833in}{5.175000in}}%
\pgfusepath{clip}%
\pgfsetbuttcap%
\pgfsetroundjoin%
\pgfsetlinewidth{1.003750pt}%
\definecolor{currentstroke}{rgb}{0.827451,0.827451,0.827451}%
\pgfsetstrokecolor{currentstroke}%
\pgfsetdash{}{0pt}%
\pgfpathmoveto{\pgfqpoint{9.101799in}{3.460532in}}%
\pgfpathcurveto{\pgfqpoint{9.112849in}{3.460532in}}{\pgfqpoint{9.123448in}{3.464923in}}{\pgfqpoint{9.131261in}{3.472736in}}%
\pgfpathcurveto{\pgfqpoint{9.139075in}{3.480550in}}{\pgfqpoint{9.143465in}{3.491149in}}{\pgfqpoint{9.143465in}{3.502199in}}%
\pgfpathcurveto{\pgfqpoint{9.143465in}{3.513249in}}{\pgfqpoint{9.139075in}{3.523848in}}{\pgfqpoint{9.131261in}{3.531662in}}%
\pgfpathcurveto{\pgfqpoint{9.123448in}{3.539475in}}{\pgfqpoint{9.112849in}{3.543866in}}{\pgfqpoint{9.101799in}{3.543866in}}%
\pgfpathcurveto{\pgfqpoint{9.090749in}{3.543866in}}{\pgfqpoint{9.080149in}{3.539475in}}{\pgfqpoint{9.072336in}{3.531662in}}%
\pgfpathcurveto{\pgfqpoint{9.064522in}{3.523848in}}{\pgfqpoint{9.060132in}{3.513249in}}{\pgfqpoint{9.060132in}{3.502199in}}%
\pgfpathcurveto{\pgfqpoint{9.060132in}{3.491149in}}{\pgfqpoint{9.064522in}{3.480550in}}{\pgfqpoint{9.072336in}{3.472736in}}%
\pgfpathcurveto{\pgfqpoint{9.080149in}{3.464923in}}{\pgfqpoint{9.090749in}{3.460532in}}{\pgfqpoint{9.101799in}{3.460532in}}%
\pgfpathlineto{\pgfqpoint{9.101799in}{3.460532in}}%
\pgfpathclose%
\pgfusepath{stroke}%
\end{pgfscope}%
\begin{pgfscope}%
\pgfpathrectangle{\pgfqpoint{7.394209in}{0.375000in}}{\pgfqpoint{6.356833in}{5.175000in}}%
\pgfusepath{clip}%
\pgfsetbuttcap%
\pgfsetroundjoin%
\pgfsetlinewidth{1.003750pt}%
\definecolor{currentstroke}{rgb}{0.827451,0.827451,0.827451}%
\pgfsetstrokecolor{currentstroke}%
\pgfsetdash{}{0pt}%
\pgfpathmoveto{\pgfqpoint{8.715670in}{1.718756in}}%
\pgfpathcurveto{\pgfqpoint{8.726721in}{1.718756in}}{\pgfqpoint{8.737320in}{1.723146in}}{\pgfqpoint{8.745133in}{1.730960in}}%
\pgfpathcurveto{\pgfqpoint{8.752947in}{1.738773in}}{\pgfqpoint{8.757337in}{1.749372in}}{\pgfqpoint{8.757337in}{1.760422in}}%
\pgfpathcurveto{\pgfqpoint{8.757337in}{1.771472in}}{\pgfqpoint{8.752947in}{1.782071in}}{\pgfqpoint{8.745133in}{1.789885in}}%
\pgfpathcurveto{\pgfqpoint{8.737320in}{1.797699in}}{\pgfqpoint{8.726721in}{1.802089in}}{\pgfqpoint{8.715670in}{1.802089in}}%
\pgfpathcurveto{\pgfqpoint{8.704620in}{1.802089in}}{\pgfqpoint{8.694021in}{1.797699in}}{\pgfqpoint{8.686208in}{1.789885in}}%
\pgfpathcurveto{\pgfqpoint{8.678394in}{1.782071in}}{\pgfqpoint{8.674004in}{1.771472in}}{\pgfqpoint{8.674004in}{1.760422in}}%
\pgfpathcurveto{\pgfqpoint{8.674004in}{1.749372in}}{\pgfqpoint{8.678394in}{1.738773in}}{\pgfqpoint{8.686208in}{1.730960in}}%
\pgfpathcurveto{\pgfqpoint{8.694021in}{1.723146in}}{\pgfqpoint{8.704620in}{1.718756in}}{\pgfqpoint{8.715670in}{1.718756in}}%
\pgfpathlineto{\pgfqpoint{8.715670in}{1.718756in}}%
\pgfpathclose%
\pgfusepath{stroke}%
\end{pgfscope}%
\begin{pgfscope}%
\pgfpathrectangle{\pgfqpoint{7.394209in}{0.375000in}}{\pgfqpoint{6.356833in}{5.175000in}}%
\pgfusepath{clip}%
\pgfsetbuttcap%
\pgfsetroundjoin%
\pgfsetlinewidth{1.003750pt}%
\definecolor{currentstroke}{rgb}{0.827451,0.827451,0.827451}%
\pgfsetstrokecolor{currentstroke}%
\pgfsetdash{}{0pt}%
\pgfpathmoveto{\pgfqpoint{10.140111in}{5.108207in}}%
\pgfpathcurveto{\pgfqpoint{10.151161in}{5.108207in}}{\pgfqpoint{10.161760in}{5.112598in}}{\pgfqpoint{10.169573in}{5.120411in}}%
\pgfpathcurveto{\pgfqpoint{10.177387in}{5.128225in}}{\pgfqpoint{10.181777in}{5.138824in}}{\pgfqpoint{10.181777in}{5.149874in}}%
\pgfpathcurveto{\pgfqpoint{10.181777in}{5.160924in}}{\pgfqpoint{10.177387in}{5.171523in}}{\pgfqpoint{10.169573in}{5.179337in}}%
\pgfpathcurveto{\pgfqpoint{10.161760in}{5.187151in}}{\pgfqpoint{10.151161in}{5.191541in}}{\pgfqpoint{10.140111in}{5.191541in}}%
\pgfpathcurveto{\pgfqpoint{10.129061in}{5.191541in}}{\pgfqpoint{10.118462in}{5.187151in}}{\pgfqpoint{10.110648in}{5.179337in}}%
\pgfpathcurveto{\pgfqpoint{10.102834in}{5.171523in}}{\pgfqpoint{10.098444in}{5.160924in}}{\pgfqpoint{10.098444in}{5.149874in}}%
\pgfpathcurveto{\pgfqpoint{10.098444in}{5.138824in}}{\pgfqpoint{10.102834in}{5.128225in}}{\pgfqpoint{10.110648in}{5.120411in}}%
\pgfpathcurveto{\pgfqpoint{10.118462in}{5.112598in}}{\pgfqpoint{10.129061in}{5.108207in}}{\pgfqpoint{10.140111in}{5.108207in}}%
\pgfpathlineto{\pgfqpoint{10.140111in}{5.108207in}}%
\pgfpathclose%
\pgfusepath{stroke}%
\end{pgfscope}%
\begin{pgfscope}%
\pgfpathrectangle{\pgfqpoint{7.394209in}{0.375000in}}{\pgfqpoint{6.356833in}{5.175000in}}%
\pgfusepath{clip}%
\pgfsetbuttcap%
\pgfsetroundjoin%
\pgfsetlinewidth{1.003750pt}%
\definecolor{currentstroke}{rgb}{0.827451,0.827451,0.827451}%
\pgfsetstrokecolor{currentstroke}%
\pgfsetdash{}{0pt}%
\pgfpathmoveto{\pgfqpoint{7.720296in}{0.431929in}}%
\pgfpathcurveto{\pgfqpoint{7.731346in}{0.431929in}}{\pgfqpoint{7.741945in}{0.436319in}}{\pgfqpoint{7.749759in}{0.444133in}}%
\pgfpathcurveto{\pgfqpoint{7.757572in}{0.451946in}}{\pgfqpoint{7.761962in}{0.462545in}}{\pgfqpoint{7.761962in}{0.473595in}}%
\pgfpathcurveto{\pgfqpoint{7.761962in}{0.484645in}}{\pgfqpoint{7.757572in}{0.495244in}}{\pgfqpoint{7.749759in}{0.503058in}}%
\pgfpathcurveto{\pgfqpoint{7.741945in}{0.510872in}}{\pgfqpoint{7.731346in}{0.515262in}}{\pgfqpoint{7.720296in}{0.515262in}}%
\pgfpathcurveto{\pgfqpoint{7.709246in}{0.515262in}}{\pgfqpoint{7.698647in}{0.510872in}}{\pgfqpoint{7.690833in}{0.503058in}}%
\pgfpathcurveto{\pgfqpoint{7.683019in}{0.495244in}}{\pgfqpoint{7.678629in}{0.484645in}}{\pgfqpoint{7.678629in}{0.473595in}}%
\pgfpathcurveto{\pgfqpoint{7.678629in}{0.462545in}}{\pgfqpoint{7.683019in}{0.451946in}}{\pgfqpoint{7.690833in}{0.444133in}}%
\pgfpathcurveto{\pgfqpoint{7.698647in}{0.436319in}}{\pgfqpoint{7.709246in}{0.431929in}}{\pgfqpoint{7.720296in}{0.431929in}}%
\pgfpathlineto{\pgfqpoint{7.720296in}{0.431929in}}%
\pgfpathclose%
\pgfusepath{stroke}%
\end{pgfscope}%
\begin{pgfscope}%
\pgfpathrectangle{\pgfqpoint{7.394209in}{0.375000in}}{\pgfqpoint{6.356833in}{5.175000in}}%
\pgfusepath{clip}%
\pgfsetbuttcap%
\pgfsetroundjoin%
\pgfsetlinewidth{1.003750pt}%
\definecolor{currentstroke}{rgb}{0.827451,0.827451,0.827451}%
\pgfsetstrokecolor{currentstroke}%
\pgfsetdash{}{0pt}%
\pgfpathmoveto{\pgfqpoint{8.963594in}{3.559615in}}%
\pgfpathcurveto{\pgfqpoint{8.974644in}{3.559615in}}{\pgfqpoint{8.985243in}{3.564005in}}{\pgfqpoint{8.993057in}{3.571819in}}%
\pgfpathcurveto{\pgfqpoint{9.000870in}{3.579632in}}{\pgfqpoint{9.005261in}{3.590232in}}{\pgfqpoint{9.005261in}{3.601282in}}%
\pgfpathcurveto{\pgfqpoint{9.005261in}{3.612332in}}{\pgfqpoint{9.000870in}{3.622931in}}{\pgfqpoint{8.993057in}{3.630744in}}%
\pgfpathcurveto{\pgfqpoint{8.985243in}{3.638558in}}{\pgfqpoint{8.974644in}{3.642948in}}{\pgfqpoint{8.963594in}{3.642948in}}%
\pgfpathcurveto{\pgfqpoint{8.952544in}{3.642948in}}{\pgfqpoint{8.941945in}{3.638558in}}{\pgfqpoint{8.934131in}{3.630744in}}%
\pgfpathcurveto{\pgfqpoint{8.926318in}{3.622931in}}{\pgfqpoint{8.921927in}{3.612332in}}{\pgfqpoint{8.921927in}{3.601282in}}%
\pgfpathcurveto{\pgfqpoint{8.921927in}{3.590232in}}{\pgfqpoint{8.926318in}{3.579632in}}{\pgfqpoint{8.934131in}{3.571819in}}%
\pgfpathcurveto{\pgfqpoint{8.941945in}{3.564005in}}{\pgfqpoint{8.952544in}{3.559615in}}{\pgfqpoint{8.963594in}{3.559615in}}%
\pgfpathlineto{\pgfqpoint{8.963594in}{3.559615in}}%
\pgfpathclose%
\pgfusepath{stroke}%
\end{pgfscope}%
\begin{pgfscope}%
\pgfpathrectangle{\pgfqpoint{7.394209in}{0.375000in}}{\pgfqpoint{6.356833in}{5.175000in}}%
\pgfusepath{clip}%
\pgfsetbuttcap%
\pgfsetroundjoin%
\pgfsetlinewidth{1.003750pt}%
\definecolor{currentstroke}{rgb}{0.827451,0.827451,0.827451}%
\pgfsetstrokecolor{currentstroke}%
\pgfsetdash{}{0pt}%
\pgfpathmoveto{\pgfqpoint{13.510179in}{5.498734in}}%
\pgfpathcurveto{\pgfqpoint{13.521229in}{5.498734in}}{\pgfqpoint{13.531828in}{5.503125in}}{\pgfqpoint{13.539641in}{5.510938in}}%
\pgfpathcurveto{\pgfqpoint{13.547455in}{5.518752in}}{\pgfqpoint{13.551845in}{5.529351in}}{\pgfqpoint{13.551845in}{5.540401in}}%
\pgfpathcurveto{\pgfqpoint{13.551845in}{5.551451in}}{\pgfqpoint{13.547455in}{5.562050in}}{\pgfqpoint{13.539641in}{5.569864in}}%
\pgfpathcurveto{\pgfqpoint{13.531828in}{5.577677in}}{\pgfqpoint{13.521229in}{5.582068in}}{\pgfqpoint{13.510179in}{5.582068in}}%
\pgfpathcurveto{\pgfqpoint{13.499129in}{5.582068in}}{\pgfqpoint{13.488530in}{5.577677in}}{\pgfqpoint{13.480716in}{5.569864in}}%
\pgfpathcurveto{\pgfqpoint{13.472902in}{5.562050in}}{\pgfqpoint{13.468512in}{5.551451in}}{\pgfqpoint{13.468512in}{5.540401in}}%
\pgfpathcurveto{\pgfqpoint{13.468512in}{5.529351in}}{\pgfqpoint{13.472902in}{5.518752in}}{\pgfqpoint{13.480716in}{5.510938in}}%
\pgfpathcurveto{\pgfqpoint{13.488530in}{5.503125in}}{\pgfqpoint{13.499129in}{5.498734in}}{\pgfqpoint{13.510179in}{5.498734in}}%
\pgfpathlineto{\pgfqpoint{13.510179in}{5.498734in}}%
\pgfpathclose%
\pgfusepath{stroke}%
\end{pgfscope}%
\begin{pgfscope}%
\pgfpathrectangle{\pgfqpoint{7.394209in}{0.375000in}}{\pgfqpoint{6.356833in}{5.175000in}}%
\pgfusepath{clip}%
\pgfsetbuttcap%
\pgfsetroundjoin%
\pgfsetlinewidth{1.003750pt}%
\definecolor{currentstroke}{rgb}{0.827451,0.827451,0.827451}%
\pgfsetstrokecolor{currentstroke}%
\pgfsetdash{}{0pt}%
\pgfpathmoveto{\pgfqpoint{10.279981in}{5.505277in}}%
\pgfpathcurveto{\pgfqpoint{10.291031in}{5.505277in}}{\pgfqpoint{10.301630in}{5.509667in}}{\pgfqpoint{10.309444in}{5.517481in}}%
\pgfpathcurveto{\pgfqpoint{10.317257in}{5.525295in}}{\pgfqpoint{10.321648in}{5.535894in}}{\pgfqpoint{10.321648in}{5.546944in}}%
\pgfpathcurveto{\pgfqpoint{10.321648in}{5.557994in}}{\pgfqpoint{10.317257in}{5.568593in}}{\pgfqpoint{10.309444in}{5.576407in}}%
\pgfpathcurveto{\pgfqpoint{10.301630in}{5.584220in}}{\pgfqpoint{10.291031in}{5.588610in}}{\pgfqpoint{10.279981in}{5.588610in}}%
\pgfpathcurveto{\pgfqpoint{10.268931in}{5.588610in}}{\pgfqpoint{10.258332in}{5.584220in}}{\pgfqpoint{10.250518in}{5.576407in}}%
\pgfpathcurveto{\pgfqpoint{10.242705in}{5.568593in}}{\pgfqpoint{10.238314in}{5.557994in}}{\pgfqpoint{10.238314in}{5.546944in}}%
\pgfpathcurveto{\pgfqpoint{10.238314in}{5.535894in}}{\pgfqpoint{10.242705in}{5.525295in}}{\pgfqpoint{10.250518in}{5.517481in}}%
\pgfpathcurveto{\pgfqpoint{10.258332in}{5.509667in}}{\pgfqpoint{10.268931in}{5.505277in}}{\pgfqpoint{10.279981in}{5.505277in}}%
\pgfpathlineto{\pgfqpoint{10.279981in}{5.505277in}}%
\pgfpathclose%
\pgfusepath{stroke}%
\end{pgfscope}%
\begin{pgfscope}%
\pgfpathrectangle{\pgfqpoint{7.394209in}{0.375000in}}{\pgfqpoint{6.356833in}{5.175000in}}%
\pgfusepath{clip}%
\pgfsetbuttcap%
\pgfsetroundjoin%
\pgfsetlinewidth{1.003750pt}%
\definecolor{currentstroke}{rgb}{0.827451,0.827451,0.827451}%
\pgfsetstrokecolor{currentstroke}%
\pgfsetdash{}{0pt}%
\pgfpathmoveto{\pgfqpoint{8.657131in}{3.181864in}}%
\pgfpathcurveto{\pgfqpoint{8.668181in}{3.181864in}}{\pgfqpoint{8.678780in}{3.186254in}}{\pgfqpoint{8.686594in}{3.194068in}}%
\pgfpathcurveto{\pgfqpoint{8.694407in}{3.201882in}}{\pgfqpoint{8.698798in}{3.212481in}}{\pgfqpoint{8.698798in}{3.223531in}}%
\pgfpathcurveto{\pgfqpoint{8.698798in}{3.234581in}}{\pgfqpoint{8.694407in}{3.245180in}}{\pgfqpoint{8.686594in}{3.252994in}}%
\pgfpathcurveto{\pgfqpoint{8.678780in}{3.260807in}}{\pgfqpoint{8.668181in}{3.265197in}}{\pgfqpoint{8.657131in}{3.265197in}}%
\pgfpathcurveto{\pgfqpoint{8.646081in}{3.265197in}}{\pgfqpoint{8.635482in}{3.260807in}}{\pgfqpoint{8.627668in}{3.252994in}}%
\pgfpathcurveto{\pgfqpoint{8.619855in}{3.245180in}}{\pgfqpoint{8.615464in}{3.234581in}}{\pgfqpoint{8.615464in}{3.223531in}}%
\pgfpathcurveto{\pgfqpoint{8.615464in}{3.212481in}}{\pgfqpoint{8.619855in}{3.201882in}}{\pgfqpoint{8.627668in}{3.194068in}}%
\pgfpathcurveto{\pgfqpoint{8.635482in}{3.186254in}}{\pgfqpoint{8.646081in}{3.181864in}}{\pgfqpoint{8.657131in}{3.181864in}}%
\pgfpathlineto{\pgfqpoint{8.657131in}{3.181864in}}%
\pgfpathclose%
\pgfusepath{stroke}%
\end{pgfscope}%
\begin{pgfscope}%
\pgfpathrectangle{\pgfqpoint{7.394209in}{0.375000in}}{\pgfqpoint{6.356833in}{5.175000in}}%
\pgfusepath{clip}%
\pgfsetbuttcap%
\pgfsetroundjoin%
\pgfsetlinewidth{1.003750pt}%
\definecolor{currentstroke}{rgb}{0.827451,0.827451,0.827451}%
\pgfsetstrokecolor{currentstroke}%
\pgfsetdash{}{0pt}%
\pgfpathmoveto{\pgfqpoint{9.049744in}{3.473606in}}%
\pgfpathcurveto{\pgfqpoint{9.060794in}{3.473606in}}{\pgfqpoint{9.071393in}{3.477997in}}{\pgfqpoint{9.079206in}{3.485810in}}%
\pgfpathcurveto{\pgfqpoint{9.087020in}{3.493624in}}{\pgfqpoint{9.091410in}{3.504223in}}{\pgfqpoint{9.091410in}{3.515273in}}%
\pgfpathcurveto{\pgfqpoint{9.091410in}{3.526323in}}{\pgfqpoint{9.087020in}{3.536922in}}{\pgfqpoint{9.079206in}{3.544736in}}%
\pgfpathcurveto{\pgfqpoint{9.071393in}{3.552550in}}{\pgfqpoint{9.060794in}{3.556940in}}{\pgfqpoint{9.049744in}{3.556940in}}%
\pgfpathcurveto{\pgfqpoint{9.038693in}{3.556940in}}{\pgfqpoint{9.028094in}{3.552550in}}{\pgfqpoint{9.020281in}{3.544736in}}%
\pgfpathcurveto{\pgfqpoint{9.012467in}{3.536922in}}{\pgfqpoint{9.008077in}{3.526323in}}{\pgfqpoint{9.008077in}{3.515273in}}%
\pgfpathcurveto{\pgfqpoint{9.008077in}{3.504223in}}{\pgfqpoint{9.012467in}{3.493624in}}{\pgfqpoint{9.020281in}{3.485810in}}%
\pgfpathcurveto{\pgfqpoint{9.028094in}{3.477997in}}{\pgfqpoint{9.038693in}{3.473606in}}{\pgfqpoint{9.049744in}{3.473606in}}%
\pgfpathlineto{\pgfqpoint{9.049744in}{3.473606in}}%
\pgfpathclose%
\pgfusepath{stroke}%
\end{pgfscope}%
\begin{pgfscope}%
\pgfpathrectangle{\pgfqpoint{7.394209in}{0.375000in}}{\pgfqpoint{6.356833in}{5.175000in}}%
\pgfusepath{clip}%
\pgfsetbuttcap%
\pgfsetroundjoin%
\pgfsetlinewidth{1.003750pt}%
\definecolor{currentstroke}{rgb}{0.827451,0.827451,0.827451}%
\pgfsetstrokecolor{currentstroke}%
\pgfsetdash{}{0pt}%
\pgfpathmoveto{\pgfqpoint{9.672088in}{4.614975in}}%
\pgfpathcurveto{\pgfqpoint{9.683138in}{4.614975in}}{\pgfqpoint{9.693737in}{4.619365in}}{\pgfqpoint{9.701551in}{4.627179in}}%
\pgfpathcurveto{\pgfqpoint{9.709364in}{4.634992in}}{\pgfqpoint{9.713755in}{4.645591in}}{\pgfqpoint{9.713755in}{4.656642in}}%
\pgfpathcurveto{\pgfqpoint{9.713755in}{4.667692in}}{\pgfqpoint{9.709364in}{4.678291in}}{\pgfqpoint{9.701551in}{4.686104in}}%
\pgfpathcurveto{\pgfqpoint{9.693737in}{4.693918in}}{\pgfqpoint{9.683138in}{4.698308in}}{\pgfqpoint{9.672088in}{4.698308in}}%
\pgfpathcurveto{\pgfqpoint{9.661038in}{4.698308in}}{\pgfqpoint{9.650439in}{4.693918in}}{\pgfqpoint{9.642625in}{4.686104in}}%
\pgfpathcurveto{\pgfqpoint{9.634811in}{4.678291in}}{\pgfqpoint{9.630421in}{4.667692in}}{\pgfqpoint{9.630421in}{4.656642in}}%
\pgfpathcurveto{\pgfqpoint{9.630421in}{4.645591in}}{\pgfqpoint{9.634811in}{4.634992in}}{\pgfqpoint{9.642625in}{4.627179in}}%
\pgfpathcurveto{\pgfqpoint{9.650439in}{4.619365in}}{\pgfqpoint{9.661038in}{4.614975in}}{\pgfqpoint{9.672088in}{4.614975in}}%
\pgfpathlineto{\pgfqpoint{9.672088in}{4.614975in}}%
\pgfpathclose%
\pgfusepath{stroke}%
\end{pgfscope}%
\begin{pgfscope}%
\pgfpathrectangle{\pgfqpoint{7.394209in}{0.375000in}}{\pgfqpoint{6.356833in}{5.175000in}}%
\pgfusepath{clip}%
\pgfsetbuttcap%
\pgfsetroundjoin%
\pgfsetlinewidth{1.003750pt}%
\definecolor{currentstroke}{rgb}{0.827451,0.827451,0.827451}%
\pgfsetstrokecolor{currentstroke}%
\pgfsetdash{}{0pt}%
\pgfpathmoveto{\pgfqpoint{9.780777in}{4.933591in}}%
\pgfpathcurveto{\pgfqpoint{9.791827in}{4.933591in}}{\pgfqpoint{9.802426in}{4.937981in}}{\pgfqpoint{9.810240in}{4.945795in}}%
\pgfpathcurveto{\pgfqpoint{9.818054in}{4.953609in}}{\pgfqpoint{9.822444in}{4.964208in}}{\pgfqpoint{9.822444in}{4.975258in}}%
\pgfpathcurveto{\pgfqpoint{9.822444in}{4.986308in}}{\pgfqpoint{9.818054in}{4.996907in}}{\pgfqpoint{9.810240in}{5.004721in}}%
\pgfpathcurveto{\pgfqpoint{9.802426in}{5.012534in}}{\pgfqpoint{9.791827in}{5.016924in}}{\pgfqpoint{9.780777in}{5.016924in}}%
\pgfpathcurveto{\pgfqpoint{9.769727in}{5.016924in}}{\pgfqpoint{9.759128in}{5.012534in}}{\pgfqpoint{9.751314in}{5.004721in}}%
\pgfpathcurveto{\pgfqpoint{9.743501in}{4.996907in}}{\pgfqpoint{9.739111in}{4.986308in}}{\pgfqpoint{9.739111in}{4.975258in}}%
\pgfpathcurveto{\pgfqpoint{9.739111in}{4.964208in}}{\pgfqpoint{9.743501in}{4.953609in}}{\pgfqpoint{9.751314in}{4.945795in}}%
\pgfpathcurveto{\pgfqpoint{9.759128in}{4.937981in}}{\pgfqpoint{9.769727in}{4.933591in}}{\pgfqpoint{9.780777in}{4.933591in}}%
\pgfpathlineto{\pgfqpoint{9.780777in}{4.933591in}}%
\pgfpathclose%
\pgfusepath{stroke}%
\end{pgfscope}%
\begin{pgfscope}%
\pgfpathrectangle{\pgfqpoint{7.394209in}{0.375000in}}{\pgfqpoint{6.356833in}{5.175000in}}%
\pgfusepath{clip}%
\pgfsetbuttcap%
\pgfsetroundjoin%
\pgfsetlinewidth{1.003750pt}%
\definecolor{currentstroke}{rgb}{0.827451,0.827451,0.827451}%
\pgfsetstrokecolor{currentstroke}%
\pgfsetdash{}{0pt}%
\pgfpathmoveto{\pgfqpoint{11.291232in}{4.838533in}}%
\pgfpathcurveto{\pgfqpoint{11.302283in}{4.838533in}}{\pgfqpoint{11.312882in}{4.842923in}}{\pgfqpoint{11.320695in}{4.850736in}}%
\pgfpathcurveto{\pgfqpoint{11.328509in}{4.858550in}}{\pgfqpoint{11.332899in}{4.869149in}}{\pgfqpoint{11.332899in}{4.880199in}}%
\pgfpathcurveto{\pgfqpoint{11.332899in}{4.891249in}}{\pgfqpoint{11.328509in}{4.901848in}}{\pgfqpoint{11.320695in}{4.909662in}}%
\pgfpathcurveto{\pgfqpoint{11.312882in}{4.917476in}}{\pgfqpoint{11.302283in}{4.921866in}}{\pgfqpoint{11.291232in}{4.921866in}}%
\pgfpathcurveto{\pgfqpoint{11.280182in}{4.921866in}}{\pgfqpoint{11.269583in}{4.917476in}}{\pgfqpoint{11.261770in}{4.909662in}}%
\pgfpathcurveto{\pgfqpoint{11.253956in}{4.901848in}}{\pgfqpoint{11.249566in}{4.891249in}}{\pgfqpoint{11.249566in}{4.880199in}}%
\pgfpathcurveto{\pgfqpoint{11.249566in}{4.869149in}}{\pgfqpoint{11.253956in}{4.858550in}}{\pgfqpoint{11.261770in}{4.850736in}}%
\pgfpathcurveto{\pgfqpoint{11.269583in}{4.842923in}}{\pgfqpoint{11.280182in}{4.838533in}}{\pgfqpoint{11.291232in}{4.838533in}}%
\pgfpathlineto{\pgfqpoint{11.291232in}{4.838533in}}%
\pgfpathclose%
\pgfusepath{stroke}%
\end{pgfscope}%
\begin{pgfscope}%
\pgfpathrectangle{\pgfqpoint{7.394209in}{0.375000in}}{\pgfqpoint{6.356833in}{5.175000in}}%
\pgfusepath{clip}%
\pgfsetbuttcap%
\pgfsetroundjoin%
\pgfsetlinewidth{1.003750pt}%
\definecolor{currentstroke}{rgb}{0.827451,0.827451,0.827451}%
\pgfsetstrokecolor{currentstroke}%
\pgfsetdash{}{0pt}%
\pgfpathmoveto{\pgfqpoint{7.515847in}{0.734484in}}%
\pgfpathcurveto{\pgfqpoint{7.526897in}{0.734484in}}{\pgfqpoint{7.537496in}{0.738874in}}{\pgfqpoint{7.545310in}{0.746688in}}%
\pgfpathcurveto{\pgfqpoint{7.553123in}{0.754502in}}{\pgfqpoint{7.557514in}{0.765101in}}{\pgfqpoint{7.557514in}{0.776151in}}%
\pgfpathcurveto{\pgfqpoint{7.557514in}{0.787201in}}{\pgfqpoint{7.553123in}{0.797800in}}{\pgfqpoint{7.545310in}{0.805614in}}%
\pgfpathcurveto{\pgfqpoint{7.537496in}{0.813427in}}{\pgfqpoint{7.526897in}{0.817817in}}{\pgfqpoint{7.515847in}{0.817817in}}%
\pgfpathcurveto{\pgfqpoint{7.504797in}{0.817817in}}{\pgfqpoint{7.494198in}{0.813427in}}{\pgfqpoint{7.486384in}{0.805614in}}%
\pgfpathcurveto{\pgfqpoint{7.478570in}{0.797800in}}{\pgfqpoint{7.474180in}{0.787201in}}{\pgfqpoint{7.474180in}{0.776151in}}%
\pgfpathcurveto{\pgfqpoint{7.474180in}{0.765101in}}{\pgfqpoint{7.478570in}{0.754502in}}{\pgfqpoint{7.486384in}{0.746688in}}%
\pgfpathcurveto{\pgfqpoint{7.494198in}{0.738874in}}{\pgfqpoint{7.504797in}{0.734484in}}{\pgfqpoint{7.515847in}{0.734484in}}%
\pgfpathlineto{\pgfqpoint{7.515847in}{0.734484in}}%
\pgfpathclose%
\pgfusepath{stroke}%
\end{pgfscope}%
\begin{pgfscope}%
\pgfpathrectangle{\pgfqpoint{7.394209in}{0.375000in}}{\pgfqpoint{6.356833in}{5.175000in}}%
\pgfusepath{clip}%
\pgfsetbuttcap%
\pgfsetroundjoin%
\pgfsetlinewidth{1.003750pt}%
\definecolor{currentstroke}{rgb}{0.827451,0.827451,0.827451}%
\pgfsetstrokecolor{currentstroke}%
\pgfsetdash{}{0pt}%
\pgfpathmoveto{\pgfqpoint{11.281291in}{5.045876in}}%
\pgfpathcurveto{\pgfqpoint{11.292341in}{5.045876in}}{\pgfqpoint{11.302940in}{5.050266in}}{\pgfqpoint{11.310754in}{5.058079in}}%
\pgfpathcurveto{\pgfqpoint{11.318568in}{5.065893in}}{\pgfqpoint{11.322958in}{5.076492in}}{\pgfqpoint{11.322958in}{5.087542in}}%
\pgfpathcurveto{\pgfqpoint{11.322958in}{5.098592in}}{\pgfqpoint{11.318568in}{5.109191in}}{\pgfqpoint{11.310754in}{5.117005in}}%
\pgfpathcurveto{\pgfqpoint{11.302940in}{5.124819in}}{\pgfqpoint{11.292341in}{5.129209in}}{\pgfqpoint{11.281291in}{5.129209in}}%
\pgfpathcurveto{\pgfqpoint{11.270241in}{5.129209in}}{\pgfqpoint{11.259642in}{5.124819in}}{\pgfqpoint{11.251828in}{5.117005in}}%
\pgfpathcurveto{\pgfqpoint{11.244015in}{5.109191in}}{\pgfqpoint{11.239624in}{5.098592in}}{\pgfqpoint{11.239624in}{5.087542in}}%
\pgfpathcurveto{\pgfqpoint{11.239624in}{5.076492in}}{\pgfqpoint{11.244015in}{5.065893in}}{\pgfqpoint{11.251828in}{5.058079in}}%
\pgfpathcurveto{\pgfqpoint{11.259642in}{5.050266in}}{\pgfqpoint{11.270241in}{5.045876in}}{\pgfqpoint{11.281291in}{5.045876in}}%
\pgfpathlineto{\pgfqpoint{11.281291in}{5.045876in}}%
\pgfpathclose%
\pgfusepath{stroke}%
\end{pgfscope}%
\begin{pgfscope}%
\pgfpathrectangle{\pgfqpoint{7.394209in}{0.375000in}}{\pgfqpoint{6.356833in}{5.175000in}}%
\pgfusepath{clip}%
\pgfsetbuttcap%
\pgfsetroundjoin%
\pgfsetlinewidth{1.003750pt}%
\definecolor{currentstroke}{rgb}{0.827451,0.827451,0.827451}%
\pgfsetstrokecolor{currentstroke}%
\pgfsetdash{}{0pt}%
\pgfpathmoveto{\pgfqpoint{8.174087in}{1.973506in}}%
\pgfpathcurveto{\pgfqpoint{8.185137in}{1.973506in}}{\pgfqpoint{8.195736in}{1.977896in}}{\pgfqpoint{8.203550in}{1.985710in}}%
\pgfpathcurveto{\pgfqpoint{8.211364in}{1.993523in}}{\pgfqpoint{8.215754in}{2.004122in}}{\pgfqpoint{8.215754in}{2.015173in}}%
\pgfpathcurveto{\pgfqpoint{8.215754in}{2.026223in}}{\pgfqpoint{8.211364in}{2.036822in}}{\pgfqpoint{8.203550in}{2.044635in}}%
\pgfpathcurveto{\pgfqpoint{8.195736in}{2.052449in}}{\pgfqpoint{8.185137in}{2.056839in}}{\pgfqpoint{8.174087in}{2.056839in}}%
\pgfpathcurveto{\pgfqpoint{8.163037in}{2.056839in}}{\pgfqpoint{8.152438in}{2.052449in}}{\pgfqpoint{8.144624in}{2.044635in}}%
\pgfpathcurveto{\pgfqpoint{8.136811in}{2.036822in}}{\pgfqpoint{8.132421in}{2.026223in}}{\pgfqpoint{8.132421in}{2.015173in}}%
\pgfpathcurveto{\pgfqpoint{8.132421in}{2.004122in}}{\pgfqpoint{8.136811in}{1.993523in}}{\pgfqpoint{8.144624in}{1.985710in}}%
\pgfpathcurveto{\pgfqpoint{8.152438in}{1.977896in}}{\pgfqpoint{8.163037in}{1.973506in}}{\pgfqpoint{8.174087in}{1.973506in}}%
\pgfpathlineto{\pgfqpoint{8.174087in}{1.973506in}}%
\pgfpathclose%
\pgfusepath{stroke}%
\end{pgfscope}%
\begin{pgfscope}%
\pgfpathrectangle{\pgfqpoint{7.394209in}{0.375000in}}{\pgfqpoint{6.356833in}{5.175000in}}%
\pgfusepath{clip}%
\pgfsetbuttcap%
\pgfsetroundjoin%
\pgfsetlinewidth{1.003750pt}%
\definecolor{currentstroke}{rgb}{0.827451,0.827451,0.827451}%
\pgfsetstrokecolor{currentstroke}%
\pgfsetdash{}{0pt}%
\pgfpathmoveto{\pgfqpoint{11.485377in}{5.129770in}}%
\pgfpathcurveto{\pgfqpoint{11.496427in}{5.129770in}}{\pgfqpoint{11.507026in}{5.134160in}}{\pgfqpoint{11.514840in}{5.141973in}}%
\pgfpathcurveto{\pgfqpoint{11.522653in}{5.149787in}}{\pgfqpoint{11.527043in}{5.160386in}}{\pgfqpoint{11.527043in}{5.171436in}}%
\pgfpathcurveto{\pgfqpoint{11.527043in}{5.182486in}}{\pgfqpoint{11.522653in}{5.193085in}}{\pgfqpoint{11.514840in}{5.200899in}}%
\pgfpathcurveto{\pgfqpoint{11.507026in}{5.208713in}}{\pgfqpoint{11.496427in}{5.213103in}}{\pgfqpoint{11.485377in}{5.213103in}}%
\pgfpathcurveto{\pgfqpoint{11.474327in}{5.213103in}}{\pgfqpoint{11.463728in}{5.208713in}}{\pgfqpoint{11.455914in}{5.200899in}}%
\pgfpathcurveto{\pgfqpoint{11.448100in}{5.193085in}}{\pgfqpoint{11.443710in}{5.182486in}}{\pgfqpoint{11.443710in}{5.171436in}}%
\pgfpathcurveto{\pgfqpoint{11.443710in}{5.160386in}}{\pgfqpoint{11.448100in}{5.149787in}}{\pgfqpoint{11.455914in}{5.141973in}}%
\pgfpathcurveto{\pgfqpoint{11.463728in}{5.134160in}}{\pgfqpoint{11.474327in}{5.129770in}}{\pgfqpoint{11.485377in}{5.129770in}}%
\pgfpathlineto{\pgfqpoint{11.485377in}{5.129770in}}%
\pgfpathclose%
\pgfusepath{stroke}%
\end{pgfscope}%
\begin{pgfscope}%
\pgfpathrectangle{\pgfqpoint{7.394209in}{0.375000in}}{\pgfqpoint{6.356833in}{5.175000in}}%
\pgfusepath{clip}%
\pgfsetbuttcap%
\pgfsetroundjoin%
\pgfsetlinewidth{1.003750pt}%
\definecolor{currentstroke}{rgb}{0.827451,0.827451,0.827451}%
\pgfsetstrokecolor{currentstroke}%
\pgfsetdash{}{0pt}%
\pgfpathmoveto{\pgfqpoint{8.174087in}{2.044655in}}%
\pgfpathcurveto{\pgfqpoint{8.185137in}{2.044655in}}{\pgfqpoint{8.195736in}{2.049045in}}{\pgfqpoint{8.203550in}{2.056859in}}%
\pgfpathcurveto{\pgfqpoint{8.211364in}{2.064673in}}{\pgfqpoint{8.215754in}{2.075272in}}{\pgfqpoint{8.215754in}{2.086322in}}%
\pgfpathcurveto{\pgfqpoint{8.215754in}{2.097372in}}{\pgfqpoint{8.211364in}{2.107971in}}{\pgfqpoint{8.203550in}{2.115785in}}%
\pgfpathcurveto{\pgfqpoint{8.195736in}{2.123598in}}{\pgfqpoint{8.185137in}{2.127989in}}{\pgfqpoint{8.174087in}{2.127989in}}%
\pgfpathcurveto{\pgfqpoint{8.163037in}{2.127989in}}{\pgfqpoint{8.152438in}{2.123598in}}{\pgfqpoint{8.144624in}{2.115785in}}%
\pgfpathcurveto{\pgfqpoint{8.136811in}{2.107971in}}{\pgfqpoint{8.132421in}{2.097372in}}{\pgfqpoint{8.132421in}{2.086322in}}%
\pgfpathcurveto{\pgfqpoint{8.132421in}{2.075272in}}{\pgfqpoint{8.136811in}{2.064673in}}{\pgfqpoint{8.144624in}{2.056859in}}%
\pgfpathcurveto{\pgfqpoint{8.152438in}{2.049045in}}{\pgfqpoint{8.163037in}{2.044655in}}{\pgfqpoint{8.174087in}{2.044655in}}%
\pgfpathlineto{\pgfqpoint{8.174087in}{2.044655in}}%
\pgfpathclose%
\pgfusepath{stroke}%
\end{pgfscope}%
\begin{pgfscope}%
\pgfpathrectangle{\pgfqpoint{7.394209in}{0.375000in}}{\pgfqpoint{6.356833in}{5.175000in}}%
\pgfusepath{clip}%
\pgfsetbuttcap%
\pgfsetroundjoin%
\pgfsetlinewidth{1.003750pt}%
\definecolor{currentstroke}{rgb}{0.827451,0.827451,0.827451}%
\pgfsetstrokecolor{currentstroke}%
\pgfsetdash{}{0pt}%
\pgfpathmoveto{\pgfqpoint{7.397274in}{0.542553in}}%
\pgfpathcurveto{\pgfqpoint{7.408324in}{0.542553in}}{\pgfqpoint{7.418923in}{0.546944in}}{\pgfqpoint{7.426737in}{0.554757in}}%
\pgfpathcurveto{\pgfqpoint{7.434550in}{0.562571in}}{\pgfqpoint{7.438940in}{0.573170in}}{\pgfqpoint{7.438940in}{0.584220in}}%
\pgfpathcurveto{\pgfqpoint{7.438940in}{0.595270in}}{\pgfqpoint{7.434550in}{0.605869in}}{\pgfqpoint{7.426737in}{0.613683in}}%
\pgfpathcurveto{\pgfqpoint{7.418923in}{0.621496in}}{\pgfqpoint{7.408324in}{0.625887in}}{\pgfqpoint{7.397274in}{0.625887in}}%
\pgfpathcurveto{\pgfqpoint{7.386224in}{0.625887in}}{\pgfqpoint{7.375625in}{0.621496in}}{\pgfqpoint{7.367811in}{0.613683in}}%
\pgfpathcurveto{\pgfqpoint{7.359997in}{0.605869in}}{\pgfqpoint{7.355607in}{0.595270in}}{\pgfqpoint{7.355607in}{0.584220in}}%
\pgfpathcurveto{\pgfqpoint{7.355607in}{0.573170in}}{\pgfqpoint{7.359997in}{0.562571in}}{\pgfqpoint{7.367811in}{0.554757in}}%
\pgfpathcurveto{\pgfqpoint{7.375625in}{0.546944in}}{\pgfqpoint{7.386224in}{0.542553in}}{\pgfqpoint{7.397274in}{0.542553in}}%
\pgfpathlineto{\pgfqpoint{7.397274in}{0.542553in}}%
\pgfpathclose%
\pgfusepath{stroke}%
\end{pgfscope}%
\begin{pgfscope}%
\pgfpathrectangle{\pgfqpoint{7.394209in}{0.375000in}}{\pgfqpoint{6.356833in}{5.175000in}}%
\pgfusepath{clip}%
\pgfsetbuttcap%
\pgfsetroundjoin%
\pgfsetlinewidth{1.003750pt}%
\definecolor{currentstroke}{rgb}{0.827451,0.827451,0.827451}%
\pgfsetstrokecolor{currentstroke}%
\pgfsetdash{}{0pt}%
\pgfpathmoveto{\pgfqpoint{8.162049in}{0.851328in}}%
\pgfpathcurveto{\pgfqpoint{8.173099in}{0.851328in}}{\pgfqpoint{8.183698in}{0.855718in}}{\pgfqpoint{8.191512in}{0.863532in}}%
\pgfpathcurveto{\pgfqpoint{8.199325in}{0.871345in}}{\pgfqpoint{8.203716in}{0.881944in}}{\pgfqpoint{8.203716in}{0.892994in}}%
\pgfpathcurveto{\pgfqpoint{8.203716in}{0.904045in}}{\pgfqpoint{8.199325in}{0.914644in}}{\pgfqpoint{8.191512in}{0.922457in}}%
\pgfpathcurveto{\pgfqpoint{8.183698in}{0.930271in}}{\pgfqpoint{8.173099in}{0.934661in}}{\pgfqpoint{8.162049in}{0.934661in}}%
\pgfpathcurveto{\pgfqpoint{8.150999in}{0.934661in}}{\pgfqpoint{8.140400in}{0.930271in}}{\pgfqpoint{8.132586in}{0.922457in}}%
\pgfpathcurveto{\pgfqpoint{8.124773in}{0.914644in}}{\pgfqpoint{8.120382in}{0.904045in}}{\pgfqpoint{8.120382in}{0.892994in}}%
\pgfpathcurveto{\pgfqpoint{8.120382in}{0.881944in}}{\pgfqpoint{8.124773in}{0.871345in}}{\pgfqpoint{8.132586in}{0.863532in}}%
\pgfpathcurveto{\pgfqpoint{8.140400in}{0.855718in}}{\pgfqpoint{8.150999in}{0.851328in}}{\pgfqpoint{8.162049in}{0.851328in}}%
\pgfpathlineto{\pgfqpoint{8.162049in}{0.851328in}}%
\pgfpathclose%
\pgfusepath{stroke}%
\end{pgfscope}%
\begin{pgfscope}%
\pgfpathrectangle{\pgfqpoint{7.394209in}{0.375000in}}{\pgfqpoint{6.356833in}{5.175000in}}%
\pgfusepath{clip}%
\pgfsetbuttcap%
\pgfsetroundjoin%
\pgfsetlinewidth{1.003750pt}%
\definecolor{currentstroke}{rgb}{0.827451,0.827451,0.827451}%
\pgfsetstrokecolor{currentstroke}%
\pgfsetdash{}{0pt}%
\pgfpathmoveto{\pgfqpoint{12.103123in}{5.440849in}}%
\pgfpathcurveto{\pgfqpoint{12.114173in}{5.440849in}}{\pgfqpoint{12.124772in}{5.445239in}}{\pgfqpoint{12.132586in}{5.453053in}}%
\pgfpathcurveto{\pgfqpoint{12.140399in}{5.460866in}}{\pgfqpoint{12.144790in}{5.471465in}}{\pgfqpoint{12.144790in}{5.482516in}}%
\pgfpathcurveto{\pgfqpoint{12.144790in}{5.493566in}}{\pgfqpoint{12.140399in}{5.504165in}}{\pgfqpoint{12.132586in}{5.511978in}}%
\pgfpathcurveto{\pgfqpoint{12.124772in}{5.519792in}}{\pgfqpoint{12.114173in}{5.524182in}}{\pgfqpoint{12.103123in}{5.524182in}}%
\pgfpathcurveto{\pgfqpoint{12.092073in}{5.524182in}}{\pgfqpoint{12.081474in}{5.519792in}}{\pgfqpoint{12.073660in}{5.511978in}}%
\pgfpathcurveto{\pgfqpoint{12.065847in}{5.504165in}}{\pgfqpoint{12.061456in}{5.493566in}}{\pgfqpoint{12.061456in}{5.482516in}}%
\pgfpathcurveto{\pgfqpoint{12.061456in}{5.471465in}}{\pgfqpoint{12.065847in}{5.460866in}}{\pgfqpoint{12.073660in}{5.453053in}}%
\pgfpathcurveto{\pgfqpoint{12.081474in}{5.445239in}}{\pgfqpoint{12.092073in}{5.440849in}}{\pgfqpoint{12.103123in}{5.440849in}}%
\pgfpathlineto{\pgfqpoint{12.103123in}{5.440849in}}%
\pgfpathclose%
\pgfusepath{stroke}%
\end{pgfscope}%
\begin{pgfscope}%
\pgfpathrectangle{\pgfqpoint{7.394209in}{0.375000in}}{\pgfqpoint{6.356833in}{5.175000in}}%
\pgfusepath{clip}%
\pgfsetbuttcap%
\pgfsetroundjoin%
\pgfsetlinewidth{1.003750pt}%
\definecolor{currentstroke}{rgb}{0.827451,0.827451,0.827451}%
\pgfsetstrokecolor{currentstroke}%
\pgfsetdash{}{0pt}%
\pgfpathmoveto{\pgfqpoint{9.694836in}{4.589001in}}%
\pgfpathcurveto{\pgfqpoint{9.705886in}{4.589001in}}{\pgfqpoint{9.716486in}{4.593392in}}{\pgfqpoint{9.724299in}{4.601205in}}%
\pgfpathcurveto{\pgfqpoint{9.732113in}{4.609019in}}{\pgfqpoint{9.736503in}{4.619618in}}{\pgfqpoint{9.736503in}{4.630668in}}%
\pgfpathcurveto{\pgfqpoint{9.736503in}{4.641718in}}{\pgfqpoint{9.732113in}{4.652317in}}{\pgfqpoint{9.724299in}{4.660131in}}%
\pgfpathcurveto{\pgfqpoint{9.716486in}{4.667944in}}{\pgfqpoint{9.705886in}{4.672335in}}{\pgfqpoint{9.694836in}{4.672335in}}%
\pgfpathcurveto{\pgfqpoint{9.683786in}{4.672335in}}{\pgfqpoint{9.673187in}{4.667944in}}{\pgfqpoint{9.665374in}{4.660131in}}%
\pgfpathcurveto{\pgfqpoint{9.657560in}{4.652317in}}{\pgfqpoint{9.653170in}{4.641718in}}{\pgfqpoint{9.653170in}{4.630668in}}%
\pgfpathcurveto{\pgfqpoint{9.653170in}{4.619618in}}{\pgfqpoint{9.657560in}{4.609019in}}{\pgfqpoint{9.665374in}{4.601205in}}%
\pgfpathcurveto{\pgfqpoint{9.673187in}{4.593392in}}{\pgfqpoint{9.683786in}{4.589001in}}{\pgfqpoint{9.694836in}{4.589001in}}%
\pgfpathlineto{\pgfqpoint{9.694836in}{4.589001in}}%
\pgfpathclose%
\pgfusepath{stroke}%
\end{pgfscope}%
\begin{pgfscope}%
\pgfpathrectangle{\pgfqpoint{7.394209in}{0.375000in}}{\pgfqpoint{6.356833in}{5.175000in}}%
\pgfusepath{clip}%
\pgfsetbuttcap%
\pgfsetroundjoin%
\pgfsetlinewidth{1.003750pt}%
\definecolor{currentstroke}{rgb}{0.827451,0.827451,0.827451}%
\pgfsetstrokecolor{currentstroke}%
\pgfsetdash{}{0pt}%
\pgfpathmoveto{\pgfqpoint{10.132828in}{3.770260in}}%
\pgfpathcurveto{\pgfqpoint{10.143878in}{3.770260in}}{\pgfqpoint{10.154477in}{3.774650in}}{\pgfqpoint{10.162291in}{3.782464in}}%
\pgfpathcurveto{\pgfqpoint{10.170105in}{3.790278in}}{\pgfqpoint{10.174495in}{3.800877in}}{\pgfqpoint{10.174495in}{3.811927in}}%
\pgfpathcurveto{\pgfqpoint{10.174495in}{3.822977in}}{\pgfqpoint{10.170105in}{3.833576in}}{\pgfqpoint{10.162291in}{3.841390in}}%
\pgfpathcurveto{\pgfqpoint{10.154477in}{3.849203in}}{\pgfqpoint{10.143878in}{3.853593in}}{\pgfqpoint{10.132828in}{3.853593in}}%
\pgfpathcurveto{\pgfqpoint{10.121778in}{3.853593in}}{\pgfqpoint{10.111179in}{3.849203in}}{\pgfqpoint{10.103365in}{3.841390in}}%
\pgfpathcurveto{\pgfqpoint{10.095552in}{3.833576in}}{\pgfqpoint{10.091162in}{3.822977in}}{\pgfqpoint{10.091162in}{3.811927in}}%
\pgfpathcurveto{\pgfqpoint{10.091162in}{3.800877in}}{\pgfqpoint{10.095552in}{3.790278in}}{\pgfqpoint{10.103365in}{3.782464in}}%
\pgfpathcurveto{\pgfqpoint{10.111179in}{3.774650in}}{\pgfqpoint{10.121778in}{3.770260in}}{\pgfqpoint{10.132828in}{3.770260in}}%
\pgfpathlineto{\pgfqpoint{10.132828in}{3.770260in}}%
\pgfpathclose%
\pgfusepath{stroke}%
\end{pgfscope}%
\begin{pgfscope}%
\pgfpathrectangle{\pgfqpoint{7.394209in}{0.375000in}}{\pgfqpoint{6.356833in}{5.175000in}}%
\pgfusepath{clip}%
\pgfsetbuttcap%
\pgfsetroundjoin%
\pgfsetlinewidth{1.003750pt}%
\definecolor{currentstroke}{rgb}{0.827451,0.827451,0.827451}%
\pgfsetstrokecolor{currentstroke}%
\pgfsetdash{}{0pt}%
\pgfpathmoveto{\pgfqpoint{8.742298in}{1.326663in}}%
\pgfpathcurveto{\pgfqpoint{8.753348in}{1.326663in}}{\pgfqpoint{8.763947in}{1.331053in}}{\pgfqpoint{8.771761in}{1.338866in}}%
\pgfpathcurveto{\pgfqpoint{8.779574in}{1.346680in}}{\pgfqpoint{8.783965in}{1.357279in}}{\pgfqpoint{8.783965in}{1.368329in}}%
\pgfpathcurveto{\pgfqpoint{8.783965in}{1.379379in}}{\pgfqpoint{8.779574in}{1.389978in}}{\pgfqpoint{8.771761in}{1.397792in}}%
\pgfpathcurveto{\pgfqpoint{8.763947in}{1.405606in}}{\pgfqpoint{8.753348in}{1.409996in}}{\pgfqpoint{8.742298in}{1.409996in}}%
\pgfpathcurveto{\pgfqpoint{8.731248in}{1.409996in}}{\pgfqpoint{8.720649in}{1.405606in}}{\pgfqpoint{8.712835in}{1.397792in}}%
\pgfpathcurveto{\pgfqpoint{8.705022in}{1.389978in}}{\pgfqpoint{8.700631in}{1.379379in}}{\pgfqpoint{8.700631in}{1.368329in}}%
\pgfpathcurveto{\pgfqpoint{8.700631in}{1.357279in}}{\pgfqpoint{8.705022in}{1.346680in}}{\pgfqpoint{8.712835in}{1.338866in}}%
\pgfpathcurveto{\pgfqpoint{8.720649in}{1.331053in}}{\pgfqpoint{8.731248in}{1.326663in}}{\pgfqpoint{8.742298in}{1.326663in}}%
\pgfpathlineto{\pgfqpoint{8.742298in}{1.326663in}}%
\pgfpathclose%
\pgfusepath{stroke}%
\end{pgfscope}%
\begin{pgfscope}%
\pgfpathrectangle{\pgfqpoint{7.394209in}{0.375000in}}{\pgfqpoint{6.356833in}{5.175000in}}%
\pgfusepath{clip}%
\pgfsetbuttcap%
\pgfsetroundjoin%
\pgfsetlinewidth{1.003750pt}%
\definecolor{currentstroke}{rgb}{0.827451,0.827451,0.827451}%
\pgfsetstrokecolor{currentstroke}%
\pgfsetdash{}{0pt}%
\pgfpathmoveto{\pgfqpoint{8.092044in}{1.857108in}}%
\pgfpathcurveto{\pgfqpoint{8.103094in}{1.857108in}}{\pgfqpoint{8.113693in}{1.861498in}}{\pgfqpoint{8.121507in}{1.869312in}}%
\pgfpathcurveto{\pgfqpoint{8.129320in}{1.877125in}}{\pgfqpoint{8.133711in}{1.887724in}}{\pgfqpoint{8.133711in}{1.898774in}}%
\pgfpathcurveto{\pgfqpoint{8.133711in}{1.909825in}}{\pgfqpoint{8.129320in}{1.920424in}}{\pgfqpoint{8.121507in}{1.928237in}}%
\pgfpathcurveto{\pgfqpoint{8.113693in}{1.936051in}}{\pgfqpoint{8.103094in}{1.940441in}}{\pgfqpoint{8.092044in}{1.940441in}}%
\pgfpathcurveto{\pgfqpoint{8.080994in}{1.940441in}}{\pgfqpoint{8.070395in}{1.936051in}}{\pgfqpoint{8.062581in}{1.928237in}}%
\pgfpathcurveto{\pgfqpoint{8.054768in}{1.920424in}}{\pgfqpoint{8.050377in}{1.909825in}}{\pgfqpoint{8.050377in}{1.898774in}}%
\pgfpathcurveto{\pgfqpoint{8.050377in}{1.887724in}}{\pgfqpoint{8.054768in}{1.877125in}}{\pgfqpoint{8.062581in}{1.869312in}}%
\pgfpathcurveto{\pgfqpoint{8.070395in}{1.861498in}}{\pgfqpoint{8.080994in}{1.857108in}}{\pgfqpoint{8.092044in}{1.857108in}}%
\pgfpathlineto{\pgfqpoint{8.092044in}{1.857108in}}%
\pgfpathclose%
\pgfusepath{stroke}%
\end{pgfscope}%
\begin{pgfscope}%
\pgfpathrectangle{\pgfqpoint{7.394209in}{0.375000in}}{\pgfqpoint{6.356833in}{5.175000in}}%
\pgfusepath{clip}%
\pgfsetbuttcap%
\pgfsetroundjoin%
\pgfsetlinewidth{1.003750pt}%
\definecolor{currentstroke}{rgb}{0.827451,0.827451,0.827451}%
\pgfsetstrokecolor{currentstroke}%
\pgfsetdash{}{0pt}%
\pgfpathmoveto{\pgfqpoint{10.005918in}{4.802768in}}%
\pgfpathcurveto{\pgfqpoint{10.016968in}{4.802768in}}{\pgfqpoint{10.027568in}{4.807159in}}{\pgfqpoint{10.035381in}{4.814972in}}%
\pgfpathcurveto{\pgfqpoint{10.043195in}{4.822786in}}{\pgfqpoint{10.047585in}{4.833385in}}{\pgfqpoint{10.047585in}{4.844435in}}%
\pgfpathcurveto{\pgfqpoint{10.047585in}{4.855485in}}{\pgfqpoint{10.043195in}{4.866084in}}{\pgfqpoint{10.035381in}{4.873898in}}%
\pgfpathcurveto{\pgfqpoint{10.027568in}{4.881712in}}{\pgfqpoint{10.016968in}{4.886102in}}{\pgfqpoint{10.005918in}{4.886102in}}%
\pgfpathcurveto{\pgfqpoint{9.994868in}{4.886102in}}{\pgfqpoint{9.984269in}{4.881712in}}{\pgfqpoint{9.976456in}{4.873898in}}%
\pgfpathcurveto{\pgfqpoint{9.968642in}{4.866084in}}{\pgfqpoint{9.964252in}{4.855485in}}{\pgfqpoint{9.964252in}{4.844435in}}%
\pgfpathcurveto{\pgfqpoint{9.964252in}{4.833385in}}{\pgfqpoint{9.968642in}{4.822786in}}{\pgfqpoint{9.976456in}{4.814972in}}%
\pgfpathcurveto{\pgfqpoint{9.984269in}{4.807159in}}{\pgfqpoint{9.994868in}{4.802768in}}{\pgfqpoint{10.005918in}{4.802768in}}%
\pgfpathlineto{\pgfqpoint{10.005918in}{4.802768in}}%
\pgfpathclose%
\pgfusepath{stroke}%
\end{pgfscope}%
\begin{pgfscope}%
\pgfpathrectangle{\pgfqpoint{7.394209in}{0.375000in}}{\pgfqpoint{6.356833in}{5.175000in}}%
\pgfusepath{clip}%
\pgfsetbuttcap%
\pgfsetroundjoin%
\pgfsetlinewidth{1.003750pt}%
\definecolor{currentstroke}{rgb}{0.827451,0.827451,0.827451}%
\pgfsetstrokecolor{currentstroke}%
\pgfsetdash{}{0pt}%
\pgfpathmoveto{\pgfqpoint{7.401160in}{0.530606in}}%
\pgfpathcurveto{\pgfqpoint{7.412210in}{0.530606in}}{\pgfqpoint{7.422809in}{0.534996in}}{\pgfqpoint{7.430623in}{0.542810in}}%
\pgfpathcurveto{\pgfqpoint{7.438437in}{0.550623in}}{\pgfqpoint{7.442827in}{0.561222in}}{\pgfqpoint{7.442827in}{0.572272in}}%
\pgfpathcurveto{\pgfqpoint{7.442827in}{0.583322in}}{\pgfqpoint{7.438437in}{0.593921in}}{\pgfqpoint{7.430623in}{0.601735in}}%
\pgfpathcurveto{\pgfqpoint{7.422809in}{0.609549in}}{\pgfqpoint{7.412210in}{0.613939in}}{\pgfqpoint{7.401160in}{0.613939in}}%
\pgfpathcurveto{\pgfqpoint{7.390110in}{0.613939in}}{\pgfqpoint{7.379511in}{0.609549in}}{\pgfqpoint{7.371697in}{0.601735in}}%
\pgfpathcurveto{\pgfqpoint{7.363884in}{0.593921in}}{\pgfqpoint{7.359493in}{0.583322in}}{\pgfqpoint{7.359493in}{0.572272in}}%
\pgfpathcurveto{\pgfqpoint{7.359493in}{0.561222in}}{\pgfqpoint{7.363884in}{0.550623in}}{\pgfqpoint{7.371697in}{0.542810in}}%
\pgfpathcurveto{\pgfqpoint{7.379511in}{0.534996in}}{\pgfqpoint{7.390110in}{0.530606in}}{\pgfqpoint{7.401160in}{0.530606in}}%
\pgfpathlineto{\pgfqpoint{7.401160in}{0.530606in}}%
\pgfpathclose%
\pgfusepath{stroke}%
\end{pgfscope}%
\begin{pgfscope}%
\pgfpathrectangle{\pgfqpoint{7.394209in}{0.375000in}}{\pgfqpoint{6.356833in}{5.175000in}}%
\pgfusepath{clip}%
\pgfsetbuttcap%
\pgfsetroundjoin%
\pgfsetlinewidth{1.003750pt}%
\definecolor{currentstroke}{rgb}{0.827451,0.827451,0.827451}%
\pgfsetstrokecolor{currentstroke}%
\pgfsetdash{}{0pt}%
\pgfpathmoveto{\pgfqpoint{12.831077in}{5.454066in}}%
\pgfpathcurveto{\pgfqpoint{12.842127in}{5.454066in}}{\pgfqpoint{12.852726in}{5.458456in}}{\pgfqpoint{12.860539in}{5.466270in}}%
\pgfpathcurveto{\pgfqpoint{12.868353in}{5.474084in}}{\pgfqpoint{12.872743in}{5.484683in}}{\pgfqpoint{12.872743in}{5.495733in}}%
\pgfpathcurveto{\pgfqpoint{12.872743in}{5.506783in}}{\pgfqpoint{12.868353in}{5.517382in}}{\pgfqpoint{12.860539in}{5.525196in}}%
\pgfpathcurveto{\pgfqpoint{12.852726in}{5.533009in}}{\pgfqpoint{12.842127in}{5.537399in}}{\pgfqpoint{12.831077in}{5.537399in}}%
\pgfpathcurveto{\pgfqpoint{12.820026in}{5.537399in}}{\pgfqpoint{12.809427in}{5.533009in}}{\pgfqpoint{12.801614in}{5.525196in}}%
\pgfpathcurveto{\pgfqpoint{12.793800in}{5.517382in}}{\pgfqpoint{12.789410in}{5.506783in}}{\pgfqpoint{12.789410in}{5.495733in}}%
\pgfpathcurveto{\pgfqpoint{12.789410in}{5.484683in}}{\pgfqpoint{12.793800in}{5.474084in}}{\pgfqpoint{12.801614in}{5.466270in}}%
\pgfpathcurveto{\pgfqpoint{12.809427in}{5.458456in}}{\pgfqpoint{12.820026in}{5.454066in}}{\pgfqpoint{12.831077in}{5.454066in}}%
\pgfpathlineto{\pgfqpoint{12.831077in}{5.454066in}}%
\pgfpathclose%
\pgfusepath{stroke}%
\end{pgfscope}%
\begin{pgfscope}%
\pgfpathrectangle{\pgfqpoint{7.394209in}{0.375000in}}{\pgfqpoint{6.356833in}{5.175000in}}%
\pgfusepath{clip}%
\pgfsetbuttcap%
\pgfsetroundjoin%
\pgfsetlinewidth{1.003750pt}%
\definecolor{currentstroke}{rgb}{0.827451,0.827451,0.827451}%
\pgfsetstrokecolor{currentstroke}%
\pgfsetdash{}{0pt}%
\pgfpathmoveto{\pgfqpoint{8.882396in}{3.533894in}}%
\pgfpathcurveto{\pgfqpoint{8.893446in}{3.533894in}}{\pgfqpoint{8.904045in}{3.538284in}}{\pgfqpoint{8.911859in}{3.546098in}}%
\pgfpathcurveto{\pgfqpoint{8.919672in}{3.553911in}}{\pgfqpoint{8.924062in}{3.564510in}}{\pgfqpoint{8.924062in}{3.575561in}}%
\pgfpathcurveto{\pgfqpoint{8.924062in}{3.586611in}}{\pgfqpoint{8.919672in}{3.597210in}}{\pgfqpoint{8.911859in}{3.605023in}}%
\pgfpathcurveto{\pgfqpoint{8.904045in}{3.612837in}}{\pgfqpoint{8.893446in}{3.617227in}}{\pgfqpoint{8.882396in}{3.617227in}}%
\pgfpathcurveto{\pgfqpoint{8.871346in}{3.617227in}}{\pgfqpoint{8.860747in}{3.612837in}}{\pgfqpoint{8.852933in}{3.605023in}}%
\pgfpathcurveto{\pgfqpoint{8.845119in}{3.597210in}}{\pgfqpoint{8.840729in}{3.586611in}}{\pgfqpoint{8.840729in}{3.575561in}}%
\pgfpathcurveto{\pgfqpoint{8.840729in}{3.564510in}}{\pgfqpoint{8.845119in}{3.553911in}}{\pgfqpoint{8.852933in}{3.546098in}}%
\pgfpathcurveto{\pgfqpoint{8.860747in}{3.538284in}}{\pgfqpoint{8.871346in}{3.533894in}}{\pgfqpoint{8.882396in}{3.533894in}}%
\pgfpathlineto{\pgfqpoint{8.882396in}{3.533894in}}%
\pgfpathclose%
\pgfusepath{stroke}%
\end{pgfscope}%
\begin{pgfscope}%
\pgfpathrectangle{\pgfqpoint{7.394209in}{0.375000in}}{\pgfqpoint{6.356833in}{5.175000in}}%
\pgfusepath{clip}%
\pgfsetbuttcap%
\pgfsetroundjoin%
\pgfsetlinewidth{1.003750pt}%
\definecolor{currentstroke}{rgb}{0.827451,0.827451,0.827451}%
\pgfsetstrokecolor{currentstroke}%
\pgfsetdash{}{0pt}%
\pgfpathmoveto{\pgfqpoint{13.461816in}{5.502450in}}%
\pgfpathcurveto{\pgfqpoint{13.472866in}{5.502450in}}{\pgfqpoint{13.483465in}{5.506840in}}{\pgfqpoint{13.491278in}{5.514654in}}%
\pgfpathcurveto{\pgfqpoint{13.499092in}{5.522467in}}{\pgfqpoint{13.503482in}{5.533066in}}{\pgfqpoint{13.503482in}{5.544116in}}%
\pgfpathcurveto{\pgfqpoint{13.503482in}{5.555167in}}{\pgfqpoint{13.499092in}{5.565766in}}{\pgfqpoint{13.491278in}{5.573579in}}%
\pgfpathcurveto{\pgfqpoint{13.483465in}{5.581393in}}{\pgfqpoint{13.472866in}{5.585783in}}{\pgfqpoint{13.461816in}{5.585783in}}%
\pgfpathcurveto{\pgfqpoint{13.450765in}{5.585783in}}{\pgfqpoint{13.440166in}{5.581393in}}{\pgfqpoint{13.432353in}{5.573579in}}%
\pgfpathcurveto{\pgfqpoint{13.424539in}{5.565766in}}{\pgfqpoint{13.420149in}{5.555167in}}{\pgfqpoint{13.420149in}{5.544116in}}%
\pgfpathcurveto{\pgfqpoint{13.420149in}{5.533066in}}{\pgfqpoint{13.424539in}{5.522467in}}{\pgfqpoint{13.432353in}{5.514654in}}%
\pgfpathcurveto{\pgfqpoint{13.440166in}{5.506840in}}{\pgfqpoint{13.450765in}{5.502450in}}{\pgfqpoint{13.461816in}{5.502450in}}%
\pgfpathlineto{\pgfqpoint{13.461816in}{5.502450in}}%
\pgfpathclose%
\pgfusepath{stroke}%
\end{pgfscope}%
\begin{pgfscope}%
\pgfpathrectangle{\pgfqpoint{7.394209in}{0.375000in}}{\pgfqpoint{6.356833in}{5.175000in}}%
\pgfusepath{clip}%
\pgfsetbuttcap%
\pgfsetroundjoin%
\pgfsetlinewidth{1.003750pt}%
\definecolor{currentstroke}{rgb}{0.827451,0.827451,0.827451}%
\pgfsetstrokecolor{currentstroke}%
\pgfsetdash{}{0pt}%
\pgfpathmoveto{\pgfqpoint{12.790840in}{5.493764in}}%
\pgfpathcurveto{\pgfqpoint{12.801890in}{5.493764in}}{\pgfqpoint{12.812489in}{5.498154in}}{\pgfqpoint{12.820303in}{5.505968in}}%
\pgfpathcurveto{\pgfqpoint{12.828117in}{5.513782in}}{\pgfqpoint{12.832507in}{5.524381in}}{\pgfqpoint{12.832507in}{5.535431in}}%
\pgfpathcurveto{\pgfqpoint{12.832507in}{5.546481in}}{\pgfqpoint{12.828117in}{5.557080in}}{\pgfqpoint{12.820303in}{5.564894in}}%
\pgfpathcurveto{\pgfqpoint{12.812489in}{5.572707in}}{\pgfqpoint{12.801890in}{5.577097in}}{\pgfqpoint{12.790840in}{5.577097in}}%
\pgfpathcurveto{\pgfqpoint{12.779790in}{5.577097in}}{\pgfqpoint{12.769191in}{5.572707in}}{\pgfqpoint{12.761378in}{5.564894in}}%
\pgfpathcurveto{\pgfqpoint{12.753564in}{5.557080in}}{\pgfqpoint{12.749174in}{5.546481in}}{\pgfqpoint{12.749174in}{5.535431in}}%
\pgfpathcurveto{\pgfqpoint{12.749174in}{5.524381in}}{\pgfqpoint{12.753564in}{5.513782in}}{\pgfqpoint{12.761378in}{5.505968in}}%
\pgfpathcurveto{\pgfqpoint{12.769191in}{5.498154in}}{\pgfqpoint{12.779790in}{5.493764in}}{\pgfqpoint{12.790840in}{5.493764in}}%
\pgfpathlineto{\pgfqpoint{12.790840in}{5.493764in}}%
\pgfpathclose%
\pgfusepath{stroke}%
\end{pgfscope}%
\begin{pgfscope}%
\pgfpathrectangle{\pgfqpoint{7.394209in}{0.375000in}}{\pgfqpoint{6.356833in}{5.175000in}}%
\pgfusepath{clip}%
\pgfsetbuttcap%
\pgfsetroundjoin%
\pgfsetlinewidth{1.003750pt}%
\definecolor{currentstroke}{rgb}{0.827451,0.827451,0.827451}%
\pgfsetstrokecolor{currentstroke}%
\pgfsetdash{}{0pt}%
\pgfpathmoveto{\pgfqpoint{8.742298in}{1.402196in}}%
\pgfpathcurveto{\pgfqpoint{8.753348in}{1.402196in}}{\pgfqpoint{8.763947in}{1.406586in}}{\pgfqpoint{8.771761in}{1.414400in}}%
\pgfpathcurveto{\pgfqpoint{8.779574in}{1.422214in}}{\pgfqpoint{8.783965in}{1.432813in}}{\pgfqpoint{8.783965in}{1.443863in}}%
\pgfpathcurveto{\pgfqpoint{8.783965in}{1.454913in}}{\pgfqpoint{8.779574in}{1.465512in}}{\pgfqpoint{8.771761in}{1.473326in}}%
\pgfpathcurveto{\pgfqpoint{8.763947in}{1.481139in}}{\pgfqpoint{8.753348in}{1.485529in}}{\pgfqpoint{8.742298in}{1.485529in}}%
\pgfpathcurveto{\pgfqpoint{8.731248in}{1.485529in}}{\pgfqpoint{8.720649in}{1.481139in}}{\pgfqpoint{8.712835in}{1.473326in}}%
\pgfpathcurveto{\pgfqpoint{8.705022in}{1.465512in}}{\pgfqpoint{8.700631in}{1.454913in}}{\pgfqpoint{8.700631in}{1.443863in}}%
\pgfpathcurveto{\pgfqpoint{8.700631in}{1.432813in}}{\pgfqpoint{8.705022in}{1.422214in}}{\pgfqpoint{8.712835in}{1.414400in}}%
\pgfpathcurveto{\pgfqpoint{8.720649in}{1.406586in}}{\pgfqpoint{8.731248in}{1.402196in}}{\pgfqpoint{8.742298in}{1.402196in}}%
\pgfpathlineto{\pgfqpoint{8.742298in}{1.402196in}}%
\pgfpathclose%
\pgfusepath{stroke}%
\end{pgfscope}%
\begin{pgfscope}%
\pgfpathrectangle{\pgfqpoint{7.394209in}{0.375000in}}{\pgfqpoint{6.356833in}{5.175000in}}%
\pgfusepath{clip}%
\pgfsetbuttcap%
\pgfsetroundjoin%
\pgfsetlinewidth{1.003750pt}%
\definecolor{currentstroke}{rgb}{0.827451,0.827451,0.827451}%
\pgfsetstrokecolor{currentstroke}%
\pgfsetdash{}{0pt}%
\pgfpathmoveto{\pgfqpoint{8.997871in}{3.074328in}}%
\pgfpathcurveto{\pgfqpoint{9.008921in}{3.074328in}}{\pgfqpoint{9.019520in}{3.078718in}}{\pgfqpoint{9.027334in}{3.086532in}}%
\pgfpathcurveto{\pgfqpoint{9.035148in}{3.094345in}}{\pgfqpoint{9.039538in}{3.104944in}}{\pgfqpoint{9.039538in}{3.115995in}}%
\pgfpathcurveto{\pgfqpoint{9.039538in}{3.127045in}}{\pgfqpoint{9.035148in}{3.137644in}}{\pgfqpoint{9.027334in}{3.145457in}}%
\pgfpathcurveto{\pgfqpoint{9.019520in}{3.153271in}}{\pgfqpoint{9.008921in}{3.157661in}}{\pgfqpoint{8.997871in}{3.157661in}}%
\pgfpathcurveto{\pgfqpoint{8.986821in}{3.157661in}}{\pgfqpoint{8.976222in}{3.153271in}}{\pgfqpoint{8.968408in}{3.145457in}}%
\pgfpathcurveto{\pgfqpoint{8.960595in}{3.137644in}}{\pgfqpoint{8.956204in}{3.127045in}}{\pgfqpoint{8.956204in}{3.115995in}}%
\pgfpathcurveto{\pgfqpoint{8.956204in}{3.104944in}}{\pgfqpoint{8.960595in}{3.094345in}}{\pgfqpoint{8.968408in}{3.086532in}}%
\pgfpathcurveto{\pgfqpoint{8.976222in}{3.078718in}}{\pgfqpoint{8.986821in}{3.074328in}}{\pgfqpoint{8.997871in}{3.074328in}}%
\pgfpathlineto{\pgfqpoint{8.997871in}{3.074328in}}%
\pgfpathclose%
\pgfusepath{stroke}%
\end{pgfscope}%
\begin{pgfscope}%
\pgfpathrectangle{\pgfqpoint{7.394209in}{0.375000in}}{\pgfqpoint{6.356833in}{5.175000in}}%
\pgfusepath{clip}%
\pgfsetbuttcap%
\pgfsetroundjoin%
\pgfsetlinewidth{1.003750pt}%
\definecolor{currentstroke}{rgb}{0.827451,0.827451,0.827451}%
\pgfsetstrokecolor{currentstroke}%
\pgfsetdash{}{0pt}%
\pgfpathmoveto{\pgfqpoint{9.100364in}{3.218330in}}%
\pgfpathcurveto{\pgfqpoint{9.111414in}{3.218330in}}{\pgfqpoint{9.122013in}{3.222720in}}{\pgfqpoint{9.129827in}{3.230534in}}%
\pgfpathcurveto{\pgfqpoint{9.137641in}{3.238347in}}{\pgfqpoint{9.142031in}{3.248946in}}{\pgfqpoint{9.142031in}{3.259996in}}%
\pgfpathcurveto{\pgfqpoint{9.142031in}{3.271046in}}{\pgfqpoint{9.137641in}{3.281645in}}{\pgfqpoint{9.129827in}{3.289459in}}%
\pgfpathcurveto{\pgfqpoint{9.122013in}{3.297273in}}{\pgfqpoint{9.111414in}{3.301663in}}{\pgfqpoint{9.100364in}{3.301663in}}%
\pgfpathcurveto{\pgfqpoint{9.089314in}{3.301663in}}{\pgfqpoint{9.078715in}{3.297273in}}{\pgfqpoint{9.070901in}{3.289459in}}%
\pgfpathcurveto{\pgfqpoint{9.063088in}{3.281645in}}{\pgfqpoint{9.058698in}{3.271046in}}{\pgfqpoint{9.058698in}{3.259996in}}%
\pgfpathcurveto{\pgfqpoint{9.058698in}{3.248946in}}{\pgfqpoint{9.063088in}{3.238347in}}{\pgfqpoint{9.070901in}{3.230534in}}%
\pgfpathcurveto{\pgfqpoint{9.078715in}{3.222720in}}{\pgfqpoint{9.089314in}{3.218330in}}{\pgfqpoint{9.100364in}{3.218330in}}%
\pgfpathlineto{\pgfqpoint{9.100364in}{3.218330in}}%
\pgfpathclose%
\pgfusepath{stroke}%
\end{pgfscope}%
\begin{pgfscope}%
\pgfpathrectangle{\pgfqpoint{7.394209in}{0.375000in}}{\pgfqpoint{6.356833in}{5.175000in}}%
\pgfusepath{clip}%
\pgfsetbuttcap%
\pgfsetroundjoin%
\pgfsetlinewidth{1.003750pt}%
\definecolor{currentstroke}{rgb}{0.827451,0.827451,0.827451}%
\pgfsetstrokecolor{currentstroke}%
\pgfsetdash{}{0pt}%
\pgfpathmoveto{\pgfqpoint{9.070004in}{3.491886in}}%
\pgfpathcurveto{\pgfqpoint{9.081054in}{3.491886in}}{\pgfqpoint{9.091653in}{3.496276in}}{\pgfqpoint{9.099466in}{3.504090in}}%
\pgfpathcurveto{\pgfqpoint{9.107280in}{3.511903in}}{\pgfqpoint{9.111670in}{3.522502in}}{\pgfqpoint{9.111670in}{3.533552in}}%
\pgfpathcurveto{\pgfqpoint{9.111670in}{3.544602in}}{\pgfqpoint{9.107280in}{3.555201in}}{\pgfqpoint{9.099466in}{3.563015in}}%
\pgfpathcurveto{\pgfqpoint{9.091653in}{3.570829in}}{\pgfqpoint{9.081054in}{3.575219in}}{\pgfqpoint{9.070004in}{3.575219in}}%
\pgfpathcurveto{\pgfqpoint{9.058953in}{3.575219in}}{\pgfqpoint{9.048354in}{3.570829in}}{\pgfqpoint{9.040541in}{3.563015in}}%
\pgfpathcurveto{\pgfqpoint{9.032727in}{3.555201in}}{\pgfqpoint{9.028337in}{3.544602in}}{\pgfqpoint{9.028337in}{3.533552in}}%
\pgfpathcurveto{\pgfqpoint{9.028337in}{3.522502in}}{\pgfqpoint{9.032727in}{3.511903in}}{\pgfqpoint{9.040541in}{3.504090in}}%
\pgfpathcurveto{\pgfqpoint{9.048354in}{3.496276in}}{\pgfqpoint{9.058953in}{3.491886in}}{\pgfqpoint{9.070004in}{3.491886in}}%
\pgfpathlineto{\pgfqpoint{9.070004in}{3.491886in}}%
\pgfpathclose%
\pgfusepath{stroke}%
\end{pgfscope}%
\begin{pgfscope}%
\pgfpathrectangle{\pgfqpoint{7.394209in}{0.375000in}}{\pgfqpoint{6.356833in}{5.175000in}}%
\pgfusepath{clip}%
\pgfsetbuttcap%
\pgfsetroundjoin%
\pgfsetlinewidth{1.003750pt}%
\definecolor{currentstroke}{rgb}{0.827451,0.827451,0.827451}%
\pgfsetstrokecolor{currentstroke}%
\pgfsetdash{}{0pt}%
\pgfpathmoveto{\pgfqpoint{11.384484in}{5.215918in}}%
\pgfpathcurveto{\pgfqpoint{11.395534in}{5.215918in}}{\pgfqpoint{11.406133in}{5.220308in}}{\pgfqpoint{11.413947in}{5.228122in}}%
\pgfpathcurveto{\pgfqpoint{11.421761in}{5.235935in}}{\pgfqpoint{11.426151in}{5.246534in}}{\pgfqpoint{11.426151in}{5.257584in}}%
\pgfpathcurveto{\pgfqpoint{11.426151in}{5.268634in}}{\pgfqpoint{11.421761in}{5.279234in}}{\pgfqpoint{11.413947in}{5.287047in}}%
\pgfpathcurveto{\pgfqpoint{11.406133in}{5.294861in}}{\pgfqpoint{11.395534in}{5.299251in}}{\pgfqpoint{11.384484in}{5.299251in}}%
\pgfpathcurveto{\pgfqpoint{11.373434in}{5.299251in}}{\pgfqpoint{11.362835in}{5.294861in}}{\pgfqpoint{11.355021in}{5.287047in}}%
\pgfpathcurveto{\pgfqpoint{11.347208in}{5.279234in}}{\pgfqpoint{11.342818in}{5.268634in}}{\pgfqpoint{11.342818in}{5.257584in}}%
\pgfpathcurveto{\pgfqpoint{11.342818in}{5.246534in}}{\pgfqpoint{11.347208in}{5.235935in}}{\pgfqpoint{11.355021in}{5.228122in}}%
\pgfpathcurveto{\pgfqpoint{11.362835in}{5.220308in}}{\pgfqpoint{11.373434in}{5.215918in}}{\pgfqpoint{11.384484in}{5.215918in}}%
\pgfpathlineto{\pgfqpoint{11.384484in}{5.215918in}}%
\pgfpathclose%
\pgfusepath{stroke}%
\end{pgfscope}%
\begin{pgfscope}%
\pgfpathrectangle{\pgfqpoint{7.394209in}{0.375000in}}{\pgfqpoint{6.356833in}{5.175000in}}%
\pgfusepath{clip}%
\pgfsetbuttcap%
\pgfsetroundjoin%
\pgfsetlinewidth{1.003750pt}%
\definecolor{currentstroke}{rgb}{0.827451,0.827451,0.827451}%
\pgfsetstrokecolor{currentstroke}%
\pgfsetdash{}{0pt}%
\pgfpathmoveto{\pgfqpoint{8.750870in}{1.267555in}}%
\pgfpathcurveto{\pgfqpoint{8.761920in}{1.267555in}}{\pgfqpoint{8.772519in}{1.271945in}}{\pgfqpoint{8.780332in}{1.279758in}}%
\pgfpathcurveto{\pgfqpoint{8.788146in}{1.287572in}}{\pgfqpoint{8.792536in}{1.298171in}}{\pgfqpoint{8.792536in}{1.309221in}}%
\pgfpathcurveto{\pgfqpoint{8.792536in}{1.320271in}}{\pgfqpoint{8.788146in}{1.330870in}}{\pgfqpoint{8.780332in}{1.338684in}}%
\pgfpathcurveto{\pgfqpoint{8.772519in}{1.346498in}}{\pgfqpoint{8.761920in}{1.350888in}}{\pgfqpoint{8.750870in}{1.350888in}}%
\pgfpathcurveto{\pgfqpoint{8.739820in}{1.350888in}}{\pgfqpoint{8.729221in}{1.346498in}}{\pgfqpoint{8.721407in}{1.338684in}}%
\pgfpathcurveto{\pgfqpoint{8.713593in}{1.330870in}}{\pgfqpoint{8.709203in}{1.320271in}}{\pgfqpoint{8.709203in}{1.309221in}}%
\pgfpathcurveto{\pgfqpoint{8.709203in}{1.298171in}}{\pgfqpoint{8.713593in}{1.287572in}}{\pgfqpoint{8.721407in}{1.279758in}}%
\pgfpathcurveto{\pgfqpoint{8.729221in}{1.271945in}}{\pgfqpoint{8.739820in}{1.267555in}}{\pgfqpoint{8.750870in}{1.267555in}}%
\pgfpathlineto{\pgfqpoint{8.750870in}{1.267555in}}%
\pgfpathclose%
\pgfusepath{stroke}%
\end{pgfscope}%
\begin{pgfscope}%
\pgfpathrectangle{\pgfqpoint{7.394209in}{0.375000in}}{\pgfqpoint{6.356833in}{5.175000in}}%
\pgfusepath{clip}%
\pgfsetbuttcap%
\pgfsetroundjoin%
\pgfsetlinewidth{1.003750pt}%
\definecolor{currentstroke}{rgb}{0.827451,0.827451,0.827451}%
\pgfsetstrokecolor{currentstroke}%
\pgfsetdash{}{0pt}%
\pgfpathmoveto{\pgfqpoint{11.611576in}{5.257520in}}%
\pgfpathcurveto{\pgfqpoint{11.622626in}{5.257520in}}{\pgfqpoint{11.633225in}{5.261910in}}{\pgfqpoint{11.641039in}{5.269724in}}%
\pgfpathcurveto{\pgfqpoint{11.648853in}{5.277538in}}{\pgfqpoint{11.653243in}{5.288137in}}{\pgfqpoint{11.653243in}{5.299187in}}%
\pgfpathcurveto{\pgfqpoint{11.653243in}{5.310237in}}{\pgfqpoint{11.648853in}{5.320836in}}{\pgfqpoint{11.641039in}{5.328649in}}%
\pgfpathcurveto{\pgfqpoint{11.633225in}{5.336463in}}{\pgfqpoint{11.622626in}{5.340853in}}{\pgfqpoint{11.611576in}{5.340853in}}%
\pgfpathcurveto{\pgfqpoint{11.600526in}{5.340853in}}{\pgfqpoint{11.589927in}{5.336463in}}{\pgfqpoint{11.582113in}{5.328649in}}%
\pgfpathcurveto{\pgfqpoint{11.574300in}{5.320836in}}{\pgfqpoint{11.569909in}{5.310237in}}{\pgfqpoint{11.569909in}{5.299187in}}%
\pgfpathcurveto{\pgfqpoint{11.569909in}{5.288137in}}{\pgfqpoint{11.574300in}{5.277538in}}{\pgfqpoint{11.582113in}{5.269724in}}%
\pgfpathcurveto{\pgfqpoint{11.589927in}{5.261910in}}{\pgfqpoint{11.600526in}{5.257520in}}{\pgfqpoint{11.611576in}{5.257520in}}%
\pgfpathlineto{\pgfqpoint{11.611576in}{5.257520in}}%
\pgfpathclose%
\pgfusepath{stroke}%
\end{pgfscope}%
\begin{pgfscope}%
\pgfpathrectangle{\pgfqpoint{7.394209in}{0.375000in}}{\pgfqpoint{6.356833in}{5.175000in}}%
\pgfusepath{clip}%
\pgfsetbuttcap%
\pgfsetroundjoin%
\pgfsetlinewidth{1.003750pt}%
\definecolor{currentstroke}{rgb}{0.827451,0.827451,0.827451}%
\pgfsetstrokecolor{currentstroke}%
\pgfsetdash{}{0pt}%
\pgfpathmoveto{\pgfqpoint{8.742298in}{1.821548in}}%
\pgfpathcurveto{\pgfqpoint{8.753348in}{1.821548in}}{\pgfqpoint{8.763947in}{1.825938in}}{\pgfqpoint{8.771761in}{1.833751in}}%
\pgfpathcurveto{\pgfqpoint{8.779574in}{1.841565in}}{\pgfqpoint{8.783965in}{1.852164in}}{\pgfqpoint{8.783965in}{1.863214in}}%
\pgfpathcurveto{\pgfqpoint{8.783965in}{1.874264in}}{\pgfqpoint{8.779574in}{1.884863in}}{\pgfqpoint{8.771761in}{1.892677in}}%
\pgfpathcurveto{\pgfqpoint{8.763947in}{1.900491in}}{\pgfqpoint{8.753348in}{1.904881in}}{\pgfqpoint{8.742298in}{1.904881in}}%
\pgfpathcurveto{\pgfqpoint{8.731248in}{1.904881in}}{\pgfqpoint{8.720649in}{1.900491in}}{\pgfqpoint{8.712835in}{1.892677in}}%
\pgfpathcurveto{\pgfqpoint{8.705022in}{1.884863in}}{\pgfqpoint{8.700631in}{1.874264in}}{\pgfqpoint{8.700631in}{1.863214in}}%
\pgfpathcurveto{\pgfqpoint{8.700631in}{1.852164in}}{\pgfqpoint{8.705022in}{1.841565in}}{\pgfqpoint{8.712835in}{1.833751in}}%
\pgfpathcurveto{\pgfqpoint{8.720649in}{1.825938in}}{\pgfqpoint{8.731248in}{1.821548in}}{\pgfqpoint{8.742298in}{1.821548in}}%
\pgfpathlineto{\pgfqpoint{8.742298in}{1.821548in}}%
\pgfpathclose%
\pgfusepath{stroke}%
\end{pgfscope}%
\begin{pgfscope}%
\pgfpathrectangle{\pgfqpoint{7.394209in}{0.375000in}}{\pgfqpoint{6.356833in}{5.175000in}}%
\pgfusepath{clip}%
\pgfsetbuttcap%
\pgfsetroundjoin%
\pgfsetlinewidth{1.003750pt}%
\definecolor{currentstroke}{rgb}{0.827451,0.827451,0.827451}%
\pgfsetstrokecolor{currentstroke}%
\pgfsetdash{}{0pt}%
\pgfpathmoveto{\pgfqpoint{10.376594in}{5.371564in}}%
\pgfpathcurveto{\pgfqpoint{10.387644in}{5.371564in}}{\pgfqpoint{10.398243in}{5.375954in}}{\pgfqpoint{10.406057in}{5.383768in}}%
\pgfpathcurveto{\pgfqpoint{10.413871in}{5.391581in}}{\pgfqpoint{10.418261in}{5.402180in}}{\pgfqpoint{10.418261in}{5.413231in}}%
\pgfpathcurveto{\pgfqpoint{10.418261in}{5.424281in}}{\pgfqpoint{10.413871in}{5.434880in}}{\pgfqpoint{10.406057in}{5.442693in}}%
\pgfpathcurveto{\pgfqpoint{10.398243in}{5.450507in}}{\pgfqpoint{10.387644in}{5.454897in}}{\pgfqpoint{10.376594in}{5.454897in}}%
\pgfpathcurveto{\pgfqpoint{10.365544in}{5.454897in}}{\pgfqpoint{10.354945in}{5.450507in}}{\pgfqpoint{10.347132in}{5.442693in}}%
\pgfpathcurveto{\pgfqpoint{10.339318in}{5.434880in}}{\pgfqpoint{10.334928in}{5.424281in}}{\pgfqpoint{10.334928in}{5.413231in}}%
\pgfpathcurveto{\pgfqpoint{10.334928in}{5.402180in}}{\pgfqpoint{10.339318in}{5.391581in}}{\pgfqpoint{10.347132in}{5.383768in}}%
\pgfpathcurveto{\pgfqpoint{10.354945in}{5.375954in}}{\pgfqpoint{10.365544in}{5.371564in}}{\pgfqpoint{10.376594in}{5.371564in}}%
\pgfpathlineto{\pgfqpoint{10.376594in}{5.371564in}}%
\pgfpathclose%
\pgfusepath{stroke}%
\end{pgfscope}%
\begin{pgfscope}%
\pgfpathrectangle{\pgfqpoint{7.394209in}{0.375000in}}{\pgfqpoint{6.356833in}{5.175000in}}%
\pgfusepath{clip}%
\pgfsetbuttcap%
\pgfsetroundjoin%
\pgfsetlinewidth{1.003750pt}%
\definecolor{currentstroke}{rgb}{0.827451,0.827451,0.827451}%
\pgfsetstrokecolor{currentstroke}%
\pgfsetdash{}{0pt}%
\pgfpathmoveto{\pgfqpoint{7.928696in}{1.639544in}}%
\pgfpathcurveto{\pgfqpoint{7.939746in}{1.639544in}}{\pgfqpoint{7.950345in}{1.643934in}}{\pgfqpoint{7.958159in}{1.651748in}}%
\pgfpathcurveto{\pgfqpoint{7.965973in}{1.659561in}}{\pgfqpoint{7.970363in}{1.670161in}}{\pgfqpoint{7.970363in}{1.681211in}}%
\pgfpathcurveto{\pgfqpoint{7.970363in}{1.692261in}}{\pgfqpoint{7.965973in}{1.702860in}}{\pgfqpoint{7.958159in}{1.710673in}}%
\pgfpathcurveto{\pgfqpoint{7.950345in}{1.718487in}}{\pgfqpoint{7.939746in}{1.722877in}}{\pgfqpoint{7.928696in}{1.722877in}}%
\pgfpathcurveto{\pgfqpoint{7.917646in}{1.722877in}}{\pgfqpoint{7.907047in}{1.718487in}}{\pgfqpoint{7.899233in}{1.710673in}}%
\pgfpathcurveto{\pgfqpoint{7.891420in}{1.702860in}}{\pgfqpoint{7.887030in}{1.692261in}}{\pgfqpoint{7.887030in}{1.681211in}}%
\pgfpathcurveto{\pgfqpoint{7.887030in}{1.670161in}}{\pgfqpoint{7.891420in}{1.659561in}}{\pgfqpoint{7.899233in}{1.651748in}}%
\pgfpathcurveto{\pgfqpoint{7.907047in}{1.643934in}}{\pgfqpoint{7.917646in}{1.639544in}}{\pgfqpoint{7.928696in}{1.639544in}}%
\pgfpathlineto{\pgfqpoint{7.928696in}{1.639544in}}%
\pgfpathclose%
\pgfusepath{stroke}%
\end{pgfscope}%
\begin{pgfscope}%
\pgfpathrectangle{\pgfqpoint{7.394209in}{0.375000in}}{\pgfqpoint{6.356833in}{5.175000in}}%
\pgfusepath{clip}%
\pgfsetbuttcap%
\pgfsetroundjoin%
\pgfsetlinewidth{1.003750pt}%
\definecolor{currentstroke}{rgb}{0.827451,0.827451,0.827451}%
\pgfsetstrokecolor{currentstroke}%
\pgfsetdash{}{0pt}%
\pgfpathmoveto{\pgfqpoint{9.168638in}{2.136741in}}%
\pgfpathcurveto{\pgfqpoint{9.179688in}{2.136741in}}{\pgfqpoint{9.190287in}{2.141131in}}{\pgfqpoint{9.198100in}{2.148945in}}%
\pgfpathcurveto{\pgfqpoint{9.205914in}{2.156759in}}{\pgfqpoint{9.210304in}{2.167358in}}{\pgfqpoint{9.210304in}{2.178408in}}%
\pgfpathcurveto{\pgfqpoint{9.210304in}{2.189458in}}{\pgfqpoint{9.205914in}{2.200057in}}{\pgfqpoint{9.198100in}{2.207871in}}%
\pgfpathcurveto{\pgfqpoint{9.190287in}{2.215684in}}{\pgfqpoint{9.179688in}{2.220074in}}{\pgfqpoint{9.168638in}{2.220074in}}%
\pgfpathcurveto{\pgfqpoint{9.157588in}{2.220074in}}{\pgfqpoint{9.146989in}{2.215684in}}{\pgfqpoint{9.139175in}{2.207871in}}%
\pgfpathcurveto{\pgfqpoint{9.131361in}{2.200057in}}{\pgfqpoint{9.126971in}{2.189458in}}{\pgfqpoint{9.126971in}{2.178408in}}%
\pgfpathcurveto{\pgfqpoint{9.126971in}{2.167358in}}{\pgfqpoint{9.131361in}{2.156759in}}{\pgfqpoint{9.139175in}{2.148945in}}%
\pgfpathcurveto{\pgfqpoint{9.146989in}{2.141131in}}{\pgfqpoint{9.157588in}{2.136741in}}{\pgfqpoint{9.168638in}{2.136741in}}%
\pgfpathlineto{\pgfqpoint{9.168638in}{2.136741in}}%
\pgfpathclose%
\pgfusepath{stroke}%
\end{pgfscope}%
\begin{pgfscope}%
\pgfpathrectangle{\pgfqpoint{7.394209in}{0.375000in}}{\pgfqpoint{6.356833in}{5.175000in}}%
\pgfusepath{clip}%
\pgfsetbuttcap%
\pgfsetroundjoin%
\pgfsetlinewidth{1.003750pt}%
\definecolor{currentstroke}{rgb}{0.827451,0.827451,0.827451}%
\pgfsetstrokecolor{currentstroke}%
\pgfsetdash{}{0pt}%
\pgfpathmoveto{\pgfqpoint{7.756898in}{1.392572in}}%
\pgfpathcurveto{\pgfqpoint{7.767948in}{1.392572in}}{\pgfqpoint{7.778547in}{1.396962in}}{\pgfqpoint{7.786361in}{1.404775in}}%
\pgfpathcurveto{\pgfqpoint{7.794175in}{1.412589in}}{\pgfqpoint{7.798565in}{1.423188in}}{\pgfqpoint{7.798565in}{1.434238in}}%
\pgfpathcurveto{\pgfqpoint{7.798565in}{1.445288in}}{\pgfqpoint{7.794175in}{1.455887in}}{\pgfqpoint{7.786361in}{1.463701in}}%
\pgfpathcurveto{\pgfqpoint{7.778547in}{1.471515in}}{\pgfqpoint{7.767948in}{1.475905in}}{\pgfqpoint{7.756898in}{1.475905in}}%
\pgfpathcurveto{\pgfqpoint{7.745848in}{1.475905in}}{\pgfqpoint{7.735249in}{1.471515in}}{\pgfqpoint{7.727435in}{1.463701in}}%
\pgfpathcurveto{\pgfqpoint{7.719622in}{1.455887in}}{\pgfqpoint{7.715232in}{1.445288in}}{\pgfqpoint{7.715232in}{1.434238in}}%
\pgfpathcurveto{\pgfqpoint{7.715232in}{1.423188in}}{\pgfqpoint{7.719622in}{1.412589in}}{\pgfqpoint{7.727435in}{1.404775in}}%
\pgfpathcurveto{\pgfqpoint{7.735249in}{1.396962in}}{\pgfqpoint{7.745848in}{1.392572in}}{\pgfqpoint{7.756898in}{1.392572in}}%
\pgfpathlineto{\pgfqpoint{7.756898in}{1.392572in}}%
\pgfpathclose%
\pgfusepath{stroke}%
\end{pgfscope}%
\begin{pgfscope}%
\pgfpathrectangle{\pgfqpoint{7.394209in}{0.375000in}}{\pgfqpoint{6.356833in}{5.175000in}}%
\pgfusepath{clip}%
\pgfsetbuttcap%
\pgfsetroundjoin%
\pgfsetlinewidth{1.003750pt}%
\definecolor{currentstroke}{rgb}{0.827451,0.827451,0.827451}%
\pgfsetstrokecolor{currentstroke}%
\pgfsetdash{}{0pt}%
\pgfpathmoveto{\pgfqpoint{9.865298in}{5.356069in}}%
\pgfpathcurveto{\pgfqpoint{9.876348in}{5.356069in}}{\pgfqpoint{9.886947in}{5.360459in}}{\pgfqpoint{9.894760in}{5.368273in}}%
\pgfpathcurveto{\pgfqpoint{9.902574in}{5.376086in}}{\pgfqpoint{9.906964in}{5.386686in}}{\pgfqpoint{9.906964in}{5.397736in}}%
\pgfpathcurveto{\pgfqpoint{9.906964in}{5.408786in}}{\pgfqpoint{9.902574in}{5.419385in}}{\pgfqpoint{9.894760in}{5.427198in}}%
\pgfpathcurveto{\pgfqpoint{9.886947in}{5.435012in}}{\pgfqpoint{9.876348in}{5.439402in}}{\pgfqpoint{9.865298in}{5.439402in}}%
\pgfpathcurveto{\pgfqpoint{9.854248in}{5.439402in}}{\pgfqpoint{9.843649in}{5.435012in}}{\pgfqpoint{9.835835in}{5.427198in}}%
\pgfpathcurveto{\pgfqpoint{9.828021in}{5.419385in}}{\pgfqpoint{9.823631in}{5.408786in}}{\pgfqpoint{9.823631in}{5.397736in}}%
\pgfpathcurveto{\pgfqpoint{9.823631in}{5.386686in}}{\pgfqpoint{9.828021in}{5.376086in}}{\pgfqpoint{9.835835in}{5.368273in}}%
\pgfpathcurveto{\pgfqpoint{9.843649in}{5.360459in}}{\pgfqpoint{9.854248in}{5.356069in}}{\pgfqpoint{9.865298in}{5.356069in}}%
\pgfpathlineto{\pgfqpoint{9.865298in}{5.356069in}}%
\pgfpathclose%
\pgfusepath{stroke}%
\end{pgfscope}%
\begin{pgfscope}%
\pgfpathrectangle{\pgfqpoint{7.394209in}{0.375000in}}{\pgfqpoint{6.356833in}{5.175000in}}%
\pgfusepath{clip}%
\pgfsetbuttcap%
\pgfsetroundjoin%
\pgfsetlinewidth{1.003750pt}%
\definecolor{currentstroke}{rgb}{0.827451,0.827451,0.827451}%
\pgfsetstrokecolor{currentstroke}%
\pgfsetdash{}{0pt}%
\pgfpathmoveto{\pgfqpoint{7.947835in}{1.639544in}}%
\pgfpathcurveto{\pgfqpoint{7.958886in}{1.639544in}}{\pgfqpoint{7.969485in}{1.643934in}}{\pgfqpoint{7.977298in}{1.651748in}}%
\pgfpathcurveto{\pgfqpoint{7.985112in}{1.659561in}}{\pgfqpoint{7.989502in}{1.670161in}}{\pgfqpoint{7.989502in}{1.681211in}}%
\pgfpathcurveto{\pgfqpoint{7.989502in}{1.692261in}}{\pgfqpoint{7.985112in}{1.702860in}}{\pgfqpoint{7.977298in}{1.710673in}}%
\pgfpathcurveto{\pgfqpoint{7.969485in}{1.718487in}}{\pgfqpoint{7.958886in}{1.722877in}}{\pgfqpoint{7.947835in}{1.722877in}}%
\pgfpathcurveto{\pgfqpoint{7.936785in}{1.722877in}}{\pgfqpoint{7.926186in}{1.718487in}}{\pgfqpoint{7.918373in}{1.710673in}}%
\pgfpathcurveto{\pgfqpoint{7.910559in}{1.702860in}}{\pgfqpoint{7.906169in}{1.692261in}}{\pgfqpoint{7.906169in}{1.681211in}}%
\pgfpathcurveto{\pgfqpoint{7.906169in}{1.670161in}}{\pgfqpoint{7.910559in}{1.659561in}}{\pgfqpoint{7.918373in}{1.651748in}}%
\pgfpathcurveto{\pgfqpoint{7.926186in}{1.643934in}}{\pgfqpoint{7.936785in}{1.639544in}}{\pgfqpoint{7.947835in}{1.639544in}}%
\pgfpathlineto{\pgfqpoint{7.947835in}{1.639544in}}%
\pgfpathclose%
\pgfusepath{stroke}%
\end{pgfscope}%
\begin{pgfscope}%
\pgfpathrectangle{\pgfqpoint{7.394209in}{0.375000in}}{\pgfqpoint{6.356833in}{5.175000in}}%
\pgfusepath{clip}%
\pgfsetbuttcap%
\pgfsetroundjoin%
\pgfsetlinewidth{1.003750pt}%
\definecolor{currentstroke}{rgb}{0.827451,0.827451,0.827451}%
\pgfsetstrokecolor{currentstroke}%
\pgfsetdash{}{0pt}%
\pgfpathmoveto{\pgfqpoint{10.412514in}{5.012719in}}%
\pgfpathcurveto{\pgfqpoint{10.423564in}{5.012719in}}{\pgfqpoint{10.434163in}{5.017109in}}{\pgfqpoint{10.441977in}{5.024923in}}%
\pgfpathcurveto{\pgfqpoint{10.449791in}{5.032736in}}{\pgfqpoint{10.454181in}{5.043335in}}{\pgfqpoint{10.454181in}{5.054385in}}%
\pgfpathcurveto{\pgfqpoint{10.454181in}{5.065436in}}{\pgfqpoint{10.449791in}{5.076035in}}{\pgfqpoint{10.441977in}{5.083848in}}%
\pgfpathcurveto{\pgfqpoint{10.434163in}{5.091662in}}{\pgfqpoint{10.423564in}{5.096052in}}{\pgfqpoint{10.412514in}{5.096052in}}%
\pgfpathcurveto{\pgfqpoint{10.401464in}{5.096052in}}{\pgfqpoint{10.390865in}{5.091662in}}{\pgfqpoint{10.383051in}{5.083848in}}%
\pgfpathcurveto{\pgfqpoint{10.375238in}{5.076035in}}{\pgfqpoint{10.370847in}{5.065436in}}{\pgfqpoint{10.370847in}{5.054385in}}%
\pgfpathcurveto{\pgfqpoint{10.370847in}{5.043335in}}{\pgfqpoint{10.375238in}{5.032736in}}{\pgfqpoint{10.383051in}{5.024923in}}%
\pgfpathcurveto{\pgfqpoint{10.390865in}{5.017109in}}{\pgfqpoint{10.401464in}{5.012719in}}{\pgfqpoint{10.412514in}{5.012719in}}%
\pgfpathlineto{\pgfqpoint{10.412514in}{5.012719in}}%
\pgfpathclose%
\pgfusepath{stroke}%
\end{pgfscope}%
\begin{pgfscope}%
\pgfpathrectangle{\pgfqpoint{7.394209in}{0.375000in}}{\pgfqpoint{6.356833in}{5.175000in}}%
\pgfusepath{clip}%
\pgfsetbuttcap%
\pgfsetroundjoin%
\pgfsetlinewidth{1.003750pt}%
\definecolor{currentstroke}{rgb}{0.827451,0.827451,0.827451}%
\pgfsetstrokecolor{currentstroke}%
\pgfsetdash{}{0pt}%
\pgfpathmoveto{\pgfqpoint{8.591271in}{1.267555in}}%
\pgfpathcurveto{\pgfqpoint{8.602321in}{1.267555in}}{\pgfqpoint{8.612920in}{1.271945in}}{\pgfqpoint{8.620734in}{1.279758in}}%
\pgfpathcurveto{\pgfqpoint{8.628548in}{1.287572in}}{\pgfqpoint{8.632938in}{1.298171in}}{\pgfqpoint{8.632938in}{1.309221in}}%
\pgfpathcurveto{\pgfqpoint{8.632938in}{1.320271in}}{\pgfqpoint{8.628548in}{1.330870in}}{\pgfqpoint{8.620734in}{1.338684in}}%
\pgfpathcurveto{\pgfqpoint{8.612920in}{1.346498in}}{\pgfqpoint{8.602321in}{1.350888in}}{\pgfqpoint{8.591271in}{1.350888in}}%
\pgfpathcurveto{\pgfqpoint{8.580221in}{1.350888in}}{\pgfqpoint{8.569622in}{1.346498in}}{\pgfqpoint{8.561808in}{1.338684in}}%
\pgfpathcurveto{\pgfqpoint{8.553995in}{1.330870in}}{\pgfqpoint{8.549605in}{1.320271in}}{\pgfqpoint{8.549605in}{1.309221in}}%
\pgfpathcurveto{\pgfqpoint{8.549605in}{1.298171in}}{\pgfqpoint{8.553995in}{1.287572in}}{\pgfqpoint{8.561808in}{1.279758in}}%
\pgfpathcurveto{\pgfqpoint{8.569622in}{1.271945in}}{\pgfqpoint{8.580221in}{1.267555in}}{\pgfqpoint{8.591271in}{1.267555in}}%
\pgfpathlineto{\pgfqpoint{8.591271in}{1.267555in}}%
\pgfpathclose%
\pgfusepath{stroke}%
\end{pgfscope}%
\begin{pgfscope}%
\pgfpathrectangle{\pgfqpoint{7.394209in}{0.375000in}}{\pgfqpoint{6.356833in}{5.175000in}}%
\pgfusepath{clip}%
\pgfsetbuttcap%
\pgfsetroundjoin%
\pgfsetlinewidth{1.003750pt}%
\definecolor{currentstroke}{rgb}{0.827451,0.827451,0.827451}%
\pgfsetstrokecolor{currentstroke}%
\pgfsetdash{}{0pt}%
\pgfpathmoveto{\pgfqpoint{8.778512in}{1.914910in}}%
\pgfpathcurveto{\pgfqpoint{8.789563in}{1.914910in}}{\pgfqpoint{8.800162in}{1.919300in}}{\pgfqpoint{8.807975in}{1.927114in}}%
\pgfpathcurveto{\pgfqpoint{8.815789in}{1.934928in}}{\pgfqpoint{8.820179in}{1.945527in}}{\pgfqpoint{8.820179in}{1.956577in}}%
\pgfpathcurveto{\pgfqpoint{8.820179in}{1.967627in}}{\pgfqpoint{8.815789in}{1.978226in}}{\pgfqpoint{8.807975in}{1.986040in}}%
\pgfpathcurveto{\pgfqpoint{8.800162in}{1.993853in}}{\pgfqpoint{8.789563in}{1.998243in}}{\pgfqpoint{8.778512in}{1.998243in}}%
\pgfpathcurveto{\pgfqpoint{8.767462in}{1.998243in}}{\pgfqpoint{8.756863in}{1.993853in}}{\pgfqpoint{8.749050in}{1.986040in}}%
\pgfpathcurveto{\pgfqpoint{8.741236in}{1.978226in}}{\pgfqpoint{8.736846in}{1.967627in}}{\pgfqpoint{8.736846in}{1.956577in}}%
\pgfpathcurveto{\pgfqpoint{8.736846in}{1.945527in}}{\pgfqpoint{8.741236in}{1.934928in}}{\pgfqpoint{8.749050in}{1.927114in}}%
\pgfpathcurveto{\pgfqpoint{8.756863in}{1.919300in}}{\pgfqpoint{8.767462in}{1.914910in}}{\pgfqpoint{8.778512in}{1.914910in}}%
\pgfpathlineto{\pgfqpoint{8.778512in}{1.914910in}}%
\pgfpathclose%
\pgfusepath{stroke}%
\end{pgfscope}%
\begin{pgfscope}%
\pgfpathrectangle{\pgfqpoint{7.394209in}{0.375000in}}{\pgfqpoint{6.356833in}{5.175000in}}%
\pgfusepath{clip}%
\pgfsetbuttcap%
\pgfsetroundjoin%
\pgfsetlinewidth{1.003750pt}%
\definecolor{currentstroke}{rgb}{0.827451,0.827451,0.827451}%
\pgfsetstrokecolor{currentstroke}%
\pgfsetdash{}{0pt}%
\pgfpathmoveto{\pgfqpoint{8.536257in}{2.550576in}}%
\pgfpathcurveto{\pgfqpoint{8.547307in}{2.550576in}}{\pgfqpoint{8.557906in}{2.554966in}}{\pgfqpoint{8.565720in}{2.562780in}}%
\pgfpathcurveto{\pgfqpoint{8.573534in}{2.570594in}}{\pgfqpoint{8.577924in}{2.581193in}}{\pgfqpoint{8.577924in}{2.592243in}}%
\pgfpathcurveto{\pgfqpoint{8.577924in}{2.603293in}}{\pgfqpoint{8.573534in}{2.613892in}}{\pgfqpoint{8.565720in}{2.621705in}}%
\pgfpathcurveto{\pgfqpoint{8.557906in}{2.629519in}}{\pgfqpoint{8.547307in}{2.633909in}}{\pgfqpoint{8.536257in}{2.633909in}}%
\pgfpathcurveto{\pgfqpoint{8.525207in}{2.633909in}}{\pgfqpoint{8.514608in}{2.629519in}}{\pgfqpoint{8.506794in}{2.621705in}}%
\pgfpathcurveto{\pgfqpoint{8.498981in}{2.613892in}}{\pgfqpoint{8.494591in}{2.603293in}}{\pgfqpoint{8.494591in}{2.592243in}}%
\pgfpathcurveto{\pgfqpoint{8.494591in}{2.581193in}}{\pgfqpoint{8.498981in}{2.570594in}}{\pgfqpoint{8.506794in}{2.562780in}}%
\pgfpathcurveto{\pgfqpoint{8.514608in}{2.554966in}}{\pgfqpoint{8.525207in}{2.550576in}}{\pgfqpoint{8.536257in}{2.550576in}}%
\pgfpathlineto{\pgfqpoint{8.536257in}{2.550576in}}%
\pgfpathclose%
\pgfusepath{stroke}%
\end{pgfscope}%
\begin{pgfscope}%
\pgfpathrectangle{\pgfqpoint{7.394209in}{0.375000in}}{\pgfqpoint{6.356833in}{5.175000in}}%
\pgfusepath{clip}%
\pgfsetbuttcap%
\pgfsetroundjoin%
\pgfsetlinewidth{1.003750pt}%
\definecolor{currentstroke}{rgb}{0.827451,0.827451,0.827451}%
\pgfsetstrokecolor{currentstroke}%
\pgfsetdash{}{0pt}%
\pgfpathmoveto{\pgfqpoint{8.142966in}{0.511085in}}%
\pgfpathcurveto{\pgfqpoint{8.154016in}{0.511085in}}{\pgfqpoint{8.164615in}{0.515476in}}{\pgfqpoint{8.172429in}{0.523289in}}%
\pgfpathcurveto{\pgfqpoint{8.180243in}{0.531103in}}{\pgfqpoint{8.184633in}{0.541702in}}{\pgfqpoint{8.184633in}{0.552752in}}%
\pgfpathcurveto{\pgfqpoint{8.184633in}{0.563802in}}{\pgfqpoint{8.180243in}{0.574401in}}{\pgfqpoint{8.172429in}{0.582215in}}%
\pgfpathcurveto{\pgfqpoint{8.164615in}{0.590028in}}{\pgfqpoint{8.154016in}{0.594419in}}{\pgfqpoint{8.142966in}{0.594419in}}%
\pgfpathcurveto{\pgfqpoint{8.131916in}{0.594419in}}{\pgfqpoint{8.121317in}{0.590028in}}{\pgfqpoint{8.113503in}{0.582215in}}%
\pgfpathcurveto{\pgfqpoint{8.105690in}{0.574401in}}{\pgfqpoint{8.101300in}{0.563802in}}{\pgfqpoint{8.101300in}{0.552752in}}%
\pgfpathcurveto{\pgfqpoint{8.101300in}{0.541702in}}{\pgfqpoint{8.105690in}{0.531103in}}{\pgfqpoint{8.113503in}{0.523289in}}%
\pgfpathcurveto{\pgfqpoint{8.121317in}{0.515476in}}{\pgfqpoint{8.131916in}{0.511085in}}{\pgfqpoint{8.142966in}{0.511085in}}%
\pgfpathlineto{\pgfqpoint{8.142966in}{0.511085in}}%
\pgfpathclose%
\pgfusepath{stroke}%
\end{pgfscope}%
\begin{pgfscope}%
\pgfpathrectangle{\pgfqpoint{7.394209in}{0.375000in}}{\pgfqpoint{6.356833in}{5.175000in}}%
\pgfusepath{clip}%
\pgfsetbuttcap%
\pgfsetroundjoin%
\pgfsetlinewidth{1.003750pt}%
\definecolor{currentstroke}{rgb}{0.827451,0.827451,0.827451}%
\pgfsetstrokecolor{currentstroke}%
\pgfsetdash{}{0pt}%
\pgfpathmoveto{\pgfqpoint{7.416320in}{0.514864in}}%
\pgfpathcurveto{\pgfqpoint{7.427370in}{0.514864in}}{\pgfqpoint{7.437969in}{0.519254in}}{\pgfqpoint{7.445783in}{0.527068in}}%
\pgfpathcurveto{\pgfqpoint{7.453596in}{0.534882in}}{\pgfqpoint{7.457987in}{0.545481in}}{\pgfqpoint{7.457987in}{0.556531in}}%
\pgfpathcurveto{\pgfqpoint{7.457987in}{0.567581in}}{\pgfqpoint{7.453596in}{0.578180in}}{\pgfqpoint{7.445783in}{0.585994in}}%
\pgfpathcurveto{\pgfqpoint{7.437969in}{0.593807in}}{\pgfqpoint{7.427370in}{0.598198in}}{\pgfqpoint{7.416320in}{0.598198in}}%
\pgfpathcurveto{\pgfqpoint{7.405270in}{0.598198in}}{\pgfqpoint{7.394671in}{0.593807in}}{\pgfqpoint{7.386857in}{0.585994in}}%
\pgfpathcurveto{\pgfqpoint{7.379044in}{0.578180in}}{\pgfqpoint{7.374653in}{0.567581in}}{\pgfqpoint{7.374653in}{0.556531in}}%
\pgfpathcurveto{\pgfqpoint{7.374653in}{0.545481in}}{\pgfqpoint{7.379044in}{0.534882in}}{\pgfqpoint{7.386857in}{0.527068in}}%
\pgfpathcurveto{\pgfqpoint{7.394671in}{0.519254in}}{\pgfqpoint{7.405270in}{0.514864in}}{\pgfqpoint{7.416320in}{0.514864in}}%
\pgfpathlineto{\pgfqpoint{7.416320in}{0.514864in}}%
\pgfpathclose%
\pgfusepath{stroke}%
\end{pgfscope}%
\begin{pgfscope}%
\pgfpathrectangle{\pgfqpoint{7.394209in}{0.375000in}}{\pgfqpoint{6.356833in}{5.175000in}}%
\pgfusepath{clip}%
\pgfsetbuttcap%
\pgfsetroundjoin%
\pgfsetlinewidth{1.003750pt}%
\definecolor{currentstroke}{rgb}{0.827451,0.827451,0.827451}%
\pgfsetstrokecolor{currentstroke}%
\pgfsetdash{}{0pt}%
\pgfpathmoveto{\pgfqpoint{13.039385in}{5.482640in}}%
\pgfpathcurveto{\pgfqpoint{13.050435in}{5.482640in}}{\pgfqpoint{13.061034in}{5.487031in}}{\pgfqpoint{13.068848in}{5.494844in}}%
\pgfpathcurveto{\pgfqpoint{13.076662in}{5.502658in}}{\pgfqpoint{13.081052in}{5.513257in}}{\pgfqpoint{13.081052in}{5.524307in}}%
\pgfpathcurveto{\pgfqpoint{13.081052in}{5.535357in}}{\pgfqpoint{13.076662in}{5.545956in}}{\pgfqpoint{13.068848in}{5.553770in}}%
\pgfpathcurveto{\pgfqpoint{13.061034in}{5.561584in}}{\pgfqpoint{13.050435in}{5.565974in}}{\pgfqpoint{13.039385in}{5.565974in}}%
\pgfpathcurveto{\pgfqpoint{13.028335in}{5.565974in}}{\pgfqpoint{13.017736in}{5.561584in}}{\pgfqpoint{13.009922in}{5.553770in}}%
\pgfpathcurveto{\pgfqpoint{13.002109in}{5.545956in}}{\pgfqpoint{12.997718in}{5.535357in}}{\pgfqpoint{12.997718in}{5.524307in}}%
\pgfpathcurveto{\pgfqpoint{12.997718in}{5.513257in}}{\pgfqpoint{13.002109in}{5.502658in}}{\pgfqpoint{13.009922in}{5.494844in}}%
\pgfpathcurveto{\pgfqpoint{13.017736in}{5.487031in}}{\pgfqpoint{13.028335in}{5.482640in}}{\pgfqpoint{13.039385in}{5.482640in}}%
\pgfpathlineto{\pgfqpoint{13.039385in}{5.482640in}}%
\pgfpathclose%
\pgfusepath{stroke}%
\end{pgfscope}%
\begin{pgfscope}%
\pgfpathrectangle{\pgfqpoint{7.394209in}{0.375000in}}{\pgfqpoint{6.356833in}{5.175000in}}%
\pgfusepath{clip}%
\pgfsetbuttcap%
\pgfsetroundjoin%
\pgfsetlinewidth{1.003750pt}%
\definecolor{currentstroke}{rgb}{0.827451,0.827451,0.827451}%
\pgfsetstrokecolor{currentstroke}%
\pgfsetdash{}{0pt}%
\pgfpathmoveto{\pgfqpoint{7.934823in}{0.788369in}}%
\pgfpathcurveto{\pgfqpoint{7.945873in}{0.788369in}}{\pgfqpoint{7.956472in}{0.792759in}}{\pgfqpoint{7.964286in}{0.800573in}}%
\pgfpathcurveto{\pgfqpoint{7.972100in}{0.808386in}}{\pgfqpoint{7.976490in}{0.818985in}}{\pgfqpoint{7.976490in}{0.830035in}}%
\pgfpathcurveto{\pgfqpoint{7.976490in}{0.841085in}}{\pgfqpoint{7.972100in}{0.851685in}}{\pgfqpoint{7.964286in}{0.859498in}}%
\pgfpathcurveto{\pgfqpoint{7.956472in}{0.867312in}}{\pgfqpoint{7.945873in}{0.871702in}}{\pgfqpoint{7.934823in}{0.871702in}}%
\pgfpathcurveto{\pgfqpoint{7.923773in}{0.871702in}}{\pgfqpoint{7.913174in}{0.867312in}}{\pgfqpoint{7.905361in}{0.859498in}}%
\pgfpathcurveto{\pgfqpoint{7.897547in}{0.851685in}}{\pgfqpoint{7.893157in}{0.841085in}}{\pgfqpoint{7.893157in}{0.830035in}}%
\pgfpathcurveto{\pgfqpoint{7.893157in}{0.818985in}}{\pgfqpoint{7.897547in}{0.808386in}}{\pgfqpoint{7.905361in}{0.800573in}}%
\pgfpathcurveto{\pgfqpoint{7.913174in}{0.792759in}}{\pgfqpoint{7.923773in}{0.788369in}}{\pgfqpoint{7.934823in}{0.788369in}}%
\pgfpathlineto{\pgfqpoint{7.934823in}{0.788369in}}%
\pgfpathclose%
\pgfusepath{stroke}%
\end{pgfscope}%
\begin{pgfscope}%
\pgfpathrectangle{\pgfqpoint{7.394209in}{0.375000in}}{\pgfqpoint{6.356833in}{5.175000in}}%
\pgfusepath{clip}%
\pgfsetbuttcap%
\pgfsetroundjoin%
\pgfsetlinewidth{1.003750pt}%
\definecolor{currentstroke}{rgb}{0.827451,0.827451,0.827451}%
\pgfsetstrokecolor{currentstroke}%
\pgfsetdash{}{0pt}%
\pgfpathmoveto{\pgfqpoint{9.256497in}{3.964768in}}%
\pgfpathcurveto{\pgfqpoint{9.267547in}{3.964768in}}{\pgfqpoint{9.278146in}{3.969158in}}{\pgfqpoint{9.285959in}{3.976972in}}%
\pgfpathcurveto{\pgfqpoint{9.293773in}{3.984785in}}{\pgfqpoint{9.298163in}{3.995384in}}{\pgfqpoint{9.298163in}{4.006435in}}%
\pgfpathcurveto{\pgfqpoint{9.298163in}{4.017485in}}{\pgfqpoint{9.293773in}{4.028084in}}{\pgfqpoint{9.285959in}{4.035897in}}%
\pgfpathcurveto{\pgfqpoint{9.278146in}{4.043711in}}{\pgfqpoint{9.267547in}{4.048101in}}{\pgfqpoint{9.256497in}{4.048101in}}%
\pgfpathcurveto{\pgfqpoint{9.245446in}{4.048101in}}{\pgfqpoint{9.234847in}{4.043711in}}{\pgfqpoint{9.227034in}{4.035897in}}%
\pgfpathcurveto{\pgfqpoint{9.219220in}{4.028084in}}{\pgfqpoint{9.214830in}{4.017485in}}{\pgfqpoint{9.214830in}{4.006435in}}%
\pgfpathcurveto{\pgfqpoint{9.214830in}{3.995384in}}{\pgfqpoint{9.219220in}{3.984785in}}{\pgfqpoint{9.227034in}{3.976972in}}%
\pgfpathcurveto{\pgfqpoint{9.234847in}{3.969158in}}{\pgfqpoint{9.245446in}{3.964768in}}{\pgfqpoint{9.256497in}{3.964768in}}%
\pgfpathlineto{\pgfqpoint{9.256497in}{3.964768in}}%
\pgfpathclose%
\pgfusepath{stroke}%
\end{pgfscope}%
\begin{pgfscope}%
\pgfpathrectangle{\pgfqpoint{7.394209in}{0.375000in}}{\pgfqpoint{6.356833in}{5.175000in}}%
\pgfusepath{clip}%
\pgfsetbuttcap%
\pgfsetroundjoin%
\pgfsetlinewidth{1.003750pt}%
\definecolor{currentstroke}{rgb}{0.827451,0.827451,0.827451}%
\pgfsetstrokecolor{currentstroke}%
\pgfsetdash{}{0pt}%
\pgfpathmoveto{\pgfqpoint{8.849784in}{1.629878in}}%
\pgfpathcurveto{\pgfqpoint{8.860834in}{1.629878in}}{\pgfqpoint{8.871433in}{1.634268in}}{\pgfqpoint{8.879247in}{1.642081in}}%
\pgfpathcurveto{\pgfqpoint{8.887061in}{1.649895in}}{\pgfqpoint{8.891451in}{1.660494in}}{\pgfqpoint{8.891451in}{1.671544in}}%
\pgfpathcurveto{\pgfqpoint{8.891451in}{1.682594in}}{\pgfqpoint{8.887061in}{1.693193in}}{\pgfqpoint{8.879247in}{1.701007in}}%
\pgfpathcurveto{\pgfqpoint{8.871433in}{1.708821in}}{\pgfqpoint{8.860834in}{1.713211in}}{\pgfqpoint{8.849784in}{1.713211in}}%
\pgfpathcurveto{\pgfqpoint{8.838734in}{1.713211in}}{\pgfqpoint{8.828135in}{1.708821in}}{\pgfqpoint{8.820321in}{1.701007in}}%
\pgfpathcurveto{\pgfqpoint{8.812508in}{1.693193in}}{\pgfqpoint{8.808117in}{1.682594in}}{\pgfqpoint{8.808117in}{1.671544in}}%
\pgfpathcurveto{\pgfqpoint{8.808117in}{1.660494in}}{\pgfqpoint{8.812508in}{1.649895in}}{\pgfqpoint{8.820321in}{1.642081in}}%
\pgfpathcurveto{\pgfqpoint{8.828135in}{1.634268in}}{\pgfqpoint{8.838734in}{1.629878in}}{\pgfqpoint{8.849784in}{1.629878in}}%
\pgfpathlineto{\pgfqpoint{8.849784in}{1.629878in}}%
\pgfpathclose%
\pgfusepath{stroke}%
\end{pgfscope}%
\begin{pgfscope}%
\pgfpathrectangle{\pgfqpoint{7.394209in}{0.375000in}}{\pgfqpoint{6.356833in}{5.175000in}}%
\pgfusepath{clip}%
\pgfsetbuttcap%
\pgfsetroundjoin%
\pgfsetlinewidth{1.003750pt}%
\definecolor{currentstroke}{rgb}{0.827451,0.827451,0.827451}%
\pgfsetstrokecolor{currentstroke}%
\pgfsetdash{}{0pt}%
\pgfpathmoveto{\pgfqpoint{9.908150in}{4.510658in}}%
\pgfpathcurveto{\pgfqpoint{9.919200in}{4.510658in}}{\pgfqpoint{9.929800in}{4.515049in}}{\pgfqpoint{9.937613in}{4.522862in}}%
\pgfpathcurveto{\pgfqpoint{9.945427in}{4.530676in}}{\pgfqpoint{9.949817in}{4.541275in}}{\pgfqpoint{9.949817in}{4.552325in}}%
\pgfpathcurveto{\pgfqpoint{9.949817in}{4.563375in}}{\pgfqpoint{9.945427in}{4.573974in}}{\pgfqpoint{9.937613in}{4.581788in}}%
\pgfpathcurveto{\pgfqpoint{9.929800in}{4.589602in}}{\pgfqpoint{9.919200in}{4.593992in}}{\pgfqpoint{9.908150in}{4.593992in}}%
\pgfpathcurveto{\pgfqpoint{9.897100in}{4.593992in}}{\pgfqpoint{9.886501in}{4.589602in}}{\pgfqpoint{9.878688in}{4.581788in}}%
\pgfpathcurveto{\pgfqpoint{9.870874in}{4.573974in}}{\pgfqpoint{9.866484in}{4.563375in}}{\pgfqpoint{9.866484in}{4.552325in}}%
\pgfpathcurveto{\pgfqpoint{9.866484in}{4.541275in}}{\pgfqpoint{9.870874in}{4.530676in}}{\pgfqpoint{9.878688in}{4.522862in}}%
\pgfpathcurveto{\pgfqpoint{9.886501in}{4.515049in}}{\pgfqpoint{9.897100in}{4.510658in}}{\pgfqpoint{9.908150in}{4.510658in}}%
\pgfpathlineto{\pgfqpoint{9.908150in}{4.510658in}}%
\pgfpathclose%
\pgfusepath{stroke}%
\end{pgfscope}%
\begin{pgfscope}%
\pgfpathrectangle{\pgfqpoint{7.394209in}{0.375000in}}{\pgfqpoint{6.356833in}{5.175000in}}%
\pgfusepath{clip}%
\pgfsetbuttcap%
\pgfsetroundjoin%
\pgfsetlinewidth{1.003750pt}%
\definecolor{currentstroke}{rgb}{0.827451,0.827451,0.827451}%
\pgfsetstrokecolor{currentstroke}%
\pgfsetdash{}{0pt}%
\pgfpathmoveto{\pgfqpoint{8.721964in}{1.576154in}}%
\pgfpathcurveto{\pgfqpoint{8.733014in}{1.576154in}}{\pgfqpoint{8.743613in}{1.580544in}}{\pgfqpoint{8.751427in}{1.588357in}}%
\pgfpathcurveto{\pgfqpoint{8.759240in}{1.596171in}}{\pgfqpoint{8.763631in}{1.606770in}}{\pgfqpoint{8.763631in}{1.617820in}}%
\pgfpathcurveto{\pgfqpoint{8.763631in}{1.628870in}}{\pgfqpoint{8.759240in}{1.639469in}}{\pgfqpoint{8.751427in}{1.647283in}}%
\pgfpathcurveto{\pgfqpoint{8.743613in}{1.655097in}}{\pgfqpoint{8.733014in}{1.659487in}}{\pgfqpoint{8.721964in}{1.659487in}}%
\pgfpathcurveto{\pgfqpoint{8.710914in}{1.659487in}}{\pgfqpoint{8.700315in}{1.655097in}}{\pgfqpoint{8.692501in}{1.647283in}}%
\pgfpathcurveto{\pgfqpoint{8.684687in}{1.639469in}}{\pgfqpoint{8.680297in}{1.628870in}}{\pgfqpoint{8.680297in}{1.617820in}}%
\pgfpathcurveto{\pgfqpoint{8.680297in}{1.606770in}}{\pgfqpoint{8.684687in}{1.596171in}}{\pgfqpoint{8.692501in}{1.588357in}}%
\pgfpathcurveto{\pgfqpoint{8.700315in}{1.580544in}}{\pgfqpoint{8.710914in}{1.576154in}}{\pgfqpoint{8.721964in}{1.576154in}}%
\pgfpathlineto{\pgfqpoint{8.721964in}{1.576154in}}%
\pgfpathclose%
\pgfusepath{stroke}%
\end{pgfscope}%
\begin{pgfscope}%
\pgfpathrectangle{\pgfqpoint{7.394209in}{0.375000in}}{\pgfqpoint{6.356833in}{5.175000in}}%
\pgfusepath{clip}%
\pgfsetbuttcap%
\pgfsetroundjoin%
\pgfsetlinewidth{1.003750pt}%
\definecolor{currentstroke}{rgb}{0.827451,0.827451,0.827451}%
\pgfsetstrokecolor{currentstroke}%
\pgfsetdash{}{0pt}%
\pgfpathmoveto{\pgfqpoint{7.647349in}{0.334737in}}%
\pgfpathcurveto{\pgfqpoint{7.658399in}{0.334737in}}{\pgfqpoint{7.668998in}{0.339127in}}{\pgfqpoint{7.676812in}{0.346941in}}%
\pgfpathcurveto{\pgfqpoint{7.684625in}{0.354754in}}{\pgfqpoint{7.689016in}{0.365353in}}{\pgfqpoint{7.689016in}{0.376404in}}%
\pgfpathcurveto{\pgfqpoint{7.689016in}{0.387454in}}{\pgfqpoint{7.684625in}{0.398053in}}{\pgfqpoint{7.676812in}{0.405866in}}%
\pgfpathcurveto{\pgfqpoint{7.668998in}{0.413680in}}{\pgfqpoint{7.658399in}{0.418070in}}{\pgfqpoint{7.647349in}{0.418070in}}%
\pgfpathcurveto{\pgfqpoint{7.636299in}{0.418070in}}{\pgfqpoint{7.625700in}{0.413680in}}{\pgfqpoint{7.617886in}{0.405866in}}%
\pgfpathcurveto{\pgfqpoint{7.610073in}{0.398053in}}{\pgfqpoint{7.605682in}{0.387454in}}{\pgfqpoint{7.605682in}{0.376404in}}%
\pgfpathcurveto{\pgfqpoint{7.605682in}{0.365353in}}{\pgfqpoint{7.610073in}{0.354754in}}{\pgfqpoint{7.617886in}{0.346941in}}%
\pgfpathcurveto{\pgfqpoint{7.625700in}{0.339127in}}{\pgfqpoint{7.636299in}{0.334737in}}{\pgfqpoint{7.647349in}{0.334737in}}%
\pgfusepath{stroke}%
\end{pgfscope}%
\begin{pgfscope}%
\pgfpathrectangle{\pgfqpoint{7.394209in}{0.375000in}}{\pgfqpoint{6.356833in}{5.175000in}}%
\pgfusepath{clip}%
\pgfsetbuttcap%
\pgfsetroundjoin%
\pgfsetlinewidth{1.003750pt}%
\definecolor{currentstroke}{rgb}{0.827451,0.827451,0.827451}%
\pgfsetstrokecolor{currentstroke}%
\pgfsetdash{}{0pt}%
\pgfpathmoveto{\pgfqpoint{11.335066in}{4.955739in}}%
\pgfpathcurveto{\pgfqpoint{11.346116in}{4.955739in}}{\pgfqpoint{11.356715in}{4.960129in}}{\pgfqpoint{11.364528in}{4.967943in}}%
\pgfpathcurveto{\pgfqpoint{11.372342in}{4.975757in}}{\pgfqpoint{11.376732in}{4.986356in}}{\pgfqpoint{11.376732in}{4.997406in}}%
\pgfpathcurveto{\pgfqpoint{11.376732in}{5.008456in}}{\pgfqpoint{11.372342in}{5.019055in}}{\pgfqpoint{11.364528in}{5.026868in}}%
\pgfpathcurveto{\pgfqpoint{11.356715in}{5.034682in}}{\pgfqpoint{11.346116in}{5.039072in}}{\pgfqpoint{11.335066in}{5.039072in}}%
\pgfpathcurveto{\pgfqpoint{11.324015in}{5.039072in}}{\pgfqpoint{11.313416in}{5.034682in}}{\pgfqpoint{11.305603in}{5.026868in}}%
\pgfpathcurveto{\pgfqpoint{11.297789in}{5.019055in}}{\pgfqpoint{11.293399in}{5.008456in}}{\pgfqpoint{11.293399in}{4.997406in}}%
\pgfpathcurveto{\pgfqpoint{11.293399in}{4.986356in}}{\pgfqpoint{11.297789in}{4.975757in}}{\pgfqpoint{11.305603in}{4.967943in}}%
\pgfpathcurveto{\pgfqpoint{11.313416in}{4.960129in}}{\pgfqpoint{11.324015in}{4.955739in}}{\pgfqpoint{11.335066in}{4.955739in}}%
\pgfpathlineto{\pgfqpoint{11.335066in}{4.955739in}}%
\pgfpathclose%
\pgfusepath{stroke}%
\end{pgfscope}%
\begin{pgfscope}%
\pgfpathrectangle{\pgfqpoint{7.394209in}{0.375000in}}{\pgfqpoint{6.356833in}{5.175000in}}%
\pgfusepath{clip}%
\pgfsetbuttcap%
\pgfsetroundjoin%
\pgfsetlinewidth{1.003750pt}%
\definecolor{currentstroke}{rgb}{0.827451,0.827451,0.827451}%
\pgfsetstrokecolor{currentstroke}%
\pgfsetdash{}{0pt}%
\pgfpathmoveto{\pgfqpoint{9.074964in}{1.824429in}}%
\pgfpathcurveto{\pgfqpoint{9.086015in}{1.824429in}}{\pgfqpoint{9.096614in}{1.828819in}}{\pgfqpoint{9.104427in}{1.836633in}}%
\pgfpathcurveto{\pgfqpoint{9.112241in}{1.844446in}}{\pgfqpoint{9.116631in}{1.855045in}}{\pgfqpoint{9.116631in}{1.866095in}}%
\pgfpathcurveto{\pgfqpoint{9.116631in}{1.877146in}}{\pgfqpoint{9.112241in}{1.887745in}}{\pgfqpoint{9.104427in}{1.895558in}}%
\pgfpathcurveto{\pgfqpoint{9.096614in}{1.903372in}}{\pgfqpoint{9.086015in}{1.907762in}}{\pgfqpoint{9.074964in}{1.907762in}}%
\pgfpathcurveto{\pgfqpoint{9.063914in}{1.907762in}}{\pgfqpoint{9.053315in}{1.903372in}}{\pgfqpoint{9.045502in}{1.895558in}}%
\pgfpathcurveto{\pgfqpoint{9.037688in}{1.887745in}}{\pgfqpoint{9.033298in}{1.877146in}}{\pgfqpoint{9.033298in}{1.866095in}}%
\pgfpathcurveto{\pgfqpoint{9.033298in}{1.855045in}}{\pgfqpoint{9.037688in}{1.844446in}}{\pgfqpoint{9.045502in}{1.836633in}}%
\pgfpathcurveto{\pgfqpoint{9.053315in}{1.828819in}}{\pgfqpoint{9.063914in}{1.824429in}}{\pgfqpoint{9.074964in}{1.824429in}}%
\pgfpathlineto{\pgfqpoint{9.074964in}{1.824429in}}%
\pgfpathclose%
\pgfusepath{stroke}%
\end{pgfscope}%
\begin{pgfscope}%
\pgfpathrectangle{\pgfqpoint{7.394209in}{0.375000in}}{\pgfqpoint{6.356833in}{5.175000in}}%
\pgfusepath{clip}%
\pgfsetbuttcap%
\pgfsetroundjoin%
\pgfsetlinewidth{1.003750pt}%
\definecolor{currentstroke}{rgb}{0.827451,0.827451,0.827451}%
\pgfsetstrokecolor{currentstroke}%
\pgfsetdash{}{0pt}%
\pgfpathmoveto{\pgfqpoint{8.804717in}{1.885434in}}%
\pgfpathcurveto{\pgfqpoint{8.815767in}{1.885434in}}{\pgfqpoint{8.826366in}{1.889824in}}{\pgfqpoint{8.834180in}{1.897638in}}%
\pgfpathcurveto{\pgfqpoint{8.841993in}{1.905452in}}{\pgfqpoint{8.846384in}{1.916051in}}{\pgfqpoint{8.846384in}{1.927101in}}%
\pgfpathcurveto{\pgfqpoint{8.846384in}{1.938151in}}{\pgfqpoint{8.841993in}{1.948750in}}{\pgfqpoint{8.834180in}{1.956564in}}%
\pgfpathcurveto{\pgfqpoint{8.826366in}{1.964377in}}{\pgfqpoint{8.815767in}{1.968768in}}{\pgfqpoint{8.804717in}{1.968768in}}%
\pgfpathcurveto{\pgfqpoint{8.793667in}{1.968768in}}{\pgfqpoint{8.783068in}{1.964377in}}{\pgfqpoint{8.775254in}{1.956564in}}%
\pgfpathcurveto{\pgfqpoint{8.767441in}{1.948750in}}{\pgfqpoint{8.763050in}{1.938151in}}{\pgfqpoint{8.763050in}{1.927101in}}%
\pgfpathcurveto{\pgfqpoint{8.763050in}{1.916051in}}{\pgfqpoint{8.767441in}{1.905452in}}{\pgfqpoint{8.775254in}{1.897638in}}%
\pgfpathcurveto{\pgfqpoint{8.783068in}{1.889824in}}{\pgfqpoint{8.793667in}{1.885434in}}{\pgfqpoint{8.804717in}{1.885434in}}%
\pgfpathlineto{\pgfqpoint{8.804717in}{1.885434in}}%
\pgfpathclose%
\pgfusepath{stroke}%
\end{pgfscope}%
\begin{pgfscope}%
\pgfpathrectangle{\pgfqpoint{7.394209in}{0.375000in}}{\pgfqpoint{6.356833in}{5.175000in}}%
\pgfusepath{clip}%
\pgfsetbuttcap%
\pgfsetroundjoin%
\pgfsetlinewidth{1.003750pt}%
\definecolor{currentstroke}{rgb}{0.827451,0.827451,0.827451}%
\pgfsetstrokecolor{currentstroke}%
\pgfsetdash{}{0pt}%
\pgfpathmoveto{\pgfqpoint{11.247632in}{4.922623in}}%
\pgfpathcurveto{\pgfqpoint{11.258682in}{4.922623in}}{\pgfqpoint{11.269281in}{4.927013in}}{\pgfqpoint{11.277095in}{4.934826in}}%
\pgfpathcurveto{\pgfqpoint{11.284909in}{4.942640in}}{\pgfqpoint{11.289299in}{4.953239in}}{\pgfqpoint{11.289299in}{4.964289in}}%
\pgfpathcurveto{\pgfqpoint{11.289299in}{4.975339in}}{\pgfqpoint{11.284909in}{4.985938in}}{\pgfqpoint{11.277095in}{4.993752in}}%
\pgfpathcurveto{\pgfqpoint{11.269281in}{5.001566in}}{\pgfqpoint{11.258682in}{5.005956in}}{\pgfqpoint{11.247632in}{5.005956in}}%
\pgfpathcurveto{\pgfqpoint{11.236582in}{5.005956in}}{\pgfqpoint{11.225983in}{5.001566in}}{\pgfqpoint{11.218170in}{4.993752in}}%
\pgfpathcurveto{\pgfqpoint{11.210356in}{4.985938in}}{\pgfqpoint{11.205966in}{4.975339in}}{\pgfqpoint{11.205966in}{4.964289in}}%
\pgfpathcurveto{\pgfqpoint{11.205966in}{4.953239in}}{\pgfqpoint{11.210356in}{4.942640in}}{\pgfqpoint{11.218170in}{4.934826in}}%
\pgfpathcurveto{\pgfqpoint{11.225983in}{4.927013in}}{\pgfqpoint{11.236582in}{4.922623in}}{\pgfqpoint{11.247632in}{4.922623in}}%
\pgfpathlineto{\pgfqpoint{11.247632in}{4.922623in}}%
\pgfpathclose%
\pgfusepath{stroke}%
\end{pgfscope}%
\begin{pgfscope}%
\pgfpathrectangle{\pgfqpoint{7.394209in}{0.375000in}}{\pgfqpoint{6.356833in}{5.175000in}}%
\pgfusepath{clip}%
\pgfsetbuttcap%
\pgfsetroundjoin%
\pgfsetlinewidth{1.003750pt}%
\definecolor{currentstroke}{rgb}{0.827451,0.827451,0.827451}%
\pgfsetstrokecolor{currentstroke}%
\pgfsetdash{}{0pt}%
\pgfpathmoveto{\pgfqpoint{7.504760in}{1.063278in}}%
\pgfpathcurveto{\pgfqpoint{7.515811in}{1.063278in}}{\pgfqpoint{7.526410in}{1.067668in}}{\pgfqpoint{7.534223in}{1.075482in}}%
\pgfpathcurveto{\pgfqpoint{7.542037in}{1.083295in}}{\pgfqpoint{7.546427in}{1.093894in}}{\pgfqpoint{7.546427in}{1.104944in}}%
\pgfpathcurveto{\pgfqpoint{7.546427in}{1.115995in}}{\pgfqpoint{7.542037in}{1.126594in}}{\pgfqpoint{7.534223in}{1.134407in}}%
\pgfpathcurveto{\pgfqpoint{7.526410in}{1.142221in}}{\pgfqpoint{7.515811in}{1.146611in}}{\pgfqpoint{7.504760in}{1.146611in}}%
\pgfpathcurveto{\pgfqpoint{7.493710in}{1.146611in}}{\pgfqpoint{7.483111in}{1.142221in}}{\pgfqpoint{7.475298in}{1.134407in}}%
\pgfpathcurveto{\pgfqpoint{7.467484in}{1.126594in}}{\pgfqpoint{7.463094in}{1.115995in}}{\pgfqpoint{7.463094in}{1.104944in}}%
\pgfpathcurveto{\pgfqpoint{7.463094in}{1.093894in}}{\pgfqpoint{7.467484in}{1.083295in}}{\pgfqpoint{7.475298in}{1.075482in}}%
\pgfpathcurveto{\pgfqpoint{7.483111in}{1.067668in}}{\pgfqpoint{7.493710in}{1.063278in}}{\pgfqpoint{7.504760in}{1.063278in}}%
\pgfpathlineto{\pgfqpoint{7.504760in}{1.063278in}}%
\pgfpathclose%
\pgfusepath{stroke}%
\end{pgfscope}%
\begin{pgfscope}%
\pgfpathrectangle{\pgfqpoint{7.394209in}{0.375000in}}{\pgfqpoint{6.356833in}{5.175000in}}%
\pgfusepath{clip}%
\pgfsetbuttcap%
\pgfsetroundjoin%
\pgfsetlinewidth{1.003750pt}%
\definecolor{currentstroke}{rgb}{0.827451,0.827451,0.827451}%
\pgfsetstrokecolor{currentstroke}%
\pgfsetdash{}{0pt}%
\pgfpathmoveto{\pgfqpoint{9.465947in}{3.503628in}}%
\pgfpathcurveto{\pgfqpoint{9.476997in}{3.503628in}}{\pgfqpoint{9.487596in}{3.508019in}}{\pgfqpoint{9.495410in}{3.515832in}}%
\pgfpathcurveto{\pgfqpoint{9.503224in}{3.523646in}}{\pgfqpoint{9.507614in}{3.534245in}}{\pgfqpoint{9.507614in}{3.545295in}}%
\pgfpathcurveto{\pgfqpoint{9.507614in}{3.556345in}}{\pgfqpoint{9.503224in}{3.566944in}}{\pgfqpoint{9.495410in}{3.574758in}}%
\pgfpathcurveto{\pgfqpoint{9.487596in}{3.582572in}}{\pgfqpoint{9.476997in}{3.586962in}}{\pgfqpoint{9.465947in}{3.586962in}}%
\pgfpathcurveto{\pgfqpoint{9.454897in}{3.586962in}}{\pgfqpoint{9.444298in}{3.582572in}}{\pgfqpoint{9.436485in}{3.574758in}}%
\pgfpathcurveto{\pgfqpoint{9.428671in}{3.566944in}}{\pgfqpoint{9.424281in}{3.556345in}}{\pgfqpoint{9.424281in}{3.545295in}}%
\pgfpathcurveto{\pgfqpoint{9.424281in}{3.534245in}}{\pgfqpoint{9.428671in}{3.523646in}}{\pgfqpoint{9.436485in}{3.515832in}}%
\pgfpathcurveto{\pgfqpoint{9.444298in}{3.508019in}}{\pgfqpoint{9.454897in}{3.503628in}}{\pgfqpoint{9.465947in}{3.503628in}}%
\pgfpathlineto{\pgfqpoint{9.465947in}{3.503628in}}%
\pgfpathclose%
\pgfusepath{stroke}%
\end{pgfscope}%
\begin{pgfscope}%
\pgfpathrectangle{\pgfqpoint{7.394209in}{0.375000in}}{\pgfqpoint{6.356833in}{5.175000in}}%
\pgfusepath{clip}%
\pgfsetbuttcap%
\pgfsetroundjoin%
\pgfsetlinewidth{1.003750pt}%
\definecolor{currentstroke}{rgb}{0.827451,0.827451,0.827451}%
\pgfsetstrokecolor{currentstroke}%
\pgfsetdash{}{0pt}%
\pgfpathmoveto{\pgfqpoint{10.412514in}{5.226330in}}%
\pgfpathcurveto{\pgfqpoint{10.423564in}{5.226330in}}{\pgfqpoint{10.434163in}{5.230721in}}{\pgfqpoint{10.441977in}{5.238534in}}%
\pgfpathcurveto{\pgfqpoint{10.449791in}{5.246348in}}{\pgfqpoint{10.454181in}{5.256947in}}{\pgfqpoint{10.454181in}{5.267997in}}%
\pgfpathcurveto{\pgfqpoint{10.454181in}{5.279047in}}{\pgfqpoint{10.449791in}{5.289646in}}{\pgfqpoint{10.441977in}{5.297460in}}%
\pgfpathcurveto{\pgfqpoint{10.434163in}{5.305274in}}{\pgfqpoint{10.423564in}{5.309664in}}{\pgfqpoint{10.412514in}{5.309664in}}%
\pgfpathcurveto{\pgfqpoint{10.401464in}{5.309664in}}{\pgfqpoint{10.390865in}{5.305274in}}{\pgfqpoint{10.383051in}{5.297460in}}%
\pgfpathcurveto{\pgfqpoint{10.375238in}{5.289646in}}{\pgfqpoint{10.370847in}{5.279047in}}{\pgfqpoint{10.370847in}{5.267997in}}%
\pgfpathcurveto{\pgfqpoint{10.370847in}{5.256947in}}{\pgfqpoint{10.375238in}{5.246348in}}{\pgfqpoint{10.383051in}{5.238534in}}%
\pgfpathcurveto{\pgfqpoint{10.390865in}{5.230721in}}{\pgfqpoint{10.401464in}{5.226330in}}{\pgfqpoint{10.412514in}{5.226330in}}%
\pgfpathlineto{\pgfqpoint{10.412514in}{5.226330in}}%
\pgfpathclose%
\pgfusepath{stroke}%
\end{pgfscope}%
\begin{pgfscope}%
\pgfpathrectangle{\pgfqpoint{7.394209in}{0.375000in}}{\pgfqpoint{6.356833in}{5.175000in}}%
\pgfusepath{clip}%
\pgfsetbuttcap%
\pgfsetroundjoin%
\pgfsetlinewidth{1.003750pt}%
\definecolor{currentstroke}{rgb}{0.827451,0.827451,0.827451}%
\pgfsetstrokecolor{currentstroke}%
\pgfsetdash{}{0pt}%
\pgfpathmoveto{\pgfqpoint{9.457210in}{3.730755in}}%
\pgfpathcurveto{\pgfqpoint{9.468260in}{3.730755in}}{\pgfqpoint{9.478859in}{3.735146in}}{\pgfqpoint{9.486672in}{3.742959in}}%
\pgfpathcurveto{\pgfqpoint{9.494486in}{3.750773in}}{\pgfqpoint{9.498876in}{3.761372in}}{\pgfqpoint{9.498876in}{3.772422in}}%
\pgfpathcurveto{\pgfqpoint{9.498876in}{3.783472in}}{\pgfqpoint{9.494486in}{3.794071in}}{\pgfqpoint{9.486672in}{3.801885in}}%
\pgfpathcurveto{\pgfqpoint{9.478859in}{3.809698in}}{\pgfqpoint{9.468260in}{3.814089in}}{\pgfqpoint{9.457210in}{3.814089in}}%
\pgfpathcurveto{\pgfqpoint{9.446160in}{3.814089in}}{\pgfqpoint{9.435561in}{3.809698in}}{\pgfqpoint{9.427747in}{3.801885in}}%
\pgfpathcurveto{\pgfqpoint{9.419933in}{3.794071in}}{\pgfqpoint{9.415543in}{3.783472in}}{\pgfqpoint{9.415543in}{3.772422in}}%
\pgfpathcurveto{\pgfqpoint{9.415543in}{3.761372in}}{\pgfqpoint{9.419933in}{3.750773in}}{\pgfqpoint{9.427747in}{3.742959in}}%
\pgfpathcurveto{\pgfqpoint{9.435561in}{3.735146in}}{\pgfqpoint{9.446160in}{3.730755in}}{\pgfqpoint{9.457210in}{3.730755in}}%
\pgfpathlineto{\pgfqpoint{9.457210in}{3.730755in}}%
\pgfpathclose%
\pgfusepath{stroke}%
\end{pgfscope}%
\begin{pgfscope}%
\pgfpathrectangle{\pgfqpoint{7.394209in}{0.375000in}}{\pgfqpoint{6.356833in}{5.175000in}}%
\pgfusepath{clip}%
\pgfsetbuttcap%
\pgfsetroundjoin%
\pgfsetlinewidth{1.003750pt}%
\definecolor{currentstroke}{rgb}{0.827451,0.827451,0.827451}%
\pgfsetstrokecolor{currentstroke}%
\pgfsetdash{}{0pt}%
\pgfpathmoveto{\pgfqpoint{10.032258in}{4.491928in}}%
\pgfpathcurveto{\pgfqpoint{10.043308in}{4.491928in}}{\pgfqpoint{10.053907in}{4.496318in}}{\pgfqpoint{10.061721in}{4.504132in}}%
\pgfpathcurveto{\pgfqpoint{10.069535in}{4.511945in}}{\pgfqpoint{10.073925in}{4.522544in}}{\pgfqpoint{10.073925in}{4.533594in}}%
\pgfpathcurveto{\pgfqpoint{10.073925in}{4.544645in}}{\pgfqpoint{10.069535in}{4.555244in}}{\pgfqpoint{10.061721in}{4.563057in}}%
\pgfpathcurveto{\pgfqpoint{10.053907in}{4.570871in}}{\pgfqpoint{10.043308in}{4.575261in}}{\pgfqpoint{10.032258in}{4.575261in}}%
\pgfpathcurveto{\pgfqpoint{10.021208in}{4.575261in}}{\pgfqpoint{10.010609in}{4.570871in}}{\pgfqpoint{10.002795in}{4.563057in}}%
\pgfpathcurveto{\pgfqpoint{9.994982in}{4.555244in}}{\pgfqpoint{9.990592in}{4.544645in}}{\pgfqpoint{9.990592in}{4.533594in}}%
\pgfpathcurveto{\pgfqpoint{9.990592in}{4.522544in}}{\pgfqpoint{9.994982in}{4.511945in}}{\pgfqpoint{10.002795in}{4.504132in}}%
\pgfpathcurveto{\pgfqpoint{10.010609in}{4.496318in}}{\pgfqpoint{10.021208in}{4.491928in}}{\pgfqpoint{10.032258in}{4.491928in}}%
\pgfpathlineto{\pgfqpoint{10.032258in}{4.491928in}}%
\pgfpathclose%
\pgfusepath{stroke}%
\end{pgfscope}%
\begin{pgfscope}%
\pgfpathrectangle{\pgfqpoint{7.394209in}{0.375000in}}{\pgfqpoint{6.356833in}{5.175000in}}%
\pgfusepath{clip}%
\pgfsetbuttcap%
\pgfsetroundjoin%
\pgfsetlinewidth{1.003750pt}%
\definecolor{currentstroke}{rgb}{0.827451,0.827451,0.827451}%
\pgfsetstrokecolor{currentstroke}%
\pgfsetdash{}{0pt}%
\pgfpathmoveto{\pgfqpoint{8.594940in}{1.522739in}}%
\pgfpathcurveto{\pgfqpoint{8.605990in}{1.522739in}}{\pgfqpoint{8.616589in}{1.527129in}}{\pgfqpoint{8.624403in}{1.534943in}}%
\pgfpathcurveto{\pgfqpoint{8.632217in}{1.542756in}}{\pgfqpoint{8.636607in}{1.553355in}}{\pgfqpoint{8.636607in}{1.564405in}}%
\pgfpathcurveto{\pgfqpoint{8.636607in}{1.575455in}}{\pgfqpoint{8.632217in}{1.586054in}}{\pgfqpoint{8.624403in}{1.593868in}}%
\pgfpathcurveto{\pgfqpoint{8.616589in}{1.601682in}}{\pgfqpoint{8.605990in}{1.606072in}}{\pgfqpoint{8.594940in}{1.606072in}}%
\pgfpathcurveto{\pgfqpoint{8.583890in}{1.606072in}}{\pgfqpoint{8.573291in}{1.601682in}}{\pgfqpoint{8.565477in}{1.593868in}}%
\pgfpathcurveto{\pgfqpoint{8.557664in}{1.586054in}}{\pgfqpoint{8.553274in}{1.575455in}}{\pgfqpoint{8.553274in}{1.564405in}}%
\pgfpathcurveto{\pgfqpoint{8.553274in}{1.553355in}}{\pgfqpoint{8.557664in}{1.542756in}}{\pgfqpoint{8.565477in}{1.534943in}}%
\pgfpathcurveto{\pgfqpoint{8.573291in}{1.527129in}}{\pgfqpoint{8.583890in}{1.522739in}}{\pgfqpoint{8.594940in}{1.522739in}}%
\pgfpathlineto{\pgfqpoint{8.594940in}{1.522739in}}%
\pgfpathclose%
\pgfusepath{stroke}%
\end{pgfscope}%
\begin{pgfscope}%
\pgfpathrectangle{\pgfqpoint{7.394209in}{0.375000in}}{\pgfqpoint{6.356833in}{5.175000in}}%
\pgfusepath{clip}%
\pgfsetbuttcap%
\pgfsetroundjoin%
\pgfsetlinewidth{1.003750pt}%
\definecolor{currentstroke}{rgb}{0.827451,0.827451,0.827451}%
\pgfsetstrokecolor{currentstroke}%
\pgfsetdash{}{0pt}%
\pgfpathmoveto{\pgfqpoint{8.306845in}{2.468248in}}%
\pgfpathcurveto{\pgfqpoint{8.317895in}{2.468248in}}{\pgfqpoint{8.328494in}{2.472639in}}{\pgfqpoint{8.336307in}{2.480452in}}%
\pgfpathcurveto{\pgfqpoint{8.344121in}{2.488266in}}{\pgfqpoint{8.348511in}{2.498865in}}{\pgfqpoint{8.348511in}{2.509915in}}%
\pgfpathcurveto{\pgfqpoint{8.348511in}{2.520965in}}{\pgfqpoint{8.344121in}{2.531564in}}{\pgfqpoint{8.336307in}{2.539378in}}%
\pgfpathcurveto{\pgfqpoint{8.328494in}{2.547191in}}{\pgfqpoint{8.317895in}{2.551582in}}{\pgfqpoint{8.306845in}{2.551582in}}%
\pgfpathcurveto{\pgfqpoint{8.295795in}{2.551582in}}{\pgfqpoint{8.285196in}{2.547191in}}{\pgfqpoint{8.277382in}{2.539378in}}%
\pgfpathcurveto{\pgfqpoint{8.269568in}{2.531564in}}{\pgfqpoint{8.265178in}{2.520965in}}{\pgfqpoint{8.265178in}{2.509915in}}%
\pgfpathcurveto{\pgfqpoint{8.265178in}{2.498865in}}{\pgfqpoint{8.269568in}{2.488266in}}{\pgfqpoint{8.277382in}{2.480452in}}%
\pgfpathcurveto{\pgfqpoint{8.285196in}{2.472639in}}{\pgfqpoint{8.295795in}{2.468248in}}{\pgfqpoint{8.306845in}{2.468248in}}%
\pgfpathlineto{\pgfqpoint{8.306845in}{2.468248in}}%
\pgfpathclose%
\pgfusepath{stroke}%
\end{pgfscope}%
\begin{pgfscope}%
\pgfpathrectangle{\pgfqpoint{7.394209in}{0.375000in}}{\pgfqpoint{6.356833in}{5.175000in}}%
\pgfusepath{clip}%
\pgfsetbuttcap%
\pgfsetroundjoin%
\pgfsetlinewidth{1.003750pt}%
\definecolor{currentstroke}{rgb}{0.827451,0.827451,0.827451}%
\pgfsetstrokecolor{currentstroke}%
\pgfsetdash{}{0pt}%
\pgfpathmoveto{\pgfqpoint{10.217111in}{3.156381in}}%
\pgfpathcurveto{\pgfqpoint{10.228161in}{3.156381in}}{\pgfqpoint{10.238760in}{3.160771in}}{\pgfqpoint{10.246574in}{3.168585in}}%
\pgfpathcurveto{\pgfqpoint{10.254387in}{3.176399in}}{\pgfqpoint{10.258778in}{3.186998in}}{\pgfqpoint{10.258778in}{3.198048in}}%
\pgfpathcurveto{\pgfqpoint{10.258778in}{3.209098in}}{\pgfqpoint{10.254387in}{3.219697in}}{\pgfqpoint{10.246574in}{3.227511in}}%
\pgfpathcurveto{\pgfqpoint{10.238760in}{3.235324in}}{\pgfqpoint{10.228161in}{3.239715in}}{\pgfqpoint{10.217111in}{3.239715in}}%
\pgfpathcurveto{\pgfqpoint{10.206061in}{3.239715in}}{\pgfqpoint{10.195462in}{3.235324in}}{\pgfqpoint{10.187648in}{3.227511in}}%
\pgfpathcurveto{\pgfqpoint{10.179835in}{3.219697in}}{\pgfqpoint{10.175444in}{3.209098in}}{\pgfqpoint{10.175444in}{3.198048in}}%
\pgfpathcurveto{\pgfqpoint{10.175444in}{3.186998in}}{\pgfqpoint{10.179835in}{3.176399in}}{\pgfqpoint{10.187648in}{3.168585in}}%
\pgfpathcurveto{\pgfqpoint{10.195462in}{3.160771in}}{\pgfqpoint{10.206061in}{3.156381in}}{\pgfqpoint{10.217111in}{3.156381in}}%
\pgfpathlineto{\pgfqpoint{10.217111in}{3.156381in}}%
\pgfpathclose%
\pgfusepath{stroke}%
\end{pgfscope}%
\begin{pgfscope}%
\pgfpathrectangle{\pgfqpoint{7.394209in}{0.375000in}}{\pgfqpoint{6.356833in}{5.175000in}}%
\pgfusepath{clip}%
\pgfsetbuttcap%
\pgfsetroundjoin%
\pgfsetlinewidth{1.003750pt}%
\definecolor{currentstroke}{rgb}{0.827451,0.827451,0.827451}%
\pgfsetstrokecolor{currentstroke}%
\pgfsetdash{}{0pt}%
\pgfpathmoveto{\pgfqpoint{7.508632in}{0.905431in}}%
\pgfpathcurveto{\pgfqpoint{7.519682in}{0.905431in}}{\pgfqpoint{7.530281in}{0.909821in}}{\pgfqpoint{7.538094in}{0.917635in}}%
\pgfpathcurveto{\pgfqpoint{7.545908in}{0.925448in}}{\pgfqpoint{7.550298in}{0.936047in}}{\pgfqpoint{7.550298in}{0.947097in}}%
\pgfpathcurveto{\pgfqpoint{7.550298in}{0.958148in}}{\pgfqpoint{7.545908in}{0.968747in}}{\pgfqpoint{7.538094in}{0.976560in}}%
\pgfpathcurveto{\pgfqpoint{7.530281in}{0.984374in}}{\pgfqpoint{7.519682in}{0.988764in}}{\pgfqpoint{7.508632in}{0.988764in}}%
\pgfpathcurveto{\pgfqpoint{7.497581in}{0.988764in}}{\pgfqpoint{7.486982in}{0.984374in}}{\pgfqpoint{7.479169in}{0.976560in}}%
\pgfpathcurveto{\pgfqpoint{7.471355in}{0.968747in}}{\pgfqpoint{7.466965in}{0.958148in}}{\pgfqpoint{7.466965in}{0.947097in}}%
\pgfpathcurveto{\pgfqpoint{7.466965in}{0.936047in}}{\pgfqpoint{7.471355in}{0.925448in}}{\pgfqpoint{7.479169in}{0.917635in}}%
\pgfpathcurveto{\pgfqpoint{7.486982in}{0.909821in}}{\pgfqpoint{7.497581in}{0.905431in}}{\pgfqpoint{7.508632in}{0.905431in}}%
\pgfpathlineto{\pgfqpoint{7.508632in}{0.905431in}}%
\pgfpathclose%
\pgfusepath{stroke}%
\end{pgfscope}%
\begin{pgfscope}%
\pgfpathrectangle{\pgfqpoint{7.394209in}{0.375000in}}{\pgfqpoint{6.356833in}{5.175000in}}%
\pgfusepath{clip}%
\pgfsetbuttcap%
\pgfsetroundjoin%
\pgfsetlinewidth{1.003750pt}%
\definecolor{currentstroke}{rgb}{0.827451,0.827451,0.827451}%
\pgfsetstrokecolor{currentstroke}%
\pgfsetdash{}{0pt}%
\pgfpathmoveto{\pgfqpoint{11.287333in}{5.029816in}}%
\pgfpathcurveto{\pgfqpoint{11.298383in}{5.029816in}}{\pgfqpoint{11.308982in}{5.034206in}}{\pgfqpoint{11.316796in}{5.042020in}}%
\pgfpathcurveto{\pgfqpoint{11.324610in}{5.049833in}}{\pgfqpoint{11.329000in}{5.060432in}}{\pgfqpoint{11.329000in}{5.071482in}}%
\pgfpathcurveto{\pgfqpoint{11.329000in}{5.082532in}}{\pgfqpoint{11.324610in}{5.093132in}}{\pgfqpoint{11.316796in}{5.100945in}}%
\pgfpathcurveto{\pgfqpoint{11.308982in}{5.108759in}}{\pgfqpoint{11.298383in}{5.113149in}}{\pgfqpoint{11.287333in}{5.113149in}}%
\pgfpathcurveto{\pgfqpoint{11.276283in}{5.113149in}}{\pgfqpoint{11.265684in}{5.108759in}}{\pgfqpoint{11.257871in}{5.100945in}}%
\pgfpathcurveto{\pgfqpoint{11.250057in}{5.093132in}}{\pgfqpoint{11.245667in}{5.082532in}}{\pgfqpoint{11.245667in}{5.071482in}}%
\pgfpathcurveto{\pgfqpoint{11.245667in}{5.060432in}}{\pgfqpoint{11.250057in}{5.049833in}}{\pgfqpoint{11.257871in}{5.042020in}}%
\pgfpathcurveto{\pgfqpoint{11.265684in}{5.034206in}}{\pgfqpoint{11.276283in}{5.029816in}}{\pgfqpoint{11.287333in}{5.029816in}}%
\pgfpathlineto{\pgfqpoint{11.287333in}{5.029816in}}%
\pgfpathclose%
\pgfusepath{stroke}%
\end{pgfscope}%
\begin{pgfscope}%
\pgfpathrectangle{\pgfqpoint{7.394209in}{0.375000in}}{\pgfqpoint{6.356833in}{5.175000in}}%
\pgfusepath{clip}%
\pgfsetbuttcap%
\pgfsetroundjoin%
\pgfsetlinewidth{1.003750pt}%
\definecolor{currentstroke}{rgb}{0.827451,0.827451,0.827451}%
\pgfsetstrokecolor{currentstroke}%
\pgfsetdash{}{0pt}%
\pgfpathmoveto{\pgfqpoint{13.040973in}{5.496428in}}%
\pgfpathcurveto{\pgfqpoint{13.052023in}{5.496428in}}{\pgfqpoint{13.062622in}{5.500819in}}{\pgfqpoint{13.070436in}{5.508632in}}%
\pgfpathcurveto{\pgfqpoint{13.078250in}{5.516446in}}{\pgfqpoint{13.082640in}{5.527045in}}{\pgfqpoint{13.082640in}{5.538095in}}%
\pgfpathcurveto{\pgfqpoint{13.082640in}{5.549145in}}{\pgfqpoint{13.078250in}{5.559744in}}{\pgfqpoint{13.070436in}{5.567558in}}%
\pgfpathcurveto{\pgfqpoint{13.062622in}{5.575372in}}{\pgfqpoint{13.052023in}{5.579762in}}{\pgfqpoint{13.040973in}{5.579762in}}%
\pgfpathcurveto{\pgfqpoint{13.029923in}{5.579762in}}{\pgfqpoint{13.019324in}{5.575372in}}{\pgfqpoint{13.011511in}{5.567558in}}%
\pgfpathcurveto{\pgfqpoint{13.003697in}{5.559744in}}{\pgfqpoint{12.999307in}{5.549145in}}{\pgfqpoint{12.999307in}{5.538095in}}%
\pgfpathcurveto{\pgfqpoint{12.999307in}{5.527045in}}{\pgfqpoint{13.003697in}{5.516446in}}{\pgfqpoint{13.011511in}{5.508632in}}%
\pgfpathcurveto{\pgfqpoint{13.019324in}{5.500819in}}{\pgfqpoint{13.029923in}{5.496428in}}{\pgfqpoint{13.040973in}{5.496428in}}%
\pgfpathlineto{\pgfqpoint{13.040973in}{5.496428in}}%
\pgfpathclose%
\pgfusepath{stroke}%
\end{pgfscope}%
\begin{pgfscope}%
\pgfpathrectangle{\pgfqpoint{7.394209in}{0.375000in}}{\pgfqpoint{6.356833in}{5.175000in}}%
\pgfusepath{clip}%
\pgfsetbuttcap%
\pgfsetroundjoin%
\pgfsetlinewidth{1.003750pt}%
\definecolor{currentstroke}{rgb}{0.827451,0.827451,0.827451}%
\pgfsetstrokecolor{currentstroke}%
\pgfsetdash{}{0pt}%
\pgfpathmoveto{\pgfqpoint{8.265256in}{2.574417in}}%
\pgfpathcurveto{\pgfqpoint{8.276306in}{2.574417in}}{\pgfqpoint{8.286905in}{2.578807in}}{\pgfqpoint{8.294718in}{2.586620in}}%
\pgfpathcurveto{\pgfqpoint{8.302532in}{2.594434in}}{\pgfqpoint{8.306922in}{2.605033in}}{\pgfqpoint{8.306922in}{2.616083in}}%
\pgfpathcurveto{\pgfqpoint{8.306922in}{2.627133in}}{\pgfqpoint{8.302532in}{2.637732in}}{\pgfqpoint{8.294718in}{2.645546in}}%
\pgfpathcurveto{\pgfqpoint{8.286905in}{2.653360in}}{\pgfqpoint{8.276306in}{2.657750in}}{\pgfqpoint{8.265256in}{2.657750in}}%
\pgfpathcurveto{\pgfqpoint{8.254205in}{2.657750in}}{\pgfqpoint{8.243606in}{2.653360in}}{\pgfqpoint{8.235793in}{2.645546in}}%
\pgfpathcurveto{\pgfqpoint{8.227979in}{2.637732in}}{\pgfqpoint{8.223589in}{2.627133in}}{\pgfqpoint{8.223589in}{2.616083in}}%
\pgfpathcurveto{\pgfqpoint{8.223589in}{2.605033in}}{\pgfqpoint{8.227979in}{2.594434in}}{\pgfqpoint{8.235793in}{2.586620in}}%
\pgfpathcurveto{\pgfqpoint{8.243606in}{2.578807in}}{\pgfqpoint{8.254205in}{2.574417in}}{\pgfqpoint{8.265256in}{2.574417in}}%
\pgfpathlineto{\pgfqpoint{8.265256in}{2.574417in}}%
\pgfpathclose%
\pgfusepath{stroke}%
\end{pgfscope}%
\begin{pgfscope}%
\pgfpathrectangle{\pgfqpoint{7.394209in}{0.375000in}}{\pgfqpoint{6.356833in}{5.175000in}}%
\pgfusepath{clip}%
\pgfsetbuttcap%
\pgfsetroundjoin%
\pgfsetlinewidth{1.003750pt}%
\definecolor{currentstroke}{rgb}{0.827451,0.827451,0.827451}%
\pgfsetstrokecolor{currentstroke}%
\pgfsetdash{}{0pt}%
\pgfpathmoveto{\pgfqpoint{8.099354in}{0.750356in}}%
\pgfpathcurveto{\pgfqpoint{8.110404in}{0.750356in}}{\pgfqpoint{8.121004in}{0.754746in}}{\pgfqpoint{8.128817in}{0.762559in}}%
\pgfpathcurveto{\pgfqpoint{8.136631in}{0.770373in}}{\pgfqpoint{8.141021in}{0.780972in}}{\pgfqpoint{8.141021in}{0.792022in}}%
\pgfpathcurveto{\pgfqpoint{8.141021in}{0.803072in}}{\pgfqpoint{8.136631in}{0.813671in}}{\pgfqpoint{8.128817in}{0.821485in}}%
\pgfpathcurveto{\pgfqpoint{8.121004in}{0.829299in}}{\pgfqpoint{8.110404in}{0.833689in}}{\pgfqpoint{8.099354in}{0.833689in}}%
\pgfpathcurveto{\pgfqpoint{8.088304in}{0.833689in}}{\pgfqpoint{8.077705in}{0.829299in}}{\pgfqpoint{8.069892in}{0.821485in}}%
\pgfpathcurveto{\pgfqpoint{8.062078in}{0.813671in}}{\pgfqpoint{8.057688in}{0.803072in}}{\pgfqpoint{8.057688in}{0.792022in}}%
\pgfpathcurveto{\pgfqpoint{8.057688in}{0.780972in}}{\pgfqpoint{8.062078in}{0.770373in}}{\pgfqpoint{8.069892in}{0.762559in}}%
\pgfpathcurveto{\pgfqpoint{8.077705in}{0.754746in}}{\pgfqpoint{8.088304in}{0.750356in}}{\pgfqpoint{8.099354in}{0.750356in}}%
\pgfpathlineto{\pgfqpoint{8.099354in}{0.750356in}}%
\pgfpathclose%
\pgfusepath{stroke}%
\end{pgfscope}%
\begin{pgfscope}%
\pgfpathrectangle{\pgfqpoint{7.394209in}{0.375000in}}{\pgfqpoint{6.356833in}{5.175000in}}%
\pgfusepath{clip}%
\pgfsetbuttcap%
\pgfsetroundjoin%
\pgfsetlinewidth{1.003750pt}%
\definecolor{currentstroke}{rgb}{0.827451,0.827451,0.827451}%
\pgfsetstrokecolor{currentstroke}%
\pgfsetdash{}{0pt}%
\pgfpathmoveto{\pgfqpoint{10.013934in}{2.892768in}}%
\pgfpathcurveto{\pgfqpoint{10.024984in}{2.892768in}}{\pgfqpoint{10.035583in}{2.897158in}}{\pgfqpoint{10.043396in}{2.904972in}}%
\pgfpathcurveto{\pgfqpoint{10.051210in}{2.912786in}}{\pgfqpoint{10.055600in}{2.923385in}}{\pgfqpoint{10.055600in}{2.934435in}}%
\pgfpathcurveto{\pgfqpoint{10.055600in}{2.945485in}}{\pgfqpoint{10.051210in}{2.956084in}}{\pgfqpoint{10.043396in}{2.963898in}}%
\pgfpathcurveto{\pgfqpoint{10.035583in}{2.971711in}}{\pgfqpoint{10.024984in}{2.976101in}}{\pgfqpoint{10.013934in}{2.976101in}}%
\pgfpathcurveto{\pgfqpoint{10.002883in}{2.976101in}}{\pgfqpoint{9.992284in}{2.971711in}}{\pgfqpoint{9.984471in}{2.963898in}}%
\pgfpathcurveto{\pgfqpoint{9.976657in}{2.956084in}}{\pgfqpoint{9.972267in}{2.945485in}}{\pgfqpoint{9.972267in}{2.934435in}}%
\pgfpathcurveto{\pgfqpoint{9.972267in}{2.923385in}}{\pgfqpoint{9.976657in}{2.912786in}}{\pgfqpoint{9.984471in}{2.904972in}}%
\pgfpathcurveto{\pgfqpoint{9.992284in}{2.897158in}}{\pgfqpoint{10.002883in}{2.892768in}}{\pgfqpoint{10.013934in}{2.892768in}}%
\pgfpathlineto{\pgfqpoint{10.013934in}{2.892768in}}%
\pgfpathclose%
\pgfusepath{stroke}%
\end{pgfscope}%
\begin{pgfscope}%
\pgfpathrectangle{\pgfqpoint{7.394209in}{0.375000in}}{\pgfqpoint{6.356833in}{5.175000in}}%
\pgfusepath{clip}%
\pgfsetbuttcap%
\pgfsetroundjoin%
\pgfsetlinewidth{1.003750pt}%
\definecolor{currentstroke}{rgb}{0.827451,0.827451,0.827451}%
\pgfsetstrokecolor{currentstroke}%
\pgfsetdash{}{0pt}%
\pgfpathmoveto{\pgfqpoint{7.506296in}{0.757559in}}%
\pgfpathcurveto{\pgfqpoint{7.517346in}{0.757559in}}{\pgfqpoint{7.527945in}{0.761950in}}{\pgfqpoint{7.535759in}{0.769763in}}%
\pgfpathcurveto{\pgfqpoint{7.543573in}{0.777577in}}{\pgfqpoint{7.547963in}{0.788176in}}{\pgfqpoint{7.547963in}{0.799226in}}%
\pgfpathcurveto{\pgfqpoint{7.547963in}{0.810276in}}{\pgfqpoint{7.543573in}{0.820875in}}{\pgfqpoint{7.535759in}{0.828689in}}%
\pgfpathcurveto{\pgfqpoint{7.527945in}{0.836502in}}{\pgfqpoint{7.517346in}{0.840893in}}{\pgfqpoint{7.506296in}{0.840893in}}%
\pgfpathcurveto{\pgfqpoint{7.495246in}{0.840893in}}{\pgfqpoint{7.484647in}{0.836502in}}{\pgfqpoint{7.476833in}{0.828689in}}%
\pgfpathcurveto{\pgfqpoint{7.469020in}{0.820875in}}{\pgfqpoint{7.464630in}{0.810276in}}{\pgfqpoint{7.464630in}{0.799226in}}%
\pgfpathcurveto{\pgfqpoint{7.464630in}{0.788176in}}{\pgfqpoint{7.469020in}{0.777577in}}{\pgfqpoint{7.476833in}{0.769763in}}%
\pgfpathcurveto{\pgfqpoint{7.484647in}{0.761950in}}{\pgfqpoint{7.495246in}{0.757559in}}{\pgfqpoint{7.506296in}{0.757559in}}%
\pgfpathlineto{\pgfqpoint{7.506296in}{0.757559in}}%
\pgfpathclose%
\pgfusepath{stroke}%
\end{pgfscope}%
\begin{pgfscope}%
\pgfpathrectangle{\pgfqpoint{7.394209in}{0.375000in}}{\pgfqpoint{6.356833in}{5.175000in}}%
\pgfusepath{clip}%
\pgfsetbuttcap%
\pgfsetroundjoin%
\pgfsetlinewidth{1.003750pt}%
\definecolor{currentstroke}{rgb}{0.827451,0.827451,0.827451}%
\pgfsetstrokecolor{currentstroke}%
\pgfsetdash{}{0pt}%
\pgfpathmoveto{\pgfqpoint{7.822158in}{0.514864in}}%
\pgfpathcurveto{\pgfqpoint{7.833208in}{0.514864in}}{\pgfqpoint{7.843807in}{0.519254in}}{\pgfqpoint{7.851620in}{0.527068in}}%
\pgfpathcurveto{\pgfqpoint{7.859434in}{0.534882in}}{\pgfqpoint{7.863824in}{0.545481in}}{\pgfqpoint{7.863824in}{0.556531in}}%
\pgfpathcurveto{\pgfqpoint{7.863824in}{0.567581in}}{\pgfqpoint{7.859434in}{0.578180in}}{\pgfqpoint{7.851620in}{0.585994in}}%
\pgfpathcurveto{\pgfqpoint{7.843807in}{0.593807in}}{\pgfqpoint{7.833208in}{0.598198in}}{\pgfqpoint{7.822158in}{0.598198in}}%
\pgfpathcurveto{\pgfqpoint{7.811107in}{0.598198in}}{\pgfqpoint{7.800508in}{0.593807in}}{\pgfqpoint{7.792695in}{0.585994in}}%
\pgfpathcurveto{\pgfqpoint{7.784881in}{0.578180in}}{\pgfqpoint{7.780491in}{0.567581in}}{\pgfqpoint{7.780491in}{0.556531in}}%
\pgfpathcurveto{\pgfqpoint{7.780491in}{0.545481in}}{\pgfqpoint{7.784881in}{0.534882in}}{\pgfqpoint{7.792695in}{0.527068in}}%
\pgfpathcurveto{\pgfqpoint{7.800508in}{0.519254in}}{\pgfqpoint{7.811107in}{0.514864in}}{\pgfqpoint{7.822158in}{0.514864in}}%
\pgfpathlineto{\pgfqpoint{7.822158in}{0.514864in}}%
\pgfpathclose%
\pgfusepath{stroke}%
\end{pgfscope}%
\begin{pgfscope}%
\pgfpathrectangle{\pgfqpoint{7.394209in}{0.375000in}}{\pgfqpoint{6.356833in}{5.175000in}}%
\pgfusepath{clip}%
\pgfsetbuttcap%
\pgfsetroundjoin%
\pgfsetlinewidth{1.003750pt}%
\definecolor{currentstroke}{rgb}{0.827451,0.827451,0.827451}%
\pgfsetstrokecolor{currentstroke}%
\pgfsetdash{}{0pt}%
\pgfpathmoveto{\pgfqpoint{9.627852in}{4.073652in}}%
\pgfpathcurveto{\pgfqpoint{9.638902in}{4.073652in}}{\pgfqpoint{9.649501in}{4.078043in}}{\pgfqpoint{9.657315in}{4.085856in}}%
\pgfpathcurveto{\pgfqpoint{9.665129in}{4.093670in}}{\pgfqpoint{9.669519in}{4.104269in}}{\pgfqpoint{9.669519in}{4.115319in}}%
\pgfpathcurveto{\pgfqpoint{9.669519in}{4.126369in}}{\pgfqpoint{9.665129in}{4.136968in}}{\pgfqpoint{9.657315in}{4.144782in}}%
\pgfpathcurveto{\pgfqpoint{9.649501in}{4.152595in}}{\pgfqpoint{9.638902in}{4.156986in}}{\pgfqpoint{9.627852in}{4.156986in}}%
\pgfpathcurveto{\pgfqpoint{9.616802in}{4.156986in}}{\pgfqpoint{9.606203in}{4.152595in}}{\pgfqpoint{9.598389in}{4.144782in}}%
\pgfpathcurveto{\pgfqpoint{9.590576in}{4.136968in}}{\pgfqpoint{9.586186in}{4.126369in}}{\pgfqpoint{9.586186in}{4.115319in}}%
\pgfpathcurveto{\pgfqpoint{9.586186in}{4.104269in}}{\pgfqpoint{9.590576in}{4.093670in}}{\pgfqpoint{9.598389in}{4.085856in}}%
\pgfpathcurveto{\pgfqpoint{9.606203in}{4.078043in}}{\pgfqpoint{9.616802in}{4.073652in}}{\pgfqpoint{9.627852in}{4.073652in}}%
\pgfpathlineto{\pgfqpoint{9.627852in}{4.073652in}}%
\pgfpathclose%
\pgfusepath{stroke}%
\end{pgfscope}%
\begin{pgfscope}%
\pgfpathrectangle{\pgfqpoint{7.394209in}{0.375000in}}{\pgfqpoint{6.356833in}{5.175000in}}%
\pgfusepath{clip}%
\pgfsetbuttcap%
\pgfsetroundjoin%
\pgfsetlinewidth{1.003750pt}%
\definecolor{currentstroke}{rgb}{0.827451,0.827451,0.827451}%
\pgfsetstrokecolor{currentstroke}%
\pgfsetdash{}{0pt}%
\pgfpathmoveto{\pgfqpoint{11.439122in}{4.807663in}}%
\pgfpathcurveto{\pgfqpoint{11.450172in}{4.807663in}}{\pgfqpoint{11.460771in}{4.812053in}}{\pgfqpoint{11.468585in}{4.819867in}}%
\pgfpathcurveto{\pgfqpoint{11.476398in}{4.827680in}}{\pgfqpoint{11.480789in}{4.838280in}}{\pgfqpoint{11.480789in}{4.849330in}}%
\pgfpathcurveto{\pgfqpoint{11.480789in}{4.860380in}}{\pgfqpoint{11.476398in}{4.870979in}}{\pgfqpoint{11.468585in}{4.878792in}}%
\pgfpathcurveto{\pgfqpoint{11.460771in}{4.886606in}}{\pgfqpoint{11.450172in}{4.890996in}}{\pgfqpoint{11.439122in}{4.890996in}}%
\pgfpathcurveto{\pgfqpoint{11.428072in}{4.890996in}}{\pgfqpoint{11.417473in}{4.886606in}}{\pgfqpoint{11.409659in}{4.878792in}}%
\pgfpathcurveto{\pgfqpoint{11.401846in}{4.870979in}}{\pgfqpoint{11.397455in}{4.860380in}}{\pgfqpoint{11.397455in}{4.849330in}}%
\pgfpathcurveto{\pgfqpoint{11.397455in}{4.838280in}}{\pgfqpoint{11.401846in}{4.827680in}}{\pgfqpoint{11.409659in}{4.819867in}}%
\pgfpathcurveto{\pgfqpoint{11.417473in}{4.812053in}}{\pgfqpoint{11.428072in}{4.807663in}}{\pgfqpoint{11.439122in}{4.807663in}}%
\pgfpathlineto{\pgfqpoint{11.439122in}{4.807663in}}%
\pgfpathclose%
\pgfusepath{stroke}%
\end{pgfscope}%
\begin{pgfscope}%
\pgfpathrectangle{\pgfqpoint{7.394209in}{0.375000in}}{\pgfqpoint{6.356833in}{5.175000in}}%
\pgfusepath{clip}%
\pgfsetbuttcap%
\pgfsetroundjoin%
\pgfsetlinewidth{1.003750pt}%
\definecolor{currentstroke}{rgb}{0.827451,0.827451,0.827451}%
\pgfsetstrokecolor{currentstroke}%
\pgfsetdash{}{0pt}%
\pgfpathmoveto{\pgfqpoint{8.291906in}{1.157401in}}%
\pgfpathcurveto{\pgfqpoint{8.302956in}{1.157401in}}{\pgfqpoint{8.313555in}{1.161791in}}{\pgfqpoint{8.321369in}{1.169605in}}%
\pgfpathcurveto{\pgfqpoint{8.329182in}{1.177419in}}{\pgfqpoint{8.333573in}{1.188018in}}{\pgfqpoint{8.333573in}{1.199068in}}%
\pgfpathcurveto{\pgfqpoint{8.333573in}{1.210118in}}{\pgfqpoint{8.329182in}{1.220717in}}{\pgfqpoint{8.321369in}{1.228531in}}%
\pgfpathcurveto{\pgfqpoint{8.313555in}{1.236344in}}{\pgfqpoint{8.302956in}{1.240734in}}{\pgfqpoint{8.291906in}{1.240734in}}%
\pgfpathcurveto{\pgfqpoint{8.280856in}{1.240734in}}{\pgfqpoint{8.270257in}{1.236344in}}{\pgfqpoint{8.262443in}{1.228531in}}%
\pgfpathcurveto{\pgfqpoint{8.254630in}{1.220717in}}{\pgfqpoint{8.250239in}{1.210118in}}{\pgfqpoint{8.250239in}{1.199068in}}%
\pgfpathcurveto{\pgfqpoint{8.250239in}{1.188018in}}{\pgfqpoint{8.254630in}{1.177419in}}{\pgfqpoint{8.262443in}{1.169605in}}%
\pgfpathcurveto{\pgfqpoint{8.270257in}{1.161791in}}{\pgfqpoint{8.280856in}{1.157401in}}{\pgfqpoint{8.291906in}{1.157401in}}%
\pgfpathlineto{\pgfqpoint{8.291906in}{1.157401in}}%
\pgfpathclose%
\pgfusepath{stroke}%
\end{pgfscope}%
\begin{pgfscope}%
\pgfpathrectangle{\pgfqpoint{7.394209in}{0.375000in}}{\pgfqpoint{6.356833in}{5.175000in}}%
\pgfusepath{clip}%
\pgfsetbuttcap%
\pgfsetroundjoin%
\pgfsetlinewidth{1.003750pt}%
\definecolor{currentstroke}{rgb}{0.827451,0.827451,0.827451}%
\pgfsetstrokecolor{currentstroke}%
\pgfsetdash{}{0pt}%
\pgfpathmoveto{\pgfqpoint{8.884034in}{2.755708in}}%
\pgfpathcurveto{\pgfqpoint{8.895084in}{2.755708in}}{\pgfqpoint{8.905683in}{2.760098in}}{\pgfqpoint{8.913496in}{2.767912in}}%
\pgfpathcurveto{\pgfqpoint{8.921310in}{2.775726in}}{\pgfqpoint{8.925700in}{2.786325in}}{\pgfqpoint{8.925700in}{2.797375in}}%
\pgfpathcurveto{\pgfqpoint{8.925700in}{2.808425in}}{\pgfqpoint{8.921310in}{2.819024in}}{\pgfqpoint{8.913496in}{2.826837in}}%
\pgfpathcurveto{\pgfqpoint{8.905683in}{2.834651in}}{\pgfqpoint{8.895084in}{2.839041in}}{\pgfqpoint{8.884034in}{2.839041in}}%
\pgfpathcurveto{\pgfqpoint{8.872983in}{2.839041in}}{\pgfqpoint{8.862384in}{2.834651in}}{\pgfqpoint{8.854571in}{2.826837in}}%
\pgfpathcurveto{\pgfqpoint{8.846757in}{2.819024in}}{\pgfqpoint{8.842367in}{2.808425in}}{\pgfqpoint{8.842367in}{2.797375in}}%
\pgfpathcurveto{\pgfqpoint{8.842367in}{2.786325in}}{\pgfqpoint{8.846757in}{2.775726in}}{\pgfqpoint{8.854571in}{2.767912in}}%
\pgfpathcurveto{\pgfqpoint{8.862384in}{2.760098in}}{\pgfqpoint{8.872983in}{2.755708in}}{\pgfqpoint{8.884034in}{2.755708in}}%
\pgfpathlineto{\pgfqpoint{8.884034in}{2.755708in}}%
\pgfpathclose%
\pgfusepath{stroke}%
\end{pgfscope}%
\begin{pgfscope}%
\pgfpathrectangle{\pgfqpoint{7.394209in}{0.375000in}}{\pgfqpoint{6.356833in}{5.175000in}}%
\pgfusepath{clip}%
\pgfsetbuttcap%
\pgfsetroundjoin%
\pgfsetlinewidth{1.003750pt}%
\definecolor{currentstroke}{rgb}{0.827451,0.827451,0.827451}%
\pgfsetstrokecolor{currentstroke}%
\pgfsetdash{}{0pt}%
\pgfpathmoveto{\pgfqpoint{7.482027in}{0.754133in}}%
\pgfpathcurveto{\pgfqpoint{7.493077in}{0.754133in}}{\pgfqpoint{7.503676in}{0.758523in}}{\pgfqpoint{7.511490in}{0.766337in}}%
\pgfpathcurveto{\pgfqpoint{7.519303in}{0.774151in}}{\pgfqpoint{7.523693in}{0.784750in}}{\pgfqpoint{7.523693in}{0.795800in}}%
\pgfpathcurveto{\pgfqpoint{7.523693in}{0.806850in}}{\pgfqpoint{7.519303in}{0.817449in}}{\pgfqpoint{7.511490in}{0.825263in}}%
\pgfpathcurveto{\pgfqpoint{7.503676in}{0.833076in}}{\pgfqpoint{7.493077in}{0.837466in}}{\pgfqpoint{7.482027in}{0.837466in}}%
\pgfpathcurveto{\pgfqpoint{7.470977in}{0.837466in}}{\pgfqpoint{7.460378in}{0.833076in}}{\pgfqpoint{7.452564in}{0.825263in}}%
\pgfpathcurveto{\pgfqpoint{7.444750in}{0.817449in}}{\pgfqpoint{7.440360in}{0.806850in}}{\pgfqpoint{7.440360in}{0.795800in}}%
\pgfpathcurveto{\pgfqpoint{7.440360in}{0.784750in}}{\pgfqpoint{7.444750in}{0.774151in}}{\pgfqpoint{7.452564in}{0.766337in}}%
\pgfpathcurveto{\pgfqpoint{7.460378in}{0.758523in}}{\pgfqpoint{7.470977in}{0.754133in}}{\pgfqpoint{7.482027in}{0.754133in}}%
\pgfpathlineto{\pgfqpoint{7.482027in}{0.754133in}}%
\pgfpathclose%
\pgfusepath{stroke}%
\end{pgfscope}%
\begin{pgfscope}%
\pgfpathrectangle{\pgfqpoint{7.394209in}{0.375000in}}{\pgfqpoint{6.356833in}{5.175000in}}%
\pgfusepath{clip}%
\pgfsetbuttcap%
\pgfsetroundjoin%
\pgfsetlinewidth{1.003750pt}%
\definecolor{currentstroke}{rgb}{0.827451,0.827451,0.827451}%
\pgfsetstrokecolor{currentstroke}%
\pgfsetdash{}{0pt}%
\pgfpathmoveto{\pgfqpoint{11.518034in}{4.807663in}}%
\pgfpathcurveto{\pgfqpoint{11.529084in}{4.807663in}}{\pgfqpoint{11.539683in}{4.812053in}}{\pgfqpoint{11.547497in}{4.819867in}}%
\pgfpathcurveto{\pgfqpoint{11.555310in}{4.827680in}}{\pgfqpoint{11.559701in}{4.838280in}}{\pgfqpoint{11.559701in}{4.849330in}}%
\pgfpathcurveto{\pgfqpoint{11.559701in}{4.860380in}}{\pgfqpoint{11.555310in}{4.870979in}}{\pgfqpoint{11.547497in}{4.878792in}}%
\pgfpathcurveto{\pgfqpoint{11.539683in}{4.886606in}}{\pgfqpoint{11.529084in}{4.890996in}}{\pgfqpoint{11.518034in}{4.890996in}}%
\pgfpathcurveto{\pgfqpoint{11.506984in}{4.890996in}}{\pgfqpoint{11.496385in}{4.886606in}}{\pgfqpoint{11.488571in}{4.878792in}}%
\pgfpathcurveto{\pgfqpoint{11.480758in}{4.870979in}}{\pgfqpoint{11.476367in}{4.860380in}}{\pgfqpoint{11.476367in}{4.849330in}}%
\pgfpathcurveto{\pgfqpoint{11.476367in}{4.838280in}}{\pgfqpoint{11.480758in}{4.827680in}}{\pgfqpoint{11.488571in}{4.819867in}}%
\pgfpathcurveto{\pgfqpoint{11.496385in}{4.812053in}}{\pgfqpoint{11.506984in}{4.807663in}}{\pgfqpoint{11.518034in}{4.807663in}}%
\pgfpathlineto{\pgfqpoint{11.518034in}{4.807663in}}%
\pgfpathclose%
\pgfusepath{stroke}%
\end{pgfscope}%
\begin{pgfscope}%
\pgfpathrectangle{\pgfqpoint{7.394209in}{0.375000in}}{\pgfqpoint{6.356833in}{5.175000in}}%
\pgfusepath{clip}%
\pgfsetbuttcap%
\pgfsetroundjoin%
\pgfsetlinewidth{1.003750pt}%
\definecolor{currentstroke}{rgb}{0.827451,0.827451,0.827451}%
\pgfsetstrokecolor{currentstroke}%
\pgfsetdash{}{0pt}%
\pgfpathmoveto{\pgfqpoint{11.266817in}{5.388832in}}%
\pgfpathcurveto{\pgfqpoint{11.277867in}{5.388832in}}{\pgfqpoint{11.288466in}{5.393222in}}{\pgfqpoint{11.296280in}{5.401036in}}%
\pgfpathcurveto{\pgfqpoint{11.304093in}{5.408850in}}{\pgfqpoint{11.308484in}{5.419449in}}{\pgfqpoint{11.308484in}{5.430499in}}%
\pgfpathcurveto{\pgfqpoint{11.308484in}{5.441549in}}{\pgfqpoint{11.304093in}{5.452148in}}{\pgfqpoint{11.296280in}{5.459962in}}%
\pgfpathcurveto{\pgfqpoint{11.288466in}{5.467775in}}{\pgfqpoint{11.277867in}{5.472165in}}{\pgfqpoint{11.266817in}{5.472165in}}%
\pgfpathcurveto{\pgfqpoint{11.255767in}{5.472165in}}{\pgfqpoint{11.245168in}{5.467775in}}{\pgfqpoint{11.237354in}{5.459962in}}%
\pgfpathcurveto{\pgfqpoint{11.229541in}{5.452148in}}{\pgfqpoint{11.225150in}{5.441549in}}{\pgfqpoint{11.225150in}{5.430499in}}%
\pgfpathcurveto{\pgfqpoint{11.225150in}{5.419449in}}{\pgfqpoint{11.229541in}{5.408850in}}{\pgfqpoint{11.237354in}{5.401036in}}%
\pgfpathcurveto{\pgfqpoint{11.245168in}{5.393222in}}{\pgfqpoint{11.255767in}{5.388832in}}{\pgfqpoint{11.266817in}{5.388832in}}%
\pgfpathlineto{\pgfqpoint{11.266817in}{5.388832in}}%
\pgfpathclose%
\pgfusepath{stroke}%
\end{pgfscope}%
\begin{pgfscope}%
\pgfpathrectangle{\pgfqpoint{7.394209in}{0.375000in}}{\pgfqpoint{6.356833in}{5.175000in}}%
\pgfusepath{clip}%
\pgfsetbuttcap%
\pgfsetroundjoin%
\pgfsetlinewidth{1.003750pt}%
\definecolor{currentstroke}{rgb}{0.827451,0.827451,0.827451}%
\pgfsetstrokecolor{currentstroke}%
\pgfsetdash{}{0pt}%
\pgfpathmoveto{\pgfqpoint{9.499476in}{3.419732in}}%
\pgfpathcurveto{\pgfqpoint{9.510526in}{3.419732in}}{\pgfqpoint{9.521125in}{3.424122in}}{\pgfqpoint{9.528938in}{3.431936in}}%
\pgfpathcurveto{\pgfqpoint{9.536752in}{3.439749in}}{\pgfqpoint{9.541142in}{3.450348in}}{\pgfqpoint{9.541142in}{3.461399in}}%
\pgfpathcurveto{\pgfqpoint{9.541142in}{3.472449in}}{\pgfqpoint{9.536752in}{3.483048in}}{\pgfqpoint{9.528938in}{3.490861in}}%
\pgfpathcurveto{\pgfqpoint{9.521125in}{3.498675in}}{\pgfqpoint{9.510526in}{3.503065in}}{\pgfqpoint{9.499476in}{3.503065in}}%
\pgfpathcurveto{\pgfqpoint{9.488426in}{3.503065in}}{\pgfqpoint{9.477827in}{3.498675in}}{\pgfqpoint{9.470013in}{3.490861in}}%
\pgfpathcurveto{\pgfqpoint{9.462199in}{3.483048in}}{\pgfqpoint{9.457809in}{3.472449in}}{\pgfqpoint{9.457809in}{3.461399in}}%
\pgfpathcurveto{\pgfqpoint{9.457809in}{3.450348in}}{\pgfqpoint{9.462199in}{3.439749in}}{\pgfqpoint{9.470013in}{3.431936in}}%
\pgfpathcurveto{\pgfqpoint{9.477827in}{3.424122in}}{\pgfqpoint{9.488426in}{3.419732in}}{\pgfqpoint{9.499476in}{3.419732in}}%
\pgfpathlineto{\pgfqpoint{9.499476in}{3.419732in}}%
\pgfpathclose%
\pgfusepath{stroke}%
\end{pgfscope}%
\begin{pgfscope}%
\pgfpathrectangle{\pgfqpoint{7.394209in}{0.375000in}}{\pgfqpoint{6.356833in}{5.175000in}}%
\pgfusepath{clip}%
\pgfsetbuttcap%
\pgfsetroundjoin%
\pgfsetlinewidth{1.003750pt}%
\definecolor{currentstroke}{rgb}{0.827451,0.827451,0.827451}%
\pgfsetstrokecolor{currentstroke}%
\pgfsetdash{}{0pt}%
\pgfpathmoveto{\pgfqpoint{7.470228in}{0.931414in}}%
\pgfpathcurveto{\pgfqpoint{7.481278in}{0.931414in}}{\pgfqpoint{7.491877in}{0.935804in}}{\pgfqpoint{7.499691in}{0.943618in}}%
\pgfpathcurveto{\pgfqpoint{7.507505in}{0.951431in}}{\pgfqpoint{7.511895in}{0.962031in}}{\pgfqpoint{7.511895in}{0.973081in}}%
\pgfpathcurveto{\pgfqpoint{7.511895in}{0.984131in}}{\pgfqpoint{7.507505in}{0.994730in}}{\pgfqpoint{7.499691in}{1.002543in}}%
\pgfpathcurveto{\pgfqpoint{7.491877in}{1.010357in}}{\pgfqpoint{7.481278in}{1.014747in}}{\pgfqpoint{7.470228in}{1.014747in}}%
\pgfpathcurveto{\pgfqpoint{7.459178in}{1.014747in}}{\pgfqpoint{7.448579in}{1.010357in}}{\pgfqpoint{7.440765in}{1.002543in}}%
\pgfpathcurveto{\pgfqpoint{7.432952in}{0.994730in}}{\pgfqpoint{7.428562in}{0.984131in}}{\pgfqpoint{7.428562in}{0.973081in}}%
\pgfpathcurveto{\pgfqpoint{7.428562in}{0.962031in}}{\pgfqpoint{7.432952in}{0.951431in}}{\pgfqpoint{7.440765in}{0.943618in}}%
\pgfpathcurveto{\pgfqpoint{7.448579in}{0.935804in}}{\pgfqpoint{7.459178in}{0.931414in}}{\pgfqpoint{7.470228in}{0.931414in}}%
\pgfpathlineto{\pgfqpoint{7.470228in}{0.931414in}}%
\pgfpathclose%
\pgfusepath{stroke}%
\end{pgfscope}%
\begin{pgfscope}%
\pgfpathrectangle{\pgfqpoint{7.394209in}{0.375000in}}{\pgfqpoint{6.356833in}{5.175000in}}%
\pgfusepath{clip}%
\pgfsetbuttcap%
\pgfsetroundjoin%
\pgfsetlinewidth{1.003750pt}%
\definecolor{currentstroke}{rgb}{0.827451,0.827451,0.827451}%
\pgfsetstrokecolor{currentstroke}%
\pgfsetdash{}{0pt}%
\pgfpathmoveto{\pgfqpoint{10.389601in}{5.394086in}}%
\pgfpathcurveto{\pgfqpoint{10.400652in}{5.394086in}}{\pgfqpoint{10.411251in}{5.398477in}}{\pgfqpoint{10.419064in}{5.406290in}}%
\pgfpathcurveto{\pgfqpoint{10.426878in}{5.414104in}}{\pgfqpoint{10.431268in}{5.424703in}}{\pgfqpoint{10.431268in}{5.435753in}}%
\pgfpathcurveto{\pgfqpoint{10.431268in}{5.446803in}}{\pgfqpoint{10.426878in}{5.457402in}}{\pgfqpoint{10.419064in}{5.465216in}}%
\pgfpathcurveto{\pgfqpoint{10.411251in}{5.473030in}}{\pgfqpoint{10.400652in}{5.477420in}}{\pgfqpoint{10.389601in}{5.477420in}}%
\pgfpathcurveto{\pgfqpoint{10.378551in}{5.477420in}}{\pgfqpoint{10.367952in}{5.473030in}}{\pgfqpoint{10.360139in}{5.465216in}}%
\pgfpathcurveto{\pgfqpoint{10.352325in}{5.457402in}}{\pgfqpoint{10.347935in}{5.446803in}}{\pgfqpoint{10.347935in}{5.435753in}}%
\pgfpathcurveto{\pgfqpoint{10.347935in}{5.424703in}}{\pgfqpoint{10.352325in}{5.414104in}}{\pgfqpoint{10.360139in}{5.406290in}}%
\pgfpathcurveto{\pgfqpoint{10.367952in}{5.398477in}}{\pgfqpoint{10.378551in}{5.394086in}}{\pgfqpoint{10.389601in}{5.394086in}}%
\pgfpathlineto{\pgfqpoint{10.389601in}{5.394086in}}%
\pgfpathclose%
\pgfusepath{stroke}%
\end{pgfscope}%
\begin{pgfscope}%
\pgfpathrectangle{\pgfqpoint{7.394209in}{0.375000in}}{\pgfqpoint{6.356833in}{5.175000in}}%
\pgfusepath{clip}%
\pgfsetbuttcap%
\pgfsetroundjoin%
\pgfsetlinewidth{1.003750pt}%
\definecolor{currentstroke}{rgb}{0.827451,0.827451,0.827451}%
\pgfsetstrokecolor{currentstroke}%
\pgfsetdash{}{0pt}%
\pgfpathmoveto{\pgfqpoint{11.555431in}{5.308935in}}%
\pgfpathcurveto{\pgfqpoint{11.566481in}{5.308935in}}{\pgfqpoint{11.577080in}{5.313325in}}{\pgfqpoint{11.584894in}{5.321139in}}%
\pgfpathcurveto{\pgfqpoint{11.592707in}{5.328953in}}{\pgfqpoint{11.597098in}{5.339552in}}{\pgfqpoint{11.597098in}{5.350602in}}%
\pgfpathcurveto{\pgfqpoint{11.597098in}{5.361652in}}{\pgfqpoint{11.592707in}{5.372251in}}{\pgfqpoint{11.584894in}{5.380065in}}%
\pgfpathcurveto{\pgfqpoint{11.577080in}{5.387878in}}{\pgfqpoint{11.566481in}{5.392268in}}{\pgfqpoint{11.555431in}{5.392268in}}%
\pgfpathcurveto{\pgfqpoint{11.544381in}{5.392268in}}{\pgfqpoint{11.533782in}{5.387878in}}{\pgfqpoint{11.525968in}{5.380065in}}%
\pgfpathcurveto{\pgfqpoint{11.518155in}{5.372251in}}{\pgfqpoint{11.513764in}{5.361652in}}{\pgfqpoint{11.513764in}{5.350602in}}%
\pgfpathcurveto{\pgfqpoint{11.513764in}{5.339552in}}{\pgfqpoint{11.518155in}{5.328953in}}{\pgfqpoint{11.525968in}{5.321139in}}%
\pgfpathcurveto{\pgfqpoint{11.533782in}{5.313325in}}{\pgfqpoint{11.544381in}{5.308935in}}{\pgfqpoint{11.555431in}{5.308935in}}%
\pgfpathlineto{\pgfqpoint{11.555431in}{5.308935in}}%
\pgfpathclose%
\pgfusepath{stroke}%
\end{pgfscope}%
\begin{pgfscope}%
\pgfpathrectangle{\pgfqpoint{7.394209in}{0.375000in}}{\pgfqpoint{6.356833in}{5.175000in}}%
\pgfusepath{clip}%
\pgfsetbuttcap%
\pgfsetroundjoin%
\pgfsetlinewidth{1.003750pt}%
\definecolor{currentstroke}{rgb}{0.827451,0.827451,0.827451}%
\pgfsetstrokecolor{currentstroke}%
\pgfsetdash{}{0pt}%
\pgfpathmoveto{\pgfqpoint{10.214507in}{5.039669in}}%
\pgfpathcurveto{\pgfqpoint{10.225558in}{5.039669in}}{\pgfqpoint{10.236157in}{5.044060in}}{\pgfqpoint{10.243970in}{5.051873in}}%
\pgfpathcurveto{\pgfqpoint{10.251784in}{5.059687in}}{\pgfqpoint{10.256174in}{5.070286in}}{\pgfqpoint{10.256174in}{5.081336in}}%
\pgfpathcurveto{\pgfqpoint{10.256174in}{5.092386in}}{\pgfqpoint{10.251784in}{5.102985in}}{\pgfqpoint{10.243970in}{5.110799in}}%
\pgfpathcurveto{\pgfqpoint{10.236157in}{5.118612in}}{\pgfqpoint{10.225558in}{5.123003in}}{\pgfqpoint{10.214507in}{5.123003in}}%
\pgfpathcurveto{\pgfqpoint{10.203457in}{5.123003in}}{\pgfqpoint{10.192858in}{5.118612in}}{\pgfqpoint{10.185045in}{5.110799in}}%
\pgfpathcurveto{\pgfqpoint{10.177231in}{5.102985in}}{\pgfqpoint{10.172841in}{5.092386in}}{\pgfqpoint{10.172841in}{5.081336in}}%
\pgfpathcurveto{\pgfqpoint{10.172841in}{5.070286in}}{\pgfqpoint{10.177231in}{5.059687in}}{\pgfqpoint{10.185045in}{5.051873in}}%
\pgfpathcurveto{\pgfqpoint{10.192858in}{5.044060in}}{\pgfqpoint{10.203457in}{5.039669in}}{\pgfqpoint{10.214507in}{5.039669in}}%
\pgfpathlineto{\pgfqpoint{10.214507in}{5.039669in}}%
\pgfpathclose%
\pgfusepath{stroke}%
\end{pgfscope}%
\begin{pgfscope}%
\pgfpathrectangle{\pgfqpoint{7.394209in}{0.375000in}}{\pgfqpoint{6.356833in}{5.175000in}}%
\pgfusepath{clip}%
\pgfsetbuttcap%
\pgfsetroundjoin%
\pgfsetlinewidth{1.003750pt}%
\definecolor{currentstroke}{rgb}{0.827451,0.827451,0.827451}%
\pgfsetstrokecolor{currentstroke}%
\pgfsetdash{}{0pt}%
\pgfpathmoveto{\pgfqpoint{11.515299in}{5.359395in}}%
\pgfpathcurveto{\pgfqpoint{11.526349in}{5.359395in}}{\pgfqpoint{11.536948in}{5.363785in}}{\pgfqpoint{11.544762in}{5.371599in}}%
\pgfpathcurveto{\pgfqpoint{11.552576in}{5.379413in}}{\pgfqpoint{11.556966in}{5.390012in}}{\pgfqpoint{11.556966in}{5.401062in}}%
\pgfpathcurveto{\pgfqpoint{11.556966in}{5.412112in}}{\pgfqpoint{11.552576in}{5.422711in}}{\pgfqpoint{11.544762in}{5.430524in}}%
\pgfpathcurveto{\pgfqpoint{11.536948in}{5.438338in}}{\pgfqpoint{11.526349in}{5.442728in}}{\pgfqpoint{11.515299in}{5.442728in}}%
\pgfpathcurveto{\pgfqpoint{11.504249in}{5.442728in}}{\pgfqpoint{11.493650in}{5.438338in}}{\pgfqpoint{11.485837in}{5.430524in}}%
\pgfpathcurveto{\pgfqpoint{11.478023in}{5.422711in}}{\pgfqpoint{11.473633in}{5.412112in}}{\pgfqpoint{11.473633in}{5.401062in}}%
\pgfpathcurveto{\pgfqpoint{11.473633in}{5.390012in}}{\pgfqpoint{11.478023in}{5.379413in}}{\pgfqpoint{11.485837in}{5.371599in}}%
\pgfpathcurveto{\pgfqpoint{11.493650in}{5.363785in}}{\pgfqpoint{11.504249in}{5.359395in}}{\pgfqpoint{11.515299in}{5.359395in}}%
\pgfpathlineto{\pgfqpoint{11.515299in}{5.359395in}}%
\pgfpathclose%
\pgfusepath{stroke}%
\end{pgfscope}%
\begin{pgfscope}%
\pgfpathrectangle{\pgfqpoint{7.394209in}{0.375000in}}{\pgfqpoint{6.356833in}{5.175000in}}%
\pgfusepath{clip}%
\pgfsetbuttcap%
\pgfsetroundjoin%
\pgfsetlinewidth{1.003750pt}%
\definecolor{currentstroke}{rgb}{0.827451,0.827451,0.827451}%
\pgfsetstrokecolor{currentstroke}%
\pgfsetdash{}{0pt}%
\pgfpathmoveto{\pgfqpoint{8.159759in}{2.658461in}}%
\pgfpathcurveto{\pgfqpoint{8.170809in}{2.658461in}}{\pgfqpoint{8.181408in}{2.662852in}}{\pgfqpoint{8.189222in}{2.670665in}}%
\pgfpathcurveto{\pgfqpoint{8.197036in}{2.678479in}}{\pgfqpoint{8.201426in}{2.689078in}}{\pgfqpoint{8.201426in}{2.700128in}}%
\pgfpathcurveto{\pgfqpoint{8.201426in}{2.711178in}}{\pgfqpoint{8.197036in}{2.721777in}}{\pgfqpoint{8.189222in}{2.729591in}}%
\pgfpathcurveto{\pgfqpoint{8.181408in}{2.737405in}}{\pgfqpoint{8.170809in}{2.741795in}}{\pgfqpoint{8.159759in}{2.741795in}}%
\pgfpathcurveto{\pgfqpoint{8.148709in}{2.741795in}}{\pgfqpoint{8.138110in}{2.737405in}}{\pgfqpoint{8.130296in}{2.729591in}}%
\pgfpathcurveto{\pgfqpoint{8.122483in}{2.721777in}}{\pgfqpoint{8.118093in}{2.711178in}}{\pgfqpoint{8.118093in}{2.700128in}}%
\pgfpathcurveto{\pgfqpoint{8.118093in}{2.689078in}}{\pgfqpoint{8.122483in}{2.678479in}}{\pgfqpoint{8.130296in}{2.670665in}}%
\pgfpathcurveto{\pgfqpoint{8.138110in}{2.662852in}}{\pgfqpoint{8.148709in}{2.658461in}}{\pgfqpoint{8.159759in}{2.658461in}}%
\pgfpathlineto{\pgfqpoint{8.159759in}{2.658461in}}%
\pgfpathclose%
\pgfusepath{stroke}%
\end{pgfscope}%
\begin{pgfscope}%
\pgfpathrectangle{\pgfqpoint{7.394209in}{0.375000in}}{\pgfqpoint{6.356833in}{5.175000in}}%
\pgfusepath{clip}%
\pgfsetbuttcap%
\pgfsetroundjoin%
\pgfsetlinewidth{1.003750pt}%
\definecolor{currentstroke}{rgb}{0.827451,0.827451,0.827451}%
\pgfsetstrokecolor{currentstroke}%
\pgfsetdash{}{0pt}%
\pgfpathmoveto{\pgfqpoint{11.587325in}{5.424796in}}%
\pgfpathcurveto{\pgfqpoint{11.598376in}{5.424796in}}{\pgfqpoint{11.608975in}{5.429187in}}{\pgfqpoint{11.616788in}{5.437000in}}%
\pgfpathcurveto{\pgfqpoint{11.624602in}{5.444814in}}{\pgfqpoint{11.628992in}{5.455413in}}{\pgfqpoint{11.628992in}{5.466463in}}%
\pgfpathcurveto{\pgfqpoint{11.628992in}{5.477513in}}{\pgfqpoint{11.624602in}{5.488112in}}{\pgfqpoint{11.616788in}{5.495926in}}%
\pgfpathcurveto{\pgfqpoint{11.608975in}{5.503739in}}{\pgfqpoint{11.598376in}{5.508130in}}{\pgfqpoint{11.587325in}{5.508130in}}%
\pgfpathcurveto{\pgfqpoint{11.576275in}{5.508130in}}{\pgfqpoint{11.565676in}{5.503739in}}{\pgfqpoint{11.557863in}{5.495926in}}%
\pgfpathcurveto{\pgfqpoint{11.550049in}{5.488112in}}{\pgfqpoint{11.545659in}{5.477513in}}{\pgfqpoint{11.545659in}{5.466463in}}%
\pgfpathcurveto{\pgfqpoint{11.545659in}{5.455413in}}{\pgfqpoint{11.550049in}{5.444814in}}{\pgfqpoint{11.557863in}{5.437000in}}%
\pgfpathcurveto{\pgfqpoint{11.565676in}{5.429187in}}{\pgfqpoint{11.576275in}{5.424796in}}{\pgfqpoint{11.587325in}{5.424796in}}%
\pgfpathlineto{\pgfqpoint{11.587325in}{5.424796in}}%
\pgfpathclose%
\pgfusepath{stroke}%
\end{pgfscope}%
\begin{pgfscope}%
\pgfpathrectangle{\pgfqpoint{7.394209in}{0.375000in}}{\pgfqpoint{6.356833in}{5.175000in}}%
\pgfusepath{clip}%
\pgfsetbuttcap%
\pgfsetroundjoin%
\pgfsetlinewidth{1.003750pt}%
\definecolor{currentstroke}{rgb}{0.827451,0.827451,0.827451}%
\pgfsetstrokecolor{currentstroke}%
\pgfsetdash{}{0pt}%
\pgfpathmoveto{\pgfqpoint{7.672517in}{1.362470in}}%
\pgfpathcurveto{\pgfqpoint{7.683567in}{1.362470in}}{\pgfqpoint{7.694166in}{1.366861in}}{\pgfqpoint{7.701980in}{1.374674in}}%
\pgfpathcurveto{\pgfqpoint{7.709793in}{1.382488in}}{\pgfqpoint{7.714183in}{1.393087in}}{\pgfqpoint{7.714183in}{1.404137in}}%
\pgfpathcurveto{\pgfqpoint{7.714183in}{1.415187in}}{\pgfqpoint{7.709793in}{1.425786in}}{\pgfqpoint{7.701980in}{1.433600in}}%
\pgfpathcurveto{\pgfqpoint{7.694166in}{1.441414in}}{\pgfqpoint{7.683567in}{1.445804in}}{\pgfqpoint{7.672517in}{1.445804in}}%
\pgfpathcurveto{\pgfqpoint{7.661467in}{1.445804in}}{\pgfqpoint{7.650868in}{1.441414in}}{\pgfqpoint{7.643054in}{1.433600in}}%
\pgfpathcurveto{\pgfqpoint{7.635240in}{1.425786in}}{\pgfqpoint{7.630850in}{1.415187in}}{\pgfqpoint{7.630850in}{1.404137in}}%
\pgfpathcurveto{\pgfqpoint{7.630850in}{1.393087in}}{\pgfqpoint{7.635240in}{1.382488in}}{\pgfqpoint{7.643054in}{1.374674in}}%
\pgfpathcurveto{\pgfqpoint{7.650868in}{1.366861in}}{\pgfqpoint{7.661467in}{1.362470in}}{\pgfqpoint{7.672517in}{1.362470in}}%
\pgfpathlineto{\pgfqpoint{7.672517in}{1.362470in}}%
\pgfpathclose%
\pgfusepath{stroke}%
\end{pgfscope}%
\begin{pgfscope}%
\pgfpathrectangle{\pgfqpoint{7.394209in}{0.375000in}}{\pgfqpoint{6.356833in}{5.175000in}}%
\pgfusepath{clip}%
\pgfsetbuttcap%
\pgfsetroundjoin%
\pgfsetlinewidth{1.003750pt}%
\definecolor{currentstroke}{rgb}{0.827451,0.827451,0.827451}%
\pgfsetstrokecolor{currentstroke}%
\pgfsetdash{}{0pt}%
\pgfpathmoveto{\pgfqpoint{8.309795in}{2.491493in}}%
\pgfpathcurveto{\pgfqpoint{8.320845in}{2.491493in}}{\pgfqpoint{8.331444in}{2.495883in}}{\pgfqpoint{8.339258in}{2.503697in}}%
\pgfpathcurveto{\pgfqpoint{8.347071in}{2.511510in}}{\pgfqpoint{8.351461in}{2.522109in}}{\pgfqpoint{8.351461in}{2.533159in}}%
\pgfpathcurveto{\pgfqpoint{8.351461in}{2.544210in}}{\pgfqpoint{8.347071in}{2.554809in}}{\pgfqpoint{8.339258in}{2.562622in}}%
\pgfpathcurveto{\pgfqpoint{8.331444in}{2.570436in}}{\pgfqpoint{8.320845in}{2.574826in}}{\pgfqpoint{8.309795in}{2.574826in}}%
\pgfpathcurveto{\pgfqpoint{8.298745in}{2.574826in}}{\pgfqpoint{8.288146in}{2.570436in}}{\pgfqpoint{8.280332in}{2.562622in}}%
\pgfpathcurveto{\pgfqpoint{8.272518in}{2.554809in}}{\pgfqpoint{8.268128in}{2.544210in}}{\pgfqpoint{8.268128in}{2.533159in}}%
\pgfpathcurveto{\pgfqpoint{8.268128in}{2.522109in}}{\pgfqpoint{8.272518in}{2.511510in}}{\pgfqpoint{8.280332in}{2.503697in}}%
\pgfpathcurveto{\pgfqpoint{8.288146in}{2.495883in}}{\pgfqpoint{8.298745in}{2.491493in}}{\pgfqpoint{8.309795in}{2.491493in}}%
\pgfpathlineto{\pgfqpoint{8.309795in}{2.491493in}}%
\pgfpathclose%
\pgfusepath{stroke}%
\end{pgfscope}%
\begin{pgfscope}%
\pgfpathrectangle{\pgfqpoint{7.394209in}{0.375000in}}{\pgfqpoint{6.356833in}{5.175000in}}%
\pgfusepath{clip}%
\pgfsetbuttcap%
\pgfsetroundjoin%
\pgfsetlinewidth{1.003750pt}%
\definecolor{currentstroke}{rgb}{0.827451,0.827451,0.827451}%
\pgfsetstrokecolor{currentstroke}%
\pgfsetdash{}{0pt}%
\pgfpathmoveto{\pgfqpoint{11.192695in}{4.145108in}}%
\pgfpathcurveto{\pgfqpoint{11.203745in}{4.145108in}}{\pgfqpoint{11.214344in}{4.149498in}}{\pgfqpoint{11.222158in}{4.157312in}}%
\pgfpathcurveto{\pgfqpoint{11.229971in}{4.165126in}}{\pgfqpoint{11.234362in}{4.175725in}}{\pgfqpoint{11.234362in}{4.186775in}}%
\pgfpathcurveto{\pgfqpoint{11.234362in}{4.197825in}}{\pgfqpoint{11.229971in}{4.208424in}}{\pgfqpoint{11.222158in}{4.216237in}}%
\pgfpathcurveto{\pgfqpoint{11.214344in}{4.224051in}}{\pgfqpoint{11.203745in}{4.228441in}}{\pgfqpoint{11.192695in}{4.228441in}}%
\pgfpathcurveto{\pgfqpoint{11.181645in}{4.228441in}}{\pgfqpoint{11.171046in}{4.224051in}}{\pgfqpoint{11.163232in}{4.216237in}}%
\pgfpathcurveto{\pgfqpoint{11.155419in}{4.208424in}}{\pgfqpoint{11.151028in}{4.197825in}}{\pgfqpoint{11.151028in}{4.186775in}}%
\pgfpathcurveto{\pgfqpoint{11.151028in}{4.175725in}}{\pgfqpoint{11.155419in}{4.165126in}}{\pgfqpoint{11.163232in}{4.157312in}}%
\pgfpathcurveto{\pgfqpoint{11.171046in}{4.149498in}}{\pgfqpoint{11.181645in}{4.145108in}}{\pgfqpoint{11.192695in}{4.145108in}}%
\pgfpathlineto{\pgfqpoint{11.192695in}{4.145108in}}%
\pgfpathclose%
\pgfusepath{stroke}%
\end{pgfscope}%
\begin{pgfscope}%
\pgfpathrectangle{\pgfqpoint{7.394209in}{0.375000in}}{\pgfqpoint{6.356833in}{5.175000in}}%
\pgfusepath{clip}%
\pgfsetbuttcap%
\pgfsetroundjoin%
\pgfsetlinewidth{1.003750pt}%
\definecolor{currentstroke}{rgb}{0.827451,0.827451,0.827451}%
\pgfsetstrokecolor{currentstroke}%
\pgfsetdash{}{0pt}%
\pgfpathmoveto{\pgfqpoint{9.849864in}{3.187768in}}%
\pgfpathcurveto{\pgfqpoint{9.860914in}{3.187768in}}{\pgfqpoint{9.871513in}{3.192158in}}{\pgfqpoint{9.879327in}{3.199972in}}%
\pgfpathcurveto{\pgfqpoint{9.887140in}{3.207785in}}{\pgfqpoint{9.891531in}{3.218384in}}{\pgfqpoint{9.891531in}{3.229434in}}%
\pgfpathcurveto{\pgfqpoint{9.891531in}{3.240484in}}{\pgfqpoint{9.887140in}{3.251084in}}{\pgfqpoint{9.879327in}{3.258897in}}%
\pgfpathcurveto{\pgfqpoint{9.871513in}{3.266711in}}{\pgfqpoint{9.860914in}{3.271101in}}{\pgfqpoint{9.849864in}{3.271101in}}%
\pgfpathcurveto{\pgfqpoint{9.838814in}{3.271101in}}{\pgfqpoint{9.828215in}{3.266711in}}{\pgfqpoint{9.820401in}{3.258897in}}%
\pgfpathcurveto{\pgfqpoint{9.812588in}{3.251084in}}{\pgfqpoint{9.808197in}{3.240484in}}{\pgfqpoint{9.808197in}{3.229434in}}%
\pgfpathcurveto{\pgfqpoint{9.808197in}{3.218384in}}{\pgfqpoint{9.812588in}{3.207785in}}{\pgfqpoint{9.820401in}{3.199972in}}%
\pgfpathcurveto{\pgfqpoint{9.828215in}{3.192158in}}{\pgfqpoint{9.838814in}{3.187768in}}{\pgfqpoint{9.849864in}{3.187768in}}%
\pgfpathlineto{\pgfqpoint{9.849864in}{3.187768in}}%
\pgfpathclose%
\pgfusepath{stroke}%
\end{pgfscope}%
\begin{pgfscope}%
\pgfpathrectangle{\pgfqpoint{7.394209in}{0.375000in}}{\pgfqpoint{6.356833in}{5.175000in}}%
\pgfusepath{clip}%
\pgfsetbuttcap%
\pgfsetroundjoin%
\pgfsetlinewidth{1.003750pt}%
\definecolor{currentstroke}{rgb}{0.827451,0.827451,0.827451}%
\pgfsetstrokecolor{currentstroke}%
\pgfsetdash{}{0pt}%
\pgfpathmoveto{\pgfqpoint{11.178919in}{5.172242in}}%
\pgfpathcurveto{\pgfqpoint{11.189969in}{5.172242in}}{\pgfqpoint{11.200568in}{5.176632in}}{\pgfqpoint{11.208382in}{5.184446in}}%
\pgfpathcurveto{\pgfqpoint{11.216195in}{5.192260in}}{\pgfqpoint{11.220586in}{5.202859in}}{\pgfqpoint{11.220586in}{5.213909in}}%
\pgfpathcurveto{\pgfqpoint{11.220586in}{5.224959in}}{\pgfqpoint{11.216195in}{5.235558in}}{\pgfqpoint{11.208382in}{5.243372in}}%
\pgfpathcurveto{\pgfqpoint{11.200568in}{5.251185in}}{\pgfqpoint{11.189969in}{5.255575in}}{\pgfqpoint{11.178919in}{5.255575in}}%
\pgfpathcurveto{\pgfqpoint{11.167869in}{5.255575in}}{\pgfqpoint{11.157270in}{5.251185in}}{\pgfqpoint{11.149456in}{5.243372in}}%
\pgfpathcurveto{\pgfqpoint{11.141642in}{5.235558in}}{\pgfqpoint{11.137252in}{5.224959in}}{\pgfqpoint{11.137252in}{5.213909in}}%
\pgfpathcurveto{\pgfqpoint{11.137252in}{5.202859in}}{\pgfqpoint{11.141642in}{5.192260in}}{\pgfqpoint{11.149456in}{5.184446in}}%
\pgfpathcurveto{\pgfqpoint{11.157270in}{5.176632in}}{\pgfqpoint{11.167869in}{5.172242in}}{\pgfqpoint{11.178919in}{5.172242in}}%
\pgfpathlineto{\pgfqpoint{11.178919in}{5.172242in}}%
\pgfpathclose%
\pgfusepath{stroke}%
\end{pgfscope}%
\begin{pgfscope}%
\pgfpathrectangle{\pgfqpoint{7.394209in}{0.375000in}}{\pgfqpoint{6.356833in}{5.175000in}}%
\pgfusepath{clip}%
\pgfsetbuttcap%
\pgfsetroundjoin%
\pgfsetlinewidth{1.003750pt}%
\definecolor{currentstroke}{rgb}{0.827451,0.827451,0.827451}%
\pgfsetstrokecolor{currentstroke}%
\pgfsetdash{}{0pt}%
\pgfpathmoveto{\pgfqpoint{8.989779in}{3.066622in}}%
\pgfpathcurveto{\pgfqpoint{9.000829in}{3.066622in}}{\pgfqpoint{9.011428in}{3.071012in}}{\pgfqpoint{9.019242in}{3.078825in}}%
\pgfpathcurveto{\pgfqpoint{9.027055in}{3.086639in}}{\pgfqpoint{9.031445in}{3.097238in}}{\pgfqpoint{9.031445in}{3.108288in}}%
\pgfpathcurveto{\pgfqpoint{9.031445in}{3.119338in}}{\pgfqpoint{9.027055in}{3.129937in}}{\pgfqpoint{9.019242in}{3.137751in}}%
\pgfpathcurveto{\pgfqpoint{9.011428in}{3.145565in}}{\pgfqpoint{9.000829in}{3.149955in}}{\pgfqpoint{8.989779in}{3.149955in}}%
\pgfpathcurveto{\pgfqpoint{8.978729in}{3.149955in}}{\pgfqpoint{8.968130in}{3.145565in}}{\pgfqpoint{8.960316in}{3.137751in}}%
\pgfpathcurveto{\pgfqpoint{8.952502in}{3.129937in}}{\pgfqpoint{8.948112in}{3.119338in}}{\pgfqpoint{8.948112in}{3.108288in}}%
\pgfpathcurveto{\pgfqpoint{8.948112in}{3.097238in}}{\pgfqpoint{8.952502in}{3.086639in}}{\pgfqpoint{8.960316in}{3.078825in}}%
\pgfpathcurveto{\pgfqpoint{8.968130in}{3.071012in}}{\pgfqpoint{8.978729in}{3.066622in}}{\pgfqpoint{8.989779in}{3.066622in}}%
\pgfpathlineto{\pgfqpoint{8.989779in}{3.066622in}}%
\pgfpathclose%
\pgfusepath{stroke}%
\end{pgfscope}%
\begin{pgfscope}%
\pgfpathrectangle{\pgfqpoint{7.394209in}{0.375000in}}{\pgfqpoint{6.356833in}{5.175000in}}%
\pgfusepath{clip}%
\pgfsetbuttcap%
\pgfsetroundjoin%
\pgfsetlinewidth{1.003750pt}%
\definecolor{currentstroke}{rgb}{0.827451,0.827451,0.827451}%
\pgfsetstrokecolor{currentstroke}%
\pgfsetdash{}{0pt}%
\pgfpathmoveto{\pgfqpoint{10.377941in}{3.615888in}}%
\pgfpathcurveto{\pgfqpoint{10.388991in}{3.615888in}}{\pgfqpoint{10.399590in}{3.620278in}}{\pgfqpoint{10.407404in}{3.628092in}}%
\pgfpathcurveto{\pgfqpoint{10.415217in}{3.635905in}}{\pgfqpoint{10.419607in}{3.646504in}}{\pgfqpoint{10.419607in}{3.657555in}}%
\pgfpathcurveto{\pgfqpoint{10.419607in}{3.668605in}}{\pgfqpoint{10.415217in}{3.679204in}}{\pgfqpoint{10.407404in}{3.687017in}}%
\pgfpathcurveto{\pgfqpoint{10.399590in}{3.694831in}}{\pgfqpoint{10.388991in}{3.699221in}}{\pgfqpoint{10.377941in}{3.699221in}}%
\pgfpathcurveto{\pgfqpoint{10.366891in}{3.699221in}}{\pgfqpoint{10.356292in}{3.694831in}}{\pgfqpoint{10.348478in}{3.687017in}}%
\pgfpathcurveto{\pgfqpoint{10.340664in}{3.679204in}}{\pgfqpoint{10.336274in}{3.668605in}}{\pgfqpoint{10.336274in}{3.657555in}}%
\pgfpathcurveto{\pgfqpoint{10.336274in}{3.646504in}}{\pgfqpoint{10.340664in}{3.635905in}}{\pgfqpoint{10.348478in}{3.628092in}}%
\pgfpathcurveto{\pgfqpoint{10.356292in}{3.620278in}}{\pgfqpoint{10.366891in}{3.615888in}}{\pgfqpoint{10.377941in}{3.615888in}}%
\pgfpathlineto{\pgfqpoint{10.377941in}{3.615888in}}%
\pgfpathclose%
\pgfusepath{stroke}%
\end{pgfscope}%
\begin{pgfscope}%
\pgfpathrectangle{\pgfqpoint{7.394209in}{0.375000in}}{\pgfqpoint{6.356833in}{5.175000in}}%
\pgfusepath{clip}%
\pgfsetbuttcap%
\pgfsetroundjoin%
\pgfsetlinewidth{1.003750pt}%
\definecolor{currentstroke}{rgb}{0.827451,0.827451,0.827451}%
\pgfsetstrokecolor{currentstroke}%
\pgfsetdash{}{0pt}%
\pgfpathmoveto{\pgfqpoint{9.093802in}{1.481039in}}%
\pgfpathcurveto{\pgfqpoint{9.104852in}{1.481039in}}{\pgfqpoint{9.115451in}{1.485429in}}{\pgfqpoint{9.123265in}{1.493242in}}%
\pgfpathcurveto{\pgfqpoint{9.131078in}{1.501056in}}{\pgfqpoint{9.135469in}{1.511655in}}{\pgfqpoint{9.135469in}{1.522705in}}%
\pgfpathcurveto{\pgfqpoint{9.135469in}{1.533755in}}{\pgfqpoint{9.131078in}{1.544354in}}{\pgfqpoint{9.123265in}{1.552168in}}%
\pgfpathcurveto{\pgfqpoint{9.115451in}{1.559982in}}{\pgfqpoint{9.104852in}{1.564372in}}{\pgfqpoint{9.093802in}{1.564372in}}%
\pgfpathcurveto{\pgfqpoint{9.082752in}{1.564372in}}{\pgfqpoint{9.072153in}{1.559982in}}{\pgfqpoint{9.064339in}{1.552168in}}%
\pgfpathcurveto{\pgfqpoint{9.056525in}{1.544354in}}{\pgfqpoint{9.052135in}{1.533755in}}{\pgfqpoint{9.052135in}{1.522705in}}%
\pgfpathcurveto{\pgfqpoint{9.052135in}{1.511655in}}{\pgfqpoint{9.056525in}{1.501056in}}{\pgfqpoint{9.064339in}{1.493242in}}%
\pgfpathcurveto{\pgfqpoint{9.072153in}{1.485429in}}{\pgfqpoint{9.082752in}{1.481039in}}{\pgfqpoint{9.093802in}{1.481039in}}%
\pgfpathlineto{\pgfqpoint{9.093802in}{1.481039in}}%
\pgfpathclose%
\pgfusepath{stroke}%
\end{pgfscope}%
\begin{pgfscope}%
\pgfpathrectangle{\pgfqpoint{7.394209in}{0.375000in}}{\pgfqpoint{6.356833in}{5.175000in}}%
\pgfusepath{clip}%
\pgfsetbuttcap%
\pgfsetroundjoin%
\pgfsetlinewidth{1.003750pt}%
\definecolor{currentstroke}{rgb}{0.827451,0.827451,0.827451}%
\pgfsetstrokecolor{currentstroke}%
\pgfsetdash{}{0pt}%
\pgfpathmoveto{\pgfqpoint{9.274202in}{4.084545in}}%
\pgfpathcurveto{\pgfqpoint{9.285252in}{4.084545in}}{\pgfqpoint{9.295852in}{4.088935in}}{\pgfqpoint{9.303665in}{4.096749in}}%
\pgfpathcurveto{\pgfqpoint{9.311479in}{4.104563in}}{\pgfqpoint{9.315869in}{4.115162in}}{\pgfqpoint{9.315869in}{4.126212in}}%
\pgfpathcurveto{\pgfqpoint{9.315869in}{4.137262in}}{\pgfqpoint{9.311479in}{4.147861in}}{\pgfqpoint{9.303665in}{4.155674in}}%
\pgfpathcurveto{\pgfqpoint{9.295852in}{4.163488in}}{\pgfqpoint{9.285252in}{4.167878in}}{\pgfqpoint{9.274202in}{4.167878in}}%
\pgfpathcurveto{\pgfqpoint{9.263152in}{4.167878in}}{\pgfqpoint{9.252553in}{4.163488in}}{\pgfqpoint{9.244740in}{4.155674in}}%
\pgfpathcurveto{\pgfqpoint{9.236926in}{4.147861in}}{\pgfqpoint{9.232536in}{4.137262in}}{\pgfqpoint{9.232536in}{4.126212in}}%
\pgfpathcurveto{\pgfqpoint{9.232536in}{4.115162in}}{\pgfqpoint{9.236926in}{4.104563in}}{\pgfqpoint{9.244740in}{4.096749in}}%
\pgfpathcurveto{\pgfqpoint{9.252553in}{4.088935in}}{\pgfqpoint{9.263152in}{4.084545in}}{\pgfqpoint{9.274202in}{4.084545in}}%
\pgfpathlineto{\pgfqpoint{9.274202in}{4.084545in}}%
\pgfpathclose%
\pgfusepath{stroke}%
\end{pgfscope}%
\begin{pgfscope}%
\pgfpathrectangle{\pgfqpoint{7.394209in}{0.375000in}}{\pgfqpoint{6.356833in}{5.175000in}}%
\pgfusepath{clip}%
\pgfsetbuttcap%
\pgfsetroundjoin%
\pgfsetlinewidth{1.003750pt}%
\definecolor{currentstroke}{rgb}{0.827451,0.827451,0.827451}%
\pgfsetstrokecolor{currentstroke}%
\pgfsetdash{}{0pt}%
\pgfpathmoveto{\pgfqpoint{10.936288in}{5.505487in}}%
\pgfpathcurveto{\pgfqpoint{10.947338in}{5.505487in}}{\pgfqpoint{10.957937in}{5.509877in}}{\pgfqpoint{10.965751in}{5.517691in}}%
\pgfpathcurveto{\pgfqpoint{10.973564in}{5.525504in}}{\pgfqpoint{10.977955in}{5.536104in}}{\pgfqpoint{10.977955in}{5.547154in}}%
\pgfpathcurveto{\pgfqpoint{10.977955in}{5.558204in}}{\pgfqpoint{10.973564in}{5.568803in}}{\pgfqpoint{10.965751in}{5.576616in}}%
\pgfpathcurveto{\pgfqpoint{10.957937in}{5.584430in}}{\pgfqpoint{10.947338in}{5.588820in}}{\pgfqpoint{10.936288in}{5.588820in}}%
\pgfpathcurveto{\pgfqpoint{10.925238in}{5.588820in}}{\pgfqpoint{10.914639in}{5.584430in}}{\pgfqpoint{10.906825in}{5.576616in}}%
\pgfpathcurveto{\pgfqpoint{10.899011in}{5.568803in}}{\pgfqpoint{10.894621in}{5.558204in}}{\pgfqpoint{10.894621in}{5.547154in}}%
\pgfpathcurveto{\pgfqpoint{10.894621in}{5.536104in}}{\pgfqpoint{10.899011in}{5.525504in}}{\pgfqpoint{10.906825in}{5.517691in}}%
\pgfpathcurveto{\pgfqpoint{10.914639in}{5.509877in}}{\pgfqpoint{10.925238in}{5.505487in}}{\pgfqpoint{10.936288in}{5.505487in}}%
\pgfpathlineto{\pgfqpoint{10.936288in}{5.505487in}}%
\pgfpathclose%
\pgfusepath{stroke}%
\end{pgfscope}%
\begin{pgfscope}%
\pgfpathrectangle{\pgfqpoint{7.394209in}{0.375000in}}{\pgfqpoint{6.356833in}{5.175000in}}%
\pgfusepath{clip}%
\pgfsetbuttcap%
\pgfsetroundjoin%
\pgfsetlinewidth{1.003750pt}%
\definecolor{currentstroke}{rgb}{0.827451,0.827451,0.827451}%
\pgfsetstrokecolor{currentstroke}%
\pgfsetdash{}{0pt}%
\pgfpathmoveto{\pgfqpoint{8.321769in}{2.350835in}}%
\pgfpathcurveto{\pgfqpoint{8.332820in}{2.350835in}}{\pgfqpoint{8.343419in}{2.355226in}}{\pgfqpoint{8.351232in}{2.363039in}}%
\pgfpathcurveto{\pgfqpoint{8.359046in}{2.370853in}}{\pgfqpoint{8.363436in}{2.381452in}}{\pgfqpoint{8.363436in}{2.392502in}}%
\pgfpathcurveto{\pgfqpoint{8.363436in}{2.403552in}}{\pgfqpoint{8.359046in}{2.414151in}}{\pgfqpoint{8.351232in}{2.421965in}}%
\pgfpathcurveto{\pgfqpoint{8.343419in}{2.429778in}}{\pgfqpoint{8.332820in}{2.434169in}}{\pgfqpoint{8.321769in}{2.434169in}}%
\pgfpathcurveto{\pgfqpoint{8.310719in}{2.434169in}}{\pgfqpoint{8.300120in}{2.429778in}}{\pgfqpoint{8.292307in}{2.421965in}}%
\pgfpathcurveto{\pgfqpoint{8.284493in}{2.414151in}}{\pgfqpoint{8.280103in}{2.403552in}}{\pgfqpoint{8.280103in}{2.392502in}}%
\pgfpathcurveto{\pgfqpoint{8.280103in}{2.381452in}}{\pgfqpoint{8.284493in}{2.370853in}}{\pgfqpoint{8.292307in}{2.363039in}}%
\pgfpathcurveto{\pgfqpoint{8.300120in}{2.355226in}}{\pgfqpoint{8.310719in}{2.350835in}}{\pgfqpoint{8.321769in}{2.350835in}}%
\pgfpathlineto{\pgfqpoint{8.321769in}{2.350835in}}%
\pgfpathclose%
\pgfusepath{stroke}%
\end{pgfscope}%
\begin{pgfscope}%
\pgfpathrectangle{\pgfqpoint{7.394209in}{0.375000in}}{\pgfqpoint{6.356833in}{5.175000in}}%
\pgfusepath{clip}%
\pgfsetbuttcap%
\pgfsetroundjoin%
\pgfsetlinewidth{1.003750pt}%
\definecolor{currentstroke}{rgb}{0.827451,0.827451,0.827451}%
\pgfsetstrokecolor{currentstroke}%
\pgfsetdash{}{0pt}%
\pgfpathmoveto{\pgfqpoint{13.452953in}{5.503958in}}%
\pgfpathcurveto{\pgfqpoint{13.464003in}{5.503958in}}{\pgfqpoint{13.474603in}{5.508349in}}{\pgfqpoint{13.482416in}{5.516162in}}%
\pgfpathcurveto{\pgfqpoint{13.490230in}{5.523976in}}{\pgfqpoint{13.494620in}{5.534575in}}{\pgfqpoint{13.494620in}{5.545625in}}%
\pgfpathcurveto{\pgfqpoint{13.494620in}{5.556675in}}{\pgfqpoint{13.490230in}{5.567274in}}{\pgfqpoint{13.482416in}{5.575088in}}%
\pgfpathcurveto{\pgfqpoint{13.474603in}{5.582902in}}{\pgfqpoint{13.464003in}{5.587292in}}{\pgfqpoint{13.452953in}{5.587292in}}%
\pgfpathcurveto{\pgfqpoint{13.441903in}{5.587292in}}{\pgfqpoint{13.431304in}{5.582902in}}{\pgfqpoint{13.423491in}{5.575088in}}%
\pgfpathcurveto{\pgfqpoint{13.415677in}{5.567274in}}{\pgfqpoint{13.411287in}{5.556675in}}{\pgfqpoint{13.411287in}{5.545625in}}%
\pgfpathcurveto{\pgfqpoint{13.411287in}{5.534575in}}{\pgfqpoint{13.415677in}{5.523976in}}{\pgfqpoint{13.423491in}{5.516162in}}%
\pgfpathcurveto{\pgfqpoint{13.431304in}{5.508349in}}{\pgfqpoint{13.441903in}{5.503958in}}{\pgfqpoint{13.452953in}{5.503958in}}%
\pgfpathlineto{\pgfqpoint{13.452953in}{5.503958in}}%
\pgfpathclose%
\pgfusepath{stroke}%
\end{pgfscope}%
\begin{pgfscope}%
\pgfpathrectangle{\pgfqpoint{7.394209in}{0.375000in}}{\pgfqpoint{6.356833in}{5.175000in}}%
\pgfusepath{clip}%
\pgfsetbuttcap%
\pgfsetroundjoin%
\pgfsetlinewidth{1.003750pt}%
\definecolor{currentstroke}{rgb}{0.827451,0.827451,0.827451}%
\pgfsetstrokecolor{currentstroke}%
\pgfsetdash{}{0pt}%
\pgfpathmoveto{\pgfqpoint{7.674631in}{0.333497in}}%
\pgfpathcurveto{\pgfqpoint{7.685681in}{0.333497in}}{\pgfqpoint{7.696280in}{0.337887in}}{\pgfqpoint{7.704094in}{0.345701in}}%
\pgfpathcurveto{\pgfqpoint{7.711907in}{0.353515in}}{\pgfqpoint{7.716297in}{0.364114in}}{\pgfqpoint{7.716297in}{0.375164in}}%
\pgfpathcurveto{\pgfqpoint{7.716297in}{0.386214in}}{\pgfqpoint{7.711907in}{0.396813in}}{\pgfqpoint{7.704094in}{0.404626in}}%
\pgfpathcurveto{\pgfqpoint{7.696280in}{0.412440in}}{\pgfqpoint{7.685681in}{0.416830in}}{\pgfqpoint{7.674631in}{0.416830in}}%
\pgfpathcurveto{\pgfqpoint{7.663581in}{0.416830in}}{\pgfqpoint{7.652982in}{0.412440in}}{\pgfqpoint{7.645168in}{0.404626in}}%
\pgfpathcurveto{\pgfqpoint{7.637354in}{0.396813in}}{\pgfqpoint{7.632964in}{0.386214in}}{\pgfqpoint{7.632964in}{0.375164in}}%
\pgfpathcurveto{\pgfqpoint{7.632964in}{0.364114in}}{\pgfqpoint{7.637354in}{0.353515in}}{\pgfqpoint{7.645168in}{0.345701in}}%
\pgfpathcurveto{\pgfqpoint{7.652982in}{0.337887in}}{\pgfqpoint{7.663581in}{0.333497in}}{\pgfqpoint{7.674631in}{0.333497in}}%
\pgfusepath{stroke}%
\end{pgfscope}%
\begin{pgfscope}%
\pgfpathrectangle{\pgfqpoint{7.394209in}{0.375000in}}{\pgfqpoint{6.356833in}{5.175000in}}%
\pgfusepath{clip}%
\pgfsetbuttcap%
\pgfsetroundjoin%
\pgfsetlinewidth{1.003750pt}%
\definecolor{currentstroke}{rgb}{0.827451,0.827451,0.827451}%
\pgfsetstrokecolor{currentstroke}%
\pgfsetdash{}{0pt}%
\pgfpathmoveto{\pgfqpoint{11.529681in}{5.337320in}}%
\pgfpathcurveto{\pgfqpoint{11.540731in}{5.337320in}}{\pgfqpoint{11.551330in}{5.341710in}}{\pgfqpoint{11.559144in}{5.349523in}}%
\pgfpathcurveto{\pgfqpoint{11.566957in}{5.357337in}}{\pgfqpoint{11.571347in}{5.367936in}}{\pgfqpoint{11.571347in}{5.378986in}}%
\pgfpathcurveto{\pgfqpoint{11.571347in}{5.390036in}}{\pgfqpoint{11.566957in}{5.400635in}}{\pgfqpoint{11.559144in}{5.408449in}}%
\pgfpathcurveto{\pgfqpoint{11.551330in}{5.416263in}}{\pgfqpoint{11.540731in}{5.420653in}}{\pgfqpoint{11.529681in}{5.420653in}}%
\pgfpathcurveto{\pgfqpoint{11.518631in}{5.420653in}}{\pgfqpoint{11.508032in}{5.416263in}}{\pgfqpoint{11.500218in}{5.408449in}}%
\pgfpathcurveto{\pgfqpoint{11.492404in}{5.400635in}}{\pgfqpoint{11.488014in}{5.390036in}}{\pgfqpoint{11.488014in}{5.378986in}}%
\pgfpathcurveto{\pgfqpoint{11.488014in}{5.367936in}}{\pgfqpoint{11.492404in}{5.357337in}}{\pgfqpoint{11.500218in}{5.349523in}}%
\pgfpathcurveto{\pgfqpoint{11.508032in}{5.341710in}}{\pgfqpoint{11.518631in}{5.337320in}}{\pgfqpoint{11.529681in}{5.337320in}}%
\pgfpathlineto{\pgfqpoint{11.529681in}{5.337320in}}%
\pgfpathclose%
\pgfusepath{stroke}%
\end{pgfscope}%
\begin{pgfscope}%
\pgfpathrectangle{\pgfqpoint{7.394209in}{0.375000in}}{\pgfqpoint{6.356833in}{5.175000in}}%
\pgfusepath{clip}%
\pgfsetbuttcap%
\pgfsetroundjoin%
\pgfsetlinewidth{1.003750pt}%
\definecolor{currentstroke}{rgb}{0.827451,0.827451,0.827451}%
\pgfsetstrokecolor{currentstroke}%
\pgfsetdash{}{0pt}%
\pgfpathmoveto{\pgfqpoint{8.302787in}{2.350835in}}%
\pgfpathcurveto{\pgfqpoint{8.313838in}{2.350835in}}{\pgfqpoint{8.324437in}{2.355226in}}{\pgfqpoint{8.332250in}{2.363039in}}%
\pgfpathcurveto{\pgfqpoint{8.340064in}{2.370853in}}{\pgfqpoint{8.344454in}{2.381452in}}{\pgfqpoint{8.344454in}{2.392502in}}%
\pgfpathcurveto{\pgfqpoint{8.344454in}{2.403552in}}{\pgfqpoint{8.340064in}{2.414151in}}{\pgfqpoint{8.332250in}{2.421965in}}%
\pgfpathcurveto{\pgfqpoint{8.324437in}{2.429778in}}{\pgfqpoint{8.313838in}{2.434169in}}{\pgfqpoint{8.302787in}{2.434169in}}%
\pgfpathcurveto{\pgfqpoint{8.291737in}{2.434169in}}{\pgfqpoint{8.281138in}{2.429778in}}{\pgfqpoint{8.273325in}{2.421965in}}%
\pgfpathcurveto{\pgfqpoint{8.265511in}{2.414151in}}{\pgfqpoint{8.261121in}{2.403552in}}{\pgfqpoint{8.261121in}{2.392502in}}%
\pgfpathcurveto{\pgfqpoint{8.261121in}{2.381452in}}{\pgfqpoint{8.265511in}{2.370853in}}{\pgfqpoint{8.273325in}{2.363039in}}%
\pgfpathcurveto{\pgfqpoint{8.281138in}{2.355226in}}{\pgfqpoint{8.291737in}{2.350835in}}{\pgfqpoint{8.302787in}{2.350835in}}%
\pgfpathlineto{\pgfqpoint{8.302787in}{2.350835in}}%
\pgfpathclose%
\pgfusepath{stroke}%
\end{pgfscope}%
\begin{pgfscope}%
\pgfpathrectangle{\pgfqpoint{7.394209in}{0.375000in}}{\pgfqpoint{6.356833in}{5.175000in}}%
\pgfusepath{clip}%
\pgfsetbuttcap%
\pgfsetroundjoin%
\pgfsetlinewidth{1.003750pt}%
\definecolor{currentstroke}{rgb}{0.827451,0.827451,0.827451}%
\pgfsetstrokecolor{currentstroke}%
\pgfsetdash{}{0pt}%
\pgfpathmoveto{\pgfqpoint{13.039649in}{5.489617in}}%
\pgfpathcurveto{\pgfqpoint{13.050699in}{5.489617in}}{\pgfqpoint{13.061298in}{5.494007in}}{\pgfqpoint{13.069112in}{5.501821in}}%
\pgfpathcurveto{\pgfqpoint{13.076926in}{5.509635in}}{\pgfqpoint{13.081316in}{5.520234in}}{\pgfqpoint{13.081316in}{5.531284in}}%
\pgfpathcurveto{\pgfqpoint{13.081316in}{5.542334in}}{\pgfqpoint{13.076926in}{5.552933in}}{\pgfqpoint{13.069112in}{5.560747in}}%
\pgfpathcurveto{\pgfqpoint{13.061298in}{5.568560in}}{\pgfqpoint{13.050699in}{5.572951in}}{\pgfqpoint{13.039649in}{5.572951in}}%
\pgfpathcurveto{\pgfqpoint{13.028599in}{5.572951in}}{\pgfqpoint{13.018000in}{5.568560in}}{\pgfqpoint{13.010186in}{5.560747in}}%
\pgfpathcurveto{\pgfqpoint{13.002373in}{5.552933in}}{\pgfqpoint{12.997983in}{5.542334in}}{\pgfqpoint{12.997983in}{5.531284in}}%
\pgfpathcurveto{\pgfqpoint{12.997983in}{5.520234in}}{\pgfqpoint{13.002373in}{5.509635in}}{\pgfqpoint{13.010186in}{5.501821in}}%
\pgfpathcurveto{\pgfqpoint{13.018000in}{5.494007in}}{\pgfqpoint{13.028599in}{5.489617in}}{\pgfqpoint{13.039649in}{5.489617in}}%
\pgfpathlineto{\pgfqpoint{13.039649in}{5.489617in}}%
\pgfpathclose%
\pgfusepath{stroke}%
\end{pgfscope}%
\begin{pgfscope}%
\pgfpathrectangle{\pgfqpoint{7.394209in}{0.375000in}}{\pgfqpoint{6.356833in}{5.175000in}}%
\pgfusepath{clip}%
\pgfsetbuttcap%
\pgfsetroundjoin%
\pgfsetlinewidth{1.003750pt}%
\definecolor{currentstroke}{rgb}{0.827451,0.827451,0.827451}%
\pgfsetstrokecolor{currentstroke}%
\pgfsetdash{}{0pt}%
\pgfpathmoveto{\pgfqpoint{10.044895in}{4.478092in}}%
\pgfpathcurveto{\pgfqpoint{10.055946in}{4.478092in}}{\pgfqpoint{10.066545in}{4.482482in}}{\pgfqpoint{10.074358in}{4.490296in}}%
\pgfpathcurveto{\pgfqpoint{10.082172in}{4.498110in}}{\pgfqpoint{10.086562in}{4.508709in}}{\pgfqpoint{10.086562in}{4.519759in}}%
\pgfpathcurveto{\pgfqpoint{10.086562in}{4.530809in}}{\pgfqpoint{10.082172in}{4.541408in}}{\pgfqpoint{10.074358in}{4.549222in}}%
\pgfpathcurveto{\pgfqpoint{10.066545in}{4.557035in}}{\pgfqpoint{10.055946in}{4.561425in}}{\pgfqpoint{10.044895in}{4.561425in}}%
\pgfpathcurveto{\pgfqpoint{10.033845in}{4.561425in}}{\pgfqpoint{10.023246in}{4.557035in}}{\pgfqpoint{10.015433in}{4.549222in}}%
\pgfpathcurveto{\pgfqpoint{10.007619in}{4.541408in}}{\pgfqpoint{10.003229in}{4.530809in}}{\pgfqpoint{10.003229in}{4.519759in}}%
\pgfpathcurveto{\pgfqpoint{10.003229in}{4.508709in}}{\pgfqpoint{10.007619in}{4.498110in}}{\pgfqpoint{10.015433in}{4.490296in}}%
\pgfpathcurveto{\pgfqpoint{10.023246in}{4.482482in}}{\pgfqpoint{10.033845in}{4.478092in}}{\pgfqpoint{10.044895in}{4.478092in}}%
\pgfpathlineto{\pgfqpoint{10.044895in}{4.478092in}}%
\pgfpathclose%
\pgfusepath{stroke}%
\end{pgfscope}%
\begin{pgfscope}%
\pgfpathrectangle{\pgfqpoint{7.394209in}{0.375000in}}{\pgfqpoint{6.356833in}{5.175000in}}%
\pgfusepath{clip}%
\pgfsetbuttcap%
\pgfsetroundjoin%
\pgfsetlinewidth{1.003750pt}%
\definecolor{currentstroke}{rgb}{0.827451,0.827451,0.827451}%
\pgfsetstrokecolor{currentstroke}%
\pgfsetdash{}{0pt}%
\pgfpathmoveto{\pgfqpoint{12.549375in}{5.392564in}}%
\pgfpathcurveto{\pgfqpoint{12.560425in}{5.392564in}}{\pgfqpoint{12.571024in}{5.396954in}}{\pgfqpoint{12.578838in}{5.404768in}}%
\pgfpathcurveto{\pgfqpoint{12.586651in}{5.412581in}}{\pgfqpoint{12.591041in}{5.423180in}}{\pgfqpoint{12.591041in}{5.434231in}}%
\pgfpathcurveto{\pgfqpoint{12.591041in}{5.445281in}}{\pgfqpoint{12.586651in}{5.455880in}}{\pgfqpoint{12.578838in}{5.463693in}}%
\pgfpathcurveto{\pgfqpoint{12.571024in}{5.471507in}}{\pgfqpoint{12.560425in}{5.475897in}}{\pgfqpoint{12.549375in}{5.475897in}}%
\pgfpathcurveto{\pgfqpoint{12.538325in}{5.475897in}}{\pgfqpoint{12.527726in}{5.471507in}}{\pgfqpoint{12.519912in}{5.463693in}}%
\pgfpathcurveto{\pgfqpoint{12.512098in}{5.455880in}}{\pgfqpoint{12.507708in}{5.445281in}}{\pgfqpoint{12.507708in}{5.434231in}}%
\pgfpathcurveto{\pgfqpoint{12.507708in}{5.423180in}}{\pgfqpoint{12.512098in}{5.412581in}}{\pgfqpoint{12.519912in}{5.404768in}}%
\pgfpathcurveto{\pgfqpoint{12.527726in}{5.396954in}}{\pgfqpoint{12.538325in}{5.392564in}}{\pgfqpoint{12.549375in}{5.392564in}}%
\pgfpathlineto{\pgfqpoint{12.549375in}{5.392564in}}%
\pgfpathclose%
\pgfusepath{stroke}%
\end{pgfscope}%
\begin{pgfscope}%
\pgfpathrectangle{\pgfqpoint{7.394209in}{0.375000in}}{\pgfqpoint{6.356833in}{5.175000in}}%
\pgfusepath{clip}%
\pgfsetbuttcap%
\pgfsetroundjoin%
\pgfsetlinewidth{1.003750pt}%
\definecolor{currentstroke}{rgb}{0.827451,0.827451,0.827451}%
\pgfsetstrokecolor{currentstroke}%
\pgfsetdash{}{0pt}%
\pgfpathmoveto{\pgfqpoint{11.990499in}{5.443266in}}%
\pgfpathcurveto{\pgfqpoint{12.001549in}{5.443266in}}{\pgfqpoint{12.012148in}{5.447656in}}{\pgfqpoint{12.019962in}{5.455470in}}%
\pgfpathcurveto{\pgfqpoint{12.027775in}{5.463284in}}{\pgfqpoint{12.032166in}{5.473883in}}{\pgfqpoint{12.032166in}{5.484933in}}%
\pgfpathcurveto{\pgfqpoint{12.032166in}{5.495983in}}{\pgfqpoint{12.027775in}{5.506582in}}{\pgfqpoint{12.019962in}{5.514396in}}%
\pgfpathcurveto{\pgfqpoint{12.012148in}{5.522209in}}{\pgfqpoint{12.001549in}{5.526600in}}{\pgfqpoint{11.990499in}{5.526600in}}%
\pgfpathcurveto{\pgfqpoint{11.979449in}{5.526600in}}{\pgfqpoint{11.968850in}{5.522209in}}{\pgfqpoint{11.961036in}{5.514396in}}%
\pgfpathcurveto{\pgfqpoint{11.953223in}{5.506582in}}{\pgfqpoint{11.948832in}{5.495983in}}{\pgfqpoint{11.948832in}{5.484933in}}%
\pgfpathcurveto{\pgfqpoint{11.948832in}{5.473883in}}{\pgfqpoint{11.953223in}{5.463284in}}{\pgfqpoint{11.961036in}{5.455470in}}%
\pgfpathcurveto{\pgfqpoint{11.968850in}{5.447656in}}{\pgfqpoint{11.979449in}{5.443266in}}{\pgfqpoint{11.990499in}{5.443266in}}%
\pgfpathlineto{\pgfqpoint{11.990499in}{5.443266in}}%
\pgfpathclose%
\pgfusepath{stroke}%
\end{pgfscope}%
\begin{pgfscope}%
\pgfpathrectangle{\pgfqpoint{7.394209in}{0.375000in}}{\pgfqpoint{6.356833in}{5.175000in}}%
\pgfusepath{clip}%
\pgfsetbuttcap%
\pgfsetroundjoin%
\pgfsetlinewidth{1.003750pt}%
\definecolor{currentstroke}{rgb}{0.827451,0.827451,0.827451}%
\pgfsetstrokecolor{currentstroke}%
\pgfsetdash{}{0pt}%
\pgfpathmoveto{\pgfqpoint{8.593996in}{2.496484in}}%
\pgfpathcurveto{\pgfqpoint{8.605046in}{2.496484in}}{\pgfqpoint{8.615645in}{2.500874in}}{\pgfqpoint{8.623459in}{2.508688in}}%
\pgfpathcurveto{\pgfqpoint{8.631273in}{2.516501in}}{\pgfqpoint{8.635663in}{2.527100in}}{\pgfqpoint{8.635663in}{2.538151in}}%
\pgfpathcurveto{\pgfqpoint{8.635663in}{2.549201in}}{\pgfqpoint{8.631273in}{2.559800in}}{\pgfqpoint{8.623459in}{2.567613in}}%
\pgfpathcurveto{\pgfqpoint{8.615645in}{2.575427in}}{\pgfqpoint{8.605046in}{2.579817in}}{\pgfqpoint{8.593996in}{2.579817in}}%
\pgfpathcurveto{\pgfqpoint{8.582946in}{2.579817in}}{\pgfqpoint{8.572347in}{2.575427in}}{\pgfqpoint{8.564533in}{2.567613in}}%
\pgfpathcurveto{\pgfqpoint{8.556720in}{2.559800in}}{\pgfqpoint{8.552329in}{2.549201in}}{\pgfqpoint{8.552329in}{2.538151in}}%
\pgfpathcurveto{\pgfqpoint{8.552329in}{2.527100in}}{\pgfqpoint{8.556720in}{2.516501in}}{\pgfqpoint{8.564533in}{2.508688in}}%
\pgfpathcurveto{\pgfqpoint{8.572347in}{2.500874in}}{\pgfqpoint{8.582946in}{2.496484in}}{\pgfqpoint{8.593996in}{2.496484in}}%
\pgfpathlineto{\pgfqpoint{8.593996in}{2.496484in}}%
\pgfpathclose%
\pgfusepath{stroke}%
\end{pgfscope}%
\begin{pgfscope}%
\pgfpathrectangle{\pgfqpoint{7.394209in}{0.375000in}}{\pgfqpoint{6.356833in}{5.175000in}}%
\pgfusepath{clip}%
\pgfsetbuttcap%
\pgfsetroundjoin%
\pgfsetlinewidth{1.003750pt}%
\definecolor{currentstroke}{rgb}{0.827451,0.827451,0.827451}%
\pgfsetstrokecolor{currentstroke}%
\pgfsetdash{}{0pt}%
\pgfpathmoveto{\pgfqpoint{8.470201in}{2.181484in}}%
\pgfpathcurveto{\pgfqpoint{8.481251in}{2.181484in}}{\pgfqpoint{8.491850in}{2.185875in}}{\pgfqpoint{8.499663in}{2.193688in}}%
\pgfpathcurveto{\pgfqpoint{8.507477in}{2.201502in}}{\pgfqpoint{8.511867in}{2.212101in}}{\pgfqpoint{8.511867in}{2.223151in}}%
\pgfpathcurveto{\pgfqpoint{8.511867in}{2.234201in}}{\pgfqpoint{8.507477in}{2.244800in}}{\pgfqpoint{8.499663in}{2.252614in}}%
\pgfpathcurveto{\pgfqpoint{8.491850in}{2.260427in}}{\pgfqpoint{8.481251in}{2.264818in}}{\pgfqpoint{8.470201in}{2.264818in}}%
\pgfpathcurveto{\pgfqpoint{8.459150in}{2.264818in}}{\pgfqpoint{8.448551in}{2.260427in}}{\pgfqpoint{8.440738in}{2.252614in}}%
\pgfpathcurveto{\pgfqpoint{8.432924in}{2.244800in}}{\pgfqpoint{8.428534in}{2.234201in}}{\pgfqpoint{8.428534in}{2.223151in}}%
\pgfpathcurveto{\pgfqpoint{8.428534in}{2.212101in}}{\pgfqpoint{8.432924in}{2.201502in}}{\pgfqpoint{8.440738in}{2.193688in}}%
\pgfpathcurveto{\pgfqpoint{8.448551in}{2.185875in}}{\pgfqpoint{8.459150in}{2.181484in}}{\pgfqpoint{8.470201in}{2.181484in}}%
\pgfpathlineto{\pgfqpoint{8.470201in}{2.181484in}}%
\pgfpathclose%
\pgfusepath{stroke}%
\end{pgfscope}%
\begin{pgfscope}%
\pgfpathrectangle{\pgfqpoint{7.394209in}{0.375000in}}{\pgfqpoint{6.356833in}{5.175000in}}%
\pgfusepath{clip}%
\pgfsetbuttcap%
\pgfsetroundjoin%
\pgfsetlinewidth{1.003750pt}%
\definecolor{currentstroke}{rgb}{0.827451,0.827451,0.827451}%
\pgfsetstrokecolor{currentstroke}%
\pgfsetdash{}{0pt}%
\pgfpathmoveto{\pgfqpoint{12.297587in}{5.434154in}}%
\pgfpathcurveto{\pgfqpoint{12.308637in}{5.434154in}}{\pgfqpoint{12.319236in}{5.438544in}}{\pgfqpoint{12.327050in}{5.446358in}}%
\pgfpathcurveto{\pgfqpoint{12.334863in}{5.454172in}}{\pgfqpoint{12.339253in}{5.464771in}}{\pgfqpoint{12.339253in}{5.475821in}}%
\pgfpathcurveto{\pgfqpoint{12.339253in}{5.486871in}}{\pgfqpoint{12.334863in}{5.497470in}}{\pgfqpoint{12.327050in}{5.505284in}}%
\pgfpathcurveto{\pgfqpoint{12.319236in}{5.513097in}}{\pgfqpoint{12.308637in}{5.517487in}}{\pgfqpoint{12.297587in}{5.517487in}}%
\pgfpathcurveto{\pgfqpoint{12.286537in}{5.517487in}}{\pgfqpoint{12.275938in}{5.513097in}}{\pgfqpoint{12.268124in}{5.505284in}}%
\pgfpathcurveto{\pgfqpoint{12.260310in}{5.497470in}}{\pgfqpoint{12.255920in}{5.486871in}}{\pgfqpoint{12.255920in}{5.475821in}}%
\pgfpathcurveto{\pgfqpoint{12.255920in}{5.464771in}}{\pgfqpoint{12.260310in}{5.454172in}}{\pgfqpoint{12.268124in}{5.446358in}}%
\pgfpathcurveto{\pgfqpoint{12.275938in}{5.438544in}}{\pgfqpoint{12.286537in}{5.434154in}}{\pgfqpoint{12.297587in}{5.434154in}}%
\pgfpathlineto{\pgfqpoint{12.297587in}{5.434154in}}%
\pgfpathclose%
\pgfusepath{stroke}%
\end{pgfscope}%
\begin{pgfscope}%
\pgfpathrectangle{\pgfqpoint{7.394209in}{0.375000in}}{\pgfqpoint{6.356833in}{5.175000in}}%
\pgfusepath{clip}%
\pgfsetbuttcap%
\pgfsetroundjoin%
\pgfsetlinewidth{1.003750pt}%
\definecolor{currentstroke}{rgb}{0.827451,0.827451,0.827451}%
\pgfsetstrokecolor{currentstroke}%
\pgfsetdash{}{0pt}%
\pgfpathmoveto{\pgfqpoint{9.834929in}{2.809812in}}%
\pgfpathcurveto{\pgfqpoint{9.845979in}{2.809812in}}{\pgfqpoint{9.856578in}{2.814202in}}{\pgfqpoint{9.864392in}{2.822016in}}%
\pgfpathcurveto{\pgfqpoint{9.872205in}{2.829830in}}{\pgfqpoint{9.876596in}{2.840429in}}{\pgfqpoint{9.876596in}{2.851479in}}%
\pgfpathcurveto{\pgfqpoint{9.876596in}{2.862529in}}{\pgfqpoint{9.872205in}{2.873128in}}{\pgfqpoint{9.864392in}{2.880942in}}%
\pgfpathcurveto{\pgfqpoint{9.856578in}{2.888755in}}{\pgfqpoint{9.845979in}{2.893145in}}{\pgfqpoint{9.834929in}{2.893145in}}%
\pgfpathcurveto{\pgfqpoint{9.823879in}{2.893145in}}{\pgfqpoint{9.813280in}{2.888755in}}{\pgfqpoint{9.805466in}{2.880942in}}%
\pgfpathcurveto{\pgfqpoint{9.797653in}{2.873128in}}{\pgfqpoint{9.793262in}{2.862529in}}{\pgfqpoint{9.793262in}{2.851479in}}%
\pgfpathcurveto{\pgfqpoint{9.793262in}{2.840429in}}{\pgfqpoint{9.797653in}{2.829830in}}{\pgfqpoint{9.805466in}{2.822016in}}%
\pgfpathcurveto{\pgfqpoint{9.813280in}{2.814202in}}{\pgfqpoint{9.823879in}{2.809812in}}{\pgfqpoint{9.834929in}{2.809812in}}%
\pgfpathlineto{\pgfqpoint{9.834929in}{2.809812in}}%
\pgfpathclose%
\pgfusepath{stroke}%
\end{pgfscope}%
\begin{pgfscope}%
\pgfpathrectangle{\pgfqpoint{7.394209in}{0.375000in}}{\pgfqpoint{6.356833in}{5.175000in}}%
\pgfusepath{clip}%
\pgfsetbuttcap%
\pgfsetroundjoin%
\pgfsetlinewidth{1.003750pt}%
\definecolor{currentstroke}{rgb}{0.827451,0.827451,0.827451}%
\pgfsetstrokecolor{currentstroke}%
\pgfsetdash{}{0pt}%
\pgfpathmoveto{\pgfqpoint{10.064246in}{3.028900in}}%
\pgfpathcurveto{\pgfqpoint{10.075296in}{3.028900in}}{\pgfqpoint{10.085895in}{3.033291in}}{\pgfqpoint{10.093709in}{3.041104in}}%
\pgfpathcurveto{\pgfqpoint{10.101522in}{3.048918in}}{\pgfqpoint{10.105912in}{3.059517in}}{\pgfqpoint{10.105912in}{3.070567in}}%
\pgfpathcurveto{\pgfqpoint{10.105912in}{3.081617in}}{\pgfqpoint{10.101522in}{3.092216in}}{\pgfqpoint{10.093709in}{3.100030in}}%
\pgfpathcurveto{\pgfqpoint{10.085895in}{3.107843in}}{\pgfqpoint{10.075296in}{3.112234in}}{\pgfqpoint{10.064246in}{3.112234in}}%
\pgfpathcurveto{\pgfqpoint{10.053196in}{3.112234in}}{\pgfqpoint{10.042597in}{3.107843in}}{\pgfqpoint{10.034783in}{3.100030in}}%
\pgfpathcurveto{\pgfqpoint{10.026969in}{3.092216in}}{\pgfqpoint{10.022579in}{3.081617in}}{\pgfqpoint{10.022579in}{3.070567in}}%
\pgfpathcurveto{\pgfqpoint{10.022579in}{3.059517in}}{\pgfqpoint{10.026969in}{3.048918in}}{\pgfqpoint{10.034783in}{3.041104in}}%
\pgfpathcurveto{\pgfqpoint{10.042597in}{3.033291in}}{\pgfqpoint{10.053196in}{3.028900in}}{\pgfqpoint{10.064246in}{3.028900in}}%
\pgfpathlineto{\pgfqpoint{10.064246in}{3.028900in}}%
\pgfpathclose%
\pgfusepath{stroke}%
\end{pgfscope}%
\begin{pgfscope}%
\pgfpathrectangle{\pgfqpoint{7.394209in}{0.375000in}}{\pgfqpoint{6.356833in}{5.175000in}}%
\pgfusepath{clip}%
\pgfsetbuttcap%
\pgfsetroundjoin%
\pgfsetlinewidth{1.003750pt}%
\definecolor{currentstroke}{rgb}{0.827451,0.827451,0.827451}%
\pgfsetstrokecolor{currentstroke}%
\pgfsetdash{}{0pt}%
\pgfpathmoveto{\pgfqpoint{9.868553in}{4.331872in}}%
\pgfpathcurveto{\pgfqpoint{9.879603in}{4.331872in}}{\pgfqpoint{9.890202in}{4.336262in}}{\pgfqpoint{9.898016in}{4.344075in}}%
\pgfpathcurveto{\pgfqpoint{9.905830in}{4.351889in}}{\pgfqpoint{9.910220in}{4.362488in}}{\pgfqpoint{9.910220in}{4.373538in}}%
\pgfpathcurveto{\pgfqpoint{9.910220in}{4.384588in}}{\pgfqpoint{9.905830in}{4.395187in}}{\pgfqpoint{9.898016in}{4.403001in}}%
\pgfpathcurveto{\pgfqpoint{9.890202in}{4.410815in}}{\pgfqpoint{9.879603in}{4.415205in}}{\pgfqpoint{9.868553in}{4.415205in}}%
\pgfpathcurveto{\pgfqpoint{9.857503in}{4.415205in}}{\pgfqpoint{9.846904in}{4.410815in}}{\pgfqpoint{9.839090in}{4.403001in}}%
\pgfpathcurveto{\pgfqpoint{9.831277in}{4.395187in}}{\pgfqpoint{9.826887in}{4.384588in}}{\pgfqpoint{9.826887in}{4.373538in}}%
\pgfpathcurveto{\pgfqpoint{9.826887in}{4.362488in}}{\pgfqpoint{9.831277in}{4.351889in}}{\pgfqpoint{9.839090in}{4.344075in}}%
\pgfpathcurveto{\pgfqpoint{9.846904in}{4.336262in}}{\pgfqpoint{9.857503in}{4.331872in}}{\pgfqpoint{9.868553in}{4.331872in}}%
\pgfpathlineto{\pgfqpoint{9.868553in}{4.331872in}}%
\pgfpathclose%
\pgfusepath{stroke}%
\end{pgfscope}%
\begin{pgfscope}%
\pgfpathrectangle{\pgfqpoint{7.394209in}{0.375000in}}{\pgfqpoint{6.356833in}{5.175000in}}%
\pgfusepath{clip}%
\pgfsetbuttcap%
\pgfsetroundjoin%
\pgfsetlinewidth{1.003750pt}%
\definecolor{currentstroke}{rgb}{0.827451,0.827451,0.827451}%
\pgfsetstrokecolor{currentstroke}%
\pgfsetdash{}{0pt}%
\pgfpathmoveto{\pgfqpoint{10.505392in}{2.869210in}}%
\pgfpathcurveto{\pgfqpoint{10.516442in}{2.869210in}}{\pgfqpoint{10.527041in}{2.873600in}}{\pgfqpoint{10.534854in}{2.881414in}}%
\pgfpathcurveto{\pgfqpoint{10.542668in}{2.889228in}}{\pgfqpoint{10.547058in}{2.899827in}}{\pgfqpoint{10.547058in}{2.910877in}}%
\pgfpathcurveto{\pgfqpoint{10.547058in}{2.921927in}}{\pgfqpoint{10.542668in}{2.932526in}}{\pgfqpoint{10.534854in}{2.940340in}}%
\pgfpathcurveto{\pgfqpoint{10.527041in}{2.948153in}}{\pgfqpoint{10.516442in}{2.952543in}}{\pgfqpoint{10.505392in}{2.952543in}}%
\pgfpathcurveto{\pgfqpoint{10.494341in}{2.952543in}}{\pgfqpoint{10.483742in}{2.948153in}}{\pgfqpoint{10.475929in}{2.940340in}}%
\pgfpathcurveto{\pgfqpoint{10.468115in}{2.932526in}}{\pgfqpoint{10.463725in}{2.921927in}}{\pgfqpoint{10.463725in}{2.910877in}}%
\pgfpathcurveto{\pgfqpoint{10.463725in}{2.899827in}}{\pgfqpoint{10.468115in}{2.889228in}}{\pgfqpoint{10.475929in}{2.881414in}}%
\pgfpathcurveto{\pgfqpoint{10.483742in}{2.873600in}}{\pgfqpoint{10.494341in}{2.869210in}}{\pgfqpoint{10.505392in}{2.869210in}}%
\pgfpathlineto{\pgfqpoint{10.505392in}{2.869210in}}%
\pgfpathclose%
\pgfusepath{stroke}%
\end{pgfscope}%
\begin{pgfscope}%
\pgfpathrectangle{\pgfqpoint{7.394209in}{0.375000in}}{\pgfqpoint{6.356833in}{5.175000in}}%
\pgfusepath{clip}%
\pgfsetbuttcap%
\pgfsetroundjoin%
\pgfsetlinewidth{1.003750pt}%
\definecolor{currentstroke}{rgb}{0.827451,0.827451,0.827451}%
\pgfsetstrokecolor{currentstroke}%
\pgfsetdash{}{0pt}%
\pgfpathmoveto{\pgfqpoint{8.558914in}{2.882781in}}%
\pgfpathcurveto{\pgfqpoint{8.569964in}{2.882781in}}{\pgfqpoint{8.580563in}{2.887172in}}{\pgfqpoint{8.588377in}{2.894985in}}%
\pgfpathcurveto{\pgfqpoint{8.596191in}{2.902799in}}{\pgfqpoint{8.600581in}{2.913398in}}{\pgfqpoint{8.600581in}{2.924448in}}%
\pgfpathcurveto{\pgfqpoint{8.600581in}{2.935498in}}{\pgfqpoint{8.596191in}{2.946097in}}{\pgfqpoint{8.588377in}{2.953911in}}%
\pgfpathcurveto{\pgfqpoint{8.580563in}{2.961724in}}{\pgfqpoint{8.569964in}{2.966115in}}{\pgfqpoint{8.558914in}{2.966115in}}%
\pgfpathcurveto{\pgfqpoint{8.547864in}{2.966115in}}{\pgfqpoint{8.537265in}{2.961724in}}{\pgfqpoint{8.529451in}{2.953911in}}%
\pgfpathcurveto{\pgfqpoint{8.521638in}{2.946097in}}{\pgfqpoint{8.517248in}{2.935498in}}{\pgfqpoint{8.517248in}{2.924448in}}%
\pgfpathcurveto{\pgfqpoint{8.517248in}{2.913398in}}{\pgfqpoint{8.521638in}{2.902799in}}{\pgfqpoint{8.529451in}{2.894985in}}%
\pgfpathcurveto{\pgfqpoint{8.537265in}{2.887172in}}{\pgfqpoint{8.547864in}{2.882781in}}{\pgfqpoint{8.558914in}{2.882781in}}%
\pgfpathlineto{\pgfqpoint{8.558914in}{2.882781in}}%
\pgfpathclose%
\pgfusepath{stroke}%
\end{pgfscope}%
\begin{pgfscope}%
\pgfpathrectangle{\pgfqpoint{7.394209in}{0.375000in}}{\pgfqpoint{6.356833in}{5.175000in}}%
\pgfusepath{clip}%
\pgfsetbuttcap%
\pgfsetroundjoin%
\pgfsetlinewidth{1.003750pt}%
\definecolor{currentstroke}{rgb}{0.827451,0.827451,0.827451}%
\pgfsetstrokecolor{currentstroke}%
\pgfsetdash{}{0pt}%
\pgfpathmoveto{\pgfqpoint{11.806727in}{5.223153in}}%
\pgfpathcurveto{\pgfqpoint{11.817777in}{5.223153in}}{\pgfqpoint{11.828376in}{5.227543in}}{\pgfqpoint{11.836189in}{5.235357in}}%
\pgfpathcurveto{\pgfqpoint{11.844003in}{5.243171in}}{\pgfqpoint{11.848393in}{5.253770in}}{\pgfqpoint{11.848393in}{5.264820in}}%
\pgfpathcurveto{\pgfqpoint{11.848393in}{5.275870in}}{\pgfqpoint{11.844003in}{5.286469in}}{\pgfqpoint{11.836189in}{5.294283in}}%
\pgfpathcurveto{\pgfqpoint{11.828376in}{5.302096in}}{\pgfqpoint{11.817777in}{5.306487in}}{\pgfqpoint{11.806727in}{5.306487in}}%
\pgfpathcurveto{\pgfqpoint{11.795676in}{5.306487in}}{\pgfqpoint{11.785077in}{5.302096in}}{\pgfqpoint{11.777264in}{5.294283in}}%
\pgfpathcurveto{\pgfqpoint{11.769450in}{5.286469in}}{\pgfqpoint{11.765060in}{5.275870in}}{\pgfqpoint{11.765060in}{5.264820in}}%
\pgfpathcurveto{\pgfqpoint{11.765060in}{5.253770in}}{\pgfqpoint{11.769450in}{5.243171in}}{\pgfqpoint{11.777264in}{5.235357in}}%
\pgfpathcurveto{\pgfqpoint{11.785077in}{5.227543in}}{\pgfqpoint{11.795676in}{5.223153in}}{\pgfqpoint{11.806727in}{5.223153in}}%
\pgfpathlineto{\pgfqpoint{11.806727in}{5.223153in}}%
\pgfpathclose%
\pgfusepath{stroke}%
\end{pgfscope}%
\begin{pgfscope}%
\pgfpathrectangle{\pgfqpoint{7.394209in}{0.375000in}}{\pgfqpoint{6.356833in}{5.175000in}}%
\pgfusepath{clip}%
\pgfsetbuttcap%
\pgfsetroundjoin%
\pgfsetlinewidth{1.003750pt}%
\definecolor{currentstroke}{rgb}{0.827451,0.827451,0.827451}%
\pgfsetstrokecolor{currentstroke}%
\pgfsetdash{}{0pt}%
\pgfpathmoveto{\pgfqpoint{11.850296in}{5.443266in}}%
\pgfpathcurveto{\pgfqpoint{11.861346in}{5.443266in}}{\pgfqpoint{11.871945in}{5.447656in}}{\pgfqpoint{11.879759in}{5.455470in}}%
\pgfpathcurveto{\pgfqpoint{11.887572in}{5.463284in}}{\pgfqpoint{11.891963in}{5.473883in}}{\pgfqpoint{11.891963in}{5.484933in}}%
\pgfpathcurveto{\pgfqpoint{11.891963in}{5.495983in}}{\pgfqpoint{11.887572in}{5.506582in}}{\pgfqpoint{11.879759in}{5.514396in}}%
\pgfpathcurveto{\pgfqpoint{11.871945in}{5.522209in}}{\pgfqpoint{11.861346in}{5.526600in}}{\pgfqpoint{11.850296in}{5.526600in}}%
\pgfpathcurveto{\pgfqpoint{11.839246in}{5.526600in}}{\pgfqpoint{11.828647in}{5.522209in}}{\pgfqpoint{11.820833in}{5.514396in}}%
\pgfpathcurveto{\pgfqpoint{11.813020in}{5.506582in}}{\pgfqpoint{11.808629in}{5.495983in}}{\pgfqpoint{11.808629in}{5.484933in}}%
\pgfpathcurveto{\pgfqpoint{11.808629in}{5.473883in}}{\pgfqpoint{11.813020in}{5.463284in}}{\pgfqpoint{11.820833in}{5.455470in}}%
\pgfpathcurveto{\pgfqpoint{11.828647in}{5.447656in}}{\pgfqpoint{11.839246in}{5.443266in}}{\pgfqpoint{11.850296in}{5.443266in}}%
\pgfpathlineto{\pgfqpoint{11.850296in}{5.443266in}}%
\pgfpathclose%
\pgfusepath{stroke}%
\end{pgfscope}%
\begin{pgfscope}%
\pgfpathrectangle{\pgfqpoint{7.394209in}{0.375000in}}{\pgfqpoint{6.356833in}{5.175000in}}%
\pgfusepath{clip}%
\pgfsetbuttcap%
\pgfsetroundjoin%
\pgfsetlinewidth{1.003750pt}%
\definecolor{currentstroke}{rgb}{0.827451,0.827451,0.827451}%
\pgfsetstrokecolor{currentstroke}%
\pgfsetdash{}{0pt}%
\pgfpathmoveto{\pgfqpoint{10.044118in}{4.636133in}}%
\pgfpathcurveto{\pgfqpoint{10.055169in}{4.636133in}}{\pgfqpoint{10.065768in}{4.640524in}}{\pgfqpoint{10.073581in}{4.648337in}}%
\pgfpathcurveto{\pgfqpoint{10.081395in}{4.656151in}}{\pgfqpoint{10.085785in}{4.666750in}}{\pgfqpoint{10.085785in}{4.677800in}}%
\pgfpathcurveto{\pgfqpoint{10.085785in}{4.688850in}}{\pgfqpoint{10.081395in}{4.699449in}}{\pgfqpoint{10.073581in}{4.707263in}}%
\pgfpathcurveto{\pgfqpoint{10.065768in}{4.715076in}}{\pgfqpoint{10.055169in}{4.719467in}}{\pgfqpoint{10.044118in}{4.719467in}}%
\pgfpathcurveto{\pgfqpoint{10.033068in}{4.719467in}}{\pgfqpoint{10.022469in}{4.715076in}}{\pgfqpoint{10.014656in}{4.707263in}}%
\pgfpathcurveto{\pgfqpoint{10.006842in}{4.699449in}}{\pgfqpoint{10.002452in}{4.688850in}}{\pgfqpoint{10.002452in}{4.677800in}}%
\pgfpathcurveto{\pgfqpoint{10.002452in}{4.666750in}}{\pgfqpoint{10.006842in}{4.656151in}}{\pgfqpoint{10.014656in}{4.648337in}}%
\pgfpathcurveto{\pgfqpoint{10.022469in}{4.640524in}}{\pgfqpoint{10.033068in}{4.636133in}}{\pgfqpoint{10.044118in}{4.636133in}}%
\pgfpathlineto{\pgfqpoint{10.044118in}{4.636133in}}%
\pgfpathclose%
\pgfusepath{stroke}%
\end{pgfscope}%
\begin{pgfscope}%
\pgfpathrectangle{\pgfqpoint{7.394209in}{0.375000in}}{\pgfqpoint{6.356833in}{5.175000in}}%
\pgfusepath{clip}%
\pgfsetbuttcap%
\pgfsetroundjoin%
\pgfsetlinewidth{1.003750pt}%
\definecolor{currentstroke}{rgb}{0.827451,0.827451,0.827451}%
\pgfsetstrokecolor{currentstroke}%
\pgfsetdash{}{0pt}%
\pgfpathmoveto{\pgfqpoint{8.666196in}{2.407173in}}%
\pgfpathcurveto{\pgfqpoint{8.677246in}{2.407173in}}{\pgfqpoint{8.687845in}{2.411563in}}{\pgfqpoint{8.695659in}{2.419376in}}%
\pgfpathcurveto{\pgfqpoint{8.703472in}{2.427190in}}{\pgfqpoint{8.707863in}{2.437789in}}{\pgfqpoint{8.707863in}{2.448839in}}%
\pgfpathcurveto{\pgfqpoint{8.707863in}{2.459889in}}{\pgfqpoint{8.703472in}{2.470488in}}{\pgfqpoint{8.695659in}{2.478302in}}%
\pgfpathcurveto{\pgfqpoint{8.687845in}{2.486116in}}{\pgfqpoint{8.677246in}{2.490506in}}{\pgfqpoint{8.666196in}{2.490506in}}%
\pgfpathcurveto{\pgfqpoint{8.655146in}{2.490506in}}{\pgfqpoint{8.644547in}{2.486116in}}{\pgfqpoint{8.636733in}{2.478302in}}%
\pgfpathcurveto{\pgfqpoint{8.628919in}{2.470488in}}{\pgfqpoint{8.624529in}{2.459889in}}{\pgfqpoint{8.624529in}{2.448839in}}%
\pgfpathcurveto{\pgfqpoint{8.624529in}{2.437789in}}{\pgfqpoint{8.628919in}{2.427190in}}{\pgfqpoint{8.636733in}{2.419376in}}%
\pgfpathcurveto{\pgfqpoint{8.644547in}{2.411563in}}{\pgfqpoint{8.655146in}{2.407173in}}{\pgfqpoint{8.666196in}{2.407173in}}%
\pgfpathlineto{\pgfqpoint{8.666196in}{2.407173in}}%
\pgfpathclose%
\pgfusepath{stroke}%
\end{pgfscope}%
\begin{pgfscope}%
\pgfpathrectangle{\pgfqpoint{7.394209in}{0.375000in}}{\pgfqpoint{6.356833in}{5.175000in}}%
\pgfusepath{clip}%
\pgfsetbuttcap%
\pgfsetroundjoin%
\pgfsetlinewidth{1.003750pt}%
\definecolor{currentstroke}{rgb}{0.827451,0.827451,0.827451}%
\pgfsetstrokecolor{currentstroke}%
\pgfsetdash{}{0pt}%
\pgfpathmoveto{\pgfqpoint{9.398487in}{4.417732in}}%
\pgfpathcurveto{\pgfqpoint{9.409537in}{4.417732in}}{\pgfqpoint{9.420136in}{4.422122in}}{\pgfqpoint{9.427950in}{4.429936in}}%
\pgfpathcurveto{\pgfqpoint{9.435764in}{4.437749in}}{\pgfqpoint{9.440154in}{4.448348in}}{\pgfqpoint{9.440154in}{4.459398in}}%
\pgfpathcurveto{\pgfqpoint{9.440154in}{4.470449in}}{\pgfqpoint{9.435764in}{4.481048in}}{\pgfqpoint{9.427950in}{4.488861in}}%
\pgfpathcurveto{\pgfqpoint{9.420136in}{4.496675in}}{\pgfqpoint{9.409537in}{4.501065in}}{\pgfqpoint{9.398487in}{4.501065in}}%
\pgfpathcurveto{\pgfqpoint{9.387437in}{4.501065in}}{\pgfqpoint{9.376838in}{4.496675in}}{\pgfqpoint{9.369024in}{4.488861in}}%
\pgfpathcurveto{\pgfqpoint{9.361211in}{4.481048in}}{\pgfqpoint{9.356820in}{4.470449in}}{\pgfqpoint{9.356820in}{4.459398in}}%
\pgfpathcurveto{\pgfqpoint{9.356820in}{4.448348in}}{\pgfqpoint{9.361211in}{4.437749in}}{\pgfqpoint{9.369024in}{4.429936in}}%
\pgfpathcurveto{\pgfqpoint{9.376838in}{4.422122in}}{\pgfqpoint{9.387437in}{4.417732in}}{\pgfqpoint{9.398487in}{4.417732in}}%
\pgfpathlineto{\pgfqpoint{9.398487in}{4.417732in}}%
\pgfpathclose%
\pgfusepath{stroke}%
\end{pgfscope}%
\begin{pgfscope}%
\pgfpathrectangle{\pgfqpoint{7.394209in}{0.375000in}}{\pgfqpoint{6.356833in}{5.175000in}}%
\pgfusepath{clip}%
\pgfsetbuttcap%
\pgfsetroundjoin%
\pgfsetlinewidth{1.003750pt}%
\definecolor{currentstroke}{rgb}{0.827451,0.827451,0.827451}%
\pgfsetstrokecolor{currentstroke}%
\pgfsetdash{}{0pt}%
\pgfpathmoveto{\pgfqpoint{10.353839in}{3.159540in}}%
\pgfpathcurveto{\pgfqpoint{10.364889in}{3.159540in}}{\pgfqpoint{10.375488in}{3.163930in}}{\pgfqpoint{10.383302in}{3.171744in}}%
\pgfpathcurveto{\pgfqpoint{10.391115in}{3.179557in}}{\pgfqpoint{10.395505in}{3.190156in}}{\pgfqpoint{10.395505in}{3.201207in}}%
\pgfpathcurveto{\pgfqpoint{10.395505in}{3.212257in}}{\pgfqpoint{10.391115in}{3.222856in}}{\pgfqpoint{10.383302in}{3.230669in}}%
\pgfpathcurveto{\pgfqpoint{10.375488in}{3.238483in}}{\pgfqpoint{10.364889in}{3.242873in}}{\pgfqpoint{10.353839in}{3.242873in}}%
\pgfpathcurveto{\pgfqpoint{10.342789in}{3.242873in}}{\pgfqpoint{10.332190in}{3.238483in}}{\pgfqpoint{10.324376in}{3.230669in}}%
\pgfpathcurveto{\pgfqpoint{10.316562in}{3.222856in}}{\pgfqpoint{10.312172in}{3.212257in}}{\pgfqpoint{10.312172in}{3.201207in}}%
\pgfpathcurveto{\pgfqpoint{10.312172in}{3.190156in}}{\pgfqpoint{10.316562in}{3.179557in}}{\pgfqpoint{10.324376in}{3.171744in}}%
\pgfpathcurveto{\pgfqpoint{10.332190in}{3.163930in}}{\pgfqpoint{10.342789in}{3.159540in}}{\pgfqpoint{10.353839in}{3.159540in}}%
\pgfpathlineto{\pgfqpoint{10.353839in}{3.159540in}}%
\pgfpathclose%
\pgfusepath{stroke}%
\end{pgfscope}%
\begin{pgfscope}%
\pgfpathrectangle{\pgfqpoint{7.394209in}{0.375000in}}{\pgfqpoint{6.356833in}{5.175000in}}%
\pgfusepath{clip}%
\pgfsetbuttcap%
\pgfsetroundjoin%
\pgfsetlinewidth{1.003750pt}%
\definecolor{currentstroke}{rgb}{0.827451,0.827451,0.827451}%
\pgfsetstrokecolor{currentstroke}%
\pgfsetdash{}{0pt}%
\pgfpathmoveto{\pgfqpoint{9.808763in}{3.240034in}}%
\pgfpathcurveto{\pgfqpoint{9.819813in}{3.240034in}}{\pgfqpoint{9.830413in}{3.244424in}}{\pgfqpoint{9.838226in}{3.252238in}}%
\pgfpathcurveto{\pgfqpoint{9.846040in}{3.260052in}}{\pgfqpoint{9.850430in}{3.270651in}}{\pgfqpoint{9.850430in}{3.281701in}}%
\pgfpathcurveto{\pgfqpoint{9.850430in}{3.292751in}}{\pgfqpoint{9.846040in}{3.303350in}}{\pgfqpoint{9.838226in}{3.311163in}}%
\pgfpathcurveto{\pgfqpoint{9.830413in}{3.318977in}}{\pgfqpoint{9.819813in}{3.323367in}}{\pgfqpoint{9.808763in}{3.323367in}}%
\pgfpathcurveto{\pgfqpoint{9.797713in}{3.323367in}}{\pgfqpoint{9.787114in}{3.318977in}}{\pgfqpoint{9.779301in}{3.311163in}}%
\pgfpathcurveto{\pgfqpoint{9.771487in}{3.303350in}}{\pgfqpoint{9.767097in}{3.292751in}}{\pgfqpoint{9.767097in}{3.281701in}}%
\pgfpathcurveto{\pgfqpoint{9.767097in}{3.270651in}}{\pgfqpoint{9.771487in}{3.260052in}}{\pgfqpoint{9.779301in}{3.252238in}}%
\pgfpathcurveto{\pgfqpoint{9.787114in}{3.244424in}}{\pgfqpoint{9.797713in}{3.240034in}}{\pgfqpoint{9.808763in}{3.240034in}}%
\pgfpathlineto{\pgfqpoint{9.808763in}{3.240034in}}%
\pgfpathclose%
\pgfusepath{stroke}%
\end{pgfscope}%
\begin{pgfscope}%
\pgfpathrectangle{\pgfqpoint{7.394209in}{0.375000in}}{\pgfqpoint{6.356833in}{5.175000in}}%
\pgfusepath{clip}%
\pgfsetbuttcap%
\pgfsetroundjoin%
\pgfsetlinewidth{1.003750pt}%
\definecolor{currentstroke}{rgb}{0.827451,0.827451,0.827451}%
\pgfsetstrokecolor{currentstroke}%
\pgfsetdash{}{0pt}%
\pgfpathmoveto{\pgfqpoint{10.305287in}{3.159540in}}%
\pgfpathcurveto{\pgfqpoint{10.316338in}{3.159540in}}{\pgfqpoint{10.326937in}{3.163930in}}{\pgfqpoint{10.334750in}{3.171744in}}%
\pgfpathcurveto{\pgfqpoint{10.342564in}{3.179557in}}{\pgfqpoint{10.346954in}{3.190156in}}{\pgfqpoint{10.346954in}{3.201207in}}%
\pgfpathcurveto{\pgfqpoint{10.346954in}{3.212257in}}{\pgfqpoint{10.342564in}{3.222856in}}{\pgfqpoint{10.334750in}{3.230669in}}%
\pgfpathcurveto{\pgfqpoint{10.326937in}{3.238483in}}{\pgfqpoint{10.316338in}{3.242873in}}{\pgfqpoint{10.305287in}{3.242873in}}%
\pgfpathcurveto{\pgfqpoint{10.294237in}{3.242873in}}{\pgfqpoint{10.283638in}{3.238483in}}{\pgfqpoint{10.275825in}{3.230669in}}%
\pgfpathcurveto{\pgfqpoint{10.268011in}{3.222856in}}{\pgfqpoint{10.263621in}{3.212257in}}{\pgfqpoint{10.263621in}{3.201207in}}%
\pgfpathcurveto{\pgfqpoint{10.263621in}{3.190156in}}{\pgfqpoint{10.268011in}{3.179557in}}{\pgfqpoint{10.275825in}{3.171744in}}%
\pgfpathcurveto{\pgfqpoint{10.283638in}{3.163930in}}{\pgfqpoint{10.294237in}{3.159540in}}{\pgfqpoint{10.305287in}{3.159540in}}%
\pgfpathlineto{\pgfqpoint{10.305287in}{3.159540in}}%
\pgfpathclose%
\pgfusepath{stroke}%
\end{pgfscope}%
\begin{pgfscope}%
\pgfpathrectangle{\pgfqpoint{7.394209in}{0.375000in}}{\pgfqpoint{6.356833in}{5.175000in}}%
\pgfusepath{clip}%
\pgfsetbuttcap%
\pgfsetroundjoin%
\pgfsetlinewidth{1.003750pt}%
\definecolor{currentstroke}{rgb}{0.827451,0.827451,0.827451}%
\pgfsetstrokecolor{currentstroke}%
\pgfsetdash{}{0pt}%
\pgfpathmoveto{\pgfqpoint{12.263616in}{5.461562in}}%
\pgfpathcurveto{\pgfqpoint{12.274667in}{5.461562in}}{\pgfqpoint{12.285266in}{5.465952in}}{\pgfqpoint{12.293079in}{5.473765in}}%
\pgfpathcurveto{\pgfqpoint{12.300893in}{5.481579in}}{\pgfqpoint{12.305283in}{5.492178in}}{\pgfqpoint{12.305283in}{5.503228in}}%
\pgfpathcurveto{\pgfqpoint{12.305283in}{5.514278in}}{\pgfqpoint{12.300893in}{5.524877in}}{\pgfqpoint{12.293079in}{5.532691in}}%
\pgfpathcurveto{\pgfqpoint{12.285266in}{5.540505in}}{\pgfqpoint{12.274667in}{5.544895in}}{\pgfqpoint{12.263616in}{5.544895in}}%
\pgfpathcurveto{\pgfqpoint{12.252566in}{5.544895in}}{\pgfqpoint{12.241967in}{5.540505in}}{\pgfqpoint{12.234154in}{5.532691in}}%
\pgfpathcurveto{\pgfqpoint{12.226340in}{5.524877in}}{\pgfqpoint{12.221950in}{5.514278in}}{\pgfqpoint{12.221950in}{5.503228in}}%
\pgfpathcurveto{\pgfqpoint{12.221950in}{5.492178in}}{\pgfqpoint{12.226340in}{5.481579in}}{\pgfqpoint{12.234154in}{5.473765in}}%
\pgfpathcurveto{\pgfqpoint{12.241967in}{5.465952in}}{\pgfqpoint{12.252566in}{5.461562in}}{\pgfqpoint{12.263616in}{5.461562in}}%
\pgfpathlineto{\pgfqpoint{12.263616in}{5.461562in}}%
\pgfpathclose%
\pgfusepath{stroke}%
\end{pgfscope}%
\begin{pgfscope}%
\pgfpathrectangle{\pgfqpoint{7.394209in}{0.375000in}}{\pgfqpoint{6.356833in}{5.175000in}}%
\pgfusepath{clip}%
\pgfsetbuttcap%
\pgfsetroundjoin%
\pgfsetlinewidth{1.003750pt}%
\definecolor{currentstroke}{rgb}{0.827451,0.827451,0.827451}%
\pgfsetstrokecolor{currentstroke}%
\pgfsetdash{}{0pt}%
\pgfpathmoveto{\pgfqpoint{9.435779in}{4.266970in}}%
\pgfpathcurveto{\pgfqpoint{9.446829in}{4.266970in}}{\pgfqpoint{9.457428in}{4.271360in}}{\pgfqpoint{9.465241in}{4.279174in}}%
\pgfpathcurveto{\pgfqpoint{9.473055in}{4.286987in}}{\pgfqpoint{9.477445in}{4.297586in}}{\pgfqpoint{9.477445in}{4.308636in}}%
\pgfpathcurveto{\pgfqpoint{9.477445in}{4.319686in}}{\pgfqpoint{9.473055in}{4.330285in}}{\pgfqpoint{9.465241in}{4.338099in}}%
\pgfpathcurveto{\pgfqpoint{9.457428in}{4.345913in}}{\pgfqpoint{9.446829in}{4.350303in}}{\pgfqpoint{9.435779in}{4.350303in}}%
\pgfpathcurveto{\pgfqpoint{9.424728in}{4.350303in}}{\pgfqpoint{9.414129in}{4.345913in}}{\pgfqpoint{9.406316in}{4.338099in}}%
\pgfpathcurveto{\pgfqpoint{9.398502in}{4.330285in}}{\pgfqpoint{9.394112in}{4.319686in}}{\pgfqpoint{9.394112in}{4.308636in}}%
\pgfpathcurveto{\pgfqpoint{9.394112in}{4.297586in}}{\pgfqpoint{9.398502in}{4.286987in}}{\pgfqpoint{9.406316in}{4.279174in}}%
\pgfpathcurveto{\pgfqpoint{9.414129in}{4.271360in}}{\pgfqpoint{9.424728in}{4.266970in}}{\pgfqpoint{9.435779in}{4.266970in}}%
\pgfpathlineto{\pgfqpoint{9.435779in}{4.266970in}}%
\pgfpathclose%
\pgfusepath{stroke}%
\end{pgfscope}%
\begin{pgfscope}%
\pgfpathrectangle{\pgfqpoint{7.394209in}{0.375000in}}{\pgfqpoint{6.356833in}{5.175000in}}%
\pgfusepath{clip}%
\pgfsetbuttcap%
\pgfsetroundjoin%
\pgfsetlinewidth{1.003750pt}%
\definecolor{currentstroke}{rgb}{0.827451,0.827451,0.827451}%
\pgfsetstrokecolor{currentstroke}%
\pgfsetdash{}{0pt}%
\pgfpathmoveto{\pgfqpoint{8.329754in}{0.935410in}}%
\pgfpathcurveto{\pgfqpoint{8.340804in}{0.935410in}}{\pgfqpoint{8.351403in}{0.939800in}}{\pgfqpoint{8.359217in}{0.947614in}}%
\pgfpathcurveto{\pgfqpoint{8.367031in}{0.955428in}}{\pgfqpoint{8.371421in}{0.966027in}}{\pgfqpoint{8.371421in}{0.977077in}}%
\pgfpathcurveto{\pgfqpoint{8.371421in}{0.988127in}}{\pgfqpoint{8.367031in}{0.998726in}}{\pgfqpoint{8.359217in}{1.006540in}}%
\pgfpathcurveto{\pgfqpoint{8.351403in}{1.014353in}}{\pgfqpoint{8.340804in}{1.018743in}}{\pgfqpoint{8.329754in}{1.018743in}}%
\pgfpathcurveto{\pgfqpoint{8.318704in}{1.018743in}}{\pgfqpoint{8.308105in}{1.014353in}}{\pgfqpoint{8.300292in}{1.006540in}}%
\pgfpathcurveto{\pgfqpoint{8.292478in}{0.998726in}}{\pgfqpoint{8.288088in}{0.988127in}}{\pgfqpoint{8.288088in}{0.977077in}}%
\pgfpathcurveto{\pgfqpoint{8.288088in}{0.966027in}}{\pgfqpoint{8.292478in}{0.955428in}}{\pgfqpoint{8.300292in}{0.947614in}}%
\pgfpathcurveto{\pgfqpoint{8.308105in}{0.939800in}}{\pgfqpoint{8.318704in}{0.935410in}}{\pgfqpoint{8.329754in}{0.935410in}}%
\pgfpathlineto{\pgfqpoint{8.329754in}{0.935410in}}%
\pgfpathclose%
\pgfusepath{stroke}%
\end{pgfscope}%
\begin{pgfscope}%
\pgfpathrectangle{\pgfqpoint{7.394209in}{0.375000in}}{\pgfqpoint{6.356833in}{5.175000in}}%
\pgfusepath{clip}%
\pgfsetbuttcap%
\pgfsetroundjoin%
\pgfsetlinewidth{1.003750pt}%
\definecolor{currentstroke}{rgb}{0.827451,0.827451,0.827451}%
\pgfsetstrokecolor{currentstroke}%
\pgfsetdash{}{0pt}%
\pgfpathmoveto{\pgfqpoint{11.965622in}{5.478407in}}%
\pgfpathcurveto{\pgfqpoint{11.976672in}{5.478407in}}{\pgfqpoint{11.987271in}{5.482797in}}{\pgfqpoint{11.995085in}{5.490611in}}%
\pgfpathcurveto{\pgfqpoint{12.002898in}{5.498425in}}{\pgfqpoint{12.007289in}{5.509024in}}{\pgfqpoint{12.007289in}{5.520074in}}%
\pgfpathcurveto{\pgfqpoint{12.007289in}{5.531124in}}{\pgfqpoint{12.002898in}{5.541723in}}{\pgfqpoint{11.995085in}{5.549537in}}%
\pgfpathcurveto{\pgfqpoint{11.987271in}{5.557350in}}{\pgfqpoint{11.976672in}{5.561740in}}{\pgfqpoint{11.965622in}{5.561740in}}%
\pgfpathcurveto{\pgfqpoint{11.954572in}{5.561740in}}{\pgfqpoint{11.943973in}{5.557350in}}{\pgfqpoint{11.936159in}{5.549537in}}%
\pgfpathcurveto{\pgfqpoint{11.928346in}{5.541723in}}{\pgfqpoint{11.923955in}{5.531124in}}{\pgfqpoint{11.923955in}{5.520074in}}%
\pgfpathcurveto{\pgfqpoint{11.923955in}{5.509024in}}{\pgfqpoint{11.928346in}{5.498425in}}{\pgfqpoint{11.936159in}{5.490611in}}%
\pgfpathcurveto{\pgfqpoint{11.943973in}{5.482797in}}{\pgfqpoint{11.954572in}{5.478407in}}{\pgfqpoint{11.965622in}{5.478407in}}%
\pgfpathlineto{\pgfqpoint{11.965622in}{5.478407in}}%
\pgfpathclose%
\pgfusepath{stroke}%
\end{pgfscope}%
\begin{pgfscope}%
\pgfpathrectangle{\pgfqpoint{7.394209in}{0.375000in}}{\pgfqpoint{6.356833in}{5.175000in}}%
\pgfusepath{clip}%
\pgfsetbuttcap%
\pgfsetroundjoin%
\pgfsetlinewidth{1.003750pt}%
\definecolor{currentstroke}{rgb}{0.827451,0.827451,0.827451}%
\pgfsetstrokecolor{currentstroke}%
\pgfsetdash{}{0pt}%
\pgfpathmoveto{\pgfqpoint{8.081334in}{2.181484in}}%
\pgfpathcurveto{\pgfqpoint{8.092384in}{2.181484in}}{\pgfqpoint{8.102983in}{2.185875in}}{\pgfqpoint{8.110797in}{2.193688in}}%
\pgfpathcurveto{\pgfqpoint{8.118610in}{2.201502in}}{\pgfqpoint{8.123001in}{2.212101in}}{\pgfqpoint{8.123001in}{2.223151in}}%
\pgfpathcurveto{\pgfqpoint{8.123001in}{2.234201in}}{\pgfqpoint{8.118610in}{2.244800in}}{\pgfqpoint{8.110797in}{2.252614in}}%
\pgfpathcurveto{\pgfqpoint{8.102983in}{2.260427in}}{\pgfqpoint{8.092384in}{2.264818in}}{\pgfqpoint{8.081334in}{2.264818in}}%
\pgfpathcurveto{\pgfqpoint{8.070284in}{2.264818in}}{\pgfqpoint{8.059685in}{2.260427in}}{\pgfqpoint{8.051871in}{2.252614in}}%
\pgfpathcurveto{\pgfqpoint{8.044058in}{2.244800in}}{\pgfqpoint{8.039667in}{2.234201in}}{\pgfqpoint{8.039667in}{2.223151in}}%
\pgfpathcurveto{\pgfqpoint{8.039667in}{2.212101in}}{\pgfqpoint{8.044058in}{2.201502in}}{\pgfqpoint{8.051871in}{2.193688in}}%
\pgfpathcurveto{\pgfqpoint{8.059685in}{2.185875in}}{\pgfqpoint{8.070284in}{2.181484in}}{\pgfqpoint{8.081334in}{2.181484in}}%
\pgfpathlineto{\pgfqpoint{8.081334in}{2.181484in}}%
\pgfpathclose%
\pgfusepath{stroke}%
\end{pgfscope}%
\begin{pgfscope}%
\pgfpathrectangle{\pgfqpoint{7.394209in}{0.375000in}}{\pgfqpoint{6.356833in}{5.175000in}}%
\pgfusepath{clip}%
\pgfsetbuttcap%
\pgfsetroundjoin%
\pgfsetlinewidth{1.003750pt}%
\definecolor{currentstroke}{rgb}{0.827451,0.827451,0.827451}%
\pgfsetstrokecolor{currentstroke}%
\pgfsetdash{}{0pt}%
\pgfpathmoveto{\pgfqpoint{7.424763in}{0.497807in}}%
\pgfpathcurveto{\pgfqpoint{7.435813in}{0.497807in}}{\pgfqpoint{7.446412in}{0.502197in}}{\pgfqpoint{7.454226in}{0.510010in}}%
\pgfpathcurveto{\pgfqpoint{7.462039in}{0.517824in}}{\pgfqpoint{7.466429in}{0.528423in}}{\pgfqpoint{7.466429in}{0.539473in}}%
\pgfpathcurveto{\pgfqpoint{7.466429in}{0.550523in}}{\pgfqpoint{7.462039in}{0.561122in}}{\pgfqpoint{7.454226in}{0.568936in}}%
\pgfpathcurveto{\pgfqpoint{7.446412in}{0.576750in}}{\pgfqpoint{7.435813in}{0.581140in}}{\pgfqpoint{7.424763in}{0.581140in}}%
\pgfpathcurveto{\pgfqpoint{7.413713in}{0.581140in}}{\pgfqpoint{7.403114in}{0.576750in}}{\pgfqpoint{7.395300in}{0.568936in}}%
\pgfpathcurveto{\pgfqpoint{7.387486in}{0.561122in}}{\pgfqpoint{7.383096in}{0.550523in}}{\pgfqpoint{7.383096in}{0.539473in}}%
\pgfpathcurveto{\pgfqpoint{7.383096in}{0.528423in}}{\pgfqpoint{7.387486in}{0.517824in}}{\pgfqpoint{7.395300in}{0.510010in}}%
\pgfpathcurveto{\pgfqpoint{7.403114in}{0.502197in}}{\pgfqpoint{7.413713in}{0.497807in}}{\pgfqpoint{7.424763in}{0.497807in}}%
\pgfpathlineto{\pgfqpoint{7.424763in}{0.497807in}}%
\pgfpathclose%
\pgfusepath{stroke}%
\end{pgfscope}%
\begin{pgfscope}%
\pgfpathrectangle{\pgfqpoint{7.394209in}{0.375000in}}{\pgfqpoint{6.356833in}{5.175000in}}%
\pgfusepath{clip}%
\pgfsetbuttcap%
\pgfsetroundjoin%
\pgfsetlinewidth{1.003750pt}%
\definecolor{currentstroke}{rgb}{0.827451,0.827451,0.827451}%
\pgfsetstrokecolor{currentstroke}%
\pgfsetdash{}{0pt}%
\pgfpathmoveto{\pgfqpoint{12.263616in}{5.459081in}}%
\pgfpathcurveto{\pgfqpoint{12.274667in}{5.459081in}}{\pgfqpoint{12.285266in}{5.463471in}}{\pgfqpoint{12.293079in}{5.471285in}}%
\pgfpathcurveto{\pgfqpoint{12.300893in}{5.479098in}}{\pgfqpoint{12.305283in}{5.489697in}}{\pgfqpoint{12.305283in}{5.500747in}}%
\pgfpathcurveto{\pgfqpoint{12.305283in}{5.511798in}}{\pgfqpoint{12.300893in}{5.522397in}}{\pgfqpoint{12.293079in}{5.530210in}}%
\pgfpathcurveto{\pgfqpoint{12.285266in}{5.538024in}}{\pgfqpoint{12.274667in}{5.542414in}}{\pgfqpoint{12.263616in}{5.542414in}}%
\pgfpathcurveto{\pgfqpoint{12.252566in}{5.542414in}}{\pgfqpoint{12.241967in}{5.538024in}}{\pgfqpoint{12.234154in}{5.530210in}}%
\pgfpathcurveto{\pgfqpoint{12.226340in}{5.522397in}}{\pgfqpoint{12.221950in}{5.511798in}}{\pgfqpoint{12.221950in}{5.500747in}}%
\pgfpathcurveto{\pgfqpoint{12.221950in}{5.489697in}}{\pgfqpoint{12.226340in}{5.479098in}}{\pgfqpoint{12.234154in}{5.471285in}}%
\pgfpathcurveto{\pgfqpoint{12.241967in}{5.463471in}}{\pgfqpoint{12.252566in}{5.459081in}}{\pgfqpoint{12.263616in}{5.459081in}}%
\pgfpathlineto{\pgfqpoint{12.263616in}{5.459081in}}%
\pgfpathclose%
\pgfusepath{stroke}%
\end{pgfscope}%
\begin{pgfscope}%
\pgfpathrectangle{\pgfqpoint{7.394209in}{0.375000in}}{\pgfqpoint{6.356833in}{5.175000in}}%
\pgfusepath{clip}%
\pgfsetbuttcap%
\pgfsetroundjoin%
\pgfsetlinewidth{1.003750pt}%
\definecolor{currentstroke}{rgb}{0.827451,0.827451,0.827451}%
\pgfsetstrokecolor{currentstroke}%
\pgfsetdash{}{0pt}%
\pgfpathmoveto{\pgfqpoint{7.869324in}{0.698769in}}%
\pgfpathcurveto{\pgfqpoint{7.880374in}{0.698769in}}{\pgfqpoint{7.890973in}{0.703159in}}{\pgfqpoint{7.898787in}{0.710973in}}%
\pgfpathcurveto{\pgfqpoint{7.906601in}{0.718786in}}{\pgfqpoint{7.910991in}{0.729385in}}{\pgfqpoint{7.910991in}{0.740436in}}%
\pgfpathcurveto{\pgfqpoint{7.910991in}{0.751486in}}{\pgfqpoint{7.906601in}{0.762085in}}{\pgfqpoint{7.898787in}{0.769898in}}%
\pgfpathcurveto{\pgfqpoint{7.890973in}{0.777712in}}{\pgfqpoint{7.880374in}{0.782102in}}{\pgfqpoint{7.869324in}{0.782102in}}%
\pgfpathcurveto{\pgfqpoint{7.858274in}{0.782102in}}{\pgfqpoint{7.847675in}{0.777712in}}{\pgfqpoint{7.839862in}{0.769898in}}%
\pgfpathcurveto{\pgfqpoint{7.832048in}{0.762085in}}{\pgfqpoint{7.827658in}{0.751486in}}{\pgfqpoint{7.827658in}{0.740436in}}%
\pgfpathcurveto{\pgfqpoint{7.827658in}{0.729385in}}{\pgfqpoint{7.832048in}{0.718786in}}{\pgfqpoint{7.839862in}{0.710973in}}%
\pgfpathcurveto{\pgfqpoint{7.847675in}{0.703159in}}{\pgfqpoint{7.858274in}{0.698769in}}{\pgfqpoint{7.869324in}{0.698769in}}%
\pgfpathlineto{\pgfqpoint{7.869324in}{0.698769in}}%
\pgfpathclose%
\pgfusepath{stroke}%
\end{pgfscope}%
\begin{pgfscope}%
\pgfpathrectangle{\pgfqpoint{7.394209in}{0.375000in}}{\pgfqpoint{6.356833in}{5.175000in}}%
\pgfusepath{clip}%
\pgfsetbuttcap%
\pgfsetroundjoin%
\pgfsetlinewidth{1.003750pt}%
\definecolor{currentstroke}{rgb}{0.827451,0.827451,0.827451}%
\pgfsetstrokecolor{currentstroke}%
\pgfsetdash{}{0pt}%
\pgfpathmoveto{\pgfqpoint{11.277417in}{5.372688in}}%
\pgfpathcurveto{\pgfqpoint{11.288467in}{5.372688in}}{\pgfqpoint{11.299066in}{5.377078in}}{\pgfqpoint{11.306879in}{5.384892in}}%
\pgfpathcurveto{\pgfqpoint{11.314693in}{5.392705in}}{\pgfqpoint{11.319083in}{5.403304in}}{\pgfqpoint{11.319083in}{5.414355in}}%
\pgfpathcurveto{\pgfqpoint{11.319083in}{5.425405in}}{\pgfqpoint{11.314693in}{5.436004in}}{\pgfqpoint{11.306879in}{5.443817in}}%
\pgfpathcurveto{\pgfqpoint{11.299066in}{5.451631in}}{\pgfqpoint{11.288467in}{5.456021in}}{\pgfqpoint{11.277417in}{5.456021in}}%
\pgfpathcurveto{\pgfqpoint{11.266367in}{5.456021in}}{\pgfqpoint{11.255768in}{5.451631in}}{\pgfqpoint{11.247954in}{5.443817in}}%
\pgfpathcurveto{\pgfqpoint{11.240140in}{5.436004in}}{\pgfqpoint{11.235750in}{5.425405in}}{\pgfqpoint{11.235750in}{5.414355in}}%
\pgfpathcurveto{\pgfqpoint{11.235750in}{5.403304in}}{\pgfqpoint{11.240140in}{5.392705in}}{\pgfqpoint{11.247954in}{5.384892in}}%
\pgfpathcurveto{\pgfqpoint{11.255768in}{5.377078in}}{\pgfqpoint{11.266367in}{5.372688in}}{\pgfqpoint{11.277417in}{5.372688in}}%
\pgfpathlineto{\pgfqpoint{11.277417in}{5.372688in}}%
\pgfpathclose%
\pgfusepath{stroke}%
\end{pgfscope}%
\begin{pgfscope}%
\pgfpathrectangle{\pgfqpoint{7.394209in}{0.375000in}}{\pgfqpoint{6.356833in}{5.175000in}}%
\pgfusepath{clip}%
\pgfsetbuttcap%
\pgfsetroundjoin%
\pgfsetlinewidth{1.003750pt}%
\definecolor{currentstroke}{rgb}{0.827451,0.827451,0.827451}%
\pgfsetstrokecolor{currentstroke}%
\pgfsetdash{}{0pt}%
\pgfpathmoveto{\pgfqpoint{7.417878in}{1.026764in}}%
\pgfpathcurveto{\pgfqpoint{7.428928in}{1.026764in}}{\pgfqpoint{7.439527in}{1.031154in}}{\pgfqpoint{7.447341in}{1.038968in}}%
\pgfpathcurveto{\pgfqpoint{7.455154in}{1.046782in}}{\pgfqpoint{7.459544in}{1.057381in}}{\pgfqpoint{7.459544in}{1.068431in}}%
\pgfpathcurveto{\pgfqpoint{7.459544in}{1.079481in}}{\pgfqpoint{7.455154in}{1.090080in}}{\pgfqpoint{7.447341in}{1.097894in}}%
\pgfpathcurveto{\pgfqpoint{7.439527in}{1.105707in}}{\pgfqpoint{7.428928in}{1.110098in}}{\pgfqpoint{7.417878in}{1.110098in}}%
\pgfpathcurveto{\pgfqpoint{7.406828in}{1.110098in}}{\pgfqpoint{7.396229in}{1.105707in}}{\pgfqpoint{7.388415in}{1.097894in}}%
\pgfpathcurveto{\pgfqpoint{7.380601in}{1.090080in}}{\pgfqpoint{7.376211in}{1.079481in}}{\pgfqpoint{7.376211in}{1.068431in}}%
\pgfpathcurveto{\pgfqpoint{7.376211in}{1.057381in}}{\pgfqpoint{7.380601in}{1.046782in}}{\pgfqpoint{7.388415in}{1.038968in}}%
\pgfpathcurveto{\pgfqpoint{7.396229in}{1.031154in}}{\pgfqpoint{7.406828in}{1.026764in}}{\pgfqpoint{7.417878in}{1.026764in}}%
\pgfpathlineto{\pgfqpoint{7.417878in}{1.026764in}}%
\pgfpathclose%
\pgfusepath{stroke}%
\end{pgfscope}%
\begin{pgfscope}%
\pgfpathrectangle{\pgfqpoint{7.394209in}{0.375000in}}{\pgfqpoint{6.356833in}{5.175000in}}%
\pgfusepath{clip}%
\pgfsetbuttcap%
\pgfsetroundjoin%
\pgfsetlinewidth{1.003750pt}%
\definecolor{currentstroke}{rgb}{0.827451,0.827451,0.827451}%
\pgfsetstrokecolor{currentstroke}%
\pgfsetdash{}{0pt}%
\pgfpathmoveto{\pgfqpoint{8.174049in}{2.136785in}}%
\pgfpathcurveto{\pgfqpoint{8.185099in}{2.136785in}}{\pgfqpoint{8.195698in}{2.141175in}}{\pgfqpoint{8.203512in}{2.148988in}}%
\pgfpathcurveto{\pgfqpoint{8.211325in}{2.156802in}}{\pgfqpoint{8.215715in}{2.167401in}}{\pgfqpoint{8.215715in}{2.178451in}}%
\pgfpathcurveto{\pgfqpoint{8.215715in}{2.189501in}}{\pgfqpoint{8.211325in}{2.200100in}}{\pgfqpoint{8.203512in}{2.207914in}}%
\pgfpathcurveto{\pgfqpoint{8.195698in}{2.215728in}}{\pgfqpoint{8.185099in}{2.220118in}}{\pgfqpoint{8.174049in}{2.220118in}}%
\pgfpathcurveto{\pgfqpoint{8.162999in}{2.220118in}}{\pgfqpoint{8.152400in}{2.215728in}}{\pgfqpoint{8.144586in}{2.207914in}}%
\pgfpathcurveto{\pgfqpoint{8.136772in}{2.200100in}}{\pgfqpoint{8.132382in}{2.189501in}}{\pgfqpoint{8.132382in}{2.178451in}}%
\pgfpathcurveto{\pgfqpoint{8.132382in}{2.167401in}}{\pgfqpoint{8.136772in}{2.156802in}}{\pgfqpoint{8.144586in}{2.148988in}}%
\pgfpathcurveto{\pgfqpoint{8.152400in}{2.141175in}}{\pgfqpoint{8.162999in}{2.136785in}}{\pgfqpoint{8.174049in}{2.136785in}}%
\pgfpathlineto{\pgfqpoint{8.174049in}{2.136785in}}%
\pgfpathclose%
\pgfusepath{stroke}%
\end{pgfscope}%
\begin{pgfscope}%
\pgfpathrectangle{\pgfqpoint{7.394209in}{0.375000in}}{\pgfqpoint{6.356833in}{5.175000in}}%
\pgfusepath{clip}%
\pgfsetbuttcap%
\pgfsetroundjoin%
\pgfsetlinewidth{1.003750pt}%
\definecolor{currentstroke}{rgb}{0.827451,0.827451,0.827451}%
\pgfsetstrokecolor{currentstroke}%
\pgfsetdash{}{0pt}%
\pgfpathmoveto{\pgfqpoint{9.917688in}{3.098792in}}%
\pgfpathcurveto{\pgfqpoint{9.928739in}{3.098792in}}{\pgfqpoint{9.939338in}{3.103182in}}{\pgfqpoint{9.947151in}{3.110996in}}%
\pgfpathcurveto{\pgfqpoint{9.954965in}{3.118809in}}{\pgfqpoint{9.959355in}{3.129409in}}{\pgfqpoint{9.959355in}{3.140459in}}%
\pgfpathcurveto{\pgfqpoint{9.959355in}{3.151509in}}{\pgfqpoint{9.954965in}{3.162108in}}{\pgfqpoint{9.947151in}{3.169921in}}%
\pgfpathcurveto{\pgfqpoint{9.939338in}{3.177735in}}{\pgfqpoint{9.928739in}{3.182125in}}{\pgfqpoint{9.917688in}{3.182125in}}%
\pgfpathcurveto{\pgfqpoint{9.906638in}{3.182125in}}{\pgfqpoint{9.896039in}{3.177735in}}{\pgfqpoint{9.888226in}{3.169921in}}%
\pgfpathcurveto{\pgfqpoint{9.880412in}{3.162108in}}{\pgfqpoint{9.876022in}{3.151509in}}{\pgfqpoint{9.876022in}{3.140459in}}%
\pgfpathcurveto{\pgfqpoint{9.876022in}{3.129409in}}{\pgfqpoint{9.880412in}{3.118809in}}{\pgfqpoint{9.888226in}{3.110996in}}%
\pgfpathcurveto{\pgfqpoint{9.896039in}{3.103182in}}{\pgfqpoint{9.906638in}{3.098792in}}{\pgfqpoint{9.917688in}{3.098792in}}%
\pgfpathlineto{\pgfqpoint{9.917688in}{3.098792in}}%
\pgfpathclose%
\pgfusepath{stroke}%
\end{pgfscope}%
\begin{pgfscope}%
\pgfpathrectangle{\pgfqpoint{7.394209in}{0.375000in}}{\pgfqpoint{6.356833in}{5.175000in}}%
\pgfusepath{clip}%
\pgfsetbuttcap%
\pgfsetroundjoin%
\pgfsetlinewidth{1.003750pt}%
\definecolor{currentstroke}{rgb}{0.827451,0.827451,0.827451}%
\pgfsetstrokecolor{currentstroke}%
\pgfsetdash{}{0pt}%
\pgfpathmoveto{\pgfqpoint{10.810528in}{4.206547in}}%
\pgfpathcurveto{\pgfqpoint{10.821578in}{4.206547in}}{\pgfqpoint{10.832177in}{4.210937in}}{\pgfqpoint{10.839991in}{4.218751in}}%
\pgfpathcurveto{\pgfqpoint{10.847804in}{4.226565in}}{\pgfqpoint{10.852195in}{4.237164in}}{\pgfqpoint{10.852195in}{4.248214in}}%
\pgfpathcurveto{\pgfqpoint{10.852195in}{4.259264in}}{\pgfqpoint{10.847804in}{4.269863in}}{\pgfqpoint{10.839991in}{4.277677in}}%
\pgfpathcurveto{\pgfqpoint{10.832177in}{4.285490in}}{\pgfqpoint{10.821578in}{4.289881in}}{\pgfqpoint{10.810528in}{4.289881in}}%
\pgfpathcurveto{\pgfqpoint{10.799478in}{4.289881in}}{\pgfqpoint{10.788879in}{4.285490in}}{\pgfqpoint{10.781065in}{4.277677in}}%
\pgfpathcurveto{\pgfqpoint{10.773252in}{4.269863in}}{\pgfqpoint{10.768861in}{4.259264in}}{\pgfqpoint{10.768861in}{4.248214in}}%
\pgfpathcurveto{\pgfqpoint{10.768861in}{4.237164in}}{\pgfqpoint{10.773252in}{4.226565in}}{\pgfqpoint{10.781065in}{4.218751in}}%
\pgfpathcurveto{\pgfqpoint{10.788879in}{4.210937in}}{\pgfqpoint{10.799478in}{4.206547in}}{\pgfqpoint{10.810528in}{4.206547in}}%
\pgfpathlineto{\pgfqpoint{10.810528in}{4.206547in}}%
\pgfpathclose%
\pgfusepath{stroke}%
\end{pgfscope}%
\begin{pgfscope}%
\pgfpathrectangle{\pgfqpoint{7.394209in}{0.375000in}}{\pgfqpoint{6.356833in}{5.175000in}}%
\pgfusepath{clip}%
\pgfsetbuttcap%
\pgfsetroundjoin%
\pgfsetlinewidth{1.003750pt}%
\definecolor{currentstroke}{rgb}{0.827451,0.827451,0.827451}%
\pgfsetstrokecolor{currentstroke}%
\pgfsetdash{}{0pt}%
\pgfpathmoveto{\pgfqpoint{12.789422in}{5.474188in}}%
\pgfpathcurveto{\pgfqpoint{12.800472in}{5.474188in}}{\pgfqpoint{12.811071in}{5.478579in}}{\pgfqpoint{12.818885in}{5.486392in}}%
\pgfpathcurveto{\pgfqpoint{12.826698in}{5.494206in}}{\pgfqpoint{12.831089in}{5.504805in}}{\pgfqpoint{12.831089in}{5.515855in}}%
\pgfpathcurveto{\pgfqpoint{12.831089in}{5.526905in}}{\pgfqpoint{12.826698in}{5.537504in}}{\pgfqpoint{12.818885in}{5.545318in}}%
\pgfpathcurveto{\pgfqpoint{12.811071in}{5.553131in}}{\pgfqpoint{12.800472in}{5.557522in}}{\pgfqpoint{12.789422in}{5.557522in}}%
\pgfpathcurveto{\pgfqpoint{12.778372in}{5.557522in}}{\pgfqpoint{12.767773in}{5.553131in}}{\pgfqpoint{12.759959in}{5.545318in}}%
\pgfpathcurveto{\pgfqpoint{12.752146in}{5.537504in}}{\pgfqpoint{12.747755in}{5.526905in}}{\pgfqpoint{12.747755in}{5.515855in}}%
\pgfpathcurveto{\pgfqpoint{12.747755in}{5.504805in}}{\pgfqpoint{12.752146in}{5.494206in}}{\pgfqpoint{12.759959in}{5.486392in}}%
\pgfpathcurveto{\pgfqpoint{12.767773in}{5.478579in}}{\pgfqpoint{12.778372in}{5.474188in}}{\pgfqpoint{12.789422in}{5.474188in}}%
\pgfpathlineto{\pgfqpoint{12.789422in}{5.474188in}}%
\pgfpathclose%
\pgfusepath{stroke}%
\end{pgfscope}%
\begin{pgfscope}%
\pgfpathrectangle{\pgfqpoint{7.394209in}{0.375000in}}{\pgfqpoint{6.356833in}{5.175000in}}%
\pgfusepath{clip}%
\pgfsetbuttcap%
\pgfsetroundjoin%
\pgfsetlinewidth{1.003750pt}%
\definecolor{currentstroke}{rgb}{0.827451,0.827451,0.827451}%
\pgfsetstrokecolor{currentstroke}%
\pgfsetdash{}{0pt}%
\pgfpathmoveto{\pgfqpoint{9.828626in}{2.801950in}}%
\pgfpathcurveto{\pgfqpoint{9.839677in}{2.801950in}}{\pgfqpoint{9.850276in}{2.806341in}}{\pgfqpoint{9.858089in}{2.814154in}}%
\pgfpathcurveto{\pgfqpoint{9.865903in}{2.821968in}}{\pgfqpoint{9.870293in}{2.832567in}}{\pgfqpoint{9.870293in}{2.843617in}}%
\pgfpathcurveto{\pgfqpoint{9.870293in}{2.854667in}}{\pgfqpoint{9.865903in}{2.865266in}}{\pgfqpoint{9.858089in}{2.873080in}}%
\pgfpathcurveto{\pgfqpoint{9.850276in}{2.880893in}}{\pgfqpoint{9.839677in}{2.885284in}}{\pgfqpoint{9.828626in}{2.885284in}}%
\pgfpathcurveto{\pgfqpoint{9.817576in}{2.885284in}}{\pgfqpoint{9.806977in}{2.880893in}}{\pgfqpoint{9.799164in}{2.873080in}}%
\pgfpathcurveto{\pgfqpoint{9.791350in}{2.865266in}}{\pgfqpoint{9.786960in}{2.854667in}}{\pgfqpoint{9.786960in}{2.843617in}}%
\pgfpathcurveto{\pgfqpoint{9.786960in}{2.832567in}}{\pgfqpoint{9.791350in}{2.821968in}}{\pgfqpoint{9.799164in}{2.814154in}}%
\pgfpathcurveto{\pgfqpoint{9.806977in}{2.806341in}}{\pgfqpoint{9.817576in}{2.801950in}}{\pgfqpoint{9.828626in}{2.801950in}}%
\pgfpathlineto{\pgfqpoint{9.828626in}{2.801950in}}%
\pgfpathclose%
\pgfusepath{stroke}%
\end{pgfscope}%
\begin{pgfscope}%
\pgfpathrectangle{\pgfqpoint{7.394209in}{0.375000in}}{\pgfqpoint{6.356833in}{5.175000in}}%
\pgfusepath{clip}%
\pgfsetbuttcap%
\pgfsetroundjoin%
\pgfsetlinewidth{1.003750pt}%
\definecolor{currentstroke}{rgb}{0.827451,0.827451,0.827451}%
\pgfsetstrokecolor{currentstroke}%
\pgfsetdash{}{0pt}%
\pgfpathmoveto{\pgfqpoint{12.864303in}{5.449974in}}%
\pgfpathcurveto{\pgfqpoint{12.875353in}{5.449974in}}{\pgfqpoint{12.885952in}{5.454364in}}{\pgfqpoint{12.893766in}{5.462178in}}%
\pgfpathcurveto{\pgfqpoint{12.901579in}{5.469992in}}{\pgfqpoint{12.905970in}{5.480591in}}{\pgfqpoint{12.905970in}{5.491641in}}%
\pgfpathcurveto{\pgfqpoint{12.905970in}{5.502691in}}{\pgfqpoint{12.901579in}{5.513290in}}{\pgfqpoint{12.893766in}{5.521104in}}%
\pgfpathcurveto{\pgfqpoint{12.885952in}{5.528917in}}{\pgfqpoint{12.875353in}{5.533308in}}{\pgfqpoint{12.864303in}{5.533308in}}%
\pgfpathcurveto{\pgfqpoint{12.853253in}{5.533308in}}{\pgfqpoint{12.842654in}{5.528917in}}{\pgfqpoint{12.834840in}{5.521104in}}%
\pgfpathcurveto{\pgfqpoint{12.827027in}{5.513290in}}{\pgfqpoint{12.822636in}{5.502691in}}{\pgfqpoint{12.822636in}{5.491641in}}%
\pgfpathcurveto{\pgfqpoint{12.822636in}{5.480591in}}{\pgfqpoint{12.827027in}{5.469992in}}{\pgfqpoint{12.834840in}{5.462178in}}%
\pgfpathcurveto{\pgfqpoint{12.842654in}{5.454364in}}{\pgfqpoint{12.853253in}{5.449974in}}{\pgfqpoint{12.864303in}{5.449974in}}%
\pgfpathlineto{\pgfqpoint{12.864303in}{5.449974in}}%
\pgfpathclose%
\pgfusepath{stroke}%
\end{pgfscope}%
\begin{pgfscope}%
\pgfpathrectangle{\pgfqpoint{7.394209in}{0.375000in}}{\pgfqpoint{6.356833in}{5.175000in}}%
\pgfusepath{clip}%
\pgfsetbuttcap%
\pgfsetroundjoin%
\pgfsetlinewidth{1.003750pt}%
\definecolor{currentstroke}{rgb}{0.827451,0.827451,0.827451}%
\pgfsetstrokecolor{currentstroke}%
\pgfsetdash{}{0pt}%
\pgfpathmoveto{\pgfqpoint{12.261477in}{5.443780in}}%
\pgfpathcurveto{\pgfqpoint{12.272527in}{5.443780in}}{\pgfqpoint{12.283126in}{5.448170in}}{\pgfqpoint{12.290940in}{5.455984in}}%
\pgfpathcurveto{\pgfqpoint{12.298753in}{5.463797in}}{\pgfqpoint{12.303144in}{5.474396in}}{\pgfqpoint{12.303144in}{5.485446in}}%
\pgfpathcurveto{\pgfqpoint{12.303144in}{5.496497in}}{\pgfqpoint{12.298753in}{5.507096in}}{\pgfqpoint{12.290940in}{5.514909in}}%
\pgfpathcurveto{\pgfqpoint{12.283126in}{5.522723in}}{\pgfqpoint{12.272527in}{5.527113in}}{\pgfqpoint{12.261477in}{5.527113in}}%
\pgfpathcurveto{\pgfqpoint{12.250427in}{5.527113in}}{\pgfqpoint{12.239828in}{5.522723in}}{\pgfqpoint{12.232014in}{5.514909in}}%
\pgfpathcurveto{\pgfqpoint{12.224201in}{5.507096in}}{\pgfqpoint{12.219810in}{5.496497in}}{\pgfqpoint{12.219810in}{5.485446in}}%
\pgfpathcurveto{\pgfqpoint{12.219810in}{5.474396in}}{\pgfqpoint{12.224201in}{5.463797in}}{\pgfqpoint{12.232014in}{5.455984in}}%
\pgfpathcurveto{\pgfqpoint{12.239828in}{5.448170in}}{\pgfqpoint{12.250427in}{5.443780in}}{\pgfqpoint{12.261477in}{5.443780in}}%
\pgfpathlineto{\pgfqpoint{12.261477in}{5.443780in}}%
\pgfpathclose%
\pgfusepath{stroke}%
\end{pgfscope}%
\begin{pgfscope}%
\pgfpathrectangle{\pgfqpoint{7.394209in}{0.375000in}}{\pgfqpoint{6.356833in}{5.175000in}}%
\pgfusepath{clip}%
\pgfsetbuttcap%
\pgfsetroundjoin%
\pgfsetlinewidth{1.003750pt}%
\definecolor{currentstroke}{rgb}{0.827451,0.827451,0.827451}%
\pgfsetstrokecolor{currentstroke}%
\pgfsetdash{}{0pt}%
\pgfpathmoveto{\pgfqpoint{8.508522in}{2.532978in}}%
\pgfpathcurveto{\pgfqpoint{8.519572in}{2.532978in}}{\pgfqpoint{8.530171in}{2.537369in}}{\pgfqpoint{8.537985in}{2.545182in}}%
\pgfpathcurveto{\pgfqpoint{8.545799in}{2.552996in}}{\pgfqpoint{8.550189in}{2.563595in}}{\pgfqpoint{8.550189in}{2.574645in}}%
\pgfpathcurveto{\pgfqpoint{8.550189in}{2.585695in}}{\pgfqpoint{8.545799in}{2.596294in}}{\pgfqpoint{8.537985in}{2.604108in}}%
\pgfpathcurveto{\pgfqpoint{8.530171in}{2.611921in}}{\pgfqpoint{8.519572in}{2.616312in}}{\pgfqpoint{8.508522in}{2.616312in}}%
\pgfpathcurveto{\pgfqpoint{8.497472in}{2.616312in}}{\pgfqpoint{8.486873in}{2.611921in}}{\pgfqpoint{8.479059in}{2.604108in}}%
\pgfpathcurveto{\pgfqpoint{8.471246in}{2.596294in}}{\pgfqpoint{8.466856in}{2.585695in}}{\pgfqpoint{8.466856in}{2.574645in}}%
\pgfpathcurveto{\pgfqpoint{8.466856in}{2.563595in}}{\pgfqpoint{8.471246in}{2.552996in}}{\pgfqpoint{8.479059in}{2.545182in}}%
\pgfpathcurveto{\pgfqpoint{8.486873in}{2.537369in}}{\pgfqpoint{8.497472in}{2.532978in}}{\pgfqpoint{8.508522in}{2.532978in}}%
\pgfpathlineto{\pgfqpoint{8.508522in}{2.532978in}}%
\pgfpathclose%
\pgfusepath{stroke}%
\end{pgfscope}%
\begin{pgfscope}%
\pgfpathrectangle{\pgfqpoint{7.394209in}{0.375000in}}{\pgfqpoint{6.356833in}{5.175000in}}%
\pgfusepath{clip}%
\pgfsetbuttcap%
\pgfsetroundjoin%
\pgfsetlinewidth{1.003750pt}%
\definecolor{currentstroke}{rgb}{0.827451,0.827451,0.827451}%
\pgfsetstrokecolor{currentstroke}%
\pgfsetdash{}{0pt}%
\pgfpathmoveto{\pgfqpoint{10.198574in}{2.854355in}}%
\pgfpathcurveto{\pgfqpoint{10.209625in}{2.854355in}}{\pgfqpoint{10.220224in}{2.858745in}}{\pgfqpoint{10.228037in}{2.866559in}}%
\pgfpathcurveto{\pgfqpoint{10.235851in}{2.874372in}}{\pgfqpoint{10.240241in}{2.884971in}}{\pgfqpoint{10.240241in}{2.896021in}}%
\pgfpathcurveto{\pgfqpoint{10.240241in}{2.907072in}}{\pgfqpoint{10.235851in}{2.917671in}}{\pgfqpoint{10.228037in}{2.925484in}}%
\pgfpathcurveto{\pgfqpoint{10.220224in}{2.933298in}}{\pgfqpoint{10.209625in}{2.937688in}}{\pgfqpoint{10.198574in}{2.937688in}}%
\pgfpathcurveto{\pgfqpoint{10.187524in}{2.937688in}}{\pgfqpoint{10.176925in}{2.933298in}}{\pgfqpoint{10.169112in}{2.925484in}}%
\pgfpathcurveto{\pgfqpoint{10.161298in}{2.917671in}}{\pgfqpoint{10.156908in}{2.907072in}}{\pgfqpoint{10.156908in}{2.896021in}}%
\pgfpathcurveto{\pgfqpoint{10.156908in}{2.884971in}}{\pgfqpoint{10.161298in}{2.874372in}}{\pgfqpoint{10.169112in}{2.866559in}}%
\pgfpathcurveto{\pgfqpoint{10.176925in}{2.858745in}}{\pgfqpoint{10.187524in}{2.854355in}}{\pgfqpoint{10.198574in}{2.854355in}}%
\pgfpathlineto{\pgfqpoint{10.198574in}{2.854355in}}%
\pgfpathclose%
\pgfusepath{stroke}%
\end{pgfscope}%
\begin{pgfscope}%
\pgfpathrectangle{\pgfqpoint{7.394209in}{0.375000in}}{\pgfqpoint{6.356833in}{5.175000in}}%
\pgfusepath{clip}%
\pgfsetbuttcap%
\pgfsetroundjoin%
\pgfsetlinewidth{1.003750pt}%
\definecolor{currentstroke}{rgb}{0.827451,0.827451,0.827451}%
\pgfsetstrokecolor{currentstroke}%
\pgfsetdash{}{0pt}%
\pgfpathmoveto{\pgfqpoint{7.417878in}{0.542504in}}%
\pgfpathcurveto{\pgfqpoint{7.428928in}{0.542504in}}{\pgfqpoint{7.439527in}{0.546894in}}{\pgfqpoint{7.447341in}{0.554708in}}%
\pgfpathcurveto{\pgfqpoint{7.455154in}{0.562522in}}{\pgfqpoint{7.459544in}{0.573121in}}{\pgfqpoint{7.459544in}{0.584171in}}%
\pgfpathcurveto{\pgfqpoint{7.459544in}{0.595221in}}{\pgfqpoint{7.455154in}{0.605820in}}{\pgfqpoint{7.447341in}{0.613634in}}%
\pgfpathcurveto{\pgfqpoint{7.439527in}{0.621447in}}{\pgfqpoint{7.428928in}{0.625837in}}{\pgfqpoint{7.417878in}{0.625837in}}%
\pgfpathcurveto{\pgfqpoint{7.406828in}{0.625837in}}{\pgfqpoint{7.396229in}{0.621447in}}{\pgfqpoint{7.388415in}{0.613634in}}%
\pgfpathcurveto{\pgfqpoint{7.380601in}{0.605820in}}{\pgfqpoint{7.376211in}{0.595221in}}{\pgfqpoint{7.376211in}{0.584171in}}%
\pgfpathcurveto{\pgfqpoint{7.376211in}{0.573121in}}{\pgfqpoint{7.380601in}{0.562522in}}{\pgfqpoint{7.388415in}{0.554708in}}%
\pgfpathcurveto{\pgfqpoint{7.396229in}{0.546894in}}{\pgfqpoint{7.406828in}{0.542504in}}{\pgfqpoint{7.417878in}{0.542504in}}%
\pgfpathlineto{\pgfqpoint{7.417878in}{0.542504in}}%
\pgfpathclose%
\pgfusepath{stroke}%
\end{pgfscope}%
\begin{pgfscope}%
\pgfpathrectangle{\pgfqpoint{7.394209in}{0.375000in}}{\pgfqpoint{6.356833in}{5.175000in}}%
\pgfusepath{clip}%
\pgfsetbuttcap%
\pgfsetroundjoin%
\pgfsetlinewidth{1.003750pt}%
\definecolor{currentstroke}{rgb}{0.827451,0.827451,0.827451}%
\pgfsetstrokecolor{currentstroke}%
\pgfsetdash{}{0pt}%
\pgfpathmoveto{\pgfqpoint{10.503290in}{5.084489in}}%
\pgfpathcurveto{\pgfqpoint{10.514340in}{5.084489in}}{\pgfqpoint{10.524939in}{5.088880in}}{\pgfqpoint{10.532753in}{5.096693in}}%
\pgfpathcurveto{\pgfqpoint{10.540566in}{5.104507in}}{\pgfqpoint{10.544957in}{5.115106in}}{\pgfqpoint{10.544957in}{5.126156in}}%
\pgfpathcurveto{\pgfqpoint{10.544957in}{5.137206in}}{\pgfqpoint{10.540566in}{5.147805in}}{\pgfqpoint{10.532753in}{5.155619in}}%
\pgfpathcurveto{\pgfqpoint{10.524939in}{5.163432in}}{\pgfqpoint{10.514340in}{5.167823in}}{\pgfqpoint{10.503290in}{5.167823in}}%
\pgfpathcurveto{\pgfqpoint{10.492240in}{5.167823in}}{\pgfqpoint{10.481641in}{5.163432in}}{\pgfqpoint{10.473827in}{5.155619in}}%
\pgfpathcurveto{\pgfqpoint{10.466013in}{5.147805in}}{\pgfqpoint{10.461623in}{5.137206in}}{\pgfqpoint{10.461623in}{5.126156in}}%
\pgfpathcurveto{\pgfqpoint{10.461623in}{5.115106in}}{\pgfqpoint{10.466013in}{5.104507in}}{\pgfqpoint{10.473827in}{5.096693in}}%
\pgfpathcurveto{\pgfqpoint{10.481641in}{5.088880in}}{\pgfqpoint{10.492240in}{5.084489in}}{\pgfqpoint{10.503290in}{5.084489in}}%
\pgfpathlineto{\pgfqpoint{10.503290in}{5.084489in}}%
\pgfpathclose%
\pgfusepath{stroke}%
\end{pgfscope}%
\begin{pgfscope}%
\pgfpathrectangle{\pgfqpoint{7.394209in}{0.375000in}}{\pgfqpoint{6.356833in}{5.175000in}}%
\pgfusepath{clip}%
\pgfsetbuttcap%
\pgfsetroundjoin%
\pgfsetlinewidth{1.003750pt}%
\definecolor{currentstroke}{rgb}{0.827451,0.827451,0.827451}%
\pgfsetstrokecolor{currentstroke}%
\pgfsetdash{}{0pt}%
\pgfpathmoveto{\pgfqpoint{8.174049in}{2.214929in}}%
\pgfpathcurveto{\pgfqpoint{8.185099in}{2.214929in}}{\pgfqpoint{8.195698in}{2.219320in}}{\pgfqpoint{8.203512in}{2.227133in}}%
\pgfpathcurveto{\pgfqpoint{8.211325in}{2.234947in}}{\pgfqpoint{8.215715in}{2.245546in}}{\pgfqpoint{8.215715in}{2.256596in}}%
\pgfpathcurveto{\pgfqpoint{8.215715in}{2.267646in}}{\pgfqpoint{8.211325in}{2.278245in}}{\pgfqpoint{8.203512in}{2.286059in}}%
\pgfpathcurveto{\pgfqpoint{8.195698in}{2.293872in}}{\pgfqpoint{8.185099in}{2.298263in}}{\pgfqpoint{8.174049in}{2.298263in}}%
\pgfpathcurveto{\pgfqpoint{8.162999in}{2.298263in}}{\pgfqpoint{8.152400in}{2.293872in}}{\pgfqpoint{8.144586in}{2.286059in}}%
\pgfpathcurveto{\pgfqpoint{8.136772in}{2.278245in}}{\pgfqpoint{8.132382in}{2.267646in}}{\pgfqpoint{8.132382in}{2.256596in}}%
\pgfpathcurveto{\pgfqpoint{8.132382in}{2.245546in}}{\pgfqpoint{8.136772in}{2.234947in}}{\pgfqpoint{8.144586in}{2.227133in}}%
\pgfpathcurveto{\pgfqpoint{8.152400in}{2.219320in}}{\pgfqpoint{8.162999in}{2.214929in}}{\pgfqpoint{8.174049in}{2.214929in}}%
\pgfpathlineto{\pgfqpoint{8.174049in}{2.214929in}}%
\pgfpathclose%
\pgfusepath{stroke}%
\end{pgfscope}%
\begin{pgfscope}%
\pgfpathrectangle{\pgfqpoint{7.394209in}{0.375000in}}{\pgfqpoint{6.356833in}{5.175000in}}%
\pgfusepath{clip}%
\pgfsetbuttcap%
\pgfsetroundjoin%
\pgfsetlinewidth{1.003750pt}%
\definecolor{currentstroke}{rgb}{0.827451,0.827451,0.827451}%
\pgfsetstrokecolor{currentstroke}%
\pgfsetdash{}{0pt}%
\pgfpathmoveto{\pgfqpoint{12.794324in}{5.484131in}}%
\pgfpathcurveto{\pgfqpoint{12.805374in}{5.484131in}}{\pgfqpoint{12.815973in}{5.488521in}}{\pgfqpoint{12.823786in}{5.496335in}}%
\pgfpathcurveto{\pgfqpoint{12.831600in}{5.504148in}}{\pgfqpoint{12.835990in}{5.514747in}}{\pgfqpoint{12.835990in}{5.525797in}}%
\pgfpathcurveto{\pgfqpoint{12.835990in}{5.536847in}}{\pgfqpoint{12.831600in}{5.547446in}}{\pgfqpoint{12.823786in}{5.555260in}}%
\pgfpathcurveto{\pgfqpoint{12.815973in}{5.563074in}}{\pgfqpoint{12.805374in}{5.567464in}}{\pgfqpoint{12.794324in}{5.567464in}}%
\pgfpathcurveto{\pgfqpoint{12.783274in}{5.567464in}}{\pgfqpoint{12.772674in}{5.563074in}}{\pgfqpoint{12.764861in}{5.555260in}}%
\pgfpathcurveto{\pgfqpoint{12.757047in}{5.547446in}}{\pgfqpoint{12.752657in}{5.536847in}}{\pgfqpoint{12.752657in}{5.525797in}}%
\pgfpathcurveto{\pgfqpoint{12.752657in}{5.514747in}}{\pgfqpoint{12.757047in}{5.504148in}}{\pgfqpoint{12.764861in}{5.496335in}}%
\pgfpathcurveto{\pgfqpoint{12.772674in}{5.488521in}}{\pgfqpoint{12.783274in}{5.484131in}}{\pgfqpoint{12.794324in}{5.484131in}}%
\pgfpathlineto{\pgfqpoint{12.794324in}{5.484131in}}%
\pgfpathclose%
\pgfusepath{stroke}%
\end{pgfscope}%
\begin{pgfscope}%
\pgfpathrectangle{\pgfqpoint{7.394209in}{0.375000in}}{\pgfqpoint{6.356833in}{5.175000in}}%
\pgfusepath{clip}%
\pgfsetbuttcap%
\pgfsetroundjoin%
\pgfsetlinewidth{1.003750pt}%
\definecolor{currentstroke}{rgb}{0.827451,0.827451,0.827451}%
\pgfsetstrokecolor{currentstroke}%
\pgfsetdash{}{0pt}%
\pgfpathmoveto{\pgfqpoint{10.139276in}{2.854355in}}%
\pgfpathcurveto{\pgfqpoint{10.150326in}{2.854355in}}{\pgfqpoint{10.160925in}{2.858745in}}{\pgfqpoint{10.168738in}{2.866559in}}%
\pgfpathcurveto{\pgfqpoint{10.176552in}{2.874372in}}{\pgfqpoint{10.180942in}{2.884971in}}{\pgfqpoint{10.180942in}{2.896021in}}%
\pgfpathcurveto{\pgfqpoint{10.180942in}{2.907072in}}{\pgfqpoint{10.176552in}{2.917671in}}{\pgfqpoint{10.168738in}{2.925484in}}%
\pgfpathcurveto{\pgfqpoint{10.160925in}{2.933298in}}{\pgfqpoint{10.150326in}{2.937688in}}{\pgfqpoint{10.139276in}{2.937688in}}%
\pgfpathcurveto{\pgfqpoint{10.128226in}{2.937688in}}{\pgfqpoint{10.117627in}{2.933298in}}{\pgfqpoint{10.109813in}{2.925484in}}%
\pgfpathcurveto{\pgfqpoint{10.101999in}{2.917671in}}{\pgfqpoint{10.097609in}{2.907072in}}{\pgfqpoint{10.097609in}{2.896021in}}%
\pgfpathcurveto{\pgfqpoint{10.097609in}{2.884971in}}{\pgfqpoint{10.101999in}{2.874372in}}{\pgfqpoint{10.109813in}{2.866559in}}%
\pgfpathcurveto{\pgfqpoint{10.117627in}{2.858745in}}{\pgfqpoint{10.128226in}{2.854355in}}{\pgfqpoint{10.139276in}{2.854355in}}%
\pgfpathlineto{\pgfqpoint{10.139276in}{2.854355in}}%
\pgfpathclose%
\pgfusepath{stroke}%
\end{pgfscope}%
\begin{pgfscope}%
\pgfpathrectangle{\pgfqpoint{7.394209in}{0.375000in}}{\pgfqpoint{6.356833in}{5.175000in}}%
\pgfusepath{clip}%
\pgfsetbuttcap%
\pgfsetroundjoin%
\pgfsetlinewidth{1.003750pt}%
\definecolor{currentstroke}{rgb}{0.827451,0.827451,0.827451}%
\pgfsetstrokecolor{currentstroke}%
\pgfsetdash{}{0pt}%
\pgfpathmoveto{\pgfqpoint{11.692569in}{5.184597in}}%
\pgfpathcurveto{\pgfqpoint{11.703619in}{5.184597in}}{\pgfqpoint{11.714218in}{5.188987in}}{\pgfqpoint{11.722032in}{5.196801in}}%
\pgfpathcurveto{\pgfqpoint{11.729845in}{5.204614in}}{\pgfqpoint{11.734236in}{5.215213in}}{\pgfqpoint{11.734236in}{5.226263in}}%
\pgfpathcurveto{\pgfqpoint{11.734236in}{5.237314in}}{\pgfqpoint{11.729845in}{5.247913in}}{\pgfqpoint{11.722032in}{5.255726in}}%
\pgfpathcurveto{\pgfqpoint{11.714218in}{5.263540in}}{\pgfqpoint{11.703619in}{5.267930in}}{\pgfqpoint{11.692569in}{5.267930in}}%
\pgfpathcurveto{\pgfqpoint{11.681519in}{5.267930in}}{\pgfqpoint{11.670920in}{5.263540in}}{\pgfqpoint{11.663106in}{5.255726in}}%
\pgfpathcurveto{\pgfqpoint{11.655292in}{5.247913in}}{\pgfqpoint{11.650902in}{5.237314in}}{\pgfqpoint{11.650902in}{5.226263in}}%
\pgfpathcurveto{\pgfqpoint{11.650902in}{5.215213in}}{\pgfqpoint{11.655292in}{5.204614in}}{\pgfqpoint{11.663106in}{5.196801in}}%
\pgfpathcurveto{\pgfqpoint{11.670920in}{5.188987in}}{\pgfqpoint{11.681519in}{5.184597in}}{\pgfqpoint{11.692569in}{5.184597in}}%
\pgfpathlineto{\pgfqpoint{11.692569in}{5.184597in}}%
\pgfpathclose%
\pgfusepath{stroke}%
\end{pgfscope}%
\begin{pgfscope}%
\pgfpathrectangle{\pgfqpoint{7.394209in}{0.375000in}}{\pgfqpoint{6.356833in}{5.175000in}}%
\pgfusepath{clip}%
\pgfsetbuttcap%
\pgfsetroundjoin%
\pgfsetlinewidth{1.003750pt}%
\definecolor{currentstroke}{rgb}{0.827451,0.827451,0.827451}%
\pgfsetstrokecolor{currentstroke}%
\pgfsetdash{}{0pt}%
\pgfpathmoveto{\pgfqpoint{9.662846in}{4.205352in}}%
\pgfpathcurveto{\pgfqpoint{9.673896in}{4.205352in}}{\pgfqpoint{9.684495in}{4.209742in}}{\pgfqpoint{9.692309in}{4.217555in}}%
\pgfpathcurveto{\pgfqpoint{9.700123in}{4.225369in}}{\pgfqpoint{9.704513in}{4.235968in}}{\pgfqpoint{9.704513in}{4.247018in}}%
\pgfpathcurveto{\pgfqpoint{9.704513in}{4.258068in}}{\pgfqpoint{9.700123in}{4.268667in}}{\pgfqpoint{9.692309in}{4.276481in}}%
\pgfpathcurveto{\pgfqpoint{9.684495in}{4.284295in}}{\pgfqpoint{9.673896in}{4.288685in}}{\pgfqpoint{9.662846in}{4.288685in}}%
\pgfpathcurveto{\pgfqpoint{9.651796in}{4.288685in}}{\pgfqpoint{9.641197in}{4.284295in}}{\pgfqpoint{9.633383in}{4.276481in}}%
\pgfpathcurveto{\pgfqpoint{9.625570in}{4.268667in}}{\pgfqpoint{9.621180in}{4.258068in}}{\pgfqpoint{9.621180in}{4.247018in}}%
\pgfpathcurveto{\pgfqpoint{9.621180in}{4.235968in}}{\pgfqpoint{9.625570in}{4.225369in}}{\pgfqpoint{9.633383in}{4.217555in}}%
\pgfpathcurveto{\pgfqpoint{9.641197in}{4.209742in}}{\pgfqpoint{9.651796in}{4.205352in}}{\pgfqpoint{9.662846in}{4.205352in}}%
\pgfpathlineto{\pgfqpoint{9.662846in}{4.205352in}}%
\pgfpathclose%
\pgfusepath{stroke}%
\end{pgfscope}%
\begin{pgfscope}%
\pgfpathrectangle{\pgfqpoint{7.394209in}{0.375000in}}{\pgfqpoint{6.356833in}{5.175000in}}%
\pgfusepath{clip}%
\pgfsetbuttcap%
\pgfsetroundjoin%
\pgfsetlinewidth{1.003750pt}%
\definecolor{currentstroke}{rgb}{0.827451,0.827451,0.827451}%
\pgfsetstrokecolor{currentstroke}%
\pgfsetdash{}{0pt}%
\pgfpathmoveto{\pgfqpoint{11.818076in}{5.075949in}}%
\pgfpathcurveto{\pgfqpoint{11.829126in}{5.075949in}}{\pgfqpoint{11.839725in}{5.080339in}}{\pgfqpoint{11.847539in}{5.088153in}}%
\pgfpathcurveto{\pgfqpoint{11.855353in}{5.095967in}}{\pgfqpoint{11.859743in}{5.106566in}}{\pgfqpoint{11.859743in}{5.117616in}}%
\pgfpathcurveto{\pgfqpoint{11.859743in}{5.128666in}}{\pgfqpoint{11.855353in}{5.139265in}}{\pgfqpoint{11.847539in}{5.147079in}}%
\pgfpathcurveto{\pgfqpoint{11.839725in}{5.154892in}}{\pgfqpoint{11.829126in}{5.159283in}}{\pgfqpoint{11.818076in}{5.159283in}}%
\pgfpathcurveto{\pgfqpoint{11.807026in}{5.159283in}}{\pgfqpoint{11.796427in}{5.154892in}}{\pgfqpoint{11.788613in}{5.147079in}}%
\pgfpathcurveto{\pgfqpoint{11.780800in}{5.139265in}}{\pgfqpoint{11.776409in}{5.128666in}}{\pgfqpoint{11.776409in}{5.117616in}}%
\pgfpathcurveto{\pgfqpoint{11.776409in}{5.106566in}}{\pgfqpoint{11.780800in}{5.095967in}}{\pgfqpoint{11.788613in}{5.088153in}}%
\pgfpathcurveto{\pgfqpoint{11.796427in}{5.080339in}}{\pgfqpoint{11.807026in}{5.075949in}}{\pgfqpoint{11.818076in}{5.075949in}}%
\pgfpathlineto{\pgfqpoint{11.818076in}{5.075949in}}%
\pgfpathclose%
\pgfusepath{stroke}%
\end{pgfscope}%
\begin{pgfscope}%
\pgfpathrectangle{\pgfqpoint{7.394209in}{0.375000in}}{\pgfqpoint{6.356833in}{5.175000in}}%
\pgfusepath{clip}%
\pgfsetbuttcap%
\pgfsetroundjoin%
\pgfsetlinewidth{1.003750pt}%
\definecolor{currentstroke}{rgb}{0.827451,0.827451,0.827451}%
\pgfsetstrokecolor{currentstroke}%
\pgfsetdash{}{0pt}%
\pgfpathmoveto{\pgfqpoint{9.337062in}{2.161463in}}%
\pgfpathcurveto{\pgfqpoint{9.348112in}{2.161463in}}{\pgfqpoint{9.358712in}{2.165854in}}{\pgfqpoint{9.366525in}{2.173667in}}%
\pgfpathcurveto{\pgfqpoint{9.374339in}{2.181481in}}{\pgfqpoint{9.378729in}{2.192080in}}{\pgfqpoint{9.378729in}{2.203130in}}%
\pgfpathcurveto{\pgfqpoint{9.378729in}{2.214180in}}{\pgfqpoint{9.374339in}{2.224779in}}{\pgfqpoint{9.366525in}{2.232593in}}%
\pgfpathcurveto{\pgfqpoint{9.358712in}{2.240406in}}{\pgfqpoint{9.348112in}{2.244797in}}{\pgfqpoint{9.337062in}{2.244797in}}%
\pgfpathcurveto{\pgfqpoint{9.326012in}{2.244797in}}{\pgfqpoint{9.315413in}{2.240406in}}{\pgfqpoint{9.307600in}{2.232593in}}%
\pgfpathcurveto{\pgfqpoint{9.299786in}{2.224779in}}{\pgfqpoint{9.295396in}{2.214180in}}{\pgfqpoint{9.295396in}{2.203130in}}%
\pgfpathcurveto{\pgfqpoint{9.295396in}{2.192080in}}{\pgfqpoint{9.299786in}{2.181481in}}{\pgfqpoint{9.307600in}{2.173667in}}%
\pgfpathcurveto{\pgfqpoint{9.315413in}{2.165854in}}{\pgfqpoint{9.326012in}{2.161463in}}{\pgfqpoint{9.337062in}{2.161463in}}%
\pgfpathlineto{\pgfqpoint{9.337062in}{2.161463in}}%
\pgfpathclose%
\pgfusepath{stroke}%
\end{pgfscope}%
\begin{pgfscope}%
\pgfpathrectangle{\pgfqpoint{7.394209in}{0.375000in}}{\pgfqpoint{6.356833in}{5.175000in}}%
\pgfusepath{clip}%
\pgfsetbuttcap%
\pgfsetroundjoin%
\pgfsetlinewidth{1.003750pt}%
\definecolor{currentstroke}{rgb}{0.827451,0.827451,0.827451}%
\pgfsetstrokecolor{currentstroke}%
\pgfsetdash{}{0pt}%
\pgfpathmoveto{\pgfqpoint{8.088134in}{0.801824in}}%
\pgfpathcurveto{\pgfqpoint{8.099184in}{0.801824in}}{\pgfqpoint{8.109783in}{0.806214in}}{\pgfqpoint{8.117597in}{0.814028in}}%
\pgfpathcurveto{\pgfqpoint{8.125410in}{0.821842in}}{\pgfqpoint{8.129800in}{0.832441in}}{\pgfqpoint{8.129800in}{0.843491in}}%
\pgfpathcurveto{\pgfqpoint{8.129800in}{0.854541in}}{\pgfqpoint{8.125410in}{0.865140in}}{\pgfqpoint{8.117597in}{0.872954in}}%
\pgfpathcurveto{\pgfqpoint{8.109783in}{0.880767in}}{\pgfqpoint{8.099184in}{0.885157in}}{\pgfqpoint{8.088134in}{0.885157in}}%
\pgfpathcurveto{\pgfqpoint{8.077084in}{0.885157in}}{\pgfqpoint{8.066485in}{0.880767in}}{\pgfqpoint{8.058671in}{0.872954in}}%
\pgfpathcurveto{\pgfqpoint{8.050857in}{0.865140in}}{\pgfqpoint{8.046467in}{0.854541in}}{\pgfqpoint{8.046467in}{0.843491in}}%
\pgfpathcurveto{\pgfqpoint{8.046467in}{0.832441in}}{\pgfqpoint{8.050857in}{0.821842in}}{\pgfqpoint{8.058671in}{0.814028in}}%
\pgfpathcurveto{\pgfqpoint{8.066485in}{0.806214in}}{\pgfqpoint{8.077084in}{0.801824in}}{\pgfqpoint{8.088134in}{0.801824in}}%
\pgfpathlineto{\pgfqpoint{8.088134in}{0.801824in}}%
\pgfpathclose%
\pgfusepath{stroke}%
\end{pgfscope}%
\begin{pgfscope}%
\pgfpathrectangle{\pgfqpoint{7.394209in}{0.375000in}}{\pgfqpoint{6.356833in}{5.175000in}}%
\pgfusepath{clip}%
\pgfsetbuttcap%
\pgfsetroundjoin%
\pgfsetlinewidth{1.003750pt}%
\definecolor{currentstroke}{rgb}{0.827451,0.827451,0.827451}%
\pgfsetstrokecolor{currentstroke}%
\pgfsetdash{}{0pt}%
\pgfpathmoveto{\pgfqpoint{12.864303in}{5.420712in}}%
\pgfpathcurveto{\pgfqpoint{12.875353in}{5.420712in}}{\pgfqpoint{12.885952in}{5.425102in}}{\pgfqpoint{12.893766in}{5.432916in}}%
\pgfpathcurveto{\pgfqpoint{12.901579in}{5.440729in}}{\pgfqpoint{12.905970in}{5.451328in}}{\pgfqpoint{12.905970in}{5.462378in}}%
\pgfpathcurveto{\pgfqpoint{12.905970in}{5.473428in}}{\pgfqpoint{12.901579in}{5.484027in}}{\pgfqpoint{12.893766in}{5.491841in}}%
\pgfpathcurveto{\pgfqpoint{12.885952in}{5.499655in}}{\pgfqpoint{12.875353in}{5.504045in}}{\pgfqpoint{12.864303in}{5.504045in}}%
\pgfpathcurveto{\pgfqpoint{12.853253in}{5.504045in}}{\pgfqpoint{12.842654in}{5.499655in}}{\pgfqpoint{12.834840in}{5.491841in}}%
\pgfpathcurveto{\pgfqpoint{12.827027in}{5.484027in}}{\pgfqpoint{12.822636in}{5.473428in}}{\pgfqpoint{12.822636in}{5.462378in}}%
\pgfpathcurveto{\pgfqpoint{12.822636in}{5.451328in}}{\pgfqpoint{12.827027in}{5.440729in}}{\pgfqpoint{12.834840in}{5.432916in}}%
\pgfpathcurveto{\pgfqpoint{12.842654in}{5.425102in}}{\pgfqpoint{12.853253in}{5.420712in}}{\pgfqpoint{12.864303in}{5.420712in}}%
\pgfpathlineto{\pgfqpoint{12.864303in}{5.420712in}}%
\pgfpathclose%
\pgfusepath{stroke}%
\end{pgfscope}%
\begin{pgfscope}%
\pgfpathrectangle{\pgfqpoint{7.394209in}{0.375000in}}{\pgfqpoint{6.356833in}{5.175000in}}%
\pgfusepath{clip}%
\pgfsetbuttcap%
\pgfsetroundjoin%
\pgfsetlinewidth{1.003750pt}%
\definecolor{currentstroke}{rgb}{0.827451,0.827451,0.827451}%
\pgfsetstrokecolor{currentstroke}%
\pgfsetdash{}{0pt}%
\pgfpathmoveto{\pgfqpoint{9.205969in}{3.642569in}}%
\pgfpathcurveto{\pgfqpoint{9.217020in}{3.642569in}}{\pgfqpoint{9.227619in}{3.646959in}}{\pgfqpoint{9.235432in}{3.654772in}}%
\pgfpathcurveto{\pgfqpoint{9.243246in}{3.662586in}}{\pgfqpoint{9.247636in}{3.673185in}}{\pgfqpoint{9.247636in}{3.684235in}}%
\pgfpathcurveto{\pgfqpoint{9.247636in}{3.695285in}}{\pgfqpoint{9.243246in}{3.705884in}}{\pgfqpoint{9.235432in}{3.713698in}}%
\pgfpathcurveto{\pgfqpoint{9.227619in}{3.721512in}}{\pgfqpoint{9.217020in}{3.725902in}}{\pgfqpoint{9.205969in}{3.725902in}}%
\pgfpathcurveto{\pgfqpoint{9.194919in}{3.725902in}}{\pgfqpoint{9.184320in}{3.721512in}}{\pgfqpoint{9.176507in}{3.713698in}}%
\pgfpathcurveto{\pgfqpoint{9.168693in}{3.705884in}}{\pgfqpoint{9.164303in}{3.695285in}}{\pgfqpoint{9.164303in}{3.684235in}}%
\pgfpathcurveto{\pgfqpoint{9.164303in}{3.673185in}}{\pgfqpoint{9.168693in}{3.662586in}}{\pgfqpoint{9.176507in}{3.654772in}}%
\pgfpathcurveto{\pgfqpoint{9.184320in}{3.646959in}}{\pgfqpoint{9.194919in}{3.642569in}}{\pgfqpoint{9.205969in}{3.642569in}}%
\pgfpathlineto{\pgfqpoint{9.205969in}{3.642569in}}%
\pgfpathclose%
\pgfusepath{stroke}%
\end{pgfscope}%
\begin{pgfscope}%
\pgfpathrectangle{\pgfqpoint{7.394209in}{0.375000in}}{\pgfqpoint{6.356833in}{5.175000in}}%
\pgfusepath{clip}%
\pgfsetbuttcap%
\pgfsetroundjoin%
\pgfsetlinewidth{1.003750pt}%
\definecolor{currentstroke}{rgb}{0.827451,0.827451,0.827451}%
\pgfsetstrokecolor{currentstroke}%
\pgfsetdash{}{0pt}%
\pgfpathmoveto{\pgfqpoint{10.960140in}{5.467370in}}%
\pgfpathcurveto{\pgfqpoint{10.971190in}{5.467370in}}{\pgfqpoint{10.981789in}{5.471761in}}{\pgfqpoint{10.989602in}{5.479574in}}%
\pgfpathcurveto{\pgfqpoint{10.997416in}{5.487388in}}{\pgfqpoint{11.001806in}{5.497987in}}{\pgfqpoint{11.001806in}{5.509037in}}%
\pgfpathcurveto{\pgfqpoint{11.001806in}{5.520087in}}{\pgfqpoint{10.997416in}{5.530686in}}{\pgfqpoint{10.989602in}{5.538500in}}%
\pgfpathcurveto{\pgfqpoint{10.981789in}{5.546314in}}{\pgfqpoint{10.971190in}{5.550704in}}{\pgfqpoint{10.960140in}{5.550704in}}%
\pgfpathcurveto{\pgfqpoint{10.949090in}{5.550704in}}{\pgfqpoint{10.938490in}{5.546314in}}{\pgfqpoint{10.930677in}{5.538500in}}%
\pgfpathcurveto{\pgfqpoint{10.922863in}{5.530686in}}{\pgfqpoint{10.918473in}{5.520087in}}{\pgfqpoint{10.918473in}{5.509037in}}%
\pgfpathcurveto{\pgfqpoint{10.918473in}{5.497987in}}{\pgfqpoint{10.922863in}{5.487388in}}{\pgfqpoint{10.930677in}{5.479574in}}%
\pgfpathcurveto{\pgfqpoint{10.938490in}{5.471761in}}{\pgfqpoint{10.949090in}{5.467370in}}{\pgfqpoint{10.960140in}{5.467370in}}%
\pgfpathlineto{\pgfqpoint{10.960140in}{5.467370in}}%
\pgfpathclose%
\pgfusepath{stroke}%
\end{pgfscope}%
\begin{pgfscope}%
\pgfpathrectangle{\pgfqpoint{7.394209in}{0.375000in}}{\pgfqpoint{6.356833in}{5.175000in}}%
\pgfusepath{clip}%
\pgfsetbuttcap%
\pgfsetroundjoin%
\pgfsetlinewidth{1.003750pt}%
\definecolor{currentstroke}{rgb}{0.827451,0.827451,0.827451}%
\pgfsetstrokecolor{currentstroke}%
\pgfsetdash{}{0pt}%
\pgfpathmoveto{\pgfqpoint{10.340293in}{3.760205in}}%
\pgfpathcurveto{\pgfqpoint{10.351343in}{3.760205in}}{\pgfqpoint{10.361943in}{3.764595in}}{\pgfqpoint{10.369756in}{3.772409in}}%
\pgfpathcurveto{\pgfqpoint{10.377570in}{3.780223in}}{\pgfqpoint{10.381960in}{3.790822in}}{\pgfqpoint{10.381960in}{3.801872in}}%
\pgfpathcurveto{\pgfqpoint{10.381960in}{3.812922in}}{\pgfqpoint{10.377570in}{3.823521in}}{\pgfqpoint{10.369756in}{3.831335in}}%
\pgfpathcurveto{\pgfqpoint{10.361943in}{3.839148in}}{\pgfqpoint{10.351343in}{3.843538in}}{\pgfqpoint{10.340293in}{3.843538in}}%
\pgfpathcurveto{\pgfqpoint{10.329243in}{3.843538in}}{\pgfqpoint{10.318644in}{3.839148in}}{\pgfqpoint{10.310831in}{3.831335in}}%
\pgfpathcurveto{\pgfqpoint{10.303017in}{3.823521in}}{\pgfqpoint{10.298627in}{3.812922in}}{\pgfqpoint{10.298627in}{3.801872in}}%
\pgfpathcurveto{\pgfqpoint{10.298627in}{3.790822in}}{\pgfqpoint{10.303017in}{3.780223in}}{\pgfqpoint{10.310831in}{3.772409in}}%
\pgfpathcurveto{\pgfqpoint{10.318644in}{3.764595in}}{\pgfqpoint{10.329243in}{3.760205in}}{\pgfqpoint{10.340293in}{3.760205in}}%
\pgfpathlineto{\pgfqpoint{10.340293in}{3.760205in}}%
\pgfpathclose%
\pgfusepath{stroke}%
\end{pgfscope}%
\begin{pgfscope}%
\pgfpathrectangle{\pgfqpoint{7.394209in}{0.375000in}}{\pgfqpoint{6.356833in}{5.175000in}}%
\pgfusepath{clip}%
\pgfsetbuttcap%
\pgfsetroundjoin%
\pgfsetlinewidth{1.003750pt}%
\definecolor{currentstroke}{rgb}{0.827451,0.827451,0.827451}%
\pgfsetstrokecolor{currentstroke}%
\pgfsetdash{}{0pt}%
\pgfpathmoveto{\pgfqpoint{8.189920in}{0.801824in}}%
\pgfpathcurveto{\pgfqpoint{8.200970in}{0.801824in}}{\pgfqpoint{8.211569in}{0.806214in}}{\pgfqpoint{8.219383in}{0.814028in}}%
\pgfpathcurveto{\pgfqpoint{8.227197in}{0.821842in}}{\pgfqpoint{8.231587in}{0.832441in}}{\pgfqpoint{8.231587in}{0.843491in}}%
\pgfpathcurveto{\pgfqpoint{8.231587in}{0.854541in}}{\pgfqpoint{8.227197in}{0.865140in}}{\pgfqpoint{8.219383in}{0.872954in}}%
\pgfpathcurveto{\pgfqpoint{8.211569in}{0.880767in}}{\pgfqpoint{8.200970in}{0.885157in}}{\pgfqpoint{8.189920in}{0.885157in}}%
\pgfpathcurveto{\pgfqpoint{8.178870in}{0.885157in}}{\pgfqpoint{8.168271in}{0.880767in}}{\pgfqpoint{8.160457in}{0.872954in}}%
\pgfpathcurveto{\pgfqpoint{8.152644in}{0.865140in}}{\pgfqpoint{8.148253in}{0.854541in}}{\pgfqpoint{8.148253in}{0.843491in}}%
\pgfpathcurveto{\pgfqpoint{8.148253in}{0.832441in}}{\pgfqpoint{8.152644in}{0.821842in}}{\pgfqpoint{8.160457in}{0.814028in}}%
\pgfpathcurveto{\pgfqpoint{8.168271in}{0.806214in}}{\pgfqpoint{8.178870in}{0.801824in}}{\pgfqpoint{8.189920in}{0.801824in}}%
\pgfpathlineto{\pgfqpoint{8.189920in}{0.801824in}}%
\pgfpathclose%
\pgfusepath{stroke}%
\end{pgfscope}%
\begin{pgfscope}%
\pgfpathrectangle{\pgfqpoint{7.394209in}{0.375000in}}{\pgfqpoint{6.356833in}{5.175000in}}%
\pgfusepath{clip}%
\pgfsetbuttcap%
\pgfsetroundjoin%
\pgfsetlinewidth{1.003750pt}%
\definecolor{currentstroke}{rgb}{0.827451,0.827451,0.827451}%
\pgfsetstrokecolor{currentstroke}%
\pgfsetdash{}{0pt}%
\pgfpathmoveto{\pgfqpoint{9.980225in}{2.854355in}}%
\pgfpathcurveto{\pgfqpoint{9.991275in}{2.854355in}}{\pgfqpoint{10.001874in}{2.858745in}}{\pgfqpoint{10.009688in}{2.866559in}}%
\pgfpathcurveto{\pgfqpoint{10.017502in}{2.874372in}}{\pgfqpoint{10.021892in}{2.884971in}}{\pgfqpoint{10.021892in}{2.896021in}}%
\pgfpathcurveto{\pgfqpoint{10.021892in}{2.907072in}}{\pgfqpoint{10.017502in}{2.917671in}}{\pgfqpoint{10.009688in}{2.925484in}}%
\pgfpathcurveto{\pgfqpoint{10.001874in}{2.933298in}}{\pgfqpoint{9.991275in}{2.937688in}}{\pgfqpoint{9.980225in}{2.937688in}}%
\pgfpathcurveto{\pgfqpoint{9.969175in}{2.937688in}}{\pgfqpoint{9.958576in}{2.933298in}}{\pgfqpoint{9.950762in}{2.925484in}}%
\pgfpathcurveto{\pgfqpoint{9.942949in}{2.917671in}}{\pgfqpoint{9.938559in}{2.907072in}}{\pgfqpoint{9.938559in}{2.896021in}}%
\pgfpathcurveto{\pgfqpoint{9.938559in}{2.884971in}}{\pgfqpoint{9.942949in}{2.874372in}}{\pgfqpoint{9.950762in}{2.866559in}}%
\pgfpathcurveto{\pgfqpoint{9.958576in}{2.858745in}}{\pgfqpoint{9.969175in}{2.854355in}}{\pgfqpoint{9.980225in}{2.854355in}}%
\pgfpathlineto{\pgfqpoint{9.980225in}{2.854355in}}%
\pgfpathclose%
\pgfusepath{stroke}%
\end{pgfscope}%
\begin{pgfscope}%
\pgfpathrectangle{\pgfqpoint{7.394209in}{0.375000in}}{\pgfqpoint{6.356833in}{5.175000in}}%
\pgfusepath{clip}%
\pgfsetbuttcap%
\pgfsetroundjoin%
\pgfsetlinewidth{1.003750pt}%
\definecolor{currentstroke}{rgb}{0.827451,0.827451,0.827451}%
\pgfsetstrokecolor{currentstroke}%
\pgfsetdash{}{0pt}%
\pgfpathmoveto{\pgfqpoint{8.602590in}{1.643406in}}%
\pgfpathcurveto{\pgfqpoint{8.613640in}{1.643406in}}{\pgfqpoint{8.624239in}{1.647797in}}{\pgfqpoint{8.632053in}{1.655610in}}%
\pgfpathcurveto{\pgfqpoint{8.639866in}{1.663424in}}{\pgfqpoint{8.644257in}{1.674023in}}{\pgfqpoint{8.644257in}{1.685073in}}%
\pgfpathcurveto{\pgfqpoint{8.644257in}{1.696123in}}{\pgfqpoint{8.639866in}{1.706722in}}{\pgfqpoint{8.632053in}{1.714536in}}%
\pgfpathcurveto{\pgfqpoint{8.624239in}{1.722350in}}{\pgfqpoint{8.613640in}{1.726740in}}{\pgfqpoint{8.602590in}{1.726740in}}%
\pgfpathcurveto{\pgfqpoint{8.591540in}{1.726740in}}{\pgfqpoint{8.580941in}{1.722350in}}{\pgfqpoint{8.573127in}{1.714536in}}%
\pgfpathcurveto{\pgfqpoint{8.565314in}{1.706722in}}{\pgfqpoint{8.560923in}{1.696123in}}{\pgfqpoint{8.560923in}{1.685073in}}%
\pgfpathcurveto{\pgfqpoint{8.560923in}{1.674023in}}{\pgfqpoint{8.565314in}{1.663424in}}{\pgfqpoint{8.573127in}{1.655610in}}%
\pgfpathcurveto{\pgfqpoint{8.580941in}{1.647797in}}{\pgfqpoint{8.591540in}{1.643406in}}{\pgfqpoint{8.602590in}{1.643406in}}%
\pgfpathlineto{\pgfqpoint{8.602590in}{1.643406in}}%
\pgfpathclose%
\pgfusepath{stroke}%
\end{pgfscope}%
\begin{pgfscope}%
\pgfpathrectangle{\pgfqpoint{7.394209in}{0.375000in}}{\pgfqpoint{6.356833in}{5.175000in}}%
\pgfusepath{clip}%
\pgfsetbuttcap%
\pgfsetroundjoin%
\pgfsetlinewidth{1.003750pt}%
\definecolor{currentstroke}{rgb}{0.827451,0.827451,0.827451}%
\pgfsetstrokecolor{currentstroke}%
\pgfsetdash{}{0pt}%
\pgfpathmoveto{\pgfqpoint{11.360669in}{5.267878in}}%
\pgfpathcurveto{\pgfqpoint{11.371720in}{5.267878in}}{\pgfqpoint{11.382319in}{5.272269in}}{\pgfqpoint{11.390132in}{5.280082in}}%
\pgfpathcurveto{\pgfqpoint{11.397946in}{5.287896in}}{\pgfqpoint{11.402336in}{5.298495in}}{\pgfqpoint{11.402336in}{5.309545in}}%
\pgfpathcurveto{\pgfqpoint{11.402336in}{5.320595in}}{\pgfqpoint{11.397946in}{5.331194in}}{\pgfqpoint{11.390132in}{5.339008in}}%
\pgfpathcurveto{\pgfqpoint{11.382319in}{5.346821in}}{\pgfqpoint{11.371720in}{5.351212in}}{\pgfqpoint{11.360669in}{5.351212in}}%
\pgfpathcurveto{\pgfqpoint{11.349619in}{5.351212in}}{\pgfqpoint{11.339020in}{5.346821in}}{\pgfqpoint{11.331207in}{5.339008in}}%
\pgfpathcurveto{\pgfqpoint{11.323393in}{5.331194in}}{\pgfqpoint{11.319003in}{5.320595in}}{\pgfqpoint{11.319003in}{5.309545in}}%
\pgfpathcurveto{\pgfqpoint{11.319003in}{5.298495in}}{\pgfqpoint{11.323393in}{5.287896in}}{\pgfqpoint{11.331207in}{5.280082in}}%
\pgfpathcurveto{\pgfqpoint{11.339020in}{5.272269in}}{\pgfqpoint{11.349619in}{5.267878in}}{\pgfqpoint{11.360669in}{5.267878in}}%
\pgfpathlineto{\pgfqpoint{11.360669in}{5.267878in}}%
\pgfpathclose%
\pgfusepath{stroke}%
\end{pgfscope}%
\begin{pgfscope}%
\pgfpathrectangle{\pgfqpoint{7.394209in}{0.375000in}}{\pgfqpoint{6.356833in}{5.175000in}}%
\pgfusepath{clip}%
\pgfsetbuttcap%
\pgfsetroundjoin%
\pgfsetlinewidth{1.003750pt}%
\definecolor{currentstroke}{rgb}{0.827451,0.827451,0.827451}%
\pgfsetstrokecolor{currentstroke}%
\pgfsetdash{}{0pt}%
\pgfpathmoveto{\pgfqpoint{10.665262in}{4.396398in}}%
\pgfpathcurveto{\pgfqpoint{10.676312in}{4.396398in}}{\pgfqpoint{10.686911in}{4.400788in}}{\pgfqpoint{10.694725in}{4.408602in}}%
\pgfpathcurveto{\pgfqpoint{10.702538in}{4.416416in}}{\pgfqpoint{10.706929in}{4.427015in}}{\pgfqpoint{10.706929in}{4.438065in}}%
\pgfpathcurveto{\pgfqpoint{10.706929in}{4.449115in}}{\pgfqpoint{10.702538in}{4.459714in}}{\pgfqpoint{10.694725in}{4.467527in}}%
\pgfpathcurveto{\pgfqpoint{10.686911in}{4.475341in}}{\pgfqpoint{10.676312in}{4.479731in}}{\pgfqpoint{10.665262in}{4.479731in}}%
\pgfpathcurveto{\pgfqpoint{10.654212in}{4.479731in}}{\pgfqpoint{10.643613in}{4.475341in}}{\pgfqpoint{10.635799in}{4.467527in}}%
\pgfpathcurveto{\pgfqpoint{10.627986in}{4.459714in}}{\pgfqpoint{10.623595in}{4.449115in}}{\pgfqpoint{10.623595in}{4.438065in}}%
\pgfpathcurveto{\pgfqpoint{10.623595in}{4.427015in}}{\pgfqpoint{10.627986in}{4.416416in}}{\pgfqpoint{10.635799in}{4.408602in}}%
\pgfpathcurveto{\pgfqpoint{10.643613in}{4.400788in}}{\pgfqpoint{10.654212in}{4.396398in}}{\pgfqpoint{10.665262in}{4.396398in}}%
\pgfpathlineto{\pgfqpoint{10.665262in}{4.396398in}}%
\pgfpathclose%
\pgfusepath{stroke}%
\end{pgfscope}%
\begin{pgfscope}%
\pgfpathrectangle{\pgfqpoint{7.394209in}{0.375000in}}{\pgfqpoint{6.356833in}{5.175000in}}%
\pgfusepath{clip}%
\pgfsetbuttcap%
\pgfsetroundjoin%
\pgfsetlinewidth{1.003750pt}%
\definecolor{currentstroke}{rgb}{0.827451,0.827451,0.827451}%
\pgfsetstrokecolor{currentstroke}%
\pgfsetdash{}{0pt}%
\pgfpathmoveto{\pgfqpoint{11.821017in}{5.476920in}}%
\pgfpathcurveto{\pgfqpoint{11.832067in}{5.476920in}}{\pgfqpoint{11.842666in}{5.481310in}}{\pgfqpoint{11.850480in}{5.489124in}}%
\pgfpathcurveto{\pgfqpoint{11.858294in}{5.496937in}}{\pgfqpoint{11.862684in}{5.507536in}}{\pgfqpoint{11.862684in}{5.518587in}}%
\pgfpathcurveto{\pgfqpoint{11.862684in}{5.529637in}}{\pgfqpoint{11.858294in}{5.540236in}}{\pgfqpoint{11.850480in}{5.548049in}}%
\pgfpathcurveto{\pgfqpoint{11.842666in}{5.555863in}}{\pgfqpoint{11.832067in}{5.560253in}}{\pgfqpoint{11.821017in}{5.560253in}}%
\pgfpathcurveto{\pgfqpoint{11.809967in}{5.560253in}}{\pgfqpoint{11.799368in}{5.555863in}}{\pgfqpoint{11.791554in}{5.548049in}}%
\pgfpathcurveto{\pgfqpoint{11.783741in}{5.540236in}}{\pgfqpoint{11.779351in}{5.529637in}}{\pgfqpoint{11.779351in}{5.518587in}}%
\pgfpathcurveto{\pgfqpoint{11.779351in}{5.507536in}}{\pgfqpoint{11.783741in}{5.496937in}}{\pgfqpoint{11.791554in}{5.489124in}}%
\pgfpathcurveto{\pgfqpoint{11.799368in}{5.481310in}}{\pgfqpoint{11.809967in}{5.476920in}}{\pgfqpoint{11.821017in}{5.476920in}}%
\pgfpathlineto{\pgfqpoint{11.821017in}{5.476920in}}%
\pgfpathclose%
\pgfusepath{stroke}%
\end{pgfscope}%
\begin{pgfscope}%
\pgfpathrectangle{\pgfqpoint{7.394209in}{0.375000in}}{\pgfqpoint{6.356833in}{5.175000in}}%
\pgfusepath{clip}%
\pgfsetbuttcap%
\pgfsetroundjoin%
\pgfsetlinewidth{1.003750pt}%
\definecolor{currentstroke}{rgb}{0.827451,0.827451,0.827451}%
\pgfsetstrokecolor{currentstroke}%
\pgfsetdash{}{0pt}%
\pgfpathmoveto{\pgfqpoint{9.834690in}{3.206240in}}%
\pgfpathcurveto{\pgfqpoint{9.845740in}{3.206240in}}{\pgfqpoint{9.856339in}{3.210631in}}{\pgfqpoint{9.864153in}{3.218444in}}%
\pgfpathcurveto{\pgfqpoint{9.871966in}{3.226258in}}{\pgfqpoint{9.876356in}{3.236857in}}{\pgfqpoint{9.876356in}{3.247907in}}%
\pgfpathcurveto{\pgfqpoint{9.876356in}{3.258957in}}{\pgfqpoint{9.871966in}{3.269556in}}{\pgfqpoint{9.864153in}{3.277370in}}%
\pgfpathcurveto{\pgfqpoint{9.856339in}{3.285184in}}{\pgfqpoint{9.845740in}{3.289574in}}{\pgfqpoint{9.834690in}{3.289574in}}%
\pgfpathcurveto{\pgfqpoint{9.823640in}{3.289574in}}{\pgfqpoint{9.813041in}{3.285184in}}{\pgfqpoint{9.805227in}{3.277370in}}%
\pgfpathcurveto{\pgfqpoint{9.797413in}{3.269556in}}{\pgfqpoint{9.793023in}{3.258957in}}{\pgfqpoint{9.793023in}{3.247907in}}%
\pgfpathcurveto{\pgfqpoint{9.793023in}{3.236857in}}{\pgfqpoint{9.797413in}{3.226258in}}{\pgfqpoint{9.805227in}{3.218444in}}%
\pgfpathcurveto{\pgfqpoint{9.813041in}{3.210631in}}{\pgfqpoint{9.823640in}{3.206240in}}{\pgfqpoint{9.834690in}{3.206240in}}%
\pgfpathlineto{\pgfqpoint{9.834690in}{3.206240in}}%
\pgfpathclose%
\pgfusepath{stroke}%
\end{pgfscope}%
\begin{pgfscope}%
\pgfpathrectangle{\pgfqpoint{7.394209in}{0.375000in}}{\pgfqpoint{6.356833in}{5.175000in}}%
\pgfusepath{clip}%
\pgfsetbuttcap%
\pgfsetroundjoin%
\pgfsetlinewidth{1.003750pt}%
\definecolor{currentstroke}{rgb}{0.827451,0.827451,0.827451}%
\pgfsetstrokecolor{currentstroke}%
\pgfsetdash{}{0pt}%
\pgfpathmoveto{\pgfqpoint{13.267669in}{5.477466in}}%
\pgfpathcurveto{\pgfqpoint{13.278720in}{5.477466in}}{\pgfqpoint{13.289319in}{5.481857in}}{\pgfqpoint{13.297132in}{5.489670in}}%
\pgfpathcurveto{\pgfqpoint{13.304946in}{5.497484in}}{\pgfqpoint{13.309336in}{5.508083in}}{\pgfqpoint{13.309336in}{5.519133in}}%
\pgfpathcurveto{\pgfqpoint{13.309336in}{5.530183in}}{\pgfqpoint{13.304946in}{5.540782in}}{\pgfqpoint{13.297132in}{5.548596in}}%
\pgfpathcurveto{\pgfqpoint{13.289319in}{5.556409in}}{\pgfqpoint{13.278720in}{5.560800in}}{\pgfqpoint{13.267669in}{5.560800in}}%
\pgfpathcurveto{\pgfqpoint{13.256619in}{5.560800in}}{\pgfqpoint{13.246020in}{5.556409in}}{\pgfqpoint{13.238207in}{5.548596in}}%
\pgfpathcurveto{\pgfqpoint{13.230393in}{5.540782in}}{\pgfqpoint{13.226003in}{5.530183in}}{\pgfqpoint{13.226003in}{5.519133in}}%
\pgfpathcurveto{\pgfqpoint{13.226003in}{5.508083in}}{\pgfqpoint{13.230393in}{5.497484in}}{\pgfqpoint{13.238207in}{5.489670in}}%
\pgfpathcurveto{\pgfqpoint{13.246020in}{5.481857in}}{\pgfqpoint{13.256619in}{5.477466in}}{\pgfqpoint{13.267669in}{5.477466in}}%
\pgfpathlineto{\pgfqpoint{13.267669in}{5.477466in}}%
\pgfpathclose%
\pgfusepath{stroke}%
\end{pgfscope}%
\begin{pgfscope}%
\pgfpathrectangle{\pgfqpoint{7.394209in}{0.375000in}}{\pgfqpoint{6.356833in}{5.175000in}}%
\pgfusepath{clip}%
\pgfsetbuttcap%
\pgfsetroundjoin%
\pgfsetlinewidth{1.003750pt}%
\definecolor{currentstroke}{rgb}{0.827451,0.827451,0.827451}%
\pgfsetstrokecolor{currentstroke}%
\pgfsetdash{}{0pt}%
\pgfpathmoveto{\pgfqpoint{7.661120in}{1.136791in}}%
\pgfpathcurveto{\pgfqpoint{7.672170in}{1.136791in}}{\pgfqpoint{7.682770in}{1.141181in}}{\pgfqpoint{7.690583in}{1.148995in}}%
\pgfpathcurveto{\pgfqpoint{7.698397in}{1.156808in}}{\pgfqpoint{7.702787in}{1.167407in}}{\pgfqpoint{7.702787in}{1.178457in}}%
\pgfpathcurveto{\pgfqpoint{7.702787in}{1.189508in}}{\pgfqpoint{7.698397in}{1.200107in}}{\pgfqpoint{7.690583in}{1.207920in}}%
\pgfpathcurveto{\pgfqpoint{7.682770in}{1.215734in}}{\pgfqpoint{7.672170in}{1.220124in}}{\pgfqpoint{7.661120in}{1.220124in}}%
\pgfpathcurveto{\pgfqpoint{7.650070in}{1.220124in}}{\pgfqpoint{7.639471in}{1.215734in}}{\pgfqpoint{7.631658in}{1.207920in}}%
\pgfpathcurveto{\pgfqpoint{7.623844in}{1.200107in}}{\pgfqpoint{7.619454in}{1.189508in}}{\pgfqpoint{7.619454in}{1.178457in}}%
\pgfpathcurveto{\pgfqpoint{7.619454in}{1.167407in}}{\pgfqpoint{7.623844in}{1.156808in}}{\pgfqpoint{7.631658in}{1.148995in}}%
\pgfpathcurveto{\pgfqpoint{7.639471in}{1.141181in}}{\pgfqpoint{7.650070in}{1.136791in}}{\pgfqpoint{7.661120in}{1.136791in}}%
\pgfpathlineto{\pgfqpoint{7.661120in}{1.136791in}}%
\pgfpathclose%
\pgfusepath{stroke}%
\end{pgfscope}%
\begin{pgfscope}%
\pgfpathrectangle{\pgfqpoint{7.394209in}{0.375000in}}{\pgfqpoint{6.356833in}{5.175000in}}%
\pgfusepath{clip}%
\pgfsetbuttcap%
\pgfsetroundjoin%
\pgfsetlinewidth{1.003750pt}%
\definecolor{currentstroke}{rgb}{0.827451,0.827451,0.827451}%
\pgfsetstrokecolor{currentstroke}%
\pgfsetdash{}{0pt}%
\pgfpathmoveto{\pgfqpoint{9.834690in}{3.210155in}}%
\pgfpathcurveto{\pgfqpoint{9.845740in}{3.210155in}}{\pgfqpoint{9.856339in}{3.214546in}}{\pgfqpoint{9.864153in}{3.222359in}}%
\pgfpathcurveto{\pgfqpoint{9.871966in}{3.230173in}}{\pgfqpoint{9.876356in}{3.240772in}}{\pgfqpoint{9.876356in}{3.251822in}}%
\pgfpathcurveto{\pgfqpoint{9.876356in}{3.262872in}}{\pgfqpoint{9.871966in}{3.273471in}}{\pgfqpoint{9.864153in}{3.281285in}}%
\pgfpathcurveto{\pgfqpoint{9.856339in}{3.289098in}}{\pgfqpoint{9.845740in}{3.293489in}}{\pgfqpoint{9.834690in}{3.293489in}}%
\pgfpathcurveto{\pgfqpoint{9.823640in}{3.293489in}}{\pgfqpoint{9.813041in}{3.289098in}}{\pgfqpoint{9.805227in}{3.281285in}}%
\pgfpathcurveto{\pgfqpoint{9.797413in}{3.273471in}}{\pgfqpoint{9.793023in}{3.262872in}}{\pgfqpoint{9.793023in}{3.251822in}}%
\pgfpathcurveto{\pgfqpoint{9.793023in}{3.240772in}}{\pgfqpoint{9.797413in}{3.230173in}}{\pgfqpoint{9.805227in}{3.222359in}}%
\pgfpathcurveto{\pgfqpoint{9.813041in}{3.214546in}}{\pgfqpoint{9.823640in}{3.210155in}}{\pgfqpoint{9.834690in}{3.210155in}}%
\pgfpathlineto{\pgfqpoint{9.834690in}{3.210155in}}%
\pgfpathclose%
\pgfusepath{stroke}%
\end{pgfscope}%
\begin{pgfscope}%
\pgfpathrectangle{\pgfqpoint{7.394209in}{0.375000in}}{\pgfqpoint{6.356833in}{5.175000in}}%
\pgfusepath{clip}%
\pgfsetbuttcap%
\pgfsetroundjoin%
\pgfsetlinewidth{1.003750pt}%
\definecolor{currentstroke}{rgb}{0.827451,0.827451,0.827451}%
\pgfsetstrokecolor{currentstroke}%
\pgfsetdash{}{0pt}%
\pgfpathmoveto{\pgfqpoint{10.142236in}{3.373903in}}%
\pgfpathcurveto{\pgfqpoint{10.153286in}{3.373903in}}{\pgfqpoint{10.163885in}{3.378293in}}{\pgfqpoint{10.171699in}{3.386107in}}%
\pgfpathcurveto{\pgfqpoint{10.179513in}{3.393920in}}{\pgfqpoint{10.183903in}{3.404519in}}{\pgfqpoint{10.183903in}{3.415569in}}%
\pgfpathcurveto{\pgfqpoint{10.183903in}{3.426619in}}{\pgfqpoint{10.179513in}{3.437218in}}{\pgfqpoint{10.171699in}{3.445032in}}%
\pgfpathcurveto{\pgfqpoint{10.163885in}{3.452846in}}{\pgfqpoint{10.153286in}{3.457236in}}{\pgfqpoint{10.142236in}{3.457236in}}%
\pgfpathcurveto{\pgfqpoint{10.131186in}{3.457236in}}{\pgfqpoint{10.120587in}{3.452846in}}{\pgfqpoint{10.112773in}{3.445032in}}%
\pgfpathcurveto{\pgfqpoint{10.104960in}{3.437218in}}{\pgfqpoint{10.100570in}{3.426619in}}{\pgfqpoint{10.100570in}{3.415569in}}%
\pgfpathcurveto{\pgfqpoint{10.100570in}{3.404519in}}{\pgfqpoint{10.104960in}{3.393920in}}{\pgfqpoint{10.112773in}{3.386107in}}%
\pgfpathcurveto{\pgfqpoint{10.120587in}{3.378293in}}{\pgfqpoint{10.131186in}{3.373903in}}{\pgfqpoint{10.142236in}{3.373903in}}%
\pgfpathlineto{\pgfqpoint{10.142236in}{3.373903in}}%
\pgfpathclose%
\pgfusepath{stroke}%
\end{pgfscope}%
\begin{pgfscope}%
\pgfpathrectangle{\pgfqpoint{7.394209in}{0.375000in}}{\pgfqpoint{6.356833in}{5.175000in}}%
\pgfusepath{clip}%
\pgfsetbuttcap%
\pgfsetroundjoin%
\pgfsetlinewidth{1.003750pt}%
\definecolor{currentstroke}{rgb}{0.827451,0.827451,0.827451}%
\pgfsetstrokecolor{currentstroke}%
\pgfsetdash{}{0pt}%
\pgfpathmoveto{\pgfqpoint{12.598832in}{5.394129in}}%
\pgfpathcurveto{\pgfqpoint{12.609882in}{5.394129in}}{\pgfqpoint{12.620481in}{5.398520in}}{\pgfqpoint{12.628295in}{5.406333in}}%
\pgfpathcurveto{\pgfqpoint{12.636108in}{5.414147in}}{\pgfqpoint{12.640499in}{5.424746in}}{\pgfqpoint{12.640499in}{5.435796in}}%
\pgfpathcurveto{\pgfqpoint{12.640499in}{5.446846in}}{\pgfqpoint{12.636108in}{5.457445in}}{\pgfqpoint{12.628295in}{5.465259in}}%
\pgfpathcurveto{\pgfqpoint{12.620481in}{5.473073in}}{\pgfqpoint{12.609882in}{5.477463in}}{\pgfqpoint{12.598832in}{5.477463in}}%
\pgfpathcurveto{\pgfqpoint{12.587782in}{5.477463in}}{\pgfqpoint{12.577183in}{5.473073in}}{\pgfqpoint{12.569369in}{5.465259in}}%
\pgfpathcurveto{\pgfqpoint{12.561556in}{5.457445in}}{\pgfqpoint{12.557165in}{5.446846in}}{\pgfqpoint{12.557165in}{5.435796in}}%
\pgfpathcurveto{\pgfqpoint{12.557165in}{5.424746in}}{\pgfqpoint{12.561556in}{5.414147in}}{\pgfqpoint{12.569369in}{5.406333in}}%
\pgfpathcurveto{\pgfqpoint{12.577183in}{5.398520in}}{\pgfqpoint{12.587782in}{5.394129in}}{\pgfqpoint{12.598832in}{5.394129in}}%
\pgfpathlineto{\pgfqpoint{12.598832in}{5.394129in}}%
\pgfpathclose%
\pgfusepath{stroke}%
\end{pgfscope}%
\begin{pgfscope}%
\pgfpathrectangle{\pgfqpoint{7.394209in}{0.375000in}}{\pgfqpoint{6.356833in}{5.175000in}}%
\pgfusepath{clip}%
\pgfsetbuttcap%
\pgfsetroundjoin%
\pgfsetlinewidth{1.003750pt}%
\definecolor{currentstroke}{rgb}{0.827451,0.827451,0.827451}%
\pgfsetstrokecolor{currentstroke}%
\pgfsetdash{}{0pt}%
\pgfpathmoveto{\pgfqpoint{12.254457in}{5.442977in}}%
\pgfpathcurveto{\pgfqpoint{12.265507in}{5.442977in}}{\pgfqpoint{12.276106in}{5.447367in}}{\pgfqpoint{12.283920in}{5.455181in}}%
\pgfpathcurveto{\pgfqpoint{12.291734in}{5.462994in}}{\pgfqpoint{12.296124in}{5.473593in}}{\pgfqpoint{12.296124in}{5.484644in}}%
\pgfpathcurveto{\pgfqpoint{12.296124in}{5.495694in}}{\pgfqpoint{12.291734in}{5.506293in}}{\pgfqpoint{12.283920in}{5.514106in}}%
\pgfpathcurveto{\pgfqpoint{12.276106in}{5.521920in}}{\pgfqpoint{12.265507in}{5.526310in}}{\pgfqpoint{12.254457in}{5.526310in}}%
\pgfpathcurveto{\pgfqpoint{12.243407in}{5.526310in}}{\pgfqpoint{12.232808in}{5.521920in}}{\pgfqpoint{12.224994in}{5.514106in}}%
\pgfpathcurveto{\pgfqpoint{12.217181in}{5.506293in}}{\pgfqpoint{12.212790in}{5.495694in}}{\pgfqpoint{12.212790in}{5.484644in}}%
\pgfpathcurveto{\pgfqpoint{12.212790in}{5.473593in}}{\pgfqpoint{12.217181in}{5.462994in}}{\pgfqpoint{12.224994in}{5.455181in}}%
\pgfpathcurveto{\pgfqpoint{12.232808in}{5.447367in}}{\pgfqpoint{12.243407in}{5.442977in}}{\pgfqpoint{12.254457in}{5.442977in}}%
\pgfpathlineto{\pgfqpoint{12.254457in}{5.442977in}}%
\pgfpathclose%
\pgfusepath{stroke}%
\end{pgfscope}%
\begin{pgfscope}%
\pgfpathrectangle{\pgfqpoint{7.394209in}{0.375000in}}{\pgfqpoint{6.356833in}{5.175000in}}%
\pgfusepath{clip}%
\pgfsetbuttcap%
\pgfsetroundjoin%
\pgfsetlinewidth{1.003750pt}%
\definecolor{currentstroke}{rgb}{0.827451,0.827451,0.827451}%
\pgfsetstrokecolor{currentstroke}%
\pgfsetdash{}{0pt}%
\pgfpathmoveto{\pgfqpoint{10.652016in}{5.489379in}}%
\pgfpathcurveto{\pgfqpoint{10.663066in}{5.489379in}}{\pgfqpoint{10.673665in}{5.493770in}}{\pgfqpoint{10.681479in}{5.501583in}}%
\pgfpathcurveto{\pgfqpoint{10.689293in}{5.509397in}}{\pgfqpoint{10.693683in}{5.519996in}}{\pgfqpoint{10.693683in}{5.531046in}}%
\pgfpathcurveto{\pgfqpoint{10.693683in}{5.542096in}}{\pgfqpoint{10.689293in}{5.552695in}}{\pgfqpoint{10.681479in}{5.560509in}}%
\pgfpathcurveto{\pgfqpoint{10.673665in}{5.568322in}}{\pgfqpoint{10.663066in}{5.572713in}}{\pgfqpoint{10.652016in}{5.572713in}}%
\pgfpathcurveto{\pgfqpoint{10.640966in}{5.572713in}}{\pgfqpoint{10.630367in}{5.568322in}}{\pgfqpoint{10.622553in}{5.560509in}}%
\pgfpathcurveto{\pgfqpoint{10.614740in}{5.552695in}}{\pgfqpoint{10.610350in}{5.542096in}}{\pgfqpoint{10.610350in}{5.531046in}}%
\pgfpathcurveto{\pgfqpoint{10.610350in}{5.519996in}}{\pgfqpoint{10.614740in}{5.509397in}}{\pgfqpoint{10.622553in}{5.501583in}}%
\pgfpathcurveto{\pgfqpoint{10.630367in}{5.493770in}}{\pgfqpoint{10.640966in}{5.489379in}}{\pgfqpoint{10.652016in}{5.489379in}}%
\pgfpathlineto{\pgfqpoint{10.652016in}{5.489379in}}%
\pgfpathclose%
\pgfusepath{stroke}%
\end{pgfscope}%
\begin{pgfscope}%
\pgfpathrectangle{\pgfqpoint{7.394209in}{0.375000in}}{\pgfqpoint{6.356833in}{5.175000in}}%
\pgfusepath{clip}%
\pgfsetbuttcap%
\pgfsetroundjoin%
\pgfsetlinewidth{1.003750pt}%
\definecolor{currentstroke}{rgb}{0.827451,0.827451,0.827451}%
\pgfsetstrokecolor{currentstroke}%
\pgfsetdash{}{0pt}%
\pgfpathmoveto{\pgfqpoint{7.568121in}{1.114558in}}%
\pgfpathcurveto{\pgfqpoint{7.579172in}{1.114558in}}{\pgfqpoint{7.589771in}{1.118948in}}{\pgfqpoint{7.597584in}{1.126762in}}%
\pgfpathcurveto{\pgfqpoint{7.605398in}{1.134575in}}{\pgfqpoint{7.609788in}{1.145174in}}{\pgfqpoint{7.609788in}{1.156224in}}%
\pgfpathcurveto{\pgfqpoint{7.609788in}{1.167274in}}{\pgfqpoint{7.605398in}{1.177874in}}{\pgfqpoint{7.597584in}{1.185687in}}%
\pgfpathcurveto{\pgfqpoint{7.589771in}{1.193501in}}{\pgfqpoint{7.579172in}{1.197891in}}{\pgfqpoint{7.568121in}{1.197891in}}%
\pgfpathcurveto{\pgfqpoint{7.557071in}{1.197891in}}{\pgfqpoint{7.546472in}{1.193501in}}{\pgfqpoint{7.538659in}{1.185687in}}%
\pgfpathcurveto{\pgfqpoint{7.530845in}{1.177874in}}{\pgfqpoint{7.526455in}{1.167274in}}{\pgfqpoint{7.526455in}{1.156224in}}%
\pgfpathcurveto{\pgfqpoint{7.526455in}{1.145174in}}{\pgfqpoint{7.530845in}{1.134575in}}{\pgfqpoint{7.538659in}{1.126762in}}%
\pgfpathcurveto{\pgfqpoint{7.546472in}{1.118948in}}{\pgfqpoint{7.557071in}{1.114558in}}{\pgfqpoint{7.568121in}{1.114558in}}%
\pgfpathlineto{\pgfqpoint{7.568121in}{1.114558in}}%
\pgfpathclose%
\pgfusepath{stroke}%
\end{pgfscope}%
\begin{pgfscope}%
\pgfpathrectangle{\pgfqpoint{7.394209in}{0.375000in}}{\pgfqpoint{6.356833in}{5.175000in}}%
\pgfusepath{clip}%
\pgfsetbuttcap%
\pgfsetroundjoin%
\pgfsetlinewidth{1.003750pt}%
\definecolor{currentstroke}{rgb}{0.827451,0.827451,0.827451}%
\pgfsetstrokecolor{currentstroke}%
\pgfsetdash{}{0pt}%
\pgfpathmoveto{\pgfqpoint{9.927865in}{3.190393in}}%
\pgfpathcurveto{\pgfqpoint{9.938915in}{3.190393in}}{\pgfqpoint{9.949514in}{3.194783in}}{\pgfqpoint{9.957328in}{3.202597in}}%
\pgfpathcurveto{\pgfqpoint{9.965142in}{3.210410in}}{\pgfqpoint{9.969532in}{3.221009in}}{\pgfqpoint{9.969532in}{3.232059in}}%
\pgfpathcurveto{\pgfqpoint{9.969532in}{3.243110in}}{\pgfqpoint{9.965142in}{3.253709in}}{\pgfqpoint{9.957328in}{3.261522in}}%
\pgfpathcurveto{\pgfqpoint{9.949514in}{3.269336in}}{\pgfqpoint{9.938915in}{3.273726in}}{\pgfqpoint{9.927865in}{3.273726in}}%
\pgfpathcurveto{\pgfqpoint{9.916815in}{3.273726in}}{\pgfqpoint{9.906216in}{3.269336in}}{\pgfqpoint{9.898403in}{3.261522in}}%
\pgfpathcurveto{\pgfqpoint{9.890589in}{3.253709in}}{\pgfqpoint{9.886199in}{3.243110in}}{\pgfqpoint{9.886199in}{3.232059in}}%
\pgfpathcurveto{\pgfqpoint{9.886199in}{3.221009in}}{\pgfqpoint{9.890589in}{3.210410in}}{\pgfqpoint{9.898403in}{3.202597in}}%
\pgfpathcurveto{\pgfqpoint{9.906216in}{3.194783in}}{\pgfqpoint{9.916815in}{3.190393in}}{\pgfqpoint{9.927865in}{3.190393in}}%
\pgfpathlineto{\pgfqpoint{9.927865in}{3.190393in}}%
\pgfpathclose%
\pgfusepath{stroke}%
\end{pgfscope}%
\begin{pgfscope}%
\pgfpathrectangle{\pgfqpoint{7.394209in}{0.375000in}}{\pgfqpoint{6.356833in}{5.175000in}}%
\pgfusepath{clip}%
\pgfsetbuttcap%
\pgfsetroundjoin%
\pgfsetlinewidth{1.003750pt}%
\definecolor{currentstroke}{rgb}{0.827451,0.827451,0.827451}%
\pgfsetstrokecolor{currentstroke}%
\pgfsetdash{}{0pt}%
\pgfpathmoveto{\pgfqpoint{9.633913in}{2.637760in}}%
\pgfpathcurveto{\pgfqpoint{9.644963in}{2.637760in}}{\pgfqpoint{9.655562in}{2.642150in}}{\pgfqpoint{9.663376in}{2.649964in}}%
\pgfpathcurveto{\pgfqpoint{9.671189in}{2.657777in}}{\pgfqpoint{9.675579in}{2.668376in}}{\pgfqpoint{9.675579in}{2.679426in}}%
\pgfpathcurveto{\pgfqpoint{9.675579in}{2.690477in}}{\pgfqpoint{9.671189in}{2.701076in}}{\pgfqpoint{9.663376in}{2.708889in}}%
\pgfpathcurveto{\pgfqpoint{9.655562in}{2.716703in}}{\pgfqpoint{9.644963in}{2.721093in}}{\pgfqpoint{9.633913in}{2.721093in}}%
\pgfpathcurveto{\pgfqpoint{9.622863in}{2.721093in}}{\pgfqpoint{9.612264in}{2.716703in}}{\pgfqpoint{9.604450in}{2.708889in}}%
\pgfpathcurveto{\pgfqpoint{9.596636in}{2.701076in}}{\pgfqpoint{9.592246in}{2.690477in}}{\pgfqpoint{9.592246in}{2.679426in}}%
\pgfpathcurveto{\pgfqpoint{9.592246in}{2.668376in}}{\pgfqpoint{9.596636in}{2.657777in}}{\pgfqpoint{9.604450in}{2.649964in}}%
\pgfpathcurveto{\pgfqpoint{9.612264in}{2.642150in}}{\pgfqpoint{9.622863in}{2.637760in}}{\pgfqpoint{9.633913in}{2.637760in}}%
\pgfpathlineto{\pgfqpoint{9.633913in}{2.637760in}}%
\pgfpathclose%
\pgfusepath{stroke}%
\end{pgfscope}%
\begin{pgfscope}%
\pgfpathrectangle{\pgfqpoint{7.394209in}{0.375000in}}{\pgfqpoint{6.356833in}{5.175000in}}%
\pgfusepath{clip}%
\pgfsetbuttcap%
\pgfsetroundjoin%
\pgfsetlinewidth{1.003750pt}%
\definecolor{currentstroke}{rgb}{0.827451,0.827451,0.827451}%
\pgfsetstrokecolor{currentstroke}%
\pgfsetdash{}{0pt}%
\pgfpathmoveto{\pgfqpoint{11.715424in}{5.019642in}}%
\pgfpathcurveto{\pgfqpoint{11.726474in}{5.019642in}}{\pgfqpoint{11.737073in}{5.024032in}}{\pgfqpoint{11.744887in}{5.031846in}}%
\pgfpathcurveto{\pgfqpoint{11.752700in}{5.039659in}}{\pgfqpoint{11.757090in}{5.050258in}}{\pgfqpoint{11.757090in}{5.061309in}}%
\pgfpathcurveto{\pgfqpoint{11.757090in}{5.072359in}}{\pgfqpoint{11.752700in}{5.082958in}}{\pgfqpoint{11.744887in}{5.090771in}}%
\pgfpathcurveto{\pgfqpoint{11.737073in}{5.098585in}}{\pgfqpoint{11.726474in}{5.102975in}}{\pgfqpoint{11.715424in}{5.102975in}}%
\pgfpathcurveto{\pgfqpoint{11.704374in}{5.102975in}}{\pgfqpoint{11.693775in}{5.098585in}}{\pgfqpoint{11.685961in}{5.090771in}}%
\pgfpathcurveto{\pgfqpoint{11.678147in}{5.082958in}}{\pgfqpoint{11.673757in}{5.072359in}}{\pgfqpoint{11.673757in}{5.061309in}}%
\pgfpathcurveto{\pgfqpoint{11.673757in}{5.050258in}}{\pgfqpoint{11.678147in}{5.039659in}}{\pgfqpoint{11.685961in}{5.031846in}}%
\pgfpathcurveto{\pgfqpoint{11.693775in}{5.024032in}}{\pgfqpoint{11.704374in}{5.019642in}}{\pgfqpoint{11.715424in}{5.019642in}}%
\pgfpathlineto{\pgfqpoint{11.715424in}{5.019642in}}%
\pgfpathclose%
\pgfusepath{stroke}%
\end{pgfscope}%
\begin{pgfscope}%
\pgfpathrectangle{\pgfqpoint{7.394209in}{0.375000in}}{\pgfqpoint{6.356833in}{5.175000in}}%
\pgfusepath{clip}%
\pgfsetbuttcap%
\pgfsetroundjoin%
\pgfsetlinewidth{1.003750pt}%
\definecolor{currentstroke}{rgb}{0.827451,0.827451,0.827451}%
\pgfsetstrokecolor{currentstroke}%
\pgfsetdash{}{0pt}%
\pgfpathmoveto{\pgfqpoint{10.283345in}{3.373903in}}%
\pgfpathcurveto{\pgfqpoint{10.294395in}{3.373903in}}{\pgfqpoint{10.304994in}{3.378293in}}{\pgfqpoint{10.312807in}{3.386107in}}%
\pgfpathcurveto{\pgfqpoint{10.320621in}{3.393920in}}{\pgfqpoint{10.325011in}{3.404519in}}{\pgfqpoint{10.325011in}{3.415569in}}%
\pgfpathcurveto{\pgfqpoint{10.325011in}{3.426619in}}{\pgfqpoint{10.320621in}{3.437218in}}{\pgfqpoint{10.312807in}{3.445032in}}%
\pgfpathcurveto{\pgfqpoint{10.304994in}{3.452846in}}{\pgfqpoint{10.294395in}{3.457236in}}{\pgfqpoint{10.283345in}{3.457236in}}%
\pgfpathcurveto{\pgfqpoint{10.272295in}{3.457236in}}{\pgfqpoint{10.261696in}{3.452846in}}{\pgfqpoint{10.253882in}{3.445032in}}%
\pgfpathcurveto{\pgfqpoint{10.246068in}{3.437218in}}{\pgfqpoint{10.241678in}{3.426619in}}{\pgfqpoint{10.241678in}{3.415569in}}%
\pgfpathcurveto{\pgfqpoint{10.241678in}{3.404519in}}{\pgfqpoint{10.246068in}{3.393920in}}{\pgfqpoint{10.253882in}{3.386107in}}%
\pgfpathcurveto{\pgfqpoint{10.261696in}{3.378293in}}{\pgfqpoint{10.272295in}{3.373903in}}{\pgfqpoint{10.283345in}{3.373903in}}%
\pgfpathlineto{\pgfqpoint{10.283345in}{3.373903in}}%
\pgfpathclose%
\pgfusepath{stroke}%
\end{pgfscope}%
\begin{pgfscope}%
\pgfpathrectangle{\pgfqpoint{7.394209in}{0.375000in}}{\pgfqpoint{6.356833in}{5.175000in}}%
\pgfusepath{clip}%
\pgfsetbuttcap%
\pgfsetroundjoin%
\pgfsetlinewidth{1.003750pt}%
\definecolor{currentstroke}{rgb}{0.827451,0.827451,0.827451}%
\pgfsetstrokecolor{currentstroke}%
\pgfsetdash{}{0pt}%
\pgfpathmoveto{\pgfqpoint{13.267418in}{5.495726in}}%
\pgfpathcurveto{\pgfqpoint{13.278468in}{5.495726in}}{\pgfqpoint{13.289067in}{5.500116in}}{\pgfqpoint{13.296881in}{5.507929in}}%
\pgfpathcurveto{\pgfqpoint{13.304694in}{5.515743in}}{\pgfqpoint{13.309084in}{5.526342in}}{\pgfqpoint{13.309084in}{5.537392in}}%
\pgfpathcurveto{\pgfqpoint{13.309084in}{5.548442in}}{\pgfqpoint{13.304694in}{5.559041in}}{\pgfqpoint{13.296881in}{5.566855in}}%
\pgfpathcurveto{\pgfqpoint{13.289067in}{5.574669in}}{\pgfqpoint{13.278468in}{5.579059in}}{\pgfqpoint{13.267418in}{5.579059in}}%
\pgfpathcurveto{\pgfqpoint{13.256368in}{5.579059in}}{\pgfqpoint{13.245769in}{5.574669in}}{\pgfqpoint{13.237955in}{5.566855in}}%
\pgfpathcurveto{\pgfqpoint{13.230141in}{5.559041in}}{\pgfqpoint{13.225751in}{5.548442in}}{\pgfqpoint{13.225751in}{5.537392in}}%
\pgfpathcurveto{\pgfqpoint{13.225751in}{5.526342in}}{\pgfqpoint{13.230141in}{5.515743in}}{\pgfqpoint{13.237955in}{5.507929in}}%
\pgfpathcurveto{\pgfqpoint{13.245769in}{5.500116in}}{\pgfqpoint{13.256368in}{5.495726in}}{\pgfqpoint{13.267418in}{5.495726in}}%
\pgfpathlineto{\pgfqpoint{13.267418in}{5.495726in}}%
\pgfpathclose%
\pgfusepath{stroke}%
\end{pgfscope}%
\begin{pgfscope}%
\pgfpathrectangle{\pgfqpoint{7.394209in}{0.375000in}}{\pgfqpoint{6.356833in}{5.175000in}}%
\pgfusepath{clip}%
\pgfsetbuttcap%
\pgfsetroundjoin%
\pgfsetlinewidth{1.003750pt}%
\definecolor{currentstroke}{rgb}{0.827451,0.827451,0.827451}%
\pgfsetstrokecolor{currentstroke}%
\pgfsetdash{}{0pt}%
\pgfpathmoveto{\pgfqpoint{8.963666in}{1.893318in}}%
\pgfpathcurveto{\pgfqpoint{8.974717in}{1.893318in}}{\pgfqpoint{8.985316in}{1.897708in}}{\pgfqpoint{8.993129in}{1.905522in}}%
\pgfpathcurveto{\pgfqpoint{9.000943in}{1.913335in}}{\pgfqpoint{9.005333in}{1.923934in}}{\pgfqpoint{9.005333in}{1.934984in}}%
\pgfpathcurveto{\pgfqpoint{9.005333in}{1.946035in}}{\pgfqpoint{9.000943in}{1.956634in}}{\pgfqpoint{8.993129in}{1.964447in}}%
\pgfpathcurveto{\pgfqpoint{8.985316in}{1.972261in}}{\pgfqpoint{8.974717in}{1.976651in}}{\pgfqpoint{8.963666in}{1.976651in}}%
\pgfpathcurveto{\pgfqpoint{8.952616in}{1.976651in}}{\pgfqpoint{8.942017in}{1.972261in}}{\pgfqpoint{8.934204in}{1.964447in}}%
\pgfpathcurveto{\pgfqpoint{8.926390in}{1.956634in}}{\pgfqpoint{8.922000in}{1.946035in}}{\pgfqpoint{8.922000in}{1.934984in}}%
\pgfpathcurveto{\pgfqpoint{8.922000in}{1.923934in}}{\pgfqpoint{8.926390in}{1.913335in}}{\pgfqpoint{8.934204in}{1.905522in}}%
\pgfpathcurveto{\pgfqpoint{8.942017in}{1.897708in}}{\pgfqpoint{8.952616in}{1.893318in}}{\pgfqpoint{8.963666in}{1.893318in}}%
\pgfpathlineto{\pgfqpoint{8.963666in}{1.893318in}}%
\pgfpathclose%
\pgfusepath{stroke}%
\end{pgfscope}%
\begin{pgfscope}%
\pgfpathrectangle{\pgfqpoint{7.394209in}{0.375000in}}{\pgfqpoint{6.356833in}{5.175000in}}%
\pgfusepath{clip}%
\pgfsetbuttcap%
\pgfsetroundjoin%
\pgfsetlinewidth{1.003750pt}%
\definecolor{currentstroke}{rgb}{0.827451,0.827451,0.827451}%
\pgfsetstrokecolor{currentstroke}%
\pgfsetdash{}{0pt}%
\pgfpathmoveto{\pgfqpoint{11.282527in}{5.015788in}}%
\pgfpathcurveto{\pgfqpoint{11.293577in}{5.015788in}}{\pgfqpoint{11.304176in}{5.020179in}}{\pgfqpoint{11.311990in}{5.027992in}}%
\pgfpathcurveto{\pgfqpoint{11.319804in}{5.035806in}}{\pgfqpoint{11.324194in}{5.046405in}}{\pgfqpoint{11.324194in}{5.057455in}}%
\pgfpathcurveto{\pgfqpoint{11.324194in}{5.068505in}}{\pgfqpoint{11.319804in}{5.079104in}}{\pgfqpoint{11.311990in}{5.086918in}}%
\pgfpathcurveto{\pgfqpoint{11.304176in}{5.094731in}}{\pgfqpoint{11.293577in}{5.099122in}}{\pgfqpoint{11.282527in}{5.099122in}}%
\pgfpathcurveto{\pgfqpoint{11.271477in}{5.099122in}}{\pgfqpoint{11.260878in}{5.094731in}}{\pgfqpoint{11.253064in}{5.086918in}}%
\pgfpathcurveto{\pgfqpoint{11.245251in}{5.079104in}}{\pgfqpoint{11.240861in}{5.068505in}}{\pgfqpoint{11.240861in}{5.057455in}}%
\pgfpathcurveto{\pgfqpoint{11.240861in}{5.046405in}}{\pgfqpoint{11.245251in}{5.035806in}}{\pgfqpoint{11.253064in}{5.027992in}}%
\pgfpathcurveto{\pgfqpoint{11.260878in}{5.020179in}}{\pgfqpoint{11.271477in}{5.015788in}}{\pgfqpoint{11.282527in}{5.015788in}}%
\pgfpathlineto{\pgfqpoint{11.282527in}{5.015788in}}%
\pgfpathclose%
\pgfusepath{stroke}%
\end{pgfscope}%
\begin{pgfscope}%
\pgfpathrectangle{\pgfqpoint{7.394209in}{0.375000in}}{\pgfqpoint{6.356833in}{5.175000in}}%
\pgfusepath{clip}%
\pgfsetbuttcap%
\pgfsetroundjoin%
\pgfsetlinewidth{1.003750pt}%
\definecolor{currentstroke}{rgb}{0.827451,0.827451,0.827451}%
\pgfsetstrokecolor{currentstroke}%
\pgfsetdash{}{0pt}%
\pgfpathmoveto{\pgfqpoint{8.899083in}{1.984170in}}%
\pgfpathcurveto{\pgfqpoint{8.910134in}{1.984170in}}{\pgfqpoint{8.920733in}{1.988560in}}{\pgfqpoint{8.928546in}{1.996374in}}%
\pgfpathcurveto{\pgfqpoint{8.936360in}{2.004188in}}{\pgfqpoint{8.940750in}{2.014787in}}{\pgfqpoint{8.940750in}{2.025837in}}%
\pgfpathcurveto{\pgfqpoint{8.940750in}{2.036887in}}{\pgfqpoint{8.936360in}{2.047486in}}{\pgfqpoint{8.928546in}{2.055299in}}%
\pgfpathcurveto{\pgfqpoint{8.920733in}{2.063113in}}{\pgfqpoint{8.910134in}{2.067503in}}{\pgfqpoint{8.899083in}{2.067503in}}%
\pgfpathcurveto{\pgfqpoint{8.888033in}{2.067503in}}{\pgfqpoint{8.877434in}{2.063113in}}{\pgfqpoint{8.869621in}{2.055299in}}%
\pgfpathcurveto{\pgfqpoint{8.861807in}{2.047486in}}{\pgfqpoint{8.857417in}{2.036887in}}{\pgfqpoint{8.857417in}{2.025837in}}%
\pgfpathcurveto{\pgfqpoint{8.857417in}{2.014787in}}{\pgfqpoint{8.861807in}{2.004188in}}{\pgfqpoint{8.869621in}{1.996374in}}%
\pgfpathcurveto{\pgfqpoint{8.877434in}{1.988560in}}{\pgfqpoint{8.888033in}{1.984170in}}{\pgfqpoint{8.899083in}{1.984170in}}%
\pgfpathlineto{\pgfqpoint{8.899083in}{1.984170in}}%
\pgfpathclose%
\pgfusepath{stroke}%
\end{pgfscope}%
\begin{pgfscope}%
\pgfpathrectangle{\pgfqpoint{7.394209in}{0.375000in}}{\pgfqpoint{6.356833in}{5.175000in}}%
\pgfusepath{clip}%
\pgfsetbuttcap%
\pgfsetroundjoin%
\pgfsetlinewidth{1.003750pt}%
\definecolor{currentstroke}{rgb}{0.827451,0.827451,0.827451}%
\pgfsetstrokecolor{currentstroke}%
\pgfsetdash{}{0pt}%
\pgfpathmoveto{\pgfqpoint{8.633110in}{2.613891in}}%
\pgfpathcurveto{\pgfqpoint{8.644160in}{2.613891in}}{\pgfqpoint{8.654759in}{2.618281in}}{\pgfqpoint{8.662573in}{2.626095in}}%
\pgfpathcurveto{\pgfqpoint{8.670387in}{2.633909in}}{\pgfqpoint{8.674777in}{2.644508in}}{\pgfqpoint{8.674777in}{2.655558in}}%
\pgfpathcurveto{\pgfqpoint{8.674777in}{2.666608in}}{\pgfqpoint{8.670387in}{2.677207in}}{\pgfqpoint{8.662573in}{2.685021in}}%
\pgfpathcurveto{\pgfqpoint{8.654759in}{2.692834in}}{\pgfqpoint{8.644160in}{2.697224in}}{\pgfqpoint{8.633110in}{2.697224in}}%
\pgfpathcurveto{\pgfqpoint{8.622060in}{2.697224in}}{\pgfqpoint{8.611461in}{2.692834in}}{\pgfqpoint{8.603647in}{2.685021in}}%
\pgfpathcurveto{\pgfqpoint{8.595834in}{2.677207in}}{\pgfqpoint{8.591444in}{2.666608in}}{\pgfqpoint{8.591444in}{2.655558in}}%
\pgfpathcurveto{\pgfqpoint{8.591444in}{2.644508in}}{\pgfqpoint{8.595834in}{2.633909in}}{\pgfqpoint{8.603647in}{2.626095in}}%
\pgfpathcurveto{\pgfqpoint{8.611461in}{2.618281in}}{\pgfqpoint{8.622060in}{2.613891in}}{\pgfqpoint{8.633110in}{2.613891in}}%
\pgfpathlineto{\pgfqpoint{8.633110in}{2.613891in}}%
\pgfpathclose%
\pgfusepath{stroke}%
\end{pgfscope}%
\begin{pgfscope}%
\pgfpathrectangle{\pgfqpoint{7.394209in}{0.375000in}}{\pgfqpoint{6.356833in}{5.175000in}}%
\pgfusepath{clip}%
\pgfsetbuttcap%
\pgfsetroundjoin%
\pgfsetlinewidth{1.003750pt}%
\definecolor{currentstroke}{rgb}{0.827451,0.827451,0.827451}%
\pgfsetstrokecolor{currentstroke}%
\pgfsetdash{}{0pt}%
\pgfpathmoveto{\pgfqpoint{9.843597in}{2.597315in}}%
\pgfpathcurveto{\pgfqpoint{9.854648in}{2.597315in}}{\pgfqpoint{9.865247in}{2.601705in}}{\pgfqpoint{9.873060in}{2.609519in}}%
\pgfpathcurveto{\pgfqpoint{9.880874in}{2.617332in}}{\pgfqpoint{9.885264in}{2.627931in}}{\pgfqpoint{9.885264in}{2.638982in}}%
\pgfpathcurveto{\pgfqpoint{9.885264in}{2.650032in}}{\pgfqpoint{9.880874in}{2.660631in}}{\pgfqpoint{9.873060in}{2.668444in}}%
\pgfpathcurveto{\pgfqpoint{9.865247in}{2.676258in}}{\pgfqpoint{9.854648in}{2.680648in}}{\pgfqpoint{9.843597in}{2.680648in}}%
\pgfpathcurveto{\pgfqpoint{9.832547in}{2.680648in}}{\pgfqpoint{9.821948in}{2.676258in}}{\pgfqpoint{9.814135in}{2.668444in}}%
\pgfpathcurveto{\pgfqpoint{9.806321in}{2.660631in}}{\pgfqpoint{9.801931in}{2.650032in}}{\pgfqpoint{9.801931in}{2.638982in}}%
\pgfpathcurveto{\pgfqpoint{9.801931in}{2.627931in}}{\pgfqpoint{9.806321in}{2.617332in}}{\pgfqpoint{9.814135in}{2.609519in}}%
\pgfpathcurveto{\pgfqpoint{9.821948in}{2.601705in}}{\pgfqpoint{9.832547in}{2.597315in}}{\pgfqpoint{9.843597in}{2.597315in}}%
\pgfpathlineto{\pgfqpoint{9.843597in}{2.597315in}}%
\pgfpathclose%
\pgfusepath{stroke}%
\end{pgfscope}%
\begin{pgfscope}%
\pgfpathrectangle{\pgfqpoint{7.394209in}{0.375000in}}{\pgfqpoint{6.356833in}{5.175000in}}%
\pgfusepath{clip}%
\pgfsetbuttcap%
\pgfsetroundjoin%
\pgfsetlinewidth{1.003750pt}%
\definecolor{currentstroke}{rgb}{0.827451,0.827451,0.827451}%
\pgfsetstrokecolor{currentstroke}%
\pgfsetdash{}{0pt}%
\pgfpathmoveto{\pgfqpoint{10.385212in}{3.600815in}}%
\pgfpathcurveto{\pgfqpoint{10.396262in}{3.600815in}}{\pgfqpoint{10.406861in}{3.605206in}}{\pgfqpoint{10.414674in}{3.613019in}}%
\pgfpathcurveto{\pgfqpoint{10.422488in}{3.620833in}}{\pgfqpoint{10.426878in}{3.631432in}}{\pgfqpoint{10.426878in}{3.642482in}}%
\pgfpathcurveto{\pgfqpoint{10.426878in}{3.653532in}}{\pgfqpoint{10.422488in}{3.664131in}}{\pgfqpoint{10.414674in}{3.671945in}}%
\pgfpathcurveto{\pgfqpoint{10.406861in}{3.679759in}}{\pgfqpoint{10.396262in}{3.684149in}}{\pgfqpoint{10.385212in}{3.684149in}}%
\pgfpathcurveto{\pgfqpoint{10.374162in}{3.684149in}}{\pgfqpoint{10.363562in}{3.679759in}}{\pgfqpoint{10.355749in}{3.671945in}}%
\pgfpathcurveto{\pgfqpoint{10.347935in}{3.664131in}}{\pgfqpoint{10.343545in}{3.653532in}}{\pgfqpoint{10.343545in}{3.642482in}}%
\pgfpathcurveto{\pgfqpoint{10.343545in}{3.631432in}}{\pgfqpoint{10.347935in}{3.620833in}}{\pgfqpoint{10.355749in}{3.613019in}}%
\pgfpathcurveto{\pgfqpoint{10.363562in}{3.605206in}}{\pgfqpoint{10.374162in}{3.600815in}}{\pgfqpoint{10.385212in}{3.600815in}}%
\pgfpathlineto{\pgfqpoint{10.385212in}{3.600815in}}%
\pgfpathclose%
\pgfusepath{stroke}%
\end{pgfscope}%
\begin{pgfscope}%
\pgfpathrectangle{\pgfqpoint{7.394209in}{0.375000in}}{\pgfqpoint{6.356833in}{5.175000in}}%
\pgfusepath{clip}%
\pgfsetbuttcap%
\pgfsetroundjoin%
\pgfsetlinewidth{1.003750pt}%
\definecolor{currentstroke}{rgb}{0.827451,0.827451,0.827451}%
\pgfsetstrokecolor{currentstroke}%
\pgfsetdash{}{0pt}%
\pgfpathmoveto{\pgfqpoint{8.567326in}{1.140380in}}%
\pgfpathcurveto{\pgfqpoint{8.578377in}{1.140380in}}{\pgfqpoint{8.588976in}{1.144770in}}{\pgfqpoint{8.596789in}{1.152584in}}%
\pgfpathcurveto{\pgfqpoint{8.604603in}{1.160398in}}{\pgfqpoint{8.608993in}{1.170997in}}{\pgfqpoint{8.608993in}{1.182047in}}%
\pgfpathcurveto{\pgfqpoint{8.608993in}{1.193097in}}{\pgfqpoint{8.604603in}{1.203696in}}{\pgfqpoint{8.596789in}{1.211510in}}%
\pgfpathcurveto{\pgfqpoint{8.588976in}{1.219323in}}{\pgfqpoint{8.578377in}{1.223713in}}{\pgfqpoint{8.567326in}{1.223713in}}%
\pgfpathcurveto{\pgfqpoint{8.556276in}{1.223713in}}{\pgfqpoint{8.545677in}{1.219323in}}{\pgfqpoint{8.537864in}{1.211510in}}%
\pgfpathcurveto{\pgfqpoint{8.530050in}{1.203696in}}{\pgfqpoint{8.525660in}{1.193097in}}{\pgfqpoint{8.525660in}{1.182047in}}%
\pgfpathcurveto{\pgfqpoint{8.525660in}{1.170997in}}{\pgfqpoint{8.530050in}{1.160398in}}{\pgfqpoint{8.537864in}{1.152584in}}%
\pgfpathcurveto{\pgfqpoint{8.545677in}{1.144770in}}{\pgfqpoint{8.556276in}{1.140380in}}{\pgfqpoint{8.567326in}{1.140380in}}%
\pgfpathlineto{\pgfqpoint{8.567326in}{1.140380in}}%
\pgfpathclose%
\pgfusepath{stroke}%
\end{pgfscope}%
\begin{pgfscope}%
\pgfpathrectangle{\pgfqpoint{7.394209in}{0.375000in}}{\pgfqpoint{6.356833in}{5.175000in}}%
\pgfusepath{clip}%
\pgfsetbuttcap%
\pgfsetroundjoin%
\pgfsetlinewidth{1.003750pt}%
\definecolor{currentstroke}{rgb}{0.827451,0.827451,0.827451}%
\pgfsetstrokecolor{currentstroke}%
\pgfsetdash{}{0pt}%
\pgfpathmoveto{\pgfqpoint{7.627475in}{1.201624in}}%
\pgfpathcurveto{\pgfqpoint{7.638525in}{1.201624in}}{\pgfqpoint{7.649124in}{1.206014in}}{\pgfqpoint{7.656937in}{1.213828in}}%
\pgfpathcurveto{\pgfqpoint{7.664751in}{1.221642in}}{\pgfqpoint{7.669141in}{1.232241in}}{\pgfqpoint{7.669141in}{1.243291in}}%
\pgfpathcurveto{\pgfqpoint{7.669141in}{1.254341in}}{\pgfqpoint{7.664751in}{1.264940in}}{\pgfqpoint{7.656937in}{1.272754in}}%
\pgfpathcurveto{\pgfqpoint{7.649124in}{1.280567in}}{\pgfqpoint{7.638525in}{1.284957in}}{\pgfqpoint{7.627475in}{1.284957in}}%
\pgfpathcurveto{\pgfqpoint{7.616424in}{1.284957in}}{\pgfqpoint{7.605825in}{1.280567in}}{\pgfqpoint{7.598012in}{1.272754in}}%
\pgfpathcurveto{\pgfqpoint{7.590198in}{1.264940in}}{\pgfqpoint{7.585808in}{1.254341in}}{\pgfqpoint{7.585808in}{1.243291in}}%
\pgfpathcurveto{\pgfqpoint{7.585808in}{1.232241in}}{\pgfqpoint{7.590198in}{1.221642in}}{\pgfqpoint{7.598012in}{1.213828in}}%
\pgfpathcurveto{\pgfqpoint{7.605825in}{1.206014in}}{\pgfqpoint{7.616424in}{1.201624in}}{\pgfqpoint{7.627475in}{1.201624in}}%
\pgfpathlineto{\pgfqpoint{7.627475in}{1.201624in}}%
\pgfpathclose%
\pgfusepath{stroke}%
\end{pgfscope}%
\begin{pgfscope}%
\pgfpathrectangle{\pgfqpoint{7.394209in}{0.375000in}}{\pgfqpoint{6.356833in}{5.175000in}}%
\pgfusepath{clip}%
\pgfsetbuttcap%
\pgfsetroundjoin%
\pgfsetlinewidth{1.003750pt}%
\definecolor{currentstroke}{rgb}{0.827451,0.827451,0.827451}%
\pgfsetstrokecolor{currentstroke}%
\pgfsetdash{}{0pt}%
\pgfpathmoveto{\pgfqpoint{13.058680in}{5.489338in}}%
\pgfpathcurveto{\pgfqpoint{13.069730in}{5.489338in}}{\pgfqpoint{13.080329in}{5.493729in}}{\pgfqpoint{13.088143in}{5.501542in}}%
\pgfpathcurveto{\pgfqpoint{13.095956in}{5.509356in}}{\pgfqpoint{13.100347in}{5.519955in}}{\pgfqpoint{13.100347in}{5.531005in}}%
\pgfpathcurveto{\pgfqpoint{13.100347in}{5.542055in}}{\pgfqpoint{13.095956in}{5.552654in}}{\pgfqpoint{13.088143in}{5.560468in}}%
\pgfpathcurveto{\pgfqpoint{13.080329in}{5.568282in}}{\pgfqpoint{13.069730in}{5.572672in}}{\pgfqpoint{13.058680in}{5.572672in}}%
\pgfpathcurveto{\pgfqpoint{13.047630in}{5.572672in}}{\pgfqpoint{13.037031in}{5.568282in}}{\pgfqpoint{13.029217in}{5.560468in}}%
\pgfpathcurveto{\pgfqpoint{13.021404in}{5.552654in}}{\pgfqpoint{13.017013in}{5.542055in}}{\pgfqpoint{13.017013in}{5.531005in}}%
\pgfpathcurveto{\pgfqpoint{13.017013in}{5.519955in}}{\pgfqpoint{13.021404in}{5.509356in}}{\pgfqpoint{13.029217in}{5.501542in}}%
\pgfpathcurveto{\pgfqpoint{13.037031in}{5.493729in}}{\pgfqpoint{13.047630in}{5.489338in}}{\pgfqpoint{13.058680in}{5.489338in}}%
\pgfpathlineto{\pgfqpoint{13.058680in}{5.489338in}}%
\pgfpathclose%
\pgfusepath{stroke}%
\end{pgfscope}%
\begin{pgfscope}%
\pgfpathrectangle{\pgfqpoint{7.394209in}{0.375000in}}{\pgfqpoint{6.356833in}{5.175000in}}%
\pgfusepath{clip}%
\pgfsetbuttcap%
\pgfsetroundjoin%
\pgfsetlinewidth{1.003750pt}%
\definecolor{currentstroke}{rgb}{0.827451,0.827451,0.827451}%
\pgfsetstrokecolor{currentstroke}%
\pgfsetdash{}{0pt}%
\pgfpathmoveto{\pgfqpoint{9.570860in}{2.764582in}}%
\pgfpathcurveto{\pgfqpoint{9.581910in}{2.764582in}}{\pgfqpoint{9.592509in}{2.768972in}}{\pgfqpoint{9.600323in}{2.776786in}}%
\pgfpathcurveto{\pgfqpoint{9.608136in}{2.784599in}}{\pgfqpoint{9.612527in}{2.795198in}}{\pgfqpoint{9.612527in}{2.806248in}}%
\pgfpathcurveto{\pgfqpoint{9.612527in}{2.817299in}}{\pgfqpoint{9.608136in}{2.827898in}}{\pgfqpoint{9.600323in}{2.835711in}}%
\pgfpathcurveto{\pgfqpoint{9.592509in}{2.843525in}}{\pgfqpoint{9.581910in}{2.847915in}}{\pgfqpoint{9.570860in}{2.847915in}}%
\pgfpathcurveto{\pgfqpoint{9.559810in}{2.847915in}}{\pgfqpoint{9.549211in}{2.843525in}}{\pgfqpoint{9.541397in}{2.835711in}}%
\pgfpathcurveto{\pgfqpoint{9.533584in}{2.827898in}}{\pgfqpoint{9.529193in}{2.817299in}}{\pgfqpoint{9.529193in}{2.806248in}}%
\pgfpathcurveto{\pgfqpoint{9.529193in}{2.795198in}}{\pgfqpoint{9.533584in}{2.784599in}}{\pgfqpoint{9.541397in}{2.776786in}}%
\pgfpathcurveto{\pgfqpoint{9.549211in}{2.768972in}}{\pgfqpoint{9.559810in}{2.764582in}}{\pgfqpoint{9.570860in}{2.764582in}}%
\pgfpathlineto{\pgfqpoint{9.570860in}{2.764582in}}%
\pgfpathclose%
\pgfusepath{stroke}%
\end{pgfscope}%
\begin{pgfscope}%
\pgfpathrectangle{\pgfqpoint{7.394209in}{0.375000in}}{\pgfqpoint{6.356833in}{5.175000in}}%
\pgfusepath{clip}%
\pgfsetbuttcap%
\pgfsetroundjoin%
\pgfsetlinewidth{1.003750pt}%
\definecolor{currentstroke}{rgb}{0.827451,0.827451,0.827451}%
\pgfsetstrokecolor{currentstroke}%
\pgfsetdash{}{0pt}%
\pgfpathmoveto{\pgfqpoint{8.499898in}{1.114929in}}%
\pgfpathcurveto{\pgfqpoint{8.510948in}{1.114929in}}{\pgfqpoint{8.521547in}{1.119319in}}{\pgfqpoint{8.529361in}{1.127133in}}%
\pgfpathcurveto{\pgfqpoint{8.537174in}{1.134947in}}{\pgfqpoint{8.541564in}{1.145546in}}{\pgfqpoint{8.541564in}{1.156596in}}%
\pgfpathcurveto{\pgfqpoint{8.541564in}{1.167646in}}{\pgfqpoint{8.537174in}{1.178245in}}{\pgfqpoint{8.529361in}{1.186058in}}%
\pgfpathcurveto{\pgfqpoint{8.521547in}{1.193872in}}{\pgfqpoint{8.510948in}{1.198262in}}{\pgfqpoint{8.499898in}{1.198262in}}%
\pgfpathcurveto{\pgfqpoint{8.488848in}{1.198262in}}{\pgfqpoint{8.478249in}{1.193872in}}{\pgfqpoint{8.470435in}{1.186058in}}%
\pgfpathcurveto{\pgfqpoint{8.462621in}{1.178245in}}{\pgfqpoint{8.458231in}{1.167646in}}{\pgfqpoint{8.458231in}{1.156596in}}%
\pgfpathcurveto{\pgfqpoint{8.458231in}{1.145546in}}{\pgfqpoint{8.462621in}{1.134947in}}{\pgfqpoint{8.470435in}{1.127133in}}%
\pgfpathcurveto{\pgfqpoint{8.478249in}{1.119319in}}{\pgfqpoint{8.488848in}{1.114929in}}{\pgfqpoint{8.499898in}{1.114929in}}%
\pgfpathlineto{\pgfqpoint{8.499898in}{1.114929in}}%
\pgfpathclose%
\pgfusepath{stroke}%
\end{pgfscope}%
\begin{pgfscope}%
\pgfpathrectangle{\pgfqpoint{7.394209in}{0.375000in}}{\pgfqpoint{6.356833in}{5.175000in}}%
\pgfusepath{clip}%
\pgfsetbuttcap%
\pgfsetroundjoin%
\pgfsetlinewidth{1.003750pt}%
\definecolor{currentstroke}{rgb}{0.827451,0.827451,0.827451}%
\pgfsetstrokecolor{currentstroke}%
\pgfsetdash{}{0pt}%
\pgfpathmoveto{\pgfqpoint{8.567326in}{1.158676in}}%
\pgfpathcurveto{\pgfqpoint{8.578377in}{1.158676in}}{\pgfqpoint{8.588976in}{1.163066in}}{\pgfqpoint{8.596789in}{1.170880in}}%
\pgfpathcurveto{\pgfqpoint{8.604603in}{1.178693in}}{\pgfqpoint{8.608993in}{1.189292in}}{\pgfqpoint{8.608993in}{1.200342in}}%
\pgfpathcurveto{\pgfqpoint{8.608993in}{1.211393in}}{\pgfqpoint{8.604603in}{1.221992in}}{\pgfqpoint{8.596789in}{1.229805in}}%
\pgfpathcurveto{\pgfqpoint{8.588976in}{1.237619in}}{\pgfqpoint{8.578377in}{1.242009in}}{\pgfqpoint{8.567326in}{1.242009in}}%
\pgfpathcurveto{\pgfqpoint{8.556276in}{1.242009in}}{\pgfqpoint{8.545677in}{1.237619in}}{\pgfqpoint{8.537864in}{1.229805in}}%
\pgfpathcurveto{\pgfqpoint{8.530050in}{1.221992in}}{\pgfqpoint{8.525660in}{1.211393in}}{\pgfqpoint{8.525660in}{1.200342in}}%
\pgfpathcurveto{\pgfqpoint{8.525660in}{1.189292in}}{\pgfqpoint{8.530050in}{1.178693in}}{\pgfqpoint{8.537864in}{1.170880in}}%
\pgfpathcurveto{\pgfqpoint{8.545677in}{1.163066in}}{\pgfqpoint{8.556276in}{1.158676in}}{\pgfqpoint{8.567326in}{1.158676in}}%
\pgfpathlineto{\pgfqpoint{8.567326in}{1.158676in}}%
\pgfpathclose%
\pgfusepath{stroke}%
\end{pgfscope}%
\begin{pgfscope}%
\pgfpathrectangle{\pgfqpoint{7.394209in}{0.375000in}}{\pgfqpoint{6.356833in}{5.175000in}}%
\pgfusepath{clip}%
\pgfsetbuttcap%
\pgfsetroundjoin%
\pgfsetlinewidth{1.003750pt}%
\definecolor{currentstroke}{rgb}{0.827451,0.827451,0.827451}%
\pgfsetstrokecolor{currentstroke}%
\pgfsetdash{}{0pt}%
\pgfpathmoveto{\pgfqpoint{7.977823in}{0.697512in}}%
\pgfpathcurveto{\pgfqpoint{7.988873in}{0.697512in}}{\pgfqpoint{7.999472in}{0.701902in}}{\pgfqpoint{8.007286in}{0.709716in}}%
\pgfpathcurveto{\pgfqpoint{8.015100in}{0.717529in}}{\pgfqpoint{8.019490in}{0.728128in}}{\pgfqpoint{8.019490in}{0.739178in}}%
\pgfpathcurveto{\pgfqpoint{8.019490in}{0.750229in}}{\pgfqpoint{8.015100in}{0.760828in}}{\pgfqpoint{8.007286in}{0.768641in}}%
\pgfpathcurveto{\pgfqpoint{7.999472in}{0.776455in}}{\pgfqpoint{7.988873in}{0.780845in}}{\pgfqpoint{7.977823in}{0.780845in}}%
\pgfpathcurveto{\pgfqpoint{7.966773in}{0.780845in}}{\pgfqpoint{7.956174in}{0.776455in}}{\pgfqpoint{7.948360in}{0.768641in}}%
\pgfpathcurveto{\pgfqpoint{7.940547in}{0.760828in}}{\pgfqpoint{7.936157in}{0.750229in}}{\pgfqpoint{7.936157in}{0.739178in}}%
\pgfpathcurveto{\pgfqpoint{7.936157in}{0.728128in}}{\pgfqpoint{7.940547in}{0.717529in}}{\pgfqpoint{7.948360in}{0.709716in}}%
\pgfpathcurveto{\pgfqpoint{7.956174in}{0.701902in}}{\pgfqpoint{7.966773in}{0.697512in}}{\pgfqpoint{7.977823in}{0.697512in}}%
\pgfpathlineto{\pgfqpoint{7.977823in}{0.697512in}}%
\pgfpathclose%
\pgfusepath{stroke}%
\end{pgfscope}%
\begin{pgfscope}%
\pgfpathrectangle{\pgfqpoint{7.394209in}{0.375000in}}{\pgfqpoint{6.356833in}{5.175000in}}%
\pgfusepath{clip}%
\pgfsetbuttcap%
\pgfsetroundjoin%
\pgfsetlinewidth{1.003750pt}%
\definecolor{currentstroke}{rgb}{0.827451,0.827451,0.827451}%
\pgfsetstrokecolor{currentstroke}%
\pgfsetdash{}{0pt}%
\pgfpathmoveto{\pgfqpoint{10.652016in}{5.485929in}}%
\pgfpathcurveto{\pgfqpoint{10.663066in}{5.485929in}}{\pgfqpoint{10.673665in}{5.490319in}}{\pgfqpoint{10.681479in}{5.498133in}}%
\pgfpathcurveto{\pgfqpoint{10.689293in}{5.505947in}}{\pgfqpoint{10.693683in}{5.516546in}}{\pgfqpoint{10.693683in}{5.527596in}}%
\pgfpathcurveto{\pgfqpoint{10.693683in}{5.538646in}}{\pgfqpoint{10.689293in}{5.549245in}}{\pgfqpoint{10.681479in}{5.557059in}}%
\pgfpathcurveto{\pgfqpoint{10.673665in}{5.564872in}}{\pgfqpoint{10.663066in}{5.569262in}}{\pgfqpoint{10.652016in}{5.569262in}}%
\pgfpathcurveto{\pgfqpoint{10.640966in}{5.569262in}}{\pgfqpoint{10.630367in}{5.564872in}}{\pgfqpoint{10.622553in}{5.557059in}}%
\pgfpathcurveto{\pgfqpoint{10.614740in}{5.549245in}}{\pgfqpoint{10.610350in}{5.538646in}}{\pgfqpoint{10.610350in}{5.527596in}}%
\pgfpathcurveto{\pgfqpoint{10.610350in}{5.516546in}}{\pgfqpoint{10.614740in}{5.505947in}}{\pgfqpoint{10.622553in}{5.498133in}}%
\pgfpathcurveto{\pgfqpoint{10.630367in}{5.490319in}}{\pgfqpoint{10.640966in}{5.485929in}}{\pgfqpoint{10.652016in}{5.485929in}}%
\pgfpathlineto{\pgfqpoint{10.652016in}{5.485929in}}%
\pgfpathclose%
\pgfusepath{stroke}%
\end{pgfscope}%
\begin{pgfscope}%
\pgfpathrectangle{\pgfqpoint{7.394209in}{0.375000in}}{\pgfqpoint{6.356833in}{5.175000in}}%
\pgfusepath{clip}%
\pgfsetbuttcap%
\pgfsetroundjoin%
\pgfsetlinewidth{1.003750pt}%
\definecolor{currentstroke}{rgb}{0.827451,0.827451,0.827451}%
\pgfsetstrokecolor{currentstroke}%
\pgfsetdash{}{0pt}%
\pgfpathmoveto{\pgfqpoint{9.712490in}{3.135713in}}%
\pgfpathcurveto{\pgfqpoint{9.723541in}{3.135713in}}{\pgfqpoint{9.734140in}{3.140103in}}{\pgfqpoint{9.741953in}{3.147917in}}%
\pgfpathcurveto{\pgfqpoint{9.749767in}{3.155730in}}{\pgfqpoint{9.754157in}{3.166329in}}{\pgfqpoint{9.754157in}{3.177380in}}%
\pgfpathcurveto{\pgfqpoint{9.754157in}{3.188430in}}{\pgfqpoint{9.749767in}{3.199029in}}{\pgfqpoint{9.741953in}{3.206842in}}%
\pgfpathcurveto{\pgfqpoint{9.734140in}{3.214656in}}{\pgfqpoint{9.723541in}{3.219046in}}{\pgfqpoint{9.712490in}{3.219046in}}%
\pgfpathcurveto{\pgfqpoint{9.701440in}{3.219046in}}{\pgfqpoint{9.690841in}{3.214656in}}{\pgfqpoint{9.683028in}{3.206842in}}%
\pgfpathcurveto{\pgfqpoint{9.675214in}{3.199029in}}{\pgfqpoint{9.670824in}{3.188430in}}{\pgfqpoint{9.670824in}{3.177380in}}%
\pgfpathcurveto{\pgfqpoint{9.670824in}{3.166329in}}{\pgfqpoint{9.675214in}{3.155730in}}{\pgfqpoint{9.683028in}{3.147917in}}%
\pgfpathcurveto{\pgfqpoint{9.690841in}{3.140103in}}{\pgfqpoint{9.701440in}{3.135713in}}{\pgfqpoint{9.712490in}{3.135713in}}%
\pgfpathlineto{\pgfqpoint{9.712490in}{3.135713in}}%
\pgfpathclose%
\pgfusepath{stroke}%
\end{pgfscope}%
\begin{pgfscope}%
\pgfpathrectangle{\pgfqpoint{7.394209in}{0.375000in}}{\pgfqpoint{6.356833in}{5.175000in}}%
\pgfusepath{clip}%
\pgfsetbuttcap%
\pgfsetroundjoin%
\pgfsetlinewidth{1.003750pt}%
\definecolor{currentstroke}{rgb}{0.827451,0.827451,0.827451}%
\pgfsetstrokecolor{currentstroke}%
\pgfsetdash{}{0pt}%
\pgfpathmoveto{\pgfqpoint{9.990712in}{2.960645in}}%
\pgfpathcurveto{\pgfqpoint{10.001762in}{2.960645in}}{\pgfqpoint{10.012361in}{2.965035in}}{\pgfqpoint{10.020174in}{2.972849in}}%
\pgfpathcurveto{\pgfqpoint{10.027988in}{2.980662in}}{\pgfqpoint{10.032378in}{2.991261in}}{\pgfqpoint{10.032378in}{3.002311in}}%
\pgfpathcurveto{\pgfqpoint{10.032378in}{3.013361in}}{\pgfqpoint{10.027988in}{3.023961in}}{\pgfqpoint{10.020174in}{3.031774in}}%
\pgfpathcurveto{\pgfqpoint{10.012361in}{3.039588in}}{\pgfqpoint{10.001762in}{3.043978in}}{\pgfqpoint{9.990712in}{3.043978in}}%
\pgfpathcurveto{\pgfqpoint{9.979661in}{3.043978in}}{\pgfqpoint{9.969062in}{3.039588in}}{\pgfqpoint{9.961249in}{3.031774in}}%
\pgfpathcurveto{\pgfqpoint{9.953435in}{3.023961in}}{\pgfqpoint{9.949045in}{3.013361in}}{\pgfqpoint{9.949045in}{3.002311in}}%
\pgfpathcurveto{\pgfqpoint{9.949045in}{2.991261in}}{\pgfqpoint{9.953435in}{2.980662in}}{\pgfqpoint{9.961249in}{2.972849in}}%
\pgfpathcurveto{\pgfqpoint{9.969062in}{2.965035in}}{\pgfqpoint{9.979661in}{2.960645in}}{\pgfqpoint{9.990712in}{2.960645in}}%
\pgfpathlineto{\pgfqpoint{9.990712in}{2.960645in}}%
\pgfpathclose%
\pgfusepath{stroke}%
\end{pgfscope}%
\begin{pgfscope}%
\pgfpathrectangle{\pgfqpoint{7.394209in}{0.375000in}}{\pgfqpoint{6.356833in}{5.175000in}}%
\pgfusepath{clip}%
\pgfsetbuttcap%
\pgfsetroundjoin%
\pgfsetlinewidth{1.003750pt}%
\definecolor{currentstroke}{rgb}{0.827451,0.827451,0.827451}%
\pgfsetstrokecolor{currentstroke}%
\pgfsetdash{}{0pt}%
\pgfpathmoveto{\pgfqpoint{11.047772in}{4.156854in}}%
\pgfpathcurveto{\pgfqpoint{11.058822in}{4.156854in}}{\pgfqpoint{11.069421in}{4.161245in}}{\pgfqpoint{11.077234in}{4.169058in}}%
\pgfpathcurveto{\pgfqpoint{11.085048in}{4.176872in}}{\pgfqpoint{11.089438in}{4.187471in}}{\pgfqpoint{11.089438in}{4.198521in}}%
\pgfpathcurveto{\pgfqpoint{11.089438in}{4.209571in}}{\pgfqpoint{11.085048in}{4.220170in}}{\pgfqpoint{11.077234in}{4.227984in}}%
\pgfpathcurveto{\pgfqpoint{11.069421in}{4.235797in}}{\pgfqpoint{11.058822in}{4.240188in}}{\pgfqpoint{11.047772in}{4.240188in}}%
\pgfpathcurveto{\pgfqpoint{11.036721in}{4.240188in}}{\pgfqpoint{11.026122in}{4.235797in}}{\pgfqpoint{11.018309in}{4.227984in}}%
\pgfpathcurveto{\pgfqpoint{11.010495in}{4.220170in}}{\pgfqpoint{11.006105in}{4.209571in}}{\pgfqpoint{11.006105in}{4.198521in}}%
\pgfpathcurveto{\pgfqpoint{11.006105in}{4.187471in}}{\pgfqpoint{11.010495in}{4.176872in}}{\pgfqpoint{11.018309in}{4.169058in}}%
\pgfpathcurveto{\pgfqpoint{11.026122in}{4.161245in}}{\pgfqpoint{11.036721in}{4.156854in}}{\pgfqpoint{11.047772in}{4.156854in}}%
\pgfpathlineto{\pgfqpoint{11.047772in}{4.156854in}}%
\pgfpathclose%
\pgfusepath{stroke}%
\end{pgfscope}%
\begin{pgfscope}%
\pgfpathrectangle{\pgfqpoint{7.394209in}{0.375000in}}{\pgfqpoint{6.356833in}{5.175000in}}%
\pgfusepath{clip}%
\pgfsetbuttcap%
\pgfsetroundjoin%
\pgfsetlinewidth{1.003750pt}%
\definecolor{currentstroke}{rgb}{0.827451,0.827451,0.827451}%
\pgfsetstrokecolor{currentstroke}%
\pgfsetdash{}{0pt}%
\pgfpathmoveto{\pgfqpoint{7.490947in}{0.918476in}}%
\pgfpathcurveto{\pgfqpoint{7.501998in}{0.918476in}}{\pgfqpoint{7.512597in}{0.922866in}}{\pgfqpoint{7.520410in}{0.930679in}}%
\pgfpathcurveto{\pgfqpoint{7.528224in}{0.938493in}}{\pgfqpoint{7.532614in}{0.949092in}}{\pgfqpoint{7.532614in}{0.960142in}}%
\pgfpathcurveto{\pgfqpoint{7.532614in}{0.971192in}}{\pgfqpoint{7.528224in}{0.981791in}}{\pgfqpoint{7.520410in}{0.989605in}}%
\pgfpathcurveto{\pgfqpoint{7.512597in}{0.997419in}}{\pgfqpoint{7.501998in}{1.001809in}}{\pgfqpoint{7.490947in}{1.001809in}}%
\pgfpathcurveto{\pgfqpoint{7.479897in}{1.001809in}}{\pgfqpoint{7.469298in}{0.997419in}}{\pgfqpoint{7.461485in}{0.989605in}}%
\pgfpathcurveto{\pgfqpoint{7.453671in}{0.981791in}}{\pgfqpoint{7.449281in}{0.971192in}}{\pgfqpoint{7.449281in}{0.960142in}}%
\pgfpathcurveto{\pgfqpoint{7.449281in}{0.949092in}}{\pgfqpoint{7.453671in}{0.938493in}}{\pgfqpoint{7.461485in}{0.930679in}}%
\pgfpathcurveto{\pgfqpoint{7.469298in}{0.922866in}}{\pgfqpoint{7.479897in}{0.918476in}}{\pgfqpoint{7.490947in}{0.918476in}}%
\pgfpathlineto{\pgfqpoint{7.490947in}{0.918476in}}%
\pgfpathclose%
\pgfusepath{stroke}%
\end{pgfscope}%
\begin{pgfscope}%
\pgfpathrectangle{\pgfqpoint{7.394209in}{0.375000in}}{\pgfqpoint{6.356833in}{5.175000in}}%
\pgfusepath{clip}%
\pgfsetbuttcap%
\pgfsetroundjoin%
\pgfsetlinewidth{1.003750pt}%
\definecolor{currentstroke}{rgb}{0.827451,0.827451,0.827451}%
\pgfsetstrokecolor{currentstroke}%
\pgfsetdash{}{0pt}%
\pgfpathmoveto{\pgfqpoint{12.365886in}{5.487257in}}%
\pgfpathcurveto{\pgfqpoint{12.376936in}{5.487257in}}{\pgfqpoint{12.387535in}{5.491647in}}{\pgfqpoint{12.395349in}{5.499461in}}%
\pgfpathcurveto{\pgfqpoint{12.403163in}{5.507274in}}{\pgfqpoint{12.407553in}{5.517873in}}{\pgfqpoint{12.407553in}{5.528924in}}%
\pgfpathcurveto{\pgfqpoint{12.407553in}{5.539974in}}{\pgfqpoint{12.403163in}{5.550573in}}{\pgfqpoint{12.395349in}{5.558386in}}%
\pgfpathcurveto{\pgfqpoint{12.387535in}{5.566200in}}{\pgfqpoint{12.376936in}{5.570590in}}{\pgfqpoint{12.365886in}{5.570590in}}%
\pgfpathcurveto{\pgfqpoint{12.354836in}{5.570590in}}{\pgfqpoint{12.344237in}{5.566200in}}{\pgfqpoint{12.336423in}{5.558386in}}%
\pgfpathcurveto{\pgfqpoint{12.328610in}{5.550573in}}{\pgfqpoint{12.324219in}{5.539974in}}{\pgfqpoint{12.324219in}{5.528924in}}%
\pgfpathcurveto{\pgfqpoint{12.324219in}{5.517873in}}{\pgfqpoint{12.328610in}{5.507274in}}{\pgfqpoint{12.336423in}{5.499461in}}%
\pgfpathcurveto{\pgfqpoint{12.344237in}{5.491647in}}{\pgfqpoint{12.354836in}{5.487257in}}{\pgfqpoint{12.365886in}{5.487257in}}%
\pgfpathlineto{\pgfqpoint{12.365886in}{5.487257in}}%
\pgfpathclose%
\pgfusepath{stroke}%
\end{pgfscope}%
\begin{pgfscope}%
\pgfpathrectangle{\pgfqpoint{7.394209in}{0.375000in}}{\pgfqpoint{6.356833in}{5.175000in}}%
\pgfusepath{clip}%
\pgfsetbuttcap%
\pgfsetroundjoin%
\pgfsetlinewidth{1.003750pt}%
\definecolor{currentstroke}{rgb}{0.827451,0.827451,0.827451}%
\pgfsetstrokecolor{currentstroke}%
\pgfsetdash{}{0pt}%
\pgfpathmoveto{\pgfqpoint{7.427688in}{0.507172in}}%
\pgfpathcurveto{\pgfqpoint{7.438738in}{0.507172in}}{\pgfqpoint{7.449337in}{0.511562in}}{\pgfqpoint{7.457151in}{0.519376in}}%
\pgfpathcurveto{\pgfqpoint{7.464964in}{0.527189in}}{\pgfqpoint{7.469354in}{0.537789in}}{\pgfqpoint{7.469354in}{0.548839in}}%
\pgfpathcurveto{\pgfqpoint{7.469354in}{0.559889in}}{\pgfqpoint{7.464964in}{0.570488in}}{\pgfqpoint{7.457151in}{0.578301in}}%
\pgfpathcurveto{\pgfqpoint{7.449337in}{0.586115in}}{\pgfqpoint{7.438738in}{0.590505in}}{\pgfqpoint{7.427688in}{0.590505in}}%
\pgfpathcurveto{\pgfqpoint{7.416638in}{0.590505in}}{\pgfqpoint{7.406039in}{0.586115in}}{\pgfqpoint{7.398225in}{0.578301in}}%
\pgfpathcurveto{\pgfqpoint{7.390411in}{0.570488in}}{\pgfqpoint{7.386021in}{0.559889in}}{\pgfqpoint{7.386021in}{0.548839in}}%
\pgfpathcurveto{\pgfqpoint{7.386021in}{0.537789in}}{\pgfqpoint{7.390411in}{0.527189in}}{\pgfqpoint{7.398225in}{0.519376in}}%
\pgfpathcurveto{\pgfqpoint{7.406039in}{0.511562in}}{\pgfqpoint{7.416638in}{0.507172in}}{\pgfqpoint{7.427688in}{0.507172in}}%
\pgfpathlineto{\pgfqpoint{7.427688in}{0.507172in}}%
\pgfpathclose%
\pgfusepath{stroke}%
\end{pgfscope}%
\begin{pgfscope}%
\pgfpathrectangle{\pgfqpoint{7.394209in}{0.375000in}}{\pgfqpoint{6.356833in}{5.175000in}}%
\pgfusepath{clip}%
\pgfsetbuttcap%
\pgfsetroundjoin%
\pgfsetlinewidth{1.003750pt}%
\definecolor{currentstroke}{rgb}{0.827451,0.827451,0.827451}%
\pgfsetstrokecolor{currentstroke}%
\pgfsetdash{}{0pt}%
\pgfpathmoveto{\pgfqpoint{13.090484in}{5.470421in}}%
\pgfpathcurveto{\pgfqpoint{13.101534in}{5.470421in}}{\pgfqpoint{13.112133in}{5.474812in}}{\pgfqpoint{13.119946in}{5.482625in}}%
\pgfpathcurveto{\pgfqpoint{13.127760in}{5.490439in}}{\pgfqpoint{13.132150in}{5.501038in}}{\pgfqpoint{13.132150in}{5.512088in}}%
\pgfpathcurveto{\pgfqpoint{13.132150in}{5.523138in}}{\pgfqpoint{13.127760in}{5.533737in}}{\pgfqpoint{13.119946in}{5.541551in}}%
\pgfpathcurveto{\pgfqpoint{13.112133in}{5.549364in}}{\pgfqpoint{13.101534in}{5.553755in}}{\pgfqpoint{13.090484in}{5.553755in}}%
\pgfpathcurveto{\pgfqpoint{13.079434in}{5.553755in}}{\pgfqpoint{13.068834in}{5.549364in}}{\pgfqpoint{13.061021in}{5.541551in}}%
\pgfpathcurveto{\pgfqpoint{13.053207in}{5.533737in}}{\pgfqpoint{13.048817in}{5.523138in}}{\pgfqpoint{13.048817in}{5.512088in}}%
\pgfpathcurveto{\pgfqpoint{13.048817in}{5.501038in}}{\pgfqpoint{13.053207in}{5.490439in}}{\pgfqpoint{13.061021in}{5.482625in}}%
\pgfpathcurveto{\pgfqpoint{13.068834in}{5.474812in}}{\pgfqpoint{13.079434in}{5.470421in}}{\pgfqpoint{13.090484in}{5.470421in}}%
\pgfpathlineto{\pgfqpoint{13.090484in}{5.470421in}}%
\pgfpathclose%
\pgfusepath{stroke}%
\end{pgfscope}%
\begin{pgfscope}%
\pgfpathrectangle{\pgfqpoint{7.394209in}{0.375000in}}{\pgfqpoint{6.356833in}{5.175000in}}%
\pgfusepath{clip}%
\pgfsetbuttcap%
\pgfsetroundjoin%
\pgfsetlinewidth{1.003750pt}%
\definecolor{currentstroke}{rgb}{0.827451,0.827451,0.827451}%
\pgfsetstrokecolor{currentstroke}%
\pgfsetdash{}{0pt}%
\pgfpathmoveto{\pgfqpoint{12.087319in}{5.359475in}}%
\pgfpathcurveto{\pgfqpoint{12.098369in}{5.359475in}}{\pgfqpoint{12.108968in}{5.363866in}}{\pgfqpoint{12.116782in}{5.371679in}}%
\pgfpathcurveto{\pgfqpoint{12.124595in}{5.379493in}}{\pgfqpoint{12.128986in}{5.390092in}}{\pgfqpoint{12.128986in}{5.401142in}}%
\pgfpathcurveto{\pgfqpoint{12.128986in}{5.412192in}}{\pgfqpoint{12.124595in}{5.422791in}}{\pgfqpoint{12.116782in}{5.430605in}}%
\pgfpathcurveto{\pgfqpoint{12.108968in}{5.438418in}}{\pgfqpoint{12.098369in}{5.442809in}}{\pgfqpoint{12.087319in}{5.442809in}}%
\pgfpathcurveto{\pgfqpoint{12.076269in}{5.442809in}}{\pgfqpoint{12.065670in}{5.438418in}}{\pgfqpoint{12.057856in}{5.430605in}}%
\pgfpathcurveto{\pgfqpoint{12.050043in}{5.422791in}}{\pgfqpoint{12.045652in}{5.412192in}}{\pgfqpoint{12.045652in}{5.401142in}}%
\pgfpathcurveto{\pgfqpoint{12.045652in}{5.390092in}}{\pgfqpoint{12.050043in}{5.379493in}}{\pgfqpoint{12.057856in}{5.371679in}}%
\pgfpathcurveto{\pgfqpoint{12.065670in}{5.363866in}}{\pgfqpoint{12.076269in}{5.359475in}}{\pgfqpoint{12.087319in}{5.359475in}}%
\pgfpathlineto{\pgfqpoint{12.087319in}{5.359475in}}%
\pgfpathclose%
\pgfusepath{stroke}%
\end{pgfscope}%
\begin{pgfscope}%
\pgfpathrectangle{\pgfqpoint{7.394209in}{0.375000in}}{\pgfqpoint{6.356833in}{5.175000in}}%
\pgfusepath{clip}%
\pgfsetbuttcap%
\pgfsetroundjoin%
\pgfsetlinewidth{1.003750pt}%
\definecolor{currentstroke}{rgb}{0.827451,0.827451,0.827451}%
\pgfsetstrokecolor{currentstroke}%
\pgfsetdash{}{0pt}%
\pgfpathmoveto{\pgfqpoint{11.014334in}{4.553691in}}%
\pgfpathcurveto{\pgfqpoint{11.025384in}{4.553691in}}{\pgfqpoint{11.035983in}{4.558081in}}{\pgfqpoint{11.043797in}{4.565895in}}%
\pgfpathcurveto{\pgfqpoint{11.051611in}{4.573709in}}{\pgfqpoint{11.056001in}{4.584308in}}{\pgfqpoint{11.056001in}{4.595358in}}%
\pgfpathcurveto{\pgfqpoint{11.056001in}{4.606408in}}{\pgfqpoint{11.051611in}{4.617007in}}{\pgfqpoint{11.043797in}{4.624821in}}%
\pgfpathcurveto{\pgfqpoint{11.035983in}{4.632634in}}{\pgfqpoint{11.025384in}{4.637024in}}{\pgfqpoint{11.014334in}{4.637024in}}%
\pgfpathcurveto{\pgfqpoint{11.003284in}{4.637024in}}{\pgfqpoint{10.992685in}{4.632634in}}{\pgfqpoint{10.984871in}{4.624821in}}%
\pgfpathcurveto{\pgfqpoint{10.977058in}{4.617007in}}{\pgfqpoint{10.972668in}{4.606408in}}{\pgfqpoint{10.972668in}{4.595358in}}%
\pgfpathcurveto{\pgfqpoint{10.972668in}{4.584308in}}{\pgfqpoint{10.977058in}{4.573709in}}{\pgfqpoint{10.984871in}{4.565895in}}%
\pgfpathcurveto{\pgfqpoint{10.992685in}{4.558081in}}{\pgfqpoint{11.003284in}{4.553691in}}{\pgfqpoint{11.014334in}{4.553691in}}%
\pgfpathlineto{\pgfqpoint{11.014334in}{4.553691in}}%
\pgfpathclose%
\pgfusepath{stroke}%
\end{pgfscope}%
\begin{pgfscope}%
\pgfpathrectangle{\pgfqpoint{7.394209in}{0.375000in}}{\pgfqpoint{6.356833in}{5.175000in}}%
\pgfusepath{clip}%
\pgfsetbuttcap%
\pgfsetroundjoin%
\pgfsetlinewidth{1.003750pt}%
\definecolor{currentstroke}{rgb}{0.827451,0.827451,0.827451}%
\pgfsetstrokecolor{currentstroke}%
\pgfsetdash{}{0pt}%
\pgfpathmoveto{\pgfqpoint{13.045446in}{5.488592in}}%
\pgfpathcurveto{\pgfqpoint{13.056496in}{5.488592in}}{\pgfqpoint{13.067095in}{5.492982in}}{\pgfqpoint{13.074909in}{5.500796in}}%
\pgfpathcurveto{\pgfqpoint{13.082722in}{5.508609in}}{\pgfqpoint{13.087113in}{5.519208in}}{\pgfqpoint{13.087113in}{5.530258in}}%
\pgfpathcurveto{\pgfqpoint{13.087113in}{5.541308in}}{\pgfqpoint{13.082722in}{5.551907in}}{\pgfqpoint{13.074909in}{5.559721in}}%
\pgfpathcurveto{\pgfqpoint{13.067095in}{5.567535in}}{\pgfqpoint{13.056496in}{5.571925in}}{\pgfqpoint{13.045446in}{5.571925in}}%
\pgfpathcurveto{\pgfqpoint{13.034396in}{5.571925in}}{\pgfqpoint{13.023797in}{5.567535in}}{\pgfqpoint{13.015983in}{5.559721in}}%
\pgfpathcurveto{\pgfqpoint{13.008170in}{5.551907in}}{\pgfqpoint{13.003779in}{5.541308in}}{\pgfqpoint{13.003779in}{5.530258in}}%
\pgfpathcurveto{\pgfqpoint{13.003779in}{5.519208in}}{\pgfqpoint{13.008170in}{5.508609in}}{\pgfqpoint{13.015983in}{5.500796in}}%
\pgfpathcurveto{\pgfqpoint{13.023797in}{5.492982in}}{\pgfqpoint{13.034396in}{5.488592in}}{\pgfqpoint{13.045446in}{5.488592in}}%
\pgfpathlineto{\pgfqpoint{13.045446in}{5.488592in}}%
\pgfpathclose%
\pgfusepath{stroke}%
\end{pgfscope}%
\begin{pgfscope}%
\pgfpathrectangle{\pgfqpoint{7.394209in}{0.375000in}}{\pgfqpoint{6.356833in}{5.175000in}}%
\pgfusepath{clip}%
\pgfsetbuttcap%
\pgfsetroundjoin%
\pgfsetlinewidth{1.003750pt}%
\definecolor{currentstroke}{rgb}{0.827451,0.827451,0.827451}%
\pgfsetstrokecolor{currentstroke}%
\pgfsetdash{}{0pt}%
\pgfpathmoveto{\pgfqpoint{9.563820in}{3.686579in}}%
\pgfpathcurveto{\pgfqpoint{9.574870in}{3.686579in}}{\pgfqpoint{9.585469in}{3.690969in}}{\pgfqpoint{9.593283in}{3.698783in}}%
\pgfpathcurveto{\pgfqpoint{9.601097in}{3.706597in}}{\pgfqpoint{9.605487in}{3.717196in}}{\pgfqpoint{9.605487in}{3.728246in}}%
\pgfpathcurveto{\pgfqpoint{9.605487in}{3.739296in}}{\pgfqpoint{9.601097in}{3.749895in}}{\pgfqpoint{9.593283in}{3.757709in}}%
\pgfpathcurveto{\pgfqpoint{9.585469in}{3.765522in}}{\pgfqpoint{9.574870in}{3.769912in}}{\pgfqpoint{9.563820in}{3.769912in}}%
\pgfpathcurveto{\pgfqpoint{9.552770in}{3.769912in}}{\pgfqpoint{9.542171in}{3.765522in}}{\pgfqpoint{9.534358in}{3.757709in}}%
\pgfpathcurveto{\pgfqpoint{9.526544in}{3.749895in}}{\pgfqpoint{9.522154in}{3.739296in}}{\pgfqpoint{9.522154in}{3.728246in}}%
\pgfpathcurveto{\pgfqpoint{9.522154in}{3.717196in}}{\pgfqpoint{9.526544in}{3.706597in}}{\pgfqpoint{9.534358in}{3.698783in}}%
\pgfpathcurveto{\pgfqpoint{9.542171in}{3.690969in}}{\pgfqpoint{9.552770in}{3.686579in}}{\pgfqpoint{9.563820in}{3.686579in}}%
\pgfpathlineto{\pgfqpoint{9.563820in}{3.686579in}}%
\pgfpathclose%
\pgfusepath{stroke}%
\end{pgfscope}%
\begin{pgfscope}%
\pgfpathrectangle{\pgfqpoint{7.394209in}{0.375000in}}{\pgfqpoint{6.356833in}{5.175000in}}%
\pgfusepath{clip}%
\pgfsetbuttcap%
\pgfsetroundjoin%
\pgfsetlinewidth{1.003750pt}%
\definecolor{currentstroke}{rgb}{0.827451,0.827451,0.827451}%
\pgfsetstrokecolor{currentstroke}%
\pgfsetdash{}{0pt}%
\pgfpathmoveto{\pgfqpoint{7.915117in}{1.592114in}}%
\pgfpathcurveto{\pgfqpoint{7.926167in}{1.592114in}}{\pgfqpoint{7.936766in}{1.596505in}}{\pgfqpoint{7.944579in}{1.604318in}}%
\pgfpathcurveto{\pgfqpoint{7.952393in}{1.612132in}}{\pgfqpoint{7.956783in}{1.622731in}}{\pgfqpoint{7.956783in}{1.633781in}}%
\pgfpathcurveto{\pgfqpoint{7.956783in}{1.644831in}}{\pgfqpoint{7.952393in}{1.655430in}}{\pgfqpoint{7.944579in}{1.663244in}}%
\pgfpathcurveto{\pgfqpoint{7.936766in}{1.671058in}}{\pgfqpoint{7.926167in}{1.675448in}}{\pgfqpoint{7.915117in}{1.675448in}}%
\pgfpathcurveto{\pgfqpoint{7.904067in}{1.675448in}}{\pgfqpoint{7.893468in}{1.671058in}}{\pgfqpoint{7.885654in}{1.663244in}}%
\pgfpathcurveto{\pgfqpoint{7.877840in}{1.655430in}}{\pgfqpoint{7.873450in}{1.644831in}}{\pgfqpoint{7.873450in}{1.633781in}}%
\pgfpathcurveto{\pgfqpoint{7.873450in}{1.622731in}}{\pgfqpoint{7.877840in}{1.612132in}}{\pgfqpoint{7.885654in}{1.604318in}}%
\pgfpathcurveto{\pgfqpoint{7.893468in}{1.596505in}}{\pgfqpoint{7.904067in}{1.592114in}}{\pgfqpoint{7.915117in}{1.592114in}}%
\pgfpathlineto{\pgfqpoint{7.915117in}{1.592114in}}%
\pgfpathclose%
\pgfusepath{stroke}%
\end{pgfscope}%
\begin{pgfscope}%
\pgfpathrectangle{\pgfqpoint{7.394209in}{0.375000in}}{\pgfqpoint{6.356833in}{5.175000in}}%
\pgfusepath{clip}%
\pgfsetbuttcap%
\pgfsetroundjoin%
\pgfsetlinewidth{1.003750pt}%
\definecolor{currentstroke}{rgb}{0.827451,0.827451,0.827451}%
\pgfsetstrokecolor{currentstroke}%
\pgfsetdash{}{0pt}%
\pgfpathmoveto{\pgfqpoint{7.424176in}{0.670108in}}%
\pgfpathcurveto{\pgfqpoint{7.435226in}{0.670108in}}{\pgfqpoint{7.445825in}{0.674498in}}{\pgfqpoint{7.453639in}{0.682311in}}%
\pgfpathcurveto{\pgfqpoint{7.461453in}{0.690125in}}{\pgfqpoint{7.465843in}{0.700724in}}{\pgfqpoint{7.465843in}{0.711774in}}%
\pgfpathcurveto{\pgfqpoint{7.465843in}{0.722824in}}{\pgfqpoint{7.461453in}{0.733423in}}{\pgfqpoint{7.453639in}{0.741237in}}%
\pgfpathcurveto{\pgfqpoint{7.445825in}{0.749051in}}{\pgfqpoint{7.435226in}{0.753441in}}{\pgfqpoint{7.424176in}{0.753441in}}%
\pgfpathcurveto{\pgfqpoint{7.413126in}{0.753441in}}{\pgfqpoint{7.402527in}{0.749051in}}{\pgfqpoint{7.394713in}{0.741237in}}%
\pgfpathcurveto{\pgfqpoint{7.386900in}{0.733423in}}{\pgfqpoint{7.382509in}{0.722824in}}{\pgfqpoint{7.382509in}{0.711774in}}%
\pgfpathcurveto{\pgfqpoint{7.382509in}{0.700724in}}{\pgfqpoint{7.386900in}{0.690125in}}{\pgfqpoint{7.394713in}{0.682311in}}%
\pgfpathcurveto{\pgfqpoint{7.402527in}{0.674498in}}{\pgfqpoint{7.413126in}{0.670108in}}{\pgfqpoint{7.424176in}{0.670108in}}%
\pgfpathlineto{\pgfqpoint{7.424176in}{0.670108in}}%
\pgfpathclose%
\pgfusepath{stroke}%
\end{pgfscope}%
\begin{pgfscope}%
\pgfpathrectangle{\pgfqpoint{7.394209in}{0.375000in}}{\pgfqpoint{6.356833in}{5.175000in}}%
\pgfusepath{clip}%
\pgfsetbuttcap%
\pgfsetroundjoin%
\pgfsetlinewidth{1.003750pt}%
\definecolor{currentstroke}{rgb}{0.827451,0.827451,0.827451}%
\pgfsetstrokecolor{currentstroke}%
\pgfsetdash{}{0pt}%
\pgfpathmoveto{\pgfqpoint{10.496452in}{4.013703in}}%
\pgfpathcurveto{\pgfqpoint{10.507502in}{4.013703in}}{\pgfqpoint{10.518101in}{4.018093in}}{\pgfqpoint{10.525915in}{4.025907in}}%
\pgfpathcurveto{\pgfqpoint{10.533728in}{4.033721in}}{\pgfqpoint{10.538118in}{4.044320in}}{\pgfqpoint{10.538118in}{4.055370in}}%
\pgfpathcurveto{\pgfqpoint{10.538118in}{4.066420in}}{\pgfqpoint{10.533728in}{4.077019in}}{\pgfqpoint{10.525915in}{4.084833in}}%
\pgfpathcurveto{\pgfqpoint{10.518101in}{4.092646in}}{\pgfqpoint{10.507502in}{4.097037in}}{\pgfqpoint{10.496452in}{4.097037in}}%
\pgfpathcurveto{\pgfqpoint{10.485402in}{4.097037in}}{\pgfqpoint{10.474803in}{4.092646in}}{\pgfqpoint{10.466989in}{4.084833in}}%
\pgfpathcurveto{\pgfqpoint{10.459175in}{4.077019in}}{\pgfqpoint{10.454785in}{4.066420in}}{\pgfqpoint{10.454785in}{4.055370in}}%
\pgfpathcurveto{\pgfqpoint{10.454785in}{4.044320in}}{\pgfqpoint{10.459175in}{4.033721in}}{\pgfqpoint{10.466989in}{4.025907in}}%
\pgfpathcurveto{\pgfqpoint{10.474803in}{4.018093in}}{\pgfqpoint{10.485402in}{4.013703in}}{\pgfqpoint{10.496452in}{4.013703in}}%
\pgfpathlineto{\pgfqpoint{10.496452in}{4.013703in}}%
\pgfpathclose%
\pgfusepath{stroke}%
\end{pgfscope}%
\begin{pgfscope}%
\pgfpathrectangle{\pgfqpoint{7.394209in}{0.375000in}}{\pgfqpoint{6.356833in}{5.175000in}}%
\pgfusepath{clip}%
\pgfsetbuttcap%
\pgfsetroundjoin%
\pgfsetlinewidth{1.003750pt}%
\definecolor{currentstroke}{rgb}{0.827451,0.827451,0.827451}%
\pgfsetstrokecolor{currentstroke}%
\pgfsetdash{}{0pt}%
\pgfpathmoveto{\pgfqpoint{11.715424in}{5.252273in}}%
\pgfpathcurveto{\pgfqpoint{11.726474in}{5.252273in}}{\pgfqpoint{11.737073in}{5.256663in}}{\pgfqpoint{11.744887in}{5.264476in}}%
\pgfpathcurveto{\pgfqpoint{11.752700in}{5.272290in}}{\pgfqpoint{11.757090in}{5.282889in}}{\pgfqpoint{11.757090in}{5.293939in}}%
\pgfpathcurveto{\pgfqpoint{11.757090in}{5.304989in}}{\pgfqpoint{11.752700in}{5.315588in}}{\pgfqpoint{11.744887in}{5.323402in}}%
\pgfpathcurveto{\pgfqpoint{11.737073in}{5.331216in}}{\pgfqpoint{11.726474in}{5.335606in}}{\pgfqpoint{11.715424in}{5.335606in}}%
\pgfpathcurveto{\pgfqpoint{11.704374in}{5.335606in}}{\pgfqpoint{11.693775in}{5.331216in}}{\pgfqpoint{11.685961in}{5.323402in}}%
\pgfpathcurveto{\pgfqpoint{11.678147in}{5.315588in}}{\pgfqpoint{11.673757in}{5.304989in}}{\pgfqpoint{11.673757in}{5.293939in}}%
\pgfpathcurveto{\pgfqpoint{11.673757in}{5.282889in}}{\pgfqpoint{11.678147in}{5.272290in}}{\pgfqpoint{11.685961in}{5.264476in}}%
\pgfpathcurveto{\pgfqpoint{11.693775in}{5.256663in}}{\pgfqpoint{11.704374in}{5.252273in}}{\pgfqpoint{11.715424in}{5.252273in}}%
\pgfpathlineto{\pgfqpoint{11.715424in}{5.252273in}}%
\pgfpathclose%
\pgfusepath{stroke}%
\end{pgfscope}%
\begin{pgfscope}%
\pgfpathrectangle{\pgfqpoint{7.394209in}{0.375000in}}{\pgfqpoint{6.356833in}{5.175000in}}%
\pgfusepath{clip}%
\pgfsetbuttcap%
\pgfsetroundjoin%
\pgfsetlinewidth{1.003750pt}%
\definecolor{currentstroke}{rgb}{0.827451,0.827451,0.827451}%
\pgfsetstrokecolor{currentstroke}%
\pgfsetdash{}{0pt}%
\pgfpathmoveto{\pgfqpoint{10.710680in}{5.411206in}}%
\pgfpathcurveto{\pgfqpoint{10.721730in}{5.411206in}}{\pgfqpoint{10.732329in}{5.415597in}}{\pgfqpoint{10.740143in}{5.423410in}}%
\pgfpathcurveto{\pgfqpoint{10.747956in}{5.431224in}}{\pgfqpoint{10.752347in}{5.441823in}}{\pgfqpoint{10.752347in}{5.452873in}}%
\pgfpathcurveto{\pgfqpoint{10.752347in}{5.463923in}}{\pgfqpoint{10.747956in}{5.474522in}}{\pgfqpoint{10.740143in}{5.482336in}}%
\pgfpathcurveto{\pgfqpoint{10.732329in}{5.490149in}}{\pgfqpoint{10.721730in}{5.494540in}}{\pgfqpoint{10.710680in}{5.494540in}}%
\pgfpathcurveto{\pgfqpoint{10.699630in}{5.494540in}}{\pgfqpoint{10.689031in}{5.490149in}}{\pgfqpoint{10.681217in}{5.482336in}}%
\pgfpathcurveto{\pgfqpoint{10.673403in}{5.474522in}}{\pgfqpoint{10.669013in}{5.463923in}}{\pgfqpoint{10.669013in}{5.452873in}}%
\pgfpathcurveto{\pgfqpoint{10.669013in}{5.441823in}}{\pgfqpoint{10.673403in}{5.431224in}}{\pgfqpoint{10.681217in}{5.423410in}}%
\pgfpathcurveto{\pgfqpoint{10.689031in}{5.415597in}}{\pgfqpoint{10.699630in}{5.411206in}}{\pgfqpoint{10.710680in}{5.411206in}}%
\pgfpathlineto{\pgfqpoint{10.710680in}{5.411206in}}%
\pgfpathclose%
\pgfusepath{stroke}%
\end{pgfscope}%
\begin{pgfscope}%
\pgfpathrectangle{\pgfqpoint{7.394209in}{0.375000in}}{\pgfqpoint{6.356833in}{5.175000in}}%
\pgfusepath{clip}%
\pgfsetbuttcap%
\pgfsetroundjoin%
\pgfsetlinewidth{1.003750pt}%
\definecolor{currentstroke}{rgb}{0.827451,0.827451,0.827451}%
\pgfsetstrokecolor{currentstroke}%
\pgfsetdash{}{0pt}%
\pgfpathmoveto{\pgfqpoint{12.451528in}{5.397663in}}%
\pgfpathcurveto{\pgfqpoint{12.462579in}{5.397663in}}{\pgfqpoint{12.473178in}{5.402053in}}{\pgfqpoint{12.480991in}{5.409867in}}%
\pgfpathcurveto{\pgfqpoint{12.488805in}{5.417681in}}{\pgfqpoint{12.493195in}{5.428280in}}{\pgfqpoint{12.493195in}{5.439330in}}%
\pgfpathcurveto{\pgfqpoint{12.493195in}{5.450380in}}{\pgfqpoint{12.488805in}{5.460979in}}{\pgfqpoint{12.480991in}{5.468793in}}%
\pgfpathcurveto{\pgfqpoint{12.473178in}{5.476606in}}{\pgfqpoint{12.462579in}{5.480997in}}{\pgfqpoint{12.451528in}{5.480997in}}%
\pgfpathcurveto{\pgfqpoint{12.440478in}{5.480997in}}{\pgfqpoint{12.429879in}{5.476606in}}{\pgfqpoint{12.422066in}{5.468793in}}%
\pgfpathcurveto{\pgfqpoint{12.414252in}{5.460979in}}{\pgfqpoint{12.409862in}{5.450380in}}{\pgfqpoint{12.409862in}{5.439330in}}%
\pgfpathcurveto{\pgfqpoint{12.409862in}{5.428280in}}{\pgfqpoint{12.414252in}{5.417681in}}{\pgfqpoint{12.422066in}{5.409867in}}%
\pgfpathcurveto{\pgfqpoint{12.429879in}{5.402053in}}{\pgfqpoint{12.440478in}{5.397663in}}{\pgfqpoint{12.451528in}{5.397663in}}%
\pgfpathlineto{\pgfqpoint{12.451528in}{5.397663in}}%
\pgfpathclose%
\pgfusepath{stroke}%
\end{pgfscope}%
\begin{pgfscope}%
\pgfpathrectangle{\pgfqpoint{7.394209in}{0.375000in}}{\pgfqpoint{6.356833in}{5.175000in}}%
\pgfusepath{clip}%
\pgfsetbuttcap%
\pgfsetroundjoin%
\pgfsetlinewidth{1.003750pt}%
\definecolor{currentstroke}{rgb}{0.827451,0.827451,0.827451}%
\pgfsetstrokecolor{currentstroke}%
\pgfsetdash{}{0pt}%
\pgfpathmoveto{\pgfqpoint{10.710680in}{5.507216in}}%
\pgfpathcurveto{\pgfqpoint{10.721730in}{5.507216in}}{\pgfqpoint{10.732329in}{5.511606in}}{\pgfqpoint{10.740143in}{5.519419in}}%
\pgfpathcurveto{\pgfqpoint{10.747956in}{5.527233in}}{\pgfqpoint{10.752347in}{5.537832in}}{\pgfqpoint{10.752347in}{5.548882in}}%
\pgfpathcurveto{\pgfqpoint{10.752347in}{5.559932in}}{\pgfqpoint{10.747956in}{5.570531in}}{\pgfqpoint{10.740143in}{5.578345in}}%
\pgfpathcurveto{\pgfqpoint{10.732329in}{5.586159in}}{\pgfqpoint{10.721730in}{5.590549in}}{\pgfqpoint{10.710680in}{5.590549in}}%
\pgfpathcurveto{\pgfqpoint{10.699630in}{5.590549in}}{\pgfqpoint{10.689031in}{5.586159in}}{\pgfqpoint{10.681217in}{5.578345in}}%
\pgfpathcurveto{\pgfqpoint{10.673403in}{5.570531in}}{\pgfqpoint{10.669013in}{5.559932in}}{\pgfqpoint{10.669013in}{5.548882in}}%
\pgfpathcurveto{\pgfqpoint{10.669013in}{5.537832in}}{\pgfqpoint{10.673403in}{5.527233in}}{\pgfqpoint{10.681217in}{5.519419in}}%
\pgfpathcurveto{\pgfqpoint{10.689031in}{5.511606in}}{\pgfqpoint{10.699630in}{5.507216in}}{\pgfqpoint{10.710680in}{5.507216in}}%
\pgfpathlineto{\pgfqpoint{10.710680in}{5.507216in}}%
\pgfpathclose%
\pgfusepath{stroke}%
\end{pgfscope}%
\begin{pgfscope}%
\pgfpathrectangle{\pgfqpoint{7.394209in}{0.375000in}}{\pgfqpoint{6.356833in}{5.175000in}}%
\pgfusepath{clip}%
\pgfsetbuttcap%
\pgfsetroundjoin%
\pgfsetlinewidth{1.003750pt}%
\definecolor{currentstroke}{rgb}{0.827451,0.827451,0.827451}%
\pgfsetstrokecolor{currentstroke}%
\pgfsetdash{}{0pt}%
\pgfpathmoveto{\pgfqpoint{7.402870in}{0.497148in}}%
\pgfpathcurveto{\pgfqpoint{7.413920in}{0.497148in}}{\pgfqpoint{7.424519in}{0.501538in}}{\pgfqpoint{7.432332in}{0.509351in}}%
\pgfpathcurveto{\pgfqpoint{7.440146in}{0.517165in}}{\pgfqpoint{7.444536in}{0.527764in}}{\pgfqpoint{7.444536in}{0.538814in}}%
\pgfpathcurveto{\pgfqpoint{7.444536in}{0.549864in}}{\pgfqpoint{7.440146in}{0.560463in}}{\pgfqpoint{7.432332in}{0.568277in}}%
\pgfpathcurveto{\pgfqpoint{7.424519in}{0.576091in}}{\pgfqpoint{7.413920in}{0.580481in}}{\pgfqpoint{7.402870in}{0.580481in}}%
\pgfpathcurveto{\pgfqpoint{7.391819in}{0.580481in}}{\pgfqpoint{7.381220in}{0.576091in}}{\pgfqpoint{7.373407in}{0.568277in}}%
\pgfpathcurveto{\pgfqpoint{7.365593in}{0.560463in}}{\pgfqpoint{7.361203in}{0.549864in}}{\pgfqpoint{7.361203in}{0.538814in}}%
\pgfpathcurveto{\pgfqpoint{7.361203in}{0.527764in}}{\pgfqpoint{7.365593in}{0.517165in}}{\pgfqpoint{7.373407in}{0.509351in}}%
\pgfpathcurveto{\pgfqpoint{7.381220in}{0.501538in}}{\pgfqpoint{7.391819in}{0.497148in}}{\pgfqpoint{7.402870in}{0.497148in}}%
\pgfpathlineto{\pgfqpoint{7.402870in}{0.497148in}}%
\pgfpathclose%
\pgfusepath{stroke}%
\end{pgfscope}%
\begin{pgfscope}%
\pgfpathrectangle{\pgfqpoint{7.394209in}{0.375000in}}{\pgfqpoint{6.356833in}{5.175000in}}%
\pgfusepath{clip}%
\pgfsetbuttcap%
\pgfsetroundjoin%
\pgfsetlinewidth{1.003750pt}%
\definecolor{currentstroke}{rgb}{0.827451,0.827451,0.827451}%
\pgfsetstrokecolor{currentstroke}%
\pgfsetdash{}{0pt}%
\pgfpathmoveto{\pgfqpoint{7.900091in}{1.592114in}}%
\pgfpathcurveto{\pgfqpoint{7.911141in}{1.592114in}}{\pgfqpoint{7.921740in}{1.596505in}}{\pgfqpoint{7.929553in}{1.604318in}}%
\pgfpathcurveto{\pgfqpoint{7.937367in}{1.612132in}}{\pgfqpoint{7.941757in}{1.622731in}}{\pgfqpoint{7.941757in}{1.633781in}}%
\pgfpathcurveto{\pgfqpoint{7.941757in}{1.644831in}}{\pgfqpoint{7.937367in}{1.655430in}}{\pgfqpoint{7.929553in}{1.663244in}}%
\pgfpathcurveto{\pgfqpoint{7.921740in}{1.671058in}}{\pgfqpoint{7.911141in}{1.675448in}}{\pgfqpoint{7.900091in}{1.675448in}}%
\pgfpathcurveto{\pgfqpoint{7.889040in}{1.675448in}}{\pgfqpoint{7.878441in}{1.671058in}}{\pgfqpoint{7.870628in}{1.663244in}}%
\pgfpathcurveto{\pgfqpoint{7.862814in}{1.655430in}}{\pgfqpoint{7.858424in}{1.644831in}}{\pgfqpoint{7.858424in}{1.633781in}}%
\pgfpathcurveto{\pgfqpoint{7.858424in}{1.622731in}}{\pgfqpoint{7.862814in}{1.612132in}}{\pgfqpoint{7.870628in}{1.604318in}}%
\pgfpathcurveto{\pgfqpoint{7.878441in}{1.596505in}}{\pgfqpoint{7.889040in}{1.592114in}}{\pgfqpoint{7.900091in}{1.592114in}}%
\pgfpathlineto{\pgfqpoint{7.900091in}{1.592114in}}%
\pgfpathclose%
\pgfusepath{stroke}%
\end{pgfscope}%
\begin{pgfscope}%
\pgfpathrectangle{\pgfqpoint{7.394209in}{0.375000in}}{\pgfqpoint{6.356833in}{5.175000in}}%
\pgfusepath{clip}%
\pgfsetbuttcap%
\pgfsetroundjoin%
\pgfsetlinewidth{1.003750pt}%
\definecolor{currentstroke}{rgb}{0.827451,0.827451,0.827451}%
\pgfsetstrokecolor{currentstroke}%
\pgfsetdash{}{0pt}%
\pgfpathmoveto{\pgfqpoint{7.730811in}{1.851997in}}%
\pgfpathcurveto{\pgfqpoint{7.741862in}{1.851997in}}{\pgfqpoint{7.752461in}{1.856387in}}{\pgfqpoint{7.760274in}{1.864201in}}%
\pgfpathcurveto{\pgfqpoint{7.768088in}{1.872015in}}{\pgfqpoint{7.772478in}{1.882614in}}{\pgfqpoint{7.772478in}{1.893664in}}%
\pgfpathcurveto{\pgfqpoint{7.772478in}{1.904714in}}{\pgfqpoint{7.768088in}{1.915313in}}{\pgfqpoint{7.760274in}{1.923127in}}%
\pgfpathcurveto{\pgfqpoint{7.752461in}{1.930940in}}{\pgfqpoint{7.741862in}{1.935330in}}{\pgfqpoint{7.730811in}{1.935330in}}%
\pgfpathcurveto{\pgfqpoint{7.719761in}{1.935330in}}{\pgfqpoint{7.709162in}{1.930940in}}{\pgfqpoint{7.701349in}{1.923127in}}%
\pgfpathcurveto{\pgfqpoint{7.693535in}{1.915313in}}{\pgfqpoint{7.689145in}{1.904714in}}{\pgfqpoint{7.689145in}{1.893664in}}%
\pgfpathcurveto{\pgfqpoint{7.689145in}{1.882614in}}{\pgfqpoint{7.693535in}{1.872015in}}{\pgfqpoint{7.701349in}{1.864201in}}%
\pgfpathcurveto{\pgfqpoint{7.709162in}{1.856387in}}{\pgfqpoint{7.719761in}{1.851997in}}{\pgfqpoint{7.730811in}{1.851997in}}%
\pgfpathlineto{\pgfqpoint{7.730811in}{1.851997in}}%
\pgfpathclose%
\pgfusepath{stroke}%
\end{pgfscope}%
\begin{pgfscope}%
\pgfpathrectangle{\pgfqpoint{7.394209in}{0.375000in}}{\pgfqpoint{6.356833in}{5.175000in}}%
\pgfusepath{clip}%
\pgfsetbuttcap%
\pgfsetroundjoin%
\pgfsetlinewidth{1.003750pt}%
\definecolor{currentstroke}{rgb}{0.827451,0.827451,0.827451}%
\pgfsetstrokecolor{currentstroke}%
\pgfsetdash{}{0pt}%
\pgfpathmoveto{\pgfqpoint{7.900091in}{1.573324in}}%
\pgfpathcurveto{\pgfqpoint{7.911141in}{1.573324in}}{\pgfqpoint{7.921740in}{1.577714in}}{\pgfqpoint{7.929553in}{1.585528in}}%
\pgfpathcurveto{\pgfqpoint{7.937367in}{1.593341in}}{\pgfqpoint{7.941757in}{1.603940in}}{\pgfqpoint{7.941757in}{1.614990in}}%
\pgfpathcurveto{\pgfqpoint{7.941757in}{1.626040in}}{\pgfqpoint{7.937367in}{1.636639in}}{\pgfqpoint{7.929553in}{1.644453in}}%
\pgfpathcurveto{\pgfqpoint{7.921740in}{1.652267in}}{\pgfqpoint{7.911141in}{1.656657in}}{\pgfqpoint{7.900091in}{1.656657in}}%
\pgfpathcurveto{\pgfqpoint{7.889040in}{1.656657in}}{\pgfqpoint{7.878441in}{1.652267in}}{\pgfqpoint{7.870628in}{1.644453in}}%
\pgfpathcurveto{\pgfqpoint{7.862814in}{1.636639in}}{\pgfqpoint{7.858424in}{1.626040in}}{\pgfqpoint{7.858424in}{1.614990in}}%
\pgfpathcurveto{\pgfqpoint{7.858424in}{1.603940in}}{\pgfqpoint{7.862814in}{1.593341in}}{\pgfqpoint{7.870628in}{1.585528in}}%
\pgfpathcurveto{\pgfqpoint{7.878441in}{1.577714in}}{\pgfqpoint{7.889040in}{1.573324in}}{\pgfqpoint{7.900091in}{1.573324in}}%
\pgfpathlineto{\pgfqpoint{7.900091in}{1.573324in}}%
\pgfpathclose%
\pgfusepath{stroke}%
\end{pgfscope}%
\begin{pgfscope}%
\pgfpathrectangle{\pgfqpoint{7.394209in}{0.375000in}}{\pgfqpoint{6.356833in}{5.175000in}}%
\pgfusepath{clip}%
\pgfsetbuttcap%
\pgfsetroundjoin%
\pgfsetlinewidth{1.003750pt}%
\definecolor{currentstroke}{rgb}{0.827451,0.827451,0.827451}%
\pgfsetstrokecolor{currentstroke}%
\pgfsetdash{}{0pt}%
\pgfpathmoveto{\pgfqpoint{8.515906in}{2.810697in}}%
\pgfpathcurveto{\pgfqpoint{8.526956in}{2.810697in}}{\pgfqpoint{8.537556in}{2.815087in}}{\pgfqpoint{8.545369in}{2.822901in}}%
\pgfpathcurveto{\pgfqpoint{8.553183in}{2.830715in}}{\pgfqpoint{8.557573in}{2.841314in}}{\pgfqpoint{8.557573in}{2.852364in}}%
\pgfpathcurveto{\pgfqpoint{8.557573in}{2.863414in}}{\pgfqpoint{8.553183in}{2.874013in}}{\pgfqpoint{8.545369in}{2.881827in}}%
\pgfpathcurveto{\pgfqpoint{8.537556in}{2.889640in}}{\pgfqpoint{8.526956in}{2.894030in}}{\pgfqpoint{8.515906in}{2.894030in}}%
\pgfpathcurveto{\pgfqpoint{8.504856in}{2.894030in}}{\pgfqpoint{8.494257in}{2.889640in}}{\pgfqpoint{8.486444in}{2.881827in}}%
\pgfpathcurveto{\pgfqpoint{8.478630in}{2.874013in}}{\pgfqpoint{8.474240in}{2.863414in}}{\pgfqpoint{8.474240in}{2.852364in}}%
\pgfpathcurveto{\pgfqpoint{8.474240in}{2.841314in}}{\pgfqpoint{8.478630in}{2.830715in}}{\pgfqpoint{8.486444in}{2.822901in}}%
\pgfpathcurveto{\pgfqpoint{8.494257in}{2.815087in}}{\pgfqpoint{8.504856in}{2.810697in}}{\pgfqpoint{8.515906in}{2.810697in}}%
\pgfpathlineto{\pgfqpoint{8.515906in}{2.810697in}}%
\pgfpathclose%
\pgfusepath{stroke}%
\end{pgfscope}%
\begin{pgfscope}%
\pgfpathrectangle{\pgfqpoint{7.394209in}{0.375000in}}{\pgfqpoint{6.356833in}{5.175000in}}%
\pgfusepath{clip}%
\pgfsetbuttcap%
\pgfsetroundjoin%
\pgfsetlinewidth{1.003750pt}%
\definecolor{currentstroke}{rgb}{0.827451,0.827451,0.827451}%
\pgfsetstrokecolor{currentstroke}%
\pgfsetdash{}{0pt}%
\pgfpathmoveto{\pgfqpoint{10.293577in}{3.541270in}}%
\pgfpathcurveto{\pgfqpoint{10.304627in}{3.541270in}}{\pgfqpoint{10.315226in}{3.545660in}}{\pgfqpoint{10.323039in}{3.553474in}}%
\pgfpathcurveto{\pgfqpoint{10.330853in}{3.561287in}}{\pgfqpoint{10.335243in}{3.571886in}}{\pgfqpoint{10.335243in}{3.582937in}}%
\pgfpathcurveto{\pgfqpoint{10.335243in}{3.593987in}}{\pgfqpoint{10.330853in}{3.604586in}}{\pgfqpoint{10.323039in}{3.612399in}}%
\pgfpathcurveto{\pgfqpoint{10.315226in}{3.620213in}}{\pgfqpoint{10.304627in}{3.624603in}}{\pgfqpoint{10.293577in}{3.624603in}}%
\pgfpathcurveto{\pgfqpoint{10.282527in}{3.624603in}}{\pgfqpoint{10.271927in}{3.620213in}}{\pgfqpoint{10.264114in}{3.612399in}}%
\pgfpathcurveto{\pgfqpoint{10.256300in}{3.604586in}}{\pgfqpoint{10.251910in}{3.593987in}}{\pgfqpoint{10.251910in}{3.582937in}}%
\pgfpathcurveto{\pgfqpoint{10.251910in}{3.571886in}}{\pgfqpoint{10.256300in}{3.561287in}}{\pgfqpoint{10.264114in}{3.553474in}}%
\pgfpathcurveto{\pgfqpoint{10.271927in}{3.545660in}}{\pgfqpoint{10.282527in}{3.541270in}}{\pgfqpoint{10.293577in}{3.541270in}}%
\pgfpathlineto{\pgfqpoint{10.293577in}{3.541270in}}%
\pgfpathclose%
\pgfusepath{stroke}%
\end{pgfscope}%
\begin{pgfscope}%
\pgfpathrectangle{\pgfqpoint{7.394209in}{0.375000in}}{\pgfqpoint{6.356833in}{5.175000in}}%
\pgfusepath{clip}%
\pgfsetbuttcap%
\pgfsetroundjoin%
\pgfsetlinewidth{1.003750pt}%
\definecolor{currentstroke}{rgb}{0.827451,0.827451,0.827451}%
\pgfsetstrokecolor{currentstroke}%
\pgfsetdash{}{0pt}%
\pgfpathmoveto{\pgfqpoint{10.158256in}{4.953039in}}%
\pgfpathcurveto{\pgfqpoint{10.169306in}{4.953039in}}{\pgfqpoint{10.179906in}{4.957430in}}{\pgfqpoint{10.187719in}{4.965243in}}%
\pgfpathcurveto{\pgfqpoint{10.195533in}{4.973057in}}{\pgfqpoint{10.199923in}{4.983656in}}{\pgfqpoint{10.199923in}{4.994706in}}%
\pgfpathcurveto{\pgfqpoint{10.199923in}{5.005756in}}{\pgfqpoint{10.195533in}{5.016355in}}{\pgfqpoint{10.187719in}{5.024169in}}%
\pgfpathcurveto{\pgfqpoint{10.179906in}{5.031982in}}{\pgfqpoint{10.169306in}{5.036373in}}{\pgfqpoint{10.158256in}{5.036373in}}%
\pgfpathcurveto{\pgfqpoint{10.147206in}{5.036373in}}{\pgfqpoint{10.136607in}{5.031982in}}{\pgfqpoint{10.128794in}{5.024169in}}%
\pgfpathcurveto{\pgfqpoint{10.120980in}{5.016355in}}{\pgfqpoint{10.116590in}{5.005756in}}{\pgfqpoint{10.116590in}{4.994706in}}%
\pgfpathcurveto{\pgfqpoint{10.116590in}{4.983656in}}{\pgfqpoint{10.120980in}{4.973057in}}{\pgfqpoint{10.128794in}{4.965243in}}%
\pgfpathcurveto{\pgfqpoint{10.136607in}{4.957430in}}{\pgfqpoint{10.147206in}{4.953039in}}{\pgfqpoint{10.158256in}{4.953039in}}%
\pgfpathlineto{\pgfqpoint{10.158256in}{4.953039in}}%
\pgfpathclose%
\pgfusepath{stroke}%
\end{pgfscope}%
\begin{pgfscope}%
\pgfpathrectangle{\pgfqpoint{7.394209in}{0.375000in}}{\pgfqpoint{6.356833in}{5.175000in}}%
\pgfusepath{clip}%
\pgfsetbuttcap%
\pgfsetroundjoin%
\pgfsetlinewidth{1.003750pt}%
\definecolor{currentstroke}{rgb}{0.827451,0.827451,0.827451}%
\pgfsetstrokecolor{currentstroke}%
\pgfsetdash{}{0pt}%
\pgfpathmoveto{\pgfqpoint{8.125122in}{0.781922in}}%
\pgfpathcurveto{\pgfqpoint{8.136172in}{0.781922in}}{\pgfqpoint{8.146771in}{0.786313in}}{\pgfqpoint{8.154585in}{0.794126in}}%
\pgfpathcurveto{\pgfqpoint{8.162398in}{0.801940in}}{\pgfqpoint{8.166789in}{0.812539in}}{\pgfqpoint{8.166789in}{0.823589in}}%
\pgfpathcurveto{\pgfqpoint{8.166789in}{0.834639in}}{\pgfqpoint{8.162398in}{0.845238in}}{\pgfqpoint{8.154585in}{0.853052in}}%
\pgfpathcurveto{\pgfqpoint{8.146771in}{0.860866in}}{\pgfqpoint{8.136172in}{0.865256in}}{\pgfqpoint{8.125122in}{0.865256in}}%
\pgfpathcurveto{\pgfqpoint{8.114072in}{0.865256in}}{\pgfqpoint{8.103473in}{0.860866in}}{\pgfqpoint{8.095659in}{0.853052in}}%
\pgfpathcurveto{\pgfqpoint{8.087845in}{0.845238in}}{\pgfqpoint{8.083455in}{0.834639in}}{\pgfqpoint{8.083455in}{0.823589in}}%
\pgfpathcurveto{\pgfqpoint{8.083455in}{0.812539in}}{\pgfqpoint{8.087845in}{0.801940in}}{\pgfqpoint{8.095659in}{0.794126in}}%
\pgfpathcurveto{\pgfqpoint{8.103473in}{0.786313in}}{\pgfqpoint{8.114072in}{0.781922in}}{\pgfqpoint{8.125122in}{0.781922in}}%
\pgfpathlineto{\pgfqpoint{8.125122in}{0.781922in}}%
\pgfpathclose%
\pgfusepath{stroke}%
\end{pgfscope}%
\begin{pgfscope}%
\pgfpathrectangle{\pgfqpoint{7.394209in}{0.375000in}}{\pgfqpoint{6.356833in}{5.175000in}}%
\pgfusepath{clip}%
\pgfsetbuttcap%
\pgfsetroundjoin%
\pgfsetlinewidth{1.003750pt}%
\definecolor{currentstroke}{rgb}{0.827451,0.827451,0.827451}%
\pgfsetstrokecolor{currentstroke}%
\pgfsetdash{}{0pt}%
\pgfpathmoveto{\pgfqpoint{11.616520in}{5.231433in}}%
\pgfpathcurveto{\pgfqpoint{11.627570in}{5.231433in}}{\pgfqpoint{11.638169in}{5.235824in}}{\pgfqpoint{11.645983in}{5.243637in}}%
\pgfpathcurveto{\pgfqpoint{11.653797in}{5.251451in}}{\pgfqpoint{11.658187in}{5.262050in}}{\pgfqpoint{11.658187in}{5.273100in}}%
\pgfpathcurveto{\pgfqpoint{11.658187in}{5.284150in}}{\pgfqpoint{11.653797in}{5.294749in}}{\pgfqpoint{11.645983in}{5.302563in}}%
\pgfpathcurveto{\pgfqpoint{11.638169in}{5.310376in}}{\pgfqpoint{11.627570in}{5.314767in}}{\pgfqpoint{11.616520in}{5.314767in}}%
\pgfpathcurveto{\pgfqpoint{11.605470in}{5.314767in}}{\pgfqpoint{11.594871in}{5.310376in}}{\pgfqpoint{11.587058in}{5.302563in}}%
\pgfpathcurveto{\pgfqpoint{11.579244in}{5.294749in}}{\pgfqpoint{11.574854in}{5.284150in}}{\pgfqpoint{11.574854in}{5.273100in}}%
\pgfpathcurveto{\pgfqpoint{11.574854in}{5.262050in}}{\pgfqpoint{11.579244in}{5.251451in}}{\pgfqpoint{11.587058in}{5.243637in}}%
\pgfpathcurveto{\pgfqpoint{11.594871in}{5.235824in}}{\pgfqpoint{11.605470in}{5.231433in}}{\pgfqpoint{11.616520in}{5.231433in}}%
\pgfpathlineto{\pgfqpoint{11.616520in}{5.231433in}}%
\pgfpathclose%
\pgfusepath{stroke}%
\end{pgfscope}%
\begin{pgfscope}%
\pgfpathrectangle{\pgfqpoint{7.394209in}{0.375000in}}{\pgfqpoint{6.356833in}{5.175000in}}%
\pgfusepath{clip}%
\pgfsetbuttcap%
\pgfsetroundjoin%
\pgfsetlinewidth{1.003750pt}%
\definecolor{currentstroke}{rgb}{0.827451,0.827451,0.827451}%
\pgfsetstrokecolor{currentstroke}%
\pgfsetdash{}{0pt}%
\pgfpathmoveto{\pgfqpoint{10.333022in}{5.116329in}}%
\pgfpathcurveto{\pgfqpoint{10.344072in}{5.116329in}}{\pgfqpoint{10.354671in}{5.120720in}}{\pgfqpoint{10.362484in}{5.128533in}}%
\pgfpathcurveto{\pgfqpoint{10.370298in}{5.136347in}}{\pgfqpoint{10.374688in}{5.146946in}}{\pgfqpoint{10.374688in}{5.157996in}}%
\pgfpathcurveto{\pgfqpoint{10.374688in}{5.169046in}}{\pgfqpoint{10.370298in}{5.179645in}}{\pgfqpoint{10.362484in}{5.187459in}}%
\pgfpathcurveto{\pgfqpoint{10.354671in}{5.195273in}}{\pgfqpoint{10.344072in}{5.199663in}}{\pgfqpoint{10.333022in}{5.199663in}}%
\pgfpathcurveto{\pgfqpoint{10.321972in}{5.199663in}}{\pgfqpoint{10.311372in}{5.195273in}}{\pgfqpoint{10.303559in}{5.187459in}}%
\pgfpathcurveto{\pgfqpoint{10.295745in}{5.179645in}}{\pgfqpoint{10.291355in}{5.169046in}}{\pgfqpoint{10.291355in}{5.157996in}}%
\pgfpathcurveto{\pgfqpoint{10.291355in}{5.146946in}}{\pgfqpoint{10.295745in}{5.136347in}}{\pgfqpoint{10.303559in}{5.128533in}}%
\pgfpathcurveto{\pgfqpoint{10.311372in}{5.120720in}}{\pgfqpoint{10.321972in}{5.116329in}}{\pgfqpoint{10.333022in}{5.116329in}}%
\pgfpathlineto{\pgfqpoint{10.333022in}{5.116329in}}%
\pgfpathclose%
\pgfusepath{stroke}%
\end{pgfscope}%
\begin{pgfscope}%
\pgfpathrectangle{\pgfqpoint{7.394209in}{0.375000in}}{\pgfqpoint{6.356833in}{5.175000in}}%
\pgfusepath{clip}%
\pgfsetbuttcap%
\pgfsetroundjoin%
\pgfsetlinewidth{1.003750pt}%
\definecolor{currentstroke}{rgb}{0.827451,0.827451,0.827451}%
\pgfsetstrokecolor{currentstroke}%
\pgfsetdash{}{0pt}%
\pgfpathmoveto{\pgfqpoint{9.874278in}{4.125293in}}%
\pgfpathcurveto{\pgfqpoint{9.885328in}{4.125293in}}{\pgfqpoint{9.895927in}{4.129683in}}{\pgfqpoint{9.903741in}{4.137497in}}%
\pgfpathcurveto{\pgfqpoint{9.911555in}{4.145310in}}{\pgfqpoint{9.915945in}{4.155909in}}{\pgfqpoint{9.915945in}{4.166959in}}%
\pgfpathcurveto{\pgfqpoint{9.915945in}{4.178010in}}{\pgfqpoint{9.911555in}{4.188609in}}{\pgfqpoint{9.903741in}{4.196422in}}%
\pgfpathcurveto{\pgfqpoint{9.895927in}{4.204236in}}{\pgfqpoint{9.885328in}{4.208626in}}{\pgfqpoint{9.874278in}{4.208626in}}%
\pgfpathcurveto{\pgfqpoint{9.863228in}{4.208626in}}{\pgfqpoint{9.852629in}{4.204236in}}{\pgfqpoint{9.844815in}{4.196422in}}%
\pgfpathcurveto{\pgfqpoint{9.837002in}{4.188609in}}{\pgfqpoint{9.832611in}{4.178010in}}{\pgfqpoint{9.832611in}{4.166959in}}%
\pgfpathcurveto{\pgfqpoint{9.832611in}{4.155909in}}{\pgfqpoint{9.837002in}{4.145310in}}{\pgfqpoint{9.844815in}{4.137497in}}%
\pgfpathcurveto{\pgfqpoint{9.852629in}{4.129683in}}{\pgfqpoint{9.863228in}{4.125293in}}{\pgfqpoint{9.874278in}{4.125293in}}%
\pgfpathlineto{\pgfqpoint{9.874278in}{4.125293in}}%
\pgfpathclose%
\pgfusepath{stroke}%
\end{pgfscope}%
\begin{pgfscope}%
\pgfpathrectangle{\pgfqpoint{7.394209in}{0.375000in}}{\pgfqpoint{6.356833in}{5.175000in}}%
\pgfusepath{clip}%
\pgfsetbuttcap%
\pgfsetroundjoin%
\pgfsetlinewidth{1.003750pt}%
\definecolor{currentstroke}{rgb}{0.827451,0.827451,0.827451}%
\pgfsetstrokecolor{currentstroke}%
\pgfsetdash{}{0pt}%
\pgfpathmoveto{\pgfqpoint{10.341569in}{5.128266in}}%
\pgfpathcurveto{\pgfqpoint{10.352619in}{5.128266in}}{\pgfqpoint{10.363218in}{5.132656in}}{\pgfqpoint{10.371032in}{5.140469in}}%
\pgfpathcurveto{\pgfqpoint{10.378846in}{5.148283in}}{\pgfqpoint{10.383236in}{5.158882in}}{\pgfqpoint{10.383236in}{5.169932in}}%
\pgfpathcurveto{\pgfqpoint{10.383236in}{5.180982in}}{\pgfqpoint{10.378846in}{5.191581in}}{\pgfqpoint{10.371032in}{5.199395in}}%
\pgfpathcurveto{\pgfqpoint{10.363218in}{5.207209in}}{\pgfqpoint{10.352619in}{5.211599in}}{\pgfqpoint{10.341569in}{5.211599in}}%
\pgfpathcurveto{\pgfqpoint{10.330519in}{5.211599in}}{\pgfqpoint{10.319920in}{5.207209in}}{\pgfqpoint{10.312106in}{5.199395in}}%
\pgfpathcurveto{\pgfqpoint{10.304293in}{5.191581in}}{\pgfqpoint{10.299902in}{5.180982in}}{\pgfqpoint{10.299902in}{5.169932in}}%
\pgfpathcurveto{\pgfqpoint{10.299902in}{5.158882in}}{\pgfqpoint{10.304293in}{5.148283in}}{\pgfqpoint{10.312106in}{5.140469in}}%
\pgfpathcurveto{\pgfqpoint{10.319920in}{5.132656in}}{\pgfqpoint{10.330519in}{5.128266in}}{\pgfqpoint{10.341569in}{5.128266in}}%
\pgfpathlineto{\pgfqpoint{10.341569in}{5.128266in}}%
\pgfpathclose%
\pgfusepath{stroke}%
\end{pgfscope}%
\begin{pgfscope}%
\pgfpathrectangle{\pgfqpoint{7.394209in}{0.375000in}}{\pgfqpoint{6.356833in}{5.175000in}}%
\pgfusepath{clip}%
\pgfsetbuttcap%
\pgfsetroundjoin%
\pgfsetlinewidth{1.003750pt}%
\definecolor{currentstroke}{rgb}{0.827451,0.827451,0.827451}%
\pgfsetstrokecolor{currentstroke}%
\pgfsetdash{}{0pt}%
\pgfpathmoveto{\pgfqpoint{9.022617in}{3.178770in}}%
\pgfpathcurveto{\pgfqpoint{9.033667in}{3.178770in}}{\pgfqpoint{9.044266in}{3.183160in}}{\pgfqpoint{9.052079in}{3.190974in}}%
\pgfpathcurveto{\pgfqpoint{9.059893in}{3.198788in}}{\pgfqpoint{9.064283in}{3.209387in}}{\pgfqpoint{9.064283in}{3.220437in}}%
\pgfpathcurveto{\pgfqpoint{9.064283in}{3.231487in}}{\pgfqpoint{9.059893in}{3.242086in}}{\pgfqpoint{9.052079in}{3.249899in}}%
\pgfpathcurveto{\pgfqpoint{9.044266in}{3.257713in}}{\pgfqpoint{9.033667in}{3.262103in}}{\pgfqpoint{9.022617in}{3.262103in}}%
\pgfpathcurveto{\pgfqpoint{9.011567in}{3.262103in}}{\pgfqpoint{9.000967in}{3.257713in}}{\pgfqpoint{8.993154in}{3.249899in}}%
\pgfpathcurveto{\pgfqpoint{8.985340in}{3.242086in}}{\pgfqpoint{8.980950in}{3.231487in}}{\pgfqpoint{8.980950in}{3.220437in}}%
\pgfpathcurveto{\pgfqpoint{8.980950in}{3.209387in}}{\pgfqpoint{8.985340in}{3.198788in}}{\pgfqpoint{8.993154in}{3.190974in}}%
\pgfpathcurveto{\pgfqpoint{9.000967in}{3.183160in}}{\pgfqpoint{9.011567in}{3.178770in}}{\pgfqpoint{9.022617in}{3.178770in}}%
\pgfpathlineto{\pgfqpoint{9.022617in}{3.178770in}}%
\pgfpathclose%
\pgfusepath{stroke}%
\end{pgfscope}%
\begin{pgfscope}%
\pgfpathrectangle{\pgfqpoint{7.394209in}{0.375000in}}{\pgfqpoint{6.356833in}{5.175000in}}%
\pgfusepath{clip}%
\pgfsetbuttcap%
\pgfsetroundjoin%
\pgfsetlinewidth{1.003750pt}%
\definecolor{currentstroke}{rgb}{0.827451,0.827451,0.827451}%
\pgfsetstrokecolor{currentstroke}%
\pgfsetdash{}{0pt}%
\pgfpathmoveto{\pgfqpoint{8.996512in}{2.887803in}}%
\pgfpathcurveto{\pgfqpoint{9.007563in}{2.887803in}}{\pgfqpoint{9.018162in}{2.892193in}}{\pgfqpoint{9.025975in}{2.900007in}}%
\pgfpathcurveto{\pgfqpoint{9.033789in}{2.907820in}}{\pgfqpoint{9.038179in}{2.918419in}}{\pgfqpoint{9.038179in}{2.929469in}}%
\pgfpathcurveto{\pgfqpoint{9.038179in}{2.940519in}}{\pgfqpoint{9.033789in}{2.951118in}}{\pgfqpoint{9.025975in}{2.958932in}}%
\pgfpathcurveto{\pgfqpoint{9.018162in}{2.966746in}}{\pgfqpoint{9.007563in}{2.971136in}}{\pgfqpoint{8.996512in}{2.971136in}}%
\pgfpathcurveto{\pgfqpoint{8.985462in}{2.971136in}}{\pgfqpoint{8.974863in}{2.966746in}}{\pgfqpoint{8.967050in}{2.958932in}}%
\pgfpathcurveto{\pgfqpoint{8.959236in}{2.951118in}}{\pgfqpoint{8.954846in}{2.940519in}}{\pgfqpoint{8.954846in}{2.929469in}}%
\pgfpathcurveto{\pgfqpoint{8.954846in}{2.918419in}}{\pgfqpoint{8.959236in}{2.907820in}}{\pgfqpoint{8.967050in}{2.900007in}}%
\pgfpathcurveto{\pgfqpoint{8.974863in}{2.892193in}}{\pgfqpoint{8.985462in}{2.887803in}}{\pgfqpoint{8.996512in}{2.887803in}}%
\pgfpathlineto{\pgfqpoint{8.996512in}{2.887803in}}%
\pgfpathclose%
\pgfusepath{stroke}%
\end{pgfscope}%
\begin{pgfscope}%
\pgfpathrectangle{\pgfqpoint{7.394209in}{0.375000in}}{\pgfqpoint{6.356833in}{5.175000in}}%
\pgfusepath{clip}%
\pgfsetbuttcap%
\pgfsetroundjoin%
\pgfsetlinewidth{1.003750pt}%
\definecolor{currentstroke}{rgb}{0.827451,0.827451,0.827451}%
\pgfsetstrokecolor{currentstroke}%
\pgfsetdash{}{0pt}%
\pgfpathmoveto{\pgfqpoint{10.532402in}{3.242717in}}%
\pgfpathcurveto{\pgfqpoint{10.543452in}{3.242717in}}{\pgfqpoint{10.554051in}{3.247107in}}{\pgfqpoint{10.561864in}{3.254921in}}%
\pgfpathcurveto{\pgfqpoint{10.569678in}{3.262734in}}{\pgfqpoint{10.574068in}{3.273333in}}{\pgfqpoint{10.574068in}{3.284383in}}%
\pgfpathcurveto{\pgfqpoint{10.574068in}{3.295434in}}{\pgfqpoint{10.569678in}{3.306033in}}{\pgfqpoint{10.561864in}{3.313846in}}%
\pgfpathcurveto{\pgfqpoint{10.554051in}{3.321660in}}{\pgfqpoint{10.543452in}{3.326050in}}{\pgfqpoint{10.532402in}{3.326050in}}%
\pgfpathcurveto{\pgfqpoint{10.521351in}{3.326050in}}{\pgfqpoint{10.510752in}{3.321660in}}{\pgfqpoint{10.502939in}{3.313846in}}%
\pgfpathcurveto{\pgfqpoint{10.495125in}{3.306033in}}{\pgfqpoint{10.490735in}{3.295434in}}{\pgfqpoint{10.490735in}{3.284383in}}%
\pgfpathcurveto{\pgfqpoint{10.490735in}{3.273333in}}{\pgfqpoint{10.495125in}{3.262734in}}{\pgfqpoint{10.502939in}{3.254921in}}%
\pgfpathcurveto{\pgfqpoint{10.510752in}{3.247107in}}{\pgfqpoint{10.521351in}{3.242717in}}{\pgfqpoint{10.532402in}{3.242717in}}%
\pgfpathlineto{\pgfqpoint{10.532402in}{3.242717in}}%
\pgfpathclose%
\pgfusepath{stroke}%
\end{pgfscope}%
\begin{pgfscope}%
\pgfpathrectangle{\pgfqpoint{7.394209in}{0.375000in}}{\pgfqpoint{6.356833in}{5.175000in}}%
\pgfusepath{clip}%
\pgfsetbuttcap%
\pgfsetroundjoin%
\pgfsetlinewidth{1.003750pt}%
\definecolor{currentstroke}{rgb}{0.827451,0.827451,0.827451}%
\pgfsetstrokecolor{currentstroke}%
\pgfsetdash{}{0pt}%
\pgfpathmoveto{\pgfqpoint{7.682298in}{1.107527in}}%
\pgfpathcurveto{\pgfqpoint{7.693348in}{1.107527in}}{\pgfqpoint{7.703947in}{1.111917in}}{\pgfqpoint{7.711760in}{1.119730in}}%
\pgfpathcurveto{\pgfqpoint{7.719574in}{1.127544in}}{\pgfqpoint{7.723964in}{1.138143in}}{\pgfqpoint{7.723964in}{1.149193in}}%
\pgfpathcurveto{\pgfqpoint{7.723964in}{1.160243in}}{\pgfqpoint{7.719574in}{1.170842in}}{\pgfqpoint{7.711760in}{1.178656in}}%
\pgfpathcurveto{\pgfqpoint{7.703947in}{1.186470in}}{\pgfqpoint{7.693348in}{1.190860in}}{\pgfqpoint{7.682298in}{1.190860in}}%
\pgfpathcurveto{\pgfqpoint{7.671248in}{1.190860in}}{\pgfqpoint{7.660648in}{1.186470in}}{\pgfqpoint{7.652835in}{1.178656in}}%
\pgfpathcurveto{\pgfqpoint{7.645021in}{1.170842in}}{\pgfqpoint{7.640631in}{1.160243in}}{\pgfqpoint{7.640631in}{1.149193in}}%
\pgfpathcurveto{\pgfqpoint{7.640631in}{1.138143in}}{\pgfqpoint{7.645021in}{1.127544in}}{\pgfqpoint{7.652835in}{1.119730in}}%
\pgfpathcurveto{\pgfqpoint{7.660648in}{1.111917in}}{\pgfqpoint{7.671248in}{1.107527in}}{\pgfqpoint{7.682298in}{1.107527in}}%
\pgfpathlineto{\pgfqpoint{7.682298in}{1.107527in}}%
\pgfpathclose%
\pgfusepath{stroke}%
\end{pgfscope}%
\begin{pgfscope}%
\pgfpathrectangle{\pgfqpoint{7.394209in}{0.375000in}}{\pgfqpoint{6.356833in}{5.175000in}}%
\pgfusepath{clip}%
\pgfsetbuttcap%
\pgfsetroundjoin%
\pgfsetlinewidth{1.003750pt}%
\definecolor{currentstroke}{rgb}{0.827451,0.827451,0.827451}%
\pgfsetstrokecolor{currentstroke}%
\pgfsetdash{}{0pt}%
\pgfpathmoveto{\pgfqpoint{7.840832in}{0.337935in}}%
\pgfpathcurveto{\pgfqpoint{7.851882in}{0.337935in}}{\pgfqpoint{7.862481in}{0.342325in}}{\pgfqpoint{7.870295in}{0.350139in}}%
\pgfpathcurveto{\pgfqpoint{7.878108in}{0.357953in}}{\pgfqpoint{7.882499in}{0.368552in}}{\pgfqpoint{7.882499in}{0.379602in}}%
\pgfpathcurveto{\pgfqpoint{7.882499in}{0.390652in}}{\pgfqpoint{7.878108in}{0.401251in}}{\pgfqpoint{7.870295in}{0.409065in}}%
\pgfpathcurveto{\pgfqpoint{7.862481in}{0.416878in}}{\pgfqpoint{7.851882in}{0.421269in}}{\pgfqpoint{7.840832in}{0.421269in}}%
\pgfpathcurveto{\pgfqpoint{7.829782in}{0.421269in}}{\pgfqpoint{7.819183in}{0.416878in}}{\pgfqpoint{7.811369in}{0.409065in}}%
\pgfpathcurveto{\pgfqpoint{7.803556in}{0.401251in}}{\pgfqpoint{7.799165in}{0.390652in}}{\pgfqpoint{7.799165in}{0.379602in}}%
\pgfpathcurveto{\pgfqpoint{7.799165in}{0.368552in}}{\pgfqpoint{7.803556in}{0.357953in}}{\pgfqpoint{7.811369in}{0.350139in}}%
\pgfpathcurveto{\pgfqpoint{7.819183in}{0.342325in}}{\pgfqpoint{7.829782in}{0.337935in}}{\pgfqpoint{7.840832in}{0.337935in}}%
\pgfusepath{stroke}%
\end{pgfscope}%
\begin{pgfscope}%
\pgfpathrectangle{\pgfqpoint{7.394209in}{0.375000in}}{\pgfqpoint{6.356833in}{5.175000in}}%
\pgfusepath{clip}%
\pgfsetbuttcap%
\pgfsetroundjoin%
\pgfsetlinewidth{1.003750pt}%
\definecolor{currentstroke}{rgb}{0.827451,0.827451,0.827451}%
\pgfsetstrokecolor{currentstroke}%
\pgfsetdash{}{0pt}%
\pgfpathmoveto{\pgfqpoint{11.380997in}{4.931407in}}%
\pgfpathcurveto{\pgfqpoint{11.392048in}{4.931407in}}{\pgfqpoint{11.402647in}{4.935797in}}{\pgfqpoint{11.410460in}{4.943611in}}%
\pgfpathcurveto{\pgfqpoint{11.418274in}{4.951424in}}{\pgfqpoint{11.422664in}{4.962023in}}{\pgfqpoint{11.422664in}{4.973073in}}%
\pgfpathcurveto{\pgfqpoint{11.422664in}{4.984124in}}{\pgfqpoint{11.418274in}{4.994723in}}{\pgfqpoint{11.410460in}{5.002536in}}%
\pgfpathcurveto{\pgfqpoint{11.402647in}{5.010350in}}{\pgfqpoint{11.392048in}{5.014740in}}{\pgfqpoint{11.380997in}{5.014740in}}%
\pgfpathcurveto{\pgfqpoint{11.369947in}{5.014740in}}{\pgfqpoint{11.359348in}{5.010350in}}{\pgfqpoint{11.351535in}{5.002536in}}%
\pgfpathcurveto{\pgfqpoint{11.343721in}{4.994723in}}{\pgfqpoint{11.339331in}{4.984124in}}{\pgfqpoint{11.339331in}{4.973073in}}%
\pgfpathcurveto{\pgfqpoint{11.339331in}{4.962023in}}{\pgfqpoint{11.343721in}{4.951424in}}{\pgfqpoint{11.351535in}{4.943611in}}%
\pgfpathcurveto{\pgfqpoint{11.359348in}{4.935797in}}{\pgfqpoint{11.369947in}{4.931407in}}{\pgfqpoint{11.380997in}{4.931407in}}%
\pgfpathlineto{\pgfqpoint{11.380997in}{4.931407in}}%
\pgfpathclose%
\pgfusepath{stroke}%
\end{pgfscope}%
\begin{pgfscope}%
\pgfpathrectangle{\pgfqpoint{7.394209in}{0.375000in}}{\pgfqpoint{6.356833in}{5.175000in}}%
\pgfusepath{clip}%
\pgfsetbuttcap%
\pgfsetroundjoin%
\pgfsetlinewidth{1.003750pt}%
\definecolor{currentstroke}{rgb}{0.827451,0.827451,0.827451}%
\pgfsetstrokecolor{currentstroke}%
\pgfsetdash{}{0pt}%
\pgfpathmoveto{\pgfqpoint{10.477254in}{4.926434in}}%
\pgfpathcurveto{\pgfqpoint{10.488304in}{4.926434in}}{\pgfqpoint{10.498903in}{4.930824in}}{\pgfqpoint{10.506717in}{4.938638in}}%
\pgfpathcurveto{\pgfqpoint{10.514530in}{4.946451in}}{\pgfqpoint{10.518921in}{4.957051in}}{\pgfqpoint{10.518921in}{4.968101in}}%
\pgfpathcurveto{\pgfqpoint{10.518921in}{4.979151in}}{\pgfqpoint{10.514530in}{4.989750in}}{\pgfqpoint{10.506717in}{4.997563in}}%
\pgfpathcurveto{\pgfqpoint{10.498903in}{5.005377in}}{\pgfqpoint{10.488304in}{5.009767in}}{\pgfqpoint{10.477254in}{5.009767in}}%
\pgfpathcurveto{\pgfqpoint{10.466204in}{5.009767in}}{\pgfqpoint{10.455605in}{5.005377in}}{\pgfqpoint{10.447791in}{4.997563in}}%
\pgfpathcurveto{\pgfqpoint{10.439978in}{4.989750in}}{\pgfqpoint{10.435587in}{4.979151in}}{\pgfqpoint{10.435587in}{4.968101in}}%
\pgfpathcurveto{\pgfqpoint{10.435587in}{4.957051in}}{\pgfqpoint{10.439978in}{4.946451in}}{\pgfqpoint{10.447791in}{4.938638in}}%
\pgfpathcurveto{\pgfqpoint{10.455605in}{4.930824in}}{\pgfqpoint{10.466204in}{4.926434in}}{\pgfqpoint{10.477254in}{4.926434in}}%
\pgfpathlineto{\pgfqpoint{10.477254in}{4.926434in}}%
\pgfpathclose%
\pgfusepath{stroke}%
\end{pgfscope}%
\begin{pgfscope}%
\pgfpathrectangle{\pgfqpoint{7.394209in}{0.375000in}}{\pgfqpoint{6.356833in}{5.175000in}}%
\pgfusepath{clip}%
\pgfsetbuttcap%
\pgfsetroundjoin%
\pgfsetlinewidth{1.003750pt}%
\definecolor{currentstroke}{rgb}{0.827451,0.827451,0.827451}%
\pgfsetstrokecolor{currentstroke}%
\pgfsetdash{}{0pt}%
\pgfpathmoveto{\pgfqpoint{11.258194in}{4.660891in}}%
\pgfpathcurveto{\pgfqpoint{11.269244in}{4.660891in}}{\pgfqpoint{11.279843in}{4.665281in}}{\pgfqpoint{11.287657in}{4.673095in}}%
\pgfpathcurveto{\pgfqpoint{11.295470in}{4.680908in}}{\pgfqpoint{11.299861in}{4.691507in}}{\pgfqpoint{11.299861in}{4.702558in}}%
\pgfpathcurveto{\pgfqpoint{11.299861in}{4.713608in}}{\pgfqpoint{11.295470in}{4.724207in}}{\pgfqpoint{11.287657in}{4.732020in}}%
\pgfpathcurveto{\pgfqpoint{11.279843in}{4.739834in}}{\pgfqpoint{11.269244in}{4.744224in}}{\pgfqpoint{11.258194in}{4.744224in}}%
\pgfpathcurveto{\pgfqpoint{11.247144in}{4.744224in}}{\pgfqpoint{11.236545in}{4.739834in}}{\pgfqpoint{11.228731in}{4.732020in}}%
\pgfpathcurveto{\pgfqpoint{11.220917in}{4.724207in}}{\pgfqpoint{11.216527in}{4.713608in}}{\pgfqpoint{11.216527in}{4.702558in}}%
\pgfpathcurveto{\pgfqpoint{11.216527in}{4.691507in}}{\pgfqpoint{11.220917in}{4.680908in}}{\pgfqpoint{11.228731in}{4.673095in}}%
\pgfpathcurveto{\pgfqpoint{11.236545in}{4.665281in}}{\pgfqpoint{11.247144in}{4.660891in}}{\pgfqpoint{11.258194in}{4.660891in}}%
\pgfpathlineto{\pgfqpoint{11.258194in}{4.660891in}}%
\pgfpathclose%
\pgfusepath{stroke}%
\end{pgfscope}%
\begin{pgfscope}%
\pgfpathrectangle{\pgfqpoint{7.394209in}{0.375000in}}{\pgfqpoint{6.356833in}{5.175000in}}%
\pgfusepath{clip}%
\pgfsetbuttcap%
\pgfsetroundjoin%
\pgfsetlinewidth{1.003750pt}%
\definecolor{currentstroke}{rgb}{0.827451,0.827451,0.827451}%
\pgfsetstrokecolor{currentstroke}%
\pgfsetdash{}{0pt}%
\pgfpathmoveto{\pgfqpoint{7.682298in}{1.118626in}}%
\pgfpathcurveto{\pgfqpoint{7.693348in}{1.118626in}}{\pgfqpoint{7.703947in}{1.123016in}}{\pgfqpoint{7.711760in}{1.130830in}}%
\pgfpathcurveto{\pgfqpoint{7.719574in}{1.138643in}}{\pgfqpoint{7.723964in}{1.149242in}}{\pgfqpoint{7.723964in}{1.160292in}}%
\pgfpathcurveto{\pgfqpoint{7.723964in}{1.171342in}}{\pgfqpoint{7.719574in}{1.181941in}}{\pgfqpoint{7.711760in}{1.189755in}}%
\pgfpathcurveto{\pgfqpoint{7.703947in}{1.197569in}}{\pgfqpoint{7.693348in}{1.201959in}}{\pgfqpoint{7.682298in}{1.201959in}}%
\pgfpathcurveto{\pgfqpoint{7.671248in}{1.201959in}}{\pgfqpoint{7.660648in}{1.197569in}}{\pgfqpoint{7.652835in}{1.189755in}}%
\pgfpathcurveto{\pgfqpoint{7.645021in}{1.181941in}}{\pgfqpoint{7.640631in}{1.171342in}}{\pgfqpoint{7.640631in}{1.160292in}}%
\pgfpathcurveto{\pgfqpoint{7.640631in}{1.149242in}}{\pgfqpoint{7.645021in}{1.138643in}}{\pgfqpoint{7.652835in}{1.130830in}}%
\pgfpathcurveto{\pgfqpoint{7.660648in}{1.123016in}}{\pgfqpoint{7.671248in}{1.118626in}}{\pgfqpoint{7.682298in}{1.118626in}}%
\pgfpathlineto{\pgfqpoint{7.682298in}{1.118626in}}%
\pgfpathclose%
\pgfusepath{stroke}%
\end{pgfscope}%
\begin{pgfscope}%
\pgfpathrectangle{\pgfqpoint{7.394209in}{0.375000in}}{\pgfqpoint{6.356833in}{5.175000in}}%
\pgfusepath{clip}%
\pgfsetbuttcap%
\pgfsetroundjoin%
\pgfsetlinewidth{1.003750pt}%
\definecolor{currentstroke}{rgb}{0.827451,0.827451,0.827451}%
\pgfsetstrokecolor{currentstroke}%
\pgfsetdash{}{0pt}%
\pgfpathmoveto{\pgfqpoint{7.507171in}{0.378857in}}%
\pgfpathcurveto{\pgfqpoint{7.518221in}{0.378857in}}{\pgfqpoint{7.528820in}{0.383247in}}{\pgfqpoint{7.536634in}{0.391061in}}%
\pgfpathcurveto{\pgfqpoint{7.544448in}{0.398875in}}{\pgfqpoint{7.548838in}{0.409474in}}{\pgfqpoint{7.548838in}{0.420524in}}%
\pgfpathcurveto{\pgfqpoint{7.548838in}{0.431574in}}{\pgfqpoint{7.544448in}{0.442173in}}{\pgfqpoint{7.536634in}{0.449987in}}%
\pgfpathcurveto{\pgfqpoint{7.528820in}{0.457800in}}{\pgfqpoint{7.518221in}{0.462190in}}{\pgfqpoint{7.507171in}{0.462190in}}%
\pgfpathcurveto{\pgfqpoint{7.496121in}{0.462190in}}{\pgfqpoint{7.485522in}{0.457800in}}{\pgfqpoint{7.477708in}{0.449987in}}%
\pgfpathcurveto{\pgfqpoint{7.469895in}{0.442173in}}{\pgfqpoint{7.465505in}{0.431574in}}{\pgfqpoint{7.465505in}{0.420524in}}%
\pgfpathcurveto{\pgfqpoint{7.465505in}{0.409474in}}{\pgfqpoint{7.469895in}{0.398875in}}{\pgfqpoint{7.477708in}{0.391061in}}%
\pgfpathcurveto{\pgfqpoint{7.485522in}{0.383247in}}{\pgfqpoint{7.496121in}{0.378857in}}{\pgfqpoint{7.507171in}{0.378857in}}%
\pgfpathlineto{\pgfqpoint{7.507171in}{0.378857in}}%
\pgfpathclose%
\pgfusepath{stroke}%
\end{pgfscope}%
\begin{pgfscope}%
\pgfpathrectangle{\pgfqpoint{7.394209in}{0.375000in}}{\pgfqpoint{6.356833in}{5.175000in}}%
\pgfusepath{clip}%
\pgfsetbuttcap%
\pgfsetroundjoin%
\pgfsetlinewidth{1.003750pt}%
\definecolor{currentstroke}{rgb}{0.827451,0.827451,0.827451}%
\pgfsetstrokecolor{currentstroke}%
\pgfsetdash{}{0pt}%
\pgfpathmoveto{\pgfqpoint{10.692333in}{4.355743in}}%
\pgfpathcurveto{\pgfqpoint{10.703383in}{4.355743in}}{\pgfqpoint{10.713982in}{4.360133in}}{\pgfqpoint{10.721796in}{4.367947in}}%
\pgfpathcurveto{\pgfqpoint{10.729609in}{4.375760in}}{\pgfqpoint{10.734000in}{4.386359in}}{\pgfqpoint{10.734000in}{4.397409in}}%
\pgfpathcurveto{\pgfqpoint{10.734000in}{4.408460in}}{\pgfqpoint{10.729609in}{4.419059in}}{\pgfqpoint{10.721796in}{4.426872in}}%
\pgfpathcurveto{\pgfqpoint{10.713982in}{4.434686in}}{\pgfqpoint{10.703383in}{4.439076in}}{\pgfqpoint{10.692333in}{4.439076in}}%
\pgfpathcurveto{\pgfqpoint{10.681283in}{4.439076in}}{\pgfqpoint{10.670684in}{4.434686in}}{\pgfqpoint{10.662870in}{4.426872in}}%
\pgfpathcurveto{\pgfqpoint{10.655056in}{4.419059in}}{\pgfqpoint{10.650666in}{4.408460in}}{\pgfqpoint{10.650666in}{4.397409in}}%
\pgfpathcurveto{\pgfqpoint{10.650666in}{4.386359in}}{\pgfqpoint{10.655056in}{4.375760in}}{\pgfqpoint{10.662870in}{4.367947in}}%
\pgfpathcurveto{\pgfqpoint{10.670684in}{4.360133in}}{\pgfqpoint{10.681283in}{4.355743in}}{\pgfqpoint{10.692333in}{4.355743in}}%
\pgfpathlineto{\pgfqpoint{10.692333in}{4.355743in}}%
\pgfpathclose%
\pgfusepath{stroke}%
\end{pgfscope}%
\begin{pgfscope}%
\pgfpathrectangle{\pgfqpoint{7.394209in}{0.375000in}}{\pgfqpoint{6.356833in}{5.175000in}}%
\pgfusepath{clip}%
\pgfsetbuttcap%
\pgfsetroundjoin%
\pgfsetlinewidth{1.003750pt}%
\definecolor{currentstroke}{rgb}{0.827451,0.827451,0.827451}%
\pgfsetstrokecolor{currentstroke}%
\pgfsetdash{}{0pt}%
\pgfpathmoveto{\pgfqpoint{10.143309in}{4.962153in}}%
\pgfpathcurveto{\pgfqpoint{10.154359in}{4.962153in}}{\pgfqpoint{10.164958in}{4.966543in}}{\pgfqpoint{10.172772in}{4.974357in}}%
\pgfpathcurveto{\pgfqpoint{10.180585in}{4.982170in}}{\pgfqpoint{10.184976in}{4.992769in}}{\pgfqpoint{10.184976in}{5.003819in}}%
\pgfpathcurveto{\pgfqpoint{10.184976in}{5.014869in}}{\pgfqpoint{10.180585in}{5.025468in}}{\pgfqpoint{10.172772in}{5.033282in}}%
\pgfpathcurveto{\pgfqpoint{10.164958in}{5.041096in}}{\pgfqpoint{10.154359in}{5.045486in}}{\pgfqpoint{10.143309in}{5.045486in}}%
\pgfpathcurveto{\pgfqpoint{10.132259in}{5.045486in}}{\pgfqpoint{10.121660in}{5.041096in}}{\pgfqpoint{10.113846in}{5.033282in}}%
\pgfpathcurveto{\pgfqpoint{10.106032in}{5.025468in}}{\pgfqpoint{10.101642in}{5.014869in}}{\pgfqpoint{10.101642in}{5.003819in}}%
\pgfpathcurveto{\pgfqpoint{10.101642in}{4.992769in}}{\pgfqpoint{10.106032in}{4.982170in}}{\pgfqpoint{10.113846in}{4.974357in}}%
\pgfpathcurveto{\pgfqpoint{10.121660in}{4.966543in}}{\pgfqpoint{10.132259in}{4.962153in}}{\pgfqpoint{10.143309in}{4.962153in}}%
\pgfpathlineto{\pgfqpoint{10.143309in}{4.962153in}}%
\pgfpathclose%
\pgfusepath{stroke}%
\end{pgfscope}%
\begin{pgfscope}%
\pgfpathrectangle{\pgfqpoint{7.394209in}{0.375000in}}{\pgfqpoint{6.356833in}{5.175000in}}%
\pgfusepath{clip}%
\pgfsetbuttcap%
\pgfsetroundjoin%
\pgfsetlinewidth{1.003750pt}%
\definecolor{currentstroke}{rgb}{0.827451,0.827451,0.827451}%
\pgfsetstrokecolor{currentstroke}%
\pgfsetdash{}{0pt}%
\pgfpathmoveto{\pgfqpoint{10.708053in}{5.494601in}}%
\pgfpathcurveto{\pgfqpoint{10.719103in}{5.494601in}}{\pgfqpoint{10.729702in}{5.498992in}}{\pgfqpoint{10.737515in}{5.506805in}}%
\pgfpathcurveto{\pgfqpoint{10.745329in}{5.514619in}}{\pgfqpoint{10.749719in}{5.525218in}}{\pgfqpoint{10.749719in}{5.536268in}}%
\pgfpathcurveto{\pgfqpoint{10.749719in}{5.547318in}}{\pgfqpoint{10.745329in}{5.557917in}}{\pgfqpoint{10.737515in}{5.565731in}}%
\pgfpathcurveto{\pgfqpoint{10.729702in}{5.573545in}}{\pgfqpoint{10.719103in}{5.577935in}}{\pgfqpoint{10.708053in}{5.577935in}}%
\pgfpathcurveto{\pgfqpoint{10.697002in}{5.577935in}}{\pgfqpoint{10.686403in}{5.573545in}}{\pgfqpoint{10.678590in}{5.565731in}}%
\pgfpathcurveto{\pgfqpoint{10.670776in}{5.557917in}}{\pgfqpoint{10.666386in}{5.547318in}}{\pgfqpoint{10.666386in}{5.536268in}}%
\pgfpathcurveto{\pgfqpoint{10.666386in}{5.525218in}}{\pgfqpoint{10.670776in}{5.514619in}}{\pgfqpoint{10.678590in}{5.506805in}}%
\pgfpathcurveto{\pgfqpoint{10.686403in}{5.498992in}}{\pgfqpoint{10.697002in}{5.494601in}}{\pgfqpoint{10.708053in}{5.494601in}}%
\pgfpathlineto{\pgfqpoint{10.708053in}{5.494601in}}%
\pgfpathclose%
\pgfusepath{stroke}%
\end{pgfscope}%
\begin{pgfscope}%
\pgfpathrectangle{\pgfqpoint{7.394209in}{0.375000in}}{\pgfqpoint{6.356833in}{5.175000in}}%
\pgfusepath{clip}%
\pgfsetbuttcap%
\pgfsetroundjoin%
\pgfsetlinewidth{1.003750pt}%
\definecolor{currentstroke}{rgb}{0.827451,0.827451,0.827451}%
\pgfsetstrokecolor{currentstroke}%
\pgfsetdash{}{0pt}%
\pgfpathmoveto{\pgfqpoint{10.379657in}{4.780741in}}%
\pgfpathcurveto{\pgfqpoint{10.390707in}{4.780741in}}{\pgfqpoint{10.401306in}{4.785131in}}{\pgfqpoint{10.409119in}{4.792944in}}%
\pgfpathcurveto{\pgfqpoint{10.416933in}{4.800758in}}{\pgfqpoint{10.421323in}{4.811357in}}{\pgfqpoint{10.421323in}{4.822407in}}%
\pgfpathcurveto{\pgfqpoint{10.421323in}{4.833457in}}{\pgfqpoint{10.416933in}{4.844056in}}{\pgfqpoint{10.409119in}{4.851870in}}%
\pgfpathcurveto{\pgfqpoint{10.401306in}{4.859684in}}{\pgfqpoint{10.390707in}{4.864074in}}{\pgfqpoint{10.379657in}{4.864074in}}%
\pgfpathcurveto{\pgfqpoint{10.368607in}{4.864074in}}{\pgfqpoint{10.358008in}{4.859684in}}{\pgfqpoint{10.350194in}{4.851870in}}%
\pgfpathcurveto{\pgfqpoint{10.342380in}{4.844056in}}{\pgfqpoint{10.337990in}{4.833457in}}{\pgfqpoint{10.337990in}{4.822407in}}%
\pgfpathcurveto{\pgfqpoint{10.337990in}{4.811357in}}{\pgfqpoint{10.342380in}{4.800758in}}{\pgfqpoint{10.350194in}{4.792944in}}%
\pgfpathcurveto{\pgfqpoint{10.358008in}{4.785131in}}{\pgfqpoint{10.368607in}{4.780741in}}{\pgfqpoint{10.379657in}{4.780741in}}%
\pgfpathlineto{\pgfqpoint{10.379657in}{4.780741in}}%
\pgfpathclose%
\pgfusepath{stroke}%
\end{pgfscope}%
\begin{pgfscope}%
\pgfpathrectangle{\pgfqpoint{7.394209in}{0.375000in}}{\pgfqpoint{6.356833in}{5.175000in}}%
\pgfusepath{clip}%
\pgfsetbuttcap%
\pgfsetroundjoin%
\pgfsetlinewidth{1.003750pt}%
\definecolor{currentstroke}{rgb}{0.827451,0.827451,0.827451}%
\pgfsetstrokecolor{currentstroke}%
\pgfsetdash{}{0pt}%
\pgfpathmoveto{\pgfqpoint{9.753826in}{2.069544in}}%
\pgfpathcurveto{\pgfqpoint{9.764876in}{2.069544in}}{\pgfqpoint{9.775475in}{2.073934in}}{\pgfqpoint{9.783288in}{2.081748in}}%
\pgfpathcurveto{\pgfqpoint{9.791102in}{2.089561in}}{\pgfqpoint{9.795492in}{2.100160in}}{\pgfqpoint{9.795492in}{2.111210in}}%
\pgfpathcurveto{\pgfqpoint{9.795492in}{2.122260in}}{\pgfqpoint{9.791102in}{2.132859in}}{\pgfqpoint{9.783288in}{2.140673in}}%
\pgfpathcurveto{\pgfqpoint{9.775475in}{2.148487in}}{\pgfqpoint{9.764876in}{2.152877in}}{\pgfqpoint{9.753826in}{2.152877in}}%
\pgfpathcurveto{\pgfqpoint{9.742775in}{2.152877in}}{\pgfqpoint{9.732176in}{2.148487in}}{\pgfqpoint{9.724363in}{2.140673in}}%
\pgfpathcurveto{\pgfqpoint{9.716549in}{2.132859in}}{\pgfqpoint{9.712159in}{2.122260in}}{\pgfqpoint{9.712159in}{2.111210in}}%
\pgfpathcurveto{\pgfqpoint{9.712159in}{2.100160in}}{\pgfqpoint{9.716549in}{2.089561in}}{\pgfqpoint{9.724363in}{2.081748in}}%
\pgfpathcurveto{\pgfqpoint{9.732176in}{2.073934in}}{\pgfqpoint{9.742775in}{2.069544in}}{\pgfqpoint{9.753826in}{2.069544in}}%
\pgfpathlineto{\pgfqpoint{9.753826in}{2.069544in}}%
\pgfpathclose%
\pgfusepath{stroke}%
\end{pgfscope}%
\begin{pgfscope}%
\pgfpathrectangle{\pgfqpoint{7.394209in}{0.375000in}}{\pgfqpoint{6.356833in}{5.175000in}}%
\pgfusepath{clip}%
\pgfsetbuttcap%
\pgfsetroundjoin%
\pgfsetlinewidth{1.003750pt}%
\definecolor{currentstroke}{rgb}{0.827451,0.827451,0.827451}%
\pgfsetstrokecolor{currentstroke}%
\pgfsetdash{}{0pt}%
\pgfpathmoveto{\pgfqpoint{9.092078in}{3.456938in}}%
\pgfpathcurveto{\pgfqpoint{9.103129in}{3.456938in}}{\pgfqpoint{9.113728in}{3.461328in}}{\pgfqpoint{9.121541in}{3.469142in}}%
\pgfpathcurveto{\pgfqpoint{9.129355in}{3.476955in}}{\pgfqpoint{9.133745in}{3.487554in}}{\pgfqpoint{9.133745in}{3.498604in}}%
\pgfpathcurveto{\pgfqpoint{9.133745in}{3.509655in}}{\pgfqpoint{9.129355in}{3.520254in}}{\pgfqpoint{9.121541in}{3.528067in}}%
\pgfpathcurveto{\pgfqpoint{9.113728in}{3.535881in}}{\pgfqpoint{9.103129in}{3.540271in}}{\pgfqpoint{9.092078in}{3.540271in}}%
\pgfpathcurveto{\pgfqpoint{9.081028in}{3.540271in}}{\pgfqpoint{9.070429in}{3.535881in}}{\pgfqpoint{9.062616in}{3.528067in}}%
\pgfpathcurveto{\pgfqpoint{9.054802in}{3.520254in}}{\pgfqpoint{9.050412in}{3.509655in}}{\pgfqpoint{9.050412in}{3.498604in}}%
\pgfpathcurveto{\pgfqpoint{9.050412in}{3.487554in}}{\pgfqpoint{9.054802in}{3.476955in}}{\pgfqpoint{9.062616in}{3.469142in}}%
\pgfpathcurveto{\pgfqpoint{9.070429in}{3.461328in}}{\pgfqpoint{9.081028in}{3.456938in}}{\pgfqpoint{9.092078in}{3.456938in}}%
\pgfpathlineto{\pgfqpoint{9.092078in}{3.456938in}}%
\pgfpathclose%
\pgfusepath{stroke}%
\end{pgfscope}%
\begin{pgfscope}%
\pgfpathrectangle{\pgfqpoint{7.394209in}{0.375000in}}{\pgfqpoint{6.356833in}{5.175000in}}%
\pgfusepath{clip}%
\pgfsetbuttcap%
\pgfsetroundjoin%
\pgfsetlinewidth{1.003750pt}%
\definecolor{currentstroke}{rgb}{0.827451,0.827451,0.827451}%
\pgfsetstrokecolor{currentstroke}%
\pgfsetdash{}{0pt}%
\pgfpathmoveto{\pgfqpoint{13.211396in}{5.498811in}}%
\pgfpathcurveto{\pgfqpoint{13.222446in}{5.498811in}}{\pgfqpoint{13.233045in}{5.503201in}}{\pgfqpoint{13.240858in}{5.511015in}}%
\pgfpathcurveto{\pgfqpoint{13.248672in}{5.518829in}}{\pgfqpoint{13.253062in}{5.529428in}}{\pgfqpoint{13.253062in}{5.540478in}}%
\pgfpathcurveto{\pgfqpoint{13.253062in}{5.551528in}}{\pgfqpoint{13.248672in}{5.562127in}}{\pgfqpoint{13.240858in}{5.569940in}}%
\pgfpathcurveto{\pgfqpoint{13.233045in}{5.577754in}}{\pgfqpoint{13.222446in}{5.582144in}}{\pgfqpoint{13.211396in}{5.582144in}}%
\pgfpathcurveto{\pgfqpoint{13.200346in}{5.582144in}}{\pgfqpoint{13.189747in}{5.577754in}}{\pgfqpoint{13.181933in}{5.569940in}}%
\pgfpathcurveto{\pgfqpoint{13.174119in}{5.562127in}}{\pgfqpoint{13.169729in}{5.551528in}}{\pgfqpoint{13.169729in}{5.540478in}}%
\pgfpathcurveto{\pgfqpoint{13.169729in}{5.529428in}}{\pgfqpoint{13.174119in}{5.518829in}}{\pgfqpoint{13.181933in}{5.511015in}}%
\pgfpathcurveto{\pgfqpoint{13.189747in}{5.503201in}}{\pgfqpoint{13.200346in}{5.498811in}}{\pgfqpoint{13.211396in}{5.498811in}}%
\pgfpathlineto{\pgfqpoint{13.211396in}{5.498811in}}%
\pgfpathclose%
\pgfusepath{stroke}%
\end{pgfscope}%
\begin{pgfscope}%
\pgfpathrectangle{\pgfqpoint{7.394209in}{0.375000in}}{\pgfqpoint{6.356833in}{5.175000in}}%
\pgfusepath{clip}%
\pgfsetbuttcap%
\pgfsetroundjoin%
\pgfsetlinewidth{1.003750pt}%
\definecolor{currentstroke}{rgb}{0.827451,0.827451,0.827451}%
\pgfsetstrokecolor{currentstroke}%
\pgfsetdash{}{0pt}%
\pgfpathmoveto{\pgfqpoint{9.432795in}{2.773668in}}%
\pgfpathcurveto{\pgfqpoint{9.443845in}{2.773668in}}{\pgfqpoint{9.454444in}{2.778059in}}{\pgfqpoint{9.462258in}{2.785872in}}%
\pgfpathcurveto{\pgfqpoint{9.470071in}{2.793686in}}{\pgfqpoint{9.474461in}{2.804285in}}{\pgfqpoint{9.474461in}{2.815335in}}%
\pgfpathcurveto{\pgfqpoint{9.474461in}{2.826385in}}{\pgfqpoint{9.470071in}{2.836984in}}{\pgfqpoint{9.462258in}{2.844798in}}%
\pgfpathcurveto{\pgfqpoint{9.454444in}{2.852611in}}{\pgfqpoint{9.443845in}{2.857002in}}{\pgfqpoint{9.432795in}{2.857002in}}%
\pgfpathcurveto{\pgfqpoint{9.421745in}{2.857002in}}{\pgfqpoint{9.411146in}{2.852611in}}{\pgfqpoint{9.403332in}{2.844798in}}%
\pgfpathcurveto{\pgfqpoint{9.395518in}{2.836984in}}{\pgfqpoint{9.391128in}{2.826385in}}{\pgfqpoint{9.391128in}{2.815335in}}%
\pgfpathcurveto{\pgfqpoint{9.391128in}{2.804285in}}{\pgfqpoint{9.395518in}{2.793686in}}{\pgfqpoint{9.403332in}{2.785872in}}%
\pgfpathcurveto{\pgfqpoint{9.411146in}{2.778059in}}{\pgfqpoint{9.421745in}{2.773668in}}{\pgfqpoint{9.432795in}{2.773668in}}%
\pgfpathlineto{\pgfqpoint{9.432795in}{2.773668in}}%
\pgfpathclose%
\pgfusepath{stroke}%
\end{pgfscope}%
\begin{pgfscope}%
\pgfpathrectangle{\pgfqpoint{7.394209in}{0.375000in}}{\pgfqpoint{6.356833in}{5.175000in}}%
\pgfusepath{clip}%
\pgfsetbuttcap%
\pgfsetroundjoin%
\pgfsetlinewidth{1.003750pt}%
\definecolor{currentstroke}{rgb}{0.827451,0.827451,0.827451}%
\pgfsetstrokecolor{currentstroke}%
\pgfsetdash{}{0pt}%
\pgfpathmoveto{\pgfqpoint{10.166386in}{4.933490in}}%
\pgfpathcurveto{\pgfqpoint{10.177436in}{4.933490in}}{\pgfqpoint{10.188035in}{4.937881in}}{\pgfqpoint{10.195849in}{4.945694in}}%
\pgfpathcurveto{\pgfqpoint{10.203662in}{4.953508in}}{\pgfqpoint{10.208053in}{4.964107in}}{\pgfqpoint{10.208053in}{4.975157in}}%
\pgfpathcurveto{\pgfqpoint{10.208053in}{4.986207in}}{\pgfqpoint{10.203662in}{4.996806in}}{\pgfqpoint{10.195849in}{5.004620in}}%
\pgfpathcurveto{\pgfqpoint{10.188035in}{5.012434in}}{\pgfqpoint{10.177436in}{5.016824in}}{\pgfqpoint{10.166386in}{5.016824in}}%
\pgfpathcurveto{\pgfqpoint{10.155336in}{5.016824in}}{\pgfqpoint{10.144737in}{5.012434in}}{\pgfqpoint{10.136923in}{5.004620in}}%
\pgfpathcurveto{\pgfqpoint{10.129110in}{4.996806in}}{\pgfqpoint{10.124719in}{4.986207in}}{\pgfqpoint{10.124719in}{4.975157in}}%
\pgfpathcurveto{\pgfqpoint{10.124719in}{4.964107in}}{\pgfqpoint{10.129110in}{4.953508in}}{\pgfqpoint{10.136923in}{4.945694in}}%
\pgfpathcurveto{\pgfqpoint{10.144737in}{4.937881in}}{\pgfqpoint{10.155336in}{4.933490in}}{\pgfqpoint{10.166386in}{4.933490in}}%
\pgfpathlineto{\pgfqpoint{10.166386in}{4.933490in}}%
\pgfpathclose%
\pgfusepath{stroke}%
\end{pgfscope}%
\begin{pgfscope}%
\pgfpathrectangle{\pgfqpoint{7.394209in}{0.375000in}}{\pgfqpoint{6.356833in}{5.175000in}}%
\pgfusepath{clip}%
\pgfsetbuttcap%
\pgfsetroundjoin%
\pgfsetlinewidth{1.003750pt}%
\definecolor{currentstroke}{rgb}{0.827451,0.827451,0.827451}%
\pgfsetstrokecolor{currentstroke}%
\pgfsetdash{}{0pt}%
\pgfpathmoveto{\pgfqpoint{8.887015in}{1.727815in}}%
\pgfpathcurveto{\pgfqpoint{8.898066in}{1.727815in}}{\pgfqpoint{8.908665in}{1.732205in}}{\pgfqpoint{8.916478in}{1.740019in}}%
\pgfpathcurveto{\pgfqpoint{8.924292in}{1.747832in}}{\pgfqpoint{8.928682in}{1.758431in}}{\pgfqpoint{8.928682in}{1.769481in}}%
\pgfpathcurveto{\pgfqpoint{8.928682in}{1.780531in}}{\pgfqpoint{8.924292in}{1.791131in}}{\pgfqpoint{8.916478in}{1.798944in}}%
\pgfpathcurveto{\pgfqpoint{8.908665in}{1.806758in}}{\pgfqpoint{8.898066in}{1.811148in}}{\pgfqpoint{8.887015in}{1.811148in}}%
\pgfpathcurveto{\pgfqpoint{8.875965in}{1.811148in}}{\pgfqpoint{8.865366in}{1.806758in}}{\pgfqpoint{8.857553in}{1.798944in}}%
\pgfpathcurveto{\pgfqpoint{8.849739in}{1.791131in}}{\pgfqpoint{8.845349in}{1.780531in}}{\pgfqpoint{8.845349in}{1.769481in}}%
\pgfpathcurveto{\pgfqpoint{8.845349in}{1.758431in}}{\pgfqpoint{8.849739in}{1.747832in}}{\pgfqpoint{8.857553in}{1.740019in}}%
\pgfpathcurveto{\pgfqpoint{8.865366in}{1.732205in}}{\pgfqpoint{8.875965in}{1.727815in}}{\pgfqpoint{8.887015in}{1.727815in}}%
\pgfpathlineto{\pgfqpoint{8.887015in}{1.727815in}}%
\pgfpathclose%
\pgfusepath{stroke}%
\end{pgfscope}%
\begin{pgfscope}%
\pgfpathrectangle{\pgfqpoint{7.394209in}{0.375000in}}{\pgfqpoint{6.356833in}{5.175000in}}%
\pgfusepath{clip}%
\pgfsetbuttcap%
\pgfsetroundjoin%
\pgfsetlinewidth{1.003750pt}%
\definecolor{currentstroke}{rgb}{0.827451,0.827451,0.827451}%
\pgfsetstrokecolor{currentstroke}%
\pgfsetdash{}{0pt}%
\pgfpathmoveto{\pgfqpoint{9.450165in}{3.482117in}}%
\pgfpathcurveto{\pgfqpoint{9.461215in}{3.482117in}}{\pgfqpoint{9.471814in}{3.486508in}}{\pgfqpoint{9.479628in}{3.494321in}}%
\pgfpathcurveto{\pgfqpoint{9.487442in}{3.502135in}}{\pgfqpoint{9.491832in}{3.512734in}}{\pgfqpoint{9.491832in}{3.523784in}}%
\pgfpathcurveto{\pgfqpoint{9.491832in}{3.534834in}}{\pgfqpoint{9.487442in}{3.545433in}}{\pgfqpoint{9.479628in}{3.553247in}}%
\pgfpathcurveto{\pgfqpoint{9.471814in}{3.561060in}}{\pgfqpoint{9.461215in}{3.565451in}}{\pgfqpoint{9.450165in}{3.565451in}}%
\pgfpathcurveto{\pgfqpoint{9.439115in}{3.565451in}}{\pgfqpoint{9.428516in}{3.561060in}}{\pgfqpoint{9.420703in}{3.553247in}}%
\pgfpathcurveto{\pgfqpoint{9.412889in}{3.545433in}}{\pgfqpoint{9.408499in}{3.534834in}}{\pgfqpoint{9.408499in}{3.523784in}}%
\pgfpathcurveto{\pgfqpoint{9.408499in}{3.512734in}}{\pgfqpoint{9.412889in}{3.502135in}}{\pgfqpoint{9.420703in}{3.494321in}}%
\pgfpathcurveto{\pgfqpoint{9.428516in}{3.486508in}}{\pgfqpoint{9.439115in}{3.482117in}}{\pgfqpoint{9.450165in}{3.482117in}}%
\pgfpathlineto{\pgfqpoint{9.450165in}{3.482117in}}%
\pgfpathclose%
\pgfusepath{stroke}%
\end{pgfscope}%
\begin{pgfscope}%
\pgfpathrectangle{\pgfqpoint{7.394209in}{0.375000in}}{\pgfqpoint{6.356833in}{5.175000in}}%
\pgfusepath{clip}%
\pgfsetbuttcap%
\pgfsetroundjoin%
\pgfsetlinewidth{1.003750pt}%
\definecolor{currentstroke}{rgb}{0.827451,0.827451,0.827451}%
\pgfsetstrokecolor{currentstroke}%
\pgfsetdash{}{0pt}%
\pgfpathmoveto{\pgfqpoint{10.379657in}{5.045949in}}%
\pgfpathcurveto{\pgfqpoint{10.390707in}{5.045949in}}{\pgfqpoint{10.401306in}{5.050339in}}{\pgfqpoint{10.409119in}{5.058153in}}%
\pgfpathcurveto{\pgfqpoint{10.416933in}{5.065966in}}{\pgfqpoint{10.421323in}{5.076565in}}{\pgfqpoint{10.421323in}{5.087616in}}%
\pgfpathcurveto{\pgfqpoint{10.421323in}{5.098666in}}{\pgfqpoint{10.416933in}{5.109265in}}{\pgfqpoint{10.409119in}{5.117078in}}%
\pgfpathcurveto{\pgfqpoint{10.401306in}{5.124892in}}{\pgfqpoint{10.390707in}{5.129282in}}{\pgfqpoint{10.379657in}{5.129282in}}%
\pgfpathcurveto{\pgfqpoint{10.368607in}{5.129282in}}{\pgfqpoint{10.358008in}{5.124892in}}{\pgfqpoint{10.350194in}{5.117078in}}%
\pgfpathcurveto{\pgfqpoint{10.342380in}{5.109265in}}{\pgfqpoint{10.337990in}{5.098666in}}{\pgfqpoint{10.337990in}{5.087616in}}%
\pgfpathcurveto{\pgfqpoint{10.337990in}{5.076565in}}{\pgfqpoint{10.342380in}{5.065966in}}{\pgfqpoint{10.350194in}{5.058153in}}%
\pgfpathcurveto{\pgfqpoint{10.358008in}{5.050339in}}{\pgfqpoint{10.368607in}{5.045949in}}{\pgfqpoint{10.379657in}{5.045949in}}%
\pgfpathlineto{\pgfqpoint{10.379657in}{5.045949in}}%
\pgfpathclose%
\pgfusepath{stroke}%
\end{pgfscope}%
\begin{pgfscope}%
\pgfpathrectangle{\pgfqpoint{7.394209in}{0.375000in}}{\pgfqpoint{6.356833in}{5.175000in}}%
\pgfusepath{clip}%
\pgfsetbuttcap%
\pgfsetroundjoin%
\pgfsetlinewidth{1.003750pt}%
\definecolor{currentstroke}{rgb}{0.827451,0.827451,0.827451}%
\pgfsetstrokecolor{currentstroke}%
\pgfsetdash{}{0pt}%
\pgfpathmoveto{\pgfqpoint{9.817025in}{3.225670in}}%
\pgfpathcurveto{\pgfqpoint{9.828075in}{3.225670in}}{\pgfqpoint{9.838674in}{3.230060in}}{\pgfqpoint{9.846488in}{3.237874in}}%
\pgfpathcurveto{\pgfqpoint{9.854301in}{3.245688in}}{\pgfqpoint{9.858691in}{3.256287in}}{\pgfqpoint{9.858691in}{3.267337in}}%
\pgfpathcurveto{\pgfqpoint{9.858691in}{3.278387in}}{\pgfqpoint{9.854301in}{3.288986in}}{\pgfqpoint{9.846488in}{3.296799in}}%
\pgfpathcurveto{\pgfqpoint{9.838674in}{3.304613in}}{\pgfqpoint{9.828075in}{3.309003in}}{\pgfqpoint{9.817025in}{3.309003in}}%
\pgfpathcurveto{\pgfqpoint{9.805975in}{3.309003in}}{\pgfqpoint{9.795376in}{3.304613in}}{\pgfqpoint{9.787562in}{3.296799in}}%
\pgfpathcurveto{\pgfqpoint{9.779748in}{3.288986in}}{\pgfqpoint{9.775358in}{3.278387in}}{\pgfqpoint{9.775358in}{3.267337in}}%
\pgfpathcurveto{\pgfqpoint{9.775358in}{3.256287in}}{\pgfqpoint{9.779748in}{3.245688in}}{\pgfqpoint{9.787562in}{3.237874in}}%
\pgfpathcurveto{\pgfqpoint{9.795376in}{3.230060in}}{\pgfqpoint{9.805975in}{3.225670in}}{\pgfqpoint{9.817025in}{3.225670in}}%
\pgfpathlineto{\pgfqpoint{9.817025in}{3.225670in}}%
\pgfpathclose%
\pgfusepath{stroke}%
\end{pgfscope}%
\begin{pgfscope}%
\pgfpathrectangle{\pgfqpoint{7.394209in}{0.375000in}}{\pgfqpoint{6.356833in}{5.175000in}}%
\pgfusepath{clip}%
\pgfsetbuttcap%
\pgfsetroundjoin%
\pgfsetlinewidth{1.003750pt}%
\definecolor{currentstroke}{rgb}{0.827451,0.827451,0.827451}%
\pgfsetstrokecolor{currentstroke}%
\pgfsetdash{}{0pt}%
\pgfpathmoveto{\pgfqpoint{8.876873in}{2.769368in}}%
\pgfpathcurveto{\pgfqpoint{8.887924in}{2.769368in}}{\pgfqpoint{8.898523in}{2.773759in}}{\pgfqpoint{8.906336in}{2.781572in}}%
\pgfpathcurveto{\pgfqpoint{8.914150in}{2.789386in}}{\pgfqpoint{8.918540in}{2.799985in}}{\pgfqpoint{8.918540in}{2.811035in}}%
\pgfpathcurveto{\pgfqpoint{8.918540in}{2.822085in}}{\pgfqpoint{8.914150in}{2.832684in}}{\pgfqpoint{8.906336in}{2.840498in}}%
\pgfpathcurveto{\pgfqpoint{8.898523in}{2.848311in}}{\pgfqpoint{8.887924in}{2.852702in}}{\pgfqpoint{8.876873in}{2.852702in}}%
\pgfpathcurveto{\pgfqpoint{8.865823in}{2.852702in}}{\pgfqpoint{8.855224in}{2.848311in}}{\pgfqpoint{8.847411in}{2.840498in}}%
\pgfpathcurveto{\pgfqpoint{8.839597in}{2.832684in}}{\pgfqpoint{8.835207in}{2.822085in}}{\pgfqpoint{8.835207in}{2.811035in}}%
\pgfpathcurveto{\pgfqpoint{8.835207in}{2.799985in}}{\pgfqpoint{8.839597in}{2.789386in}}{\pgfqpoint{8.847411in}{2.781572in}}%
\pgfpathcurveto{\pgfqpoint{8.855224in}{2.773759in}}{\pgfqpoint{8.865823in}{2.769368in}}{\pgfqpoint{8.876873in}{2.769368in}}%
\pgfpathlineto{\pgfqpoint{8.876873in}{2.769368in}}%
\pgfpathclose%
\pgfusepath{stroke}%
\end{pgfscope}%
\begin{pgfscope}%
\pgfpathrectangle{\pgfqpoint{7.394209in}{0.375000in}}{\pgfqpoint{6.356833in}{5.175000in}}%
\pgfusepath{clip}%
\pgfsetbuttcap%
\pgfsetroundjoin%
\pgfsetlinewidth{1.003750pt}%
\definecolor{currentstroke}{rgb}{0.827451,0.827451,0.827451}%
\pgfsetstrokecolor{currentstroke}%
\pgfsetdash{}{0pt}%
\pgfpathmoveto{\pgfqpoint{11.271207in}{5.187759in}}%
\pgfpathcurveto{\pgfqpoint{11.282257in}{5.187759in}}{\pgfqpoint{11.292856in}{5.192149in}}{\pgfqpoint{11.300670in}{5.199963in}}%
\pgfpathcurveto{\pgfqpoint{11.308484in}{5.207776in}}{\pgfqpoint{11.312874in}{5.218376in}}{\pgfqpoint{11.312874in}{5.229426in}}%
\pgfpathcurveto{\pgfqpoint{11.312874in}{5.240476in}}{\pgfqpoint{11.308484in}{5.251075in}}{\pgfqpoint{11.300670in}{5.258888in}}%
\pgfpathcurveto{\pgfqpoint{11.292856in}{5.266702in}}{\pgfqpoint{11.282257in}{5.271092in}}{\pgfqpoint{11.271207in}{5.271092in}}%
\pgfpathcurveto{\pgfqpoint{11.260157in}{5.271092in}}{\pgfqpoint{11.249558in}{5.266702in}}{\pgfqpoint{11.241745in}{5.258888in}}%
\pgfpathcurveto{\pgfqpoint{11.233931in}{5.251075in}}{\pgfqpoint{11.229541in}{5.240476in}}{\pgfqpoint{11.229541in}{5.229426in}}%
\pgfpathcurveto{\pgfqpoint{11.229541in}{5.218376in}}{\pgfqpoint{11.233931in}{5.207776in}}{\pgfqpoint{11.241745in}{5.199963in}}%
\pgfpathcurveto{\pgfqpoint{11.249558in}{5.192149in}}{\pgfqpoint{11.260157in}{5.187759in}}{\pgfqpoint{11.271207in}{5.187759in}}%
\pgfpathlineto{\pgfqpoint{11.271207in}{5.187759in}}%
\pgfpathclose%
\pgfusepath{stroke}%
\end{pgfscope}%
\begin{pgfscope}%
\pgfpathrectangle{\pgfqpoint{7.394209in}{0.375000in}}{\pgfqpoint{6.356833in}{5.175000in}}%
\pgfusepath{clip}%
\pgfsetbuttcap%
\pgfsetroundjoin%
\pgfsetlinewidth{1.003750pt}%
\definecolor{currentstroke}{rgb}{0.827451,0.827451,0.827451}%
\pgfsetstrokecolor{currentstroke}%
\pgfsetdash{}{0pt}%
\pgfpathmoveto{\pgfqpoint{12.079577in}{5.444631in}}%
\pgfpathcurveto{\pgfqpoint{12.090627in}{5.444631in}}{\pgfqpoint{12.101226in}{5.449021in}}{\pgfqpoint{12.109040in}{5.456835in}}%
\pgfpathcurveto{\pgfqpoint{12.116853in}{5.464648in}}{\pgfqpoint{12.121244in}{5.475247in}}{\pgfqpoint{12.121244in}{5.486297in}}%
\pgfpathcurveto{\pgfqpoint{12.121244in}{5.497348in}}{\pgfqpoint{12.116853in}{5.507947in}}{\pgfqpoint{12.109040in}{5.515760in}}%
\pgfpathcurveto{\pgfqpoint{12.101226in}{5.523574in}}{\pgfqpoint{12.090627in}{5.527964in}}{\pgfqpoint{12.079577in}{5.527964in}}%
\pgfpathcurveto{\pgfqpoint{12.068527in}{5.527964in}}{\pgfqpoint{12.057928in}{5.523574in}}{\pgfqpoint{12.050114in}{5.515760in}}%
\pgfpathcurveto{\pgfqpoint{12.042301in}{5.507947in}}{\pgfqpoint{12.037910in}{5.497348in}}{\pgfqpoint{12.037910in}{5.486297in}}%
\pgfpathcurveto{\pgfqpoint{12.037910in}{5.475247in}}{\pgfqpoint{12.042301in}{5.464648in}}{\pgfqpoint{12.050114in}{5.456835in}}%
\pgfpathcurveto{\pgfqpoint{12.057928in}{5.449021in}}{\pgfqpoint{12.068527in}{5.444631in}}{\pgfqpoint{12.079577in}{5.444631in}}%
\pgfpathlineto{\pgfqpoint{12.079577in}{5.444631in}}%
\pgfpathclose%
\pgfusepath{stroke}%
\end{pgfscope}%
\begin{pgfscope}%
\pgfpathrectangle{\pgfqpoint{7.394209in}{0.375000in}}{\pgfqpoint{6.356833in}{5.175000in}}%
\pgfusepath{clip}%
\pgfsetbuttcap%
\pgfsetroundjoin%
\pgfsetlinewidth{1.003750pt}%
\definecolor{currentstroke}{rgb}{0.827451,0.827451,0.827451}%
\pgfsetstrokecolor{currentstroke}%
\pgfsetdash{}{0pt}%
\pgfpathmoveto{\pgfqpoint{13.597439in}{5.502952in}}%
\pgfpathcurveto{\pgfqpoint{13.608489in}{5.502952in}}{\pgfqpoint{13.619088in}{5.507342in}}{\pgfqpoint{13.626902in}{5.515156in}}%
\pgfpathcurveto{\pgfqpoint{13.634715in}{5.522969in}}{\pgfqpoint{13.639106in}{5.533568in}}{\pgfqpoint{13.639106in}{5.544618in}}%
\pgfpathcurveto{\pgfqpoint{13.639106in}{5.555669in}}{\pgfqpoint{13.634715in}{5.566268in}}{\pgfqpoint{13.626902in}{5.574081in}}%
\pgfpathcurveto{\pgfqpoint{13.619088in}{5.581895in}}{\pgfqpoint{13.608489in}{5.586285in}}{\pgfqpoint{13.597439in}{5.586285in}}%
\pgfpathcurveto{\pgfqpoint{13.586389in}{5.586285in}}{\pgfqpoint{13.575790in}{5.581895in}}{\pgfqpoint{13.567976in}{5.574081in}}%
\pgfpathcurveto{\pgfqpoint{13.560162in}{5.566268in}}{\pgfqpoint{13.555772in}{5.555669in}}{\pgfqpoint{13.555772in}{5.544618in}}%
\pgfpathcurveto{\pgfqpoint{13.555772in}{5.533568in}}{\pgfqpoint{13.560162in}{5.522969in}}{\pgfqpoint{13.567976in}{5.515156in}}%
\pgfpathcurveto{\pgfqpoint{13.575790in}{5.507342in}}{\pgfqpoint{13.586389in}{5.502952in}}{\pgfqpoint{13.597439in}{5.502952in}}%
\pgfpathlineto{\pgfqpoint{13.597439in}{5.502952in}}%
\pgfpathclose%
\pgfusepath{stroke}%
\end{pgfscope}%
\begin{pgfscope}%
\pgfpathrectangle{\pgfqpoint{7.394209in}{0.375000in}}{\pgfqpoint{6.356833in}{5.175000in}}%
\pgfusepath{clip}%
\pgfsetbuttcap%
\pgfsetroundjoin%
\pgfsetlinewidth{1.003750pt}%
\definecolor{currentstroke}{rgb}{0.827451,0.827451,0.827451}%
\pgfsetstrokecolor{currentstroke}%
\pgfsetdash{}{0pt}%
\pgfpathmoveto{\pgfqpoint{10.478735in}{5.114344in}}%
\pgfpathcurveto{\pgfqpoint{10.489785in}{5.114344in}}{\pgfqpoint{10.500384in}{5.118735in}}{\pgfqpoint{10.508198in}{5.126548in}}%
\pgfpathcurveto{\pgfqpoint{10.516011in}{5.134362in}}{\pgfqpoint{10.520402in}{5.144961in}}{\pgfqpoint{10.520402in}{5.156011in}}%
\pgfpathcurveto{\pgfqpoint{10.520402in}{5.167061in}}{\pgfqpoint{10.516011in}{5.177660in}}{\pgfqpoint{10.508198in}{5.185474in}}%
\pgfpathcurveto{\pgfqpoint{10.500384in}{5.193287in}}{\pgfqpoint{10.489785in}{5.197678in}}{\pgfqpoint{10.478735in}{5.197678in}}%
\pgfpathcurveto{\pgfqpoint{10.467685in}{5.197678in}}{\pgfqpoint{10.457086in}{5.193287in}}{\pgfqpoint{10.449272in}{5.185474in}}%
\pgfpathcurveto{\pgfqpoint{10.441458in}{5.177660in}}{\pgfqpoint{10.437068in}{5.167061in}}{\pgfqpoint{10.437068in}{5.156011in}}%
\pgfpathcurveto{\pgfqpoint{10.437068in}{5.144961in}}{\pgfqpoint{10.441458in}{5.134362in}}{\pgfqpoint{10.449272in}{5.126548in}}%
\pgfpathcurveto{\pgfqpoint{10.457086in}{5.118735in}}{\pgfqpoint{10.467685in}{5.114344in}}{\pgfqpoint{10.478735in}{5.114344in}}%
\pgfpathlineto{\pgfqpoint{10.478735in}{5.114344in}}%
\pgfpathclose%
\pgfusepath{stroke}%
\end{pgfscope}%
\begin{pgfscope}%
\pgfpathrectangle{\pgfqpoint{7.394209in}{0.375000in}}{\pgfqpoint{6.356833in}{5.175000in}}%
\pgfusepath{clip}%
\pgfsetbuttcap%
\pgfsetroundjoin%
\pgfsetlinewidth{1.003750pt}%
\definecolor{currentstroke}{rgb}{0.827451,0.827451,0.827451}%
\pgfsetstrokecolor{currentstroke}%
\pgfsetdash{}{0pt}%
\pgfpathmoveto{\pgfqpoint{10.311256in}{3.278763in}}%
\pgfpathcurveto{\pgfqpoint{10.322306in}{3.278763in}}{\pgfqpoint{10.332905in}{3.283153in}}{\pgfqpoint{10.340719in}{3.290967in}}%
\pgfpathcurveto{\pgfqpoint{10.348533in}{3.298781in}}{\pgfqpoint{10.352923in}{3.309380in}}{\pgfqpoint{10.352923in}{3.320430in}}%
\pgfpathcurveto{\pgfqpoint{10.352923in}{3.331480in}}{\pgfqpoint{10.348533in}{3.342079in}}{\pgfqpoint{10.340719in}{3.349893in}}%
\pgfpathcurveto{\pgfqpoint{10.332905in}{3.357706in}}{\pgfqpoint{10.322306in}{3.362096in}}{\pgfqpoint{10.311256in}{3.362096in}}%
\pgfpathcurveto{\pgfqpoint{10.300206in}{3.362096in}}{\pgfqpoint{10.289607in}{3.357706in}}{\pgfqpoint{10.281793in}{3.349893in}}%
\pgfpathcurveto{\pgfqpoint{10.273980in}{3.342079in}}{\pgfqpoint{10.269590in}{3.331480in}}{\pgfqpoint{10.269590in}{3.320430in}}%
\pgfpathcurveto{\pgfqpoint{10.269590in}{3.309380in}}{\pgfqpoint{10.273980in}{3.298781in}}{\pgfqpoint{10.281793in}{3.290967in}}%
\pgfpathcurveto{\pgfqpoint{10.289607in}{3.283153in}}{\pgfqpoint{10.300206in}{3.278763in}}{\pgfqpoint{10.311256in}{3.278763in}}%
\pgfpathlineto{\pgfqpoint{10.311256in}{3.278763in}}%
\pgfpathclose%
\pgfusepath{stroke}%
\end{pgfscope}%
\begin{pgfscope}%
\pgfpathrectangle{\pgfqpoint{7.394209in}{0.375000in}}{\pgfqpoint{6.356833in}{5.175000in}}%
\pgfusepath{clip}%
\pgfsetbuttcap%
\pgfsetroundjoin%
\pgfsetlinewidth{1.003750pt}%
\definecolor{currentstroke}{rgb}{0.827451,0.827451,0.827451}%
\pgfsetstrokecolor{currentstroke}%
\pgfsetdash{}{0pt}%
\pgfpathmoveto{\pgfqpoint{9.138688in}{3.321622in}}%
\pgfpathcurveto{\pgfqpoint{9.149738in}{3.321622in}}{\pgfqpoint{9.160338in}{3.326012in}}{\pgfqpoint{9.168151in}{3.333826in}}%
\pgfpathcurveto{\pgfqpoint{9.175965in}{3.341639in}}{\pgfqpoint{9.180355in}{3.352238in}}{\pgfqpoint{9.180355in}{3.363288in}}%
\pgfpathcurveto{\pgfqpoint{9.180355in}{3.374339in}}{\pgfqpoint{9.175965in}{3.384938in}}{\pgfqpoint{9.168151in}{3.392751in}}%
\pgfpathcurveto{\pgfqpoint{9.160338in}{3.400565in}}{\pgfqpoint{9.149738in}{3.404955in}}{\pgfqpoint{9.138688in}{3.404955in}}%
\pgfpathcurveto{\pgfqpoint{9.127638in}{3.404955in}}{\pgfqpoint{9.117039in}{3.400565in}}{\pgfqpoint{9.109226in}{3.392751in}}%
\pgfpathcurveto{\pgfqpoint{9.101412in}{3.384938in}}{\pgfqpoint{9.097022in}{3.374339in}}{\pgfqpoint{9.097022in}{3.363288in}}%
\pgfpathcurveto{\pgfqpoint{9.097022in}{3.352238in}}{\pgfqpoint{9.101412in}{3.341639in}}{\pgfqpoint{9.109226in}{3.333826in}}%
\pgfpathcurveto{\pgfqpoint{9.117039in}{3.326012in}}{\pgfqpoint{9.127638in}{3.321622in}}{\pgfqpoint{9.138688in}{3.321622in}}%
\pgfpathlineto{\pgfqpoint{9.138688in}{3.321622in}}%
\pgfpathclose%
\pgfusepath{stroke}%
\end{pgfscope}%
\begin{pgfscope}%
\pgfpathrectangle{\pgfqpoint{7.394209in}{0.375000in}}{\pgfqpoint{6.356833in}{5.175000in}}%
\pgfusepath{clip}%
\pgfsetbuttcap%
\pgfsetroundjoin%
\pgfsetlinewidth{1.003750pt}%
\definecolor{currentstroke}{rgb}{0.827451,0.827451,0.827451}%
\pgfsetstrokecolor{currentstroke}%
\pgfsetdash{}{0pt}%
\pgfpathmoveto{\pgfqpoint{9.375490in}{2.380577in}}%
\pgfpathcurveto{\pgfqpoint{9.386540in}{2.380577in}}{\pgfqpoint{9.397139in}{2.384967in}}{\pgfqpoint{9.404952in}{2.392780in}}%
\pgfpathcurveto{\pgfqpoint{9.412766in}{2.400594in}}{\pgfqpoint{9.417156in}{2.411193in}}{\pgfqpoint{9.417156in}{2.422243in}}%
\pgfpathcurveto{\pgfqpoint{9.417156in}{2.433293in}}{\pgfqpoint{9.412766in}{2.443892in}}{\pgfqpoint{9.404952in}{2.451706in}}%
\pgfpathcurveto{\pgfqpoint{9.397139in}{2.459520in}}{\pgfqpoint{9.386540in}{2.463910in}}{\pgfqpoint{9.375490in}{2.463910in}}%
\pgfpathcurveto{\pgfqpoint{9.364439in}{2.463910in}}{\pgfqpoint{9.353840in}{2.459520in}}{\pgfqpoint{9.346027in}{2.451706in}}%
\pgfpathcurveto{\pgfqpoint{9.338213in}{2.443892in}}{\pgfqpoint{9.333823in}{2.433293in}}{\pgfqpoint{9.333823in}{2.422243in}}%
\pgfpathcurveto{\pgfqpoint{9.333823in}{2.411193in}}{\pgfqpoint{9.338213in}{2.400594in}}{\pgfqpoint{9.346027in}{2.392780in}}%
\pgfpathcurveto{\pgfqpoint{9.353840in}{2.384967in}}{\pgfqpoint{9.364439in}{2.380577in}}{\pgfqpoint{9.375490in}{2.380577in}}%
\pgfpathlineto{\pgfqpoint{9.375490in}{2.380577in}}%
\pgfpathclose%
\pgfusepath{stroke}%
\end{pgfscope}%
\begin{pgfscope}%
\pgfpathrectangle{\pgfqpoint{7.394209in}{0.375000in}}{\pgfqpoint{6.356833in}{5.175000in}}%
\pgfusepath{clip}%
\pgfsetbuttcap%
\pgfsetroundjoin%
\pgfsetlinewidth{1.003750pt}%
\definecolor{currentstroke}{rgb}{0.827451,0.827451,0.827451}%
\pgfsetstrokecolor{currentstroke}%
\pgfsetdash{}{0pt}%
\pgfpathmoveto{\pgfqpoint{13.081922in}{5.473877in}}%
\pgfpathcurveto{\pgfqpoint{13.092973in}{5.473877in}}{\pgfqpoint{13.103572in}{5.478267in}}{\pgfqpoint{13.111385in}{5.486081in}}%
\pgfpathcurveto{\pgfqpoint{13.119199in}{5.493895in}}{\pgfqpoint{13.123589in}{5.504494in}}{\pgfqpoint{13.123589in}{5.515544in}}%
\pgfpathcurveto{\pgfqpoint{13.123589in}{5.526594in}}{\pgfqpoint{13.119199in}{5.537193in}}{\pgfqpoint{13.111385in}{5.545007in}}%
\pgfpathcurveto{\pgfqpoint{13.103572in}{5.552820in}}{\pgfqpoint{13.092973in}{5.557210in}}{\pgfqpoint{13.081922in}{5.557210in}}%
\pgfpathcurveto{\pgfqpoint{13.070872in}{5.557210in}}{\pgfqpoint{13.060273in}{5.552820in}}{\pgfqpoint{13.052460in}{5.545007in}}%
\pgfpathcurveto{\pgfqpoint{13.044646in}{5.537193in}}{\pgfqpoint{13.040256in}{5.526594in}}{\pgfqpoint{13.040256in}{5.515544in}}%
\pgfpathcurveto{\pgfqpoint{13.040256in}{5.504494in}}{\pgfqpoint{13.044646in}{5.493895in}}{\pgfqpoint{13.052460in}{5.486081in}}%
\pgfpathcurveto{\pgfqpoint{13.060273in}{5.478267in}}{\pgfqpoint{13.070872in}{5.473877in}}{\pgfqpoint{13.081922in}{5.473877in}}%
\pgfpathlineto{\pgfqpoint{13.081922in}{5.473877in}}%
\pgfpathclose%
\pgfusepath{stroke}%
\end{pgfscope}%
\begin{pgfscope}%
\pgfpathrectangle{\pgfqpoint{7.394209in}{0.375000in}}{\pgfqpoint{6.356833in}{5.175000in}}%
\pgfusepath{clip}%
\pgfsetbuttcap%
\pgfsetroundjoin%
\pgfsetlinewidth{1.003750pt}%
\definecolor{currentstroke}{rgb}{0.827451,0.827451,0.827451}%
\pgfsetstrokecolor{currentstroke}%
\pgfsetdash{}{0pt}%
\pgfpathmoveto{\pgfqpoint{12.296381in}{5.421733in}}%
\pgfpathcurveto{\pgfqpoint{12.307431in}{5.421733in}}{\pgfqpoint{12.318030in}{5.426124in}}{\pgfqpoint{12.325843in}{5.433937in}}%
\pgfpathcurveto{\pgfqpoint{12.333657in}{5.441751in}}{\pgfqpoint{12.338047in}{5.452350in}}{\pgfqpoint{12.338047in}{5.463400in}}%
\pgfpathcurveto{\pgfqpoint{12.338047in}{5.474450in}}{\pgfqpoint{12.333657in}{5.485049in}}{\pgfqpoint{12.325843in}{5.492863in}}%
\pgfpathcurveto{\pgfqpoint{12.318030in}{5.500676in}}{\pgfqpoint{12.307431in}{5.505067in}}{\pgfqpoint{12.296381in}{5.505067in}}%
\pgfpathcurveto{\pgfqpoint{12.285330in}{5.505067in}}{\pgfqpoint{12.274731in}{5.500676in}}{\pgfqpoint{12.266918in}{5.492863in}}%
\pgfpathcurveto{\pgfqpoint{12.259104in}{5.485049in}}{\pgfqpoint{12.254714in}{5.474450in}}{\pgfqpoint{12.254714in}{5.463400in}}%
\pgfpathcurveto{\pgfqpoint{12.254714in}{5.452350in}}{\pgfqpoint{12.259104in}{5.441751in}}{\pgfqpoint{12.266918in}{5.433937in}}%
\pgfpathcurveto{\pgfqpoint{12.274731in}{5.426124in}}{\pgfqpoint{12.285330in}{5.421733in}}{\pgfqpoint{12.296381in}{5.421733in}}%
\pgfpathlineto{\pgfqpoint{12.296381in}{5.421733in}}%
\pgfpathclose%
\pgfusepath{stroke}%
\end{pgfscope}%
\begin{pgfscope}%
\pgfpathrectangle{\pgfqpoint{7.394209in}{0.375000in}}{\pgfqpoint{6.356833in}{5.175000in}}%
\pgfusepath{clip}%
\pgfsetbuttcap%
\pgfsetroundjoin%
\pgfsetlinewidth{1.003750pt}%
\definecolor{currentstroke}{rgb}{0.827451,0.827451,0.827451}%
\pgfsetstrokecolor{currentstroke}%
\pgfsetdash{}{0pt}%
\pgfpathmoveto{\pgfqpoint{7.794945in}{0.550177in}}%
\pgfpathcurveto{\pgfqpoint{7.805995in}{0.550177in}}{\pgfqpoint{7.816594in}{0.554568in}}{\pgfqpoint{7.824408in}{0.562381in}}%
\pgfpathcurveto{\pgfqpoint{7.832222in}{0.570195in}}{\pgfqpoint{7.836612in}{0.580794in}}{\pgfqpoint{7.836612in}{0.591844in}}%
\pgfpathcurveto{\pgfqpoint{7.836612in}{0.602894in}}{\pgfqpoint{7.832222in}{0.613493in}}{\pgfqpoint{7.824408in}{0.621307in}}%
\pgfpathcurveto{\pgfqpoint{7.816594in}{0.629120in}}{\pgfqpoint{7.805995in}{0.633511in}}{\pgfqpoint{7.794945in}{0.633511in}}%
\pgfpathcurveto{\pgfqpoint{7.783895in}{0.633511in}}{\pgfqpoint{7.773296in}{0.629120in}}{\pgfqpoint{7.765482in}{0.621307in}}%
\pgfpathcurveto{\pgfqpoint{7.757669in}{0.613493in}}{\pgfqpoint{7.753279in}{0.602894in}}{\pgfqpoint{7.753279in}{0.591844in}}%
\pgfpathcurveto{\pgfqpoint{7.753279in}{0.580794in}}{\pgfqpoint{7.757669in}{0.570195in}}{\pgfqpoint{7.765482in}{0.562381in}}%
\pgfpathcurveto{\pgfqpoint{7.773296in}{0.554568in}}{\pgfqpoint{7.783895in}{0.550177in}}{\pgfqpoint{7.794945in}{0.550177in}}%
\pgfpathlineto{\pgfqpoint{7.794945in}{0.550177in}}%
\pgfpathclose%
\pgfusepath{stroke}%
\end{pgfscope}%
\begin{pgfscope}%
\pgfpathrectangle{\pgfqpoint{7.394209in}{0.375000in}}{\pgfqpoint{6.356833in}{5.175000in}}%
\pgfusepath{clip}%
\pgfsetbuttcap%
\pgfsetroundjoin%
\pgfsetlinewidth{1.003750pt}%
\definecolor{currentstroke}{rgb}{0.827451,0.827451,0.827451}%
\pgfsetstrokecolor{currentstroke}%
\pgfsetdash{}{0pt}%
\pgfpathmoveto{\pgfqpoint{10.996320in}{4.418628in}}%
\pgfpathcurveto{\pgfqpoint{11.007370in}{4.418628in}}{\pgfqpoint{11.017969in}{4.423019in}}{\pgfqpoint{11.025783in}{4.430832in}}%
\pgfpathcurveto{\pgfqpoint{11.033596in}{4.438646in}}{\pgfqpoint{11.037987in}{4.449245in}}{\pgfqpoint{11.037987in}{4.460295in}}%
\pgfpathcurveto{\pgfqpoint{11.037987in}{4.471345in}}{\pgfqpoint{11.033596in}{4.481944in}}{\pgfqpoint{11.025783in}{4.489758in}}%
\pgfpathcurveto{\pgfqpoint{11.017969in}{4.497571in}}{\pgfqpoint{11.007370in}{4.501962in}}{\pgfqpoint{10.996320in}{4.501962in}}%
\pgfpathcurveto{\pgfqpoint{10.985270in}{4.501962in}}{\pgfqpoint{10.974671in}{4.497571in}}{\pgfqpoint{10.966857in}{4.489758in}}%
\pgfpathcurveto{\pgfqpoint{10.959044in}{4.481944in}}{\pgfqpoint{10.954653in}{4.471345in}}{\pgfqpoint{10.954653in}{4.460295in}}%
\pgfpathcurveto{\pgfqpoint{10.954653in}{4.449245in}}{\pgfqpoint{10.959044in}{4.438646in}}{\pgfqpoint{10.966857in}{4.430832in}}%
\pgfpathcurveto{\pgfqpoint{10.974671in}{4.423019in}}{\pgfqpoint{10.985270in}{4.418628in}}{\pgfqpoint{10.996320in}{4.418628in}}%
\pgfpathlineto{\pgfqpoint{10.996320in}{4.418628in}}%
\pgfpathclose%
\pgfusepath{stroke}%
\end{pgfscope}%
\begin{pgfscope}%
\pgfpathrectangle{\pgfqpoint{7.394209in}{0.375000in}}{\pgfqpoint{6.356833in}{5.175000in}}%
\pgfusepath{clip}%
\pgfsetbuttcap%
\pgfsetroundjoin%
\pgfsetlinewidth{1.003750pt}%
\definecolor{currentstroke}{rgb}{0.827451,0.827451,0.827451}%
\pgfsetstrokecolor{currentstroke}%
\pgfsetdash{}{0pt}%
\pgfpathmoveto{\pgfqpoint{10.327373in}{5.114344in}}%
\pgfpathcurveto{\pgfqpoint{10.338423in}{5.114344in}}{\pgfqpoint{10.349022in}{5.118735in}}{\pgfqpoint{10.356835in}{5.126548in}}%
\pgfpathcurveto{\pgfqpoint{10.364649in}{5.134362in}}{\pgfqpoint{10.369039in}{5.144961in}}{\pgfqpoint{10.369039in}{5.156011in}}%
\pgfpathcurveto{\pgfqpoint{10.369039in}{5.167061in}}{\pgfqpoint{10.364649in}{5.177660in}}{\pgfqpoint{10.356835in}{5.185474in}}%
\pgfpathcurveto{\pgfqpoint{10.349022in}{5.193287in}}{\pgfqpoint{10.338423in}{5.197678in}}{\pgfqpoint{10.327373in}{5.197678in}}%
\pgfpathcurveto{\pgfqpoint{10.316322in}{5.197678in}}{\pgfqpoint{10.305723in}{5.193287in}}{\pgfqpoint{10.297910in}{5.185474in}}%
\pgfpathcurveto{\pgfqpoint{10.290096in}{5.177660in}}{\pgfqpoint{10.285706in}{5.167061in}}{\pgfqpoint{10.285706in}{5.156011in}}%
\pgfpathcurveto{\pgfqpoint{10.285706in}{5.144961in}}{\pgfqpoint{10.290096in}{5.134362in}}{\pgfqpoint{10.297910in}{5.126548in}}%
\pgfpathcurveto{\pgfqpoint{10.305723in}{5.118735in}}{\pgfqpoint{10.316322in}{5.114344in}}{\pgfqpoint{10.327373in}{5.114344in}}%
\pgfpathlineto{\pgfqpoint{10.327373in}{5.114344in}}%
\pgfpathclose%
\pgfusepath{stroke}%
\end{pgfscope}%
\begin{pgfscope}%
\pgfpathrectangle{\pgfqpoint{7.394209in}{0.375000in}}{\pgfqpoint{6.356833in}{5.175000in}}%
\pgfusepath{clip}%
\pgfsetbuttcap%
\pgfsetroundjoin%
\pgfsetlinewidth{1.003750pt}%
\definecolor{currentstroke}{rgb}{0.827451,0.827451,0.827451}%
\pgfsetstrokecolor{currentstroke}%
\pgfsetdash{}{0pt}%
\pgfpathmoveto{\pgfqpoint{8.206934in}{1.911121in}}%
\pgfpathcurveto{\pgfqpoint{8.217984in}{1.911121in}}{\pgfqpoint{8.228583in}{1.915511in}}{\pgfqpoint{8.236397in}{1.923325in}}%
\pgfpathcurveto{\pgfqpoint{8.244211in}{1.931139in}}{\pgfqpoint{8.248601in}{1.941738in}}{\pgfqpoint{8.248601in}{1.952788in}}%
\pgfpathcurveto{\pgfqpoint{8.248601in}{1.963838in}}{\pgfqpoint{8.244211in}{1.974437in}}{\pgfqpoint{8.236397in}{1.982250in}}%
\pgfpathcurveto{\pgfqpoint{8.228583in}{1.990064in}}{\pgfqpoint{8.217984in}{1.994454in}}{\pgfqpoint{8.206934in}{1.994454in}}%
\pgfpathcurveto{\pgfqpoint{8.195884in}{1.994454in}}{\pgfqpoint{8.185285in}{1.990064in}}{\pgfqpoint{8.177471in}{1.982250in}}%
\pgfpathcurveto{\pgfqpoint{8.169658in}{1.974437in}}{\pgfqpoint{8.165268in}{1.963838in}}{\pgfqpoint{8.165268in}{1.952788in}}%
\pgfpathcurveto{\pgfqpoint{8.165268in}{1.941738in}}{\pgfqpoint{8.169658in}{1.931139in}}{\pgfqpoint{8.177471in}{1.923325in}}%
\pgfpathcurveto{\pgfqpoint{8.185285in}{1.915511in}}{\pgfqpoint{8.195884in}{1.911121in}}{\pgfqpoint{8.206934in}{1.911121in}}%
\pgfpathlineto{\pgfqpoint{8.206934in}{1.911121in}}%
\pgfpathclose%
\pgfusepath{stroke}%
\end{pgfscope}%
\begin{pgfscope}%
\pgfpathrectangle{\pgfqpoint{7.394209in}{0.375000in}}{\pgfqpoint{6.356833in}{5.175000in}}%
\pgfusepath{clip}%
\pgfsetbuttcap%
\pgfsetroundjoin%
\pgfsetlinewidth{1.003750pt}%
\definecolor{currentstroke}{rgb}{0.827451,0.827451,0.827451}%
\pgfsetstrokecolor{currentstroke}%
\pgfsetdash{}{0pt}%
\pgfpathmoveto{\pgfqpoint{11.953556in}{5.341626in}}%
\pgfpathcurveto{\pgfqpoint{11.964606in}{5.341626in}}{\pgfqpoint{11.975205in}{5.346016in}}{\pgfqpoint{11.983019in}{5.353829in}}%
\pgfpathcurveto{\pgfqpoint{11.990833in}{5.361643in}}{\pgfqpoint{11.995223in}{5.372242in}}{\pgfqpoint{11.995223in}{5.383292in}}%
\pgfpathcurveto{\pgfqpoint{11.995223in}{5.394342in}}{\pgfqpoint{11.990833in}{5.404941in}}{\pgfqpoint{11.983019in}{5.412755in}}%
\pgfpathcurveto{\pgfqpoint{11.975205in}{5.420569in}}{\pgfqpoint{11.964606in}{5.424959in}}{\pgfqpoint{11.953556in}{5.424959in}}%
\pgfpathcurveto{\pgfqpoint{11.942506in}{5.424959in}}{\pgfqpoint{11.931907in}{5.420569in}}{\pgfqpoint{11.924093in}{5.412755in}}%
\pgfpathcurveto{\pgfqpoint{11.916280in}{5.404941in}}{\pgfqpoint{11.911890in}{5.394342in}}{\pgfqpoint{11.911890in}{5.383292in}}%
\pgfpathcurveto{\pgfqpoint{11.911890in}{5.372242in}}{\pgfqpoint{11.916280in}{5.361643in}}{\pgfqpoint{11.924093in}{5.353829in}}%
\pgfpathcurveto{\pgfqpoint{11.931907in}{5.346016in}}{\pgfqpoint{11.942506in}{5.341626in}}{\pgfqpoint{11.953556in}{5.341626in}}%
\pgfpathlineto{\pgfqpoint{11.953556in}{5.341626in}}%
\pgfpathclose%
\pgfusepath{stroke}%
\end{pgfscope}%
\begin{pgfscope}%
\pgfpathrectangle{\pgfqpoint{7.394209in}{0.375000in}}{\pgfqpoint{6.356833in}{5.175000in}}%
\pgfusepath{clip}%
\pgfsetbuttcap%
\pgfsetroundjoin%
\pgfsetlinewidth{1.003750pt}%
\definecolor{currentstroke}{rgb}{0.827451,0.827451,0.827451}%
\pgfsetstrokecolor{currentstroke}%
\pgfsetdash{}{0pt}%
\pgfpathmoveto{\pgfqpoint{8.173793in}{2.162080in}}%
\pgfpathcurveto{\pgfqpoint{8.184844in}{2.162080in}}{\pgfqpoint{8.195443in}{2.166471in}}{\pgfqpoint{8.203256in}{2.174284in}}%
\pgfpathcurveto{\pgfqpoint{8.211070in}{2.182098in}}{\pgfqpoint{8.215460in}{2.192697in}}{\pgfqpoint{8.215460in}{2.203747in}}%
\pgfpathcurveto{\pgfqpoint{8.215460in}{2.214797in}}{\pgfqpoint{8.211070in}{2.225396in}}{\pgfqpoint{8.203256in}{2.233210in}}%
\pgfpathcurveto{\pgfqpoint{8.195443in}{2.241023in}}{\pgfqpoint{8.184844in}{2.245414in}}{\pgfqpoint{8.173793in}{2.245414in}}%
\pgfpathcurveto{\pgfqpoint{8.162743in}{2.245414in}}{\pgfqpoint{8.152144in}{2.241023in}}{\pgfqpoint{8.144331in}{2.233210in}}%
\pgfpathcurveto{\pgfqpoint{8.136517in}{2.225396in}}{\pgfqpoint{8.132127in}{2.214797in}}{\pgfqpoint{8.132127in}{2.203747in}}%
\pgfpathcurveto{\pgfqpoint{8.132127in}{2.192697in}}{\pgfqpoint{8.136517in}{2.182098in}}{\pgfqpoint{8.144331in}{2.174284in}}%
\pgfpathcurveto{\pgfqpoint{8.152144in}{2.166471in}}{\pgfqpoint{8.162743in}{2.162080in}}{\pgfqpoint{8.173793in}{2.162080in}}%
\pgfpathlineto{\pgfqpoint{8.173793in}{2.162080in}}%
\pgfpathclose%
\pgfusepath{stroke}%
\end{pgfscope}%
\begin{pgfscope}%
\pgfpathrectangle{\pgfqpoint{7.394209in}{0.375000in}}{\pgfqpoint{6.356833in}{5.175000in}}%
\pgfusepath{clip}%
\pgfsetbuttcap%
\pgfsetroundjoin%
\pgfsetlinewidth{1.003750pt}%
\definecolor{currentstroke}{rgb}{0.827451,0.827451,0.827451}%
\pgfsetstrokecolor{currentstroke}%
\pgfsetdash{}{0pt}%
\pgfpathmoveto{\pgfqpoint{8.309936in}{2.162080in}}%
\pgfpathcurveto{\pgfqpoint{8.320986in}{2.162080in}}{\pgfqpoint{8.331585in}{2.166471in}}{\pgfqpoint{8.339399in}{2.174284in}}%
\pgfpathcurveto{\pgfqpoint{8.347212in}{2.182098in}}{\pgfqpoint{8.351602in}{2.192697in}}{\pgfqpoint{8.351602in}{2.203747in}}%
\pgfpathcurveto{\pgfqpoint{8.351602in}{2.214797in}}{\pgfqpoint{8.347212in}{2.225396in}}{\pgfqpoint{8.339399in}{2.233210in}}%
\pgfpathcurveto{\pgfqpoint{8.331585in}{2.241023in}}{\pgfqpoint{8.320986in}{2.245414in}}{\pgfqpoint{8.309936in}{2.245414in}}%
\pgfpathcurveto{\pgfqpoint{8.298886in}{2.245414in}}{\pgfqpoint{8.288287in}{2.241023in}}{\pgfqpoint{8.280473in}{2.233210in}}%
\pgfpathcurveto{\pgfqpoint{8.272659in}{2.225396in}}{\pgfqpoint{8.268269in}{2.214797in}}{\pgfqpoint{8.268269in}{2.203747in}}%
\pgfpathcurveto{\pgfqpoint{8.268269in}{2.192697in}}{\pgfqpoint{8.272659in}{2.182098in}}{\pgfqpoint{8.280473in}{2.174284in}}%
\pgfpathcurveto{\pgfqpoint{8.288287in}{2.166471in}}{\pgfqpoint{8.298886in}{2.162080in}}{\pgfqpoint{8.309936in}{2.162080in}}%
\pgfpathlineto{\pgfqpoint{8.309936in}{2.162080in}}%
\pgfpathclose%
\pgfusepath{stroke}%
\end{pgfscope}%
\begin{pgfscope}%
\pgfpathrectangle{\pgfqpoint{7.394209in}{0.375000in}}{\pgfqpoint{6.356833in}{5.175000in}}%
\pgfusepath{clip}%
\pgfsetbuttcap%
\pgfsetroundjoin%
\pgfsetlinewidth{1.003750pt}%
\definecolor{currentstroke}{rgb}{0.827451,0.827451,0.827451}%
\pgfsetstrokecolor{currentstroke}%
\pgfsetdash{}{0pt}%
\pgfpathmoveto{\pgfqpoint{8.994221in}{1.864779in}}%
\pgfpathcurveto{\pgfqpoint{9.005271in}{1.864779in}}{\pgfqpoint{9.015870in}{1.869169in}}{\pgfqpoint{9.023683in}{1.876982in}}%
\pgfpathcurveto{\pgfqpoint{9.031497in}{1.884796in}}{\pgfqpoint{9.035887in}{1.895395in}}{\pgfqpoint{9.035887in}{1.906445in}}%
\pgfpathcurveto{\pgfqpoint{9.035887in}{1.917495in}}{\pgfqpoint{9.031497in}{1.928094in}}{\pgfqpoint{9.023683in}{1.935908in}}%
\pgfpathcurveto{\pgfqpoint{9.015870in}{1.943722in}}{\pgfqpoint{9.005271in}{1.948112in}}{\pgfqpoint{8.994221in}{1.948112in}}%
\pgfpathcurveto{\pgfqpoint{8.983170in}{1.948112in}}{\pgfqpoint{8.972571in}{1.943722in}}{\pgfqpoint{8.964758in}{1.935908in}}%
\pgfpathcurveto{\pgfqpoint{8.956944in}{1.928094in}}{\pgfqpoint{8.952554in}{1.917495in}}{\pgfqpoint{8.952554in}{1.906445in}}%
\pgfpathcurveto{\pgfqpoint{8.952554in}{1.895395in}}{\pgfqpoint{8.956944in}{1.884796in}}{\pgfqpoint{8.964758in}{1.876982in}}%
\pgfpathcurveto{\pgfqpoint{8.972571in}{1.869169in}}{\pgfqpoint{8.983170in}{1.864779in}}{\pgfqpoint{8.994221in}{1.864779in}}%
\pgfpathlineto{\pgfqpoint{8.994221in}{1.864779in}}%
\pgfpathclose%
\pgfusepath{stroke}%
\end{pgfscope}%
\begin{pgfscope}%
\pgfpathrectangle{\pgfqpoint{7.394209in}{0.375000in}}{\pgfqpoint{6.356833in}{5.175000in}}%
\pgfusepath{clip}%
\pgfsetbuttcap%
\pgfsetroundjoin%
\pgfsetlinewidth{1.003750pt}%
\definecolor{currentstroke}{rgb}{0.827451,0.827451,0.827451}%
\pgfsetstrokecolor{currentstroke}%
\pgfsetdash{}{0pt}%
\pgfpathmoveto{\pgfqpoint{10.118660in}{3.321655in}}%
\pgfpathcurveto{\pgfqpoint{10.129711in}{3.321655in}}{\pgfqpoint{10.140310in}{3.326046in}}{\pgfqpoint{10.148123in}{3.333859in}}%
\pgfpathcurveto{\pgfqpoint{10.155937in}{3.341673in}}{\pgfqpoint{10.160327in}{3.352272in}}{\pgfqpoint{10.160327in}{3.363322in}}%
\pgfpathcurveto{\pgfqpoint{10.160327in}{3.374372in}}{\pgfqpoint{10.155937in}{3.384971in}}{\pgfqpoint{10.148123in}{3.392785in}}%
\pgfpathcurveto{\pgfqpoint{10.140310in}{3.400598in}}{\pgfqpoint{10.129711in}{3.404989in}}{\pgfqpoint{10.118660in}{3.404989in}}%
\pgfpathcurveto{\pgfqpoint{10.107610in}{3.404989in}}{\pgfqpoint{10.097011in}{3.400598in}}{\pgfqpoint{10.089198in}{3.392785in}}%
\pgfpathcurveto{\pgfqpoint{10.081384in}{3.384971in}}{\pgfqpoint{10.076994in}{3.374372in}}{\pgfqpoint{10.076994in}{3.363322in}}%
\pgfpathcurveto{\pgfqpoint{10.076994in}{3.352272in}}{\pgfqpoint{10.081384in}{3.341673in}}{\pgfqpoint{10.089198in}{3.333859in}}%
\pgfpathcurveto{\pgfqpoint{10.097011in}{3.326046in}}{\pgfqpoint{10.107610in}{3.321655in}}{\pgfqpoint{10.118660in}{3.321655in}}%
\pgfpathlineto{\pgfqpoint{10.118660in}{3.321655in}}%
\pgfpathclose%
\pgfusepath{stroke}%
\end{pgfscope}%
\begin{pgfscope}%
\pgfpathrectangle{\pgfqpoint{7.394209in}{0.375000in}}{\pgfqpoint{6.356833in}{5.175000in}}%
\pgfusepath{clip}%
\pgfsetbuttcap%
\pgfsetroundjoin%
\pgfsetlinewidth{1.003750pt}%
\definecolor{currentstroke}{rgb}{0.827451,0.827451,0.827451}%
\pgfsetstrokecolor{currentstroke}%
\pgfsetdash{}{0pt}%
\pgfpathmoveto{\pgfqpoint{12.319286in}{5.421733in}}%
\pgfpathcurveto{\pgfqpoint{12.330336in}{5.421733in}}{\pgfqpoint{12.340935in}{5.426124in}}{\pgfqpoint{12.348748in}{5.433937in}}%
\pgfpathcurveto{\pgfqpoint{12.356562in}{5.441751in}}{\pgfqpoint{12.360952in}{5.452350in}}{\pgfqpoint{12.360952in}{5.463400in}}%
\pgfpathcurveto{\pgfqpoint{12.360952in}{5.474450in}}{\pgfqpoint{12.356562in}{5.485049in}}{\pgfqpoint{12.348748in}{5.492863in}}%
\pgfpathcurveto{\pgfqpoint{12.340935in}{5.500676in}}{\pgfqpoint{12.330336in}{5.505067in}}{\pgfqpoint{12.319286in}{5.505067in}}%
\pgfpathcurveto{\pgfqpoint{12.308235in}{5.505067in}}{\pgfqpoint{12.297636in}{5.500676in}}{\pgfqpoint{12.289823in}{5.492863in}}%
\pgfpathcurveto{\pgfqpoint{12.282009in}{5.485049in}}{\pgfqpoint{12.277619in}{5.474450in}}{\pgfqpoint{12.277619in}{5.463400in}}%
\pgfpathcurveto{\pgfqpoint{12.277619in}{5.452350in}}{\pgfqpoint{12.282009in}{5.441751in}}{\pgfqpoint{12.289823in}{5.433937in}}%
\pgfpathcurveto{\pgfqpoint{12.297636in}{5.426124in}}{\pgfqpoint{12.308235in}{5.421733in}}{\pgfqpoint{12.319286in}{5.421733in}}%
\pgfpathlineto{\pgfqpoint{12.319286in}{5.421733in}}%
\pgfpathclose%
\pgfusepath{stroke}%
\end{pgfscope}%
\begin{pgfscope}%
\pgfpathrectangle{\pgfqpoint{7.394209in}{0.375000in}}{\pgfqpoint{6.356833in}{5.175000in}}%
\pgfusepath{clip}%
\pgfsetbuttcap%
\pgfsetroundjoin%
\pgfsetlinewidth{1.003750pt}%
\definecolor{currentstroke}{rgb}{0.827451,0.827451,0.827451}%
\pgfsetstrokecolor{currentstroke}%
\pgfsetdash{}{0pt}%
\pgfpathmoveto{\pgfqpoint{9.154452in}{3.708114in}}%
\pgfpathcurveto{\pgfqpoint{9.165502in}{3.708114in}}{\pgfqpoint{9.176101in}{3.712505in}}{\pgfqpoint{9.183915in}{3.720318in}}%
\pgfpathcurveto{\pgfqpoint{9.191728in}{3.728132in}}{\pgfqpoint{9.196118in}{3.738731in}}{\pgfqpoint{9.196118in}{3.749781in}}%
\pgfpathcurveto{\pgfqpoint{9.196118in}{3.760831in}}{\pgfqpoint{9.191728in}{3.771430in}}{\pgfqpoint{9.183915in}{3.779244in}}%
\pgfpathcurveto{\pgfqpoint{9.176101in}{3.787057in}}{\pgfqpoint{9.165502in}{3.791448in}}{\pgfqpoint{9.154452in}{3.791448in}}%
\pgfpathcurveto{\pgfqpoint{9.143402in}{3.791448in}}{\pgfqpoint{9.132803in}{3.787057in}}{\pgfqpoint{9.124989in}{3.779244in}}%
\pgfpathcurveto{\pgfqpoint{9.117175in}{3.771430in}}{\pgfqpoint{9.112785in}{3.760831in}}{\pgfqpoint{9.112785in}{3.749781in}}%
\pgfpathcurveto{\pgfqpoint{9.112785in}{3.738731in}}{\pgfqpoint{9.117175in}{3.728132in}}{\pgfqpoint{9.124989in}{3.720318in}}%
\pgfpathcurveto{\pgfqpoint{9.132803in}{3.712505in}}{\pgfqpoint{9.143402in}{3.708114in}}{\pgfqpoint{9.154452in}{3.708114in}}%
\pgfpathlineto{\pgfqpoint{9.154452in}{3.708114in}}%
\pgfpathclose%
\pgfusepath{stroke}%
\end{pgfscope}%
\begin{pgfscope}%
\pgfpathrectangle{\pgfqpoint{7.394209in}{0.375000in}}{\pgfqpoint{6.356833in}{5.175000in}}%
\pgfusepath{clip}%
\pgfsetbuttcap%
\pgfsetroundjoin%
\pgfsetlinewidth{1.003750pt}%
\definecolor{currentstroke}{rgb}{0.827451,0.827451,0.827451}%
\pgfsetstrokecolor{currentstroke}%
\pgfsetdash{}{0pt}%
\pgfpathmoveto{\pgfqpoint{7.603036in}{1.060709in}}%
\pgfpathcurveto{\pgfqpoint{7.614086in}{1.060709in}}{\pgfqpoint{7.624685in}{1.065099in}}{\pgfqpoint{7.632499in}{1.072913in}}%
\pgfpathcurveto{\pgfqpoint{7.640312in}{1.080726in}}{\pgfqpoint{7.644703in}{1.091325in}}{\pgfqpoint{7.644703in}{1.102376in}}%
\pgfpathcurveto{\pgfqpoint{7.644703in}{1.113426in}}{\pgfqpoint{7.640312in}{1.124025in}}{\pgfqpoint{7.632499in}{1.131838in}}%
\pgfpathcurveto{\pgfqpoint{7.624685in}{1.139652in}}{\pgfqpoint{7.614086in}{1.144042in}}{\pgfqpoint{7.603036in}{1.144042in}}%
\pgfpathcurveto{\pgfqpoint{7.591986in}{1.144042in}}{\pgfqpoint{7.581387in}{1.139652in}}{\pgfqpoint{7.573573in}{1.131838in}}%
\pgfpathcurveto{\pgfqpoint{7.565760in}{1.124025in}}{\pgfqpoint{7.561369in}{1.113426in}}{\pgfqpoint{7.561369in}{1.102376in}}%
\pgfpathcurveto{\pgfqpoint{7.561369in}{1.091325in}}{\pgfqpoint{7.565760in}{1.080726in}}{\pgfqpoint{7.573573in}{1.072913in}}%
\pgfpathcurveto{\pgfqpoint{7.581387in}{1.065099in}}{\pgfqpoint{7.591986in}{1.060709in}}{\pgfqpoint{7.603036in}{1.060709in}}%
\pgfpathlineto{\pgfqpoint{7.603036in}{1.060709in}}%
\pgfpathclose%
\pgfusepath{stroke}%
\end{pgfscope}%
\begin{pgfscope}%
\pgfpathrectangle{\pgfqpoint{7.394209in}{0.375000in}}{\pgfqpoint{6.356833in}{5.175000in}}%
\pgfusepath{clip}%
\pgfsetbuttcap%
\pgfsetroundjoin%
\pgfsetlinewidth{1.003750pt}%
\definecolor{currentstroke}{rgb}{0.827451,0.827451,0.827451}%
\pgfsetstrokecolor{currentstroke}%
\pgfsetdash{}{0pt}%
\pgfpathmoveto{\pgfqpoint{8.842337in}{1.518369in}}%
\pgfpathcurveto{\pgfqpoint{8.853387in}{1.518369in}}{\pgfqpoint{8.863986in}{1.522759in}}{\pgfqpoint{8.871800in}{1.530573in}}%
\pgfpathcurveto{\pgfqpoint{8.879613in}{1.538386in}}{\pgfqpoint{8.884004in}{1.548985in}}{\pgfqpoint{8.884004in}{1.560036in}}%
\pgfpathcurveto{\pgfqpoint{8.884004in}{1.571086in}}{\pgfqpoint{8.879613in}{1.581685in}}{\pgfqpoint{8.871800in}{1.589498in}}%
\pgfpathcurveto{\pgfqpoint{8.863986in}{1.597312in}}{\pgfqpoint{8.853387in}{1.601702in}}{\pgfqpoint{8.842337in}{1.601702in}}%
\pgfpathcurveto{\pgfqpoint{8.831287in}{1.601702in}}{\pgfqpoint{8.820688in}{1.597312in}}{\pgfqpoint{8.812874in}{1.589498in}}%
\pgfpathcurveto{\pgfqpoint{8.805061in}{1.581685in}}{\pgfqpoint{8.800670in}{1.571086in}}{\pgfqpoint{8.800670in}{1.560036in}}%
\pgfpathcurveto{\pgfqpoint{8.800670in}{1.548985in}}{\pgfqpoint{8.805061in}{1.538386in}}{\pgfqpoint{8.812874in}{1.530573in}}%
\pgfpathcurveto{\pgfqpoint{8.820688in}{1.522759in}}{\pgfqpoint{8.831287in}{1.518369in}}{\pgfqpoint{8.842337in}{1.518369in}}%
\pgfpathlineto{\pgfqpoint{8.842337in}{1.518369in}}%
\pgfpathclose%
\pgfusepath{stroke}%
\end{pgfscope}%
\begin{pgfscope}%
\pgfpathrectangle{\pgfqpoint{7.394209in}{0.375000in}}{\pgfqpoint{6.356833in}{5.175000in}}%
\pgfusepath{clip}%
\pgfsetbuttcap%
\pgfsetroundjoin%
\pgfsetlinewidth{1.003750pt}%
\definecolor{currentstroke}{rgb}{0.827451,0.827451,0.827451}%
\pgfsetstrokecolor{currentstroke}%
\pgfsetdash{}{0pt}%
\pgfpathmoveto{\pgfqpoint{10.669881in}{3.749305in}}%
\pgfpathcurveto{\pgfqpoint{10.680932in}{3.749305in}}{\pgfqpoint{10.691531in}{3.753696in}}{\pgfqpoint{10.699344in}{3.761509in}}%
\pgfpathcurveto{\pgfqpoint{10.707158in}{3.769323in}}{\pgfqpoint{10.711548in}{3.779922in}}{\pgfqpoint{10.711548in}{3.790972in}}%
\pgfpathcurveto{\pgfqpoint{10.711548in}{3.802022in}}{\pgfqpoint{10.707158in}{3.812621in}}{\pgfqpoint{10.699344in}{3.820435in}}%
\pgfpathcurveto{\pgfqpoint{10.691531in}{3.828249in}}{\pgfqpoint{10.680932in}{3.832639in}}{\pgfqpoint{10.669881in}{3.832639in}}%
\pgfpathcurveto{\pgfqpoint{10.658831in}{3.832639in}}{\pgfqpoint{10.648232in}{3.828249in}}{\pgfqpoint{10.640419in}{3.820435in}}%
\pgfpathcurveto{\pgfqpoint{10.632605in}{3.812621in}}{\pgfqpoint{10.628215in}{3.802022in}}{\pgfqpoint{10.628215in}{3.790972in}}%
\pgfpathcurveto{\pgfqpoint{10.628215in}{3.779922in}}{\pgfqpoint{10.632605in}{3.769323in}}{\pgfqpoint{10.640419in}{3.761509in}}%
\pgfpathcurveto{\pgfqpoint{10.648232in}{3.753696in}}{\pgfqpoint{10.658831in}{3.749305in}}{\pgfqpoint{10.669881in}{3.749305in}}%
\pgfpathlineto{\pgfqpoint{10.669881in}{3.749305in}}%
\pgfpathclose%
\pgfusepath{stroke}%
\end{pgfscope}%
\begin{pgfscope}%
\pgfpathrectangle{\pgfqpoint{7.394209in}{0.375000in}}{\pgfqpoint{6.356833in}{5.175000in}}%
\pgfusepath{clip}%
\pgfsetbuttcap%
\pgfsetroundjoin%
\pgfsetlinewidth{1.003750pt}%
\definecolor{currentstroke}{rgb}{0.827451,0.827451,0.827451}%
\pgfsetstrokecolor{currentstroke}%
\pgfsetdash{}{0pt}%
\pgfpathmoveto{\pgfqpoint{8.778881in}{1.481397in}}%
\pgfpathcurveto{\pgfqpoint{8.789931in}{1.481397in}}{\pgfqpoint{8.800530in}{1.485787in}}{\pgfqpoint{8.808344in}{1.493600in}}%
\pgfpathcurveto{\pgfqpoint{8.816157in}{1.501414in}}{\pgfqpoint{8.820548in}{1.512013in}}{\pgfqpoint{8.820548in}{1.523063in}}%
\pgfpathcurveto{\pgfqpoint{8.820548in}{1.534113in}}{\pgfqpoint{8.816157in}{1.544712in}}{\pgfqpoint{8.808344in}{1.552526in}}%
\pgfpathcurveto{\pgfqpoint{8.800530in}{1.560340in}}{\pgfqpoint{8.789931in}{1.564730in}}{\pgfqpoint{8.778881in}{1.564730in}}%
\pgfpathcurveto{\pgfqpoint{8.767831in}{1.564730in}}{\pgfqpoint{8.757232in}{1.560340in}}{\pgfqpoint{8.749418in}{1.552526in}}%
\pgfpathcurveto{\pgfqpoint{8.741605in}{1.544712in}}{\pgfqpoint{8.737214in}{1.534113in}}{\pgfqpoint{8.737214in}{1.523063in}}%
\pgfpathcurveto{\pgfqpoint{8.737214in}{1.512013in}}{\pgfqpoint{8.741605in}{1.501414in}}{\pgfqpoint{8.749418in}{1.493600in}}%
\pgfpathcurveto{\pgfqpoint{8.757232in}{1.485787in}}{\pgfqpoint{8.767831in}{1.481397in}}{\pgfqpoint{8.778881in}{1.481397in}}%
\pgfpathlineto{\pgfqpoint{8.778881in}{1.481397in}}%
\pgfpathclose%
\pgfusepath{stroke}%
\end{pgfscope}%
\begin{pgfscope}%
\pgfpathrectangle{\pgfqpoint{7.394209in}{0.375000in}}{\pgfqpoint{6.356833in}{5.175000in}}%
\pgfusepath{clip}%
\pgfsetbuttcap%
\pgfsetroundjoin%
\pgfsetlinewidth{1.003750pt}%
\definecolor{currentstroke}{rgb}{0.827451,0.827451,0.827451}%
\pgfsetstrokecolor{currentstroke}%
\pgfsetdash{}{0pt}%
\pgfpathmoveto{\pgfqpoint{10.705973in}{3.850544in}}%
\pgfpathcurveto{\pgfqpoint{10.717024in}{3.850544in}}{\pgfqpoint{10.727623in}{3.854934in}}{\pgfqpoint{10.735436in}{3.862748in}}%
\pgfpathcurveto{\pgfqpoint{10.743250in}{3.870562in}}{\pgfqpoint{10.747640in}{3.881161in}}{\pgfqpoint{10.747640in}{3.892211in}}%
\pgfpathcurveto{\pgfqpoint{10.747640in}{3.903261in}}{\pgfqpoint{10.743250in}{3.913860in}}{\pgfqpoint{10.735436in}{3.921674in}}%
\pgfpathcurveto{\pgfqpoint{10.727623in}{3.929487in}}{\pgfqpoint{10.717024in}{3.933878in}}{\pgfqpoint{10.705973in}{3.933878in}}%
\pgfpathcurveto{\pgfqpoint{10.694923in}{3.933878in}}{\pgfqpoint{10.684324in}{3.929487in}}{\pgfqpoint{10.676511in}{3.921674in}}%
\pgfpathcurveto{\pgfqpoint{10.668697in}{3.913860in}}{\pgfqpoint{10.664307in}{3.903261in}}{\pgfqpoint{10.664307in}{3.892211in}}%
\pgfpathcurveto{\pgfqpoint{10.664307in}{3.881161in}}{\pgfqpoint{10.668697in}{3.870562in}}{\pgfqpoint{10.676511in}{3.862748in}}%
\pgfpathcurveto{\pgfqpoint{10.684324in}{3.854934in}}{\pgfqpoint{10.694923in}{3.850544in}}{\pgfqpoint{10.705973in}{3.850544in}}%
\pgfpathlineto{\pgfqpoint{10.705973in}{3.850544in}}%
\pgfpathclose%
\pgfusepath{stroke}%
\end{pgfscope}%
\begin{pgfscope}%
\pgfpathrectangle{\pgfqpoint{7.394209in}{0.375000in}}{\pgfqpoint{6.356833in}{5.175000in}}%
\pgfusepath{clip}%
\pgfsetbuttcap%
\pgfsetroundjoin%
\pgfsetlinewidth{1.003750pt}%
\definecolor{currentstroke}{rgb}{0.827451,0.827451,0.827451}%
\pgfsetstrokecolor{currentstroke}%
\pgfsetdash{}{0pt}%
\pgfpathmoveto{\pgfqpoint{9.455491in}{3.717376in}}%
\pgfpathcurveto{\pgfqpoint{9.466541in}{3.717376in}}{\pgfqpoint{9.477140in}{3.721766in}}{\pgfqpoint{9.484953in}{3.729580in}}%
\pgfpathcurveto{\pgfqpoint{9.492767in}{3.737393in}}{\pgfqpoint{9.497157in}{3.747992in}}{\pgfqpoint{9.497157in}{3.759042in}}%
\pgfpathcurveto{\pgfqpoint{9.497157in}{3.770093in}}{\pgfqpoint{9.492767in}{3.780692in}}{\pgfqpoint{9.484953in}{3.788505in}}%
\pgfpathcurveto{\pgfqpoint{9.477140in}{3.796319in}}{\pgfqpoint{9.466541in}{3.800709in}}{\pgfqpoint{9.455491in}{3.800709in}}%
\pgfpathcurveto{\pgfqpoint{9.444441in}{3.800709in}}{\pgfqpoint{9.433841in}{3.796319in}}{\pgfqpoint{9.426028in}{3.788505in}}%
\pgfpathcurveto{\pgfqpoint{9.418214in}{3.780692in}}{\pgfqpoint{9.413824in}{3.770093in}}{\pgfqpoint{9.413824in}{3.759042in}}%
\pgfpathcurveto{\pgfqpoint{9.413824in}{3.747992in}}{\pgfqpoint{9.418214in}{3.737393in}}{\pgfqpoint{9.426028in}{3.729580in}}%
\pgfpathcurveto{\pgfqpoint{9.433841in}{3.721766in}}{\pgfqpoint{9.444441in}{3.717376in}}{\pgfqpoint{9.455491in}{3.717376in}}%
\pgfpathlineto{\pgfqpoint{9.455491in}{3.717376in}}%
\pgfpathclose%
\pgfusepath{stroke}%
\end{pgfscope}%
\begin{pgfscope}%
\pgfpathrectangle{\pgfqpoint{7.394209in}{0.375000in}}{\pgfqpoint{6.356833in}{5.175000in}}%
\pgfusepath{clip}%
\pgfsetbuttcap%
\pgfsetroundjoin%
\pgfsetlinewidth{1.003750pt}%
\definecolor{currentstroke}{rgb}{0.827451,0.827451,0.827451}%
\pgfsetstrokecolor{currentstroke}%
\pgfsetdash{}{0pt}%
\pgfpathmoveto{\pgfqpoint{10.473956in}{3.883308in}}%
\pgfpathcurveto{\pgfqpoint{10.485006in}{3.883308in}}{\pgfqpoint{10.495605in}{3.887699in}}{\pgfqpoint{10.503419in}{3.895512in}}%
\pgfpathcurveto{\pgfqpoint{10.511233in}{3.903326in}}{\pgfqpoint{10.515623in}{3.913925in}}{\pgfqpoint{10.515623in}{3.924975in}}%
\pgfpathcurveto{\pgfqpoint{10.515623in}{3.936025in}}{\pgfqpoint{10.511233in}{3.946624in}}{\pgfqpoint{10.503419in}{3.954438in}}%
\pgfpathcurveto{\pgfqpoint{10.495605in}{3.962251in}}{\pgfqpoint{10.485006in}{3.966642in}}{\pgfqpoint{10.473956in}{3.966642in}}%
\pgfpathcurveto{\pgfqpoint{10.462906in}{3.966642in}}{\pgfqpoint{10.452307in}{3.962251in}}{\pgfqpoint{10.444493in}{3.954438in}}%
\pgfpathcurveto{\pgfqpoint{10.436680in}{3.946624in}}{\pgfqpoint{10.432290in}{3.936025in}}{\pgfqpoint{10.432290in}{3.924975in}}%
\pgfpathcurveto{\pgfqpoint{10.432290in}{3.913925in}}{\pgfqpoint{10.436680in}{3.903326in}}{\pgfqpoint{10.444493in}{3.895512in}}%
\pgfpathcurveto{\pgfqpoint{10.452307in}{3.887699in}}{\pgfqpoint{10.462906in}{3.883308in}}{\pgfqpoint{10.473956in}{3.883308in}}%
\pgfpathlineto{\pgfqpoint{10.473956in}{3.883308in}}%
\pgfpathclose%
\pgfusepath{stroke}%
\end{pgfscope}%
\begin{pgfscope}%
\pgfpathrectangle{\pgfqpoint{7.394209in}{0.375000in}}{\pgfqpoint{6.356833in}{5.175000in}}%
\pgfusepath{clip}%
\pgfsetbuttcap%
\pgfsetroundjoin%
\pgfsetlinewidth{1.003750pt}%
\definecolor{currentstroke}{rgb}{0.827451,0.827451,0.827451}%
\pgfsetstrokecolor{currentstroke}%
\pgfsetdash{}{0pt}%
\pgfpathmoveto{\pgfqpoint{8.816430in}{2.832538in}}%
\pgfpathcurveto{\pgfqpoint{8.827480in}{2.832538in}}{\pgfqpoint{8.838079in}{2.836928in}}{\pgfqpoint{8.845892in}{2.844742in}}%
\pgfpathcurveto{\pgfqpoint{8.853706in}{2.852555in}}{\pgfqpoint{8.858096in}{2.863154in}}{\pgfqpoint{8.858096in}{2.874205in}}%
\pgfpathcurveto{\pgfqpoint{8.858096in}{2.885255in}}{\pgfqpoint{8.853706in}{2.895854in}}{\pgfqpoint{8.845892in}{2.903667in}}%
\pgfpathcurveto{\pgfqpoint{8.838079in}{2.911481in}}{\pgfqpoint{8.827480in}{2.915871in}}{\pgfqpoint{8.816430in}{2.915871in}}%
\pgfpathcurveto{\pgfqpoint{8.805379in}{2.915871in}}{\pgfqpoint{8.794780in}{2.911481in}}{\pgfqpoint{8.786967in}{2.903667in}}%
\pgfpathcurveto{\pgfqpoint{8.779153in}{2.895854in}}{\pgfqpoint{8.774763in}{2.885255in}}{\pgfqpoint{8.774763in}{2.874205in}}%
\pgfpathcurveto{\pgfqpoint{8.774763in}{2.863154in}}{\pgfqpoint{8.779153in}{2.852555in}}{\pgfqpoint{8.786967in}{2.844742in}}%
\pgfpathcurveto{\pgfqpoint{8.794780in}{2.836928in}}{\pgfqpoint{8.805379in}{2.832538in}}{\pgfqpoint{8.816430in}{2.832538in}}%
\pgfpathlineto{\pgfqpoint{8.816430in}{2.832538in}}%
\pgfpathclose%
\pgfusepath{stroke}%
\end{pgfscope}%
\begin{pgfscope}%
\pgfpathrectangle{\pgfqpoint{7.394209in}{0.375000in}}{\pgfqpoint{6.356833in}{5.175000in}}%
\pgfusepath{clip}%
\pgfsetbuttcap%
\pgfsetroundjoin%
\pgfsetlinewidth{1.003750pt}%
\definecolor{currentstroke}{rgb}{0.827451,0.827451,0.827451}%
\pgfsetstrokecolor{currentstroke}%
\pgfsetdash{}{0pt}%
\pgfpathmoveto{\pgfqpoint{11.140595in}{3.998184in}}%
\pgfpathcurveto{\pgfqpoint{11.151645in}{3.998184in}}{\pgfqpoint{11.162244in}{4.002574in}}{\pgfqpoint{11.170058in}{4.010388in}}%
\pgfpathcurveto{\pgfqpoint{11.177871in}{4.018201in}}{\pgfqpoint{11.182261in}{4.028800in}}{\pgfqpoint{11.182261in}{4.039850in}}%
\pgfpathcurveto{\pgfqpoint{11.182261in}{4.050901in}}{\pgfqpoint{11.177871in}{4.061500in}}{\pgfqpoint{11.170058in}{4.069313in}}%
\pgfpathcurveto{\pgfqpoint{11.162244in}{4.077127in}}{\pgfqpoint{11.151645in}{4.081517in}}{\pgfqpoint{11.140595in}{4.081517in}}%
\pgfpathcurveto{\pgfqpoint{11.129545in}{4.081517in}}{\pgfqpoint{11.118946in}{4.077127in}}{\pgfqpoint{11.111132in}{4.069313in}}%
\pgfpathcurveto{\pgfqpoint{11.103318in}{4.061500in}}{\pgfqpoint{11.098928in}{4.050901in}}{\pgfqpoint{11.098928in}{4.039850in}}%
\pgfpathcurveto{\pgfqpoint{11.098928in}{4.028800in}}{\pgfqpoint{11.103318in}{4.018201in}}{\pgfqpoint{11.111132in}{4.010388in}}%
\pgfpathcurveto{\pgfqpoint{11.118946in}{4.002574in}}{\pgfqpoint{11.129545in}{3.998184in}}{\pgfqpoint{11.140595in}{3.998184in}}%
\pgfpathlineto{\pgfqpoint{11.140595in}{3.998184in}}%
\pgfpathclose%
\pgfusepath{stroke}%
\end{pgfscope}%
\begin{pgfscope}%
\pgfpathrectangle{\pgfqpoint{7.394209in}{0.375000in}}{\pgfqpoint{6.356833in}{5.175000in}}%
\pgfusepath{clip}%
\pgfsetbuttcap%
\pgfsetroundjoin%
\pgfsetlinewidth{1.003750pt}%
\definecolor{currentstroke}{rgb}{0.827451,0.827451,0.827451}%
\pgfsetstrokecolor{currentstroke}%
\pgfsetdash{}{0pt}%
\pgfpathmoveto{\pgfqpoint{8.904100in}{1.975545in}}%
\pgfpathcurveto{\pgfqpoint{8.915150in}{1.975545in}}{\pgfqpoint{8.925749in}{1.979936in}}{\pgfqpoint{8.933562in}{1.987749in}}%
\pgfpathcurveto{\pgfqpoint{8.941376in}{1.995563in}}{\pgfqpoint{8.945766in}{2.006162in}}{\pgfqpoint{8.945766in}{2.017212in}}%
\pgfpathcurveto{\pgfqpoint{8.945766in}{2.028262in}}{\pgfqpoint{8.941376in}{2.038861in}}{\pgfqpoint{8.933562in}{2.046675in}}%
\pgfpathcurveto{\pgfqpoint{8.925749in}{2.054488in}}{\pgfqpoint{8.915150in}{2.058879in}}{\pgfqpoint{8.904100in}{2.058879in}}%
\pgfpathcurveto{\pgfqpoint{8.893050in}{2.058879in}}{\pgfqpoint{8.882450in}{2.054488in}}{\pgfqpoint{8.874637in}{2.046675in}}%
\pgfpathcurveto{\pgfqpoint{8.866823in}{2.038861in}}{\pgfqpoint{8.862433in}{2.028262in}}{\pgfqpoint{8.862433in}{2.017212in}}%
\pgfpathcurveto{\pgfqpoint{8.862433in}{2.006162in}}{\pgfqpoint{8.866823in}{1.995563in}}{\pgfqpoint{8.874637in}{1.987749in}}%
\pgfpathcurveto{\pgfqpoint{8.882450in}{1.979936in}}{\pgfqpoint{8.893050in}{1.975545in}}{\pgfqpoint{8.904100in}{1.975545in}}%
\pgfpathlineto{\pgfqpoint{8.904100in}{1.975545in}}%
\pgfpathclose%
\pgfusepath{stroke}%
\end{pgfscope}%
\begin{pgfscope}%
\pgfpathrectangle{\pgfqpoint{7.394209in}{0.375000in}}{\pgfqpoint{6.356833in}{5.175000in}}%
\pgfusepath{clip}%
\pgfsetbuttcap%
\pgfsetroundjoin%
\pgfsetlinewidth{1.003750pt}%
\definecolor{currentstroke}{rgb}{0.827451,0.827451,0.827451}%
\pgfsetstrokecolor{currentstroke}%
\pgfsetdash{}{0pt}%
\pgfpathmoveto{\pgfqpoint{9.439789in}{2.469753in}}%
\pgfpathcurveto{\pgfqpoint{9.450839in}{2.469753in}}{\pgfqpoint{9.461438in}{2.474144in}}{\pgfqpoint{9.469252in}{2.481957in}}%
\pgfpathcurveto{\pgfqpoint{9.477066in}{2.489771in}}{\pgfqpoint{9.481456in}{2.500370in}}{\pgfqpoint{9.481456in}{2.511420in}}%
\pgfpathcurveto{\pgfqpoint{9.481456in}{2.522470in}}{\pgfqpoint{9.477066in}{2.533069in}}{\pgfqpoint{9.469252in}{2.540883in}}%
\pgfpathcurveto{\pgfqpoint{9.461438in}{2.548697in}}{\pgfqpoint{9.450839in}{2.553087in}}{\pgfqpoint{9.439789in}{2.553087in}}%
\pgfpathcurveto{\pgfqpoint{9.428739in}{2.553087in}}{\pgfqpoint{9.418140in}{2.548697in}}{\pgfqpoint{9.410326in}{2.540883in}}%
\pgfpathcurveto{\pgfqpoint{9.402513in}{2.533069in}}{\pgfqpoint{9.398123in}{2.522470in}}{\pgfqpoint{9.398123in}{2.511420in}}%
\pgfpathcurveto{\pgfqpoint{9.398123in}{2.500370in}}{\pgfqpoint{9.402513in}{2.489771in}}{\pgfqpoint{9.410326in}{2.481957in}}%
\pgfpathcurveto{\pgfqpoint{9.418140in}{2.474144in}}{\pgfqpoint{9.428739in}{2.469753in}}{\pgfqpoint{9.439789in}{2.469753in}}%
\pgfpathlineto{\pgfqpoint{9.439789in}{2.469753in}}%
\pgfpathclose%
\pgfusepath{stroke}%
\end{pgfscope}%
\begin{pgfscope}%
\pgfpathrectangle{\pgfqpoint{7.394209in}{0.375000in}}{\pgfqpoint{6.356833in}{5.175000in}}%
\pgfusepath{clip}%
\pgfsetbuttcap%
\pgfsetroundjoin%
\pgfsetlinewidth{1.003750pt}%
\definecolor{currentstroke}{rgb}{0.827451,0.827451,0.827451}%
\pgfsetstrokecolor{currentstroke}%
\pgfsetdash{}{0pt}%
\pgfpathmoveto{\pgfqpoint{9.397687in}{2.333774in}}%
\pgfpathcurveto{\pgfqpoint{9.408737in}{2.333774in}}{\pgfqpoint{9.419336in}{2.338164in}}{\pgfqpoint{9.427149in}{2.345978in}}%
\pgfpathcurveto{\pgfqpoint{9.434963in}{2.353792in}}{\pgfqpoint{9.439353in}{2.364391in}}{\pgfqpoint{9.439353in}{2.375441in}}%
\pgfpathcurveto{\pgfqpoint{9.439353in}{2.386491in}}{\pgfqpoint{9.434963in}{2.397090in}}{\pgfqpoint{9.427149in}{2.404904in}}%
\pgfpathcurveto{\pgfqpoint{9.419336in}{2.412717in}}{\pgfqpoint{9.408737in}{2.417107in}}{\pgfqpoint{9.397687in}{2.417107in}}%
\pgfpathcurveto{\pgfqpoint{9.386636in}{2.417107in}}{\pgfqpoint{9.376037in}{2.412717in}}{\pgfqpoint{9.368224in}{2.404904in}}%
\pgfpathcurveto{\pgfqpoint{9.360410in}{2.397090in}}{\pgfqpoint{9.356020in}{2.386491in}}{\pgfqpoint{9.356020in}{2.375441in}}%
\pgfpathcurveto{\pgfqpoint{9.356020in}{2.364391in}}{\pgfqpoint{9.360410in}{2.353792in}}{\pgfqpoint{9.368224in}{2.345978in}}%
\pgfpathcurveto{\pgfqpoint{9.376037in}{2.338164in}}{\pgfqpoint{9.386636in}{2.333774in}}{\pgfqpoint{9.397687in}{2.333774in}}%
\pgfpathlineto{\pgfqpoint{9.397687in}{2.333774in}}%
\pgfpathclose%
\pgfusepath{stroke}%
\end{pgfscope}%
\begin{pgfscope}%
\pgfpathrectangle{\pgfqpoint{7.394209in}{0.375000in}}{\pgfqpoint{6.356833in}{5.175000in}}%
\pgfusepath{clip}%
\pgfsetbuttcap%
\pgfsetroundjoin%
\pgfsetlinewidth{1.003750pt}%
\definecolor{currentstroke}{rgb}{0.827451,0.827451,0.827451}%
\pgfsetstrokecolor{currentstroke}%
\pgfsetdash{}{0pt}%
\pgfpathmoveto{\pgfqpoint{9.172662in}{3.525895in}}%
\pgfpathcurveto{\pgfqpoint{9.183712in}{3.525895in}}{\pgfqpoint{9.194311in}{3.530285in}}{\pgfqpoint{9.202125in}{3.538098in}}%
\pgfpathcurveto{\pgfqpoint{9.209939in}{3.545912in}}{\pgfqpoint{9.214329in}{3.556511in}}{\pgfqpoint{9.214329in}{3.567561in}}%
\pgfpathcurveto{\pgfqpoint{9.214329in}{3.578611in}}{\pgfqpoint{9.209939in}{3.589210in}}{\pgfqpoint{9.202125in}{3.597024in}}%
\pgfpathcurveto{\pgfqpoint{9.194311in}{3.604838in}}{\pgfqpoint{9.183712in}{3.609228in}}{\pgfqpoint{9.172662in}{3.609228in}}%
\pgfpathcurveto{\pgfqpoint{9.161612in}{3.609228in}}{\pgfqpoint{9.151013in}{3.604838in}}{\pgfqpoint{9.143200in}{3.597024in}}%
\pgfpathcurveto{\pgfqpoint{9.135386in}{3.589210in}}{\pgfqpoint{9.130996in}{3.578611in}}{\pgfqpoint{9.130996in}{3.567561in}}%
\pgfpathcurveto{\pgfqpoint{9.130996in}{3.556511in}}{\pgfqpoint{9.135386in}{3.545912in}}{\pgfqpoint{9.143200in}{3.538098in}}%
\pgfpathcurveto{\pgfqpoint{9.151013in}{3.530285in}}{\pgfqpoint{9.161612in}{3.525895in}}{\pgfqpoint{9.172662in}{3.525895in}}%
\pgfpathlineto{\pgfqpoint{9.172662in}{3.525895in}}%
\pgfpathclose%
\pgfusepath{stroke}%
\end{pgfscope}%
\begin{pgfscope}%
\pgfpathrectangle{\pgfqpoint{7.394209in}{0.375000in}}{\pgfqpoint{6.356833in}{5.175000in}}%
\pgfusepath{clip}%
\pgfsetbuttcap%
\pgfsetroundjoin%
\pgfsetlinewidth{1.003750pt}%
\definecolor{currentstroke}{rgb}{0.827451,0.827451,0.827451}%
\pgfsetstrokecolor{currentstroke}%
\pgfsetdash{}{0pt}%
\pgfpathmoveto{\pgfqpoint{7.683662in}{0.365611in}}%
\pgfpathcurveto{\pgfqpoint{7.694712in}{0.365611in}}{\pgfqpoint{7.705311in}{0.370001in}}{\pgfqpoint{7.713125in}{0.377815in}}%
\pgfpathcurveto{\pgfqpoint{7.720938in}{0.385629in}}{\pgfqpoint{7.725329in}{0.396228in}}{\pgfqpoint{7.725329in}{0.407278in}}%
\pgfpathcurveto{\pgfqpoint{7.725329in}{0.418328in}}{\pgfqpoint{7.720938in}{0.428927in}}{\pgfqpoint{7.713125in}{0.436741in}}%
\pgfpathcurveto{\pgfqpoint{7.705311in}{0.444554in}}{\pgfqpoint{7.694712in}{0.448944in}}{\pgfqpoint{7.683662in}{0.448944in}}%
\pgfpathcurveto{\pgfqpoint{7.672612in}{0.448944in}}{\pgfqpoint{7.662013in}{0.444554in}}{\pgfqpoint{7.654199in}{0.436741in}}%
\pgfpathcurveto{\pgfqpoint{7.646386in}{0.428927in}}{\pgfqpoint{7.641995in}{0.418328in}}{\pgfqpoint{7.641995in}{0.407278in}}%
\pgfpathcurveto{\pgfqpoint{7.641995in}{0.396228in}}{\pgfqpoint{7.646386in}{0.385629in}}{\pgfqpoint{7.654199in}{0.377815in}}%
\pgfpathcurveto{\pgfqpoint{7.662013in}{0.370001in}}{\pgfqpoint{7.672612in}{0.365611in}}{\pgfqpoint{7.683662in}{0.365611in}}%
\pgfusepath{stroke}%
\end{pgfscope}%
\begin{pgfscope}%
\pgfpathrectangle{\pgfqpoint{7.394209in}{0.375000in}}{\pgfqpoint{6.356833in}{5.175000in}}%
\pgfusepath{clip}%
\pgfsetbuttcap%
\pgfsetroundjoin%
\pgfsetlinewidth{1.003750pt}%
\definecolor{currentstroke}{rgb}{0.827451,0.827451,0.827451}%
\pgfsetstrokecolor{currentstroke}%
\pgfsetdash{}{0pt}%
\pgfpathmoveto{\pgfqpoint{8.299421in}{1.157259in}}%
\pgfpathcurveto{\pgfqpoint{8.310471in}{1.157259in}}{\pgfqpoint{8.321070in}{1.161649in}}{\pgfqpoint{8.328884in}{1.169463in}}%
\pgfpathcurveto{\pgfqpoint{8.336697in}{1.177276in}}{\pgfqpoint{8.341087in}{1.187875in}}{\pgfqpoint{8.341087in}{1.198926in}}%
\pgfpathcurveto{\pgfqpoint{8.341087in}{1.209976in}}{\pgfqpoint{8.336697in}{1.220575in}}{\pgfqpoint{8.328884in}{1.228388in}}%
\pgfpathcurveto{\pgfqpoint{8.321070in}{1.236202in}}{\pgfqpoint{8.310471in}{1.240592in}}{\pgfqpoint{8.299421in}{1.240592in}}%
\pgfpathcurveto{\pgfqpoint{8.288371in}{1.240592in}}{\pgfqpoint{8.277772in}{1.236202in}}{\pgfqpoint{8.269958in}{1.228388in}}%
\pgfpathcurveto{\pgfqpoint{8.262144in}{1.220575in}}{\pgfqpoint{8.257754in}{1.209976in}}{\pgfqpoint{8.257754in}{1.198926in}}%
\pgfpathcurveto{\pgfqpoint{8.257754in}{1.187875in}}{\pgfqpoint{8.262144in}{1.177276in}}{\pgfqpoint{8.269958in}{1.169463in}}%
\pgfpathcurveto{\pgfqpoint{8.277772in}{1.161649in}}{\pgfqpoint{8.288371in}{1.157259in}}{\pgfqpoint{8.299421in}{1.157259in}}%
\pgfpathlineto{\pgfqpoint{8.299421in}{1.157259in}}%
\pgfpathclose%
\pgfusepath{stroke}%
\end{pgfscope}%
\begin{pgfscope}%
\pgfpathrectangle{\pgfqpoint{7.394209in}{0.375000in}}{\pgfqpoint{6.356833in}{5.175000in}}%
\pgfusepath{clip}%
\pgfsetbuttcap%
\pgfsetroundjoin%
\pgfsetlinewidth{1.003750pt}%
\definecolor{currentstroke}{rgb}{0.827451,0.827451,0.827451}%
\pgfsetstrokecolor{currentstroke}%
\pgfsetdash{}{0pt}%
\pgfpathmoveto{\pgfqpoint{9.954529in}{2.558771in}}%
\pgfpathcurveto{\pgfqpoint{9.965579in}{2.558771in}}{\pgfqpoint{9.976178in}{2.563161in}}{\pgfqpoint{9.983991in}{2.570975in}}%
\pgfpathcurveto{\pgfqpoint{9.991805in}{2.578788in}}{\pgfqpoint{9.996195in}{2.589387in}}{\pgfqpoint{9.996195in}{2.600438in}}%
\pgfpathcurveto{\pgfqpoint{9.996195in}{2.611488in}}{\pgfqpoint{9.991805in}{2.622087in}}{\pgfqpoint{9.983991in}{2.629900in}}%
\pgfpathcurveto{\pgfqpoint{9.976178in}{2.637714in}}{\pgfqpoint{9.965579in}{2.642104in}}{\pgfqpoint{9.954529in}{2.642104in}}%
\pgfpathcurveto{\pgfqpoint{9.943479in}{2.642104in}}{\pgfqpoint{9.932879in}{2.637714in}}{\pgfqpoint{9.925066in}{2.629900in}}%
\pgfpathcurveto{\pgfqpoint{9.917252in}{2.622087in}}{\pgfqpoint{9.912862in}{2.611488in}}{\pgfqpoint{9.912862in}{2.600438in}}%
\pgfpathcurveto{\pgfqpoint{9.912862in}{2.589387in}}{\pgfqpoint{9.917252in}{2.578788in}}{\pgfqpoint{9.925066in}{2.570975in}}%
\pgfpathcurveto{\pgfqpoint{9.932879in}{2.563161in}}{\pgfqpoint{9.943479in}{2.558771in}}{\pgfqpoint{9.954529in}{2.558771in}}%
\pgfpathlineto{\pgfqpoint{9.954529in}{2.558771in}}%
\pgfpathclose%
\pgfusepath{stroke}%
\end{pgfscope}%
\begin{pgfscope}%
\pgfpathrectangle{\pgfqpoint{7.394209in}{0.375000in}}{\pgfqpoint{6.356833in}{5.175000in}}%
\pgfusepath{clip}%
\pgfsetbuttcap%
\pgfsetroundjoin%
\pgfsetlinewidth{1.003750pt}%
\definecolor{currentstroke}{rgb}{0.827451,0.827451,0.827451}%
\pgfsetstrokecolor{currentstroke}%
\pgfsetdash{}{0pt}%
\pgfpathmoveto{\pgfqpoint{10.320140in}{5.468570in}}%
\pgfpathcurveto{\pgfqpoint{10.331191in}{5.468570in}}{\pgfqpoint{10.341790in}{5.472961in}}{\pgfqpoint{10.349603in}{5.480774in}}%
\pgfpathcurveto{\pgfqpoint{10.357417in}{5.488588in}}{\pgfqpoint{10.361807in}{5.499187in}}{\pgfqpoint{10.361807in}{5.510237in}}%
\pgfpathcurveto{\pgfqpoint{10.361807in}{5.521287in}}{\pgfqpoint{10.357417in}{5.531886in}}{\pgfqpoint{10.349603in}{5.539700in}}%
\pgfpathcurveto{\pgfqpoint{10.341790in}{5.547513in}}{\pgfqpoint{10.331191in}{5.551904in}}{\pgfqpoint{10.320140in}{5.551904in}}%
\pgfpathcurveto{\pgfqpoint{10.309090in}{5.551904in}}{\pgfqpoint{10.298491in}{5.547513in}}{\pgfqpoint{10.290678in}{5.539700in}}%
\pgfpathcurveto{\pgfqpoint{10.282864in}{5.531886in}}{\pgfqpoint{10.278474in}{5.521287in}}{\pgfqpoint{10.278474in}{5.510237in}}%
\pgfpathcurveto{\pgfqpoint{10.278474in}{5.499187in}}{\pgfqpoint{10.282864in}{5.488588in}}{\pgfqpoint{10.290678in}{5.480774in}}%
\pgfpathcurveto{\pgfqpoint{10.298491in}{5.472961in}}{\pgfqpoint{10.309090in}{5.468570in}}{\pgfqpoint{10.320140in}{5.468570in}}%
\pgfpathlineto{\pgfqpoint{10.320140in}{5.468570in}}%
\pgfpathclose%
\pgfusepath{stroke}%
\end{pgfscope}%
\begin{pgfscope}%
\pgfpathrectangle{\pgfqpoint{7.394209in}{0.375000in}}{\pgfqpoint{6.356833in}{5.175000in}}%
\pgfusepath{clip}%
\pgfsetbuttcap%
\pgfsetroundjoin%
\pgfsetlinewidth{1.003750pt}%
\definecolor{currentstroke}{rgb}{0.827451,0.827451,0.827451}%
\pgfsetstrokecolor{currentstroke}%
\pgfsetdash{}{0pt}%
\pgfpathmoveto{\pgfqpoint{10.645469in}{3.836090in}}%
\pgfpathcurveto{\pgfqpoint{10.656519in}{3.836090in}}{\pgfqpoint{10.667118in}{3.840481in}}{\pgfqpoint{10.674932in}{3.848294in}}%
\pgfpathcurveto{\pgfqpoint{10.682745in}{3.856108in}}{\pgfqpoint{10.687136in}{3.866707in}}{\pgfqpoint{10.687136in}{3.877757in}}%
\pgfpathcurveto{\pgfqpoint{10.687136in}{3.888807in}}{\pgfqpoint{10.682745in}{3.899406in}}{\pgfqpoint{10.674932in}{3.907220in}}%
\pgfpathcurveto{\pgfqpoint{10.667118in}{3.915033in}}{\pgfqpoint{10.656519in}{3.919424in}}{\pgfqpoint{10.645469in}{3.919424in}}%
\pgfpathcurveto{\pgfqpoint{10.634419in}{3.919424in}}{\pgfqpoint{10.623820in}{3.915033in}}{\pgfqpoint{10.616006in}{3.907220in}}%
\pgfpathcurveto{\pgfqpoint{10.608193in}{3.899406in}}{\pgfqpoint{10.603802in}{3.888807in}}{\pgfqpoint{10.603802in}{3.877757in}}%
\pgfpathcurveto{\pgfqpoint{10.603802in}{3.866707in}}{\pgfqpoint{10.608193in}{3.856108in}}{\pgfqpoint{10.616006in}{3.848294in}}%
\pgfpathcurveto{\pgfqpoint{10.623820in}{3.840481in}}{\pgfqpoint{10.634419in}{3.836090in}}{\pgfqpoint{10.645469in}{3.836090in}}%
\pgfpathlineto{\pgfqpoint{10.645469in}{3.836090in}}%
\pgfpathclose%
\pgfusepath{stroke}%
\end{pgfscope}%
\begin{pgfscope}%
\pgfpathrectangle{\pgfqpoint{7.394209in}{0.375000in}}{\pgfqpoint{6.356833in}{5.175000in}}%
\pgfusepath{clip}%
\pgfsetbuttcap%
\pgfsetroundjoin%
\pgfsetlinewidth{1.003750pt}%
\definecolor{currentstroke}{rgb}{0.827451,0.827451,0.827451}%
\pgfsetstrokecolor{currentstroke}%
\pgfsetdash{}{0pt}%
\pgfpathmoveto{\pgfqpoint{11.173632in}{4.770539in}}%
\pgfpathcurveto{\pgfqpoint{11.184682in}{4.770539in}}{\pgfqpoint{11.195281in}{4.774930in}}{\pgfqpoint{11.203095in}{4.782743in}}%
\pgfpathcurveto{\pgfqpoint{11.210908in}{4.790557in}}{\pgfqpoint{11.215298in}{4.801156in}}{\pgfqpoint{11.215298in}{4.812206in}}%
\pgfpathcurveto{\pgfqpoint{11.215298in}{4.823256in}}{\pgfqpoint{11.210908in}{4.833855in}}{\pgfqpoint{11.203095in}{4.841669in}}%
\pgfpathcurveto{\pgfqpoint{11.195281in}{4.849482in}}{\pgfqpoint{11.184682in}{4.853873in}}{\pgfqpoint{11.173632in}{4.853873in}}%
\pgfpathcurveto{\pgfqpoint{11.162582in}{4.853873in}}{\pgfqpoint{11.151983in}{4.849482in}}{\pgfqpoint{11.144169in}{4.841669in}}%
\pgfpathcurveto{\pgfqpoint{11.136355in}{4.833855in}}{\pgfqpoint{11.131965in}{4.823256in}}{\pgfqpoint{11.131965in}{4.812206in}}%
\pgfpathcurveto{\pgfqpoint{11.131965in}{4.801156in}}{\pgfqpoint{11.136355in}{4.790557in}}{\pgfqpoint{11.144169in}{4.782743in}}%
\pgfpathcurveto{\pgfqpoint{11.151983in}{4.774930in}}{\pgfqpoint{11.162582in}{4.770539in}}{\pgfqpoint{11.173632in}{4.770539in}}%
\pgfpathlineto{\pgfqpoint{11.173632in}{4.770539in}}%
\pgfpathclose%
\pgfusepath{stroke}%
\end{pgfscope}%
\begin{pgfscope}%
\pgfpathrectangle{\pgfqpoint{7.394209in}{0.375000in}}{\pgfqpoint{6.356833in}{5.175000in}}%
\pgfusepath{clip}%
\pgfsetbuttcap%
\pgfsetroundjoin%
\pgfsetlinewidth{1.003750pt}%
\definecolor{currentstroke}{rgb}{0.827451,0.827451,0.827451}%
\pgfsetstrokecolor{currentstroke}%
\pgfsetdash{}{0pt}%
\pgfpathmoveto{\pgfqpoint{10.087522in}{5.508079in}}%
\pgfpathcurveto{\pgfqpoint{10.098572in}{5.508079in}}{\pgfqpoint{10.109171in}{5.512469in}}{\pgfqpoint{10.116984in}{5.520283in}}%
\pgfpathcurveto{\pgfqpoint{10.124798in}{5.528097in}}{\pgfqpoint{10.129188in}{5.538696in}}{\pgfqpoint{10.129188in}{5.549746in}}%
\pgfpathcurveto{\pgfqpoint{10.129188in}{5.560796in}}{\pgfqpoint{10.124798in}{5.571395in}}{\pgfqpoint{10.116984in}{5.579209in}}%
\pgfpathcurveto{\pgfqpoint{10.109171in}{5.587022in}}{\pgfqpoint{10.098572in}{5.591412in}}{\pgfqpoint{10.087522in}{5.591412in}}%
\pgfpathcurveto{\pgfqpoint{10.076471in}{5.591412in}}{\pgfqpoint{10.065872in}{5.587022in}}{\pgfqpoint{10.058059in}{5.579209in}}%
\pgfpathcurveto{\pgfqpoint{10.050245in}{5.571395in}}{\pgfqpoint{10.045855in}{5.560796in}}{\pgfqpoint{10.045855in}{5.549746in}}%
\pgfpathcurveto{\pgfqpoint{10.045855in}{5.538696in}}{\pgfqpoint{10.050245in}{5.528097in}}{\pgfqpoint{10.058059in}{5.520283in}}%
\pgfpathcurveto{\pgfqpoint{10.065872in}{5.512469in}}{\pgfqpoint{10.076471in}{5.508079in}}{\pgfqpoint{10.087522in}{5.508079in}}%
\pgfpathlineto{\pgfqpoint{10.087522in}{5.508079in}}%
\pgfpathclose%
\pgfusepath{stroke}%
\end{pgfscope}%
\begin{pgfscope}%
\pgfpathrectangle{\pgfqpoint{7.394209in}{0.375000in}}{\pgfqpoint{6.356833in}{5.175000in}}%
\pgfusepath{clip}%
\pgfsetbuttcap%
\pgfsetroundjoin%
\pgfsetlinewidth{1.003750pt}%
\definecolor{currentstroke}{rgb}{0.827451,0.827451,0.827451}%
\pgfsetstrokecolor{currentstroke}%
\pgfsetdash{}{0pt}%
\pgfpathmoveto{\pgfqpoint{10.337309in}{3.516159in}}%
\pgfpathcurveto{\pgfqpoint{10.348360in}{3.516159in}}{\pgfqpoint{10.358959in}{3.520549in}}{\pgfqpoint{10.366772in}{3.528362in}}%
\pgfpathcurveto{\pgfqpoint{10.374586in}{3.536176in}}{\pgfqpoint{10.378976in}{3.546775in}}{\pgfqpoint{10.378976in}{3.557825in}}%
\pgfpathcurveto{\pgfqpoint{10.378976in}{3.568875in}}{\pgfqpoint{10.374586in}{3.579474in}}{\pgfqpoint{10.366772in}{3.587288in}}%
\pgfpathcurveto{\pgfqpoint{10.358959in}{3.595102in}}{\pgfqpoint{10.348360in}{3.599492in}}{\pgfqpoint{10.337309in}{3.599492in}}%
\pgfpathcurveto{\pgfqpoint{10.326259in}{3.599492in}}{\pgfqpoint{10.315660in}{3.595102in}}{\pgfqpoint{10.307847in}{3.587288in}}%
\pgfpathcurveto{\pgfqpoint{10.300033in}{3.579474in}}{\pgfqpoint{10.295643in}{3.568875in}}{\pgfqpoint{10.295643in}{3.557825in}}%
\pgfpathcurveto{\pgfqpoint{10.295643in}{3.546775in}}{\pgfqpoint{10.300033in}{3.536176in}}{\pgfqpoint{10.307847in}{3.528362in}}%
\pgfpathcurveto{\pgfqpoint{10.315660in}{3.520549in}}{\pgfqpoint{10.326259in}{3.516159in}}{\pgfqpoint{10.337309in}{3.516159in}}%
\pgfpathlineto{\pgfqpoint{10.337309in}{3.516159in}}%
\pgfpathclose%
\pgfusepath{stroke}%
\end{pgfscope}%
\begin{pgfscope}%
\pgfpathrectangle{\pgfqpoint{7.394209in}{0.375000in}}{\pgfqpoint{6.356833in}{5.175000in}}%
\pgfusepath{clip}%
\pgfsetbuttcap%
\pgfsetroundjoin%
\pgfsetlinewidth{1.003750pt}%
\definecolor{currentstroke}{rgb}{0.827451,0.827451,0.827451}%
\pgfsetstrokecolor{currentstroke}%
\pgfsetdash{}{0pt}%
\pgfpathmoveto{\pgfqpoint{7.908814in}{0.648950in}}%
\pgfpathcurveto{\pgfqpoint{7.919864in}{0.648950in}}{\pgfqpoint{7.930463in}{0.653340in}}{\pgfqpoint{7.938276in}{0.661154in}}%
\pgfpathcurveto{\pgfqpoint{7.946090in}{0.668967in}}{\pgfqpoint{7.950480in}{0.679566in}}{\pgfqpoint{7.950480in}{0.690616in}}%
\pgfpathcurveto{\pgfqpoint{7.950480in}{0.701666in}}{\pgfqpoint{7.946090in}{0.712266in}}{\pgfqpoint{7.938276in}{0.720079in}}%
\pgfpathcurveto{\pgfqpoint{7.930463in}{0.727893in}}{\pgfqpoint{7.919864in}{0.732283in}}{\pgfqpoint{7.908814in}{0.732283in}}%
\pgfpathcurveto{\pgfqpoint{7.897763in}{0.732283in}}{\pgfqpoint{7.887164in}{0.727893in}}{\pgfqpoint{7.879351in}{0.720079in}}%
\pgfpathcurveto{\pgfqpoint{7.871537in}{0.712266in}}{\pgfqpoint{7.867147in}{0.701666in}}{\pgfqpoint{7.867147in}{0.690616in}}%
\pgfpathcurveto{\pgfqpoint{7.867147in}{0.679566in}}{\pgfqpoint{7.871537in}{0.668967in}}{\pgfqpoint{7.879351in}{0.661154in}}%
\pgfpathcurveto{\pgfqpoint{7.887164in}{0.653340in}}{\pgfqpoint{7.897763in}{0.648950in}}{\pgfqpoint{7.908814in}{0.648950in}}%
\pgfpathlineto{\pgfqpoint{7.908814in}{0.648950in}}%
\pgfpathclose%
\pgfusepath{stroke}%
\end{pgfscope}%
\begin{pgfscope}%
\pgfpathrectangle{\pgfqpoint{7.394209in}{0.375000in}}{\pgfqpoint{6.356833in}{5.175000in}}%
\pgfusepath{clip}%
\pgfsetbuttcap%
\pgfsetroundjoin%
\pgfsetlinewidth{1.003750pt}%
\definecolor{currentstroke}{rgb}{0.827451,0.827451,0.827451}%
\pgfsetstrokecolor{currentstroke}%
\pgfsetdash{}{0pt}%
\pgfpathmoveto{\pgfqpoint{10.123035in}{5.508152in}}%
\pgfpathcurveto{\pgfqpoint{10.134085in}{5.508152in}}{\pgfqpoint{10.144684in}{5.512543in}}{\pgfqpoint{10.152497in}{5.520356in}}%
\pgfpathcurveto{\pgfqpoint{10.160311in}{5.528170in}}{\pgfqpoint{10.164701in}{5.538769in}}{\pgfqpoint{10.164701in}{5.549819in}}%
\pgfpathcurveto{\pgfqpoint{10.164701in}{5.560869in}}{\pgfqpoint{10.160311in}{5.571468in}}{\pgfqpoint{10.152497in}{5.579282in}}%
\pgfpathcurveto{\pgfqpoint{10.144684in}{5.587095in}}{\pgfqpoint{10.134085in}{5.591486in}}{\pgfqpoint{10.123035in}{5.591486in}}%
\pgfpathcurveto{\pgfqpoint{10.111984in}{5.591486in}}{\pgfqpoint{10.101385in}{5.587095in}}{\pgfqpoint{10.093572in}{5.579282in}}%
\pgfpathcurveto{\pgfqpoint{10.085758in}{5.571468in}}{\pgfqpoint{10.081368in}{5.560869in}}{\pgfqpoint{10.081368in}{5.549819in}}%
\pgfpathcurveto{\pgfqpoint{10.081368in}{5.538769in}}{\pgfqpoint{10.085758in}{5.528170in}}{\pgfqpoint{10.093572in}{5.520356in}}%
\pgfpathcurveto{\pgfqpoint{10.101385in}{5.512543in}}{\pgfqpoint{10.111984in}{5.508152in}}{\pgfqpoint{10.123035in}{5.508152in}}%
\pgfpathlineto{\pgfqpoint{10.123035in}{5.508152in}}%
\pgfpathclose%
\pgfusepath{stroke}%
\end{pgfscope}%
\begin{pgfscope}%
\pgfpathrectangle{\pgfqpoint{7.394209in}{0.375000in}}{\pgfqpoint{6.356833in}{5.175000in}}%
\pgfusepath{clip}%
\pgfsetbuttcap%
\pgfsetroundjoin%
\pgfsetlinewidth{1.003750pt}%
\definecolor{currentstroke}{rgb}{0.827451,0.827451,0.827451}%
\pgfsetstrokecolor{currentstroke}%
\pgfsetdash{}{0pt}%
\pgfpathmoveto{\pgfqpoint{9.171508in}{3.708114in}}%
\pgfpathcurveto{\pgfqpoint{9.182558in}{3.708114in}}{\pgfqpoint{9.193157in}{3.712505in}}{\pgfqpoint{9.200971in}{3.720318in}}%
\pgfpathcurveto{\pgfqpoint{9.208785in}{3.728132in}}{\pgfqpoint{9.213175in}{3.738731in}}{\pgfqpoint{9.213175in}{3.749781in}}%
\pgfpathcurveto{\pgfqpoint{9.213175in}{3.760831in}}{\pgfqpoint{9.208785in}{3.771430in}}{\pgfqpoint{9.200971in}{3.779244in}}%
\pgfpathcurveto{\pgfqpoint{9.193157in}{3.787057in}}{\pgfqpoint{9.182558in}{3.791448in}}{\pgfqpoint{9.171508in}{3.791448in}}%
\pgfpathcurveto{\pgfqpoint{9.160458in}{3.791448in}}{\pgfqpoint{9.149859in}{3.787057in}}{\pgfqpoint{9.142045in}{3.779244in}}%
\pgfpathcurveto{\pgfqpoint{9.134232in}{3.771430in}}{\pgfqpoint{9.129842in}{3.760831in}}{\pgfqpoint{9.129842in}{3.749781in}}%
\pgfpathcurveto{\pgfqpoint{9.129842in}{3.738731in}}{\pgfqpoint{9.134232in}{3.728132in}}{\pgfqpoint{9.142045in}{3.720318in}}%
\pgfpathcurveto{\pgfqpoint{9.149859in}{3.712505in}}{\pgfqpoint{9.160458in}{3.708114in}}{\pgfqpoint{9.171508in}{3.708114in}}%
\pgfpathlineto{\pgfqpoint{9.171508in}{3.708114in}}%
\pgfpathclose%
\pgfusepath{stroke}%
\end{pgfscope}%
\begin{pgfscope}%
\pgfpathrectangle{\pgfqpoint{7.394209in}{0.375000in}}{\pgfqpoint{6.356833in}{5.175000in}}%
\pgfusepath{clip}%
\pgfsetbuttcap%
\pgfsetroundjoin%
\pgfsetlinewidth{1.003750pt}%
\definecolor{currentstroke}{rgb}{0.827451,0.827451,0.827451}%
\pgfsetstrokecolor{currentstroke}%
\pgfsetdash{}{0pt}%
\pgfpathmoveto{\pgfqpoint{8.805731in}{3.127856in}}%
\pgfpathcurveto{\pgfqpoint{8.816782in}{3.127856in}}{\pgfqpoint{8.827381in}{3.132246in}}{\pgfqpoint{8.835194in}{3.140060in}}%
\pgfpathcurveto{\pgfqpoint{8.843008in}{3.147873in}}{\pgfqpoint{8.847398in}{3.158472in}}{\pgfqpoint{8.847398in}{3.169523in}}%
\pgfpathcurveto{\pgfqpoint{8.847398in}{3.180573in}}{\pgfqpoint{8.843008in}{3.191172in}}{\pgfqpoint{8.835194in}{3.198985in}}%
\pgfpathcurveto{\pgfqpoint{8.827381in}{3.206799in}}{\pgfqpoint{8.816782in}{3.211189in}}{\pgfqpoint{8.805731in}{3.211189in}}%
\pgfpathcurveto{\pgfqpoint{8.794681in}{3.211189in}}{\pgfqpoint{8.784082in}{3.206799in}}{\pgfqpoint{8.776269in}{3.198985in}}%
\pgfpathcurveto{\pgfqpoint{8.768455in}{3.191172in}}{\pgfqpoint{8.764065in}{3.180573in}}{\pgfqpoint{8.764065in}{3.169523in}}%
\pgfpathcurveto{\pgfqpoint{8.764065in}{3.158472in}}{\pgfqpoint{8.768455in}{3.147873in}}{\pgfqpoint{8.776269in}{3.140060in}}%
\pgfpathcurveto{\pgfqpoint{8.784082in}{3.132246in}}{\pgfqpoint{8.794681in}{3.127856in}}{\pgfqpoint{8.805731in}{3.127856in}}%
\pgfpathlineto{\pgfqpoint{8.805731in}{3.127856in}}%
\pgfpathclose%
\pgfusepath{stroke}%
\end{pgfscope}%
\begin{pgfscope}%
\pgfpathrectangle{\pgfqpoint{7.394209in}{0.375000in}}{\pgfqpoint{6.356833in}{5.175000in}}%
\pgfusepath{clip}%
\pgfsetbuttcap%
\pgfsetroundjoin%
\pgfsetlinewidth{1.003750pt}%
\definecolor{currentstroke}{rgb}{0.827451,0.827451,0.827451}%
\pgfsetstrokecolor{currentstroke}%
\pgfsetdash{}{0pt}%
\pgfpathmoveto{\pgfqpoint{11.173695in}{4.055104in}}%
\pgfpathcurveto{\pgfqpoint{11.184745in}{4.055104in}}{\pgfqpoint{11.195344in}{4.059494in}}{\pgfqpoint{11.203157in}{4.067308in}}%
\pgfpathcurveto{\pgfqpoint{11.210971in}{4.075122in}}{\pgfqpoint{11.215361in}{4.085721in}}{\pgfqpoint{11.215361in}{4.096771in}}%
\pgfpathcurveto{\pgfqpoint{11.215361in}{4.107821in}}{\pgfqpoint{11.210971in}{4.118420in}}{\pgfqpoint{11.203157in}{4.126234in}}%
\pgfpathcurveto{\pgfqpoint{11.195344in}{4.134047in}}{\pgfqpoint{11.184745in}{4.138437in}}{\pgfqpoint{11.173695in}{4.138437in}}%
\pgfpathcurveto{\pgfqpoint{11.162645in}{4.138437in}}{\pgfqpoint{11.152046in}{4.134047in}}{\pgfqpoint{11.144232in}{4.126234in}}%
\pgfpathcurveto{\pgfqpoint{11.136418in}{4.118420in}}{\pgfqpoint{11.132028in}{4.107821in}}{\pgfqpoint{11.132028in}{4.096771in}}%
\pgfpathcurveto{\pgfqpoint{11.132028in}{4.085721in}}{\pgfqpoint{11.136418in}{4.075122in}}{\pgfqpoint{11.144232in}{4.067308in}}%
\pgfpathcurveto{\pgfqpoint{11.152046in}{4.059494in}}{\pgfqpoint{11.162645in}{4.055104in}}{\pgfqpoint{11.173695in}{4.055104in}}%
\pgfpathlineto{\pgfqpoint{11.173695in}{4.055104in}}%
\pgfpathclose%
\pgfusepath{stroke}%
\end{pgfscope}%
\begin{pgfscope}%
\pgfpathrectangle{\pgfqpoint{7.394209in}{0.375000in}}{\pgfqpoint{6.356833in}{5.175000in}}%
\pgfusepath{clip}%
\pgfsetbuttcap%
\pgfsetroundjoin%
\pgfsetlinewidth{1.003750pt}%
\definecolor{currentstroke}{rgb}{0.827451,0.827451,0.827451}%
\pgfsetstrokecolor{currentstroke}%
\pgfsetdash{}{0pt}%
\pgfpathmoveto{\pgfqpoint{9.991629in}{4.414677in}}%
\pgfpathcurveto{\pgfqpoint{10.002679in}{4.414677in}}{\pgfqpoint{10.013279in}{4.419067in}}{\pgfqpoint{10.021092in}{4.426881in}}%
\pgfpathcurveto{\pgfqpoint{10.028906in}{4.434694in}}{\pgfqpoint{10.033296in}{4.445293in}}{\pgfqpoint{10.033296in}{4.456343in}}%
\pgfpathcurveto{\pgfqpoint{10.033296in}{4.467394in}}{\pgfqpoint{10.028906in}{4.477993in}}{\pgfqpoint{10.021092in}{4.485806in}}%
\pgfpathcurveto{\pgfqpoint{10.013279in}{4.493620in}}{\pgfqpoint{10.002679in}{4.498010in}}{\pgfqpoint{9.991629in}{4.498010in}}%
\pgfpathcurveto{\pgfqpoint{9.980579in}{4.498010in}}{\pgfqpoint{9.969980in}{4.493620in}}{\pgfqpoint{9.962167in}{4.485806in}}%
\pgfpathcurveto{\pgfqpoint{9.954353in}{4.477993in}}{\pgfqpoint{9.949963in}{4.467394in}}{\pgfqpoint{9.949963in}{4.456343in}}%
\pgfpathcurveto{\pgfqpoint{9.949963in}{4.445293in}}{\pgfqpoint{9.954353in}{4.434694in}}{\pgfqpoint{9.962167in}{4.426881in}}%
\pgfpathcurveto{\pgfqpoint{9.969980in}{4.419067in}}{\pgfqpoint{9.980579in}{4.414677in}}{\pgfqpoint{9.991629in}{4.414677in}}%
\pgfpathlineto{\pgfqpoint{9.991629in}{4.414677in}}%
\pgfpathclose%
\pgfusepath{stroke}%
\end{pgfscope}%
\begin{pgfscope}%
\pgfpathrectangle{\pgfqpoint{7.394209in}{0.375000in}}{\pgfqpoint{6.356833in}{5.175000in}}%
\pgfusepath{clip}%
\pgfsetbuttcap%
\pgfsetroundjoin%
\pgfsetlinewidth{1.003750pt}%
\definecolor{currentstroke}{rgb}{0.827451,0.827451,0.827451}%
\pgfsetstrokecolor{currentstroke}%
\pgfsetdash{}{0pt}%
\pgfpathmoveto{\pgfqpoint{13.086899in}{5.467176in}}%
\pgfpathcurveto{\pgfqpoint{13.097949in}{5.467176in}}{\pgfqpoint{13.108548in}{5.471566in}}{\pgfqpoint{13.116361in}{5.479380in}}%
\pgfpathcurveto{\pgfqpoint{13.124175in}{5.487194in}}{\pgfqpoint{13.128565in}{5.497793in}}{\pgfqpoint{13.128565in}{5.508843in}}%
\pgfpathcurveto{\pgfqpoint{13.128565in}{5.519893in}}{\pgfqpoint{13.124175in}{5.530492in}}{\pgfqpoint{13.116361in}{5.538306in}}%
\pgfpathcurveto{\pgfqpoint{13.108548in}{5.546119in}}{\pgfqpoint{13.097949in}{5.550509in}}{\pgfqpoint{13.086899in}{5.550509in}}%
\pgfpathcurveto{\pgfqpoint{13.075848in}{5.550509in}}{\pgfqpoint{13.065249in}{5.546119in}}{\pgfqpoint{13.057436in}{5.538306in}}%
\pgfpathcurveto{\pgfqpoint{13.049622in}{5.530492in}}{\pgfqpoint{13.045232in}{5.519893in}}{\pgfqpoint{13.045232in}{5.508843in}}%
\pgfpathcurveto{\pgfqpoint{13.045232in}{5.497793in}}{\pgfqpoint{13.049622in}{5.487194in}}{\pgfqpoint{13.057436in}{5.479380in}}%
\pgfpathcurveto{\pgfqpoint{13.065249in}{5.471566in}}{\pgfqpoint{13.075848in}{5.467176in}}{\pgfqpoint{13.086899in}{5.467176in}}%
\pgfpathlineto{\pgfqpoint{13.086899in}{5.467176in}}%
\pgfpathclose%
\pgfusepath{stroke}%
\end{pgfscope}%
\begin{pgfscope}%
\pgfpathrectangle{\pgfqpoint{7.394209in}{0.375000in}}{\pgfqpoint{6.356833in}{5.175000in}}%
\pgfusepath{clip}%
\pgfsetbuttcap%
\pgfsetroundjoin%
\pgfsetlinewidth{1.003750pt}%
\definecolor{currentstroke}{rgb}{0.827451,0.827451,0.827451}%
\pgfsetstrokecolor{currentstroke}%
\pgfsetdash{}{0pt}%
\pgfpathmoveto{\pgfqpoint{8.538948in}{1.477963in}}%
\pgfpathcurveto{\pgfqpoint{8.549998in}{1.477963in}}{\pgfqpoint{8.560597in}{1.482354in}}{\pgfqpoint{8.568411in}{1.490167in}}%
\pgfpathcurveto{\pgfqpoint{8.576225in}{1.497981in}}{\pgfqpoint{8.580615in}{1.508580in}}{\pgfqpoint{8.580615in}{1.519630in}}%
\pgfpathcurveto{\pgfqpoint{8.580615in}{1.530680in}}{\pgfqpoint{8.576225in}{1.541279in}}{\pgfqpoint{8.568411in}{1.549093in}}%
\pgfpathcurveto{\pgfqpoint{8.560597in}{1.556906in}}{\pgfqpoint{8.549998in}{1.561297in}}{\pgfqpoint{8.538948in}{1.561297in}}%
\pgfpathcurveto{\pgfqpoint{8.527898in}{1.561297in}}{\pgfqpoint{8.517299in}{1.556906in}}{\pgfqpoint{8.509485in}{1.549093in}}%
\pgfpathcurveto{\pgfqpoint{8.501672in}{1.541279in}}{\pgfqpoint{8.497281in}{1.530680in}}{\pgfqpoint{8.497281in}{1.519630in}}%
\pgfpathcurveto{\pgfqpoint{8.497281in}{1.508580in}}{\pgfqpoint{8.501672in}{1.497981in}}{\pgfqpoint{8.509485in}{1.490167in}}%
\pgfpathcurveto{\pgfqpoint{8.517299in}{1.482354in}}{\pgfqpoint{8.527898in}{1.477963in}}{\pgfqpoint{8.538948in}{1.477963in}}%
\pgfpathlineto{\pgfqpoint{8.538948in}{1.477963in}}%
\pgfpathclose%
\pgfusepath{stroke}%
\end{pgfscope}%
\begin{pgfscope}%
\pgfpathrectangle{\pgfqpoint{7.394209in}{0.375000in}}{\pgfqpoint{6.356833in}{5.175000in}}%
\pgfusepath{clip}%
\pgfsetbuttcap%
\pgfsetroundjoin%
\pgfsetlinewidth{1.003750pt}%
\definecolor{currentstroke}{rgb}{0.827451,0.827451,0.827451}%
\pgfsetstrokecolor{currentstroke}%
\pgfsetdash{}{0pt}%
\pgfpathmoveto{\pgfqpoint{8.477754in}{2.784299in}}%
\pgfpathcurveto{\pgfqpoint{8.488804in}{2.784299in}}{\pgfqpoint{8.499403in}{2.788689in}}{\pgfqpoint{8.507217in}{2.796502in}}%
\pgfpathcurveto{\pgfqpoint{8.515030in}{2.804316in}}{\pgfqpoint{8.519421in}{2.814915in}}{\pgfqpoint{8.519421in}{2.825965in}}%
\pgfpathcurveto{\pgfqpoint{8.519421in}{2.837015in}}{\pgfqpoint{8.515030in}{2.847614in}}{\pgfqpoint{8.507217in}{2.855428in}}%
\pgfpathcurveto{\pgfqpoint{8.499403in}{2.863242in}}{\pgfqpoint{8.488804in}{2.867632in}}{\pgfqpoint{8.477754in}{2.867632in}}%
\pgfpathcurveto{\pgfqpoint{8.466704in}{2.867632in}}{\pgfqpoint{8.456105in}{2.863242in}}{\pgfqpoint{8.448291in}{2.855428in}}%
\pgfpathcurveto{\pgfqpoint{8.440478in}{2.847614in}}{\pgfqpoint{8.436087in}{2.837015in}}{\pgfqpoint{8.436087in}{2.825965in}}%
\pgfpathcurveto{\pgfqpoint{8.436087in}{2.814915in}}{\pgfqpoint{8.440478in}{2.804316in}}{\pgfqpoint{8.448291in}{2.796502in}}%
\pgfpathcurveto{\pgfqpoint{8.456105in}{2.788689in}}{\pgfqpoint{8.466704in}{2.784299in}}{\pgfqpoint{8.477754in}{2.784299in}}%
\pgfpathlineto{\pgfqpoint{8.477754in}{2.784299in}}%
\pgfpathclose%
\pgfusepath{stroke}%
\end{pgfscope}%
\begin{pgfscope}%
\pgfpathrectangle{\pgfqpoint{7.394209in}{0.375000in}}{\pgfqpoint{6.356833in}{5.175000in}}%
\pgfusepath{clip}%
\pgfsetbuttcap%
\pgfsetroundjoin%
\pgfsetlinewidth{1.003750pt}%
\definecolor{currentstroke}{rgb}{0.827451,0.827451,0.827451}%
\pgfsetstrokecolor{currentstroke}%
\pgfsetdash{}{0pt}%
\pgfpathmoveto{\pgfqpoint{11.180736in}{4.181055in}}%
\pgfpathcurveto{\pgfqpoint{11.191786in}{4.181055in}}{\pgfqpoint{11.202385in}{4.185445in}}{\pgfqpoint{11.210198in}{4.193259in}}%
\pgfpathcurveto{\pgfqpoint{11.218012in}{4.201073in}}{\pgfqpoint{11.222402in}{4.211672in}}{\pgfqpoint{11.222402in}{4.222722in}}%
\pgfpathcurveto{\pgfqpoint{11.222402in}{4.233772in}}{\pgfqpoint{11.218012in}{4.244371in}}{\pgfqpoint{11.210198in}{4.252185in}}%
\pgfpathcurveto{\pgfqpoint{11.202385in}{4.259998in}}{\pgfqpoint{11.191786in}{4.264388in}}{\pgfqpoint{11.180736in}{4.264388in}}%
\pgfpathcurveto{\pgfqpoint{11.169685in}{4.264388in}}{\pgfqpoint{11.159086in}{4.259998in}}{\pgfqpoint{11.151273in}{4.252185in}}%
\pgfpathcurveto{\pgfqpoint{11.143459in}{4.244371in}}{\pgfqpoint{11.139069in}{4.233772in}}{\pgfqpoint{11.139069in}{4.222722in}}%
\pgfpathcurveto{\pgfqpoint{11.139069in}{4.211672in}}{\pgfqpoint{11.143459in}{4.201073in}}{\pgfqpoint{11.151273in}{4.193259in}}%
\pgfpathcurveto{\pgfqpoint{11.159086in}{4.185445in}}{\pgfqpoint{11.169685in}{4.181055in}}{\pgfqpoint{11.180736in}{4.181055in}}%
\pgfpathlineto{\pgfqpoint{11.180736in}{4.181055in}}%
\pgfpathclose%
\pgfusepath{stroke}%
\end{pgfscope}%
\begin{pgfscope}%
\pgfpathrectangle{\pgfqpoint{7.394209in}{0.375000in}}{\pgfqpoint{6.356833in}{5.175000in}}%
\pgfusepath{clip}%
\pgfsetbuttcap%
\pgfsetroundjoin%
\pgfsetlinewidth{1.003750pt}%
\definecolor{currentstroke}{rgb}{0.827451,0.827451,0.827451}%
\pgfsetstrokecolor{currentstroke}%
\pgfsetdash{}{0pt}%
\pgfpathmoveto{\pgfqpoint{13.202710in}{5.498078in}}%
\pgfpathcurveto{\pgfqpoint{13.213760in}{5.498078in}}{\pgfqpoint{13.224359in}{5.502468in}}{\pgfqpoint{13.232172in}{5.510282in}}%
\pgfpathcurveto{\pgfqpoint{13.239986in}{5.518096in}}{\pgfqpoint{13.244376in}{5.528695in}}{\pgfqpoint{13.244376in}{5.539745in}}%
\pgfpathcurveto{\pgfqpoint{13.244376in}{5.550795in}}{\pgfqpoint{13.239986in}{5.561394in}}{\pgfqpoint{13.232172in}{5.569207in}}%
\pgfpathcurveto{\pgfqpoint{13.224359in}{5.577021in}}{\pgfqpoint{13.213760in}{5.581411in}}{\pgfqpoint{13.202710in}{5.581411in}}%
\pgfpathcurveto{\pgfqpoint{13.191659in}{5.581411in}}{\pgfqpoint{13.181060in}{5.577021in}}{\pgfqpoint{13.173247in}{5.569207in}}%
\pgfpathcurveto{\pgfqpoint{13.165433in}{5.561394in}}{\pgfqpoint{13.161043in}{5.550795in}}{\pgfqpoint{13.161043in}{5.539745in}}%
\pgfpathcurveto{\pgfqpoint{13.161043in}{5.528695in}}{\pgfqpoint{13.165433in}{5.518096in}}{\pgfqpoint{13.173247in}{5.510282in}}%
\pgfpathcurveto{\pgfqpoint{13.181060in}{5.502468in}}{\pgfqpoint{13.191659in}{5.498078in}}{\pgfqpoint{13.202710in}{5.498078in}}%
\pgfpathlineto{\pgfqpoint{13.202710in}{5.498078in}}%
\pgfpathclose%
\pgfusepath{stroke}%
\end{pgfscope}%
\begin{pgfscope}%
\pgfpathrectangle{\pgfqpoint{7.394209in}{0.375000in}}{\pgfqpoint{6.356833in}{5.175000in}}%
\pgfusepath{clip}%
\pgfsetbuttcap%
\pgfsetroundjoin%
\pgfsetlinewidth{1.003750pt}%
\definecolor{currentstroke}{rgb}{0.827451,0.827451,0.827451}%
\pgfsetstrokecolor{currentstroke}%
\pgfsetdash{}{0pt}%
\pgfpathmoveto{\pgfqpoint{8.737431in}{3.131158in}}%
\pgfpathcurveto{\pgfqpoint{8.748481in}{3.131158in}}{\pgfqpoint{8.759080in}{3.135549in}}{\pgfqpoint{8.766894in}{3.143362in}}%
\pgfpathcurveto{\pgfqpoint{8.774708in}{3.151176in}}{\pgfqpoint{8.779098in}{3.161775in}}{\pgfqpoint{8.779098in}{3.172825in}}%
\pgfpathcurveto{\pgfqpoint{8.779098in}{3.183875in}}{\pgfqpoint{8.774708in}{3.194474in}}{\pgfqpoint{8.766894in}{3.202288in}}%
\pgfpathcurveto{\pgfqpoint{8.759080in}{3.210101in}}{\pgfqpoint{8.748481in}{3.214492in}}{\pgfqpoint{8.737431in}{3.214492in}}%
\pgfpathcurveto{\pgfqpoint{8.726381in}{3.214492in}}{\pgfqpoint{8.715782in}{3.210101in}}{\pgfqpoint{8.707969in}{3.202288in}}%
\pgfpathcurveto{\pgfqpoint{8.700155in}{3.194474in}}{\pgfqpoint{8.695765in}{3.183875in}}{\pgfqpoint{8.695765in}{3.172825in}}%
\pgfpathcurveto{\pgfqpoint{8.695765in}{3.161775in}}{\pgfqpoint{8.700155in}{3.151176in}}{\pgfqpoint{8.707969in}{3.143362in}}%
\pgfpathcurveto{\pgfqpoint{8.715782in}{3.135549in}}{\pgfqpoint{8.726381in}{3.131158in}}{\pgfqpoint{8.737431in}{3.131158in}}%
\pgfpathlineto{\pgfqpoint{8.737431in}{3.131158in}}%
\pgfpathclose%
\pgfusepath{stroke}%
\end{pgfscope}%
\begin{pgfscope}%
\pgfpathrectangle{\pgfqpoint{7.394209in}{0.375000in}}{\pgfqpoint{6.356833in}{5.175000in}}%
\pgfusepath{clip}%
\pgfsetbuttcap%
\pgfsetroundjoin%
\pgfsetlinewidth{1.003750pt}%
\definecolor{currentstroke}{rgb}{0.827451,0.827451,0.827451}%
\pgfsetstrokecolor{currentstroke}%
\pgfsetdash{}{0pt}%
\pgfpathmoveto{\pgfqpoint{7.731431in}{1.477963in}}%
\pgfpathcurveto{\pgfqpoint{7.742481in}{1.477963in}}{\pgfqpoint{7.753081in}{1.482354in}}{\pgfqpoint{7.760894in}{1.490167in}}%
\pgfpathcurveto{\pgfqpoint{7.768708in}{1.497981in}}{\pgfqpoint{7.773098in}{1.508580in}}{\pgfqpoint{7.773098in}{1.519630in}}%
\pgfpathcurveto{\pgfqpoint{7.773098in}{1.530680in}}{\pgfqpoint{7.768708in}{1.541279in}}{\pgfqpoint{7.760894in}{1.549093in}}%
\pgfpathcurveto{\pgfqpoint{7.753081in}{1.556906in}}{\pgfqpoint{7.742481in}{1.561297in}}{\pgfqpoint{7.731431in}{1.561297in}}%
\pgfpathcurveto{\pgfqpoint{7.720381in}{1.561297in}}{\pgfqpoint{7.709782in}{1.556906in}}{\pgfqpoint{7.701969in}{1.549093in}}%
\pgfpathcurveto{\pgfqpoint{7.694155in}{1.541279in}}{\pgfqpoint{7.689765in}{1.530680in}}{\pgfqpoint{7.689765in}{1.519630in}}%
\pgfpathcurveto{\pgfqpoint{7.689765in}{1.508580in}}{\pgfqpoint{7.694155in}{1.497981in}}{\pgfqpoint{7.701969in}{1.490167in}}%
\pgfpathcurveto{\pgfqpoint{7.709782in}{1.482354in}}{\pgfqpoint{7.720381in}{1.477963in}}{\pgfqpoint{7.731431in}{1.477963in}}%
\pgfpathlineto{\pgfqpoint{7.731431in}{1.477963in}}%
\pgfpathclose%
\pgfusepath{stroke}%
\end{pgfscope}%
\begin{pgfscope}%
\pgfpathrectangle{\pgfqpoint{7.394209in}{0.375000in}}{\pgfqpoint{6.356833in}{5.175000in}}%
\pgfusepath{clip}%
\pgfsetbuttcap%
\pgfsetroundjoin%
\pgfsetlinewidth{1.003750pt}%
\definecolor{currentstroke}{rgb}{0.827451,0.827451,0.827451}%
\pgfsetstrokecolor{currentstroke}%
\pgfsetdash{}{0pt}%
\pgfpathmoveto{\pgfqpoint{7.971516in}{2.579256in}}%
\pgfpathcurveto{\pgfqpoint{7.982566in}{2.579256in}}{\pgfqpoint{7.993165in}{2.583646in}}{\pgfqpoint{8.000979in}{2.591460in}}%
\pgfpathcurveto{\pgfqpoint{8.008792in}{2.599273in}}{\pgfqpoint{8.013183in}{2.609872in}}{\pgfqpoint{8.013183in}{2.620922in}}%
\pgfpathcurveto{\pgfqpoint{8.013183in}{2.631973in}}{\pgfqpoint{8.008792in}{2.642572in}}{\pgfqpoint{8.000979in}{2.650385in}}%
\pgfpathcurveto{\pgfqpoint{7.993165in}{2.658199in}}{\pgfqpoint{7.982566in}{2.662589in}}{\pgfqpoint{7.971516in}{2.662589in}}%
\pgfpathcurveto{\pgfqpoint{7.960466in}{2.662589in}}{\pgfqpoint{7.949867in}{2.658199in}}{\pgfqpoint{7.942053in}{2.650385in}}%
\pgfpathcurveto{\pgfqpoint{7.934240in}{2.642572in}}{\pgfqpoint{7.929849in}{2.631973in}}{\pgfqpoint{7.929849in}{2.620922in}}%
\pgfpathcurveto{\pgfqpoint{7.929849in}{2.609872in}}{\pgfqpoint{7.934240in}{2.599273in}}{\pgfqpoint{7.942053in}{2.591460in}}%
\pgfpathcurveto{\pgfqpoint{7.949867in}{2.583646in}}{\pgfqpoint{7.960466in}{2.579256in}}{\pgfqpoint{7.971516in}{2.579256in}}%
\pgfpathlineto{\pgfqpoint{7.971516in}{2.579256in}}%
\pgfpathclose%
\pgfusepath{stroke}%
\end{pgfscope}%
\begin{pgfscope}%
\pgfpathrectangle{\pgfqpoint{7.394209in}{0.375000in}}{\pgfqpoint{6.356833in}{5.175000in}}%
\pgfusepath{clip}%
\pgfsetbuttcap%
\pgfsetroundjoin%
\pgfsetlinewidth{1.003750pt}%
\definecolor{currentstroke}{rgb}{0.827451,0.827451,0.827451}%
\pgfsetstrokecolor{currentstroke}%
\pgfsetdash{}{0pt}%
\pgfpathmoveto{\pgfqpoint{9.895796in}{4.277697in}}%
\pgfpathcurveto{\pgfqpoint{9.906846in}{4.277697in}}{\pgfqpoint{9.917445in}{4.282087in}}{\pgfqpoint{9.925258in}{4.289901in}}%
\pgfpathcurveto{\pgfqpoint{9.933072in}{4.297715in}}{\pgfqpoint{9.937462in}{4.308314in}}{\pgfqpoint{9.937462in}{4.319364in}}%
\pgfpathcurveto{\pgfqpoint{9.937462in}{4.330414in}}{\pgfqpoint{9.933072in}{4.341013in}}{\pgfqpoint{9.925258in}{4.348827in}}%
\pgfpathcurveto{\pgfqpoint{9.917445in}{4.356640in}}{\pgfqpoint{9.906846in}{4.361030in}}{\pgfqpoint{9.895796in}{4.361030in}}%
\pgfpathcurveto{\pgfqpoint{9.884745in}{4.361030in}}{\pgfqpoint{9.874146in}{4.356640in}}{\pgfqpoint{9.866333in}{4.348827in}}%
\pgfpathcurveto{\pgfqpoint{9.858519in}{4.341013in}}{\pgfqpoint{9.854129in}{4.330414in}}{\pgfqpoint{9.854129in}{4.319364in}}%
\pgfpathcurveto{\pgfqpoint{9.854129in}{4.308314in}}{\pgfqpoint{9.858519in}{4.297715in}}{\pgfqpoint{9.866333in}{4.289901in}}%
\pgfpathcurveto{\pgfqpoint{9.874146in}{4.282087in}}{\pgfqpoint{9.884745in}{4.277697in}}{\pgfqpoint{9.895796in}{4.277697in}}%
\pgfpathlineto{\pgfqpoint{9.895796in}{4.277697in}}%
\pgfpathclose%
\pgfusepath{stroke}%
\end{pgfscope}%
\begin{pgfscope}%
\pgfpathrectangle{\pgfqpoint{7.394209in}{0.375000in}}{\pgfqpoint{6.356833in}{5.175000in}}%
\pgfusepath{clip}%
\pgfsetbuttcap%
\pgfsetroundjoin%
\pgfsetlinewidth{1.003750pt}%
\definecolor{currentstroke}{rgb}{0.827451,0.827451,0.827451}%
\pgfsetstrokecolor{currentstroke}%
\pgfsetdash{}{0pt}%
\pgfpathmoveto{\pgfqpoint{8.679195in}{2.395323in}}%
\pgfpathcurveto{\pgfqpoint{8.690245in}{2.395323in}}{\pgfqpoint{8.700844in}{2.399713in}}{\pgfqpoint{8.708657in}{2.407527in}}%
\pgfpathcurveto{\pgfqpoint{8.716471in}{2.415340in}}{\pgfqpoint{8.720861in}{2.425939in}}{\pgfqpoint{8.720861in}{2.436990in}}%
\pgfpathcurveto{\pgfqpoint{8.720861in}{2.448040in}}{\pgfqpoint{8.716471in}{2.458639in}}{\pgfqpoint{8.708657in}{2.466452in}}%
\pgfpathcurveto{\pgfqpoint{8.700844in}{2.474266in}}{\pgfqpoint{8.690245in}{2.478656in}}{\pgfqpoint{8.679195in}{2.478656in}}%
\pgfpathcurveto{\pgfqpoint{8.668145in}{2.478656in}}{\pgfqpoint{8.657546in}{2.474266in}}{\pgfqpoint{8.649732in}{2.466452in}}%
\pgfpathcurveto{\pgfqpoint{8.641918in}{2.458639in}}{\pgfqpoint{8.637528in}{2.448040in}}{\pgfqpoint{8.637528in}{2.436990in}}%
\pgfpathcurveto{\pgfqpoint{8.637528in}{2.425939in}}{\pgfqpoint{8.641918in}{2.415340in}}{\pgfqpoint{8.649732in}{2.407527in}}%
\pgfpathcurveto{\pgfqpoint{8.657546in}{2.399713in}}{\pgfqpoint{8.668145in}{2.395323in}}{\pgfqpoint{8.679195in}{2.395323in}}%
\pgfpathlineto{\pgfqpoint{8.679195in}{2.395323in}}%
\pgfpathclose%
\pgfusepath{stroke}%
\end{pgfscope}%
\begin{pgfscope}%
\pgfpathrectangle{\pgfqpoint{7.394209in}{0.375000in}}{\pgfqpoint{6.356833in}{5.175000in}}%
\pgfusepath{clip}%
\pgfsetbuttcap%
\pgfsetroundjoin%
\pgfsetlinewidth{1.003750pt}%
\definecolor{currentstroke}{rgb}{0.827451,0.827451,0.827451}%
\pgfsetstrokecolor{currentstroke}%
\pgfsetdash{}{0pt}%
\pgfpathmoveto{\pgfqpoint{11.535837in}{5.352981in}}%
\pgfpathcurveto{\pgfqpoint{11.546887in}{5.352981in}}{\pgfqpoint{11.557486in}{5.357371in}}{\pgfqpoint{11.565300in}{5.365185in}}%
\pgfpathcurveto{\pgfqpoint{11.573114in}{5.372998in}}{\pgfqpoint{11.577504in}{5.383598in}}{\pgfqpoint{11.577504in}{5.394648in}}%
\pgfpathcurveto{\pgfqpoint{11.577504in}{5.405698in}}{\pgfqpoint{11.573114in}{5.416297in}}{\pgfqpoint{11.565300in}{5.424110in}}%
\pgfpathcurveto{\pgfqpoint{11.557486in}{5.431924in}}{\pgfqpoint{11.546887in}{5.436314in}}{\pgfqpoint{11.535837in}{5.436314in}}%
\pgfpathcurveto{\pgfqpoint{11.524787in}{5.436314in}}{\pgfqpoint{11.514188in}{5.431924in}}{\pgfqpoint{11.506374in}{5.424110in}}%
\pgfpathcurveto{\pgfqpoint{11.498561in}{5.416297in}}{\pgfqpoint{11.494170in}{5.405698in}}{\pgfqpoint{11.494170in}{5.394648in}}%
\pgfpathcurveto{\pgfqpoint{11.494170in}{5.383598in}}{\pgfqpoint{11.498561in}{5.372998in}}{\pgfqpoint{11.506374in}{5.365185in}}%
\pgfpathcurveto{\pgfqpoint{11.514188in}{5.357371in}}{\pgfqpoint{11.524787in}{5.352981in}}{\pgfqpoint{11.535837in}{5.352981in}}%
\pgfpathlineto{\pgfqpoint{11.535837in}{5.352981in}}%
\pgfpathclose%
\pgfusepath{stroke}%
\end{pgfscope}%
\begin{pgfscope}%
\pgfpathrectangle{\pgfqpoint{7.394209in}{0.375000in}}{\pgfqpoint{6.356833in}{5.175000in}}%
\pgfusepath{clip}%
\pgfsetbuttcap%
\pgfsetroundjoin%
\pgfsetlinewidth{1.003750pt}%
\definecolor{currentstroke}{rgb}{0.827451,0.827451,0.827451}%
\pgfsetstrokecolor{currentstroke}%
\pgfsetdash{}{0pt}%
\pgfpathmoveto{\pgfqpoint{8.715339in}{2.734967in}}%
\pgfpathcurveto{\pgfqpoint{8.726389in}{2.734967in}}{\pgfqpoint{8.736988in}{2.739357in}}{\pgfqpoint{8.744802in}{2.747171in}}%
\pgfpathcurveto{\pgfqpoint{8.752615in}{2.754985in}}{\pgfqpoint{8.757006in}{2.765584in}}{\pgfqpoint{8.757006in}{2.776634in}}%
\pgfpathcurveto{\pgfqpoint{8.757006in}{2.787684in}}{\pgfqpoint{8.752615in}{2.798283in}}{\pgfqpoint{8.744802in}{2.806097in}}%
\pgfpathcurveto{\pgfqpoint{8.736988in}{2.813910in}}{\pgfqpoint{8.726389in}{2.818300in}}{\pgfqpoint{8.715339in}{2.818300in}}%
\pgfpathcurveto{\pgfqpoint{8.704289in}{2.818300in}}{\pgfqpoint{8.693690in}{2.813910in}}{\pgfqpoint{8.685876in}{2.806097in}}%
\pgfpathcurveto{\pgfqpoint{8.678062in}{2.798283in}}{\pgfqpoint{8.673672in}{2.787684in}}{\pgfqpoint{8.673672in}{2.776634in}}%
\pgfpathcurveto{\pgfqpoint{8.673672in}{2.765584in}}{\pgfqpoint{8.678062in}{2.754985in}}{\pgfqpoint{8.685876in}{2.747171in}}%
\pgfpathcurveto{\pgfqpoint{8.693690in}{2.739357in}}{\pgfqpoint{8.704289in}{2.734967in}}{\pgfqpoint{8.715339in}{2.734967in}}%
\pgfpathlineto{\pgfqpoint{8.715339in}{2.734967in}}%
\pgfpathclose%
\pgfusepath{stroke}%
\end{pgfscope}%
\begin{pgfscope}%
\pgfpathrectangle{\pgfqpoint{7.394209in}{0.375000in}}{\pgfqpoint{6.356833in}{5.175000in}}%
\pgfusepath{clip}%
\pgfsetbuttcap%
\pgfsetroundjoin%
\pgfsetlinewidth{1.003750pt}%
\definecolor{currentstroke}{rgb}{0.827451,0.827451,0.827451}%
\pgfsetstrokecolor{currentstroke}%
\pgfsetdash{}{0pt}%
\pgfpathmoveto{\pgfqpoint{8.962481in}{2.655121in}}%
\pgfpathcurveto{\pgfqpoint{8.973531in}{2.655121in}}{\pgfqpoint{8.984130in}{2.659511in}}{\pgfqpoint{8.991943in}{2.667325in}}%
\pgfpathcurveto{\pgfqpoint{8.999757in}{2.675138in}}{\pgfqpoint{9.004147in}{2.685737in}}{\pgfqpoint{9.004147in}{2.696787in}}%
\pgfpathcurveto{\pgfqpoint{9.004147in}{2.707837in}}{\pgfqpoint{8.999757in}{2.718437in}}{\pgfqpoint{8.991943in}{2.726250in}}%
\pgfpathcurveto{\pgfqpoint{8.984130in}{2.734064in}}{\pgfqpoint{8.973531in}{2.738454in}}{\pgfqpoint{8.962481in}{2.738454in}}%
\pgfpathcurveto{\pgfqpoint{8.951430in}{2.738454in}}{\pgfqpoint{8.940831in}{2.734064in}}{\pgfqpoint{8.933018in}{2.726250in}}%
\pgfpathcurveto{\pgfqpoint{8.925204in}{2.718437in}}{\pgfqpoint{8.920814in}{2.707837in}}{\pgfqpoint{8.920814in}{2.696787in}}%
\pgfpathcurveto{\pgfqpoint{8.920814in}{2.685737in}}{\pgfqpoint{8.925204in}{2.675138in}}{\pgfqpoint{8.933018in}{2.667325in}}%
\pgfpathcurveto{\pgfqpoint{8.940831in}{2.659511in}}{\pgfqpoint{8.951430in}{2.655121in}}{\pgfqpoint{8.962481in}{2.655121in}}%
\pgfpathlineto{\pgfqpoint{8.962481in}{2.655121in}}%
\pgfpathclose%
\pgfusepath{stroke}%
\end{pgfscope}%
\begin{pgfscope}%
\pgfpathrectangle{\pgfqpoint{7.394209in}{0.375000in}}{\pgfqpoint{6.356833in}{5.175000in}}%
\pgfusepath{clip}%
\pgfsetbuttcap%
\pgfsetroundjoin%
\pgfsetlinewidth{1.003750pt}%
\definecolor{currentstroke}{rgb}{0.827451,0.827451,0.827451}%
\pgfsetstrokecolor{currentstroke}%
\pgfsetdash{}{0pt}%
\pgfpathmoveto{\pgfqpoint{8.679195in}{1.663045in}}%
\pgfpathcurveto{\pgfqpoint{8.690245in}{1.663045in}}{\pgfqpoint{8.700844in}{1.667435in}}{\pgfqpoint{8.708657in}{1.675249in}}%
\pgfpathcurveto{\pgfqpoint{8.716471in}{1.683062in}}{\pgfqpoint{8.720861in}{1.693661in}}{\pgfqpoint{8.720861in}{1.704711in}}%
\pgfpathcurveto{\pgfqpoint{8.720861in}{1.715762in}}{\pgfqpoint{8.716471in}{1.726361in}}{\pgfqpoint{8.708657in}{1.734174in}}%
\pgfpathcurveto{\pgfqpoint{8.700844in}{1.741988in}}{\pgfqpoint{8.690245in}{1.746378in}}{\pgfqpoint{8.679195in}{1.746378in}}%
\pgfpathcurveto{\pgfqpoint{8.668145in}{1.746378in}}{\pgfqpoint{8.657546in}{1.741988in}}{\pgfqpoint{8.649732in}{1.734174in}}%
\pgfpathcurveto{\pgfqpoint{8.641918in}{1.726361in}}{\pgfqpoint{8.637528in}{1.715762in}}{\pgfqpoint{8.637528in}{1.704711in}}%
\pgfpathcurveto{\pgfqpoint{8.637528in}{1.693661in}}{\pgfqpoint{8.641918in}{1.683062in}}{\pgfqpoint{8.649732in}{1.675249in}}%
\pgfpathcurveto{\pgfqpoint{8.657546in}{1.667435in}}{\pgfqpoint{8.668145in}{1.663045in}}{\pgfqpoint{8.679195in}{1.663045in}}%
\pgfpathlineto{\pgfqpoint{8.679195in}{1.663045in}}%
\pgfpathclose%
\pgfusepath{stroke}%
\end{pgfscope}%
\begin{pgfscope}%
\pgfpathrectangle{\pgfqpoint{7.394209in}{0.375000in}}{\pgfqpoint{6.356833in}{5.175000in}}%
\pgfusepath{clip}%
\pgfsetbuttcap%
\pgfsetroundjoin%
\pgfsetlinewidth{1.003750pt}%
\definecolor{currentstroke}{rgb}{0.827451,0.827451,0.827451}%
\pgfsetstrokecolor{currentstroke}%
\pgfsetdash{}{0pt}%
\pgfpathmoveto{\pgfqpoint{8.901809in}{3.445372in}}%
\pgfpathcurveto{\pgfqpoint{8.912859in}{3.445372in}}{\pgfqpoint{8.923458in}{3.449762in}}{\pgfqpoint{8.931272in}{3.457576in}}%
\pgfpathcurveto{\pgfqpoint{8.939085in}{3.465389in}}{\pgfqpoint{8.943476in}{3.475988in}}{\pgfqpoint{8.943476in}{3.487038in}}%
\pgfpathcurveto{\pgfqpoint{8.943476in}{3.498089in}}{\pgfqpoint{8.939085in}{3.508688in}}{\pgfqpoint{8.931272in}{3.516501in}}%
\pgfpathcurveto{\pgfqpoint{8.923458in}{3.524315in}}{\pgfqpoint{8.912859in}{3.528705in}}{\pgfqpoint{8.901809in}{3.528705in}}%
\pgfpathcurveto{\pgfqpoint{8.890759in}{3.528705in}}{\pgfqpoint{8.880160in}{3.524315in}}{\pgfqpoint{8.872346in}{3.516501in}}%
\pgfpathcurveto{\pgfqpoint{8.864533in}{3.508688in}}{\pgfqpoint{8.860142in}{3.498089in}}{\pgfqpoint{8.860142in}{3.487038in}}%
\pgfpathcurveto{\pgfqpoint{8.860142in}{3.475988in}}{\pgfqpoint{8.864533in}{3.465389in}}{\pgfqpoint{8.872346in}{3.457576in}}%
\pgfpathcurveto{\pgfqpoint{8.880160in}{3.449762in}}{\pgfqpoint{8.890759in}{3.445372in}}{\pgfqpoint{8.901809in}{3.445372in}}%
\pgfpathlineto{\pgfqpoint{8.901809in}{3.445372in}}%
\pgfpathclose%
\pgfusepath{stroke}%
\end{pgfscope}%
\begin{pgfscope}%
\pgfpathrectangle{\pgfqpoint{7.394209in}{0.375000in}}{\pgfqpoint{6.356833in}{5.175000in}}%
\pgfusepath{clip}%
\pgfsetbuttcap%
\pgfsetroundjoin%
\pgfsetlinewidth{1.003750pt}%
\definecolor{currentstroke}{rgb}{0.827451,0.827451,0.827451}%
\pgfsetstrokecolor{currentstroke}%
\pgfsetdash{}{0pt}%
\pgfpathmoveto{\pgfqpoint{8.645217in}{2.600127in}}%
\pgfpathcurveto{\pgfqpoint{8.656267in}{2.600127in}}{\pgfqpoint{8.666867in}{2.604517in}}{\pgfqpoint{8.674680in}{2.612331in}}%
\pgfpathcurveto{\pgfqpoint{8.682494in}{2.620144in}}{\pgfqpoint{8.686884in}{2.630743in}}{\pgfqpoint{8.686884in}{2.641793in}}%
\pgfpathcurveto{\pgfqpoint{8.686884in}{2.652844in}}{\pgfqpoint{8.682494in}{2.663443in}}{\pgfqpoint{8.674680in}{2.671256in}}%
\pgfpathcurveto{\pgfqpoint{8.666867in}{2.679070in}}{\pgfqpoint{8.656267in}{2.683460in}}{\pgfqpoint{8.645217in}{2.683460in}}%
\pgfpathcurveto{\pgfqpoint{8.634167in}{2.683460in}}{\pgfqpoint{8.623568in}{2.679070in}}{\pgfqpoint{8.615755in}{2.671256in}}%
\pgfpathcurveto{\pgfqpoint{8.607941in}{2.663443in}}{\pgfqpoint{8.603551in}{2.652844in}}{\pgfqpoint{8.603551in}{2.641793in}}%
\pgfpathcurveto{\pgfqpoint{8.603551in}{2.630743in}}{\pgfqpoint{8.607941in}{2.620144in}}{\pgfqpoint{8.615755in}{2.612331in}}%
\pgfpathcurveto{\pgfqpoint{8.623568in}{2.604517in}}{\pgfqpoint{8.634167in}{2.600127in}}{\pgfqpoint{8.645217in}{2.600127in}}%
\pgfpathlineto{\pgfqpoint{8.645217in}{2.600127in}}%
\pgfpathclose%
\pgfusepath{stroke}%
\end{pgfscope}%
\begin{pgfscope}%
\pgfpathrectangle{\pgfqpoint{7.394209in}{0.375000in}}{\pgfqpoint{6.356833in}{5.175000in}}%
\pgfusepath{clip}%
\pgfsetbuttcap%
\pgfsetroundjoin%
\pgfsetlinewidth{1.003750pt}%
\definecolor{currentstroke}{rgb}{0.827451,0.827451,0.827451}%
\pgfsetstrokecolor{currentstroke}%
\pgfsetdash{}{0pt}%
\pgfpathmoveto{\pgfqpoint{7.744383in}{0.437957in}}%
\pgfpathcurveto{\pgfqpoint{7.755433in}{0.437957in}}{\pgfqpoint{7.766032in}{0.442347in}}{\pgfqpoint{7.773846in}{0.450161in}}%
\pgfpathcurveto{\pgfqpoint{7.781659in}{0.457974in}}{\pgfqpoint{7.786049in}{0.468573in}}{\pgfqpoint{7.786049in}{0.479624in}}%
\pgfpathcurveto{\pgfqpoint{7.786049in}{0.490674in}}{\pgfqpoint{7.781659in}{0.501273in}}{\pgfqpoint{7.773846in}{0.509086in}}%
\pgfpathcurveto{\pgfqpoint{7.766032in}{0.516900in}}{\pgfqpoint{7.755433in}{0.521290in}}{\pgfqpoint{7.744383in}{0.521290in}}%
\pgfpathcurveto{\pgfqpoint{7.733333in}{0.521290in}}{\pgfqpoint{7.722734in}{0.516900in}}{\pgfqpoint{7.714920in}{0.509086in}}%
\pgfpathcurveto{\pgfqpoint{7.707106in}{0.501273in}}{\pgfqpoint{7.702716in}{0.490674in}}{\pgfqpoint{7.702716in}{0.479624in}}%
\pgfpathcurveto{\pgfqpoint{7.702716in}{0.468573in}}{\pgfqpoint{7.707106in}{0.457974in}}{\pgfqpoint{7.714920in}{0.450161in}}%
\pgfpathcurveto{\pgfqpoint{7.722734in}{0.442347in}}{\pgfqpoint{7.733333in}{0.437957in}}{\pgfqpoint{7.744383in}{0.437957in}}%
\pgfpathlineto{\pgfqpoint{7.744383in}{0.437957in}}%
\pgfpathclose%
\pgfusepath{stroke}%
\end{pgfscope}%
\begin{pgfscope}%
\pgfpathrectangle{\pgfqpoint{7.394209in}{0.375000in}}{\pgfqpoint{6.356833in}{5.175000in}}%
\pgfusepath{clip}%
\pgfsetbuttcap%
\pgfsetroundjoin%
\pgfsetlinewidth{1.003750pt}%
\definecolor{currentstroke}{rgb}{0.827451,0.827451,0.827451}%
\pgfsetstrokecolor{currentstroke}%
\pgfsetdash{}{0pt}%
\pgfpathmoveto{\pgfqpoint{8.219919in}{1.903503in}}%
\pgfpathcurveto{\pgfqpoint{8.230969in}{1.903503in}}{\pgfqpoint{8.241568in}{1.907894in}}{\pgfqpoint{8.249381in}{1.915707in}}%
\pgfpathcurveto{\pgfqpoint{8.257195in}{1.923521in}}{\pgfqpoint{8.261585in}{1.934120in}}{\pgfqpoint{8.261585in}{1.945170in}}%
\pgfpathcurveto{\pgfqpoint{8.261585in}{1.956220in}}{\pgfqpoint{8.257195in}{1.966819in}}{\pgfqpoint{8.249381in}{1.974633in}}%
\pgfpathcurveto{\pgfqpoint{8.241568in}{1.982447in}}{\pgfqpoint{8.230969in}{1.986837in}}{\pgfqpoint{8.219919in}{1.986837in}}%
\pgfpathcurveto{\pgfqpoint{8.208869in}{1.986837in}}{\pgfqpoint{8.198269in}{1.982447in}}{\pgfqpoint{8.190456in}{1.974633in}}%
\pgfpathcurveto{\pgfqpoint{8.182642in}{1.966819in}}{\pgfqpoint{8.178252in}{1.956220in}}{\pgfqpoint{8.178252in}{1.945170in}}%
\pgfpathcurveto{\pgfqpoint{8.178252in}{1.934120in}}{\pgfqpoint{8.182642in}{1.923521in}}{\pgfqpoint{8.190456in}{1.915707in}}%
\pgfpathcurveto{\pgfqpoint{8.198269in}{1.907894in}}{\pgfqpoint{8.208869in}{1.903503in}}{\pgfqpoint{8.219919in}{1.903503in}}%
\pgfpathlineto{\pgfqpoint{8.219919in}{1.903503in}}%
\pgfpathclose%
\pgfusepath{stroke}%
\end{pgfscope}%
\begin{pgfscope}%
\pgfpathrectangle{\pgfqpoint{7.394209in}{0.375000in}}{\pgfqpoint{6.356833in}{5.175000in}}%
\pgfusepath{clip}%
\pgfsetbuttcap%
\pgfsetroundjoin%
\pgfsetlinewidth{1.003750pt}%
\definecolor{currentstroke}{rgb}{0.827451,0.827451,0.827451}%
\pgfsetstrokecolor{currentstroke}%
\pgfsetdash{}{0pt}%
\pgfpathmoveto{\pgfqpoint{10.462835in}{3.445372in}}%
\pgfpathcurveto{\pgfqpoint{10.473885in}{3.445372in}}{\pgfqpoint{10.484484in}{3.449762in}}{\pgfqpoint{10.492298in}{3.457576in}}%
\pgfpathcurveto{\pgfqpoint{10.500111in}{3.465389in}}{\pgfqpoint{10.504502in}{3.475988in}}{\pgfqpoint{10.504502in}{3.487038in}}%
\pgfpathcurveto{\pgfqpoint{10.504502in}{3.498089in}}{\pgfqpoint{10.500111in}{3.508688in}}{\pgfqpoint{10.492298in}{3.516501in}}%
\pgfpathcurveto{\pgfqpoint{10.484484in}{3.524315in}}{\pgfqpoint{10.473885in}{3.528705in}}{\pgfqpoint{10.462835in}{3.528705in}}%
\pgfpathcurveto{\pgfqpoint{10.451785in}{3.528705in}}{\pgfqpoint{10.441186in}{3.524315in}}{\pgfqpoint{10.433372in}{3.516501in}}%
\pgfpathcurveto{\pgfqpoint{10.425558in}{3.508688in}}{\pgfqpoint{10.421168in}{3.498089in}}{\pgfqpoint{10.421168in}{3.487038in}}%
\pgfpathcurveto{\pgfqpoint{10.421168in}{3.475988in}}{\pgfqpoint{10.425558in}{3.465389in}}{\pgfqpoint{10.433372in}{3.457576in}}%
\pgfpathcurveto{\pgfqpoint{10.441186in}{3.449762in}}{\pgfqpoint{10.451785in}{3.445372in}}{\pgfqpoint{10.462835in}{3.445372in}}%
\pgfpathlineto{\pgfqpoint{10.462835in}{3.445372in}}%
\pgfpathclose%
\pgfusepath{stroke}%
\end{pgfscope}%
\begin{pgfscope}%
\pgfpathrectangle{\pgfqpoint{7.394209in}{0.375000in}}{\pgfqpoint{6.356833in}{5.175000in}}%
\pgfusepath{clip}%
\pgfsetbuttcap%
\pgfsetroundjoin%
\pgfsetlinewidth{1.003750pt}%
\definecolor{currentstroke}{rgb}{0.827451,0.827451,0.827451}%
\pgfsetstrokecolor{currentstroke}%
\pgfsetdash{}{0pt}%
\pgfpathmoveto{\pgfqpoint{13.468544in}{5.485001in}}%
\pgfpathcurveto{\pgfqpoint{13.479594in}{5.485001in}}{\pgfqpoint{13.490193in}{5.489391in}}{\pgfqpoint{13.498007in}{5.497205in}}%
\pgfpathcurveto{\pgfqpoint{13.505821in}{5.505018in}}{\pgfqpoint{13.510211in}{5.515618in}}{\pgfqpoint{13.510211in}{5.526668in}}%
\pgfpathcurveto{\pgfqpoint{13.510211in}{5.537718in}}{\pgfqpoint{13.505821in}{5.548317in}}{\pgfqpoint{13.498007in}{5.556130in}}%
\pgfpathcurveto{\pgfqpoint{13.490193in}{5.563944in}}{\pgfqpoint{13.479594in}{5.568334in}}{\pgfqpoint{13.468544in}{5.568334in}}%
\pgfpathcurveto{\pgfqpoint{13.457494in}{5.568334in}}{\pgfqpoint{13.446895in}{5.563944in}}{\pgfqpoint{13.439081in}{5.556130in}}%
\pgfpathcurveto{\pgfqpoint{13.431268in}{5.548317in}}{\pgfqpoint{13.426878in}{5.537718in}}{\pgfqpoint{13.426878in}{5.526668in}}%
\pgfpathcurveto{\pgfqpoint{13.426878in}{5.515618in}}{\pgfqpoint{13.431268in}{5.505018in}}{\pgfqpoint{13.439081in}{5.497205in}}%
\pgfpathcurveto{\pgfqpoint{13.446895in}{5.489391in}}{\pgfqpoint{13.457494in}{5.485001in}}{\pgfqpoint{13.468544in}{5.485001in}}%
\pgfpathlineto{\pgfqpoint{13.468544in}{5.485001in}}%
\pgfpathclose%
\pgfusepath{stroke}%
\end{pgfscope}%
\begin{pgfscope}%
\pgfpathrectangle{\pgfqpoint{7.394209in}{0.375000in}}{\pgfqpoint{6.356833in}{5.175000in}}%
\pgfusepath{clip}%
\pgfsetbuttcap%
\pgfsetroundjoin%
\pgfsetlinewidth{1.003750pt}%
\definecolor{currentstroke}{rgb}{0.827451,0.827451,0.827451}%
\pgfsetstrokecolor{currentstroke}%
\pgfsetdash{}{0pt}%
\pgfpathmoveto{\pgfqpoint{13.202710in}{5.498700in}}%
\pgfpathcurveto{\pgfqpoint{13.213760in}{5.498700in}}{\pgfqpoint{13.224359in}{5.503090in}}{\pgfqpoint{13.232172in}{5.510904in}}%
\pgfpathcurveto{\pgfqpoint{13.239986in}{5.518718in}}{\pgfqpoint{13.244376in}{5.529317in}}{\pgfqpoint{13.244376in}{5.540367in}}%
\pgfpathcurveto{\pgfqpoint{13.244376in}{5.551417in}}{\pgfqpoint{13.239986in}{5.562016in}}{\pgfqpoint{13.232172in}{5.569830in}}%
\pgfpathcurveto{\pgfqpoint{13.224359in}{5.577643in}}{\pgfqpoint{13.213760in}{5.582033in}}{\pgfqpoint{13.202710in}{5.582033in}}%
\pgfpathcurveto{\pgfqpoint{13.191659in}{5.582033in}}{\pgfqpoint{13.181060in}{5.577643in}}{\pgfqpoint{13.173247in}{5.569830in}}%
\pgfpathcurveto{\pgfqpoint{13.165433in}{5.562016in}}{\pgfqpoint{13.161043in}{5.551417in}}{\pgfqpoint{13.161043in}{5.540367in}}%
\pgfpathcurveto{\pgfqpoint{13.161043in}{5.529317in}}{\pgfqpoint{13.165433in}{5.518718in}}{\pgfqpoint{13.173247in}{5.510904in}}%
\pgfpathcurveto{\pgfqpoint{13.181060in}{5.503090in}}{\pgfqpoint{13.191659in}{5.498700in}}{\pgfqpoint{13.202710in}{5.498700in}}%
\pgfpathlineto{\pgfqpoint{13.202710in}{5.498700in}}%
\pgfpathclose%
\pgfusepath{stroke}%
\end{pgfscope}%
\begin{pgfscope}%
\pgfpathrectangle{\pgfqpoint{7.394209in}{0.375000in}}{\pgfqpoint{6.356833in}{5.175000in}}%
\pgfusepath{clip}%
\pgfsetbuttcap%
\pgfsetroundjoin%
\pgfsetlinewidth{1.003750pt}%
\definecolor{currentstroke}{rgb}{0.827451,0.827451,0.827451}%
\pgfsetstrokecolor{currentstroke}%
\pgfsetdash{}{0pt}%
\pgfpathmoveto{\pgfqpoint{9.995888in}{4.555244in}}%
\pgfpathcurveto{\pgfqpoint{10.006938in}{4.555244in}}{\pgfqpoint{10.017537in}{4.559634in}}{\pgfqpoint{10.025351in}{4.567448in}}%
\pgfpathcurveto{\pgfqpoint{10.033164in}{4.575261in}}{\pgfqpoint{10.037554in}{4.585860in}}{\pgfqpoint{10.037554in}{4.596911in}}%
\pgfpathcurveto{\pgfqpoint{10.037554in}{4.607961in}}{\pgfqpoint{10.033164in}{4.618560in}}{\pgfqpoint{10.025351in}{4.626373in}}%
\pgfpathcurveto{\pgfqpoint{10.017537in}{4.634187in}}{\pgfqpoint{10.006938in}{4.638577in}}{\pgfqpoint{9.995888in}{4.638577in}}%
\pgfpathcurveto{\pgfqpoint{9.984838in}{4.638577in}}{\pgfqpoint{9.974239in}{4.634187in}}{\pgfqpoint{9.966425in}{4.626373in}}%
\pgfpathcurveto{\pgfqpoint{9.958611in}{4.618560in}}{\pgfqpoint{9.954221in}{4.607961in}}{\pgfqpoint{9.954221in}{4.596911in}}%
\pgfpathcurveto{\pgfqpoint{9.954221in}{4.585860in}}{\pgfqpoint{9.958611in}{4.575261in}}{\pgfqpoint{9.966425in}{4.567448in}}%
\pgfpathcurveto{\pgfqpoint{9.974239in}{4.559634in}}{\pgfqpoint{9.984838in}{4.555244in}}{\pgfqpoint{9.995888in}{4.555244in}}%
\pgfpathlineto{\pgfqpoint{9.995888in}{4.555244in}}%
\pgfpathclose%
\pgfusepath{stroke}%
\end{pgfscope}%
\begin{pgfscope}%
\pgfpathrectangle{\pgfqpoint{7.394209in}{0.375000in}}{\pgfqpoint{6.356833in}{5.175000in}}%
\pgfusepath{clip}%
\pgfsetbuttcap%
\pgfsetroundjoin%
\pgfsetlinewidth{1.003750pt}%
\definecolor{currentstroke}{rgb}{0.827451,0.827451,0.827451}%
\pgfsetstrokecolor{currentstroke}%
\pgfsetdash{}{0pt}%
\pgfpathmoveto{\pgfqpoint{12.276880in}{5.400099in}}%
\pgfpathcurveto{\pgfqpoint{12.287930in}{5.400099in}}{\pgfqpoint{12.298529in}{5.404489in}}{\pgfqpoint{12.306343in}{5.412303in}}%
\pgfpathcurveto{\pgfqpoint{12.314156in}{5.420117in}}{\pgfqpoint{12.318546in}{5.430716in}}{\pgfqpoint{12.318546in}{5.441766in}}%
\pgfpathcurveto{\pgfqpoint{12.318546in}{5.452816in}}{\pgfqpoint{12.314156in}{5.463415in}}{\pgfqpoint{12.306343in}{5.471229in}}%
\pgfpathcurveto{\pgfqpoint{12.298529in}{5.479042in}}{\pgfqpoint{12.287930in}{5.483433in}}{\pgfqpoint{12.276880in}{5.483433in}}%
\pgfpathcurveto{\pgfqpoint{12.265830in}{5.483433in}}{\pgfqpoint{12.255231in}{5.479042in}}{\pgfqpoint{12.247417in}{5.471229in}}%
\pgfpathcurveto{\pgfqpoint{12.239603in}{5.463415in}}{\pgfqpoint{12.235213in}{5.452816in}}{\pgfqpoint{12.235213in}{5.441766in}}%
\pgfpathcurveto{\pgfqpoint{12.235213in}{5.430716in}}{\pgfqpoint{12.239603in}{5.420117in}}{\pgfqpoint{12.247417in}{5.412303in}}%
\pgfpathcurveto{\pgfqpoint{12.255231in}{5.404489in}}{\pgfqpoint{12.265830in}{5.400099in}}{\pgfqpoint{12.276880in}{5.400099in}}%
\pgfpathlineto{\pgfqpoint{12.276880in}{5.400099in}}%
\pgfpathclose%
\pgfusepath{stroke}%
\end{pgfscope}%
\begin{pgfscope}%
\pgfpathrectangle{\pgfqpoint{7.394209in}{0.375000in}}{\pgfqpoint{6.356833in}{5.175000in}}%
\pgfusepath{clip}%
\pgfsetbuttcap%
\pgfsetroundjoin%
\pgfsetlinewidth{1.003750pt}%
\definecolor{currentstroke}{rgb}{0.827451,0.827451,0.827451}%
\pgfsetstrokecolor{currentstroke}%
\pgfsetdash{}{0pt}%
\pgfpathmoveto{\pgfqpoint{9.492816in}{2.202827in}}%
\pgfpathcurveto{\pgfqpoint{9.503866in}{2.202827in}}{\pgfqpoint{9.514465in}{2.207217in}}{\pgfqpoint{9.522278in}{2.215030in}}%
\pgfpathcurveto{\pgfqpoint{9.530092in}{2.222844in}}{\pgfqpoint{9.534482in}{2.233443in}}{\pgfqpoint{9.534482in}{2.244493in}}%
\pgfpathcurveto{\pgfqpoint{9.534482in}{2.255543in}}{\pgfqpoint{9.530092in}{2.266142in}}{\pgfqpoint{9.522278in}{2.273956in}}%
\pgfpathcurveto{\pgfqpoint{9.514465in}{2.281770in}}{\pgfqpoint{9.503866in}{2.286160in}}{\pgfqpoint{9.492816in}{2.286160in}}%
\pgfpathcurveto{\pgfqpoint{9.481766in}{2.286160in}}{\pgfqpoint{9.471166in}{2.281770in}}{\pgfqpoint{9.463353in}{2.273956in}}%
\pgfpathcurveto{\pgfqpoint{9.455539in}{2.266142in}}{\pgfqpoint{9.451149in}{2.255543in}}{\pgfqpoint{9.451149in}{2.244493in}}%
\pgfpathcurveto{\pgfqpoint{9.451149in}{2.233443in}}{\pgfqpoint{9.455539in}{2.222844in}}{\pgfqpoint{9.463353in}{2.215030in}}%
\pgfpathcurveto{\pgfqpoint{9.471166in}{2.207217in}}{\pgfqpoint{9.481766in}{2.202827in}}{\pgfqpoint{9.492816in}{2.202827in}}%
\pgfpathlineto{\pgfqpoint{9.492816in}{2.202827in}}%
\pgfpathclose%
\pgfusepath{stroke}%
\end{pgfscope}%
\begin{pgfscope}%
\pgfpathrectangle{\pgfqpoint{7.394209in}{0.375000in}}{\pgfqpoint{6.356833in}{5.175000in}}%
\pgfusepath{clip}%
\pgfsetbuttcap%
\pgfsetroundjoin%
\pgfsetlinewidth{1.003750pt}%
\definecolor{currentstroke}{rgb}{0.827451,0.827451,0.827451}%
\pgfsetstrokecolor{currentstroke}%
\pgfsetdash{}{0pt}%
\pgfpathmoveto{\pgfqpoint{8.598900in}{2.164183in}}%
\pgfpathcurveto{\pgfqpoint{8.609951in}{2.164183in}}{\pgfqpoint{8.620550in}{2.168573in}}{\pgfqpoint{8.628363in}{2.176386in}}%
\pgfpathcurveto{\pgfqpoint{8.636177in}{2.184200in}}{\pgfqpoint{8.640567in}{2.194799in}}{\pgfqpoint{8.640567in}{2.205849in}}%
\pgfpathcurveto{\pgfqpoint{8.640567in}{2.216899in}}{\pgfqpoint{8.636177in}{2.227498in}}{\pgfqpoint{8.628363in}{2.235312in}}%
\pgfpathcurveto{\pgfqpoint{8.620550in}{2.243126in}}{\pgfqpoint{8.609951in}{2.247516in}}{\pgfqpoint{8.598900in}{2.247516in}}%
\pgfpathcurveto{\pgfqpoint{8.587850in}{2.247516in}}{\pgfqpoint{8.577251in}{2.243126in}}{\pgfqpoint{8.569438in}{2.235312in}}%
\pgfpathcurveto{\pgfqpoint{8.561624in}{2.227498in}}{\pgfqpoint{8.557234in}{2.216899in}}{\pgfqpoint{8.557234in}{2.205849in}}%
\pgfpathcurveto{\pgfqpoint{8.557234in}{2.194799in}}{\pgfqpoint{8.561624in}{2.184200in}}{\pgfqpoint{8.569438in}{2.176386in}}%
\pgfpathcurveto{\pgfqpoint{8.577251in}{2.168573in}}{\pgfqpoint{8.587850in}{2.164183in}}{\pgfqpoint{8.598900in}{2.164183in}}%
\pgfpathlineto{\pgfqpoint{8.598900in}{2.164183in}}%
\pgfpathclose%
\pgfusepath{stroke}%
\end{pgfscope}%
\begin{pgfscope}%
\pgfpathrectangle{\pgfqpoint{7.394209in}{0.375000in}}{\pgfqpoint{6.356833in}{5.175000in}}%
\pgfusepath{clip}%
\pgfsetbuttcap%
\pgfsetroundjoin%
\pgfsetlinewidth{1.003750pt}%
\definecolor{currentstroke}{rgb}{0.827451,0.827451,0.827451}%
\pgfsetstrokecolor{currentstroke}%
\pgfsetdash{}{0pt}%
\pgfpathmoveto{\pgfqpoint{10.511306in}{5.420057in}}%
\pgfpathcurveto{\pgfqpoint{10.522356in}{5.420057in}}{\pgfqpoint{10.532955in}{5.424447in}}{\pgfqpoint{10.540768in}{5.432261in}}%
\pgfpathcurveto{\pgfqpoint{10.548582in}{5.440074in}}{\pgfqpoint{10.552972in}{5.450673in}}{\pgfqpoint{10.552972in}{5.461724in}}%
\pgfpathcurveto{\pgfqpoint{10.552972in}{5.472774in}}{\pgfqpoint{10.548582in}{5.483373in}}{\pgfqpoint{10.540768in}{5.491186in}}%
\pgfpathcurveto{\pgfqpoint{10.532955in}{5.499000in}}{\pgfqpoint{10.522356in}{5.503390in}}{\pgfqpoint{10.511306in}{5.503390in}}%
\pgfpathcurveto{\pgfqpoint{10.500255in}{5.503390in}}{\pgfqpoint{10.489656in}{5.499000in}}{\pgfqpoint{10.481843in}{5.491186in}}%
\pgfpathcurveto{\pgfqpoint{10.474029in}{5.483373in}}{\pgfqpoint{10.469639in}{5.472774in}}{\pgfqpoint{10.469639in}{5.461724in}}%
\pgfpathcurveto{\pgfqpoint{10.469639in}{5.450673in}}{\pgfqpoint{10.474029in}{5.440074in}}{\pgfqpoint{10.481843in}{5.432261in}}%
\pgfpathcurveto{\pgfqpoint{10.489656in}{5.424447in}}{\pgfqpoint{10.500255in}{5.420057in}}{\pgfqpoint{10.511306in}{5.420057in}}%
\pgfpathlineto{\pgfqpoint{10.511306in}{5.420057in}}%
\pgfpathclose%
\pgfusepath{stroke}%
\end{pgfscope}%
\begin{pgfscope}%
\pgfpathrectangle{\pgfqpoint{7.394209in}{0.375000in}}{\pgfqpoint{6.356833in}{5.175000in}}%
\pgfusepath{clip}%
\pgfsetbuttcap%
\pgfsetroundjoin%
\pgfsetlinewidth{1.003750pt}%
\definecolor{currentstroke}{rgb}{0.827451,0.827451,0.827451}%
\pgfsetstrokecolor{currentstroke}%
\pgfsetdash{}{0pt}%
\pgfpathmoveto{\pgfqpoint{8.724763in}{2.734967in}}%
\pgfpathcurveto{\pgfqpoint{8.735813in}{2.734967in}}{\pgfqpoint{8.746412in}{2.739357in}}{\pgfqpoint{8.754225in}{2.747171in}}%
\pgfpathcurveto{\pgfqpoint{8.762039in}{2.754985in}}{\pgfqpoint{8.766429in}{2.765584in}}{\pgfqpoint{8.766429in}{2.776634in}}%
\pgfpathcurveto{\pgfqpoint{8.766429in}{2.787684in}}{\pgfqpoint{8.762039in}{2.798283in}}{\pgfqpoint{8.754225in}{2.806097in}}%
\pgfpathcurveto{\pgfqpoint{8.746412in}{2.813910in}}{\pgfqpoint{8.735813in}{2.818300in}}{\pgfqpoint{8.724763in}{2.818300in}}%
\pgfpathcurveto{\pgfqpoint{8.713712in}{2.818300in}}{\pgfqpoint{8.703113in}{2.813910in}}{\pgfqpoint{8.695300in}{2.806097in}}%
\pgfpathcurveto{\pgfqpoint{8.687486in}{2.798283in}}{\pgfqpoint{8.683096in}{2.787684in}}{\pgfqpoint{8.683096in}{2.776634in}}%
\pgfpathcurveto{\pgfqpoint{8.683096in}{2.765584in}}{\pgfqpoint{8.687486in}{2.754985in}}{\pgfqpoint{8.695300in}{2.747171in}}%
\pgfpathcurveto{\pgfqpoint{8.703113in}{2.739357in}}{\pgfqpoint{8.713712in}{2.734967in}}{\pgfqpoint{8.724763in}{2.734967in}}%
\pgfpathlineto{\pgfqpoint{8.724763in}{2.734967in}}%
\pgfpathclose%
\pgfusepath{stroke}%
\end{pgfscope}%
\begin{pgfscope}%
\pgfpathrectangle{\pgfqpoint{7.394209in}{0.375000in}}{\pgfqpoint{6.356833in}{5.175000in}}%
\pgfusepath{clip}%
\pgfsetbuttcap%
\pgfsetroundjoin%
\pgfsetlinewidth{1.003750pt}%
\definecolor{currentstroke}{rgb}{0.827451,0.827451,0.827451}%
\pgfsetstrokecolor{currentstroke}%
\pgfsetdash{}{0pt}%
\pgfpathmoveto{\pgfqpoint{8.887338in}{3.481284in}}%
\pgfpathcurveto{\pgfqpoint{8.898388in}{3.481284in}}{\pgfqpoint{8.908987in}{3.485674in}}{\pgfqpoint{8.916801in}{3.493487in}}%
\pgfpathcurveto{\pgfqpoint{8.924615in}{3.501301in}}{\pgfqpoint{8.929005in}{3.511900in}}{\pgfqpoint{8.929005in}{3.522950in}}%
\pgfpathcurveto{\pgfqpoint{8.929005in}{3.534000in}}{\pgfqpoint{8.924615in}{3.544599in}}{\pgfqpoint{8.916801in}{3.552413in}}%
\pgfpathcurveto{\pgfqpoint{8.908987in}{3.560227in}}{\pgfqpoint{8.898388in}{3.564617in}}{\pgfqpoint{8.887338in}{3.564617in}}%
\pgfpathcurveto{\pgfqpoint{8.876288in}{3.564617in}}{\pgfqpoint{8.865689in}{3.560227in}}{\pgfqpoint{8.857876in}{3.552413in}}%
\pgfpathcurveto{\pgfqpoint{8.850062in}{3.544599in}}{\pgfqpoint{8.845672in}{3.534000in}}{\pgfqpoint{8.845672in}{3.522950in}}%
\pgfpathcurveto{\pgfqpoint{8.845672in}{3.511900in}}{\pgfqpoint{8.850062in}{3.501301in}}{\pgfqpoint{8.857876in}{3.493487in}}%
\pgfpathcurveto{\pgfqpoint{8.865689in}{3.485674in}}{\pgfqpoint{8.876288in}{3.481284in}}{\pgfqpoint{8.887338in}{3.481284in}}%
\pgfpathlineto{\pgfqpoint{8.887338in}{3.481284in}}%
\pgfpathclose%
\pgfusepath{stroke}%
\end{pgfscope}%
\begin{pgfscope}%
\pgfpathrectangle{\pgfqpoint{7.394209in}{0.375000in}}{\pgfqpoint{6.356833in}{5.175000in}}%
\pgfusepath{clip}%
\pgfsetbuttcap%
\pgfsetroundjoin%
\pgfsetlinewidth{1.003750pt}%
\definecolor{currentstroke}{rgb}{0.827451,0.827451,0.827451}%
\pgfsetstrokecolor{currentstroke}%
\pgfsetdash{}{0pt}%
\pgfpathmoveto{\pgfqpoint{8.699122in}{1.625851in}}%
\pgfpathcurveto{\pgfqpoint{8.710172in}{1.625851in}}{\pgfqpoint{8.720771in}{1.630241in}}{\pgfqpoint{8.728584in}{1.638055in}}%
\pgfpathcurveto{\pgfqpoint{8.736398in}{1.645868in}}{\pgfqpoint{8.740788in}{1.656467in}}{\pgfqpoint{8.740788in}{1.667518in}}%
\pgfpathcurveto{\pgfqpoint{8.740788in}{1.678568in}}{\pgfqpoint{8.736398in}{1.689167in}}{\pgfqpoint{8.728584in}{1.696980in}}%
\pgfpathcurveto{\pgfqpoint{8.720771in}{1.704794in}}{\pgfqpoint{8.710172in}{1.709184in}}{\pgfqpoint{8.699122in}{1.709184in}}%
\pgfpathcurveto{\pgfqpoint{8.688072in}{1.709184in}}{\pgfqpoint{8.677473in}{1.704794in}}{\pgfqpoint{8.669659in}{1.696980in}}%
\pgfpathcurveto{\pgfqpoint{8.661845in}{1.689167in}}{\pgfqpoint{8.657455in}{1.678568in}}{\pgfqpoint{8.657455in}{1.667518in}}%
\pgfpathcurveto{\pgfqpoint{8.657455in}{1.656467in}}{\pgfqpoint{8.661845in}{1.645868in}}{\pgfqpoint{8.669659in}{1.638055in}}%
\pgfpathcurveto{\pgfqpoint{8.677473in}{1.630241in}}{\pgfqpoint{8.688072in}{1.625851in}}{\pgfqpoint{8.699122in}{1.625851in}}%
\pgfpathlineto{\pgfqpoint{8.699122in}{1.625851in}}%
\pgfpathclose%
\pgfusepath{stroke}%
\end{pgfscope}%
\begin{pgfscope}%
\pgfpathrectangle{\pgfqpoint{7.394209in}{0.375000in}}{\pgfqpoint{6.356833in}{5.175000in}}%
\pgfusepath{clip}%
\pgfsetbuttcap%
\pgfsetroundjoin%
\pgfsetlinewidth{1.003750pt}%
\definecolor{currentstroke}{rgb}{0.827451,0.827451,0.827451}%
\pgfsetstrokecolor{currentstroke}%
\pgfsetdash{}{0pt}%
\pgfpathmoveto{\pgfqpoint{11.784810in}{5.047073in}}%
\pgfpathcurveto{\pgfqpoint{11.795860in}{5.047073in}}{\pgfqpoint{11.806459in}{5.051463in}}{\pgfqpoint{11.814273in}{5.059277in}}%
\pgfpathcurveto{\pgfqpoint{11.822086in}{5.067090in}}{\pgfqpoint{11.826476in}{5.077690in}}{\pgfqpoint{11.826476in}{5.088740in}}%
\pgfpathcurveto{\pgfqpoint{11.826476in}{5.099790in}}{\pgfqpoint{11.822086in}{5.110389in}}{\pgfqpoint{11.814273in}{5.118202in}}%
\pgfpathcurveto{\pgfqpoint{11.806459in}{5.126016in}}{\pgfqpoint{11.795860in}{5.130406in}}{\pgfqpoint{11.784810in}{5.130406in}}%
\pgfpathcurveto{\pgfqpoint{11.773760in}{5.130406in}}{\pgfqpoint{11.763161in}{5.126016in}}{\pgfqpoint{11.755347in}{5.118202in}}%
\pgfpathcurveto{\pgfqpoint{11.747533in}{5.110389in}}{\pgfqpoint{11.743143in}{5.099790in}}{\pgfqpoint{11.743143in}{5.088740in}}%
\pgfpathcurveto{\pgfqpoint{11.743143in}{5.077690in}}{\pgfqpoint{11.747533in}{5.067090in}}{\pgfqpoint{11.755347in}{5.059277in}}%
\pgfpathcurveto{\pgfqpoint{11.763161in}{5.051463in}}{\pgfqpoint{11.773760in}{5.047073in}}{\pgfqpoint{11.784810in}{5.047073in}}%
\pgfpathlineto{\pgfqpoint{11.784810in}{5.047073in}}%
\pgfpathclose%
\pgfusepath{stroke}%
\end{pgfscope}%
\begin{pgfscope}%
\pgfpathrectangle{\pgfqpoint{7.394209in}{0.375000in}}{\pgfqpoint{6.356833in}{5.175000in}}%
\pgfusepath{clip}%
\pgfsetbuttcap%
\pgfsetroundjoin%
\pgfsetlinewidth{1.003750pt}%
\definecolor{currentstroke}{rgb}{0.827451,0.827451,0.827451}%
\pgfsetstrokecolor{currentstroke}%
\pgfsetdash{}{0pt}%
\pgfpathmoveto{\pgfqpoint{11.801966in}{5.118411in}}%
\pgfpathcurveto{\pgfqpoint{11.813016in}{5.118411in}}{\pgfqpoint{11.823615in}{5.122802in}}{\pgfqpoint{11.831428in}{5.130615in}}%
\pgfpathcurveto{\pgfqpoint{11.839242in}{5.138429in}}{\pgfqpoint{11.843632in}{5.149028in}}{\pgfqpoint{11.843632in}{5.160078in}}%
\pgfpathcurveto{\pgfqpoint{11.843632in}{5.171128in}}{\pgfqpoint{11.839242in}{5.181727in}}{\pgfqpoint{11.831428in}{5.189541in}}%
\pgfpathcurveto{\pgfqpoint{11.823615in}{5.197354in}}{\pgfqpoint{11.813016in}{5.201745in}}{\pgfqpoint{11.801966in}{5.201745in}}%
\pgfpathcurveto{\pgfqpoint{11.790915in}{5.201745in}}{\pgfqpoint{11.780316in}{5.197354in}}{\pgfqpoint{11.772503in}{5.189541in}}%
\pgfpathcurveto{\pgfqpoint{11.764689in}{5.181727in}}{\pgfqpoint{11.760299in}{5.171128in}}{\pgfqpoint{11.760299in}{5.160078in}}%
\pgfpathcurveto{\pgfqpoint{11.760299in}{5.149028in}}{\pgfqpoint{11.764689in}{5.138429in}}{\pgfqpoint{11.772503in}{5.130615in}}%
\pgfpathcurveto{\pgfqpoint{11.780316in}{5.122802in}}{\pgfqpoint{11.790915in}{5.118411in}}{\pgfqpoint{11.801966in}{5.118411in}}%
\pgfpathlineto{\pgfqpoint{11.801966in}{5.118411in}}%
\pgfpathclose%
\pgfusepath{stroke}%
\end{pgfscope}%
\begin{pgfscope}%
\pgfpathrectangle{\pgfqpoint{7.394209in}{0.375000in}}{\pgfqpoint{6.356833in}{5.175000in}}%
\pgfusepath{clip}%
\pgfsetbuttcap%
\pgfsetroundjoin%
\pgfsetlinewidth{1.003750pt}%
\definecolor{currentstroke}{rgb}{0.827451,0.827451,0.827451}%
\pgfsetstrokecolor{currentstroke}%
\pgfsetdash{}{0pt}%
\pgfpathmoveto{\pgfqpoint{12.232275in}{5.465289in}}%
\pgfpathcurveto{\pgfqpoint{12.243325in}{5.465289in}}{\pgfqpoint{12.253924in}{5.469680in}}{\pgfqpoint{12.261738in}{5.477493in}}%
\pgfpathcurveto{\pgfqpoint{12.269552in}{5.485307in}}{\pgfqpoint{12.273942in}{5.495906in}}{\pgfqpoint{12.273942in}{5.506956in}}%
\pgfpathcurveto{\pgfqpoint{12.273942in}{5.518006in}}{\pgfqpoint{12.269552in}{5.528605in}}{\pgfqpoint{12.261738in}{5.536419in}}%
\pgfpathcurveto{\pgfqpoint{12.253924in}{5.544232in}}{\pgfqpoint{12.243325in}{5.548623in}}{\pgfqpoint{12.232275in}{5.548623in}}%
\pgfpathcurveto{\pgfqpoint{12.221225in}{5.548623in}}{\pgfqpoint{12.210626in}{5.544232in}}{\pgfqpoint{12.202812in}{5.536419in}}%
\pgfpathcurveto{\pgfqpoint{12.194999in}{5.528605in}}{\pgfqpoint{12.190609in}{5.518006in}}{\pgfqpoint{12.190609in}{5.506956in}}%
\pgfpathcurveto{\pgfqpoint{12.190609in}{5.495906in}}{\pgfqpoint{12.194999in}{5.485307in}}{\pgfqpoint{12.202812in}{5.477493in}}%
\pgfpathcurveto{\pgfqpoint{12.210626in}{5.469680in}}{\pgfqpoint{12.221225in}{5.465289in}}{\pgfqpoint{12.232275in}{5.465289in}}%
\pgfpathlineto{\pgfqpoint{12.232275in}{5.465289in}}%
\pgfpathclose%
\pgfusepath{stroke}%
\end{pgfscope}%
\begin{pgfscope}%
\pgfpathrectangle{\pgfqpoint{7.394209in}{0.375000in}}{\pgfqpoint{6.356833in}{5.175000in}}%
\pgfusepath{clip}%
\pgfsetbuttcap%
\pgfsetroundjoin%
\pgfsetlinewidth{1.003750pt}%
\definecolor{currentstroke}{rgb}{0.827451,0.827451,0.827451}%
\pgfsetstrokecolor{currentstroke}%
\pgfsetdash{}{0pt}%
\pgfpathmoveto{\pgfqpoint{11.646362in}{5.181089in}}%
\pgfpathcurveto{\pgfqpoint{11.657412in}{5.181089in}}{\pgfqpoint{11.668011in}{5.185479in}}{\pgfqpoint{11.675824in}{5.193293in}}%
\pgfpathcurveto{\pgfqpoint{11.683638in}{5.201106in}}{\pgfqpoint{11.688028in}{5.211705in}}{\pgfqpoint{11.688028in}{5.222755in}}%
\pgfpathcurveto{\pgfqpoint{11.688028in}{5.233806in}}{\pgfqpoint{11.683638in}{5.244405in}}{\pgfqpoint{11.675824in}{5.252218in}}%
\pgfpathcurveto{\pgfqpoint{11.668011in}{5.260032in}}{\pgfqpoint{11.657412in}{5.264422in}}{\pgfqpoint{11.646362in}{5.264422in}}%
\pgfpathcurveto{\pgfqpoint{11.635311in}{5.264422in}}{\pgfqpoint{11.624712in}{5.260032in}}{\pgfqpoint{11.616899in}{5.252218in}}%
\pgfpathcurveto{\pgfqpoint{11.609085in}{5.244405in}}{\pgfqpoint{11.604695in}{5.233806in}}{\pgfqpoint{11.604695in}{5.222755in}}%
\pgfpathcurveto{\pgfqpoint{11.604695in}{5.211705in}}{\pgfqpoint{11.609085in}{5.201106in}}{\pgfqpoint{11.616899in}{5.193293in}}%
\pgfpathcurveto{\pgfqpoint{11.624712in}{5.185479in}}{\pgfqpoint{11.635311in}{5.181089in}}{\pgfqpoint{11.646362in}{5.181089in}}%
\pgfpathlineto{\pgfqpoint{11.646362in}{5.181089in}}%
\pgfpathclose%
\pgfusepath{stroke}%
\end{pgfscope}%
\begin{pgfscope}%
\pgfpathrectangle{\pgfqpoint{7.394209in}{0.375000in}}{\pgfqpoint{6.356833in}{5.175000in}}%
\pgfusepath{clip}%
\pgfsetbuttcap%
\pgfsetroundjoin%
\pgfsetlinewidth{1.003750pt}%
\definecolor{currentstroke}{rgb}{0.827451,0.827451,0.827451}%
\pgfsetstrokecolor{currentstroke}%
\pgfsetdash{}{0pt}%
\pgfpathmoveto{\pgfqpoint{9.787238in}{4.216543in}}%
\pgfpathcurveto{\pgfqpoint{9.798288in}{4.216543in}}{\pgfqpoint{9.808887in}{4.220933in}}{\pgfqpoint{9.816701in}{4.228746in}}%
\pgfpathcurveto{\pgfqpoint{9.824514in}{4.236560in}}{\pgfqpoint{9.828905in}{4.247159in}}{\pgfqpoint{9.828905in}{4.258209in}}%
\pgfpathcurveto{\pgfqpoint{9.828905in}{4.269259in}}{\pgfqpoint{9.824514in}{4.279858in}}{\pgfqpoint{9.816701in}{4.287672in}}%
\pgfpathcurveto{\pgfqpoint{9.808887in}{4.295486in}}{\pgfqpoint{9.798288in}{4.299876in}}{\pgfqpoint{9.787238in}{4.299876in}}%
\pgfpathcurveto{\pgfqpoint{9.776188in}{4.299876in}}{\pgfqpoint{9.765589in}{4.295486in}}{\pgfqpoint{9.757775in}{4.287672in}}%
\pgfpathcurveto{\pgfqpoint{9.749962in}{4.279858in}}{\pgfqpoint{9.745571in}{4.269259in}}{\pgfqpoint{9.745571in}{4.258209in}}%
\pgfpathcurveto{\pgfqpoint{9.745571in}{4.247159in}}{\pgfqpoint{9.749962in}{4.236560in}}{\pgfqpoint{9.757775in}{4.228746in}}%
\pgfpathcurveto{\pgfqpoint{9.765589in}{4.220933in}}{\pgfqpoint{9.776188in}{4.216543in}}{\pgfqpoint{9.787238in}{4.216543in}}%
\pgfpathlineto{\pgfqpoint{9.787238in}{4.216543in}}%
\pgfpathclose%
\pgfusepath{stroke}%
\end{pgfscope}%
\begin{pgfscope}%
\pgfpathrectangle{\pgfqpoint{7.394209in}{0.375000in}}{\pgfqpoint{6.356833in}{5.175000in}}%
\pgfusepath{clip}%
\pgfsetbuttcap%
\pgfsetroundjoin%
\pgfsetlinewidth{1.003750pt}%
\definecolor{currentstroke}{rgb}{0.827451,0.827451,0.827451}%
\pgfsetstrokecolor{currentstroke}%
\pgfsetdash{}{0pt}%
\pgfpathmoveto{\pgfqpoint{10.056381in}{2.954280in}}%
\pgfpathcurveto{\pgfqpoint{10.067431in}{2.954280in}}{\pgfqpoint{10.078030in}{2.958670in}}{\pgfqpoint{10.085844in}{2.966484in}}%
\pgfpathcurveto{\pgfqpoint{10.093657in}{2.974298in}}{\pgfqpoint{10.098048in}{2.984897in}}{\pgfqpoint{10.098048in}{2.995947in}}%
\pgfpathcurveto{\pgfqpoint{10.098048in}{3.006997in}}{\pgfqpoint{10.093657in}{3.017596in}}{\pgfqpoint{10.085844in}{3.025410in}}%
\pgfpathcurveto{\pgfqpoint{10.078030in}{3.033223in}}{\pgfqpoint{10.067431in}{3.037613in}}{\pgfqpoint{10.056381in}{3.037613in}}%
\pgfpathcurveto{\pgfqpoint{10.045331in}{3.037613in}}{\pgfqpoint{10.034732in}{3.033223in}}{\pgfqpoint{10.026918in}{3.025410in}}%
\pgfpathcurveto{\pgfqpoint{10.019104in}{3.017596in}}{\pgfqpoint{10.014714in}{3.006997in}}{\pgfqpoint{10.014714in}{2.995947in}}%
\pgfpathcurveto{\pgfqpoint{10.014714in}{2.984897in}}{\pgfqpoint{10.019104in}{2.974298in}}{\pgfqpoint{10.026918in}{2.966484in}}%
\pgfpathcurveto{\pgfqpoint{10.034732in}{2.958670in}}{\pgfqpoint{10.045331in}{2.954280in}}{\pgfqpoint{10.056381in}{2.954280in}}%
\pgfpathlineto{\pgfqpoint{10.056381in}{2.954280in}}%
\pgfpathclose%
\pgfusepath{stroke}%
\end{pgfscope}%
\begin{pgfscope}%
\pgfpathrectangle{\pgfqpoint{7.394209in}{0.375000in}}{\pgfqpoint{6.356833in}{5.175000in}}%
\pgfusepath{clip}%
\pgfsetbuttcap%
\pgfsetroundjoin%
\pgfsetlinewidth{1.003750pt}%
\definecolor{currentstroke}{rgb}{0.827451,0.827451,0.827451}%
\pgfsetstrokecolor{currentstroke}%
\pgfsetdash{}{0pt}%
\pgfpathmoveto{\pgfqpoint{8.015256in}{0.647487in}}%
\pgfpathcurveto{\pgfqpoint{8.026306in}{0.647487in}}{\pgfqpoint{8.036905in}{0.651878in}}{\pgfqpoint{8.044719in}{0.659691in}}%
\pgfpathcurveto{\pgfqpoint{8.052532in}{0.667505in}}{\pgfqpoint{8.056923in}{0.678104in}}{\pgfqpoint{8.056923in}{0.689154in}}%
\pgfpathcurveto{\pgfqpoint{8.056923in}{0.700204in}}{\pgfqpoint{8.052532in}{0.710803in}}{\pgfqpoint{8.044719in}{0.718617in}}%
\pgfpathcurveto{\pgfqpoint{8.036905in}{0.726430in}}{\pgfqpoint{8.026306in}{0.730821in}}{\pgfqpoint{8.015256in}{0.730821in}}%
\pgfpathcurveto{\pgfqpoint{8.004206in}{0.730821in}}{\pgfqpoint{7.993607in}{0.726430in}}{\pgfqpoint{7.985793in}{0.718617in}}%
\pgfpathcurveto{\pgfqpoint{7.977979in}{0.710803in}}{\pgfqpoint{7.973589in}{0.700204in}}{\pgfqpoint{7.973589in}{0.689154in}}%
\pgfpathcurveto{\pgfqpoint{7.973589in}{0.678104in}}{\pgfqpoint{7.977979in}{0.667505in}}{\pgfqpoint{7.985793in}{0.659691in}}%
\pgfpathcurveto{\pgfqpoint{7.993607in}{0.651878in}}{\pgfqpoint{8.004206in}{0.647487in}}{\pgfqpoint{8.015256in}{0.647487in}}%
\pgfpathlineto{\pgfqpoint{8.015256in}{0.647487in}}%
\pgfpathclose%
\pgfusepath{stroke}%
\end{pgfscope}%
\begin{pgfscope}%
\pgfpathrectangle{\pgfqpoint{7.394209in}{0.375000in}}{\pgfqpoint{6.356833in}{5.175000in}}%
\pgfusepath{clip}%
\pgfsetbuttcap%
\pgfsetroundjoin%
\pgfsetlinewidth{1.003750pt}%
\definecolor{currentstroke}{rgb}{0.827451,0.827451,0.827451}%
\pgfsetstrokecolor{currentstroke}%
\pgfsetdash{}{0pt}%
\pgfpathmoveto{\pgfqpoint{12.535453in}{5.405834in}}%
\pgfpathcurveto{\pgfqpoint{12.546504in}{5.405834in}}{\pgfqpoint{12.557103in}{5.410224in}}{\pgfqpoint{12.564916in}{5.418038in}}%
\pgfpathcurveto{\pgfqpoint{12.572730in}{5.425851in}}{\pgfqpoint{12.577120in}{5.436450in}}{\pgfqpoint{12.577120in}{5.447500in}}%
\pgfpathcurveto{\pgfqpoint{12.577120in}{5.458550in}}{\pgfqpoint{12.572730in}{5.469149in}}{\pgfqpoint{12.564916in}{5.476963in}}%
\pgfpathcurveto{\pgfqpoint{12.557103in}{5.484777in}}{\pgfqpoint{12.546504in}{5.489167in}}{\pgfqpoint{12.535453in}{5.489167in}}%
\pgfpathcurveto{\pgfqpoint{12.524403in}{5.489167in}}{\pgfqpoint{12.513804in}{5.484777in}}{\pgfqpoint{12.505991in}{5.476963in}}%
\pgfpathcurveto{\pgfqpoint{12.498177in}{5.469149in}}{\pgfqpoint{12.493787in}{5.458550in}}{\pgfqpoint{12.493787in}{5.447500in}}%
\pgfpathcurveto{\pgfqpoint{12.493787in}{5.436450in}}{\pgfqpoint{12.498177in}{5.425851in}}{\pgfqpoint{12.505991in}{5.418038in}}%
\pgfpathcurveto{\pgfqpoint{12.513804in}{5.410224in}}{\pgfqpoint{12.524403in}{5.405834in}}{\pgfqpoint{12.535453in}{5.405834in}}%
\pgfpathlineto{\pgfqpoint{12.535453in}{5.405834in}}%
\pgfpathclose%
\pgfusepath{stroke}%
\end{pgfscope}%
\begin{pgfscope}%
\pgfpathrectangle{\pgfqpoint{7.394209in}{0.375000in}}{\pgfqpoint{6.356833in}{5.175000in}}%
\pgfusepath{clip}%
\pgfsetbuttcap%
\pgfsetroundjoin%
\pgfsetlinewidth{1.003750pt}%
\definecolor{currentstroke}{rgb}{0.827451,0.827451,0.827451}%
\pgfsetstrokecolor{currentstroke}%
\pgfsetdash{}{0pt}%
\pgfpathmoveto{\pgfqpoint{7.819633in}{0.444034in}}%
\pgfpathcurveto{\pgfqpoint{7.830684in}{0.444034in}}{\pgfqpoint{7.841283in}{0.448424in}}{\pgfqpoint{7.849096in}{0.456238in}}%
\pgfpathcurveto{\pgfqpoint{7.856910in}{0.464051in}}{\pgfqpoint{7.861300in}{0.474650in}}{\pgfqpoint{7.861300in}{0.485700in}}%
\pgfpathcurveto{\pgfqpoint{7.861300in}{0.496750in}}{\pgfqpoint{7.856910in}{0.507350in}}{\pgfqpoint{7.849096in}{0.515163in}}%
\pgfpathcurveto{\pgfqpoint{7.841283in}{0.522977in}}{\pgfqpoint{7.830684in}{0.527367in}}{\pgfqpoint{7.819633in}{0.527367in}}%
\pgfpathcurveto{\pgfqpoint{7.808583in}{0.527367in}}{\pgfqpoint{7.797984in}{0.522977in}}{\pgfqpoint{7.790171in}{0.515163in}}%
\pgfpathcurveto{\pgfqpoint{7.782357in}{0.507350in}}{\pgfqpoint{7.777967in}{0.496750in}}{\pgfqpoint{7.777967in}{0.485700in}}%
\pgfpathcurveto{\pgfqpoint{7.777967in}{0.474650in}}{\pgfqpoint{7.782357in}{0.464051in}}{\pgfqpoint{7.790171in}{0.456238in}}%
\pgfpathcurveto{\pgfqpoint{7.797984in}{0.448424in}}{\pgfqpoint{7.808583in}{0.444034in}}{\pgfqpoint{7.819633in}{0.444034in}}%
\pgfpathlineto{\pgfqpoint{7.819633in}{0.444034in}}%
\pgfpathclose%
\pgfusepath{stroke}%
\end{pgfscope}%
\begin{pgfscope}%
\pgfpathrectangle{\pgfqpoint{7.394209in}{0.375000in}}{\pgfqpoint{6.356833in}{5.175000in}}%
\pgfusepath{clip}%
\pgfsetbuttcap%
\pgfsetroundjoin%
\pgfsetlinewidth{1.003750pt}%
\definecolor{currentstroke}{rgb}{0.827451,0.827451,0.827451}%
\pgfsetstrokecolor{currentstroke}%
\pgfsetdash{}{0pt}%
\pgfpathmoveto{\pgfqpoint{7.960705in}{0.557678in}}%
\pgfpathcurveto{\pgfqpoint{7.971755in}{0.557678in}}{\pgfqpoint{7.982354in}{0.562068in}}{\pgfqpoint{7.990168in}{0.569882in}}%
\pgfpathcurveto{\pgfqpoint{7.997981in}{0.577695in}}{\pgfqpoint{8.002372in}{0.588295in}}{\pgfqpoint{8.002372in}{0.599345in}}%
\pgfpathcurveto{\pgfqpoint{8.002372in}{0.610395in}}{\pgfqpoint{7.997981in}{0.620994in}}{\pgfqpoint{7.990168in}{0.628807in}}%
\pgfpathcurveto{\pgfqpoint{7.982354in}{0.636621in}}{\pgfqpoint{7.971755in}{0.641011in}}{\pgfqpoint{7.960705in}{0.641011in}}%
\pgfpathcurveto{\pgfqpoint{7.949655in}{0.641011in}}{\pgfqpoint{7.939056in}{0.636621in}}{\pgfqpoint{7.931242in}{0.628807in}}%
\pgfpathcurveto{\pgfqpoint{7.923429in}{0.620994in}}{\pgfqpoint{7.919038in}{0.610395in}}{\pgfqpoint{7.919038in}{0.599345in}}%
\pgfpathcurveto{\pgfqpoint{7.919038in}{0.588295in}}{\pgfqpoint{7.923429in}{0.577695in}}{\pgfqpoint{7.931242in}{0.569882in}}%
\pgfpathcurveto{\pgfqpoint{7.939056in}{0.562068in}}{\pgfqpoint{7.949655in}{0.557678in}}{\pgfqpoint{7.960705in}{0.557678in}}%
\pgfpathlineto{\pgfqpoint{7.960705in}{0.557678in}}%
\pgfpathclose%
\pgfusepath{stroke}%
\end{pgfscope}%
\begin{pgfscope}%
\pgfpathrectangle{\pgfqpoint{7.394209in}{0.375000in}}{\pgfqpoint{6.356833in}{5.175000in}}%
\pgfusepath{clip}%
\pgfsetbuttcap%
\pgfsetroundjoin%
\pgfsetlinewidth{1.003750pt}%
\definecolor{currentstroke}{rgb}{0.827451,0.827451,0.827451}%
\pgfsetstrokecolor{currentstroke}%
\pgfsetdash{}{0pt}%
\pgfpathmoveto{\pgfqpoint{9.373327in}{2.034377in}}%
\pgfpathcurveto{\pgfqpoint{9.384377in}{2.034377in}}{\pgfqpoint{9.394976in}{2.038768in}}{\pgfqpoint{9.402790in}{2.046581in}}%
\pgfpathcurveto{\pgfqpoint{9.410603in}{2.054395in}}{\pgfqpoint{9.414993in}{2.064994in}}{\pgfqpoint{9.414993in}{2.076044in}}%
\pgfpathcurveto{\pgfqpoint{9.414993in}{2.087094in}}{\pgfqpoint{9.410603in}{2.097693in}}{\pgfqpoint{9.402790in}{2.105507in}}%
\pgfpathcurveto{\pgfqpoint{9.394976in}{2.113320in}}{\pgfqpoint{9.384377in}{2.117711in}}{\pgfqpoint{9.373327in}{2.117711in}}%
\pgfpathcurveto{\pgfqpoint{9.362277in}{2.117711in}}{\pgfqpoint{9.351678in}{2.113320in}}{\pgfqpoint{9.343864in}{2.105507in}}%
\pgfpathcurveto{\pgfqpoint{9.336050in}{2.097693in}}{\pgfqpoint{9.331660in}{2.087094in}}{\pgfqpoint{9.331660in}{2.076044in}}%
\pgfpathcurveto{\pgfqpoint{9.331660in}{2.064994in}}{\pgfqpoint{9.336050in}{2.054395in}}{\pgfqpoint{9.343864in}{2.046581in}}%
\pgfpathcurveto{\pgfqpoint{9.351678in}{2.038768in}}{\pgfqpoint{9.362277in}{2.034377in}}{\pgfqpoint{9.373327in}{2.034377in}}%
\pgfpathlineto{\pgfqpoint{9.373327in}{2.034377in}}%
\pgfpathclose%
\pgfusepath{stroke}%
\end{pgfscope}%
\begin{pgfscope}%
\pgfpathrectangle{\pgfqpoint{7.394209in}{0.375000in}}{\pgfqpoint{6.356833in}{5.175000in}}%
\pgfusepath{clip}%
\pgfsetbuttcap%
\pgfsetroundjoin%
\pgfsetlinewidth{1.003750pt}%
\definecolor{currentstroke}{rgb}{0.827451,0.827451,0.827451}%
\pgfsetstrokecolor{currentstroke}%
\pgfsetdash{}{0pt}%
\pgfpathmoveto{\pgfqpoint{8.714016in}{2.759559in}}%
\pgfpathcurveto{\pgfqpoint{8.725066in}{2.759559in}}{\pgfqpoint{8.735665in}{2.763949in}}{\pgfqpoint{8.743479in}{2.771763in}}%
\pgfpathcurveto{\pgfqpoint{8.751292in}{2.779577in}}{\pgfqpoint{8.755683in}{2.790176in}}{\pgfqpoint{8.755683in}{2.801226in}}%
\pgfpathcurveto{\pgfqpoint{8.755683in}{2.812276in}}{\pgfqpoint{8.751292in}{2.822875in}}{\pgfqpoint{8.743479in}{2.830688in}}%
\pgfpathcurveto{\pgfqpoint{8.735665in}{2.838502in}}{\pgfqpoint{8.725066in}{2.842892in}}{\pgfqpoint{8.714016in}{2.842892in}}%
\pgfpathcurveto{\pgfqpoint{8.702966in}{2.842892in}}{\pgfqpoint{8.692367in}{2.838502in}}{\pgfqpoint{8.684553in}{2.830688in}}%
\pgfpathcurveto{\pgfqpoint{8.676740in}{2.822875in}}{\pgfqpoint{8.672349in}{2.812276in}}{\pgfqpoint{8.672349in}{2.801226in}}%
\pgfpathcurveto{\pgfqpoint{8.672349in}{2.790176in}}{\pgfqpoint{8.676740in}{2.779577in}}{\pgfqpoint{8.684553in}{2.771763in}}%
\pgfpathcurveto{\pgfqpoint{8.692367in}{2.763949in}}{\pgfqpoint{8.702966in}{2.759559in}}{\pgfqpoint{8.714016in}{2.759559in}}%
\pgfpathlineto{\pgfqpoint{8.714016in}{2.759559in}}%
\pgfpathclose%
\pgfusepath{stroke}%
\end{pgfscope}%
\begin{pgfscope}%
\pgfpathrectangle{\pgfqpoint{7.394209in}{0.375000in}}{\pgfqpoint{6.356833in}{5.175000in}}%
\pgfusepath{clip}%
\pgfsetbuttcap%
\pgfsetroundjoin%
\pgfsetlinewidth{1.003750pt}%
\definecolor{currentstroke}{rgb}{0.827451,0.827451,0.827451}%
\pgfsetstrokecolor{currentstroke}%
\pgfsetdash{}{0pt}%
\pgfpathmoveto{\pgfqpoint{11.595516in}{5.237762in}}%
\pgfpathcurveto{\pgfqpoint{11.606566in}{5.237762in}}{\pgfqpoint{11.617165in}{5.242153in}}{\pgfqpoint{11.624979in}{5.249966in}}%
\pgfpathcurveto{\pgfqpoint{11.632792in}{5.257780in}}{\pgfqpoint{11.637183in}{5.268379in}}{\pgfqpoint{11.637183in}{5.279429in}}%
\pgfpathcurveto{\pgfqpoint{11.637183in}{5.290479in}}{\pgfqpoint{11.632792in}{5.301078in}}{\pgfqpoint{11.624979in}{5.308892in}}%
\pgfpathcurveto{\pgfqpoint{11.617165in}{5.316705in}}{\pgfqpoint{11.606566in}{5.321096in}}{\pgfqpoint{11.595516in}{5.321096in}}%
\pgfpathcurveto{\pgfqpoint{11.584466in}{5.321096in}}{\pgfqpoint{11.573867in}{5.316705in}}{\pgfqpoint{11.566053in}{5.308892in}}%
\pgfpathcurveto{\pgfqpoint{11.558240in}{5.301078in}}{\pgfqpoint{11.553849in}{5.290479in}}{\pgfqpoint{11.553849in}{5.279429in}}%
\pgfpathcurveto{\pgfqpoint{11.553849in}{5.268379in}}{\pgfqpoint{11.558240in}{5.257780in}}{\pgfqpoint{11.566053in}{5.249966in}}%
\pgfpathcurveto{\pgfqpoint{11.573867in}{5.242153in}}{\pgfqpoint{11.584466in}{5.237762in}}{\pgfqpoint{11.595516in}{5.237762in}}%
\pgfpathlineto{\pgfqpoint{11.595516in}{5.237762in}}%
\pgfpathclose%
\pgfusepath{stroke}%
\end{pgfscope}%
\begin{pgfscope}%
\pgfpathrectangle{\pgfqpoint{7.394209in}{0.375000in}}{\pgfqpoint{6.356833in}{5.175000in}}%
\pgfusepath{clip}%
\pgfsetbuttcap%
\pgfsetroundjoin%
\pgfsetlinewidth{1.003750pt}%
\definecolor{currentstroke}{rgb}{0.827451,0.827451,0.827451}%
\pgfsetstrokecolor{currentstroke}%
\pgfsetdash{}{0pt}%
\pgfpathmoveto{\pgfqpoint{9.975698in}{3.125376in}}%
\pgfpathcurveto{\pgfqpoint{9.986748in}{3.125376in}}{\pgfqpoint{9.997347in}{3.129766in}}{\pgfqpoint{10.005161in}{3.137580in}}%
\pgfpathcurveto{\pgfqpoint{10.012975in}{3.145394in}}{\pgfqpoint{10.017365in}{3.155993in}}{\pgfqpoint{10.017365in}{3.167043in}}%
\pgfpathcurveto{\pgfqpoint{10.017365in}{3.178093in}}{\pgfqpoint{10.012975in}{3.188692in}}{\pgfqpoint{10.005161in}{3.196505in}}%
\pgfpathcurveto{\pgfqpoint{9.997347in}{3.204319in}}{\pgfqpoint{9.986748in}{3.208709in}}{\pgfqpoint{9.975698in}{3.208709in}}%
\pgfpathcurveto{\pgfqpoint{9.964648in}{3.208709in}}{\pgfqpoint{9.954049in}{3.204319in}}{\pgfqpoint{9.946235in}{3.196505in}}%
\pgfpathcurveto{\pgfqpoint{9.938422in}{3.188692in}}{\pgfqpoint{9.934031in}{3.178093in}}{\pgfqpoint{9.934031in}{3.167043in}}%
\pgfpathcurveto{\pgfqpoint{9.934031in}{3.155993in}}{\pgfqpoint{9.938422in}{3.145394in}}{\pgfqpoint{9.946235in}{3.137580in}}%
\pgfpathcurveto{\pgfqpoint{9.954049in}{3.129766in}}{\pgfqpoint{9.964648in}{3.125376in}}{\pgfqpoint{9.975698in}{3.125376in}}%
\pgfpathlineto{\pgfqpoint{9.975698in}{3.125376in}}%
\pgfpathclose%
\pgfusepath{stroke}%
\end{pgfscope}%
\begin{pgfscope}%
\pgfpathrectangle{\pgfqpoint{7.394209in}{0.375000in}}{\pgfqpoint{6.356833in}{5.175000in}}%
\pgfusepath{clip}%
\pgfsetbuttcap%
\pgfsetroundjoin%
\pgfsetlinewidth{1.003750pt}%
\definecolor{currentstroke}{rgb}{0.827451,0.827451,0.827451}%
\pgfsetstrokecolor{currentstroke}%
\pgfsetdash{}{0pt}%
\pgfpathmoveto{\pgfqpoint{7.486430in}{0.913648in}}%
\pgfpathcurveto{\pgfqpoint{7.497480in}{0.913648in}}{\pgfqpoint{7.508079in}{0.918038in}}{\pgfqpoint{7.515892in}{0.925852in}}%
\pgfpathcurveto{\pgfqpoint{7.523706in}{0.933666in}}{\pgfqpoint{7.528096in}{0.944265in}}{\pgfqpoint{7.528096in}{0.955315in}}%
\pgfpathcurveto{\pgfqpoint{7.528096in}{0.966365in}}{\pgfqpoint{7.523706in}{0.976964in}}{\pgfqpoint{7.515892in}{0.984778in}}%
\pgfpathcurveto{\pgfqpoint{7.508079in}{0.992591in}}{\pgfqpoint{7.497480in}{0.996981in}}{\pgfqpoint{7.486430in}{0.996981in}}%
\pgfpathcurveto{\pgfqpoint{7.475379in}{0.996981in}}{\pgfqpoint{7.464780in}{0.992591in}}{\pgfqpoint{7.456967in}{0.984778in}}%
\pgfpathcurveto{\pgfqpoint{7.449153in}{0.976964in}}{\pgfqpoint{7.444763in}{0.966365in}}{\pgfqpoint{7.444763in}{0.955315in}}%
\pgfpathcurveto{\pgfqpoint{7.444763in}{0.944265in}}{\pgfqpoint{7.449153in}{0.933666in}}{\pgfqpoint{7.456967in}{0.925852in}}%
\pgfpathcurveto{\pgfqpoint{7.464780in}{0.918038in}}{\pgfqpoint{7.475379in}{0.913648in}}{\pgfqpoint{7.486430in}{0.913648in}}%
\pgfpathlineto{\pgfqpoint{7.486430in}{0.913648in}}%
\pgfpathclose%
\pgfusepath{stroke}%
\end{pgfscope}%
\begin{pgfscope}%
\pgfpathrectangle{\pgfqpoint{7.394209in}{0.375000in}}{\pgfqpoint{6.356833in}{5.175000in}}%
\pgfusepath{clip}%
\pgfsetbuttcap%
\pgfsetroundjoin%
\pgfsetlinewidth{1.003750pt}%
\definecolor{currentstroke}{rgb}{0.827451,0.827451,0.827451}%
\pgfsetstrokecolor{currentstroke}%
\pgfsetdash{}{0pt}%
\pgfpathmoveto{\pgfqpoint{7.486430in}{0.761760in}}%
\pgfpathcurveto{\pgfqpoint{7.497480in}{0.761760in}}{\pgfqpoint{7.508079in}{0.766151in}}{\pgfqpoint{7.515892in}{0.773964in}}%
\pgfpathcurveto{\pgfqpoint{7.523706in}{0.781778in}}{\pgfqpoint{7.528096in}{0.792377in}}{\pgfqpoint{7.528096in}{0.803427in}}%
\pgfpathcurveto{\pgfqpoint{7.528096in}{0.814477in}}{\pgfqpoint{7.523706in}{0.825076in}}{\pgfqpoint{7.515892in}{0.832890in}}%
\pgfpathcurveto{\pgfqpoint{7.508079in}{0.840703in}}{\pgfqpoint{7.497480in}{0.845094in}}{\pgfqpoint{7.486430in}{0.845094in}}%
\pgfpathcurveto{\pgfqpoint{7.475379in}{0.845094in}}{\pgfqpoint{7.464780in}{0.840703in}}{\pgfqpoint{7.456967in}{0.832890in}}%
\pgfpathcurveto{\pgfqpoint{7.449153in}{0.825076in}}{\pgfqpoint{7.444763in}{0.814477in}}{\pgfqpoint{7.444763in}{0.803427in}}%
\pgfpathcurveto{\pgfqpoint{7.444763in}{0.792377in}}{\pgfqpoint{7.449153in}{0.781778in}}{\pgfqpoint{7.456967in}{0.773964in}}%
\pgfpathcurveto{\pgfqpoint{7.464780in}{0.766151in}}{\pgfqpoint{7.475379in}{0.761760in}}{\pgfqpoint{7.486430in}{0.761760in}}%
\pgfpathlineto{\pgfqpoint{7.486430in}{0.761760in}}%
\pgfpathclose%
\pgfusepath{stroke}%
\end{pgfscope}%
\begin{pgfscope}%
\pgfpathrectangle{\pgfqpoint{7.394209in}{0.375000in}}{\pgfqpoint{6.356833in}{5.175000in}}%
\pgfusepath{clip}%
\pgfsetbuttcap%
\pgfsetroundjoin%
\pgfsetlinewidth{1.003750pt}%
\definecolor{currentstroke}{rgb}{0.827451,0.827451,0.827451}%
\pgfsetstrokecolor{currentstroke}%
\pgfsetdash{}{0pt}%
\pgfpathmoveto{\pgfqpoint{9.167498in}{3.130791in}}%
\pgfpathcurveto{\pgfqpoint{9.178548in}{3.130791in}}{\pgfqpoint{9.189147in}{3.135182in}}{\pgfqpoint{9.196961in}{3.142995in}}%
\pgfpathcurveto{\pgfqpoint{9.204774in}{3.150809in}}{\pgfqpoint{9.209165in}{3.161408in}}{\pgfqpoint{9.209165in}{3.172458in}}%
\pgfpathcurveto{\pgfqpoint{9.209165in}{3.183508in}}{\pgfqpoint{9.204774in}{3.194107in}}{\pgfqpoint{9.196961in}{3.201921in}}%
\pgfpathcurveto{\pgfqpoint{9.189147in}{3.209734in}}{\pgfqpoint{9.178548in}{3.214125in}}{\pgfqpoint{9.167498in}{3.214125in}}%
\pgfpathcurveto{\pgfqpoint{9.156448in}{3.214125in}}{\pgfqpoint{9.145849in}{3.209734in}}{\pgfqpoint{9.138035in}{3.201921in}}%
\pgfpathcurveto{\pgfqpoint{9.130221in}{3.194107in}}{\pgfqpoint{9.125831in}{3.183508in}}{\pgfqpoint{9.125831in}{3.172458in}}%
\pgfpathcurveto{\pgfqpoint{9.125831in}{3.161408in}}{\pgfqpoint{9.130221in}{3.150809in}}{\pgfqpoint{9.138035in}{3.142995in}}%
\pgfpathcurveto{\pgfqpoint{9.145849in}{3.135182in}}{\pgfqpoint{9.156448in}{3.130791in}}{\pgfqpoint{9.167498in}{3.130791in}}%
\pgfpathlineto{\pgfqpoint{9.167498in}{3.130791in}}%
\pgfpathclose%
\pgfusepath{stroke}%
\end{pgfscope}%
\begin{pgfscope}%
\pgfpathrectangle{\pgfqpoint{7.394209in}{0.375000in}}{\pgfqpoint{6.356833in}{5.175000in}}%
\pgfusepath{clip}%
\pgfsetbuttcap%
\pgfsetroundjoin%
\pgfsetlinewidth{1.003750pt}%
\definecolor{currentstroke}{rgb}{0.827451,0.827451,0.827451}%
\pgfsetstrokecolor{currentstroke}%
\pgfsetdash{}{0pt}%
\pgfpathmoveto{\pgfqpoint{11.232043in}{4.795586in}}%
\pgfpathcurveto{\pgfqpoint{11.243093in}{4.795586in}}{\pgfqpoint{11.253692in}{4.799976in}}{\pgfqpoint{11.261506in}{4.807790in}}%
\pgfpathcurveto{\pgfqpoint{11.269319in}{4.815603in}}{\pgfqpoint{11.273709in}{4.826202in}}{\pgfqpoint{11.273709in}{4.837252in}}%
\pgfpathcurveto{\pgfqpoint{11.273709in}{4.848303in}}{\pgfqpoint{11.269319in}{4.858902in}}{\pgfqpoint{11.261506in}{4.866715in}}%
\pgfpathcurveto{\pgfqpoint{11.253692in}{4.874529in}}{\pgfqpoint{11.243093in}{4.878919in}}{\pgfqpoint{11.232043in}{4.878919in}}%
\pgfpathcurveto{\pgfqpoint{11.220993in}{4.878919in}}{\pgfqpoint{11.210394in}{4.874529in}}{\pgfqpoint{11.202580in}{4.866715in}}%
\pgfpathcurveto{\pgfqpoint{11.194766in}{4.858902in}}{\pgfqpoint{11.190376in}{4.848303in}}{\pgfqpoint{11.190376in}{4.837252in}}%
\pgfpathcurveto{\pgfqpoint{11.190376in}{4.826202in}}{\pgfqpoint{11.194766in}{4.815603in}}{\pgfqpoint{11.202580in}{4.807790in}}%
\pgfpathcurveto{\pgfqpoint{11.210394in}{4.799976in}}{\pgfqpoint{11.220993in}{4.795586in}}{\pgfqpoint{11.232043in}{4.795586in}}%
\pgfpathlineto{\pgfqpoint{11.232043in}{4.795586in}}%
\pgfpathclose%
\pgfusepath{stroke}%
\end{pgfscope}%
\begin{pgfscope}%
\pgfpathrectangle{\pgfqpoint{7.394209in}{0.375000in}}{\pgfqpoint{6.356833in}{5.175000in}}%
\pgfusepath{clip}%
\pgfsetbuttcap%
\pgfsetroundjoin%
\pgfsetlinewidth{1.003750pt}%
\definecolor{currentstroke}{rgb}{0.827451,0.827451,0.827451}%
\pgfsetstrokecolor{currentstroke}%
\pgfsetdash{}{0pt}%
\pgfpathmoveto{\pgfqpoint{8.222050in}{2.213466in}}%
\pgfpathcurveto{\pgfqpoint{8.233100in}{2.213466in}}{\pgfqpoint{8.243699in}{2.217856in}}{\pgfqpoint{8.251513in}{2.225670in}}%
\pgfpathcurveto{\pgfqpoint{8.259326in}{2.233484in}}{\pgfqpoint{8.263717in}{2.244083in}}{\pgfqpoint{8.263717in}{2.255133in}}%
\pgfpathcurveto{\pgfqpoint{8.263717in}{2.266183in}}{\pgfqpoint{8.259326in}{2.276782in}}{\pgfqpoint{8.251513in}{2.284595in}}%
\pgfpathcurveto{\pgfqpoint{8.243699in}{2.292409in}}{\pgfqpoint{8.233100in}{2.296799in}}{\pgfqpoint{8.222050in}{2.296799in}}%
\pgfpathcurveto{\pgfqpoint{8.211000in}{2.296799in}}{\pgfqpoint{8.200401in}{2.292409in}}{\pgfqpoint{8.192587in}{2.284595in}}%
\pgfpathcurveto{\pgfqpoint{8.184774in}{2.276782in}}{\pgfqpoint{8.180383in}{2.266183in}}{\pgfqpoint{8.180383in}{2.255133in}}%
\pgfpathcurveto{\pgfqpoint{8.180383in}{2.244083in}}{\pgfqpoint{8.184774in}{2.233484in}}{\pgfqpoint{8.192587in}{2.225670in}}%
\pgfpathcurveto{\pgfqpoint{8.200401in}{2.217856in}}{\pgfqpoint{8.211000in}{2.213466in}}{\pgfqpoint{8.222050in}{2.213466in}}%
\pgfpathlineto{\pgfqpoint{8.222050in}{2.213466in}}%
\pgfpathclose%
\pgfusepath{stroke}%
\end{pgfscope}%
\begin{pgfscope}%
\pgfpathrectangle{\pgfqpoint{7.394209in}{0.375000in}}{\pgfqpoint{6.356833in}{5.175000in}}%
\pgfusepath{clip}%
\pgfsetbuttcap%
\pgfsetroundjoin%
\pgfsetlinewidth{1.003750pt}%
\definecolor{currentstroke}{rgb}{0.827451,0.827451,0.827451}%
\pgfsetstrokecolor{currentstroke}%
\pgfsetdash{}{0pt}%
\pgfpathmoveto{\pgfqpoint{10.512975in}{3.830660in}}%
\pgfpathcurveto{\pgfqpoint{10.524025in}{3.830660in}}{\pgfqpoint{10.534625in}{3.835050in}}{\pgfqpoint{10.542438in}{3.842864in}}%
\pgfpathcurveto{\pgfqpoint{10.550252in}{3.850677in}}{\pgfqpoint{10.554642in}{3.861276in}}{\pgfqpoint{10.554642in}{3.872326in}}%
\pgfpathcurveto{\pgfqpoint{10.554642in}{3.883376in}}{\pgfqpoint{10.550252in}{3.893976in}}{\pgfqpoint{10.542438in}{3.901789in}}%
\pgfpathcurveto{\pgfqpoint{10.534625in}{3.909603in}}{\pgfqpoint{10.524025in}{3.913993in}}{\pgfqpoint{10.512975in}{3.913993in}}%
\pgfpathcurveto{\pgfqpoint{10.501925in}{3.913993in}}{\pgfqpoint{10.491326in}{3.909603in}}{\pgfqpoint{10.483513in}{3.901789in}}%
\pgfpathcurveto{\pgfqpoint{10.475699in}{3.893976in}}{\pgfqpoint{10.471309in}{3.883376in}}{\pgfqpoint{10.471309in}{3.872326in}}%
\pgfpathcurveto{\pgfqpoint{10.471309in}{3.861276in}}{\pgfqpoint{10.475699in}{3.850677in}}{\pgfqpoint{10.483513in}{3.842864in}}%
\pgfpathcurveto{\pgfqpoint{10.491326in}{3.835050in}}{\pgfqpoint{10.501925in}{3.830660in}}{\pgfqpoint{10.512975in}{3.830660in}}%
\pgfpathlineto{\pgfqpoint{10.512975in}{3.830660in}}%
\pgfpathclose%
\pgfusepath{stroke}%
\end{pgfscope}%
\begin{pgfscope}%
\pgfpathrectangle{\pgfqpoint{7.394209in}{0.375000in}}{\pgfqpoint{6.356833in}{5.175000in}}%
\pgfusepath{clip}%
\pgfsetbuttcap%
\pgfsetroundjoin%
\pgfsetlinewidth{1.003750pt}%
\definecolor{currentstroke}{rgb}{0.827451,0.827451,0.827451}%
\pgfsetstrokecolor{currentstroke}%
\pgfsetdash{}{0pt}%
\pgfpathmoveto{\pgfqpoint{10.272892in}{3.865396in}}%
\pgfpathcurveto{\pgfqpoint{10.283942in}{3.865396in}}{\pgfqpoint{10.294541in}{3.869786in}}{\pgfqpoint{10.302355in}{3.877600in}}%
\pgfpathcurveto{\pgfqpoint{10.310168in}{3.885413in}}{\pgfqpoint{10.314559in}{3.896012in}}{\pgfqpoint{10.314559in}{3.907062in}}%
\pgfpathcurveto{\pgfqpoint{10.314559in}{3.918112in}}{\pgfqpoint{10.310168in}{3.928711in}}{\pgfqpoint{10.302355in}{3.936525in}}%
\pgfpathcurveto{\pgfqpoint{10.294541in}{3.944339in}}{\pgfqpoint{10.283942in}{3.948729in}}{\pgfqpoint{10.272892in}{3.948729in}}%
\pgfpathcurveto{\pgfqpoint{10.261842in}{3.948729in}}{\pgfqpoint{10.251243in}{3.944339in}}{\pgfqpoint{10.243429in}{3.936525in}}%
\pgfpathcurveto{\pgfqpoint{10.235615in}{3.928711in}}{\pgfqpoint{10.231225in}{3.918112in}}{\pgfqpoint{10.231225in}{3.907062in}}%
\pgfpathcurveto{\pgfqpoint{10.231225in}{3.896012in}}{\pgfqpoint{10.235615in}{3.885413in}}{\pgfqpoint{10.243429in}{3.877600in}}%
\pgfpathcurveto{\pgfqpoint{10.251243in}{3.869786in}}{\pgfqpoint{10.261842in}{3.865396in}}{\pgfqpoint{10.272892in}{3.865396in}}%
\pgfpathlineto{\pgfqpoint{10.272892in}{3.865396in}}%
\pgfpathclose%
\pgfusepath{stroke}%
\end{pgfscope}%
\begin{pgfscope}%
\pgfpathrectangle{\pgfqpoint{7.394209in}{0.375000in}}{\pgfqpoint{6.356833in}{5.175000in}}%
\pgfusepath{clip}%
\pgfsetbuttcap%
\pgfsetroundjoin%
\pgfsetlinewidth{1.003750pt}%
\definecolor{currentstroke}{rgb}{0.827451,0.827451,0.827451}%
\pgfsetstrokecolor{currentstroke}%
\pgfsetdash{}{0pt}%
\pgfpathmoveto{\pgfqpoint{11.521047in}{5.237762in}}%
\pgfpathcurveto{\pgfqpoint{11.532097in}{5.237762in}}{\pgfqpoint{11.542696in}{5.242153in}}{\pgfqpoint{11.550509in}{5.249966in}}%
\pgfpathcurveto{\pgfqpoint{11.558323in}{5.257780in}}{\pgfqpoint{11.562713in}{5.268379in}}{\pgfqpoint{11.562713in}{5.279429in}}%
\pgfpathcurveto{\pgfqpoint{11.562713in}{5.290479in}}{\pgfqpoint{11.558323in}{5.301078in}}{\pgfqpoint{11.550509in}{5.308892in}}%
\pgfpathcurveto{\pgfqpoint{11.542696in}{5.316705in}}{\pgfqpoint{11.532097in}{5.321096in}}{\pgfqpoint{11.521047in}{5.321096in}}%
\pgfpathcurveto{\pgfqpoint{11.509996in}{5.321096in}}{\pgfqpoint{11.499397in}{5.316705in}}{\pgfqpoint{11.491584in}{5.308892in}}%
\pgfpathcurveto{\pgfqpoint{11.483770in}{5.301078in}}{\pgfqpoint{11.479380in}{5.290479in}}{\pgfqpoint{11.479380in}{5.279429in}}%
\pgfpathcurveto{\pgfqpoint{11.479380in}{5.268379in}}{\pgfqpoint{11.483770in}{5.257780in}}{\pgfqpoint{11.491584in}{5.249966in}}%
\pgfpathcurveto{\pgfqpoint{11.499397in}{5.242153in}}{\pgfqpoint{11.509996in}{5.237762in}}{\pgfqpoint{11.521047in}{5.237762in}}%
\pgfpathlineto{\pgfqpoint{11.521047in}{5.237762in}}%
\pgfpathclose%
\pgfusepath{stroke}%
\end{pgfscope}%
\begin{pgfscope}%
\pgfpathrectangle{\pgfqpoint{7.394209in}{0.375000in}}{\pgfqpoint{6.356833in}{5.175000in}}%
\pgfusepath{clip}%
\pgfsetbuttcap%
\pgfsetroundjoin%
\pgfsetlinewidth{1.003750pt}%
\definecolor{currentstroke}{rgb}{0.827451,0.827451,0.827451}%
\pgfsetstrokecolor{currentstroke}%
\pgfsetdash{}{0pt}%
\pgfpathmoveto{\pgfqpoint{9.392769in}{2.213466in}}%
\pgfpathcurveto{\pgfqpoint{9.403819in}{2.213466in}}{\pgfqpoint{9.414418in}{2.217856in}}{\pgfqpoint{9.422232in}{2.225670in}}%
\pgfpathcurveto{\pgfqpoint{9.430046in}{2.233484in}}{\pgfqpoint{9.434436in}{2.244083in}}{\pgfqpoint{9.434436in}{2.255133in}}%
\pgfpathcurveto{\pgfqpoint{9.434436in}{2.266183in}}{\pgfqpoint{9.430046in}{2.276782in}}{\pgfqpoint{9.422232in}{2.284595in}}%
\pgfpathcurveto{\pgfqpoint{9.414418in}{2.292409in}}{\pgfqpoint{9.403819in}{2.296799in}}{\pgfqpoint{9.392769in}{2.296799in}}%
\pgfpathcurveto{\pgfqpoint{9.381719in}{2.296799in}}{\pgfqpoint{9.371120in}{2.292409in}}{\pgfqpoint{9.363306in}{2.284595in}}%
\pgfpathcurveto{\pgfqpoint{9.355493in}{2.276782in}}{\pgfqpoint{9.351103in}{2.266183in}}{\pgfqpoint{9.351103in}{2.255133in}}%
\pgfpathcurveto{\pgfqpoint{9.351103in}{2.244083in}}{\pgfqpoint{9.355493in}{2.233484in}}{\pgfqpoint{9.363306in}{2.225670in}}%
\pgfpathcurveto{\pgfqpoint{9.371120in}{2.217856in}}{\pgfqpoint{9.381719in}{2.213466in}}{\pgfqpoint{9.392769in}{2.213466in}}%
\pgfpathlineto{\pgfqpoint{9.392769in}{2.213466in}}%
\pgfpathclose%
\pgfusepath{stroke}%
\end{pgfscope}%
\begin{pgfscope}%
\pgfpathrectangle{\pgfqpoint{7.394209in}{0.375000in}}{\pgfqpoint{6.356833in}{5.175000in}}%
\pgfusepath{clip}%
\pgfsetbuttcap%
\pgfsetroundjoin%
\pgfsetlinewidth{1.003750pt}%
\definecolor{currentstroke}{rgb}{0.827451,0.827451,0.827451}%
\pgfsetstrokecolor{currentstroke}%
\pgfsetdash{}{0pt}%
\pgfpathmoveto{\pgfqpoint{8.385168in}{2.167237in}}%
\pgfpathcurveto{\pgfqpoint{8.396218in}{2.167237in}}{\pgfqpoint{8.406817in}{2.171627in}}{\pgfqpoint{8.414630in}{2.179441in}}%
\pgfpathcurveto{\pgfqpoint{8.422444in}{2.187255in}}{\pgfqpoint{8.426834in}{2.197854in}}{\pgfqpoint{8.426834in}{2.208904in}}%
\pgfpathcurveto{\pgfqpoint{8.426834in}{2.219954in}}{\pgfqpoint{8.422444in}{2.230553in}}{\pgfqpoint{8.414630in}{2.238367in}}%
\pgfpathcurveto{\pgfqpoint{8.406817in}{2.246180in}}{\pgfqpoint{8.396218in}{2.250570in}}{\pgfqpoint{8.385168in}{2.250570in}}%
\pgfpathcurveto{\pgfqpoint{8.374117in}{2.250570in}}{\pgfqpoint{8.363518in}{2.246180in}}{\pgfqpoint{8.355705in}{2.238367in}}%
\pgfpathcurveto{\pgfqpoint{8.347891in}{2.230553in}}{\pgfqpoint{8.343501in}{2.219954in}}{\pgfqpoint{8.343501in}{2.208904in}}%
\pgfpathcurveto{\pgfqpoint{8.343501in}{2.197854in}}{\pgfqpoint{8.347891in}{2.187255in}}{\pgfqpoint{8.355705in}{2.179441in}}%
\pgfpathcurveto{\pgfqpoint{8.363518in}{2.171627in}}{\pgfqpoint{8.374117in}{2.167237in}}{\pgfqpoint{8.385168in}{2.167237in}}%
\pgfpathlineto{\pgfqpoint{8.385168in}{2.167237in}}%
\pgfpathclose%
\pgfusepath{stroke}%
\end{pgfscope}%
\begin{pgfscope}%
\pgfpathrectangle{\pgfqpoint{7.394209in}{0.375000in}}{\pgfqpoint{6.356833in}{5.175000in}}%
\pgfusepath{clip}%
\pgfsetbuttcap%
\pgfsetroundjoin%
\pgfsetlinewidth{1.003750pt}%
\definecolor{currentstroke}{rgb}{0.827451,0.827451,0.827451}%
\pgfsetstrokecolor{currentstroke}%
\pgfsetdash{}{0pt}%
\pgfpathmoveto{\pgfqpoint{8.015256in}{0.501764in}}%
\pgfpathcurveto{\pgfqpoint{8.026306in}{0.501764in}}{\pgfqpoint{8.036905in}{0.506154in}}{\pgfqpoint{8.044719in}{0.513968in}}%
\pgfpathcurveto{\pgfqpoint{8.052532in}{0.521781in}}{\pgfqpoint{8.056923in}{0.532380in}}{\pgfqpoint{8.056923in}{0.543430in}}%
\pgfpathcurveto{\pgfqpoint{8.056923in}{0.554481in}}{\pgfqpoint{8.052532in}{0.565080in}}{\pgfqpoint{8.044719in}{0.572893in}}%
\pgfpathcurveto{\pgfqpoint{8.036905in}{0.580707in}}{\pgfqpoint{8.026306in}{0.585097in}}{\pgfqpoint{8.015256in}{0.585097in}}%
\pgfpathcurveto{\pgfqpoint{8.004206in}{0.585097in}}{\pgfqpoint{7.993607in}{0.580707in}}{\pgfqpoint{7.985793in}{0.572893in}}%
\pgfpathcurveto{\pgfqpoint{7.977979in}{0.565080in}}{\pgfqpoint{7.973589in}{0.554481in}}{\pgfqpoint{7.973589in}{0.543430in}}%
\pgfpathcurveto{\pgfqpoint{7.973589in}{0.532380in}}{\pgfqpoint{7.977979in}{0.521781in}}{\pgfqpoint{7.985793in}{0.513968in}}%
\pgfpathcurveto{\pgfqpoint{7.993607in}{0.506154in}}{\pgfqpoint{8.004206in}{0.501764in}}{\pgfqpoint{8.015256in}{0.501764in}}%
\pgfpathlineto{\pgfqpoint{8.015256in}{0.501764in}}%
\pgfpathclose%
\pgfusepath{stroke}%
\end{pgfscope}%
\begin{pgfscope}%
\pgfpathrectangle{\pgfqpoint{7.394209in}{0.375000in}}{\pgfqpoint{6.356833in}{5.175000in}}%
\pgfusepath{clip}%
\pgfsetbuttcap%
\pgfsetroundjoin%
\pgfsetlinewidth{1.003750pt}%
\definecolor{currentstroke}{rgb}{0.827451,0.827451,0.827451}%
\pgfsetstrokecolor{currentstroke}%
\pgfsetdash{}{0pt}%
\pgfpathmoveto{\pgfqpoint{11.521047in}{5.086536in}}%
\pgfpathcurveto{\pgfqpoint{11.532097in}{5.086536in}}{\pgfqpoint{11.542696in}{5.090927in}}{\pgfqpoint{11.550509in}{5.098740in}}%
\pgfpathcurveto{\pgfqpoint{11.558323in}{5.106554in}}{\pgfqpoint{11.562713in}{5.117153in}}{\pgfqpoint{11.562713in}{5.128203in}}%
\pgfpathcurveto{\pgfqpoint{11.562713in}{5.139253in}}{\pgfqpoint{11.558323in}{5.149852in}}{\pgfqpoint{11.550509in}{5.157666in}}%
\pgfpathcurveto{\pgfqpoint{11.542696in}{5.165479in}}{\pgfqpoint{11.532097in}{5.169870in}}{\pgfqpoint{11.521047in}{5.169870in}}%
\pgfpathcurveto{\pgfqpoint{11.509996in}{5.169870in}}{\pgfqpoint{11.499397in}{5.165479in}}{\pgfqpoint{11.491584in}{5.157666in}}%
\pgfpathcurveto{\pgfqpoint{11.483770in}{5.149852in}}{\pgfqpoint{11.479380in}{5.139253in}}{\pgfqpoint{11.479380in}{5.128203in}}%
\pgfpathcurveto{\pgfqpoint{11.479380in}{5.117153in}}{\pgfqpoint{11.483770in}{5.106554in}}{\pgfqpoint{11.491584in}{5.098740in}}%
\pgfpathcurveto{\pgfqpoint{11.499397in}{5.090927in}}{\pgfqpoint{11.509996in}{5.086536in}}{\pgfqpoint{11.521047in}{5.086536in}}%
\pgfpathlineto{\pgfqpoint{11.521047in}{5.086536in}}%
\pgfpathclose%
\pgfusepath{stroke}%
\end{pgfscope}%
\begin{pgfscope}%
\pgfpathrectangle{\pgfqpoint{7.394209in}{0.375000in}}{\pgfqpoint{6.356833in}{5.175000in}}%
\pgfusepath{clip}%
\pgfsetbuttcap%
\pgfsetroundjoin%
\pgfsetlinewidth{1.003750pt}%
\definecolor{currentstroke}{rgb}{0.827451,0.827451,0.827451}%
\pgfsetstrokecolor{currentstroke}%
\pgfsetdash{}{0pt}%
\pgfpathmoveto{\pgfqpoint{8.729741in}{1.837853in}}%
\pgfpathcurveto{\pgfqpoint{8.740791in}{1.837853in}}{\pgfqpoint{8.751391in}{1.842243in}}{\pgfqpoint{8.759204in}{1.850056in}}%
\pgfpathcurveto{\pgfqpoint{8.767018in}{1.857870in}}{\pgfqpoint{8.771408in}{1.868469in}}{\pgfqpoint{8.771408in}{1.879519in}}%
\pgfpathcurveto{\pgfqpoint{8.771408in}{1.890569in}}{\pgfqpoint{8.767018in}{1.901168in}}{\pgfqpoint{8.759204in}{1.908982in}}%
\pgfpathcurveto{\pgfqpoint{8.751391in}{1.916796in}}{\pgfqpoint{8.740791in}{1.921186in}}{\pgfqpoint{8.729741in}{1.921186in}}%
\pgfpathcurveto{\pgfqpoint{8.718691in}{1.921186in}}{\pgfqpoint{8.708092in}{1.916796in}}{\pgfqpoint{8.700279in}{1.908982in}}%
\pgfpathcurveto{\pgfqpoint{8.692465in}{1.901168in}}{\pgfqpoint{8.688075in}{1.890569in}}{\pgfqpoint{8.688075in}{1.879519in}}%
\pgfpathcurveto{\pgfqpoint{8.688075in}{1.868469in}}{\pgfqpoint{8.692465in}{1.857870in}}{\pgfqpoint{8.700279in}{1.850056in}}%
\pgfpathcurveto{\pgfqpoint{8.708092in}{1.842243in}}{\pgfqpoint{8.718691in}{1.837853in}}{\pgfqpoint{8.729741in}{1.837853in}}%
\pgfpathlineto{\pgfqpoint{8.729741in}{1.837853in}}%
\pgfpathclose%
\pgfusepath{stroke}%
\end{pgfscope}%
\begin{pgfscope}%
\pgfpathrectangle{\pgfqpoint{7.394209in}{0.375000in}}{\pgfqpoint{6.356833in}{5.175000in}}%
\pgfusepath{clip}%
\pgfsetbuttcap%
\pgfsetroundjoin%
\pgfsetlinewidth{1.003750pt}%
\definecolor{currentstroke}{rgb}{0.827451,0.827451,0.827451}%
\pgfsetstrokecolor{currentstroke}%
\pgfsetdash{}{0pt}%
\pgfpathmoveto{\pgfqpoint{11.521047in}{5.227222in}}%
\pgfpathcurveto{\pgfqpoint{11.532097in}{5.227222in}}{\pgfqpoint{11.542696in}{5.231612in}}{\pgfqpoint{11.550509in}{5.239426in}}%
\pgfpathcurveto{\pgfqpoint{11.558323in}{5.247239in}}{\pgfqpoint{11.562713in}{5.257838in}}{\pgfqpoint{11.562713in}{5.268889in}}%
\pgfpathcurveto{\pgfqpoint{11.562713in}{5.279939in}}{\pgfqpoint{11.558323in}{5.290538in}}{\pgfqpoint{11.550509in}{5.298351in}}%
\pgfpathcurveto{\pgfqpoint{11.542696in}{5.306165in}}{\pgfqpoint{11.532097in}{5.310555in}}{\pgfqpoint{11.521047in}{5.310555in}}%
\pgfpathcurveto{\pgfqpoint{11.509996in}{5.310555in}}{\pgfqpoint{11.499397in}{5.306165in}}{\pgfqpoint{11.491584in}{5.298351in}}%
\pgfpathcurveto{\pgfqpoint{11.483770in}{5.290538in}}{\pgfqpoint{11.479380in}{5.279939in}}{\pgfqpoint{11.479380in}{5.268889in}}%
\pgfpathcurveto{\pgfqpoint{11.479380in}{5.257838in}}{\pgfqpoint{11.483770in}{5.247239in}}{\pgfqpoint{11.491584in}{5.239426in}}%
\pgfpathcurveto{\pgfqpoint{11.499397in}{5.231612in}}{\pgfqpoint{11.509996in}{5.227222in}}{\pgfqpoint{11.521047in}{5.227222in}}%
\pgfpathlineto{\pgfqpoint{11.521047in}{5.227222in}}%
\pgfpathclose%
\pgfusepath{stroke}%
\end{pgfscope}%
\begin{pgfscope}%
\pgfpathrectangle{\pgfqpoint{7.394209in}{0.375000in}}{\pgfqpoint{6.356833in}{5.175000in}}%
\pgfusepath{clip}%
\pgfsetbuttcap%
\pgfsetroundjoin%
\pgfsetlinewidth{1.003750pt}%
\definecolor{currentstroke}{rgb}{0.827451,0.827451,0.827451}%
\pgfsetstrokecolor{currentstroke}%
\pgfsetdash{}{0pt}%
\pgfpathmoveto{\pgfqpoint{10.150255in}{5.140313in}}%
\pgfpathcurveto{\pgfqpoint{10.161305in}{5.140313in}}{\pgfqpoint{10.171904in}{5.144703in}}{\pgfqpoint{10.179717in}{5.152517in}}%
\pgfpathcurveto{\pgfqpoint{10.187531in}{5.160330in}}{\pgfqpoint{10.191921in}{5.170929in}}{\pgfqpoint{10.191921in}{5.181979in}}%
\pgfpathcurveto{\pgfqpoint{10.191921in}{5.193029in}}{\pgfqpoint{10.187531in}{5.203628in}}{\pgfqpoint{10.179717in}{5.211442in}}%
\pgfpathcurveto{\pgfqpoint{10.171904in}{5.219256in}}{\pgfqpoint{10.161305in}{5.223646in}}{\pgfqpoint{10.150255in}{5.223646in}}%
\pgfpathcurveto{\pgfqpoint{10.139205in}{5.223646in}}{\pgfqpoint{10.128605in}{5.219256in}}{\pgfqpoint{10.120792in}{5.211442in}}%
\pgfpathcurveto{\pgfqpoint{10.112978in}{5.203628in}}{\pgfqpoint{10.108588in}{5.193029in}}{\pgfqpoint{10.108588in}{5.181979in}}%
\pgfpathcurveto{\pgfqpoint{10.108588in}{5.170929in}}{\pgfqpoint{10.112978in}{5.160330in}}{\pgfqpoint{10.120792in}{5.152517in}}%
\pgfpathcurveto{\pgfqpoint{10.128605in}{5.144703in}}{\pgfqpoint{10.139205in}{5.140313in}}{\pgfqpoint{10.150255in}{5.140313in}}%
\pgfpathlineto{\pgfqpoint{10.150255in}{5.140313in}}%
\pgfpathclose%
\pgfusepath{stroke}%
\end{pgfscope}%
\begin{pgfscope}%
\pgfpathrectangle{\pgfqpoint{7.394209in}{0.375000in}}{\pgfqpoint{6.356833in}{5.175000in}}%
\pgfusepath{clip}%
\pgfsetbuttcap%
\pgfsetroundjoin%
\pgfsetlinewidth{1.003750pt}%
\definecolor{currentstroke}{rgb}{0.827451,0.827451,0.827451}%
\pgfsetstrokecolor{currentstroke}%
\pgfsetdash{}{0pt}%
\pgfpathmoveto{\pgfqpoint{11.495187in}{5.118763in}}%
\pgfpathcurveto{\pgfqpoint{11.506238in}{5.118763in}}{\pgfqpoint{11.516837in}{5.123154in}}{\pgfqpoint{11.524650in}{5.130967in}}%
\pgfpathcurveto{\pgfqpoint{11.532464in}{5.138781in}}{\pgfqpoint{11.536854in}{5.149380in}}{\pgfqpoint{11.536854in}{5.160430in}}%
\pgfpathcurveto{\pgfqpoint{11.536854in}{5.171480in}}{\pgfqpoint{11.532464in}{5.182079in}}{\pgfqpoint{11.524650in}{5.189893in}}%
\pgfpathcurveto{\pgfqpoint{11.516837in}{5.197707in}}{\pgfqpoint{11.506238in}{5.202097in}}{\pgfqpoint{11.495187in}{5.202097in}}%
\pgfpathcurveto{\pgfqpoint{11.484137in}{5.202097in}}{\pgfqpoint{11.473538in}{5.197707in}}{\pgfqpoint{11.465725in}{5.189893in}}%
\pgfpathcurveto{\pgfqpoint{11.457911in}{5.182079in}}{\pgfqpoint{11.453521in}{5.171480in}}{\pgfqpoint{11.453521in}{5.160430in}}%
\pgfpathcurveto{\pgfqpoint{11.453521in}{5.149380in}}{\pgfqpoint{11.457911in}{5.138781in}}{\pgfqpoint{11.465725in}{5.130967in}}%
\pgfpathcurveto{\pgfqpoint{11.473538in}{5.123154in}}{\pgfqpoint{11.484137in}{5.118763in}}{\pgfqpoint{11.495187in}{5.118763in}}%
\pgfpathlineto{\pgfqpoint{11.495187in}{5.118763in}}%
\pgfpathclose%
\pgfusepath{stroke}%
\end{pgfscope}%
\begin{pgfscope}%
\pgfpathrectangle{\pgfqpoint{7.394209in}{0.375000in}}{\pgfqpoint{6.356833in}{5.175000in}}%
\pgfusepath{clip}%
\pgfsetbuttcap%
\pgfsetroundjoin%
\pgfsetlinewidth{1.003750pt}%
\definecolor{currentstroke}{rgb}{0.827451,0.827451,0.827451}%
\pgfsetstrokecolor{currentstroke}%
\pgfsetdash{}{0pt}%
\pgfpathmoveto{\pgfqpoint{9.446955in}{3.542745in}}%
\pgfpathcurveto{\pgfqpoint{9.458005in}{3.542745in}}{\pgfqpoint{9.468605in}{3.547135in}}{\pgfqpoint{9.476418in}{3.554949in}}%
\pgfpathcurveto{\pgfqpoint{9.484232in}{3.562762in}}{\pgfqpoint{9.488622in}{3.573361in}}{\pgfqpoint{9.488622in}{3.584411in}}%
\pgfpathcurveto{\pgfqpoint{9.488622in}{3.595462in}}{\pgfqpoint{9.484232in}{3.606061in}}{\pgfqpoint{9.476418in}{3.613874in}}%
\pgfpathcurveto{\pgfqpoint{9.468605in}{3.621688in}}{\pgfqpoint{9.458005in}{3.626078in}}{\pgfqpoint{9.446955in}{3.626078in}}%
\pgfpathcurveto{\pgfqpoint{9.435905in}{3.626078in}}{\pgfqpoint{9.425306in}{3.621688in}}{\pgfqpoint{9.417493in}{3.613874in}}%
\pgfpathcurveto{\pgfqpoint{9.409679in}{3.606061in}}{\pgfqpoint{9.405289in}{3.595462in}}{\pgfqpoint{9.405289in}{3.584411in}}%
\pgfpathcurveto{\pgfqpoint{9.405289in}{3.573361in}}{\pgfqpoint{9.409679in}{3.562762in}}{\pgfqpoint{9.417493in}{3.554949in}}%
\pgfpathcurveto{\pgfqpoint{9.425306in}{3.547135in}}{\pgfqpoint{9.435905in}{3.542745in}}{\pgfqpoint{9.446955in}{3.542745in}}%
\pgfpathlineto{\pgfqpoint{9.446955in}{3.542745in}}%
\pgfpathclose%
\pgfusepath{stroke}%
\end{pgfscope}%
\begin{pgfscope}%
\pgfpathrectangle{\pgfqpoint{7.394209in}{0.375000in}}{\pgfqpoint{6.356833in}{5.175000in}}%
\pgfusepath{clip}%
\pgfsetbuttcap%
\pgfsetroundjoin%
\pgfsetlinewidth{1.003750pt}%
\definecolor{currentstroke}{rgb}{0.827451,0.827451,0.827451}%
\pgfsetstrokecolor{currentstroke}%
\pgfsetdash{}{0pt}%
\pgfpathmoveto{\pgfqpoint{12.113774in}{5.412800in}}%
\pgfpathcurveto{\pgfqpoint{12.124825in}{5.412800in}}{\pgfqpoint{12.135424in}{5.417191in}}{\pgfqpoint{12.143237in}{5.425004in}}%
\pgfpathcurveto{\pgfqpoint{12.151051in}{5.432818in}}{\pgfqpoint{12.155441in}{5.443417in}}{\pgfqpoint{12.155441in}{5.454467in}}%
\pgfpathcurveto{\pgfqpoint{12.155441in}{5.465517in}}{\pgfqpoint{12.151051in}{5.476116in}}{\pgfqpoint{12.143237in}{5.483930in}}%
\pgfpathcurveto{\pgfqpoint{12.135424in}{5.491744in}}{\pgfqpoint{12.124825in}{5.496134in}}{\pgfqpoint{12.113774in}{5.496134in}}%
\pgfpathcurveto{\pgfqpoint{12.102724in}{5.496134in}}{\pgfqpoint{12.092125in}{5.491744in}}{\pgfqpoint{12.084312in}{5.483930in}}%
\pgfpathcurveto{\pgfqpoint{12.076498in}{5.476116in}}{\pgfqpoint{12.072108in}{5.465517in}}{\pgfqpoint{12.072108in}{5.454467in}}%
\pgfpathcurveto{\pgfqpoint{12.072108in}{5.443417in}}{\pgfqpoint{12.076498in}{5.432818in}}{\pgfqpoint{12.084312in}{5.425004in}}%
\pgfpathcurveto{\pgfqpoint{12.092125in}{5.417191in}}{\pgfqpoint{12.102724in}{5.412800in}}{\pgfqpoint{12.113774in}{5.412800in}}%
\pgfpathlineto{\pgfqpoint{12.113774in}{5.412800in}}%
\pgfpathclose%
\pgfusepath{stroke}%
\end{pgfscope}%
\begin{pgfscope}%
\pgfpathrectangle{\pgfqpoint{7.394209in}{0.375000in}}{\pgfqpoint{6.356833in}{5.175000in}}%
\pgfusepath{clip}%
\pgfsetbuttcap%
\pgfsetroundjoin%
\pgfsetlinewidth{1.003750pt}%
\definecolor{currentstroke}{rgb}{0.827451,0.827451,0.827451}%
\pgfsetstrokecolor{currentstroke}%
\pgfsetdash{}{0pt}%
\pgfpathmoveto{\pgfqpoint{9.615497in}{2.470388in}}%
\pgfpathcurveto{\pgfqpoint{9.626548in}{2.470388in}}{\pgfqpoint{9.637147in}{2.474778in}}{\pgfqpoint{9.644960in}{2.482592in}}%
\pgfpathcurveto{\pgfqpoint{9.652774in}{2.490406in}}{\pgfqpoint{9.657164in}{2.501005in}}{\pgfqpoint{9.657164in}{2.512055in}}%
\pgfpathcurveto{\pgfqpoint{9.657164in}{2.523105in}}{\pgfqpoint{9.652774in}{2.533704in}}{\pgfqpoint{9.644960in}{2.541518in}}%
\pgfpathcurveto{\pgfqpoint{9.637147in}{2.549331in}}{\pgfqpoint{9.626548in}{2.553721in}}{\pgfqpoint{9.615497in}{2.553721in}}%
\pgfpathcurveto{\pgfqpoint{9.604447in}{2.553721in}}{\pgfqpoint{9.593848in}{2.549331in}}{\pgfqpoint{9.586035in}{2.541518in}}%
\pgfpathcurveto{\pgfqpoint{9.578221in}{2.533704in}}{\pgfqpoint{9.573831in}{2.523105in}}{\pgfqpoint{9.573831in}{2.512055in}}%
\pgfpathcurveto{\pgfqpoint{9.573831in}{2.501005in}}{\pgfqpoint{9.578221in}{2.490406in}}{\pgfqpoint{9.586035in}{2.482592in}}%
\pgfpathcurveto{\pgfqpoint{9.593848in}{2.474778in}}{\pgfqpoint{9.604447in}{2.470388in}}{\pgfqpoint{9.615497in}{2.470388in}}%
\pgfpathlineto{\pgfqpoint{9.615497in}{2.470388in}}%
\pgfpathclose%
\pgfusepath{stroke}%
\end{pgfscope}%
\begin{pgfscope}%
\pgfpathrectangle{\pgfqpoint{7.394209in}{0.375000in}}{\pgfqpoint{6.356833in}{5.175000in}}%
\pgfusepath{clip}%
\pgfsetbuttcap%
\pgfsetroundjoin%
\pgfsetlinewidth{1.003750pt}%
\definecolor{currentstroke}{rgb}{0.827451,0.827451,0.827451}%
\pgfsetstrokecolor{currentstroke}%
\pgfsetdash{}{0pt}%
\pgfpathmoveto{\pgfqpoint{8.871232in}{2.782517in}}%
\pgfpathcurveto{\pgfqpoint{8.882283in}{2.782517in}}{\pgfqpoint{8.892882in}{2.786907in}}{\pgfqpoint{8.900695in}{2.794721in}}%
\pgfpathcurveto{\pgfqpoint{8.908509in}{2.802535in}}{\pgfqpoint{8.912899in}{2.813134in}}{\pgfqpoint{8.912899in}{2.824184in}}%
\pgfpathcurveto{\pgfqpoint{8.912899in}{2.835234in}}{\pgfqpoint{8.908509in}{2.845833in}}{\pgfqpoint{8.900695in}{2.853647in}}%
\pgfpathcurveto{\pgfqpoint{8.892882in}{2.861460in}}{\pgfqpoint{8.882283in}{2.865851in}}{\pgfqpoint{8.871232in}{2.865851in}}%
\pgfpathcurveto{\pgfqpoint{8.860182in}{2.865851in}}{\pgfqpoint{8.849583in}{2.861460in}}{\pgfqpoint{8.841770in}{2.853647in}}%
\pgfpathcurveto{\pgfqpoint{8.833956in}{2.845833in}}{\pgfqpoint{8.829566in}{2.835234in}}{\pgfqpoint{8.829566in}{2.824184in}}%
\pgfpathcurveto{\pgfqpoint{8.829566in}{2.813134in}}{\pgfqpoint{8.833956in}{2.802535in}}{\pgfqpoint{8.841770in}{2.794721in}}%
\pgfpathcurveto{\pgfqpoint{8.849583in}{2.786907in}}{\pgfqpoint{8.860182in}{2.782517in}}{\pgfqpoint{8.871232in}{2.782517in}}%
\pgfpathlineto{\pgfqpoint{8.871232in}{2.782517in}}%
\pgfpathclose%
\pgfusepath{stroke}%
\end{pgfscope}%
\begin{pgfscope}%
\pgfpathrectangle{\pgfqpoint{7.394209in}{0.375000in}}{\pgfqpoint{6.356833in}{5.175000in}}%
\pgfusepath{clip}%
\pgfsetbuttcap%
\pgfsetroundjoin%
\pgfsetlinewidth{1.003750pt}%
\definecolor{currentstroke}{rgb}{0.827451,0.827451,0.827451}%
\pgfsetstrokecolor{currentstroke}%
\pgfsetdash{}{0pt}%
\pgfpathmoveto{\pgfqpoint{10.491822in}{5.255210in}}%
\pgfpathcurveto{\pgfqpoint{10.502872in}{5.255210in}}{\pgfqpoint{10.513471in}{5.259600in}}{\pgfqpoint{10.521285in}{5.267414in}}%
\pgfpathcurveto{\pgfqpoint{10.529099in}{5.275227in}}{\pgfqpoint{10.533489in}{5.285827in}}{\pgfqpoint{10.533489in}{5.296877in}}%
\pgfpathcurveto{\pgfqpoint{10.533489in}{5.307927in}}{\pgfqpoint{10.529099in}{5.318526in}}{\pgfqpoint{10.521285in}{5.326339in}}%
\pgfpathcurveto{\pgfqpoint{10.513471in}{5.334153in}}{\pgfqpoint{10.502872in}{5.338543in}}{\pgfqpoint{10.491822in}{5.338543in}}%
\pgfpathcurveto{\pgfqpoint{10.480772in}{5.338543in}}{\pgfqpoint{10.470173in}{5.334153in}}{\pgfqpoint{10.462359in}{5.326339in}}%
\pgfpathcurveto{\pgfqpoint{10.454546in}{5.318526in}}{\pgfqpoint{10.450155in}{5.307927in}}{\pgfqpoint{10.450155in}{5.296877in}}%
\pgfpathcurveto{\pgfqpoint{10.450155in}{5.285827in}}{\pgfqpoint{10.454546in}{5.275227in}}{\pgfqpoint{10.462359in}{5.267414in}}%
\pgfpathcurveto{\pgfqpoint{10.470173in}{5.259600in}}{\pgfqpoint{10.480772in}{5.255210in}}{\pgfqpoint{10.491822in}{5.255210in}}%
\pgfpathlineto{\pgfqpoint{10.491822in}{5.255210in}}%
\pgfpathclose%
\pgfusepath{stroke}%
\end{pgfscope}%
\begin{pgfscope}%
\pgfpathrectangle{\pgfqpoint{7.394209in}{0.375000in}}{\pgfqpoint{6.356833in}{5.175000in}}%
\pgfusepath{clip}%
\pgfsetbuttcap%
\pgfsetroundjoin%
\pgfsetlinewidth{1.003750pt}%
\definecolor{currentstroke}{rgb}{0.827451,0.827451,0.827451}%
\pgfsetstrokecolor{currentstroke}%
\pgfsetdash{}{0pt}%
\pgfpathmoveto{\pgfqpoint{8.425418in}{2.366936in}}%
\pgfpathcurveto{\pgfqpoint{8.436468in}{2.366936in}}{\pgfqpoint{8.447067in}{2.371326in}}{\pgfqpoint{8.454881in}{2.379140in}}%
\pgfpathcurveto{\pgfqpoint{8.462694in}{2.386953in}}{\pgfqpoint{8.467085in}{2.397552in}}{\pgfqpoint{8.467085in}{2.408602in}}%
\pgfpathcurveto{\pgfqpoint{8.467085in}{2.419652in}}{\pgfqpoint{8.462694in}{2.430252in}}{\pgfqpoint{8.454881in}{2.438065in}}%
\pgfpathcurveto{\pgfqpoint{8.447067in}{2.445879in}}{\pgfqpoint{8.436468in}{2.450269in}}{\pgfqpoint{8.425418in}{2.450269in}}%
\pgfpathcurveto{\pgfqpoint{8.414368in}{2.450269in}}{\pgfqpoint{8.403769in}{2.445879in}}{\pgfqpoint{8.395955in}{2.438065in}}%
\pgfpathcurveto{\pgfqpoint{8.388142in}{2.430252in}}{\pgfqpoint{8.383751in}{2.419652in}}{\pgfqpoint{8.383751in}{2.408602in}}%
\pgfpathcurveto{\pgfqpoint{8.383751in}{2.397552in}}{\pgfqpoint{8.388142in}{2.386953in}}{\pgfqpoint{8.395955in}{2.379140in}}%
\pgfpathcurveto{\pgfqpoint{8.403769in}{2.371326in}}{\pgfqpoint{8.414368in}{2.366936in}}{\pgfqpoint{8.425418in}{2.366936in}}%
\pgfpathlineto{\pgfqpoint{8.425418in}{2.366936in}}%
\pgfpathclose%
\pgfusepath{stroke}%
\end{pgfscope}%
\begin{pgfscope}%
\pgfpathrectangle{\pgfqpoint{7.394209in}{0.375000in}}{\pgfqpoint{6.356833in}{5.175000in}}%
\pgfusepath{clip}%
\pgfsetbuttcap%
\pgfsetroundjoin%
\pgfsetlinewidth{1.003750pt}%
\definecolor{currentstroke}{rgb}{0.827451,0.827451,0.827451}%
\pgfsetstrokecolor{currentstroke}%
\pgfsetdash{}{0pt}%
\pgfpathmoveto{\pgfqpoint{9.947095in}{4.300308in}}%
\pgfpathcurveto{\pgfqpoint{9.958146in}{4.300308in}}{\pgfqpoint{9.968745in}{4.304699in}}{\pgfqpoint{9.976558in}{4.312512in}}%
\pgfpathcurveto{\pgfqpoint{9.984372in}{4.320326in}}{\pgfqpoint{9.988762in}{4.330925in}}{\pgfqpoint{9.988762in}{4.341975in}}%
\pgfpathcurveto{\pgfqpoint{9.988762in}{4.353025in}}{\pgfqpoint{9.984372in}{4.363624in}}{\pgfqpoint{9.976558in}{4.371438in}}%
\pgfpathcurveto{\pgfqpoint{9.968745in}{4.379252in}}{\pgfqpoint{9.958146in}{4.383642in}}{\pgfqpoint{9.947095in}{4.383642in}}%
\pgfpathcurveto{\pgfqpoint{9.936045in}{4.383642in}}{\pgfqpoint{9.925446in}{4.379252in}}{\pgfqpoint{9.917633in}{4.371438in}}%
\pgfpathcurveto{\pgfqpoint{9.909819in}{4.363624in}}{\pgfqpoint{9.905429in}{4.353025in}}{\pgfqpoint{9.905429in}{4.341975in}}%
\pgfpathcurveto{\pgfqpoint{9.905429in}{4.330925in}}{\pgfqpoint{9.909819in}{4.320326in}}{\pgfqpoint{9.917633in}{4.312512in}}%
\pgfpathcurveto{\pgfqpoint{9.925446in}{4.304699in}}{\pgfqpoint{9.936045in}{4.300308in}}{\pgfqpoint{9.947095in}{4.300308in}}%
\pgfpathlineto{\pgfqpoint{9.947095in}{4.300308in}}%
\pgfpathclose%
\pgfusepath{stroke}%
\end{pgfscope}%
\begin{pgfscope}%
\pgfpathrectangle{\pgfqpoint{7.394209in}{0.375000in}}{\pgfqpoint{6.356833in}{5.175000in}}%
\pgfusepath{clip}%
\pgfsetbuttcap%
\pgfsetroundjoin%
\pgfsetlinewidth{1.003750pt}%
\definecolor{currentstroke}{rgb}{0.827451,0.827451,0.827451}%
\pgfsetstrokecolor{currentstroke}%
\pgfsetdash{}{0pt}%
\pgfpathmoveto{\pgfqpoint{8.516488in}{1.290099in}}%
\pgfpathcurveto{\pgfqpoint{8.527538in}{1.290099in}}{\pgfqpoint{8.538137in}{1.294489in}}{\pgfqpoint{8.545951in}{1.302303in}}%
\pgfpathcurveto{\pgfqpoint{8.553764in}{1.310117in}}{\pgfqpoint{8.558155in}{1.320716in}}{\pgfqpoint{8.558155in}{1.331766in}}%
\pgfpathcurveto{\pgfqpoint{8.558155in}{1.342816in}}{\pgfqpoint{8.553764in}{1.353415in}}{\pgfqpoint{8.545951in}{1.361229in}}%
\pgfpathcurveto{\pgfqpoint{8.538137in}{1.369042in}}{\pgfqpoint{8.527538in}{1.373432in}}{\pgfqpoint{8.516488in}{1.373432in}}%
\pgfpathcurveto{\pgfqpoint{8.505438in}{1.373432in}}{\pgfqpoint{8.494839in}{1.369042in}}{\pgfqpoint{8.487025in}{1.361229in}}%
\pgfpathcurveto{\pgfqpoint{8.479211in}{1.353415in}}{\pgfqpoint{8.474821in}{1.342816in}}{\pgfqpoint{8.474821in}{1.331766in}}%
\pgfpathcurveto{\pgfqpoint{8.474821in}{1.320716in}}{\pgfqpoint{8.479211in}{1.310117in}}{\pgfqpoint{8.487025in}{1.302303in}}%
\pgfpathcurveto{\pgfqpoint{8.494839in}{1.294489in}}{\pgfqpoint{8.505438in}{1.290099in}}{\pgfqpoint{8.516488in}{1.290099in}}%
\pgfpathlineto{\pgfqpoint{8.516488in}{1.290099in}}%
\pgfpathclose%
\pgfusepath{stroke}%
\end{pgfscope}%
\begin{pgfscope}%
\pgfpathrectangle{\pgfqpoint{7.394209in}{0.375000in}}{\pgfqpoint{6.356833in}{5.175000in}}%
\pgfusepath{clip}%
\pgfsetbuttcap%
\pgfsetroundjoin%
\pgfsetlinewidth{1.003750pt}%
\definecolor{currentstroke}{rgb}{0.827451,0.827451,0.827451}%
\pgfsetstrokecolor{currentstroke}%
\pgfsetdash{}{0pt}%
\pgfpathmoveto{\pgfqpoint{10.248406in}{3.624431in}}%
\pgfpathcurveto{\pgfqpoint{10.259456in}{3.624431in}}{\pgfqpoint{10.270055in}{3.628821in}}{\pgfqpoint{10.277869in}{3.636635in}}%
\pgfpathcurveto{\pgfqpoint{10.285682in}{3.644448in}}{\pgfqpoint{10.290072in}{3.655047in}}{\pgfqpoint{10.290072in}{3.666097in}}%
\pgfpathcurveto{\pgfqpoint{10.290072in}{3.677148in}}{\pgfqpoint{10.285682in}{3.687747in}}{\pgfqpoint{10.277869in}{3.695560in}}%
\pgfpathcurveto{\pgfqpoint{10.270055in}{3.703374in}}{\pgfqpoint{10.259456in}{3.707764in}}{\pgfqpoint{10.248406in}{3.707764in}}%
\pgfpathcurveto{\pgfqpoint{10.237356in}{3.707764in}}{\pgfqpoint{10.226757in}{3.703374in}}{\pgfqpoint{10.218943in}{3.695560in}}%
\pgfpathcurveto{\pgfqpoint{10.211129in}{3.687747in}}{\pgfqpoint{10.206739in}{3.677148in}}{\pgfqpoint{10.206739in}{3.666097in}}%
\pgfpathcurveto{\pgfqpoint{10.206739in}{3.655047in}}{\pgfqpoint{10.211129in}{3.644448in}}{\pgfqpoint{10.218943in}{3.636635in}}%
\pgfpathcurveto{\pgfqpoint{10.226757in}{3.628821in}}{\pgfqpoint{10.237356in}{3.624431in}}{\pgfqpoint{10.248406in}{3.624431in}}%
\pgfpathlineto{\pgfqpoint{10.248406in}{3.624431in}}%
\pgfpathclose%
\pgfusepath{stroke}%
\end{pgfscope}%
\begin{pgfscope}%
\pgfpathrectangle{\pgfqpoint{7.394209in}{0.375000in}}{\pgfqpoint{6.356833in}{5.175000in}}%
\pgfusepath{clip}%
\pgfsetbuttcap%
\pgfsetroundjoin%
\pgfsetlinewidth{1.003750pt}%
\definecolor{currentstroke}{rgb}{0.827451,0.827451,0.827451}%
\pgfsetstrokecolor{currentstroke}%
\pgfsetdash{}{0pt}%
\pgfpathmoveto{\pgfqpoint{7.922670in}{1.535099in}}%
\pgfpathcurveto{\pgfqpoint{7.933720in}{1.535099in}}{\pgfqpoint{7.944319in}{1.539489in}}{\pgfqpoint{7.952133in}{1.547303in}}%
\pgfpathcurveto{\pgfqpoint{7.959947in}{1.555117in}}{\pgfqpoint{7.964337in}{1.565716in}}{\pgfqpoint{7.964337in}{1.576766in}}%
\pgfpathcurveto{\pgfqpoint{7.964337in}{1.587816in}}{\pgfqpoint{7.959947in}{1.598415in}}{\pgfqpoint{7.952133in}{1.606229in}}%
\pgfpathcurveto{\pgfqpoint{7.944319in}{1.614042in}}{\pgfqpoint{7.933720in}{1.618432in}}{\pgfqpoint{7.922670in}{1.618432in}}%
\pgfpathcurveto{\pgfqpoint{7.911620in}{1.618432in}}{\pgfqpoint{7.901021in}{1.614042in}}{\pgfqpoint{7.893207in}{1.606229in}}%
\pgfpathcurveto{\pgfqpoint{7.885394in}{1.598415in}}{\pgfqpoint{7.881004in}{1.587816in}}{\pgfqpoint{7.881004in}{1.576766in}}%
\pgfpathcurveto{\pgfqpoint{7.881004in}{1.565716in}}{\pgfqpoint{7.885394in}{1.555117in}}{\pgfqpoint{7.893207in}{1.547303in}}%
\pgfpathcurveto{\pgfqpoint{7.901021in}{1.539489in}}{\pgfqpoint{7.911620in}{1.535099in}}{\pgfqpoint{7.922670in}{1.535099in}}%
\pgfpathlineto{\pgfqpoint{7.922670in}{1.535099in}}%
\pgfpathclose%
\pgfusepath{stroke}%
\end{pgfscope}%
\begin{pgfscope}%
\pgfpathrectangle{\pgfqpoint{7.394209in}{0.375000in}}{\pgfqpoint{6.356833in}{5.175000in}}%
\pgfusepath{clip}%
\pgfsetbuttcap%
\pgfsetroundjoin%
\pgfsetlinewidth{1.003750pt}%
\definecolor{currentstroke}{rgb}{0.827451,0.827451,0.827451}%
\pgfsetstrokecolor{currentstroke}%
\pgfsetdash{}{0pt}%
\pgfpathmoveto{\pgfqpoint{8.417072in}{2.413470in}}%
\pgfpathcurveto{\pgfqpoint{8.428122in}{2.413470in}}{\pgfqpoint{8.438721in}{2.417861in}}{\pgfqpoint{8.446535in}{2.425674in}}%
\pgfpathcurveto{\pgfqpoint{8.454348in}{2.433488in}}{\pgfqpoint{8.458739in}{2.444087in}}{\pgfqpoint{8.458739in}{2.455137in}}%
\pgfpathcurveto{\pgfqpoint{8.458739in}{2.466187in}}{\pgfqpoint{8.454348in}{2.476786in}}{\pgfqpoint{8.446535in}{2.484600in}}%
\pgfpathcurveto{\pgfqpoint{8.438721in}{2.492413in}}{\pgfqpoint{8.428122in}{2.496804in}}{\pgfqpoint{8.417072in}{2.496804in}}%
\pgfpathcurveto{\pgfqpoint{8.406022in}{2.496804in}}{\pgfqpoint{8.395423in}{2.492413in}}{\pgfqpoint{8.387609in}{2.484600in}}%
\pgfpathcurveto{\pgfqpoint{8.379795in}{2.476786in}}{\pgfqpoint{8.375405in}{2.466187in}}{\pgfqpoint{8.375405in}{2.455137in}}%
\pgfpathcurveto{\pgfqpoint{8.375405in}{2.444087in}}{\pgfqpoint{8.379795in}{2.433488in}}{\pgfqpoint{8.387609in}{2.425674in}}%
\pgfpathcurveto{\pgfqpoint{8.395423in}{2.417861in}}{\pgfqpoint{8.406022in}{2.413470in}}{\pgfqpoint{8.417072in}{2.413470in}}%
\pgfpathlineto{\pgfqpoint{8.417072in}{2.413470in}}%
\pgfpathclose%
\pgfusepath{stroke}%
\end{pgfscope}%
\begin{pgfscope}%
\pgfpathrectangle{\pgfqpoint{7.394209in}{0.375000in}}{\pgfqpoint{6.356833in}{5.175000in}}%
\pgfusepath{clip}%
\pgfsetbuttcap%
\pgfsetroundjoin%
\pgfsetlinewidth{1.003750pt}%
\definecolor{currentstroke}{rgb}{0.827451,0.827451,0.827451}%
\pgfsetstrokecolor{currentstroke}%
\pgfsetdash{}{0pt}%
\pgfpathmoveto{\pgfqpoint{11.428820in}{5.167453in}}%
\pgfpathcurveto{\pgfqpoint{11.439870in}{5.167453in}}{\pgfqpoint{11.450469in}{5.171843in}}{\pgfqpoint{11.458283in}{5.179657in}}%
\pgfpathcurveto{\pgfqpoint{11.466097in}{5.187470in}}{\pgfqpoint{11.470487in}{5.198069in}}{\pgfqpoint{11.470487in}{5.209119in}}%
\pgfpathcurveto{\pgfqpoint{11.470487in}{5.220170in}}{\pgfqpoint{11.466097in}{5.230769in}}{\pgfqpoint{11.458283in}{5.238582in}}%
\pgfpathcurveto{\pgfqpoint{11.450469in}{5.246396in}}{\pgfqpoint{11.439870in}{5.250786in}}{\pgfqpoint{11.428820in}{5.250786in}}%
\pgfpathcurveto{\pgfqpoint{11.417770in}{5.250786in}}{\pgfqpoint{11.407171in}{5.246396in}}{\pgfqpoint{11.399357in}{5.238582in}}%
\pgfpathcurveto{\pgfqpoint{11.391544in}{5.230769in}}{\pgfqpoint{11.387154in}{5.220170in}}{\pgfqpoint{11.387154in}{5.209119in}}%
\pgfpathcurveto{\pgfqpoint{11.387154in}{5.198069in}}{\pgfqpoint{11.391544in}{5.187470in}}{\pgfqpoint{11.399357in}{5.179657in}}%
\pgfpathcurveto{\pgfqpoint{11.407171in}{5.171843in}}{\pgfqpoint{11.417770in}{5.167453in}}{\pgfqpoint{11.428820in}{5.167453in}}%
\pgfpathlineto{\pgfqpoint{11.428820in}{5.167453in}}%
\pgfpathclose%
\pgfusepath{stroke}%
\end{pgfscope}%
\begin{pgfscope}%
\pgfpathrectangle{\pgfqpoint{7.394209in}{0.375000in}}{\pgfqpoint{6.356833in}{5.175000in}}%
\pgfusepath{clip}%
\pgfsetbuttcap%
\pgfsetroundjoin%
\pgfsetlinewidth{1.003750pt}%
\definecolor{currentstroke}{rgb}{0.827451,0.827451,0.827451}%
\pgfsetstrokecolor{currentstroke}%
\pgfsetdash{}{0pt}%
\pgfpathmoveto{\pgfqpoint{8.880062in}{1.617250in}}%
\pgfpathcurveto{\pgfqpoint{8.891112in}{1.617250in}}{\pgfqpoint{8.901711in}{1.621641in}}{\pgfqpoint{8.909524in}{1.629454in}}%
\pgfpathcurveto{\pgfqpoint{8.917338in}{1.637268in}}{\pgfqpoint{8.921728in}{1.647867in}}{\pgfqpoint{8.921728in}{1.658917in}}%
\pgfpathcurveto{\pgfqpoint{8.921728in}{1.669967in}}{\pgfqpoint{8.917338in}{1.680566in}}{\pgfqpoint{8.909524in}{1.688380in}}%
\pgfpathcurveto{\pgfqpoint{8.901711in}{1.696193in}}{\pgfqpoint{8.891112in}{1.700584in}}{\pgfqpoint{8.880062in}{1.700584in}}%
\pgfpathcurveto{\pgfqpoint{8.869011in}{1.700584in}}{\pgfqpoint{8.858412in}{1.696193in}}{\pgfqpoint{8.850599in}{1.688380in}}%
\pgfpathcurveto{\pgfqpoint{8.842785in}{1.680566in}}{\pgfqpoint{8.838395in}{1.669967in}}{\pgfqpoint{8.838395in}{1.658917in}}%
\pgfpathcurveto{\pgfqpoint{8.838395in}{1.647867in}}{\pgfqpoint{8.842785in}{1.637268in}}{\pgfqpoint{8.850599in}{1.629454in}}%
\pgfpathcurveto{\pgfqpoint{8.858412in}{1.621641in}}{\pgfqpoint{8.869011in}{1.617250in}}{\pgfqpoint{8.880062in}{1.617250in}}%
\pgfpathlineto{\pgfqpoint{8.880062in}{1.617250in}}%
\pgfpathclose%
\pgfusepath{stroke}%
\end{pgfscope}%
\begin{pgfscope}%
\pgfpathrectangle{\pgfqpoint{7.394209in}{0.375000in}}{\pgfqpoint{6.356833in}{5.175000in}}%
\pgfusepath{clip}%
\pgfsetbuttcap%
\pgfsetroundjoin%
\pgfsetlinewidth{1.003750pt}%
\definecolor{currentstroke}{rgb}{0.827451,0.827451,0.827451}%
\pgfsetstrokecolor{currentstroke}%
\pgfsetdash{}{0pt}%
\pgfpathmoveto{\pgfqpoint{9.613951in}{2.392628in}}%
\pgfpathcurveto{\pgfqpoint{9.625001in}{2.392628in}}{\pgfqpoint{9.635601in}{2.397018in}}{\pgfqpoint{9.643414in}{2.404831in}}%
\pgfpathcurveto{\pgfqpoint{9.651228in}{2.412645in}}{\pgfqpoint{9.655618in}{2.423244in}}{\pgfqpoint{9.655618in}{2.434294in}}%
\pgfpathcurveto{\pgfqpoint{9.655618in}{2.445344in}}{\pgfqpoint{9.651228in}{2.455943in}}{\pgfqpoint{9.643414in}{2.463757in}}%
\pgfpathcurveto{\pgfqpoint{9.635601in}{2.471571in}}{\pgfqpoint{9.625001in}{2.475961in}}{\pgfqpoint{9.613951in}{2.475961in}}%
\pgfpathcurveto{\pgfqpoint{9.602901in}{2.475961in}}{\pgfqpoint{9.592302in}{2.471571in}}{\pgfqpoint{9.584489in}{2.463757in}}%
\pgfpathcurveto{\pgfqpoint{9.576675in}{2.455943in}}{\pgfqpoint{9.572285in}{2.445344in}}{\pgfqpoint{9.572285in}{2.434294in}}%
\pgfpathcurveto{\pgfqpoint{9.572285in}{2.423244in}}{\pgfqpoint{9.576675in}{2.412645in}}{\pgfqpoint{9.584489in}{2.404831in}}%
\pgfpathcurveto{\pgfqpoint{9.592302in}{2.397018in}}{\pgfqpoint{9.602901in}{2.392628in}}{\pgfqpoint{9.613951in}{2.392628in}}%
\pgfpathlineto{\pgfqpoint{9.613951in}{2.392628in}}%
\pgfpathclose%
\pgfusepath{stroke}%
\end{pgfscope}%
\begin{pgfscope}%
\pgfpathrectangle{\pgfqpoint{7.394209in}{0.375000in}}{\pgfqpoint{6.356833in}{5.175000in}}%
\pgfusepath{clip}%
\pgfsetbuttcap%
\pgfsetroundjoin%
\pgfsetlinewidth{1.003750pt}%
\definecolor{currentstroke}{rgb}{0.827451,0.827451,0.827451}%
\pgfsetstrokecolor{currentstroke}%
\pgfsetdash{}{0pt}%
\pgfpathmoveto{\pgfqpoint{8.490275in}{1.292819in}}%
\pgfpathcurveto{\pgfqpoint{8.501325in}{1.292819in}}{\pgfqpoint{8.511924in}{1.297209in}}{\pgfqpoint{8.519738in}{1.305023in}}%
\pgfpathcurveto{\pgfqpoint{8.527551in}{1.312837in}}{\pgfqpoint{8.531941in}{1.323436in}}{\pgfqpoint{8.531941in}{1.334486in}}%
\pgfpathcurveto{\pgfqpoint{8.531941in}{1.345536in}}{\pgfqpoint{8.527551in}{1.356135in}}{\pgfqpoint{8.519738in}{1.363949in}}%
\pgfpathcurveto{\pgfqpoint{8.511924in}{1.371762in}}{\pgfqpoint{8.501325in}{1.376153in}}{\pgfqpoint{8.490275in}{1.376153in}}%
\pgfpathcurveto{\pgfqpoint{8.479225in}{1.376153in}}{\pgfqpoint{8.468626in}{1.371762in}}{\pgfqpoint{8.460812in}{1.363949in}}%
\pgfpathcurveto{\pgfqpoint{8.452998in}{1.356135in}}{\pgfqpoint{8.448608in}{1.345536in}}{\pgfqpoint{8.448608in}{1.334486in}}%
\pgfpathcurveto{\pgfqpoint{8.448608in}{1.323436in}}{\pgfqpoint{8.452998in}{1.312837in}}{\pgfqpoint{8.460812in}{1.305023in}}%
\pgfpathcurveto{\pgfqpoint{8.468626in}{1.297209in}}{\pgfqpoint{8.479225in}{1.292819in}}{\pgfqpoint{8.490275in}{1.292819in}}%
\pgfpathlineto{\pgfqpoint{8.490275in}{1.292819in}}%
\pgfpathclose%
\pgfusepath{stroke}%
\end{pgfscope}%
\begin{pgfscope}%
\pgfpathrectangle{\pgfqpoint{7.394209in}{0.375000in}}{\pgfqpoint{6.356833in}{5.175000in}}%
\pgfusepath{clip}%
\pgfsetbuttcap%
\pgfsetroundjoin%
\pgfsetlinewidth{1.003750pt}%
\definecolor{currentstroke}{rgb}{0.827451,0.827451,0.827451}%
\pgfsetstrokecolor{currentstroke}%
\pgfsetdash{}{0pt}%
\pgfpathmoveto{\pgfqpoint{10.203802in}{4.867614in}}%
\pgfpathcurveto{\pgfqpoint{10.214852in}{4.867614in}}{\pgfqpoint{10.225451in}{4.872005in}}{\pgfqpoint{10.233265in}{4.879818in}}%
\pgfpathcurveto{\pgfqpoint{10.241079in}{4.887632in}}{\pgfqpoint{10.245469in}{4.898231in}}{\pgfqpoint{10.245469in}{4.909281in}}%
\pgfpathcurveto{\pgfqpoint{10.245469in}{4.920331in}}{\pgfqpoint{10.241079in}{4.930930in}}{\pgfqpoint{10.233265in}{4.938744in}}%
\pgfpathcurveto{\pgfqpoint{10.225451in}{4.946557in}}{\pgfqpoint{10.214852in}{4.950948in}}{\pgfqpoint{10.203802in}{4.950948in}}%
\pgfpathcurveto{\pgfqpoint{10.192752in}{4.950948in}}{\pgfqpoint{10.182153in}{4.946557in}}{\pgfqpoint{10.174339in}{4.938744in}}%
\pgfpathcurveto{\pgfqpoint{10.166526in}{4.930930in}}{\pgfqpoint{10.162136in}{4.920331in}}{\pgfqpoint{10.162136in}{4.909281in}}%
\pgfpathcurveto{\pgfqpoint{10.162136in}{4.898231in}}{\pgfqpoint{10.166526in}{4.887632in}}{\pgfqpoint{10.174339in}{4.879818in}}%
\pgfpathcurveto{\pgfqpoint{10.182153in}{4.872005in}}{\pgfqpoint{10.192752in}{4.867614in}}{\pgfqpoint{10.203802in}{4.867614in}}%
\pgfpathlineto{\pgfqpoint{10.203802in}{4.867614in}}%
\pgfpathclose%
\pgfusepath{stroke}%
\end{pgfscope}%
\begin{pgfscope}%
\pgfpathrectangle{\pgfqpoint{7.394209in}{0.375000in}}{\pgfqpoint{6.356833in}{5.175000in}}%
\pgfusepath{clip}%
\pgfsetbuttcap%
\pgfsetroundjoin%
\pgfsetlinewidth{1.003750pt}%
\definecolor{currentstroke}{rgb}{0.827451,0.827451,0.827451}%
\pgfsetstrokecolor{currentstroke}%
\pgfsetdash{}{0pt}%
\pgfpathmoveto{\pgfqpoint{10.551105in}{3.776556in}}%
\pgfpathcurveto{\pgfqpoint{10.562155in}{3.776556in}}{\pgfqpoint{10.572754in}{3.780946in}}{\pgfqpoint{10.580567in}{3.788759in}}%
\pgfpathcurveto{\pgfqpoint{10.588381in}{3.796573in}}{\pgfqpoint{10.592771in}{3.807172in}}{\pgfqpoint{10.592771in}{3.818222in}}%
\pgfpathcurveto{\pgfqpoint{10.592771in}{3.829272in}}{\pgfqpoint{10.588381in}{3.839871in}}{\pgfqpoint{10.580567in}{3.847685in}}%
\pgfpathcurveto{\pgfqpoint{10.572754in}{3.855499in}}{\pgfqpoint{10.562155in}{3.859889in}}{\pgfqpoint{10.551105in}{3.859889in}}%
\pgfpathcurveto{\pgfqpoint{10.540054in}{3.859889in}}{\pgfqpoint{10.529455in}{3.855499in}}{\pgfqpoint{10.521642in}{3.847685in}}%
\pgfpathcurveto{\pgfqpoint{10.513828in}{3.839871in}}{\pgfqpoint{10.509438in}{3.829272in}}{\pgfqpoint{10.509438in}{3.818222in}}%
\pgfpathcurveto{\pgfqpoint{10.509438in}{3.807172in}}{\pgfqpoint{10.513828in}{3.796573in}}{\pgfqpoint{10.521642in}{3.788759in}}%
\pgfpathcurveto{\pgfqpoint{10.529455in}{3.780946in}}{\pgfqpoint{10.540054in}{3.776556in}}{\pgfqpoint{10.551105in}{3.776556in}}%
\pgfpathlineto{\pgfqpoint{10.551105in}{3.776556in}}%
\pgfpathclose%
\pgfusepath{stroke}%
\end{pgfscope}%
\begin{pgfscope}%
\pgfpathrectangle{\pgfqpoint{7.394209in}{0.375000in}}{\pgfqpoint{6.356833in}{5.175000in}}%
\pgfusepath{clip}%
\pgfsetbuttcap%
\pgfsetroundjoin%
\pgfsetlinewidth{1.003750pt}%
\definecolor{currentstroke}{rgb}{0.827451,0.827451,0.827451}%
\pgfsetstrokecolor{currentstroke}%
\pgfsetdash{}{0pt}%
\pgfpathmoveto{\pgfqpoint{10.360868in}{3.217116in}}%
\pgfpathcurveto{\pgfqpoint{10.371918in}{3.217116in}}{\pgfqpoint{10.382517in}{3.221506in}}{\pgfqpoint{10.390331in}{3.229320in}}%
\pgfpathcurveto{\pgfqpoint{10.398144in}{3.237133in}}{\pgfqpoint{10.402534in}{3.247732in}}{\pgfqpoint{10.402534in}{3.258782in}}%
\pgfpathcurveto{\pgfqpoint{10.402534in}{3.269833in}}{\pgfqpoint{10.398144in}{3.280432in}}{\pgfqpoint{10.390331in}{3.288245in}}%
\pgfpathcurveto{\pgfqpoint{10.382517in}{3.296059in}}{\pgfqpoint{10.371918in}{3.300449in}}{\pgfqpoint{10.360868in}{3.300449in}}%
\pgfpathcurveto{\pgfqpoint{10.349818in}{3.300449in}}{\pgfqpoint{10.339219in}{3.296059in}}{\pgfqpoint{10.331405in}{3.288245in}}%
\pgfpathcurveto{\pgfqpoint{10.323591in}{3.280432in}}{\pgfqpoint{10.319201in}{3.269833in}}{\pgfqpoint{10.319201in}{3.258782in}}%
\pgfpathcurveto{\pgfqpoint{10.319201in}{3.247732in}}{\pgfqpoint{10.323591in}{3.237133in}}{\pgfqpoint{10.331405in}{3.229320in}}%
\pgfpathcurveto{\pgfqpoint{10.339219in}{3.221506in}}{\pgfqpoint{10.349818in}{3.217116in}}{\pgfqpoint{10.360868in}{3.217116in}}%
\pgfpathlineto{\pgfqpoint{10.360868in}{3.217116in}}%
\pgfpathclose%
\pgfusepath{stroke}%
\end{pgfscope}%
\begin{pgfscope}%
\pgfpathrectangle{\pgfqpoint{7.394209in}{0.375000in}}{\pgfqpoint{6.356833in}{5.175000in}}%
\pgfusepath{clip}%
\pgfsetbuttcap%
\pgfsetroundjoin%
\pgfsetlinewidth{1.003750pt}%
\definecolor{currentstroke}{rgb}{0.827451,0.827451,0.827451}%
\pgfsetstrokecolor{currentstroke}%
\pgfsetdash{}{0pt}%
\pgfpathmoveto{\pgfqpoint{7.500262in}{0.401888in}}%
\pgfpathcurveto{\pgfqpoint{7.511312in}{0.401888in}}{\pgfqpoint{7.521911in}{0.406278in}}{\pgfqpoint{7.529725in}{0.414092in}}%
\pgfpathcurveto{\pgfqpoint{7.537539in}{0.421906in}}{\pgfqpoint{7.541929in}{0.432505in}}{\pgfqpoint{7.541929in}{0.443555in}}%
\pgfpathcurveto{\pgfqpoint{7.541929in}{0.454605in}}{\pgfqpoint{7.537539in}{0.465204in}}{\pgfqpoint{7.529725in}{0.473018in}}%
\pgfpathcurveto{\pgfqpoint{7.521911in}{0.480831in}}{\pgfqpoint{7.511312in}{0.485222in}}{\pgfqpoint{7.500262in}{0.485222in}}%
\pgfpathcurveto{\pgfqpoint{7.489212in}{0.485222in}}{\pgfqpoint{7.478613in}{0.480831in}}{\pgfqpoint{7.470800in}{0.473018in}}%
\pgfpathcurveto{\pgfqpoint{7.462986in}{0.465204in}}{\pgfqpoint{7.458596in}{0.454605in}}{\pgfqpoint{7.458596in}{0.443555in}}%
\pgfpathcurveto{\pgfqpoint{7.458596in}{0.432505in}}{\pgfqpoint{7.462986in}{0.421906in}}{\pgfqpoint{7.470800in}{0.414092in}}%
\pgfpathcurveto{\pgfqpoint{7.478613in}{0.406278in}}{\pgfqpoint{7.489212in}{0.401888in}}{\pgfqpoint{7.500262in}{0.401888in}}%
\pgfpathlineto{\pgfqpoint{7.500262in}{0.401888in}}%
\pgfpathclose%
\pgfusepath{stroke}%
\end{pgfscope}%
\begin{pgfscope}%
\pgfpathrectangle{\pgfqpoint{7.394209in}{0.375000in}}{\pgfqpoint{6.356833in}{5.175000in}}%
\pgfusepath{clip}%
\pgfsetbuttcap%
\pgfsetroundjoin%
\pgfsetlinewidth{1.003750pt}%
\definecolor{currentstroke}{rgb}{0.827451,0.827451,0.827451}%
\pgfsetstrokecolor{currentstroke}%
\pgfsetdash{}{0pt}%
\pgfpathmoveto{\pgfqpoint{8.489229in}{1.156348in}}%
\pgfpathcurveto{\pgfqpoint{8.500279in}{1.156348in}}{\pgfqpoint{8.510878in}{1.160738in}}{\pgfqpoint{8.518692in}{1.168552in}}%
\pgfpathcurveto{\pgfqpoint{8.526505in}{1.176366in}}{\pgfqpoint{8.530895in}{1.186965in}}{\pgfqpoint{8.530895in}{1.198015in}}%
\pgfpathcurveto{\pgfqpoint{8.530895in}{1.209065in}}{\pgfqpoint{8.526505in}{1.219664in}}{\pgfqpoint{8.518692in}{1.227477in}}%
\pgfpathcurveto{\pgfqpoint{8.510878in}{1.235291in}}{\pgfqpoint{8.500279in}{1.239681in}}{\pgfqpoint{8.489229in}{1.239681in}}%
\pgfpathcurveto{\pgfqpoint{8.478179in}{1.239681in}}{\pgfqpoint{8.467580in}{1.235291in}}{\pgfqpoint{8.459766in}{1.227477in}}%
\pgfpathcurveto{\pgfqpoint{8.451952in}{1.219664in}}{\pgfqpoint{8.447562in}{1.209065in}}{\pgfqpoint{8.447562in}{1.198015in}}%
\pgfpathcurveto{\pgfqpoint{8.447562in}{1.186965in}}{\pgfqpoint{8.451952in}{1.176366in}}{\pgfqpoint{8.459766in}{1.168552in}}%
\pgfpathcurveto{\pgfqpoint{8.467580in}{1.160738in}}{\pgfqpoint{8.478179in}{1.156348in}}{\pgfqpoint{8.489229in}{1.156348in}}%
\pgfpathlineto{\pgfqpoint{8.489229in}{1.156348in}}%
\pgfpathclose%
\pgfusepath{stroke}%
\end{pgfscope}%
\begin{pgfscope}%
\pgfpathrectangle{\pgfqpoint{7.394209in}{0.375000in}}{\pgfqpoint{6.356833in}{5.175000in}}%
\pgfusepath{clip}%
\pgfsetbuttcap%
\pgfsetroundjoin%
\pgfsetlinewidth{1.003750pt}%
\definecolor{currentstroke}{rgb}{0.827451,0.827451,0.827451}%
\pgfsetstrokecolor{currentstroke}%
\pgfsetdash{}{0pt}%
\pgfpathmoveto{\pgfqpoint{9.637325in}{2.485896in}}%
\pgfpathcurveto{\pgfqpoint{9.648375in}{2.485896in}}{\pgfqpoint{9.658974in}{2.490287in}}{\pgfqpoint{9.666788in}{2.498100in}}%
\pgfpathcurveto{\pgfqpoint{9.674602in}{2.505914in}}{\pgfqpoint{9.678992in}{2.516513in}}{\pgfqpoint{9.678992in}{2.527563in}}%
\pgfpathcurveto{\pgfqpoint{9.678992in}{2.538613in}}{\pgfqpoint{9.674602in}{2.549212in}}{\pgfqpoint{9.666788in}{2.557026in}}%
\pgfpathcurveto{\pgfqpoint{9.658974in}{2.564839in}}{\pgfqpoint{9.648375in}{2.569230in}}{\pgfqpoint{9.637325in}{2.569230in}}%
\pgfpathcurveto{\pgfqpoint{9.626275in}{2.569230in}}{\pgfqpoint{9.615676in}{2.564839in}}{\pgfqpoint{9.607863in}{2.557026in}}%
\pgfpathcurveto{\pgfqpoint{9.600049in}{2.549212in}}{\pgfqpoint{9.595659in}{2.538613in}}{\pgfqpoint{9.595659in}{2.527563in}}%
\pgfpathcurveto{\pgfqpoint{9.595659in}{2.516513in}}{\pgfqpoint{9.600049in}{2.505914in}}{\pgfqpoint{9.607863in}{2.498100in}}%
\pgfpathcurveto{\pgfqpoint{9.615676in}{2.490287in}}{\pgfqpoint{9.626275in}{2.485896in}}{\pgfqpoint{9.637325in}{2.485896in}}%
\pgfpathlineto{\pgfqpoint{9.637325in}{2.485896in}}%
\pgfpathclose%
\pgfusepath{stroke}%
\end{pgfscope}%
\begin{pgfscope}%
\pgfpathrectangle{\pgfqpoint{7.394209in}{0.375000in}}{\pgfqpoint{6.356833in}{5.175000in}}%
\pgfusepath{clip}%
\pgfsetbuttcap%
\pgfsetroundjoin%
\pgfsetlinewidth{1.003750pt}%
\definecolor{currentstroke}{rgb}{0.827451,0.827451,0.827451}%
\pgfsetstrokecolor{currentstroke}%
\pgfsetdash{}{0pt}%
\pgfpathmoveto{\pgfqpoint{10.184789in}{4.942308in}}%
\pgfpathcurveto{\pgfqpoint{10.195839in}{4.942308in}}{\pgfqpoint{10.206439in}{4.946699in}}{\pgfqpoint{10.214252in}{4.954512in}}%
\pgfpathcurveto{\pgfqpoint{10.222066in}{4.962326in}}{\pgfqpoint{10.226456in}{4.972925in}}{\pgfqpoint{10.226456in}{4.983975in}}%
\pgfpathcurveto{\pgfqpoint{10.226456in}{4.995025in}}{\pgfqpoint{10.222066in}{5.005624in}}{\pgfqpoint{10.214252in}{5.013438in}}%
\pgfpathcurveto{\pgfqpoint{10.206439in}{5.021252in}}{\pgfqpoint{10.195839in}{5.025642in}}{\pgfqpoint{10.184789in}{5.025642in}}%
\pgfpathcurveto{\pgfqpoint{10.173739in}{5.025642in}}{\pgfqpoint{10.163140in}{5.021252in}}{\pgfqpoint{10.155327in}{5.013438in}}%
\pgfpathcurveto{\pgfqpoint{10.147513in}{5.005624in}}{\pgfqpoint{10.143123in}{4.995025in}}{\pgfqpoint{10.143123in}{4.983975in}}%
\pgfpathcurveto{\pgfqpoint{10.143123in}{4.972925in}}{\pgfqpoint{10.147513in}{4.962326in}}{\pgfqpoint{10.155327in}{4.954512in}}%
\pgfpathcurveto{\pgfqpoint{10.163140in}{4.946699in}}{\pgfqpoint{10.173739in}{4.942308in}}{\pgfqpoint{10.184789in}{4.942308in}}%
\pgfpathlineto{\pgfqpoint{10.184789in}{4.942308in}}%
\pgfpathclose%
\pgfusepath{stroke}%
\end{pgfscope}%
\begin{pgfscope}%
\pgfpathrectangle{\pgfqpoint{7.394209in}{0.375000in}}{\pgfqpoint{6.356833in}{5.175000in}}%
\pgfusepath{clip}%
\pgfsetbuttcap%
\pgfsetroundjoin%
\pgfsetlinewidth{1.003750pt}%
\definecolor{currentstroke}{rgb}{0.827451,0.827451,0.827451}%
\pgfsetstrokecolor{currentstroke}%
\pgfsetdash{}{0pt}%
\pgfpathmoveto{\pgfqpoint{11.572131in}{4.858364in}}%
\pgfpathcurveto{\pgfqpoint{11.583181in}{4.858364in}}{\pgfqpoint{11.593780in}{4.862754in}}{\pgfqpoint{11.601593in}{4.870568in}}%
\pgfpathcurveto{\pgfqpoint{11.609407in}{4.878382in}}{\pgfqpoint{11.613797in}{4.888981in}}{\pgfqpoint{11.613797in}{4.900031in}}%
\pgfpathcurveto{\pgfqpoint{11.613797in}{4.911081in}}{\pgfqpoint{11.609407in}{4.921680in}}{\pgfqpoint{11.601593in}{4.929494in}}%
\pgfpathcurveto{\pgfqpoint{11.593780in}{4.937307in}}{\pgfqpoint{11.583181in}{4.941697in}}{\pgfqpoint{11.572131in}{4.941697in}}%
\pgfpathcurveto{\pgfqpoint{11.561080in}{4.941697in}}{\pgfqpoint{11.550481in}{4.937307in}}{\pgfqpoint{11.542668in}{4.929494in}}%
\pgfpathcurveto{\pgfqpoint{11.534854in}{4.921680in}}{\pgfqpoint{11.530464in}{4.911081in}}{\pgfqpoint{11.530464in}{4.900031in}}%
\pgfpathcurveto{\pgfqpoint{11.530464in}{4.888981in}}{\pgfqpoint{11.534854in}{4.878382in}}{\pgfqpoint{11.542668in}{4.870568in}}%
\pgfpathcurveto{\pgfqpoint{11.550481in}{4.862754in}}{\pgfqpoint{11.561080in}{4.858364in}}{\pgfqpoint{11.572131in}{4.858364in}}%
\pgfpathlineto{\pgfqpoint{11.572131in}{4.858364in}}%
\pgfpathclose%
\pgfusepath{stroke}%
\end{pgfscope}%
\begin{pgfscope}%
\pgfpathrectangle{\pgfqpoint{7.394209in}{0.375000in}}{\pgfqpoint{6.356833in}{5.175000in}}%
\pgfusepath{clip}%
\pgfsetbuttcap%
\pgfsetroundjoin%
\pgfsetlinewidth{1.003750pt}%
\definecolor{currentstroke}{rgb}{0.827451,0.827451,0.827451}%
\pgfsetstrokecolor{currentstroke}%
\pgfsetdash{}{0pt}%
\pgfpathmoveto{\pgfqpoint{8.134063in}{0.717554in}}%
\pgfpathcurveto{\pgfqpoint{8.145113in}{0.717554in}}{\pgfqpoint{8.155712in}{0.721944in}}{\pgfqpoint{8.163526in}{0.729757in}}%
\pgfpathcurveto{\pgfqpoint{8.171340in}{0.737571in}}{\pgfqpoint{8.175730in}{0.748170in}}{\pgfqpoint{8.175730in}{0.759220in}}%
\pgfpathcurveto{\pgfqpoint{8.175730in}{0.770270in}}{\pgfqpoint{8.171340in}{0.780869in}}{\pgfqpoint{8.163526in}{0.788683in}}%
\pgfpathcurveto{\pgfqpoint{8.155712in}{0.796497in}}{\pgfqpoint{8.145113in}{0.800887in}}{\pgfqpoint{8.134063in}{0.800887in}}%
\pgfpathcurveto{\pgfqpoint{8.123013in}{0.800887in}}{\pgfqpoint{8.112414in}{0.796497in}}{\pgfqpoint{8.104600in}{0.788683in}}%
\pgfpathcurveto{\pgfqpoint{8.096787in}{0.780869in}}{\pgfqpoint{8.092397in}{0.770270in}}{\pgfqpoint{8.092397in}{0.759220in}}%
\pgfpathcurveto{\pgfqpoint{8.092397in}{0.748170in}}{\pgfqpoint{8.096787in}{0.737571in}}{\pgfqpoint{8.104600in}{0.729757in}}%
\pgfpathcurveto{\pgfqpoint{8.112414in}{0.721944in}}{\pgfqpoint{8.123013in}{0.717554in}}{\pgfqpoint{8.134063in}{0.717554in}}%
\pgfpathlineto{\pgfqpoint{8.134063in}{0.717554in}}%
\pgfpathclose%
\pgfusepath{stroke}%
\end{pgfscope}%
\begin{pgfscope}%
\pgfpathrectangle{\pgfqpoint{7.394209in}{0.375000in}}{\pgfqpoint{6.356833in}{5.175000in}}%
\pgfusepath{clip}%
\pgfsetbuttcap%
\pgfsetroundjoin%
\pgfsetlinewidth{1.003750pt}%
\definecolor{currentstroke}{rgb}{0.827451,0.827451,0.827451}%
\pgfsetstrokecolor{currentstroke}%
\pgfsetdash{}{0pt}%
\pgfpathmoveto{\pgfqpoint{8.769121in}{2.682216in}}%
\pgfpathcurveto{\pgfqpoint{8.780171in}{2.682216in}}{\pgfqpoint{8.790770in}{2.686606in}}{\pgfqpoint{8.798584in}{2.694420in}}%
\pgfpathcurveto{\pgfqpoint{8.806398in}{2.702234in}}{\pgfqpoint{8.810788in}{2.712833in}}{\pgfqpoint{8.810788in}{2.723883in}}%
\pgfpathcurveto{\pgfqpoint{8.810788in}{2.734933in}}{\pgfqpoint{8.806398in}{2.745532in}}{\pgfqpoint{8.798584in}{2.753346in}}%
\pgfpathcurveto{\pgfqpoint{8.790770in}{2.761159in}}{\pgfqpoint{8.780171in}{2.765550in}}{\pgfqpoint{8.769121in}{2.765550in}}%
\pgfpathcurveto{\pgfqpoint{8.758071in}{2.765550in}}{\pgfqpoint{8.747472in}{2.761159in}}{\pgfqpoint{8.739659in}{2.753346in}}%
\pgfpathcurveto{\pgfqpoint{8.731845in}{2.745532in}}{\pgfqpoint{8.727455in}{2.734933in}}{\pgfqpoint{8.727455in}{2.723883in}}%
\pgfpathcurveto{\pgfqpoint{8.727455in}{2.712833in}}{\pgfqpoint{8.731845in}{2.702234in}}{\pgfqpoint{8.739659in}{2.694420in}}%
\pgfpathcurveto{\pgfqpoint{8.747472in}{2.686606in}}{\pgfqpoint{8.758071in}{2.682216in}}{\pgfqpoint{8.769121in}{2.682216in}}%
\pgfpathlineto{\pgfqpoint{8.769121in}{2.682216in}}%
\pgfpathclose%
\pgfusepath{stroke}%
\end{pgfscope}%
\begin{pgfscope}%
\pgfpathrectangle{\pgfqpoint{7.394209in}{0.375000in}}{\pgfqpoint{6.356833in}{5.175000in}}%
\pgfusepath{clip}%
\pgfsetbuttcap%
\pgfsetroundjoin%
\pgfsetlinewidth{1.003750pt}%
\definecolor{currentstroke}{rgb}{0.827451,0.827451,0.827451}%
\pgfsetstrokecolor{currentstroke}%
\pgfsetdash{}{0pt}%
\pgfpathmoveto{\pgfqpoint{10.683353in}{5.433903in}}%
\pgfpathcurveto{\pgfqpoint{10.694403in}{5.433903in}}{\pgfqpoint{10.705002in}{5.438293in}}{\pgfqpoint{10.712815in}{5.446106in}}%
\pgfpathcurveto{\pgfqpoint{10.720629in}{5.453920in}}{\pgfqpoint{10.725019in}{5.464519in}}{\pgfqpoint{10.725019in}{5.475569in}}%
\pgfpathcurveto{\pgfqpoint{10.725019in}{5.486619in}}{\pgfqpoint{10.720629in}{5.497218in}}{\pgfqpoint{10.712815in}{5.505032in}}%
\pgfpathcurveto{\pgfqpoint{10.705002in}{5.512846in}}{\pgfqpoint{10.694403in}{5.517236in}}{\pgfqpoint{10.683353in}{5.517236in}}%
\pgfpathcurveto{\pgfqpoint{10.672303in}{5.517236in}}{\pgfqpoint{10.661704in}{5.512846in}}{\pgfqpoint{10.653890in}{5.505032in}}%
\pgfpathcurveto{\pgfqpoint{10.646076in}{5.497218in}}{\pgfqpoint{10.641686in}{5.486619in}}{\pgfqpoint{10.641686in}{5.475569in}}%
\pgfpathcurveto{\pgfqpoint{10.641686in}{5.464519in}}{\pgfqpoint{10.646076in}{5.453920in}}{\pgfqpoint{10.653890in}{5.446106in}}%
\pgfpathcurveto{\pgfqpoint{10.661704in}{5.438293in}}{\pgfqpoint{10.672303in}{5.433903in}}{\pgfqpoint{10.683353in}{5.433903in}}%
\pgfpathlineto{\pgfqpoint{10.683353in}{5.433903in}}%
\pgfpathclose%
\pgfusepath{stroke}%
\end{pgfscope}%
\begin{pgfscope}%
\pgfpathrectangle{\pgfqpoint{7.394209in}{0.375000in}}{\pgfqpoint{6.356833in}{5.175000in}}%
\pgfusepath{clip}%
\pgfsetbuttcap%
\pgfsetroundjoin%
\pgfsetlinewidth{1.003750pt}%
\definecolor{currentstroke}{rgb}{0.827451,0.827451,0.827451}%
\pgfsetstrokecolor{currentstroke}%
\pgfsetdash{}{0pt}%
\pgfpathmoveto{\pgfqpoint{11.473885in}{4.865695in}}%
\pgfpathcurveto{\pgfqpoint{11.484935in}{4.865695in}}{\pgfqpoint{11.495534in}{4.870085in}}{\pgfqpoint{11.503348in}{4.877899in}}%
\pgfpathcurveto{\pgfqpoint{11.511162in}{4.885713in}}{\pgfqpoint{11.515552in}{4.896312in}}{\pgfqpoint{11.515552in}{4.907362in}}%
\pgfpathcurveto{\pgfqpoint{11.515552in}{4.918412in}}{\pgfqpoint{11.511162in}{4.929011in}}{\pgfqpoint{11.503348in}{4.936824in}}%
\pgfpathcurveto{\pgfqpoint{11.495534in}{4.944638in}}{\pgfqpoint{11.484935in}{4.949028in}}{\pgfqpoint{11.473885in}{4.949028in}}%
\pgfpathcurveto{\pgfqpoint{11.462835in}{4.949028in}}{\pgfqpoint{11.452236in}{4.944638in}}{\pgfqpoint{11.444422in}{4.936824in}}%
\pgfpathcurveto{\pgfqpoint{11.436609in}{4.929011in}}{\pgfqpoint{11.432219in}{4.918412in}}{\pgfqpoint{11.432219in}{4.907362in}}%
\pgfpathcurveto{\pgfqpoint{11.432219in}{4.896312in}}{\pgfqpoint{11.436609in}{4.885713in}}{\pgfqpoint{11.444422in}{4.877899in}}%
\pgfpathcurveto{\pgfqpoint{11.452236in}{4.870085in}}{\pgfqpoint{11.462835in}{4.865695in}}{\pgfqpoint{11.473885in}{4.865695in}}%
\pgfpathlineto{\pgfqpoint{11.473885in}{4.865695in}}%
\pgfpathclose%
\pgfusepath{stroke}%
\end{pgfscope}%
\begin{pgfscope}%
\pgfpathrectangle{\pgfqpoint{7.394209in}{0.375000in}}{\pgfqpoint{6.356833in}{5.175000in}}%
\pgfusepath{clip}%
\pgfsetbuttcap%
\pgfsetroundjoin%
\pgfsetlinewidth{1.003750pt}%
\definecolor{currentstroke}{rgb}{0.827451,0.827451,0.827451}%
\pgfsetstrokecolor{currentstroke}%
\pgfsetdash{}{0pt}%
\pgfpathmoveto{\pgfqpoint{7.963791in}{0.717554in}}%
\pgfpathcurveto{\pgfqpoint{7.974841in}{0.717554in}}{\pgfqpoint{7.985440in}{0.721944in}}{\pgfqpoint{7.993254in}{0.729757in}}%
\pgfpathcurveto{\pgfqpoint{8.001068in}{0.737571in}}{\pgfqpoint{8.005458in}{0.748170in}}{\pgfqpoint{8.005458in}{0.759220in}}%
\pgfpathcurveto{\pgfqpoint{8.005458in}{0.770270in}}{\pgfqpoint{8.001068in}{0.780869in}}{\pgfqpoint{7.993254in}{0.788683in}}%
\pgfpathcurveto{\pgfqpoint{7.985440in}{0.796497in}}{\pgfqpoint{7.974841in}{0.800887in}}{\pgfqpoint{7.963791in}{0.800887in}}%
\pgfpathcurveto{\pgfqpoint{7.952741in}{0.800887in}}{\pgfqpoint{7.942142in}{0.796497in}}{\pgfqpoint{7.934328in}{0.788683in}}%
\pgfpathcurveto{\pgfqpoint{7.926515in}{0.780869in}}{\pgfqpoint{7.922125in}{0.770270in}}{\pgfqpoint{7.922125in}{0.759220in}}%
\pgfpathcurveto{\pgfqpoint{7.922125in}{0.748170in}}{\pgfqpoint{7.926515in}{0.737571in}}{\pgfqpoint{7.934328in}{0.729757in}}%
\pgfpathcurveto{\pgfqpoint{7.942142in}{0.721944in}}{\pgfqpoint{7.952741in}{0.717554in}}{\pgfqpoint{7.963791in}{0.717554in}}%
\pgfpathlineto{\pgfqpoint{7.963791in}{0.717554in}}%
\pgfpathclose%
\pgfusepath{stroke}%
\end{pgfscope}%
\begin{pgfscope}%
\pgfpathrectangle{\pgfqpoint{7.394209in}{0.375000in}}{\pgfqpoint{6.356833in}{5.175000in}}%
\pgfusepath{clip}%
\pgfsetbuttcap%
\pgfsetroundjoin%
\pgfsetlinewidth{1.003750pt}%
\definecolor{currentstroke}{rgb}{0.827451,0.827451,0.827451}%
\pgfsetstrokecolor{currentstroke}%
\pgfsetdash{}{0pt}%
\pgfpathmoveto{\pgfqpoint{8.614667in}{2.874355in}}%
\pgfpathcurveto{\pgfqpoint{8.625717in}{2.874355in}}{\pgfqpoint{8.636316in}{2.878745in}}{\pgfqpoint{8.644129in}{2.886559in}}%
\pgfpathcurveto{\pgfqpoint{8.651943in}{2.894373in}}{\pgfqpoint{8.656333in}{2.904972in}}{\pgfqpoint{8.656333in}{2.916022in}}%
\pgfpathcurveto{\pgfqpoint{8.656333in}{2.927072in}}{\pgfqpoint{8.651943in}{2.937671in}}{\pgfqpoint{8.644129in}{2.945485in}}%
\pgfpathcurveto{\pgfqpoint{8.636316in}{2.953298in}}{\pgfqpoint{8.625717in}{2.957689in}}{\pgfqpoint{8.614667in}{2.957689in}}%
\pgfpathcurveto{\pgfqpoint{8.603616in}{2.957689in}}{\pgfqpoint{8.593017in}{2.953298in}}{\pgfqpoint{8.585204in}{2.945485in}}%
\pgfpathcurveto{\pgfqpoint{8.577390in}{2.937671in}}{\pgfqpoint{8.573000in}{2.927072in}}{\pgfqpoint{8.573000in}{2.916022in}}%
\pgfpathcurveto{\pgfqpoint{8.573000in}{2.904972in}}{\pgfqpoint{8.577390in}{2.894373in}}{\pgfqpoint{8.585204in}{2.886559in}}%
\pgfpathcurveto{\pgfqpoint{8.593017in}{2.878745in}}{\pgfqpoint{8.603616in}{2.874355in}}{\pgfqpoint{8.614667in}{2.874355in}}%
\pgfpathlineto{\pgfqpoint{8.614667in}{2.874355in}}%
\pgfpathclose%
\pgfusepath{stroke}%
\end{pgfscope}%
\begin{pgfscope}%
\pgfpathrectangle{\pgfqpoint{7.394209in}{0.375000in}}{\pgfqpoint{6.356833in}{5.175000in}}%
\pgfusepath{clip}%
\pgfsetbuttcap%
\pgfsetroundjoin%
\pgfsetlinewidth{1.003750pt}%
\definecolor{currentstroke}{rgb}{0.827451,0.827451,0.827451}%
\pgfsetstrokecolor{currentstroke}%
\pgfsetdash{}{0pt}%
\pgfpathmoveto{\pgfqpoint{9.413549in}{3.574677in}}%
\pgfpathcurveto{\pgfqpoint{9.424599in}{3.574677in}}{\pgfqpoint{9.435198in}{3.579068in}}{\pgfqpoint{9.443012in}{3.586881in}}%
\pgfpathcurveto{\pgfqpoint{9.450825in}{3.594695in}}{\pgfqpoint{9.455216in}{3.605294in}}{\pgfqpoint{9.455216in}{3.616344in}}%
\pgfpathcurveto{\pgfqpoint{9.455216in}{3.627394in}}{\pgfqpoint{9.450825in}{3.637993in}}{\pgfqpoint{9.443012in}{3.645807in}}%
\pgfpathcurveto{\pgfqpoint{9.435198in}{3.653620in}}{\pgfqpoint{9.424599in}{3.658011in}}{\pgfqpoint{9.413549in}{3.658011in}}%
\pgfpathcurveto{\pgfqpoint{9.402499in}{3.658011in}}{\pgfqpoint{9.391900in}{3.653620in}}{\pgfqpoint{9.384086in}{3.645807in}}%
\pgfpathcurveto{\pgfqpoint{9.376273in}{3.637993in}}{\pgfqpoint{9.371882in}{3.627394in}}{\pgfqpoint{9.371882in}{3.616344in}}%
\pgfpathcurveto{\pgfqpoint{9.371882in}{3.605294in}}{\pgfqpoint{9.376273in}{3.594695in}}{\pgfqpoint{9.384086in}{3.586881in}}%
\pgfpathcurveto{\pgfqpoint{9.391900in}{3.579068in}}{\pgfqpoint{9.402499in}{3.574677in}}{\pgfqpoint{9.413549in}{3.574677in}}%
\pgfpathlineto{\pgfqpoint{9.413549in}{3.574677in}}%
\pgfpathclose%
\pgfusepath{stroke}%
\end{pgfscope}%
\begin{pgfscope}%
\pgfpathrectangle{\pgfqpoint{7.394209in}{0.375000in}}{\pgfqpoint{6.356833in}{5.175000in}}%
\pgfusepath{clip}%
\pgfsetbuttcap%
\pgfsetroundjoin%
\pgfsetlinewidth{1.003750pt}%
\definecolor{currentstroke}{rgb}{0.827451,0.827451,0.827451}%
\pgfsetstrokecolor{currentstroke}%
\pgfsetdash{}{0pt}%
\pgfpathmoveto{\pgfqpoint{10.248406in}{3.388226in}}%
\pgfpathcurveto{\pgfqpoint{10.259456in}{3.388226in}}{\pgfqpoint{10.270055in}{3.392616in}}{\pgfqpoint{10.277869in}{3.400430in}}%
\pgfpathcurveto{\pgfqpoint{10.285682in}{3.408243in}}{\pgfqpoint{10.290072in}{3.418842in}}{\pgfqpoint{10.290072in}{3.429893in}}%
\pgfpathcurveto{\pgfqpoint{10.290072in}{3.440943in}}{\pgfqpoint{10.285682in}{3.451542in}}{\pgfqpoint{10.277869in}{3.459355in}}%
\pgfpathcurveto{\pgfqpoint{10.270055in}{3.467169in}}{\pgfqpoint{10.259456in}{3.471559in}}{\pgfqpoint{10.248406in}{3.471559in}}%
\pgfpathcurveto{\pgfqpoint{10.237356in}{3.471559in}}{\pgfqpoint{10.226757in}{3.467169in}}{\pgfqpoint{10.218943in}{3.459355in}}%
\pgfpathcurveto{\pgfqpoint{10.211129in}{3.451542in}}{\pgfqpoint{10.206739in}{3.440943in}}{\pgfqpoint{10.206739in}{3.429893in}}%
\pgfpathcurveto{\pgfqpoint{10.206739in}{3.418842in}}{\pgfqpoint{10.211129in}{3.408243in}}{\pgfqpoint{10.218943in}{3.400430in}}%
\pgfpathcurveto{\pgfqpoint{10.226757in}{3.392616in}}{\pgfqpoint{10.237356in}{3.388226in}}{\pgfqpoint{10.248406in}{3.388226in}}%
\pgfpathlineto{\pgfqpoint{10.248406in}{3.388226in}}%
\pgfpathclose%
\pgfusepath{stroke}%
\end{pgfscope}%
\begin{pgfscope}%
\pgfpathrectangle{\pgfqpoint{7.394209in}{0.375000in}}{\pgfqpoint{6.356833in}{5.175000in}}%
\pgfusepath{clip}%
\pgfsetbuttcap%
\pgfsetroundjoin%
\pgfsetlinewidth{1.003750pt}%
\definecolor{currentstroke}{rgb}{0.827451,0.827451,0.827451}%
\pgfsetstrokecolor{currentstroke}%
\pgfsetdash{}{0pt}%
\pgfpathmoveto{\pgfqpoint{10.623324in}{3.724382in}}%
\pgfpathcurveto{\pgfqpoint{10.634374in}{3.724382in}}{\pgfqpoint{10.644973in}{3.728773in}}{\pgfqpoint{10.652787in}{3.736586in}}%
\pgfpathcurveto{\pgfqpoint{10.660601in}{3.744400in}}{\pgfqpoint{10.664991in}{3.754999in}}{\pgfqpoint{10.664991in}{3.766049in}}%
\pgfpathcurveto{\pgfqpoint{10.664991in}{3.777099in}}{\pgfqpoint{10.660601in}{3.787698in}}{\pgfqpoint{10.652787in}{3.795512in}}%
\pgfpathcurveto{\pgfqpoint{10.644973in}{3.803325in}}{\pgfqpoint{10.634374in}{3.807716in}}{\pgfqpoint{10.623324in}{3.807716in}}%
\pgfpathcurveto{\pgfqpoint{10.612274in}{3.807716in}}{\pgfqpoint{10.601675in}{3.803325in}}{\pgfqpoint{10.593862in}{3.795512in}}%
\pgfpathcurveto{\pgfqpoint{10.586048in}{3.787698in}}{\pgfqpoint{10.581658in}{3.777099in}}{\pgfqpoint{10.581658in}{3.766049in}}%
\pgfpathcurveto{\pgfqpoint{10.581658in}{3.754999in}}{\pgfqpoint{10.586048in}{3.744400in}}{\pgfqpoint{10.593862in}{3.736586in}}%
\pgfpathcurveto{\pgfqpoint{10.601675in}{3.728773in}}{\pgfqpoint{10.612274in}{3.724382in}}{\pgfqpoint{10.623324in}{3.724382in}}%
\pgfpathlineto{\pgfqpoint{10.623324in}{3.724382in}}%
\pgfpathclose%
\pgfusepath{stroke}%
\end{pgfscope}%
\begin{pgfscope}%
\pgfpathrectangle{\pgfqpoint{7.394209in}{0.375000in}}{\pgfqpoint{6.356833in}{5.175000in}}%
\pgfusepath{clip}%
\pgfsetbuttcap%
\pgfsetroundjoin%
\pgfsetlinewidth{1.003750pt}%
\definecolor{currentstroke}{rgb}{0.827451,0.827451,0.827451}%
\pgfsetstrokecolor{currentstroke}%
\pgfsetdash{}{0pt}%
\pgfpathmoveto{\pgfqpoint{7.482475in}{1.222796in}}%
\pgfpathcurveto{\pgfqpoint{7.493525in}{1.222796in}}{\pgfqpoint{7.504124in}{1.227187in}}{\pgfqpoint{7.511937in}{1.235000in}}%
\pgfpathcurveto{\pgfqpoint{7.519751in}{1.242814in}}{\pgfqpoint{7.524141in}{1.253413in}}{\pgfqpoint{7.524141in}{1.264463in}}%
\pgfpathcurveto{\pgfqpoint{7.524141in}{1.275513in}}{\pgfqpoint{7.519751in}{1.286112in}}{\pgfqpoint{7.511937in}{1.293926in}}%
\pgfpathcurveto{\pgfqpoint{7.504124in}{1.301740in}}{\pgfqpoint{7.493525in}{1.306130in}}{\pgfqpoint{7.482475in}{1.306130in}}%
\pgfpathcurveto{\pgfqpoint{7.471424in}{1.306130in}}{\pgfqpoint{7.460825in}{1.301740in}}{\pgfqpoint{7.453012in}{1.293926in}}%
\pgfpathcurveto{\pgfqpoint{7.445198in}{1.286112in}}{\pgfqpoint{7.440808in}{1.275513in}}{\pgfqpoint{7.440808in}{1.264463in}}%
\pgfpathcurveto{\pgfqpoint{7.440808in}{1.253413in}}{\pgfqpoint{7.445198in}{1.242814in}}{\pgfqpoint{7.453012in}{1.235000in}}%
\pgfpathcurveto{\pgfqpoint{7.460825in}{1.227187in}}{\pgfqpoint{7.471424in}{1.222796in}}{\pgfqpoint{7.482475in}{1.222796in}}%
\pgfpathlineto{\pgfqpoint{7.482475in}{1.222796in}}%
\pgfpathclose%
\pgfusepath{stroke}%
\end{pgfscope}%
\begin{pgfscope}%
\pgfpathrectangle{\pgfqpoint{7.394209in}{0.375000in}}{\pgfqpoint{6.356833in}{5.175000in}}%
\pgfusepath{clip}%
\pgfsetbuttcap%
\pgfsetroundjoin%
\pgfsetlinewidth{1.003750pt}%
\definecolor{currentstroke}{rgb}{0.827451,0.827451,0.827451}%
\pgfsetstrokecolor{currentstroke}%
\pgfsetdash{}{0pt}%
\pgfpathmoveto{\pgfqpoint{8.899125in}{3.318767in}}%
\pgfpathcurveto{\pgfqpoint{8.910175in}{3.318767in}}{\pgfqpoint{8.920774in}{3.323157in}}{\pgfqpoint{8.928587in}{3.330970in}}%
\pgfpathcurveto{\pgfqpoint{8.936401in}{3.338784in}}{\pgfqpoint{8.940791in}{3.349383in}}{\pgfqpoint{8.940791in}{3.360433in}}%
\pgfpathcurveto{\pgfqpoint{8.940791in}{3.371483in}}{\pgfqpoint{8.936401in}{3.382082in}}{\pgfqpoint{8.928587in}{3.389896in}}%
\pgfpathcurveto{\pgfqpoint{8.920774in}{3.397710in}}{\pgfqpoint{8.910175in}{3.402100in}}{\pgfqpoint{8.899125in}{3.402100in}}%
\pgfpathcurveto{\pgfqpoint{8.888074in}{3.402100in}}{\pgfqpoint{8.877475in}{3.397710in}}{\pgfqpoint{8.869662in}{3.389896in}}%
\pgfpathcurveto{\pgfqpoint{8.861848in}{3.382082in}}{\pgfqpoint{8.857458in}{3.371483in}}{\pgfqpoint{8.857458in}{3.360433in}}%
\pgfpathcurveto{\pgfqpoint{8.857458in}{3.349383in}}{\pgfqpoint{8.861848in}{3.338784in}}{\pgfqpoint{8.869662in}{3.330970in}}%
\pgfpathcurveto{\pgfqpoint{8.877475in}{3.323157in}}{\pgfqpoint{8.888074in}{3.318767in}}{\pgfqpoint{8.899125in}{3.318767in}}%
\pgfpathlineto{\pgfqpoint{8.899125in}{3.318767in}}%
\pgfpathclose%
\pgfusepath{stroke}%
\end{pgfscope}%
\begin{pgfscope}%
\pgfpathrectangle{\pgfqpoint{7.394209in}{0.375000in}}{\pgfqpoint{6.356833in}{5.175000in}}%
\pgfusepath{clip}%
\pgfsetbuttcap%
\pgfsetroundjoin%
\pgfsetlinewidth{1.003750pt}%
\definecolor{currentstroke}{rgb}{0.827451,0.827451,0.827451}%
\pgfsetstrokecolor{currentstroke}%
\pgfsetdash{}{0pt}%
\pgfpathmoveto{\pgfqpoint{8.618638in}{2.184752in}}%
\pgfpathcurveto{\pgfqpoint{8.629688in}{2.184752in}}{\pgfqpoint{8.640287in}{2.189143in}}{\pgfqpoint{8.648101in}{2.196956in}}%
\pgfpathcurveto{\pgfqpoint{8.655915in}{2.204770in}}{\pgfqpoint{8.660305in}{2.215369in}}{\pgfqpoint{8.660305in}{2.226419in}}%
\pgfpathcurveto{\pgfqpoint{8.660305in}{2.237469in}}{\pgfqpoint{8.655915in}{2.248068in}}{\pgfqpoint{8.648101in}{2.255882in}}%
\pgfpathcurveto{\pgfqpoint{8.640287in}{2.263695in}}{\pgfqpoint{8.629688in}{2.268086in}}{\pgfqpoint{8.618638in}{2.268086in}}%
\pgfpathcurveto{\pgfqpoint{8.607588in}{2.268086in}}{\pgfqpoint{8.596989in}{2.263695in}}{\pgfqpoint{8.589176in}{2.255882in}}%
\pgfpathcurveto{\pgfqpoint{8.581362in}{2.248068in}}{\pgfqpoint{8.576972in}{2.237469in}}{\pgfqpoint{8.576972in}{2.226419in}}%
\pgfpathcurveto{\pgfqpoint{8.576972in}{2.215369in}}{\pgfqpoint{8.581362in}{2.204770in}}{\pgfqpoint{8.589176in}{2.196956in}}%
\pgfpathcurveto{\pgfqpoint{8.596989in}{2.189143in}}{\pgfqpoint{8.607588in}{2.184752in}}{\pgfqpoint{8.618638in}{2.184752in}}%
\pgfpathlineto{\pgfqpoint{8.618638in}{2.184752in}}%
\pgfpathclose%
\pgfusepath{stroke}%
\end{pgfscope}%
\begin{pgfscope}%
\pgfpathrectangle{\pgfqpoint{7.394209in}{0.375000in}}{\pgfqpoint{6.356833in}{5.175000in}}%
\pgfusepath{clip}%
\pgfsetbuttcap%
\pgfsetroundjoin%
\pgfsetlinewidth{1.003750pt}%
\definecolor{currentstroke}{rgb}{0.827451,0.827451,0.827451}%
\pgfsetstrokecolor{currentstroke}%
\pgfsetdash{}{0pt}%
\pgfpathmoveto{\pgfqpoint{8.790581in}{3.388226in}}%
\pgfpathcurveto{\pgfqpoint{8.801631in}{3.388226in}}{\pgfqpoint{8.812230in}{3.392616in}}{\pgfqpoint{8.820044in}{3.400430in}}%
\pgfpathcurveto{\pgfqpoint{8.827858in}{3.408243in}}{\pgfqpoint{8.832248in}{3.418842in}}{\pgfqpoint{8.832248in}{3.429893in}}%
\pgfpathcurveto{\pgfqpoint{8.832248in}{3.440943in}}{\pgfqpoint{8.827858in}{3.451542in}}{\pgfqpoint{8.820044in}{3.459355in}}%
\pgfpathcurveto{\pgfqpoint{8.812230in}{3.467169in}}{\pgfqpoint{8.801631in}{3.471559in}}{\pgfqpoint{8.790581in}{3.471559in}}%
\pgfpathcurveto{\pgfqpoint{8.779531in}{3.471559in}}{\pgfqpoint{8.768932in}{3.467169in}}{\pgfqpoint{8.761119in}{3.459355in}}%
\pgfpathcurveto{\pgfqpoint{8.753305in}{3.451542in}}{\pgfqpoint{8.748915in}{3.440943in}}{\pgfqpoint{8.748915in}{3.429893in}}%
\pgfpathcurveto{\pgfqpoint{8.748915in}{3.418842in}}{\pgfqpoint{8.753305in}{3.408243in}}{\pgfqpoint{8.761119in}{3.400430in}}%
\pgfpathcurveto{\pgfqpoint{8.768932in}{3.392616in}}{\pgfqpoint{8.779531in}{3.388226in}}{\pgfqpoint{8.790581in}{3.388226in}}%
\pgfpathlineto{\pgfqpoint{8.790581in}{3.388226in}}%
\pgfpathclose%
\pgfusepath{stroke}%
\end{pgfscope}%
\begin{pgfscope}%
\pgfpathrectangle{\pgfqpoint{7.394209in}{0.375000in}}{\pgfqpoint{6.356833in}{5.175000in}}%
\pgfusepath{clip}%
\pgfsetbuttcap%
\pgfsetroundjoin%
\pgfsetlinewidth{1.003750pt}%
\definecolor{currentstroke}{rgb}{0.827451,0.827451,0.827451}%
\pgfsetstrokecolor{currentstroke}%
\pgfsetdash{}{0pt}%
\pgfpathmoveto{\pgfqpoint{12.872812in}{5.499772in}}%
\pgfpathcurveto{\pgfqpoint{12.883862in}{5.499772in}}{\pgfqpoint{12.894461in}{5.504163in}}{\pgfqpoint{12.902275in}{5.511976in}}%
\pgfpathcurveto{\pgfqpoint{12.910089in}{5.519790in}}{\pgfqpoint{12.914479in}{5.530389in}}{\pgfqpoint{12.914479in}{5.541439in}}%
\pgfpathcurveto{\pgfqpoint{12.914479in}{5.552489in}}{\pgfqpoint{12.910089in}{5.563088in}}{\pgfqpoint{12.902275in}{5.570902in}}%
\pgfpathcurveto{\pgfqpoint{12.894461in}{5.578715in}}{\pgfqpoint{12.883862in}{5.583106in}}{\pgfqpoint{12.872812in}{5.583106in}}%
\pgfpathcurveto{\pgfqpoint{12.861762in}{5.583106in}}{\pgfqpoint{12.851163in}{5.578715in}}{\pgfqpoint{12.843349in}{5.570902in}}%
\pgfpathcurveto{\pgfqpoint{12.835536in}{5.563088in}}{\pgfqpoint{12.831145in}{5.552489in}}{\pgfqpoint{12.831145in}{5.541439in}}%
\pgfpathcurveto{\pgfqpoint{12.831145in}{5.530389in}}{\pgfqpoint{12.835536in}{5.519790in}}{\pgfqpoint{12.843349in}{5.511976in}}%
\pgfpathcurveto{\pgfqpoint{12.851163in}{5.504163in}}{\pgfqpoint{12.861762in}{5.499772in}}{\pgfqpoint{12.872812in}{5.499772in}}%
\pgfpathlineto{\pgfqpoint{12.872812in}{5.499772in}}%
\pgfpathclose%
\pgfusepath{stroke}%
\end{pgfscope}%
\begin{pgfscope}%
\pgfpathrectangle{\pgfqpoint{7.394209in}{0.375000in}}{\pgfqpoint{6.356833in}{5.175000in}}%
\pgfusepath{clip}%
\pgfsetbuttcap%
\pgfsetroundjoin%
\pgfsetlinewidth{1.003750pt}%
\definecolor{currentstroke}{rgb}{0.827451,0.827451,0.827451}%
\pgfsetstrokecolor{currentstroke}%
\pgfsetdash{}{0pt}%
\pgfpathmoveto{\pgfqpoint{10.756322in}{5.438209in}}%
\pgfpathcurveto{\pgfqpoint{10.767372in}{5.438209in}}{\pgfqpoint{10.777971in}{5.442600in}}{\pgfqpoint{10.785785in}{5.450413in}}%
\pgfpathcurveto{\pgfqpoint{10.793599in}{5.458227in}}{\pgfqpoint{10.797989in}{5.468826in}}{\pgfqpoint{10.797989in}{5.479876in}}%
\pgfpathcurveto{\pgfqpoint{10.797989in}{5.490926in}}{\pgfqpoint{10.793599in}{5.501525in}}{\pgfqpoint{10.785785in}{5.509339in}}%
\pgfpathcurveto{\pgfqpoint{10.777971in}{5.517153in}}{\pgfqpoint{10.767372in}{5.521543in}}{\pgfqpoint{10.756322in}{5.521543in}}%
\pgfpathcurveto{\pgfqpoint{10.745272in}{5.521543in}}{\pgfqpoint{10.734673in}{5.517153in}}{\pgfqpoint{10.726860in}{5.509339in}}%
\pgfpathcurveto{\pgfqpoint{10.719046in}{5.501525in}}{\pgfqpoint{10.714656in}{5.490926in}}{\pgfqpoint{10.714656in}{5.479876in}}%
\pgfpathcurveto{\pgfqpoint{10.714656in}{5.468826in}}{\pgfqpoint{10.719046in}{5.458227in}}{\pgfqpoint{10.726860in}{5.450413in}}%
\pgfpathcurveto{\pgfqpoint{10.734673in}{5.442600in}}{\pgfqpoint{10.745272in}{5.438209in}}{\pgfqpoint{10.756322in}{5.438209in}}%
\pgfpathlineto{\pgfqpoint{10.756322in}{5.438209in}}%
\pgfpathclose%
\pgfusepath{stroke}%
\end{pgfscope}%
\begin{pgfscope}%
\pgfpathrectangle{\pgfqpoint{7.394209in}{0.375000in}}{\pgfqpoint{6.356833in}{5.175000in}}%
\pgfusepath{clip}%
\pgfsetbuttcap%
\pgfsetroundjoin%
\pgfsetlinewidth{1.003750pt}%
\definecolor{currentstroke}{rgb}{0.827451,0.827451,0.827451}%
\pgfsetstrokecolor{currentstroke}%
\pgfsetdash{}{0pt}%
\pgfpathmoveto{\pgfqpoint{11.686513in}{5.306473in}}%
\pgfpathcurveto{\pgfqpoint{11.697564in}{5.306473in}}{\pgfqpoint{11.708163in}{5.310863in}}{\pgfqpoint{11.715976in}{5.318677in}}%
\pgfpathcurveto{\pgfqpoint{11.723790in}{5.326491in}}{\pgfqpoint{11.728180in}{5.337090in}}{\pgfqpoint{11.728180in}{5.348140in}}%
\pgfpathcurveto{\pgfqpoint{11.728180in}{5.359190in}}{\pgfqpoint{11.723790in}{5.369789in}}{\pgfqpoint{11.715976in}{5.377603in}}%
\pgfpathcurveto{\pgfqpoint{11.708163in}{5.385416in}}{\pgfqpoint{11.697564in}{5.389806in}}{\pgfqpoint{11.686513in}{5.389806in}}%
\pgfpathcurveto{\pgfqpoint{11.675463in}{5.389806in}}{\pgfqpoint{11.664864in}{5.385416in}}{\pgfqpoint{11.657051in}{5.377603in}}%
\pgfpathcurveto{\pgfqpoint{11.649237in}{5.369789in}}{\pgfqpoint{11.644847in}{5.359190in}}{\pgfqpoint{11.644847in}{5.348140in}}%
\pgfpathcurveto{\pgfqpoint{11.644847in}{5.337090in}}{\pgfqpoint{11.649237in}{5.326491in}}{\pgfqpoint{11.657051in}{5.318677in}}%
\pgfpathcurveto{\pgfqpoint{11.664864in}{5.310863in}}{\pgfqpoint{11.675463in}{5.306473in}}{\pgfqpoint{11.686513in}{5.306473in}}%
\pgfpathlineto{\pgfqpoint{11.686513in}{5.306473in}}%
\pgfpathclose%
\pgfusepath{stroke}%
\end{pgfscope}%
\begin{pgfscope}%
\pgfpathrectangle{\pgfqpoint{7.394209in}{0.375000in}}{\pgfqpoint{6.356833in}{5.175000in}}%
\pgfusepath{clip}%
\pgfsetbuttcap%
\pgfsetroundjoin%
\pgfsetlinewidth{1.003750pt}%
\definecolor{currentstroke}{rgb}{0.827451,0.827451,0.827451}%
\pgfsetstrokecolor{currentstroke}%
\pgfsetdash{}{0pt}%
\pgfpathmoveto{\pgfqpoint{8.135635in}{1.811547in}}%
\pgfpathcurveto{\pgfqpoint{8.146685in}{1.811547in}}{\pgfqpoint{8.157284in}{1.815937in}}{\pgfqpoint{8.165097in}{1.823750in}}%
\pgfpathcurveto{\pgfqpoint{8.172911in}{1.831564in}}{\pgfqpoint{8.177301in}{1.842163in}}{\pgfqpoint{8.177301in}{1.853213in}}%
\pgfpathcurveto{\pgfqpoint{8.177301in}{1.864263in}}{\pgfqpoint{8.172911in}{1.874862in}}{\pgfqpoint{8.165097in}{1.882676in}}%
\pgfpathcurveto{\pgfqpoint{8.157284in}{1.890490in}}{\pgfqpoint{8.146685in}{1.894880in}}{\pgfqpoint{8.135635in}{1.894880in}}%
\pgfpathcurveto{\pgfqpoint{8.124585in}{1.894880in}}{\pgfqpoint{8.113986in}{1.890490in}}{\pgfqpoint{8.106172in}{1.882676in}}%
\pgfpathcurveto{\pgfqpoint{8.098358in}{1.874862in}}{\pgfqpoint{8.093968in}{1.864263in}}{\pgfqpoint{8.093968in}{1.853213in}}%
\pgfpathcurveto{\pgfqpoint{8.093968in}{1.842163in}}{\pgfqpoint{8.098358in}{1.831564in}}{\pgfqpoint{8.106172in}{1.823750in}}%
\pgfpathcurveto{\pgfqpoint{8.113986in}{1.815937in}}{\pgfqpoint{8.124585in}{1.811547in}}{\pgfqpoint{8.135635in}{1.811547in}}%
\pgfpathlineto{\pgfqpoint{8.135635in}{1.811547in}}%
\pgfpathclose%
\pgfusepath{stroke}%
\end{pgfscope}%
\begin{pgfscope}%
\pgfpathrectangle{\pgfqpoint{7.394209in}{0.375000in}}{\pgfqpoint{6.356833in}{5.175000in}}%
\pgfusepath{clip}%
\pgfsetbuttcap%
\pgfsetroundjoin%
\pgfsetlinewidth{1.003750pt}%
\definecolor{currentstroke}{rgb}{0.827451,0.827451,0.827451}%
\pgfsetstrokecolor{currentstroke}%
\pgfsetdash{}{0pt}%
\pgfpathmoveto{\pgfqpoint{10.591475in}{3.574677in}}%
\pgfpathcurveto{\pgfqpoint{10.602525in}{3.574677in}}{\pgfqpoint{10.613124in}{3.579068in}}{\pgfqpoint{10.620937in}{3.586881in}}%
\pgfpathcurveto{\pgfqpoint{10.628751in}{3.594695in}}{\pgfqpoint{10.633141in}{3.605294in}}{\pgfqpoint{10.633141in}{3.616344in}}%
\pgfpathcurveto{\pgfqpoint{10.633141in}{3.627394in}}{\pgfqpoint{10.628751in}{3.637993in}}{\pgfqpoint{10.620937in}{3.645807in}}%
\pgfpathcurveto{\pgfqpoint{10.613124in}{3.653620in}}{\pgfqpoint{10.602525in}{3.658011in}}{\pgfqpoint{10.591475in}{3.658011in}}%
\pgfpathcurveto{\pgfqpoint{10.580424in}{3.658011in}}{\pgfqpoint{10.569825in}{3.653620in}}{\pgfqpoint{10.562012in}{3.645807in}}%
\pgfpathcurveto{\pgfqpoint{10.554198in}{3.637993in}}{\pgfqpoint{10.549808in}{3.627394in}}{\pgfqpoint{10.549808in}{3.616344in}}%
\pgfpathcurveto{\pgfqpoint{10.549808in}{3.605294in}}{\pgfqpoint{10.554198in}{3.594695in}}{\pgfqpoint{10.562012in}{3.586881in}}%
\pgfpathcurveto{\pgfqpoint{10.569825in}{3.579068in}}{\pgfqpoint{10.580424in}{3.574677in}}{\pgfqpoint{10.591475in}{3.574677in}}%
\pgfpathlineto{\pgfqpoint{10.591475in}{3.574677in}}%
\pgfpathclose%
\pgfusepath{stroke}%
\end{pgfscope}%
\begin{pgfscope}%
\pgfpathrectangle{\pgfqpoint{7.394209in}{0.375000in}}{\pgfqpoint{6.356833in}{5.175000in}}%
\pgfusepath{clip}%
\pgfsetbuttcap%
\pgfsetroundjoin%
\pgfsetlinewidth{1.003750pt}%
\definecolor{currentstroke}{rgb}{0.827451,0.827451,0.827451}%
\pgfsetstrokecolor{currentstroke}%
\pgfsetdash{}{0pt}%
\pgfpathmoveto{\pgfqpoint{9.539048in}{4.238367in}}%
\pgfpathcurveto{\pgfqpoint{9.550098in}{4.238367in}}{\pgfqpoint{9.560697in}{4.242757in}}{\pgfqpoint{9.568511in}{4.250571in}}%
\pgfpathcurveto{\pgfqpoint{9.576324in}{4.258384in}}{\pgfqpoint{9.580715in}{4.268983in}}{\pgfqpoint{9.580715in}{4.280033in}}%
\pgfpathcurveto{\pgfqpoint{9.580715in}{4.291083in}}{\pgfqpoint{9.576324in}{4.301682in}}{\pgfqpoint{9.568511in}{4.309496in}}%
\pgfpathcurveto{\pgfqpoint{9.560697in}{4.317310in}}{\pgfqpoint{9.550098in}{4.321700in}}{\pgfqpoint{9.539048in}{4.321700in}}%
\pgfpathcurveto{\pgfqpoint{9.527998in}{4.321700in}}{\pgfqpoint{9.517399in}{4.317310in}}{\pgfqpoint{9.509585in}{4.309496in}}%
\pgfpathcurveto{\pgfqpoint{9.501772in}{4.301682in}}{\pgfqpoint{9.497381in}{4.291083in}}{\pgfqpoint{9.497381in}{4.280033in}}%
\pgfpathcurveto{\pgfqpoint{9.497381in}{4.268983in}}{\pgfqpoint{9.501772in}{4.258384in}}{\pgfqpoint{9.509585in}{4.250571in}}%
\pgfpathcurveto{\pgfqpoint{9.517399in}{4.242757in}}{\pgfqpoint{9.527998in}{4.238367in}}{\pgfqpoint{9.539048in}{4.238367in}}%
\pgfpathlineto{\pgfqpoint{9.539048in}{4.238367in}}%
\pgfpathclose%
\pgfusepath{stroke}%
\end{pgfscope}%
\begin{pgfscope}%
\pgfpathrectangle{\pgfqpoint{7.394209in}{0.375000in}}{\pgfqpoint{6.356833in}{5.175000in}}%
\pgfusepath{clip}%
\pgfsetbuttcap%
\pgfsetroundjoin%
\pgfsetlinewidth{1.003750pt}%
\definecolor{currentstroke}{rgb}{0.827451,0.827451,0.827451}%
\pgfsetstrokecolor{currentstroke}%
\pgfsetdash{}{0pt}%
\pgfpathmoveto{\pgfqpoint{11.564258in}{5.157362in}}%
\pgfpathcurveto{\pgfqpoint{11.575308in}{5.157362in}}{\pgfqpoint{11.585907in}{5.161752in}}{\pgfqpoint{11.593720in}{5.169566in}}%
\pgfpathcurveto{\pgfqpoint{11.601534in}{5.177379in}}{\pgfqpoint{11.605924in}{5.187978in}}{\pgfqpoint{11.605924in}{5.199028in}}%
\pgfpathcurveto{\pgfqpoint{11.605924in}{5.210078in}}{\pgfqpoint{11.601534in}{5.220677in}}{\pgfqpoint{11.593720in}{5.228491in}}%
\pgfpathcurveto{\pgfqpoint{11.585907in}{5.236305in}}{\pgfqpoint{11.575308in}{5.240695in}}{\pgfqpoint{11.564258in}{5.240695in}}%
\pgfpathcurveto{\pgfqpoint{11.553207in}{5.240695in}}{\pgfqpoint{11.542608in}{5.236305in}}{\pgfqpoint{11.534795in}{5.228491in}}%
\pgfpathcurveto{\pgfqpoint{11.526981in}{5.220677in}}{\pgfqpoint{11.522591in}{5.210078in}}{\pgfqpoint{11.522591in}{5.199028in}}%
\pgfpathcurveto{\pgfqpoint{11.522591in}{5.187978in}}{\pgfqpoint{11.526981in}{5.177379in}}{\pgfqpoint{11.534795in}{5.169566in}}%
\pgfpathcurveto{\pgfqpoint{11.542608in}{5.161752in}}{\pgfqpoint{11.553207in}{5.157362in}}{\pgfqpoint{11.564258in}{5.157362in}}%
\pgfpathlineto{\pgfqpoint{11.564258in}{5.157362in}}%
\pgfpathclose%
\pgfusepath{stroke}%
\end{pgfscope}%
\begin{pgfscope}%
\pgfpathrectangle{\pgfqpoint{7.394209in}{0.375000in}}{\pgfqpoint{6.356833in}{5.175000in}}%
\pgfusepath{clip}%
\pgfsetbuttcap%
\pgfsetroundjoin%
\pgfsetlinewidth{1.003750pt}%
\definecolor{currentstroke}{rgb}{0.827451,0.827451,0.827451}%
\pgfsetstrokecolor{currentstroke}%
\pgfsetdash{}{0pt}%
\pgfpathmoveto{\pgfqpoint{8.808211in}{2.852700in}}%
\pgfpathcurveto{\pgfqpoint{8.819261in}{2.852700in}}{\pgfqpoint{8.829860in}{2.857090in}}{\pgfqpoint{8.837674in}{2.864904in}}%
\pgfpathcurveto{\pgfqpoint{8.845487in}{2.872718in}}{\pgfqpoint{8.849878in}{2.883317in}}{\pgfqpoint{8.849878in}{2.894367in}}%
\pgfpathcurveto{\pgfqpoint{8.849878in}{2.905417in}}{\pgfqpoint{8.845487in}{2.916016in}}{\pgfqpoint{8.837674in}{2.923830in}}%
\pgfpathcurveto{\pgfqpoint{8.829860in}{2.931643in}}{\pgfqpoint{8.819261in}{2.936034in}}{\pgfqpoint{8.808211in}{2.936034in}}%
\pgfpathcurveto{\pgfqpoint{8.797161in}{2.936034in}}{\pgfqpoint{8.786562in}{2.931643in}}{\pgfqpoint{8.778748in}{2.923830in}}%
\pgfpathcurveto{\pgfqpoint{8.770935in}{2.916016in}}{\pgfqpoint{8.766544in}{2.905417in}}{\pgfqpoint{8.766544in}{2.894367in}}%
\pgfpathcurveto{\pgfqpoint{8.766544in}{2.883317in}}{\pgfqpoint{8.770935in}{2.872718in}}{\pgfqpoint{8.778748in}{2.864904in}}%
\pgfpathcurveto{\pgfqpoint{8.786562in}{2.857090in}}{\pgfqpoint{8.797161in}{2.852700in}}{\pgfqpoint{8.808211in}{2.852700in}}%
\pgfpathlineto{\pgfqpoint{8.808211in}{2.852700in}}%
\pgfpathclose%
\pgfusepath{stroke}%
\end{pgfscope}%
\begin{pgfscope}%
\pgfpathrectangle{\pgfqpoint{7.394209in}{0.375000in}}{\pgfqpoint{6.356833in}{5.175000in}}%
\pgfusepath{clip}%
\pgfsetbuttcap%
\pgfsetroundjoin%
\pgfsetlinewidth{1.003750pt}%
\definecolor{currentstroke}{rgb}{0.827451,0.827451,0.827451}%
\pgfsetstrokecolor{currentstroke}%
\pgfsetdash{}{0pt}%
\pgfpathmoveto{\pgfqpoint{11.617468in}{5.238738in}}%
\pgfpathcurveto{\pgfqpoint{11.628519in}{5.238738in}}{\pgfqpoint{11.639118in}{5.243129in}}{\pgfqpoint{11.646931in}{5.250942in}}%
\pgfpathcurveto{\pgfqpoint{11.654745in}{5.258756in}}{\pgfqpoint{11.659135in}{5.269355in}}{\pgfqpoint{11.659135in}{5.280405in}}%
\pgfpathcurveto{\pgfqpoint{11.659135in}{5.291455in}}{\pgfqpoint{11.654745in}{5.302054in}}{\pgfqpoint{11.646931in}{5.309868in}}%
\pgfpathcurveto{\pgfqpoint{11.639118in}{5.317681in}}{\pgfqpoint{11.628519in}{5.322072in}}{\pgfqpoint{11.617468in}{5.322072in}}%
\pgfpathcurveto{\pgfqpoint{11.606418in}{5.322072in}}{\pgfqpoint{11.595819in}{5.317681in}}{\pgfqpoint{11.588006in}{5.309868in}}%
\pgfpathcurveto{\pgfqpoint{11.580192in}{5.302054in}}{\pgfqpoint{11.575802in}{5.291455in}}{\pgfqpoint{11.575802in}{5.280405in}}%
\pgfpathcurveto{\pgfqpoint{11.575802in}{5.269355in}}{\pgfqpoint{11.580192in}{5.258756in}}{\pgfqpoint{11.588006in}{5.250942in}}%
\pgfpathcurveto{\pgfqpoint{11.595819in}{5.243129in}}{\pgfqpoint{11.606418in}{5.238738in}}{\pgfqpoint{11.617468in}{5.238738in}}%
\pgfpathlineto{\pgfqpoint{11.617468in}{5.238738in}}%
\pgfpathclose%
\pgfusepath{stroke}%
\end{pgfscope}%
\begin{pgfscope}%
\pgfpathrectangle{\pgfqpoint{7.394209in}{0.375000in}}{\pgfqpoint{6.356833in}{5.175000in}}%
\pgfusepath{clip}%
\pgfsetbuttcap%
\pgfsetroundjoin%
\pgfsetlinewidth{1.003750pt}%
\definecolor{currentstroke}{rgb}{0.827451,0.827451,0.827451}%
\pgfsetstrokecolor{currentstroke}%
\pgfsetdash{}{0pt}%
\pgfpathmoveto{\pgfqpoint{9.069877in}{2.267946in}}%
\pgfpathcurveto{\pgfqpoint{9.080927in}{2.267946in}}{\pgfqpoint{9.091526in}{2.272336in}}{\pgfqpoint{9.099340in}{2.280150in}}%
\pgfpathcurveto{\pgfqpoint{9.107153in}{2.287964in}}{\pgfqpoint{9.111543in}{2.298563in}}{\pgfqpoint{9.111543in}{2.309613in}}%
\pgfpathcurveto{\pgfqpoint{9.111543in}{2.320663in}}{\pgfqpoint{9.107153in}{2.331262in}}{\pgfqpoint{9.099340in}{2.339076in}}%
\pgfpathcurveto{\pgfqpoint{9.091526in}{2.346889in}}{\pgfqpoint{9.080927in}{2.351279in}}{\pgfqpoint{9.069877in}{2.351279in}}%
\pgfpathcurveto{\pgfqpoint{9.058827in}{2.351279in}}{\pgfqpoint{9.048228in}{2.346889in}}{\pgfqpoint{9.040414in}{2.339076in}}%
\pgfpathcurveto{\pgfqpoint{9.032600in}{2.331262in}}{\pgfqpoint{9.028210in}{2.320663in}}{\pgfqpoint{9.028210in}{2.309613in}}%
\pgfpathcurveto{\pgfqpoint{9.028210in}{2.298563in}}{\pgfqpoint{9.032600in}{2.287964in}}{\pgfqpoint{9.040414in}{2.280150in}}%
\pgfpathcurveto{\pgfqpoint{9.048228in}{2.272336in}}{\pgfqpoint{9.058827in}{2.267946in}}{\pgfqpoint{9.069877in}{2.267946in}}%
\pgfpathlineto{\pgfqpoint{9.069877in}{2.267946in}}%
\pgfpathclose%
\pgfusepath{stroke}%
\end{pgfscope}%
\begin{pgfscope}%
\pgfpathrectangle{\pgfqpoint{7.394209in}{0.375000in}}{\pgfqpoint{6.356833in}{5.175000in}}%
\pgfusepath{clip}%
\pgfsetbuttcap%
\pgfsetroundjoin%
\pgfsetlinewidth{1.003750pt}%
\definecolor{currentstroke}{rgb}{0.827451,0.827451,0.827451}%
\pgfsetstrokecolor{currentstroke}%
\pgfsetdash{}{0pt}%
\pgfpathmoveto{\pgfqpoint{7.833091in}{0.492443in}}%
\pgfpathcurveto{\pgfqpoint{7.844141in}{0.492443in}}{\pgfqpoint{7.854740in}{0.496833in}}{\pgfqpoint{7.862554in}{0.504647in}}%
\pgfpathcurveto{\pgfqpoint{7.870367in}{0.512460in}}{\pgfqpoint{7.874758in}{0.523059in}}{\pgfqpoint{7.874758in}{0.534109in}}%
\pgfpathcurveto{\pgfqpoint{7.874758in}{0.545159in}}{\pgfqpoint{7.870367in}{0.555758in}}{\pgfqpoint{7.862554in}{0.563572in}}%
\pgfpathcurveto{\pgfqpoint{7.854740in}{0.571386in}}{\pgfqpoint{7.844141in}{0.575776in}}{\pgfqpoint{7.833091in}{0.575776in}}%
\pgfpathcurveto{\pgfqpoint{7.822041in}{0.575776in}}{\pgfqpoint{7.811442in}{0.571386in}}{\pgfqpoint{7.803628in}{0.563572in}}%
\pgfpathcurveto{\pgfqpoint{7.795815in}{0.555758in}}{\pgfqpoint{7.791424in}{0.545159in}}{\pgfqpoint{7.791424in}{0.534109in}}%
\pgfpathcurveto{\pgfqpoint{7.791424in}{0.523059in}}{\pgfqpoint{7.795815in}{0.512460in}}{\pgfqpoint{7.803628in}{0.504647in}}%
\pgfpathcurveto{\pgfqpoint{7.811442in}{0.496833in}}{\pgfqpoint{7.822041in}{0.492443in}}{\pgfqpoint{7.833091in}{0.492443in}}%
\pgfpathlineto{\pgfqpoint{7.833091in}{0.492443in}}%
\pgfpathclose%
\pgfusepath{stroke}%
\end{pgfscope}%
\begin{pgfscope}%
\pgfpathrectangle{\pgfqpoint{7.394209in}{0.375000in}}{\pgfqpoint{6.356833in}{5.175000in}}%
\pgfusepath{clip}%
\pgfsetbuttcap%
\pgfsetroundjoin%
\pgfsetlinewidth{1.003750pt}%
\definecolor{currentstroke}{rgb}{0.827451,0.827451,0.827451}%
\pgfsetstrokecolor{currentstroke}%
\pgfsetdash{}{0pt}%
\pgfpathmoveto{\pgfqpoint{11.568172in}{5.446835in}}%
\pgfpathcurveto{\pgfqpoint{11.579223in}{5.446835in}}{\pgfqpoint{11.589822in}{5.451225in}}{\pgfqpoint{11.597635in}{5.459039in}}%
\pgfpathcurveto{\pgfqpoint{11.605449in}{5.466853in}}{\pgfqpoint{11.609839in}{5.477452in}}{\pgfqpoint{11.609839in}{5.488502in}}%
\pgfpathcurveto{\pgfqpoint{11.609839in}{5.499552in}}{\pgfqpoint{11.605449in}{5.510151in}}{\pgfqpoint{11.597635in}{5.517964in}}%
\pgfpathcurveto{\pgfqpoint{11.589822in}{5.525778in}}{\pgfqpoint{11.579223in}{5.530168in}}{\pgfqpoint{11.568172in}{5.530168in}}%
\pgfpathcurveto{\pgfqpoint{11.557122in}{5.530168in}}{\pgfqpoint{11.546523in}{5.525778in}}{\pgfqpoint{11.538710in}{5.517964in}}%
\pgfpathcurveto{\pgfqpoint{11.530896in}{5.510151in}}{\pgfqpoint{11.526506in}{5.499552in}}{\pgfqpoint{11.526506in}{5.488502in}}%
\pgfpathcurveto{\pgfqpoint{11.526506in}{5.477452in}}{\pgfqpoint{11.530896in}{5.466853in}}{\pgfqpoint{11.538710in}{5.459039in}}%
\pgfpathcurveto{\pgfqpoint{11.546523in}{5.451225in}}{\pgfqpoint{11.557122in}{5.446835in}}{\pgfqpoint{11.568172in}{5.446835in}}%
\pgfpathlineto{\pgfqpoint{11.568172in}{5.446835in}}%
\pgfpathclose%
\pgfusepath{stroke}%
\end{pgfscope}%
\begin{pgfscope}%
\pgfpathrectangle{\pgfqpoint{7.394209in}{0.375000in}}{\pgfqpoint{6.356833in}{5.175000in}}%
\pgfusepath{clip}%
\pgfsetbuttcap%
\pgfsetroundjoin%
\pgfsetlinewidth{1.003750pt}%
\definecolor{currentstroke}{rgb}{0.827451,0.827451,0.827451}%
\pgfsetstrokecolor{currentstroke}%
\pgfsetdash{}{0pt}%
\pgfpathmoveto{\pgfqpoint{8.037691in}{1.514427in}}%
\pgfpathcurveto{\pgfqpoint{8.048741in}{1.514427in}}{\pgfqpoint{8.059340in}{1.518817in}}{\pgfqpoint{8.067154in}{1.526631in}}%
\pgfpathcurveto{\pgfqpoint{8.074968in}{1.534444in}}{\pgfqpoint{8.079358in}{1.545043in}}{\pgfqpoint{8.079358in}{1.556093in}}%
\pgfpathcurveto{\pgfqpoint{8.079358in}{1.567143in}}{\pgfqpoint{8.074968in}{1.577742in}}{\pgfqpoint{8.067154in}{1.585556in}}%
\pgfpathcurveto{\pgfqpoint{8.059340in}{1.593370in}}{\pgfqpoint{8.048741in}{1.597760in}}{\pgfqpoint{8.037691in}{1.597760in}}%
\pgfpathcurveto{\pgfqpoint{8.026641in}{1.597760in}}{\pgfqpoint{8.016042in}{1.593370in}}{\pgfqpoint{8.008228in}{1.585556in}}%
\pgfpathcurveto{\pgfqpoint{8.000415in}{1.577742in}}{\pgfqpoint{7.996024in}{1.567143in}}{\pgfqpoint{7.996024in}{1.556093in}}%
\pgfpathcurveto{\pgfqpoint{7.996024in}{1.545043in}}{\pgfqpoint{8.000415in}{1.534444in}}{\pgfqpoint{8.008228in}{1.526631in}}%
\pgfpathcurveto{\pgfqpoint{8.016042in}{1.518817in}}{\pgfqpoint{8.026641in}{1.514427in}}{\pgfqpoint{8.037691in}{1.514427in}}%
\pgfpathlineto{\pgfqpoint{8.037691in}{1.514427in}}%
\pgfpathclose%
\pgfusepath{stroke}%
\end{pgfscope}%
\begin{pgfscope}%
\pgfpathrectangle{\pgfqpoint{7.394209in}{0.375000in}}{\pgfqpoint{6.356833in}{5.175000in}}%
\pgfusepath{clip}%
\pgfsetbuttcap%
\pgfsetroundjoin%
\pgfsetlinewidth{1.003750pt}%
\definecolor{currentstroke}{rgb}{0.827451,0.827451,0.827451}%
\pgfsetstrokecolor{currentstroke}%
\pgfsetdash{}{0pt}%
\pgfpathmoveto{\pgfqpoint{9.491754in}{3.681066in}}%
\pgfpathcurveto{\pgfqpoint{9.502804in}{3.681066in}}{\pgfqpoint{9.513403in}{3.685456in}}{\pgfqpoint{9.521216in}{3.693270in}}%
\pgfpathcurveto{\pgfqpoint{9.529030in}{3.701083in}}{\pgfqpoint{9.533420in}{3.711682in}}{\pgfqpoint{9.533420in}{3.722732in}}%
\pgfpathcurveto{\pgfqpoint{9.533420in}{3.733782in}}{\pgfqpoint{9.529030in}{3.744381in}}{\pgfqpoint{9.521216in}{3.752195in}}%
\pgfpathcurveto{\pgfqpoint{9.513403in}{3.760009in}}{\pgfqpoint{9.502804in}{3.764399in}}{\pgfqpoint{9.491754in}{3.764399in}}%
\pgfpathcurveto{\pgfqpoint{9.480704in}{3.764399in}}{\pgfqpoint{9.470104in}{3.760009in}}{\pgfqpoint{9.462291in}{3.752195in}}%
\pgfpathcurveto{\pgfqpoint{9.454477in}{3.744381in}}{\pgfqpoint{9.450087in}{3.733782in}}{\pgfqpoint{9.450087in}{3.722732in}}%
\pgfpathcurveto{\pgfqpoint{9.450087in}{3.711682in}}{\pgfqpoint{9.454477in}{3.701083in}}{\pgfqpoint{9.462291in}{3.693270in}}%
\pgfpathcurveto{\pgfqpoint{9.470104in}{3.685456in}}{\pgfqpoint{9.480704in}{3.681066in}}{\pgfqpoint{9.491754in}{3.681066in}}%
\pgfpathlineto{\pgfqpoint{9.491754in}{3.681066in}}%
\pgfpathclose%
\pgfusepath{stroke}%
\end{pgfscope}%
\begin{pgfscope}%
\pgfpathrectangle{\pgfqpoint{7.394209in}{0.375000in}}{\pgfqpoint{6.356833in}{5.175000in}}%
\pgfusepath{clip}%
\pgfsetbuttcap%
\pgfsetroundjoin%
\pgfsetlinewidth{1.003750pt}%
\definecolor{currentstroke}{rgb}{0.827451,0.827451,0.827451}%
\pgfsetstrokecolor{currentstroke}%
\pgfsetdash{}{0pt}%
\pgfpathmoveto{\pgfqpoint{10.480251in}{4.025189in}}%
\pgfpathcurveto{\pgfqpoint{10.491301in}{4.025189in}}{\pgfqpoint{10.501900in}{4.029579in}}{\pgfqpoint{10.509714in}{4.037393in}}%
\pgfpathcurveto{\pgfqpoint{10.517527in}{4.045206in}}{\pgfqpoint{10.521918in}{4.055805in}}{\pgfqpoint{10.521918in}{4.066856in}}%
\pgfpathcurveto{\pgfqpoint{10.521918in}{4.077906in}}{\pgfqpoint{10.517527in}{4.088505in}}{\pgfqpoint{10.509714in}{4.096318in}}%
\pgfpathcurveto{\pgfqpoint{10.501900in}{4.104132in}}{\pgfqpoint{10.491301in}{4.108522in}}{\pgfqpoint{10.480251in}{4.108522in}}%
\pgfpathcurveto{\pgfqpoint{10.469201in}{4.108522in}}{\pgfqpoint{10.458602in}{4.104132in}}{\pgfqpoint{10.450788in}{4.096318in}}%
\pgfpathcurveto{\pgfqpoint{10.442975in}{4.088505in}}{\pgfqpoint{10.438584in}{4.077906in}}{\pgfqpoint{10.438584in}{4.066856in}}%
\pgfpathcurveto{\pgfqpoint{10.438584in}{4.055805in}}{\pgfqpoint{10.442975in}{4.045206in}}{\pgfqpoint{10.450788in}{4.037393in}}%
\pgfpathcurveto{\pgfqpoint{10.458602in}{4.029579in}}{\pgfqpoint{10.469201in}{4.025189in}}{\pgfqpoint{10.480251in}{4.025189in}}%
\pgfpathlineto{\pgfqpoint{10.480251in}{4.025189in}}%
\pgfpathclose%
\pgfusepath{stroke}%
\end{pgfscope}%
\begin{pgfscope}%
\pgfpathrectangle{\pgfqpoint{7.394209in}{0.375000in}}{\pgfqpoint{6.356833in}{5.175000in}}%
\pgfusepath{clip}%
\pgfsetbuttcap%
\pgfsetroundjoin%
\pgfsetlinewidth{1.003750pt}%
\definecolor{currentstroke}{rgb}{0.827451,0.827451,0.827451}%
\pgfsetstrokecolor{currentstroke}%
\pgfsetdash{}{0pt}%
\pgfpathmoveto{\pgfqpoint{12.729939in}{5.407106in}}%
\pgfpathcurveto{\pgfqpoint{12.740989in}{5.407106in}}{\pgfqpoint{12.751588in}{5.411496in}}{\pgfqpoint{12.759402in}{5.419309in}}%
\pgfpathcurveto{\pgfqpoint{12.767215in}{5.427123in}}{\pgfqpoint{12.771606in}{5.437722in}}{\pgfqpoint{12.771606in}{5.448772in}}%
\pgfpathcurveto{\pgfqpoint{12.771606in}{5.459822in}}{\pgfqpoint{12.767215in}{5.470421in}}{\pgfqpoint{12.759402in}{5.478235in}}%
\pgfpathcurveto{\pgfqpoint{12.751588in}{5.486049in}}{\pgfqpoint{12.740989in}{5.490439in}}{\pgfqpoint{12.729939in}{5.490439in}}%
\pgfpathcurveto{\pgfqpoint{12.718889in}{5.490439in}}{\pgfqpoint{12.708290in}{5.486049in}}{\pgfqpoint{12.700476in}{5.478235in}}%
\pgfpathcurveto{\pgfqpoint{12.692663in}{5.470421in}}{\pgfqpoint{12.688272in}{5.459822in}}{\pgfqpoint{12.688272in}{5.448772in}}%
\pgfpathcurveto{\pgfqpoint{12.688272in}{5.437722in}}{\pgfqpoint{12.692663in}{5.427123in}}{\pgfqpoint{12.700476in}{5.419309in}}%
\pgfpathcurveto{\pgfqpoint{12.708290in}{5.411496in}}{\pgfqpoint{12.718889in}{5.407106in}}{\pgfqpoint{12.729939in}{5.407106in}}%
\pgfpathlineto{\pgfqpoint{12.729939in}{5.407106in}}%
\pgfpathclose%
\pgfusepath{stroke}%
\end{pgfscope}%
\begin{pgfscope}%
\pgfpathrectangle{\pgfqpoint{7.394209in}{0.375000in}}{\pgfqpoint{6.356833in}{5.175000in}}%
\pgfusepath{clip}%
\pgfsetbuttcap%
\pgfsetroundjoin%
\pgfsetlinewidth{1.003750pt}%
\definecolor{currentstroke}{rgb}{0.827451,0.827451,0.827451}%
\pgfsetstrokecolor{currentstroke}%
\pgfsetdash{}{0pt}%
\pgfpathmoveto{\pgfqpoint{8.944098in}{1.496126in}}%
\pgfpathcurveto{\pgfqpoint{8.955148in}{1.496126in}}{\pgfqpoint{8.965747in}{1.500516in}}{\pgfqpoint{8.973561in}{1.508330in}}%
\pgfpathcurveto{\pgfqpoint{8.981374in}{1.516143in}}{\pgfqpoint{8.985765in}{1.526742in}}{\pgfqpoint{8.985765in}{1.537792in}}%
\pgfpathcurveto{\pgfqpoint{8.985765in}{1.548843in}}{\pgfqpoint{8.981374in}{1.559442in}}{\pgfqpoint{8.973561in}{1.567255in}}%
\pgfpathcurveto{\pgfqpoint{8.965747in}{1.575069in}}{\pgfqpoint{8.955148in}{1.579459in}}{\pgfqpoint{8.944098in}{1.579459in}}%
\pgfpathcurveto{\pgfqpoint{8.933048in}{1.579459in}}{\pgfqpoint{8.922449in}{1.575069in}}{\pgfqpoint{8.914635in}{1.567255in}}%
\pgfpathcurveto{\pgfqpoint{8.906821in}{1.559442in}}{\pgfqpoint{8.902431in}{1.548843in}}{\pgfqpoint{8.902431in}{1.537792in}}%
\pgfpathcurveto{\pgfqpoint{8.902431in}{1.526742in}}{\pgfqpoint{8.906821in}{1.516143in}}{\pgfqpoint{8.914635in}{1.508330in}}%
\pgfpathcurveto{\pgfqpoint{8.922449in}{1.500516in}}{\pgfqpoint{8.933048in}{1.496126in}}{\pgfqpoint{8.944098in}{1.496126in}}%
\pgfpathlineto{\pgfqpoint{8.944098in}{1.496126in}}%
\pgfpathclose%
\pgfusepath{stroke}%
\end{pgfscope}%
\begin{pgfscope}%
\pgfpathrectangle{\pgfqpoint{7.394209in}{0.375000in}}{\pgfqpoint{6.356833in}{5.175000in}}%
\pgfusepath{clip}%
\pgfsetbuttcap%
\pgfsetroundjoin%
\pgfsetlinewidth{1.003750pt}%
\definecolor{currentstroke}{rgb}{0.827451,0.827451,0.827451}%
\pgfsetstrokecolor{currentstroke}%
\pgfsetdash{}{0pt}%
\pgfpathmoveto{\pgfqpoint{8.565096in}{2.740994in}}%
\pgfpathcurveto{\pgfqpoint{8.576146in}{2.740994in}}{\pgfqpoint{8.586746in}{2.745384in}}{\pgfqpoint{8.594559in}{2.753198in}}%
\pgfpathcurveto{\pgfqpoint{8.602373in}{2.761011in}}{\pgfqpoint{8.606763in}{2.771610in}}{\pgfqpoint{8.606763in}{2.782661in}}%
\pgfpathcurveto{\pgfqpoint{8.606763in}{2.793711in}}{\pgfqpoint{8.602373in}{2.804310in}}{\pgfqpoint{8.594559in}{2.812123in}}%
\pgfpathcurveto{\pgfqpoint{8.586746in}{2.819937in}}{\pgfqpoint{8.576146in}{2.824327in}}{\pgfqpoint{8.565096in}{2.824327in}}%
\pgfpathcurveto{\pgfqpoint{8.554046in}{2.824327in}}{\pgfqpoint{8.543447in}{2.819937in}}{\pgfqpoint{8.535634in}{2.812123in}}%
\pgfpathcurveto{\pgfqpoint{8.527820in}{2.804310in}}{\pgfqpoint{8.523430in}{2.793711in}}{\pgfqpoint{8.523430in}{2.782661in}}%
\pgfpathcurveto{\pgfqpoint{8.523430in}{2.771610in}}{\pgfqpoint{8.527820in}{2.761011in}}{\pgfqpoint{8.535634in}{2.753198in}}%
\pgfpathcurveto{\pgfqpoint{8.543447in}{2.745384in}}{\pgfqpoint{8.554046in}{2.740994in}}{\pgfqpoint{8.565096in}{2.740994in}}%
\pgfpathlineto{\pgfqpoint{8.565096in}{2.740994in}}%
\pgfpathclose%
\pgfusepath{stroke}%
\end{pgfscope}%
\begin{pgfscope}%
\pgfpathrectangle{\pgfqpoint{7.394209in}{0.375000in}}{\pgfqpoint{6.356833in}{5.175000in}}%
\pgfusepath{clip}%
\pgfsetbuttcap%
\pgfsetroundjoin%
\pgfsetlinewidth{1.003750pt}%
\definecolor{currentstroke}{rgb}{0.827451,0.827451,0.827451}%
\pgfsetstrokecolor{currentstroke}%
\pgfsetdash{}{0pt}%
\pgfpathmoveto{\pgfqpoint{9.143380in}{1.820514in}}%
\pgfpathcurveto{\pgfqpoint{9.154430in}{1.820514in}}{\pgfqpoint{9.165029in}{1.824905in}}{\pgfqpoint{9.172843in}{1.832718in}}%
\pgfpathcurveto{\pgfqpoint{9.180656in}{1.840532in}}{\pgfqpoint{9.185047in}{1.851131in}}{\pgfqpoint{9.185047in}{1.862181in}}%
\pgfpathcurveto{\pgfqpoint{9.185047in}{1.873231in}}{\pgfqpoint{9.180656in}{1.883830in}}{\pgfqpoint{9.172843in}{1.891644in}}%
\pgfpathcurveto{\pgfqpoint{9.165029in}{1.899457in}}{\pgfqpoint{9.154430in}{1.903848in}}{\pgfqpoint{9.143380in}{1.903848in}}%
\pgfpathcurveto{\pgfqpoint{9.132330in}{1.903848in}}{\pgfqpoint{9.121731in}{1.899457in}}{\pgfqpoint{9.113917in}{1.891644in}}%
\pgfpathcurveto{\pgfqpoint{9.106104in}{1.883830in}}{\pgfqpoint{9.101713in}{1.873231in}}{\pgfqpoint{9.101713in}{1.862181in}}%
\pgfpathcurveto{\pgfqpoint{9.101713in}{1.851131in}}{\pgfqpoint{9.106104in}{1.840532in}}{\pgfqpoint{9.113917in}{1.832718in}}%
\pgfpathcurveto{\pgfqpoint{9.121731in}{1.824905in}}{\pgfqpoint{9.132330in}{1.820514in}}{\pgfqpoint{9.143380in}{1.820514in}}%
\pgfpathlineto{\pgfqpoint{9.143380in}{1.820514in}}%
\pgfpathclose%
\pgfusepath{stroke}%
\end{pgfscope}%
\begin{pgfscope}%
\pgfpathrectangle{\pgfqpoint{7.394209in}{0.375000in}}{\pgfqpoint{6.356833in}{5.175000in}}%
\pgfusepath{clip}%
\pgfsetbuttcap%
\pgfsetroundjoin%
\pgfsetlinewidth{1.003750pt}%
\definecolor{currentstroke}{rgb}{0.827451,0.827451,0.827451}%
\pgfsetstrokecolor{currentstroke}%
\pgfsetdash{}{0pt}%
\pgfpathmoveto{\pgfqpoint{11.617468in}{4.928474in}}%
\pgfpathcurveto{\pgfqpoint{11.628519in}{4.928474in}}{\pgfqpoint{11.639118in}{4.932864in}}{\pgfqpoint{11.646931in}{4.940678in}}%
\pgfpathcurveto{\pgfqpoint{11.654745in}{4.948491in}}{\pgfqpoint{11.659135in}{4.959090in}}{\pgfqpoint{11.659135in}{4.970140in}}%
\pgfpathcurveto{\pgfqpoint{11.659135in}{4.981190in}}{\pgfqpoint{11.654745in}{4.991789in}}{\pgfqpoint{11.646931in}{4.999603in}}%
\pgfpathcurveto{\pgfqpoint{11.639118in}{5.007417in}}{\pgfqpoint{11.628519in}{5.011807in}}{\pgfqpoint{11.617468in}{5.011807in}}%
\pgfpathcurveto{\pgfqpoint{11.606418in}{5.011807in}}{\pgfqpoint{11.595819in}{5.007417in}}{\pgfqpoint{11.588006in}{4.999603in}}%
\pgfpathcurveto{\pgfqpoint{11.580192in}{4.991789in}}{\pgfqpoint{11.575802in}{4.981190in}}{\pgfqpoint{11.575802in}{4.970140in}}%
\pgfpathcurveto{\pgfqpoint{11.575802in}{4.959090in}}{\pgfqpoint{11.580192in}{4.948491in}}{\pgfqpoint{11.588006in}{4.940678in}}%
\pgfpathcurveto{\pgfqpoint{11.595819in}{4.932864in}}{\pgfqpoint{11.606418in}{4.928474in}}{\pgfqpoint{11.617468in}{4.928474in}}%
\pgfpathlineto{\pgfqpoint{11.617468in}{4.928474in}}%
\pgfpathclose%
\pgfusepath{stroke}%
\end{pgfscope}%
\begin{pgfscope}%
\pgfpathrectangle{\pgfqpoint{7.394209in}{0.375000in}}{\pgfqpoint{6.356833in}{5.175000in}}%
\pgfusepath{clip}%
\pgfsetbuttcap%
\pgfsetroundjoin%
\pgfsetlinewidth{1.003750pt}%
\definecolor{currentstroke}{rgb}{0.827451,0.827451,0.827451}%
\pgfsetstrokecolor{currentstroke}%
\pgfsetdash{}{0pt}%
\pgfpathmoveto{\pgfqpoint{9.357761in}{3.811653in}}%
\pgfpathcurveto{\pgfqpoint{9.368811in}{3.811653in}}{\pgfqpoint{9.379410in}{3.816043in}}{\pgfqpoint{9.387224in}{3.823857in}}%
\pgfpathcurveto{\pgfqpoint{9.395038in}{3.831671in}}{\pgfqpoint{9.399428in}{3.842270in}}{\pgfqpoint{9.399428in}{3.853320in}}%
\pgfpathcurveto{\pgfqpoint{9.399428in}{3.864370in}}{\pgfqpoint{9.395038in}{3.874969in}}{\pgfqpoint{9.387224in}{3.882783in}}%
\pgfpathcurveto{\pgfqpoint{9.379410in}{3.890596in}}{\pgfqpoint{9.368811in}{3.894986in}}{\pgfqpoint{9.357761in}{3.894986in}}%
\pgfpathcurveto{\pgfqpoint{9.346711in}{3.894986in}}{\pgfqpoint{9.336112in}{3.890596in}}{\pgfqpoint{9.328298in}{3.882783in}}%
\pgfpathcurveto{\pgfqpoint{9.320485in}{3.874969in}}{\pgfqpoint{9.316094in}{3.864370in}}{\pgfqpoint{9.316094in}{3.853320in}}%
\pgfpathcurveto{\pgfqpoint{9.316094in}{3.842270in}}{\pgfqpoint{9.320485in}{3.831671in}}{\pgfqpoint{9.328298in}{3.823857in}}%
\pgfpathcurveto{\pgfqpoint{9.336112in}{3.816043in}}{\pgfqpoint{9.346711in}{3.811653in}}{\pgfqpoint{9.357761in}{3.811653in}}%
\pgfpathlineto{\pgfqpoint{9.357761in}{3.811653in}}%
\pgfpathclose%
\pgfusepath{stroke}%
\end{pgfscope}%
\begin{pgfscope}%
\pgfpathrectangle{\pgfqpoint{7.394209in}{0.375000in}}{\pgfqpoint{6.356833in}{5.175000in}}%
\pgfusepath{clip}%
\pgfsetbuttcap%
\pgfsetroundjoin%
\pgfsetlinewidth{1.003750pt}%
\definecolor{currentstroke}{rgb}{0.827451,0.827451,0.827451}%
\pgfsetstrokecolor{currentstroke}%
\pgfsetdash{}{0pt}%
\pgfpathmoveto{\pgfqpoint{7.648161in}{1.359791in}}%
\pgfpathcurveto{\pgfqpoint{7.659211in}{1.359791in}}{\pgfqpoint{7.669810in}{1.364181in}}{\pgfqpoint{7.677624in}{1.371995in}}%
\pgfpathcurveto{\pgfqpoint{7.685437in}{1.379808in}}{\pgfqpoint{7.689828in}{1.390407in}}{\pgfqpoint{7.689828in}{1.401458in}}%
\pgfpathcurveto{\pgfqpoint{7.689828in}{1.412508in}}{\pgfqpoint{7.685437in}{1.423107in}}{\pgfqpoint{7.677624in}{1.430920in}}%
\pgfpathcurveto{\pgfqpoint{7.669810in}{1.438734in}}{\pgfqpoint{7.659211in}{1.443124in}}{\pgfqpoint{7.648161in}{1.443124in}}%
\pgfpathcurveto{\pgfqpoint{7.637111in}{1.443124in}}{\pgfqpoint{7.626512in}{1.438734in}}{\pgfqpoint{7.618698in}{1.430920in}}%
\pgfpathcurveto{\pgfqpoint{7.610884in}{1.423107in}}{\pgfqpoint{7.606494in}{1.412508in}}{\pgfqpoint{7.606494in}{1.401458in}}%
\pgfpathcurveto{\pgfqpoint{7.606494in}{1.390407in}}{\pgfqpoint{7.610884in}{1.379808in}}{\pgfqpoint{7.618698in}{1.371995in}}%
\pgfpathcurveto{\pgfqpoint{7.626512in}{1.364181in}}{\pgfqpoint{7.637111in}{1.359791in}}{\pgfqpoint{7.648161in}{1.359791in}}%
\pgfpathlineto{\pgfqpoint{7.648161in}{1.359791in}}%
\pgfpathclose%
\pgfusepath{stroke}%
\end{pgfscope}%
\begin{pgfscope}%
\pgfpathrectangle{\pgfqpoint{7.394209in}{0.375000in}}{\pgfqpoint{6.356833in}{5.175000in}}%
\pgfusepath{clip}%
\pgfsetbuttcap%
\pgfsetroundjoin%
\pgfsetlinewidth{1.003750pt}%
\definecolor{currentstroke}{rgb}{0.827451,0.827451,0.827451}%
\pgfsetstrokecolor{currentstroke}%
\pgfsetdash{}{0pt}%
\pgfpathmoveto{\pgfqpoint{8.944098in}{1.545638in}}%
\pgfpathcurveto{\pgfqpoint{8.955148in}{1.545638in}}{\pgfqpoint{8.965747in}{1.550028in}}{\pgfqpoint{8.973561in}{1.557842in}}%
\pgfpathcurveto{\pgfqpoint{8.981374in}{1.565655in}}{\pgfqpoint{8.985765in}{1.576254in}}{\pgfqpoint{8.985765in}{1.587304in}}%
\pgfpathcurveto{\pgfqpoint{8.985765in}{1.598355in}}{\pgfqpoint{8.981374in}{1.608954in}}{\pgfqpoint{8.973561in}{1.616767in}}%
\pgfpathcurveto{\pgfqpoint{8.965747in}{1.624581in}}{\pgfqpoint{8.955148in}{1.628971in}}{\pgfqpoint{8.944098in}{1.628971in}}%
\pgfpathcurveto{\pgfqpoint{8.933048in}{1.628971in}}{\pgfqpoint{8.922449in}{1.624581in}}{\pgfqpoint{8.914635in}{1.616767in}}%
\pgfpathcurveto{\pgfqpoint{8.906821in}{1.608954in}}{\pgfqpoint{8.902431in}{1.598355in}}{\pgfqpoint{8.902431in}{1.587304in}}%
\pgfpathcurveto{\pgfqpoint{8.902431in}{1.576254in}}{\pgfqpoint{8.906821in}{1.565655in}}{\pgfqpoint{8.914635in}{1.557842in}}%
\pgfpathcurveto{\pgfqpoint{8.922449in}{1.550028in}}{\pgfqpoint{8.933048in}{1.545638in}}{\pgfqpoint{8.944098in}{1.545638in}}%
\pgfpathlineto{\pgfqpoint{8.944098in}{1.545638in}}%
\pgfpathclose%
\pgfusepath{stroke}%
\end{pgfscope}%
\begin{pgfscope}%
\pgfpathrectangle{\pgfqpoint{7.394209in}{0.375000in}}{\pgfqpoint{6.356833in}{5.175000in}}%
\pgfusepath{clip}%
\pgfsetbuttcap%
\pgfsetroundjoin%
\pgfsetlinewidth{1.003750pt}%
\definecolor{currentstroke}{rgb}{0.827451,0.827451,0.827451}%
\pgfsetstrokecolor{currentstroke}%
\pgfsetdash{}{0pt}%
\pgfpathmoveto{\pgfqpoint{10.157063in}{3.365025in}}%
\pgfpathcurveto{\pgfqpoint{10.168113in}{3.365025in}}{\pgfqpoint{10.178712in}{3.369415in}}{\pgfqpoint{10.186526in}{3.377229in}}%
\pgfpathcurveto{\pgfqpoint{10.194339in}{3.385043in}}{\pgfqpoint{10.198729in}{3.395642in}}{\pgfqpoint{10.198729in}{3.406692in}}%
\pgfpathcurveto{\pgfqpoint{10.198729in}{3.417742in}}{\pgfqpoint{10.194339in}{3.428341in}}{\pgfqpoint{10.186526in}{3.436155in}}%
\pgfpathcurveto{\pgfqpoint{10.178712in}{3.443968in}}{\pgfqpoint{10.168113in}{3.448359in}}{\pgfqpoint{10.157063in}{3.448359in}}%
\pgfpathcurveto{\pgfqpoint{10.146013in}{3.448359in}}{\pgfqpoint{10.135414in}{3.443968in}}{\pgfqpoint{10.127600in}{3.436155in}}%
\pgfpathcurveto{\pgfqpoint{10.119786in}{3.428341in}}{\pgfqpoint{10.115396in}{3.417742in}}{\pgfqpoint{10.115396in}{3.406692in}}%
\pgfpathcurveto{\pgfqpoint{10.115396in}{3.395642in}}{\pgfqpoint{10.119786in}{3.385043in}}{\pgfqpoint{10.127600in}{3.377229in}}%
\pgfpathcurveto{\pgfqpoint{10.135414in}{3.369415in}}{\pgfqpoint{10.146013in}{3.365025in}}{\pgfqpoint{10.157063in}{3.365025in}}%
\pgfpathlineto{\pgfqpoint{10.157063in}{3.365025in}}%
\pgfpathclose%
\pgfusepath{stroke}%
\end{pgfscope}%
\begin{pgfscope}%
\pgfpathrectangle{\pgfqpoint{7.394209in}{0.375000in}}{\pgfqpoint{6.356833in}{5.175000in}}%
\pgfusepath{clip}%
\pgfsetbuttcap%
\pgfsetroundjoin%
\pgfsetlinewidth{1.003750pt}%
\definecolor{currentstroke}{rgb}{0.827451,0.827451,0.827451}%
\pgfsetstrokecolor{currentstroke}%
\pgfsetdash{}{0pt}%
\pgfpathmoveto{\pgfqpoint{10.157063in}{3.364671in}}%
\pgfpathcurveto{\pgfqpoint{10.168113in}{3.364671in}}{\pgfqpoint{10.178712in}{3.369061in}}{\pgfqpoint{10.186526in}{3.376875in}}%
\pgfpathcurveto{\pgfqpoint{10.194339in}{3.384688in}}{\pgfqpoint{10.198729in}{3.395287in}}{\pgfqpoint{10.198729in}{3.406337in}}%
\pgfpathcurveto{\pgfqpoint{10.198729in}{3.417388in}}{\pgfqpoint{10.194339in}{3.427987in}}{\pgfqpoint{10.186526in}{3.435800in}}%
\pgfpathcurveto{\pgfqpoint{10.178712in}{3.443614in}}{\pgfqpoint{10.168113in}{3.448004in}}{\pgfqpoint{10.157063in}{3.448004in}}%
\pgfpathcurveto{\pgfqpoint{10.146013in}{3.448004in}}{\pgfqpoint{10.135414in}{3.443614in}}{\pgfqpoint{10.127600in}{3.435800in}}%
\pgfpathcurveto{\pgfqpoint{10.119786in}{3.427987in}}{\pgfqpoint{10.115396in}{3.417388in}}{\pgfqpoint{10.115396in}{3.406337in}}%
\pgfpathcurveto{\pgfqpoint{10.115396in}{3.395287in}}{\pgfqpoint{10.119786in}{3.384688in}}{\pgfqpoint{10.127600in}{3.376875in}}%
\pgfpathcurveto{\pgfqpoint{10.135414in}{3.369061in}}{\pgfqpoint{10.146013in}{3.364671in}}{\pgfqpoint{10.157063in}{3.364671in}}%
\pgfpathlineto{\pgfqpoint{10.157063in}{3.364671in}}%
\pgfpathclose%
\pgfusepath{stroke}%
\end{pgfscope}%
\begin{pgfscope}%
\pgfpathrectangle{\pgfqpoint{7.394209in}{0.375000in}}{\pgfqpoint{6.356833in}{5.175000in}}%
\pgfusepath{clip}%
\pgfsetbuttcap%
\pgfsetroundjoin%
\pgfsetlinewidth{1.003750pt}%
\definecolor{currentstroke}{rgb}{0.827451,0.827451,0.827451}%
\pgfsetstrokecolor{currentstroke}%
\pgfsetdash{}{0pt}%
\pgfpathmoveto{\pgfqpoint{9.516494in}{2.676749in}}%
\pgfpathcurveto{\pgfqpoint{9.527544in}{2.676749in}}{\pgfqpoint{9.538143in}{2.681139in}}{\pgfqpoint{9.545956in}{2.688953in}}%
\pgfpathcurveto{\pgfqpoint{9.553770in}{2.696766in}}{\pgfqpoint{9.558160in}{2.707365in}}{\pgfqpoint{9.558160in}{2.718415in}}%
\pgfpathcurveto{\pgfqpoint{9.558160in}{2.729466in}}{\pgfqpoint{9.553770in}{2.740065in}}{\pgfqpoint{9.545956in}{2.747878in}}%
\pgfpathcurveto{\pgfqpoint{9.538143in}{2.755692in}}{\pgfqpoint{9.527544in}{2.760082in}}{\pgfqpoint{9.516494in}{2.760082in}}%
\pgfpathcurveto{\pgfqpoint{9.505443in}{2.760082in}}{\pgfqpoint{9.494844in}{2.755692in}}{\pgfqpoint{9.487031in}{2.747878in}}%
\pgfpathcurveto{\pgfqpoint{9.479217in}{2.740065in}}{\pgfqpoint{9.474827in}{2.729466in}}{\pgfqpoint{9.474827in}{2.718415in}}%
\pgfpathcurveto{\pgfqpoint{9.474827in}{2.707365in}}{\pgfqpoint{9.479217in}{2.696766in}}{\pgfqpoint{9.487031in}{2.688953in}}%
\pgfpathcurveto{\pgfqpoint{9.494844in}{2.681139in}}{\pgfqpoint{9.505443in}{2.676749in}}{\pgfqpoint{9.516494in}{2.676749in}}%
\pgfpathlineto{\pgfqpoint{9.516494in}{2.676749in}}%
\pgfpathclose%
\pgfusepath{stroke}%
\end{pgfscope}%
\begin{pgfscope}%
\pgfpathrectangle{\pgfqpoint{7.394209in}{0.375000in}}{\pgfqpoint{6.356833in}{5.175000in}}%
\pgfusepath{clip}%
\pgfsetbuttcap%
\pgfsetroundjoin%
\pgfsetlinewidth{1.003750pt}%
\definecolor{currentstroke}{rgb}{0.827451,0.827451,0.827451}%
\pgfsetstrokecolor{currentstroke}%
\pgfsetdash{}{0pt}%
\pgfpathmoveto{\pgfqpoint{10.813716in}{4.067080in}}%
\pgfpathcurveto{\pgfqpoint{10.824766in}{4.067080in}}{\pgfqpoint{10.835365in}{4.071470in}}{\pgfqpoint{10.843179in}{4.079284in}}%
\pgfpathcurveto{\pgfqpoint{10.850992in}{4.087098in}}{\pgfqpoint{10.855383in}{4.097697in}}{\pgfqpoint{10.855383in}{4.108747in}}%
\pgfpathcurveto{\pgfqpoint{10.855383in}{4.119797in}}{\pgfqpoint{10.850992in}{4.130396in}}{\pgfqpoint{10.843179in}{4.138210in}}%
\pgfpathcurveto{\pgfqpoint{10.835365in}{4.146023in}}{\pgfqpoint{10.824766in}{4.150413in}}{\pgfqpoint{10.813716in}{4.150413in}}%
\pgfpathcurveto{\pgfqpoint{10.802666in}{4.150413in}}{\pgfqpoint{10.792067in}{4.146023in}}{\pgfqpoint{10.784253in}{4.138210in}}%
\pgfpathcurveto{\pgfqpoint{10.776439in}{4.130396in}}{\pgfqpoint{10.772049in}{4.119797in}}{\pgfqpoint{10.772049in}{4.108747in}}%
\pgfpathcurveto{\pgfqpoint{10.772049in}{4.097697in}}{\pgfqpoint{10.776439in}{4.087098in}}{\pgfqpoint{10.784253in}{4.079284in}}%
\pgfpathcurveto{\pgfqpoint{10.792067in}{4.071470in}}{\pgfqpoint{10.802666in}{4.067080in}}{\pgfqpoint{10.813716in}{4.067080in}}%
\pgfpathlineto{\pgfqpoint{10.813716in}{4.067080in}}%
\pgfpathclose%
\pgfusepath{stroke}%
\end{pgfscope}%
\begin{pgfscope}%
\pgfpathrectangle{\pgfqpoint{7.394209in}{0.375000in}}{\pgfqpoint{6.356833in}{5.175000in}}%
\pgfusepath{clip}%
\pgfsetbuttcap%
\pgfsetroundjoin%
\pgfsetlinewidth{1.003750pt}%
\definecolor{currentstroke}{rgb}{0.827451,0.827451,0.827451}%
\pgfsetstrokecolor{currentstroke}%
\pgfsetdash{}{0pt}%
\pgfpathmoveto{\pgfqpoint{12.800019in}{5.403918in}}%
\pgfpathcurveto{\pgfqpoint{12.811069in}{5.403918in}}{\pgfqpoint{12.821668in}{5.408309in}}{\pgfqpoint{12.829481in}{5.416122in}}%
\pgfpathcurveto{\pgfqpoint{12.837295in}{5.423936in}}{\pgfqpoint{12.841685in}{5.434535in}}{\pgfqpoint{12.841685in}{5.445585in}}%
\pgfpathcurveto{\pgfqpoint{12.841685in}{5.456635in}}{\pgfqpoint{12.837295in}{5.467234in}}{\pgfqpoint{12.829481in}{5.475048in}}%
\pgfpathcurveto{\pgfqpoint{12.821668in}{5.482862in}}{\pgfqpoint{12.811069in}{5.487252in}}{\pgfqpoint{12.800019in}{5.487252in}}%
\pgfpathcurveto{\pgfqpoint{12.788969in}{5.487252in}}{\pgfqpoint{12.778370in}{5.482862in}}{\pgfqpoint{12.770556in}{5.475048in}}%
\pgfpathcurveto{\pgfqpoint{12.762742in}{5.467234in}}{\pgfqpoint{12.758352in}{5.456635in}}{\pgfqpoint{12.758352in}{5.445585in}}%
\pgfpathcurveto{\pgfqpoint{12.758352in}{5.434535in}}{\pgfqpoint{12.762742in}{5.423936in}}{\pgfqpoint{12.770556in}{5.416122in}}%
\pgfpathcurveto{\pgfqpoint{12.778370in}{5.408309in}}{\pgfqpoint{12.788969in}{5.403918in}}{\pgfqpoint{12.800019in}{5.403918in}}%
\pgfpathlineto{\pgfqpoint{12.800019in}{5.403918in}}%
\pgfpathclose%
\pgfusepath{stroke}%
\end{pgfscope}%
\begin{pgfscope}%
\pgfpathrectangle{\pgfqpoint{7.394209in}{0.375000in}}{\pgfqpoint{6.356833in}{5.175000in}}%
\pgfusepath{clip}%
\pgfsetbuttcap%
\pgfsetroundjoin%
\pgfsetlinewidth{1.003750pt}%
\definecolor{currentstroke}{rgb}{0.827451,0.827451,0.827451}%
\pgfsetstrokecolor{currentstroke}%
\pgfsetdash{}{0pt}%
\pgfpathmoveto{\pgfqpoint{11.400693in}{5.399137in}}%
\pgfpathcurveto{\pgfqpoint{11.411743in}{5.399137in}}{\pgfqpoint{11.422342in}{5.403527in}}{\pgfqpoint{11.430156in}{5.411341in}}%
\pgfpathcurveto{\pgfqpoint{11.437970in}{5.419155in}}{\pgfqpoint{11.442360in}{5.429754in}}{\pgfqpoint{11.442360in}{5.440804in}}%
\pgfpathcurveto{\pgfqpoint{11.442360in}{5.451854in}}{\pgfqpoint{11.437970in}{5.462453in}}{\pgfqpoint{11.430156in}{5.470266in}}%
\pgfpathcurveto{\pgfqpoint{11.422342in}{5.478080in}}{\pgfqpoint{11.411743in}{5.482470in}}{\pgfqpoint{11.400693in}{5.482470in}}%
\pgfpathcurveto{\pgfqpoint{11.389643in}{5.482470in}}{\pgfqpoint{11.379044in}{5.478080in}}{\pgfqpoint{11.371230in}{5.470266in}}%
\pgfpathcurveto{\pgfqpoint{11.363417in}{5.462453in}}{\pgfqpoint{11.359027in}{5.451854in}}{\pgfqpoint{11.359027in}{5.440804in}}%
\pgfpathcurveto{\pgfqpoint{11.359027in}{5.429754in}}{\pgfqpoint{11.363417in}{5.419155in}}{\pgfqpoint{11.371230in}{5.411341in}}%
\pgfpathcurveto{\pgfqpoint{11.379044in}{5.403527in}}{\pgfqpoint{11.389643in}{5.399137in}}{\pgfqpoint{11.400693in}{5.399137in}}%
\pgfpathlineto{\pgfqpoint{11.400693in}{5.399137in}}%
\pgfpathclose%
\pgfusepath{stroke}%
\end{pgfscope}%
\begin{pgfscope}%
\pgfpathrectangle{\pgfqpoint{7.394209in}{0.375000in}}{\pgfqpoint{6.356833in}{5.175000in}}%
\pgfusepath{clip}%
\pgfsetbuttcap%
\pgfsetroundjoin%
\pgfsetlinewidth{1.003750pt}%
\definecolor{currentstroke}{rgb}{0.827451,0.827451,0.827451}%
\pgfsetstrokecolor{currentstroke}%
\pgfsetdash{}{0pt}%
\pgfpathmoveto{\pgfqpoint{12.718343in}{5.434402in}}%
\pgfpathcurveto{\pgfqpoint{12.729393in}{5.434402in}}{\pgfqpoint{12.739992in}{5.438792in}}{\pgfqpoint{12.747806in}{5.446606in}}%
\pgfpathcurveto{\pgfqpoint{12.755620in}{5.454419in}}{\pgfqpoint{12.760010in}{5.465018in}}{\pgfqpoint{12.760010in}{5.476069in}}%
\pgfpathcurveto{\pgfqpoint{12.760010in}{5.487119in}}{\pgfqpoint{12.755620in}{5.497718in}}{\pgfqpoint{12.747806in}{5.505531in}}%
\pgfpathcurveto{\pgfqpoint{12.739992in}{5.513345in}}{\pgfqpoint{12.729393in}{5.517735in}}{\pgfqpoint{12.718343in}{5.517735in}}%
\pgfpathcurveto{\pgfqpoint{12.707293in}{5.517735in}}{\pgfqpoint{12.696694in}{5.513345in}}{\pgfqpoint{12.688880in}{5.505531in}}%
\pgfpathcurveto{\pgfqpoint{12.681067in}{5.497718in}}{\pgfqpoint{12.676677in}{5.487119in}}{\pgfqpoint{12.676677in}{5.476069in}}%
\pgfpathcurveto{\pgfqpoint{12.676677in}{5.465018in}}{\pgfqpoint{12.681067in}{5.454419in}}{\pgfqpoint{12.688880in}{5.446606in}}%
\pgfpathcurveto{\pgfqpoint{12.696694in}{5.438792in}}{\pgfqpoint{12.707293in}{5.434402in}}{\pgfqpoint{12.718343in}{5.434402in}}%
\pgfpathlineto{\pgfqpoint{12.718343in}{5.434402in}}%
\pgfpathclose%
\pgfusepath{stroke}%
\end{pgfscope}%
\begin{pgfscope}%
\pgfpathrectangle{\pgfqpoint{7.394209in}{0.375000in}}{\pgfqpoint{6.356833in}{5.175000in}}%
\pgfusepath{clip}%
\pgfsetbuttcap%
\pgfsetroundjoin%
\pgfsetlinewidth{1.003750pt}%
\definecolor{currentstroke}{rgb}{0.827451,0.827451,0.827451}%
\pgfsetstrokecolor{currentstroke}%
\pgfsetdash{}{0pt}%
\pgfpathmoveto{\pgfqpoint{11.268570in}{5.191450in}}%
\pgfpathcurveto{\pgfqpoint{11.279621in}{5.191450in}}{\pgfqpoint{11.290220in}{5.195841in}}{\pgfqpoint{11.298033in}{5.203654in}}%
\pgfpathcurveto{\pgfqpoint{11.305847in}{5.211468in}}{\pgfqpoint{11.310237in}{5.222067in}}{\pgfqpoint{11.310237in}{5.233117in}}%
\pgfpathcurveto{\pgfqpoint{11.310237in}{5.244167in}}{\pgfqpoint{11.305847in}{5.254766in}}{\pgfqpoint{11.298033in}{5.262580in}}%
\pgfpathcurveto{\pgfqpoint{11.290220in}{5.270394in}}{\pgfqpoint{11.279621in}{5.274784in}}{\pgfqpoint{11.268570in}{5.274784in}}%
\pgfpathcurveto{\pgfqpoint{11.257520in}{5.274784in}}{\pgfqpoint{11.246921in}{5.270394in}}{\pgfqpoint{11.239108in}{5.262580in}}%
\pgfpathcurveto{\pgfqpoint{11.231294in}{5.254766in}}{\pgfqpoint{11.226904in}{5.244167in}}{\pgfqpoint{11.226904in}{5.233117in}}%
\pgfpathcurveto{\pgfqpoint{11.226904in}{5.222067in}}{\pgfqpoint{11.231294in}{5.211468in}}{\pgfqpoint{11.239108in}{5.203654in}}%
\pgfpathcurveto{\pgfqpoint{11.246921in}{5.195841in}}{\pgfqpoint{11.257520in}{5.191450in}}{\pgfqpoint{11.268570in}{5.191450in}}%
\pgfpathlineto{\pgfqpoint{11.268570in}{5.191450in}}%
\pgfpathclose%
\pgfusepath{stroke}%
\end{pgfscope}%
\begin{pgfscope}%
\pgfpathrectangle{\pgfqpoint{7.394209in}{0.375000in}}{\pgfqpoint{6.356833in}{5.175000in}}%
\pgfusepath{clip}%
\pgfsetbuttcap%
\pgfsetroundjoin%
\pgfsetlinewidth{1.003750pt}%
\definecolor{currentstroke}{rgb}{0.827451,0.827451,0.827451}%
\pgfsetstrokecolor{currentstroke}%
\pgfsetdash{}{0pt}%
\pgfpathmoveto{\pgfqpoint{7.859209in}{1.595055in}}%
\pgfpathcurveto{\pgfqpoint{7.870259in}{1.595055in}}{\pgfqpoint{7.880858in}{1.599445in}}{\pgfqpoint{7.888672in}{1.607259in}}%
\pgfpathcurveto{\pgfqpoint{7.896485in}{1.615072in}}{\pgfqpoint{7.900876in}{1.625671in}}{\pgfqpoint{7.900876in}{1.636721in}}%
\pgfpathcurveto{\pgfqpoint{7.900876in}{1.647772in}}{\pgfqpoint{7.896485in}{1.658371in}}{\pgfqpoint{7.888672in}{1.666184in}}%
\pgfpathcurveto{\pgfqpoint{7.880858in}{1.673998in}}{\pgfqpoint{7.870259in}{1.678388in}}{\pgfqpoint{7.859209in}{1.678388in}}%
\pgfpathcurveto{\pgfqpoint{7.848159in}{1.678388in}}{\pgfqpoint{7.837560in}{1.673998in}}{\pgfqpoint{7.829746in}{1.666184in}}%
\pgfpathcurveto{\pgfqpoint{7.821933in}{1.658371in}}{\pgfqpoint{7.817542in}{1.647772in}}{\pgfqpoint{7.817542in}{1.636721in}}%
\pgfpathcurveto{\pgfqpoint{7.817542in}{1.625671in}}{\pgfqpoint{7.821933in}{1.615072in}}{\pgfqpoint{7.829746in}{1.607259in}}%
\pgfpathcurveto{\pgfqpoint{7.837560in}{1.599445in}}{\pgfqpoint{7.848159in}{1.595055in}}{\pgfqpoint{7.859209in}{1.595055in}}%
\pgfpathlineto{\pgfqpoint{7.859209in}{1.595055in}}%
\pgfpathclose%
\pgfusepath{stroke}%
\end{pgfscope}%
\begin{pgfscope}%
\pgfpathrectangle{\pgfqpoint{7.394209in}{0.375000in}}{\pgfqpoint{6.356833in}{5.175000in}}%
\pgfusepath{clip}%
\pgfsetbuttcap%
\pgfsetroundjoin%
\pgfsetlinewidth{1.003750pt}%
\definecolor{currentstroke}{rgb}{0.827451,0.827451,0.827451}%
\pgfsetstrokecolor{currentstroke}%
\pgfsetdash{}{0pt}%
\pgfpathmoveto{\pgfqpoint{9.794824in}{2.655096in}}%
\pgfpathcurveto{\pgfqpoint{9.805874in}{2.655096in}}{\pgfqpoint{9.816473in}{2.659487in}}{\pgfqpoint{9.824286in}{2.667300in}}%
\pgfpathcurveto{\pgfqpoint{9.832100in}{2.675114in}}{\pgfqpoint{9.836490in}{2.685713in}}{\pgfqpoint{9.836490in}{2.696763in}}%
\pgfpathcurveto{\pgfqpoint{9.836490in}{2.707813in}}{\pgfqpoint{9.832100in}{2.718412in}}{\pgfqpoint{9.824286in}{2.726226in}}%
\pgfpathcurveto{\pgfqpoint{9.816473in}{2.734040in}}{\pgfqpoint{9.805874in}{2.738430in}}{\pgfqpoint{9.794824in}{2.738430in}}%
\pgfpathcurveto{\pgfqpoint{9.783773in}{2.738430in}}{\pgfqpoint{9.773174in}{2.734040in}}{\pgfqpoint{9.765361in}{2.726226in}}%
\pgfpathcurveto{\pgfqpoint{9.757547in}{2.718412in}}{\pgfqpoint{9.753157in}{2.707813in}}{\pgfqpoint{9.753157in}{2.696763in}}%
\pgfpathcurveto{\pgfqpoint{9.753157in}{2.685713in}}{\pgfqpoint{9.757547in}{2.675114in}}{\pgfqpoint{9.765361in}{2.667300in}}%
\pgfpathcurveto{\pgfqpoint{9.773174in}{2.659487in}}{\pgfqpoint{9.783773in}{2.655096in}}{\pgfqpoint{9.794824in}{2.655096in}}%
\pgfpathlineto{\pgfqpoint{9.794824in}{2.655096in}}%
\pgfpathclose%
\pgfusepath{stroke}%
\end{pgfscope}%
\begin{pgfscope}%
\pgfpathrectangle{\pgfqpoint{7.394209in}{0.375000in}}{\pgfqpoint{6.356833in}{5.175000in}}%
\pgfusepath{clip}%
\pgfsetbuttcap%
\pgfsetroundjoin%
\pgfsetlinewidth{1.003750pt}%
\definecolor{currentstroke}{rgb}{0.827451,0.827451,0.827451}%
\pgfsetstrokecolor{currentstroke}%
\pgfsetdash{}{0pt}%
\pgfpathmoveto{\pgfqpoint{12.411355in}{5.460846in}}%
\pgfpathcurveto{\pgfqpoint{12.422405in}{5.460846in}}{\pgfqpoint{12.433004in}{5.465236in}}{\pgfqpoint{12.440818in}{5.473050in}}%
\pgfpathcurveto{\pgfqpoint{12.448632in}{5.480864in}}{\pgfqpoint{12.453022in}{5.491463in}}{\pgfqpoint{12.453022in}{5.502513in}}%
\pgfpathcurveto{\pgfqpoint{12.453022in}{5.513563in}}{\pgfqpoint{12.448632in}{5.524162in}}{\pgfqpoint{12.440818in}{5.531976in}}%
\pgfpathcurveto{\pgfqpoint{12.433004in}{5.539789in}}{\pgfqpoint{12.422405in}{5.544179in}}{\pgfqpoint{12.411355in}{5.544179in}}%
\pgfpathcurveto{\pgfqpoint{12.400305in}{5.544179in}}{\pgfqpoint{12.389706in}{5.539789in}}{\pgfqpoint{12.381892in}{5.531976in}}%
\pgfpathcurveto{\pgfqpoint{12.374079in}{5.524162in}}{\pgfqpoint{12.369689in}{5.513563in}}{\pgfqpoint{12.369689in}{5.502513in}}%
\pgfpathcurveto{\pgfqpoint{12.369689in}{5.491463in}}{\pgfqpoint{12.374079in}{5.480864in}}{\pgfqpoint{12.381892in}{5.473050in}}%
\pgfpathcurveto{\pgfqpoint{12.389706in}{5.465236in}}{\pgfqpoint{12.400305in}{5.460846in}}{\pgfqpoint{12.411355in}{5.460846in}}%
\pgfpathlineto{\pgfqpoint{12.411355in}{5.460846in}}%
\pgfpathclose%
\pgfusepath{stroke}%
\end{pgfscope}%
\begin{pgfscope}%
\pgfpathrectangle{\pgfqpoint{7.394209in}{0.375000in}}{\pgfqpoint{6.356833in}{5.175000in}}%
\pgfusepath{clip}%
\pgfsetbuttcap%
\pgfsetroundjoin%
\pgfsetlinewidth{1.003750pt}%
\definecolor{currentstroke}{rgb}{0.827451,0.827451,0.827451}%
\pgfsetstrokecolor{currentstroke}%
\pgfsetdash{}{0pt}%
\pgfpathmoveto{\pgfqpoint{12.010529in}{5.403918in}}%
\pgfpathcurveto{\pgfqpoint{12.021580in}{5.403918in}}{\pgfqpoint{12.032179in}{5.408309in}}{\pgfqpoint{12.039992in}{5.416122in}}%
\pgfpathcurveto{\pgfqpoint{12.047806in}{5.423936in}}{\pgfqpoint{12.052196in}{5.434535in}}{\pgfqpoint{12.052196in}{5.445585in}}%
\pgfpathcurveto{\pgfqpoint{12.052196in}{5.456635in}}{\pgfqpoint{12.047806in}{5.467234in}}{\pgfqpoint{12.039992in}{5.475048in}}%
\pgfpathcurveto{\pgfqpoint{12.032179in}{5.482862in}}{\pgfqpoint{12.021580in}{5.487252in}}{\pgfqpoint{12.010529in}{5.487252in}}%
\pgfpathcurveto{\pgfqpoint{11.999479in}{5.487252in}}{\pgfqpoint{11.988880in}{5.482862in}}{\pgfqpoint{11.981067in}{5.475048in}}%
\pgfpathcurveto{\pgfqpoint{11.973253in}{5.467234in}}{\pgfqpoint{11.968863in}{5.456635in}}{\pgfqpoint{11.968863in}{5.445585in}}%
\pgfpathcurveto{\pgfqpoint{11.968863in}{5.434535in}}{\pgfqpoint{11.973253in}{5.423936in}}{\pgfqpoint{11.981067in}{5.416122in}}%
\pgfpathcurveto{\pgfqpoint{11.988880in}{5.408309in}}{\pgfqpoint{11.999479in}{5.403918in}}{\pgfqpoint{12.010529in}{5.403918in}}%
\pgfpathlineto{\pgfqpoint{12.010529in}{5.403918in}}%
\pgfpathclose%
\pgfusepath{stroke}%
\end{pgfscope}%
\begin{pgfscope}%
\pgfpathrectangle{\pgfqpoint{7.394209in}{0.375000in}}{\pgfqpoint{6.356833in}{5.175000in}}%
\pgfusepath{clip}%
\pgfsetbuttcap%
\pgfsetroundjoin%
\pgfsetlinewidth{1.003750pt}%
\definecolor{currentstroke}{rgb}{0.827451,0.827451,0.827451}%
\pgfsetstrokecolor{currentstroke}%
\pgfsetdash{}{0pt}%
\pgfpathmoveto{\pgfqpoint{10.749492in}{5.446066in}}%
\pgfpathcurveto{\pgfqpoint{10.760542in}{5.446066in}}{\pgfqpoint{10.771142in}{5.450456in}}{\pgfqpoint{10.778955in}{5.458269in}}%
\pgfpathcurveto{\pgfqpoint{10.786769in}{5.466083in}}{\pgfqpoint{10.791159in}{5.476682in}}{\pgfqpoint{10.791159in}{5.487732in}}%
\pgfpathcurveto{\pgfqpoint{10.791159in}{5.498782in}}{\pgfqpoint{10.786769in}{5.509381in}}{\pgfqpoint{10.778955in}{5.517195in}}%
\pgfpathcurveto{\pgfqpoint{10.771142in}{5.525009in}}{\pgfqpoint{10.760542in}{5.529399in}}{\pgfqpoint{10.749492in}{5.529399in}}%
\pgfpathcurveto{\pgfqpoint{10.738442in}{5.529399in}}{\pgfqpoint{10.727843in}{5.525009in}}{\pgfqpoint{10.720030in}{5.517195in}}%
\pgfpathcurveto{\pgfqpoint{10.712216in}{5.509381in}}{\pgfqpoint{10.707826in}{5.498782in}}{\pgfqpoint{10.707826in}{5.487732in}}%
\pgfpathcurveto{\pgfqpoint{10.707826in}{5.476682in}}{\pgfqpoint{10.712216in}{5.466083in}}{\pgfqpoint{10.720030in}{5.458269in}}%
\pgfpathcurveto{\pgfqpoint{10.727843in}{5.450456in}}{\pgfqpoint{10.738442in}{5.446066in}}{\pgfqpoint{10.749492in}{5.446066in}}%
\pgfpathlineto{\pgfqpoint{10.749492in}{5.446066in}}%
\pgfpathclose%
\pgfusepath{stroke}%
\end{pgfscope}%
\begin{pgfscope}%
\pgfpathrectangle{\pgfqpoint{7.394209in}{0.375000in}}{\pgfqpoint{6.356833in}{5.175000in}}%
\pgfusepath{clip}%
\pgfsetbuttcap%
\pgfsetroundjoin%
\pgfsetlinewidth{1.003750pt}%
\definecolor{currentstroke}{rgb}{0.827451,0.827451,0.827451}%
\pgfsetstrokecolor{currentstroke}%
\pgfsetdash{}{0pt}%
\pgfpathmoveto{\pgfqpoint{10.630478in}{5.508333in}}%
\pgfpathcurveto{\pgfqpoint{10.641528in}{5.508333in}}{\pgfqpoint{10.652127in}{5.512724in}}{\pgfqpoint{10.659941in}{5.520537in}}%
\pgfpathcurveto{\pgfqpoint{10.667754in}{5.528351in}}{\pgfqpoint{10.672145in}{5.538950in}}{\pgfqpoint{10.672145in}{5.550000in}}%
\pgfpathcurveto{\pgfqpoint{10.672145in}{5.561050in}}{\pgfqpoint{10.667754in}{5.571649in}}{\pgfqpoint{10.659941in}{5.579463in}}%
\pgfpathcurveto{\pgfqpoint{10.652127in}{5.587276in}}{\pgfqpoint{10.641528in}{5.591667in}}{\pgfqpoint{10.630478in}{5.591667in}}%
\pgfpathcurveto{\pgfqpoint{10.619428in}{5.591667in}}{\pgfqpoint{10.608829in}{5.587276in}}{\pgfqpoint{10.601015in}{5.579463in}}%
\pgfpathcurveto{\pgfqpoint{10.593202in}{5.571649in}}{\pgfqpoint{10.588811in}{5.561050in}}{\pgfqpoint{10.588811in}{5.550000in}}%
\pgfpathcurveto{\pgfqpoint{10.588811in}{5.538950in}}{\pgfqpoint{10.593202in}{5.528351in}}{\pgfqpoint{10.601015in}{5.520537in}}%
\pgfpathcurveto{\pgfqpoint{10.608829in}{5.512724in}}{\pgfqpoint{10.619428in}{5.508333in}}{\pgfqpoint{10.630478in}{5.508333in}}%
\pgfpathlineto{\pgfqpoint{10.630478in}{5.508333in}}%
\pgfpathclose%
\pgfusepath{stroke}%
\end{pgfscope}%
\begin{pgfscope}%
\pgfpathrectangle{\pgfqpoint{7.394209in}{0.375000in}}{\pgfqpoint{6.356833in}{5.175000in}}%
\pgfusepath{clip}%
\pgfsetbuttcap%
\pgfsetroundjoin%
\pgfsetlinewidth{1.003750pt}%
\definecolor{currentstroke}{rgb}{0.827451,0.827451,0.827451}%
\pgfsetstrokecolor{currentstroke}%
\pgfsetdash{}{0pt}%
\pgfpathmoveto{\pgfqpoint{9.935511in}{4.722830in}}%
\pgfpathcurveto{\pgfqpoint{9.946561in}{4.722830in}}{\pgfqpoint{9.957160in}{4.727220in}}{\pgfqpoint{9.964974in}{4.735034in}}%
\pgfpathcurveto{\pgfqpoint{9.972787in}{4.742848in}}{\pgfqpoint{9.977178in}{4.753447in}}{\pgfqpoint{9.977178in}{4.764497in}}%
\pgfpathcurveto{\pgfqpoint{9.977178in}{4.775547in}}{\pgfqpoint{9.972787in}{4.786146in}}{\pgfqpoint{9.964974in}{4.793959in}}%
\pgfpathcurveto{\pgfqpoint{9.957160in}{4.801773in}}{\pgfqpoint{9.946561in}{4.806163in}}{\pgfqpoint{9.935511in}{4.806163in}}%
\pgfpathcurveto{\pgfqpoint{9.924461in}{4.806163in}}{\pgfqpoint{9.913862in}{4.801773in}}{\pgfqpoint{9.906048in}{4.793959in}}%
\pgfpathcurveto{\pgfqpoint{9.898235in}{4.786146in}}{\pgfqpoint{9.893844in}{4.775547in}}{\pgfqpoint{9.893844in}{4.764497in}}%
\pgfpathcurveto{\pgfqpoint{9.893844in}{4.753447in}}{\pgfqpoint{9.898235in}{4.742848in}}{\pgfqpoint{9.906048in}{4.735034in}}%
\pgfpathcurveto{\pgfqpoint{9.913862in}{4.727220in}}{\pgfqpoint{9.924461in}{4.722830in}}{\pgfqpoint{9.935511in}{4.722830in}}%
\pgfpathlineto{\pgfqpoint{9.935511in}{4.722830in}}%
\pgfpathclose%
\pgfusepath{stroke}%
\end{pgfscope}%
\begin{pgfscope}%
\pgfpathrectangle{\pgfqpoint{7.394209in}{0.375000in}}{\pgfqpoint{6.356833in}{5.175000in}}%
\pgfusepath{clip}%
\pgfsetbuttcap%
\pgfsetroundjoin%
\pgfsetlinewidth{1.003750pt}%
\definecolor{currentstroke}{rgb}{0.827451,0.827451,0.827451}%
\pgfsetstrokecolor{currentstroke}%
\pgfsetdash{}{0pt}%
\pgfpathmoveto{\pgfqpoint{12.084123in}{5.439756in}}%
\pgfpathcurveto{\pgfqpoint{12.095173in}{5.439756in}}{\pgfqpoint{12.105772in}{5.444146in}}{\pgfqpoint{12.113586in}{5.451959in}}%
\pgfpathcurveto{\pgfqpoint{12.121399in}{5.459773in}}{\pgfqpoint{12.125790in}{5.470372in}}{\pgfqpoint{12.125790in}{5.481422in}}%
\pgfpathcurveto{\pgfqpoint{12.125790in}{5.492472in}}{\pgfqpoint{12.121399in}{5.503071in}}{\pgfqpoint{12.113586in}{5.510885in}}%
\pgfpathcurveto{\pgfqpoint{12.105772in}{5.518699in}}{\pgfqpoint{12.095173in}{5.523089in}}{\pgfqpoint{12.084123in}{5.523089in}}%
\pgfpathcurveto{\pgfqpoint{12.073073in}{5.523089in}}{\pgfqpoint{12.062474in}{5.518699in}}{\pgfqpoint{12.054660in}{5.510885in}}%
\pgfpathcurveto{\pgfqpoint{12.046847in}{5.503071in}}{\pgfqpoint{12.042456in}{5.492472in}}{\pgfqpoint{12.042456in}{5.481422in}}%
\pgfpathcurveto{\pgfqpoint{12.042456in}{5.470372in}}{\pgfqpoint{12.046847in}{5.459773in}}{\pgfqpoint{12.054660in}{5.451959in}}%
\pgfpathcurveto{\pgfqpoint{12.062474in}{5.444146in}}{\pgfqpoint{12.073073in}{5.439756in}}{\pgfqpoint{12.084123in}{5.439756in}}%
\pgfpathlineto{\pgfqpoint{12.084123in}{5.439756in}}%
\pgfpathclose%
\pgfusepath{stroke}%
\end{pgfscope}%
\begin{pgfscope}%
\pgfpathrectangle{\pgfqpoint{7.394209in}{0.375000in}}{\pgfqpoint{6.356833in}{5.175000in}}%
\pgfusepath{clip}%
\pgfsetbuttcap%
\pgfsetroundjoin%
\pgfsetlinewidth{1.003750pt}%
\definecolor{currentstroke}{rgb}{0.827451,0.827451,0.827451}%
\pgfsetstrokecolor{currentstroke}%
\pgfsetdash{}{0pt}%
\pgfpathmoveto{\pgfqpoint{9.002044in}{1.804535in}}%
\pgfpathcurveto{\pgfqpoint{9.013095in}{1.804535in}}{\pgfqpoint{9.023694in}{1.808926in}}{\pgfqpoint{9.031507in}{1.816739in}}%
\pgfpathcurveto{\pgfqpoint{9.039321in}{1.824553in}}{\pgfqpoint{9.043711in}{1.835152in}}{\pgfqpoint{9.043711in}{1.846202in}}%
\pgfpathcurveto{\pgfqpoint{9.043711in}{1.857252in}}{\pgfqpoint{9.039321in}{1.867851in}}{\pgfqpoint{9.031507in}{1.875665in}}%
\pgfpathcurveto{\pgfqpoint{9.023694in}{1.883479in}}{\pgfqpoint{9.013095in}{1.887869in}}{\pgfqpoint{9.002044in}{1.887869in}}%
\pgfpathcurveto{\pgfqpoint{8.990994in}{1.887869in}}{\pgfqpoint{8.980395in}{1.883479in}}{\pgfqpoint{8.972582in}{1.875665in}}%
\pgfpathcurveto{\pgfqpoint{8.964768in}{1.867851in}}{\pgfqpoint{8.960378in}{1.857252in}}{\pgfqpoint{8.960378in}{1.846202in}}%
\pgfpathcurveto{\pgfqpoint{8.960378in}{1.835152in}}{\pgfqpoint{8.964768in}{1.824553in}}{\pgfqpoint{8.972582in}{1.816739in}}%
\pgfpathcurveto{\pgfqpoint{8.980395in}{1.808926in}}{\pgfqpoint{8.990994in}{1.804535in}}{\pgfqpoint{9.002044in}{1.804535in}}%
\pgfpathlineto{\pgfqpoint{9.002044in}{1.804535in}}%
\pgfpathclose%
\pgfusepath{stroke}%
\end{pgfscope}%
\begin{pgfscope}%
\pgfpathrectangle{\pgfqpoint{7.394209in}{0.375000in}}{\pgfqpoint{6.356833in}{5.175000in}}%
\pgfusepath{clip}%
\pgfsetbuttcap%
\pgfsetroundjoin%
\pgfsetlinewidth{1.003750pt}%
\definecolor{currentstroke}{rgb}{0.827451,0.827451,0.827451}%
\pgfsetstrokecolor{currentstroke}%
\pgfsetdash{}{0pt}%
\pgfpathmoveto{\pgfqpoint{9.097304in}{1.790848in}}%
\pgfpathcurveto{\pgfqpoint{9.108354in}{1.790848in}}{\pgfqpoint{9.118953in}{1.795238in}}{\pgfqpoint{9.126767in}{1.803052in}}%
\pgfpathcurveto{\pgfqpoint{9.134580in}{1.810866in}}{\pgfqpoint{9.138971in}{1.821465in}}{\pgfqpoint{9.138971in}{1.832515in}}%
\pgfpathcurveto{\pgfqpoint{9.138971in}{1.843565in}}{\pgfqpoint{9.134580in}{1.854164in}}{\pgfqpoint{9.126767in}{1.861978in}}%
\pgfpathcurveto{\pgfqpoint{9.118953in}{1.869791in}}{\pgfqpoint{9.108354in}{1.874181in}}{\pgfqpoint{9.097304in}{1.874181in}}%
\pgfpathcurveto{\pgfqpoint{9.086254in}{1.874181in}}{\pgfqpoint{9.075655in}{1.869791in}}{\pgfqpoint{9.067841in}{1.861978in}}%
\pgfpathcurveto{\pgfqpoint{9.060028in}{1.854164in}}{\pgfqpoint{9.055637in}{1.843565in}}{\pgfqpoint{9.055637in}{1.832515in}}%
\pgfpathcurveto{\pgfqpoint{9.055637in}{1.821465in}}{\pgfqpoint{9.060028in}{1.810866in}}{\pgfqpoint{9.067841in}{1.803052in}}%
\pgfpathcurveto{\pgfqpoint{9.075655in}{1.795238in}}{\pgfqpoint{9.086254in}{1.790848in}}{\pgfqpoint{9.097304in}{1.790848in}}%
\pgfpathlineto{\pgfqpoint{9.097304in}{1.790848in}}%
\pgfpathclose%
\pgfusepath{stroke}%
\end{pgfscope}%
\begin{pgfscope}%
\pgfpathrectangle{\pgfqpoint{7.394209in}{0.375000in}}{\pgfqpoint{6.356833in}{5.175000in}}%
\pgfusepath{clip}%
\pgfsetbuttcap%
\pgfsetroundjoin%
\pgfsetlinewidth{1.003750pt}%
\definecolor{currentstroke}{rgb}{0.827451,0.827451,0.827451}%
\pgfsetstrokecolor{currentstroke}%
\pgfsetdash{}{0pt}%
\pgfpathmoveto{\pgfqpoint{10.165500in}{3.364671in}}%
\pgfpathcurveto{\pgfqpoint{10.176550in}{3.364671in}}{\pgfqpoint{10.187149in}{3.369061in}}{\pgfqpoint{10.194963in}{3.376875in}}%
\pgfpathcurveto{\pgfqpoint{10.202776in}{3.384688in}}{\pgfqpoint{10.207166in}{3.395287in}}{\pgfqpoint{10.207166in}{3.406337in}}%
\pgfpathcurveto{\pgfqpoint{10.207166in}{3.417388in}}{\pgfqpoint{10.202776in}{3.427987in}}{\pgfqpoint{10.194963in}{3.435800in}}%
\pgfpathcurveto{\pgfqpoint{10.187149in}{3.443614in}}{\pgfqpoint{10.176550in}{3.448004in}}{\pgfqpoint{10.165500in}{3.448004in}}%
\pgfpathcurveto{\pgfqpoint{10.154450in}{3.448004in}}{\pgfqpoint{10.143851in}{3.443614in}}{\pgfqpoint{10.136037in}{3.435800in}}%
\pgfpathcurveto{\pgfqpoint{10.128223in}{3.427987in}}{\pgfqpoint{10.123833in}{3.417388in}}{\pgfqpoint{10.123833in}{3.406337in}}%
\pgfpathcurveto{\pgfqpoint{10.123833in}{3.395287in}}{\pgfqpoint{10.128223in}{3.384688in}}{\pgfqpoint{10.136037in}{3.376875in}}%
\pgfpathcurveto{\pgfqpoint{10.143851in}{3.369061in}}{\pgfqpoint{10.154450in}{3.364671in}}{\pgfqpoint{10.165500in}{3.364671in}}%
\pgfpathlineto{\pgfqpoint{10.165500in}{3.364671in}}%
\pgfpathclose%
\pgfusepath{stroke}%
\end{pgfscope}%
\begin{pgfscope}%
\pgfpathrectangle{\pgfqpoint{7.394209in}{0.375000in}}{\pgfqpoint{6.356833in}{5.175000in}}%
\pgfusepath{clip}%
\pgfsetbuttcap%
\pgfsetroundjoin%
\pgfsetlinewidth{1.003750pt}%
\definecolor{currentstroke}{rgb}{0.827451,0.827451,0.827451}%
\pgfsetstrokecolor{currentstroke}%
\pgfsetdash{}{0pt}%
\pgfpathmoveto{\pgfqpoint{10.365111in}{3.654716in}}%
\pgfpathcurveto{\pgfqpoint{10.376162in}{3.654716in}}{\pgfqpoint{10.386761in}{3.659106in}}{\pgfqpoint{10.394574in}{3.666920in}}%
\pgfpathcurveto{\pgfqpoint{10.402388in}{3.674733in}}{\pgfqpoint{10.406778in}{3.685332in}}{\pgfqpoint{10.406778in}{3.696383in}}%
\pgfpathcurveto{\pgfqpoint{10.406778in}{3.707433in}}{\pgfqpoint{10.402388in}{3.718032in}}{\pgfqpoint{10.394574in}{3.725845in}}%
\pgfpathcurveto{\pgfqpoint{10.386761in}{3.733659in}}{\pgfqpoint{10.376162in}{3.738049in}}{\pgfqpoint{10.365111in}{3.738049in}}%
\pgfpathcurveto{\pgfqpoint{10.354061in}{3.738049in}}{\pgfqpoint{10.343462in}{3.733659in}}{\pgfqpoint{10.335649in}{3.725845in}}%
\pgfpathcurveto{\pgfqpoint{10.327835in}{3.718032in}}{\pgfqpoint{10.323445in}{3.707433in}}{\pgfqpoint{10.323445in}{3.696383in}}%
\pgfpathcurveto{\pgfqpoint{10.323445in}{3.685332in}}{\pgfqpoint{10.327835in}{3.674733in}}{\pgfqpoint{10.335649in}{3.666920in}}%
\pgfpathcurveto{\pgfqpoint{10.343462in}{3.659106in}}{\pgfqpoint{10.354061in}{3.654716in}}{\pgfqpoint{10.365111in}{3.654716in}}%
\pgfpathlineto{\pgfqpoint{10.365111in}{3.654716in}}%
\pgfpathclose%
\pgfusepath{stroke}%
\end{pgfscope}%
\begin{pgfscope}%
\pgfpathrectangle{\pgfqpoint{7.394209in}{0.375000in}}{\pgfqpoint{6.356833in}{5.175000in}}%
\pgfusepath{clip}%
\pgfsetbuttcap%
\pgfsetroundjoin%
\pgfsetlinewidth{1.003750pt}%
\definecolor{currentstroke}{rgb}{0.827451,0.827451,0.827451}%
\pgfsetstrokecolor{currentstroke}%
\pgfsetdash{}{0pt}%
\pgfpathmoveto{\pgfqpoint{7.886478in}{0.686383in}}%
\pgfpathcurveto{\pgfqpoint{7.897528in}{0.686383in}}{\pgfqpoint{7.908127in}{0.690774in}}{\pgfqpoint{7.915941in}{0.698587in}}%
\pgfpathcurveto{\pgfqpoint{7.923755in}{0.706401in}}{\pgfqpoint{7.928145in}{0.717000in}}{\pgfqpoint{7.928145in}{0.728050in}}%
\pgfpathcurveto{\pgfqpoint{7.928145in}{0.739100in}}{\pgfqpoint{7.923755in}{0.749699in}}{\pgfqpoint{7.915941in}{0.757513in}}%
\pgfpathcurveto{\pgfqpoint{7.908127in}{0.765327in}}{\pgfqpoint{7.897528in}{0.769717in}}{\pgfqpoint{7.886478in}{0.769717in}}%
\pgfpathcurveto{\pgfqpoint{7.875428in}{0.769717in}}{\pgfqpoint{7.864829in}{0.765327in}}{\pgfqpoint{7.857016in}{0.757513in}}%
\pgfpathcurveto{\pgfqpoint{7.849202in}{0.749699in}}{\pgfqpoint{7.844812in}{0.739100in}}{\pgfqpoint{7.844812in}{0.728050in}}%
\pgfpathcurveto{\pgfqpoint{7.844812in}{0.717000in}}{\pgfqpoint{7.849202in}{0.706401in}}{\pgfqpoint{7.857016in}{0.698587in}}%
\pgfpathcurveto{\pgfqpoint{7.864829in}{0.690774in}}{\pgfqpoint{7.875428in}{0.686383in}}{\pgfqpoint{7.886478in}{0.686383in}}%
\pgfpathlineto{\pgfqpoint{7.886478in}{0.686383in}}%
\pgfpathclose%
\pgfusepath{stroke}%
\end{pgfscope}%
\begin{pgfscope}%
\pgfpathrectangle{\pgfqpoint{7.394209in}{0.375000in}}{\pgfqpoint{6.356833in}{5.175000in}}%
\pgfusepath{clip}%
\pgfsetbuttcap%
\pgfsetroundjoin%
\pgfsetlinewidth{1.003750pt}%
\definecolor{currentstroke}{rgb}{0.827451,0.827451,0.827451}%
\pgfsetstrokecolor{currentstroke}%
\pgfsetdash{}{0pt}%
\pgfpathmoveto{\pgfqpoint{12.473494in}{5.460814in}}%
\pgfpathcurveto{\pgfqpoint{12.484544in}{5.460814in}}{\pgfqpoint{12.495143in}{5.465204in}}{\pgfqpoint{12.502957in}{5.473018in}}%
\pgfpathcurveto{\pgfqpoint{12.510770in}{5.480832in}}{\pgfqpoint{12.515160in}{5.491431in}}{\pgfqpoint{12.515160in}{5.502481in}}%
\pgfpathcurveto{\pgfqpoint{12.515160in}{5.513531in}}{\pgfqpoint{12.510770in}{5.524130in}}{\pgfqpoint{12.502957in}{5.531944in}}%
\pgfpathcurveto{\pgfqpoint{12.495143in}{5.539757in}}{\pgfqpoint{12.484544in}{5.544148in}}{\pgfqpoint{12.473494in}{5.544148in}}%
\pgfpathcurveto{\pgfqpoint{12.462444in}{5.544148in}}{\pgfqpoint{12.451845in}{5.539757in}}{\pgfqpoint{12.444031in}{5.531944in}}%
\pgfpathcurveto{\pgfqpoint{12.436217in}{5.524130in}}{\pgfqpoint{12.431827in}{5.513531in}}{\pgfqpoint{12.431827in}{5.502481in}}%
\pgfpathcurveto{\pgfqpoint{12.431827in}{5.491431in}}{\pgfqpoint{12.436217in}{5.480832in}}{\pgfqpoint{12.444031in}{5.473018in}}%
\pgfpathcurveto{\pgfqpoint{12.451845in}{5.465204in}}{\pgfqpoint{12.462444in}{5.460814in}}{\pgfqpoint{12.473494in}{5.460814in}}%
\pgfpathlineto{\pgfqpoint{12.473494in}{5.460814in}}%
\pgfpathclose%
\pgfusepath{stroke}%
\end{pgfscope}%
\begin{pgfscope}%
\pgfpathrectangle{\pgfqpoint{7.394209in}{0.375000in}}{\pgfqpoint{6.356833in}{5.175000in}}%
\pgfusepath{clip}%
\pgfsetbuttcap%
\pgfsetroundjoin%
\pgfsetlinewidth{1.003750pt}%
\definecolor{currentstroke}{rgb}{0.827451,0.827451,0.827451}%
\pgfsetstrokecolor{currentstroke}%
\pgfsetdash{}{0pt}%
\pgfpathmoveto{\pgfqpoint{8.661756in}{1.791483in}}%
\pgfpathcurveto{\pgfqpoint{8.672806in}{1.791483in}}{\pgfqpoint{8.683405in}{1.795873in}}{\pgfqpoint{8.691219in}{1.803687in}}%
\pgfpathcurveto{\pgfqpoint{8.699033in}{1.811500in}}{\pgfqpoint{8.703423in}{1.822099in}}{\pgfqpoint{8.703423in}{1.833149in}}%
\pgfpathcurveto{\pgfqpoint{8.703423in}{1.844199in}}{\pgfqpoint{8.699033in}{1.854798in}}{\pgfqpoint{8.691219in}{1.862612in}}%
\pgfpathcurveto{\pgfqpoint{8.683405in}{1.870426in}}{\pgfqpoint{8.672806in}{1.874816in}}{\pgfqpoint{8.661756in}{1.874816in}}%
\pgfpathcurveto{\pgfqpoint{8.650706in}{1.874816in}}{\pgfqpoint{8.640107in}{1.870426in}}{\pgfqpoint{8.632294in}{1.862612in}}%
\pgfpathcurveto{\pgfqpoint{8.624480in}{1.854798in}}{\pgfqpoint{8.620090in}{1.844199in}}{\pgfqpoint{8.620090in}{1.833149in}}%
\pgfpathcurveto{\pgfqpoint{8.620090in}{1.822099in}}{\pgfqpoint{8.624480in}{1.811500in}}{\pgfqpoint{8.632294in}{1.803687in}}%
\pgfpathcurveto{\pgfqpoint{8.640107in}{1.795873in}}{\pgfqpoint{8.650706in}{1.791483in}}{\pgfqpoint{8.661756in}{1.791483in}}%
\pgfpathlineto{\pgfqpoint{8.661756in}{1.791483in}}%
\pgfpathclose%
\pgfusepath{stroke}%
\end{pgfscope}%
\begin{pgfscope}%
\pgfpathrectangle{\pgfqpoint{7.394209in}{0.375000in}}{\pgfqpoint{6.356833in}{5.175000in}}%
\pgfusepath{clip}%
\pgfsetbuttcap%
\pgfsetroundjoin%
\pgfsetlinewidth{1.003750pt}%
\definecolor{currentstroke}{rgb}{0.827451,0.827451,0.827451}%
\pgfsetstrokecolor{currentstroke}%
\pgfsetdash{}{0pt}%
\pgfpathmoveto{\pgfqpoint{8.617774in}{3.093013in}}%
\pgfpathcurveto{\pgfqpoint{8.628824in}{3.093013in}}{\pgfqpoint{8.639423in}{3.097403in}}{\pgfqpoint{8.647237in}{3.105217in}}%
\pgfpathcurveto{\pgfqpoint{8.655051in}{3.113030in}}{\pgfqpoint{8.659441in}{3.123630in}}{\pgfqpoint{8.659441in}{3.134680in}}%
\pgfpathcurveto{\pgfqpoint{8.659441in}{3.145730in}}{\pgfqpoint{8.655051in}{3.156329in}}{\pgfqpoint{8.647237in}{3.164142in}}%
\pgfpathcurveto{\pgfqpoint{8.639423in}{3.171956in}}{\pgfqpoint{8.628824in}{3.176346in}}{\pgfqpoint{8.617774in}{3.176346in}}%
\pgfpathcurveto{\pgfqpoint{8.606724in}{3.176346in}}{\pgfqpoint{8.596125in}{3.171956in}}{\pgfqpoint{8.588311in}{3.164142in}}%
\pgfpathcurveto{\pgfqpoint{8.580498in}{3.156329in}}{\pgfqpoint{8.576107in}{3.145730in}}{\pgfqpoint{8.576107in}{3.134680in}}%
\pgfpathcurveto{\pgfqpoint{8.576107in}{3.123630in}}{\pgfqpoint{8.580498in}{3.113030in}}{\pgfqpoint{8.588311in}{3.105217in}}%
\pgfpathcurveto{\pgfqpoint{8.596125in}{3.097403in}}{\pgfqpoint{8.606724in}{3.093013in}}{\pgfqpoint{8.617774in}{3.093013in}}%
\pgfpathlineto{\pgfqpoint{8.617774in}{3.093013in}}%
\pgfpathclose%
\pgfusepath{stroke}%
\end{pgfscope}%
\begin{pgfscope}%
\pgfpathrectangle{\pgfqpoint{7.394209in}{0.375000in}}{\pgfqpoint{6.356833in}{5.175000in}}%
\pgfusepath{clip}%
\pgfsetbuttcap%
\pgfsetroundjoin%
\pgfsetlinewidth{1.003750pt}%
\definecolor{currentstroke}{rgb}{0.827451,0.827451,0.827451}%
\pgfsetstrokecolor{currentstroke}%
\pgfsetdash{}{0pt}%
\pgfpathmoveto{\pgfqpoint{12.272658in}{5.460846in}}%
\pgfpathcurveto{\pgfqpoint{12.283708in}{5.460846in}}{\pgfqpoint{12.294307in}{5.465236in}}{\pgfqpoint{12.302121in}{5.473050in}}%
\pgfpathcurveto{\pgfqpoint{12.309935in}{5.480864in}}{\pgfqpoint{12.314325in}{5.491463in}}{\pgfqpoint{12.314325in}{5.502513in}}%
\pgfpathcurveto{\pgfqpoint{12.314325in}{5.513563in}}{\pgfqpoint{12.309935in}{5.524162in}}{\pgfqpoint{12.302121in}{5.531976in}}%
\pgfpathcurveto{\pgfqpoint{12.294307in}{5.539789in}}{\pgfqpoint{12.283708in}{5.544179in}}{\pgfqpoint{12.272658in}{5.544179in}}%
\pgfpathcurveto{\pgfqpoint{12.261608in}{5.544179in}}{\pgfqpoint{12.251009in}{5.539789in}}{\pgfqpoint{12.243196in}{5.531976in}}%
\pgfpathcurveto{\pgfqpoint{12.235382in}{5.524162in}}{\pgfqpoint{12.230992in}{5.513563in}}{\pgfqpoint{12.230992in}{5.502513in}}%
\pgfpathcurveto{\pgfqpoint{12.230992in}{5.491463in}}{\pgfqpoint{12.235382in}{5.480864in}}{\pgfqpoint{12.243196in}{5.473050in}}%
\pgfpathcurveto{\pgfqpoint{12.251009in}{5.465236in}}{\pgfqpoint{12.261608in}{5.460846in}}{\pgfqpoint{12.272658in}{5.460846in}}%
\pgfpathlineto{\pgfqpoint{12.272658in}{5.460846in}}%
\pgfpathclose%
\pgfusepath{stroke}%
\end{pgfscope}%
\begin{pgfscope}%
\pgfpathrectangle{\pgfqpoint{7.394209in}{0.375000in}}{\pgfqpoint{6.356833in}{5.175000in}}%
\pgfusepath{clip}%
\pgfsetbuttcap%
\pgfsetroundjoin%
\pgfsetlinewidth{1.003750pt}%
\definecolor{currentstroke}{rgb}{0.827451,0.827451,0.827451}%
\pgfsetstrokecolor{currentstroke}%
\pgfsetdash{}{0pt}%
\pgfpathmoveto{\pgfqpoint{11.487180in}{5.067284in}}%
\pgfpathcurveto{\pgfqpoint{11.498230in}{5.067284in}}{\pgfqpoint{11.508829in}{5.071674in}}{\pgfqpoint{11.516643in}{5.079488in}}%
\pgfpathcurveto{\pgfqpoint{11.524456in}{5.087301in}}{\pgfqpoint{11.528847in}{5.097900in}}{\pgfqpoint{11.528847in}{5.108951in}}%
\pgfpathcurveto{\pgfqpoint{11.528847in}{5.120001in}}{\pgfqpoint{11.524456in}{5.130600in}}{\pgfqpoint{11.516643in}{5.138413in}}%
\pgfpathcurveto{\pgfqpoint{11.508829in}{5.146227in}}{\pgfqpoint{11.498230in}{5.150617in}}{\pgfqpoint{11.487180in}{5.150617in}}%
\pgfpathcurveto{\pgfqpoint{11.476130in}{5.150617in}}{\pgfqpoint{11.465531in}{5.146227in}}{\pgfqpoint{11.457717in}{5.138413in}}%
\pgfpathcurveto{\pgfqpoint{11.449903in}{5.130600in}}{\pgfqpoint{11.445513in}{5.120001in}}{\pgfqpoint{11.445513in}{5.108951in}}%
\pgfpathcurveto{\pgfqpoint{11.445513in}{5.097900in}}{\pgfqpoint{11.449903in}{5.087301in}}{\pgfqpoint{11.457717in}{5.079488in}}%
\pgfpathcurveto{\pgfqpoint{11.465531in}{5.071674in}}{\pgfqpoint{11.476130in}{5.067284in}}{\pgfqpoint{11.487180in}{5.067284in}}%
\pgfpathlineto{\pgfqpoint{11.487180in}{5.067284in}}%
\pgfpathclose%
\pgfusepath{stroke}%
\end{pgfscope}%
\begin{pgfscope}%
\pgfpathrectangle{\pgfqpoint{7.394209in}{0.375000in}}{\pgfqpoint{6.356833in}{5.175000in}}%
\pgfusepath{clip}%
\pgfsetbuttcap%
\pgfsetroundjoin%
\pgfsetlinewidth{1.003750pt}%
\definecolor{currentstroke}{rgb}{0.827451,0.827451,0.827451}%
\pgfsetstrokecolor{currentstroke}%
\pgfsetdash{}{0pt}%
\pgfpathmoveto{\pgfqpoint{10.300595in}{3.816946in}}%
\pgfpathcurveto{\pgfqpoint{10.311645in}{3.816946in}}{\pgfqpoint{10.322244in}{3.821336in}}{\pgfqpoint{10.330058in}{3.829150in}}%
\pgfpathcurveto{\pgfqpoint{10.337872in}{3.836963in}}{\pgfqpoint{10.342262in}{3.847562in}}{\pgfqpoint{10.342262in}{3.858612in}}%
\pgfpathcurveto{\pgfqpoint{10.342262in}{3.869663in}}{\pgfqpoint{10.337872in}{3.880262in}}{\pgfqpoint{10.330058in}{3.888075in}}%
\pgfpathcurveto{\pgfqpoint{10.322244in}{3.895889in}}{\pgfqpoint{10.311645in}{3.900279in}}{\pgfqpoint{10.300595in}{3.900279in}}%
\pgfpathcurveto{\pgfqpoint{10.289545in}{3.900279in}}{\pgfqpoint{10.278946in}{3.895889in}}{\pgfqpoint{10.271132in}{3.888075in}}%
\pgfpathcurveto{\pgfqpoint{10.263319in}{3.880262in}}{\pgfqpoint{10.258929in}{3.869663in}}{\pgfqpoint{10.258929in}{3.858612in}}%
\pgfpathcurveto{\pgfqpoint{10.258929in}{3.847562in}}{\pgfqpoint{10.263319in}{3.836963in}}{\pgfqpoint{10.271132in}{3.829150in}}%
\pgfpathcurveto{\pgfqpoint{10.278946in}{3.821336in}}{\pgfqpoint{10.289545in}{3.816946in}}{\pgfqpoint{10.300595in}{3.816946in}}%
\pgfpathlineto{\pgfqpoint{10.300595in}{3.816946in}}%
\pgfpathclose%
\pgfusepath{stroke}%
\end{pgfscope}%
\begin{pgfscope}%
\pgfpathrectangle{\pgfqpoint{7.394209in}{0.375000in}}{\pgfqpoint{6.356833in}{5.175000in}}%
\pgfusepath{clip}%
\pgfsetbuttcap%
\pgfsetroundjoin%
\pgfsetlinewidth{1.003750pt}%
\definecolor{currentstroke}{rgb}{0.827451,0.827451,0.827451}%
\pgfsetstrokecolor{currentstroke}%
\pgfsetdash{}{0pt}%
\pgfpathmoveto{\pgfqpoint{9.220234in}{2.067669in}}%
\pgfpathcurveto{\pgfqpoint{9.231284in}{2.067669in}}{\pgfqpoint{9.241883in}{2.072059in}}{\pgfqpoint{9.249696in}{2.079873in}}%
\pgfpathcurveto{\pgfqpoint{9.257510in}{2.087686in}}{\pgfqpoint{9.261900in}{2.098286in}}{\pgfqpoint{9.261900in}{2.109336in}}%
\pgfpathcurveto{\pgfqpoint{9.261900in}{2.120386in}}{\pgfqpoint{9.257510in}{2.130985in}}{\pgfqpoint{9.249696in}{2.138798in}}%
\pgfpathcurveto{\pgfqpoint{9.241883in}{2.146612in}}{\pgfqpoint{9.231284in}{2.151002in}}{\pgfqpoint{9.220234in}{2.151002in}}%
\pgfpathcurveto{\pgfqpoint{9.209183in}{2.151002in}}{\pgfqpoint{9.198584in}{2.146612in}}{\pgfqpoint{9.190771in}{2.138798in}}%
\pgfpathcurveto{\pgfqpoint{9.182957in}{2.130985in}}{\pgfqpoint{9.178567in}{2.120386in}}{\pgfqpoint{9.178567in}{2.109336in}}%
\pgfpathcurveto{\pgfqpoint{9.178567in}{2.098286in}}{\pgfqpoint{9.182957in}{2.087686in}}{\pgfqpoint{9.190771in}{2.079873in}}%
\pgfpathcurveto{\pgfqpoint{9.198584in}{2.072059in}}{\pgfqpoint{9.209183in}{2.067669in}}{\pgfqpoint{9.220234in}{2.067669in}}%
\pgfpathlineto{\pgfqpoint{9.220234in}{2.067669in}}%
\pgfpathclose%
\pgfusepath{stroke}%
\end{pgfscope}%
\begin{pgfscope}%
\pgfpathrectangle{\pgfqpoint{7.394209in}{0.375000in}}{\pgfqpoint{6.356833in}{5.175000in}}%
\pgfusepath{clip}%
\pgfsetbuttcap%
\pgfsetroundjoin%
\pgfsetlinewidth{1.003750pt}%
\definecolor{currentstroke}{rgb}{0.827451,0.827451,0.827451}%
\pgfsetstrokecolor{currentstroke}%
\pgfsetdash{}{0pt}%
\pgfpathmoveto{\pgfqpoint{11.784934in}{5.190475in}}%
\pgfpathcurveto{\pgfqpoint{11.795984in}{5.190475in}}{\pgfqpoint{11.806583in}{5.194865in}}{\pgfqpoint{11.814397in}{5.202679in}}%
\pgfpathcurveto{\pgfqpoint{11.822210in}{5.210492in}}{\pgfqpoint{11.826601in}{5.221091in}}{\pgfqpoint{11.826601in}{5.232141in}}%
\pgfpathcurveto{\pgfqpoint{11.826601in}{5.243192in}}{\pgfqpoint{11.822210in}{5.253791in}}{\pgfqpoint{11.814397in}{5.261604in}}%
\pgfpathcurveto{\pgfqpoint{11.806583in}{5.269418in}}{\pgfqpoint{11.795984in}{5.273808in}}{\pgfqpoint{11.784934in}{5.273808in}}%
\pgfpathcurveto{\pgfqpoint{11.773884in}{5.273808in}}{\pgfqpoint{11.763285in}{5.269418in}}{\pgfqpoint{11.755471in}{5.261604in}}%
\pgfpathcurveto{\pgfqpoint{11.747658in}{5.253791in}}{\pgfqpoint{11.743267in}{5.243192in}}{\pgfqpoint{11.743267in}{5.232141in}}%
\pgfpathcurveto{\pgfqpoint{11.743267in}{5.221091in}}{\pgfqpoint{11.747658in}{5.210492in}}{\pgfqpoint{11.755471in}{5.202679in}}%
\pgfpathcurveto{\pgfqpoint{11.763285in}{5.194865in}}{\pgfqpoint{11.773884in}{5.190475in}}{\pgfqpoint{11.784934in}{5.190475in}}%
\pgfpathlineto{\pgfqpoint{11.784934in}{5.190475in}}%
\pgfpathclose%
\pgfusepath{stroke}%
\end{pgfscope}%
\begin{pgfscope}%
\pgfpathrectangle{\pgfqpoint{7.394209in}{0.375000in}}{\pgfqpoint{6.356833in}{5.175000in}}%
\pgfusepath{clip}%
\pgfsetbuttcap%
\pgfsetroundjoin%
\pgfsetlinewidth{1.003750pt}%
\definecolor{currentstroke}{rgb}{0.827451,0.827451,0.827451}%
\pgfsetstrokecolor{currentstroke}%
\pgfsetdash{}{0pt}%
\pgfpathmoveto{\pgfqpoint{13.046953in}{5.460814in}}%
\pgfpathcurveto{\pgfqpoint{13.058003in}{5.460814in}}{\pgfqpoint{13.068602in}{5.465204in}}{\pgfqpoint{13.076416in}{5.473018in}}%
\pgfpathcurveto{\pgfqpoint{13.084229in}{5.480832in}}{\pgfqpoint{13.088620in}{5.491431in}}{\pgfqpoint{13.088620in}{5.502481in}}%
\pgfpathcurveto{\pgfqpoint{13.088620in}{5.513531in}}{\pgfqpoint{13.084229in}{5.524130in}}{\pgfqpoint{13.076416in}{5.531944in}}%
\pgfpathcurveto{\pgfqpoint{13.068602in}{5.539757in}}{\pgfqpoint{13.058003in}{5.544148in}}{\pgfqpoint{13.046953in}{5.544148in}}%
\pgfpathcurveto{\pgfqpoint{13.035903in}{5.544148in}}{\pgfqpoint{13.025304in}{5.539757in}}{\pgfqpoint{13.017490in}{5.531944in}}%
\pgfpathcurveto{\pgfqpoint{13.009676in}{5.524130in}}{\pgfqpoint{13.005286in}{5.513531in}}{\pgfqpoint{13.005286in}{5.502481in}}%
\pgfpathcurveto{\pgfqpoint{13.005286in}{5.491431in}}{\pgfqpoint{13.009676in}{5.480832in}}{\pgfqpoint{13.017490in}{5.473018in}}%
\pgfpathcurveto{\pgfqpoint{13.025304in}{5.465204in}}{\pgfqpoint{13.035903in}{5.460814in}}{\pgfqpoint{13.046953in}{5.460814in}}%
\pgfpathlineto{\pgfqpoint{13.046953in}{5.460814in}}%
\pgfpathclose%
\pgfusepath{stroke}%
\end{pgfscope}%
\begin{pgfscope}%
\pgfpathrectangle{\pgfqpoint{7.394209in}{0.375000in}}{\pgfqpoint{6.356833in}{5.175000in}}%
\pgfusepath{clip}%
\pgfsetbuttcap%
\pgfsetroundjoin%
\pgfsetlinewidth{1.003750pt}%
\definecolor{currentstroke}{rgb}{0.827451,0.827451,0.827451}%
\pgfsetstrokecolor{currentstroke}%
\pgfsetdash{}{0pt}%
\pgfpathmoveto{\pgfqpoint{11.664021in}{5.190475in}}%
\pgfpathcurveto{\pgfqpoint{11.675071in}{5.190475in}}{\pgfqpoint{11.685670in}{5.194865in}}{\pgfqpoint{11.693483in}{5.202679in}}%
\pgfpathcurveto{\pgfqpoint{11.701297in}{5.210492in}}{\pgfqpoint{11.705687in}{5.221091in}}{\pgfqpoint{11.705687in}{5.232141in}}%
\pgfpathcurveto{\pgfqpoint{11.705687in}{5.243192in}}{\pgfqpoint{11.701297in}{5.253791in}}{\pgfqpoint{11.693483in}{5.261604in}}%
\pgfpathcurveto{\pgfqpoint{11.685670in}{5.269418in}}{\pgfqpoint{11.675071in}{5.273808in}}{\pgfqpoint{11.664021in}{5.273808in}}%
\pgfpathcurveto{\pgfqpoint{11.652970in}{5.273808in}}{\pgfqpoint{11.642371in}{5.269418in}}{\pgfqpoint{11.634558in}{5.261604in}}%
\pgfpathcurveto{\pgfqpoint{11.626744in}{5.253791in}}{\pgfqpoint{11.622354in}{5.243192in}}{\pgfqpoint{11.622354in}{5.232141in}}%
\pgfpathcurveto{\pgfqpoint{11.622354in}{5.221091in}}{\pgfqpoint{11.626744in}{5.210492in}}{\pgfqpoint{11.634558in}{5.202679in}}%
\pgfpathcurveto{\pgfqpoint{11.642371in}{5.194865in}}{\pgfqpoint{11.652970in}{5.190475in}}{\pgfqpoint{11.664021in}{5.190475in}}%
\pgfpathlineto{\pgfqpoint{11.664021in}{5.190475in}}%
\pgfpathclose%
\pgfusepath{stroke}%
\end{pgfscope}%
\begin{pgfscope}%
\pgfpathrectangle{\pgfqpoint{7.394209in}{0.375000in}}{\pgfqpoint{6.356833in}{5.175000in}}%
\pgfusepath{clip}%
\pgfsetbuttcap%
\pgfsetroundjoin%
\pgfsetlinewidth{1.003750pt}%
\definecolor{currentstroke}{rgb}{0.827451,0.827451,0.827451}%
\pgfsetstrokecolor{currentstroke}%
\pgfsetdash{}{0pt}%
\pgfpathmoveto{\pgfqpoint{9.891746in}{2.733699in}}%
\pgfpathcurveto{\pgfqpoint{9.902796in}{2.733699in}}{\pgfqpoint{9.913395in}{2.738089in}}{\pgfqpoint{9.921208in}{2.745903in}}%
\pgfpathcurveto{\pgfqpoint{9.929022in}{2.753717in}}{\pgfqpoint{9.933412in}{2.764316in}}{\pgfqpoint{9.933412in}{2.775366in}}%
\pgfpathcurveto{\pgfqpoint{9.933412in}{2.786416in}}{\pgfqpoint{9.929022in}{2.797015in}}{\pgfqpoint{9.921208in}{2.804829in}}%
\pgfpathcurveto{\pgfqpoint{9.913395in}{2.812642in}}{\pgfqpoint{9.902796in}{2.817033in}}{\pgfqpoint{9.891746in}{2.817033in}}%
\pgfpathcurveto{\pgfqpoint{9.880695in}{2.817033in}}{\pgfqpoint{9.870096in}{2.812642in}}{\pgfqpoint{9.862283in}{2.804829in}}%
\pgfpathcurveto{\pgfqpoint{9.854469in}{2.797015in}}{\pgfqpoint{9.850079in}{2.786416in}}{\pgfqpoint{9.850079in}{2.775366in}}%
\pgfpathcurveto{\pgfqpoint{9.850079in}{2.764316in}}{\pgfqpoint{9.854469in}{2.753717in}}{\pgfqpoint{9.862283in}{2.745903in}}%
\pgfpathcurveto{\pgfqpoint{9.870096in}{2.738089in}}{\pgfqpoint{9.880695in}{2.733699in}}{\pgfqpoint{9.891746in}{2.733699in}}%
\pgfpathlineto{\pgfqpoint{9.891746in}{2.733699in}}%
\pgfpathclose%
\pgfusepath{stroke}%
\end{pgfscope}%
\begin{pgfscope}%
\pgfpathrectangle{\pgfqpoint{7.394209in}{0.375000in}}{\pgfqpoint{6.356833in}{5.175000in}}%
\pgfusepath{clip}%
\pgfsetbuttcap%
\pgfsetroundjoin%
\pgfsetlinewidth{1.003750pt}%
\definecolor{currentstroke}{rgb}{0.827451,0.827451,0.827451}%
\pgfsetstrokecolor{currentstroke}%
\pgfsetdash{}{0pt}%
\pgfpathmoveto{\pgfqpoint{11.332034in}{4.976915in}}%
\pgfpathcurveto{\pgfqpoint{11.343084in}{4.976915in}}{\pgfqpoint{11.353683in}{4.981305in}}{\pgfqpoint{11.361497in}{4.989118in}}%
\pgfpathcurveto{\pgfqpoint{11.369310in}{4.996932in}}{\pgfqpoint{11.373701in}{5.007531in}}{\pgfqpoint{11.373701in}{5.018581in}}%
\pgfpathcurveto{\pgfqpoint{11.373701in}{5.029631in}}{\pgfqpoint{11.369310in}{5.040230in}}{\pgfqpoint{11.361497in}{5.048044in}}%
\pgfpathcurveto{\pgfqpoint{11.353683in}{5.055858in}}{\pgfqpoint{11.343084in}{5.060248in}}{\pgfqpoint{11.332034in}{5.060248in}}%
\pgfpathcurveto{\pgfqpoint{11.320984in}{5.060248in}}{\pgfqpoint{11.310385in}{5.055858in}}{\pgfqpoint{11.302571in}{5.048044in}}%
\pgfpathcurveto{\pgfqpoint{11.294758in}{5.040230in}}{\pgfqpoint{11.290367in}{5.029631in}}{\pgfqpoint{11.290367in}{5.018581in}}%
\pgfpathcurveto{\pgfqpoint{11.290367in}{5.007531in}}{\pgfqpoint{11.294758in}{4.996932in}}{\pgfqpoint{11.302571in}{4.989118in}}%
\pgfpathcurveto{\pgfqpoint{11.310385in}{4.981305in}}{\pgfqpoint{11.320984in}{4.976915in}}{\pgfqpoint{11.332034in}{4.976915in}}%
\pgfpathlineto{\pgfqpoint{11.332034in}{4.976915in}}%
\pgfpathclose%
\pgfusepath{stroke}%
\end{pgfscope}%
\begin{pgfscope}%
\pgfpathrectangle{\pgfqpoint{7.394209in}{0.375000in}}{\pgfqpoint{6.356833in}{5.175000in}}%
\pgfusepath{clip}%
\pgfsetbuttcap%
\pgfsetroundjoin%
\pgfsetlinewidth{1.003750pt}%
\definecolor{currentstroke}{rgb}{0.827451,0.827451,0.827451}%
\pgfsetstrokecolor{currentstroke}%
\pgfsetdash{}{0pt}%
\pgfpathmoveto{\pgfqpoint{10.336941in}{3.499835in}}%
\pgfpathcurveto{\pgfqpoint{10.347991in}{3.499835in}}{\pgfqpoint{10.358590in}{3.504225in}}{\pgfqpoint{10.366404in}{3.512039in}}%
\pgfpathcurveto{\pgfqpoint{10.374217in}{3.519853in}}{\pgfqpoint{10.378608in}{3.530452in}}{\pgfqpoint{10.378608in}{3.541502in}}%
\pgfpathcurveto{\pgfqpoint{10.378608in}{3.552552in}}{\pgfqpoint{10.374217in}{3.563151in}}{\pgfqpoint{10.366404in}{3.570964in}}%
\pgfpathcurveto{\pgfqpoint{10.358590in}{3.578778in}}{\pgfqpoint{10.347991in}{3.583168in}}{\pgfqpoint{10.336941in}{3.583168in}}%
\pgfpathcurveto{\pgfqpoint{10.325891in}{3.583168in}}{\pgfqpoint{10.315292in}{3.578778in}}{\pgfqpoint{10.307478in}{3.570964in}}%
\pgfpathcurveto{\pgfqpoint{10.299665in}{3.563151in}}{\pgfqpoint{10.295274in}{3.552552in}}{\pgfqpoint{10.295274in}{3.541502in}}%
\pgfpathcurveto{\pgfqpoint{10.295274in}{3.530452in}}{\pgfqpoint{10.299665in}{3.519853in}}{\pgfqpoint{10.307478in}{3.512039in}}%
\pgfpathcurveto{\pgfqpoint{10.315292in}{3.504225in}}{\pgfqpoint{10.325891in}{3.499835in}}{\pgfqpoint{10.336941in}{3.499835in}}%
\pgfpathlineto{\pgfqpoint{10.336941in}{3.499835in}}%
\pgfpathclose%
\pgfusepath{stroke}%
\end{pgfscope}%
\begin{pgfscope}%
\pgfpathrectangle{\pgfqpoint{7.394209in}{0.375000in}}{\pgfqpoint{6.356833in}{5.175000in}}%
\pgfusepath{clip}%
\pgfsetbuttcap%
\pgfsetroundjoin%
\pgfsetlinewidth{1.003750pt}%
\definecolor{currentstroke}{rgb}{0.827451,0.827451,0.827451}%
\pgfsetstrokecolor{currentstroke}%
\pgfsetdash{}{0pt}%
\pgfpathmoveto{\pgfqpoint{7.785970in}{0.480946in}}%
\pgfpathcurveto{\pgfqpoint{7.797020in}{0.480946in}}{\pgfqpoint{7.807620in}{0.485336in}}{\pgfqpoint{7.815433in}{0.493150in}}%
\pgfpathcurveto{\pgfqpoint{7.823247in}{0.500963in}}{\pgfqpoint{7.827637in}{0.511562in}}{\pgfqpoint{7.827637in}{0.522612in}}%
\pgfpathcurveto{\pgfqpoint{7.827637in}{0.533662in}}{\pgfqpoint{7.823247in}{0.544261in}}{\pgfqpoint{7.815433in}{0.552075in}}%
\pgfpathcurveto{\pgfqpoint{7.807620in}{0.559889in}}{\pgfqpoint{7.797020in}{0.564279in}}{\pgfqpoint{7.785970in}{0.564279in}}%
\pgfpathcurveto{\pgfqpoint{7.774920in}{0.564279in}}{\pgfqpoint{7.764321in}{0.559889in}}{\pgfqpoint{7.756508in}{0.552075in}}%
\pgfpathcurveto{\pgfqpoint{7.748694in}{0.544261in}}{\pgfqpoint{7.744304in}{0.533662in}}{\pgfqpoint{7.744304in}{0.522612in}}%
\pgfpathcurveto{\pgfqpoint{7.744304in}{0.511562in}}{\pgfqpoint{7.748694in}{0.500963in}}{\pgfqpoint{7.756508in}{0.493150in}}%
\pgfpathcurveto{\pgfqpoint{7.764321in}{0.485336in}}{\pgfqpoint{7.774920in}{0.480946in}}{\pgfqpoint{7.785970in}{0.480946in}}%
\pgfpathlineto{\pgfqpoint{7.785970in}{0.480946in}}%
\pgfpathclose%
\pgfusepath{stroke}%
\end{pgfscope}%
\begin{pgfscope}%
\pgfpathrectangle{\pgfqpoint{7.394209in}{0.375000in}}{\pgfqpoint{6.356833in}{5.175000in}}%
\pgfusepath{clip}%
\pgfsetbuttcap%
\pgfsetroundjoin%
\pgfsetlinewidth{1.003750pt}%
\definecolor{currentstroke}{rgb}{0.827451,0.827451,0.827451}%
\pgfsetstrokecolor{currentstroke}%
\pgfsetdash{}{0pt}%
\pgfpathmoveto{\pgfqpoint{11.398055in}{5.396900in}}%
\pgfpathcurveto{\pgfqpoint{11.409105in}{5.396900in}}{\pgfqpoint{11.419704in}{5.401291in}}{\pgfqpoint{11.427518in}{5.409104in}}%
\pgfpathcurveto{\pgfqpoint{11.435331in}{5.416918in}}{\pgfqpoint{11.439721in}{5.427517in}}{\pgfqpoint{11.439721in}{5.438567in}}%
\pgfpathcurveto{\pgfqpoint{11.439721in}{5.449617in}}{\pgfqpoint{11.435331in}{5.460216in}}{\pgfqpoint{11.427518in}{5.468030in}}%
\pgfpathcurveto{\pgfqpoint{11.419704in}{5.475843in}}{\pgfqpoint{11.409105in}{5.480234in}}{\pgfqpoint{11.398055in}{5.480234in}}%
\pgfpathcurveto{\pgfqpoint{11.387005in}{5.480234in}}{\pgfqpoint{11.376406in}{5.475843in}}{\pgfqpoint{11.368592in}{5.468030in}}%
\pgfpathcurveto{\pgfqpoint{11.360778in}{5.460216in}}{\pgfqpoint{11.356388in}{5.449617in}}{\pgfqpoint{11.356388in}{5.438567in}}%
\pgfpathcurveto{\pgfqpoint{11.356388in}{5.427517in}}{\pgfqpoint{11.360778in}{5.416918in}}{\pgfqpoint{11.368592in}{5.409104in}}%
\pgfpathcurveto{\pgfqpoint{11.376406in}{5.401291in}}{\pgfqpoint{11.387005in}{5.396900in}}{\pgfqpoint{11.398055in}{5.396900in}}%
\pgfpathlineto{\pgfqpoint{11.398055in}{5.396900in}}%
\pgfpathclose%
\pgfusepath{stroke}%
\end{pgfscope}%
\begin{pgfscope}%
\pgfpathrectangle{\pgfqpoint{7.394209in}{0.375000in}}{\pgfqpoint{6.356833in}{5.175000in}}%
\pgfusepath{clip}%
\pgfsetbuttcap%
\pgfsetroundjoin%
\pgfsetlinewidth{1.003750pt}%
\definecolor{currentstroke}{rgb}{0.827451,0.827451,0.827451}%
\pgfsetstrokecolor{currentstroke}%
\pgfsetdash{}{0pt}%
\pgfpathmoveto{\pgfqpoint{7.483124in}{0.944208in}}%
\pgfpathcurveto{\pgfqpoint{7.494174in}{0.944208in}}{\pgfqpoint{7.504773in}{0.948598in}}{\pgfqpoint{7.512587in}{0.956412in}}%
\pgfpathcurveto{\pgfqpoint{7.520401in}{0.964225in}}{\pgfqpoint{7.524791in}{0.974824in}}{\pgfqpoint{7.524791in}{0.985875in}}%
\pgfpathcurveto{\pgfqpoint{7.524791in}{0.996925in}}{\pgfqpoint{7.520401in}{1.007524in}}{\pgfqpoint{7.512587in}{1.015337in}}%
\pgfpathcurveto{\pgfqpoint{7.504773in}{1.023151in}}{\pgfqpoint{7.494174in}{1.027541in}}{\pgfqpoint{7.483124in}{1.027541in}}%
\pgfpathcurveto{\pgfqpoint{7.472074in}{1.027541in}}{\pgfqpoint{7.461475in}{1.023151in}}{\pgfqpoint{7.453661in}{1.015337in}}%
\pgfpathcurveto{\pgfqpoint{7.445848in}{1.007524in}}{\pgfqpoint{7.441458in}{0.996925in}}{\pgfqpoint{7.441458in}{0.985875in}}%
\pgfpathcurveto{\pgfqpoint{7.441458in}{0.974824in}}{\pgfqpoint{7.445848in}{0.964225in}}{\pgfqpoint{7.453661in}{0.956412in}}%
\pgfpathcurveto{\pgfqpoint{7.461475in}{0.948598in}}{\pgfqpoint{7.472074in}{0.944208in}}{\pgfqpoint{7.483124in}{0.944208in}}%
\pgfpathlineto{\pgfqpoint{7.483124in}{0.944208in}}%
\pgfpathclose%
\pgfusepath{stroke}%
\end{pgfscope}%
\begin{pgfscope}%
\pgfpathrectangle{\pgfqpoint{7.394209in}{0.375000in}}{\pgfqpoint{6.356833in}{5.175000in}}%
\pgfusepath{clip}%
\pgfsetbuttcap%
\pgfsetroundjoin%
\pgfsetlinewidth{1.003750pt}%
\definecolor{currentstroke}{rgb}{0.827451,0.827451,0.827451}%
\pgfsetstrokecolor{currentstroke}%
\pgfsetdash{}{0pt}%
\pgfpathmoveto{\pgfqpoint{8.659029in}{1.791483in}}%
\pgfpathcurveto{\pgfqpoint{8.670079in}{1.791483in}}{\pgfqpoint{8.680678in}{1.795873in}}{\pgfqpoint{8.688492in}{1.803687in}}%
\pgfpathcurveto{\pgfqpoint{8.696305in}{1.811500in}}{\pgfqpoint{8.700695in}{1.822099in}}{\pgfqpoint{8.700695in}{1.833149in}}%
\pgfpathcurveto{\pgfqpoint{8.700695in}{1.844199in}}{\pgfqpoint{8.696305in}{1.854798in}}{\pgfqpoint{8.688492in}{1.862612in}}%
\pgfpathcurveto{\pgfqpoint{8.680678in}{1.870426in}}{\pgfqpoint{8.670079in}{1.874816in}}{\pgfqpoint{8.659029in}{1.874816in}}%
\pgfpathcurveto{\pgfqpoint{8.647979in}{1.874816in}}{\pgfqpoint{8.637380in}{1.870426in}}{\pgfqpoint{8.629566in}{1.862612in}}%
\pgfpathcurveto{\pgfqpoint{8.621752in}{1.854798in}}{\pgfqpoint{8.617362in}{1.844199in}}{\pgfqpoint{8.617362in}{1.833149in}}%
\pgfpathcurveto{\pgfqpoint{8.617362in}{1.822099in}}{\pgfqpoint{8.621752in}{1.811500in}}{\pgfqpoint{8.629566in}{1.803687in}}%
\pgfpathcurveto{\pgfqpoint{8.637380in}{1.795873in}}{\pgfqpoint{8.647979in}{1.791483in}}{\pgfqpoint{8.659029in}{1.791483in}}%
\pgfpathlineto{\pgfqpoint{8.659029in}{1.791483in}}%
\pgfpathclose%
\pgfusepath{stroke}%
\end{pgfscope}%
\begin{pgfscope}%
\pgfpathrectangle{\pgfqpoint{7.394209in}{0.375000in}}{\pgfqpoint{6.356833in}{5.175000in}}%
\pgfusepath{clip}%
\pgfsetbuttcap%
\pgfsetroundjoin%
\pgfsetlinewidth{1.003750pt}%
\definecolor{currentstroke}{rgb}{0.827451,0.827451,0.827451}%
\pgfsetstrokecolor{currentstroke}%
\pgfsetdash{}{0pt}%
\pgfpathmoveto{\pgfqpoint{8.381126in}{2.406061in}}%
\pgfpathcurveto{\pgfqpoint{8.392176in}{2.406061in}}{\pgfqpoint{8.402775in}{2.410451in}}{\pgfqpoint{8.410588in}{2.418265in}}%
\pgfpathcurveto{\pgfqpoint{8.418402in}{2.426078in}}{\pgfqpoint{8.422792in}{2.436677in}}{\pgfqpoint{8.422792in}{2.447727in}}%
\pgfpathcurveto{\pgfqpoint{8.422792in}{2.458777in}}{\pgfqpoint{8.418402in}{2.469376in}}{\pgfqpoint{8.410588in}{2.477190in}}%
\pgfpathcurveto{\pgfqpoint{8.402775in}{2.485004in}}{\pgfqpoint{8.392176in}{2.489394in}}{\pgfqpoint{8.381126in}{2.489394in}}%
\pgfpathcurveto{\pgfqpoint{8.370075in}{2.489394in}}{\pgfqpoint{8.359476in}{2.485004in}}{\pgfqpoint{8.351663in}{2.477190in}}%
\pgfpathcurveto{\pgfqpoint{8.343849in}{2.469376in}}{\pgfqpoint{8.339459in}{2.458777in}}{\pgfqpoint{8.339459in}{2.447727in}}%
\pgfpathcurveto{\pgfqpoint{8.339459in}{2.436677in}}{\pgfqpoint{8.343849in}{2.426078in}}{\pgfqpoint{8.351663in}{2.418265in}}%
\pgfpathcurveto{\pgfqpoint{8.359476in}{2.410451in}}{\pgfqpoint{8.370075in}{2.406061in}}{\pgfqpoint{8.381126in}{2.406061in}}%
\pgfpathlineto{\pgfqpoint{8.381126in}{2.406061in}}%
\pgfpathclose%
\pgfusepath{stroke}%
\end{pgfscope}%
\begin{pgfscope}%
\pgfpathrectangle{\pgfqpoint{7.394209in}{0.375000in}}{\pgfqpoint{6.356833in}{5.175000in}}%
\pgfusepath{clip}%
\pgfsetbuttcap%
\pgfsetroundjoin%
\pgfsetlinewidth{1.003750pt}%
\definecolor{currentstroke}{rgb}{0.827451,0.827451,0.827451}%
\pgfsetstrokecolor{currentstroke}%
\pgfsetdash{}{0pt}%
\pgfpathmoveto{\pgfqpoint{11.564705in}{5.291194in}}%
\pgfpathcurveto{\pgfqpoint{11.575755in}{5.291194in}}{\pgfqpoint{11.586354in}{5.295584in}}{\pgfqpoint{11.594168in}{5.303398in}}%
\pgfpathcurveto{\pgfqpoint{11.601981in}{5.311212in}}{\pgfqpoint{11.606371in}{5.321811in}}{\pgfqpoint{11.606371in}{5.332861in}}%
\pgfpathcurveto{\pgfqpoint{11.606371in}{5.343911in}}{\pgfqpoint{11.601981in}{5.354510in}}{\pgfqpoint{11.594168in}{5.362324in}}%
\pgfpathcurveto{\pgfqpoint{11.586354in}{5.370137in}}{\pgfqpoint{11.575755in}{5.374527in}}{\pgfqpoint{11.564705in}{5.374527in}}%
\pgfpathcurveto{\pgfqpoint{11.553655in}{5.374527in}}{\pgfqpoint{11.543056in}{5.370137in}}{\pgfqpoint{11.535242in}{5.362324in}}%
\pgfpathcurveto{\pgfqpoint{11.527428in}{5.354510in}}{\pgfqpoint{11.523038in}{5.343911in}}{\pgfqpoint{11.523038in}{5.332861in}}%
\pgfpathcurveto{\pgfqpoint{11.523038in}{5.321811in}}{\pgfqpoint{11.527428in}{5.311212in}}{\pgfqpoint{11.535242in}{5.303398in}}%
\pgfpathcurveto{\pgfqpoint{11.543056in}{5.295584in}}{\pgfqpoint{11.553655in}{5.291194in}}{\pgfqpoint{11.564705in}{5.291194in}}%
\pgfpathlineto{\pgfqpoint{11.564705in}{5.291194in}}%
\pgfpathclose%
\pgfusepath{stroke}%
\end{pgfscope}%
\begin{pgfscope}%
\pgfpathrectangle{\pgfqpoint{7.394209in}{0.375000in}}{\pgfqpoint{6.356833in}{5.175000in}}%
\pgfusepath{clip}%
\pgfsetbuttcap%
\pgfsetroundjoin%
\pgfsetlinewidth{1.003750pt}%
\definecolor{currentstroke}{rgb}{0.827451,0.827451,0.827451}%
\pgfsetstrokecolor{currentstroke}%
\pgfsetdash{}{0pt}%
\pgfpathmoveto{\pgfqpoint{10.349686in}{5.488004in}}%
\pgfpathcurveto{\pgfqpoint{10.360737in}{5.488004in}}{\pgfqpoint{10.371336in}{5.492394in}}{\pgfqpoint{10.379149in}{5.500208in}}%
\pgfpathcurveto{\pgfqpoint{10.386963in}{5.508021in}}{\pgfqpoint{10.391353in}{5.518620in}}{\pgfqpoint{10.391353in}{5.529670in}}%
\pgfpathcurveto{\pgfqpoint{10.391353in}{5.540720in}}{\pgfqpoint{10.386963in}{5.551319in}}{\pgfqpoint{10.379149in}{5.559133in}}%
\pgfpathcurveto{\pgfqpoint{10.371336in}{5.566947in}}{\pgfqpoint{10.360737in}{5.571337in}}{\pgfqpoint{10.349686in}{5.571337in}}%
\pgfpathcurveto{\pgfqpoint{10.338636in}{5.571337in}}{\pgfqpoint{10.328037in}{5.566947in}}{\pgfqpoint{10.320224in}{5.559133in}}%
\pgfpathcurveto{\pgfqpoint{10.312410in}{5.551319in}}{\pgfqpoint{10.308020in}{5.540720in}}{\pgfqpoint{10.308020in}{5.529670in}}%
\pgfpathcurveto{\pgfqpoint{10.308020in}{5.518620in}}{\pgfqpoint{10.312410in}{5.508021in}}{\pgfqpoint{10.320224in}{5.500208in}}%
\pgfpathcurveto{\pgfqpoint{10.328037in}{5.492394in}}{\pgfqpoint{10.338636in}{5.488004in}}{\pgfqpoint{10.349686in}{5.488004in}}%
\pgfpathlineto{\pgfqpoint{10.349686in}{5.488004in}}%
\pgfpathclose%
\pgfusepath{stroke}%
\end{pgfscope}%
\begin{pgfscope}%
\pgfpathrectangle{\pgfqpoint{7.394209in}{0.375000in}}{\pgfqpoint{6.356833in}{5.175000in}}%
\pgfusepath{clip}%
\pgfsetbuttcap%
\pgfsetroundjoin%
\pgfsetlinewidth{1.003750pt}%
\definecolor{currentstroke}{rgb}{0.827451,0.827451,0.827451}%
\pgfsetstrokecolor{currentstroke}%
\pgfsetdash{}{0pt}%
\pgfpathmoveto{\pgfqpoint{9.731679in}{4.494882in}}%
\pgfpathcurveto{\pgfqpoint{9.742729in}{4.494882in}}{\pgfqpoint{9.753328in}{4.499273in}}{\pgfqpoint{9.761142in}{4.507086in}}%
\pgfpathcurveto{\pgfqpoint{9.768955in}{4.514900in}}{\pgfqpoint{9.773345in}{4.525499in}}{\pgfqpoint{9.773345in}{4.536549in}}%
\pgfpathcurveto{\pgfqpoint{9.773345in}{4.547599in}}{\pgfqpoint{9.768955in}{4.558198in}}{\pgfqpoint{9.761142in}{4.566012in}}%
\pgfpathcurveto{\pgfqpoint{9.753328in}{4.573826in}}{\pgfqpoint{9.742729in}{4.578216in}}{\pgfqpoint{9.731679in}{4.578216in}}%
\pgfpathcurveto{\pgfqpoint{9.720629in}{4.578216in}}{\pgfqpoint{9.710030in}{4.573826in}}{\pgfqpoint{9.702216in}{4.566012in}}%
\pgfpathcurveto{\pgfqpoint{9.694402in}{4.558198in}}{\pgfqpoint{9.690012in}{4.547599in}}{\pgfqpoint{9.690012in}{4.536549in}}%
\pgfpathcurveto{\pgfqpoint{9.690012in}{4.525499in}}{\pgfqpoint{9.694402in}{4.514900in}}{\pgfqpoint{9.702216in}{4.507086in}}%
\pgfpathcurveto{\pgfqpoint{9.710030in}{4.499273in}}{\pgfqpoint{9.720629in}{4.494882in}}{\pgfqpoint{9.731679in}{4.494882in}}%
\pgfpathlineto{\pgfqpoint{9.731679in}{4.494882in}}%
\pgfpathclose%
\pgfusepath{stroke}%
\end{pgfscope}%
\begin{pgfscope}%
\pgfpathrectangle{\pgfqpoint{7.394209in}{0.375000in}}{\pgfqpoint{6.356833in}{5.175000in}}%
\pgfusepath{clip}%
\pgfsetbuttcap%
\pgfsetroundjoin%
\pgfsetlinewidth{1.003750pt}%
\definecolor{currentstroke}{rgb}{0.827451,0.827451,0.827451}%
\pgfsetstrokecolor{currentstroke}%
\pgfsetdash{}{0pt}%
\pgfpathmoveto{\pgfqpoint{10.327331in}{3.287881in}}%
\pgfpathcurveto{\pgfqpoint{10.338381in}{3.287881in}}{\pgfqpoint{10.348980in}{3.292271in}}{\pgfqpoint{10.356794in}{3.300085in}}%
\pgfpathcurveto{\pgfqpoint{10.364607in}{3.307899in}}{\pgfqpoint{10.368998in}{3.318498in}}{\pgfqpoint{10.368998in}{3.329548in}}%
\pgfpathcurveto{\pgfqpoint{10.368998in}{3.340598in}}{\pgfqpoint{10.364607in}{3.351197in}}{\pgfqpoint{10.356794in}{3.359011in}}%
\pgfpathcurveto{\pgfqpoint{10.348980in}{3.366824in}}{\pgfqpoint{10.338381in}{3.371215in}}{\pgfqpoint{10.327331in}{3.371215in}}%
\pgfpathcurveto{\pgfqpoint{10.316281in}{3.371215in}}{\pgfqpoint{10.305682in}{3.366824in}}{\pgfqpoint{10.297868in}{3.359011in}}%
\pgfpathcurveto{\pgfqpoint{10.290055in}{3.351197in}}{\pgfqpoint{10.285664in}{3.340598in}}{\pgfqpoint{10.285664in}{3.329548in}}%
\pgfpathcurveto{\pgfqpoint{10.285664in}{3.318498in}}{\pgfqpoint{10.290055in}{3.307899in}}{\pgfqpoint{10.297868in}{3.300085in}}%
\pgfpathcurveto{\pgfqpoint{10.305682in}{3.292271in}}{\pgfqpoint{10.316281in}{3.287881in}}{\pgfqpoint{10.327331in}{3.287881in}}%
\pgfpathlineto{\pgfqpoint{10.327331in}{3.287881in}}%
\pgfpathclose%
\pgfusepath{stroke}%
\end{pgfscope}%
\begin{pgfscope}%
\pgfpathrectangle{\pgfqpoint{7.394209in}{0.375000in}}{\pgfqpoint{6.356833in}{5.175000in}}%
\pgfusepath{clip}%
\pgfsetbuttcap%
\pgfsetroundjoin%
\pgfsetlinewidth{1.003750pt}%
\definecolor{currentstroke}{rgb}{0.827451,0.827451,0.827451}%
\pgfsetstrokecolor{currentstroke}%
\pgfsetdash{}{0pt}%
\pgfpathmoveto{\pgfqpoint{9.884974in}{4.550699in}}%
\pgfpathcurveto{\pgfqpoint{9.896024in}{4.550699in}}{\pgfqpoint{9.906623in}{4.555089in}}{\pgfqpoint{9.914436in}{4.562903in}}%
\pgfpathcurveto{\pgfqpoint{9.922250in}{4.570716in}}{\pgfqpoint{9.926640in}{4.581315in}}{\pgfqpoint{9.926640in}{4.592365in}}%
\pgfpathcurveto{\pgfqpoint{9.926640in}{4.603416in}}{\pgfqpoint{9.922250in}{4.614015in}}{\pgfqpoint{9.914436in}{4.621828in}}%
\pgfpathcurveto{\pgfqpoint{9.906623in}{4.629642in}}{\pgfqpoint{9.896024in}{4.634032in}}{\pgfqpoint{9.884974in}{4.634032in}}%
\pgfpathcurveto{\pgfqpoint{9.873923in}{4.634032in}}{\pgfqpoint{9.863324in}{4.629642in}}{\pgfqpoint{9.855511in}{4.621828in}}%
\pgfpathcurveto{\pgfqpoint{9.847697in}{4.614015in}}{\pgfqpoint{9.843307in}{4.603416in}}{\pgfqpoint{9.843307in}{4.592365in}}%
\pgfpathcurveto{\pgfqpoint{9.843307in}{4.581315in}}{\pgfqpoint{9.847697in}{4.570716in}}{\pgfqpoint{9.855511in}{4.562903in}}%
\pgfpathcurveto{\pgfqpoint{9.863324in}{4.555089in}}{\pgfqpoint{9.873923in}{4.550699in}}{\pgfqpoint{9.884974in}{4.550699in}}%
\pgfpathlineto{\pgfqpoint{9.884974in}{4.550699in}}%
\pgfpathclose%
\pgfusepath{stroke}%
\end{pgfscope}%
\begin{pgfscope}%
\pgfpathrectangle{\pgfqpoint{7.394209in}{0.375000in}}{\pgfqpoint{6.356833in}{5.175000in}}%
\pgfusepath{clip}%
\pgfsetbuttcap%
\pgfsetroundjoin%
\pgfsetlinewidth{1.003750pt}%
\definecolor{currentstroke}{rgb}{0.827451,0.827451,0.827451}%
\pgfsetstrokecolor{currentstroke}%
\pgfsetdash{}{0pt}%
\pgfpathmoveto{\pgfqpoint{8.382243in}{1.303828in}}%
\pgfpathcurveto{\pgfqpoint{8.393293in}{1.303828in}}{\pgfqpoint{8.403892in}{1.308218in}}{\pgfqpoint{8.411706in}{1.316031in}}%
\pgfpathcurveto{\pgfqpoint{8.419519in}{1.323845in}}{\pgfqpoint{8.423909in}{1.334444in}}{\pgfqpoint{8.423909in}{1.345494in}}%
\pgfpathcurveto{\pgfqpoint{8.423909in}{1.356544in}}{\pgfqpoint{8.419519in}{1.367143in}}{\pgfqpoint{8.411706in}{1.374957in}}%
\pgfpathcurveto{\pgfqpoint{8.403892in}{1.382771in}}{\pgfqpoint{8.393293in}{1.387161in}}{\pgfqpoint{8.382243in}{1.387161in}}%
\pgfpathcurveto{\pgfqpoint{8.371193in}{1.387161in}}{\pgfqpoint{8.360594in}{1.382771in}}{\pgfqpoint{8.352780in}{1.374957in}}%
\pgfpathcurveto{\pgfqpoint{8.344966in}{1.367143in}}{\pgfqpoint{8.340576in}{1.356544in}}{\pgfqpoint{8.340576in}{1.345494in}}%
\pgfpathcurveto{\pgfqpoint{8.340576in}{1.334444in}}{\pgfqpoint{8.344966in}{1.323845in}}{\pgfqpoint{8.352780in}{1.316031in}}%
\pgfpathcurveto{\pgfqpoint{8.360594in}{1.308218in}}{\pgfqpoint{8.371193in}{1.303828in}}{\pgfqpoint{8.382243in}{1.303828in}}%
\pgfpathlineto{\pgfqpoint{8.382243in}{1.303828in}}%
\pgfpathclose%
\pgfusepath{stroke}%
\end{pgfscope}%
\begin{pgfscope}%
\pgfpathrectangle{\pgfqpoint{7.394209in}{0.375000in}}{\pgfqpoint{6.356833in}{5.175000in}}%
\pgfusepath{clip}%
\pgfsetbuttcap%
\pgfsetroundjoin%
\pgfsetlinewidth{1.003750pt}%
\definecolor{currentstroke}{rgb}{0.827451,0.827451,0.827451}%
\pgfsetstrokecolor{currentstroke}%
\pgfsetdash{}{0pt}%
\pgfpathmoveto{\pgfqpoint{10.655278in}{5.503873in}}%
\pgfpathcurveto{\pgfqpoint{10.666328in}{5.503873in}}{\pgfqpoint{10.676927in}{5.508264in}}{\pgfqpoint{10.684740in}{5.516077in}}%
\pgfpathcurveto{\pgfqpoint{10.692554in}{5.523891in}}{\pgfqpoint{10.696944in}{5.534490in}}{\pgfqpoint{10.696944in}{5.545540in}}%
\pgfpathcurveto{\pgfqpoint{10.696944in}{5.556590in}}{\pgfqpoint{10.692554in}{5.567189in}}{\pgfqpoint{10.684740in}{5.575003in}}%
\pgfpathcurveto{\pgfqpoint{10.676927in}{5.582816in}}{\pgfqpoint{10.666328in}{5.587207in}}{\pgfqpoint{10.655278in}{5.587207in}}%
\pgfpathcurveto{\pgfqpoint{10.644227in}{5.587207in}}{\pgfqpoint{10.633628in}{5.582816in}}{\pgfqpoint{10.625815in}{5.575003in}}%
\pgfpathcurveto{\pgfqpoint{10.618001in}{5.567189in}}{\pgfqpoint{10.613611in}{5.556590in}}{\pgfqpoint{10.613611in}{5.545540in}}%
\pgfpathcurveto{\pgfqpoint{10.613611in}{5.534490in}}{\pgfqpoint{10.618001in}{5.523891in}}{\pgfqpoint{10.625815in}{5.516077in}}%
\pgfpathcurveto{\pgfqpoint{10.633628in}{5.508264in}}{\pgfqpoint{10.644227in}{5.503873in}}{\pgfqpoint{10.655278in}{5.503873in}}%
\pgfpathlineto{\pgfqpoint{10.655278in}{5.503873in}}%
\pgfpathclose%
\pgfusepath{stroke}%
\end{pgfscope}%
\begin{pgfscope}%
\pgfpathrectangle{\pgfqpoint{7.394209in}{0.375000in}}{\pgfqpoint{6.356833in}{5.175000in}}%
\pgfusepath{clip}%
\pgfsetbuttcap%
\pgfsetroundjoin%
\pgfsetlinewidth{1.003750pt}%
\definecolor{currentstroke}{rgb}{0.827451,0.827451,0.827451}%
\pgfsetstrokecolor{currentstroke}%
\pgfsetdash{}{0pt}%
\pgfpathmoveto{\pgfqpoint{11.291712in}{4.846948in}}%
\pgfpathcurveto{\pgfqpoint{11.302762in}{4.846948in}}{\pgfqpoint{11.313361in}{4.851339in}}{\pgfqpoint{11.321175in}{4.859152in}}%
\pgfpathcurveto{\pgfqpoint{11.328988in}{4.866966in}}{\pgfqpoint{11.333379in}{4.877565in}}{\pgfqpoint{11.333379in}{4.888615in}}%
\pgfpathcurveto{\pgfqpoint{11.333379in}{4.899665in}}{\pgfqpoint{11.328988in}{4.910264in}}{\pgfqpoint{11.321175in}{4.918078in}}%
\pgfpathcurveto{\pgfqpoint{11.313361in}{4.925891in}}{\pgfqpoint{11.302762in}{4.930282in}}{\pgfqpoint{11.291712in}{4.930282in}}%
\pgfpathcurveto{\pgfqpoint{11.280662in}{4.930282in}}{\pgfqpoint{11.270063in}{4.925891in}}{\pgfqpoint{11.262249in}{4.918078in}}%
\pgfpathcurveto{\pgfqpoint{11.254436in}{4.910264in}}{\pgfqpoint{11.250045in}{4.899665in}}{\pgfqpoint{11.250045in}{4.888615in}}%
\pgfpathcurveto{\pgfqpoint{11.250045in}{4.877565in}}{\pgfqpoint{11.254436in}{4.866966in}}{\pgfqpoint{11.262249in}{4.859152in}}%
\pgfpathcurveto{\pgfqpoint{11.270063in}{4.851339in}}{\pgfqpoint{11.280662in}{4.846948in}}{\pgfqpoint{11.291712in}{4.846948in}}%
\pgfpathlineto{\pgfqpoint{11.291712in}{4.846948in}}%
\pgfpathclose%
\pgfusepath{stroke}%
\end{pgfscope}%
\begin{pgfscope}%
\pgfpathrectangle{\pgfqpoint{7.394209in}{0.375000in}}{\pgfqpoint{6.356833in}{5.175000in}}%
\pgfusepath{clip}%
\pgfsetbuttcap%
\pgfsetroundjoin%
\pgfsetlinewidth{1.003750pt}%
\definecolor{currentstroke}{rgb}{0.827451,0.827451,0.827451}%
\pgfsetstrokecolor{currentstroke}%
\pgfsetdash{}{0pt}%
\pgfpathmoveto{\pgfqpoint{7.633421in}{0.333333in}}%
\pgfpathcurveto{\pgfqpoint{7.644471in}{0.333333in}}{\pgfqpoint{7.655070in}{0.337723in}}{\pgfqpoint{7.662884in}{0.345537in}}%
\pgfpathcurveto{\pgfqpoint{7.670697in}{0.353350in}}{\pgfqpoint{7.675087in}{0.363949in}}{\pgfqpoint{7.675087in}{0.375000in}}%
\pgfpathcurveto{\pgfqpoint{7.675087in}{0.386050in}}{\pgfqpoint{7.670697in}{0.396649in}}{\pgfqpoint{7.662884in}{0.404462in}}%
\pgfpathcurveto{\pgfqpoint{7.655070in}{0.412276in}}{\pgfqpoint{7.644471in}{0.416666in}}{\pgfqpoint{7.633421in}{0.416666in}}%
\pgfpathcurveto{\pgfqpoint{7.622371in}{0.416666in}}{\pgfqpoint{7.611772in}{0.412276in}}{\pgfqpoint{7.603958in}{0.404462in}}%
\pgfpathcurveto{\pgfqpoint{7.596144in}{0.396649in}}{\pgfqpoint{7.591754in}{0.386050in}}{\pgfqpoint{7.591754in}{0.375000in}}%
\pgfpathcurveto{\pgfqpoint{7.591754in}{0.363949in}}{\pgfqpoint{7.596144in}{0.353350in}}{\pgfqpoint{7.603958in}{0.345537in}}%
\pgfpathcurveto{\pgfqpoint{7.611772in}{0.337723in}}{\pgfqpoint{7.622371in}{0.333333in}}{\pgfqpoint{7.633421in}{0.333333in}}%
\pgfusepath{stroke}%
\end{pgfscope}%
\begin{pgfscope}%
\pgfpathrectangle{\pgfqpoint{7.394209in}{0.375000in}}{\pgfqpoint{6.356833in}{5.175000in}}%
\pgfusepath{clip}%
\pgfsetbuttcap%
\pgfsetroundjoin%
\pgfsetlinewidth{1.003750pt}%
\definecolor{currentstroke}{rgb}{0.827451,0.827451,0.827451}%
\pgfsetstrokecolor{currentstroke}%
\pgfsetdash{}{0pt}%
\pgfpathmoveto{\pgfqpoint{8.302414in}{2.046835in}}%
\pgfpathcurveto{\pgfqpoint{8.313464in}{2.046835in}}{\pgfqpoint{8.324063in}{2.051225in}}{\pgfqpoint{8.331877in}{2.059039in}}%
\pgfpathcurveto{\pgfqpoint{8.339690in}{2.066853in}}{\pgfqpoint{8.344081in}{2.077452in}}{\pgfqpoint{8.344081in}{2.088502in}}%
\pgfpathcurveto{\pgfqpoint{8.344081in}{2.099552in}}{\pgfqpoint{8.339690in}{2.110151in}}{\pgfqpoint{8.331877in}{2.117965in}}%
\pgfpathcurveto{\pgfqpoint{8.324063in}{2.125778in}}{\pgfqpoint{8.313464in}{2.130168in}}{\pgfqpoint{8.302414in}{2.130168in}}%
\pgfpathcurveto{\pgfqpoint{8.291364in}{2.130168in}}{\pgfqpoint{8.280765in}{2.125778in}}{\pgfqpoint{8.272951in}{2.117965in}}%
\pgfpathcurveto{\pgfqpoint{8.265138in}{2.110151in}}{\pgfqpoint{8.260747in}{2.099552in}}{\pgfqpoint{8.260747in}{2.088502in}}%
\pgfpathcurveto{\pgfqpoint{8.260747in}{2.077452in}}{\pgfqpoint{8.265138in}{2.066853in}}{\pgfqpoint{8.272951in}{2.059039in}}%
\pgfpathcurveto{\pgfqpoint{8.280765in}{2.051225in}}{\pgfqpoint{8.291364in}{2.046835in}}{\pgfqpoint{8.302414in}{2.046835in}}%
\pgfpathlineto{\pgfqpoint{8.302414in}{2.046835in}}%
\pgfpathclose%
\pgfusepath{stroke}%
\end{pgfscope}%
\begin{pgfscope}%
\pgfpathrectangle{\pgfqpoint{7.394209in}{0.375000in}}{\pgfqpoint{6.356833in}{5.175000in}}%
\pgfusepath{clip}%
\pgfsetbuttcap%
\pgfsetroundjoin%
\pgfsetlinewidth{1.003750pt}%
\definecolor{currentstroke}{rgb}{0.827451,0.827451,0.827451}%
\pgfsetstrokecolor{currentstroke}%
\pgfsetdash{}{0pt}%
\pgfpathmoveto{\pgfqpoint{11.160578in}{4.025189in}}%
\pgfpathcurveto{\pgfqpoint{11.171628in}{4.025189in}}{\pgfqpoint{11.182227in}{4.029579in}}{\pgfqpoint{11.190041in}{4.037393in}}%
\pgfpathcurveto{\pgfqpoint{11.197854in}{4.045206in}}{\pgfqpoint{11.202245in}{4.055805in}}{\pgfqpoint{11.202245in}{4.066856in}}%
\pgfpathcurveto{\pgfqpoint{11.202245in}{4.077906in}}{\pgfqpoint{11.197854in}{4.088505in}}{\pgfqpoint{11.190041in}{4.096318in}}%
\pgfpathcurveto{\pgfqpoint{11.182227in}{4.104132in}}{\pgfqpoint{11.171628in}{4.108522in}}{\pgfqpoint{11.160578in}{4.108522in}}%
\pgfpathcurveto{\pgfqpoint{11.149528in}{4.108522in}}{\pgfqpoint{11.138929in}{4.104132in}}{\pgfqpoint{11.131115in}{4.096318in}}%
\pgfpathcurveto{\pgfqpoint{11.123302in}{4.088505in}}{\pgfqpoint{11.118911in}{4.077906in}}{\pgfqpoint{11.118911in}{4.066856in}}%
\pgfpathcurveto{\pgfqpoint{11.118911in}{4.055805in}}{\pgfqpoint{11.123302in}{4.045206in}}{\pgfqpoint{11.131115in}{4.037393in}}%
\pgfpathcurveto{\pgfqpoint{11.138929in}{4.029579in}}{\pgfqpoint{11.149528in}{4.025189in}}{\pgfqpoint{11.160578in}{4.025189in}}%
\pgfpathlineto{\pgfqpoint{11.160578in}{4.025189in}}%
\pgfpathclose%
\pgfusepath{stroke}%
\end{pgfscope}%
\begin{pgfscope}%
\pgfpathrectangle{\pgfqpoint{7.394209in}{0.375000in}}{\pgfqpoint{6.356833in}{5.175000in}}%
\pgfusepath{clip}%
\pgfsetbuttcap%
\pgfsetroundjoin%
\pgfsetlinewidth{1.003750pt}%
\definecolor{currentstroke}{rgb}{0.827451,0.827451,0.827451}%
\pgfsetstrokecolor{currentstroke}%
\pgfsetdash{}{0pt}%
\pgfpathmoveto{\pgfqpoint{9.705320in}{2.507655in}}%
\pgfpathcurveto{\pgfqpoint{9.716370in}{2.507655in}}{\pgfqpoint{9.726969in}{2.512045in}}{\pgfqpoint{9.734782in}{2.519858in}}%
\pgfpathcurveto{\pgfqpoint{9.742596in}{2.527672in}}{\pgfqpoint{9.746986in}{2.538271in}}{\pgfqpoint{9.746986in}{2.549321in}}%
\pgfpathcurveto{\pgfqpoint{9.746986in}{2.560371in}}{\pgfqpoint{9.742596in}{2.570970in}}{\pgfqpoint{9.734782in}{2.578784in}}%
\pgfpathcurveto{\pgfqpoint{9.726969in}{2.586598in}}{\pgfqpoint{9.716370in}{2.590988in}}{\pgfqpoint{9.705320in}{2.590988in}}%
\pgfpathcurveto{\pgfqpoint{9.694269in}{2.590988in}}{\pgfqpoint{9.683670in}{2.586598in}}{\pgfqpoint{9.675857in}{2.578784in}}%
\pgfpathcurveto{\pgfqpoint{9.668043in}{2.570970in}}{\pgfqpoint{9.663653in}{2.560371in}}{\pgfqpoint{9.663653in}{2.549321in}}%
\pgfpathcurveto{\pgfqpoint{9.663653in}{2.538271in}}{\pgfqpoint{9.668043in}{2.527672in}}{\pgfqpoint{9.675857in}{2.519858in}}%
\pgfpathcurveto{\pgfqpoint{9.683670in}{2.512045in}}{\pgfqpoint{9.694269in}{2.507655in}}{\pgfqpoint{9.705320in}{2.507655in}}%
\pgfpathlineto{\pgfqpoint{9.705320in}{2.507655in}}%
\pgfpathclose%
\pgfusepath{stroke}%
\end{pgfscope}%
\begin{pgfscope}%
\pgfpathrectangle{\pgfqpoint{7.394209in}{0.375000in}}{\pgfqpoint{6.356833in}{5.175000in}}%
\pgfusepath{clip}%
\pgfsetbuttcap%
\pgfsetroundjoin%
\pgfsetlinewidth{1.003750pt}%
\definecolor{currentstroke}{rgb}{0.827451,0.827451,0.827451}%
\pgfsetstrokecolor{currentstroke}%
\pgfsetdash{}{0pt}%
\pgfpathmoveto{\pgfqpoint{8.993612in}{3.617331in}}%
\pgfpathcurveto{\pgfqpoint{9.004662in}{3.617331in}}{\pgfqpoint{9.015261in}{3.621721in}}{\pgfqpoint{9.023075in}{3.629535in}}%
\pgfpathcurveto{\pgfqpoint{9.030888in}{3.637348in}}{\pgfqpoint{9.035279in}{3.647947in}}{\pgfqpoint{9.035279in}{3.658997in}}%
\pgfpathcurveto{\pgfqpoint{9.035279in}{3.670048in}}{\pgfqpoint{9.030888in}{3.680647in}}{\pgfqpoint{9.023075in}{3.688460in}}%
\pgfpathcurveto{\pgfqpoint{9.015261in}{3.696274in}}{\pgfqpoint{9.004662in}{3.700664in}}{\pgfqpoint{8.993612in}{3.700664in}}%
\pgfpathcurveto{\pgfqpoint{8.982562in}{3.700664in}}{\pgfqpoint{8.971963in}{3.696274in}}{\pgfqpoint{8.964149in}{3.688460in}}%
\pgfpathcurveto{\pgfqpoint{8.956336in}{3.680647in}}{\pgfqpoint{8.951945in}{3.670048in}}{\pgfqpoint{8.951945in}{3.658997in}}%
\pgfpathcurveto{\pgfqpoint{8.951945in}{3.647947in}}{\pgfqpoint{8.956336in}{3.637348in}}{\pgfqpoint{8.964149in}{3.629535in}}%
\pgfpathcurveto{\pgfqpoint{8.971963in}{3.621721in}}{\pgfqpoint{8.982562in}{3.617331in}}{\pgfqpoint{8.993612in}{3.617331in}}%
\pgfpathlineto{\pgfqpoint{8.993612in}{3.617331in}}%
\pgfpathclose%
\pgfusepath{stroke}%
\end{pgfscope}%
\begin{pgfscope}%
\pgfpathrectangle{\pgfqpoint{7.394209in}{0.375000in}}{\pgfqpoint{6.356833in}{5.175000in}}%
\pgfusepath{clip}%
\pgfsetbuttcap%
\pgfsetroundjoin%
\pgfsetlinewidth{1.003750pt}%
\definecolor{currentstroke}{rgb}{0.827451,0.827451,0.827451}%
\pgfsetstrokecolor{currentstroke}%
\pgfsetdash{}{0pt}%
\pgfpathmoveto{\pgfqpoint{7.423064in}{0.644221in}}%
\pgfpathcurveto{\pgfqpoint{7.434114in}{0.644221in}}{\pgfqpoint{7.444713in}{0.648612in}}{\pgfqpoint{7.452526in}{0.656425in}}%
\pgfpathcurveto{\pgfqpoint{7.460340in}{0.664239in}}{\pgfqpoint{7.464730in}{0.674838in}}{\pgfqpoint{7.464730in}{0.685888in}}%
\pgfpathcurveto{\pgfqpoint{7.464730in}{0.696938in}}{\pgfqpoint{7.460340in}{0.707537in}}{\pgfqpoint{7.452526in}{0.715351in}}%
\pgfpathcurveto{\pgfqpoint{7.444713in}{0.723164in}}{\pgfqpoint{7.434114in}{0.727555in}}{\pgfqpoint{7.423064in}{0.727555in}}%
\pgfpathcurveto{\pgfqpoint{7.412013in}{0.727555in}}{\pgfqpoint{7.401414in}{0.723164in}}{\pgfqpoint{7.393601in}{0.715351in}}%
\pgfpathcurveto{\pgfqpoint{7.385787in}{0.707537in}}{\pgfqpoint{7.381397in}{0.696938in}}{\pgfqpoint{7.381397in}{0.685888in}}%
\pgfpathcurveto{\pgfqpoint{7.381397in}{0.674838in}}{\pgfqpoint{7.385787in}{0.664239in}}{\pgfqpoint{7.393601in}{0.656425in}}%
\pgfpathcurveto{\pgfqpoint{7.401414in}{0.648612in}}{\pgfqpoint{7.412013in}{0.644221in}}{\pgfqpoint{7.423064in}{0.644221in}}%
\pgfpathlineto{\pgfqpoint{7.423064in}{0.644221in}}%
\pgfpathclose%
\pgfusepath{stroke}%
\end{pgfscope}%
\begin{pgfscope}%
\pgfpathrectangle{\pgfqpoint{7.394209in}{0.375000in}}{\pgfqpoint{6.356833in}{5.175000in}}%
\pgfusepath{clip}%
\pgfsetbuttcap%
\pgfsetroundjoin%
\pgfsetlinewidth{1.003750pt}%
\definecolor{currentstroke}{rgb}{0.827451,0.827451,0.827451}%
\pgfsetstrokecolor{currentstroke}%
\pgfsetdash{}{0pt}%
\pgfpathmoveto{\pgfqpoint{11.966408in}{5.481955in}}%
\pgfpathcurveto{\pgfqpoint{11.977459in}{5.481955in}}{\pgfqpoint{11.988058in}{5.486346in}}{\pgfqpoint{11.995871in}{5.494159in}}%
\pgfpathcurveto{\pgfqpoint{12.003685in}{5.501973in}}{\pgfqpoint{12.008075in}{5.512572in}}{\pgfqpoint{12.008075in}{5.523622in}}%
\pgfpathcurveto{\pgfqpoint{12.008075in}{5.534672in}}{\pgfqpoint{12.003685in}{5.545271in}}{\pgfqpoint{11.995871in}{5.553085in}}%
\pgfpathcurveto{\pgfqpoint{11.988058in}{5.560899in}}{\pgfqpoint{11.977459in}{5.565289in}}{\pgfqpoint{11.966408in}{5.565289in}}%
\pgfpathcurveto{\pgfqpoint{11.955358in}{5.565289in}}{\pgfqpoint{11.944759in}{5.560899in}}{\pgfqpoint{11.936946in}{5.553085in}}%
\pgfpathcurveto{\pgfqpoint{11.929132in}{5.545271in}}{\pgfqpoint{11.924742in}{5.534672in}}{\pgfqpoint{11.924742in}{5.523622in}}%
\pgfpathcurveto{\pgfqpoint{11.924742in}{5.512572in}}{\pgfqpoint{11.929132in}{5.501973in}}{\pgfqpoint{11.936946in}{5.494159in}}%
\pgfpathcurveto{\pgfqpoint{11.944759in}{5.486346in}}{\pgfqpoint{11.955358in}{5.481955in}}{\pgfqpoint{11.966408in}{5.481955in}}%
\pgfpathlineto{\pgfqpoint{11.966408in}{5.481955in}}%
\pgfpathclose%
\pgfusepath{stroke}%
\end{pgfscope}%
\begin{pgfscope}%
\pgfpathrectangle{\pgfqpoint{7.394209in}{0.375000in}}{\pgfqpoint{6.356833in}{5.175000in}}%
\pgfusepath{clip}%
\pgfsetbuttcap%
\pgfsetroundjoin%
\pgfsetlinewidth{1.003750pt}%
\definecolor{currentstroke}{rgb}{0.827451,0.827451,0.827451}%
\pgfsetstrokecolor{currentstroke}%
\pgfsetdash{}{0pt}%
\pgfpathmoveto{\pgfqpoint{10.446544in}{5.495947in}}%
\pgfpathcurveto{\pgfqpoint{10.457594in}{5.495947in}}{\pgfqpoint{10.468193in}{5.500337in}}{\pgfqpoint{10.476006in}{5.508151in}}%
\pgfpathcurveto{\pgfqpoint{10.483820in}{5.515965in}}{\pgfqpoint{10.488210in}{5.526564in}}{\pgfqpoint{10.488210in}{5.537614in}}%
\pgfpathcurveto{\pgfqpoint{10.488210in}{5.548664in}}{\pgfqpoint{10.483820in}{5.559263in}}{\pgfqpoint{10.476006in}{5.567077in}}%
\pgfpathcurveto{\pgfqpoint{10.468193in}{5.574890in}}{\pgfqpoint{10.457594in}{5.579281in}}{\pgfqpoint{10.446544in}{5.579281in}}%
\pgfpathcurveto{\pgfqpoint{10.435494in}{5.579281in}}{\pgfqpoint{10.424895in}{5.574890in}}{\pgfqpoint{10.417081in}{5.567077in}}%
\pgfpathcurveto{\pgfqpoint{10.409267in}{5.559263in}}{\pgfqpoint{10.404877in}{5.548664in}}{\pgfqpoint{10.404877in}{5.537614in}}%
\pgfpathcurveto{\pgfqpoint{10.404877in}{5.526564in}}{\pgfqpoint{10.409267in}{5.515965in}}{\pgfqpoint{10.417081in}{5.508151in}}%
\pgfpathcurveto{\pgfqpoint{10.424895in}{5.500337in}}{\pgfqpoint{10.435494in}{5.495947in}}{\pgfqpoint{10.446544in}{5.495947in}}%
\pgfpathlineto{\pgfqpoint{10.446544in}{5.495947in}}%
\pgfpathclose%
\pgfusepath{stroke}%
\end{pgfscope}%
\begin{pgfscope}%
\pgfpathrectangle{\pgfqpoint{7.394209in}{0.375000in}}{\pgfqpoint{6.356833in}{5.175000in}}%
\pgfusepath{clip}%
\pgfsetbuttcap%
\pgfsetroundjoin%
\pgfsetlinewidth{1.003750pt}%
\definecolor{currentstroke}{rgb}{0.827451,0.827451,0.827451}%
\pgfsetstrokecolor{currentstroke}%
\pgfsetdash{}{0pt}%
\pgfpathmoveto{\pgfqpoint{8.199966in}{1.846013in}}%
\pgfpathcurveto{\pgfqpoint{8.211016in}{1.846013in}}{\pgfqpoint{8.221615in}{1.850403in}}{\pgfqpoint{8.229428in}{1.858216in}}%
\pgfpathcurveto{\pgfqpoint{8.237242in}{1.866030in}}{\pgfqpoint{8.241632in}{1.876629in}}{\pgfqpoint{8.241632in}{1.887679in}}%
\pgfpathcurveto{\pgfqpoint{8.241632in}{1.898729in}}{\pgfqpoint{8.237242in}{1.909328in}}{\pgfqpoint{8.229428in}{1.917142in}}%
\pgfpathcurveto{\pgfqpoint{8.221615in}{1.924956in}}{\pgfqpoint{8.211016in}{1.929346in}}{\pgfqpoint{8.199966in}{1.929346in}}%
\pgfpathcurveto{\pgfqpoint{8.188915in}{1.929346in}}{\pgfqpoint{8.178316in}{1.924956in}}{\pgfqpoint{8.170503in}{1.917142in}}%
\pgfpathcurveto{\pgfqpoint{8.162689in}{1.909328in}}{\pgfqpoint{8.158299in}{1.898729in}}{\pgfqpoint{8.158299in}{1.887679in}}%
\pgfpathcurveto{\pgfqpoint{8.158299in}{1.876629in}}{\pgfqpoint{8.162689in}{1.866030in}}{\pgfqpoint{8.170503in}{1.858216in}}%
\pgfpathcurveto{\pgfqpoint{8.178316in}{1.850403in}}{\pgfqpoint{8.188915in}{1.846013in}}{\pgfqpoint{8.199966in}{1.846013in}}%
\pgfpathlineto{\pgfqpoint{8.199966in}{1.846013in}}%
\pgfpathclose%
\pgfusepath{stroke}%
\end{pgfscope}%
\begin{pgfscope}%
\pgfpathrectangle{\pgfqpoint{7.394209in}{0.375000in}}{\pgfqpoint{6.356833in}{5.175000in}}%
\pgfusepath{clip}%
\pgfsetbuttcap%
\pgfsetroundjoin%
\pgfsetlinewidth{1.003750pt}%
\definecolor{currentstroke}{rgb}{0.827451,0.827451,0.827451}%
\pgfsetstrokecolor{currentstroke}%
\pgfsetdash{}{0pt}%
\pgfpathmoveto{\pgfqpoint{7.818881in}{1.608352in}}%
\pgfpathcurveto{\pgfqpoint{7.829932in}{1.608352in}}{\pgfqpoint{7.840531in}{1.612742in}}{\pgfqpoint{7.848344in}{1.620556in}}%
\pgfpathcurveto{\pgfqpoint{7.856158in}{1.628370in}}{\pgfqpoint{7.860548in}{1.638969in}}{\pgfqpoint{7.860548in}{1.650019in}}%
\pgfpathcurveto{\pgfqpoint{7.860548in}{1.661069in}}{\pgfqpoint{7.856158in}{1.671668in}}{\pgfqpoint{7.848344in}{1.679481in}}%
\pgfpathcurveto{\pgfqpoint{7.840531in}{1.687295in}}{\pgfqpoint{7.829932in}{1.691685in}}{\pgfqpoint{7.818881in}{1.691685in}}%
\pgfpathcurveto{\pgfqpoint{7.807831in}{1.691685in}}{\pgfqpoint{7.797232in}{1.687295in}}{\pgfqpoint{7.789419in}{1.679481in}}%
\pgfpathcurveto{\pgfqpoint{7.781605in}{1.671668in}}{\pgfqpoint{7.777215in}{1.661069in}}{\pgfqpoint{7.777215in}{1.650019in}}%
\pgfpathcurveto{\pgfqpoint{7.777215in}{1.638969in}}{\pgfqpoint{7.781605in}{1.628370in}}{\pgfqpoint{7.789419in}{1.620556in}}%
\pgfpathcurveto{\pgfqpoint{7.797232in}{1.612742in}}{\pgfqpoint{7.807831in}{1.608352in}}{\pgfqpoint{7.818881in}{1.608352in}}%
\pgfpathlineto{\pgfqpoint{7.818881in}{1.608352in}}%
\pgfpathclose%
\pgfusepath{stroke}%
\end{pgfscope}%
\begin{pgfscope}%
\pgfpathrectangle{\pgfqpoint{7.394209in}{0.375000in}}{\pgfqpoint{6.356833in}{5.175000in}}%
\pgfusepath{clip}%
\pgfsetbuttcap%
\pgfsetroundjoin%
\pgfsetlinewidth{1.003750pt}%
\definecolor{currentstroke}{rgb}{0.827451,0.827451,0.827451}%
\pgfsetstrokecolor{currentstroke}%
\pgfsetdash{}{0pt}%
\pgfpathmoveto{\pgfqpoint{9.455110in}{2.121343in}}%
\pgfpathcurveto{\pgfqpoint{9.466161in}{2.121343in}}{\pgfqpoint{9.476760in}{2.125733in}}{\pgfqpoint{9.484573in}{2.133547in}}%
\pgfpathcurveto{\pgfqpoint{9.492387in}{2.141360in}}{\pgfqpoint{9.496777in}{2.151959in}}{\pgfqpoint{9.496777in}{2.163009in}}%
\pgfpathcurveto{\pgfqpoint{9.496777in}{2.174059in}}{\pgfqpoint{9.492387in}{2.184659in}}{\pgfqpoint{9.484573in}{2.192472in}}%
\pgfpathcurveto{\pgfqpoint{9.476760in}{2.200286in}}{\pgfqpoint{9.466161in}{2.204676in}}{\pgfqpoint{9.455110in}{2.204676in}}%
\pgfpathcurveto{\pgfqpoint{9.444060in}{2.204676in}}{\pgfqpoint{9.433461in}{2.200286in}}{\pgfqpoint{9.425648in}{2.192472in}}%
\pgfpathcurveto{\pgfqpoint{9.417834in}{2.184659in}}{\pgfqpoint{9.413444in}{2.174059in}}{\pgfqpoint{9.413444in}{2.163009in}}%
\pgfpathcurveto{\pgfqpoint{9.413444in}{2.151959in}}{\pgfqpoint{9.417834in}{2.141360in}}{\pgfqpoint{9.425648in}{2.133547in}}%
\pgfpathcurveto{\pgfqpoint{9.433461in}{2.125733in}}{\pgfqpoint{9.444060in}{2.121343in}}{\pgfqpoint{9.455110in}{2.121343in}}%
\pgfpathlineto{\pgfqpoint{9.455110in}{2.121343in}}%
\pgfpathclose%
\pgfusepath{stroke}%
\end{pgfscope}%
\begin{pgfscope}%
\pgfpathrectangle{\pgfqpoint{7.394209in}{0.375000in}}{\pgfqpoint{6.356833in}{5.175000in}}%
\pgfusepath{clip}%
\pgfsetbuttcap%
\pgfsetroundjoin%
\pgfsetlinewidth{1.003750pt}%
\definecolor{currentstroke}{rgb}{0.827451,0.827451,0.827451}%
\pgfsetstrokecolor{currentstroke}%
\pgfsetdash{}{0pt}%
\pgfpathmoveto{\pgfqpoint{8.823201in}{1.910131in}}%
\pgfpathcurveto{\pgfqpoint{8.834251in}{1.910131in}}{\pgfqpoint{8.844850in}{1.914521in}}{\pgfqpoint{8.852663in}{1.922335in}}%
\pgfpathcurveto{\pgfqpoint{8.860477in}{1.930149in}}{\pgfqpoint{8.864867in}{1.940748in}}{\pgfqpoint{8.864867in}{1.951798in}}%
\pgfpathcurveto{\pgfqpoint{8.864867in}{1.962848in}}{\pgfqpoint{8.860477in}{1.973447in}}{\pgfqpoint{8.852663in}{1.981261in}}%
\pgfpathcurveto{\pgfqpoint{8.844850in}{1.989074in}}{\pgfqpoint{8.834251in}{1.993464in}}{\pgfqpoint{8.823201in}{1.993464in}}%
\pgfpathcurveto{\pgfqpoint{8.812151in}{1.993464in}}{\pgfqpoint{8.801551in}{1.989074in}}{\pgfqpoint{8.793738in}{1.981261in}}%
\pgfpathcurveto{\pgfqpoint{8.785924in}{1.973447in}}{\pgfqpoint{8.781534in}{1.962848in}}{\pgfqpoint{8.781534in}{1.951798in}}%
\pgfpathcurveto{\pgfqpoint{8.781534in}{1.940748in}}{\pgfqpoint{8.785924in}{1.930149in}}{\pgfqpoint{8.793738in}{1.922335in}}%
\pgfpathcurveto{\pgfqpoint{8.801551in}{1.914521in}}{\pgfqpoint{8.812151in}{1.910131in}}{\pgfqpoint{8.823201in}{1.910131in}}%
\pgfpathlineto{\pgfqpoint{8.823201in}{1.910131in}}%
\pgfpathclose%
\pgfusepath{stroke}%
\end{pgfscope}%
\begin{pgfscope}%
\pgfpathrectangle{\pgfqpoint{7.394209in}{0.375000in}}{\pgfqpoint{6.356833in}{5.175000in}}%
\pgfusepath{clip}%
\pgfsetbuttcap%
\pgfsetroundjoin%
\pgfsetlinewidth{1.003750pt}%
\definecolor{currentstroke}{rgb}{0.827451,0.827451,0.827451}%
\pgfsetstrokecolor{currentstroke}%
\pgfsetdash{}{0pt}%
\pgfpathmoveto{\pgfqpoint{7.426809in}{0.650070in}}%
\pgfpathcurveto{\pgfqpoint{7.437859in}{0.650070in}}{\pgfqpoint{7.448458in}{0.654460in}}{\pgfqpoint{7.456272in}{0.662274in}}%
\pgfpathcurveto{\pgfqpoint{7.464085in}{0.670088in}}{\pgfqpoint{7.468475in}{0.680687in}}{\pgfqpoint{7.468475in}{0.691737in}}%
\pgfpathcurveto{\pgfqpoint{7.468475in}{0.702787in}}{\pgfqpoint{7.464085in}{0.713386in}}{\pgfqpoint{7.456272in}{0.721200in}}%
\pgfpathcurveto{\pgfqpoint{7.448458in}{0.729013in}}{\pgfqpoint{7.437859in}{0.733403in}}{\pgfqpoint{7.426809in}{0.733403in}}%
\pgfpathcurveto{\pgfqpoint{7.415759in}{0.733403in}}{\pgfqpoint{7.405160in}{0.729013in}}{\pgfqpoint{7.397346in}{0.721200in}}%
\pgfpathcurveto{\pgfqpoint{7.389532in}{0.713386in}}{\pgfqpoint{7.385142in}{0.702787in}}{\pgfqpoint{7.385142in}{0.691737in}}%
\pgfpathcurveto{\pgfqpoint{7.385142in}{0.680687in}}{\pgfqpoint{7.389532in}{0.670088in}}{\pgfqpoint{7.397346in}{0.662274in}}%
\pgfpathcurveto{\pgfqpoint{7.405160in}{0.654460in}}{\pgfqpoint{7.415759in}{0.650070in}}{\pgfqpoint{7.426809in}{0.650070in}}%
\pgfpathlineto{\pgfqpoint{7.426809in}{0.650070in}}%
\pgfpathclose%
\pgfusepath{stroke}%
\end{pgfscope}%
\begin{pgfscope}%
\pgfpathrectangle{\pgfqpoint{7.394209in}{0.375000in}}{\pgfqpoint{6.356833in}{5.175000in}}%
\pgfusepath{clip}%
\pgfsetbuttcap%
\pgfsetroundjoin%
\pgfsetlinewidth{1.003750pt}%
\definecolor{currentstroke}{rgb}{0.827451,0.827451,0.827451}%
\pgfsetstrokecolor{currentstroke}%
\pgfsetdash{}{0pt}%
\pgfpathmoveto{\pgfqpoint{10.659876in}{4.261834in}}%
\pgfpathcurveto{\pgfqpoint{10.670926in}{4.261834in}}{\pgfqpoint{10.681525in}{4.266225in}}{\pgfqpoint{10.689339in}{4.274038in}}%
\pgfpathcurveto{\pgfqpoint{10.697152in}{4.281852in}}{\pgfqpoint{10.701543in}{4.292451in}}{\pgfqpoint{10.701543in}{4.303501in}}%
\pgfpathcurveto{\pgfqpoint{10.701543in}{4.314551in}}{\pgfqpoint{10.697152in}{4.325150in}}{\pgfqpoint{10.689339in}{4.332964in}}%
\pgfpathcurveto{\pgfqpoint{10.681525in}{4.340777in}}{\pgfqpoint{10.670926in}{4.345168in}}{\pgfqpoint{10.659876in}{4.345168in}}%
\pgfpathcurveto{\pgfqpoint{10.648826in}{4.345168in}}{\pgfqpoint{10.638227in}{4.340777in}}{\pgfqpoint{10.630413in}{4.332964in}}%
\pgfpathcurveto{\pgfqpoint{10.622600in}{4.325150in}}{\pgfqpoint{10.618209in}{4.314551in}}{\pgfqpoint{10.618209in}{4.303501in}}%
\pgfpathcurveto{\pgfqpoint{10.618209in}{4.292451in}}{\pgfqpoint{10.622600in}{4.281852in}}{\pgfqpoint{10.630413in}{4.274038in}}%
\pgfpathcurveto{\pgfqpoint{10.638227in}{4.266225in}}{\pgfqpoint{10.648826in}{4.261834in}}{\pgfqpoint{10.659876in}{4.261834in}}%
\pgfpathlineto{\pgfqpoint{10.659876in}{4.261834in}}%
\pgfpathclose%
\pgfusepath{stroke}%
\end{pgfscope}%
\begin{pgfscope}%
\pgfpathrectangle{\pgfqpoint{7.394209in}{0.375000in}}{\pgfqpoint{6.356833in}{5.175000in}}%
\pgfusepath{clip}%
\pgfsetbuttcap%
\pgfsetroundjoin%
\pgfsetlinewidth{1.003750pt}%
\definecolor{currentstroke}{rgb}{0.827451,0.827451,0.827451}%
\pgfsetstrokecolor{currentstroke}%
\pgfsetdash{}{0pt}%
\pgfpathmoveto{\pgfqpoint{13.334908in}{5.475282in}}%
\pgfpathcurveto{\pgfqpoint{13.345959in}{5.475282in}}{\pgfqpoint{13.356558in}{5.479672in}}{\pgfqpoint{13.364371in}{5.487485in}}%
\pgfpathcurveto{\pgfqpoint{13.372185in}{5.495299in}}{\pgfqpoint{13.376575in}{5.505898in}}{\pgfqpoint{13.376575in}{5.516948in}}%
\pgfpathcurveto{\pgfqpoint{13.376575in}{5.527998in}}{\pgfqpoint{13.372185in}{5.538597in}}{\pgfqpoint{13.364371in}{5.546411in}}%
\pgfpathcurveto{\pgfqpoint{13.356558in}{5.554225in}}{\pgfqpoint{13.345959in}{5.558615in}}{\pgfqpoint{13.334908in}{5.558615in}}%
\pgfpathcurveto{\pgfqpoint{13.323858in}{5.558615in}}{\pgfqpoint{13.313259in}{5.554225in}}{\pgfqpoint{13.305446in}{5.546411in}}%
\pgfpathcurveto{\pgfqpoint{13.297632in}{5.538597in}}{\pgfqpoint{13.293242in}{5.527998in}}{\pgfqpoint{13.293242in}{5.516948in}}%
\pgfpathcurveto{\pgfqpoint{13.293242in}{5.505898in}}{\pgfqpoint{13.297632in}{5.495299in}}{\pgfqpoint{13.305446in}{5.487485in}}%
\pgfpathcurveto{\pgfqpoint{13.313259in}{5.479672in}}{\pgfqpoint{13.323858in}{5.475282in}}{\pgfqpoint{13.334908in}{5.475282in}}%
\pgfpathlineto{\pgfqpoint{13.334908in}{5.475282in}}%
\pgfpathclose%
\pgfusepath{stroke}%
\end{pgfscope}%
\begin{pgfscope}%
\pgfpathrectangle{\pgfqpoint{7.394209in}{0.375000in}}{\pgfqpoint{6.356833in}{5.175000in}}%
\pgfusepath{clip}%
\pgfsetbuttcap%
\pgfsetroundjoin%
\pgfsetlinewidth{1.003750pt}%
\definecolor{currentstroke}{rgb}{0.827451,0.827451,0.827451}%
\pgfsetstrokecolor{currentstroke}%
\pgfsetdash{}{0pt}%
\pgfpathmoveto{\pgfqpoint{8.283512in}{1.927198in}}%
\pgfpathcurveto{\pgfqpoint{8.294562in}{1.927198in}}{\pgfqpoint{8.305161in}{1.931588in}}{\pgfqpoint{8.312974in}{1.939402in}}%
\pgfpathcurveto{\pgfqpoint{8.320788in}{1.947215in}}{\pgfqpoint{8.325178in}{1.957814in}}{\pgfqpoint{8.325178in}{1.968865in}}%
\pgfpathcurveto{\pgfqpoint{8.325178in}{1.979915in}}{\pgfqpoint{8.320788in}{1.990514in}}{\pgfqpoint{8.312974in}{1.998327in}}%
\pgfpathcurveto{\pgfqpoint{8.305161in}{2.006141in}}{\pgfqpoint{8.294562in}{2.010531in}}{\pgfqpoint{8.283512in}{2.010531in}}%
\pgfpathcurveto{\pgfqpoint{8.272461in}{2.010531in}}{\pgfqpoint{8.261862in}{2.006141in}}{\pgfqpoint{8.254049in}{1.998327in}}%
\pgfpathcurveto{\pgfqpoint{8.246235in}{1.990514in}}{\pgfqpoint{8.241845in}{1.979915in}}{\pgfqpoint{8.241845in}{1.968865in}}%
\pgfpathcurveto{\pgfqpoint{8.241845in}{1.957814in}}{\pgfqpoint{8.246235in}{1.947215in}}{\pgfqpoint{8.254049in}{1.939402in}}%
\pgfpathcurveto{\pgfqpoint{8.261862in}{1.931588in}}{\pgfqpoint{8.272461in}{1.927198in}}{\pgfqpoint{8.283512in}{1.927198in}}%
\pgfpathlineto{\pgfqpoint{8.283512in}{1.927198in}}%
\pgfpathclose%
\pgfusepath{stroke}%
\end{pgfscope}%
\begin{pgfscope}%
\pgfpathrectangle{\pgfqpoint{7.394209in}{0.375000in}}{\pgfqpoint{6.356833in}{5.175000in}}%
\pgfusepath{clip}%
\pgfsetbuttcap%
\pgfsetroundjoin%
\pgfsetlinewidth{1.003750pt}%
\definecolor{currentstroke}{rgb}{1.000000,0.000000,0.000000}%
\pgfsetstrokecolor{currentstroke}%
\pgfsetdash{}{0pt}%
\pgfpathmoveto{\pgfqpoint{7.394209in}{0.333333in}}%
\pgfpathcurveto{\pgfqpoint{7.405260in}{0.333333in}}{\pgfqpoint{7.415859in}{0.337723in}}{\pgfqpoint{7.423672in}{0.345537in}}%
\pgfpathcurveto{\pgfqpoint{7.431486in}{0.353350in}}{\pgfqpoint{7.435876in}{0.363949in}}{\pgfqpoint{7.435876in}{0.375000in}}%
\pgfpathcurveto{\pgfqpoint{7.435876in}{0.386050in}}{\pgfqpoint{7.431486in}{0.396649in}}{\pgfqpoint{7.423672in}{0.404462in}}%
\pgfpathcurveto{\pgfqpoint{7.415859in}{0.412276in}}{\pgfqpoint{7.405260in}{0.416666in}}{\pgfqpoint{7.394209in}{0.416666in}}%
\pgfpathcurveto{\pgfqpoint{7.383159in}{0.416666in}}{\pgfqpoint{7.372560in}{0.412276in}}{\pgfqpoint{7.364747in}{0.404462in}}%
\pgfpathcurveto{\pgfqpoint{7.356933in}{0.396649in}}{\pgfqpoint{7.352543in}{0.386050in}}{\pgfqpoint{7.352543in}{0.375000in}}%
\pgfpathcurveto{\pgfqpoint{7.352543in}{0.363949in}}{\pgfqpoint{7.356933in}{0.353350in}}{\pgfqpoint{7.364747in}{0.345537in}}%
\pgfpathcurveto{\pgfqpoint{7.372560in}{0.337723in}}{\pgfqpoint{7.383159in}{0.333333in}}{\pgfqpoint{7.394209in}{0.333333in}}%
\pgfusepath{stroke}%
\end{pgfscope}%
\begin{pgfscope}%
\pgfpathrectangle{\pgfqpoint{7.394209in}{0.375000in}}{\pgfqpoint{6.356833in}{5.175000in}}%
\pgfusepath{clip}%
\pgfsetbuttcap%
\pgfsetroundjoin%
\pgfsetlinewidth{1.003750pt}%
\definecolor{currentstroke}{rgb}{1.000000,0.000000,0.000000}%
\pgfsetstrokecolor{currentstroke}%
\pgfsetdash{}{0pt}%
\pgfpathmoveto{\pgfqpoint{13.751042in}{5.508333in}}%
\pgfpathcurveto{\pgfqpoint{13.762092in}{5.508333in}}{\pgfqpoint{13.772691in}{5.512724in}}{\pgfqpoint{13.780505in}{5.520537in}}%
\pgfpathcurveto{\pgfqpoint{13.788319in}{5.528351in}}{\pgfqpoint{13.792709in}{5.538950in}}{\pgfqpoint{13.792709in}{5.550000in}}%
\pgfpathcurveto{\pgfqpoint{13.792709in}{5.561050in}}{\pgfqpoint{13.788319in}{5.571649in}}{\pgfqpoint{13.780505in}{5.579463in}}%
\pgfpathcurveto{\pgfqpoint{13.772691in}{5.587276in}}{\pgfqpoint{13.762092in}{5.591667in}}{\pgfqpoint{13.751042in}{5.591667in}}%
\pgfpathcurveto{\pgfqpoint{13.739992in}{5.591667in}}{\pgfqpoint{13.729393in}{5.587276in}}{\pgfqpoint{13.721580in}{5.579463in}}%
\pgfpathcurveto{\pgfqpoint{13.713766in}{5.571649in}}{\pgfqpoint{13.709376in}{5.561050in}}{\pgfqpoint{13.709376in}{5.550000in}}%
\pgfpathcurveto{\pgfqpoint{13.709376in}{5.538950in}}{\pgfqpoint{13.713766in}{5.528351in}}{\pgfqpoint{13.721580in}{5.520537in}}%
\pgfpathcurveto{\pgfqpoint{13.729393in}{5.512724in}}{\pgfqpoint{13.739992in}{5.508333in}}{\pgfqpoint{13.751042in}{5.508333in}}%
\pgfpathlineto{\pgfqpoint{13.751042in}{5.508333in}}%
\pgfpathclose%
\pgfusepath{stroke}%
\end{pgfscope}%
\begin{pgfscope}%
\pgfpathrectangle{\pgfqpoint{7.394209in}{0.375000in}}{\pgfqpoint{6.356833in}{5.175000in}}%
\pgfusepath{clip}%
\pgfsetbuttcap%
\pgfsetroundjoin%
\pgfsetlinewidth{1.003750pt}%
\definecolor{currentstroke}{rgb}{1.000000,0.000000,0.000000}%
\pgfsetstrokecolor{currentstroke}%
\pgfsetdash{}{0pt}%
\pgfpathmoveto{\pgfqpoint{11.115956in}{4.598437in}}%
\pgfpathcurveto{\pgfqpoint{11.127006in}{4.598437in}}{\pgfqpoint{11.137605in}{4.602827in}}{\pgfqpoint{11.145419in}{4.610641in}}%
\pgfpathcurveto{\pgfqpoint{11.153233in}{4.618455in}}{\pgfqpoint{11.157623in}{4.629054in}}{\pgfqpoint{11.157623in}{4.640104in}}%
\pgfpathcurveto{\pgfqpoint{11.157623in}{4.651154in}}{\pgfqpoint{11.153233in}{4.661753in}}{\pgfqpoint{11.145419in}{4.669567in}}%
\pgfpathcurveto{\pgfqpoint{11.137605in}{4.677380in}}{\pgfqpoint{11.127006in}{4.681771in}}{\pgfqpoint{11.115956in}{4.681771in}}%
\pgfpathcurveto{\pgfqpoint{11.104906in}{4.681771in}}{\pgfqpoint{11.094307in}{4.677380in}}{\pgfqpoint{11.086493in}{4.669567in}}%
\pgfpathcurveto{\pgfqpoint{11.078680in}{4.661753in}}{\pgfqpoint{11.074290in}{4.651154in}}{\pgfqpoint{11.074290in}{4.640104in}}%
\pgfpathcurveto{\pgfqpoint{11.074290in}{4.629054in}}{\pgfqpoint{11.078680in}{4.618455in}}{\pgfqpoint{11.086493in}{4.610641in}}%
\pgfpathcurveto{\pgfqpoint{11.094307in}{4.602827in}}{\pgfqpoint{11.104906in}{4.598437in}}{\pgfqpoint{11.115956in}{4.598437in}}%
\pgfpathlineto{\pgfqpoint{11.115956in}{4.598437in}}%
\pgfpathclose%
\pgfusepath{stroke}%
\end{pgfscope}%
\begin{pgfscope}%
\pgfpathrectangle{\pgfqpoint{7.394209in}{0.375000in}}{\pgfqpoint{6.356833in}{5.175000in}}%
\pgfusepath{clip}%
\pgfsetbuttcap%
\pgfsetroundjoin%
\pgfsetlinewidth{1.003750pt}%
\definecolor{currentstroke}{rgb}{1.000000,0.000000,0.000000}%
\pgfsetstrokecolor{currentstroke}%
\pgfsetdash{}{0pt}%
\pgfpathmoveto{\pgfqpoint{7.818881in}{0.777190in}}%
\pgfpathcurveto{\pgfqpoint{7.829932in}{0.777190in}}{\pgfqpoint{7.840531in}{0.781581in}}{\pgfqpoint{7.848344in}{0.789394in}}%
\pgfpathcurveto{\pgfqpoint{7.856158in}{0.797208in}}{\pgfqpoint{7.860548in}{0.807807in}}{\pgfqpoint{7.860548in}{0.818857in}}%
\pgfpathcurveto{\pgfqpoint{7.860548in}{0.829907in}}{\pgfqpoint{7.856158in}{0.840506in}}{\pgfqpoint{7.848344in}{0.848320in}}%
\pgfpathcurveto{\pgfqpoint{7.840531in}{0.856133in}}{\pgfqpoint{7.829932in}{0.860524in}}{\pgfqpoint{7.818881in}{0.860524in}}%
\pgfpathcurveto{\pgfqpoint{7.807831in}{0.860524in}}{\pgfqpoint{7.797232in}{0.856133in}}{\pgfqpoint{7.789419in}{0.848320in}}%
\pgfpathcurveto{\pgfqpoint{7.781605in}{0.840506in}}{\pgfqpoint{7.777215in}{0.829907in}}{\pgfqpoint{7.777215in}{0.818857in}}%
\pgfpathcurveto{\pgfqpoint{7.777215in}{0.807807in}}{\pgfqpoint{7.781605in}{0.797208in}}{\pgfqpoint{7.789419in}{0.789394in}}%
\pgfpathcurveto{\pgfqpoint{7.797232in}{0.781581in}}{\pgfqpoint{7.807831in}{0.777190in}}{\pgfqpoint{7.818881in}{0.777190in}}%
\pgfpathlineto{\pgfqpoint{7.818881in}{0.777190in}}%
\pgfpathclose%
\pgfusepath{stroke}%
\end{pgfscope}%
\begin{pgfscope}%
\pgfpathrectangle{\pgfqpoint{7.394209in}{0.375000in}}{\pgfqpoint{6.356833in}{5.175000in}}%
\pgfusepath{clip}%
\pgfsetbuttcap%
\pgfsetroundjoin%
\pgfsetlinewidth{1.003750pt}%
\definecolor{currentstroke}{rgb}{1.000000,0.000000,0.000000}%
\pgfsetstrokecolor{currentstroke}%
\pgfsetdash{}{0pt}%
\pgfpathmoveto{\pgfqpoint{12.855865in}{5.508319in}}%
\pgfpathcurveto{\pgfqpoint{12.866915in}{5.508319in}}{\pgfqpoint{12.877514in}{5.512709in}}{\pgfqpoint{12.885328in}{5.520523in}}%
\pgfpathcurveto{\pgfqpoint{12.893142in}{5.528336in}}{\pgfqpoint{12.897532in}{5.538935in}}{\pgfqpoint{12.897532in}{5.549986in}}%
\pgfpathcurveto{\pgfqpoint{12.897532in}{5.561036in}}{\pgfqpoint{12.893142in}{5.571635in}}{\pgfqpoint{12.885328in}{5.579448in}}%
\pgfpathcurveto{\pgfqpoint{12.877514in}{5.587262in}}{\pgfqpoint{12.866915in}{5.591652in}}{\pgfqpoint{12.855865in}{5.591652in}}%
\pgfpathcurveto{\pgfqpoint{12.844815in}{5.591652in}}{\pgfqpoint{12.834216in}{5.587262in}}{\pgfqpoint{12.826402in}{5.579448in}}%
\pgfpathcurveto{\pgfqpoint{12.818589in}{5.571635in}}{\pgfqpoint{12.814199in}{5.561036in}}{\pgfqpoint{12.814199in}{5.549986in}}%
\pgfpathcurveto{\pgfqpoint{12.814199in}{5.538935in}}{\pgfqpoint{12.818589in}{5.528336in}}{\pgfqpoint{12.826402in}{5.520523in}}%
\pgfpathcurveto{\pgfqpoint{12.834216in}{5.512709in}}{\pgfqpoint{12.844815in}{5.508319in}}{\pgfqpoint{12.855865in}{5.508319in}}%
\pgfpathlineto{\pgfqpoint{12.855865in}{5.508319in}}%
\pgfpathclose%
\pgfusepath{stroke}%
\end{pgfscope}%
\begin{pgfscope}%
\pgfpathrectangle{\pgfqpoint{7.394209in}{0.375000in}}{\pgfqpoint{6.356833in}{5.175000in}}%
\pgfusepath{clip}%
\pgfsetbuttcap%
\pgfsetroundjoin%
\pgfsetlinewidth{1.003750pt}%
\definecolor{currentstroke}{rgb}{1.000000,0.000000,0.000000}%
\pgfsetstrokecolor{currentstroke}%
\pgfsetdash{}{0pt}%
\pgfpathmoveto{\pgfqpoint{7.726762in}{0.650070in}}%
\pgfpathcurveto{\pgfqpoint{7.737812in}{0.650070in}}{\pgfqpoint{7.748411in}{0.654460in}}{\pgfqpoint{7.756225in}{0.662274in}}%
\pgfpathcurveto{\pgfqpoint{7.764038in}{0.670088in}}{\pgfqpoint{7.768429in}{0.680687in}}{\pgfqpoint{7.768429in}{0.691737in}}%
\pgfpathcurveto{\pgfqpoint{7.768429in}{0.702787in}}{\pgfqpoint{7.764038in}{0.713386in}}{\pgfqpoint{7.756225in}{0.721200in}}%
\pgfpathcurveto{\pgfqpoint{7.748411in}{0.729013in}}{\pgfqpoint{7.737812in}{0.733403in}}{\pgfqpoint{7.726762in}{0.733403in}}%
\pgfpathcurveto{\pgfqpoint{7.715712in}{0.733403in}}{\pgfqpoint{7.705113in}{0.729013in}}{\pgfqpoint{7.697299in}{0.721200in}}%
\pgfpathcurveto{\pgfqpoint{7.689486in}{0.713386in}}{\pgfqpoint{7.685095in}{0.702787in}}{\pgfqpoint{7.685095in}{0.691737in}}%
\pgfpathcurveto{\pgfqpoint{7.685095in}{0.680687in}}{\pgfqpoint{7.689486in}{0.670088in}}{\pgfqpoint{7.697299in}{0.662274in}}%
\pgfpathcurveto{\pgfqpoint{7.705113in}{0.654460in}}{\pgfqpoint{7.715712in}{0.650070in}}{\pgfqpoint{7.726762in}{0.650070in}}%
\pgfpathlineto{\pgfqpoint{7.726762in}{0.650070in}}%
\pgfpathclose%
\pgfusepath{stroke}%
\end{pgfscope}%
\begin{pgfscope}%
\pgfpathrectangle{\pgfqpoint{7.394209in}{0.375000in}}{\pgfqpoint{6.356833in}{5.175000in}}%
\pgfusepath{clip}%
\pgfsetbuttcap%
\pgfsetroundjoin%
\pgfsetlinewidth{1.003750pt}%
\definecolor{currentstroke}{rgb}{1.000000,0.000000,0.000000}%
\pgfsetstrokecolor{currentstroke}%
\pgfsetdash{}{0pt}%
\pgfpathmoveto{\pgfqpoint{10.608044in}{5.094991in}}%
\pgfpathcurveto{\pgfqpoint{10.619094in}{5.094991in}}{\pgfqpoint{10.629693in}{5.099382in}}{\pgfqpoint{10.637507in}{5.107195in}}%
\pgfpathcurveto{\pgfqpoint{10.645321in}{5.115009in}}{\pgfqpoint{10.649711in}{5.125608in}}{\pgfqpoint{10.649711in}{5.136658in}}%
\pgfpathcurveto{\pgfqpoint{10.649711in}{5.147708in}}{\pgfqpoint{10.645321in}{5.158307in}}{\pgfqpoint{10.637507in}{5.166121in}}%
\pgfpathcurveto{\pgfqpoint{10.629693in}{5.173934in}}{\pgfqpoint{10.619094in}{5.178325in}}{\pgfqpoint{10.608044in}{5.178325in}}%
\pgfpathcurveto{\pgfqpoint{10.596994in}{5.178325in}}{\pgfqpoint{10.586395in}{5.173934in}}{\pgfqpoint{10.578581in}{5.166121in}}%
\pgfpathcurveto{\pgfqpoint{10.570768in}{5.158307in}}{\pgfqpoint{10.566378in}{5.147708in}}{\pgfqpoint{10.566378in}{5.136658in}}%
\pgfpathcurveto{\pgfqpoint{10.566378in}{5.125608in}}{\pgfqpoint{10.570768in}{5.115009in}}{\pgfqpoint{10.578581in}{5.107195in}}%
\pgfpathcurveto{\pgfqpoint{10.586395in}{5.099382in}}{\pgfqpoint{10.596994in}{5.094991in}}{\pgfqpoint{10.608044in}{5.094991in}}%
\pgfpathlineto{\pgfqpoint{10.608044in}{5.094991in}}%
\pgfpathclose%
\pgfusepath{stroke}%
\end{pgfscope}%
\begin{pgfscope}%
\pgfpathrectangle{\pgfqpoint{7.394209in}{0.375000in}}{\pgfqpoint{6.356833in}{5.175000in}}%
\pgfusepath{clip}%
\pgfsetbuttcap%
\pgfsetroundjoin%
\pgfsetlinewidth{1.003750pt}%
\definecolor{currentstroke}{rgb}{1.000000,0.000000,0.000000}%
\pgfsetstrokecolor{currentstroke}%
\pgfsetdash{}{0pt}%
\pgfpathmoveto{\pgfqpoint{13.221294in}{5.508326in}}%
\pgfpathcurveto{\pgfqpoint{13.232344in}{5.508326in}}{\pgfqpoint{13.242943in}{5.512716in}}{\pgfqpoint{13.250757in}{5.520530in}}%
\pgfpathcurveto{\pgfqpoint{13.258570in}{5.528343in}}{\pgfqpoint{13.262961in}{5.538942in}}{\pgfqpoint{13.262961in}{5.549992in}}%
\pgfpathcurveto{\pgfqpoint{13.262961in}{5.561043in}}{\pgfqpoint{13.258570in}{5.571642in}}{\pgfqpoint{13.250757in}{5.579455in}}%
\pgfpathcurveto{\pgfqpoint{13.242943in}{5.587269in}}{\pgfqpoint{13.232344in}{5.591659in}}{\pgfqpoint{13.221294in}{5.591659in}}%
\pgfpathcurveto{\pgfqpoint{13.210244in}{5.591659in}}{\pgfqpoint{13.199645in}{5.587269in}}{\pgfqpoint{13.191831in}{5.579455in}}%
\pgfpathcurveto{\pgfqpoint{13.184018in}{5.571642in}}{\pgfqpoint{13.179627in}{5.561043in}}{\pgfqpoint{13.179627in}{5.549992in}}%
\pgfpathcurveto{\pgfqpoint{13.179627in}{5.538942in}}{\pgfqpoint{13.184018in}{5.528343in}}{\pgfqpoint{13.191831in}{5.520530in}}%
\pgfpathcurveto{\pgfqpoint{13.199645in}{5.512716in}}{\pgfqpoint{13.210244in}{5.508326in}}{\pgfqpoint{13.221294in}{5.508326in}}%
\pgfpathlineto{\pgfqpoint{13.221294in}{5.508326in}}%
\pgfpathclose%
\pgfusepath{stroke}%
\end{pgfscope}%
\begin{pgfscope}%
\pgfpathrectangle{\pgfqpoint{7.394209in}{0.375000in}}{\pgfqpoint{6.356833in}{5.175000in}}%
\pgfusepath{clip}%
\pgfsetbuttcap%
\pgfsetroundjoin%
\pgfsetlinewidth{1.003750pt}%
\definecolor{currentstroke}{rgb}{1.000000,0.000000,0.000000}%
\pgfsetstrokecolor{currentstroke}%
\pgfsetdash{}{0pt}%
\pgfpathmoveto{\pgfqpoint{13.448595in}{5.479742in}}%
\pgfpathcurveto{\pgfqpoint{13.459645in}{5.479742in}}{\pgfqpoint{13.470244in}{5.484132in}}{\pgfqpoint{13.478058in}{5.491945in}}%
\pgfpathcurveto{\pgfqpoint{13.485872in}{5.499759in}}{\pgfqpoint{13.490262in}{5.510358in}}{\pgfqpoint{13.490262in}{5.521408in}}%
\pgfpathcurveto{\pgfqpoint{13.490262in}{5.532458in}}{\pgfqpoint{13.485872in}{5.543057in}}{\pgfqpoint{13.478058in}{5.550871in}}%
\pgfpathcurveto{\pgfqpoint{13.470244in}{5.558685in}}{\pgfqpoint{13.459645in}{5.563075in}}{\pgfqpoint{13.448595in}{5.563075in}}%
\pgfpathcurveto{\pgfqpoint{13.437545in}{5.563075in}}{\pgfqpoint{13.426946in}{5.558685in}}{\pgfqpoint{13.419132in}{5.550871in}}%
\pgfpathcurveto{\pgfqpoint{13.411319in}{5.543057in}}{\pgfqpoint{13.406929in}{5.532458in}}{\pgfqpoint{13.406929in}{5.521408in}}%
\pgfpathcurveto{\pgfqpoint{13.406929in}{5.510358in}}{\pgfqpoint{13.411319in}{5.499759in}}{\pgfqpoint{13.419132in}{5.491945in}}%
\pgfpathcurveto{\pgfqpoint{13.426946in}{5.484132in}}{\pgfqpoint{13.437545in}{5.479742in}}{\pgfqpoint{13.448595in}{5.479742in}}%
\pgfpathlineto{\pgfqpoint{13.448595in}{5.479742in}}%
\pgfpathclose%
\pgfusepath{stroke}%
\end{pgfscope}%
\begin{pgfscope}%
\pgfpathrectangle{\pgfqpoint{7.394209in}{0.375000in}}{\pgfqpoint{6.356833in}{5.175000in}}%
\pgfusepath{clip}%
\pgfsetbuttcap%
\pgfsetroundjoin%
\pgfsetlinewidth{1.003750pt}%
\definecolor{currentstroke}{rgb}{1.000000,0.000000,0.000000}%
\pgfsetstrokecolor{currentstroke}%
\pgfsetdash{}{0pt}%
\pgfpathmoveto{\pgfqpoint{10.766230in}{5.332455in}}%
\pgfpathcurveto{\pgfqpoint{10.777281in}{5.332455in}}{\pgfqpoint{10.787880in}{5.336845in}}{\pgfqpoint{10.795693in}{5.344659in}}%
\pgfpathcurveto{\pgfqpoint{10.803507in}{5.352472in}}{\pgfqpoint{10.807897in}{5.363071in}}{\pgfqpoint{10.807897in}{5.374122in}}%
\pgfpathcurveto{\pgfqpoint{10.807897in}{5.385172in}}{\pgfqpoint{10.803507in}{5.395771in}}{\pgfqpoint{10.795693in}{5.403584in}}%
\pgfpathcurveto{\pgfqpoint{10.787880in}{5.411398in}}{\pgfqpoint{10.777281in}{5.415788in}}{\pgfqpoint{10.766230in}{5.415788in}}%
\pgfpathcurveto{\pgfqpoint{10.755180in}{5.415788in}}{\pgfqpoint{10.744581in}{5.411398in}}{\pgfqpoint{10.736768in}{5.403584in}}%
\pgfpathcurveto{\pgfqpoint{10.728954in}{5.395771in}}{\pgfqpoint{10.724564in}{5.385172in}}{\pgfqpoint{10.724564in}{5.374122in}}%
\pgfpathcurveto{\pgfqpoint{10.724564in}{5.363071in}}{\pgfqpoint{10.728954in}{5.352472in}}{\pgfqpoint{10.736768in}{5.344659in}}%
\pgfpathcurveto{\pgfqpoint{10.744581in}{5.336845in}}{\pgfqpoint{10.755180in}{5.332455in}}{\pgfqpoint{10.766230in}{5.332455in}}%
\pgfpathlineto{\pgfqpoint{10.766230in}{5.332455in}}%
\pgfpathclose%
\pgfusepath{stroke}%
\end{pgfscope}%
\begin{pgfscope}%
\pgfpathrectangle{\pgfqpoint{7.394209in}{0.375000in}}{\pgfqpoint{6.356833in}{5.175000in}}%
\pgfusepath{clip}%
\pgfsetbuttcap%
\pgfsetroundjoin%
\pgfsetlinewidth{1.003750pt}%
\definecolor{currentstroke}{rgb}{1.000000,0.000000,0.000000}%
\pgfsetstrokecolor{currentstroke}%
\pgfsetdash{}{0pt}%
\pgfpathmoveto{\pgfqpoint{8.230316in}{1.257079in}}%
\pgfpathcurveto{\pgfqpoint{8.241366in}{1.257079in}}{\pgfqpoint{8.251965in}{1.261469in}}{\pgfqpoint{8.259779in}{1.269283in}}%
\pgfpathcurveto{\pgfqpoint{8.267593in}{1.277097in}}{\pgfqpoint{8.271983in}{1.287696in}}{\pgfqpoint{8.271983in}{1.298746in}}%
\pgfpathcurveto{\pgfqpoint{8.271983in}{1.309796in}}{\pgfqpoint{8.267593in}{1.320395in}}{\pgfqpoint{8.259779in}{1.328209in}}%
\pgfpathcurveto{\pgfqpoint{8.251965in}{1.336022in}}{\pgfqpoint{8.241366in}{1.340412in}}{\pgfqpoint{8.230316in}{1.340412in}}%
\pgfpathcurveto{\pgfqpoint{8.219266in}{1.340412in}}{\pgfqpoint{8.208667in}{1.336022in}}{\pgfqpoint{8.200853in}{1.328209in}}%
\pgfpathcurveto{\pgfqpoint{8.193040in}{1.320395in}}{\pgfqpoint{8.188650in}{1.309796in}}{\pgfqpoint{8.188650in}{1.298746in}}%
\pgfpathcurveto{\pgfqpoint{8.188650in}{1.287696in}}{\pgfqpoint{8.193040in}{1.277097in}}{\pgfqpoint{8.200853in}{1.269283in}}%
\pgfpathcurveto{\pgfqpoint{8.208667in}{1.261469in}}{\pgfqpoint{8.219266in}{1.257079in}}{\pgfqpoint{8.230316in}{1.257079in}}%
\pgfpathlineto{\pgfqpoint{8.230316in}{1.257079in}}%
\pgfpathclose%
\pgfusepath{stroke}%
\end{pgfscope}%
\begin{pgfscope}%
\pgfpathrectangle{\pgfqpoint{7.394209in}{0.375000in}}{\pgfqpoint{6.356833in}{5.175000in}}%
\pgfusepath{clip}%
\pgfsetbuttcap%
\pgfsetroundjoin%
\pgfsetlinewidth{1.003750pt}%
\definecolor{currentstroke}{rgb}{1.000000,0.000000,0.000000}%
\pgfsetstrokecolor{currentstroke}%
\pgfsetdash{}{0pt}%
\pgfpathmoveto{\pgfqpoint{12.346492in}{5.508108in}}%
\pgfpathcurveto{\pgfqpoint{12.357542in}{5.508108in}}{\pgfqpoint{12.368141in}{5.512498in}}{\pgfqpoint{12.375955in}{5.520312in}}%
\pgfpathcurveto{\pgfqpoint{12.383768in}{5.528125in}}{\pgfqpoint{12.388159in}{5.538724in}}{\pgfqpoint{12.388159in}{5.549774in}}%
\pgfpathcurveto{\pgfqpoint{12.388159in}{5.560824in}}{\pgfqpoint{12.383768in}{5.571424in}}{\pgfqpoint{12.375955in}{5.579237in}}%
\pgfpathcurveto{\pgfqpoint{12.368141in}{5.587051in}}{\pgfqpoint{12.357542in}{5.591441in}}{\pgfqpoint{12.346492in}{5.591441in}}%
\pgfpathcurveto{\pgfqpoint{12.335442in}{5.591441in}}{\pgfqpoint{12.324843in}{5.587051in}}{\pgfqpoint{12.317029in}{5.579237in}}%
\pgfpathcurveto{\pgfqpoint{12.309216in}{5.571424in}}{\pgfqpoint{12.304825in}{5.560824in}}{\pgfqpoint{12.304825in}{5.549774in}}%
\pgfpathcurveto{\pgfqpoint{12.304825in}{5.538724in}}{\pgfqpoint{12.309216in}{5.528125in}}{\pgfqpoint{12.317029in}{5.520312in}}%
\pgfpathcurveto{\pgfqpoint{12.324843in}{5.512498in}}{\pgfqpoint{12.335442in}{5.508108in}}{\pgfqpoint{12.346492in}{5.508108in}}%
\pgfpathlineto{\pgfqpoint{12.346492in}{5.508108in}}%
\pgfpathclose%
\pgfusepath{stroke}%
\end{pgfscope}%
\begin{pgfscope}%
\pgfpathrectangle{\pgfqpoint{7.394209in}{0.375000in}}{\pgfqpoint{6.356833in}{5.175000in}}%
\pgfusepath{clip}%
\pgfsetbuttcap%
\pgfsetroundjoin%
\pgfsetlinewidth{1.003750pt}%
\definecolor{currentstroke}{rgb}{1.000000,0.000000,0.000000}%
\pgfsetstrokecolor{currentstroke}%
\pgfsetdash{}{0pt}%
\pgfpathmoveto{\pgfqpoint{8.727403in}{2.506266in}}%
\pgfpathcurveto{\pgfqpoint{8.738453in}{2.506266in}}{\pgfqpoint{8.749052in}{2.510656in}}{\pgfqpoint{8.756866in}{2.518470in}}%
\pgfpathcurveto{\pgfqpoint{8.764680in}{2.526283in}}{\pgfqpoint{8.769070in}{2.536882in}}{\pgfqpoint{8.769070in}{2.547932in}}%
\pgfpathcurveto{\pgfqpoint{8.769070in}{2.558983in}}{\pgfqpoint{8.764680in}{2.569582in}}{\pgfqpoint{8.756866in}{2.577395in}}%
\pgfpathcurveto{\pgfqpoint{8.749052in}{2.585209in}}{\pgfqpoint{8.738453in}{2.589599in}}{\pgfqpoint{8.727403in}{2.589599in}}%
\pgfpathcurveto{\pgfqpoint{8.716353in}{2.589599in}}{\pgfqpoint{8.705754in}{2.585209in}}{\pgfqpoint{8.697941in}{2.577395in}}%
\pgfpathcurveto{\pgfqpoint{8.690127in}{2.569582in}}{\pgfqpoint{8.685737in}{2.558983in}}{\pgfqpoint{8.685737in}{2.547932in}}%
\pgfpathcurveto{\pgfqpoint{8.685737in}{2.536882in}}{\pgfqpoint{8.690127in}{2.526283in}}{\pgfqpoint{8.697941in}{2.518470in}}%
\pgfpathcurveto{\pgfqpoint{8.705754in}{2.510656in}}{\pgfqpoint{8.716353in}{2.506266in}}{\pgfqpoint{8.727403in}{2.506266in}}%
\pgfpathlineto{\pgfqpoint{8.727403in}{2.506266in}}%
\pgfpathclose%
\pgfusepath{stroke}%
\end{pgfscope}%
\begin{pgfscope}%
\pgfpathrectangle{\pgfqpoint{7.394209in}{0.375000in}}{\pgfqpoint{6.356833in}{5.175000in}}%
\pgfusepath{clip}%
\pgfsetbuttcap%
\pgfsetroundjoin%
\pgfsetlinewidth{1.003750pt}%
\definecolor{currentstroke}{rgb}{1.000000,0.000000,0.000000}%
\pgfsetstrokecolor{currentstroke}%
\pgfsetdash{}{0pt}%
\pgfpathmoveto{\pgfqpoint{11.346825in}{5.467983in}}%
\pgfpathcurveto{\pgfqpoint{11.357875in}{5.467983in}}{\pgfqpoint{11.368474in}{5.472374in}}{\pgfqpoint{11.376287in}{5.480187in}}%
\pgfpathcurveto{\pgfqpoint{11.384101in}{5.488001in}}{\pgfqpoint{11.388491in}{5.498600in}}{\pgfqpoint{11.388491in}{5.509650in}}%
\pgfpathcurveto{\pgfqpoint{11.388491in}{5.520700in}}{\pgfqpoint{11.384101in}{5.531299in}}{\pgfqpoint{11.376287in}{5.539113in}}%
\pgfpathcurveto{\pgfqpoint{11.368474in}{5.546927in}}{\pgfqpoint{11.357875in}{5.551317in}}{\pgfqpoint{11.346825in}{5.551317in}}%
\pgfpathcurveto{\pgfqpoint{11.335775in}{5.551317in}}{\pgfqpoint{11.325176in}{5.546927in}}{\pgfqpoint{11.317362in}{5.539113in}}%
\pgfpathcurveto{\pgfqpoint{11.309548in}{5.531299in}}{\pgfqpoint{11.305158in}{5.520700in}}{\pgfqpoint{11.305158in}{5.509650in}}%
\pgfpathcurveto{\pgfqpoint{11.305158in}{5.498600in}}{\pgfqpoint{11.309548in}{5.488001in}}{\pgfqpoint{11.317362in}{5.480187in}}%
\pgfpathcurveto{\pgfqpoint{11.325176in}{5.472374in}}{\pgfqpoint{11.335775in}{5.467983in}}{\pgfqpoint{11.346825in}{5.467983in}}%
\pgfpathlineto{\pgfqpoint{11.346825in}{5.467983in}}%
\pgfpathclose%
\pgfusepath{stroke}%
\end{pgfscope}%
\begin{pgfscope}%
\pgfpathrectangle{\pgfqpoint{7.394209in}{0.375000in}}{\pgfqpoint{6.356833in}{5.175000in}}%
\pgfusepath{clip}%
\pgfsetbuttcap%
\pgfsetroundjoin%
\pgfsetlinewidth{1.003750pt}%
\definecolor{currentstroke}{rgb}{1.000000,0.000000,0.000000}%
\pgfsetstrokecolor{currentstroke}%
\pgfsetdash{}{0pt}%
\pgfpathmoveto{\pgfqpoint{10.374644in}{3.898847in}}%
\pgfpathcurveto{\pgfqpoint{10.385694in}{3.898847in}}{\pgfqpoint{10.396293in}{3.903238in}}{\pgfqpoint{10.404107in}{3.911051in}}%
\pgfpathcurveto{\pgfqpoint{10.411921in}{3.918865in}}{\pgfqpoint{10.416311in}{3.929464in}}{\pgfqpoint{10.416311in}{3.940514in}}%
\pgfpathcurveto{\pgfqpoint{10.416311in}{3.951564in}}{\pgfqpoint{10.411921in}{3.962163in}}{\pgfqpoint{10.404107in}{3.969977in}}%
\pgfpathcurveto{\pgfqpoint{10.396293in}{3.977790in}}{\pgfqpoint{10.385694in}{3.982181in}}{\pgfqpoint{10.374644in}{3.982181in}}%
\pgfpathcurveto{\pgfqpoint{10.363594in}{3.982181in}}{\pgfqpoint{10.352995in}{3.977790in}}{\pgfqpoint{10.345182in}{3.969977in}}%
\pgfpathcurveto{\pgfqpoint{10.337368in}{3.962163in}}{\pgfqpoint{10.332978in}{3.951564in}}{\pgfqpoint{10.332978in}{3.940514in}}%
\pgfpathcurveto{\pgfqpoint{10.332978in}{3.929464in}}{\pgfqpoint{10.337368in}{3.918865in}}{\pgfqpoint{10.345182in}{3.911051in}}%
\pgfpathcurveto{\pgfqpoint{10.352995in}{3.903238in}}{\pgfqpoint{10.363594in}{3.898847in}}{\pgfqpoint{10.374644in}{3.898847in}}%
\pgfpathlineto{\pgfqpoint{10.374644in}{3.898847in}}%
\pgfpathclose%
\pgfusepath{stroke}%
\end{pgfscope}%
\begin{pgfscope}%
\pgfpathrectangle{\pgfqpoint{7.394209in}{0.375000in}}{\pgfqpoint{6.356833in}{5.175000in}}%
\pgfusepath{clip}%
\pgfsetbuttcap%
\pgfsetroundjoin%
\pgfsetlinewidth{1.003750pt}%
\definecolor{currentstroke}{rgb}{1.000000,0.000000,0.000000}%
\pgfsetstrokecolor{currentstroke}%
\pgfsetdash{}{0pt}%
\pgfpathmoveto{\pgfqpoint{8.242772in}{1.497561in}}%
\pgfpathcurveto{\pgfqpoint{8.253822in}{1.497561in}}{\pgfqpoint{8.264421in}{1.501951in}}{\pgfqpoint{8.272235in}{1.509765in}}%
\pgfpathcurveto{\pgfqpoint{8.280048in}{1.517578in}}{\pgfqpoint{8.284439in}{1.528177in}}{\pgfqpoint{8.284439in}{1.539228in}}%
\pgfpathcurveto{\pgfqpoint{8.284439in}{1.550278in}}{\pgfqpoint{8.280048in}{1.560877in}}{\pgfqpoint{8.272235in}{1.568690in}}%
\pgfpathcurveto{\pgfqpoint{8.264421in}{1.576504in}}{\pgfqpoint{8.253822in}{1.580894in}}{\pgfqpoint{8.242772in}{1.580894in}}%
\pgfpathcurveto{\pgfqpoint{8.231722in}{1.580894in}}{\pgfqpoint{8.221123in}{1.576504in}}{\pgfqpoint{8.213309in}{1.568690in}}%
\pgfpathcurveto{\pgfqpoint{8.205496in}{1.560877in}}{\pgfqpoint{8.201105in}{1.550278in}}{\pgfqpoint{8.201105in}{1.539228in}}%
\pgfpathcurveto{\pgfqpoint{8.201105in}{1.528177in}}{\pgfqpoint{8.205496in}{1.517578in}}{\pgfqpoint{8.213309in}{1.509765in}}%
\pgfpathcurveto{\pgfqpoint{8.221123in}{1.501951in}}{\pgfqpoint{8.231722in}{1.497561in}}{\pgfqpoint{8.242772in}{1.497561in}}%
\pgfpathlineto{\pgfqpoint{8.242772in}{1.497561in}}%
\pgfpathclose%
\pgfusepath{stroke}%
\end{pgfscope}%
\begin{pgfscope}%
\pgfpathrectangle{\pgfqpoint{7.394209in}{0.375000in}}{\pgfqpoint{6.356833in}{5.175000in}}%
\pgfusepath{clip}%
\pgfsetbuttcap%
\pgfsetroundjoin%
\pgfsetlinewidth{1.003750pt}%
\definecolor{currentstroke}{rgb}{1.000000,0.000000,0.000000}%
\pgfsetstrokecolor{currentstroke}%
\pgfsetdash{}{0pt}%
\pgfpathmoveto{\pgfqpoint{9.595606in}{3.534932in}}%
\pgfpathcurveto{\pgfqpoint{9.606656in}{3.534932in}}{\pgfqpoint{9.617255in}{3.539322in}}{\pgfqpoint{9.625069in}{3.547136in}}%
\pgfpathcurveto{\pgfqpoint{9.632883in}{3.554949in}}{\pgfqpoint{9.637273in}{3.565548in}}{\pgfqpoint{9.637273in}{3.576599in}}%
\pgfpathcurveto{\pgfqpoint{9.637273in}{3.587649in}}{\pgfqpoint{9.632883in}{3.598248in}}{\pgfqpoint{9.625069in}{3.606061in}}%
\pgfpathcurveto{\pgfqpoint{9.617255in}{3.613875in}}{\pgfqpoint{9.606656in}{3.618265in}}{\pgfqpoint{9.595606in}{3.618265in}}%
\pgfpathcurveto{\pgfqpoint{9.584556in}{3.618265in}}{\pgfqpoint{9.573957in}{3.613875in}}{\pgfqpoint{9.566144in}{3.606061in}}%
\pgfpathcurveto{\pgfqpoint{9.558330in}{3.598248in}}{\pgfqpoint{9.553940in}{3.587649in}}{\pgfqpoint{9.553940in}{3.576599in}}%
\pgfpathcurveto{\pgfqpoint{9.553940in}{3.565548in}}{\pgfqpoint{9.558330in}{3.554949in}}{\pgfqpoint{9.566144in}{3.547136in}}%
\pgfpathcurveto{\pgfqpoint{9.573957in}{3.539322in}}{\pgfqpoint{9.584556in}{3.534932in}}{\pgfqpoint{9.595606in}{3.534932in}}%
\pgfpathlineto{\pgfqpoint{9.595606in}{3.534932in}}%
\pgfpathclose%
\pgfusepath{stroke}%
\end{pgfscope}%
\begin{pgfscope}%
\pgfpathrectangle{\pgfqpoint{7.394209in}{0.375000in}}{\pgfqpoint{6.356833in}{5.175000in}}%
\pgfusepath{clip}%
\pgfsetbuttcap%
\pgfsetroundjoin%
\pgfsetlinewidth{1.003750pt}%
\definecolor{currentstroke}{rgb}{1.000000,0.000000,0.000000}%
\pgfsetstrokecolor{currentstroke}%
\pgfsetdash{}{0pt}%
\pgfpathmoveto{\pgfqpoint{12.766769in}{5.507351in}}%
\pgfpathcurveto{\pgfqpoint{12.777819in}{5.507351in}}{\pgfqpoint{12.788418in}{5.511742in}}{\pgfqpoint{12.796232in}{5.519555in}}%
\pgfpathcurveto{\pgfqpoint{12.804045in}{5.527369in}}{\pgfqpoint{12.808435in}{5.537968in}}{\pgfqpoint{12.808435in}{5.549018in}}%
\pgfpathcurveto{\pgfqpoint{12.808435in}{5.560068in}}{\pgfqpoint{12.804045in}{5.570667in}}{\pgfqpoint{12.796232in}{5.578481in}}%
\pgfpathcurveto{\pgfqpoint{12.788418in}{5.586294in}}{\pgfqpoint{12.777819in}{5.590685in}}{\pgfqpoint{12.766769in}{5.590685in}}%
\pgfpathcurveto{\pgfqpoint{12.755719in}{5.590685in}}{\pgfqpoint{12.745120in}{5.586294in}}{\pgfqpoint{12.737306in}{5.578481in}}%
\pgfpathcurveto{\pgfqpoint{12.729492in}{5.570667in}}{\pgfqpoint{12.725102in}{5.560068in}}{\pgfqpoint{12.725102in}{5.549018in}}%
\pgfpathcurveto{\pgfqpoint{12.725102in}{5.537968in}}{\pgfqpoint{12.729492in}{5.527369in}}{\pgfqpoint{12.737306in}{5.519555in}}%
\pgfpathcurveto{\pgfqpoint{12.745120in}{5.511742in}}{\pgfqpoint{12.755719in}{5.507351in}}{\pgfqpoint{12.766769in}{5.507351in}}%
\pgfpathlineto{\pgfqpoint{12.766769in}{5.507351in}}%
\pgfpathclose%
\pgfusepath{stroke}%
\end{pgfscope}%
\begin{pgfscope}%
\pgfpathrectangle{\pgfqpoint{7.394209in}{0.375000in}}{\pgfqpoint{6.356833in}{5.175000in}}%
\pgfusepath{clip}%
\pgfsetbuttcap%
\pgfsetroundjoin%
\pgfsetlinewidth{1.003750pt}%
\definecolor{currentstroke}{rgb}{1.000000,0.000000,0.000000}%
\pgfsetstrokecolor{currentstroke}%
\pgfsetdash{}{0pt}%
\pgfpathmoveto{\pgfqpoint{10.444964in}{4.692478in}}%
\pgfpathcurveto{\pgfqpoint{10.456014in}{4.692478in}}{\pgfqpoint{10.466613in}{4.696868in}}{\pgfqpoint{10.474427in}{4.704682in}}%
\pgfpathcurveto{\pgfqpoint{10.482241in}{4.712495in}}{\pgfqpoint{10.486631in}{4.723095in}}{\pgfqpoint{10.486631in}{4.734145in}}%
\pgfpathcurveto{\pgfqpoint{10.486631in}{4.745195in}}{\pgfqpoint{10.482241in}{4.755794in}}{\pgfqpoint{10.474427in}{4.763607in}}%
\pgfpathcurveto{\pgfqpoint{10.466613in}{4.771421in}}{\pgfqpoint{10.456014in}{4.775811in}}{\pgfqpoint{10.444964in}{4.775811in}}%
\pgfpathcurveto{\pgfqpoint{10.433914in}{4.775811in}}{\pgfqpoint{10.423315in}{4.771421in}}{\pgfqpoint{10.415501in}{4.763607in}}%
\pgfpathcurveto{\pgfqpoint{10.407688in}{4.755794in}}{\pgfqpoint{10.403297in}{4.745195in}}{\pgfqpoint{10.403297in}{4.734145in}}%
\pgfpathcurveto{\pgfqpoint{10.403297in}{4.723095in}}{\pgfqpoint{10.407688in}{4.712495in}}{\pgfqpoint{10.415501in}{4.704682in}}%
\pgfpathcurveto{\pgfqpoint{10.423315in}{4.696868in}}{\pgfqpoint{10.433914in}{4.692478in}}{\pgfqpoint{10.444964in}{4.692478in}}%
\pgfpathlineto{\pgfqpoint{10.444964in}{4.692478in}}%
\pgfpathclose%
\pgfusepath{stroke}%
\end{pgfscope}%
\begin{pgfscope}%
\pgfpathrectangle{\pgfqpoint{7.394209in}{0.375000in}}{\pgfqpoint{6.356833in}{5.175000in}}%
\pgfusepath{clip}%
\pgfsetbuttcap%
\pgfsetroundjoin%
\pgfsetlinewidth{1.003750pt}%
\definecolor{currentstroke}{rgb}{1.000000,0.000000,0.000000}%
\pgfsetstrokecolor{currentstroke}%
\pgfsetdash{}{0pt}%
\pgfpathmoveto{\pgfqpoint{13.567013in}{5.508101in}}%
\pgfpathcurveto{\pgfqpoint{13.578063in}{5.508101in}}{\pgfqpoint{13.588662in}{5.512491in}}{\pgfqpoint{13.596475in}{5.520305in}}%
\pgfpathcurveto{\pgfqpoint{13.604289in}{5.528118in}}{\pgfqpoint{13.608679in}{5.538717in}}{\pgfqpoint{13.608679in}{5.549767in}}%
\pgfpathcurveto{\pgfqpoint{13.608679in}{5.560817in}}{\pgfqpoint{13.604289in}{5.571416in}}{\pgfqpoint{13.596475in}{5.579230in}}%
\pgfpathcurveto{\pgfqpoint{13.588662in}{5.587044in}}{\pgfqpoint{13.578063in}{5.591434in}}{\pgfqpoint{13.567013in}{5.591434in}}%
\pgfpathcurveto{\pgfqpoint{13.555962in}{5.591434in}}{\pgfqpoint{13.545363in}{5.587044in}}{\pgfqpoint{13.537550in}{5.579230in}}%
\pgfpathcurveto{\pgfqpoint{13.529736in}{5.571416in}}{\pgfqpoint{13.525346in}{5.560817in}}{\pgfqpoint{13.525346in}{5.549767in}}%
\pgfpathcurveto{\pgfqpoint{13.525346in}{5.538717in}}{\pgfqpoint{13.529736in}{5.528118in}}{\pgfqpoint{13.537550in}{5.520305in}}%
\pgfpathcurveto{\pgfqpoint{13.545363in}{5.512491in}}{\pgfqpoint{13.555962in}{5.508101in}}{\pgfqpoint{13.567013in}{5.508101in}}%
\pgfpathlineto{\pgfqpoint{13.567013in}{5.508101in}}%
\pgfpathclose%
\pgfusepath{stroke}%
\end{pgfscope}%
\begin{pgfscope}%
\pgfpathrectangle{\pgfqpoint{7.394209in}{0.375000in}}{\pgfqpoint{6.356833in}{5.175000in}}%
\pgfusepath{clip}%
\pgfsetbuttcap%
\pgfsetroundjoin%
\pgfsetlinewidth{1.003750pt}%
\definecolor{currentstroke}{rgb}{1.000000,0.000000,0.000000}%
\pgfsetstrokecolor{currentstroke}%
\pgfsetdash{}{0pt}%
\pgfpathmoveto{\pgfqpoint{7.808254in}{0.974483in}}%
\pgfpathcurveto{\pgfqpoint{7.819304in}{0.974483in}}{\pgfqpoint{7.829903in}{0.978874in}}{\pgfqpoint{7.837716in}{0.986687in}}%
\pgfpathcurveto{\pgfqpoint{7.845530in}{0.994501in}}{\pgfqpoint{7.849920in}{1.005100in}}{\pgfqpoint{7.849920in}{1.016150in}}%
\pgfpathcurveto{\pgfqpoint{7.849920in}{1.027200in}}{\pgfqpoint{7.845530in}{1.037799in}}{\pgfqpoint{7.837716in}{1.045613in}}%
\pgfpathcurveto{\pgfqpoint{7.829903in}{1.053426in}}{\pgfqpoint{7.819304in}{1.057817in}}{\pgfqpoint{7.808254in}{1.057817in}}%
\pgfpathcurveto{\pgfqpoint{7.797203in}{1.057817in}}{\pgfqpoint{7.786604in}{1.053426in}}{\pgfqpoint{7.778791in}{1.045613in}}%
\pgfpathcurveto{\pgfqpoint{7.770977in}{1.037799in}}{\pgfqpoint{7.766587in}{1.027200in}}{\pgfqpoint{7.766587in}{1.016150in}}%
\pgfpathcurveto{\pgfqpoint{7.766587in}{1.005100in}}{\pgfqpoint{7.770977in}{0.994501in}}{\pgfqpoint{7.778791in}{0.986687in}}%
\pgfpathcurveto{\pgfqpoint{7.786604in}{0.978874in}}{\pgfqpoint{7.797203in}{0.974483in}}{\pgfqpoint{7.808254in}{0.974483in}}%
\pgfpathlineto{\pgfqpoint{7.808254in}{0.974483in}}%
\pgfpathclose%
\pgfusepath{stroke}%
\end{pgfscope}%
\begin{pgfscope}%
\pgfpathrectangle{\pgfqpoint{7.394209in}{0.375000in}}{\pgfqpoint{6.356833in}{5.175000in}}%
\pgfusepath{clip}%
\pgfsetbuttcap%
\pgfsetroundjoin%
\pgfsetlinewidth{1.003750pt}%
\definecolor{currentstroke}{rgb}{1.000000,0.000000,0.000000}%
\pgfsetstrokecolor{currentstroke}%
\pgfsetdash{}{0pt}%
\pgfpathmoveto{\pgfqpoint{10.374644in}{4.025189in}}%
\pgfpathcurveto{\pgfqpoint{10.385694in}{4.025189in}}{\pgfqpoint{10.396293in}{4.029579in}}{\pgfqpoint{10.404107in}{4.037393in}}%
\pgfpathcurveto{\pgfqpoint{10.411921in}{4.045206in}}{\pgfqpoint{10.416311in}{4.055805in}}{\pgfqpoint{10.416311in}{4.066856in}}%
\pgfpathcurveto{\pgfqpoint{10.416311in}{4.077906in}}{\pgfqpoint{10.411921in}{4.088505in}}{\pgfqpoint{10.404107in}{4.096318in}}%
\pgfpathcurveto{\pgfqpoint{10.396293in}{4.104132in}}{\pgfqpoint{10.385694in}{4.108522in}}{\pgfqpoint{10.374644in}{4.108522in}}%
\pgfpathcurveto{\pgfqpoint{10.363594in}{4.108522in}}{\pgfqpoint{10.352995in}{4.104132in}}{\pgfqpoint{10.345182in}{4.096318in}}%
\pgfpathcurveto{\pgfqpoint{10.337368in}{4.088505in}}{\pgfqpoint{10.332978in}{4.077906in}}{\pgfqpoint{10.332978in}{4.066856in}}%
\pgfpathcurveto{\pgfqpoint{10.332978in}{4.055805in}}{\pgfqpoint{10.337368in}{4.045206in}}{\pgfqpoint{10.345182in}{4.037393in}}%
\pgfpathcurveto{\pgfqpoint{10.352995in}{4.029579in}}{\pgfqpoint{10.363594in}{4.025189in}}{\pgfqpoint{10.374644in}{4.025189in}}%
\pgfpathlineto{\pgfqpoint{10.374644in}{4.025189in}}%
\pgfpathclose%
\pgfusepath{stroke}%
\end{pgfscope}%
\begin{pgfscope}%
\pgfpathrectangle{\pgfqpoint{7.394209in}{0.375000in}}{\pgfqpoint{6.356833in}{5.175000in}}%
\pgfusepath{clip}%
\pgfsetbuttcap%
\pgfsetroundjoin%
\pgfsetlinewidth{1.003750pt}%
\definecolor{currentstroke}{rgb}{1.000000,0.000000,0.000000}%
\pgfsetstrokecolor{currentstroke}%
\pgfsetdash{}{0pt}%
\pgfpathmoveto{\pgfqpoint{11.029754in}{4.035615in}}%
\pgfpathcurveto{\pgfqpoint{11.040804in}{4.035615in}}{\pgfqpoint{11.051403in}{4.040006in}}{\pgfqpoint{11.059217in}{4.047819in}}%
\pgfpathcurveto{\pgfqpoint{11.067031in}{4.055633in}}{\pgfqpoint{11.071421in}{4.066232in}}{\pgfqpoint{11.071421in}{4.077282in}}%
\pgfpathcurveto{\pgfqpoint{11.071421in}{4.088332in}}{\pgfqpoint{11.067031in}{4.098931in}}{\pgfqpoint{11.059217in}{4.106745in}}%
\pgfpathcurveto{\pgfqpoint{11.051403in}{4.114559in}}{\pgfqpoint{11.040804in}{4.118949in}}{\pgfqpoint{11.029754in}{4.118949in}}%
\pgfpathcurveto{\pgfqpoint{11.018704in}{4.118949in}}{\pgfqpoint{11.008105in}{4.114559in}}{\pgfqpoint{11.000291in}{4.106745in}}%
\pgfpathcurveto{\pgfqpoint{10.992478in}{4.098931in}}{\pgfqpoint{10.988088in}{4.088332in}}{\pgfqpoint{10.988088in}{4.077282in}}%
\pgfpathcurveto{\pgfqpoint{10.988088in}{4.066232in}}{\pgfqpoint{10.992478in}{4.055633in}}{\pgfqpoint{11.000291in}{4.047819in}}%
\pgfpathcurveto{\pgfqpoint{11.008105in}{4.040006in}}{\pgfqpoint{11.018704in}{4.035615in}}{\pgfqpoint{11.029754in}{4.035615in}}%
\pgfpathlineto{\pgfqpoint{11.029754in}{4.035615in}}%
\pgfpathclose%
\pgfusepath{stroke}%
\end{pgfscope}%
\begin{pgfscope}%
\pgfpathrectangle{\pgfqpoint{7.394209in}{0.375000in}}{\pgfqpoint{6.356833in}{5.175000in}}%
\pgfusepath{clip}%
\pgfsetbuttcap%
\pgfsetroundjoin%
\pgfsetlinewidth{1.003750pt}%
\definecolor{currentstroke}{rgb}{1.000000,0.000000,0.000000}%
\pgfsetstrokecolor{currentstroke}%
\pgfsetdash{}{0pt}%
\pgfpathmoveto{\pgfqpoint{11.757350in}{5.343816in}}%
\pgfpathcurveto{\pgfqpoint{11.768401in}{5.343816in}}{\pgfqpoint{11.779000in}{5.348206in}}{\pgfqpoint{11.786813in}{5.356020in}}%
\pgfpathcurveto{\pgfqpoint{11.794627in}{5.363833in}}{\pgfqpoint{11.799017in}{5.374432in}}{\pgfqpoint{11.799017in}{5.385482in}}%
\pgfpathcurveto{\pgfqpoint{11.799017in}{5.396532in}}{\pgfqpoint{11.794627in}{5.407132in}}{\pgfqpoint{11.786813in}{5.414945in}}%
\pgfpathcurveto{\pgfqpoint{11.779000in}{5.422759in}}{\pgfqpoint{11.768401in}{5.427149in}}{\pgfqpoint{11.757350in}{5.427149in}}%
\pgfpathcurveto{\pgfqpoint{11.746300in}{5.427149in}}{\pgfqpoint{11.735701in}{5.422759in}}{\pgfqpoint{11.727888in}{5.414945in}}%
\pgfpathcurveto{\pgfqpoint{11.720074in}{5.407132in}}{\pgfqpoint{11.715684in}{5.396532in}}{\pgfqpoint{11.715684in}{5.385482in}}%
\pgfpathcurveto{\pgfqpoint{11.715684in}{5.374432in}}{\pgfqpoint{11.720074in}{5.363833in}}{\pgfqpoint{11.727888in}{5.356020in}}%
\pgfpathcurveto{\pgfqpoint{11.735701in}{5.348206in}}{\pgfqpoint{11.746300in}{5.343816in}}{\pgfqpoint{11.757350in}{5.343816in}}%
\pgfpathlineto{\pgfqpoint{11.757350in}{5.343816in}}%
\pgfpathclose%
\pgfusepath{stroke}%
\end{pgfscope}%
\begin{pgfscope}%
\pgfpathrectangle{\pgfqpoint{7.394209in}{0.375000in}}{\pgfqpoint{6.356833in}{5.175000in}}%
\pgfusepath{clip}%
\pgfsetbuttcap%
\pgfsetroundjoin%
\pgfsetlinewidth{1.003750pt}%
\definecolor{currentstroke}{rgb}{1.000000,0.000000,0.000000}%
\pgfsetstrokecolor{currentstroke}%
\pgfsetdash{}{0pt}%
\pgfpathmoveto{\pgfqpoint{7.658579in}{0.572285in}}%
\pgfpathcurveto{\pgfqpoint{7.669629in}{0.572285in}}{\pgfqpoint{7.680228in}{0.576676in}}{\pgfqpoint{7.688042in}{0.584489in}}%
\pgfpathcurveto{\pgfqpoint{7.695856in}{0.592303in}}{\pgfqpoint{7.700246in}{0.602902in}}{\pgfqpoint{7.700246in}{0.613952in}}%
\pgfpathcurveto{\pgfqpoint{7.700246in}{0.625002in}}{\pgfqpoint{7.695856in}{0.635601in}}{\pgfqpoint{7.688042in}{0.643415in}}%
\pgfpathcurveto{\pgfqpoint{7.680228in}{0.651228in}}{\pgfqpoint{7.669629in}{0.655619in}}{\pgfqpoint{7.658579in}{0.655619in}}%
\pgfpathcurveto{\pgfqpoint{7.647529in}{0.655619in}}{\pgfqpoint{7.636930in}{0.651228in}}{\pgfqpoint{7.629117in}{0.643415in}}%
\pgfpathcurveto{\pgfqpoint{7.621303in}{0.635601in}}{\pgfqpoint{7.616913in}{0.625002in}}{\pgfqpoint{7.616913in}{0.613952in}}%
\pgfpathcurveto{\pgfqpoint{7.616913in}{0.602902in}}{\pgfqpoint{7.621303in}{0.592303in}}{\pgfqpoint{7.629117in}{0.584489in}}%
\pgfpathcurveto{\pgfqpoint{7.636930in}{0.576676in}}{\pgfqpoint{7.647529in}{0.572285in}}{\pgfqpoint{7.658579in}{0.572285in}}%
\pgfpathlineto{\pgfqpoint{7.658579in}{0.572285in}}%
\pgfpathclose%
\pgfusepath{stroke}%
\end{pgfscope}%
\begin{pgfscope}%
\pgfpathrectangle{\pgfqpoint{7.394209in}{0.375000in}}{\pgfqpoint{6.356833in}{5.175000in}}%
\pgfusepath{clip}%
\pgfsetbuttcap%
\pgfsetroundjoin%
\pgfsetlinewidth{1.003750pt}%
\definecolor{currentstroke}{rgb}{1.000000,0.000000,0.000000}%
\pgfsetstrokecolor{currentstroke}%
\pgfsetdash{}{0pt}%
\pgfpathmoveto{\pgfqpoint{10.196360in}{3.964095in}}%
\pgfpathcurveto{\pgfqpoint{10.207410in}{3.964095in}}{\pgfqpoint{10.218009in}{3.968485in}}{\pgfqpoint{10.225822in}{3.976299in}}%
\pgfpathcurveto{\pgfqpoint{10.233636in}{3.984113in}}{\pgfqpoint{10.238026in}{3.994712in}}{\pgfqpoint{10.238026in}{4.005762in}}%
\pgfpathcurveto{\pgfqpoint{10.238026in}{4.016812in}}{\pgfqpoint{10.233636in}{4.027411in}}{\pgfqpoint{10.225822in}{4.035225in}}%
\pgfpathcurveto{\pgfqpoint{10.218009in}{4.043038in}}{\pgfqpoint{10.207410in}{4.047428in}}{\pgfqpoint{10.196360in}{4.047428in}}%
\pgfpathcurveto{\pgfqpoint{10.185309in}{4.047428in}}{\pgfqpoint{10.174710in}{4.043038in}}{\pgfqpoint{10.166897in}{4.035225in}}%
\pgfpathcurveto{\pgfqpoint{10.159083in}{4.027411in}}{\pgfqpoint{10.154693in}{4.016812in}}{\pgfqpoint{10.154693in}{4.005762in}}%
\pgfpathcurveto{\pgfqpoint{10.154693in}{3.994712in}}{\pgfqpoint{10.159083in}{3.984113in}}{\pgfqpoint{10.166897in}{3.976299in}}%
\pgfpathcurveto{\pgfqpoint{10.174710in}{3.968485in}}{\pgfqpoint{10.185309in}{3.964095in}}{\pgfqpoint{10.196360in}{3.964095in}}%
\pgfpathlineto{\pgfqpoint{10.196360in}{3.964095in}}%
\pgfpathclose%
\pgfusepath{stroke}%
\end{pgfscope}%
\begin{pgfscope}%
\pgfpathrectangle{\pgfqpoint{7.394209in}{0.375000in}}{\pgfqpoint{6.356833in}{5.175000in}}%
\pgfusepath{clip}%
\pgfsetbuttcap%
\pgfsetroundjoin%
\pgfsetlinewidth{1.003750pt}%
\definecolor{currentstroke}{rgb}{1.000000,0.000000,0.000000}%
\pgfsetstrokecolor{currentstroke}%
\pgfsetdash{}{0pt}%
\pgfpathmoveto{\pgfqpoint{9.016520in}{2.342188in}}%
\pgfpathcurveto{\pgfqpoint{9.027570in}{2.342188in}}{\pgfqpoint{9.038169in}{2.346578in}}{\pgfqpoint{9.045983in}{2.354392in}}%
\pgfpathcurveto{\pgfqpoint{9.053797in}{2.362206in}}{\pgfqpoint{9.058187in}{2.372805in}}{\pgfqpoint{9.058187in}{2.383855in}}%
\pgfpathcurveto{\pgfqpoint{9.058187in}{2.394905in}}{\pgfqpoint{9.053797in}{2.405504in}}{\pgfqpoint{9.045983in}{2.413318in}}%
\pgfpathcurveto{\pgfqpoint{9.038169in}{2.421131in}}{\pgfqpoint{9.027570in}{2.425521in}}{\pgfqpoint{9.016520in}{2.425521in}}%
\pgfpathcurveto{\pgfqpoint{9.005470in}{2.425521in}}{\pgfqpoint{8.994871in}{2.421131in}}{\pgfqpoint{8.987058in}{2.413318in}}%
\pgfpathcurveto{\pgfqpoint{8.979244in}{2.405504in}}{\pgfqpoint{8.974854in}{2.394905in}}{\pgfqpoint{8.974854in}{2.383855in}}%
\pgfpathcurveto{\pgfqpoint{8.974854in}{2.372805in}}{\pgfqpoint{8.979244in}{2.362206in}}{\pgfqpoint{8.987058in}{2.354392in}}%
\pgfpathcurveto{\pgfqpoint{8.994871in}{2.346578in}}{\pgfqpoint{9.005470in}{2.342188in}}{\pgfqpoint{9.016520in}{2.342188in}}%
\pgfpathlineto{\pgfqpoint{9.016520in}{2.342188in}}%
\pgfpathclose%
\pgfusepath{stroke}%
\end{pgfscope}%
\begin{pgfscope}%
\pgfpathrectangle{\pgfqpoint{7.394209in}{0.375000in}}{\pgfqpoint{6.356833in}{5.175000in}}%
\pgfusepath{clip}%
\pgfsetbuttcap%
\pgfsetroundjoin%
\pgfsetlinewidth{1.003750pt}%
\definecolor{currentstroke}{rgb}{1.000000,0.000000,0.000000}%
\pgfsetstrokecolor{currentstroke}%
\pgfsetdash{}{0pt}%
\pgfpathmoveto{\pgfqpoint{12.193765in}{5.508318in}}%
\pgfpathcurveto{\pgfqpoint{12.204815in}{5.508318in}}{\pgfqpoint{12.215414in}{5.512709in}}{\pgfqpoint{12.223228in}{5.520522in}}%
\pgfpathcurveto{\pgfqpoint{12.231041in}{5.528336in}}{\pgfqpoint{12.235432in}{5.538935in}}{\pgfqpoint{12.235432in}{5.549985in}}%
\pgfpathcurveto{\pgfqpoint{12.235432in}{5.561035in}}{\pgfqpoint{12.231041in}{5.571634in}}{\pgfqpoint{12.223228in}{5.579448in}}%
\pgfpathcurveto{\pgfqpoint{12.215414in}{5.587262in}}{\pgfqpoint{12.204815in}{5.591652in}}{\pgfqpoint{12.193765in}{5.591652in}}%
\pgfpathcurveto{\pgfqpoint{12.182715in}{5.591652in}}{\pgfqpoint{12.172116in}{5.587262in}}{\pgfqpoint{12.164302in}{5.579448in}}%
\pgfpathcurveto{\pgfqpoint{12.156489in}{5.571634in}}{\pgfqpoint{12.152098in}{5.561035in}}{\pgfqpoint{12.152098in}{5.549985in}}%
\pgfpathcurveto{\pgfqpoint{12.152098in}{5.538935in}}{\pgfqpoint{12.156489in}{5.528336in}}{\pgfqpoint{12.164302in}{5.520522in}}%
\pgfpathcurveto{\pgfqpoint{12.172116in}{5.512709in}}{\pgfqpoint{12.182715in}{5.508318in}}{\pgfqpoint{12.193765in}{5.508318in}}%
\pgfpathlineto{\pgfqpoint{12.193765in}{5.508318in}}%
\pgfpathclose%
\pgfusepath{stroke}%
\end{pgfscope}%
\begin{pgfscope}%
\pgfpathrectangle{\pgfqpoint{7.394209in}{0.375000in}}{\pgfqpoint{6.356833in}{5.175000in}}%
\pgfusepath{clip}%
\pgfsetbuttcap%
\pgfsetroundjoin%
\pgfsetlinewidth{1.003750pt}%
\definecolor{currentstroke}{rgb}{1.000000,0.000000,0.000000}%
\pgfsetstrokecolor{currentstroke}%
\pgfsetdash{}{0pt}%
\pgfpathmoveto{\pgfqpoint{10.845332in}{4.623892in}}%
\pgfpathcurveto{\pgfqpoint{10.856382in}{4.623892in}}{\pgfqpoint{10.866981in}{4.628282in}}{\pgfqpoint{10.874795in}{4.636096in}}%
\pgfpathcurveto{\pgfqpoint{10.882609in}{4.643910in}}{\pgfqpoint{10.886999in}{4.654509in}}{\pgfqpoint{10.886999in}{4.665559in}}%
\pgfpathcurveto{\pgfqpoint{10.886999in}{4.676609in}}{\pgfqpoint{10.882609in}{4.687208in}}{\pgfqpoint{10.874795in}{4.695022in}}%
\pgfpathcurveto{\pgfqpoint{10.866981in}{4.702835in}}{\pgfqpoint{10.856382in}{4.707225in}}{\pgfqpoint{10.845332in}{4.707225in}}%
\pgfpathcurveto{\pgfqpoint{10.834282in}{4.707225in}}{\pgfqpoint{10.823683in}{4.702835in}}{\pgfqpoint{10.815869in}{4.695022in}}%
\pgfpathcurveto{\pgfqpoint{10.808056in}{4.687208in}}{\pgfqpoint{10.803665in}{4.676609in}}{\pgfqpoint{10.803665in}{4.665559in}}%
\pgfpathcurveto{\pgfqpoint{10.803665in}{4.654509in}}{\pgfqpoint{10.808056in}{4.643910in}}{\pgfqpoint{10.815869in}{4.636096in}}%
\pgfpathcurveto{\pgfqpoint{10.823683in}{4.628282in}}{\pgfqpoint{10.834282in}{4.623892in}}{\pgfqpoint{10.845332in}{4.623892in}}%
\pgfpathlineto{\pgfqpoint{10.845332in}{4.623892in}}%
\pgfpathclose%
\pgfusepath{stroke}%
\end{pgfscope}%
\begin{pgfscope}%
\pgfpathrectangle{\pgfqpoint{7.394209in}{0.375000in}}{\pgfqpoint{6.356833in}{5.175000in}}%
\pgfusepath{clip}%
\pgfsetbuttcap%
\pgfsetroundjoin%
\pgfsetlinewidth{1.003750pt}%
\definecolor{currentstroke}{rgb}{1.000000,0.000000,0.000000}%
\pgfsetstrokecolor{currentstroke}%
\pgfsetdash{}{0pt}%
\pgfpathmoveto{\pgfqpoint{8.455602in}{1.848394in}}%
\pgfpathcurveto{\pgfqpoint{8.466652in}{1.848394in}}{\pgfqpoint{8.477251in}{1.852785in}}{\pgfqpoint{8.485064in}{1.860598in}}%
\pgfpathcurveto{\pgfqpoint{8.492878in}{1.868412in}}{\pgfqpoint{8.497268in}{1.879011in}}{\pgfqpoint{8.497268in}{1.890061in}}%
\pgfpathcurveto{\pgfqpoint{8.497268in}{1.901111in}}{\pgfqpoint{8.492878in}{1.911710in}}{\pgfqpoint{8.485064in}{1.919524in}}%
\pgfpathcurveto{\pgfqpoint{8.477251in}{1.927337in}}{\pgfqpoint{8.466652in}{1.931728in}}{\pgfqpoint{8.455602in}{1.931728in}}%
\pgfpathcurveto{\pgfqpoint{8.444551in}{1.931728in}}{\pgfqpoint{8.433952in}{1.927337in}}{\pgfqpoint{8.426139in}{1.919524in}}%
\pgfpathcurveto{\pgfqpoint{8.418325in}{1.911710in}}{\pgfqpoint{8.413935in}{1.901111in}}{\pgfqpoint{8.413935in}{1.890061in}}%
\pgfpathcurveto{\pgfqpoint{8.413935in}{1.879011in}}{\pgfqpoint{8.418325in}{1.868412in}}{\pgfqpoint{8.426139in}{1.860598in}}%
\pgfpathcurveto{\pgfqpoint{8.433952in}{1.852785in}}{\pgfqpoint{8.444551in}{1.848394in}}{\pgfqpoint{8.455602in}{1.848394in}}%
\pgfpathlineto{\pgfqpoint{8.455602in}{1.848394in}}%
\pgfpathclose%
\pgfusepath{stroke}%
\end{pgfscope}%
\begin{pgfscope}%
\pgfpathrectangle{\pgfqpoint{7.394209in}{0.375000in}}{\pgfqpoint{6.356833in}{5.175000in}}%
\pgfusepath{clip}%
\pgfsetbuttcap%
\pgfsetroundjoin%
\pgfsetlinewidth{1.003750pt}%
\definecolor{currentstroke}{rgb}{1.000000,0.000000,0.000000}%
\pgfsetstrokecolor{currentstroke}%
\pgfsetdash{}{0pt}%
\pgfpathmoveto{\pgfqpoint{10.177504in}{3.898169in}}%
\pgfpathcurveto{\pgfqpoint{10.188554in}{3.898169in}}{\pgfqpoint{10.199153in}{3.902559in}}{\pgfqpoint{10.206966in}{3.910373in}}%
\pgfpathcurveto{\pgfqpoint{10.214780in}{3.918186in}}{\pgfqpoint{10.219170in}{3.928785in}}{\pgfqpoint{10.219170in}{3.939835in}}%
\pgfpathcurveto{\pgfqpoint{10.219170in}{3.950885in}}{\pgfqpoint{10.214780in}{3.961485in}}{\pgfqpoint{10.206966in}{3.969298in}}%
\pgfpathcurveto{\pgfqpoint{10.199153in}{3.977112in}}{\pgfqpoint{10.188554in}{3.981502in}}{\pgfqpoint{10.177504in}{3.981502in}}%
\pgfpathcurveto{\pgfqpoint{10.166453in}{3.981502in}}{\pgfqpoint{10.155854in}{3.977112in}}{\pgfqpoint{10.148041in}{3.969298in}}%
\pgfpathcurveto{\pgfqpoint{10.140227in}{3.961485in}}{\pgfqpoint{10.135837in}{3.950885in}}{\pgfqpoint{10.135837in}{3.939835in}}%
\pgfpathcurveto{\pgfqpoint{10.135837in}{3.928785in}}{\pgfqpoint{10.140227in}{3.918186in}}{\pgfqpoint{10.148041in}{3.910373in}}%
\pgfpathcurveto{\pgfqpoint{10.155854in}{3.902559in}}{\pgfqpoint{10.166453in}{3.898169in}}{\pgfqpoint{10.177504in}{3.898169in}}%
\pgfpathlineto{\pgfqpoint{10.177504in}{3.898169in}}%
\pgfpathclose%
\pgfusepath{stroke}%
\end{pgfscope}%
\begin{pgfscope}%
\pgfpathrectangle{\pgfqpoint{7.394209in}{0.375000in}}{\pgfqpoint{6.356833in}{5.175000in}}%
\pgfusepath{clip}%
\pgfsetbuttcap%
\pgfsetroundjoin%
\pgfsetlinewidth{1.003750pt}%
\definecolor{currentstroke}{rgb}{1.000000,0.000000,0.000000}%
\pgfsetstrokecolor{currentstroke}%
\pgfsetdash{}{0pt}%
\pgfpathmoveto{\pgfqpoint{12.070870in}{5.475282in}}%
\pgfpathcurveto{\pgfqpoint{12.081920in}{5.475282in}}{\pgfqpoint{12.092519in}{5.479672in}}{\pgfqpoint{12.100333in}{5.487485in}}%
\pgfpathcurveto{\pgfqpoint{12.108147in}{5.495299in}}{\pgfqpoint{12.112537in}{5.505898in}}{\pgfqpoint{12.112537in}{5.516948in}}%
\pgfpathcurveto{\pgfqpoint{12.112537in}{5.527998in}}{\pgfqpoint{12.108147in}{5.538597in}}{\pgfqpoint{12.100333in}{5.546411in}}%
\pgfpathcurveto{\pgfqpoint{12.092519in}{5.554225in}}{\pgfqpoint{12.081920in}{5.558615in}}{\pgfqpoint{12.070870in}{5.558615in}}%
\pgfpathcurveto{\pgfqpoint{12.059820in}{5.558615in}}{\pgfqpoint{12.049221in}{5.554225in}}{\pgfqpoint{12.041408in}{5.546411in}}%
\pgfpathcurveto{\pgfqpoint{12.033594in}{5.538597in}}{\pgfqpoint{12.029204in}{5.527998in}}{\pgfqpoint{12.029204in}{5.516948in}}%
\pgfpathcurveto{\pgfqpoint{12.029204in}{5.505898in}}{\pgfqpoint{12.033594in}{5.495299in}}{\pgfqpoint{12.041408in}{5.487485in}}%
\pgfpathcurveto{\pgfqpoint{12.049221in}{5.479672in}}{\pgfqpoint{12.059820in}{5.475282in}}{\pgfqpoint{12.070870in}{5.475282in}}%
\pgfpathlineto{\pgfqpoint{12.070870in}{5.475282in}}%
\pgfpathclose%
\pgfusepath{stroke}%
\end{pgfscope}%
\begin{pgfscope}%
\pgfpathrectangle{\pgfqpoint{7.394209in}{0.375000in}}{\pgfqpoint{6.356833in}{5.175000in}}%
\pgfusepath{clip}%
\pgfsetbuttcap%
\pgfsetroundjoin%
\pgfsetlinewidth{1.003750pt}%
\definecolor{currentstroke}{rgb}{1.000000,0.000000,0.000000}%
\pgfsetstrokecolor{currentstroke}%
\pgfsetdash{}{0pt}%
\pgfpathmoveto{\pgfqpoint{13.011222in}{5.508326in}}%
\pgfpathcurveto{\pgfqpoint{13.022272in}{5.508326in}}{\pgfqpoint{13.032871in}{5.512716in}}{\pgfqpoint{13.040685in}{5.520530in}}%
\pgfpathcurveto{\pgfqpoint{13.048499in}{5.528343in}}{\pgfqpoint{13.052889in}{5.538942in}}{\pgfqpoint{13.052889in}{5.549992in}}%
\pgfpathcurveto{\pgfqpoint{13.052889in}{5.561043in}}{\pgfqpoint{13.048499in}{5.571642in}}{\pgfqpoint{13.040685in}{5.579455in}}%
\pgfpathcurveto{\pgfqpoint{13.032871in}{5.587269in}}{\pgfqpoint{13.022272in}{5.591659in}}{\pgfqpoint{13.011222in}{5.591659in}}%
\pgfpathcurveto{\pgfqpoint{13.000172in}{5.591659in}}{\pgfqpoint{12.989573in}{5.587269in}}{\pgfqpoint{12.981760in}{5.579455in}}%
\pgfpathcurveto{\pgfqpoint{12.973946in}{5.571642in}}{\pgfqpoint{12.969556in}{5.561043in}}{\pgfqpoint{12.969556in}{5.549992in}}%
\pgfpathcurveto{\pgfqpoint{12.969556in}{5.538942in}}{\pgfqpoint{12.973946in}{5.528343in}}{\pgfqpoint{12.981760in}{5.520530in}}%
\pgfpathcurveto{\pgfqpoint{12.989573in}{5.512716in}}{\pgfqpoint{13.000172in}{5.508326in}}{\pgfqpoint{13.011222in}{5.508326in}}%
\pgfpathlineto{\pgfqpoint{13.011222in}{5.508326in}}%
\pgfpathclose%
\pgfusepath{stroke}%
\end{pgfscope}%
\begin{pgfscope}%
\pgfpathrectangle{\pgfqpoint{7.394209in}{0.375000in}}{\pgfqpoint{6.356833in}{5.175000in}}%
\pgfusepath{clip}%
\pgfsetbuttcap%
\pgfsetroundjoin%
\pgfsetlinewidth{1.003750pt}%
\definecolor{currentstroke}{rgb}{1.000000,0.000000,0.000000}%
\pgfsetstrokecolor{currentstroke}%
\pgfsetdash{}{0pt}%
\pgfpathmoveto{\pgfqpoint{9.635428in}{3.282844in}}%
\pgfpathcurveto{\pgfqpoint{9.646478in}{3.282844in}}{\pgfqpoint{9.657078in}{3.287234in}}{\pgfqpoint{9.664891in}{3.295048in}}%
\pgfpathcurveto{\pgfqpoint{9.672705in}{3.302862in}}{\pgfqpoint{9.677095in}{3.313461in}}{\pgfqpoint{9.677095in}{3.324511in}}%
\pgfpathcurveto{\pgfqpoint{9.677095in}{3.335561in}}{\pgfqpoint{9.672705in}{3.346160in}}{\pgfqpoint{9.664891in}{3.353974in}}%
\pgfpathcurveto{\pgfqpoint{9.657078in}{3.361787in}}{\pgfqpoint{9.646478in}{3.366177in}}{\pgfqpoint{9.635428in}{3.366177in}}%
\pgfpathcurveto{\pgfqpoint{9.624378in}{3.366177in}}{\pgfqpoint{9.613779in}{3.361787in}}{\pgfqpoint{9.605966in}{3.353974in}}%
\pgfpathcurveto{\pgfqpoint{9.598152in}{3.346160in}}{\pgfqpoint{9.593762in}{3.335561in}}{\pgfqpoint{9.593762in}{3.324511in}}%
\pgfpathcurveto{\pgfqpoint{9.593762in}{3.313461in}}{\pgfqpoint{9.598152in}{3.302862in}}{\pgfqpoint{9.605966in}{3.295048in}}%
\pgfpathcurveto{\pgfqpoint{9.613779in}{3.287234in}}{\pgfqpoint{9.624378in}{3.282844in}}{\pgfqpoint{9.635428in}{3.282844in}}%
\pgfpathlineto{\pgfqpoint{9.635428in}{3.282844in}}%
\pgfpathclose%
\pgfusepath{stroke}%
\end{pgfscope}%
\begin{pgfscope}%
\pgfpathrectangle{\pgfqpoint{7.394209in}{0.375000in}}{\pgfqpoint{6.356833in}{5.175000in}}%
\pgfusepath{clip}%
\pgfsetbuttcap%
\pgfsetroundjoin%
\pgfsetlinewidth{1.003750pt}%
\definecolor{currentstroke}{rgb}{1.000000,0.000000,0.000000}%
\pgfsetstrokecolor{currentstroke}%
\pgfsetdash{}{0pt}%
\pgfpathmoveto{\pgfqpoint{9.483886in}{2.879169in}}%
\pgfpathcurveto{\pgfqpoint{9.494936in}{2.879169in}}{\pgfqpoint{9.505535in}{2.883559in}}{\pgfqpoint{9.513349in}{2.891373in}}%
\pgfpathcurveto{\pgfqpoint{9.521163in}{2.899186in}}{\pgfqpoint{9.525553in}{2.909785in}}{\pgfqpoint{9.525553in}{2.920835in}}%
\pgfpathcurveto{\pgfqpoint{9.525553in}{2.931885in}}{\pgfqpoint{9.521163in}{2.942484in}}{\pgfqpoint{9.513349in}{2.950298in}}%
\pgfpathcurveto{\pgfqpoint{9.505535in}{2.958112in}}{\pgfqpoint{9.494936in}{2.962502in}}{\pgfqpoint{9.483886in}{2.962502in}}%
\pgfpathcurveto{\pgfqpoint{9.472836in}{2.962502in}}{\pgfqpoint{9.462237in}{2.958112in}}{\pgfqpoint{9.454423in}{2.950298in}}%
\pgfpathcurveto{\pgfqpoint{9.446610in}{2.942484in}}{\pgfqpoint{9.442220in}{2.931885in}}{\pgfqpoint{9.442220in}{2.920835in}}%
\pgfpathcurveto{\pgfqpoint{9.442220in}{2.909785in}}{\pgfqpoint{9.446610in}{2.899186in}}{\pgfqpoint{9.454423in}{2.891373in}}%
\pgfpathcurveto{\pgfqpoint{9.462237in}{2.883559in}}{\pgfqpoint{9.472836in}{2.879169in}}{\pgfqpoint{9.483886in}{2.879169in}}%
\pgfpathlineto{\pgfqpoint{9.483886in}{2.879169in}}%
\pgfpathclose%
\pgfusepath{stroke}%
\end{pgfscope}%
\begin{pgfscope}%
\pgfpathrectangle{\pgfqpoint{7.394209in}{0.375000in}}{\pgfqpoint{6.356833in}{5.175000in}}%
\pgfusepath{clip}%
\pgfsetbuttcap%
\pgfsetroundjoin%
\pgfsetlinewidth{1.003750pt}%
\definecolor{currentstroke}{rgb}{1.000000,0.000000,0.000000}%
\pgfsetstrokecolor{currentstroke}%
\pgfsetdash{}{0pt}%
\pgfpathmoveto{\pgfqpoint{10.287902in}{4.296932in}}%
\pgfpathcurveto{\pgfqpoint{10.298952in}{4.296932in}}{\pgfqpoint{10.309551in}{4.301322in}}{\pgfqpoint{10.317364in}{4.309136in}}%
\pgfpathcurveto{\pgfqpoint{10.325178in}{4.316949in}}{\pgfqpoint{10.329568in}{4.327548in}}{\pgfqpoint{10.329568in}{4.338599in}}%
\pgfpathcurveto{\pgfqpoint{10.329568in}{4.349649in}}{\pgfqpoint{10.325178in}{4.360248in}}{\pgfqpoint{10.317364in}{4.368061in}}%
\pgfpathcurveto{\pgfqpoint{10.309551in}{4.375875in}}{\pgfqpoint{10.298952in}{4.380265in}}{\pgfqpoint{10.287902in}{4.380265in}}%
\pgfpathcurveto{\pgfqpoint{10.276851in}{4.380265in}}{\pgfqpoint{10.266252in}{4.375875in}}{\pgfqpoint{10.258439in}{4.368061in}}%
\pgfpathcurveto{\pgfqpoint{10.250625in}{4.360248in}}{\pgfqpoint{10.246235in}{4.349649in}}{\pgfqpoint{10.246235in}{4.338599in}}%
\pgfpathcurveto{\pgfqpoint{10.246235in}{4.327548in}}{\pgfqpoint{10.250625in}{4.316949in}}{\pgfqpoint{10.258439in}{4.309136in}}%
\pgfpathcurveto{\pgfqpoint{10.266252in}{4.301322in}}{\pgfqpoint{10.276851in}{4.296932in}}{\pgfqpoint{10.287902in}{4.296932in}}%
\pgfpathlineto{\pgfqpoint{10.287902in}{4.296932in}}%
\pgfpathclose%
\pgfusepath{stroke}%
\end{pgfscope}%
\begin{pgfscope}%
\pgfpathrectangle{\pgfqpoint{7.394209in}{0.375000in}}{\pgfqpoint{6.356833in}{5.175000in}}%
\pgfusepath{clip}%
\pgfsetbuttcap%
\pgfsetroundjoin%
\pgfsetlinewidth{1.003750pt}%
\definecolor{currentstroke}{rgb}{1.000000,0.000000,0.000000}%
\pgfsetstrokecolor{currentstroke}%
\pgfsetdash{}{0pt}%
\pgfpathmoveto{\pgfqpoint{8.475694in}{1.680189in}}%
\pgfpathcurveto{\pgfqpoint{8.486745in}{1.680189in}}{\pgfqpoint{8.497344in}{1.684579in}}{\pgfqpoint{8.505157in}{1.692393in}}%
\pgfpathcurveto{\pgfqpoint{8.512971in}{1.700207in}}{\pgfqpoint{8.517361in}{1.710806in}}{\pgfqpoint{8.517361in}{1.721856in}}%
\pgfpathcurveto{\pgfqpoint{8.517361in}{1.732906in}}{\pgfqpoint{8.512971in}{1.743505in}}{\pgfqpoint{8.505157in}{1.751319in}}%
\pgfpathcurveto{\pgfqpoint{8.497344in}{1.759132in}}{\pgfqpoint{8.486745in}{1.763522in}}{\pgfqpoint{8.475694in}{1.763522in}}%
\pgfpathcurveto{\pgfqpoint{8.464644in}{1.763522in}}{\pgfqpoint{8.454045in}{1.759132in}}{\pgfqpoint{8.446232in}{1.751319in}}%
\pgfpathcurveto{\pgfqpoint{8.438418in}{1.743505in}}{\pgfqpoint{8.434028in}{1.732906in}}{\pgfqpoint{8.434028in}{1.721856in}}%
\pgfpathcurveto{\pgfqpoint{8.434028in}{1.710806in}}{\pgfqpoint{8.438418in}{1.700207in}}{\pgfqpoint{8.446232in}{1.692393in}}%
\pgfpathcurveto{\pgfqpoint{8.454045in}{1.684579in}}{\pgfqpoint{8.464644in}{1.680189in}}{\pgfqpoint{8.475694in}{1.680189in}}%
\pgfpathlineto{\pgfqpoint{8.475694in}{1.680189in}}%
\pgfpathclose%
\pgfusepath{stroke}%
\end{pgfscope}%
\begin{pgfscope}%
\pgfpathrectangle{\pgfqpoint{7.394209in}{0.375000in}}{\pgfqpoint{6.356833in}{5.175000in}}%
\pgfusepath{clip}%
\pgfsetbuttcap%
\pgfsetroundjoin%
\pgfsetlinewidth{1.003750pt}%
\definecolor{currentstroke}{rgb}{1.000000,0.000000,0.000000}%
\pgfsetstrokecolor{currentstroke}%
\pgfsetdash{}{0pt}%
\pgfpathmoveto{\pgfqpoint{9.595606in}{2.804348in}}%
\pgfpathcurveto{\pgfqpoint{9.606656in}{2.804348in}}{\pgfqpoint{9.617255in}{2.808738in}}{\pgfqpoint{9.625069in}{2.816552in}}%
\pgfpathcurveto{\pgfqpoint{9.632883in}{2.824366in}}{\pgfqpoint{9.637273in}{2.834965in}}{\pgfqpoint{9.637273in}{2.846015in}}%
\pgfpathcurveto{\pgfqpoint{9.637273in}{2.857065in}}{\pgfqpoint{9.632883in}{2.867664in}}{\pgfqpoint{9.625069in}{2.875477in}}%
\pgfpathcurveto{\pgfqpoint{9.617255in}{2.883291in}}{\pgfqpoint{9.606656in}{2.887681in}}{\pgfqpoint{9.595606in}{2.887681in}}%
\pgfpathcurveto{\pgfqpoint{9.584556in}{2.887681in}}{\pgfqpoint{9.573957in}{2.883291in}}{\pgfqpoint{9.566144in}{2.875477in}}%
\pgfpathcurveto{\pgfqpoint{9.558330in}{2.867664in}}{\pgfqpoint{9.553940in}{2.857065in}}{\pgfqpoint{9.553940in}{2.846015in}}%
\pgfpathcurveto{\pgfqpoint{9.553940in}{2.834965in}}{\pgfqpoint{9.558330in}{2.824366in}}{\pgfqpoint{9.566144in}{2.816552in}}%
\pgfpathcurveto{\pgfqpoint{9.573957in}{2.808738in}}{\pgfqpoint{9.584556in}{2.804348in}}{\pgfqpoint{9.595606in}{2.804348in}}%
\pgfpathlineto{\pgfqpoint{9.595606in}{2.804348in}}%
\pgfpathclose%
\pgfusepath{stroke}%
\end{pgfscope}%
\begin{pgfscope}%
\pgfpathrectangle{\pgfqpoint{7.394209in}{0.375000in}}{\pgfqpoint{6.356833in}{5.175000in}}%
\pgfusepath{clip}%
\pgfsetbuttcap%
\pgfsetroundjoin%
\pgfsetlinewidth{1.003750pt}%
\definecolor{currentstroke}{rgb}{1.000000,0.000000,0.000000}%
\pgfsetstrokecolor{currentstroke}%
\pgfsetdash{}{0pt}%
\pgfpathmoveto{\pgfqpoint{12.226610in}{5.508333in}}%
\pgfpathcurveto{\pgfqpoint{12.237660in}{5.508333in}}{\pgfqpoint{12.248259in}{5.512723in}}{\pgfqpoint{12.256072in}{5.520536in}}%
\pgfpathcurveto{\pgfqpoint{12.263886in}{5.528350in}}{\pgfqpoint{12.268276in}{5.538949in}}{\pgfqpoint{12.268276in}{5.549999in}}%
\pgfpathcurveto{\pgfqpoint{12.268276in}{5.561049in}}{\pgfqpoint{12.263886in}{5.571648in}}{\pgfqpoint{12.256072in}{5.579462in}}%
\pgfpathcurveto{\pgfqpoint{12.248259in}{5.587276in}}{\pgfqpoint{12.237660in}{5.591666in}}{\pgfqpoint{12.226610in}{5.591666in}}%
\pgfpathcurveto{\pgfqpoint{12.215560in}{5.591666in}}{\pgfqpoint{12.204961in}{5.587276in}}{\pgfqpoint{12.197147in}{5.579462in}}%
\pgfpathcurveto{\pgfqpoint{12.189333in}{5.571648in}}{\pgfqpoint{12.184943in}{5.561049in}}{\pgfqpoint{12.184943in}{5.549999in}}%
\pgfpathcurveto{\pgfqpoint{12.184943in}{5.538949in}}{\pgfqpoint{12.189333in}{5.528350in}}{\pgfqpoint{12.197147in}{5.520536in}}%
\pgfpathcurveto{\pgfqpoint{12.204961in}{5.512723in}}{\pgfqpoint{12.215560in}{5.508333in}}{\pgfqpoint{12.226610in}{5.508333in}}%
\pgfpathlineto{\pgfqpoint{12.226610in}{5.508333in}}%
\pgfpathclose%
\pgfusepath{stroke}%
\end{pgfscope}%
\begin{pgfscope}%
\pgfpathrectangle{\pgfqpoint{7.394209in}{0.375000in}}{\pgfqpoint{6.356833in}{5.175000in}}%
\pgfusepath{clip}%
\pgfsetbuttcap%
\pgfsetroundjoin%
\pgfsetlinewidth{1.003750pt}%
\definecolor{currentstroke}{rgb}{1.000000,0.000000,0.000000}%
\pgfsetstrokecolor{currentstroke}%
\pgfsetdash{}{0pt}%
\pgfpathmoveto{\pgfqpoint{11.035697in}{5.507121in}}%
\pgfpathcurveto{\pgfqpoint{11.046747in}{5.507121in}}{\pgfqpoint{11.057346in}{5.511512in}}{\pgfqpoint{11.065160in}{5.519325in}}%
\pgfpathcurveto{\pgfqpoint{11.072973in}{5.527139in}}{\pgfqpoint{11.077364in}{5.537738in}}{\pgfqpoint{11.077364in}{5.548788in}}%
\pgfpathcurveto{\pgfqpoint{11.077364in}{5.559838in}}{\pgfqpoint{11.072973in}{5.570437in}}{\pgfqpoint{11.065160in}{5.578251in}}%
\pgfpathcurveto{\pgfqpoint{11.057346in}{5.586065in}}{\pgfqpoint{11.046747in}{5.590455in}}{\pgfqpoint{11.035697in}{5.590455in}}%
\pgfpathcurveto{\pgfqpoint{11.024647in}{5.590455in}}{\pgfqpoint{11.014048in}{5.586065in}}{\pgfqpoint{11.006234in}{5.578251in}}%
\pgfpathcurveto{\pgfqpoint{10.998420in}{5.570437in}}{\pgfqpoint{10.994030in}{5.559838in}}{\pgfqpoint{10.994030in}{5.548788in}}%
\pgfpathcurveto{\pgfqpoint{10.994030in}{5.537738in}}{\pgfqpoint{10.998420in}{5.527139in}}{\pgfqpoint{11.006234in}{5.519325in}}%
\pgfpathcurveto{\pgfqpoint{11.014048in}{5.511512in}}{\pgfqpoint{11.024647in}{5.507121in}}{\pgfqpoint{11.035697in}{5.507121in}}%
\pgfpathlineto{\pgfqpoint{11.035697in}{5.507121in}}%
\pgfpathclose%
\pgfusepath{stroke}%
\end{pgfscope}%
\begin{pgfscope}%
\pgfpathrectangle{\pgfqpoint{7.394209in}{0.375000in}}{\pgfqpoint{6.356833in}{5.175000in}}%
\pgfusepath{clip}%
\pgfsetbuttcap%
\pgfsetroundjoin%
\pgfsetlinewidth{1.003750pt}%
\definecolor{currentstroke}{rgb}{1.000000,0.000000,0.000000}%
\pgfsetstrokecolor{currentstroke}%
\pgfsetdash{}{0pt}%
\pgfpathmoveto{\pgfqpoint{11.793624in}{5.508330in}}%
\pgfpathcurveto{\pgfqpoint{11.804674in}{5.508330in}}{\pgfqpoint{11.815273in}{5.512720in}}{\pgfqpoint{11.823087in}{5.520534in}}%
\pgfpathcurveto{\pgfqpoint{11.830900in}{5.528347in}}{\pgfqpoint{11.835291in}{5.538946in}}{\pgfqpoint{11.835291in}{5.549997in}}%
\pgfpathcurveto{\pgfqpoint{11.835291in}{5.561047in}}{\pgfqpoint{11.830900in}{5.571646in}}{\pgfqpoint{11.823087in}{5.579459in}}%
\pgfpathcurveto{\pgfqpoint{11.815273in}{5.587273in}}{\pgfqpoint{11.804674in}{5.591663in}}{\pgfqpoint{11.793624in}{5.591663in}}%
\pgfpathcurveto{\pgfqpoint{11.782574in}{5.591663in}}{\pgfqpoint{11.771975in}{5.587273in}}{\pgfqpoint{11.764161in}{5.579459in}}%
\pgfpathcurveto{\pgfqpoint{11.756348in}{5.571646in}}{\pgfqpoint{11.751957in}{5.561047in}}{\pgfqpoint{11.751957in}{5.549997in}}%
\pgfpathcurveto{\pgfqpoint{11.751957in}{5.538946in}}{\pgfqpoint{11.756348in}{5.528347in}}{\pgfqpoint{11.764161in}{5.520534in}}%
\pgfpathcurveto{\pgfqpoint{11.771975in}{5.512720in}}{\pgfqpoint{11.782574in}{5.508330in}}{\pgfqpoint{11.793624in}{5.508330in}}%
\pgfpathlineto{\pgfqpoint{11.793624in}{5.508330in}}%
\pgfpathclose%
\pgfusepath{stroke}%
\end{pgfscope}%
\begin{pgfscope}%
\pgfpathrectangle{\pgfqpoint{7.394209in}{0.375000in}}{\pgfqpoint{6.356833in}{5.175000in}}%
\pgfusepath{clip}%
\pgfsetbuttcap%
\pgfsetroundjoin%
\pgfsetlinewidth{1.003750pt}%
\definecolor{currentstroke}{rgb}{1.000000,0.000000,0.000000}%
\pgfsetstrokecolor{currentstroke}%
\pgfsetdash{}{0pt}%
\pgfpathmoveto{\pgfqpoint{11.190685in}{5.490657in}}%
\pgfpathcurveto{\pgfqpoint{11.201735in}{5.490657in}}{\pgfqpoint{11.212334in}{5.495047in}}{\pgfqpoint{11.220148in}{5.502861in}}%
\pgfpathcurveto{\pgfqpoint{11.227961in}{5.510675in}}{\pgfqpoint{11.232352in}{5.521274in}}{\pgfqpoint{11.232352in}{5.532324in}}%
\pgfpathcurveto{\pgfqpoint{11.232352in}{5.543374in}}{\pgfqpoint{11.227961in}{5.553973in}}{\pgfqpoint{11.220148in}{5.561787in}}%
\pgfpathcurveto{\pgfqpoint{11.212334in}{5.569600in}}{\pgfqpoint{11.201735in}{5.573990in}}{\pgfqpoint{11.190685in}{5.573990in}}%
\pgfpathcurveto{\pgfqpoint{11.179635in}{5.573990in}}{\pgfqpoint{11.169036in}{5.569600in}}{\pgfqpoint{11.161222in}{5.561787in}}%
\pgfpathcurveto{\pgfqpoint{11.153409in}{5.553973in}}{\pgfqpoint{11.149018in}{5.543374in}}{\pgfqpoint{11.149018in}{5.532324in}}%
\pgfpathcurveto{\pgfqpoint{11.149018in}{5.521274in}}{\pgfqpoint{11.153409in}{5.510675in}}{\pgfqpoint{11.161222in}{5.502861in}}%
\pgfpathcurveto{\pgfqpoint{11.169036in}{5.495047in}}{\pgfqpoint{11.179635in}{5.490657in}}{\pgfqpoint{11.190685in}{5.490657in}}%
\pgfpathlineto{\pgfqpoint{11.190685in}{5.490657in}}%
\pgfpathclose%
\pgfusepath{stroke}%
\end{pgfscope}%
\begin{pgfscope}%
\pgfpathrectangle{\pgfqpoint{7.394209in}{0.375000in}}{\pgfqpoint{6.356833in}{5.175000in}}%
\pgfusepath{clip}%
\pgfsetbuttcap%
\pgfsetroundjoin%
\pgfsetlinewidth{1.003750pt}%
\definecolor{currentstroke}{rgb}{1.000000,0.000000,0.000000}%
\pgfsetstrokecolor{currentstroke}%
\pgfsetdash{}{0pt}%
\pgfpathmoveto{\pgfqpoint{10.183435in}{3.517213in}}%
\pgfpathcurveto{\pgfqpoint{10.194485in}{3.517213in}}{\pgfqpoint{10.205084in}{3.521603in}}{\pgfqpoint{10.212898in}{3.529417in}}%
\pgfpathcurveto{\pgfqpoint{10.220711in}{3.537230in}}{\pgfqpoint{10.225102in}{3.547829in}}{\pgfqpoint{10.225102in}{3.558879in}}%
\pgfpathcurveto{\pgfqpoint{10.225102in}{3.569929in}}{\pgfqpoint{10.220711in}{3.580528in}}{\pgfqpoint{10.212898in}{3.588342in}}%
\pgfpathcurveto{\pgfqpoint{10.205084in}{3.596156in}}{\pgfqpoint{10.194485in}{3.600546in}}{\pgfqpoint{10.183435in}{3.600546in}}%
\pgfpathcurveto{\pgfqpoint{10.172385in}{3.600546in}}{\pgfqpoint{10.161786in}{3.596156in}}{\pgfqpoint{10.153972in}{3.588342in}}%
\pgfpathcurveto{\pgfqpoint{10.146159in}{3.580528in}}{\pgfqpoint{10.141768in}{3.569929in}}{\pgfqpoint{10.141768in}{3.558879in}}%
\pgfpathcurveto{\pgfqpoint{10.141768in}{3.547829in}}{\pgfqpoint{10.146159in}{3.537230in}}{\pgfqpoint{10.153972in}{3.529417in}}%
\pgfpathcurveto{\pgfqpoint{10.161786in}{3.521603in}}{\pgfqpoint{10.172385in}{3.517213in}}{\pgfqpoint{10.183435in}{3.517213in}}%
\pgfpathlineto{\pgfqpoint{10.183435in}{3.517213in}}%
\pgfpathclose%
\pgfusepath{stroke}%
\end{pgfscope}%
\begin{pgfscope}%
\pgfpathrectangle{\pgfqpoint{7.394209in}{0.375000in}}{\pgfqpoint{6.356833in}{5.175000in}}%
\pgfusepath{clip}%
\pgfsetbuttcap%
\pgfsetroundjoin%
\pgfsetlinewidth{1.003750pt}%
\definecolor{currentstroke}{rgb}{1.000000,0.000000,0.000000}%
\pgfsetstrokecolor{currentstroke}%
\pgfsetdash{}{0pt}%
\pgfpathmoveto{\pgfqpoint{10.073328in}{3.854908in}}%
\pgfpathcurveto{\pgfqpoint{10.084378in}{3.854908in}}{\pgfqpoint{10.094977in}{3.859298in}}{\pgfqpoint{10.102791in}{3.867112in}}%
\pgfpathcurveto{\pgfqpoint{10.110604in}{3.874925in}}{\pgfqpoint{10.114995in}{3.885524in}}{\pgfqpoint{10.114995in}{3.896574in}}%
\pgfpathcurveto{\pgfqpoint{10.114995in}{3.907625in}}{\pgfqpoint{10.110604in}{3.918224in}}{\pgfqpoint{10.102791in}{3.926037in}}%
\pgfpathcurveto{\pgfqpoint{10.094977in}{3.933851in}}{\pgfqpoint{10.084378in}{3.938241in}}{\pgfqpoint{10.073328in}{3.938241in}}%
\pgfpathcurveto{\pgfqpoint{10.062278in}{3.938241in}}{\pgfqpoint{10.051679in}{3.933851in}}{\pgfqpoint{10.043865in}{3.926037in}}%
\pgfpathcurveto{\pgfqpoint{10.036052in}{3.918224in}}{\pgfqpoint{10.031661in}{3.907625in}}{\pgfqpoint{10.031661in}{3.896574in}}%
\pgfpathcurveto{\pgfqpoint{10.031661in}{3.885524in}}{\pgfqpoint{10.036052in}{3.874925in}}{\pgfqpoint{10.043865in}{3.867112in}}%
\pgfpathcurveto{\pgfqpoint{10.051679in}{3.859298in}}{\pgfqpoint{10.062278in}{3.854908in}}{\pgfqpoint{10.073328in}{3.854908in}}%
\pgfpathlineto{\pgfqpoint{10.073328in}{3.854908in}}%
\pgfpathclose%
\pgfusepath{stroke}%
\end{pgfscope}%
\begin{pgfscope}%
\pgfpathrectangle{\pgfqpoint{7.394209in}{0.375000in}}{\pgfqpoint{6.356833in}{5.175000in}}%
\pgfusepath{clip}%
\pgfsetbuttcap%
\pgfsetroundjoin%
\pgfsetlinewidth{1.003750pt}%
\definecolor{currentstroke}{rgb}{1.000000,0.000000,0.000000}%
\pgfsetstrokecolor{currentstroke}%
\pgfsetdash{}{0pt}%
\pgfpathmoveto{\pgfqpoint{13.448595in}{5.507766in}}%
\pgfpathcurveto{\pgfqpoint{13.459645in}{5.507766in}}{\pgfqpoint{13.470244in}{5.512156in}}{\pgfqpoint{13.478058in}{5.519970in}}%
\pgfpathcurveto{\pgfqpoint{13.485872in}{5.527783in}}{\pgfqpoint{13.490262in}{5.538382in}}{\pgfqpoint{13.490262in}{5.549432in}}%
\pgfpathcurveto{\pgfqpoint{13.490262in}{5.560483in}}{\pgfqpoint{13.485872in}{5.571082in}}{\pgfqpoint{13.478058in}{5.578895in}}%
\pgfpathcurveto{\pgfqpoint{13.470244in}{5.586709in}}{\pgfqpoint{13.459645in}{5.591099in}}{\pgfqpoint{13.448595in}{5.591099in}}%
\pgfpathcurveto{\pgfqpoint{13.437545in}{5.591099in}}{\pgfqpoint{13.426946in}{5.586709in}}{\pgfqpoint{13.419132in}{5.578895in}}%
\pgfpathcurveto{\pgfqpoint{13.411319in}{5.571082in}}{\pgfqpoint{13.406929in}{5.560483in}}{\pgfqpoint{13.406929in}{5.549432in}}%
\pgfpathcurveto{\pgfqpoint{13.406929in}{5.538382in}}{\pgfqpoint{13.411319in}{5.527783in}}{\pgfqpoint{13.419132in}{5.519970in}}%
\pgfpathcurveto{\pgfqpoint{13.426946in}{5.512156in}}{\pgfqpoint{13.437545in}{5.507766in}}{\pgfqpoint{13.448595in}{5.507766in}}%
\pgfpathlineto{\pgfqpoint{13.448595in}{5.507766in}}%
\pgfpathclose%
\pgfusepath{stroke}%
\end{pgfscope}%
\begin{pgfscope}%
\pgfpathrectangle{\pgfqpoint{7.394209in}{0.375000in}}{\pgfqpoint{6.356833in}{5.175000in}}%
\pgfusepath{clip}%
\pgfsetbuttcap%
\pgfsetroundjoin%
\pgfsetlinewidth{1.003750pt}%
\definecolor{currentstroke}{rgb}{1.000000,0.000000,0.000000}%
\pgfsetstrokecolor{currentstroke}%
\pgfsetdash{}{0pt}%
\pgfpathmoveto{\pgfqpoint{11.941168in}{5.490657in}}%
\pgfpathcurveto{\pgfqpoint{11.952218in}{5.490657in}}{\pgfqpoint{11.962817in}{5.495047in}}{\pgfqpoint{11.970631in}{5.502861in}}%
\pgfpathcurveto{\pgfqpoint{11.978444in}{5.510675in}}{\pgfqpoint{11.982835in}{5.521274in}}{\pgfqpoint{11.982835in}{5.532324in}}%
\pgfpathcurveto{\pgfqpoint{11.982835in}{5.543374in}}{\pgfqpoint{11.978444in}{5.553973in}}{\pgfqpoint{11.970631in}{5.561787in}}%
\pgfpathcurveto{\pgfqpoint{11.962817in}{5.569600in}}{\pgfqpoint{11.952218in}{5.573990in}}{\pgfqpoint{11.941168in}{5.573990in}}%
\pgfpathcurveto{\pgfqpoint{11.930118in}{5.573990in}}{\pgfqpoint{11.919519in}{5.569600in}}{\pgfqpoint{11.911705in}{5.561787in}}%
\pgfpathcurveto{\pgfqpoint{11.903892in}{5.553973in}}{\pgfqpoint{11.899501in}{5.543374in}}{\pgfqpoint{11.899501in}{5.532324in}}%
\pgfpathcurveto{\pgfqpoint{11.899501in}{5.521274in}}{\pgfqpoint{11.903892in}{5.510675in}}{\pgfqpoint{11.911705in}{5.502861in}}%
\pgfpathcurveto{\pgfqpoint{11.919519in}{5.495047in}}{\pgfqpoint{11.930118in}{5.490657in}}{\pgfqpoint{11.941168in}{5.490657in}}%
\pgfpathlineto{\pgfqpoint{11.941168in}{5.490657in}}%
\pgfpathclose%
\pgfusepath{stroke}%
\end{pgfscope}%
\begin{pgfscope}%
\pgfpathrectangle{\pgfqpoint{7.394209in}{0.375000in}}{\pgfqpoint{6.356833in}{5.175000in}}%
\pgfusepath{clip}%
\pgfsetbuttcap%
\pgfsetroundjoin%
\pgfsetlinewidth{1.003750pt}%
\definecolor{currentstroke}{rgb}{1.000000,0.000000,0.000000}%
\pgfsetstrokecolor{currentstroke}%
\pgfsetdash{}{0pt}%
\pgfpathmoveto{\pgfqpoint{12.686101in}{5.484797in}}%
\pgfpathcurveto{\pgfqpoint{12.697151in}{5.484797in}}{\pgfqpoint{12.707750in}{5.489187in}}{\pgfqpoint{12.715564in}{5.497001in}}%
\pgfpathcurveto{\pgfqpoint{12.723377in}{5.504814in}}{\pgfqpoint{12.727768in}{5.515413in}}{\pgfqpoint{12.727768in}{5.526463in}}%
\pgfpathcurveto{\pgfqpoint{12.727768in}{5.537513in}}{\pgfqpoint{12.723377in}{5.548113in}}{\pgfqpoint{12.715564in}{5.555926in}}%
\pgfpathcurveto{\pgfqpoint{12.707750in}{5.563740in}}{\pgfqpoint{12.697151in}{5.568130in}}{\pgfqpoint{12.686101in}{5.568130in}}%
\pgfpathcurveto{\pgfqpoint{12.675051in}{5.568130in}}{\pgfqpoint{12.664452in}{5.563740in}}{\pgfqpoint{12.656638in}{5.555926in}}%
\pgfpathcurveto{\pgfqpoint{12.648825in}{5.548113in}}{\pgfqpoint{12.644434in}{5.537513in}}{\pgfqpoint{12.644434in}{5.526463in}}%
\pgfpathcurveto{\pgfqpoint{12.644434in}{5.515413in}}{\pgfqpoint{12.648825in}{5.504814in}}{\pgfqpoint{12.656638in}{5.497001in}}%
\pgfpathcurveto{\pgfqpoint{12.664452in}{5.489187in}}{\pgfqpoint{12.675051in}{5.484797in}}{\pgfqpoint{12.686101in}{5.484797in}}%
\pgfpathlineto{\pgfqpoint{12.686101in}{5.484797in}}%
\pgfpathclose%
\pgfusepath{stroke}%
\end{pgfscope}%
\begin{pgfscope}%
\pgfpathrectangle{\pgfqpoint{7.394209in}{0.375000in}}{\pgfqpoint{6.356833in}{5.175000in}}%
\pgfusepath{clip}%
\pgfsetbuttcap%
\pgfsetroundjoin%
\pgfsetlinewidth{1.003750pt}%
\definecolor{currentstroke}{rgb}{1.000000,0.000000,0.000000}%
\pgfsetstrokecolor{currentstroke}%
\pgfsetdash{}{0pt}%
\pgfpathmoveto{\pgfqpoint{8.619591in}{1.754582in}}%
\pgfpathcurveto{\pgfqpoint{8.630642in}{1.754582in}}{\pgfqpoint{8.641241in}{1.758972in}}{\pgfqpoint{8.649054in}{1.766785in}}%
\pgfpathcurveto{\pgfqpoint{8.656868in}{1.774599in}}{\pgfqpoint{8.661258in}{1.785198in}}{\pgfqpoint{8.661258in}{1.796248in}}%
\pgfpathcurveto{\pgfqpoint{8.661258in}{1.807298in}}{\pgfqpoint{8.656868in}{1.817897in}}{\pgfqpoint{8.649054in}{1.825711in}}%
\pgfpathcurveto{\pgfqpoint{8.641241in}{1.833525in}}{\pgfqpoint{8.630642in}{1.837915in}}{\pgfqpoint{8.619591in}{1.837915in}}%
\pgfpathcurveto{\pgfqpoint{8.608541in}{1.837915in}}{\pgfqpoint{8.597942in}{1.833525in}}{\pgfqpoint{8.590129in}{1.825711in}}%
\pgfpathcurveto{\pgfqpoint{8.582315in}{1.817897in}}{\pgfqpoint{8.577925in}{1.807298in}}{\pgfqpoint{8.577925in}{1.796248in}}%
\pgfpathcurveto{\pgfqpoint{8.577925in}{1.785198in}}{\pgfqpoint{8.582315in}{1.774599in}}{\pgfqpoint{8.590129in}{1.766785in}}%
\pgfpathcurveto{\pgfqpoint{8.597942in}{1.758972in}}{\pgfqpoint{8.608541in}{1.754582in}}{\pgfqpoint{8.619591in}{1.754582in}}%
\pgfpathlineto{\pgfqpoint{8.619591in}{1.754582in}}%
\pgfpathclose%
\pgfusepath{stroke}%
\end{pgfscope}%
\begin{pgfscope}%
\pgfpathrectangle{\pgfqpoint{7.394209in}{0.375000in}}{\pgfqpoint{6.356833in}{5.175000in}}%
\pgfusepath{clip}%
\pgfsetbuttcap%
\pgfsetroundjoin%
\pgfsetlinewidth{1.003750pt}%
\definecolor{currentstroke}{rgb}{1.000000,0.000000,0.000000}%
\pgfsetstrokecolor{currentstroke}%
\pgfsetdash{}{0pt}%
\pgfpathmoveto{\pgfqpoint{9.220234in}{2.777172in}}%
\pgfpathcurveto{\pgfqpoint{9.231284in}{2.777172in}}{\pgfqpoint{9.241883in}{2.781562in}}{\pgfqpoint{9.249696in}{2.789376in}}%
\pgfpathcurveto{\pgfqpoint{9.257510in}{2.797189in}}{\pgfqpoint{9.261900in}{2.807788in}}{\pgfqpoint{9.261900in}{2.818839in}}%
\pgfpathcurveto{\pgfqpoint{9.261900in}{2.829889in}}{\pgfqpoint{9.257510in}{2.840488in}}{\pgfqpoint{9.249696in}{2.848301in}}%
\pgfpathcurveto{\pgfqpoint{9.241883in}{2.856115in}}{\pgfqpoint{9.231284in}{2.860505in}}{\pgfqpoint{9.220234in}{2.860505in}}%
\pgfpathcurveto{\pgfqpoint{9.209183in}{2.860505in}}{\pgfqpoint{9.198584in}{2.856115in}}{\pgfqpoint{9.190771in}{2.848301in}}%
\pgfpathcurveto{\pgfqpoint{9.182957in}{2.840488in}}{\pgfqpoint{9.178567in}{2.829889in}}{\pgfqpoint{9.178567in}{2.818839in}}%
\pgfpathcurveto{\pgfqpoint{9.178567in}{2.807788in}}{\pgfqpoint{9.182957in}{2.797189in}}{\pgfqpoint{9.190771in}{2.789376in}}%
\pgfpathcurveto{\pgfqpoint{9.198584in}{2.781562in}}{\pgfqpoint{9.209183in}{2.777172in}}{\pgfqpoint{9.220234in}{2.777172in}}%
\pgfpathlineto{\pgfqpoint{9.220234in}{2.777172in}}%
\pgfpathclose%
\pgfusepath{stroke}%
\end{pgfscope}%
\begin{pgfscope}%
\pgfpathrectangle{\pgfqpoint{7.394209in}{0.375000in}}{\pgfqpoint{6.356833in}{5.175000in}}%
\pgfusepath{clip}%
\pgfsetbuttcap%
\pgfsetroundjoin%
\pgfsetlinewidth{1.003750pt}%
\definecolor{currentstroke}{rgb}{1.000000,0.000000,0.000000}%
\pgfsetstrokecolor{currentstroke}%
\pgfsetdash{}{0pt}%
\pgfpathmoveto{\pgfqpoint{9.601185in}{3.282844in}}%
\pgfpathcurveto{\pgfqpoint{9.612235in}{3.282844in}}{\pgfqpoint{9.622835in}{3.287234in}}{\pgfqpoint{9.630648in}{3.295048in}}%
\pgfpathcurveto{\pgfqpoint{9.638462in}{3.302862in}}{\pgfqpoint{9.642852in}{3.313461in}}{\pgfqpoint{9.642852in}{3.324511in}}%
\pgfpathcurveto{\pgfqpoint{9.642852in}{3.335561in}}{\pgfqpoint{9.638462in}{3.346160in}}{\pgfqpoint{9.630648in}{3.353974in}}%
\pgfpathcurveto{\pgfqpoint{9.622835in}{3.361787in}}{\pgfqpoint{9.612235in}{3.366177in}}{\pgfqpoint{9.601185in}{3.366177in}}%
\pgfpathcurveto{\pgfqpoint{9.590135in}{3.366177in}}{\pgfqpoint{9.579536in}{3.361787in}}{\pgfqpoint{9.571723in}{3.353974in}}%
\pgfpathcurveto{\pgfqpoint{9.563909in}{3.346160in}}{\pgfqpoint{9.559519in}{3.335561in}}{\pgfqpoint{9.559519in}{3.324511in}}%
\pgfpathcurveto{\pgfqpoint{9.559519in}{3.313461in}}{\pgfqpoint{9.563909in}{3.302862in}}{\pgfqpoint{9.571723in}{3.295048in}}%
\pgfpathcurveto{\pgfqpoint{9.579536in}{3.287234in}}{\pgfqpoint{9.590135in}{3.282844in}}{\pgfqpoint{9.601185in}{3.282844in}}%
\pgfpathlineto{\pgfqpoint{9.601185in}{3.282844in}}%
\pgfpathclose%
\pgfusepath{stroke}%
\end{pgfscope}%
\begin{pgfscope}%
\pgfpathrectangle{\pgfqpoint{7.394209in}{0.375000in}}{\pgfqpoint{6.356833in}{5.175000in}}%
\pgfusepath{clip}%
\pgfsetbuttcap%
\pgfsetroundjoin%
\pgfsetlinewidth{1.003750pt}%
\definecolor{currentstroke}{rgb}{1.000000,0.000000,0.000000}%
\pgfsetstrokecolor{currentstroke}%
\pgfsetdash{}{0pt}%
\pgfpathmoveto{\pgfqpoint{8.403664in}{1.577470in}}%
\pgfpathcurveto{\pgfqpoint{8.414714in}{1.577470in}}{\pgfqpoint{8.425313in}{1.581860in}}{\pgfqpoint{8.433127in}{1.589674in}}%
\pgfpathcurveto{\pgfqpoint{8.440941in}{1.597488in}}{\pgfqpoint{8.445331in}{1.608087in}}{\pgfqpoint{8.445331in}{1.619137in}}%
\pgfpathcurveto{\pgfqpoint{8.445331in}{1.630187in}}{\pgfqpoint{8.440941in}{1.640786in}}{\pgfqpoint{8.433127in}{1.648600in}}%
\pgfpathcurveto{\pgfqpoint{8.425313in}{1.656413in}}{\pgfqpoint{8.414714in}{1.660804in}}{\pgfqpoint{8.403664in}{1.660804in}}%
\pgfpathcurveto{\pgfqpoint{8.392614in}{1.660804in}}{\pgfqpoint{8.382015in}{1.656413in}}{\pgfqpoint{8.374201in}{1.648600in}}%
\pgfpathcurveto{\pgfqpoint{8.366388in}{1.640786in}}{\pgfqpoint{8.361998in}{1.630187in}}{\pgfqpoint{8.361998in}{1.619137in}}%
\pgfpathcurveto{\pgfqpoint{8.361998in}{1.608087in}}{\pgfqpoint{8.366388in}{1.597488in}}{\pgfqpoint{8.374201in}{1.589674in}}%
\pgfpathcurveto{\pgfqpoint{8.382015in}{1.581860in}}{\pgfqpoint{8.392614in}{1.577470in}}{\pgfqpoint{8.403664in}{1.577470in}}%
\pgfpathlineto{\pgfqpoint{8.403664in}{1.577470in}}%
\pgfpathclose%
\pgfusepath{stroke}%
\end{pgfscope}%
\begin{pgfscope}%
\pgfpathrectangle{\pgfqpoint{7.394209in}{0.375000in}}{\pgfqpoint{6.356833in}{5.175000in}}%
\pgfusepath{clip}%
\pgfsetbuttcap%
\pgfsetroundjoin%
\pgfsetlinewidth{1.003750pt}%
\definecolor{currentstroke}{rgb}{1.000000,0.000000,0.000000}%
\pgfsetstrokecolor{currentstroke}%
\pgfsetdash{}{0pt}%
\pgfpathmoveto{\pgfqpoint{11.228862in}{5.460287in}}%
\pgfpathcurveto{\pgfqpoint{11.239912in}{5.460287in}}{\pgfqpoint{11.250511in}{5.464677in}}{\pgfqpoint{11.258325in}{5.472491in}}%
\pgfpathcurveto{\pgfqpoint{11.266138in}{5.480305in}}{\pgfqpoint{11.270529in}{5.490904in}}{\pgfqpoint{11.270529in}{5.501954in}}%
\pgfpathcurveto{\pgfqpoint{11.270529in}{5.513004in}}{\pgfqpoint{11.266138in}{5.523603in}}{\pgfqpoint{11.258325in}{5.531417in}}%
\pgfpathcurveto{\pgfqpoint{11.250511in}{5.539230in}}{\pgfqpoint{11.239912in}{5.543621in}}{\pgfqpoint{11.228862in}{5.543621in}}%
\pgfpathcurveto{\pgfqpoint{11.217812in}{5.543621in}}{\pgfqpoint{11.207213in}{5.539230in}}{\pgfqpoint{11.199399in}{5.531417in}}%
\pgfpathcurveto{\pgfqpoint{11.191585in}{5.523603in}}{\pgfqpoint{11.187195in}{5.513004in}}{\pgfqpoint{11.187195in}{5.501954in}}%
\pgfpathcurveto{\pgfqpoint{11.187195in}{5.490904in}}{\pgfqpoint{11.191585in}{5.480305in}}{\pgfqpoint{11.199399in}{5.472491in}}%
\pgfpathcurveto{\pgfqpoint{11.207213in}{5.464677in}}{\pgfqpoint{11.217812in}{5.460287in}}{\pgfqpoint{11.228862in}{5.460287in}}%
\pgfpathlineto{\pgfqpoint{11.228862in}{5.460287in}}%
\pgfpathclose%
\pgfusepath{stroke}%
\end{pgfscope}%
\begin{pgfscope}%
\pgfpathrectangle{\pgfqpoint{7.394209in}{0.375000in}}{\pgfqpoint{6.356833in}{5.175000in}}%
\pgfusepath{clip}%
\pgfsetbuttcap%
\pgfsetroundjoin%
\pgfsetlinewidth{1.003750pt}%
\definecolor{currentstroke}{rgb}{1.000000,0.000000,0.000000}%
\pgfsetstrokecolor{currentstroke}%
\pgfsetdash{}{0pt}%
\pgfpathmoveto{\pgfqpoint{11.793624in}{5.508330in}}%
\pgfpathcurveto{\pgfqpoint{11.804674in}{5.508330in}}{\pgfqpoint{11.815273in}{5.512720in}}{\pgfqpoint{11.823087in}{5.520534in}}%
\pgfpathcurveto{\pgfqpoint{11.830900in}{5.528347in}}{\pgfqpoint{11.835291in}{5.538947in}}{\pgfqpoint{11.835291in}{5.549997in}}%
\pgfpathcurveto{\pgfqpoint{11.835291in}{5.561047in}}{\pgfqpoint{11.830900in}{5.571646in}}{\pgfqpoint{11.823087in}{5.579459in}}%
\pgfpathcurveto{\pgfqpoint{11.815273in}{5.587273in}}{\pgfqpoint{11.804674in}{5.591663in}}{\pgfqpoint{11.793624in}{5.591663in}}%
\pgfpathcurveto{\pgfqpoint{11.782574in}{5.591663in}}{\pgfqpoint{11.771975in}{5.587273in}}{\pgfqpoint{11.764161in}{5.579459in}}%
\pgfpathcurveto{\pgfqpoint{11.756348in}{5.571646in}}{\pgfqpoint{11.751957in}{5.561047in}}{\pgfqpoint{11.751957in}{5.549997in}}%
\pgfpathcurveto{\pgfqpoint{11.751957in}{5.538947in}}{\pgfqpoint{11.756348in}{5.528347in}}{\pgfqpoint{11.764161in}{5.520534in}}%
\pgfpathcurveto{\pgfqpoint{11.771975in}{5.512720in}}{\pgfqpoint{11.782574in}{5.508330in}}{\pgfqpoint{11.793624in}{5.508330in}}%
\pgfpathlineto{\pgfqpoint{11.793624in}{5.508330in}}%
\pgfpathclose%
\pgfusepath{stroke}%
\end{pgfscope}%
\begin{pgfscope}%
\pgfpathrectangle{\pgfqpoint{7.394209in}{0.375000in}}{\pgfqpoint{6.356833in}{5.175000in}}%
\pgfusepath{clip}%
\pgfsetbuttcap%
\pgfsetroundjoin%
\pgfsetlinewidth{1.003750pt}%
\definecolor{currentstroke}{rgb}{1.000000,0.000000,0.000000}%
\pgfsetstrokecolor{currentstroke}%
\pgfsetdash{}{0pt}%
\pgfpathmoveto{\pgfqpoint{8.619591in}{1.848394in}}%
\pgfpathcurveto{\pgfqpoint{8.630642in}{1.848394in}}{\pgfqpoint{8.641241in}{1.852785in}}{\pgfqpoint{8.649054in}{1.860598in}}%
\pgfpathcurveto{\pgfqpoint{8.656868in}{1.868412in}}{\pgfqpoint{8.661258in}{1.879011in}}{\pgfqpoint{8.661258in}{1.890061in}}%
\pgfpathcurveto{\pgfqpoint{8.661258in}{1.901111in}}{\pgfqpoint{8.656868in}{1.911710in}}{\pgfqpoint{8.649054in}{1.919524in}}%
\pgfpathcurveto{\pgfqpoint{8.641241in}{1.927337in}}{\pgfqpoint{8.630642in}{1.931728in}}{\pgfqpoint{8.619591in}{1.931728in}}%
\pgfpathcurveto{\pgfqpoint{8.608541in}{1.931728in}}{\pgfqpoint{8.597942in}{1.927337in}}{\pgfqpoint{8.590129in}{1.919524in}}%
\pgfpathcurveto{\pgfqpoint{8.582315in}{1.911710in}}{\pgfqpoint{8.577925in}{1.901111in}}{\pgfqpoint{8.577925in}{1.890061in}}%
\pgfpathcurveto{\pgfqpoint{8.577925in}{1.879011in}}{\pgfqpoint{8.582315in}{1.868412in}}{\pgfqpoint{8.590129in}{1.860598in}}%
\pgfpathcurveto{\pgfqpoint{8.597942in}{1.852785in}}{\pgfqpoint{8.608541in}{1.848394in}}{\pgfqpoint{8.619591in}{1.848394in}}%
\pgfpathlineto{\pgfqpoint{8.619591in}{1.848394in}}%
\pgfpathclose%
\pgfusepath{stroke}%
\end{pgfscope}%
\begin{pgfscope}%
\pgfpathrectangle{\pgfqpoint{7.394209in}{0.375000in}}{\pgfqpoint{6.356833in}{5.175000in}}%
\pgfusepath{clip}%
\pgfsetbuttcap%
\pgfsetroundjoin%
\pgfsetlinewidth{1.003750pt}%
\definecolor{currentstroke}{rgb}{1.000000,0.000000,0.000000}%
\pgfsetstrokecolor{currentstroke}%
\pgfsetdash{}{0pt}%
\pgfpathmoveto{\pgfqpoint{7.429198in}{0.333333in}}%
\pgfpathcurveto{\pgfqpoint{7.440248in}{0.333333in}}{\pgfqpoint{7.450847in}{0.337723in}}{\pgfqpoint{7.458661in}{0.345537in}}%
\pgfpathcurveto{\pgfqpoint{7.466474in}{0.353350in}}{\pgfqpoint{7.470865in}{0.363949in}}{\pgfqpoint{7.470865in}{0.375000in}}%
\pgfpathcurveto{\pgfqpoint{7.470865in}{0.386050in}}{\pgfqpoint{7.466474in}{0.396649in}}{\pgfqpoint{7.458661in}{0.404462in}}%
\pgfpathcurveto{\pgfqpoint{7.450847in}{0.412276in}}{\pgfqpoint{7.440248in}{0.416666in}}{\pgfqpoint{7.429198in}{0.416666in}}%
\pgfpathcurveto{\pgfqpoint{7.418148in}{0.416666in}}{\pgfqpoint{7.407549in}{0.412276in}}{\pgfqpoint{7.399735in}{0.404462in}}%
\pgfpathcurveto{\pgfqpoint{7.391921in}{0.396649in}}{\pgfqpoint{7.387531in}{0.386050in}}{\pgfqpoint{7.387531in}{0.375000in}}%
\pgfpathcurveto{\pgfqpoint{7.387531in}{0.363949in}}{\pgfqpoint{7.391921in}{0.353350in}}{\pgfqpoint{7.399735in}{0.345537in}}%
\pgfpathcurveto{\pgfqpoint{7.407549in}{0.337723in}}{\pgfqpoint{7.418148in}{0.333333in}}{\pgfqpoint{7.429198in}{0.333333in}}%
\pgfusepath{stroke}%
\end{pgfscope}%
\begin{pgfscope}%
\pgfpathrectangle{\pgfqpoint{7.394209in}{0.375000in}}{\pgfqpoint{6.356833in}{5.175000in}}%
\pgfusepath{clip}%
\pgfsetbuttcap%
\pgfsetroundjoin%
\pgfsetlinewidth{1.003750pt}%
\definecolor{currentstroke}{rgb}{1.000000,0.000000,0.000000}%
\pgfsetstrokecolor{currentstroke}%
\pgfsetdash{}{0pt}%
\pgfpathmoveto{\pgfqpoint{10.056116in}{3.509422in}}%
\pgfpathcurveto{\pgfqpoint{10.067166in}{3.509422in}}{\pgfqpoint{10.077765in}{3.513812in}}{\pgfqpoint{10.085579in}{3.521626in}}%
\pgfpathcurveto{\pgfqpoint{10.093392in}{3.529439in}}{\pgfqpoint{10.097783in}{3.540038in}}{\pgfqpoint{10.097783in}{3.551088in}}%
\pgfpathcurveto{\pgfqpoint{10.097783in}{3.562139in}}{\pgfqpoint{10.093392in}{3.572738in}}{\pgfqpoint{10.085579in}{3.580551in}}%
\pgfpathcurveto{\pgfqpoint{10.077765in}{3.588365in}}{\pgfqpoint{10.067166in}{3.592755in}}{\pgfqpoint{10.056116in}{3.592755in}}%
\pgfpathcurveto{\pgfqpoint{10.045066in}{3.592755in}}{\pgfqpoint{10.034467in}{3.588365in}}{\pgfqpoint{10.026653in}{3.580551in}}%
\pgfpathcurveto{\pgfqpoint{10.018840in}{3.572738in}}{\pgfqpoint{10.014449in}{3.562139in}}{\pgfqpoint{10.014449in}{3.551088in}}%
\pgfpathcurveto{\pgfqpoint{10.014449in}{3.540038in}}{\pgfqpoint{10.018840in}{3.529439in}}{\pgfqpoint{10.026653in}{3.521626in}}%
\pgfpathcurveto{\pgfqpoint{10.034467in}{3.513812in}}{\pgfqpoint{10.045066in}{3.509422in}}{\pgfqpoint{10.056116in}{3.509422in}}%
\pgfpathlineto{\pgfqpoint{10.056116in}{3.509422in}}%
\pgfpathclose%
\pgfusepath{stroke}%
\end{pgfscope}%
\begin{pgfscope}%
\pgfpathrectangle{\pgfqpoint{7.394209in}{0.375000in}}{\pgfqpoint{6.356833in}{5.175000in}}%
\pgfusepath{clip}%
\pgfsetbuttcap%
\pgfsetroundjoin%
\pgfsetlinewidth{1.003750pt}%
\definecolor{currentstroke}{rgb}{1.000000,0.000000,0.000000}%
\pgfsetstrokecolor{currentstroke}%
\pgfsetdash{}{0pt}%
\pgfpathmoveto{\pgfqpoint{8.134093in}{0.952646in}}%
\pgfpathcurveto{\pgfqpoint{8.145143in}{0.952646in}}{\pgfqpoint{8.155742in}{0.957036in}}{\pgfqpoint{8.163556in}{0.964850in}}%
\pgfpathcurveto{\pgfqpoint{8.171370in}{0.972664in}}{\pgfqpoint{8.175760in}{0.983263in}}{\pgfqpoint{8.175760in}{0.994313in}}%
\pgfpathcurveto{\pgfqpoint{8.175760in}{1.005363in}}{\pgfqpoint{8.171370in}{1.015962in}}{\pgfqpoint{8.163556in}{1.023776in}}%
\pgfpathcurveto{\pgfqpoint{8.155742in}{1.031589in}}{\pgfqpoint{8.145143in}{1.035980in}}{\pgfqpoint{8.134093in}{1.035980in}}%
\pgfpathcurveto{\pgfqpoint{8.123043in}{1.035980in}}{\pgfqpoint{8.112444in}{1.031589in}}{\pgfqpoint{8.104630in}{1.023776in}}%
\pgfpathcurveto{\pgfqpoint{8.096817in}{1.015962in}}{\pgfqpoint{8.092427in}{1.005363in}}{\pgfqpoint{8.092427in}{0.994313in}}%
\pgfpathcurveto{\pgfqpoint{8.092427in}{0.983263in}}{\pgfqpoint{8.096817in}{0.972664in}}{\pgfqpoint{8.104630in}{0.964850in}}%
\pgfpathcurveto{\pgfqpoint{8.112444in}{0.957036in}}{\pgfqpoint{8.123043in}{0.952646in}}{\pgfqpoint{8.134093in}{0.952646in}}%
\pgfpathlineto{\pgfqpoint{8.134093in}{0.952646in}}%
\pgfpathclose%
\pgfusepath{stroke}%
\end{pgfscope}%
\begin{pgfscope}%
\pgfpathrectangle{\pgfqpoint{7.394209in}{0.375000in}}{\pgfqpoint{6.356833in}{5.175000in}}%
\pgfusepath{clip}%
\pgfsetbuttcap%
\pgfsetroundjoin%
\pgfsetlinewidth{1.003750pt}%
\definecolor{currentstroke}{rgb}{1.000000,0.000000,0.000000}%
\pgfsetstrokecolor{currentstroke}%
\pgfsetdash{}{0pt}%
\pgfpathmoveto{\pgfqpoint{9.323730in}{2.928911in}}%
\pgfpathcurveto{\pgfqpoint{9.334781in}{2.928911in}}{\pgfqpoint{9.345380in}{2.933301in}}{\pgfqpoint{9.353193in}{2.941115in}}%
\pgfpathcurveto{\pgfqpoint{9.361007in}{2.948929in}}{\pgfqpoint{9.365397in}{2.959528in}}{\pgfqpoint{9.365397in}{2.970578in}}%
\pgfpathcurveto{\pgfqpoint{9.365397in}{2.981628in}}{\pgfqpoint{9.361007in}{2.992227in}}{\pgfqpoint{9.353193in}{3.000040in}}%
\pgfpathcurveto{\pgfqpoint{9.345380in}{3.007854in}}{\pgfqpoint{9.334781in}{3.012244in}}{\pgfqpoint{9.323730in}{3.012244in}}%
\pgfpathcurveto{\pgfqpoint{9.312680in}{3.012244in}}{\pgfqpoint{9.302081in}{3.007854in}}{\pgfqpoint{9.294268in}{3.000040in}}%
\pgfpathcurveto{\pgfqpoint{9.286454in}{2.992227in}}{\pgfqpoint{9.282064in}{2.981628in}}{\pgfqpoint{9.282064in}{2.970578in}}%
\pgfpathcurveto{\pgfqpoint{9.282064in}{2.959528in}}{\pgfqpoint{9.286454in}{2.948929in}}{\pgfqpoint{9.294268in}{2.941115in}}%
\pgfpathcurveto{\pgfqpoint{9.302081in}{2.933301in}}{\pgfqpoint{9.312680in}{2.928911in}}{\pgfqpoint{9.323730in}{2.928911in}}%
\pgfpathlineto{\pgfqpoint{9.323730in}{2.928911in}}%
\pgfpathclose%
\pgfusepath{stroke}%
\end{pgfscope}%
\begin{pgfscope}%
\pgfpathrectangle{\pgfqpoint{7.394209in}{0.375000in}}{\pgfqpoint{6.356833in}{5.175000in}}%
\pgfusepath{clip}%
\pgfsetbuttcap%
\pgfsetroundjoin%
\pgfsetlinewidth{1.003750pt}%
\definecolor{currentstroke}{rgb}{1.000000,0.000000,0.000000}%
\pgfsetstrokecolor{currentstroke}%
\pgfsetdash{}{0pt}%
\pgfpathmoveto{\pgfqpoint{9.734935in}{2.937946in}}%
\pgfpathcurveto{\pgfqpoint{9.745985in}{2.937946in}}{\pgfqpoint{9.756584in}{2.942336in}}{\pgfqpoint{9.764397in}{2.950150in}}%
\pgfpathcurveto{\pgfqpoint{9.772211in}{2.957963in}}{\pgfqpoint{9.776601in}{2.968562in}}{\pgfqpoint{9.776601in}{2.979612in}}%
\pgfpathcurveto{\pgfqpoint{9.776601in}{2.990663in}}{\pgfqpoint{9.772211in}{3.001262in}}{\pgfqpoint{9.764397in}{3.009075in}}%
\pgfpathcurveto{\pgfqpoint{9.756584in}{3.016889in}}{\pgfqpoint{9.745985in}{3.021279in}}{\pgfqpoint{9.734935in}{3.021279in}}%
\pgfpathcurveto{\pgfqpoint{9.723885in}{3.021279in}}{\pgfqpoint{9.713286in}{3.016889in}}{\pgfqpoint{9.705472in}{3.009075in}}%
\pgfpathcurveto{\pgfqpoint{9.697658in}{3.001262in}}{\pgfqpoint{9.693268in}{2.990663in}}{\pgfqpoint{9.693268in}{2.979612in}}%
\pgfpathcurveto{\pgfqpoint{9.693268in}{2.968562in}}{\pgfqpoint{9.697658in}{2.957963in}}{\pgfqpoint{9.705472in}{2.950150in}}%
\pgfpathcurveto{\pgfqpoint{9.713286in}{2.942336in}}{\pgfqpoint{9.723885in}{2.937946in}}{\pgfqpoint{9.734935in}{2.937946in}}%
\pgfpathlineto{\pgfqpoint{9.734935in}{2.937946in}}%
\pgfpathclose%
\pgfusepath{stroke}%
\end{pgfscope}%
\begin{pgfscope}%
\pgfpathrectangle{\pgfqpoint{7.394209in}{0.375000in}}{\pgfqpoint{6.356833in}{5.175000in}}%
\pgfusepath{clip}%
\pgfsetbuttcap%
\pgfsetroundjoin%
\pgfsetlinewidth{1.003750pt}%
\definecolor{currentstroke}{rgb}{1.000000,0.000000,0.000000}%
\pgfsetstrokecolor{currentstroke}%
\pgfsetdash{}{0pt}%
\pgfpathmoveto{\pgfqpoint{9.049371in}{3.146881in}}%
\pgfpathcurveto{\pgfqpoint{9.060421in}{3.146881in}}{\pgfqpoint{9.071020in}{3.151271in}}{\pgfqpoint{9.078834in}{3.159085in}}%
\pgfpathcurveto{\pgfqpoint{9.086648in}{3.166899in}}{\pgfqpoint{9.091038in}{3.177498in}}{\pgfqpoint{9.091038in}{3.188548in}}%
\pgfpathcurveto{\pgfqpoint{9.091038in}{3.199598in}}{\pgfqpoint{9.086648in}{3.210197in}}{\pgfqpoint{9.078834in}{3.218011in}}%
\pgfpathcurveto{\pgfqpoint{9.071020in}{3.225824in}}{\pgfqpoint{9.060421in}{3.230214in}}{\pgfqpoint{9.049371in}{3.230214in}}%
\pgfpathcurveto{\pgfqpoint{9.038321in}{3.230214in}}{\pgfqpoint{9.027722in}{3.225824in}}{\pgfqpoint{9.019908in}{3.218011in}}%
\pgfpathcurveto{\pgfqpoint{9.012095in}{3.210197in}}{\pgfqpoint{9.007705in}{3.199598in}}{\pgfqpoint{9.007705in}{3.188548in}}%
\pgfpathcurveto{\pgfqpoint{9.007705in}{3.177498in}}{\pgfqpoint{9.012095in}{3.166899in}}{\pgfqpoint{9.019908in}{3.159085in}}%
\pgfpathcurveto{\pgfqpoint{9.027722in}{3.151271in}}{\pgfqpoint{9.038321in}{3.146881in}}{\pgfqpoint{9.049371in}{3.146881in}}%
\pgfpathlineto{\pgfqpoint{9.049371in}{3.146881in}}%
\pgfpathclose%
\pgfusepath{stroke}%
\end{pgfscope}%
\begin{pgfscope}%
\pgfpathrectangle{\pgfqpoint{7.394209in}{0.375000in}}{\pgfqpoint{6.356833in}{5.175000in}}%
\pgfusepath{clip}%
\pgfsetbuttcap%
\pgfsetroundjoin%
\pgfsetlinewidth{1.003750pt}%
\definecolor{currentstroke}{rgb}{1.000000,0.000000,0.000000}%
\pgfsetstrokecolor{currentstroke}%
\pgfsetdash{}{0pt}%
\pgfpathmoveto{\pgfqpoint{9.635428in}{2.912504in}}%
\pgfpathcurveto{\pgfqpoint{9.646478in}{2.912504in}}{\pgfqpoint{9.657078in}{2.916894in}}{\pgfqpoint{9.664891in}{2.924708in}}%
\pgfpathcurveto{\pgfqpoint{9.672705in}{2.932521in}}{\pgfqpoint{9.677095in}{2.943120in}}{\pgfqpoint{9.677095in}{2.954171in}}%
\pgfpathcurveto{\pgfqpoint{9.677095in}{2.965221in}}{\pgfqpoint{9.672705in}{2.975820in}}{\pgfqpoint{9.664891in}{2.983633in}}%
\pgfpathcurveto{\pgfqpoint{9.657078in}{2.991447in}}{\pgfqpoint{9.646478in}{2.995837in}}{\pgfqpoint{9.635428in}{2.995837in}}%
\pgfpathcurveto{\pgfqpoint{9.624378in}{2.995837in}}{\pgfqpoint{9.613779in}{2.991447in}}{\pgfqpoint{9.605966in}{2.983633in}}%
\pgfpathcurveto{\pgfqpoint{9.598152in}{2.975820in}}{\pgfqpoint{9.593762in}{2.965221in}}{\pgfqpoint{9.593762in}{2.954171in}}%
\pgfpathcurveto{\pgfqpoint{9.593762in}{2.943120in}}{\pgfqpoint{9.598152in}{2.932521in}}{\pgfqpoint{9.605966in}{2.924708in}}%
\pgfpathcurveto{\pgfqpoint{9.613779in}{2.916894in}}{\pgfqpoint{9.624378in}{2.912504in}}{\pgfqpoint{9.635428in}{2.912504in}}%
\pgfpathlineto{\pgfqpoint{9.635428in}{2.912504in}}%
\pgfpathclose%
\pgfusepath{stroke}%
\end{pgfscope}%
\begin{pgfscope}%
\pgfpathrectangle{\pgfqpoint{7.394209in}{0.375000in}}{\pgfqpoint{6.356833in}{5.175000in}}%
\pgfusepath{clip}%
\pgfsetbuttcap%
\pgfsetroundjoin%
\pgfsetlinewidth{1.003750pt}%
\definecolor{currentstroke}{rgb}{1.000000,0.000000,0.000000}%
\pgfsetstrokecolor{currentstroke}%
\pgfsetdash{}{0pt}%
\pgfpathmoveto{\pgfqpoint{9.795843in}{4.260964in}}%
\pgfpathcurveto{\pgfqpoint{9.806893in}{4.260964in}}{\pgfqpoint{9.817492in}{4.265355in}}{\pgfqpoint{9.825306in}{4.273168in}}%
\pgfpathcurveto{\pgfqpoint{9.833119in}{4.280982in}}{\pgfqpoint{9.837509in}{4.291581in}}{\pgfqpoint{9.837509in}{4.302631in}}%
\pgfpathcurveto{\pgfqpoint{9.837509in}{4.313681in}}{\pgfqpoint{9.833119in}{4.324280in}}{\pgfqpoint{9.825306in}{4.332094in}}%
\pgfpathcurveto{\pgfqpoint{9.817492in}{4.339907in}}{\pgfqpoint{9.806893in}{4.344298in}}{\pgfqpoint{9.795843in}{4.344298in}}%
\pgfpathcurveto{\pgfqpoint{9.784793in}{4.344298in}}{\pgfqpoint{9.774194in}{4.339907in}}{\pgfqpoint{9.766380in}{4.332094in}}%
\pgfpathcurveto{\pgfqpoint{9.758566in}{4.324280in}}{\pgfqpoint{9.754176in}{4.313681in}}{\pgfqpoint{9.754176in}{4.302631in}}%
\pgfpathcurveto{\pgfqpoint{9.754176in}{4.291581in}}{\pgfqpoint{9.758566in}{4.280982in}}{\pgfqpoint{9.766380in}{4.273168in}}%
\pgfpathcurveto{\pgfqpoint{9.774194in}{4.265355in}}{\pgfqpoint{9.784793in}{4.260964in}}{\pgfqpoint{9.795843in}{4.260964in}}%
\pgfpathlineto{\pgfqpoint{9.795843in}{4.260964in}}%
\pgfpathclose%
\pgfusepath{stroke}%
\end{pgfscope}%
\begin{pgfscope}%
\pgfpathrectangle{\pgfqpoint{7.394209in}{0.375000in}}{\pgfqpoint{6.356833in}{5.175000in}}%
\pgfusepath{clip}%
\pgfsetbuttcap%
\pgfsetroundjoin%
\pgfsetlinewidth{1.003750pt}%
\definecolor{currentstroke}{rgb}{1.000000,0.000000,0.000000}%
\pgfsetstrokecolor{currentstroke}%
\pgfsetdash{}{0pt}%
\pgfpathmoveto{\pgfqpoint{8.513151in}{1.514737in}}%
\pgfpathcurveto{\pgfqpoint{8.524201in}{1.514737in}}{\pgfqpoint{8.534800in}{1.519127in}}{\pgfqpoint{8.542614in}{1.526941in}}%
\pgfpathcurveto{\pgfqpoint{8.550427in}{1.534754in}}{\pgfqpoint{8.554818in}{1.545353in}}{\pgfqpoint{8.554818in}{1.556404in}}%
\pgfpathcurveto{\pgfqpoint{8.554818in}{1.567454in}}{\pgfqpoint{8.550427in}{1.578053in}}{\pgfqpoint{8.542614in}{1.585866in}}%
\pgfpathcurveto{\pgfqpoint{8.534800in}{1.593680in}}{\pgfqpoint{8.524201in}{1.598070in}}{\pgfqpoint{8.513151in}{1.598070in}}%
\pgfpathcurveto{\pgfqpoint{8.502101in}{1.598070in}}{\pgfqpoint{8.491502in}{1.593680in}}{\pgfqpoint{8.483688in}{1.585866in}}%
\pgfpathcurveto{\pgfqpoint{8.475875in}{1.578053in}}{\pgfqpoint{8.471484in}{1.567454in}}{\pgfqpoint{8.471484in}{1.556404in}}%
\pgfpathcurveto{\pgfqpoint{8.471484in}{1.545353in}}{\pgfqpoint{8.475875in}{1.534754in}}{\pgfqpoint{8.483688in}{1.526941in}}%
\pgfpathcurveto{\pgfqpoint{8.491502in}{1.519127in}}{\pgfqpoint{8.502101in}{1.514737in}}{\pgfqpoint{8.513151in}{1.514737in}}%
\pgfpathlineto{\pgfqpoint{8.513151in}{1.514737in}}%
\pgfpathclose%
\pgfusepath{stroke}%
\end{pgfscope}%
\begin{pgfscope}%
\pgfpathrectangle{\pgfqpoint{7.394209in}{0.375000in}}{\pgfqpoint{6.356833in}{5.175000in}}%
\pgfusepath{clip}%
\pgfsetbuttcap%
\pgfsetroundjoin%
\pgfsetlinewidth{1.003750pt}%
\definecolor{currentstroke}{rgb}{1.000000,0.000000,0.000000}%
\pgfsetstrokecolor{currentstroke}%
\pgfsetdash{}{0pt}%
\pgfpathmoveto{\pgfqpoint{7.868271in}{1.122217in}}%
\pgfpathcurveto{\pgfqpoint{7.879321in}{1.122217in}}{\pgfqpoint{7.889920in}{1.126608in}}{\pgfqpoint{7.897734in}{1.134421in}}%
\pgfpathcurveto{\pgfqpoint{7.905548in}{1.142235in}}{\pgfqpoint{7.909938in}{1.152834in}}{\pgfqpoint{7.909938in}{1.163884in}}%
\pgfpathcurveto{\pgfqpoint{7.909938in}{1.174934in}}{\pgfqpoint{7.905548in}{1.185533in}}{\pgfqpoint{7.897734in}{1.193347in}}%
\pgfpathcurveto{\pgfqpoint{7.889920in}{1.201161in}}{\pgfqpoint{7.879321in}{1.205551in}}{\pgfqpoint{7.868271in}{1.205551in}}%
\pgfpathcurveto{\pgfqpoint{7.857221in}{1.205551in}}{\pgfqpoint{7.846622in}{1.201161in}}{\pgfqpoint{7.838809in}{1.193347in}}%
\pgfpathcurveto{\pgfqpoint{7.830995in}{1.185533in}}{\pgfqpoint{7.826605in}{1.174934in}}{\pgfqpoint{7.826605in}{1.163884in}}%
\pgfpathcurveto{\pgfqpoint{7.826605in}{1.152834in}}{\pgfqpoint{7.830995in}{1.142235in}}{\pgfqpoint{7.838809in}{1.134421in}}%
\pgfpathcurveto{\pgfqpoint{7.846622in}{1.126608in}}{\pgfqpoint{7.857221in}{1.122217in}}{\pgfqpoint{7.868271in}{1.122217in}}%
\pgfpathlineto{\pgfqpoint{7.868271in}{1.122217in}}%
\pgfpathclose%
\pgfusepath{stroke}%
\end{pgfscope}%
\begin{pgfscope}%
\pgfpathrectangle{\pgfqpoint{7.394209in}{0.375000in}}{\pgfqpoint{6.356833in}{5.175000in}}%
\pgfusepath{clip}%
\pgfsetbuttcap%
\pgfsetroundjoin%
\pgfsetlinewidth{1.003750pt}%
\definecolor{currentstroke}{rgb}{1.000000,0.000000,0.000000}%
\pgfsetstrokecolor{currentstroke}%
\pgfsetdash{}{0pt}%
\pgfpathmoveto{\pgfqpoint{10.175962in}{3.491154in}}%
\pgfpathcurveto{\pgfqpoint{10.187012in}{3.491154in}}{\pgfqpoint{10.197611in}{3.495544in}}{\pgfqpoint{10.205424in}{3.503357in}}%
\pgfpathcurveto{\pgfqpoint{10.213238in}{3.511171in}}{\pgfqpoint{10.217628in}{3.521770in}}{\pgfqpoint{10.217628in}{3.532820in}}%
\pgfpathcurveto{\pgfqpoint{10.217628in}{3.543870in}}{\pgfqpoint{10.213238in}{3.554469in}}{\pgfqpoint{10.205424in}{3.562283in}}%
\pgfpathcurveto{\pgfqpoint{10.197611in}{3.570097in}}{\pgfqpoint{10.187012in}{3.574487in}}{\pgfqpoint{10.175962in}{3.574487in}}%
\pgfpathcurveto{\pgfqpoint{10.164911in}{3.574487in}}{\pgfqpoint{10.154312in}{3.570097in}}{\pgfqpoint{10.146499in}{3.562283in}}%
\pgfpathcurveto{\pgfqpoint{10.138685in}{3.554469in}}{\pgfqpoint{10.134295in}{3.543870in}}{\pgfqpoint{10.134295in}{3.532820in}}%
\pgfpathcurveto{\pgfqpoint{10.134295in}{3.521770in}}{\pgfqpoint{10.138685in}{3.511171in}}{\pgfqpoint{10.146499in}{3.503357in}}%
\pgfpathcurveto{\pgfqpoint{10.154312in}{3.495544in}}{\pgfqpoint{10.164911in}{3.491154in}}{\pgfqpoint{10.175962in}{3.491154in}}%
\pgfpathlineto{\pgfqpoint{10.175962in}{3.491154in}}%
\pgfpathclose%
\pgfusepath{stroke}%
\end{pgfscope}%
\begin{pgfscope}%
\pgfpathrectangle{\pgfqpoint{7.394209in}{0.375000in}}{\pgfqpoint{6.356833in}{5.175000in}}%
\pgfusepath{clip}%
\pgfsetbuttcap%
\pgfsetroundjoin%
\pgfsetlinewidth{1.003750pt}%
\definecolor{currentstroke}{rgb}{1.000000,0.000000,0.000000}%
\pgfsetstrokecolor{currentstroke}%
\pgfsetdash{}{0pt}%
\pgfpathmoveto{\pgfqpoint{8.658904in}{2.083306in}}%
\pgfpathcurveto{\pgfqpoint{8.669954in}{2.083306in}}{\pgfqpoint{8.680553in}{2.087696in}}{\pgfqpoint{8.688367in}{2.095509in}}%
\pgfpathcurveto{\pgfqpoint{8.696180in}{2.103323in}}{\pgfqpoint{8.700570in}{2.113922in}}{\pgfqpoint{8.700570in}{2.124972in}}%
\pgfpathcurveto{\pgfqpoint{8.700570in}{2.136022in}}{\pgfqpoint{8.696180in}{2.146621in}}{\pgfqpoint{8.688367in}{2.154435in}}%
\pgfpathcurveto{\pgfqpoint{8.680553in}{2.162249in}}{\pgfqpoint{8.669954in}{2.166639in}}{\pgfqpoint{8.658904in}{2.166639in}}%
\pgfpathcurveto{\pgfqpoint{8.647854in}{2.166639in}}{\pgfqpoint{8.637255in}{2.162249in}}{\pgfqpoint{8.629441in}{2.154435in}}%
\pgfpathcurveto{\pgfqpoint{8.621627in}{2.146621in}}{\pgfqpoint{8.617237in}{2.136022in}}{\pgfqpoint{8.617237in}{2.124972in}}%
\pgfpathcurveto{\pgfqpoint{8.617237in}{2.113922in}}{\pgfqpoint{8.621627in}{2.103323in}}{\pgfqpoint{8.629441in}{2.095509in}}%
\pgfpathcurveto{\pgfqpoint{8.637255in}{2.087696in}}{\pgfqpoint{8.647854in}{2.083306in}}{\pgfqpoint{8.658904in}{2.083306in}}%
\pgfpathlineto{\pgfqpoint{8.658904in}{2.083306in}}%
\pgfpathclose%
\pgfusepath{stroke}%
\end{pgfscope}%
\begin{pgfscope}%
\pgfpathrectangle{\pgfqpoint{7.394209in}{0.375000in}}{\pgfqpoint{6.356833in}{5.175000in}}%
\pgfusepath{clip}%
\pgfsetbuttcap%
\pgfsetroundjoin%
\pgfsetlinewidth{1.003750pt}%
\definecolor{currentstroke}{rgb}{1.000000,0.000000,0.000000}%
\pgfsetstrokecolor{currentstroke}%
\pgfsetdash{}{0pt}%
\pgfpathmoveto{\pgfqpoint{9.016520in}{2.585381in}}%
\pgfpathcurveto{\pgfqpoint{9.027570in}{2.585381in}}{\pgfqpoint{9.038169in}{2.589772in}}{\pgfqpoint{9.045983in}{2.597585in}}%
\pgfpathcurveto{\pgfqpoint{9.053797in}{2.605399in}}{\pgfqpoint{9.058187in}{2.615998in}}{\pgfqpoint{9.058187in}{2.627048in}}%
\pgfpathcurveto{\pgfqpoint{9.058187in}{2.638098in}}{\pgfqpoint{9.053797in}{2.648697in}}{\pgfqpoint{9.045983in}{2.656511in}}%
\pgfpathcurveto{\pgfqpoint{9.038169in}{2.664324in}}{\pgfqpoint{9.027570in}{2.668715in}}{\pgfqpoint{9.016520in}{2.668715in}}%
\pgfpathcurveto{\pgfqpoint{9.005470in}{2.668715in}}{\pgfqpoint{8.994871in}{2.664324in}}{\pgfqpoint{8.987058in}{2.656511in}}%
\pgfpathcurveto{\pgfqpoint{8.979244in}{2.648697in}}{\pgfqpoint{8.974854in}{2.638098in}}{\pgfqpoint{8.974854in}{2.627048in}}%
\pgfpathcurveto{\pgfqpoint{8.974854in}{2.615998in}}{\pgfqpoint{8.979244in}{2.605399in}}{\pgfqpoint{8.987058in}{2.597585in}}%
\pgfpathcurveto{\pgfqpoint{8.994871in}{2.589772in}}{\pgfqpoint{9.005470in}{2.585381in}}{\pgfqpoint{9.016520in}{2.585381in}}%
\pgfpathlineto{\pgfqpoint{9.016520in}{2.585381in}}%
\pgfpathclose%
\pgfusepath{stroke}%
\end{pgfscope}%
\begin{pgfscope}%
\pgfpathrectangle{\pgfqpoint{7.394209in}{0.375000in}}{\pgfqpoint{6.356833in}{5.175000in}}%
\pgfusepath{clip}%
\pgfsetbuttcap%
\pgfsetroundjoin%
\pgfsetlinewidth{1.003750pt}%
\definecolor{currentstroke}{rgb}{1.000000,0.000000,0.000000}%
\pgfsetstrokecolor{currentstroke}%
\pgfsetdash{}{0pt}%
\pgfpathmoveto{\pgfqpoint{11.721604in}{5.343816in}}%
\pgfpathcurveto{\pgfqpoint{11.732655in}{5.343816in}}{\pgfqpoint{11.743254in}{5.348206in}}{\pgfqpoint{11.751067in}{5.356020in}}%
\pgfpathcurveto{\pgfqpoint{11.758881in}{5.363833in}}{\pgfqpoint{11.763271in}{5.374432in}}{\pgfqpoint{11.763271in}{5.385482in}}%
\pgfpathcurveto{\pgfqpoint{11.763271in}{5.396532in}}{\pgfqpoint{11.758881in}{5.407132in}}{\pgfqpoint{11.751067in}{5.414945in}}%
\pgfpathcurveto{\pgfqpoint{11.743254in}{5.422759in}}{\pgfqpoint{11.732655in}{5.427149in}}{\pgfqpoint{11.721604in}{5.427149in}}%
\pgfpathcurveto{\pgfqpoint{11.710554in}{5.427149in}}{\pgfqpoint{11.699955in}{5.422759in}}{\pgfqpoint{11.692142in}{5.414945in}}%
\pgfpathcurveto{\pgfqpoint{11.684328in}{5.407132in}}{\pgfqpoint{11.679938in}{5.396532in}}{\pgfqpoint{11.679938in}{5.385482in}}%
\pgfpathcurveto{\pgfqpoint{11.679938in}{5.374432in}}{\pgfqpoint{11.684328in}{5.363833in}}{\pgfqpoint{11.692142in}{5.356020in}}%
\pgfpathcurveto{\pgfqpoint{11.699955in}{5.348206in}}{\pgfqpoint{11.710554in}{5.343816in}}{\pgfqpoint{11.721604in}{5.343816in}}%
\pgfpathlineto{\pgfqpoint{11.721604in}{5.343816in}}%
\pgfpathclose%
\pgfusepath{stroke}%
\end{pgfscope}%
\begin{pgfscope}%
\pgfpathrectangle{\pgfqpoint{7.394209in}{0.375000in}}{\pgfqpoint{6.356833in}{5.175000in}}%
\pgfusepath{clip}%
\pgfsetbuttcap%
\pgfsetroundjoin%
\pgfsetlinewidth{1.003750pt}%
\definecolor{currentstroke}{rgb}{1.000000,0.000000,0.000000}%
\pgfsetstrokecolor{currentstroke}%
\pgfsetdash{}{0pt}%
\pgfpathmoveto{\pgfqpoint{7.455734in}{0.466429in}}%
\pgfpathcurveto{\pgfqpoint{7.466784in}{0.466429in}}{\pgfqpoint{7.477383in}{0.470820in}}{\pgfqpoint{7.485197in}{0.478633in}}%
\pgfpathcurveto{\pgfqpoint{7.493010in}{0.486447in}}{\pgfqpoint{7.497401in}{0.497046in}}{\pgfqpoint{7.497401in}{0.508096in}}%
\pgfpathcurveto{\pgfqpoint{7.497401in}{0.519146in}}{\pgfqpoint{7.493010in}{0.529745in}}{\pgfqpoint{7.485197in}{0.537559in}}%
\pgfpathcurveto{\pgfqpoint{7.477383in}{0.545372in}}{\pgfqpoint{7.466784in}{0.549763in}}{\pgfqpoint{7.455734in}{0.549763in}}%
\pgfpathcurveto{\pgfqpoint{7.444684in}{0.549763in}}{\pgfqpoint{7.434085in}{0.545372in}}{\pgfqpoint{7.426271in}{0.537559in}}%
\pgfpathcurveto{\pgfqpoint{7.418458in}{0.529745in}}{\pgfqpoint{7.414067in}{0.519146in}}{\pgfqpoint{7.414067in}{0.508096in}}%
\pgfpathcurveto{\pgfqpoint{7.414067in}{0.497046in}}{\pgfqpoint{7.418458in}{0.486447in}}{\pgfqpoint{7.426271in}{0.478633in}}%
\pgfpathcurveto{\pgfqpoint{7.434085in}{0.470820in}}{\pgfqpoint{7.444684in}{0.466429in}}{\pgfqpoint{7.455734in}{0.466429in}}%
\pgfpathlineto{\pgfqpoint{7.455734in}{0.466429in}}%
\pgfpathclose%
\pgfusepath{stroke}%
\end{pgfscope}%
\begin{pgfscope}%
\pgfpathrectangle{\pgfqpoint{7.394209in}{0.375000in}}{\pgfqpoint{6.356833in}{5.175000in}}%
\pgfusepath{clip}%
\pgfsetbuttcap%
\pgfsetroundjoin%
\pgfsetlinewidth{1.003750pt}%
\definecolor{currentstroke}{rgb}{1.000000,0.000000,0.000000}%
\pgfsetstrokecolor{currentstroke}%
\pgfsetdash{}{0pt}%
\pgfpathmoveto{\pgfqpoint{11.518484in}{5.508101in}}%
\pgfpathcurveto{\pgfqpoint{11.529534in}{5.508101in}}{\pgfqpoint{11.540133in}{5.512491in}}{\pgfqpoint{11.547946in}{5.520305in}}%
\pgfpathcurveto{\pgfqpoint{11.555760in}{5.528118in}}{\pgfqpoint{11.560150in}{5.538717in}}{\pgfqpoint{11.560150in}{5.549767in}}%
\pgfpathcurveto{\pgfqpoint{11.560150in}{5.560817in}}{\pgfqpoint{11.555760in}{5.571416in}}{\pgfqpoint{11.547946in}{5.579230in}}%
\pgfpathcurveto{\pgfqpoint{11.540133in}{5.587044in}}{\pgfqpoint{11.529534in}{5.591434in}}{\pgfqpoint{11.518484in}{5.591434in}}%
\pgfpathcurveto{\pgfqpoint{11.507434in}{5.591434in}}{\pgfqpoint{11.496834in}{5.587044in}}{\pgfqpoint{11.489021in}{5.579230in}}%
\pgfpathcurveto{\pgfqpoint{11.481207in}{5.571416in}}{\pgfqpoint{11.476817in}{5.560817in}}{\pgfqpoint{11.476817in}{5.549767in}}%
\pgfpathcurveto{\pgfqpoint{11.476817in}{5.538717in}}{\pgfqpoint{11.481207in}{5.528118in}}{\pgfqpoint{11.489021in}{5.520305in}}%
\pgfpathcurveto{\pgfqpoint{11.496834in}{5.512491in}}{\pgfqpoint{11.507434in}{5.508101in}}{\pgfqpoint{11.518484in}{5.508101in}}%
\pgfpathlineto{\pgfqpoint{11.518484in}{5.508101in}}%
\pgfpathclose%
\pgfusepath{stroke}%
\end{pgfscope}%
\begin{pgfscope}%
\pgfpathrectangle{\pgfqpoint{7.394209in}{0.375000in}}{\pgfqpoint{6.356833in}{5.175000in}}%
\pgfusepath{clip}%
\pgfsetbuttcap%
\pgfsetroundjoin%
\pgfsetlinewidth{1.003750pt}%
\definecolor{currentstroke}{rgb}{1.000000,0.000000,0.000000}%
\pgfsetstrokecolor{currentstroke}%
\pgfsetdash{}{0pt}%
\pgfpathmoveto{\pgfqpoint{8.132603in}{0.890877in}}%
\pgfpathcurveto{\pgfqpoint{8.143653in}{0.890877in}}{\pgfqpoint{8.154252in}{0.895267in}}{\pgfqpoint{8.162066in}{0.903081in}}%
\pgfpathcurveto{\pgfqpoint{8.169879in}{0.910895in}}{\pgfqpoint{8.174270in}{0.921494in}}{\pgfqpoint{8.174270in}{0.932544in}}%
\pgfpathcurveto{\pgfqpoint{8.174270in}{0.943594in}}{\pgfqpoint{8.169879in}{0.954193in}}{\pgfqpoint{8.162066in}{0.962007in}}%
\pgfpathcurveto{\pgfqpoint{8.154252in}{0.969820in}}{\pgfqpoint{8.143653in}{0.974210in}}{\pgfqpoint{8.132603in}{0.974210in}}%
\pgfpathcurveto{\pgfqpoint{8.121553in}{0.974210in}}{\pgfqpoint{8.110954in}{0.969820in}}{\pgfqpoint{8.103140in}{0.962007in}}%
\pgfpathcurveto{\pgfqpoint{8.095327in}{0.954193in}}{\pgfqpoint{8.090936in}{0.943594in}}{\pgfqpoint{8.090936in}{0.932544in}}%
\pgfpathcurveto{\pgfqpoint{8.090936in}{0.921494in}}{\pgfqpoint{8.095327in}{0.910895in}}{\pgfqpoint{8.103140in}{0.903081in}}%
\pgfpathcurveto{\pgfqpoint{8.110954in}{0.895267in}}{\pgfqpoint{8.121553in}{0.890877in}}{\pgfqpoint{8.132603in}{0.890877in}}%
\pgfpathlineto{\pgfqpoint{8.132603in}{0.890877in}}%
\pgfpathclose%
\pgfusepath{stroke}%
\end{pgfscope}%
\begin{pgfscope}%
\pgfpathrectangle{\pgfqpoint{7.394209in}{0.375000in}}{\pgfqpoint{6.356833in}{5.175000in}}%
\pgfusepath{clip}%
\pgfsetbuttcap%
\pgfsetroundjoin%
\pgfsetlinewidth{1.003750pt}%
\definecolor{currentstroke}{rgb}{1.000000,0.000000,0.000000}%
\pgfsetstrokecolor{currentstroke}%
\pgfsetdash{}{0pt}%
\pgfpathmoveto{\pgfqpoint{10.426214in}{4.588854in}}%
\pgfpathcurveto{\pgfqpoint{10.437264in}{4.588854in}}{\pgfqpoint{10.447864in}{4.593244in}}{\pgfqpoint{10.455677in}{4.601057in}}%
\pgfpathcurveto{\pgfqpoint{10.463491in}{4.608871in}}{\pgfqpoint{10.467881in}{4.619470in}}{\pgfqpoint{10.467881in}{4.630520in}}%
\pgfpathcurveto{\pgfqpoint{10.467881in}{4.641570in}}{\pgfqpoint{10.463491in}{4.652169in}}{\pgfqpoint{10.455677in}{4.659983in}}%
\pgfpathcurveto{\pgfqpoint{10.447864in}{4.667797in}}{\pgfqpoint{10.437264in}{4.672187in}}{\pgfqpoint{10.426214in}{4.672187in}}%
\pgfpathcurveto{\pgfqpoint{10.415164in}{4.672187in}}{\pgfqpoint{10.404565in}{4.667797in}}{\pgfqpoint{10.396752in}{4.659983in}}%
\pgfpathcurveto{\pgfqpoint{10.388938in}{4.652169in}}{\pgfqpoint{10.384548in}{4.641570in}}{\pgfqpoint{10.384548in}{4.630520in}}%
\pgfpathcurveto{\pgfqpoint{10.384548in}{4.619470in}}{\pgfqpoint{10.388938in}{4.608871in}}{\pgfqpoint{10.396752in}{4.601057in}}%
\pgfpathcurveto{\pgfqpoint{10.404565in}{4.593244in}}{\pgfqpoint{10.415164in}{4.588854in}}{\pgfqpoint{10.426214in}{4.588854in}}%
\pgfpathlineto{\pgfqpoint{10.426214in}{4.588854in}}%
\pgfpathclose%
\pgfusepath{stroke}%
\end{pgfscope}%
\begin{pgfscope}%
\pgfpathrectangle{\pgfqpoint{7.394209in}{0.375000in}}{\pgfqpoint{6.356833in}{5.175000in}}%
\pgfusepath{clip}%
\pgfsetbuttcap%
\pgfsetroundjoin%
\pgfsetlinewidth{1.003750pt}%
\definecolor{currentstroke}{rgb}{1.000000,0.000000,0.000000}%
\pgfsetstrokecolor{currentstroke}%
\pgfsetdash{}{0pt}%
\pgfpathmoveto{\pgfqpoint{12.459097in}{5.481955in}}%
\pgfpathcurveto{\pgfqpoint{12.470147in}{5.481955in}}{\pgfqpoint{12.480746in}{5.486346in}}{\pgfqpoint{12.488560in}{5.494159in}}%
\pgfpathcurveto{\pgfqpoint{12.496373in}{5.501973in}}{\pgfqpoint{12.500764in}{5.512572in}}{\pgfqpoint{12.500764in}{5.523622in}}%
\pgfpathcurveto{\pgfqpoint{12.500764in}{5.534672in}}{\pgfqpoint{12.496373in}{5.545271in}}{\pgfqpoint{12.488560in}{5.553085in}}%
\pgfpathcurveto{\pgfqpoint{12.480746in}{5.560899in}}{\pgfqpoint{12.470147in}{5.565289in}}{\pgfqpoint{12.459097in}{5.565289in}}%
\pgfpathcurveto{\pgfqpoint{12.448047in}{5.565289in}}{\pgfqpoint{12.437448in}{5.560899in}}{\pgfqpoint{12.429634in}{5.553085in}}%
\pgfpathcurveto{\pgfqpoint{12.421820in}{5.545271in}}{\pgfqpoint{12.417430in}{5.534672in}}{\pgfqpoint{12.417430in}{5.523622in}}%
\pgfpathcurveto{\pgfqpoint{12.417430in}{5.512572in}}{\pgfqpoint{12.421820in}{5.501973in}}{\pgfqpoint{12.429634in}{5.494159in}}%
\pgfpathcurveto{\pgfqpoint{12.437448in}{5.486346in}}{\pgfqpoint{12.448047in}{5.481955in}}{\pgfqpoint{12.459097in}{5.481955in}}%
\pgfpathlineto{\pgfqpoint{12.459097in}{5.481955in}}%
\pgfpathclose%
\pgfusepath{stroke}%
\end{pgfscope}%
\begin{pgfscope}%
\pgfpathrectangle{\pgfqpoint{7.394209in}{0.375000in}}{\pgfqpoint{6.356833in}{5.175000in}}%
\pgfusepath{clip}%
\pgfsetbuttcap%
\pgfsetroundjoin%
\pgfsetlinewidth{1.003750pt}%
\definecolor{currentstroke}{rgb}{1.000000,0.000000,0.000000}%
\pgfsetstrokecolor{currentstroke}%
\pgfsetdash{}{0pt}%
\pgfpathmoveto{\pgfqpoint{12.553179in}{5.467983in}}%
\pgfpathcurveto{\pgfqpoint{12.564229in}{5.467983in}}{\pgfqpoint{12.574828in}{5.472374in}}{\pgfqpoint{12.582642in}{5.480187in}}%
\pgfpathcurveto{\pgfqpoint{12.590455in}{5.488001in}}{\pgfqpoint{12.594846in}{5.498600in}}{\pgfqpoint{12.594846in}{5.509650in}}%
\pgfpathcurveto{\pgfqpoint{12.594846in}{5.520700in}}{\pgfqpoint{12.590455in}{5.531299in}}{\pgfqpoint{12.582642in}{5.539113in}}%
\pgfpathcurveto{\pgfqpoint{12.574828in}{5.546927in}}{\pgfqpoint{12.564229in}{5.551317in}}{\pgfqpoint{12.553179in}{5.551317in}}%
\pgfpathcurveto{\pgfqpoint{12.542129in}{5.551317in}}{\pgfqpoint{12.531530in}{5.546927in}}{\pgfqpoint{12.523716in}{5.539113in}}%
\pgfpathcurveto{\pgfqpoint{12.515903in}{5.531299in}}{\pgfqpoint{12.511512in}{5.520700in}}{\pgfqpoint{12.511512in}{5.509650in}}%
\pgfpathcurveto{\pgfqpoint{12.511512in}{5.498600in}}{\pgfqpoint{12.515903in}{5.488001in}}{\pgfqpoint{12.523716in}{5.480187in}}%
\pgfpathcurveto{\pgfqpoint{12.531530in}{5.472374in}}{\pgfqpoint{12.542129in}{5.467983in}}{\pgfqpoint{12.553179in}{5.467983in}}%
\pgfpathlineto{\pgfqpoint{12.553179in}{5.467983in}}%
\pgfpathclose%
\pgfusepath{stroke}%
\end{pgfscope}%
\begin{pgfscope}%
\pgfpathrectangle{\pgfqpoint{7.394209in}{0.375000in}}{\pgfqpoint{6.356833in}{5.175000in}}%
\pgfusepath{clip}%
\pgfsetbuttcap%
\pgfsetroundjoin%
\pgfsetlinewidth{1.003750pt}%
\definecolor{currentstroke}{rgb}{1.000000,0.000000,0.000000}%
\pgfsetstrokecolor{currentstroke}%
\pgfsetdash{}{0pt}%
\pgfpathmoveto{\pgfqpoint{11.138995in}{4.916273in}}%
\pgfpathcurveto{\pgfqpoint{11.150045in}{4.916273in}}{\pgfqpoint{11.160645in}{4.920663in}}{\pgfqpoint{11.168458in}{4.928477in}}%
\pgfpathcurveto{\pgfqpoint{11.176272in}{4.936290in}}{\pgfqpoint{11.180662in}{4.946889in}}{\pgfqpoint{11.180662in}{4.957939in}}%
\pgfpathcurveto{\pgfqpoint{11.180662in}{4.968989in}}{\pgfqpoint{11.176272in}{4.979588in}}{\pgfqpoint{11.168458in}{4.987402in}}%
\pgfpathcurveto{\pgfqpoint{11.160645in}{4.995216in}}{\pgfqpoint{11.150045in}{4.999606in}}{\pgfqpoint{11.138995in}{4.999606in}}%
\pgfpathcurveto{\pgfqpoint{11.127945in}{4.999606in}}{\pgfqpoint{11.117346in}{4.995216in}}{\pgfqpoint{11.109533in}{4.987402in}}%
\pgfpathcurveto{\pgfqpoint{11.101719in}{4.979588in}}{\pgfqpoint{11.097329in}{4.968989in}}{\pgfqpoint{11.097329in}{4.957939in}}%
\pgfpathcurveto{\pgfqpoint{11.097329in}{4.946889in}}{\pgfqpoint{11.101719in}{4.936290in}}{\pgfqpoint{11.109533in}{4.928477in}}%
\pgfpathcurveto{\pgfqpoint{11.117346in}{4.920663in}}{\pgfqpoint{11.127945in}{4.916273in}}{\pgfqpoint{11.138995in}{4.916273in}}%
\pgfpathlineto{\pgfqpoint{11.138995in}{4.916273in}}%
\pgfpathclose%
\pgfusepath{stroke}%
\end{pgfscope}%
\begin{pgfscope}%
\pgfpathrectangle{\pgfqpoint{7.394209in}{0.375000in}}{\pgfqpoint{6.356833in}{5.175000in}}%
\pgfusepath{clip}%
\pgfsetbuttcap%
\pgfsetroundjoin%
\pgfsetlinewidth{1.003750pt}%
\definecolor{currentstroke}{rgb}{1.000000,0.000000,0.000000}%
\pgfsetstrokecolor{currentstroke}%
\pgfsetdash{}{0pt}%
\pgfpathmoveto{\pgfqpoint{8.193762in}{1.208724in}}%
\pgfpathcurveto{\pgfqpoint{8.204812in}{1.208724in}}{\pgfqpoint{8.215411in}{1.213114in}}{\pgfqpoint{8.223225in}{1.220928in}}%
\pgfpathcurveto{\pgfqpoint{8.231039in}{1.228741in}}{\pgfqpoint{8.235429in}{1.239340in}}{\pgfqpoint{8.235429in}{1.250391in}}%
\pgfpathcurveto{\pgfqpoint{8.235429in}{1.261441in}}{\pgfqpoint{8.231039in}{1.272040in}}{\pgfqpoint{8.223225in}{1.279853in}}%
\pgfpathcurveto{\pgfqpoint{8.215411in}{1.287667in}}{\pgfqpoint{8.204812in}{1.292057in}}{\pgfqpoint{8.193762in}{1.292057in}}%
\pgfpathcurveto{\pgfqpoint{8.182712in}{1.292057in}}{\pgfqpoint{8.172113in}{1.287667in}}{\pgfqpoint{8.164299in}{1.279853in}}%
\pgfpathcurveto{\pgfqpoint{8.156486in}{1.272040in}}{\pgfqpoint{8.152095in}{1.261441in}}{\pgfqpoint{8.152095in}{1.250391in}}%
\pgfpathcurveto{\pgfqpoint{8.152095in}{1.239340in}}{\pgfqpoint{8.156486in}{1.228741in}}{\pgfqpoint{8.164299in}{1.220928in}}%
\pgfpathcurveto{\pgfqpoint{8.172113in}{1.213114in}}{\pgfqpoint{8.182712in}{1.208724in}}{\pgfqpoint{8.193762in}{1.208724in}}%
\pgfpathlineto{\pgfqpoint{8.193762in}{1.208724in}}%
\pgfpathclose%
\pgfusepath{stroke}%
\end{pgfscope}%
\begin{pgfscope}%
\pgfpathrectangle{\pgfqpoint{7.394209in}{0.375000in}}{\pgfqpoint{6.356833in}{5.175000in}}%
\pgfusepath{clip}%
\pgfsetbuttcap%
\pgfsetroundjoin%
\pgfsetlinewidth{1.003750pt}%
\definecolor{currentstroke}{rgb}{1.000000,0.000000,0.000000}%
\pgfsetstrokecolor{currentstroke}%
\pgfsetdash{}{0pt}%
\pgfpathmoveto{\pgfqpoint{9.128346in}{1.848705in}}%
\pgfpathcurveto{\pgfqpoint{9.139396in}{1.848705in}}{\pgfqpoint{9.149995in}{1.853096in}}{\pgfqpoint{9.157809in}{1.860909in}}%
\pgfpathcurveto{\pgfqpoint{9.165623in}{1.868723in}}{\pgfqpoint{9.170013in}{1.879322in}}{\pgfqpoint{9.170013in}{1.890372in}}%
\pgfpathcurveto{\pgfqpoint{9.170013in}{1.901422in}}{\pgfqpoint{9.165623in}{1.912021in}}{\pgfqpoint{9.157809in}{1.919835in}}%
\pgfpathcurveto{\pgfqpoint{9.149995in}{1.927648in}}{\pgfqpoint{9.139396in}{1.932039in}}{\pgfqpoint{9.128346in}{1.932039in}}%
\pgfpathcurveto{\pgfqpoint{9.117296in}{1.932039in}}{\pgfqpoint{9.106697in}{1.927648in}}{\pgfqpoint{9.098884in}{1.919835in}}%
\pgfpathcurveto{\pgfqpoint{9.091070in}{1.912021in}}{\pgfqpoint{9.086680in}{1.901422in}}{\pgfqpoint{9.086680in}{1.890372in}}%
\pgfpathcurveto{\pgfqpoint{9.086680in}{1.879322in}}{\pgfqpoint{9.091070in}{1.868723in}}{\pgfqpoint{9.098884in}{1.860909in}}%
\pgfpathcurveto{\pgfqpoint{9.106697in}{1.853096in}}{\pgfqpoint{9.117296in}{1.848705in}}{\pgfqpoint{9.128346in}{1.848705in}}%
\pgfpathlineto{\pgfqpoint{9.128346in}{1.848705in}}%
\pgfpathclose%
\pgfusepath{stroke}%
\end{pgfscope}%
\begin{pgfscope}%
\pgfpathrectangle{\pgfqpoint{7.394209in}{0.375000in}}{\pgfqpoint{6.356833in}{5.175000in}}%
\pgfusepath{clip}%
\pgfsetbuttcap%
\pgfsetroundjoin%
\pgfsetlinewidth{1.003750pt}%
\definecolor{currentstroke}{rgb}{1.000000,0.000000,0.000000}%
\pgfsetstrokecolor{currentstroke}%
\pgfsetdash{}{0pt}%
\pgfpathmoveto{\pgfqpoint{10.608044in}{4.745499in}}%
\pgfpathcurveto{\pgfqpoint{10.619094in}{4.745499in}}{\pgfqpoint{10.629693in}{4.749889in}}{\pgfqpoint{10.637507in}{4.757702in}}%
\pgfpathcurveto{\pgfqpoint{10.645321in}{4.765516in}}{\pgfqpoint{10.649711in}{4.776115in}}{\pgfqpoint{10.649711in}{4.787165in}}%
\pgfpathcurveto{\pgfqpoint{10.649711in}{4.798215in}}{\pgfqpoint{10.645321in}{4.808814in}}{\pgfqpoint{10.637507in}{4.816628in}}%
\pgfpathcurveto{\pgfqpoint{10.629693in}{4.824442in}}{\pgfqpoint{10.619094in}{4.828832in}}{\pgfqpoint{10.608044in}{4.828832in}}%
\pgfpathcurveto{\pgfqpoint{10.596994in}{4.828832in}}{\pgfqpoint{10.586395in}{4.824442in}}{\pgfqpoint{10.578581in}{4.816628in}}%
\pgfpathcurveto{\pgfqpoint{10.570768in}{4.808814in}}{\pgfqpoint{10.566378in}{4.798215in}}{\pgfqpoint{10.566378in}{4.787165in}}%
\pgfpathcurveto{\pgfqpoint{10.566378in}{4.776115in}}{\pgfqpoint{10.570768in}{4.765516in}}{\pgfqpoint{10.578581in}{4.757702in}}%
\pgfpathcurveto{\pgfqpoint{10.586395in}{4.749889in}}{\pgfqpoint{10.596994in}{4.745499in}}{\pgfqpoint{10.608044in}{4.745499in}}%
\pgfpathlineto{\pgfqpoint{10.608044in}{4.745499in}}%
\pgfpathclose%
\pgfusepath{stroke}%
\end{pgfscope}%
\begin{pgfscope}%
\pgfpathrectangle{\pgfqpoint{7.394209in}{0.375000in}}{\pgfqpoint{6.356833in}{5.175000in}}%
\pgfusepath{clip}%
\pgfsetbuttcap%
\pgfsetroundjoin%
\pgfsetlinewidth{1.003750pt}%
\definecolor{currentstroke}{rgb}{1.000000,0.000000,0.000000}%
\pgfsetstrokecolor{currentstroke}%
\pgfsetdash{}{0pt}%
\pgfpathmoveto{\pgfqpoint{7.498262in}{0.438974in}}%
\pgfpathcurveto{\pgfqpoint{7.509312in}{0.438974in}}{\pgfqpoint{7.519911in}{0.443364in}}{\pgfqpoint{7.527725in}{0.451178in}}%
\pgfpathcurveto{\pgfqpoint{7.535538in}{0.458991in}}{\pgfqpoint{7.539929in}{0.469590in}}{\pgfqpoint{7.539929in}{0.480641in}}%
\pgfpathcurveto{\pgfqpoint{7.539929in}{0.491691in}}{\pgfqpoint{7.535538in}{0.502290in}}{\pgfqpoint{7.527725in}{0.510103in}}%
\pgfpathcurveto{\pgfqpoint{7.519911in}{0.517917in}}{\pgfqpoint{7.509312in}{0.522307in}}{\pgfqpoint{7.498262in}{0.522307in}}%
\pgfpathcurveto{\pgfqpoint{7.487212in}{0.522307in}}{\pgfqpoint{7.476613in}{0.517917in}}{\pgfqpoint{7.468799in}{0.510103in}}%
\pgfpathcurveto{\pgfqpoint{7.460986in}{0.502290in}}{\pgfqpoint{7.456595in}{0.491691in}}{\pgfqpoint{7.456595in}{0.480641in}}%
\pgfpathcurveto{\pgfqpoint{7.456595in}{0.469590in}}{\pgfqpoint{7.460986in}{0.458991in}}{\pgfqpoint{7.468799in}{0.451178in}}%
\pgfpathcurveto{\pgfqpoint{7.476613in}{0.443364in}}{\pgfqpoint{7.487212in}{0.438974in}}{\pgfqpoint{7.498262in}{0.438974in}}%
\pgfpathlineto{\pgfqpoint{7.498262in}{0.438974in}}%
\pgfpathclose%
\pgfusepath{stroke}%
\end{pgfscope}%
\begin{pgfscope}%
\pgfpathrectangle{\pgfqpoint{7.394209in}{0.375000in}}{\pgfqpoint{6.356833in}{5.175000in}}%
\pgfusepath{clip}%
\pgfsetbuttcap%
\pgfsetroundjoin%
\pgfsetlinewidth{1.003750pt}%
\definecolor{currentstroke}{rgb}{1.000000,0.000000,0.000000}%
\pgfsetstrokecolor{currentstroke}%
\pgfsetdash{}{0pt}%
\pgfpathmoveto{\pgfqpoint{9.843959in}{3.307301in}}%
\pgfpathcurveto{\pgfqpoint{9.855009in}{3.307301in}}{\pgfqpoint{9.865608in}{3.311691in}}{\pgfqpoint{9.873422in}{3.319505in}}%
\pgfpathcurveto{\pgfqpoint{9.881235in}{3.327318in}}{\pgfqpoint{9.885626in}{3.337917in}}{\pgfqpoint{9.885626in}{3.348968in}}%
\pgfpathcurveto{\pgfqpoint{9.885626in}{3.360018in}}{\pgfqpoint{9.881235in}{3.370617in}}{\pgfqpoint{9.873422in}{3.378430in}}%
\pgfpathcurveto{\pgfqpoint{9.865608in}{3.386244in}}{\pgfqpoint{9.855009in}{3.390634in}}{\pgfqpoint{9.843959in}{3.390634in}}%
\pgfpathcurveto{\pgfqpoint{9.832909in}{3.390634in}}{\pgfqpoint{9.822310in}{3.386244in}}{\pgfqpoint{9.814496in}{3.378430in}}%
\pgfpathcurveto{\pgfqpoint{9.806683in}{3.370617in}}{\pgfqpoint{9.802292in}{3.360018in}}{\pgfqpoint{9.802292in}{3.348968in}}%
\pgfpathcurveto{\pgfqpoint{9.802292in}{3.337917in}}{\pgfqpoint{9.806683in}{3.327318in}}{\pgfqpoint{9.814496in}{3.319505in}}%
\pgfpathcurveto{\pgfqpoint{9.822310in}{3.311691in}}{\pgfqpoint{9.832909in}{3.307301in}}{\pgfqpoint{9.843959in}{3.307301in}}%
\pgfpathlineto{\pgfqpoint{9.843959in}{3.307301in}}%
\pgfpathclose%
\pgfusepath{stroke}%
\end{pgfscope}%
\begin{pgfscope}%
\pgfpathrectangle{\pgfqpoint{7.394209in}{0.375000in}}{\pgfqpoint{6.356833in}{5.175000in}}%
\pgfusepath{clip}%
\pgfsetbuttcap%
\pgfsetroundjoin%
\pgfsetlinewidth{1.003750pt}%
\definecolor{currentstroke}{rgb}{1.000000,0.000000,0.000000}%
\pgfsetstrokecolor{currentstroke}%
\pgfsetdash{}{0pt}%
\pgfpathmoveto{\pgfqpoint{7.659903in}{0.428020in}}%
\pgfpathcurveto{\pgfqpoint{7.670953in}{0.428020in}}{\pgfqpoint{7.681552in}{0.432410in}}{\pgfqpoint{7.689365in}{0.440224in}}%
\pgfpathcurveto{\pgfqpoint{7.697179in}{0.448037in}}{\pgfqpoint{7.701569in}{0.458636in}}{\pgfqpoint{7.701569in}{0.469686in}}%
\pgfpathcurveto{\pgfqpoint{7.701569in}{0.480737in}}{\pgfqpoint{7.697179in}{0.491336in}}{\pgfqpoint{7.689365in}{0.499149in}}%
\pgfpathcurveto{\pgfqpoint{7.681552in}{0.506963in}}{\pgfqpoint{7.670953in}{0.511353in}}{\pgfqpoint{7.659903in}{0.511353in}}%
\pgfpathcurveto{\pgfqpoint{7.648853in}{0.511353in}}{\pgfqpoint{7.638254in}{0.506963in}}{\pgfqpoint{7.630440in}{0.499149in}}%
\pgfpathcurveto{\pgfqpoint{7.622626in}{0.491336in}}{\pgfqpoint{7.618236in}{0.480737in}}{\pgfqpoint{7.618236in}{0.469686in}}%
\pgfpathcurveto{\pgfqpoint{7.618236in}{0.458636in}}{\pgfqpoint{7.622626in}{0.448037in}}{\pgfqpoint{7.630440in}{0.440224in}}%
\pgfpathcurveto{\pgfqpoint{7.638254in}{0.432410in}}{\pgfqpoint{7.648853in}{0.428020in}}{\pgfqpoint{7.659903in}{0.428020in}}%
\pgfpathlineto{\pgfqpoint{7.659903in}{0.428020in}}%
\pgfpathclose%
\pgfusepath{stroke}%
\end{pgfscope}%
\begin{pgfscope}%
\pgfpathrectangle{\pgfqpoint{7.394209in}{0.375000in}}{\pgfqpoint{6.356833in}{5.175000in}}%
\pgfusepath{clip}%
\pgfsetbuttcap%
\pgfsetroundjoin%
\pgfsetlinewidth{1.003750pt}%
\definecolor{currentstroke}{rgb}{1.000000,0.000000,0.000000}%
\pgfsetstrokecolor{currentstroke}%
\pgfsetdash{}{0pt}%
\pgfpathmoveto{\pgfqpoint{10.671880in}{3.898847in}}%
\pgfpathcurveto{\pgfqpoint{10.682930in}{3.898847in}}{\pgfqpoint{10.693529in}{3.903238in}}{\pgfqpoint{10.701343in}{3.911051in}}%
\pgfpathcurveto{\pgfqpoint{10.709156in}{3.918865in}}{\pgfqpoint{10.713547in}{3.929464in}}{\pgfqpoint{10.713547in}{3.940514in}}%
\pgfpathcurveto{\pgfqpoint{10.713547in}{3.951564in}}{\pgfqpoint{10.709156in}{3.962163in}}{\pgfqpoint{10.701343in}{3.969977in}}%
\pgfpathcurveto{\pgfqpoint{10.693529in}{3.977790in}}{\pgfqpoint{10.682930in}{3.982181in}}{\pgfqpoint{10.671880in}{3.982181in}}%
\pgfpathcurveto{\pgfqpoint{10.660830in}{3.982181in}}{\pgfqpoint{10.650231in}{3.977790in}}{\pgfqpoint{10.642417in}{3.969977in}}%
\pgfpathcurveto{\pgfqpoint{10.634604in}{3.962163in}}{\pgfqpoint{10.630213in}{3.951564in}}{\pgfqpoint{10.630213in}{3.940514in}}%
\pgfpathcurveto{\pgfqpoint{10.630213in}{3.929464in}}{\pgfqpoint{10.634604in}{3.918865in}}{\pgfqpoint{10.642417in}{3.911051in}}%
\pgfpathcurveto{\pgfqpoint{10.650231in}{3.903238in}}{\pgfqpoint{10.660830in}{3.898847in}}{\pgfqpoint{10.671880in}{3.898847in}}%
\pgfpathlineto{\pgfqpoint{10.671880in}{3.898847in}}%
\pgfpathclose%
\pgfusepath{stroke}%
\end{pgfscope}%
\begin{pgfscope}%
\pgfpathrectangle{\pgfqpoint{7.394209in}{0.375000in}}{\pgfqpoint{6.356833in}{5.175000in}}%
\pgfusepath{clip}%
\pgfsetbuttcap%
\pgfsetroundjoin%
\pgfsetlinewidth{1.003750pt}%
\definecolor{currentstroke}{rgb}{1.000000,0.000000,0.000000}%
\pgfsetstrokecolor{currentstroke}%
\pgfsetdash{}{0pt}%
\pgfpathmoveto{\pgfqpoint{8.727403in}{2.332225in}}%
\pgfpathcurveto{\pgfqpoint{8.738453in}{2.332225in}}{\pgfqpoint{8.749052in}{2.336615in}}{\pgfqpoint{8.756866in}{2.344429in}}%
\pgfpathcurveto{\pgfqpoint{8.764680in}{2.352243in}}{\pgfqpoint{8.769070in}{2.362842in}}{\pgfqpoint{8.769070in}{2.373892in}}%
\pgfpathcurveto{\pgfqpoint{8.769070in}{2.384942in}}{\pgfqpoint{8.764680in}{2.395541in}}{\pgfqpoint{8.756866in}{2.403355in}}%
\pgfpathcurveto{\pgfqpoint{8.749052in}{2.411168in}}{\pgfqpoint{8.738453in}{2.415558in}}{\pgfqpoint{8.727403in}{2.415558in}}%
\pgfpathcurveto{\pgfqpoint{8.716353in}{2.415558in}}{\pgfqpoint{8.705754in}{2.411168in}}{\pgfqpoint{8.697941in}{2.403355in}}%
\pgfpathcurveto{\pgfqpoint{8.690127in}{2.395541in}}{\pgfqpoint{8.685737in}{2.384942in}}{\pgfqpoint{8.685737in}{2.373892in}}%
\pgfpathcurveto{\pgfqpoint{8.685737in}{2.362842in}}{\pgfqpoint{8.690127in}{2.352243in}}{\pgfqpoint{8.697941in}{2.344429in}}%
\pgfpathcurveto{\pgfqpoint{8.705754in}{2.336615in}}{\pgfqpoint{8.716353in}{2.332225in}}{\pgfqpoint{8.727403in}{2.332225in}}%
\pgfpathlineto{\pgfqpoint{8.727403in}{2.332225in}}%
\pgfpathclose%
\pgfusepath{stroke}%
\end{pgfscope}%
\begin{pgfscope}%
\pgfpathrectangle{\pgfqpoint{7.394209in}{0.375000in}}{\pgfqpoint{6.356833in}{5.175000in}}%
\pgfusepath{clip}%
\pgfsetbuttcap%
\pgfsetroundjoin%
\pgfsetlinewidth{1.003750pt}%
\definecolor{currentstroke}{rgb}{1.000000,0.000000,0.000000}%
\pgfsetstrokecolor{currentstroke}%
\pgfsetdash{}{0pt}%
\pgfpathmoveto{\pgfqpoint{12.032662in}{5.507766in}}%
\pgfpathcurveto{\pgfqpoint{12.043712in}{5.507766in}}{\pgfqpoint{12.054311in}{5.512156in}}{\pgfqpoint{12.062125in}{5.519970in}}%
\pgfpathcurveto{\pgfqpoint{12.069938in}{5.527783in}}{\pgfqpoint{12.074329in}{5.538382in}}{\pgfqpoint{12.074329in}{5.549432in}}%
\pgfpathcurveto{\pgfqpoint{12.074329in}{5.560483in}}{\pgfqpoint{12.069938in}{5.571082in}}{\pgfqpoint{12.062125in}{5.578895in}}%
\pgfpathcurveto{\pgfqpoint{12.054311in}{5.586709in}}{\pgfqpoint{12.043712in}{5.591099in}}{\pgfqpoint{12.032662in}{5.591099in}}%
\pgfpathcurveto{\pgfqpoint{12.021612in}{5.591099in}}{\pgfqpoint{12.011013in}{5.586709in}}{\pgfqpoint{12.003199in}{5.578895in}}%
\pgfpathcurveto{\pgfqpoint{11.995386in}{5.571082in}}{\pgfqpoint{11.990995in}{5.560483in}}{\pgfqpoint{11.990995in}{5.549432in}}%
\pgfpathcurveto{\pgfqpoint{11.990995in}{5.538382in}}{\pgfqpoint{11.995386in}{5.527783in}}{\pgfqpoint{12.003199in}{5.519970in}}%
\pgfpathcurveto{\pgfqpoint{12.011013in}{5.512156in}}{\pgfqpoint{12.021612in}{5.507766in}}{\pgfqpoint{12.032662in}{5.507766in}}%
\pgfpathlineto{\pgfqpoint{12.032662in}{5.507766in}}%
\pgfpathclose%
\pgfusepath{stroke}%
\end{pgfscope}%
\begin{pgfscope}%
\pgfpathrectangle{\pgfqpoint{7.394209in}{0.375000in}}{\pgfqpoint{6.356833in}{5.175000in}}%
\pgfusepath{clip}%
\pgfsetbuttcap%
\pgfsetroundjoin%
\pgfsetlinewidth{1.003750pt}%
\definecolor{currentstroke}{rgb}{1.000000,0.000000,0.000000}%
\pgfsetstrokecolor{currentstroke}%
\pgfsetdash{}{0pt}%
\pgfpathmoveto{\pgfqpoint{10.062150in}{3.399198in}}%
\pgfpathcurveto{\pgfqpoint{10.073201in}{3.399198in}}{\pgfqpoint{10.083800in}{3.403588in}}{\pgfqpoint{10.091613in}{3.411402in}}%
\pgfpathcurveto{\pgfqpoint{10.099427in}{3.419215in}}{\pgfqpoint{10.103817in}{3.429815in}}{\pgfqpoint{10.103817in}{3.440865in}}%
\pgfpathcurveto{\pgfqpoint{10.103817in}{3.451915in}}{\pgfqpoint{10.099427in}{3.462514in}}{\pgfqpoint{10.091613in}{3.470327in}}%
\pgfpathcurveto{\pgfqpoint{10.083800in}{3.478141in}}{\pgfqpoint{10.073201in}{3.482531in}}{\pgfqpoint{10.062150in}{3.482531in}}%
\pgfpathcurveto{\pgfqpoint{10.051100in}{3.482531in}}{\pgfqpoint{10.040501in}{3.478141in}}{\pgfqpoint{10.032688in}{3.470327in}}%
\pgfpathcurveto{\pgfqpoint{10.024874in}{3.462514in}}{\pgfqpoint{10.020484in}{3.451915in}}{\pgfqpoint{10.020484in}{3.440865in}}%
\pgfpathcurveto{\pgfqpoint{10.020484in}{3.429815in}}{\pgfqpoint{10.024874in}{3.419215in}}{\pgfqpoint{10.032688in}{3.411402in}}%
\pgfpathcurveto{\pgfqpoint{10.040501in}{3.403588in}}{\pgfqpoint{10.051100in}{3.399198in}}{\pgfqpoint{10.062150in}{3.399198in}}%
\pgfpathlineto{\pgfqpoint{10.062150in}{3.399198in}}%
\pgfpathclose%
\pgfusepath{stroke}%
\end{pgfscope}%
\begin{pgfscope}%
\pgfpathrectangle{\pgfqpoint{7.394209in}{0.375000in}}{\pgfqpoint{6.356833in}{5.175000in}}%
\pgfusepath{clip}%
\pgfsetbuttcap%
\pgfsetroundjoin%
\pgfsetlinewidth{1.003750pt}%
\definecolor{currentstroke}{rgb}{1.000000,0.000000,0.000000}%
\pgfsetstrokecolor{currentstroke}%
\pgfsetdash{}{0pt}%
\pgfpathmoveto{\pgfqpoint{8.871144in}{2.932216in}}%
\pgfpathcurveto{\pgfqpoint{8.882194in}{2.932216in}}{\pgfqpoint{8.892793in}{2.936606in}}{\pgfqpoint{8.900607in}{2.944420in}}%
\pgfpathcurveto{\pgfqpoint{8.908420in}{2.952234in}}{\pgfqpoint{8.912811in}{2.962833in}}{\pgfqpoint{8.912811in}{2.973883in}}%
\pgfpathcurveto{\pgfqpoint{8.912811in}{2.984933in}}{\pgfqpoint{8.908420in}{2.995532in}}{\pgfqpoint{8.900607in}{3.003346in}}%
\pgfpathcurveto{\pgfqpoint{8.892793in}{3.011159in}}{\pgfqpoint{8.882194in}{3.015549in}}{\pgfqpoint{8.871144in}{3.015549in}}%
\pgfpathcurveto{\pgfqpoint{8.860094in}{3.015549in}}{\pgfqpoint{8.849495in}{3.011159in}}{\pgfqpoint{8.841681in}{3.003346in}}%
\pgfpathcurveto{\pgfqpoint{8.833867in}{2.995532in}}{\pgfqpoint{8.829477in}{2.984933in}}{\pgfqpoint{8.829477in}{2.973883in}}%
\pgfpathcurveto{\pgfqpoint{8.829477in}{2.962833in}}{\pgfqpoint{8.833867in}{2.952234in}}{\pgfqpoint{8.841681in}{2.944420in}}%
\pgfpathcurveto{\pgfqpoint{8.849495in}{2.936606in}}{\pgfqpoint{8.860094in}{2.932216in}}{\pgfqpoint{8.871144in}{2.932216in}}%
\pgfpathlineto{\pgfqpoint{8.871144in}{2.932216in}}%
\pgfpathclose%
\pgfusepath{stroke}%
\end{pgfscope}%
\begin{pgfscope}%
\pgfpathrectangle{\pgfqpoint{7.394209in}{0.375000in}}{\pgfqpoint{6.356833in}{5.175000in}}%
\pgfusepath{clip}%
\pgfsetbuttcap%
\pgfsetroundjoin%
\pgfsetlinewidth{1.003750pt}%
\definecolor{currentstroke}{rgb}{1.000000,0.000000,0.000000}%
\pgfsetstrokecolor{currentstroke}%
\pgfsetdash{}{0pt}%
\pgfpathmoveto{\pgfqpoint{13.187292in}{5.507351in}}%
\pgfpathcurveto{\pgfqpoint{13.198342in}{5.507351in}}{\pgfqpoint{13.208941in}{5.511742in}}{\pgfqpoint{13.216755in}{5.519555in}}%
\pgfpathcurveto{\pgfqpoint{13.224569in}{5.527369in}}{\pgfqpoint{13.228959in}{5.537968in}}{\pgfqpoint{13.228959in}{5.549018in}}%
\pgfpathcurveto{\pgfqpoint{13.228959in}{5.560068in}}{\pgfqpoint{13.224569in}{5.570667in}}{\pgfqpoint{13.216755in}{5.578481in}}%
\pgfpathcurveto{\pgfqpoint{13.208941in}{5.586294in}}{\pgfqpoint{13.198342in}{5.590685in}}{\pgfqpoint{13.187292in}{5.590685in}}%
\pgfpathcurveto{\pgfqpoint{13.176242in}{5.590685in}}{\pgfqpoint{13.165643in}{5.586294in}}{\pgfqpoint{13.157829in}{5.578481in}}%
\pgfpathcurveto{\pgfqpoint{13.150016in}{5.570667in}}{\pgfqpoint{13.145626in}{5.560068in}}{\pgfqpoint{13.145626in}{5.549018in}}%
\pgfpathcurveto{\pgfqpoint{13.145626in}{5.537968in}}{\pgfqpoint{13.150016in}{5.527369in}}{\pgfqpoint{13.157829in}{5.519555in}}%
\pgfpathcurveto{\pgfqpoint{13.165643in}{5.511742in}}{\pgfqpoint{13.176242in}{5.507351in}}{\pgfqpoint{13.187292in}{5.507351in}}%
\pgfpathlineto{\pgfqpoint{13.187292in}{5.507351in}}%
\pgfpathclose%
\pgfusepath{stroke}%
\end{pgfscope}%
\begin{pgfscope}%
\pgfpathrectangle{\pgfqpoint{7.394209in}{0.375000in}}{\pgfqpoint{6.356833in}{5.175000in}}%
\pgfusepath{clip}%
\pgfsetbuttcap%
\pgfsetroundjoin%
\pgfsetlinewidth{1.003750pt}%
\definecolor{currentstroke}{rgb}{1.000000,0.000000,0.000000}%
\pgfsetstrokecolor{currentstroke}%
\pgfsetdash{}{0pt}%
\pgfpathmoveto{\pgfqpoint{8.813985in}{2.111749in}}%
\pgfpathcurveto{\pgfqpoint{8.825035in}{2.111749in}}{\pgfqpoint{8.835634in}{2.116140in}}{\pgfqpoint{8.843448in}{2.123953in}}%
\pgfpathcurveto{\pgfqpoint{8.851261in}{2.131767in}}{\pgfqpoint{8.855652in}{2.142366in}}{\pgfqpoint{8.855652in}{2.153416in}}%
\pgfpathcurveto{\pgfqpoint{8.855652in}{2.164466in}}{\pgfqpoint{8.851261in}{2.175065in}}{\pgfqpoint{8.843448in}{2.182879in}}%
\pgfpathcurveto{\pgfqpoint{8.835634in}{2.190693in}}{\pgfqpoint{8.825035in}{2.195083in}}{\pgfqpoint{8.813985in}{2.195083in}}%
\pgfpathcurveto{\pgfqpoint{8.802935in}{2.195083in}}{\pgfqpoint{8.792336in}{2.190693in}}{\pgfqpoint{8.784522in}{2.182879in}}%
\pgfpathcurveto{\pgfqpoint{8.776709in}{2.175065in}}{\pgfqpoint{8.772318in}{2.164466in}}{\pgfqpoint{8.772318in}{2.153416in}}%
\pgfpathcurveto{\pgfqpoint{8.772318in}{2.142366in}}{\pgfqpoint{8.776709in}{2.131767in}}{\pgfqpoint{8.784522in}{2.123953in}}%
\pgfpathcurveto{\pgfqpoint{8.792336in}{2.116140in}}{\pgfqpoint{8.802935in}{2.111749in}}{\pgfqpoint{8.813985in}{2.111749in}}%
\pgfpathlineto{\pgfqpoint{8.813985in}{2.111749in}}%
\pgfpathclose%
\pgfusepath{stroke}%
\end{pgfscope}%
\begin{pgfscope}%
\pgfpathrectangle{\pgfqpoint{7.394209in}{0.375000in}}{\pgfqpoint{6.356833in}{5.175000in}}%
\pgfusepath{clip}%
\pgfsetbuttcap%
\pgfsetroundjoin%
\pgfsetlinewidth{1.003750pt}%
\definecolor{currentstroke}{rgb}{1.000000,0.000000,0.000000}%
\pgfsetstrokecolor{currentstroke}%
\pgfsetdash{}{0pt}%
\pgfpathmoveto{\pgfqpoint{8.205712in}{1.106959in}}%
\pgfpathcurveto{\pgfqpoint{8.216762in}{1.106959in}}{\pgfqpoint{8.227361in}{1.111349in}}{\pgfqpoint{8.235175in}{1.119163in}}%
\pgfpathcurveto{\pgfqpoint{8.242988in}{1.126976in}}{\pgfqpoint{8.247379in}{1.137575in}}{\pgfqpoint{8.247379in}{1.148626in}}%
\pgfpathcurveto{\pgfqpoint{8.247379in}{1.159676in}}{\pgfqpoint{8.242988in}{1.170275in}}{\pgfqpoint{8.235175in}{1.178088in}}%
\pgfpathcurveto{\pgfqpoint{8.227361in}{1.185902in}}{\pgfqpoint{8.216762in}{1.190292in}}{\pgfqpoint{8.205712in}{1.190292in}}%
\pgfpathcurveto{\pgfqpoint{8.194662in}{1.190292in}}{\pgfqpoint{8.184063in}{1.185902in}}{\pgfqpoint{8.176249in}{1.178088in}}%
\pgfpathcurveto{\pgfqpoint{8.168436in}{1.170275in}}{\pgfqpoint{8.164045in}{1.159676in}}{\pgfqpoint{8.164045in}{1.148626in}}%
\pgfpathcurveto{\pgfqpoint{8.164045in}{1.137575in}}{\pgfqpoint{8.168436in}{1.126976in}}{\pgfqpoint{8.176249in}{1.119163in}}%
\pgfpathcurveto{\pgfqpoint{8.184063in}{1.111349in}}{\pgfqpoint{8.194662in}{1.106959in}}{\pgfqpoint{8.205712in}{1.106959in}}%
\pgfpathlineto{\pgfqpoint{8.205712in}{1.106959in}}%
\pgfpathclose%
\pgfusepath{stroke}%
\end{pgfscope}%
\begin{pgfscope}%
\pgfpathrectangle{\pgfqpoint{7.394209in}{0.375000in}}{\pgfqpoint{6.356833in}{5.175000in}}%
\pgfusepath{clip}%
\pgfsetbuttcap%
\pgfsetroundjoin%
\pgfsetlinewidth{1.003750pt}%
\definecolor{currentstroke}{rgb}{1.000000,0.000000,0.000000}%
\pgfsetstrokecolor{currentstroke}%
\pgfsetdash{}{0pt}%
\pgfpathmoveto{\pgfqpoint{10.816465in}{5.482308in}}%
\pgfpathcurveto{\pgfqpoint{10.827515in}{5.482308in}}{\pgfqpoint{10.838114in}{5.486698in}}{\pgfqpoint{10.845928in}{5.494512in}}%
\pgfpathcurveto{\pgfqpoint{10.853741in}{5.502326in}}{\pgfqpoint{10.858132in}{5.512925in}}{\pgfqpoint{10.858132in}{5.523975in}}%
\pgfpathcurveto{\pgfqpoint{10.858132in}{5.535025in}}{\pgfqpoint{10.853741in}{5.545624in}}{\pgfqpoint{10.845928in}{5.553438in}}%
\pgfpathcurveto{\pgfqpoint{10.838114in}{5.561251in}}{\pgfqpoint{10.827515in}{5.565641in}}{\pgfqpoint{10.816465in}{5.565641in}}%
\pgfpathcurveto{\pgfqpoint{10.805415in}{5.565641in}}{\pgfqpoint{10.794816in}{5.561251in}}{\pgfqpoint{10.787002in}{5.553438in}}%
\pgfpathcurveto{\pgfqpoint{10.779189in}{5.545624in}}{\pgfqpoint{10.774798in}{5.535025in}}{\pgfqpoint{10.774798in}{5.523975in}}%
\pgfpathcurveto{\pgfqpoint{10.774798in}{5.512925in}}{\pgfqpoint{10.779189in}{5.502326in}}{\pgfqpoint{10.787002in}{5.494512in}}%
\pgfpathcurveto{\pgfqpoint{10.794816in}{5.486698in}}{\pgfqpoint{10.805415in}{5.482308in}}{\pgfqpoint{10.816465in}{5.482308in}}%
\pgfpathlineto{\pgfqpoint{10.816465in}{5.482308in}}%
\pgfpathclose%
\pgfusepath{stroke}%
\end{pgfscope}%
\begin{pgfscope}%
\pgfpathrectangle{\pgfqpoint{7.394209in}{0.375000in}}{\pgfqpoint{6.356833in}{5.175000in}}%
\pgfusepath{clip}%
\pgfsetbuttcap%
\pgfsetroundjoin%
\pgfsetlinewidth{1.003750pt}%
\definecolor{currentstroke}{rgb}{1.000000,0.000000,0.000000}%
\pgfsetstrokecolor{currentstroke}%
\pgfsetdash{}{0pt}%
\pgfpathmoveto{\pgfqpoint{11.426198in}{5.490013in}}%
\pgfpathcurveto{\pgfqpoint{11.437248in}{5.490013in}}{\pgfqpoint{11.447847in}{5.494404in}}{\pgfqpoint{11.455661in}{5.502217in}}%
\pgfpathcurveto{\pgfqpoint{11.463474in}{5.510031in}}{\pgfqpoint{11.467864in}{5.520630in}}{\pgfqpoint{11.467864in}{5.531680in}}%
\pgfpathcurveto{\pgfqpoint{11.467864in}{5.542730in}}{\pgfqpoint{11.463474in}{5.553329in}}{\pgfqpoint{11.455661in}{5.561143in}}%
\pgfpathcurveto{\pgfqpoint{11.447847in}{5.568956in}}{\pgfqpoint{11.437248in}{5.573347in}}{\pgfqpoint{11.426198in}{5.573347in}}%
\pgfpathcurveto{\pgfqpoint{11.415148in}{5.573347in}}{\pgfqpoint{11.404549in}{5.568956in}}{\pgfqpoint{11.396735in}{5.561143in}}%
\pgfpathcurveto{\pgfqpoint{11.388921in}{5.553329in}}{\pgfqpoint{11.384531in}{5.542730in}}{\pgfqpoint{11.384531in}{5.531680in}}%
\pgfpathcurveto{\pgfqpoint{11.384531in}{5.520630in}}{\pgfqpoint{11.388921in}{5.510031in}}{\pgfqpoint{11.396735in}{5.502217in}}%
\pgfpathcurveto{\pgfqpoint{11.404549in}{5.494404in}}{\pgfqpoint{11.415148in}{5.490013in}}{\pgfqpoint{11.426198in}{5.490013in}}%
\pgfpathlineto{\pgfqpoint{11.426198in}{5.490013in}}%
\pgfpathclose%
\pgfusepath{stroke}%
\end{pgfscope}%
\begin{pgfscope}%
\pgfpathrectangle{\pgfqpoint{7.394209in}{0.375000in}}{\pgfqpoint{6.356833in}{5.175000in}}%
\pgfusepath{clip}%
\pgfsetbuttcap%
\pgfsetroundjoin%
\pgfsetlinewidth{1.003750pt}%
\definecolor{currentstroke}{rgb}{1.000000,0.000000,0.000000}%
\pgfsetstrokecolor{currentstroke}%
\pgfsetdash{}{0pt}%
\pgfpathmoveto{\pgfqpoint{10.816465in}{5.504464in}}%
\pgfpathcurveto{\pgfqpoint{10.827515in}{5.504464in}}{\pgfqpoint{10.838114in}{5.508854in}}{\pgfqpoint{10.845928in}{5.516668in}}%
\pgfpathcurveto{\pgfqpoint{10.853741in}{5.524481in}}{\pgfqpoint{10.858132in}{5.535080in}}{\pgfqpoint{10.858132in}{5.546130in}}%
\pgfpathcurveto{\pgfqpoint{10.858132in}{5.557180in}}{\pgfqpoint{10.853741in}{5.567780in}}{\pgfqpoint{10.845928in}{5.575593in}}%
\pgfpathcurveto{\pgfqpoint{10.838114in}{5.583407in}}{\pgfqpoint{10.827515in}{5.587797in}}{\pgfqpoint{10.816465in}{5.587797in}}%
\pgfpathcurveto{\pgfqpoint{10.805415in}{5.587797in}}{\pgfqpoint{10.794816in}{5.583407in}}{\pgfqpoint{10.787002in}{5.575593in}}%
\pgfpathcurveto{\pgfqpoint{10.779189in}{5.567780in}}{\pgfqpoint{10.774798in}{5.557180in}}{\pgfqpoint{10.774798in}{5.546130in}}%
\pgfpathcurveto{\pgfqpoint{10.774798in}{5.535080in}}{\pgfqpoint{10.779189in}{5.524481in}}{\pgfqpoint{10.787002in}{5.516668in}}%
\pgfpathcurveto{\pgfqpoint{10.794816in}{5.508854in}}{\pgfqpoint{10.805415in}{5.504464in}}{\pgfqpoint{10.816465in}{5.504464in}}%
\pgfpathlineto{\pgfqpoint{10.816465in}{5.504464in}}%
\pgfpathclose%
\pgfusepath{stroke}%
\end{pgfscope}%
\begin{pgfscope}%
\pgfpathrectangle{\pgfqpoint{7.394209in}{0.375000in}}{\pgfqpoint{6.356833in}{5.175000in}}%
\pgfusepath{clip}%
\pgfsetbuttcap%
\pgfsetroundjoin%
\pgfsetlinewidth{1.003750pt}%
\definecolor{currentstroke}{rgb}{1.000000,0.000000,0.000000}%
\pgfsetstrokecolor{currentstroke}%
\pgfsetdash{}{0pt}%
\pgfpathmoveto{\pgfqpoint{9.165190in}{2.653773in}}%
\pgfpathcurveto{\pgfqpoint{9.176240in}{2.653773in}}{\pgfqpoint{9.186839in}{2.658163in}}{\pgfqpoint{9.194653in}{2.665977in}}%
\pgfpathcurveto{\pgfqpoint{9.202466in}{2.673791in}}{\pgfqpoint{9.206856in}{2.684390in}}{\pgfqpoint{9.206856in}{2.695440in}}%
\pgfpathcurveto{\pgfqpoint{9.206856in}{2.706490in}}{\pgfqpoint{9.202466in}{2.717089in}}{\pgfqpoint{9.194653in}{2.724903in}}%
\pgfpathcurveto{\pgfqpoint{9.186839in}{2.732716in}}{\pgfqpoint{9.176240in}{2.737106in}}{\pgfqpoint{9.165190in}{2.737106in}}%
\pgfpathcurveto{\pgfqpoint{9.154140in}{2.737106in}}{\pgfqpoint{9.143541in}{2.732716in}}{\pgfqpoint{9.135727in}{2.724903in}}%
\pgfpathcurveto{\pgfqpoint{9.127913in}{2.717089in}}{\pgfqpoint{9.123523in}{2.706490in}}{\pgfqpoint{9.123523in}{2.695440in}}%
\pgfpathcurveto{\pgfqpoint{9.123523in}{2.684390in}}{\pgfqpoint{9.127913in}{2.673791in}}{\pgfqpoint{9.135727in}{2.665977in}}%
\pgfpathcurveto{\pgfqpoint{9.143541in}{2.658163in}}{\pgfqpoint{9.154140in}{2.653773in}}{\pgfqpoint{9.165190in}{2.653773in}}%
\pgfpathlineto{\pgfqpoint{9.165190in}{2.653773in}}%
\pgfpathclose%
\pgfusepath{stroke}%
\end{pgfscope}%
\begin{pgfscope}%
\pgfpathrectangle{\pgfqpoint{7.394209in}{0.375000in}}{\pgfqpoint{6.356833in}{5.175000in}}%
\pgfusepath{clip}%
\pgfsetbuttcap%
\pgfsetroundjoin%
\pgfsetlinewidth{1.003750pt}%
\definecolor{currentstroke}{rgb}{1.000000,0.000000,0.000000}%
\pgfsetstrokecolor{currentstroke}%
\pgfsetdash{}{0pt}%
\pgfpathmoveto{\pgfqpoint{11.941168in}{5.507138in}}%
\pgfpathcurveto{\pgfqpoint{11.952218in}{5.507138in}}{\pgfqpoint{11.962817in}{5.511529in}}{\pgfqpoint{11.970631in}{5.519342in}}%
\pgfpathcurveto{\pgfqpoint{11.978444in}{5.527156in}}{\pgfqpoint{11.982835in}{5.537755in}}{\pgfqpoint{11.982835in}{5.548805in}}%
\pgfpathcurveto{\pgfqpoint{11.982835in}{5.559855in}}{\pgfqpoint{11.978444in}{5.570454in}}{\pgfqpoint{11.970631in}{5.578268in}}%
\pgfpathcurveto{\pgfqpoint{11.962817in}{5.586082in}}{\pgfqpoint{11.952218in}{5.590472in}}{\pgfqpoint{11.941168in}{5.590472in}}%
\pgfpathcurveto{\pgfqpoint{11.930118in}{5.590472in}}{\pgfqpoint{11.919519in}{5.586082in}}{\pgfqpoint{11.911705in}{5.578268in}}%
\pgfpathcurveto{\pgfqpoint{11.903892in}{5.570454in}}{\pgfqpoint{11.899501in}{5.559855in}}{\pgfqpoint{11.899501in}{5.548805in}}%
\pgfpathcurveto{\pgfqpoint{11.899501in}{5.537755in}}{\pgfqpoint{11.903892in}{5.527156in}}{\pgfqpoint{11.911705in}{5.519342in}}%
\pgfpathcurveto{\pgfqpoint{11.919519in}{5.511529in}}{\pgfqpoint{11.930118in}{5.507138in}}{\pgfqpoint{11.941168in}{5.507138in}}%
\pgfpathlineto{\pgfqpoint{11.941168in}{5.507138in}}%
\pgfpathclose%
\pgfusepath{stroke}%
\end{pgfscope}%
\begin{pgfscope}%
\pgfpathrectangle{\pgfqpoint{7.394209in}{0.375000in}}{\pgfqpoint{6.356833in}{5.175000in}}%
\pgfusepath{clip}%
\pgfsetbuttcap%
\pgfsetroundjoin%
\pgfsetlinewidth{1.003750pt}%
\definecolor{currentstroke}{rgb}{1.000000,0.000000,0.000000}%
\pgfsetstrokecolor{currentstroke}%
\pgfsetdash{}{0pt}%
\pgfpathmoveto{\pgfqpoint{11.029754in}{5.491716in}}%
\pgfpathcurveto{\pgfqpoint{11.040804in}{5.491716in}}{\pgfqpoint{11.051403in}{5.496106in}}{\pgfqpoint{11.059217in}{5.503920in}}%
\pgfpathcurveto{\pgfqpoint{11.067031in}{5.511733in}}{\pgfqpoint{11.071421in}{5.522332in}}{\pgfqpoint{11.071421in}{5.533382in}}%
\pgfpathcurveto{\pgfqpoint{11.071421in}{5.544432in}}{\pgfqpoint{11.067031in}{5.555032in}}{\pgfqpoint{11.059217in}{5.562845in}}%
\pgfpathcurveto{\pgfqpoint{11.051403in}{5.570659in}}{\pgfqpoint{11.040804in}{5.575049in}}{\pgfqpoint{11.029754in}{5.575049in}}%
\pgfpathcurveto{\pgfqpoint{11.018704in}{5.575049in}}{\pgfqpoint{11.008105in}{5.570659in}}{\pgfqpoint{11.000291in}{5.562845in}}%
\pgfpathcurveto{\pgfqpoint{10.992478in}{5.555032in}}{\pgfqpoint{10.988088in}{5.544432in}}{\pgfqpoint{10.988088in}{5.533382in}}%
\pgfpathcurveto{\pgfqpoint{10.988088in}{5.522332in}}{\pgfqpoint{10.992478in}{5.511733in}}{\pgfqpoint{11.000291in}{5.503920in}}%
\pgfpathcurveto{\pgfqpoint{11.008105in}{5.496106in}}{\pgfqpoint{11.018704in}{5.491716in}}{\pgfqpoint{11.029754in}{5.491716in}}%
\pgfpathlineto{\pgfqpoint{11.029754in}{5.491716in}}%
\pgfpathclose%
\pgfusepath{stroke}%
\end{pgfscope}%
\begin{pgfscope}%
\pgfpathrectangle{\pgfqpoint{7.394209in}{0.375000in}}{\pgfqpoint{6.356833in}{5.175000in}}%
\pgfusepath{clip}%
\pgfsetbuttcap%
\pgfsetroundjoin%
\pgfsetlinewidth{1.003750pt}%
\definecolor{currentstroke}{rgb}{1.000000,0.000000,0.000000}%
\pgfsetstrokecolor{currentstroke}%
\pgfsetdash{}{0pt}%
\pgfpathmoveto{\pgfqpoint{8.658904in}{1.937732in}}%
\pgfpathcurveto{\pgfqpoint{8.669954in}{1.937732in}}{\pgfqpoint{8.680553in}{1.942122in}}{\pgfqpoint{8.688367in}{1.949936in}}%
\pgfpathcurveto{\pgfqpoint{8.696180in}{1.957749in}}{\pgfqpoint{8.700570in}{1.968348in}}{\pgfqpoint{8.700570in}{1.979399in}}%
\pgfpathcurveto{\pgfqpoint{8.700570in}{1.990449in}}{\pgfqpoint{8.696180in}{2.001048in}}{\pgfqpoint{8.688367in}{2.008861in}}%
\pgfpathcurveto{\pgfqpoint{8.680553in}{2.016675in}}{\pgfqpoint{8.669954in}{2.021065in}}{\pgfqpoint{8.658904in}{2.021065in}}%
\pgfpathcurveto{\pgfqpoint{8.647854in}{2.021065in}}{\pgfqpoint{8.637255in}{2.016675in}}{\pgfqpoint{8.629441in}{2.008861in}}%
\pgfpathcurveto{\pgfqpoint{8.621627in}{2.001048in}}{\pgfqpoint{8.617237in}{1.990449in}}{\pgfqpoint{8.617237in}{1.979399in}}%
\pgfpathcurveto{\pgfqpoint{8.617237in}{1.968348in}}{\pgfqpoint{8.621627in}{1.957749in}}{\pgfqpoint{8.629441in}{1.949936in}}%
\pgfpathcurveto{\pgfqpoint{8.637255in}{1.942122in}}{\pgfqpoint{8.647854in}{1.937732in}}{\pgfqpoint{8.658904in}{1.937732in}}%
\pgfpathlineto{\pgfqpoint{8.658904in}{1.937732in}}%
\pgfpathclose%
\pgfusepath{stroke}%
\end{pgfscope}%
\begin{pgfscope}%
\pgfpathrectangle{\pgfqpoint{7.394209in}{0.375000in}}{\pgfqpoint{6.356833in}{5.175000in}}%
\pgfusepath{clip}%
\pgfsetbuttcap%
\pgfsetroundjoin%
\pgfsetlinewidth{1.003750pt}%
\definecolor{currentstroke}{rgb}{1.000000,0.000000,0.000000}%
\pgfsetstrokecolor{currentstroke}%
\pgfsetdash{}{0pt}%
\pgfpathmoveto{\pgfqpoint{8.732054in}{2.030065in}}%
\pgfpathcurveto{\pgfqpoint{8.743104in}{2.030065in}}{\pgfqpoint{8.753703in}{2.034455in}}{\pgfqpoint{8.761517in}{2.042269in}}%
\pgfpathcurveto{\pgfqpoint{8.769330in}{2.050083in}}{\pgfqpoint{8.773720in}{2.060682in}}{\pgfqpoint{8.773720in}{2.071732in}}%
\pgfpathcurveto{\pgfqpoint{8.773720in}{2.082782in}}{\pgfqpoint{8.769330in}{2.093381in}}{\pgfqpoint{8.761517in}{2.101194in}}%
\pgfpathcurveto{\pgfqpoint{8.753703in}{2.109008in}}{\pgfqpoint{8.743104in}{2.113398in}}{\pgfqpoint{8.732054in}{2.113398in}}%
\pgfpathcurveto{\pgfqpoint{8.721004in}{2.113398in}}{\pgfqpoint{8.710405in}{2.109008in}}{\pgfqpoint{8.702591in}{2.101194in}}%
\pgfpathcurveto{\pgfqpoint{8.694777in}{2.093381in}}{\pgfqpoint{8.690387in}{2.082782in}}{\pgfqpoint{8.690387in}{2.071732in}}%
\pgfpathcurveto{\pgfqpoint{8.690387in}{2.060682in}}{\pgfqpoint{8.694777in}{2.050083in}}{\pgfqpoint{8.702591in}{2.042269in}}%
\pgfpathcurveto{\pgfqpoint{8.710405in}{2.034455in}}{\pgfqpoint{8.721004in}{2.030065in}}{\pgfqpoint{8.732054in}{2.030065in}}%
\pgfpathlineto{\pgfqpoint{8.732054in}{2.030065in}}%
\pgfpathclose%
\pgfusepath{stroke}%
\end{pgfscope}%
\begin{pgfscope}%
\pgfpathrectangle{\pgfqpoint{7.394209in}{0.375000in}}{\pgfqpoint{6.356833in}{5.175000in}}%
\pgfusepath{clip}%
\pgfsetbuttcap%
\pgfsetroundjoin%
\pgfsetlinewidth{1.003750pt}%
\definecolor{currentstroke}{rgb}{1.000000,0.000000,0.000000}%
\pgfsetstrokecolor{currentstroke}%
\pgfsetdash{}{0pt}%
\pgfpathmoveto{\pgfqpoint{11.044655in}{5.349644in}}%
\pgfpathcurveto{\pgfqpoint{11.055705in}{5.349644in}}{\pgfqpoint{11.066304in}{5.354034in}}{\pgfqpoint{11.074118in}{5.361848in}}%
\pgfpathcurveto{\pgfqpoint{11.081932in}{5.369662in}}{\pgfqpoint{11.086322in}{5.380261in}}{\pgfqpoint{11.086322in}{5.391311in}}%
\pgfpathcurveto{\pgfqpoint{11.086322in}{5.402361in}}{\pgfqpoint{11.081932in}{5.412960in}}{\pgfqpoint{11.074118in}{5.420773in}}%
\pgfpathcurveto{\pgfqpoint{11.066304in}{5.428587in}}{\pgfqpoint{11.055705in}{5.432977in}}{\pgfqpoint{11.044655in}{5.432977in}}%
\pgfpathcurveto{\pgfqpoint{11.033605in}{5.432977in}}{\pgfqpoint{11.023006in}{5.428587in}}{\pgfqpoint{11.015193in}{5.420773in}}%
\pgfpathcurveto{\pgfqpoint{11.007379in}{5.412960in}}{\pgfqpoint{11.002989in}{5.402361in}}{\pgfqpoint{11.002989in}{5.391311in}}%
\pgfpathcurveto{\pgfqpoint{11.002989in}{5.380261in}}{\pgfqpoint{11.007379in}{5.369662in}}{\pgfqpoint{11.015193in}{5.361848in}}%
\pgfpathcurveto{\pgfqpoint{11.023006in}{5.354034in}}{\pgfqpoint{11.033605in}{5.349644in}}{\pgfqpoint{11.044655in}{5.349644in}}%
\pgfpathlineto{\pgfqpoint{11.044655in}{5.349644in}}%
\pgfpathclose%
\pgfusepath{stroke}%
\end{pgfscope}%
\begin{pgfscope}%
\pgfpathrectangle{\pgfqpoint{7.394209in}{0.375000in}}{\pgfqpoint{6.356833in}{5.175000in}}%
\pgfusepath{clip}%
\pgfsetbuttcap%
\pgfsetroundjoin%
\pgfsetlinewidth{1.003750pt}%
\definecolor{currentstroke}{rgb}{1.000000,0.000000,0.000000}%
\pgfsetstrokecolor{currentstroke}%
\pgfsetdash{}{0pt}%
\pgfpathmoveto{\pgfqpoint{8.420651in}{1.422035in}}%
\pgfpathcurveto{\pgfqpoint{8.431701in}{1.422035in}}{\pgfqpoint{8.442300in}{1.426425in}}{\pgfqpoint{8.450114in}{1.434239in}}%
\pgfpathcurveto{\pgfqpoint{8.457928in}{1.442053in}}{\pgfqpoint{8.462318in}{1.452652in}}{\pgfqpoint{8.462318in}{1.463702in}}%
\pgfpathcurveto{\pgfqpoint{8.462318in}{1.474752in}}{\pgfqpoint{8.457928in}{1.485351in}}{\pgfqpoint{8.450114in}{1.493165in}}%
\pgfpathcurveto{\pgfqpoint{8.442300in}{1.500978in}}{\pgfqpoint{8.431701in}{1.505369in}}{\pgfqpoint{8.420651in}{1.505369in}}%
\pgfpathcurveto{\pgfqpoint{8.409601in}{1.505369in}}{\pgfqpoint{8.399002in}{1.500978in}}{\pgfqpoint{8.391188in}{1.493165in}}%
\pgfpathcurveto{\pgfqpoint{8.383375in}{1.485351in}}{\pgfqpoint{8.378985in}{1.474752in}}{\pgfqpoint{8.378985in}{1.463702in}}%
\pgfpathcurveto{\pgfqpoint{8.378985in}{1.452652in}}{\pgfqpoint{8.383375in}{1.442053in}}{\pgfqpoint{8.391188in}{1.434239in}}%
\pgfpathcurveto{\pgfqpoint{8.399002in}{1.426425in}}{\pgfqpoint{8.409601in}{1.422035in}}{\pgfqpoint{8.420651in}{1.422035in}}%
\pgfpathlineto{\pgfqpoint{8.420651in}{1.422035in}}%
\pgfpathclose%
\pgfusepath{stroke}%
\end{pgfscope}%
\begin{pgfscope}%
\pgfpathrectangle{\pgfqpoint{7.394209in}{0.375000in}}{\pgfqpoint{6.356833in}{5.175000in}}%
\pgfusepath{clip}%
\pgfsetbuttcap%
\pgfsetroundjoin%
\pgfsetlinewidth{1.003750pt}%
\definecolor{currentstroke}{rgb}{1.000000,0.000000,0.000000}%
\pgfsetstrokecolor{currentstroke}%
\pgfsetdash{}{0pt}%
\pgfpathmoveto{\pgfqpoint{7.604295in}{0.450054in}}%
\pgfpathcurveto{\pgfqpoint{7.615345in}{0.450054in}}{\pgfqpoint{7.625944in}{0.454444in}}{\pgfqpoint{7.633758in}{0.462258in}}%
\pgfpathcurveto{\pgfqpoint{7.641571in}{0.470071in}}{\pgfqpoint{7.645962in}{0.480670in}}{\pgfqpoint{7.645962in}{0.491721in}}%
\pgfpathcurveto{\pgfqpoint{7.645962in}{0.502771in}}{\pgfqpoint{7.641571in}{0.513370in}}{\pgfqpoint{7.633758in}{0.521183in}}%
\pgfpathcurveto{\pgfqpoint{7.625944in}{0.528997in}}{\pgfqpoint{7.615345in}{0.533387in}}{\pgfqpoint{7.604295in}{0.533387in}}%
\pgfpathcurveto{\pgfqpoint{7.593245in}{0.533387in}}{\pgfqpoint{7.582646in}{0.528997in}}{\pgfqpoint{7.574832in}{0.521183in}}%
\pgfpathcurveto{\pgfqpoint{7.567019in}{0.513370in}}{\pgfqpoint{7.562628in}{0.502771in}}{\pgfqpoint{7.562628in}{0.491721in}}%
\pgfpathcurveto{\pgfqpoint{7.562628in}{0.480670in}}{\pgfqpoint{7.567019in}{0.470071in}}{\pgfqpoint{7.574832in}{0.462258in}}%
\pgfpathcurveto{\pgfqpoint{7.582646in}{0.454444in}}{\pgfqpoint{7.593245in}{0.450054in}}{\pgfqpoint{7.604295in}{0.450054in}}%
\pgfpathlineto{\pgfqpoint{7.604295in}{0.450054in}}%
\pgfpathclose%
\pgfusepath{stroke}%
\end{pgfscope}%
\begin{pgfscope}%
\pgfpathrectangle{\pgfqpoint{7.394209in}{0.375000in}}{\pgfqpoint{6.356833in}{5.175000in}}%
\pgfusepath{clip}%
\pgfsetbuttcap%
\pgfsetroundjoin%
\definecolor{currentfill}{rgb}{0.121569,0.466667,0.705882}%
\pgfsetfillcolor{currentfill}%
\pgfsetlinewidth{1.003750pt}%
\definecolor{currentstroke}{rgb}{0.121569,0.466667,0.705882}%
\pgfsetstrokecolor{currentstroke}%
\pgfsetdash{}{0pt}%
\pgfsys@defobject{currentmarker}{\pgfqpoint{-0.069444in}{-0.069444in}}{\pgfqpoint{0.069444in}{0.069444in}}{%
\pgfpathmoveto{\pgfqpoint{0.000000in}{-0.069444in}}%
\pgfpathcurveto{\pgfqpoint{0.018417in}{-0.069444in}}{\pgfqpoint{0.036082in}{-0.062127in}}{\pgfqpoint{0.049105in}{-0.049105in}}%
\pgfpathcurveto{\pgfqpoint{0.062127in}{-0.036082in}}{\pgfqpoint{0.069444in}{-0.018417in}}{\pgfqpoint{0.069444in}{0.000000in}}%
\pgfpathcurveto{\pgfqpoint{0.069444in}{0.018417in}}{\pgfqpoint{0.062127in}{0.036082in}}{\pgfqpoint{0.049105in}{0.049105in}}%
\pgfpathcurveto{\pgfqpoint{0.036082in}{0.062127in}}{\pgfqpoint{0.018417in}{0.069444in}}{\pgfqpoint{0.000000in}{0.069444in}}%
\pgfpathcurveto{\pgfqpoint{-0.018417in}{0.069444in}}{\pgfqpoint{-0.036082in}{0.062127in}}{\pgfqpoint{-0.049105in}{0.049105in}}%
\pgfpathcurveto{\pgfqpoint{-0.062127in}{0.036082in}}{\pgfqpoint{-0.069444in}{0.018417in}}{\pgfqpoint{-0.069444in}{0.000000in}}%
\pgfpathcurveto{\pgfqpoint{-0.069444in}{-0.018417in}}{\pgfqpoint{-0.062127in}{-0.036082in}}{\pgfqpoint{-0.049105in}{-0.049105in}}%
\pgfpathcurveto{\pgfqpoint{-0.036082in}{-0.062127in}}{\pgfqpoint{-0.018417in}{-0.069444in}}{\pgfqpoint{0.000000in}{-0.069444in}}%
\pgfpathlineto{\pgfqpoint{0.000000in}{-0.069444in}}%
\pgfpathclose%
\pgfusepath{stroke,fill}%
}%
\begin{pgfscope}%
\pgfsys@transformshift{8.302414in}{2.088502in}%
\pgfsys@useobject{currentmarker}{}%
\end{pgfscope}%
\end{pgfscope}%
\begin{pgfscope}%
\pgfpathrectangle{\pgfqpoint{7.394209in}{0.375000in}}{\pgfqpoint{6.356833in}{5.175000in}}%
\pgfusepath{clip}%
\pgfsetbuttcap%
\pgfsetroundjoin%
\definecolor{currentfill}{rgb}{0.172549,0.627451,0.172549}%
\pgfsetfillcolor{currentfill}%
\pgfsetlinewidth{1.003750pt}%
\definecolor{currentstroke}{rgb}{0.172549,0.627451,0.172549}%
\pgfsetstrokecolor{currentstroke}%
\pgfsetdash{}{0pt}%
\pgfsys@defobject{currentmarker}{\pgfqpoint{-0.069444in}{-0.069444in}}{\pgfqpoint{0.069444in}{0.069444in}}{%
\pgfpathmoveto{\pgfqpoint{0.000000in}{-0.069444in}}%
\pgfpathcurveto{\pgfqpoint{0.018417in}{-0.069444in}}{\pgfqpoint{0.036082in}{-0.062127in}}{\pgfqpoint{0.049105in}{-0.049105in}}%
\pgfpathcurveto{\pgfqpoint{0.062127in}{-0.036082in}}{\pgfqpoint{0.069444in}{-0.018417in}}{\pgfqpoint{0.069444in}{0.000000in}}%
\pgfpathcurveto{\pgfqpoint{0.069444in}{0.018417in}}{\pgfqpoint{0.062127in}{0.036082in}}{\pgfqpoint{0.049105in}{0.049105in}}%
\pgfpathcurveto{\pgfqpoint{0.036082in}{0.062127in}}{\pgfqpoint{0.018417in}{0.069444in}}{\pgfqpoint{0.000000in}{0.069444in}}%
\pgfpathcurveto{\pgfqpoint{-0.018417in}{0.069444in}}{\pgfqpoint{-0.036082in}{0.062127in}}{\pgfqpoint{-0.049105in}{0.049105in}}%
\pgfpathcurveto{\pgfqpoint{-0.062127in}{0.036082in}}{\pgfqpoint{-0.069444in}{0.018417in}}{\pgfqpoint{-0.069444in}{0.000000in}}%
\pgfpathcurveto{\pgfqpoint{-0.069444in}{-0.018417in}}{\pgfqpoint{-0.062127in}{-0.036082in}}{\pgfqpoint{-0.049105in}{-0.049105in}}%
\pgfpathcurveto{\pgfqpoint{-0.036082in}{-0.062127in}}{\pgfqpoint{-0.018417in}{-0.069444in}}{\pgfqpoint{0.000000in}{-0.069444in}}%
\pgfpathlineto{\pgfqpoint{0.000000in}{-0.069444in}}%
\pgfpathclose%
\pgfusepath{stroke,fill}%
}%
\begin{pgfscope}%
\pgfsys@transformshift{13.694003in}{5.523664in}%
\pgfsys@useobject{currentmarker}{}%
\end{pgfscope}%
\end{pgfscope}%
\begin{pgfscope}%
\pgfpathrectangle{\pgfqpoint{7.394209in}{0.375000in}}{\pgfqpoint{6.356833in}{5.175000in}}%
\pgfusepath{clip}%
\pgfsetbuttcap%
\pgfsetroundjoin%
\definecolor{currentfill}{rgb}{0.580392,0.403922,0.741176}%
\pgfsetfillcolor{currentfill}%
\pgfsetlinewidth{1.003750pt}%
\definecolor{currentstroke}{rgb}{0.580392,0.403922,0.741176}%
\pgfsetstrokecolor{currentstroke}%
\pgfsetdash{}{0pt}%
\pgfsys@defobject{currentmarker}{\pgfqpoint{-0.069444in}{-0.069444in}}{\pgfqpoint{0.069444in}{0.069444in}}{%
\pgfpathmoveto{\pgfqpoint{0.000000in}{-0.069444in}}%
\pgfpathcurveto{\pgfqpoint{0.018417in}{-0.069444in}}{\pgfqpoint{0.036082in}{-0.062127in}}{\pgfqpoint{0.049105in}{-0.049105in}}%
\pgfpathcurveto{\pgfqpoint{0.062127in}{-0.036082in}}{\pgfqpoint{0.069444in}{-0.018417in}}{\pgfqpoint{0.069444in}{0.000000in}}%
\pgfpathcurveto{\pgfqpoint{0.069444in}{0.018417in}}{\pgfqpoint{0.062127in}{0.036082in}}{\pgfqpoint{0.049105in}{0.049105in}}%
\pgfpathcurveto{\pgfqpoint{0.036082in}{0.062127in}}{\pgfqpoint{0.018417in}{0.069444in}}{\pgfqpoint{0.000000in}{0.069444in}}%
\pgfpathcurveto{\pgfqpoint{-0.018417in}{0.069444in}}{\pgfqpoint{-0.036082in}{0.062127in}}{\pgfqpoint{-0.049105in}{0.049105in}}%
\pgfpathcurveto{\pgfqpoint{-0.062127in}{0.036082in}}{\pgfqpoint{-0.069444in}{0.018417in}}{\pgfqpoint{-0.069444in}{0.000000in}}%
\pgfpathcurveto{\pgfqpoint{-0.069444in}{-0.018417in}}{\pgfqpoint{-0.062127in}{-0.036082in}}{\pgfqpoint{-0.049105in}{-0.049105in}}%
\pgfpathcurveto{\pgfqpoint{-0.036082in}{-0.062127in}}{\pgfqpoint{-0.018417in}{-0.069444in}}{\pgfqpoint{0.000000in}{-0.069444in}}%
\pgfpathlineto{\pgfqpoint{0.000000in}{-0.069444in}}%
\pgfpathclose%
\pgfusepath{stroke,fill}%
}%
\begin{pgfscope}%
\pgfsys@transformshift{7.402870in}{0.538814in}%
\pgfsys@useobject{currentmarker}{}%
\end{pgfscope}%
\end{pgfscope}%
\begin{pgfscope}%
\pgfpathrectangle{\pgfqpoint{7.394209in}{0.375000in}}{\pgfqpoint{6.356833in}{5.175000in}}%
\pgfusepath{clip}%
\pgfsetbuttcap%
\pgfsetroundjoin%
\definecolor{currentfill}{rgb}{0.890196,0.466667,0.760784}%
\pgfsetfillcolor{currentfill}%
\pgfsetlinewidth{1.003750pt}%
\definecolor{currentstroke}{rgb}{0.890196,0.466667,0.760784}%
\pgfsetstrokecolor{currentstroke}%
\pgfsetdash{}{0pt}%
\pgfsys@defobject{currentmarker}{\pgfqpoint{-0.069444in}{-0.069444in}}{\pgfqpoint{0.069444in}{0.069444in}}{%
\pgfpathmoveto{\pgfqpoint{0.000000in}{-0.069444in}}%
\pgfpathcurveto{\pgfqpoint{0.018417in}{-0.069444in}}{\pgfqpoint{0.036082in}{-0.062127in}}{\pgfqpoint{0.049105in}{-0.049105in}}%
\pgfpathcurveto{\pgfqpoint{0.062127in}{-0.036082in}}{\pgfqpoint{0.069444in}{-0.018417in}}{\pgfqpoint{0.069444in}{0.000000in}}%
\pgfpathcurveto{\pgfqpoint{0.069444in}{0.018417in}}{\pgfqpoint{0.062127in}{0.036082in}}{\pgfqpoint{0.049105in}{0.049105in}}%
\pgfpathcurveto{\pgfqpoint{0.036082in}{0.062127in}}{\pgfqpoint{0.018417in}{0.069444in}}{\pgfqpoint{0.000000in}{0.069444in}}%
\pgfpathcurveto{\pgfqpoint{-0.018417in}{0.069444in}}{\pgfqpoint{-0.036082in}{0.062127in}}{\pgfqpoint{-0.049105in}{0.049105in}}%
\pgfpathcurveto{\pgfqpoint{-0.062127in}{0.036082in}}{\pgfqpoint{-0.069444in}{0.018417in}}{\pgfqpoint{-0.069444in}{0.000000in}}%
\pgfpathcurveto{\pgfqpoint{-0.069444in}{-0.018417in}}{\pgfqpoint{-0.062127in}{-0.036082in}}{\pgfqpoint{-0.049105in}{-0.049105in}}%
\pgfpathcurveto{\pgfqpoint{-0.036082in}{-0.062127in}}{\pgfqpoint{-0.018417in}{-0.069444in}}{\pgfqpoint{0.000000in}{-0.069444in}}%
\pgfpathlineto{\pgfqpoint{0.000000in}{-0.069444in}}%
\pgfpathclose%
\pgfusepath{stroke,fill}%
}%
\begin{pgfscope}%
\pgfsys@transformshift{13.749892in}{5.115406in}%
\pgfsys@useobject{currentmarker}{}%
\end{pgfscope}%
\end{pgfscope}%
\begin{pgfscope}%
\pgfpathrectangle{\pgfqpoint{7.394209in}{0.375000in}}{\pgfqpoint{6.356833in}{5.175000in}}%
\pgfusepath{clip}%
\pgfsetbuttcap%
\pgfsetroundjoin%
\definecolor{currentfill}{rgb}{0.737255,0.741176,0.133333}%
\pgfsetfillcolor{currentfill}%
\pgfsetlinewidth{1.003750pt}%
\definecolor{currentstroke}{rgb}{0.737255,0.741176,0.133333}%
\pgfsetstrokecolor{currentstroke}%
\pgfsetdash{}{0pt}%
\pgfsys@defobject{currentmarker}{\pgfqpoint{-0.069444in}{-0.069444in}}{\pgfqpoint{0.069444in}{0.069444in}}{%
\pgfpathmoveto{\pgfqpoint{0.000000in}{-0.069444in}}%
\pgfpathcurveto{\pgfqpoint{0.018417in}{-0.069444in}}{\pgfqpoint{0.036082in}{-0.062127in}}{\pgfqpoint{0.049105in}{-0.049105in}}%
\pgfpathcurveto{\pgfqpoint{0.062127in}{-0.036082in}}{\pgfqpoint{0.069444in}{-0.018417in}}{\pgfqpoint{0.069444in}{0.000000in}}%
\pgfpathcurveto{\pgfqpoint{0.069444in}{0.018417in}}{\pgfqpoint{0.062127in}{0.036082in}}{\pgfqpoint{0.049105in}{0.049105in}}%
\pgfpathcurveto{\pgfqpoint{0.036082in}{0.062127in}}{\pgfqpoint{0.018417in}{0.069444in}}{\pgfqpoint{0.000000in}{0.069444in}}%
\pgfpathcurveto{\pgfqpoint{-0.018417in}{0.069444in}}{\pgfqpoint{-0.036082in}{0.062127in}}{\pgfqpoint{-0.049105in}{0.049105in}}%
\pgfpathcurveto{\pgfqpoint{-0.062127in}{0.036082in}}{\pgfqpoint{-0.069444in}{0.018417in}}{\pgfqpoint{-0.069444in}{0.000000in}}%
\pgfpathcurveto{\pgfqpoint{-0.069444in}{-0.018417in}}{\pgfqpoint{-0.062127in}{-0.036082in}}{\pgfqpoint{-0.049105in}{-0.049105in}}%
\pgfpathcurveto{\pgfqpoint{-0.036082in}{-0.062127in}}{\pgfqpoint{-0.018417in}{-0.069444in}}{\pgfqpoint{0.000000in}{-0.069444in}}%
\pgfpathlineto{\pgfqpoint{0.000000in}{-0.069444in}}%
\pgfpathclose%
\pgfusepath{stroke,fill}%
}%
\begin{pgfscope}%
\pgfsys@transformshift{7.695239in}{3.867509in}%
\pgfsys@useobject{currentmarker}{}%
\end{pgfscope}%
\end{pgfscope}%
\begin{pgfscope}%
\pgfpathrectangle{\pgfqpoint{7.394209in}{0.375000in}}{\pgfqpoint{6.356833in}{5.175000in}}%
\pgfusepath{clip}%
\pgfsetbuttcap%
\pgfsetroundjoin%
\definecolor{currentfill}{rgb}{0.090196,0.745098,0.811765}%
\pgfsetfillcolor{currentfill}%
\pgfsetlinewidth{1.003750pt}%
\definecolor{currentstroke}{rgb}{0.090196,0.745098,0.811765}%
\pgfsetstrokecolor{currentstroke}%
\pgfsetdash{}{0pt}%
\pgfsys@defobject{currentmarker}{\pgfqpoint{-0.069444in}{-0.069444in}}{\pgfqpoint{0.069444in}{0.069444in}}{%
\pgfpathmoveto{\pgfqpoint{0.000000in}{-0.069444in}}%
\pgfpathcurveto{\pgfqpoint{0.018417in}{-0.069444in}}{\pgfqpoint{0.036082in}{-0.062127in}}{\pgfqpoint{0.049105in}{-0.049105in}}%
\pgfpathcurveto{\pgfqpoint{0.062127in}{-0.036082in}}{\pgfqpoint{0.069444in}{-0.018417in}}{\pgfqpoint{0.069444in}{0.000000in}}%
\pgfpathcurveto{\pgfqpoint{0.069444in}{0.018417in}}{\pgfqpoint{0.062127in}{0.036082in}}{\pgfqpoint{0.049105in}{0.049105in}}%
\pgfpathcurveto{\pgfqpoint{0.036082in}{0.062127in}}{\pgfqpoint{0.018417in}{0.069444in}}{\pgfqpoint{0.000000in}{0.069444in}}%
\pgfpathcurveto{\pgfqpoint{-0.018417in}{0.069444in}}{\pgfqpoint{-0.036082in}{0.062127in}}{\pgfqpoint{-0.049105in}{0.049105in}}%
\pgfpathcurveto{\pgfqpoint{-0.062127in}{0.036082in}}{\pgfqpoint{-0.069444in}{0.018417in}}{\pgfqpoint{-0.069444in}{0.000000in}}%
\pgfpathcurveto{\pgfqpoint{-0.069444in}{-0.018417in}}{\pgfqpoint{-0.062127in}{-0.036082in}}{\pgfqpoint{-0.049105in}{-0.049105in}}%
\pgfpathcurveto{\pgfqpoint{-0.036082in}{-0.062127in}}{\pgfqpoint{-0.018417in}{-0.069444in}}{\pgfqpoint{0.000000in}{-0.069444in}}%
\pgfpathlineto{\pgfqpoint{0.000000in}{-0.069444in}}%
\pgfpathclose%
\pgfusepath{stroke,fill}%
}%
\begin{pgfscope}%
\pgfsys@transformshift{13.687757in}{5.532387in}%
\pgfsys@useobject{currentmarker}{}%
\end{pgfscope}%
\end{pgfscope}%
\begin{pgfscope}%
\pgfpathrectangle{\pgfqpoint{7.394209in}{0.375000in}}{\pgfqpoint{6.356833in}{5.175000in}}%
\pgfusepath{clip}%
\pgfsetroundcap%
\pgfsetroundjoin%
\definecolor{currentfill}{rgb}{0.172549,0.627451,0.172549}%
\pgfsetfillcolor{currentfill}%
\pgfsetlinewidth{3.011250pt}%
\definecolor{currentstroke}{rgb}{0.172549,0.627451,0.172549}%
\pgfsetstrokecolor{currentstroke}%
\pgfsetdash{}{0pt}%
\pgfpathmoveto{\pgfqpoint{8.323980in}{2.106359in}}%
\pgfpathquadraticcurveto{\pgfqpoint{10.949487in}{3.779158in}}{\pgfqpoint{13.574994in}{5.451957in}}%
\pgfpathlineto{\pgfqpoint{13.561934in}{5.472455in}}%
\pgfpathquadraticcurveto{\pgfqpoint{13.616253in}{5.490595in}}{\pgfqpoint{13.670571in}{5.508735in}}%
\pgfpathquadraticcurveto{\pgfqpoint{13.631179in}{5.467168in}}{\pgfqpoint{13.591786in}{5.425601in}}%
\pgfpathlineto{\pgfqpoint{13.578726in}{5.446100in}}%
\pgfpathquadraticcurveto{\pgfqpoint{10.953219in}{3.773301in}}{\pgfqpoint{8.327711in}{2.100502in}}%
\pgfpathlineto{\pgfqpoint{8.323980in}{2.106359in}}%
\pgfpathlineto{\pgfqpoint{8.323980in}{2.106359in}}%
\pgfpathclose%
\pgfusepath{stroke,fill}%
\end{pgfscope}%
\begin{pgfscope}%
\pgfpathrectangle{\pgfqpoint{7.394209in}{0.375000in}}{\pgfqpoint{6.356833in}{5.175000in}}%
\pgfusepath{clip}%
\pgfsetroundcap%
\pgfsetroundjoin%
\definecolor{currentfill}{rgb}{0.580392,0.403922,0.741176}%
\pgfsetfillcolor{currentfill}%
\pgfsetlinewidth{3.011250pt}%
\definecolor{currentstroke}{rgb}{0.580392,0.403922,0.741176}%
\pgfsetstrokecolor{currentstroke}%
\pgfsetdash{}{0pt}%
\pgfpathmoveto{\pgfqpoint{13.674386in}{5.503690in}}%
\pgfpathquadraticcurveto{\pgfqpoint{10.594136in}{3.063020in}}{\pgfqpoint{7.513886in}{0.622349in}}%
\pgfpathlineto{\pgfqpoint{7.528980in}{0.603299in}}%
\pgfpathquadraticcurveto{\pgfqpoint{7.476810in}{0.579681in}}{\pgfqpoint{7.424639in}{0.556063in}}%
\pgfpathquadraticcurveto{\pgfqpoint{7.459559in}{0.601453in}}{\pgfqpoint{7.494478in}{0.646842in}}%
\pgfpathlineto{\pgfqpoint{7.509573in}{0.627792in}}%
\pgfpathquadraticcurveto{\pgfqpoint{10.589823in}{3.068463in}}{\pgfqpoint{13.670073in}{5.509133in}}%
\pgfpathlineto{\pgfqpoint{13.674386in}{5.503690in}}%
\pgfpathlineto{\pgfqpoint{13.674386in}{5.503690in}}%
\pgfpathclose%
\pgfusepath{stroke,fill}%
\end{pgfscope}%
\begin{pgfscope}%
\pgfpathrectangle{\pgfqpoint{7.394209in}{0.375000in}}{\pgfqpoint{6.356833in}{5.175000in}}%
\pgfusepath{clip}%
\pgfsetroundcap%
\pgfsetroundjoin%
\definecolor{currentfill}{rgb}{0.890196,0.466667,0.760784}%
\pgfsetfillcolor{currentfill}%
\pgfsetlinewidth{3.011250pt}%
\definecolor{currentstroke}{rgb}{0.890196,0.466667,0.760784}%
\pgfsetstrokecolor{currentstroke}%
\pgfsetdash{}{0pt}%
\pgfpathmoveto{\pgfqpoint{7.423374in}{0.557880in}}%
\pgfpathquadraticcurveto{\pgfqpoint{10.529293in}{2.797438in}}{\pgfqpoint{13.635213in}{5.036997in}}%
\pgfpathlineto{\pgfqpoint{13.620997in}{5.056712in}}%
\pgfpathquadraticcurveto{\pgfqpoint{13.674181in}{5.077937in}}{\pgfqpoint{13.727364in}{5.099162in}}%
\pgfpathquadraticcurveto{\pgfqpoint{13.690427in}{5.055406in}}{\pgfqpoint{13.653490in}{5.011649in}}%
\pgfpathlineto{\pgfqpoint{13.639274in}{5.031364in}}%
\pgfpathquadraticcurveto{\pgfqpoint{10.533355in}{2.791805in}}{\pgfqpoint{7.427436in}{0.552247in}}%
\pgfpathlineto{\pgfqpoint{7.423374in}{0.557880in}}%
\pgfpathlineto{\pgfqpoint{7.423374in}{0.557880in}}%
\pgfpathclose%
\pgfusepath{stroke,fill}%
\end{pgfscope}%
\begin{pgfscope}%
\pgfpathrectangle{\pgfqpoint{7.394209in}{0.375000in}}{\pgfqpoint{6.356833in}{5.175000in}}%
\pgfusepath{clip}%
\pgfsetroundcap%
\pgfsetroundjoin%
\definecolor{currentfill}{rgb}{0.737255,0.741176,0.133333}%
\pgfsetfillcolor{currentfill}%
\pgfsetlinewidth{3.011250pt}%
\definecolor{currentstroke}{rgb}{0.737255,0.741176,0.133333}%
\pgfsetstrokecolor{currentstroke}%
\pgfsetdash{}{0pt}%
\pgfpathmoveto{\pgfqpoint{13.723390in}{5.106399in}}%
\pgfpathquadraticcurveto{\pgfqpoint{10.777679in}{4.499271in}}{\pgfqpoint{7.831967in}{3.892144in}}%
\pgfpathlineto{\pgfqpoint{7.836873in}{3.868339in}}%
\pgfpathquadraticcurveto{\pgfqpoint{7.779659in}{3.870727in}}{\pgfqpoint{7.722445in}{3.873116in}}%
\pgfpathquadraticcurveto{\pgfqpoint{7.774052in}{3.897933in}}{\pgfqpoint{7.825659in}{3.922751in}}%
\pgfpathlineto{\pgfqpoint{7.830565in}{3.898945in}}%
\pgfpathquadraticcurveto{\pgfqpoint{10.776277in}{4.506073in}}{\pgfqpoint{13.721989in}{5.113200in}}%
\pgfpathlineto{\pgfqpoint{13.723390in}{5.106399in}}%
\pgfpathlineto{\pgfqpoint{13.723390in}{5.106399in}}%
\pgfpathclose%
\pgfusepath{stroke,fill}%
\end{pgfscope}%
\begin{pgfscope}%
\pgfpathrectangle{\pgfqpoint{7.394209in}{0.375000in}}{\pgfqpoint{6.356833in}{5.175000in}}%
\pgfusepath{clip}%
\pgfsetroundcap%
\pgfsetroundjoin%
\definecolor{currentfill}{rgb}{0.090196,0.745098,0.811765}%
\pgfsetfillcolor{currentfill}%
\pgfsetlinewidth{3.011250pt}%
\definecolor{currentstroke}{rgb}{0.090196,0.745098,0.811765}%
\pgfsetstrokecolor{currentstroke}%
\pgfsetdash{}{0pt}%
\pgfpathmoveto{\pgfqpoint{7.721072in}{3.878290in}}%
\pgfpathquadraticcurveto{\pgfqpoint{10.637039in}{4.688421in}}{\pgfqpoint{13.553006in}{5.498553in}}%
\pgfpathlineto{\pgfqpoint{13.546500in}{5.521972in}}%
\pgfpathquadraticcurveto{\pgfqpoint{13.603744in}{5.523461in}}{\pgfqpoint{13.660988in}{5.524950in}}%
\pgfpathquadraticcurveto{\pgfqpoint{13.611180in}{5.496697in}}{\pgfqpoint{13.561372in}{5.468444in}}%
\pgfpathlineto{\pgfqpoint{13.554865in}{5.491862in}}%
\pgfpathquadraticcurveto{\pgfqpoint{10.638898in}{4.681730in}}{\pgfqpoint{7.722931in}{3.871599in}}%
\pgfpathlineto{\pgfqpoint{7.721072in}{3.878290in}}%
\pgfpathlineto{\pgfqpoint{7.721072in}{3.878290in}}%
\pgfpathclose%
\pgfusepath{stroke,fill}%
\end{pgfscope}%
\begin{pgfscope}%
\pgfsetbuttcap%
\pgfsetroundjoin%
\definecolor{currentfill}{rgb}{0.000000,0.000000,0.000000}%
\pgfsetfillcolor{currentfill}%
\pgfsetlinewidth{0.803000pt}%
\definecolor{currentstroke}{rgb}{0.000000,0.000000,0.000000}%
\pgfsetstrokecolor{currentstroke}%
\pgfsetdash{}{0pt}%
\pgfsys@defobject{currentmarker}{\pgfqpoint{0.000000in}{-0.048611in}}{\pgfqpoint{0.000000in}{0.000000in}}{%
\pgfpathmoveto{\pgfqpoint{0.000000in}{0.000000in}}%
\pgfpathlineto{\pgfqpoint{0.000000in}{-0.048611in}}%
\pgfusepath{stroke,fill}%
}%
\begin{pgfscope}%
\pgfsys@transformshift{7.394209in}{0.375000in}%
\pgfsys@useobject{currentmarker}{}%
\end{pgfscope}%
\end{pgfscope}%
\begin{pgfscope}%
\definecolor{textcolor}{rgb}{0.000000,0.000000,0.000000}%
\pgfsetstrokecolor{textcolor}%
\pgfsetfillcolor{textcolor}%
\pgftext[x=7.394209in,y=0.277777in,,top]{\color{textcolor}{\rmfamily\fontsize{14.000000}{16.800000}\selectfont\catcode`\^=\active\def^{\ifmmode\sp\else\^{}\fi}\catcode`\%=\active\def%{\%}$\mathdefault{0}$}}%
\end{pgfscope}%
\begin{pgfscope}%
\pgfsetbuttcap%
\pgfsetroundjoin%
\definecolor{currentfill}{rgb}{0.000000,0.000000,0.000000}%
\pgfsetfillcolor{currentfill}%
\pgfsetlinewidth{0.803000pt}%
\definecolor{currentstroke}{rgb}{0.000000,0.000000,0.000000}%
\pgfsetstrokecolor{currentstroke}%
\pgfsetdash{}{0pt}%
\pgfsys@defobject{currentmarker}{\pgfqpoint{0.000000in}{-0.048611in}}{\pgfqpoint{0.000000in}{0.000000in}}{%
\pgfpathmoveto{\pgfqpoint{0.000000in}{0.000000in}}%
\pgfpathlineto{\pgfqpoint{0.000000in}{-0.048611in}}%
\pgfusepath{stroke,fill}%
}%
\begin{pgfscope}%
\pgfsys@transformshift{8.665576in}{0.375000in}%
\pgfsys@useobject{currentmarker}{}%
\end{pgfscope}%
\end{pgfscope}%
\begin{pgfscope}%
\definecolor{textcolor}{rgb}{0.000000,0.000000,0.000000}%
\pgfsetstrokecolor{textcolor}%
\pgfsetfillcolor{textcolor}%
\pgftext[x=8.665576in,y=0.277777in,,top]{\color{textcolor}{\rmfamily\fontsize{14.000000}{16.800000}\selectfont\catcode`\^=\active\def^{\ifmmode\sp\else\^{}\fi}\catcode`\%=\active\def%{\%}$\mathdefault{1}$}}%
\end{pgfscope}%
\begin{pgfscope}%
\pgfsetbuttcap%
\pgfsetroundjoin%
\definecolor{currentfill}{rgb}{0.000000,0.000000,0.000000}%
\pgfsetfillcolor{currentfill}%
\pgfsetlinewidth{0.803000pt}%
\definecolor{currentstroke}{rgb}{0.000000,0.000000,0.000000}%
\pgfsetstrokecolor{currentstroke}%
\pgfsetdash{}{0pt}%
\pgfsys@defobject{currentmarker}{\pgfqpoint{0.000000in}{-0.048611in}}{\pgfqpoint{0.000000in}{0.000000in}}{%
\pgfpathmoveto{\pgfqpoint{0.000000in}{0.000000in}}%
\pgfpathlineto{\pgfqpoint{0.000000in}{-0.048611in}}%
\pgfusepath{stroke,fill}%
}%
\begin{pgfscope}%
\pgfsys@transformshift{9.936943in}{0.375000in}%
\pgfsys@useobject{currentmarker}{}%
\end{pgfscope}%
\end{pgfscope}%
\begin{pgfscope}%
\definecolor{textcolor}{rgb}{0.000000,0.000000,0.000000}%
\pgfsetstrokecolor{textcolor}%
\pgfsetfillcolor{textcolor}%
\pgftext[x=9.936943in,y=0.277777in,,top]{\color{textcolor}{\rmfamily\fontsize{14.000000}{16.800000}\selectfont\catcode`\^=\active\def^{\ifmmode\sp\else\^{}\fi}\catcode`\%=\active\def%{\%}$\mathdefault{2}$}}%
\end{pgfscope}%
\begin{pgfscope}%
\pgfsetbuttcap%
\pgfsetroundjoin%
\definecolor{currentfill}{rgb}{0.000000,0.000000,0.000000}%
\pgfsetfillcolor{currentfill}%
\pgfsetlinewidth{0.803000pt}%
\definecolor{currentstroke}{rgb}{0.000000,0.000000,0.000000}%
\pgfsetstrokecolor{currentstroke}%
\pgfsetdash{}{0pt}%
\pgfsys@defobject{currentmarker}{\pgfqpoint{0.000000in}{-0.048611in}}{\pgfqpoint{0.000000in}{0.000000in}}{%
\pgfpathmoveto{\pgfqpoint{0.000000in}{0.000000in}}%
\pgfpathlineto{\pgfqpoint{0.000000in}{-0.048611in}}%
\pgfusepath{stroke,fill}%
}%
\begin{pgfscope}%
\pgfsys@transformshift{11.208309in}{0.375000in}%
\pgfsys@useobject{currentmarker}{}%
\end{pgfscope}%
\end{pgfscope}%
\begin{pgfscope}%
\definecolor{textcolor}{rgb}{0.000000,0.000000,0.000000}%
\pgfsetstrokecolor{textcolor}%
\pgfsetfillcolor{textcolor}%
\pgftext[x=11.208309in,y=0.277777in,,top]{\color{textcolor}{\rmfamily\fontsize{14.000000}{16.800000}\selectfont\catcode`\^=\active\def^{\ifmmode\sp\else\^{}\fi}\catcode`\%=\active\def%{\%}$\mathdefault{3}$}}%
\end{pgfscope}%
\begin{pgfscope}%
\pgfsetbuttcap%
\pgfsetroundjoin%
\definecolor{currentfill}{rgb}{0.000000,0.000000,0.000000}%
\pgfsetfillcolor{currentfill}%
\pgfsetlinewidth{0.803000pt}%
\definecolor{currentstroke}{rgb}{0.000000,0.000000,0.000000}%
\pgfsetstrokecolor{currentstroke}%
\pgfsetdash{}{0pt}%
\pgfsys@defobject{currentmarker}{\pgfqpoint{0.000000in}{-0.048611in}}{\pgfqpoint{0.000000in}{0.000000in}}{%
\pgfpathmoveto{\pgfqpoint{0.000000in}{0.000000in}}%
\pgfpathlineto{\pgfqpoint{0.000000in}{-0.048611in}}%
\pgfusepath{stroke,fill}%
}%
\begin{pgfscope}%
\pgfsys@transformshift{12.479676in}{0.375000in}%
\pgfsys@useobject{currentmarker}{}%
\end{pgfscope}%
\end{pgfscope}%
\begin{pgfscope}%
\definecolor{textcolor}{rgb}{0.000000,0.000000,0.000000}%
\pgfsetstrokecolor{textcolor}%
\pgfsetfillcolor{textcolor}%
\pgftext[x=12.479676in,y=0.277777in,,top]{\color{textcolor}{\rmfamily\fontsize{14.000000}{16.800000}\selectfont\catcode`\^=\active\def^{\ifmmode\sp\else\^{}\fi}\catcode`\%=\active\def%{\%}$\mathdefault{4}$}}%
\end{pgfscope}%
\begin{pgfscope}%
\pgfsetbuttcap%
\pgfsetroundjoin%
\definecolor{currentfill}{rgb}{0.000000,0.000000,0.000000}%
\pgfsetfillcolor{currentfill}%
\pgfsetlinewidth{0.803000pt}%
\definecolor{currentstroke}{rgb}{0.000000,0.000000,0.000000}%
\pgfsetstrokecolor{currentstroke}%
\pgfsetdash{}{0pt}%
\pgfsys@defobject{currentmarker}{\pgfqpoint{0.000000in}{-0.048611in}}{\pgfqpoint{0.000000in}{0.000000in}}{%
\pgfpathmoveto{\pgfqpoint{0.000000in}{0.000000in}}%
\pgfpathlineto{\pgfqpoint{0.000000in}{-0.048611in}}%
\pgfusepath{stroke,fill}%
}%
\begin{pgfscope}%
\pgfsys@transformshift{13.751042in}{0.375000in}%
\pgfsys@useobject{currentmarker}{}%
\end{pgfscope}%
\end{pgfscope}%
\begin{pgfscope}%
\definecolor{textcolor}{rgb}{0.000000,0.000000,0.000000}%
\pgfsetstrokecolor{textcolor}%
\pgfsetfillcolor{textcolor}%
\pgftext[x=13.751042in,y=0.277777in,,top]{\color{textcolor}{\rmfamily\fontsize{14.000000}{16.800000}\selectfont\catcode`\^=\active\def^{\ifmmode\sp\else\^{}\fi}\catcode`\%=\active\def%{\%}$\mathdefault{5}$}}%
\end{pgfscope}%
\begin{pgfscope}%
\pgfsetbuttcap%
\pgfsetroundjoin%
\definecolor{currentfill}{rgb}{0.000000,0.000000,0.000000}%
\pgfsetfillcolor{currentfill}%
\pgfsetlinewidth{0.803000pt}%
\definecolor{currentstroke}{rgb}{0.000000,0.000000,0.000000}%
\pgfsetstrokecolor{currentstroke}%
\pgfsetdash{}{0pt}%
\pgfsys@defobject{currentmarker}{\pgfqpoint{-0.048611in}{0.000000in}}{\pgfqpoint{-0.000000in}{0.000000in}}{%
\pgfpathmoveto{\pgfqpoint{-0.000000in}{0.000000in}}%
\pgfpathlineto{\pgfqpoint{-0.048611in}{0.000000in}}%
\pgfusepath{stroke,fill}%
}%
\begin{pgfscope}%
\pgfsys@transformshift{7.394209in}{0.375000in}%
\pgfsys@useobject{currentmarker}{}%
\end{pgfscope}%
\end{pgfscope}%
\begin{pgfscope}%
\definecolor{textcolor}{rgb}{0.000000,0.000000,0.000000}%
\pgfsetstrokecolor{textcolor}%
\pgfsetfillcolor{textcolor}%
\pgftext[x=7.046759in, y=0.305555in, left, base]{\color{textcolor}{\rmfamily\fontsize{14.000000}{16.800000}\selectfont\catcode`\^=\active\def^{\ifmmode\sp\else\^{}\fi}\catcode`\%=\active\def%{\%}$\mathdefault{0.0}$}}%
\end{pgfscope}%
\begin{pgfscope}%
\pgfsetbuttcap%
\pgfsetroundjoin%
\definecolor{currentfill}{rgb}{0.000000,0.000000,0.000000}%
\pgfsetfillcolor{currentfill}%
\pgfsetlinewidth{0.803000pt}%
\definecolor{currentstroke}{rgb}{0.000000,0.000000,0.000000}%
\pgfsetstrokecolor{currentstroke}%
\pgfsetdash{}{0pt}%
\pgfsys@defobject{currentmarker}{\pgfqpoint{-0.048611in}{0.000000in}}{\pgfqpoint{-0.000000in}{0.000000in}}{%
\pgfpathmoveto{\pgfqpoint{-0.000000in}{0.000000in}}%
\pgfpathlineto{\pgfqpoint{-0.048611in}{0.000000in}}%
\pgfusepath{stroke,fill}%
}%
\begin{pgfscope}%
\pgfsys@transformshift{7.394209in}{1.237500in}%
\pgfsys@useobject{currentmarker}{}%
\end{pgfscope}%
\end{pgfscope}%
\begin{pgfscope}%
\definecolor{textcolor}{rgb}{0.000000,0.000000,0.000000}%
\pgfsetstrokecolor{textcolor}%
\pgfsetfillcolor{textcolor}%
\pgftext[x=7.046759in, y=1.168055in, left, base]{\color{textcolor}{\rmfamily\fontsize{14.000000}{16.800000}\selectfont\catcode`\^=\active\def^{\ifmmode\sp\else\^{}\fi}\catcode`\%=\active\def%{\%}$\mathdefault{0.5}$}}%
\end{pgfscope}%
\begin{pgfscope}%
\pgfsetbuttcap%
\pgfsetroundjoin%
\definecolor{currentfill}{rgb}{0.000000,0.000000,0.000000}%
\pgfsetfillcolor{currentfill}%
\pgfsetlinewidth{0.803000pt}%
\definecolor{currentstroke}{rgb}{0.000000,0.000000,0.000000}%
\pgfsetstrokecolor{currentstroke}%
\pgfsetdash{}{0pt}%
\pgfsys@defobject{currentmarker}{\pgfqpoint{-0.048611in}{0.000000in}}{\pgfqpoint{-0.000000in}{0.000000in}}{%
\pgfpathmoveto{\pgfqpoint{-0.000000in}{0.000000in}}%
\pgfpathlineto{\pgfqpoint{-0.048611in}{0.000000in}}%
\pgfusepath{stroke,fill}%
}%
\begin{pgfscope}%
\pgfsys@transformshift{7.394209in}{2.100000in}%
\pgfsys@useobject{currentmarker}{}%
\end{pgfscope}%
\end{pgfscope}%
\begin{pgfscope}%
\definecolor{textcolor}{rgb}{0.000000,0.000000,0.000000}%
\pgfsetstrokecolor{textcolor}%
\pgfsetfillcolor{textcolor}%
\pgftext[x=7.046759in, y=2.030555in, left, base]{\color{textcolor}{\rmfamily\fontsize{14.000000}{16.800000}\selectfont\catcode`\^=\active\def^{\ifmmode\sp\else\^{}\fi}\catcode`\%=\active\def%{\%}$\mathdefault{1.0}$}}%
\end{pgfscope}%
\begin{pgfscope}%
\pgfsetbuttcap%
\pgfsetroundjoin%
\definecolor{currentfill}{rgb}{0.000000,0.000000,0.000000}%
\pgfsetfillcolor{currentfill}%
\pgfsetlinewidth{0.803000pt}%
\definecolor{currentstroke}{rgb}{0.000000,0.000000,0.000000}%
\pgfsetstrokecolor{currentstroke}%
\pgfsetdash{}{0pt}%
\pgfsys@defobject{currentmarker}{\pgfqpoint{-0.048611in}{0.000000in}}{\pgfqpoint{-0.000000in}{0.000000in}}{%
\pgfpathmoveto{\pgfqpoint{-0.000000in}{0.000000in}}%
\pgfpathlineto{\pgfqpoint{-0.048611in}{0.000000in}}%
\pgfusepath{stroke,fill}%
}%
\begin{pgfscope}%
\pgfsys@transformshift{7.394209in}{2.962500in}%
\pgfsys@useobject{currentmarker}{}%
\end{pgfscope}%
\end{pgfscope}%
\begin{pgfscope}%
\definecolor{textcolor}{rgb}{0.000000,0.000000,0.000000}%
\pgfsetstrokecolor{textcolor}%
\pgfsetfillcolor{textcolor}%
\pgftext[x=7.046759in, y=2.893055in, left, base]{\color{textcolor}{\rmfamily\fontsize{14.000000}{16.800000}\selectfont\catcode`\^=\active\def^{\ifmmode\sp\else\^{}\fi}\catcode`\%=\active\def%{\%}$\mathdefault{1.5}$}}%
\end{pgfscope}%
\begin{pgfscope}%
\pgfsetbuttcap%
\pgfsetroundjoin%
\definecolor{currentfill}{rgb}{0.000000,0.000000,0.000000}%
\pgfsetfillcolor{currentfill}%
\pgfsetlinewidth{0.803000pt}%
\definecolor{currentstroke}{rgb}{0.000000,0.000000,0.000000}%
\pgfsetstrokecolor{currentstroke}%
\pgfsetdash{}{0pt}%
\pgfsys@defobject{currentmarker}{\pgfqpoint{-0.048611in}{0.000000in}}{\pgfqpoint{-0.000000in}{0.000000in}}{%
\pgfpathmoveto{\pgfqpoint{-0.000000in}{0.000000in}}%
\pgfpathlineto{\pgfqpoint{-0.048611in}{0.000000in}}%
\pgfusepath{stroke,fill}%
}%
\begin{pgfscope}%
\pgfsys@transformshift{7.394209in}{3.825000in}%
\pgfsys@useobject{currentmarker}{}%
\end{pgfscope}%
\end{pgfscope}%
\begin{pgfscope}%
\definecolor{textcolor}{rgb}{0.000000,0.000000,0.000000}%
\pgfsetstrokecolor{textcolor}%
\pgfsetfillcolor{textcolor}%
\pgftext[x=7.046759in, y=3.755556in, left, base]{\color{textcolor}{\rmfamily\fontsize{14.000000}{16.800000}\selectfont\catcode`\^=\active\def^{\ifmmode\sp\else\^{}\fi}\catcode`\%=\active\def%{\%}$\mathdefault{2.0}$}}%
\end{pgfscope}%
\begin{pgfscope}%
\pgfsetbuttcap%
\pgfsetroundjoin%
\definecolor{currentfill}{rgb}{0.000000,0.000000,0.000000}%
\pgfsetfillcolor{currentfill}%
\pgfsetlinewidth{0.803000pt}%
\definecolor{currentstroke}{rgb}{0.000000,0.000000,0.000000}%
\pgfsetstrokecolor{currentstroke}%
\pgfsetdash{}{0pt}%
\pgfsys@defobject{currentmarker}{\pgfqpoint{-0.048611in}{0.000000in}}{\pgfqpoint{-0.000000in}{0.000000in}}{%
\pgfpathmoveto{\pgfqpoint{-0.000000in}{0.000000in}}%
\pgfpathlineto{\pgfqpoint{-0.048611in}{0.000000in}}%
\pgfusepath{stroke,fill}%
}%
\begin{pgfscope}%
\pgfsys@transformshift{7.394209in}{4.687500in}%
\pgfsys@useobject{currentmarker}{}%
\end{pgfscope}%
\end{pgfscope}%
\begin{pgfscope}%
\definecolor{textcolor}{rgb}{0.000000,0.000000,0.000000}%
\pgfsetstrokecolor{textcolor}%
\pgfsetfillcolor{textcolor}%
\pgftext[x=7.046759in, y=4.618056in, left, base]{\color{textcolor}{\rmfamily\fontsize{14.000000}{16.800000}\selectfont\catcode`\^=\active\def^{\ifmmode\sp\else\^{}\fi}\catcode`\%=\active\def%{\%}$\mathdefault{2.5}$}}%
\end{pgfscope}%
\begin{pgfscope}%
\pgfsetbuttcap%
\pgfsetroundjoin%
\definecolor{currentfill}{rgb}{0.000000,0.000000,0.000000}%
\pgfsetfillcolor{currentfill}%
\pgfsetlinewidth{0.803000pt}%
\definecolor{currentstroke}{rgb}{0.000000,0.000000,0.000000}%
\pgfsetstrokecolor{currentstroke}%
\pgfsetdash{}{0pt}%
\pgfsys@defobject{currentmarker}{\pgfqpoint{-0.048611in}{0.000000in}}{\pgfqpoint{-0.000000in}{0.000000in}}{%
\pgfpathmoveto{\pgfqpoint{-0.000000in}{0.000000in}}%
\pgfpathlineto{\pgfqpoint{-0.048611in}{0.000000in}}%
\pgfusepath{stroke,fill}%
}%
\begin{pgfscope}%
\pgfsys@transformshift{7.394209in}{5.550000in}%
\pgfsys@useobject{currentmarker}{}%
\end{pgfscope}%
\end{pgfscope}%
\begin{pgfscope}%
\definecolor{textcolor}{rgb}{0.000000,0.000000,0.000000}%
\pgfsetstrokecolor{textcolor}%
\pgfsetfillcolor{textcolor}%
\pgftext[x=7.046759in, y=5.480556in, left, base]{\color{textcolor}{\rmfamily\fontsize{14.000000}{16.800000}\selectfont\catcode`\^=\active\def^{\ifmmode\sp\else\^{}\fi}\catcode`\%=\active\def%{\%}$\mathdefault{3.0}$}}%
\end{pgfscope}%
\begin{pgfscope}%
\pgfsetrectcap%
\pgfsetmiterjoin%
\pgfsetlinewidth{0.803000pt}%
\definecolor{currentstroke}{rgb}{0.000000,0.000000,0.000000}%
\pgfsetstrokecolor{currentstroke}%
\pgfsetdash{}{0pt}%
\pgfpathmoveto{\pgfqpoint{7.394209in}{0.375000in}}%
\pgfpathlineto{\pgfqpoint{7.394209in}{5.550000in}}%
\pgfusepath{stroke}%
\end{pgfscope}%
\begin{pgfscope}%
\pgfsetrectcap%
\pgfsetmiterjoin%
\pgfsetlinewidth{0.803000pt}%
\definecolor{currentstroke}{rgb}{0.000000,0.000000,0.000000}%
\pgfsetstrokecolor{currentstroke}%
\pgfsetdash{}{0pt}%
\pgfpathmoveto{\pgfqpoint{13.751042in}{0.375000in}}%
\pgfpathlineto{\pgfqpoint{13.751042in}{5.550000in}}%
\pgfusepath{stroke}%
\end{pgfscope}%
\begin{pgfscope}%
\pgfsetrectcap%
\pgfsetmiterjoin%
\pgfsetlinewidth{0.803000pt}%
\definecolor{currentstroke}{rgb}{0.000000,0.000000,0.000000}%
\pgfsetstrokecolor{currentstroke}%
\pgfsetdash{}{0pt}%
\pgfpathmoveto{\pgfqpoint{7.394209in}{0.375000in}}%
\pgfpathlineto{\pgfqpoint{13.751042in}{0.375000in}}%
\pgfusepath{stroke}%
\end{pgfscope}%
\begin{pgfscope}%
\pgfsetrectcap%
\pgfsetmiterjoin%
\pgfsetlinewidth{0.803000pt}%
\definecolor{currentstroke}{rgb}{0.000000,0.000000,0.000000}%
\pgfsetstrokecolor{currentstroke}%
\pgfsetdash{}{0pt}%
\pgfpathmoveto{\pgfqpoint{7.394209in}{5.550000in}}%
\pgfpathlineto{\pgfqpoint{13.751042in}{5.550000in}}%
\pgfusepath{stroke}%
\end{pgfscope}%
\begin{pgfscope}%
\definecolor{textcolor}{rgb}{0.000000,0.000000,0.000000}%
\pgfsetstrokecolor{textcolor}%
\pgfsetfillcolor{textcolor}%
\pgftext[x=10.572626in,y=5.633333in,,base]{\color{textcolor}{\rmfamily\fontsize{16.000000}{19.200000}\selectfont\catcode`\^=\active\def^{\ifmmode\sp\else\^{}\fi}\catcode`\%=\active\def%{\%}Design Space}}%
\end{pgfscope}%
\end{pgfpicture}%
\makeatother%
\endgroup%
}
  \caption{Demonstrates the ``farthest-first-traversal'' algorithm with an example problem.}
  \label{fig:mga-fft}
\end{figure}
\section{Limitations of \ac{osier}}

Although \ac{osier} is the first \ac{esom} to allow arbitrary multi-objective
optimization it still has some limitations. This section describes some of the limitations in \ac{osier}'s modeling details
and scope.

\begin{enumerate}
    \item \ac{osier} has a limited time horizon and is not currently set up to
    optimize multiple decades into the future.
    \item \ac{osier} does not model interactions between the environment and the
    energy system. Therefore, temperature feedbacks and geoengineering
    technologies cannot be modeled adequately. Although, \ac{osier} could handle
    a negative emissions technology.
    \item Currently, \ac{osier}'s dispatch models can only model one energy carrier. That is, technologies
    could produce heat (measured in MW$_{th}$) or electricity. But there is no method
    to convert between the two endogenously.
    \item There is no concept of transmission or distribution within \ac{osier}'s dispatch
    models. 
    \item Demand for energy is treated as perfectly inelastic. Therefore, \ac{osier} cannot model 
    optional or deferrable loads.
    \item Ancillary services, capacity auctions, and other markets are not included in \ac{osier}. The dispatch
    model is strictly day-ahead or real time for the \ac{lp} and hierarchical formulations respectively.
\end{enumerate}

Many of these limitations are due to the underlying dispatch model. These limitations could be addressed 
by coupling \ac{osier}'s genetic algorithm to an external dispatch model, such as \ac{pypsa} or \ac{temoa}.
Other features, such as geoengineering and environmental interactions, are much more challenging to enclose.
Avenues of future work are detailed in Section \ref{section:future-work}.

% This responds to some of Cliff's comments on \ac{osier}.


\chapter{Benchmark Results}
\label{chapter:benchmark-results}
Due to its novelty, it is necessary to demonstrate that \ac{osier} generates
reproducible and reliable answers consistent with the results from an
established framework and that \ac{osier} is internally consistent when applying
different methods. To that end, this chapter illustrates some of the differences
between the logical and optimal dispatch algorithms introduced in Section
\ref{section:dispatch_model} by testing the two dispatch algorithms  
on some simple cases, as well as comparing the time scaling of these two
methods. Second, this chapter verifies that solutions calculated by \ac{osier}
agree with a more established \ac{esom}, \ac{temoa} \cite{hunter_modeling_2013}.
Finally, this chapter shows some of \ac{osier}'s advanced features, such as
many-objective objective problems and combining \ac{moo} with \ac{mga}. The next
section introduces some of the data and methods used to prepare the later
example problems.


\section{Technology Data}
Another feature of \ac{osier} is automatically exporting technology data to a
\texttt{pandas} dataframe or a \LaTeX table. Table \ref{tab:tech-table}
summarizes the technology data used in this thesis and was generated by
\ac{osier}. The cost data comes from \ac{nrel}'s \ac{atb}
\cite{national_renewable_energy_laboratory_2023_2023}. Carbon intensity data
come from a life cycle analysis from the \ac{unece}
\cite{united_nations_economic_commission_for_europe_carbon_2022}.

\textcolor{red}{CITE OTHER SOURCES HERE!}

\begin{sidewaystable}[!ht]
  \centering
  \caption{Summary of Technologies and Parameters available in \ac{osier}. This
  table was generated by \ac{osier}.}
  \label{tab:tech-table}
  \resizebox{\textheight}{!}{
  \input{tables/technology_database2}
  } % end resize box
\end{sidewaystable}

\FloatBarrier
\section{Data for the dispatch algorithm comparison}
The following sections describe the synthetic data generated to run the experiments.

\subsubsection{Energy Demand}
The synthetic demand data for this exercise were generated with

\begin{align}
    E(t) = -\sin\left(\frac{\pi t}{12}\right) + \sin\left(\frac{\pi t}{8760}\right) + y 
\end{align}

\subsubsection{Wind Speed}
\section{Comparing dispatch algorithms}
The first exercise in this section directly compares the results of two dispatch
algorithms introduced in Section \ref{section:dispatch_model} to confirm that
the two algorithms give similar answers. The next exercise shows how the two
algorithms scale. The final exercise focuses on the logical dispatch algorithm
and investigates how that algorithm scales with the number of threads used for
parallelization.
% \subsection{Exercise 0: Demonstrating \ac{osier}}
\textcolor{red}{This section is new in the dissertation!}

This section demonstrates some of the basic features of Osier. This section should be based 
on the examples in \ac{osier}'s documentation.

\subsection{Exercise 1: Validating a Simplified Approach}

This exercise considers three different cases with different technology mixes.
The first case includes natural gas and nuclear resources, the second case adds
a wind resource, and the last case adds battery storage. Each case had seven
days with an hourly resolution (168 timesteps). Both algorithms were allowed to
curtail excess energy and both were required to meet demand at all time steps.
Table \ref{tab:dispatch-results} summarizes the technologies available, the
optimizer used, and the value of the objective function. 

\begin{table}[ht!]
    \centering
    \caption{Summary results for the three dispatch test cases.}
    \label{tab:dispatch-results}
    \begin{tabular}{llllllr}
\toprule
Case & Natural Gas & Nuclear & Wind & Storage & Optimizer & Value \\
\midrule
1&\checkmark&\checkmark&&& Logical & 1.84954 \\
1&\checkmark&\checkmark&&& Optimal & 1.84954 \\
2&\checkmark&\checkmark&\checkmark&& Logical & 0.85499 \\
2&\checkmark&\checkmark&\checkmark&& Optimal & 0.85499 \\
3&\checkmark&\checkmark&\checkmark&\checkmark& Logical & 0.73009 \\
3&\checkmark&\checkmark&\checkmark&\checkmark& Optimal & 0.58797 \\
\bottomrule
\end{tabular}

\end{table}

% \noindent
The first two cases in Table \ref{tab:dispatch-results} show perfect agreement between 
the two algorithms. However, they disagree on the final case
with a battery storage technology. Figure \ref{fig:dispatch-comparison} compares
the dispatch results for the two methods. Figure
\ref{fig:dispatch-comparison}a was calculated with the logical dispatch
algorithm and Figure \ref{fig:dispatch-comparison}b was calculated with the
optimal dispatch algorithm. These plots show that the two
algorithms dispatch the same amounts of wind and nuclear energy. However, the
two algorithms differ in their usage of battery storage which causes further
differences in the dispatch of natural gas and total curtailment.

\begin{figure}[ht!]
    \centering
    \resizebox{0.95\columnwidth}{!}{%% Creator: Matplotlib, PGF backend
%%
%% To include the figure in your LaTeX document, write
%%   \input{<filename>.pgf}
%%
%% Make sure the required packages are loaded in your preamble
%%   \usepackage{pgf}
%%
%% Also ensure that all the required font packages are loaded; for instance,
%% the lmodern package is sometimes necessary when using math font.
%%   \usepackage{lmodern}
%%
%% Figures using additional raster images can only be included by \input if
%% they are in the same directory as the main LaTeX file. For loading figures
%% from other directories you can use the `import` package
%%   \usepackage{import}
%%
%% and then include the figures with
%%   \import{<path to file>}{<filename>.pgf}
%%
%% Matplotlib used the following preamble
%%   \def\mathdefault#1{#1}
%%   \everymath=\expandafter{\the\everymath\displaystyle}
%%   \IfFileExists{scrextend.sty}{
%%     \usepackage[fontsize=10.000000pt]{scrextend}
%%   }{
%%     \renewcommand{\normalsize}{\fontsize{10.000000}{12.000000}\selectfont}
%%     \normalsize
%%   }
%%   
%%   \makeatletter\@ifpackageloaded{underscore}{}{\usepackage[strings]{underscore}}\makeatother
%%
\begingroup%
\makeatletter%
\begin{pgfpicture}%
\pgfpathrectangle{\pgfpointorigin}{\pgfqpoint{9.900000in}{7.900000in}}%
\pgfusepath{use as bounding box, clip}%
\begin{pgfscope}%
\pgfsetbuttcap%
\pgfsetmiterjoin%
\definecolor{currentfill}{rgb}{1.000000,1.000000,1.000000}%
\pgfsetfillcolor{currentfill}%
\pgfsetlinewidth{0.000000pt}%
\definecolor{currentstroke}{rgb}{0.000000,0.000000,0.000000}%
\pgfsetstrokecolor{currentstroke}%
\pgfsetdash{}{0pt}%
\pgfpathmoveto{\pgfqpoint{0.000000in}{0.000000in}}%
\pgfpathlineto{\pgfqpoint{9.900000in}{0.000000in}}%
\pgfpathlineto{\pgfqpoint{9.900000in}{7.900000in}}%
\pgfpathlineto{\pgfqpoint{0.000000in}{7.900000in}}%
\pgfpathlineto{\pgfqpoint{0.000000in}{0.000000in}}%
\pgfpathclose%
\pgfusepath{fill}%
\end{pgfscope}%
\begin{pgfscope}%
\pgfsetbuttcap%
\pgfsetmiterjoin%
\definecolor{currentfill}{rgb}{1.000000,1.000000,1.000000}%
\pgfsetfillcolor{currentfill}%
\pgfsetlinewidth{0.000000pt}%
\definecolor{currentstroke}{rgb}{0.000000,0.000000,0.000000}%
\pgfsetstrokecolor{currentstroke}%
\pgfsetstrokeopacity{0.000000}%
\pgfsetdash{}{0pt}%
\pgfpathmoveto{\pgfqpoint{0.941663in}{4.334375in}}%
\pgfpathlineto{\pgfqpoint{9.800000in}{4.334375in}}%
\pgfpathlineto{\pgfqpoint{9.800000in}{7.800000in}}%
\pgfpathlineto{\pgfqpoint{0.941663in}{7.800000in}}%
\pgfpathlineto{\pgfqpoint{0.941663in}{4.334375in}}%
\pgfpathclose%
\pgfusepath{fill}%
\end{pgfscope}%
\begin{pgfscope}%
\pgfpathrectangle{\pgfqpoint{0.941663in}{4.334375in}}{\pgfqpoint{8.858337in}{3.465625in}}%
\pgfusepath{clip}%
\pgfsetrectcap%
\pgfsetroundjoin%
\pgfsetlinewidth{0.803000pt}%
\definecolor{currentstroke}{rgb}{0.690196,0.690196,0.690196}%
\pgfsetstrokecolor{currentstroke}%
\pgfsetdash{}{0pt}%
\pgfpathmoveto{\pgfqpoint{0.941663in}{4.334375in}}%
\pgfpathlineto{\pgfqpoint{0.941663in}{7.800000in}}%
\pgfusepath{stroke}%
\end{pgfscope}%
\begin{pgfscope}%
\pgfsetbuttcap%
\pgfsetroundjoin%
\definecolor{currentfill}{rgb}{0.000000,0.000000,0.000000}%
\pgfsetfillcolor{currentfill}%
\pgfsetlinewidth{0.803000pt}%
\definecolor{currentstroke}{rgb}{0.000000,0.000000,0.000000}%
\pgfsetstrokecolor{currentstroke}%
\pgfsetdash{}{0pt}%
\pgfsys@defobject{currentmarker}{\pgfqpoint{0.000000in}{-0.048611in}}{\pgfqpoint{0.000000in}{0.000000in}}{%
\pgfpathmoveto{\pgfqpoint{0.000000in}{0.000000in}}%
\pgfpathlineto{\pgfqpoint{0.000000in}{-0.048611in}}%
\pgfusepath{stroke,fill}%
}%
\begin{pgfscope}%
\pgfsys@transformshift{0.941663in}{4.334375in}%
\pgfsys@useobject{currentmarker}{}%
\end{pgfscope}%
\end{pgfscope}%
\begin{pgfscope}%
\pgfpathrectangle{\pgfqpoint{0.941663in}{4.334375in}}{\pgfqpoint{8.858337in}{3.465625in}}%
\pgfusepath{clip}%
\pgfsetrectcap%
\pgfsetroundjoin%
\pgfsetlinewidth{0.803000pt}%
\definecolor{currentstroke}{rgb}{0.690196,0.690196,0.690196}%
\pgfsetstrokecolor{currentstroke}%
\pgfsetdash{}{0pt}%
\pgfpathmoveto{\pgfqpoint{2.002542in}{4.334375in}}%
\pgfpathlineto{\pgfqpoint{2.002542in}{7.800000in}}%
\pgfusepath{stroke}%
\end{pgfscope}%
\begin{pgfscope}%
\pgfsetbuttcap%
\pgfsetroundjoin%
\definecolor{currentfill}{rgb}{0.000000,0.000000,0.000000}%
\pgfsetfillcolor{currentfill}%
\pgfsetlinewidth{0.803000pt}%
\definecolor{currentstroke}{rgb}{0.000000,0.000000,0.000000}%
\pgfsetstrokecolor{currentstroke}%
\pgfsetdash{}{0pt}%
\pgfsys@defobject{currentmarker}{\pgfqpoint{0.000000in}{-0.048611in}}{\pgfqpoint{0.000000in}{0.000000in}}{%
\pgfpathmoveto{\pgfqpoint{0.000000in}{0.000000in}}%
\pgfpathlineto{\pgfqpoint{0.000000in}{-0.048611in}}%
\pgfusepath{stroke,fill}%
}%
\begin{pgfscope}%
\pgfsys@transformshift{2.002542in}{4.334375in}%
\pgfsys@useobject{currentmarker}{}%
\end{pgfscope}%
\end{pgfscope}%
\begin{pgfscope}%
\pgfpathrectangle{\pgfqpoint{0.941663in}{4.334375in}}{\pgfqpoint{8.858337in}{3.465625in}}%
\pgfusepath{clip}%
\pgfsetrectcap%
\pgfsetroundjoin%
\pgfsetlinewidth{0.803000pt}%
\definecolor{currentstroke}{rgb}{0.690196,0.690196,0.690196}%
\pgfsetstrokecolor{currentstroke}%
\pgfsetdash{}{0pt}%
\pgfpathmoveto{\pgfqpoint{3.063420in}{4.334375in}}%
\pgfpathlineto{\pgfqpoint{3.063420in}{7.800000in}}%
\pgfusepath{stroke}%
\end{pgfscope}%
\begin{pgfscope}%
\pgfsetbuttcap%
\pgfsetroundjoin%
\definecolor{currentfill}{rgb}{0.000000,0.000000,0.000000}%
\pgfsetfillcolor{currentfill}%
\pgfsetlinewidth{0.803000pt}%
\definecolor{currentstroke}{rgb}{0.000000,0.000000,0.000000}%
\pgfsetstrokecolor{currentstroke}%
\pgfsetdash{}{0pt}%
\pgfsys@defobject{currentmarker}{\pgfqpoint{0.000000in}{-0.048611in}}{\pgfqpoint{0.000000in}{0.000000in}}{%
\pgfpathmoveto{\pgfqpoint{0.000000in}{0.000000in}}%
\pgfpathlineto{\pgfqpoint{0.000000in}{-0.048611in}}%
\pgfusepath{stroke,fill}%
}%
\begin{pgfscope}%
\pgfsys@transformshift{3.063420in}{4.334375in}%
\pgfsys@useobject{currentmarker}{}%
\end{pgfscope}%
\end{pgfscope}%
\begin{pgfscope}%
\pgfpathrectangle{\pgfqpoint{0.941663in}{4.334375in}}{\pgfqpoint{8.858337in}{3.465625in}}%
\pgfusepath{clip}%
\pgfsetrectcap%
\pgfsetroundjoin%
\pgfsetlinewidth{0.803000pt}%
\definecolor{currentstroke}{rgb}{0.690196,0.690196,0.690196}%
\pgfsetstrokecolor{currentstroke}%
\pgfsetdash{}{0pt}%
\pgfpathmoveto{\pgfqpoint{4.124299in}{4.334375in}}%
\pgfpathlineto{\pgfqpoint{4.124299in}{7.800000in}}%
\pgfusepath{stroke}%
\end{pgfscope}%
\begin{pgfscope}%
\pgfsetbuttcap%
\pgfsetroundjoin%
\definecolor{currentfill}{rgb}{0.000000,0.000000,0.000000}%
\pgfsetfillcolor{currentfill}%
\pgfsetlinewidth{0.803000pt}%
\definecolor{currentstroke}{rgb}{0.000000,0.000000,0.000000}%
\pgfsetstrokecolor{currentstroke}%
\pgfsetdash{}{0pt}%
\pgfsys@defobject{currentmarker}{\pgfqpoint{0.000000in}{-0.048611in}}{\pgfqpoint{0.000000in}{0.000000in}}{%
\pgfpathmoveto{\pgfqpoint{0.000000in}{0.000000in}}%
\pgfpathlineto{\pgfqpoint{0.000000in}{-0.048611in}}%
\pgfusepath{stroke,fill}%
}%
\begin{pgfscope}%
\pgfsys@transformshift{4.124299in}{4.334375in}%
\pgfsys@useobject{currentmarker}{}%
\end{pgfscope}%
\end{pgfscope}%
\begin{pgfscope}%
\pgfpathrectangle{\pgfqpoint{0.941663in}{4.334375in}}{\pgfqpoint{8.858337in}{3.465625in}}%
\pgfusepath{clip}%
\pgfsetrectcap%
\pgfsetroundjoin%
\pgfsetlinewidth{0.803000pt}%
\definecolor{currentstroke}{rgb}{0.690196,0.690196,0.690196}%
\pgfsetstrokecolor{currentstroke}%
\pgfsetdash{}{0pt}%
\pgfpathmoveto{\pgfqpoint{5.185178in}{4.334375in}}%
\pgfpathlineto{\pgfqpoint{5.185178in}{7.800000in}}%
\pgfusepath{stroke}%
\end{pgfscope}%
\begin{pgfscope}%
\pgfsetbuttcap%
\pgfsetroundjoin%
\definecolor{currentfill}{rgb}{0.000000,0.000000,0.000000}%
\pgfsetfillcolor{currentfill}%
\pgfsetlinewidth{0.803000pt}%
\definecolor{currentstroke}{rgb}{0.000000,0.000000,0.000000}%
\pgfsetstrokecolor{currentstroke}%
\pgfsetdash{}{0pt}%
\pgfsys@defobject{currentmarker}{\pgfqpoint{0.000000in}{-0.048611in}}{\pgfqpoint{0.000000in}{0.000000in}}{%
\pgfpathmoveto{\pgfqpoint{0.000000in}{0.000000in}}%
\pgfpathlineto{\pgfqpoint{0.000000in}{-0.048611in}}%
\pgfusepath{stroke,fill}%
}%
\begin{pgfscope}%
\pgfsys@transformshift{5.185178in}{4.334375in}%
\pgfsys@useobject{currentmarker}{}%
\end{pgfscope}%
\end{pgfscope}%
\begin{pgfscope}%
\pgfpathrectangle{\pgfqpoint{0.941663in}{4.334375in}}{\pgfqpoint{8.858337in}{3.465625in}}%
\pgfusepath{clip}%
\pgfsetrectcap%
\pgfsetroundjoin%
\pgfsetlinewidth{0.803000pt}%
\definecolor{currentstroke}{rgb}{0.690196,0.690196,0.690196}%
\pgfsetstrokecolor{currentstroke}%
\pgfsetdash{}{0pt}%
\pgfpathmoveto{\pgfqpoint{6.246056in}{4.334375in}}%
\pgfpathlineto{\pgfqpoint{6.246056in}{7.800000in}}%
\pgfusepath{stroke}%
\end{pgfscope}%
\begin{pgfscope}%
\pgfsetbuttcap%
\pgfsetroundjoin%
\definecolor{currentfill}{rgb}{0.000000,0.000000,0.000000}%
\pgfsetfillcolor{currentfill}%
\pgfsetlinewidth{0.803000pt}%
\definecolor{currentstroke}{rgb}{0.000000,0.000000,0.000000}%
\pgfsetstrokecolor{currentstroke}%
\pgfsetdash{}{0pt}%
\pgfsys@defobject{currentmarker}{\pgfqpoint{0.000000in}{-0.048611in}}{\pgfqpoint{0.000000in}{0.000000in}}{%
\pgfpathmoveto{\pgfqpoint{0.000000in}{0.000000in}}%
\pgfpathlineto{\pgfqpoint{0.000000in}{-0.048611in}}%
\pgfusepath{stroke,fill}%
}%
\begin{pgfscope}%
\pgfsys@transformshift{6.246056in}{4.334375in}%
\pgfsys@useobject{currentmarker}{}%
\end{pgfscope}%
\end{pgfscope}%
\begin{pgfscope}%
\pgfpathrectangle{\pgfqpoint{0.941663in}{4.334375in}}{\pgfqpoint{8.858337in}{3.465625in}}%
\pgfusepath{clip}%
\pgfsetrectcap%
\pgfsetroundjoin%
\pgfsetlinewidth{0.803000pt}%
\definecolor{currentstroke}{rgb}{0.690196,0.690196,0.690196}%
\pgfsetstrokecolor{currentstroke}%
\pgfsetdash{}{0pt}%
\pgfpathmoveto{\pgfqpoint{7.306935in}{4.334375in}}%
\pgfpathlineto{\pgfqpoint{7.306935in}{7.800000in}}%
\pgfusepath{stroke}%
\end{pgfscope}%
\begin{pgfscope}%
\pgfsetbuttcap%
\pgfsetroundjoin%
\definecolor{currentfill}{rgb}{0.000000,0.000000,0.000000}%
\pgfsetfillcolor{currentfill}%
\pgfsetlinewidth{0.803000pt}%
\definecolor{currentstroke}{rgb}{0.000000,0.000000,0.000000}%
\pgfsetstrokecolor{currentstroke}%
\pgfsetdash{}{0pt}%
\pgfsys@defobject{currentmarker}{\pgfqpoint{0.000000in}{-0.048611in}}{\pgfqpoint{0.000000in}{0.000000in}}{%
\pgfpathmoveto{\pgfqpoint{0.000000in}{0.000000in}}%
\pgfpathlineto{\pgfqpoint{0.000000in}{-0.048611in}}%
\pgfusepath{stroke,fill}%
}%
\begin{pgfscope}%
\pgfsys@transformshift{7.306935in}{4.334375in}%
\pgfsys@useobject{currentmarker}{}%
\end{pgfscope}%
\end{pgfscope}%
\begin{pgfscope}%
\pgfpathrectangle{\pgfqpoint{0.941663in}{4.334375in}}{\pgfqpoint{8.858337in}{3.465625in}}%
\pgfusepath{clip}%
\pgfsetrectcap%
\pgfsetroundjoin%
\pgfsetlinewidth{0.803000pt}%
\definecolor{currentstroke}{rgb}{0.690196,0.690196,0.690196}%
\pgfsetstrokecolor{currentstroke}%
\pgfsetdash{}{0pt}%
\pgfpathmoveto{\pgfqpoint{8.367814in}{4.334375in}}%
\pgfpathlineto{\pgfqpoint{8.367814in}{7.800000in}}%
\pgfusepath{stroke}%
\end{pgfscope}%
\begin{pgfscope}%
\pgfsetbuttcap%
\pgfsetroundjoin%
\definecolor{currentfill}{rgb}{0.000000,0.000000,0.000000}%
\pgfsetfillcolor{currentfill}%
\pgfsetlinewidth{0.803000pt}%
\definecolor{currentstroke}{rgb}{0.000000,0.000000,0.000000}%
\pgfsetstrokecolor{currentstroke}%
\pgfsetdash{}{0pt}%
\pgfsys@defobject{currentmarker}{\pgfqpoint{0.000000in}{-0.048611in}}{\pgfqpoint{0.000000in}{0.000000in}}{%
\pgfpathmoveto{\pgfqpoint{0.000000in}{0.000000in}}%
\pgfpathlineto{\pgfqpoint{0.000000in}{-0.048611in}}%
\pgfusepath{stroke,fill}%
}%
\begin{pgfscope}%
\pgfsys@transformshift{8.367814in}{4.334375in}%
\pgfsys@useobject{currentmarker}{}%
\end{pgfscope}%
\end{pgfscope}%
\begin{pgfscope}%
\pgfpathrectangle{\pgfqpoint{0.941663in}{4.334375in}}{\pgfqpoint{8.858337in}{3.465625in}}%
\pgfusepath{clip}%
\pgfsetrectcap%
\pgfsetroundjoin%
\pgfsetlinewidth{0.803000pt}%
\definecolor{currentstroke}{rgb}{0.690196,0.690196,0.690196}%
\pgfsetstrokecolor{currentstroke}%
\pgfsetdash{}{0pt}%
\pgfpathmoveto{\pgfqpoint{9.428692in}{4.334375in}}%
\pgfpathlineto{\pgfqpoint{9.428692in}{7.800000in}}%
\pgfusepath{stroke}%
\end{pgfscope}%
\begin{pgfscope}%
\pgfsetbuttcap%
\pgfsetroundjoin%
\definecolor{currentfill}{rgb}{0.000000,0.000000,0.000000}%
\pgfsetfillcolor{currentfill}%
\pgfsetlinewidth{0.803000pt}%
\definecolor{currentstroke}{rgb}{0.000000,0.000000,0.000000}%
\pgfsetstrokecolor{currentstroke}%
\pgfsetdash{}{0pt}%
\pgfsys@defobject{currentmarker}{\pgfqpoint{0.000000in}{-0.048611in}}{\pgfqpoint{0.000000in}{0.000000in}}{%
\pgfpathmoveto{\pgfqpoint{0.000000in}{0.000000in}}%
\pgfpathlineto{\pgfqpoint{0.000000in}{-0.048611in}}%
\pgfusepath{stroke,fill}%
}%
\begin{pgfscope}%
\pgfsys@transformshift{9.428692in}{4.334375in}%
\pgfsys@useobject{currentmarker}{}%
\end{pgfscope}%
\end{pgfscope}%
\begin{pgfscope}%
\pgfpathrectangle{\pgfqpoint{0.941663in}{4.334375in}}{\pgfqpoint{8.858337in}{3.465625in}}%
\pgfusepath{clip}%
\pgfsetrectcap%
\pgfsetroundjoin%
\pgfsetlinewidth{0.803000pt}%
\definecolor{currentstroke}{rgb}{0.690196,0.690196,0.690196}%
\pgfsetstrokecolor{currentstroke}%
\pgfsetdash{}{0pt}%
\pgfpathmoveto{\pgfqpoint{0.941663in}{4.859785in}}%
\pgfpathlineto{\pgfqpoint{9.800000in}{4.859785in}}%
\pgfusepath{stroke}%
\end{pgfscope}%
\begin{pgfscope}%
\pgfsetbuttcap%
\pgfsetroundjoin%
\definecolor{currentfill}{rgb}{0.000000,0.000000,0.000000}%
\pgfsetfillcolor{currentfill}%
\pgfsetlinewidth{0.803000pt}%
\definecolor{currentstroke}{rgb}{0.000000,0.000000,0.000000}%
\pgfsetstrokecolor{currentstroke}%
\pgfsetdash{}{0pt}%
\pgfsys@defobject{currentmarker}{\pgfqpoint{-0.048611in}{0.000000in}}{\pgfqpoint{-0.000000in}{0.000000in}}{%
\pgfpathmoveto{\pgfqpoint{-0.000000in}{0.000000in}}%
\pgfpathlineto{\pgfqpoint{-0.048611in}{0.000000in}}%
\pgfusepath{stroke,fill}%
}%
\begin{pgfscope}%
\pgfsys@transformshift{0.941663in}{4.859785in}%
\pgfsys@useobject{currentmarker}{}%
\end{pgfscope}%
\end{pgfscope}%
\begin{pgfscope}%
\definecolor{textcolor}{rgb}{0.000000,0.000000,0.000000}%
\pgfsetstrokecolor{textcolor}%
\pgfsetfillcolor{textcolor}%
\pgftext[x=0.395138in, y=4.790340in, left, base]{\color{textcolor}{\rmfamily\fontsize{14.000000}{16.800000}\selectfont\catcode`\^=\active\def^{\ifmmode\sp\else\^{}\fi}\catcode`\%=\active\def%{\%}$\mathdefault{\ensuremath{-}500}$}}%
\end{pgfscope}%
\begin{pgfscope}%
\pgfpathrectangle{\pgfqpoint{0.941663in}{4.334375in}}{\pgfqpoint{8.858337in}{3.465625in}}%
\pgfusepath{clip}%
\pgfsetrectcap%
\pgfsetroundjoin%
\pgfsetlinewidth{0.803000pt}%
\definecolor{currentstroke}{rgb}{0.690196,0.690196,0.690196}%
\pgfsetstrokecolor{currentstroke}%
\pgfsetdash{}{0pt}%
\pgfpathmoveto{\pgfqpoint{0.941663in}{5.555456in}}%
\pgfpathlineto{\pgfqpoint{9.800000in}{5.555456in}}%
\pgfusepath{stroke}%
\end{pgfscope}%
\begin{pgfscope}%
\pgfsetbuttcap%
\pgfsetroundjoin%
\definecolor{currentfill}{rgb}{0.000000,0.000000,0.000000}%
\pgfsetfillcolor{currentfill}%
\pgfsetlinewidth{0.803000pt}%
\definecolor{currentstroke}{rgb}{0.000000,0.000000,0.000000}%
\pgfsetstrokecolor{currentstroke}%
\pgfsetdash{}{0pt}%
\pgfsys@defobject{currentmarker}{\pgfqpoint{-0.048611in}{0.000000in}}{\pgfqpoint{-0.000000in}{0.000000in}}{%
\pgfpathmoveto{\pgfqpoint{-0.000000in}{0.000000in}}%
\pgfpathlineto{\pgfqpoint{-0.048611in}{0.000000in}}%
\pgfusepath{stroke,fill}%
}%
\begin{pgfscope}%
\pgfsys@transformshift{0.941663in}{5.555456in}%
\pgfsys@useobject{currentmarker}{}%
\end{pgfscope}%
\end{pgfscope}%
\begin{pgfscope}%
\definecolor{textcolor}{rgb}{0.000000,0.000000,0.000000}%
\pgfsetstrokecolor{textcolor}%
\pgfsetfillcolor{textcolor}%
\pgftext[x=0.746525in, y=5.486012in, left, base]{\color{textcolor}{\rmfamily\fontsize{14.000000}{16.800000}\selectfont\catcode`\^=\active\def^{\ifmmode\sp\else\^{}\fi}\catcode`\%=\active\def%{\%}$\mathdefault{0}$}}%
\end{pgfscope}%
\begin{pgfscope}%
\pgfpathrectangle{\pgfqpoint{0.941663in}{4.334375in}}{\pgfqpoint{8.858337in}{3.465625in}}%
\pgfusepath{clip}%
\pgfsetrectcap%
\pgfsetroundjoin%
\pgfsetlinewidth{0.803000pt}%
\definecolor{currentstroke}{rgb}{0.690196,0.690196,0.690196}%
\pgfsetstrokecolor{currentstroke}%
\pgfsetdash{}{0pt}%
\pgfpathmoveto{\pgfqpoint{0.941663in}{6.251128in}}%
\pgfpathlineto{\pgfqpoint{9.800000in}{6.251128in}}%
\pgfusepath{stroke}%
\end{pgfscope}%
\begin{pgfscope}%
\pgfsetbuttcap%
\pgfsetroundjoin%
\definecolor{currentfill}{rgb}{0.000000,0.000000,0.000000}%
\pgfsetfillcolor{currentfill}%
\pgfsetlinewidth{0.803000pt}%
\definecolor{currentstroke}{rgb}{0.000000,0.000000,0.000000}%
\pgfsetstrokecolor{currentstroke}%
\pgfsetdash{}{0pt}%
\pgfsys@defobject{currentmarker}{\pgfqpoint{-0.048611in}{0.000000in}}{\pgfqpoint{-0.000000in}{0.000000in}}{%
\pgfpathmoveto{\pgfqpoint{-0.000000in}{0.000000in}}%
\pgfpathlineto{\pgfqpoint{-0.048611in}{0.000000in}}%
\pgfusepath{stroke,fill}%
}%
\begin{pgfscope}%
\pgfsys@transformshift{0.941663in}{6.251128in}%
\pgfsys@useobject{currentmarker}{}%
\end{pgfscope}%
\end{pgfscope}%
\begin{pgfscope}%
\definecolor{textcolor}{rgb}{0.000000,0.000000,0.000000}%
\pgfsetstrokecolor{textcolor}%
\pgfsetfillcolor{textcolor}%
\pgftext[x=0.550694in, y=6.181684in, left, base]{\color{textcolor}{\rmfamily\fontsize{14.000000}{16.800000}\selectfont\catcode`\^=\active\def^{\ifmmode\sp\else\^{}\fi}\catcode`\%=\active\def%{\%}$\mathdefault{500}$}}%
\end{pgfscope}%
\begin{pgfscope}%
\pgfpathrectangle{\pgfqpoint{0.941663in}{4.334375in}}{\pgfqpoint{8.858337in}{3.465625in}}%
\pgfusepath{clip}%
\pgfsetrectcap%
\pgfsetroundjoin%
\pgfsetlinewidth{0.803000pt}%
\definecolor{currentstroke}{rgb}{0.690196,0.690196,0.690196}%
\pgfsetstrokecolor{currentstroke}%
\pgfsetdash{}{0pt}%
\pgfpathmoveto{\pgfqpoint{0.941663in}{6.946800in}}%
\pgfpathlineto{\pgfqpoint{9.800000in}{6.946800in}}%
\pgfusepath{stroke}%
\end{pgfscope}%
\begin{pgfscope}%
\pgfsetbuttcap%
\pgfsetroundjoin%
\definecolor{currentfill}{rgb}{0.000000,0.000000,0.000000}%
\pgfsetfillcolor{currentfill}%
\pgfsetlinewidth{0.803000pt}%
\definecolor{currentstroke}{rgb}{0.000000,0.000000,0.000000}%
\pgfsetstrokecolor{currentstroke}%
\pgfsetdash{}{0pt}%
\pgfsys@defobject{currentmarker}{\pgfqpoint{-0.048611in}{0.000000in}}{\pgfqpoint{-0.000000in}{0.000000in}}{%
\pgfpathmoveto{\pgfqpoint{-0.000000in}{0.000000in}}%
\pgfpathlineto{\pgfqpoint{-0.048611in}{0.000000in}}%
\pgfusepath{stroke,fill}%
}%
\begin{pgfscope}%
\pgfsys@transformshift{0.941663in}{6.946800in}%
\pgfsys@useobject{currentmarker}{}%
\end{pgfscope}%
\end{pgfscope}%
\begin{pgfscope}%
\definecolor{textcolor}{rgb}{0.000000,0.000000,0.000000}%
\pgfsetstrokecolor{textcolor}%
\pgfsetfillcolor{textcolor}%
\pgftext[x=0.452779in, y=6.877356in, left, base]{\color{textcolor}{\rmfamily\fontsize{14.000000}{16.800000}\selectfont\catcode`\^=\active\def^{\ifmmode\sp\else\^{}\fi}\catcode`\%=\active\def%{\%}$\mathdefault{1000}$}}%
\end{pgfscope}%
\begin{pgfscope}%
\pgfpathrectangle{\pgfqpoint{0.941663in}{4.334375in}}{\pgfqpoint{8.858337in}{3.465625in}}%
\pgfusepath{clip}%
\pgfsetrectcap%
\pgfsetroundjoin%
\pgfsetlinewidth{0.803000pt}%
\definecolor{currentstroke}{rgb}{0.690196,0.690196,0.690196}%
\pgfsetstrokecolor{currentstroke}%
\pgfsetdash{}{0pt}%
\pgfpathmoveto{\pgfqpoint{0.941663in}{7.642472in}}%
\pgfpathlineto{\pgfqpoint{9.800000in}{7.642472in}}%
\pgfusepath{stroke}%
\end{pgfscope}%
\begin{pgfscope}%
\pgfsetbuttcap%
\pgfsetroundjoin%
\definecolor{currentfill}{rgb}{0.000000,0.000000,0.000000}%
\pgfsetfillcolor{currentfill}%
\pgfsetlinewidth{0.803000pt}%
\definecolor{currentstroke}{rgb}{0.000000,0.000000,0.000000}%
\pgfsetstrokecolor{currentstroke}%
\pgfsetdash{}{0pt}%
\pgfsys@defobject{currentmarker}{\pgfqpoint{-0.048611in}{0.000000in}}{\pgfqpoint{-0.000000in}{0.000000in}}{%
\pgfpathmoveto{\pgfqpoint{-0.000000in}{0.000000in}}%
\pgfpathlineto{\pgfqpoint{-0.048611in}{0.000000in}}%
\pgfusepath{stroke,fill}%
}%
\begin{pgfscope}%
\pgfsys@transformshift{0.941663in}{7.642472in}%
\pgfsys@useobject{currentmarker}{}%
\end{pgfscope}%
\end{pgfscope}%
\begin{pgfscope}%
\definecolor{textcolor}{rgb}{0.000000,0.000000,0.000000}%
\pgfsetstrokecolor{textcolor}%
\pgfsetfillcolor{textcolor}%
\pgftext[x=0.452779in, y=7.573027in, left, base]{\color{textcolor}{\rmfamily\fontsize{14.000000}{16.800000}\selectfont\catcode`\^=\active\def^{\ifmmode\sp\else\^{}\fi}\catcode`\%=\active\def%{\%}$\mathdefault{1500}$}}%
\end{pgfscope}%
\begin{pgfscope}%
\definecolor{textcolor}{rgb}{0.000000,0.000000,0.000000}%
\pgfsetstrokecolor{textcolor}%
\pgfsetfillcolor{textcolor}%
\pgftext[x=0.339583in,y=6.067187in,,bottom,rotate=90.000000]{\color{textcolor}{\rmfamily\fontsize{18.000000}{21.600000}\selectfont\catcode`\^=\active\def^{\ifmmode\sp\else\^{}\fi}\catcode`\%=\active\def%{\%}Energy [MWh]}}%
\end{pgfscope}%
\begin{pgfscope}%
\pgfpathrectangle{\pgfqpoint{0.941663in}{4.334375in}}{\pgfqpoint{8.858337in}{3.465625in}}%
\pgfusepath{clip}%
\pgfsetrectcap%
\pgfsetroundjoin%
\pgfsetlinewidth{1.505625pt}%
\definecolor{currentstroke}{rgb}{0.121569,0.466667,0.705882}%
\pgfsetstrokecolor{currentstroke}%
\pgfsetdash{}{0pt}%
\pgfpathmoveto{\pgfqpoint{0.941663in}{6.251128in}}%
\pgfpathlineto{\pgfqpoint{9.800000in}{6.251128in}}%
\pgfpathlineto{\pgfqpoint{9.800000in}{6.251128in}}%
\pgfusepath{stroke}%
\end{pgfscope}%
\begin{pgfscope}%
\pgfpathrectangle{\pgfqpoint{0.941663in}{4.334375in}}{\pgfqpoint{8.858337in}{3.465625in}}%
\pgfusepath{clip}%
\pgfsetbuttcap%
\pgfsetroundjoin%
\definecolor{currentfill}{rgb}{0.121569,0.466667,0.705882}%
\pgfsetfillcolor{currentfill}%
\pgfsetlinewidth{1.003750pt}%
\definecolor{currentstroke}{rgb}{0.121569,0.466667,0.705882}%
\pgfsetstrokecolor{currentstroke}%
\pgfsetdash{}{0pt}%
\pgfsys@defobject{currentmarker}{\pgfqpoint{0.941663in}{5.555456in}}{\pgfqpoint{9.800000in}{6.251128in}}{%
\pgfpathmoveto{\pgfqpoint{0.941663in}{6.251128in}}%
\pgfpathlineto{\pgfqpoint{0.941663in}{5.555456in}}%
\pgfpathlineto{\pgfqpoint{0.994707in}{5.555456in}}%
\pgfpathlineto{\pgfqpoint{1.047751in}{5.555456in}}%
\pgfpathlineto{\pgfqpoint{1.100795in}{5.555456in}}%
\pgfpathlineto{\pgfqpoint{1.153839in}{5.555456in}}%
\pgfpathlineto{\pgfqpoint{1.206883in}{5.555456in}}%
\pgfpathlineto{\pgfqpoint{1.259927in}{5.555456in}}%
\pgfpathlineto{\pgfqpoint{1.312970in}{5.555456in}}%
\pgfpathlineto{\pgfqpoint{1.366014in}{5.555456in}}%
\pgfpathlineto{\pgfqpoint{1.419058in}{5.555456in}}%
\pgfpathlineto{\pgfqpoint{1.472102in}{5.555456in}}%
\pgfpathlineto{\pgfqpoint{1.525146in}{5.555456in}}%
\pgfpathlineto{\pgfqpoint{1.578190in}{5.555456in}}%
\pgfpathlineto{\pgfqpoint{1.631234in}{5.555456in}}%
\pgfpathlineto{\pgfqpoint{1.684278in}{5.555456in}}%
\pgfpathlineto{\pgfqpoint{1.737322in}{5.555456in}}%
\pgfpathlineto{\pgfqpoint{1.790366in}{5.555456in}}%
\pgfpathlineto{\pgfqpoint{1.843410in}{5.555456in}}%
\pgfpathlineto{\pgfqpoint{1.896454in}{5.555456in}}%
\pgfpathlineto{\pgfqpoint{1.949498in}{5.555456in}}%
\pgfpathlineto{\pgfqpoint{2.002542in}{5.555456in}}%
\pgfpathlineto{\pgfqpoint{2.055586in}{5.555456in}}%
\pgfpathlineto{\pgfqpoint{2.108629in}{5.555456in}}%
\pgfpathlineto{\pgfqpoint{2.161673in}{5.555456in}}%
\pgfpathlineto{\pgfqpoint{2.214717in}{5.555456in}}%
\pgfpathlineto{\pgfqpoint{2.267761in}{5.555456in}}%
\pgfpathlineto{\pgfqpoint{2.320805in}{5.555456in}}%
\pgfpathlineto{\pgfqpoint{2.373849in}{5.555456in}}%
\pgfpathlineto{\pgfqpoint{2.426893in}{5.555456in}}%
\pgfpathlineto{\pgfqpoint{2.479937in}{5.555456in}}%
\pgfpathlineto{\pgfqpoint{2.532981in}{5.555456in}}%
\pgfpathlineto{\pgfqpoint{2.586025in}{5.555456in}}%
\pgfpathlineto{\pgfqpoint{2.639069in}{5.555456in}}%
\pgfpathlineto{\pgfqpoint{2.692113in}{5.555456in}}%
\pgfpathlineto{\pgfqpoint{2.745157in}{5.555456in}}%
\pgfpathlineto{\pgfqpoint{2.798201in}{5.555456in}}%
\pgfpathlineto{\pgfqpoint{2.851245in}{5.555456in}}%
\pgfpathlineto{\pgfqpoint{2.904288in}{5.555456in}}%
\pgfpathlineto{\pgfqpoint{2.957332in}{5.555456in}}%
\pgfpathlineto{\pgfqpoint{3.010376in}{5.555456in}}%
\pgfpathlineto{\pgfqpoint{3.063420in}{5.555456in}}%
\pgfpathlineto{\pgfqpoint{3.116464in}{5.555456in}}%
\pgfpathlineto{\pgfqpoint{3.169508in}{5.555456in}}%
\pgfpathlineto{\pgfqpoint{3.222552in}{5.555456in}}%
\pgfpathlineto{\pgfqpoint{3.275596in}{5.555456in}}%
\pgfpathlineto{\pgfqpoint{3.328640in}{5.555456in}}%
\pgfpathlineto{\pgfqpoint{3.381684in}{5.555456in}}%
\pgfpathlineto{\pgfqpoint{3.434728in}{5.555456in}}%
\pgfpathlineto{\pgfqpoint{3.487772in}{5.555456in}}%
\pgfpathlineto{\pgfqpoint{3.540816in}{5.555456in}}%
\pgfpathlineto{\pgfqpoint{3.593860in}{5.555456in}}%
\pgfpathlineto{\pgfqpoint{3.646904in}{5.555456in}}%
\pgfpathlineto{\pgfqpoint{3.699948in}{5.555456in}}%
\pgfpathlineto{\pgfqpoint{3.752991in}{5.555456in}}%
\pgfpathlineto{\pgfqpoint{3.806035in}{5.555456in}}%
\pgfpathlineto{\pgfqpoint{3.859079in}{5.555456in}}%
\pgfpathlineto{\pgfqpoint{3.912123in}{5.555456in}}%
\pgfpathlineto{\pgfqpoint{3.965167in}{5.555456in}}%
\pgfpathlineto{\pgfqpoint{4.018211in}{5.555456in}}%
\pgfpathlineto{\pgfqpoint{4.071255in}{5.555456in}}%
\pgfpathlineto{\pgfqpoint{4.124299in}{5.555456in}}%
\pgfpathlineto{\pgfqpoint{4.177343in}{5.555456in}}%
\pgfpathlineto{\pgfqpoint{4.230387in}{5.555456in}}%
\pgfpathlineto{\pgfqpoint{4.283431in}{5.555456in}}%
\pgfpathlineto{\pgfqpoint{4.336475in}{5.555456in}}%
\pgfpathlineto{\pgfqpoint{4.389519in}{5.555456in}}%
\pgfpathlineto{\pgfqpoint{4.442563in}{5.555456in}}%
\pgfpathlineto{\pgfqpoint{4.495607in}{5.555456in}}%
\pgfpathlineto{\pgfqpoint{4.548650in}{5.555456in}}%
\pgfpathlineto{\pgfqpoint{4.601694in}{5.555456in}}%
\pgfpathlineto{\pgfqpoint{4.654738in}{5.555456in}}%
\pgfpathlineto{\pgfqpoint{4.707782in}{5.555456in}}%
\pgfpathlineto{\pgfqpoint{4.760826in}{5.555456in}}%
\pgfpathlineto{\pgfqpoint{4.813870in}{5.555456in}}%
\pgfpathlineto{\pgfqpoint{4.866914in}{5.555456in}}%
\pgfpathlineto{\pgfqpoint{4.919958in}{5.555456in}}%
\pgfpathlineto{\pgfqpoint{4.973002in}{5.555456in}}%
\pgfpathlineto{\pgfqpoint{5.026046in}{5.555456in}}%
\pgfpathlineto{\pgfqpoint{5.079090in}{5.555456in}}%
\pgfpathlineto{\pgfqpoint{5.132134in}{5.555456in}}%
\pgfpathlineto{\pgfqpoint{5.185178in}{5.555456in}}%
\pgfpathlineto{\pgfqpoint{5.238222in}{5.555456in}}%
\pgfpathlineto{\pgfqpoint{5.291266in}{5.555456in}}%
\pgfpathlineto{\pgfqpoint{5.344309in}{5.555456in}}%
\pgfpathlineto{\pgfqpoint{5.397353in}{5.555456in}}%
\pgfpathlineto{\pgfqpoint{5.450397in}{5.555456in}}%
\pgfpathlineto{\pgfqpoint{5.503441in}{5.555456in}}%
\pgfpathlineto{\pgfqpoint{5.556485in}{5.555456in}}%
\pgfpathlineto{\pgfqpoint{5.609529in}{5.555456in}}%
\pgfpathlineto{\pgfqpoint{5.662573in}{5.555456in}}%
\pgfpathlineto{\pgfqpoint{5.715617in}{5.555456in}}%
\pgfpathlineto{\pgfqpoint{5.768661in}{5.555456in}}%
\pgfpathlineto{\pgfqpoint{5.821705in}{5.555456in}}%
\pgfpathlineto{\pgfqpoint{5.874749in}{5.555456in}}%
\pgfpathlineto{\pgfqpoint{5.927793in}{5.555456in}}%
\pgfpathlineto{\pgfqpoint{5.980837in}{5.555456in}}%
\pgfpathlineto{\pgfqpoint{6.033881in}{5.555456in}}%
\pgfpathlineto{\pgfqpoint{6.086925in}{5.555456in}}%
\pgfpathlineto{\pgfqpoint{6.139969in}{5.555456in}}%
\pgfpathlineto{\pgfqpoint{6.193012in}{5.555456in}}%
\pgfpathlineto{\pgfqpoint{6.246056in}{5.555456in}}%
\pgfpathlineto{\pgfqpoint{6.299100in}{5.555456in}}%
\pgfpathlineto{\pgfqpoint{6.352144in}{5.555456in}}%
\pgfpathlineto{\pgfqpoint{6.405188in}{5.555456in}}%
\pgfpathlineto{\pgfqpoint{6.458232in}{5.555456in}}%
\pgfpathlineto{\pgfqpoint{6.511276in}{5.555456in}}%
\pgfpathlineto{\pgfqpoint{6.564320in}{5.555456in}}%
\pgfpathlineto{\pgfqpoint{6.617364in}{5.555456in}}%
\pgfpathlineto{\pgfqpoint{6.670408in}{5.555456in}}%
\pgfpathlineto{\pgfqpoint{6.723452in}{5.555456in}}%
\pgfpathlineto{\pgfqpoint{6.776496in}{5.555456in}}%
\pgfpathlineto{\pgfqpoint{6.829540in}{5.555456in}}%
\pgfpathlineto{\pgfqpoint{6.882584in}{5.555456in}}%
\pgfpathlineto{\pgfqpoint{6.935628in}{5.555456in}}%
\pgfpathlineto{\pgfqpoint{6.988671in}{5.555456in}}%
\pgfpathlineto{\pgfqpoint{7.041715in}{5.555456in}}%
\pgfpathlineto{\pgfqpoint{7.094759in}{5.555456in}}%
\pgfpathlineto{\pgfqpoint{7.147803in}{5.555456in}}%
\pgfpathlineto{\pgfqpoint{7.200847in}{5.555456in}}%
\pgfpathlineto{\pgfqpoint{7.253891in}{5.555456in}}%
\pgfpathlineto{\pgfqpoint{7.306935in}{5.555456in}}%
\pgfpathlineto{\pgfqpoint{7.359979in}{5.555456in}}%
\pgfpathlineto{\pgfqpoint{7.413023in}{5.555456in}}%
\pgfpathlineto{\pgfqpoint{7.466067in}{5.555456in}}%
\pgfpathlineto{\pgfqpoint{7.519111in}{5.555456in}}%
\pgfpathlineto{\pgfqpoint{7.572155in}{5.555456in}}%
\pgfpathlineto{\pgfqpoint{7.625199in}{5.555456in}}%
\pgfpathlineto{\pgfqpoint{7.678243in}{5.555456in}}%
\pgfpathlineto{\pgfqpoint{7.731287in}{5.555456in}}%
\pgfpathlineto{\pgfqpoint{7.784330in}{5.555456in}}%
\pgfpathlineto{\pgfqpoint{7.837374in}{5.555456in}}%
\pgfpathlineto{\pgfqpoint{7.890418in}{5.555456in}}%
\pgfpathlineto{\pgfqpoint{7.943462in}{5.555456in}}%
\pgfpathlineto{\pgfqpoint{7.996506in}{5.555456in}}%
\pgfpathlineto{\pgfqpoint{8.049550in}{5.555456in}}%
\pgfpathlineto{\pgfqpoint{8.102594in}{5.555456in}}%
\pgfpathlineto{\pgfqpoint{8.155638in}{5.555456in}}%
\pgfpathlineto{\pgfqpoint{8.208682in}{5.555456in}}%
\pgfpathlineto{\pgfqpoint{8.261726in}{5.555456in}}%
\pgfpathlineto{\pgfqpoint{8.314770in}{5.555456in}}%
\pgfpathlineto{\pgfqpoint{8.367814in}{5.555456in}}%
\pgfpathlineto{\pgfqpoint{8.420858in}{5.555456in}}%
\pgfpathlineto{\pgfqpoint{8.473902in}{5.555456in}}%
\pgfpathlineto{\pgfqpoint{8.526946in}{5.555456in}}%
\pgfpathlineto{\pgfqpoint{8.579990in}{5.555456in}}%
\pgfpathlineto{\pgfqpoint{8.633033in}{5.555456in}}%
\pgfpathlineto{\pgfqpoint{8.686077in}{5.555456in}}%
\pgfpathlineto{\pgfqpoint{8.739121in}{5.555456in}}%
\pgfpathlineto{\pgfqpoint{8.792165in}{5.555456in}}%
\pgfpathlineto{\pgfqpoint{8.845209in}{5.555456in}}%
\pgfpathlineto{\pgfqpoint{8.898253in}{5.555456in}}%
\pgfpathlineto{\pgfqpoint{8.951297in}{5.555456in}}%
\pgfpathlineto{\pgfqpoint{9.004341in}{5.555456in}}%
\pgfpathlineto{\pgfqpoint{9.057385in}{5.555456in}}%
\pgfpathlineto{\pgfqpoint{9.110429in}{5.555456in}}%
\pgfpathlineto{\pgfqpoint{9.163473in}{5.555456in}}%
\pgfpathlineto{\pgfqpoint{9.216517in}{5.555456in}}%
\pgfpathlineto{\pgfqpoint{9.269561in}{5.555456in}}%
\pgfpathlineto{\pgfqpoint{9.322605in}{5.555456in}}%
\pgfpathlineto{\pgfqpoint{9.375649in}{5.555456in}}%
\pgfpathlineto{\pgfqpoint{9.428692in}{5.555456in}}%
\pgfpathlineto{\pgfqpoint{9.481736in}{5.555456in}}%
\pgfpathlineto{\pgfqpoint{9.534780in}{5.555456in}}%
\pgfpathlineto{\pgfqpoint{9.587824in}{5.555456in}}%
\pgfpathlineto{\pgfqpoint{9.640868in}{5.555456in}}%
\pgfpathlineto{\pgfqpoint{9.693912in}{5.555456in}}%
\pgfpathlineto{\pgfqpoint{9.746956in}{5.555456in}}%
\pgfpathlineto{\pgfqpoint{9.800000in}{5.555456in}}%
\pgfpathlineto{\pgfqpoint{9.800000in}{6.251128in}}%
\pgfpathlineto{\pgfqpoint{9.800000in}{6.251128in}}%
\pgfpathlineto{\pgfqpoint{9.746956in}{6.251128in}}%
\pgfpathlineto{\pgfqpoint{9.693912in}{6.251128in}}%
\pgfpathlineto{\pgfqpoint{9.640868in}{6.251128in}}%
\pgfpathlineto{\pgfqpoint{9.587824in}{6.251128in}}%
\pgfpathlineto{\pgfqpoint{9.534780in}{6.251128in}}%
\pgfpathlineto{\pgfqpoint{9.481736in}{6.251128in}}%
\pgfpathlineto{\pgfqpoint{9.428692in}{6.251128in}}%
\pgfpathlineto{\pgfqpoint{9.375649in}{6.251128in}}%
\pgfpathlineto{\pgfqpoint{9.322605in}{6.251128in}}%
\pgfpathlineto{\pgfqpoint{9.269561in}{6.251128in}}%
\pgfpathlineto{\pgfqpoint{9.216517in}{6.251128in}}%
\pgfpathlineto{\pgfqpoint{9.163473in}{6.251128in}}%
\pgfpathlineto{\pgfqpoint{9.110429in}{6.251128in}}%
\pgfpathlineto{\pgfqpoint{9.057385in}{6.251128in}}%
\pgfpathlineto{\pgfqpoint{9.004341in}{6.251128in}}%
\pgfpathlineto{\pgfqpoint{8.951297in}{6.251128in}}%
\pgfpathlineto{\pgfqpoint{8.898253in}{6.251128in}}%
\pgfpathlineto{\pgfqpoint{8.845209in}{6.251128in}}%
\pgfpathlineto{\pgfqpoint{8.792165in}{6.251128in}}%
\pgfpathlineto{\pgfqpoint{8.739121in}{6.251128in}}%
\pgfpathlineto{\pgfqpoint{8.686077in}{6.251128in}}%
\pgfpathlineto{\pgfqpoint{8.633033in}{6.251128in}}%
\pgfpathlineto{\pgfqpoint{8.579990in}{6.251128in}}%
\pgfpathlineto{\pgfqpoint{8.526946in}{6.251128in}}%
\pgfpathlineto{\pgfqpoint{8.473902in}{6.251128in}}%
\pgfpathlineto{\pgfqpoint{8.420858in}{6.251128in}}%
\pgfpathlineto{\pgfqpoint{8.367814in}{6.251128in}}%
\pgfpathlineto{\pgfqpoint{8.314770in}{6.251128in}}%
\pgfpathlineto{\pgfqpoint{8.261726in}{6.251128in}}%
\pgfpathlineto{\pgfqpoint{8.208682in}{6.251128in}}%
\pgfpathlineto{\pgfqpoint{8.155638in}{6.251128in}}%
\pgfpathlineto{\pgfqpoint{8.102594in}{6.251128in}}%
\pgfpathlineto{\pgfqpoint{8.049550in}{6.251128in}}%
\pgfpathlineto{\pgfqpoint{7.996506in}{6.251128in}}%
\pgfpathlineto{\pgfqpoint{7.943462in}{6.251128in}}%
\pgfpathlineto{\pgfqpoint{7.890418in}{6.251128in}}%
\pgfpathlineto{\pgfqpoint{7.837374in}{6.251128in}}%
\pgfpathlineto{\pgfqpoint{7.784330in}{6.251128in}}%
\pgfpathlineto{\pgfqpoint{7.731287in}{6.251128in}}%
\pgfpathlineto{\pgfqpoint{7.678243in}{6.251128in}}%
\pgfpathlineto{\pgfqpoint{7.625199in}{6.251128in}}%
\pgfpathlineto{\pgfqpoint{7.572155in}{6.251128in}}%
\pgfpathlineto{\pgfqpoint{7.519111in}{6.251128in}}%
\pgfpathlineto{\pgfqpoint{7.466067in}{6.251128in}}%
\pgfpathlineto{\pgfqpoint{7.413023in}{6.251128in}}%
\pgfpathlineto{\pgfqpoint{7.359979in}{6.251128in}}%
\pgfpathlineto{\pgfqpoint{7.306935in}{6.251128in}}%
\pgfpathlineto{\pgfqpoint{7.253891in}{6.251128in}}%
\pgfpathlineto{\pgfqpoint{7.200847in}{6.251128in}}%
\pgfpathlineto{\pgfqpoint{7.147803in}{6.251128in}}%
\pgfpathlineto{\pgfqpoint{7.094759in}{6.251128in}}%
\pgfpathlineto{\pgfqpoint{7.041715in}{6.251128in}}%
\pgfpathlineto{\pgfqpoint{6.988671in}{6.251128in}}%
\pgfpathlineto{\pgfqpoint{6.935628in}{6.251128in}}%
\pgfpathlineto{\pgfqpoint{6.882584in}{6.251128in}}%
\pgfpathlineto{\pgfqpoint{6.829540in}{6.251128in}}%
\pgfpathlineto{\pgfqpoint{6.776496in}{6.251128in}}%
\pgfpathlineto{\pgfqpoint{6.723452in}{6.251128in}}%
\pgfpathlineto{\pgfqpoint{6.670408in}{6.251128in}}%
\pgfpathlineto{\pgfqpoint{6.617364in}{6.251128in}}%
\pgfpathlineto{\pgfqpoint{6.564320in}{6.251128in}}%
\pgfpathlineto{\pgfqpoint{6.511276in}{6.251128in}}%
\pgfpathlineto{\pgfqpoint{6.458232in}{6.251128in}}%
\pgfpathlineto{\pgfqpoint{6.405188in}{6.251128in}}%
\pgfpathlineto{\pgfqpoint{6.352144in}{6.251128in}}%
\pgfpathlineto{\pgfqpoint{6.299100in}{6.251128in}}%
\pgfpathlineto{\pgfqpoint{6.246056in}{6.251128in}}%
\pgfpathlineto{\pgfqpoint{6.193012in}{6.251128in}}%
\pgfpathlineto{\pgfqpoint{6.139969in}{6.251128in}}%
\pgfpathlineto{\pgfqpoint{6.086925in}{6.251128in}}%
\pgfpathlineto{\pgfqpoint{6.033881in}{6.251128in}}%
\pgfpathlineto{\pgfqpoint{5.980837in}{6.251128in}}%
\pgfpathlineto{\pgfqpoint{5.927793in}{6.251128in}}%
\pgfpathlineto{\pgfqpoint{5.874749in}{6.251128in}}%
\pgfpathlineto{\pgfqpoint{5.821705in}{6.251128in}}%
\pgfpathlineto{\pgfqpoint{5.768661in}{6.251128in}}%
\pgfpathlineto{\pgfqpoint{5.715617in}{6.251128in}}%
\pgfpathlineto{\pgfqpoint{5.662573in}{6.251128in}}%
\pgfpathlineto{\pgfqpoint{5.609529in}{6.251128in}}%
\pgfpathlineto{\pgfqpoint{5.556485in}{6.251128in}}%
\pgfpathlineto{\pgfqpoint{5.503441in}{6.251128in}}%
\pgfpathlineto{\pgfqpoint{5.450397in}{6.251128in}}%
\pgfpathlineto{\pgfqpoint{5.397353in}{6.251128in}}%
\pgfpathlineto{\pgfqpoint{5.344309in}{6.251128in}}%
\pgfpathlineto{\pgfqpoint{5.291266in}{6.251128in}}%
\pgfpathlineto{\pgfqpoint{5.238222in}{6.251128in}}%
\pgfpathlineto{\pgfqpoint{5.185178in}{6.251128in}}%
\pgfpathlineto{\pgfqpoint{5.132134in}{6.251128in}}%
\pgfpathlineto{\pgfqpoint{5.079090in}{6.251128in}}%
\pgfpathlineto{\pgfqpoint{5.026046in}{6.251128in}}%
\pgfpathlineto{\pgfqpoint{4.973002in}{6.251128in}}%
\pgfpathlineto{\pgfqpoint{4.919958in}{6.251128in}}%
\pgfpathlineto{\pgfqpoint{4.866914in}{6.251128in}}%
\pgfpathlineto{\pgfqpoint{4.813870in}{6.251128in}}%
\pgfpathlineto{\pgfqpoint{4.760826in}{6.251128in}}%
\pgfpathlineto{\pgfqpoint{4.707782in}{6.251128in}}%
\pgfpathlineto{\pgfqpoint{4.654738in}{6.251128in}}%
\pgfpathlineto{\pgfqpoint{4.601694in}{6.251128in}}%
\pgfpathlineto{\pgfqpoint{4.548650in}{6.251128in}}%
\pgfpathlineto{\pgfqpoint{4.495607in}{6.251128in}}%
\pgfpathlineto{\pgfqpoint{4.442563in}{6.251128in}}%
\pgfpathlineto{\pgfqpoint{4.389519in}{6.251128in}}%
\pgfpathlineto{\pgfqpoint{4.336475in}{6.251128in}}%
\pgfpathlineto{\pgfqpoint{4.283431in}{6.251128in}}%
\pgfpathlineto{\pgfqpoint{4.230387in}{6.251128in}}%
\pgfpathlineto{\pgfqpoint{4.177343in}{6.251128in}}%
\pgfpathlineto{\pgfqpoint{4.124299in}{6.251128in}}%
\pgfpathlineto{\pgfqpoint{4.071255in}{6.251128in}}%
\pgfpathlineto{\pgfqpoint{4.018211in}{6.251128in}}%
\pgfpathlineto{\pgfqpoint{3.965167in}{6.251128in}}%
\pgfpathlineto{\pgfqpoint{3.912123in}{6.251128in}}%
\pgfpathlineto{\pgfqpoint{3.859079in}{6.251128in}}%
\pgfpathlineto{\pgfqpoint{3.806035in}{6.251128in}}%
\pgfpathlineto{\pgfqpoint{3.752991in}{6.251128in}}%
\pgfpathlineto{\pgfqpoint{3.699948in}{6.251128in}}%
\pgfpathlineto{\pgfqpoint{3.646904in}{6.251128in}}%
\pgfpathlineto{\pgfqpoint{3.593860in}{6.251128in}}%
\pgfpathlineto{\pgfqpoint{3.540816in}{6.251128in}}%
\pgfpathlineto{\pgfqpoint{3.487772in}{6.251128in}}%
\pgfpathlineto{\pgfqpoint{3.434728in}{6.251128in}}%
\pgfpathlineto{\pgfqpoint{3.381684in}{6.251128in}}%
\pgfpathlineto{\pgfqpoint{3.328640in}{6.251128in}}%
\pgfpathlineto{\pgfqpoint{3.275596in}{6.251128in}}%
\pgfpathlineto{\pgfqpoint{3.222552in}{6.251128in}}%
\pgfpathlineto{\pgfqpoint{3.169508in}{6.251128in}}%
\pgfpathlineto{\pgfqpoint{3.116464in}{6.251128in}}%
\pgfpathlineto{\pgfqpoint{3.063420in}{6.251128in}}%
\pgfpathlineto{\pgfqpoint{3.010376in}{6.251128in}}%
\pgfpathlineto{\pgfqpoint{2.957332in}{6.251128in}}%
\pgfpathlineto{\pgfqpoint{2.904288in}{6.251128in}}%
\pgfpathlineto{\pgfqpoint{2.851245in}{6.251128in}}%
\pgfpathlineto{\pgfqpoint{2.798201in}{6.251128in}}%
\pgfpathlineto{\pgfqpoint{2.745157in}{6.251128in}}%
\pgfpathlineto{\pgfqpoint{2.692113in}{6.251128in}}%
\pgfpathlineto{\pgfqpoint{2.639069in}{6.251128in}}%
\pgfpathlineto{\pgfqpoint{2.586025in}{6.251128in}}%
\pgfpathlineto{\pgfqpoint{2.532981in}{6.251128in}}%
\pgfpathlineto{\pgfqpoint{2.479937in}{6.251128in}}%
\pgfpathlineto{\pgfqpoint{2.426893in}{6.251128in}}%
\pgfpathlineto{\pgfqpoint{2.373849in}{6.251128in}}%
\pgfpathlineto{\pgfqpoint{2.320805in}{6.251128in}}%
\pgfpathlineto{\pgfqpoint{2.267761in}{6.251128in}}%
\pgfpathlineto{\pgfqpoint{2.214717in}{6.251128in}}%
\pgfpathlineto{\pgfqpoint{2.161673in}{6.251128in}}%
\pgfpathlineto{\pgfqpoint{2.108629in}{6.251128in}}%
\pgfpathlineto{\pgfqpoint{2.055586in}{6.251128in}}%
\pgfpathlineto{\pgfqpoint{2.002542in}{6.251128in}}%
\pgfpathlineto{\pgfqpoint{1.949498in}{6.251128in}}%
\pgfpathlineto{\pgfqpoint{1.896454in}{6.251128in}}%
\pgfpathlineto{\pgfqpoint{1.843410in}{6.251128in}}%
\pgfpathlineto{\pgfqpoint{1.790366in}{6.251128in}}%
\pgfpathlineto{\pgfqpoint{1.737322in}{6.251128in}}%
\pgfpathlineto{\pgfqpoint{1.684278in}{6.251128in}}%
\pgfpathlineto{\pgfqpoint{1.631234in}{6.251128in}}%
\pgfpathlineto{\pgfqpoint{1.578190in}{6.251128in}}%
\pgfpathlineto{\pgfqpoint{1.525146in}{6.251128in}}%
\pgfpathlineto{\pgfqpoint{1.472102in}{6.251128in}}%
\pgfpathlineto{\pgfqpoint{1.419058in}{6.251128in}}%
\pgfpathlineto{\pgfqpoint{1.366014in}{6.251128in}}%
\pgfpathlineto{\pgfqpoint{1.312970in}{6.251128in}}%
\pgfpathlineto{\pgfqpoint{1.259927in}{6.251128in}}%
\pgfpathlineto{\pgfqpoint{1.206883in}{6.251128in}}%
\pgfpathlineto{\pgfqpoint{1.153839in}{6.251128in}}%
\pgfpathlineto{\pgfqpoint{1.100795in}{6.251128in}}%
\pgfpathlineto{\pgfqpoint{1.047751in}{6.251128in}}%
\pgfpathlineto{\pgfqpoint{0.994707in}{6.251128in}}%
\pgfpathlineto{\pgfqpoint{0.941663in}{6.251128in}}%
\pgfpathlineto{\pgfqpoint{0.941663in}{6.251128in}}%
\pgfpathclose%
\pgfusepath{stroke,fill}%
}%
\begin{pgfscope}%
\pgfsys@transformshift{0.000000in}{0.000000in}%
\pgfsys@useobject{currentmarker}{}%
\end{pgfscope}%
\end{pgfscope}%
\begin{pgfscope}%
\pgfpathrectangle{\pgfqpoint{0.941663in}{4.334375in}}{\pgfqpoint{8.858337in}{3.465625in}}%
\pgfusepath{clip}%
\pgfsetrectcap%
\pgfsetroundjoin%
\pgfsetlinewidth{1.505625pt}%
\definecolor{currentstroke}{rgb}{0.501961,0.000000,0.501961}%
\pgfsetstrokecolor{currentstroke}%
\pgfsetdash{}{0pt}%
\pgfpathmoveto{\pgfqpoint{0.941663in}{6.251128in}}%
\pgfpathlineto{\pgfqpoint{0.994707in}{6.261680in}}%
\pgfpathlineto{\pgfqpoint{1.047751in}{6.251128in}}%
\pgfpathlineto{\pgfqpoint{1.153839in}{6.251128in}}%
\pgfpathlineto{\pgfqpoint{1.206883in}{6.598964in}}%
\pgfpathlineto{\pgfqpoint{1.312970in}{6.598964in}}%
\pgfpathlineto{\pgfqpoint{1.366014in}{6.251128in}}%
\pgfpathlineto{\pgfqpoint{1.578190in}{6.251128in}}%
\pgfpathlineto{\pgfqpoint{1.631234in}{6.598964in}}%
\pgfpathlineto{\pgfqpoint{1.684278in}{6.598964in}}%
\pgfpathlineto{\pgfqpoint{1.737322in}{6.251128in}}%
\pgfpathlineto{\pgfqpoint{1.790366in}{6.251128in}}%
\pgfpathlineto{\pgfqpoint{1.843410in}{6.598964in}}%
\pgfpathlineto{\pgfqpoint{1.896454in}{6.598964in}}%
\pgfpathlineto{\pgfqpoint{1.949498in}{6.251128in}}%
\pgfpathlineto{\pgfqpoint{2.002542in}{6.505445in}}%
\pgfpathlineto{\pgfqpoint{2.055586in}{6.251128in}}%
\pgfpathlineto{\pgfqpoint{2.108629in}{6.348747in}}%
\pgfpathlineto{\pgfqpoint{2.161673in}{6.251128in}}%
\pgfpathlineto{\pgfqpoint{2.214717in}{6.459099in}}%
\pgfpathlineto{\pgfqpoint{2.267761in}{6.251128in}}%
\pgfpathlineto{\pgfqpoint{2.320805in}{6.251128in}}%
\pgfpathlineto{\pgfqpoint{2.373849in}{6.541500in}}%
\pgfpathlineto{\pgfqpoint{2.426893in}{6.251128in}}%
\pgfpathlineto{\pgfqpoint{2.798201in}{6.251128in}}%
\pgfpathlineto{\pgfqpoint{2.851245in}{6.598964in}}%
\pgfpathlineto{\pgfqpoint{2.904288in}{6.251128in}}%
\pgfpathlineto{\pgfqpoint{2.957332in}{6.598964in}}%
\pgfpathlineto{\pgfqpoint{3.010376in}{6.251128in}}%
\pgfpathlineto{\pgfqpoint{3.063420in}{6.598964in}}%
\pgfpathlineto{\pgfqpoint{3.116464in}{6.598964in}}%
\pgfpathlineto{\pgfqpoint{3.169508in}{6.251128in}}%
\pgfpathlineto{\pgfqpoint{3.328640in}{6.251128in}}%
\pgfpathlineto{\pgfqpoint{3.381684in}{6.598964in}}%
\pgfpathlineto{\pgfqpoint{3.434728in}{6.251128in}}%
\pgfpathlineto{\pgfqpoint{3.487772in}{6.598964in}}%
\pgfpathlineto{\pgfqpoint{3.540816in}{6.427614in}}%
\pgfpathlineto{\pgfqpoint{3.593860in}{6.251128in}}%
\pgfpathlineto{\pgfqpoint{3.699948in}{6.251128in}}%
\pgfpathlineto{\pgfqpoint{3.752991in}{6.510498in}}%
\pgfpathlineto{\pgfqpoint{3.806035in}{6.251128in}}%
\pgfpathlineto{\pgfqpoint{3.859079in}{6.572998in}}%
\pgfpathlineto{\pgfqpoint{3.912123in}{6.481634in}}%
\pgfpathlineto{\pgfqpoint{3.965167in}{6.340259in}}%
\pgfpathlineto{\pgfqpoint{4.018211in}{6.251128in}}%
\pgfpathlineto{\pgfqpoint{4.283431in}{6.251128in}}%
\pgfpathlineto{\pgfqpoint{4.336475in}{6.598964in}}%
\pgfpathlineto{\pgfqpoint{4.389519in}{6.299258in}}%
\pgfpathlineto{\pgfqpoint{4.442563in}{6.598964in}}%
\pgfpathlineto{\pgfqpoint{4.495607in}{6.437705in}}%
\pgfpathlineto{\pgfqpoint{4.548650in}{6.251128in}}%
\pgfpathlineto{\pgfqpoint{4.654738in}{6.251128in}}%
\pgfpathlineto{\pgfqpoint{4.707782in}{6.351811in}}%
\pgfpathlineto{\pgfqpoint{4.760826in}{6.251128in}}%
\pgfpathlineto{\pgfqpoint{4.813870in}{6.251128in}}%
\pgfpathlineto{\pgfqpoint{4.866914in}{6.471548in}}%
\pgfpathlineto{\pgfqpoint{4.919958in}{6.251128in}}%
\pgfpathlineto{\pgfqpoint{4.973002in}{6.251128in}}%
\pgfpathlineto{\pgfqpoint{5.026046in}{6.590550in}}%
\pgfpathlineto{\pgfqpoint{5.079090in}{6.251128in}}%
\pgfpathlineto{\pgfqpoint{5.132134in}{6.598964in}}%
\pgfpathlineto{\pgfqpoint{5.185178in}{6.251128in}}%
\pgfpathlineto{\pgfqpoint{5.291266in}{6.251128in}}%
\pgfpathlineto{\pgfqpoint{5.344309in}{6.471024in}}%
\pgfpathlineto{\pgfqpoint{5.397353in}{6.251128in}}%
\pgfpathlineto{\pgfqpoint{5.503441in}{6.251128in}}%
\pgfpathlineto{\pgfqpoint{5.556485in}{6.329843in}}%
\pgfpathlineto{\pgfqpoint{5.609529in}{6.251128in}}%
\pgfpathlineto{\pgfqpoint{5.715617in}{6.251128in}}%
\pgfpathlineto{\pgfqpoint{5.768661in}{6.324051in}}%
\pgfpathlineto{\pgfqpoint{5.821705in}{6.251128in}}%
\pgfpathlineto{\pgfqpoint{5.980837in}{6.251128in}}%
\pgfpathlineto{\pgfqpoint{6.033881in}{6.400407in}}%
\pgfpathlineto{\pgfqpoint{6.086925in}{6.251128in}}%
\pgfpathlineto{\pgfqpoint{6.139969in}{6.251128in}}%
\pgfpathlineto{\pgfqpoint{6.193012in}{6.598964in}}%
\pgfpathlineto{\pgfqpoint{6.246056in}{6.362599in}}%
\pgfpathlineto{\pgfqpoint{6.299100in}{6.251128in}}%
\pgfpathlineto{\pgfqpoint{6.352144in}{6.569516in}}%
\pgfpathlineto{\pgfqpoint{6.405188in}{6.251128in}}%
\pgfpathlineto{\pgfqpoint{6.723452in}{6.251128in}}%
\pgfpathlineto{\pgfqpoint{6.776496in}{6.385890in}}%
\pgfpathlineto{\pgfqpoint{6.829540in}{6.251128in}}%
\pgfpathlineto{\pgfqpoint{6.882584in}{6.598964in}}%
\pgfpathlineto{\pgfqpoint{6.935628in}{6.251128in}}%
\pgfpathlineto{\pgfqpoint{6.988671in}{6.598964in}}%
\pgfpathlineto{\pgfqpoint{7.041715in}{6.251128in}}%
\pgfpathlineto{\pgfqpoint{7.147803in}{6.251128in}}%
\pgfpathlineto{\pgfqpoint{7.200847in}{6.598964in}}%
\pgfpathlineto{\pgfqpoint{7.253891in}{6.451777in}}%
\pgfpathlineto{\pgfqpoint{7.306935in}{6.251128in}}%
\pgfpathlineto{\pgfqpoint{7.359979in}{6.350974in}}%
\pgfpathlineto{\pgfqpoint{7.413023in}{6.251128in}}%
\pgfpathlineto{\pgfqpoint{7.996506in}{6.251128in}}%
\pgfpathlineto{\pgfqpoint{8.049550in}{6.598964in}}%
\pgfpathlineto{\pgfqpoint{8.102594in}{6.598964in}}%
\pgfpathlineto{\pgfqpoint{8.155638in}{6.251128in}}%
\pgfpathlineto{\pgfqpoint{8.208682in}{6.251128in}}%
\pgfpathlineto{\pgfqpoint{8.261726in}{6.598964in}}%
\pgfpathlineto{\pgfqpoint{8.314770in}{6.598964in}}%
\pgfpathlineto{\pgfqpoint{8.367814in}{6.251128in}}%
\pgfpathlineto{\pgfqpoint{8.420858in}{6.371312in}}%
\pgfpathlineto{\pgfqpoint{8.473902in}{6.251128in}}%
\pgfpathlineto{\pgfqpoint{8.526946in}{6.251128in}}%
\pgfpathlineto{\pgfqpoint{8.579990in}{6.560479in}}%
\pgfpathlineto{\pgfqpoint{8.633033in}{6.251128in}}%
\pgfpathlineto{\pgfqpoint{8.951297in}{6.251128in}}%
\pgfpathlineto{\pgfqpoint{9.004341in}{6.568668in}}%
\pgfpathlineto{\pgfqpoint{9.057385in}{6.251128in}}%
\pgfpathlineto{\pgfqpoint{9.110429in}{6.251128in}}%
\pgfpathlineto{\pgfqpoint{9.163473in}{6.598964in}}%
\pgfpathlineto{\pgfqpoint{9.216517in}{6.334558in}}%
\pgfpathlineto{\pgfqpoint{9.269561in}{6.251128in}}%
\pgfpathlineto{\pgfqpoint{9.428692in}{6.251128in}}%
\pgfpathlineto{\pgfqpoint{9.481736in}{6.256205in}}%
\pgfpathlineto{\pgfqpoint{9.534780in}{6.287774in}}%
\pgfpathlineto{\pgfqpoint{9.587824in}{6.251128in}}%
\pgfpathlineto{\pgfqpoint{9.800000in}{6.251128in}}%
\pgfpathlineto{\pgfqpoint{9.800000in}{6.251128in}}%
\pgfusepath{stroke}%
\end{pgfscope}%
\begin{pgfscope}%
\pgfpathrectangle{\pgfqpoint{0.941663in}{4.334375in}}{\pgfqpoint{8.858337in}{3.465625in}}%
\pgfusepath{clip}%
\pgfsetbuttcap%
\pgfsetroundjoin%
\definecolor{currentfill}{rgb}{0.501961,0.000000,0.501961}%
\pgfsetfillcolor{currentfill}%
\pgfsetlinewidth{1.003750pt}%
\definecolor{currentstroke}{rgb}{0.501961,0.000000,0.501961}%
\pgfsetstrokecolor{currentstroke}%
\pgfsetdash{}{0pt}%
\pgfsys@defobject{currentmarker}{\pgfqpoint{0.941663in}{6.251128in}}{\pgfqpoint{9.800000in}{6.598964in}}{%
\pgfpathmoveto{\pgfqpoint{0.941663in}{6.251128in}}%
\pgfpathlineto{\pgfqpoint{0.941663in}{6.251128in}}%
\pgfpathlineto{\pgfqpoint{0.994707in}{6.251128in}}%
\pgfpathlineto{\pgfqpoint{1.047751in}{6.251128in}}%
\pgfpathlineto{\pgfqpoint{1.100795in}{6.251128in}}%
\pgfpathlineto{\pgfqpoint{1.153839in}{6.251128in}}%
\pgfpathlineto{\pgfqpoint{1.206883in}{6.251128in}}%
\pgfpathlineto{\pgfqpoint{1.259927in}{6.251128in}}%
\pgfpathlineto{\pgfqpoint{1.312970in}{6.251128in}}%
\pgfpathlineto{\pgfqpoint{1.366014in}{6.251128in}}%
\pgfpathlineto{\pgfqpoint{1.419058in}{6.251128in}}%
\pgfpathlineto{\pgfqpoint{1.472102in}{6.251128in}}%
\pgfpathlineto{\pgfqpoint{1.525146in}{6.251128in}}%
\pgfpathlineto{\pgfqpoint{1.578190in}{6.251128in}}%
\pgfpathlineto{\pgfqpoint{1.631234in}{6.251128in}}%
\pgfpathlineto{\pgfqpoint{1.684278in}{6.251128in}}%
\pgfpathlineto{\pgfqpoint{1.737322in}{6.251128in}}%
\pgfpathlineto{\pgfqpoint{1.790366in}{6.251128in}}%
\pgfpathlineto{\pgfqpoint{1.843410in}{6.251128in}}%
\pgfpathlineto{\pgfqpoint{1.896454in}{6.251128in}}%
\pgfpathlineto{\pgfqpoint{1.949498in}{6.251128in}}%
\pgfpathlineto{\pgfqpoint{2.002542in}{6.251128in}}%
\pgfpathlineto{\pgfqpoint{2.055586in}{6.251128in}}%
\pgfpathlineto{\pgfqpoint{2.108629in}{6.251128in}}%
\pgfpathlineto{\pgfqpoint{2.161673in}{6.251128in}}%
\pgfpathlineto{\pgfqpoint{2.214717in}{6.251128in}}%
\pgfpathlineto{\pgfqpoint{2.267761in}{6.251128in}}%
\pgfpathlineto{\pgfqpoint{2.320805in}{6.251128in}}%
\pgfpathlineto{\pgfqpoint{2.373849in}{6.251128in}}%
\pgfpathlineto{\pgfqpoint{2.426893in}{6.251128in}}%
\pgfpathlineto{\pgfqpoint{2.479937in}{6.251128in}}%
\pgfpathlineto{\pgfqpoint{2.532981in}{6.251128in}}%
\pgfpathlineto{\pgfqpoint{2.586025in}{6.251128in}}%
\pgfpathlineto{\pgfqpoint{2.639069in}{6.251128in}}%
\pgfpathlineto{\pgfqpoint{2.692113in}{6.251128in}}%
\pgfpathlineto{\pgfqpoint{2.745157in}{6.251128in}}%
\pgfpathlineto{\pgfqpoint{2.798201in}{6.251128in}}%
\pgfpathlineto{\pgfqpoint{2.851245in}{6.251128in}}%
\pgfpathlineto{\pgfqpoint{2.904288in}{6.251128in}}%
\pgfpathlineto{\pgfqpoint{2.957332in}{6.251128in}}%
\pgfpathlineto{\pgfqpoint{3.010376in}{6.251128in}}%
\pgfpathlineto{\pgfqpoint{3.063420in}{6.251128in}}%
\pgfpathlineto{\pgfqpoint{3.116464in}{6.251128in}}%
\pgfpathlineto{\pgfqpoint{3.169508in}{6.251128in}}%
\pgfpathlineto{\pgfqpoint{3.222552in}{6.251128in}}%
\pgfpathlineto{\pgfqpoint{3.275596in}{6.251128in}}%
\pgfpathlineto{\pgfqpoint{3.328640in}{6.251128in}}%
\pgfpathlineto{\pgfqpoint{3.381684in}{6.251128in}}%
\pgfpathlineto{\pgfqpoint{3.434728in}{6.251128in}}%
\pgfpathlineto{\pgfqpoint{3.487772in}{6.251128in}}%
\pgfpathlineto{\pgfqpoint{3.540816in}{6.251128in}}%
\pgfpathlineto{\pgfqpoint{3.593860in}{6.251128in}}%
\pgfpathlineto{\pgfqpoint{3.646904in}{6.251128in}}%
\pgfpathlineto{\pgfqpoint{3.699948in}{6.251128in}}%
\pgfpathlineto{\pgfqpoint{3.752991in}{6.251128in}}%
\pgfpathlineto{\pgfqpoint{3.806035in}{6.251128in}}%
\pgfpathlineto{\pgfqpoint{3.859079in}{6.251128in}}%
\pgfpathlineto{\pgfqpoint{3.912123in}{6.251128in}}%
\pgfpathlineto{\pgfqpoint{3.965167in}{6.251128in}}%
\pgfpathlineto{\pgfqpoint{4.018211in}{6.251128in}}%
\pgfpathlineto{\pgfqpoint{4.071255in}{6.251128in}}%
\pgfpathlineto{\pgfqpoint{4.124299in}{6.251128in}}%
\pgfpathlineto{\pgfqpoint{4.177343in}{6.251128in}}%
\pgfpathlineto{\pgfqpoint{4.230387in}{6.251128in}}%
\pgfpathlineto{\pgfqpoint{4.283431in}{6.251128in}}%
\pgfpathlineto{\pgfqpoint{4.336475in}{6.251128in}}%
\pgfpathlineto{\pgfqpoint{4.389519in}{6.251128in}}%
\pgfpathlineto{\pgfqpoint{4.442563in}{6.251128in}}%
\pgfpathlineto{\pgfqpoint{4.495607in}{6.251128in}}%
\pgfpathlineto{\pgfqpoint{4.548650in}{6.251128in}}%
\pgfpathlineto{\pgfqpoint{4.601694in}{6.251128in}}%
\pgfpathlineto{\pgfqpoint{4.654738in}{6.251128in}}%
\pgfpathlineto{\pgfqpoint{4.707782in}{6.251128in}}%
\pgfpathlineto{\pgfqpoint{4.760826in}{6.251128in}}%
\pgfpathlineto{\pgfqpoint{4.813870in}{6.251128in}}%
\pgfpathlineto{\pgfqpoint{4.866914in}{6.251128in}}%
\pgfpathlineto{\pgfqpoint{4.919958in}{6.251128in}}%
\pgfpathlineto{\pgfqpoint{4.973002in}{6.251128in}}%
\pgfpathlineto{\pgfqpoint{5.026046in}{6.251128in}}%
\pgfpathlineto{\pgfqpoint{5.079090in}{6.251128in}}%
\pgfpathlineto{\pgfqpoint{5.132134in}{6.251128in}}%
\pgfpathlineto{\pgfqpoint{5.185178in}{6.251128in}}%
\pgfpathlineto{\pgfqpoint{5.238222in}{6.251128in}}%
\pgfpathlineto{\pgfqpoint{5.291266in}{6.251128in}}%
\pgfpathlineto{\pgfqpoint{5.344309in}{6.251128in}}%
\pgfpathlineto{\pgfqpoint{5.397353in}{6.251128in}}%
\pgfpathlineto{\pgfqpoint{5.450397in}{6.251128in}}%
\pgfpathlineto{\pgfqpoint{5.503441in}{6.251128in}}%
\pgfpathlineto{\pgfqpoint{5.556485in}{6.251128in}}%
\pgfpathlineto{\pgfqpoint{5.609529in}{6.251128in}}%
\pgfpathlineto{\pgfqpoint{5.662573in}{6.251128in}}%
\pgfpathlineto{\pgfqpoint{5.715617in}{6.251128in}}%
\pgfpathlineto{\pgfqpoint{5.768661in}{6.251128in}}%
\pgfpathlineto{\pgfqpoint{5.821705in}{6.251128in}}%
\pgfpathlineto{\pgfqpoint{5.874749in}{6.251128in}}%
\pgfpathlineto{\pgfqpoint{5.927793in}{6.251128in}}%
\pgfpathlineto{\pgfqpoint{5.980837in}{6.251128in}}%
\pgfpathlineto{\pgfqpoint{6.033881in}{6.251128in}}%
\pgfpathlineto{\pgfqpoint{6.086925in}{6.251128in}}%
\pgfpathlineto{\pgfqpoint{6.139969in}{6.251128in}}%
\pgfpathlineto{\pgfqpoint{6.193012in}{6.251128in}}%
\pgfpathlineto{\pgfqpoint{6.246056in}{6.251128in}}%
\pgfpathlineto{\pgfqpoint{6.299100in}{6.251128in}}%
\pgfpathlineto{\pgfqpoint{6.352144in}{6.251128in}}%
\pgfpathlineto{\pgfqpoint{6.405188in}{6.251128in}}%
\pgfpathlineto{\pgfqpoint{6.458232in}{6.251128in}}%
\pgfpathlineto{\pgfqpoint{6.511276in}{6.251128in}}%
\pgfpathlineto{\pgfqpoint{6.564320in}{6.251128in}}%
\pgfpathlineto{\pgfqpoint{6.617364in}{6.251128in}}%
\pgfpathlineto{\pgfqpoint{6.670408in}{6.251128in}}%
\pgfpathlineto{\pgfqpoint{6.723452in}{6.251128in}}%
\pgfpathlineto{\pgfqpoint{6.776496in}{6.251128in}}%
\pgfpathlineto{\pgfqpoint{6.829540in}{6.251128in}}%
\pgfpathlineto{\pgfqpoint{6.882584in}{6.251128in}}%
\pgfpathlineto{\pgfqpoint{6.935628in}{6.251128in}}%
\pgfpathlineto{\pgfqpoint{6.988671in}{6.251128in}}%
\pgfpathlineto{\pgfqpoint{7.041715in}{6.251128in}}%
\pgfpathlineto{\pgfqpoint{7.094759in}{6.251128in}}%
\pgfpathlineto{\pgfqpoint{7.147803in}{6.251128in}}%
\pgfpathlineto{\pgfqpoint{7.200847in}{6.251128in}}%
\pgfpathlineto{\pgfqpoint{7.253891in}{6.251128in}}%
\pgfpathlineto{\pgfqpoint{7.306935in}{6.251128in}}%
\pgfpathlineto{\pgfqpoint{7.359979in}{6.251128in}}%
\pgfpathlineto{\pgfqpoint{7.413023in}{6.251128in}}%
\pgfpathlineto{\pgfqpoint{7.466067in}{6.251128in}}%
\pgfpathlineto{\pgfqpoint{7.519111in}{6.251128in}}%
\pgfpathlineto{\pgfqpoint{7.572155in}{6.251128in}}%
\pgfpathlineto{\pgfqpoint{7.625199in}{6.251128in}}%
\pgfpathlineto{\pgfqpoint{7.678243in}{6.251128in}}%
\pgfpathlineto{\pgfqpoint{7.731287in}{6.251128in}}%
\pgfpathlineto{\pgfqpoint{7.784330in}{6.251128in}}%
\pgfpathlineto{\pgfqpoint{7.837374in}{6.251128in}}%
\pgfpathlineto{\pgfqpoint{7.890418in}{6.251128in}}%
\pgfpathlineto{\pgfqpoint{7.943462in}{6.251128in}}%
\pgfpathlineto{\pgfqpoint{7.996506in}{6.251128in}}%
\pgfpathlineto{\pgfqpoint{8.049550in}{6.251128in}}%
\pgfpathlineto{\pgfqpoint{8.102594in}{6.251128in}}%
\pgfpathlineto{\pgfqpoint{8.155638in}{6.251128in}}%
\pgfpathlineto{\pgfqpoint{8.208682in}{6.251128in}}%
\pgfpathlineto{\pgfqpoint{8.261726in}{6.251128in}}%
\pgfpathlineto{\pgfqpoint{8.314770in}{6.251128in}}%
\pgfpathlineto{\pgfqpoint{8.367814in}{6.251128in}}%
\pgfpathlineto{\pgfqpoint{8.420858in}{6.251128in}}%
\pgfpathlineto{\pgfqpoint{8.473902in}{6.251128in}}%
\pgfpathlineto{\pgfqpoint{8.526946in}{6.251128in}}%
\pgfpathlineto{\pgfqpoint{8.579990in}{6.251128in}}%
\pgfpathlineto{\pgfqpoint{8.633033in}{6.251128in}}%
\pgfpathlineto{\pgfqpoint{8.686077in}{6.251128in}}%
\pgfpathlineto{\pgfqpoint{8.739121in}{6.251128in}}%
\pgfpathlineto{\pgfqpoint{8.792165in}{6.251128in}}%
\pgfpathlineto{\pgfqpoint{8.845209in}{6.251128in}}%
\pgfpathlineto{\pgfqpoint{8.898253in}{6.251128in}}%
\pgfpathlineto{\pgfqpoint{8.951297in}{6.251128in}}%
\pgfpathlineto{\pgfqpoint{9.004341in}{6.251128in}}%
\pgfpathlineto{\pgfqpoint{9.057385in}{6.251128in}}%
\pgfpathlineto{\pgfqpoint{9.110429in}{6.251128in}}%
\pgfpathlineto{\pgfqpoint{9.163473in}{6.251128in}}%
\pgfpathlineto{\pgfqpoint{9.216517in}{6.251128in}}%
\pgfpathlineto{\pgfqpoint{9.269561in}{6.251128in}}%
\pgfpathlineto{\pgfqpoint{9.322605in}{6.251128in}}%
\pgfpathlineto{\pgfqpoint{9.375649in}{6.251128in}}%
\pgfpathlineto{\pgfqpoint{9.428692in}{6.251128in}}%
\pgfpathlineto{\pgfqpoint{9.481736in}{6.251128in}}%
\pgfpathlineto{\pgfqpoint{9.534780in}{6.251128in}}%
\pgfpathlineto{\pgfqpoint{9.587824in}{6.251128in}}%
\pgfpathlineto{\pgfqpoint{9.640868in}{6.251128in}}%
\pgfpathlineto{\pgfqpoint{9.693912in}{6.251128in}}%
\pgfpathlineto{\pgfqpoint{9.746956in}{6.251128in}}%
\pgfpathlineto{\pgfqpoint{9.800000in}{6.251128in}}%
\pgfpathlineto{\pgfqpoint{9.800000in}{6.251128in}}%
\pgfpathlineto{\pgfqpoint{9.800000in}{6.251128in}}%
\pgfpathlineto{\pgfqpoint{9.746956in}{6.251128in}}%
\pgfpathlineto{\pgfqpoint{9.693912in}{6.251128in}}%
\pgfpathlineto{\pgfqpoint{9.640868in}{6.251128in}}%
\pgfpathlineto{\pgfqpoint{9.587824in}{6.251128in}}%
\pgfpathlineto{\pgfqpoint{9.534780in}{6.287774in}}%
\pgfpathlineto{\pgfqpoint{9.481736in}{6.256205in}}%
\pgfpathlineto{\pgfqpoint{9.428692in}{6.251128in}}%
\pgfpathlineto{\pgfqpoint{9.375649in}{6.251128in}}%
\pgfpathlineto{\pgfqpoint{9.322605in}{6.251128in}}%
\pgfpathlineto{\pgfqpoint{9.269561in}{6.251128in}}%
\pgfpathlineto{\pgfqpoint{9.216517in}{6.334558in}}%
\pgfpathlineto{\pgfqpoint{9.163473in}{6.598964in}}%
\pgfpathlineto{\pgfqpoint{9.110429in}{6.251128in}}%
\pgfpathlineto{\pgfqpoint{9.057385in}{6.251128in}}%
\pgfpathlineto{\pgfqpoint{9.004341in}{6.568668in}}%
\pgfpathlineto{\pgfqpoint{8.951297in}{6.251128in}}%
\pgfpathlineto{\pgfqpoint{8.898253in}{6.251128in}}%
\pgfpathlineto{\pgfqpoint{8.845209in}{6.251128in}}%
\pgfpathlineto{\pgfqpoint{8.792165in}{6.251128in}}%
\pgfpathlineto{\pgfqpoint{8.739121in}{6.251128in}}%
\pgfpathlineto{\pgfqpoint{8.686077in}{6.251128in}}%
\pgfpathlineto{\pgfqpoint{8.633033in}{6.251128in}}%
\pgfpathlineto{\pgfqpoint{8.579990in}{6.560479in}}%
\pgfpathlineto{\pgfqpoint{8.526946in}{6.251128in}}%
\pgfpathlineto{\pgfqpoint{8.473902in}{6.251128in}}%
\pgfpathlineto{\pgfqpoint{8.420858in}{6.371312in}}%
\pgfpathlineto{\pgfqpoint{8.367814in}{6.251128in}}%
\pgfpathlineto{\pgfqpoint{8.314770in}{6.598964in}}%
\pgfpathlineto{\pgfqpoint{8.261726in}{6.598964in}}%
\pgfpathlineto{\pgfqpoint{8.208682in}{6.251128in}}%
\pgfpathlineto{\pgfqpoint{8.155638in}{6.251128in}}%
\pgfpathlineto{\pgfqpoint{8.102594in}{6.598964in}}%
\pgfpathlineto{\pgfqpoint{8.049550in}{6.598964in}}%
\pgfpathlineto{\pgfqpoint{7.996506in}{6.251128in}}%
\pgfpathlineto{\pgfqpoint{7.943462in}{6.251128in}}%
\pgfpathlineto{\pgfqpoint{7.890418in}{6.251128in}}%
\pgfpathlineto{\pgfqpoint{7.837374in}{6.251128in}}%
\pgfpathlineto{\pgfqpoint{7.784330in}{6.251128in}}%
\pgfpathlineto{\pgfqpoint{7.731287in}{6.251128in}}%
\pgfpathlineto{\pgfqpoint{7.678243in}{6.251128in}}%
\pgfpathlineto{\pgfqpoint{7.625199in}{6.251128in}}%
\pgfpathlineto{\pgfqpoint{7.572155in}{6.251128in}}%
\pgfpathlineto{\pgfqpoint{7.519111in}{6.251128in}}%
\pgfpathlineto{\pgfqpoint{7.466067in}{6.251128in}}%
\pgfpathlineto{\pgfqpoint{7.413023in}{6.251128in}}%
\pgfpathlineto{\pgfqpoint{7.359979in}{6.350974in}}%
\pgfpathlineto{\pgfqpoint{7.306935in}{6.251128in}}%
\pgfpathlineto{\pgfqpoint{7.253891in}{6.451777in}}%
\pgfpathlineto{\pgfqpoint{7.200847in}{6.598964in}}%
\pgfpathlineto{\pgfqpoint{7.147803in}{6.251128in}}%
\pgfpathlineto{\pgfqpoint{7.094759in}{6.251128in}}%
\pgfpathlineto{\pgfqpoint{7.041715in}{6.251128in}}%
\pgfpathlineto{\pgfqpoint{6.988671in}{6.598964in}}%
\pgfpathlineto{\pgfqpoint{6.935628in}{6.251128in}}%
\pgfpathlineto{\pgfqpoint{6.882584in}{6.598964in}}%
\pgfpathlineto{\pgfqpoint{6.829540in}{6.251128in}}%
\pgfpathlineto{\pgfqpoint{6.776496in}{6.385890in}}%
\pgfpathlineto{\pgfqpoint{6.723452in}{6.251128in}}%
\pgfpathlineto{\pgfqpoint{6.670408in}{6.251128in}}%
\pgfpathlineto{\pgfqpoint{6.617364in}{6.251128in}}%
\pgfpathlineto{\pgfqpoint{6.564320in}{6.251128in}}%
\pgfpathlineto{\pgfqpoint{6.511276in}{6.251128in}}%
\pgfpathlineto{\pgfqpoint{6.458232in}{6.251128in}}%
\pgfpathlineto{\pgfqpoint{6.405188in}{6.251128in}}%
\pgfpathlineto{\pgfqpoint{6.352144in}{6.569516in}}%
\pgfpathlineto{\pgfqpoint{6.299100in}{6.251128in}}%
\pgfpathlineto{\pgfqpoint{6.246056in}{6.362599in}}%
\pgfpathlineto{\pgfqpoint{6.193012in}{6.598964in}}%
\pgfpathlineto{\pgfqpoint{6.139969in}{6.251128in}}%
\pgfpathlineto{\pgfqpoint{6.086925in}{6.251128in}}%
\pgfpathlineto{\pgfqpoint{6.033881in}{6.400407in}}%
\pgfpathlineto{\pgfqpoint{5.980837in}{6.251128in}}%
\pgfpathlineto{\pgfqpoint{5.927793in}{6.251128in}}%
\pgfpathlineto{\pgfqpoint{5.874749in}{6.251128in}}%
\pgfpathlineto{\pgfqpoint{5.821705in}{6.251128in}}%
\pgfpathlineto{\pgfqpoint{5.768661in}{6.324051in}}%
\pgfpathlineto{\pgfqpoint{5.715617in}{6.251128in}}%
\pgfpathlineto{\pgfqpoint{5.662573in}{6.251128in}}%
\pgfpathlineto{\pgfqpoint{5.609529in}{6.251128in}}%
\pgfpathlineto{\pgfqpoint{5.556485in}{6.329843in}}%
\pgfpathlineto{\pgfqpoint{5.503441in}{6.251128in}}%
\pgfpathlineto{\pgfqpoint{5.450397in}{6.251128in}}%
\pgfpathlineto{\pgfqpoint{5.397353in}{6.251128in}}%
\pgfpathlineto{\pgfqpoint{5.344309in}{6.471024in}}%
\pgfpathlineto{\pgfqpoint{5.291266in}{6.251128in}}%
\pgfpathlineto{\pgfqpoint{5.238222in}{6.251128in}}%
\pgfpathlineto{\pgfqpoint{5.185178in}{6.251128in}}%
\pgfpathlineto{\pgfqpoint{5.132134in}{6.598964in}}%
\pgfpathlineto{\pgfqpoint{5.079090in}{6.251128in}}%
\pgfpathlineto{\pgfqpoint{5.026046in}{6.590550in}}%
\pgfpathlineto{\pgfqpoint{4.973002in}{6.251128in}}%
\pgfpathlineto{\pgfqpoint{4.919958in}{6.251128in}}%
\pgfpathlineto{\pgfqpoint{4.866914in}{6.471548in}}%
\pgfpathlineto{\pgfqpoint{4.813870in}{6.251128in}}%
\pgfpathlineto{\pgfqpoint{4.760826in}{6.251128in}}%
\pgfpathlineto{\pgfqpoint{4.707782in}{6.351811in}}%
\pgfpathlineto{\pgfqpoint{4.654738in}{6.251128in}}%
\pgfpathlineto{\pgfqpoint{4.601694in}{6.251128in}}%
\pgfpathlineto{\pgfqpoint{4.548650in}{6.251128in}}%
\pgfpathlineto{\pgfqpoint{4.495607in}{6.437705in}}%
\pgfpathlineto{\pgfqpoint{4.442563in}{6.598964in}}%
\pgfpathlineto{\pgfqpoint{4.389519in}{6.299258in}}%
\pgfpathlineto{\pgfqpoint{4.336475in}{6.598964in}}%
\pgfpathlineto{\pgfqpoint{4.283431in}{6.251128in}}%
\pgfpathlineto{\pgfqpoint{4.230387in}{6.251128in}}%
\pgfpathlineto{\pgfqpoint{4.177343in}{6.251128in}}%
\pgfpathlineto{\pgfqpoint{4.124299in}{6.251128in}}%
\pgfpathlineto{\pgfqpoint{4.071255in}{6.251128in}}%
\pgfpathlineto{\pgfqpoint{4.018211in}{6.251128in}}%
\pgfpathlineto{\pgfqpoint{3.965167in}{6.340259in}}%
\pgfpathlineto{\pgfqpoint{3.912123in}{6.481634in}}%
\pgfpathlineto{\pgfqpoint{3.859079in}{6.572998in}}%
\pgfpathlineto{\pgfqpoint{3.806035in}{6.251128in}}%
\pgfpathlineto{\pgfqpoint{3.752991in}{6.510498in}}%
\pgfpathlineto{\pgfqpoint{3.699948in}{6.251128in}}%
\pgfpathlineto{\pgfqpoint{3.646904in}{6.251128in}}%
\pgfpathlineto{\pgfqpoint{3.593860in}{6.251128in}}%
\pgfpathlineto{\pgfqpoint{3.540816in}{6.427614in}}%
\pgfpathlineto{\pgfqpoint{3.487772in}{6.598964in}}%
\pgfpathlineto{\pgfqpoint{3.434728in}{6.251128in}}%
\pgfpathlineto{\pgfqpoint{3.381684in}{6.598964in}}%
\pgfpathlineto{\pgfqpoint{3.328640in}{6.251128in}}%
\pgfpathlineto{\pgfqpoint{3.275596in}{6.251128in}}%
\pgfpathlineto{\pgfqpoint{3.222552in}{6.251128in}}%
\pgfpathlineto{\pgfqpoint{3.169508in}{6.251128in}}%
\pgfpathlineto{\pgfqpoint{3.116464in}{6.598964in}}%
\pgfpathlineto{\pgfqpoint{3.063420in}{6.598964in}}%
\pgfpathlineto{\pgfqpoint{3.010376in}{6.251128in}}%
\pgfpathlineto{\pgfqpoint{2.957332in}{6.598964in}}%
\pgfpathlineto{\pgfqpoint{2.904288in}{6.251128in}}%
\pgfpathlineto{\pgfqpoint{2.851245in}{6.598964in}}%
\pgfpathlineto{\pgfqpoint{2.798201in}{6.251128in}}%
\pgfpathlineto{\pgfqpoint{2.745157in}{6.251128in}}%
\pgfpathlineto{\pgfqpoint{2.692113in}{6.251128in}}%
\pgfpathlineto{\pgfqpoint{2.639069in}{6.251128in}}%
\pgfpathlineto{\pgfqpoint{2.586025in}{6.251128in}}%
\pgfpathlineto{\pgfqpoint{2.532981in}{6.251128in}}%
\pgfpathlineto{\pgfqpoint{2.479937in}{6.251128in}}%
\pgfpathlineto{\pgfqpoint{2.426893in}{6.251128in}}%
\pgfpathlineto{\pgfqpoint{2.373849in}{6.541500in}}%
\pgfpathlineto{\pgfqpoint{2.320805in}{6.251128in}}%
\pgfpathlineto{\pgfqpoint{2.267761in}{6.251128in}}%
\pgfpathlineto{\pgfqpoint{2.214717in}{6.459099in}}%
\pgfpathlineto{\pgfqpoint{2.161673in}{6.251128in}}%
\pgfpathlineto{\pgfqpoint{2.108629in}{6.348747in}}%
\pgfpathlineto{\pgfqpoint{2.055586in}{6.251128in}}%
\pgfpathlineto{\pgfqpoint{2.002542in}{6.505445in}}%
\pgfpathlineto{\pgfqpoint{1.949498in}{6.251128in}}%
\pgfpathlineto{\pgfqpoint{1.896454in}{6.598964in}}%
\pgfpathlineto{\pgfqpoint{1.843410in}{6.598964in}}%
\pgfpathlineto{\pgfqpoint{1.790366in}{6.251128in}}%
\pgfpathlineto{\pgfqpoint{1.737322in}{6.251128in}}%
\pgfpathlineto{\pgfqpoint{1.684278in}{6.598964in}}%
\pgfpathlineto{\pgfqpoint{1.631234in}{6.598964in}}%
\pgfpathlineto{\pgfqpoint{1.578190in}{6.251128in}}%
\pgfpathlineto{\pgfqpoint{1.525146in}{6.251128in}}%
\pgfpathlineto{\pgfqpoint{1.472102in}{6.251128in}}%
\pgfpathlineto{\pgfqpoint{1.419058in}{6.251128in}}%
\pgfpathlineto{\pgfqpoint{1.366014in}{6.251128in}}%
\pgfpathlineto{\pgfqpoint{1.312970in}{6.598964in}}%
\pgfpathlineto{\pgfqpoint{1.259927in}{6.598964in}}%
\pgfpathlineto{\pgfqpoint{1.206883in}{6.598964in}}%
\pgfpathlineto{\pgfqpoint{1.153839in}{6.251128in}}%
\pgfpathlineto{\pgfqpoint{1.100795in}{6.251128in}}%
\pgfpathlineto{\pgfqpoint{1.047751in}{6.251128in}}%
\pgfpathlineto{\pgfqpoint{0.994707in}{6.261680in}}%
\pgfpathlineto{\pgfqpoint{0.941663in}{6.251128in}}%
\pgfpathlineto{\pgfqpoint{0.941663in}{6.251128in}}%
\pgfpathclose%
\pgfusepath{stroke,fill}%
}%
\begin{pgfscope}%
\pgfsys@transformshift{0.000000in}{0.000000in}%
\pgfsys@useobject{currentmarker}{}%
\end{pgfscope}%
\end{pgfscope}%
\begin{pgfscope}%
\pgfpathrectangle{\pgfqpoint{0.941663in}{4.334375in}}{\pgfqpoint{8.858337in}{3.465625in}}%
\pgfusepath{clip}%
\pgfsetrectcap%
\pgfsetroundjoin%
\pgfsetlinewidth{1.505625pt}%
\definecolor{currentstroke}{rgb}{0.549020,0.337255,0.294118}%
\pgfsetstrokecolor{currentstroke}%
\pgfsetdash{}{0pt}%
\pgfpathmoveto{\pgfqpoint{0.941663in}{6.251128in}}%
\pgfpathlineto{\pgfqpoint{0.994707in}{6.261680in}}%
\pgfpathlineto{\pgfqpoint{1.047751in}{6.251128in}}%
\pgfpathlineto{\pgfqpoint{1.153839in}{6.251128in}}%
\pgfpathlineto{\pgfqpoint{1.206883in}{6.598964in}}%
\pgfpathlineto{\pgfqpoint{1.312970in}{6.598964in}}%
\pgfpathlineto{\pgfqpoint{1.366014in}{6.251128in}}%
\pgfpathlineto{\pgfqpoint{1.578190in}{6.251128in}}%
\pgfpathlineto{\pgfqpoint{1.631234in}{6.598964in}}%
\pgfpathlineto{\pgfqpoint{1.684278in}{6.598964in}}%
\pgfpathlineto{\pgfqpoint{1.737322in}{6.251128in}}%
\pgfpathlineto{\pgfqpoint{1.790366in}{6.251128in}}%
\pgfpathlineto{\pgfqpoint{1.843410in}{6.734592in}}%
\pgfpathlineto{\pgfqpoint{1.896454in}{6.883587in}}%
\pgfpathlineto{\pgfqpoint{1.949498in}{6.251128in}}%
\pgfpathlineto{\pgfqpoint{2.002542in}{6.724358in}}%
\pgfpathlineto{\pgfqpoint{2.055586in}{6.251128in}}%
\pgfpathlineto{\pgfqpoint{2.108629in}{6.348747in}}%
\pgfpathlineto{\pgfqpoint{2.161673in}{6.251128in}}%
\pgfpathlineto{\pgfqpoint{2.214717in}{6.574971in}}%
\pgfpathlineto{\pgfqpoint{2.267761in}{6.251128in}}%
\pgfpathlineto{\pgfqpoint{2.320805in}{6.251128in}}%
\pgfpathlineto{\pgfqpoint{2.373849in}{6.541500in}}%
\pgfpathlineto{\pgfqpoint{2.426893in}{6.606651in}}%
\pgfpathlineto{\pgfqpoint{2.479937in}{6.251128in}}%
\pgfpathlineto{\pgfqpoint{2.798201in}{6.251128in}}%
\pgfpathlineto{\pgfqpoint{2.851245in}{6.598964in}}%
\pgfpathlineto{\pgfqpoint{2.904288in}{6.251128in}}%
\pgfpathlineto{\pgfqpoint{2.957332in}{6.744748in}}%
\pgfpathlineto{\pgfqpoint{3.010376in}{6.251128in}}%
\pgfpathlineto{\pgfqpoint{3.063420in}{6.598964in}}%
\pgfpathlineto{\pgfqpoint{3.116464in}{6.598964in}}%
\pgfpathlineto{\pgfqpoint{3.169508in}{6.251128in}}%
\pgfpathlineto{\pgfqpoint{3.328640in}{6.251128in}}%
\pgfpathlineto{\pgfqpoint{3.381684in}{6.598964in}}%
\pgfpathlineto{\pgfqpoint{3.434728in}{6.251128in}}%
\pgfpathlineto{\pgfqpoint{3.487772in}{6.598964in}}%
\pgfpathlineto{\pgfqpoint{3.540816in}{6.572638in}}%
\pgfpathlineto{\pgfqpoint{3.593860in}{6.251128in}}%
\pgfpathlineto{\pgfqpoint{3.699948in}{6.251128in}}%
\pgfpathlineto{\pgfqpoint{3.752991in}{6.510498in}}%
\pgfpathlineto{\pgfqpoint{3.806035in}{6.251128in}}%
\pgfpathlineto{\pgfqpoint{3.859079in}{6.572998in}}%
\pgfpathlineto{\pgfqpoint{3.912123in}{6.481634in}}%
\pgfpathlineto{\pgfqpoint{3.965167in}{6.340259in}}%
\pgfpathlineto{\pgfqpoint{4.018211in}{6.251128in}}%
\pgfpathlineto{\pgfqpoint{4.283431in}{6.251128in}}%
\pgfpathlineto{\pgfqpoint{4.336475in}{6.889082in}}%
\pgfpathlineto{\pgfqpoint{4.389519in}{6.299258in}}%
\pgfpathlineto{\pgfqpoint{4.442563in}{6.598964in}}%
\pgfpathlineto{\pgfqpoint{4.495607in}{6.633012in}}%
\pgfpathlineto{\pgfqpoint{4.548650in}{6.251128in}}%
\pgfpathlineto{\pgfqpoint{4.601694in}{6.592100in}}%
\pgfpathlineto{\pgfqpoint{4.654738in}{6.251128in}}%
\pgfpathlineto{\pgfqpoint{4.707782in}{6.766257in}}%
\pgfpathlineto{\pgfqpoint{4.760826in}{6.251128in}}%
\pgfpathlineto{\pgfqpoint{4.813870in}{6.251128in}}%
\pgfpathlineto{\pgfqpoint{4.866914in}{6.548881in}}%
\pgfpathlineto{\pgfqpoint{4.919958in}{6.528186in}}%
\pgfpathlineto{\pgfqpoint{4.973002in}{6.251128in}}%
\pgfpathlineto{\pgfqpoint{5.026046in}{6.590550in}}%
\pgfpathlineto{\pgfqpoint{5.079090in}{6.251128in}}%
\pgfpathlineto{\pgfqpoint{5.132134in}{6.598964in}}%
\pgfpathlineto{\pgfqpoint{5.185178in}{6.574350in}}%
\pgfpathlineto{\pgfqpoint{5.238222in}{6.501650in}}%
\pgfpathlineto{\pgfqpoint{5.291266in}{6.251128in}}%
\pgfpathlineto{\pgfqpoint{5.344309in}{6.471024in}}%
\pgfpathlineto{\pgfqpoint{5.397353in}{6.325005in}}%
\pgfpathlineto{\pgfqpoint{5.450397in}{6.770026in}}%
\pgfpathlineto{\pgfqpoint{5.503441in}{6.251128in}}%
\pgfpathlineto{\pgfqpoint{5.556485in}{6.329843in}}%
\pgfpathlineto{\pgfqpoint{5.609529in}{6.514446in}}%
\pgfpathlineto{\pgfqpoint{5.662573in}{6.375963in}}%
\pgfpathlineto{\pgfqpoint{5.715617in}{6.251128in}}%
\pgfpathlineto{\pgfqpoint{5.768661in}{6.913209in}}%
\pgfpathlineto{\pgfqpoint{5.821705in}{6.872997in}}%
\pgfpathlineto{\pgfqpoint{5.874749in}{6.702075in}}%
\pgfpathlineto{\pgfqpoint{5.927793in}{6.598351in}}%
\pgfpathlineto{\pgfqpoint{5.980837in}{6.251128in}}%
\pgfpathlineto{\pgfqpoint{6.033881in}{6.400407in}}%
\pgfpathlineto{\pgfqpoint{6.086925in}{6.251128in}}%
\pgfpathlineto{\pgfqpoint{6.139969in}{6.251128in}}%
\pgfpathlineto{\pgfqpoint{6.193012in}{6.635398in}}%
\pgfpathlineto{\pgfqpoint{6.246056in}{6.691111in}}%
\pgfpathlineto{\pgfqpoint{6.299100in}{6.251128in}}%
\pgfpathlineto{\pgfqpoint{6.352144in}{6.569516in}}%
\pgfpathlineto{\pgfqpoint{6.405188in}{6.364054in}}%
\pgfpathlineto{\pgfqpoint{6.458232in}{6.251128in}}%
\pgfpathlineto{\pgfqpoint{6.723452in}{6.251128in}}%
\pgfpathlineto{\pgfqpoint{6.776496in}{6.385890in}}%
\pgfpathlineto{\pgfqpoint{6.829540in}{6.251128in}}%
\pgfpathlineto{\pgfqpoint{6.882584in}{6.780238in}}%
\pgfpathlineto{\pgfqpoint{6.935628in}{6.251128in}}%
\pgfpathlineto{\pgfqpoint{6.988671in}{6.598964in}}%
\pgfpathlineto{\pgfqpoint{7.041715in}{6.251128in}}%
\pgfpathlineto{\pgfqpoint{7.147803in}{6.251128in}}%
\pgfpathlineto{\pgfqpoint{7.200847in}{6.729225in}}%
\pgfpathlineto{\pgfqpoint{7.253891in}{6.451777in}}%
\pgfpathlineto{\pgfqpoint{7.306935in}{6.251128in}}%
\pgfpathlineto{\pgfqpoint{7.359979in}{6.470643in}}%
\pgfpathlineto{\pgfqpoint{7.413023in}{6.529690in}}%
\pgfpathlineto{\pgfqpoint{7.466067in}{6.251128in}}%
\pgfpathlineto{\pgfqpoint{7.519111in}{6.251128in}}%
\pgfpathlineto{\pgfqpoint{7.572155in}{6.487962in}}%
\pgfpathlineto{\pgfqpoint{7.625199in}{6.652740in}}%
\pgfpathlineto{\pgfqpoint{7.678243in}{6.251128in}}%
\pgfpathlineto{\pgfqpoint{7.996506in}{6.251128in}}%
\pgfpathlineto{\pgfqpoint{8.049550in}{6.766484in}}%
\pgfpathlineto{\pgfqpoint{8.102594in}{6.782745in}}%
\pgfpathlineto{\pgfqpoint{8.155638in}{6.251128in}}%
\pgfpathlineto{\pgfqpoint{8.208682in}{6.251128in}}%
\pgfpathlineto{\pgfqpoint{8.261726in}{6.924712in}}%
\pgfpathlineto{\pgfqpoint{8.314770in}{6.876087in}}%
\pgfpathlineto{\pgfqpoint{8.367814in}{6.251128in}}%
\pgfpathlineto{\pgfqpoint{8.420858in}{6.653636in}}%
\pgfpathlineto{\pgfqpoint{8.473902in}{6.251128in}}%
\pgfpathlineto{\pgfqpoint{8.526946in}{6.251128in}}%
\pgfpathlineto{\pgfqpoint{8.579990in}{6.685501in}}%
\pgfpathlineto{\pgfqpoint{8.633033in}{6.520595in}}%
\pgfpathlineto{\pgfqpoint{8.686077in}{6.723687in}}%
\pgfpathlineto{\pgfqpoint{8.739121in}{6.489640in}}%
\pgfpathlineto{\pgfqpoint{8.792165in}{6.251128in}}%
\pgfpathlineto{\pgfqpoint{8.845209in}{6.299108in}}%
\pgfpathlineto{\pgfqpoint{8.898253in}{6.354195in}}%
\pgfpathlineto{\pgfqpoint{8.951297in}{6.251128in}}%
\pgfpathlineto{\pgfqpoint{9.004341in}{6.568668in}}%
\pgfpathlineto{\pgfqpoint{9.057385in}{6.251128in}}%
\pgfpathlineto{\pgfqpoint{9.110429in}{6.251128in}}%
\pgfpathlineto{\pgfqpoint{9.163473in}{6.668425in}}%
\pgfpathlineto{\pgfqpoint{9.216517in}{6.804492in}}%
\pgfpathlineto{\pgfqpoint{9.269561in}{6.628250in}}%
\pgfpathlineto{\pgfqpoint{9.322605in}{6.727450in}}%
\pgfpathlineto{\pgfqpoint{9.375649in}{6.754769in}}%
\pgfpathlineto{\pgfqpoint{9.428692in}{6.251128in}}%
\pgfpathlineto{\pgfqpoint{9.481736in}{6.256205in}}%
\pgfpathlineto{\pgfqpoint{9.534780in}{6.946800in}}%
\pgfpathlineto{\pgfqpoint{9.587824in}{6.621011in}}%
\pgfpathlineto{\pgfqpoint{9.640868in}{6.251128in}}%
\pgfpathlineto{\pgfqpoint{9.693912in}{6.632876in}}%
\pgfpathlineto{\pgfqpoint{9.746956in}{6.779712in}}%
\pgfpathlineto{\pgfqpoint{9.800000in}{6.573653in}}%
\pgfpathlineto{\pgfqpoint{9.800000in}{6.573653in}}%
\pgfusepath{stroke}%
\end{pgfscope}%
\begin{pgfscope}%
\pgfpathrectangle{\pgfqpoint{0.941663in}{4.334375in}}{\pgfqpoint{8.858337in}{3.465625in}}%
\pgfusepath{clip}%
\pgfsetbuttcap%
\pgfsetroundjoin%
\definecolor{currentfill}{rgb}{0.549020,0.337255,0.294118}%
\pgfsetfillcolor{currentfill}%
\pgfsetlinewidth{1.003750pt}%
\definecolor{currentstroke}{rgb}{0.549020,0.337255,0.294118}%
\pgfsetstrokecolor{currentstroke}%
\pgfsetdash{}{0pt}%
\pgfsys@defobject{currentmarker}{\pgfqpoint{0.941663in}{6.251128in}}{\pgfqpoint{9.800000in}{6.946800in}}{%
\pgfpathmoveto{\pgfqpoint{0.941663in}{6.251128in}}%
\pgfpathlineto{\pgfqpoint{0.941663in}{6.251128in}}%
\pgfpathlineto{\pgfqpoint{0.994707in}{6.261680in}}%
\pgfpathlineto{\pgfqpoint{1.047751in}{6.251128in}}%
\pgfpathlineto{\pgfqpoint{1.100795in}{6.251128in}}%
\pgfpathlineto{\pgfqpoint{1.153839in}{6.251128in}}%
\pgfpathlineto{\pgfqpoint{1.206883in}{6.598964in}}%
\pgfpathlineto{\pgfqpoint{1.259927in}{6.598964in}}%
\pgfpathlineto{\pgfqpoint{1.312970in}{6.598964in}}%
\pgfpathlineto{\pgfqpoint{1.366014in}{6.251128in}}%
\pgfpathlineto{\pgfqpoint{1.419058in}{6.251128in}}%
\pgfpathlineto{\pgfqpoint{1.472102in}{6.251128in}}%
\pgfpathlineto{\pgfqpoint{1.525146in}{6.251128in}}%
\pgfpathlineto{\pgfqpoint{1.578190in}{6.251128in}}%
\pgfpathlineto{\pgfqpoint{1.631234in}{6.598964in}}%
\pgfpathlineto{\pgfqpoint{1.684278in}{6.598964in}}%
\pgfpathlineto{\pgfqpoint{1.737322in}{6.251128in}}%
\pgfpathlineto{\pgfqpoint{1.790366in}{6.251128in}}%
\pgfpathlineto{\pgfqpoint{1.843410in}{6.598964in}}%
\pgfpathlineto{\pgfqpoint{1.896454in}{6.598964in}}%
\pgfpathlineto{\pgfqpoint{1.949498in}{6.251128in}}%
\pgfpathlineto{\pgfqpoint{2.002542in}{6.505445in}}%
\pgfpathlineto{\pgfqpoint{2.055586in}{6.251128in}}%
\pgfpathlineto{\pgfqpoint{2.108629in}{6.348747in}}%
\pgfpathlineto{\pgfqpoint{2.161673in}{6.251128in}}%
\pgfpathlineto{\pgfqpoint{2.214717in}{6.459099in}}%
\pgfpathlineto{\pgfqpoint{2.267761in}{6.251128in}}%
\pgfpathlineto{\pgfqpoint{2.320805in}{6.251128in}}%
\pgfpathlineto{\pgfqpoint{2.373849in}{6.541500in}}%
\pgfpathlineto{\pgfqpoint{2.426893in}{6.251128in}}%
\pgfpathlineto{\pgfqpoint{2.479937in}{6.251128in}}%
\pgfpathlineto{\pgfqpoint{2.532981in}{6.251128in}}%
\pgfpathlineto{\pgfqpoint{2.586025in}{6.251128in}}%
\pgfpathlineto{\pgfqpoint{2.639069in}{6.251128in}}%
\pgfpathlineto{\pgfqpoint{2.692113in}{6.251128in}}%
\pgfpathlineto{\pgfqpoint{2.745157in}{6.251128in}}%
\pgfpathlineto{\pgfqpoint{2.798201in}{6.251128in}}%
\pgfpathlineto{\pgfqpoint{2.851245in}{6.598964in}}%
\pgfpathlineto{\pgfqpoint{2.904288in}{6.251128in}}%
\pgfpathlineto{\pgfqpoint{2.957332in}{6.598964in}}%
\pgfpathlineto{\pgfqpoint{3.010376in}{6.251128in}}%
\pgfpathlineto{\pgfqpoint{3.063420in}{6.598964in}}%
\pgfpathlineto{\pgfqpoint{3.116464in}{6.598964in}}%
\pgfpathlineto{\pgfqpoint{3.169508in}{6.251128in}}%
\pgfpathlineto{\pgfqpoint{3.222552in}{6.251128in}}%
\pgfpathlineto{\pgfqpoint{3.275596in}{6.251128in}}%
\pgfpathlineto{\pgfqpoint{3.328640in}{6.251128in}}%
\pgfpathlineto{\pgfqpoint{3.381684in}{6.598964in}}%
\pgfpathlineto{\pgfqpoint{3.434728in}{6.251128in}}%
\pgfpathlineto{\pgfqpoint{3.487772in}{6.598964in}}%
\pgfpathlineto{\pgfqpoint{3.540816in}{6.427614in}}%
\pgfpathlineto{\pgfqpoint{3.593860in}{6.251128in}}%
\pgfpathlineto{\pgfqpoint{3.646904in}{6.251128in}}%
\pgfpathlineto{\pgfqpoint{3.699948in}{6.251128in}}%
\pgfpathlineto{\pgfqpoint{3.752991in}{6.510498in}}%
\pgfpathlineto{\pgfqpoint{3.806035in}{6.251128in}}%
\pgfpathlineto{\pgfqpoint{3.859079in}{6.572998in}}%
\pgfpathlineto{\pgfqpoint{3.912123in}{6.481634in}}%
\pgfpathlineto{\pgfqpoint{3.965167in}{6.340259in}}%
\pgfpathlineto{\pgfqpoint{4.018211in}{6.251128in}}%
\pgfpathlineto{\pgfqpoint{4.071255in}{6.251128in}}%
\pgfpathlineto{\pgfqpoint{4.124299in}{6.251128in}}%
\pgfpathlineto{\pgfqpoint{4.177343in}{6.251128in}}%
\pgfpathlineto{\pgfqpoint{4.230387in}{6.251128in}}%
\pgfpathlineto{\pgfqpoint{4.283431in}{6.251128in}}%
\pgfpathlineto{\pgfqpoint{4.336475in}{6.598964in}}%
\pgfpathlineto{\pgfqpoint{4.389519in}{6.299258in}}%
\pgfpathlineto{\pgfqpoint{4.442563in}{6.598964in}}%
\pgfpathlineto{\pgfqpoint{4.495607in}{6.437705in}}%
\pgfpathlineto{\pgfqpoint{4.548650in}{6.251128in}}%
\pgfpathlineto{\pgfqpoint{4.601694in}{6.251128in}}%
\pgfpathlineto{\pgfqpoint{4.654738in}{6.251128in}}%
\pgfpathlineto{\pgfqpoint{4.707782in}{6.351811in}}%
\pgfpathlineto{\pgfqpoint{4.760826in}{6.251128in}}%
\pgfpathlineto{\pgfqpoint{4.813870in}{6.251128in}}%
\pgfpathlineto{\pgfqpoint{4.866914in}{6.471548in}}%
\pgfpathlineto{\pgfqpoint{4.919958in}{6.251128in}}%
\pgfpathlineto{\pgfqpoint{4.973002in}{6.251128in}}%
\pgfpathlineto{\pgfqpoint{5.026046in}{6.590550in}}%
\pgfpathlineto{\pgfqpoint{5.079090in}{6.251128in}}%
\pgfpathlineto{\pgfqpoint{5.132134in}{6.598964in}}%
\pgfpathlineto{\pgfqpoint{5.185178in}{6.251128in}}%
\pgfpathlineto{\pgfqpoint{5.238222in}{6.251128in}}%
\pgfpathlineto{\pgfqpoint{5.291266in}{6.251128in}}%
\pgfpathlineto{\pgfqpoint{5.344309in}{6.471024in}}%
\pgfpathlineto{\pgfqpoint{5.397353in}{6.251128in}}%
\pgfpathlineto{\pgfqpoint{5.450397in}{6.251128in}}%
\pgfpathlineto{\pgfqpoint{5.503441in}{6.251128in}}%
\pgfpathlineto{\pgfqpoint{5.556485in}{6.329843in}}%
\pgfpathlineto{\pgfqpoint{5.609529in}{6.251128in}}%
\pgfpathlineto{\pgfqpoint{5.662573in}{6.251128in}}%
\pgfpathlineto{\pgfqpoint{5.715617in}{6.251128in}}%
\pgfpathlineto{\pgfqpoint{5.768661in}{6.324051in}}%
\pgfpathlineto{\pgfqpoint{5.821705in}{6.251128in}}%
\pgfpathlineto{\pgfqpoint{5.874749in}{6.251128in}}%
\pgfpathlineto{\pgfqpoint{5.927793in}{6.251128in}}%
\pgfpathlineto{\pgfqpoint{5.980837in}{6.251128in}}%
\pgfpathlineto{\pgfqpoint{6.033881in}{6.400407in}}%
\pgfpathlineto{\pgfqpoint{6.086925in}{6.251128in}}%
\pgfpathlineto{\pgfqpoint{6.139969in}{6.251128in}}%
\pgfpathlineto{\pgfqpoint{6.193012in}{6.598964in}}%
\pgfpathlineto{\pgfqpoint{6.246056in}{6.362599in}}%
\pgfpathlineto{\pgfqpoint{6.299100in}{6.251128in}}%
\pgfpathlineto{\pgfqpoint{6.352144in}{6.569516in}}%
\pgfpathlineto{\pgfqpoint{6.405188in}{6.251128in}}%
\pgfpathlineto{\pgfqpoint{6.458232in}{6.251128in}}%
\pgfpathlineto{\pgfqpoint{6.511276in}{6.251128in}}%
\pgfpathlineto{\pgfqpoint{6.564320in}{6.251128in}}%
\pgfpathlineto{\pgfqpoint{6.617364in}{6.251128in}}%
\pgfpathlineto{\pgfqpoint{6.670408in}{6.251128in}}%
\pgfpathlineto{\pgfqpoint{6.723452in}{6.251128in}}%
\pgfpathlineto{\pgfqpoint{6.776496in}{6.385890in}}%
\pgfpathlineto{\pgfqpoint{6.829540in}{6.251128in}}%
\pgfpathlineto{\pgfqpoint{6.882584in}{6.598964in}}%
\pgfpathlineto{\pgfqpoint{6.935628in}{6.251128in}}%
\pgfpathlineto{\pgfqpoint{6.988671in}{6.598964in}}%
\pgfpathlineto{\pgfqpoint{7.041715in}{6.251128in}}%
\pgfpathlineto{\pgfqpoint{7.094759in}{6.251128in}}%
\pgfpathlineto{\pgfqpoint{7.147803in}{6.251128in}}%
\pgfpathlineto{\pgfqpoint{7.200847in}{6.598964in}}%
\pgfpathlineto{\pgfqpoint{7.253891in}{6.451777in}}%
\pgfpathlineto{\pgfqpoint{7.306935in}{6.251128in}}%
\pgfpathlineto{\pgfqpoint{7.359979in}{6.350974in}}%
\pgfpathlineto{\pgfqpoint{7.413023in}{6.251128in}}%
\pgfpathlineto{\pgfqpoint{7.466067in}{6.251128in}}%
\pgfpathlineto{\pgfqpoint{7.519111in}{6.251128in}}%
\pgfpathlineto{\pgfqpoint{7.572155in}{6.251128in}}%
\pgfpathlineto{\pgfqpoint{7.625199in}{6.251128in}}%
\pgfpathlineto{\pgfqpoint{7.678243in}{6.251128in}}%
\pgfpathlineto{\pgfqpoint{7.731287in}{6.251128in}}%
\pgfpathlineto{\pgfqpoint{7.784330in}{6.251128in}}%
\pgfpathlineto{\pgfqpoint{7.837374in}{6.251128in}}%
\pgfpathlineto{\pgfqpoint{7.890418in}{6.251128in}}%
\pgfpathlineto{\pgfqpoint{7.943462in}{6.251128in}}%
\pgfpathlineto{\pgfqpoint{7.996506in}{6.251128in}}%
\pgfpathlineto{\pgfqpoint{8.049550in}{6.598964in}}%
\pgfpathlineto{\pgfqpoint{8.102594in}{6.598964in}}%
\pgfpathlineto{\pgfqpoint{8.155638in}{6.251128in}}%
\pgfpathlineto{\pgfqpoint{8.208682in}{6.251128in}}%
\pgfpathlineto{\pgfqpoint{8.261726in}{6.598964in}}%
\pgfpathlineto{\pgfqpoint{8.314770in}{6.598964in}}%
\pgfpathlineto{\pgfqpoint{8.367814in}{6.251128in}}%
\pgfpathlineto{\pgfqpoint{8.420858in}{6.371312in}}%
\pgfpathlineto{\pgfqpoint{8.473902in}{6.251128in}}%
\pgfpathlineto{\pgfqpoint{8.526946in}{6.251128in}}%
\pgfpathlineto{\pgfqpoint{8.579990in}{6.560479in}}%
\pgfpathlineto{\pgfqpoint{8.633033in}{6.251128in}}%
\pgfpathlineto{\pgfqpoint{8.686077in}{6.251128in}}%
\pgfpathlineto{\pgfqpoint{8.739121in}{6.251128in}}%
\pgfpathlineto{\pgfqpoint{8.792165in}{6.251128in}}%
\pgfpathlineto{\pgfqpoint{8.845209in}{6.251128in}}%
\pgfpathlineto{\pgfqpoint{8.898253in}{6.251128in}}%
\pgfpathlineto{\pgfqpoint{8.951297in}{6.251128in}}%
\pgfpathlineto{\pgfqpoint{9.004341in}{6.568668in}}%
\pgfpathlineto{\pgfqpoint{9.057385in}{6.251128in}}%
\pgfpathlineto{\pgfqpoint{9.110429in}{6.251128in}}%
\pgfpathlineto{\pgfqpoint{9.163473in}{6.598964in}}%
\pgfpathlineto{\pgfqpoint{9.216517in}{6.334558in}}%
\pgfpathlineto{\pgfqpoint{9.269561in}{6.251128in}}%
\pgfpathlineto{\pgfqpoint{9.322605in}{6.251128in}}%
\pgfpathlineto{\pgfqpoint{9.375649in}{6.251128in}}%
\pgfpathlineto{\pgfqpoint{9.428692in}{6.251128in}}%
\pgfpathlineto{\pgfqpoint{9.481736in}{6.256205in}}%
\pgfpathlineto{\pgfqpoint{9.534780in}{6.287774in}}%
\pgfpathlineto{\pgfqpoint{9.587824in}{6.251128in}}%
\pgfpathlineto{\pgfqpoint{9.640868in}{6.251128in}}%
\pgfpathlineto{\pgfqpoint{9.693912in}{6.251128in}}%
\pgfpathlineto{\pgfqpoint{9.746956in}{6.251128in}}%
\pgfpathlineto{\pgfqpoint{9.800000in}{6.251128in}}%
\pgfpathlineto{\pgfqpoint{9.800000in}{6.573653in}}%
\pgfpathlineto{\pgfqpoint{9.800000in}{6.573653in}}%
\pgfpathlineto{\pgfqpoint{9.746956in}{6.779712in}}%
\pgfpathlineto{\pgfqpoint{9.693912in}{6.632876in}}%
\pgfpathlineto{\pgfqpoint{9.640868in}{6.251128in}}%
\pgfpathlineto{\pgfqpoint{9.587824in}{6.621011in}}%
\pgfpathlineto{\pgfqpoint{9.534780in}{6.946800in}}%
\pgfpathlineto{\pgfqpoint{9.481736in}{6.256205in}}%
\pgfpathlineto{\pgfqpoint{9.428692in}{6.251128in}}%
\pgfpathlineto{\pgfqpoint{9.375649in}{6.754769in}}%
\pgfpathlineto{\pgfqpoint{9.322605in}{6.727450in}}%
\pgfpathlineto{\pgfqpoint{9.269561in}{6.628250in}}%
\pgfpathlineto{\pgfqpoint{9.216517in}{6.804492in}}%
\pgfpathlineto{\pgfqpoint{9.163473in}{6.668425in}}%
\pgfpathlineto{\pgfqpoint{9.110429in}{6.251128in}}%
\pgfpathlineto{\pgfqpoint{9.057385in}{6.251128in}}%
\pgfpathlineto{\pgfqpoint{9.004341in}{6.568668in}}%
\pgfpathlineto{\pgfqpoint{8.951297in}{6.251128in}}%
\pgfpathlineto{\pgfqpoint{8.898253in}{6.354195in}}%
\pgfpathlineto{\pgfqpoint{8.845209in}{6.299108in}}%
\pgfpathlineto{\pgfqpoint{8.792165in}{6.251128in}}%
\pgfpathlineto{\pgfqpoint{8.739121in}{6.489640in}}%
\pgfpathlineto{\pgfqpoint{8.686077in}{6.723687in}}%
\pgfpathlineto{\pgfqpoint{8.633033in}{6.520595in}}%
\pgfpathlineto{\pgfqpoint{8.579990in}{6.685501in}}%
\pgfpathlineto{\pgfqpoint{8.526946in}{6.251128in}}%
\pgfpathlineto{\pgfqpoint{8.473902in}{6.251128in}}%
\pgfpathlineto{\pgfqpoint{8.420858in}{6.653636in}}%
\pgfpathlineto{\pgfqpoint{8.367814in}{6.251128in}}%
\pgfpathlineto{\pgfqpoint{8.314770in}{6.876087in}}%
\pgfpathlineto{\pgfqpoint{8.261726in}{6.924712in}}%
\pgfpathlineto{\pgfqpoint{8.208682in}{6.251128in}}%
\pgfpathlineto{\pgfqpoint{8.155638in}{6.251128in}}%
\pgfpathlineto{\pgfqpoint{8.102594in}{6.782745in}}%
\pgfpathlineto{\pgfqpoint{8.049550in}{6.766484in}}%
\pgfpathlineto{\pgfqpoint{7.996506in}{6.251128in}}%
\pgfpathlineto{\pgfqpoint{7.943462in}{6.251128in}}%
\pgfpathlineto{\pgfqpoint{7.890418in}{6.251128in}}%
\pgfpathlineto{\pgfqpoint{7.837374in}{6.251128in}}%
\pgfpathlineto{\pgfqpoint{7.784330in}{6.251128in}}%
\pgfpathlineto{\pgfqpoint{7.731287in}{6.251128in}}%
\pgfpathlineto{\pgfqpoint{7.678243in}{6.251128in}}%
\pgfpathlineto{\pgfqpoint{7.625199in}{6.652740in}}%
\pgfpathlineto{\pgfqpoint{7.572155in}{6.487962in}}%
\pgfpathlineto{\pgfqpoint{7.519111in}{6.251128in}}%
\pgfpathlineto{\pgfqpoint{7.466067in}{6.251128in}}%
\pgfpathlineto{\pgfqpoint{7.413023in}{6.529690in}}%
\pgfpathlineto{\pgfqpoint{7.359979in}{6.470643in}}%
\pgfpathlineto{\pgfqpoint{7.306935in}{6.251128in}}%
\pgfpathlineto{\pgfqpoint{7.253891in}{6.451777in}}%
\pgfpathlineto{\pgfqpoint{7.200847in}{6.729225in}}%
\pgfpathlineto{\pgfqpoint{7.147803in}{6.251128in}}%
\pgfpathlineto{\pgfqpoint{7.094759in}{6.251128in}}%
\pgfpathlineto{\pgfqpoint{7.041715in}{6.251128in}}%
\pgfpathlineto{\pgfqpoint{6.988671in}{6.598964in}}%
\pgfpathlineto{\pgfqpoint{6.935628in}{6.251128in}}%
\pgfpathlineto{\pgfqpoint{6.882584in}{6.780238in}}%
\pgfpathlineto{\pgfqpoint{6.829540in}{6.251128in}}%
\pgfpathlineto{\pgfqpoint{6.776496in}{6.385890in}}%
\pgfpathlineto{\pgfqpoint{6.723452in}{6.251128in}}%
\pgfpathlineto{\pgfqpoint{6.670408in}{6.251128in}}%
\pgfpathlineto{\pgfqpoint{6.617364in}{6.251128in}}%
\pgfpathlineto{\pgfqpoint{6.564320in}{6.251128in}}%
\pgfpathlineto{\pgfqpoint{6.511276in}{6.251128in}}%
\pgfpathlineto{\pgfqpoint{6.458232in}{6.251128in}}%
\pgfpathlineto{\pgfqpoint{6.405188in}{6.364054in}}%
\pgfpathlineto{\pgfqpoint{6.352144in}{6.569516in}}%
\pgfpathlineto{\pgfqpoint{6.299100in}{6.251128in}}%
\pgfpathlineto{\pgfqpoint{6.246056in}{6.691111in}}%
\pgfpathlineto{\pgfqpoint{6.193012in}{6.635398in}}%
\pgfpathlineto{\pgfqpoint{6.139969in}{6.251128in}}%
\pgfpathlineto{\pgfqpoint{6.086925in}{6.251128in}}%
\pgfpathlineto{\pgfqpoint{6.033881in}{6.400407in}}%
\pgfpathlineto{\pgfqpoint{5.980837in}{6.251128in}}%
\pgfpathlineto{\pgfqpoint{5.927793in}{6.598351in}}%
\pgfpathlineto{\pgfqpoint{5.874749in}{6.702075in}}%
\pgfpathlineto{\pgfqpoint{5.821705in}{6.872997in}}%
\pgfpathlineto{\pgfqpoint{5.768661in}{6.913209in}}%
\pgfpathlineto{\pgfqpoint{5.715617in}{6.251128in}}%
\pgfpathlineto{\pgfqpoint{5.662573in}{6.375963in}}%
\pgfpathlineto{\pgfqpoint{5.609529in}{6.514446in}}%
\pgfpathlineto{\pgfqpoint{5.556485in}{6.329843in}}%
\pgfpathlineto{\pgfqpoint{5.503441in}{6.251128in}}%
\pgfpathlineto{\pgfqpoint{5.450397in}{6.770026in}}%
\pgfpathlineto{\pgfqpoint{5.397353in}{6.325005in}}%
\pgfpathlineto{\pgfqpoint{5.344309in}{6.471024in}}%
\pgfpathlineto{\pgfqpoint{5.291266in}{6.251128in}}%
\pgfpathlineto{\pgfqpoint{5.238222in}{6.501650in}}%
\pgfpathlineto{\pgfqpoint{5.185178in}{6.574350in}}%
\pgfpathlineto{\pgfqpoint{5.132134in}{6.598964in}}%
\pgfpathlineto{\pgfqpoint{5.079090in}{6.251128in}}%
\pgfpathlineto{\pgfqpoint{5.026046in}{6.590550in}}%
\pgfpathlineto{\pgfqpoint{4.973002in}{6.251128in}}%
\pgfpathlineto{\pgfqpoint{4.919958in}{6.528186in}}%
\pgfpathlineto{\pgfqpoint{4.866914in}{6.548881in}}%
\pgfpathlineto{\pgfqpoint{4.813870in}{6.251128in}}%
\pgfpathlineto{\pgfqpoint{4.760826in}{6.251128in}}%
\pgfpathlineto{\pgfqpoint{4.707782in}{6.766257in}}%
\pgfpathlineto{\pgfqpoint{4.654738in}{6.251128in}}%
\pgfpathlineto{\pgfqpoint{4.601694in}{6.592100in}}%
\pgfpathlineto{\pgfqpoint{4.548650in}{6.251128in}}%
\pgfpathlineto{\pgfqpoint{4.495607in}{6.633012in}}%
\pgfpathlineto{\pgfqpoint{4.442563in}{6.598964in}}%
\pgfpathlineto{\pgfqpoint{4.389519in}{6.299258in}}%
\pgfpathlineto{\pgfqpoint{4.336475in}{6.889082in}}%
\pgfpathlineto{\pgfqpoint{4.283431in}{6.251128in}}%
\pgfpathlineto{\pgfqpoint{4.230387in}{6.251128in}}%
\pgfpathlineto{\pgfqpoint{4.177343in}{6.251128in}}%
\pgfpathlineto{\pgfqpoint{4.124299in}{6.251128in}}%
\pgfpathlineto{\pgfqpoint{4.071255in}{6.251128in}}%
\pgfpathlineto{\pgfqpoint{4.018211in}{6.251128in}}%
\pgfpathlineto{\pgfqpoint{3.965167in}{6.340259in}}%
\pgfpathlineto{\pgfqpoint{3.912123in}{6.481634in}}%
\pgfpathlineto{\pgfqpoint{3.859079in}{6.572998in}}%
\pgfpathlineto{\pgfqpoint{3.806035in}{6.251128in}}%
\pgfpathlineto{\pgfqpoint{3.752991in}{6.510498in}}%
\pgfpathlineto{\pgfqpoint{3.699948in}{6.251128in}}%
\pgfpathlineto{\pgfqpoint{3.646904in}{6.251128in}}%
\pgfpathlineto{\pgfqpoint{3.593860in}{6.251128in}}%
\pgfpathlineto{\pgfqpoint{3.540816in}{6.572638in}}%
\pgfpathlineto{\pgfqpoint{3.487772in}{6.598964in}}%
\pgfpathlineto{\pgfqpoint{3.434728in}{6.251128in}}%
\pgfpathlineto{\pgfqpoint{3.381684in}{6.598964in}}%
\pgfpathlineto{\pgfqpoint{3.328640in}{6.251128in}}%
\pgfpathlineto{\pgfqpoint{3.275596in}{6.251128in}}%
\pgfpathlineto{\pgfqpoint{3.222552in}{6.251128in}}%
\pgfpathlineto{\pgfqpoint{3.169508in}{6.251128in}}%
\pgfpathlineto{\pgfqpoint{3.116464in}{6.598964in}}%
\pgfpathlineto{\pgfqpoint{3.063420in}{6.598964in}}%
\pgfpathlineto{\pgfqpoint{3.010376in}{6.251128in}}%
\pgfpathlineto{\pgfqpoint{2.957332in}{6.744748in}}%
\pgfpathlineto{\pgfqpoint{2.904288in}{6.251128in}}%
\pgfpathlineto{\pgfqpoint{2.851245in}{6.598964in}}%
\pgfpathlineto{\pgfqpoint{2.798201in}{6.251128in}}%
\pgfpathlineto{\pgfqpoint{2.745157in}{6.251128in}}%
\pgfpathlineto{\pgfqpoint{2.692113in}{6.251128in}}%
\pgfpathlineto{\pgfqpoint{2.639069in}{6.251128in}}%
\pgfpathlineto{\pgfqpoint{2.586025in}{6.251128in}}%
\pgfpathlineto{\pgfqpoint{2.532981in}{6.251128in}}%
\pgfpathlineto{\pgfqpoint{2.479937in}{6.251128in}}%
\pgfpathlineto{\pgfqpoint{2.426893in}{6.606651in}}%
\pgfpathlineto{\pgfqpoint{2.373849in}{6.541500in}}%
\pgfpathlineto{\pgfqpoint{2.320805in}{6.251128in}}%
\pgfpathlineto{\pgfqpoint{2.267761in}{6.251128in}}%
\pgfpathlineto{\pgfqpoint{2.214717in}{6.574971in}}%
\pgfpathlineto{\pgfqpoint{2.161673in}{6.251128in}}%
\pgfpathlineto{\pgfqpoint{2.108629in}{6.348747in}}%
\pgfpathlineto{\pgfqpoint{2.055586in}{6.251128in}}%
\pgfpathlineto{\pgfqpoint{2.002542in}{6.724358in}}%
\pgfpathlineto{\pgfqpoint{1.949498in}{6.251128in}}%
\pgfpathlineto{\pgfqpoint{1.896454in}{6.883587in}}%
\pgfpathlineto{\pgfqpoint{1.843410in}{6.734592in}}%
\pgfpathlineto{\pgfqpoint{1.790366in}{6.251128in}}%
\pgfpathlineto{\pgfqpoint{1.737322in}{6.251128in}}%
\pgfpathlineto{\pgfqpoint{1.684278in}{6.598964in}}%
\pgfpathlineto{\pgfqpoint{1.631234in}{6.598964in}}%
\pgfpathlineto{\pgfqpoint{1.578190in}{6.251128in}}%
\pgfpathlineto{\pgfqpoint{1.525146in}{6.251128in}}%
\pgfpathlineto{\pgfqpoint{1.472102in}{6.251128in}}%
\pgfpathlineto{\pgfqpoint{1.419058in}{6.251128in}}%
\pgfpathlineto{\pgfqpoint{1.366014in}{6.251128in}}%
\pgfpathlineto{\pgfqpoint{1.312970in}{6.598964in}}%
\pgfpathlineto{\pgfqpoint{1.259927in}{6.598964in}}%
\pgfpathlineto{\pgfqpoint{1.206883in}{6.598964in}}%
\pgfpathlineto{\pgfqpoint{1.153839in}{6.251128in}}%
\pgfpathlineto{\pgfqpoint{1.100795in}{6.251128in}}%
\pgfpathlineto{\pgfqpoint{1.047751in}{6.251128in}}%
\pgfpathlineto{\pgfqpoint{0.994707in}{6.261680in}}%
\pgfpathlineto{\pgfqpoint{0.941663in}{6.251128in}}%
\pgfpathlineto{\pgfqpoint{0.941663in}{6.251128in}}%
\pgfpathclose%
\pgfusepath{stroke,fill}%
}%
\begin{pgfscope}%
\pgfsys@transformshift{0.000000in}{0.000000in}%
\pgfsys@useobject{currentmarker}{}%
\end{pgfscope}%
\end{pgfscope}%
\begin{pgfscope}%
\pgfpathrectangle{\pgfqpoint{0.941663in}{4.334375in}}{\pgfqpoint{8.858337in}{3.465625in}}%
\pgfusepath{clip}%
\pgfsetrectcap%
\pgfsetroundjoin%
\pgfsetlinewidth{1.505625pt}%
\definecolor{currentstroke}{rgb}{1.000000,0.647059,0.000000}%
\pgfsetstrokecolor{currentstroke}%
\pgfsetdash{}{0pt}%
\pgfpathmoveto{\pgfqpoint{0.941663in}{5.302080in}}%
\pgfpathlineto{\pgfqpoint{0.994707in}{5.555456in}}%
\pgfpathlineto{\pgfqpoint{1.047751in}{5.298031in}}%
\pgfpathlineto{\pgfqpoint{1.100795in}{5.267250in}}%
\pgfpathlineto{\pgfqpoint{1.153839in}{5.253587in}}%
\pgfpathlineto{\pgfqpoint{1.206883in}{5.555456in}}%
\pgfpathlineto{\pgfqpoint{1.312970in}{5.555456in}}%
\pgfpathlineto{\pgfqpoint{1.366014in}{5.207621in}}%
\pgfpathlineto{\pgfqpoint{1.419058in}{5.263460in}}%
\pgfpathlineto{\pgfqpoint{1.472102in}{5.268823in}}%
\pgfpathlineto{\pgfqpoint{1.525146in}{5.349185in}}%
\pgfpathlineto{\pgfqpoint{1.578190in}{5.343668in}}%
\pgfpathlineto{\pgfqpoint{1.631234in}{5.555456in}}%
\pgfpathlineto{\pgfqpoint{1.684278in}{5.555456in}}%
\pgfpathlineto{\pgfqpoint{1.737322in}{5.491300in}}%
\pgfpathlineto{\pgfqpoint{1.790366in}{5.473721in}}%
\pgfpathlineto{\pgfqpoint{1.843410in}{5.555456in}}%
\pgfpathlineto{\pgfqpoint{1.896454in}{5.555456in}}%
\pgfpathlineto{\pgfqpoint{1.949498in}{5.447032in}}%
\pgfpathlineto{\pgfqpoint{2.002542in}{5.555456in}}%
\pgfpathlineto{\pgfqpoint{2.055586in}{5.457838in}}%
\pgfpathlineto{\pgfqpoint{2.108629in}{5.555456in}}%
\pgfpathlineto{\pgfqpoint{2.161673in}{5.347486in}}%
\pgfpathlineto{\pgfqpoint{2.214717in}{5.555456in}}%
\pgfpathlineto{\pgfqpoint{2.267761in}{5.542464in}}%
\pgfpathlineto{\pgfqpoint{2.320805in}{5.278077in}}%
\pgfpathlineto{\pgfqpoint{2.373849in}{5.555456in}}%
\pgfpathlineto{\pgfqpoint{2.426893in}{5.555456in}}%
\pgfpathlineto{\pgfqpoint{2.479937in}{5.397997in}}%
\pgfpathlineto{\pgfqpoint{2.532981in}{5.253753in}}%
\pgfpathlineto{\pgfqpoint{2.586025in}{5.231942in}}%
\pgfpathlineto{\pgfqpoint{2.639069in}{5.227448in}}%
\pgfpathlineto{\pgfqpoint{2.692113in}{5.274799in}}%
\pgfpathlineto{\pgfqpoint{2.745157in}{5.555456in}}%
\pgfpathlineto{\pgfqpoint{2.851245in}{5.555456in}}%
\pgfpathlineto{\pgfqpoint{2.904288in}{5.357472in}}%
\pgfpathlineto{\pgfqpoint{2.957332in}{5.555456in}}%
\pgfpathlineto{\pgfqpoint{3.010376in}{5.443237in}}%
\pgfpathlineto{\pgfqpoint{3.063420in}{5.555456in}}%
\pgfpathlineto{\pgfqpoint{3.116464in}{5.555456in}}%
\pgfpathlineto{\pgfqpoint{3.169508in}{5.475056in}}%
\pgfpathlineto{\pgfqpoint{3.222552in}{5.505579in}}%
\pgfpathlineto{\pgfqpoint{3.275596in}{5.421720in}}%
\pgfpathlineto{\pgfqpoint{3.328640in}{5.480294in}}%
\pgfpathlineto{\pgfqpoint{3.381684in}{5.555456in}}%
\pgfpathlineto{\pgfqpoint{3.434728in}{5.332680in}}%
\pgfpathlineto{\pgfqpoint{3.487772in}{5.555456in}}%
\pgfpathlineto{\pgfqpoint{3.593860in}{5.555456in}}%
\pgfpathlineto{\pgfqpoint{3.646904in}{5.270980in}}%
\pgfpathlineto{\pgfqpoint{3.699948in}{5.227131in}}%
\pgfpathlineto{\pgfqpoint{3.752991in}{5.555456in}}%
\pgfpathlineto{\pgfqpoint{3.806035in}{5.267382in}}%
\pgfpathlineto{\pgfqpoint{3.859079in}{5.555456in}}%
\pgfpathlineto{\pgfqpoint{3.965167in}{5.555456in}}%
\pgfpathlineto{\pgfqpoint{4.018211in}{5.273356in}}%
\pgfpathlineto{\pgfqpoint{4.071255in}{5.355447in}}%
\pgfpathlineto{\pgfqpoint{4.124299in}{5.483615in}}%
\pgfpathlineto{\pgfqpoint{4.177343in}{5.399336in}}%
\pgfpathlineto{\pgfqpoint{4.230387in}{5.437444in}}%
\pgfpathlineto{\pgfqpoint{4.283431in}{5.453162in}}%
\pgfpathlineto{\pgfqpoint{4.336475in}{5.555456in}}%
\pgfpathlineto{\pgfqpoint{4.601694in}{5.555456in}}%
\pgfpathlineto{\pgfqpoint{4.654738in}{5.454774in}}%
\pgfpathlineto{\pgfqpoint{4.707782in}{5.555456in}}%
\pgfpathlineto{\pgfqpoint{4.760826in}{5.555456in}}%
\pgfpathlineto{\pgfqpoint{4.813870in}{5.335037in}}%
\pgfpathlineto{\pgfqpoint{4.866914in}{5.555456in}}%
\pgfpathlineto{\pgfqpoint{4.919958in}{5.555456in}}%
\pgfpathlineto{\pgfqpoint{4.973002in}{5.216034in}}%
\pgfpathlineto{\pgfqpoint{5.026046in}{5.555456in}}%
\pgfpathlineto{\pgfqpoint{5.079090in}{5.207621in}}%
\pgfpathlineto{\pgfqpoint{5.132134in}{5.555456in}}%
\pgfpathlineto{\pgfqpoint{5.238222in}{5.555456in}}%
\pgfpathlineto{\pgfqpoint{5.291266in}{5.335561in}}%
\pgfpathlineto{\pgfqpoint{5.344309in}{5.555456in}}%
\pgfpathlineto{\pgfqpoint{5.450397in}{5.555456in}}%
\pgfpathlineto{\pgfqpoint{5.503441in}{5.476742in}}%
\pgfpathlineto{\pgfqpoint{5.556485in}{5.555456in}}%
\pgfpathlineto{\pgfqpoint{5.662573in}{5.555456in}}%
\pgfpathlineto{\pgfqpoint{5.715617in}{5.482534in}}%
\pgfpathlineto{\pgfqpoint{5.768661in}{5.555456in}}%
\pgfpathlineto{\pgfqpoint{5.927793in}{5.555456in}}%
\pgfpathlineto{\pgfqpoint{5.980837in}{5.406177in}}%
\pgfpathlineto{\pgfqpoint{6.033881in}{5.555456in}}%
\pgfpathlineto{\pgfqpoint{6.086925in}{5.335040in}}%
\pgfpathlineto{\pgfqpoint{6.139969in}{5.316566in}}%
\pgfpathlineto{\pgfqpoint{6.193012in}{5.555456in}}%
\pgfpathlineto{\pgfqpoint{6.246056in}{5.555456in}}%
\pgfpathlineto{\pgfqpoint{6.299100in}{5.237069in}}%
\pgfpathlineto{\pgfqpoint{6.352144in}{5.555456in}}%
\pgfpathlineto{\pgfqpoint{6.458232in}{5.555456in}}%
\pgfpathlineto{\pgfqpoint{6.511276in}{5.306663in}}%
\pgfpathlineto{\pgfqpoint{6.564320in}{5.303012in}}%
\pgfpathlineto{\pgfqpoint{6.617364in}{5.369970in}}%
\pgfpathlineto{\pgfqpoint{6.670408in}{5.399912in}}%
\pgfpathlineto{\pgfqpoint{6.723452in}{5.445993in}}%
\pgfpathlineto{\pgfqpoint{6.776496in}{5.555456in}}%
\pgfpathlineto{\pgfqpoint{6.829540in}{5.457960in}}%
\pgfpathlineto{\pgfqpoint{6.882584in}{5.555456in}}%
\pgfpathlineto{\pgfqpoint{6.935628in}{5.451613in}}%
\pgfpathlineto{\pgfqpoint{6.988671in}{5.555456in}}%
\pgfpathlineto{\pgfqpoint{7.041715in}{5.470876in}}%
\pgfpathlineto{\pgfqpoint{7.094759in}{5.493927in}}%
\pgfpathlineto{\pgfqpoint{7.147803in}{5.475721in}}%
\pgfpathlineto{\pgfqpoint{7.200847in}{5.555456in}}%
\pgfpathlineto{\pgfqpoint{7.253891in}{5.555456in}}%
\pgfpathlineto{\pgfqpoint{7.306935in}{5.455611in}}%
\pgfpathlineto{\pgfqpoint{7.359979in}{5.555456in}}%
\pgfpathlineto{\pgfqpoint{7.678243in}{5.555456in}}%
\pgfpathlineto{\pgfqpoint{7.731287in}{5.268845in}}%
\pgfpathlineto{\pgfqpoint{7.784330in}{5.304877in}}%
\pgfpathlineto{\pgfqpoint{7.837374in}{5.313128in}}%
\pgfpathlineto{\pgfqpoint{7.890418in}{5.354431in}}%
\pgfpathlineto{\pgfqpoint{7.943462in}{5.352176in}}%
\pgfpathlineto{\pgfqpoint{7.996506in}{5.430879in}}%
\pgfpathlineto{\pgfqpoint{8.049550in}{5.555456in}}%
\pgfpathlineto{\pgfqpoint{8.102594in}{5.555456in}}%
\pgfpathlineto{\pgfqpoint{8.155638in}{5.497516in}}%
\pgfpathlineto{\pgfqpoint{8.208682in}{5.491656in}}%
\pgfpathlineto{\pgfqpoint{8.261726in}{5.555456in}}%
\pgfpathlineto{\pgfqpoint{8.314770in}{5.555456in}}%
\pgfpathlineto{\pgfqpoint{8.367814in}{5.474073in}}%
\pgfpathlineto{\pgfqpoint{8.420858in}{5.555456in}}%
\pgfpathlineto{\pgfqpoint{8.473902in}{5.414851in}}%
\pgfpathlineto{\pgfqpoint{8.526946in}{5.386710in}}%
\pgfpathlineto{\pgfqpoint{8.579990in}{5.555456in}}%
\pgfpathlineto{\pgfqpoint{8.898253in}{5.555456in}}%
\pgfpathlineto{\pgfqpoint{8.951297in}{5.237916in}}%
\pgfpathlineto{\pgfqpoint{9.004341in}{5.555456in}}%
\pgfpathlineto{\pgfqpoint{9.057385in}{5.322277in}}%
\pgfpathlineto{\pgfqpoint{9.110429in}{5.357370in}}%
\pgfpathlineto{\pgfqpoint{9.163473in}{5.555456in}}%
\pgfpathlineto{\pgfqpoint{9.375649in}{5.555456in}}%
\pgfpathlineto{\pgfqpoint{9.428692in}{5.513734in}}%
\pgfpathlineto{\pgfqpoint{9.481736in}{5.555456in}}%
\pgfpathlineto{\pgfqpoint{9.800000in}{5.555456in}}%
\pgfpathlineto{\pgfqpoint{9.800000in}{5.555456in}}%
\pgfusepath{stroke}%
\end{pgfscope}%
\begin{pgfscope}%
\pgfpathrectangle{\pgfqpoint{0.941663in}{4.334375in}}{\pgfqpoint{8.858337in}{3.465625in}}%
\pgfusepath{clip}%
\pgfsetbuttcap%
\pgfsetroundjoin%
\definecolor{currentfill}{rgb}{1.000000,0.647059,0.000000}%
\pgfsetfillcolor{currentfill}%
\pgfsetlinewidth{1.003750pt}%
\definecolor{currentstroke}{rgb}{1.000000,0.647059,0.000000}%
\pgfsetstrokecolor{currentstroke}%
\pgfsetdash{}{0pt}%
\pgfsys@defobject{currentmarker}{\pgfqpoint{0.941663in}{5.207621in}}{\pgfqpoint{9.800000in}{5.555456in}}{%
\pgfpathmoveto{\pgfqpoint{0.941663in}{5.302080in}}%
\pgfpathlineto{\pgfqpoint{0.941663in}{5.555456in}}%
\pgfpathlineto{\pgfqpoint{0.994707in}{5.555456in}}%
\pgfpathlineto{\pgfqpoint{1.047751in}{5.555456in}}%
\pgfpathlineto{\pgfqpoint{1.100795in}{5.555456in}}%
\pgfpathlineto{\pgfqpoint{1.153839in}{5.555456in}}%
\pgfpathlineto{\pgfqpoint{1.206883in}{5.555456in}}%
\pgfpathlineto{\pgfqpoint{1.259927in}{5.555456in}}%
\pgfpathlineto{\pgfqpoint{1.312970in}{5.555456in}}%
\pgfpathlineto{\pgfqpoint{1.366014in}{5.555456in}}%
\pgfpathlineto{\pgfqpoint{1.419058in}{5.555456in}}%
\pgfpathlineto{\pgfqpoint{1.472102in}{5.555456in}}%
\pgfpathlineto{\pgfqpoint{1.525146in}{5.555456in}}%
\pgfpathlineto{\pgfqpoint{1.578190in}{5.555456in}}%
\pgfpathlineto{\pgfqpoint{1.631234in}{5.555456in}}%
\pgfpathlineto{\pgfqpoint{1.684278in}{5.555456in}}%
\pgfpathlineto{\pgfqpoint{1.737322in}{5.555456in}}%
\pgfpathlineto{\pgfqpoint{1.790366in}{5.555456in}}%
\pgfpathlineto{\pgfqpoint{1.843410in}{5.555456in}}%
\pgfpathlineto{\pgfqpoint{1.896454in}{5.555456in}}%
\pgfpathlineto{\pgfqpoint{1.949498in}{5.555456in}}%
\pgfpathlineto{\pgfqpoint{2.002542in}{5.555456in}}%
\pgfpathlineto{\pgfqpoint{2.055586in}{5.555456in}}%
\pgfpathlineto{\pgfqpoint{2.108629in}{5.555456in}}%
\pgfpathlineto{\pgfqpoint{2.161673in}{5.555456in}}%
\pgfpathlineto{\pgfqpoint{2.214717in}{5.555456in}}%
\pgfpathlineto{\pgfqpoint{2.267761in}{5.555456in}}%
\pgfpathlineto{\pgfqpoint{2.320805in}{5.555456in}}%
\pgfpathlineto{\pgfqpoint{2.373849in}{5.555456in}}%
\pgfpathlineto{\pgfqpoint{2.426893in}{5.555456in}}%
\pgfpathlineto{\pgfqpoint{2.479937in}{5.555456in}}%
\pgfpathlineto{\pgfqpoint{2.532981in}{5.555456in}}%
\pgfpathlineto{\pgfqpoint{2.586025in}{5.555456in}}%
\pgfpathlineto{\pgfqpoint{2.639069in}{5.555456in}}%
\pgfpathlineto{\pgfqpoint{2.692113in}{5.555456in}}%
\pgfpathlineto{\pgfqpoint{2.745157in}{5.555456in}}%
\pgfpathlineto{\pgfqpoint{2.798201in}{5.555456in}}%
\pgfpathlineto{\pgfqpoint{2.851245in}{5.555456in}}%
\pgfpathlineto{\pgfqpoint{2.904288in}{5.555456in}}%
\pgfpathlineto{\pgfqpoint{2.957332in}{5.555456in}}%
\pgfpathlineto{\pgfqpoint{3.010376in}{5.555456in}}%
\pgfpathlineto{\pgfqpoint{3.063420in}{5.555456in}}%
\pgfpathlineto{\pgfqpoint{3.116464in}{5.555456in}}%
\pgfpathlineto{\pgfqpoint{3.169508in}{5.555456in}}%
\pgfpathlineto{\pgfqpoint{3.222552in}{5.555456in}}%
\pgfpathlineto{\pgfqpoint{3.275596in}{5.555456in}}%
\pgfpathlineto{\pgfqpoint{3.328640in}{5.555456in}}%
\pgfpathlineto{\pgfqpoint{3.381684in}{5.555456in}}%
\pgfpathlineto{\pgfqpoint{3.434728in}{5.555456in}}%
\pgfpathlineto{\pgfqpoint{3.487772in}{5.555456in}}%
\pgfpathlineto{\pgfqpoint{3.540816in}{5.555456in}}%
\pgfpathlineto{\pgfqpoint{3.593860in}{5.555456in}}%
\pgfpathlineto{\pgfqpoint{3.646904in}{5.555456in}}%
\pgfpathlineto{\pgfqpoint{3.699948in}{5.555456in}}%
\pgfpathlineto{\pgfqpoint{3.752991in}{5.555456in}}%
\pgfpathlineto{\pgfqpoint{3.806035in}{5.555456in}}%
\pgfpathlineto{\pgfqpoint{3.859079in}{5.555456in}}%
\pgfpathlineto{\pgfqpoint{3.912123in}{5.555456in}}%
\pgfpathlineto{\pgfqpoint{3.965167in}{5.555456in}}%
\pgfpathlineto{\pgfqpoint{4.018211in}{5.555456in}}%
\pgfpathlineto{\pgfqpoint{4.071255in}{5.555456in}}%
\pgfpathlineto{\pgfqpoint{4.124299in}{5.555456in}}%
\pgfpathlineto{\pgfqpoint{4.177343in}{5.555456in}}%
\pgfpathlineto{\pgfqpoint{4.230387in}{5.555456in}}%
\pgfpathlineto{\pgfqpoint{4.283431in}{5.555456in}}%
\pgfpathlineto{\pgfqpoint{4.336475in}{5.555456in}}%
\pgfpathlineto{\pgfqpoint{4.389519in}{5.555456in}}%
\pgfpathlineto{\pgfqpoint{4.442563in}{5.555456in}}%
\pgfpathlineto{\pgfqpoint{4.495607in}{5.555456in}}%
\pgfpathlineto{\pgfqpoint{4.548650in}{5.555456in}}%
\pgfpathlineto{\pgfqpoint{4.601694in}{5.555456in}}%
\pgfpathlineto{\pgfqpoint{4.654738in}{5.555456in}}%
\pgfpathlineto{\pgfqpoint{4.707782in}{5.555456in}}%
\pgfpathlineto{\pgfqpoint{4.760826in}{5.555456in}}%
\pgfpathlineto{\pgfqpoint{4.813870in}{5.555456in}}%
\pgfpathlineto{\pgfqpoint{4.866914in}{5.555456in}}%
\pgfpathlineto{\pgfqpoint{4.919958in}{5.555456in}}%
\pgfpathlineto{\pgfqpoint{4.973002in}{5.555456in}}%
\pgfpathlineto{\pgfqpoint{5.026046in}{5.555456in}}%
\pgfpathlineto{\pgfqpoint{5.079090in}{5.555456in}}%
\pgfpathlineto{\pgfqpoint{5.132134in}{5.555456in}}%
\pgfpathlineto{\pgfqpoint{5.185178in}{5.555456in}}%
\pgfpathlineto{\pgfqpoint{5.238222in}{5.555456in}}%
\pgfpathlineto{\pgfqpoint{5.291266in}{5.555456in}}%
\pgfpathlineto{\pgfqpoint{5.344309in}{5.555456in}}%
\pgfpathlineto{\pgfqpoint{5.397353in}{5.555456in}}%
\pgfpathlineto{\pgfqpoint{5.450397in}{5.555456in}}%
\pgfpathlineto{\pgfqpoint{5.503441in}{5.555456in}}%
\pgfpathlineto{\pgfqpoint{5.556485in}{5.555456in}}%
\pgfpathlineto{\pgfqpoint{5.609529in}{5.555456in}}%
\pgfpathlineto{\pgfqpoint{5.662573in}{5.555456in}}%
\pgfpathlineto{\pgfqpoint{5.715617in}{5.555456in}}%
\pgfpathlineto{\pgfqpoint{5.768661in}{5.555456in}}%
\pgfpathlineto{\pgfqpoint{5.821705in}{5.555456in}}%
\pgfpathlineto{\pgfqpoint{5.874749in}{5.555456in}}%
\pgfpathlineto{\pgfqpoint{5.927793in}{5.555456in}}%
\pgfpathlineto{\pgfqpoint{5.980837in}{5.555456in}}%
\pgfpathlineto{\pgfqpoint{6.033881in}{5.555456in}}%
\pgfpathlineto{\pgfqpoint{6.086925in}{5.555456in}}%
\pgfpathlineto{\pgfqpoint{6.139969in}{5.555456in}}%
\pgfpathlineto{\pgfqpoint{6.193012in}{5.555456in}}%
\pgfpathlineto{\pgfqpoint{6.246056in}{5.555456in}}%
\pgfpathlineto{\pgfqpoint{6.299100in}{5.555456in}}%
\pgfpathlineto{\pgfqpoint{6.352144in}{5.555456in}}%
\pgfpathlineto{\pgfqpoint{6.405188in}{5.555456in}}%
\pgfpathlineto{\pgfqpoint{6.458232in}{5.555456in}}%
\pgfpathlineto{\pgfqpoint{6.511276in}{5.555456in}}%
\pgfpathlineto{\pgfqpoint{6.564320in}{5.555456in}}%
\pgfpathlineto{\pgfqpoint{6.617364in}{5.555456in}}%
\pgfpathlineto{\pgfqpoint{6.670408in}{5.555456in}}%
\pgfpathlineto{\pgfqpoint{6.723452in}{5.555456in}}%
\pgfpathlineto{\pgfqpoint{6.776496in}{5.555456in}}%
\pgfpathlineto{\pgfqpoint{6.829540in}{5.555456in}}%
\pgfpathlineto{\pgfqpoint{6.882584in}{5.555456in}}%
\pgfpathlineto{\pgfqpoint{6.935628in}{5.555456in}}%
\pgfpathlineto{\pgfqpoint{6.988671in}{5.555456in}}%
\pgfpathlineto{\pgfqpoint{7.041715in}{5.555456in}}%
\pgfpathlineto{\pgfqpoint{7.094759in}{5.555456in}}%
\pgfpathlineto{\pgfqpoint{7.147803in}{5.555456in}}%
\pgfpathlineto{\pgfqpoint{7.200847in}{5.555456in}}%
\pgfpathlineto{\pgfqpoint{7.253891in}{5.555456in}}%
\pgfpathlineto{\pgfqpoint{7.306935in}{5.555456in}}%
\pgfpathlineto{\pgfqpoint{7.359979in}{5.555456in}}%
\pgfpathlineto{\pgfqpoint{7.413023in}{5.555456in}}%
\pgfpathlineto{\pgfqpoint{7.466067in}{5.555456in}}%
\pgfpathlineto{\pgfqpoint{7.519111in}{5.555456in}}%
\pgfpathlineto{\pgfqpoint{7.572155in}{5.555456in}}%
\pgfpathlineto{\pgfqpoint{7.625199in}{5.555456in}}%
\pgfpathlineto{\pgfqpoint{7.678243in}{5.555456in}}%
\pgfpathlineto{\pgfqpoint{7.731287in}{5.555456in}}%
\pgfpathlineto{\pgfqpoint{7.784330in}{5.555456in}}%
\pgfpathlineto{\pgfqpoint{7.837374in}{5.555456in}}%
\pgfpathlineto{\pgfqpoint{7.890418in}{5.555456in}}%
\pgfpathlineto{\pgfqpoint{7.943462in}{5.555456in}}%
\pgfpathlineto{\pgfqpoint{7.996506in}{5.555456in}}%
\pgfpathlineto{\pgfqpoint{8.049550in}{5.555456in}}%
\pgfpathlineto{\pgfqpoint{8.102594in}{5.555456in}}%
\pgfpathlineto{\pgfqpoint{8.155638in}{5.555456in}}%
\pgfpathlineto{\pgfqpoint{8.208682in}{5.555456in}}%
\pgfpathlineto{\pgfqpoint{8.261726in}{5.555456in}}%
\pgfpathlineto{\pgfqpoint{8.314770in}{5.555456in}}%
\pgfpathlineto{\pgfqpoint{8.367814in}{5.555456in}}%
\pgfpathlineto{\pgfqpoint{8.420858in}{5.555456in}}%
\pgfpathlineto{\pgfqpoint{8.473902in}{5.555456in}}%
\pgfpathlineto{\pgfqpoint{8.526946in}{5.555456in}}%
\pgfpathlineto{\pgfqpoint{8.579990in}{5.555456in}}%
\pgfpathlineto{\pgfqpoint{8.633033in}{5.555456in}}%
\pgfpathlineto{\pgfqpoint{8.686077in}{5.555456in}}%
\pgfpathlineto{\pgfqpoint{8.739121in}{5.555456in}}%
\pgfpathlineto{\pgfqpoint{8.792165in}{5.555456in}}%
\pgfpathlineto{\pgfqpoint{8.845209in}{5.555456in}}%
\pgfpathlineto{\pgfqpoint{8.898253in}{5.555456in}}%
\pgfpathlineto{\pgfqpoint{8.951297in}{5.555456in}}%
\pgfpathlineto{\pgfqpoint{9.004341in}{5.555456in}}%
\pgfpathlineto{\pgfqpoint{9.057385in}{5.555456in}}%
\pgfpathlineto{\pgfqpoint{9.110429in}{5.555456in}}%
\pgfpathlineto{\pgfqpoint{9.163473in}{5.555456in}}%
\pgfpathlineto{\pgfqpoint{9.216517in}{5.555456in}}%
\pgfpathlineto{\pgfqpoint{9.269561in}{5.555456in}}%
\pgfpathlineto{\pgfqpoint{9.322605in}{5.555456in}}%
\pgfpathlineto{\pgfqpoint{9.375649in}{5.555456in}}%
\pgfpathlineto{\pgfqpoint{9.428692in}{5.555456in}}%
\pgfpathlineto{\pgfqpoint{9.481736in}{5.555456in}}%
\pgfpathlineto{\pgfqpoint{9.534780in}{5.555456in}}%
\pgfpathlineto{\pgfqpoint{9.587824in}{5.555456in}}%
\pgfpathlineto{\pgfqpoint{9.640868in}{5.555456in}}%
\pgfpathlineto{\pgfqpoint{9.693912in}{5.555456in}}%
\pgfpathlineto{\pgfqpoint{9.746956in}{5.555456in}}%
\pgfpathlineto{\pgfqpoint{9.800000in}{5.555456in}}%
\pgfpathlineto{\pgfqpoint{9.800000in}{5.555456in}}%
\pgfpathlineto{\pgfqpoint{9.800000in}{5.555456in}}%
\pgfpathlineto{\pgfqpoint{9.746956in}{5.555456in}}%
\pgfpathlineto{\pgfqpoint{9.693912in}{5.555456in}}%
\pgfpathlineto{\pgfqpoint{9.640868in}{5.555456in}}%
\pgfpathlineto{\pgfqpoint{9.587824in}{5.555456in}}%
\pgfpathlineto{\pgfqpoint{9.534780in}{5.555456in}}%
\pgfpathlineto{\pgfqpoint{9.481736in}{5.555456in}}%
\pgfpathlineto{\pgfqpoint{9.428692in}{5.513734in}}%
\pgfpathlineto{\pgfqpoint{9.375649in}{5.555456in}}%
\pgfpathlineto{\pgfqpoint{9.322605in}{5.555456in}}%
\pgfpathlineto{\pgfqpoint{9.269561in}{5.555456in}}%
\pgfpathlineto{\pgfqpoint{9.216517in}{5.555456in}}%
\pgfpathlineto{\pgfqpoint{9.163473in}{5.555456in}}%
\pgfpathlineto{\pgfqpoint{9.110429in}{5.357370in}}%
\pgfpathlineto{\pgfqpoint{9.057385in}{5.322277in}}%
\pgfpathlineto{\pgfqpoint{9.004341in}{5.555456in}}%
\pgfpathlineto{\pgfqpoint{8.951297in}{5.237916in}}%
\pgfpathlineto{\pgfqpoint{8.898253in}{5.555456in}}%
\pgfpathlineto{\pgfqpoint{8.845209in}{5.555456in}}%
\pgfpathlineto{\pgfqpoint{8.792165in}{5.555456in}}%
\pgfpathlineto{\pgfqpoint{8.739121in}{5.555456in}}%
\pgfpathlineto{\pgfqpoint{8.686077in}{5.555456in}}%
\pgfpathlineto{\pgfqpoint{8.633033in}{5.555456in}}%
\pgfpathlineto{\pgfqpoint{8.579990in}{5.555456in}}%
\pgfpathlineto{\pgfqpoint{8.526946in}{5.386710in}}%
\pgfpathlineto{\pgfqpoint{8.473902in}{5.414851in}}%
\pgfpathlineto{\pgfqpoint{8.420858in}{5.555456in}}%
\pgfpathlineto{\pgfqpoint{8.367814in}{5.474073in}}%
\pgfpathlineto{\pgfqpoint{8.314770in}{5.555456in}}%
\pgfpathlineto{\pgfqpoint{8.261726in}{5.555456in}}%
\pgfpathlineto{\pgfqpoint{8.208682in}{5.491656in}}%
\pgfpathlineto{\pgfqpoint{8.155638in}{5.497516in}}%
\pgfpathlineto{\pgfqpoint{8.102594in}{5.555456in}}%
\pgfpathlineto{\pgfqpoint{8.049550in}{5.555456in}}%
\pgfpathlineto{\pgfqpoint{7.996506in}{5.430879in}}%
\pgfpathlineto{\pgfqpoint{7.943462in}{5.352176in}}%
\pgfpathlineto{\pgfqpoint{7.890418in}{5.354431in}}%
\pgfpathlineto{\pgfqpoint{7.837374in}{5.313128in}}%
\pgfpathlineto{\pgfqpoint{7.784330in}{5.304877in}}%
\pgfpathlineto{\pgfqpoint{7.731287in}{5.268845in}}%
\pgfpathlineto{\pgfqpoint{7.678243in}{5.555456in}}%
\pgfpathlineto{\pgfqpoint{7.625199in}{5.555456in}}%
\pgfpathlineto{\pgfqpoint{7.572155in}{5.555456in}}%
\pgfpathlineto{\pgfqpoint{7.519111in}{5.555456in}}%
\pgfpathlineto{\pgfqpoint{7.466067in}{5.555456in}}%
\pgfpathlineto{\pgfqpoint{7.413023in}{5.555456in}}%
\pgfpathlineto{\pgfqpoint{7.359979in}{5.555456in}}%
\pgfpathlineto{\pgfqpoint{7.306935in}{5.455611in}}%
\pgfpathlineto{\pgfqpoint{7.253891in}{5.555456in}}%
\pgfpathlineto{\pgfqpoint{7.200847in}{5.555456in}}%
\pgfpathlineto{\pgfqpoint{7.147803in}{5.475721in}}%
\pgfpathlineto{\pgfqpoint{7.094759in}{5.493927in}}%
\pgfpathlineto{\pgfqpoint{7.041715in}{5.470876in}}%
\pgfpathlineto{\pgfqpoint{6.988671in}{5.555456in}}%
\pgfpathlineto{\pgfqpoint{6.935628in}{5.451613in}}%
\pgfpathlineto{\pgfqpoint{6.882584in}{5.555456in}}%
\pgfpathlineto{\pgfqpoint{6.829540in}{5.457960in}}%
\pgfpathlineto{\pgfqpoint{6.776496in}{5.555456in}}%
\pgfpathlineto{\pgfqpoint{6.723452in}{5.445993in}}%
\pgfpathlineto{\pgfqpoint{6.670408in}{5.399912in}}%
\pgfpathlineto{\pgfqpoint{6.617364in}{5.369970in}}%
\pgfpathlineto{\pgfqpoint{6.564320in}{5.303012in}}%
\pgfpathlineto{\pgfqpoint{6.511276in}{5.306663in}}%
\pgfpathlineto{\pgfqpoint{6.458232in}{5.555456in}}%
\pgfpathlineto{\pgfqpoint{6.405188in}{5.555456in}}%
\pgfpathlineto{\pgfqpoint{6.352144in}{5.555456in}}%
\pgfpathlineto{\pgfqpoint{6.299100in}{5.237069in}}%
\pgfpathlineto{\pgfqpoint{6.246056in}{5.555456in}}%
\pgfpathlineto{\pgfqpoint{6.193012in}{5.555456in}}%
\pgfpathlineto{\pgfqpoint{6.139969in}{5.316566in}}%
\pgfpathlineto{\pgfqpoint{6.086925in}{5.335040in}}%
\pgfpathlineto{\pgfqpoint{6.033881in}{5.555456in}}%
\pgfpathlineto{\pgfqpoint{5.980837in}{5.406177in}}%
\pgfpathlineto{\pgfqpoint{5.927793in}{5.555456in}}%
\pgfpathlineto{\pgfqpoint{5.874749in}{5.555456in}}%
\pgfpathlineto{\pgfqpoint{5.821705in}{5.555456in}}%
\pgfpathlineto{\pgfqpoint{5.768661in}{5.555456in}}%
\pgfpathlineto{\pgfqpoint{5.715617in}{5.482534in}}%
\pgfpathlineto{\pgfqpoint{5.662573in}{5.555456in}}%
\pgfpathlineto{\pgfqpoint{5.609529in}{5.555456in}}%
\pgfpathlineto{\pgfqpoint{5.556485in}{5.555456in}}%
\pgfpathlineto{\pgfqpoint{5.503441in}{5.476742in}}%
\pgfpathlineto{\pgfqpoint{5.450397in}{5.555456in}}%
\pgfpathlineto{\pgfqpoint{5.397353in}{5.555456in}}%
\pgfpathlineto{\pgfqpoint{5.344309in}{5.555456in}}%
\pgfpathlineto{\pgfqpoint{5.291266in}{5.335561in}}%
\pgfpathlineto{\pgfqpoint{5.238222in}{5.555456in}}%
\pgfpathlineto{\pgfqpoint{5.185178in}{5.555456in}}%
\pgfpathlineto{\pgfqpoint{5.132134in}{5.555456in}}%
\pgfpathlineto{\pgfqpoint{5.079090in}{5.207621in}}%
\pgfpathlineto{\pgfqpoint{5.026046in}{5.555456in}}%
\pgfpathlineto{\pgfqpoint{4.973002in}{5.216034in}}%
\pgfpathlineto{\pgfqpoint{4.919958in}{5.555456in}}%
\pgfpathlineto{\pgfqpoint{4.866914in}{5.555456in}}%
\pgfpathlineto{\pgfqpoint{4.813870in}{5.335037in}}%
\pgfpathlineto{\pgfqpoint{4.760826in}{5.555456in}}%
\pgfpathlineto{\pgfqpoint{4.707782in}{5.555456in}}%
\pgfpathlineto{\pgfqpoint{4.654738in}{5.454774in}}%
\pgfpathlineto{\pgfqpoint{4.601694in}{5.555456in}}%
\pgfpathlineto{\pgfqpoint{4.548650in}{5.555456in}}%
\pgfpathlineto{\pgfqpoint{4.495607in}{5.555456in}}%
\pgfpathlineto{\pgfqpoint{4.442563in}{5.555456in}}%
\pgfpathlineto{\pgfqpoint{4.389519in}{5.555456in}}%
\pgfpathlineto{\pgfqpoint{4.336475in}{5.555456in}}%
\pgfpathlineto{\pgfqpoint{4.283431in}{5.453162in}}%
\pgfpathlineto{\pgfqpoint{4.230387in}{5.437444in}}%
\pgfpathlineto{\pgfqpoint{4.177343in}{5.399336in}}%
\pgfpathlineto{\pgfqpoint{4.124299in}{5.483615in}}%
\pgfpathlineto{\pgfqpoint{4.071255in}{5.355447in}}%
\pgfpathlineto{\pgfqpoint{4.018211in}{5.273356in}}%
\pgfpathlineto{\pgfqpoint{3.965167in}{5.555456in}}%
\pgfpathlineto{\pgfqpoint{3.912123in}{5.555456in}}%
\pgfpathlineto{\pgfqpoint{3.859079in}{5.555456in}}%
\pgfpathlineto{\pgfqpoint{3.806035in}{5.267382in}}%
\pgfpathlineto{\pgfqpoint{3.752991in}{5.555456in}}%
\pgfpathlineto{\pgfqpoint{3.699948in}{5.227131in}}%
\pgfpathlineto{\pgfqpoint{3.646904in}{5.270980in}}%
\pgfpathlineto{\pgfqpoint{3.593860in}{5.555456in}}%
\pgfpathlineto{\pgfqpoint{3.540816in}{5.555456in}}%
\pgfpathlineto{\pgfqpoint{3.487772in}{5.555456in}}%
\pgfpathlineto{\pgfqpoint{3.434728in}{5.332680in}}%
\pgfpathlineto{\pgfqpoint{3.381684in}{5.555456in}}%
\pgfpathlineto{\pgfqpoint{3.328640in}{5.480294in}}%
\pgfpathlineto{\pgfqpoint{3.275596in}{5.421720in}}%
\pgfpathlineto{\pgfqpoint{3.222552in}{5.505579in}}%
\pgfpathlineto{\pgfqpoint{3.169508in}{5.475056in}}%
\pgfpathlineto{\pgfqpoint{3.116464in}{5.555456in}}%
\pgfpathlineto{\pgfqpoint{3.063420in}{5.555456in}}%
\pgfpathlineto{\pgfqpoint{3.010376in}{5.443237in}}%
\pgfpathlineto{\pgfqpoint{2.957332in}{5.555456in}}%
\pgfpathlineto{\pgfqpoint{2.904288in}{5.357472in}}%
\pgfpathlineto{\pgfqpoint{2.851245in}{5.555456in}}%
\pgfpathlineto{\pgfqpoint{2.798201in}{5.555456in}}%
\pgfpathlineto{\pgfqpoint{2.745157in}{5.555456in}}%
\pgfpathlineto{\pgfqpoint{2.692113in}{5.274799in}}%
\pgfpathlineto{\pgfqpoint{2.639069in}{5.227448in}}%
\pgfpathlineto{\pgfqpoint{2.586025in}{5.231942in}}%
\pgfpathlineto{\pgfqpoint{2.532981in}{5.253753in}}%
\pgfpathlineto{\pgfqpoint{2.479937in}{5.397997in}}%
\pgfpathlineto{\pgfqpoint{2.426893in}{5.555456in}}%
\pgfpathlineto{\pgfqpoint{2.373849in}{5.555456in}}%
\pgfpathlineto{\pgfqpoint{2.320805in}{5.278077in}}%
\pgfpathlineto{\pgfqpoint{2.267761in}{5.542464in}}%
\pgfpathlineto{\pgfqpoint{2.214717in}{5.555456in}}%
\pgfpathlineto{\pgfqpoint{2.161673in}{5.347486in}}%
\pgfpathlineto{\pgfqpoint{2.108629in}{5.555456in}}%
\pgfpathlineto{\pgfqpoint{2.055586in}{5.457838in}}%
\pgfpathlineto{\pgfqpoint{2.002542in}{5.555456in}}%
\pgfpathlineto{\pgfqpoint{1.949498in}{5.447032in}}%
\pgfpathlineto{\pgfqpoint{1.896454in}{5.555456in}}%
\pgfpathlineto{\pgfqpoint{1.843410in}{5.555456in}}%
\pgfpathlineto{\pgfqpoint{1.790366in}{5.473721in}}%
\pgfpathlineto{\pgfqpoint{1.737322in}{5.491300in}}%
\pgfpathlineto{\pgfqpoint{1.684278in}{5.555456in}}%
\pgfpathlineto{\pgfqpoint{1.631234in}{5.555456in}}%
\pgfpathlineto{\pgfqpoint{1.578190in}{5.343668in}}%
\pgfpathlineto{\pgfqpoint{1.525146in}{5.349185in}}%
\pgfpathlineto{\pgfqpoint{1.472102in}{5.268823in}}%
\pgfpathlineto{\pgfqpoint{1.419058in}{5.263460in}}%
\pgfpathlineto{\pgfqpoint{1.366014in}{5.207621in}}%
\pgfpathlineto{\pgfqpoint{1.312970in}{5.555456in}}%
\pgfpathlineto{\pgfqpoint{1.259927in}{5.555456in}}%
\pgfpathlineto{\pgfqpoint{1.206883in}{5.555456in}}%
\pgfpathlineto{\pgfqpoint{1.153839in}{5.253587in}}%
\pgfpathlineto{\pgfqpoint{1.100795in}{5.267250in}}%
\pgfpathlineto{\pgfqpoint{1.047751in}{5.298031in}}%
\pgfpathlineto{\pgfqpoint{0.994707in}{5.555456in}}%
\pgfpathlineto{\pgfqpoint{0.941663in}{5.302080in}}%
\pgfpathlineto{\pgfqpoint{0.941663in}{5.302080in}}%
\pgfpathclose%
\pgfusepath{stroke,fill}%
}%
\begin{pgfscope}%
\pgfsys@transformshift{0.000000in}{0.000000in}%
\pgfsys@useobject{currentmarker}{}%
\end{pgfscope}%
\end{pgfscope}%
\begin{pgfscope}%
\pgfpathrectangle{\pgfqpoint{0.941663in}{4.334375in}}{\pgfqpoint{8.858337in}{3.465625in}}%
\pgfusepath{clip}%
\pgfsetrectcap%
\pgfsetroundjoin%
\pgfsetlinewidth{1.505625pt}%
\definecolor{currentstroke}{rgb}{0.501961,0.501961,0.501961}%
\pgfsetstrokecolor{currentstroke}%
\pgfsetdash{}{0pt}%
\pgfpathmoveto{\pgfqpoint{0.941663in}{4.606409in}}%
\pgfpathlineto{\pgfqpoint{0.994707in}{4.859785in}}%
\pgfpathlineto{\pgfqpoint{1.047751in}{4.602360in}}%
\pgfpathlineto{\pgfqpoint{1.100795in}{4.571579in}}%
\pgfpathlineto{\pgfqpoint{1.153839in}{4.557915in}}%
\pgfpathlineto{\pgfqpoint{1.206883in}{5.127392in}}%
\pgfpathlineto{\pgfqpoint{1.259927in}{5.369914in}}%
\pgfpathlineto{\pgfqpoint{1.312970in}{5.172617in}}%
\pgfpathlineto{\pgfqpoint{1.366014in}{4.493572in}}%
\pgfpathlineto{\pgfqpoint{1.419058in}{4.567788in}}%
\pgfpathlineto{\pgfqpoint{1.472102in}{4.573151in}}%
\pgfpathlineto{\pgfqpoint{1.525146in}{4.653514in}}%
\pgfpathlineto{\pgfqpoint{1.578190in}{4.630644in}}%
\pgfpathlineto{\pgfqpoint{1.631234in}{5.423055in}}%
\pgfpathlineto{\pgfqpoint{1.684278in}{4.875476in}}%
\pgfpathlineto{\pgfqpoint{1.737322in}{4.795628in}}%
\pgfpathlineto{\pgfqpoint{1.790366in}{4.778049in}}%
\pgfpathlineto{\pgfqpoint{1.843410in}{5.555456in}}%
\pgfpathlineto{\pgfqpoint{1.896454in}{5.555456in}}%
\pgfpathlineto{\pgfqpoint{1.949498in}{4.751360in}}%
\pgfpathlineto{\pgfqpoint{2.002542in}{5.555456in}}%
\pgfpathlineto{\pgfqpoint{2.055586in}{4.762166in}}%
\pgfpathlineto{\pgfqpoint{2.108629in}{5.048281in}}%
\pgfpathlineto{\pgfqpoint{2.161673in}{4.651814in}}%
\pgfpathlineto{\pgfqpoint{2.214717in}{5.555456in}}%
\pgfpathlineto{\pgfqpoint{2.267761in}{4.846792in}}%
\pgfpathlineto{\pgfqpoint{2.320805in}{4.582406in}}%
\pgfpathlineto{\pgfqpoint{2.373849in}{5.523078in}}%
\pgfpathlineto{\pgfqpoint{2.426893in}{5.555456in}}%
\pgfpathlineto{\pgfqpoint{2.479937in}{4.702326in}}%
\pgfpathlineto{\pgfqpoint{2.532981in}{4.558081in}}%
\pgfpathlineto{\pgfqpoint{2.586025in}{4.536270in}}%
\pgfpathlineto{\pgfqpoint{2.639069in}{4.531777in}}%
\pgfpathlineto{\pgfqpoint{2.692113in}{4.501574in}}%
\pgfpathlineto{\pgfqpoint{2.745157in}{4.572498in}}%
\pgfpathlineto{\pgfqpoint{2.798201in}{4.614894in}}%
\pgfpathlineto{\pgfqpoint{2.851245in}{5.097809in}}%
\pgfpathlineto{\pgfqpoint{2.904288in}{4.661801in}}%
\pgfpathlineto{\pgfqpoint{2.957332in}{5.555456in}}%
\pgfpathlineto{\pgfqpoint{3.010376in}{4.747565in}}%
\pgfpathlineto{\pgfqpoint{3.063420in}{5.529745in}}%
\pgfpathlineto{\pgfqpoint{3.116464in}{4.873985in}}%
\pgfpathlineto{\pgfqpoint{3.169508in}{4.779384in}}%
\pgfpathlineto{\pgfqpoint{3.222552in}{4.809907in}}%
\pgfpathlineto{\pgfqpoint{3.275596in}{4.726048in}}%
\pgfpathlineto{\pgfqpoint{3.328640in}{4.784623in}}%
\pgfpathlineto{\pgfqpoint{3.381684in}{4.982873in}}%
\pgfpathlineto{\pgfqpoint{3.434728in}{4.637008in}}%
\pgfpathlineto{\pgfqpoint{3.487772in}{5.331044in}}%
\pgfpathlineto{\pgfqpoint{3.540816in}{5.555456in}}%
\pgfpathlineto{\pgfqpoint{3.593860in}{5.052711in}}%
\pgfpathlineto{\pgfqpoint{3.646904in}{4.575309in}}%
\pgfpathlineto{\pgfqpoint{3.699948in}{4.531459in}}%
\pgfpathlineto{\pgfqpoint{3.752991in}{4.859785in}}%
\pgfpathlineto{\pgfqpoint{3.806035in}{4.571710in}}%
\pgfpathlineto{\pgfqpoint{3.859079in}{4.859785in}}%
\pgfpathlineto{\pgfqpoint{3.912123in}{4.859785in}}%
\pgfpathlineto{\pgfqpoint{3.965167in}{5.299182in}}%
\pgfpathlineto{\pgfqpoint{4.018211in}{4.577684in}}%
\pgfpathlineto{\pgfqpoint{4.071255in}{4.659775in}}%
\pgfpathlineto{\pgfqpoint{4.124299in}{4.787943in}}%
\pgfpathlineto{\pgfqpoint{4.177343in}{4.703664in}}%
\pgfpathlineto{\pgfqpoint{4.230387in}{4.741772in}}%
\pgfpathlineto{\pgfqpoint{4.283431in}{4.757491in}}%
\pgfpathlineto{\pgfqpoint{4.336475in}{5.555456in}}%
\pgfpathlineto{\pgfqpoint{4.389519in}{4.859785in}}%
\pgfpathlineto{\pgfqpoint{4.442563in}{4.896628in}}%
\pgfpathlineto{\pgfqpoint{4.495607in}{5.555456in}}%
\pgfpathlineto{\pgfqpoint{4.548650in}{5.101868in}}%
\pgfpathlineto{\pgfqpoint{4.601694in}{5.555456in}}%
\pgfpathlineto{\pgfqpoint{4.654738in}{4.759102in}}%
\pgfpathlineto{\pgfqpoint{4.707782in}{5.555456in}}%
\pgfpathlineto{\pgfqpoint{4.760826in}{4.951917in}}%
\pgfpathlineto{\pgfqpoint{4.813870in}{4.639365in}}%
\pgfpathlineto{\pgfqpoint{4.866914in}{5.555456in}}%
\pgfpathlineto{\pgfqpoint{4.919958in}{5.555456in}}%
\pgfpathlineto{\pgfqpoint{4.973002in}{4.520363in}}%
\pgfpathlineto{\pgfqpoint{5.026046in}{5.387465in}}%
\pgfpathlineto{\pgfqpoint{5.079090in}{4.491903in}}%
\pgfpathlineto{\pgfqpoint{5.132134in}{5.444876in}}%
\pgfpathlineto{\pgfqpoint{5.185178in}{5.555456in}}%
\pgfpathlineto{\pgfqpoint{5.238222in}{5.555456in}}%
\pgfpathlineto{\pgfqpoint{5.291266in}{4.639889in}}%
\pgfpathlineto{\pgfqpoint{5.344309in}{5.453871in}}%
\pgfpathlineto{\pgfqpoint{5.397353in}{5.555456in}}%
\pgfpathlineto{\pgfqpoint{5.450397in}{5.555456in}}%
\pgfpathlineto{\pgfqpoint{5.503441in}{4.781070in}}%
\pgfpathlineto{\pgfqpoint{5.556485in}{5.135041in}}%
\pgfpathlineto{\pgfqpoint{5.609529in}{5.555456in}}%
\pgfpathlineto{\pgfqpoint{5.662573in}{5.555456in}}%
\pgfpathlineto{\pgfqpoint{5.715617in}{4.786862in}}%
\pgfpathlineto{\pgfqpoint{5.768661in}{5.555456in}}%
\pgfpathlineto{\pgfqpoint{5.927793in}{5.555456in}}%
\pgfpathlineto{\pgfqpoint{5.980837in}{4.710506in}}%
\pgfpathlineto{\pgfqpoint{6.033881in}{5.193362in}}%
\pgfpathlineto{\pgfqpoint{6.086925in}{4.639368in}}%
\pgfpathlineto{\pgfqpoint{6.139969in}{4.620894in}}%
\pgfpathlineto{\pgfqpoint{6.193012in}{5.555456in}}%
\pgfpathlineto{\pgfqpoint{6.246056in}{5.555456in}}%
\pgfpathlineto{\pgfqpoint{6.299100in}{4.541397in}}%
\pgfpathlineto{\pgfqpoint{6.352144in}{5.178906in}}%
\pgfpathlineto{\pgfqpoint{6.405188in}{5.555456in}}%
\pgfpathlineto{\pgfqpoint{6.458232in}{5.077516in}}%
\pgfpathlineto{\pgfqpoint{6.511276in}{4.610991in}}%
\pgfpathlineto{\pgfqpoint{6.564320in}{4.607341in}}%
\pgfpathlineto{\pgfqpoint{6.617364in}{4.674298in}}%
\pgfpathlineto{\pgfqpoint{6.670408in}{4.704240in}}%
\pgfpathlineto{\pgfqpoint{6.723452in}{4.750321in}}%
\pgfpathlineto{\pgfqpoint{6.776496in}{4.859785in}}%
\pgfpathlineto{\pgfqpoint{6.829540in}{4.762288in}}%
\pgfpathlineto{\pgfqpoint{6.882584in}{5.555456in}}%
\pgfpathlineto{\pgfqpoint{6.935628in}{4.755942in}}%
\pgfpathlineto{\pgfqpoint{6.988671in}{5.000909in}}%
\pgfpathlineto{\pgfqpoint{7.041715in}{4.775204in}}%
\pgfpathlineto{\pgfqpoint{7.094759in}{4.798255in}}%
\pgfpathlineto{\pgfqpoint{7.147803in}{4.780049in}}%
\pgfpathlineto{\pgfqpoint{7.200847in}{5.555456in}}%
\pgfpathlineto{\pgfqpoint{7.253891in}{4.923953in}}%
\pgfpathlineto{\pgfqpoint{7.306935in}{4.759939in}}%
\pgfpathlineto{\pgfqpoint{7.359979in}{5.555456in}}%
\pgfpathlineto{\pgfqpoint{7.413023in}{5.555456in}}%
\pgfpathlineto{\pgfqpoint{7.466067in}{5.147602in}}%
\pgfpathlineto{\pgfqpoint{7.519111in}{5.074761in}}%
\pgfpathlineto{\pgfqpoint{7.572155in}{5.555456in}}%
\pgfpathlineto{\pgfqpoint{7.625199in}{5.555456in}}%
\pgfpathlineto{\pgfqpoint{7.678243in}{5.513983in}}%
\pgfpathlineto{\pgfqpoint{7.731287in}{4.573173in}}%
\pgfpathlineto{\pgfqpoint{7.784330in}{4.609205in}}%
\pgfpathlineto{\pgfqpoint{7.837374in}{4.617456in}}%
\pgfpathlineto{\pgfqpoint{7.890418in}{4.658759in}}%
\pgfpathlineto{\pgfqpoint{7.943462in}{4.656505in}}%
\pgfpathlineto{\pgfqpoint{7.996506in}{4.735208in}}%
\pgfpathlineto{\pgfqpoint{8.049550in}{5.555456in}}%
\pgfpathlineto{\pgfqpoint{8.102594in}{5.555456in}}%
\pgfpathlineto{\pgfqpoint{8.155638in}{4.801844in}}%
\pgfpathlineto{\pgfqpoint{8.208682in}{4.795985in}}%
\pgfpathlineto{\pgfqpoint{8.261726in}{5.555456in}}%
\pgfpathlineto{\pgfqpoint{8.314770in}{5.555456in}}%
\pgfpathlineto{\pgfqpoint{8.367814in}{4.778402in}}%
\pgfpathlineto{\pgfqpoint{8.420858in}{5.555456in}}%
\pgfpathlineto{\pgfqpoint{8.473902in}{4.719180in}}%
\pgfpathlineto{\pgfqpoint{8.526946in}{4.691039in}}%
\pgfpathlineto{\pgfqpoint{8.579990in}{5.555456in}}%
\pgfpathlineto{\pgfqpoint{8.739121in}{5.555456in}}%
\pgfpathlineto{\pgfqpoint{8.792165in}{5.420011in}}%
\pgfpathlineto{\pgfqpoint{8.845209in}{5.555456in}}%
\pgfpathlineto{\pgfqpoint{8.898253in}{5.555456in}}%
\pgfpathlineto{\pgfqpoint{8.951297in}{4.542245in}}%
\pgfpathlineto{\pgfqpoint{9.004341in}{4.990837in}}%
\pgfpathlineto{\pgfqpoint{9.057385in}{4.626605in}}%
\pgfpathlineto{\pgfqpoint{9.110429in}{4.661699in}}%
\pgfpathlineto{\pgfqpoint{9.163473in}{5.555456in}}%
\pgfpathlineto{\pgfqpoint{9.375649in}{5.555456in}}%
\pgfpathlineto{\pgfqpoint{9.428692in}{4.818062in}}%
\pgfpathlineto{\pgfqpoint{9.481736in}{4.859785in}}%
\pgfpathlineto{\pgfqpoint{9.534780in}{5.555456in}}%
\pgfpathlineto{\pgfqpoint{9.587824in}{5.555456in}}%
\pgfpathlineto{\pgfqpoint{9.640868in}{5.281518in}}%
\pgfpathlineto{\pgfqpoint{9.693912in}{5.555456in}}%
\pgfpathlineto{\pgfqpoint{9.800000in}{5.555456in}}%
\pgfpathlineto{\pgfqpoint{9.800000in}{5.555456in}}%
\pgfusepath{stroke}%
\end{pgfscope}%
\begin{pgfscope}%
\pgfpathrectangle{\pgfqpoint{0.941663in}{4.334375in}}{\pgfqpoint{8.858337in}{3.465625in}}%
\pgfusepath{clip}%
\pgfsetbuttcap%
\pgfsetroundjoin%
\definecolor{currentfill}{rgb}{0.501961,0.501961,0.501961}%
\pgfsetfillcolor{currentfill}%
\pgfsetlinewidth{1.003750pt}%
\definecolor{currentstroke}{rgb}{0.501961,0.501961,0.501961}%
\pgfsetstrokecolor{currentstroke}%
\pgfsetdash{}{0pt}%
\pgfsys@defobject{currentmarker}{\pgfqpoint{0.941663in}{4.491903in}}{\pgfqpoint{9.800000in}{5.555456in}}{%
\pgfpathmoveto{\pgfqpoint{0.941663in}{4.606409in}}%
\pgfpathlineto{\pgfqpoint{0.941663in}{5.302080in}}%
\pgfpathlineto{\pgfqpoint{0.994707in}{5.555456in}}%
\pgfpathlineto{\pgfqpoint{1.047751in}{5.298031in}}%
\pgfpathlineto{\pgfqpoint{1.100795in}{5.267250in}}%
\pgfpathlineto{\pgfqpoint{1.153839in}{5.253587in}}%
\pgfpathlineto{\pgfqpoint{1.206883in}{5.555456in}}%
\pgfpathlineto{\pgfqpoint{1.259927in}{5.555456in}}%
\pgfpathlineto{\pgfqpoint{1.312970in}{5.555456in}}%
\pgfpathlineto{\pgfqpoint{1.366014in}{5.207621in}}%
\pgfpathlineto{\pgfqpoint{1.419058in}{5.263460in}}%
\pgfpathlineto{\pgfqpoint{1.472102in}{5.268823in}}%
\pgfpathlineto{\pgfqpoint{1.525146in}{5.349185in}}%
\pgfpathlineto{\pgfqpoint{1.578190in}{5.343668in}}%
\pgfpathlineto{\pgfqpoint{1.631234in}{5.555456in}}%
\pgfpathlineto{\pgfqpoint{1.684278in}{5.555456in}}%
\pgfpathlineto{\pgfqpoint{1.737322in}{5.491300in}}%
\pgfpathlineto{\pgfqpoint{1.790366in}{5.473721in}}%
\pgfpathlineto{\pgfqpoint{1.843410in}{5.555456in}}%
\pgfpathlineto{\pgfqpoint{1.896454in}{5.555456in}}%
\pgfpathlineto{\pgfqpoint{1.949498in}{5.447032in}}%
\pgfpathlineto{\pgfqpoint{2.002542in}{5.555456in}}%
\pgfpathlineto{\pgfqpoint{2.055586in}{5.457838in}}%
\pgfpathlineto{\pgfqpoint{2.108629in}{5.555456in}}%
\pgfpathlineto{\pgfqpoint{2.161673in}{5.347486in}}%
\pgfpathlineto{\pgfqpoint{2.214717in}{5.555456in}}%
\pgfpathlineto{\pgfqpoint{2.267761in}{5.542464in}}%
\pgfpathlineto{\pgfqpoint{2.320805in}{5.278077in}}%
\pgfpathlineto{\pgfqpoint{2.373849in}{5.555456in}}%
\pgfpathlineto{\pgfqpoint{2.426893in}{5.555456in}}%
\pgfpathlineto{\pgfqpoint{2.479937in}{5.397997in}}%
\pgfpathlineto{\pgfqpoint{2.532981in}{5.253753in}}%
\pgfpathlineto{\pgfqpoint{2.586025in}{5.231942in}}%
\pgfpathlineto{\pgfqpoint{2.639069in}{5.227448in}}%
\pgfpathlineto{\pgfqpoint{2.692113in}{5.274799in}}%
\pgfpathlineto{\pgfqpoint{2.745157in}{5.555456in}}%
\pgfpathlineto{\pgfqpoint{2.798201in}{5.555456in}}%
\pgfpathlineto{\pgfqpoint{2.851245in}{5.555456in}}%
\pgfpathlineto{\pgfqpoint{2.904288in}{5.357472in}}%
\pgfpathlineto{\pgfqpoint{2.957332in}{5.555456in}}%
\pgfpathlineto{\pgfqpoint{3.010376in}{5.443237in}}%
\pgfpathlineto{\pgfqpoint{3.063420in}{5.555456in}}%
\pgfpathlineto{\pgfqpoint{3.116464in}{5.555456in}}%
\pgfpathlineto{\pgfqpoint{3.169508in}{5.475056in}}%
\pgfpathlineto{\pgfqpoint{3.222552in}{5.505579in}}%
\pgfpathlineto{\pgfqpoint{3.275596in}{5.421720in}}%
\pgfpathlineto{\pgfqpoint{3.328640in}{5.480294in}}%
\pgfpathlineto{\pgfqpoint{3.381684in}{5.555456in}}%
\pgfpathlineto{\pgfqpoint{3.434728in}{5.332680in}}%
\pgfpathlineto{\pgfqpoint{3.487772in}{5.555456in}}%
\pgfpathlineto{\pgfqpoint{3.540816in}{5.555456in}}%
\pgfpathlineto{\pgfqpoint{3.593860in}{5.555456in}}%
\pgfpathlineto{\pgfqpoint{3.646904in}{5.270980in}}%
\pgfpathlineto{\pgfqpoint{3.699948in}{5.227131in}}%
\pgfpathlineto{\pgfqpoint{3.752991in}{5.555456in}}%
\pgfpathlineto{\pgfqpoint{3.806035in}{5.267382in}}%
\pgfpathlineto{\pgfqpoint{3.859079in}{5.555456in}}%
\pgfpathlineto{\pgfqpoint{3.912123in}{5.555456in}}%
\pgfpathlineto{\pgfqpoint{3.965167in}{5.555456in}}%
\pgfpathlineto{\pgfqpoint{4.018211in}{5.273356in}}%
\pgfpathlineto{\pgfqpoint{4.071255in}{5.355447in}}%
\pgfpathlineto{\pgfqpoint{4.124299in}{5.483615in}}%
\pgfpathlineto{\pgfqpoint{4.177343in}{5.399336in}}%
\pgfpathlineto{\pgfqpoint{4.230387in}{5.437444in}}%
\pgfpathlineto{\pgfqpoint{4.283431in}{5.453162in}}%
\pgfpathlineto{\pgfqpoint{4.336475in}{5.555456in}}%
\pgfpathlineto{\pgfqpoint{4.389519in}{5.555456in}}%
\pgfpathlineto{\pgfqpoint{4.442563in}{5.555456in}}%
\pgfpathlineto{\pgfqpoint{4.495607in}{5.555456in}}%
\pgfpathlineto{\pgfqpoint{4.548650in}{5.555456in}}%
\pgfpathlineto{\pgfqpoint{4.601694in}{5.555456in}}%
\pgfpathlineto{\pgfqpoint{4.654738in}{5.454774in}}%
\pgfpathlineto{\pgfqpoint{4.707782in}{5.555456in}}%
\pgfpathlineto{\pgfqpoint{4.760826in}{5.555456in}}%
\pgfpathlineto{\pgfqpoint{4.813870in}{5.335037in}}%
\pgfpathlineto{\pgfqpoint{4.866914in}{5.555456in}}%
\pgfpathlineto{\pgfqpoint{4.919958in}{5.555456in}}%
\pgfpathlineto{\pgfqpoint{4.973002in}{5.216034in}}%
\pgfpathlineto{\pgfqpoint{5.026046in}{5.555456in}}%
\pgfpathlineto{\pgfqpoint{5.079090in}{5.207621in}}%
\pgfpathlineto{\pgfqpoint{5.132134in}{5.555456in}}%
\pgfpathlineto{\pgfqpoint{5.185178in}{5.555456in}}%
\pgfpathlineto{\pgfqpoint{5.238222in}{5.555456in}}%
\pgfpathlineto{\pgfqpoint{5.291266in}{5.335561in}}%
\pgfpathlineto{\pgfqpoint{5.344309in}{5.555456in}}%
\pgfpathlineto{\pgfqpoint{5.397353in}{5.555456in}}%
\pgfpathlineto{\pgfqpoint{5.450397in}{5.555456in}}%
\pgfpathlineto{\pgfqpoint{5.503441in}{5.476742in}}%
\pgfpathlineto{\pgfqpoint{5.556485in}{5.555456in}}%
\pgfpathlineto{\pgfqpoint{5.609529in}{5.555456in}}%
\pgfpathlineto{\pgfqpoint{5.662573in}{5.555456in}}%
\pgfpathlineto{\pgfqpoint{5.715617in}{5.482534in}}%
\pgfpathlineto{\pgfqpoint{5.768661in}{5.555456in}}%
\pgfpathlineto{\pgfqpoint{5.821705in}{5.555456in}}%
\pgfpathlineto{\pgfqpoint{5.874749in}{5.555456in}}%
\pgfpathlineto{\pgfqpoint{5.927793in}{5.555456in}}%
\pgfpathlineto{\pgfqpoint{5.980837in}{5.406177in}}%
\pgfpathlineto{\pgfqpoint{6.033881in}{5.555456in}}%
\pgfpathlineto{\pgfqpoint{6.086925in}{5.335040in}}%
\pgfpathlineto{\pgfqpoint{6.139969in}{5.316566in}}%
\pgfpathlineto{\pgfqpoint{6.193012in}{5.555456in}}%
\pgfpathlineto{\pgfqpoint{6.246056in}{5.555456in}}%
\pgfpathlineto{\pgfqpoint{6.299100in}{5.237069in}}%
\pgfpathlineto{\pgfqpoint{6.352144in}{5.555456in}}%
\pgfpathlineto{\pgfqpoint{6.405188in}{5.555456in}}%
\pgfpathlineto{\pgfqpoint{6.458232in}{5.555456in}}%
\pgfpathlineto{\pgfqpoint{6.511276in}{5.306663in}}%
\pgfpathlineto{\pgfqpoint{6.564320in}{5.303012in}}%
\pgfpathlineto{\pgfqpoint{6.617364in}{5.369970in}}%
\pgfpathlineto{\pgfqpoint{6.670408in}{5.399912in}}%
\pgfpathlineto{\pgfqpoint{6.723452in}{5.445993in}}%
\pgfpathlineto{\pgfqpoint{6.776496in}{5.555456in}}%
\pgfpathlineto{\pgfqpoint{6.829540in}{5.457960in}}%
\pgfpathlineto{\pgfqpoint{6.882584in}{5.555456in}}%
\pgfpathlineto{\pgfqpoint{6.935628in}{5.451613in}}%
\pgfpathlineto{\pgfqpoint{6.988671in}{5.555456in}}%
\pgfpathlineto{\pgfqpoint{7.041715in}{5.470876in}}%
\pgfpathlineto{\pgfqpoint{7.094759in}{5.493927in}}%
\pgfpathlineto{\pgfqpoint{7.147803in}{5.475721in}}%
\pgfpathlineto{\pgfqpoint{7.200847in}{5.555456in}}%
\pgfpathlineto{\pgfqpoint{7.253891in}{5.555456in}}%
\pgfpathlineto{\pgfqpoint{7.306935in}{5.455611in}}%
\pgfpathlineto{\pgfqpoint{7.359979in}{5.555456in}}%
\pgfpathlineto{\pgfqpoint{7.413023in}{5.555456in}}%
\pgfpathlineto{\pgfqpoint{7.466067in}{5.555456in}}%
\pgfpathlineto{\pgfqpoint{7.519111in}{5.555456in}}%
\pgfpathlineto{\pgfqpoint{7.572155in}{5.555456in}}%
\pgfpathlineto{\pgfqpoint{7.625199in}{5.555456in}}%
\pgfpathlineto{\pgfqpoint{7.678243in}{5.555456in}}%
\pgfpathlineto{\pgfqpoint{7.731287in}{5.268845in}}%
\pgfpathlineto{\pgfqpoint{7.784330in}{5.304877in}}%
\pgfpathlineto{\pgfqpoint{7.837374in}{5.313128in}}%
\pgfpathlineto{\pgfqpoint{7.890418in}{5.354431in}}%
\pgfpathlineto{\pgfqpoint{7.943462in}{5.352176in}}%
\pgfpathlineto{\pgfqpoint{7.996506in}{5.430879in}}%
\pgfpathlineto{\pgfqpoint{8.049550in}{5.555456in}}%
\pgfpathlineto{\pgfqpoint{8.102594in}{5.555456in}}%
\pgfpathlineto{\pgfqpoint{8.155638in}{5.497516in}}%
\pgfpathlineto{\pgfqpoint{8.208682in}{5.491656in}}%
\pgfpathlineto{\pgfqpoint{8.261726in}{5.555456in}}%
\pgfpathlineto{\pgfqpoint{8.314770in}{5.555456in}}%
\pgfpathlineto{\pgfqpoint{8.367814in}{5.474073in}}%
\pgfpathlineto{\pgfqpoint{8.420858in}{5.555456in}}%
\pgfpathlineto{\pgfqpoint{8.473902in}{5.414851in}}%
\pgfpathlineto{\pgfqpoint{8.526946in}{5.386710in}}%
\pgfpathlineto{\pgfqpoint{8.579990in}{5.555456in}}%
\pgfpathlineto{\pgfqpoint{8.633033in}{5.555456in}}%
\pgfpathlineto{\pgfqpoint{8.686077in}{5.555456in}}%
\pgfpathlineto{\pgfqpoint{8.739121in}{5.555456in}}%
\pgfpathlineto{\pgfqpoint{8.792165in}{5.555456in}}%
\pgfpathlineto{\pgfqpoint{8.845209in}{5.555456in}}%
\pgfpathlineto{\pgfqpoint{8.898253in}{5.555456in}}%
\pgfpathlineto{\pgfqpoint{8.951297in}{5.237916in}}%
\pgfpathlineto{\pgfqpoint{9.004341in}{5.555456in}}%
\pgfpathlineto{\pgfqpoint{9.057385in}{5.322277in}}%
\pgfpathlineto{\pgfqpoint{9.110429in}{5.357370in}}%
\pgfpathlineto{\pgfqpoint{9.163473in}{5.555456in}}%
\pgfpathlineto{\pgfqpoint{9.216517in}{5.555456in}}%
\pgfpathlineto{\pgfqpoint{9.269561in}{5.555456in}}%
\pgfpathlineto{\pgfqpoint{9.322605in}{5.555456in}}%
\pgfpathlineto{\pgfqpoint{9.375649in}{5.555456in}}%
\pgfpathlineto{\pgfqpoint{9.428692in}{5.513734in}}%
\pgfpathlineto{\pgfqpoint{9.481736in}{5.555456in}}%
\pgfpathlineto{\pgfqpoint{9.534780in}{5.555456in}}%
\pgfpathlineto{\pgfqpoint{9.587824in}{5.555456in}}%
\pgfpathlineto{\pgfqpoint{9.640868in}{5.555456in}}%
\pgfpathlineto{\pgfqpoint{9.693912in}{5.555456in}}%
\pgfpathlineto{\pgfqpoint{9.746956in}{5.555456in}}%
\pgfpathlineto{\pgfqpoint{9.800000in}{5.555456in}}%
\pgfpathlineto{\pgfqpoint{9.800000in}{5.555456in}}%
\pgfpathlineto{\pgfqpoint{9.800000in}{5.555456in}}%
\pgfpathlineto{\pgfqpoint{9.746956in}{5.555456in}}%
\pgfpathlineto{\pgfqpoint{9.693912in}{5.555456in}}%
\pgfpathlineto{\pgfqpoint{9.640868in}{5.281518in}}%
\pgfpathlineto{\pgfqpoint{9.587824in}{5.555456in}}%
\pgfpathlineto{\pgfqpoint{9.534780in}{5.555456in}}%
\pgfpathlineto{\pgfqpoint{9.481736in}{4.859785in}}%
\pgfpathlineto{\pgfqpoint{9.428692in}{4.818062in}}%
\pgfpathlineto{\pgfqpoint{9.375649in}{5.555456in}}%
\pgfpathlineto{\pgfqpoint{9.322605in}{5.555456in}}%
\pgfpathlineto{\pgfqpoint{9.269561in}{5.555456in}}%
\pgfpathlineto{\pgfqpoint{9.216517in}{5.555456in}}%
\pgfpathlineto{\pgfqpoint{9.163473in}{5.555456in}}%
\pgfpathlineto{\pgfqpoint{9.110429in}{4.661699in}}%
\pgfpathlineto{\pgfqpoint{9.057385in}{4.626605in}}%
\pgfpathlineto{\pgfqpoint{9.004341in}{4.990837in}}%
\pgfpathlineto{\pgfqpoint{8.951297in}{4.542245in}}%
\pgfpathlineto{\pgfqpoint{8.898253in}{5.555456in}}%
\pgfpathlineto{\pgfqpoint{8.845209in}{5.555456in}}%
\pgfpathlineto{\pgfqpoint{8.792165in}{5.420011in}}%
\pgfpathlineto{\pgfqpoint{8.739121in}{5.555456in}}%
\pgfpathlineto{\pgfqpoint{8.686077in}{5.555456in}}%
\pgfpathlineto{\pgfqpoint{8.633033in}{5.555456in}}%
\pgfpathlineto{\pgfqpoint{8.579990in}{5.555456in}}%
\pgfpathlineto{\pgfqpoint{8.526946in}{4.691039in}}%
\pgfpathlineto{\pgfqpoint{8.473902in}{4.719180in}}%
\pgfpathlineto{\pgfqpoint{8.420858in}{5.555456in}}%
\pgfpathlineto{\pgfqpoint{8.367814in}{4.778402in}}%
\pgfpathlineto{\pgfqpoint{8.314770in}{5.555456in}}%
\pgfpathlineto{\pgfqpoint{8.261726in}{5.555456in}}%
\pgfpathlineto{\pgfqpoint{8.208682in}{4.795985in}}%
\pgfpathlineto{\pgfqpoint{8.155638in}{4.801844in}}%
\pgfpathlineto{\pgfqpoint{8.102594in}{5.555456in}}%
\pgfpathlineto{\pgfqpoint{8.049550in}{5.555456in}}%
\pgfpathlineto{\pgfqpoint{7.996506in}{4.735208in}}%
\pgfpathlineto{\pgfqpoint{7.943462in}{4.656505in}}%
\pgfpathlineto{\pgfqpoint{7.890418in}{4.658759in}}%
\pgfpathlineto{\pgfqpoint{7.837374in}{4.617456in}}%
\pgfpathlineto{\pgfqpoint{7.784330in}{4.609205in}}%
\pgfpathlineto{\pgfqpoint{7.731287in}{4.573173in}}%
\pgfpathlineto{\pgfqpoint{7.678243in}{5.513983in}}%
\pgfpathlineto{\pgfqpoint{7.625199in}{5.555456in}}%
\pgfpathlineto{\pgfqpoint{7.572155in}{5.555456in}}%
\pgfpathlineto{\pgfqpoint{7.519111in}{5.074761in}}%
\pgfpathlineto{\pgfqpoint{7.466067in}{5.147602in}}%
\pgfpathlineto{\pgfqpoint{7.413023in}{5.555456in}}%
\pgfpathlineto{\pgfqpoint{7.359979in}{5.555456in}}%
\pgfpathlineto{\pgfqpoint{7.306935in}{4.759939in}}%
\pgfpathlineto{\pgfqpoint{7.253891in}{4.923953in}}%
\pgfpathlineto{\pgfqpoint{7.200847in}{5.555456in}}%
\pgfpathlineto{\pgfqpoint{7.147803in}{4.780049in}}%
\pgfpathlineto{\pgfqpoint{7.094759in}{4.798255in}}%
\pgfpathlineto{\pgfqpoint{7.041715in}{4.775204in}}%
\pgfpathlineto{\pgfqpoint{6.988671in}{5.000909in}}%
\pgfpathlineto{\pgfqpoint{6.935628in}{4.755942in}}%
\pgfpathlineto{\pgfqpoint{6.882584in}{5.555456in}}%
\pgfpathlineto{\pgfqpoint{6.829540in}{4.762288in}}%
\pgfpathlineto{\pgfqpoint{6.776496in}{4.859785in}}%
\pgfpathlineto{\pgfqpoint{6.723452in}{4.750321in}}%
\pgfpathlineto{\pgfqpoint{6.670408in}{4.704240in}}%
\pgfpathlineto{\pgfqpoint{6.617364in}{4.674298in}}%
\pgfpathlineto{\pgfqpoint{6.564320in}{4.607341in}}%
\pgfpathlineto{\pgfqpoint{6.511276in}{4.610991in}}%
\pgfpathlineto{\pgfqpoint{6.458232in}{5.077516in}}%
\pgfpathlineto{\pgfqpoint{6.405188in}{5.555456in}}%
\pgfpathlineto{\pgfqpoint{6.352144in}{5.178906in}}%
\pgfpathlineto{\pgfqpoint{6.299100in}{4.541397in}}%
\pgfpathlineto{\pgfqpoint{6.246056in}{5.555456in}}%
\pgfpathlineto{\pgfqpoint{6.193012in}{5.555456in}}%
\pgfpathlineto{\pgfqpoint{6.139969in}{4.620894in}}%
\pgfpathlineto{\pgfqpoint{6.086925in}{4.639368in}}%
\pgfpathlineto{\pgfqpoint{6.033881in}{5.193362in}}%
\pgfpathlineto{\pgfqpoint{5.980837in}{4.710506in}}%
\pgfpathlineto{\pgfqpoint{5.927793in}{5.555456in}}%
\pgfpathlineto{\pgfqpoint{5.874749in}{5.555456in}}%
\pgfpathlineto{\pgfqpoint{5.821705in}{5.555456in}}%
\pgfpathlineto{\pgfqpoint{5.768661in}{5.555456in}}%
\pgfpathlineto{\pgfqpoint{5.715617in}{4.786862in}}%
\pgfpathlineto{\pgfqpoint{5.662573in}{5.555456in}}%
\pgfpathlineto{\pgfqpoint{5.609529in}{5.555456in}}%
\pgfpathlineto{\pgfqpoint{5.556485in}{5.135041in}}%
\pgfpathlineto{\pgfqpoint{5.503441in}{4.781070in}}%
\pgfpathlineto{\pgfqpoint{5.450397in}{5.555456in}}%
\pgfpathlineto{\pgfqpoint{5.397353in}{5.555456in}}%
\pgfpathlineto{\pgfqpoint{5.344309in}{5.453871in}}%
\pgfpathlineto{\pgfqpoint{5.291266in}{4.639889in}}%
\pgfpathlineto{\pgfqpoint{5.238222in}{5.555456in}}%
\pgfpathlineto{\pgfqpoint{5.185178in}{5.555456in}}%
\pgfpathlineto{\pgfqpoint{5.132134in}{5.444876in}}%
\pgfpathlineto{\pgfqpoint{5.079090in}{4.491903in}}%
\pgfpathlineto{\pgfqpoint{5.026046in}{5.387465in}}%
\pgfpathlineto{\pgfqpoint{4.973002in}{4.520363in}}%
\pgfpathlineto{\pgfqpoint{4.919958in}{5.555456in}}%
\pgfpathlineto{\pgfqpoint{4.866914in}{5.555456in}}%
\pgfpathlineto{\pgfqpoint{4.813870in}{4.639365in}}%
\pgfpathlineto{\pgfqpoint{4.760826in}{4.951917in}}%
\pgfpathlineto{\pgfqpoint{4.707782in}{5.555456in}}%
\pgfpathlineto{\pgfqpoint{4.654738in}{4.759102in}}%
\pgfpathlineto{\pgfqpoint{4.601694in}{5.555456in}}%
\pgfpathlineto{\pgfqpoint{4.548650in}{5.101868in}}%
\pgfpathlineto{\pgfqpoint{4.495607in}{5.555456in}}%
\pgfpathlineto{\pgfqpoint{4.442563in}{4.896628in}}%
\pgfpathlineto{\pgfqpoint{4.389519in}{4.859785in}}%
\pgfpathlineto{\pgfqpoint{4.336475in}{5.555456in}}%
\pgfpathlineto{\pgfqpoint{4.283431in}{4.757491in}}%
\pgfpathlineto{\pgfqpoint{4.230387in}{4.741772in}}%
\pgfpathlineto{\pgfqpoint{4.177343in}{4.703664in}}%
\pgfpathlineto{\pgfqpoint{4.124299in}{4.787943in}}%
\pgfpathlineto{\pgfqpoint{4.071255in}{4.659775in}}%
\pgfpathlineto{\pgfqpoint{4.018211in}{4.577684in}}%
\pgfpathlineto{\pgfqpoint{3.965167in}{5.299182in}}%
\pgfpathlineto{\pgfqpoint{3.912123in}{4.859785in}}%
\pgfpathlineto{\pgfqpoint{3.859079in}{4.859785in}}%
\pgfpathlineto{\pgfqpoint{3.806035in}{4.571710in}}%
\pgfpathlineto{\pgfqpoint{3.752991in}{4.859785in}}%
\pgfpathlineto{\pgfqpoint{3.699948in}{4.531459in}}%
\pgfpathlineto{\pgfqpoint{3.646904in}{4.575309in}}%
\pgfpathlineto{\pgfqpoint{3.593860in}{5.052711in}}%
\pgfpathlineto{\pgfqpoint{3.540816in}{5.555456in}}%
\pgfpathlineto{\pgfqpoint{3.487772in}{5.331044in}}%
\pgfpathlineto{\pgfqpoint{3.434728in}{4.637008in}}%
\pgfpathlineto{\pgfqpoint{3.381684in}{4.982873in}}%
\pgfpathlineto{\pgfqpoint{3.328640in}{4.784623in}}%
\pgfpathlineto{\pgfqpoint{3.275596in}{4.726048in}}%
\pgfpathlineto{\pgfqpoint{3.222552in}{4.809907in}}%
\pgfpathlineto{\pgfqpoint{3.169508in}{4.779384in}}%
\pgfpathlineto{\pgfqpoint{3.116464in}{4.873985in}}%
\pgfpathlineto{\pgfqpoint{3.063420in}{5.529745in}}%
\pgfpathlineto{\pgfqpoint{3.010376in}{4.747565in}}%
\pgfpathlineto{\pgfqpoint{2.957332in}{5.555456in}}%
\pgfpathlineto{\pgfqpoint{2.904288in}{4.661801in}}%
\pgfpathlineto{\pgfqpoint{2.851245in}{5.097809in}}%
\pgfpathlineto{\pgfqpoint{2.798201in}{4.614894in}}%
\pgfpathlineto{\pgfqpoint{2.745157in}{4.572498in}}%
\pgfpathlineto{\pgfqpoint{2.692113in}{4.501574in}}%
\pgfpathlineto{\pgfqpoint{2.639069in}{4.531777in}}%
\pgfpathlineto{\pgfqpoint{2.586025in}{4.536270in}}%
\pgfpathlineto{\pgfqpoint{2.532981in}{4.558081in}}%
\pgfpathlineto{\pgfqpoint{2.479937in}{4.702326in}}%
\pgfpathlineto{\pgfqpoint{2.426893in}{5.555456in}}%
\pgfpathlineto{\pgfqpoint{2.373849in}{5.523078in}}%
\pgfpathlineto{\pgfqpoint{2.320805in}{4.582406in}}%
\pgfpathlineto{\pgfqpoint{2.267761in}{4.846792in}}%
\pgfpathlineto{\pgfqpoint{2.214717in}{5.555456in}}%
\pgfpathlineto{\pgfqpoint{2.161673in}{4.651814in}}%
\pgfpathlineto{\pgfqpoint{2.108629in}{5.048281in}}%
\pgfpathlineto{\pgfqpoint{2.055586in}{4.762166in}}%
\pgfpathlineto{\pgfqpoint{2.002542in}{5.555456in}}%
\pgfpathlineto{\pgfqpoint{1.949498in}{4.751360in}}%
\pgfpathlineto{\pgfqpoint{1.896454in}{5.555456in}}%
\pgfpathlineto{\pgfqpoint{1.843410in}{5.555456in}}%
\pgfpathlineto{\pgfqpoint{1.790366in}{4.778049in}}%
\pgfpathlineto{\pgfqpoint{1.737322in}{4.795628in}}%
\pgfpathlineto{\pgfqpoint{1.684278in}{4.875476in}}%
\pgfpathlineto{\pgfqpoint{1.631234in}{5.423055in}}%
\pgfpathlineto{\pgfqpoint{1.578190in}{4.630644in}}%
\pgfpathlineto{\pgfqpoint{1.525146in}{4.653514in}}%
\pgfpathlineto{\pgfqpoint{1.472102in}{4.573151in}}%
\pgfpathlineto{\pgfqpoint{1.419058in}{4.567788in}}%
\pgfpathlineto{\pgfqpoint{1.366014in}{4.493572in}}%
\pgfpathlineto{\pgfqpoint{1.312970in}{5.172617in}}%
\pgfpathlineto{\pgfqpoint{1.259927in}{5.369914in}}%
\pgfpathlineto{\pgfqpoint{1.206883in}{5.127392in}}%
\pgfpathlineto{\pgfqpoint{1.153839in}{4.557915in}}%
\pgfpathlineto{\pgfqpoint{1.100795in}{4.571579in}}%
\pgfpathlineto{\pgfqpoint{1.047751in}{4.602360in}}%
\pgfpathlineto{\pgfqpoint{0.994707in}{4.859785in}}%
\pgfpathlineto{\pgfqpoint{0.941663in}{4.606409in}}%
\pgfpathlineto{\pgfqpoint{0.941663in}{4.606409in}}%
\pgfpathclose%
\pgfusepath{stroke,fill}%
}%
\begin{pgfscope}%
\pgfsys@transformshift{0.000000in}{0.000000in}%
\pgfsys@useobject{currentmarker}{}%
\end{pgfscope}%
\end{pgfscope}%
\begin{pgfscope}%
\pgfpathrectangle{\pgfqpoint{0.941663in}{4.334375in}}{\pgfqpoint{8.858337in}{3.465625in}}%
\pgfusepath{clip}%
\pgfsetrectcap%
\pgfsetroundjoin%
\pgfsetlinewidth{1.505625pt}%
\definecolor{currentstroke}{rgb}{0.090196,0.745098,0.811765}%
\pgfsetstrokecolor{currentstroke}%
\pgfsetdash{}{0pt}%
\pgfpathmoveto{\pgfqpoint{0.941663in}{7.642472in}}%
\pgfpathlineto{\pgfqpoint{0.994707in}{7.398889in}}%
\pgfpathlineto{\pgfqpoint{1.047751in}{7.642472in}}%
\pgfpathlineto{\pgfqpoint{1.100795in}{7.630782in}}%
\pgfpathlineto{\pgfqpoint{1.153839in}{7.642472in}}%
\pgfpathlineto{\pgfqpoint{1.206883in}{7.113972in}}%
\pgfpathlineto{\pgfqpoint{1.259927in}{6.752271in}}%
\pgfpathlineto{\pgfqpoint{1.312970in}{7.018171in}}%
\pgfpathlineto{\pgfqpoint{1.366014in}{7.642472in}}%
\pgfpathlineto{\pgfqpoint{1.578190in}{7.642472in}}%
\pgfpathlineto{\pgfqpoint{1.631234in}{6.920473in}}%
\pgfpathlineto{\pgfqpoint{1.684278in}{7.452967in}}%
\pgfpathlineto{\pgfqpoint{1.737322in}{7.536394in}}%
\pgfpathlineto{\pgfqpoint{1.790366in}{7.642472in}}%
\pgfpathlineto{\pgfqpoint{1.843410in}{6.911654in}}%
\pgfpathlineto{\pgfqpoint{1.896454in}{6.883587in}}%
\pgfpathlineto{\pgfqpoint{1.949498in}{7.642472in}}%
\pgfpathlineto{\pgfqpoint{2.002542in}{6.857739in}}%
\pgfpathlineto{\pgfqpoint{2.055586in}{7.642472in}}%
\pgfpathlineto{\pgfqpoint{2.108629in}{7.339659in}}%
\pgfpathlineto{\pgfqpoint{2.161673in}{7.642472in}}%
\pgfpathlineto{\pgfqpoint{2.214717in}{6.753348in}}%
\pgfpathlineto{\pgfqpoint{2.267761in}{7.419218in}}%
\pgfpathlineto{\pgfqpoint{2.320805in}{7.642472in}}%
\pgfpathlineto{\pgfqpoint{2.373849in}{6.676762in}}%
\pgfpathlineto{\pgfqpoint{2.426893in}{6.606651in}}%
\pgfpathlineto{\pgfqpoint{2.479937in}{7.449769in}}%
\pgfpathlineto{\pgfqpoint{2.532981in}{7.642472in}}%
\pgfpathlineto{\pgfqpoint{2.798201in}{7.642472in}}%
\pgfpathlineto{\pgfqpoint{2.851245in}{7.173688in}}%
\pgfpathlineto{\pgfqpoint{2.904288in}{7.642472in}}%
\pgfpathlineto{\pgfqpoint{2.957332in}{6.836531in}}%
\pgfpathlineto{\pgfqpoint{3.010376in}{7.642472in}}%
\pgfpathlineto{\pgfqpoint{3.063420in}{6.841623in}}%
\pgfpathlineto{\pgfqpoint{3.116464in}{7.518262in}}%
\pgfpathlineto{\pgfqpoint{3.169508in}{7.642472in}}%
\pgfpathlineto{\pgfqpoint{3.328640in}{7.642472in}}%
\pgfpathlineto{\pgfqpoint{3.381684in}{7.361194in}}%
\pgfpathlineto{\pgfqpoint{3.434728in}{7.642472in}}%
\pgfpathlineto{\pgfqpoint{3.487772in}{6.972573in}}%
\pgfpathlineto{\pgfqpoint{3.540816in}{6.738883in}}%
\pgfpathlineto{\pgfqpoint{3.593860in}{7.167623in}}%
\pgfpathlineto{\pgfqpoint{3.646904in}{7.642472in}}%
\pgfpathlineto{\pgfqpoint{3.699948in}{7.642472in}}%
\pgfpathlineto{\pgfqpoint{3.752991in}{7.311866in}}%
\pgfpathlineto{\pgfqpoint{3.806035in}{7.642472in}}%
\pgfpathlineto{\pgfqpoint{3.859079in}{7.341795in}}%
\pgfpathlineto{\pgfqpoint{3.912123in}{7.302849in}}%
\pgfpathlineto{\pgfqpoint{3.965167in}{6.991885in}}%
\pgfpathlineto{\pgfqpoint{4.018211in}{7.642472in}}%
\pgfpathlineto{\pgfqpoint{4.071255in}{7.642472in}}%
\pgfpathlineto{\pgfqpoint{4.124299in}{7.539721in}}%
\pgfpathlineto{\pgfqpoint{4.177343in}{7.642472in}}%
\pgfpathlineto{\pgfqpoint{4.283431in}{7.642472in}}%
\pgfpathlineto{\pgfqpoint{4.336475in}{6.889082in}}%
\pgfpathlineto{\pgfqpoint{4.389519in}{7.554209in}}%
\pgfpathlineto{\pgfqpoint{4.442563in}{7.506419in}}%
\pgfpathlineto{\pgfqpoint{4.495607in}{6.883010in}}%
\pgfpathlineto{\pgfqpoint{4.548650in}{7.353421in}}%
\pgfpathlineto{\pgfqpoint{4.601694in}{6.827112in}}%
\pgfpathlineto{\pgfqpoint{4.654738in}{7.642472in}}%
\pgfpathlineto{\pgfqpoint{4.707782in}{6.766257in}}%
\pgfpathlineto{\pgfqpoint{4.760826in}{7.378709in}}%
\pgfpathlineto{\pgfqpoint{4.813870in}{7.642472in}}%
\pgfpathlineto{\pgfqpoint{4.866914in}{6.684929in}}%
\pgfpathlineto{\pgfqpoint{4.919958in}{6.671298in}}%
\pgfpathlineto{\pgfqpoint{4.973002in}{7.642472in}}%
\pgfpathlineto{\pgfqpoint{5.026046in}{6.790800in}}%
\pgfpathlineto{\pgfqpoint{5.079090in}{7.642472in}}%
\pgfpathlineto{\pgfqpoint{5.132134in}{6.692849in}}%
\pgfpathlineto{\pgfqpoint{5.185178in}{6.670254in}}%
\pgfpathlineto{\pgfqpoint{5.238222in}{6.666672in}}%
\pgfpathlineto{\pgfqpoint{5.291266in}{7.642472in}}%
\pgfpathlineto{\pgfqpoint{5.344309in}{6.838988in}}%
\pgfpathlineto{\pgfqpoint{5.397353in}{6.748316in}}%
\pgfpathlineto{\pgfqpoint{5.450397in}{6.877918in}}%
\pgfpathlineto{\pgfqpoint{5.503441in}{7.642472in}}%
\pgfpathlineto{\pgfqpoint{5.556485in}{7.281307in}}%
\pgfpathlineto{\pgfqpoint{5.609529in}{6.874196in}}%
\pgfpathlineto{\pgfqpoint{5.662573in}{6.925790in}}%
\pgfpathlineto{\pgfqpoint{5.715617in}{7.642472in}}%
\pgfpathlineto{\pgfqpoint{5.768661in}{6.913209in}}%
\pgfpathlineto{\pgfqpoint{5.821705in}{6.872997in}}%
\pgfpathlineto{\pgfqpoint{5.874749in}{6.796671in}}%
\pgfpathlineto{\pgfqpoint{5.927793in}{6.820928in}}%
\pgfpathlineto{\pgfqpoint{5.980837in}{7.642472in}}%
\pgfpathlineto{\pgfqpoint{6.033881in}{7.110408in}}%
\pgfpathlineto{\pgfqpoint{6.086925in}{7.642472in}}%
\pgfpathlineto{\pgfqpoint{6.139969in}{7.642472in}}%
\pgfpathlineto{\pgfqpoint{6.193012in}{6.635398in}}%
\pgfpathlineto{\pgfqpoint{6.246056in}{6.691111in}}%
\pgfpathlineto{\pgfqpoint{6.299100in}{7.642472in}}%
\pgfpathlineto{\pgfqpoint{6.352144in}{7.023749in}}%
\pgfpathlineto{\pgfqpoint{6.405188in}{6.593987in}}%
\pgfpathlineto{\pgfqpoint{6.458232in}{7.158194in}}%
\pgfpathlineto{\pgfqpoint{6.511276in}{7.642472in}}%
\pgfpathlineto{\pgfqpoint{6.723452in}{7.642472in}}%
\pgfpathlineto{\pgfqpoint{6.776496in}{7.554798in}}%
\pgfpathlineto{\pgfqpoint{6.829540in}{7.642472in}}%
\pgfpathlineto{\pgfqpoint{6.882584in}{6.881359in}}%
\pgfpathlineto{\pgfqpoint{6.935628in}{7.642472in}}%
\pgfpathlineto{\pgfqpoint{6.988671in}{7.441180in}}%
\pgfpathlineto{\pgfqpoint{7.041715in}{7.642472in}}%
\pgfpathlineto{\pgfqpoint{7.147803in}{7.642472in}}%
\pgfpathlineto{\pgfqpoint{7.200847in}{6.818171in}}%
\pgfpathlineto{\pgfqpoint{7.253891in}{7.350789in}}%
\pgfpathlineto{\pgfqpoint{7.306935in}{7.542837in}}%
\pgfpathlineto{\pgfqpoint{7.359979in}{6.672189in}}%
\pgfpathlineto{\pgfqpoint{7.413023in}{6.684093in}}%
\pgfpathlineto{\pgfqpoint{7.466067in}{7.061991in}}%
\pgfpathlineto{\pgfqpoint{7.519111in}{7.086960in}}%
\pgfpathlineto{\pgfqpoint{7.572155in}{6.593688in}}%
\pgfpathlineto{\pgfqpoint{7.625199in}{6.652740in}}%
\pgfpathlineto{\pgfqpoint{7.678243in}{6.739129in}}%
\pgfpathlineto{\pgfqpoint{7.731287in}{7.642472in}}%
\pgfpathlineto{\pgfqpoint{7.996506in}{7.642472in}}%
\pgfpathlineto{\pgfqpoint{8.049550in}{6.853606in}}%
\pgfpathlineto{\pgfqpoint{8.102594in}{6.863050in}}%
\pgfpathlineto{\pgfqpoint{8.155638in}{7.642472in}}%
\pgfpathlineto{\pgfqpoint{8.208682in}{7.642472in}}%
\pgfpathlineto{\pgfqpoint{8.261726in}{6.924712in}}%
\pgfpathlineto{\pgfqpoint{8.314770in}{6.876087in}}%
\pgfpathlineto{\pgfqpoint{8.367814in}{7.642472in}}%
\pgfpathlineto{\pgfqpoint{8.420858in}{6.857476in}}%
\pgfpathlineto{\pgfqpoint{8.473902in}{7.642472in}}%
\pgfpathlineto{\pgfqpoint{8.526946in}{7.642472in}}%
\pgfpathlineto{\pgfqpoint{8.579990in}{6.685501in}}%
\pgfpathlineto{\pgfqpoint{8.633033in}{6.694774in}}%
\pgfpathlineto{\pgfqpoint{8.686077in}{6.723687in}}%
\pgfpathlineto{\pgfqpoint{8.739121in}{6.662088in}}%
\pgfpathlineto{\pgfqpoint{8.792165in}{6.800188in}}%
\pgfpathlineto{\pgfqpoint{8.845209in}{6.642906in}}%
\pgfpathlineto{\pgfqpoint{8.898253in}{6.685692in}}%
\pgfpathlineto{\pgfqpoint{8.951297in}{7.642472in}}%
\pgfpathlineto{\pgfqpoint{9.004341in}{7.212818in}}%
\pgfpathlineto{\pgfqpoint{9.057385in}{7.642472in}}%
\pgfpathlineto{\pgfqpoint{9.110429in}{7.642472in}}%
\pgfpathlineto{\pgfqpoint{9.163473in}{6.770647in}}%
\pgfpathlineto{\pgfqpoint{9.216517in}{6.804492in}}%
\pgfpathlineto{\pgfqpoint{9.269561in}{6.811485in}}%
\pgfpathlineto{\pgfqpoint{9.322605in}{6.883905in}}%
\pgfpathlineto{\pgfqpoint{9.375649in}{6.879220in}}%
\pgfpathlineto{\pgfqpoint{9.428692in}{7.642472in}}%
\pgfpathlineto{\pgfqpoint{9.481736in}{7.580172in}}%
\pgfpathlineto{\pgfqpoint{9.534780in}{6.946800in}}%
\pgfpathlineto{\pgfqpoint{9.587824in}{6.868462in}}%
\pgfpathlineto{\pgfqpoint{9.640868in}{7.137950in}}%
\pgfpathlineto{\pgfqpoint{9.693912in}{6.803680in}}%
\pgfpathlineto{\pgfqpoint{9.746956in}{6.779712in}}%
\pgfpathlineto{\pgfqpoint{9.800000in}{6.765861in}}%
\pgfpathlineto{\pgfqpoint{9.800000in}{6.765861in}}%
\pgfusepath{stroke}%
\end{pgfscope}%
\begin{pgfscope}%
\pgfpathrectangle{\pgfqpoint{0.941663in}{4.334375in}}{\pgfqpoint{8.858337in}{3.465625in}}%
\pgfusepath{clip}%
\pgfsetbuttcap%
\pgfsetroundjoin%
\definecolor{currentfill}{rgb}{0.090196,0.745098,0.811765}%
\pgfsetfillcolor{currentfill}%
\pgfsetlinewidth{1.003750pt}%
\definecolor{currentstroke}{rgb}{0.090196,0.745098,0.811765}%
\pgfsetstrokecolor{currentstroke}%
\pgfsetdash{}{0pt}%
\pgfsys@defobject{currentmarker}{\pgfqpoint{0.941663in}{6.251128in}}{\pgfqpoint{9.800000in}{7.642472in}}{%
\pgfpathmoveto{\pgfqpoint{0.941663in}{7.642472in}}%
\pgfpathlineto{\pgfqpoint{0.941663in}{6.251128in}}%
\pgfpathlineto{\pgfqpoint{0.994707in}{6.261680in}}%
\pgfpathlineto{\pgfqpoint{1.047751in}{6.251128in}}%
\pgfpathlineto{\pgfqpoint{1.100795in}{6.251128in}}%
\pgfpathlineto{\pgfqpoint{1.153839in}{6.251128in}}%
\pgfpathlineto{\pgfqpoint{1.206883in}{6.598964in}}%
\pgfpathlineto{\pgfqpoint{1.259927in}{6.598964in}}%
\pgfpathlineto{\pgfqpoint{1.312970in}{6.598964in}}%
\pgfpathlineto{\pgfqpoint{1.366014in}{6.251128in}}%
\pgfpathlineto{\pgfqpoint{1.419058in}{6.251128in}}%
\pgfpathlineto{\pgfqpoint{1.472102in}{6.251128in}}%
\pgfpathlineto{\pgfqpoint{1.525146in}{6.251128in}}%
\pgfpathlineto{\pgfqpoint{1.578190in}{6.251128in}}%
\pgfpathlineto{\pgfqpoint{1.631234in}{6.598964in}}%
\pgfpathlineto{\pgfqpoint{1.684278in}{6.598964in}}%
\pgfpathlineto{\pgfqpoint{1.737322in}{6.251128in}}%
\pgfpathlineto{\pgfqpoint{1.790366in}{6.251128in}}%
\pgfpathlineto{\pgfqpoint{1.843410in}{6.734592in}}%
\pgfpathlineto{\pgfqpoint{1.896454in}{6.883587in}}%
\pgfpathlineto{\pgfqpoint{1.949498in}{6.251128in}}%
\pgfpathlineto{\pgfqpoint{2.002542in}{6.724358in}}%
\pgfpathlineto{\pgfqpoint{2.055586in}{6.251128in}}%
\pgfpathlineto{\pgfqpoint{2.108629in}{6.348747in}}%
\pgfpathlineto{\pgfqpoint{2.161673in}{6.251128in}}%
\pgfpathlineto{\pgfqpoint{2.214717in}{6.574971in}}%
\pgfpathlineto{\pgfqpoint{2.267761in}{6.251128in}}%
\pgfpathlineto{\pgfqpoint{2.320805in}{6.251128in}}%
\pgfpathlineto{\pgfqpoint{2.373849in}{6.541500in}}%
\pgfpathlineto{\pgfqpoint{2.426893in}{6.606651in}}%
\pgfpathlineto{\pgfqpoint{2.479937in}{6.251128in}}%
\pgfpathlineto{\pgfqpoint{2.532981in}{6.251128in}}%
\pgfpathlineto{\pgfqpoint{2.586025in}{6.251128in}}%
\pgfpathlineto{\pgfqpoint{2.639069in}{6.251128in}}%
\pgfpathlineto{\pgfqpoint{2.692113in}{6.251128in}}%
\pgfpathlineto{\pgfqpoint{2.745157in}{6.251128in}}%
\pgfpathlineto{\pgfqpoint{2.798201in}{6.251128in}}%
\pgfpathlineto{\pgfqpoint{2.851245in}{6.598964in}}%
\pgfpathlineto{\pgfqpoint{2.904288in}{6.251128in}}%
\pgfpathlineto{\pgfqpoint{2.957332in}{6.744748in}}%
\pgfpathlineto{\pgfqpoint{3.010376in}{6.251128in}}%
\pgfpathlineto{\pgfqpoint{3.063420in}{6.598964in}}%
\pgfpathlineto{\pgfqpoint{3.116464in}{6.598964in}}%
\pgfpathlineto{\pgfqpoint{3.169508in}{6.251128in}}%
\pgfpathlineto{\pgfqpoint{3.222552in}{6.251128in}}%
\pgfpathlineto{\pgfqpoint{3.275596in}{6.251128in}}%
\pgfpathlineto{\pgfqpoint{3.328640in}{6.251128in}}%
\pgfpathlineto{\pgfqpoint{3.381684in}{6.598964in}}%
\pgfpathlineto{\pgfqpoint{3.434728in}{6.251128in}}%
\pgfpathlineto{\pgfqpoint{3.487772in}{6.598964in}}%
\pgfpathlineto{\pgfqpoint{3.540816in}{6.572638in}}%
\pgfpathlineto{\pgfqpoint{3.593860in}{6.251128in}}%
\pgfpathlineto{\pgfqpoint{3.646904in}{6.251128in}}%
\pgfpathlineto{\pgfqpoint{3.699948in}{6.251128in}}%
\pgfpathlineto{\pgfqpoint{3.752991in}{6.510498in}}%
\pgfpathlineto{\pgfqpoint{3.806035in}{6.251128in}}%
\pgfpathlineto{\pgfqpoint{3.859079in}{6.572998in}}%
\pgfpathlineto{\pgfqpoint{3.912123in}{6.481634in}}%
\pgfpathlineto{\pgfqpoint{3.965167in}{6.340259in}}%
\pgfpathlineto{\pgfqpoint{4.018211in}{6.251128in}}%
\pgfpathlineto{\pgfqpoint{4.071255in}{6.251128in}}%
\pgfpathlineto{\pgfqpoint{4.124299in}{6.251128in}}%
\pgfpathlineto{\pgfqpoint{4.177343in}{6.251128in}}%
\pgfpathlineto{\pgfqpoint{4.230387in}{6.251128in}}%
\pgfpathlineto{\pgfqpoint{4.283431in}{6.251128in}}%
\pgfpathlineto{\pgfqpoint{4.336475in}{6.889082in}}%
\pgfpathlineto{\pgfqpoint{4.389519in}{6.299258in}}%
\pgfpathlineto{\pgfqpoint{4.442563in}{6.598964in}}%
\pgfpathlineto{\pgfqpoint{4.495607in}{6.633012in}}%
\pgfpathlineto{\pgfqpoint{4.548650in}{6.251128in}}%
\pgfpathlineto{\pgfqpoint{4.601694in}{6.592100in}}%
\pgfpathlineto{\pgfqpoint{4.654738in}{6.251128in}}%
\pgfpathlineto{\pgfqpoint{4.707782in}{6.766257in}}%
\pgfpathlineto{\pgfqpoint{4.760826in}{6.251128in}}%
\pgfpathlineto{\pgfqpoint{4.813870in}{6.251128in}}%
\pgfpathlineto{\pgfqpoint{4.866914in}{6.548881in}}%
\pgfpathlineto{\pgfqpoint{4.919958in}{6.528186in}}%
\pgfpathlineto{\pgfqpoint{4.973002in}{6.251128in}}%
\pgfpathlineto{\pgfqpoint{5.026046in}{6.590550in}}%
\pgfpathlineto{\pgfqpoint{5.079090in}{6.251128in}}%
\pgfpathlineto{\pgfqpoint{5.132134in}{6.598964in}}%
\pgfpathlineto{\pgfqpoint{5.185178in}{6.574350in}}%
\pgfpathlineto{\pgfqpoint{5.238222in}{6.501650in}}%
\pgfpathlineto{\pgfqpoint{5.291266in}{6.251128in}}%
\pgfpathlineto{\pgfqpoint{5.344309in}{6.471024in}}%
\pgfpathlineto{\pgfqpoint{5.397353in}{6.325005in}}%
\pgfpathlineto{\pgfqpoint{5.450397in}{6.770026in}}%
\pgfpathlineto{\pgfqpoint{5.503441in}{6.251128in}}%
\pgfpathlineto{\pgfqpoint{5.556485in}{6.329843in}}%
\pgfpathlineto{\pgfqpoint{5.609529in}{6.514446in}}%
\pgfpathlineto{\pgfqpoint{5.662573in}{6.375963in}}%
\pgfpathlineto{\pgfqpoint{5.715617in}{6.251128in}}%
\pgfpathlineto{\pgfqpoint{5.768661in}{6.913209in}}%
\pgfpathlineto{\pgfqpoint{5.821705in}{6.872997in}}%
\pgfpathlineto{\pgfqpoint{5.874749in}{6.702075in}}%
\pgfpathlineto{\pgfqpoint{5.927793in}{6.598351in}}%
\pgfpathlineto{\pgfqpoint{5.980837in}{6.251128in}}%
\pgfpathlineto{\pgfqpoint{6.033881in}{6.400407in}}%
\pgfpathlineto{\pgfqpoint{6.086925in}{6.251128in}}%
\pgfpathlineto{\pgfqpoint{6.139969in}{6.251128in}}%
\pgfpathlineto{\pgfqpoint{6.193012in}{6.635398in}}%
\pgfpathlineto{\pgfqpoint{6.246056in}{6.691111in}}%
\pgfpathlineto{\pgfqpoint{6.299100in}{6.251128in}}%
\pgfpathlineto{\pgfqpoint{6.352144in}{6.569516in}}%
\pgfpathlineto{\pgfqpoint{6.405188in}{6.364054in}}%
\pgfpathlineto{\pgfqpoint{6.458232in}{6.251128in}}%
\pgfpathlineto{\pgfqpoint{6.511276in}{6.251128in}}%
\pgfpathlineto{\pgfqpoint{6.564320in}{6.251128in}}%
\pgfpathlineto{\pgfqpoint{6.617364in}{6.251128in}}%
\pgfpathlineto{\pgfqpoint{6.670408in}{6.251128in}}%
\pgfpathlineto{\pgfqpoint{6.723452in}{6.251128in}}%
\pgfpathlineto{\pgfqpoint{6.776496in}{6.385890in}}%
\pgfpathlineto{\pgfqpoint{6.829540in}{6.251128in}}%
\pgfpathlineto{\pgfqpoint{6.882584in}{6.780238in}}%
\pgfpathlineto{\pgfqpoint{6.935628in}{6.251128in}}%
\pgfpathlineto{\pgfqpoint{6.988671in}{6.598964in}}%
\pgfpathlineto{\pgfqpoint{7.041715in}{6.251128in}}%
\pgfpathlineto{\pgfqpoint{7.094759in}{6.251128in}}%
\pgfpathlineto{\pgfqpoint{7.147803in}{6.251128in}}%
\pgfpathlineto{\pgfqpoint{7.200847in}{6.729225in}}%
\pgfpathlineto{\pgfqpoint{7.253891in}{6.451777in}}%
\pgfpathlineto{\pgfqpoint{7.306935in}{6.251128in}}%
\pgfpathlineto{\pgfqpoint{7.359979in}{6.470643in}}%
\pgfpathlineto{\pgfqpoint{7.413023in}{6.529690in}}%
\pgfpathlineto{\pgfqpoint{7.466067in}{6.251128in}}%
\pgfpathlineto{\pgfqpoint{7.519111in}{6.251128in}}%
\pgfpathlineto{\pgfqpoint{7.572155in}{6.487962in}}%
\pgfpathlineto{\pgfqpoint{7.625199in}{6.652740in}}%
\pgfpathlineto{\pgfqpoint{7.678243in}{6.251128in}}%
\pgfpathlineto{\pgfqpoint{7.731287in}{6.251128in}}%
\pgfpathlineto{\pgfqpoint{7.784330in}{6.251128in}}%
\pgfpathlineto{\pgfqpoint{7.837374in}{6.251128in}}%
\pgfpathlineto{\pgfqpoint{7.890418in}{6.251128in}}%
\pgfpathlineto{\pgfqpoint{7.943462in}{6.251128in}}%
\pgfpathlineto{\pgfqpoint{7.996506in}{6.251128in}}%
\pgfpathlineto{\pgfqpoint{8.049550in}{6.766484in}}%
\pgfpathlineto{\pgfqpoint{8.102594in}{6.782745in}}%
\pgfpathlineto{\pgfqpoint{8.155638in}{6.251128in}}%
\pgfpathlineto{\pgfqpoint{8.208682in}{6.251128in}}%
\pgfpathlineto{\pgfqpoint{8.261726in}{6.924712in}}%
\pgfpathlineto{\pgfqpoint{8.314770in}{6.876087in}}%
\pgfpathlineto{\pgfqpoint{8.367814in}{6.251128in}}%
\pgfpathlineto{\pgfqpoint{8.420858in}{6.653636in}}%
\pgfpathlineto{\pgfqpoint{8.473902in}{6.251128in}}%
\pgfpathlineto{\pgfqpoint{8.526946in}{6.251128in}}%
\pgfpathlineto{\pgfqpoint{8.579990in}{6.685501in}}%
\pgfpathlineto{\pgfqpoint{8.633033in}{6.520595in}}%
\pgfpathlineto{\pgfqpoint{8.686077in}{6.723687in}}%
\pgfpathlineto{\pgfqpoint{8.739121in}{6.489640in}}%
\pgfpathlineto{\pgfqpoint{8.792165in}{6.251128in}}%
\pgfpathlineto{\pgfqpoint{8.845209in}{6.299108in}}%
\pgfpathlineto{\pgfqpoint{8.898253in}{6.354195in}}%
\pgfpathlineto{\pgfqpoint{8.951297in}{6.251128in}}%
\pgfpathlineto{\pgfqpoint{9.004341in}{6.568668in}}%
\pgfpathlineto{\pgfqpoint{9.057385in}{6.251128in}}%
\pgfpathlineto{\pgfqpoint{9.110429in}{6.251128in}}%
\pgfpathlineto{\pgfqpoint{9.163473in}{6.668425in}}%
\pgfpathlineto{\pgfqpoint{9.216517in}{6.804492in}}%
\pgfpathlineto{\pgfqpoint{9.269561in}{6.628250in}}%
\pgfpathlineto{\pgfqpoint{9.322605in}{6.727450in}}%
\pgfpathlineto{\pgfqpoint{9.375649in}{6.754769in}}%
\pgfpathlineto{\pgfqpoint{9.428692in}{6.251128in}}%
\pgfpathlineto{\pgfqpoint{9.481736in}{6.256205in}}%
\pgfpathlineto{\pgfqpoint{9.534780in}{6.946800in}}%
\pgfpathlineto{\pgfqpoint{9.587824in}{6.621011in}}%
\pgfpathlineto{\pgfqpoint{9.640868in}{6.251128in}}%
\pgfpathlineto{\pgfqpoint{9.693912in}{6.632876in}}%
\pgfpathlineto{\pgfqpoint{9.746956in}{6.779712in}}%
\pgfpathlineto{\pgfqpoint{9.800000in}{6.573653in}}%
\pgfpathlineto{\pgfqpoint{9.800000in}{6.765861in}}%
\pgfpathlineto{\pgfqpoint{9.800000in}{6.765861in}}%
\pgfpathlineto{\pgfqpoint{9.746956in}{6.779712in}}%
\pgfpathlineto{\pgfqpoint{9.693912in}{6.803680in}}%
\pgfpathlineto{\pgfqpoint{9.640868in}{7.137950in}}%
\pgfpathlineto{\pgfqpoint{9.587824in}{6.868462in}}%
\pgfpathlineto{\pgfqpoint{9.534780in}{6.946800in}}%
\pgfpathlineto{\pgfqpoint{9.481736in}{7.580172in}}%
\pgfpathlineto{\pgfqpoint{9.428692in}{7.642472in}}%
\pgfpathlineto{\pgfqpoint{9.375649in}{6.879220in}}%
\pgfpathlineto{\pgfqpoint{9.322605in}{6.883905in}}%
\pgfpathlineto{\pgfqpoint{9.269561in}{6.811485in}}%
\pgfpathlineto{\pgfqpoint{9.216517in}{6.804492in}}%
\pgfpathlineto{\pgfqpoint{9.163473in}{6.770647in}}%
\pgfpathlineto{\pgfqpoint{9.110429in}{7.642472in}}%
\pgfpathlineto{\pgfqpoint{9.057385in}{7.642472in}}%
\pgfpathlineto{\pgfqpoint{9.004341in}{7.212818in}}%
\pgfpathlineto{\pgfqpoint{8.951297in}{7.642472in}}%
\pgfpathlineto{\pgfqpoint{8.898253in}{6.685692in}}%
\pgfpathlineto{\pgfqpoint{8.845209in}{6.642906in}}%
\pgfpathlineto{\pgfqpoint{8.792165in}{6.800188in}}%
\pgfpathlineto{\pgfqpoint{8.739121in}{6.662088in}}%
\pgfpathlineto{\pgfqpoint{8.686077in}{6.723687in}}%
\pgfpathlineto{\pgfqpoint{8.633033in}{6.694774in}}%
\pgfpathlineto{\pgfqpoint{8.579990in}{6.685501in}}%
\pgfpathlineto{\pgfqpoint{8.526946in}{7.642472in}}%
\pgfpathlineto{\pgfqpoint{8.473902in}{7.642472in}}%
\pgfpathlineto{\pgfqpoint{8.420858in}{6.857476in}}%
\pgfpathlineto{\pgfqpoint{8.367814in}{7.642472in}}%
\pgfpathlineto{\pgfqpoint{8.314770in}{6.876087in}}%
\pgfpathlineto{\pgfqpoint{8.261726in}{6.924712in}}%
\pgfpathlineto{\pgfqpoint{8.208682in}{7.642472in}}%
\pgfpathlineto{\pgfqpoint{8.155638in}{7.642472in}}%
\pgfpathlineto{\pgfqpoint{8.102594in}{6.863050in}}%
\pgfpathlineto{\pgfqpoint{8.049550in}{6.853606in}}%
\pgfpathlineto{\pgfqpoint{7.996506in}{7.642472in}}%
\pgfpathlineto{\pgfqpoint{7.943462in}{7.642472in}}%
\pgfpathlineto{\pgfqpoint{7.890418in}{7.642472in}}%
\pgfpathlineto{\pgfqpoint{7.837374in}{7.642472in}}%
\pgfpathlineto{\pgfqpoint{7.784330in}{7.642472in}}%
\pgfpathlineto{\pgfqpoint{7.731287in}{7.642472in}}%
\pgfpathlineto{\pgfqpoint{7.678243in}{6.739129in}}%
\pgfpathlineto{\pgfqpoint{7.625199in}{6.652740in}}%
\pgfpathlineto{\pgfqpoint{7.572155in}{6.593688in}}%
\pgfpathlineto{\pgfqpoint{7.519111in}{7.086960in}}%
\pgfpathlineto{\pgfqpoint{7.466067in}{7.061991in}}%
\pgfpathlineto{\pgfqpoint{7.413023in}{6.684093in}}%
\pgfpathlineto{\pgfqpoint{7.359979in}{6.672189in}}%
\pgfpathlineto{\pgfqpoint{7.306935in}{7.542837in}}%
\pgfpathlineto{\pgfqpoint{7.253891in}{7.350789in}}%
\pgfpathlineto{\pgfqpoint{7.200847in}{6.818171in}}%
\pgfpathlineto{\pgfqpoint{7.147803in}{7.642472in}}%
\pgfpathlineto{\pgfqpoint{7.094759in}{7.642472in}}%
\pgfpathlineto{\pgfqpoint{7.041715in}{7.642472in}}%
\pgfpathlineto{\pgfqpoint{6.988671in}{7.441180in}}%
\pgfpathlineto{\pgfqpoint{6.935628in}{7.642472in}}%
\pgfpathlineto{\pgfqpoint{6.882584in}{6.881359in}}%
\pgfpathlineto{\pgfqpoint{6.829540in}{7.642472in}}%
\pgfpathlineto{\pgfqpoint{6.776496in}{7.554798in}}%
\pgfpathlineto{\pgfqpoint{6.723452in}{7.642472in}}%
\pgfpathlineto{\pgfqpoint{6.670408in}{7.642472in}}%
\pgfpathlineto{\pgfqpoint{6.617364in}{7.642472in}}%
\pgfpathlineto{\pgfqpoint{6.564320in}{7.642472in}}%
\pgfpathlineto{\pgfqpoint{6.511276in}{7.642472in}}%
\pgfpathlineto{\pgfqpoint{6.458232in}{7.158194in}}%
\pgfpathlineto{\pgfqpoint{6.405188in}{6.593987in}}%
\pgfpathlineto{\pgfqpoint{6.352144in}{7.023749in}}%
\pgfpathlineto{\pgfqpoint{6.299100in}{7.642472in}}%
\pgfpathlineto{\pgfqpoint{6.246056in}{6.691111in}}%
\pgfpathlineto{\pgfqpoint{6.193012in}{6.635398in}}%
\pgfpathlineto{\pgfqpoint{6.139969in}{7.642472in}}%
\pgfpathlineto{\pgfqpoint{6.086925in}{7.642472in}}%
\pgfpathlineto{\pgfqpoint{6.033881in}{7.110408in}}%
\pgfpathlineto{\pgfqpoint{5.980837in}{7.642472in}}%
\pgfpathlineto{\pgfqpoint{5.927793in}{6.820928in}}%
\pgfpathlineto{\pgfqpoint{5.874749in}{6.796671in}}%
\pgfpathlineto{\pgfqpoint{5.821705in}{6.872997in}}%
\pgfpathlineto{\pgfqpoint{5.768661in}{6.913209in}}%
\pgfpathlineto{\pgfqpoint{5.715617in}{7.642472in}}%
\pgfpathlineto{\pgfqpoint{5.662573in}{6.925790in}}%
\pgfpathlineto{\pgfqpoint{5.609529in}{6.874196in}}%
\pgfpathlineto{\pgfqpoint{5.556485in}{7.281307in}}%
\pgfpathlineto{\pgfqpoint{5.503441in}{7.642472in}}%
\pgfpathlineto{\pgfqpoint{5.450397in}{6.877918in}}%
\pgfpathlineto{\pgfqpoint{5.397353in}{6.748316in}}%
\pgfpathlineto{\pgfqpoint{5.344309in}{6.838988in}}%
\pgfpathlineto{\pgfqpoint{5.291266in}{7.642472in}}%
\pgfpathlineto{\pgfqpoint{5.238222in}{6.666672in}}%
\pgfpathlineto{\pgfqpoint{5.185178in}{6.670254in}}%
\pgfpathlineto{\pgfqpoint{5.132134in}{6.692849in}}%
\pgfpathlineto{\pgfqpoint{5.079090in}{7.642472in}}%
\pgfpathlineto{\pgfqpoint{5.026046in}{6.790800in}}%
\pgfpathlineto{\pgfqpoint{4.973002in}{7.642472in}}%
\pgfpathlineto{\pgfqpoint{4.919958in}{6.671298in}}%
\pgfpathlineto{\pgfqpoint{4.866914in}{6.684929in}}%
\pgfpathlineto{\pgfqpoint{4.813870in}{7.642472in}}%
\pgfpathlineto{\pgfqpoint{4.760826in}{7.378709in}}%
\pgfpathlineto{\pgfqpoint{4.707782in}{6.766257in}}%
\pgfpathlineto{\pgfqpoint{4.654738in}{7.642472in}}%
\pgfpathlineto{\pgfqpoint{4.601694in}{6.827112in}}%
\pgfpathlineto{\pgfqpoint{4.548650in}{7.353421in}}%
\pgfpathlineto{\pgfqpoint{4.495607in}{6.883010in}}%
\pgfpathlineto{\pgfqpoint{4.442563in}{7.506419in}}%
\pgfpathlineto{\pgfqpoint{4.389519in}{7.554209in}}%
\pgfpathlineto{\pgfqpoint{4.336475in}{6.889082in}}%
\pgfpathlineto{\pgfqpoint{4.283431in}{7.642472in}}%
\pgfpathlineto{\pgfqpoint{4.230387in}{7.642472in}}%
\pgfpathlineto{\pgfqpoint{4.177343in}{7.642472in}}%
\pgfpathlineto{\pgfqpoint{4.124299in}{7.539721in}}%
\pgfpathlineto{\pgfqpoint{4.071255in}{7.642472in}}%
\pgfpathlineto{\pgfqpoint{4.018211in}{7.642472in}}%
\pgfpathlineto{\pgfqpoint{3.965167in}{6.991885in}}%
\pgfpathlineto{\pgfqpoint{3.912123in}{7.302849in}}%
\pgfpathlineto{\pgfqpoint{3.859079in}{7.341795in}}%
\pgfpathlineto{\pgfqpoint{3.806035in}{7.642472in}}%
\pgfpathlineto{\pgfqpoint{3.752991in}{7.311866in}}%
\pgfpathlineto{\pgfqpoint{3.699948in}{7.642472in}}%
\pgfpathlineto{\pgfqpoint{3.646904in}{7.642472in}}%
\pgfpathlineto{\pgfqpoint{3.593860in}{7.167623in}}%
\pgfpathlineto{\pgfqpoint{3.540816in}{6.738883in}}%
\pgfpathlineto{\pgfqpoint{3.487772in}{6.972573in}}%
\pgfpathlineto{\pgfqpoint{3.434728in}{7.642472in}}%
\pgfpathlineto{\pgfqpoint{3.381684in}{7.361194in}}%
\pgfpathlineto{\pgfqpoint{3.328640in}{7.642472in}}%
\pgfpathlineto{\pgfqpoint{3.275596in}{7.642472in}}%
\pgfpathlineto{\pgfqpoint{3.222552in}{7.642472in}}%
\pgfpathlineto{\pgfqpoint{3.169508in}{7.642472in}}%
\pgfpathlineto{\pgfqpoint{3.116464in}{7.518262in}}%
\pgfpathlineto{\pgfqpoint{3.063420in}{6.841623in}}%
\pgfpathlineto{\pgfqpoint{3.010376in}{7.642472in}}%
\pgfpathlineto{\pgfqpoint{2.957332in}{6.836531in}}%
\pgfpathlineto{\pgfqpoint{2.904288in}{7.642472in}}%
\pgfpathlineto{\pgfqpoint{2.851245in}{7.173688in}}%
\pgfpathlineto{\pgfqpoint{2.798201in}{7.642472in}}%
\pgfpathlineto{\pgfqpoint{2.745157in}{7.642472in}}%
\pgfpathlineto{\pgfqpoint{2.692113in}{7.642472in}}%
\pgfpathlineto{\pgfqpoint{2.639069in}{7.642472in}}%
\pgfpathlineto{\pgfqpoint{2.586025in}{7.642472in}}%
\pgfpathlineto{\pgfqpoint{2.532981in}{7.642472in}}%
\pgfpathlineto{\pgfqpoint{2.479937in}{7.449769in}}%
\pgfpathlineto{\pgfqpoint{2.426893in}{6.606651in}}%
\pgfpathlineto{\pgfqpoint{2.373849in}{6.676762in}}%
\pgfpathlineto{\pgfqpoint{2.320805in}{7.642472in}}%
\pgfpathlineto{\pgfqpoint{2.267761in}{7.419218in}}%
\pgfpathlineto{\pgfqpoint{2.214717in}{6.753348in}}%
\pgfpathlineto{\pgfqpoint{2.161673in}{7.642472in}}%
\pgfpathlineto{\pgfqpoint{2.108629in}{7.339659in}}%
\pgfpathlineto{\pgfqpoint{2.055586in}{7.642472in}}%
\pgfpathlineto{\pgfqpoint{2.002542in}{6.857739in}}%
\pgfpathlineto{\pgfqpoint{1.949498in}{7.642472in}}%
\pgfpathlineto{\pgfqpoint{1.896454in}{6.883587in}}%
\pgfpathlineto{\pgfqpoint{1.843410in}{6.911654in}}%
\pgfpathlineto{\pgfqpoint{1.790366in}{7.642472in}}%
\pgfpathlineto{\pgfqpoint{1.737322in}{7.536394in}}%
\pgfpathlineto{\pgfqpoint{1.684278in}{7.452967in}}%
\pgfpathlineto{\pgfqpoint{1.631234in}{6.920473in}}%
\pgfpathlineto{\pgfqpoint{1.578190in}{7.642472in}}%
\pgfpathlineto{\pgfqpoint{1.525146in}{7.642472in}}%
\pgfpathlineto{\pgfqpoint{1.472102in}{7.642472in}}%
\pgfpathlineto{\pgfqpoint{1.419058in}{7.642472in}}%
\pgfpathlineto{\pgfqpoint{1.366014in}{7.642472in}}%
\pgfpathlineto{\pgfqpoint{1.312970in}{7.018171in}}%
\pgfpathlineto{\pgfqpoint{1.259927in}{6.752271in}}%
\pgfpathlineto{\pgfqpoint{1.206883in}{7.113972in}}%
\pgfpathlineto{\pgfqpoint{1.153839in}{7.642472in}}%
\pgfpathlineto{\pgfqpoint{1.100795in}{7.630782in}}%
\pgfpathlineto{\pgfqpoint{1.047751in}{7.642472in}}%
\pgfpathlineto{\pgfqpoint{0.994707in}{7.398889in}}%
\pgfpathlineto{\pgfqpoint{0.941663in}{7.642472in}}%
\pgfpathlineto{\pgfqpoint{0.941663in}{7.642472in}}%
\pgfpathclose%
\pgfusepath{stroke,fill}%
}%
\begin{pgfscope}%
\pgfsys@transformshift{0.000000in}{0.000000in}%
\pgfsys@useobject{currentmarker}{}%
\end{pgfscope}%
\end{pgfscope}%
\begin{pgfscope}%
\pgfsetrectcap%
\pgfsetmiterjoin%
\pgfsetlinewidth{0.803000pt}%
\definecolor{currentstroke}{rgb}{0.000000,0.000000,0.000000}%
\pgfsetstrokecolor{currentstroke}%
\pgfsetdash{}{0pt}%
\pgfpathmoveto{\pgfqpoint{0.941663in}{4.334375in}}%
\pgfpathlineto{\pgfqpoint{0.941663in}{7.800000in}}%
\pgfusepath{stroke}%
\end{pgfscope}%
\begin{pgfscope}%
\pgfsetrectcap%
\pgfsetmiterjoin%
\pgfsetlinewidth{0.803000pt}%
\definecolor{currentstroke}{rgb}{0.000000,0.000000,0.000000}%
\pgfsetstrokecolor{currentstroke}%
\pgfsetdash{}{0pt}%
\pgfpathmoveto{\pgfqpoint{9.800000in}{4.334375in}}%
\pgfpathlineto{\pgfqpoint{9.800000in}{7.800000in}}%
\pgfusepath{stroke}%
\end{pgfscope}%
\begin{pgfscope}%
\pgfsetrectcap%
\pgfsetmiterjoin%
\pgfsetlinewidth{0.803000pt}%
\definecolor{currentstroke}{rgb}{0.000000,0.000000,0.000000}%
\pgfsetstrokecolor{currentstroke}%
\pgfsetdash{}{0pt}%
\pgfpathmoveto{\pgfqpoint{0.941663in}{4.334375in}}%
\pgfpathlineto{\pgfqpoint{9.800000in}{4.334375in}}%
\pgfusepath{stroke}%
\end{pgfscope}%
\begin{pgfscope}%
\pgfsetrectcap%
\pgfsetmiterjoin%
\pgfsetlinewidth{0.803000pt}%
\definecolor{currentstroke}{rgb}{0.000000,0.000000,0.000000}%
\pgfsetstrokecolor{currentstroke}%
\pgfsetdash{}{0pt}%
\pgfpathmoveto{\pgfqpoint{0.941663in}{7.800000in}}%
\pgfpathlineto{\pgfqpoint{9.800000in}{7.800000in}}%
\pgfusepath{stroke}%
\end{pgfscope}%
\begin{pgfscope}%
\pgfpathrectangle{\pgfqpoint{0.941663in}{4.334375in}}{\pgfqpoint{8.858337in}{3.465625in}}%
\pgfusepath{clip}%
\pgfsetbuttcap%
\pgfsetroundjoin%
\pgfsetlinewidth{1.505625pt}%
\definecolor{currentstroke}{rgb}{0.000000,0.000000,0.000000}%
\pgfsetstrokecolor{currentstroke}%
\pgfsetdash{{5.550000pt}{2.400000pt}}{0.000000pt}%
\pgfpathmoveto{\pgfqpoint{0.941663in}{6.693424in}}%
\pgfpathlineto{\pgfqpoint{0.994707in}{6.703217in}}%
\pgfpathlineto{\pgfqpoint{1.047751in}{6.689375in}}%
\pgfpathlineto{\pgfqpoint{1.100795in}{6.646904in}}%
\pgfpathlineto{\pgfqpoint{1.153839in}{6.644931in}}%
\pgfpathlineto{\pgfqpoint{1.206883in}{6.685908in}}%
\pgfpathlineto{\pgfqpoint{1.259927in}{6.566729in}}%
\pgfpathlineto{\pgfqpoint{1.312970in}{6.635332in}}%
\pgfpathlineto{\pgfqpoint{1.366014in}{6.580587in}}%
\pgfpathlineto{\pgfqpoint{1.419058in}{6.654803in}}%
\pgfpathlineto{\pgfqpoint{1.472102in}{6.660166in}}%
\pgfpathlineto{\pgfqpoint{1.525146in}{6.740529in}}%
\pgfpathlineto{\pgfqpoint{1.578190in}{6.717659in}}%
\pgfpathlineto{\pgfqpoint{1.631234in}{6.788071in}}%
\pgfpathlineto{\pgfqpoint{1.684278in}{6.772987in}}%
\pgfpathlineto{\pgfqpoint{1.737322in}{6.776566in}}%
\pgfpathlineto{\pgfqpoint{1.790366in}{6.865064in}}%
\pgfpathlineto{\pgfqpoint{1.843410in}{6.911654in}}%
\pgfpathlineto{\pgfqpoint{1.896454in}{6.883587in}}%
\pgfpathlineto{\pgfqpoint{1.949498in}{6.838375in}}%
\pgfpathlineto{\pgfqpoint{2.002542in}{6.857739in}}%
\pgfpathlineto{\pgfqpoint{2.055586in}{6.849181in}}%
\pgfpathlineto{\pgfqpoint{2.108629in}{6.832484in}}%
\pgfpathlineto{\pgfqpoint{2.161673in}{6.738829in}}%
\pgfpathlineto{\pgfqpoint{2.214717in}{6.753348in}}%
\pgfpathlineto{\pgfqpoint{2.267761in}{6.710554in}}%
\pgfpathlineto{\pgfqpoint{2.320805in}{6.669421in}}%
\pgfpathlineto{\pgfqpoint{2.373849in}{6.644384in}}%
\pgfpathlineto{\pgfqpoint{2.426893in}{6.606651in}}%
\pgfpathlineto{\pgfqpoint{2.479937in}{6.596638in}}%
\pgfpathlineto{\pgfqpoint{2.532981in}{6.645096in}}%
\pgfpathlineto{\pgfqpoint{2.586025in}{6.623285in}}%
\pgfpathlineto{\pgfqpoint{2.639069in}{6.618792in}}%
\pgfpathlineto{\pgfqpoint{2.692113in}{6.588590in}}%
\pgfpathlineto{\pgfqpoint{2.745157in}{6.659513in}}%
\pgfpathlineto{\pgfqpoint{2.798201in}{6.701909in}}%
\pgfpathlineto{\pgfqpoint{2.851245in}{6.716040in}}%
\pgfpathlineto{\pgfqpoint{2.904288in}{6.748816in}}%
\pgfpathlineto{\pgfqpoint{2.957332in}{6.836531in}}%
\pgfpathlineto{\pgfqpoint{3.010376in}{6.834580in}}%
\pgfpathlineto{\pgfqpoint{3.063420in}{6.815911in}}%
\pgfpathlineto{\pgfqpoint{3.116464in}{6.836791in}}%
\pgfpathlineto{\pgfqpoint{3.169508in}{6.866400in}}%
\pgfpathlineto{\pgfqpoint{3.222552in}{6.896923in}}%
\pgfpathlineto{\pgfqpoint{3.275596in}{6.813063in}}%
\pgfpathlineto{\pgfqpoint{3.328640in}{6.871638in}}%
\pgfpathlineto{\pgfqpoint{3.381684in}{6.788610in}}%
\pgfpathlineto{\pgfqpoint{3.434728in}{6.724023in}}%
\pgfpathlineto{\pgfqpoint{3.487772in}{6.748160in}}%
\pgfpathlineto{\pgfqpoint{3.540816in}{6.738883in}}%
\pgfpathlineto{\pgfqpoint{3.593860in}{6.664878in}}%
\pgfpathlineto{\pgfqpoint{3.646904in}{6.662324in}}%
\pgfpathlineto{\pgfqpoint{3.699948in}{6.618474in}}%
\pgfpathlineto{\pgfqpoint{3.752991in}{6.616195in}}%
\pgfpathlineto{\pgfqpoint{3.806035in}{6.658725in}}%
\pgfpathlineto{\pgfqpoint{3.859079in}{6.646123in}}%
\pgfpathlineto{\pgfqpoint{3.912123in}{6.607177in}}%
\pgfpathlineto{\pgfqpoint{3.965167in}{6.735611in}}%
\pgfpathlineto{\pgfqpoint{4.018211in}{6.664699in}}%
\pgfpathlineto{\pgfqpoint{4.071255in}{6.746790in}}%
\pgfpathlineto{\pgfqpoint{4.124299in}{6.772208in}}%
\pgfpathlineto{\pgfqpoint{4.177343in}{6.790679in}}%
\pgfpathlineto{\pgfqpoint{4.230387in}{6.828787in}}%
\pgfpathlineto{\pgfqpoint{4.283431in}{6.844506in}}%
\pgfpathlineto{\pgfqpoint{4.336475in}{6.889082in}}%
\pgfpathlineto{\pgfqpoint{4.389519in}{6.858537in}}%
\pgfpathlineto{\pgfqpoint{4.442563in}{6.847591in}}%
\pgfpathlineto{\pgfqpoint{4.495607in}{6.883010in}}%
\pgfpathlineto{\pgfqpoint{4.548650in}{6.899833in}}%
\pgfpathlineto{\pgfqpoint{4.601694in}{6.827112in}}%
\pgfpathlineto{\pgfqpoint{4.654738in}{6.846117in}}%
\pgfpathlineto{\pgfqpoint{4.707782in}{6.766257in}}%
\pgfpathlineto{\pgfqpoint{4.760826in}{6.775170in}}%
\pgfpathlineto{\pgfqpoint{4.813870in}{6.726380in}}%
\pgfpathlineto{\pgfqpoint{4.866914in}{6.684929in}}%
\pgfpathlineto{\pgfqpoint{4.919958in}{6.671298in}}%
\pgfpathlineto{\pgfqpoint{4.973002in}{6.607378in}}%
\pgfpathlineto{\pgfqpoint{5.026046in}{6.622809in}}%
\pgfpathlineto{\pgfqpoint{5.079090in}{6.578918in}}%
\pgfpathlineto{\pgfqpoint{5.132134in}{6.582269in}}%
\pgfpathlineto{\pgfqpoint{5.185178in}{6.670254in}}%
\pgfpathlineto{\pgfqpoint{5.238222in}{6.666672in}}%
\pgfpathlineto{\pgfqpoint{5.291266in}{6.726904in}}%
\pgfpathlineto{\pgfqpoint{5.344309in}{6.737403in}}%
\pgfpathlineto{\pgfqpoint{5.397353in}{6.748316in}}%
\pgfpathlineto{\pgfqpoint{5.450397in}{6.877918in}}%
\pgfpathlineto{\pgfqpoint{5.503441in}{6.868085in}}%
\pgfpathlineto{\pgfqpoint{5.556485in}{6.860891in}}%
\pgfpathlineto{\pgfqpoint{5.609529in}{6.874196in}}%
\pgfpathlineto{\pgfqpoint{5.662573in}{6.925790in}}%
\pgfpathlineto{\pgfqpoint{5.715617in}{6.873877in}}%
\pgfpathlineto{\pgfqpoint{5.768661in}{6.913209in}}%
\pgfpathlineto{\pgfqpoint{5.821705in}{6.872997in}}%
\pgfpathlineto{\pgfqpoint{5.874749in}{6.796671in}}%
\pgfpathlineto{\pgfqpoint{5.927793in}{6.820928in}}%
\pgfpathlineto{\pgfqpoint{5.980837in}{6.797521in}}%
\pgfpathlineto{\pgfqpoint{6.033881in}{6.748313in}}%
\pgfpathlineto{\pgfqpoint{6.086925in}{6.726383in}}%
\pgfpathlineto{\pgfqpoint{6.139969in}{6.707909in}}%
\pgfpathlineto{\pgfqpoint{6.193012in}{6.635398in}}%
\pgfpathlineto{\pgfqpoint{6.246056in}{6.691111in}}%
\pgfpathlineto{\pgfqpoint{6.299100in}{6.628412in}}%
\pgfpathlineto{\pgfqpoint{6.352144in}{6.647199in}}%
\pgfpathlineto{\pgfqpoint{6.405188in}{6.593987in}}%
\pgfpathlineto{\pgfqpoint{6.458232in}{6.680253in}}%
\pgfpathlineto{\pgfqpoint{6.511276in}{6.698006in}}%
\pgfpathlineto{\pgfqpoint{6.564320in}{6.694356in}}%
\pgfpathlineto{\pgfqpoint{6.617364in}{6.761313in}}%
\pgfpathlineto{\pgfqpoint{6.670408in}{6.791255in}}%
\pgfpathlineto{\pgfqpoint{6.723452in}{6.837336in}}%
\pgfpathlineto{\pgfqpoint{6.776496in}{6.859127in}}%
\pgfpathlineto{\pgfqpoint{6.829540in}{6.849303in}}%
\pgfpathlineto{\pgfqpoint{6.882584in}{6.881359in}}%
\pgfpathlineto{\pgfqpoint{6.935628in}{6.842957in}}%
\pgfpathlineto{\pgfqpoint{6.988671in}{6.886633in}}%
\pgfpathlineto{\pgfqpoint{7.041715in}{6.862219in}}%
\pgfpathlineto{\pgfqpoint{7.094759in}{6.885270in}}%
\pgfpathlineto{\pgfqpoint{7.147803in}{6.867065in}}%
\pgfpathlineto{\pgfqpoint{7.200847in}{6.818171in}}%
\pgfpathlineto{\pgfqpoint{7.253891in}{6.719285in}}%
\pgfpathlineto{\pgfqpoint{7.306935in}{6.747320in}}%
\pgfpathlineto{\pgfqpoint{7.359979in}{6.672189in}}%
\pgfpathlineto{\pgfqpoint{7.413023in}{6.684093in}}%
\pgfpathlineto{\pgfqpoint{7.466067in}{6.654137in}}%
\pgfpathlineto{\pgfqpoint{7.519111in}{6.606265in}}%
\pgfpathlineto{\pgfqpoint{7.572155in}{6.593688in}}%
\pgfpathlineto{\pgfqpoint{7.625199in}{6.652740in}}%
\pgfpathlineto{\pgfqpoint{7.678243in}{6.697655in}}%
\pgfpathlineto{\pgfqpoint{7.731287in}{6.660188in}}%
\pgfpathlineto{\pgfqpoint{7.784330in}{6.696220in}}%
\pgfpathlineto{\pgfqpoint{7.837374in}{6.704471in}}%
\pgfpathlineto{\pgfqpoint{7.890418in}{6.745774in}}%
\pgfpathlineto{\pgfqpoint{7.943462in}{6.743520in}}%
\pgfpathlineto{\pgfqpoint{7.996506in}{6.822223in}}%
\pgfpathlineto{\pgfqpoint{8.049550in}{6.853606in}}%
\pgfpathlineto{\pgfqpoint{8.102594in}{6.863050in}}%
\pgfpathlineto{\pgfqpoint{8.155638in}{6.888859in}}%
\pgfpathlineto{\pgfqpoint{8.208682in}{6.883000in}}%
\pgfpathlineto{\pgfqpoint{8.261726in}{6.924712in}}%
\pgfpathlineto{\pgfqpoint{8.314770in}{6.876087in}}%
\pgfpathlineto{\pgfqpoint{8.367814in}{6.865417in}}%
\pgfpathlineto{\pgfqpoint{8.420858in}{6.857476in}}%
\pgfpathlineto{\pgfqpoint{8.473902in}{6.806195in}}%
\pgfpathlineto{\pgfqpoint{8.526946in}{6.778054in}}%
\pgfpathlineto{\pgfqpoint{8.579990in}{6.685501in}}%
\pgfpathlineto{\pgfqpoint{8.633033in}{6.694774in}}%
\pgfpathlineto{\pgfqpoint{8.686077in}{6.723687in}}%
\pgfpathlineto{\pgfqpoint{8.739121in}{6.662088in}}%
\pgfpathlineto{\pgfqpoint{8.792165in}{6.664742in}}%
\pgfpathlineto{\pgfqpoint{8.845209in}{6.642906in}}%
\pgfpathlineto{\pgfqpoint{8.898253in}{6.685692in}}%
\pgfpathlineto{\pgfqpoint{8.951297in}{6.629260in}}%
\pgfpathlineto{\pgfqpoint{9.004341in}{6.648198in}}%
\pgfpathlineto{\pgfqpoint{9.057385in}{6.713621in}}%
\pgfpathlineto{\pgfqpoint{9.110429in}{6.748714in}}%
\pgfpathlineto{\pgfqpoint{9.163473in}{6.770647in}}%
\pgfpathlineto{\pgfqpoint{9.216517in}{6.804492in}}%
\pgfpathlineto{\pgfqpoint{9.269561in}{6.811485in}}%
\pgfpathlineto{\pgfqpoint{9.322605in}{6.883905in}}%
\pgfpathlineto{\pgfqpoint{9.375649in}{6.879220in}}%
\pgfpathlineto{\pgfqpoint{9.428692in}{6.905077in}}%
\pgfpathlineto{\pgfqpoint{9.481736in}{6.884500in}}%
\pgfpathlineto{\pgfqpoint{9.534780in}{6.946800in}}%
\pgfpathlineto{\pgfqpoint{9.587824in}{6.868462in}}%
\pgfpathlineto{\pgfqpoint{9.640868in}{6.864012in}}%
\pgfpathlineto{\pgfqpoint{9.693912in}{6.803680in}}%
\pgfpathlineto{\pgfqpoint{9.746956in}{6.779712in}}%
\pgfpathlineto{\pgfqpoint{9.800000in}{6.765861in}}%
\pgfpathlineto{\pgfqpoint{9.800000in}{6.765861in}}%
\pgfusepath{stroke}%
\end{pgfscope}%
\begin{pgfscope}%
\pgfsetbuttcap%
\pgfsetmiterjoin%
\definecolor{currentfill}{rgb}{1.000000,1.000000,1.000000}%
\pgfsetfillcolor{currentfill}%
\pgfsetlinewidth{1.003750pt}%
\definecolor{currentstroke}{rgb}{0.000000,0.000000,0.000000}%
\pgfsetstrokecolor{currentstroke}%
\pgfsetdash{}{0pt}%
\pgfpathmoveto{\pgfqpoint{1.017884in}{7.466516in}}%
\pgfpathlineto{\pgfqpoint{1.304733in}{7.466516in}}%
\pgfpathlineto{\pgfqpoint{1.304733in}{7.779293in}}%
\pgfpathlineto{\pgfqpoint{1.017884in}{7.779293in}}%
\pgfpathlineto{\pgfqpoint{1.017884in}{7.466516in}}%
\pgfpathclose%
\pgfusepath{stroke,fill}%
\end{pgfscope}%
\begin{pgfscope}%
\definecolor{textcolor}{rgb}{0.000000,0.000000,0.000000}%
\pgfsetstrokecolor{textcolor}%
\pgfsetfillcolor{textcolor}%
\pgftext[x=1.074273in,y=7.572904in,left,base]{\color{textcolor}{\rmfamily\fontsize{14.000000}{16.800000}\selectfont\catcode`\^=\active\def^{\ifmmode\sp\else\^{}\fi}\catcode`\%=\active\def%{\%}a)}}%
\end{pgfscope}%
\begin{pgfscope}%
\pgfsetbuttcap%
\pgfsetmiterjoin%
\definecolor{currentfill}{rgb}{1.000000,1.000000,1.000000}%
\pgfsetfillcolor{currentfill}%
\pgfsetlinewidth{0.000000pt}%
\definecolor{currentstroke}{rgb}{0.000000,0.000000,0.000000}%
\pgfsetstrokecolor{currentstroke}%
\pgfsetstrokeopacity{0.000000}%
\pgfsetdash{}{0pt}%
\pgfpathmoveto{\pgfqpoint{0.941663in}{0.670138in}}%
\pgfpathlineto{\pgfqpoint{9.800000in}{0.670138in}}%
\pgfpathlineto{\pgfqpoint{9.800000in}{4.135763in}}%
\pgfpathlineto{\pgfqpoint{0.941663in}{4.135763in}}%
\pgfpathlineto{\pgfqpoint{0.941663in}{0.670138in}}%
\pgfpathclose%
\pgfusepath{fill}%
\end{pgfscope}%
\begin{pgfscope}%
\pgfpathrectangle{\pgfqpoint{0.941663in}{0.670138in}}{\pgfqpoint{8.858337in}{3.465625in}}%
\pgfusepath{clip}%
\pgfsetrectcap%
\pgfsetroundjoin%
\pgfsetlinewidth{0.803000pt}%
\definecolor{currentstroke}{rgb}{0.690196,0.690196,0.690196}%
\pgfsetstrokecolor{currentstroke}%
\pgfsetdash{}{0pt}%
\pgfpathmoveto{\pgfqpoint{0.941663in}{0.670138in}}%
\pgfpathlineto{\pgfqpoint{0.941663in}{4.135763in}}%
\pgfusepath{stroke}%
\end{pgfscope}%
\begin{pgfscope}%
\pgfsetbuttcap%
\pgfsetroundjoin%
\definecolor{currentfill}{rgb}{0.000000,0.000000,0.000000}%
\pgfsetfillcolor{currentfill}%
\pgfsetlinewidth{0.803000pt}%
\definecolor{currentstroke}{rgb}{0.000000,0.000000,0.000000}%
\pgfsetstrokecolor{currentstroke}%
\pgfsetdash{}{0pt}%
\pgfsys@defobject{currentmarker}{\pgfqpoint{0.000000in}{-0.048611in}}{\pgfqpoint{0.000000in}{0.000000in}}{%
\pgfpathmoveto{\pgfqpoint{0.000000in}{0.000000in}}%
\pgfpathlineto{\pgfqpoint{0.000000in}{-0.048611in}}%
\pgfusepath{stroke,fill}%
}%
\begin{pgfscope}%
\pgfsys@transformshift{0.941663in}{0.670138in}%
\pgfsys@useobject{currentmarker}{}%
\end{pgfscope}%
\end{pgfscope}%
\begin{pgfscope}%
\definecolor{textcolor}{rgb}{0.000000,0.000000,0.000000}%
\pgfsetstrokecolor{textcolor}%
\pgfsetfillcolor{textcolor}%
\pgftext[x=0.941663in,y=0.572916in,,top]{\color{textcolor}{\rmfamily\fontsize{14.000000}{16.800000}\selectfont\catcode`\^=\active\def^{\ifmmode\sp\else\^{}\fi}\catcode`\%=\active\def%{\%}$\mathdefault{0}$}}%
\end{pgfscope}%
\begin{pgfscope}%
\pgfpathrectangle{\pgfqpoint{0.941663in}{0.670138in}}{\pgfqpoint{8.858337in}{3.465625in}}%
\pgfusepath{clip}%
\pgfsetrectcap%
\pgfsetroundjoin%
\pgfsetlinewidth{0.803000pt}%
\definecolor{currentstroke}{rgb}{0.690196,0.690196,0.690196}%
\pgfsetstrokecolor{currentstroke}%
\pgfsetdash{}{0pt}%
\pgfpathmoveto{\pgfqpoint{2.002542in}{0.670138in}}%
\pgfpathlineto{\pgfqpoint{2.002542in}{4.135763in}}%
\pgfusepath{stroke}%
\end{pgfscope}%
\begin{pgfscope}%
\pgfsetbuttcap%
\pgfsetroundjoin%
\definecolor{currentfill}{rgb}{0.000000,0.000000,0.000000}%
\pgfsetfillcolor{currentfill}%
\pgfsetlinewidth{0.803000pt}%
\definecolor{currentstroke}{rgb}{0.000000,0.000000,0.000000}%
\pgfsetstrokecolor{currentstroke}%
\pgfsetdash{}{0pt}%
\pgfsys@defobject{currentmarker}{\pgfqpoint{0.000000in}{-0.048611in}}{\pgfqpoint{0.000000in}{0.000000in}}{%
\pgfpathmoveto{\pgfqpoint{0.000000in}{0.000000in}}%
\pgfpathlineto{\pgfqpoint{0.000000in}{-0.048611in}}%
\pgfusepath{stroke,fill}%
}%
\begin{pgfscope}%
\pgfsys@transformshift{2.002542in}{0.670138in}%
\pgfsys@useobject{currentmarker}{}%
\end{pgfscope}%
\end{pgfscope}%
\begin{pgfscope}%
\definecolor{textcolor}{rgb}{0.000000,0.000000,0.000000}%
\pgfsetstrokecolor{textcolor}%
\pgfsetfillcolor{textcolor}%
\pgftext[x=2.002542in,y=0.572916in,,top]{\color{textcolor}{\rmfamily\fontsize{14.000000}{16.800000}\selectfont\catcode`\^=\active\def^{\ifmmode\sp\else\^{}\fi}\catcode`\%=\active\def%{\%}$\mathdefault{20}$}}%
\end{pgfscope}%
\begin{pgfscope}%
\pgfpathrectangle{\pgfqpoint{0.941663in}{0.670138in}}{\pgfqpoint{8.858337in}{3.465625in}}%
\pgfusepath{clip}%
\pgfsetrectcap%
\pgfsetroundjoin%
\pgfsetlinewidth{0.803000pt}%
\definecolor{currentstroke}{rgb}{0.690196,0.690196,0.690196}%
\pgfsetstrokecolor{currentstroke}%
\pgfsetdash{}{0pt}%
\pgfpathmoveto{\pgfqpoint{3.063420in}{0.670138in}}%
\pgfpathlineto{\pgfqpoint{3.063420in}{4.135763in}}%
\pgfusepath{stroke}%
\end{pgfscope}%
\begin{pgfscope}%
\pgfsetbuttcap%
\pgfsetroundjoin%
\definecolor{currentfill}{rgb}{0.000000,0.000000,0.000000}%
\pgfsetfillcolor{currentfill}%
\pgfsetlinewidth{0.803000pt}%
\definecolor{currentstroke}{rgb}{0.000000,0.000000,0.000000}%
\pgfsetstrokecolor{currentstroke}%
\pgfsetdash{}{0pt}%
\pgfsys@defobject{currentmarker}{\pgfqpoint{0.000000in}{-0.048611in}}{\pgfqpoint{0.000000in}{0.000000in}}{%
\pgfpathmoveto{\pgfqpoint{0.000000in}{0.000000in}}%
\pgfpathlineto{\pgfqpoint{0.000000in}{-0.048611in}}%
\pgfusepath{stroke,fill}%
}%
\begin{pgfscope}%
\pgfsys@transformshift{3.063420in}{0.670138in}%
\pgfsys@useobject{currentmarker}{}%
\end{pgfscope}%
\end{pgfscope}%
\begin{pgfscope}%
\definecolor{textcolor}{rgb}{0.000000,0.000000,0.000000}%
\pgfsetstrokecolor{textcolor}%
\pgfsetfillcolor{textcolor}%
\pgftext[x=3.063420in,y=0.572916in,,top]{\color{textcolor}{\rmfamily\fontsize{14.000000}{16.800000}\selectfont\catcode`\^=\active\def^{\ifmmode\sp\else\^{}\fi}\catcode`\%=\active\def%{\%}$\mathdefault{40}$}}%
\end{pgfscope}%
\begin{pgfscope}%
\pgfpathrectangle{\pgfqpoint{0.941663in}{0.670138in}}{\pgfqpoint{8.858337in}{3.465625in}}%
\pgfusepath{clip}%
\pgfsetrectcap%
\pgfsetroundjoin%
\pgfsetlinewidth{0.803000pt}%
\definecolor{currentstroke}{rgb}{0.690196,0.690196,0.690196}%
\pgfsetstrokecolor{currentstroke}%
\pgfsetdash{}{0pt}%
\pgfpathmoveto{\pgfqpoint{4.124299in}{0.670138in}}%
\pgfpathlineto{\pgfqpoint{4.124299in}{4.135763in}}%
\pgfusepath{stroke}%
\end{pgfscope}%
\begin{pgfscope}%
\pgfsetbuttcap%
\pgfsetroundjoin%
\definecolor{currentfill}{rgb}{0.000000,0.000000,0.000000}%
\pgfsetfillcolor{currentfill}%
\pgfsetlinewidth{0.803000pt}%
\definecolor{currentstroke}{rgb}{0.000000,0.000000,0.000000}%
\pgfsetstrokecolor{currentstroke}%
\pgfsetdash{}{0pt}%
\pgfsys@defobject{currentmarker}{\pgfqpoint{0.000000in}{-0.048611in}}{\pgfqpoint{0.000000in}{0.000000in}}{%
\pgfpathmoveto{\pgfqpoint{0.000000in}{0.000000in}}%
\pgfpathlineto{\pgfqpoint{0.000000in}{-0.048611in}}%
\pgfusepath{stroke,fill}%
}%
\begin{pgfscope}%
\pgfsys@transformshift{4.124299in}{0.670138in}%
\pgfsys@useobject{currentmarker}{}%
\end{pgfscope}%
\end{pgfscope}%
\begin{pgfscope}%
\definecolor{textcolor}{rgb}{0.000000,0.000000,0.000000}%
\pgfsetstrokecolor{textcolor}%
\pgfsetfillcolor{textcolor}%
\pgftext[x=4.124299in,y=0.572916in,,top]{\color{textcolor}{\rmfamily\fontsize{14.000000}{16.800000}\selectfont\catcode`\^=\active\def^{\ifmmode\sp\else\^{}\fi}\catcode`\%=\active\def%{\%}$\mathdefault{60}$}}%
\end{pgfscope}%
\begin{pgfscope}%
\pgfpathrectangle{\pgfqpoint{0.941663in}{0.670138in}}{\pgfqpoint{8.858337in}{3.465625in}}%
\pgfusepath{clip}%
\pgfsetrectcap%
\pgfsetroundjoin%
\pgfsetlinewidth{0.803000pt}%
\definecolor{currentstroke}{rgb}{0.690196,0.690196,0.690196}%
\pgfsetstrokecolor{currentstroke}%
\pgfsetdash{}{0pt}%
\pgfpathmoveto{\pgfqpoint{5.185178in}{0.670138in}}%
\pgfpathlineto{\pgfqpoint{5.185178in}{4.135763in}}%
\pgfusepath{stroke}%
\end{pgfscope}%
\begin{pgfscope}%
\pgfsetbuttcap%
\pgfsetroundjoin%
\definecolor{currentfill}{rgb}{0.000000,0.000000,0.000000}%
\pgfsetfillcolor{currentfill}%
\pgfsetlinewidth{0.803000pt}%
\definecolor{currentstroke}{rgb}{0.000000,0.000000,0.000000}%
\pgfsetstrokecolor{currentstroke}%
\pgfsetdash{}{0pt}%
\pgfsys@defobject{currentmarker}{\pgfqpoint{0.000000in}{-0.048611in}}{\pgfqpoint{0.000000in}{0.000000in}}{%
\pgfpathmoveto{\pgfqpoint{0.000000in}{0.000000in}}%
\pgfpathlineto{\pgfqpoint{0.000000in}{-0.048611in}}%
\pgfusepath{stroke,fill}%
}%
\begin{pgfscope}%
\pgfsys@transformshift{5.185178in}{0.670138in}%
\pgfsys@useobject{currentmarker}{}%
\end{pgfscope}%
\end{pgfscope}%
\begin{pgfscope}%
\definecolor{textcolor}{rgb}{0.000000,0.000000,0.000000}%
\pgfsetstrokecolor{textcolor}%
\pgfsetfillcolor{textcolor}%
\pgftext[x=5.185178in,y=0.572916in,,top]{\color{textcolor}{\rmfamily\fontsize{14.000000}{16.800000}\selectfont\catcode`\^=\active\def^{\ifmmode\sp\else\^{}\fi}\catcode`\%=\active\def%{\%}$\mathdefault{80}$}}%
\end{pgfscope}%
\begin{pgfscope}%
\pgfpathrectangle{\pgfqpoint{0.941663in}{0.670138in}}{\pgfqpoint{8.858337in}{3.465625in}}%
\pgfusepath{clip}%
\pgfsetrectcap%
\pgfsetroundjoin%
\pgfsetlinewidth{0.803000pt}%
\definecolor{currentstroke}{rgb}{0.690196,0.690196,0.690196}%
\pgfsetstrokecolor{currentstroke}%
\pgfsetdash{}{0pt}%
\pgfpathmoveto{\pgfqpoint{6.246056in}{0.670138in}}%
\pgfpathlineto{\pgfqpoint{6.246056in}{4.135763in}}%
\pgfusepath{stroke}%
\end{pgfscope}%
\begin{pgfscope}%
\pgfsetbuttcap%
\pgfsetroundjoin%
\definecolor{currentfill}{rgb}{0.000000,0.000000,0.000000}%
\pgfsetfillcolor{currentfill}%
\pgfsetlinewidth{0.803000pt}%
\definecolor{currentstroke}{rgb}{0.000000,0.000000,0.000000}%
\pgfsetstrokecolor{currentstroke}%
\pgfsetdash{}{0pt}%
\pgfsys@defobject{currentmarker}{\pgfqpoint{0.000000in}{-0.048611in}}{\pgfqpoint{0.000000in}{0.000000in}}{%
\pgfpathmoveto{\pgfqpoint{0.000000in}{0.000000in}}%
\pgfpathlineto{\pgfqpoint{0.000000in}{-0.048611in}}%
\pgfusepath{stroke,fill}%
}%
\begin{pgfscope}%
\pgfsys@transformshift{6.246056in}{0.670138in}%
\pgfsys@useobject{currentmarker}{}%
\end{pgfscope}%
\end{pgfscope}%
\begin{pgfscope}%
\definecolor{textcolor}{rgb}{0.000000,0.000000,0.000000}%
\pgfsetstrokecolor{textcolor}%
\pgfsetfillcolor{textcolor}%
\pgftext[x=6.246056in,y=0.572916in,,top]{\color{textcolor}{\rmfamily\fontsize{14.000000}{16.800000}\selectfont\catcode`\^=\active\def^{\ifmmode\sp\else\^{}\fi}\catcode`\%=\active\def%{\%}$\mathdefault{100}$}}%
\end{pgfscope}%
\begin{pgfscope}%
\pgfpathrectangle{\pgfqpoint{0.941663in}{0.670138in}}{\pgfqpoint{8.858337in}{3.465625in}}%
\pgfusepath{clip}%
\pgfsetrectcap%
\pgfsetroundjoin%
\pgfsetlinewidth{0.803000pt}%
\definecolor{currentstroke}{rgb}{0.690196,0.690196,0.690196}%
\pgfsetstrokecolor{currentstroke}%
\pgfsetdash{}{0pt}%
\pgfpathmoveto{\pgfqpoint{7.306935in}{0.670138in}}%
\pgfpathlineto{\pgfqpoint{7.306935in}{4.135763in}}%
\pgfusepath{stroke}%
\end{pgfscope}%
\begin{pgfscope}%
\pgfsetbuttcap%
\pgfsetroundjoin%
\definecolor{currentfill}{rgb}{0.000000,0.000000,0.000000}%
\pgfsetfillcolor{currentfill}%
\pgfsetlinewidth{0.803000pt}%
\definecolor{currentstroke}{rgb}{0.000000,0.000000,0.000000}%
\pgfsetstrokecolor{currentstroke}%
\pgfsetdash{}{0pt}%
\pgfsys@defobject{currentmarker}{\pgfqpoint{0.000000in}{-0.048611in}}{\pgfqpoint{0.000000in}{0.000000in}}{%
\pgfpathmoveto{\pgfqpoint{0.000000in}{0.000000in}}%
\pgfpathlineto{\pgfqpoint{0.000000in}{-0.048611in}}%
\pgfusepath{stroke,fill}%
}%
\begin{pgfscope}%
\pgfsys@transformshift{7.306935in}{0.670138in}%
\pgfsys@useobject{currentmarker}{}%
\end{pgfscope}%
\end{pgfscope}%
\begin{pgfscope}%
\definecolor{textcolor}{rgb}{0.000000,0.000000,0.000000}%
\pgfsetstrokecolor{textcolor}%
\pgfsetfillcolor{textcolor}%
\pgftext[x=7.306935in,y=0.572916in,,top]{\color{textcolor}{\rmfamily\fontsize{14.000000}{16.800000}\selectfont\catcode`\^=\active\def^{\ifmmode\sp\else\^{}\fi}\catcode`\%=\active\def%{\%}$\mathdefault{120}$}}%
\end{pgfscope}%
\begin{pgfscope}%
\pgfpathrectangle{\pgfqpoint{0.941663in}{0.670138in}}{\pgfqpoint{8.858337in}{3.465625in}}%
\pgfusepath{clip}%
\pgfsetrectcap%
\pgfsetroundjoin%
\pgfsetlinewidth{0.803000pt}%
\definecolor{currentstroke}{rgb}{0.690196,0.690196,0.690196}%
\pgfsetstrokecolor{currentstroke}%
\pgfsetdash{}{0pt}%
\pgfpathmoveto{\pgfqpoint{8.367814in}{0.670138in}}%
\pgfpathlineto{\pgfqpoint{8.367814in}{4.135763in}}%
\pgfusepath{stroke}%
\end{pgfscope}%
\begin{pgfscope}%
\pgfsetbuttcap%
\pgfsetroundjoin%
\definecolor{currentfill}{rgb}{0.000000,0.000000,0.000000}%
\pgfsetfillcolor{currentfill}%
\pgfsetlinewidth{0.803000pt}%
\definecolor{currentstroke}{rgb}{0.000000,0.000000,0.000000}%
\pgfsetstrokecolor{currentstroke}%
\pgfsetdash{}{0pt}%
\pgfsys@defobject{currentmarker}{\pgfqpoint{0.000000in}{-0.048611in}}{\pgfqpoint{0.000000in}{0.000000in}}{%
\pgfpathmoveto{\pgfqpoint{0.000000in}{0.000000in}}%
\pgfpathlineto{\pgfqpoint{0.000000in}{-0.048611in}}%
\pgfusepath{stroke,fill}%
}%
\begin{pgfscope}%
\pgfsys@transformshift{8.367814in}{0.670138in}%
\pgfsys@useobject{currentmarker}{}%
\end{pgfscope}%
\end{pgfscope}%
\begin{pgfscope}%
\definecolor{textcolor}{rgb}{0.000000,0.000000,0.000000}%
\pgfsetstrokecolor{textcolor}%
\pgfsetfillcolor{textcolor}%
\pgftext[x=8.367814in,y=0.572916in,,top]{\color{textcolor}{\rmfamily\fontsize{14.000000}{16.800000}\selectfont\catcode`\^=\active\def^{\ifmmode\sp\else\^{}\fi}\catcode`\%=\active\def%{\%}$\mathdefault{140}$}}%
\end{pgfscope}%
\begin{pgfscope}%
\pgfpathrectangle{\pgfqpoint{0.941663in}{0.670138in}}{\pgfqpoint{8.858337in}{3.465625in}}%
\pgfusepath{clip}%
\pgfsetrectcap%
\pgfsetroundjoin%
\pgfsetlinewidth{0.803000pt}%
\definecolor{currentstroke}{rgb}{0.690196,0.690196,0.690196}%
\pgfsetstrokecolor{currentstroke}%
\pgfsetdash{}{0pt}%
\pgfpathmoveto{\pgfqpoint{9.428692in}{0.670138in}}%
\pgfpathlineto{\pgfqpoint{9.428692in}{4.135763in}}%
\pgfusepath{stroke}%
\end{pgfscope}%
\begin{pgfscope}%
\pgfsetbuttcap%
\pgfsetroundjoin%
\definecolor{currentfill}{rgb}{0.000000,0.000000,0.000000}%
\pgfsetfillcolor{currentfill}%
\pgfsetlinewidth{0.803000pt}%
\definecolor{currentstroke}{rgb}{0.000000,0.000000,0.000000}%
\pgfsetstrokecolor{currentstroke}%
\pgfsetdash{}{0pt}%
\pgfsys@defobject{currentmarker}{\pgfqpoint{0.000000in}{-0.048611in}}{\pgfqpoint{0.000000in}{0.000000in}}{%
\pgfpathmoveto{\pgfqpoint{0.000000in}{0.000000in}}%
\pgfpathlineto{\pgfqpoint{0.000000in}{-0.048611in}}%
\pgfusepath{stroke,fill}%
}%
\begin{pgfscope}%
\pgfsys@transformshift{9.428692in}{0.670138in}%
\pgfsys@useobject{currentmarker}{}%
\end{pgfscope}%
\end{pgfscope}%
\begin{pgfscope}%
\definecolor{textcolor}{rgb}{0.000000,0.000000,0.000000}%
\pgfsetstrokecolor{textcolor}%
\pgfsetfillcolor{textcolor}%
\pgftext[x=9.428692in,y=0.572916in,,top]{\color{textcolor}{\rmfamily\fontsize{14.000000}{16.800000}\selectfont\catcode`\^=\active\def^{\ifmmode\sp\else\^{}\fi}\catcode`\%=\active\def%{\%}$\mathdefault{160}$}}%
\end{pgfscope}%
\begin{pgfscope}%
\definecolor{textcolor}{rgb}{0.000000,0.000000,0.000000}%
\pgfsetstrokecolor{textcolor}%
\pgfsetfillcolor{textcolor}%
\pgftext[x=5.370831in,y=0.339583in,,top]{\color{textcolor}{\rmfamily\fontsize{18.000000}{21.600000}\selectfont\catcode`\^=\active\def^{\ifmmode\sp\else\^{}\fi}\catcode`\%=\active\def%{\%}Time [hours]}}%
\end{pgfscope}%
\begin{pgfscope}%
\pgfpathrectangle{\pgfqpoint{0.941663in}{0.670138in}}{\pgfqpoint{8.858337in}{3.465625in}}%
\pgfusepath{clip}%
\pgfsetrectcap%
\pgfsetroundjoin%
\pgfsetlinewidth{0.803000pt}%
\definecolor{currentstroke}{rgb}{0.690196,0.690196,0.690196}%
\pgfsetstrokecolor{currentstroke}%
\pgfsetdash{}{0pt}%
\pgfpathmoveto{\pgfqpoint{0.941663in}{1.195548in}}%
\pgfpathlineto{\pgfqpoint{9.800000in}{1.195548in}}%
\pgfusepath{stroke}%
\end{pgfscope}%
\begin{pgfscope}%
\pgfsetbuttcap%
\pgfsetroundjoin%
\definecolor{currentfill}{rgb}{0.000000,0.000000,0.000000}%
\pgfsetfillcolor{currentfill}%
\pgfsetlinewidth{0.803000pt}%
\definecolor{currentstroke}{rgb}{0.000000,0.000000,0.000000}%
\pgfsetstrokecolor{currentstroke}%
\pgfsetdash{}{0pt}%
\pgfsys@defobject{currentmarker}{\pgfqpoint{-0.048611in}{0.000000in}}{\pgfqpoint{-0.000000in}{0.000000in}}{%
\pgfpathmoveto{\pgfqpoint{-0.000000in}{0.000000in}}%
\pgfpathlineto{\pgfqpoint{-0.048611in}{0.000000in}}%
\pgfusepath{stroke,fill}%
}%
\begin{pgfscope}%
\pgfsys@transformshift{0.941663in}{1.195548in}%
\pgfsys@useobject{currentmarker}{}%
\end{pgfscope}%
\end{pgfscope}%
\begin{pgfscope}%
\definecolor{textcolor}{rgb}{0.000000,0.000000,0.000000}%
\pgfsetstrokecolor{textcolor}%
\pgfsetfillcolor{textcolor}%
\pgftext[x=0.395138in, y=1.126104in, left, base]{\color{textcolor}{\rmfamily\fontsize{14.000000}{16.800000}\selectfont\catcode`\^=\active\def^{\ifmmode\sp\else\^{}\fi}\catcode`\%=\active\def%{\%}$\mathdefault{\ensuremath{-}500}$}}%
\end{pgfscope}%
\begin{pgfscope}%
\pgfpathrectangle{\pgfqpoint{0.941663in}{0.670138in}}{\pgfqpoint{8.858337in}{3.465625in}}%
\pgfusepath{clip}%
\pgfsetrectcap%
\pgfsetroundjoin%
\pgfsetlinewidth{0.803000pt}%
\definecolor{currentstroke}{rgb}{0.690196,0.690196,0.690196}%
\pgfsetstrokecolor{currentstroke}%
\pgfsetdash{}{0pt}%
\pgfpathmoveto{\pgfqpoint{0.941663in}{1.891220in}}%
\pgfpathlineto{\pgfqpoint{9.800000in}{1.891220in}}%
\pgfusepath{stroke}%
\end{pgfscope}%
\begin{pgfscope}%
\pgfsetbuttcap%
\pgfsetroundjoin%
\definecolor{currentfill}{rgb}{0.000000,0.000000,0.000000}%
\pgfsetfillcolor{currentfill}%
\pgfsetlinewidth{0.803000pt}%
\definecolor{currentstroke}{rgb}{0.000000,0.000000,0.000000}%
\pgfsetstrokecolor{currentstroke}%
\pgfsetdash{}{0pt}%
\pgfsys@defobject{currentmarker}{\pgfqpoint{-0.048611in}{0.000000in}}{\pgfqpoint{-0.000000in}{0.000000in}}{%
\pgfpathmoveto{\pgfqpoint{-0.000000in}{0.000000in}}%
\pgfpathlineto{\pgfqpoint{-0.048611in}{0.000000in}}%
\pgfusepath{stroke,fill}%
}%
\begin{pgfscope}%
\pgfsys@transformshift{0.941663in}{1.891220in}%
\pgfsys@useobject{currentmarker}{}%
\end{pgfscope}%
\end{pgfscope}%
\begin{pgfscope}%
\definecolor{textcolor}{rgb}{0.000000,0.000000,0.000000}%
\pgfsetstrokecolor{textcolor}%
\pgfsetfillcolor{textcolor}%
\pgftext[x=0.746525in, y=1.821776in, left, base]{\color{textcolor}{\rmfamily\fontsize{14.000000}{16.800000}\selectfont\catcode`\^=\active\def^{\ifmmode\sp\else\^{}\fi}\catcode`\%=\active\def%{\%}$\mathdefault{0}$}}%
\end{pgfscope}%
\begin{pgfscope}%
\pgfpathrectangle{\pgfqpoint{0.941663in}{0.670138in}}{\pgfqpoint{8.858337in}{3.465625in}}%
\pgfusepath{clip}%
\pgfsetrectcap%
\pgfsetroundjoin%
\pgfsetlinewidth{0.803000pt}%
\definecolor{currentstroke}{rgb}{0.690196,0.690196,0.690196}%
\pgfsetstrokecolor{currentstroke}%
\pgfsetdash{}{0pt}%
\pgfpathmoveto{\pgfqpoint{0.941663in}{2.586892in}}%
\pgfpathlineto{\pgfqpoint{9.800000in}{2.586892in}}%
\pgfusepath{stroke}%
\end{pgfscope}%
\begin{pgfscope}%
\pgfsetbuttcap%
\pgfsetroundjoin%
\definecolor{currentfill}{rgb}{0.000000,0.000000,0.000000}%
\pgfsetfillcolor{currentfill}%
\pgfsetlinewidth{0.803000pt}%
\definecolor{currentstroke}{rgb}{0.000000,0.000000,0.000000}%
\pgfsetstrokecolor{currentstroke}%
\pgfsetdash{}{0pt}%
\pgfsys@defobject{currentmarker}{\pgfqpoint{-0.048611in}{0.000000in}}{\pgfqpoint{-0.000000in}{0.000000in}}{%
\pgfpathmoveto{\pgfqpoint{-0.000000in}{0.000000in}}%
\pgfpathlineto{\pgfqpoint{-0.048611in}{0.000000in}}%
\pgfusepath{stroke,fill}%
}%
\begin{pgfscope}%
\pgfsys@transformshift{0.941663in}{2.586892in}%
\pgfsys@useobject{currentmarker}{}%
\end{pgfscope}%
\end{pgfscope}%
\begin{pgfscope}%
\definecolor{textcolor}{rgb}{0.000000,0.000000,0.000000}%
\pgfsetstrokecolor{textcolor}%
\pgfsetfillcolor{textcolor}%
\pgftext[x=0.550694in, y=2.517447in, left, base]{\color{textcolor}{\rmfamily\fontsize{14.000000}{16.800000}\selectfont\catcode`\^=\active\def^{\ifmmode\sp\else\^{}\fi}\catcode`\%=\active\def%{\%}$\mathdefault{500}$}}%
\end{pgfscope}%
\begin{pgfscope}%
\pgfpathrectangle{\pgfqpoint{0.941663in}{0.670138in}}{\pgfqpoint{8.858337in}{3.465625in}}%
\pgfusepath{clip}%
\pgfsetrectcap%
\pgfsetroundjoin%
\pgfsetlinewidth{0.803000pt}%
\definecolor{currentstroke}{rgb}{0.690196,0.690196,0.690196}%
\pgfsetstrokecolor{currentstroke}%
\pgfsetdash{}{0pt}%
\pgfpathmoveto{\pgfqpoint{0.941663in}{3.282563in}}%
\pgfpathlineto{\pgfqpoint{9.800000in}{3.282563in}}%
\pgfusepath{stroke}%
\end{pgfscope}%
\begin{pgfscope}%
\pgfsetbuttcap%
\pgfsetroundjoin%
\definecolor{currentfill}{rgb}{0.000000,0.000000,0.000000}%
\pgfsetfillcolor{currentfill}%
\pgfsetlinewidth{0.803000pt}%
\definecolor{currentstroke}{rgb}{0.000000,0.000000,0.000000}%
\pgfsetstrokecolor{currentstroke}%
\pgfsetdash{}{0pt}%
\pgfsys@defobject{currentmarker}{\pgfqpoint{-0.048611in}{0.000000in}}{\pgfqpoint{-0.000000in}{0.000000in}}{%
\pgfpathmoveto{\pgfqpoint{-0.000000in}{0.000000in}}%
\pgfpathlineto{\pgfqpoint{-0.048611in}{0.000000in}}%
\pgfusepath{stroke,fill}%
}%
\begin{pgfscope}%
\pgfsys@transformshift{0.941663in}{3.282563in}%
\pgfsys@useobject{currentmarker}{}%
\end{pgfscope}%
\end{pgfscope}%
\begin{pgfscope}%
\definecolor{textcolor}{rgb}{0.000000,0.000000,0.000000}%
\pgfsetstrokecolor{textcolor}%
\pgfsetfillcolor{textcolor}%
\pgftext[x=0.452779in, y=3.213119in, left, base]{\color{textcolor}{\rmfamily\fontsize{14.000000}{16.800000}\selectfont\catcode`\^=\active\def^{\ifmmode\sp\else\^{}\fi}\catcode`\%=\active\def%{\%}$\mathdefault{1000}$}}%
\end{pgfscope}%
\begin{pgfscope}%
\pgfpathrectangle{\pgfqpoint{0.941663in}{0.670138in}}{\pgfqpoint{8.858337in}{3.465625in}}%
\pgfusepath{clip}%
\pgfsetrectcap%
\pgfsetroundjoin%
\pgfsetlinewidth{0.803000pt}%
\definecolor{currentstroke}{rgb}{0.690196,0.690196,0.690196}%
\pgfsetstrokecolor{currentstroke}%
\pgfsetdash{}{0pt}%
\pgfpathmoveto{\pgfqpoint{0.941663in}{3.978235in}}%
\pgfpathlineto{\pgfqpoint{9.800000in}{3.978235in}}%
\pgfusepath{stroke}%
\end{pgfscope}%
\begin{pgfscope}%
\pgfsetbuttcap%
\pgfsetroundjoin%
\definecolor{currentfill}{rgb}{0.000000,0.000000,0.000000}%
\pgfsetfillcolor{currentfill}%
\pgfsetlinewidth{0.803000pt}%
\definecolor{currentstroke}{rgb}{0.000000,0.000000,0.000000}%
\pgfsetstrokecolor{currentstroke}%
\pgfsetdash{}{0pt}%
\pgfsys@defobject{currentmarker}{\pgfqpoint{-0.048611in}{0.000000in}}{\pgfqpoint{-0.000000in}{0.000000in}}{%
\pgfpathmoveto{\pgfqpoint{-0.000000in}{0.000000in}}%
\pgfpathlineto{\pgfqpoint{-0.048611in}{0.000000in}}%
\pgfusepath{stroke,fill}%
}%
\begin{pgfscope}%
\pgfsys@transformshift{0.941663in}{3.978235in}%
\pgfsys@useobject{currentmarker}{}%
\end{pgfscope}%
\end{pgfscope}%
\begin{pgfscope}%
\definecolor{textcolor}{rgb}{0.000000,0.000000,0.000000}%
\pgfsetstrokecolor{textcolor}%
\pgfsetfillcolor{textcolor}%
\pgftext[x=0.452779in, y=3.908791in, left, base]{\color{textcolor}{\rmfamily\fontsize{14.000000}{16.800000}\selectfont\catcode`\^=\active\def^{\ifmmode\sp\else\^{}\fi}\catcode`\%=\active\def%{\%}$\mathdefault{1500}$}}%
\end{pgfscope}%
\begin{pgfscope}%
\definecolor{textcolor}{rgb}{0.000000,0.000000,0.000000}%
\pgfsetstrokecolor{textcolor}%
\pgfsetfillcolor{textcolor}%
\pgftext[x=0.339583in,y=2.402951in,,bottom,rotate=90.000000]{\color{textcolor}{\rmfamily\fontsize{18.000000}{21.600000}\selectfont\catcode`\^=\active\def^{\ifmmode\sp\else\^{}\fi}\catcode`\%=\active\def%{\%}Energy [MWh]}}%
\end{pgfscope}%
\begin{pgfscope}%
\pgfpathrectangle{\pgfqpoint{0.941663in}{0.670138in}}{\pgfqpoint{8.858337in}{3.465625in}}%
\pgfusepath{clip}%
\pgfsetrectcap%
\pgfsetroundjoin%
\pgfsetlinewidth{1.505625pt}%
\definecolor{currentstroke}{rgb}{0.121569,0.466667,0.705882}%
\pgfsetstrokecolor{currentstroke}%
\pgfsetdash{}{0pt}%
\pgfpathmoveto{\pgfqpoint{0.941663in}{2.586892in}}%
\pgfpathlineto{\pgfqpoint{9.800000in}{2.586892in}}%
\pgfpathlineto{\pgfqpoint{9.800000in}{2.586892in}}%
\pgfusepath{stroke}%
\end{pgfscope}%
\begin{pgfscope}%
\pgfpathrectangle{\pgfqpoint{0.941663in}{0.670138in}}{\pgfqpoint{8.858337in}{3.465625in}}%
\pgfusepath{clip}%
\pgfsetbuttcap%
\pgfsetroundjoin%
\definecolor{currentfill}{rgb}{0.121569,0.466667,0.705882}%
\pgfsetfillcolor{currentfill}%
\pgfsetlinewidth{1.003750pt}%
\definecolor{currentstroke}{rgb}{0.121569,0.466667,0.705882}%
\pgfsetstrokecolor{currentstroke}%
\pgfsetdash{}{0pt}%
\pgfsys@defobject{currentmarker}{\pgfqpoint{0.941663in}{1.891220in}}{\pgfqpoint{9.800000in}{2.586892in}}{%
\pgfpathmoveto{\pgfqpoint{0.941663in}{2.586892in}}%
\pgfpathlineto{\pgfqpoint{0.941663in}{1.891220in}}%
\pgfpathlineto{\pgfqpoint{0.994707in}{1.891220in}}%
\pgfpathlineto{\pgfqpoint{1.047751in}{1.891220in}}%
\pgfpathlineto{\pgfqpoint{1.100795in}{1.891220in}}%
\pgfpathlineto{\pgfqpoint{1.153839in}{1.891220in}}%
\pgfpathlineto{\pgfqpoint{1.206883in}{1.891220in}}%
\pgfpathlineto{\pgfqpoint{1.259927in}{1.891220in}}%
\pgfpathlineto{\pgfqpoint{1.312970in}{1.891220in}}%
\pgfpathlineto{\pgfqpoint{1.366014in}{1.891220in}}%
\pgfpathlineto{\pgfqpoint{1.419058in}{1.891220in}}%
\pgfpathlineto{\pgfqpoint{1.472102in}{1.891220in}}%
\pgfpathlineto{\pgfqpoint{1.525146in}{1.891220in}}%
\pgfpathlineto{\pgfqpoint{1.578190in}{1.891220in}}%
\pgfpathlineto{\pgfqpoint{1.631234in}{1.891220in}}%
\pgfpathlineto{\pgfqpoint{1.684278in}{1.891220in}}%
\pgfpathlineto{\pgfqpoint{1.737322in}{1.891220in}}%
\pgfpathlineto{\pgfqpoint{1.790366in}{1.891220in}}%
\pgfpathlineto{\pgfqpoint{1.843410in}{1.891220in}}%
\pgfpathlineto{\pgfqpoint{1.896454in}{1.891220in}}%
\pgfpathlineto{\pgfqpoint{1.949498in}{1.891220in}}%
\pgfpathlineto{\pgfqpoint{2.002542in}{1.891220in}}%
\pgfpathlineto{\pgfqpoint{2.055586in}{1.891220in}}%
\pgfpathlineto{\pgfqpoint{2.108629in}{1.891220in}}%
\pgfpathlineto{\pgfqpoint{2.161673in}{1.891220in}}%
\pgfpathlineto{\pgfqpoint{2.214717in}{1.891220in}}%
\pgfpathlineto{\pgfqpoint{2.267761in}{1.891220in}}%
\pgfpathlineto{\pgfqpoint{2.320805in}{1.891220in}}%
\pgfpathlineto{\pgfqpoint{2.373849in}{1.891220in}}%
\pgfpathlineto{\pgfqpoint{2.426893in}{1.891220in}}%
\pgfpathlineto{\pgfqpoint{2.479937in}{1.891220in}}%
\pgfpathlineto{\pgfqpoint{2.532981in}{1.891220in}}%
\pgfpathlineto{\pgfqpoint{2.586025in}{1.891220in}}%
\pgfpathlineto{\pgfqpoint{2.639069in}{1.891220in}}%
\pgfpathlineto{\pgfqpoint{2.692113in}{1.891220in}}%
\pgfpathlineto{\pgfqpoint{2.745157in}{1.891220in}}%
\pgfpathlineto{\pgfqpoint{2.798201in}{1.891220in}}%
\pgfpathlineto{\pgfqpoint{2.851245in}{1.891220in}}%
\pgfpathlineto{\pgfqpoint{2.904288in}{1.891220in}}%
\pgfpathlineto{\pgfqpoint{2.957332in}{1.891220in}}%
\pgfpathlineto{\pgfqpoint{3.010376in}{1.891220in}}%
\pgfpathlineto{\pgfqpoint{3.063420in}{1.891220in}}%
\pgfpathlineto{\pgfqpoint{3.116464in}{1.891220in}}%
\pgfpathlineto{\pgfqpoint{3.169508in}{1.891220in}}%
\pgfpathlineto{\pgfqpoint{3.222552in}{1.891220in}}%
\pgfpathlineto{\pgfqpoint{3.275596in}{1.891220in}}%
\pgfpathlineto{\pgfqpoint{3.328640in}{1.891220in}}%
\pgfpathlineto{\pgfqpoint{3.381684in}{1.891220in}}%
\pgfpathlineto{\pgfqpoint{3.434728in}{1.891220in}}%
\pgfpathlineto{\pgfqpoint{3.487772in}{1.891220in}}%
\pgfpathlineto{\pgfqpoint{3.540816in}{1.891220in}}%
\pgfpathlineto{\pgfqpoint{3.593860in}{1.891220in}}%
\pgfpathlineto{\pgfqpoint{3.646904in}{1.891220in}}%
\pgfpathlineto{\pgfqpoint{3.699948in}{1.891220in}}%
\pgfpathlineto{\pgfqpoint{3.752991in}{1.891220in}}%
\pgfpathlineto{\pgfqpoint{3.806035in}{1.891220in}}%
\pgfpathlineto{\pgfqpoint{3.859079in}{1.891220in}}%
\pgfpathlineto{\pgfqpoint{3.912123in}{1.891220in}}%
\pgfpathlineto{\pgfqpoint{3.965167in}{1.891220in}}%
\pgfpathlineto{\pgfqpoint{4.018211in}{1.891220in}}%
\pgfpathlineto{\pgfqpoint{4.071255in}{1.891220in}}%
\pgfpathlineto{\pgfqpoint{4.124299in}{1.891220in}}%
\pgfpathlineto{\pgfqpoint{4.177343in}{1.891220in}}%
\pgfpathlineto{\pgfqpoint{4.230387in}{1.891220in}}%
\pgfpathlineto{\pgfqpoint{4.283431in}{1.891220in}}%
\pgfpathlineto{\pgfqpoint{4.336475in}{1.891220in}}%
\pgfpathlineto{\pgfqpoint{4.389519in}{1.891220in}}%
\pgfpathlineto{\pgfqpoint{4.442563in}{1.891220in}}%
\pgfpathlineto{\pgfqpoint{4.495607in}{1.891220in}}%
\pgfpathlineto{\pgfqpoint{4.548650in}{1.891220in}}%
\pgfpathlineto{\pgfqpoint{4.601694in}{1.891220in}}%
\pgfpathlineto{\pgfqpoint{4.654738in}{1.891220in}}%
\pgfpathlineto{\pgfqpoint{4.707782in}{1.891220in}}%
\pgfpathlineto{\pgfqpoint{4.760826in}{1.891220in}}%
\pgfpathlineto{\pgfqpoint{4.813870in}{1.891220in}}%
\pgfpathlineto{\pgfqpoint{4.866914in}{1.891220in}}%
\pgfpathlineto{\pgfqpoint{4.919958in}{1.891220in}}%
\pgfpathlineto{\pgfqpoint{4.973002in}{1.891220in}}%
\pgfpathlineto{\pgfqpoint{5.026046in}{1.891220in}}%
\pgfpathlineto{\pgfqpoint{5.079090in}{1.891220in}}%
\pgfpathlineto{\pgfqpoint{5.132134in}{1.891220in}}%
\pgfpathlineto{\pgfqpoint{5.185178in}{1.891220in}}%
\pgfpathlineto{\pgfqpoint{5.238222in}{1.891220in}}%
\pgfpathlineto{\pgfqpoint{5.291266in}{1.891220in}}%
\pgfpathlineto{\pgfqpoint{5.344309in}{1.891220in}}%
\pgfpathlineto{\pgfqpoint{5.397353in}{1.891220in}}%
\pgfpathlineto{\pgfqpoint{5.450397in}{1.891220in}}%
\pgfpathlineto{\pgfqpoint{5.503441in}{1.891220in}}%
\pgfpathlineto{\pgfqpoint{5.556485in}{1.891220in}}%
\pgfpathlineto{\pgfqpoint{5.609529in}{1.891220in}}%
\pgfpathlineto{\pgfqpoint{5.662573in}{1.891220in}}%
\pgfpathlineto{\pgfqpoint{5.715617in}{1.891220in}}%
\pgfpathlineto{\pgfqpoint{5.768661in}{1.891220in}}%
\pgfpathlineto{\pgfqpoint{5.821705in}{1.891220in}}%
\pgfpathlineto{\pgfqpoint{5.874749in}{1.891220in}}%
\pgfpathlineto{\pgfqpoint{5.927793in}{1.891220in}}%
\pgfpathlineto{\pgfqpoint{5.980837in}{1.891220in}}%
\pgfpathlineto{\pgfqpoint{6.033881in}{1.891220in}}%
\pgfpathlineto{\pgfqpoint{6.086925in}{1.891220in}}%
\pgfpathlineto{\pgfqpoint{6.139969in}{1.891220in}}%
\pgfpathlineto{\pgfqpoint{6.193012in}{1.891220in}}%
\pgfpathlineto{\pgfqpoint{6.246056in}{1.891220in}}%
\pgfpathlineto{\pgfqpoint{6.299100in}{1.891220in}}%
\pgfpathlineto{\pgfqpoint{6.352144in}{1.891220in}}%
\pgfpathlineto{\pgfqpoint{6.405188in}{1.891220in}}%
\pgfpathlineto{\pgfqpoint{6.458232in}{1.891220in}}%
\pgfpathlineto{\pgfqpoint{6.511276in}{1.891220in}}%
\pgfpathlineto{\pgfqpoint{6.564320in}{1.891220in}}%
\pgfpathlineto{\pgfqpoint{6.617364in}{1.891220in}}%
\pgfpathlineto{\pgfqpoint{6.670408in}{1.891220in}}%
\pgfpathlineto{\pgfqpoint{6.723452in}{1.891220in}}%
\pgfpathlineto{\pgfqpoint{6.776496in}{1.891220in}}%
\pgfpathlineto{\pgfqpoint{6.829540in}{1.891220in}}%
\pgfpathlineto{\pgfqpoint{6.882584in}{1.891220in}}%
\pgfpathlineto{\pgfqpoint{6.935628in}{1.891220in}}%
\pgfpathlineto{\pgfqpoint{6.988671in}{1.891220in}}%
\pgfpathlineto{\pgfqpoint{7.041715in}{1.891220in}}%
\pgfpathlineto{\pgfqpoint{7.094759in}{1.891220in}}%
\pgfpathlineto{\pgfqpoint{7.147803in}{1.891220in}}%
\pgfpathlineto{\pgfqpoint{7.200847in}{1.891220in}}%
\pgfpathlineto{\pgfqpoint{7.253891in}{1.891220in}}%
\pgfpathlineto{\pgfqpoint{7.306935in}{1.891220in}}%
\pgfpathlineto{\pgfqpoint{7.359979in}{1.891220in}}%
\pgfpathlineto{\pgfqpoint{7.413023in}{1.891220in}}%
\pgfpathlineto{\pgfqpoint{7.466067in}{1.891220in}}%
\pgfpathlineto{\pgfqpoint{7.519111in}{1.891220in}}%
\pgfpathlineto{\pgfqpoint{7.572155in}{1.891220in}}%
\pgfpathlineto{\pgfqpoint{7.625199in}{1.891220in}}%
\pgfpathlineto{\pgfqpoint{7.678243in}{1.891220in}}%
\pgfpathlineto{\pgfqpoint{7.731287in}{1.891220in}}%
\pgfpathlineto{\pgfqpoint{7.784330in}{1.891220in}}%
\pgfpathlineto{\pgfqpoint{7.837374in}{1.891220in}}%
\pgfpathlineto{\pgfqpoint{7.890418in}{1.891220in}}%
\pgfpathlineto{\pgfqpoint{7.943462in}{1.891220in}}%
\pgfpathlineto{\pgfqpoint{7.996506in}{1.891220in}}%
\pgfpathlineto{\pgfqpoint{8.049550in}{1.891220in}}%
\pgfpathlineto{\pgfqpoint{8.102594in}{1.891220in}}%
\pgfpathlineto{\pgfqpoint{8.155638in}{1.891220in}}%
\pgfpathlineto{\pgfqpoint{8.208682in}{1.891220in}}%
\pgfpathlineto{\pgfqpoint{8.261726in}{1.891220in}}%
\pgfpathlineto{\pgfqpoint{8.314770in}{1.891220in}}%
\pgfpathlineto{\pgfqpoint{8.367814in}{1.891220in}}%
\pgfpathlineto{\pgfqpoint{8.420858in}{1.891220in}}%
\pgfpathlineto{\pgfqpoint{8.473902in}{1.891220in}}%
\pgfpathlineto{\pgfqpoint{8.526946in}{1.891220in}}%
\pgfpathlineto{\pgfqpoint{8.579990in}{1.891220in}}%
\pgfpathlineto{\pgfqpoint{8.633033in}{1.891220in}}%
\pgfpathlineto{\pgfqpoint{8.686077in}{1.891220in}}%
\pgfpathlineto{\pgfqpoint{8.739121in}{1.891220in}}%
\pgfpathlineto{\pgfqpoint{8.792165in}{1.891220in}}%
\pgfpathlineto{\pgfqpoint{8.845209in}{1.891220in}}%
\pgfpathlineto{\pgfqpoint{8.898253in}{1.891220in}}%
\pgfpathlineto{\pgfqpoint{8.951297in}{1.891220in}}%
\pgfpathlineto{\pgfqpoint{9.004341in}{1.891220in}}%
\pgfpathlineto{\pgfqpoint{9.057385in}{1.891220in}}%
\pgfpathlineto{\pgfqpoint{9.110429in}{1.891220in}}%
\pgfpathlineto{\pgfqpoint{9.163473in}{1.891220in}}%
\pgfpathlineto{\pgfqpoint{9.216517in}{1.891220in}}%
\pgfpathlineto{\pgfqpoint{9.269561in}{1.891220in}}%
\pgfpathlineto{\pgfqpoint{9.322605in}{1.891220in}}%
\pgfpathlineto{\pgfqpoint{9.375649in}{1.891220in}}%
\pgfpathlineto{\pgfqpoint{9.428692in}{1.891220in}}%
\pgfpathlineto{\pgfqpoint{9.481736in}{1.891220in}}%
\pgfpathlineto{\pgfqpoint{9.534780in}{1.891220in}}%
\pgfpathlineto{\pgfqpoint{9.587824in}{1.891220in}}%
\pgfpathlineto{\pgfqpoint{9.640868in}{1.891220in}}%
\pgfpathlineto{\pgfqpoint{9.693912in}{1.891220in}}%
\pgfpathlineto{\pgfqpoint{9.746956in}{1.891220in}}%
\pgfpathlineto{\pgfqpoint{9.800000in}{1.891220in}}%
\pgfpathlineto{\pgfqpoint{9.800000in}{2.586892in}}%
\pgfpathlineto{\pgfqpoint{9.800000in}{2.586892in}}%
\pgfpathlineto{\pgfqpoint{9.746956in}{2.586892in}}%
\pgfpathlineto{\pgfqpoint{9.693912in}{2.586892in}}%
\pgfpathlineto{\pgfqpoint{9.640868in}{2.586892in}}%
\pgfpathlineto{\pgfqpoint{9.587824in}{2.586892in}}%
\pgfpathlineto{\pgfqpoint{9.534780in}{2.586892in}}%
\pgfpathlineto{\pgfqpoint{9.481736in}{2.586892in}}%
\pgfpathlineto{\pgfqpoint{9.428692in}{2.586892in}}%
\pgfpathlineto{\pgfqpoint{9.375649in}{2.586892in}}%
\pgfpathlineto{\pgfqpoint{9.322605in}{2.586892in}}%
\pgfpathlineto{\pgfqpoint{9.269561in}{2.586892in}}%
\pgfpathlineto{\pgfqpoint{9.216517in}{2.586892in}}%
\pgfpathlineto{\pgfqpoint{9.163473in}{2.586892in}}%
\pgfpathlineto{\pgfqpoint{9.110429in}{2.586892in}}%
\pgfpathlineto{\pgfqpoint{9.057385in}{2.586892in}}%
\pgfpathlineto{\pgfqpoint{9.004341in}{2.586892in}}%
\pgfpathlineto{\pgfqpoint{8.951297in}{2.586892in}}%
\pgfpathlineto{\pgfqpoint{8.898253in}{2.586892in}}%
\pgfpathlineto{\pgfqpoint{8.845209in}{2.586892in}}%
\pgfpathlineto{\pgfqpoint{8.792165in}{2.586892in}}%
\pgfpathlineto{\pgfqpoint{8.739121in}{2.586892in}}%
\pgfpathlineto{\pgfqpoint{8.686077in}{2.586892in}}%
\pgfpathlineto{\pgfqpoint{8.633033in}{2.586892in}}%
\pgfpathlineto{\pgfqpoint{8.579990in}{2.586892in}}%
\pgfpathlineto{\pgfqpoint{8.526946in}{2.586892in}}%
\pgfpathlineto{\pgfqpoint{8.473902in}{2.586892in}}%
\pgfpathlineto{\pgfqpoint{8.420858in}{2.586892in}}%
\pgfpathlineto{\pgfqpoint{8.367814in}{2.586892in}}%
\pgfpathlineto{\pgfqpoint{8.314770in}{2.586892in}}%
\pgfpathlineto{\pgfqpoint{8.261726in}{2.586892in}}%
\pgfpathlineto{\pgfqpoint{8.208682in}{2.586892in}}%
\pgfpathlineto{\pgfqpoint{8.155638in}{2.586892in}}%
\pgfpathlineto{\pgfqpoint{8.102594in}{2.586892in}}%
\pgfpathlineto{\pgfqpoint{8.049550in}{2.586892in}}%
\pgfpathlineto{\pgfqpoint{7.996506in}{2.586892in}}%
\pgfpathlineto{\pgfqpoint{7.943462in}{2.586892in}}%
\pgfpathlineto{\pgfqpoint{7.890418in}{2.586892in}}%
\pgfpathlineto{\pgfqpoint{7.837374in}{2.586892in}}%
\pgfpathlineto{\pgfqpoint{7.784330in}{2.586892in}}%
\pgfpathlineto{\pgfqpoint{7.731287in}{2.586892in}}%
\pgfpathlineto{\pgfqpoint{7.678243in}{2.586892in}}%
\pgfpathlineto{\pgfqpoint{7.625199in}{2.586892in}}%
\pgfpathlineto{\pgfqpoint{7.572155in}{2.586892in}}%
\pgfpathlineto{\pgfqpoint{7.519111in}{2.586892in}}%
\pgfpathlineto{\pgfqpoint{7.466067in}{2.586892in}}%
\pgfpathlineto{\pgfqpoint{7.413023in}{2.586892in}}%
\pgfpathlineto{\pgfqpoint{7.359979in}{2.586892in}}%
\pgfpathlineto{\pgfqpoint{7.306935in}{2.586892in}}%
\pgfpathlineto{\pgfqpoint{7.253891in}{2.586892in}}%
\pgfpathlineto{\pgfqpoint{7.200847in}{2.586892in}}%
\pgfpathlineto{\pgfqpoint{7.147803in}{2.586892in}}%
\pgfpathlineto{\pgfqpoint{7.094759in}{2.586892in}}%
\pgfpathlineto{\pgfqpoint{7.041715in}{2.586892in}}%
\pgfpathlineto{\pgfqpoint{6.988671in}{2.586892in}}%
\pgfpathlineto{\pgfqpoint{6.935628in}{2.586892in}}%
\pgfpathlineto{\pgfqpoint{6.882584in}{2.586892in}}%
\pgfpathlineto{\pgfqpoint{6.829540in}{2.586892in}}%
\pgfpathlineto{\pgfqpoint{6.776496in}{2.586892in}}%
\pgfpathlineto{\pgfqpoint{6.723452in}{2.586892in}}%
\pgfpathlineto{\pgfqpoint{6.670408in}{2.586892in}}%
\pgfpathlineto{\pgfqpoint{6.617364in}{2.586892in}}%
\pgfpathlineto{\pgfqpoint{6.564320in}{2.586892in}}%
\pgfpathlineto{\pgfqpoint{6.511276in}{2.586892in}}%
\pgfpathlineto{\pgfqpoint{6.458232in}{2.586892in}}%
\pgfpathlineto{\pgfqpoint{6.405188in}{2.586892in}}%
\pgfpathlineto{\pgfqpoint{6.352144in}{2.586892in}}%
\pgfpathlineto{\pgfqpoint{6.299100in}{2.586892in}}%
\pgfpathlineto{\pgfqpoint{6.246056in}{2.586892in}}%
\pgfpathlineto{\pgfqpoint{6.193012in}{2.586892in}}%
\pgfpathlineto{\pgfqpoint{6.139969in}{2.586892in}}%
\pgfpathlineto{\pgfqpoint{6.086925in}{2.586892in}}%
\pgfpathlineto{\pgfqpoint{6.033881in}{2.586892in}}%
\pgfpathlineto{\pgfqpoint{5.980837in}{2.586892in}}%
\pgfpathlineto{\pgfqpoint{5.927793in}{2.586892in}}%
\pgfpathlineto{\pgfqpoint{5.874749in}{2.586892in}}%
\pgfpathlineto{\pgfqpoint{5.821705in}{2.586892in}}%
\pgfpathlineto{\pgfqpoint{5.768661in}{2.586892in}}%
\pgfpathlineto{\pgfqpoint{5.715617in}{2.586892in}}%
\pgfpathlineto{\pgfqpoint{5.662573in}{2.586892in}}%
\pgfpathlineto{\pgfqpoint{5.609529in}{2.586892in}}%
\pgfpathlineto{\pgfqpoint{5.556485in}{2.586892in}}%
\pgfpathlineto{\pgfqpoint{5.503441in}{2.586892in}}%
\pgfpathlineto{\pgfqpoint{5.450397in}{2.586892in}}%
\pgfpathlineto{\pgfqpoint{5.397353in}{2.586892in}}%
\pgfpathlineto{\pgfqpoint{5.344309in}{2.586892in}}%
\pgfpathlineto{\pgfqpoint{5.291266in}{2.586892in}}%
\pgfpathlineto{\pgfqpoint{5.238222in}{2.586892in}}%
\pgfpathlineto{\pgfqpoint{5.185178in}{2.586892in}}%
\pgfpathlineto{\pgfqpoint{5.132134in}{2.586892in}}%
\pgfpathlineto{\pgfqpoint{5.079090in}{2.586892in}}%
\pgfpathlineto{\pgfqpoint{5.026046in}{2.586892in}}%
\pgfpathlineto{\pgfqpoint{4.973002in}{2.586892in}}%
\pgfpathlineto{\pgfqpoint{4.919958in}{2.586892in}}%
\pgfpathlineto{\pgfqpoint{4.866914in}{2.586892in}}%
\pgfpathlineto{\pgfqpoint{4.813870in}{2.586892in}}%
\pgfpathlineto{\pgfqpoint{4.760826in}{2.586892in}}%
\pgfpathlineto{\pgfqpoint{4.707782in}{2.586892in}}%
\pgfpathlineto{\pgfqpoint{4.654738in}{2.586892in}}%
\pgfpathlineto{\pgfqpoint{4.601694in}{2.586892in}}%
\pgfpathlineto{\pgfqpoint{4.548650in}{2.586892in}}%
\pgfpathlineto{\pgfqpoint{4.495607in}{2.586892in}}%
\pgfpathlineto{\pgfqpoint{4.442563in}{2.586892in}}%
\pgfpathlineto{\pgfqpoint{4.389519in}{2.586892in}}%
\pgfpathlineto{\pgfqpoint{4.336475in}{2.586892in}}%
\pgfpathlineto{\pgfqpoint{4.283431in}{2.586892in}}%
\pgfpathlineto{\pgfqpoint{4.230387in}{2.586892in}}%
\pgfpathlineto{\pgfqpoint{4.177343in}{2.586892in}}%
\pgfpathlineto{\pgfqpoint{4.124299in}{2.586892in}}%
\pgfpathlineto{\pgfqpoint{4.071255in}{2.586892in}}%
\pgfpathlineto{\pgfqpoint{4.018211in}{2.586892in}}%
\pgfpathlineto{\pgfqpoint{3.965167in}{2.586892in}}%
\pgfpathlineto{\pgfqpoint{3.912123in}{2.586892in}}%
\pgfpathlineto{\pgfqpoint{3.859079in}{2.586892in}}%
\pgfpathlineto{\pgfqpoint{3.806035in}{2.586892in}}%
\pgfpathlineto{\pgfqpoint{3.752991in}{2.586892in}}%
\pgfpathlineto{\pgfqpoint{3.699948in}{2.586892in}}%
\pgfpathlineto{\pgfqpoint{3.646904in}{2.586892in}}%
\pgfpathlineto{\pgfqpoint{3.593860in}{2.586892in}}%
\pgfpathlineto{\pgfqpoint{3.540816in}{2.586892in}}%
\pgfpathlineto{\pgfqpoint{3.487772in}{2.586892in}}%
\pgfpathlineto{\pgfqpoint{3.434728in}{2.586892in}}%
\pgfpathlineto{\pgfqpoint{3.381684in}{2.586892in}}%
\pgfpathlineto{\pgfqpoint{3.328640in}{2.586892in}}%
\pgfpathlineto{\pgfqpoint{3.275596in}{2.586892in}}%
\pgfpathlineto{\pgfqpoint{3.222552in}{2.586892in}}%
\pgfpathlineto{\pgfqpoint{3.169508in}{2.586892in}}%
\pgfpathlineto{\pgfqpoint{3.116464in}{2.586892in}}%
\pgfpathlineto{\pgfqpoint{3.063420in}{2.586892in}}%
\pgfpathlineto{\pgfqpoint{3.010376in}{2.586892in}}%
\pgfpathlineto{\pgfqpoint{2.957332in}{2.586892in}}%
\pgfpathlineto{\pgfqpoint{2.904288in}{2.586892in}}%
\pgfpathlineto{\pgfqpoint{2.851245in}{2.586892in}}%
\pgfpathlineto{\pgfqpoint{2.798201in}{2.586892in}}%
\pgfpathlineto{\pgfqpoint{2.745157in}{2.586892in}}%
\pgfpathlineto{\pgfqpoint{2.692113in}{2.586892in}}%
\pgfpathlineto{\pgfqpoint{2.639069in}{2.586892in}}%
\pgfpathlineto{\pgfqpoint{2.586025in}{2.586892in}}%
\pgfpathlineto{\pgfqpoint{2.532981in}{2.586892in}}%
\pgfpathlineto{\pgfqpoint{2.479937in}{2.586892in}}%
\pgfpathlineto{\pgfqpoint{2.426893in}{2.586892in}}%
\pgfpathlineto{\pgfqpoint{2.373849in}{2.586892in}}%
\pgfpathlineto{\pgfqpoint{2.320805in}{2.586892in}}%
\pgfpathlineto{\pgfqpoint{2.267761in}{2.586892in}}%
\pgfpathlineto{\pgfqpoint{2.214717in}{2.586892in}}%
\pgfpathlineto{\pgfqpoint{2.161673in}{2.586892in}}%
\pgfpathlineto{\pgfqpoint{2.108629in}{2.586892in}}%
\pgfpathlineto{\pgfqpoint{2.055586in}{2.586892in}}%
\pgfpathlineto{\pgfqpoint{2.002542in}{2.586892in}}%
\pgfpathlineto{\pgfqpoint{1.949498in}{2.586892in}}%
\pgfpathlineto{\pgfqpoint{1.896454in}{2.586892in}}%
\pgfpathlineto{\pgfqpoint{1.843410in}{2.586892in}}%
\pgfpathlineto{\pgfqpoint{1.790366in}{2.586892in}}%
\pgfpathlineto{\pgfqpoint{1.737322in}{2.586892in}}%
\pgfpathlineto{\pgfqpoint{1.684278in}{2.586892in}}%
\pgfpathlineto{\pgfqpoint{1.631234in}{2.586892in}}%
\pgfpathlineto{\pgfqpoint{1.578190in}{2.586892in}}%
\pgfpathlineto{\pgfqpoint{1.525146in}{2.586892in}}%
\pgfpathlineto{\pgfqpoint{1.472102in}{2.586892in}}%
\pgfpathlineto{\pgfqpoint{1.419058in}{2.586892in}}%
\pgfpathlineto{\pgfqpoint{1.366014in}{2.586892in}}%
\pgfpathlineto{\pgfqpoint{1.312970in}{2.586892in}}%
\pgfpathlineto{\pgfqpoint{1.259927in}{2.586892in}}%
\pgfpathlineto{\pgfqpoint{1.206883in}{2.586892in}}%
\pgfpathlineto{\pgfqpoint{1.153839in}{2.586892in}}%
\pgfpathlineto{\pgfqpoint{1.100795in}{2.586892in}}%
\pgfpathlineto{\pgfqpoint{1.047751in}{2.586892in}}%
\pgfpathlineto{\pgfqpoint{0.994707in}{2.586892in}}%
\pgfpathlineto{\pgfqpoint{0.941663in}{2.586892in}}%
\pgfpathlineto{\pgfqpoint{0.941663in}{2.586892in}}%
\pgfpathclose%
\pgfusepath{stroke,fill}%
}%
\begin{pgfscope}%
\pgfsys@transformshift{0.000000in}{0.000000in}%
\pgfsys@useobject{currentmarker}{}%
\end{pgfscope}%
\end{pgfscope}%
\begin{pgfscope}%
\pgfpathrectangle{\pgfqpoint{0.941663in}{0.670138in}}{\pgfqpoint{8.858337in}{3.465625in}}%
\pgfusepath{clip}%
\pgfsetrectcap%
\pgfsetroundjoin%
\pgfsetlinewidth{1.505625pt}%
\definecolor{currentstroke}{rgb}{0.501961,0.000000,0.501961}%
\pgfsetstrokecolor{currentstroke}%
\pgfsetdash{}{0pt}%
\pgfpathmoveto{\pgfqpoint{0.941663in}{2.586892in}}%
\pgfpathlineto{\pgfqpoint{1.206883in}{2.586892in}}%
\pgfpathlineto{\pgfqpoint{1.259927in}{2.749185in}}%
\pgfpathlineto{\pgfqpoint{1.312970in}{2.586892in}}%
\pgfpathlineto{\pgfqpoint{1.578190in}{2.586892in}}%
\pgfpathlineto{\pgfqpoint{1.631234in}{2.802326in}}%
\pgfpathlineto{\pgfqpoint{1.684278in}{2.586892in}}%
\pgfpathlineto{\pgfqpoint{1.790366in}{2.586892in}}%
\pgfpathlineto{\pgfqpoint{1.843410in}{2.934727in}}%
\pgfpathlineto{\pgfqpoint{1.896454in}{2.934727in}}%
\pgfpathlineto{\pgfqpoint{1.949498in}{2.586892in}}%
\pgfpathlineto{\pgfqpoint{2.002542in}{2.934727in}}%
\pgfpathlineto{\pgfqpoint{2.055586in}{2.586892in}}%
\pgfpathlineto{\pgfqpoint{2.161673in}{2.586892in}}%
\pgfpathlineto{\pgfqpoint{2.214717in}{2.910735in}}%
\pgfpathlineto{\pgfqpoint{2.267761in}{2.586892in}}%
\pgfpathlineto{\pgfqpoint{2.320805in}{2.586892in}}%
\pgfpathlineto{\pgfqpoint{2.373849in}{2.844885in}}%
\pgfpathlineto{\pgfqpoint{2.426893in}{2.934727in}}%
\pgfpathlineto{\pgfqpoint{2.479937in}{2.586892in}}%
\pgfpathlineto{\pgfqpoint{2.904288in}{2.586892in}}%
\pgfpathlineto{\pgfqpoint{2.957332in}{2.934727in}}%
\pgfpathlineto{\pgfqpoint{3.010376in}{2.586892in}}%
\pgfpathlineto{\pgfqpoint{3.063420in}{2.909016in}}%
\pgfpathlineto{\pgfqpoint{3.116464in}{2.586892in}}%
\pgfpathlineto{\pgfqpoint{3.434728in}{2.586892in}}%
\pgfpathlineto{\pgfqpoint{3.487772in}{2.710315in}}%
\pgfpathlineto{\pgfqpoint{3.540816in}{2.908402in}}%
\pgfpathlineto{\pgfqpoint{3.593860in}{2.586892in}}%
\pgfpathlineto{\pgfqpoint{4.283431in}{2.586892in}}%
\pgfpathlineto{\pgfqpoint{4.336475in}{2.934727in}}%
\pgfpathlineto{\pgfqpoint{4.389519in}{2.586892in}}%
\pgfpathlineto{\pgfqpoint{4.442563in}{2.586892in}}%
\pgfpathlineto{\pgfqpoint{4.495607in}{2.934727in}}%
\pgfpathlineto{\pgfqpoint{4.548650in}{2.586892in}}%
\pgfpathlineto{\pgfqpoint{4.601694in}{2.927863in}}%
\pgfpathlineto{\pgfqpoint{4.654738in}{2.586892in}}%
\pgfpathlineto{\pgfqpoint{4.707782in}{2.934727in}}%
\pgfpathlineto{\pgfqpoint{4.760826in}{2.586892in}}%
\pgfpathlineto{\pgfqpoint{4.813870in}{2.586892in}}%
\pgfpathlineto{\pgfqpoint{4.866914in}{2.884645in}}%
\pgfpathlineto{\pgfqpoint{4.919958in}{2.863950in}}%
\pgfpathlineto{\pgfqpoint{4.973002in}{2.586892in}}%
\pgfpathlineto{\pgfqpoint{5.026046in}{2.758322in}}%
\pgfpathlineto{\pgfqpoint{5.079090in}{2.586892in}}%
\pgfpathlineto{\pgfqpoint{5.132134in}{2.824147in}}%
\pgfpathlineto{\pgfqpoint{5.185178in}{2.910114in}}%
\pgfpathlineto{\pgfqpoint{5.238222in}{2.837414in}}%
\pgfpathlineto{\pgfqpoint{5.291266in}{2.586892in}}%
\pgfpathlineto{\pgfqpoint{5.344309in}{2.705202in}}%
\pgfpathlineto{\pgfqpoint{5.397353in}{2.660769in}}%
\pgfpathlineto{\pgfqpoint{5.450397in}{2.934727in}}%
\pgfpathlineto{\pgfqpoint{5.503441in}{2.586892in}}%
\pgfpathlineto{\pgfqpoint{5.556485in}{2.586892in}}%
\pgfpathlineto{\pgfqpoint{5.609529in}{2.850209in}}%
\pgfpathlineto{\pgfqpoint{5.662573in}{2.711726in}}%
\pgfpathlineto{\pgfqpoint{5.715617in}{2.586892in}}%
\pgfpathlineto{\pgfqpoint{5.768661in}{2.934727in}}%
\pgfpathlineto{\pgfqpoint{5.874749in}{2.934727in}}%
\pgfpathlineto{\pgfqpoint{5.927793in}{2.643205in}}%
\pgfpathlineto{\pgfqpoint{5.980837in}{2.586892in}}%
\pgfpathlineto{\pgfqpoint{6.139969in}{2.586892in}}%
\pgfpathlineto{\pgfqpoint{6.193012in}{2.934727in}}%
\pgfpathlineto{\pgfqpoint{6.246056in}{2.934727in}}%
\pgfpathlineto{\pgfqpoint{6.299100in}{2.586892in}}%
\pgfpathlineto{\pgfqpoint{6.352144in}{2.586892in}}%
\pgfpathlineto{\pgfqpoint{6.405188in}{2.699818in}}%
\pgfpathlineto{\pgfqpoint{6.458232in}{2.586892in}}%
\pgfpathlineto{\pgfqpoint{6.829540in}{2.586892in}}%
\pgfpathlineto{\pgfqpoint{6.882584in}{2.934727in}}%
\pgfpathlineto{\pgfqpoint{6.935628in}{2.586892in}}%
\pgfpathlineto{\pgfqpoint{7.147803in}{2.586892in}}%
\pgfpathlineto{\pgfqpoint{7.200847in}{2.934727in}}%
\pgfpathlineto{\pgfqpoint{7.253891in}{2.586892in}}%
\pgfpathlineto{\pgfqpoint{7.306935in}{2.586892in}}%
\pgfpathlineto{\pgfqpoint{7.359979in}{2.806406in}}%
\pgfpathlineto{\pgfqpoint{7.413023in}{2.865454in}}%
\pgfpathlineto{\pgfqpoint{7.466067in}{2.586892in}}%
\pgfpathlineto{\pgfqpoint{7.519111in}{2.586892in}}%
\pgfpathlineto{\pgfqpoint{7.572155in}{2.823725in}}%
\pgfpathlineto{\pgfqpoint{7.625199in}{2.934727in}}%
\pgfpathlineto{\pgfqpoint{7.678243in}{2.586892in}}%
\pgfpathlineto{\pgfqpoint{7.996506in}{2.586892in}}%
\pgfpathlineto{\pgfqpoint{8.049550in}{2.934727in}}%
\pgfpathlineto{\pgfqpoint{8.102594in}{2.934727in}}%
\pgfpathlineto{\pgfqpoint{8.155638in}{2.586892in}}%
\pgfpathlineto{\pgfqpoint{8.208682in}{2.586892in}}%
\pgfpathlineto{\pgfqpoint{8.261726in}{2.934727in}}%
\pgfpathlineto{\pgfqpoint{8.314770in}{2.934727in}}%
\pgfpathlineto{\pgfqpoint{8.367814in}{2.586892in}}%
\pgfpathlineto{\pgfqpoint{8.420858in}{2.934727in}}%
\pgfpathlineto{\pgfqpoint{8.473902in}{2.586892in}}%
\pgfpathlineto{\pgfqpoint{8.526946in}{2.586892in}}%
\pgfpathlineto{\pgfqpoint{8.579990in}{2.934727in}}%
\pgfpathlineto{\pgfqpoint{8.633033in}{2.856358in}}%
\pgfpathlineto{\pgfqpoint{8.686077in}{2.934727in}}%
\pgfpathlineto{\pgfqpoint{8.739121in}{2.825404in}}%
\pgfpathlineto{\pgfqpoint{8.792165in}{2.586892in}}%
\pgfpathlineto{\pgfqpoint{8.845209in}{2.634872in}}%
\pgfpathlineto{\pgfqpoint{8.898253in}{2.689958in}}%
\pgfpathlineto{\pgfqpoint{8.951297in}{2.586892in}}%
\pgfpathlineto{\pgfqpoint{9.110429in}{2.586892in}}%
\pgfpathlineto{\pgfqpoint{9.163473in}{2.934727in}}%
\pgfpathlineto{\pgfqpoint{9.322605in}{2.934727in}}%
\pgfpathlineto{\pgfqpoint{9.375649in}{2.695480in}}%
\pgfpathlineto{\pgfqpoint{9.428692in}{2.586892in}}%
\pgfpathlineto{\pgfqpoint{9.481736in}{2.586892in}}%
\pgfpathlineto{\pgfqpoint{9.534780in}{2.934727in}}%
\pgfpathlineto{\pgfqpoint{9.587824in}{2.916055in}}%
\pgfpathlineto{\pgfqpoint{9.640868in}{2.586892in}}%
\pgfpathlineto{\pgfqpoint{9.693912in}{2.865140in}}%
\pgfpathlineto{\pgfqpoint{9.746956in}{2.608608in}}%
\pgfpathlineto{\pgfqpoint{9.800000in}{2.588586in}}%
\pgfpathlineto{\pgfqpoint{9.800000in}{2.588586in}}%
\pgfusepath{stroke}%
\end{pgfscope}%
\begin{pgfscope}%
\pgfpathrectangle{\pgfqpoint{0.941663in}{0.670138in}}{\pgfqpoint{8.858337in}{3.465625in}}%
\pgfusepath{clip}%
\pgfsetbuttcap%
\pgfsetroundjoin%
\definecolor{currentfill}{rgb}{0.501961,0.000000,0.501961}%
\pgfsetfillcolor{currentfill}%
\pgfsetlinewidth{1.003750pt}%
\definecolor{currentstroke}{rgb}{0.501961,0.000000,0.501961}%
\pgfsetstrokecolor{currentstroke}%
\pgfsetdash{}{0pt}%
\pgfsys@defobject{currentmarker}{\pgfqpoint{0.941663in}{2.586892in}}{\pgfqpoint{9.800000in}{2.934727in}}{%
\pgfpathmoveto{\pgfqpoint{0.941663in}{2.586892in}}%
\pgfpathlineto{\pgfqpoint{0.941663in}{2.586892in}}%
\pgfpathlineto{\pgfqpoint{0.994707in}{2.586892in}}%
\pgfpathlineto{\pgfqpoint{1.047751in}{2.586892in}}%
\pgfpathlineto{\pgfqpoint{1.100795in}{2.586892in}}%
\pgfpathlineto{\pgfqpoint{1.153839in}{2.586892in}}%
\pgfpathlineto{\pgfqpoint{1.206883in}{2.586892in}}%
\pgfpathlineto{\pgfqpoint{1.259927in}{2.586892in}}%
\pgfpathlineto{\pgfqpoint{1.312970in}{2.586892in}}%
\pgfpathlineto{\pgfqpoint{1.366014in}{2.586892in}}%
\pgfpathlineto{\pgfqpoint{1.419058in}{2.586892in}}%
\pgfpathlineto{\pgfqpoint{1.472102in}{2.586892in}}%
\pgfpathlineto{\pgfqpoint{1.525146in}{2.586892in}}%
\pgfpathlineto{\pgfqpoint{1.578190in}{2.586892in}}%
\pgfpathlineto{\pgfqpoint{1.631234in}{2.586892in}}%
\pgfpathlineto{\pgfqpoint{1.684278in}{2.586892in}}%
\pgfpathlineto{\pgfqpoint{1.737322in}{2.586892in}}%
\pgfpathlineto{\pgfqpoint{1.790366in}{2.586892in}}%
\pgfpathlineto{\pgfqpoint{1.843410in}{2.586892in}}%
\pgfpathlineto{\pgfqpoint{1.896454in}{2.586892in}}%
\pgfpathlineto{\pgfqpoint{1.949498in}{2.586892in}}%
\pgfpathlineto{\pgfqpoint{2.002542in}{2.586892in}}%
\pgfpathlineto{\pgfqpoint{2.055586in}{2.586892in}}%
\pgfpathlineto{\pgfqpoint{2.108629in}{2.586892in}}%
\pgfpathlineto{\pgfqpoint{2.161673in}{2.586892in}}%
\pgfpathlineto{\pgfqpoint{2.214717in}{2.586892in}}%
\pgfpathlineto{\pgfqpoint{2.267761in}{2.586892in}}%
\pgfpathlineto{\pgfqpoint{2.320805in}{2.586892in}}%
\pgfpathlineto{\pgfqpoint{2.373849in}{2.586892in}}%
\pgfpathlineto{\pgfqpoint{2.426893in}{2.586892in}}%
\pgfpathlineto{\pgfqpoint{2.479937in}{2.586892in}}%
\pgfpathlineto{\pgfqpoint{2.532981in}{2.586892in}}%
\pgfpathlineto{\pgfqpoint{2.586025in}{2.586892in}}%
\pgfpathlineto{\pgfqpoint{2.639069in}{2.586892in}}%
\pgfpathlineto{\pgfqpoint{2.692113in}{2.586892in}}%
\pgfpathlineto{\pgfqpoint{2.745157in}{2.586892in}}%
\pgfpathlineto{\pgfqpoint{2.798201in}{2.586892in}}%
\pgfpathlineto{\pgfqpoint{2.851245in}{2.586892in}}%
\pgfpathlineto{\pgfqpoint{2.904288in}{2.586892in}}%
\pgfpathlineto{\pgfqpoint{2.957332in}{2.586892in}}%
\pgfpathlineto{\pgfqpoint{3.010376in}{2.586892in}}%
\pgfpathlineto{\pgfqpoint{3.063420in}{2.586892in}}%
\pgfpathlineto{\pgfqpoint{3.116464in}{2.586892in}}%
\pgfpathlineto{\pgfqpoint{3.169508in}{2.586892in}}%
\pgfpathlineto{\pgfqpoint{3.222552in}{2.586892in}}%
\pgfpathlineto{\pgfqpoint{3.275596in}{2.586892in}}%
\pgfpathlineto{\pgfqpoint{3.328640in}{2.586892in}}%
\pgfpathlineto{\pgfqpoint{3.381684in}{2.586892in}}%
\pgfpathlineto{\pgfqpoint{3.434728in}{2.586892in}}%
\pgfpathlineto{\pgfqpoint{3.487772in}{2.586892in}}%
\pgfpathlineto{\pgfqpoint{3.540816in}{2.586892in}}%
\pgfpathlineto{\pgfqpoint{3.593860in}{2.586892in}}%
\pgfpathlineto{\pgfqpoint{3.646904in}{2.586892in}}%
\pgfpathlineto{\pgfqpoint{3.699948in}{2.586892in}}%
\pgfpathlineto{\pgfqpoint{3.752991in}{2.586892in}}%
\pgfpathlineto{\pgfqpoint{3.806035in}{2.586892in}}%
\pgfpathlineto{\pgfqpoint{3.859079in}{2.586892in}}%
\pgfpathlineto{\pgfqpoint{3.912123in}{2.586892in}}%
\pgfpathlineto{\pgfqpoint{3.965167in}{2.586892in}}%
\pgfpathlineto{\pgfqpoint{4.018211in}{2.586892in}}%
\pgfpathlineto{\pgfqpoint{4.071255in}{2.586892in}}%
\pgfpathlineto{\pgfqpoint{4.124299in}{2.586892in}}%
\pgfpathlineto{\pgfqpoint{4.177343in}{2.586892in}}%
\pgfpathlineto{\pgfqpoint{4.230387in}{2.586892in}}%
\pgfpathlineto{\pgfqpoint{4.283431in}{2.586892in}}%
\pgfpathlineto{\pgfqpoint{4.336475in}{2.586892in}}%
\pgfpathlineto{\pgfqpoint{4.389519in}{2.586892in}}%
\pgfpathlineto{\pgfqpoint{4.442563in}{2.586892in}}%
\pgfpathlineto{\pgfqpoint{4.495607in}{2.586892in}}%
\pgfpathlineto{\pgfqpoint{4.548650in}{2.586892in}}%
\pgfpathlineto{\pgfqpoint{4.601694in}{2.586892in}}%
\pgfpathlineto{\pgfqpoint{4.654738in}{2.586892in}}%
\pgfpathlineto{\pgfqpoint{4.707782in}{2.586892in}}%
\pgfpathlineto{\pgfqpoint{4.760826in}{2.586892in}}%
\pgfpathlineto{\pgfqpoint{4.813870in}{2.586892in}}%
\pgfpathlineto{\pgfqpoint{4.866914in}{2.586892in}}%
\pgfpathlineto{\pgfqpoint{4.919958in}{2.586892in}}%
\pgfpathlineto{\pgfqpoint{4.973002in}{2.586892in}}%
\pgfpathlineto{\pgfqpoint{5.026046in}{2.586892in}}%
\pgfpathlineto{\pgfqpoint{5.079090in}{2.586892in}}%
\pgfpathlineto{\pgfqpoint{5.132134in}{2.586892in}}%
\pgfpathlineto{\pgfqpoint{5.185178in}{2.586892in}}%
\pgfpathlineto{\pgfqpoint{5.238222in}{2.586892in}}%
\pgfpathlineto{\pgfqpoint{5.291266in}{2.586892in}}%
\pgfpathlineto{\pgfqpoint{5.344309in}{2.586892in}}%
\pgfpathlineto{\pgfqpoint{5.397353in}{2.586892in}}%
\pgfpathlineto{\pgfqpoint{5.450397in}{2.586892in}}%
\pgfpathlineto{\pgfqpoint{5.503441in}{2.586892in}}%
\pgfpathlineto{\pgfqpoint{5.556485in}{2.586892in}}%
\pgfpathlineto{\pgfqpoint{5.609529in}{2.586892in}}%
\pgfpathlineto{\pgfqpoint{5.662573in}{2.586892in}}%
\pgfpathlineto{\pgfqpoint{5.715617in}{2.586892in}}%
\pgfpathlineto{\pgfqpoint{5.768661in}{2.586892in}}%
\pgfpathlineto{\pgfqpoint{5.821705in}{2.586892in}}%
\pgfpathlineto{\pgfqpoint{5.874749in}{2.586892in}}%
\pgfpathlineto{\pgfqpoint{5.927793in}{2.586892in}}%
\pgfpathlineto{\pgfqpoint{5.980837in}{2.586892in}}%
\pgfpathlineto{\pgfqpoint{6.033881in}{2.586892in}}%
\pgfpathlineto{\pgfqpoint{6.086925in}{2.586892in}}%
\pgfpathlineto{\pgfqpoint{6.139969in}{2.586892in}}%
\pgfpathlineto{\pgfqpoint{6.193012in}{2.586892in}}%
\pgfpathlineto{\pgfqpoint{6.246056in}{2.586892in}}%
\pgfpathlineto{\pgfqpoint{6.299100in}{2.586892in}}%
\pgfpathlineto{\pgfqpoint{6.352144in}{2.586892in}}%
\pgfpathlineto{\pgfqpoint{6.405188in}{2.586892in}}%
\pgfpathlineto{\pgfqpoint{6.458232in}{2.586892in}}%
\pgfpathlineto{\pgfqpoint{6.511276in}{2.586892in}}%
\pgfpathlineto{\pgfqpoint{6.564320in}{2.586892in}}%
\pgfpathlineto{\pgfqpoint{6.617364in}{2.586892in}}%
\pgfpathlineto{\pgfqpoint{6.670408in}{2.586892in}}%
\pgfpathlineto{\pgfqpoint{6.723452in}{2.586892in}}%
\pgfpathlineto{\pgfqpoint{6.776496in}{2.586892in}}%
\pgfpathlineto{\pgfqpoint{6.829540in}{2.586892in}}%
\pgfpathlineto{\pgfqpoint{6.882584in}{2.586892in}}%
\pgfpathlineto{\pgfqpoint{6.935628in}{2.586892in}}%
\pgfpathlineto{\pgfqpoint{6.988671in}{2.586892in}}%
\pgfpathlineto{\pgfqpoint{7.041715in}{2.586892in}}%
\pgfpathlineto{\pgfqpoint{7.094759in}{2.586892in}}%
\pgfpathlineto{\pgfqpoint{7.147803in}{2.586892in}}%
\pgfpathlineto{\pgfqpoint{7.200847in}{2.586892in}}%
\pgfpathlineto{\pgfqpoint{7.253891in}{2.586892in}}%
\pgfpathlineto{\pgfqpoint{7.306935in}{2.586892in}}%
\pgfpathlineto{\pgfqpoint{7.359979in}{2.586892in}}%
\pgfpathlineto{\pgfqpoint{7.413023in}{2.586892in}}%
\pgfpathlineto{\pgfqpoint{7.466067in}{2.586892in}}%
\pgfpathlineto{\pgfqpoint{7.519111in}{2.586892in}}%
\pgfpathlineto{\pgfqpoint{7.572155in}{2.586892in}}%
\pgfpathlineto{\pgfqpoint{7.625199in}{2.586892in}}%
\pgfpathlineto{\pgfqpoint{7.678243in}{2.586892in}}%
\pgfpathlineto{\pgfqpoint{7.731287in}{2.586892in}}%
\pgfpathlineto{\pgfqpoint{7.784330in}{2.586892in}}%
\pgfpathlineto{\pgfqpoint{7.837374in}{2.586892in}}%
\pgfpathlineto{\pgfqpoint{7.890418in}{2.586892in}}%
\pgfpathlineto{\pgfqpoint{7.943462in}{2.586892in}}%
\pgfpathlineto{\pgfqpoint{7.996506in}{2.586892in}}%
\pgfpathlineto{\pgfqpoint{8.049550in}{2.586892in}}%
\pgfpathlineto{\pgfqpoint{8.102594in}{2.586892in}}%
\pgfpathlineto{\pgfqpoint{8.155638in}{2.586892in}}%
\pgfpathlineto{\pgfqpoint{8.208682in}{2.586892in}}%
\pgfpathlineto{\pgfqpoint{8.261726in}{2.586892in}}%
\pgfpathlineto{\pgfqpoint{8.314770in}{2.586892in}}%
\pgfpathlineto{\pgfqpoint{8.367814in}{2.586892in}}%
\pgfpathlineto{\pgfqpoint{8.420858in}{2.586892in}}%
\pgfpathlineto{\pgfqpoint{8.473902in}{2.586892in}}%
\pgfpathlineto{\pgfqpoint{8.526946in}{2.586892in}}%
\pgfpathlineto{\pgfqpoint{8.579990in}{2.586892in}}%
\pgfpathlineto{\pgfqpoint{8.633033in}{2.586892in}}%
\pgfpathlineto{\pgfqpoint{8.686077in}{2.586892in}}%
\pgfpathlineto{\pgfqpoint{8.739121in}{2.586892in}}%
\pgfpathlineto{\pgfqpoint{8.792165in}{2.586892in}}%
\pgfpathlineto{\pgfqpoint{8.845209in}{2.586892in}}%
\pgfpathlineto{\pgfqpoint{8.898253in}{2.586892in}}%
\pgfpathlineto{\pgfqpoint{8.951297in}{2.586892in}}%
\pgfpathlineto{\pgfqpoint{9.004341in}{2.586892in}}%
\pgfpathlineto{\pgfqpoint{9.057385in}{2.586892in}}%
\pgfpathlineto{\pgfqpoint{9.110429in}{2.586892in}}%
\pgfpathlineto{\pgfqpoint{9.163473in}{2.586892in}}%
\pgfpathlineto{\pgfqpoint{9.216517in}{2.586892in}}%
\pgfpathlineto{\pgfqpoint{9.269561in}{2.586892in}}%
\pgfpathlineto{\pgfqpoint{9.322605in}{2.586892in}}%
\pgfpathlineto{\pgfqpoint{9.375649in}{2.586892in}}%
\pgfpathlineto{\pgfqpoint{9.428692in}{2.586892in}}%
\pgfpathlineto{\pgfqpoint{9.481736in}{2.586892in}}%
\pgfpathlineto{\pgfqpoint{9.534780in}{2.586892in}}%
\pgfpathlineto{\pgfqpoint{9.587824in}{2.586892in}}%
\pgfpathlineto{\pgfqpoint{9.640868in}{2.586892in}}%
\pgfpathlineto{\pgfqpoint{9.693912in}{2.586892in}}%
\pgfpathlineto{\pgfqpoint{9.746956in}{2.586892in}}%
\pgfpathlineto{\pgfqpoint{9.800000in}{2.586892in}}%
\pgfpathlineto{\pgfqpoint{9.800000in}{2.588586in}}%
\pgfpathlineto{\pgfqpoint{9.800000in}{2.588586in}}%
\pgfpathlineto{\pgfqpoint{9.746956in}{2.608608in}}%
\pgfpathlineto{\pgfqpoint{9.693912in}{2.865140in}}%
\pgfpathlineto{\pgfqpoint{9.640868in}{2.586892in}}%
\pgfpathlineto{\pgfqpoint{9.587824in}{2.916055in}}%
\pgfpathlineto{\pgfqpoint{9.534780in}{2.934727in}}%
\pgfpathlineto{\pgfqpoint{9.481736in}{2.586892in}}%
\pgfpathlineto{\pgfqpoint{9.428692in}{2.586892in}}%
\pgfpathlineto{\pgfqpoint{9.375649in}{2.695480in}}%
\pgfpathlineto{\pgfqpoint{9.322605in}{2.934727in}}%
\pgfpathlineto{\pgfqpoint{9.269561in}{2.934727in}}%
\pgfpathlineto{\pgfqpoint{9.216517in}{2.934727in}}%
\pgfpathlineto{\pgfqpoint{9.163473in}{2.934727in}}%
\pgfpathlineto{\pgfqpoint{9.110429in}{2.586892in}}%
\pgfpathlineto{\pgfqpoint{9.057385in}{2.586892in}}%
\pgfpathlineto{\pgfqpoint{9.004341in}{2.586892in}}%
\pgfpathlineto{\pgfqpoint{8.951297in}{2.586892in}}%
\pgfpathlineto{\pgfqpoint{8.898253in}{2.689958in}}%
\pgfpathlineto{\pgfqpoint{8.845209in}{2.634872in}}%
\pgfpathlineto{\pgfqpoint{8.792165in}{2.586892in}}%
\pgfpathlineto{\pgfqpoint{8.739121in}{2.825404in}}%
\pgfpathlineto{\pgfqpoint{8.686077in}{2.934727in}}%
\pgfpathlineto{\pgfqpoint{8.633033in}{2.856358in}}%
\pgfpathlineto{\pgfqpoint{8.579990in}{2.934727in}}%
\pgfpathlineto{\pgfqpoint{8.526946in}{2.586892in}}%
\pgfpathlineto{\pgfqpoint{8.473902in}{2.586892in}}%
\pgfpathlineto{\pgfqpoint{8.420858in}{2.934727in}}%
\pgfpathlineto{\pgfqpoint{8.367814in}{2.586892in}}%
\pgfpathlineto{\pgfqpoint{8.314770in}{2.934727in}}%
\pgfpathlineto{\pgfqpoint{8.261726in}{2.934727in}}%
\pgfpathlineto{\pgfqpoint{8.208682in}{2.586892in}}%
\pgfpathlineto{\pgfqpoint{8.155638in}{2.586892in}}%
\pgfpathlineto{\pgfqpoint{8.102594in}{2.934727in}}%
\pgfpathlineto{\pgfqpoint{8.049550in}{2.934727in}}%
\pgfpathlineto{\pgfqpoint{7.996506in}{2.586892in}}%
\pgfpathlineto{\pgfqpoint{7.943462in}{2.586892in}}%
\pgfpathlineto{\pgfqpoint{7.890418in}{2.586892in}}%
\pgfpathlineto{\pgfqpoint{7.837374in}{2.586892in}}%
\pgfpathlineto{\pgfqpoint{7.784330in}{2.586892in}}%
\pgfpathlineto{\pgfqpoint{7.731287in}{2.586892in}}%
\pgfpathlineto{\pgfqpoint{7.678243in}{2.586892in}}%
\pgfpathlineto{\pgfqpoint{7.625199in}{2.934727in}}%
\pgfpathlineto{\pgfqpoint{7.572155in}{2.823725in}}%
\pgfpathlineto{\pgfqpoint{7.519111in}{2.586892in}}%
\pgfpathlineto{\pgfqpoint{7.466067in}{2.586892in}}%
\pgfpathlineto{\pgfqpoint{7.413023in}{2.865454in}}%
\pgfpathlineto{\pgfqpoint{7.359979in}{2.806406in}}%
\pgfpathlineto{\pgfqpoint{7.306935in}{2.586892in}}%
\pgfpathlineto{\pgfqpoint{7.253891in}{2.586892in}}%
\pgfpathlineto{\pgfqpoint{7.200847in}{2.934727in}}%
\pgfpathlineto{\pgfqpoint{7.147803in}{2.586892in}}%
\pgfpathlineto{\pgfqpoint{7.094759in}{2.586892in}}%
\pgfpathlineto{\pgfqpoint{7.041715in}{2.586892in}}%
\pgfpathlineto{\pgfqpoint{6.988671in}{2.586892in}}%
\pgfpathlineto{\pgfqpoint{6.935628in}{2.586892in}}%
\pgfpathlineto{\pgfqpoint{6.882584in}{2.934727in}}%
\pgfpathlineto{\pgfqpoint{6.829540in}{2.586892in}}%
\pgfpathlineto{\pgfqpoint{6.776496in}{2.586892in}}%
\pgfpathlineto{\pgfqpoint{6.723452in}{2.586892in}}%
\pgfpathlineto{\pgfqpoint{6.670408in}{2.586892in}}%
\pgfpathlineto{\pgfqpoint{6.617364in}{2.586892in}}%
\pgfpathlineto{\pgfqpoint{6.564320in}{2.586892in}}%
\pgfpathlineto{\pgfqpoint{6.511276in}{2.586892in}}%
\pgfpathlineto{\pgfqpoint{6.458232in}{2.586892in}}%
\pgfpathlineto{\pgfqpoint{6.405188in}{2.699818in}}%
\pgfpathlineto{\pgfqpoint{6.352144in}{2.586892in}}%
\pgfpathlineto{\pgfqpoint{6.299100in}{2.586892in}}%
\pgfpathlineto{\pgfqpoint{6.246056in}{2.934727in}}%
\pgfpathlineto{\pgfqpoint{6.193012in}{2.934727in}}%
\pgfpathlineto{\pgfqpoint{6.139969in}{2.586892in}}%
\pgfpathlineto{\pgfqpoint{6.086925in}{2.586892in}}%
\pgfpathlineto{\pgfqpoint{6.033881in}{2.586892in}}%
\pgfpathlineto{\pgfqpoint{5.980837in}{2.586892in}}%
\pgfpathlineto{\pgfqpoint{5.927793in}{2.643205in}}%
\pgfpathlineto{\pgfqpoint{5.874749in}{2.934727in}}%
\pgfpathlineto{\pgfqpoint{5.821705in}{2.934727in}}%
\pgfpathlineto{\pgfqpoint{5.768661in}{2.934727in}}%
\pgfpathlineto{\pgfqpoint{5.715617in}{2.586892in}}%
\pgfpathlineto{\pgfqpoint{5.662573in}{2.711726in}}%
\pgfpathlineto{\pgfqpoint{5.609529in}{2.850209in}}%
\pgfpathlineto{\pgfqpoint{5.556485in}{2.586892in}}%
\pgfpathlineto{\pgfqpoint{5.503441in}{2.586892in}}%
\pgfpathlineto{\pgfqpoint{5.450397in}{2.934727in}}%
\pgfpathlineto{\pgfqpoint{5.397353in}{2.660769in}}%
\pgfpathlineto{\pgfqpoint{5.344309in}{2.705202in}}%
\pgfpathlineto{\pgfqpoint{5.291266in}{2.586892in}}%
\pgfpathlineto{\pgfqpoint{5.238222in}{2.837414in}}%
\pgfpathlineto{\pgfqpoint{5.185178in}{2.910114in}}%
\pgfpathlineto{\pgfqpoint{5.132134in}{2.824147in}}%
\pgfpathlineto{\pgfqpoint{5.079090in}{2.586892in}}%
\pgfpathlineto{\pgfqpoint{5.026046in}{2.758322in}}%
\pgfpathlineto{\pgfqpoint{4.973002in}{2.586892in}}%
\pgfpathlineto{\pgfqpoint{4.919958in}{2.863950in}}%
\pgfpathlineto{\pgfqpoint{4.866914in}{2.884645in}}%
\pgfpathlineto{\pgfqpoint{4.813870in}{2.586892in}}%
\pgfpathlineto{\pgfqpoint{4.760826in}{2.586892in}}%
\pgfpathlineto{\pgfqpoint{4.707782in}{2.934727in}}%
\pgfpathlineto{\pgfqpoint{4.654738in}{2.586892in}}%
\pgfpathlineto{\pgfqpoint{4.601694in}{2.927863in}}%
\pgfpathlineto{\pgfqpoint{4.548650in}{2.586892in}}%
\pgfpathlineto{\pgfqpoint{4.495607in}{2.934727in}}%
\pgfpathlineto{\pgfqpoint{4.442563in}{2.586892in}}%
\pgfpathlineto{\pgfqpoint{4.389519in}{2.586892in}}%
\pgfpathlineto{\pgfqpoint{4.336475in}{2.934727in}}%
\pgfpathlineto{\pgfqpoint{4.283431in}{2.586892in}}%
\pgfpathlineto{\pgfqpoint{4.230387in}{2.586892in}}%
\pgfpathlineto{\pgfqpoint{4.177343in}{2.586892in}}%
\pgfpathlineto{\pgfqpoint{4.124299in}{2.586892in}}%
\pgfpathlineto{\pgfqpoint{4.071255in}{2.586892in}}%
\pgfpathlineto{\pgfqpoint{4.018211in}{2.586892in}}%
\pgfpathlineto{\pgfqpoint{3.965167in}{2.586892in}}%
\pgfpathlineto{\pgfqpoint{3.912123in}{2.586892in}}%
\pgfpathlineto{\pgfqpoint{3.859079in}{2.586892in}}%
\pgfpathlineto{\pgfqpoint{3.806035in}{2.586892in}}%
\pgfpathlineto{\pgfqpoint{3.752991in}{2.586892in}}%
\pgfpathlineto{\pgfqpoint{3.699948in}{2.586892in}}%
\pgfpathlineto{\pgfqpoint{3.646904in}{2.586892in}}%
\pgfpathlineto{\pgfqpoint{3.593860in}{2.586892in}}%
\pgfpathlineto{\pgfqpoint{3.540816in}{2.908402in}}%
\pgfpathlineto{\pgfqpoint{3.487772in}{2.710315in}}%
\pgfpathlineto{\pgfqpoint{3.434728in}{2.586892in}}%
\pgfpathlineto{\pgfqpoint{3.381684in}{2.586892in}}%
\pgfpathlineto{\pgfqpoint{3.328640in}{2.586892in}}%
\pgfpathlineto{\pgfqpoint{3.275596in}{2.586892in}}%
\pgfpathlineto{\pgfqpoint{3.222552in}{2.586892in}}%
\pgfpathlineto{\pgfqpoint{3.169508in}{2.586892in}}%
\pgfpathlineto{\pgfqpoint{3.116464in}{2.586892in}}%
\pgfpathlineto{\pgfqpoint{3.063420in}{2.909016in}}%
\pgfpathlineto{\pgfqpoint{3.010376in}{2.586892in}}%
\pgfpathlineto{\pgfqpoint{2.957332in}{2.934727in}}%
\pgfpathlineto{\pgfqpoint{2.904288in}{2.586892in}}%
\pgfpathlineto{\pgfqpoint{2.851245in}{2.586892in}}%
\pgfpathlineto{\pgfqpoint{2.798201in}{2.586892in}}%
\pgfpathlineto{\pgfqpoint{2.745157in}{2.586892in}}%
\pgfpathlineto{\pgfqpoint{2.692113in}{2.586892in}}%
\pgfpathlineto{\pgfqpoint{2.639069in}{2.586892in}}%
\pgfpathlineto{\pgfqpoint{2.586025in}{2.586892in}}%
\pgfpathlineto{\pgfqpoint{2.532981in}{2.586892in}}%
\pgfpathlineto{\pgfqpoint{2.479937in}{2.586892in}}%
\pgfpathlineto{\pgfqpoint{2.426893in}{2.934727in}}%
\pgfpathlineto{\pgfqpoint{2.373849in}{2.844885in}}%
\pgfpathlineto{\pgfqpoint{2.320805in}{2.586892in}}%
\pgfpathlineto{\pgfqpoint{2.267761in}{2.586892in}}%
\pgfpathlineto{\pgfqpoint{2.214717in}{2.910735in}}%
\pgfpathlineto{\pgfqpoint{2.161673in}{2.586892in}}%
\pgfpathlineto{\pgfqpoint{2.108629in}{2.586892in}}%
\pgfpathlineto{\pgfqpoint{2.055586in}{2.586892in}}%
\pgfpathlineto{\pgfqpoint{2.002542in}{2.934727in}}%
\pgfpathlineto{\pgfqpoint{1.949498in}{2.586892in}}%
\pgfpathlineto{\pgfqpoint{1.896454in}{2.934727in}}%
\pgfpathlineto{\pgfqpoint{1.843410in}{2.934727in}}%
\pgfpathlineto{\pgfqpoint{1.790366in}{2.586892in}}%
\pgfpathlineto{\pgfqpoint{1.737322in}{2.586892in}}%
\pgfpathlineto{\pgfqpoint{1.684278in}{2.586892in}}%
\pgfpathlineto{\pgfqpoint{1.631234in}{2.802326in}}%
\pgfpathlineto{\pgfqpoint{1.578190in}{2.586892in}}%
\pgfpathlineto{\pgfqpoint{1.525146in}{2.586892in}}%
\pgfpathlineto{\pgfqpoint{1.472102in}{2.586892in}}%
\pgfpathlineto{\pgfqpoint{1.419058in}{2.586892in}}%
\pgfpathlineto{\pgfqpoint{1.366014in}{2.586892in}}%
\pgfpathlineto{\pgfqpoint{1.312970in}{2.586892in}}%
\pgfpathlineto{\pgfqpoint{1.259927in}{2.749185in}}%
\pgfpathlineto{\pgfqpoint{1.206883in}{2.586892in}}%
\pgfpathlineto{\pgfqpoint{1.153839in}{2.586892in}}%
\pgfpathlineto{\pgfqpoint{1.100795in}{2.586892in}}%
\pgfpathlineto{\pgfqpoint{1.047751in}{2.586892in}}%
\pgfpathlineto{\pgfqpoint{0.994707in}{2.586892in}}%
\pgfpathlineto{\pgfqpoint{0.941663in}{2.586892in}}%
\pgfpathlineto{\pgfqpoint{0.941663in}{2.586892in}}%
\pgfpathclose%
\pgfusepath{stroke,fill}%
}%
\begin{pgfscope}%
\pgfsys@transformshift{0.000000in}{0.000000in}%
\pgfsys@useobject{currentmarker}{}%
\end{pgfscope}%
\end{pgfscope}%
\begin{pgfscope}%
\pgfpathrectangle{\pgfqpoint{0.941663in}{0.670138in}}{\pgfqpoint{8.858337in}{3.465625in}}%
\pgfusepath{clip}%
\pgfsetrectcap%
\pgfsetroundjoin%
\pgfsetlinewidth{1.505625pt}%
\definecolor{currentstroke}{rgb}{0.549020,0.337255,0.294118}%
\pgfsetstrokecolor{currentstroke}%
\pgfsetdash{}{0pt}%
\pgfpathmoveto{\pgfqpoint{0.941663in}{2.586892in}}%
\pgfpathlineto{\pgfqpoint{1.206883in}{2.586892in}}%
\pgfpathlineto{\pgfqpoint{1.259927in}{2.749185in}}%
\pgfpathlineto{\pgfqpoint{1.312970in}{2.586892in}}%
\pgfpathlineto{\pgfqpoint{1.578190in}{2.586892in}}%
\pgfpathlineto{\pgfqpoint{1.631234in}{2.802326in}}%
\pgfpathlineto{\pgfqpoint{1.684278in}{2.586892in}}%
\pgfpathlineto{\pgfqpoint{1.790366in}{2.586892in}}%
\pgfpathlineto{\pgfqpoint{1.843410in}{3.070356in}}%
\pgfpathlineto{\pgfqpoint{1.896454in}{3.219351in}}%
\pgfpathlineto{\pgfqpoint{1.949498in}{2.586892in}}%
\pgfpathlineto{\pgfqpoint{2.002542in}{3.060121in}}%
\pgfpathlineto{\pgfqpoint{2.055586in}{2.586892in}}%
\pgfpathlineto{\pgfqpoint{2.161673in}{2.586892in}}%
\pgfpathlineto{\pgfqpoint{2.214717in}{2.910735in}}%
\pgfpathlineto{\pgfqpoint{2.267761in}{2.586892in}}%
\pgfpathlineto{\pgfqpoint{2.320805in}{2.586892in}}%
\pgfpathlineto{\pgfqpoint{2.373849in}{2.844885in}}%
\pgfpathlineto{\pgfqpoint{2.426893in}{2.942415in}}%
\pgfpathlineto{\pgfqpoint{2.479937in}{2.586892in}}%
\pgfpathlineto{\pgfqpoint{2.904288in}{2.586892in}}%
\pgfpathlineto{\pgfqpoint{2.957332in}{3.080511in}}%
\pgfpathlineto{\pgfqpoint{3.010376in}{2.586892in}}%
\pgfpathlineto{\pgfqpoint{3.063420in}{2.909016in}}%
\pgfpathlineto{\pgfqpoint{3.116464in}{2.586892in}}%
\pgfpathlineto{\pgfqpoint{3.434728in}{2.586892in}}%
\pgfpathlineto{\pgfqpoint{3.487772in}{2.710315in}}%
\pgfpathlineto{\pgfqpoint{3.540816in}{2.908402in}}%
\pgfpathlineto{\pgfqpoint{3.593860in}{2.586892in}}%
\pgfpathlineto{\pgfqpoint{4.283431in}{2.586892in}}%
\pgfpathlineto{\pgfqpoint{4.336475in}{3.224846in}}%
\pgfpathlineto{\pgfqpoint{4.389519in}{2.586892in}}%
\pgfpathlineto{\pgfqpoint{4.442563in}{2.586892in}}%
\pgfpathlineto{\pgfqpoint{4.495607in}{2.968775in}}%
\pgfpathlineto{\pgfqpoint{4.548650in}{2.586892in}}%
\pgfpathlineto{\pgfqpoint{4.601694in}{2.927863in}}%
\pgfpathlineto{\pgfqpoint{4.654738in}{2.586892in}}%
\pgfpathlineto{\pgfqpoint{4.707782in}{3.102021in}}%
\pgfpathlineto{\pgfqpoint{4.760826in}{2.586892in}}%
\pgfpathlineto{\pgfqpoint{4.813870in}{2.586892in}}%
\pgfpathlineto{\pgfqpoint{4.866914in}{2.884645in}}%
\pgfpathlineto{\pgfqpoint{4.919958in}{2.863950in}}%
\pgfpathlineto{\pgfqpoint{4.973002in}{2.586892in}}%
\pgfpathlineto{\pgfqpoint{5.026046in}{2.758322in}}%
\pgfpathlineto{\pgfqpoint{5.079090in}{2.586892in}}%
\pgfpathlineto{\pgfqpoint{5.132134in}{2.824147in}}%
\pgfpathlineto{\pgfqpoint{5.185178in}{2.910114in}}%
\pgfpathlineto{\pgfqpoint{5.238222in}{2.837414in}}%
\pgfpathlineto{\pgfqpoint{5.291266in}{2.586892in}}%
\pgfpathlineto{\pgfqpoint{5.344309in}{2.705202in}}%
\pgfpathlineto{\pgfqpoint{5.397353in}{2.660769in}}%
\pgfpathlineto{\pgfqpoint{5.450397in}{3.105789in}}%
\pgfpathlineto{\pgfqpoint{5.503441in}{2.586892in}}%
\pgfpathlineto{\pgfqpoint{5.556485in}{2.586892in}}%
\pgfpathlineto{\pgfqpoint{5.609529in}{2.850209in}}%
\pgfpathlineto{\pgfqpoint{5.662573in}{2.711726in}}%
\pgfpathlineto{\pgfqpoint{5.715617in}{2.586892in}}%
\pgfpathlineto{\pgfqpoint{5.768661in}{3.248973in}}%
\pgfpathlineto{\pgfqpoint{5.821705in}{3.208761in}}%
\pgfpathlineto{\pgfqpoint{5.874749in}{3.037838in}}%
\pgfpathlineto{\pgfqpoint{5.927793in}{2.934114in}}%
\pgfpathlineto{\pgfqpoint{5.980837in}{2.586892in}}%
\pgfpathlineto{\pgfqpoint{6.139969in}{2.586892in}}%
\pgfpathlineto{\pgfqpoint{6.193012in}{2.971161in}}%
\pgfpathlineto{\pgfqpoint{6.246056in}{3.026875in}}%
\pgfpathlineto{\pgfqpoint{6.299100in}{2.586892in}}%
\pgfpathlineto{\pgfqpoint{6.352144in}{2.586892in}}%
\pgfpathlineto{\pgfqpoint{6.405188in}{2.699818in}}%
\pgfpathlineto{\pgfqpoint{6.458232in}{2.586892in}}%
\pgfpathlineto{\pgfqpoint{6.829540in}{2.586892in}}%
\pgfpathlineto{\pgfqpoint{6.882584in}{3.116001in}}%
\pgfpathlineto{\pgfqpoint{6.935628in}{2.586892in}}%
\pgfpathlineto{\pgfqpoint{7.147803in}{2.586892in}}%
\pgfpathlineto{\pgfqpoint{7.200847in}{3.064989in}}%
\pgfpathlineto{\pgfqpoint{7.253891in}{2.586892in}}%
\pgfpathlineto{\pgfqpoint{7.306935in}{2.586892in}}%
\pgfpathlineto{\pgfqpoint{7.359979in}{2.806406in}}%
\pgfpathlineto{\pgfqpoint{7.413023in}{2.865454in}}%
\pgfpathlineto{\pgfqpoint{7.466067in}{2.586892in}}%
\pgfpathlineto{\pgfqpoint{7.519111in}{2.586892in}}%
\pgfpathlineto{\pgfqpoint{7.572155in}{2.823725in}}%
\pgfpathlineto{\pgfqpoint{7.625199in}{2.988504in}}%
\pgfpathlineto{\pgfqpoint{7.678243in}{2.586892in}}%
\pgfpathlineto{\pgfqpoint{7.996506in}{2.586892in}}%
\pgfpathlineto{\pgfqpoint{8.049550in}{3.102247in}}%
\pgfpathlineto{\pgfqpoint{8.102594in}{3.118508in}}%
\pgfpathlineto{\pgfqpoint{8.155638in}{2.586892in}}%
\pgfpathlineto{\pgfqpoint{8.208682in}{2.586892in}}%
\pgfpathlineto{\pgfqpoint{8.261726in}{3.260476in}}%
\pgfpathlineto{\pgfqpoint{8.314770in}{3.211850in}}%
\pgfpathlineto{\pgfqpoint{8.367814in}{2.586892in}}%
\pgfpathlineto{\pgfqpoint{8.420858in}{2.989399in}}%
\pgfpathlineto{\pgfqpoint{8.473902in}{2.586892in}}%
\pgfpathlineto{\pgfqpoint{8.526946in}{2.586892in}}%
\pgfpathlineto{\pgfqpoint{8.579990in}{3.021265in}}%
\pgfpathlineto{\pgfqpoint{8.633033in}{2.856358in}}%
\pgfpathlineto{\pgfqpoint{8.686077in}{3.059450in}}%
\pgfpathlineto{\pgfqpoint{8.739121in}{2.825404in}}%
\pgfpathlineto{\pgfqpoint{8.792165in}{2.586892in}}%
\pgfpathlineto{\pgfqpoint{8.845209in}{2.634872in}}%
\pgfpathlineto{\pgfqpoint{8.898253in}{2.689958in}}%
\pgfpathlineto{\pgfqpoint{8.951297in}{2.586892in}}%
\pgfpathlineto{\pgfqpoint{9.110429in}{2.586892in}}%
\pgfpathlineto{\pgfqpoint{9.163473in}{3.004188in}}%
\pgfpathlineto{\pgfqpoint{9.216517in}{3.140256in}}%
\pgfpathlineto{\pgfqpoint{9.269561in}{2.964013in}}%
\pgfpathlineto{\pgfqpoint{9.322605in}{3.063213in}}%
\pgfpathlineto{\pgfqpoint{9.375649in}{3.090533in}}%
\pgfpathlineto{\pgfqpoint{9.428692in}{2.586892in}}%
\pgfpathlineto{\pgfqpoint{9.481736in}{2.586892in}}%
\pgfpathlineto{\pgfqpoint{9.534780in}{3.282563in}}%
\pgfpathlineto{\pgfqpoint{9.587824in}{2.956774in}}%
\pgfpathlineto{\pgfqpoint{9.640868in}{2.586892in}}%
\pgfpathlineto{\pgfqpoint{9.693912in}{2.968640in}}%
\pgfpathlineto{\pgfqpoint{9.746956in}{3.115475in}}%
\pgfpathlineto{\pgfqpoint{9.800000in}{2.909416in}}%
\pgfpathlineto{\pgfqpoint{9.800000in}{2.909416in}}%
\pgfusepath{stroke}%
\end{pgfscope}%
\begin{pgfscope}%
\pgfpathrectangle{\pgfqpoint{0.941663in}{0.670138in}}{\pgfqpoint{8.858337in}{3.465625in}}%
\pgfusepath{clip}%
\pgfsetbuttcap%
\pgfsetroundjoin%
\definecolor{currentfill}{rgb}{0.549020,0.337255,0.294118}%
\pgfsetfillcolor{currentfill}%
\pgfsetlinewidth{1.003750pt}%
\definecolor{currentstroke}{rgb}{0.549020,0.337255,0.294118}%
\pgfsetstrokecolor{currentstroke}%
\pgfsetdash{}{0pt}%
\pgfsys@defobject{currentmarker}{\pgfqpoint{0.941663in}{2.586892in}}{\pgfqpoint{9.800000in}{3.282563in}}{%
\pgfpathmoveto{\pgfqpoint{0.941663in}{2.586892in}}%
\pgfpathlineto{\pgfqpoint{0.941663in}{2.586892in}}%
\pgfpathlineto{\pgfqpoint{0.994707in}{2.586892in}}%
\pgfpathlineto{\pgfqpoint{1.047751in}{2.586892in}}%
\pgfpathlineto{\pgfqpoint{1.100795in}{2.586892in}}%
\pgfpathlineto{\pgfqpoint{1.153839in}{2.586892in}}%
\pgfpathlineto{\pgfqpoint{1.206883in}{2.586892in}}%
\pgfpathlineto{\pgfqpoint{1.259927in}{2.749185in}}%
\pgfpathlineto{\pgfqpoint{1.312970in}{2.586892in}}%
\pgfpathlineto{\pgfqpoint{1.366014in}{2.586892in}}%
\pgfpathlineto{\pgfqpoint{1.419058in}{2.586892in}}%
\pgfpathlineto{\pgfqpoint{1.472102in}{2.586892in}}%
\pgfpathlineto{\pgfqpoint{1.525146in}{2.586892in}}%
\pgfpathlineto{\pgfqpoint{1.578190in}{2.586892in}}%
\pgfpathlineto{\pgfqpoint{1.631234in}{2.802326in}}%
\pgfpathlineto{\pgfqpoint{1.684278in}{2.586892in}}%
\pgfpathlineto{\pgfqpoint{1.737322in}{2.586892in}}%
\pgfpathlineto{\pgfqpoint{1.790366in}{2.586892in}}%
\pgfpathlineto{\pgfqpoint{1.843410in}{2.934727in}}%
\pgfpathlineto{\pgfqpoint{1.896454in}{2.934727in}}%
\pgfpathlineto{\pgfqpoint{1.949498in}{2.586892in}}%
\pgfpathlineto{\pgfqpoint{2.002542in}{2.934727in}}%
\pgfpathlineto{\pgfqpoint{2.055586in}{2.586892in}}%
\pgfpathlineto{\pgfqpoint{2.108629in}{2.586892in}}%
\pgfpathlineto{\pgfqpoint{2.161673in}{2.586892in}}%
\pgfpathlineto{\pgfqpoint{2.214717in}{2.910735in}}%
\pgfpathlineto{\pgfqpoint{2.267761in}{2.586892in}}%
\pgfpathlineto{\pgfqpoint{2.320805in}{2.586892in}}%
\pgfpathlineto{\pgfqpoint{2.373849in}{2.844885in}}%
\pgfpathlineto{\pgfqpoint{2.426893in}{2.934727in}}%
\pgfpathlineto{\pgfqpoint{2.479937in}{2.586892in}}%
\pgfpathlineto{\pgfqpoint{2.532981in}{2.586892in}}%
\pgfpathlineto{\pgfqpoint{2.586025in}{2.586892in}}%
\pgfpathlineto{\pgfqpoint{2.639069in}{2.586892in}}%
\pgfpathlineto{\pgfqpoint{2.692113in}{2.586892in}}%
\pgfpathlineto{\pgfqpoint{2.745157in}{2.586892in}}%
\pgfpathlineto{\pgfqpoint{2.798201in}{2.586892in}}%
\pgfpathlineto{\pgfqpoint{2.851245in}{2.586892in}}%
\pgfpathlineto{\pgfqpoint{2.904288in}{2.586892in}}%
\pgfpathlineto{\pgfqpoint{2.957332in}{2.934727in}}%
\pgfpathlineto{\pgfqpoint{3.010376in}{2.586892in}}%
\pgfpathlineto{\pgfqpoint{3.063420in}{2.909016in}}%
\pgfpathlineto{\pgfqpoint{3.116464in}{2.586892in}}%
\pgfpathlineto{\pgfqpoint{3.169508in}{2.586892in}}%
\pgfpathlineto{\pgfqpoint{3.222552in}{2.586892in}}%
\pgfpathlineto{\pgfqpoint{3.275596in}{2.586892in}}%
\pgfpathlineto{\pgfqpoint{3.328640in}{2.586892in}}%
\pgfpathlineto{\pgfqpoint{3.381684in}{2.586892in}}%
\pgfpathlineto{\pgfqpoint{3.434728in}{2.586892in}}%
\pgfpathlineto{\pgfqpoint{3.487772in}{2.710315in}}%
\pgfpathlineto{\pgfqpoint{3.540816in}{2.908402in}}%
\pgfpathlineto{\pgfqpoint{3.593860in}{2.586892in}}%
\pgfpathlineto{\pgfqpoint{3.646904in}{2.586892in}}%
\pgfpathlineto{\pgfqpoint{3.699948in}{2.586892in}}%
\pgfpathlineto{\pgfqpoint{3.752991in}{2.586892in}}%
\pgfpathlineto{\pgfqpoint{3.806035in}{2.586892in}}%
\pgfpathlineto{\pgfqpoint{3.859079in}{2.586892in}}%
\pgfpathlineto{\pgfqpoint{3.912123in}{2.586892in}}%
\pgfpathlineto{\pgfqpoint{3.965167in}{2.586892in}}%
\pgfpathlineto{\pgfqpoint{4.018211in}{2.586892in}}%
\pgfpathlineto{\pgfqpoint{4.071255in}{2.586892in}}%
\pgfpathlineto{\pgfqpoint{4.124299in}{2.586892in}}%
\pgfpathlineto{\pgfqpoint{4.177343in}{2.586892in}}%
\pgfpathlineto{\pgfqpoint{4.230387in}{2.586892in}}%
\pgfpathlineto{\pgfqpoint{4.283431in}{2.586892in}}%
\pgfpathlineto{\pgfqpoint{4.336475in}{2.934727in}}%
\pgfpathlineto{\pgfqpoint{4.389519in}{2.586892in}}%
\pgfpathlineto{\pgfqpoint{4.442563in}{2.586892in}}%
\pgfpathlineto{\pgfqpoint{4.495607in}{2.934727in}}%
\pgfpathlineto{\pgfqpoint{4.548650in}{2.586892in}}%
\pgfpathlineto{\pgfqpoint{4.601694in}{2.927863in}}%
\pgfpathlineto{\pgfqpoint{4.654738in}{2.586892in}}%
\pgfpathlineto{\pgfqpoint{4.707782in}{2.934727in}}%
\pgfpathlineto{\pgfqpoint{4.760826in}{2.586892in}}%
\pgfpathlineto{\pgfqpoint{4.813870in}{2.586892in}}%
\pgfpathlineto{\pgfqpoint{4.866914in}{2.884645in}}%
\pgfpathlineto{\pgfqpoint{4.919958in}{2.863950in}}%
\pgfpathlineto{\pgfqpoint{4.973002in}{2.586892in}}%
\pgfpathlineto{\pgfqpoint{5.026046in}{2.758322in}}%
\pgfpathlineto{\pgfqpoint{5.079090in}{2.586892in}}%
\pgfpathlineto{\pgfqpoint{5.132134in}{2.824147in}}%
\pgfpathlineto{\pgfqpoint{5.185178in}{2.910114in}}%
\pgfpathlineto{\pgfqpoint{5.238222in}{2.837414in}}%
\pgfpathlineto{\pgfqpoint{5.291266in}{2.586892in}}%
\pgfpathlineto{\pgfqpoint{5.344309in}{2.705202in}}%
\pgfpathlineto{\pgfqpoint{5.397353in}{2.660769in}}%
\pgfpathlineto{\pgfqpoint{5.450397in}{2.934727in}}%
\pgfpathlineto{\pgfqpoint{5.503441in}{2.586892in}}%
\pgfpathlineto{\pgfqpoint{5.556485in}{2.586892in}}%
\pgfpathlineto{\pgfqpoint{5.609529in}{2.850209in}}%
\pgfpathlineto{\pgfqpoint{5.662573in}{2.711726in}}%
\pgfpathlineto{\pgfqpoint{5.715617in}{2.586892in}}%
\pgfpathlineto{\pgfqpoint{5.768661in}{2.934727in}}%
\pgfpathlineto{\pgfqpoint{5.821705in}{2.934727in}}%
\pgfpathlineto{\pgfqpoint{5.874749in}{2.934727in}}%
\pgfpathlineto{\pgfqpoint{5.927793in}{2.643205in}}%
\pgfpathlineto{\pgfqpoint{5.980837in}{2.586892in}}%
\pgfpathlineto{\pgfqpoint{6.033881in}{2.586892in}}%
\pgfpathlineto{\pgfqpoint{6.086925in}{2.586892in}}%
\pgfpathlineto{\pgfqpoint{6.139969in}{2.586892in}}%
\pgfpathlineto{\pgfqpoint{6.193012in}{2.934727in}}%
\pgfpathlineto{\pgfqpoint{6.246056in}{2.934727in}}%
\pgfpathlineto{\pgfqpoint{6.299100in}{2.586892in}}%
\pgfpathlineto{\pgfqpoint{6.352144in}{2.586892in}}%
\pgfpathlineto{\pgfqpoint{6.405188in}{2.699818in}}%
\pgfpathlineto{\pgfqpoint{6.458232in}{2.586892in}}%
\pgfpathlineto{\pgfqpoint{6.511276in}{2.586892in}}%
\pgfpathlineto{\pgfqpoint{6.564320in}{2.586892in}}%
\pgfpathlineto{\pgfqpoint{6.617364in}{2.586892in}}%
\pgfpathlineto{\pgfqpoint{6.670408in}{2.586892in}}%
\pgfpathlineto{\pgfqpoint{6.723452in}{2.586892in}}%
\pgfpathlineto{\pgfqpoint{6.776496in}{2.586892in}}%
\pgfpathlineto{\pgfqpoint{6.829540in}{2.586892in}}%
\pgfpathlineto{\pgfqpoint{6.882584in}{2.934727in}}%
\pgfpathlineto{\pgfqpoint{6.935628in}{2.586892in}}%
\pgfpathlineto{\pgfqpoint{6.988671in}{2.586892in}}%
\pgfpathlineto{\pgfqpoint{7.041715in}{2.586892in}}%
\pgfpathlineto{\pgfqpoint{7.094759in}{2.586892in}}%
\pgfpathlineto{\pgfqpoint{7.147803in}{2.586892in}}%
\pgfpathlineto{\pgfqpoint{7.200847in}{2.934727in}}%
\pgfpathlineto{\pgfqpoint{7.253891in}{2.586892in}}%
\pgfpathlineto{\pgfqpoint{7.306935in}{2.586892in}}%
\pgfpathlineto{\pgfqpoint{7.359979in}{2.806406in}}%
\pgfpathlineto{\pgfqpoint{7.413023in}{2.865454in}}%
\pgfpathlineto{\pgfqpoint{7.466067in}{2.586892in}}%
\pgfpathlineto{\pgfqpoint{7.519111in}{2.586892in}}%
\pgfpathlineto{\pgfqpoint{7.572155in}{2.823725in}}%
\pgfpathlineto{\pgfqpoint{7.625199in}{2.934727in}}%
\pgfpathlineto{\pgfqpoint{7.678243in}{2.586892in}}%
\pgfpathlineto{\pgfqpoint{7.731287in}{2.586892in}}%
\pgfpathlineto{\pgfqpoint{7.784330in}{2.586892in}}%
\pgfpathlineto{\pgfqpoint{7.837374in}{2.586892in}}%
\pgfpathlineto{\pgfqpoint{7.890418in}{2.586892in}}%
\pgfpathlineto{\pgfqpoint{7.943462in}{2.586892in}}%
\pgfpathlineto{\pgfqpoint{7.996506in}{2.586892in}}%
\pgfpathlineto{\pgfqpoint{8.049550in}{2.934727in}}%
\pgfpathlineto{\pgfqpoint{8.102594in}{2.934727in}}%
\pgfpathlineto{\pgfqpoint{8.155638in}{2.586892in}}%
\pgfpathlineto{\pgfqpoint{8.208682in}{2.586892in}}%
\pgfpathlineto{\pgfqpoint{8.261726in}{2.934727in}}%
\pgfpathlineto{\pgfqpoint{8.314770in}{2.934727in}}%
\pgfpathlineto{\pgfqpoint{8.367814in}{2.586892in}}%
\pgfpathlineto{\pgfqpoint{8.420858in}{2.934727in}}%
\pgfpathlineto{\pgfqpoint{8.473902in}{2.586892in}}%
\pgfpathlineto{\pgfqpoint{8.526946in}{2.586892in}}%
\pgfpathlineto{\pgfqpoint{8.579990in}{2.934727in}}%
\pgfpathlineto{\pgfqpoint{8.633033in}{2.856358in}}%
\pgfpathlineto{\pgfqpoint{8.686077in}{2.934727in}}%
\pgfpathlineto{\pgfqpoint{8.739121in}{2.825404in}}%
\pgfpathlineto{\pgfqpoint{8.792165in}{2.586892in}}%
\pgfpathlineto{\pgfqpoint{8.845209in}{2.634872in}}%
\pgfpathlineto{\pgfqpoint{8.898253in}{2.689958in}}%
\pgfpathlineto{\pgfqpoint{8.951297in}{2.586892in}}%
\pgfpathlineto{\pgfqpoint{9.004341in}{2.586892in}}%
\pgfpathlineto{\pgfqpoint{9.057385in}{2.586892in}}%
\pgfpathlineto{\pgfqpoint{9.110429in}{2.586892in}}%
\pgfpathlineto{\pgfqpoint{9.163473in}{2.934727in}}%
\pgfpathlineto{\pgfqpoint{9.216517in}{2.934727in}}%
\pgfpathlineto{\pgfqpoint{9.269561in}{2.934727in}}%
\pgfpathlineto{\pgfqpoint{9.322605in}{2.934727in}}%
\pgfpathlineto{\pgfqpoint{9.375649in}{2.695480in}}%
\pgfpathlineto{\pgfqpoint{9.428692in}{2.586892in}}%
\pgfpathlineto{\pgfqpoint{9.481736in}{2.586892in}}%
\pgfpathlineto{\pgfqpoint{9.534780in}{2.934727in}}%
\pgfpathlineto{\pgfqpoint{9.587824in}{2.916055in}}%
\pgfpathlineto{\pgfqpoint{9.640868in}{2.586892in}}%
\pgfpathlineto{\pgfqpoint{9.693912in}{2.865140in}}%
\pgfpathlineto{\pgfqpoint{9.746956in}{2.608608in}}%
\pgfpathlineto{\pgfqpoint{9.800000in}{2.588586in}}%
\pgfpathlineto{\pgfqpoint{9.800000in}{2.909416in}}%
\pgfpathlineto{\pgfqpoint{9.800000in}{2.909416in}}%
\pgfpathlineto{\pgfqpoint{9.746956in}{3.115475in}}%
\pgfpathlineto{\pgfqpoint{9.693912in}{2.968640in}}%
\pgfpathlineto{\pgfqpoint{9.640868in}{2.586892in}}%
\pgfpathlineto{\pgfqpoint{9.587824in}{2.956774in}}%
\pgfpathlineto{\pgfqpoint{9.534780in}{3.282563in}}%
\pgfpathlineto{\pgfqpoint{9.481736in}{2.586892in}}%
\pgfpathlineto{\pgfqpoint{9.428692in}{2.586892in}}%
\pgfpathlineto{\pgfqpoint{9.375649in}{3.090533in}}%
\pgfpathlineto{\pgfqpoint{9.322605in}{3.063213in}}%
\pgfpathlineto{\pgfqpoint{9.269561in}{2.964013in}}%
\pgfpathlineto{\pgfqpoint{9.216517in}{3.140256in}}%
\pgfpathlineto{\pgfqpoint{9.163473in}{3.004188in}}%
\pgfpathlineto{\pgfqpoint{9.110429in}{2.586892in}}%
\pgfpathlineto{\pgfqpoint{9.057385in}{2.586892in}}%
\pgfpathlineto{\pgfqpoint{9.004341in}{2.586892in}}%
\pgfpathlineto{\pgfqpoint{8.951297in}{2.586892in}}%
\pgfpathlineto{\pgfqpoint{8.898253in}{2.689958in}}%
\pgfpathlineto{\pgfqpoint{8.845209in}{2.634872in}}%
\pgfpathlineto{\pgfqpoint{8.792165in}{2.586892in}}%
\pgfpathlineto{\pgfqpoint{8.739121in}{2.825404in}}%
\pgfpathlineto{\pgfqpoint{8.686077in}{3.059450in}}%
\pgfpathlineto{\pgfqpoint{8.633033in}{2.856358in}}%
\pgfpathlineto{\pgfqpoint{8.579990in}{3.021265in}}%
\pgfpathlineto{\pgfqpoint{8.526946in}{2.586892in}}%
\pgfpathlineto{\pgfqpoint{8.473902in}{2.586892in}}%
\pgfpathlineto{\pgfqpoint{8.420858in}{2.989399in}}%
\pgfpathlineto{\pgfqpoint{8.367814in}{2.586892in}}%
\pgfpathlineto{\pgfqpoint{8.314770in}{3.211850in}}%
\pgfpathlineto{\pgfqpoint{8.261726in}{3.260476in}}%
\pgfpathlineto{\pgfqpoint{8.208682in}{2.586892in}}%
\pgfpathlineto{\pgfqpoint{8.155638in}{2.586892in}}%
\pgfpathlineto{\pgfqpoint{8.102594in}{3.118508in}}%
\pgfpathlineto{\pgfqpoint{8.049550in}{3.102247in}}%
\pgfpathlineto{\pgfqpoint{7.996506in}{2.586892in}}%
\pgfpathlineto{\pgfqpoint{7.943462in}{2.586892in}}%
\pgfpathlineto{\pgfqpoint{7.890418in}{2.586892in}}%
\pgfpathlineto{\pgfqpoint{7.837374in}{2.586892in}}%
\pgfpathlineto{\pgfqpoint{7.784330in}{2.586892in}}%
\pgfpathlineto{\pgfqpoint{7.731287in}{2.586892in}}%
\pgfpathlineto{\pgfqpoint{7.678243in}{2.586892in}}%
\pgfpathlineto{\pgfqpoint{7.625199in}{2.988504in}}%
\pgfpathlineto{\pgfqpoint{7.572155in}{2.823725in}}%
\pgfpathlineto{\pgfqpoint{7.519111in}{2.586892in}}%
\pgfpathlineto{\pgfqpoint{7.466067in}{2.586892in}}%
\pgfpathlineto{\pgfqpoint{7.413023in}{2.865454in}}%
\pgfpathlineto{\pgfqpoint{7.359979in}{2.806406in}}%
\pgfpathlineto{\pgfqpoint{7.306935in}{2.586892in}}%
\pgfpathlineto{\pgfqpoint{7.253891in}{2.586892in}}%
\pgfpathlineto{\pgfqpoint{7.200847in}{3.064989in}}%
\pgfpathlineto{\pgfqpoint{7.147803in}{2.586892in}}%
\pgfpathlineto{\pgfqpoint{7.094759in}{2.586892in}}%
\pgfpathlineto{\pgfqpoint{7.041715in}{2.586892in}}%
\pgfpathlineto{\pgfqpoint{6.988671in}{2.586892in}}%
\pgfpathlineto{\pgfqpoint{6.935628in}{2.586892in}}%
\pgfpathlineto{\pgfqpoint{6.882584in}{3.116001in}}%
\pgfpathlineto{\pgfqpoint{6.829540in}{2.586892in}}%
\pgfpathlineto{\pgfqpoint{6.776496in}{2.586892in}}%
\pgfpathlineto{\pgfqpoint{6.723452in}{2.586892in}}%
\pgfpathlineto{\pgfqpoint{6.670408in}{2.586892in}}%
\pgfpathlineto{\pgfqpoint{6.617364in}{2.586892in}}%
\pgfpathlineto{\pgfqpoint{6.564320in}{2.586892in}}%
\pgfpathlineto{\pgfqpoint{6.511276in}{2.586892in}}%
\pgfpathlineto{\pgfqpoint{6.458232in}{2.586892in}}%
\pgfpathlineto{\pgfqpoint{6.405188in}{2.699818in}}%
\pgfpathlineto{\pgfqpoint{6.352144in}{2.586892in}}%
\pgfpathlineto{\pgfqpoint{6.299100in}{2.586892in}}%
\pgfpathlineto{\pgfqpoint{6.246056in}{3.026875in}}%
\pgfpathlineto{\pgfqpoint{6.193012in}{2.971161in}}%
\pgfpathlineto{\pgfqpoint{6.139969in}{2.586892in}}%
\pgfpathlineto{\pgfqpoint{6.086925in}{2.586892in}}%
\pgfpathlineto{\pgfqpoint{6.033881in}{2.586892in}}%
\pgfpathlineto{\pgfqpoint{5.980837in}{2.586892in}}%
\pgfpathlineto{\pgfqpoint{5.927793in}{2.934114in}}%
\pgfpathlineto{\pgfqpoint{5.874749in}{3.037838in}}%
\pgfpathlineto{\pgfqpoint{5.821705in}{3.208761in}}%
\pgfpathlineto{\pgfqpoint{5.768661in}{3.248973in}}%
\pgfpathlineto{\pgfqpoint{5.715617in}{2.586892in}}%
\pgfpathlineto{\pgfqpoint{5.662573in}{2.711726in}}%
\pgfpathlineto{\pgfqpoint{5.609529in}{2.850209in}}%
\pgfpathlineto{\pgfqpoint{5.556485in}{2.586892in}}%
\pgfpathlineto{\pgfqpoint{5.503441in}{2.586892in}}%
\pgfpathlineto{\pgfqpoint{5.450397in}{3.105789in}}%
\pgfpathlineto{\pgfqpoint{5.397353in}{2.660769in}}%
\pgfpathlineto{\pgfqpoint{5.344309in}{2.705202in}}%
\pgfpathlineto{\pgfqpoint{5.291266in}{2.586892in}}%
\pgfpathlineto{\pgfqpoint{5.238222in}{2.837414in}}%
\pgfpathlineto{\pgfqpoint{5.185178in}{2.910114in}}%
\pgfpathlineto{\pgfqpoint{5.132134in}{2.824147in}}%
\pgfpathlineto{\pgfqpoint{5.079090in}{2.586892in}}%
\pgfpathlineto{\pgfqpoint{5.026046in}{2.758322in}}%
\pgfpathlineto{\pgfqpoint{4.973002in}{2.586892in}}%
\pgfpathlineto{\pgfqpoint{4.919958in}{2.863950in}}%
\pgfpathlineto{\pgfqpoint{4.866914in}{2.884645in}}%
\pgfpathlineto{\pgfqpoint{4.813870in}{2.586892in}}%
\pgfpathlineto{\pgfqpoint{4.760826in}{2.586892in}}%
\pgfpathlineto{\pgfqpoint{4.707782in}{3.102021in}}%
\pgfpathlineto{\pgfqpoint{4.654738in}{2.586892in}}%
\pgfpathlineto{\pgfqpoint{4.601694in}{2.927863in}}%
\pgfpathlineto{\pgfqpoint{4.548650in}{2.586892in}}%
\pgfpathlineto{\pgfqpoint{4.495607in}{2.968775in}}%
\pgfpathlineto{\pgfqpoint{4.442563in}{2.586892in}}%
\pgfpathlineto{\pgfqpoint{4.389519in}{2.586892in}}%
\pgfpathlineto{\pgfqpoint{4.336475in}{3.224846in}}%
\pgfpathlineto{\pgfqpoint{4.283431in}{2.586892in}}%
\pgfpathlineto{\pgfqpoint{4.230387in}{2.586892in}}%
\pgfpathlineto{\pgfqpoint{4.177343in}{2.586892in}}%
\pgfpathlineto{\pgfqpoint{4.124299in}{2.586892in}}%
\pgfpathlineto{\pgfqpoint{4.071255in}{2.586892in}}%
\pgfpathlineto{\pgfqpoint{4.018211in}{2.586892in}}%
\pgfpathlineto{\pgfqpoint{3.965167in}{2.586892in}}%
\pgfpathlineto{\pgfqpoint{3.912123in}{2.586892in}}%
\pgfpathlineto{\pgfqpoint{3.859079in}{2.586892in}}%
\pgfpathlineto{\pgfqpoint{3.806035in}{2.586892in}}%
\pgfpathlineto{\pgfqpoint{3.752991in}{2.586892in}}%
\pgfpathlineto{\pgfqpoint{3.699948in}{2.586892in}}%
\pgfpathlineto{\pgfqpoint{3.646904in}{2.586892in}}%
\pgfpathlineto{\pgfqpoint{3.593860in}{2.586892in}}%
\pgfpathlineto{\pgfqpoint{3.540816in}{2.908402in}}%
\pgfpathlineto{\pgfqpoint{3.487772in}{2.710315in}}%
\pgfpathlineto{\pgfqpoint{3.434728in}{2.586892in}}%
\pgfpathlineto{\pgfqpoint{3.381684in}{2.586892in}}%
\pgfpathlineto{\pgfqpoint{3.328640in}{2.586892in}}%
\pgfpathlineto{\pgfqpoint{3.275596in}{2.586892in}}%
\pgfpathlineto{\pgfqpoint{3.222552in}{2.586892in}}%
\pgfpathlineto{\pgfqpoint{3.169508in}{2.586892in}}%
\pgfpathlineto{\pgfqpoint{3.116464in}{2.586892in}}%
\pgfpathlineto{\pgfqpoint{3.063420in}{2.909016in}}%
\pgfpathlineto{\pgfqpoint{3.010376in}{2.586892in}}%
\pgfpathlineto{\pgfqpoint{2.957332in}{3.080511in}}%
\pgfpathlineto{\pgfqpoint{2.904288in}{2.586892in}}%
\pgfpathlineto{\pgfqpoint{2.851245in}{2.586892in}}%
\pgfpathlineto{\pgfqpoint{2.798201in}{2.586892in}}%
\pgfpathlineto{\pgfqpoint{2.745157in}{2.586892in}}%
\pgfpathlineto{\pgfqpoint{2.692113in}{2.586892in}}%
\pgfpathlineto{\pgfqpoint{2.639069in}{2.586892in}}%
\pgfpathlineto{\pgfqpoint{2.586025in}{2.586892in}}%
\pgfpathlineto{\pgfqpoint{2.532981in}{2.586892in}}%
\pgfpathlineto{\pgfqpoint{2.479937in}{2.586892in}}%
\pgfpathlineto{\pgfqpoint{2.426893in}{2.942415in}}%
\pgfpathlineto{\pgfqpoint{2.373849in}{2.844885in}}%
\pgfpathlineto{\pgfqpoint{2.320805in}{2.586892in}}%
\pgfpathlineto{\pgfqpoint{2.267761in}{2.586892in}}%
\pgfpathlineto{\pgfqpoint{2.214717in}{2.910735in}}%
\pgfpathlineto{\pgfqpoint{2.161673in}{2.586892in}}%
\pgfpathlineto{\pgfqpoint{2.108629in}{2.586892in}}%
\pgfpathlineto{\pgfqpoint{2.055586in}{2.586892in}}%
\pgfpathlineto{\pgfqpoint{2.002542in}{3.060121in}}%
\pgfpathlineto{\pgfqpoint{1.949498in}{2.586892in}}%
\pgfpathlineto{\pgfqpoint{1.896454in}{3.219351in}}%
\pgfpathlineto{\pgfqpoint{1.843410in}{3.070356in}}%
\pgfpathlineto{\pgfqpoint{1.790366in}{2.586892in}}%
\pgfpathlineto{\pgfqpoint{1.737322in}{2.586892in}}%
\pgfpathlineto{\pgfqpoint{1.684278in}{2.586892in}}%
\pgfpathlineto{\pgfqpoint{1.631234in}{2.802326in}}%
\pgfpathlineto{\pgfqpoint{1.578190in}{2.586892in}}%
\pgfpathlineto{\pgfqpoint{1.525146in}{2.586892in}}%
\pgfpathlineto{\pgfqpoint{1.472102in}{2.586892in}}%
\pgfpathlineto{\pgfqpoint{1.419058in}{2.586892in}}%
\pgfpathlineto{\pgfqpoint{1.366014in}{2.586892in}}%
\pgfpathlineto{\pgfqpoint{1.312970in}{2.586892in}}%
\pgfpathlineto{\pgfqpoint{1.259927in}{2.749185in}}%
\pgfpathlineto{\pgfqpoint{1.206883in}{2.586892in}}%
\pgfpathlineto{\pgfqpoint{1.153839in}{2.586892in}}%
\pgfpathlineto{\pgfqpoint{1.100795in}{2.586892in}}%
\pgfpathlineto{\pgfqpoint{1.047751in}{2.586892in}}%
\pgfpathlineto{\pgfqpoint{0.994707in}{2.586892in}}%
\pgfpathlineto{\pgfqpoint{0.941663in}{2.586892in}}%
\pgfpathlineto{\pgfqpoint{0.941663in}{2.586892in}}%
\pgfpathclose%
\pgfusepath{stroke,fill}%
}%
\begin{pgfscope}%
\pgfsys@transformshift{0.000000in}{0.000000in}%
\pgfsys@useobject{currentmarker}{}%
\end{pgfscope}%
\end{pgfscope}%
\begin{pgfscope}%
\pgfpathrectangle{\pgfqpoint{0.941663in}{0.670138in}}{\pgfqpoint{8.858337in}{3.465625in}}%
\pgfusepath{clip}%
\pgfsetrectcap%
\pgfsetroundjoin%
\pgfsetlinewidth{1.505625pt}%
\definecolor{currentstroke}{rgb}{1.000000,0.647059,0.000000}%
\pgfsetstrokecolor{currentstroke}%
\pgfsetdash{}{0pt}%
\pgfpathmoveto{\pgfqpoint{0.941663in}{1.891220in}}%
\pgfpathlineto{\pgfqpoint{0.994707in}{1.543384in}}%
\pgfpathlineto{\pgfqpoint{1.047751in}{1.891220in}}%
\pgfpathlineto{\pgfqpoint{1.419058in}{1.891220in}}%
\pgfpathlineto{\pgfqpoint{1.472102in}{1.843091in}}%
\pgfpathlineto{\pgfqpoint{1.525146in}{1.891220in}}%
\pgfpathlineto{\pgfqpoint{1.631234in}{1.891220in}}%
\pgfpathlineto{\pgfqpoint{1.684278in}{1.880012in}}%
\pgfpathlineto{\pgfqpoint{1.737322in}{1.543384in}}%
\pgfpathlineto{\pgfqpoint{1.790366in}{1.543384in}}%
\pgfpathlineto{\pgfqpoint{1.843410in}{1.891220in}}%
\pgfpathlineto{\pgfqpoint{1.896454in}{1.891220in}}%
\pgfpathlineto{\pgfqpoint{1.949498in}{1.543384in}}%
\pgfpathlineto{\pgfqpoint{2.002542in}{1.891220in}}%
\pgfpathlineto{\pgfqpoint{2.055586in}{1.891220in}}%
\pgfpathlineto{\pgfqpoint{2.108629in}{1.569408in}}%
\pgfpathlineto{\pgfqpoint{2.161673in}{1.891220in}}%
\pgfpathlineto{\pgfqpoint{2.214717in}{1.891220in}}%
\pgfpathlineto{\pgfqpoint{2.267761in}{1.631196in}}%
\pgfpathlineto{\pgfqpoint{2.320805in}{1.543384in}}%
\pgfpathlineto{\pgfqpoint{2.373849in}{1.891220in}}%
\pgfpathlineto{\pgfqpoint{2.851245in}{1.891220in}}%
\pgfpathlineto{\pgfqpoint{2.904288in}{1.543384in}}%
\pgfpathlineto{\pgfqpoint{2.957332in}{1.891220in}}%
\pgfpathlineto{\pgfqpoint{3.010376in}{1.571272in}}%
\pgfpathlineto{\pgfqpoint{3.063420in}{1.891220in}}%
\pgfpathlineto{\pgfqpoint{3.328640in}{1.891220in}}%
\pgfpathlineto{\pgfqpoint{3.381684in}{1.794175in}}%
\pgfpathlineto{\pgfqpoint{3.434728in}{1.543384in}}%
\pgfpathlineto{\pgfqpoint{3.487772in}{1.891220in}}%
\pgfpathlineto{\pgfqpoint{4.071255in}{1.891220in}}%
\pgfpathlineto{\pgfqpoint{4.124299in}{1.770677in}}%
\pgfpathlineto{\pgfqpoint{4.177343in}{1.891220in}}%
\pgfpathlineto{\pgfqpoint{4.230387in}{1.543384in}}%
\pgfpathlineto{\pgfqpoint{4.283431in}{1.543384in}}%
\pgfpathlineto{\pgfqpoint{4.336475in}{1.891220in}}%
\pgfpathlineto{\pgfqpoint{4.389519in}{1.543384in}}%
\pgfpathlineto{\pgfqpoint{4.442563in}{1.580227in}}%
\pgfpathlineto{\pgfqpoint{4.495607in}{1.891220in}}%
\pgfpathlineto{\pgfqpoint{4.548650in}{1.543384in}}%
\pgfpathlineto{\pgfqpoint{4.601694in}{1.891220in}}%
\pgfpathlineto{\pgfqpoint{4.654738in}{1.543384in}}%
\pgfpathlineto{\pgfqpoint{4.707782in}{1.891220in}}%
\pgfpathlineto{\pgfqpoint{4.760826in}{1.543384in}}%
\pgfpathlineto{\pgfqpoint{4.813870in}{1.543384in}}%
\pgfpathlineto{\pgfqpoint{4.866914in}{1.891220in}}%
\pgfpathlineto{\pgfqpoint{4.919958in}{1.891220in}}%
\pgfpathlineto{\pgfqpoint{4.973002in}{1.543384in}}%
\pgfpathlineto{\pgfqpoint{5.026046in}{1.891220in}}%
\pgfpathlineto{\pgfqpoint{5.079090in}{1.543384in}}%
\pgfpathlineto{\pgfqpoint{5.132134in}{1.891220in}}%
\pgfpathlineto{\pgfqpoint{5.238222in}{1.891220in}}%
\pgfpathlineto{\pgfqpoint{5.291266in}{1.543384in}}%
\pgfpathlineto{\pgfqpoint{5.344309in}{1.891220in}}%
\pgfpathlineto{\pgfqpoint{5.450397in}{1.891220in}}%
\pgfpathlineto{\pgfqpoint{5.503441in}{1.543384in}}%
\pgfpathlineto{\pgfqpoint{5.556485in}{1.549519in}}%
\pgfpathlineto{\pgfqpoint{5.609529in}{1.891220in}}%
\pgfpathlineto{\pgfqpoint{5.662573in}{1.891220in}}%
\pgfpathlineto{\pgfqpoint{5.715617in}{1.543384in}}%
\pgfpathlineto{\pgfqpoint{5.768661in}{1.891220in}}%
\pgfpathlineto{\pgfqpoint{5.980837in}{1.891220in}}%
\pgfpathlineto{\pgfqpoint{6.033881in}{1.678405in}}%
\pgfpathlineto{\pgfqpoint{6.086925in}{1.543384in}}%
\pgfpathlineto{\pgfqpoint{6.139969in}{1.731521in}}%
\pgfpathlineto{\pgfqpoint{6.193012in}{1.891220in}}%
\pgfpathlineto{\pgfqpoint{6.246056in}{1.891220in}}%
\pgfpathlineto{\pgfqpoint{6.299100in}{1.798179in}}%
\pgfpathlineto{\pgfqpoint{6.352144in}{1.891220in}}%
\pgfpathlineto{\pgfqpoint{6.670408in}{1.891220in}}%
\pgfpathlineto{\pgfqpoint{6.723452in}{1.871334in}}%
\pgfpathlineto{\pgfqpoint{6.776496in}{1.543384in}}%
\pgfpathlineto{\pgfqpoint{6.829540in}{1.891220in}}%
\pgfpathlineto{\pgfqpoint{7.041715in}{1.891220in}}%
\pgfpathlineto{\pgfqpoint{7.094759in}{1.543384in}}%
\pgfpathlineto{\pgfqpoint{7.147803in}{1.891220in}}%
\pgfpathlineto{\pgfqpoint{7.200847in}{1.891220in}}%
\pgfpathlineto{\pgfqpoint{7.253891in}{1.543384in}}%
\pgfpathlineto{\pgfqpoint{7.306935in}{1.746843in}}%
\pgfpathlineto{\pgfqpoint{7.359979in}{1.891220in}}%
\pgfpathlineto{\pgfqpoint{7.413023in}{1.891220in}}%
\pgfpathlineto{\pgfqpoint{7.466067in}{1.648522in}}%
\pgfpathlineto{\pgfqpoint{7.519111in}{1.543384in}}%
\pgfpathlineto{\pgfqpoint{7.572155in}{1.891220in}}%
\pgfpathlineto{\pgfqpoint{7.625199in}{1.891220in}}%
\pgfpathlineto{\pgfqpoint{7.678243in}{1.849746in}}%
\pgfpathlineto{\pgfqpoint{7.731287in}{1.891220in}}%
\pgfpathlineto{\pgfqpoint{7.784330in}{1.567858in}}%
\pgfpathlineto{\pgfqpoint{7.837374in}{1.543384in}}%
\pgfpathlineto{\pgfqpoint{7.890418in}{1.891220in}}%
\pgfpathlineto{\pgfqpoint{7.943462in}{1.543384in}}%
\pgfpathlineto{\pgfqpoint{7.996506in}{1.543384in}}%
\pgfpathlineto{\pgfqpoint{8.049550in}{1.891220in}}%
\pgfpathlineto{\pgfqpoint{8.102594in}{1.891220in}}%
\pgfpathlineto{\pgfqpoint{8.155638in}{1.543384in}}%
\pgfpathlineto{\pgfqpoint{8.208682in}{1.543384in}}%
\pgfpathlineto{\pgfqpoint{8.261726in}{1.891220in}}%
\pgfpathlineto{\pgfqpoint{8.314770in}{1.891220in}}%
\pgfpathlineto{\pgfqpoint{8.367814in}{1.543384in}}%
\pgfpathlineto{\pgfqpoint{8.420858in}{1.891220in}}%
\pgfpathlineto{\pgfqpoint{8.473902in}{1.543384in}}%
\pgfpathlineto{\pgfqpoint{8.526946in}{1.543384in}}%
\pgfpathlineto{\pgfqpoint{8.579990in}{1.891220in}}%
\pgfpathlineto{\pgfqpoint{8.739121in}{1.891220in}}%
\pgfpathlineto{\pgfqpoint{8.792165in}{1.755774in}}%
\pgfpathlineto{\pgfqpoint{8.845209in}{1.891220in}}%
\pgfpathlineto{\pgfqpoint{8.898253in}{1.891220in}}%
\pgfpathlineto{\pgfqpoint{8.951297in}{1.543384in}}%
\pgfpathlineto{\pgfqpoint{9.004341in}{1.644141in}}%
\pgfpathlineto{\pgfqpoint{9.057385in}{1.543384in}}%
\pgfpathlineto{\pgfqpoint{9.110429in}{1.543384in}}%
\pgfpathlineto{\pgfqpoint{9.163473in}{1.891220in}}%
\pgfpathlineto{\pgfqpoint{9.375649in}{1.891220in}}%
\pgfpathlineto{\pgfqpoint{9.428692in}{1.543384in}}%
\pgfpathlineto{\pgfqpoint{9.481736in}{1.543384in}}%
\pgfpathlineto{\pgfqpoint{9.534780in}{1.891220in}}%
\pgfpathlineto{\pgfqpoint{9.587824in}{1.891220in}}%
\pgfpathlineto{\pgfqpoint{9.640868in}{1.617282in}}%
\pgfpathlineto{\pgfqpoint{9.693912in}{1.891220in}}%
\pgfpathlineto{\pgfqpoint{9.800000in}{1.891220in}}%
\pgfpathlineto{\pgfqpoint{9.800000in}{1.891220in}}%
\pgfusepath{stroke}%
\end{pgfscope}%
\begin{pgfscope}%
\pgfpathrectangle{\pgfqpoint{0.941663in}{0.670138in}}{\pgfqpoint{8.858337in}{3.465625in}}%
\pgfusepath{clip}%
\pgfsetbuttcap%
\pgfsetroundjoin%
\definecolor{currentfill}{rgb}{1.000000,0.647059,0.000000}%
\pgfsetfillcolor{currentfill}%
\pgfsetlinewidth{1.003750pt}%
\definecolor{currentstroke}{rgb}{1.000000,0.647059,0.000000}%
\pgfsetstrokecolor{currentstroke}%
\pgfsetdash{}{0pt}%
\pgfsys@defobject{currentmarker}{\pgfqpoint{0.941663in}{1.543384in}}{\pgfqpoint{9.800000in}{1.891220in}}{%
\pgfpathmoveto{\pgfqpoint{0.941663in}{1.891220in}}%
\pgfpathlineto{\pgfqpoint{0.941663in}{1.891220in}}%
\pgfpathlineto{\pgfqpoint{0.994707in}{1.891220in}}%
\pgfpathlineto{\pgfqpoint{1.047751in}{1.891220in}}%
\pgfpathlineto{\pgfqpoint{1.100795in}{1.891220in}}%
\pgfpathlineto{\pgfqpoint{1.153839in}{1.891220in}}%
\pgfpathlineto{\pgfqpoint{1.206883in}{1.891220in}}%
\pgfpathlineto{\pgfqpoint{1.259927in}{1.891220in}}%
\pgfpathlineto{\pgfqpoint{1.312970in}{1.891220in}}%
\pgfpathlineto{\pgfqpoint{1.366014in}{1.891220in}}%
\pgfpathlineto{\pgfqpoint{1.419058in}{1.891220in}}%
\pgfpathlineto{\pgfqpoint{1.472102in}{1.891220in}}%
\pgfpathlineto{\pgfqpoint{1.525146in}{1.891220in}}%
\pgfpathlineto{\pgfqpoint{1.578190in}{1.891220in}}%
\pgfpathlineto{\pgfqpoint{1.631234in}{1.891220in}}%
\pgfpathlineto{\pgfqpoint{1.684278in}{1.891220in}}%
\pgfpathlineto{\pgfqpoint{1.737322in}{1.891220in}}%
\pgfpathlineto{\pgfqpoint{1.790366in}{1.891220in}}%
\pgfpathlineto{\pgfqpoint{1.843410in}{1.891220in}}%
\pgfpathlineto{\pgfqpoint{1.896454in}{1.891220in}}%
\pgfpathlineto{\pgfqpoint{1.949498in}{1.891220in}}%
\pgfpathlineto{\pgfqpoint{2.002542in}{1.891220in}}%
\pgfpathlineto{\pgfqpoint{2.055586in}{1.891220in}}%
\pgfpathlineto{\pgfqpoint{2.108629in}{1.891220in}}%
\pgfpathlineto{\pgfqpoint{2.161673in}{1.891220in}}%
\pgfpathlineto{\pgfqpoint{2.214717in}{1.891220in}}%
\pgfpathlineto{\pgfqpoint{2.267761in}{1.891220in}}%
\pgfpathlineto{\pgfqpoint{2.320805in}{1.891220in}}%
\pgfpathlineto{\pgfqpoint{2.373849in}{1.891220in}}%
\pgfpathlineto{\pgfqpoint{2.426893in}{1.891220in}}%
\pgfpathlineto{\pgfqpoint{2.479937in}{1.891220in}}%
\pgfpathlineto{\pgfqpoint{2.532981in}{1.891220in}}%
\pgfpathlineto{\pgfqpoint{2.586025in}{1.891220in}}%
\pgfpathlineto{\pgfqpoint{2.639069in}{1.891220in}}%
\pgfpathlineto{\pgfqpoint{2.692113in}{1.891220in}}%
\pgfpathlineto{\pgfqpoint{2.745157in}{1.891220in}}%
\pgfpathlineto{\pgfqpoint{2.798201in}{1.891220in}}%
\pgfpathlineto{\pgfqpoint{2.851245in}{1.891220in}}%
\pgfpathlineto{\pgfqpoint{2.904288in}{1.891220in}}%
\pgfpathlineto{\pgfqpoint{2.957332in}{1.891220in}}%
\pgfpathlineto{\pgfqpoint{3.010376in}{1.891220in}}%
\pgfpathlineto{\pgfqpoint{3.063420in}{1.891220in}}%
\pgfpathlineto{\pgfqpoint{3.116464in}{1.891220in}}%
\pgfpathlineto{\pgfqpoint{3.169508in}{1.891220in}}%
\pgfpathlineto{\pgfqpoint{3.222552in}{1.891220in}}%
\pgfpathlineto{\pgfqpoint{3.275596in}{1.891220in}}%
\pgfpathlineto{\pgfqpoint{3.328640in}{1.891220in}}%
\pgfpathlineto{\pgfqpoint{3.381684in}{1.891220in}}%
\pgfpathlineto{\pgfqpoint{3.434728in}{1.891220in}}%
\pgfpathlineto{\pgfqpoint{3.487772in}{1.891220in}}%
\pgfpathlineto{\pgfqpoint{3.540816in}{1.891220in}}%
\pgfpathlineto{\pgfqpoint{3.593860in}{1.891220in}}%
\pgfpathlineto{\pgfqpoint{3.646904in}{1.891220in}}%
\pgfpathlineto{\pgfqpoint{3.699948in}{1.891220in}}%
\pgfpathlineto{\pgfqpoint{3.752991in}{1.891220in}}%
\pgfpathlineto{\pgfqpoint{3.806035in}{1.891220in}}%
\pgfpathlineto{\pgfqpoint{3.859079in}{1.891220in}}%
\pgfpathlineto{\pgfqpoint{3.912123in}{1.891220in}}%
\pgfpathlineto{\pgfqpoint{3.965167in}{1.891220in}}%
\pgfpathlineto{\pgfqpoint{4.018211in}{1.891220in}}%
\pgfpathlineto{\pgfqpoint{4.071255in}{1.891220in}}%
\pgfpathlineto{\pgfqpoint{4.124299in}{1.891220in}}%
\pgfpathlineto{\pgfqpoint{4.177343in}{1.891220in}}%
\pgfpathlineto{\pgfqpoint{4.230387in}{1.891220in}}%
\pgfpathlineto{\pgfqpoint{4.283431in}{1.891220in}}%
\pgfpathlineto{\pgfqpoint{4.336475in}{1.891220in}}%
\pgfpathlineto{\pgfqpoint{4.389519in}{1.891220in}}%
\pgfpathlineto{\pgfqpoint{4.442563in}{1.891220in}}%
\pgfpathlineto{\pgfqpoint{4.495607in}{1.891220in}}%
\pgfpathlineto{\pgfqpoint{4.548650in}{1.891220in}}%
\pgfpathlineto{\pgfqpoint{4.601694in}{1.891220in}}%
\pgfpathlineto{\pgfqpoint{4.654738in}{1.891220in}}%
\pgfpathlineto{\pgfqpoint{4.707782in}{1.891220in}}%
\pgfpathlineto{\pgfqpoint{4.760826in}{1.891220in}}%
\pgfpathlineto{\pgfqpoint{4.813870in}{1.891220in}}%
\pgfpathlineto{\pgfqpoint{4.866914in}{1.891220in}}%
\pgfpathlineto{\pgfqpoint{4.919958in}{1.891220in}}%
\pgfpathlineto{\pgfqpoint{4.973002in}{1.891220in}}%
\pgfpathlineto{\pgfqpoint{5.026046in}{1.891220in}}%
\pgfpathlineto{\pgfqpoint{5.079090in}{1.891220in}}%
\pgfpathlineto{\pgfqpoint{5.132134in}{1.891220in}}%
\pgfpathlineto{\pgfqpoint{5.185178in}{1.891220in}}%
\pgfpathlineto{\pgfqpoint{5.238222in}{1.891220in}}%
\pgfpathlineto{\pgfqpoint{5.291266in}{1.891220in}}%
\pgfpathlineto{\pgfqpoint{5.344309in}{1.891220in}}%
\pgfpathlineto{\pgfqpoint{5.397353in}{1.891220in}}%
\pgfpathlineto{\pgfqpoint{5.450397in}{1.891220in}}%
\pgfpathlineto{\pgfqpoint{5.503441in}{1.891220in}}%
\pgfpathlineto{\pgfqpoint{5.556485in}{1.891220in}}%
\pgfpathlineto{\pgfqpoint{5.609529in}{1.891220in}}%
\pgfpathlineto{\pgfqpoint{5.662573in}{1.891220in}}%
\pgfpathlineto{\pgfqpoint{5.715617in}{1.891220in}}%
\pgfpathlineto{\pgfqpoint{5.768661in}{1.891220in}}%
\pgfpathlineto{\pgfqpoint{5.821705in}{1.891220in}}%
\pgfpathlineto{\pgfqpoint{5.874749in}{1.891220in}}%
\pgfpathlineto{\pgfqpoint{5.927793in}{1.891220in}}%
\pgfpathlineto{\pgfqpoint{5.980837in}{1.891220in}}%
\pgfpathlineto{\pgfqpoint{6.033881in}{1.891220in}}%
\pgfpathlineto{\pgfqpoint{6.086925in}{1.891220in}}%
\pgfpathlineto{\pgfqpoint{6.139969in}{1.891220in}}%
\pgfpathlineto{\pgfqpoint{6.193012in}{1.891220in}}%
\pgfpathlineto{\pgfqpoint{6.246056in}{1.891220in}}%
\pgfpathlineto{\pgfqpoint{6.299100in}{1.891220in}}%
\pgfpathlineto{\pgfqpoint{6.352144in}{1.891220in}}%
\pgfpathlineto{\pgfqpoint{6.405188in}{1.891220in}}%
\pgfpathlineto{\pgfqpoint{6.458232in}{1.891220in}}%
\pgfpathlineto{\pgfqpoint{6.511276in}{1.891220in}}%
\pgfpathlineto{\pgfqpoint{6.564320in}{1.891220in}}%
\pgfpathlineto{\pgfqpoint{6.617364in}{1.891220in}}%
\pgfpathlineto{\pgfqpoint{6.670408in}{1.891220in}}%
\pgfpathlineto{\pgfqpoint{6.723452in}{1.891220in}}%
\pgfpathlineto{\pgfqpoint{6.776496in}{1.891220in}}%
\pgfpathlineto{\pgfqpoint{6.829540in}{1.891220in}}%
\pgfpathlineto{\pgfqpoint{6.882584in}{1.891220in}}%
\pgfpathlineto{\pgfqpoint{6.935628in}{1.891220in}}%
\pgfpathlineto{\pgfqpoint{6.988671in}{1.891220in}}%
\pgfpathlineto{\pgfqpoint{7.041715in}{1.891220in}}%
\pgfpathlineto{\pgfqpoint{7.094759in}{1.891220in}}%
\pgfpathlineto{\pgfqpoint{7.147803in}{1.891220in}}%
\pgfpathlineto{\pgfqpoint{7.200847in}{1.891220in}}%
\pgfpathlineto{\pgfqpoint{7.253891in}{1.891220in}}%
\pgfpathlineto{\pgfqpoint{7.306935in}{1.891220in}}%
\pgfpathlineto{\pgfqpoint{7.359979in}{1.891220in}}%
\pgfpathlineto{\pgfqpoint{7.413023in}{1.891220in}}%
\pgfpathlineto{\pgfqpoint{7.466067in}{1.891220in}}%
\pgfpathlineto{\pgfqpoint{7.519111in}{1.891220in}}%
\pgfpathlineto{\pgfqpoint{7.572155in}{1.891220in}}%
\pgfpathlineto{\pgfqpoint{7.625199in}{1.891220in}}%
\pgfpathlineto{\pgfqpoint{7.678243in}{1.891220in}}%
\pgfpathlineto{\pgfqpoint{7.731287in}{1.891220in}}%
\pgfpathlineto{\pgfqpoint{7.784330in}{1.891220in}}%
\pgfpathlineto{\pgfqpoint{7.837374in}{1.891220in}}%
\pgfpathlineto{\pgfqpoint{7.890418in}{1.891220in}}%
\pgfpathlineto{\pgfqpoint{7.943462in}{1.891220in}}%
\pgfpathlineto{\pgfqpoint{7.996506in}{1.891220in}}%
\pgfpathlineto{\pgfqpoint{8.049550in}{1.891220in}}%
\pgfpathlineto{\pgfqpoint{8.102594in}{1.891220in}}%
\pgfpathlineto{\pgfqpoint{8.155638in}{1.891220in}}%
\pgfpathlineto{\pgfqpoint{8.208682in}{1.891220in}}%
\pgfpathlineto{\pgfqpoint{8.261726in}{1.891220in}}%
\pgfpathlineto{\pgfqpoint{8.314770in}{1.891220in}}%
\pgfpathlineto{\pgfqpoint{8.367814in}{1.891220in}}%
\pgfpathlineto{\pgfqpoint{8.420858in}{1.891220in}}%
\pgfpathlineto{\pgfqpoint{8.473902in}{1.891220in}}%
\pgfpathlineto{\pgfqpoint{8.526946in}{1.891220in}}%
\pgfpathlineto{\pgfqpoint{8.579990in}{1.891220in}}%
\pgfpathlineto{\pgfqpoint{8.633033in}{1.891220in}}%
\pgfpathlineto{\pgfqpoint{8.686077in}{1.891220in}}%
\pgfpathlineto{\pgfqpoint{8.739121in}{1.891220in}}%
\pgfpathlineto{\pgfqpoint{8.792165in}{1.891220in}}%
\pgfpathlineto{\pgfqpoint{8.845209in}{1.891220in}}%
\pgfpathlineto{\pgfqpoint{8.898253in}{1.891220in}}%
\pgfpathlineto{\pgfqpoint{8.951297in}{1.891220in}}%
\pgfpathlineto{\pgfqpoint{9.004341in}{1.891220in}}%
\pgfpathlineto{\pgfqpoint{9.057385in}{1.891220in}}%
\pgfpathlineto{\pgfqpoint{9.110429in}{1.891220in}}%
\pgfpathlineto{\pgfqpoint{9.163473in}{1.891220in}}%
\pgfpathlineto{\pgfqpoint{9.216517in}{1.891220in}}%
\pgfpathlineto{\pgfqpoint{9.269561in}{1.891220in}}%
\pgfpathlineto{\pgfqpoint{9.322605in}{1.891220in}}%
\pgfpathlineto{\pgfqpoint{9.375649in}{1.891220in}}%
\pgfpathlineto{\pgfqpoint{9.428692in}{1.891220in}}%
\pgfpathlineto{\pgfqpoint{9.481736in}{1.891220in}}%
\pgfpathlineto{\pgfqpoint{9.534780in}{1.891220in}}%
\pgfpathlineto{\pgfqpoint{9.587824in}{1.891220in}}%
\pgfpathlineto{\pgfqpoint{9.640868in}{1.891220in}}%
\pgfpathlineto{\pgfqpoint{9.693912in}{1.891220in}}%
\pgfpathlineto{\pgfqpoint{9.746956in}{1.891220in}}%
\pgfpathlineto{\pgfqpoint{9.800000in}{1.891220in}}%
\pgfpathlineto{\pgfqpoint{9.800000in}{1.891220in}}%
\pgfpathlineto{\pgfqpoint{9.800000in}{1.891220in}}%
\pgfpathlineto{\pgfqpoint{9.746956in}{1.891220in}}%
\pgfpathlineto{\pgfqpoint{9.693912in}{1.891220in}}%
\pgfpathlineto{\pgfqpoint{9.640868in}{1.617282in}}%
\pgfpathlineto{\pgfqpoint{9.587824in}{1.891220in}}%
\pgfpathlineto{\pgfqpoint{9.534780in}{1.891220in}}%
\pgfpathlineto{\pgfqpoint{9.481736in}{1.543384in}}%
\pgfpathlineto{\pgfqpoint{9.428692in}{1.543384in}}%
\pgfpathlineto{\pgfqpoint{9.375649in}{1.891220in}}%
\pgfpathlineto{\pgfqpoint{9.322605in}{1.891220in}}%
\pgfpathlineto{\pgfqpoint{9.269561in}{1.891220in}}%
\pgfpathlineto{\pgfqpoint{9.216517in}{1.891220in}}%
\pgfpathlineto{\pgfqpoint{9.163473in}{1.891220in}}%
\pgfpathlineto{\pgfqpoint{9.110429in}{1.543384in}}%
\pgfpathlineto{\pgfqpoint{9.057385in}{1.543384in}}%
\pgfpathlineto{\pgfqpoint{9.004341in}{1.644141in}}%
\pgfpathlineto{\pgfqpoint{8.951297in}{1.543384in}}%
\pgfpathlineto{\pgfqpoint{8.898253in}{1.891220in}}%
\pgfpathlineto{\pgfqpoint{8.845209in}{1.891220in}}%
\pgfpathlineto{\pgfqpoint{8.792165in}{1.755774in}}%
\pgfpathlineto{\pgfqpoint{8.739121in}{1.891220in}}%
\pgfpathlineto{\pgfqpoint{8.686077in}{1.891220in}}%
\pgfpathlineto{\pgfqpoint{8.633033in}{1.891220in}}%
\pgfpathlineto{\pgfqpoint{8.579990in}{1.891220in}}%
\pgfpathlineto{\pgfqpoint{8.526946in}{1.543384in}}%
\pgfpathlineto{\pgfqpoint{8.473902in}{1.543384in}}%
\pgfpathlineto{\pgfqpoint{8.420858in}{1.891220in}}%
\pgfpathlineto{\pgfqpoint{8.367814in}{1.543384in}}%
\pgfpathlineto{\pgfqpoint{8.314770in}{1.891220in}}%
\pgfpathlineto{\pgfqpoint{8.261726in}{1.891220in}}%
\pgfpathlineto{\pgfqpoint{8.208682in}{1.543384in}}%
\pgfpathlineto{\pgfqpoint{8.155638in}{1.543384in}}%
\pgfpathlineto{\pgfqpoint{8.102594in}{1.891220in}}%
\pgfpathlineto{\pgfqpoint{8.049550in}{1.891220in}}%
\pgfpathlineto{\pgfqpoint{7.996506in}{1.543384in}}%
\pgfpathlineto{\pgfqpoint{7.943462in}{1.543384in}}%
\pgfpathlineto{\pgfqpoint{7.890418in}{1.891220in}}%
\pgfpathlineto{\pgfqpoint{7.837374in}{1.543384in}}%
\pgfpathlineto{\pgfqpoint{7.784330in}{1.567858in}}%
\pgfpathlineto{\pgfqpoint{7.731287in}{1.891220in}}%
\pgfpathlineto{\pgfqpoint{7.678243in}{1.849746in}}%
\pgfpathlineto{\pgfqpoint{7.625199in}{1.891220in}}%
\pgfpathlineto{\pgfqpoint{7.572155in}{1.891220in}}%
\pgfpathlineto{\pgfqpoint{7.519111in}{1.543384in}}%
\pgfpathlineto{\pgfqpoint{7.466067in}{1.648522in}}%
\pgfpathlineto{\pgfqpoint{7.413023in}{1.891220in}}%
\pgfpathlineto{\pgfqpoint{7.359979in}{1.891220in}}%
\pgfpathlineto{\pgfqpoint{7.306935in}{1.746843in}}%
\pgfpathlineto{\pgfqpoint{7.253891in}{1.543384in}}%
\pgfpathlineto{\pgfqpoint{7.200847in}{1.891220in}}%
\pgfpathlineto{\pgfqpoint{7.147803in}{1.891220in}}%
\pgfpathlineto{\pgfqpoint{7.094759in}{1.543384in}}%
\pgfpathlineto{\pgfqpoint{7.041715in}{1.891220in}}%
\pgfpathlineto{\pgfqpoint{6.988671in}{1.891220in}}%
\pgfpathlineto{\pgfqpoint{6.935628in}{1.891220in}}%
\pgfpathlineto{\pgfqpoint{6.882584in}{1.891220in}}%
\pgfpathlineto{\pgfqpoint{6.829540in}{1.891220in}}%
\pgfpathlineto{\pgfqpoint{6.776496in}{1.543384in}}%
\pgfpathlineto{\pgfqpoint{6.723452in}{1.871334in}}%
\pgfpathlineto{\pgfqpoint{6.670408in}{1.891220in}}%
\pgfpathlineto{\pgfqpoint{6.617364in}{1.891220in}}%
\pgfpathlineto{\pgfqpoint{6.564320in}{1.891220in}}%
\pgfpathlineto{\pgfqpoint{6.511276in}{1.891220in}}%
\pgfpathlineto{\pgfqpoint{6.458232in}{1.891220in}}%
\pgfpathlineto{\pgfqpoint{6.405188in}{1.891220in}}%
\pgfpathlineto{\pgfqpoint{6.352144in}{1.891220in}}%
\pgfpathlineto{\pgfqpoint{6.299100in}{1.798179in}}%
\pgfpathlineto{\pgfqpoint{6.246056in}{1.891220in}}%
\pgfpathlineto{\pgfqpoint{6.193012in}{1.891220in}}%
\pgfpathlineto{\pgfqpoint{6.139969in}{1.731521in}}%
\pgfpathlineto{\pgfqpoint{6.086925in}{1.543384in}}%
\pgfpathlineto{\pgfqpoint{6.033881in}{1.678405in}}%
\pgfpathlineto{\pgfqpoint{5.980837in}{1.891220in}}%
\pgfpathlineto{\pgfqpoint{5.927793in}{1.891220in}}%
\pgfpathlineto{\pgfqpoint{5.874749in}{1.891220in}}%
\pgfpathlineto{\pgfqpoint{5.821705in}{1.891220in}}%
\pgfpathlineto{\pgfqpoint{5.768661in}{1.891220in}}%
\pgfpathlineto{\pgfqpoint{5.715617in}{1.543384in}}%
\pgfpathlineto{\pgfqpoint{5.662573in}{1.891220in}}%
\pgfpathlineto{\pgfqpoint{5.609529in}{1.891220in}}%
\pgfpathlineto{\pgfqpoint{5.556485in}{1.549519in}}%
\pgfpathlineto{\pgfqpoint{5.503441in}{1.543384in}}%
\pgfpathlineto{\pgfqpoint{5.450397in}{1.891220in}}%
\pgfpathlineto{\pgfqpoint{5.397353in}{1.891220in}}%
\pgfpathlineto{\pgfqpoint{5.344309in}{1.891220in}}%
\pgfpathlineto{\pgfqpoint{5.291266in}{1.543384in}}%
\pgfpathlineto{\pgfqpoint{5.238222in}{1.891220in}}%
\pgfpathlineto{\pgfqpoint{5.185178in}{1.891220in}}%
\pgfpathlineto{\pgfqpoint{5.132134in}{1.891220in}}%
\pgfpathlineto{\pgfqpoint{5.079090in}{1.543384in}}%
\pgfpathlineto{\pgfqpoint{5.026046in}{1.891220in}}%
\pgfpathlineto{\pgfqpoint{4.973002in}{1.543384in}}%
\pgfpathlineto{\pgfqpoint{4.919958in}{1.891220in}}%
\pgfpathlineto{\pgfqpoint{4.866914in}{1.891220in}}%
\pgfpathlineto{\pgfqpoint{4.813870in}{1.543384in}}%
\pgfpathlineto{\pgfqpoint{4.760826in}{1.543384in}}%
\pgfpathlineto{\pgfqpoint{4.707782in}{1.891220in}}%
\pgfpathlineto{\pgfqpoint{4.654738in}{1.543384in}}%
\pgfpathlineto{\pgfqpoint{4.601694in}{1.891220in}}%
\pgfpathlineto{\pgfqpoint{4.548650in}{1.543384in}}%
\pgfpathlineto{\pgfqpoint{4.495607in}{1.891220in}}%
\pgfpathlineto{\pgfqpoint{4.442563in}{1.580227in}}%
\pgfpathlineto{\pgfqpoint{4.389519in}{1.543384in}}%
\pgfpathlineto{\pgfqpoint{4.336475in}{1.891220in}}%
\pgfpathlineto{\pgfqpoint{4.283431in}{1.543384in}}%
\pgfpathlineto{\pgfqpoint{4.230387in}{1.543384in}}%
\pgfpathlineto{\pgfqpoint{4.177343in}{1.891220in}}%
\pgfpathlineto{\pgfqpoint{4.124299in}{1.770677in}}%
\pgfpathlineto{\pgfqpoint{4.071255in}{1.891220in}}%
\pgfpathlineto{\pgfqpoint{4.018211in}{1.891220in}}%
\pgfpathlineto{\pgfqpoint{3.965167in}{1.891220in}}%
\pgfpathlineto{\pgfqpoint{3.912123in}{1.891220in}}%
\pgfpathlineto{\pgfqpoint{3.859079in}{1.891220in}}%
\pgfpathlineto{\pgfqpoint{3.806035in}{1.891220in}}%
\pgfpathlineto{\pgfqpoint{3.752991in}{1.891220in}}%
\pgfpathlineto{\pgfqpoint{3.699948in}{1.891220in}}%
\pgfpathlineto{\pgfqpoint{3.646904in}{1.891220in}}%
\pgfpathlineto{\pgfqpoint{3.593860in}{1.891220in}}%
\pgfpathlineto{\pgfqpoint{3.540816in}{1.891220in}}%
\pgfpathlineto{\pgfqpoint{3.487772in}{1.891220in}}%
\pgfpathlineto{\pgfqpoint{3.434728in}{1.543384in}}%
\pgfpathlineto{\pgfqpoint{3.381684in}{1.794175in}}%
\pgfpathlineto{\pgfqpoint{3.328640in}{1.891220in}}%
\pgfpathlineto{\pgfqpoint{3.275596in}{1.891220in}}%
\pgfpathlineto{\pgfqpoint{3.222552in}{1.891220in}}%
\pgfpathlineto{\pgfqpoint{3.169508in}{1.891220in}}%
\pgfpathlineto{\pgfqpoint{3.116464in}{1.891220in}}%
\pgfpathlineto{\pgfqpoint{3.063420in}{1.891220in}}%
\pgfpathlineto{\pgfqpoint{3.010376in}{1.571272in}}%
\pgfpathlineto{\pgfqpoint{2.957332in}{1.891220in}}%
\pgfpathlineto{\pgfqpoint{2.904288in}{1.543384in}}%
\pgfpathlineto{\pgfqpoint{2.851245in}{1.891220in}}%
\pgfpathlineto{\pgfqpoint{2.798201in}{1.891220in}}%
\pgfpathlineto{\pgfqpoint{2.745157in}{1.891220in}}%
\pgfpathlineto{\pgfqpoint{2.692113in}{1.891220in}}%
\pgfpathlineto{\pgfqpoint{2.639069in}{1.891220in}}%
\pgfpathlineto{\pgfqpoint{2.586025in}{1.891220in}}%
\pgfpathlineto{\pgfqpoint{2.532981in}{1.891220in}}%
\pgfpathlineto{\pgfqpoint{2.479937in}{1.891220in}}%
\pgfpathlineto{\pgfqpoint{2.426893in}{1.891220in}}%
\pgfpathlineto{\pgfqpoint{2.373849in}{1.891220in}}%
\pgfpathlineto{\pgfqpoint{2.320805in}{1.543384in}}%
\pgfpathlineto{\pgfqpoint{2.267761in}{1.631196in}}%
\pgfpathlineto{\pgfqpoint{2.214717in}{1.891220in}}%
\pgfpathlineto{\pgfqpoint{2.161673in}{1.891220in}}%
\pgfpathlineto{\pgfqpoint{2.108629in}{1.569408in}}%
\pgfpathlineto{\pgfqpoint{2.055586in}{1.891220in}}%
\pgfpathlineto{\pgfqpoint{2.002542in}{1.891220in}}%
\pgfpathlineto{\pgfqpoint{1.949498in}{1.543384in}}%
\pgfpathlineto{\pgfqpoint{1.896454in}{1.891220in}}%
\pgfpathlineto{\pgfqpoint{1.843410in}{1.891220in}}%
\pgfpathlineto{\pgfqpoint{1.790366in}{1.543384in}}%
\pgfpathlineto{\pgfqpoint{1.737322in}{1.543384in}}%
\pgfpathlineto{\pgfqpoint{1.684278in}{1.880012in}}%
\pgfpathlineto{\pgfqpoint{1.631234in}{1.891220in}}%
\pgfpathlineto{\pgfqpoint{1.578190in}{1.891220in}}%
\pgfpathlineto{\pgfqpoint{1.525146in}{1.891220in}}%
\pgfpathlineto{\pgfqpoint{1.472102in}{1.843091in}}%
\pgfpathlineto{\pgfqpoint{1.419058in}{1.891220in}}%
\pgfpathlineto{\pgfqpoint{1.366014in}{1.891220in}}%
\pgfpathlineto{\pgfqpoint{1.312970in}{1.891220in}}%
\pgfpathlineto{\pgfqpoint{1.259927in}{1.891220in}}%
\pgfpathlineto{\pgfqpoint{1.206883in}{1.891220in}}%
\pgfpathlineto{\pgfqpoint{1.153839in}{1.891220in}}%
\pgfpathlineto{\pgfqpoint{1.100795in}{1.891220in}}%
\pgfpathlineto{\pgfqpoint{1.047751in}{1.891220in}}%
\pgfpathlineto{\pgfqpoint{0.994707in}{1.543384in}}%
\pgfpathlineto{\pgfqpoint{0.941663in}{1.891220in}}%
\pgfpathlineto{\pgfqpoint{0.941663in}{1.891220in}}%
\pgfpathclose%
\pgfusepath{stroke,fill}%
}%
\begin{pgfscope}%
\pgfsys@transformshift{0.000000in}{0.000000in}%
\pgfsys@useobject{currentmarker}{}%
\end{pgfscope}%
\end{pgfscope}%
\begin{pgfscope}%
\pgfpathrectangle{\pgfqpoint{0.941663in}{0.670138in}}{\pgfqpoint{8.858337in}{3.465625in}}%
\pgfusepath{clip}%
\pgfsetrectcap%
\pgfsetroundjoin%
\pgfsetlinewidth{1.505625pt}%
\definecolor{currentstroke}{rgb}{0.501961,0.501961,0.501961}%
\pgfsetstrokecolor{currentstroke}%
\pgfsetdash{}{0pt}%
\pgfpathmoveto{\pgfqpoint{0.941663in}{0.942172in}}%
\pgfpathlineto{\pgfqpoint{0.994707in}{1.206100in}}%
\pgfpathlineto{\pgfqpoint{1.047751in}{0.938123in}}%
\pgfpathlineto{\pgfqpoint{1.100795in}{0.907342in}}%
\pgfpathlineto{\pgfqpoint{1.153839in}{0.893679in}}%
\pgfpathlineto{\pgfqpoint{1.206883in}{1.810992in}}%
\pgfpathlineto{\pgfqpoint{1.259927in}{1.891220in}}%
\pgfpathlineto{\pgfqpoint{1.312970in}{1.856216in}}%
\pgfpathlineto{\pgfqpoint{1.366014in}{0.829335in}}%
\pgfpathlineto{\pgfqpoint{1.419058in}{0.903552in}}%
\pgfpathlineto{\pgfqpoint{1.472102in}{0.908915in}}%
\pgfpathlineto{\pgfqpoint{1.525146in}{0.989277in}}%
\pgfpathlineto{\pgfqpoint{1.578190in}{0.966407in}}%
\pgfpathlineto{\pgfqpoint{1.631234in}{1.891220in}}%
\pgfpathlineto{\pgfqpoint{1.684278in}{1.559075in}}%
\pgfpathlineto{\pgfqpoint{1.737322in}{1.131392in}}%
\pgfpathlineto{\pgfqpoint{1.790366in}{1.113813in}}%
\pgfpathlineto{\pgfqpoint{1.843410in}{1.891220in}}%
\pgfpathlineto{\pgfqpoint{1.896454in}{1.891220in}}%
\pgfpathlineto{\pgfqpoint{1.949498in}{1.087124in}}%
\pgfpathlineto{\pgfqpoint{2.002542in}{1.891220in}}%
\pgfpathlineto{\pgfqpoint{2.055586in}{1.097929in}}%
\pgfpathlineto{\pgfqpoint{2.108629in}{1.481664in}}%
\pgfpathlineto{\pgfqpoint{2.161673in}{0.987578in}}%
\pgfpathlineto{\pgfqpoint{2.214717in}{1.891220in}}%
\pgfpathlineto{\pgfqpoint{2.267761in}{1.182556in}}%
\pgfpathlineto{\pgfqpoint{2.320805in}{0.918169in}}%
\pgfpathlineto{\pgfqpoint{2.373849in}{1.891220in}}%
\pgfpathlineto{\pgfqpoint{2.426893in}{1.891220in}}%
\pgfpathlineto{\pgfqpoint{2.479937in}{1.038089in}}%
\pgfpathlineto{\pgfqpoint{2.532981in}{0.893845in}}%
\pgfpathlineto{\pgfqpoint{2.586025in}{0.872034in}}%
\pgfpathlineto{\pgfqpoint{2.639069in}{0.867540in}}%
\pgfpathlineto{\pgfqpoint{2.692113in}{0.837338in}}%
\pgfpathlineto{\pgfqpoint{2.745157in}{0.908262in}}%
\pgfpathlineto{\pgfqpoint{2.798201in}{0.950657in}}%
\pgfpathlineto{\pgfqpoint{2.851245in}{1.781408in}}%
\pgfpathlineto{\pgfqpoint{2.904288in}{0.997564in}}%
\pgfpathlineto{\pgfqpoint{2.957332in}{1.891220in}}%
\pgfpathlineto{\pgfqpoint{3.010376in}{1.083328in}}%
\pgfpathlineto{\pgfqpoint{3.063420in}{1.891220in}}%
\pgfpathlineto{\pgfqpoint{3.116464in}{1.557584in}}%
\pgfpathlineto{\pgfqpoint{3.169508in}{1.115148in}}%
\pgfpathlineto{\pgfqpoint{3.222552in}{1.145671in}}%
\pgfpathlineto{\pgfqpoint{3.275596in}{1.061811in}}%
\pgfpathlineto{\pgfqpoint{3.328640in}{1.120386in}}%
\pgfpathlineto{\pgfqpoint{3.381684in}{1.666472in}}%
\pgfpathlineto{\pgfqpoint{3.434728in}{0.972772in}}%
\pgfpathlineto{\pgfqpoint{3.487772in}{1.891220in}}%
\pgfpathlineto{\pgfqpoint{3.540816in}{1.891220in}}%
\pgfpathlineto{\pgfqpoint{3.593860in}{1.388474in}}%
\pgfpathlineto{\pgfqpoint{3.646904in}{0.911072in}}%
\pgfpathlineto{\pgfqpoint{3.699948in}{0.867222in}}%
\pgfpathlineto{\pgfqpoint{3.752991in}{1.454919in}}%
\pgfpathlineto{\pgfqpoint{3.806035in}{0.907473in}}%
\pgfpathlineto{\pgfqpoint{3.859079in}{1.517418in}}%
\pgfpathlineto{\pgfqpoint{3.912123in}{1.426054in}}%
\pgfpathlineto{\pgfqpoint{3.965167in}{1.724076in}}%
\pgfpathlineto{\pgfqpoint{4.018211in}{0.913448in}}%
\pgfpathlineto{\pgfqpoint{4.071255in}{0.995539in}}%
\pgfpathlineto{\pgfqpoint{4.124299in}{1.123707in}}%
\pgfpathlineto{\pgfqpoint{4.177343in}{1.039428in}}%
\pgfpathlineto{\pgfqpoint{4.230387in}{1.077535in}}%
\pgfpathlineto{\pgfqpoint{4.283431in}{1.093254in}}%
\pgfpathlineto{\pgfqpoint{4.336475in}{1.891220in}}%
\pgfpathlineto{\pgfqpoint{4.389519in}{1.243678in}}%
\pgfpathlineto{\pgfqpoint{4.442563in}{1.580227in}}%
\pgfpathlineto{\pgfqpoint{4.495607in}{1.891220in}}%
\pgfpathlineto{\pgfqpoint{4.548650in}{1.437632in}}%
\pgfpathlineto{\pgfqpoint{4.601694in}{1.891220in}}%
\pgfpathlineto{\pgfqpoint{4.654738in}{1.094865in}}%
\pgfpathlineto{\pgfqpoint{4.707782in}{1.891220in}}%
\pgfpathlineto{\pgfqpoint{4.760826in}{1.287681in}}%
\pgfpathlineto{\pgfqpoint{4.813870in}{0.975129in}}%
\pgfpathlineto{\pgfqpoint{4.866914in}{1.891220in}}%
\pgfpathlineto{\pgfqpoint{4.919958in}{1.891220in}}%
\pgfpathlineto{\pgfqpoint{4.973002in}{0.856126in}}%
\pgfpathlineto{\pgfqpoint{5.026046in}{1.891220in}}%
\pgfpathlineto{\pgfqpoint{5.079090in}{0.827666in}}%
\pgfpathlineto{\pgfqpoint{5.132134in}{1.891220in}}%
\pgfpathlineto{\pgfqpoint{5.238222in}{1.891220in}}%
\pgfpathlineto{\pgfqpoint{5.291266in}{0.975653in}}%
\pgfpathlineto{\pgfqpoint{5.344309in}{1.891220in}}%
\pgfpathlineto{\pgfqpoint{5.450397in}{1.891220in}}%
\pgfpathlineto{\pgfqpoint{5.503441in}{1.116833in}}%
\pgfpathlineto{\pgfqpoint{5.556485in}{1.549519in}}%
\pgfpathlineto{\pgfqpoint{5.609529in}{1.891220in}}%
\pgfpathlineto{\pgfqpoint{5.662573in}{1.891220in}}%
\pgfpathlineto{\pgfqpoint{5.715617in}{1.122626in}}%
\pgfpathlineto{\pgfqpoint{5.768661in}{1.891220in}}%
\pgfpathlineto{\pgfqpoint{5.927793in}{1.891220in}}%
\pgfpathlineto{\pgfqpoint{5.980837in}{1.046269in}}%
\pgfpathlineto{\pgfqpoint{6.033881in}{1.678405in}}%
\pgfpathlineto{\pgfqpoint{6.086925in}{0.975132in}}%
\pgfpathlineto{\pgfqpoint{6.139969in}{0.956658in}}%
\pgfpathlineto{\pgfqpoint{6.193012in}{1.891220in}}%
\pgfpathlineto{\pgfqpoint{6.246056in}{1.891220in}}%
\pgfpathlineto{\pgfqpoint{6.299100in}{0.877160in}}%
\pgfpathlineto{\pgfqpoint{6.352144in}{1.833058in}}%
\pgfpathlineto{\pgfqpoint{6.405188in}{1.891220in}}%
\pgfpathlineto{\pgfqpoint{6.458232in}{1.413279in}}%
\pgfpathlineto{\pgfqpoint{6.511276in}{0.946754in}}%
\pgfpathlineto{\pgfqpoint{6.564320in}{0.943104in}}%
\pgfpathlineto{\pgfqpoint{6.617364in}{1.010061in}}%
\pgfpathlineto{\pgfqpoint{6.670408in}{1.040003in}}%
\pgfpathlineto{\pgfqpoint{6.723452in}{1.086085in}}%
\pgfpathlineto{\pgfqpoint{6.776496in}{1.330310in}}%
\pgfpathlineto{\pgfqpoint{6.829540in}{1.098052in}}%
\pgfpathlineto{\pgfqpoint{6.882584in}{1.891220in}}%
\pgfpathlineto{\pgfqpoint{6.935628in}{1.091705in}}%
\pgfpathlineto{\pgfqpoint{6.988671in}{1.684508in}}%
\pgfpathlineto{\pgfqpoint{7.041715in}{1.110968in}}%
\pgfpathlineto{\pgfqpoint{7.094759in}{1.134019in}}%
\pgfpathlineto{\pgfqpoint{7.147803in}{1.115813in}}%
\pgfpathlineto{\pgfqpoint{7.200847in}{1.891220in}}%
\pgfpathlineto{\pgfqpoint{7.253891in}{1.460365in}}%
\pgfpathlineto{\pgfqpoint{7.306935in}{1.095703in}}%
\pgfpathlineto{\pgfqpoint{7.359979in}{1.891220in}}%
\pgfpathlineto{\pgfqpoint{7.413023in}{1.891220in}}%
\pgfpathlineto{\pgfqpoint{7.466067in}{1.483365in}}%
\pgfpathlineto{\pgfqpoint{7.519111in}{1.410524in}}%
\pgfpathlineto{\pgfqpoint{7.572155in}{1.891220in}}%
\pgfpathlineto{\pgfqpoint{7.625199in}{1.891220in}}%
\pgfpathlineto{\pgfqpoint{7.678243in}{1.849746in}}%
\pgfpathlineto{\pgfqpoint{7.731287in}{0.908937in}}%
\pgfpathlineto{\pgfqpoint{7.784330in}{0.944968in}}%
\pgfpathlineto{\pgfqpoint{7.837374in}{0.953219in}}%
\pgfpathlineto{\pgfqpoint{7.890418in}{0.994522in}}%
\pgfpathlineto{\pgfqpoint{7.943462in}{0.992268in}}%
\pgfpathlineto{\pgfqpoint{7.996506in}{1.070971in}}%
\pgfpathlineto{\pgfqpoint{8.049550in}{1.891220in}}%
\pgfpathlineto{\pgfqpoint{8.102594in}{1.891220in}}%
\pgfpathlineto{\pgfqpoint{8.155638in}{1.137608in}}%
\pgfpathlineto{\pgfqpoint{8.208682in}{1.131748in}}%
\pgfpathlineto{\pgfqpoint{8.261726in}{1.891220in}}%
\pgfpathlineto{\pgfqpoint{8.314770in}{1.891220in}}%
\pgfpathlineto{\pgfqpoint{8.367814in}{1.114165in}}%
\pgfpathlineto{\pgfqpoint{8.420858in}{1.891220in}}%
\pgfpathlineto{\pgfqpoint{8.473902in}{1.054943in}}%
\pgfpathlineto{\pgfqpoint{8.526946in}{1.026802in}}%
\pgfpathlineto{\pgfqpoint{8.579990in}{1.891220in}}%
\pgfpathlineto{\pgfqpoint{8.739121in}{1.891220in}}%
\pgfpathlineto{\pgfqpoint{8.792165in}{1.755774in}}%
\pgfpathlineto{\pgfqpoint{8.845209in}{1.891220in}}%
\pgfpathlineto{\pgfqpoint{8.898253in}{1.891220in}}%
\pgfpathlineto{\pgfqpoint{8.951297in}{0.878008in}}%
\pgfpathlineto{\pgfqpoint{9.004341in}{1.644141in}}%
\pgfpathlineto{\pgfqpoint{9.057385in}{0.962369in}}%
\pgfpathlineto{\pgfqpoint{9.110429in}{0.997462in}}%
\pgfpathlineto{\pgfqpoint{9.163473in}{1.891220in}}%
\pgfpathlineto{\pgfqpoint{9.375649in}{1.891220in}}%
\pgfpathlineto{\pgfqpoint{9.428692in}{1.153825in}}%
\pgfpathlineto{\pgfqpoint{9.481736in}{1.200625in}}%
\pgfpathlineto{\pgfqpoint{9.534780in}{1.891220in}}%
\pgfpathlineto{\pgfqpoint{9.587824in}{1.891220in}}%
\pgfpathlineto{\pgfqpoint{9.640868in}{1.617282in}}%
\pgfpathlineto{\pgfqpoint{9.693912in}{1.891220in}}%
\pgfpathlineto{\pgfqpoint{9.800000in}{1.891220in}}%
\pgfpathlineto{\pgfqpoint{9.800000in}{1.891220in}}%
\pgfusepath{stroke}%
\end{pgfscope}%
\begin{pgfscope}%
\pgfpathrectangle{\pgfqpoint{0.941663in}{0.670138in}}{\pgfqpoint{8.858337in}{3.465625in}}%
\pgfusepath{clip}%
\pgfsetbuttcap%
\pgfsetroundjoin%
\definecolor{currentfill}{rgb}{0.501961,0.501961,0.501961}%
\pgfsetfillcolor{currentfill}%
\pgfsetlinewidth{1.003750pt}%
\definecolor{currentstroke}{rgb}{0.501961,0.501961,0.501961}%
\pgfsetstrokecolor{currentstroke}%
\pgfsetdash{}{0pt}%
\pgfsys@defobject{currentmarker}{\pgfqpoint{0.941663in}{0.827666in}}{\pgfqpoint{9.800000in}{1.891220in}}{%
\pgfpathmoveto{\pgfqpoint{0.941663in}{0.942172in}}%
\pgfpathlineto{\pgfqpoint{0.941663in}{1.891220in}}%
\pgfpathlineto{\pgfqpoint{0.994707in}{1.543384in}}%
\pgfpathlineto{\pgfqpoint{1.047751in}{1.891220in}}%
\pgfpathlineto{\pgfqpoint{1.100795in}{1.891220in}}%
\pgfpathlineto{\pgfqpoint{1.153839in}{1.891220in}}%
\pgfpathlineto{\pgfqpoint{1.206883in}{1.891220in}}%
\pgfpathlineto{\pgfqpoint{1.259927in}{1.891220in}}%
\pgfpathlineto{\pgfqpoint{1.312970in}{1.891220in}}%
\pgfpathlineto{\pgfqpoint{1.366014in}{1.891220in}}%
\pgfpathlineto{\pgfqpoint{1.419058in}{1.891220in}}%
\pgfpathlineto{\pgfqpoint{1.472102in}{1.843091in}}%
\pgfpathlineto{\pgfqpoint{1.525146in}{1.891220in}}%
\pgfpathlineto{\pgfqpoint{1.578190in}{1.891220in}}%
\pgfpathlineto{\pgfqpoint{1.631234in}{1.891220in}}%
\pgfpathlineto{\pgfqpoint{1.684278in}{1.880012in}}%
\pgfpathlineto{\pgfqpoint{1.737322in}{1.543384in}}%
\pgfpathlineto{\pgfqpoint{1.790366in}{1.543384in}}%
\pgfpathlineto{\pgfqpoint{1.843410in}{1.891220in}}%
\pgfpathlineto{\pgfqpoint{1.896454in}{1.891220in}}%
\pgfpathlineto{\pgfqpoint{1.949498in}{1.543384in}}%
\pgfpathlineto{\pgfqpoint{2.002542in}{1.891220in}}%
\pgfpathlineto{\pgfqpoint{2.055586in}{1.891220in}}%
\pgfpathlineto{\pgfqpoint{2.108629in}{1.569408in}}%
\pgfpathlineto{\pgfqpoint{2.161673in}{1.891220in}}%
\pgfpathlineto{\pgfqpoint{2.214717in}{1.891220in}}%
\pgfpathlineto{\pgfqpoint{2.267761in}{1.631196in}}%
\pgfpathlineto{\pgfqpoint{2.320805in}{1.543384in}}%
\pgfpathlineto{\pgfqpoint{2.373849in}{1.891220in}}%
\pgfpathlineto{\pgfqpoint{2.426893in}{1.891220in}}%
\pgfpathlineto{\pgfqpoint{2.479937in}{1.891220in}}%
\pgfpathlineto{\pgfqpoint{2.532981in}{1.891220in}}%
\pgfpathlineto{\pgfqpoint{2.586025in}{1.891220in}}%
\pgfpathlineto{\pgfqpoint{2.639069in}{1.891220in}}%
\pgfpathlineto{\pgfqpoint{2.692113in}{1.891220in}}%
\pgfpathlineto{\pgfqpoint{2.745157in}{1.891220in}}%
\pgfpathlineto{\pgfqpoint{2.798201in}{1.891220in}}%
\pgfpathlineto{\pgfqpoint{2.851245in}{1.891220in}}%
\pgfpathlineto{\pgfqpoint{2.904288in}{1.543384in}}%
\pgfpathlineto{\pgfqpoint{2.957332in}{1.891220in}}%
\pgfpathlineto{\pgfqpoint{3.010376in}{1.571272in}}%
\pgfpathlineto{\pgfqpoint{3.063420in}{1.891220in}}%
\pgfpathlineto{\pgfqpoint{3.116464in}{1.891220in}}%
\pgfpathlineto{\pgfqpoint{3.169508in}{1.891220in}}%
\pgfpathlineto{\pgfqpoint{3.222552in}{1.891220in}}%
\pgfpathlineto{\pgfqpoint{3.275596in}{1.891220in}}%
\pgfpathlineto{\pgfqpoint{3.328640in}{1.891220in}}%
\pgfpathlineto{\pgfqpoint{3.381684in}{1.794175in}}%
\pgfpathlineto{\pgfqpoint{3.434728in}{1.543384in}}%
\pgfpathlineto{\pgfqpoint{3.487772in}{1.891220in}}%
\pgfpathlineto{\pgfqpoint{3.540816in}{1.891220in}}%
\pgfpathlineto{\pgfqpoint{3.593860in}{1.891220in}}%
\pgfpathlineto{\pgfqpoint{3.646904in}{1.891220in}}%
\pgfpathlineto{\pgfqpoint{3.699948in}{1.891220in}}%
\pgfpathlineto{\pgfqpoint{3.752991in}{1.891220in}}%
\pgfpathlineto{\pgfqpoint{3.806035in}{1.891220in}}%
\pgfpathlineto{\pgfqpoint{3.859079in}{1.891220in}}%
\pgfpathlineto{\pgfqpoint{3.912123in}{1.891220in}}%
\pgfpathlineto{\pgfqpoint{3.965167in}{1.891220in}}%
\pgfpathlineto{\pgfqpoint{4.018211in}{1.891220in}}%
\pgfpathlineto{\pgfqpoint{4.071255in}{1.891220in}}%
\pgfpathlineto{\pgfqpoint{4.124299in}{1.770677in}}%
\pgfpathlineto{\pgfqpoint{4.177343in}{1.891220in}}%
\pgfpathlineto{\pgfqpoint{4.230387in}{1.543384in}}%
\pgfpathlineto{\pgfqpoint{4.283431in}{1.543384in}}%
\pgfpathlineto{\pgfqpoint{4.336475in}{1.891220in}}%
\pgfpathlineto{\pgfqpoint{4.389519in}{1.543384in}}%
\pgfpathlineto{\pgfqpoint{4.442563in}{1.580227in}}%
\pgfpathlineto{\pgfqpoint{4.495607in}{1.891220in}}%
\pgfpathlineto{\pgfqpoint{4.548650in}{1.543384in}}%
\pgfpathlineto{\pgfqpoint{4.601694in}{1.891220in}}%
\pgfpathlineto{\pgfqpoint{4.654738in}{1.543384in}}%
\pgfpathlineto{\pgfqpoint{4.707782in}{1.891220in}}%
\pgfpathlineto{\pgfqpoint{4.760826in}{1.543384in}}%
\pgfpathlineto{\pgfqpoint{4.813870in}{1.543384in}}%
\pgfpathlineto{\pgfqpoint{4.866914in}{1.891220in}}%
\pgfpathlineto{\pgfqpoint{4.919958in}{1.891220in}}%
\pgfpathlineto{\pgfqpoint{4.973002in}{1.543384in}}%
\pgfpathlineto{\pgfqpoint{5.026046in}{1.891220in}}%
\pgfpathlineto{\pgfqpoint{5.079090in}{1.543384in}}%
\pgfpathlineto{\pgfqpoint{5.132134in}{1.891220in}}%
\pgfpathlineto{\pgfqpoint{5.185178in}{1.891220in}}%
\pgfpathlineto{\pgfqpoint{5.238222in}{1.891220in}}%
\pgfpathlineto{\pgfqpoint{5.291266in}{1.543384in}}%
\pgfpathlineto{\pgfqpoint{5.344309in}{1.891220in}}%
\pgfpathlineto{\pgfqpoint{5.397353in}{1.891220in}}%
\pgfpathlineto{\pgfqpoint{5.450397in}{1.891220in}}%
\pgfpathlineto{\pgfqpoint{5.503441in}{1.543384in}}%
\pgfpathlineto{\pgfqpoint{5.556485in}{1.549519in}}%
\pgfpathlineto{\pgfqpoint{5.609529in}{1.891220in}}%
\pgfpathlineto{\pgfqpoint{5.662573in}{1.891220in}}%
\pgfpathlineto{\pgfqpoint{5.715617in}{1.543384in}}%
\pgfpathlineto{\pgfqpoint{5.768661in}{1.891220in}}%
\pgfpathlineto{\pgfqpoint{5.821705in}{1.891220in}}%
\pgfpathlineto{\pgfqpoint{5.874749in}{1.891220in}}%
\pgfpathlineto{\pgfqpoint{5.927793in}{1.891220in}}%
\pgfpathlineto{\pgfqpoint{5.980837in}{1.891220in}}%
\pgfpathlineto{\pgfqpoint{6.033881in}{1.678405in}}%
\pgfpathlineto{\pgfqpoint{6.086925in}{1.543384in}}%
\pgfpathlineto{\pgfqpoint{6.139969in}{1.731521in}}%
\pgfpathlineto{\pgfqpoint{6.193012in}{1.891220in}}%
\pgfpathlineto{\pgfqpoint{6.246056in}{1.891220in}}%
\pgfpathlineto{\pgfqpoint{6.299100in}{1.798179in}}%
\pgfpathlineto{\pgfqpoint{6.352144in}{1.891220in}}%
\pgfpathlineto{\pgfqpoint{6.405188in}{1.891220in}}%
\pgfpathlineto{\pgfqpoint{6.458232in}{1.891220in}}%
\pgfpathlineto{\pgfqpoint{6.511276in}{1.891220in}}%
\pgfpathlineto{\pgfqpoint{6.564320in}{1.891220in}}%
\pgfpathlineto{\pgfqpoint{6.617364in}{1.891220in}}%
\pgfpathlineto{\pgfqpoint{6.670408in}{1.891220in}}%
\pgfpathlineto{\pgfqpoint{6.723452in}{1.871334in}}%
\pgfpathlineto{\pgfqpoint{6.776496in}{1.543384in}}%
\pgfpathlineto{\pgfqpoint{6.829540in}{1.891220in}}%
\pgfpathlineto{\pgfqpoint{6.882584in}{1.891220in}}%
\pgfpathlineto{\pgfqpoint{6.935628in}{1.891220in}}%
\pgfpathlineto{\pgfqpoint{6.988671in}{1.891220in}}%
\pgfpathlineto{\pgfqpoint{7.041715in}{1.891220in}}%
\pgfpathlineto{\pgfqpoint{7.094759in}{1.543384in}}%
\pgfpathlineto{\pgfqpoint{7.147803in}{1.891220in}}%
\pgfpathlineto{\pgfqpoint{7.200847in}{1.891220in}}%
\pgfpathlineto{\pgfqpoint{7.253891in}{1.543384in}}%
\pgfpathlineto{\pgfqpoint{7.306935in}{1.746843in}}%
\pgfpathlineto{\pgfqpoint{7.359979in}{1.891220in}}%
\pgfpathlineto{\pgfqpoint{7.413023in}{1.891220in}}%
\pgfpathlineto{\pgfqpoint{7.466067in}{1.648522in}}%
\pgfpathlineto{\pgfqpoint{7.519111in}{1.543384in}}%
\pgfpathlineto{\pgfqpoint{7.572155in}{1.891220in}}%
\pgfpathlineto{\pgfqpoint{7.625199in}{1.891220in}}%
\pgfpathlineto{\pgfqpoint{7.678243in}{1.849746in}}%
\pgfpathlineto{\pgfqpoint{7.731287in}{1.891220in}}%
\pgfpathlineto{\pgfqpoint{7.784330in}{1.567858in}}%
\pgfpathlineto{\pgfqpoint{7.837374in}{1.543384in}}%
\pgfpathlineto{\pgfqpoint{7.890418in}{1.891220in}}%
\pgfpathlineto{\pgfqpoint{7.943462in}{1.543384in}}%
\pgfpathlineto{\pgfqpoint{7.996506in}{1.543384in}}%
\pgfpathlineto{\pgfqpoint{8.049550in}{1.891220in}}%
\pgfpathlineto{\pgfqpoint{8.102594in}{1.891220in}}%
\pgfpathlineto{\pgfqpoint{8.155638in}{1.543384in}}%
\pgfpathlineto{\pgfqpoint{8.208682in}{1.543384in}}%
\pgfpathlineto{\pgfqpoint{8.261726in}{1.891220in}}%
\pgfpathlineto{\pgfqpoint{8.314770in}{1.891220in}}%
\pgfpathlineto{\pgfqpoint{8.367814in}{1.543384in}}%
\pgfpathlineto{\pgfqpoint{8.420858in}{1.891220in}}%
\pgfpathlineto{\pgfqpoint{8.473902in}{1.543384in}}%
\pgfpathlineto{\pgfqpoint{8.526946in}{1.543384in}}%
\pgfpathlineto{\pgfqpoint{8.579990in}{1.891220in}}%
\pgfpathlineto{\pgfqpoint{8.633033in}{1.891220in}}%
\pgfpathlineto{\pgfqpoint{8.686077in}{1.891220in}}%
\pgfpathlineto{\pgfqpoint{8.739121in}{1.891220in}}%
\pgfpathlineto{\pgfqpoint{8.792165in}{1.755774in}}%
\pgfpathlineto{\pgfqpoint{8.845209in}{1.891220in}}%
\pgfpathlineto{\pgfqpoint{8.898253in}{1.891220in}}%
\pgfpathlineto{\pgfqpoint{8.951297in}{1.543384in}}%
\pgfpathlineto{\pgfqpoint{9.004341in}{1.644141in}}%
\pgfpathlineto{\pgfqpoint{9.057385in}{1.543384in}}%
\pgfpathlineto{\pgfqpoint{9.110429in}{1.543384in}}%
\pgfpathlineto{\pgfqpoint{9.163473in}{1.891220in}}%
\pgfpathlineto{\pgfqpoint{9.216517in}{1.891220in}}%
\pgfpathlineto{\pgfqpoint{9.269561in}{1.891220in}}%
\pgfpathlineto{\pgfqpoint{9.322605in}{1.891220in}}%
\pgfpathlineto{\pgfqpoint{9.375649in}{1.891220in}}%
\pgfpathlineto{\pgfqpoint{9.428692in}{1.543384in}}%
\pgfpathlineto{\pgfqpoint{9.481736in}{1.543384in}}%
\pgfpathlineto{\pgfqpoint{9.534780in}{1.891220in}}%
\pgfpathlineto{\pgfqpoint{9.587824in}{1.891220in}}%
\pgfpathlineto{\pgfqpoint{9.640868in}{1.617282in}}%
\pgfpathlineto{\pgfqpoint{9.693912in}{1.891220in}}%
\pgfpathlineto{\pgfqpoint{9.746956in}{1.891220in}}%
\pgfpathlineto{\pgfqpoint{9.800000in}{1.891220in}}%
\pgfpathlineto{\pgfqpoint{9.800000in}{1.891220in}}%
\pgfpathlineto{\pgfqpoint{9.800000in}{1.891220in}}%
\pgfpathlineto{\pgfqpoint{9.746956in}{1.891220in}}%
\pgfpathlineto{\pgfqpoint{9.693912in}{1.891220in}}%
\pgfpathlineto{\pgfqpoint{9.640868in}{1.617282in}}%
\pgfpathlineto{\pgfqpoint{9.587824in}{1.891220in}}%
\pgfpathlineto{\pgfqpoint{9.534780in}{1.891220in}}%
\pgfpathlineto{\pgfqpoint{9.481736in}{1.200625in}}%
\pgfpathlineto{\pgfqpoint{9.428692in}{1.153825in}}%
\pgfpathlineto{\pgfqpoint{9.375649in}{1.891220in}}%
\pgfpathlineto{\pgfqpoint{9.322605in}{1.891220in}}%
\pgfpathlineto{\pgfqpoint{9.269561in}{1.891220in}}%
\pgfpathlineto{\pgfqpoint{9.216517in}{1.891220in}}%
\pgfpathlineto{\pgfqpoint{9.163473in}{1.891220in}}%
\pgfpathlineto{\pgfqpoint{9.110429in}{0.997462in}}%
\pgfpathlineto{\pgfqpoint{9.057385in}{0.962369in}}%
\pgfpathlineto{\pgfqpoint{9.004341in}{1.644141in}}%
\pgfpathlineto{\pgfqpoint{8.951297in}{0.878008in}}%
\pgfpathlineto{\pgfqpoint{8.898253in}{1.891220in}}%
\pgfpathlineto{\pgfqpoint{8.845209in}{1.891220in}}%
\pgfpathlineto{\pgfqpoint{8.792165in}{1.755774in}}%
\pgfpathlineto{\pgfqpoint{8.739121in}{1.891220in}}%
\pgfpathlineto{\pgfqpoint{8.686077in}{1.891220in}}%
\pgfpathlineto{\pgfqpoint{8.633033in}{1.891220in}}%
\pgfpathlineto{\pgfqpoint{8.579990in}{1.891220in}}%
\pgfpathlineto{\pgfqpoint{8.526946in}{1.026802in}}%
\pgfpathlineto{\pgfqpoint{8.473902in}{1.054943in}}%
\pgfpathlineto{\pgfqpoint{8.420858in}{1.891220in}}%
\pgfpathlineto{\pgfqpoint{8.367814in}{1.114165in}}%
\pgfpathlineto{\pgfqpoint{8.314770in}{1.891220in}}%
\pgfpathlineto{\pgfqpoint{8.261726in}{1.891220in}}%
\pgfpathlineto{\pgfqpoint{8.208682in}{1.131748in}}%
\pgfpathlineto{\pgfqpoint{8.155638in}{1.137608in}}%
\pgfpathlineto{\pgfqpoint{8.102594in}{1.891220in}}%
\pgfpathlineto{\pgfqpoint{8.049550in}{1.891220in}}%
\pgfpathlineto{\pgfqpoint{7.996506in}{1.070971in}}%
\pgfpathlineto{\pgfqpoint{7.943462in}{0.992268in}}%
\pgfpathlineto{\pgfqpoint{7.890418in}{0.994522in}}%
\pgfpathlineto{\pgfqpoint{7.837374in}{0.953219in}}%
\pgfpathlineto{\pgfqpoint{7.784330in}{0.944968in}}%
\pgfpathlineto{\pgfqpoint{7.731287in}{0.908937in}}%
\pgfpathlineto{\pgfqpoint{7.678243in}{1.849746in}}%
\pgfpathlineto{\pgfqpoint{7.625199in}{1.891220in}}%
\pgfpathlineto{\pgfqpoint{7.572155in}{1.891220in}}%
\pgfpathlineto{\pgfqpoint{7.519111in}{1.410524in}}%
\pgfpathlineto{\pgfqpoint{7.466067in}{1.483365in}}%
\pgfpathlineto{\pgfqpoint{7.413023in}{1.891220in}}%
\pgfpathlineto{\pgfqpoint{7.359979in}{1.891220in}}%
\pgfpathlineto{\pgfqpoint{7.306935in}{1.095703in}}%
\pgfpathlineto{\pgfqpoint{7.253891in}{1.460365in}}%
\pgfpathlineto{\pgfqpoint{7.200847in}{1.891220in}}%
\pgfpathlineto{\pgfqpoint{7.147803in}{1.115813in}}%
\pgfpathlineto{\pgfqpoint{7.094759in}{1.134019in}}%
\pgfpathlineto{\pgfqpoint{7.041715in}{1.110968in}}%
\pgfpathlineto{\pgfqpoint{6.988671in}{1.684508in}}%
\pgfpathlineto{\pgfqpoint{6.935628in}{1.091705in}}%
\pgfpathlineto{\pgfqpoint{6.882584in}{1.891220in}}%
\pgfpathlineto{\pgfqpoint{6.829540in}{1.098052in}}%
\pgfpathlineto{\pgfqpoint{6.776496in}{1.330310in}}%
\pgfpathlineto{\pgfqpoint{6.723452in}{1.086085in}}%
\pgfpathlineto{\pgfqpoint{6.670408in}{1.040003in}}%
\pgfpathlineto{\pgfqpoint{6.617364in}{1.010061in}}%
\pgfpathlineto{\pgfqpoint{6.564320in}{0.943104in}}%
\pgfpathlineto{\pgfqpoint{6.511276in}{0.946754in}}%
\pgfpathlineto{\pgfqpoint{6.458232in}{1.413279in}}%
\pgfpathlineto{\pgfqpoint{6.405188in}{1.891220in}}%
\pgfpathlineto{\pgfqpoint{6.352144in}{1.833058in}}%
\pgfpathlineto{\pgfqpoint{6.299100in}{0.877160in}}%
\pgfpathlineto{\pgfqpoint{6.246056in}{1.891220in}}%
\pgfpathlineto{\pgfqpoint{6.193012in}{1.891220in}}%
\pgfpathlineto{\pgfqpoint{6.139969in}{0.956658in}}%
\pgfpathlineto{\pgfqpoint{6.086925in}{0.975132in}}%
\pgfpathlineto{\pgfqpoint{6.033881in}{1.678405in}}%
\pgfpathlineto{\pgfqpoint{5.980837in}{1.046269in}}%
\pgfpathlineto{\pgfqpoint{5.927793in}{1.891220in}}%
\pgfpathlineto{\pgfqpoint{5.874749in}{1.891220in}}%
\pgfpathlineto{\pgfqpoint{5.821705in}{1.891220in}}%
\pgfpathlineto{\pgfqpoint{5.768661in}{1.891220in}}%
\pgfpathlineto{\pgfqpoint{5.715617in}{1.122626in}}%
\pgfpathlineto{\pgfqpoint{5.662573in}{1.891220in}}%
\pgfpathlineto{\pgfqpoint{5.609529in}{1.891220in}}%
\pgfpathlineto{\pgfqpoint{5.556485in}{1.549519in}}%
\pgfpathlineto{\pgfqpoint{5.503441in}{1.116833in}}%
\pgfpathlineto{\pgfqpoint{5.450397in}{1.891220in}}%
\pgfpathlineto{\pgfqpoint{5.397353in}{1.891220in}}%
\pgfpathlineto{\pgfqpoint{5.344309in}{1.891220in}}%
\pgfpathlineto{\pgfqpoint{5.291266in}{0.975653in}}%
\pgfpathlineto{\pgfqpoint{5.238222in}{1.891220in}}%
\pgfpathlineto{\pgfqpoint{5.185178in}{1.891220in}}%
\pgfpathlineto{\pgfqpoint{5.132134in}{1.891220in}}%
\pgfpathlineto{\pgfqpoint{5.079090in}{0.827666in}}%
\pgfpathlineto{\pgfqpoint{5.026046in}{1.891220in}}%
\pgfpathlineto{\pgfqpoint{4.973002in}{0.856126in}}%
\pgfpathlineto{\pgfqpoint{4.919958in}{1.891220in}}%
\pgfpathlineto{\pgfqpoint{4.866914in}{1.891220in}}%
\pgfpathlineto{\pgfqpoint{4.813870in}{0.975129in}}%
\pgfpathlineto{\pgfqpoint{4.760826in}{1.287681in}}%
\pgfpathlineto{\pgfqpoint{4.707782in}{1.891220in}}%
\pgfpathlineto{\pgfqpoint{4.654738in}{1.094865in}}%
\pgfpathlineto{\pgfqpoint{4.601694in}{1.891220in}}%
\pgfpathlineto{\pgfqpoint{4.548650in}{1.437632in}}%
\pgfpathlineto{\pgfqpoint{4.495607in}{1.891220in}}%
\pgfpathlineto{\pgfqpoint{4.442563in}{1.580227in}}%
\pgfpathlineto{\pgfqpoint{4.389519in}{1.243678in}}%
\pgfpathlineto{\pgfqpoint{4.336475in}{1.891220in}}%
\pgfpathlineto{\pgfqpoint{4.283431in}{1.093254in}}%
\pgfpathlineto{\pgfqpoint{4.230387in}{1.077535in}}%
\pgfpathlineto{\pgfqpoint{4.177343in}{1.039428in}}%
\pgfpathlineto{\pgfqpoint{4.124299in}{1.123707in}}%
\pgfpathlineto{\pgfqpoint{4.071255in}{0.995539in}}%
\pgfpathlineto{\pgfqpoint{4.018211in}{0.913448in}}%
\pgfpathlineto{\pgfqpoint{3.965167in}{1.724076in}}%
\pgfpathlineto{\pgfqpoint{3.912123in}{1.426054in}}%
\pgfpathlineto{\pgfqpoint{3.859079in}{1.517418in}}%
\pgfpathlineto{\pgfqpoint{3.806035in}{0.907473in}}%
\pgfpathlineto{\pgfqpoint{3.752991in}{1.454919in}}%
\pgfpathlineto{\pgfqpoint{3.699948in}{0.867222in}}%
\pgfpathlineto{\pgfqpoint{3.646904in}{0.911072in}}%
\pgfpathlineto{\pgfqpoint{3.593860in}{1.388474in}}%
\pgfpathlineto{\pgfqpoint{3.540816in}{1.891220in}}%
\pgfpathlineto{\pgfqpoint{3.487772in}{1.891220in}}%
\pgfpathlineto{\pgfqpoint{3.434728in}{0.972772in}}%
\pgfpathlineto{\pgfqpoint{3.381684in}{1.666472in}}%
\pgfpathlineto{\pgfqpoint{3.328640in}{1.120386in}}%
\pgfpathlineto{\pgfqpoint{3.275596in}{1.061811in}}%
\pgfpathlineto{\pgfqpoint{3.222552in}{1.145671in}}%
\pgfpathlineto{\pgfqpoint{3.169508in}{1.115148in}}%
\pgfpathlineto{\pgfqpoint{3.116464in}{1.557584in}}%
\pgfpathlineto{\pgfqpoint{3.063420in}{1.891220in}}%
\pgfpathlineto{\pgfqpoint{3.010376in}{1.083328in}}%
\pgfpathlineto{\pgfqpoint{2.957332in}{1.891220in}}%
\pgfpathlineto{\pgfqpoint{2.904288in}{0.997564in}}%
\pgfpathlineto{\pgfqpoint{2.851245in}{1.781408in}}%
\pgfpathlineto{\pgfqpoint{2.798201in}{0.950657in}}%
\pgfpathlineto{\pgfqpoint{2.745157in}{0.908262in}}%
\pgfpathlineto{\pgfqpoint{2.692113in}{0.837338in}}%
\pgfpathlineto{\pgfqpoint{2.639069in}{0.867540in}}%
\pgfpathlineto{\pgfqpoint{2.586025in}{0.872034in}}%
\pgfpathlineto{\pgfqpoint{2.532981in}{0.893845in}}%
\pgfpathlineto{\pgfqpoint{2.479937in}{1.038089in}}%
\pgfpathlineto{\pgfqpoint{2.426893in}{1.891220in}}%
\pgfpathlineto{\pgfqpoint{2.373849in}{1.891220in}}%
\pgfpathlineto{\pgfqpoint{2.320805in}{0.918169in}}%
\pgfpathlineto{\pgfqpoint{2.267761in}{1.182556in}}%
\pgfpathlineto{\pgfqpoint{2.214717in}{1.891220in}}%
\pgfpathlineto{\pgfqpoint{2.161673in}{0.987578in}}%
\pgfpathlineto{\pgfqpoint{2.108629in}{1.481664in}}%
\pgfpathlineto{\pgfqpoint{2.055586in}{1.097929in}}%
\pgfpathlineto{\pgfqpoint{2.002542in}{1.891220in}}%
\pgfpathlineto{\pgfqpoint{1.949498in}{1.087124in}}%
\pgfpathlineto{\pgfqpoint{1.896454in}{1.891220in}}%
\pgfpathlineto{\pgfqpoint{1.843410in}{1.891220in}}%
\pgfpathlineto{\pgfqpoint{1.790366in}{1.113813in}}%
\pgfpathlineto{\pgfqpoint{1.737322in}{1.131392in}}%
\pgfpathlineto{\pgfqpoint{1.684278in}{1.559075in}}%
\pgfpathlineto{\pgfqpoint{1.631234in}{1.891220in}}%
\pgfpathlineto{\pgfqpoint{1.578190in}{0.966407in}}%
\pgfpathlineto{\pgfqpoint{1.525146in}{0.989277in}}%
\pgfpathlineto{\pgfqpoint{1.472102in}{0.908915in}}%
\pgfpathlineto{\pgfqpoint{1.419058in}{0.903552in}}%
\pgfpathlineto{\pgfqpoint{1.366014in}{0.829335in}}%
\pgfpathlineto{\pgfqpoint{1.312970in}{1.856216in}}%
\pgfpathlineto{\pgfqpoint{1.259927in}{1.891220in}}%
\pgfpathlineto{\pgfqpoint{1.206883in}{1.810992in}}%
\pgfpathlineto{\pgfqpoint{1.153839in}{0.893679in}}%
\pgfpathlineto{\pgfqpoint{1.100795in}{0.907342in}}%
\pgfpathlineto{\pgfqpoint{1.047751in}{0.938123in}}%
\pgfpathlineto{\pgfqpoint{0.994707in}{1.206100in}}%
\pgfpathlineto{\pgfqpoint{0.941663in}{0.942172in}}%
\pgfpathlineto{\pgfqpoint{0.941663in}{0.942172in}}%
\pgfpathclose%
\pgfusepath{stroke,fill}%
}%
\begin{pgfscope}%
\pgfsys@transformshift{0.000000in}{0.000000in}%
\pgfsys@useobject{currentmarker}{}%
\end{pgfscope}%
\end{pgfscope}%
\begin{pgfscope}%
\pgfpathrectangle{\pgfqpoint{0.941663in}{0.670138in}}{\pgfqpoint{8.858337in}{3.465625in}}%
\pgfusepath{clip}%
\pgfsetrectcap%
\pgfsetroundjoin%
\pgfsetlinewidth{1.505625pt}%
\definecolor{currentstroke}{rgb}{0.090196,0.745098,0.811765}%
\pgfsetstrokecolor{currentstroke}%
\pgfsetdash{}{0pt}%
\pgfpathmoveto{\pgfqpoint{0.941663in}{3.978235in}}%
\pgfpathlineto{\pgfqpoint{0.994707in}{3.724101in}}%
\pgfpathlineto{\pgfqpoint{1.047751in}{3.978235in}}%
\pgfpathlineto{\pgfqpoint{1.100795in}{3.966545in}}%
\pgfpathlineto{\pgfqpoint{1.153839in}{3.978235in}}%
\pgfpathlineto{\pgfqpoint{1.206883in}{3.101900in}}%
\pgfpathlineto{\pgfqpoint{1.259927in}{2.902493in}}%
\pgfpathlineto{\pgfqpoint{1.312970in}{3.006099in}}%
\pgfpathlineto{\pgfqpoint{1.366014in}{3.978235in}}%
\pgfpathlineto{\pgfqpoint{1.578190in}{3.978235in}}%
\pgfpathlineto{\pgfqpoint{1.631234in}{3.123835in}}%
\pgfpathlineto{\pgfqpoint{1.684278in}{3.440895in}}%
\pgfpathlineto{\pgfqpoint{1.737322in}{3.872158in}}%
\pgfpathlineto{\pgfqpoint{1.790366in}{3.978235in}}%
\pgfpathlineto{\pgfqpoint{1.843410in}{3.247418in}}%
\pgfpathlineto{\pgfqpoint{1.896454in}{3.219351in}}%
\pgfpathlineto{\pgfqpoint{1.949498in}{3.978235in}}%
\pgfpathlineto{\pgfqpoint{2.002542in}{3.193502in}}%
\pgfpathlineto{\pgfqpoint{2.055586in}{3.978235in}}%
\pgfpathlineto{\pgfqpoint{2.108629in}{3.577804in}}%
\pgfpathlineto{\pgfqpoint{2.161673in}{3.978235in}}%
\pgfpathlineto{\pgfqpoint{2.214717in}{3.089112in}}%
\pgfpathlineto{\pgfqpoint{2.267761in}{3.754982in}}%
\pgfpathlineto{\pgfqpoint{2.320805in}{3.978235in}}%
\pgfpathlineto{\pgfqpoint{2.373849in}{2.980147in}}%
\pgfpathlineto{\pgfqpoint{2.426893in}{2.942415in}}%
\pgfpathlineto{\pgfqpoint{2.479937in}{3.785532in}}%
\pgfpathlineto{\pgfqpoint{2.532981in}{3.978235in}}%
\pgfpathlineto{\pgfqpoint{2.798201in}{3.978235in}}%
\pgfpathlineto{\pgfqpoint{2.851245in}{3.161616in}}%
\pgfpathlineto{\pgfqpoint{2.904288in}{3.978235in}}%
\pgfpathlineto{\pgfqpoint{2.957332in}{3.172295in}}%
\pgfpathlineto{\pgfqpoint{3.010376in}{3.978235in}}%
\pgfpathlineto{\pgfqpoint{3.063420in}{3.151674in}}%
\pgfpathlineto{\pgfqpoint{3.116464in}{3.506190in}}%
\pgfpathlineto{\pgfqpoint{3.169508in}{3.978235in}}%
\pgfpathlineto{\pgfqpoint{3.328640in}{3.978235in}}%
\pgfpathlineto{\pgfqpoint{3.381684in}{3.349122in}}%
\pgfpathlineto{\pgfqpoint{3.434728in}{3.978235in}}%
\pgfpathlineto{\pgfqpoint{3.487772in}{3.083924in}}%
\pgfpathlineto{\pgfqpoint{3.540816in}{3.074646in}}%
\pgfpathlineto{\pgfqpoint{3.593860in}{3.503387in}}%
\pgfpathlineto{\pgfqpoint{3.646904in}{3.978235in}}%
\pgfpathlineto{\pgfqpoint{3.699948in}{3.978235in}}%
\pgfpathlineto{\pgfqpoint{3.752991in}{3.388260in}}%
\pgfpathlineto{\pgfqpoint{3.806035in}{3.978235in}}%
\pgfpathlineto{\pgfqpoint{3.859079in}{3.355689in}}%
\pgfpathlineto{\pgfqpoint{3.912123in}{3.408107in}}%
\pgfpathlineto{\pgfqpoint{3.965167in}{3.238518in}}%
\pgfpathlineto{\pgfqpoint{4.018211in}{3.978235in}}%
\pgfpathlineto{\pgfqpoint{4.071255in}{3.978235in}}%
\pgfpathlineto{\pgfqpoint{4.124299in}{3.875484in}}%
\pgfpathlineto{\pgfqpoint{4.177343in}{3.978235in}}%
\pgfpathlineto{\pgfqpoint{4.283431in}{3.978235in}}%
\pgfpathlineto{\pgfqpoint{4.336475in}{3.224846in}}%
\pgfpathlineto{\pgfqpoint{4.389519in}{3.841843in}}%
\pgfpathlineto{\pgfqpoint{4.442563in}{3.494347in}}%
\pgfpathlineto{\pgfqpoint{4.495607in}{3.218773in}}%
\pgfpathlineto{\pgfqpoint{4.548650in}{3.689185in}}%
\pgfpathlineto{\pgfqpoint{4.601694in}{3.162876in}}%
\pgfpathlineto{\pgfqpoint{4.654738in}{3.978235in}}%
\pgfpathlineto{\pgfqpoint{4.707782in}{3.102021in}}%
\pgfpathlineto{\pgfqpoint{4.760826in}{3.714472in}}%
\pgfpathlineto{\pgfqpoint{4.813870in}{3.978235in}}%
\pgfpathlineto{\pgfqpoint{4.866914in}{3.020692in}}%
\pgfpathlineto{\pgfqpoint{4.919958in}{3.007061in}}%
\pgfpathlineto{\pgfqpoint{4.973002in}{3.978235in}}%
\pgfpathlineto{\pgfqpoint{5.026046in}{2.958572in}}%
\pgfpathlineto{\pgfqpoint{5.079090in}{3.978235in}}%
\pgfpathlineto{\pgfqpoint{5.132134in}{2.918032in}}%
\pgfpathlineto{\pgfqpoint{5.185178in}{3.006017in}}%
\pgfpathlineto{\pgfqpoint{5.238222in}{3.002435in}}%
\pgfpathlineto{\pgfqpoint{5.291266in}{3.978235in}}%
\pgfpathlineto{\pgfqpoint{5.344309in}{3.073166in}}%
\pgfpathlineto{\pgfqpoint{5.397353in}{3.084080in}}%
\pgfpathlineto{\pgfqpoint{5.450397in}{3.213681in}}%
\pgfpathlineto{\pgfqpoint{5.503441in}{3.978235in}}%
\pgfpathlineto{\pgfqpoint{5.556485in}{3.538355in}}%
\pgfpathlineto{\pgfqpoint{5.609529in}{3.209959in}}%
\pgfpathlineto{\pgfqpoint{5.662573in}{3.261554in}}%
\pgfpathlineto{\pgfqpoint{5.715617in}{3.978235in}}%
\pgfpathlineto{\pgfqpoint{5.768661in}{3.248973in}}%
\pgfpathlineto{\pgfqpoint{5.821705in}{3.208761in}}%
\pgfpathlineto{\pgfqpoint{5.874749in}{3.132435in}}%
\pgfpathlineto{\pgfqpoint{5.927793in}{3.156691in}}%
\pgfpathlineto{\pgfqpoint{5.980837in}{3.978235in}}%
\pgfpathlineto{\pgfqpoint{6.033881in}{3.296892in}}%
\pgfpathlineto{\pgfqpoint{6.086925in}{3.978235in}}%
\pgfpathlineto{\pgfqpoint{6.139969in}{3.978235in}}%
\pgfpathlineto{\pgfqpoint{6.193012in}{2.971161in}}%
\pgfpathlineto{\pgfqpoint{6.246056in}{3.026875in}}%
\pgfpathlineto{\pgfqpoint{6.299100in}{3.978235in}}%
\pgfpathlineto{\pgfqpoint{6.352144in}{3.041125in}}%
\pgfpathlineto{\pgfqpoint{6.405188in}{2.929750in}}%
\pgfpathlineto{\pgfqpoint{6.458232in}{3.493957in}}%
\pgfpathlineto{\pgfqpoint{6.511276in}{3.978235in}}%
\pgfpathlineto{\pgfqpoint{6.723452in}{3.978235in}}%
\pgfpathlineto{\pgfqpoint{6.776496in}{3.755800in}}%
\pgfpathlineto{\pgfqpoint{6.829540in}{3.978235in}}%
\pgfpathlineto{\pgfqpoint{6.882584in}{3.217123in}}%
\pgfpathlineto{\pgfqpoint{6.935628in}{3.978235in}}%
\pgfpathlineto{\pgfqpoint{6.988671in}{3.429108in}}%
\pgfpathlineto{\pgfqpoint{7.041715in}{3.978235in}}%
\pgfpathlineto{\pgfqpoint{7.147803in}{3.978235in}}%
\pgfpathlineto{\pgfqpoint{7.200847in}{3.153935in}}%
\pgfpathlineto{\pgfqpoint{7.253891in}{3.485903in}}%
\pgfpathlineto{\pgfqpoint{7.306935in}{3.878601in}}%
\pgfpathlineto{\pgfqpoint{7.359979in}{3.007952in}}%
\pgfpathlineto{\pgfqpoint{7.413023in}{3.019856in}}%
\pgfpathlineto{\pgfqpoint{7.466067in}{3.397755in}}%
\pgfpathlineto{\pgfqpoint{7.519111in}{3.422724in}}%
\pgfpathlineto{\pgfqpoint{7.572155in}{2.929451in}}%
\pgfpathlineto{\pgfqpoint{7.625199in}{2.988504in}}%
\pgfpathlineto{\pgfqpoint{7.678243in}{3.074893in}}%
\pgfpathlineto{\pgfqpoint{7.731287in}{3.978235in}}%
\pgfpathlineto{\pgfqpoint{7.996506in}{3.978235in}}%
\pgfpathlineto{\pgfqpoint{8.049550in}{3.189369in}}%
\pgfpathlineto{\pgfqpoint{8.102594in}{3.198813in}}%
\pgfpathlineto{\pgfqpoint{8.155638in}{3.978235in}}%
\pgfpathlineto{\pgfqpoint{8.208682in}{3.978235in}}%
\pgfpathlineto{\pgfqpoint{8.261726in}{3.260476in}}%
\pgfpathlineto{\pgfqpoint{8.314770in}{3.211850in}}%
\pgfpathlineto{\pgfqpoint{8.367814in}{3.978235in}}%
\pgfpathlineto{\pgfqpoint{8.420858in}{3.193239in}}%
\pgfpathlineto{\pgfqpoint{8.473902in}{3.978235in}}%
\pgfpathlineto{\pgfqpoint{8.526946in}{3.978235in}}%
\pgfpathlineto{\pgfqpoint{8.579990in}{3.021265in}}%
\pgfpathlineto{\pgfqpoint{8.633033in}{3.030538in}}%
\pgfpathlineto{\pgfqpoint{8.686077in}{3.059450in}}%
\pgfpathlineto{\pgfqpoint{8.739121in}{2.997851in}}%
\pgfpathlineto{\pgfqpoint{8.792165in}{3.135951in}}%
\pgfpathlineto{\pgfqpoint{8.845209in}{2.978670in}}%
\pgfpathlineto{\pgfqpoint{8.898253in}{3.021456in}}%
\pgfpathlineto{\pgfqpoint{8.951297in}{3.978235in}}%
\pgfpathlineto{\pgfqpoint{9.004341in}{3.231041in}}%
\pgfpathlineto{\pgfqpoint{9.057385in}{3.978235in}}%
\pgfpathlineto{\pgfqpoint{9.110429in}{3.978235in}}%
\pgfpathlineto{\pgfqpoint{9.163473in}{3.106410in}}%
\pgfpathlineto{\pgfqpoint{9.216517in}{3.140256in}}%
\pgfpathlineto{\pgfqpoint{9.269561in}{3.147248in}}%
\pgfpathlineto{\pgfqpoint{9.322605in}{3.219668in}}%
\pgfpathlineto{\pgfqpoint{9.375649in}{3.214984in}}%
\pgfpathlineto{\pgfqpoint{9.428692in}{3.978235in}}%
\pgfpathlineto{\pgfqpoint{9.481736in}{3.910858in}}%
\pgfpathlineto{\pgfqpoint{9.534780in}{3.282563in}}%
\pgfpathlineto{\pgfqpoint{9.587824in}{3.204226in}}%
\pgfpathlineto{\pgfqpoint{9.640868in}{3.473714in}}%
\pgfpathlineto{\pgfqpoint{9.693912in}{3.139443in}}%
\pgfpathlineto{\pgfqpoint{9.746956in}{3.115475in}}%
\pgfpathlineto{\pgfqpoint{9.800000in}{3.101625in}}%
\pgfpathlineto{\pgfqpoint{9.800000in}{3.101625in}}%
\pgfusepath{stroke}%
\end{pgfscope}%
\begin{pgfscope}%
\pgfpathrectangle{\pgfqpoint{0.941663in}{0.670138in}}{\pgfqpoint{8.858337in}{3.465625in}}%
\pgfusepath{clip}%
\pgfsetbuttcap%
\pgfsetroundjoin%
\definecolor{currentfill}{rgb}{0.090196,0.745098,0.811765}%
\pgfsetfillcolor{currentfill}%
\pgfsetlinewidth{1.003750pt}%
\definecolor{currentstroke}{rgb}{0.090196,0.745098,0.811765}%
\pgfsetstrokecolor{currentstroke}%
\pgfsetdash{}{0pt}%
\pgfsys@defobject{currentmarker}{\pgfqpoint{0.941663in}{2.586892in}}{\pgfqpoint{9.800000in}{3.978235in}}{%
\pgfpathmoveto{\pgfqpoint{0.941663in}{3.978235in}}%
\pgfpathlineto{\pgfqpoint{0.941663in}{2.586892in}}%
\pgfpathlineto{\pgfqpoint{0.994707in}{2.586892in}}%
\pgfpathlineto{\pgfqpoint{1.047751in}{2.586892in}}%
\pgfpathlineto{\pgfqpoint{1.100795in}{2.586892in}}%
\pgfpathlineto{\pgfqpoint{1.153839in}{2.586892in}}%
\pgfpathlineto{\pgfqpoint{1.206883in}{2.586892in}}%
\pgfpathlineto{\pgfqpoint{1.259927in}{2.749185in}}%
\pgfpathlineto{\pgfqpoint{1.312970in}{2.586892in}}%
\pgfpathlineto{\pgfqpoint{1.366014in}{2.586892in}}%
\pgfpathlineto{\pgfqpoint{1.419058in}{2.586892in}}%
\pgfpathlineto{\pgfqpoint{1.472102in}{2.586892in}}%
\pgfpathlineto{\pgfqpoint{1.525146in}{2.586892in}}%
\pgfpathlineto{\pgfqpoint{1.578190in}{2.586892in}}%
\pgfpathlineto{\pgfqpoint{1.631234in}{2.802326in}}%
\pgfpathlineto{\pgfqpoint{1.684278in}{2.586892in}}%
\pgfpathlineto{\pgfqpoint{1.737322in}{2.586892in}}%
\pgfpathlineto{\pgfqpoint{1.790366in}{2.586892in}}%
\pgfpathlineto{\pgfqpoint{1.843410in}{3.070356in}}%
\pgfpathlineto{\pgfqpoint{1.896454in}{3.219351in}}%
\pgfpathlineto{\pgfqpoint{1.949498in}{2.586892in}}%
\pgfpathlineto{\pgfqpoint{2.002542in}{3.060121in}}%
\pgfpathlineto{\pgfqpoint{2.055586in}{2.586892in}}%
\pgfpathlineto{\pgfqpoint{2.108629in}{2.586892in}}%
\pgfpathlineto{\pgfqpoint{2.161673in}{2.586892in}}%
\pgfpathlineto{\pgfqpoint{2.214717in}{2.910735in}}%
\pgfpathlineto{\pgfqpoint{2.267761in}{2.586892in}}%
\pgfpathlineto{\pgfqpoint{2.320805in}{2.586892in}}%
\pgfpathlineto{\pgfqpoint{2.373849in}{2.844885in}}%
\pgfpathlineto{\pgfqpoint{2.426893in}{2.942415in}}%
\pgfpathlineto{\pgfqpoint{2.479937in}{2.586892in}}%
\pgfpathlineto{\pgfqpoint{2.532981in}{2.586892in}}%
\pgfpathlineto{\pgfqpoint{2.586025in}{2.586892in}}%
\pgfpathlineto{\pgfqpoint{2.639069in}{2.586892in}}%
\pgfpathlineto{\pgfqpoint{2.692113in}{2.586892in}}%
\pgfpathlineto{\pgfqpoint{2.745157in}{2.586892in}}%
\pgfpathlineto{\pgfqpoint{2.798201in}{2.586892in}}%
\pgfpathlineto{\pgfqpoint{2.851245in}{2.586892in}}%
\pgfpathlineto{\pgfqpoint{2.904288in}{2.586892in}}%
\pgfpathlineto{\pgfqpoint{2.957332in}{3.080511in}}%
\pgfpathlineto{\pgfqpoint{3.010376in}{2.586892in}}%
\pgfpathlineto{\pgfqpoint{3.063420in}{2.909016in}}%
\pgfpathlineto{\pgfqpoint{3.116464in}{2.586892in}}%
\pgfpathlineto{\pgfqpoint{3.169508in}{2.586892in}}%
\pgfpathlineto{\pgfqpoint{3.222552in}{2.586892in}}%
\pgfpathlineto{\pgfqpoint{3.275596in}{2.586892in}}%
\pgfpathlineto{\pgfqpoint{3.328640in}{2.586892in}}%
\pgfpathlineto{\pgfqpoint{3.381684in}{2.586892in}}%
\pgfpathlineto{\pgfqpoint{3.434728in}{2.586892in}}%
\pgfpathlineto{\pgfqpoint{3.487772in}{2.710315in}}%
\pgfpathlineto{\pgfqpoint{3.540816in}{2.908402in}}%
\pgfpathlineto{\pgfqpoint{3.593860in}{2.586892in}}%
\pgfpathlineto{\pgfqpoint{3.646904in}{2.586892in}}%
\pgfpathlineto{\pgfqpoint{3.699948in}{2.586892in}}%
\pgfpathlineto{\pgfqpoint{3.752991in}{2.586892in}}%
\pgfpathlineto{\pgfqpoint{3.806035in}{2.586892in}}%
\pgfpathlineto{\pgfqpoint{3.859079in}{2.586892in}}%
\pgfpathlineto{\pgfqpoint{3.912123in}{2.586892in}}%
\pgfpathlineto{\pgfqpoint{3.965167in}{2.586892in}}%
\pgfpathlineto{\pgfqpoint{4.018211in}{2.586892in}}%
\pgfpathlineto{\pgfqpoint{4.071255in}{2.586892in}}%
\pgfpathlineto{\pgfqpoint{4.124299in}{2.586892in}}%
\pgfpathlineto{\pgfqpoint{4.177343in}{2.586892in}}%
\pgfpathlineto{\pgfqpoint{4.230387in}{2.586892in}}%
\pgfpathlineto{\pgfqpoint{4.283431in}{2.586892in}}%
\pgfpathlineto{\pgfqpoint{4.336475in}{3.224846in}}%
\pgfpathlineto{\pgfqpoint{4.389519in}{2.586892in}}%
\pgfpathlineto{\pgfqpoint{4.442563in}{2.586892in}}%
\pgfpathlineto{\pgfqpoint{4.495607in}{2.968775in}}%
\pgfpathlineto{\pgfqpoint{4.548650in}{2.586892in}}%
\pgfpathlineto{\pgfqpoint{4.601694in}{2.927863in}}%
\pgfpathlineto{\pgfqpoint{4.654738in}{2.586892in}}%
\pgfpathlineto{\pgfqpoint{4.707782in}{3.102021in}}%
\pgfpathlineto{\pgfqpoint{4.760826in}{2.586892in}}%
\pgfpathlineto{\pgfqpoint{4.813870in}{2.586892in}}%
\pgfpathlineto{\pgfqpoint{4.866914in}{2.884645in}}%
\pgfpathlineto{\pgfqpoint{4.919958in}{2.863950in}}%
\pgfpathlineto{\pgfqpoint{4.973002in}{2.586892in}}%
\pgfpathlineto{\pgfqpoint{5.026046in}{2.758322in}}%
\pgfpathlineto{\pgfqpoint{5.079090in}{2.586892in}}%
\pgfpathlineto{\pgfqpoint{5.132134in}{2.824147in}}%
\pgfpathlineto{\pgfqpoint{5.185178in}{2.910114in}}%
\pgfpathlineto{\pgfqpoint{5.238222in}{2.837414in}}%
\pgfpathlineto{\pgfqpoint{5.291266in}{2.586892in}}%
\pgfpathlineto{\pgfqpoint{5.344309in}{2.705202in}}%
\pgfpathlineto{\pgfqpoint{5.397353in}{2.660769in}}%
\pgfpathlineto{\pgfqpoint{5.450397in}{3.105789in}}%
\pgfpathlineto{\pgfqpoint{5.503441in}{2.586892in}}%
\pgfpathlineto{\pgfqpoint{5.556485in}{2.586892in}}%
\pgfpathlineto{\pgfqpoint{5.609529in}{2.850209in}}%
\pgfpathlineto{\pgfqpoint{5.662573in}{2.711726in}}%
\pgfpathlineto{\pgfqpoint{5.715617in}{2.586892in}}%
\pgfpathlineto{\pgfqpoint{5.768661in}{3.248973in}}%
\pgfpathlineto{\pgfqpoint{5.821705in}{3.208761in}}%
\pgfpathlineto{\pgfqpoint{5.874749in}{3.037838in}}%
\pgfpathlineto{\pgfqpoint{5.927793in}{2.934114in}}%
\pgfpathlineto{\pgfqpoint{5.980837in}{2.586892in}}%
\pgfpathlineto{\pgfqpoint{6.033881in}{2.586892in}}%
\pgfpathlineto{\pgfqpoint{6.086925in}{2.586892in}}%
\pgfpathlineto{\pgfqpoint{6.139969in}{2.586892in}}%
\pgfpathlineto{\pgfqpoint{6.193012in}{2.971161in}}%
\pgfpathlineto{\pgfqpoint{6.246056in}{3.026875in}}%
\pgfpathlineto{\pgfqpoint{6.299100in}{2.586892in}}%
\pgfpathlineto{\pgfqpoint{6.352144in}{2.586892in}}%
\pgfpathlineto{\pgfqpoint{6.405188in}{2.699818in}}%
\pgfpathlineto{\pgfqpoint{6.458232in}{2.586892in}}%
\pgfpathlineto{\pgfqpoint{6.511276in}{2.586892in}}%
\pgfpathlineto{\pgfqpoint{6.564320in}{2.586892in}}%
\pgfpathlineto{\pgfqpoint{6.617364in}{2.586892in}}%
\pgfpathlineto{\pgfqpoint{6.670408in}{2.586892in}}%
\pgfpathlineto{\pgfqpoint{6.723452in}{2.586892in}}%
\pgfpathlineto{\pgfqpoint{6.776496in}{2.586892in}}%
\pgfpathlineto{\pgfqpoint{6.829540in}{2.586892in}}%
\pgfpathlineto{\pgfqpoint{6.882584in}{3.116001in}}%
\pgfpathlineto{\pgfqpoint{6.935628in}{2.586892in}}%
\pgfpathlineto{\pgfqpoint{6.988671in}{2.586892in}}%
\pgfpathlineto{\pgfqpoint{7.041715in}{2.586892in}}%
\pgfpathlineto{\pgfqpoint{7.094759in}{2.586892in}}%
\pgfpathlineto{\pgfqpoint{7.147803in}{2.586892in}}%
\pgfpathlineto{\pgfqpoint{7.200847in}{3.064989in}}%
\pgfpathlineto{\pgfqpoint{7.253891in}{2.586892in}}%
\pgfpathlineto{\pgfqpoint{7.306935in}{2.586892in}}%
\pgfpathlineto{\pgfqpoint{7.359979in}{2.806406in}}%
\pgfpathlineto{\pgfqpoint{7.413023in}{2.865454in}}%
\pgfpathlineto{\pgfqpoint{7.466067in}{2.586892in}}%
\pgfpathlineto{\pgfqpoint{7.519111in}{2.586892in}}%
\pgfpathlineto{\pgfqpoint{7.572155in}{2.823725in}}%
\pgfpathlineto{\pgfqpoint{7.625199in}{2.988504in}}%
\pgfpathlineto{\pgfqpoint{7.678243in}{2.586892in}}%
\pgfpathlineto{\pgfqpoint{7.731287in}{2.586892in}}%
\pgfpathlineto{\pgfqpoint{7.784330in}{2.586892in}}%
\pgfpathlineto{\pgfqpoint{7.837374in}{2.586892in}}%
\pgfpathlineto{\pgfqpoint{7.890418in}{2.586892in}}%
\pgfpathlineto{\pgfqpoint{7.943462in}{2.586892in}}%
\pgfpathlineto{\pgfqpoint{7.996506in}{2.586892in}}%
\pgfpathlineto{\pgfqpoint{8.049550in}{3.102247in}}%
\pgfpathlineto{\pgfqpoint{8.102594in}{3.118508in}}%
\pgfpathlineto{\pgfqpoint{8.155638in}{2.586892in}}%
\pgfpathlineto{\pgfqpoint{8.208682in}{2.586892in}}%
\pgfpathlineto{\pgfqpoint{8.261726in}{3.260476in}}%
\pgfpathlineto{\pgfqpoint{8.314770in}{3.211850in}}%
\pgfpathlineto{\pgfqpoint{8.367814in}{2.586892in}}%
\pgfpathlineto{\pgfqpoint{8.420858in}{2.989399in}}%
\pgfpathlineto{\pgfqpoint{8.473902in}{2.586892in}}%
\pgfpathlineto{\pgfqpoint{8.526946in}{2.586892in}}%
\pgfpathlineto{\pgfqpoint{8.579990in}{3.021265in}}%
\pgfpathlineto{\pgfqpoint{8.633033in}{2.856358in}}%
\pgfpathlineto{\pgfqpoint{8.686077in}{3.059450in}}%
\pgfpathlineto{\pgfqpoint{8.739121in}{2.825404in}}%
\pgfpathlineto{\pgfqpoint{8.792165in}{2.586892in}}%
\pgfpathlineto{\pgfqpoint{8.845209in}{2.634872in}}%
\pgfpathlineto{\pgfqpoint{8.898253in}{2.689958in}}%
\pgfpathlineto{\pgfqpoint{8.951297in}{2.586892in}}%
\pgfpathlineto{\pgfqpoint{9.004341in}{2.586892in}}%
\pgfpathlineto{\pgfqpoint{9.057385in}{2.586892in}}%
\pgfpathlineto{\pgfqpoint{9.110429in}{2.586892in}}%
\pgfpathlineto{\pgfqpoint{9.163473in}{3.004188in}}%
\pgfpathlineto{\pgfqpoint{9.216517in}{3.140256in}}%
\pgfpathlineto{\pgfqpoint{9.269561in}{2.964013in}}%
\pgfpathlineto{\pgfqpoint{9.322605in}{3.063213in}}%
\pgfpathlineto{\pgfqpoint{9.375649in}{3.090533in}}%
\pgfpathlineto{\pgfqpoint{9.428692in}{2.586892in}}%
\pgfpathlineto{\pgfqpoint{9.481736in}{2.586892in}}%
\pgfpathlineto{\pgfqpoint{9.534780in}{3.282563in}}%
\pgfpathlineto{\pgfqpoint{9.587824in}{2.956774in}}%
\pgfpathlineto{\pgfqpoint{9.640868in}{2.586892in}}%
\pgfpathlineto{\pgfqpoint{9.693912in}{2.968640in}}%
\pgfpathlineto{\pgfqpoint{9.746956in}{3.115475in}}%
\pgfpathlineto{\pgfqpoint{9.800000in}{2.909416in}}%
\pgfpathlineto{\pgfqpoint{9.800000in}{3.101625in}}%
\pgfpathlineto{\pgfqpoint{9.800000in}{3.101625in}}%
\pgfpathlineto{\pgfqpoint{9.746956in}{3.115475in}}%
\pgfpathlineto{\pgfqpoint{9.693912in}{3.139443in}}%
\pgfpathlineto{\pgfqpoint{9.640868in}{3.473714in}}%
\pgfpathlineto{\pgfqpoint{9.587824in}{3.204226in}}%
\pgfpathlineto{\pgfqpoint{9.534780in}{3.282563in}}%
\pgfpathlineto{\pgfqpoint{9.481736in}{3.910858in}}%
\pgfpathlineto{\pgfqpoint{9.428692in}{3.978235in}}%
\pgfpathlineto{\pgfqpoint{9.375649in}{3.214984in}}%
\pgfpathlineto{\pgfqpoint{9.322605in}{3.219668in}}%
\pgfpathlineto{\pgfqpoint{9.269561in}{3.147248in}}%
\pgfpathlineto{\pgfqpoint{9.216517in}{3.140256in}}%
\pgfpathlineto{\pgfqpoint{9.163473in}{3.106410in}}%
\pgfpathlineto{\pgfqpoint{9.110429in}{3.978235in}}%
\pgfpathlineto{\pgfqpoint{9.057385in}{3.978235in}}%
\pgfpathlineto{\pgfqpoint{9.004341in}{3.231041in}}%
\pgfpathlineto{\pgfqpoint{8.951297in}{3.978235in}}%
\pgfpathlineto{\pgfqpoint{8.898253in}{3.021456in}}%
\pgfpathlineto{\pgfqpoint{8.845209in}{2.978670in}}%
\pgfpathlineto{\pgfqpoint{8.792165in}{3.135951in}}%
\pgfpathlineto{\pgfqpoint{8.739121in}{2.997851in}}%
\pgfpathlineto{\pgfqpoint{8.686077in}{3.059450in}}%
\pgfpathlineto{\pgfqpoint{8.633033in}{3.030538in}}%
\pgfpathlineto{\pgfqpoint{8.579990in}{3.021265in}}%
\pgfpathlineto{\pgfqpoint{8.526946in}{3.978235in}}%
\pgfpathlineto{\pgfqpoint{8.473902in}{3.978235in}}%
\pgfpathlineto{\pgfqpoint{8.420858in}{3.193239in}}%
\pgfpathlineto{\pgfqpoint{8.367814in}{3.978235in}}%
\pgfpathlineto{\pgfqpoint{8.314770in}{3.211850in}}%
\pgfpathlineto{\pgfqpoint{8.261726in}{3.260476in}}%
\pgfpathlineto{\pgfqpoint{8.208682in}{3.978235in}}%
\pgfpathlineto{\pgfqpoint{8.155638in}{3.978235in}}%
\pgfpathlineto{\pgfqpoint{8.102594in}{3.198813in}}%
\pgfpathlineto{\pgfqpoint{8.049550in}{3.189369in}}%
\pgfpathlineto{\pgfqpoint{7.996506in}{3.978235in}}%
\pgfpathlineto{\pgfqpoint{7.943462in}{3.978235in}}%
\pgfpathlineto{\pgfqpoint{7.890418in}{3.978235in}}%
\pgfpathlineto{\pgfqpoint{7.837374in}{3.978235in}}%
\pgfpathlineto{\pgfqpoint{7.784330in}{3.978235in}}%
\pgfpathlineto{\pgfqpoint{7.731287in}{3.978235in}}%
\pgfpathlineto{\pgfqpoint{7.678243in}{3.074893in}}%
\pgfpathlineto{\pgfqpoint{7.625199in}{2.988504in}}%
\pgfpathlineto{\pgfqpoint{7.572155in}{2.929451in}}%
\pgfpathlineto{\pgfqpoint{7.519111in}{3.422724in}}%
\pgfpathlineto{\pgfqpoint{7.466067in}{3.397755in}}%
\pgfpathlineto{\pgfqpoint{7.413023in}{3.019856in}}%
\pgfpathlineto{\pgfqpoint{7.359979in}{3.007952in}}%
\pgfpathlineto{\pgfqpoint{7.306935in}{3.878601in}}%
\pgfpathlineto{\pgfqpoint{7.253891in}{3.485903in}}%
\pgfpathlineto{\pgfqpoint{7.200847in}{3.153935in}}%
\pgfpathlineto{\pgfqpoint{7.147803in}{3.978235in}}%
\pgfpathlineto{\pgfqpoint{7.094759in}{3.978235in}}%
\pgfpathlineto{\pgfqpoint{7.041715in}{3.978235in}}%
\pgfpathlineto{\pgfqpoint{6.988671in}{3.429108in}}%
\pgfpathlineto{\pgfqpoint{6.935628in}{3.978235in}}%
\pgfpathlineto{\pgfqpoint{6.882584in}{3.217123in}}%
\pgfpathlineto{\pgfqpoint{6.829540in}{3.978235in}}%
\pgfpathlineto{\pgfqpoint{6.776496in}{3.755800in}}%
\pgfpathlineto{\pgfqpoint{6.723452in}{3.978235in}}%
\pgfpathlineto{\pgfqpoint{6.670408in}{3.978235in}}%
\pgfpathlineto{\pgfqpoint{6.617364in}{3.978235in}}%
\pgfpathlineto{\pgfqpoint{6.564320in}{3.978235in}}%
\pgfpathlineto{\pgfqpoint{6.511276in}{3.978235in}}%
\pgfpathlineto{\pgfqpoint{6.458232in}{3.493957in}}%
\pgfpathlineto{\pgfqpoint{6.405188in}{2.929750in}}%
\pgfpathlineto{\pgfqpoint{6.352144in}{3.041125in}}%
\pgfpathlineto{\pgfqpoint{6.299100in}{3.978235in}}%
\pgfpathlineto{\pgfqpoint{6.246056in}{3.026875in}}%
\pgfpathlineto{\pgfqpoint{6.193012in}{2.971161in}}%
\pgfpathlineto{\pgfqpoint{6.139969in}{3.978235in}}%
\pgfpathlineto{\pgfqpoint{6.086925in}{3.978235in}}%
\pgfpathlineto{\pgfqpoint{6.033881in}{3.296892in}}%
\pgfpathlineto{\pgfqpoint{5.980837in}{3.978235in}}%
\pgfpathlineto{\pgfqpoint{5.927793in}{3.156691in}}%
\pgfpathlineto{\pgfqpoint{5.874749in}{3.132435in}}%
\pgfpathlineto{\pgfqpoint{5.821705in}{3.208761in}}%
\pgfpathlineto{\pgfqpoint{5.768661in}{3.248973in}}%
\pgfpathlineto{\pgfqpoint{5.715617in}{3.978235in}}%
\pgfpathlineto{\pgfqpoint{5.662573in}{3.261554in}}%
\pgfpathlineto{\pgfqpoint{5.609529in}{3.209959in}}%
\pgfpathlineto{\pgfqpoint{5.556485in}{3.538355in}}%
\pgfpathlineto{\pgfqpoint{5.503441in}{3.978235in}}%
\pgfpathlineto{\pgfqpoint{5.450397in}{3.213681in}}%
\pgfpathlineto{\pgfqpoint{5.397353in}{3.084080in}}%
\pgfpathlineto{\pgfqpoint{5.344309in}{3.073166in}}%
\pgfpathlineto{\pgfqpoint{5.291266in}{3.978235in}}%
\pgfpathlineto{\pgfqpoint{5.238222in}{3.002435in}}%
\pgfpathlineto{\pgfqpoint{5.185178in}{3.006017in}}%
\pgfpathlineto{\pgfqpoint{5.132134in}{2.918032in}}%
\pgfpathlineto{\pgfqpoint{5.079090in}{3.978235in}}%
\pgfpathlineto{\pgfqpoint{5.026046in}{2.958572in}}%
\pgfpathlineto{\pgfqpoint{4.973002in}{3.978235in}}%
\pgfpathlineto{\pgfqpoint{4.919958in}{3.007061in}}%
\pgfpathlineto{\pgfqpoint{4.866914in}{3.020692in}}%
\pgfpathlineto{\pgfqpoint{4.813870in}{3.978235in}}%
\pgfpathlineto{\pgfqpoint{4.760826in}{3.714472in}}%
\pgfpathlineto{\pgfqpoint{4.707782in}{3.102021in}}%
\pgfpathlineto{\pgfqpoint{4.654738in}{3.978235in}}%
\pgfpathlineto{\pgfqpoint{4.601694in}{3.162876in}}%
\pgfpathlineto{\pgfqpoint{4.548650in}{3.689185in}}%
\pgfpathlineto{\pgfqpoint{4.495607in}{3.218773in}}%
\pgfpathlineto{\pgfqpoint{4.442563in}{3.494347in}}%
\pgfpathlineto{\pgfqpoint{4.389519in}{3.841843in}}%
\pgfpathlineto{\pgfqpoint{4.336475in}{3.224846in}}%
\pgfpathlineto{\pgfqpoint{4.283431in}{3.978235in}}%
\pgfpathlineto{\pgfqpoint{4.230387in}{3.978235in}}%
\pgfpathlineto{\pgfqpoint{4.177343in}{3.978235in}}%
\pgfpathlineto{\pgfqpoint{4.124299in}{3.875484in}}%
\pgfpathlineto{\pgfqpoint{4.071255in}{3.978235in}}%
\pgfpathlineto{\pgfqpoint{4.018211in}{3.978235in}}%
\pgfpathlineto{\pgfqpoint{3.965167in}{3.238518in}}%
\pgfpathlineto{\pgfqpoint{3.912123in}{3.408107in}}%
\pgfpathlineto{\pgfqpoint{3.859079in}{3.355689in}}%
\pgfpathlineto{\pgfqpoint{3.806035in}{3.978235in}}%
\pgfpathlineto{\pgfqpoint{3.752991in}{3.388260in}}%
\pgfpathlineto{\pgfqpoint{3.699948in}{3.978235in}}%
\pgfpathlineto{\pgfqpoint{3.646904in}{3.978235in}}%
\pgfpathlineto{\pgfqpoint{3.593860in}{3.503387in}}%
\pgfpathlineto{\pgfqpoint{3.540816in}{3.074646in}}%
\pgfpathlineto{\pgfqpoint{3.487772in}{3.083924in}}%
\pgfpathlineto{\pgfqpoint{3.434728in}{3.978235in}}%
\pgfpathlineto{\pgfqpoint{3.381684in}{3.349122in}}%
\pgfpathlineto{\pgfqpoint{3.328640in}{3.978235in}}%
\pgfpathlineto{\pgfqpoint{3.275596in}{3.978235in}}%
\pgfpathlineto{\pgfqpoint{3.222552in}{3.978235in}}%
\pgfpathlineto{\pgfqpoint{3.169508in}{3.978235in}}%
\pgfpathlineto{\pgfqpoint{3.116464in}{3.506190in}}%
\pgfpathlineto{\pgfqpoint{3.063420in}{3.151674in}}%
\pgfpathlineto{\pgfqpoint{3.010376in}{3.978235in}}%
\pgfpathlineto{\pgfqpoint{2.957332in}{3.172295in}}%
\pgfpathlineto{\pgfqpoint{2.904288in}{3.978235in}}%
\pgfpathlineto{\pgfqpoint{2.851245in}{3.161616in}}%
\pgfpathlineto{\pgfqpoint{2.798201in}{3.978235in}}%
\pgfpathlineto{\pgfqpoint{2.745157in}{3.978235in}}%
\pgfpathlineto{\pgfqpoint{2.692113in}{3.978235in}}%
\pgfpathlineto{\pgfqpoint{2.639069in}{3.978235in}}%
\pgfpathlineto{\pgfqpoint{2.586025in}{3.978235in}}%
\pgfpathlineto{\pgfqpoint{2.532981in}{3.978235in}}%
\pgfpathlineto{\pgfqpoint{2.479937in}{3.785532in}}%
\pgfpathlineto{\pgfqpoint{2.426893in}{2.942415in}}%
\pgfpathlineto{\pgfqpoint{2.373849in}{2.980147in}}%
\pgfpathlineto{\pgfqpoint{2.320805in}{3.978235in}}%
\pgfpathlineto{\pgfqpoint{2.267761in}{3.754982in}}%
\pgfpathlineto{\pgfqpoint{2.214717in}{3.089112in}}%
\pgfpathlineto{\pgfqpoint{2.161673in}{3.978235in}}%
\pgfpathlineto{\pgfqpoint{2.108629in}{3.577804in}}%
\pgfpathlineto{\pgfqpoint{2.055586in}{3.978235in}}%
\pgfpathlineto{\pgfqpoint{2.002542in}{3.193502in}}%
\pgfpathlineto{\pgfqpoint{1.949498in}{3.978235in}}%
\pgfpathlineto{\pgfqpoint{1.896454in}{3.219351in}}%
\pgfpathlineto{\pgfqpoint{1.843410in}{3.247418in}}%
\pgfpathlineto{\pgfqpoint{1.790366in}{3.978235in}}%
\pgfpathlineto{\pgfqpoint{1.737322in}{3.872158in}}%
\pgfpathlineto{\pgfqpoint{1.684278in}{3.440895in}}%
\pgfpathlineto{\pgfqpoint{1.631234in}{3.123835in}}%
\pgfpathlineto{\pgfqpoint{1.578190in}{3.978235in}}%
\pgfpathlineto{\pgfqpoint{1.525146in}{3.978235in}}%
\pgfpathlineto{\pgfqpoint{1.472102in}{3.978235in}}%
\pgfpathlineto{\pgfqpoint{1.419058in}{3.978235in}}%
\pgfpathlineto{\pgfqpoint{1.366014in}{3.978235in}}%
\pgfpathlineto{\pgfqpoint{1.312970in}{3.006099in}}%
\pgfpathlineto{\pgfqpoint{1.259927in}{2.902493in}}%
\pgfpathlineto{\pgfqpoint{1.206883in}{3.101900in}}%
\pgfpathlineto{\pgfqpoint{1.153839in}{3.978235in}}%
\pgfpathlineto{\pgfqpoint{1.100795in}{3.966545in}}%
\pgfpathlineto{\pgfqpoint{1.047751in}{3.978235in}}%
\pgfpathlineto{\pgfqpoint{0.994707in}{3.724101in}}%
\pgfpathlineto{\pgfqpoint{0.941663in}{3.978235in}}%
\pgfpathlineto{\pgfqpoint{0.941663in}{3.978235in}}%
\pgfpathclose%
\pgfusepath{stroke,fill}%
}%
\begin{pgfscope}%
\pgfsys@transformshift{0.000000in}{0.000000in}%
\pgfsys@useobject{currentmarker}{}%
\end{pgfscope}%
\end{pgfscope}%
\begin{pgfscope}%
\pgfsetrectcap%
\pgfsetmiterjoin%
\pgfsetlinewidth{0.803000pt}%
\definecolor{currentstroke}{rgb}{0.000000,0.000000,0.000000}%
\pgfsetstrokecolor{currentstroke}%
\pgfsetdash{}{0pt}%
\pgfpathmoveto{\pgfqpoint{0.941663in}{0.670138in}}%
\pgfpathlineto{\pgfqpoint{0.941663in}{4.135763in}}%
\pgfusepath{stroke}%
\end{pgfscope}%
\begin{pgfscope}%
\pgfsetrectcap%
\pgfsetmiterjoin%
\pgfsetlinewidth{0.803000pt}%
\definecolor{currentstroke}{rgb}{0.000000,0.000000,0.000000}%
\pgfsetstrokecolor{currentstroke}%
\pgfsetdash{}{0pt}%
\pgfpathmoveto{\pgfqpoint{9.800000in}{0.670138in}}%
\pgfpathlineto{\pgfqpoint{9.800000in}{4.135763in}}%
\pgfusepath{stroke}%
\end{pgfscope}%
\begin{pgfscope}%
\pgfsetrectcap%
\pgfsetmiterjoin%
\pgfsetlinewidth{0.803000pt}%
\definecolor{currentstroke}{rgb}{0.000000,0.000000,0.000000}%
\pgfsetstrokecolor{currentstroke}%
\pgfsetdash{}{0pt}%
\pgfpathmoveto{\pgfqpoint{0.941663in}{0.670138in}}%
\pgfpathlineto{\pgfqpoint{9.800000in}{0.670138in}}%
\pgfusepath{stroke}%
\end{pgfscope}%
\begin{pgfscope}%
\pgfsetrectcap%
\pgfsetmiterjoin%
\pgfsetlinewidth{0.803000pt}%
\definecolor{currentstroke}{rgb}{0.000000,0.000000,0.000000}%
\pgfsetstrokecolor{currentstroke}%
\pgfsetdash{}{0pt}%
\pgfpathmoveto{\pgfqpoint{0.941663in}{4.135763in}}%
\pgfpathlineto{\pgfqpoint{9.800000in}{4.135763in}}%
\pgfusepath{stroke}%
\end{pgfscope}%
\begin{pgfscope}%
\pgfpathrectangle{\pgfqpoint{0.941663in}{0.670138in}}{\pgfqpoint{8.858337in}{3.465625in}}%
\pgfusepath{clip}%
\pgfsetbuttcap%
\pgfsetroundjoin%
\pgfsetlinewidth{1.505625pt}%
\definecolor{currentstroke}{rgb}{0.000000,0.000000,0.000000}%
\pgfsetstrokecolor{currentstroke}%
\pgfsetdash{{5.550000pt}{2.400000pt}}{0.000000pt}%
\pgfpathmoveto{\pgfqpoint{0.941663in}{3.029187in}}%
\pgfpathlineto{\pgfqpoint{0.994707in}{3.038981in}}%
\pgfpathlineto{\pgfqpoint{1.047751in}{3.025138in}}%
\pgfpathlineto{\pgfqpoint{1.100795in}{2.982667in}}%
\pgfpathlineto{\pgfqpoint{1.153839in}{2.980694in}}%
\pgfpathlineto{\pgfqpoint{1.206883in}{3.021671in}}%
\pgfpathlineto{\pgfqpoint{1.259927in}{2.902493in}}%
\pgfpathlineto{\pgfqpoint{1.312970in}{2.971095in}}%
\pgfpathlineto{\pgfqpoint{1.366014in}{2.916350in}}%
\pgfpathlineto{\pgfqpoint{1.419058in}{2.990567in}}%
\pgfpathlineto{\pgfqpoint{1.472102in}{2.995930in}}%
\pgfpathlineto{\pgfqpoint{1.525146in}{3.076292in}}%
\pgfpathlineto{\pgfqpoint{1.578190in}{3.053422in}}%
\pgfpathlineto{\pgfqpoint{1.631234in}{3.123835in}}%
\pgfpathlineto{\pgfqpoint{1.684278in}{3.108750in}}%
\pgfpathlineto{\pgfqpoint{1.737322in}{3.112329in}}%
\pgfpathlineto{\pgfqpoint{1.790366in}{3.200828in}}%
\pgfpathlineto{\pgfqpoint{1.843410in}{3.247418in}}%
\pgfpathlineto{\pgfqpoint{1.896454in}{3.219351in}}%
\pgfpathlineto{\pgfqpoint{1.949498in}{3.174139in}}%
\pgfpathlineto{\pgfqpoint{2.002542in}{3.193502in}}%
\pgfpathlineto{\pgfqpoint{2.055586in}{3.184945in}}%
\pgfpathlineto{\pgfqpoint{2.108629in}{3.168248in}}%
\pgfpathlineto{\pgfqpoint{2.161673in}{3.074593in}}%
\pgfpathlineto{\pgfqpoint{2.214717in}{3.089112in}}%
\pgfpathlineto{\pgfqpoint{2.267761in}{3.046318in}}%
\pgfpathlineto{\pgfqpoint{2.320805in}{3.005184in}}%
\pgfpathlineto{\pgfqpoint{2.373849in}{2.980147in}}%
\pgfpathlineto{\pgfqpoint{2.426893in}{2.942415in}}%
\pgfpathlineto{\pgfqpoint{2.479937in}{2.932402in}}%
\pgfpathlineto{\pgfqpoint{2.532981in}{2.980860in}}%
\pgfpathlineto{\pgfqpoint{2.586025in}{2.959049in}}%
\pgfpathlineto{\pgfqpoint{2.639069in}{2.954555in}}%
\pgfpathlineto{\pgfqpoint{2.692113in}{2.924353in}}%
\pgfpathlineto{\pgfqpoint{2.745157in}{2.995277in}}%
\pgfpathlineto{\pgfqpoint{2.798201in}{3.037672in}}%
\pgfpathlineto{\pgfqpoint{2.851245in}{3.051804in}}%
\pgfpathlineto{\pgfqpoint{2.904288in}{3.084579in}}%
\pgfpathlineto{\pgfqpoint{2.957332in}{3.172295in}}%
\pgfpathlineto{\pgfqpoint{3.010376in}{3.170343in}}%
\pgfpathlineto{\pgfqpoint{3.063420in}{3.151674in}}%
\pgfpathlineto{\pgfqpoint{3.116464in}{3.172554in}}%
\pgfpathlineto{\pgfqpoint{3.169508in}{3.202163in}}%
\pgfpathlineto{\pgfqpoint{3.222552in}{3.232686in}}%
\pgfpathlineto{\pgfqpoint{3.275596in}{3.148826in}}%
\pgfpathlineto{\pgfqpoint{3.328640in}{3.207401in}}%
\pgfpathlineto{\pgfqpoint{3.381684in}{3.124374in}}%
\pgfpathlineto{\pgfqpoint{3.434728in}{3.059787in}}%
\pgfpathlineto{\pgfqpoint{3.487772in}{3.083924in}}%
\pgfpathlineto{\pgfqpoint{3.540816in}{3.074646in}}%
\pgfpathlineto{\pgfqpoint{3.593860in}{3.000641in}}%
\pgfpathlineto{\pgfqpoint{3.646904in}{2.998087in}}%
\pgfpathlineto{\pgfqpoint{3.699948in}{2.954238in}}%
\pgfpathlineto{\pgfqpoint{3.752991in}{2.951958in}}%
\pgfpathlineto{\pgfqpoint{3.806035in}{2.994488in}}%
\pgfpathlineto{\pgfqpoint{3.859079in}{2.981887in}}%
\pgfpathlineto{\pgfqpoint{3.912123in}{2.942941in}}%
\pgfpathlineto{\pgfqpoint{3.965167in}{3.071374in}}%
\pgfpathlineto{\pgfqpoint{4.018211in}{3.000463in}}%
\pgfpathlineto{\pgfqpoint{4.071255in}{3.082554in}}%
\pgfpathlineto{\pgfqpoint{4.124299in}{3.107971in}}%
\pgfpathlineto{\pgfqpoint{4.177343in}{3.126443in}}%
\pgfpathlineto{\pgfqpoint{4.230387in}{3.164551in}}%
\pgfpathlineto{\pgfqpoint{4.283431in}{3.180269in}}%
\pgfpathlineto{\pgfqpoint{4.336475in}{3.224846in}}%
\pgfpathlineto{\pgfqpoint{4.389519in}{3.194301in}}%
\pgfpathlineto{\pgfqpoint{4.442563in}{3.183354in}}%
\pgfpathlineto{\pgfqpoint{4.495607in}{3.218773in}}%
\pgfpathlineto{\pgfqpoint{4.548650in}{3.235597in}}%
\pgfpathlineto{\pgfqpoint{4.601694in}{3.162876in}}%
\pgfpathlineto{\pgfqpoint{4.654738in}{3.181880in}}%
\pgfpathlineto{\pgfqpoint{4.707782in}{3.102021in}}%
\pgfpathlineto{\pgfqpoint{4.760826in}{3.110933in}}%
\pgfpathlineto{\pgfqpoint{4.813870in}{3.062144in}}%
\pgfpathlineto{\pgfqpoint{4.866914in}{3.020692in}}%
\pgfpathlineto{\pgfqpoint{4.919958in}{3.007061in}}%
\pgfpathlineto{\pgfqpoint{4.973002in}{2.943141in}}%
\pgfpathlineto{\pgfqpoint{5.026046in}{2.958572in}}%
\pgfpathlineto{\pgfqpoint{5.079090in}{2.914682in}}%
\pgfpathlineto{\pgfqpoint{5.132134in}{2.918032in}}%
\pgfpathlineto{\pgfqpoint{5.185178in}{3.006017in}}%
\pgfpathlineto{\pgfqpoint{5.238222in}{3.002435in}}%
\pgfpathlineto{\pgfqpoint{5.291266in}{3.062668in}}%
\pgfpathlineto{\pgfqpoint{5.344309in}{3.073166in}}%
\pgfpathlineto{\pgfqpoint{5.397353in}{3.084080in}}%
\pgfpathlineto{\pgfqpoint{5.450397in}{3.213681in}}%
\pgfpathlineto{\pgfqpoint{5.503441in}{3.203849in}}%
\pgfpathlineto{\pgfqpoint{5.556485in}{3.196655in}}%
\pgfpathlineto{\pgfqpoint{5.609529in}{3.209959in}}%
\pgfpathlineto{\pgfqpoint{5.662573in}{3.261554in}}%
\pgfpathlineto{\pgfqpoint{5.715617in}{3.209641in}}%
\pgfpathlineto{\pgfqpoint{5.768661in}{3.248973in}}%
\pgfpathlineto{\pgfqpoint{5.821705in}{3.208761in}}%
\pgfpathlineto{\pgfqpoint{5.874749in}{3.132435in}}%
\pgfpathlineto{\pgfqpoint{5.927793in}{3.156691in}}%
\pgfpathlineto{\pgfqpoint{5.980837in}{3.133284in}}%
\pgfpathlineto{\pgfqpoint{6.033881in}{3.084077in}}%
\pgfpathlineto{\pgfqpoint{6.086925in}{3.062147in}}%
\pgfpathlineto{\pgfqpoint{6.139969in}{3.043673in}}%
\pgfpathlineto{\pgfqpoint{6.193012in}{2.971161in}}%
\pgfpathlineto{\pgfqpoint{6.246056in}{3.026875in}}%
\pgfpathlineto{\pgfqpoint{6.299100in}{2.964176in}}%
\pgfpathlineto{\pgfqpoint{6.352144in}{2.982962in}}%
\pgfpathlineto{\pgfqpoint{6.405188in}{2.929750in}}%
\pgfpathlineto{\pgfqpoint{6.458232in}{3.016017in}}%
\pgfpathlineto{\pgfqpoint{6.511276in}{3.033770in}}%
\pgfpathlineto{\pgfqpoint{6.564320in}{3.030119in}}%
\pgfpathlineto{\pgfqpoint{6.617364in}{3.097076in}}%
\pgfpathlineto{\pgfqpoint{6.670408in}{3.127018in}}%
\pgfpathlineto{\pgfqpoint{6.723452in}{3.173100in}}%
\pgfpathlineto{\pgfqpoint{6.776496in}{3.194890in}}%
\pgfpathlineto{\pgfqpoint{6.829540in}{3.185067in}}%
\pgfpathlineto{\pgfqpoint{6.882584in}{3.217123in}}%
\pgfpathlineto{\pgfqpoint{6.935628in}{3.178720in}}%
\pgfpathlineto{\pgfqpoint{6.988671in}{3.222397in}}%
\pgfpathlineto{\pgfqpoint{7.041715in}{3.197983in}}%
\pgfpathlineto{\pgfqpoint{7.094759in}{3.221034in}}%
\pgfpathlineto{\pgfqpoint{7.147803in}{3.202828in}}%
\pgfpathlineto{\pgfqpoint{7.200847in}{3.153935in}}%
\pgfpathlineto{\pgfqpoint{7.253891in}{3.055048in}}%
\pgfpathlineto{\pgfqpoint{7.306935in}{3.083084in}}%
\pgfpathlineto{\pgfqpoint{7.359979in}{3.007952in}}%
\pgfpathlineto{\pgfqpoint{7.413023in}{3.019856in}}%
\pgfpathlineto{\pgfqpoint{7.466067in}{2.989900in}}%
\pgfpathlineto{\pgfqpoint{7.519111in}{2.942028in}}%
\pgfpathlineto{\pgfqpoint{7.572155in}{2.929451in}}%
\pgfpathlineto{\pgfqpoint{7.625199in}{2.988504in}}%
\pgfpathlineto{\pgfqpoint{7.678243in}{3.033419in}}%
\pgfpathlineto{\pgfqpoint{7.731287in}{2.995952in}}%
\pgfpathlineto{\pgfqpoint{7.784330in}{3.031983in}}%
\pgfpathlineto{\pgfqpoint{7.837374in}{3.040235in}}%
\pgfpathlineto{\pgfqpoint{7.890418in}{3.081537in}}%
\pgfpathlineto{\pgfqpoint{7.943462in}{3.079283in}}%
\pgfpathlineto{\pgfqpoint{7.996506in}{3.157986in}}%
\pgfpathlineto{\pgfqpoint{8.049550in}{3.189369in}}%
\pgfpathlineto{\pgfqpoint{8.102594in}{3.198813in}}%
\pgfpathlineto{\pgfqpoint{8.155638in}{3.224623in}}%
\pgfpathlineto{\pgfqpoint{8.208682in}{3.218763in}}%
\pgfpathlineto{\pgfqpoint{8.261726in}{3.260476in}}%
\pgfpathlineto{\pgfqpoint{8.314770in}{3.211850in}}%
\pgfpathlineto{\pgfqpoint{8.367814in}{3.201180in}}%
\pgfpathlineto{\pgfqpoint{8.420858in}{3.193239in}}%
\pgfpathlineto{\pgfqpoint{8.473902in}{3.141958in}}%
\pgfpathlineto{\pgfqpoint{8.526946in}{3.113817in}}%
\pgfpathlineto{\pgfqpoint{8.579990in}{3.021265in}}%
\pgfpathlineto{\pgfqpoint{8.633033in}{3.030538in}}%
\pgfpathlineto{\pgfqpoint{8.686077in}{3.059450in}}%
\pgfpathlineto{\pgfqpoint{8.739121in}{2.997851in}}%
\pgfpathlineto{\pgfqpoint{8.792165in}{3.000505in}}%
\pgfpathlineto{\pgfqpoint{8.845209in}{2.978670in}}%
\pgfpathlineto{\pgfqpoint{8.898253in}{3.021456in}}%
\pgfpathlineto{\pgfqpoint{8.951297in}{2.965023in}}%
\pgfpathlineto{\pgfqpoint{9.004341in}{2.983962in}}%
\pgfpathlineto{\pgfqpoint{9.057385in}{3.049384in}}%
\pgfpathlineto{\pgfqpoint{9.110429in}{3.084477in}}%
\pgfpathlineto{\pgfqpoint{9.163473in}{3.106410in}}%
\pgfpathlineto{\pgfqpoint{9.216517in}{3.140256in}}%
\pgfpathlineto{\pgfqpoint{9.269561in}{3.147248in}}%
\pgfpathlineto{\pgfqpoint{9.322605in}{3.219668in}}%
\pgfpathlineto{\pgfqpoint{9.375649in}{3.214984in}}%
\pgfpathlineto{\pgfqpoint{9.428692in}{3.240840in}}%
\pgfpathlineto{\pgfqpoint{9.481736in}{3.220264in}}%
\pgfpathlineto{\pgfqpoint{9.534780in}{3.282563in}}%
\pgfpathlineto{\pgfqpoint{9.587824in}{3.204226in}}%
\pgfpathlineto{\pgfqpoint{9.640868in}{3.199776in}}%
\pgfpathlineto{\pgfqpoint{9.693912in}{3.139443in}}%
\pgfpathlineto{\pgfqpoint{9.746956in}{3.115475in}}%
\pgfpathlineto{\pgfqpoint{9.800000in}{3.101625in}}%
\pgfpathlineto{\pgfqpoint{9.800000in}{3.101625in}}%
\pgfusepath{stroke}%
\end{pgfscope}%
\begin{pgfscope}%
\pgfsetbuttcap%
\pgfsetmiterjoin%
\definecolor{currentfill}{rgb}{1.000000,1.000000,1.000000}%
\pgfsetfillcolor{currentfill}%
\pgfsetlinewidth{1.003750pt}%
\definecolor{currentstroke}{rgb}{0.000000,0.000000,0.000000}%
\pgfsetstrokecolor{currentstroke}%
\pgfsetdash{}{0pt}%
\pgfpathmoveto{\pgfqpoint{1.017884in}{3.802279in}}%
\pgfpathlineto{\pgfqpoint{1.315613in}{3.802279in}}%
\pgfpathlineto{\pgfqpoint{1.315613in}{4.115057in}}%
\pgfpathlineto{\pgfqpoint{1.017884in}{4.115057in}}%
\pgfpathlineto{\pgfqpoint{1.017884in}{3.802279in}}%
\pgfpathclose%
\pgfusepath{stroke,fill}%
\end{pgfscope}%
\begin{pgfscope}%
\definecolor{textcolor}{rgb}{0.000000,0.000000,0.000000}%
\pgfsetstrokecolor{textcolor}%
\pgfsetfillcolor{textcolor}%
\pgftext[x=1.074273in,y=3.908668in,left,base]{\color{textcolor}{\rmfamily\fontsize{14.000000}{16.800000}\selectfont\catcode`\^=\active\def^{\ifmmode\sp\else\^{}\fi}\catcode`\%=\active\def%{\%}b)}}%
\end{pgfscope}%
\begin{pgfscope}%
\pgfsetbuttcap%
\pgfsetmiterjoin%
\definecolor{currentfill}{rgb}{1.000000,1.000000,1.000000}%
\pgfsetfillcolor{currentfill}%
\pgfsetfillopacity{0.800000}%
\pgfsetlinewidth{1.003750pt}%
\definecolor{currentstroke}{rgb}{0.800000,0.800000,0.800000}%
\pgfsetstrokecolor{currentstroke}%
\pgfsetstrokeopacity{0.800000}%
\pgfsetdash{}{0pt}%
\pgfpathmoveto{\pgfqpoint{1.058330in}{0.753471in}}%
\pgfpathlineto{\pgfqpoint{6.450599in}{0.753471in}}%
\pgfpathquadraticcurveto{\pgfqpoint{6.483933in}{0.753471in}}{\pgfqpoint{6.483933in}{0.786805in}}%
\pgfpathlineto{\pgfqpoint{6.483933in}{1.467359in}}%
\pgfpathquadraticcurveto{\pgfqpoint{6.483933in}{1.500693in}}{\pgfqpoint{6.450599in}{1.500693in}}%
\pgfpathlineto{\pgfqpoint{1.058330in}{1.500693in}}%
\pgfpathquadraticcurveto{\pgfqpoint{1.024996in}{1.500693in}}{\pgfqpoint{1.024996in}{1.467359in}}%
\pgfpathlineto{\pgfqpoint{1.024996in}{0.786805in}}%
\pgfpathquadraticcurveto{\pgfqpoint{1.024996in}{0.753471in}}{\pgfqpoint{1.058330in}{0.753471in}}%
\pgfpathlineto{\pgfqpoint{1.058330in}{0.753471in}}%
\pgfpathclose%
\pgfusepath{stroke,fill}%
\end{pgfscope}%
\begin{pgfscope}%
\pgfsetbuttcap%
\pgfsetmiterjoin%
\definecolor{currentfill}{rgb}{0.121569,0.466667,0.705882}%
\pgfsetfillcolor{currentfill}%
\pgfsetlinewidth{1.003750pt}%
\definecolor{currentstroke}{rgb}{0.121569,0.466667,0.705882}%
\pgfsetstrokecolor{currentstroke}%
\pgfsetdash{}{0pt}%
\pgfpathmoveto{\pgfqpoint{1.091663in}{1.317359in}}%
\pgfpathlineto{\pgfqpoint{1.424996in}{1.317359in}}%
\pgfpathlineto{\pgfqpoint{1.424996in}{1.434026in}}%
\pgfpathlineto{\pgfqpoint{1.091663in}{1.434026in}}%
\pgfpathlineto{\pgfqpoint{1.091663in}{1.317359in}}%
\pgfpathclose%
\pgfusepath{stroke,fill}%
\end{pgfscope}%
\begin{pgfscope}%
\definecolor{textcolor}{rgb}{0.000000,0.000000,0.000000}%
\pgfsetstrokecolor{textcolor}%
\pgfsetfillcolor{textcolor}%
\pgftext[x=1.558330in,y=1.317359in,left,base]{\color{textcolor}{\rmfamily\fontsize{12.000000}{14.400000}\selectfont\catcode`\^=\active\def^{\ifmmode\sp\else\^{}\fi}\catcode`\%=\active\def%{\%}Nuclear}}%
\end{pgfscope}%
\begin{pgfscope}%
\pgfsetbuttcap%
\pgfsetmiterjoin%
\definecolor{currentfill}{rgb}{0.501961,0.000000,0.501961}%
\pgfsetfillcolor{currentfill}%
\pgfsetlinewidth{1.003750pt}%
\definecolor{currentstroke}{rgb}{0.501961,0.000000,0.501961}%
\pgfsetstrokecolor{currentstroke}%
\pgfsetdash{}{0pt}%
\pgfpathmoveto{\pgfqpoint{1.091663in}{1.084952in}}%
\pgfpathlineto{\pgfqpoint{1.424996in}{1.084952in}}%
\pgfpathlineto{\pgfqpoint{1.424996in}{1.201619in}}%
\pgfpathlineto{\pgfqpoint{1.091663in}{1.201619in}}%
\pgfpathlineto{\pgfqpoint{1.091663in}{1.084952in}}%
\pgfpathclose%
\pgfusepath{stroke,fill}%
\end{pgfscope}%
\begin{pgfscope}%
\definecolor{textcolor}{rgb}{0.000000,0.000000,0.000000}%
\pgfsetstrokecolor{textcolor}%
\pgfsetfillcolor{textcolor}%
\pgftext[x=1.558330in,y=1.084952in,left,base]{\color{textcolor}{\rmfamily\fontsize{12.000000}{14.400000}\selectfont\catcode`\^=\active\def^{\ifmmode\sp\else\^{}\fi}\catcode`\%=\active\def%{\%}Battery}}%
\end{pgfscope}%
\begin{pgfscope}%
\pgfsetbuttcap%
\pgfsetmiterjoin%
\definecolor{currentfill}{rgb}{0.549020,0.337255,0.294118}%
\pgfsetfillcolor{currentfill}%
\pgfsetlinewidth{1.003750pt}%
\definecolor{currentstroke}{rgb}{0.549020,0.337255,0.294118}%
\pgfsetstrokecolor{currentstroke}%
\pgfsetdash{}{0pt}%
\pgfpathmoveto{\pgfqpoint{1.091663in}{0.852545in}}%
\pgfpathlineto{\pgfqpoint{1.424996in}{0.852545in}}%
\pgfpathlineto{\pgfqpoint{1.424996in}{0.969212in}}%
\pgfpathlineto{\pgfqpoint{1.091663in}{0.969212in}}%
\pgfpathlineto{\pgfqpoint{1.091663in}{0.852545in}}%
\pgfpathclose%
\pgfusepath{stroke,fill}%
\end{pgfscope}%
\begin{pgfscope}%
\definecolor{textcolor}{rgb}{0.000000,0.000000,0.000000}%
\pgfsetstrokecolor{textcolor}%
\pgfsetfillcolor{textcolor}%
\pgftext[x=1.558330in,y=0.852545in,left,base]{\color{textcolor}{\rmfamily\fontsize{12.000000}{14.400000}\selectfont\catcode`\^=\active\def^{\ifmmode\sp\else\^{}\fi}\catcode`\%=\active\def%{\%}NaturalGas Conv}}%
\end{pgfscope}%
\begin{pgfscope}%
\pgfsetbuttcap%
\pgfsetmiterjoin%
\definecolor{currentfill}{rgb}{1.000000,0.647059,0.000000}%
\pgfsetfillcolor{currentfill}%
\pgfsetlinewidth{1.003750pt}%
\definecolor{currentstroke}{rgb}{1.000000,0.647059,0.000000}%
\pgfsetstrokecolor{currentstroke}%
\pgfsetdash{}{0pt}%
\pgfpathmoveto{\pgfqpoint{3.140242in}{1.317359in}}%
\pgfpathlineto{\pgfqpoint{3.473575in}{1.317359in}}%
\pgfpathlineto{\pgfqpoint{3.473575in}{1.434026in}}%
\pgfpathlineto{\pgfqpoint{3.140242in}{1.434026in}}%
\pgfpathlineto{\pgfqpoint{3.140242in}{1.317359in}}%
\pgfpathclose%
\pgfusepath{stroke,fill}%
\end{pgfscope}%
\begin{pgfscope}%
\definecolor{textcolor}{rgb}{0.000000,0.000000,0.000000}%
\pgfsetstrokecolor{textcolor}%
\pgfsetfillcolor{textcolor}%
\pgftext[x=3.606909in,y=1.317359in,left,base]{\color{textcolor}{\rmfamily\fontsize{12.000000}{14.400000}\selectfont\catcode`\^=\active\def^{\ifmmode\sp\else\^{}\fi}\catcode`\%=\active\def%{\%}Battery charge}}%
\end{pgfscope}%
\begin{pgfscope}%
\pgfsetbuttcap%
\pgfsetmiterjoin%
\definecolor{currentfill}{rgb}{0.501961,0.501961,0.501961}%
\pgfsetfillcolor{currentfill}%
\pgfsetlinewidth{1.003750pt}%
\definecolor{currentstroke}{rgb}{0.501961,0.501961,0.501961}%
\pgfsetstrokecolor{currentstroke}%
\pgfsetdash{}{0pt}%
\pgfpathmoveto{\pgfqpoint{3.140242in}{1.084952in}}%
\pgfpathlineto{\pgfqpoint{3.473575in}{1.084952in}}%
\pgfpathlineto{\pgfqpoint{3.473575in}{1.201619in}}%
\pgfpathlineto{\pgfqpoint{3.140242in}{1.201619in}}%
\pgfpathlineto{\pgfqpoint{3.140242in}{1.084952in}}%
\pgfpathclose%
\pgfusepath{stroke,fill}%
\end{pgfscope}%
\begin{pgfscope}%
\definecolor{textcolor}{rgb}{0.000000,0.000000,0.000000}%
\pgfsetstrokecolor{textcolor}%
\pgfsetfillcolor{textcolor}%
\pgftext[x=3.606909in,y=1.084952in,left,base]{\color{textcolor}{\rmfamily\fontsize{12.000000}{14.400000}\selectfont\catcode`\^=\active\def^{\ifmmode\sp\else\^{}\fi}\catcode`\%=\active\def%{\%}Curtailment}}%
\end{pgfscope}%
\begin{pgfscope}%
\pgfsetbuttcap%
\pgfsetmiterjoin%
\definecolor{currentfill}{rgb}{0.090196,0.745098,0.811765}%
\pgfsetfillcolor{currentfill}%
\pgfsetlinewidth{1.003750pt}%
\definecolor{currentstroke}{rgb}{0.090196,0.745098,0.811765}%
\pgfsetstrokecolor{currentstroke}%
\pgfsetdash{}{0pt}%
\pgfpathmoveto{\pgfqpoint{4.998699in}{1.317359in}}%
\pgfpathlineto{\pgfqpoint{5.332033in}{1.317359in}}%
\pgfpathlineto{\pgfqpoint{5.332033in}{1.434026in}}%
\pgfpathlineto{\pgfqpoint{4.998699in}{1.434026in}}%
\pgfpathlineto{\pgfqpoint{4.998699in}{1.317359in}}%
\pgfpathclose%
\pgfusepath{stroke,fill}%
\end{pgfscope}%
\begin{pgfscope}%
\definecolor{textcolor}{rgb}{0.000000,0.000000,0.000000}%
\pgfsetstrokecolor{textcolor}%
\pgfsetfillcolor{textcolor}%
\pgftext[x=5.465366in,y=1.317359in,left,base]{\color{textcolor}{\rmfamily\fontsize{12.000000}{14.400000}\selectfont\catcode`\^=\active\def^{\ifmmode\sp\else\^{}\fi}\catcode`\%=\active\def%{\%}WindTurbine}}%
\end{pgfscope}%
\begin{pgfscope}%
\pgfsetbuttcap%
\pgfsetroundjoin%
\pgfsetlinewidth{1.505625pt}%
\definecolor{currentstroke}{rgb}{0.000000,0.000000,0.000000}%
\pgfsetstrokecolor{currentstroke}%
\pgfsetdash{{5.550000pt}{2.400000pt}}{0.000000pt}%
\pgfpathmoveto{\pgfqpoint{4.998699in}{1.143286in}}%
\pgfpathlineto{\pgfqpoint{5.165366in}{1.143286in}}%
\pgfpathlineto{\pgfqpoint{5.332033in}{1.143286in}}%
\pgfusepath{stroke}%
\end{pgfscope}%
\begin{pgfscope}%
\definecolor{textcolor}{rgb}{0.000000,0.000000,0.000000}%
\pgfsetstrokecolor{textcolor}%
\pgfsetfillcolor{textcolor}%
\pgftext[x=5.465366in,y=1.084952in,left,base]{\color{textcolor}{\rmfamily\fontsize{12.000000}{14.400000}\selectfont\catcode`\^=\active\def^{\ifmmode\sp\else\^{}\fi}\catcode`\%=\active\def%{\%}Demand}}%
\end{pgfscope}%
\end{pgfpicture}%
\makeatother%
\endgroup%
}
    \caption{Comparison between dispatch results for two algorithms. Plot a) was
    calculated with a logical dispatch algorithm and plot b) was calculated with
    optimal dispatch.}
    \label{fig:dispatch-comparison}
\end{figure}

The optimal dispatch algorithm uses a linear programming formulation to arrive
at an optimal solution with perfect foresight. The logical dispatch algorithm
uses a rule-based approach to dispatch energy according to merit order.
However, this algorithm is myopic since dispatch is calculated serially. These
differences totally account for the differences in their dispatch results. The
optimal dispatch algorithm uses battery storage more effectively than its
rule-based counterpart because it optimizes the entire time series at once.
Since the logical algorithm uses battery storage imperfectly, it fills the
energy gaps with natural gas and more energy is curtailed rather than used.
Although, the optimal dispatch solution performs better on a pure cost basis, the
myopia of the logical dispatch algorithm is possibly more realistic.
Further, since the logical dispatch algorithm does not have an energy balance
constraint for all time steps, users can more easily estimate reliability and
calculate costs from energy shortfalls.


\FloatBarrier

\subsection{Exercise 2: Time Scaling}

This exercise considers how the two dispatch algorithms scale with simulation duration.
For this exercise, the two algorithms were placed within a
\texttt{CapacityExpansion} problem with the parameters described in Table
\ref{tab:scaling-ga-params}.

\begin{table}[htbp!]
    \centering
    \caption{Capacity expansion parameters for the algorithm comparison exercise.}
    \label{tab:scaling-ga-params}
    \begin{tabular}{ll}
        \toprule
        Parameter & Value \\
        \midrule
        Algorithm & \acs{nsga2}\\
        Termination Criterion & Maximum generations\\
        Generations & 10 \\
        Population Size & 20 \\
        Objectives & 2 (cost, emissions)\\
        Threads & 1 \\
        \bottomrule
    \end{tabular}
\end{table}

\noindent The available technologies were the same four as in the previous
exercise. Rather than scaling the problem by number of objectives, technologies,
or population size, this exercise scales the problem by the length of the time
series. This is preferred because time series data typically increases the
problem size more dramatically than the number of objectives or number of
technologies. Further, scaling by population size would obfuscate the
differences between the two algorithms since neither are affected by population
size. Figure \ref{fig:alg-scaling} shows results of this scaling study. The
x-axis measures the number of modeled days at an hourly resolution.

\begin{figure}[htbp!]
    \centering
    \resizebox{0.75\columnwidth}{!}{%% Creator: Matplotlib, PGF backend
%%
%% To include the figure in your LaTeX document, write
%%   \input{<filename>.pgf}
%%
%% Make sure the required packages are loaded in your preamble
%%   \usepackage{pgf}
%%
%% Also ensure that all the required font packages are loaded; for instance,
%% the lmodern package is sometimes necessary when using math font.
%%   \usepackage{lmodern}
%%
%% Figures using additional raster images can only be included by \input if
%% they are in the same directory as the main LaTeX file. For loading figures
%% from other directories you can use the `import` package
%%   \usepackage{import}
%%
%% and then include the figures with
%%   \import{<path to file>}{<filename>.pgf}
%%
%% Matplotlib used the following preamble
%%   \def\mathdefault#1{#1}
%%   \everymath=\expandafter{\the\everymath\displaystyle}
%%   \IfFileExists{scrextend.sty}{
%%     \usepackage[fontsize=10.000000pt]{scrextend}
%%   }{
%%     \renewcommand{\normalsize}{\fontsize{10.000000}{12.000000}\selectfont}
%%     \normalsize
%%   }
%%   
%%   \makeatletter\@ifpackageloaded{underscore}{}{\usepackage[strings]{underscore}}\makeatother
%%
\begingroup%
\makeatletter%
\begin{pgfpicture}%
\pgfpathrectangle{\pgfpointorigin}{\pgfqpoint{7.064581in}{5.411797in}}%
\pgfusepath{use as bounding box, clip}%
\begin{pgfscope}%
\pgfsetbuttcap%
\pgfsetmiterjoin%
\definecolor{currentfill}{rgb}{1.000000,1.000000,1.000000}%
\pgfsetfillcolor{currentfill}%
\pgfsetlinewidth{0.000000pt}%
\definecolor{currentstroke}{rgb}{0.000000,0.000000,0.000000}%
\pgfsetstrokecolor{currentstroke}%
\pgfsetdash{}{0pt}%
\pgfpathmoveto{\pgfqpoint{0.000000in}{0.000000in}}%
\pgfpathlineto{\pgfqpoint{7.064581in}{0.000000in}}%
\pgfpathlineto{\pgfqpoint{7.064581in}{5.411797in}}%
\pgfpathlineto{\pgfqpoint{0.000000in}{5.411797in}}%
\pgfpathlineto{\pgfqpoint{0.000000in}{0.000000in}}%
\pgfpathclose%
\pgfusepath{fill}%
\end{pgfscope}%
\begin{pgfscope}%
\pgfsetbuttcap%
\pgfsetmiterjoin%
\definecolor{currentfill}{rgb}{1.000000,1.000000,1.000000}%
\pgfsetfillcolor{currentfill}%
\pgfsetlinewidth{0.000000pt}%
\definecolor{currentstroke}{rgb}{0.000000,0.000000,0.000000}%
\pgfsetstrokecolor{currentstroke}%
\pgfsetstrokeopacity{0.000000}%
\pgfsetdash{}{0pt}%
\pgfpathmoveto{\pgfqpoint{0.764581in}{0.643904in}}%
\pgfpathlineto{\pgfqpoint{6.964581in}{0.643904in}}%
\pgfpathlineto{\pgfqpoint{6.964581in}{5.263904in}}%
\pgfpathlineto{\pgfqpoint{0.764581in}{5.263904in}}%
\pgfpathlineto{\pgfqpoint{0.764581in}{0.643904in}}%
\pgfpathclose%
\pgfusepath{fill}%
\end{pgfscope}%
\begin{pgfscope}%
\pgfpathrectangle{\pgfqpoint{0.764581in}{0.643904in}}{\pgfqpoint{6.200000in}{4.620000in}}%
\pgfusepath{clip}%
\pgfsetrectcap%
\pgfsetroundjoin%
\pgfsetlinewidth{0.803000pt}%
\definecolor{currentstroke}{rgb}{0.690196,0.690196,0.690196}%
\pgfsetstrokecolor{currentstroke}%
\pgfsetdash{}{0pt}%
\pgfpathmoveto{\pgfqpoint{0.764581in}{0.643904in}}%
\pgfpathlineto{\pgfqpoint{0.764581in}{5.263904in}}%
\pgfusepath{stroke}%
\end{pgfscope}%
\begin{pgfscope}%
\pgfsetbuttcap%
\pgfsetroundjoin%
\definecolor{currentfill}{rgb}{0.000000,0.000000,0.000000}%
\pgfsetfillcolor{currentfill}%
\pgfsetlinewidth{0.803000pt}%
\definecolor{currentstroke}{rgb}{0.000000,0.000000,0.000000}%
\pgfsetstrokecolor{currentstroke}%
\pgfsetdash{}{0pt}%
\pgfsys@defobject{currentmarker}{\pgfqpoint{0.000000in}{-0.048611in}}{\pgfqpoint{0.000000in}{0.000000in}}{%
\pgfpathmoveto{\pgfqpoint{0.000000in}{0.000000in}}%
\pgfpathlineto{\pgfqpoint{0.000000in}{-0.048611in}}%
\pgfusepath{stroke,fill}%
}%
\begin{pgfscope}%
\pgfsys@transformshift{0.764581in}{0.643904in}%
\pgfsys@useobject{currentmarker}{}%
\end{pgfscope}%
\end{pgfscope}%
\begin{pgfscope}%
\definecolor{textcolor}{rgb}{0.000000,0.000000,0.000000}%
\pgfsetstrokecolor{textcolor}%
\pgfsetfillcolor{textcolor}%
\pgftext[x=0.764581in,y=0.546682in,,top]{\color{textcolor}{\rmfamily\fontsize{14.000000}{16.800000}\selectfont\catcode`\^=\active\def^{\ifmmode\sp\else\^{}\fi}\catcode`\%=\active\def%{\%}$\mathdefault{10^{0}}$}}%
\end{pgfscope}%
\begin{pgfscope}%
\pgfpathrectangle{\pgfqpoint{0.764581in}{0.643904in}}{\pgfqpoint{6.200000in}{4.620000in}}%
\pgfusepath{clip}%
\pgfsetrectcap%
\pgfsetroundjoin%
\pgfsetlinewidth{0.803000pt}%
\definecolor{currentstroke}{rgb}{0.690196,0.690196,0.690196}%
\pgfsetstrokecolor{currentstroke}%
\pgfsetdash{}{0pt}%
\pgfpathmoveto{\pgfqpoint{3.184288in}{0.643904in}}%
\pgfpathlineto{\pgfqpoint{3.184288in}{5.263904in}}%
\pgfusepath{stroke}%
\end{pgfscope}%
\begin{pgfscope}%
\pgfsetbuttcap%
\pgfsetroundjoin%
\definecolor{currentfill}{rgb}{0.000000,0.000000,0.000000}%
\pgfsetfillcolor{currentfill}%
\pgfsetlinewidth{0.803000pt}%
\definecolor{currentstroke}{rgb}{0.000000,0.000000,0.000000}%
\pgfsetstrokecolor{currentstroke}%
\pgfsetdash{}{0pt}%
\pgfsys@defobject{currentmarker}{\pgfqpoint{0.000000in}{-0.048611in}}{\pgfqpoint{0.000000in}{0.000000in}}{%
\pgfpathmoveto{\pgfqpoint{0.000000in}{0.000000in}}%
\pgfpathlineto{\pgfqpoint{0.000000in}{-0.048611in}}%
\pgfusepath{stroke,fill}%
}%
\begin{pgfscope}%
\pgfsys@transformshift{3.184288in}{0.643904in}%
\pgfsys@useobject{currentmarker}{}%
\end{pgfscope}%
\end{pgfscope}%
\begin{pgfscope}%
\definecolor{textcolor}{rgb}{0.000000,0.000000,0.000000}%
\pgfsetstrokecolor{textcolor}%
\pgfsetfillcolor{textcolor}%
\pgftext[x=3.184288in,y=0.546682in,,top]{\color{textcolor}{\rmfamily\fontsize{14.000000}{16.800000}\selectfont\catcode`\^=\active\def^{\ifmmode\sp\else\^{}\fi}\catcode`\%=\active\def%{\%}$\mathdefault{10^{1}}$}}%
\end{pgfscope}%
\begin{pgfscope}%
\pgfpathrectangle{\pgfqpoint{0.764581in}{0.643904in}}{\pgfqpoint{6.200000in}{4.620000in}}%
\pgfusepath{clip}%
\pgfsetrectcap%
\pgfsetroundjoin%
\pgfsetlinewidth{0.803000pt}%
\definecolor{currentstroke}{rgb}{0.690196,0.690196,0.690196}%
\pgfsetstrokecolor{currentstroke}%
\pgfsetdash{}{0pt}%
\pgfpathmoveto{\pgfqpoint{5.603996in}{0.643904in}}%
\pgfpathlineto{\pgfqpoint{5.603996in}{5.263904in}}%
\pgfusepath{stroke}%
\end{pgfscope}%
\begin{pgfscope}%
\pgfsetbuttcap%
\pgfsetroundjoin%
\definecolor{currentfill}{rgb}{0.000000,0.000000,0.000000}%
\pgfsetfillcolor{currentfill}%
\pgfsetlinewidth{0.803000pt}%
\definecolor{currentstroke}{rgb}{0.000000,0.000000,0.000000}%
\pgfsetstrokecolor{currentstroke}%
\pgfsetdash{}{0pt}%
\pgfsys@defobject{currentmarker}{\pgfqpoint{0.000000in}{-0.048611in}}{\pgfqpoint{0.000000in}{0.000000in}}{%
\pgfpathmoveto{\pgfqpoint{0.000000in}{0.000000in}}%
\pgfpathlineto{\pgfqpoint{0.000000in}{-0.048611in}}%
\pgfusepath{stroke,fill}%
}%
\begin{pgfscope}%
\pgfsys@transformshift{5.603996in}{0.643904in}%
\pgfsys@useobject{currentmarker}{}%
\end{pgfscope}%
\end{pgfscope}%
\begin{pgfscope}%
\definecolor{textcolor}{rgb}{0.000000,0.000000,0.000000}%
\pgfsetstrokecolor{textcolor}%
\pgfsetfillcolor{textcolor}%
\pgftext[x=5.603996in,y=0.546682in,,top]{\color{textcolor}{\rmfamily\fontsize{14.000000}{16.800000}\selectfont\catcode`\^=\active\def^{\ifmmode\sp\else\^{}\fi}\catcode`\%=\active\def%{\%}$\mathdefault{10^{2}}$}}%
\end{pgfscope}%
\begin{pgfscope}%
\pgfpathrectangle{\pgfqpoint{0.764581in}{0.643904in}}{\pgfqpoint{6.200000in}{4.620000in}}%
\pgfusepath{clip}%
\pgfsetbuttcap%
\pgfsetroundjoin%
\pgfsetlinewidth{0.803000pt}%
\definecolor{currentstroke}{rgb}{0.690196,0.690196,0.690196}%
\pgfsetstrokecolor{currentstroke}%
\pgfsetstrokeopacity{0.200000}%
\pgfsetdash{{2.960000pt}{1.280000pt}}{0.000000pt}%
\pgfpathmoveto{\pgfqpoint{1.492985in}{0.643904in}}%
\pgfpathlineto{\pgfqpoint{1.492985in}{5.263904in}}%
\pgfusepath{stroke}%
\end{pgfscope}%
\begin{pgfscope}%
\pgfsetbuttcap%
\pgfsetroundjoin%
\definecolor{currentfill}{rgb}{0.000000,0.000000,0.000000}%
\pgfsetfillcolor{currentfill}%
\pgfsetlinewidth{0.602250pt}%
\definecolor{currentstroke}{rgb}{0.000000,0.000000,0.000000}%
\pgfsetstrokecolor{currentstroke}%
\pgfsetdash{}{0pt}%
\pgfsys@defobject{currentmarker}{\pgfqpoint{0.000000in}{-0.027778in}}{\pgfqpoint{0.000000in}{0.000000in}}{%
\pgfpathmoveto{\pgfqpoint{0.000000in}{0.000000in}}%
\pgfpathlineto{\pgfqpoint{0.000000in}{-0.027778in}}%
\pgfusepath{stroke,fill}%
}%
\begin{pgfscope}%
\pgfsys@transformshift{1.492985in}{0.643904in}%
\pgfsys@useobject{currentmarker}{}%
\end{pgfscope}%
\end{pgfscope}%
\begin{pgfscope}%
\pgfpathrectangle{\pgfqpoint{0.764581in}{0.643904in}}{\pgfqpoint{6.200000in}{4.620000in}}%
\pgfusepath{clip}%
\pgfsetbuttcap%
\pgfsetroundjoin%
\pgfsetlinewidth{0.803000pt}%
\definecolor{currentstroke}{rgb}{0.690196,0.690196,0.690196}%
\pgfsetstrokecolor{currentstroke}%
\pgfsetstrokeopacity{0.200000}%
\pgfsetdash{{2.960000pt}{1.280000pt}}{0.000000pt}%
\pgfpathmoveto{\pgfqpoint{1.919075in}{0.643904in}}%
\pgfpathlineto{\pgfqpoint{1.919075in}{5.263904in}}%
\pgfusepath{stroke}%
\end{pgfscope}%
\begin{pgfscope}%
\pgfsetbuttcap%
\pgfsetroundjoin%
\definecolor{currentfill}{rgb}{0.000000,0.000000,0.000000}%
\pgfsetfillcolor{currentfill}%
\pgfsetlinewidth{0.602250pt}%
\definecolor{currentstroke}{rgb}{0.000000,0.000000,0.000000}%
\pgfsetstrokecolor{currentstroke}%
\pgfsetdash{}{0pt}%
\pgfsys@defobject{currentmarker}{\pgfqpoint{0.000000in}{-0.027778in}}{\pgfqpoint{0.000000in}{0.000000in}}{%
\pgfpathmoveto{\pgfqpoint{0.000000in}{0.000000in}}%
\pgfpathlineto{\pgfqpoint{0.000000in}{-0.027778in}}%
\pgfusepath{stroke,fill}%
}%
\begin{pgfscope}%
\pgfsys@transformshift{1.919075in}{0.643904in}%
\pgfsys@useobject{currentmarker}{}%
\end{pgfscope}%
\end{pgfscope}%
\begin{pgfscope}%
\pgfpathrectangle{\pgfqpoint{0.764581in}{0.643904in}}{\pgfqpoint{6.200000in}{4.620000in}}%
\pgfusepath{clip}%
\pgfsetbuttcap%
\pgfsetroundjoin%
\pgfsetlinewidth{0.803000pt}%
\definecolor{currentstroke}{rgb}{0.690196,0.690196,0.690196}%
\pgfsetstrokecolor{currentstroke}%
\pgfsetstrokeopacity{0.200000}%
\pgfsetdash{{2.960000pt}{1.280000pt}}{0.000000pt}%
\pgfpathmoveto{\pgfqpoint{2.221390in}{0.643904in}}%
\pgfpathlineto{\pgfqpoint{2.221390in}{5.263904in}}%
\pgfusepath{stroke}%
\end{pgfscope}%
\begin{pgfscope}%
\pgfsetbuttcap%
\pgfsetroundjoin%
\definecolor{currentfill}{rgb}{0.000000,0.000000,0.000000}%
\pgfsetfillcolor{currentfill}%
\pgfsetlinewidth{0.602250pt}%
\definecolor{currentstroke}{rgb}{0.000000,0.000000,0.000000}%
\pgfsetstrokecolor{currentstroke}%
\pgfsetdash{}{0pt}%
\pgfsys@defobject{currentmarker}{\pgfqpoint{0.000000in}{-0.027778in}}{\pgfqpoint{0.000000in}{0.000000in}}{%
\pgfpathmoveto{\pgfqpoint{0.000000in}{0.000000in}}%
\pgfpathlineto{\pgfqpoint{0.000000in}{-0.027778in}}%
\pgfusepath{stroke,fill}%
}%
\begin{pgfscope}%
\pgfsys@transformshift{2.221390in}{0.643904in}%
\pgfsys@useobject{currentmarker}{}%
\end{pgfscope}%
\end{pgfscope}%
\begin{pgfscope}%
\pgfpathrectangle{\pgfqpoint{0.764581in}{0.643904in}}{\pgfqpoint{6.200000in}{4.620000in}}%
\pgfusepath{clip}%
\pgfsetbuttcap%
\pgfsetroundjoin%
\pgfsetlinewidth{0.803000pt}%
\definecolor{currentstroke}{rgb}{0.690196,0.690196,0.690196}%
\pgfsetstrokecolor{currentstroke}%
\pgfsetstrokeopacity{0.200000}%
\pgfsetdash{{2.960000pt}{1.280000pt}}{0.000000pt}%
\pgfpathmoveto{\pgfqpoint{2.455884in}{0.643904in}}%
\pgfpathlineto{\pgfqpoint{2.455884in}{5.263904in}}%
\pgfusepath{stroke}%
\end{pgfscope}%
\begin{pgfscope}%
\pgfsetbuttcap%
\pgfsetroundjoin%
\definecolor{currentfill}{rgb}{0.000000,0.000000,0.000000}%
\pgfsetfillcolor{currentfill}%
\pgfsetlinewidth{0.602250pt}%
\definecolor{currentstroke}{rgb}{0.000000,0.000000,0.000000}%
\pgfsetstrokecolor{currentstroke}%
\pgfsetdash{}{0pt}%
\pgfsys@defobject{currentmarker}{\pgfqpoint{0.000000in}{-0.027778in}}{\pgfqpoint{0.000000in}{0.000000in}}{%
\pgfpathmoveto{\pgfqpoint{0.000000in}{0.000000in}}%
\pgfpathlineto{\pgfqpoint{0.000000in}{-0.027778in}}%
\pgfusepath{stroke,fill}%
}%
\begin{pgfscope}%
\pgfsys@transformshift{2.455884in}{0.643904in}%
\pgfsys@useobject{currentmarker}{}%
\end{pgfscope}%
\end{pgfscope}%
\begin{pgfscope}%
\pgfpathrectangle{\pgfqpoint{0.764581in}{0.643904in}}{\pgfqpoint{6.200000in}{4.620000in}}%
\pgfusepath{clip}%
\pgfsetbuttcap%
\pgfsetroundjoin%
\pgfsetlinewidth{0.803000pt}%
\definecolor{currentstroke}{rgb}{0.690196,0.690196,0.690196}%
\pgfsetstrokecolor{currentstroke}%
\pgfsetstrokeopacity{0.200000}%
\pgfsetdash{{2.960000pt}{1.280000pt}}{0.000000pt}%
\pgfpathmoveto{\pgfqpoint{2.647479in}{0.643904in}}%
\pgfpathlineto{\pgfqpoint{2.647479in}{5.263904in}}%
\pgfusepath{stroke}%
\end{pgfscope}%
\begin{pgfscope}%
\pgfsetbuttcap%
\pgfsetroundjoin%
\definecolor{currentfill}{rgb}{0.000000,0.000000,0.000000}%
\pgfsetfillcolor{currentfill}%
\pgfsetlinewidth{0.602250pt}%
\definecolor{currentstroke}{rgb}{0.000000,0.000000,0.000000}%
\pgfsetstrokecolor{currentstroke}%
\pgfsetdash{}{0pt}%
\pgfsys@defobject{currentmarker}{\pgfqpoint{0.000000in}{-0.027778in}}{\pgfqpoint{0.000000in}{0.000000in}}{%
\pgfpathmoveto{\pgfqpoint{0.000000in}{0.000000in}}%
\pgfpathlineto{\pgfqpoint{0.000000in}{-0.027778in}}%
\pgfusepath{stroke,fill}%
}%
\begin{pgfscope}%
\pgfsys@transformshift{2.647479in}{0.643904in}%
\pgfsys@useobject{currentmarker}{}%
\end{pgfscope}%
\end{pgfscope}%
\begin{pgfscope}%
\pgfpathrectangle{\pgfqpoint{0.764581in}{0.643904in}}{\pgfqpoint{6.200000in}{4.620000in}}%
\pgfusepath{clip}%
\pgfsetbuttcap%
\pgfsetroundjoin%
\pgfsetlinewidth{0.803000pt}%
\definecolor{currentstroke}{rgb}{0.690196,0.690196,0.690196}%
\pgfsetstrokecolor{currentstroke}%
\pgfsetstrokeopacity{0.200000}%
\pgfsetdash{{2.960000pt}{1.280000pt}}{0.000000pt}%
\pgfpathmoveto{\pgfqpoint{2.809471in}{0.643904in}}%
\pgfpathlineto{\pgfqpoint{2.809471in}{5.263904in}}%
\pgfusepath{stroke}%
\end{pgfscope}%
\begin{pgfscope}%
\pgfsetbuttcap%
\pgfsetroundjoin%
\definecolor{currentfill}{rgb}{0.000000,0.000000,0.000000}%
\pgfsetfillcolor{currentfill}%
\pgfsetlinewidth{0.602250pt}%
\definecolor{currentstroke}{rgb}{0.000000,0.000000,0.000000}%
\pgfsetstrokecolor{currentstroke}%
\pgfsetdash{}{0pt}%
\pgfsys@defobject{currentmarker}{\pgfqpoint{0.000000in}{-0.027778in}}{\pgfqpoint{0.000000in}{0.000000in}}{%
\pgfpathmoveto{\pgfqpoint{0.000000in}{0.000000in}}%
\pgfpathlineto{\pgfqpoint{0.000000in}{-0.027778in}}%
\pgfusepath{stroke,fill}%
}%
\begin{pgfscope}%
\pgfsys@transformshift{2.809471in}{0.643904in}%
\pgfsys@useobject{currentmarker}{}%
\end{pgfscope}%
\end{pgfscope}%
\begin{pgfscope}%
\pgfpathrectangle{\pgfqpoint{0.764581in}{0.643904in}}{\pgfqpoint{6.200000in}{4.620000in}}%
\pgfusepath{clip}%
\pgfsetbuttcap%
\pgfsetroundjoin%
\pgfsetlinewidth{0.803000pt}%
\definecolor{currentstroke}{rgb}{0.690196,0.690196,0.690196}%
\pgfsetstrokecolor{currentstroke}%
\pgfsetstrokeopacity{0.200000}%
\pgfsetdash{{2.960000pt}{1.280000pt}}{0.000000pt}%
\pgfpathmoveto{\pgfqpoint{2.949795in}{0.643904in}}%
\pgfpathlineto{\pgfqpoint{2.949795in}{5.263904in}}%
\pgfusepath{stroke}%
\end{pgfscope}%
\begin{pgfscope}%
\pgfsetbuttcap%
\pgfsetroundjoin%
\definecolor{currentfill}{rgb}{0.000000,0.000000,0.000000}%
\pgfsetfillcolor{currentfill}%
\pgfsetlinewidth{0.602250pt}%
\definecolor{currentstroke}{rgb}{0.000000,0.000000,0.000000}%
\pgfsetstrokecolor{currentstroke}%
\pgfsetdash{}{0pt}%
\pgfsys@defobject{currentmarker}{\pgfqpoint{0.000000in}{-0.027778in}}{\pgfqpoint{0.000000in}{0.000000in}}{%
\pgfpathmoveto{\pgfqpoint{0.000000in}{0.000000in}}%
\pgfpathlineto{\pgfqpoint{0.000000in}{-0.027778in}}%
\pgfusepath{stroke,fill}%
}%
\begin{pgfscope}%
\pgfsys@transformshift{2.949795in}{0.643904in}%
\pgfsys@useobject{currentmarker}{}%
\end{pgfscope}%
\end{pgfscope}%
\begin{pgfscope}%
\pgfpathrectangle{\pgfqpoint{0.764581in}{0.643904in}}{\pgfqpoint{6.200000in}{4.620000in}}%
\pgfusepath{clip}%
\pgfsetbuttcap%
\pgfsetroundjoin%
\pgfsetlinewidth{0.803000pt}%
\definecolor{currentstroke}{rgb}{0.690196,0.690196,0.690196}%
\pgfsetstrokecolor{currentstroke}%
\pgfsetstrokeopacity{0.200000}%
\pgfsetdash{{2.960000pt}{1.280000pt}}{0.000000pt}%
\pgfpathmoveto{\pgfqpoint{3.073569in}{0.643904in}}%
\pgfpathlineto{\pgfqpoint{3.073569in}{5.263904in}}%
\pgfusepath{stroke}%
\end{pgfscope}%
\begin{pgfscope}%
\pgfsetbuttcap%
\pgfsetroundjoin%
\definecolor{currentfill}{rgb}{0.000000,0.000000,0.000000}%
\pgfsetfillcolor{currentfill}%
\pgfsetlinewidth{0.602250pt}%
\definecolor{currentstroke}{rgb}{0.000000,0.000000,0.000000}%
\pgfsetstrokecolor{currentstroke}%
\pgfsetdash{}{0pt}%
\pgfsys@defobject{currentmarker}{\pgfqpoint{0.000000in}{-0.027778in}}{\pgfqpoint{0.000000in}{0.000000in}}{%
\pgfpathmoveto{\pgfqpoint{0.000000in}{0.000000in}}%
\pgfpathlineto{\pgfqpoint{0.000000in}{-0.027778in}}%
\pgfusepath{stroke,fill}%
}%
\begin{pgfscope}%
\pgfsys@transformshift{3.073569in}{0.643904in}%
\pgfsys@useobject{currentmarker}{}%
\end{pgfscope}%
\end{pgfscope}%
\begin{pgfscope}%
\pgfpathrectangle{\pgfqpoint{0.764581in}{0.643904in}}{\pgfqpoint{6.200000in}{4.620000in}}%
\pgfusepath{clip}%
\pgfsetbuttcap%
\pgfsetroundjoin%
\pgfsetlinewidth{0.803000pt}%
\definecolor{currentstroke}{rgb}{0.690196,0.690196,0.690196}%
\pgfsetstrokecolor{currentstroke}%
\pgfsetstrokeopacity{0.200000}%
\pgfsetdash{{2.960000pt}{1.280000pt}}{0.000000pt}%
\pgfpathmoveto{\pgfqpoint{3.912693in}{0.643904in}}%
\pgfpathlineto{\pgfqpoint{3.912693in}{5.263904in}}%
\pgfusepath{stroke}%
\end{pgfscope}%
\begin{pgfscope}%
\pgfsetbuttcap%
\pgfsetroundjoin%
\definecolor{currentfill}{rgb}{0.000000,0.000000,0.000000}%
\pgfsetfillcolor{currentfill}%
\pgfsetlinewidth{0.602250pt}%
\definecolor{currentstroke}{rgb}{0.000000,0.000000,0.000000}%
\pgfsetstrokecolor{currentstroke}%
\pgfsetdash{}{0pt}%
\pgfsys@defobject{currentmarker}{\pgfqpoint{0.000000in}{-0.027778in}}{\pgfqpoint{0.000000in}{0.000000in}}{%
\pgfpathmoveto{\pgfqpoint{0.000000in}{0.000000in}}%
\pgfpathlineto{\pgfqpoint{0.000000in}{-0.027778in}}%
\pgfusepath{stroke,fill}%
}%
\begin{pgfscope}%
\pgfsys@transformshift{3.912693in}{0.643904in}%
\pgfsys@useobject{currentmarker}{}%
\end{pgfscope}%
\end{pgfscope}%
\begin{pgfscope}%
\pgfpathrectangle{\pgfqpoint{0.764581in}{0.643904in}}{\pgfqpoint{6.200000in}{4.620000in}}%
\pgfusepath{clip}%
\pgfsetbuttcap%
\pgfsetroundjoin%
\pgfsetlinewidth{0.803000pt}%
\definecolor{currentstroke}{rgb}{0.690196,0.690196,0.690196}%
\pgfsetstrokecolor{currentstroke}%
\pgfsetstrokeopacity{0.200000}%
\pgfsetdash{{2.960000pt}{1.280000pt}}{0.000000pt}%
\pgfpathmoveto{\pgfqpoint{4.338782in}{0.643904in}}%
\pgfpathlineto{\pgfqpoint{4.338782in}{5.263904in}}%
\pgfusepath{stroke}%
\end{pgfscope}%
\begin{pgfscope}%
\pgfsetbuttcap%
\pgfsetroundjoin%
\definecolor{currentfill}{rgb}{0.000000,0.000000,0.000000}%
\pgfsetfillcolor{currentfill}%
\pgfsetlinewidth{0.602250pt}%
\definecolor{currentstroke}{rgb}{0.000000,0.000000,0.000000}%
\pgfsetstrokecolor{currentstroke}%
\pgfsetdash{}{0pt}%
\pgfsys@defobject{currentmarker}{\pgfqpoint{0.000000in}{-0.027778in}}{\pgfqpoint{0.000000in}{0.000000in}}{%
\pgfpathmoveto{\pgfqpoint{0.000000in}{0.000000in}}%
\pgfpathlineto{\pgfqpoint{0.000000in}{-0.027778in}}%
\pgfusepath{stroke,fill}%
}%
\begin{pgfscope}%
\pgfsys@transformshift{4.338782in}{0.643904in}%
\pgfsys@useobject{currentmarker}{}%
\end{pgfscope}%
\end{pgfscope}%
\begin{pgfscope}%
\pgfpathrectangle{\pgfqpoint{0.764581in}{0.643904in}}{\pgfqpoint{6.200000in}{4.620000in}}%
\pgfusepath{clip}%
\pgfsetbuttcap%
\pgfsetroundjoin%
\pgfsetlinewidth{0.803000pt}%
\definecolor{currentstroke}{rgb}{0.690196,0.690196,0.690196}%
\pgfsetstrokecolor{currentstroke}%
\pgfsetstrokeopacity{0.200000}%
\pgfsetdash{{2.960000pt}{1.280000pt}}{0.000000pt}%
\pgfpathmoveto{\pgfqpoint{4.641098in}{0.643904in}}%
\pgfpathlineto{\pgfqpoint{4.641098in}{5.263904in}}%
\pgfusepath{stroke}%
\end{pgfscope}%
\begin{pgfscope}%
\pgfsetbuttcap%
\pgfsetroundjoin%
\definecolor{currentfill}{rgb}{0.000000,0.000000,0.000000}%
\pgfsetfillcolor{currentfill}%
\pgfsetlinewidth{0.602250pt}%
\definecolor{currentstroke}{rgb}{0.000000,0.000000,0.000000}%
\pgfsetstrokecolor{currentstroke}%
\pgfsetdash{}{0pt}%
\pgfsys@defobject{currentmarker}{\pgfqpoint{0.000000in}{-0.027778in}}{\pgfqpoint{0.000000in}{0.000000in}}{%
\pgfpathmoveto{\pgfqpoint{0.000000in}{0.000000in}}%
\pgfpathlineto{\pgfqpoint{0.000000in}{-0.027778in}}%
\pgfusepath{stroke,fill}%
}%
\begin{pgfscope}%
\pgfsys@transformshift{4.641098in}{0.643904in}%
\pgfsys@useobject{currentmarker}{}%
\end{pgfscope}%
\end{pgfscope}%
\begin{pgfscope}%
\pgfpathrectangle{\pgfqpoint{0.764581in}{0.643904in}}{\pgfqpoint{6.200000in}{4.620000in}}%
\pgfusepath{clip}%
\pgfsetbuttcap%
\pgfsetroundjoin%
\pgfsetlinewidth{0.803000pt}%
\definecolor{currentstroke}{rgb}{0.690196,0.690196,0.690196}%
\pgfsetstrokecolor{currentstroke}%
\pgfsetstrokeopacity{0.200000}%
\pgfsetdash{{2.960000pt}{1.280000pt}}{0.000000pt}%
\pgfpathmoveto{\pgfqpoint{4.875592in}{0.643904in}}%
\pgfpathlineto{\pgfqpoint{4.875592in}{5.263904in}}%
\pgfusepath{stroke}%
\end{pgfscope}%
\begin{pgfscope}%
\pgfsetbuttcap%
\pgfsetroundjoin%
\definecolor{currentfill}{rgb}{0.000000,0.000000,0.000000}%
\pgfsetfillcolor{currentfill}%
\pgfsetlinewidth{0.602250pt}%
\definecolor{currentstroke}{rgb}{0.000000,0.000000,0.000000}%
\pgfsetstrokecolor{currentstroke}%
\pgfsetdash{}{0pt}%
\pgfsys@defobject{currentmarker}{\pgfqpoint{0.000000in}{-0.027778in}}{\pgfqpoint{0.000000in}{0.000000in}}{%
\pgfpathmoveto{\pgfqpoint{0.000000in}{0.000000in}}%
\pgfpathlineto{\pgfqpoint{0.000000in}{-0.027778in}}%
\pgfusepath{stroke,fill}%
}%
\begin{pgfscope}%
\pgfsys@transformshift{4.875592in}{0.643904in}%
\pgfsys@useobject{currentmarker}{}%
\end{pgfscope}%
\end{pgfscope}%
\begin{pgfscope}%
\pgfpathrectangle{\pgfqpoint{0.764581in}{0.643904in}}{\pgfqpoint{6.200000in}{4.620000in}}%
\pgfusepath{clip}%
\pgfsetbuttcap%
\pgfsetroundjoin%
\pgfsetlinewidth{0.803000pt}%
\definecolor{currentstroke}{rgb}{0.690196,0.690196,0.690196}%
\pgfsetstrokecolor{currentstroke}%
\pgfsetstrokeopacity{0.200000}%
\pgfsetdash{{2.960000pt}{1.280000pt}}{0.000000pt}%
\pgfpathmoveto{\pgfqpoint{5.067187in}{0.643904in}}%
\pgfpathlineto{\pgfqpoint{5.067187in}{5.263904in}}%
\pgfusepath{stroke}%
\end{pgfscope}%
\begin{pgfscope}%
\pgfsetbuttcap%
\pgfsetroundjoin%
\definecolor{currentfill}{rgb}{0.000000,0.000000,0.000000}%
\pgfsetfillcolor{currentfill}%
\pgfsetlinewidth{0.602250pt}%
\definecolor{currentstroke}{rgb}{0.000000,0.000000,0.000000}%
\pgfsetstrokecolor{currentstroke}%
\pgfsetdash{}{0pt}%
\pgfsys@defobject{currentmarker}{\pgfqpoint{0.000000in}{-0.027778in}}{\pgfqpoint{0.000000in}{0.000000in}}{%
\pgfpathmoveto{\pgfqpoint{0.000000in}{0.000000in}}%
\pgfpathlineto{\pgfqpoint{0.000000in}{-0.027778in}}%
\pgfusepath{stroke,fill}%
}%
\begin{pgfscope}%
\pgfsys@transformshift{5.067187in}{0.643904in}%
\pgfsys@useobject{currentmarker}{}%
\end{pgfscope}%
\end{pgfscope}%
\begin{pgfscope}%
\pgfpathrectangle{\pgfqpoint{0.764581in}{0.643904in}}{\pgfqpoint{6.200000in}{4.620000in}}%
\pgfusepath{clip}%
\pgfsetbuttcap%
\pgfsetroundjoin%
\pgfsetlinewidth{0.803000pt}%
\definecolor{currentstroke}{rgb}{0.690196,0.690196,0.690196}%
\pgfsetstrokecolor{currentstroke}%
\pgfsetstrokeopacity{0.200000}%
\pgfsetdash{{2.960000pt}{1.280000pt}}{0.000000pt}%
\pgfpathmoveto{\pgfqpoint{5.229179in}{0.643904in}}%
\pgfpathlineto{\pgfqpoint{5.229179in}{5.263904in}}%
\pgfusepath{stroke}%
\end{pgfscope}%
\begin{pgfscope}%
\pgfsetbuttcap%
\pgfsetroundjoin%
\definecolor{currentfill}{rgb}{0.000000,0.000000,0.000000}%
\pgfsetfillcolor{currentfill}%
\pgfsetlinewidth{0.602250pt}%
\definecolor{currentstroke}{rgb}{0.000000,0.000000,0.000000}%
\pgfsetstrokecolor{currentstroke}%
\pgfsetdash{}{0pt}%
\pgfsys@defobject{currentmarker}{\pgfqpoint{0.000000in}{-0.027778in}}{\pgfqpoint{0.000000in}{0.000000in}}{%
\pgfpathmoveto{\pgfqpoint{0.000000in}{0.000000in}}%
\pgfpathlineto{\pgfqpoint{0.000000in}{-0.027778in}}%
\pgfusepath{stroke,fill}%
}%
\begin{pgfscope}%
\pgfsys@transformshift{5.229179in}{0.643904in}%
\pgfsys@useobject{currentmarker}{}%
\end{pgfscope}%
\end{pgfscope}%
\begin{pgfscope}%
\pgfpathrectangle{\pgfqpoint{0.764581in}{0.643904in}}{\pgfqpoint{6.200000in}{4.620000in}}%
\pgfusepath{clip}%
\pgfsetbuttcap%
\pgfsetroundjoin%
\pgfsetlinewidth{0.803000pt}%
\definecolor{currentstroke}{rgb}{0.690196,0.690196,0.690196}%
\pgfsetstrokecolor{currentstroke}%
\pgfsetstrokeopacity{0.200000}%
\pgfsetdash{{2.960000pt}{1.280000pt}}{0.000000pt}%
\pgfpathmoveto{\pgfqpoint{5.369502in}{0.643904in}}%
\pgfpathlineto{\pgfqpoint{5.369502in}{5.263904in}}%
\pgfusepath{stroke}%
\end{pgfscope}%
\begin{pgfscope}%
\pgfsetbuttcap%
\pgfsetroundjoin%
\definecolor{currentfill}{rgb}{0.000000,0.000000,0.000000}%
\pgfsetfillcolor{currentfill}%
\pgfsetlinewidth{0.602250pt}%
\definecolor{currentstroke}{rgb}{0.000000,0.000000,0.000000}%
\pgfsetstrokecolor{currentstroke}%
\pgfsetdash{}{0pt}%
\pgfsys@defobject{currentmarker}{\pgfqpoint{0.000000in}{-0.027778in}}{\pgfqpoint{0.000000in}{0.000000in}}{%
\pgfpathmoveto{\pgfqpoint{0.000000in}{0.000000in}}%
\pgfpathlineto{\pgfqpoint{0.000000in}{-0.027778in}}%
\pgfusepath{stroke,fill}%
}%
\begin{pgfscope}%
\pgfsys@transformshift{5.369502in}{0.643904in}%
\pgfsys@useobject{currentmarker}{}%
\end{pgfscope}%
\end{pgfscope}%
\begin{pgfscope}%
\pgfpathrectangle{\pgfqpoint{0.764581in}{0.643904in}}{\pgfqpoint{6.200000in}{4.620000in}}%
\pgfusepath{clip}%
\pgfsetbuttcap%
\pgfsetroundjoin%
\pgfsetlinewidth{0.803000pt}%
\definecolor{currentstroke}{rgb}{0.690196,0.690196,0.690196}%
\pgfsetstrokecolor{currentstroke}%
\pgfsetstrokeopacity{0.200000}%
\pgfsetdash{{2.960000pt}{1.280000pt}}{0.000000pt}%
\pgfpathmoveto{\pgfqpoint{5.493277in}{0.643904in}}%
\pgfpathlineto{\pgfqpoint{5.493277in}{5.263904in}}%
\pgfusepath{stroke}%
\end{pgfscope}%
\begin{pgfscope}%
\pgfsetbuttcap%
\pgfsetroundjoin%
\definecolor{currentfill}{rgb}{0.000000,0.000000,0.000000}%
\pgfsetfillcolor{currentfill}%
\pgfsetlinewidth{0.602250pt}%
\definecolor{currentstroke}{rgb}{0.000000,0.000000,0.000000}%
\pgfsetstrokecolor{currentstroke}%
\pgfsetdash{}{0pt}%
\pgfsys@defobject{currentmarker}{\pgfqpoint{0.000000in}{-0.027778in}}{\pgfqpoint{0.000000in}{0.000000in}}{%
\pgfpathmoveto{\pgfqpoint{0.000000in}{0.000000in}}%
\pgfpathlineto{\pgfqpoint{0.000000in}{-0.027778in}}%
\pgfusepath{stroke,fill}%
}%
\begin{pgfscope}%
\pgfsys@transformshift{5.493277in}{0.643904in}%
\pgfsys@useobject{currentmarker}{}%
\end{pgfscope}%
\end{pgfscope}%
\begin{pgfscope}%
\pgfpathrectangle{\pgfqpoint{0.764581in}{0.643904in}}{\pgfqpoint{6.200000in}{4.620000in}}%
\pgfusepath{clip}%
\pgfsetbuttcap%
\pgfsetroundjoin%
\pgfsetlinewidth{0.803000pt}%
\definecolor{currentstroke}{rgb}{0.690196,0.690196,0.690196}%
\pgfsetstrokecolor{currentstroke}%
\pgfsetstrokeopacity{0.200000}%
\pgfsetdash{{2.960000pt}{1.280000pt}}{0.000000pt}%
\pgfpathmoveto{\pgfqpoint{6.332401in}{0.643904in}}%
\pgfpathlineto{\pgfqpoint{6.332401in}{5.263904in}}%
\pgfusepath{stroke}%
\end{pgfscope}%
\begin{pgfscope}%
\pgfsetbuttcap%
\pgfsetroundjoin%
\definecolor{currentfill}{rgb}{0.000000,0.000000,0.000000}%
\pgfsetfillcolor{currentfill}%
\pgfsetlinewidth{0.602250pt}%
\definecolor{currentstroke}{rgb}{0.000000,0.000000,0.000000}%
\pgfsetstrokecolor{currentstroke}%
\pgfsetdash{}{0pt}%
\pgfsys@defobject{currentmarker}{\pgfqpoint{0.000000in}{-0.027778in}}{\pgfqpoint{0.000000in}{0.000000in}}{%
\pgfpathmoveto{\pgfqpoint{0.000000in}{0.000000in}}%
\pgfpathlineto{\pgfqpoint{0.000000in}{-0.027778in}}%
\pgfusepath{stroke,fill}%
}%
\begin{pgfscope}%
\pgfsys@transformshift{6.332401in}{0.643904in}%
\pgfsys@useobject{currentmarker}{}%
\end{pgfscope}%
\end{pgfscope}%
\begin{pgfscope}%
\pgfpathrectangle{\pgfqpoint{0.764581in}{0.643904in}}{\pgfqpoint{6.200000in}{4.620000in}}%
\pgfusepath{clip}%
\pgfsetbuttcap%
\pgfsetroundjoin%
\pgfsetlinewidth{0.803000pt}%
\definecolor{currentstroke}{rgb}{0.690196,0.690196,0.690196}%
\pgfsetstrokecolor{currentstroke}%
\pgfsetstrokeopacity{0.200000}%
\pgfsetdash{{2.960000pt}{1.280000pt}}{0.000000pt}%
\pgfpathmoveto{\pgfqpoint{6.758490in}{0.643904in}}%
\pgfpathlineto{\pgfqpoint{6.758490in}{5.263904in}}%
\pgfusepath{stroke}%
\end{pgfscope}%
\begin{pgfscope}%
\pgfsetbuttcap%
\pgfsetroundjoin%
\definecolor{currentfill}{rgb}{0.000000,0.000000,0.000000}%
\pgfsetfillcolor{currentfill}%
\pgfsetlinewidth{0.602250pt}%
\definecolor{currentstroke}{rgb}{0.000000,0.000000,0.000000}%
\pgfsetstrokecolor{currentstroke}%
\pgfsetdash{}{0pt}%
\pgfsys@defobject{currentmarker}{\pgfqpoint{0.000000in}{-0.027778in}}{\pgfqpoint{0.000000in}{0.000000in}}{%
\pgfpathmoveto{\pgfqpoint{0.000000in}{0.000000in}}%
\pgfpathlineto{\pgfqpoint{0.000000in}{-0.027778in}}%
\pgfusepath{stroke,fill}%
}%
\begin{pgfscope}%
\pgfsys@transformshift{6.758490in}{0.643904in}%
\pgfsys@useobject{currentmarker}{}%
\end{pgfscope}%
\end{pgfscope}%
\begin{pgfscope}%
\definecolor{textcolor}{rgb}{0.000000,0.000000,0.000000}%
\pgfsetstrokecolor{textcolor}%
\pgfsetfillcolor{textcolor}%
\pgftext[x=3.864581in,y=0.313349in,,top]{\color{textcolor}{\rmfamily\fontsize{18.000000}{21.600000}\selectfont\catcode`\^=\active\def^{\ifmmode\sp\else\^{}\fi}\catcode`\%=\active\def%{\%}Number of Modeled Days}}%
\end{pgfscope}%
\begin{pgfscope}%
\pgfpathrectangle{\pgfqpoint{0.764581in}{0.643904in}}{\pgfqpoint{6.200000in}{4.620000in}}%
\pgfusepath{clip}%
\pgfsetrectcap%
\pgfsetroundjoin%
\pgfsetlinewidth{0.803000pt}%
\definecolor{currentstroke}{rgb}{0.690196,0.690196,0.690196}%
\pgfsetstrokecolor{currentstroke}%
\pgfsetdash{}{0pt}%
\pgfpathmoveto{\pgfqpoint{0.764581in}{2.067685in}}%
\pgfpathlineto{\pgfqpoint{6.964581in}{2.067685in}}%
\pgfusepath{stroke}%
\end{pgfscope}%
\begin{pgfscope}%
\pgfsetbuttcap%
\pgfsetroundjoin%
\definecolor{currentfill}{rgb}{0.000000,0.000000,0.000000}%
\pgfsetfillcolor{currentfill}%
\pgfsetlinewidth{0.803000pt}%
\definecolor{currentstroke}{rgb}{0.000000,0.000000,0.000000}%
\pgfsetstrokecolor{currentstroke}%
\pgfsetdash{}{0pt}%
\pgfsys@defobject{currentmarker}{\pgfqpoint{-0.048611in}{0.000000in}}{\pgfqpoint{-0.000000in}{0.000000in}}{%
\pgfpathmoveto{\pgfqpoint{-0.000000in}{0.000000in}}%
\pgfpathlineto{\pgfqpoint{-0.048611in}{0.000000in}}%
\pgfusepath{stroke,fill}%
}%
\begin{pgfscope}%
\pgfsys@transformshift{0.764581in}{2.067685in}%
\pgfsys@useobject{currentmarker}{}%
\end{pgfscope}%
\end{pgfscope}%
\begin{pgfscope}%
\definecolor{textcolor}{rgb}{0.000000,0.000000,0.000000}%
\pgfsetstrokecolor{textcolor}%
\pgfsetfillcolor{textcolor}%
\pgftext[x=0.395138in, y=1.998240in, left, base]{\color{textcolor}{\rmfamily\fontsize{14.000000}{16.800000}\selectfont\catcode`\^=\active\def^{\ifmmode\sp\else\^{}\fi}\catcode`\%=\active\def%{\%}$\mathdefault{10^{1}}$}}%
\end{pgfscope}%
\begin{pgfscope}%
\pgfpathrectangle{\pgfqpoint{0.764581in}{0.643904in}}{\pgfqpoint{6.200000in}{4.620000in}}%
\pgfusepath{clip}%
\pgfsetrectcap%
\pgfsetroundjoin%
\pgfsetlinewidth{0.803000pt}%
\definecolor{currentstroke}{rgb}{0.690196,0.690196,0.690196}%
\pgfsetstrokecolor{currentstroke}%
\pgfsetdash{}{0pt}%
\pgfpathmoveto{\pgfqpoint{0.764581in}{3.655018in}}%
\pgfpathlineto{\pgfqpoint{6.964581in}{3.655018in}}%
\pgfusepath{stroke}%
\end{pgfscope}%
\begin{pgfscope}%
\pgfsetbuttcap%
\pgfsetroundjoin%
\definecolor{currentfill}{rgb}{0.000000,0.000000,0.000000}%
\pgfsetfillcolor{currentfill}%
\pgfsetlinewidth{0.803000pt}%
\definecolor{currentstroke}{rgb}{0.000000,0.000000,0.000000}%
\pgfsetstrokecolor{currentstroke}%
\pgfsetdash{}{0pt}%
\pgfsys@defobject{currentmarker}{\pgfqpoint{-0.048611in}{0.000000in}}{\pgfqpoint{-0.000000in}{0.000000in}}{%
\pgfpathmoveto{\pgfqpoint{-0.000000in}{0.000000in}}%
\pgfpathlineto{\pgfqpoint{-0.048611in}{0.000000in}}%
\pgfusepath{stroke,fill}%
}%
\begin{pgfscope}%
\pgfsys@transformshift{0.764581in}{3.655018in}%
\pgfsys@useobject{currentmarker}{}%
\end{pgfscope}%
\end{pgfscope}%
\begin{pgfscope}%
\definecolor{textcolor}{rgb}{0.000000,0.000000,0.000000}%
\pgfsetstrokecolor{textcolor}%
\pgfsetfillcolor{textcolor}%
\pgftext[x=0.395138in, y=3.585574in, left, base]{\color{textcolor}{\rmfamily\fontsize{14.000000}{16.800000}\selectfont\catcode`\^=\active\def^{\ifmmode\sp\else\^{}\fi}\catcode`\%=\active\def%{\%}$\mathdefault{10^{2}}$}}%
\end{pgfscope}%
\begin{pgfscope}%
\pgfpathrectangle{\pgfqpoint{0.764581in}{0.643904in}}{\pgfqpoint{6.200000in}{4.620000in}}%
\pgfusepath{clip}%
\pgfsetrectcap%
\pgfsetroundjoin%
\pgfsetlinewidth{0.803000pt}%
\definecolor{currentstroke}{rgb}{0.690196,0.690196,0.690196}%
\pgfsetstrokecolor{currentstroke}%
\pgfsetdash{}{0pt}%
\pgfpathmoveto{\pgfqpoint{0.764581in}{5.242352in}}%
\pgfpathlineto{\pgfqpoint{6.964581in}{5.242352in}}%
\pgfusepath{stroke}%
\end{pgfscope}%
\begin{pgfscope}%
\pgfsetbuttcap%
\pgfsetroundjoin%
\definecolor{currentfill}{rgb}{0.000000,0.000000,0.000000}%
\pgfsetfillcolor{currentfill}%
\pgfsetlinewidth{0.803000pt}%
\definecolor{currentstroke}{rgb}{0.000000,0.000000,0.000000}%
\pgfsetstrokecolor{currentstroke}%
\pgfsetdash{}{0pt}%
\pgfsys@defobject{currentmarker}{\pgfqpoint{-0.048611in}{0.000000in}}{\pgfqpoint{-0.000000in}{0.000000in}}{%
\pgfpathmoveto{\pgfqpoint{-0.000000in}{0.000000in}}%
\pgfpathlineto{\pgfqpoint{-0.048611in}{0.000000in}}%
\pgfusepath{stroke,fill}%
}%
\begin{pgfscope}%
\pgfsys@transformshift{0.764581in}{5.242352in}%
\pgfsys@useobject{currentmarker}{}%
\end{pgfscope}%
\end{pgfscope}%
\begin{pgfscope}%
\definecolor{textcolor}{rgb}{0.000000,0.000000,0.000000}%
\pgfsetstrokecolor{textcolor}%
\pgfsetfillcolor{textcolor}%
\pgftext[x=0.395138in, y=5.172908in, left, base]{\color{textcolor}{\rmfamily\fontsize{14.000000}{16.800000}\selectfont\catcode`\^=\active\def^{\ifmmode\sp\else\^{}\fi}\catcode`\%=\active\def%{\%}$\mathdefault{10^{3}}$}}%
\end{pgfscope}%
\begin{pgfscope}%
\pgfpathrectangle{\pgfqpoint{0.764581in}{0.643904in}}{\pgfqpoint{6.200000in}{4.620000in}}%
\pgfusepath{clip}%
\pgfsetbuttcap%
\pgfsetroundjoin%
\pgfsetlinewidth{0.803000pt}%
\definecolor{currentstroke}{rgb}{0.690196,0.690196,0.690196}%
\pgfsetstrokecolor{currentstroke}%
\pgfsetstrokeopacity{0.200000}%
\pgfsetdash{{2.960000pt}{1.280000pt}}{0.000000pt}%
\pgfpathmoveto{\pgfqpoint{0.764581in}{0.958186in}}%
\pgfpathlineto{\pgfqpoint{6.964581in}{0.958186in}}%
\pgfusepath{stroke}%
\end{pgfscope}%
\begin{pgfscope}%
\pgfsetbuttcap%
\pgfsetroundjoin%
\definecolor{currentfill}{rgb}{0.000000,0.000000,0.000000}%
\pgfsetfillcolor{currentfill}%
\pgfsetlinewidth{0.602250pt}%
\definecolor{currentstroke}{rgb}{0.000000,0.000000,0.000000}%
\pgfsetstrokecolor{currentstroke}%
\pgfsetdash{}{0pt}%
\pgfsys@defobject{currentmarker}{\pgfqpoint{-0.027778in}{0.000000in}}{\pgfqpoint{-0.000000in}{0.000000in}}{%
\pgfpathmoveto{\pgfqpoint{-0.000000in}{0.000000in}}%
\pgfpathlineto{\pgfqpoint{-0.027778in}{0.000000in}}%
\pgfusepath{stroke,fill}%
}%
\begin{pgfscope}%
\pgfsys@transformshift{0.764581in}{0.958186in}%
\pgfsys@useobject{currentmarker}{}%
\end{pgfscope}%
\end{pgfscope}%
\begin{pgfscope}%
\pgfpathrectangle{\pgfqpoint{0.764581in}{0.643904in}}{\pgfqpoint{6.200000in}{4.620000in}}%
\pgfusepath{clip}%
\pgfsetbuttcap%
\pgfsetroundjoin%
\pgfsetlinewidth{0.803000pt}%
\definecolor{currentstroke}{rgb}{0.690196,0.690196,0.690196}%
\pgfsetstrokecolor{currentstroke}%
\pgfsetstrokeopacity{0.200000}%
\pgfsetdash{{2.960000pt}{1.280000pt}}{0.000000pt}%
\pgfpathmoveto{\pgfqpoint{0.764581in}{1.237701in}}%
\pgfpathlineto{\pgfqpoint{6.964581in}{1.237701in}}%
\pgfusepath{stroke}%
\end{pgfscope}%
\begin{pgfscope}%
\pgfsetbuttcap%
\pgfsetroundjoin%
\definecolor{currentfill}{rgb}{0.000000,0.000000,0.000000}%
\pgfsetfillcolor{currentfill}%
\pgfsetlinewidth{0.602250pt}%
\definecolor{currentstroke}{rgb}{0.000000,0.000000,0.000000}%
\pgfsetstrokecolor{currentstroke}%
\pgfsetdash{}{0pt}%
\pgfsys@defobject{currentmarker}{\pgfqpoint{-0.027778in}{0.000000in}}{\pgfqpoint{-0.000000in}{0.000000in}}{%
\pgfpathmoveto{\pgfqpoint{-0.000000in}{0.000000in}}%
\pgfpathlineto{\pgfqpoint{-0.027778in}{0.000000in}}%
\pgfusepath{stroke,fill}%
}%
\begin{pgfscope}%
\pgfsys@transformshift{0.764581in}{1.237701in}%
\pgfsys@useobject{currentmarker}{}%
\end{pgfscope}%
\end{pgfscope}%
\begin{pgfscope}%
\pgfpathrectangle{\pgfqpoint{0.764581in}{0.643904in}}{\pgfqpoint{6.200000in}{4.620000in}}%
\pgfusepath{clip}%
\pgfsetbuttcap%
\pgfsetroundjoin%
\pgfsetlinewidth{0.803000pt}%
\definecolor{currentstroke}{rgb}{0.690196,0.690196,0.690196}%
\pgfsetstrokecolor{currentstroke}%
\pgfsetstrokeopacity{0.200000}%
\pgfsetdash{{2.960000pt}{1.280000pt}}{0.000000pt}%
\pgfpathmoveto{\pgfqpoint{0.764581in}{1.436021in}}%
\pgfpathlineto{\pgfqpoint{6.964581in}{1.436021in}}%
\pgfusepath{stroke}%
\end{pgfscope}%
\begin{pgfscope}%
\pgfsetbuttcap%
\pgfsetroundjoin%
\definecolor{currentfill}{rgb}{0.000000,0.000000,0.000000}%
\pgfsetfillcolor{currentfill}%
\pgfsetlinewidth{0.602250pt}%
\definecolor{currentstroke}{rgb}{0.000000,0.000000,0.000000}%
\pgfsetstrokecolor{currentstroke}%
\pgfsetdash{}{0pt}%
\pgfsys@defobject{currentmarker}{\pgfqpoint{-0.027778in}{0.000000in}}{\pgfqpoint{-0.000000in}{0.000000in}}{%
\pgfpathmoveto{\pgfqpoint{-0.000000in}{0.000000in}}%
\pgfpathlineto{\pgfqpoint{-0.027778in}{0.000000in}}%
\pgfusepath{stroke,fill}%
}%
\begin{pgfscope}%
\pgfsys@transformshift{0.764581in}{1.436021in}%
\pgfsys@useobject{currentmarker}{}%
\end{pgfscope}%
\end{pgfscope}%
\begin{pgfscope}%
\pgfpathrectangle{\pgfqpoint{0.764581in}{0.643904in}}{\pgfqpoint{6.200000in}{4.620000in}}%
\pgfusepath{clip}%
\pgfsetbuttcap%
\pgfsetroundjoin%
\pgfsetlinewidth{0.803000pt}%
\definecolor{currentstroke}{rgb}{0.690196,0.690196,0.690196}%
\pgfsetstrokecolor{currentstroke}%
\pgfsetstrokeopacity{0.200000}%
\pgfsetdash{{2.960000pt}{1.280000pt}}{0.000000pt}%
\pgfpathmoveto{\pgfqpoint{0.764581in}{1.589849in}}%
\pgfpathlineto{\pgfqpoint{6.964581in}{1.589849in}}%
\pgfusepath{stroke}%
\end{pgfscope}%
\begin{pgfscope}%
\pgfsetbuttcap%
\pgfsetroundjoin%
\definecolor{currentfill}{rgb}{0.000000,0.000000,0.000000}%
\pgfsetfillcolor{currentfill}%
\pgfsetlinewidth{0.602250pt}%
\definecolor{currentstroke}{rgb}{0.000000,0.000000,0.000000}%
\pgfsetstrokecolor{currentstroke}%
\pgfsetdash{}{0pt}%
\pgfsys@defobject{currentmarker}{\pgfqpoint{-0.027778in}{0.000000in}}{\pgfqpoint{-0.000000in}{0.000000in}}{%
\pgfpathmoveto{\pgfqpoint{-0.000000in}{0.000000in}}%
\pgfpathlineto{\pgfqpoint{-0.027778in}{0.000000in}}%
\pgfusepath{stroke,fill}%
}%
\begin{pgfscope}%
\pgfsys@transformshift{0.764581in}{1.589849in}%
\pgfsys@useobject{currentmarker}{}%
\end{pgfscope}%
\end{pgfscope}%
\begin{pgfscope}%
\pgfpathrectangle{\pgfqpoint{0.764581in}{0.643904in}}{\pgfqpoint{6.200000in}{4.620000in}}%
\pgfusepath{clip}%
\pgfsetbuttcap%
\pgfsetroundjoin%
\pgfsetlinewidth{0.803000pt}%
\definecolor{currentstroke}{rgb}{0.690196,0.690196,0.690196}%
\pgfsetstrokecolor{currentstroke}%
\pgfsetstrokeopacity{0.200000}%
\pgfsetdash{{2.960000pt}{1.280000pt}}{0.000000pt}%
\pgfpathmoveto{\pgfqpoint{0.764581in}{1.715536in}}%
\pgfpathlineto{\pgfqpoint{6.964581in}{1.715536in}}%
\pgfusepath{stroke}%
\end{pgfscope}%
\begin{pgfscope}%
\pgfsetbuttcap%
\pgfsetroundjoin%
\definecolor{currentfill}{rgb}{0.000000,0.000000,0.000000}%
\pgfsetfillcolor{currentfill}%
\pgfsetlinewidth{0.602250pt}%
\definecolor{currentstroke}{rgb}{0.000000,0.000000,0.000000}%
\pgfsetstrokecolor{currentstroke}%
\pgfsetdash{}{0pt}%
\pgfsys@defobject{currentmarker}{\pgfqpoint{-0.027778in}{0.000000in}}{\pgfqpoint{-0.000000in}{0.000000in}}{%
\pgfpathmoveto{\pgfqpoint{-0.000000in}{0.000000in}}%
\pgfpathlineto{\pgfqpoint{-0.027778in}{0.000000in}}%
\pgfusepath{stroke,fill}%
}%
\begin{pgfscope}%
\pgfsys@transformshift{0.764581in}{1.715536in}%
\pgfsys@useobject{currentmarker}{}%
\end{pgfscope}%
\end{pgfscope}%
\begin{pgfscope}%
\pgfpathrectangle{\pgfqpoint{0.764581in}{0.643904in}}{\pgfqpoint{6.200000in}{4.620000in}}%
\pgfusepath{clip}%
\pgfsetbuttcap%
\pgfsetroundjoin%
\pgfsetlinewidth{0.803000pt}%
\definecolor{currentstroke}{rgb}{0.690196,0.690196,0.690196}%
\pgfsetstrokecolor{currentstroke}%
\pgfsetstrokeopacity{0.200000}%
\pgfsetdash{{2.960000pt}{1.280000pt}}{0.000000pt}%
\pgfpathmoveto{\pgfqpoint{0.764581in}{1.821803in}}%
\pgfpathlineto{\pgfqpoint{6.964581in}{1.821803in}}%
\pgfusepath{stroke}%
\end{pgfscope}%
\begin{pgfscope}%
\pgfsetbuttcap%
\pgfsetroundjoin%
\definecolor{currentfill}{rgb}{0.000000,0.000000,0.000000}%
\pgfsetfillcolor{currentfill}%
\pgfsetlinewidth{0.602250pt}%
\definecolor{currentstroke}{rgb}{0.000000,0.000000,0.000000}%
\pgfsetstrokecolor{currentstroke}%
\pgfsetdash{}{0pt}%
\pgfsys@defobject{currentmarker}{\pgfqpoint{-0.027778in}{0.000000in}}{\pgfqpoint{-0.000000in}{0.000000in}}{%
\pgfpathmoveto{\pgfqpoint{-0.000000in}{0.000000in}}%
\pgfpathlineto{\pgfqpoint{-0.027778in}{0.000000in}}%
\pgfusepath{stroke,fill}%
}%
\begin{pgfscope}%
\pgfsys@transformshift{0.764581in}{1.821803in}%
\pgfsys@useobject{currentmarker}{}%
\end{pgfscope}%
\end{pgfscope}%
\begin{pgfscope}%
\pgfpathrectangle{\pgfqpoint{0.764581in}{0.643904in}}{\pgfqpoint{6.200000in}{4.620000in}}%
\pgfusepath{clip}%
\pgfsetbuttcap%
\pgfsetroundjoin%
\pgfsetlinewidth{0.803000pt}%
\definecolor{currentstroke}{rgb}{0.690196,0.690196,0.690196}%
\pgfsetstrokecolor{currentstroke}%
\pgfsetstrokeopacity{0.200000}%
\pgfsetdash{{2.960000pt}{1.280000pt}}{0.000000pt}%
\pgfpathmoveto{\pgfqpoint{0.764581in}{1.913856in}}%
\pgfpathlineto{\pgfqpoint{6.964581in}{1.913856in}}%
\pgfusepath{stroke}%
\end{pgfscope}%
\begin{pgfscope}%
\pgfsetbuttcap%
\pgfsetroundjoin%
\definecolor{currentfill}{rgb}{0.000000,0.000000,0.000000}%
\pgfsetfillcolor{currentfill}%
\pgfsetlinewidth{0.602250pt}%
\definecolor{currentstroke}{rgb}{0.000000,0.000000,0.000000}%
\pgfsetstrokecolor{currentstroke}%
\pgfsetdash{}{0pt}%
\pgfsys@defobject{currentmarker}{\pgfqpoint{-0.027778in}{0.000000in}}{\pgfqpoint{-0.000000in}{0.000000in}}{%
\pgfpathmoveto{\pgfqpoint{-0.000000in}{0.000000in}}%
\pgfpathlineto{\pgfqpoint{-0.027778in}{0.000000in}}%
\pgfusepath{stroke,fill}%
}%
\begin{pgfscope}%
\pgfsys@transformshift{0.764581in}{1.913856in}%
\pgfsys@useobject{currentmarker}{}%
\end{pgfscope}%
\end{pgfscope}%
\begin{pgfscope}%
\pgfpathrectangle{\pgfqpoint{0.764581in}{0.643904in}}{\pgfqpoint{6.200000in}{4.620000in}}%
\pgfusepath{clip}%
\pgfsetbuttcap%
\pgfsetroundjoin%
\pgfsetlinewidth{0.803000pt}%
\definecolor{currentstroke}{rgb}{0.690196,0.690196,0.690196}%
\pgfsetstrokecolor{currentstroke}%
\pgfsetstrokeopacity{0.200000}%
\pgfsetdash{{2.960000pt}{1.280000pt}}{0.000000pt}%
\pgfpathmoveto{\pgfqpoint{0.764581in}{1.995052in}}%
\pgfpathlineto{\pgfqpoint{6.964581in}{1.995052in}}%
\pgfusepath{stroke}%
\end{pgfscope}%
\begin{pgfscope}%
\pgfsetbuttcap%
\pgfsetroundjoin%
\definecolor{currentfill}{rgb}{0.000000,0.000000,0.000000}%
\pgfsetfillcolor{currentfill}%
\pgfsetlinewidth{0.602250pt}%
\definecolor{currentstroke}{rgb}{0.000000,0.000000,0.000000}%
\pgfsetstrokecolor{currentstroke}%
\pgfsetdash{}{0pt}%
\pgfsys@defobject{currentmarker}{\pgfqpoint{-0.027778in}{0.000000in}}{\pgfqpoint{-0.000000in}{0.000000in}}{%
\pgfpathmoveto{\pgfqpoint{-0.000000in}{0.000000in}}%
\pgfpathlineto{\pgfqpoint{-0.027778in}{0.000000in}}%
\pgfusepath{stroke,fill}%
}%
\begin{pgfscope}%
\pgfsys@transformshift{0.764581in}{1.995052in}%
\pgfsys@useobject{currentmarker}{}%
\end{pgfscope}%
\end{pgfscope}%
\begin{pgfscope}%
\pgfpathrectangle{\pgfqpoint{0.764581in}{0.643904in}}{\pgfqpoint{6.200000in}{4.620000in}}%
\pgfusepath{clip}%
\pgfsetbuttcap%
\pgfsetroundjoin%
\pgfsetlinewidth{0.803000pt}%
\definecolor{currentstroke}{rgb}{0.690196,0.690196,0.690196}%
\pgfsetstrokecolor{currentstroke}%
\pgfsetstrokeopacity{0.200000}%
\pgfsetdash{{2.960000pt}{1.280000pt}}{0.000000pt}%
\pgfpathmoveto{\pgfqpoint{0.764581in}{2.545520in}}%
\pgfpathlineto{\pgfqpoint{6.964581in}{2.545520in}}%
\pgfusepath{stroke}%
\end{pgfscope}%
\begin{pgfscope}%
\pgfsetbuttcap%
\pgfsetroundjoin%
\definecolor{currentfill}{rgb}{0.000000,0.000000,0.000000}%
\pgfsetfillcolor{currentfill}%
\pgfsetlinewidth{0.602250pt}%
\definecolor{currentstroke}{rgb}{0.000000,0.000000,0.000000}%
\pgfsetstrokecolor{currentstroke}%
\pgfsetdash{}{0pt}%
\pgfsys@defobject{currentmarker}{\pgfqpoint{-0.027778in}{0.000000in}}{\pgfqpoint{-0.000000in}{0.000000in}}{%
\pgfpathmoveto{\pgfqpoint{-0.000000in}{0.000000in}}%
\pgfpathlineto{\pgfqpoint{-0.027778in}{0.000000in}}%
\pgfusepath{stroke,fill}%
}%
\begin{pgfscope}%
\pgfsys@transformshift{0.764581in}{2.545520in}%
\pgfsys@useobject{currentmarker}{}%
\end{pgfscope}%
\end{pgfscope}%
\begin{pgfscope}%
\pgfpathrectangle{\pgfqpoint{0.764581in}{0.643904in}}{\pgfqpoint{6.200000in}{4.620000in}}%
\pgfusepath{clip}%
\pgfsetbuttcap%
\pgfsetroundjoin%
\pgfsetlinewidth{0.803000pt}%
\definecolor{currentstroke}{rgb}{0.690196,0.690196,0.690196}%
\pgfsetstrokecolor{currentstroke}%
\pgfsetstrokeopacity{0.200000}%
\pgfsetdash{{2.960000pt}{1.280000pt}}{0.000000pt}%
\pgfpathmoveto{\pgfqpoint{0.764581in}{2.825035in}}%
\pgfpathlineto{\pgfqpoint{6.964581in}{2.825035in}}%
\pgfusepath{stroke}%
\end{pgfscope}%
\begin{pgfscope}%
\pgfsetbuttcap%
\pgfsetroundjoin%
\definecolor{currentfill}{rgb}{0.000000,0.000000,0.000000}%
\pgfsetfillcolor{currentfill}%
\pgfsetlinewidth{0.602250pt}%
\definecolor{currentstroke}{rgb}{0.000000,0.000000,0.000000}%
\pgfsetstrokecolor{currentstroke}%
\pgfsetdash{}{0pt}%
\pgfsys@defobject{currentmarker}{\pgfqpoint{-0.027778in}{0.000000in}}{\pgfqpoint{-0.000000in}{0.000000in}}{%
\pgfpathmoveto{\pgfqpoint{-0.000000in}{0.000000in}}%
\pgfpathlineto{\pgfqpoint{-0.027778in}{0.000000in}}%
\pgfusepath{stroke,fill}%
}%
\begin{pgfscope}%
\pgfsys@transformshift{0.764581in}{2.825035in}%
\pgfsys@useobject{currentmarker}{}%
\end{pgfscope}%
\end{pgfscope}%
\begin{pgfscope}%
\pgfpathrectangle{\pgfqpoint{0.764581in}{0.643904in}}{\pgfqpoint{6.200000in}{4.620000in}}%
\pgfusepath{clip}%
\pgfsetbuttcap%
\pgfsetroundjoin%
\pgfsetlinewidth{0.803000pt}%
\definecolor{currentstroke}{rgb}{0.690196,0.690196,0.690196}%
\pgfsetstrokecolor{currentstroke}%
\pgfsetstrokeopacity{0.200000}%
\pgfsetdash{{2.960000pt}{1.280000pt}}{0.000000pt}%
\pgfpathmoveto{\pgfqpoint{0.764581in}{3.023355in}}%
\pgfpathlineto{\pgfqpoint{6.964581in}{3.023355in}}%
\pgfusepath{stroke}%
\end{pgfscope}%
\begin{pgfscope}%
\pgfsetbuttcap%
\pgfsetroundjoin%
\definecolor{currentfill}{rgb}{0.000000,0.000000,0.000000}%
\pgfsetfillcolor{currentfill}%
\pgfsetlinewidth{0.602250pt}%
\definecolor{currentstroke}{rgb}{0.000000,0.000000,0.000000}%
\pgfsetstrokecolor{currentstroke}%
\pgfsetdash{}{0pt}%
\pgfsys@defobject{currentmarker}{\pgfqpoint{-0.027778in}{0.000000in}}{\pgfqpoint{-0.000000in}{0.000000in}}{%
\pgfpathmoveto{\pgfqpoint{-0.000000in}{0.000000in}}%
\pgfpathlineto{\pgfqpoint{-0.027778in}{0.000000in}}%
\pgfusepath{stroke,fill}%
}%
\begin{pgfscope}%
\pgfsys@transformshift{0.764581in}{3.023355in}%
\pgfsys@useobject{currentmarker}{}%
\end{pgfscope}%
\end{pgfscope}%
\begin{pgfscope}%
\pgfpathrectangle{\pgfqpoint{0.764581in}{0.643904in}}{\pgfqpoint{6.200000in}{4.620000in}}%
\pgfusepath{clip}%
\pgfsetbuttcap%
\pgfsetroundjoin%
\pgfsetlinewidth{0.803000pt}%
\definecolor{currentstroke}{rgb}{0.690196,0.690196,0.690196}%
\pgfsetstrokecolor{currentstroke}%
\pgfsetstrokeopacity{0.200000}%
\pgfsetdash{{2.960000pt}{1.280000pt}}{0.000000pt}%
\pgfpathmoveto{\pgfqpoint{0.764581in}{3.177183in}}%
\pgfpathlineto{\pgfqpoint{6.964581in}{3.177183in}}%
\pgfusepath{stroke}%
\end{pgfscope}%
\begin{pgfscope}%
\pgfsetbuttcap%
\pgfsetroundjoin%
\definecolor{currentfill}{rgb}{0.000000,0.000000,0.000000}%
\pgfsetfillcolor{currentfill}%
\pgfsetlinewidth{0.602250pt}%
\definecolor{currentstroke}{rgb}{0.000000,0.000000,0.000000}%
\pgfsetstrokecolor{currentstroke}%
\pgfsetdash{}{0pt}%
\pgfsys@defobject{currentmarker}{\pgfqpoint{-0.027778in}{0.000000in}}{\pgfqpoint{-0.000000in}{0.000000in}}{%
\pgfpathmoveto{\pgfqpoint{-0.000000in}{0.000000in}}%
\pgfpathlineto{\pgfqpoint{-0.027778in}{0.000000in}}%
\pgfusepath{stroke,fill}%
}%
\begin{pgfscope}%
\pgfsys@transformshift{0.764581in}{3.177183in}%
\pgfsys@useobject{currentmarker}{}%
\end{pgfscope}%
\end{pgfscope}%
\begin{pgfscope}%
\pgfpathrectangle{\pgfqpoint{0.764581in}{0.643904in}}{\pgfqpoint{6.200000in}{4.620000in}}%
\pgfusepath{clip}%
\pgfsetbuttcap%
\pgfsetroundjoin%
\pgfsetlinewidth{0.803000pt}%
\definecolor{currentstroke}{rgb}{0.690196,0.690196,0.690196}%
\pgfsetstrokecolor{currentstroke}%
\pgfsetstrokeopacity{0.200000}%
\pgfsetdash{{2.960000pt}{1.280000pt}}{0.000000pt}%
\pgfpathmoveto{\pgfqpoint{0.764581in}{3.302870in}}%
\pgfpathlineto{\pgfqpoint{6.964581in}{3.302870in}}%
\pgfusepath{stroke}%
\end{pgfscope}%
\begin{pgfscope}%
\pgfsetbuttcap%
\pgfsetroundjoin%
\definecolor{currentfill}{rgb}{0.000000,0.000000,0.000000}%
\pgfsetfillcolor{currentfill}%
\pgfsetlinewidth{0.602250pt}%
\definecolor{currentstroke}{rgb}{0.000000,0.000000,0.000000}%
\pgfsetstrokecolor{currentstroke}%
\pgfsetdash{}{0pt}%
\pgfsys@defobject{currentmarker}{\pgfqpoint{-0.027778in}{0.000000in}}{\pgfqpoint{-0.000000in}{0.000000in}}{%
\pgfpathmoveto{\pgfqpoint{-0.000000in}{0.000000in}}%
\pgfpathlineto{\pgfqpoint{-0.027778in}{0.000000in}}%
\pgfusepath{stroke,fill}%
}%
\begin{pgfscope}%
\pgfsys@transformshift{0.764581in}{3.302870in}%
\pgfsys@useobject{currentmarker}{}%
\end{pgfscope}%
\end{pgfscope}%
\begin{pgfscope}%
\pgfpathrectangle{\pgfqpoint{0.764581in}{0.643904in}}{\pgfqpoint{6.200000in}{4.620000in}}%
\pgfusepath{clip}%
\pgfsetbuttcap%
\pgfsetroundjoin%
\pgfsetlinewidth{0.803000pt}%
\definecolor{currentstroke}{rgb}{0.690196,0.690196,0.690196}%
\pgfsetstrokecolor{currentstroke}%
\pgfsetstrokeopacity{0.200000}%
\pgfsetdash{{2.960000pt}{1.280000pt}}{0.000000pt}%
\pgfpathmoveto{\pgfqpoint{0.764581in}{3.409137in}}%
\pgfpathlineto{\pgfqpoint{6.964581in}{3.409137in}}%
\pgfusepath{stroke}%
\end{pgfscope}%
\begin{pgfscope}%
\pgfsetbuttcap%
\pgfsetroundjoin%
\definecolor{currentfill}{rgb}{0.000000,0.000000,0.000000}%
\pgfsetfillcolor{currentfill}%
\pgfsetlinewidth{0.602250pt}%
\definecolor{currentstroke}{rgb}{0.000000,0.000000,0.000000}%
\pgfsetstrokecolor{currentstroke}%
\pgfsetdash{}{0pt}%
\pgfsys@defobject{currentmarker}{\pgfqpoint{-0.027778in}{0.000000in}}{\pgfqpoint{-0.000000in}{0.000000in}}{%
\pgfpathmoveto{\pgfqpoint{-0.000000in}{0.000000in}}%
\pgfpathlineto{\pgfqpoint{-0.027778in}{0.000000in}}%
\pgfusepath{stroke,fill}%
}%
\begin{pgfscope}%
\pgfsys@transformshift{0.764581in}{3.409137in}%
\pgfsys@useobject{currentmarker}{}%
\end{pgfscope}%
\end{pgfscope}%
\begin{pgfscope}%
\pgfpathrectangle{\pgfqpoint{0.764581in}{0.643904in}}{\pgfqpoint{6.200000in}{4.620000in}}%
\pgfusepath{clip}%
\pgfsetbuttcap%
\pgfsetroundjoin%
\pgfsetlinewidth{0.803000pt}%
\definecolor{currentstroke}{rgb}{0.690196,0.690196,0.690196}%
\pgfsetstrokecolor{currentstroke}%
\pgfsetstrokeopacity{0.200000}%
\pgfsetdash{{2.960000pt}{1.280000pt}}{0.000000pt}%
\pgfpathmoveto{\pgfqpoint{0.764581in}{3.501190in}}%
\pgfpathlineto{\pgfqpoint{6.964581in}{3.501190in}}%
\pgfusepath{stroke}%
\end{pgfscope}%
\begin{pgfscope}%
\pgfsetbuttcap%
\pgfsetroundjoin%
\definecolor{currentfill}{rgb}{0.000000,0.000000,0.000000}%
\pgfsetfillcolor{currentfill}%
\pgfsetlinewidth{0.602250pt}%
\definecolor{currentstroke}{rgb}{0.000000,0.000000,0.000000}%
\pgfsetstrokecolor{currentstroke}%
\pgfsetdash{}{0pt}%
\pgfsys@defobject{currentmarker}{\pgfqpoint{-0.027778in}{0.000000in}}{\pgfqpoint{-0.000000in}{0.000000in}}{%
\pgfpathmoveto{\pgfqpoint{-0.000000in}{0.000000in}}%
\pgfpathlineto{\pgfqpoint{-0.027778in}{0.000000in}}%
\pgfusepath{stroke,fill}%
}%
\begin{pgfscope}%
\pgfsys@transformshift{0.764581in}{3.501190in}%
\pgfsys@useobject{currentmarker}{}%
\end{pgfscope}%
\end{pgfscope}%
\begin{pgfscope}%
\pgfpathrectangle{\pgfqpoint{0.764581in}{0.643904in}}{\pgfqpoint{6.200000in}{4.620000in}}%
\pgfusepath{clip}%
\pgfsetbuttcap%
\pgfsetroundjoin%
\pgfsetlinewidth{0.803000pt}%
\definecolor{currentstroke}{rgb}{0.690196,0.690196,0.690196}%
\pgfsetstrokecolor{currentstroke}%
\pgfsetstrokeopacity{0.200000}%
\pgfsetdash{{2.960000pt}{1.280000pt}}{0.000000pt}%
\pgfpathmoveto{\pgfqpoint{0.764581in}{3.582386in}}%
\pgfpathlineto{\pgfqpoint{6.964581in}{3.582386in}}%
\pgfusepath{stroke}%
\end{pgfscope}%
\begin{pgfscope}%
\pgfsetbuttcap%
\pgfsetroundjoin%
\definecolor{currentfill}{rgb}{0.000000,0.000000,0.000000}%
\pgfsetfillcolor{currentfill}%
\pgfsetlinewidth{0.602250pt}%
\definecolor{currentstroke}{rgb}{0.000000,0.000000,0.000000}%
\pgfsetstrokecolor{currentstroke}%
\pgfsetdash{}{0pt}%
\pgfsys@defobject{currentmarker}{\pgfqpoint{-0.027778in}{0.000000in}}{\pgfqpoint{-0.000000in}{0.000000in}}{%
\pgfpathmoveto{\pgfqpoint{-0.000000in}{0.000000in}}%
\pgfpathlineto{\pgfqpoint{-0.027778in}{0.000000in}}%
\pgfusepath{stroke,fill}%
}%
\begin{pgfscope}%
\pgfsys@transformshift{0.764581in}{3.582386in}%
\pgfsys@useobject{currentmarker}{}%
\end{pgfscope}%
\end{pgfscope}%
\begin{pgfscope}%
\pgfpathrectangle{\pgfqpoint{0.764581in}{0.643904in}}{\pgfqpoint{6.200000in}{4.620000in}}%
\pgfusepath{clip}%
\pgfsetbuttcap%
\pgfsetroundjoin%
\pgfsetlinewidth{0.803000pt}%
\definecolor{currentstroke}{rgb}{0.690196,0.690196,0.690196}%
\pgfsetstrokecolor{currentstroke}%
\pgfsetstrokeopacity{0.200000}%
\pgfsetdash{{2.960000pt}{1.280000pt}}{0.000000pt}%
\pgfpathmoveto{\pgfqpoint{0.764581in}{4.132854in}}%
\pgfpathlineto{\pgfqpoint{6.964581in}{4.132854in}}%
\pgfusepath{stroke}%
\end{pgfscope}%
\begin{pgfscope}%
\pgfsetbuttcap%
\pgfsetroundjoin%
\definecolor{currentfill}{rgb}{0.000000,0.000000,0.000000}%
\pgfsetfillcolor{currentfill}%
\pgfsetlinewidth{0.602250pt}%
\definecolor{currentstroke}{rgb}{0.000000,0.000000,0.000000}%
\pgfsetstrokecolor{currentstroke}%
\pgfsetdash{}{0pt}%
\pgfsys@defobject{currentmarker}{\pgfqpoint{-0.027778in}{0.000000in}}{\pgfqpoint{-0.000000in}{0.000000in}}{%
\pgfpathmoveto{\pgfqpoint{-0.000000in}{0.000000in}}%
\pgfpathlineto{\pgfqpoint{-0.027778in}{0.000000in}}%
\pgfusepath{stroke,fill}%
}%
\begin{pgfscope}%
\pgfsys@transformshift{0.764581in}{4.132854in}%
\pgfsys@useobject{currentmarker}{}%
\end{pgfscope}%
\end{pgfscope}%
\begin{pgfscope}%
\pgfpathrectangle{\pgfqpoint{0.764581in}{0.643904in}}{\pgfqpoint{6.200000in}{4.620000in}}%
\pgfusepath{clip}%
\pgfsetbuttcap%
\pgfsetroundjoin%
\pgfsetlinewidth{0.803000pt}%
\definecolor{currentstroke}{rgb}{0.690196,0.690196,0.690196}%
\pgfsetstrokecolor{currentstroke}%
\pgfsetstrokeopacity{0.200000}%
\pgfsetdash{{2.960000pt}{1.280000pt}}{0.000000pt}%
\pgfpathmoveto{\pgfqpoint{0.764581in}{4.412369in}}%
\pgfpathlineto{\pgfqpoint{6.964581in}{4.412369in}}%
\pgfusepath{stroke}%
\end{pgfscope}%
\begin{pgfscope}%
\pgfsetbuttcap%
\pgfsetroundjoin%
\definecolor{currentfill}{rgb}{0.000000,0.000000,0.000000}%
\pgfsetfillcolor{currentfill}%
\pgfsetlinewidth{0.602250pt}%
\definecolor{currentstroke}{rgb}{0.000000,0.000000,0.000000}%
\pgfsetstrokecolor{currentstroke}%
\pgfsetdash{}{0pt}%
\pgfsys@defobject{currentmarker}{\pgfqpoint{-0.027778in}{0.000000in}}{\pgfqpoint{-0.000000in}{0.000000in}}{%
\pgfpathmoveto{\pgfqpoint{-0.000000in}{0.000000in}}%
\pgfpathlineto{\pgfqpoint{-0.027778in}{0.000000in}}%
\pgfusepath{stroke,fill}%
}%
\begin{pgfscope}%
\pgfsys@transformshift{0.764581in}{4.412369in}%
\pgfsys@useobject{currentmarker}{}%
\end{pgfscope}%
\end{pgfscope}%
\begin{pgfscope}%
\pgfpathrectangle{\pgfqpoint{0.764581in}{0.643904in}}{\pgfqpoint{6.200000in}{4.620000in}}%
\pgfusepath{clip}%
\pgfsetbuttcap%
\pgfsetroundjoin%
\pgfsetlinewidth{0.803000pt}%
\definecolor{currentstroke}{rgb}{0.690196,0.690196,0.690196}%
\pgfsetstrokecolor{currentstroke}%
\pgfsetstrokeopacity{0.200000}%
\pgfsetdash{{2.960000pt}{1.280000pt}}{0.000000pt}%
\pgfpathmoveto{\pgfqpoint{0.764581in}{4.610689in}}%
\pgfpathlineto{\pgfqpoint{6.964581in}{4.610689in}}%
\pgfusepath{stroke}%
\end{pgfscope}%
\begin{pgfscope}%
\pgfsetbuttcap%
\pgfsetroundjoin%
\definecolor{currentfill}{rgb}{0.000000,0.000000,0.000000}%
\pgfsetfillcolor{currentfill}%
\pgfsetlinewidth{0.602250pt}%
\definecolor{currentstroke}{rgb}{0.000000,0.000000,0.000000}%
\pgfsetstrokecolor{currentstroke}%
\pgfsetdash{}{0pt}%
\pgfsys@defobject{currentmarker}{\pgfqpoint{-0.027778in}{0.000000in}}{\pgfqpoint{-0.000000in}{0.000000in}}{%
\pgfpathmoveto{\pgfqpoint{-0.000000in}{0.000000in}}%
\pgfpathlineto{\pgfqpoint{-0.027778in}{0.000000in}}%
\pgfusepath{stroke,fill}%
}%
\begin{pgfscope}%
\pgfsys@transformshift{0.764581in}{4.610689in}%
\pgfsys@useobject{currentmarker}{}%
\end{pgfscope}%
\end{pgfscope}%
\begin{pgfscope}%
\pgfpathrectangle{\pgfqpoint{0.764581in}{0.643904in}}{\pgfqpoint{6.200000in}{4.620000in}}%
\pgfusepath{clip}%
\pgfsetbuttcap%
\pgfsetroundjoin%
\pgfsetlinewidth{0.803000pt}%
\definecolor{currentstroke}{rgb}{0.690196,0.690196,0.690196}%
\pgfsetstrokecolor{currentstroke}%
\pgfsetstrokeopacity{0.200000}%
\pgfsetdash{{2.960000pt}{1.280000pt}}{0.000000pt}%
\pgfpathmoveto{\pgfqpoint{0.764581in}{4.764517in}}%
\pgfpathlineto{\pgfqpoint{6.964581in}{4.764517in}}%
\pgfusepath{stroke}%
\end{pgfscope}%
\begin{pgfscope}%
\pgfsetbuttcap%
\pgfsetroundjoin%
\definecolor{currentfill}{rgb}{0.000000,0.000000,0.000000}%
\pgfsetfillcolor{currentfill}%
\pgfsetlinewidth{0.602250pt}%
\definecolor{currentstroke}{rgb}{0.000000,0.000000,0.000000}%
\pgfsetstrokecolor{currentstroke}%
\pgfsetdash{}{0pt}%
\pgfsys@defobject{currentmarker}{\pgfqpoint{-0.027778in}{0.000000in}}{\pgfqpoint{-0.000000in}{0.000000in}}{%
\pgfpathmoveto{\pgfqpoint{-0.000000in}{0.000000in}}%
\pgfpathlineto{\pgfqpoint{-0.027778in}{0.000000in}}%
\pgfusepath{stroke,fill}%
}%
\begin{pgfscope}%
\pgfsys@transformshift{0.764581in}{4.764517in}%
\pgfsys@useobject{currentmarker}{}%
\end{pgfscope}%
\end{pgfscope}%
\begin{pgfscope}%
\pgfpathrectangle{\pgfqpoint{0.764581in}{0.643904in}}{\pgfqpoint{6.200000in}{4.620000in}}%
\pgfusepath{clip}%
\pgfsetbuttcap%
\pgfsetroundjoin%
\pgfsetlinewidth{0.803000pt}%
\definecolor{currentstroke}{rgb}{0.690196,0.690196,0.690196}%
\pgfsetstrokecolor{currentstroke}%
\pgfsetstrokeopacity{0.200000}%
\pgfsetdash{{2.960000pt}{1.280000pt}}{0.000000pt}%
\pgfpathmoveto{\pgfqpoint{0.764581in}{4.890204in}}%
\pgfpathlineto{\pgfqpoint{6.964581in}{4.890204in}}%
\pgfusepath{stroke}%
\end{pgfscope}%
\begin{pgfscope}%
\pgfsetbuttcap%
\pgfsetroundjoin%
\definecolor{currentfill}{rgb}{0.000000,0.000000,0.000000}%
\pgfsetfillcolor{currentfill}%
\pgfsetlinewidth{0.602250pt}%
\definecolor{currentstroke}{rgb}{0.000000,0.000000,0.000000}%
\pgfsetstrokecolor{currentstroke}%
\pgfsetdash{}{0pt}%
\pgfsys@defobject{currentmarker}{\pgfqpoint{-0.027778in}{0.000000in}}{\pgfqpoint{-0.000000in}{0.000000in}}{%
\pgfpathmoveto{\pgfqpoint{-0.000000in}{0.000000in}}%
\pgfpathlineto{\pgfqpoint{-0.027778in}{0.000000in}}%
\pgfusepath{stroke,fill}%
}%
\begin{pgfscope}%
\pgfsys@transformshift{0.764581in}{4.890204in}%
\pgfsys@useobject{currentmarker}{}%
\end{pgfscope}%
\end{pgfscope}%
\begin{pgfscope}%
\pgfpathrectangle{\pgfqpoint{0.764581in}{0.643904in}}{\pgfqpoint{6.200000in}{4.620000in}}%
\pgfusepath{clip}%
\pgfsetbuttcap%
\pgfsetroundjoin%
\pgfsetlinewidth{0.803000pt}%
\definecolor{currentstroke}{rgb}{0.690196,0.690196,0.690196}%
\pgfsetstrokecolor{currentstroke}%
\pgfsetstrokeopacity{0.200000}%
\pgfsetdash{{2.960000pt}{1.280000pt}}{0.000000pt}%
\pgfpathmoveto{\pgfqpoint{0.764581in}{4.996471in}}%
\pgfpathlineto{\pgfqpoint{6.964581in}{4.996471in}}%
\pgfusepath{stroke}%
\end{pgfscope}%
\begin{pgfscope}%
\pgfsetbuttcap%
\pgfsetroundjoin%
\definecolor{currentfill}{rgb}{0.000000,0.000000,0.000000}%
\pgfsetfillcolor{currentfill}%
\pgfsetlinewidth{0.602250pt}%
\definecolor{currentstroke}{rgb}{0.000000,0.000000,0.000000}%
\pgfsetstrokecolor{currentstroke}%
\pgfsetdash{}{0pt}%
\pgfsys@defobject{currentmarker}{\pgfqpoint{-0.027778in}{0.000000in}}{\pgfqpoint{-0.000000in}{0.000000in}}{%
\pgfpathmoveto{\pgfqpoint{-0.000000in}{0.000000in}}%
\pgfpathlineto{\pgfqpoint{-0.027778in}{0.000000in}}%
\pgfusepath{stroke,fill}%
}%
\begin{pgfscope}%
\pgfsys@transformshift{0.764581in}{4.996471in}%
\pgfsys@useobject{currentmarker}{}%
\end{pgfscope}%
\end{pgfscope}%
\begin{pgfscope}%
\pgfpathrectangle{\pgfqpoint{0.764581in}{0.643904in}}{\pgfqpoint{6.200000in}{4.620000in}}%
\pgfusepath{clip}%
\pgfsetbuttcap%
\pgfsetroundjoin%
\pgfsetlinewidth{0.803000pt}%
\definecolor{currentstroke}{rgb}{0.690196,0.690196,0.690196}%
\pgfsetstrokecolor{currentstroke}%
\pgfsetstrokeopacity{0.200000}%
\pgfsetdash{{2.960000pt}{1.280000pt}}{0.000000pt}%
\pgfpathmoveto{\pgfqpoint{0.764581in}{5.088524in}}%
\pgfpathlineto{\pgfqpoint{6.964581in}{5.088524in}}%
\pgfusepath{stroke}%
\end{pgfscope}%
\begin{pgfscope}%
\pgfsetbuttcap%
\pgfsetroundjoin%
\definecolor{currentfill}{rgb}{0.000000,0.000000,0.000000}%
\pgfsetfillcolor{currentfill}%
\pgfsetlinewidth{0.602250pt}%
\definecolor{currentstroke}{rgb}{0.000000,0.000000,0.000000}%
\pgfsetstrokecolor{currentstroke}%
\pgfsetdash{}{0pt}%
\pgfsys@defobject{currentmarker}{\pgfqpoint{-0.027778in}{0.000000in}}{\pgfqpoint{-0.000000in}{0.000000in}}{%
\pgfpathmoveto{\pgfqpoint{-0.000000in}{0.000000in}}%
\pgfpathlineto{\pgfqpoint{-0.027778in}{0.000000in}}%
\pgfusepath{stroke,fill}%
}%
\begin{pgfscope}%
\pgfsys@transformshift{0.764581in}{5.088524in}%
\pgfsys@useobject{currentmarker}{}%
\end{pgfscope}%
\end{pgfscope}%
\begin{pgfscope}%
\pgfpathrectangle{\pgfqpoint{0.764581in}{0.643904in}}{\pgfqpoint{6.200000in}{4.620000in}}%
\pgfusepath{clip}%
\pgfsetbuttcap%
\pgfsetroundjoin%
\pgfsetlinewidth{0.803000pt}%
\definecolor{currentstroke}{rgb}{0.690196,0.690196,0.690196}%
\pgfsetstrokecolor{currentstroke}%
\pgfsetstrokeopacity{0.200000}%
\pgfsetdash{{2.960000pt}{1.280000pt}}{0.000000pt}%
\pgfpathmoveto{\pgfqpoint{0.764581in}{5.169720in}}%
\pgfpathlineto{\pgfqpoint{6.964581in}{5.169720in}}%
\pgfusepath{stroke}%
\end{pgfscope}%
\begin{pgfscope}%
\pgfsetbuttcap%
\pgfsetroundjoin%
\definecolor{currentfill}{rgb}{0.000000,0.000000,0.000000}%
\pgfsetfillcolor{currentfill}%
\pgfsetlinewidth{0.602250pt}%
\definecolor{currentstroke}{rgb}{0.000000,0.000000,0.000000}%
\pgfsetstrokecolor{currentstroke}%
\pgfsetdash{}{0pt}%
\pgfsys@defobject{currentmarker}{\pgfqpoint{-0.027778in}{0.000000in}}{\pgfqpoint{-0.000000in}{0.000000in}}{%
\pgfpathmoveto{\pgfqpoint{-0.000000in}{0.000000in}}%
\pgfpathlineto{\pgfqpoint{-0.027778in}{0.000000in}}%
\pgfusepath{stroke,fill}%
}%
\begin{pgfscope}%
\pgfsys@transformshift{0.764581in}{5.169720in}%
\pgfsys@useobject{currentmarker}{}%
\end{pgfscope}%
\end{pgfscope}%
\begin{pgfscope}%
\definecolor{textcolor}{rgb}{0.000000,0.000000,0.000000}%
\pgfsetstrokecolor{textcolor}%
\pgfsetfillcolor{textcolor}%
\pgftext[x=0.339583in,y=2.953904in,,bottom,rotate=90.000000]{\color{textcolor}{\rmfamily\fontsize{18.000000}{21.600000}\selectfont\catcode`\^=\active\def^{\ifmmode\sp\else\^{}\fi}\catcode`\%=\active\def%{\%}Time [seconds]}}%
\end{pgfscope}%
\begin{pgfscope}%
\pgfsetrectcap%
\pgfsetmiterjoin%
\pgfsetlinewidth{0.803000pt}%
\definecolor{currentstroke}{rgb}{0.000000,0.000000,0.000000}%
\pgfsetstrokecolor{currentstroke}%
\pgfsetdash{}{0pt}%
\pgfpathmoveto{\pgfqpoint{0.764581in}{0.643904in}}%
\pgfpathlineto{\pgfqpoint{0.764581in}{5.263904in}}%
\pgfusepath{stroke}%
\end{pgfscope}%
\begin{pgfscope}%
\pgfsetrectcap%
\pgfsetmiterjoin%
\pgfsetlinewidth{0.803000pt}%
\definecolor{currentstroke}{rgb}{0.000000,0.000000,0.000000}%
\pgfsetstrokecolor{currentstroke}%
\pgfsetdash{}{0pt}%
\pgfpathmoveto{\pgfqpoint{6.964581in}{0.643904in}}%
\pgfpathlineto{\pgfqpoint{6.964581in}{5.263904in}}%
\pgfusepath{stroke}%
\end{pgfscope}%
\begin{pgfscope}%
\pgfsetrectcap%
\pgfsetmiterjoin%
\pgfsetlinewidth{0.803000pt}%
\definecolor{currentstroke}{rgb}{0.000000,0.000000,0.000000}%
\pgfsetstrokecolor{currentstroke}%
\pgfsetdash{}{0pt}%
\pgfpathmoveto{\pgfqpoint{0.764581in}{0.643904in}}%
\pgfpathlineto{\pgfqpoint{6.964581in}{0.643904in}}%
\pgfusepath{stroke}%
\end{pgfscope}%
\begin{pgfscope}%
\pgfsetrectcap%
\pgfsetmiterjoin%
\pgfsetlinewidth{0.803000pt}%
\definecolor{currentstroke}{rgb}{0.000000,0.000000,0.000000}%
\pgfsetstrokecolor{currentstroke}%
\pgfsetdash{}{0pt}%
\pgfpathmoveto{\pgfqpoint{0.764581in}{5.263904in}}%
\pgfpathlineto{\pgfqpoint{6.964581in}{5.263904in}}%
\pgfusepath{stroke}%
\end{pgfscope}%
\begin{pgfscope}%
\pgfpathrectangle{\pgfqpoint{0.764581in}{0.643904in}}{\pgfqpoint{6.200000in}{4.620000in}}%
\pgfusepath{clip}%
\pgfsetrectcap%
\pgfsetroundjoin%
\pgfsetlinewidth{1.505625pt}%
\definecolor{currentstroke}{rgb}{0.000000,0.000000,1.000000}%
\pgfsetstrokecolor{currentstroke}%
\pgfsetdash{}{0pt}%
\pgfpathmoveto{\pgfqpoint{0.764581in}{0.853904in}}%
\pgfpathlineto{\pgfqpoint{2.455884in}{1.625606in}}%
\pgfpathlineto{\pgfqpoint{3.184288in}{2.056435in}}%
\pgfpathlineto{\pgfqpoint{4.147187in}{2.648049in}}%
\pgfpathlineto{\pgfqpoint{4.875592in}{3.110216in}}%
\pgfpathlineto{\pgfqpoint{5.603996in}{3.550323in}}%
\pgfpathlineto{\pgfqpoint{6.332401in}{4.052710in}}%
\pgfpathlineto{\pgfqpoint{6.758490in}{4.325160in}}%
\pgfpathlineto{\pgfqpoint{6.964581in}{4.448402in}}%
\pgfusepath{stroke}%
\end{pgfscope}%
\begin{pgfscope}%
\pgfpathrectangle{\pgfqpoint{0.764581in}{0.643904in}}{\pgfqpoint{6.200000in}{4.620000in}}%
\pgfusepath{clip}%
\pgfsetbuttcap%
\pgfsetroundjoin%
\pgfsetlinewidth{1.505625pt}%
\definecolor{currentstroke}{rgb}{1.000000,0.000000,0.000000}%
\pgfsetstrokecolor{currentstroke}%
\pgfsetdash{{5.550000pt}{2.400000pt}}{0.000000pt}%
\pgfpathmoveto{\pgfqpoint{0.764581in}{3.393336in}}%
\pgfpathlineto{\pgfqpoint{2.455884in}{3.469107in}}%
\pgfpathlineto{\pgfqpoint{3.184288in}{3.545769in}}%
\pgfpathlineto{\pgfqpoint{4.147187in}{3.728731in}}%
\pgfpathlineto{\pgfqpoint{4.875592in}{3.946418in}}%
\pgfpathlineto{\pgfqpoint{5.603996in}{4.254357in}}%
\pgfpathlineto{\pgfqpoint{6.332401in}{4.649383in}}%
\pgfpathlineto{\pgfqpoint{6.758490in}{4.916608in}}%
\pgfpathlineto{\pgfqpoint{6.964581in}{5.053904in}}%
\pgfusepath{stroke}%
\end{pgfscope}%
\begin{pgfscope}%
\pgfsetbuttcap%
\pgfsetmiterjoin%
\definecolor{currentfill}{rgb}{1.000000,1.000000,1.000000}%
\pgfsetfillcolor{currentfill}%
\pgfsetfillopacity{0.800000}%
\pgfsetlinewidth{1.003750pt}%
\definecolor{currentstroke}{rgb}{0.800000,0.800000,0.800000}%
\pgfsetstrokecolor{currentstroke}%
\pgfsetstrokeopacity{0.800000}%
\pgfsetdash{}{0pt}%
\pgfpathmoveto{\pgfqpoint{0.920136in}{4.437207in}}%
\pgfpathlineto{\pgfqpoint{3.236863in}{4.437207in}}%
\pgfpathquadraticcurveto{\pgfqpoint{3.281307in}{4.437207in}}{\pgfqpoint{3.281307in}{4.481651in}}%
\pgfpathlineto{\pgfqpoint{3.281307in}{5.108348in}}%
\pgfpathquadraticcurveto{\pgfqpoint{3.281307in}{5.152793in}}{\pgfqpoint{3.236863in}{5.152793in}}%
\pgfpathlineto{\pgfqpoint{0.920136in}{5.152793in}}%
\pgfpathquadraticcurveto{\pgfqpoint{0.875692in}{5.152793in}}{\pgfqpoint{0.875692in}{5.108348in}}%
\pgfpathlineto{\pgfqpoint{0.875692in}{4.481651in}}%
\pgfpathquadraticcurveto{\pgfqpoint{0.875692in}{4.437207in}}{\pgfqpoint{0.920136in}{4.437207in}}%
\pgfpathlineto{\pgfqpoint{0.920136in}{4.437207in}}%
\pgfpathclose%
\pgfusepath{stroke,fill}%
\end{pgfscope}%
\begin{pgfscope}%
\pgfsetrectcap%
\pgfsetroundjoin%
\pgfsetlinewidth{1.505625pt}%
\definecolor{currentstroke}{rgb}{0.000000,0.000000,1.000000}%
\pgfsetstrokecolor{currentstroke}%
\pgfsetdash{}{0pt}%
\pgfpathmoveto{\pgfqpoint{0.964581in}{4.975015in}}%
\pgfpathlineto{\pgfqpoint{1.186803in}{4.975015in}}%
\pgfpathlineto{\pgfqpoint{1.409025in}{4.975015in}}%
\pgfusepath{stroke}%
\end{pgfscope}%
\begin{pgfscope}%
\definecolor{textcolor}{rgb}{0.000000,0.000000,0.000000}%
\pgfsetstrokecolor{textcolor}%
\pgfsetfillcolor{textcolor}%
\pgftext[x=1.586803in,y=4.897237in,left,base]{\color{textcolor}{\rmfamily\fontsize{16.000000}{19.200000}\selectfont\catcode`\^=\active\def^{\ifmmode\sp\else\^{}\fi}\catcode`\%=\active\def%{\%}logical dispatch}}%
\end{pgfscope}%
\begin{pgfscope}%
\pgfsetbuttcap%
\pgfsetroundjoin%
\pgfsetlinewidth{1.505625pt}%
\definecolor{currentstroke}{rgb}{1.000000,0.000000,0.000000}%
\pgfsetstrokecolor{currentstroke}%
\pgfsetdash{{5.550000pt}{2.400000pt}}{0.000000pt}%
\pgfpathmoveto{\pgfqpoint{0.964581in}{4.650555in}}%
\pgfpathlineto{\pgfqpoint{1.186803in}{4.650555in}}%
\pgfpathlineto{\pgfqpoint{1.409025in}{4.650555in}}%
\pgfusepath{stroke}%
\end{pgfscope}%
\begin{pgfscope}%
\definecolor{textcolor}{rgb}{0.000000,0.000000,0.000000}%
\pgfsetstrokecolor{textcolor}%
\pgfsetfillcolor{textcolor}%
\pgftext[x=1.586803in,y=4.572777in,left,base]{\color{textcolor}{\rmfamily\fontsize{16.000000}{19.200000}\selectfont\catcode`\^=\active\def^{\ifmmode\sp\else\^{}\fi}\catcode`\%=\active\def%{\%}optimal dispatch}}%
\end{pgfscope}%
\end{pgfpicture}%
\makeatother%
\endgroup%
}
    \caption{Time scaling of a capacity expansion problem using either an optimal or logical dispatch algorithm.}
    \label{fig:alg-scaling}
\end{figure}

\noindent Initially, the logical dispatch algorithm outperforms the optimal
dispatch algorithm by nearly two orders of magnitude. This is because the linear
program has some overhead when writing and copying equations that the rule-based
calculation does not. The logical algorithm initially grows more quickly until
the models reach 100 modeled days after which the two algorithms scale
similarly and the logical dispatch algorithm remains approximately 2.5 times
faster than its optimal counterpart.

\subsection{Exercise 3: Parallelization}

\Acp{ga} are considered ``embarrassingly parallelizable'' since the performance
of each individual in a population is independent from the others. However, there
a some difficulties with solving multiple parallel instances of \ac{lp} solvers
since these solvers frequently have some parallel optimizations built-in. For
now, this restricts capacity expansion problems within \ac{osier} that use linear 
programming to serial calculations. This is not so for the logical dispatch algorithm 
since it does
not use an \ac{lp} solver. Therefore, this exercise looks exclusively at how the
logical dispatch algorithm scales with number of threads available. Once again,
the dispatch algorithm is driven by \ac{osier}'s \texttt{CapacityExpansion}
class whose parameters are described in Table \ref{tab:thread-scaling-params}. 

\begin{table}[htbp!]
    \centering
    \caption{Capacity expansion parameters for the parallelization exercise.}
    \label{tab:thread-scaling-params}
    \begin{tabular}{ll}
        \toprule
        Parameter & Value \\
        \midrule
        Algorithm & \acs{nsga2}\\
        Termination Criterion & Maximum Generations\\
        Generations & 10 \\
        Objectives & 2 (cost, emissions)\\
        Timesteps & 120 (5 days x 24 hours)\\
        \bottomrule
    \end{tabular}
\end{table}

\noindent In this exercise, the problem is scaled by the population size of each
generation. The study was performed on a 2024 MacBook Pro with an M4 Pro CPU, 
48 GB of RAM, and the macOS Sequoia 15.5 operating system. Figure
\ref{fig:thread-scaling} shows the results for this exercise.

\begin{figure}[htbp!]
    \centering
    \resizebox{0.75\columnwidth}{!}{%% Creator: Matplotlib, PGF backend
%%
%% To include the figure in your LaTeX document, write
%%   \input{<filename>.pgf}
%%
%% Make sure the required packages are loaded in your preamble
%%   \usepackage{pgf}
%%
%% Also ensure that all the required font packages are loaded; for instance,
%% the lmodern package is sometimes necessary when using math font.
%%   \usepackage{lmodern}
%%
%% Figures using additional raster images can only be included by \input if
%% they are in the same directory as the main LaTeX file. For loading figures
%% from other directories you can use the `import` package
%%   \usepackage{import}
%%
%% and then include the figures with
%%   \import{<path to file>}{<filename>.pgf}
%%
%% Matplotlib used the following preamble
%%   \def\mathdefault#1{#1}
%%   \everymath=\expandafter{\the\everymath\displaystyle}
%%   \IfFileExists{scrextend.sty}{
%%     \usepackage[fontsize=10.000000pt]{scrextend}
%%   }{
%%     \renewcommand{\normalsize}{\fontsize{10.000000}{12.000000}\selectfont}
%%     \normalsize
%%   }
%%   
%%   \makeatletter\@ifpackageloaded{underscore}{}{\usepackage[strings]{underscore}}\makeatother
%%
\begingroup%
\makeatletter%
\begin{pgfpicture}%
\pgfpathrectangle{\pgfpointorigin}{\pgfqpoint{7.135065in}{5.363904in}}%
\pgfusepath{use as bounding box, clip}%
\begin{pgfscope}%
\pgfsetbuttcap%
\pgfsetmiterjoin%
\definecolor{currentfill}{rgb}{1.000000,1.000000,1.000000}%
\pgfsetfillcolor{currentfill}%
\pgfsetlinewidth{0.000000pt}%
\definecolor{currentstroke}{rgb}{0.000000,0.000000,0.000000}%
\pgfsetstrokecolor{currentstroke}%
\pgfsetdash{}{0pt}%
\pgfpathmoveto{\pgfqpoint{0.000000in}{0.000000in}}%
\pgfpathlineto{\pgfqpoint{7.135065in}{0.000000in}}%
\pgfpathlineto{\pgfqpoint{7.135065in}{5.363904in}}%
\pgfpathlineto{\pgfqpoint{0.000000in}{5.363904in}}%
\pgfpathlineto{\pgfqpoint{0.000000in}{0.000000in}}%
\pgfpathclose%
\pgfusepath{fill}%
\end{pgfscope}%
\begin{pgfscope}%
\pgfsetbuttcap%
\pgfsetmiterjoin%
\definecolor{currentfill}{rgb}{1.000000,1.000000,1.000000}%
\pgfsetfillcolor{currentfill}%
\pgfsetlinewidth{0.000000pt}%
\definecolor{currentstroke}{rgb}{0.000000,0.000000,0.000000}%
\pgfsetstrokecolor{currentstroke}%
\pgfsetstrokeopacity{0.000000}%
\pgfsetdash{}{0pt}%
\pgfpathmoveto{\pgfqpoint{0.688192in}{0.643904in}}%
\pgfpathlineto{\pgfqpoint{6.888192in}{0.643904in}}%
\pgfpathlineto{\pgfqpoint{6.888192in}{5.263904in}}%
\pgfpathlineto{\pgfqpoint{0.688192in}{5.263904in}}%
\pgfpathlineto{\pgfqpoint{0.688192in}{0.643904in}}%
\pgfpathclose%
\pgfusepath{fill}%
\end{pgfscope}%
\begin{pgfscope}%
\pgfpathrectangle{\pgfqpoint{0.688192in}{0.643904in}}{\pgfqpoint{6.200000in}{4.620000in}}%
\pgfusepath{clip}%
\pgfsetrectcap%
\pgfsetroundjoin%
\pgfsetlinewidth{0.803000pt}%
\definecolor{currentstroke}{rgb}{0.690196,0.690196,0.690196}%
\pgfsetstrokecolor{currentstroke}%
\pgfsetdash{}{0pt}%
\pgfpathmoveto{\pgfqpoint{1.219620in}{0.643904in}}%
\pgfpathlineto{\pgfqpoint{1.219620in}{5.263904in}}%
\pgfusepath{stroke}%
\end{pgfscope}%
\begin{pgfscope}%
\pgfsetbuttcap%
\pgfsetroundjoin%
\definecolor{currentfill}{rgb}{0.000000,0.000000,0.000000}%
\pgfsetfillcolor{currentfill}%
\pgfsetlinewidth{0.803000pt}%
\definecolor{currentstroke}{rgb}{0.000000,0.000000,0.000000}%
\pgfsetstrokecolor{currentstroke}%
\pgfsetdash{}{0pt}%
\pgfsys@defobject{currentmarker}{\pgfqpoint{0.000000in}{-0.048611in}}{\pgfqpoint{0.000000in}{0.000000in}}{%
\pgfpathmoveto{\pgfqpoint{0.000000in}{0.000000in}}%
\pgfpathlineto{\pgfqpoint{0.000000in}{-0.048611in}}%
\pgfusepath{stroke,fill}%
}%
\begin{pgfscope}%
\pgfsys@transformshift{1.219620in}{0.643904in}%
\pgfsys@useobject{currentmarker}{}%
\end{pgfscope}%
\end{pgfscope}%
\begin{pgfscope}%
\definecolor{textcolor}{rgb}{0.000000,0.000000,0.000000}%
\pgfsetstrokecolor{textcolor}%
\pgfsetfillcolor{textcolor}%
\pgftext[x=1.219620in,y=0.546682in,,top]{\color{textcolor}{\rmfamily\fontsize{14.000000}{16.800000}\selectfont\catcode`\^=\active\def^{\ifmmode\sp\else\^{}\fi}\catcode`\%=\active\def%{\%}$\mathdefault{40}$}}%
\end{pgfscope}%
\begin{pgfscope}%
\pgfpathrectangle{\pgfqpoint{0.688192in}{0.643904in}}{\pgfqpoint{6.200000in}{4.620000in}}%
\pgfusepath{clip}%
\pgfsetrectcap%
\pgfsetroundjoin%
\pgfsetlinewidth{0.803000pt}%
\definecolor{currentstroke}{rgb}{0.690196,0.690196,0.690196}%
\pgfsetstrokecolor{currentstroke}%
\pgfsetdash{}{0pt}%
\pgfpathmoveto{\pgfqpoint{1.928192in}{0.643904in}}%
\pgfpathlineto{\pgfqpoint{1.928192in}{5.263904in}}%
\pgfusepath{stroke}%
\end{pgfscope}%
\begin{pgfscope}%
\pgfsetbuttcap%
\pgfsetroundjoin%
\definecolor{currentfill}{rgb}{0.000000,0.000000,0.000000}%
\pgfsetfillcolor{currentfill}%
\pgfsetlinewidth{0.803000pt}%
\definecolor{currentstroke}{rgb}{0.000000,0.000000,0.000000}%
\pgfsetstrokecolor{currentstroke}%
\pgfsetdash{}{0pt}%
\pgfsys@defobject{currentmarker}{\pgfqpoint{0.000000in}{-0.048611in}}{\pgfqpoint{0.000000in}{0.000000in}}{%
\pgfpathmoveto{\pgfqpoint{0.000000in}{0.000000in}}%
\pgfpathlineto{\pgfqpoint{0.000000in}{-0.048611in}}%
\pgfusepath{stroke,fill}%
}%
\begin{pgfscope}%
\pgfsys@transformshift{1.928192in}{0.643904in}%
\pgfsys@useobject{currentmarker}{}%
\end{pgfscope}%
\end{pgfscope}%
\begin{pgfscope}%
\definecolor{textcolor}{rgb}{0.000000,0.000000,0.000000}%
\pgfsetstrokecolor{textcolor}%
\pgfsetfillcolor{textcolor}%
\pgftext[x=1.928192in,y=0.546682in,,top]{\color{textcolor}{\rmfamily\fontsize{14.000000}{16.800000}\selectfont\catcode`\^=\active\def^{\ifmmode\sp\else\^{}\fi}\catcode`\%=\active\def%{\%}$\mathdefault{60}$}}%
\end{pgfscope}%
\begin{pgfscope}%
\pgfpathrectangle{\pgfqpoint{0.688192in}{0.643904in}}{\pgfqpoint{6.200000in}{4.620000in}}%
\pgfusepath{clip}%
\pgfsetrectcap%
\pgfsetroundjoin%
\pgfsetlinewidth{0.803000pt}%
\definecolor{currentstroke}{rgb}{0.690196,0.690196,0.690196}%
\pgfsetstrokecolor{currentstroke}%
\pgfsetdash{}{0pt}%
\pgfpathmoveto{\pgfqpoint{2.636763in}{0.643904in}}%
\pgfpathlineto{\pgfqpoint{2.636763in}{5.263904in}}%
\pgfusepath{stroke}%
\end{pgfscope}%
\begin{pgfscope}%
\pgfsetbuttcap%
\pgfsetroundjoin%
\definecolor{currentfill}{rgb}{0.000000,0.000000,0.000000}%
\pgfsetfillcolor{currentfill}%
\pgfsetlinewidth{0.803000pt}%
\definecolor{currentstroke}{rgb}{0.000000,0.000000,0.000000}%
\pgfsetstrokecolor{currentstroke}%
\pgfsetdash{}{0pt}%
\pgfsys@defobject{currentmarker}{\pgfqpoint{0.000000in}{-0.048611in}}{\pgfqpoint{0.000000in}{0.000000in}}{%
\pgfpathmoveto{\pgfqpoint{0.000000in}{0.000000in}}%
\pgfpathlineto{\pgfqpoint{0.000000in}{-0.048611in}}%
\pgfusepath{stroke,fill}%
}%
\begin{pgfscope}%
\pgfsys@transformshift{2.636763in}{0.643904in}%
\pgfsys@useobject{currentmarker}{}%
\end{pgfscope}%
\end{pgfscope}%
\begin{pgfscope}%
\definecolor{textcolor}{rgb}{0.000000,0.000000,0.000000}%
\pgfsetstrokecolor{textcolor}%
\pgfsetfillcolor{textcolor}%
\pgftext[x=2.636763in,y=0.546682in,,top]{\color{textcolor}{\rmfamily\fontsize{14.000000}{16.800000}\selectfont\catcode`\^=\active\def^{\ifmmode\sp\else\^{}\fi}\catcode`\%=\active\def%{\%}$\mathdefault{80}$}}%
\end{pgfscope}%
\begin{pgfscope}%
\pgfpathrectangle{\pgfqpoint{0.688192in}{0.643904in}}{\pgfqpoint{6.200000in}{4.620000in}}%
\pgfusepath{clip}%
\pgfsetrectcap%
\pgfsetroundjoin%
\pgfsetlinewidth{0.803000pt}%
\definecolor{currentstroke}{rgb}{0.690196,0.690196,0.690196}%
\pgfsetstrokecolor{currentstroke}%
\pgfsetdash{}{0pt}%
\pgfpathmoveto{\pgfqpoint{3.345334in}{0.643904in}}%
\pgfpathlineto{\pgfqpoint{3.345334in}{5.263904in}}%
\pgfusepath{stroke}%
\end{pgfscope}%
\begin{pgfscope}%
\pgfsetbuttcap%
\pgfsetroundjoin%
\definecolor{currentfill}{rgb}{0.000000,0.000000,0.000000}%
\pgfsetfillcolor{currentfill}%
\pgfsetlinewidth{0.803000pt}%
\definecolor{currentstroke}{rgb}{0.000000,0.000000,0.000000}%
\pgfsetstrokecolor{currentstroke}%
\pgfsetdash{}{0pt}%
\pgfsys@defobject{currentmarker}{\pgfqpoint{0.000000in}{-0.048611in}}{\pgfqpoint{0.000000in}{0.000000in}}{%
\pgfpathmoveto{\pgfqpoint{0.000000in}{0.000000in}}%
\pgfpathlineto{\pgfqpoint{0.000000in}{-0.048611in}}%
\pgfusepath{stroke,fill}%
}%
\begin{pgfscope}%
\pgfsys@transformshift{3.345334in}{0.643904in}%
\pgfsys@useobject{currentmarker}{}%
\end{pgfscope}%
\end{pgfscope}%
\begin{pgfscope}%
\definecolor{textcolor}{rgb}{0.000000,0.000000,0.000000}%
\pgfsetstrokecolor{textcolor}%
\pgfsetfillcolor{textcolor}%
\pgftext[x=3.345334in,y=0.546682in,,top]{\color{textcolor}{\rmfamily\fontsize{14.000000}{16.800000}\selectfont\catcode`\^=\active\def^{\ifmmode\sp\else\^{}\fi}\catcode`\%=\active\def%{\%}$\mathdefault{100}$}}%
\end{pgfscope}%
\begin{pgfscope}%
\pgfpathrectangle{\pgfqpoint{0.688192in}{0.643904in}}{\pgfqpoint{6.200000in}{4.620000in}}%
\pgfusepath{clip}%
\pgfsetrectcap%
\pgfsetroundjoin%
\pgfsetlinewidth{0.803000pt}%
\definecolor{currentstroke}{rgb}{0.690196,0.690196,0.690196}%
\pgfsetstrokecolor{currentstroke}%
\pgfsetdash{}{0pt}%
\pgfpathmoveto{\pgfqpoint{4.053906in}{0.643904in}}%
\pgfpathlineto{\pgfqpoint{4.053906in}{5.263904in}}%
\pgfusepath{stroke}%
\end{pgfscope}%
\begin{pgfscope}%
\pgfsetbuttcap%
\pgfsetroundjoin%
\definecolor{currentfill}{rgb}{0.000000,0.000000,0.000000}%
\pgfsetfillcolor{currentfill}%
\pgfsetlinewidth{0.803000pt}%
\definecolor{currentstroke}{rgb}{0.000000,0.000000,0.000000}%
\pgfsetstrokecolor{currentstroke}%
\pgfsetdash{}{0pt}%
\pgfsys@defobject{currentmarker}{\pgfqpoint{0.000000in}{-0.048611in}}{\pgfqpoint{0.000000in}{0.000000in}}{%
\pgfpathmoveto{\pgfqpoint{0.000000in}{0.000000in}}%
\pgfpathlineto{\pgfqpoint{0.000000in}{-0.048611in}}%
\pgfusepath{stroke,fill}%
}%
\begin{pgfscope}%
\pgfsys@transformshift{4.053906in}{0.643904in}%
\pgfsys@useobject{currentmarker}{}%
\end{pgfscope}%
\end{pgfscope}%
\begin{pgfscope}%
\definecolor{textcolor}{rgb}{0.000000,0.000000,0.000000}%
\pgfsetstrokecolor{textcolor}%
\pgfsetfillcolor{textcolor}%
\pgftext[x=4.053906in,y=0.546682in,,top]{\color{textcolor}{\rmfamily\fontsize{14.000000}{16.800000}\selectfont\catcode`\^=\active\def^{\ifmmode\sp\else\^{}\fi}\catcode`\%=\active\def%{\%}$\mathdefault{120}$}}%
\end{pgfscope}%
\begin{pgfscope}%
\pgfpathrectangle{\pgfqpoint{0.688192in}{0.643904in}}{\pgfqpoint{6.200000in}{4.620000in}}%
\pgfusepath{clip}%
\pgfsetrectcap%
\pgfsetroundjoin%
\pgfsetlinewidth{0.803000pt}%
\definecolor{currentstroke}{rgb}{0.690196,0.690196,0.690196}%
\pgfsetstrokecolor{currentstroke}%
\pgfsetdash{}{0pt}%
\pgfpathmoveto{\pgfqpoint{4.762477in}{0.643904in}}%
\pgfpathlineto{\pgfqpoint{4.762477in}{5.263904in}}%
\pgfusepath{stroke}%
\end{pgfscope}%
\begin{pgfscope}%
\pgfsetbuttcap%
\pgfsetroundjoin%
\definecolor{currentfill}{rgb}{0.000000,0.000000,0.000000}%
\pgfsetfillcolor{currentfill}%
\pgfsetlinewidth{0.803000pt}%
\definecolor{currentstroke}{rgb}{0.000000,0.000000,0.000000}%
\pgfsetstrokecolor{currentstroke}%
\pgfsetdash{}{0pt}%
\pgfsys@defobject{currentmarker}{\pgfqpoint{0.000000in}{-0.048611in}}{\pgfqpoint{0.000000in}{0.000000in}}{%
\pgfpathmoveto{\pgfqpoint{0.000000in}{0.000000in}}%
\pgfpathlineto{\pgfqpoint{0.000000in}{-0.048611in}}%
\pgfusepath{stroke,fill}%
}%
\begin{pgfscope}%
\pgfsys@transformshift{4.762477in}{0.643904in}%
\pgfsys@useobject{currentmarker}{}%
\end{pgfscope}%
\end{pgfscope}%
\begin{pgfscope}%
\definecolor{textcolor}{rgb}{0.000000,0.000000,0.000000}%
\pgfsetstrokecolor{textcolor}%
\pgfsetfillcolor{textcolor}%
\pgftext[x=4.762477in,y=0.546682in,,top]{\color{textcolor}{\rmfamily\fontsize{14.000000}{16.800000}\selectfont\catcode`\^=\active\def^{\ifmmode\sp\else\^{}\fi}\catcode`\%=\active\def%{\%}$\mathdefault{140}$}}%
\end{pgfscope}%
\begin{pgfscope}%
\pgfpathrectangle{\pgfqpoint{0.688192in}{0.643904in}}{\pgfqpoint{6.200000in}{4.620000in}}%
\pgfusepath{clip}%
\pgfsetrectcap%
\pgfsetroundjoin%
\pgfsetlinewidth{0.803000pt}%
\definecolor{currentstroke}{rgb}{0.690196,0.690196,0.690196}%
\pgfsetstrokecolor{currentstroke}%
\pgfsetdash{}{0pt}%
\pgfpathmoveto{\pgfqpoint{5.471049in}{0.643904in}}%
\pgfpathlineto{\pgfqpoint{5.471049in}{5.263904in}}%
\pgfusepath{stroke}%
\end{pgfscope}%
\begin{pgfscope}%
\pgfsetbuttcap%
\pgfsetroundjoin%
\definecolor{currentfill}{rgb}{0.000000,0.000000,0.000000}%
\pgfsetfillcolor{currentfill}%
\pgfsetlinewidth{0.803000pt}%
\definecolor{currentstroke}{rgb}{0.000000,0.000000,0.000000}%
\pgfsetstrokecolor{currentstroke}%
\pgfsetdash{}{0pt}%
\pgfsys@defobject{currentmarker}{\pgfqpoint{0.000000in}{-0.048611in}}{\pgfqpoint{0.000000in}{0.000000in}}{%
\pgfpathmoveto{\pgfqpoint{0.000000in}{0.000000in}}%
\pgfpathlineto{\pgfqpoint{0.000000in}{-0.048611in}}%
\pgfusepath{stroke,fill}%
}%
\begin{pgfscope}%
\pgfsys@transformshift{5.471049in}{0.643904in}%
\pgfsys@useobject{currentmarker}{}%
\end{pgfscope}%
\end{pgfscope}%
\begin{pgfscope}%
\definecolor{textcolor}{rgb}{0.000000,0.000000,0.000000}%
\pgfsetstrokecolor{textcolor}%
\pgfsetfillcolor{textcolor}%
\pgftext[x=5.471049in,y=0.546682in,,top]{\color{textcolor}{\rmfamily\fontsize{14.000000}{16.800000}\selectfont\catcode`\^=\active\def^{\ifmmode\sp\else\^{}\fi}\catcode`\%=\active\def%{\%}$\mathdefault{160}$}}%
\end{pgfscope}%
\begin{pgfscope}%
\pgfpathrectangle{\pgfqpoint{0.688192in}{0.643904in}}{\pgfqpoint{6.200000in}{4.620000in}}%
\pgfusepath{clip}%
\pgfsetrectcap%
\pgfsetroundjoin%
\pgfsetlinewidth{0.803000pt}%
\definecolor{currentstroke}{rgb}{0.690196,0.690196,0.690196}%
\pgfsetstrokecolor{currentstroke}%
\pgfsetdash{}{0pt}%
\pgfpathmoveto{\pgfqpoint{6.179620in}{0.643904in}}%
\pgfpathlineto{\pgfqpoint{6.179620in}{5.263904in}}%
\pgfusepath{stroke}%
\end{pgfscope}%
\begin{pgfscope}%
\pgfsetbuttcap%
\pgfsetroundjoin%
\definecolor{currentfill}{rgb}{0.000000,0.000000,0.000000}%
\pgfsetfillcolor{currentfill}%
\pgfsetlinewidth{0.803000pt}%
\definecolor{currentstroke}{rgb}{0.000000,0.000000,0.000000}%
\pgfsetstrokecolor{currentstroke}%
\pgfsetdash{}{0pt}%
\pgfsys@defobject{currentmarker}{\pgfqpoint{0.000000in}{-0.048611in}}{\pgfqpoint{0.000000in}{0.000000in}}{%
\pgfpathmoveto{\pgfqpoint{0.000000in}{0.000000in}}%
\pgfpathlineto{\pgfqpoint{0.000000in}{-0.048611in}}%
\pgfusepath{stroke,fill}%
}%
\begin{pgfscope}%
\pgfsys@transformshift{6.179620in}{0.643904in}%
\pgfsys@useobject{currentmarker}{}%
\end{pgfscope}%
\end{pgfscope}%
\begin{pgfscope}%
\definecolor{textcolor}{rgb}{0.000000,0.000000,0.000000}%
\pgfsetstrokecolor{textcolor}%
\pgfsetfillcolor{textcolor}%
\pgftext[x=6.179620in,y=0.546682in,,top]{\color{textcolor}{\rmfamily\fontsize{14.000000}{16.800000}\selectfont\catcode`\^=\active\def^{\ifmmode\sp\else\^{}\fi}\catcode`\%=\active\def%{\%}$\mathdefault{180}$}}%
\end{pgfscope}%
\begin{pgfscope}%
\pgfpathrectangle{\pgfqpoint{0.688192in}{0.643904in}}{\pgfqpoint{6.200000in}{4.620000in}}%
\pgfusepath{clip}%
\pgfsetrectcap%
\pgfsetroundjoin%
\pgfsetlinewidth{0.803000pt}%
\definecolor{currentstroke}{rgb}{0.690196,0.690196,0.690196}%
\pgfsetstrokecolor{currentstroke}%
\pgfsetdash{}{0pt}%
\pgfpathmoveto{\pgfqpoint{6.888192in}{0.643904in}}%
\pgfpathlineto{\pgfqpoint{6.888192in}{5.263904in}}%
\pgfusepath{stroke}%
\end{pgfscope}%
\begin{pgfscope}%
\pgfsetbuttcap%
\pgfsetroundjoin%
\definecolor{currentfill}{rgb}{0.000000,0.000000,0.000000}%
\pgfsetfillcolor{currentfill}%
\pgfsetlinewidth{0.803000pt}%
\definecolor{currentstroke}{rgb}{0.000000,0.000000,0.000000}%
\pgfsetstrokecolor{currentstroke}%
\pgfsetdash{}{0pt}%
\pgfsys@defobject{currentmarker}{\pgfqpoint{0.000000in}{-0.048611in}}{\pgfqpoint{0.000000in}{0.000000in}}{%
\pgfpathmoveto{\pgfqpoint{0.000000in}{0.000000in}}%
\pgfpathlineto{\pgfqpoint{0.000000in}{-0.048611in}}%
\pgfusepath{stroke,fill}%
}%
\begin{pgfscope}%
\pgfsys@transformshift{6.888192in}{0.643904in}%
\pgfsys@useobject{currentmarker}{}%
\end{pgfscope}%
\end{pgfscope}%
\begin{pgfscope}%
\definecolor{textcolor}{rgb}{0.000000,0.000000,0.000000}%
\pgfsetstrokecolor{textcolor}%
\pgfsetfillcolor{textcolor}%
\pgftext[x=6.888192in,y=0.546682in,,top]{\color{textcolor}{\rmfamily\fontsize{14.000000}{16.800000}\selectfont\catcode`\^=\active\def^{\ifmmode\sp\else\^{}\fi}\catcode`\%=\active\def%{\%}$\mathdefault{200}$}}%
\end{pgfscope}%
\begin{pgfscope}%
\pgfpathrectangle{\pgfqpoint{0.688192in}{0.643904in}}{\pgfqpoint{6.200000in}{4.620000in}}%
\pgfusepath{clip}%
\pgfsetbuttcap%
\pgfsetroundjoin%
\pgfsetlinewidth{0.803000pt}%
\definecolor{currentstroke}{rgb}{0.690196,0.690196,0.690196}%
\pgfsetstrokecolor{currentstroke}%
\pgfsetstrokeopacity{0.200000}%
\pgfsetdash{{2.960000pt}{1.280000pt}}{0.000000pt}%
\pgfpathmoveto{\pgfqpoint{0.688192in}{0.643904in}}%
\pgfpathlineto{\pgfqpoint{0.688192in}{5.263904in}}%
\pgfusepath{stroke}%
\end{pgfscope}%
\begin{pgfscope}%
\pgfsetbuttcap%
\pgfsetroundjoin%
\definecolor{currentfill}{rgb}{0.000000,0.000000,0.000000}%
\pgfsetfillcolor{currentfill}%
\pgfsetlinewidth{0.602250pt}%
\definecolor{currentstroke}{rgb}{0.000000,0.000000,0.000000}%
\pgfsetstrokecolor{currentstroke}%
\pgfsetdash{}{0pt}%
\pgfsys@defobject{currentmarker}{\pgfqpoint{0.000000in}{-0.027778in}}{\pgfqpoint{0.000000in}{0.000000in}}{%
\pgfpathmoveto{\pgfqpoint{0.000000in}{0.000000in}}%
\pgfpathlineto{\pgfqpoint{0.000000in}{-0.027778in}}%
\pgfusepath{stroke,fill}%
}%
\begin{pgfscope}%
\pgfsys@transformshift{0.688192in}{0.643904in}%
\pgfsys@useobject{currentmarker}{}%
\end{pgfscope}%
\end{pgfscope}%
\begin{pgfscope}%
\pgfpathrectangle{\pgfqpoint{0.688192in}{0.643904in}}{\pgfqpoint{6.200000in}{4.620000in}}%
\pgfusepath{clip}%
\pgfsetbuttcap%
\pgfsetroundjoin%
\pgfsetlinewidth{0.803000pt}%
\definecolor{currentstroke}{rgb}{0.690196,0.690196,0.690196}%
\pgfsetstrokecolor{currentstroke}%
\pgfsetstrokeopacity{0.200000}%
\pgfsetdash{{2.960000pt}{1.280000pt}}{0.000000pt}%
\pgfpathmoveto{\pgfqpoint{0.865334in}{0.643904in}}%
\pgfpathlineto{\pgfqpoint{0.865334in}{5.263904in}}%
\pgfusepath{stroke}%
\end{pgfscope}%
\begin{pgfscope}%
\pgfsetbuttcap%
\pgfsetroundjoin%
\definecolor{currentfill}{rgb}{0.000000,0.000000,0.000000}%
\pgfsetfillcolor{currentfill}%
\pgfsetlinewidth{0.602250pt}%
\definecolor{currentstroke}{rgb}{0.000000,0.000000,0.000000}%
\pgfsetstrokecolor{currentstroke}%
\pgfsetdash{}{0pt}%
\pgfsys@defobject{currentmarker}{\pgfqpoint{0.000000in}{-0.027778in}}{\pgfqpoint{0.000000in}{0.000000in}}{%
\pgfpathmoveto{\pgfqpoint{0.000000in}{0.000000in}}%
\pgfpathlineto{\pgfqpoint{0.000000in}{-0.027778in}}%
\pgfusepath{stroke,fill}%
}%
\begin{pgfscope}%
\pgfsys@transformshift{0.865334in}{0.643904in}%
\pgfsys@useobject{currentmarker}{}%
\end{pgfscope}%
\end{pgfscope}%
\begin{pgfscope}%
\pgfpathrectangle{\pgfqpoint{0.688192in}{0.643904in}}{\pgfqpoint{6.200000in}{4.620000in}}%
\pgfusepath{clip}%
\pgfsetbuttcap%
\pgfsetroundjoin%
\pgfsetlinewidth{0.803000pt}%
\definecolor{currentstroke}{rgb}{0.690196,0.690196,0.690196}%
\pgfsetstrokecolor{currentstroke}%
\pgfsetstrokeopacity{0.200000}%
\pgfsetdash{{2.960000pt}{1.280000pt}}{0.000000pt}%
\pgfpathmoveto{\pgfqpoint{1.042477in}{0.643904in}}%
\pgfpathlineto{\pgfqpoint{1.042477in}{5.263904in}}%
\pgfusepath{stroke}%
\end{pgfscope}%
\begin{pgfscope}%
\pgfsetbuttcap%
\pgfsetroundjoin%
\definecolor{currentfill}{rgb}{0.000000,0.000000,0.000000}%
\pgfsetfillcolor{currentfill}%
\pgfsetlinewidth{0.602250pt}%
\definecolor{currentstroke}{rgb}{0.000000,0.000000,0.000000}%
\pgfsetstrokecolor{currentstroke}%
\pgfsetdash{}{0pt}%
\pgfsys@defobject{currentmarker}{\pgfqpoint{0.000000in}{-0.027778in}}{\pgfqpoint{0.000000in}{0.000000in}}{%
\pgfpathmoveto{\pgfqpoint{0.000000in}{0.000000in}}%
\pgfpathlineto{\pgfqpoint{0.000000in}{-0.027778in}}%
\pgfusepath{stroke,fill}%
}%
\begin{pgfscope}%
\pgfsys@transformshift{1.042477in}{0.643904in}%
\pgfsys@useobject{currentmarker}{}%
\end{pgfscope}%
\end{pgfscope}%
\begin{pgfscope}%
\pgfpathrectangle{\pgfqpoint{0.688192in}{0.643904in}}{\pgfqpoint{6.200000in}{4.620000in}}%
\pgfusepath{clip}%
\pgfsetbuttcap%
\pgfsetroundjoin%
\pgfsetlinewidth{0.803000pt}%
\definecolor{currentstroke}{rgb}{0.690196,0.690196,0.690196}%
\pgfsetstrokecolor{currentstroke}%
\pgfsetstrokeopacity{0.200000}%
\pgfsetdash{{2.960000pt}{1.280000pt}}{0.000000pt}%
\pgfpathmoveto{\pgfqpoint{1.396763in}{0.643904in}}%
\pgfpathlineto{\pgfqpoint{1.396763in}{5.263904in}}%
\pgfusepath{stroke}%
\end{pgfscope}%
\begin{pgfscope}%
\pgfsetbuttcap%
\pgfsetroundjoin%
\definecolor{currentfill}{rgb}{0.000000,0.000000,0.000000}%
\pgfsetfillcolor{currentfill}%
\pgfsetlinewidth{0.602250pt}%
\definecolor{currentstroke}{rgb}{0.000000,0.000000,0.000000}%
\pgfsetstrokecolor{currentstroke}%
\pgfsetdash{}{0pt}%
\pgfsys@defobject{currentmarker}{\pgfqpoint{0.000000in}{-0.027778in}}{\pgfqpoint{0.000000in}{0.000000in}}{%
\pgfpathmoveto{\pgfqpoint{0.000000in}{0.000000in}}%
\pgfpathlineto{\pgfqpoint{0.000000in}{-0.027778in}}%
\pgfusepath{stroke,fill}%
}%
\begin{pgfscope}%
\pgfsys@transformshift{1.396763in}{0.643904in}%
\pgfsys@useobject{currentmarker}{}%
\end{pgfscope}%
\end{pgfscope}%
\begin{pgfscope}%
\pgfpathrectangle{\pgfqpoint{0.688192in}{0.643904in}}{\pgfqpoint{6.200000in}{4.620000in}}%
\pgfusepath{clip}%
\pgfsetbuttcap%
\pgfsetroundjoin%
\pgfsetlinewidth{0.803000pt}%
\definecolor{currentstroke}{rgb}{0.690196,0.690196,0.690196}%
\pgfsetstrokecolor{currentstroke}%
\pgfsetstrokeopacity{0.200000}%
\pgfsetdash{{2.960000pt}{1.280000pt}}{0.000000pt}%
\pgfpathmoveto{\pgfqpoint{1.573906in}{0.643904in}}%
\pgfpathlineto{\pgfqpoint{1.573906in}{5.263904in}}%
\pgfusepath{stroke}%
\end{pgfscope}%
\begin{pgfscope}%
\pgfsetbuttcap%
\pgfsetroundjoin%
\definecolor{currentfill}{rgb}{0.000000,0.000000,0.000000}%
\pgfsetfillcolor{currentfill}%
\pgfsetlinewidth{0.602250pt}%
\definecolor{currentstroke}{rgb}{0.000000,0.000000,0.000000}%
\pgfsetstrokecolor{currentstroke}%
\pgfsetdash{}{0pt}%
\pgfsys@defobject{currentmarker}{\pgfqpoint{0.000000in}{-0.027778in}}{\pgfqpoint{0.000000in}{0.000000in}}{%
\pgfpathmoveto{\pgfqpoint{0.000000in}{0.000000in}}%
\pgfpathlineto{\pgfqpoint{0.000000in}{-0.027778in}}%
\pgfusepath{stroke,fill}%
}%
\begin{pgfscope}%
\pgfsys@transformshift{1.573906in}{0.643904in}%
\pgfsys@useobject{currentmarker}{}%
\end{pgfscope}%
\end{pgfscope}%
\begin{pgfscope}%
\pgfpathrectangle{\pgfqpoint{0.688192in}{0.643904in}}{\pgfqpoint{6.200000in}{4.620000in}}%
\pgfusepath{clip}%
\pgfsetbuttcap%
\pgfsetroundjoin%
\pgfsetlinewidth{0.803000pt}%
\definecolor{currentstroke}{rgb}{0.690196,0.690196,0.690196}%
\pgfsetstrokecolor{currentstroke}%
\pgfsetstrokeopacity{0.200000}%
\pgfsetdash{{2.960000pt}{1.280000pt}}{0.000000pt}%
\pgfpathmoveto{\pgfqpoint{1.751049in}{0.643904in}}%
\pgfpathlineto{\pgfqpoint{1.751049in}{5.263904in}}%
\pgfusepath{stroke}%
\end{pgfscope}%
\begin{pgfscope}%
\pgfsetbuttcap%
\pgfsetroundjoin%
\definecolor{currentfill}{rgb}{0.000000,0.000000,0.000000}%
\pgfsetfillcolor{currentfill}%
\pgfsetlinewidth{0.602250pt}%
\definecolor{currentstroke}{rgb}{0.000000,0.000000,0.000000}%
\pgfsetstrokecolor{currentstroke}%
\pgfsetdash{}{0pt}%
\pgfsys@defobject{currentmarker}{\pgfqpoint{0.000000in}{-0.027778in}}{\pgfqpoint{0.000000in}{0.000000in}}{%
\pgfpathmoveto{\pgfqpoint{0.000000in}{0.000000in}}%
\pgfpathlineto{\pgfqpoint{0.000000in}{-0.027778in}}%
\pgfusepath{stroke,fill}%
}%
\begin{pgfscope}%
\pgfsys@transformshift{1.751049in}{0.643904in}%
\pgfsys@useobject{currentmarker}{}%
\end{pgfscope}%
\end{pgfscope}%
\begin{pgfscope}%
\pgfpathrectangle{\pgfqpoint{0.688192in}{0.643904in}}{\pgfqpoint{6.200000in}{4.620000in}}%
\pgfusepath{clip}%
\pgfsetbuttcap%
\pgfsetroundjoin%
\pgfsetlinewidth{0.803000pt}%
\definecolor{currentstroke}{rgb}{0.690196,0.690196,0.690196}%
\pgfsetstrokecolor{currentstroke}%
\pgfsetstrokeopacity{0.200000}%
\pgfsetdash{{2.960000pt}{1.280000pt}}{0.000000pt}%
\pgfpathmoveto{\pgfqpoint{2.105334in}{0.643904in}}%
\pgfpathlineto{\pgfqpoint{2.105334in}{5.263904in}}%
\pgfusepath{stroke}%
\end{pgfscope}%
\begin{pgfscope}%
\pgfsetbuttcap%
\pgfsetroundjoin%
\definecolor{currentfill}{rgb}{0.000000,0.000000,0.000000}%
\pgfsetfillcolor{currentfill}%
\pgfsetlinewidth{0.602250pt}%
\definecolor{currentstroke}{rgb}{0.000000,0.000000,0.000000}%
\pgfsetstrokecolor{currentstroke}%
\pgfsetdash{}{0pt}%
\pgfsys@defobject{currentmarker}{\pgfqpoint{0.000000in}{-0.027778in}}{\pgfqpoint{0.000000in}{0.000000in}}{%
\pgfpathmoveto{\pgfqpoint{0.000000in}{0.000000in}}%
\pgfpathlineto{\pgfqpoint{0.000000in}{-0.027778in}}%
\pgfusepath{stroke,fill}%
}%
\begin{pgfscope}%
\pgfsys@transformshift{2.105334in}{0.643904in}%
\pgfsys@useobject{currentmarker}{}%
\end{pgfscope}%
\end{pgfscope}%
\begin{pgfscope}%
\pgfpathrectangle{\pgfqpoint{0.688192in}{0.643904in}}{\pgfqpoint{6.200000in}{4.620000in}}%
\pgfusepath{clip}%
\pgfsetbuttcap%
\pgfsetroundjoin%
\pgfsetlinewidth{0.803000pt}%
\definecolor{currentstroke}{rgb}{0.690196,0.690196,0.690196}%
\pgfsetstrokecolor{currentstroke}%
\pgfsetstrokeopacity{0.200000}%
\pgfsetdash{{2.960000pt}{1.280000pt}}{0.000000pt}%
\pgfpathmoveto{\pgfqpoint{2.282477in}{0.643904in}}%
\pgfpathlineto{\pgfqpoint{2.282477in}{5.263904in}}%
\pgfusepath{stroke}%
\end{pgfscope}%
\begin{pgfscope}%
\pgfsetbuttcap%
\pgfsetroundjoin%
\definecolor{currentfill}{rgb}{0.000000,0.000000,0.000000}%
\pgfsetfillcolor{currentfill}%
\pgfsetlinewidth{0.602250pt}%
\definecolor{currentstroke}{rgb}{0.000000,0.000000,0.000000}%
\pgfsetstrokecolor{currentstroke}%
\pgfsetdash{}{0pt}%
\pgfsys@defobject{currentmarker}{\pgfqpoint{0.000000in}{-0.027778in}}{\pgfqpoint{0.000000in}{0.000000in}}{%
\pgfpathmoveto{\pgfqpoint{0.000000in}{0.000000in}}%
\pgfpathlineto{\pgfqpoint{0.000000in}{-0.027778in}}%
\pgfusepath{stroke,fill}%
}%
\begin{pgfscope}%
\pgfsys@transformshift{2.282477in}{0.643904in}%
\pgfsys@useobject{currentmarker}{}%
\end{pgfscope}%
\end{pgfscope}%
\begin{pgfscope}%
\pgfpathrectangle{\pgfqpoint{0.688192in}{0.643904in}}{\pgfqpoint{6.200000in}{4.620000in}}%
\pgfusepath{clip}%
\pgfsetbuttcap%
\pgfsetroundjoin%
\pgfsetlinewidth{0.803000pt}%
\definecolor{currentstroke}{rgb}{0.690196,0.690196,0.690196}%
\pgfsetstrokecolor{currentstroke}%
\pgfsetstrokeopacity{0.200000}%
\pgfsetdash{{2.960000pt}{1.280000pt}}{0.000000pt}%
\pgfpathmoveto{\pgfqpoint{2.459620in}{0.643904in}}%
\pgfpathlineto{\pgfqpoint{2.459620in}{5.263904in}}%
\pgfusepath{stroke}%
\end{pgfscope}%
\begin{pgfscope}%
\pgfsetbuttcap%
\pgfsetroundjoin%
\definecolor{currentfill}{rgb}{0.000000,0.000000,0.000000}%
\pgfsetfillcolor{currentfill}%
\pgfsetlinewidth{0.602250pt}%
\definecolor{currentstroke}{rgb}{0.000000,0.000000,0.000000}%
\pgfsetstrokecolor{currentstroke}%
\pgfsetdash{}{0pt}%
\pgfsys@defobject{currentmarker}{\pgfqpoint{0.000000in}{-0.027778in}}{\pgfqpoint{0.000000in}{0.000000in}}{%
\pgfpathmoveto{\pgfqpoint{0.000000in}{0.000000in}}%
\pgfpathlineto{\pgfqpoint{0.000000in}{-0.027778in}}%
\pgfusepath{stroke,fill}%
}%
\begin{pgfscope}%
\pgfsys@transformshift{2.459620in}{0.643904in}%
\pgfsys@useobject{currentmarker}{}%
\end{pgfscope}%
\end{pgfscope}%
\begin{pgfscope}%
\pgfpathrectangle{\pgfqpoint{0.688192in}{0.643904in}}{\pgfqpoint{6.200000in}{4.620000in}}%
\pgfusepath{clip}%
\pgfsetbuttcap%
\pgfsetroundjoin%
\pgfsetlinewidth{0.803000pt}%
\definecolor{currentstroke}{rgb}{0.690196,0.690196,0.690196}%
\pgfsetstrokecolor{currentstroke}%
\pgfsetstrokeopacity{0.200000}%
\pgfsetdash{{2.960000pt}{1.280000pt}}{0.000000pt}%
\pgfpathmoveto{\pgfqpoint{2.813906in}{0.643904in}}%
\pgfpathlineto{\pgfqpoint{2.813906in}{5.263904in}}%
\pgfusepath{stroke}%
\end{pgfscope}%
\begin{pgfscope}%
\pgfsetbuttcap%
\pgfsetroundjoin%
\definecolor{currentfill}{rgb}{0.000000,0.000000,0.000000}%
\pgfsetfillcolor{currentfill}%
\pgfsetlinewidth{0.602250pt}%
\definecolor{currentstroke}{rgb}{0.000000,0.000000,0.000000}%
\pgfsetstrokecolor{currentstroke}%
\pgfsetdash{}{0pt}%
\pgfsys@defobject{currentmarker}{\pgfqpoint{0.000000in}{-0.027778in}}{\pgfqpoint{0.000000in}{0.000000in}}{%
\pgfpathmoveto{\pgfqpoint{0.000000in}{0.000000in}}%
\pgfpathlineto{\pgfqpoint{0.000000in}{-0.027778in}}%
\pgfusepath{stroke,fill}%
}%
\begin{pgfscope}%
\pgfsys@transformshift{2.813906in}{0.643904in}%
\pgfsys@useobject{currentmarker}{}%
\end{pgfscope}%
\end{pgfscope}%
\begin{pgfscope}%
\pgfpathrectangle{\pgfqpoint{0.688192in}{0.643904in}}{\pgfqpoint{6.200000in}{4.620000in}}%
\pgfusepath{clip}%
\pgfsetbuttcap%
\pgfsetroundjoin%
\pgfsetlinewidth{0.803000pt}%
\definecolor{currentstroke}{rgb}{0.690196,0.690196,0.690196}%
\pgfsetstrokecolor{currentstroke}%
\pgfsetstrokeopacity{0.200000}%
\pgfsetdash{{2.960000pt}{1.280000pt}}{0.000000pt}%
\pgfpathmoveto{\pgfqpoint{2.991049in}{0.643904in}}%
\pgfpathlineto{\pgfqpoint{2.991049in}{5.263904in}}%
\pgfusepath{stroke}%
\end{pgfscope}%
\begin{pgfscope}%
\pgfsetbuttcap%
\pgfsetroundjoin%
\definecolor{currentfill}{rgb}{0.000000,0.000000,0.000000}%
\pgfsetfillcolor{currentfill}%
\pgfsetlinewidth{0.602250pt}%
\definecolor{currentstroke}{rgb}{0.000000,0.000000,0.000000}%
\pgfsetstrokecolor{currentstroke}%
\pgfsetdash{}{0pt}%
\pgfsys@defobject{currentmarker}{\pgfqpoint{0.000000in}{-0.027778in}}{\pgfqpoint{0.000000in}{0.000000in}}{%
\pgfpathmoveto{\pgfqpoint{0.000000in}{0.000000in}}%
\pgfpathlineto{\pgfqpoint{0.000000in}{-0.027778in}}%
\pgfusepath{stroke,fill}%
}%
\begin{pgfscope}%
\pgfsys@transformshift{2.991049in}{0.643904in}%
\pgfsys@useobject{currentmarker}{}%
\end{pgfscope}%
\end{pgfscope}%
\begin{pgfscope}%
\pgfpathrectangle{\pgfqpoint{0.688192in}{0.643904in}}{\pgfqpoint{6.200000in}{4.620000in}}%
\pgfusepath{clip}%
\pgfsetbuttcap%
\pgfsetroundjoin%
\pgfsetlinewidth{0.803000pt}%
\definecolor{currentstroke}{rgb}{0.690196,0.690196,0.690196}%
\pgfsetstrokecolor{currentstroke}%
\pgfsetstrokeopacity{0.200000}%
\pgfsetdash{{2.960000pt}{1.280000pt}}{0.000000pt}%
\pgfpathmoveto{\pgfqpoint{3.168192in}{0.643904in}}%
\pgfpathlineto{\pgfqpoint{3.168192in}{5.263904in}}%
\pgfusepath{stroke}%
\end{pgfscope}%
\begin{pgfscope}%
\pgfsetbuttcap%
\pgfsetroundjoin%
\definecolor{currentfill}{rgb}{0.000000,0.000000,0.000000}%
\pgfsetfillcolor{currentfill}%
\pgfsetlinewidth{0.602250pt}%
\definecolor{currentstroke}{rgb}{0.000000,0.000000,0.000000}%
\pgfsetstrokecolor{currentstroke}%
\pgfsetdash{}{0pt}%
\pgfsys@defobject{currentmarker}{\pgfqpoint{0.000000in}{-0.027778in}}{\pgfqpoint{0.000000in}{0.000000in}}{%
\pgfpathmoveto{\pgfqpoint{0.000000in}{0.000000in}}%
\pgfpathlineto{\pgfqpoint{0.000000in}{-0.027778in}}%
\pgfusepath{stroke,fill}%
}%
\begin{pgfscope}%
\pgfsys@transformshift{3.168192in}{0.643904in}%
\pgfsys@useobject{currentmarker}{}%
\end{pgfscope}%
\end{pgfscope}%
\begin{pgfscope}%
\pgfpathrectangle{\pgfqpoint{0.688192in}{0.643904in}}{\pgfqpoint{6.200000in}{4.620000in}}%
\pgfusepath{clip}%
\pgfsetbuttcap%
\pgfsetroundjoin%
\pgfsetlinewidth{0.803000pt}%
\definecolor{currentstroke}{rgb}{0.690196,0.690196,0.690196}%
\pgfsetstrokecolor{currentstroke}%
\pgfsetstrokeopacity{0.200000}%
\pgfsetdash{{2.960000pt}{1.280000pt}}{0.000000pt}%
\pgfpathmoveto{\pgfqpoint{3.522477in}{0.643904in}}%
\pgfpathlineto{\pgfqpoint{3.522477in}{5.263904in}}%
\pgfusepath{stroke}%
\end{pgfscope}%
\begin{pgfscope}%
\pgfsetbuttcap%
\pgfsetroundjoin%
\definecolor{currentfill}{rgb}{0.000000,0.000000,0.000000}%
\pgfsetfillcolor{currentfill}%
\pgfsetlinewidth{0.602250pt}%
\definecolor{currentstroke}{rgb}{0.000000,0.000000,0.000000}%
\pgfsetstrokecolor{currentstroke}%
\pgfsetdash{}{0pt}%
\pgfsys@defobject{currentmarker}{\pgfqpoint{0.000000in}{-0.027778in}}{\pgfqpoint{0.000000in}{0.000000in}}{%
\pgfpathmoveto{\pgfqpoint{0.000000in}{0.000000in}}%
\pgfpathlineto{\pgfqpoint{0.000000in}{-0.027778in}}%
\pgfusepath{stroke,fill}%
}%
\begin{pgfscope}%
\pgfsys@transformshift{3.522477in}{0.643904in}%
\pgfsys@useobject{currentmarker}{}%
\end{pgfscope}%
\end{pgfscope}%
\begin{pgfscope}%
\pgfpathrectangle{\pgfqpoint{0.688192in}{0.643904in}}{\pgfqpoint{6.200000in}{4.620000in}}%
\pgfusepath{clip}%
\pgfsetbuttcap%
\pgfsetroundjoin%
\pgfsetlinewidth{0.803000pt}%
\definecolor{currentstroke}{rgb}{0.690196,0.690196,0.690196}%
\pgfsetstrokecolor{currentstroke}%
\pgfsetstrokeopacity{0.200000}%
\pgfsetdash{{2.960000pt}{1.280000pt}}{0.000000pt}%
\pgfpathmoveto{\pgfqpoint{3.699620in}{0.643904in}}%
\pgfpathlineto{\pgfqpoint{3.699620in}{5.263904in}}%
\pgfusepath{stroke}%
\end{pgfscope}%
\begin{pgfscope}%
\pgfsetbuttcap%
\pgfsetroundjoin%
\definecolor{currentfill}{rgb}{0.000000,0.000000,0.000000}%
\pgfsetfillcolor{currentfill}%
\pgfsetlinewidth{0.602250pt}%
\definecolor{currentstroke}{rgb}{0.000000,0.000000,0.000000}%
\pgfsetstrokecolor{currentstroke}%
\pgfsetdash{}{0pt}%
\pgfsys@defobject{currentmarker}{\pgfqpoint{0.000000in}{-0.027778in}}{\pgfqpoint{0.000000in}{0.000000in}}{%
\pgfpathmoveto{\pgfqpoint{0.000000in}{0.000000in}}%
\pgfpathlineto{\pgfqpoint{0.000000in}{-0.027778in}}%
\pgfusepath{stroke,fill}%
}%
\begin{pgfscope}%
\pgfsys@transformshift{3.699620in}{0.643904in}%
\pgfsys@useobject{currentmarker}{}%
\end{pgfscope}%
\end{pgfscope}%
\begin{pgfscope}%
\pgfpathrectangle{\pgfqpoint{0.688192in}{0.643904in}}{\pgfqpoint{6.200000in}{4.620000in}}%
\pgfusepath{clip}%
\pgfsetbuttcap%
\pgfsetroundjoin%
\pgfsetlinewidth{0.803000pt}%
\definecolor{currentstroke}{rgb}{0.690196,0.690196,0.690196}%
\pgfsetstrokecolor{currentstroke}%
\pgfsetstrokeopacity{0.200000}%
\pgfsetdash{{2.960000pt}{1.280000pt}}{0.000000pt}%
\pgfpathmoveto{\pgfqpoint{3.876763in}{0.643904in}}%
\pgfpathlineto{\pgfqpoint{3.876763in}{5.263904in}}%
\pgfusepath{stroke}%
\end{pgfscope}%
\begin{pgfscope}%
\pgfsetbuttcap%
\pgfsetroundjoin%
\definecolor{currentfill}{rgb}{0.000000,0.000000,0.000000}%
\pgfsetfillcolor{currentfill}%
\pgfsetlinewidth{0.602250pt}%
\definecolor{currentstroke}{rgb}{0.000000,0.000000,0.000000}%
\pgfsetstrokecolor{currentstroke}%
\pgfsetdash{}{0pt}%
\pgfsys@defobject{currentmarker}{\pgfqpoint{0.000000in}{-0.027778in}}{\pgfqpoint{0.000000in}{0.000000in}}{%
\pgfpathmoveto{\pgfqpoint{0.000000in}{0.000000in}}%
\pgfpathlineto{\pgfqpoint{0.000000in}{-0.027778in}}%
\pgfusepath{stroke,fill}%
}%
\begin{pgfscope}%
\pgfsys@transformshift{3.876763in}{0.643904in}%
\pgfsys@useobject{currentmarker}{}%
\end{pgfscope}%
\end{pgfscope}%
\begin{pgfscope}%
\pgfpathrectangle{\pgfqpoint{0.688192in}{0.643904in}}{\pgfqpoint{6.200000in}{4.620000in}}%
\pgfusepath{clip}%
\pgfsetbuttcap%
\pgfsetroundjoin%
\pgfsetlinewidth{0.803000pt}%
\definecolor{currentstroke}{rgb}{0.690196,0.690196,0.690196}%
\pgfsetstrokecolor{currentstroke}%
\pgfsetstrokeopacity{0.200000}%
\pgfsetdash{{2.960000pt}{1.280000pt}}{0.000000pt}%
\pgfpathmoveto{\pgfqpoint{4.231049in}{0.643904in}}%
\pgfpathlineto{\pgfqpoint{4.231049in}{5.263904in}}%
\pgfusepath{stroke}%
\end{pgfscope}%
\begin{pgfscope}%
\pgfsetbuttcap%
\pgfsetroundjoin%
\definecolor{currentfill}{rgb}{0.000000,0.000000,0.000000}%
\pgfsetfillcolor{currentfill}%
\pgfsetlinewidth{0.602250pt}%
\definecolor{currentstroke}{rgb}{0.000000,0.000000,0.000000}%
\pgfsetstrokecolor{currentstroke}%
\pgfsetdash{}{0pt}%
\pgfsys@defobject{currentmarker}{\pgfqpoint{0.000000in}{-0.027778in}}{\pgfqpoint{0.000000in}{0.000000in}}{%
\pgfpathmoveto{\pgfqpoint{0.000000in}{0.000000in}}%
\pgfpathlineto{\pgfqpoint{0.000000in}{-0.027778in}}%
\pgfusepath{stroke,fill}%
}%
\begin{pgfscope}%
\pgfsys@transformshift{4.231049in}{0.643904in}%
\pgfsys@useobject{currentmarker}{}%
\end{pgfscope}%
\end{pgfscope}%
\begin{pgfscope}%
\pgfpathrectangle{\pgfqpoint{0.688192in}{0.643904in}}{\pgfqpoint{6.200000in}{4.620000in}}%
\pgfusepath{clip}%
\pgfsetbuttcap%
\pgfsetroundjoin%
\pgfsetlinewidth{0.803000pt}%
\definecolor{currentstroke}{rgb}{0.690196,0.690196,0.690196}%
\pgfsetstrokecolor{currentstroke}%
\pgfsetstrokeopacity{0.200000}%
\pgfsetdash{{2.960000pt}{1.280000pt}}{0.000000pt}%
\pgfpathmoveto{\pgfqpoint{4.408192in}{0.643904in}}%
\pgfpathlineto{\pgfqpoint{4.408192in}{5.263904in}}%
\pgfusepath{stroke}%
\end{pgfscope}%
\begin{pgfscope}%
\pgfsetbuttcap%
\pgfsetroundjoin%
\definecolor{currentfill}{rgb}{0.000000,0.000000,0.000000}%
\pgfsetfillcolor{currentfill}%
\pgfsetlinewidth{0.602250pt}%
\definecolor{currentstroke}{rgb}{0.000000,0.000000,0.000000}%
\pgfsetstrokecolor{currentstroke}%
\pgfsetdash{}{0pt}%
\pgfsys@defobject{currentmarker}{\pgfqpoint{0.000000in}{-0.027778in}}{\pgfqpoint{0.000000in}{0.000000in}}{%
\pgfpathmoveto{\pgfqpoint{0.000000in}{0.000000in}}%
\pgfpathlineto{\pgfqpoint{0.000000in}{-0.027778in}}%
\pgfusepath{stroke,fill}%
}%
\begin{pgfscope}%
\pgfsys@transformshift{4.408192in}{0.643904in}%
\pgfsys@useobject{currentmarker}{}%
\end{pgfscope}%
\end{pgfscope}%
\begin{pgfscope}%
\pgfpathrectangle{\pgfqpoint{0.688192in}{0.643904in}}{\pgfqpoint{6.200000in}{4.620000in}}%
\pgfusepath{clip}%
\pgfsetbuttcap%
\pgfsetroundjoin%
\pgfsetlinewidth{0.803000pt}%
\definecolor{currentstroke}{rgb}{0.690196,0.690196,0.690196}%
\pgfsetstrokecolor{currentstroke}%
\pgfsetstrokeopacity{0.200000}%
\pgfsetdash{{2.960000pt}{1.280000pt}}{0.000000pt}%
\pgfpathmoveto{\pgfqpoint{4.585334in}{0.643904in}}%
\pgfpathlineto{\pgfqpoint{4.585334in}{5.263904in}}%
\pgfusepath{stroke}%
\end{pgfscope}%
\begin{pgfscope}%
\pgfsetbuttcap%
\pgfsetroundjoin%
\definecolor{currentfill}{rgb}{0.000000,0.000000,0.000000}%
\pgfsetfillcolor{currentfill}%
\pgfsetlinewidth{0.602250pt}%
\definecolor{currentstroke}{rgb}{0.000000,0.000000,0.000000}%
\pgfsetstrokecolor{currentstroke}%
\pgfsetdash{}{0pt}%
\pgfsys@defobject{currentmarker}{\pgfqpoint{0.000000in}{-0.027778in}}{\pgfqpoint{0.000000in}{0.000000in}}{%
\pgfpathmoveto{\pgfqpoint{0.000000in}{0.000000in}}%
\pgfpathlineto{\pgfqpoint{0.000000in}{-0.027778in}}%
\pgfusepath{stroke,fill}%
}%
\begin{pgfscope}%
\pgfsys@transformshift{4.585334in}{0.643904in}%
\pgfsys@useobject{currentmarker}{}%
\end{pgfscope}%
\end{pgfscope}%
\begin{pgfscope}%
\pgfpathrectangle{\pgfqpoint{0.688192in}{0.643904in}}{\pgfqpoint{6.200000in}{4.620000in}}%
\pgfusepath{clip}%
\pgfsetbuttcap%
\pgfsetroundjoin%
\pgfsetlinewidth{0.803000pt}%
\definecolor{currentstroke}{rgb}{0.690196,0.690196,0.690196}%
\pgfsetstrokecolor{currentstroke}%
\pgfsetstrokeopacity{0.200000}%
\pgfsetdash{{2.960000pt}{1.280000pt}}{0.000000pt}%
\pgfpathmoveto{\pgfqpoint{4.939620in}{0.643904in}}%
\pgfpathlineto{\pgfqpoint{4.939620in}{5.263904in}}%
\pgfusepath{stroke}%
\end{pgfscope}%
\begin{pgfscope}%
\pgfsetbuttcap%
\pgfsetroundjoin%
\definecolor{currentfill}{rgb}{0.000000,0.000000,0.000000}%
\pgfsetfillcolor{currentfill}%
\pgfsetlinewidth{0.602250pt}%
\definecolor{currentstroke}{rgb}{0.000000,0.000000,0.000000}%
\pgfsetstrokecolor{currentstroke}%
\pgfsetdash{}{0pt}%
\pgfsys@defobject{currentmarker}{\pgfqpoint{0.000000in}{-0.027778in}}{\pgfqpoint{0.000000in}{0.000000in}}{%
\pgfpathmoveto{\pgfqpoint{0.000000in}{0.000000in}}%
\pgfpathlineto{\pgfqpoint{0.000000in}{-0.027778in}}%
\pgfusepath{stroke,fill}%
}%
\begin{pgfscope}%
\pgfsys@transformshift{4.939620in}{0.643904in}%
\pgfsys@useobject{currentmarker}{}%
\end{pgfscope}%
\end{pgfscope}%
\begin{pgfscope}%
\pgfpathrectangle{\pgfqpoint{0.688192in}{0.643904in}}{\pgfqpoint{6.200000in}{4.620000in}}%
\pgfusepath{clip}%
\pgfsetbuttcap%
\pgfsetroundjoin%
\pgfsetlinewidth{0.803000pt}%
\definecolor{currentstroke}{rgb}{0.690196,0.690196,0.690196}%
\pgfsetstrokecolor{currentstroke}%
\pgfsetstrokeopacity{0.200000}%
\pgfsetdash{{2.960000pt}{1.280000pt}}{0.000000pt}%
\pgfpathmoveto{\pgfqpoint{5.116763in}{0.643904in}}%
\pgfpathlineto{\pgfqpoint{5.116763in}{5.263904in}}%
\pgfusepath{stroke}%
\end{pgfscope}%
\begin{pgfscope}%
\pgfsetbuttcap%
\pgfsetroundjoin%
\definecolor{currentfill}{rgb}{0.000000,0.000000,0.000000}%
\pgfsetfillcolor{currentfill}%
\pgfsetlinewidth{0.602250pt}%
\definecolor{currentstroke}{rgb}{0.000000,0.000000,0.000000}%
\pgfsetstrokecolor{currentstroke}%
\pgfsetdash{}{0pt}%
\pgfsys@defobject{currentmarker}{\pgfqpoint{0.000000in}{-0.027778in}}{\pgfqpoint{0.000000in}{0.000000in}}{%
\pgfpathmoveto{\pgfqpoint{0.000000in}{0.000000in}}%
\pgfpathlineto{\pgfqpoint{0.000000in}{-0.027778in}}%
\pgfusepath{stroke,fill}%
}%
\begin{pgfscope}%
\pgfsys@transformshift{5.116763in}{0.643904in}%
\pgfsys@useobject{currentmarker}{}%
\end{pgfscope}%
\end{pgfscope}%
\begin{pgfscope}%
\pgfpathrectangle{\pgfqpoint{0.688192in}{0.643904in}}{\pgfqpoint{6.200000in}{4.620000in}}%
\pgfusepath{clip}%
\pgfsetbuttcap%
\pgfsetroundjoin%
\pgfsetlinewidth{0.803000pt}%
\definecolor{currentstroke}{rgb}{0.690196,0.690196,0.690196}%
\pgfsetstrokecolor{currentstroke}%
\pgfsetstrokeopacity{0.200000}%
\pgfsetdash{{2.960000pt}{1.280000pt}}{0.000000pt}%
\pgfpathmoveto{\pgfqpoint{5.293906in}{0.643904in}}%
\pgfpathlineto{\pgfqpoint{5.293906in}{5.263904in}}%
\pgfusepath{stroke}%
\end{pgfscope}%
\begin{pgfscope}%
\pgfsetbuttcap%
\pgfsetroundjoin%
\definecolor{currentfill}{rgb}{0.000000,0.000000,0.000000}%
\pgfsetfillcolor{currentfill}%
\pgfsetlinewidth{0.602250pt}%
\definecolor{currentstroke}{rgb}{0.000000,0.000000,0.000000}%
\pgfsetstrokecolor{currentstroke}%
\pgfsetdash{}{0pt}%
\pgfsys@defobject{currentmarker}{\pgfqpoint{0.000000in}{-0.027778in}}{\pgfqpoint{0.000000in}{0.000000in}}{%
\pgfpathmoveto{\pgfqpoint{0.000000in}{0.000000in}}%
\pgfpathlineto{\pgfqpoint{0.000000in}{-0.027778in}}%
\pgfusepath{stroke,fill}%
}%
\begin{pgfscope}%
\pgfsys@transformshift{5.293906in}{0.643904in}%
\pgfsys@useobject{currentmarker}{}%
\end{pgfscope}%
\end{pgfscope}%
\begin{pgfscope}%
\pgfpathrectangle{\pgfqpoint{0.688192in}{0.643904in}}{\pgfqpoint{6.200000in}{4.620000in}}%
\pgfusepath{clip}%
\pgfsetbuttcap%
\pgfsetroundjoin%
\pgfsetlinewidth{0.803000pt}%
\definecolor{currentstroke}{rgb}{0.690196,0.690196,0.690196}%
\pgfsetstrokecolor{currentstroke}%
\pgfsetstrokeopacity{0.200000}%
\pgfsetdash{{2.960000pt}{1.280000pt}}{0.000000pt}%
\pgfpathmoveto{\pgfqpoint{5.648192in}{0.643904in}}%
\pgfpathlineto{\pgfqpoint{5.648192in}{5.263904in}}%
\pgfusepath{stroke}%
\end{pgfscope}%
\begin{pgfscope}%
\pgfsetbuttcap%
\pgfsetroundjoin%
\definecolor{currentfill}{rgb}{0.000000,0.000000,0.000000}%
\pgfsetfillcolor{currentfill}%
\pgfsetlinewidth{0.602250pt}%
\definecolor{currentstroke}{rgb}{0.000000,0.000000,0.000000}%
\pgfsetstrokecolor{currentstroke}%
\pgfsetdash{}{0pt}%
\pgfsys@defobject{currentmarker}{\pgfqpoint{0.000000in}{-0.027778in}}{\pgfqpoint{0.000000in}{0.000000in}}{%
\pgfpathmoveto{\pgfqpoint{0.000000in}{0.000000in}}%
\pgfpathlineto{\pgfqpoint{0.000000in}{-0.027778in}}%
\pgfusepath{stroke,fill}%
}%
\begin{pgfscope}%
\pgfsys@transformshift{5.648192in}{0.643904in}%
\pgfsys@useobject{currentmarker}{}%
\end{pgfscope}%
\end{pgfscope}%
\begin{pgfscope}%
\pgfpathrectangle{\pgfqpoint{0.688192in}{0.643904in}}{\pgfqpoint{6.200000in}{4.620000in}}%
\pgfusepath{clip}%
\pgfsetbuttcap%
\pgfsetroundjoin%
\pgfsetlinewidth{0.803000pt}%
\definecolor{currentstroke}{rgb}{0.690196,0.690196,0.690196}%
\pgfsetstrokecolor{currentstroke}%
\pgfsetstrokeopacity{0.200000}%
\pgfsetdash{{2.960000pt}{1.280000pt}}{0.000000pt}%
\pgfpathmoveto{\pgfqpoint{5.825334in}{0.643904in}}%
\pgfpathlineto{\pgfqpoint{5.825334in}{5.263904in}}%
\pgfusepath{stroke}%
\end{pgfscope}%
\begin{pgfscope}%
\pgfsetbuttcap%
\pgfsetroundjoin%
\definecolor{currentfill}{rgb}{0.000000,0.000000,0.000000}%
\pgfsetfillcolor{currentfill}%
\pgfsetlinewidth{0.602250pt}%
\definecolor{currentstroke}{rgb}{0.000000,0.000000,0.000000}%
\pgfsetstrokecolor{currentstroke}%
\pgfsetdash{}{0pt}%
\pgfsys@defobject{currentmarker}{\pgfqpoint{0.000000in}{-0.027778in}}{\pgfqpoint{0.000000in}{0.000000in}}{%
\pgfpathmoveto{\pgfqpoint{0.000000in}{0.000000in}}%
\pgfpathlineto{\pgfqpoint{0.000000in}{-0.027778in}}%
\pgfusepath{stroke,fill}%
}%
\begin{pgfscope}%
\pgfsys@transformshift{5.825334in}{0.643904in}%
\pgfsys@useobject{currentmarker}{}%
\end{pgfscope}%
\end{pgfscope}%
\begin{pgfscope}%
\pgfpathrectangle{\pgfqpoint{0.688192in}{0.643904in}}{\pgfqpoint{6.200000in}{4.620000in}}%
\pgfusepath{clip}%
\pgfsetbuttcap%
\pgfsetroundjoin%
\pgfsetlinewidth{0.803000pt}%
\definecolor{currentstroke}{rgb}{0.690196,0.690196,0.690196}%
\pgfsetstrokecolor{currentstroke}%
\pgfsetstrokeopacity{0.200000}%
\pgfsetdash{{2.960000pt}{1.280000pt}}{0.000000pt}%
\pgfpathmoveto{\pgfqpoint{6.002477in}{0.643904in}}%
\pgfpathlineto{\pgfqpoint{6.002477in}{5.263904in}}%
\pgfusepath{stroke}%
\end{pgfscope}%
\begin{pgfscope}%
\pgfsetbuttcap%
\pgfsetroundjoin%
\definecolor{currentfill}{rgb}{0.000000,0.000000,0.000000}%
\pgfsetfillcolor{currentfill}%
\pgfsetlinewidth{0.602250pt}%
\definecolor{currentstroke}{rgb}{0.000000,0.000000,0.000000}%
\pgfsetstrokecolor{currentstroke}%
\pgfsetdash{}{0pt}%
\pgfsys@defobject{currentmarker}{\pgfqpoint{0.000000in}{-0.027778in}}{\pgfqpoint{0.000000in}{0.000000in}}{%
\pgfpathmoveto{\pgfqpoint{0.000000in}{0.000000in}}%
\pgfpathlineto{\pgfqpoint{0.000000in}{-0.027778in}}%
\pgfusepath{stroke,fill}%
}%
\begin{pgfscope}%
\pgfsys@transformshift{6.002477in}{0.643904in}%
\pgfsys@useobject{currentmarker}{}%
\end{pgfscope}%
\end{pgfscope}%
\begin{pgfscope}%
\pgfpathrectangle{\pgfqpoint{0.688192in}{0.643904in}}{\pgfqpoint{6.200000in}{4.620000in}}%
\pgfusepath{clip}%
\pgfsetbuttcap%
\pgfsetroundjoin%
\pgfsetlinewidth{0.803000pt}%
\definecolor{currentstroke}{rgb}{0.690196,0.690196,0.690196}%
\pgfsetstrokecolor{currentstroke}%
\pgfsetstrokeopacity{0.200000}%
\pgfsetdash{{2.960000pt}{1.280000pt}}{0.000000pt}%
\pgfpathmoveto{\pgfqpoint{6.356763in}{0.643904in}}%
\pgfpathlineto{\pgfqpoint{6.356763in}{5.263904in}}%
\pgfusepath{stroke}%
\end{pgfscope}%
\begin{pgfscope}%
\pgfsetbuttcap%
\pgfsetroundjoin%
\definecolor{currentfill}{rgb}{0.000000,0.000000,0.000000}%
\pgfsetfillcolor{currentfill}%
\pgfsetlinewidth{0.602250pt}%
\definecolor{currentstroke}{rgb}{0.000000,0.000000,0.000000}%
\pgfsetstrokecolor{currentstroke}%
\pgfsetdash{}{0pt}%
\pgfsys@defobject{currentmarker}{\pgfqpoint{0.000000in}{-0.027778in}}{\pgfqpoint{0.000000in}{0.000000in}}{%
\pgfpathmoveto{\pgfqpoint{0.000000in}{0.000000in}}%
\pgfpathlineto{\pgfqpoint{0.000000in}{-0.027778in}}%
\pgfusepath{stroke,fill}%
}%
\begin{pgfscope}%
\pgfsys@transformshift{6.356763in}{0.643904in}%
\pgfsys@useobject{currentmarker}{}%
\end{pgfscope}%
\end{pgfscope}%
\begin{pgfscope}%
\pgfpathrectangle{\pgfqpoint{0.688192in}{0.643904in}}{\pgfqpoint{6.200000in}{4.620000in}}%
\pgfusepath{clip}%
\pgfsetbuttcap%
\pgfsetroundjoin%
\pgfsetlinewidth{0.803000pt}%
\definecolor{currentstroke}{rgb}{0.690196,0.690196,0.690196}%
\pgfsetstrokecolor{currentstroke}%
\pgfsetstrokeopacity{0.200000}%
\pgfsetdash{{2.960000pt}{1.280000pt}}{0.000000pt}%
\pgfpathmoveto{\pgfqpoint{6.533906in}{0.643904in}}%
\pgfpathlineto{\pgfqpoint{6.533906in}{5.263904in}}%
\pgfusepath{stroke}%
\end{pgfscope}%
\begin{pgfscope}%
\pgfsetbuttcap%
\pgfsetroundjoin%
\definecolor{currentfill}{rgb}{0.000000,0.000000,0.000000}%
\pgfsetfillcolor{currentfill}%
\pgfsetlinewidth{0.602250pt}%
\definecolor{currentstroke}{rgb}{0.000000,0.000000,0.000000}%
\pgfsetstrokecolor{currentstroke}%
\pgfsetdash{}{0pt}%
\pgfsys@defobject{currentmarker}{\pgfqpoint{0.000000in}{-0.027778in}}{\pgfqpoint{0.000000in}{0.000000in}}{%
\pgfpathmoveto{\pgfqpoint{0.000000in}{0.000000in}}%
\pgfpathlineto{\pgfqpoint{0.000000in}{-0.027778in}}%
\pgfusepath{stroke,fill}%
}%
\begin{pgfscope}%
\pgfsys@transformshift{6.533906in}{0.643904in}%
\pgfsys@useobject{currentmarker}{}%
\end{pgfscope}%
\end{pgfscope}%
\begin{pgfscope}%
\pgfpathrectangle{\pgfqpoint{0.688192in}{0.643904in}}{\pgfqpoint{6.200000in}{4.620000in}}%
\pgfusepath{clip}%
\pgfsetbuttcap%
\pgfsetroundjoin%
\pgfsetlinewidth{0.803000pt}%
\definecolor{currentstroke}{rgb}{0.690196,0.690196,0.690196}%
\pgfsetstrokecolor{currentstroke}%
\pgfsetstrokeopacity{0.200000}%
\pgfsetdash{{2.960000pt}{1.280000pt}}{0.000000pt}%
\pgfpathmoveto{\pgfqpoint{6.711049in}{0.643904in}}%
\pgfpathlineto{\pgfqpoint{6.711049in}{5.263904in}}%
\pgfusepath{stroke}%
\end{pgfscope}%
\begin{pgfscope}%
\pgfsetbuttcap%
\pgfsetroundjoin%
\definecolor{currentfill}{rgb}{0.000000,0.000000,0.000000}%
\pgfsetfillcolor{currentfill}%
\pgfsetlinewidth{0.602250pt}%
\definecolor{currentstroke}{rgb}{0.000000,0.000000,0.000000}%
\pgfsetstrokecolor{currentstroke}%
\pgfsetdash{}{0pt}%
\pgfsys@defobject{currentmarker}{\pgfqpoint{0.000000in}{-0.027778in}}{\pgfqpoint{0.000000in}{0.000000in}}{%
\pgfpathmoveto{\pgfqpoint{0.000000in}{0.000000in}}%
\pgfpathlineto{\pgfqpoint{0.000000in}{-0.027778in}}%
\pgfusepath{stroke,fill}%
}%
\begin{pgfscope}%
\pgfsys@transformshift{6.711049in}{0.643904in}%
\pgfsys@useobject{currentmarker}{}%
\end{pgfscope}%
\end{pgfscope}%
\begin{pgfscope}%
\definecolor{textcolor}{rgb}{0.000000,0.000000,0.000000}%
\pgfsetstrokecolor{textcolor}%
\pgfsetfillcolor{textcolor}%
\pgftext[x=3.788192in,y=0.313349in,,top]{\color{textcolor}{\rmfamily\fontsize{18.000000}{21.600000}\selectfont\catcode`\^=\active\def^{\ifmmode\sp\else\^{}\fi}\catcode`\%=\active\def%{\%}Population per Generation}}%
\end{pgfscope}%
\begin{pgfscope}%
\pgfpathrectangle{\pgfqpoint{0.688192in}{0.643904in}}{\pgfqpoint{6.200000in}{4.620000in}}%
\pgfusepath{clip}%
\pgfsetrectcap%
\pgfsetroundjoin%
\pgfsetlinewidth{0.803000pt}%
\definecolor{currentstroke}{rgb}{0.690196,0.690196,0.690196}%
\pgfsetstrokecolor{currentstroke}%
\pgfsetdash{}{0pt}%
\pgfpathmoveto{\pgfqpoint{0.688192in}{1.118118in}}%
\pgfpathlineto{\pgfqpoint{6.888192in}{1.118118in}}%
\pgfusepath{stroke}%
\end{pgfscope}%
\begin{pgfscope}%
\pgfsetbuttcap%
\pgfsetroundjoin%
\definecolor{currentfill}{rgb}{0.000000,0.000000,0.000000}%
\pgfsetfillcolor{currentfill}%
\pgfsetlinewidth{0.803000pt}%
\definecolor{currentstroke}{rgb}{0.000000,0.000000,0.000000}%
\pgfsetstrokecolor{currentstroke}%
\pgfsetdash{}{0pt}%
\pgfsys@defobject{currentmarker}{\pgfqpoint{-0.048611in}{0.000000in}}{\pgfqpoint{-0.000000in}{0.000000in}}{%
\pgfpathmoveto{\pgfqpoint{-0.000000in}{0.000000in}}%
\pgfpathlineto{\pgfqpoint{-0.048611in}{0.000000in}}%
\pgfusepath{stroke,fill}%
}%
\begin{pgfscope}%
\pgfsys@transformshift{0.688192in}{1.118118in}%
\pgfsys@useobject{currentmarker}{}%
\end{pgfscope}%
\end{pgfscope}%
\begin{pgfscope}%
\definecolor{textcolor}{rgb}{0.000000,0.000000,0.000000}%
\pgfsetstrokecolor{textcolor}%
\pgfsetfillcolor{textcolor}%
\pgftext[x=0.493054in, y=1.048674in, left, base]{\color{textcolor}{\rmfamily\fontsize{14.000000}{16.800000}\selectfont\catcode`\^=\active\def^{\ifmmode\sp\else\^{}\fi}\catcode`\%=\active\def%{\%}$\mathdefault{5}$}}%
\end{pgfscope}%
\begin{pgfscope}%
\pgfpathrectangle{\pgfqpoint{0.688192in}{0.643904in}}{\pgfqpoint{6.200000in}{4.620000in}}%
\pgfusepath{clip}%
\pgfsetrectcap%
\pgfsetroundjoin%
\pgfsetlinewidth{0.803000pt}%
\definecolor{currentstroke}{rgb}{0.690196,0.690196,0.690196}%
\pgfsetstrokecolor{currentstroke}%
\pgfsetdash{}{0pt}%
\pgfpathmoveto{\pgfqpoint{0.688192in}{1.925862in}}%
\pgfpathlineto{\pgfqpoint{6.888192in}{1.925862in}}%
\pgfusepath{stroke}%
\end{pgfscope}%
\begin{pgfscope}%
\pgfsetbuttcap%
\pgfsetroundjoin%
\definecolor{currentfill}{rgb}{0.000000,0.000000,0.000000}%
\pgfsetfillcolor{currentfill}%
\pgfsetlinewidth{0.803000pt}%
\definecolor{currentstroke}{rgb}{0.000000,0.000000,0.000000}%
\pgfsetstrokecolor{currentstroke}%
\pgfsetdash{}{0pt}%
\pgfsys@defobject{currentmarker}{\pgfqpoint{-0.048611in}{0.000000in}}{\pgfqpoint{-0.000000in}{0.000000in}}{%
\pgfpathmoveto{\pgfqpoint{-0.000000in}{0.000000in}}%
\pgfpathlineto{\pgfqpoint{-0.048611in}{0.000000in}}%
\pgfusepath{stroke,fill}%
}%
\begin{pgfscope}%
\pgfsys@transformshift{0.688192in}{1.925862in}%
\pgfsys@useobject{currentmarker}{}%
\end{pgfscope}%
\end{pgfscope}%
\begin{pgfscope}%
\definecolor{textcolor}{rgb}{0.000000,0.000000,0.000000}%
\pgfsetstrokecolor{textcolor}%
\pgfsetfillcolor{textcolor}%
\pgftext[x=0.395138in, y=1.856418in, left, base]{\color{textcolor}{\rmfamily\fontsize{14.000000}{16.800000}\selectfont\catcode`\^=\active\def^{\ifmmode\sp\else\^{}\fi}\catcode`\%=\active\def%{\%}$\mathdefault{10}$}}%
\end{pgfscope}%
\begin{pgfscope}%
\pgfpathrectangle{\pgfqpoint{0.688192in}{0.643904in}}{\pgfqpoint{6.200000in}{4.620000in}}%
\pgfusepath{clip}%
\pgfsetrectcap%
\pgfsetroundjoin%
\pgfsetlinewidth{0.803000pt}%
\definecolor{currentstroke}{rgb}{0.690196,0.690196,0.690196}%
\pgfsetstrokecolor{currentstroke}%
\pgfsetdash{}{0pt}%
\pgfpathmoveto{\pgfqpoint{0.688192in}{2.733606in}}%
\pgfpathlineto{\pgfqpoint{6.888192in}{2.733606in}}%
\pgfusepath{stroke}%
\end{pgfscope}%
\begin{pgfscope}%
\pgfsetbuttcap%
\pgfsetroundjoin%
\definecolor{currentfill}{rgb}{0.000000,0.000000,0.000000}%
\pgfsetfillcolor{currentfill}%
\pgfsetlinewidth{0.803000pt}%
\definecolor{currentstroke}{rgb}{0.000000,0.000000,0.000000}%
\pgfsetstrokecolor{currentstroke}%
\pgfsetdash{}{0pt}%
\pgfsys@defobject{currentmarker}{\pgfqpoint{-0.048611in}{0.000000in}}{\pgfqpoint{-0.000000in}{0.000000in}}{%
\pgfpathmoveto{\pgfqpoint{-0.000000in}{0.000000in}}%
\pgfpathlineto{\pgfqpoint{-0.048611in}{0.000000in}}%
\pgfusepath{stroke,fill}%
}%
\begin{pgfscope}%
\pgfsys@transformshift{0.688192in}{2.733606in}%
\pgfsys@useobject{currentmarker}{}%
\end{pgfscope}%
\end{pgfscope}%
\begin{pgfscope}%
\definecolor{textcolor}{rgb}{0.000000,0.000000,0.000000}%
\pgfsetstrokecolor{textcolor}%
\pgfsetfillcolor{textcolor}%
\pgftext[x=0.395138in, y=2.664162in, left, base]{\color{textcolor}{\rmfamily\fontsize{14.000000}{16.800000}\selectfont\catcode`\^=\active\def^{\ifmmode\sp\else\^{}\fi}\catcode`\%=\active\def%{\%}$\mathdefault{15}$}}%
\end{pgfscope}%
\begin{pgfscope}%
\pgfpathrectangle{\pgfqpoint{0.688192in}{0.643904in}}{\pgfqpoint{6.200000in}{4.620000in}}%
\pgfusepath{clip}%
\pgfsetrectcap%
\pgfsetroundjoin%
\pgfsetlinewidth{0.803000pt}%
\definecolor{currentstroke}{rgb}{0.690196,0.690196,0.690196}%
\pgfsetstrokecolor{currentstroke}%
\pgfsetdash{}{0pt}%
\pgfpathmoveto{\pgfqpoint{0.688192in}{3.541350in}}%
\pgfpathlineto{\pgfqpoint{6.888192in}{3.541350in}}%
\pgfusepath{stroke}%
\end{pgfscope}%
\begin{pgfscope}%
\pgfsetbuttcap%
\pgfsetroundjoin%
\definecolor{currentfill}{rgb}{0.000000,0.000000,0.000000}%
\pgfsetfillcolor{currentfill}%
\pgfsetlinewidth{0.803000pt}%
\definecolor{currentstroke}{rgb}{0.000000,0.000000,0.000000}%
\pgfsetstrokecolor{currentstroke}%
\pgfsetdash{}{0pt}%
\pgfsys@defobject{currentmarker}{\pgfqpoint{-0.048611in}{0.000000in}}{\pgfqpoint{-0.000000in}{0.000000in}}{%
\pgfpathmoveto{\pgfqpoint{-0.000000in}{0.000000in}}%
\pgfpathlineto{\pgfqpoint{-0.048611in}{0.000000in}}%
\pgfusepath{stroke,fill}%
}%
\begin{pgfscope}%
\pgfsys@transformshift{0.688192in}{3.541350in}%
\pgfsys@useobject{currentmarker}{}%
\end{pgfscope}%
\end{pgfscope}%
\begin{pgfscope}%
\definecolor{textcolor}{rgb}{0.000000,0.000000,0.000000}%
\pgfsetstrokecolor{textcolor}%
\pgfsetfillcolor{textcolor}%
\pgftext[x=0.395138in, y=3.471906in, left, base]{\color{textcolor}{\rmfamily\fontsize{14.000000}{16.800000}\selectfont\catcode`\^=\active\def^{\ifmmode\sp\else\^{}\fi}\catcode`\%=\active\def%{\%}$\mathdefault{20}$}}%
\end{pgfscope}%
\begin{pgfscope}%
\pgfpathrectangle{\pgfqpoint{0.688192in}{0.643904in}}{\pgfqpoint{6.200000in}{4.620000in}}%
\pgfusepath{clip}%
\pgfsetrectcap%
\pgfsetroundjoin%
\pgfsetlinewidth{0.803000pt}%
\definecolor{currentstroke}{rgb}{0.690196,0.690196,0.690196}%
\pgfsetstrokecolor{currentstroke}%
\pgfsetdash{}{0pt}%
\pgfpathmoveto{\pgfqpoint{0.688192in}{4.349094in}}%
\pgfpathlineto{\pgfqpoint{6.888192in}{4.349094in}}%
\pgfusepath{stroke}%
\end{pgfscope}%
\begin{pgfscope}%
\pgfsetbuttcap%
\pgfsetroundjoin%
\definecolor{currentfill}{rgb}{0.000000,0.000000,0.000000}%
\pgfsetfillcolor{currentfill}%
\pgfsetlinewidth{0.803000pt}%
\definecolor{currentstroke}{rgb}{0.000000,0.000000,0.000000}%
\pgfsetstrokecolor{currentstroke}%
\pgfsetdash{}{0pt}%
\pgfsys@defobject{currentmarker}{\pgfqpoint{-0.048611in}{0.000000in}}{\pgfqpoint{-0.000000in}{0.000000in}}{%
\pgfpathmoveto{\pgfqpoint{-0.000000in}{0.000000in}}%
\pgfpathlineto{\pgfqpoint{-0.048611in}{0.000000in}}%
\pgfusepath{stroke,fill}%
}%
\begin{pgfscope}%
\pgfsys@transformshift{0.688192in}{4.349094in}%
\pgfsys@useobject{currentmarker}{}%
\end{pgfscope}%
\end{pgfscope}%
\begin{pgfscope}%
\definecolor{textcolor}{rgb}{0.000000,0.000000,0.000000}%
\pgfsetstrokecolor{textcolor}%
\pgfsetfillcolor{textcolor}%
\pgftext[x=0.395138in, y=4.279650in, left, base]{\color{textcolor}{\rmfamily\fontsize{14.000000}{16.800000}\selectfont\catcode`\^=\active\def^{\ifmmode\sp\else\^{}\fi}\catcode`\%=\active\def%{\%}$\mathdefault{25}$}}%
\end{pgfscope}%
\begin{pgfscope}%
\pgfpathrectangle{\pgfqpoint{0.688192in}{0.643904in}}{\pgfqpoint{6.200000in}{4.620000in}}%
\pgfusepath{clip}%
\pgfsetrectcap%
\pgfsetroundjoin%
\pgfsetlinewidth{0.803000pt}%
\definecolor{currentstroke}{rgb}{0.690196,0.690196,0.690196}%
\pgfsetstrokecolor{currentstroke}%
\pgfsetdash{}{0pt}%
\pgfpathmoveto{\pgfqpoint{0.688192in}{5.156838in}}%
\pgfpathlineto{\pgfqpoint{6.888192in}{5.156838in}}%
\pgfusepath{stroke}%
\end{pgfscope}%
\begin{pgfscope}%
\pgfsetbuttcap%
\pgfsetroundjoin%
\definecolor{currentfill}{rgb}{0.000000,0.000000,0.000000}%
\pgfsetfillcolor{currentfill}%
\pgfsetlinewidth{0.803000pt}%
\definecolor{currentstroke}{rgb}{0.000000,0.000000,0.000000}%
\pgfsetstrokecolor{currentstroke}%
\pgfsetdash{}{0pt}%
\pgfsys@defobject{currentmarker}{\pgfqpoint{-0.048611in}{0.000000in}}{\pgfqpoint{-0.000000in}{0.000000in}}{%
\pgfpathmoveto{\pgfqpoint{-0.000000in}{0.000000in}}%
\pgfpathlineto{\pgfqpoint{-0.048611in}{0.000000in}}%
\pgfusepath{stroke,fill}%
}%
\begin{pgfscope}%
\pgfsys@transformshift{0.688192in}{5.156838in}%
\pgfsys@useobject{currentmarker}{}%
\end{pgfscope}%
\end{pgfscope}%
\begin{pgfscope}%
\definecolor{textcolor}{rgb}{0.000000,0.000000,0.000000}%
\pgfsetstrokecolor{textcolor}%
\pgfsetfillcolor{textcolor}%
\pgftext[x=0.395138in, y=5.087394in, left, base]{\color{textcolor}{\rmfamily\fontsize{14.000000}{16.800000}\selectfont\catcode`\^=\active\def^{\ifmmode\sp\else\^{}\fi}\catcode`\%=\active\def%{\%}$\mathdefault{30}$}}%
\end{pgfscope}%
\begin{pgfscope}%
\pgfpathrectangle{\pgfqpoint{0.688192in}{0.643904in}}{\pgfqpoint{6.200000in}{4.620000in}}%
\pgfusepath{clip}%
\pgfsetbuttcap%
\pgfsetroundjoin%
\pgfsetlinewidth{0.803000pt}%
\definecolor{currentstroke}{rgb}{0.690196,0.690196,0.690196}%
\pgfsetstrokecolor{currentstroke}%
\pgfsetstrokeopacity{0.200000}%
\pgfsetdash{{2.960000pt}{1.280000pt}}{0.000000pt}%
\pgfpathmoveto{\pgfqpoint{0.688192in}{0.795020in}}%
\pgfpathlineto{\pgfqpoint{6.888192in}{0.795020in}}%
\pgfusepath{stroke}%
\end{pgfscope}%
\begin{pgfscope}%
\pgfsetbuttcap%
\pgfsetroundjoin%
\definecolor{currentfill}{rgb}{0.000000,0.000000,0.000000}%
\pgfsetfillcolor{currentfill}%
\pgfsetlinewidth{0.602250pt}%
\definecolor{currentstroke}{rgb}{0.000000,0.000000,0.000000}%
\pgfsetstrokecolor{currentstroke}%
\pgfsetdash{}{0pt}%
\pgfsys@defobject{currentmarker}{\pgfqpoint{-0.027778in}{0.000000in}}{\pgfqpoint{-0.000000in}{0.000000in}}{%
\pgfpathmoveto{\pgfqpoint{-0.000000in}{0.000000in}}%
\pgfpathlineto{\pgfqpoint{-0.027778in}{0.000000in}}%
\pgfusepath{stroke,fill}%
}%
\begin{pgfscope}%
\pgfsys@transformshift{0.688192in}{0.795020in}%
\pgfsys@useobject{currentmarker}{}%
\end{pgfscope}%
\end{pgfscope}%
\begin{pgfscope}%
\pgfpathrectangle{\pgfqpoint{0.688192in}{0.643904in}}{\pgfqpoint{6.200000in}{4.620000in}}%
\pgfusepath{clip}%
\pgfsetbuttcap%
\pgfsetroundjoin%
\pgfsetlinewidth{0.803000pt}%
\definecolor{currentstroke}{rgb}{0.690196,0.690196,0.690196}%
\pgfsetstrokecolor{currentstroke}%
\pgfsetstrokeopacity{0.200000}%
\pgfsetdash{{2.960000pt}{1.280000pt}}{0.000000pt}%
\pgfpathmoveto{\pgfqpoint{0.688192in}{0.956569in}}%
\pgfpathlineto{\pgfqpoint{6.888192in}{0.956569in}}%
\pgfusepath{stroke}%
\end{pgfscope}%
\begin{pgfscope}%
\pgfsetbuttcap%
\pgfsetroundjoin%
\definecolor{currentfill}{rgb}{0.000000,0.000000,0.000000}%
\pgfsetfillcolor{currentfill}%
\pgfsetlinewidth{0.602250pt}%
\definecolor{currentstroke}{rgb}{0.000000,0.000000,0.000000}%
\pgfsetstrokecolor{currentstroke}%
\pgfsetdash{}{0pt}%
\pgfsys@defobject{currentmarker}{\pgfqpoint{-0.027778in}{0.000000in}}{\pgfqpoint{-0.000000in}{0.000000in}}{%
\pgfpathmoveto{\pgfqpoint{-0.000000in}{0.000000in}}%
\pgfpathlineto{\pgfqpoint{-0.027778in}{0.000000in}}%
\pgfusepath{stroke,fill}%
}%
\begin{pgfscope}%
\pgfsys@transformshift{0.688192in}{0.956569in}%
\pgfsys@useobject{currentmarker}{}%
\end{pgfscope}%
\end{pgfscope}%
\begin{pgfscope}%
\pgfpathrectangle{\pgfqpoint{0.688192in}{0.643904in}}{\pgfqpoint{6.200000in}{4.620000in}}%
\pgfusepath{clip}%
\pgfsetbuttcap%
\pgfsetroundjoin%
\pgfsetlinewidth{0.803000pt}%
\definecolor{currentstroke}{rgb}{0.690196,0.690196,0.690196}%
\pgfsetstrokecolor{currentstroke}%
\pgfsetstrokeopacity{0.200000}%
\pgfsetdash{{2.960000pt}{1.280000pt}}{0.000000pt}%
\pgfpathmoveto{\pgfqpoint{0.688192in}{1.279667in}}%
\pgfpathlineto{\pgfqpoint{6.888192in}{1.279667in}}%
\pgfusepath{stroke}%
\end{pgfscope}%
\begin{pgfscope}%
\pgfsetbuttcap%
\pgfsetroundjoin%
\definecolor{currentfill}{rgb}{0.000000,0.000000,0.000000}%
\pgfsetfillcolor{currentfill}%
\pgfsetlinewidth{0.602250pt}%
\definecolor{currentstroke}{rgb}{0.000000,0.000000,0.000000}%
\pgfsetstrokecolor{currentstroke}%
\pgfsetdash{}{0pt}%
\pgfsys@defobject{currentmarker}{\pgfqpoint{-0.027778in}{0.000000in}}{\pgfqpoint{-0.000000in}{0.000000in}}{%
\pgfpathmoveto{\pgfqpoint{-0.000000in}{0.000000in}}%
\pgfpathlineto{\pgfqpoint{-0.027778in}{0.000000in}}%
\pgfusepath{stroke,fill}%
}%
\begin{pgfscope}%
\pgfsys@transformshift{0.688192in}{1.279667in}%
\pgfsys@useobject{currentmarker}{}%
\end{pgfscope}%
\end{pgfscope}%
\begin{pgfscope}%
\pgfpathrectangle{\pgfqpoint{0.688192in}{0.643904in}}{\pgfqpoint{6.200000in}{4.620000in}}%
\pgfusepath{clip}%
\pgfsetbuttcap%
\pgfsetroundjoin%
\pgfsetlinewidth{0.803000pt}%
\definecolor{currentstroke}{rgb}{0.690196,0.690196,0.690196}%
\pgfsetstrokecolor{currentstroke}%
\pgfsetstrokeopacity{0.200000}%
\pgfsetdash{{2.960000pt}{1.280000pt}}{0.000000pt}%
\pgfpathmoveto{\pgfqpoint{0.688192in}{1.441215in}}%
\pgfpathlineto{\pgfqpoint{6.888192in}{1.441215in}}%
\pgfusepath{stroke}%
\end{pgfscope}%
\begin{pgfscope}%
\pgfsetbuttcap%
\pgfsetroundjoin%
\definecolor{currentfill}{rgb}{0.000000,0.000000,0.000000}%
\pgfsetfillcolor{currentfill}%
\pgfsetlinewidth{0.602250pt}%
\definecolor{currentstroke}{rgb}{0.000000,0.000000,0.000000}%
\pgfsetstrokecolor{currentstroke}%
\pgfsetdash{}{0pt}%
\pgfsys@defobject{currentmarker}{\pgfqpoint{-0.027778in}{0.000000in}}{\pgfqpoint{-0.000000in}{0.000000in}}{%
\pgfpathmoveto{\pgfqpoint{-0.000000in}{0.000000in}}%
\pgfpathlineto{\pgfqpoint{-0.027778in}{0.000000in}}%
\pgfusepath{stroke,fill}%
}%
\begin{pgfscope}%
\pgfsys@transformshift{0.688192in}{1.441215in}%
\pgfsys@useobject{currentmarker}{}%
\end{pgfscope}%
\end{pgfscope}%
\begin{pgfscope}%
\pgfpathrectangle{\pgfqpoint{0.688192in}{0.643904in}}{\pgfqpoint{6.200000in}{4.620000in}}%
\pgfusepath{clip}%
\pgfsetbuttcap%
\pgfsetroundjoin%
\pgfsetlinewidth{0.803000pt}%
\definecolor{currentstroke}{rgb}{0.690196,0.690196,0.690196}%
\pgfsetstrokecolor{currentstroke}%
\pgfsetstrokeopacity{0.200000}%
\pgfsetdash{{2.960000pt}{1.280000pt}}{0.000000pt}%
\pgfpathmoveto{\pgfqpoint{0.688192in}{1.602764in}}%
\pgfpathlineto{\pgfqpoint{6.888192in}{1.602764in}}%
\pgfusepath{stroke}%
\end{pgfscope}%
\begin{pgfscope}%
\pgfsetbuttcap%
\pgfsetroundjoin%
\definecolor{currentfill}{rgb}{0.000000,0.000000,0.000000}%
\pgfsetfillcolor{currentfill}%
\pgfsetlinewidth{0.602250pt}%
\definecolor{currentstroke}{rgb}{0.000000,0.000000,0.000000}%
\pgfsetstrokecolor{currentstroke}%
\pgfsetdash{}{0pt}%
\pgfsys@defobject{currentmarker}{\pgfqpoint{-0.027778in}{0.000000in}}{\pgfqpoint{-0.000000in}{0.000000in}}{%
\pgfpathmoveto{\pgfqpoint{-0.000000in}{0.000000in}}%
\pgfpathlineto{\pgfqpoint{-0.027778in}{0.000000in}}%
\pgfusepath{stroke,fill}%
}%
\begin{pgfscope}%
\pgfsys@transformshift{0.688192in}{1.602764in}%
\pgfsys@useobject{currentmarker}{}%
\end{pgfscope}%
\end{pgfscope}%
\begin{pgfscope}%
\pgfpathrectangle{\pgfqpoint{0.688192in}{0.643904in}}{\pgfqpoint{6.200000in}{4.620000in}}%
\pgfusepath{clip}%
\pgfsetbuttcap%
\pgfsetroundjoin%
\pgfsetlinewidth{0.803000pt}%
\definecolor{currentstroke}{rgb}{0.690196,0.690196,0.690196}%
\pgfsetstrokecolor{currentstroke}%
\pgfsetstrokeopacity{0.200000}%
\pgfsetdash{{2.960000pt}{1.280000pt}}{0.000000pt}%
\pgfpathmoveto{\pgfqpoint{0.688192in}{1.764313in}}%
\pgfpathlineto{\pgfqpoint{6.888192in}{1.764313in}}%
\pgfusepath{stroke}%
\end{pgfscope}%
\begin{pgfscope}%
\pgfsetbuttcap%
\pgfsetroundjoin%
\definecolor{currentfill}{rgb}{0.000000,0.000000,0.000000}%
\pgfsetfillcolor{currentfill}%
\pgfsetlinewidth{0.602250pt}%
\definecolor{currentstroke}{rgb}{0.000000,0.000000,0.000000}%
\pgfsetstrokecolor{currentstroke}%
\pgfsetdash{}{0pt}%
\pgfsys@defobject{currentmarker}{\pgfqpoint{-0.027778in}{0.000000in}}{\pgfqpoint{-0.000000in}{0.000000in}}{%
\pgfpathmoveto{\pgfqpoint{-0.000000in}{0.000000in}}%
\pgfpathlineto{\pgfqpoint{-0.027778in}{0.000000in}}%
\pgfusepath{stroke,fill}%
}%
\begin{pgfscope}%
\pgfsys@transformshift{0.688192in}{1.764313in}%
\pgfsys@useobject{currentmarker}{}%
\end{pgfscope}%
\end{pgfscope}%
\begin{pgfscope}%
\pgfpathrectangle{\pgfqpoint{0.688192in}{0.643904in}}{\pgfqpoint{6.200000in}{4.620000in}}%
\pgfusepath{clip}%
\pgfsetbuttcap%
\pgfsetroundjoin%
\pgfsetlinewidth{0.803000pt}%
\definecolor{currentstroke}{rgb}{0.690196,0.690196,0.690196}%
\pgfsetstrokecolor{currentstroke}%
\pgfsetstrokeopacity{0.200000}%
\pgfsetdash{{2.960000pt}{1.280000pt}}{0.000000pt}%
\pgfpathmoveto{\pgfqpoint{0.688192in}{2.087411in}}%
\pgfpathlineto{\pgfqpoint{6.888192in}{2.087411in}}%
\pgfusepath{stroke}%
\end{pgfscope}%
\begin{pgfscope}%
\pgfsetbuttcap%
\pgfsetroundjoin%
\definecolor{currentfill}{rgb}{0.000000,0.000000,0.000000}%
\pgfsetfillcolor{currentfill}%
\pgfsetlinewidth{0.602250pt}%
\definecolor{currentstroke}{rgb}{0.000000,0.000000,0.000000}%
\pgfsetstrokecolor{currentstroke}%
\pgfsetdash{}{0pt}%
\pgfsys@defobject{currentmarker}{\pgfqpoint{-0.027778in}{0.000000in}}{\pgfqpoint{-0.000000in}{0.000000in}}{%
\pgfpathmoveto{\pgfqpoint{-0.000000in}{0.000000in}}%
\pgfpathlineto{\pgfqpoint{-0.027778in}{0.000000in}}%
\pgfusepath{stroke,fill}%
}%
\begin{pgfscope}%
\pgfsys@transformshift{0.688192in}{2.087411in}%
\pgfsys@useobject{currentmarker}{}%
\end{pgfscope}%
\end{pgfscope}%
\begin{pgfscope}%
\pgfpathrectangle{\pgfqpoint{0.688192in}{0.643904in}}{\pgfqpoint{6.200000in}{4.620000in}}%
\pgfusepath{clip}%
\pgfsetbuttcap%
\pgfsetroundjoin%
\pgfsetlinewidth{0.803000pt}%
\definecolor{currentstroke}{rgb}{0.690196,0.690196,0.690196}%
\pgfsetstrokecolor{currentstroke}%
\pgfsetstrokeopacity{0.200000}%
\pgfsetdash{{2.960000pt}{1.280000pt}}{0.000000pt}%
\pgfpathmoveto{\pgfqpoint{0.688192in}{2.248960in}}%
\pgfpathlineto{\pgfqpoint{6.888192in}{2.248960in}}%
\pgfusepath{stroke}%
\end{pgfscope}%
\begin{pgfscope}%
\pgfsetbuttcap%
\pgfsetroundjoin%
\definecolor{currentfill}{rgb}{0.000000,0.000000,0.000000}%
\pgfsetfillcolor{currentfill}%
\pgfsetlinewidth{0.602250pt}%
\definecolor{currentstroke}{rgb}{0.000000,0.000000,0.000000}%
\pgfsetstrokecolor{currentstroke}%
\pgfsetdash{}{0pt}%
\pgfsys@defobject{currentmarker}{\pgfqpoint{-0.027778in}{0.000000in}}{\pgfqpoint{-0.000000in}{0.000000in}}{%
\pgfpathmoveto{\pgfqpoint{-0.000000in}{0.000000in}}%
\pgfpathlineto{\pgfqpoint{-0.027778in}{0.000000in}}%
\pgfusepath{stroke,fill}%
}%
\begin{pgfscope}%
\pgfsys@transformshift{0.688192in}{2.248960in}%
\pgfsys@useobject{currentmarker}{}%
\end{pgfscope}%
\end{pgfscope}%
\begin{pgfscope}%
\pgfpathrectangle{\pgfqpoint{0.688192in}{0.643904in}}{\pgfqpoint{6.200000in}{4.620000in}}%
\pgfusepath{clip}%
\pgfsetbuttcap%
\pgfsetroundjoin%
\pgfsetlinewidth{0.803000pt}%
\definecolor{currentstroke}{rgb}{0.690196,0.690196,0.690196}%
\pgfsetstrokecolor{currentstroke}%
\pgfsetstrokeopacity{0.200000}%
\pgfsetdash{{2.960000pt}{1.280000pt}}{0.000000pt}%
\pgfpathmoveto{\pgfqpoint{0.688192in}{2.410508in}}%
\pgfpathlineto{\pgfqpoint{6.888192in}{2.410508in}}%
\pgfusepath{stroke}%
\end{pgfscope}%
\begin{pgfscope}%
\pgfsetbuttcap%
\pgfsetroundjoin%
\definecolor{currentfill}{rgb}{0.000000,0.000000,0.000000}%
\pgfsetfillcolor{currentfill}%
\pgfsetlinewidth{0.602250pt}%
\definecolor{currentstroke}{rgb}{0.000000,0.000000,0.000000}%
\pgfsetstrokecolor{currentstroke}%
\pgfsetdash{}{0pt}%
\pgfsys@defobject{currentmarker}{\pgfqpoint{-0.027778in}{0.000000in}}{\pgfqpoint{-0.000000in}{0.000000in}}{%
\pgfpathmoveto{\pgfqpoint{-0.000000in}{0.000000in}}%
\pgfpathlineto{\pgfqpoint{-0.027778in}{0.000000in}}%
\pgfusepath{stroke,fill}%
}%
\begin{pgfscope}%
\pgfsys@transformshift{0.688192in}{2.410508in}%
\pgfsys@useobject{currentmarker}{}%
\end{pgfscope}%
\end{pgfscope}%
\begin{pgfscope}%
\pgfpathrectangle{\pgfqpoint{0.688192in}{0.643904in}}{\pgfqpoint{6.200000in}{4.620000in}}%
\pgfusepath{clip}%
\pgfsetbuttcap%
\pgfsetroundjoin%
\pgfsetlinewidth{0.803000pt}%
\definecolor{currentstroke}{rgb}{0.690196,0.690196,0.690196}%
\pgfsetstrokecolor{currentstroke}%
\pgfsetstrokeopacity{0.200000}%
\pgfsetdash{{2.960000pt}{1.280000pt}}{0.000000pt}%
\pgfpathmoveto{\pgfqpoint{0.688192in}{2.572057in}}%
\pgfpathlineto{\pgfqpoint{6.888192in}{2.572057in}}%
\pgfusepath{stroke}%
\end{pgfscope}%
\begin{pgfscope}%
\pgfsetbuttcap%
\pgfsetroundjoin%
\definecolor{currentfill}{rgb}{0.000000,0.000000,0.000000}%
\pgfsetfillcolor{currentfill}%
\pgfsetlinewidth{0.602250pt}%
\definecolor{currentstroke}{rgb}{0.000000,0.000000,0.000000}%
\pgfsetstrokecolor{currentstroke}%
\pgfsetdash{}{0pt}%
\pgfsys@defobject{currentmarker}{\pgfqpoint{-0.027778in}{0.000000in}}{\pgfqpoint{-0.000000in}{0.000000in}}{%
\pgfpathmoveto{\pgfqpoint{-0.000000in}{0.000000in}}%
\pgfpathlineto{\pgfqpoint{-0.027778in}{0.000000in}}%
\pgfusepath{stroke,fill}%
}%
\begin{pgfscope}%
\pgfsys@transformshift{0.688192in}{2.572057in}%
\pgfsys@useobject{currentmarker}{}%
\end{pgfscope}%
\end{pgfscope}%
\begin{pgfscope}%
\pgfpathrectangle{\pgfqpoint{0.688192in}{0.643904in}}{\pgfqpoint{6.200000in}{4.620000in}}%
\pgfusepath{clip}%
\pgfsetbuttcap%
\pgfsetroundjoin%
\pgfsetlinewidth{0.803000pt}%
\definecolor{currentstroke}{rgb}{0.690196,0.690196,0.690196}%
\pgfsetstrokecolor{currentstroke}%
\pgfsetstrokeopacity{0.200000}%
\pgfsetdash{{2.960000pt}{1.280000pt}}{0.000000pt}%
\pgfpathmoveto{\pgfqpoint{0.688192in}{2.895155in}}%
\pgfpathlineto{\pgfqpoint{6.888192in}{2.895155in}}%
\pgfusepath{stroke}%
\end{pgfscope}%
\begin{pgfscope}%
\pgfsetbuttcap%
\pgfsetroundjoin%
\definecolor{currentfill}{rgb}{0.000000,0.000000,0.000000}%
\pgfsetfillcolor{currentfill}%
\pgfsetlinewidth{0.602250pt}%
\definecolor{currentstroke}{rgb}{0.000000,0.000000,0.000000}%
\pgfsetstrokecolor{currentstroke}%
\pgfsetdash{}{0pt}%
\pgfsys@defobject{currentmarker}{\pgfqpoint{-0.027778in}{0.000000in}}{\pgfqpoint{-0.000000in}{0.000000in}}{%
\pgfpathmoveto{\pgfqpoint{-0.000000in}{0.000000in}}%
\pgfpathlineto{\pgfqpoint{-0.027778in}{0.000000in}}%
\pgfusepath{stroke,fill}%
}%
\begin{pgfscope}%
\pgfsys@transformshift{0.688192in}{2.895155in}%
\pgfsys@useobject{currentmarker}{}%
\end{pgfscope}%
\end{pgfscope}%
\begin{pgfscope}%
\pgfpathrectangle{\pgfqpoint{0.688192in}{0.643904in}}{\pgfqpoint{6.200000in}{4.620000in}}%
\pgfusepath{clip}%
\pgfsetbuttcap%
\pgfsetroundjoin%
\pgfsetlinewidth{0.803000pt}%
\definecolor{currentstroke}{rgb}{0.690196,0.690196,0.690196}%
\pgfsetstrokecolor{currentstroke}%
\pgfsetstrokeopacity{0.200000}%
\pgfsetdash{{2.960000pt}{1.280000pt}}{0.000000pt}%
\pgfpathmoveto{\pgfqpoint{0.688192in}{3.056704in}}%
\pgfpathlineto{\pgfqpoint{6.888192in}{3.056704in}}%
\pgfusepath{stroke}%
\end{pgfscope}%
\begin{pgfscope}%
\pgfsetbuttcap%
\pgfsetroundjoin%
\definecolor{currentfill}{rgb}{0.000000,0.000000,0.000000}%
\pgfsetfillcolor{currentfill}%
\pgfsetlinewidth{0.602250pt}%
\definecolor{currentstroke}{rgb}{0.000000,0.000000,0.000000}%
\pgfsetstrokecolor{currentstroke}%
\pgfsetdash{}{0pt}%
\pgfsys@defobject{currentmarker}{\pgfqpoint{-0.027778in}{0.000000in}}{\pgfqpoint{-0.000000in}{0.000000in}}{%
\pgfpathmoveto{\pgfqpoint{-0.000000in}{0.000000in}}%
\pgfpathlineto{\pgfqpoint{-0.027778in}{0.000000in}}%
\pgfusepath{stroke,fill}%
}%
\begin{pgfscope}%
\pgfsys@transformshift{0.688192in}{3.056704in}%
\pgfsys@useobject{currentmarker}{}%
\end{pgfscope}%
\end{pgfscope}%
\begin{pgfscope}%
\pgfpathrectangle{\pgfqpoint{0.688192in}{0.643904in}}{\pgfqpoint{6.200000in}{4.620000in}}%
\pgfusepath{clip}%
\pgfsetbuttcap%
\pgfsetroundjoin%
\pgfsetlinewidth{0.803000pt}%
\definecolor{currentstroke}{rgb}{0.690196,0.690196,0.690196}%
\pgfsetstrokecolor{currentstroke}%
\pgfsetstrokeopacity{0.200000}%
\pgfsetdash{{2.960000pt}{1.280000pt}}{0.000000pt}%
\pgfpathmoveto{\pgfqpoint{0.688192in}{3.218253in}}%
\pgfpathlineto{\pgfqpoint{6.888192in}{3.218253in}}%
\pgfusepath{stroke}%
\end{pgfscope}%
\begin{pgfscope}%
\pgfsetbuttcap%
\pgfsetroundjoin%
\definecolor{currentfill}{rgb}{0.000000,0.000000,0.000000}%
\pgfsetfillcolor{currentfill}%
\pgfsetlinewidth{0.602250pt}%
\definecolor{currentstroke}{rgb}{0.000000,0.000000,0.000000}%
\pgfsetstrokecolor{currentstroke}%
\pgfsetdash{}{0pt}%
\pgfsys@defobject{currentmarker}{\pgfqpoint{-0.027778in}{0.000000in}}{\pgfqpoint{-0.000000in}{0.000000in}}{%
\pgfpathmoveto{\pgfqpoint{-0.000000in}{0.000000in}}%
\pgfpathlineto{\pgfqpoint{-0.027778in}{0.000000in}}%
\pgfusepath{stroke,fill}%
}%
\begin{pgfscope}%
\pgfsys@transformshift{0.688192in}{3.218253in}%
\pgfsys@useobject{currentmarker}{}%
\end{pgfscope}%
\end{pgfscope}%
\begin{pgfscope}%
\pgfpathrectangle{\pgfqpoint{0.688192in}{0.643904in}}{\pgfqpoint{6.200000in}{4.620000in}}%
\pgfusepath{clip}%
\pgfsetbuttcap%
\pgfsetroundjoin%
\pgfsetlinewidth{0.803000pt}%
\definecolor{currentstroke}{rgb}{0.690196,0.690196,0.690196}%
\pgfsetstrokecolor{currentstroke}%
\pgfsetstrokeopacity{0.200000}%
\pgfsetdash{{2.960000pt}{1.280000pt}}{0.000000pt}%
\pgfpathmoveto{\pgfqpoint{0.688192in}{3.379801in}}%
\pgfpathlineto{\pgfqpoint{6.888192in}{3.379801in}}%
\pgfusepath{stroke}%
\end{pgfscope}%
\begin{pgfscope}%
\pgfsetbuttcap%
\pgfsetroundjoin%
\definecolor{currentfill}{rgb}{0.000000,0.000000,0.000000}%
\pgfsetfillcolor{currentfill}%
\pgfsetlinewidth{0.602250pt}%
\definecolor{currentstroke}{rgb}{0.000000,0.000000,0.000000}%
\pgfsetstrokecolor{currentstroke}%
\pgfsetdash{}{0pt}%
\pgfsys@defobject{currentmarker}{\pgfqpoint{-0.027778in}{0.000000in}}{\pgfqpoint{-0.000000in}{0.000000in}}{%
\pgfpathmoveto{\pgfqpoint{-0.000000in}{0.000000in}}%
\pgfpathlineto{\pgfqpoint{-0.027778in}{0.000000in}}%
\pgfusepath{stroke,fill}%
}%
\begin{pgfscope}%
\pgfsys@transformshift{0.688192in}{3.379801in}%
\pgfsys@useobject{currentmarker}{}%
\end{pgfscope}%
\end{pgfscope}%
\begin{pgfscope}%
\pgfpathrectangle{\pgfqpoint{0.688192in}{0.643904in}}{\pgfqpoint{6.200000in}{4.620000in}}%
\pgfusepath{clip}%
\pgfsetbuttcap%
\pgfsetroundjoin%
\pgfsetlinewidth{0.803000pt}%
\definecolor{currentstroke}{rgb}{0.690196,0.690196,0.690196}%
\pgfsetstrokecolor{currentstroke}%
\pgfsetstrokeopacity{0.200000}%
\pgfsetdash{{2.960000pt}{1.280000pt}}{0.000000pt}%
\pgfpathmoveto{\pgfqpoint{0.688192in}{3.702899in}}%
\pgfpathlineto{\pgfqpoint{6.888192in}{3.702899in}}%
\pgfusepath{stroke}%
\end{pgfscope}%
\begin{pgfscope}%
\pgfsetbuttcap%
\pgfsetroundjoin%
\definecolor{currentfill}{rgb}{0.000000,0.000000,0.000000}%
\pgfsetfillcolor{currentfill}%
\pgfsetlinewidth{0.602250pt}%
\definecolor{currentstroke}{rgb}{0.000000,0.000000,0.000000}%
\pgfsetstrokecolor{currentstroke}%
\pgfsetdash{}{0pt}%
\pgfsys@defobject{currentmarker}{\pgfqpoint{-0.027778in}{0.000000in}}{\pgfqpoint{-0.000000in}{0.000000in}}{%
\pgfpathmoveto{\pgfqpoint{-0.000000in}{0.000000in}}%
\pgfpathlineto{\pgfqpoint{-0.027778in}{0.000000in}}%
\pgfusepath{stroke,fill}%
}%
\begin{pgfscope}%
\pgfsys@transformshift{0.688192in}{3.702899in}%
\pgfsys@useobject{currentmarker}{}%
\end{pgfscope}%
\end{pgfscope}%
\begin{pgfscope}%
\pgfpathrectangle{\pgfqpoint{0.688192in}{0.643904in}}{\pgfqpoint{6.200000in}{4.620000in}}%
\pgfusepath{clip}%
\pgfsetbuttcap%
\pgfsetroundjoin%
\pgfsetlinewidth{0.803000pt}%
\definecolor{currentstroke}{rgb}{0.690196,0.690196,0.690196}%
\pgfsetstrokecolor{currentstroke}%
\pgfsetstrokeopacity{0.200000}%
\pgfsetdash{{2.960000pt}{1.280000pt}}{0.000000pt}%
\pgfpathmoveto{\pgfqpoint{0.688192in}{3.864448in}}%
\pgfpathlineto{\pgfqpoint{6.888192in}{3.864448in}}%
\pgfusepath{stroke}%
\end{pgfscope}%
\begin{pgfscope}%
\pgfsetbuttcap%
\pgfsetroundjoin%
\definecolor{currentfill}{rgb}{0.000000,0.000000,0.000000}%
\pgfsetfillcolor{currentfill}%
\pgfsetlinewidth{0.602250pt}%
\definecolor{currentstroke}{rgb}{0.000000,0.000000,0.000000}%
\pgfsetstrokecolor{currentstroke}%
\pgfsetdash{}{0pt}%
\pgfsys@defobject{currentmarker}{\pgfqpoint{-0.027778in}{0.000000in}}{\pgfqpoint{-0.000000in}{0.000000in}}{%
\pgfpathmoveto{\pgfqpoint{-0.000000in}{0.000000in}}%
\pgfpathlineto{\pgfqpoint{-0.027778in}{0.000000in}}%
\pgfusepath{stroke,fill}%
}%
\begin{pgfscope}%
\pgfsys@transformshift{0.688192in}{3.864448in}%
\pgfsys@useobject{currentmarker}{}%
\end{pgfscope}%
\end{pgfscope}%
\begin{pgfscope}%
\pgfpathrectangle{\pgfqpoint{0.688192in}{0.643904in}}{\pgfqpoint{6.200000in}{4.620000in}}%
\pgfusepath{clip}%
\pgfsetbuttcap%
\pgfsetroundjoin%
\pgfsetlinewidth{0.803000pt}%
\definecolor{currentstroke}{rgb}{0.690196,0.690196,0.690196}%
\pgfsetstrokecolor{currentstroke}%
\pgfsetstrokeopacity{0.200000}%
\pgfsetdash{{2.960000pt}{1.280000pt}}{0.000000pt}%
\pgfpathmoveto{\pgfqpoint{0.688192in}{4.025997in}}%
\pgfpathlineto{\pgfqpoint{6.888192in}{4.025997in}}%
\pgfusepath{stroke}%
\end{pgfscope}%
\begin{pgfscope}%
\pgfsetbuttcap%
\pgfsetroundjoin%
\definecolor{currentfill}{rgb}{0.000000,0.000000,0.000000}%
\pgfsetfillcolor{currentfill}%
\pgfsetlinewidth{0.602250pt}%
\definecolor{currentstroke}{rgb}{0.000000,0.000000,0.000000}%
\pgfsetstrokecolor{currentstroke}%
\pgfsetdash{}{0pt}%
\pgfsys@defobject{currentmarker}{\pgfqpoint{-0.027778in}{0.000000in}}{\pgfqpoint{-0.000000in}{0.000000in}}{%
\pgfpathmoveto{\pgfqpoint{-0.000000in}{0.000000in}}%
\pgfpathlineto{\pgfqpoint{-0.027778in}{0.000000in}}%
\pgfusepath{stroke,fill}%
}%
\begin{pgfscope}%
\pgfsys@transformshift{0.688192in}{4.025997in}%
\pgfsys@useobject{currentmarker}{}%
\end{pgfscope}%
\end{pgfscope}%
\begin{pgfscope}%
\pgfpathrectangle{\pgfqpoint{0.688192in}{0.643904in}}{\pgfqpoint{6.200000in}{4.620000in}}%
\pgfusepath{clip}%
\pgfsetbuttcap%
\pgfsetroundjoin%
\pgfsetlinewidth{0.803000pt}%
\definecolor{currentstroke}{rgb}{0.690196,0.690196,0.690196}%
\pgfsetstrokecolor{currentstroke}%
\pgfsetstrokeopacity{0.200000}%
\pgfsetdash{{2.960000pt}{1.280000pt}}{0.000000pt}%
\pgfpathmoveto{\pgfqpoint{0.688192in}{4.187545in}}%
\pgfpathlineto{\pgfqpoint{6.888192in}{4.187545in}}%
\pgfusepath{stroke}%
\end{pgfscope}%
\begin{pgfscope}%
\pgfsetbuttcap%
\pgfsetroundjoin%
\definecolor{currentfill}{rgb}{0.000000,0.000000,0.000000}%
\pgfsetfillcolor{currentfill}%
\pgfsetlinewidth{0.602250pt}%
\definecolor{currentstroke}{rgb}{0.000000,0.000000,0.000000}%
\pgfsetstrokecolor{currentstroke}%
\pgfsetdash{}{0pt}%
\pgfsys@defobject{currentmarker}{\pgfqpoint{-0.027778in}{0.000000in}}{\pgfqpoint{-0.000000in}{0.000000in}}{%
\pgfpathmoveto{\pgfqpoint{-0.000000in}{0.000000in}}%
\pgfpathlineto{\pgfqpoint{-0.027778in}{0.000000in}}%
\pgfusepath{stroke,fill}%
}%
\begin{pgfscope}%
\pgfsys@transformshift{0.688192in}{4.187545in}%
\pgfsys@useobject{currentmarker}{}%
\end{pgfscope}%
\end{pgfscope}%
\begin{pgfscope}%
\pgfpathrectangle{\pgfqpoint{0.688192in}{0.643904in}}{\pgfqpoint{6.200000in}{4.620000in}}%
\pgfusepath{clip}%
\pgfsetbuttcap%
\pgfsetroundjoin%
\pgfsetlinewidth{0.803000pt}%
\definecolor{currentstroke}{rgb}{0.690196,0.690196,0.690196}%
\pgfsetstrokecolor{currentstroke}%
\pgfsetstrokeopacity{0.200000}%
\pgfsetdash{{2.960000pt}{1.280000pt}}{0.000000pt}%
\pgfpathmoveto{\pgfqpoint{0.688192in}{4.510643in}}%
\pgfpathlineto{\pgfqpoint{6.888192in}{4.510643in}}%
\pgfusepath{stroke}%
\end{pgfscope}%
\begin{pgfscope}%
\pgfsetbuttcap%
\pgfsetroundjoin%
\definecolor{currentfill}{rgb}{0.000000,0.000000,0.000000}%
\pgfsetfillcolor{currentfill}%
\pgfsetlinewidth{0.602250pt}%
\definecolor{currentstroke}{rgb}{0.000000,0.000000,0.000000}%
\pgfsetstrokecolor{currentstroke}%
\pgfsetdash{}{0pt}%
\pgfsys@defobject{currentmarker}{\pgfqpoint{-0.027778in}{0.000000in}}{\pgfqpoint{-0.000000in}{0.000000in}}{%
\pgfpathmoveto{\pgfqpoint{-0.000000in}{0.000000in}}%
\pgfpathlineto{\pgfqpoint{-0.027778in}{0.000000in}}%
\pgfusepath{stroke,fill}%
}%
\begin{pgfscope}%
\pgfsys@transformshift{0.688192in}{4.510643in}%
\pgfsys@useobject{currentmarker}{}%
\end{pgfscope}%
\end{pgfscope}%
\begin{pgfscope}%
\pgfpathrectangle{\pgfqpoint{0.688192in}{0.643904in}}{\pgfqpoint{6.200000in}{4.620000in}}%
\pgfusepath{clip}%
\pgfsetbuttcap%
\pgfsetroundjoin%
\pgfsetlinewidth{0.803000pt}%
\definecolor{currentstroke}{rgb}{0.690196,0.690196,0.690196}%
\pgfsetstrokecolor{currentstroke}%
\pgfsetstrokeopacity{0.200000}%
\pgfsetdash{{2.960000pt}{1.280000pt}}{0.000000pt}%
\pgfpathmoveto{\pgfqpoint{0.688192in}{4.672192in}}%
\pgfpathlineto{\pgfqpoint{6.888192in}{4.672192in}}%
\pgfusepath{stroke}%
\end{pgfscope}%
\begin{pgfscope}%
\pgfsetbuttcap%
\pgfsetroundjoin%
\definecolor{currentfill}{rgb}{0.000000,0.000000,0.000000}%
\pgfsetfillcolor{currentfill}%
\pgfsetlinewidth{0.602250pt}%
\definecolor{currentstroke}{rgb}{0.000000,0.000000,0.000000}%
\pgfsetstrokecolor{currentstroke}%
\pgfsetdash{}{0pt}%
\pgfsys@defobject{currentmarker}{\pgfqpoint{-0.027778in}{0.000000in}}{\pgfqpoint{-0.000000in}{0.000000in}}{%
\pgfpathmoveto{\pgfqpoint{-0.000000in}{0.000000in}}%
\pgfpathlineto{\pgfqpoint{-0.027778in}{0.000000in}}%
\pgfusepath{stroke,fill}%
}%
\begin{pgfscope}%
\pgfsys@transformshift{0.688192in}{4.672192in}%
\pgfsys@useobject{currentmarker}{}%
\end{pgfscope}%
\end{pgfscope}%
\begin{pgfscope}%
\pgfpathrectangle{\pgfqpoint{0.688192in}{0.643904in}}{\pgfqpoint{6.200000in}{4.620000in}}%
\pgfusepath{clip}%
\pgfsetbuttcap%
\pgfsetroundjoin%
\pgfsetlinewidth{0.803000pt}%
\definecolor{currentstroke}{rgb}{0.690196,0.690196,0.690196}%
\pgfsetstrokecolor{currentstroke}%
\pgfsetstrokeopacity{0.200000}%
\pgfsetdash{{2.960000pt}{1.280000pt}}{0.000000pt}%
\pgfpathmoveto{\pgfqpoint{0.688192in}{4.833741in}}%
\pgfpathlineto{\pgfqpoint{6.888192in}{4.833741in}}%
\pgfusepath{stroke}%
\end{pgfscope}%
\begin{pgfscope}%
\pgfsetbuttcap%
\pgfsetroundjoin%
\definecolor{currentfill}{rgb}{0.000000,0.000000,0.000000}%
\pgfsetfillcolor{currentfill}%
\pgfsetlinewidth{0.602250pt}%
\definecolor{currentstroke}{rgb}{0.000000,0.000000,0.000000}%
\pgfsetstrokecolor{currentstroke}%
\pgfsetdash{}{0pt}%
\pgfsys@defobject{currentmarker}{\pgfqpoint{-0.027778in}{0.000000in}}{\pgfqpoint{-0.000000in}{0.000000in}}{%
\pgfpathmoveto{\pgfqpoint{-0.000000in}{0.000000in}}%
\pgfpathlineto{\pgfqpoint{-0.027778in}{0.000000in}}%
\pgfusepath{stroke,fill}%
}%
\begin{pgfscope}%
\pgfsys@transformshift{0.688192in}{4.833741in}%
\pgfsys@useobject{currentmarker}{}%
\end{pgfscope}%
\end{pgfscope}%
\begin{pgfscope}%
\pgfpathrectangle{\pgfqpoint{0.688192in}{0.643904in}}{\pgfqpoint{6.200000in}{4.620000in}}%
\pgfusepath{clip}%
\pgfsetbuttcap%
\pgfsetroundjoin%
\pgfsetlinewidth{0.803000pt}%
\definecolor{currentstroke}{rgb}{0.690196,0.690196,0.690196}%
\pgfsetstrokecolor{currentstroke}%
\pgfsetstrokeopacity{0.200000}%
\pgfsetdash{{2.960000pt}{1.280000pt}}{0.000000pt}%
\pgfpathmoveto{\pgfqpoint{0.688192in}{4.995290in}}%
\pgfpathlineto{\pgfqpoint{6.888192in}{4.995290in}}%
\pgfusepath{stroke}%
\end{pgfscope}%
\begin{pgfscope}%
\pgfsetbuttcap%
\pgfsetroundjoin%
\definecolor{currentfill}{rgb}{0.000000,0.000000,0.000000}%
\pgfsetfillcolor{currentfill}%
\pgfsetlinewidth{0.602250pt}%
\definecolor{currentstroke}{rgb}{0.000000,0.000000,0.000000}%
\pgfsetstrokecolor{currentstroke}%
\pgfsetdash{}{0pt}%
\pgfsys@defobject{currentmarker}{\pgfqpoint{-0.027778in}{0.000000in}}{\pgfqpoint{-0.000000in}{0.000000in}}{%
\pgfpathmoveto{\pgfqpoint{-0.000000in}{0.000000in}}%
\pgfpathlineto{\pgfqpoint{-0.027778in}{0.000000in}}%
\pgfusepath{stroke,fill}%
}%
\begin{pgfscope}%
\pgfsys@transformshift{0.688192in}{4.995290in}%
\pgfsys@useobject{currentmarker}{}%
\end{pgfscope}%
\end{pgfscope}%
\begin{pgfscope}%
\definecolor{textcolor}{rgb}{0.000000,0.000000,0.000000}%
\pgfsetstrokecolor{textcolor}%
\pgfsetfillcolor{textcolor}%
\pgftext[x=0.339583in,y=2.953904in,,bottom,rotate=90.000000]{\color{textcolor}{\rmfamily\fontsize{18.000000}{21.600000}\selectfont\catcode`\^=\active\def^{\ifmmode\sp\else\^{}\fi}\catcode`\%=\active\def%{\%}Time [seconds]}}%
\end{pgfscope}%
\begin{pgfscope}%
\pgfpathrectangle{\pgfqpoint{0.688192in}{0.643904in}}{\pgfqpoint{6.200000in}{4.620000in}}%
\pgfusepath{clip}%
\pgfsetrectcap%
\pgfsetroundjoin%
\pgfsetlinewidth{1.505625pt}%
\definecolor{currentstroke}{rgb}{0.007843,0.243137,1.000000}%
\pgfsetstrokecolor{currentstroke}%
\pgfsetdash{}{0pt}%
\pgfpathmoveto{\pgfqpoint{0.688192in}{0.944593in}}%
\pgfpathlineto{\pgfqpoint{1.573906in}{1.613312in}}%
\pgfpathlineto{\pgfqpoint{3.345334in}{2.774081in}}%
\pgfpathlineto{\pgfqpoint{6.888192in}{5.053904in}}%
\pgfusepath{stroke}%
\end{pgfscope}%
\begin{pgfscope}%
\pgfpathrectangle{\pgfqpoint{0.688192in}{0.643904in}}{\pgfqpoint{6.200000in}{4.620000in}}%
\pgfusepath{clip}%
\pgfsetbuttcap%
\pgfsetroundjoin%
\definecolor{currentfill}{rgb}{0.007843,0.243137,1.000000}%
\pgfsetfillcolor{currentfill}%
\pgfsetlinewidth{0.752812pt}%
\definecolor{currentstroke}{rgb}{1.000000,1.000000,1.000000}%
\pgfsetstrokecolor{currentstroke}%
\pgfsetdash{}{0pt}%
\pgfsys@defobject{currentmarker}{\pgfqpoint{-0.041667in}{-0.041667in}}{\pgfqpoint{0.041667in}{0.041667in}}{%
\pgfpathmoveto{\pgfqpoint{0.000000in}{-0.041667in}}%
\pgfpathcurveto{\pgfqpoint{0.011050in}{-0.041667in}}{\pgfqpoint{0.021649in}{-0.037276in}}{\pgfqpoint{0.029463in}{-0.029463in}}%
\pgfpathcurveto{\pgfqpoint{0.037276in}{-0.021649in}}{\pgfqpoint{0.041667in}{-0.011050in}}{\pgfqpoint{0.041667in}{0.000000in}}%
\pgfpathcurveto{\pgfqpoint{0.041667in}{0.011050in}}{\pgfqpoint{0.037276in}{0.021649in}}{\pgfqpoint{0.029463in}{0.029463in}}%
\pgfpathcurveto{\pgfqpoint{0.021649in}{0.037276in}}{\pgfqpoint{0.011050in}{0.041667in}}{\pgfqpoint{0.000000in}{0.041667in}}%
\pgfpathcurveto{\pgfqpoint{-0.011050in}{0.041667in}}{\pgfqpoint{-0.021649in}{0.037276in}}{\pgfqpoint{-0.029463in}{0.029463in}}%
\pgfpathcurveto{\pgfqpoint{-0.037276in}{0.021649in}}{\pgfqpoint{-0.041667in}{0.011050in}}{\pgfqpoint{-0.041667in}{0.000000in}}%
\pgfpathcurveto{\pgfqpoint{-0.041667in}{-0.011050in}}{\pgfqpoint{-0.037276in}{-0.021649in}}{\pgfqpoint{-0.029463in}{-0.029463in}}%
\pgfpathcurveto{\pgfqpoint{-0.021649in}{-0.037276in}}{\pgfqpoint{-0.011050in}{-0.041667in}}{\pgfqpoint{0.000000in}{-0.041667in}}%
\pgfpathlineto{\pgfqpoint{0.000000in}{-0.041667in}}%
\pgfpathclose%
\pgfusepath{stroke,fill}%
}%
\begin{pgfscope}%
\pgfsys@transformshift{0.688192in}{0.944593in}%
\pgfsys@useobject{currentmarker}{}%
\end{pgfscope}%
\begin{pgfscope}%
\pgfsys@transformshift{1.573906in}{1.613312in}%
\pgfsys@useobject{currentmarker}{}%
\end{pgfscope}%
\begin{pgfscope}%
\pgfsys@transformshift{3.345334in}{2.774081in}%
\pgfsys@useobject{currentmarker}{}%
\end{pgfscope}%
\begin{pgfscope}%
\pgfsys@transformshift{6.888192in}{5.053904in}%
\pgfsys@useobject{currentmarker}{}%
\end{pgfscope}%
\end{pgfscope}%
\begin{pgfscope}%
\pgfpathrectangle{\pgfqpoint{0.688192in}{0.643904in}}{\pgfqpoint{6.200000in}{4.620000in}}%
\pgfusepath{clip}%
\pgfsetbuttcap%
\pgfsetroundjoin%
\pgfsetlinewidth{1.505625pt}%
\definecolor{currentstroke}{rgb}{1.000000,0.486275,0.000000}%
\pgfsetstrokecolor{currentstroke}%
\pgfsetdash{{6.000000pt}{2.250000pt}}{0.000000pt}%
\pgfpathmoveto{\pgfqpoint{0.688192in}{0.937993in}}%
\pgfpathlineto{\pgfqpoint{1.573906in}{1.570977in}}%
\pgfpathlineto{\pgfqpoint{3.345334in}{2.596766in}}%
\pgfpathlineto{\pgfqpoint{6.888192in}{4.725093in}}%
\pgfusepath{stroke}%
\end{pgfscope}%
\begin{pgfscope}%
\pgfpathrectangle{\pgfqpoint{0.688192in}{0.643904in}}{\pgfqpoint{6.200000in}{4.620000in}}%
\pgfusepath{clip}%
\pgfsetbuttcap%
\pgfsetroundjoin%
\definecolor{currentfill}{rgb}{1.000000,0.486275,0.000000}%
\pgfsetfillcolor{currentfill}%
\pgfsetlinewidth{0.752812pt}%
\definecolor{currentstroke}{rgb}{1.000000,1.000000,1.000000}%
\pgfsetstrokecolor{currentstroke}%
\pgfsetdash{}{0pt}%
\pgfsys@defobject{currentmarker}{\pgfqpoint{-0.041667in}{-0.041667in}}{\pgfqpoint{0.041667in}{0.041667in}}{%
\pgfpathmoveto{\pgfqpoint{0.000000in}{-0.041667in}}%
\pgfpathcurveto{\pgfqpoint{0.011050in}{-0.041667in}}{\pgfqpoint{0.021649in}{-0.037276in}}{\pgfqpoint{0.029463in}{-0.029463in}}%
\pgfpathcurveto{\pgfqpoint{0.037276in}{-0.021649in}}{\pgfqpoint{0.041667in}{-0.011050in}}{\pgfqpoint{0.041667in}{0.000000in}}%
\pgfpathcurveto{\pgfqpoint{0.041667in}{0.011050in}}{\pgfqpoint{0.037276in}{0.021649in}}{\pgfqpoint{0.029463in}{0.029463in}}%
\pgfpathcurveto{\pgfqpoint{0.021649in}{0.037276in}}{\pgfqpoint{0.011050in}{0.041667in}}{\pgfqpoint{0.000000in}{0.041667in}}%
\pgfpathcurveto{\pgfqpoint{-0.011050in}{0.041667in}}{\pgfqpoint{-0.021649in}{0.037276in}}{\pgfqpoint{-0.029463in}{0.029463in}}%
\pgfpathcurveto{\pgfqpoint{-0.037276in}{0.021649in}}{\pgfqpoint{-0.041667in}{0.011050in}}{\pgfqpoint{-0.041667in}{0.000000in}}%
\pgfpathcurveto{\pgfqpoint{-0.041667in}{-0.011050in}}{\pgfqpoint{-0.037276in}{-0.021649in}}{\pgfqpoint{-0.029463in}{-0.029463in}}%
\pgfpathcurveto{\pgfqpoint{-0.021649in}{-0.037276in}}{\pgfqpoint{-0.011050in}{-0.041667in}}{\pgfqpoint{0.000000in}{-0.041667in}}%
\pgfpathlineto{\pgfqpoint{0.000000in}{-0.041667in}}%
\pgfpathclose%
\pgfusepath{stroke,fill}%
}%
\begin{pgfscope}%
\pgfsys@transformshift{0.688192in}{0.937993in}%
\pgfsys@useobject{currentmarker}{}%
\end{pgfscope}%
\begin{pgfscope}%
\pgfsys@transformshift{1.573906in}{1.570977in}%
\pgfsys@useobject{currentmarker}{}%
\end{pgfscope}%
\begin{pgfscope}%
\pgfsys@transformshift{3.345334in}{2.596766in}%
\pgfsys@useobject{currentmarker}{}%
\end{pgfscope}%
\begin{pgfscope}%
\pgfsys@transformshift{6.888192in}{4.725093in}%
\pgfsys@useobject{currentmarker}{}%
\end{pgfscope}%
\end{pgfscope}%
\begin{pgfscope}%
\pgfpathrectangle{\pgfqpoint{0.688192in}{0.643904in}}{\pgfqpoint{6.200000in}{4.620000in}}%
\pgfusepath{clip}%
\pgfsetbuttcap%
\pgfsetroundjoin%
\pgfsetlinewidth{1.505625pt}%
\definecolor{currentstroke}{rgb}{0.101961,0.788235,0.219608}%
\pgfsetstrokecolor{currentstroke}%
\pgfsetdash{{1.500000pt}{1.500000pt}}{0.000000pt}%
\pgfpathmoveto{\pgfqpoint{0.688192in}{0.853904in}}%
\pgfpathlineto{\pgfqpoint{1.573906in}{1.519257in}}%
\pgfpathlineto{\pgfqpoint{3.345334in}{2.702531in}}%
\pgfpathlineto{\pgfqpoint{6.888192in}{4.463327in}}%
\pgfusepath{stroke}%
\end{pgfscope}%
\begin{pgfscope}%
\pgfpathrectangle{\pgfqpoint{0.688192in}{0.643904in}}{\pgfqpoint{6.200000in}{4.620000in}}%
\pgfusepath{clip}%
\pgfsetbuttcap%
\pgfsetroundjoin%
\definecolor{currentfill}{rgb}{0.101961,0.788235,0.219608}%
\pgfsetfillcolor{currentfill}%
\pgfsetlinewidth{0.752812pt}%
\definecolor{currentstroke}{rgb}{1.000000,1.000000,1.000000}%
\pgfsetstrokecolor{currentstroke}%
\pgfsetdash{}{0pt}%
\pgfsys@defobject{currentmarker}{\pgfqpoint{-0.041667in}{-0.041667in}}{\pgfqpoint{0.041667in}{0.041667in}}{%
\pgfpathmoveto{\pgfqpoint{0.000000in}{-0.041667in}}%
\pgfpathcurveto{\pgfqpoint{0.011050in}{-0.041667in}}{\pgfqpoint{0.021649in}{-0.037276in}}{\pgfqpoint{0.029463in}{-0.029463in}}%
\pgfpathcurveto{\pgfqpoint{0.037276in}{-0.021649in}}{\pgfqpoint{0.041667in}{-0.011050in}}{\pgfqpoint{0.041667in}{0.000000in}}%
\pgfpathcurveto{\pgfqpoint{0.041667in}{0.011050in}}{\pgfqpoint{0.037276in}{0.021649in}}{\pgfqpoint{0.029463in}{0.029463in}}%
\pgfpathcurveto{\pgfqpoint{0.021649in}{0.037276in}}{\pgfqpoint{0.011050in}{0.041667in}}{\pgfqpoint{0.000000in}{0.041667in}}%
\pgfpathcurveto{\pgfqpoint{-0.011050in}{0.041667in}}{\pgfqpoint{-0.021649in}{0.037276in}}{\pgfqpoint{-0.029463in}{0.029463in}}%
\pgfpathcurveto{\pgfqpoint{-0.037276in}{0.021649in}}{\pgfqpoint{-0.041667in}{0.011050in}}{\pgfqpoint{-0.041667in}{0.000000in}}%
\pgfpathcurveto{\pgfqpoint{-0.041667in}{-0.011050in}}{\pgfqpoint{-0.037276in}{-0.021649in}}{\pgfqpoint{-0.029463in}{-0.029463in}}%
\pgfpathcurveto{\pgfqpoint{-0.021649in}{-0.037276in}}{\pgfqpoint{-0.011050in}{-0.041667in}}{\pgfqpoint{0.000000in}{-0.041667in}}%
\pgfpathlineto{\pgfqpoint{0.000000in}{-0.041667in}}%
\pgfpathclose%
\pgfusepath{stroke,fill}%
}%
\begin{pgfscope}%
\pgfsys@transformshift{0.688192in}{0.853904in}%
\pgfsys@useobject{currentmarker}{}%
\end{pgfscope}%
\begin{pgfscope}%
\pgfsys@transformshift{1.573906in}{1.519257in}%
\pgfsys@useobject{currentmarker}{}%
\end{pgfscope}%
\begin{pgfscope}%
\pgfsys@transformshift{3.345334in}{2.702531in}%
\pgfsys@useobject{currentmarker}{}%
\end{pgfscope}%
\begin{pgfscope}%
\pgfsys@transformshift{6.888192in}{4.463327in}%
\pgfsys@useobject{currentmarker}{}%
\end{pgfscope}%
\end{pgfscope}%
\begin{pgfscope}%
\pgfpathrectangle{\pgfqpoint{0.688192in}{0.643904in}}{\pgfqpoint{6.200000in}{4.620000in}}%
\pgfusepath{clip}%
\pgfsetbuttcap%
\pgfsetroundjoin%
\pgfsetlinewidth{1.505625pt}%
\definecolor{currentstroke}{rgb}{0.909804,0.000000,0.043137}%
\pgfsetstrokecolor{currentstroke}%
\pgfsetdash{{4.500000pt}{1.875000pt}{2.250000pt}{1.875000pt}}{0.000000pt}%
\pgfpathmoveto{\pgfqpoint{0.688192in}{0.858156in}}%
\pgfpathlineto{\pgfqpoint{1.573906in}{1.407561in}}%
\pgfpathlineto{\pgfqpoint{3.345334in}{2.117402in}}%
\pgfpathlineto{\pgfqpoint{6.888192in}{4.376203in}}%
\pgfusepath{stroke}%
\end{pgfscope}%
\begin{pgfscope}%
\pgfpathrectangle{\pgfqpoint{0.688192in}{0.643904in}}{\pgfqpoint{6.200000in}{4.620000in}}%
\pgfusepath{clip}%
\pgfsetbuttcap%
\pgfsetroundjoin%
\definecolor{currentfill}{rgb}{0.909804,0.000000,0.043137}%
\pgfsetfillcolor{currentfill}%
\pgfsetlinewidth{0.752812pt}%
\definecolor{currentstroke}{rgb}{1.000000,1.000000,1.000000}%
\pgfsetstrokecolor{currentstroke}%
\pgfsetdash{}{0pt}%
\pgfsys@defobject{currentmarker}{\pgfqpoint{-0.041667in}{-0.041667in}}{\pgfqpoint{0.041667in}{0.041667in}}{%
\pgfpathmoveto{\pgfqpoint{0.000000in}{-0.041667in}}%
\pgfpathcurveto{\pgfqpoint{0.011050in}{-0.041667in}}{\pgfqpoint{0.021649in}{-0.037276in}}{\pgfqpoint{0.029463in}{-0.029463in}}%
\pgfpathcurveto{\pgfqpoint{0.037276in}{-0.021649in}}{\pgfqpoint{0.041667in}{-0.011050in}}{\pgfqpoint{0.041667in}{0.000000in}}%
\pgfpathcurveto{\pgfqpoint{0.041667in}{0.011050in}}{\pgfqpoint{0.037276in}{0.021649in}}{\pgfqpoint{0.029463in}{0.029463in}}%
\pgfpathcurveto{\pgfqpoint{0.021649in}{0.037276in}}{\pgfqpoint{0.011050in}{0.041667in}}{\pgfqpoint{0.000000in}{0.041667in}}%
\pgfpathcurveto{\pgfqpoint{-0.011050in}{0.041667in}}{\pgfqpoint{-0.021649in}{0.037276in}}{\pgfqpoint{-0.029463in}{0.029463in}}%
\pgfpathcurveto{\pgfqpoint{-0.037276in}{0.021649in}}{\pgfqpoint{-0.041667in}{0.011050in}}{\pgfqpoint{-0.041667in}{0.000000in}}%
\pgfpathcurveto{\pgfqpoint{-0.041667in}{-0.011050in}}{\pgfqpoint{-0.037276in}{-0.021649in}}{\pgfqpoint{-0.029463in}{-0.029463in}}%
\pgfpathcurveto{\pgfqpoint{-0.021649in}{-0.037276in}}{\pgfqpoint{-0.011050in}{-0.041667in}}{\pgfqpoint{0.000000in}{-0.041667in}}%
\pgfpathlineto{\pgfqpoint{0.000000in}{-0.041667in}}%
\pgfpathclose%
\pgfusepath{stroke,fill}%
}%
\begin{pgfscope}%
\pgfsys@transformshift{0.688192in}{0.858156in}%
\pgfsys@useobject{currentmarker}{}%
\end{pgfscope}%
\begin{pgfscope}%
\pgfsys@transformshift{1.573906in}{1.407561in}%
\pgfsys@useobject{currentmarker}{}%
\end{pgfscope}%
\begin{pgfscope}%
\pgfsys@transformshift{3.345334in}{2.117402in}%
\pgfsys@useobject{currentmarker}{}%
\end{pgfscope}%
\begin{pgfscope}%
\pgfsys@transformshift{6.888192in}{4.376203in}%
\pgfsys@useobject{currentmarker}{}%
\end{pgfscope}%
\end{pgfscope}%
\begin{pgfscope}%
\pgfpathrectangle{\pgfqpoint{0.688192in}{0.643904in}}{\pgfqpoint{6.200000in}{4.620000in}}%
\pgfusepath{clip}%
\pgfsetbuttcap%
\pgfsetroundjoin%
\pgfsetlinewidth{1.505625pt}%
\definecolor{currentstroke}{rgb}{0.545098,0.168627,0.886275}%
\pgfsetstrokecolor{currentstroke}%
\pgfsetdash{{7.500000pt}{1.500000pt}{1.500000pt}{1.500000pt}}{0.000000pt}%
\pgfpathmoveto{\pgfqpoint{0.688192in}{0.876810in}}%
\pgfpathlineto{\pgfqpoint{1.573906in}{1.419924in}}%
\pgfpathlineto{\pgfqpoint{3.345334in}{2.445519in}}%
\pgfpathlineto{\pgfqpoint{6.888192in}{4.301315in}}%
\pgfusepath{stroke}%
\end{pgfscope}%
\begin{pgfscope}%
\pgfpathrectangle{\pgfqpoint{0.688192in}{0.643904in}}{\pgfqpoint{6.200000in}{4.620000in}}%
\pgfusepath{clip}%
\pgfsetbuttcap%
\pgfsetroundjoin%
\definecolor{currentfill}{rgb}{0.545098,0.168627,0.886275}%
\pgfsetfillcolor{currentfill}%
\pgfsetlinewidth{0.752812pt}%
\definecolor{currentstroke}{rgb}{1.000000,1.000000,1.000000}%
\pgfsetstrokecolor{currentstroke}%
\pgfsetdash{}{0pt}%
\pgfsys@defobject{currentmarker}{\pgfqpoint{-0.041667in}{-0.041667in}}{\pgfqpoint{0.041667in}{0.041667in}}{%
\pgfpathmoveto{\pgfqpoint{0.000000in}{-0.041667in}}%
\pgfpathcurveto{\pgfqpoint{0.011050in}{-0.041667in}}{\pgfqpoint{0.021649in}{-0.037276in}}{\pgfqpoint{0.029463in}{-0.029463in}}%
\pgfpathcurveto{\pgfqpoint{0.037276in}{-0.021649in}}{\pgfqpoint{0.041667in}{-0.011050in}}{\pgfqpoint{0.041667in}{0.000000in}}%
\pgfpathcurveto{\pgfqpoint{0.041667in}{0.011050in}}{\pgfqpoint{0.037276in}{0.021649in}}{\pgfqpoint{0.029463in}{0.029463in}}%
\pgfpathcurveto{\pgfqpoint{0.021649in}{0.037276in}}{\pgfqpoint{0.011050in}{0.041667in}}{\pgfqpoint{0.000000in}{0.041667in}}%
\pgfpathcurveto{\pgfqpoint{-0.011050in}{0.041667in}}{\pgfqpoint{-0.021649in}{0.037276in}}{\pgfqpoint{-0.029463in}{0.029463in}}%
\pgfpathcurveto{\pgfqpoint{-0.037276in}{0.021649in}}{\pgfqpoint{-0.041667in}{0.011050in}}{\pgfqpoint{-0.041667in}{0.000000in}}%
\pgfpathcurveto{\pgfqpoint{-0.041667in}{-0.011050in}}{\pgfqpoint{-0.037276in}{-0.021649in}}{\pgfqpoint{-0.029463in}{-0.029463in}}%
\pgfpathcurveto{\pgfqpoint{-0.021649in}{-0.037276in}}{\pgfqpoint{-0.011050in}{-0.041667in}}{\pgfqpoint{0.000000in}{-0.041667in}}%
\pgfpathlineto{\pgfqpoint{0.000000in}{-0.041667in}}%
\pgfpathclose%
\pgfusepath{stroke,fill}%
}%
\begin{pgfscope}%
\pgfsys@transformshift{0.688192in}{0.876810in}%
\pgfsys@useobject{currentmarker}{}%
\end{pgfscope}%
\begin{pgfscope}%
\pgfsys@transformshift{1.573906in}{1.419924in}%
\pgfsys@useobject{currentmarker}{}%
\end{pgfscope}%
\begin{pgfscope}%
\pgfsys@transformshift{3.345334in}{2.445519in}%
\pgfsys@useobject{currentmarker}{}%
\end{pgfscope}%
\begin{pgfscope}%
\pgfsys@transformshift{6.888192in}{4.301315in}%
\pgfsys@useobject{currentmarker}{}%
\end{pgfscope}%
\end{pgfscope}%
\begin{pgfscope}%
\pgfsetrectcap%
\pgfsetmiterjoin%
\pgfsetlinewidth{0.803000pt}%
\definecolor{currentstroke}{rgb}{0.000000,0.000000,0.000000}%
\pgfsetstrokecolor{currentstroke}%
\pgfsetdash{}{0pt}%
\pgfpathmoveto{\pgfqpoint{0.688192in}{0.643904in}}%
\pgfpathlineto{\pgfqpoint{0.688192in}{5.263904in}}%
\pgfusepath{stroke}%
\end{pgfscope}%
\begin{pgfscope}%
\pgfsetrectcap%
\pgfsetmiterjoin%
\pgfsetlinewidth{0.803000pt}%
\definecolor{currentstroke}{rgb}{0.000000,0.000000,0.000000}%
\pgfsetstrokecolor{currentstroke}%
\pgfsetdash{}{0pt}%
\pgfpathmoveto{\pgfqpoint{6.888192in}{0.643904in}}%
\pgfpathlineto{\pgfqpoint{6.888192in}{5.263904in}}%
\pgfusepath{stroke}%
\end{pgfscope}%
\begin{pgfscope}%
\pgfsetrectcap%
\pgfsetmiterjoin%
\pgfsetlinewidth{0.803000pt}%
\definecolor{currentstroke}{rgb}{0.000000,0.000000,0.000000}%
\pgfsetstrokecolor{currentstroke}%
\pgfsetdash{}{0pt}%
\pgfpathmoveto{\pgfqpoint{0.688192in}{0.643904in}}%
\pgfpathlineto{\pgfqpoint{6.888192in}{0.643904in}}%
\pgfusepath{stroke}%
\end{pgfscope}%
\begin{pgfscope}%
\pgfsetrectcap%
\pgfsetmiterjoin%
\pgfsetlinewidth{0.803000pt}%
\definecolor{currentstroke}{rgb}{0.000000,0.000000,0.000000}%
\pgfsetstrokecolor{currentstroke}%
\pgfsetdash{}{0pt}%
\pgfpathmoveto{\pgfqpoint{0.688192in}{5.263904in}}%
\pgfpathlineto{\pgfqpoint{6.888192in}{5.263904in}}%
\pgfusepath{stroke}%
\end{pgfscope}%
\begin{pgfscope}%
\pgfsetbuttcap%
\pgfsetmiterjoin%
\definecolor{currentfill}{rgb}{1.000000,1.000000,1.000000}%
\pgfsetfillcolor{currentfill}%
\pgfsetfillopacity{0.800000}%
\pgfsetlinewidth{1.003750pt}%
\definecolor{currentstroke}{rgb}{0.800000,0.800000,0.800000}%
\pgfsetstrokecolor{currentstroke}%
\pgfsetstrokeopacity{0.800000}%
\pgfsetdash{}{0pt}%
\pgfpathmoveto{\pgfqpoint{0.824303in}{3.458351in}}%
\pgfpathlineto{\pgfqpoint{2.578541in}{3.458351in}}%
\pgfpathquadraticcurveto{\pgfqpoint{2.617430in}{3.458351in}}{\pgfqpoint{2.617430in}{3.497240in}}%
\pgfpathlineto{\pgfqpoint{2.617430in}{5.127793in}}%
\pgfpathquadraticcurveto{\pgfqpoint{2.617430in}{5.166682in}}{\pgfqpoint{2.578541in}{5.166682in}}%
\pgfpathlineto{\pgfqpoint{0.824303in}{5.166682in}}%
\pgfpathquadraticcurveto{\pgfqpoint{0.785414in}{5.166682in}}{\pgfqpoint{0.785414in}{5.127793in}}%
\pgfpathlineto{\pgfqpoint{0.785414in}{3.497240in}}%
\pgfpathquadraticcurveto{\pgfqpoint{0.785414in}{3.458351in}}{\pgfqpoint{0.824303in}{3.458351in}}%
\pgfpathlineto{\pgfqpoint{0.824303in}{3.458351in}}%
\pgfpathclose%
\pgfusepath{stroke,fill}%
\end{pgfscope}%
\begin{pgfscope}%
\definecolor{textcolor}{rgb}{0.000000,0.000000,0.000000}%
\pgfsetstrokecolor{textcolor}%
\pgfsetfillcolor{textcolor}%
\pgftext[x=0.863192in,y=4.950015in,left,base]{\color{textcolor}{\rmfamily\fontsize{14.000000}{16.800000}\selectfont\catcode`\^=\active\def^{\ifmmode\sp\else\^{}\fi}\catcode`\%=\active\def%{\%}Number of Threads}}%
\end{pgfscope}%
\begin{pgfscope}%
\pgfsetrectcap%
\pgfsetroundjoin%
\pgfsetlinewidth{1.505625pt}%
\definecolor{currentstroke}{rgb}{0.007843,0.243137,1.000000}%
\pgfsetstrokecolor{currentstroke}%
\pgfsetdash{}{0pt}%
\pgfpathmoveto{\pgfqpoint{1.331284in}{4.743071in}}%
\pgfpathlineto{\pgfqpoint{1.525729in}{4.743071in}}%
\pgfpathlineto{\pgfqpoint{1.720173in}{4.743071in}}%
\pgfusepath{stroke}%
\end{pgfscope}%
\begin{pgfscope}%
\pgfsetbuttcap%
\pgfsetroundjoin%
\definecolor{currentfill}{rgb}{0.007843,0.243137,1.000000}%
\pgfsetfillcolor{currentfill}%
\pgfsetlinewidth{0.752812pt}%
\definecolor{currentstroke}{rgb}{1.000000,1.000000,1.000000}%
\pgfsetstrokecolor{currentstroke}%
\pgfsetdash{}{0pt}%
\pgfsys@defobject{currentmarker}{\pgfqpoint{-0.041667in}{-0.041667in}}{\pgfqpoint{0.041667in}{0.041667in}}{%
\pgfpathmoveto{\pgfqpoint{0.000000in}{-0.041667in}}%
\pgfpathcurveto{\pgfqpoint{0.011050in}{-0.041667in}}{\pgfqpoint{0.021649in}{-0.037276in}}{\pgfqpoint{0.029463in}{-0.029463in}}%
\pgfpathcurveto{\pgfqpoint{0.037276in}{-0.021649in}}{\pgfqpoint{0.041667in}{-0.011050in}}{\pgfqpoint{0.041667in}{0.000000in}}%
\pgfpathcurveto{\pgfqpoint{0.041667in}{0.011050in}}{\pgfqpoint{0.037276in}{0.021649in}}{\pgfqpoint{0.029463in}{0.029463in}}%
\pgfpathcurveto{\pgfqpoint{0.021649in}{0.037276in}}{\pgfqpoint{0.011050in}{0.041667in}}{\pgfqpoint{0.000000in}{0.041667in}}%
\pgfpathcurveto{\pgfqpoint{-0.011050in}{0.041667in}}{\pgfqpoint{-0.021649in}{0.037276in}}{\pgfqpoint{-0.029463in}{0.029463in}}%
\pgfpathcurveto{\pgfqpoint{-0.037276in}{0.021649in}}{\pgfqpoint{-0.041667in}{0.011050in}}{\pgfqpoint{-0.041667in}{0.000000in}}%
\pgfpathcurveto{\pgfqpoint{-0.041667in}{-0.011050in}}{\pgfqpoint{-0.037276in}{-0.021649in}}{\pgfqpoint{-0.029463in}{-0.029463in}}%
\pgfpathcurveto{\pgfqpoint{-0.021649in}{-0.037276in}}{\pgfqpoint{-0.011050in}{-0.041667in}}{\pgfqpoint{0.000000in}{-0.041667in}}%
\pgfpathlineto{\pgfqpoint{0.000000in}{-0.041667in}}%
\pgfpathclose%
\pgfusepath{stroke,fill}%
}%
\begin{pgfscope}%
\pgfsys@transformshift{1.525729in}{4.743071in}%
\pgfsys@useobject{currentmarker}{}%
\end{pgfscope}%
\end{pgfscope}%
\begin{pgfscope}%
\definecolor{textcolor}{rgb}{0.000000,0.000000,0.000000}%
\pgfsetstrokecolor{textcolor}%
\pgfsetfillcolor{textcolor}%
\pgftext[x=1.875729in,y=4.675016in,left,base]{\color{textcolor}{\rmfamily\fontsize{14.000000}{16.800000}\selectfont\catcode`\^=\active\def^{\ifmmode\sp\else\^{}\fi}\catcode`\%=\active\def%{\%}1}}%
\end{pgfscope}%
\begin{pgfscope}%
\pgfsetbuttcap%
\pgfsetroundjoin%
\pgfsetlinewidth{1.505625pt}%
\definecolor{currentstroke}{rgb}{1.000000,0.486275,0.000000}%
\pgfsetstrokecolor{currentstroke}%
\pgfsetdash{{6.000000pt}{2.250000pt}}{0.000000pt}%
\pgfpathmoveto{\pgfqpoint{1.331284in}{4.468072in}}%
\pgfpathlineto{\pgfqpoint{1.525729in}{4.468072in}}%
\pgfpathlineto{\pgfqpoint{1.720173in}{4.468072in}}%
\pgfusepath{stroke}%
\end{pgfscope}%
\begin{pgfscope}%
\pgfsetbuttcap%
\pgfsetroundjoin%
\definecolor{currentfill}{rgb}{1.000000,0.486275,0.000000}%
\pgfsetfillcolor{currentfill}%
\pgfsetlinewidth{0.752812pt}%
\definecolor{currentstroke}{rgb}{1.000000,1.000000,1.000000}%
\pgfsetstrokecolor{currentstroke}%
\pgfsetdash{}{0pt}%
\pgfsys@defobject{currentmarker}{\pgfqpoint{-0.041667in}{-0.041667in}}{\pgfqpoint{0.041667in}{0.041667in}}{%
\pgfpathmoveto{\pgfqpoint{0.000000in}{-0.041667in}}%
\pgfpathcurveto{\pgfqpoint{0.011050in}{-0.041667in}}{\pgfqpoint{0.021649in}{-0.037276in}}{\pgfqpoint{0.029463in}{-0.029463in}}%
\pgfpathcurveto{\pgfqpoint{0.037276in}{-0.021649in}}{\pgfqpoint{0.041667in}{-0.011050in}}{\pgfqpoint{0.041667in}{0.000000in}}%
\pgfpathcurveto{\pgfqpoint{0.041667in}{0.011050in}}{\pgfqpoint{0.037276in}{0.021649in}}{\pgfqpoint{0.029463in}{0.029463in}}%
\pgfpathcurveto{\pgfqpoint{0.021649in}{0.037276in}}{\pgfqpoint{0.011050in}{0.041667in}}{\pgfqpoint{0.000000in}{0.041667in}}%
\pgfpathcurveto{\pgfqpoint{-0.011050in}{0.041667in}}{\pgfqpoint{-0.021649in}{0.037276in}}{\pgfqpoint{-0.029463in}{0.029463in}}%
\pgfpathcurveto{\pgfqpoint{-0.037276in}{0.021649in}}{\pgfqpoint{-0.041667in}{0.011050in}}{\pgfqpoint{-0.041667in}{0.000000in}}%
\pgfpathcurveto{\pgfqpoint{-0.041667in}{-0.011050in}}{\pgfqpoint{-0.037276in}{-0.021649in}}{\pgfqpoint{-0.029463in}{-0.029463in}}%
\pgfpathcurveto{\pgfqpoint{-0.021649in}{-0.037276in}}{\pgfqpoint{-0.011050in}{-0.041667in}}{\pgfqpoint{0.000000in}{-0.041667in}}%
\pgfpathlineto{\pgfqpoint{0.000000in}{-0.041667in}}%
\pgfpathclose%
\pgfusepath{stroke,fill}%
}%
\begin{pgfscope}%
\pgfsys@transformshift{1.525729in}{4.468072in}%
\pgfsys@useobject{currentmarker}{}%
\end{pgfscope}%
\end{pgfscope}%
\begin{pgfscope}%
\definecolor{textcolor}{rgb}{0.000000,0.000000,0.000000}%
\pgfsetstrokecolor{textcolor}%
\pgfsetfillcolor{textcolor}%
\pgftext[x=1.875729in,y=4.400016in,left,base]{\color{textcolor}{\rmfamily\fontsize{14.000000}{16.800000}\selectfont\catcode`\^=\active\def^{\ifmmode\sp\else\^{}\fi}\catcode`\%=\active\def%{\%}2}}%
\end{pgfscope}%
\begin{pgfscope}%
\pgfsetbuttcap%
\pgfsetroundjoin%
\pgfsetlinewidth{1.505625pt}%
\definecolor{currentstroke}{rgb}{0.101961,0.788235,0.219608}%
\pgfsetstrokecolor{currentstroke}%
\pgfsetdash{{1.500000pt}{1.500000pt}}{0.000000pt}%
\pgfpathmoveto{\pgfqpoint{1.331284in}{4.193072in}}%
\pgfpathlineto{\pgfqpoint{1.525729in}{4.193072in}}%
\pgfpathlineto{\pgfqpoint{1.720173in}{4.193072in}}%
\pgfusepath{stroke}%
\end{pgfscope}%
\begin{pgfscope}%
\pgfsetbuttcap%
\pgfsetroundjoin%
\definecolor{currentfill}{rgb}{0.101961,0.788235,0.219608}%
\pgfsetfillcolor{currentfill}%
\pgfsetlinewidth{0.752812pt}%
\definecolor{currentstroke}{rgb}{1.000000,1.000000,1.000000}%
\pgfsetstrokecolor{currentstroke}%
\pgfsetdash{}{0pt}%
\pgfsys@defobject{currentmarker}{\pgfqpoint{-0.041667in}{-0.041667in}}{\pgfqpoint{0.041667in}{0.041667in}}{%
\pgfpathmoveto{\pgfqpoint{0.000000in}{-0.041667in}}%
\pgfpathcurveto{\pgfqpoint{0.011050in}{-0.041667in}}{\pgfqpoint{0.021649in}{-0.037276in}}{\pgfqpoint{0.029463in}{-0.029463in}}%
\pgfpathcurveto{\pgfqpoint{0.037276in}{-0.021649in}}{\pgfqpoint{0.041667in}{-0.011050in}}{\pgfqpoint{0.041667in}{0.000000in}}%
\pgfpathcurveto{\pgfqpoint{0.041667in}{0.011050in}}{\pgfqpoint{0.037276in}{0.021649in}}{\pgfqpoint{0.029463in}{0.029463in}}%
\pgfpathcurveto{\pgfqpoint{0.021649in}{0.037276in}}{\pgfqpoint{0.011050in}{0.041667in}}{\pgfqpoint{0.000000in}{0.041667in}}%
\pgfpathcurveto{\pgfqpoint{-0.011050in}{0.041667in}}{\pgfqpoint{-0.021649in}{0.037276in}}{\pgfqpoint{-0.029463in}{0.029463in}}%
\pgfpathcurveto{\pgfqpoint{-0.037276in}{0.021649in}}{\pgfqpoint{-0.041667in}{0.011050in}}{\pgfqpoint{-0.041667in}{0.000000in}}%
\pgfpathcurveto{\pgfqpoint{-0.041667in}{-0.011050in}}{\pgfqpoint{-0.037276in}{-0.021649in}}{\pgfqpoint{-0.029463in}{-0.029463in}}%
\pgfpathcurveto{\pgfqpoint{-0.021649in}{-0.037276in}}{\pgfqpoint{-0.011050in}{-0.041667in}}{\pgfqpoint{0.000000in}{-0.041667in}}%
\pgfpathlineto{\pgfqpoint{0.000000in}{-0.041667in}}%
\pgfpathclose%
\pgfusepath{stroke,fill}%
}%
\begin{pgfscope}%
\pgfsys@transformshift{1.525729in}{4.193072in}%
\pgfsys@useobject{currentmarker}{}%
\end{pgfscope}%
\end{pgfscope}%
\begin{pgfscope}%
\definecolor{textcolor}{rgb}{0.000000,0.000000,0.000000}%
\pgfsetstrokecolor{textcolor}%
\pgfsetfillcolor{textcolor}%
\pgftext[x=1.875729in,y=4.125016in,left,base]{\color{textcolor}{\rmfamily\fontsize{14.000000}{16.800000}\selectfont\catcode`\^=\active\def^{\ifmmode\sp\else\^{}\fi}\catcode`\%=\active\def%{\%}4}}%
\end{pgfscope}%
\begin{pgfscope}%
\pgfsetbuttcap%
\pgfsetroundjoin%
\pgfsetlinewidth{1.505625pt}%
\definecolor{currentstroke}{rgb}{0.909804,0.000000,0.043137}%
\pgfsetstrokecolor{currentstroke}%
\pgfsetdash{{4.500000pt}{1.875000pt}{2.250000pt}{1.875000pt}}{0.000000pt}%
\pgfpathmoveto{\pgfqpoint{1.331284in}{3.918072in}}%
\pgfpathlineto{\pgfqpoint{1.525729in}{3.918072in}}%
\pgfpathlineto{\pgfqpoint{1.720173in}{3.918072in}}%
\pgfusepath{stroke}%
\end{pgfscope}%
\begin{pgfscope}%
\pgfsetbuttcap%
\pgfsetroundjoin%
\definecolor{currentfill}{rgb}{0.909804,0.000000,0.043137}%
\pgfsetfillcolor{currentfill}%
\pgfsetlinewidth{0.752812pt}%
\definecolor{currentstroke}{rgb}{1.000000,1.000000,1.000000}%
\pgfsetstrokecolor{currentstroke}%
\pgfsetdash{}{0pt}%
\pgfsys@defobject{currentmarker}{\pgfqpoint{-0.041667in}{-0.041667in}}{\pgfqpoint{0.041667in}{0.041667in}}{%
\pgfpathmoveto{\pgfqpoint{0.000000in}{-0.041667in}}%
\pgfpathcurveto{\pgfqpoint{0.011050in}{-0.041667in}}{\pgfqpoint{0.021649in}{-0.037276in}}{\pgfqpoint{0.029463in}{-0.029463in}}%
\pgfpathcurveto{\pgfqpoint{0.037276in}{-0.021649in}}{\pgfqpoint{0.041667in}{-0.011050in}}{\pgfqpoint{0.041667in}{0.000000in}}%
\pgfpathcurveto{\pgfqpoint{0.041667in}{0.011050in}}{\pgfqpoint{0.037276in}{0.021649in}}{\pgfqpoint{0.029463in}{0.029463in}}%
\pgfpathcurveto{\pgfqpoint{0.021649in}{0.037276in}}{\pgfqpoint{0.011050in}{0.041667in}}{\pgfqpoint{0.000000in}{0.041667in}}%
\pgfpathcurveto{\pgfqpoint{-0.011050in}{0.041667in}}{\pgfqpoint{-0.021649in}{0.037276in}}{\pgfqpoint{-0.029463in}{0.029463in}}%
\pgfpathcurveto{\pgfqpoint{-0.037276in}{0.021649in}}{\pgfqpoint{-0.041667in}{0.011050in}}{\pgfqpoint{-0.041667in}{0.000000in}}%
\pgfpathcurveto{\pgfqpoint{-0.041667in}{-0.011050in}}{\pgfqpoint{-0.037276in}{-0.021649in}}{\pgfqpoint{-0.029463in}{-0.029463in}}%
\pgfpathcurveto{\pgfqpoint{-0.021649in}{-0.037276in}}{\pgfqpoint{-0.011050in}{-0.041667in}}{\pgfqpoint{0.000000in}{-0.041667in}}%
\pgfpathlineto{\pgfqpoint{0.000000in}{-0.041667in}}%
\pgfpathclose%
\pgfusepath{stroke,fill}%
}%
\begin{pgfscope}%
\pgfsys@transformshift{1.525729in}{3.918072in}%
\pgfsys@useobject{currentmarker}{}%
\end{pgfscope}%
\end{pgfscope}%
\begin{pgfscope}%
\definecolor{textcolor}{rgb}{0.000000,0.000000,0.000000}%
\pgfsetstrokecolor{textcolor}%
\pgfsetfillcolor{textcolor}%
\pgftext[x=1.875729in,y=3.850017in,left,base]{\color{textcolor}{\rmfamily\fontsize{14.000000}{16.800000}\selectfont\catcode`\^=\active\def^{\ifmmode\sp\else\^{}\fi}\catcode`\%=\active\def%{\%}8}}%
\end{pgfscope}%
\begin{pgfscope}%
\pgfsetbuttcap%
\pgfsetroundjoin%
\pgfsetlinewidth{1.505625pt}%
\definecolor{currentstroke}{rgb}{0.545098,0.168627,0.886275}%
\pgfsetstrokecolor{currentstroke}%
\pgfsetdash{{7.500000pt}{1.500000pt}{1.500000pt}{1.500000pt}}{0.000000pt}%
\pgfpathmoveto{\pgfqpoint{1.331284in}{3.643073in}}%
\pgfpathlineto{\pgfqpoint{1.525729in}{3.643073in}}%
\pgfpathlineto{\pgfqpoint{1.720173in}{3.643073in}}%
\pgfusepath{stroke}%
\end{pgfscope}%
\begin{pgfscope}%
\pgfsetbuttcap%
\pgfsetroundjoin%
\definecolor{currentfill}{rgb}{0.545098,0.168627,0.886275}%
\pgfsetfillcolor{currentfill}%
\pgfsetlinewidth{0.752812pt}%
\definecolor{currentstroke}{rgb}{1.000000,1.000000,1.000000}%
\pgfsetstrokecolor{currentstroke}%
\pgfsetdash{}{0pt}%
\pgfsys@defobject{currentmarker}{\pgfqpoint{-0.041667in}{-0.041667in}}{\pgfqpoint{0.041667in}{0.041667in}}{%
\pgfpathmoveto{\pgfqpoint{0.000000in}{-0.041667in}}%
\pgfpathcurveto{\pgfqpoint{0.011050in}{-0.041667in}}{\pgfqpoint{0.021649in}{-0.037276in}}{\pgfqpoint{0.029463in}{-0.029463in}}%
\pgfpathcurveto{\pgfqpoint{0.037276in}{-0.021649in}}{\pgfqpoint{0.041667in}{-0.011050in}}{\pgfqpoint{0.041667in}{0.000000in}}%
\pgfpathcurveto{\pgfqpoint{0.041667in}{0.011050in}}{\pgfqpoint{0.037276in}{0.021649in}}{\pgfqpoint{0.029463in}{0.029463in}}%
\pgfpathcurveto{\pgfqpoint{0.021649in}{0.037276in}}{\pgfqpoint{0.011050in}{0.041667in}}{\pgfqpoint{0.000000in}{0.041667in}}%
\pgfpathcurveto{\pgfqpoint{-0.011050in}{0.041667in}}{\pgfqpoint{-0.021649in}{0.037276in}}{\pgfqpoint{-0.029463in}{0.029463in}}%
\pgfpathcurveto{\pgfqpoint{-0.037276in}{0.021649in}}{\pgfqpoint{-0.041667in}{0.011050in}}{\pgfqpoint{-0.041667in}{0.000000in}}%
\pgfpathcurveto{\pgfqpoint{-0.041667in}{-0.011050in}}{\pgfqpoint{-0.037276in}{-0.021649in}}{\pgfqpoint{-0.029463in}{-0.029463in}}%
\pgfpathcurveto{\pgfqpoint{-0.021649in}{-0.037276in}}{\pgfqpoint{-0.011050in}{-0.041667in}}{\pgfqpoint{0.000000in}{-0.041667in}}%
\pgfpathlineto{\pgfqpoint{0.000000in}{-0.041667in}}%
\pgfpathclose%
\pgfusepath{stroke,fill}%
}%
\begin{pgfscope}%
\pgfsys@transformshift{1.525729in}{3.643073in}%
\pgfsys@useobject{currentmarker}{}%
\end{pgfscope}%
\end{pgfscope}%
\begin{pgfscope}%
\definecolor{textcolor}{rgb}{0.000000,0.000000,0.000000}%
\pgfsetstrokecolor{textcolor}%
\pgfsetfillcolor{textcolor}%
\pgftext[x=1.875729in,y=3.575017in,left,base]{\color{textcolor}{\rmfamily\fontsize{14.000000}{16.800000}\selectfont\catcode`\^=\active\def^{\ifmmode\sp\else\^{}\fi}\catcode`\%=\active\def%{\%}12}}%
\end{pgfscope}%
\end{pgfpicture}%
\makeatother%
\endgroup%
}
    \caption{Time scaling of a capacity expansion problem using a range of
    threads for parallelization.}
    \label{fig:thread-scaling}
\end{figure}

All simulations perform similarly when the problem size is small
because multithreading has some overhead. Multiple threads outperformed the
single threaded simulation in every case. Eight threads outperformed twelve
threads in the middle of the population range, but otherwise performed
similarly. Simulations with two and four threads performed similarly until the
higher end of the population range where four threads proved faster. The overall
speed improvement from multithreading was modest with a roughly four second
improvement at best. This suggests that further code optimization could improve
performance and that better computer architecture might be needed to fully
realize the parallelization enhancement.

\FloatBarrier
\section{Benchmarking \ac{osier}}
The previous sections demonstrated internal consistency between two dispatch algorithms
within \ac{osier} and showed how these algorithms scale with problem size and the number
of available threads. This section explores the performance different \acp{ga}, compares 
\ac{osier} to the established \ac{esom}, \ac{temoa}, and shows how \ac{osier} exceeds
current state-of-the-art.
\subsection{\acs{temoa} and \acs{pygen}}
\label{section:temoa}

This thesis uses the tools \ac{temoa} and \ac{pygen} to establish benchmark
results for a typical \ac{esom}. \ac{temoa} is an open-source \ac{esom}
developed at North Carolina State University that uses \ac{milp} to develop
capacity-expansion scenarios \cite{decarolis_temoa_2010}. The key benefits of
\ac{temoa} are its open-source code, open data, and built-in uncertainty
analysis capabilities. These features address the need for greater transparency
in \ac{esom} modeling and robust assessment of future uncertainties
\cite{hunter_modeling_2013, fattahi_systemic_2020}. \ac{pygen} is another
open-source code, developed by this thesis' author, that facilitates rapid
development of \ac{temoa} models and enables sensitivity analyses using a
templated approach \cite{dotson_influence_2022, dotson_python_2021}. These
features of \ac{pygen} reduce the learning curve and the cost of producing
unique models in \ac{temoa} \cite{dotson_influence_2022}.

A single \ac{temoa} run minimizes total system cost \cite{decarolis_temoa_2010},

\begin{align}
  C_{total} &= C_{loans} + C_{fixed} + C_{variable}
  \intertext{where}
  C_{loans} &= \text{the sum of all investment loan costs},\nonumber\\
  C_{fixed} &= \text{the sum of all fixed operating costs},\nonumber\\
  C_{variable} &= \text{the sum of all variable operating costs}.\nonumber
\end{align}

Each of these terms is amortized over the model time horizon. The decision
variables include the generation from each technology at time, $t$, and the
capacity of each technology in year, $y$. The dispatch model deviates slightly
from the model described in Section \ref{section:dispatch} by making the initial
storage level for energy storage technologies a decision variable, whereas the
dispatch model used in this thesis does not optimize initial storage and assumes
energy storage starts at zero. The detailed formulation of \ac{temoa}'s
constraints and equations are available online \cite{decarolis_temoa_2010}.
% (\textcolor{red}{maybe in an appendix?}). 

\subsection{\acl{mga}}
\label{section:mga}
\ac{temoa}'s built-in method for uncertainty analysis is the \ac{hsj}
formulation of \ac{mga}. This algorithm is designed to handle
\textit{structural} uncertainty, which presumes to account for unmodeled
objectives. The steps for \ac{hsj} are \cite{decarolis_using_2011,
dotson_influence_2022}:
\begin{enumerate}
  \item obtain an optimal solution by any method,
  \item add a user-specified amount of slack to the objective function value
  from the first step,
  \item use the adjusted objective function value as an upper bound constraint,
  \item generate a new objective function that minimizes the sum of all decision
  variables,
  \item iterate the procedure,
  \item stop the \ac{mga} when no significant changes are observed.
\end{enumerate}
The mathematical formulation of this algorithm is to
\begin{align}
  \intertext{minimize:}
  p &= \sum_{k\in K} x_k,
  \intertext{subject to:}
  f_j\left(\vec{x}\right) &\leq T_j \quad\forall \quad j,\\
  \vec{x}&\in X,
  \intertext{where}
  p &= \text{the new objective function}\nonumber,\\
  x_k &= \text{the $k^{th}$ decision variable with a nonzero value in previous solutions}\nonumber,\\
  f_j\left(\vec{x}\right) &= \text{the $j^{th}$ original objective function},\nonumber\\
  T_j &= \text{the slack-adjusted target value},\nonumber\\
  X &= \text{the set of all feasible solutions}.\nonumber
\end{align}

Figure \ref{fig:standard_mga} illustrates this algorithm for a simple \ac{lp}
with two decision variables and a slack value of 10\%.

\begin{align}
  \intertext{Minimize:}
  f(x_1, x_2) &= c_1x_1 + c_2x_2,
  \intertext{subject to:}
  x_1 &+ x_2 = 1,
  \intertext{where}
  x_{k} &= \text{the $k^{th}$ decision variable,} \nonumber \\
  c_{k} &= \text{the $k^{th}$ cost.}\nonumber
\end{align}

The optimal solution occurs where the objective and constraint functions
intersect at $\left(1,0\right)$. Relaxing the objective function by 10\% gives a
new constraint shown by the dashed line. Since the constraint is written as an
equality, the feasible space is given by all points on the constraint curve. The
new \ac{mga} solution is now at the intersection between the \ac{mga} constraint and the
original constraint, $\left(0.6, 0.4\right)$.

\begin{figure}[h]
  \centering
  \resizebox{0.75\columnwidth}{!}{%% Creator: Matplotlib, PGF backend
%%
%% To include the figure in your LaTeX document, write
%%   \input{<filename>.pgf}
%%
%% Make sure the required packages are loaded in your preamble
%%   \usepackage{pgf}
%%
%% Also ensure that all the required font packages are loaded; for instance,
%% the lmodern package is sometimes necessary when using math font.
%%   \usepackage{lmodern}
%%
%% Figures using additional raster images can only be included by \input if
%% they are in the same directory as the main LaTeX file. For loading figures
%% from other directories you can use the `import` package
%%   \usepackage{import}
%%
%% and then include the figures with
%%   \import{<path to file>}{<filename>.pgf}
%%
%% Matplotlib used the following preamble
%%   \def\mathdefault#1{#1}
%%   \everymath=\expandafter{\the\everymath\displaystyle}
%%   \IfFileExists{scrextend.sty}{
%%     \usepackage[fontsize=10.000000pt]{scrextend}
%%   }{
%%     \renewcommand{\normalsize}{\fontsize{10.000000}{12.000000}\selectfont}
%%     \normalsize
%%   }
%%   
%%   \ifdefined\pdftexversion\else  % non-pdftex case.
%%     \usepackage{fontspec}
%%     \setmainfont{DejaVuSerif.ttf}[Path=\detokenize{/Users/samdotson/miniforge3/envs/2025-dotson-thesis/lib/python3.11/site-packages/matplotlib/mpl-data/fonts/ttf/}]
%%     \setsansfont{DejaVuSans.ttf}[Path=\detokenize{/Users/samdotson/miniforge3/envs/2025-dotson-thesis/lib/python3.11/site-packages/matplotlib/mpl-data/fonts/ttf/}]
%%     \setmonofont{DejaVuSansMono.ttf}[Path=\detokenize{/Users/samdotson/miniforge3/envs/2025-dotson-thesis/lib/python3.11/site-packages/matplotlib/mpl-data/fonts/ttf/}]
%%   \fi
%%   \makeatletter\@ifpackageloaded{underscore}{}{\usepackage[strings]{underscore}}\makeatother
%%
\begingroup%
\makeatletter%
\begin{pgfpicture}%
\pgfpathrectangle{\pgfpointorigin}{\pgfqpoint{7.930676in}{5.968802in}}%
\pgfusepath{use as bounding box, clip}%
\begin{pgfscope}%
\pgfsetbuttcap%
\pgfsetmiterjoin%
\definecolor{currentfill}{rgb}{1.000000,1.000000,1.000000}%
\pgfsetfillcolor{currentfill}%
\pgfsetlinewidth{0.000000pt}%
\definecolor{currentstroke}{rgb}{0.000000,0.000000,0.000000}%
\pgfsetstrokecolor{currentstroke}%
\pgfsetdash{}{0pt}%
\pgfpathmoveto{\pgfqpoint{0.000000in}{0.000000in}}%
\pgfpathlineto{\pgfqpoint{7.930676in}{0.000000in}}%
\pgfpathlineto{\pgfqpoint{7.930676in}{5.968802in}}%
\pgfpathlineto{\pgfqpoint{0.000000in}{5.968802in}}%
\pgfpathlineto{\pgfqpoint{0.000000in}{0.000000in}}%
\pgfpathclose%
\pgfusepath{fill}%
\end{pgfscope}%
\begin{pgfscope}%
\pgfsetbuttcap%
\pgfsetmiterjoin%
\definecolor{currentfill}{rgb}{1.000000,1.000000,1.000000}%
\pgfsetfillcolor{currentfill}%
\pgfsetlinewidth{0.000000pt}%
\definecolor{currentstroke}{rgb}{0.000000,0.000000,0.000000}%
\pgfsetstrokecolor{currentstroke}%
\pgfsetstrokeopacity{0.000000}%
\pgfsetdash{}{0pt}%
\pgfpathmoveto{\pgfqpoint{0.771832in}{0.709782in}}%
\pgfpathlineto{\pgfqpoint{7.830676in}{0.709782in}}%
\pgfpathlineto{\pgfqpoint{7.830676in}{5.118077in}}%
\pgfpathlineto{\pgfqpoint{0.771832in}{5.118077in}}%
\pgfpathlineto{\pgfqpoint{0.771832in}{0.709782in}}%
\pgfpathclose%
\pgfusepath{fill}%
\end{pgfscope}%
\begin{pgfscope}%
\pgfpathrectangle{\pgfqpoint{0.771832in}{0.709782in}}{\pgfqpoint{7.058844in}{4.408296in}}%
\pgfusepath{clip}%
\pgfsetbuttcap%
\pgfsetroundjoin%
\definecolor{currentfill}{rgb}{0.000000,0.000000,0.000000}%
\pgfsetfillcolor{currentfill}%
\pgfsetlinewidth{1.003750pt}%
\definecolor{currentstroke}{rgb}{0.000000,0.000000,0.000000}%
\pgfsetstrokecolor{currentstroke}%
\pgfsetdash{}{0pt}%
\pgfsys@defobject{currentmarker}{\pgfqpoint{-0.065881in}{-0.065881in}}{\pgfqpoint{0.065881in}{0.065881in}}{%
\pgfpathmoveto{\pgfqpoint{0.000000in}{-0.065881in}}%
\pgfpathcurveto{\pgfqpoint{0.017472in}{-0.065881in}}{\pgfqpoint{0.034230in}{-0.058939in}}{\pgfqpoint{0.046585in}{-0.046585in}}%
\pgfpathcurveto{\pgfqpoint{0.058939in}{-0.034230in}}{\pgfqpoint{0.065881in}{-0.017472in}}{\pgfqpoint{0.065881in}{0.000000in}}%
\pgfpathcurveto{\pgfqpoint{0.065881in}{0.017472in}}{\pgfqpoint{0.058939in}{0.034230in}}{\pgfqpoint{0.046585in}{0.046585in}}%
\pgfpathcurveto{\pgfqpoint{0.034230in}{0.058939in}}{\pgfqpoint{0.017472in}{0.065881in}}{\pgfqpoint{0.000000in}{0.065881in}}%
\pgfpathcurveto{\pgfqpoint{-0.017472in}{0.065881in}}{\pgfqpoint{-0.034230in}{0.058939in}}{\pgfqpoint{-0.046585in}{0.046585in}}%
\pgfpathcurveto{\pgfqpoint{-0.058939in}{0.034230in}}{\pgfqpoint{-0.065881in}{0.017472in}}{\pgfqpoint{-0.065881in}{0.000000in}}%
\pgfpathcurveto{\pgfqpoint{-0.065881in}{-0.017472in}}{\pgfqpoint{-0.058939in}{-0.034230in}}{\pgfqpoint{-0.046585in}{-0.046585in}}%
\pgfpathcurveto{\pgfqpoint{-0.034230in}{-0.058939in}}{\pgfqpoint{-0.017472in}{-0.065881in}}{\pgfqpoint{0.000000in}{-0.065881in}}%
\pgfpathlineto{\pgfqpoint{0.000000in}{-0.065881in}}%
\pgfpathclose%
\pgfusepath{stroke,fill}%
}%
\begin{pgfscope}%
\pgfsys@transformshift{4.622110in}{2.312798in}%
\pgfsys@useobject{currentmarker}{}%
\end{pgfscope}%
\end{pgfscope}%
\begin{pgfscope}%
\pgfpathrectangle{\pgfqpoint{0.771832in}{0.709782in}}{\pgfqpoint{7.058844in}{4.408296in}}%
\pgfusepath{clip}%
\pgfsetbuttcap%
\pgfsetroundjoin%
\definecolor{currentfill}{rgb}{1.000000,1.000000,1.000000}%
\pgfsetfillcolor{currentfill}%
\pgfsetlinewidth{1.003750pt}%
\definecolor{currentstroke}{rgb}{0.000000,0.000000,0.000000}%
\pgfsetstrokecolor{currentstroke}%
\pgfsetdash{}{0pt}%
\pgfsys@defobject{currentmarker}{\pgfqpoint{-0.065881in}{-0.065881in}}{\pgfqpoint{0.065881in}{0.065881in}}{%
\pgfpathmoveto{\pgfqpoint{0.000000in}{-0.065881in}}%
\pgfpathcurveto{\pgfqpoint{0.017472in}{-0.065881in}}{\pgfqpoint{0.034230in}{-0.058939in}}{\pgfqpoint{0.046585in}{-0.046585in}}%
\pgfpathcurveto{\pgfqpoint{0.058939in}{-0.034230in}}{\pgfqpoint{0.065881in}{-0.017472in}}{\pgfqpoint{0.065881in}{0.000000in}}%
\pgfpathcurveto{\pgfqpoint{0.065881in}{0.017472in}}{\pgfqpoint{0.058939in}{0.034230in}}{\pgfqpoint{0.046585in}{0.046585in}}%
\pgfpathcurveto{\pgfqpoint{0.034230in}{0.058939in}}{\pgfqpoint{0.017472in}{0.065881in}}{\pgfqpoint{0.000000in}{0.065881in}}%
\pgfpathcurveto{\pgfqpoint{-0.017472in}{0.065881in}}{\pgfqpoint{-0.034230in}{0.058939in}}{\pgfqpoint{-0.046585in}{0.046585in}}%
\pgfpathcurveto{\pgfqpoint{-0.058939in}{0.034230in}}{\pgfqpoint{-0.065881in}{0.017472in}}{\pgfqpoint{-0.065881in}{0.000000in}}%
\pgfpathcurveto{\pgfqpoint{-0.065881in}{-0.017472in}}{\pgfqpoint{-0.058939in}{-0.034230in}}{\pgfqpoint{-0.046585in}{-0.046585in}}%
\pgfpathcurveto{\pgfqpoint{-0.034230in}{-0.058939in}}{\pgfqpoint{-0.017472in}{-0.065881in}}{\pgfqpoint{0.000000in}{-0.065881in}}%
\pgfpathlineto{\pgfqpoint{0.000000in}{-0.065881in}}%
\pgfpathclose%
\pgfusepath{stroke,fill}%
}%
\begin{pgfscope}%
\pgfsys@transformshift{7.188963in}{0.709782in}%
\pgfsys@useobject{currentmarker}{}%
\end{pgfscope}%
\end{pgfscope}%
\begin{pgfscope}%
\pgfpathrectangle{\pgfqpoint{0.771832in}{0.709782in}}{\pgfqpoint{7.058844in}{4.408296in}}%
\pgfusepath{clip}%
\pgfsetrectcap%
\pgfsetroundjoin%
\pgfsetlinewidth{0.803000pt}%
\definecolor{currentstroke}{rgb}{0.501961,0.501961,0.501961}%
\pgfsetstrokecolor{currentstroke}%
\pgfsetstrokeopacity{0.300000}%
\pgfsetdash{}{0pt}%
\pgfpathmoveto{\pgfqpoint{0.771832in}{0.709782in}}%
\pgfpathlineto{\pgfqpoint{0.771832in}{5.118077in}}%
\pgfusepath{stroke}%
\end{pgfscope}%
\begin{pgfscope}%
\pgfsetbuttcap%
\pgfsetroundjoin%
\definecolor{currentfill}{rgb}{0.000000,0.000000,0.000000}%
\pgfsetfillcolor{currentfill}%
\pgfsetlinewidth{0.803000pt}%
\definecolor{currentstroke}{rgb}{0.000000,0.000000,0.000000}%
\pgfsetstrokecolor{currentstroke}%
\pgfsetdash{}{0pt}%
\pgfsys@defobject{currentmarker}{\pgfqpoint{0.000000in}{-0.048611in}}{\pgfqpoint{0.000000in}{0.000000in}}{%
\pgfpathmoveto{\pgfqpoint{0.000000in}{0.000000in}}%
\pgfpathlineto{\pgfqpoint{0.000000in}{-0.048611in}}%
\pgfusepath{stroke,fill}%
}%
\begin{pgfscope}%
\pgfsys@transformshift{0.771832in}{0.709782in}%
\pgfsys@useobject{currentmarker}{}%
\end{pgfscope}%
\end{pgfscope}%
\begin{pgfscope}%
\definecolor{textcolor}{rgb}{0.000000,0.000000,0.000000}%
\pgfsetstrokecolor{textcolor}%
\pgfsetfillcolor{textcolor}%
\pgftext[x=0.771832in,y=0.612559in,,top]{\color{textcolor}{\rmfamily\fontsize{14.000000}{16.800000}\selectfont\catcode`\^=\active\def^{\ifmmode\sp\else\^{}\fi}\catcode`\%=\active\def%{\%}$\mathdefault{0.0}$}}%
\end{pgfscope}%
\begin{pgfscope}%
\pgfpathrectangle{\pgfqpoint{0.771832in}{0.709782in}}{\pgfqpoint{7.058844in}{4.408296in}}%
\pgfusepath{clip}%
\pgfsetrectcap%
\pgfsetroundjoin%
\pgfsetlinewidth{0.803000pt}%
\definecolor{currentstroke}{rgb}{0.501961,0.501961,0.501961}%
\pgfsetstrokecolor{currentstroke}%
\pgfsetstrokeopacity{0.300000}%
\pgfsetdash{}{0pt}%
\pgfpathmoveto{\pgfqpoint{2.055258in}{0.709782in}}%
\pgfpathlineto{\pgfqpoint{2.055258in}{5.118077in}}%
\pgfusepath{stroke}%
\end{pgfscope}%
\begin{pgfscope}%
\pgfsetbuttcap%
\pgfsetroundjoin%
\definecolor{currentfill}{rgb}{0.000000,0.000000,0.000000}%
\pgfsetfillcolor{currentfill}%
\pgfsetlinewidth{0.803000pt}%
\definecolor{currentstroke}{rgb}{0.000000,0.000000,0.000000}%
\pgfsetstrokecolor{currentstroke}%
\pgfsetdash{}{0pt}%
\pgfsys@defobject{currentmarker}{\pgfqpoint{0.000000in}{-0.048611in}}{\pgfqpoint{0.000000in}{0.000000in}}{%
\pgfpathmoveto{\pgfqpoint{0.000000in}{0.000000in}}%
\pgfpathlineto{\pgfqpoint{0.000000in}{-0.048611in}}%
\pgfusepath{stroke,fill}%
}%
\begin{pgfscope}%
\pgfsys@transformshift{2.055258in}{0.709782in}%
\pgfsys@useobject{currentmarker}{}%
\end{pgfscope}%
\end{pgfscope}%
\begin{pgfscope}%
\definecolor{textcolor}{rgb}{0.000000,0.000000,0.000000}%
\pgfsetstrokecolor{textcolor}%
\pgfsetfillcolor{textcolor}%
\pgftext[x=2.055258in,y=0.612559in,,top]{\color{textcolor}{\rmfamily\fontsize{14.000000}{16.800000}\selectfont\catcode`\^=\active\def^{\ifmmode\sp\else\^{}\fi}\catcode`\%=\active\def%{\%}$\mathdefault{0.2}$}}%
\end{pgfscope}%
\begin{pgfscope}%
\pgfpathrectangle{\pgfqpoint{0.771832in}{0.709782in}}{\pgfqpoint{7.058844in}{4.408296in}}%
\pgfusepath{clip}%
\pgfsetrectcap%
\pgfsetroundjoin%
\pgfsetlinewidth{0.803000pt}%
\definecolor{currentstroke}{rgb}{0.501961,0.501961,0.501961}%
\pgfsetstrokecolor{currentstroke}%
\pgfsetstrokeopacity{0.300000}%
\pgfsetdash{}{0pt}%
\pgfpathmoveto{\pgfqpoint{3.338684in}{0.709782in}}%
\pgfpathlineto{\pgfqpoint{3.338684in}{5.118077in}}%
\pgfusepath{stroke}%
\end{pgfscope}%
\begin{pgfscope}%
\pgfsetbuttcap%
\pgfsetroundjoin%
\definecolor{currentfill}{rgb}{0.000000,0.000000,0.000000}%
\pgfsetfillcolor{currentfill}%
\pgfsetlinewidth{0.803000pt}%
\definecolor{currentstroke}{rgb}{0.000000,0.000000,0.000000}%
\pgfsetstrokecolor{currentstroke}%
\pgfsetdash{}{0pt}%
\pgfsys@defobject{currentmarker}{\pgfqpoint{0.000000in}{-0.048611in}}{\pgfqpoint{0.000000in}{0.000000in}}{%
\pgfpathmoveto{\pgfqpoint{0.000000in}{0.000000in}}%
\pgfpathlineto{\pgfqpoint{0.000000in}{-0.048611in}}%
\pgfusepath{stroke,fill}%
}%
\begin{pgfscope}%
\pgfsys@transformshift{3.338684in}{0.709782in}%
\pgfsys@useobject{currentmarker}{}%
\end{pgfscope}%
\end{pgfscope}%
\begin{pgfscope}%
\definecolor{textcolor}{rgb}{0.000000,0.000000,0.000000}%
\pgfsetstrokecolor{textcolor}%
\pgfsetfillcolor{textcolor}%
\pgftext[x=3.338684in,y=0.612559in,,top]{\color{textcolor}{\rmfamily\fontsize{14.000000}{16.800000}\selectfont\catcode`\^=\active\def^{\ifmmode\sp\else\^{}\fi}\catcode`\%=\active\def%{\%}$\mathdefault{0.4}$}}%
\end{pgfscope}%
\begin{pgfscope}%
\pgfpathrectangle{\pgfqpoint{0.771832in}{0.709782in}}{\pgfqpoint{7.058844in}{4.408296in}}%
\pgfusepath{clip}%
\pgfsetrectcap%
\pgfsetroundjoin%
\pgfsetlinewidth{0.803000pt}%
\definecolor{currentstroke}{rgb}{0.501961,0.501961,0.501961}%
\pgfsetstrokecolor{currentstroke}%
\pgfsetstrokeopacity{0.300000}%
\pgfsetdash{}{0pt}%
\pgfpathmoveto{\pgfqpoint{4.622110in}{0.709782in}}%
\pgfpathlineto{\pgfqpoint{4.622110in}{5.118077in}}%
\pgfusepath{stroke}%
\end{pgfscope}%
\begin{pgfscope}%
\pgfsetbuttcap%
\pgfsetroundjoin%
\definecolor{currentfill}{rgb}{0.000000,0.000000,0.000000}%
\pgfsetfillcolor{currentfill}%
\pgfsetlinewidth{0.803000pt}%
\definecolor{currentstroke}{rgb}{0.000000,0.000000,0.000000}%
\pgfsetstrokecolor{currentstroke}%
\pgfsetdash{}{0pt}%
\pgfsys@defobject{currentmarker}{\pgfqpoint{0.000000in}{-0.048611in}}{\pgfqpoint{0.000000in}{0.000000in}}{%
\pgfpathmoveto{\pgfqpoint{0.000000in}{0.000000in}}%
\pgfpathlineto{\pgfqpoint{0.000000in}{-0.048611in}}%
\pgfusepath{stroke,fill}%
}%
\begin{pgfscope}%
\pgfsys@transformshift{4.622110in}{0.709782in}%
\pgfsys@useobject{currentmarker}{}%
\end{pgfscope}%
\end{pgfscope}%
\begin{pgfscope}%
\definecolor{textcolor}{rgb}{0.000000,0.000000,0.000000}%
\pgfsetstrokecolor{textcolor}%
\pgfsetfillcolor{textcolor}%
\pgftext[x=4.622110in,y=0.612559in,,top]{\color{textcolor}{\rmfamily\fontsize{14.000000}{16.800000}\selectfont\catcode`\^=\active\def^{\ifmmode\sp\else\^{}\fi}\catcode`\%=\active\def%{\%}$\mathdefault{0.6}$}}%
\end{pgfscope}%
\begin{pgfscope}%
\pgfpathrectangle{\pgfqpoint{0.771832in}{0.709782in}}{\pgfqpoint{7.058844in}{4.408296in}}%
\pgfusepath{clip}%
\pgfsetrectcap%
\pgfsetroundjoin%
\pgfsetlinewidth{0.803000pt}%
\definecolor{currentstroke}{rgb}{0.501961,0.501961,0.501961}%
\pgfsetstrokecolor{currentstroke}%
\pgfsetstrokeopacity{0.300000}%
\pgfsetdash{}{0pt}%
\pgfpathmoveto{\pgfqpoint{5.905537in}{0.709782in}}%
\pgfpathlineto{\pgfqpoint{5.905537in}{5.118077in}}%
\pgfusepath{stroke}%
\end{pgfscope}%
\begin{pgfscope}%
\pgfsetbuttcap%
\pgfsetroundjoin%
\definecolor{currentfill}{rgb}{0.000000,0.000000,0.000000}%
\pgfsetfillcolor{currentfill}%
\pgfsetlinewidth{0.803000pt}%
\definecolor{currentstroke}{rgb}{0.000000,0.000000,0.000000}%
\pgfsetstrokecolor{currentstroke}%
\pgfsetdash{}{0pt}%
\pgfsys@defobject{currentmarker}{\pgfqpoint{0.000000in}{-0.048611in}}{\pgfqpoint{0.000000in}{0.000000in}}{%
\pgfpathmoveto{\pgfqpoint{0.000000in}{0.000000in}}%
\pgfpathlineto{\pgfqpoint{0.000000in}{-0.048611in}}%
\pgfusepath{stroke,fill}%
}%
\begin{pgfscope}%
\pgfsys@transformshift{5.905537in}{0.709782in}%
\pgfsys@useobject{currentmarker}{}%
\end{pgfscope}%
\end{pgfscope}%
\begin{pgfscope}%
\definecolor{textcolor}{rgb}{0.000000,0.000000,0.000000}%
\pgfsetstrokecolor{textcolor}%
\pgfsetfillcolor{textcolor}%
\pgftext[x=5.905537in,y=0.612559in,,top]{\color{textcolor}{\rmfamily\fontsize{14.000000}{16.800000}\selectfont\catcode`\^=\active\def^{\ifmmode\sp\else\^{}\fi}\catcode`\%=\active\def%{\%}$\mathdefault{0.8}$}}%
\end{pgfscope}%
\begin{pgfscope}%
\pgfpathrectangle{\pgfqpoint{0.771832in}{0.709782in}}{\pgfqpoint{7.058844in}{4.408296in}}%
\pgfusepath{clip}%
\pgfsetrectcap%
\pgfsetroundjoin%
\pgfsetlinewidth{0.803000pt}%
\definecolor{currentstroke}{rgb}{0.501961,0.501961,0.501961}%
\pgfsetstrokecolor{currentstroke}%
\pgfsetstrokeopacity{0.300000}%
\pgfsetdash{}{0pt}%
\pgfpathmoveto{\pgfqpoint{7.188963in}{0.709782in}}%
\pgfpathlineto{\pgfqpoint{7.188963in}{5.118077in}}%
\pgfusepath{stroke}%
\end{pgfscope}%
\begin{pgfscope}%
\pgfsetbuttcap%
\pgfsetroundjoin%
\definecolor{currentfill}{rgb}{0.000000,0.000000,0.000000}%
\pgfsetfillcolor{currentfill}%
\pgfsetlinewidth{0.803000pt}%
\definecolor{currentstroke}{rgb}{0.000000,0.000000,0.000000}%
\pgfsetstrokecolor{currentstroke}%
\pgfsetdash{}{0pt}%
\pgfsys@defobject{currentmarker}{\pgfqpoint{0.000000in}{-0.048611in}}{\pgfqpoint{0.000000in}{0.000000in}}{%
\pgfpathmoveto{\pgfqpoint{0.000000in}{0.000000in}}%
\pgfpathlineto{\pgfqpoint{0.000000in}{-0.048611in}}%
\pgfusepath{stroke,fill}%
}%
\begin{pgfscope}%
\pgfsys@transformshift{7.188963in}{0.709782in}%
\pgfsys@useobject{currentmarker}{}%
\end{pgfscope}%
\end{pgfscope}%
\begin{pgfscope}%
\definecolor{textcolor}{rgb}{0.000000,0.000000,0.000000}%
\pgfsetstrokecolor{textcolor}%
\pgfsetfillcolor{textcolor}%
\pgftext[x=7.188963in,y=0.612559in,,top]{\color{textcolor}{\rmfamily\fontsize{14.000000}{16.800000}\selectfont\catcode`\^=\active\def^{\ifmmode\sp\else\^{}\fi}\catcode`\%=\active\def%{\%}$\mathdefault{1.0}$}}%
\end{pgfscope}%
\begin{pgfscope}%
\definecolor{textcolor}{rgb}{0.000000,0.000000,0.000000}%
\pgfsetstrokecolor{textcolor}%
\pgfsetfillcolor{textcolor}%
\pgftext[x=4.301254in,y=0.368826in,,top]{\color{textcolor}{\rmfamily\fontsize{20.000000}{24.000000}\selectfont\catcode`\^=\active\def^{\ifmmode\sp\else\^{}\fi}\catcode`\%=\active\def%{\%}x$_1$}}%
\end{pgfscope}%
\begin{pgfscope}%
\pgfpathrectangle{\pgfqpoint{0.771832in}{0.709782in}}{\pgfqpoint{7.058844in}{4.408296in}}%
\pgfusepath{clip}%
\pgfsetrectcap%
\pgfsetroundjoin%
\pgfsetlinewidth{0.803000pt}%
\definecolor{currentstroke}{rgb}{0.501961,0.501961,0.501961}%
\pgfsetstrokecolor{currentstroke}%
\pgfsetstrokeopacity{0.300000}%
\pgfsetdash{}{0pt}%
\pgfpathmoveto{\pgfqpoint{0.771832in}{0.709782in}}%
\pgfpathlineto{\pgfqpoint{7.830676in}{0.709782in}}%
\pgfusepath{stroke}%
\end{pgfscope}%
\begin{pgfscope}%
\pgfsetbuttcap%
\pgfsetroundjoin%
\definecolor{currentfill}{rgb}{0.000000,0.000000,0.000000}%
\pgfsetfillcolor{currentfill}%
\pgfsetlinewidth{0.803000pt}%
\definecolor{currentstroke}{rgb}{0.000000,0.000000,0.000000}%
\pgfsetstrokecolor{currentstroke}%
\pgfsetdash{}{0pt}%
\pgfsys@defobject{currentmarker}{\pgfqpoint{-0.048611in}{0.000000in}}{\pgfqpoint{-0.000000in}{0.000000in}}{%
\pgfpathmoveto{\pgfqpoint{-0.000000in}{0.000000in}}%
\pgfpathlineto{\pgfqpoint{-0.048611in}{0.000000in}}%
\pgfusepath{stroke,fill}%
}%
\begin{pgfscope}%
\pgfsys@transformshift{0.771832in}{0.709782in}%
\pgfsys@useobject{currentmarker}{}%
\end{pgfscope}%
\end{pgfscope}%
\begin{pgfscope}%
\definecolor{textcolor}{rgb}{0.000000,0.000000,0.000000}%
\pgfsetstrokecolor{textcolor}%
\pgfsetfillcolor{textcolor}%
\pgftext[x=0.424381in, y=0.635915in, left, base]{\color{textcolor}{\rmfamily\fontsize{14.000000}{16.800000}\selectfont\catcode`\^=\active\def^{\ifmmode\sp\else\^{}\fi}\catcode`\%=\active\def%{\%}$\mathdefault{0.0}$}}%
\end{pgfscope}%
\begin{pgfscope}%
\pgfpathrectangle{\pgfqpoint{0.771832in}{0.709782in}}{\pgfqpoint{7.058844in}{4.408296in}}%
\pgfusepath{clip}%
\pgfsetrectcap%
\pgfsetroundjoin%
\pgfsetlinewidth{0.803000pt}%
\definecolor{currentstroke}{rgb}{0.501961,0.501961,0.501961}%
\pgfsetstrokecolor{currentstroke}%
\pgfsetstrokeopacity{0.300000}%
\pgfsetdash{}{0pt}%
\pgfpathmoveto{\pgfqpoint{0.771832in}{1.511290in}}%
\pgfpathlineto{\pgfqpoint{7.830676in}{1.511290in}}%
\pgfusepath{stroke}%
\end{pgfscope}%
\begin{pgfscope}%
\pgfsetbuttcap%
\pgfsetroundjoin%
\definecolor{currentfill}{rgb}{0.000000,0.000000,0.000000}%
\pgfsetfillcolor{currentfill}%
\pgfsetlinewidth{0.803000pt}%
\definecolor{currentstroke}{rgb}{0.000000,0.000000,0.000000}%
\pgfsetstrokecolor{currentstroke}%
\pgfsetdash{}{0pt}%
\pgfsys@defobject{currentmarker}{\pgfqpoint{-0.048611in}{0.000000in}}{\pgfqpoint{-0.000000in}{0.000000in}}{%
\pgfpathmoveto{\pgfqpoint{-0.000000in}{0.000000in}}%
\pgfpathlineto{\pgfqpoint{-0.048611in}{0.000000in}}%
\pgfusepath{stroke,fill}%
}%
\begin{pgfscope}%
\pgfsys@transformshift{0.771832in}{1.511290in}%
\pgfsys@useobject{currentmarker}{}%
\end{pgfscope}%
\end{pgfscope}%
\begin{pgfscope}%
\definecolor{textcolor}{rgb}{0.000000,0.000000,0.000000}%
\pgfsetstrokecolor{textcolor}%
\pgfsetfillcolor{textcolor}%
\pgftext[x=0.424381in, y=1.437424in, left, base]{\color{textcolor}{\rmfamily\fontsize{14.000000}{16.800000}\selectfont\catcode`\^=\active\def^{\ifmmode\sp\else\^{}\fi}\catcode`\%=\active\def%{\%}$\mathdefault{0.2}$}}%
\end{pgfscope}%
\begin{pgfscope}%
\pgfpathrectangle{\pgfqpoint{0.771832in}{0.709782in}}{\pgfqpoint{7.058844in}{4.408296in}}%
\pgfusepath{clip}%
\pgfsetrectcap%
\pgfsetroundjoin%
\pgfsetlinewidth{0.803000pt}%
\definecolor{currentstroke}{rgb}{0.501961,0.501961,0.501961}%
\pgfsetstrokecolor{currentstroke}%
\pgfsetstrokeopacity{0.300000}%
\pgfsetdash{}{0pt}%
\pgfpathmoveto{\pgfqpoint{0.771832in}{2.312798in}}%
\pgfpathlineto{\pgfqpoint{7.830676in}{2.312798in}}%
\pgfusepath{stroke}%
\end{pgfscope}%
\begin{pgfscope}%
\pgfsetbuttcap%
\pgfsetroundjoin%
\definecolor{currentfill}{rgb}{0.000000,0.000000,0.000000}%
\pgfsetfillcolor{currentfill}%
\pgfsetlinewidth{0.803000pt}%
\definecolor{currentstroke}{rgb}{0.000000,0.000000,0.000000}%
\pgfsetstrokecolor{currentstroke}%
\pgfsetdash{}{0pt}%
\pgfsys@defobject{currentmarker}{\pgfqpoint{-0.048611in}{0.000000in}}{\pgfqpoint{-0.000000in}{0.000000in}}{%
\pgfpathmoveto{\pgfqpoint{-0.000000in}{0.000000in}}%
\pgfpathlineto{\pgfqpoint{-0.048611in}{0.000000in}}%
\pgfusepath{stroke,fill}%
}%
\begin{pgfscope}%
\pgfsys@transformshift{0.771832in}{2.312798in}%
\pgfsys@useobject{currentmarker}{}%
\end{pgfscope}%
\end{pgfscope}%
\begin{pgfscope}%
\definecolor{textcolor}{rgb}{0.000000,0.000000,0.000000}%
\pgfsetstrokecolor{textcolor}%
\pgfsetfillcolor{textcolor}%
\pgftext[x=0.424381in, y=2.238932in, left, base]{\color{textcolor}{\rmfamily\fontsize{14.000000}{16.800000}\selectfont\catcode`\^=\active\def^{\ifmmode\sp\else\^{}\fi}\catcode`\%=\active\def%{\%}$\mathdefault{0.4}$}}%
\end{pgfscope}%
\begin{pgfscope}%
\pgfpathrectangle{\pgfqpoint{0.771832in}{0.709782in}}{\pgfqpoint{7.058844in}{4.408296in}}%
\pgfusepath{clip}%
\pgfsetrectcap%
\pgfsetroundjoin%
\pgfsetlinewidth{0.803000pt}%
\definecolor{currentstroke}{rgb}{0.501961,0.501961,0.501961}%
\pgfsetstrokecolor{currentstroke}%
\pgfsetstrokeopacity{0.300000}%
\pgfsetdash{}{0pt}%
\pgfpathmoveto{\pgfqpoint{0.771832in}{3.114307in}}%
\pgfpathlineto{\pgfqpoint{7.830676in}{3.114307in}}%
\pgfusepath{stroke}%
\end{pgfscope}%
\begin{pgfscope}%
\pgfsetbuttcap%
\pgfsetroundjoin%
\definecolor{currentfill}{rgb}{0.000000,0.000000,0.000000}%
\pgfsetfillcolor{currentfill}%
\pgfsetlinewidth{0.803000pt}%
\definecolor{currentstroke}{rgb}{0.000000,0.000000,0.000000}%
\pgfsetstrokecolor{currentstroke}%
\pgfsetdash{}{0pt}%
\pgfsys@defobject{currentmarker}{\pgfqpoint{-0.048611in}{0.000000in}}{\pgfqpoint{-0.000000in}{0.000000in}}{%
\pgfpathmoveto{\pgfqpoint{-0.000000in}{0.000000in}}%
\pgfpathlineto{\pgfqpoint{-0.048611in}{0.000000in}}%
\pgfusepath{stroke,fill}%
}%
\begin{pgfscope}%
\pgfsys@transformshift{0.771832in}{3.114307in}%
\pgfsys@useobject{currentmarker}{}%
\end{pgfscope}%
\end{pgfscope}%
\begin{pgfscope}%
\definecolor{textcolor}{rgb}{0.000000,0.000000,0.000000}%
\pgfsetstrokecolor{textcolor}%
\pgfsetfillcolor{textcolor}%
\pgftext[x=0.424381in, y=3.040440in, left, base]{\color{textcolor}{\rmfamily\fontsize{14.000000}{16.800000}\selectfont\catcode`\^=\active\def^{\ifmmode\sp\else\^{}\fi}\catcode`\%=\active\def%{\%}$\mathdefault{0.6}$}}%
\end{pgfscope}%
\begin{pgfscope}%
\pgfpathrectangle{\pgfqpoint{0.771832in}{0.709782in}}{\pgfqpoint{7.058844in}{4.408296in}}%
\pgfusepath{clip}%
\pgfsetrectcap%
\pgfsetroundjoin%
\pgfsetlinewidth{0.803000pt}%
\definecolor{currentstroke}{rgb}{0.501961,0.501961,0.501961}%
\pgfsetstrokecolor{currentstroke}%
\pgfsetstrokeopacity{0.300000}%
\pgfsetdash{}{0pt}%
\pgfpathmoveto{\pgfqpoint{0.771832in}{3.915815in}}%
\pgfpathlineto{\pgfqpoint{7.830676in}{3.915815in}}%
\pgfusepath{stroke}%
\end{pgfscope}%
\begin{pgfscope}%
\pgfsetbuttcap%
\pgfsetroundjoin%
\definecolor{currentfill}{rgb}{0.000000,0.000000,0.000000}%
\pgfsetfillcolor{currentfill}%
\pgfsetlinewidth{0.803000pt}%
\definecolor{currentstroke}{rgb}{0.000000,0.000000,0.000000}%
\pgfsetstrokecolor{currentstroke}%
\pgfsetdash{}{0pt}%
\pgfsys@defobject{currentmarker}{\pgfqpoint{-0.048611in}{0.000000in}}{\pgfqpoint{-0.000000in}{0.000000in}}{%
\pgfpathmoveto{\pgfqpoint{-0.000000in}{0.000000in}}%
\pgfpathlineto{\pgfqpoint{-0.048611in}{0.000000in}}%
\pgfusepath{stroke,fill}%
}%
\begin{pgfscope}%
\pgfsys@transformshift{0.771832in}{3.915815in}%
\pgfsys@useobject{currentmarker}{}%
\end{pgfscope}%
\end{pgfscope}%
\begin{pgfscope}%
\definecolor{textcolor}{rgb}{0.000000,0.000000,0.000000}%
\pgfsetstrokecolor{textcolor}%
\pgfsetfillcolor{textcolor}%
\pgftext[x=0.424381in, y=3.841949in, left, base]{\color{textcolor}{\rmfamily\fontsize{14.000000}{16.800000}\selectfont\catcode`\^=\active\def^{\ifmmode\sp\else\^{}\fi}\catcode`\%=\active\def%{\%}$\mathdefault{0.8}$}}%
\end{pgfscope}%
\begin{pgfscope}%
\pgfpathrectangle{\pgfqpoint{0.771832in}{0.709782in}}{\pgfqpoint{7.058844in}{4.408296in}}%
\pgfusepath{clip}%
\pgfsetrectcap%
\pgfsetroundjoin%
\pgfsetlinewidth{0.803000pt}%
\definecolor{currentstroke}{rgb}{0.501961,0.501961,0.501961}%
\pgfsetstrokecolor{currentstroke}%
\pgfsetstrokeopacity{0.300000}%
\pgfsetdash{}{0pt}%
\pgfpathmoveto{\pgfqpoint{0.771832in}{4.717323in}}%
\pgfpathlineto{\pgfqpoint{7.830676in}{4.717323in}}%
\pgfusepath{stroke}%
\end{pgfscope}%
\begin{pgfscope}%
\pgfsetbuttcap%
\pgfsetroundjoin%
\definecolor{currentfill}{rgb}{0.000000,0.000000,0.000000}%
\pgfsetfillcolor{currentfill}%
\pgfsetlinewidth{0.803000pt}%
\definecolor{currentstroke}{rgb}{0.000000,0.000000,0.000000}%
\pgfsetstrokecolor{currentstroke}%
\pgfsetdash{}{0pt}%
\pgfsys@defobject{currentmarker}{\pgfqpoint{-0.048611in}{0.000000in}}{\pgfqpoint{-0.000000in}{0.000000in}}{%
\pgfpathmoveto{\pgfqpoint{-0.000000in}{0.000000in}}%
\pgfpathlineto{\pgfqpoint{-0.048611in}{0.000000in}}%
\pgfusepath{stroke,fill}%
}%
\begin{pgfscope}%
\pgfsys@transformshift{0.771832in}{4.717323in}%
\pgfsys@useobject{currentmarker}{}%
\end{pgfscope}%
\end{pgfscope}%
\begin{pgfscope}%
\definecolor{textcolor}{rgb}{0.000000,0.000000,0.000000}%
\pgfsetstrokecolor{textcolor}%
\pgfsetfillcolor{textcolor}%
\pgftext[x=0.424381in, y=4.643457in, left, base]{\color{textcolor}{\rmfamily\fontsize{14.000000}{16.800000}\selectfont\catcode`\^=\active\def^{\ifmmode\sp\else\^{}\fi}\catcode`\%=\active\def%{\%}$\mathdefault{1.0}$}}%
\end{pgfscope}%
\begin{pgfscope}%
\definecolor{textcolor}{rgb}{0.000000,0.000000,0.000000}%
\pgfsetstrokecolor{textcolor}%
\pgfsetfillcolor{textcolor}%
\pgftext[x=0.368826in,y=2.913929in,,bottom,rotate=90.000000]{\color{textcolor}{\rmfamily\fontsize{20.000000}{24.000000}\selectfont\catcode`\^=\active\def^{\ifmmode\sp\else\^{}\fi}\catcode`\%=\active\def%{\%}x$_2$}}%
\end{pgfscope}%
\begin{pgfscope}%
\pgfpathrectangle{\pgfqpoint{0.771832in}{0.709782in}}{\pgfqpoint{7.058844in}{4.408296in}}%
\pgfusepath{clip}%
\pgfsetrectcap%
\pgfsetroundjoin%
\pgfsetlinewidth{1.505625pt}%
\definecolor{currentstroke}{rgb}{0.000000,0.000000,0.000000}%
\pgfsetstrokecolor{currentstroke}%
\pgfsetdash{}{0pt}%
\pgfpathmoveto{\pgfqpoint{7.188963in}{0.709782in}}%
\pgfpathlineto{\pgfqpoint{0.771832in}{4.717323in}}%
\pgfusepath{stroke}%
\end{pgfscope}%
\begin{pgfscope}%
\pgfpathrectangle{\pgfqpoint{0.771832in}{0.709782in}}{\pgfqpoint{7.058844in}{4.408296in}}%
\pgfusepath{clip}%
\pgfsetrectcap%
\pgfsetroundjoin%
\pgfsetlinewidth{1.505625pt}%
\definecolor{currentstroke}{rgb}{0.501961,0.501961,0.501961}%
\pgfsetstrokecolor{currentstroke}%
\pgfsetdash{}{0pt}%
\pgfpathmoveto{\pgfqpoint{7.188963in}{0.709782in}}%
\pgfpathlineto{\pgfqpoint{0.771832in}{3.915815in}}%
\pgfusepath{stroke}%
\end{pgfscope}%
\begin{pgfscope}%
\pgfpathrectangle{\pgfqpoint{0.771832in}{0.709782in}}{\pgfqpoint{7.058844in}{4.408296in}}%
\pgfusepath{clip}%
\pgfsetbuttcap%
\pgfsetroundjoin%
\pgfsetlinewidth{1.505625pt}%
\definecolor{currentstroke}{rgb}{0.501961,0.501961,0.501961}%
\pgfsetstrokecolor{currentstroke}%
\pgfsetdash{{5.550000pt}{2.400000pt}}{0.000000pt}%
\pgfpathmoveto{\pgfqpoint{7.830676in}{0.709782in}}%
\pgfpathlineto{\pgfqpoint{0.771832in}{4.236418in}}%
\pgfusepath{stroke}%
\end{pgfscope}%
\begin{pgfscope}%
\pgfsetrectcap%
\pgfsetmiterjoin%
\pgfsetlinewidth{0.803000pt}%
\definecolor{currentstroke}{rgb}{0.000000,0.000000,0.000000}%
\pgfsetstrokecolor{currentstroke}%
\pgfsetdash{}{0pt}%
\pgfpathmoveto{\pgfqpoint{0.771832in}{0.709782in}}%
\pgfpathlineto{\pgfqpoint{0.771832in}{5.118077in}}%
\pgfusepath{stroke}%
\end{pgfscope}%
\begin{pgfscope}%
\pgfsetrectcap%
\pgfsetmiterjoin%
\pgfsetlinewidth{0.803000pt}%
\definecolor{currentstroke}{rgb}{0.000000,0.000000,0.000000}%
\pgfsetstrokecolor{currentstroke}%
\pgfsetdash{}{0pt}%
\pgfpathmoveto{\pgfqpoint{7.830676in}{0.709782in}}%
\pgfpathlineto{\pgfqpoint{7.830676in}{5.118077in}}%
\pgfusepath{stroke}%
\end{pgfscope}%
\begin{pgfscope}%
\pgfsetrectcap%
\pgfsetmiterjoin%
\pgfsetlinewidth{0.803000pt}%
\definecolor{currentstroke}{rgb}{0.000000,0.000000,0.000000}%
\pgfsetstrokecolor{currentstroke}%
\pgfsetdash{}{0pt}%
\pgfpathmoveto{\pgfqpoint{0.771832in}{0.709782in}}%
\pgfpathlineto{\pgfqpoint{7.830676in}{0.709782in}}%
\pgfusepath{stroke}%
\end{pgfscope}%
\begin{pgfscope}%
\pgfsetrectcap%
\pgfsetmiterjoin%
\pgfsetlinewidth{0.803000pt}%
\definecolor{currentstroke}{rgb}{0.000000,0.000000,0.000000}%
\pgfsetstrokecolor{currentstroke}%
\pgfsetdash{}{0pt}%
\pgfpathmoveto{\pgfqpoint{0.771832in}{5.118077in}}%
\pgfpathlineto{\pgfqpoint{7.830676in}{5.118077in}}%
\pgfusepath{stroke}%
\end{pgfscope}%
\begin{pgfscope}%
\pgfsetroundcap%
\pgfsetroundjoin%
\pgfsetlinewidth{1.003750pt}%
\definecolor{currentstroke}{rgb}{0.000000,0.000000,0.000000}%
\pgfsetstrokecolor{currentstroke}%
\pgfsetdash{}{0pt}%
\pgfpathmoveto{\pgfqpoint{5.449588in}{1.037174in}}%
\pgfpathquadraticcurveto{\pgfqpoint{6.305624in}{0.876048in}}{\pgfqpoint{7.151295in}{0.716871in}}%
\pgfusepath{stroke}%
\end{pgfscope}%
\begin{pgfscope}%
\pgfsetroundcap%
\pgfsetroundjoin%
\pgfsetlinewidth{1.003750pt}%
\definecolor{currentstroke}{rgb}{0.000000,0.000000,0.000000}%
\pgfsetstrokecolor{currentstroke}%
\pgfsetdash{}{0pt}%
\pgfpathmoveto{\pgfqpoint{7.087448in}{0.798140in}}%
\pgfpathlineto{\pgfqpoint{7.151295in}{0.716871in}}%
\pgfpathlineto{\pgfqpoint{7.062271in}{0.664377in}}%
\pgfusepath{stroke}%
\end{pgfscope}%
\begin{pgfscope}%
\definecolor{textcolor}{rgb}{0.000000,0.000000,0.000000}%
\pgfsetstrokecolor{textcolor}%
\pgfsetfillcolor{textcolor}%
\pgftext[x=3.980397in,y=1.110536in,left,base]{\color{textcolor}{\rmfamily\fontsize{14.000000}{16.800000}\selectfont\catcode`\^=\active\def^{\ifmmode\sp\else\^{}\fi}\catcode`\%=\active\def%{\%}Optimum: (1, 0)}}%
\end{pgfscope}%
\begin{pgfscope}%
\pgfsetroundcap%
\pgfsetroundjoin%
\pgfsetlinewidth{1.003750pt}%
\definecolor{currentstroke}{rgb}{0.000000,0.000000,0.000000}%
\pgfsetstrokecolor{currentstroke}%
\pgfsetdash{}{0pt}%
\pgfpathmoveto{\pgfqpoint{5.949744in}{2.638706in}}%
\pgfpathquadraticcurveto{\pgfqpoint{5.299414in}{2.479063in}}{\pgfqpoint{4.659326in}{2.321934in}}%
\pgfusepath{stroke}%
\end{pgfscope}%
\begin{pgfscope}%
\pgfsetroundcap%
\pgfsetroundjoin%
\pgfsetlinewidth{1.003750pt}%
\definecolor{currentstroke}{rgb}{0.000000,0.000000,0.000000}%
\pgfsetstrokecolor{currentstroke}%
\pgfsetdash{}{0pt}%
\pgfpathmoveto{\pgfqpoint{4.751085in}{2.274383in}}%
\pgfpathlineto{\pgfqpoint{4.659326in}{2.321934in}}%
\pgfpathlineto{\pgfqpoint{4.718636in}{2.406569in}}%
\pgfusepath{stroke}%
\end{pgfscope}%
\begin{pgfscope}%
\definecolor{textcolor}{rgb}{0.000000,0.000000,0.000000}%
\pgfsetstrokecolor{textcolor}%
\pgfsetfillcolor{textcolor}%
\pgftext[x=5.263823in,y=2.713552in,left,base]{\color{textcolor}{\rmfamily\fontsize{14.000000}{16.800000}\selectfont\catcode`\^=\active\def^{\ifmmode\sp\else\^{}\fi}\catcode`\%=\active\def%{\%}MGA Solution: (0.6, 0.4)}}%
\end{pgfscope}%
\begin{pgfscope}%
\definecolor{textcolor}{rgb}{0.000000,0.000000,0.000000}%
\pgfsetstrokecolor{textcolor}%
\pgfsetfillcolor{textcolor}%
\pgftext[x=4.301254in,y=5.201411in,,base]{\color{textcolor}{\rmfamily\fontsize{20.000000}{24.000000}\selectfont\catcode`\^=\active\def^{\ifmmode\sp\else\^{}\fi}\catcode`\%=\active\def%{\%}Design Space}}%
\end{pgfscope}%
\begin{pgfscope}%
\pgfsetbuttcap%
\pgfsetmiterjoin%
\definecolor{currentfill}{rgb}{0.300000,0.300000,0.300000}%
\pgfsetfillcolor{currentfill}%
\pgfsetfillopacity{0.500000}%
\pgfsetlinewidth{1.003750pt}%
\definecolor{currentstroke}{rgb}{0.300000,0.300000,0.300000}%
\pgfsetstrokecolor{currentstroke}%
\pgfsetstrokeopacity{0.500000}%
\pgfsetdash{}{0pt}%
\pgfpathmoveto{\pgfqpoint{4.681326in}{3.934007in}}%
\pgfpathlineto{\pgfqpoint{7.702898in}{3.934007in}}%
\pgfpathquadraticcurveto{\pgfqpoint{7.747343in}{3.934007in}}{\pgfqpoint{7.747343in}{3.978451in}}%
\pgfpathlineto{\pgfqpoint{7.747343in}{4.934744in}}%
\pgfpathquadraticcurveto{\pgfqpoint{7.747343in}{4.979189in}}{\pgfqpoint{7.702898in}{4.979189in}}%
\pgfpathlineto{\pgfqpoint{4.681326in}{4.979189in}}%
\pgfpathquadraticcurveto{\pgfqpoint{4.636882in}{4.979189in}}{\pgfqpoint{4.636882in}{4.934744in}}%
\pgfpathlineto{\pgfqpoint{4.636882in}{3.978451in}}%
\pgfpathquadraticcurveto{\pgfqpoint{4.636882in}{3.934007in}}{\pgfqpoint{4.681326in}{3.934007in}}%
\pgfpathlineto{\pgfqpoint{4.681326in}{3.934007in}}%
\pgfpathclose%
\pgfusepath{stroke,fill}%
\end{pgfscope}%
\begin{pgfscope}%
\pgfsetbuttcap%
\pgfsetmiterjoin%
\definecolor{currentfill}{rgb}{1.000000,1.000000,1.000000}%
\pgfsetfillcolor{currentfill}%
\pgfsetlinewidth{1.003750pt}%
\definecolor{currentstroke}{rgb}{0.800000,0.800000,0.800000}%
\pgfsetstrokecolor{currentstroke}%
\pgfsetdash{}{0pt}%
\pgfpathmoveto{\pgfqpoint{4.653548in}{3.961785in}}%
\pgfpathlineto{\pgfqpoint{7.675120in}{3.961785in}}%
\pgfpathquadraticcurveto{\pgfqpoint{7.719565in}{3.961785in}}{\pgfqpoint{7.719565in}{4.006229in}}%
\pgfpathlineto{\pgfqpoint{7.719565in}{4.962522in}}%
\pgfpathquadraticcurveto{\pgfqpoint{7.719565in}{5.006966in}}{\pgfqpoint{7.675120in}{5.006966in}}%
\pgfpathlineto{\pgfqpoint{4.653548in}{5.006966in}}%
\pgfpathquadraticcurveto{\pgfqpoint{4.609104in}{5.006966in}}{\pgfqpoint{4.609104in}{4.962522in}}%
\pgfpathlineto{\pgfqpoint{4.609104in}{4.006229in}}%
\pgfpathquadraticcurveto{\pgfqpoint{4.609104in}{3.961785in}}{\pgfqpoint{4.653548in}{3.961785in}}%
\pgfpathlineto{\pgfqpoint{4.653548in}{3.961785in}}%
\pgfpathclose%
\pgfusepath{stroke,fill}%
\end{pgfscope}%
\begin{pgfscope}%
\pgfsetrectcap%
\pgfsetroundjoin%
\pgfsetlinewidth{1.505625pt}%
\definecolor{currentstroke}{rgb}{0.000000,0.000000,0.000000}%
\pgfsetstrokecolor{currentstroke}%
\pgfsetdash{}{0pt}%
\pgfpathmoveto{\pgfqpoint{4.697993in}{4.827018in}}%
\pgfpathlineto{\pgfqpoint{4.920215in}{4.827018in}}%
\pgfpathlineto{\pgfqpoint{5.142437in}{4.827018in}}%
\pgfusepath{stroke}%
\end{pgfscope}%
\begin{pgfscope}%
\definecolor{textcolor}{rgb}{0.000000,0.000000,0.000000}%
\pgfsetstrokecolor{textcolor}%
\pgfsetfillcolor{textcolor}%
\pgftext[x=5.320215in,y=4.749241in,left,base]{\color{textcolor}{\rmfamily\fontsize{16.000000}{19.200000}\selectfont\catcode`\^=\active\def^{\ifmmode\sp\else\^{}\fi}\catcode`\%=\active\def%{\%}x$_1$ + x$_2$ = 1}}%
\end{pgfscope}%
\begin{pgfscope}%
\pgfsetrectcap%
\pgfsetroundjoin%
\pgfsetlinewidth{1.505625pt}%
\definecolor{currentstroke}{rgb}{0.501961,0.501961,0.501961}%
\pgfsetstrokecolor{currentstroke}%
\pgfsetdash{}{0pt}%
\pgfpathmoveto{\pgfqpoint{4.697993in}{4.500847in}}%
\pgfpathlineto{\pgfqpoint{4.920215in}{4.500847in}}%
\pgfpathlineto{\pgfqpoint{5.142437in}{4.500847in}}%
\pgfusepath{stroke}%
\end{pgfscope}%
\begin{pgfscope}%
\definecolor{textcolor}{rgb}{0.000000,0.000000,0.000000}%
\pgfsetstrokecolor{textcolor}%
\pgfsetfillcolor{textcolor}%
\pgftext[x=5.320215in,y=4.423069in,left,base]{\color{textcolor}{\rmfamily\fontsize{16.000000}{19.200000}\selectfont\catcode`\^=\active\def^{\ifmmode\sp\else\^{}\fi}\catcode`\%=\active\def%{\%}min(c$_1$x$_1$ + c$_2$x$_2$)}}%
\end{pgfscope}%
\begin{pgfscope}%
\pgfsetbuttcap%
\pgfsetroundjoin%
\pgfsetlinewidth{1.505625pt}%
\definecolor{currentstroke}{rgb}{0.501961,0.501961,0.501961}%
\pgfsetstrokecolor{currentstroke}%
\pgfsetdash{{5.550000pt}{2.400000pt}}{0.000000pt}%
\pgfpathmoveto{\pgfqpoint{4.697993in}{4.174675in}}%
\pgfpathlineto{\pgfqpoint{4.920215in}{4.174675in}}%
\pgfpathlineto{\pgfqpoint{5.142437in}{4.174675in}}%
\pgfusepath{stroke}%
\end{pgfscope}%
\begin{pgfscope}%
\definecolor{textcolor}{rgb}{0.000000,0.000000,0.000000}%
\pgfsetstrokecolor{textcolor}%
\pgfsetfillcolor{textcolor}%
\pgftext[x=5.320215in,y=4.096897in,left,base]{\color{textcolor}{\rmfamily\fontsize{16.000000}{19.200000}\selectfont\catcode`\^=\active\def^{\ifmmode\sp\else\^{}\fi}\catcode`\%=\active\def%{\%}c$_1$x$_1$ + c$_2$x$_2$ $\leq c_1\cdot$slack}}%
\end{pgfscope}%
\begin{pgfscope}%
\definecolor{textcolor}{rgb}{0.000000,0.000000,0.000000}%
\pgfsetstrokecolor{textcolor}%
\pgfsetfillcolor{textcolor}%
\pgftext[x=3.980676in,y=5.868802in,,top]{\color{textcolor}{\rmfamily\fontsize{24.000000}{28.800000}\selectfont\catcode`\^=\active\def^{\ifmmode\sp\else\^{}\fi}\catcode`\%=\active\def%{\%}Modeling-to-Generate-Alternatives}}%
\end{pgfscope}%
\end{pgfpicture}%
\makeatother%
\endgroup%
}
  \caption{Simple demonstration of the standard \ac{mga} algorithm.}
  \label{fig:standard_mga}
\end{figure}

This procedure results in a small set of maximally different solutions for
modelers to interpret. In this way, \ac{mga} efficiently proposes alternatives
that may capture unmodeled objectives, such as political expediency or social
acceptance. However, this method depends on a single objective function which
does not guarantee that these alternative solutions will be optimal or
near-optimal for any other measurable objective.


\ac{pygen} was an initial exploration on repeatable analysis and functionality
extension for an existing \ac{esom}. While successful in that regard, \ac{pygen}
could not overcome \ac{temoa}'s inherent limitations on optimizing multiple
objectives and the inability to modify its objective function. Addressing these
limits led to the development of \ac{osier}.


\subsection{Data for benchmark problems}
 In order to verify \ac{osier}'s accuracy, this section analyzes an energy
system and compare the results against a representative \ac{esom}, \ac{temoa}.
For this problem, I chose to model the state of Illinois broadly and using
weather data from the Champaign-Urbana region due to its geographic centrality.
Chapter \ref{chapter:benchmark-results} presents the results from this problem,
with a variety of optimization criteria. This section describes the data used in
both models. The basic inputs for \ac{osier} and \ac{temoa} are
\begin{enumerate}
    \item Time series data for
    \begin{itemize}
      \item electricity demand
      \item \ac{vre} production (e.g., solar or wind),
    \end{itemize} 
    \item and technology data.
\end{enumerate}
\noindent
The time series data for electricity demand, wind energy, and solar energy, come
from \ac{uiuc}. All of the time series are averaged across several years to
simulate a ``typical'' year. I re-scaled the demand data by the total energy
demand for Illinois in order for the hourly demand to be on the same scale as
the default power units (MW) for \ac{osier} technologies. However, this
normalization choice is somewhat arbitrary. \ac{osier} automatically normalizes
the \ac{vre} time series because \ac{vre} capacity is a decision variable.
Figure \ref{fig:normalized_ldc} shows the normalized demand and load duration
curves.


 \begin{figure}[h]
  \centering
  \resizebox{1\columnwidth}{!}{%% Creator: Matplotlib, PGF backend
%%
%% To include the figure in your LaTeX document, write
%%   \input{<filename>.pgf}
%%
%% Make sure the required packages are loaded in your preamble
%%   \usepackage{pgf}
%%
%% Also ensure that all the required font packages are loaded; for instance,
%% the lmodern package is sometimes necessary when using math font.
%%   \usepackage{lmodern}
%%
%% Figures using additional raster images can only be included by \input if
%% they are in the same directory as the main LaTeX file. For loading figures
%% from other directories you can use the `import` package
%%   \usepackage{import}
%%
%% and then include the figures with
%%   \import{<path to file>}{<filename>.pgf}
%%
%% Matplotlib used the following preamble
%%   \def\mathdefault#1{#1}
%%   \everymath=\expandafter{\the\everymath\displaystyle}
%%   \IfFileExists{scrextend.sty}{
%%     \usepackage[fontsize=10.000000pt]{scrextend}
%%   }{
%%     \renewcommand{\normalsize}{\fontsize{10.000000}{12.000000}\selectfont}
%%     \normalsize
%%   }
%%   
%%   \makeatletter\@ifpackageloaded{underscore}{}{\usepackage[strings]{underscore}}\makeatother
%%
\begingroup%
\makeatletter%
\begin{pgfpicture}%
\pgfpathrectangle{\pgfpointorigin}{\pgfqpoint{11.893610in}{5.900000in}}%
\pgfusepath{use as bounding box, clip}%
\begin{pgfscope}%
\pgfsetbuttcap%
\pgfsetmiterjoin%
\definecolor{currentfill}{rgb}{1.000000,1.000000,1.000000}%
\pgfsetfillcolor{currentfill}%
\pgfsetlinewidth{0.000000pt}%
\definecolor{currentstroke}{rgb}{0.000000,0.000000,0.000000}%
\pgfsetstrokecolor{currentstroke}%
\pgfsetdash{}{0pt}%
\pgfpathmoveto{\pgfqpoint{0.000000in}{0.000000in}}%
\pgfpathlineto{\pgfqpoint{11.893610in}{0.000000in}}%
\pgfpathlineto{\pgfqpoint{11.893610in}{5.900000in}}%
\pgfpathlineto{\pgfqpoint{0.000000in}{5.900000in}}%
\pgfpathlineto{\pgfqpoint{0.000000in}{0.000000in}}%
\pgfpathclose%
\pgfusepath{fill}%
\end{pgfscope}%
\begin{pgfscope}%
\pgfsetbuttcap%
\pgfsetmiterjoin%
\definecolor{currentfill}{rgb}{1.000000,1.000000,1.000000}%
\pgfsetfillcolor{currentfill}%
\pgfsetlinewidth{0.000000pt}%
\definecolor{currentstroke}{rgb}{0.000000,0.000000,0.000000}%
\pgfsetstrokecolor{currentstroke}%
\pgfsetstrokeopacity{0.000000}%
\pgfsetdash{}{0pt}%
\pgfpathmoveto{\pgfqpoint{0.742589in}{0.670138in}}%
\pgfpathlineto{\pgfqpoint{8.410881in}{0.670138in}}%
\pgfpathlineto{\pgfqpoint{8.410881in}{5.516628in}}%
\pgfpathlineto{\pgfqpoint{0.742589in}{5.516628in}}%
\pgfpathlineto{\pgfqpoint{0.742589in}{0.670138in}}%
\pgfpathclose%
\pgfusepath{fill}%
\end{pgfscope}%
\begin{pgfscope}%
\pgfpathrectangle{\pgfqpoint{0.742589in}{0.670138in}}{\pgfqpoint{7.668292in}{4.846490in}}%
\pgfusepath{clip}%
\pgfsetrectcap%
\pgfsetroundjoin%
\pgfsetlinewidth{0.803000pt}%
\definecolor{currentstroke}{rgb}{0.690196,0.690196,0.690196}%
\pgfsetstrokecolor{currentstroke}%
\pgfsetdash{}{0pt}%
\pgfpathmoveto{\pgfqpoint{0.742589in}{0.670138in}}%
\pgfpathlineto{\pgfqpoint{0.742589in}{5.516628in}}%
\pgfusepath{stroke}%
\end{pgfscope}%
\begin{pgfscope}%
\pgfsetbuttcap%
\pgfsetroundjoin%
\definecolor{currentfill}{rgb}{0.000000,0.000000,0.000000}%
\pgfsetfillcolor{currentfill}%
\pgfsetlinewidth{0.803000pt}%
\definecolor{currentstroke}{rgb}{0.000000,0.000000,0.000000}%
\pgfsetstrokecolor{currentstroke}%
\pgfsetdash{}{0pt}%
\pgfsys@defobject{currentmarker}{\pgfqpoint{0.000000in}{-0.048611in}}{\pgfqpoint{0.000000in}{0.000000in}}{%
\pgfpathmoveto{\pgfqpoint{0.000000in}{0.000000in}}%
\pgfpathlineto{\pgfqpoint{0.000000in}{-0.048611in}}%
\pgfusepath{stroke,fill}%
}%
\begin{pgfscope}%
\pgfsys@transformshift{0.742589in}{0.670138in}%
\pgfsys@useobject{currentmarker}{}%
\end{pgfscope}%
\end{pgfscope}%
\begin{pgfscope}%
\definecolor{textcolor}{rgb}{0.000000,0.000000,0.000000}%
\pgfsetstrokecolor{textcolor}%
\pgfsetfillcolor{textcolor}%
\pgftext[x=0.742589in,y=0.572916in,,top]{\color{textcolor}{\rmfamily\fontsize{14.000000}{16.800000}\selectfont\catcode`\^=\active\def^{\ifmmode\sp\else\^{}\fi}\catcode`\%=\active\def%{\%}$\mathdefault{0}$}}%
\end{pgfscope}%
\begin{pgfscope}%
\pgfpathrectangle{\pgfqpoint{0.742589in}{0.670138in}}{\pgfqpoint{7.668292in}{4.846490in}}%
\pgfusepath{clip}%
\pgfsetrectcap%
\pgfsetroundjoin%
\pgfsetlinewidth{0.803000pt}%
\definecolor{currentstroke}{rgb}{0.690196,0.690196,0.690196}%
\pgfsetstrokecolor{currentstroke}%
\pgfsetdash{}{0pt}%
\pgfpathmoveto{\pgfqpoint{1.617965in}{0.670138in}}%
\pgfpathlineto{\pgfqpoint{1.617965in}{5.516628in}}%
\pgfusepath{stroke}%
\end{pgfscope}%
\begin{pgfscope}%
\pgfsetbuttcap%
\pgfsetroundjoin%
\definecolor{currentfill}{rgb}{0.000000,0.000000,0.000000}%
\pgfsetfillcolor{currentfill}%
\pgfsetlinewidth{0.803000pt}%
\definecolor{currentstroke}{rgb}{0.000000,0.000000,0.000000}%
\pgfsetstrokecolor{currentstroke}%
\pgfsetdash{}{0pt}%
\pgfsys@defobject{currentmarker}{\pgfqpoint{0.000000in}{-0.048611in}}{\pgfqpoint{0.000000in}{0.000000in}}{%
\pgfpathmoveto{\pgfqpoint{0.000000in}{0.000000in}}%
\pgfpathlineto{\pgfqpoint{0.000000in}{-0.048611in}}%
\pgfusepath{stroke,fill}%
}%
\begin{pgfscope}%
\pgfsys@transformshift{1.617965in}{0.670138in}%
\pgfsys@useobject{currentmarker}{}%
\end{pgfscope}%
\end{pgfscope}%
\begin{pgfscope}%
\definecolor{textcolor}{rgb}{0.000000,0.000000,0.000000}%
\pgfsetstrokecolor{textcolor}%
\pgfsetfillcolor{textcolor}%
\pgftext[x=1.617965in,y=0.572916in,,top]{\color{textcolor}{\rmfamily\fontsize{14.000000}{16.800000}\selectfont\catcode`\^=\active\def^{\ifmmode\sp\else\^{}\fi}\catcode`\%=\active\def%{\%}$\mathdefault{1000}$}}%
\end{pgfscope}%
\begin{pgfscope}%
\pgfpathrectangle{\pgfqpoint{0.742589in}{0.670138in}}{\pgfqpoint{7.668292in}{4.846490in}}%
\pgfusepath{clip}%
\pgfsetrectcap%
\pgfsetroundjoin%
\pgfsetlinewidth{0.803000pt}%
\definecolor{currentstroke}{rgb}{0.690196,0.690196,0.690196}%
\pgfsetstrokecolor{currentstroke}%
\pgfsetdash{}{0pt}%
\pgfpathmoveto{\pgfqpoint{2.493341in}{0.670138in}}%
\pgfpathlineto{\pgfqpoint{2.493341in}{5.516628in}}%
\pgfusepath{stroke}%
\end{pgfscope}%
\begin{pgfscope}%
\pgfsetbuttcap%
\pgfsetroundjoin%
\definecolor{currentfill}{rgb}{0.000000,0.000000,0.000000}%
\pgfsetfillcolor{currentfill}%
\pgfsetlinewidth{0.803000pt}%
\definecolor{currentstroke}{rgb}{0.000000,0.000000,0.000000}%
\pgfsetstrokecolor{currentstroke}%
\pgfsetdash{}{0pt}%
\pgfsys@defobject{currentmarker}{\pgfqpoint{0.000000in}{-0.048611in}}{\pgfqpoint{0.000000in}{0.000000in}}{%
\pgfpathmoveto{\pgfqpoint{0.000000in}{0.000000in}}%
\pgfpathlineto{\pgfqpoint{0.000000in}{-0.048611in}}%
\pgfusepath{stroke,fill}%
}%
\begin{pgfscope}%
\pgfsys@transformshift{2.493341in}{0.670138in}%
\pgfsys@useobject{currentmarker}{}%
\end{pgfscope}%
\end{pgfscope}%
\begin{pgfscope}%
\definecolor{textcolor}{rgb}{0.000000,0.000000,0.000000}%
\pgfsetstrokecolor{textcolor}%
\pgfsetfillcolor{textcolor}%
\pgftext[x=2.493341in,y=0.572916in,,top]{\color{textcolor}{\rmfamily\fontsize{14.000000}{16.800000}\selectfont\catcode`\^=\active\def^{\ifmmode\sp\else\^{}\fi}\catcode`\%=\active\def%{\%}$\mathdefault{2000}$}}%
\end{pgfscope}%
\begin{pgfscope}%
\pgfpathrectangle{\pgfqpoint{0.742589in}{0.670138in}}{\pgfqpoint{7.668292in}{4.846490in}}%
\pgfusepath{clip}%
\pgfsetrectcap%
\pgfsetroundjoin%
\pgfsetlinewidth{0.803000pt}%
\definecolor{currentstroke}{rgb}{0.690196,0.690196,0.690196}%
\pgfsetstrokecolor{currentstroke}%
\pgfsetdash{}{0pt}%
\pgfpathmoveto{\pgfqpoint{3.368716in}{0.670138in}}%
\pgfpathlineto{\pgfqpoint{3.368716in}{5.516628in}}%
\pgfusepath{stroke}%
\end{pgfscope}%
\begin{pgfscope}%
\pgfsetbuttcap%
\pgfsetroundjoin%
\definecolor{currentfill}{rgb}{0.000000,0.000000,0.000000}%
\pgfsetfillcolor{currentfill}%
\pgfsetlinewidth{0.803000pt}%
\definecolor{currentstroke}{rgb}{0.000000,0.000000,0.000000}%
\pgfsetstrokecolor{currentstroke}%
\pgfsetdash{}{0pt}%
\pgfsys@defobject{currentmarker}{\pgfqpoint{0.000000in}{-0.048611in}}{\pgfqpoint{0.000000in}{0.000000in}}{%
\pgfpathmoveto{\pgfqpoint{0.000000in}{0.000000in}}%
\pgfpathlineto{\pgfqpoint{0.000000in}{-0.048611in}}%
\pgfusepath{stroke,fill}%
}%
\begin{pgfscope}%
\pgfsys@transformshift{3.368716in}{0.670138in}%
\pgfsys@useobject{currentmarker}{}%
\end{pgfscope}%
\end{pgfscope}%
\begin{pgfscope}%
\definecolor{textcolor}{rgb}{0.000000,0.000000,0.000000}%
\pgfsetstrokecolor{textcolor}%
\pgfsetfillcolor{textcolor}%
\pgftext[x=3.368716in,y=0.572916in,,top]{\color{textcolor}{\rmfamily\fontsize{14.000000}{16.800000}\selectfont\catcode`\^=\active\def^{\ifmmode\sp\else\^{}\fi}\catcode`\%=\active\def%{\%}$\mathdefault{3000}$}}%
\end{pgfscope}%
\begin{pgfscope}%
\pgfpathrectangle{\pgfqpoint{0.742589in}{0.670138in}}{\pgfqpoint{7.668292in}{4.846490in}}%
\pgfusepath{clip}%
\pgfsetrectcap%
\pgfsetroundjoin%
\pgfsetlinewidth{0.803000pt}%
\definecolor{currentstroke}{rgb}{0.690196,0.690196,0.690196}%
\pgfsetstrokecolor{currentstroke}%
\pgfsetdash{}{0pt}%
\pgfpathmoveto{\pgfqpoint{4.244092in}{0.670138in}}%
\pgfpathlineto{\pgfqpoint{4.244092in}{5.516628in}}%
\pgfusepath{stroke}%
\end{pgfscope}%
\begin{pgfscope}%
\pgfsetbuttcap%
\pgfsetroundjoin%
\definecolor{currentfill}{rgb}{0.000000,0.000000,0.000000}%
\pgfsetfillcolor{currentfill}%
\pgfsetlinewidth{0.803000pt}%
\definecolor{currentstroke}{rgb}{0.000000,0.000000,0.000000}%
\pgfsetstrokecolor{currentstroke}%
\pgfsetdash{}{0pt}%
\pgfsys@defobject{currentmarker}{\pgfqpoint{0.000000in}{-0.048611in}}{\pgfqpoint{0.000000in}{0.000000in}}{%
\pgfpathmoveto{\pgfqpoint{0.000000in}{0.000000in}}%
\pgfpathlineto{\pgfqpoint{0.000000in}{-0.048611in}}%
\pgfusepath{stroke,fill}%
}%
\begin{pgfscope}%
\pgfsys@transformshift{4.244092in}{0.670138in}%
\pgfsys@useobject{currentmarker}{}%
\end{pgfscope}%
\end{pgfscope}%
\begin{pgfscope}%
\definecolor{textcolor}{rgb}{0.000000,0.000000,0.000000}%
\pgfsetstrokecolor{textcolor}%
\pgfsetfillcolor{textcolor}%
\pgftext[x=4.244092in,y=0.572916in,,top]{\color{textcolor}{\rmfamily\fontsize{14.000000}{16.800000}\selectfont\catcode`\^=\active\def^{\ifmmode\sp\else\^{}\fi}\catcode`\%=\active\def%{\%}$\mathdefault{4000}$}}%
\end{pgfscope}%
\begin{pgfscope}%
\pgfpathrectangle{\pgfqpoint{0.742589in}{0.670138in}}{\pgfqpoint{7.668292in}{4.846490in}}%
\pgfusepath{clip}%
\pgfsetrectcap%
\pgfsetroundjoin%
\pgfsetlinewidth{0.803000pt}%
\definecolor{currentstroke}{rgb}{0.690196,0.690196,0.690196}%
\pgfsetstrokecolor{currentstroke}%
\pgfsetdash{}{0pt}%
\pgfpathmoveto{\pgfqpoint{5.119468in}{0.670138in}}%
\pgfpathlineto{\pgfqpoint{5.119468in}{5.516628in}}%
\pgfusepath{stroke}%
\end{pgfscope}%
\begin{pgfscope}%
\pgfsetbuttcap%
\pgfsetroundjoin%
\definecolor{currentfill}{rgb}{0.000000,0.000000,0.000000}%
\pgfsetfillcolor{currentfill}%
\pgfsetlinewidth{0.803000pt}%
\definecolor{currentstroke}{rgb}{0.000000,0.000000,0.000000}%
\pgfsetstrokecolor{currentstroke}%
\pgfsetdash{}{0pt}%
\pgfsys@defobject{currentmarker}{\pgfqpoint{0.000000in}{-0.048611in}}{\pgfqpoint{0.000000in}{0.000000in}}{%
\pgfpathmoveto{\pgfqpoint{0.000000in}{0.000000in}}%
\pgfpathlineto{\pgfqpoint{0.000000in}{-0.048611in}}%
\pgfusepath{stroke,fill}%
}%
\begin{pgfscope}%
\pgfsys@transformshift{5.119468in}{0.670138in}%
\pgfsys@useobject{currentmarker}{}%
\end{pgfscope}%
\end{pgfscope}%
\begin{pgfscope}%
\definecolor{textcolor}{rgb}{0.000000,0.000000,0.000000}%
\pgfsetstrokecolor{textcolor}%
\pgfsetfillcolor{textcolor}%
\pgftext[x=5.119468in,y=0.572916in,,top]{\color{textcolor}{\rmfamily\fontsize{14.000000}{16.800000}\selectfont\catcode`\^=\active\def^{\ifmmode\sp\else\^{}\fi}\catcode`\%=\active\def%{\%}$\mathdefault{5000}$}}%
\end{pgfscope}%
\begin{pgfscope}%
\pgfpathrectangle{\pgfqpoint{0.742589in}{0.670138in}}{\pgfqpoint{7.668292in}{4.846490in}}%
\pgfusepath{clip}%
\pgfsetrectcap%
\pgfsetroundjoin%
\pgfsetlinewidth{0.803000pt}%
\definecolor{currentstroke}{rgb}{0.690196,0.690196,0.690196}%
\pgfsetstrokecolor{currentstroke}%
\pgfsetdash{}{0pt}%
\pgfpathmoveto{\pgfqpoint{5.994844in}{0.670138in}}%
\pgfpathlineto{\pgfqpoint{5.994844in}{5.516628in}}%
\pgfusepath{stroke}%
\end{pgfscope}%
\begin{pgfscope}%
\pgfsetbuttcap%
\pgfsetroundjoin%
\definecolor{currentfill}{rgb}{0.000000,0.000000,0.000000}%
\pgfsetfillcolor{currentfill}%
\pgfsetlinewidth{0.803000pt}%
\definecolor{currentstroke}{rgb}{0.000000,0.000000,0.000000}%
\pgfsetstrokecolor{currentstroke}%
\pgfsetdash{}{0pt}%
\pgfsys@defobject{currentmarker}{\pgfqpoint{0.000000in}{-0.048611in}}{\pgfqpoint{0.000000in}{0.000000in}}{%
\pgfpathmoveto{\pgfqpoint{0.000000in}{0.000000in}}%
\pgfpathlineto{\pgfqpoint{0.000000in}{-0.048611in}}%
\pgfusepath{stroke,fill}%
}%
\begin{pgfscope}%
\pgfsys@transformshift{5.994844in}{0.670138in}%
\pgfsys@useobject{currentmarker}{}%
\end{pgfscope}%
\end{pgfscope}%
\begin{pgfscope}%
\definecolor{textcolor}{rgb}{0.000000,0.000000,0.000000}%
\pgfsetstrokecolor{textcolor}%
\pgfsetfillcolor{textcolor}%
\pgftext[x=5.994844in,y=0.572916in,,top]{\color{textcolor}{\rmfamily\fontsize{14.000000}{16.800000}\selectfont\catcode`\^=\active\def^{\ifmmode\sp\else\^{}\fi}\catcode`\%=\active\def%{\%}$\mathdefault{6000}$}}%
\end{pgfscope}%
\begin{pgfscope}%
\pgfpathrectangle{\pgfqpoint{0.742589in}{0.670138in}}{\pgfqpoint{7.668292in}{4.846490in}}%
\pgfusepath{clip}%
\pgfsetrectcap%
\pgfsetroundjoin%
\pgfsetlinewidth{0.803000pt}%
\definecolor{currentstroke}{rgb}{0.690196,0.690196,0.690196}%
\pgfsetstrokecolor{currentstroke}%
\pgfsetdash{}{0pt}%
\pgfpathmoveto{\pgfqpoint{6.870220in}{0.670138in}}%
\pgfpathlineto{\pgfqpoint{6.870220in}{5.516628in}}%
\pgfusepath{stroke}%
\end{pgfscope}%
\begin{pgfscope}%
\pgfsetbuttcap%
\pgfsetroundjoin%
\definecolor{currentfill}{rgb}{0.000000,0.000000,0.000000}%
\pgfsetfillcolor{currentfill}%
\pgfsetlinewidth{0.803000pt}%
\definecolor{currentstroke}{rgb}{0.000000,0.000000,0.000000}%
\pgfsetstrokecolor{currentstroke}%
\pgfsetdash{}{0pt}%
\pgfsys@defobject{currentmarker}{\pgfqpoint{0.000000in}{-0.048611in}}{\pgfqpoint{0.000000in}{0.000000in}}{%
\pgfpathmoveto{\pgfqpoint{0.000000in}{0.000000in}}%
\pgfpathlineto{\pgfqpoint{0.000000in}{-0.048611in}}%
\pgfusepath{stroke,fill}%
}%
\begin{pgfscope}%
\pgfsys@transformshift{6.870220in}{0.670138in}%
\pgfsys@useobject{currentmarker}{}%
\end{pgfscope}%
\end{pgfscope}%
\begin{pgfscope}%
\definecolor{textcolor}{rgb}{0.000000,0.000000,0.000000}%
\pgfsetstrokecolor{textcolor}%
\pgfsetfillcolor{textcolor}%
\pgftext[x=6.870220in,y=0.572916in,,top]{\color{textcolor}{\rmfamily\fontsize{14.000000}{16.800000}\selectfont\catcode`\^=\active\def^{\ifmmode\sp\else\^{}\fi}\catcode`\%=\active\def%{\%}$\mathdefault{7000}$}}%
\end{pgfscope}%
\begin{pgfscope}%
\pgfpathrectangle{\pgfqpoint{0.742589in}{0.670138in}}{\pgfqpoint{7.668292in}{4.846490in}}%
\pgfusepath{clip}%
\pgfsetrectcap%
\pgfsetroundjoin%
\pgfsetlinewidth{0.803000pt}%
\definecolor{currentstroke}{rgb}{0.690196,0.690196,0.690196}%
\pgfsetstrokecolor{currentstroke}%
\pgfsetdash{}{0pt}%
\pgfpathmoveto{\pgfqpoint{7.745595in}{0.670138in}}%
\pgfpathlineto{\pgfqpoint{7.745595in}{5.516628in}}%
\pgfusepath{stroke}%
\end{pgfscope}%
\begin{pgfscope}%
\pgfsetbuttcap%
\pgfsetroundjoin%
\definecolor{currentfill}{rgb}{0.000000,0.000000,0.000000}%
\pgfsetfillcolor{currentfill}%
\pgfsetlinewidth{0.803000pt}%
\definecolor{currentstroke}{rgb}{0.000000,0.000000,0.000000}%
\pgfsetstrokecolor{currentstroke}%
\pgfsetdash{}{0pt}%
\pgfsys@defobject{currentmarker}{\pgfqpoint{0.000000in}{-0.048611in}}{\pgfqpoint{0.000000in}{0.000000in}}{%
\pgfpathmoveto{\pgfqpoint{0.000000in}{0.000000in}}%
\pgfpathlineto{\pgfqpoint{0.000000in}{-0.048611in}}%
\pgfusepath{stroke,fill}%
}%
\begin{pgfscope}%
\pgfsys@transformshift{7.745595in}{0.670138in}%
\pgfsys@useobject{currentmarker}{}%
\end{pgfscope}%
\end{pgfscope}%
\begin{pgfscope}%
\definecolor{textcolor}{rgb}{0.000000,0.000000,0.000000}%
\pgfsetstrokecolor{textcolor}%
\pgfsetfillcolor{textcolor}%
\pgftext[x=7.745595in,y=0.572916in,,top]{\color{textcolor}{\rmfamily\fontsize{14.000000}{16.800000}\selectfont\catcode`\^=\active\def^{\ifmmode\sp\else\^{}\fi}\catcode`\%=\active\def%{\%}$\mathdefault{8000}$}}%
\end{pgfscope}%
\begin{pgfscope}%
\pgfsetbuttcap%
\pgfsetroundjoin%
\definecolor{currentfill}{rgb}{0.000000,0.000000,0.000000}%
\pgfsetfillcolor{currentfill}%
\pgfsetlinewidth{0.602250pt}%
\definecolor{currentstroke}{rgb}{0.000000,0.000000,0.000000}%
\pgfsetstrokecolor{currentstroke}%
\pgfsetdash{}{0pt}%
\pgfsys@defobject{currentmarker}{\pgfqpoint{0.000000in}{-0.027778in}}{\pgfqpoint{0.000000in}{0.000000in}}{%
\pgfpathmoveto{\pgfqpoint{0.000000in}{0.000000in}}%
\pgfpathlineto{\pgfqpoint{0.000000in}{-0.027778in}}%
\pgfusepath{stroke,fill}%
}%
\begin{pgfscope}%
\pgfsys@transformshift{0.917664in}{0.670138in}%
\pgfsys@useobject{currentmarker}{}%
\end{pgfscope}%
\end{pgfscope}%
\begin{pgfscope}%
\pgfsetbuttcap%
\pgfsetroundjoin%
\definecolor{currentfill}{rgb}{0.000000,0.000000,0.000000}%
\pgfsetfillcolor{currentfill}%
\pgfsetlinewidth{0.602250pt}%
\definecolor{currentstroke}{rgb}{0.000000,0.000000,0.000000}%
\pgfsetstrokecolor{currentstroke}%
\pgfsetdash{}{0pt}%
\pgfsys@defobject{currentmarker}{\pgfqpoint{0.000000in}{-0.027778in}}{\pgfqpoint{0.000000in}{0.000000in}}{%
\pgfpathmoveto{\pgfqpoint{0.000000in}{0.000000in}}%
\pgfpathlineto{\pgfqpoint{0.000000in}{-0.027778in}}%
\pgfusepath{stroke,fill}%
}%
\begin{pgfscope}%
\pgfsys@transformshift{1.092739in}{0.670138in}%
\pgfsys@useobject{currentmarker}{}%
\end{pgfscope}%
\end{pgfscope}%
\begin{pgfscope}%
\pgfsetbuttcap%
\pgfsetroundjoin%
\definecolor{currentfill}{rgb}{0.000000,0.000000,0.000000}%
\pgfsetfillcolor{currentfill}%
\pgfsetlinewidth{0.602250pt}%
\definecolor{currentstroke}{rgb}{0.000000,0.000000,0.000000}%
\pgfsetstrokecolor{currentstroke}%
\pgfsetdash{}{0pt}%
\pgfsys@defobject{currentmarker}{\pgfqpoint{0.000000in}{-0.027778in}}{\pgfqpoint{0.000000in}{0.000000in}}{%
\pgfpathmoveto{\pgfqpoint{0.000000in}{0.000000in}}%
\pgfpathlineto{\pgfqpoint{0.000000in}{-0.027778in}}%
\pgfusepath{stroke,fill}%
}%
\begin{pgfscope}%
\pgfsys@transformshift{1.267814in}{0.670138in}%
\pgfsys@useobject{currentmarker}{}%
\end{pgfscope}%
\end{pgfscope}%
\begin{pgfscope}%
\pgfsetbuttcap%
\pgfsetroundjoin%
\definecolor{currentfill}{rgb}{0.000000,0.000000,0.000000}%
\pgfsetfillcolor{currentfill}%
\pgfsetlinewidth{0.602250pt}%
\definecolor{currentstroke}{rgb}{0.000000,0.000000,0.000000}%
\pgfsetstrokecolor{currentstroke}%
\pgfsetdash{}{0pt}%
\pgfsys@defobject{currentmarker}{\pgfqpoint{0.000000in}{-0.027778in}}{\pgfqpoint{0.000000in}{0.000000in}}{%
\pgfpathmoveto{\pgfqpoint{0.000000in}{0.000000in}}%
\pgfpathlineto{\pgfqpoint{0.000000in}{-0.027778in}}%
\pgfusepath{stroke,fill}%
}%
\begin{pgfscope}%
\pgfsys@transformshift{1.442890in}{0.670138in}%
\pgfsys@useobject{currentmarker}{}%
\end{pgfscope}%
\end{pgfscope}%
\begin{pgfscope}%
\pgfsetbuttcap%
\pgfsetroundjoin%
\definecolor{currentfill}{rgb}{0.000000,0.000000,0.000000}%
\pgfsetfillcolor{currentfill}%
\pgfsetlinewidth{0.602250pt}%
\definecolor{currentstroke}{rgb}{0.000000,0.000000,0.000000}%
\pgfsetstrokecolor{currentstroke}%
\pgfsetdash{}{0pt}%
\pgfsys@defobject{currentmarker}{\pgfqpoint{0.000000in}{-0.027778in}}{\pgfqpoint{0.000000in}{0.000000in}}{%
\pgfpathmoveto{\pgfqpoint{0.000000in}{0.000000in}}%
\pgfpathlineto{\pgfqpoint{0.000000in}{-0.027778in}}%
\pgfusepath{stroke,fill}%
}%
\begin{pgfscope}%
\pgfsys@transformshift{1.793040in}{0.670138in}%
\pgfsys@useobject{currentmarker}{}%
\end{pgfscope}%
\end{pgfscope}%
\begin{pgfscope}%
\pgfsetbuttcap%
\pgfsetroundjoin%
\definecolor{currentfill}{rgb}{0.000000,0.000000,0.000000}%
\pgfsetfillcolor{currentfill}%
\pgfsetlinewidth{0.602250pt}%
\definecolor{currentstroke}{rgb}{0.000000,0.000000,0.000000}%
\pgfsetstrokecolor{currentstroke}%
\pgfsetdash{}{0pt}%
\pgfsys@defobject{currentmarker}{\pgfqpoint{0.000000in}{-0.027778in}}{\pgfqpoint{0.000000in}{0.000000in}}{%
\pgfpathmoveto{\pgfqpoint{0.000000in}{0.000000in}}%
\pgfpathlineto{\pgfqpoint{0.000000in}{-0.027778in}}%
\pgfusepath{stroke,fill}%
}%
\begin{pgfscope}%
\pgfsys@transformshift{1.968115in}{0.670138in}%
\pgfsys@useobject{currentmarker}{}%
\end{pgfscope}%
\end{pgfscope}%
\begin{pgfscope}%
\pgfsetbuttcap%
\pgfsetroundjoin%
\definecolor{currentfill}{rgb}{0.000000,0.000000,0.000000}%
\pgfsetfillcolor{currentfill}%
\pgfsetlinewidth{0.602250pt}%
\definecolor{currentstroke}{rgb}{0.000000,0.000000,0.000000}%
\pgfsetstrokecolor{currentstroke}%
\pgfsetdash{}{0pt}%
\pgfsys@defobject{currentmarker}{\pgfqpoint{0.000000in}{-0.027778in}}{\pgfqpoint{0.000000in}{0.000000in}}{%
\pgfpathmoveto{\pgfqpoint{0.000000in}{0.000000in}}%
\pgfpathlineto{\pgfqpoint{0.000000in}{-0.027778in}}%
\pgfusepath{stroke,fill}%
}%
\begin{pgfscope}%
\pgfsys@transformshift{2.143190in}{0.670138in}%
\pgfsys@useobject{currentmarker}{}%
\end{pgfscope}%
\end{pgfscope}%
\begin{pgfscope}%
\pgfsetbuttcap%
\pgfsetroundjoin%
\definecolor{currentfill}{rgb}{0.000000,0.000000,0.000000}%
\pgfsetfillcolor{currentfill}%
\pgfsetlinewidth{0.602250pt}%
\definecolor{currentstroke}{rgb}{0.000000,0.000000,0.000000}%
\pgfsetstrokecolor{currentstroke}%
\pgfsetdash{}{0pt}%
\pgfsys@defobject{currentmarker}{\pgfqpoint{0.000000in}{-0.027778in}}{\pgfqpoint{0.000000in}{0.000000in}}{%
\pgfpathmoveto{\pgfqpoint{0.000000in}{0.000000in}}%
\pgfpathlineto{\pgfqpoint{0.000000in}{-0.027778in}}%
\pgfusepath{stroke,fill}%
}%
\begin{pgfscope}%
\pgfsys@transformshift{2.318265in}{0.670138in}%
\pgfsys@useobject{currentmarker}{}%
\end{pgfscope}%
\end{pgfscope}%
\begin{pgfscope}%
\pgfsetbuttcap%
\pgfsetroundjoin%
\definecolor{currentfill}{rgb}{0.000000,0.000000,0.000000}%
\pgfsetfillcolor{currentfill}%
\pgfsetlinewidth{0.602250pt}%
\definecolor{currentstroke}{rgb}{0.000000,0.000000,0.000000}%
\pgfsetstrokecolor{currentstroke}%
\pgfsetdash{}{0pt}%
\pgfsys@defobject{currentmarker}{\pgfqpoint{0.000000in}{-0.027778in}}{\pgfqpoint{0.000000in}{0.000000in}}{%
\pgfpathmoveto{\pgfqpoint{0.000000in}{0.000000in}}%
\pgfpathlineto{\pgfqpoint{0.000000in}{-0.027778in}}%
\pgfusepath{stroke,fill}%
}%
\begin{pgfscope}%
\pgfsys@transformshift{2.668416in}{0.670138in}%
\pgfsys@useobject{currentmarker}{}%
\end{pgfscope}%
\end{pgfscope}%
\begin{pgfscope}%
\pgfsetbuttcap%
\pgfsetroundjoin%
\definecolor{currentfill}{rgb}{0.000000,0.000000,0.000000}%
\pgfsetfillcolor{currentfill}%
\pgfsetlinewidth{0.602250pt}%
\definecolor{currentstroke}{rgb}{0.000000,0.000000,0.000000}%
\pgfsetstrokecolor{currentstroke}%
\pgfsetdash{}{0pt}%
\pgfsys@defobject{currentmarker}{\pgfqpoint{0.000000in}{-0.027778in}}{\pgfqpoint{0.000000in}{0.000000in}}{%
\pgfpathmoveto{\pgfqpoint{0.000000in}{0.000000in}}%
\pgfpathlineto{\pgfqpoint{0.000000in}{-0.027778in}}%
\pgfusepath{stroke,fill}%
}%
\begin{pgfscope}%
\pgfsys@transformshift{2.843491in}{0.670138in}%
\pgfsys@useobject{currentmarker}{}%
\end{pgfscope}%
\end{pgfscope}%
\begin{pgfscope}%
\pgfsetbuttcap%
\pgfsetroundjoin%
\definecolor{currentfill}{rgb}{0.000000,0.000000,0.000000}%
\pgfsetfillcolor{currentfill}%
\pgfsetlinewidth{0.602250pt}%
\definecolor{currentstroke}{rgb}{0.000000,0.000000,0.000000}%
\pgfsetstrokecolor{currentstroke}%
\pgfsetdash{}{0pt}%
\pgfsys@defobject{currentmarker}{\pgfqpoint{0.000000in}{-0.027778in}}{\pgfqpoint{0.000000in}{0.000000in}}{%
\pgfpathmoveto{\pgfqpoint{0.000000in}{0.000000in}}%
\pgfpathlineto{\pgfqpoint{0.000000in}{-0.027778in}}%
\pgfusepath{stroke,fill}%
}%
\begin{pgfscope}%
\pgfsys@transformshift{3.018566in}{0.670138in}%
\pgfsys@useobject{currentmarker}{}%
\end{pgfscope}%
\end{pgfscope}%
\begin{pgfscope}%
\pgfsetbuttcap%
\pgfsetroundjoin%
\definecolor{currentfill}{rgb}{0.000000,0.000000,0.000000}%
\pgfsetfillcolor{currentfill}%
\pgfsetlinewidth{0.602250pt}%
\definecolor{currentstroke}{rgb}{0.000000,0.000000,0.000000}%
\pgfsetstrokecolor{currentstroke}%
\pgfsetdash{}{0pt}%
\pgfsys@defobject{currentmarker}{\pgfqpoint{0.000000in}{-0.027778in}}{\pgfqpoint{0.000000in}{0.000000in}}{%
\pgfpathmoveto{\pgfqpoint{0.000000in}{0.000000in}}%
\pgfpathlineto{\pgfqpoint{0.000000in}{-0.027778in}}%
\pgfusepath{stroke,fill}%
}%
\begin{pgfscope}%
\pgfsys@transformshift{3.193641in}{0.670138in}%
\pgfsys@useobject{currentmarker}{}%
\end{pgfscope}%
\end{pgfscope}%
\begin{pgfscope}%
\pgfsetbuttcap%
\pgfsetroundjoin%
\definecolor{currentfill}{rgb}{0.000000,0.000000,0.000000}%
\pgfsetfillcolor{currentfill}%
\pgfsetlinewidth{0.602250pt}%
\definecolor{currentstroke}{rgb}{0.000000,0.000000,0.000000}%
\pgfsetstrokecolor{currentstroke}%
\pgfsetdash{}{0pt}%
\pgfsys@defobject{currentmarker}{\pgfqpoint{0.000000in}{-0.027778in}}{\pgfqpoint{0.000000in}{0.000000in}}{%
\pgfpathmoveto{\pgfqpoint{0.000000in}{0.000000in}}%
\pgfpathlineto{\pgfqpoint{0.000000in}{-0.027778in}}%
\pgfusepath{stroke,fill}%
}%
\begin{pgfscope}%
\pgfsys@transformshift{3.543792in}{0.670138in}%
\pgfsys@useobject{currentmarker}{}%
\end{pgfscope}%
\end{pgfscope}%
\begin{pgfscope}%
\pgfsetbuttcap%
\pgfsetroundjoin%
\definecolor{currentfill}{rgb}{0.000000,0.000000,0.000000}%
\pgfsetfillcolor{currentfill}%
\pgfsetlinewidth{0.602250pt}%
\definecolor{currentstroke}{rgb}{0.000000,0.000000,0.000000}%
\pgfsetstrokecolor{currentstroke}%
\pgfsetdash{}{0pt}%
\pgfsys@defobject{currentmarker}{\pgfqpoint{0.000000in}{-0.027778in}}{\pgfqpoint{0.000000in}{0.000000in}}{%
\pgfpathmoveto{\pgfqpoint{0.000000in}{0.000000in}}%
\pgfpathlineto{\pgfqpoint{0.000000in}{-0.027778in}}%
\pgfusepath{stroke,fill}%
}%
\begin{pgfscope}%
\pgfsys@transformshift{3.718867in}{0.670138in}%
\pgfsys@useobject{currentmarker}{}%
\end{pgfscope}%
\end{pgfscope}%
\begin{pgfscope}%
\pgfsetbuttcap%
\pgfsetroundjoin%
\definecolor{currentfill}{rgb}{0.000000,0.000000,0.000000}%
\pgfsetfillcolor{currentfill}%
\pgfsetlinewidth{0.602250pt}%
\definecolor{currentstroke}{rgb}{0.000000,0.000000,0.000000}%
\pgfsetstrokecolor{currentstroke}%
\pgfsetdash{}{0pt}%
\pgfsys@defobject{currentmarker}{\pgfqpoint{0.000000in}{-0.027778in}}{\pgfqpoint{0.000000in}{0.000000in}}{%
\pgfpathmoveto{\pgfqpoint{0.000000in}{0.000000in}}%
\pgfpathlineto{\pgfqpoint{0.000000in}{-0.027778in}}%
\pgfusepath{stroke,fill}%
}%
\begin{pgfscope}%
\pgfsys@transformshift{3.893942in}{0.670138in}%
\pgfsys@useobject{currentmarker}{}%
\end{pgfscope}%
\end{pgfscope}%
\begin{pgfscope}%
\pgfsetbuttcap%
\pgfsetroundjoin%
\definecolor{currentfill}{rgb}{0.000000,0.000000,0.000000}%
\pgfsetfillcolor{currentfill}%
\pgfsetlinewidth{0.602250pt}%
\definecolor{currentstroke}{rgb}{0.000000,0.000000,0.000000}%
\pgfsetstrokecolor{currentstroke}%
\pgfsetdash{}{0pt}%
\pgfsys@defobject{currentmarker}{\pgfqpoint{0.000000in}{-0.027778in}}{\pgfqpoint{0.000000in}{0.000000in}}{%
\pgfpathmoveto{\pgfqpoint{0.000000in}{0.000000in}}%
\pgfpathlineto{\pgfqpoint{0.000000in}{-0.027778in}}%
\pgfusepath{stroke,fill}%
}%
\begin{pgfscope}%
\pgfsys@transformshift{4.069017in}{0.670138in}%
\pgfsys@useobject{currentmarker}{}%
\end{pgfscope}%
\end{pgfscope}%
\begin{pgfscope}%
\pgfsetbuttcap%
\pgfsetroundjoin%
\definecolor{currentfill}{rgb}{0.000000,0.000000,0.000000}%
\pgfsetfillcolor{currentfill}%
\pgfsetlinewidth{0.602250pt}%
\definecolor{currentstroke}{rgb}{0.000000,0.000000,0.000000}%
\pgfsetstrokecolor{currentstroke}%
\pgfsetdash{}{0pt}%
\pgfsys@defobject{currentmarker}{\pgfqpoint{0.000000in}{-0.027778in}}{\pgfqpoint{0.000000in}{0.000000in}}{%
\pgfpathmoveto{\pgfqpoint{0.000000in}{0.000000in}}%
\pgfpathlineto{\pgfqpoint{0.000000in}{-0.027778in}}%
\pgfusepath{stroke,fill}%
}%
\begin{pgfscope}%
\pgfsys@transformshift{4.419167in}{0.670138in}%
\pgfsys@useobject{currentmarker}{}%
\end{pgfscope}%
\end{pgfscope}%
\begin{pgfscope}%
\pgfsetbuttcap%
\pgfsetroundjoin%
\definecolor{currentfill}{rgb}{0.000000,0.000000,0.000000}%
\pgfsetfillcolor{currentfill}%
\pgfsetlinewidth{0.602250pt}%
\definecolor{currentstroke}{rgb}{0.000000,0.000000,0.000000}%
\pgfsetstrokecolor{currentstroke}%
\pgfsetdash{}{0pt}%
\pgfsys@defobject{currentmarker}{\pgfqpoint{0.000000in}{-0.027778in}}{\pgfqpoint{0.000000in}{0.000000in}}{%
\pgfpathmoveto{\pgfqpoint{0.000000in}{0.000000in}}%
\pgfpathlineto{\pgfqpoint{0.000000in}{-0.027778in}}%
\pgfusepath{stroke,fill}%
}%
\begin{pgfscope}%
\pgfsys@transformshift{4.594243in}{0.670138in}%
\pgfsys@useobject{currentmarker}{}%
\end{pgfscope}%
\end{pgfscope}%
\begin{pgfscope}%
\pgfsetbuttcap%
\pgfsetroundjoin%
\definecolor{currentfill}{rgb}{0.000000,0.000000,0.000000}%
\pgfsetfillcolor{currentfill}%
\pgfsetlinewidth{0.602250pt}%
\definecolor{currentstroke}{rgb}{0.000000,0.000000,0.000000}%
\pgfsetstrokecolor{currentstroke}%
\pgfsetdash{}{0pt}%
\pgfsys@defobject{currentmarker}{\pgfqpoint{0.000000in}{-0.027778in}}{\pgfqpoint{0.000000in}{0.000000in}}{%
\pgfpathmoveto{\pgfqpoint{0.000000in}{0.000000in}}%
\pgfpathlineto{\pgfqpoint{0.000000in}{-0.027778in}}%
\pgfusepath{stroke,fill}%
}%
\begin{pgfscope}%
\pgfsys@transformshift{4.769318in}{0.670138in}%
\pgfsys@useobject{currentmarker}{}%
\end{pgfscope}%
\end{pgfscope}%
\begin{pgfscope}%
\pgfsetbuttcap%
\pgfsetroundjoin%
\definecolor{currentfill}{rgb}{0.000000,0.000000,0.000000}%
\pgfsetfillcolor{currentfill}%
\pgfsetlinewidth{0.602250pt}%
\definecolor{currentstroke}{rgb}{0.000000,0.000000,0.000000}%
\pgfsetstrokecolor{currentstroke}%
\pgfsetdash{}{0pt}%
\pgfsys@defobject{currentmarker}{\pgfqpoint{0.000000in}{-0.027778in}}{\pgfqpoint{0.000000in}{0.000000in}}{%
\pgfpathmoveto{\pgfqpoint{0.000000in}{0.000000in}}%
\pgfpathlineto{\pgfqpoint{0.000000in}{-0.027778in}}%
\pgfusepath{stroke,fill}%
}%
\begin{pgfscope}%
\pgfsys@transformshift{4.944393in}{0.670138in}%
\pgfsys@useobject{currentmarker}{}%
\end{pgfscope}%
\end{pgfscope}%
\begin{pgfscope}%
\pgfsetbuttcap%
\pgfsetroundjoin%
\definecolor{currentfill}{rgb}{0.000000,0.000000,0.000000}%
\pgfsetfillcolor{currentfill}%
\pgfsetlinewidth{0.602250pt}%
\definecolor{currentstroke}{rgb}{0.000000,0.000000,0.000000}%
\pgfsetstrokecolor{currentstroke}%
\pgfsetdash{}{0pt}%
\pgfsys@defobject{currentmarker}{\pgfqpoint{0.000000in}{-0.027778in}}{\pgfqpoint{0.000000in}{0.000000in}}{%
\pgfpathmoveto{\pgfqpoint{0.000000in}{0.000000in}}%
\pgfpathlineto{\pgfqpoint{0.000000in}{-0.027778in}}%
\pgfusepath{stroke,fill}%
}%
\begin{pgfscope}%
\pgfsys@transformshift{5.294543in}{0.670138in}%
\pgfsys@useobject{currentmarker}{}%
\end{pgfscope}%
\end{pgfscope}%
\begin{pgfscope}%
\pgfsetbuttcap%
\pgfsetroundjoin%
\definecolor{currentfill}{rgb}{0.000000,0.000000,0.000000}%
\pgfsetfillcolor{currentfill}%
\pgfsetlinewidth{0.602250pt}%
\definecolor{currentstroke}{rgb}{0.000000,0.000000,0.000000}%
\pgfsetstrokecolor{currentstroke}%
\pgfsetdash{}{0pt}%
\pgfsys@defobject{currentmarker}{\pgfqpoint{0.000000in}{-0.027778in}}{\pgfqpoint{0.000000in}{0.000000in}}{%
\pgfpathmoveto{\pgfqpoint{0.000000in}{0.000000in}}%
\pgfpathlineto{\pgfqpoint{0.000000in}{-0.027778in}}%
\pgfusepath{stroke,fill}%
}%
\begin{pgfscope}%
\pgfsys@transformshift{5.469618in}{0.670138in}%
\pgfsys@useobject{currentmarker}{}%
\end{pgfscope}%
\end{pgfscope}%
\begin{pgfscope}%
\pgfsetbuttcap%
\pgfsetroundjoin%
\definecolor{currentfill}{rgb}{0.000000,0.000000,0.000000}%
\pgfsetfillcolor{currentfill}%
\pgfsetlinewidth{0.602250pt}%
\definecolor{currentstroke}{rgb}{0.000000,0.000000,0.000000}%
\pgfsetstrokecolor{currentstroke}%
\pgfsetdash{}{0pt}%
\pgfsys@defobject{currentmarker}{\pgfqpoint{0.000000in}{-0.027778in}}{\pgfqpoint{0.000000in}{0.000000in}}{%
\pgfpathmoveto{\pgfqpoint{0.000000in}{0.000000in}}%
\pgfpathlineto{\pgfqpoint{0.000000in}{-0.027778in}}%
\pgfusepath{stroke,fill}%
}%
\begin{pgfscope}%
\pgfsys@transformshift{5.644693in}{0.670138in}%
\pgfsys@useobject{currentmarker}{}%
\end{pgfscope}%
\end{pgfscope}%
\begin{pgfscope}%
\pgfsetbuttcap%
\pgfsetroundjoin%
\definecolor{currentfill}{rgb}{0.000000,0.000000,0.000000}%
\pgfsetfillcolor{currentfill}%
\pgfsetlinewidth{0.602250pt}%
\definecolor{currentstroke}{rgb}{0.000000,0.000000,0.000000}%
\pgfsetstrokecolor{currentstroke}%
\pgfsetdash{}{0pt}%
\pgfsys@defobject{currentmarker}{\pgfqpoint{0.000000in}{-0.027778in}}{\pgfqpoint{0.000000in}{0.000000in}}{%
\pgfpathmoveto{\pgfqpoint{0.000000in}{0.000000in}}%
\pgfpathlineto{\pgfqpoint{0.000000in}{-0.027778in}}%
\pgfusepath{stroke,fill}%
}%
\begin{pgfscope}%
\pgfsys@transformshift{5.819769in}{0.670138in}%
\pgfsys@useobject{currentmarker}{}%
\end{pgfscope}%
\end{pgfscope}%
\begin{pgfscope}%
\pgfsetbuttcap%
\pgfsetroundjoin%
\definecolor{currentfill}{rgb}{0.000000,0.000000,0.000000}%
\pgfsetfillcolor{currentfill}%
\pgfsetlinewidth{0.602250pt}%
\definecolor{currentstroke}{rgb}{0.000000,0.000000,0.000000}%
\pgfsetstrokecolor{currentstroke}%
\pgfsetdash{}{0pt}%
\pgfsys@defobject{currentmarker}{\pgfqpoint{0.000000in}{-0.027778in}}{\pgfqpoint{0.000000in}{0.000000in}}{%
\pgfpathmoveto{\pgfqpoint{0.000000in}{0.000000in}}%
\pgfpathlineto{\pgfqpoint{0.000000in}{-0.027778in}}%
\pgfusepath{stroke,fill}%
}%
\begin{pgfscope}%
\pgfsys@transformshift{6.169919in}{0.670138in}%
\pgfsys@useobject{currentmarker}{}%
\end{pgfscope}%
\end{pgfscope}%
\begin{pgfscope}%
\pgfsetbuttcap%
\pgfsetroundjoin%
\definecolor{currentfill}{rgb}{0.000000,0.000000,0.000000}%
\pgfsetfillcolor{currentfill}%
\pgfsetlinewidth{0.602250pt}%
\definecolor{currentstroke}{rgb}{0.000000,0.000000,0.000000}%
\pgfsetstrokecolor{currentstroke}%
\pgfsetdash{}{0pt}%
\pgfsys@defobject{currentmarker}{\pgfqpoint{0.000000in}{-0.027778in}}{\pgfqpoint{0.000000in}{0.000000in}}{%
\pgfpathmoveto{\pgfqpoint{0.000000in}{0.000000in}}%
\pgfpathlineto{\pgfqpoint{0.000000in}{-0.027778in}}%
\pgfusepath{stroke,fill}%
}%
\begin{pgfscope}%
\pgfsys@transformshift{6.344994in}{0.670138in}%
\pgfsys@useobject{currentmarker}{}%
\end{pgfscope}%
\end{pgfscope}%
\begin{pgfscope}%
\pgfsetbuttcap%
\pgfsetroundjoin%
\definecolor{currentfill}{rgb}{0.000000,0.000000,0.000000}%
\pgfsetfillcolor{currentfill}%
\pgfsetlinewidth{0.602250pt}%
\definecolor{currentstroke}{rgb}{0.000000,0.000000,0.000000}%
\pgfsetstrokecolor{currentstroke}%
\pgfsetdash{}{0pt}%
\pgfsys@defobject{currentmarker}{\pgfqpoint{0.000000in}{-0.027778in}}{\pgfqpoint{0.000000in}{0.000000in}}{%
\pgfpathmoveto{\pgfqpoint{0.000000in}{0.000000in}}%
\pgfpathlineto{\pgfqpoint{0.000000in}{-0.027778in}}%
\pgfusepath{stroke,fill}%
}%
\begin{pgfscope}%
\pgfsys@transformshift{6.520069in}{0.670138in}%
\pgfsys@useobject{currentmarker}{}%
\end{pgfscope}%
\end{pgfscope}%
\begin{pgfscope}%
\pgfsetbuttcap%
\pgfsetroundjoin%
\definecolor{currentfill}{rgb}{0.000000,0.000000,0.000000}%
\pgfsetfillcolor{currentfill}%
\pgfsetlinewidth{0.602250pt}%
\definecolor{currentstroke}{rgb}{0.000000,0.000000,0.000000}%
\pgfsetstrokecolor{currentstroke}%
\pgfsetdash{}{0pt}%
\pgfsys@defobject{currentmarker}{\pgfqpoint{0.000000in}{-0.027778in}}{\pgfqpoint{0.000000in}{0.000000in}}{%
\pgfpathmoveto{\pgfqpoint{0.000000in}{0.000000in}}%
\pgfpathlineto{\pgfqpoint{0.000000in}{-0.027778in}}%
\pgfusepath{stroke,fill}%
}%
\begin{pgfscope}%
\pgfsys@transformshift{6.695144in}{0.670138in}%
\pgfsys@useobject{currentmarker}{}%
\end{pgfscope}%
\end{pgfscope}%
\begin{pgfscope}%
\pgfsetbuttcap%
\pgfsetroundjoin%
\definecolor{currentfill}{rgb}{0.000000,0.000000,0.000000}%
\pgfsetfillcolor{currentfill}%
\pgfsetlinewidth{0.602250pt}%
\definecolor{currentstroke}{rgb}{0.000000,0.000000,0.000000}%
\pgfsetstrokecolor{currentstroke}%
\pgfsetdash{}{0pt}%
\pgfsys@defobject{currentmarker}{\pgfqpoint{0.000000in}{-0.027778in}}{\pgfqpoint{0.000000in}{0.000000in}}{%
\pgfpathmoveto{\pgfqpoint{0.000000in}{0.000000in}}%
\pgfpathlineto{\pgfqpoint{0.000000in}{-0.027778in}}%
\pgfusepath{stroke,fill}%
}%
\begin{pgfscope}%
\pgfsys@transformshift{7.045295in}{0.670138in}%
\pgfsys@useobject{currentmarker}{}%
\end{pgfscope}%
\end{pgfscope}%
\begin{pgfscope}%
\pgfsetbuttcap%
\pgfsetroundjoin%
\definecolor{currentfill}{rgb}{0.000000,0.000000,0.000000}%
\pgfsetfillcolor{currentfill}%
\pgfsetlinewidth{0.602250pt}%
\definecolor{currentstroke}{rgb}{0.000000,0.000000,0.000000}%
\pgfsetstrokecolor{currentstroke}%
\pgfsetdash{}{0pt}%
\pgfsys@defobject{currentmarker}{\pgfqpoint{0.000000in}{-0.027778in}}{\pgfqpoint{0.000000in}{0.000000in}}{%
\pgfpathmoveto{\pgfqpoint{0.000000in}{0.000000in}}%
\pgfpathlineto{\pgfqpoint{0.000000in}{-0.027778in}}%
\pgfusepath{stroke,fill}%
}%
\begin{pgfscope}%
\pgfsys@transformshift{7.220370in}{0.670138in}%
\pgfsys@useobject{currentmarker}{}%
\end{pgfscope}%
\end{pgfscope}%
\begin{pgfscope}%
\pgfsetbuttcap%
\pgfsetroundjoin%
\definecolor{currentfill}{rgb}{0.000000,0.000000,0.000000}%
\pgfsetfillcolor{currentfill}%
\pgfsetlinewidth{0.602250pt}%
\definecolor{currentstroke}{rgb}{0.000000,0.000000,0.000000}%
\pgfsetstrokecolor{currentstroke}%
\pgfsetdash{}{0pt}%
\pgfsys@defobject{currentmarker}{\pgfqpoint{0.000000in}{-0.027778in}}{\pgfqpoint{0.000000in}{0.000000in}}{%
\pgfpathmoveto{\pgfqpoint{0.000000in}{0.000000in}}%
\pgfpathlineto{\pgfqpoint{0.000000in}{-0.027778in}}%
\pgfusepath{stroke,fill}%
}%
\begin{pgfscope}%
\pgfsys@transformshift{7.395445in}{0.670138in}%
\pgfsys@useobject{currentmarker}{}%
\end{pgfscope}%
\end{pgfscope}%
\begin{pgfscope}%
\pgfsetbuttcap%
\pgfsetroundjoin%
\definecolor{currentfill}{rgb}{0.000000,0.000000,0.000000}%
\pgfsetfillcolor{currentfill}%
\pgfsetlinewidth{0.602250pt}%
\definecolor{currentstroke}{rgb}{0.000000,0.000000,0.000000}%
\pgfsetstrokecolor{currentstroke}%
\pgfsetdash{}{0pt}%
\pgfsys@defobject{currentmarker}{\pgfqpoint{0.000000in}{-0.027778in}}{\pgfqpoint{0.000000in}{0.000000in}}{%
\pgfpathmoveto{\pgfqpoint{0.000000in}{0.000000in}}%
\pgfpathlineto{\pgfqpoint{0.000000in}{-0.027778in}}%
\pgfusepath{stroke,fill}%
}%
\begin{pgfscope}%
\pgfsys@transformshift{7.570520in}{0.670138in}%
\pgfsys@useobject{currentmarker}{}%
\end{pgfscope}%
\end{pgfscope}%
\begin{pgfscope}%
\pgfsetbuttcap%
\pgfsetroundjoin%
\definecolor{currentfill}{rgb}{0.000000,0.000000,0.000000}%
\pgfsetfillcolor{currentfill}%
\pgfsetlinewidth{0.602250pt}%
\definecolor{currentstroke}{rgb}{0.000000,0.000000,0.000000}%
\pgfsetstrokecolor{currentstroke}%
\pgfsetdash{}{0pt}%
\pgfsys@defobject{currentmarker}{\pgfqpoint{0.000000in}{-0.027778in}}{\pgfqpoint{0.000000in}{0.000000in}}{%
\pgfpathmoveto{\pgfqpoint{0.000000in}{0.000000in}}%
\pgfpathlineto{\pgfqpoint{0.000000in}{-0.027778in}}%
\pgfusepath{stroke,fill}%
}%
\begin{pgfscope}%
\pgfsys@transformshift{7.920671in}{0.670138in}%
\pgfsys@useobject{currentmarker}{}%
\end{pgfscope}%
\end{pgfscope}%
\begin{pgfscope}%
\pgfsetbuttcap%
\pgfsetroundjoin%
\definecolor{currentfill}{rgb}{0.000000,0.000000,0.000000}%
\pgfsetfillcolor{currentfill}%
\pgfsetlinewidth{0.602250pt}%
\definecolor{currentstroke}{rgb}{0.000000,0.000000,0.000000}%
\pgfsetstrokecolor{currentstroke}%
\pgfsetdash{}{0pt}%
\pgfsys@defobject{currentmarker}{\pgfqpoint{0.000000in}{-0.027778in}}{\pgfqpoint{0.000000in}{0.000000in}}{%
\pgfpathmoveto{\pgfqpoint{0.000000in}{0.000000in}}%
\pgfpathlineto{\pgfqpoint{0.000000in}{-0.027778in}}%
\pgfusepath{stroke,fill}%
}%
\begin{pgfscope}%
\pgfsys@transformshift{8.095746in}{0.670138in}%
\pgfsys@useobject{currentmarker}{}%
\end{pgfscope}%
\end{pgfscope}%
\begin{pgfscope}%
\pgfsetbuttcap%
\pgfsetroundjoin%
\definecolor{currentfill}{rgb}{0.000000,0.000000,0.000000}%
\pgfsetfillcolor{currentfill}%
\pgfsetlinewidth{0.602250pt}%
\definecolor{currentstroke}{rgb}{0.000000,0.000000,0.000000}%
\pgfsetstrokecolor{currentstroke}%
\pgfsetdash{}{0pt}%
\pgfsys@defobject{currentmarker}{\pgfqpoint{0.000000in}{-0.027778in}}{\pgfqpoint{0.000000in}{0.000000in}}{%
\pgfpathmoveto{\pgfqpoint{0.000000in}{0.000000in}}%
\pgfpathlineto{\pgfqpoint{0.000000in}{-0.027778in}}%
\pgfusepath{stroke,fill}%
}%
\begin{pgfscope}%
\pgfsys@transformshift{8.270821in}{0.670138in}%
\pgfsys@useobject{currentmarker}{}%
\end{pgfscope}%
\end{pgfscope}%
\begin{pgfscope}%
\definecolor{textcolor}{rgb}{0.000000,0.000000,0.000000}%
\pgfsetstrokecolor{textcolor}%
\pgfsetfillcolor{textcolor}%
\pgftext[x=4.576735in,y=0.339583in,,top]{\color{textcolor}{\rmfamily\fontsize{18.000000}{21.600000}\selectfont\catcode`\^=\active\def^{\ifmmode\sp\else\^{}\fi}\catcode`\%=\active\def%{\%}Time [hours]}}%
\end{pgfscope}%
\begin{pgfscope}%
\pgfpathrectangle{\pgfqpoint{0.742589in}{0.670138in}}{\pgfqpoint{7.668292in}{4.846490in}}%
\pgfusepath{clip}%
\pgfsetrectcap%
\pgfsetroundjoin%
\pgfsetlinewidth{0.803000pt}%
\definecolor{currentstroke}{rgb}{0.690196,0.690196,0.690196}%
\pgfsetstrokecolor{currentstroke}%
\pgfsetdash{}{0pt}%
\pgfpathmoveto{\pgfqpoint{0.742589in}{1.314483in}}%
\pgfpathlineto{\pgfqpoint{8.410881in}{1.314483in}}%
\pgfusepath{stroke}%
\end{pgfscope}%
\begin{pgfscope}%
\pgfsetbuttcap%
\pgfsetroundjoin%
\definecolor{currentfill}{rgb}{0.000000,0.000000,0.000000}%
\pgfsetfillcolor{currentfill}%
\pgfsetlinewidth{0.803000pt}%
\definecolor{currentstroke}{rgb}{0.000000,0.000000,0.000000}%
\pgfsetstrokecolor{currentstroke}%
\pgfsetdash{}{0pt}%
\pgfsys@defobject{currentmarker}{\pgfqpoint{-0.048611in}{0.000000in}}{\pgfqpoint{-0.000000in}{0.000000in}}{%
\pgfpathmoveto{\pgfqpoint{-0.000000in}{0.000000in}}%
\pgfpathlineto{\pgfqpoint{-0.048611in}{0.000000in}}%
\pgfusepath{stroke,fill}%
}%
\begin{pgfscope}%
\pgfsys@transformshift{0.742589in}{1.314483in}%
\pgfsys@useobject{currentmarker}{}%
\end{pgfscope}%
\end{pgfscope}%
\begin{pgfscope}%
\definecolor{textcolor}{rgb}{0.000000,0.000000,0.000000}%
\pgfsetstrokecolor{textcolor}%
\pgfsetfillcolor{textcolor}%
\pgftext[x=0.395138in, y=1.245039in, left, base]{\color{textcolor}{\rmfamily\fontsize{14.000000}{16.800000}\selectfont\catcode`\^=\active\def^{\ifmmode\sp\else\^{}\fi}\catcode`\%=\active\def%{\%}$\mathdefault{0.6}$}}%
\end{pgfscope}%
\begin{pgfscope}%
\pgfpathrectangle{\pgfqpoint{0.742589in}{0.670138in}}{\pgfqpoint{7.668292in}{4.846490in}}%
\pgfusepath{clip}%
\pgfsetrectcap%
\pgfsetroundjoin%
\pgfsetlinewidth{0.803000pt}%
\definecolor{currentstroke}{rgb}{0.690196,0.690196,0.690196}%
\pgfsetstrokecolor{currentstroke}%
\pgfsetdash{}{0pt}%
\pgfpathmoveto{\pgfqpoint{0.742589in}{2.365019in}}%
\pgfpathlineto{\pgfqpoint{8.410881in}{2.365019in}}%
\pgfusepath{stroke}%
\end{pgfscope}%
\begin{pgfscope}%
\pgfsetbuttcap%
\pgfsetroundjoin%
\definecolor{currentfill}{rgb}{0.000000,0.000000,0.000000}%
\pgfsetfillcolor{currentfill}%
\pgfsetlinewidth{0.803000pt}%
\definecolor{currentstroke}{rgb}{0.000000,0.000000,0.000000}%
\pgfsetstrokecolor{currentstroke}%
\pgfsetdash{}{0pt}%
\pgfsys@defobject{currentmarker}{\pgfqpoint{-0.048611in}{0.000000in}}{\pgfqpoint{-0.000000in}{0.000000in}}{%
\pgfpathmoveto{\pgfqpoint{-0.000000in}{0.000000in}}%
\pgfpathlineto{\pgfqpoint{-0.048611in}{0.000000in}}%
\pgfusepath{stroke,fill}%
}%
\begin{pgfscope}%
\pgfsys@transformshift{0.742589in}{2.365019in}%
\pgfsys@useobject{currentmarker}{}%
\end{pgfscope}%
\end{pgfscope}%
\begin{pgfscope}%
\definecolor{textcolor}{rgb}{0.000000,0.000000,0.000000}%
\pgfsetstrokecolor{textcolor}%
\pgfsetfillcolor{textcolor}%
\pgftext[x=0.395138in, y=2.295575in, left, base]{\color{textcolor}{\rmfamily\fontsize{14.000000}{16.800000}\selectfont\catcode`\^=\active\def^{\ifmmode\sp\else\^{}\fi}\catcode`\%=\active\def%{\%}$\mathdefault{0.7}$}}%
\end{pgfscope}%
\begin{pgfscope}%
\pgfpathrectangle{\pgfqpoint{0.742589in}{0.670138in}}{\pgfqpoint{7.668292in}{4.846490in}}%
\pgfusepath{clip}%
\pgfsetrectcap%
\pgfsetroundjoin%
\pgfsetlinewidth{0.803000pt}%
\definecolor{currentstroke}{rgb}{0.690196,0.690196,0.690196}%
\pgfsetstrokecolor{currentstroke}%
\pgfsetdash{}{0pt}%
\pgfpathmoveto{\pgfqpoint{0.742589in}{3.415556in}}%
\pgfpathlineto{\pgfqpoint{8.410881in}{3.415556in}}%
\pgfusepath{stroke}%
\end{pgfscope}%
\begin{pgfscope}%
\pgfsetbuttcap%
\pgfsetroundjoin%
\definecolor{currentfill}{rgb}{0.000000,0.000000,0.000000}%
\pgfsetfillcolor{currentfill}%
\pgfsetlinewidth{0.803000pt}%
\definecolor{currentstroke}{rgb}{0.000000,0.000000,0.000000}%
\pgfsetstrokecolor{currentstroke}%
\pgfsetdash{}{0pt}%
\pgfsys@defobject{currentmarker}{\pgfqpoint{-0.048611in}{0.000000in}}{\pgfqpoint{-0.000000in}{0.000000in}}{%
\pgfpathmoveto{\pgfqpoint{-0.000000in}{0.000000in}}%
\pgfpathlineto{\pgfqpoint{-0.048611in}{0.000000in}}%
\pgfusepath{stroke,fill}%
}%
\begin{pgfscope}%
\pgfsys@transformshift{0.742589in}{3.415556in}%
\pgfsys@useobject{currentmarker}{}%
\end{pgfscope}%
\end{pgfscope}%
\begin{pgfscope}%
\definecolor{textcolor}{rgb}{0.000000,0.000000,0.000000}%
\pgfsetstrokecolor{textcolor}%
\pgfsetfillcolor{textcolor}%
\pgftext[x=0.395138in, y=3.346111in, left, base]{\color{textcolor}{\rmfamily\fontsize{14.000000}{16.800000}\selectfont\catcode`\^=\active\def^{\ifmmode\sp\else\^{}\fi}\catcode`\%=\active\def%{\%}$\mathdefault{0.8}$}}%
\end{pgfscope}%
\begin{pgfscope}%
\pgfpathrectangle{\pgfqpoint{0.742589in}{0.670138in}}{\pgfqpoint{7.668292in}{4.846490in}}%
\pgfusepath{clip}%
\pgfsetrectcap%
\pgfsetroundjoin%
\pgfsetlinewidth{0.803000pt}%
\definecolor{currentstroke}{rgb}{0.690196,0.690196,0.690196}%
\pgfsetstrokecolor{currentstroke}%
\pgfsetdash{}{0pt}%
\pgfpathmoveto{\pgfqpoint{0.742589in}{4.466092in}}%
\pgfpathlineto{\pgfqpoint{8.410881in}{4.466092in}}%
\pgfusepath{stroke}%
\end{pgfscope}%
\begin{pgfscope}%
\pgfsetbuttcap%
\pgfsetroundjoin%
\definecolor{currentfill}{rgb}{0.000000,0.000000,0.000000}%
\pgfsetfillcolor{currentfill}%
\pgfsetlinewidth{0.803000pt}%
\definecolor{currentstroke}{rgb}{0.000000,0.000000,0.000000}%
\pgfsetstrokecolor{currentstroke}%
\pgfsetdash{}{0pt}%
\pgfsys@defobject{currentmarker}{\pgfqpoint{-0.048611in}{0.000000in}}{\pgfqpoint{-0.000000in}{0.000000in}}{%
\pgfpathmoveto{\pgfqpoint{-0.000000in}{0.000000in}}%
\pgfpathlineto{\pgfqpoint{-0.048611in}{0.000000in}}%
\pgfusepath{stroke,fill}%
}%
\begin{pgfscope}%
\pgfsys@transformshift{0.742589in}{4.466092in}%
\pgfsys@useobject{currentmarker}{}%
\end{pgfscope}%
\end{pgfscope}%
\begin{pgfscope}%
\definecolor{textcolor}{rgb}{0.000000,0.000000,0.000000}%
\pgfsetstrokecolor{textcolor}%
\pgfsetfillcolor{textcolor}%
\pgftext[x=0.395138in, y=4.396648in, left, base]{\color{textcolor}{\rmfamily\fontsize{14.000000}{16.800000}\selectfont\catcode`\^=\active\def^{\ifmmode\sp\else\^{}\fi}\catcode`\%=\active\def%{\%}$\mathdefault{0.9}$}}%
\end{pgfscope}%
\begin{pgfscope}%
\pgfpathrectangle{\pgfqpoint{0.742589in}{0.670138in}}{\pgfqpoint{7.668292in}{4.846490in}}%
\pgfusepath{clip}%
\pgfsetrectcap%
\pgfsetroundjoin%
\pgfsetlinewidth{0.803000pt}%
\definecolor{currentstroke}{rgb}{0.690196,0.690196,0.690196}%
\pgfsetstrokecolor{currentstroke}%
\pgfsetdash{}{0pt}%
\pgfpathmoveto{\pgfqpoint{0.742589in}{5.516628in}}%
\pgfpathlineto{\pgfqpoint{8.410881in}{5.516628in}}%
\pgfusepath{stroke}%
\end{pgfscope}%
\begin{pgfscope}%
\pgfsetbuttcap%
\pgfsetroundjoin%
\definecolor{currentfill}{rgb}{0.000000,0.000000,0.000000}%
\pgfsetfillcolor{currentfill}%
\pgfsetlinewidth{0.803000pt}%
\definecolor{currentstroke}{rgb}{0.000000,0.000000,0.000000}%
\pgfsetstrokecolor{currentstroke}%
\pgfsetdash{}{0pt}%
\pgfsys@defobject{currentmarker}{\pgfqpoint{-0.048611in}{0.000000in}}{\pgfqpoint{-0.000000in}{0.000000in}}{%
\pgfpathmoveto{\pgfqpoint{-0.000000in}{0.000000in}}%
\pgfpathlineto{\pgfqpoint{-0.048611in}{0.000000in}}%
\pgfusepath{stroke,fill}%
}%
\begin{pgfscope}%
\pgfsys@transformshift{0.742589in}{5.516628in}%
\pgfsys@useobject{currentmarker}{}%
\end{pgfscope}%
\end{pgfscope}%
\begin{pgfscope}%
\definecolor{textcolor}{rgb}{0.000000,0.000000,0.000000}%
\pgfsetstrokecolor{textcolor}%
\pgfsetfillcolor{textcolor}%
\pgftext[x=0.395138in, y=5.447184in, left, base]{\color{textcolor}{\rmfamily\fontsize{14.000000}{16.800000}\selectfont\catcode`\^=\active\def^{\ifmmode\sp\else\^{}\fi}\catcode`\%=\active\def%{\%}$\mathdefault{1.0}$}}%
\end{pgfscope}%
\begin{pgfscope}%
\pgfsetbuttcap%
\pgfsetroundjoin%
\definecolor{currentfill}{rgb}{0.000000,0.000000,0.000000}%
\pgfsetfillcolor{currentfill}%
\pgfsetlinewidth{0.602250pt}%
\definecolor{currentstroke}{rgb}{0.000000,0.000000,0.000000}%
\pgfsetstrokecolor{currentstroke}%
\pgfsetdash{}{0pt}%
\pgfsys@defobject{currentmarker}{\pgfqpoint{-0.027778in}{0.000000in}}{\pgfqpoint{-0.000000in}{0.000000in}}{%
\pgfpathmoveto{\pgfqpoint{-0.000000in}{0.000000in}}%
\pgfpathlineto{\pgfqpoint{-0.027778in}{0.000000in}}%
\pgfusepath{stroke,fill}%
}%
\begin{pgfscope}%
\pgfsys@transformshift{0.742589in}{0.684161in}%
\pgfsys@useobject{currentmarker}{}%
\end{pgfscope}%
\end{pgfscope}%
\begin{pgfscope}%
\pgfsetbuttcap%
\pgfsetroundjoin%
\definecolor{currentfill}{rgb}{0.000000,0.000000,0.000000}%
\pgfsetfillcolor{currentfill}%
\pgfsetlinewidth{0.602250pt}%
\definecolor{currentstroke}{rgb}{0.000000,0.000000,0.000000}%
\pgfsetstrokecolor{currentstroke}%
\pgfsetdash{}{0pt}%
\pgfsys@defobject{currentmarker}{\pgfqpoint{-0.027778in}{0.000000in}}{\pgfqpoint{-0.000000in}{0.000000in}}{%
\pgfpathmoveto{\pgfqpoint{-0.000000in}{0.000000in}}%
\pgfpathlineto{\pgfqpoint{-0.027778in}{0.000000in}}%
\pgfusepath{stroke,fill}%
}%
\begin{pgfscope}%
\pgfsys@transformshift{0.742589in}{0.894269in}%
\pgfsys@useobject{currentmarker}{}%
\end{pgfscope}%
\end{pgfscope}%
\begin{pgfscope}%
\pgfsetbuttcap%
\pgfsetroundjoin%
\definecolor{currentfill}{rgb}{0.000000,0.000000,0.000000}%
\pgfsetfillcolor{currentfill}%
\pgfsetlinewidth{0.602250pt}%
\definecolor{currentstroke}{rgb}{0.000000,0.000000,0.000000}%
\pgfsetstrokecolor{currentstroke}%
\pgfsetdash{}{0pt}%
\pgfsys@defobject{currentmarker}{\pgfqpoint{-0.027778in}{0.000000in}}{\pgfqpoint{-0.000000in}{0.000000in}}{%
\pgfpathmoveto{\pgfqpoint{-0.000000in}{0.000000in}}%
\pgfpathlineto{\pgfqpoint{-0.027778in}{0.000000in}}%
\pgfusepath{stroke,fill}%
}%
\begin{pgfscope}%
\pgfsys@transformshift{0.742589in}{1.104376in}%
\pgfsys@useobject{currentmarker}{}%
\end{pgfscope}%
\end{pgfscope}%
\begin{pgfscope}%
\pgfsetbuttcap%
\pgfsetroundjoin%
\definecolor{currentfill}{rgb}{0.000000,0.000000,0.000000}%
\pgfsetfillcolor{currentfill}%
\pgfsetlinewidth{0.602250pt}%
\definecolor{currentstroke}{rgb}{0.000000,0.000000,0.000000}%
\pgfsetstrokecolor{currentstroke}%
\pgfsetdash{}{0pt}%
\pgfsys@defobject{currentmarker}{\pgfqpoint{-0.027778in}{0.000000in}}{\pgfqpoint{-0.000000in}{0.000000in}}{%
\pgfpathmoveto{\pgfqpoint{-0.000000in}{0.000000in}}%
\pgfpathlineto{\pgfqpoint{-0.027778in}{0.000000in}}%
\pgfusepath{stroke,fill}%
}%
\begin{pgfscope}%
\pgfsys@transformshift{0.742589in}{1.524590in}%
\pgfsys@useobject{currentmarker}{}%
\end{pgfscope}%
\end{pgfscope}%
\begin{pgfscope}%
\pgfsetbuttcap%
\pgfsetroundjoin%
\definecolor{currentfill}{rgb}{0.000000,0.000000,0.000000}%
\pgfsetfillcolor{currentfill}%
\pgfsetlinewidth{0.602250pt}%
\definecolor{currentstroke}{rgb}{0.000000,0.000000,0.000000}%
\pgfsetstrokecolor{currentstroke}%
\pgfsetdash{}{0pt}%
\pgfsys@defobject{currentmarker}{\pgfqpoint{-0.027778in}{0.000000in}}{\pgfqpoint{-0.000000in}{0.000000in}}{%
\pgfpathmoveto{\pgfqpoint{-0.000000in}{0.000000in}}%
\pgfpathlineto{\pgfqpoint{-0.027778in}{0.000000in}}%
\pgfusepath{stroke,fill}%
}%
\begin{pgfscope}%
\pgfsys@transformshift{0.742589in}{1.734698in}%
\pgfsys@useobject{currentmarker}{}%
\end{pgfscope}%
\end{pgfscope}%
\begin{pgfscope}%
\pgfsetbuttcap%
\pgfsetroundjoin%
\definecolor{currentfill}{rgb}{0.000000,0.000000,0.000000}%
\pgfsetfillcolor{currentfill}%
\pgfsetlinewidth{0.602250pt}%
\definecolor{currentstroke}{rgb}{0.000000,0.000000,0.000000}%
\pgfsetstrokecolor{currentstroke}%
\pgfsetdash{}{0pt}%
\pgfsys@defobject{currentmarker}{\pgfqpoint{-0.027778in}{0.000000in}}{\pgfqpoint{-0.000000in}{0.000000in}}{%
\pgfpathmoveto{\pgfqpoint{-0.000000in}{0.000000in}}%
\pgfpathlineto{\pgfqpoint{-0.027778in}{0.000000in}}%
\pgfusepath{stroke,fill}%
}%
\begin{pgfscope}%
\pgfsys@transformshift{0.742589in}{1.944805in}%
\pgfsys@useobject{currentmarker}{}%
\end{pgfscope}%
\end{pgfscope}%
\begin{pgfscope}%
\pgfsetbuttcap%
\pgfsetroundjoin%
\definecolor{currentfill}{rgb}{0.000000,0.000000,0.000000}%
\pgfsetfillcolor{currentfill}%
\pgfsetlinewidth{0.602250pt}%
\definecolor{currentstroke}{rgb}{0.000000,0.000000,0.000000}%
\pgfsetstrokecolor{currentstroke}%
\pgfsetdash{}{0pt}%
\pgfsys@defobject{currentmarker}{\pgfqpoint{-0.027778in}{0.000000in}}{\pgfqpoint{-0.000000in}{0.000000in}}{%
\pgfpathmoveto{\pgfqpoint{-0.000000in}{0.000000in}}%
\pgfpathlineto{\pgfqpoint{-0.027778in}{0.000000in}}%
\pgfusepath{stroke,fill}%
}%
\begin{pgfscope}%
\pgfsys@transformshift{0.742589in}{2.154912in}%
\pgfsys@useobject{currentmarker}{}%
\end{pgfscope}%
\end{pgfscope}%
\begin{pgfscope}%
\pgfsetbuttcap%
\pgfsetroundjoin%
\definecolor{currentfill}{rgb}{0.000000,0.000000,0.000000}%
\pgfsetfillcolor{currentfill}%
\pgfsetlinewidth{0.602250pt}%
\definecolor{currentstroke}{rgb}{0.000000,0.000000,0.000000}%
\pgfsetstrokecolor{currentstroke}%
\pgfsetdash{}{0pt}%
\pgfsys@defobject{currentmarker}{\pgfqpoint{-0.027778in}{0.000000in}}{\pgfqpoint{-0.000000in}{0.000000in}}{%
\pgfpathmoveto{\pgfqpoint{-0.000000in}{0.000000in}}%
\pgfpathlineto{\pgfqpoint{-0.027778in}{0.000000in}}%
\pgfusepath{stroke,fill}%
}%
\begin{pgfscope}%
\pgfsys@transformshift{0.742589in}{2.575127in}%
\pgfsys@useobject{currentmarker}{}%
\end{pgfscope}%
\end{pgfscope}%
\begin{pgfscope}%
\pgfsetbuttcap%
\pgfsetroundjoin%
\definecolor{currentfill}{rgb}{0.000000,0.000000,0.000000}%
\pgfsetfillcolor{currentfill}%
\pgfsetlinewidth{0.602250pt}%
\definecolor{currentstroke}{rgb}{0.000000,0.000000,0.000000}%
\pgfsetstrokecolor{currentstroke}%
\pgfsetdash{}{0pt}%
\pgfsys@defobject{currentmarker}{\pgfqpoint{-0.027778in}{0.000000in}}{\pgfqpoint{-0.000000in}{0.000000in}}{%
\pgfpathmoveto{\pgfqpoint{-0.000000in}{0.000000in}}%
\pgfpathlineto{\pgfqpoint{-0.027778in}{0.000000in}}%
\pgfusepath{stroke,fill}%
}%
\begin{pgfscope}%
\pgfsys@transformshift{0.742589in}{2.785234in}%
\pgfsys@useobject{currentmarker}{}%
\end{pgfscope}%
\end{pgfscope}%
\begin{pgfscope}%
\pgfsetbuttcap%
\pgfsetroundjoin%
\definecolor{currentfill}{rgb}{0.000000,0.000000,0.000000}%
\pgfsetfillcolor{currentfill}%
\pgfsetlinewidth{0.602250pt}%
\definecolor{currentstroke}{rgb}{0.000000,0.000000,0.000000}%
\pgfsetstrokecolor{currentstroke}%
\pgfsetdash{}{0pt}%
\pgfsys@defobject{currentmarker}{\pgfqpoint{-0.027778in}{0.000000in}}{\pgfqpoint{-0.000000in}{0.000000in}}{%
\pgfpathmoveto{\pgfqpoint{-0.000000in}{0.000000in}}%
\pgfpathlineto{\pgfqpoint{-0.027778in}{0.000000in}}%
\pgfusepath{stroke,fill}%
}%
\begin{pgfscope}%
\pgfsys@transformshift{0.742589in}{2.995341in}%
\pgfsys@useobject{currentmarker}{}%
\end{pgfscope}%
\end{pgfscope}%
\begin{pgfscope}%
\pgfsetbuttcap%
\pgfsetroundjoin%
\definecolor{currentfill}{rgb}{0.000000,0.000000,0.000000}%
\pgfsetfillcolor{currentfill}%
\pgfsetlinewidth{0.602250pt}%
\definecolor{currentstroke}{rgb}{0.000000,0.000000,0.000000}%
\pgfsetstrokecolor{currentstroke}%
\pgfsetdash{}{0pt}%
\pgfsys@defobject{currentmarker}{\pgfqpoint{-0.027778in}{0.000000in}}{\pgfqpoint{-0.000000in}{0.000000in}}{%
\pgfpathmoveto{\pgfqpoint{-0.000000in}{0.000000in}}%
\pgfpathlineto{\pgfqpoint{-0.027778in}{0.000000in}}%
\pgfusepath{stroke,fill}%
}%
\begin{pgfscope}%
\pgfsys@transformshift{0.742589in}{3.205448in}%
\pgfsys@useobject{currentmarker}{}%
\end{pgfscope}%
\end{pgfscope}%
\begin{pgfscope}%
\pgfsetbuttcap%
\pgfsetroundjoin%
\definecolor{currentfill}{rgb}{0.000000,0.000000,0.000000}%
\pgfsetfillcolor{currentfill}%
\pgfsetlinewidth{0.602250pt}%
\definecolor{currentstroke}{rgb}{0.000000,0.000000,0.000000}%
\pgfsetstrokecolor{currentstroke}%
\pgfsetdash{}{0pt}%
\pgfsys@defobject{currentmarker}{\pgfqpoint{-0.027778in}{0.000000in}}{\pgfqpoint{-0.000000in}{0.000000in}}{%
\pgfpathmoveto{\pgfqpoint{-0.000000in}{0.000000in}}%
\pgfpathlineto{\pgfqpoint{-0.027778in}{0.000000in}}%
\pgfusepath{stroke,fill}%
}%
\begin{pgfscope}%
\pgfsys@transformshift{0.742589in}{3.625663in}%
\pgfsys@useobject{currentmarker}{}%
\end{pgfscope}%
\end{pgfscope}%
\begin{pgfscope}%
\pgfsetbuttcap%
\pgfsetroundjoin%
\definecolor{currentfill}{rgb}{0.000000,0.000000,0.000000}%
\pgfsetfillcolor{currentfill}%
\pgfsetlinewidth{0.602250pt}%
\definecolor{currentstroke}{rgb}{0.000000,0.000000,0.000000}%
\pgfsetstrokecolor{currentstroke}%
\pgfsetdash{}{0pt}%
\pgfsys@defobject{currentmarker}{\pgfqpoint{-0.027778in}{0.000000in}}{\pgfqpoint{-0.000000in}{0.000000in}}{%
\pgfpathmoveto{\pgfqpoint{-0.000000in}{0.000000in}}%
\pgfpathlineto{\pgfqpoint{-0.027778in}{0.000000in}}%
\pgfusepath{stroke,fill}%
}%
\begin{pgfscope}%
\pgfsys@transformshift{0.742589in}{3.835770in}%
\pgfsys@useobject{currentmarker}{}%
\end{pgfscope}%
\end{pgfscope}%
\begin{pgfscope}%
\pgfsetbuttcap%
\pgfsetroundjoin%
\definecolor{currentfill}{rgb}{0.000000,0.000000,0.000000}%
\pgfsetfillcolor{currentfill}%
\pgfsetlinewidth{0.602250pt}%
\definecolor{currentstroke}{rgb}{0.000000,0.000000,0.000000}%
\pgfsetstrokecolor{currentstroke}%
\pgfsetdash{}{0pt}%
\pgfsys@defobject{currentmarker}{\pgfqpoint{-0.027778in}{0.000000in}}{\pgfqpoint{-0.000000in}{0.000000in}}{%
\pgfpathmoveto{\pgfqpoint{-0.000000in}{0.000000in}}%
\pgfpathlineto{\pgfqpoint{-0.027778in}{0.000000in}}%
\pgfusepath{stroke,fill}%
}%
\begin{pgfscope}%
\pgfsys@transformshift{0.742589in}{4.045877in}%
\pgfsys@useobject{currentmarker}{}%
\end{pgfscope}%
\end{pgfscope}%
\begin{pgfscope}%
\pgfsetbuttcap%
\pgfsetroundjoin%
\definecolor{currentfill}{rgb}{0.000000,0.000000,0.000000}%
\pgfsetfillcolor{currentfill}%
\pgfsetlinewidth{0.602250pt}%
\definecolor{currentstroke}{rgb}{0.000000,0.000000,0.000000}%
\pgfsetstrokecolor{currentstroke}%
\pgfsetdash{}{0pt}%
\pgfsys@defobject{currentmarker}{\pgfqpoint{-0.027778in}{0.000000in}}{\pgfqpoint{-0.000000in}{0.000000in}}{%
\pgfpathmoveto{\pgfqpoint{-0.000000in}{0.000000in}}%
\pgfpathlineto{\pgfqpoint{-0.027778in}{0.000000in}}%
\pgfusepath{stroke,fill}%
}%
\begin{pgfscope}%
\pgfsys@transformshift{0.742589in}{4.255985in}%
\pgfsys@useobject{currentmarker}{}%
\end{pgfscope}%
\end{pgfscope}%
\begin{pgfscope}%
\pgfsetbuttcap%
\pgfsetroundjoin%
\definecolor{currentfill}{rgb}{0.000000,0.000000,0.000000}%
\pgfsetfillcolor{currentfill}%
\pgfsetlinewidth{0.602250pt}%
\definecolor{currentstroke}{rgb}{0.000000,0.000000,0.000000}%
\pgfsetstrokecolor{currentstroke}%
\pgfsetdash{}{0pt}%
\pgfsys@defobject{currentmarker}{\pgfqpoint{-0.027778in}{0.000000in}}{\pgfqpoint{-0.000000in}{0.000000in}}{%
\pgfpathmoveto{\pgfqpoint{-0.000000in}{0.000000in}}%
\pgfpathlineto{\pgfqpoint{-0.027778in}{0.000000in}}%
\pgfusepath{stroke,fill}%
}%
\begin{pgfscope}%
\pgfsys@transformshift{0.742589in}{4.676199in}%
\pgfsys@useobject{currentmarker}{}%
\end{pgfscope}%
\end{pgfscope}%
\begin{pgfscope}%
\pgfsetbuttcap%
\pgfsetroundjoin%
\definecolor{currentfill}{rgb}{0.000000,0.000000,0.000000}%
\pgfsetfillcolor{currentfill}%
\pgfsetlinewidth{0.602250pt}%
\definecolor{currentstroke}{rgb}{0.000000,0.000000,0.000000}%
\pgfsetstrokecolor{currentstroke}%
\pgfsetdash{}{0pt}%
\pgfsys@defobject{currentmarker}{\pgfqpoint{-0.027778in}{0.000000in}}{\pgfqpoint{-0.000000in}{0.000000in}}{%
\pgfpathmoveto{\pgfqpoint{-0.000000in}{0.000000in}}%
\pgfpathlineto{\pgfqpoint{-0.027778in}{0.000000in}}%
\pgfusepath{stroke,fill}%
}%
\begin{pgfscope}%
\pgfsys@transformshift{0.742589in}{4.886306in}%
\pgfsys@useobject{currentmarker}{}%
\end{pgfscope}%
\end{pgfscope}%
\begin{pgfscope}%
\pgfsetbuttcap%
\pgfsetroundjoin%
\definecolor{currentfill}{rgb}{0.000000,0.000000,0.000000}%
\pgfsetfillcolor{currentfill}%
\pgfsetlinewidth{0.602250pt}%
\definecolor{currentstroke}{rgb}{0.000000,0.000000,0.000000}%
\pgfsetstrokecolor{currentstroke}%
\pgfsetdash{}{0pt}%
\pgfsys@defobject{currentmarker}{\pgfqpoint{-0.027778in}{0.000000in}}{\pgfqpoint{-0.000000in}{0.000000in}}{%
\pgfpathmoveto{\pgfqpoint{-0.000000in}{0.000000in}}%
\pgfpathlineto{\pgfqpoint{-0.027778in}{0.000000in}}%
\pgfusepath{stroke,fill}%
}%
\begin{pgfscope}%
\pgfsys@transformshift{0.742589in}{5.096414in}%
\pgfsys@useobject{currentmarker}{}%
\end{pgfscope}%
\end{pgfscope}%
\begin{pgfscope}%
\pgfsetbuttcap%
\pgfsetroundjoin%
\definecolor{currentfill}{rgb}{0.000000,0.000000,0.000000}%
\pgfsetfillcolor{currentfill}%
\pgfsetlinewidth{0.602250pt}%
\definecolor{currentstroke}{rgb}{0.000000,0.000000,0.000000}%
\pgfsetstrokecolor{currentstroke}%
\pgfsetdash{}{0pt}%
\pgfsys@defobject{currentmarker}{\pgfqpoint{-0.027778in}{0.000000in}}{\pgfqpoint{-0.000000in}{0.000000in}}{%
\pgfpathmoveto{\pgfqpoint{-0.000000in}{0.000000in}}%
\pgfpathlineto{\pgfqpoint{-0.027778in}{0.000000in}}%
\pgfusepath{stroke,fill}%
}%
\begin{pgfscope}%
\pgfsys@transformshift{0.742589in}{5.306521in}%
\pgfsys@useobject{currentmarker}{}%
\end{pgfscope}%
\end{pgfscope}%
\begin{pgfscope}%
\definecolor{textcolor}{rgb}{0.000000,0.000000,0.000000}%
\pgfsetstrokecolor{textcolor}%
\pgfsetfillcolor{textcolor}%
\pgftext[x=0.339583in,y=3.093383in,,bottom,rotate=90.000000]{\color{textcolor}{\rmfamily\fontsize{18.000000}{21.600000}\selectfont\catcode`\^=\active\def^{\ifmmode\sp\else\^{}\fi}\catcode`\%=\active\def%{\%}Demand (--)}}%
\end{pgfscope}%
\begin{pgfscope}%
\pgfpathrectangle{\pgfqpoint{0.742589in}{0.670138in}}{\pgfqpoint{7.668292in}{4.846490in}}%
\pgfusepath{clip}%
\pgfsetrectcap%
\pgfsetroundjoin%
\pgfsetlinewidth{1.505625pt}%
\definecolor{currentstroke}{rgb}{0.121569,0.466667,0.705882}%
\pgfsetstrokecolor{currentstroke}%
\pgfsetdash{}{0pt}%
\pgfpathmoveto{\pgfqpoint{0.742589in}{1.132052in}}%
\pgfpathlineto{\pgfqpoint{0.743464in}{1.245779in}}%
\pgfpathlineto{\pgfqpoint{0.744340in}{1.216338in}}%
\pgfpathlineto{\pgfqpoint{0.745215in}{1.221251in}}%
\pgfpathlineto{\pgfqpoint{0.746090in}{1.160756in}}%
\pgfpathlineto{\pgfqpoint{0.746966in}{1.165521in}}%
\pgfpathlineto{\pgfqpoint{0.748717in}{1.289336in}}%
\pgfpathlineto{\pgfqpoint{0.749592in}{1.302128in}}%
\pgfpathlineto{\pgfqpoint{0.750467in}{1.291270in}}%
\pgfpathlineto{\pgfqpoint{0.751343in}{1.247104in}}%
\pgfpathlineto{\pgfqpoint{0.752218in}{1.239478in}}%
\pgfpathlineto{\pgfqpoint{0.753093in}{1.237325in}}%
\pgfpathlineto{\pgfqpoint{0.753969in}{1.271787in}}%
\pgfpathlineto{\pgfqpoint{0.754844in}{1.335894in}}%
\pgfpathlineto{\pgfqpoint{0.755720in}{1.303995in}}%
\pgfpathlineto{\pgfqpoint{0.756595in}{1.314810in}}%
\pgfpathlineto{\pgfqpoint{0.757470in}{1.448070in}}%
\pgfpathlineto{\pgfqpoint{0.758346in}{1.390678in}}%
\pgfpathlineto{\pgfqpoint{0.759221in}{1.363946in}}%
\pgfpathlineto{\pgfqpoint{0.760972in}{1.267558in}}%
\pgfpathlineto{\pgfqpoint{0.761847in}{1.207115in}}%
\pgfpathlineto{\pgfqpoint{0.762723in}{1.193492in}}%
\pgfpathlineto{\pgfqpoint{0.763598in}{1.088857in}}%
\pgfpathlineto{\pgfqpoint{0.764473in}{1.044786in}}%
\pgfpathlineto{\pgfqpoint{0.765349in}{1.030287in}}%
\pgfpathlineto{\pgfqpoint{0.766224in}{1.001229in}}%
\pgfpathlineto{\pgfqpoint{0.767099in}{1.043669in}}%
\pgfpathlineto{\pgfqpoint{0.767975in}{1.102631in}}%
\pgfpathlineto{\pgfqpoint{0.768850in}{1.131996in}}%
\pgfpathlineto{\pgfqpoint{0.770601in}{1.488399in}}%
\pgfpathlineto{\pgfqpoint{0.771476in}{1.526882in}}%
\pgfpathlineto{\pgfqpoint{0.773227in}{1.686619in}}%
\pgfpathlineto{\pgfqpoint{0.774102in}{1.669855in}}%
\pgfpathlineto{\pgfqpoint{0.774978in}{1.703433in}}%
\pgfpathlineto{\pgfqpoint{0.775853in}{1.782670in}}%
\pgfpathlineto{\pgfqpoint{0.776729in}{1.788185in}}%
\pgfpathlineto{\pgfqpoint{0.777604in}{1.752532in}}%
\pgfpathlineto{\pgfqpoint{0.778479in}{1.775174in}}%
\pgfpathlineto{\pgfqpoint{0.779355in}{1.733376in}}%
\pgfpathlineto{\pgfqpoint{0.781105in}{1.579322in}}%
\pgfpathlineto{\pgfqpoint{0.781981in}{1.547479in}}%
\pgfpathlineto{\pgfqpoint{0.782856in}{1.572399in}}%
\pgfpathlineto{\pgfqpoint{0.783732in}{1.581552in}}%
\pgfpathlineto{\pgfqpoint{0.784607in}{1.445654in}}%
\pgfpathlineto{\pgfqpoint{0.785482in}{1.405966in}}%
\pgfpathlineto{\pgfqpoint{0.786358in}{1.385634in}}%
\pgfpathlineto{\pgfqpoint{0.787233in}{1.418710in}}%
\pgfpathlineto{\pgfqpoint{0.788108in}{1.359808in}}%
\pgfpathlineto{\pgfqpoint{0.788984in}{1.373249in}}%
\pgfpathlineto{\pgfqpoint{0.789859in}{1.376454in}}%
\pgfpathlineto{\pgfqpoint{0.790735in}{1.441425in}}%
\pgfpathlineto{\pgfqpoint{0.791610in}{1.612144in}}%
\pgfpathlineto{\pgfqpoint{0.792485in}{1.692428in}}%
\pgfpathlineto{\pgfqpoint{0.793361in}{1.740962in}}%
\pgfpathlineto{\pgfqpoint{0.794236in}{1.806357in}}%
\pgfpathlineto{\pgfqpoint{0.795111in}{1.771252in}}%
\pgfpathlineto{\pgfqpoint{0.795987in}{1.802901in}}%
\pgfpathlineto{\pgfqpoint{0.796862in}{1.872716in}}%
\pgfpathlineto{\pgfqpoint{0.797738in}{1.853225in}}%
\pgfpathlineto{\pgfqpoint{0.798613in}{1.815235in}}%
\pgfpathlineto{\pgfqpoint{0.799488in}{1.751100in}}%
\pgfpathlineto{\pgfqpoint{0.801239in}{1.442029in}}%
\pgfpathlineto{\pgfqpoint{0.802114in}{1.415689in}}%
\pgfpathlineto{\pgfqpoint{0.802990in}{1.335854in}}%
\pgfpathlineto{\pgfqpoint{0.803865in}{1.284292in}}%
\pgfpathlineto{\pgfqpoint{0.804741in}{1.259311in}}%
\pgfpathlineto{\pgfqpoint{0.805616in}{1.223547in}}%
\pgfpathlineto{\pgfqpoint{0.806491in}{1.174402in}}%
\pgfpathlineto{\pgfqpoint{0.807367in}{1.183040in}}%
\pgfpathlineto{\pgfqpoint{0.808242in}{1.208716in}}%
\pgfpathlineto{\pgfqpoint{0.809117in}{1.151535in}}%
\pgfpathlineto{\pgfqpoint{0.809993in}{1.139876in}}%
\pgfpathlineto{\pgfqpoint{0.810868in}{1.178661in}}%
\pgfpathlineto{\pgfqpoint{0.813494in}{1.825786in}}%
\pgfpathlineto{\pgfqpoint{0.814370in}{1.905465in}}%
\pgfpathlineto{\pgfqpoint{0.815245in}{1.705974in}}%
\pgfpathlineto{\pgfqpoint{0.816120in}{1.714410in}}%
\pgfpathlineto{\pgfqpoint{0.817871in}{1.975193in}}%
\pgfpathlineto{\pgfqpoint{0.818747in}{1.991798in}}%
\pgfpathlineto{\pgfqpoint{0.819622in}{1.961488in}}%
\pgfpathlineto{\pgfqpoint{0.820497in}{1.966440in}}%
\pgfpathlineto{\pgfqpoint{0.821373in}{1.720470in}}%
\pgfpathlineto{\pgfqpoint{0.823123in}{1.504556in}}%
\pgfpathlineto{\pgfqpoint{0.823999in}{1.496733in}}%
\pgfpathlineto{\pgfqpoint{0.825750in}{1.314287in}}%
\pgfpathlineto{\pgfqpoint{0.826625in}{1.208957in}}%
\pgfpathlineto{\pgfqpoint{0.827500in}{1.223305in}}%
\pgfpathlineto{\pgfqpoint{0.828376in}{1.186937in}}%
\pgfpathlineto{\pgfqpoint{0.829251in}{1.127219in}}%
\pgfpathlineto{\pgfqpoint{0.830126in}{1.125256in}}%
\pgfpathlineto{\pgfqpoint{0.831002in}{1.202946in}}%
\pgfpathlineto{\pgfqpoint{0.831877in}{1.313894in}}%
\pgfpathlineto{\pgfqpoint{0.832753in}{1.546150in}}%
\pgfpathlineto{\pgfqpoint{0.833628in}{1.659025in}}%
\pgfpathlineto{\pgfqpoint{0.834503in}{1.870682in}}%
\pgfpathlineto{\pgfqpoint{0.835379in}{1.966540in}}%
\pgfpathlineto{\pgfqpoint{0.836254in}{2.162565in}}%
\pgfpathlineto{\pgfqpoint{0.837129in}{2.150895in}}%
\pgfpathlineto{\pgfqpoint{0.838005in}{2.103478in}}%
\pgfpathlineto{\pgfqpoint{0.838880in}{2.089315in}}%
\pgfpathlineto{\pgfqpoint{0.839756in}{2.024607in}}%
\pgfpathlineto{\pgfqpoint{0.841506in}{1.806226in}}%
\pgfpathlineto{\pgfqpoint{0.842382in}{1.550681in}}%
\pgfpathlineto{\pgfqpoint{0.843257in}{1.407594in}}%
\pgfpathlineto{\pgfqpoint{0.844132in}{1.366483in}}%
\pgfpathlineto{\pgfqpoint{0.845883in}{1.323772in}}%
\pgfpathlineto{\pgfqpoint{0.846759in}{1.241369in}}%
\pgfpathlineto{\pgfqpoint{0.847634in}{1.208686in}}%
\pgfpathlineto{\pgfqpoint{0.848509in}{1.159329in}}%
\pgfpathlineto{\pgfqpoint{0.849385in}{1.165793in}}%
\pgfpathlineto{\pgfqpoint{0.850260in}{1.127219in}}%
\pgfpathlineto{\pgfqpoint{0.851135in}{1.158362in}}%
\pgfpathlineto{\pgfqpoint{0.852011in}{1.235720in}}%
\pgfpathlineto{\pgfqpoint{0.852886in}{1.247803in}}%
\pgfpathlineto{\pgfqpoint{0.853762in}{1.359777in}}%
\pgfpathlineto{\pgfqpoint{0.854637in}{1.706307in}}%
\pgfpathlineto{\pgfqpoint{0.855512in}{1.817685in}}%
\pgfpathlineto{\pgfqpoint{0.856388in}{1.759746in}}%
\pgfpathlineto{\pgfqpoint{0.857263in}{1.935605in}}%
\pgfpathlineto{\pgfqpoint{0.858138in}{1.936561in}}%
\pgfpathlineto{\pgfqpoint{0.859014in}{1.985307in}}%
\pgfpathlineto{\pgfqpoint{0.859889in}{1.965360in}}%
\pgfpathlineto{\pgfqpoint{0.860765in}{2.061121in}}%
\pgfpathlineto{\pgfqpoint{0.861640in}{1.969974in}}%
\pgfpathlineto{\pgfqpoint{0.862515in}{1.926024in}}%
\pgfpathlineto{\pgfqpoint{0.864266in}{1.618947in}}%
\pgfpathlineto{\pgfqpoint{0.865142in}{1.588016in}}%
\pgfpathlineto{\pgfqpoint{0.866017in}{1.529808in}}%
\pgfpathlineto{\pgfqpoint{0.866892in}{1.515732in}}%
\pgfpathlineto{\pgfqpoint{0.867768in}{1.449792in}}%
\pgfpathlineto{\pgfqpoint{0.868643in}{1.326943in}}%
\pgfpathlineto{\pgfqpoint{0.869518in}{1.352437in}}%
\pgfpathlineto{\pgfqpoint{0.870394in}{1.289699in}}%
\pgfpathlineto{\pgfqpoint{0.871269in}{1.284413in}}%
\pgfpathlineto{\pgfqpoint{0.872145in}{1.242909in}}%
\pgfpathlineto{\pgfqpoint{0.873020in}{1.255777in}}%
\pgfpathlineto{\pgfqpoint{0.876521in}{1.893999in}}%
\pgfpathlineto{\pgfqpoint{0.877397in}{1.869922in}}%
\pgfpathlineto{\pgfqpoint{0.878272in}{2.001243in}}%
\pgfpathlineto{\pgfqpoint{0.879148in}{2.006768in}}%
\pgfpathlineto{\pgfqpoint{0.880023in}{1.947204in}}%
\pgfpathlineto{\pgfqpoint{0.880898in}{2.027985in}}%
\pgfpathlineto{\pgfqpoint{0.881774in}{2.055950in}}%
\pgfpathlineto{\pgfqpoint{0.882649in}{2.049790in}}%
\pgfpathlineto{\pgfqpoint{0.883524in}{2.005708in}}%
\pgfpathlineto{\pgfqpoint{0.885275in}{1.641662in}}%
\pgfpathlineto{\pgfqpoint{0.886151in}{1.610943in}}%
\pgfpathlineto{\pgfqpoint{0.887901in}{1.428557in}}%
\pgfpathlineto{\pgfqpoint{0.888777in}{1.428557in}}%
\pgfpathlineto{\pgfqpoint{0.889652in}{1.407745in}}%
\pgfpathlineto{\pgfqpoint{0.891403in}{1.298187in}}%
\pgfpathlineto{\pgfqpoint{0.892278in}{1.319210in}}%
\pgfpathlineto{\pgfqpoint{0.893154in}{1.286950in}}%
\pgfpathlineto{\pgfqpoint{0.894029in}{1.242245in}}%
\pgfpathlineto{\pgfqpoint{0.894904in}{1.310420in}}%
\pgfpathlineto{\pgfqpoint{0.895780in}{1.470000in}}%
\pgfpathlineto{\pgfqpoint{0.896655in}{1.787871in}}%
\pgfpathlineto{\pgfqpoint{0.897530in}{1.968621in}}%
\pgfpathlineto{\pgfqpoint{0.899281in}{2.101663in}}%
\pgfpathlineto{\pgfqpoint{0.900157in}{2.092589in}}%
\pgfpathlineto{\pgfqpoint{0.901032in}{2.007860in}}%
\pgfpathlineto{\pgfqpoint{0.901907in}{2.148298in}}%
\pgfpathlineto{\pgfqpoint{0.902783in}{2.199252in}}%
\pgfpathlineto{\pgfqpoint{0.903658in}{2.203700in}}%
\pgfpathlineto{\pgfqpoint{0.904533in}{2.012928in}}%
\pgfpathlineto{\pgfqpoint{0.906284in}{1.786350in}}%
\pgfpathlineto{\pgfqpoint{0.907160in}{1.730559in}}%
\pgfpathlineto{\pgfqpoint{0.908910in}{1.570073in}}%
\pgfpathlineto{\pgfqpoint{0.909786in}{1.555816in}}%
\pgfpathlineto{\pgfqpoint{0.910661in}{1.442663in}}%
\pgfpathlineto{\pgfqpoint{0.912412in}{1.377539in}}%
\pgfpathlineto{\pgfqpoint{0.913287in}{1.303503in}}%
\pgfpathlineto{\pgfqpoint{0.914163in}{1.271122in}}%
\pgfpathlineto{\pgfqpoint{0.915913in}{1.427470in}}%
\pgfpathlineto{\pgfqpoint{0.916789in}{1.635772in}}%
\pgfpathlineto{\pgfqpoint{0.917664in}{1.973933in}}%
\pgfpathlineto{\pgfqpoint{0.918539in}{2.169993in}}%
\pgfpathlineto{\pgfqpoint{0.919415in}{2.138729in}}%
\pgfpathlineto{\pgfqpoint{0.920290in}{1.890687in}}%
\pgfpathlineto{\pgfqpoint{0.921166in}{2.062379in}}%
\pgfpathlineto{\pgfqpoint{0.922916in}{2.238429in}}%
\pgfpathlineto{\pgfqpoint{0.925542in}{2.067812in}}%
\pgfpathlineto{\pgfqpoint{0.927293in}{1.608435in}}%
\pgfpathlineto{\pgfqpoint{0.928169in}{1.592698in}}%
\pgfpathlineto{\pgfqpoint{0.929919in}{1.356424in}}%
\pgfpathlineto{\pgfqpoint{0.931670in}{1.410826in}}%
\pgfpathlineto{\pgfqpoint{0.933421in}{1.362013in}}%
\pgfpathlineto{\pgfqpoint{0.934296in}{1.329027in}}%
\pgfpathlineto{\pgfqpoint{0.935172in}{1.327306in}}%
\pgfpathlineto{\pgfqpoint{0.936047in}{1.364248in}}%
\pgfpathlineto{\pgfqpoint{0.936922in}{1.444838in}}%
\pgfpathlineto{\pgfqpoint{0.937798in}{1.608345in}}%
\pgfpathlineto{\pgfqpoint{0.938673in}{1.854410in}}%
\pgfpathlineto{\pgfqpoint{0.939548in}{1.985475in}}%
\pgfpathlineto{\pgfqpoint{0.940424in}{2.185401in}}%
\pgfpathlineto{\pgfqpoint{0.941299in}{2.174708in}}%
\pgfpathlineto{\pgfqpoint{0.942175in}{2.176110in}}%
\pgfpathlineto{\pgfqpoint{0.943050in}{2.143377in}}%
\pgfpathlineto{\pgfqpoint{0.943925in}{2.145558in}}%
\pgfpathlineto{\pgfqpoint{0.944801in}{2.273558in}}%
\pgfpathlineto{\pgfqpoint{0.945676in}{2.320585in}}%
\pgfpathlineto{\pgfqpoint{0.946551in}{2.262099in}}%
\pgfpathlineto{\pgfqpoint{0.947427in}{2.031384in}}%
\pgfpathlineto{\pgfqpoint{0.951804in}{1.466224in}}%
\pgfpathlineto{\pgfqpoint{0.953554in}{1.356968in}}%
\pgfpathlineto{\pgfqpoint{0.954430in}{1.343224in}}%
\pgfpathlineto{\pgfqpoint{0.955305in}{1.355186in}}%
\pgfpathlineto{\pgfqpoint{0.956181in}{1.384969in}}%
\pgfpathlineto{\pgfqpoint{0.957056in}{1.389772in}}%
\pgfpathlineto{\pgfqpoint{0.957931in}{1.480482in}}%
\pgfpathlineto{\pgfqpoint{0.960557in}{2.090452in}}%
\pgfpathlineto{\pgfqpoint{0.961433in}{2.180611in}}%
\pgfpathlineto{\pgfqpoint{0.962308in}{2.368521in}}%
\pgfpathlineto{\pgfqpoint{0.963184in}{2.365122in}}%
\pgfpathlineto{\pgfqpoint{0.964059in}{2.322674in}}%
\pgfpathlineto{\pgfqpoint{0.964934in}{2.404865in}}%
\pgfpathlineto{\pgfqpoint{0.965810in}{2.348286in}}%
\pgfpathlineto{\pgfqpoint{0.966685in}{2.353749in}}%
\pgfpathlineto{\pgfqpoint{0.967560in}{2.287654in}}%
\pgfpathlineto{\pgfqpoint{0.968436in}{1.964809in}}%
\pgfpathlineto{\pgfqpoint{0.969311in}{1.798070in}}%
\pgfpathlineto{\pgfqpoint{0.971062in}{1.709052in}}%
\pgfpathlineto{\pgfqpoint{0.971937in}{1.616500in}}%
\pgfpathlineto{\pgfqpoint{0.972813in}{1.424479in}}%
\pgfpathlineto{\pgfqpoint{0.973688in}{1.322171in}}%
\pgfpathlineto{\pgfqpoint{0.974563in}{1.336005in}}%
\pgfpathlineto{\pgfqpoint{0.975439in}{1.247168in}}%
\pgfpathlineto{\pgfqpoint{0.976314in}{1.270790in}}%
\pgfpathlineto{\pgfqpoint{0.977190in}{1.271817in}}%
\pgfpathlineto{\pgfqpoint{0.978065in}{1.222007in}}%
\pgfpathlineto{\pgfqpoint{0.978940in}{1.302959in}}%
\pgfpathlineto{\pgfqpoint{0.979816in}{1.445926in}}%
\pgfpathlineto{\pgfqpoint{0.980691in}{1.702280in}}%
\pgfpathlineto{\pgfqpoint{0.981566in}{2.068034in}}%
\pgfpathlineto{\pgfqpoint{0.982442in}{2.131220in}}%
\pgfpathlineto{\pgfqpoint{0.983317in}{2.233517in}}%
\pgfpathlineto{\pgfqpoint{0.984193in}{2.227409in}}%
\pgfpathlineto{\pgfqpoint{0.985068in}{2.194371in}}%
\pgfpathlineto{\pgfqpoint{0.985943in}{2.355786in}}%
\pgfpathlineto{\pgfqpoint{0.986819in}{2.327715in}}%
\pgfpathlineto{\pgfqpoint{0.987694in}{2.255116in}}%
\pgfpathlineto{\pgfqpoint{0.988569in}{2.150094in}}%
\pgfpathlineto{\pgfqpoint{0.989445in}{1.944571in}}%
\pgfpathlineto{\pgfqpoint{0.990320in}{1.920496in}}%
\pgfpathlineto{\pgfqpoint{0.991196in}{1.875096in}}%
\pgfpathlineto{\pgfqpoint{0.994697in}{1.600491in}}%
\pgfpathlineto{\pgfqpoint{0.997323in}{1.452541in}}%
\pgfpathlineto{\pgfqpoint{0.998199in}{1.433269in}}%
\pgfpathlineto{\pgfqpoint{0.999074in}{1.404936in}}%
\pgfpathlineto{\pgfqpoint{0.999949in}{1.428497in}}%
\pgfpathlineto{\pgfqpoint{1.000825in}{1.614597in}}%
\pgfpathlineto{\pgfqpoint{1.001700in}{1.883975in}}%
\pgfpathlineto{\pgfqpoint{1.002576in}{2.050410in}}%
\pgfpathlineto{\pgfqpoint{1.004326in}{2.272140in}}%
\pgfpathlineto{\pgfqpoint{1.005202in}{2.268959in}}%
\pgfpathlineto{\pgfqpoint{1.006077in}{2.224335in}}%
\pgfpathlineto{\pgfqpoint{1.006952in}{2.118933in}}%
\pgfpathlineto{\pgfqpoint{1.007828in}{2.231970in}}%
\pgfpathlineto{\pgfqpoint{1.008703in}{2.204189in}}%
\pgfpathlineto{\pgfqpoint{1.009579in}{2.114783in}}%
\pgfpathlineto{\pgfqpoint{1.010454in}{1.908958in}}%
\pgfpathlineto{\pgfqpoint{1.011329in}{1.786018in}}%
\pgfpathlineto{\pgfqpoint{1.012205in}{1.741524in}}%
\pgfpathlineto{\pgfqpoint{1.013080in}{1.588318in}}%
\pgfpathlineto{\pgfqpoint{1.014831in}{1.462267in}}%
\pgfpathlineto{\pgfqpoint{1.015706in}{1.337455in}}%
\pgfpathlineto{\pgfqpoint{1.016582in}{1.292115in}}%
\pgfpathlineto{\pgfqpoint{1.017457in}{1.298277in}}%
\pgfpathlineto{\pgfqpoint{1.020083in}{1.246897in}}%
\pgfpathlineto{\pgfqpoint{1.020958in}{1.302899in}}%
\pgfpathlineto{\pgfqpoint{1.021834in}{1.460304in}}%
\pgfpathlineto{\pgfqpoint{1.023585in}{1.943637in}}%
\pgfpathlineto{\pgfqpoint{1.025335in}{2.112871in}}%
\pgfpathlineto{\pgfqpoint{1.026211in}{2.126361in}}%
\pgfpathlineto{\pgfqpoint{1.027086in}{2.088747in}}%
\pgfpathlineto{\pgfqpoint{1.027961in}{2.184290in}}%
\pgfpathlineto{\pgfqpoint{1.028837in}{2.175365in}}%
\pgfpathlineto{\pgfqpoint{1.029712in}{2.021509in}}%
\pgfpathlineto{\pgfqpoint{1.030588in}{2.044946in}}%
\pgfpathlineto{\pgfqpoint{1.031463in}{1.862651in}}%
\pgfpathlineto{\pgfqpoint{1.032338in}{1.741766in}}%
\pgfpathlineto{\pgfqpoint{1.033214in}{1.413605in}}%
\pgfpathlineto{\pgfqpoint{1.034089in}{1.382674in}}%
\pgfpathlineto{\pgfqpoint{1.034964in}{1.314075in}}%
\pgfpathlineto{\pgfqpoint{1.035840in}{1.326128in}}%
\pgfpathlineto{\pgfqpoint{1.037591in}{1.178721in}}%
\pgfpathlineto{\pgfqpoint{1.038466in}{1.167001in}}%
\pgfpathlineto{\pgfqpoint{1.039341in}{1.115076in}}%
\pgfpathlineto{\pgfqpoint{1.040217in}{1.131297in}}%
\pgfpathlineto{\pgfqpoint{1.041092in}{1.138970in}}%
\pgfpathlineto{\pgfqpoint{1.041967in}{1.231522in}}%
\pgfpathlineto{\pgfqpoint{1.044594in}{1.869785in}}%
\pgfpathlineto{\pgfqpoint{1.046344in}{2.053998in}}%
\pgfpathlineto{\pgfqpoint{1.047220in}{2.030742in}}%
\pgfpathlineto{\pgfqpoint{1.048095in}{2.036268in}}%
\pgfpathlineto{\pgfqpoint{1.049846in}{2.237099in}}%
\pgfpathlineto{\pgfqpoint{1.050721in}{2.218435in}}%
\pgfpathlineto{\pgfqpoint{1.051597in}{2.155984in}}%
\pgfpathlineto{\pgfqpoint{1.052472in}{2.004591in}}%
\pgfpathlineto{\pgfqpoint{1.053347in}{1.960248in}}%
\pgfpathlineto{\pgfqpoint{1.054223in}{1.889354in}}%
\pgfpathlineto{\pgfqpoint{1.055098in}{1.842776in}}%
\pgfpathlineto{\pgfqpoint{1.055973in}{1.752066in}}%
\pgfpathlineto{\pgfqpoint{1.057724in}{1.430249in}}%
\pgfpathlineto{\pgfqpoint{1.058600in}{1.363583in}}%
\pgfpathlineto{\pgfqpoint{1.059475in}{1.367873in}}%
\pgfpathlineto{\pgfqpoint{1.060350in}{1.380197in}}%
\pgfpathlineto{\pgfqpoint{1.061226in}{1.283658in}}%
\pgfpathlineto{\pgfqpoint{1.062101in}{1.302446in}}%
\pgfpathlineto{\pgfqpoint{1.062976in}{1.184460in}}%
\pgfpathlineto{\pgfqpoint{1.063852in}{1.398472in}}%
\pgfpathlineto{\pgfqpoint{1.065603in}{2.037062in}}%
\pgfpathlineto{\pgfqpoint{1.066478in}{1.941026in}}%
\pgfpathlineto{\pgfqpoint{1.068229in}{2.036192in}}%
\pgfpathlineto{\pgfqpoint{1.070855in}{2.401842in}}%
\pgfpathlineto{\pgfqpoint{1.071730in}{2.382655in}}%
\pgfpathlineto{\pgfqpoint{1.073481in}{2.127198in}}%
\pgfpathlineto{\pgfqpoint{1.074356in}{2.096690in}}%
\pgfpathlineto{\pgfqpoint{1.075232in}{1.979972in}}%
\pgfpathlineto{\pgfqpoint{1.076107in}{1.983174in}}%
\pgfpathlineto{\pgfqpoint{1.076982in}{1.924121in}}%
\pgfpathlineto{\pgfqpoint{1.077858in}{1.758349in}}%
\pgfpathlineto{\pgfqpoint{1.079609in}{1.565331in}}%
\pgfpathlineto{\pgfqpoint{1.080484in}{1.569560in}}%
\pgfpathlineto{\pgfqpoint{1.081359in}{1.540683in}}%
\pgfpathlineto{\pgfqpoint{1.083110in}{1.425265in}}%
\pgfpathlineto{\pgfqpoint{1.083985in}{1.520837in}}%
\pgfpathlineto{\pgfqpoint{1.084861in}{1.659846in}}%
\pgfpathlineto{\pgfqpoint{1.086612in}{2.165905in}}%
\pgfpathlineto{\pgfqpoint{1.088362in}{2.366278in}}%
\pgfpathlineto{\pgfqpoint{1.089238in}{2.392085in}}%
\pgfpathlineto{\pgfqpoint{1.090113in}{2.397288in}}%
\pgfpathlineto{\pgfqpoint{1.090988in}{2.412930in}}%
\pgfpathlineto{\pgfqpoint{1.091864in}{2.447307in}}%
\pgfpathlineto{\pgfqpoint{1.092739in}{2.394622in}}%
\pgfpathlineto{\pgfqpoint{1.094490in}{2.137287in}}%
\pgfpathlineto{\pgfqpoint{1.096241in}{1.952575in}}%
\pgfpathlineto{\pgfqpoint{1.097116in}{1.913730in}}%
\pgfpathlineto{\pgfqpoint{1.098867in}{1.710533in}}%
\pgfpathlineto{\pgfqpoint{1.099742in}{1.522287in}}%
\pgfpathlineto{\pgfqpoint{1.100618in}{1.439522in}}%
\pgfpathlineto{\pgfqpoint{1.102368in}{1.326248in}}%
\pgfpathlineto{\pgfqpoint{1.103244in}{1.304047in}}%
\pgfpathlineto{\pgfqpoint{1.104119in}{1.324406in}}%
\pgfpathlineto{\pgfqpoint{1.104994in}{1.285107in}}%
\pgfpathlineto{\pgfqpoint{1.107621in}{2.063273in}}%
\pgfpathlineto{\pgfqpoint{1.108496in}{2.209091in}}%
\pgfpathlineto{\pgfqpoint{1.109371in}{2.188687in}}%
\pgfpathlineto{\pgfqpoint{1.110247in}{2.332839in}}%
\pgfpathlineto{\pgfqpoint{1.111122in}{2.312488in}}%
\pgfpathlineto{\pgfqpoint{1.111997in}{2.276186in}}%
\pgfpathlineto{\pgfqpoint{1.112873in}{2.284617in}}%
\pgfpathlineto{\pgfqpoint{1.113748in}{2.388046in}}%
\pgfpathlineto{\pgfqpoint{1.114624in}{2.307741in}}%
\pgfpathlineto{\pgfqpoint{1.119000in}{1.653050in}}%
\pgfpathlineto{\pgfqpoint{1.119876in}{1.548446in}}%
\pgfpathlineto{\pgfqpoint{1.120751in}{1.370501in}}%
\pgfpathlineto{\pgfqpoint{1.122502in}{1.203067in}}%
\pgfpathlineto{\pgfqpoint{1.123377in}{1.184581in}}%
\pgfpathlineto{\pgfqpoint{1.126003in}{1.381224in}}%
\pgfpathlineto{\pgfqpoint{1.129505in}{2.250066in}}%
\pgfpathlineto{\pgfqpoint{1.131256in}{2.386025in}}%
\pgfpathlineto{\pgfqpoint{1.132131in}{2.489330in}}%
\pgfpathlineto{\pgfqpoint{1.133007in}{2.527722in}}%
\pgfpathlineto{\pgfqpoint{1.133882in}{2.598767in}}%
\pgfpathlineto{\pgfqpoint{1.134757in}{2.576717in}}%
\pgfpathlineto{\pgfqpoint{1.135633in}{2.476190in}}%
\pgfpathlineto{\pgfqpoint{1.136508in}{2.297108in}}%
\pgfpathlineto{\pgfqpoint{1.138259in}{2.131245in}}%
\pgfpathlineto{\pgfqpoint{1.139134in}{2.102670in}}%
\pgfpathlineto{\pgfqpoint{1.140010in}{1.998398in}}%
\pgfpathlineto{\pgfqpoint{1.140885in}{1.772002in}}%
\pgfpathlineto{\pgfqpoint{1.142636in}{1.538296in}}%
\pgfpathlineto{\pgfqpoint{1.143511in}{1.493319in}}%
\pgfpathlineto{\pgfqpoint{1.144386in}{1.387084in}}%
\pgfpathlineto{\pgfqpoint{1.145262in}{1.391011in}}%
\pgfpathlineto{\pgfqpoint{1.146137in}{1.402761in}}%
\pgfpathlineto{\pgfqpoint{1.147013in}{1.514464in}}%
\pgfpathlineto{\pgfqpoint{1.148763in}{1.931793in}}%
\pgfpathlineto{\pgfqpoint{1.149639in}{2.160334in}}%
\pgfpathlineto{\pgfqpoint{1.151389in}{2.456964in}}%
\pgfpathlineto{\pgfqpoint{1.152265in}{2.461340in}}%
\pgfpathlineto{\pgfqpoint{1.154016in}{2.564399in}}%
\pgfpathlineto{\pgfqpoint{1.154891in}{2.587476in}}%
\pgfpathlineto{\pgfqpoint{1.155766in}{2.521299in}}%
\pgfpathlineto{\pgfqpoint{1.156642in}{2.405730in}}%
\pgfpathlineto{\pgfqpoint{1.157517in}{2.327043in}}%
\pgfpathlineto{\pgfqpoint{1.158392in}{2.151031in}}%
\pgfpathlineto{\pgfqpoint{1.159268in}{2.067087in}}%
\pgfpathlineto{\pgfqpoint{1.161894in}{1.705398in}}%
\pgfpathlineto{\pgfqpoint{1.162769in}{1.546513in}}%
\pgfpathlineto{\pgfqpoint{1.163645in}{1.490993in}}%
\pgfpathlineto{\pgfqpoint{1.164520in}{1.481932in}}%
\pgfpathlineto{\pgfqpoint{1.166271in}{1.371830in}}%
\pgfpathlineto{\pgfqpoint{1.167146in}{1.351501in}}%
\pgfpathlineto{\pgfqpoint{1.168897in}{1.582126in}}%
\pgfpathlineto{\pgfqpoint{1.170648in}{2.048673in}}%
\pgfpathlineto{\pgfqpoint{1.171523in}{2.101036in}}%
\pgfpathlineto{\pgfqpoint{1.172398in}{2.236910in}}%
\pgfpathlineto{\pgfqpoint{1.173274in}{2.300512in}}%
\pgfpathlineto{\pgfqpoint{1.175025in}{2.457204in}}%
\pgfpathlineto{\pgfqpoint{1.175900in}{2.498540in}}%
\pgfpathlineto{\pgfqpoint{1.176775in}{2.495110in}}%
\pgfpathlineto{\pgfqpoint{1.178526in}{2.380115in}}%
\pgfpathlineto{\pgfqpoint{1.179401in}{2.247751in}}%
\pgfpathlineto{\pgfqpoint{1.181152in}{2.125053in}}%
\pgfpathlineto{\pgfqpoint{1.184654in}{1.618132in}}%
\pgfpathlineto{\pgfqpoint{1.185529in}{1.532618in}}%
\pgfpathlineto{\pgfqpoint{1.186404in}{1.521260in}}%
\pgfpathlineto{\pgfqpoint{1.187280in}{1.480451in}}%
\pgfpathlineto{\pgfqpoint{1.189031in}{1.574755in}}%
\pgfpathlineto{\pgfqpoint{1.189906in}{1.789099in}}%
\pgfpathlineto{\pgfqpoint{1.190781in}{1.904215in}}%
\pgfpathlineto{\pgfqpoint{1.191657in}{2.331593in}}%
\pgfpathlineto{\pgfqpoint{1.192532in}{2.507770in}}%
\pgfpathlineto{\pgfqpoint{1.194283in}{2.543491in}}%
\pgfpathlineto{\pgfqpoint{1.195158in}{2.603805in}}%
\pgfpathlineto{\pgfqpoint{1.196034in}{2.834052in}}%
\pgfpathlineto{\pgfqpoint{1.197784in}{2.825567in}}%
\pgfpathlineto{\pgfqpoint{1.198660in}{2.763432in}}%
\pgfpathlineto{\pgfqpoint{1.200410in}{2.408479in}}%
\pgfpathlineto{\pgfqpoint{1.201286in}{2.286113in}}%
\pgfpathlineto{\pgfqpoint{1.203912in}{2.083791in}}%
\pgfpathlineto{\pgfqpoint{1.204787in}{1.892465in}}%
\pgfpathlineto{\pgfqpoint{1.206538in}{1.719202in}}%
\pgfpathlineto{\pgfqpoint{1.208289in}{1.624233in}}%
\pgfpathlineto{\pgfqpoint{1.209164in}{1.657732in}}%
\pgfpathlineto{\pgfqpoint{1.210040in}{1.679964in}}%
\pgfpathlineto{\pgfqpoint{1.210915in}{1.854798in}}%
\pgfpathlineto{\pgfqpoint{1.211790in}{2.279679in}}%
\pgfpathlineto{\pgfqpoint{1.213541in}{2.708003in}}%
\pgfpathlineto{\pgfqpoint{1.214416in}{2.695529in}}%
\pgfpathlineto{\pgfqpoint{1.215292in}{2.758293in}}%
\pgfpathlineto{\pgfqpoint{1.216167in}{2.777426in}}%
\pgfpathlineto{\pgfqpoint{1.217043in}{2.836386in}}%
\pgfpathlineto{\pgfqpoint{1.217918in}{2.809997in}}%
\pgfpathlineto{\pgfqpoint{1.218793in}{2.749882in}}%
\pgfpathlineto{\pgfqpoint{1.219669in}{2.661094in}}%
\pgfpathlineto{\pgfqpoint{1.221419in}{2.340847in}}%
\pgfpathlineto{\pgfqpoint{1.223170in}{2.113363in}}%
\pgfpathlineto{\pgfqpoint{1.224921in}{1.926870in}}%
\pgfpathlineto{\pgfqpoint{1.225796in}{1.733580in}}%
\pgfpathlineto{\pgfqpoint{1.226672in}{1.653715in}}%
\pgfpathlineto{\pgfqpoint{1.227547in}{1.601005in}}%
\pgfpathlineto{\pgfqpoint{1.228422in}{1.571221in}}%
\pgfpathlineto{\pgfqpoint{1.229298in}{1.507667in}}%
\pgfpathlineto{\pgfqpoint{1.230173in}{1.267467in}}%
\pgfpathlineto{\pgfqpoint{1.231924in}{1.770734in}}%
\pgfpathlineto{\pgfqpoint{1.234550in}{2.398739in}}%
\pgfpathlineto{\pgfqpoint{1.238052in}{2.664898in}}%
\pgfpathlineto{\pgfqpoint{1.238927in}{2.626800in}}%
\pgfpathlineto{\pgfqpoint{1.239802in}{2.568092in}}%
\pgfpathlineto{\pgfqpoint{1.240678in}{2.558882in}}%
\pgfpathlineto{\pgfqpoint{1.241553in}{2.493509in}}%
\pgfpathlineto{\pgfqpoint{1.242428in}{2.310822in}}%
\pgfpathlineto{\pgfqpoint{1.243304in}{2.256873in}}%
\pgfpathlineto{\pgfqpoint{1.244179in}{2.171178in}}%
\pgfpathlineto{\pgfqpoint{1.245055in}{2.133451in}}%
\pgfpathlineto{\pgfqpoint{1.245930in}{2.041261in}}%
\pgfpathlineto{\pgfqpoint{1.246805in}{1.869689in}}%
\pgfpathlineto{\pgfqpoint{1.247681in}{1.817523in}}%
\pgfpathlineto{\pgfqpoint{1.248556in}{1.725817in}}%
\pgfpathlineto{\pgfqpoint{1.249431in}{1.663411in}}%
\pgfpathlineto{\pgfqpoint{1.250307in}{1.644441in}}%
\pgfpathlineto{\pgfqpoint{1.251182in}{1.648338in}}%
\pgfpathlineto{\pgfqpoint{1.252058in}{1.770703in}}%
\pgfpathlineto{\pgfqpoint{1.254684in}{2.387132in}}%
\pgfpathlineto{\pgfqpoint{1.255559in}{2.459244in}}%
\pgfpathlineto{\pgfqpoint{1.256434in}{2.433984in}}%
\pgfpathlineto{\pgfqpoint{1.257310in}{2.546486in}}%
\pgfpathlineto{\pgfqpoint{1.258185in}{2.604012in}}%
\pgfpathlineto{\pgfqpoint{1.259061in}{2.815784in}}%
\pgfpathlineto{\pgfqpoint{1.259936in}{2.818991in}}%
\pgfpathlineto{\pgfqpoint{1.260811in}{2.862084in}}%
\pgfpathlineto{\pgfqpoint{1.261687in}{2.745611in}}%
\pgfpathlineto{\pgfqpoint{1.262562in}{2.532596in}}%
\pgfpathlineto{\pgfqpoint{1.263438in}{2.391533in}}%
\pgfpathlineto{\pgfqpoint{1.264313in}{2.388452in}}%
\pgfpathlineto{\pgfqpoint{1.265188in}{2.343414in}}%
\pgfpathlineto{\pgfqpoint{1.266064in}{2.217545in}}%
\pgfpathlineto{\pgfqpoint{1.267814in}{1.908897in}}%
\pgfpathlineto{\pgfqpoint{1.268690in}{1.874069in}}%
\pgfpathlineto{\pgfqpoint{1.269565in}{1.858755in}}%
\pgfpathlineto{\pgfqpoint{1.270441in}{1.775265in}}%
\pgfpathlineto{\pgfqpoint{1.271316in}{1.740286in}}%
\pgfpathlineto{\pgfqpoint{1.272191in}{1.777742in}}%
\pgfpathlineto{\pgfqpoint{1.273067in}{1.854888in}}%
\pgfpathlineto{\pgfqpoint{1.273942in}{2.004621in}}%
\pgfpathlineto{\pgfqpoint{1.275693in}{2.443479in}}%
\pgfpathlineto{\pgfqpoint{1.276568in}{2.507004in}}%
\pgfpathlineto{\pgfqpoint{1.277444in}{2.537042in}}%
\pgfpathlineto{\pgfqpoint{1.278319in}{2.588751in}}%
\pgfpathlineto{\pgfqpoint{1.279194in}{2.585810in}}%
\pgfpathlineto{\pgfqpoint{1.280070in}{2.783578in}}%
\pgfpathlineto{\pgfqpoint{1.280945in}{2.753740in}}%
\pgfpathlineto{\pgfqpoint{1.281820in}{2.677815in}}%
\pgfpathlineto{\pgfqpoint{1.282696in}{2.560960in}}%
\pgfpathlineto{\pgfqpoint{1.284447in}{2.261072in}}%
\pgfpathlineto{\pgfqpoint{1.285322in}{2.203318in}}%
\pgfpathlineto{\pgfqpoint{1.287073in}{2.039992in}}%
\pgfpathlineto{\pgfqpoint{1.287948in}{1.896543in}}%
\pgfpathlineto{\pgfqpoint{1.288823in}{1.636497in}}%
\pgfpathlineto{\pgfqpoint{1.289699in}{1.495283in}}%
\pgfpathlineto{\pgfqpoint{1.291450in}{1.403486in}}%
\pgfpathlineto{\pgfqpoint{1.292325in}{1.401794in}}%
\pgfpathlineto{\pgfqpoint{1.293200in}{1.373733in}}%
\pgfpathlineto{\pgfqpoint{1.294076in}{1.493108in}}%
\pgfpathlineto{\pgfqpoint{1.296702in}{2.034329in}}%
\pgfpathlineto{\pgfqpoint{1.297577in}{2.114814in}}%
\pgfpathlineto{\pgfqpoint{1.298453in}{2.070113in}}%
\pgfpathlineto{\pgfqpoint{1.299328in}{2.144057in}}%
\pgfpathlineto{\pgfqpoint{1.300203in}{2.192643in}}%
\pgfpathlineto{\pgfqpoint{1.301079in}{2.334897in}}%
\pgfpathlineto{\pgfqpoint{1.301954in}{2.376058in}}%
\pgfpathlineto{\pgfqpoint{1.302829in}{2.503671in}}%
\pgfpathlineto{\pgfqpoint{1.304580in}{2.285026in}}%
\pgfpathlineto{\pgfqpoint{1.305456in}{2.118952in}}%
\pgfpathlineto{\pgfqpoint{1.307206in}{1.928833in}}%
\pgfpathlineto{\pgfqpoint{1.308957in}{1.796711in}}%
\pgfpathlineto{\pgfqpoint{1.310708in}{1.520898in}}%
\pgfpathlineto{\pgfqpoint{1.312459in}{1.381345in}}%
\pgfpathlineto{\pgfqpoint{1.313334in}{1.388050in}}%
\pgfpathlineto{\pgfqpoint{1.314209in}{1.320842in}}%
\pgfpathlineto{\pgfqpoint{1.315960in}{1.526365in}}%
\pgfpathlineto{\pgfqpoint{1.316835in}{1.729330in}}%
\pgfpathlineto{\pgfqpoint{1.317711in}{1.999217in}}%
\pgfpathlineto{\pgfqpoint{1.319462in}{2.196725in}}%
\pgfpathlineto{\pgfqpoint{1.321212in}{2.240865in}}%
\pgfpathlineto{\pgfqpoint{1.322088in}{2.335717in}}%
\pgfpathlineto{\pgfqpoint{1.322963in}{2.492947in}}%
\pgfpathlineto{\pgfqpoint{1.323838in}{2.439360in}}%
\pgfpathlineto{\pgfqpoint{1.325589in}{2.285479in}}%
\pgfpathlineto{\pgfqpoint{1.327340in}{2.073582in}}%
\pgfpathlineto{\pgfqpoint{1.328215in}{1.997824in}}%
\pgfpathlineto{\pgfqpoint{1.329091in}{1.886001in}}%
\pgfpathlineto{\pgfqpoint{1.330841in}{1.572097in}}%
\pgfpathlineto{\pgfqpoint{1.333468in}{1.277073in}}%
\pgfpathlineto{\pgfqpoint{1.334343in}{1.265927in}}%
\pgfpathlineto{\pgfqpoint{1.335218in}{1.295710in}}%
\pgfpathlineto{\pgfqpoint{1.336969in}{1.600491in}}%
\pgfpathlineto{\pgfqpoint{1.337844in}{1.972545in}}%
\pgfpathlineto{\pgfqpoint{1.338720in}{2.184424in}}%
\pgfpathlineto{\pgfqpoint{1.339595in}{2.298645in}}%
\pgfpathlineto{\pgfqpoint{1.340471in}{2.353181in}}%
\pgfpathlineto{\pgfqpoint{1.341346in}{2.339428in}}%
\pgfpathlineto{\pgfqpoint{1.342221in}{2.445280in}}%
\pgfpathlineto{\pgfqpoint{1.343097in}{2.235501in}}%
\pgfpathlineto{\pgfqpoint{1.343972in}{2.525129in}}%
\pgfpathlineto{\pgfqpoint{1.344847in}{2.560900in}}%
\pgfpathlineto{\pgfqpoint{1.345723in}{2.558345in}}%
\pgfpathlineto{\pgfqpoint{1.346598in}{2.557577in}}%
\pgfpathlineto{\pgfqpoint{1.350100in}{2.168701in}}%
\pgfpathlineto{\pgfqpoint{1.350975in}{1.939828in}}%
\pgfpathlineto{\pgfqpoint{1.352726in}{1.692801in}}%
\pgfpathlineto{\pgfqpoint{1.353601in}{1.692560in}}%
\pgfpathlineto{\pgfqpoint{1.355352in}{1.559531in}}%
\pgfpathlineto{\pgfqpoint{1.356227in}{1.595779in}}%
\pgfpathlineto{\pgfqpoint{1.357103in}{1.694856in}}%
\pgfpathlineto{\pgfqpoint{1.357978in}{1.885064in}}%
\pgfpathlineto{\pgfqpoint{1.359729in}{2.423043in}}%
\pgfpathlineto{\pgfqpoint{1.360604in}{2.510258in}}%
\pgfpathlineto{\pgfqpoint{1.361480in}{2.502304in}}%
\pgfpathlineto{\pgfqpoint{1.362355in}{2.329403in}}%
\pgfpathlineto{\pgfqpoint{1.363230in}{2.447800in}}%
\pgfpathlineto{\pgfqpoint{1.364106in}{2.431330in}}%
\pgfpathlineto{\pgfqpoint{1.364981in}{2.473011in}}%
\pgfpathlineto{\pgfqpoint{1.365856in}{2.589654in}}%
\pgfpathlineto{\pgfqpoint{1.366732in}{2.491636in}}%
\pgfpathlineto{\pgfqpoint{1.367607in}{2.426693in}}%
\pgfpathlineto{\pgfqpoint{1.368483in}{2.266630in}}%
\pgfpathlineto{\pgfqpoint{1.370233in}{2.056153in}}%
\pgfpathlineto{\pgfqpoint{1.371109in}{2.083550in}}%
\pgfpathlineto{\pgfqpoint{1.373735in}{1.691835in}}%
\pgfpathlineto{\pgfqpoint{1.376361in}{1.527211in}}%
\pgfpathlineto{\pgfqpoint{1.377236in}{1.547721in}}%
\pgfpathlineto{\pgfqpoint{1.378112in}{1.578199in}}%
\pgfpathlineto{\pgfqpoint{1.378987in}{1.785595in}}%
\pgfpathlineto{\pgfqpoint{1.379862in}{2.142593in}}%
\pgfpathlineto{\pgfqpoint{1.380738in}{2.183843in}}%
\pgfpathlineto{\pgfqpoint{1.381613in}{2.210570in}}%
\pgfpathlineto{\pgfqpoint{1.383364in}{2.566630in}}%
\pgfpathlineto{\pgfqpoint{1.384239in}{2.562345in}}%
\pgfpathlineto{\pgfqpoint{1.385115in}{2.632943in}}%
\pgfpathlineto{\pgfqpoint{1.386865in}{2.587557in}}%
\pgfpathlineto{\pgfqpoint{1.388616in}{2.427629in}}%
\pgfpathlineto{\pgfqpoint{1.389492in}{2.241468in}}%
\pgfpathlineto{\pgfqpoint{1.390367in}{2.159307in}}%
\pgfpathlineto{\pgfqpoint{1.391242in}{2.036911in}}%
\pgfpathlineto{\pgfqpoint{1.392118in}{1.964628in}}%
\pgfpathlineto{\pgfqpoint{1.393869in}{1.637252in}}%
\pgfpathlineto{\pgfqpoint{1.394744in}{1.552101in}}%
\pgfpathlineto{\pgfqpoint{1.395619in}{1.511745in}}%
\pgfpathlineto{\pgfqpoint{1.397370in}{1.390618in}}%
\pgfpathlineto{\pgfqpoint{1.398245in}{1.385332in}}%
\pgfpathlineto{\pgfqpoint{1.399121in}{1.473142in}}%
\pgfpathlineto{\pgfqpoint{1.399996in}{1.632721in}}%
\pgfpathlineto{\pgfqpoint{1.401747in}{2.119699in}}%
\pgfpathlineto{\pgfqpoint{1.402622in}{2.061127in}}%
\pgfpathlineto{\pgfqpoint{1.403498in}{2.229131in}}%
\pgfpathlineto{\pgfqpoint{1.404373in}{2.338459in}}%
\pgfpathlineto{\pgfqpoint{1.405248in}{2.243211in}}%
\pgfpathlineto{\pgfqpoint{1.406124in}{2.302644in}}%
\pgfpathlineto{\pgfqpoint{1.406999in}{2.417308in}}%
\pgfpathlineto{\pgfqpoint{1.407875in}{2.411371in}}%
\pgfpathlineto{\pgfqpoint{1.408750in}{2.366204in}}%
\pgfpathlineto{\pgfqpoint{1.410501in}{2.126503in}}%
\pgfpathlineto{\pgfqpoint{1.411376in}{2.049658in}}%
\pgfpathlineto{\pgfqpoint{1.412251in}{1.944873in}}%
\pgfpathlineto{\pgfqpoint{1.414002in}{1.681353in}}%
\pgfpathlineto{\pgfqpoint{1.416628in}{1.498485in}}%
\pgfpathlineto{\pgfqpoint{1.417504in}{1.487731in}}%
\pgfpathlineto{\pgfqpoint{1.418379in}{1.455441in}}%
\pgfpathlineto{\pgfqpoint{1.419254in}{1.444144in}}%
\pgfpathlineto{\pgfqpoint{1.420130in}{1.476615in}}%
\pgfpathlineto{\pgfqpoint{1.422756in}{2.083608in}}%
\pgfpathlineto{\pgfqpoint{1.423631in}{2.190841in}}%
\pgfpathlineto{\pgfqpoint{1.424507in}{2.224664in}}%
\pgfpathlineto{\pgfqpoint{1.425382in}{2.270772in}}%
\pgfpathlineto{\pgfqpoint{1.426257in}{2.276648in}}%
\pgfpathlineto{\pgfqpoint{1.427133in}{2.322026in}}%
\pgfpathlineto{\pgfqpoint{1.428008in}{2.563352in}}%
\pgfpathlineto{\pgfqpoint{1.428884in}{2.592325in}}%
\pgfpathlineto{\pgfqpoint{1.430634in}{2.434063in}}%
\pgfpathlineto{\pgfqpoint{1.432385in}{2.198303in}}%
\pgfpathlineto{\pgfqpoint{1.433260in}{2.157011in}}%
\pgfpathlineto{\pgfqpoint{1.434136in}{2.049326in}}%
\pgfpathlineto{\pgfqpoint{1.435011in}{1.992780in}}%
\pgfpathlineto{\pgfqpoint{1.435887in}{1.665495in}}%
\pgfpathlineto{\pgfqpoint{1.436762in}{1.594118in}}%
\pgfpathlineto{\pgfqpoint{1.437637in}{1.577957in}}%
\pgfpathlineto{\pgfqpoint{1.438513in}{1.578018in}}%
\pgfpathlineto{\pgfqpoint{1.439388in}{1.515974in}}%
\pgfpathlineto{\pgfqpoint{1.440263in}{1.508332in}}%
\pgfpathlineto{\pgfqpoint{1.441139in}{1.489634in}}%
\pgfpathlineto{\pgfqpoint{1.442890in}{1.691973in}}%
\pgfpathlineto{\pgfqpoint{1.443765in}{1.609987in}}%
\pgfpathlineto{\pgfqpoint{1.444640in}{1.635109in}}%
\pgfpathlineto{\pgfqpoint{1.445516in}{1.716534in}}%
\pgfpathlineto{\pgfqpoint{1.446391in}{1.660605in}}%
\pgfpathlineto{\pgfqpoint{1.447266in}{1.634005in}}%
\pgfpathlineto{\pgfqpoint{1.448142in}{1.753848in}}%
\pgfpathlineto{\pgfqpoint{1.449017in}{1.778562in}}%
\pgfpathlineto{\pgfqpoint{1.449893in}{1.924935in}}%
\pgfpathlineto{\pgfqpoint{1.450768in}{1.981241in}}%
\pgfpathlineto{\pgfqpoint{1.451643in}{1.855976in}}%
\pgfpathlineto{\pgfqpoint{1.452519in}{1.848696in}}%
\pgfpathlineto{\pgfqpoint{1.454269in}{1.692801in}}%
\pgfpathlineto{\pgfqpoint{1.455145in}{1.714188in}}%
\pgfpathlineto{\pgfqpoint{1.456020in}{1.577867in}}%
\pgfpathlineto{\pgfqpoint{1.456896in}{1.541136in}}%
\pgfpathlineto{\pgfqpoint{1.457771in}{1.523375in}}%
\pgfpathlineto{\pgfqpoint{1.458646in}{1.560951in}}%
\pgfpathlineto{\pgfqpoint{1.459522in}{1.501445in}}%
\pgfpathlineto{\pgfqpoint{1.460397in}{1.465469in}}%
\pgfpathlineto{\pgfqpoint{1.461272in}{1.482505in}}%
\pgfpathlineto{\pgfqpoint{1.462148in}{1.516034in}}%
\pgfpathlineto{\pgfqpoint{1.463023in}{1.627888in}}%
\pgfpathlineto{\pgfqpoint{1.463899in}{1.697661in}}%
\pgfpathlineto{\pgfqpoint{1.464774in}{1.790114in}}%
\pgfpathlineto{\pgfqpoint{1.466525in}{2.065216in}}%
\pgfpathlineto{\pgfqpoint{1.468275in}{2.189421in}}%
\pgfpathlineto{\pgfqpoint{1.469151in}{2.701481in}}%
\pgfpathlineto{\pgfqpoint{1.470026in}{2.636690in}}%
\pgfpathlineto{\pgfqpoint{1.470902in}{2.667664in}}%
\pgfpathlineto{\pgfqpoint{1.471777in}{2.584932in}}%
\pgfpathlineto{\pgfqpoint{1.473528in}{2.355376in}}%
\pgfpathlineto{\pgfqpoint{1.476154in}{2.153538in}}%
\pgfpathlineto{\pgfqpoint{1.478780in}{1.811693in}}%
\pgfpathlineto{\pgfqpoint{1.479655in}{1.763363in}}%
\pgfpathlineto{\pgfqpoint{1.480531in}{1.683135in}}%
\pgfpathlineto{\pgfqpoint{1.481406in}{1.655829in}}%
\pgfpathlineto{\pgfqpoint{1.482281in}{1.676279in}}%
\pgfpathlineto{\pgfqpoint{1.483157in}{1.775748in}}%
\pgfpathlineto{\pgfqpoint{1.485783in}{2.281685in}}%
\pgfpathlineto{\pgfqpoint{1.486658in}{2.363910in}}%
\pgfpathlineto{\pgfqpoint{1.487534in}{2.393806in}}%
\pgfpathlineto{\pgfqpoint{1.488409in}{2.389369in}}%
\pgfpathlineto{\pgfqpoint{1.489284in}{2.467191in}}%
\pgfpathlineto{\pgfqpoint{1.490160in}{2.489987in}}%
\pgfpathlineto{\pgfqpoint{1.491035in}{2.607219in}}%
\pgfpathlineto{\pgfqpoint{1.491911in}{2.621698in}}%
\pgfpathlineto{\pgfqpoint{1.492786in}{2.554974in}}%
\pgfpathlineto{\pgfqpoint{1.493661in}{2.450918in}}%
\pgfpathlineto{\pgfqpoint{1.495412in}{2.201687in}}%
\pgfpathlineto{\pgfqpoint{1.496287in}{2.144385in}}%
\pgfpathlineto{\pgfqpoint{1.497163in}{2.049326in}}%
\pgfpathlineto{\pgfqpoint{1.498038in}{1.864554in}}%
\pgfpathlineto{\pgfqpoint{1.498914in}{1.827673in}}%
\pgfpathlineto{\pgfqpoint{1.500664in}{1.582337in}}%
\pgfpathlineto{\pgfqpoint{1.501540in}{1.576749in}}%
\pgfpathlineto{\pgfqpoint{1.502415in}{1.585690in}}%
\pgfpathlineto{\pgfqpoint{1.503290in}{1.583153in}}%
\pgfpathlineto{\pgfqpoint{1.504166in}{1.696034in}}%
\pgfpathlineto{\pgfqpoint{1.506792in}{2.540693in}}%
\pgfpathlineto{\pgfqpoint{1.507667in}{2.709123in}}%
\pgfpathlineto{\pgfqpoint{1.508543in}{2.711053in}}%
\pgfpathlineto{\pgfqpoint{1.509418in}{2.707889in}}%
\pgfpathlineto{\pgfqpoint{1.512044in}{2.843235in}}%
\pgfpathlineto{\pgfqpoint{1.512920in}{2.877969in}}%
\pgfpathlineto{\pgfqpoint{1.514670in}{2.542474in}}%
\pgfpathlineto{\pgfqpoint{1.515546in}{2.372473in}}%
\pgfpathlineto{\pgfqpoint{1.516421in}{2.325170in}}%
\pgfpathlineto{\pgfqpoint{1.517296in}{2.355014in}}%
\pgfpathlineto{\pgfqpoint{1.518172in}{2.254578in}}%
\pgfpathlineto{\pgfqpoint{1.519047in}{2.108742in}}%
\pgfpathlineto{\pgfqpoint{1.520798in}{1.922581in}}%
\pgfpathlineto{\pgfqpoint{1.521673in}{1.925692in}}%
\pgfpathlineto{\pgfqpoint{1.522549in}{1.903460in}}%
\pgfpathlineto{\pgfqpoint{1.523424in}{1.841447in}}%
\pgfpathlineto{\pgfqpoint{1.524299in}{1.813808in}}%
\pgfpathlineto{\pgfqpoint{1.525175in}{1.964416in}}%
\pgfpathlineto{\pgfqpoint{1.528676in}{2.814887in}}%
\pgfpathlineto{\pgfqpoint{1.529552in}{2.859939in}}%
\pgfpathlineto{\pgfqpoint{1.530427in}{2.881354in}}%
\pgfpathlineto{\pgfqpoint{1.532178in}{2.972975in}}%
\pgfpathlineto{\pgfqpoint{1.533053in}{2.962169in}}%
\pgfpathlineto{\pgfqpoint{1.533929in}{2.967402in}}%
\pgfpathlineto{\pgfqpoint{1.537430in}{2.368274in}}%
\pgfpathlineto{\pgfqpoint{1.538306in}{2.376762in}}%
\pgfpathlineto{\pgfqpoint{1.539181in}{2.267083in}}%
\pgfpathlineto{\pgfqpoint{1.540932in}{2.003956in}}%
\pgfpathlineto{\pgfqpoint{1.541807in}{1.907417in}}%
\pgfpathlineto{\pgfqpoint{1.542682in}{1.885669in}}%
\pgfpathlineto{\pgfqpoint{1.543558in}{1.824018in}}%
\pgfpathlineto{\pgfqpoint{1.544433in}{1.784266in}}%
\pgfpathlineto{\pgfqpoint{1.545309in}{1.791214in}}%
\pgfpathlineto{\pgfqpoint{1.546184in}{1.848636in}}%
\pgfpathlineto{\pgfqpoint{1.548810in}{2.471317in}}%
\pgfpathlineto{\pgfqpoint{1.549685in}{2.513088in}}%
\pgfpathlineto{\pgfqpoint{1.550561in}{2.696641in}}%
\pgfpathlineto{\pgfqpoint{1.551436in}{2.808739in}}%
\pgfpathlineto{\pgfqpoint{1.552312in}{2.870643in}}%
\pgfpathlineto{\pgfqpoint{1.553187in}{2.822904in}}%
\pgfpathlineto{\pgfqpoint{1.554062in}{2.839250in}}%
\pgfpathlineto{\pgfqpoint{1.555813in}{2.693602in}}%
\pgfpathlineto{\pgfqpoint{1.556688in}{2.582618in}}%
\pgfpathlineto{\pgfqpoint{1.557564in}{2.411983in}}%
\pgfpathlineto{\pgfqpoint{1.559315in}{2.195041in}}%
\pgfpathlineto{\pgfqpoint{1.560190in}{2.120764in}}%
\pgfpathlineto{\pgfqpoint{1.561065in}{1.956442in}}%
\pgfpathlineto{\pgfqpoint{1.563691in}{1.669815in}}%
\pgfpathlineto{\pgfqpoint{1.564567in}{1.616259in}}%
\pgfpathlineto{\pgfqpoint{1.565442in}{1.605113in}}%
\pgfpathlineto{\pgfqpoint{1.566318in}{1.604207in}}%
\pgfpathlineto{\pgfqpoint{1.567193in}{1.689449in}}%
\pgfpathlineto{\pgfqpoint{1.569819in}{2.434024in}}%
\pgfpathlineto{\pgfqpoint{1.570694in}{2.482933in}}%
\pgfpathlineto{\pgfqpoint{1.571570in}{2.550638in}}%
\pgfpathlineto{\pgfqpoint{1.572445in}{2.579700in}}%
\pgfpathlineto{\pgfqpoint{1.573321in}{2.535939in}}%
\pgfpathlineto{\pgfqpoint{1.574196in}{2.755471in}}%
\pgfpathlineto{\pgfqpoint{1.575071in}{2.802199in}}%
\pgfpathlineto{\pgfqpoint{1.575947in}{2.705074in}}%
\pgfpathlineto{\pgfqpoint{1.576822in}{2.709001in}}%
\pgfpathlineto{\pgfqpoint{1.577697in}{2.629981in}}%
\pgfpathlineto{\pgfqpoint{1.578573in}{2.469556in}}%
\pgfpathlineto{\pgfqpoint{1.579448in}{2.387485in}}%
\pgfpathlineto{\pgfqpoint{1.580324in}{2.335561in}}%
\pgfpathlineto{\pgfqpoint{1.581199in}{2.302515in}}%
\pgfpathlineto{\pgfqpoint{1.582074in}{2.232980in}}%
\pgfpathlineto{\pgfqpoint{1.582950in}{1.961577in}}%
\pgfpathlineto{\pgfqpoint{1.583825in}{1.948196in}}%
\pgfpathlineto{\pgfqpoint{1.584700in}{1.944118in}}%
\pgfpathlineto{\pgfqpoint{1.585576in}{1.887481in}}%
\pgfpathlineto{\pgfqpoint{1.586451in}{1.860658in}}%
\pgfpathlineto{\pgfqpoint{1.587327in}{2.002869in}}%
\pgfpathlineto{\pgfqpoint{1.588202in}{2.071014in}}%
\pgfpathlineto{\pgfqpoint{1.589077in}{2.188849in}}%
\pgfpathlineto{\pgfqpoint{1.589953in}{2.143705in}}%
\pgfpathlineto{\pgfqpoint{1.590828in}{2.374979in}}%
\pgfpathlineto{\pgfqpoint{1.591703in}{2.523482in}}%
\pgfpathlineto{\pgfqpoint{1.592579in}{2.599256in}}%
\pgfpathlineto{\pgfqpoint{1.593454in}{2.572890in}}%
\pgfpathlineto{\pgfqpoint{1.594330in}{2.504998in}}%
\pgfpathlineto{\pgfqpoint{1.595205in}{2.506143in}}%
\pgfpathlineto{\pgfqpoint{1.596080in}{2.582773in}}%
\pgfpathlineto{\pgfqpoint{1.596956in}{2.576181in}}%
\pgfpathlineto{\pgfqpoint{1.597831in}{2.515270in}}%
\pgfpathlineto{\pgfqpoint{1.598706in}{2.391201in}}%
\pgfpathlineto{\pgfqpoint{1.599582in}{2.221019in}}%
\pgfpathlineto{\pgfqpoint{1.600457in}{2.193863in}}%
\pgfpathlineto{\pgfqpoint{1.601333in}{2.142845in}}%
\pgfpathlineto{\pgfqpoint{1.602208in}{2.029843in}}%
\pgfpathlineto{\pgfqpoint{1.603083in}{1.880141in}}%
\pgfpathlineto{\pgfqpoint{1.607460in}{1.489755in}}%
\pgfpathlineto{\pgfqpoint{1.608336in}{1.473232in}}%
\pgfpathlineto{\pgfqpoint{1.609211in}{1.543764in}}%
\pgfpathlineto{\pgfqpoint{1.611837in}{2.096602in}}%
\pgfpathlineto{\pgfqpoint{1.613588in}{2.367093in}}%
\pgfpathlineto{\pgfqpoint{1.614463in}{2.355254in}}%
\pgfpathlineto{\pgfqpoint{1.615339in}{2.365940in}}%
\pgfpathlineto{\pgfqpoint{1.616214in}{2.413318in}}%
\pgfpathlineto{\pgfqpoint{1.617965in}{2.665881in}}%
\pgfpathlineto{\pgfqpoint{1.618840in}{2.640630in}}%
\pgfpathlineto{\pgfqpoint{1.619715in}{2.530844in}}%
\pgfpathlineto{\pgfqpoint{1.620591in}{2.140579in}}%
\pgfpathlineto{\pgfqpoint{1.621466in}{2.273306in}}%
\pgfpathlineto{\pgfqpoint{1.622342in}{2.170363in}}%
\pgfpathlineto{\pgfqpoint{1.624968in}{1.650483in}}%
\pgfpathlineto{\pgfqpoint{1.625843in}{1.626197in}}%
\pgfpathlineto{\pgfqpoint{1.626718in}{1.614024in}}%
\pgfpathlineto{\pgfqpoint{1.627594in}{1.567234in}}%
\pgfpathlineto{\pgfqpoint{1.628469in}{1.538991in}}%
\pgfpathlineto{\pgfqpoint{1.630220in}{1.658155in}}%
\pgfpathlineto{\pgfqpoint{1.631095in}{1.830572in}}%
\pgfpathlineto{\pgfqpoint{1.632846in}{2.286020in}}%
\pgfpathlineto{\pgfqpoint{1.633721in}{2.301611in}}%
\pgfpathlineto{\pgfqpoint{1.635472in}{2.443918in}}%
\pgfpathlineto{\pgfqpoint{1.636348in}{2.452909in}}%
\pgfpathlineto{\pgfqpoint{1.637223in}{2.568362in}}%
\pgfpathlineto{\pgfqpoint{1.638098in}{2.616677in}}%
\pgfpathlineto{\pgfqpoint{1.638974in}{2.739187in}}%
\pgfpathlineto{\pgfqpoint{1.640724in}{2.486139in}}%
\pgfpathlineto{\pgfqpoint{1.641600in}{2.414037in}}%
\pgfpathlineto{\pgfqpoint{1.642475in}{2.468227in}}%
\pgfpathlineto{\pgfqpoint{1.643351in}{2.448381in}}%
\pgfpathlineto{\pgfqpoint{1.644226in}{2.346042in}}%
\pgfpathlineto{\pgfqpoint{1.645977in}{1.998519in}}%
\pgfpathlineto{\pgfqpoint{1.646852in}{1.963087in}}%
\pgfpathlineto{\pgfqpoint{1.647727in}{1.824350in}}%
\pgfpathlineto{\pgfqpoint{1.648603in}{1.766293in}}%
\pgfpathlineto{\pgfqpoint{1.649478in}{1.752791in}}%
\pgfpathlineto{\pgfqpoint{1.650354in}{1.704280in}}%
\pgfpathlineto{\pgfqpoint{1.651229in}{1.801846in}}%
\pgfpathlineto{\pgfqpoint{1.652980in}{2.230885in}}%
\pgfpathlineto{\pgfqpoint{1.653855in}{2.348493in}}%
\pgfpathlineto{\pgfqpoint{1.654730in}{2.363511in}}%
\pgfpathlineto{\pgfqpoint{1.655606in}{2.553328in}}%
\pgfpathlineto{\pgfqpoint{1.656481in}{2.593796in}}%
\pgfpathlineto{\pgfqpoint{1.657357in}{2.464654in}}%
\pgfpathlineto{\pgfqpoint{1.658232in}{2.499636in}}%
\pgfpathlineto{\pgfqpoint{1.659107in}{2.601394in}}%
\pgfpathlineto{\pgfqpoint{1.659983in}{2.751042in}}%
\pgfpathlineto{\pgfqpoint{1.660858in}{2.675921in}}%
\pgfpathlineto{\pgfqpoint{1.662609in}{2.381082in}}%
\pgfpathlineto{\pgfqpoint{1.665235in}{2.026732in}}%
\pgfpathlineto{\pgfqpoint{1.666110in}{1.848787in}}%
\pgfpathlineto{\pgfqpoint{1.666986in}{1.732764in}}%
\pgfpathlineto{\pgfqpoint{1.667861in}{1.699477in}}%
\pgfpathlineto{\pgfqpoint{1.668737in}{1.528661in}}%
\pgfpathlineto{\pgfqpoint{1.669612in}{1.283204in}}%
\pgfpathlineto{\pgfqpoint{1.670487in}{1.218895in}}%
\pgfpathlineto{\pgfqpoint{1.671363in}{1.245930in}}%
\pgfpathlineto{\pgfqpoint{1.672238in}{1.414179in}}%
\pgfpathlineto{\pgfqpoint{1.673113in}{1.693708in}}%
\pgfpathlineto{\pgfqpoint{1.674864in}{2.096257in}}%
\pgfpathlineto{\pgfqpoint{1.675740in}{2.166599in}}%
\pgfpathlineto{\pgfqpoint{1.676615in}{2.121070in}}%
\pgfpathlineto{\pgfqpoint{1.677490in}{2.113532in}}%
\pgfpathlineto{\pgfqpoint{1.678366in}{2.164363in}}%
\pgfpathlineto{\pgfqpoint{1.679241in}{2.295529in}}%
\pgfpathlineto{\pgfqpoint{1.680116in}{2.335534in}}%
\pgfpathlineto{\pgfqpoint{1.680992in}{2.638846in}}%
\pgfpathlineto{\pgfqpoint{1.682743in}{2.408992in}}%
\pgfpathlineto{\pgfqpoint{1.683618in}{2.408932in}}%
\pgfpathlineto{\pgfqpoint{1.684493in}{2.286657in}}%
\pgfpathlineto{\pgfqpoint{1.685369in}{2.094031in}}%
\pgfpathlineto{\pgfqpoint{1.686244in}{2.128074in}}%
\pgfpathlineto{\pgfqpoint{1.688870in}{1.606291in}}%
\pgfpathlineto{\pgfqpoint{1.689746in}{1.506127in}}%
\pgfpathlineto{\pgfqpoint{1.690621in}{1.468973in}}%
\pgfpathlineto{\pgfqpoint{1.691496in}{1.448161in}}%
\pgfpathlineto{\pgfqpoint{1.694122in}{1.800396in}}%
\pgfpathlineto{\pgfqpoint{1.694998in}{1.907556in}}%
\pgfpathlineto{\pgfqpoint{1.695873in}{2.121987in}}%
\pgfpathlineto{\pgfqpoint{1.697624in}{2.313736in}}%
\pgfpathlineto{\pgfqpoint{1.698499in}{2.341638in}}%
\pgfpathlineto{\pgfqpoint{1.699375in}{2.381947in}}%
\pgfpathlineto{\pgfqpoint{1.700250in}{2.527780in}}%
\pgfpathlineto{\pgfqpoint{1.701125in}{2.207892in}}%
\pgfpathlineto{\pgfqpoint{1.702001in}{2.342311in}}%
\pgfpathlineto{\pgfqpoint{1.702876in}{2.352925in}}%
\pgfpathlineto{\pgfqpoint{1.703752in}{2.244459in}}%
\pgfpathlineto{\pgfqpoint{1.704627in}{2.012354in}}%
\pgfpathlineto{\pgfqpoint{1.705502in}{1.959493in}}%
\pgfpathlineto{\pgfqpoint{1.706378in}{1.850750in}}%
\pgfpathlineto{\pgfqpoint{1.707253in}{1.838577in}}%
\pgfpathlineto{\pgfqpoint{1.708128in}{1.636104in}}%
\pgfpathlineto{\pgfqpoint{1.709879in}{1.476525in}}%
\pgfpathlineto{\pgfqpoint{1.710755in}{1.404301in}}%
\pgfpathlineto{\pgfqpoint{1.711630in}{1.376240in}}%
\pgfpathlineto{\pgfqpoint{1.713381in}{1.344342in}}%
\pgfpathlineto{\pgfqpoint{1.714256in}{1.478307in}}%
\pgfpathlineto{\pgfqpoint{1.715131in}{1.655342in}}%
\pgfpathlineto{\pgfqpoint{1.716007in}{1.906307in}}%
\pgfpathlineto{\pgfqpoint{1.717758in}{2.244243in}}%
\pgfpathlineto{\pgfqpoint{1.718633in}{2.275571in}}%
\pgfpathlineto{\pgfqpoint{1.719508in}{2.317926in}}%
\pgfpathlineto{\pgfqpoint{1.720384in}{2.319911in}}%
\pgfpathlineto{\pgfqpoint{1.721259in}{2.253177in}}%
\pgfpathlineto{\pgfqpoint{1.722134in}{2.313147in}}%
\pgfpathlineto{\pgfqpoint{1.723010in}{2.259932in}}%
\pgfpathlineto{\pgfqpoint{1.723885in}{2.249666in}}%
\pgfpathlineto{\pgfqpoint{1.724761in}{2.134749in}}%
\pgfpathlineto{\pgfqpoint{1.725636in}{1.967165in}}%
\pgfpathlineto{\pgfqpoint{1.726511in}{1.858604in}}%
\pgfpathlineto{\pgfqpoint{1.728262in}{1.723521in}}%
\pgfpathlineto{\pgfqpoint{1.730013in}{1.487489in}}%
\pgfpathlineto{\pgfqpoint{1.732639in}{1.313622in}}%
\pgfpathlineto{\pgfqpoint{1.733514in}{1.285893in}}%
\pgfpathlineto{\pgfqpoint{1.734390in}{1.293535in}}%
\pgfpathlineto{\pgfqpoint{1.735265in}{1.358962in}}%
\pgfpathlineto{\pgfqpoint{1.737891in}{1.887858in}}%
\pgfpathlineto{\pgfqpoint{1.739642in}{1.983210in}}%
\pgfpathlineto{\pgfqpoint{1.740517in}{1.866654in}}%
\pgfpathlineto{\pgfqpoint{1.741393in}{1.933888in}}%
\pgfpathlineto{\pgfqpoint{1.742268in}{1.967501in}}%
\pgfpathlineto{\pgfqpoint{1.743143in}{2.062258in}}%
\pgfpathlineto{\pgfqpoint{1.744019in}{2.045103in}}%
\pgfpathlineto{\pgfqpoint{1.745770in}{1.872982in}}%
\pgfpathlineto{\pgfqpoint{1.746645in}{1.702075in}}%
\pgfpathlineto{\pgfqpoint{1.747520in}{1.600884in}}%
\pgfpathlineto{\pgfqpoint{1.749271in}{1.503348in}}%
\pgfpathlineto{\pgfqpoint{1.751022in}{1.234874in}}%
\pgfpathlineto{\pgfqpoint{1.751897in}{1.181512in}}%
\pgfpathlineto{\pgfqpoint{1.752773in}{1.110056in}}%
\pgfpathlineto{\pgfqpoint{1.754523in}{1.032903in}}%
\pgfpathlineto{\pgfqpoint{1.756274in}{1.133895in}}%
\pgfpathlineto{\pgfqpoint{1.758025in}{1.351508in}}%
\pgfpathlineto{\pgfqpoint{1.758900in}{1.562107in}}%
\pgfpathlineto{\pgfqpoint{1.759776in}{1.588510in}}%
\pgfpathlineto{\pgfqpoint{1.760651in}{1.703217in}}%
\pgfpathlineto{\pgfqpoint{1.761526in}{1.762834in}}%
\pgfpathlineto{\pgfqpoint{1.762402in}{1.761824in}}%
\pgfpathlineto{\pgfqpoint{1.763277in}{1.823516in}}%
\pgfpathlineto{\pgfqpoint{1.764152in}{1.998876in}}%
\pgfpathlineto{\pgfqpoint{1.765028in}{1.972674in}}%
\pgfpathlineto{\pgfqpoint{1.765903in}{1.990876in}}%
\pgfpathlineto{\pgfqpoint{1.766779in}{1.957620in}}%
\pgfpathlineto{\pgfqpoint{1.768529in}{1.830754in}}%
\pgfpathlineto{\pgfqpoint{1.769405in}{1.805894in}}%
\pgfpathlineto{\pgfqpoint{1.771155in}{1.622391in}}%
\pgfpathlineto{\pgfqpoint{1.772906in}{1.474773in}}%
\pgfpathlineto{\pgfqpoint{1.773782in}{1.406446in}}%
\pgfpathlineto{\pgfqpoint{1.774657in}{1.359898in}}%
\pgfpathlineto{\pgfqpoint{1.775532in}{1.337999in}}%
\pgfpathlineto{\pgfqpoint{1.776408in}{1.269581in}}%
\pgfpathlineto{\pgfqpoint{1.777283in}{1.367933in}}%
\pgfpathlineto{\pgfqpoint{1.779034in}{1.765511in}}%
\pgfpathlineto{\pgfqpoint{1.779909in}{2.006997in}}%
\pgfpathlineto{\pgfqpoint{1.780785in}{2.328871in}}%
\pgfpathlineto{\pgfqpoint{1.781660in}{2.369992in}}%
\pgfpathlineto{\pgfqpoint{1.782535in}{2.266769in}}%
\pgfpathlineto{\pgfqpoint{1.784286in}{2.488482in}}%
\pgfpathlineto{\pgfqpoint{1.785161in}{2.516358in}}%
\pgfpathlineto{\pgfqpoint{1.786037in}{2.714817in}}%
\pgfpathlineto{\pgfqpoint{1.787788in}{2.469374in}}%
\pgfpathlineto{\pgfqpoint{1.788663in}{2.405307in}}%
\pgfpathlineto{\pgfqpoint{1.789538in}{2.279679in}}%
\pgfpathlineto{\pgfqpoint{1.790414in}{2.293181in}}%
\pgfpathlineto{\pgfqpoint{1.791289in}{2.329097in}}%
\pgfpathlineto{\pgfqpoint{1.792164in}{2.316501in}}%
\pgfpathlineto{\pgfqpoint{1.793040in}{2.335772in}}%
\pgfpathlineto{\pgfqpoint{1.793915in}{2.156709in}}%
\pgfpathlineto{\pgfqpoint{1.794791in}{2.079925in}}%
\pgfpathlineto{\pgfqpoint{1.795666in}{2.065909in}}%
\pgfpathlineto{\pgfqpoint{1.796541in}{2.023530in}}%
\pgfpathlineto{\pgfqpoint{1.797417in}{2.008880in}}%
\pgfpathlineto{\pgfqpoint{1.798292in}{2.101190in}}%
\pgfpathlineto{\pgfqpoint{1.800918in}{2.965234in}}%
\pgfpathlineto{\pgfqpoint{1.801794in}{3.041317in}}%
\pgfpathlineto{\pgfqpoint{1.802669in}{3.169545in}}%
\pgfpathlineto{\pgfqpoint{1.803544in}{3.151692in}}%
\pgfpathlineto{\pgfqpoint{1.805295in}{3.270324in}}%
\pgfpathlineto{\pgfqpoint{1.806171in}{3.331307in}}%
\pgfpathlineto{\pgfqpoint{1.807046in}{3.292556in}}%
\pgfpathlineto{\pgfqpoint{1.808797in}{2.900629in}}%
\pgfpathlineto{\pgfqpoint{1.809672in}{2.548213in}}%
\pgfpathlineto{\pgfqpoint{1.810547in}{2.444636in}}%
\pgfpathlineto{\pgfqpoint{1.811423in}{2.484689in}}%
\pgfpathlineto{\pgfqpoint{1.812298in}{2.456023in}}%
\pgfpathlineto{\pgfqpoint{1.813174in}{2.265633in}}%
\pgfpathlineto{\pgfqpoint{1.816675in}{1.836372in}}%
\pgfpathlineto{\pgfqpoint{1.817550in}{1.773029in}}%
\pgfpathlineto{\pgfqpoint{1.818426in}{1.735725in}}%
\pgfpathlineto{\pgfqpoint{1.820177in}{1.993841in}}%
\pgfpathlineto{\pgfqpoint{1.822803in}{2.552883in}}%
\pgfpathlineto{\pgfqpoint{1.823678in}{2.564879in}}%
\pgfpathlineto{\pgfqpoint{1.824553in}{2.605270in}}%
\pgfpathlineto{\pgfqpoint{1.825429in}{2.623395in}}%
\pgfpathlineto{\pgfqpoint{1.826304in}{2.760541in}}%
\pgfpathlineto{\pgfqpoint{1.827180in}{2.770538in}}%
\pgfpathlineto{\pgfqpoint{1.828055in}{2.751455in}}%
\pgfpathlineto{\pgfqpoint{1.829806in}{2.522809in}}%
\pgfpathlineto{\pgfqpoint{1.832432in}{2.222378in}}%
\pgfpathlineto{\pgfqpoint{1.834183in}{1.932579in}}%
\pgfpathlineto{\pgfqpoint{1.835058in}{1.747837in}}%
\pgfpathlineto{\pgfqpoint{1.835933in}{1.680840in}}%
\pgfpathlineto{\pgfqpoint{1.836809in}{1.536242in}}%
\pgfpathlineto{\pgfqpoint{1.837684in}{1.506217in}}%
\pgfpathlineto{\pgfqpoint{1.838559in}{1.428044in}}%
\pgfpathlineto{\pgfqpoint{1.839435in}{1.432242in}}%
\pgfpathlineto{\pgfqpoint{1.840310in}{1.473987in}}%
\pgfpathlineto{\pgfqpoint{1.842936in}{2.126851in}}%
\pgfpathlineto{\pgfqpoint{1.844687in}{2.388733in}}%
\pgfpathlineto{\pgfqpoint{1.845562in}{2.136883in}}%
\pgfpathlineto{\pgfqpoint{1.847313in}{2.424413in}}%
\pgfpathlineto{\pgfqpoint{1.848189in}{2.580195in}}%
\pgfpathlineto{\pgfqpoint{1.849064in}{2.604482in}}%
\pgfpathlineto{\pgfqpoint{1.849939in}{2.545912in}}%
\pgfpathlineto{\pgfqpoint{1.851690in}{2.178881in}}%
\pgfpathlineto{\pgfqpoint{1.852565in}{2.026822in}}%
\pgfpathlineto{\pgfqpoint{1.853441in}{1.805894in}}%
\pgfpathlineto{\pgfqpoint{1.854316in}{1.709506in}}%
\pgfpathlineto{\pgfqpoint{1.856067in}{1.456981in}}%
\pgfpathlineto{\pgfqpoint{1.857818in}{1.226507in}}%
\pgfpathlineto{\pgfqpoint{1.858693in}{1.147548in}}%
\pgfpathlineto{\pgfqpoint{1.859568in}{1.299848in}}%
\pgfpathlineto{\pgfqpoint{1.860444in}{1.272663in}}%
\pgfpathlineto{\pgfqpoint{1.861319in}{1.359475in}}%
\pgfpathlineto{\pgfqpoint{1.863070in}{1.878717in}}%
\pgfpathlineto{\pgfqpoint{1.863945in}{2.218599in}}%
\pgfpathlineto{\pgfqpoint{1.866571in}{2.713679in}}%
\pgfpathlineto{\pgfqpoint{1.867447in}{2.677767in}}%
\pgfpathlineto{\pgfqpoint{1.868322in}{2.708800in}}%
\pgfpathlineto{\pgfqpoint{1.869198in}{2.761455in}}%
\pgfpathlineto{\pgfqpoint{1.870073in}{2.743284in}}%
\pgfpathlineto{\pgfqpoint{1.873574in}{2.046185in}}%
\pgfpathlineto{\pgfqpoint{1.874450in}{1.918835in}}%
\pgfpathlineto{\pgfqpoint{1.875325in}{1.887964in}}%
\pgfpathlineto{\pgfqpoint{1.877076in}{1.487701in}}%
\pgfpathlineto{\pgfqpoint{1.877951in}{1.378143in}}%
\pgfpathlineto{\pgfqpoint{1.878827in}{1.368869in}}%
\pgfpathlineto{\pgfqpoint{1.879702in}{1.284896in}}%
\pgfpathlineto{\pgfqpoint{1.880577in}{1.394726in}}%
\pgfpathlineto{\pgfqpoint{1.881453in}{1.330538in}}%
\pgfpathlineto{\pgfqpoint{1.882328in}{1.417109in}}%
\pgfpathlineto{\pgfqpoint{1.884079in}{1.870006in}}%
\pgfpathlineto{\pgfqpoint{1.884954in}{2.051716in}}%
\pgfpathlineto{\pgfqpoint{1.886705in}{2.230901in}}%
\pgfpathlineto{\pgfqpoint{1.887580in}{2.249852in}}%
\pgfpathlineto{\pgfqpoint{1.888456in}{2.237323in}}%
\pgfpathlineto{\pgfqpoint{1.889331in}{2.268725in}}%
\pgfpathlineto{\pgfqpoint{1.890207in}{2.459201in}}%
\pgfpathlineto{\pgfqpoint{1.891082in}{2.491905in}}%
\pgfpathlineto{\pgfqpoint{1.893708in}{2.126352in}}%
\pgfpathlineto{\pgfqpoint{1.894583in}{2.072162in}}%
\pgfpathlineto{\pgfqpoint{1.895459in}{1.869810in}}%
\pgfpathlineto{\pgfqpoint{1.896334in}{1.738806in}}%
\pgfpathlineto{\pgfqpoint{1.897210in}{1.508241in}}%
\pgfpathlineto{\pgfqpoint{1.899836in}{1.173979in}}%
\pgfpathlineto{\pgfqpoint{1.900711in}{1.101514in}}%
\pgfpathlineto{\pgfqpoint{1.901586in}{1.102994in}}%
\pgfpathlineto{\pgfqpoint{1.902462in}{1.060282in}}%
\pgfpathlineto{\pgfqpoint{1.903337in}{1.131690in}}%
\pgfpathlineto{\pgfqpoint{1.905963in}{1.692214in}}%
\pgfpathlineto{\pgfqpoint{1.906839in}{1.753525in}}%
\pgfpathlineto{\pgfqpoint{1.907714in}{1.934686in}}%
\pgfpathlineto{\pgfqpoint{1.908589in}{1.784105in}}%
\pgfpathlineto{\pgfqpoint{1.909465in}{1.768079in}}%
\pgfpathlineto{\pgfqpoint{1.910340in}{1.885379in}}%
\pgfpathlineto{\pgfqpoint{1.911216in}{2.050873in}}%
\pgfpathlineto{\pgfqpoint{1.912966in}{2.128826in}}%
\pgfpathlineto{\pgfqpoint{1.914717in}{1.866034in}}%
\pgfpathlineto{\pgfqpoint{1.915592in}{1.723914in}}%
\pgfpathlineto{\pgfqpoint{1.916468in}{1.714157in}}%
\pgfpathlineto{\pgfqpoint{1.917343in}{1.725696in}}%
\pgfpathlineto{\pgfqpoint{1.918219in}{1.489453in}}%
\pgfpathlineto{\pgfqpoint{1.919969in}{1.237774in}}%
\pgfpathlineto{\pgfqpoint{1.920845in}{1.151264in}}%
\pgfpathlineto{\pgfqpoint{1.921720in}{1.128035in}}%
\pgfpathlineto{\pgfqpoint{1.922595in}{1.094355in}}%
\pgfpathlineto{\pgfqpoint{1.923471in}{1.016272in}}%
\pgfpathlineto{\pgfqpoint{1.924346in}{1.250793in}}%
\pgfpathlineto{\pgfqpoint{1.925222in}{1.408538in}}%
\pgfpathlineto{\pgfqpoint{1.926097in}{1.431085in}}%
\pgfpathlineto{\pgfqpoint{1.926972in}{1.606103in}}%
\pgfpathlineto{\pgfqpoint{1.927848in}{1.668917in}}%
\pgfpathlineto{\pgfqpoint{1.928723in}{1.689679in}}%
\pgfpathlineto{\pgfqpoint{1.929599in}{1.667428in}}%
\pgfpathlineto{\pgfqpoint{1.930474in}{1.784291in}}%
\pgfpathlineto{\pgfqpoint{1.933100in}{2.205726in}}%
\pgfpathlineto{\pgfqpoint{1.933975in}{2.143595in}}%
\pgfpathlineto{\pgfqpoint{1.935726in}{1.770311in}}%
\pgfpathlineto{\pgfqpoint{1.936602in}{1.690325in}}%
\pgfpathlineto{\pgfqpoint{1.937477in}{1.744001in}}%
\pgfpathlineto{\pgfqpoint{1.939228in}{1.390860in}}%
\pgfpathlineto{\pgfqpoint{1.940978in}{1.139815in}}%
\pgfpathlineto{\pgfqpoint{1.941854in}{1.118731in}}%
\pgfpathlineto{\pgfqpoint{1.942729in}{1.026844in}}%
\pgfpathlineto{\pgfqpoint{1.943605in}{1.010442in}}%
\pgfpathlineto{\pgfqpoint{1.944480in}{1.071217in}}%
\pgfpathlineto{\pgfqpoint{1.945355in}{1.210347in}}%
\pgfpathlineto{\pgfqpoint{1.947981in}{1.895076in}}%
\pgfpathlineto{\pgfqpoint{1.949732in}{2.196211in}}%
\pgfpathlineto{\pgfqpoint{1.950608in}{2.214983in}}%
\pgfpathlineto{\pgfqpoint{1.951483in}{2.339220in}}%
\pgfpathlineto{\pgfqpoint{1.952358in}{2.624415in}}%
\pgfpathlineto{\pgfqpoint{1.953234in}{2.791791in}}%
\pgfpathlineto{\pgfqpoint{1.954109in}{2.750552in}}%
\pgfpathlineto{\pgfqpoint{1.954984in}{2.660083in}}%
\pgfpathlineto{\pgfqpoint{1.956735in}{2.260770in}}%
\pgfpathlineto{\pgfqpoint{1.958486in}{2.126141in}}%
\pgfpathlineto{\pgfqpoint{1.959361in}{2.010874in}}%
\pgfpathlineto{\pgfqpoint{1.960237in}{1.768740in}}%
\pgfpathlineto{\pgfqpoint{1.961987in}{1.517061in}}%
\pgfpathlineto{\pgfqpoint{1.962863in}{1.445926in}}%
\pgfpathlineto{\pgfqpoint{1.963738in}{1.306856in}}%
\pgfpathlineto{\pgfqpoint{1.964614in}{1.300150in}}%
\pgfpathlineto{\pgfqpoint{1.965489in}{1.247259in}}%
\pgfpathlineto{\pgfqpoint{1.966364in}{1.345822in}}%
\pgfpathlineto{\pgfqpoint{1.968115in}{1.759818in}}%
\pgfpathlineto{\pgfqpoint{1.968990in}{2.062718in}}%
\pgfpathlineto{\pgfqpoint{1.969866in}{2.144706in}}%
\pgfpathlineto{\pgfqpoint{1.970741in}{2.337383in}}%
\pgfpathlineto{\pgfqpoint{1.972492in}{2.441346in}}%
\pgfpathlineto{\pgfqpoint{1.973367in}{2.614123in}}%
\pgfpathlineto{\pgfqpoint{1.974243in}{2.607319in}}%
\pgfpathlineto{\pgfqpoint{1.975118in}{2.701204in}}%
\pgfpathlineto{\pgfqpoint{1.976869in}{2.542927in}}%
\pgfpathlineto{\pgfqpoint{1.977744in}{2.384706in}}%
\pgfpathlineto{\pgfqpoint{1.978620in}{2.335198in}}%
\pgfpathlineto{\pgfqpoint{1.982121in}{1.803659in}}%
\pgfpathlineto{\pgfqpoint{1.982996in}{1.585146in}}%
\pgfpathlineto{\pgfqpoint{1.983872in}{1.438132in}}%
\pgfpathlineto{\pgfqpoint{1.984747in}{1.399408in}}%
\pgfpathlineto{\pgfqpoint{1.985623in}{1.308336in}}%
\pgfpathlineto{\pgfqpoint{1.986498in}{1.298912in}}%
\pgfpathlineto{\pgfqpoint{1.987373in}{1.297250in}}%
\pgfpathlineto{\pgfqpoint{1.988249in}{1.546494in}}%
\pgfpathlineto{\pgfqpoint{1.989124in}{1.673741in}}%
\pgfpathlineto{\pgfqpoint{1.989999in}{1.994155in}}%
\pgfpathlineto{\pgfqpoint{1.990875in}{1.998858in}}%
\pgfpathlineto{\pgfqpoint{1.992626in}{2.268683in}}%
\pgfpathlineto{\pgfqpoint{1.993501in}{2.239008in}}%
\pgfpathlineto{\pgfqpoint{1.994376in}{2.305051in}}%
\pgfpathlineto{\pgfqpoint{1.995252in}{2.419853in}}%
\pgfpathlineto{\pgfqpoint{1.996127in}{2.423574in}}%
\pgfpathlineto{\pgfqpoint{1.997002in}{2.403878in}}%
\pgfpathlineto{\pgfqpoint{1.997878in}{2.198364in}}%
\pgfpathlineto{\pgfqpoint{1.998753in}{1.604297in}}%
\pgfpathlineto{\pgfqpoint{1.999629in}{1.986437in}}%
\pgfpathlineto{\pgfqpoint{2.000504in}{1.963510in}}%
\pgfpathlineto{\pgfqpoint{2.001379in}{1.958224in}}%
\pgfpathlineto{\pgfqpoint{2.002255in}{1.713704in}}%
\pgfpathlineto{\pgfqpoint{2.004881in}{1.371618in}}%
\pgfpathlineto{\pgfqpoint{2.005756in}{1.349840in}}%
\pgfpathlineto{\pgfqpoint{2.006632in}{1.249343in}}%
\pgfpathlineto{\pgfqpoint{2.007507in}{1.260791in}}%
\pgfpathlineto{\pgfqpoint{2.008382in}{1.383157in}}%
\pgfpathlineto{\pgfqpoint{2.011008in}{1.952304in}}%
\pgfpathlineto{\pgfqpoint{2.011884in}{2.063484in}}%
\pgfpathlineto{\pgfqpoint{2.012759in}{2.122664in}}%
\pgfpathlineto{\pgfqpoint{2.013635in}{2.127828in}}%
\pgfpathlineto{\pgfqpoint{2.014510in}{2.130007in}}%
\pgfpathlineto{\pgfqpoint{2.015385in}{2.244006in}}%
\pgfpathlineto{\pgfqpoint{2.016261in}{2.270873in}}%
\pgfpathlineto{\pgfqpoint{2.017136in}{2.228701in}}%
\pgfpathlineto{\pgfqpoint{2.018011in}{2.158907in}}%
\pgfpathlineto{\pgfqpoint{2.021513in}{1.654862in}}%
\pgfpathlineto{\pgfqpoint{2.022388in}{1.612423in}}%
\pgfpathlineto{\pgfqpoint{2.024139in}{1.263570in}}%
\pgfpathlineto{\pgfqpoint{2.026765in}{0.959816in}}%
\pgfpathlineto{\pgfqpoint{2.027641in}{0.953775in}}%
\pgfpathlineto{\pgfqpoint{2.028516in}{0.963532in}}%
\pgfpathlineto{\pgfqpoint{2.029391in}{1.063696in}}%
\pgfpathlineto{\pgfqpoint{2.032017in}{1.626190in}}%
\pgfpathlineto{\pgfqpoint{2.032893in}{1.760807in}}%
\pgfpathlineto{\pgfqpoint{2.033768in}{1.809115in}}%
\pgfpathlineto{\pgfqpoint{2.034644in}{1.829268in}}%
\pgfpathlineto{\pgfqpoint{2.036394in}{1.931986in}}%
\pgfpathlineto{\pgfqpoint{2.037270in}{2.039513in}}%
\pgfpathlineto{\pgfqpoint{2.038145in}{2.083621in}}%
\pgfpathlineto{\pgfqpoint{2.039020in}{2.008113in}}%
\pgfpathlineto{\pgfqpoint{2.039896in}{2.035129in}}%
\pgfpathlineto{\pgfqpoint{2.041647in}{1.845736in}}%
\pgfpathlineto{\pgfqpoint{2.042522in}{1.790549in}}%
\pgfpathlineto{\pgfqpoint{2.043397in}{1.673772in}}%
\pgfpathlineto{\pgfqpoint{2.046023in}{1.175882in}}%
\pgfpathlineto{\pgfqpoint{2.046899in}{1.136553in}}%
\pgfpathlineto{\pgfqpoint{2.047774in}{1.126139in}}%
\pgfpathlineto{\pgfqpoint{2.048650in}{1.110983in}}%
\pgfpathlineto{\pgfqpoint{2.049525in}{1.175602in}}%
\pgfpathlineto{\pgfqpoint{2.050400in}{1.172378in}}%
\pgfpathlineto{\pgfqpoint{2.053026in}{1.678059in}}%
\pgfpathlineto{\pgfqpoint{2.053902in}{1.739525in}}%
\pgfpathlineto{\pgfqpoint{2.054777in}{1.876546in}}%
\pgfpathlineto{\pgfqpoint{2.055653in}{1.908191in}}%
\pgfpathlineto{\pgfqpoint{2.059154in}{2.301608in}}%
\pgfpathlineto{\pgfqpoint{2.060030in}{2.168274in}}%
\pgfpathlineto{\pgfqpoint{2.060905in}{2.105993in}}%
\pgfpathlineto{\pgfqpoint{2.061780in}{1.955868in}}%
\pgfpathlineto{\pgfqpoint{2.062656in}{1.867454in}}%
\pgfpathlineto{\pgfqpoint{2.063531in}{1.727146in}}%
\pgfpathlineto{\pgfqpoint{2.064406in}{1.643626in}}%
\pgfpathlineto{\pgfqpoint{2.067033in}{1.195183in}}%
\pgfpathlineto{\pgfqpoint{2.068783in}{1.101846in}}%
\pgfpathlineto{\pgfqpoint{2.069659in}{1.085082in}}%
\pgfpathlineto{\pgfqpoint{2.070534in}{1.097073in}}%
\pgfpathlineto{\pgfqpoint{2.071409in}{1.197358in}}%
\pgfpathlineto{\pgfqpoint{2.072285in}{1.379599in}}%
\pgfpathlineto{\pgfqpoint{2.074036in}{1.938147in}}%
\pgfpathlineto{\pgfqpoint{2.074911in}{2.060648in}}%
\pgfpathlineto{\pgfqpoint{2.075786in}{2.248779in}}%
\pgfpathlineto{\pgfqpoint{2.077537in}{2.363967in}}%
\pgfpathlineto{\pgfqpoint{2.078412in}{2.349481in}}%
\pgfpathlineto{\pgfqpoint{2.079288in}{2.527527in}}%
\pgfpathlineto{\pgfqpoint{2.080163in}{2.500592in}}%
\pgfpathlineto{\pgfqpoint{2.081914in}{2.186553in}}%
\pgfpathlineto{\pgfqpoint{2.082789in}{2.030508in}}%
\pgfpathlineto{\pgfqpoint{2.085415in}{1.748049in}}%
\pgfpathlineto{\pgfqpoint{2.086291in}{1.523858in}}%
\pgfpathlineto{\pgfqpoint{2.087166in}{1.415599in}}%
\pgfpathlineto{\pgfqpoint{2.088917in}{1.145434in}}%
\pgfpathlineto{\pgfqpoint{2.090668in}{1.061611in}}%
\pgfpathlineto{\pgfqpoint{2.091543in}{1.074268in}}%
\pgfpathlineto{\pgfqpoint{2.092418in}{1.196694in}}%
\pgfpathlineto{\pgfqpoint{2.097671in}{2.242800in}}%
\pgfpathlineto{\pgfqpoint{2.098546in}{2.213017in}}%
\pgfpathlineto{\pgfqpoint{2.099421in}{2.447048in}}%
\pgfpathlineto{\pgfqpoint{2.100297in}{2.462668in}}%
\pgfpathlineto{\pgfqpoint{2.101172in}{2.547274in}}%
\pgfpathlineto{\pgfqpoint{2.102048in}{2.466387in}}%
\pgfpathlineto{\pgfqpoint{2.103798in}{2.146983in}}%
\pgfpathlineto{\pgfqpoint{2.104674in}{2.101371in}}%
\pgfpathlineto{\pgfqpoint{2.105549in}{2.089319in}}%
\pgfpathlineto{\pgfqpoint{2.106424in}{2.043889in}}%
\pgfpathlineto{\pgfqpoint{2.108175in}{1.688814in}}%
\pgfpathlineto{\pgfqpoint{2.109926in}{1.452299in}}%
\pgfpathlineto{\pgfqpoint{2.110801in}{1.316884in}}%
\pgfpathlineto{\pgfqpoint{2.112552in}{1.202644in}}%
\pgfpathlineto{\pgfqpoint{2.113427in}{1.249464in}}%
\pgfpathlineto{\pgfqpoint{2.114303in}{1.460444in}}%
\pgfpathlineto{\pgfqpoint{2.115178in}{1.600673in}}%
\pgfpathlineto{\pgfqpoint{2.116054in}{1.677601in}}%
\pgfpathlineto{\pgfqpoint{2.116929in}{1.846310in}}%
\pgfpathlineto{\pgfqpoint{2.117804in}{1.869057in}}%
\pgfpathlineto{\pgfqpoint{2.119555in}{2.036213in}}%
\pgfpathlineto{\pgfqpoint{2.120430in}{2.265806in}}%
\pgfpathlineto{\pgfqpoint{2.122181in}{2.453402in}}%
\pgfpathlineto{\pgfqpoint{2.123932in}{2.238025in}}%
\pgfpathlineto{\pgfqpoint{2.124807in}{2.184227in}}%
\pgfpathlineto{\pgfqpoint{2.125683in}{1.993444in}}%
\pgfpathlineto{\pgfqpoint{2.126558in}{1.927806in}}%
\pgfpathlineto{\pgfqpoint{2.129184in}{1.459911in}}%
\pgfpathlineto{\pgfqpoint{2.130060in}{1.328393in}}%
\pgfpathlineto{\pgfqpoint{2.132686in}{1.116194in}}%
\pgfpathlineto{\pgfqpoint{2.133561in}{1.158422in}}%
\pgfpathlineto{\pgfqpoint{2.134436in}{1.253965in}}%
\pgfpathlineto{\pgfqpoint{2.136187in}{1.663885in}}%
\pgfpathlineto{\pgfqpoint{2.137938in}{2.074553in}}%
\pgfpathlineto{\pgfqpoint{2.138813in}{2.186802in}}%
\pgfpathlineto{\pgfqpoint{2.139689in}{2.202067in}}%
\pgfpathlineto{\pgfqpoint{2.140564in}{2.267234in}}%
\pgfpathlineto{\pgfqpoint{2.142315in}{2.510136in}}%
\pgfpathlineto{\pgfqpoint{2.143190in}{2.537762in}}%
\pgfpathlineto{\pgfqpoint{2.144066in}{2.409167in}}%
\pgfpathlineto{\pgfqpoint{2.146692in}{1.911888in}}%
\pgfpathlineto{\pgfqpoint{2.147567in}{1.821510in}}%
\pgfpathlineto{\pgfqpoint{2.151944in}{1.035332in}}%
\pgfpathlineto{\pgfqpoint{2.152819in}{0.968425in}}%
\pgfpathlineto{\pgfqpoint{2.153695in}{0.958487in}}%
\pgfpathlineto{\pgfqpoint{2.154570in}{0.961387in}}%
\pgfpathlineto{\pgfqpoint{2.155445in}{1.075204in}}%
\pgfpathlineto{\pgfqpoint{2.157196in}{1.487405in}}%
\pgfpathlineto{\pgfqpoint{2.158072in}{1.840056in}}%
\pgfpathlineto{\pgfqpoint{2.158947in}{1.776099in}}%
\pgfpathlineto{\pgfqpoint{2.159822in}{2.140471in}}%
\pgfpathlineto{\pgfqpoint{2.160698in}{2.172551in}}%
\pgfpathlineto{\pgfqpoint{2.161573in}{2.579465in}}%
\pgfpathlineto{\pgfqpoint{2.162448in}{2.723021in}}%
\pgfpathlineto{\pgfqpoint{2.163324in}{2.802434in}}%
\pgfpathlineto{\pgfqpoint{2.164199in}{2.792702in}}%
\pgfpathlineto{\pgfqpoint{2.165075in}{2.569755in}}%
\pgfpathlineto{\pgfqpoint{2.165950in}{2.271463in}}%
\pgfpathlineto{\pgfqpoint{2.167701in}{2.139522in}}%
\pgfpathlineto{\pgfqpoint{2.168576in}{2.058448in}}%
\pgfpathlineto{\pgfqpoint{2.169451in}{1.928410in}}%
\pgfpathlineto{\pgfqpoint{2.171202in}{1.543613in}}%
\pgfpathlineto{\pgfqpoint{2.172078in}{1.490027in}}%
\pgfpathlineto{\pgfqpoint{2.172953in}{1.373280in}}%
\pgfpathlineto{\pgfqpoint{2.173828in}{1.359264in}}%
\pgfpathlineto{\pgfqpoint{2.174704in}{1.275200in}}%
\pgfpathlineto{\pgfqpoint{2.175579in}{1.340627in}}%
\pgfpathlineto{\pgfqpoint{2.176454in}{1.468399in}}%
\pgfpathlineto{\pgfqpoint{2.178205in}{1.898606in}}%
\pgfpathlineto{\pgfqpoint{2.179081in}{1.973548in}}%
\pgfpathlineto{\pgfqpoint{2.179956in}{2.083500in}}%
\pgfpathlineto{\pgfqpoint{2.180831in}{2.090000in}}%
\pgfpathlineto{\pgfqpoint{2.181707in}{2.128631in}}%
\pgfpathlineto{\pgfqpoint{2.182582in}{2.338520in}}%
\pgfpathlineto{\pgfqpoint{2.183457in}{2.344386in}}%
\pgfpathlineto{\pgfqpoint{2.184333in}{2.384661in}}%
\pgfpathlineto{\pgfqpoint{2.185208in}{2.448108in}}%
\pgfpathlineto{\pgfqpoint{2.186959in}{2.264908in}}%
\pgfpathlineto{\pgfqpoint{2.187834in}{2.124419in}}%
\pgfpathlineto{\pgfqpoint{2.188710in}{2.058781in}}%
\pgfpathlineto{\pgfqpoint{2.189585in}{2.050172in}}%
\pgfpathlineto{\pgfqpoint{2.190460in}{1.922792in}}%
\pgfpathlineto{\pgfqpoint{2.193087in}{1.404392in}}%
\pgfpathlineto{\pgfqpoint{2.193962in}{1.249434in}}%
\pgfpathlineto{\pgfqpoint{2.195713in}{1.124259in}}%
\pgfpathlineto{\pgfqpoint{2.196588in}{1.175459in}}%
\pgfpathlineto{\pgfqpoint{2.198339in}{1.451278in}}%
\pgfpathlineto{\pgfqpoint{2.200090in}{1.907653in}}%
\pgfpathlineto{\pgfqpoint{2.200965in}{1.905986in}}%
\pgfpathlineto{\pgfqpoint{2.204467in}{2.302146in}}%
\pgfpathlineto{\pgfqpoint{2.205342in}{2.295374in}}%
\pgfpathlineto{\pgfqpoint{2.208843in}{1.998882in}}%
\pgfpathlineto{\pgfqpoint{2.209719in}{1.877150in}}%
\pgfpathlineto{\pgfqpoint{2.210594in}{1.706787in}}%
\pgfpathlineto{\pgfqpoint{2.211470in}{1.602485in}}%
\pgfpathlineto{\pgfqpoint{2.212345in}{1.395481in}}%
\pgfpathlineto{\pgfqpoint{2.214096in}{1.145917in}}%
\pgfpathlineto{\pgfqpoint{2.214971in}{1.096711in}}%
\pgfpathlineto{\pgfqpoint{2.215846in}{0.987727in}}%
\pgfpathlineto{\pgfqpoint{2.216722in}{1.014399in}}%
\pgfpathlineto{\pgfqpoint{2.217597in}{1.151264in}}%
\pgfpathlineto{\pgfqpoint{2.221099in}{1.962894in}}%
\pgfpathlineto{\pgfqpoint{2.221974in}{2.030223in}}%
\pgfpathlineto{\pgfqpoint{2.222849in}{2.042338in}}%
\pgfpathlineto{\pgfqpoint{2.223725in}{2.177820in}}%
\pgfpathlineto{\pgfqpoint{2.224600in}{2.236033in}}%
\pgfpathlineto{\pgfqpoint{2.226351in}{2.509705in}}%
\pgfpathlineto{\pgfqpoint{2.228977in}{2.110101in}}%
\pgfpathlineto{\pgfqpoint{2.229852in}{2.000483in}}%
\pgfpathlineto{\pgfqpoint{2.232479in}{1.564153in}}%
\pgfpathlineto{\pgfqpoint{2.234229in}{1.214908in}}%
\pgfpathlineto{\pgfqpoint{2.235105in}{1.104565in}}%
\pgfpathlineto{\pgfqpoint{2.235980in}{1.052278in}}%
\pgfpathlineto{\pgfqpoint{2.236855in}{1.033368in}}%
\pgfpathlineto{\pgfqpoint{2.237731in}{1.059527in}}%
\pgfpathlineto{\pgfqpoint{2.238606in}{1.193613in}}%
\pgfpathlineto{\pgfqpoint{2.240357in}{1.650977in}}%
\pgfpathlineto{\pgfqpoint{2.241232in}{1.966023in}}%
\pgfpathlineto{\pgfqpoint{2.242108in}{2.116189in}}%
\pgfpathlineto{\pgfqpoint{2.242983in}{2.171401in}}%
\pgfpathlineto{\pgfqpoint{2.243858in}{2.402027in}}%
\pgfpathlineto{\pgfqpoint{2.244734in}{2.488136in}}%
\pgfpathlineto{\pgfqpoint{2.245609in}{2.430658in}}%
\pgfpathlineto{\pgfqpoint{2.246485in}{2.675963in}}%
\pgfpathlineto{\pgfqpoint{2.247360in}{2.637068in}}%
\pgfpathlineto{\pgfqpoint{2.249111in}{2.347483in}}%
\pgfpathlineto{\pgfqpoint{2.249986in}{2.307439in}}%
\pgfpathlineto{\pgfqpoint{2.250861in}{2.360632in}}%
\pgfpathlineto{\pgfqpoint{2.251737in}{2.247509in}}%
\pgfpathlineto{\pgfqpoint{2.252612in}{2.207637in}}%
\pgfpathlineto{\pgfqpoint{2.254363in}{1.819577in}}%
\pgfpathlineto{\pgfqpoint{2.255238in}{1.720682in}}%
\pgfpathlineto{\pgfqpoint{2.256114in}{1.586929in}}%
\pgfpathlineto{\pgfqpoint{2.257864in}{1.449248in}}%
\pgfpathlineto{\pgfqpoint{2.259615in}{1.479062in}}%
\pgfpathlineto{\pgfqpoint{2.261366in}{1.787943in}}%
\pgfpathlineto{\pgfqpoint{2.262241in}{2.050078in}}%
\pgfpathlineto{\pgfqpoint{2.263992in}{2.349240in}}%
\pgfpathlineto{\pgfqpoint{2.266618in}{2.734063in}}%
\pgfpathlineto{\pgfqpoint{2.267494in}{2.766014in}}%
\pgfpathlineto{\pgfqpoint{2.268369in}{2.606641in}}%
\pgfpathlineto{\pgfqpoint{2.269244in}{2.523699in}}%
\pgfpathlineto{\pgfqpoint{2.270120in}{2.367152in}}%
\pgfpathlineto{\pgfqpoint{2.273621in}{1.985410in}}%
\pgfpathlineto{\pgfqpoint{2.276247in}{1.450124in}}%
\pgfpathlineto{\pgfqpoint{2.278873in}{1.216902in}}%
\pgfpathlineto{\pgfqpoint{2.279749in}{1.200923in}}%
\pgfpathlineto{\pgfqpoint{2.280624in}{1.356606in}}%
\pgfpathlineto{\pgfqpoint{2.283250in}{2.100342in}}%
\pgfpathlineto{\pgfqpoint{2.285001in}{2.395376in}}%
\pgfpathlineto{\pgfqpoint{2.285876in}{2.376620in}}%
\pgfpathlineto{\pgfqpoint{2.286752in}{2.406511in}}%
\pgfpathlineto{\pgfqpoint{2.287627in}{2.498897in}}%
\pgfpathlineto{\pgfqpoint{2.288503in}{2.485531in}}%
\pgfpathlineto{\pgfqpoint{2.289378in}{2.493088in}}%
\pgfpathlineto{\pgfqpoint{2.290253in}{2.392609in}}%
\pgfpathlineto{\pgfqpoint{2.291129in}{2.344132in}}%
\pgfpathlineto{\pgfqpoint{2.292004in}{2.251164in}}%
\pgfpathlineto{\pgfqpoint{2.292879in}{2.271977in}}%
\pgfpathlineto{\pgfqpoint{2.293755in}{2.145382in}}%
\pgfpathlineto{\pgfqpoint{2.294630in}{2.096780in}}%
\pgfpathlineto{\pgfqpoint{2.296381in}{1.806921in}}%
\pgfpathlineto{\pgfqpoint{2.297256in}{1.695278in}}%
\pgfpathlineto{\pgfqpoint{2.298132in}{1.651993in}}%
\pgfpathlineto{\pgfqpoint{2.299007in}{1.559743in}}%
\pgfpathlineto{\pgfqpoint{2.299882in}{1.499089in}}%
\pgfpathlineto{\pgfqpoint{2.300758in}{1.494920in}}%
\pgfpathlineto{\pgfqpoint{2.301633in}{1.591610in}}%
\pgfpathlineto{\pgfqpoint{2.302509in}{1.794808in}}%
\pgfpathlineto{\pgfqpoint{2.303384in}{1.916513in}}%
\pgfpathlineto{\pgfqpoint{2.305135in}{2.197004in}}%
\pgfpathlineto{\pgfqpoint{2.306010in}{2.260791in}}%
\pgfpathlineto{\pgfqpoint{2.307761in}{2.516582in}}%
\pgfpathlineto{\pgfqpoint{2.308636in}{2.499713in}}%
\pgfpathlineto{\pgfqpoint{2.309512in}{2.416415in}}%
\pgfpathlineto{\pgfqpoint{2.310387in}{2.505974in}}%
\pgfpathlineto{\pgfqpoint{2.311262in}{2.492688in}}%
\pgfpathlineto{\pgfqpoint{2.312138in}{2.381601in}}%
\pgfpathlineto{\pgfqpoint{2.313013in}{2.315292in}}%
\pgfpathlineto{\pgfqpoint{2.313888in}{2.212953in}}%
\pgfpathlineto{\pgfqpoint{2.315639in}{1.976771in}}%
\pgfpathlineto{\pgfqpoint{2.316515in}{1.950068in}}%
\pgfpathlineto{\pgfqpoint{2.318265in}{1.692107in}}%
\pgfpathlineto{\pgfqpoint{2.320891in}{1.482596in}}%
\pgfpathlineto{\pgfqpoint{2.321767in}{1.486100in}}%
\pgfpathlineto{\pgfqpoint{2.322642in}{1.573034in}}%
\pgfpathlineto{\pgfqpoint{2.323518in}{1.700504in}}%
\pgfpathlineto{\pgfqpoint{2.325268in}{2.057689in}}%
\pgfpathlineto{\pgfqpoint{2.326144in}{2.091480in}}%
\pgfpathlineto{\pgfqpoint{2.327019in}{2.212131in}}%
\pgfpathlineto{\pgfqpoint{2.327895in}{2.282691in}}%
\pgfpathlineto{\pgfqpoint{2.328770in}{2.199021in}}%
\pgfpathlineto{\pgfqpoint{2.329645in}{2.269803in}}%
\pgfpathlineto{\pgfqpoint{2.330521in}{2.162578in}}%
\pgfpathlineto{\pgfqpoint{2.331396in}{2.228200in}}%
\pgfpathlineto{\pgfqpoint{2.332271in}{2.117809in}}%
\pgfpathlineto{\pgfqpoint{2.333147in}{1.951628in}}%
\pgfpathlineto{\pgfqpoint{2.334022in}{1.913964in}}%
\pgfpathlineto{\pgfqpoint{2.334898in}{1.948437in}}%
\pgfpathlineto{\pgfqpoint{2.337524in}{1.726572in}}%
\pgfpathlineto{\pgfqpoint{2.339274in}{1.534762in}}%
\pgfpathlineto{\pgfqpoint{2.340150in}{1.510688in}}%
\pgfpathlineto{\pgfqpoint{2.341025in}{1.457706in}}%
\pgfpathlineto{\pgfqpoint{2.341901in}{1.471752in}}%
\pgfpathlineto{\pgfqpoint{2.342776in}{1.479243in}}%
\pgfpathlineto{\pgfqpoint{2.343651in}{1.521623in}}%
\pgfpathlineto{\pgfqpoint{2.345402in}{1.839941in}}%
\pgfpathlineto{\pgfqpoint{2.346277in}{1.892938in}}%
\pgfpathlineto{\pgfqpoint{2.347153in}{1.874234in}}%
\pgfpathlineto{\pgfqpoint{2.348028in}{1.834033in}}%
\pgfpathlineto{\pgfqpoint{2.348904in}{1.934780in}}%
\pgfpathlineto{\pgfqpoint{2.349779in}{1.844902in}}%
\pgfpathlineto{\pgfqpoint{2.350654in}{2.072609in}}%
\pgfpathlineto{\pgfqpoint{2.351530in}{2.077266in}}%
\pgfpathlineto{\pgfqpoint{2.352405in}{2.071900in}}%
\pgfpathlineto{\pgfqpoint{2.354156in}{1.764830in}}%
\pgfpathlineto{\pgfqpoint{2.355031in}{1.664740in}}%
\pgfpathlineto{\pgfqpoint{2.355907in}{1.669815in}}%
\pgfpathlineto{\pgfqpoint{2.357657in}{1.428557in}}%
\pgfpathlineto{\pgfqpoint{2.359408in}{1.150478in}}%
\pgfpathlineto{\pgfqpoint{2.361159in}{0.988270in}}%
\pgfpathlineto{\pgfqpoint{2.362034in}{0.957581in}}%
\pgfpathlineto{\pgfqpoint{2.362910in}{0.907076in}}%
\pgfpathlineto{\pgfqpoint{2.363785in}{0.928100in}}%
\pgfpathlineto{\pgfqpoint{2.364660in}{1.061944in}}%
\pgfpathlineto{\pgfqpoint{2.365536in}{1.263661in}}%
\pgfpathlineto{\pgfqpoint{2.367286in}{1.526714in}}%
\pgfpathlineto{\pgfqpoint{2.368162in}{1.606407in}}%
\pgfpathlineto{\pgfqpoint{2.369037in}{1.643667in}}%
\pgfpathlineto{\pgfqpoint{2.369913in}{1.575523in}}%
\pgfpathlineto{\pgfqpoint{2.371663in}{1.906369in}}%
\pgfpathlineto{\pgfqpoint{2.372539in}{1.899096in}}%
\pgfpathlineto{\pgfqpoint{2.373414in}{1.938371in}}%
\pgfpathlineto{\pgfqpoint{2.374289in}{1.826033in}}%
\pgfpathlineto{\pgfqpoint{2.375165in}{1.654473in}}%
\pgfpathlineto{\pgfqpoint{2.376916in}{1.449188in}}%
\pgfpathlineto{\pgfqpoint{2.378666in}{1.333951in}}%
\pgfpathlineto{\pgfqpoint{2.380417in}{1.097889in}}%
\pgfpathlineto{\pgfqpoint{2.381292in}{1.022373in}}%
\pgfpathlineto{\pgfqpoint{2.383043in}{0.936285in}}%
\pgfpathlineto{\pgfqpoint{2.383919in}{0.938219in}}%
\pgfpathlineto{\pgfqpoint{2.384794in}{0.955618in}}%
\pgfpathlineto{\pgfqpoint{2.385669in}{0.995309in}}%
\pgfpathlineto{\pgfqpoint{2.387420in}{1.416886in}}%
\pgfpathlineto{\pgfqpoint{2.388295in}{1.544533in}}%
\pgfpathlineto{\pgfqpoint{2.389171in}{2.029792in}}%
\pgfpathlineto{\pgfqpoint{2.390046in}{1.846085in}}%
\pgfpathlineto{\pgfqpoint{2.390922in}{1.880128in}}%
\pgfpathlineto{\pgfqpoint{2.391797in}{1.843209in}}%
\pgfpathlineto{\pgfqpoint{2.392672in}{1.824465in}}%
\pgfpathlineto{\pgfqpoint{2.393548in}{1.927086in}}%
\pgfpathlineto{\pgfqpoint{2.394423in}{1.914012in}}%
\pgfpathlineto{\pgfqpoint{2.395298in}{1.813456in}}%
\pgfpathlineto{\pgfqpoint{2.396174in}{1.655679in}}%
\pgfpathlineto{\pgfqpoint{2.397925in}{1.512833in}}%
\pgfpathlineto{\pgfqpoint{2.399675in}{1.291481in}}%
\pgfpathlineto{\pgfqpoint{2.400551in}{1.298549in}}%
\pgfpathlineto{\pgfqpoint{2.404052in}{1.024669in}}%
\pgfpathlineto{\pgfqpoint{2.404928in}{1.026149in}}%
\pgfpathlineto{\pgfqpoint{2.405803in}{0.975282in}}%
\pgfpathlineto{\pgfqpoint{2.407554in}{1.232730in}}%
\pgfpathlineto{\pgfqpoint{2.410180in}{1.589676in}}%
\pgfpathlineto{\pgfqpoint{2.411055in}{1.475900in}}%
\pgfpathlineto{\pgfqpoint{2.411931in}{1.478735in}}%
\pgfpathlineto{\pgfqpoint{2.412806in}{1.470893in}}%
\pgfpathlineto{\pgfqpoint{2.413681in}{1.509750in}}%
\pgfpathlineto{\pgfqpoint{2.414557in}{1.583348in}}%
\pgfpathlineto{\pgfqpoint{2.415432in}{1.583214in}}%
\pgfpathlineto{\pgfqpoint{2.417183in}{1.337888in}}%
\pgfpathlineto{\pgfqpoint{2.418058in}{1.330646in}}%
\pgfpathlineto{\pgfqpoint{2.418934in}{1.310632in}}%
\pgfpathlineto{\pgfqpoint{2.419809in}{1.271726in}}%
\pgfpathlineto{\pgfqpoint{2.420684in}{1.202524in}}%
\pgfpathlineto{\pgfqpoint{2.422435in}{1.027992in}}%
\pgfpathlineto{\pgfqpoint{2.424186in}{0.917920in}}%
\pgfpathlineto{\pgfqpoint{2.425061in}{0.912876in}}%
\pgfpathlineto{\pgfqpoint{2.425937in}{0.956282in}}%
\pgfpathlineto{\pgfqpoint{2.426812in}{0.926046in}}%
\pgfpathlineto{\pgfqpoint{2.427687in}{0.975916in}}%
\pgfpathlineto{\pgfqpoint{2.428563in}{1.244994in}}%
\pgfpathlineto{\pgfqpoint{2.430313in}{1.471441in}}%
\pgfpathlineto{\pgfqpoint{2.431189in}{1.429949in}}%
\pgfpathlineto{\pgfqpoint{2.432064in}{1.318951in}}%
\pgfpathlineto{\pgfqpoint{2.432940in}{1.391523in}}%
\pgfpathlineto{\pgfqpoint{2.433815in}{1.387406in}}%
\pgfpathlineto{\pgfqpoint{2.435566in}{1.605214in}}%
\pgfpathlineto{\pgfqpoint{2.436441in}{1.550503in}}%
\pgfpathlineto{\pgfqpoint{2.437316in}{1.458409in}}%
\pgfpathlineto{\pgfqpoint{2.438192in}{1.506682in}}%
\pgfpathlineto{\pgfqpoint{2.439067in}{1.162084in}}%
\pgfpathlineto{\pgfqpoint{2.439943in}{1.191215in}}%
\pgfpathlineto{\pgfqpoint{2.441693in}{0.973254in}}%
\pgfpathlineto{\pgfqpoint{2.445195in}{0.722859in}}%
\pgfpathlineto{\pgfqpoint{2.446070in}{0.685876in}}%
\pgfpathlineto{\pgfqpoint{2.446946in}{0.698732in}}%
\pgfpathlineto{\pgfqpoint{2.447821in}{0.724517in}}%
\pgfpathlineto{\pgfqpoint{2.448696in}{0.772020in}}%
\pgfpathlineto{\pgfqpoint{2.451322in}{1.207406in}}%
\pgfpathlineto{\pgfqpoint{2.452198in}{1.198464in}}%
\pgfpathlineto{\pgfqpoint{2.453073in}{1.313933in}}%
\pgfpathlineto{\pgfqpoint{2.454824in}{1.378655in}}%
\pgfpathlineto{\pgfqpoint{2.455699in}{1.901067in}}%
\pgfpathlineto{\pgfqpoint{2.456575in}{2.169344in}}%
\pgfpathlineto{\pgfqpoint{2.458325in}{2.024961in}}%
\pgfpathlineto{\pgfqpoint{2.459201in}{1.852658in}}%
\pgfpathlineto{\pgfqpoint{2.460076in}{1.764783in}}%
\pgfpathlineto{\pgfqpoint{2.460952in}{1.715728in}}%
\pgfpathlineto{\pgfqpoint{2.464453in}{1.264990in}}%
\pgfpathlineto{\pgfqpoint{2.465329in}{1.233727in}}%
\pgfpathlineto{\pgfqpoint{2.467079in}{1.137248in}}%
\pgfpathlineto{\pgfqpoint{2.467955in}{1.123504in}}%
\pgfpathlineto{\pgfqpoint{2.468830in}{1.090942in}}%
\pgfpathlineto{\pgfqpoint{2.469705in}{1.180715in}}%
\pgfpathlineto{\pgfqpoint{2.470581in}{1.405449in}}%
\pgfpathlineto{\pgfqpoint{2.472332in}{1.588754in}}%
\pgfpathlineto{\pgfqpoint{2.473207in}{1.729813in}}%
\pgfpathlineto{\pgfqpoint{2.474082in}{1.918579in}}%
\pgfpathlineto{\pgfqpoint{2.474958in}{2.013498in}}%
\pgfpathlineto{\pgfqpoint{2.475833in}{2.038460in}}%
\pgfpathlineto{\pgfqpoint{2.476708in}{2.140939in}}%
\pgfpathlineto{\pgfqpoint{2.477584in}{2.106414in}}%
\pgfpathlineto{\pgfqpoint{2.478459in}{2.154600in}}%
\pgfpathlineto{\pgfqpoint{2.479335in}{2.064384in}}%
\pgfpathlineto{\pgfqpoint{2.480210in}{1.844742in}}%
\pgfpathlineto{\pgfqpoint{2.481085in}{1.743669in}}%
\pgfpathlineto{\pgfqpoint{2.481961in}{1.834439in}}%
\pgfpathlineto{\pgfqpoint{2.482836in}{1.766112in}}%
\pgfpathlineto{\pgfqpoint{2.483711in}{1.803024in}}%
\pgfpathlineto{\pgfqpoint{2.484587in}{1.658094in}}%
\pgfpathlineto{\pgfqpoint{2.485462in}{1.383187in}}%
\pgfpathlineto{\pgfqpoint{2.486338in}{1.412215in}}%
\pgfpathlineto{\pgfqpoint{2.487213in}{1.315646in}}%
\pgfpathlineto{\pgfqpoint{2.488088in}{1.273690in}}%
\pgfpathlineto{\pgfqpoint{2.488964in}{1.346487in}}%
\pgfpathlineto{\pgfqpoint{2.489839in}{1.340808in}}%
\pgfpathlineto{\pgfqpoint{2.490714in}{1.343798in}}%
\pgfpathlineto{\pgfqpoint{2.493341in}{1.856340in}}%
\pgfpathlineto{\pgfqpoint{2.494216in}{2.012952in}}%
\pgfpathlineto{\pgfqpoint{2.495091in}{2.030323in}}%
\pgfpathlineto{\pgfqpoint{2.495967in}{2.174585in}}%
\pgfpathlineto{\pgfqpoint{2.496842in}{1.989090in}}%
\pgfpathlineto{\pgfqpoint{2.497717in}{1.869875in}}%
\pgfpathlineto{\pgfqpoint{2.498593in}{1.890608in}}%
\pgfpathlineto{\pgfqpoint{2.500344in}{1.689949in}}%
\pgfpathlineto{\pgfqpoint{2.501219in}{1.634185in}}%
\pgfpathlineto{\pgfqpoint{2.502094in}{1.540713in}}%
\pgfpathlineto{\pgfqpoint{2.502970in}{1.595930in}}%
\pgfpathlineto{\pgfqpoint{2.503845in}{1.492474in}}%
\pgfpathlineto{\pgfqpoint{2.504720in}{1.440217in}}%
\pgfpathlineto{\pgfqpoint{2.505596in}{1.262543in}}%
\pgfpathlineto{\pgfqpoint{2.506471in}{1.179869in}}%
\pgfpathlineto{\pgfqpoint{2.507347in}{1.126102in}}%
\pgfpathlineto{\pgfqpoint{2.508222in}{1.100940in}}%
\pgfpathlineto{\pgfqpoint{2.509097in}{1.052670in}}%
\pgfpathlineto{\pgfqpoint{2.509973in}{1.091667in}}%
\pgfpathlineto{\pgfqpoint{2.510848in}{1.012164in}}%
\pgfpathlineto{\pgfqpoint{2.511723in}{1.075204in}}%
\pgfpathlineto{\pgfqpoint{2.513474in}{1.420734in}}%
\pgfpathlineto{\pgfqpoint{2.514350in}{1.680581in}}%
\pgfpathlineto{\pgfqpoint{2.515225in}{1.727905in}}%
\pgfpathlineto{\pgfqpoint{2.516100in}{1.756578in}}%
\pgfpathlineto{\pgfqpoint{2.516976in}{1.914244in}}%
\pgfpathlineto{\pgfqpoint{2.517851in}{1.925364in}}%
\pgfpathlineto{\pgfqpoint{2.519602in}{2.005255in}}%
\pgfpathlineto{\pgfqpoint{2.520477in}{1.983425in}}%
\pgfpathlineto{\pgfqpoint{2.521353in}{1.936455in}}%
\pgfpathlineto{\pgfqpoint{2.522228in}{1.763016in}}%
\pgfpathlineto{\pgfqpoint{2.523103in}{1.738017in}}%
\pgfpathlineto{\pgfqpoint{2.523979in}{1.744545in}}%
\pgfpathlineto{\pgfqpoint{2.524854in}{1.743155in}}%
\pgfpathlineto{\pgfqpoint{2.525729in}{1.670237in}}%
\pgfpathlineto{\pgfqpoint{2.528356in}{1.304288in}}%
\pgfpathlineto{\pgfqpoint{2.529231in}{1.164917in}}%
\pgfpathlineto{\pgfqpoint{2.530106in}{1.155492in}}%
\pgfpathlineto{\pgfqpoint{2.530982in}{1.083420in}}%
\pgfpathlineto{\pgfqpoint{2.531857in}{0.973318in}}%
\pgfpathlineto{\pgfqpoint{2.532732in}{1.011650in}}%
\pgfpathlineto{\pgfqpoint{2.535359in}{1.861994in}}%
\pgfpathlineto{\pgfqpoint{2.536234in}{1.991523in}}%
\pgfpathlineto{\pgfqpoint{2.537109in}{2.044366in}}%
\pgfpathlineto{\pgfqpoint{2.539735in}{2.426885in}}%
\pgfpathlineto{\pgfqpoint{2.540611in}{2.401341in}}%
\pgfpathlineto{\pgfqpoint{2.541486in}{2.479444in}}%
\pgfpathlineto{\pgfqpoint{2.542362in}{2.344013in}}%
\pgfpathlineto{\pgfqpoint{2.543237in}{2.145883in}}%
\pgfpathlineto{\pgfqpoint{2.544988in}{2.044946in}}%
\pgfpathlineto{\pgfqpoint{2.545863in}{1.803115in}}%
\pgfpathlineto{\pgfqpoint{2.546738in}{1.671446in}}%
\pgfpathlineto{\pgfqpoint{2.548489in}{1.291270in}}%
\pgfpathlineto{\pgfqpoint{2.550240in}{1.045662in}}%
\pgfpathlineto{\pgfqpoint{2.551115in}{0.970570in}}%
\pgfpathlineto{\pgfqpoint{2.551991in}{0.670138in}}%
\pgfpathlineto{\pgfqpoint{2.552866in}{0.690467in}}%
\pgfpathlineto{\pgfqpoint{2.553741in}{0.760153in}}%
\pgfpathlineto{\pgfqpoint{2.554617in}{0.909801in}}%
\pgfpathlineto{\pgfqpoint{2.556368in}{1.476311in}}%
\pgfpathlineto{\pgfqpoint{2.557243in}{1.574462in}}%
\pgfpathlineto{\pgfqpoint{2.558118in}{1.594669in}}%
\pgfpathlineto{\pgfqpoint{2.558994in}{1.679750in}}%
\pgfpathlineto{\pgfqpoint{2.559869in}{1.626369in}}%
\pgfpathlineto{\pgfqpoint{2.561620in}{2.023476in}}%
\pgfpathlineto{\pgfqpoint{2.562495in}{1.981630in}}%
\pgfpathlineto{\pgfqpoint{2.564246in}{1.602877in}}%
\pgfpathlineto{\pgfqpoint{2.565121in}{1.532859in}}%
\pgfpathlineto{\pgfqpoint{2.565997in}{1.669059in}}%
\pgfpathlineto{\pgfqpoint{2.566872in}{1.683709in}}%
\pgfpathlineto{\pgfqpoint{2.567747in}{1.673530in}}%
\pgfpathlineto{\pgfqpoint{2.568623in}{1.523284in}}%
\pgfpathlineto{\pgfqpoint{2.569498in}{1.323711in}}%
\pgfpathlineto{\pgfqpoint{2.570374in}{1.258466in}}%
\pgfpathlineto{\pgfqpoint{2.571249in}{1.156640in}}%
\pgfpathlineto{\pgfqpoint{2.572124in}{1.203671in}}%
\pgfpathlineto{\pgfqpoint{2.573000in}{1.180231in}}%
\pgfpathlineto{\pgfqpoint{2.573875in}{1.132687in}}%
\pgfpathlineto{\pgfqpoint{2.574750in}{1.140812in}}%
\pgfpathlineto{\pgfqpoint{2.577377in}{2.096138in}}%
\pgfpathlineto{\pgfqpoint{2.578252in}{2.216741in}}%
\pgfpathlineto{\pgfqpoint{2.579127in}{2.091930in}}%
\pgfpathlineto{\pgfqpoint{2.580003in}{2.332278in}}%
\pgfpathlineto{\pgfqpoint{2.580878in}{2.345719in}}%
\pgfpathlineto{\pgfqpoint{2.581753in}{2.314280in}}%
\pgfpathlineto{\pgfqpoint{2.582629in}{2.350083in}}%
\pgfpathlineto{\pgfqpoint{2.583504in}{2.198111in}}%
\pgfpathlineto{\pgfqpoint{2.584380in}{2.129331in}}%
\pgfpathlineto{\pgfqpoint{2.586130in}{1.877396in}}%
\pgfpathlineto{\pgfqpoint{2.587006in}{1.906964in}}%
\pgfpathlineto{\pgfqpoint{2.590507in}{1.358841in}}%
\pgfpathlineto{\pgfqpoint{2.591383in}{1.308397in}}%
\pgfpathlineto{\pgfqpoint{2.594009in}{1.089884in}}%
\pgfpathlineto{\pgfqpoint{2.594884in}{0.995822in}}%
\pgfpathlineto{\pgfqpoint{2.595760in}{1.116103in}}%
\pgfpathlineto{\pgfqpoint{2.598386in}{1.712819in}}%
\pgfpathlineto{\pgfqpoint{2.599261in}{1.716687in}}%
\pgfpathlineto{\pgfqpoint{2.600136in}{2.196806in}}%
\pgfpathlineto{\pgfqpoint{2.601012in}{2.432141in}}%
\pgfpathlineto{\pgfqpoint{2.601887in}{2.549027in}}%
\pgfpathlineto{\pgfqpoint{2.602763in}{2.541756in}}%
\pgfpathlineto{\pgfqpoint{2.603638in}{2.730849in}}%
\pgfpathlineto{\pgfqpoint{2.604513in}{2.778299in}}%
\pgfpathlineto{\pgfqpoint{2.605389in}{2.612976in}}%
\pgfpathlineto{\pgfqpoint{2.606264in}{2.281945in}}%
\pgfpathlineto{\pgfqpoint{2.607139in}{2.100909in}}%
\pgfpathlineto{\pgfqpoint{2.608015in}{2.118559in}}%
\pgfpathlineto{\pgfqpoint{2.610641in}{1.513135in}}%
\pgfpathlineto{\pgfqpoint{2.613267in}{1.132536in}}%
\pgfpathlineto{\pgfqpoint{2.614142in}{1.095624in}}%
\pgfpathlineto{\pgfqpoint{2.615018in}{1.037416in}}%
\pgfpathlineto{\pgfqpoint{2.615893in}{0.919798in}}%
\pgfpathlineto{\pgfqpoint{2.616769in}{1.000967in}}%
\pgfpathlineto{\pgfqpoint{2.617644in}{1.169855in}}%
\pgfpathlineto{\pgfqpoint{2.619395in}{1.579274in}}%
\pgfpathlineto{\pgfqpoint{2.620270in}{1.723106in}}%
\pgfpathlineto{\pgfqpoint{2.621145in}{1.745337in}}%
\pgfpathlineto{\pgfqpoint{2.622021in}{2.022302in}}%
\pgfpathlineto{\pgfqpoint{2.622896in}{1.996270in}}%
\pgfpathlineto{\pgfqpoint{2.623772in}{2.156922in}}%
\pgfpathlineto{\pgfqpoint{2.624647in}{2.098544in}}%
\pgfpathlineto{\pgfqpoint{2.625522in}{2.136420in}}%
\pgfpathlineto{\pgfqpoint{2.627273in}{1.883890in}}%
\pgfpathlineto{\pgfqpoint{2.628148in}{1.997643in}}%
\pgfpathlineto{\pgfqpoint{2.629024in}{1.975019in}}%
\pgfpathlineto{\pgfqpoint{2.629899in}{1.939979in}}%
\pgfpathlineto{\pgfqpoint{2.630775in}{1.799188in}}%
\pgfpathlineto{\pgfqpoint{2.631650in}{1.578229in}}%
\pgfpathlineto{\pgfqpoint{2.632525in}{1.475709in}}%
\pgfpathlineto{\pgfqpoint{2.633401in}{1.275834in}}%
\pgfpathlineto{\pgfqpoint{2.635151in}{1.137157in}}%
\pgfpathlineto{\pgfqpoint{2.636027in}{1.125679in}}%
\pgfpathlineto{\pgfqpoint{2.636902in}{1.109005in}}%
\pgfpathlineto{\pgfqpoint{2.637778in}{1.196724in}}%
\pgfpathlineto{\pgfqpoint{2.638653in}{1.427893in}}%
\pgfpathlineto{\pgfqpoint{2.639528in}{1.522511in}}%
\pgfpathlineto{\pgfqpoint{2.640404in}{1.726354in}}%
\pgfpathlineto{\pgfqpoint{2.641279in}{1.814441in}}%
\pgfpathlineto{\pgfqpoint{2.642154in}{1.968899in}}%
\pgfpathlineto{\pgfqpoint{2.643030in}{2.173801in}}%
\pgfpathlineto{\pgfqpoint{2.643905in}{2.241024in}}%
\pgfpathlineto{\pgfqpoint{2.644781in}{2.154750in}}%
\pgfpathlineto{\pgfqpoint{2.645656in}{2.173570in}}%
\pgfpathlineto{\pgfqpoint{2.646531in}{2.471496in}}%
\pgfpathlineto{\pgfqpoint{2.647407in}{2.357028in}}%
\pgfpathlineto{\pgfqpoint{2.649157in}{2.050716in}}%
\pgfpathlineto{\pgfqpoint{2.650033in}{2.100012in}}%
\pgfpathlineto{\pgfqpoint{2.650908in}{1.953089in}}%
\pgfpathlineto{\pgfqpoint{2.652659in}{1.595235in}}%
\pgfpathlineto{\pgfqpoint{2.656160in}{1.062699in}}%
\pgfpathlineto{\pgfqpoint{2.657036in}{1.049166in}}%
\pgfpathlineto{\pgfqpoint{2.657911in}{1.009083in}}%
\pgfpathlineto{\pgfqpoint{2.659662in}{1.281301in}}%
\pgfpathlineto{\pgfqpoint{2.661413in}{1.870324in}}%
\pgfpathlineto{\pgfqpoint{2.662288in}{2.065171in}}%
\pgfpathlineto{\pgfqpoint{2.663163in}{2.161635in}}%
\pgfpathlineto{\pgfqpoint{2.664039in}{2.084639in}}%
\pgfpathlineto{\pgfqpoint{2.664914in}{2.201680in}}%
\pgfpathlineto{\pgfqpoint{2.666665in}{2.538145in}}%
\pgfpathlineto{\pgfqpoint{2.667540in}{2.447677in}}%
\pgfpathlineto{\pgfqpoint{2.668416in}{2.432382in}}%
\pgfpathlineto{\pgfqpoint{2.671042in}{2.013743in}}%
\pgfpathlineto{\pgfqpoint{2.671917in}{1.950310in}}%
\pgfpathlineto{\pgfqpoint{2.672793in}{1.791032in}}%
\pgfpathlineto{\pgfqpoint{2.674543in}{1.306509in}}%
\pgfpathlineto{\pgfqpoint{2.676294in}{1.124308in}}%
\pgfpathlineto{\pgfqpoint{2.677169in}{1.055698in}}%
\pgfpathlineto{\pgfqpoint{2.678045in}{1.045144in}}%
\pgfpathlineto{\pgfqpoint{2.678920in}{1.073164in}}%
\pgfpathlineto{\pgfqpoint{2.679796in}{1.156099in}}%
\pgfpathlineto{\pgfqpoint{2.681546in}{1.681060in}}%
\pgfpathlineto{\pgfqpoint{2.682422in}{1.856590in}}%
\pgfpathlineto{\pgfqpoint{2.683297in}{1.944402in}}%
\pgfpathlineto{\pgfqpoint{2.685048in}{2.169550in}}%
\pgfpathlineto{\pgfqpoint{2.686799in}{2.705277in}}%
\pgfpathlineto{\pgfqpoint{2.687674in}{2.764540in}}%
\pgfpathlineto{\pgfqpoint{2.688549in}{2.862470in}}%
\pgfpathlineto{\pgfqpoint{2.690300in}{2.412044in}}%
\pgfpathlineto{\pgfqpoint{2.691175in}{2.211991in}}%
\pgfpathlineto{\pgfqpoint{2.692051in}{2.203348in}}%
\pgfpathlineto{\pgfqpoint{2.692926in}{2.034706in}}%
\pgfpathlineto{\pgfqpoint{2.693802in}{1.920043in}}%
\pgfpathlineto{\pgfqpoint{2.694677in}{1.671385in}}%
\pgfpathlineto{\pgfqpoint{2.696428in}{1.401429in}}%
\pgfpathlineto{\pgfqpoint{2.697303in}{1.346921in}}%
\pgfpathlineto{\pgfqpoint{2.698178in}{1.317634in}}%
\pgfpathlineto{\pgfqpoint{2.699929in}{1.276551in}}%
\pgfpathlineto{\pgfqpoint{2.700805in}{1.336313in}}%
\pgfpathlineto{\pgfqpoint{2.703431in}{1.810502in}}%
\pgfpathlineto{\pgfqpoint{2.705181in}{2.016512in}}%
\pgfpathlineto{\pgfqpoint{2.706057in}{2.004383in}}%
\pgfpathlineto{\pgfqpoint{2.706932in}{2.079426in}}%
\pgfpathlineto{\pgfqpoint{2.707808in}{2.299548in}}%
\pgfpathlineto{\pgfqpoint{2.708683in}{2.421341in}}%
\pgfpathlineto{\pgfqpoint{2.709558in}{2.449160in}}%
\pgfpathlineto{\pgfqpoint{2.710434in}{2.312380in}}%
\pgfpathlineto{\pgfqpoint{2.711309in}{2.248598in}}%
\pgfpathlineto{\pgfqpoint{2.712184in}{2.111513in}}%
\pgfpathlineto{\pgfqpoint{2.713060in}{2.073068in}}%
\pgfpathlineto{\pgfqpoint{2.714811in}{1.835466in}}%
\pgfpathlineto{\pgfqpoint{2.716561in}{1.477401in}}%
\pgfpathlineto{\pgfqpoint{2.718312in}{1.265111in}}%
\pgfpathlineto{\pgfqpoint{2.719187in}{1.248316in}}%
\pgfpathlineto{\pgfqpoint{2.720063in}{1.133895in}}%
\pgfpathlineto{\pgfqpoint{2.720938in}{1.180511in}}%
\pgfpathlineto{\pgfqpoint{2.721814in}{1.292523in}}%
\pgfpathlineto{\pgfqpoint{2.725315in}{1.965919in}}%
\pgfpathlineto{\pgfqpoint{2.726191in}{1.912793in}}%
\pgfpathlineto{\pgfqpoint{2.727066in}{2.043545in}}%
\pgfpathlineto{\pgfqpoint{2.727941in}{1.990284in}}%
\pgfpathlineto{\pgfqpoint{2.729692in}{2.234197in}}%
\pgfpathlineto{\pgfqpoint{2.730567in}{2.280349in}}%
\pgfpathlineto{\pgfqpoint{2.733194in}{1.885224in}}%
\pgfpathlineto{\pgfqpoint{2.734069in}{1.900802in}}%
\pgfpathlineto{\pgfqpoint{2.734944in}{1.873012in}}%
\pgfpathlineto{\pgfqpoint{2.735820in}{1.611396in}}%
\pgfpathlineto{\pgfqpoint{2.736695in}{1.568926in}}%
\pgfpathlineto{\pgfqpoint{2.738446in}{1.325775in}}%
\pgfpathlineto{\pgfqpoint{2.739321in}{1.264733in}}%
\pgfpathlineto{\pgfqpoint{2.741947in}{1.160898in}}%
\pgfpathlineto{\pgfqpoint{2.742823in}{1.266564in}}%
\pgfpathlineto{\pgfqpoint{2.743698in}{1.316440in}}%
\pgfpathlineto{\pgfqpoint{2.744573in}{1.298774in}}%
\pgfpathlineto{\pgfqpoint{2.746324in}{1.590327in}}%
\pgfpathlineto{\pgfqpoint{2.747200in}{1.613661in}}%
\pgfpathlineto{\pgfqpoint{2.748075in}{1.778507in}}%
\pgfpathlineto{\pgfqpoint{2.748950in}{1.762331in}}%
\pgfpathlineto{\pgfqpoint{2.749826in}{1.786994in}}%
\pgfpathlineto{\pgfqpoint{2.750701in}{1.996710in}}%
\pgfpathlineto{\pgfqpoint{2.751576in}{2.025471in}}%
\pgfpathlineto{\pgfqpoint{2.752452in}{2.007827in}}%
\pgfpathlineto{\pgfqpoint{2.753327in}{1.825007in}}%
\pgfpathlineto{\pgfqpoint{2.754203in}{1.839100in}}%
\pgfpathlineto{\pgfqpoint{2.755078in}{1.823353in}}%
\pgfpathlineto{\pgfqpoint{2.756829in}{1.382885in}}%
\pgfpathlineto{\pgfqpoint{2.757704in}{1.278825in}}%
\pgfpathlineto{\pgfqpoint{2.758579in}{1.054392in}}%
\pgfpathlineto{\pgfqpoint{2.759455in}{1.130959in}}%
\pgfpathlineto{\pgfqpoint{2.760330in}{1.099025in}}%
\pgfpathlineto{\pgfqpoint{2.761206in}{1.089022in}}%
\pgfpathlineto{\pgfqpoint{2.762956in}{1.048987in}}%
\pgfpathlineto{\pgfqpoint{2.763832in}{1.052485in}}%
\pgfpathlineto{\pgfqpoint{2.764707in}{1.305487in}}%
\pgfpathlineto{\pgfqpoint{2.765582in}{1.352150in}}%
\pgfpathlineto{\pgfqpoint{2.766458in}{1.509203in}}%
\pgfpathlineto{\pgfqpoint{2.767333in}{1.574595in}}%
\pgfpathlineto{\pgfqpoint{2.768209in}{1.589728in}}%
\pgfpathlineto{\pgfqpoint{2.769084in}{1.735900in}}%
\pgfpathlineto{\pgfqpoint{2.769959in}{1.769004in}}%
\pgfpathlineto{\pgfqpoint{2.770835in}{1.816348in}}%
\pgfpathlineto{\pgfqpoint{2.771710in}{1.776633in}}%
\pgfpathlineto{\pgfqpoint{2.772585in}{1.988179in}}%
\pgfpathlineto{\pgfqpoint{2.773461in}{2.008123in}}%
\pgfpathlineto{\pgfqpoint{2.774336in}{1.895301in}}%
\pgfpathlineto{\pgfqpoint{2.775212in}{1.925360in}}%
\pgfpathlineto{\pgfqpoint{2.776962in}{1.954146in}}%
\pgfpathlineto{\pgfqpoint{2.777838in}{1.874130in}}%
\pgfpathlineto{\pgfqpoint{2.778713in}{1.703525in}}%
\pgfpathlineto{\pgfqpoint{2.780464in}{1.566418in}}%
\pgfpathlineto{\pgfqpoint{2.781339in}{1.421247in}}%
\pgfpathlineto{\pgfqpoint{2.782215in}{1.431155in}}%
\pgfpathlineto{\pgfqpoint{2.783090in}{1.384667in}}%
\pgfpathlineto{\pgfqpoint{2.783965in}{1.291451in}}%
\pgfpathlineto{\pgfqpoint{2.784841in}{1.261396in}}%
\pgfpathlineto{\pgfqpoint{2.785716in}{1.252469in}}%
\pgfpathlineto{\pgfqpoint{2.786591in}{1.232725in}}%
\pgfpathlineto{\pgfqpoint{2.787467in}{1.524610in}}%
\pgfpathlineto{\pgfqpoint{2.789218in}{1.680750in}}%
\pgfpathlineto{\pgfqpoint{2.791844in}{2.119756in}}%
\pgfpathlineto{\pgfqpoint{2.793594in}{2.738315in}}%
\pgfpathlineto{\pgfqpoint{2.794470in}{2.659063in}}%
\pgfpathlineto{\pgfqpoint{2.795345in}{2.687586in}}%
\pgfpathlineto{\pgfqpoint{2.796221in}{2.591392in}}%
\pgfpathlineto{\pgfqpoint{2.797096in}{2.636234in}}%
\pgfpathlineto{\pgfqpoint{2.797971in}{2.534378in}}%
\pgfpathlineto{\pgfqpoint{2.798847in}{2.540691in}}%
\pgfpathlineto{\pgfqpoint{2.800597in}{2.096961in}}%
\pgfpathlineto{\pgfqpoint{2.802348in}{1.787679in}}%
\pgfpathlineto{\pgfqpoint{2.803224in}{1.731345in}}%
\pgfpathlineto{\pgfqpoint{2.804099in}{1.705730in}}%
\pgfpathlineto{\pgfqpoint{2.804974in}{1.700595in}}%
\pgfpathlineto{\pgfqpoint{2.805850in}{1.723008in}}%
\pgfpathlineto{\pgfqpoint{2.807600in}{2.003886in}}%
\pgfpathlineto{\pgfqpoint{2.808476in}{2.364815in}}%
\pgfpathlineto{\pgfqpoint{2.809351in}{2.589013in}}%
\pgfpathlineto{\pgfqpoint{2.811102in}{2.764201in}}%
\pgfpathlineto{\pgfqpoint{2.811977in}{2.753940in}}%
\pgfpathlineto{\pgfqpoint{2.812853in}{2.954457in}}%
\pgfpathlineto{\pgfqpoint{2.813728in}{2.996178in}}%
\pgfpathlineto{\pgfqpoint{2.814603in}{3.012048in}}%
\pgfpathlineto{\pgfqpoint{2.815479in}{2.965348in}}%
\pgfpathlineto{\pgfqpoint{2.817230in}{2.689931in}}%
\pgfpathlineto{\pgfqpoint{2.818105in}{2.582950in}}%
\pgfpathlineto{\pgfqpoint{2.818980in}{2.519064in}}%
\pgfpathlineto{\pgfqpoint{2.819856in}{2.515288in}}%
\pgfpathlineto{\pgfqpoint{2.820731in}{2.231258in}}%
\pgfpathlineto{\pgfqpoint{2.821606in}{2.143268in}}%
\pgfpathlineto{\pgfqpoint{2.822482in}{1.922007in}}%
\pgfpathlineto{\pgfqpoint{2.824233in}{1.677819in}}%
\pgfpathlineto{\pgfqpoint{2.825108in}{1.457102in}}%
\pgfpathlineto{\pgfqpoint{2.826859in}{1.572399in}}%
\pgfpathlineto{\pgfqpoint{2.829485in}{2.290883in}}%
\pgfpathlineto{\pgfqpoint{2.830360in}{2.390963in}}%
\pgfpathlineto{\pgfqpoint{2.831236in}{2.428457in}}%
\pgfpathlineto{\pgfqpoint{2.832986in}{2.576807in}}%
\pgfpathlineto{\pgfqpoint{2.834737in}{2.817962in}}%
\pgfpathlineto{\pgfqpoint{2.835612in}{2.652849in}}%
\pgfpathlineto{\pgfqpoint{2.836488in}{2.681271in}}%
\pgfpathlineto{\pgfqpoint{2.837363in}{2.552166in}}%
\pgfpathlineto{\pgfqpoint{2.838239in}{2.478483in}}%
\pgfpathlineto{\pgfqpoint{2.839114in}{2.469042in}}%
\pgfpathlineto{\pgfqpoint{2.839989in}{2.342418in}}%
\pgfpathlineto{\pgfqpoint{2.843491in}{1.657188in}}%
\pgfpathlineto{\pgfqpoint{2.844366in}{1.592094in}}%
\pgfpathlineto{\pgfqpoint{2.846117in}{1.406174in}}%
\pgfpathlineto{\pgfqpoint{2.846992in}{1.376089in}}%
\pgfpathlineto{\pgfqpoint{2.847868in}{1.460787in}}%
\pgfpathlineto{\pgfqpoint{2.848743in}{1.592521in}}%
\pgfpathlineto{\pgfqpoint{2.849618in}{1.868633in}}%
\pgfpathlineto{\pgfqpoint{2.850494in}{1.887882in}}%
\pgfpathlineto{\pgfqpoint{2.851369in}{2.206145in}}%
\pgfpathlineto{\pgfqpoint{2.852245in}{2.319763in}}%
\pgfpathlineto{\pgfqpoint{2.853120in}{2.503499in}}%
\pgfpathlineto{\pgfqpoint{2.853995in}{2.603807in}}%
\pgfpathlineto{\pgfqpoint{2.854871in}{2.814667in}}%
\pgfpathlineto{\pgfqpoint{2.855746in}{2.795524in}}%
\pgfpathlineto{\pgfqpoint{2.856621in}{2.930738in}}%
\pgfpathlineto{\pgfqpoint{2.858372in}{2.682795in}}%
\pgfpathlineto{\pgfqpoint{2.859248in}{2.752951in}}%
\pgfpathlineto{\pgfqpoint{2.860998in}{2.611676in}}%
\pgfpathlineto{\pgfqpoint{2.861874in}{2.429381in}}%
\pgfpathlineto{\pgfqpoint{2.863625in}{1.871864in}}%
\pgfpathlineto{\pgfqpoint{2.864500in}{1.713281in}}%
\pgfpathlineto{\pgfqpoint{2.867126in}{1.460183in}}%
\pgfpathlineto{\pgfqpoint{2.868001in}{1.461875in}}%
\pgfpathlineto{\pgfqpoint{2.868877in}{1.535034in}}%
\pgfpathlineto{\pgfqpoint{2.870628in}{1.986770in}}%
\pgfpathlineto{\pgfqpoint{2.871503in}{2.213081in}}%
\pgfpathlineto{\pgfqpoint{2.875004in}{2.642145in}}%
\pgfpathlineto{\pgfqpoint{2.875880in}{2.670830in}}%
\pgfpathlineto{\pgfqpoint{2.876755in}{2.973025in}}%
\pgfpathlineto{\pgfqpoint{2.877631in}{2.951439in}}%
\pgfpathlineto{\pgfqpoint{2.878506in}{2.880278in}}%
\pgfpathlineto{\pgfqpoint{2.879381in}{2.923560in}}%
\pgfpathlineto{\pgfqpoint{2.880257in}{2.849724in}}%
\pgfpathlineto{\pgfqpoint{2.881132in}{2.835867in}}%
\pgfpathlineto{\pgfqpoint{2.882007in}{2.777448in}}%
\pgfpathlineto{\pgfqpoint{2.882883in}{2.641520in}}%
\pgfpathlineto{\pgfqpoint{2.884634in}{2.114602in}}%
\pgfpathlineto{\pgfqpoint{2.885509in}{1.945507in}}%
\pgfpathlineto{\pgfqpoint{2.886384in}{1.900711in}}%
\pgfpathlineto{\pgfqpoint{2.887260in}{1.798010in}}%
\pgfpathlineto{\pgfqpoint{2.888135in}{1.742461in}}%
\pgfpathlineto{\pgfqpoint{2.889010in}{1.744394in}}%
\pgfpathlineto{\pgfqpoint{2.889886in}{1.781457in}}%
\pgfpathlineto{\pgfqpoint{2.891637in}{2.187560in}}%
\pgfpathlineto{\pgfqpoint{2.892512in}{2.550052in}}%
\pgfpathlineto{\pgfqpoint{2.894263in}{2.779886in}}%
\pgfpathlineto{\pgfqpoint{2.895138in}{3.105963in}}%
\pgfpathlineto{\pgfqpoint{2.896013in}{3.157606in}}%
\pgfpathlineto{\pgfqpoint{2.896889in}{3.170729in}}%
\pgfpathlineto{\pgfqpoint{2.897764in}{3.253669in}}%
\pgfpathlineto{\pgfqpoint{2.898640in}{3.142004in}}%
\pgfpathlineto{\pgfqpoint{2.899515in}{3.095455in}}%
\pgfpathlineto{\pgfqpoint{2.900390in}{3.134908in}}%
\pgfpathlineto{\pgfqpoint{2.901266in}{3.030527in}}%
\pgfpathlineto{\pgfqpoint{2.902141in}{2.988379in}}%
\pgfpathlineto{\pgfqpoint{2.903016in}{2.961284in}}%
\pgfpathlineto{\pgfqpoint{2.904767in}{2.690031in}}%
\pgfpathlineto{\pgfqpoint{2.905643in}{2.472758in}}%
\pgfpathlineto{\pgfqpoint{2.907393in}{2.203831in}}%
\pgfpathlineto{\pgfqpoint{2.908269in}{2.155924in}}%
\pgfpathlineto{\pgfqpoint{2.909144in}{1.988128in}}%
\pgfpathlineto{\pgfqpoint{2.910019in}{1.877392in}}%
\pgfpathlineto{\pgfqpoint{2.910895in}{1.830089in}}%
\pgfpathlineto{\pgfqpoint{2.911770in}{1.933972in}}%
\pgfpathlineto{\pgfqpoint{2.912646in}{2.155501in}}%
\pgfpathlineto{\pgfqpoint{2.913521in}{2.478715in}}%
\pgfpathlineto{\pgfqpoint{2.915272in}{2.786747in}}%
\pgfpathlineto{\pgfqpoint{2.916147in}{2.694162in}}%
\pgfpathlineto{\pgfqpoint{2.917898in}{3.116364in}}%
\pgfpathlineto{\pgfqpoint{2.918773in}{3.179170in}}%
\pgfpathlineto{\pgfqpoint{2.919649in}{3.094860in}}%
\pgfpathlineto{\pgfqpoint{2.920524in}{3.122632in}}%
\pgfpathlineto{\pgfqpoint{2.921399in}{3.014357in}}%
\pgfpathlineto{\pgfqpoint{2.922275in}{2.830130in}}%
\pgfpathlineto{\pgfqpoint{2.923150in}{2.799680in}}%
\pgfpathlineto{\pgfqpoint{2.924901in}{2.404250in}}%
\pgfpathlineto{\pgfqpoint{2.925776in}{2.056425in}}%
\pgfpathlineto{\pgfqpoint{2.928402in}{1.504043in}}%
\pgfpathlineto{\pgfqpoint{2.930153in}{1.413394in}}%
\pgfpathlineto{\pgfqpoint{2.931028in}{1.411551in}}%
\pgfpathlineto{\pgfqpoint{2.931904in}{1.447949in}}%
\pgfpathlineto{\pgfqpoint{2.932779in}{1.595672in}}%
\pgfpathlineto{\pgfqpoint{2.933655in}{1.828767in}}%
\pgfpathlineto{\pgfqpoint{2.934530in}{2.138621in}}%
\pgfpathlineto{\pgfqpoint{2.936281in}{2.302806in}}%
\pgfpathlineto{\pgfqpoint{2.937156in}{2.414445in}}%
\pgfpathlineto{\pgfqpoint{2.938031in}{2.634760in}}%
\pgfpathlineto{\pgfqpoint{2.938907in}{2.640429in}}%
\pgfpathlineto{\pgfqpoint{2.939782in}{2.815761in}}%
\pgfpathlineto{\pgfqpoint{2.940658in}{2.838175in}}%
\pgfpathlineto{\pgfqpoint{2.941533in}{2.819918in}}%
\pgfpathlineto{\pgfqpoint{2.943284in}{2.634516in}}%
\pgfpathlineto{\pgfqpoint{2.944159in}{2.670639in}}%
\pgfpathlineto{\pgfqpoint{2.945910in}{2.472788in}}%
\pgfpathlineto{\pgfqpoint{2.947661in}{1.972814in}}%
\pgfpathlineto{\pgfqpoint{2.949411in}{1.690355in}}%
\pgfpathlineto{\pgfqpoint{2.950287in}{1.662746in}}%
\pgfpathlineto{\pgfqpoint{2.951162in}{1.603935in}}%
\pgfpathlineto{\pgfqpoint{2.952037in}{1.569651in}}%
\pgfpathlineto{\pgfqpoint{2.952913in}{1.611516in}}%
\pgfpathlineto{\pgfqpoint{2.953788in}{1.815089in}}%
\pgfpathlineto{\pgfqpoint{2.954664in}{2.156427in}}%
\pgfpathlineto{\pgfqpoint{2.955539in}{2.626107in}}%
\pgfpathlineto{\pgfqpoint{2.956414in}{2.674789in}}%
\pgfpathlineto{\pgfqpoint{2.958165in}{3.226749in}}%
\pgfpathlineto{\pgfqpoint{2.959040in}{3.250911in}}%
\pgfpathlineto{\pgfqpoint{2.959916in}{3.464768in}}%
\pgfpathlineto{\pgfqpoint{2.960791in}{3.568298in}}%
\pgfpathlineto{\pgfqpoint{2.961667in}{3.707431in}}%
\pgfpathlineto{\pgfqpoint{2.963417in}{3.341443in}}%
\pgfpathlineto{\pgfqpoint{2.964293in}{3.262311in}}%
\pgfpathlineto{\pgfqpoint{2.965168in}{3.256701in}}%
\pgfpathlineto{\pgfqpoint{2.966919in}{2.839673in}}%
\pgfpathlineto{\pgfqpoint{2.968670in}{2.228510in}}%
\pgfpathlineto{\pgfqpoint{2.969545in}{1.918654in}}%
\pgfpathlineto{\pgfqpoint{2.972171in}{1.636165in}}%
\pgfpathlineto{\pgfqpoint{2.973046in}{1.559834in}}%
\pgfpathlineto{\pgfqpoint{2.973922in}{1.624566in}}%
\pgfpathlineto{\pgfqpoint{2.974797in}{1.877350in}}%
\pgfpathlineto{\pgfqpoint{2.975673in}{2.263025in}}%
\pgfpathlineto{\pgfqpoint{2.978299in}{3.005243in}}%
\pgfpathlineto{\pgfqpoint{2.979174in}{3.026074in}}%
\pgfpathlineto{\pgfqpoint{2.981800in}{3.356245in}}%
\pgfpathlineto{\pgfqpoint{2.982676in}{3.396951in}}%
\pgfpathlineto{\pgfqpoint{2.983551in}{3.362889in}}%
\pgfpathlineto{\pgfqpoint{2.985302in}{2.955864in}}%
\pgfpathlineto{\pgfqpoint{2.986177in}{3.138927in}}%
\pgfpathlineto{\pgfqpoint{2.987052in}{3.048972in}}%
\pgfpathlineto{\pgfqpoint{2.987928in}{3.028251in}}%
\pgfpathlineto{\pgfqpoint{2.989679in}{2.454483in}}%
\pgfpathlineto{\pgfqpoint{2.990554in}{2.199270in}}%
\pgfpathlineto{\pgfqpoint{2.992305in}{1.982933in}}%
\pgfpathlineto{\pgfqpoint{2.993180in}{1.918322in}}%
\pgfpathlineto{\pgfqpoint{2.994056in}{1.810757in}}%
\pgfpathlineto{\pgfqpoint{2.995806in}{1.992273in}}%
\pgfpathlineto{\pgfqpoint{2.996682in}{2.458746in}}%
\pgfpathlineto{\pgfqpoint{2.998432in}{2.824567in}}%
\pgfpathlineto{\pgfqpoint{2.999308in}{2.816340in}}%
\pgfpathlineto{\pgfqpoint{3.000183in}{3.009957in}}%
\pgfpathlineto{\pgfqpoint{3.001059in}{3.102287in}}%
\pgfpathlineto{\pgfqpoint{3.001934in}{3.319235in}}%
\pgfpathlineto{\pgfqpoint{3.002809in}{3.342899in}}%
\pgfpathlineto{\pgfqpoint{3.003685in}{3.305977in}}%
\pgfpathlineto{\pgfqpoint{3.006311in}{3.027890in}}%
\pgfpathlineto{\pgfqpoint{3.007186in}{3.138594in}}%
\pgfpathlineto{\pgfqpoint{3.008062in}{3.175537in}}%
\pgfpathlineto{\pgfqpoint{3.008937in}{3.046979in}}%
\pgfpathlineto{\pgfqpoint{3.010688in}{2.548032in}}%
\pgfpathlineto{\pgfqpoint{3.011563in}{2.328553in}}%
\pgfpathlineto{\pgfqpoint{3.013314in}{2.132182in}}%
\pgfpathlineto{\pgfqpoint{3.014189in}{2.061801in}}%
\pgfpathlineto{\pgfqpoint{3.015065in}{2.019452in}}%
\pgfpathlineto{\pgfqpoint{3.015940in}{2.027517in}}%
\pgfpathlineto{\pgfqpoint{3.016815in}{2.114740in}}%
\pgfpathlineto{\pgfqpoint{3.017691in}{2.270290in}}%
\pgfpathlineto{\pgfqpoint{3.019441in}{3.000378in}}%
\pgfpathlineto{\pgfqpoint{3.020317in}{3.128637in}}%
\pgfpathlineto{\pgfqpoint{3.021192in}{3.144763in}}%
\pgfpathlineto{\pgfqpoint{3.022068in}{3.183781in}}%
\pgfpathlineto{\pgfqpoint{3.022943in}{3.337884in}}%
\pgfpathlineto{\pgfqpoint{3.023818in}{3.346808in}}%
\pgfpathlineto{\pgfqpoint{3.025569in}{3.157323in}}%
\pgfpathlineto{\pgfqpoint{3.026444in}{3.105588in}}%
\pgfpathlineto{\pgfqpoint{3.027320in}{2.978936in}}%
\pgfpathlineto{\pgfqpoint{3.028195in}{3.021364in}}%
\pgfpathlineto{\pgfqpoint{3.029071in}{3.027979in}}%
\pgfpathlineto{\pgfqpoint{3.029946in}{2.860274in}}%
\pgfpathlineto{\pgfqpoint{3.031697in}{2.289375in}}%
\pgfpathlineto{\pgfqpoint{3.032572in}{2.111823in}}%
\pgfpathlineto{\pgfqpoint{3.035198in}{1.883161in}}%
\pgfpathlineto{\pgfqpoint{3.036074in}{1.793690in}}%
\pgfpathlineto{\pgfqpoint{3.036949in}{1.835164in}}%
\pgfpathlineto{\pgfqpoint{3.038700in}{2.257778in}}%
\pgfpathlineto{\pgfqpoint{3.039575in}{2.604348in}}%
\pgfpathlineto{\pgfqpoint{3.040450in}{2.410888in}}%
\pgfpathlineto{\pgfqpoint{3.043952in}{3.147041in}}%
\pgfpathlineto{\pgfqpoint{3.044827in}{3.131713in}}%
\pgfpathlineto{\pgfqpoint{3.047453in}{2.603969in}}%
\pgfpathlineto{\pgfqpoint{3.048329in}{2.580589in}}%
\pgfpathlineto{\pgfqpoint{3.049204in}{2.587390in}}%
\pgfpathlineto{\pgfqpoint{3.050080in}{2.408207in}}%
\pgfpathlineto{\pgfqpoint{3.050955in}{2.303874in}}%
\pgfpathlineto{\pgfqpoint{3.051830in}{1.906209in}}%
\pgfpathlineto{\pgfqpoint{3.052706in}{1.661145in}}%
\pgfpathlineto{\pgfqpoint{3.053581in}{1.632510in}}%
\pgfpathlineto{\pgfqpoint{3.056207in}{1.376904in}}%
\pgfpathlineto{\pgfqpoint{3.057958in}{1.349809in}}%
\pgfpathlineto{\pgfqpoint{3.058833in}{1.394486in}}%
\pgfpathlineto{\pgfqpoint{3.059709in}{1.601829in}}%
\pgfpathlineto{\pgfqpoint{3.060584in}{2.005086in}}%
\pgfpathlineto{\pgfqpoint{3.061459in}{2.125258in}}%
\pgfpathlineto{\pgfqpoint{3.062335in}{2.191784in}}%
\pgfpathlineto{\pgfqpoint{3.063210in}{2.321873in}}%
\pgfpathlineto{\pgfqpoint{3.064086in}{2.341375in}}%
\pgfpathlineto{\pgfqpoint{3.064961in}{2.523006in}}%
\pgfpathlineto{\pgfqpoint{3.065836in}{2.563056in}}%
\pgfpathlineto{\pgfqpoint{3.066712in}{2.647871in}}%
\pgfpathlineto{\pgfqpoint{3.067587in}{2.605803in}}%
\pgfpathlineto{\pgfqpoint{3.068462in}{2.495870in}}%
\pgfpathlineto{\pgfqpoint{3.069338in}{2.474570in}}%
\pgfpathlineto{\pgfqpoint{3.070213in}{2.469737in}}%
\pgfpathlineto{\pgfqpoint{3.071089in}{2.578600in}}%
\pgfpathlineto{\pgfqpoint{3.071964in}{2.450526in}}%
\pgfpathlineto{\pgfqpoint{3.072839in}{2.169940in}}%
\pgfpathlineto{\pgfqpoint{3.073715in}{1.979036in}}%
\pgfpathlineto{\pgfqpoint{3.074590in}{1.715003in}}%
\pgfpathlineto{\pgfqpoint{3.075465in}{1.563639in}}%
\pgfpathlineto{\pgfqpoint{3.077216in}{1.460334in}}%
\pgfpathlineto{\pgfqpoint{3.078092in}{1.439945in}}%
\pgfpathlineto{\pgfqpoint{3.078967in}{1.471178in}}%
\pgfpathlineto{\pgfqpoint{3.079842in}{1.537531in}}%
\pgfpathlineto{\pgfqpoint{3.080718in}{1.766511in}}%
\pgfpathlineto{\pgfqpoint{3.082468in}{2.047040in}}%
\pgfpathlineto{\pgfqpoint{3.083344in}{2.352569in}}%
\pgfpathlineto{\pgfqpoint{3.084219in}{2.498348in}}%
\pgfpathlineto{\pgfqpoint{3.085095in}{2.470944in}}%
\pgfpathlineto{\pgfqpoint{3.085970in}{2.668972in}}%
\pgfpathlineto{\pgfqpoint{3.087721in}{2.896889in}}%
\pgfpathlineto{\pgfqpoint{3.089471in}{2.812989in}}%
\pgfpathlineto{\pgfqpoint{3.090347in}{2.587028in}}%
\pgfpathlineto{\pgfqpoint{3.091222in}{2.575338in}}%
\pgfpathlineto{\pgfqpoint{3.092973in}{2.300642in}}%
\pgfpathlineto{\pgfqpoint{3.093848in}{1.982540in}}%
\pgfpathlineto{\pgfqpoint{3.096474in}{1.467281in}}%
\pgfpathlineto{\pgfqpoint{3.097350in}{1.385423in}}%
\pgfpathlineto{\pgfqpoint{3.098225in}{1.331021in}}%
\pgfpathlineto{\pgfqpoint{3.099101in}{1.241792in}}%
\pgfpathlineto{\pgfqpoint{3.099976in}{1.300150in}}%
\pgfpathlineto{\pgfqpoint{3.102602in}{2.045357in}}%
\pgfpathlineto{\pgfqpoint{3.103477in}{2.355623in}}%
\pgfpathlineto{\pgfqpoint{3.104353in}{2.483243in}}%
\pgfpathlineto{\pgfqpoint{3.105228in}{2.694030in}}%
\pgfpathlineto{\pgfqpoint{3.106104in}{2.621330in}}%
\pgfpathlineto{\pgfqpoint{3.107854in}{2.912414in}}%
\pgfpathlineto{\pgfqpoint{3.111356in}{2.486204in}}%
\pgfpathlineto{\pgfqpoint{3.112231in}{2.436812in}}%
\pgfpathlineto{\pgfqpoint{3.113107in}{2.296806in}}%
\pgfpathlineto{\pgfqpoint{3.113982in}{2.341753in}}%
\pgfpathlineto{\pgfqpoint{3.114857in}{2.148856in}}%
\pgfpathlineto{\pgfqpoint{3.116608in}{1.615896in}}%
\pgfpathlineto{\pgfqpoint{3.117483in}{1.442905in}}%
\pgfpathlineto{\pgfqpoint{3.118359in}{1.419948in}}%
\pgfpathlineto{\pgfqpoint{3.119234in}{1.334465in}}%
\pgfpathlineto{\pgfqpoint{3.120110in}{1.191921in}}%
\pgfpathlineto{\pgfqpoint{3.120985in}{1.192646in}}%
\pgfpathlineto{\pgfqpoint{3.121860in}{1.379359in}}%
\pgfpathlineto{\pgfqpoint{3.123611in}{1.958472in}}%
\pgfpathlineto{\pgfqpoint{3.124486in}{2.080138in}}%
\pgfpathlineto{\pgfqpoint{3.125362in}{2.392502in}}%
\pgfpathlineto{\pgfqpoint{3.126237in}{2.490640in}}%
\pgfpathlineto{\pgfqpoint{3.127113in}{2.666365in}}%
\pgfpathlineto{\pgfqpoint{3.127988in}{2.900551in}}%
\pgfpathlineto{\pgfqpoint{3.128863in}{2.850716in}}%
\pgfpathlineto{\pgfqpoint{3.129739in}{2.875143in}}%
\pgfpathlineto{\pgfqpoint{3.130614in}{2.825186in}}%
\pgfpathlineto{\pgfqpoint{3.131490in}{2.619222in}}%
\pgfpathlineto{\pgfqpoint{3.132365in}{2.612397in}}%
\pgfpathlineto{\pgfqpoint{3.133240in}{2.756243in}}%
\pgfpathlineto{\pgfqpoint{3.134116in}{2.744191in}}%
\pgfpathlineto{\pgfqpoint{3.134991in}{2.693686in}}%
\pgfpathlineto{\pgfqpoint{3.135866in}{2.367277in}}%
\pgfpathlineto{\pgfqpoint{3.138493in}{1.787770in}}%
\pgfpathlineto{\pgfqpoint{3.140243in}{1.650724in}}%
\pgfpathlineto{\pgfqpoint{3.141994in}{1.457374in}}%
\pgfpathlineto{\pgfqpoint{3.142869in}{1.514866in}}%
\pgfpathlineto{\pgfqpoint{3.144620in}{2.022004in}}%
\pgfpathlineto{\pgfqpoint{3.145496in}{2.184685in}}%
\pgfpathlineto{\pgfqpoint{3.147246in}{2.560289in}}%
\pgfpathlineto{\pgfqpoint{3.148997in}{2.718196in}}%
\pgfpathlineto{\pgfqpoint{3.149872in}{2.874791in}}%
\pgfpathlineto{\pgfqpoint{3.150748in}{2.928283in}}%
\pgfpathlineto{\pgfqpoint{3.151623in}{2.829922in}}%
\pgfpathlineto{\pgfqpoint{3.152499in}{2.671696in}}%
\pgfpathlineto{\pgfqpoint{3.153374in}{2.636639in}}%
\pgfpathlineto{\pgfqpoint{3.154249in}{2.714529in}}%
\pgfpathlineto{\pgfqpoint{3.155125in}{2.627716in}}%
\pgfpathlineto{\pgfqpoint{3.156000in}{2.572891in}}%
\pgfpathlineto{\pgfqpoint{3.157751in}{1.985440in}}%
\pgfpathlineto{\pgfqpoint{3.158626in}{1.809277in}}%
\pgfpathlineto{\pgfqpoint{3.159502in}{1.695158in}}%
\pgfpathlineto{\pgfqpoint{3.160377in}{1.617709in}}%
\pgfpathlineto{\pgfqpoint{3.161252in}{1.511352in}}%
\pgfpathlineto{\pgfqpoint{3.162128in}{1.505674in}}%
\pgfpathlineto{\pgfqpoint{3.163003in}{1.517031in}}%
\pgfpathlineto{\pgfqpoint{3.164754in}{2.026563in}}%
\pgfpathlineto{\pgfqpoint{3.165629in}{2.346277in}}%
\pgfpathlineto{\pgfqpoint{3.167380in}{2.535989in}}%
\pgfpathlineto{\pgfqpoint{3.169131in}{2.797465in}}%
\pgfpathlineto{\pgfqpoint{3.170006in}{2.954165in}}%
\pgfpathlineto{\pgfqpoint{3.170881in}{2.987346in}}%
\pgfpathlineto{\pgfqpoint{3.171757in}{3.310575in}}%
\pgfpathlineto{\pgfqpoint{3.172632in}{3.242640in}}%
\pgfpathlineto{\pgfqpoint{3.173508in}{3.114042in}}%
\pgfpathlineto{\pgfqpoint{3.175258in}{3.014598in}}%
\pgfpathlineto{\pgfqpoint{3.176134in}{2.914554in}}%
\pgfpathlineto{\pgfqpoint{3.177884in}{2.438927in}}%
\pgfpathlineto{\pgfqpoint{3.179635in}{1.975109in}}%
\pgfpathlineto{\pgfqpoint{3.181386in}{1.706515in}}%
\pgfpathlineto{\pgfqpoint{3.182261in}{1.647704in}}%
\pgfpathlineto{\pgfqpoint{3.183137in}{1.459881in}}%
\pgfpathlineto{\pgfqpoint{3.184887in}{1.816518in}}%
\pgfpathlineto{\pgfqpoint{3.185763in}{2.241825in}}%
\pgfpathlineto{\pgfqpoint{3.187514in}{2.706330in}}%
\pgfpathlineto{\pgfqpoint{3.189264in}{3.027375in}}%
\pgfpathlineto{\pgfqpoint{3.190140in}{3.038331in}}%
\pgfpathlineto{\pgfqpoint{3.191890in}{3.196490in}}%
\pgfpathlineto{\pgfqpoint{3.193641in}{2.958426in}}%
\pgfpathlineto{\pgfqpoint{3.197143in}{2.500910in}}%
\pgfpathlineto{\pgfqpoint{3.198018in}{2.320367in}}%
\pgfpathlineto{\pgfqpoint{3.199769in}{1.790609in}}%
\pgfpathlineto{\pgfqpoint{3.201520in}{1.479545in}}%
\pgfpathlineto{\pgfqpoint{3.202395in}{1.471631in}}%
\pgfpathlineto{\pgfqpoint{3.203270in}{1.498092in}}%
\pgfpathlineto{\pgfqpoint{3.204146in}{1.496914in}}%
\pgfpathlineto{\pgfqpoint{3.205021in}{1.582398in}}%
\pgfpathlineto{\pgfqpoint{3.205896in}{1.779630in}}%
\pgfpathlineto{\pgfqpoint{3.206772in}{1.893413in}}%
\pgfpathlineto{\pgfqpoint{3.208523in}{2.224795in}}%
\pgfpathlineto{\pgfqpoint{3.212024in}{2.689690in}}%
\pgfpathlineto{\pgfqpoint{3.212899in}{2.915441in}}%
\pgfpathlineto{\pgfqpoint{3.214650in}{2.782046in}}%
\pgfpathlineto{\pgfqpoint{3.216401in}{2.608142in}}%
\pgfpathlineto{\pgfqpoint{3.217276in}{2.586152in}}%
\pgfpathlineto{\pgfqpoint{3.218152in}{2.557033in}}%
\pgfpathlineto{\pgfqpoint{3.220778in}{1.771791in}}%
\pgfpathlineto{\pgfqpoint{3.221653in}{1.675886in}}%
\pgfpathlineto{\pgfqpoint{3.223404in}{1.383127in}}%
\pgfpathlineto{\pgfqpoint{3.224279in}{1.308034in}}%
\pgfpathlineto{\pgfqpoint{3.225155in}{1.317458in}}%
\pgfpathlineto{\pgfqpoint{3.226030in}{1.330417in}}%
\pgfpathlineto{\pgfqpoint{3.226905in}{1.470688in}}%
\pgfpathlineto{\pgfqpoint{3.227781in}{1.662358in}}%
\pgfpathlineto{\pgfqpoint{3.228656in}{2.061945in}}%
\pgfpathlineto{\pgfqpoint{3.232158in}{2.629906in}}%
\pgfpathlineto{\pgfqpoint{3.233033in}{2.716069in}}%
\pgfpathlineto{\pgfqpoint{3.233908in}{3.026258in}}%
\pgfpathlineto{\pgfqpoint{3.234784in}{3.055121in}}%
\pgfpathlineto{\pgfqpoint{3.236535in}{2.840014in}}%
\pgfpathlineto{\pgfqpoint{3.237410in}{2.753369in}}%
\pgfpathlineto{\pgfqpoint{3.238285in}{2.794334in}}%
\pgfpathlineto{\pgfqpoint{3.239161in}{2.707400in}}%
\pgfpathlineto{\pgfqpoint{3.240036in}{2.488797in}}%
\pgfpathlineto{\pgfqpoint{3.241787in}{1.913609in}}%
\pgfpathlineto{\pgfqpoint{3.243538in}{1.575873in}}%
\pgfpathlineto{\pgfqpoint{3.244413in}{1.453809in}}%
\pgfpathlineto{\pgfqpoint{3.245288in}{1.385120in}}%
\pgfpathlineto{\pgfqpoint{3.246164in}{1.355398in}}%
\pgfpathlineto{\pgfqpoint{3.247039in}{1.398019in}}%
\pgfpathlineto{\pgfqpoint{3.250541in}{2.613251in}}%
\pgfpathlineto{\pgfqpoint{3.252291in}{2.927044in}}%
\pgfpathlineto{\pgfqpoint{3.253167in}{2.981193in}}%
\pgfpathlineto{\pgfqpoint{3.254042in}{2.999511in}}%
\pgfpathlineto{\pgfqpoint{3.254917in}{3.383699in}}%
\pgfpathlineto{\pgfqpoint{3.255793in}{3.356040in}}%
\pgfpathlineto{\pgfqpoint{3.258419in}{2.833775in}}%
\pgfpathlineto{\pgfqpoint{3.259294in}{2.807050in}}%
\pgfpathlineto{\pgfqpoint{3.260170in}{2.796297in}}%
\pgfpathlineto{\pgfqpoint{3.261045in}{2.679610in}}%
\pgfpathlineto{\pgfqpoint{3.261921in}{2.518973in}}%
\pgfpathlineto{\pgfqpoint{3.263671in}{2.024708in}}%
\pgfpathlineto{\pgfqpoint{3.264547in}{1.880684in}}%
\pgfpathlineto{\pgfqpoint{3.265422in}{1.814805in}}%
\pgfpathlineto{\pgfqpoint{3.266297in}{1.791969in}}%
\pgfpathlineto{\pgfqpoint{3.267173in}{1.738775in}}%
\pgfpathlineto{\pgfqpoint{3.268048in}{1.718054in}}%
\pgfpathlineto{\pgfqpoint{3.268924in}{1.934850in}}%
\pgfpathlineto{\pgfqpoint{3.270674in}{2.638867in}}%
\pgfpathlineto{\pgfqpoint{3.271550in}{2.697900in}}%
\pgfpathlineto{\pgfqpoint{3.272425in}{2.985260in}}%
\pgfpathlineto{\pgfqpoint{3.273300in}{3.034258in}}%
\pgfpathlineto{\pgfqpoint{3.274176in}{3.119762in}}%
\pgfpathlineto{\pgfqpoint{3.275051in}{3.155152in}}%
\pgfpathlineto{\pgfqpoint{3.276802in}{3.255527in}}%
\pgfpathlineto{\pgfqpoint{3.277677in}{3.225783in}}%
\pgfpathlineto{\pgfqpoint{3.279428in}{2.892525in}}%
\pgfpathlineto{\pgfqpoint{3.280303in}{2.933796in}}%
\pgfpathlineto{\pgfqpoint{3.282054in}{3.127599in}}%
\pgfpathlineto{\pgfqpoint{3.282930in}{2.980797in}}%
\pgfpathlineto{\pgfqpoint{3.283805in}{2.597298in}}%
\pgfpathlineto{\pgfqpoint{3.284680in}{2.388180in}}%
\pgfpathlineto{\pgfqpoint{3.286431in}{2.164412in}}%
\pgfpathlineto{\pgfqpoint{3.287306in}{2.052075in}}%
\pgfpathlineto{\pgfqpoint{3.288182in}{1.900590in}}%
\pgfpathlineto{\pgfqpoint{3.289057in}{1.855734in}}%
\pgfpathlineto{\pgfqpoint{3.289933in}{1.982180in}}%
\pgfpathlineto{\pgfqpoint{3.290808in}{2.380049in}}%
\pgfpathlineto{\pgfqpoint{3.292559in}{2.758517in}}%
\pgfpathlineto{\pgfqpoint{3.293434in}{2.928443in}}%
\pgfpathlineto{\pgfqpoint{3.294309in}{2.971959in}}%
\pgfpathlineto{\pgfqpoint{3.295185in}{3.203039in}}%
\pgfpathlineto{\pgfqpoint{3.296936in}{3.498546in}}%
\pgfpathlineto{\pgfqpoint{3.297811in}{3.520035in}}%
\pgfpathlineto{\pgfqpoint{3.298686in}{3.431399in}}%
\pgfpathlineto{\pgfqpoint{3.300437in}{2.993999in}}%
\pgfpathlineto{\pgfqpoint{3.301312in}{2.870151in}}%
\pgfpathlineto{\pgfqpoint{3.302188in}{2.791373in}}%
\pgfpathlineto{\pgfqpoint{3.303063in}{2.787356in}}%
\pgfpathlineto{\pgfqpoint{3.304814in}{2.177038in}}%
\pgfpathlineto{\pgfqpoint{3.305689in}{2.143811in}}%
\pgfpathlineto{\pgfqpoint{3.307440in}{1.953965in}}%
\pgfpathlineto{\pgfqpoint{3.308315in}{1.939436in}}%
\pgfpathlineto{\pgfqpoint{3.309191in}{1.850629in}}%
\pgfpathlineto{\pgfqpoint{3.310066in}{1.868058in}}%
\pgfpathlineto{\pgfqpoint{3.310942in}{2.068825in}}%
\pgfpathlineto{\pgfqpoint{3.312692in}{2.725028in}}%
\pgfpathlineto{\pgfqpoint{3.313568in}{2.850371in}}%
\pgfpathlineto{\pgfqpoint{3.314443in}{3.051747in}}%
\pgfpathlineto{\pgfqpoint{3.315318in}{3.126504in}}%
\pgfpathlineto{\pgfqpoint{3.316194in}{3.364836in}}%
\pgfpathlineto{\pgfqpoint{3.317069in}{3.392710in}}%
\pgfpathlineto{\pgfqpoint{3.317945in}{3.504629in}}%
\pgfpathlineto{\pgfqpoint{3.318820in}{3.446226in}}%
\pgfpathlineto{\pgfqpoint{3.321446in}{2.874722in}}%
\pgfpathlineto{\pgfqpoint{3.322321in}{2.920686in}}%
\pgfpathlineto{\pgfqpoint{3.323197in}{2.783006in}}%
\pgfpathlineto{\pgfqpoint{3.324072in}{2.812065in}}%
\pgfpathlineto{\pgfqpoint{3.325823in}{2.347341in}}%
\pgfpathlineto{\pgfqpoint{3.326698in}{2.324535in}}%
\pgfpathlineto{\pgfqpoint{3.328449in}{2.127893in}}%
\pgfpathlineto{\pgfqpoint{3.329324in}{2.087597in}}%
\pgfpathlineto{\pgfqpoint{3.330200in}{2.008819in}}%
\pgfpathlineto{\pgfqpoint{3.331075in}{2.077025in}}%
\pgfpathlineto{\pgfqpoint{3.331951in}{2.283508in}}%
\pgfpathlineto{\pgfqpoint{3.334577in}{3.113969in}}%
\pgfpathlineto{\pgfqpoint{3.335452in}{3.239416in}}%
\pgfpathlineto{\pgfqpoint{3.336327in}{3.177879in}}%
\pgfpathlineto{\pgfqpoint{3.337203in}{3.400195in}}%
\pgfpathlineto{\pgfqpoint{3.338078in}{3.349469in}}%
\pgfpathlineto{\pgfqpoint{3.338954in}{3.471544in}}%
\pgfpathlineto{\pgfqpoint{3.339829in}{3.299429in}}%
\pgfpathlineto{\pgfqpoint{3.340704in}{3.198394in}}%
\pgfpathlineto{\pgfqpoint{3.343330in}{2.701177in}}%
\pgfpathlineto{\pgfqpoint{3.344206in}{2.682661in}}%
\pgfpathlineto{\pgfqpoint{3.345081in}{2.628350in}}%
\pgfpathlineto{\pgfqpoint{3.346832in}{2.068779in}}%
\pgfpathlineto{\pgfqpoint{3.347707in}{1.855432in}}%
\pgfpathlineto{\pgfqpoint{3.348583in}{1.721920in}}%
\pgfpathlineto{\pgfqpoint{3.351209in}{1.552191in}}%
\pgfpathlineto{\pgfqpoint{3.352084in}{1.625593in}}%
\pgfpathlineto{\pgfqpoint{3.355586in}{2.625228in}}%
\pgfpathlineto{\pgfqpoint{3.356461in}{2.627165in}}%
\pgfpathlineto{\pgfqpoint{3.357336in}{2.858483in}}%
\pgfpathlineto{\pgfqpoint{3.358212in}{2.850787in}}%
\pgfpathlineto{\pgfqpoint{3.359963in}{3.087614in}}%
\pgfpathlineto{\pgfqpoint{3.360838in}{3.144002in}}%
\pgfpathlineto{\pgfqpoint{3.361713in}{3.093002in}}%
\pgfpathlineto{\pgfqpoint{3.362589in}{3.014807in}}%
\pgfpathlineto{\pgfqpoint{3.363464in}{2.895061in}}%
\pgfpathlineto{\pgfqpoint{3.364339in}{2.874773in}}%
\pgfpathlineto{\pgfqpoint{3.365215in}{2.731746in}}%
\pgfpathlineto{\pgfqpoint{3.366090in}{2.540994in}}%
\pgfpathlineto{\pgfqpoint{3.367841in}{2.081012in}}%
\pgfpathlineto{\pgfqpoint{3.369592in}{1.691986in}}%
\pgfpathlineto{\pgfqpoint{3.370467in}{1.619612in}}%
\pgfpathlineto{\pgfqpoint{3.371342in}{1.507033in}}%
\pgfpathlineto{\pgfqpoint{3.373093in}{1.729381in}}%
\pgfpathlineto{\pgfqpoint{3.373969in}{2.028027in}}%
\pgfpathlineto{\pgfqpoint{3.374844in}{2.227458in}}%
\pgfpathlineto{\pgfqpoint{3.376595in}{2.800569in}}%
\pgfpathlineto{\pgfqpoint{3.377470in}{2.849903in}}%
\pgfpathlineto{\pgfqpoint{3.378345in}{3.063113in}}%
\pgfpathlineto{\pgfqpoint{3.379221in}{3.170864in}}%
\pgfpathlineto{\pgfqpoint{3.380096in}{3.144978in}}%
\pgfpathlineto{\pgfqpoint{3.380972in}{3.389418in}}%
\pgfpathlineto{\pgfqpoint{3.381847in}{3.357852in}}%
\pgfpathlineto{\pgfqpoint{3.382722in}{3.264820in}}%
\pgfpathlineto{\pgfqpoint{3.383598in}{3.266776in}}%
\pgfpathlineto{\pgfqpoint{3.384473in}{3.231855in}}%
\pgfpathlineto{\pgfqpoint{3.385348in}{3.117359in}}%
\pgfpathlineto{\pgfqpoint{3.386224in}{3.057581in}}%
\pgfpathlineto{\pgfqpoint{3.387099in}{2.865530in}}%
\pgfpathlineto{\pgfqpoint{3.387975in}{2.538637in}}%
\pgfpathlineto{\pgfqpoint{3.388850in}{2.413644in}}%
\pgfpathlineto{\pgfqpoint{3.390601in}{2.031806in}}%
\pgfpathlineto{\pgfqpoint{3.392352in}{1.805139in}}%
\pgfpathlineto{\pgfqpoint{3.393227in}{1.731375in}}%
\pgfpathlineto{\pgfqpoint{3.394102in}{1.718718in}}%
\pgfpathlineto{\pgfqpoint{3.394978in}{1.924114in}}%
\pgfpathlineto{\pgfqpoint{3.395853in}{2.390120in}}%
\pgfpathlineto{\pgfqpoint{3.396728in}{2.688233in}}%
\pgfpathlineto{\pgfqpoint{3.397604in}{2.725419in}}%
\pgfpathlineto{\pgfqpoint{3.401981in}{3.434050in}}%
\pgfpathlineto{\pgfqpoint{3.402856in}{3.631053in}}%
\pgfpathlineto{\pgfqpoint{3.403731in}{3.588965in}}%
\pgfpathlineto{\pgfqpoint{3.404607in}{3.507529in}}%
\pgfpathlineto{\pgfqpoint{3.406358in}{3.183692in}}%
\pgfpathlineto{\pgfqpoint{3.407233in}{3.216043in}}%
\pgfpathlineto{\pgfqpoint{3.408108in}{3.133580in}}%
\pgfpathlineto{\pgfqpoint{3.410734in}{2.571230in}}%
\pgfpathlineto{\pgfqpoint{3.412485in}{2.344985in}}%
\pgfpathlineto{\pgfqpoint{3.413361in}{2.316198in}}%
\pgfpathlineto{\pgfqpoint{3.415111in}{2.241166in}}%
\pgfpathlineto{\pgfqpoint{3.415987in}{2.341984in}}%
\pgfpathlineto{\pgfqpoint{3.416862in}{2.652933in}}%
\pgfpathlineto{\pgfqpoint{3.417737in}{2.762194in}}%
\pgfpathlineto{\pgfqpoint{3.418613in}{3.262996in}}%
\pgfpathlineto{\pgfqpoint{3.419488in}{3.294599in}}%
\pgfpathlineto{\pgfqpoint{3.420364in}{3.433727in}}%
\pgfpathlineto{\pgfqpoint{3.421239in}{3.498044in}}%
\pgfpathlineto{\pgfqpoint{3.422114in}{3.452797in}}%
\pgfpathlineto{\pgfqpoint{3.422990in}{3.549778in}}%
\pgfpathlineto{\pgfqpoint{3.423865in}{3.876829in}}%
\pgfpathlineto{\pgfqpoint{3.424740in}{3.899418in}}%
\pgfpathlineto{\pgfqpoint{3.426491in}{3.618769in}}%
\pgfpathlineto{\pgfqpoint{3.427367in}{3.739125in}}%
\pgfpathlineto{\pgfqpoint{3.428242in}{3.791684in}}%
\pgfpathlineto{\pgfqpoint{3.429117in}{3.752023in}}%
\pgfpathlineto{\pgfqpoint{3.430868in}{3.482402in}}%
\pgfpathlineto{\pgfqpoint{3.431743in}{3.282014in}}%
\pgfpathlineto{\pgfqpoint{3.432619in}{3.215802in}}%
\pgfpathlineto{\pgfqpoint{3.433494in}{3.095641in}}%
\pgfpathlineto{\pgfqpoint{3.434370in}{3.046616in}}%
\pgfpathlineto{\pgfqpoint{3.435245in}{2.960861in}}%
\pgfpathlineto{\pgfqpoint{3.436120in}{2.903952in}}%
\pgfpathlineto{\pgfqpoint{3.436996in}{2.981401in}}%
\pgfpathlineto{\pgfqpoint{3.437871in}{3.199688in}}%
\pgfpathlineto{\pgfqpoint{3.439622in}{3.437955in}}%
\pgfpathlineto{\pgfqpoint{3.440497in}{3.712133in}}%
\pgfpathlineto{\pgfqpoint{3.441373in}{3.825019in}}%
\pgfpathlineto{\pgfqpoint{3.442248in}{3.794291in}}%
\pgfpathlineto{\pgfqpoint{3.444874in}{4.185233in}}%
\pgfpathlineto{\pgfqpoint{3.445749in}{3.914189in}}%
\pgfpathlineto{\pgfqpoint{3.448376in}{3.598636in}}%
\pgfpathlineto{\pgfqpoint{3.449251in}{3.630383in}}%
\pgfpathlineto{\pgfqpoint{3.450126in}{3.642677in}}%
\pgfpathlineto{\pgfqpoint{3.451002in}{3.293583in}}%
\pgfpathlineto{\pgfqpoint{3.451877in}{3.089026in}}%
\pgfpathlineto{\pgfqpoint{3.454503in}{2.755881in}}%
\pgfpathlineto{\pgfqpoint{3.455379in}{2.652424in}}%
\pgfpathlineto{\pgfqpoint{3.456254in}{2.651246in}}%
\pgfpathlineto{\pgfqpoint{3.457129in}{2.657378in}}%
\pgfpathlineto{\pgfqpoint{3.458005in}{2.845180in}}%
\pgfpathlineto{\pgfqpoint{3.458880in}{3.094993in}}%
\pgfpathlineto{\pgfqpoint{3.459755in}{3.433825in}}%
\pgfpathlineto{\pgfqpoint{3.460631in}{3.621860in}}%
\pgfpathlineto{\pgfqpoint{3.461506in}{3.734153in}}%
\pgfpathlineto{\pgfqpoint{3.462382in}{3.785422in}}%
\pgfpathlineto{\pgfqpoint{3.464132in}{4.349083in}}%
\pgfpathlineto{\pgfqpoint{3.465008in}{4.399287in}}%
\pgfpathlineto{\pgfqpoint{3.465883in}{4.414606in}}%
\pgfpathlineto{\pgfqpoint{3.466758in}{4.269492in}}%
\pgfpathlineto{\pgfqpoint{3.468509in}{3.920695in}}%
\pgfpathlineto{\pgfqpoint{3.469385in}{3.863273in}}%
\pgfpathlineto{\pgfqpoint{3.470260in}{3.749395in}}%
\pgfpathlineto{\pgfqpoint{3.471135in}{3.459022in}}%
\pgfpathlineto{\pgfqpoint{3.472011in}{3.300862in}}%
\pgfpathlineto{\pgfqpoint{3.472886in}{3.029671in}}%
\pgfpathlineto{\pgfqpoint{3.474637in}{2.620980in}}%
\pgfpathlineto{\pgfqpoint{3.475512in}{2.576244in}}%
\pgfpathlineto{\pgfqpoint{3.476388in}{2.494234in}}%
\pgfpathlineto{\pgfqpoint{3.477263in}{2.467260in}}%
\pgfpathlineto{\pgfqpoint{3.478138in}{2.579537in}}%
\pgfpathlineto{\pgfqpoint{3.479889in}{3.133373in}}%
\pgfpathlineto{\pgfqpoint{3.480764in}{3.424032in}}%
\pgfpathlineto{\pgfqpoint{3.482515in}{3.777409in}}%
\pgfpathlineto{\pgfqpoint{3.483391in}{3.893276in}}%
\pgfpathlineto{\pgfqpoint{3.484266in}{3.875523in}}%
\pgfpathlineto{\pgfqpoint{3.486017in}{3.950917in}}%
\pgfpathlineto{\pgfqpoint{3.490394in}{3.324272in}}%
\pgfpathlineto{\pgfqpoint{3.491269in}{3.473824in}}%
\pgfpathlineto{\pgfqpoint{3.492144in}{3.342336in}}%
\pgfpathlineto{\pgfqpoint{3.494770in}{2.521722in}}%
\pgfpathlineto{\pgfqpoint{3.495646in}{2.370207in}}%
\pgfpathlineto{\pgfqpoint{3.496521in}{2.289194in}}%
\pgfpathlineto{\pgfqpoint{3.497397in}{2.312725in}}%
\pgfpathlineto{\pgfqpoint{3.498272in}{2.255605in}}%
\pgfpathlineto{\pgfqpoint{3.499147in}{2.260259in}}%
\pgfpathlineto{\pgfqpoint{3.500023in}{2.383651in}}%
\pgfpathlineto{\pgfqpoint{3.500898in}{2.727976in}}%
\pgfpathlineto{\pgfqpoint{3.503524in}{3.090199in}}%
\pgfpathlineto{\pgfqpoint{3.504400in}{3.310860in}}%
\pgfpathlineto{\pgfqpoint{3.506150in}{3.476474in}}%
\pgfpathlineto{\pgfqpoint{3.507026in}{3.458785in}}%
\pgfpathlineto{\pgfqpoint{3.507901in}{3.371334in}}%
\pgfpathlineto{\pgfqpoint{3.509652in}{2.681251in}}%
\pgfpathlineto{\pgfqpoint{3.510527in}{2.809039in}}%
\pgfpathlineto{\pgfqpoint{3.511403in}{2.722926in}}%
\pgfpathlineto{\pgfqpoint{3.512278in}{2.795512in}}%
\pgfpathlineto{\pgfqpoint{3.514904in}{2.412315in}}%
\pgfpathlineto{\pgfqpoint{3.515779in}{2.210809in}}%
\pgfpathlineto{\pgfqpoint{3.516655in}{2.136834in}}%
\pgfpathlineto{\pgfqpoint{3.517530in}{2.087779in}}%
\pgfpathlineto{\pgfqpoint{3.518406in}{2.000724in}}%
\pgfpathlineto{\pgfqpoint{3.519281in}{1.849904in}}%
\pgfpathlineto{\pgfqpoint{3.520156in}{1.905999in}}%
\pgfpathlineto{\pgfqpoint{3.521032in}{2.068814in}}%
\pgfpathlineto{\pgfqpoint{3.521907in}{2.328339in}}%
\pgfpathlineto{\pgfqpoint{3.523658in}{2.704060in}}%
\pgfpathlineto{\pgfqpoint{3.524533in}{2.908120in}}%
\pgfpathlineto{\pgfqpoint{3.525409in}{3.014978in}}%
\pgfpathlineto{\pgfqpoint{3.526284in}{3.019582in}}%
\pgfpathlineto{\pgfqpoint{3.528035in}{3.307190in}}%
\pgfpathlineto{\pgfqpoint{3.528910in}{3.272283in}}%
\pgfpathlineto{\pgfqpoint{3.529786in}{3.299133in}}%
\pgfpathlineto{\pgfqpoint{3.531536in}{3.007391in}}%
\pgfpathlineto{\pgfqpoint{3.533287in}{2.784607in}}%
\pgfpathlineto{\pgfqpoint{3.535038in}{2.469405in}}%
\pgfpathlineto{\pgfqpoint{3.535913in}{2.273789in}}%
\pgfpathlineto{\pgfqpoint{3.536789in}{2.012928in}}%
\pgfpathlineto{\pgfqpoint{3.537664in}{1.840269in}}%
\pgfpathlineto{\pgfqpoint{3.539415in}{1.768800in}}%
\pgfpathlineto{\pgfqpoint{3.540290in}{1.873465in}}%
\pgfpathlineto{\pgfqpoint{3.541165in}{1.929403in}}%
\pgfpathlineto{\pgfqpoint{3.542916in}{2.437462in}}%
\pgfpathlineto{\pgfqpoint{3.543792in}{2.913677in}}%
\pgfpathlineto{\pgfqpoint{3.544667in}{3.033416in}}%
\pgfpathlineto{\pgfqpoint{3.546418in}{3.176065in}}%
\pgfpathlineto{\pgfqpoint{3.547293in}{3.342126in}}%
\pgfpathlineto{\pgfqpoint{3.548168in}{3.406184in}}%
\pgfpathlineto{\pgfqpoint{3.549044in}{3.608331in}}%
\pgfpathlineto{\pgfqpoint{3.549919in}{3.542190in}}%
\pgfpathlineto{\pgfqpoint{3.551670in}{3.250232in}}%
\pgfpathlineto{\pgfqpoint{3.552545in}{3.076218in}}%
\pgfpathlineto{\pgfqpoint{3.553421in}{2.994903in}}%
\pgfpathlineto{\pgfqpoint{3.554296in}{2.990010in}}%
\pgfpathlineto{\pgfqpoint{3.556047in}{2.775968in}}%
\pgfpathlineto{\pgfqpoint{3.556922in}{2.525135in}}%
\pgfpathlineto{\pgfqpoint{3.558673in}{2.298800in}}%
\pgfpathlineto{\pgfqpoint{3.559548in}{2.264727in}}%
\pgfpathlineto{\pgfqpoint{3.560424in}{2.163838in}}%
\pgfpathlineto{\pgfqpoint{3.561299in}{2.170846in}}%
\pgfpathlineto{\pgfqpoint{3.562174in}{2.237617in}}%
\pgfpathlineto{\pgfqpoint{3.564801in}{3.024519in}}%
\pgfpathlineto{\pgfqpoint{3.565676in}{3.104970in}}%
\pgfpathlineto{\pgfqpoint{3.566551in}{3.261544in}}%
\pgfpathlineto{\pgfqpoint{3.567427in}{3.322046in}}%
\pgfpathlineto{\pgfqpoint{3.568302in}{3.554218in}}%
\pgfpathlineto{\pgfqpoint{3.569177in}{3.626772in}}%
\pgfpathlineto{\pgfqpoint{3.570053in}{3.638608in}}%
\pgfpathlineto{\pgfqpoint{3.570928in}{3.600108in}}%
\pgfpathlineto{\pgfqpoint{3.571804in}{3.544113in}}%
\pgfpathlineto{\pgfqpoint{3.573554in}{2.973861in}}%
\pgfpathlineto{\pgfqpoint{3.574430in}{2.856860in}}%
\pgfpathlineto{\pgfqpoint{3.575305in}{2.855411in}}%
\pgfpathlineto{\pgfqpoint{3.576180in}{2.898394in}}%
\pgfpathlineto{\pgfqpoint{3.577056in}{2.826413in}}%
\pgfpathlineto{\pgfqpoint{3.577931in}{2.595576in}}%
\pgfpathlineto{\pgfqpoint{3.580557in}{2.203831in}}%
\pgfpathlineto{\pgfqpoint{3.581433in}{2.167704in}}%
\pgfpathlineto{\pgfqpoint{3.582308in}{2.087356in}}%
\pgfpathlineto{\pgfqpoint{3.583183in}{2.191635in}}%
\pgfpathlineto{\pgfqpoint{3.585810in}{2.790899in}}%
\pgfpathlineto{\pgfqpoint{3.586685in}{2.787645in}}%
\pgfpathlineto{\pgfqpoint{3.587560in}{2.979633in}}%
\pgfpathlineto{\pgfqpoint{3.588436in}{3.050078in}}%
\pgfpathlineto{\pgfqpoint{3.589311in}{3.085286in}}%
\pgfpathlineto{\pgfqpoint{3.590186in}{3.202059in}}%
\pgfpathlineto{\pgfqpoint{3.591062in}{3.185860in}}%
\pgfpathlineto{\pgfqpoint{3.591937in}{3.180565in}}%
\pgfpathlineto{\pgfqpoint{3.592813in}{3.123072in}}%
\pgfpathlineto{\pgfqpoint{3.593688in}{3.131678in}}%
\pgfpathlineto{\pgfqpoint{3.594563in}{3.009673in}}%
\pgfpathlineto{\pgfqpoint{3.595439in}{2.962613in}}%
\pgfpathlineto{\pgfqpoint{3.596314in}{3.146055in}}%
\pgfpathlineto{\pgfqpoint{3.597189in}{3.088120in}}%
\pgfpathlineto{\pgfqpoint{3.598065in}{3.103525in}}%
\pgfpathlineto{\pgfqpoint{3.598940in}{2.953279in}}%
\pgfpathlineto{\pgfqpoint{3.599816in}{2.695619in}}%
\pgfpathlineto{\pgfqpoint{3.601566in}{2.540873in}}%
\pgfpathlineto{\pgfqpoint{3.603317in}{2.477953in}}%
\pgfpathlineto{\pgfqpoint{3.604192in}{2.528982in}}%
\pgfpathlineto{\pgfqpoint{3.605943in}{3.024254in}}%
\pgfpathlineto{\pgfqpoint{3.606819in}{3.324179in}}%
\pgfpathlineto{\pgfqpoint{3.607694in}{3.355883in}}%
\pgfpathlineto{\pgfqpoint{3.608569in}{3.665226in}}%
\pgfpathlineto{\pgfqpoint{3.609445in}{3.531777in}}%
\pgfpathlineto{\pgfqpoint{3.610320in}{3.548526in}}%
\pgfpathlineto{\pgfqpoint{3.611195in}{3.625584in}}%
\pgfpathlineto{\pgfqpoint{3.612071in}{3.675911in}}%
\pgfpathlineto{\pgfqpoint{3.613822in}{3.501655in}}%
\pgfpathlineto{\pgfqpoint{3.615572in}{3.052665in}}%
\pgfpathlineto{\pgfqpoint{3.616448in}{2.930564in}}%
\pgfpathlineto{\pgfqpoint{3.617323in}{3.170009in}}%
\pgfpathlineto{\pgfqpoint{3.618198in}{3.115879in}}%
\pgfpathlineto{\pgfqpoint{3.619949in}{2.875286in}}%
\pgfpathlineto{\pgfqpoint{3.620825in}{2.745430in}}%
\pgfpathlineto{\pgfqpoint{3.621700in}{2.733710in}}%
\pgfpathlineto{\pgfqpoint{3.622575in}{2.718032in}}%
\pgfpathlineto{\pgfqpoint{3.623451in}{2.722412in}}%
\pgfpathlineto{\pgfqpoint{3.624326in}{2.635962in}}%
\pgfpathlineto{\pgfqpoint{3.625201in}{2.402129in}}%
\pgfpathlineto{\pgfqpoint{3.626077in}{2.591387in}}%
\pgfpathlineto{\pgfqpoint{3.626952in}{2.686402in}}%
\pgfpathlineto{\pgfqpoint{3.627828in}{2.851317in}}%
\pgfpathlineto{\pgfqpoint{3.628703in}{2.945393in}}%
\pgfpathlineto{\pgfqpoint{3.629578in}{3.102176in}}%
\pgfpathlineto{\pgfqpoint{3.630454in}{3.149801in}}%
\pgfpathlineto{\pgfqpoint{3.631329in}{3.165962in}}%
\pgfpathlineto{\pgfqpoint{3.632204in}{3.229302in}}%
\pgfpathlineto{\pgfqpoint{3.633080in}{3.267055in}}%
\pgfpathlineto{\pgfqpoint{3.633955in}{3.263959in}}%
\pgfpathlineto{\pgfqpoint{3.634831in}{3.058639in}}%
\pgfpathlineto{\pgfqpoint{3.635706in}{3.025185in}}%
\pgfpathlineto{\pgfqpoint{3.636581in}{2.829750in}}%
\pgfpathlineto{\pgfqpoint{3.637457in}{2.875951in}}%
\pgfpathlineto{\pgfqpoint{3.638332in}{2.891084in}}%
\pgfpathlineto{\pgfqpoint{3.639207in}{2.834296in}}%
\pgfpathlineto{\pgfqpoint{3.640958in}{2.666652in}}%
\pgfpathlineto{\pgfqpoint{3.642709in}{2.332933in}}%
\pgfpathlineto{\pgfqpoint{3.643584in}{2.317920in}}%
\pgfpathlineto{\pgfqpoint{3.646210in}{2.032723in}}%
\pgfpathlineto{\pgfqpoint{3.647961in}{2.249167in}}%
\pgfpathlineto{\pgfqpoint{3.648837in}{2.463867in}}%
\pgfpathlineto{\pgfqpoint{3.649712in}{2.523161in}}%
\pgfpathlineto{\pgfqpoint{3.650587in}{2.690699in}}%
\pgfpathlineto{\pgfqpoint{3.651463in}{2.725482in}}%
\pgfpathlineto{\pgfqpoint{3.652338in}{2.927310in}}%
\pgfpathlineto{\pgfqpoint{3.653213in}{3.009565in}}%
\pgfpathlineto{\pgfqpoint{3.654089in}{2.854399in}}%
\pgfpathlineto{\pgfqpoint{3.654964in}{2.851147in}}%
\pgfpathlineto{\pgfqpoint{3.655840in}{2.810115in}}%
\pgfpathlineto{\pgfqpoint{3.657590in}{2.505559in}}%
\pgfpathlineto{\pgfqpoint{3.658466in}{2.522725in}}%
\pgfpathlineto{\pgfqpoint{3.659341in}{2.474479in}}%
\pgfpathlineto{\pgfqpoint{3.664593in}{1.551889in}}%
\pgfpathlineto{\pgfqpoint{3.665469in}{1.581250in}}%
\pgfpathlineto{\pgfqpoint{3.666344in}{1.565452in}}%
\pgfpathlineto{\pgfqpoint{3.667220in}{1.531405in}}%
\pgfpathlineto{\pgfqpoint{3.668970in}{1.877633in}}%
\pgfpathlineto{\pgfqpoint{3.669846in}{2.369219in}}%
\pgfpathlineto{\pgfqpoint{3.671596in}{2.553514in}}%
\pgfpathlineto{\pgfqpoint{3.672472in}{2.723233in}}%
\pgfpathlineto{\pgfqpoint{3.673347in}{2.652260in}}%
\pgfpathlineto{\pgfqpoint{3.674223in}{2.836883in}}%
\pgfpathlineto{\pgfqpoint{3.675098in}{2.827431in}}%
\pgfpathlineto{\pgfqpoint{3.675973in}{2.968795in}}%
\pgfpathlineto{\pgfqpoint{3.676849in}{2.616384in}}%
\pgfpathlineto{\pgfqpoint{3.677724in}{2.555206in}}%
\pgfpathlineto{\pgfqpoint{3.678599in}{2.299508in}}%
\pgfpathlineto{\pgfqpoint{3.679475in}{2.280434in}}%
\pgfpathlineto{\pgfqpoint{3.680350in}{2.365163in}}%
\pgfpathlineto{\pgfqpoint{3.682101in}{2.154232in}}%
\pgfpathlineto{\pgfqpoint{3.683852in}{1.824259in}}%
\pgfpathlineto{\pgfqpoint{3.684727in}{1.701531in}}%
\pgfpathlineto{\pgfqpoint{3.687353in}{1.563639in}}%
\pgfpathlineto{\pgfqpoint{3.689104in}{1.910878in}}%
\pgfpathlineto{\pgfqpoint{3.689979in}{2.124065in}}%
\pgfpathlineto{\pgfqpoint{3.690855in}{2.411298in}}%
\pgfpathlineto{\pgfqpoint{3.691730in}{2.413650in}}%
\pgfpathlineto{\pgfqpoint{3.693481in}{2.718952in}}%
\pgfpathlineto{\pgfqpoint{3.694356in}{2.733124in}}%
\pgfpathlineto{\pgfqpoint{3.695232in}{2.828674in}}%
\pgfpathlineto{\pgfqpoint{3.696107in}{2.872754in}}%
\pgfpathlineto{\pgfqpoint{3.696982in}{2.875518in}}%
\pgfpathlineto{\pgfqpoint{3.697858in}{2.714906in}}%
\pgfpathlineto{\pgfqpoint{3.699608in}{2.259049in}}%
\pgfpathlineto{\pgfqpoint{3.700484in}{2.250953in}}%
\pgfpathlineto{\pgfqpoint{3.701359in}{2.389388in}}%
\pgfpathlineto{\pgfqpoint{3.703985in}{2.186432in}}%
\pgfpathlineto{\pgfqpoint{3.704861in}{2.089893in}}%
\pgfpathlineto{\pgfqpoint{3.705736in}{1.915633in}}%
\pgfpathlineto{\pgfqpoint{3.707487in}{1.843621in}}%
\pgfpathlineto{\pgfqpoint{3.708362in}{1.860235in}}%
\pgfpathlineto{\pgfqpoint{3.709238in}{1.857185in}}%
\pgfpathlineto{\pgfqpoint{3.710988in}{2.373483in}}%
\pgfpathlineto{\pgfqpoint{3.711864in}{2.475090in}}%
\pgfpathlineto{\pgfqpoint{3.713614in}{2.762389in}}%
\pgfpathlineto{\pgfqpoint{3.714490in}{2.766606in}}%
\pgfpathlineto{\pgfqpoint{3.715365in}{2.862901in}}%
\pgfpathlineto{\pgfqpoint{3.716241in}{2.907627in}}%
\pgfpathlineto{\pgfqpoint{3.717116in}{3.090629in}}%
\pgfpathlineto{\pgfqpoint{3.717991in}{3.136740in}}%
\pgfpathlineto{\pgfqpoint{3.718867in}{3.043928in}}%
\pgfpathlineto{\pgfqpoint{3.720617in}{2.683942in}}%
\pgfpathlineto{\pgfqpoint{3.721493in}{2.667574in}}%
\pgfpathlineto{\pgfqpoint{3.722368in}{2.705678in}}%
\pgfpathlineto{\pgfqpoint{3.723244in}{2.567545in}}%
\pgfpathlineto{\pgfqpoint{3.724119in}{2.664718in}}%
\pgfpathlineto{\pgfqpoint{3.727620in}{2.062375in}}%
\pgfpathlineto{\pgfqpoint{3.728496in}{2.072978in}}%
\pgfpathlineto{\pgfqpoint{3.729371in}{2.100828in}}%
\pgfpathlineto{\pgfqpoint{3.730247in}{2.162971in}}%
\pgfpathlineto{\pgfqpoint{3.731122in}{2.175755in}}%
\pgfpathlineto{\pgfqpoint{3.732873in}{2.715905in}}%
\pgfpathlineto{\pgfqpoint{3.733748in}{2.750697in}}%
\pgfpathlineto{\pgfqpoint{3.734623in}{2.743572in}}%
\pgfpathlineto{\pgfqpoint{3.735499in}{2.786117in}}%
\pgfpathlineto{\pgfqpoint{3.736374in}{2.792582in}}%
\pgfpathlineto{\pgfqpoint{3.737250in}{2.957819in}}%
\pgfpathlineto{\pgfqpoint{3.739000in}{2.871179in}}%
\pgfpathlineto{\pgfqpoint{3.739876in}{2.735781in}}%
\pgfpathlineto{\pgfqpoint{3.740751in}{2.752749in}}%
\pgfpathlineto{\pgfqpoint{3.741626in}{2.628502in}}%
\pgfpathlineto{\pgfqpoint{3.742502in}{2.603814in}}%
\pgfpathlineto{\pgfqpoint{3.743377in}{2.683144in}}%
\pgfpathlineto{\pgfqpoint{3.744253in}{2.589807in}}%
\pgfpathlineto{\pgfqpoint{3.745128in}{2.541175in}}%
\pgfpathlineto{\pgfqpoint{3.746003in}{2.388029in}}%
\pgfpathlineto{\pgfqpoint{3.749505in}{2.093639in}}%
\pgfpathlineto{\pgfqpoint{3.750380in}{2.065879in}}%
\pgfpathlineto{\pgfqpoint{3.752131in}{2.265583in}}%
\pgfpathlineto{\pgfqpoint{3.753006in}{2.449660in}}%
\pgfpathlineto{\pgfqpoint{3.753882in}{2.765774in}}%
\pgfpathlineto{\pgfqpoint{3.754757in}{2.888836in}}%
\pgfpathlineto{\pgfqpoint{3.755632in}{2.936472in}}%
\pgfpathlineto{\pgfqpoint{3.756508in}{2.890292in}}%
\pgfpathlineto{\pgfqpoint{3.758259in}{3.076320in}}%
\pgfpathlineto{\pgfqpoint{3.759134in}{2.946374in}}%
\pgfpathlineto{\pgfqpoint{3.760009in}{3.055600in}}%
\pgfpathlineto{\pgfqpoint{3.762635in}{2.702590in}}%
\pgfpathlineto{\pgfqpoint{3.764386in}{2.867614in}}%
\pgfpathlineto{\pgfqpoint{3.765262in}{2.885768in}}%
\pgfpathlineto{\pgfqpoint{3.766137in}{2.848524in}}%
\pgfpathlineto{\pgfqpoint{3.767012in}{2.826835in}}%
\pgfpathlineto{\pgfqpoint{3.767888in}{2.740325in}}%
\pgfpathlineto{\pgfqpoint{3.769638in}{2.527401in}}%
\pgfpathlineto{\pgfqpoint{3.770514in}{2.517282in}}%
\pgfpathlineto{\pgfqpoint{3.771389in}{2.503810in}}%
\pgfpathlineto{\pgfqpoint{3.772265in}{2.594831in}}%
\pgfpathlineto{\pgfqpoint{3.773140in}{2.764003in}}%
\pgfpathlineto{\pgfqpoint{3.774015in}{2.727586in}}%
\pgfpathlineto{\pgfqpoint{3.775766in}{2.943641in}}%
\pgfpathlineto{\pgfqpoint{3.776641in}{2.917277in}}%
\pgfpathlineto{\pgfqpoint{3.777517in}{2.774821in}}%
\pgfpathlineto{\pgfqpoint{3.778392in}{2.767678in}}%
\pgfpathlineto{\pgfqpoint{3.779268in}{2.882446in}}%
\pgfpathlineto{\pgfqpoint{3.780143in}{2.776939in}}%
\pgfpathlineto{\pgfqpoint{3.781018in}{2.919423in}}%
\pgfpathlineto{\pgfqpoint{3.781894in}{2.921778in}}%
\pgfpathlineto{\pgfqpoint{3.782769in}{2.893404in}}%
\pgfpathlineto{\pgfqpoint{3.783644in}{2.749912in}}%
\pgfpathlineto{\pgfqpoint{3.784520in}{2.781740in}}%
\pgfpathlineto{\pgfqpoint{3.785395in}{2.838858in}}%
\pgfpathlineto{\pgfqpoint{3.786271in}{2.862932in}}%
\pgfpathlineto{\pgfqpoint{3.787146in}{2.902079in}}%
\pgfpathlineto{\pgfqpoint{3.789772in}{2.528126in}}%
\pgfpathlineto{\pgfqpoint{3.791523in}{2.477107in}}%
\pgfpathlineto{\pgfqpoint{3.792398in}{2.501091in}}%
\pgfpathlineto{\pgfqpoint{3.793274in}{2.585363in}}%
\pgfpathlineto{\pgfqpoint{3.794149in}{2.840666in}}%
\pgfpathlineto{\pgfqpoint{3.795024in}{2.873790in}}%
\pgfpathlineto{\pgfqpoint{3.795900in}{2.876191in}}%
\pgfpathlineto{\pgfqpoint{3.796775in}{2.956560in}}%
\pgfpathlineto{\pgfqpoint{3.797651in}{2.995692in}}%
\pgfpathlineto{\pgfqpoint{3.798526in}{3.016304in}}%
\pgfpathlineto{\pgfqpoint{3.799401in}{3.159938in}}%
\pgfpathlineto{\pgfqpoint{3.800277in}{3.145635in}}%
\pgfpathlineto{\pgfqpoint{3.802027in}{3.281585in}}%
\pgfpathlineto{\pgfqpoint{3.802903in}{3.225582in}}%
\pgfpathlineto{\pgfqpoint{3.803778in}{3.009032in}}%
\pgfpathlineto{\pgfqpoint{3.804654in}{2.886725in}}%
\pgfpathlineto{\pgfqpoint{3.806404in}{3.058064in}}%
\pgfpathlineto{\pgfqpoint{3.807280in}{3.083891in}}%
\pgfpathlineto{\pgfqpoint{3.808155in}{3.051812in}}%
\pgfpathlineto{\pgfqpoint{3.809030in}{2.944761in}}%
\pgfpathlineto{\pgfqpoint{3.809906in}{2.770289in}}%
\pgfpathlineto{\pgfqpoint{3.810781in}{2.665776in}}%
\pgfpathlineto{\pgfqpoint{3.811657in}{2.619620in}}%
\pgfpathlineto{\pgfqpoint{3.812532in}{2.639496in}}%
\pgfpathlineto{\pgfqpoint{3.813407in}{2.610589in}}%
\pgfpathlineto{\pgfqpoint{3.814283in}{2.506445in}}%
\pgfpathlineto{\pgfqpoint{3.815158in}{2.708186in}}%
\pgfpathlineto{\pgfqpoint{3.816033in}{2.716864in}}%
\pgfpathlineto{\pgfqpoint{3.817784in}{2.922077in}}%
\pgfpathlineto{\pgfqpoint{3.818660in}{2.964556in}}%
\pgfpathlineto{\pgfqpoint{3.819535in}{3.055542in}}%
\pgfpathlineto{\pgfqpoint{3.820410in}{3.039574in}}%
\pgfpathlineto{\pgfqpoint{3.821286in}{2.940557in}}%
\pgfpathlineto{\pgfqpoint{3.822161in}{3.005181in}}%
\pgfpathlineto{\pgfqpoint{3.823036in}{3.101964in}}%
\pgfpathlineto{\pgfqpoint{3.823912in}{3.133315in}}%
\pgfpathlineto{\pgfqpoint{3.826538in}{2.774126in}}%
\pgfpathlineto{\pgfqpoint{3.827413in}{2.861663in}}%
\pgfpathlineto{\pgfqpoint{3.828289in}{2.819012in}}%
\pgfpathlineto{\pgfqpoint{3.829164in}{2.903439in}}%
\pgfpathlineto{\pgfqpoint{3.830915in}{2.683235in}}%
\pgfpathlineto{\pgfqpoint{3.831790in}{2.611827in}}%
\pgfpathlineto{\pgfqpoint{3.832666in}{2.563105in}}%
\pgfpathlineto{\pgfqpoint{3.833541in}{2.464390in}}%
\pgfpathlineto{\pgfqpoint{3.834416in}{2.483179in}}%
\pgfpathlineto{\pgfqpoint{3.835292in}{2.700159in}}%
\pgfpathlineto{\pgfqpoint{3.837042in}{3.283798in}}%
\pgfpathlineto{\pgfqpoint{3.837918in}{3.365954in}}%
\pgfpathlineto{\pgfqpoint{3.838793in}{3.398288in}}%
\pgfpathlineto{\pgfqpoint{3.839669in}{3.354237in}}%
\pgfpathlineto{\pgfqpoint{3.840544in}{3.444196in}}%
\pgfpathlineto{\pgfqpoint{3.841419in}{3.488827in}}%
\pgfpathlineto{\pgfqpoint{3.842295in}{3.443011in}}%
\pgfpathlineto{\pgfqpoint{3.843170in}{3.544266in}}%
\pgfpathlineto{\pgfqpoint{3.844045in}{3.433721in}}%
\pgfpathlineto{\pgfqpoint{3.844921in}{3.431541in}}%
\pgfpathlineto{\pgfqpoint{3.845796in}{3.248732in}}%
\pgfpathlineto{\pgfqpoint{3.846672in}{3.273406in}}%
\pgfpathlineto{\pgfqpoint{3.847547in}{3.352394in}}%
\pgfpathlineto{\pgfqpoint{3.848422in}{3.474730in}}%
\pgfpathlineto{\pgfqpoint{3.849298in}{3.300409in}}%
\pgfpathlineto{\pgfqpoint{3.850173in}{3.218732in}}%
\pgfpathlineto{\pgfqpoint{3.851048in}{3.001911in}}%
\pgfpathlineto{\pgfqpoint{3.851924in}{2.983787in}}%
\pgfpathlineto{\pgfqpoint{3.852799in}{3.015866in}}%
\pgfpathlineto{\pgfqpoint{3.853675in}{2.954094in}}%
\pgfpathlineto{\pgfqpoint{3.854550in}{2.920717in}}%
\pgfpathlineto{\pgfqpoint{3.855425in}{2.870514in}}%
\pgfpathlineto{\pgfqpoint{3.856301in}{3.018278in}}%
\pgfpathlineto{\pgfqpoint{3.857176in}{3.353030in}}%
\pgfpathlineto{\pgfqpoint{3.858051in}{3.505862in}}%
\pgfpathlineto{\pgfqpoint{3.858927in}{3.596131in}}%
\pgfpathlineto{\pgfqpoint{3.860678in}{3.433236in}}%
\pgfpathlineto{\pgfqpoint{3.861553in}{3.672752in}}%
\pgfpathlineto{\pgfqpoint{3.863304in}{3.763064in}}%
\pgfpathlineto{\pgfqpoint{3.864179in}{3.177439in}}%
\pgfpathlineto{\pgfqpoint{3.865930in}{2.984416in}}%
\pgfpathlineto{\pgfqpoint{3.866805in}{2.933484in}}%
\pgfpathlineto{\pgfqpoint{3.867681in}{2.758963in}}%
\pgfpathlineto{\pgfqpoint{3.868556in}{2.710297in}}%
\pgfpathlineto{\pgfqpoint{3.869431in}{2.854535in}}%
\pgfpathlineto{\pgfqpoint{3.871182in}{2.693052in}}%
\pgfpathlineto{\pgfqpoint{3.872057in}{2.525165in}}%
\pgfpathlineto{\pgfqpoint{3.872933in}{2.287744in}}%
\pgfpathlineto{\pgfqpoint{3.873808in}{2.272188in}}%
\pgfpathlineto{\pgfqpoint{3.874684in}{2.291641in}}%
\pgfpathlineto{\pgfqpoint{3.875559in}{2.278471in}}%
\pgfpathlineto{\pgfqpoint{3.876434in}{2.250590in}}%
\pgfpathlineto{\pgfqpoint{3.877310in}{2.305576in}}%
\pgfpathlineto{\pgfqpoint{3.878185in}{2.428400in}}%
\pgfpathlineto{\pgfqpoint{3.879060in}{2.427185in}}%
\pgfpathlineto{\pgfqpoint{3.879936in}{2.713466in}}%
\pgfpathlineto{\pgfqpoint{3.880811in}{2.334767in}}%
\pgfpathlineto{\pgfqpoint{3.881687in}{2.415658in}}%
\pgfpathlineto{\pgfqpoint{3.883437in}{2.636792in}}%
\pgfpathlineto{\pgfqpoint{3.884313in}{2.737146in}}%
\pgfpathlineto{\pgfqpoint{3.885188in}{2.577818in}}%
\pgfpathlineto{\pgfqpoint{3.886063in}{2.584909in}}%
\pgfpathlineto{\pgfqpoint{3.886939in}{2.637539in}}%
\pgfpathlineto{\pgfqpoint{3.887814in}{2.178193in}}%
\pgfpathlineto{\pgfqpoint{3.888690in}{2.048994in}}%
\pgfpathlineto{\pgfqpoint{3.889565in}{2.034335in}}%
\pgfpathlineto{\pgfqpoint{3.890440in}{2.045973in}}%
\pgfpathlineto{\pgfqpoint{3.891316in}{2.035703in}}%
\pgfpathlineto{\pgfqpoint{3.892191in}{1.953270in}}%
\pgfpathlineto{\pgfqpoint{3.893942in}{1.857849in}}%
\pgfpathlineto{\pgfqpoint{3.894817in}{1.768952in}}%
\pgfpathlineto{\pgfqpoint{3.895693in}{1.763605in}}%
\pgfpathlineto{\pgfqpoint{3.897443in}{1.659273in}}%
\pgfpathlineto{\pgfqpoint{3.898319in}{1.686442in}}%
\pgfpathlineto{\pgfqpoint{3.899194in}{1.887919in}}%
\pgfpathlineto{\pgfqpoint{3.900069in}{1.891483in}}%
\pgfpathlineto{\pgfqpoint{3.900945in}{2.114882in}}%
\pgfpathlineto{\pgfqpoint{3.901820in}{2.229335in}}%
\pgfpathlineto{\pgfqpoint{3.902696in}{2.227906in}}%
\pgfpathlineto{\pgfqpoint{3.903571in}{2.106956in}}%
\pgfpathlineto{\pgfqpoint{3.904446in}{2.166408in}}%
\pgfpathlineto{\pgfqpoint{3.905322in}{2.289833in}}%
\pgfpathlineto{\pgfqpoint{3.906197in}{2.480165in}}%
\pgfpathlineto{\pgfqpoint{3.907072in}{2.509364in}}%
\pgfpathlineto{\pgfqpoint{3.907948in}{2.461877in}}%
\pgfpathlineto{\pgfqpoint{3.908823in}{2.106406in}}%
\pgfpathlineto{\pgfqpoint{3.909699in}{1.882110in}}%
\pgfpathlineto{\pgfqpoint{3.910574in}{1.863618in}}%
\pgfpathlineto{\pgfqpoint{3.911449in}{1.929015in}}%
\pgfpathlineto{\pgfqpoint{3.912325in}{1.801544in}}%
\pgfpathlineto{\pgfqpoint{3.913200in}{1.744968in}}%
\pgfpathlineto{\pgfqpoint{3.914075in}{1.728838in}}%
\pgfpathlineto{\pgfqpoint{3.918452in}{1.523858in}}%
\pgfpathlineto{\pgfqpoint{3.919328in}{1.574034in}}%
\pgfpathlineto{\pgfqpoint{3.921078in}{1.869771in}}%
\pgfpathlineto{\pgfqpoint{3.921954in}{2.029091in}}%
\pgfpathlineto{\pgfqpoint{3.922829in}{2.126436in}}%
\pgfpathlineto{\pgfqpoint{3.923705in}{2.270652in}}%
\pgfpathlineto{\pgfqpoint{3.924580in}{2.368461in}}%
\pgfpathlineto{\pgfqpoint{3.925455in}{2.528686in}}%
\pgfpathlineto{\pgfqpoint{3.926331in}{2.850126in}}%
\pgfpathlineto{\pgfqpoint{3.927206in}{2.898507in}}%
\pgfpathlineto{\pgfqpoint{3.928082in}{2.864524in}}%
\pgfpathlineto{\pgfqpoint{3.928957in}{2.643514in}}%
\pgfpathlineto{\pgfqpoint{3.929832in}{2.555844in}}%
\pgfpathlineto{\pgfqpoint{3.930708in}{2.295728in}}%
\pgfpathlineto{\pgfqpoint{3.931583in}{2.295326in}}%
\pgfpathlineto{\pgfqpoint{3.932458in}{2.366734in}}%
\pgfpathlineto{\pgfqpoint{3.934209in}{2.167251in}}%
\pgfpathlineto{\pgfqpoint{3.935085in}{2.011870in}}%
\pgfpathlineto{\pgfqpoint{3.935960in}{1.951156in}}%
\pgfpathlineto{\pgfqpoint{3.938586in}{1.847065in}}%
\pgfpathlineto{\pgfqpoint{3.939461in}{1.800064in}}%
\pgfpathlineto{\pgfqpoint{3.940337in}{1.720771in}}%
\pgfpathlineto{\pgfqpoint{3.942088in}{2.189537in}}%
\pgfpathlineto{\pgfqpoint{3.943838in}{2.353696in}}%
\pgfpathlineto{\pgfqpoint{3.945589in}{2.432784in}}%
\pgfpathlineto{\pgfqpoint{3.946464in}{2.630877in}}%
\pgfpathlineto{\pgfqpoint{3.947340in}{2.562016in}}%
\pgfpathlineto{\pgfqpoint{3.948215in}{2.675533in}}%
\pgfpathlineto{\pgfqpoint{3.949966in}{2.575983in}}%
\pgfpathlineto{\pgfqpoint{3.951717in}{2.410706in}}%
\pgfpathlineto{\pgfqpoint{3.952592in}{2.433324in}}%
\pgfpathlineto{\pgfqpoint{3.953467in}{2.496173in}}%
\pgfpathlineto{\pgfqpoint{3.955218in}{2.319755in}}%
\pgfpathlineto{\pgfqpoint{3.956094in}{2.177824in}}%
\pgfpathlineto{\pgfqpoint{3.956969in}{2.089380in}}%
\pgfpathlineto{\pgfqpoint{3.957844in}{1.889203in}}%
\pgfpathlineto{\pgfqpoint{3.958720in}{1.799007in}}%
\pgfpathlineto{\pgfqpoint{3.959595in}{1.760554in}}%
\pgfpathlineto{\pgfqpoint{3.960470in}{1.739682in}}%
\pgfpathlineto{\pgfqpoint{3.961346in}{1.680847in}}%
\pgfpathlineto{\pgfqpoint{3.963972in}{2.510453in}}%
\pgfpathlineto{\pgfqpoint{3.964847in}{2.515750in}}%
\pgfpathlineto{\pgfqpoint{3.965723in}{2.515863in}}%
\pgfpathlineto{\pgfqpoint{3.966598in}{2.533533in}}%
\pgfpathlineto{\pgfqpoint{3.967473in}{2.614519in}}%
\pgfpathlineto{\pgfqpoint{3.968349in}{2.654948in}}%
\pgfpathlineto{\pgfqpoint{3.969224in}{2.736828in}}%
\pgfpathlineto{\pgfqpoint{3.970975in}{2.652631in}}%
\pgfpathlineto{\pgfqpoint{3.971850in}{2.472989in}}%
\pgfpathlineto{\pgfqpoint{3.972726in}{2.355921in}}%
\pgfpathlineto{\pgfqpoint{3.974476in}{2.461219in}}%
\pgfpathlineto{\pgfqpoint{3.975352in}{2.545110in}}%
\pgfpathlineto{\pgfqpoint{3.976227in}{2.523129in}}%
\pgfpathlineto{\pgfqpoint{3.977103in}{2.420527in}}%
\pgfpathlineto{\pgfqpoint{3.977978in}{2.390299in}}%
\pgfpathlineto{\pgfqpoint{3.978853in}{2.307905in}}%
\pgfpathlineto{\pgfqpoint{3.979729in}{2.321325in}}%
\pgfpathlineto{\pgfqpoint{3.980604in}{2.284391in}}%
\pgfpathlineto{\pgfqpoint{3.981479in}{2.333183in}}%
\pgfpathlineto{\pgfqpoint{3.982355in}{2.399587in}}%
\pgfpathlineto{\pgfqpoint{3.984106in}{2.973018in}}%
\pgfpathlineto{\pgfqpoint{3.985856in}{3.310977in}}%
\pgfpathlineto{\pgfqpoint{3.986732in}{3.597117in}}%
\pgfpathlineto{\pgfqpoint{3.987607in}{3.576362in}}%
\pgfpathlineto{\pgfqpoint{3.989358in}{3.720812in}}%
\pgfpathlineto{\pgfqpoint{3.990233in}{3.889355in}}%
\pgfpathlineto{\pgfqpoint{3.991109in}{3.777422in}}%
\pgfpathlineto{\pgfqpoint{3.991984in}{3.744284in}}%
\pgfpathlineto{\pgfqpoint{3.993735in}{3.385943in}}%
\pgfpathlineto{\pgfqpoint{3.994610in}{3.375314in}}%
\pgfpathlineto{\pgfqpoint{3.995485in}{3.534448in}}%
\pgfpathlineto{\pgfqpoint{3.996361in}{3.498351in}}%
\pgfpathlineto{\pgfqpoint{3.997236in}{3.660438in}}%
\pgfpathlineto{\pgfqpoint{3.998112in}{3.566919in}}%
\pgfpathlineto{\pgfqpoint{3.999862in}{3.269629in}}%
\pgfpathlineto{\pgfqpoint{4.000738in}{3.173181in}}%
\pgfpathlineto{\pgfqpoint{4.001613in}{3.154996in}}%
\pgfpathlineto{\pgfqpoint{4.002488in}{3.204142in}}%
\pgfpathlineto{\pgfqpoint{4.003364in}{3.119172in}}%
\pgfpathlineto{\pgfqpoint{4.005990in}{3.818849in}}%
\pgfpathlineto{\pgfqpoint{4.006865in}{3.774555in}}%
\pgfpathlineto{\pgfqpoint{4.008616in}{3.862049in}}%
\pgfpathlineto{\pgfqpoint{4.009491in}{4.065783in}}%
\pgfpathlineto{\pgfqpoint{4.010367in}{4.067057in}}%
\pgfpathlineto{\pgfqpoint{4.011242in}{3.931821in}}%
\pgfpathlineto{\pgfqpoint{4.012118in}{4.006346in}}%
\pgfpathlineto{\pgfqpoint{4.012993in}{4.134537in}}%
\pgfpathlineto{\pgfqpoint{4.013868in}{3.882751in}}%
\pgfpathlineto{\pgfqpoint{4.015619in}{3.626840in}}%
\pgfpathlineto{\pgfqpoint{4.016494in}{3.581902in}}%
\pgfpathlineto{\pgfqpoint{4.018245in}{3.330646in}}%
\pgfpathlineto{\pgfqpoint{4.019121in}{3.165871in}}%
\pgfpathlineto{\pgfqpoint{4.019996in}{3.194083in}}%
\pgfpathlineto{\pgfqpoint{4.020871in}{3.056796in}}%
\pgfpathlineto{\pgfqpoint{4.022622in}{2.881086in}}%
\pgfpathlineto{\pgfqpoint{4.023497in}{2.854384in}}%
\pgfpathlineto{\pgfqpoint{4.024373in}{2.699334in}}%
\pgfpathlineto{\pgfqpoint{4.026124in}{3.092468in}}%
\pgfpathlineto{\pgfqpoint{4.026999in}{3.376599in}}%
\pgfpathlineto{\pgfqpoint{4.028750in}{3.602019in}}%
\pgfpathlineto{\pgfqpoint{4.029625in}{3.656684in}}%
\pgfpathlineto{\pgfqpoint{4.030500in}{3.613676in}}%
\pgfpathlineto{\pgfqpoint{4.031376in}{3.619716in}}%
\pgfpathlineto{\pgfqpoint{4.032251in}{3.736304in}}%
\pgfpathlineto{\pgfqpoint{4.033127in}{3.733061in}}%
\pgfpathlineto{\pgfqpoint{4.034877in}{3.320390in}}%
\pgfpathlineto{\pgfqpoint{4.035753in}{3.315865in}}%
\pgfpathlineto{\pgfqpoint{4.036628in}{3.386618in}}%
\pgfpathlineto{\pgfqpoint{4.037503in}{3.346655in}}%
\pgfpathlineto{\pgfqpoint{4.038379in}{3.140195in}}%
\pgfpathlineto{\pgfqpoint{4.039254in}{3.300711in}}%
\pgfpathlineto{\pgfqpoint{4.041880in}{2.974574in}}%
\pgfpathlineto{\pgfqpoint{4.044506in}{2.754884in}}%
\pgfpathlineto{\pgfqpoint{4.045382in}{2.780886in}}%
\pgfpathlineto{\pgfqpoint{4.046257in}{3.089697in}}%
\pgfpathlineto{\pgfqpoint{4.047133in}{3.519299in}}%
\pgfpathlineto{\pgfqpoint{4.048008in}{3.426638in}}%
\pgfpathlineto{\pgfqpoint{4.049759in}{3.854129in}}%
\pgfpathlineto{\pgfqpoint{4.051509in}{3.693991in}}%
\pgfpathlineto{\pgfqpoint{4.052385in}{3.548530in}}%
\pgfpathlineto{\pgfqpoint{4.053260in}{3.727511in}}%
\pgfpathlineto{\pgfqpoint{4.054136in}{3.778131in}}%
\pgfpathlineto{\pgfqpoint{4.055011in}{3.428366in}}%
\pgfpathlineto{\pgfqpoint{4.055886in}{3.346588in}}%
\pgfpathlineto{\pgfqpoint{4.056762in}{3.327550in}}%
\pgfpathlineto{\pgfqpoint{4.057637in}{3.274837in}}%
\pgfpathlineto{\pgfqpoint{4.058513in}{3.423349in}}%
\pgfpathlineto{\pgfqpoint{4.060263in}{3.188163in}}%
\pgfpathlineto{\pgfqpoint{4.062014in}{2.822727in}}%
\pgfpathlineto{\pgfqpoint{4.062889in}{2.726098in}}%
\pgfpathlineto{\pgfqpoint{4.064640in}{2.612311in}}%
\pgfpathlineto{\pgfqpoint{4.065516in}{2.602252in}}%
\pgfpathlineto{\pgfqpoint{4.066391in}{2.598890in}}%
\pgfpathlineto{\pgfqpoint{4.067266in}{2.827412in}}%
\pgfpathlineto{\pgfqpoint{4.068142in}{2.982659in}}%
\pgfpathlineto{\pgfqpoint{4.069892in}{3.564746in}}%
\pgfpathlineto{\pgfqpoint{4.070768in}{3.604083in}}%
\pgfpathlineto{\pgfqpoint{4.071643in}{3.867553in}}%
\pgfpathlineto{\pgfqpoint{4.072519in}{3.675681in}}%
\pgfpathlineto{\pgfqpoint{4.073394in}{3.786305in}}%
\pgfpathlineto{\pgfqpoint{4.074269in}{3.975376in}}%
\pgfpathlineto{\pgfqpoint{4.075145in}{3.933631in}}%
\pgfpathlineto{\pgfqpoint{4.076020in}{3.628332in}}%
\pgfpathlineto{\pgfqpoint{4.077771in}{3.273017in}}%
\pgfpathlineto{\pgfqpoint{4.078646in}{3.301237in}}%
\pgfpathlineto{\pgfqpoint{4.079522in}{3.364084in}}%
\pgfpathlineto{\pgfqpoint{4.083898in}{2.675955in}}%
\pgfpathlineto{\pgfqpoint{4.084774in}{2.625027in}}%
\pgfpathlineto{\pgfqpoint{4.085649in}{2.529183in}}%
\pgfpathlineto{\pgfqpoint{4.086525in}{2.519849in}}%
\pgfpathlineto{\pgfqpoint{4.087400in}{2.580225in}}%
\pgfpathlineto{\pgfqpoint{4.090026in}{3.487772in}}%
\pgfpathlineto{\pgfqpoint{4.090901in}{3.479279in}}%
\pgfpathlineto{\pgfqpoint{4.091777in}{3.411123in}}%
\pgfpathlineto{\pgfqpoint{4.092652in}{3.405547in}}%
\pgfpathlineto{\pgfqpoint{4.094403in}{3.611349in}}%
\pgfpathlineto{\pgfqpoint{4.095278in}{3.594965in}}%
\pgfpathlineto{\pgfqpoint{4.096154in}{3.693632in}}%
\pgfpathlineto{\pgfqpoint{4.097029in}{3.566328in}}%
\pgfpathlineto{\pgfqpoint{4.097904in}{3.491649in}}%
\pgfpathlineto{\pgfqpoint{4.098780in}{3.280790in}}%
\pgfpathlineto{\pgfqpoint{4.099655in}{3.236357in}}%
\pgfpathlineto{\pgfqpoint{4.100531in}{3.322762in}}%
\pgfpathlineto{\pgfqpoint{4.101406in}{3.336113in}}%
\pgfpathlineto{\pgfqpoint{4.102281in}{3.269992in}}%
\pgfpathlineto{\pgfqpoint{4.104032in}{2.897820in}}%
\pgfpathlineto{\pgfqpoint{4.104907in}{2.809346in}}%
\pgfpathlineto{\pgfqpoint{4.105783in}{2.757301in}}%
\pgfpathlineto{\pgfqpoint{4.107534in}{2.540601in}}%
\pgfpathlineto{\pgfqpoint{4.108409in}{2.477364in}}%
\pgfpathlineto{\pgfqpoint{4.109284in}{2.817620in}}%
\pgfpathlineto{\pgfqpoint{4.111035in}{3.162058in}}%
\pgfpathlineto{\pgfqpoint{4.111910in}{3.163527in}}%
\pgfpathlineto{\pgfqpoint{4.112786in}{3.273663in}}%
\pgfpathlineto{\pgfqpoint{4.113661in}{3.274922in}}%
\pgfpathlineto{\pgfqpoint{4.114537in}{3.379137in}}%
\pgfpathlineto{\pgfqpoint{4.115412in}{3.646028in}}%
\pgfpathlineto{\pgfqpoint{4.116287in}{3.570244in}}%
\pgfpathlineto{\pgfqpoint{4.117163in}{3.665432in}}%
\pgfpathlineto{\pgfqpoint{4.118038in}{3.555481in}}%
\pgfpathlineto{\pgfqpoint{4.119789in}{3.246678in}}%
\pgfpathlineto{\pgfqpoint{4.120664in}{3.164899in}}%
\pgfpathlineto{\pgfqpoint{4.121540in}{3.370579in}}%
\pgfpathlineto{\pgfqpoint{4.122415in}{3.202269in}}%
\pgfpathlineto{\pgfqpoint{4.123290in}{3.206075in}}%
\pgfpathlineto{\pgfqpoint{4.126792in}{2.688793in}}%
\pgfpathlineto{\pgfqpoint{4.127667in}{2.750383in}}%
\pgfpathlineto{\pgfqpoint{4.128543in}{2.693898in}}%
\pgfpathlineto{\pgfqpoint{4.129418in}{2.858297in}}%
\pgfpathlineto{\pgfqpoint{4.132044in}{3.950914in}}%
\pgfpathlineto{\pgfqpoint{4.132919in}{3.867207in}}%
\pgfpathlineto{\pgfqpoint{4.133795in}{4.034599in}}%
\pgfpathlineto{\pgfqpoint{4.134670in}{4.077814in}}%
\pgfpathlineto{\pgfqpoint{4.135546in}{4.060515in}}%
\pgfpathlineto{\pgfqpoint{4.136421in}{4.149206in}}%
\pgfpathlineto{\pgfqpoint{4.137296in}{4.155464in}}%
\pgfpathlineto{\pgfqpoint{4.138172in}{4.125051in}}%
\pgfpathlineto{\pgfqpoint{4.139922in}{3.752904in}}%
\pgfpathlineto{\pgfqpoint{4.140798in}{3.659095in}}%
\pgfpathlineto{\pgfqpoint{4.141673in}{3.720548in}}%
\pgfpathlineto{\pgfqpoint{4.142549in}{4.090515in}}%
\pgfpathlineto{\pgfqpoint{4.143424in}{4.019741in}}%
\pgfpathlineto{\pgfqpoint{4.145175in}{3.778454in}}%
\pgfpathlineto{\pgfqpoint{4.146050in}{3.559881in}}%
\pgfpathlineto{\pgfqpoint{4.146925in}{3.475334in}}%
\pgfpathlineto{\pgfqpoint{4.147801in}{3.347894in}}%
\pgfpathlineto{\pgfqpoint{4.148676in}{3.029912in}}%
\pgfpathlineto{\pgfqpoint{4.149552in}{2.969198in}}%
\pgfpathlineto{\pgfqpoint{4.150427in}{2.961672in}}%
\pgfpathlineto{\pgfqpoint{4.152178in}{3.666071in}}%
\pgfpathlineto{\pgfqpoint{4.153053in}{3.957783in}}%
\pgfpathlineto{\pgfqpoint{4.153928in}{4.091871in}}%
\pgfpathlineto{\pgfqpoint{4.154804in}{4.295794in}}%
\pgfpathlineto{\pgfqpoint{4.155679in}{4.369021in}}%
\pgfpathlineto{\pgfqpoint{4.156555in}{4.321622in}}%
\pgfpathlineto{\pgfqpoint{4.157430in}{4.453244in}}%
\pgfpathlineto{\pgfqpoint{4.158305in}{4.519467in}}%
\pgfpathlineto{\pgfqpoint{4.159181in}{4.475909in}}%
\pgfpathlineto{\pgfqpoint{4.160056in}{4.144774in}}%
\pgfpathlineto{\pgfqpoint{4.161807in}{3.759851in}}%
\pgfpathlineto{\pgfqpoint{4.162682in}{3.705884in}}%
\pgfpathlineto{\pgfqpoint{4.163558in}{3.836631in}}%
\pgfpathlineto{\pgfqpoint{4.164433in}{3.841011in}}%
\pgfpathlineto{\pgfqpoint{4.168810in}{3.259057in}}%
\pgfpathlineto{\pgfqpoint{4.169685in}{3.174510in}}%
\pgfpathlineto{\pgfqpoint{4.170561in}{3.164662in}}%
\pgfpathlineto{\pgfqpoint{4.171436in}{3.287005in}}%
\pgfpathlineto{\pgfqpoint{4.173187in}{3.790848in}}%
\pgfpathlineto{\pgfqpoint{4.174062in}{3.887615in}}%
\pgfpathlineto{\pgfqpoint{4.174937in}{3.845506in}}%
\pgfpathlineto{\pgfqpoint{4.175813in}{4.054455in}}%
\pgfpathlineto{\pgfqpoint{4.176688in}{4.041626in}}%
\pgfpathlineto{\pgfqpoint{4.179314in}{3.768724in}}%
\pgfpathlineto{\pgfqpoint{4.180190in}{3.863697in}}%
\pgfpathlineto{\pgfqpoint{4.181940in}{3.534113in}}%
\pgfpathlineto{\pgfqpoint{4.182816in}{3.443375in}}%
\pgfpathlineto{\pgfqpoint{4.183691in}{3.400907in}}%
\pgfpathlineto{\pgfqpoint{4.184567in}{3.328139in}}%
\pgfpathlineto{\pgfqpoint{4.185442in}{3.302645in}}%
\pgfpathlineto{\pgfqpoint{4.186317in}{3.222991in}}%
\pgfpathlineto{\pgfqpoint{4.187193in}{3.005898in}}%
\pgfpathlineto{\pgfqpoint{4.188068in}{2.896854in}}%
\pgfpathlineto{\pgfqpoint{4.188943in}{2.870423in}}%
\pgfpathlineto{\pgfqpoint{4.189819in}{2.778536in}}%
\pgfpathlineto{\pgfqpoint{4.190694in}{2.754945in}}%
\pgfpathlineto{\pgfqpoint{4.191570in}{2.683325in}}%
\pgfpathlineto{\pgfqpoint{4.192445in}{2.789989in}}%
\pgfpathlineto{\pgfqpoint{4.193320in}{3.115817in}}%
\pgfpathlineto{\pgfqpoint{4.195071in}{3.476329in}}%
\pgfpathlineto{\pgfqpoint{4.195947in}{3.509489in}}%
\pgfpathlineto{\pgfqpoint{4.196822in}{3.715192in}}%
\pgfpathlineto{\pgfqpoint{4.197697in}{3.720653in}}%
\pgfpathlineto{\pgfqpoint{4.198573in}{3.624131in}}%
\pgfpathlineto{\pgfqpoint{4.199448in}{3.761838in}}%
\pgfpathlineto{\pgfqpoint{4.200323in}{3.729690in}}%
\pgfpathlineto{\pgfqpoint{4.201199in}{3.858414in}}%
\pgfpathlineto{\pgfqpoint{4.202074in}{3.563128in}}%
\pgfpathlineto{\pgfqpoint{4.202950in}{3.152265in}}%
\pgfpathlineto{\pgfqpoint{4.203825in}{3.460927in}}%
\pgfpathlineto{\pgfqpoint{4.204700in}{3.007299in}}%
\pgfpathlineto{\pgfqpoint{4.206451in}{3.259903in}}%
\pgfpathlineto{\pgfqpoint{4.208202in}{3.165931in}}%
\pgfpathlineto{\pgfqpoint{4.209077in}{3.064559in}}%
\pgfpathlineto{\pgfqpoint{4.209953in}{3.000824in}}%
\pgfpathlineto{\pgfqpoint{4.210828in}{2.960347in}}%
\pgfpathlineto{\pgfqpoint{4.211703in}{2.955363in}}%
\pgfpathlineto{\pgfqpoint{4.212579in}{2.937572in}}%
\pgfpathlineto{\pgfqpoint{4.213454in}{3.064341in}}%
\pgfpathlineto{\pgfqpoint{4.214329in}{3.470410in}}%
\pgfpathlineto{\pgfqpoint{4.216956in}{3.943233in}}%
\pgfpathlineto{\pgfqpoint{4.217831in}{4.017074in}}%
\pgfpathlineto{\pgfqpoint{4.218706in}{3.931228in}}%
\pgfpathlineto{\pgfqpoint{4.219582in}{3.995517in}}%
\pgfpathlineto{\pgfqpoint{4.220457in}{3.996533in}}%
\pgfpathlineto{\pgfqpoint{4.221332in}{4.070882in}}%
\pgfpathlineto{\pgfqpoint{4.225709in}{3.516120in}}%
\pgfpathlineto{\pgfqpoint{4.226585in}{3.539129in}}%
\pgfpathlineto{\pgfqpoint{4.227460in}{3.653732in}}%
\pgfpathlineto{\pgfqpoint{4.228335in}{3.532726in}}%
\pgfpathlineto{\pgfqpoint{4.229211in}{3.467601in}}%
\pgfpathlineto{\pgfqpoint{4.230962in}{3.175235in}}%
\pgfpathlineto{\pgfqpoint{4.232712in}{2.969832in}}%
\pgfpathlineto{\pgfqpoint{4.233588in}{2.979951in}}%
\pgfpathlineto{\pgfqpoint{4.234463in}{3.107119in}}%
\pgfpathlineto{\pgfqpoint{4.235338in}{3.307578in}}%
\pgfpathlineto{\pgfqpoint{4.236214in}{3.367943in}}%
\pgfpathlineto{\pgfqpoint{4.237089in}{3.505302in}}%
\pgfpathlineto{\pgfqpoint{4.237965in}{3.706268in}}%
\pgfpathlineto{\pgfqpoint{4.238840in}{3.610711in}}%
\pgfpathlineto{\pgfqpoint{4.239715in}{3.573620in}}%
\pgfpathlineto{\pgfqpoint{4.241466in}{3.748426in}}%
\pgfpathlineto{\pgfqpoint{4.242341in}{3.770643in}}%
\pgfpathlineto{\pgfqpoint{4.243217in}{3.782856in}}%
\pgfpathlineto{\pgfqpoint{4.244968in}{3.335376in}}%
\pgfpathlineto{\pgfqpoint{4.245843in}{3.267138in}}%
\pgfpathlineto{\pgfqpoint{4.246718in}{3.283297in}}%
\pgfpathlineto{\pgfqpoint{4.247594in}{3.379066in}}%
\pgfpathlineto{\pgfqpoint{4.248469in}{3.399607in}}%
\pgfpathlineto{\pgfqpoint{4.249344in}{3.389367in}}%
\pgfpathlineto{\pgfqpoint{4.250220in}{3.085008in}}%
\pgfpathlineto{\pgfqpoint{4.251095in}{3.029610in}}%
\pgfpathlineto{\pgfqpoint{4.251971in}{2.942103in}}%
\pgfpathlineto{\pgfqpoint{4.252846in}{2.818438in}}%
\pgfpathlineto{\pgfqpoint{4.253721in}{2.759264in}}%
\pgfpathlineto{\pgfqpoint{4.254597in}{2.747242in}}%
\pgfpathlineto{\pgfqpoint{4.255472in}{2.766524in}}%
\pgfpathlineto{\pgfqpoint{4.256347in}{3.120390in}}%
\pgfpathlineto{\pgfqpoint{4.257223in}{3.341180in}}%
\pgfpathlineto{\pgfqpoint{4.258098in}{3.488334in}}%
\pgfpathlineto{\pgfqpoint{4.258974in}{3.510161in}}%
\pgfpathlineto{\pgfqpoint{4.259849in}{3.521285in}}%
\pgfpathlineto{\pgfqpoint{4.260724in}{3.542641in}}%
\pgfpathlineto{\pgfqpoint{4.261600in}{3.471031in}}%
\pgfpathlineto{\pgfqpoint{4.262475in}{3.579187in}}%
\pgfpathlineto{\pgfqpoint{4.263350in}{3.758892in}}%
\pgfpathlineto{\pgfqpoint{4.264226in}{3.580330in}}%
\pgfpathlineto{\pgfqpoint{4.265101in}{3.553482in}}%
\pgfpathlineto{\pgfqpoint{4.266852in}{3.249422in}}%
\pgfpathlineto{\pgfqpoint{4.268603in}{3.526564in}}%
\pgfpathlineto{\pgfqpoint{4.269478in}{3.469504in}}%
\pgfpathlineto{\pgfqpoint{4.270353in}{3.483943in}}%
\pgfpathlineto{\pgfqpoint{4.271229in}{3.345689in}}%
\pgfpathlineto{\pgfqpoint{4.272104in}{3.143156in}}%
\pgfpathlineto{\pgfqpoint{4.272980in}{3.068637in}}%
\pgfpathlineto{\pgfqpoint{4.273855in}{3.040605in}}%
\pgfpathlineto{\pgfqpoint{4.274730in}{3.035712in}}%
\pgfpathlineto{\pgfqpoint{4.275606in}{3.061236in}}%
\pgfpathlineto{\pgfqpoint{4.276481in}{3.060288in}}%
\pgfpathlineto{\pgfqpoint{4.277356in}{3.465923in}}%
\pgfpathlineto{\pgfqpoint{4.279107in}{3.734500in}}%
\pgfpathlineto{\pgfqpoint{4.279983in}{3.801747in}}%
\pgfpathlineto{\pgfqpoint{4.280858in}{3.762652in}}%
\pgfpathlineto{\pgfqpoint{4.281733in}{3.764853in}}%
\pgfpathlineto{\pgfqpoint{4.282609in}{3.731625in}}%
\pgfpathlineto{\pgfqpoint{4.283484in}{3.549532in}}%
\pgfpathlineto{\pgfqpoint{4.284359in}{3.702311in}}%
\pgfpathlineto{\pgfqpoint{4.285235in}{3.704018in}}%
\pgfpathlineto{\pgfqpoint{4.286110in}{3.579022in}}%
\pgfpathlineto{\pgfqpoint{4.287861in}{3.023191in}}%
\pgfpathlineto{\pgfqpoint{4.288736in}{3.027306in}}%
\pgfpathlineto{\pgfqpoint{4.289612in}{3.344903in}}%
\pgfpathlineto{\pgfqpoint{4.290487in}{3.256761in}}%
\pgfpathlineto{\pgfqpoint{4.291362in}{3.480862in}}%
\pgfpathlineto{\pgfqpoint{4.293113in}{3.171459in}}%
\pgfpathlineto{\pgfqpoint{4.294864in}{2.928238in}}%
\pgfpathlineto{\pgfqpoint{4.295739in}{2.924674in}}%
\pgfpathlineto{\pgfqpoint{4.296615in}{2.924069in}}%
\pgfpathlineto{\pgfqpoint{4.297490in}{2.844767in}}%
\pgfpathlineto{\pgfqpoint{4.300116in}{3.665599in}}%
\pgfpathlineto{\pgfqpoint{4.300992in}{3.679347in}}%
\pgfpathlineto{\pgfqpoint{4.301867in}{3.819699in}}%
\pgfpathlineto{\pgfqpoint{4.302742in}{3.685058in}}%
\pgfpathlineto{\pgfqpoint{4.303618in}{3.760958in}}%
\pgfpathlineto{\pgfqpoint{4.304493in}{3.799272in}}%
\pgfpathlineto{\pgfqpoint{4.306244in}{3.654061in}}%
\pgfpathlineto{\pgfqpoint{4.308870in}{2.962850in}}%
\pgfpathlineto{\pgfqpoint{4.309745in}{2.852168in}}%
\pgfpathlineto{\pgfqpoint{4.310621in}{3.044925in}}%
\pgfpathlineto{\pgfqpoint{4.311496in}{2.866617in}}%
\pgfpathlineto{\pgfqpoint{4.312371in}{2.821278in}}%
\pgfpathlineto{\pgfqpoint{4.313247in}{2.803244in}}%
\pgfpathlineto{\pgfqpoint{4.314122in}{2.636898in}}%
\pgfpathlineto{\pgfqpoint{4.314998in}{2.614485in}}%
\pgfpathlineto{\pgfqpoint{4.315873in}{2.551082in}}%
\pgfpathlineto{\pgfqpoint{4.316748in}{2.518943in}}%
\pgfpathlineto{\pgfqpoint{4.317624in}{2.466444in}}%
\pgfpathlineto{\pgfqpoint{4.318499in}{2.601630in}}%
\pgfpathlineto{\pgfqpoint{4.319374in}{2.874584in}}%
\pgfpathlineto{\pgfqpoint{4.320250in}{2.845946in}}%
\pgfpathlineto{\pgfqpoint{4.321125in}{3.070017in}}%
\pgfpathlineto{\pgfqpoint{4.322001in}{3.197247in}}%
\pgfpathlineto{\pgfqpoint{4.322876in}{3.417838in}}%
\pgfpathlineto{\pgfqpoint{4.323751in}{3.380757in}}%
\pgfpathlineto{\pgfqpoint{4.324627in}{3.306537in}}%
\pgfpathlineto{\pgfqpoint{4.326378in}{3.484385in}}%
\pgfpathlineto{\pgfqpoint{4.327253in}{3.475923in}}%
\pgfpathlineto{\pgfqpoint{4.328128in}{3.266042in}}%
\pgfpathlineto{\pgfqpoint{4.330754in}{2.929416in}}%
\pgfpathlineto{\pgfqpoint{4.331630in}{3.103253in}}%
\pgfpathlineto{\pgfqpoint{4.332505in}{3.004690in}}%
\pgfpathlineto{\pgfqpoint{4.333381in}{3.025291in}}%
\pgfpathlineto{\pgfqpoint{4.334256in}{2.851121in}}%
\pgfpathlineto{\pgfqpoint{4.337757in}{2.544226in}}%
\pgfpathlineto{\pgfqpoint{4.338633in}{2.525498in}}%
\pgfpathlineto{\pgfqpoint{4.339508in}{2.620255in}}%
\pgfpathlineto{\pgfqpoint{4.340384in}{2.992248in}}%
\pgfpathlineto{\pgfqpoint{4.341259in}{3.019394in}}%
\pgfpathlineto{\pgfqpoint{4.342134in}{3.187741in}}%
\pgfpathlineto{\pgfqpoint{4.343010in}{3.300421in}}%
\pgfpathlineto{\pgfqpoint{4.344760in}{3.452403in}}%
\pgfpathlineto{\pgfqpoint{4.345636in}{3.581533in}}%
\pgfpathlineto{\pgfqpoint{4.346511in}{3.780397in}}%
\pgfpathlineto{\pgfqpoint{4.347387in}{4.338173in}}%
\pgfpathlineto{\pgfqpoint{4.348262in}{4.160091in}}%
\pgfpathlineto{\pgfqpoint{4.349137in}{3.845220in}}%
\pgfpathlineto{\pgfqpoint{4.350888in}{3.576129in}}%
\pgfpathlineto{\pgfqpoint{4.351763in}{3.641033in}}%
\pgfpathlineto{\pgfqpoint{4.352639in}{3.661707in}}%
\pgfpathlineto{\pgfqpoint{4.354390in}{3.518348in}}%
\pgfpathlineto{\pgfqpoint{4.355265in}{3.415465in}}%
\pgfpathlineto{\pgfqpoint{4.357016in}{3.040998in}}%
\pgfpathlineto{\pgfqpoint{4.357891in}{2.929929in}}%
\pgfpathlineto{\pgfqpoint{4.359642in}{2.815901in}}%
\pgfpathlineto{\pgfqpoint{4.360517in}{2.829327in}}%
\pgfpathlineto{\pgfqpoint{4.361393in}{3.245580in}}%
\pgfpathlineto{\pgfqpoint{4.363143in}{3.772711in}}%
\pgfpathlineto{\pgfqpoint{4.364019in}{3.759542in}}%
\pgfpathlineto{\pgfqpoint{4.365769in}{4.080502in}}%
\pgfpathlineto{\pgfqpoint{4.366645in}{4.137187in}}%
\pgfpathlineto{\pgfqpoint{4.367520in}{4.372728in}}%
\pgfpathlineto{\pgfqpoint{4.368396in}{4.344229in}}%
\pgfpathlineto{\pgfqpoint{4.369271in}{4.106499in}}%
\pgfpathlineto{\pgfqpoint{4.370146in}{4.027970in}}%
\pgfpathlineto{\pgfqpoint{4.371897in}{3.595697in}}%
\pgfpathlineto{\pgfqpoint{4.372772in}{3.540074in}}%
\pgfpathlineto{\pgfqpoint{4.373648in}{3.732872in}}%
\pgfpathlineto{\pgfqpoint{4.374523in}{3.761659in}}%
\pgfpathlineto{\pgfqpoint{4.375399in}{3.669681in}}%
\pgfpathlineto{\pgfqpoint{4.376274in}{3.626486in}}%
\pgfpathlineto{\pgfqpoint{4.378900in}{3.140739in}}%
\pgfpathlineto{\pgfqpoint{4.379775in}{3.146297in}}%
\pgfpathlineto{\pgfqpoint{4.380651in}{3.131587in}}%
\pgfpathlineto{\pgfqpoint{4.381526in}{3.108902in}}%
\pgfpathlineto{\pgfqpoint{4.382402in}{3.459536in}}%
\pgfpathlineto{\pgfqpoint{4.383277in}{3.406896in}}%
\pgfpathlineto{\pgfqpoint{4.384152in}{3.456926in}}%
\pgfpathlineto{\pgfqpoint{4.385028in}{3.764502in}}%
\pgfpathlineto{\pgfqpoint{4.385903in}{3.747720in}}%
\pgfpathlineto{\pgfqpoint{4.386778in}{3.910498in}}%
\pgfpathlineto{\pgfqpoint{4.387654in}{3.876903in}}%
\pgfpathlineto{\pgfqpoint{4.388529in}{3.784843in}}%
\pgfpathlineto{\pgfqpoint{4.389405in}{3.778469in}}%
\pgfpathlineto{\pgfqpoint{4.390280in}{3.830141in}}%
\pgfpathlineto{\pgfqpoint{4.391155in}{3.839957in}}%
\pgfpathlineto{\pgfqpoint{4.392031in}{3.857642in}}%
\pgfpathlineto{\pgfqpoint{4.392906in}{3.626515in}}%
\pgfpathlineto{\pgfqpoint{4.393781in}{3.678543in}}%
\pgfpathlineto{\pgfqpoint{4.394657in}{3.796880in}}%
\pgfpathlineto{\pgfqpoint{4.395532in}{3.776732in}}%
\pgfpathlineto{\pgfqpoint{4.396408in}{3.675692in}}%
\pgfpathlineto{\pgfqpoint{4.397283in}{3.659079in}}%
\pgfpathlineto{\pgfqpoint{4.399034in}{3.409605in}}%
\pgfpathlineto{\pgfqpoint{4.399909in}{3.433287in}}%
\pgfpathlineto{\pgfqpoint{4.400784in}{3.401963in}}%
\pgfpathlineto{\pgfqpoint{4.401660in}{3.317264in}}%
\pgfpathlineto{\pgfqpoint{4.402535in}{3.458153in}}%
\pgfpathlineto{\pgfqpoint{4.403411in}{3.942678in}}%
\pgfpathlineto{\pgfqpoint{4.404286in}{4.213807in}}%
\pgfpathlineto{\pgfqpoint{4.405161in}{4.288979in}}%
\pgfpathlineto{\pgfqpoint{4.406037in}{4.335226in}}%
\pgfpathlineto{\pgfqpoint{4.406912in}{4.255569in}}%
\pgfpathlineto{\pgfqpoint{4.407787in}{4.253677in}}%
\pgfpathlineto{\pgfqpoint{4.408663in}{4.141370in}}%
\pgfpathlineto{\pgfqpoint{4.409538in}{4.279366in}}%
\pgfpathlineto{\pgfqpoint{4.411289in}{4.119724in}}%
\pgfpathlineto{\pgfqpoint{4.412164in}{4.007157in}}%
\pgfpathlineto{\pgfqpoint{4.413040in}{3.803607in}}%
\pgfpathlineto{\pgfqpoint{4.413915in}{3.686512in}}%
\pgfpathlineto{\pgfqpoint{4.414790in}{3.644951in}}%
\pgfpathlineto{\pgfqpoint{4.415666in}{3.861340in}}%
\pgfpathlineto{\pgfqpoint{4.416541in}{3.780991in}}%
\pgfpathlineto{\pgfqpoint{4.418292in}{3.572507in}}%
\pgfpathlineto{\pgfqpoint{4.420043in}{3.417459in}}%
\pgfpathlineto{\pgfqpoint{4.420918in}{2.702718in}}%
\pgfpathlineto{\pgfqpoint{4.421793in}{2.843902in}}%
\pgfpathlineto{\pgfqpoint{4.422669in}{2.738965in}}%
\pgfpathlineto{\pgfqpoint{4.423544in}{2.831163in}}%
\pgfpathlineto{\pgfqpoint{4.424420in}{3.289116in}}%
\pgfpathlineto{\pgfqpoint{4.426170in}{3.921102in}}%
\pgfpathlineto{\pgfqpoint{4.427046in}{4.051705in}}%
\pgfpathlineto{\pgfqpoint{4.427921in}{4.080831in}}%
\pgfpathlineto{\pgfqpoint{4.428796in}{4.262681in}}%
\pgfpathlineto{\pgfqpoint{4.430547in}{4.361863in}}%
\pgfpathlineto{\pgfqpoint{4.431423in}{4.531230in}}%
\pgfpathlineto{\pgfqpoint{4.432298in}{4.502303in}}%
\pgfpathlineto{\pgfqpoint{4.434924in}{3.902125in}}%
\pgfpathlineto{\pgfqpoint{4.435799in}{3.895201in}}%
\pgfpathlineto{\pgfqpoint{4.436675in}{3.983796in}}%
\pgfpathlineto{\pgfqpoint{4.437550in}{3.925981in}}%
\pgfpathlineto{\pgfqpoint{4.438426in}{3.665513in}}%
\pgfpathlineto{\pgfqpoint{4.439301in}{3.570121in}}%
\pgfpathlineto{\pgfqpoint{4.440176in}{3.408216in}}%
\pgfpathlineto{\pgfqpoint{4.441927in}{3.240903in}}%
\pgfpathlineto{\pgfqpoint{4.442802in}{3.248092in}}%
\pgfpathlineto{\pgfqpoint{4.443678in}{3.241658in}}%
\pgfpathlineto{\pgfqpoint{4.444553in}{3.380626in}}%
\pgfpathlineto{\pgfqpoint{4.446304in}{4.088152in}}%
\pgfpathlineto{\pgfqpoint{4.448055in}{4.510343in}}%
\pgfpathlineto{\pgfqpoint{4.448930in}{4.455167in}}%
\pgfpathlineto{\pgfqpoint{4.449805in}{4.521988in}}%
\pgfpathlineto{\pgfqpoint{4.450681in}{4.339688in}}%
\pgfpathlineto{\pgfqpoint{4.451556in}{4.385821in}}%
\pgfpathlineto{\pgfqpoint{4.452432in}{3.959164in}}%
\pgfpathlineto{\pgfqpoint{4.453307in}{4.010031in}}%
\pgfpathlineto{\pgfqpoint{4.454182in}{3.921709in}}%
\pgfpathlineto{\pgfqpoint{4.455058in}{3.632618in}}%
\pgfpathlineto{\pgfqpoint{4.455933in}{3.468428in}}%
\pgfpathlineto{\pgfqpoint{4.456808in}{3.596206in}}%
\pgfpathlineto{\pgfqpoint{4.457684in}{3.609087in}}%
\pgfpathlineto{\pgfqpoint{4.460310in}{3.261655in}}%
\pgfpathlineto{\pgfqpoint{4.461185in}{3.079390in}}%
\pgfpathlineto{\pgfqpoint{4.462061in}{2.960830in}}%
\pgfpathlineto{\pgfqpoint{4.462936in}{2.880421in}}%
\pgfpathlineto{\pgfqpoint{4.463812in}{2.828436in}}%
\pgfpathlineto{\pgfqpoint{4.464687in}{2.824328in}}%
\pgfpathlineto{\pgfqpoint{4.465562in}{2.897484in}}%
\pgfpathlineto{\pgfqpoint{4.466438in}{3.117517in}}%
\pgfpathlineto{\pgfqpoint{4.467313in}{3.236780in}}%
\pgfpathlineto{\pgfqpoint{4.468188in}{3.459202in}}%
\pgfpathlineto{\pgfqpoint{4.469064in}{3.600521in}}%
\pgfpathlineto{\pgfqpoint{4.469939in}{3.601012in}}%
\pgfpathlineto{\pgfqpoint{4.470815in}{3.605083in}}%
\pgfpathlineto{\pgfqpoint{4.471690in}{3.619068in}}%
\pgfpathlineto{\pgfqpoint{4.472565in}{3.745345in}}%
\pgfpathlineto{\pgfqpoint{4.473441in}{3.723787in}}%
\pgfpathlineto{\pgfqpoint{4.474316in}{3.679390in}}%
\pgfpathlineto{\pgfqpoint{4.476942in}{3.248094in}}%
\pgfpathlineto{\pgfqpoint{4.477818in}{3.277992in}}%
\pgfpathlineto{\pgfqpoint{4.478693in}{3.486571in}}%
\pgfpathlineto{\pgfqpoint{4.480444in}{3.346685in}}%
\pgfpathlineto{\pgfqpoint{4.483070in}{3.010671in}}%
\pgfpathlineto{\pgfqpoint{4.483945in}{2.947358in}}%
\pgfpathlineto{\pgfqpoint{4.484821in}{2.821912in}}%
\pgfpathlineto{\pgfqpoint{4.485696in}{2.785423in}}%
\pgfpathlineto{\pgfqpoint{4.486571in}{2.898680in}}%
\pgfpathlineto{\pgfqpoint{4.488322in}{3.562590in}}%
\pgfpathlineto{\pgfqpoint{4.489197in}{3.779913in}}%
\pgfpathlineto{\pgfqpoint{4.490073in}{3.802600in}}%
\pgfpathlineto{\pgfqpoint{4.491824in}{3.757004in}}%
\pgfpathlineto{\pgfqpoint{4.494450in}{4.091888in}}%
\pgfpathlineto{\pgfqpoint{4.495325in}{4.153099in}}%
\pgfpathlineto{\pgfqpoint{4.496200in}{3.983828in}}%
\pgfpathlineto{\pgfqpoint{4.497076in}{3.705635in}}%
\pgfpathlineto{\pgfqpoint{4.497951in}{3.609899in}}%
\pgfpathlineto{\pgfqpoint{4.498827in}{3.588149in}}%
\pgfpathlineto{\pgfqpoint{4.499702in}{3.741330in}}%
\pgfpathlineto{\pgfqpoint{4.500577in}{3.676417in}}%
\pgfpathlineto{\pgfqpoint{4.501453in}{3.710188in}}%
\pgfpathlineto{\pgfqpoint{4.503203in}{3.538586in}}%
\pgfpathlineto{\pgfqpoint{4.504954in}{3.267756in}}%
\pgfpathlineto{\pgfqpoint{4.506705in}{3.295667in}}%
\pgfpathlineto{\pgfqpoint{4.507580in}{3.465520in}}%
\pgfpathlineto{\pgfqpoint{4.508456in}{3.881603in}}%
\pgfpathlineto{\pgfqpoint{4.510206in}{4.245281in}}%
\pgfpathlineto{\pgfqpoint{4.511957in}{4.156242in}}%
\pgfpathlineto{\pgfqpoint{4.512833in}{4.189615in}}%
\pgfpathlineto{\pgfqpoint{4.513708in}{4.193966in}}%
\pgfpathlineto{\pgfqpoint{4.514583in}{4.147625in}}%
\pgfpathlineto{\pgfqpoint{4.515459in}{4.342651in}}%
\pgfpathlineto{\pgfqpoint{4.516334in}{4.352917in}}%
\pgfpathlineto{\pgfqpoint{4.518085in}{3.833732in}}%
\pgfpathlineto{\pgfqpoint{4.518960in}{3.760394in}}%
\pgfpathlineto{\pgfqpoint{4.519836in}{3.782271in}}%
\pgfpathlineto{\pgfqpoint{4.520711in}{3.852248in}}%
\pgfpathlineto{\pgfqpoint{4.521586in}{3.709221in}}%
\pgfpathlineto{\pgfqpoint{4.523337in}{3.632437in}}%
\pgfpathlineto{\pgfqpoint{4.524212in}{3.454159in}}%
\pgfpathlineto{\pgfqpoint{4.525088in}{3.390424in}}%
\pgfpathlineto{\pgfqpoint{4.525963in}{3.238305in}}%
\pgfpathlineto{\pgfqpoint{4.526839in}{3.210153in}}%
\pgfpathlineto{\pgfqpoint{4.527714in}{3.269810in}}%
\pgfpathlineto{\pgfqpoint{4.528589in}{3.351720in}}%
\pgfpathlineto{\pgfqpoint{4.529465in}{3.664884in}}%
\pgfpathlineto{\pgfqpoint{4.530340in}{3.756019in}}%
\pgfpathlineto{\pgfqpoint{4.531215in}{4.052270in}}%
\pgfpathlineto{\pgfqpoint{4.532091in}{4.196897in}}%
\pgfpathlineto{\pgfqpoint{4.532966in}{4.126529in}}%
\pgfpathlineto{\pgfqpoint{4.533842in}{4.191550in}}%
\pgfpathlineto{\pgfqpoint{4.534717in}{4.004426in}}%
\pgfpathlineto{\pgfqpoint{4.535592in}{4.190351in}}%
\pgfpathlineto{\pgfqpoint{4.536468in}{4.250669in}}%
\pgfpathlineto{\pgfqpoint{4.537343in}{4.237253in}}%
\pgfpathlineto{\pgfqpoint{4.539094in}{3.609012in}}%
\pgfpathlineto{\pgfqpoint{4.539969in}{3.443598in}}%
\pgfpathlineto{\pgfqpoint{4.540845in}{3.672161in}}%
\pgfpathlineto{\pgfqpoint{4.541720in}{3.753292in}}%
\pgfpathlineto{\pgfqpoint{4.542595in}{3.734534in}}%
\pgfpathlineto{\pgfqpoint{4.543471in}{3.576676in}}%
\pgfpathlineto{\pgfqpoint{4.544346in}{3.642707in}}%
\pgfpathlineto{\pgfqpoint{4.545221in}{3.426279in}}%
\pgfpathlineto{\pgfqpoint{4.546097in}{3.348724in}}%
\pgfpathlineto{\pgfqpoint{4.546972in}{3.239113in}}%
\pgfpathlineto{\pgfqpoint{4.547848in}{3.211021in}}%
\pgfpathlineto{\pgfqpoint{4.548723in}{3.219479in}}%
\pgfpathlineto{\pgfqpoint{4.549598in}{3.157099in}}%
\pgfpathlineto{\pgfqpoint{4.550474in}{3.391611in}}%
\pgfpathlineto{\pgfqpoint{4.551349in}{3.510132in}}%
\pgfpathlineto{\pgfqpoint{4.552224in}{3.704812in}}%
\pgfpathlineto{\pgfqpoint{4.553100in}{3.616056in}}%
\pgfpathlineto{\pgfqpoint{4.553975in}{3.661420in}}%
\pgfpathlineto{\pgfqpoint{4.554851in}{3.553529in}}%
\pgfpathlineto{\pgfqpoint{4.555726in}{3.710890in}}%
\pgfpathlineto{\pgfqpoint{4.556601in}{3.763264in}}%
\pgfpathlineto{\pgfqpoint{4.557477in}{3.651309in}}%
\pgfpathlineto{\pgfqpoint{4.558352in}{3.592822in}}%
\pgfpathlineto{\pgfqpoint{4.559227in}{3.507394in}}%
\pgfpathlineto{\pgfqpoint{4.560978in}{3.243116in}}%
\pgfpathlineto{\pgfqpoint{4.561854in}{3.275880in}}%
\pgfpathlineto{\pgfqpoint{4.562729in}{3.505646in}}%
\pgfpathlineto{\pgfqpoint{4.563604in}{3.541214in}}%
\pgfpathlineto{\pgfqpoint{4.564480in}{3.509988in}}%
\pgfpathlineto{\pgfqpoint{4.566230in}{3.217629in}}%
\pgfpathlineto{\pgfqpoint{4.567106in}{3.165599in}}%
\pgfpathlineto{\pgfqpoint{4.567981in}{3.051948in}}%
\pgfpathlineto{\pgfqpoint{4.568857in}{3.057158in}}%
\pgfpathlineto{\pgfqpoint{4.570607in}{3.174589in}}%
\pgfpathlineto{\pgfqpoint{4.571483in}{3.387122in}}%
\pgfpathlineto{\pgfqpoint{4.572358in}{3.418623in}}%
\pgfpathlineto{\pgfqpoint{4.573233in}{3.599363in}}%
\pgfpathlineto{\pgfqpoint{4.574109in}{3.398885in}}%
\pgfpathlineto{\pgfqpoint{4.574984in}{3.452081in}}%
\pgfpathlineto{\pgfqpoint{4.575860in}{3.439800in}}%
\pgfpathlineto{\pgfqpoint{4.576735in}{3.539688in}}%
\pgfpathlineto{\pgfqpoint{4.577610in}{3.732963in}}%
\pgfpathlineto{\pgfqpoint{4.578486in}{3.756053in}}%
\pgfpathlineto{\pgfqpoint{4.579361in}{3.676922in}}%
\pgfpathlineto{\pgfqpoint{4.581987in}{3.130170in}}%
\pgfpathlineto{\pgfqpoint{4.582863in}{3.149927in}}%
\pgfpathlineto{\pgfqpoint{4.583738in}{3.214042in}}%
\pgfpathlineto{\pgfqpoint{4.584613in}{3.303264in}}%
\pgfpathlineto{\pgfqpoint{4.585489in}{3.458373in}}%
\pgfpathlineto{\pgfqpoint{4.587239in}{3.108169in}}%
\pgfpathlineto{\pgfqpoint{4.588115in}{3.056214in}}%
\pgfpathlineto{\pgfqpoint{4.589866in}{2.913641in}}%
\pgfpathlineto{\pgfqpoint{4.590741in}{2.965407in}}%
\pgfpathlineto{\pgfqpoint{4.591616in}{2.977740in}}%
\pgfpathlineto{\pgfqpoint{4.592492in}{3.368115in}}%
\pgfpathlineto{\pgfqpoint{4.593367in}{3.279672in}}%
\pgfpathlineto{\pgfqpoint{4.594243in}{3.451260in}}%
\pgfpathlineto{\pgfqpoint{4.595118in}{3.494939in}}%
\pgfpathlineto{\pgfqpoint{4.595993in}{3.491347in}}%
\pgfpathlineto{\pgfqpoint{4.596869in}{3.496047in}}%
\pgfpathlineto{\pgfqpoint{4.597744in}{3.672688in}}%
\pgfpathlineto{\pgfqpoint{4.598619in}{3.579287in}}%
\pgfpathlineto{\pgfqpoint{4.599495in}{3.598507in}}%
\pgfpathlineto{\pgfqpoint{4.601246in}{3.673410in}}%
\pgfpathlineto{\pgfqpoint{4.602996in}{3.279067in}}%
\pgfpathlineto{\pgfqpoint{4.603872in}{3.181864in}}%
\pgfpathlineto{\pgfqpoint{4.604747in}{3.491449in}}%
\pgfpathlineto{\pgfqpoint{4.605622in}{3.623564in}}%
\pgfpathlineto{\pgfqpoint{4.606498in}{3.583578in}}%
\pgfpathlineto{\pgfqpoint{4.608249in}{3.225445in}}%
\pgfpathlineto{\pgfqpoint{4.609124in}{3.152157in}}%
\pgfpathlineto{\pgfqpoint{4.609999in}{3.107225in}}%
\pgfpathlineto{\pgfqpoint{4.611750in}{2.975413in}}%
\pgfpathlineto{\pgfqpoint{4.612625in}{2.941204in}}%
\pgfpathlineto{\pgfqpoint{4.613501in}{3.127823in}}%
\pgfpathlineto{\pgfqpoint{4.614376in}{3.135845in}}%
\pgfpathlineto{\pgfqpoint{4.615252in}{3.104318in}}%
\pgfpathlineto{\pgfqpoint{4.617002in}{2.946517in}}%
\pgfpathlineto{\pgfqpoint{4.617878in}{3.028913in}}%
\pgfpathlineto{\pgfqpoint{4.618753in}{3.145501in}}%
\pgfpathlineto{\pgfqpoint{4.619628in}{3.136070in}}%
\pgfpathlineto{\pgfqpoint{4.620504in}{3.072843in}}%
\pgfpathlineto{\pgfqpoint{4.621379in}{3.278484in}}%
\pgfpathlineto{\pgfqpoint{4.622255in}{3.197472in}}%
\pgfpathlineto{\pgfqpoint{4.623130in}{3.188012in}}%
\pgfpathlineto{\pgfqpoint{4.624005in}{3.139537in}}%
\pgfpathlineto{\pgfqpoint{4.624881in}{3.141114in}}%
\pgfpathlineto{\pgfqpoint{4.625756in}{3.347214in}}%
\pgfpathlineto{\pgfqpoint{4.626631in}{3.486993in}}%
\pgfpathlineto{\pgfqpoint{4.627507in}{3.553976in}}%
\pgfpathlineto{\pgfqpoint{4.628382in}{3.498245in}}%
\pgfpathlineto{\pgfqpoint{4.630133in}{3.243342in}}%
\pgfpathlineto{\pgfqpoint{4.631008in}{3.224086in}}%
\pgfpathlineto{\pgfqpoint{4.631884in}{3.181306in}}%
\pgfpathlineto{\pgfqpoint{4.632759in}{3.164391in}}%
\pgfpathlineto{\pgfqpoint{4.633634in}{3.253769in}}%
\pgfpathlineto{\pgfqpoint{4.634510in}{3.607855in}}%
\pgfpathlineto{\pgfqpoint{4.636261in}{3.786307in}}%
\pgfpathlineto{\pgfqpoint{4.637136in}{3.836997in}}%
\pgfpathlineto{\pgfqpoint{4.638011in}{3.871081in}}%
\pgfpathlineto{\pgfqpoint{4.638887in}{3.987020in}}%
\pgfpathlineto{\pgfqpoint{4.639762in}{4.061360in}}%
\pgfpathlineto{\pgfqpoint{4.640637in}{4.064095in}}%
\pgfpathlineto{\pgfqpoint{4.642388in}{4.001045in}}%
\pgfpathlineto{\pgfqpoint{4.643264in}{3.845994in}}%
\pgfpathlineto{\pgfqpoint{4.645014in}{3.456515in}}%
\pgfpathlineto{\pgfqpoint{4.645890in}{3.419708in}}%
\pgfpathlineto{\pgfqpoint{4.646765in}{3.427676in}}%
\pgfpathlineto{\pgfqpoint{4.647640in}{3.831269in}}%
\pgfpathlineto{\pgfqpoint{4.648516in}{3.814316in}}%
\pgfpathlineto{\pgfqpoint{4.650267in}{3.540459in}}%
\pgfpathlineto{\pgfqpoint{4.651142in}{3.449953in}}%
\pgfpathlineto{\pgfqpoint{4.652017in}{3.436851in}}%
\pgfpathlineto{\pgfqpoint{4.652893in}{3.395619in}}%
\pgfpathlineto{\pgfqpoint{4.653768in}{3.433000in}}%
\pgfpathlineto{\pgfqpoint{4.654643in}{3.535245in}}%
\pgfpathlineto{\pgfqpoint{4.656394in}{4.380028in}}%
\pgfpathlineto{\pgfqpoint{4.657270in}{4.593510in}}%
\pgfpathlineto{\pgfqpoint{4.658145in}{4.492895in}}%
\pgfpathlineto{\pgfqpoint{4.659020in}{4.358034in}}%
\pgfpathlineto{\pgfqpoint{4.659896in}{4.435115in}}%
\pgfpathlineto{\pgfqpoint{4.660771in}{4.614073in}}%
\pgfpathlineto{\pgfqpoint{4.661646in}{4.473567in}}%
\pgfpathlineto{\pgfqpoint{4.662522in}{4.509994in}}%
\pgfpathlineto{\pgfqpoint{4.663397in}{4.572964in}}%
\pgfpathlineto{\pgfqpoint{4.664273in}{4.272599in}}%
\pgfpathlineto{\pgfqpoint{4.665148in}{4.145509in}}%
\pgfpathlineto{\pgfqpoint{4.666023in}{3.927650in}}%
\pgfpathlineto{\pgfqpoint{4.666899in}{3.794603in}}%
\pgfpathlineto{\pgfqpoint{4.667774in}{3.814241in}}%
\pgfpathlineto{\pgfqpoint{4.668649in}{3.942844in}}%
\pgfpathlineto{\pgfqpoint{4.670400in}{3.591620in}}%
\pgfpathlineto{\pgfqpoint{4.671276in}{3.572930in}}%
\pgfpathlineto{\pgfqpoint{4.672151in}{3.522977in}}%
\pgfpathlineto{\pgfqpoint{4.673026in}{3.491826in}}%
\pgfpathlineto{\pgfqpoint{4.673902in}{3.489825in}}%
\pgfpathlineto{\pgfqpoint{4.675652in}{3.333772in}}%
\pgfpathlineto{\pgfqpoint{4.676528in}{3.820018in}}%
\pgfpathlineto{\pgfqpoint{4.677403in}{3.861872in}}%
\pgfpathlineto{\pgfqpoint{4.678279in}{4.102569in}}%
\pgfpathlineto{\pgfqpoint{4.679154in}{4.151282in}}%
\pgfpathlineto{\pgfqpoint{4.680905in}{3.832242in}}%
\pgfpathlineto{\pgfqpoint{4.681780in}{3.777479in}}%
\pgfpathlineto{\pgfqpoint{4.682655in}{3.836576in}}%
\pgfpathlineto{\pgfqpoint{4.683531in}{4.178931in}}%
\pgfpathlineto{\pgfqpoint{4.684406in}{3.958763in}}%
\pgfpathlineto{\pgfqpoint{4.685282in}{3.954096in}}%
\pgfpathlineto{\pgfqpoint{4.687032in}{3.425868in}}%
\pgfpathlineto{\pgfqpoint{4.687908in}{3.446960in}}%
\pgfpathlineto{\pgfqpoint{4.688783in}{3.444289in}}%
\pgfpathlineto{\pgfqpoint{4.689658in}{3.242398in}}%
\pgfpathlineto{\pgfqpoint{4.691409in}{3.080191in}}%
\pgfpathlineto{\pgfqpoint{4.692285in}{2.942979in}}%
\pgfpathlineto{\pgfqpoint{4.693160in}{2.865499in}}%
\pgfpathlineto{\pgfqpoint{4.694035in}{2.760495in}}%
\pgfpathlineto{\pgfqpoint{4.695786in}{2.634384in}}%
\pgfpathlineto{\pgfqpoint{4.696661in}{2.541543in}}%
\pgfpathlineto{\pgfqpoint{4.697537in}{2.760663in}}%
\pgfpathlineto{\pgfqpoint{4.698412in}{2.859298in}}%
\pgfpathlineto{\pgfqpoint{4.699288in}{3.134436in}}%
\pgfpathlineto{\pgfqpoint{4.700163in}{3.232807in}}%
\pgfpathlineto{\pgfqpoint{4.701038in}{3.449011in}}%
\pgfpathlineto{\pgfqpoint{4.701914in}{3.181507in}}%
\pgfpathlineto{\pgfqpoint{4.702789in}{3.176172in}}%
\pgfpathlineto{\pgfqpoint{4.703664in}{3.238673in}}%
\pgfpathlineto{\pgfqpoint{4.704540in}{3.392633in}}%
\pgfpathlineto{\pgfqpoint{4.705415in}{3.423893in}}%
\pgfpathlineto{\pgfqpoint{4.706291in}{3.308063in}}%
\pgfpathlineto{\pgfqpoint{4.707166in}{3.271285in}}%
\pgfpathlineto{\pgfqpoint{4.708041in}{3.143557in}}%
\pgfpathlineto{\pgfqpoint{4.708917in}{3.146536in}}%
\pgfpathlineto{\pgfqpoint{4.709792in}{3.244815in}}%
\pgfpathlineto{\pgfqpoint{4.710667in}{3.203621in}}%
\pgfpathlineto{\pgfqpoint{4.711543in}{3.194106in}}%
\pgfpathlineto{\pgfqpoint{4.714169in}{2.948076in}}%
\pgfpathlineto{\pgfqpoint{4.715920in}{2.807654in}}%
\pgfpathlineto{\pgfqpoint{4.716795in}{2.766763in}}%
\pgfpathlineto{\pgfqpoint{4.717670in}{2.815145in}}%
\pgfpathlineto{\pgfqpoint{4.719421in}{3.638942in}}%
\pgfpathlineto{\pgfqpoint{4.720297in}{3.948323in}}%
\pgfpathlineto{\pgfqpoint{4.721172in}{3.905532in}}%
\pgfpathlineto{\pgfqpoint{4.722047in}{3.930907in}}%
\pgfpathlineto{\pgfqpoint{4.722923in}{4.027446in}}%
\pgfpathlineto{\pgfqpoint{4.723798in}{3.837500in}}%
\pgfpathlineto{\pgfqpoint{4.724674in}{4.176944in}}%
\pgfpathlineto{\pgfqpoint{4.726424in}{4.285676in}}%
\pgfpathlineto{\pgfqpoint{4.728175in}{3.782667in}}%
\pgfpathlineto{\pgfqpoint{4.729050in}{3.546587in}}%
\pgfpathlineto{\pgfqpoint{4.729926in}{3.637701in}}%
\pgfpathlineto{\pgfqpoint{4.730801in}{3.960024in}}%
\pgfpathlineto{\pgfqpoint{4.731677in}{3.883375in}}%
\pgfpathlineto{\pgfqpoint{4.732552in}{3.904444in}}%
\pgfpathlineto{\pgfqpoint{4.733427in}{3.771876in}}%
\pgfpathlineto{\pgfqpoint{4.734303in}{3.556430in}}%
\pgfpathlineto{\pgfqpoint{4.735178in}{3.501077in}}%
\pgfpathlineto{\pgfqpoint{4.736053in}{3.363374in}}%
\pgfpathlineto{\pgfqpoint{4.736929in}{3.532907in}}%
\pgfpathlineto{\pgfqpoint{4.737804in}{3.449878in}}%
\pgfpathlineto{\pgfqpoint{4.738680in}{3.561815in}}%
\pgfpathlineto{\pgfqpoint{4.740430in}{4.009904in}}%
\pgfpathlineto{\pgfqpoint{4.741306in}{4.112190in}}%
\pgfpathlineto{\pgfqpoint{4.742181in}{4.045867in}}%
\pgfpathlineto{\pgfqpoint{4.743932in}{4.206401in}}%
\pgfpathlineto{\pgfqpoint{4.744807in}{4.123855in}}%
\pgfpathlineto{\pgfqpoint{4.747433in}{4.500754in}}%
\pgfpathlineto{\pgfqpoint{4.748309in}{4.135730in}}%
\pgfpathlineto{\pgfqpoint{4.750059in}{3.638828in}}%
\pgfpathlineto{\pgfqpoint{4.750935in}{3.644351in}}%
\pgfpathlineto{\pgfqpoint{4.752686in}{3.517540in}}%
\pgfpathlineto{\pgfqpoint{4.754436in}{3.235904in}}%
\pgfpathlineto{\pgfqpoint{4.755312in}{2.886417in}}%
\pgfpathlineto{\pgfqpoint{4.757062in}{2.763780in}}%
\pgfpathlineto{\pgfqpoint{4.757938in}{2.778468in}}%
\pgfpathlineto{\pgfqpoint{4.758813in}{2.823475in}}%
\pgfpathlineto{\pgfqpoint{4.759689in}{2.836801in}}%
\pgfpathlineto{\pgfqpoint{4.760564in}{3.081383in}}%
\pgfpathlineto{\pgfqpoint{4.762315in}{3.723379in}}%
\pgfpathlineto{\pgfqpoint{4.763190in}{3.784188in}}%
\pgfpathlineto{\pgfqpoint{4.764065in}{3.704196in}}%
\pgfpathlineto{\pgfqpoint{4.764941in}{3.670203in}}%
\pgfpathlineto{\pgfqpoint{4.766692in}{3.975937in}}%
\pgfpathlineto{\pgfqpoint{4.767567in}{3.990884in}}%
\pgfpathlineto{\pgfqpoint{4.768442in}{4.103800in}}%
\pgfpathlineto{\pgfqpoint{4.770193in}{3.956684in}}%
\pgfpathlineto{\pgfqpoint{4.771068in}{3.785412in}}%
\pgfpathlineto{\pgfqpoint{4.771944in}{3.754388in}}%
\pgfpathlineto{\pgfqpoint{4.772819in}{3.795286in}}%
\pgfpathlineto{\pgfqpoint{4.773695in}{3.695606in}}%
\pgfpathlineto{\pgfqpoint{4.774570in}{3.648937in}}%
\pgfpathlineto{\pgfqpoint{4.776321in}{3.325805in}}%
\pgfpathlineto{\pgfqpoint{4.777196in}{3.167373in}}%
\pgfpathlineto{\pgfqpoint{4.778947in}{3.028500in}}%
\pgfpathlineto{\pgfqpoint{4.779822in}{3.058933in}}%
\pgfpathlineto{\pgfqpoint{4.780698in}{3.034430in}}%
\pgfpathlineto{\pgfqpoint{4.783324in}{4.476046in}}%
\pgfpathlineto{\pgfqpoint{4.784199in}{4.647500in}}%
\pgfpathlineto{\pgfqpoint{4.785074in}{4.627801in}}%
\pgfpathlineto{\pgfqpoint{4.786825in}{4.701780in}}%
\pgfpathlineto{\pgfqpoint{4.787701in}{4.919848in}}%
\pgfpathlineto{\pgfqpoint{4.788576in}{5.056253in}}%
\pgfpathlineto{\pgfqpoint{4.792077in}{4.348379in}}%
\pgfpathlineto{\pgfqpoint{4.792953in}{4.105693in}}%
\pgfpathlineto{\pgfqpoint{4.793828in}{4.011865in}}%
\pgfpathlineto{\pgfqpoint{4.794704in}{4.107581in}}%
\pgfpathlineto{\pgfqpoint{4.796454in}{3.886207in}}%
\pgfpathlineto{\pgfqpoint{4.797330in}{3.690168in}}%
\pgfpathlineto{\pgfqpoint{4.798205in}{3.725661in}}%
\pgfpathlineto{\pgfqpoint{4.799080in}{3.667023in}}%
\pgfpathlineto{\pgfqpoint{4.799956in}{3.539175in}}%
\pgfpathlineto{\pgfqpoint{4.800831in}{3.565756in}}%
\pgfpathlineto{\pgfqpoint{4.801707in}{3.800576in}}%
\pgfpathlineto{\pgfqpoint{4.802582in}{4.430044in}}%
\pgfpathlineto{\pgfqpoint{4.805208in}{5.224722in}}%
\pgfpathlineto{\pgfqpoint{4.806083in}{4.907883in}}%
\pgfpathlineto{\pgfqpoint{4.806959in}{4.982463in}}%
\pgfpathlineto{\pgfqpoint{4.807834in}{4.946237in}}%
\pgfpathlineto{\pgfqpoint{4.808710in}{5.072376in}}%
\pgfpathlineto{\pgfqpoint{4.809585in}{5.255919in}}%
\pgfpathlineto{\pgfqpoint{4.811336in}{4.846430in}}%
\pgfpathlineto{\pgfqpoint{4.813086in}{4.256862in}}%
\pgfpathlineto{\pgfqpoint{4.814837in}{4.418630in}}%
\pgfpathlineto{\pgfqpoint{4.815713in}{4.602586in}}%
\pgfpathlineto{\pgfqpoint{4.816588in}{4.473795in}}%
\pgfpathlineto{\pgfqpoint{4.817463in}{4.296408in}}%
\pgfpathlineto{\pgfqpoint{4.818339in}{4.032443in}}%
\pgfpathlineto{\pgfqpoint{4.819214in}{3.899045in}}%
\pgfpathlineto{\pgfqpoint{4.820965in}{3.738196in}}%
\pgfpathlineto{\pgfqpoint{4.821840in}{3.754394in}}%
\pgfpathlineto{\pgfqpoint{4.822716in}{3.759204in}}%
\pgfpathlineto{\pgfqpoint{4.825342in}{4.825389in}}%
\pgfpathlineto{\pgfqpoint{4.826217in}{4.671875in}}%
\pgfpathlineto{\pgfqpoint{4.827092in}{4.299400in}}%
\pgfpathlineto{\pgfqpoint{4.827968in}{4.525477in}}%
\pgfpathlineto{\pgfqpoint{4.828843in}{4.441921in}}%
\pgfpathlineto{\pgfqpoint{4.829719in}{4.074084in}}%
\pgfpathlineto{\pgfqpoint{4.830594in}{4.139129in}}%
\pgfpathlineto{\pgfqpoint{4.831469in}{4.373131in}}%
\pgfpathlineto{\pgfqpoint{4.832345in}{4.176866in}}%
\pgfpathlineto{\pgfqpoint{4.833220in}{4.094857in}}%
\pgfpathlineto{\pgfqpoint{4.834095in}{3.806425in}}%
\pgfpathlineto{\pgfqpoint{4.834971in}{3.705876in}}%
\pgfpathlineto{\pgfqpoint{4.835846in}{3.698249in}}%
\pgfpathlineto{\pgfqpoint{4.836722in}{3.597473in}}%
\pgfpathlineto{\pgfqpoint{4.837597in}{3.603174in}}%
\pgfpathlineto{\pgfqpoint{4.838472in}{3.519125in}}%
\pgfpathlineto{\pgfqpoint{4.839348in}{3.285178in}}%
\pgfpathlineto{\pgfqpoint{4.841098in}{3.062709in}}%
\pgfpathlineto{\pgfqpoint{4.841974in}{3.049493in}}%
\pgfpathlineto{\pgfqpoint{4.842849in}{3.028500in}}%
\pgfpathlineto{\pgfqpoint{4.843725in}{2.952059in}}%
\pgfpathlineto{\pgfqpoint{4.844600in}{3.455594in}}%
\pgfpathlineto{\pgfqpoint{4.845475in}{3.600380in}}%
\pgfpathlineto{\pgfqpoint{4.846351in}{3.691674in}}%
\pgfpathlineto{\pgfqpoint{4.848101in}{3.782358in}}%
\pgfpathlineto{\pgfqpoint{4.848977in}{3.813174in}}%
\pgfpathlineto{\pgfqpoint{4.849852in}{4.113318in}}%
\pgfpathlineto{\pgfqpoint{4.851603in}{3.930351in}}%
\pgfpathlineto{\pgfqpoint{4.852478in}{4.107727in}}%
\pgfpathlineto{\pgfqpoint{4.855980in}{3.400648in}}%
\pgfpathlineto{\pgfqpoint{4.856855in}{3.516520in}}%
\pgfpathlineto{\pgfqpoint{4.857731in}{3.969274in}}%
\pgfpathlineto{\pgfqpoint{4.858606in}{3.935103in}}%
\pgfpathlineto{\pgfqpoint{4.859481in}{3.719922in}}%
\pgfpathlineto{\pgfqpoint{4.861232in}{3.491977in}}%
\pgfpathlineto{\pgfqpoint{4.862108in}{3.272869in}}%
\pgfpathlineto{\pgfqpoint{4.862983in}{3.209511in}}%
\pgfpathlineto{\pgfqpoint{4.863858in}{3.237867in}}%
\pgfpathlineto{\pgfqpoint{4.864734in}{3.349752in}}%
\pgfpathlineto{\pgfqpoint{4.865609in}{3.735064in}}%
\pgfpathlineto{\pgfqpoint{4.866484in}{3.842398in}}%
\pgfpathlineto{\pgfqpoint{4.867360in}{4.209587in}}%
\pgfpathlineto{\pgfqpoint{4.868235in}{4.323558in}}%
\pgfpathlineto{\pgfqpoint{4.869111in}{3.992147in}}%
\pgfpathlineto{\pgfqpoint{4.869986in}{4.044880in}}%
\pgfpathlineto{\pgfqpoint{4.870861in}{4.251546in}}%
\pgfpathlineto{\pgfqpoint{4.871737in}{4.220517in}}%
\pgfpathlineto{\pgfqpoint{4.872612in}{4.405149in}}%
\pgfpathlineto{\pgfqpoint{4.873487in}{4.194840in}}%
\pgfpathlineto{\pgfqpoint{4.875238in}{3.948092in}}%
\pgfpathlineto{\pgfqpoint{4.876114in}{3.736799in}}%
\pgfpathlineto{\pgfqpoint{4.876989in}{3.777464in}}%
\pgfpathlineto{\pgfqpoint{4.877864in}{3.761115in}}%
\pgfpathlineto{\pgfqpoint{4.878740in}{4.045281in}}%
\pgfpathlineto{\pgfqpoint{4.879615in}{4.023268in}}%
\pgfpathlineto{\pgfqpoint{4.880490in}{3.951906in}}%
\pgfpathlineto{\pgfqpoint{4.881366in}{3.670950in}}%
\pgfpathlineto{\pgfqpoint{4.882241in}{3.483973in}}%
\pgfpathlineto{\pgfqpoint{4.883117in}{3.416613in}}%
\pgfpathlineto{\pgfqpoint{4.883992in}{3.372059in}}%
\pgfpathlineto{\pgfqpoint{4.884867in}{3.404455in}}%
\pgfpathlineto{\pgfqpoint{4.887493in}{3.873936in}}%
\pgfpathlineto{\pgfqpoint{4.888369in}{3.921888in}}%
\pgfpathlineto{\pgfqpoint{4.889244in}{4.044790in}}%
\pgfpathlineto{\pgfqpoint{4.890120in}{4.002879in}}%
\pgfpathlineto{\pgfqpoint{4.890995in}{4.051662in}}%
\pgfpathlineto{\pgfqpoint{4.891870in}{3.946242in}}%
\pgfpathlineto{\pgfqpoint{4.892746in}{3.950546in}}%
\pgfpathlineto{\pgfqpoint{4.893621in}{3.929213in}}%
\pgfpathlineto{\pgfqpoint{4.894496in}{3.971313in}}%
\pgfpathlineto{\pgfqpoint{4.895372in}{3.800648in}}%
\pgfpathlineto{\pgfqpoint{4.896247in}{3.722867in}}%
\pgfpathlineto{\pgfqpoint{4.897123in}{3.618202in}}%
\pgfpathlineto{\pgfqpoint{4.897998in}{3.963422in}}%
\pgfpathlineto{\pgfqpoint{4.899749in}{4.082132in}}%
\pgfpathlineto{\pgfqpoint{4.900624in}{3.830967in}}%
\pgfpathlineto{\pgfqpoint{4.901499in}{3.476006in}}%
\pgfpathlineto{\pgfqpoint{4.904126in}{2.909676in}}%
\pgfpathlineto{\pgfqpoint{4.905001in}{2.831366in}}%
\pgfpathlineto{\pgfqpoint{4.905876in}{2.925950in}}%
\pgfpathlineto{\pgfqpoint{4.906752in}{2.965822in}}%
\pgfpathlineto{\pgfqpoint{4.907627in}{3.358050in}}%
\pgfpathlineto{\pgfqpoint{4.908502in}{3.493337in}}%
\pgfpathlineto{\pgfqpoint{4.909378in}{3.804839in}}%
\pgfpathlineto{\pgfqpoint{4.910253in}{3.765986in}}%
\pgfpathlineto{\pgfqpoint{4.911129in}{3.815034in}}%
\pgfpathlineto{\pgfqpoint{4.912004in}{3.882696in}}%
\pgfpathlineto{\pgfqpoint{4.912879in}{3.647011in}}%
\pgfpathlineto{\pgfqpoint{4.913755in}{3.800421in}}%
\pgfpathlineto{\pgfqpoint{4.914630in}{3.757566in}}%
\pgfpathlineto{\pgfqpoint{4.915505in}{3.864798in}}%
\pgfpathlineto{\pgfqpoint{4.916381in}{3.836140in}}%
\pgfpathlineto{\pgfqpoint{4.917256in}{3.560281in}}%
\pgfpathlineto{\pgfqpoint{4.918132in}{3.557676in}}%
\pgfpathlineto{\pgfqpoint{4.919882in}{3.991627in}}%
\pgfpathlineto{\pgfqpoint{4.920758in}{3.876654in}}%
\pgfpathlineto{\pgfqpoint{4.921633in}{3.811296in}}%
\pgfpathlineto{\pgfqpoint{4.922508in}{3.683750in}}%
\pgfpathlineto{\pgfqpoint{4.923384in}{3.647880in}}%
\pgfpathlineto{\pgfqpoint{4.924259in}{3.553070in}}%
\pgfpathlineto{\pgfqpoint{4.925135in}{3.579991in}}%
\pgfpathlineto{\pgfqpoint{4.926010in}{3.562962in}}%
\pgfpathlineto{\pgfqpoint{4.926885in}{3.613558in}}%
\pgfpathlineto{\pgfqpoint{4.927761in}{3.569872in}}%
\pgfpathlineto{\pgfqpoint{4.928636in}{4.108790in}}%
\pgfpathlineto{\pgfqpoint{4.929511in}{4.185174in}}%
\pgfpathlineto{\pgfqpoint{4.930387in}{4.218136in}}%
\pgfpathlineto{\pgfqpoint{4.931262in}{4.331070in}}%
\pgfpathlineto{\pgfqpoint{4.933013in}{4.028441in}}%
\pgfpathlineto{\pgfqpoint{4.933888in}{4.062272in}}%
\pgfpathlineto{\pgfqpoint{4.934764in}{4.128499in}}%
\pgfpathlineto{\pgfqpoint{4.935639in}{4.338206in}}%
\pgfpathlineto{\pgfqpoint{4.938265in}{3.970294in}}%
\pgfpathlineto{\pgfqpoint{4.939141in}{3.988304in}}%
\pgfpathlineto{\pgfqpoint{4.940016in}{4.018397in}}%
\pgfpathlineto{\pgfqpoint{4.940891in}{4.251174in}}%
\pgfpathlineto{\pgfqpoint{4.941767in}{4.220213in}}%
\pgfpathlineto{\pgfqpoint{4.942642in}{4.277529in}}%
\pgfpathlineto{\pgfqpoint{4.943517in}{4.209339in}}%
\pgfpathlineto{\pgfqpoint{4.944393in}{3.933518in}}%
\pgfpathlineto{\pgfqpoint{4.945268in}{3.766213in}}%
\pgfpathlineto{\pgfqpoint{4.946144in}{3.684542in}}%
\pgfpathlineto{\pgfqpoint{4.947019in}{3.542535in}}%
\pgfpathlineto{\pgfqpoint{4.947894in}{3.625527in}}%
\pgfpathlineto{\pgfqpoint{4.948770in}{3.753413in}}%
\pgfpathlineto{\pgfqpoint{4.951396in}{5.087057in}}%
\pgfpathlineto{\pgfqpoint{4.952271in}{4.975445in}}%
\pgfpathlineto{\pgfqpoint{4.953147in}{4.914882in}}%
\pgfpathlineto{\pgfqpoint{4.954022in}{4.913145in}}%
\pgfpathlineto{\pgfqpoint{4.954897in}{4.878445in}}%
\pgfpathlineto{\pgfqpoint{4.955773in}{4.945239in}}%
\pgfpathlineto{\pgfqpoint{4.956648in}{4.689392in}}%
\pgfpathlineto{\pgfqpoint{4.957523in}{4.561506in}}%
\pgfpathlineto{\pgfqpoint{4.958399in}{4.495694in}}%
\pgfpathlineto{\pgfqpoint{4.961025in}{4.096178in}}%
\pgfpathlineto{\pgfqpoint{4.961900in}{4.099916in}}%
\pgfpathlineto{\pgfqpoint{4.962776in}{3.972899in}}%
\pgfpathlineto{\pgfqpoint{4.963651in}{4.010921in}}%
\pgfpathlineto{\pgfqpoint{4.965402in}{3.772065in}}%
\pgfpathlineto{\pgfqpoint{4.968028in}{3.471211in}}%
\pgfpathlineto{\pgfqpoint{4.968903in}{3.413781in}}%
\pgfpathlineto{\pgfqpoint{4.969779in}{3.475100in}}%
\pgfpathlineto{\pgfqpoint{4.970654in}{3.887491in}}%
\pgfpathlineto{\pgfqpoint{4.972405in}{4.280739in}}%
\pgfpathlineto{\pgfqpoint{4.973280in}{4.283117in}}%
\pgfpathlineto{\pgfqpoint{4.975031in}{4.330315in}}%
\pgfpathlineto{\pgfqpoint{4.975906in}{4.534887in}}%
\pgfpathlineto{\pgfqpoint{4.976782in}{4.555654in}}%
\pgfpathlineto{\pgfqpoint{4.977657in}{4.291386in}}%
\pgfpathlineto{\pgfqpoint{4.978532in}{4.231842in}}%
\pgfpathlineto{\pgfqpoint{4.980283in}{3.869669in}}%
\pgfpathlineto{\pgfqpoint{4.981159in}{3.514632in}}%
\pgfpathlineto{\pgfqpoint{4.982034in}{3.697984in}}%
\pgfpathlineto{\pgfqpoint{4.983785in}{3.600154in}}%
\pgfpathlineto{\pgfqpoint{4.984660in}{3.500851in}}%
\pgfpathlineto{\pgfqpoint{4.985535in}{3.465547in}}%
\pgfpathlineto{\pgfqpoint{4.986411in}{3.466982in}}%
\pgfpathlineto{\pgfqpoint{4.987286in}{3.436096in}}%
\pgfpathlineto{\pgfqpoint{4.989037in}{3.203055in}}%
\pgfpathlineto{\pgfqpoint{4.989912in}{3.152006in}}%
\pgfpathlineto{\pgfqpoint{4.993414in}{4.190800in}}%
\pgfpathlineto{\pgfqpoint{4.994289in}{4.149606in}}%
\pgfpathlineto{\pgfqpoint{4.996040in}{4.016849in}}%
\pgfpathlineto{\pgfqpoint{4.996915in}{4.200277in}}%
\pgfpathlineto{\pgfqpoint{4.997791in}{4.224517in}}%
\pgfpathlineto{\pgfqpoint{4.998666in}{4.405113in}}%
\pgfpathlineto{\pgfqpoint{4.999542in}{4.434338in}}%
\pgfpathlineto{\pgfqpoint{5.000417in}{4.353423in}}%
\pgfpathlineto{\pgfqpoint{5.002168in}{3.905917in}}%
\pgfpathlineto{\pgfqpoint{5.003043in}{4.066878in}}%
\pgfpathlineto{\pgfqpoint{5.003918in}{4.071825in}}%
\pgfpathlineto{\pgfqpoint{5.004794in}{4.064839in}}%
\pgfpathlineto{\pgfqpoint{5.005669in}{4.036710in}}%
\pgfpathlineto{\pgfqpoint{5.006545in}{4.095046in}}%
\pgfpathlineto{\pgfqpoint{5.008295in}{3.550502in}}%
\pgfpathlineto{\pgfqpoint{5.010046in}{3.404040in}}%
\pgfpathlineto{\pgfqpoint{5.010921in}{3.450821in}}%
\pgfpathlineto{\pgfqpoint{5.011797in}{3.554807in}}%
\pgfpathlineto{\pgfqpoint{5.015298in}{4.183890in}}%
\pgfpathlineto{\pgfqpoint{5.016174in}{4.159838in}}%
\pgfpathlineto{\pgfqpoint{5.017049in}{4.266655in}}%
\pgfpathlineto{\pgfqpoint{5.017924in}{4.066312in}}%
\pgfpathlineto{\pgfqpoint{5.018800in}{4.172865in}}%
\pgfpathlineto{\pgfqpoint{5.019675in}{4.235127in}}%
\pgfpathlineto{\pgfqpoint{5.022301in}{3.920642in}}%
\pgfpathlineto{\pgfqpoint{5.023177in}{3.852754in}}%
\pgfpathlineto{\pgfqpoint{5.024052in}{3.857662in}}%
\pgfpathlineto{\pgfqpoint{5.024927in}{3.968557in}}%
\pgfpathlineto{\pgfqpoint{5.026678in}{3.937256in}}%
\pgfpathlineto{\pgfqpoint{5.027554in}{3.953189in}}%
\pgfpathlineto{\pgfqpoint{5.028429in}{3.755112in}}%
\pgfpathlineto{\pgfqpoint{5.030180in}{3.537967in}}%
\pgfpathlineto{\pgfqpoint{5.031055in}{3.435681in}}%
\pgfpathlineto{\pgfqpoint{5.032806in}{3.727133in}}%
\pgfpathlineto{\pgfqpoint{5.035432in}{4.498979in}}%
\pgfpathlineto{\pgfqpoint{5.036307in}{4.474889in}}%
\pgfpathlineto{\pgfqpoint{5.037183in}{4.470056in}}%
\pgfpathlineto{\pgfqpoint{5.039809in}{4.213265in}}%
\pgfpathlineto{\pgfqpoint{5.041560in}{4.315438in}}%
\pgfpathlineto{\pgfqpoint{5.043310in}{3.938200in}}%
\pgfpathlineto{\pgfqpoint{5.044186in}{3.798684in}}%
\pgfpathlineto{\pgfqpoint{5.045061in}{3.782675in}}%
\pgfpathlineto{\pgfqpoint{5.045936in}{3.891871in}}%
\pgfpathlineto{\pgfqpoint{5.046812in}{3.874087in}}%
\pgfpathlineto{\pgfqpoint{5.047687in}{3.889605in}}%
\pgfpathlineto{\pgfqpoint{5.048563in}{3.816166in}}%
\pgfpathlineto{\pgfqpoint{5.050313in}{3.592678in}}%
\pgfpathlineto{\pgfqpoint{5.051189in}{3.580859in}}%
\pgfpathlineto{\pgfqpoint{5.052064in}{3.526941in}}%
\pgfpathlineto{\pgfqpoint{5.052939in}{3.501077in}}%
\pgfpathlineto{\pgfqpoint{5.053815in}{3.436322in}}%
\pgfpathlineto{\pgfqpoint{5.054690in}{3.698664in}}%
\pgfpathlineto{\pgfqpoint{5.055566in}{3.729890in}}%
\pgfpathlineto{\pgfqpoint{5.056441in}{3.796797in}}%
\pgfpathlineto{\pgfqpoint{5.057316in}{3.901159in}}%
\pgfpathlineto{\pgfqpoint{5.058192in}{3.866007in}}%
\pgfpathlineto{\pgfqpoint{5.059067in}{3.894325in}}%
\pgfpathlineto{\pgfqpoint{5.059942in}{3.803404in}}%
\pgfpathlineto{\pgfqpoint{5.060818in}{4.190158in}}%
\pgfpathlineto{\pgfqpoint{5.061693in}{4.255441in}}%
\pgfpathlineto{\pgfqpoint{5.062569in}{4.075449in}}%
\pgfpathlineto{\pgfqpoint{5.063444in}{4.133294in}}%
\pgfpathlineto{\pgfqpoint{5.065195in}{3.895835in}}%
\pgfpathlineto{\pgfqpoint{5.066070in}{3.828211in}}%
\pgfpathlineto{\pgfqpoint{5.066945in}{3.975127in}}%
\pgfpathlineto{\pgfqpoint{5.067821in}{3.736006in}}%
\pgfpathlineto{\pgfqpoint{5.068696in}{3.786602in}}%
\pgfpathlineto{\pgfqpoint{5.069572in}{3.820508in}}%
\pgfpathlineto{\pgfqpoint{5.070447in}{3.821113in}}%
\pgfpathlineto{\pgfqpoint{5.072198in}{3.663436in}}%
\pgfpathlineto{\pgfqpoint{5.073073in}{3.605629in}}%
\pgfpathlineto{\pgfqpoint{5.073948in}{3.598681in}}%
\pgfpathlineto{\pgfqpoint{5.074824in}{3.640479in}}%
\pgfpathlineto{\pgfqpoint{5.077450in}{4.415761in}}%
\pgfpathlineto{\pgfqpoint{5.078325in}{4.466243in}}%
\pgfpathlineto{\pgfqpoint{5.079201in}{4.642761in}}%
\pgfpathlineto{\pgfqpoint{5.080076in}{4.689052in}}%
\pgfpathlineto{\pgfqpoint{5.082702in}{4.492409in}}%
\pgfpathlineto{\pgfqpoint{5.083578in}{4.734361in}}%
\pgfpathlineto{\pgfqpoint{5.084453in}{4.532319in}}%
\pgfpathlineto{\pgfqpoint{5.085328in}{4.215871in}}%
\pgfpathlineto{\pgfqpoint{5.086204in}{4.089797in}}%
\pgfpathlineto{\pgfqpoint{5.087079in}{4.061441in}}%
\pgfpathlineto{\pgfqpoint{5.087954in}{4.155081in}}%
\pgfpathlineto{\pgfqpoint{5.088830in}{4.079489in}}%
\pgfpathlineto{\pgfqpoint{5.089705in}{4.075147in}}%
\pgfpathlineto{\pgfqpoint{5.090581in}{3.877221in}}%
\pgfpathlineto{\pgfqpoint{5.093207in}{3.729550in}}%
\pgfpathlineto{\pgfqpoint{5.094082in}{3.682692in}}%
\pgfpathlineto{\pgfqpoint{5.094957in}{3.684014in}}%
\pgfpathlineto{\pgfqpoint{5.095833in}{3.701269in}}%
\pgfpathlineto{\pgfqpoint{5.096708in}{3.966896in}}%
\pgfpathlineto{\pgfqpoint{5.098459in}{4.654352in}}%
\pgfpathlineto{\pgfqpoint{5.099334in}{4.491390in}}%
\pgfpathlineto{\pgfqpoint{5.100210in}{4.627129in}}%
\pgfpathlineto{\pgfqpoint{5.101085in}{4.696415in}}%
\pgfpathlineto{\pgfqpoint{5.101960in}{4.719107in}}%
\pgfpathlineto{\pgfqpoint{5.102836in}{4.512987in}}%
\pgfpathlineto{\pgfqpoint{5.104587in}{4.289914in}}%
\pgfpathlineto{\pgfqpoint{5.105462in}{4.175130in}}%
\pgfpathlineto{\pgfqpoint{5.106337in}{3.846108in}}%
\pgfpathlineto{\pgfqpoint{5.108088in}{3.595434in}}%
\pgfpathlineto{\pgfqpoint{5.108963in}{3.934575in}}%
\pgfpathlineto{\pgfqpoint{5.109839in}{3.819036in}}%
\pgfpathlineto{\pgfqpoint{5.110714in}{3.775803in}}%
\pgfpathlineto{\pgfqpoint{5.111590in}{3.802007in}}%
\pgfpathlineto{\pgfqpoint{5.112465in}{3.624621in}}%
\pgfpathlineto{\pgfqpoint{5.114216in}{3.474458in}}%
\pgfpathlineto{\pgfqpoint{5.115091in}{3.435227in}}%
\pgfpathlineto{\pgfqpoint{5.116842in}{3.491600in}}%
\pgfpathlineto{\pgfqpoint{5.117717in}{3.952661in}}%
\pgfpathlineto{\pgfqpoint{5.119468in}{4.546176in}}%
\pgfpathlineto{\pgfqpoint{5.120343in}{4.561884in}}%
\pgfpathlineto{\pgfqpoint{5.121219in}{4.450120in}}%
\pgfpathlineto{\pgfqpoint{5.122094in}{4.383553in}}%
\pgfpathlineto{\pgfqpoint{5.122969in}{4.266731in}}%
\pgfpathlineto{\pgfqpoint{5.123845in}{4.022135in}}%
\pgfpathlineto{\pgfqpoint{5.124720in}{4.129217in}}%
\pgfpathlineto{\pgfqpoint{5.127346in}{3.358881in}}%
\pgfpathlineto{\pgfqpoint{5.128222in}{3.313119in}}%
\pgfpathlineto{\pgfqpoint{5.129097in}{3.323238in}}%
\pgfpathlineto{\pgfqpoint{5.129973in}{3.380252in}}%
\pgfpathlineto{\pgfqpoint{5.130848in}{3.273020in}}%
\pgfpathlineto{\pgfqpoint{5.131723in}{3.390182in}}%
\pgfpathlineto{\pgfqpoint{5.132599in}{3.423334in}}%
\pgfpathlineto{\pgfqpoint{5.133474in}{3.331129in}}%
\pgfpathlineto{\pgfqpoint{5.134349in}{3.321086in}}%
\pgfpathlineto{\pgfqpoint{5.135225in}{3.337208in}}%
\pgfpathlineto{\pgfqpoint{5.136100in}{3.381234in}}%
\pgfpathlineto{\pgfqpoint{5.136976in}{3.374551in}}%
\pgfpathlineto{\pgfqpoint{5.137851in}{3.431678in}}%
\pgfpathlineto{\pgfqpoint{5.138726in}{3.730607in}}%
\pgfpathlineto{\pgfqpoint{5.139602in}{3.785394in}}%
\pgfpathlineto{\pgfqpoint{5.140477in}{3.721583in}}%
\pgfpathlineto{\pgfqpoint{5.141352in}{3.766213in}}%
\pgfpathlineto{\pgfqpoint{5.142228in}{3.742387in}}%
\pgfpathlineto{\pgfqpoint{5.143103in}{3.703874in}}%
\pgfpathlineto{\pgfqpoint{5.143979in}{3.845693in}}%
\pgfpathlineto{\pgfqpoint{5.144854in}{3.651014in}}%
\pgfpathlineto{\pgfqpoint{5.145729in}{3.689338in}}%
\pgfpathlineto{\pgfqpoint{5.146605in}{3.711615in}}%
\pgfpathlineto{\pgfqpoint{5.147480in}{3.539288in}}%
\pgfpathlineto{\pgfqpoint{5.148355in}{3.293787in}}%
\pgfpathlineto{\pgfqpoint{5.149231in}{3.163598in}}%
\pgfpathlineto{\pgfqpoint{5.150106in}{3.141962in}}%
\pgfpathlineto{\pgfqpoint{5.151857in}{3.412120in}}%
\pgfpathlineto{\pgfqpoint{5.152732in}{3.530679in}}%
\pgfpathlineto{\pgfqpoint{5.153608in}{3.471060in}}%
\pgfpathlineto{\pgfqpoint{5.154483in}{3.445611in}}%
\pgfpathlineto{\pgfqpoint{5.155358in}{3.357409in}}%
\pgfpathlineto{\pgfqpoint{5.156234in}{3.319764in}}%
\pgfpathlineto{\pgfqpoint{5.157109in}{3.203470in}}%
\pgfpathlineto{\pgfqpoint{5.157985in}{3.187461in}}%
\pgfpathlineto{\pgfqpoint{5.158860in}{3.164239in}}%
\pgfpathlineto{\pgfqpoint{5.161486in}{3.921737in}}%
\pgfpathlineto{\pgfqpoint{5.162361in}{4.018850in}}%
\pgfpathlineto{\pgfqpoint{5.163237in}{3.911467in}}%
\pgfpathlineto{\pgfqpoint{5.164112in}{3.717769in}}%
\pgfpathlineto{\pgfqpoint{5.164988in}{3.699457in}}%
\pgfpathlineto{\pgfqpoint{5.165863in}{3.739556in}}%
\pgfpathlineto{\pgfqpoint{5.167614in}{3.928609in}}%
\pgfpathlineto{\pgfqpoint{5.170240in}{3.263618in}}%
\pgfpathlineto{\pgfqpoint{5.171115in}{3.370926in}}%
\pgfpathlineto{\pgfqpoint{5.171991in}{3.400490in}}%
\pgfpathlineto{\pgfqpoint{5.172866in}{3.418085in}}%
\pgfpathlineto{\pgfqpoint{5.173741in}{3.570740in}}%
\pgfpathlineto{\pgfqpoint{5.174617in}{3.547972in}}%
\pgfpathlineto{\pgfqpoint{5.176367in}{3.303528in}}%
\pgfpathlineto{\pgfqpoint{5.177243in}{3.283554in}}%
\pgfpathlineto{\pgfqpoint{5.178118in}{3.288010in}}%
\pgfpathlineto{\pgfqpoint{5.178994in}{3.244513in}}%
\pgfpathlineto{\pgfqpoint{5.179869in}{3.423938in}}%
\pgfpathlineto{\pgfqpoint{5.180744in}{3.691981in}}%
\pgfpathlineto{\pgfqpoint{5.181620in}{3.814505in}}%
\pgfpathlineto{\pgfqpoint{5.183370in}{4.126271in}}%
\pgfpathlineto{\pgfqpoint{5.184246in}{4.110866in}}%
\pgfpathlineto{\pgfqpoint{5.185121in}{3.925211in}}%
\pgfpathlineto{\pgfqpoint{5.185997in}{3.902330in}}%
\pgfpathlineto{\pgfqpoint{5.186872in}{3.928798in}}%
\pgfpathlineto{\pgfqpoint{5.188623in}{3.840029in}}%
\pgfpathlineto{\pgfqpoint{5.189498in}{3.642631in}}%
\pgfpathlineto{\pgfqpoint{5.191249in}{3.354539in}}%
\pgfpathlineto{\pgfqpoint{5.192124in}{3.351103in}}%
\pgfpathlineto{\pgfqpoint{5.193000in}{3.539779in}}%
\pgfpathlineto{\pgfqpoint{5.193875in}{3.491336in}}%
\pgfpathlineto{\pgfqpoint{5.194750in}{3.558771in}}%
\pgfpathlineto{\pgfqpoint{5.195626in}{3.504513in}}%
\pgfpathlineto{\pgfqpoint{5.198252in}{3.096351in}}%
\pgfpathlineto{\pgfqpoint{5.199127in}{3.129729in}}%
\pgfpathlineto{\pgfqpoint{5.200003in}{3.119194in}}%
\pgfpathlineto{\pgfqpoint{5.200878in}{3.190783in}}%
\pgfpathlineto{\pgfqpoint{5.201753in}{3.427034in}}%
\pgfpathlineto{\pgfqpoint{5.202629in}{3.422126in}}%
\pgfpathlineto{\pgfqpoint{5.203504in}{3.618126in}}%
\pgfpathlineto{\pgfqpoint{5.204379in}{3.482160in}}%
\pgfpathlineto{\pgfqpoint{5.205255in}{3.400755in}}%
\pgfpathlineto{\pgfqpoint{5.207006in}{3.454220in}}%
\pgfpathlineto{\pgfqpoint{5.207881in}{3.599096in}}%
\pgfpathlineto{\pgfqpoint{5.208756in}{3.686846in}}%
\pgfpathlineto{\pgfqpoint{5.209632in}{3.727926in}}%
\pgfpathlineto{\pgfqpoint{5.211382in}{3.382895in}}%
\pgfpathlineto{\pgfqpoint{5.212258in}{3.378666in}}%
\pgfpathlineto{\pgfqpoint{5.213133in}{3.353331in}}%
\pgfpathlineto{\pgfqpoint{5.214009in}{3.584673in}}%
\pgfpathlineto{\pgfqpoint{5.214884in}{3.587203in}}%
\pgfpathlineto{\pgfqpoint{5.215759in}{3.556808in}}%
\pgfpathlineto{\pgfqpoint{5.216635in}{3.434208in}}%
\pgfpathlineto{\pgfqpoint{5.217510in}{3.384368in}}%
\pgfpathlineto{\pgfqpoint{5.218385in}{3.267771in}}%
\pgfpathlineto{\pgfqpoint{5.219261in}{3.032087in}}%
\pgfpathlineto{\pgfqpoint{5.220136in}{3.061312in}}%
\pgfpathlineto{\pgfqpoint{5.221012in}{3.054515in}}%
\pgfpathlineto{\pgfqpoint{5.221887in}{3.179758in}}%
\pgfpathlineto{\pgfqpoint{5.223638in}{3.819187in}}%
\pgfpathlineto{\pgfqpoint{5.225388in}{4.240602in}}%
\pgfpathlineto{\pgfqpoint{5.226264in}{4.057439in}}%
\pgfpathlineto{\pgfqpoint{5.227139in}{4.250683in}}%
\pgfpathlineto{\pgfqpoint{5.228015in}{4.310341in}}%
\pgfpathlineto{\pgfqpoint{5.228890in}{4.255856in}}%
\pgfpathlineto{\pgfqpoint{5.229765in}{4.421802in}}%
\pgfpathlineto{\pgfqpoint{5.230641in}{4.240338in}}%
\pgfpathlineto{\pgfqpoint{5.231516in}{4.393144in}}%
\pgfpathlineto{\pgfqpoint{5.232391in}{4.185891in}}%
\pgfpathlineto{\pgfqpoint{5.233267in}{4.102333in}}%
\pgfpathlineto{\pgfqpoint{5.234142in}{4.120532in}}%
\pgfpathlineto{\pgfqpoint{5.235018in}{4.206998in}}%
\pgfpathlineto{\pgfqpoint{5.235893in}{4.219495in}}%
\pgfpathlineto{\pgfqpoint{5.236768in}{4.272205in}}%
\pgfpathlineto{\pgfqpoint{5.237644in}{4.215229in}}%
\pgfpathlineto{\pgfqpoint{5.238519in}{4.042487in}}%
\pgfpathlineto{\pgfqpoint{5.239394in}{3.799477in}}%
\pgfpathlineto{\pgfqpoint{5.240270in}{3.724792in}}%
\pgfpathlineto{\pgfqpoint{5.241145in}{3.753451in}}%
\pgfpathlineto{\pgfqpoint{5.242021in}{3.688809in}}%
\pgfpathlineto{\pgfqpoint{5.242896in}{3.867517in}}%
\pgfpathlineto{\pgfqpoint{5.243771in}{4.294029in}}%
\pgfpathlineto{\pgfqpoint{5.245522in}{4.611460in}}%
\pgfpathlineto{\pgfqpoint{5.246397in}{4.557957in}}%
\pgfpathlineto{\pgfqpoint{5.248148in}{4.853903in}}%
\pgfpathlineto{\pgfqpoint{5.249024in}{4.858434in}}%
\pgfpathlineto{\pgfqpoint{5.249899in}{4.912125in}}%
\pgfpathlineto{\pgfqpoint{5.250774in}{4.836156in}}%
\pgfpathlineto{\pgfqpoint{5.251650in}{4.634832in}}%
\pgfpathlineto{\pgfqpoint{5.252525in}{4.595375in}}%
\pgfpathlineto{\pgfqpoint{5.254276in}{4.062574in}}%
\pgfpathlineto{\pgfqpoint{5.255151in}{3.905803in}}%
\pgfpathlineto{\pgfqpoint{5.256027in}{3.883677in}}%
\pgfpathlineto{\pgfqpoint{5.256902in}{4.239016in}}%
\pgfpathlineto{\pgfqpoint{5.257777in}{4.240036in}}%
\pgfpathlineto{\pgfqpoint{5.258653in}{4.163803in}}%
\pgfpathlineto{\pgfqpoint{5.259528in}{3.888963in}}%
\pgfpathlineto{\pgfqpoint{5.260404in}{3.728946in}}%
\pgfpathlineto{\pgfqpoint{5.261279in}{3.685260in}}%
\pgfpathlineto{\pgfqpoint{5.262154in}{3.572326in}}%
\pgfpathlineto{\pgfqpoint{5.263030in}{3.517351in}}%
\pgfpathlineto{\pgfqpoint{5.263905in}{3.481481in}}%
\pgfpathlineto{\pgfqpoint{5.264780in}{3.737517in}}%
\pgfpathlineto{\pgfqpoint{5.265656in}{3.833761in}}%
\pgfpathlineto{\pgfqpoint{5.266531in}{3.876126in}}%
\pgfpathlineto{\pgfqpoint{5.267407in}{3.382178in}}%
\pgfpathlineto{\pgfqpoint{5.268282in}{3.471928in}}%
\pgfpathlineto{\pgfqpoint{5.269157in}{3.442930in}}%
\pgfpathlineto{\pgfqpoint{5.270033in}{3.554731in}}%
\pgfpathlineto{\pgfqpoint{5.270908in}{3.559564in}}%
\pgfpathlineto{\pgfqpoint{5.271783in}{3.521542in}}%
\pgfpathlineto{\pgfqpoint{5.272659in}{3.424844in}}%
\pgfpathlineto{\pgfqpoint{5.273534in}{3.550880in}}%
\pgfpathlineto{\pgfqpoint{5.276160in}{3.277475in}}%
\pgfpathlineto{\pgfqpoint{5.277036in}{3.292314in}}%
\pgfpathlineto{\pgfqpoint{5.277911in}{3.509724in}}%
\pgfpathlineto{\pgfqpoint{5.279662in}{3.313232in}}%
\pgfpathlineto{\pgfqpoint{5.280537in}{3.242247in}}%
\pgfpathlineto{\pgfqpoint{5.281413in}{3.205849in}}%
\pgfpathlineto{\pgfqpoint{5.282288in}{3.086911in}}%
\pgfpathlineto{\pgfqpoint{5.283163in}{3.066409in}}%
\pgfpathlineto{\pgfqpoint{5.284039in}{3.024687in}}%
\pgfpathlineto{\pgfqpoint{5.284914in}{3.085099in}}%
\pgfpathlineto{\pgfqpoint{5.285789in}{3.297336in}}%
\pgfpathlineto{\pgfqpoint{5.286665in}{3.367679in}}%
\pgfpathlineto{\pgfqpoint{5.287540in}{3.468719in}}%
\pgfpathlineto{\pgfqpoint{5.288416in}{3.476006in}}%
\pgfpathlineto{\pgfqpoint{5.289291in}{3.388823in}}%
\pgfpathlineto{\pgfqpoint{5.290166in}{3.358353in}}%
\pgfpathlineto{\pgfqpoint{5.291042in}{3.412913in}}%
\pgfpathlineto{\pgfqpoint{5.291917in}{3.397734in}}%
\pgfpathlineto{\pgfqpoint{5.293668in}{3.573912in}}%
\pgfpathlineto{\pgfqpoint{5.294543in}{3.610537in}}%
\pgfpathlineto{\pgfqpoint{5.295419in}{3.449198in}}%
\pgfpathlineto{\pgfqpoint{5.296294in}{3.473401in}}%
\pgfpathlineto{\pgfqpoint{5.297169in}{3.565114in}}%
\pgfpathlineto{\pgfqpoint{5.298045in}{3.546122in}}%
\pgfpathlineto{\pgfqpoint{5.298920in}{3.859890in}}%
\pgfpathlineto{\pgfqpoint{5.299795in}{3.894665in}}%
\pgfpathlineto{\pgfqpoint{5.300671in}{3.777049in}}%
\pgfpathlineto{\pgfqpoint{5.301546in}{3.543404in}}%
\pgfpathlineto{\pgfqpoint{5.302422in}{3.569947in}}%
\pgfpathlineto{\pgfqpoint{5.303297in}{3.556053in}}%
\pgfpathlineto{\pgfqpoint{5.304172in}{3.513915in}}%
\pgfpathlineto{\pgfqpoint{5.305048in}{3.451501in}}%
\pgfpathlineto{\pgfqpoint{5.305923in}{3.573044in}}%
\pgfpathlineto{\pgfqpoint{5.306798in}{3.909126in}}%
\pgfpathlineto{\pgfqpoint{5.307674in}{3.938200in}}%
\pgfpathlineto{\pgfqpoint{5.308549in}{3.954284in}}%
\pgfpathlineto{\pgfqpoint{5.309425in}{3.891418in}}%
\pgfpathlineto{\pgfqpoint{5.310300in}{3.734421in}}%
\pgfpathlineto{\pgfqpoint{5.311175in}{3.966745in}}%
\pgfpathlineto{\pgfqpoint{5.312051in}{3.919774in}}%
\pgfpathlineto{\pgfqpoint{5.312926in}{4.032405in}}%
\pgfpathlineto{\pgfqpoint{5.313801in}{4.067785in}}%
\pgfpathlineto{\pgfqpoint{5.314677in}{4.236524in}}%
\pgfpathlineto{\pgfqpoint{5.315552in}{3.997970in}}%
\pgfpathlineto{\pgfqpoint{5.316428in}{3.642291in}}%
\pgfpathlineto{\pgfqpoint{5.317303in}{3.542233in}}%
\pgfpathlineto{\pgfqpoint{5.318178in}{3.483029in}}%
\pgfpathlineto{\pgfqpoint{5.319929in}{3.651240in}}%
\pgfpathlineto{\pgfqpoint{5.320804in}{3.688243in}}%
\pgfpathlineto{\pgfqpoint{5.321680in}{3.595132in}}%
\pgfpathlineto{\pgfqpoint{5.322555in}{3.368887in}}%
\pgfpathlineto{\pgfqpoint{5.324306in}{3.196598in}}%
\pgfpathlineto{\pgfqpoint{5.325181in}{3.253726in}}%
\pgfpathlineto{\pgfqpoint{5.326057in}{3.151062in}}%
\pgfpathlineto{\pgfqpoint{5.326932in}{3.253197in}}%
\pgfpathlineto{\pgfqpoint{5.327807in}{3.397432in}}%
\pgfpathlineto{\pgfqpoint{5.330434in}{3.551446in}}%
\pgfpathlineto{\pgfqpoint{5.331309in}{3.366886in}}%
\pgfpathlineto{\pgfqpoint{5.332184in}{3.646822in}}%
\pgfpathlineto{\pgfqpoint{5.333060in}{3.762286in}}%
\pgfpathlineto{\pgfqpoint{5.333935in}{3.704063in}}%
\pgfpathlineto{\pgfqpoint{5.334810in}{3.760813in}}%
\pgfpathlineto{\pgfqpoint{5.335686in}{3.690093in}}%
\pgfpathlineto{\pgfqpoint{5.337437in}{3.162238in}}%
\pgfpathlineto{\pgfqpoint{5.338312in}{3.052476in}}%
\pgfpathlineto{\pgfqpoint{5.340063in}{3.296543in}}%
\pgfpathlineto{\pgfqpoint{5.340938in}{3.516520in}}%
\pgfpathlineto{\pgfqpoint{5.341813in}{3.464792in}}%
\pgfpathlineto{\pgfqpoint{5.342689in}{3.375041in}}%
\pgfpathlineto{\pgfqpoint{5.344440in}{3.088044in}}%
\pgfpathlineto{\pgfqpoint{5.345315in}{2.987533in}}%
\pgfpathlineto{\pgfqpoint{5.347066in}{2.852058in}}%
\pgfpathlineto{\pgfqpoint{5.347941in}{2.934256in}}%
\pgfpathlineto{\pgfqpoint{5.348816in}{3.201393in}}%
\pgfpathlineto{\pgfqpoint{5.349692in}{3.247760in}}%
\pgfpathlineto{\pgfqpoint{5.350567in}{3.349291in}}%
\pgfpathlineto{\pgfqpoint{5.351443in}{3.148155in}}%
\pgfpathlineto{\pgfqpoint{5.352318in}{3.361751in}}%
\pgfpathlineto{\pgfqpoint{5.353193in}{3.327920in}}%
\pgfpathlineto{\pgfqpoint{5.354069in}{3.248402in}}%
\pgfpathlineto{\pgfqpoint{5.354944in}{3.455428in}}%
\pgfpathlineto{\pgfqpoint{5.355819in}{3.423787in}}%
\pgfpathlineto{\pgfqpoint{5.356695in}{3.574667in}}%
\pgfpathlineto{\pgfqpoint{5.358446in}{3.204301in}}%
\pgfpathlineto{\pgfqpoint{5.359321in}{3.250176in}}%
\pgfpathlineto{\pgfqpoint{5.360196in}{3.224312in}}%
\pgfpathlineto{\pgfqpoint{5.361072in}{3.581048in}}%
\pgfpathlineto{\pgfqpoint{5.361947in}{3.565265in}}%
\pgfpathlineto{\pgfqpoint{5.362822in}{3.385614in}}%
\pgfpathlineto{\pgfqpoint{5.363698in}{3.359372in}}%
\pgfpathlineto{\pgfqpoint{5.365449in}{3.094425in}}%
\pgfpathlineto{\pgfqpoint{5.366324in}{3.043188in}}%
\pgfpathlineto{\pgfqpoint{5.368075in}{2.999578in}}%
\pgfpathlineto{\pgfqpoint{5.368950in}{3.104016in}}%
\pgfpathlineto{\pgfqpoint{5.369825in}{3.338190in}}%
\pgfpathlineto{\pgfqpoint{5.370701in}{3.425410in}}%
\pgfpathlineto{\pgfqpoint{5.371576in}{3.618957in}}%
\pgfpathlineto{\pgfqpoint{5.372452in}{3.631644in}}%
\pgfpathlineto{\pgfqpoint{5.373327in}{3.741481in}}%
\pgfpathlineto{\pgfqpoint{5.374202in}{3.433491in}}%
\pgfpathlineto{\pgfqpoint{5.376828in}{3.779617in}}%
\pgfpathlineto{\pgfqpoint{5.377704in}{3.805556in}}%
\pgfpathlineto{\pgfqpoint{5.379455in}{3.437531in}}%
\pgfpathlineto{\pgfqpoint{5.380330in}{3.455503in}}%
\pgfpathlineto{\pgfqpoint{5.381205in}{3.448594in}}%
\pgfpathlineto{\pgfqpoint{5.382956in}{3.719091in}}%
\pgfpathlineto{\pgfqpoint{5.383831in}{3.795853in}}%
\pgfpathlineto{\pgfqpoint{5.384707in}{3.683485in}}%
\pgfpathlineto{\pgfqpoint{5.386458in}{3.370511in}}%
\pgfpathlineto{\pgfqpoint{5.387333in}{3.309871in}}%
\pgfpathlineto{\pgfqpoint{5.388208in}{3.311873in}}%
\pgfpathlineto{\pgfqpoint{5.389084in}{3.226993in}}%
\pgfpathlineto{\pgfqpoint{5.389959in}{3.356238in}}%
\pgfpathlineto{\pgfqpoint{5.390835in}{3.578103in}}%
\pgfpathlineto{\pgfqpoint{5.392585in}{4.244869in}}%
\pgfpathlineto{\pgfqpoint{5.393461in}{4.332731in}}%
\pgfpathlineto{\pgfqpoint{5.394336in}{4.260614in}}%
\pgfpathlineto{\pgfqpoint{5.395211in}{4.392313in}}%
\pgfpathlineto{\pgfqpoint{5.396087in}{4.425125in}}%
\pgfpathlineto{\pgfqpoint{5.396962in}{4.609005in}}%
\pgfpathlineto{\pgfqpoint{5.397838in}{4.444872in}}%
\pgfpathlineto{\pgfqpoint{5.398713in}{4.499168in}}%
\pgfpathlineto{\pgfqpoint{5.399588in}{4.319969in}}%
\pgfpathlineto{\pgfqpoint{5.401339in}{3.834743in}}%
\pgfpathlineto{\pgfqpoint{5.402214in}{3.839727in}}%
\pgfpathlineto{\pgfqpoint{5.403090in}{3.916791in}}%
\pgfpathlineto{\pgfqpoint{5.403965in}{4.021569in}}%
\pgfpathlineto{\pgfqpoint{5.404841in}{3.886962in}}%
\pgfpathlineto{\pgfqpoint{5.405716in}{3.803442in}}%
\pgfpathlineto{\pgfqpoint{5.407467in}{3.467171in}}%
\pgfpathlineto{\pgfqpoint{5.408342in}{3.411327in}}%
\pgfpathlineto{\pgfqpoint{5.409217in}{3.499265in}}%
\pgfpathlineto{\pgfqpoint{5.410093in}{3.534493in}}%
\pgfpathlineto{\pgfqpoint{5.410968in}{3.646181in}}%
\pgfpathlineto{\pgfqpoint{5.411844in}{3.988418in}}%
\pgfpathlineto{\pgfqpoint{5.414470in}{4.372150in}}%
\pgfpathlineto{\pgfqpoint{5.415345in}{4.357689in}}%
\pgfpathlineto{\pgfqpoint{5.416220in}{4.305432in}}%
\pgfpathlineto{\pgfqpoint{5.417096in}{4.045621in}}%
\pgfpathlineto{\pgfqpoint{5.417971in}{4.018435in}}%
\pgfpathlineto{\pgfqpoint{5.418847in}{4.145377in}}%
\pgfpathlineto{\pgfqpoint{5.419722in}{4.060686in}}%
\pgfpathlineto{\pgfqpoint{5.422348in}{3.488164in}}%
\pgfpathlineto{\pgfqpoint{5.423223in}{3.520258in}}%
\pgfpathlineto{\pgfqpoint{5.424099in}{3.864119in}}%
\pgfpathlineto{\pgfqpoint{5.425850in}{3.626622in}}%
\pgfpathlineto{\pgfqpoint{5.426725in}{3.550049in}}%
\pgfpathlineto{\pgfqpoint{5.427600in}{3.354237in}}%
\pgfpathlineto{\pgfqpoint{5.428476in}{3.243946in}}%
\pgfpathlineto{\pgfqpoint{5.429351in}{3.086723in}}%
\pgfpathlineto{\pgfqpoint{5.430226in}{3.008413in}}%
\pgfpathlineto{\pgfqpoint{5.431102in}{2.970806in}}%
\pgfpathlineto{\pgfqpoint{5.431977in}{2.910658in}}%
\pgfpathlineto{\pgfqpoint{5.434603in}{3.452256in}}%
\pgfpathlineto{\pgfqpoint{5.435479in}{3.501379in}}%
\pgfpathlineto{\pgfqpoint{5.436354in}{3.674763in}}%
\pgfpathlineto{\pgfqpoint{5.437229in}{3.708934in}}%
\pgfpathlineto{\pgfqpoint{5.438105in}{3.525242in}}%
\pgfpathlineto{\pgfqpoint{5.438980in}{3.549369in}}%
\pgfpathlineto{\pgfqpoint{5.439856in}{3.762928in}}%
\pgfpathlineto{\pgfqpoint{5.440731in}{3.675481in}}%
\pgfpathlineto{\pgfqpoint{5.441606in}{3.424391in}}%
\pgfpathlineto{\pgfqpoint{5.442482in}{3.382933in}}%
\pgfpathlineto{\pgfqpoint{5.443357in}{3.368660in}}%
\pgfpathlineto{\pgfqpoint{5.444232in}{3.529698in}}%
\pgfpathlineto{\pgfqpoint{5.445108in}{3.637572in}}%
\pgfpathlineto{\pgfqpoint{5.445983in}{3.602986in}}%
\pgfpathlineto{\pgfqpoint{5.446859in}{3.518446in}}%
\pgfpathlineto{\pgfqpoint{5.447734in}{3.509233in}}%
\pgfpathlineto{\pgfqpoint{5.448609in}{3.310249in}}%
\pgfpathlineto{\pgfqpoint{5.449485in}{3.249723in}}%
\pgfpathlineto{\pgfqpoint{5.450360in}{3.160728in}}%
\pgfpathlineto{\pgfqpoint{5.451235in}{3.104620in}}%
\pgfpathlineto{\pgfqpoint{5.452111in}{3.072224in}}%
\pgfpathlineto{\pgfqpoint{5.452986in}{3.163145in}}%
\pgfpathlineto{\pgfqpoint{5.453862in}{3.594830in}}%
\pgfpathlineto{\pgfqpoint{5.454737in}{3.785507in}}%
\pgfpathlineto{\pgfqpoint{5.455612in}{3.854226in}}%
\pgfpathlineto{\pgfqpoint{5.456488in}{4.141488in}}%
\pgfpathlineto{\pgfqpoint{5.457363in}{3.975806in}}%
\pgfpathlineto{\pgfqpoint{5.459114in}{3.905388in}}%
\pgfpathlineto{\pgfqpoint{5.459989in}{4.076091in}}%
\pgfpathlineto{\pgfqpoint{5.460865in}{4.041279in}}%
\pgfpathlineto{\pgfqpoint{5.461740in}{3.936576in}}%
\pgfpathlineto{\pgfqpoint{5.463491in}{3.658301in}}%
\pgfpathlineto{\pgfqpoint{5.464366in}{3.437984in}}%
\pgfpathlineto{\pgfqpoint{5.465241in}{3.522788in}}%
\pgfpathlineto{\pgfqpoint{5.466117in}{3.739820in}}%
\pgfpathlineto{\pgfqpoint{5.466992in}{3.737630in}}%
\pgfpathlineto{\pgfqpoint{5.467868in}{3.586070in}}%
\pgfpathlineto{\pgfqpoint{5.468743in}{3.514481in}}%
\pgfpathlineto{\pgfqpoint{5.470494in}{3.275738in}}%
\pgfpathlineto{\pgfqpoint{5.471369in}{3.332979in}}%
\pgfpathlineto{\pgfqpoint{5.473120in}{3.300885in}}%
\pgfpathlineto{\pgfqpoint{5.473995in}{3.394109in}}%
\pgfpathlineto{\pgfqpoint{5.477497in}{4.329295in}}%
\pgfpathlineto{\pgfqpoint{5.479247in}{3.987360in}}%
\pgfpathlineto{\pgfqpoint{5.480123in}{3.989097in}}%
\pgfpathlineto{\pgfqpoint{5.480998in}{4.131671in}}%
\pgfpathlineto{\pgfqpoint{5.481874in}{4.159951in}}%
\pgfpathlineto{\pgfqpoint{5.482749in}{4.134767in}}%
\pgfpathlineto{\pgfqpoint{5.483624in}{4.045998in}}%
\pgfpathlineto{\pgfqpoint{5.484500in}{3.679407in}}%
\pgfpathlineto{\pgfqpoint{5.485375in}{3.629680in}}%
\pgfpathlineto{\pgfqpoint{5.486250in}{3.621751in}}%
\pgfpathlineto{\pgfqpoint{5.487126in}{3.749561in}}%
\pgfpathlineto{\pgfqpoint{5.488001in}{4.064953in}}%
\pgfpathlineto{\pgfqpoint{5.489752in}{4.004880in}}%
\pgfpathlineto{\pgfqpoint{5.491503in}{3.746314in}}%
\pgfpathlineto{\pgfqpoint{5.493253in}{3.561301in}}%
\pgfpathlineto{\pgfqpoint{5.494129in}{3.483860in}}%
\pgfpathlineto{\pgfqpoint{5.495004in}{3.593810in}}%
\pgfpathlineto{\pgfqpoint{5.495880in}{3.866611in}}%
\pgfpathlineto{\pgfqpoint{5.496755in}{3.970445in}}%
\pgfpathlineto{\pgfqpoint{5.497630in}{4.189931in}}%
\pgfpathlineto{\pgfqpoint{5.498506in}{4.178793in}}%
\pgfpathlineto{\pgfqpoint{5.500256in}{4.296786in}}%
\pgfpathlineto{\pgfqpoint{5.501132in}{4.256196in}}%
\pgfpathlineto{\pgfqpoint{5.502007in}{4.330466in}}%
\pgfpathlineto{\pgfqpoint{5.502883in}{4.378985in}}%
\pgfpathlineto{\pgfqpoint{5.503758in}{4.171619in}}%
\pgfpathlineto{\pgfqpoint{5.504633in}{4.084889in}}%
\pgfpathlineto{\pgfqpoint{5.505509in}{3.881601in}}%
\pgfpathlineto{\pgfqpoint{5.506384in}{3.783770in}}%
\pgfpathlineto{\pgfqpoint{5.507259in}{3.788867in}}%
\pgfpathlineto{\pgfqpoint{5.508135in}{4.133596in}}%
\pgfpathlineto{\pgfqpoint{5.509010in}{4.269638in}}%
\pgfpathlineto{\pgfqpoint{5.509886in}{4.091194in}}%
\pgfpathlineto{\pgfqpoint{5.510761in}{3.979695in}}%
\pgfpathlineto{\pgfqpoint{5.511636in}{3.658829in}}%
\pgfpathlineto{\pgfqpoint{5.512512in}{3.669590in}}%
\pgfpathlineto{\pgfqpoint{5.513387in}{3.555750in}}%
\pgfpathlineto{\pgfqpoint{5.514262in}{3.575724in}}%
\pgfpathlineto{\pgfqpoint{5.515138in}{3.460676in}}%
\pgfpathlineto{\pgfqpoint{5.516889in}{3.956437in}}%
\pgfpathlineto{\pgfqpoint{5.518639in}{4.595866in}}%
\pgfpathlineto{\pgfqpoint{5.519515in}{4.850127in}}%
\pgfpathlineto{\pgfqpoint{5.520390in}{4.737231in}}%
\pgfpathlineto{\pgfqpoint{5.521265in}{4.697887in}}%
\pgfpathlineto{\pgfqpoint{5.522141in}{4.556144in}}%
\pgfpathlineto{\pgfqpoint{5.523016in}{4.647405in}}%
\pgfpathlineto{\pgfqpoint{5.523892in}{4.673685in}}%
\pgfpathlineto{\pgfqpoint{5.524767in}{4.601492in}}%
\pgfpathlineto{\pgfqpoint{5.525642in}{4.471302in}}%
\pgfpathlineto{\pgfqpoint{5.526518in}{4.135333in}}%
\pgfpathlineto{\pgfqpoint{5.527393in}{3.979167in}}%
\pgfpathlineto{\pgfqpoint{5.528269in}{4.049812in}}%
\pgfpathlineto{\pgfqpoint{5.529144in}{4.470019in}}%
\pgfpathlineto{\pgfqpoint{5.530019in}{4.598395in}}%
\pgfpathlineto{\pgfqpoint{5.532645in}{3.996309in}}%
\pgfpathlineto{\pgfqpoint{5.533521in}{3.660302in}}%
\pgfpathlineto{\pgfqpoint{5.534396in}{3.643613in}}%
\pgfpathlineto{\pgfqpoint{5.535272in}{3.610462in}}%
\pgfpathlineto{\pgfqpoint{5.536147in}{3.588033in}}%
\pgfpathlineto{\pgfqpoint{5.537022in}{3.654600in}}%
\pgfpathlineto{\pgfqpoint{5.539648in}{4.864437in}}%
\pgfpathlineto{\pgfqpoint{5.540524in}{4.883694in}}%
\pgfpathlineto{\pgfqpoint{5.541399in}{5.055190in}}%
\pgfpathlineto{\pgfqpoint{5.542275in}{4.853110in}}%
\pgfpathlineto{\pgfqpoint{5.543150in}{4.858547in}}%
\pgfpathlineto{\pgfqpoint{5.544025in}{4.971745in}}%
\pgfpathlineto{\pgfqpoint{5.545776in}{4.931759in}}%
\pgfpathlineto{\pgfqpoint{5.548402in}{4.083265in}}%
\pgfpathlineto{\pgfqpoint{5.549278in}{4.077753in}}%
\pgfpathlineto{\pgfqpoint{5.550153in}{4.202278in}}%
\pgfpathlineto{\pgfqpoint{5.551028in}{4.257140in}}%
\pgfpathlineto{\pgfqpoint{5.551904in}{4.196010in}}%
\pgfpathlineto{\pgfqpoint{5.552779in}{4.068049in}}%
\pgfpathlineto{\pgfqpoint{5.553654in}{3.824473in}}%
\pgfpathlineto{\pgfqpoint{5.555405in}{3.702477in}}%
\pgfpathlineto{\pgfqpoint{5.556281in}{3.568022in}}%
\pgfpathlineto{\pgfqpoint{5.557156in}{3.584258in}}%
\pgfpathlineto{\pgfqpoint{5.558031in}{3.740273in}}%
\pgfpathlineto{\pgfqpoint{5.558907in}{4.105844in}}%
\pgfpathlineto{\pgfqpoint{5.560657in}{4.529638in}}%
\pgfpathlineto{\pgfqpoint{5.561533in}{4.591523in}}%
\pgfpathlineto{\pgfqpoint{5.562408in}{4.526202in}}%
\pgfpathlineto{\pgfqpoint{5.563284in}{4.428523in}}%
\pgfpathlineto{\pgfqpoint{5.564159in}{4.445740in}}%
\pgfpathlineto{\pgfqpoint{5.565034in}{4.493089in}}%
\pgfpathlineto{\pgfqpoint{5.565910in}{4.562941in}}%
\pgfpathlineto{\pgfqpoint{5.566785in}{4.712462in}}%
\pgfpathlineto{\pgfqpoint{5.567660in}{4.668738in}}%
\pgfpathlineto{\pgfqpoint{5.568536in}{4.310492in}}%
\pgfpathlineto{\pgfqpoint{5.569411in}{4.201409in}}%
\pgfpathlineto{\pgfqpoint{5.570287in}{4.166748in}}%
\pgfpathlineto{\pgfqpoint{5.571162in}{4.062725in}}%
\pgfpathlineto{\pgfqpoint{5.572037in}{4.360106in}}%
\pgfpathlineto{\pgfqpoint{5.572913in}{4.089760in}}%
\pgfpathlineto{\pgfqpoint{5.575539in}{3.635533in}}%
\pgfpathlineto{\pgfqpoint{5.576414in}{3.568475in}}%
\pgfpathlineto{\pgfqpoint{5.577290in}{3.599625in}}%
\pgfpathlineto{\pgfqpoint{5.578165in}{3.587089in}}%
\pgfpathlineto{\pgfqpoint{5.579040in}{3.613520in}}%
\pgfpathlineto{\pgfqpoint{5.579916in}{3.861325in}}%
\pgfpathlineto{\pgfqpoint{5.580791in}{3.931592in}}%
\pgfpathlineto{\pgfqpoint{5.581666in}{4.058194in}}%
\pgfpathlineto{\pgfqpoint{5.582542in}{4.075261in}}%
\pgfpathlineto{\pgfqpoint{5.583417in}{4.270808in}}%
\pgfpathlineto{\pgfqpoint{5.584293in}{4.081264in}}%
\pgfpathlineto{\pgfqpoint{5.585168in}{4.121099in}}%
\pgfpathlineto{\pgfqpoint{5.586043in}{4.086324in}}%
\pgfpathlineto{\pgfqpoint{5.586919in}{4.350968in}}%
\pgfpathlineto{\pgfqpoint{5.588669in}{4.163538in}}%
\pgfpathlineto{\pgfqpoint{5.589545in}{3.872237in}}%
\pgfpathlineto{\pgfqpoint{5.590420in}{3.776634in}}%
\pgfpathlineto{\pgfqpoint{5.591296in}{3.850866in}}%
\pgfpathlineto{\pgfqpoint{5.593046in}{4.183248in}}%
\pgfpathlineto{\pgfqpoint{5.593922in}{3.977166in}}%
\pgfpathlineto{\pgfqpoint{5.594797in}{3.857436in}}%
\pgfpathlineto{\pgfqpoint{5.595672in}{3.459468in}}%
\pgfpathlineto{\pgfqpoint{5.597423in}{3.306587in}}%
\pgfpathlineto{\pgfqpoint{5.598299in}{3.374022in}}%
\pgfpathlineto{\pgfqpoint{5.599174in}{3.371681in}}%
\pgfpathlineto{\pgfqpoint{5.600049in}{3.476006in}}%
\pgfpathlineto{\pgfqpoint{5.600925in}{3.897421in}}%
\pgfpathlineto{\pgfqpoint{5.601800in}{4.052681in}}%
\pgfpathlineto{\pgfqpoint{5.602675in}{4.336205in}}%
\pgfpathlineto{\pgfqpoint{5.603551in}{4.355915in}}%
\pgfpathlineto{\pgfqpoint{5.604426in}{4.591259in}}%
\pgfpathlineto{\pgfqpoint{5.605302in}{4.676894in}}%
\pgfpathlineto{\pgfqpoint{5.606177in}{4.590391in}}%
\pgfpathlineto{\pgfqpoint{5.607052in}{4.687428in}}%
\pgfpathlineto{\pgfqpoint{5.607928in}{4.708422in}}%
\pgfpathlineto{\pgfqpoint{5.609678in}{4.583594in}}%
\pgfpathlineto{\pgfqpoint{5.610554in}{4.214285in}}%
\pgfpathlineto{\pgfqpoint{5.611429in}{4.185136in}}%
\pgfpathlineto{\pgfqpoint{5.613180in}{4.339867in}}%
\pgfpathlineto{\pgfqpoint{5.614055in}{4.296295in}}%
\pgfpathlineto{\pgfqpoint{5.614931in}{4.158101in}}%
\pgfpathlineto{\pgfqpoint{5.615806in}{3.931970in}}%
\pgfpathlineto{\pgfqpoint{5.616681in}{3.784148in}}%
\pgfpathlineto{\pgfqpoint{5.617557in}{3.571193in}}%
\pgfpathlineto{\pgfqpoint{5.618432in}{3.564699in}}%
\pgfpathlineto{\pgfqpoint{5.619308in}{3.504249in}}%
\pgfpathlineto{\pgfqpoint{5.620183in}{3.545329in}}%
\pgfpathlineto{\pgfqpoint{5.621058in}{3.537627in}}%
\pgfpathlineto{\pgfqpoint{5.623684in}{4.502755in}}%
\pgfpathlineto{\pgfqpoint{5.624560in}{4.614707in}}%
\pgfpathlineto{\pgfqpoint{5.625435in}{4.545308in}}%
\pgfpathlineto{\pgfqpoint{5.626311in}{4.180378in}}%
\pgfpathlineto{\pgfqpoint{5.627186in}{4.138883in}}%
\pgfpathlineto{\pgfqpoint{5.628061in}{4.066161in}}%
\pgfpathlineto{\pgfqpoint{5.628937in}{4.289876in}}%
\pgfpathlineto{\pgfqpoint{5.629812in}{4.152249in}}%
\pgfpathlineto{\pgfqpoint{5.630687in}{4.141261in}}%
\pgfpathlineto{\pgfqpoint{5.631563in}{3.998348in}}%
\pgfpathlineto{\pgfqpoint{5.632438in}{3.802271in}}%
\pgfpathlineto{\pgfqpoint{5.633314in}{3.855925in}}%
\pgfpathlineto{\pgfqpoint{5.634189in}{3.875748in}}%
\pgfpathlineto{\pgfqpoint{5.635064in}{3.933555in}}%
\pgfpathlineto{\pgfqpoint{5.635940in}{3.801781in}}%
\pgfpathlineto{\pgfqpoint{5.636815in}{3.613105in}}%
\pgfpathlineto{\pgfqpoint{5.637690in}{3.364243in}}%
\pgfpathlineto{\pgfqpoint{5.638566in}{3.228088in}}%
\pgfpathlineto{\pgfqpoint{5.639441in}{3.257237in}}%
\pgfpathlineto{\pgfqpoint{5.640317in}{3.161294in}}%
\pgfpathlineto{\pgfqpoint{5.641192in}{3.097597in}}%
\pgfpathlineto{\pgfqpoint{5.642067in}{3.088271in}}%
\pgfpathlineto{\pgfqpoint{5.642943in}{3.175944in}}%
\pgfpathlineto{\pgfqpoint{5.643818in}{3.182137in}}%
\pgfpathlineto{\pgfqpoint{5.644693in}{3.374249in}}%
\pgfpathlineto{\pgfqpoint{5.645569in}{3.406267in}}%
\pgfpathlineto{\pgfqpoint{5.646444in}{3.556845in}}%
\pgfpathlineto{\pgfqpoint{5.647320in}{3.430772in}}%
\pgfpathlineto{\pgfqpoint{5.648195in}{3.511310in}}%
\pgfpathlineto{\pgfqpoint{5.649070in}{3.617787in}}%
\pgfpathlineto{\pgfqpoint{5.649946in}{3.903387in}}%
\pgfpathlineto{\pgfqpoint{5.650821in}{3.678350in}}%
\pgfpathlineto{\pgfqpoint{5.651696in}{3.716523in}}%
\pgfpathlineto{\pgfqpoint{5.652572in}{3.407551in}}%
\pgfpathlineto{\pgfqpoint{5.654323in}{3.264222in}}%
\pgfpathlineto{\pgfqpoint{5.656073in}{3.400113in}}%
\pgfpathlineto{\pgfqpoint{5.656949in}{3.375117in}}%
\pgfpathlineto{\pgfqpoint{5.659575in}{2.817283in}}%
\pgfpathlineto{\pgfqpoint{5.660450in}{2.793306in}}%
\pgfpathlineto{\pgfqpoint{5.662201in}{2.569818in}}%
\pgfpathlineto{\pgfqpoint{5.663076in}{2.637669in}}%
\pgfpathlineto{\pgfqpoint{5.663952in}{2.755133in}}%
\pgfpathlineto{\pgfqpoint{5.664827in}{2.956571in}}%
\pgfpathlineto{\pgfqpoint{5.666578in}{3.568211in}}%
\pgfpathlineto{\pgfqpoint{5.667453in}{3.499265in}}%
\pgfpathlineto{\pgfqpoint{5.668329in}{3.503531in}}%
\pgfpathlineto{\pgfqpoint{5.669204in}{3.532718in}}%
\pgfpathlineto{\pgfqpoint{5.670079in}{3.827833in}}%
\pgfpathlineto{\pgfqpoint{5.670955in}{3.681711in}}%
\pgfpathlineto{\pgfqpoint{5.671830in}{3.643651in}}%
\pgfpathlineto{\pgfqpoint{5.672706in}{3.579047in}}%
\pgfpathlineto{\pgfqpoint{5.673581in}{3.445535in}}%
\pgfpathlineto{\pgfqpoint{5.674456in}{3.358617in}}%
\pgfpathlineto{\pgfqpoint{5.675332in}{3.364167in}}%
\pgfpathlineto{\pgfqpoint{5.676207in}{3.485974in}}%
\pgfpathlineto{\pgfqpoint{5.677082in}{3.667363in}}%
\pgfpathlineto{\pgfqpoint{5.677958in}{3.605062in}}%
\pgfpathlineto{\pgfqpoint{5.679709in}{3.051344in}}%
\pgfpathlineto{\pgfqpoint{5.680584in}{2.936711in}}%
\pgfpathlineto{\pgfqpoint{5.681459in}{2.872900in}}%
\pgfpathlineto{\pgfqpoint{5.684085in}{2.741389in}}%
\pgfpathlineto{\pgfqpoint{5.686712in}{3.856907in}}%
\pgfpathlineto{\pgfqpoint{5.687587in}{3.843503in}}%
\pgfpathlineto{\pgfqpoint{5.688462in}{3.852942in}}%
\pgfpathlineto{\pgfqpoint{5.689338in}{3.999707in}}%
\pgfpathlineto{\pgfqpoint{5.690213in}{3.995478in}}%
\pgfpathlineto{\pgfqpoint{5.691964in}{4.414704in}}%
\pgfpathlineto{\pgfqpoint{5.692839in}{4.440379in}}%
\pgfpathlineto{\pgfqpoint{5.694590in}{3.919321in}}%
\pgfpathlineto{\pgfqpoint{5.695465in}{3.702591in}}%
\pgfpathlineto{\pgfqpoint{5.696341in}{3.818281in}}%
\pgfpathlineto{\pgfqpoint{5.697216in}{3.792870in}}%
\pgfpathlineto{\pgfqpoint{5.698091in}{3.712408in}}%
\pgfpathlineto{\pgfqpoint{5.701593in}{2.874901in}}%
\pgfpathlineto{\pgfqpoint{5.702468in}{2.858779in}}%
\pgfpathlineto{\pgfqpoint{5.703344in}{2.902502in}}%
\pgfpathlineto{\pgfqpoint{5.704219in}{2.889400in}}%
\pgfpathlineto{\pgfqpoint{5.705094in}{2.892836in}}%
\pgfpathlineto{\pgfqpoint{5.708596in}{4.221572in}}%
\pgfpathlineto{\pgfqpoint{5.709471in}{4.298825in}}%
\pgfpathlineto{\pgfqpoint{5.712097in}{4.768079in}}%
\pgfpathlineto{\pgfqpoint{5.712973in}{4.739950in}}%
\pgfpathlineto{\pgfqpoint{5.713848in}{4.837025in}}%
\pgfpathlineto{\pgfqpoint{5.714724in}{4.752938in}}%
\pgfpathlineto{\pgfqpoint{5.715599in}{4.487878in}}%
\pgfpathlineto{\pgfqpoint{5.716474in}{4.357312in}}%
\pgfpathlineto{\pgfqpoint{5.718225in}{4.461070in}}%
\pgfpathlineto{\pgfqpoint{5.722602in}{3.715693in}}%
\pgfpathlineto{\pgfqpoint{5.723477in}{3.674990in}}%
\pgfpathlineto{\pgfqpoint{5.724353in}{3.607970in}}%
\pgfpathlineto{\pgfqpoint{5.725228in}{3.407211in}}%
\pgfpathlineto{\pgfqpoint{5.726103in}{3.369642in}}%
\pgfpathlineto{\pgfqpoint{5.727854in}{4.004276in}}%
\pgfpathlineto{\pgfqpoint{5.728730in}{4.253440in}}%
\pgfpathlineto{\pgfqpoint{5.729605in}{4.245662in}}%
\pgfpathlineto{\pgfqpoint{5.730480in}{4.266202in}}%
\pgfpathlineto{\pgfqpoint{5.731356in}{4.346286in}}%
\pgfpathlineto{\pgfqpoint{5.732231in}{4.308528in}}%
\pgfpathlineto{\pgfqpoint{5.733106in}{4.531904in}}%
\pgfpathlineto{\pgfqpoint{5.733982in}{4.627998in}}%
\pgfpathlineto{\pgfqpoint{5.734857in}{4.509778in}}%
\pgfpathlineto{\pgfqpoint{5.736608in}{4.357161in}}%
\pgfpathlineto{\pgfqpoint{5.737483in}{4.229086in}}%
\pgfpathlineto{\pgfqpoint{5.738359in}{4.306263in}}%
\pgfpathlineto{\pgfqpoint{5.739234in}{4.487576in}}%
\pgfpathlineto{\pgfqpoint{5.740109in}{4.441776in}}%
\pgfpathlineto{\pgfqpoint{5.742736in}{3.660226in}}%
\pgfpathlineto{\pgfqpoint{5.743611in}{3.565152in}}%
\pgfpathlineto{\pgfqpoint{5.744486in}{3.437078in}}%
\pgfpathlineto{\pgfqpoint{5.745362in}{3.392335in}}%
\pgfpathlineto{\pgfqpoint{5.746237in}{3.328977in}}%
\pgfpathlineto{\pgfqpoint{5.747112in}{3.349630in}}%
\pgfpathlineto{\pgfqpoint{5.747988in}{3.418727in}}%
\pgfpathlineto{\pgfqpoint{5.748863in}{3.630398in}}%
\pgfpathlineto{\pgfqpoint{5.749739in}{4.244340in}}%
\pgfpathlineto{\pgfqpoint{5.750614in}{4.363089in}}%
\pgfpathlineto{\pgfqpoint{5.751489in}{4.588994in}}%
\pgfpathlineto{\pgfqpoint{5.752365in}{4.538776in}}%
\pgfpathlineto{\pgfqpoint{5.753240in}{4.570190in}}%
\pgfpathlineto{\pgfqpoint{5.754991in}{4.840234in}}%
\pgfpathlineto{\pgfqpoint{5.755866in}{4.736400in}}%
\pgfpathlineto{\pgfqpoint{5.756742in}{4.765096in}}%
\pgfpathlineto{\pgfqpoint{5.757617in}{4.427390in}}%
\pgfpathlineto{\pgfqpoint{5.758492in}{4.251741in}}%
\pgfpathlineto{\pgfqpoint{5.759368in}{4.403527in}}%
\pgfpathlineto{\pgfqpoint{5.761118in}{4.601038in}}%
\pgfpathlineto{\pgfqpoint{5.761994in}{4.544175in}}%
\pgfpathlineto{\pgfqpoint{5.762869in}{4.384308in}}%
\pgfpathlineto{\pgfqpoint{5.763745in}{4.069521in}}%
\pgfpathlineto{\pgfqpoint{5.764620in}{3.931101in}}%
\pgfpathlineto{\pgfqpoint{5.765495in}{3.869254in}}%
\pgfpathlineto{\pgfqpoint{5.766371in}{3.738763in}}%
\pgfpathlineto{\pgfqpoint{5.767246in}{3.723924in}}%
\pgfpathlineto{\pgfqpoint{5.768121in}{3.806009in}}%
\pgfpathlineto{\pgfqpoint{5.768997in}{4.083605in}}%
\pgfpathlineto{\pgfqpoint{5.769872in}{4.239696in}}%
\pgfpathlineto{\pgfqpoint{5.771623in}{4.811161in}}%
\pgfpathlineto{\pgfqpoint{5.772498in}{4.735230in}}%
\pgfpathlineto{\pgfqpoint{5.773374in}{4.758979in}}%
\pgfpathlineto{\pgfqpoint{5.774249in}{4.865268in}}%
\pgfpathlineto{\pgfqpoint{5.775124in}{5.195573in}}%
\pgfpathlineto{\pgfqpoint{5.777751in}{4.898646in}}%
\pgfpathlineto{\pgfqpoint{5.778626in}{4.605569in}}%
\pgfpathlineto{\pgfqpoint{5.779501in}{4.483763in}}%
\pgfpathlineto{\pgfqpoint{5.780377in}{4.466734in}}%
\pgfpathlineto{\pgfqpoint{5.782127in}{4.412627in}}%
\pgfpathlineto{\pgfqpoint{5.784754in}{3.873483in}}%
\pgfpathlineto{\pgfqpoint{5.785629in}{3.738083in}}%
\pgfpathlineto{\pgfqpoint{5.786504in}{3.649994in}}%
\pgfpathlineto{\pgfqpoint{5.787380in}{3.605364in}}%
\pgfpathlineto{\pgfqpoint{5.788255in}{3.577952in}}%
\pgfpathlineto{\pgfqpoint{5.789130in}{3.594188in}}%
\pgfpathlineto{\pgfqpoint{5.790881in}{3.724113in}}%
\pgfpathlineto{\pgfqpoint{5.791757in}{4.069333in}}%
\pgfpathlineto{\pgfqpoint{5.792632in}{4.212775in}}%
\pgfpathlineto{\pgfqpoint{5.793507in}{4.212737in}}%
\pgfpathlineto{\pgfqpoint{5.794383in}{4.485462in}}%
\pgfpathlineto{\pgfqpoint{5.795258in}{4.521785in}}%
\pgfpathlineto{\pgfqpoint{5.796134in}{4.516348in}}%
\pgfpathlineto{\pgfqpoint{5.797009in}{4.526958in}}%
\pgfpathlineto{\pgfqpoint{5.797884in}{4.652314in}}%
\pgfpathlineto{\pgfqpoint{5.798760in}{4.502491in}}%
\pgfpathlineto{\pgfqpoint{5.799635in}{4.495165in}}%
\pgfpathlineto{\pgfqpoint{5.800510in}{4.330806in}}%
\pgfpathlineto{\pgfqpoint{5.802261in}{4.673080in}}%
\pgfpathlineto{\pgfqpoint{5.803137in}{4.712537in}}%
\pgfpathlineto{\pgfqpoint{5.804887in}{4.324462in}}%
\pgfpathlineto{\pgfqpoint{5.805763in}{4.013602in}}%
\pgfpathlineto{\pgfqpoint{5.807513in}{3.794909in}}%
\pgfpathlineto{\pgfqpoint{5.808389in}{3.645312in}}%
\pgfpathlineto{\pgfqpoint{5.809264in}{3.616918in}}%
\pgfpathlineto{\pgfqpoint{5.810140in}{3.663436in}}%
\pgfpathlineto{\pgfqpoint{5.811015in}{3.825379in}}%
\pgfpathlineto{\pgfqpoint{5.811890in}{3.926079in}}%
\pgfpathlineto{\pgfqpoint{5.812766in}{4.256574in}}%
\pgfpathlineto{\pgfqpoint{5.814516in}{4.403489in}}%
\pgfpathlineto{\pgfqpoint{5.815392in}{4.590806in}}%
\pgfpathlineto{\pgfqpoint{5.816267in}{4.619502in}}%
\pgfpathlineto{\pgfqpoint{5.817143in}{4.878596in}}%
\pgfpathlineto{\pgfqpoint{5.818018in}{4.952413in}}%
\pgfpathlineto{\pgfqpoint{5.819769in}{4.850089in}}%
\pgfpathlineto{\pgfqpoint{5.820644in}{4.642081in}}%
\pgfpathlineto{\pgfqpoint{5.821519in}{4.636153in}}%
\pgfpathlineto{\pgfqpoint{5.822395in}{4.659034in}}%
\pgfpathlineto{\pgfqpoint{5.823270in}{4.872291in}}%
\pgfpathlineto{\pgfqpoint{5.824146in}{4.813540in}}%
\pgfpathlineto{\pgfqpoint{5.827647in}{3.950207in}}%
\pgfpathlineto{\pgfqpoint{5.828522in}{3.795626in}}%
\pgfpathlineto{\pgfqpoint{5.829398in}{3.769649in}}%
\pgfpathlineto{\pgfqpoint{5.830273in}{3.715466in}}%
\pgfpathlineto{\pgfqpoint{5.831149in}{3.805254in}}%
\pgfpathlineto{\pgfqpoint{5.832024in}{4.204657in}}%
\pgfpathlineto{\pgfqpoint{5.832899in}{4.389783in}}%
\pgfpathlineto{\pgfqpoint{5.833775in}{4.901289in}}%
\pgfpathlineto{\pgfqpoint{5.834650in}{5.208789in}}%
\pgfpathlineto{\pgfqpoint{5.835525in}{5.333616in}}%
\pgfpathlineto{\pgfqpoint{5.836401in}{5.409547in}}%
\pgfpathlineto{\pgfqpoint{5.837276in}{5.353741in}}%
\pgfpathlineto{\pgfqpoint{5.838152in}{5.489178in}}%
\pgfpathlineto{\pgfqpoint{5.839027in}{5.452515in}}%
\pgfpathlineto{\pgfqpoint{5.839902in}{5.516628in}}%
\pgfpathlineto{\pgfqpoint{5.840778in}{5.244017in}}%
\pgfpathlineto{\pgfqpoint{5.841653in}{4.864664in}}%
\pgfpathlineto{\pgfqpoint{5.842528in}{4.649633in}}%
\pgfpathlineto{\pgfqpoint{5.843404in}{4.887167in}}%
\pgfpathlineto{\pgfqpoint{5.844279in}{4.842160in}}%
\pgfpathlineto{\pgfqpoint{5.845155in}{4.867609in}}%
\pgfpathlineto{\pgfqpoint{5.846030in}{4.655712in}}%
\pgfpathlineto{\pgfqpoint{5.847781in}{4.144395in}}%
\pgfpathlineto{\pgfqpoint{5.849531in}{3.816959in}}%
\pgfpathlineto{\pgfqpoint{5.850407in}{3.718449in}}%
\pgfpathlineto{\pgfqpoint{5.852158in}{3.826172in}}%
\pgfpathlineto{\pgfqpoint{5.853908in}{4.075336in}}%
\pgfpathlineto{\pgfqpoint{5.855659in}{4.542929in}}%
\pgfpathlineto{\pgfqpoint{5.856534in}{4.592467in}}%
\pgfpathlineto{\pgfqpoint{5.857410in}{4.567019in}}%
\pgfpathlineto{\pgfqpoint{5.858285in}{4.733040in}}%
\pgfpathlineto{\pgfqpoint{5.859161in}{5.031553in}}%
\pgfpathlineto{\pgfqpoint{5.860036in}{5.023284in}}%
\pgfpathlineto{\pgfqpoint{5.860911in}{5.073804in}}%
\pgfpathlineto{\pgfqpoint{5.861787in}{4.943502in}}%
\pgfpathlineto{\pgfqpoint{5.862662in}{4.613083in}}%
\pgfpathlineto{\pgfqpoint{5.864413in}{4.354593in}}%
\pgfpathlineto{\pgfqpoint{5.865288in}{4.321781in}}%
\pgfpathlineto{\pgfqpoint{5.866164in}{4.240036in}}%
\pgfpathlineto{\pgfqpoint{5.867914in}{3.764476in}}%
\pgfpathlineto{\pgfqpoint{5.868790in}{3.474005in}}%
\pgfpathlineto{\pgfqpoint{5.869665in}{3.307719in}}%
\pgfpathlineto{\pgfqpoint{5.870540in}{3.249421in}}%
\pgfpathlineto{\pgfqpoint{5.871416in}{3.155178in}}%
\pgfpathlineto{\pgfqpoint{5.873167in}{3.087440in}}%
\pgfpathlineto{\pgfqpoint{5.874042in}{3.311457in}}%
\pgfpathlineto{\pgfqpoint{5.874917in}{3.355596in}}%
\pgfpathlineto{\pgfqpoint{5.875793in}{3.654034in}}%
\pgfpathlineto{\pgfqpoint{5.876668in}{3.749561in}}%
\pgfpathlineto{\pgfqpoint{5.877543in}{3.998461in}}%
\pgfpathlineto{\pgfqpoint{5.878419in}{4.027308in}}%
\pgfpathlineto{\pgfqpoint{5.879294in}{4.038447in}}%
\pgfpathlineto{\pgfqpoint{5.880170in}{4.280550in}}%
\pgfpathlineto{\pgfqpoint{5.881045in}{4.390425in}}%
\pgfpathlineto{\pgfqpoint{5.881920in}{4.191253in}}%
\pgfpathlineto{\pgfqpoint{5.882796in}{4.297390in}}%
\pgfpathlineto{\pgfqpoint{5.883671in}{4.183663in}}%
\pgfpathlineto{\pgfqpoint{5.884546in}{4.172978in}}%
\pgfpathlineto{\pgfqpoint{5.885422in}{4.172223in}}%
\pgfpathlineto{\pgfqpoint{5.886297in}{4.264276in}}%
\pgfpathlineto{\pgfqpoint{5.887173in}{4.205940in}}%
\pgfpathlineto{\pgfqpoint{5.889799in}{3.528036in}}%
\pgfpathlineto{\pgfqpoint{5.890674in}{3.428544in}}%
\pgfpathlineto{\pgfqpoint{5.891549in}{3.406720in}}%
\pgfpathlineto{\pgfqpoint{5.892425in}{3.344307in}}%
\pgfpathlineto{\pgfqpoint{5.893300in}{3.367716in}}%
\pgfpathlineto{\pgfqpoint{5.894176in}{3.340569in}}%
\pgfpathlineto{\pgfqpoint{5.896802in}{4.231465in}}%
\pgfpathlineto{\pgfqpoint{5.897677in}{4.414439in}}%
\pgfpathlineto{\pgfqpoint{5.899428in}{4.633359in}}%
\pgfpathlineto{\pgfqpoint{5.900303in}{4.587257in}}%
\pgfpathlineto{\pgfqpoint{5.901179in}{4.653862in}}%
\pgfpathlineto{\pgfqpoint{5.902054in}{4.751579in}}%
\pgfpathlineto{\pgfqpoint{5.902929in}{4.724393in}}%
\pgfpathlineto{\pgfqpoint{5.903805in}{4.574155in}}%
\pgfpathlineto{\pgfqpoint{5.904680in}{4.472058in}}%
\pgfpathlineto{\pgfqpoint{5.905555in}{4.333109in}}%
\pgfpathlineto{\pgfqpoint{5.906431in}{4.364750in}}%
\pgfpathlineto{\pgfqpoint{5.907306in}{4.362938in}}%
\pgfpathlineto{\pgfqpoint{5.908182in}{4.292632in}}%
\pgfpathlineto{\pgfqpoint{5.909932in}{3.861287in}}%
\pgfpathlineto{\pgfqpoint{5.910808in}{3.471966in}}%
\pgfpathlineto{\pgfqpoint{5.911683in}{3.371379in}}%
\pgfpathlineto{\pgfqpoint{5.913434in}{3.140830in}}%
\pgfpathlineto{\pgfqpoint{5.914309in}{3.111983in}}%
\pgfpathlineto{\pgfqpoint{5.915185in}{3.254216in}}%
\pgfpathlineto{\pgfqpoint{5.917811in}{4.076771in}}%
\pgfpathlineto{\pgfqpoint{5.918686in}{4.318006in}}%
\pgfpathlineto{\pgfqpoint{5.919561in}{4.472284in}}%
\pgfpathlineto{\pgfqpoint{5.921312in}{4.669154in}}%
\pgfpathlineto{\pgfqpoint{5.922188in}{4.493466in}}%
\pgfpathlineto{\pgfqpoint{5.923063in}{4.680783in}}%
\pgfpathlineto{\pgfqpoint{5.923938in}{4.674138in}}%
\pgfpathlineto{\pgfqpoint{5.924814in}{4.687768in}}%
\pgfpathlineto{\pgfqpoint{5.925689in}{4.403489in}}%
\pgfpathlineto{\pgfqpoint{5.926565in}{4.337413in}}%
\pgfpathlineto{\pgfqpoint{5.927440in}{4.408889in}}%
\pgfpathlineto{\pgfqpoint{5.928315in}{4.553615in}}%
\pgfpathlineto{\pgfqpoint{5.929191in}{4.565093in}}%
\pgfpathlineto{\pgfqpoint{5.930066in}{4.430297in}}%
\pgfpathlineto{\pgfqpoint{5.931817in}{3.550880in}}%
\pgfpathlineto{\pgfqpoint{5.932692in}{3.520636in}}%
\pgfpathlineto{\pgfqpoint{5.935318in}{3.243267in}}%
\pgfpathlineto{\pgfqpoint{5.936194in}{3.442855in}}%
\pgfpathlineto{\pgfqpoint{5.937069in}{3.397017in}}%
\pgfpathlineto{\pgfqpoint{5.937944in}{3.398149in}}%
\pgfpathlineto{\pgfqpoint{5.939695in}{3.964026in}}%
\pgfpathlineto{\pgfqpoint{5.940571in}{4.033425in}}%
\pgfpathlineto{\pgfqpoint{5.941446in}{4.147151in}}%
\pgfpathlineto{\pgfqpoint{5.942321in}{4.184418in}}%
\pgfpathlineto{\pgfqpoint{5.943197in}{4.256800in}}%
\pgfpathlineto{\pgfqpoint{5.944072in}{4.500225in}}%
\pgfpathlineto{\pgfqpoint{5.944947in}{4.465563in}}%
\pgfpathlineto{\pgfqpoint{5.946698in}{4.367166in}}%
\pgfpathlineto{\pgfqpoint{5.947574in}{4.267863in}}%
\pgfpathlineto{\pgfqpoint{5.948449in}{4.265673in}}%
\pgfpathlineto{\pgfqpoint{5.949324in}{4.310190in}}%
\pgfpathlineto{\pgfqpoint{5.950200in}{4.377965in}}%
\pgfpathlineto{\pgfqpoint{5.951950in}{3.841804in}}%
\pgfpathlineto{\pgfqpoint{5.952826in}{3.594566in}}%
\pgfpathlineto{\pgfqpoint{5.953701in}{3.461998in}}%
\pgfpathlineto{\pgfqpoint{5.955452in}{3.164315in}}%
\pgfpathlineto{\pgfqpoint{5.956327in}{3.182892in}}%
\pgfpathlineto{\pgfqpoint{5.957203in}{3.224274in}}%
\pgfpathlineto{\pgfqpoint{5.958078in}{3.537627in}}%
\pgfpathlineto{\pgfqpoint{5.961580in}{4.189667in}}%
\pgfpathlineto{\pgfqpoint{5.962455in}{4.259292in}}%
\pgfpathlineto{\pgfqpoint{5.963330in}{4.439171in}}%
\pgfpathlineto{\pgfqpoint{5.964206in}{4.474436in}}%
\pgfpathlineto{\pgfqpoint{5.965081in}{4.729755in}}%
\pgfpathlineto{\pgfqpoint{5.965956in}{4.635889in}}%
\pgfpathlineto{\pgfqpoint{5.966832in}{4.474399in}}%
\pgfpathlineto{\pgfqpoint{5.967707in}{4.377890in}}%
\pgfpathlineto{\pgfqpoint{5.968583in}{4.239054in}}%
\pgfpathlineto{\pgfqpoint{5.969458in}{4.351233in}}%
\pgfpathlineto{\pgfqpoint{5.971209in}{4.481761in}}%
\pgfpathlineto{\pgfqpoint{5.972959in}{4.008127in}}%
\pgfpathlineto{\pgfqpoint{5.973835in}{3.615597in}}%
\pgfpathlineto{\pgfqpoint{5.974710in}{3.556241in}}%
\pgfpathlineto{\pgfqpoint{5.975586in}{3.439872in}}%
\pgfpathlineto{\pgfqpoint{5.976461in}{3.390371in}}%
\pgfpathlineto{\pgfqpoint{5.977336in}{3.191048in}}%
\pgfpathlineto{\pgfqpoint{5.978212in}{3.495527in}}%
\pgfpathlineto{\pgfqpoint{5.980838in}{4.646650in}}%
\pgfpathlineto{\pgfqpoint{5.981713in}{4.729981in}}%
\pgfpathlineto{\pgfqpoint{5.982589in}{4.855942in}}%
\pgfpathlineto{\pgfqpoint{5.983464in}{4.786807in}}%
\pgfpathlineto{\pgfqpoint{5.984339in}{4.852166in}}%
\pgfpathlineto{\pgfqpoint{5.985215in}{4.816900in}}%
\pgfpathlineto{\pgfqpoint{5.986090in}{4.735192in}}%
\pgfpathlineto{\pgfqpoint{5.986965in}{4.736249in}}%
\pgfpathlineto{\pgfqpoint{5.987841in}{4.444306in}}%
\pgfpathlineto{\pgfqpoint{5.988716in}{4.349685in}}%
\pgfpathlineto{\pgfqpoint{5.989592in}{4.175998in}}%
\pgfpathlineto{\pgfqpoint{5.990467in}{4.278662in}}%
\pgfpathlineto{\pgfqpoint{5.992218in}{4.707478in}}%
\pgfpathlineto{\pgfqpoint{5.994844in}{3.751374in}}%
\pgfpathlineto{\pgfqpoint{5.997470in}{3.324446in}}%
\pgfpathlineto{\pgfqpoint{5.998345in}{3.317159in}}%
\pgfpathlineto{\pgfqpoint{5.999221in}{3.371681in}}%
\pgfpathlineto{\pgfqpoint{6.000971in}{3.796419in}}%
\pgfpathlineto{\pgfqpoint{6.002722in}{4.609043in}}%
\pgfpathlineto{\pgfqpoint{6.003598in}{4.630641in}}%
\pgfpathlineto{\pgfqpoint{6.004473in}{4.908236in}}%
\pgfpathlineto{\pgfqpoint{6.005348in}{4.784579in}}%
\pgfpathlineto{\pgfqpoint{6.006224in}{4.881994in}}%
\pgfpathlineto{\pgfqpoint{6.007099in}{4.814899in}}%
\pgfpathlineto{\pgfqpoint{6.007974in}{4.621730in}}%
\pgfpathlineto{\pgfqpoint{6.008850in}{4.571512in}}%
\pgfpathlineto{\pgfqpoint{6.010601in}{4.078432in}}%
\pgfpathlineto{\pgfqpoint{6.011476in}{4.080849in}}%
\pgfpathlineto{\pgfqpoint{6.012351in}{4.210509in}}%
\pgfpathlineto{\pgfqpoint{6.013227in}{4.268279in}}%
\pgfpathlineto{\pgfqpoint{6.014977in}{3.746616in}}%
\pgfpathlineto{\pgfqpoint{6.015853in}{3.484539in}}%
\pgfpathlineto{\pgfqpoint{6.017604in}{3.178361in}}%
\pgfpathlineto{\pgfqpoint{6.018479in}{3.156801in}}%
\pgfpathlineto{\pgfqpoint{6.019354in}{3.182854in}}%
\pgfpathlineto{\pgfqpoint{6.020230in}{3.304963in}}%
\pgfpathlineto{\pgfqpoint{6.021105in}{3.473212in}}%
\pgfpathlineto{\pgfqpoint{6.021980in}{3.553032in}}%
\pgfpathlineto{\pgfqpoint{6.022856in}{3.766288in}}%
\pgfpathlineto{\pgfqpoint{6.023731in}{3.872728in}}%
\pgfpathlineto{\pgfqpoint{6.024607in}{4.151041in}}%
\pgfpathlineto{\pgfqpoint{6.025482in}{4.092667in}}%
\pgfpathlineto{\pgfqpoint{6.027233in}{4.171694in}}%
\pgfpathlineto{\pgfqpoint{6.028983in}{4.565471in}}%
\pgfpathlineto{\pgfqpoint{6.030734in}{4.239318in}}%
\pgfpathlineto{\pgfqpoint{6.031610in}{4.013375in}}%
\pgfpathlineto{\pgfqpoint{6.033360in}{4.285874in}}%
\pgfpathlineto{\pgfqpoint{6.034236in}{4.211453in}}%
\pgfpathlineto{\pgfqpoint{6.035986in}{3.501002in}}%
\pgfpathlineto{\pgfqpoint{6.036862in}{3.145625in}}%
\pgfpathlineto{\pgfqpoint{6.037737in}{2.993008in}}%
\pgfpathlineto{\pgfqpoint{6.038613in}{2.791456in}}%
\pgfpathlineto{\pgfqpoint{6.039488in}{2.741578in}}%
\pgfpathlineto{\pgfqpoint{6.040363in}{2.724549in}}%
\pgfpathlineto{\pgfqpoint{6.041239in}{2.758267in}}%
\pgfpathlineto{\pgfqpoint{6.042989in}{3.223557in}}%
\pgfpathlineto{\pgfqpoint{6.043865in}{3.600267in}}%
\pgfpathlineto{\pgfqpoint{6.044740in}{3.857398in}}%
\pgfpathlineto{\pgfqpoint{6.046491in}{4.022324in}}%
\pgfpathlineto{\pgfqpoint{6.047366in}{3.980904in}}%
\pgfpathlineto{\pgfqpoint{6.048242in}{4.016925in}}%
\pgfpathlineto{\pgfqpoint{6.049117in}{4.200994in}}%
\pgfpathlineto{\pgfqpoint{6.049992in}{4.189063in}}%
\pgfpathlineto{\pgfqpoint{6.050868in}{4.163727in}}%
\pgfpathlineto{\pgfqpoint{6.052619in}{3.845051in}}%
\pgfpathlineto{\pgfqpoint{6.053494in}{4.047886in}}%
\pgfpathlineto{\pgfqpoint{6.054369in}{3.983585in}}%
\pgfpathlineto{\pgfqpoint{6.055245in}{3.846675in}}%
\pgfpathlineto{\pgfqpoint{6.057871in}{2.988288in}}%
\pgfpathlineto{\pgfqpoint{6.060497in}{2.723530in}}%
\pgfpathlineto{\pgfqpoint{6.061372in}{2.703556in}}%
\pgfpathlineto{\pgfqpoint{6.062248in}{2.757134in}}%
\pgfpathlineto{\pgfqpoint{6.063123in}{3.018985in}}%
\pgfpathlineto{\pgfqpoint{6.063999in}{3.196032in}}%
\pgfpathlineto{\pgfqpoint{6.064874in}{3.607819in}}%
\pgfpathlineto{\pgfqpoint{6.065749in}{3.842446in}}%
\pgfpathlineto{\pgfqpoint{6.066625in}{3.776105in}}%
\pgfpathlineto{\pgfqpoint{6.067500in}{3.961081in}}%
\pgfpathlineto{\pgfqpoint{6.068375in}{3.983962in}}%
\pgfpathlineto{\pgfqpoint{6.069251in}{3.994421in}}%
\pgfpathlineto{\pgfqpoint{6.070126in}{3.908258in}}%
\pgfpathlineto{\pgfqpoint{6.071002in}{3.952359in}}%
\pgfpathlineto{\pgfqpoint{6.071877in}{3.897459in}}%
\pgfpathlineto{\pgfqpoint{6.072752in}{3.749071in}}%
\pgfpathlineto{\pgfqpoint{6.073628in}{3.736799in}}%
\pgfpathlineto{\pgfqpoint{6.074503in}{3.784525in}}%
\pgfpathlineto{\pgfqpoint{6.075378in}{3.808426in}}%
\pgfpathlineto{\pgfqpoint{6.076254in}{3.772783in}}%
\pgfpathlineto{\pgfqpoint{6.077129in}{3.540459in}}%
\pgfpathlineto{\pgfqpoint{6.078005in}{3.175151in}}%
\pgfpathlineto{\pgfqpoint{6.079755in}{2.743844in}}%
\pgfpathlineto{\pgfqpoint{6.080631in}{2.605084in}}%
\pgfpathlineto{\pgfqpoint{6.081506in}{2.607462in}}%
\pgfpathlineto{\pgfqpoint{6.082381in}{2.580315in}}%
\pgfpathlineto{\pgfqpoint{6.083257in}{2.519524in}}%
\pgfpathlineto{\pgfqpoint{6.084132in}{2.540065in}}%
\pgfpathlineto{\pgfqpoint{6.085008in}{2.567930in}}%
\pgfpathlineto{\pgfqpoint{6.085883in}{2.870635in}}%
\pgfpathlineto{\pgfqpoint{6.086758in}{3.064294in}}%
\pgfpathlineto{\pgfqpoint{6.087634in}{3.077321in}}%
\pgfpathlineto{\pgfqpoint{6.088509in}{3.067315in}}%
\pgfpathlineto{\pgfqpoint{6.089384in}{3.072261in}}%
\pgfpathlineto{\pgfqpoint{6.090260in}{3.245872in}}%
\pgfpathlineto{\pgfqpoint{6.091135in}{3.481821in}}%
\pgfpathlineto{\pgfqpoint{6.092011in}{3.529244in}}%
\pgfpathlineto{\pgfqpoint{6.092886in}{3.454824in}}%
\pgfpathlineto{\pgfqpoint{6.093761in}{3.415669in}}%
\pgfpathlineto{\pgfqpoint{6.094637in}{3.406116in}}%
\pgfpathlineto{\pgfqpoint{6.095512in}{3.474345in}}%
\pgfpathlineto{\pgfqpoint{6.096387in}{3.365527in}}%
\pgfpathlineto{\pgfqpoint{6.097263in}{3.399244in}}%
\pgfpathlineto{\pgfqpoint{6.098138in}{3.330034in}}%
\pgfpathlineto{\pgfqpoint{6.100764in}{2.690454in}}%
\pgfpathlineto{\pgfqpoint{6.101640in}{2.564305in}}%
\pgfpathlineto{\pgfqpoint{6.103390in}{2.422147in}}%
\pgfpathlineto{\pgfqpoint{6.104266in}{2.645409in}}%
\pgfpathlineto{\pgfqpoint{6.105141in}{2.751018in}}%
\pgfpathlineto{\pgfqpoint{6.106017in}{2.943507in}}%
\pgfpathlineto{\pgfqpoint{6.106892in}{3.373758in}}%
\pgfpathlineto{\pgfqpoint{6.107767in}{3.537136in}}%
\pgfpathlineto{\pgfqpoint{6.108643in}{3.615597in}}%
\pgfpathlineto{\pgfqpoint{6.109518in}{3.613898in}}%
\pgfpathlineto{\pgfqpoint{6.110393in}{3.666494in}}%
\pgfpathlineto{\pgfqpoint{6.112144in}{3.990645in}}%
\pgfpathlineto{\pgfqpoint{6.113020in}{3.999179in}}%
\pgfpathlineto{\pgfqpoint{6.114770in}{3.689413in}}%
\pgfpathlineto{\pgfqpoint{6.115646in}{3.637081in}}%
\pgfpathlineto{\pgfqpoint{6.116521in}{3.782826in}}%
\pgfpathlineto{\pgfqpoint{6.117396in}{3.746881in}}%
\pgfpathlineto{\pgfqpoint{6.118272in}{3.837197in}}%
\pgfpathlineto{\pgfqpoint{6.119147in}{3.710369in}}%
\pgfpathlineto{\pgfqpoint{6.120023in}{3.452558in}}%
\pgfpathlineto{\pgfqpoint{6.120898in}{3.533171in}}%
\pgfpathlineto{\pgfqpoint{6.122649in}{3.247307in}}%
\pgfpathlineto{\pgfqpoint{6.123524in}{3.232166in}}%
\pgfpathlineto{\pgfqpoint{6.124399in}{3.209209in}}%
\pgfpathlineto{\pgfqpoint{6.125275in}{3.389427in}}%
\pgfpathlineto{\pgfqpoint{6.127901in}{4.299391in}}%
\pgfpathlineto{\pgfqpoint{6.128776in}{4.398166in}}%
\pgfpathlineto{\pgfqpoint{6.129652in}{4.058005in}}%
\pgfpathlineto{\pgfqpoint{6.130527in}{4.211642in}}%
\pgfpathlineto{\pgfqpoint{6.131402in}{4.209565in}}%
\pgfpathlineto{\pgfqpoint{6.132278in}{4.401639in}}%
\pgfpathlineto{\pgfqpoint{6.133153in}{4.506002in}}%
\pgfpathlineto{\pgfqpoint{6.134029in}{4.673609in}}%
\pgfpathlineto{\pgfqpoint{6.134904in}{4.424558in}}%
\pgfpathlineto{\pgfqpoint{6.135779in}{4.260047in}}%
\pgfpathlineto{\pgfqpoint{6.136655in}{4.163689in}}%
\pgfpathlineto{\pgfqpoint{6.137530in}{4.330315in}}%
\pgfpathlineto{\pgfqpoint{6.138405in}{4.348854in}}%
\pgfpathlineto{\pgfqpoint{6.139281in}{4.304224in}}%
\pgfpathlineto{\pgfqpoint{6.141032in}{3.912260in}}%
\pgfpathlineto{\pgfqpoint{6.141907in}{3.564888in}}%
\pgfpathlineto{\pgfqpoint{6.142782in}{3.472608in}}%
\pgfpathlineto{\pgfqpoint{6.143658in}{3.300319in}}%
\pgfpathlineto{\pgfqpoint{6.145408in}{3.130824in}}%
\pgfpathlineto{\pgfqpoint{6.146284in}{3.234016in}}%
\pgfpathlineto{\pgfqpoint{6.147159in}{3.480310in}}%
\pgfpathlineto{\pgfqpoint{6.148035in}{3.534304in}}%
\pgfpathlineto{\pgfqpoint{6.148910in}{3.937558in}}%
\pgfpathlineto{\pgfqpoint{6.150661in}{4.174035in}}%
\pgfpathlineto{\pgfqpoint{6.151536in}{4.130916in}}%
\pgfpathlineto{\pgfqpoint{6.152411in}{4.167578in}}%
\pgfpathlineto{\pgfqpoint{6.153287in}{4.238563in}}%
\pgfpathlineto{\pgfqpoint{6.154162in}{4.261633in}}%
\pgfpathlineto{\pgfqpoint{6.155038in}{4.431808in}}%
\pgfpathlineto{\pgfqpoint{6.156788in}{4.072957in}}%
\pgfpathlineto{\pgfqpoint{6.157664in}{3.974447in}}%
\pgfpathlineto{\pgfqpoint{6.158539in}{3.995289in}}%
\pgfpathlineto{\pgfqpoint{6.159414in}{4.035955in}}%
\pgfpathlineto{\pgfqpoint{6.160290in}{4.159423in}}%
\pgfpathlineto{\pgfqpoint{6.162041in}{3.701571in}}%
\pgfpathlineto{\pgfqpoint{6.162916in}{3.410949in}}%
\pgfpathlineto{\pgfqpoint{6.164667in}{3.087402in}}%
\pgfpathlineto{\pgfqpoint{6.165542in}{3.130484in}}%
\pgfpathlineto{\pgfqpoint{6.167293in}{3.092160in}}%
\pgfpathlineto{\pgfqpoint{6.168168in}{3.178738in}}%
\pgfpathlineto{\pgfqpoint{6.169044in}{3.177077in}}%
\pgfpathlineto{\pgfqpoint{6.171670in}{4.050605in}}%
\pgfpathlineto{\pgfqpoint{6.172545in}{3.780674in}}%
\pgfpathlineto{\pgfqpoint{6.173420in}{4.094706in}}%
\pgfpathlineto{\pgfqpoint{6.174296in}{4.038522in}}%
\pgfpathlineto{\pgfqpoint{6.175171in}{4.062612in}}%
\pgfpathlineto{\pgfqpoint{6.176047in}{4.223120in}}%
\pgfpathlineto{\pgfqpoint{6.176922in}{4.127857in}}%
\pgfpathlineto{\pgfqpoint{6.177797in}{4.103503in}}%
\pgfpathlineto{\pgfqpoint{6.178673in}{4.067407in}}%
\pgfpathlineto{\pgfqpoint{6.180423in}{4.407227in}}%
\pgfpathlineto{\pgfqpoint{6.181299in}{4.403829in}}%
\pgfpathlineto{\pgfqpoint{6.182174in}{4.223385in}}%
\pgfpathlineto{\pgfqpoint{6.183050in}{3.970860in}}%
\pgfpathlineto{\pgfqpoint{6.183925in}{3.615219in}}%
\pgfpathlineto{\pgfqpoint{6.184800in}{3.553787in}}%
\pgfpathlineto{\pgfqpoint{6.186551in}{3.303075in}}%
\pgfpathlineto{\pgfqpoint{6.187426in}{3.240322in}}%
\pgfpathlineto{\pgfqpoint{6.188302in}{3.447310in}}%
\pgfpathlineto{\pgfqpoint{6.189177in}{3.742123in}}%
\pgfpathlineto{\pgfqpoint{6.190053in}{3.741368in}}%
\pgfpathlineto{\pgfqpoint{6.190928in}{4.201296in}}%
\pgfpathlineto{\pgfqpoint{6.192679in}{4.664321in}}%
\pgfpathlineto{\pgfqpoint{6.193554in}{4.571210in}}%
\pgfpathlineto{\pgfqpoint{6.194430in}{4.671683in}}%
\pgfpathlineto{\pgfqpoint{6.195305in}{4.573400in}}%
\pgfpathlineto{\pgfqpoint{6.196180in}{4.669191in}}%
\pgfpathlineto{\pgfqpoint{6.197931in}{4.512723in}}%
\pgfpathlineto{\pgfqpoint{6.198806in}{4.314456in}}%
\pgfpathlineto{\pgfqpoint{6.199682in}{4.282966in}}%
\pgfpathlineto{\pgfqpoint{6.200557in}{4.315627in}}%
\pgfpathlineto{\pgfqpoint{6.202308in}{4.661375in}}%
\pgfpathlineto{\pgfqpoint{6.203183in}{4.440001in}}%
\pgfpathlineto{\pgfqpoint{6.205809in}{3.571042in}}%
\pgfpathlineto{\pgfqpoint{6.206685in}{3.533247in}}%
\pgfpathlineto{\pgfqpoint{6.207560in}{3.454862in}}%
\pgfpathlineto{\pgfqpoint{6.208436in}{3.427261in}}%
\pgfpathlineto{\pgfqpoint{6.209311in}{3.588562in}}%
\pgfpathlineto{\pgfqpoint{6.210186in}{3.912789in}}%
\pgfpathlineto{\pgfqpoint{6.211062in}{3.966216in}}%
\pgfpathlineto{\pgfqpoint{6.211937in}{4.050756in}}%
\pgfpathlineto{\pgfqpoint{6.212812in}{4.175810in}}%
\pgfpathlineto{\pgfqpoint{6.213688in}{4.352856in}}%
\pgfpathlineto{\pgfqpoint{6.214563in}{4.446949in}}%
\pgfpathlineto{\pgfqpoint{6.215439in}{4.670966in}}%
\pgfpathlineto{\pgfqpoint{6.216314in}{4.653409in}}%
\pgfpathlineto{\pgfqpoint{6.217189in}{4.848277in}}%
\pgfpathlineto{\pgfqpoint{6.218065in}{4.776877in}}%
\pgfpathlineto{\pgfqpoint{6.219815in}{4.448912in}}%
\pgfpathlineto{\pgfqpoint{6.220691in}{4.396731in}}%
\pgfpathlineto{\pgfqpoint{6.221566in}{4.497733in}}%
\pgfpathlineto{\pgfqpoint{6.223317in}{4.817353in}}%
\pgfpathlineto{\pgfqpoint{6.227694in}{3.532076in}}%
\pgfpathlineto{\pgfqpoint{6.228569in}{3.576555in}}%
\pgfpathlineto{\pgfqpoint{6.229445in}{3.437946in}}%
\pgfpathlineto{\pgfqpoint{6.230320in}{3.496697in}}%
\pgfpathlineto{\pgfqpoint{6.231195in}{3.663700in}}%
\pgfpathlineto{\pgfqpoint{6.232071in}{3.595207in}}%
\pgfpathlineto{\pgfqpoint{6.232946in}{3.669137in}}%
\pgfpathlineto{\pgfqpoint{6.233821in}{3.612614in}}%
\pgfpathlineto{\pgfqpoint{6.234697in}{3.839840in}}%
\pgfpathlineto{\pgfqpoint{6.235572in}{3.892248in}}%
\pgfpathlineto{\pgfqpoint{6.236448in}{3.879222in}}%
\pgfpathlineto{\pgfqpoint{6.237323in}{3.880845in}}%
\pgfpathlineto{\pgfqpoint{6.238198in}{3.805292in}}%
\pgfpathlineto{\pgfqpoint{6.239074in}{3.862231in}}%
\pgfpathlineto{\pgfqpoint{6.239949in}{4.016396in}}%
\pgfpathlineto{\pgfqpoint{6.241700in}{3.794833in}}%
\pgfpathlineto{\pgfqpoint{6.242575in}{3.900593in}}%
\pgfpathlineto{\pgfqpoint{6.244326in}{4.217645in}}%
\pgfpathlineto{\pgfqpoint{6.247827in}{3.220876in}}%
\pgfpathlineto{\pgfqpoint{6.248703in}{3.097484in}}%
\pgfpathlineto{\pgfqpoint{6.249578in}{3.051230in}}%
\pgfpathlineto{\pgfqpoint{6.250454in}{3.065616in}}%
\pgfpathlineto{\pgfqpoint{6.251329in}{3.186705in}}%
\pgfpathlineto{\pgfqpoint{6.252204in}{3.353444in}}%
\pgfpathlineto{\pgfqpoint{6.255706in}{4.453632in}}%
\pgfpathlineto{\pgfqpoint{6.256581in}{4.513214in}}%
\pgfpathlineto{\pgfqpoint{6.257457in}{4.506266in}}%
\pgfpathlineto{\pgfqpoint{6.258332in}{4.449252in}}%
\pgfpathlineto{\pgfqpoint{6.259207in}{4.640495in}}%
\pgfpathlineto{\pgfqpoint{6.260083in}{4.498903in}}%
\pgfpathlineto{\pgfqpoint{6.260958in}{4.439812in}}%
\pgfpathlineto{\pgfqpoint{6.262709in}{4.171430in}}%
\pgfpathlineto{\pgfqpoint{6.263584in}{4.222327in}}%
\pgfpathlineto{\pgfqpoint{6.264460in}{4.353611in}}%
\pgfpathlineto{\pgfqpoint{6.265335in}{4.561204in}}%
\pgfpathlineto{\pgfqpoint{6.267961in}{3.620543in}}%
\pgfpathlineto{\pgfqpoint{6.268836in}{3.503229in}}%
\pgfpathlineto{\pgfqpoint{6.269712in}{3.300356in}}%
\pgfpathlineto{\pgfqpoint{6.271463in}{3.089441in}}%
\pgfpathlineto{\pgfqpoint{6.272338in}{3.291785in}}%
\pgfpathlineto{\pgfqpoint{6.273213in}{3.782675in}}%
\pgfpathlineto{\pgfqpoint{6.274089in}{4.000160in}}%
\pgfpathlineto{\pgfqpoint{6.275839in}{4.931722in}}%
\pgfpathlineto{\pgfqpoint{6.276715in}{4.921753in}}%
\pgfpathlineto{\pgfqpoint{6.277590in}{4.995910in}}%
\pgfpathlineto{\pgfqpoint{6.278466in}{4.866061in}}%
\pgfpathlineto{\pgfqpoint{6.279341in}{5.080676in}}%
\pgfpathlineto{\pgfqpoint{6.280216in}{4.925982in}}%
\pgfpathlineto{\pgfqpoint{6.281092in}{4.948713in}}%
\pgfpathlineto{\pgfqpoint{6.281967in}{4.600510in}}%
\pgfpathlineto{\pgfqpoint{6.282842in}{4.390501in}}%
\pgfpathlineto{\pgfqpoint{6.283718in}{4.299165in}}%
\pgfpathlineto{\pgfqpoint{6.284593in}{4.317326in}}%
\pgfpathlineto{\pgfqpoint{6.285469in}{4.459749in}}%
\pgfpathlineto{\pgfqpoint{6.286344in}{4.364070in}}%
\pgfpathlineto{\pgfqpoint{6.288970in}{3.356389in}}%
\pgfpathlineto{\pgfqpoint{6.289845in}{3.133958in}}%
\pgfpathlineto{\pgfqpoint{6.291596in}{2.919078in}}%
\pgfpathlineto{\pgfqpoint{6.292472in}{2.929272in}}%
\pgfpathlineto{\pgfqpoint{6.293347in}{2.975262in}}%
\pgfpathlineto{\pgfqpoint{6.294222in}{3.106734in}}%
\pgfpathlineto{\pgfqpoint{6.295098in}{3.310891in}}%
\pgfpathlineto{\pgfqpoint{6.296848in}{4.001746in}}%
\pgfpathlineto{\pgfqpoint{6.297724in}{4.097689in}}%
\pgfpathlineto{\pgfqpoint{6.298599in}{4.047546in}}%
\pgfpathlineto{\pgfqpoint{6.299475in}{4.073637in}}%
\pgfpathlineto{\pgfqpoint{6.300350in}{4.243396in}}%
\pgfpathlineto{\pgfqpoint{6.301225in}{4.266580in}}%
\pgfpathlineto{\pgfqpoint{6.302101in}{4.350364in}}%
\pgfpathlineto{\pgfqpoint{6.302976in}{4.251363in}}%
\pgfpathlineto{\pgfqpoint{6.303851in}{4.114302in}}%
\pgfpathlineto{\pgfqpoint{6.304727in}{4.027195in}}%
\pgfpathlineto{\pgfqpoint{6.305602in}{3.983358in}}%
\pgfpathlineto{\pgfqpoint{6.306478in}{4.000160in}}%
\pgfpathlineto{\pgfqpoint{6.307353in}{3.878353in}}%
\pgfpathlineto{\pgfqpoint{6.309979in}{3.007431in}}%
\pgfpathlineto{\pgfqpoint{6.312605in}{2.720019in}}%
\pgfpathlineto{\pgfqpoint{6.313481in}{2.687094in}}%
\pgfpathlineto{\pgfqpoint{6.314356in}{2.665270in}}%
\pgfpathlineto{\pgfqpoint{6.315231in}{2.732063in}}%
\pgfpathlineto{\pgfqpoint{6.316107in}{2.879357in}}%
\pgfpathlineto{\pgfqpoint{6.316982in}{3.215288in}}%
\pgfpathlineto{\pgfqpoint{6.318733in}{3.620241in}}%
\pgfpathlineto{\pgfqpoint{6.319608in}{3.660831in}}%
\pgfpathlineto{\pgfqpoint{6.322234in}{3.894627in}}%
\pgfpathlineto{\pgfqpoint{6.323110in}{4.045319in}}%
\pgfpathlineto{\pgfqpoint{6.323985in}{4.005748in}}%
\pgfpathlineto{\pgfqpoint{6.324861in}{3.798873in}}%
\pgfpathlineto{\pgfqpoint{6.325736in}{3.732231in}}%
\pgfpathlineto{\pgfqpoint{6.326611in}{3.785318in}}%
\pgfpathlineto{\pgfqpoint{6.327487in}{3.788867in}}%
\pgfpathlineto{\pgfqpoint{6.328362in}{3.965461in}}%
\pgfpathlineto{\pgfqpoint{6.330988in}{3.109189in}}%
\pgfpathlineto{\pgfqpoint{6.331864in}{2.932784in}}%
\pgfpathlineto{\pgfqpoint{6.332739in}{2.868935in}}%
\pgfpathlineto{\pgfqpoint{6.333614in}{2.764875in}}%
\pgfpathlineto{\pgfqpoint{6.334490in}{2.701819in}}%
\pgfpathlineto{\pgfqpoint{6.335365in}{2.820945in}}%
\pgfpathlineto{\pgfqpoint{6.336240in}{3.118213in}}%
\pgfpathlineto{\pgfqpoint{6.337116in}{3.248666in}}%
\pgfpathlineto{\pgfqpoint{6.337991in}{3.591243in}}%
\pgfpathlineto{\pgfqpoint{6.338867in}{4.052002in}}%
\pgfpathlineto{\pgfqpoint{6.339742in}{4.266240in}}%
\pgfpathlineto{\pgfqpoint{6.340617in}{4.172789in}}%
\pgfpathlineto{\pgfqpoint{6.341493in}{4.239281in}}%
\pgfpathlineto{\pgfqpoint{6.342368in}{4.443211in}}%
\pgfpathlineto{\pgfqpoint{6.343243in}{4.389066in}}%
\pgfpathlineto{\pgfqpoint{6.344119in}{4.500074in}}%
\pgfpathlineto{\pgfqpoint{6.344994in}{4.386347in}}%
\pgfpathlineto{\pgfqpoint{6.345870in}{4.185853in}}%
\pgfpathlineto{\pgfqpoint{6.346745in}{4.134729in}}%
\pgfpathlineto{\pgfqpoint{6.347620in}{4.064500in}}%
\pgfpathlineto{\pgfqpoint{6.348496in}{4.213907in}}%
\pgfpathlineto{\pgfqpoint{6.349371in}{4.449705in}}%
\pgfpathlineto{\pgfqpoint{6.351122in}{3.930497in}}%
\pgfpathlineto{\pgfqpoint{6.351997in}{3.687110in}}%
\pgfpathlineto{\pgfqpoint{6.352873in}{3.576668in}}%
\pgfpathlineto{\pgfqpoint{6.353748in}{3.395695in}}%
\pgfpathlineto{\pgfqpoint{6.354623in}{3.368849in}}%
\pgfpathlineto{\pgfqpoint{6.355499in}{3.281553in}}%
\pgfpathlineto{\pgfqpoint{6.356374in}{3.418841in}}%
\pgfpathlineto{\pgfqpoint{6.357249in}{3.755074in}}%
\pgfpathlineto{\pgfqpoint{6.358125in}{3.945411in}}%
\pgfpathlineto{\pgfqpoint{6.359876in}{4.490144in}}%
\pgfpathlineto{\pgfqpoint{6.360751in}{4.483385in}}%
\pgfpathlineto{\pgfqpoint{6.361626in}{4.573928in}}%
\pgfpathlineto{\pgfqpoint{6.362502in}{4.503548in}}%
\pgfpathlineto{\pgfqpoint{6.363377in}{4.522427in}}%
\pgfpathlineto{\pgfqpoint{6.364252in}{4.752221in}}%
\pgfpathlineto{\pgfqpoint{6.365128in}{4.683388in}}%
\pgfpathlineto{\pgfqpoint{6.366003in}{4.590693in}}%
\pgfpathlineto{\pgfqpoint{6.366879in}{4.366260in}}%
\pgfpathlineto{\pgfqpoint{6.367754in}{4.229161in}}%
\pgfpathlineto{\pgfqpoint{6.368629in}{4.310718in}}%
\pgfpathlineto{\pgfqpoint{6.369505in}{4.328238in}}%
\pgfpathlineto{\pgfqpoint{6.370380in}{4.421387in}}%
\pgfpathlineto{\pgfqpoint{6.371255in}{4.154363in}}%
\pgfpathlineto{\pgfqpoint{6.373006in}{3.447763in}}%
\pgfpathlineto{\pgfqpoint{6.373882in}{3.324559in}}%
\pgfpathlineto{\pgfqpoint{6.374757in}{3.142566in}}%
\pgfpathlineto{\pgfqpoint{6.376508in}{3.001692in}}%
\pgfpathlineto{\pgfqpoint{6.377383in}{3.083853in}}%
\pgfpathlineto{\pgfqpoint{6.378258in}{3.136110in}}%
\pgfpathlineto{\pgfqpoint{6.379134in}{3.157481in}}%
\pgfpathlineto{\pgfqpoint{6.380009in}{3.400943in}}%
\pgfpathlineto{\pgfqpoint{6.381760in}{3.485068in}}%
\pgfpathlineto{\pgfqpoint{6.382635in}{3.534493in}}%
\pgfpathlineto{\pgfqpoint{6.383511in}{3.562811in}}%
\pgfpathlineto{\pgfqpoint{6.384386in}{3.825077in}}%
\pgfpathlineto{\pgfqpoint{6.385261in}{3.872916in}}%
\pgfpathlineto{\pgfqpoint{6.386137in}{3.903236in}}%
\pgfpathlineto{\pgfqpoint{6.387012in}{3.902821in}}%
\pgfpathlineto{\pgfqpoint{6.387888in}{3.763645in}}%
\pgfpathlineto{\pgfqpoint{6.388763in}{3.764174in}}%
\pgfpathlineto{\pgfqpoint{6.389638in}{3.760964in}}%
\pgfpathlineto{\pgfqpoint{6.390514in}{3.848563in}}%
\pgfpathlineto{\pgfqpoint{6.391389in}{4.062687in}}%
\pgfpathlineto{\pgfqpoint{6.392264in}{3.809445in}}%
\pgfpathlineto{\pgfqpoint{6.393140in}{3.387351in}}%
\pgfpathlineto{\pgfqpoint{6.394015in}{3.114966in}}%
\pgfpathlineto{\pgfqpoint{6.394891in}{3.009848in}}%
\pgfpathlineto{\pgfqpoint{6.396641in}{2.673916in}}%
\pgfpathlineto{\pgfqpoint{6.397517in}{2.608029in}}%
\pgfpathlineto{\pgfqpoint{6.398392in}{2.639859in}}%
\pgfpathlineto{\pgfqpoint{6.399267in}{2.951172in}}%
\pgfpathlineto{\pgfqpoint{6.400143in}{3.001201in}}%
\pgfpathlineto{\pgfqpoint{6.401018in}{3.116174in}}%
\pgfpathlineto{\pgfqpoint{6.401894in}{3.070298in}}%
\pgfpathlineto{\pgfqpoint{6.403644in}{3.351216in}}%
\pgfpathlineto{\pgfqpoint{6.404520in}{3.366584in}}%
\pgfpathlineto{\pgfqpoint{6.405395in}{3.525355in}}%
\pgfpathlineto{\pgfqpoint{6.406270in}{3.497188in}}%
\pgfpathlineto{\pgfqpoint{6.407146in}{3.558469in}}%
\pgfpathlineto{\pgfqpoint{6.408021in}{3.560961in}}%
\pgfpathlineto{\pgfqpoint{6.408897in}{3.454144in}}%
\pgfpathlineto{\pgfqpoint{6.409772in}{3.447234in}}%
\pgfpathlineto{\pgfqpoint{6.410647in}{3.271056in}}%
\pgfpathlineto{\pgfqpoint{6.411523in}{3.372247in}}%
\pgfpathlineto{\pgfqpoint{6.412398in}{3.360240in}}%
\pgfpathlineto{\pgfqpoint{6.414149in}{2.739992in}}%
\pgfpathlineto{\pgfqpoint{6.415024in}{2.625548in}}%
\pgfpathlineto{\pgfqpoint{6.415900in}{2.191371in}}%
\pgfpathlineto{\pgfqpoint{6.416775in}{2.228298in}}%
\pgfpathlineto{\pgfqpoint{6.418526in}{2.144514in}}%
\pgfpathlineto{\pgfqpoint{6.419401in}{2.222332in}}%
\pgfpathlineto{\pgfqpoint{6.420276in}{2.715903in}}%
\pgfpathlineto{\pgfqpoint{6.421152in}{2.915868in}}%
\pgfpathlineto{\pgfqpoint{6.422903in}{3.506137in}}%
\pgfpathlineto{\pgfqpoint{6.423778in}{3.524827in}}%
\pgfpathlineto{\pgfqpoint{6.424653in}{3.638063in}}%
\pgfpathlineto{\pgfqpoint{6.425529in}{3.671063in}}%
\pgfpathlineto{\pgfqpoint{6.426404in}{3.724717in}}%
\pgfpathlineto{\pgfqpoint{6.427279in}{3.856227in}}%
\pgfpathlineto{\pgfqpoint{6.428155in}{4.038296in}}%
\pgfpathlineto{\pgfqpoint{6.430781in}{3.412120in}}%
\pgfpathlineto{\pgfqpoint{6.432532in}{3.493714in}}%
\pgfpathlineto{\pgfqpoint{6.434282in}{3.290993in}}%
\pgfpathlineto{\pgfqpoint{6.436033in}{2.656850in}}%
\pgfpathlineto{\pgfqpoint{6.437784in}{2.435249in}}%
\pgfpathlineto{\pgfqpoint{6.438659in}{2.431398in}}%
\pgfpathlineto{\pgfqpoint{6.439535in}{2.423166in}}%
\pgfpathlineto{\pgfqpoint{6.440410in}{2.463869in}}%
\pgfpathlineto{\pgfqpoint{6.442161in}{2.666289in}}%
\pgfpathlineto{\pgfqpoint{6.443036in}{3.029368in}}%
\pgfpathlineto{\pgfqpoint{6.444787in}{2.979377in}}%
\pgfpathlineto{\pgfqpoint{6.445662in}{3.175227in}}%
\pgfpathlineto{\pgfqpoint{6.446538in}{3.257199in}}%
\pgfpathlineto{\pgfqpoint{6.447413in}{3.158840in}}%
\pgfpathlineto{\pgfqpoint{6.448288in}{3.289369in}}%
\pgfpathlineto{\pgfqpoint{6.449164in}{3.227069in}}%
\pgfpathlineto{\pgfqpoint{6.450039in}{3.197353in}}%
\pgfpathlineto{\pgfqpoint{6.450915in}{2.943620in}}%
\pgfpathlineto{\pgfqpoint{6.451790in}{2.839107in}}%
\pgfpathlineto{\pgfqpoint{6.452665in}{2.776995in}}%
\pgfpathlineto{\pgfqpoint{6.453541in}{2.806748in}}%
\pgfpathlineto{\pgfqpoint{6.454416in}{2.801500in}}%
\pgfpathlineto{\pgfqpoint{6.456167in}{2.389675in}}%
\pgfpathlineto{\pgfqpoint{6.457042in}{2.136433in}}%
\pgfpathlineto{\pgfqpoint{6.458793in}{2.078022in}}%
\pgfpathlineto{\pgfqpoint{6.460544in}{1.846869in}}%
\pgfpathlineto{\pgfqpoint{6.461419in}{1.953799in}}%
\pgfpathlineto{\pgfqpoint{6.462295in}{2.163279in}}%
\pgfpathlineto{\pgfqpoint{6.463170in}{2.112117in}}%
\pgfpathlineto{\pgfqpoint{6.465796in}{2.565665in}}%
\pgfpathlineto{\pgfqpoint{6.466671in}{2.674483in}}%
\pgfpathlineto{\pgfqpoint{6.467547in}{2.725380in}}%
\pgfpathlineto{\pgfqpoint{6.468422in}{3.063577in}}%
\pgfpathlineto{\pgfqpoint{6.469298in}{3.054402in}}%
\pgfpathlineto{\pgfqpoint{6.470173in}{3.035372in}}%
\pgfpathlineto{\pgfqpoint{6.471048in}{3.217478in}}%
\pgfpathlineto{\pgfqpoint{6.471924in}{2.935729in}}%
\pgfpathlineto{\pgfqpoint{6.472799in}{2.978169in}}%
\pgfpathlineto{\pgfqpoint{6.473674in}{2.984399in}}%
\pgfpathlineto{\pgfqpoint{6.474550in}{3.081588in}}%
\pgfpathlineto{\pgfqpoint{6.475425in}{3.123083in}}%
\pgfpathlineto{\pgfqpoint{6.476301in}{2.951550in}}%
\pgfpathlineto{\pgfqpoint{6.477176in}{2.671198in}}%
\pgfpathlineto{\pgfqpoint{6.479802in}{2.336210in}}%
\pgfpathlineto{\pgfqpoint{6.480677in}{2.229129in}}%
\pgfpathlineto{\pgfqpoint{6.481553in}{2.229620in}}%
\pgfpathlineto{\pgfqpoint{6.482428in}{2.174871in}}%
\pgfpathlineto{\pgfqpoint{6.484179in}{2.494038in}}%
\pgfpathlineto{\pgfqpoint{6.485930in}{3.107225in}}%
\pgfpathlineto{\pgfqpoint{6.487680in}{3.259691in}}%
\pgfpathlineto{\pgfqpoint{6.489431in}{3.354010in}}%
\pgfpathlineto{\pgfqpoint{6.490307in}{3.568022in}}%
\pgfpathlineto{\pgfqpoint{6.491182in}{3.563302in}}%
\pgfpathlineto{\pgfqpoint{6.492057in}{3.474722in}}%
\pgfpathlineto{\pgfqpoint{6.492933in}{3.276078in}}%
\pgfpathlineto{\pgfqpoint{6.493808in}{3.173868in}}%
\pgfpathlineto{\pgfqpoint{6.494683in}{3.117911in}}%
\pgfpathlineto{\pgfqpoint{6.495559in}{3.195578in}}%
\pgfpathlineto{\pgfqpoint{6.496434in}{3.417821in}}%
\pgfpathlineto{\pgfqpoint{6.497310in}{3.361033in}}%
\pgfpathlineto{\pgfqpoint{6.499060in}{2.852171in}}%
\pgfpathlineto{\pgfqpoint{6.499936in}{2.749092in}}%
\pgfpathlineto{\pgfqpoint{6.500811in}{2.734820in}}%
\pgfpathlineto{\pgfqpoint{6.501686in}{2.697855in}}%
\pgfpathlineto{\pgfqpoint{6.502562in}{2.635630in}}%
\pgfpathlineto{\pgfqpoint{6.503437in}{2.650544in}}%
\pgfpathlineto{\pgfqpoint{6.504313in}{2.888079in}}%
\pgfpathlineto{\pgfqpoint{6.505188in}{3.043452in}}%
\pgfpathlineto{\pgfqpoint{6.506939in}{3.489976in}}%
\pgfpathlineto{\pgfqpoint{6.507814in}{3.788565in}}%
\pgfpathlineto{\pgfqpoint{6.508689in}{3.589279in}}%
\pgfpathlineto{\pgfqpoint{6.509565in}{3.533285in}}%
\pgfpathlineto{\pgfqpoint{6.510440in}{3.614011in}}%
\pgfpathlineto{\pgfqpoint{6.512191in}{3.969878in}}%
\pgfpathlineto{\pgfqpoint{6.513066in}{3.780145in}}%
\pgfpathlineto{\pgfqpoint{6.513942in}{3.481481in}}%
\pgfpathlineto{\pgfqpoint{6.515692in}{3.199770in}}%
\pgfpathlineto{\pgfqpoint{6.516568in}{3.201846in}}%
\pgfpathlineto{\pgfqpoint{6.517443in}{3.226578in}}%
\pgfpathlineto{\pgfqpoint{6.518319in}{3.204489in}}%
\pgfpathlineto{\pgfqpoint{6.519194in}{3.089894in}}%
\pgfpathlineto{\pgfqpoint{6.520945in}{2.704613in}}%
\pgfpathlineto{\pgfqpoint{6.521820in}{2.641633in}}%
\pgfpathlineto{\pgfqpoint{6.522695in}{2.547918in}}%
\pgfpathlineto{\pgfqpoint{6.523571in}{2.569138in}}%
\pgfpathlineto{\pgfqpoint{6.524446in}{2.608482in}}%
\pgfpathlineto{\pgfqpoint{6.525322in}{2.855871in}}%
\pgfpathlineto{\pgfqpoint{6.526197in}{2.938561in}}%
\pgfpathlineto{\pgfqpoint{6.527072in}{3.128936in}}%
\pgfpathlineto{\pgfqpoint{6.527948in}{3.420426in}}%
\pgfpathlineto{\pgfqpoint{6.528823in}{3.587656in}}%
\pgfpathlineto{\pgfqpoint{6.529698in}{3.481632in}}%
\pgfpathlineto{\pgfqpoint{6.531449in}{3.608498in}}%
\pgfpathlineto{\pgfqpoint{6.532325in}{3.727322in}}%
\pgfpathlineto{\pgfqpoint{6.533200in}{3.716939in}}%
\pgfpathlineto{\pgfqpoint{6.534075in}{3.563680in}}%
\pgfpathlineto{\pgfqpoint{6.534951in}{3.356465in}}%
\pgfpathlineto{\pgfqpoint{6.535826in}{3.392184in}}%
\pgfpathlineto{\pgfqpoint{6.536701in}{3.392486in}}%
\pgfpathlineto{\pgfqpoint{6.537577in}{3.487409in}}%
\pgfpathlineto{\pgfqpoint{6.538452in}{3.517766in}}%
\pgfpathlineto{\pgfqpoint{6.539328in}{3.472494in}}%
\pgfpathlineto{\pgfqpoint{6.541954in}{2.922363in}}%
\pgfpathlineto{\pgfqpoint{6.542829in}{2.878941in}}%
\pgfpathlineto{\pgfqpoint{6.543704in}{2.851756in}}%
\pgfpathlineto{\pgfqpoint{6.544580in}{2.894649in}}%
\pgfpathlineto{\pgfqpoint{6.545455in}{2.917870in}}%
\pgfpathlineto{\pgfqpoint{6.546331in}{3.005015in}}%
\pgfpathlineto{\pgfqpoint{6.547206in}{3.187800in}}%
\pgfpathlineto{\pgfqpoint{6.548081in}{3.429035in}}%
\pgfpathlineto{\pgfqpoint{6.548957in}{3.456070in}}%
\pgfpathlineto{\pgfqpoint{6.549832in}{3.532190in}}%
\pgfpathlineto{\pgfqpoint{6.550707in}{3.575989in}}%
\pgfpathlineto{\pgfqpoint{6.551583in}{3.585126in}}%
\pgfpathlineto{\pgfqpoint{6.552458in}{3.690773in}}%
\pgfpathlineto{\pgfqpoint{6.553334in}{3.613407in}}%
\pgfpathlineto{\pgfqpoint{6.554209in}{3.704554in}}%
\pgfpathlineto{\pgfqpoint{6.555960in}{3.288236in}}%
\pgfpathlineto{\pgfqpoint{6.556835in}{3.184780in}}%
\pgfpathlineto{\pgfqpoint{6.557710in}{3.196711in}}%
\pgfpathlineto{\pgfqpoint{6.558586in}{3.187385in}}%
\pgfpathlineto{\pgfqpoint{6.559461in}{3.335698in}}%
\pgfpathlineto{\pgfqpoint{6.560337in}{3.258483in}}%
\pgfpathlineto{\pgfqpoint{6.562087in}{2.755133in}}%
\pgfpathlineto{\pgfqpoint{6.563838in}{2.434305in}}%
\pgfpathlineto{\pgfqpoint{6.564713in}{2.392809in}}%
\pgfpathlineto{\pgfqpoint{6.565589in}{2.379896in}}%
\pgfpathlineto{\pgfqpoint{6.566464in}{2.446387in}}%
\pgfpathlineto{\pgfqpoint{6.567340in}{2.857570in}}%
\pgfpathlineto{\pgfqpoint{6.568215in}{2.975903in}}%
\pgfpathlineto{\pgfqpoint{6.569090in}{3.424278in}}%
\pgfpathlineto{\pgfqpoint{6.571716in}{3.690131in}}%
\pgfpathlineto{\pgfqpoint{6.572592in}{4.022437in}}%
\pgfpathlineto{\pgfqpoint{6.574343in}{4.463902in}}%
\pgfpathlineto{\pgfqpoint{6.576093in}{4.350477in}}%
\pgfpathlineto{\pgfqpoint{6.576969in}{4.009184in}}%
\pgfpathlineto{\pgfqpoint{6.577844in}{3.913959in}}%
\pgfpathlineto{\pgfqpoint{6.578719in}{3.784676in}}%
\pgfpathlineto{\pgfqpoint{6.579595in}{3.859210in}}%
\pgfpathlineto{\pgfqpoint{6.580470in}{3.872237in}}%
\pgfpathlineto{\pgfqpoint{6.581346in}{3.753035in}}%
\pgfpathlineto{\pgfqpoint{6.582221in}{3.528867in}}%
\pgfpathlineto{\pgfqpoint{6.583096in}{3.459393in}}%
\pgfpathlineto{\pgfqpoint{6.583972in}{3.267620in}}%
\pgfpathlineto{\pgfqpoint{6.584847in}{3.264071in}}%
\pgfpathlineto{\pgfqpoint{6.586598in}{3.204905in}}%
\pgfpathlineto{\pgfqpoint{6.587473in}{3.369416in}}%
\pgfpathlineto{\pgfqpoint{6.588349in}{3.641121in}}%
\pgfpathlineto{\pgfqpoint{6.589224in}{3.672913in}}%
\pgfpathlineto{\pgfqpoint{6.590099in}{3.887642in}}%
\pgfpathlineto{\pgfqpoint{6.590975in}{4.004955in}}%
\pgfpathlineto{\pgfqpoint{6.591850in}{3.968293in}}%
\pgfpathlineto{\pgfqpoint{6.592726in}{4.170561in}}%
\pgfpathlineto{\pgfqpoint{6.593601in}{4.239092in}}%
\pgfpathlineto{\pgfqpoint{6.594476in}{4.144886in}}%
\pgfpathlineto{\pgfqpoint{6.595352in}{4.142092in}}%
\pgfpathlineto{\pgfqpoint{6.596227in}{4.125063in}}%
\pgfpathlineto{\pgfqpoint{6.597102in}{4.010921in}}%
\pgfpathlineto{\pgfqpoint{6.597978in}{3.807822in}}%
\pgfpathlineto{\pgfqpoint{6.598853in}{3.712559in}}%
\pgfpathlineto{\pgfqpoint{6.599729in}{3.663738in}}%
\pgfpathlineto{\pgfqpoint{6.600604in}{3.575045in}}%
\pgfpathlineto{\pgfqpoint{6.601479in}{3.599474in}}%
\pgfpathlineto{\pgfqpoint{6.604105in}{2.945357in}}%
\pgfpathlineto{\pgfqpoint{6.604981in}{2.872334in}}%
\pgfpathlineto{\pgfqpoint{6.605856in}{2.842958in}}%
\pgfpathlineto{\pgfqpoint{6.607607in}{2.696080in}}%
\pgfpathlineto{\pgfqpoint{6.610233in}{3.060821in}}%
\pgfpathlineto{\pgfqpoint{6.611108in}{3.429753in}}%
\pgfpathlineto{\pgfqpoint{6.612859in}{3.824926in}}%
\pgfpathlineto{\pgfqpoint{6.613735in}{3.730720in}}%
\pgfpathlineto{\pgfqpoint{6.614610in}{3.879977in}}%
\pgfpathlineto{\pgfqpoint{6.615485in}{4.180265in}}%
\pgfpathlineto{\pgfqpoint{6.616361in}{4.318572in}}%
\pgfpathlineto{\pgfqpoint{6.617236in}{4.214700in}}%
\pgfpathlineto{\pgfqpoint{6.618987in}{3.504664in}}%
\pgfpathlineto{\pgfqpoint{6.619862in}{3.562207in}}%
\pgfpathlineto{\pgfqpoint{6.620738in}{3.578179in}}%
\pgfpathlineto{\pgfqpoint{6.621613in}{3.682919in}}%
\pgfpathlineto{\pgfqpoint{6.623364in}{3.509875in}}%
\pgfpathlineto{\pgfqpoint{6.625990in}{2.903824in}}%
\pgfpathlineto{\pgfqpoint{6.626865in}{2.849113in}}%
\pgfpathlineto{\pgfqpoint{6.627741in}{2.756153in}}%
\pgfpathlineto{\pgfqpoint{6.628616in}{2.591868in}}%
\pgfpathlineto{\pgfqpoint{6.629491in}{2.797120in}}%
\pgfpathlineto{\pgfqpoint{6.630367in}{2.868747in}}%
\pgfpathlineto{\pgfqpoint{6.631242in}{3.061463in}}%
\pgfpathlineto{\pgfqpoint{6.632117in}{3.552466in}}%
\pgfpathlineto{\pgfqpoint{6.632993in}{3.747372in}}%
\pgfpathlineto{\pgfqpoint{6.633868in}{3.876088in}}%
\pgfpathlineto{\pgfqpoint{6.634744in}{3.768252in}}%
\pgfpathlineto{\pgfqpoint{6.635619in}{3.768138in}}%
\pgfpathlineto{\pgfqpoint{6.636494in}{3.939181in}}%
\pgfpathlineto{\pgfqpoint{6.637370in}{3.955040in}}%
\pgfpathlineto{\pgfqpoint{6.638245in}{3.835121in}}%
\pgfpathlineto{\pgfqpoint{6.639120in}{3.783544in}}%
\pgfpathlineto{\pgfqpoint{6.639996in}{3.600229in}}%
\pgfpathlineto{\pgfqpoint{6.640871in}{3.698135in}}%
\pgfpathlineto{\pgfqpoint{6.641747in}{3.610915in}}%
\pgfpathlineto{\pgfqpoint{6.642622in}{3.559337in}}%
\pgfpathlineto{\pgfqpoint{6.643497in}{3.683334in}}%
\pgfpathlineto{\pgfqpoint{6.644373in}{3.593810in}}%
\pgfpathlineto{\pgfqpoint{6.645248in}{3.390673in}}%
\pgfpathlineto{\pgfqpoint{6.646123in}{3.084797in}}%
\pgfpathlineto{\pgfqpoint{6.646999in}{2.895064in}}%
\pgfpathlineto{\pgfqpoint{6.648750in}{2.816037in}}%
\pgfpathlineto{\pgfqpoint{6.649625in}{2.739653in}}%
\pgfpathlineto{\pgfqpoint{6.653126in}{3.537249in}}%
\pgfpathlineto{\pgfqpoint{6.654002in}{3.562169in}}%
\pgfpathlineto{\pgfqpoint{6.654877in}{3.476006in}}%
\pgfpathlineto{\pgfqpoint{6.655753in}{3.586485in}}%
\pgfpathlineto{\pgfqpoint{6.656628in}{3.642442in}}%
\pgfpathlineto{\pgfqpoint{6.657503in}{3.786677in}}%
\pgfpathlineto{\pgfqpoint{6.658379in}{3.837311in}}%
\pgfpathlineto{\pgfqpoint{6.659254in}{3.837348in}}%
\pgfpathlineto{\pgfqpoint{6.660129in}{3.902896in}}%
\pgfpathlineto{\pgfqpoint{6.661005in}{3.635042in}}%
\pgfpathlineto{\pgfqpoint{6.662756in}{3.388521in}}%
\pgfpathlineto{\pgfqpoint{6.663631in}{3.441722in}}%
\pgfpathlineto{\pgfqpoint{6.664506in}{3.441307in}}%
\pgfpathlineto{\pgfqpoint{6.666257in}{3.028840in}}%
\pgfpathlineto{\pgfqpoint{6.668008in}{2.576992in}}%
\pgfpathlineto{\pgfqpoint{6.668883in}{2.479916in}}%
\pgfpathlineto{\pgfqpoint{6.669759in}{2.341987in}}%
\pgfpathlineto{\pgfqpoint{6.670634in}{2.328696in}}%
\pgfpathlineto{\pgfqpoint{6.671509in}{2.513143in}}%
\pgfpathlineto{\pgfqpoint{6.672385in}{2.865386in}}%
\pgfpathlineto{\pgfqpoint{6.673260in}{2.801840in}}%
\pgfpathlineto{\pgfqpoint{6.674135in}{3.053534in}}%
\pgfpathlineto{\pgfqpoint{6.675011in}{3.163711in}}%
\pgfpathlineto{\pgfqpoint{6.675886in}{3.230618in}}%
\pgfpathlineto{\pgfqpoint{6.676762in}{3.458071in}}%
\pgfpathlineto{\pgfqpoint{6.677637in}{3.514632in}}%
\pgfpathlineto{\pgfqpoint{6.678512in}{3.682617in}}%
\pgfpathlineto{\pgfqpoint{6.680263in}{3.759983in}}%
\pgfpathlineto{\pgfqpoint{6.681138in}{3.724755in}}%
\pgfpathlineto{\pgfqpoint{6.682014in}{3.502852in}}%
\pgfpathlineto{\pgfqpoint{6.682889in}{3.395997in}}%
\pgfpathlineto{\pgfqpoint{6.683765in}{3.195427in}}%
\pgfpathlineto{\pgfqpoint{6.684640in}{3.121686in}}%
\pgfpathlineto{\pgfqpoint{6.685515in}{2.999993in}}%
\pgfpathlineto{\pgfqpoint{6.688141in}{2.153273in}}%
\pgfpathlineto{\pgfqpoint{6.689892in}{1.770522in}}%
\pgfpathlineto{\pgfqpoint{6.690768in}{1.703993in}}%
\pgfpathlineto{\pgfqpoint{6.691643in}{1.713772in}}%
\pgfpathlineto{\pgfqpoint{6.692518in}{1.789401in}}%
\pgfpathlineto{\pgfqpoint{6.693394in}{2.011040in}}%
\pgfpathlineto{\pgfqpoint{6.694269in}{2.005527in}}%
\pgfpathlineto{\pgfqpoint{6.696020in}{2.687320in}}%
\pgfpathlineto{\pgfqpoint{6.698646in}{3.054364in}}%
\pgfpathlineto{\pgfqpoint{6.699521in}{3.233261in}}%
\pgfpathlineto{\pgfqpoint{6.700397in}{3.325390in}}%
\pgfpathlineto{\pgfqpoint{6.701272in}{3.353897in}}%
\pgfpathlineto{\pgfqpoint{6.702147in}{3.275814in}}%
\pgfpathlineto{\pgfqpoint{6.703023in}{3.130635in}}%
\pgfpathlineto{\pgfqpoint{6.703898in}{3.141132in}}%
\pgfpathlineto{\pgfqpoint{6.704774in}{2.934521in}}%
\pgfpathlineto{\pgfqpoint{6.705649in}{2.859836in}}%
\pgfpathlineto{\pgfqpoint{6.706524in}{2.886455in}}%
\pgfpathlineto{\pgfqpoint{6.707400in}{2.687207in}}%
\pgfpathlineto{\pgfqpoint{6.709150in}{2.186198in}}%
\pgfpathlineto{\pgfqpoint{6.710901in}{1.915890in}}%
\pgfpathlineto{\pgfqpoint{6.712652in}{1.821571in}}%
\pgfpathlineto{\pgfqpoint{6.713527in}{1.941830in}}%
\pgfpathlineto{\pgfqpoint{6.717904in}{3.283970in}}%
\pgfpathlineto{\pgfqpoint{6.718780in}{3.469021in}}%
\pgfpathlineto{\pgfqpoint{6.719655in}{3.545556in}}%
\pgfpathlineto{\pgfqpoint{6.720530in}{3.447574in}}%
\pgfpathlineto{\pgfqpoint{6.721406in}{3.521806in}}%
\pgfpathlineto{\pgfqpoint{6.723157in}{3.309683in}}%
\pgfpathlineto{\pgfqpoint{6.724032in}{2.996632in}}%
\pgfpathlineto{\pgfqpoint{6.725783in}{2.851189in}}%
\pgfpathlineto{\pgfqpoint{6.726658in}{2.788322in}}%
\pgfpathlineto{\pgfqpoint{6.727533in}{2.605575in}}%
\pgfpathlineto{\pgfqpoint{6.728409in}{2.490526in}}%
\pgfpathlineto{\pgfqpoint{6.730160in}{1.998504in}}%
\pgfpathlineto{\pgfqpoint{6.731035in}{1.855779in}}%
\pgfpathlineto{\pgfqpoint{6.732786in}{1.763839in}}%
\pgfpathlineto{\pgfqpoint{6.733661in}{1.770598in}}%
\pgfpathlineto{\pgfqpoint{6.734536in}{1.849361in}}%
\pgfpathlineto{\pgfqpoint{6.736287in}{2.224787in}}%
\pgfpathlineto{\pgfqpoint{6.737163in}{2.675200in}}%
\pgfpathlineto{\pgfqpoint{6.738038in}{2.969145in}}%
\pgfpathlineto{\pgfqpoint{6.739789in}{3.040545in}}%
\pgfpathlineto{\pgfqpoint{6.740664in}{3.043528in}}%
\pgfpathlineto{\pgfqpoint{6.741539in}{3.195465in}}%
\pgfpathlineto{\pgfqpoint{6.742415in}{3.423447in}}%
\pgfpathlineto{\pgfqpoint{6.743290in}{3.479971in}}%
\pgfpathlineto{\pgfqpoint{6.745041in}{2.947510in}}%
\pgfpathlineto{\pgfqpoint{6.746792in}{2.722624in}}%
\pgfpathlineto{\pgfqpoint{6.747667in}{2.613806in}}%
\pgfpathlineto{\pgfqpoint{6.748542in}{2.603045in}}%
\pgfpathlineto{\pgfqpoint{6.752044in}{2.004394in}}%
\pgfpathlineto{\pgfqpoint{6.752919in}{1.918948in}}%
\pgfpathlineto{\pgfqpoint{6.753795in}{1.878849in}}%
\pgfpathlineto{\pgfqpoint{6.754670in}{1.866578in}}%
\pgfpathlineto{\pgfqpoint{6.755545in}{1.872695in}}%
\pgfpathlineto{\pgfqpoint{6.757296in}{2.060653in}}%
\pgfpathlineto{\pgfqpoint{6.758172in}{2.099884in}}%
\pgfpathlineto{\pgfqpoint{6.759047in}{2.165205in}}%
\pgfpathlineto{\pgfqpoint{6.759922in}{2.269756in}}%
\pgfpathlineto{\pgfqpoint{6.760798in}{2.415766in}}%
\pgfpathlineto{\pgfqpoint{6.761673in}{2.463152in}}%
\pgfpathlineto{\pgfqpoint{6.762548in}{2.776391in}}%
\pgfpathlineto{\pgfqpoint{6.764299in}{3.015134in}}%
\pgfpathlineto{\pgfqpoint{6.765175in}{2.953664in}}%
\pgfpathlineto{\pgfqpoint{6.766050in}{2.801575in}}%
\pgfpathlineto{\pgfqpoint{6.766925in}{2.778203in}}%
\pgfpathlineto{\pgfqpoint{6.767801in}{2.663118in}}%
\pgfpathlineto{\pgfqpoint{6.768676in}{2.644239in}}%
\pgfpathlineto{\pgfqpoint{6.769551in}{2.655490in}}%
\pgfpathlineto{\pgfqpoint{6.770427in}{2.482144in}}%
\pgfpathlineto{\pgfqpoint{6.771302in}{2.180006in}}%
\pgfpathlineto{\pgfqpoint{6.773053in}{1.904940in}}%
\pgfpathlineto{\pgfqpoint{6.773928in}{1.700821in}}%
\pgfpathlineto{\pgfqpoint{6.774804in}{1.692628in}}%
\pgfpathlineto{\pgfqpoint{6.775679in}{1.631422in}}%
\pgfpathlineto{\pgfqpoint{6.776554in}{1.694327in}}%
\pgfpathlineto{\pgfqpoint{6.777430in}{1.872468in}}%
\pgfpathlineto{\pgfqpoint{6.778305in}{1.956668in}}%
\pgfpathlineto{\pgfqpoint{6.779181in}{2.137906in}}%
\pgfpathlineto{\pgfqpoint{6.780056in}{1.879152in}}%
\pgfpathlineto{\pgfqpoint{6.781807in}{2.341043in}}%
\pgfpathlineto{\pgfqpoint{6.782682in}{2.320767in}}%
\pgfpathlineto{\pgfqpoint{6.783557in}{2.337909in}}%
\pgfpathlineto{\pgfqpoint{6.784433in}{2.463076in}}%
\pgfpathlineto{\pgfqpoint{6.785308in}{2.533457in}}%
\pgfpathlineto{\pgfqpoint{6.786184in}{2.512388in}}%
\pgfpathlineto{\pgfqpoint{6.787059in}{2.426451in}}%
\pgfpathlineto{\pgfqpoint{6.787934in}{2.436608in}}%
\pgfpathlineto{\pgfqpoint{6.788810in}{2.283462in}}%
\pgfpathlineto{\pgfqpoint{6.789685in}{2.360111in}}%
\pgfpathlineto{\pgfqpoint{6.790560in}{2.349576in}}%
\pgfpathlineto{\pgfqpoint{6.791436in}{2.219614in}}%
\pgfpathlineto{\pgfqpoint{6.792311in}{1.918344in}}%
\pgfpathlineto{\pgfqpoint{6.793187in}{1.801521in}}%
\pgfpathlineto{\pgfqpoint{6.794062in}{1.726119in}}%
\pgfpathlineto{\pgfqpoint{6.794937in}{1.691382in}}%
\pgfpathlineto{\pgfqpoint{6.796688in}{1.578335in}}%
\pgfpathlineto{\pgfqpoint{6.797563in}{1.697083in}}%
\pgfpathlineto{\pgfqpoint{6.800190in}{2.371249in}}%
\pgfpathlineto{\pgfqpoint{6.801065in}{2.374836in}}%
\pgfpathlineto{\pgfqpoint{6.801940in}{2.464360in}}%
\pgfpathlineto{\pgfqpoint{6.802816in}{2.450390in}}%
\pgfpathlineto{\pgfqpoint{6.804566in}{2.643483in}}%
\pgfpathlineto{\pgfqpoint{6.805442in}{2.716960in}}%
\pgfpathlineto{\pgfqpoint{6.806317in}{2.919116in}}%
\pgfpathlineto{\pgfqpoint{6.807193in}{2.870597in}}%
\pgfpathlineto{\pgfqpoint{6.808943in}{2.477009in}}%
\pgfpathlineto{\pgfqpoint{6.809819in}{2.360451in}}%
\pgfpathlineto{\pgfqpoint{6.810694in}{2.418145in}}%
\pgfpathlineto{\pgfqpoint{6.811569in}{2.452051in}}%
\pgfpathlineto{\pgfqpoint{6.813320in}{1.970412in}}%
\pgfpathlineto{\pgfqpoint{6.815071in}{1.831954in}}%
\pgfpathlineto{\pgfqpoint{6.815946in}{1.716000in}}%
\pgfpathlineto{\pgfqpoint{6.816822in}{1.704522in}}%
\pgfpathlineto{\pgfqpoint{6.817697in}{1.652227in}}%
\pgfpathlineto{\pgfqpoint{6.818572in}{1.881870in}}%
\pgfpathlineto{\pgfqpoint{6.819448in}{1.979701in}}%
\pgfpathlineto{\pgfqpoint{6.820323in}{2.145420in}}%
\pgfpathlineto{\pgfqpoint{6.822074in}{2.194996in}}%
\pgfpathlineto{\pgfqpoint{6.822949in}{2.294299in}}%
\pgfpathlineto{\pgfqpoint{6.823825in}{2.284482in}}%
\pgfpathlineto{\pgfqpoint{6.824700in}{2.347084in}}%
\pgfpathlineto{\pgfqpoint{6.826451in}{2.607538in}}%
\pgfpathlineto{\pgfqpoint{6.827326in}{2.705633in}}%
\pgfpathlineto{\pgfqpoint{6.829077in}{2.504044in}}%
\pgfpathlineto{\pgfqpoint{6.829952in}{2.471157in}}%
\pgfpathlineto{\pgfqpoint{6.830828in}{2.369173in}}%
\pgfpathlineto{\pgfqpoint{6.831703in}{2.331302in}}%
\pgfpathlineto{\pgfqpoint{6.832578in}{2.372760in}}%
\pgfpathlineto{\pgfqpoint{6.833454in}{2.252048in}}%
\pgfpathlineto{\pgfqpoint{6.834329in}{2.033015in}}%
\pgfpathlineto{\pgfqpoint{6.836080in}{1.814586in}}%
\pgfpathlineto{\pgfqpoint{6.836955in}{1.717397in}}%
\pgfpathlineto{\pgfqpoint{6.837831in}{1.680885in}}%
\pgfpathlineto{\pgfqpoint{6.838706in}{1.660420in}}%
\pgfpathlineto{\pgfqpoint{6.839581in}{1.765651in}}%
\pgfpathlineto{\pgfqpoint{6.841332in}{2.272437in}}%
\pgfpathlineto{\pgfqpoint{6.842208in}{2.470099in}}%
\pgfpathlineto{\pgfqpoint{6.843083in}{2.530965in}}%
\pgfpathlineto{\pgfqpoint{6.843958in}{2.567552in}}%
\pgfpathlineto{\pgfqpoint{6.844834in}{2.649411in}}%
\pgfpathlineto{\pgfqpoint{6.845709in}{2.823173in}}%
\pgfpathlineto{\pgfqpoint{6.846584in}{2.830725in}}%
\pgfpathlineto{\pgfqpoint{6.847460in}{3.253046in}}%
\pgfpathlineto{\pgfqpoint{6.848335in}{3.413554in}}%
\pgfpathlineto{\pgfqpoint{6.850086in}{3.127388in}}%
\pgfpathlineto{\pgfqpoint{6.850961in}{2.906920in}}%
\pgfpathlineto{\pgfqpoint{6.852712in}{2.797158in}}%
\pgfpathlineto{\pgfqpoint{6.853587in}{2.776580in}}%
\pgfpathlineto{\pgfqpoint{6.854463in}{2.606028in}}%
\pgfpathlineto{\pgfqpoint{6.856214in}{2.123860in}}%
\pgfpathlineto{\pgfqpoint{6.857964in}{1.946360in}}%
\pgfpathlineto{\pgfqpoint{6.858840in}{1.996692in}}%
\pgfpathlineto{\pgfqpoint{6.859715in}{1.938016in}}%
\pgfpathlineto{\pgfqpoint{6.860591in}{1.991406in}}%
\pgfpathlineto{\pgfqpoint{6.863217in}{2.937164in}}%
\pgfpathlineto{\pgfqpoint{6.864092in}{3.043037in}}%
\pgfpathlineto{\pgfqpoint{6.865843in}{3.478762in}}%
\pgfpathlineto{\pgfqpoint{6.866718in}{3.594868in}}%
\pgfpathlineto{\pgfqpoint{6.867594in}{3.789169in}}%
\pgfpathlineto{\pgfqpoint{6.868469in}{3.833459in}}%
\pgfpathlineto{\pgfqpoint{6.869344in}{3.779805in}}%
\pgfpathlineto{\pgfqpoint{6.870220in}{3.663209in}}%
\pgfpathlineto{\pgfqpoint{6.871095in}{3.374551in}}%
\pgfpathlineto{\pgfqpoint{6.871970in}{3.239491in}}%
\pgfpathlineto{\pgfqpoint{6.872846in}{3.001277in}}%
\pgfpathlineto{\pgfqpoint{6.873721in}{2.919795in}}%
\pgfpathlineto{\pgfqpoint{6.874597in}{2.770539in}}%
\pgfpathlineto{\pgfqpoint{6.877223in}{1.981777in}}%
\pgfpathlineto{\pgfqpoint{6.878973in}{1.698933in}}%
\pgfpathlineto{\pgfqpoint{6.879849in}{1.613789in}}%
\pgfpathlineto{\pgfqpoint{6.880724in}{1.496891in}}%
\pgfpathlineto{\pgfqpoint{6.881600in}{1.679111in}}%
\pgfpathlineto{\pgfqpoint{6.885976in}{3.138300in}}%
\pgfpathlineto{\pgfqpoint{6.886852in}{3.125991in}}%
\pgfpathlineto{\pgfqpoint{6.887727in}{3.191123in}}%
\pgfpathlineto{\pgfqpoint{6.889478in}{3.492770in}}%
\pgfpathlineto{\pgfqpoint{6.890353in}{3.361298in}}%
\pgfpathlineto{\pgfqpoint{6.891229in}{3.294504in}}%
\pgfpathlineto{\pgfqpoint{6.893855in}{2.736783in}}%
\pgfpathlineto{\pgfqpoint{6.894730in}{2.743391in}}%
\pgfpathlineto{\pgfqpoint{6.895606in}{2.637480in}}%
\pgfpathlineto{\pgfqpoint{6.896481in}{2.484863in}}%
\pgfpathlineto{\pgfqpoint{6.898232in}{1.967618in}}%
\pgfpathlineto{\pgfqpoint{6.899982in}{1.834371in}}%
\pgfpathlineto{\pgfqpoint{6.900858in}{1.720531in}}%
\pgfpathlineto{\pgfqpoint{6.901733in}{1.694214in}}%
\pgfpathlineto{\pgfqpoint{6.902609in}{1.753531in}}%
\pgfpathlineto{\pgfqpoint{6.904359in}{2.040453in}}%
\pgfpathlineto{\pgfqpoint{6.906110in}{2.495322in}}%
\pgfpathlineto{\pgfqpoint{6.906985in}{2.580956in}}%
\pgfpathlineto{\pgfqpoint{6.907861in}{2.743693in}}%
\pgfpathlineto{\pgfqpoint{6.909612in}{3.164919in}}%
\pgfpathlineto{\pgfqpoint{6.910487in}{3.172433in}}%
\pgfpathlineto{\pgfqpoint{6.911362in}{3.242209in}}%
\pgfpathlineto{\pgfqpoint{6.912238in}{3.130069in}}%
\pgfpathlineto{\pgfqpoint{6.913113in}{3.103072in}}%
\pgfpathlineto{\pgfqpoint{6.913988in}{3.011320in}}%
\pgfpathlineto{\pgfqpoint{6.917490in}{2.262545in}}%
\pgfpathlineto{\pgfqpoint{6.919241in}{1.793781in}}%
\pgfpathlineto{\pgfqpoint{6.920116in}{1.668350in}}%
\pgfpathlineto{\pgfqpoint{6.921867in}{1.612997in}}%
\pgfpathlineto{\pgfqpoint{6.922742in}{1.599706in}}%
\pgfpathlineto{\pgfqpoint{6.923618in}{1.734426in}}%
\pgfpathlineto{\pgfqpoint{6.925368in}{2.328130in}}%
\pgfpathlineto{\pgfqpoint{6.926244in}{2.395679in}}%
\pgfpathlineto{\pgfqpoint{6.927119in}{2.400172in}}%
\pgfpathlineto{\pgfqpoint{6.928870in}{2.831593in}}%
\pgfpathlineto{\pgfqpoint{6.930621in}{3.185648in}}%
\pgfpathlineto{\pgfqpoint{6.931496in}{3.064559in}}%
\pgfpathlineto{\pgfqpoint{6.932371in}{3.115343in}}%
\pgfpathlineto{\pgfqpoint{6.934122in}{2.731421in}}%
\pgfpathlineto{\pgfqpoint{6.934997in}{2.618828in}}%
\pgfpathlineto{\pgfqpoint{6.935873in}{2.631363in}}%
\pgfpathlineto{\pgfqpoint{6.936748in}{2.568647in}}%
\pgfpathlineto{\pgfqpoint{6.938499in}{2.338438in}}%
\pgfpathlineto{\pgfqpoint{6.941125in}{1.736200in}}%
\pgfpathlineto{\pgfqpoint{6.942000in}{1.677185in}}%
\pgfpathlineto{\pgfqpoint{6.942876in}{1.575163in}}%
\pgfpathlineto{\pgfqpoint{6.943751in}{1.532346in}}%
\pgfpathlineto{\pgfqpoint{6.944627in}{1.552357in}}%
\pgfpathlineto{\pgfqpoint{6.945502in}{1.736767in}}%
\pgfpathlineto{\pgfqpoint{6.947253in}{2.158484in}}%
\pgfpathlineto{\pgfqpoint{6.948128in}{2.253030in}}%
\pgfpathlineto{\pgfqpoint{6.949003in}{2.223541in}}%
\pgfpathlineto{\pgfqpoint{6.949879in}{2.280480in}}%
\pgfpathlineto{\pgfqpoint{6.950754in}{2.439214in}}%
\pgfpathlineto{\pgfqpoint{6.951630in}{2.531192in}}%
\pgfpathlineto{\pgfqpoint{6.952505in}{2.567968in}}%
\pgfpathlineto{\pgfqpoint{6.953380in}{2.685206in}}%
\pgfpathlineto{\pgfqpoint{6.954256in}{2.665081in}}%
\pgfpathlineto{\pgfqpoint{6.956006in}{2.601912in}}%
\pgfpathlineto{\pgfqpoint{6.956882in}{2.460962in}}%
\pgfpathlineto{\pgfqpoint{6.957757in}{2.485731in}}%
\pgfpathlineto{\pgfqpoint{6.958633in}{2.418787in}}%
\pgfpathlineto{\pgfqpoint{6.960383in}{2.011644in}}%
\pgfpathlineto{\pgfqpoint{6.961259in}{1.853929in}}%
\pgfpathlineto{\pgfqpoint{6.962134in}{1.751303in}}%
\pgfpathlineto{\pgfqpoint{6.963009in}{1.681905in}}%
\pgfpathlineto{\pgfqpoint{6.963885in}{1.644109in}}%
\pgfpathlineto{\pgfqpoint{6.964760in}{1.642712in}}%
\pgfpathlineto{\pgfqpoint{6.965636in}{1.922686in}}%
\pgfpathlineto{\pgfqpoint{6.966511in}{2.028786in}}%
\pgfpathlineto{\pgfqpoint{6.967386in}{2.238266in}}%
\pgfpathlineto{\pgfqpoint{6.969137in}{2.426149in}}%
\pgfpathlineto{\pgfqpoint{6.970888in}{2.804370in}}%
\pgfpathlineto{\pgfqpoint{6.971763in}{2.679655in}}%
\pgfpathlineto{\pgfqpoint{6.972639in}{2.675578in}}%
\pgfpathlineto{\pgfqpoint{6.973514in}{2.773710in}}%
\pgfpathlineto{\pgfqpoint{6.974389in}{2.831895in}}%
\pgfpathlineto{\pgfqpoint{6.975265in}{2.787605in}}%
\pgfpathlineto{\pgfqpoint{6.976140in}{2.763629in}}%
\pgfpathlineto{\pgfqpoint{6.977015in}{2.754567in}}%
\pgfpathlineto{\pgfqpoint{6.977891in}{2.617053in}}%
\pgfpathlineto{\pgfqpoint{6.978766in}{2.631401in}}%
\pgfpathlineto{\pgfqpoint{6.979642in}{2.713222in}}%
\pgfpathlineto{\pgfqpoint{6.981392in}{2.352597in}}%
\pgfpathlineto{\pgfqpoint{6.983143in}{1.969997in}}%
\pgfpathlineto{\pgfqpoint{6.984018in}{1.903996in}}%
\pgfpathlineto{\pgfqpoint{6.985769in}{1.694667in}}%
\pgfpathlineto{\pgfqpoint{6.988395in}{1.916419in}}%
\pgfpathlineto{\pgfqpoint{6.989271in}{2.136094in}}%
\pgfpathlineto{\pgfqpoint{6.990146in}{2.088254in}}%
\pgfpathlineto{\pgfqpoint{6.991022in}{2.192466in}}%
\pgfpathlineto{\pgfqpoint{6.991897in}{2.353390in}}%
\pgfpathlineto{\pgfqpoint{6.992772in}{2.380047in}}%
\pgfpathlineto{\pgfqpoint{6.993648in}{2.446614in}}%
\pgfpathlineto{\pgfqpoint{6.994523in}{2.769708in}}%
\pgfpathlineto{\pgfqpoint{6.996274in}{2.883208in}}%
\pgfpathlineto{\pgfqpoint{6.997149in}{2.736141in}}%
\pgfpathlineto{\pgfqpoint{6.998025in}{2.712052in}}%
\pgfpathlineto{\pgfqpoint{6.999775in}{2.296300in}}%
\pgfpathlineto{\pgfqpoint{7.000651in}{2.248876in}}%
\pgfpathlineto{\pgfqpoint{7.001526in}{2.118159in}}%
\pgfpathlineto{\pgfqpoint{7.002401in}{1.786456in}}%
\pgfpathlineto{\pgfqpoint{7.004152in}{1.497155in}}%
\pgfpathlineto{\pgfqpoint{7.005028in}{1.527135in}}%
\pgfpathlineto{\pgfqpoint{7.006778in}{1.514524in}}%
\pgfpathlineto{\pgfqpoint{7.007654in}{1.595099in}}%
\pgfpathlineto{\pgfqpoint{7.008529in}{1.800389in}}%
\pgfpathlineto{\pgfqpoint{7.009404in}{2.156558in}}%
\pgfpathlineto{\pgfqpoint{7.010280in}{2.169245in}}%
\pgfpathlineto{\pgfqpoint{7.012031in}{2.479086in}}%
\pgfpathlineto{\pgfqpoint{7.013781in}{3.092160in}}%
\pgfpathlineto{\pgfqpoint{7.015532in}{3.332639in}}%
\pgfpathlineto{\pgfqpoint{7.016407in}{3.324068in}}%
\pgfpathlineto{\pgfqpoint{7.020784in}{2.428339in}}%
\pgfpathlineto{\pgfqpoint{7.021660in}{2.426716in}}%
\pgfpathlineto{\pgfqpoint{7.024286in}{1.839090in}}%
\pgfpathlineto{\pgfqpoint{7.025161in}{1.749529in}}%
\pgfpathlineto{\pgfqpoint{7.026037in}{1.743978in}}%
\pgfpathlineto{\pgfqpoint{7.026912in}{1.631385in}}%
\pgfpathlineto{\pgfqpoint{7.027787in}{1.667406in}}%
\pgfpathlineto{\pgfqpoint{7.028663in}{1.796311in}}%
\pgfpathlineto{\pgfqpoint{7.030413in}{2.382992in}}%
\pgfpathlineto{\pgfqpoint{7.031289in}{2.568043in}}%
\pgfpathlineto{\pgfqpoint{7.032164in}{2.651337in}}%
\pgfpathlineto{\pgfqpoint{7.033040in}{2.604102in}}%
\pgfpathlineto{\pgfqpoint{7.035666in}{2.989496in}}%
\pgfpathlineto{\pgfqpoint{7.036541in}{3.079096in}}%
\pgfpathlineto{\pgfqpoint{7.037416in}{3.210870in}}%
\pgfpathlineto{\pgfqpoint{7.039167in}{2.950455in}}%
\pgfpathlineto{\pgfqpoint{7.040043in}{2.880980in}}%
\pgfpathlineto{\pgfqpoint{7.040918in}{2.738407in}}%
\pgfpathlineto{\pgfqpoint{7.043544in}{2.610143in}}%
\pgfpathlineto{\pgfqpoint{7.045295in}{2.250991in}}%
\pgfpathlineto{\pgfqpoint{7.046170in}{2.143117in}}%
\pgfpathlineto{\pgfqpoint{7.047046in}{2.131223in}}%
\pgfpathlineto{\pgfqpoint{7.047921in}{1.938809in}}%
\pgfpathlineto{\pgfqpoint{7.048796in}{1.915852in}}%
\pgfpathlineto{\pgfqpoint{7.049672in}{2.055934in}}%
\pgfpathlineto{\pgfqpoint{7.052298in}{2.665081in}}%
\pgfpathlineto{\pgfqpoint{7.053173in}{2.555923in}}%
\pgfpathlineto{\pgfqpoint{7.054049in}{2.670065in}}%
\pgfpathlineto{\pgfqpoint{7.054924in}{2.734669in}}%
\pgfpathlineto{\pgfqpoint{7.055799in}{2.759249in}}%
\pgfpathlineto{\pgfqpoint{7.056675in}{2.747922in}}%
\pgfpathlineto{\pgfqpoint{7.057550in}{2.908166in}}%
\pgfpathlineto{\pgfqpoint{7.058425in}{2.834802in}}%
\pgfpathlineto{\pgfqpoint{7.060176in}{2.665836in}}%
\pgfpathlineto{\pgfqpoint{7.061052in}{2.641444in}}%
\pgfpathlineto{\pgfqpoint{7.061927in}{2.481049in}}%
\pgfpathlineto{\pgfqpoint{7.062802in}{2.517825in}}%
\pgfpathlineto{\pgfqpoint{7.063678in}{2.580541in}}%
\pgfpathlineto{\pgfqpoint{7.066304in}{2.001487in}}%
\pgfpathlineto{\pgfqpoint{7.068930in}{1.770220in}}%
\pgfpathlineto{\pgfqpoint{7.069805in}{1.734086in}}%
\pgfpathlineto{\pgfqpoint{7.070681in}{1.850833in}}%
\pgfpathlineto{\pgfqpoint{7.071556in}{2.090482in}}%
\pgfpathlineto{\pgfqpoint{7.072431in}{2.231168in}}%
\pgfpathlineto{\pgfqpoint{7.073307in}{2.205757in}}%
\pgfpathlineto{\pgfqpoint{7.074182in}{2.103018in}}%
\pgfpathlineto{\pgfqpoint{7.075058in}{2.188577in}}%
\pgfpathlineto{\pgfqpoint{7.075933in}{2.349954in}}%
\pgfpathlineto{\pgfqpoint{7.076808in}{2.442989in}}%
\pgfpathlineto{\pgfqpoint{7.077684in}{2.640954in}}%
\pgfpathlineto{\pgfqpoint{7.078559in}{2.703443in}}%
\pgfpathlineto{\pgfqpoint{7.079434in}{2.726098in}}%
\pgfpathlineto{\pgfqpoint{7.080310in}{2.851605in}}%
\pgfpathlineto{\pgfqpoint{7.081185in}{2.787983in}}%
\pgfpathlineto{\pgfqpoint{7.082936in}{2.580239in}}%
\pgfpathlineto{\pgfqpoint{7.083811in}{2.582278in}}%
\pgfpathlineto{\pgfqpoint{7.084687in}{2.686301in}}%
\pgfpathlineto{\pgfqpoint{7.086437in}{2.510576in}}%
\pgfpathlineto{\pgfqpoint{7.088188in}{2.417578in}}%
\pgfpathlineto{\pgfqpoint{7.089939in}{2.318426in}}%
\pgfpathlineto{\pgfqpoint{7.090814in}{2.333492in}}%
\pgfpathlineto{\pgfqpoint{7.091690in}{2.436873in}}%
\pgfpathlineto{\pgfqpoint{7.093440in}{2.768462in}}%
\pgfpathlineto{\pgfqpoint{7.094316in}{2.994896in}}%
\pgfpathlineto{\pgfqpoint{7.095191in}{2.944451in}}%
\pgfpathlineto{\pgfqpoint{7.096067in}{3.066522in}}%
\pgfpathlineto{\pgfqpoint{7.096942in}{3.138791in}}%
\pgfpathlineto{\pgfqpoint{7.097817in}{3.270037in}}%
\pgfpathlineto{\pgfqpoint{7.098693in}{3.251309in}}%
\pgfpathlineto{\pgfqpoint{7.099568in}{3.338228in}}%
\pgfpathlineto{\pgfqpoint{7.100443in}{3.265808in}}%
\pgfpathlineto{\pgfqpoint{7.101319in}{3.113908in}}%
\pgfpathlineto{\pgfqpoint{7.103070in}{2.912508in}}%
\pgfpathlineto{\pgfqpoint{7.103945in}{2.925421in}}%
\pgfpathlineto{\pgfqpoint{7.104820in}{2.909450in}}%
\pgfpathlineto{\pgfqpoint{7.106571in}{2.740672in}}%
\pgfpathlineto{\pgfqpoint{7.107446in}{2.549580in}}%
\pgfpathlineto{\pgfqpoint{7.109197in}{2.370834in}}%
\pgfpathlineto{\pgfqpoint{7.110073in}{2.338174in}}%
\pgfpathlineto{\pgfqpoint{7.110948in}{2.231092in}}%
\pgfpathlineto{\pgfqpoint{7.111823in}{2.221766in}}%
\pgfpathlineto{\pgfqpoint{7.112699in}{2.254691in}}%
\pgfpathlineto{\pgfqpoint{7.113574in}{2.347122in}}%
\pgfpathlineto{\pgfqpoint{7.114449in}{2.392658in}}%
\pgfpathlineto{\pgfqpoint{7.115325in}{2.556565in}}%
\pgfpathlineto{\pgfqpoint{7.116200in}{2.625737in}}%
\pgfpathlineto{\pgfqpoint{7.117076in}{2.648958in}}%
\pgfpathlineto{\pgfqpoint{7.117951in}{2.752679in}}%
\pgfpathlineto{\pgfqpoint{7.118826in}{2.705935in}}%
\pgfpathlineto{\pgfqpoint{7.119702in}{2.768386in}}%
\pgfpathlineto{\pgfqpoint{7.120577in}{2.797498in}}%
\pgfpathlineto{\pgfqpoint{7.121453in}{2.882981in}}%
\pgfpathlineto{\pgfqpoint{7.122328in}{2.892836in}}%
\pgfpathlineto{\pgfqpoint{7.123203in}{2.699138in}}%
\pgfpathlineto{\pgfqpoint{7.124079in}{2.655264in}}%
\pgfpathlineto{\pgfqpoint{7.125829in}{2.670782in}}%
\pgfpathlineto{\pgfqpoint{7.126705in}{2.858137in}}%
\pgfpathlineto{\pgfqpoint{7.129331in}{2.489922in}}%
\pgfpathlineto{\pgfqpoint{7.130206in}{2.482333in}}%
\pgfpathlineto{\pgfqpoint{7.131082in}{2.429963in}}%
\pgfpathlineto{\pgfqpoint{7.131957in}{2.413651in}}%
\pgfpathlineto{\pgfqpoint{7.132832in}{2.439704in}}%
\pgfpathlineto{\pgfqpoint{7.133708in}{2.312914in}}%
\pgfpathlineto{\pgfqpoint{7.134583in}{2.386843in}}%
\pgfpathlineto{\pgfqpoint{7.137209in}{2.764233in}}%
\pgfpathlineto{\pgfqpoint{7.138960in}{3.053005in}}%
\pgfpathlineto{\pgfqpoint{7.139835in}{3.021553in}}%
\pgfpathlineto{\pgfqpoint{7.140711in}{3.157745in}}%
\pgfpathlineto{\pgfqpoint{7.141586in}{3.228201in}}%
\pgfpathlineto{\pgfqpoint{7.142462in}{3.180400in}}%
\pgfpathlineto{\pgfqpoint{7.143337in}{3.224010in}}%
\pgfpathlineto{\pgfqpoint{7.145088in}{2.854663in}}%
\pgfpathlineto{\pgfqpoint{7.145963in}{2.812714in}}%
\pgfpathlineto{\pgfqpoint{7.146838in}{2.892761in}}%
\pgfpathlineto{\pgfqpoint{7.147714in}{2.766347in}}%
\pgfpathlineto{\pgfqpoint{7.148589in}{2.721340in}}%
\pgfpathlineto{\pgfqpoint{7.149465in}{2.506951in}}%
\pgfpathlineto{\pgfqpoint{7.150340in}{2.374006in}}%
\pgfpathlineto{\pgfqpoint{7.151215in}{2.311630in}}%
\pgfpathlineto{\pgfqpoint{7.152091in}{2.149800in}}%
\pgfpathlineto{\pgfqpoint{7.152966in}{2.128164in}}%
\pgfpathlineto{\pgfqpoint{7.153841in}{2.206134in}}%
\pgfpathlineto{\pgfqpoint{7.154717in}{2.340628in}}%
\pgfpathlineto{\pgfqpoint{7.155592in}{2.644276in}}%
\pgfpathlineto{\pgfqpoint{7.156468in}{2.789493in}}%
\pgfpathlineto{\pgfqpoint{7.157343in}{3.218762in}}%
\pgfpathlineto{\pgfqpoint{7.158218in}{3.325805in}}%
\pgfpathlineto{\pgfqpoint{7.159094in}{3.620052in}}%
\pgfpathlineto{\pgfqpoint{7.159969in}{3.707197in}}%
\pgfpathlineto{\pgfqpoint{7.160844in}{3.844069in}}%
\pgfpathlineto{\pgfqpoint{7.161720in}{3.923587in}}%
\pgfpathlineto{\pgfqpoint{7.162595in}{4.125818in}}%
\pgfpathlineto{\pgfqpoint{7.163471in}{4.218514in}}%
\pgfpathlineto{\pgfqpoint{7.164346in}{4.063631in}}%
\pgfpathlineto{\pgfqpoint{7.166972in}{3.224879in}}%
\pgfpathlineto{\pgfqpoint{7.167847in}{3.209473in}}%
\pgfpathlineto{\pgfqpoint{7.169598in}{2.864291in}}%
\pgfpathlineto{\pgfqpoint{7.171349in}{2.356524in}}%
\pgfpathlineto{\pgfqpoint{7.172224in}{2.191371in}}%
\pgfpathlineto{\pgfqpoint{7.173100in}{2.205304in}}%
\pgfpathlineto{\pgfqpoint{7.175726in}{2.141795in}}%
\pgfpathlineto{\pgfqpoint{7.176601in}{2.302115in}}%
\pgfpathlineto{\pgfqpoint{7.177477in}{2.516164in}}%
\pgfpathlineto{\pgfqpoint{7.178352in}{2.892043in}}%
\pgfpathlineto{\pgfqpoint{7.179227in}{3.083929in}}%
\pgfpathlineto{\pgfqpoint{7.180978in}{3.287217in}}%
\pgfpathlineto{\pgfqpoint{7.181853in}{3.476837in}}%
\pgfpathlineto{\pgfqpoint{7.182729in}{3.547708in}}%
\pgfpathlineto{\pgfqpoint{7.183604in}{3.575800in}}%
\pgfpathlineto{\pgfqpoint{7.184480in}{3.672196in}}%
\pgfpathlineto{\pgfqpoint{7.186230in}{3.283743in}}%
\pgfpathlineto{\pgfqpoint{7.187106in}{3.056743in}}%
\pgfpathlineto{\pgfqpoint{7.189732in}{2.876449in}}%
\pgfpathlineto{\pgfqpoint{7.192358in}{2.328281in}}%
\pgfpathlineto{\pgfqpoint{7.194109in}{2.161316in}}%
\pgfpathlineto{\pgfqpoint{7.194984in}{2.087726in}}%
\pgfpathlineto{\pgfqpoint{7.195859in}{2.090482in}}%
\pgfpathlineto{\pgfqpoint{7.196735in}{2.168339in}}%
\pgfpathlineto{\pgfqpoint{7.197610in}{2.292600in}}%
\pgfpathlineto{\pgfqpoint{7.198486in}{2.333680in}}%
\pgfpathlineto{\pgfqpoint{7.199361in}{2.414822in}}%
\pgfpathlineto{\pgfqpoint{7.200236in}{2.546295in}}%
\pgfpathlineto{\pgfqpoint{7.201112in}{2.606556in}}%
\pgfpathlineto{\pgfqpoint{7.201987in}{2.638952in}}%
\pgfpathlineto{\pgfqpoint{7.202862in}{2.711523in}}%
\pgfpathlineto{\pgfqpoint{7.203738in}{2.824457in}}%
\pgfpathlineto{\pgfqpoint{7.204613in}{2.758947in}}%
\pgfpathlineto{\pgfqpoint{7.206364in}{2.850057in}}%
\pgfpathlineto{\pgfqpoint{7.207239in}{2.830725in}}%
\pgfpathlineto{\pgfqpoint{7.208115in}{3.102505in}}%
\pgfpathlineto{\pgfqpoint{7.208990in}{2.863800in}}%
\pgfpathlineto{\pgfqpoint{7.209865in}{2.954646in}}%
\pgfpathlineto{\pgfqpoint{7.210741in}{2.950832in}}%
\pgfpathlineto{\pgfqpoint{7.211616in}{2.849830in}}%
\pgfpathlineto{\pgfqpoint{7.212492in}{2.640123in}}%
\pgfpathlineto{\pgfqpoint{7.213367in}{2.552940in}}%
\pgfpathlineto{\pgfqpoint{7.215118in}{2.255408in}}%
\pgfpathlineto{\pgfqpoint{7.215993in}{2.196091in}}%
\pgfpathlineto{\pgfqpoint{7.216868in}{2.168943in}}%
\pgfpathlineto{\pgfqpoint{7.217744in}{2.006886in}}%
\pgfpathlineto{\pgfqpoint{7.218619in}{1.997787in}}%
\pgfpathlineto{\pgfqpoint{7.219495in}{2.187520in}}%
\pgfpathlineto{\pgfqpoint{7.220370in}{2.313065in}}%
\pgfpathlineto{\pgfqpoint{7.221245in}{2.866934in}}%
\pgfpathlineto{\pgfqpoint{7.222121in}{3.027216in}}%
\pgfpathlineto{\pgfqpoint{7.222996in}{3.091858in}}%
\pgfpathlineto{\pgfqpoint{7.223871in}{3.073885in}}%
\pgfpathlineto{\pgfqpoint{7.224747in}{3.202262in}}%
\pgfpathlineto{\pgfqpoint{7.225622in}{3.218384in}}%
\pgfpathlineto{\pgfqpoint{7.226498in}{3.290200in}}%
\pgfpathlineto{\pgfqpoint{7.227373in}{3.226766in}}%
\pgfpathlineto{\pgfqpoint{7.228248in}{3.253197in}}%
\pgfpathlineto{\pgfqpoint{7.229124in}{2.805653in}}%
\pgfpathlineto{\pgfqpoint{7.229999in}{2.651262in}}%
\pgfpathlineto{\pgfqpoint{7.231750in}{2.841976in}}%
\pgfpathlineto{\pgfqpoint{7.232625in}{2.631439in}}%
\pgfpathlineto{\pgfqpoint{7.233501in}{2.544407in}}%
\pgfpathlineto{\pgfqpoint{7.234376in}{2.592019in}}%
\pgfpathlineto{\pgfqpoint{7.235251in}{2.264206in}}%
\pgfpathlineto{\pgfqpoint{7.236127in}{2.208438in}}%
\pgfpathlineto{\pgfqpoint{7.237002in}{2.191107in}}%
\pgfpathlineto{\pgfqpoint{7.237877in}{2.055971in}}%
\pgfpathlineto{\pgfqpoint{7.238753in}{1.875187in}}%
\pgfpathlineto{\pgfqpoint{7.239628in}{2.026596in}}%
\pgfpathlineto{\pgfqpoint{7.240504in}{2.124880in}}%
\pgfpathlineto{\pgfqpoint{7.241379in}{2.398360in}}%
\pgfpathlineto{\pgfqpoint{7.243130in}{2.776580in}}%
\pgfpathlineto{\pgfqpoint{7.244005in}{2.864631in}}%
\pgfpathlineto{\pgfqpoint{7.244880in}{2.746336in}}%
\pgfpathlineto{\pgfqpoint{7.246631in}{2.971675in}}%
\pgfpathlineto{\pgfqpoint{7.247507in}{2.951210in}}%
\pgfpathlineto{\pgfqpoint{7.248382in}{2.890457in}}%
\pgfpathlineto{\pgfqpoint{7.249257in}{2.874486in}}%
\pgfpathlineto{\pgfqpoint{7.250133in}{2.737312in}}%
\pgfpathlineto{\pgfqpoint{7.251008in}{2.548636in}}%
\pgfpathlineto{\pgfqpoint{7.251883in}{2.557924in}}%
\pgfpathlineto{\pgfqpoint{7.253634in}{2.326167in}}%
\pgfpathlineto{\pgfqpoint{7.254510in}{2.046683in}}%
\pgfpathlineto{\pgfqpoint{7.255385in}{2.075266in}}%
\pgfpathlineto{\pgfqpoint{7.256260in}{2.024066in}}%
\pgfpathlineto{\pgfqpoint{7.257136in}{2.010436in}}%
\pgfpathlineto{\pgfqpoint{7.258887in}{1.863973in}}%
\pgfpathlineto{\pgfqpoint{7.259762in}{1.836598in}}%
\pgfpathlineto{\pgfqpoint{7.260637in}{1.894746in}}%
\pgfpathlineto{\pgfqpoint{7.261513in}{1.890479in}}%
\pgfpathlineto{\pgfqpoint{7.262388in}{2.138133in}}%
\pgfpathlineto{\pgfqpoint{7.263263in}{2.210137in}}%
\pgfpathlineto{\pgfqpoint{7.264139in}{2.399379in}}%
\pgfpathlineto{\pgfqpoint{7.265014in}{2.447407in}}%
\pgfpathlineto{\pgfqpoint{7.266765in}{2.664477in}}%
\pgfpathlineto{\pgfqpoint{7.267640in}{2.732554in}}%
\pgfpathlineto{\pgfqpoint{7.268516in}{2.871503in}}%
\pgfpathlineto{\pgfqpoint{7.269391in}{2.828988in}}%
\pgfpathlineto{\pgfqpoint{7.270266in}{2.821247in}}%
\pgfpathlineto{\pgfqpoint{7.272017in}{2.587111in}}%
\pgfpathlineto{\pgfqpoint{7.272893in}{2.499928in}}%
\pgfpathlineto{\pgfqpoint{7.273768in}{2.507442in}}%
\pgfpathlineto{\pgfqpoint{7.274643in}{2.386239in}}%
\pgfpathlineto{\pgfqpoint{7.275519in}{2.340552in}}%
\pgfpathlineto{\pgfqpoint{7.276394in}{2.121859in}}%
\pgfpathlineto{\pgfqpoint{7.277269in}{2.022745in}}%
\pgfpathlineto{\pgfqpoint{7.278145in}{2.007377in}}%
\pgfpathlineto{\pgfqpoint{7.279896in}{1.805750in}}%
\pgfpathlineto{\pgfqpoint{7.280771in}{1.776903in}}%
\pgfpathlineto{\pgfqpoint{7.281646in}{1.868957in}}%
\pgfpathlineto{\pgfqpoint{7.282522in}{1.999184in}}%
\pgfpathlineto{\pgfqpoint{7.283397in}{1.976604in}}%
\pgfpathlineto{\pgfqpoint{7.284272in}{2.316501in}}%
\pgfpathlineto{\pgfqpoint{7.285148in}{2.536629in}}%
\pgfpathlineto{\pgfqpoint{7.286023in}{2.513181in}}%
\pgfpathlineto{\pgfqpoint{7.286899in}{2.568270in}}%
\pgfpathlineto{\pgfqpoint{7.287774in}{2.704122in}}%
\pgfpathlineto{\pgfqpoint{7.288649in}{2.744750in}}%
\pgfpathlineto{\pgfqpoint{7.289525in}{2.739728in}}%
\pgfpathlineto{\pgfqpoint{7.290400in}{2.949284in}}%
\pgfpathlineto{\pgfqpoint{7.291275in}{2.852416in}}%
\pgfpathlineto{\pgfqpoint{7.293026in}{2.461264in}}%
\pgfpathlineto{\pgfqpoint{7.293902in}{3.581917in}}%
\pgfpathlineto{\pgfqpoint{7.294777in}{3.707348in}}%
\pgfpathlineto{\pgfqpoint{7.295652in}{3.614955in}}%
\pgfpathlineto{\pgfqpoint{7.296528in}{3.562849in}}%
\pgfpathlineto{\pgfqpoint{7.299154in}{3.218762in}}%
\pgfpathlineto{\pgfqpoint{7.300029in}{3.271321in}}%
\pgfpathlineto{\pgfqpoint{7.300905in}{3.075395in}}%
\pgfpathlineto{\pgfqpoint{7.301780in}{2.967030in}}%
\pgfpathlineto{\pgfqpoint{7.302655in}{3.266676in}}%
\pgfpathlineto{\pgfqpoint{7.303531in}{3.464452in}}%
\pgfpathlineto{\pgfqpoint{7.304406in}{3.475081in}}%
\pgfpathlineto{\pgfqpoint{7.305281in}{3.936425in}}%
\pgfpathlineto{\pgfqpoint{7.306157in}{2.582542in}}%
\pgfpathlineto{\pgfqpoint{7.307032in}{3.311080in}}%
\pgfpathlineto{\pgfqpoint{7.307908in}{2.890004in}}%
\pgfpathlineto{\pgfqpoint{7.308783in}{2.789191in}}%
\pgfpathlineto{\pgfqpoint{7.309658in}{2.906580in}}%
\pgfpathlineto{\pgfqpoint{7.310534in}{2.892043in}}%
\pgfpathlineto{\pgfqpoint{7.312284in}{2.662362in}}%
\pgfpathlineto{\pgfqpoint{7.314035in}{2.181063in}}%
\pgfpathlineto{\pgfqpoint{7.314911in}{2.227365in}}%
\pgfpathlineto{\pgfqpoint{7.315786in}{2.321582in}}%
\pgfpathlineto{\pgfqpoint{7.316661in}{2.199823in}}%
\pgfpathlineto{\pgfqpoint{7.317537in}{2.167136in}}%
\pgfpathlineto{\pgfqpoint{7.320163in}{2.455676in}}%
\pgfpathlineto{\pgfqpoint{7.321038in}{2.502771in}}%
\pgfpathlineto{\pgfqpoint{7.321914in}{2.613941in}}%
\pgfpathlineto{\pgfqpoint{7.322789in}{2.593222in}}%
\pgfpathlineto{\pgfqpoint{7.323664in}{2.742150in}}%
\pgfpathlineto{\pgfqpoint{7.325415in}{3.219770in}}%
\pgfpathlineto{\pgfqpoint{7.327166in}{3.592602in}}%
\pgfpathlineto{\pgfqpoint{7.328041in}{2.565627in}}%
\pgfpathlineto{\pgfqpoint{7.328917in}{2.641444in}}%
\pgfpathlineto{\pgfqpoint{7.329792in}{2.634535in}}%
\pgfpathlineto{\pgfqpoint{7.331543in}{2.967672in}}%
\pgfpathlineto{\pgfqpoint{7.332418in}{2.927913in}}%
\pgfpathlineto{\pgfqpoint{7.333293in}{2.949586in}}%
\pgfpathlineto{\pgfqpoint{7.335044in}{2.515258in}}%
\pgfpathlineto{\pgfqpoint{7.339421in}{1.884022in}}%
\pgfpathlineto{\pgfqpoint{7.340296in}{1.740694in}}%
\pgfpathlineto{\pgfqpoint{7.341172in}{1.778602in}}%
\pgfpathlineto{\pgfqpoint{7.342047in}{1.598044in}}%
\pgfpathlineto{\pgfqpoint{7.342923in}{1.557115in}}%
\pgfpathlineto{\pgfqpoint{7.344673in}{1.778753in}}%
\pgfpathlineto{\pgfqpoint{7.346424in}{2.125710in}}%
\pgfpathlineto{\pgfqpoint{7.347299in}{2.311932in}}%
\pgfpathlineto{\pgfqpoint{7.348175in}{2.369475in}}%
\pgfpathlineto{\pgfqpoint{7.349926in}{2.704727in}}%
\pgfpathlineto{\pgfqpoint{7.350801in}{2.701744in}}%
\pgfpathlineto{\pgfqpoint{7.351676in}{2.848508in}}%
\pgfpathlineto{\pgfqpoint{7.353427in}{2.989156in}}%
\pgfpathlineto{\pgfqpoint{7.354302in}{2.977527in}}%
\pgfpathlineto{\pgfqpoint{7.355178in}{2.661456in}}%
\pgfpathlineto{\pgfqpoint{7.357804in}{2.278780in}}%
\pgfpathlineto{\pgfqpoint{7.358679in}{2.308307in}}%
\pgfpathlineto{\pgfqpoint{7.360430in}{1.998882in}}%
\pgfpathlineto{\pgfqpoint{7.361305in}{1.780037in}}%
\pgfpathlineto{\pgfqpoint{7.363056in}{1.647016in}}%
\pgfpathlineto{\pgfqpoint{7.363932in}{1.543069in}}%
\pgfpathlineto{\pgfqpoint{7.364807in}{1.591475in}}%
\pgfpathlineto{\pgfqpoint{7.365682in}{1.685718in}}%
\pgfpathlineto{\pgfqpoint{7.367433in}{2.087499in}}%
\pgfpathlineto{\pgfqpoint{7.368308in}{2.216631in}}%
\pgfpathlineto{\pgfqpoint{7.369184in}{2.441706in}}%
\pgfpathlineto{\pgfqpoint{7.370059in}{2.578653in}}%
\pgfpathlineto{\pgfqpoint{7.370935in}{2.572046in}}%
\pgfpathlineto{\pgfqpoint{7.371810in}{2.733838in}}%
\pgfpathlineto{\pgfqpoint{7.372685in}{2.699894in}}%
\pgfpathlineto{\pgfqpoint{7.373561in}{2.832763in}}%
\pgfpathlineto{\pgfqpoint{7.374436in}{2.920513in}}%
\pgfpathlineto{\pgfqpoint{7.375311in}{2.928970in}}%
\pgfpathlineto{\pgfqpoint{7.377062in}{2.503817in}}%
\pgfpathlineto{\pgfqpoint{7.378813in}{2.363547in}}%
\pgfpathlineto{\pgfqpoint{7.379688in}{2.396736in}}%
\pgfpathlineto{\pgfqpoint{7.382314in}{1.908414in}}%
\pgfpathlineto{\pgfqpoint{7.383190in}{1.896294in}}%
\pgfpathlineto{\pgfqpoint{7.384065in}{1.876508in}}%
\pgfpathlineto{\pgfqpoint{7.385816in}{1.809148in}}%
\pgfpathlineto{\pgfqpoint{7.386691in}{1.881493in}}%
\pgfpathlineto{\pgfqpoint{7.387567in}{2.132658in}}%
\pgfpathlineto{\pgfqpoint{7.388442in}{2.200018in}}%
\pgfpathlineto{\pgfqpoint{7.389318in}{2.314952in}}%
\pgfpathlineto{\pgfqpoint{7.390193in}{2.346027in}}%
\pgfpathlineto{\pgfqpoint{7.391068in}{2.459489in}}%
\pgfpathlineto{\pgfqpoint{7.391944in}{2.404174in}}%
\pgfpathlineto{\pgfqpoint{7.392819in}{2.424979in}}%
\pgfpathlineto{\pgfqpoint{7.393694in}{2.526774in}}%
\pgfpathlineto{\pgfqpoint{7.394570in}{2.461793in}}%
\pgfpathlineto{\pgfqpoint{7.396321in}{2.689963in}}%
\pgfpathlineto{\pgfqpoint{7.398071in}{2.427433in}}%
\pgfpathlineto{\pgfqpoint{7.398947in}{2.354145in}}%
\pgfpathlineto{\pgfqpoint{7.400697in}{2.107322in}}%
\pgfpathlineto{\pgfqpoint{7.402448in}{1.738994in}}%
\pgfpathlineto{\pgfqpoint{7.405074in}{1.590229in}}%
\pgfpathlineto{\pgfqpoint{7.405950in}{1.552660in}}%
\pgfpathlineto{\pgfqpoint{7.406825in}{1.488509in}}%
\pgfpathlineto{\pgfqpoint{7.407700in}{1.547185in}}%
\pgfpathlineto{\pgfqpoint{7.409451in}{1.710336in}}%
\pgfpathlineto{\pgfqpoint{7.410327in}{1.689834in}}%
\pgfpathlineto{\pgfqpoint{7.411202in}{1.804202in}}%
\pgfpathlineto{\pgfqpoint{7.412077in}{1.984269in}}%
\pgfpathlineto{\pgfqpoint{7.415579in}{2.923684in}}%
\pgfpathlineto{\pgfqpoint{7.416454in}{3.000333in}}%
\pgfpathlineto{\pgfqpoint{7.417330in}{3.011962in}}%
\pgfpathlineto{\pgfqpoint{7.418205in}{2.869691in}}%
\pgfpathlineto{\pgfqpoint{7.419080in}{2.663420in}}%
\pgfpathlineto{\pgfqpoint{7.419956in}{2.579899in}}%
\pgfpathlineto{\pgfqpoint{7.422582in}{2.135678in}}%
\pgfpathlineto{\pgfqpoint{7.423457in}{1.950703in}}%
\pgfpathlineto{\pgfqpoint{7.424333in}{1.887383in}}%
\pgfpathlineto{\pgfqpoint{7.426083in}{1.691004in}}%
\pgfpathlineto{\pgfqpoint{7.426959in}{1.672578in}}%
\pgfpathlineto{\pgfqpoint{7.427834in}{1.623153in}}%
\pgfpathlineto{\pgfqpoint{7.433086in}{2.642011in}}%
\pgfpathlineto{\pgfqpoint{7.433962in}{2.681808in}}%
\pgfpathlineto{\pgfqpoint{7.435712in}{2.966275in}}%
\pgfpathlineto{\pgfqpoint{7.436588in}{3.144605in}}%
\pgfpathlineto{\pgfqpoint{7.437463in}{3.254254in}}%
\pgfpathlineto{\pgfqpoint{7.438339in}{3.247835in}}%
\pgfpathlineto{\pgfqpoint{7.440089in}{2.769783in}}%
\pgfpathlineto{\pgfqpoint{7.440965in}{2.699101in}}%
\pgfpathlineto{\pgfqpoint{7.441840in}{2.520317in}}%
\pgfpathlineto{\pgfqpoint{7.442715in}{2.455751in}}%
\pgfpathlineto{\pgfqpoint{7.443591in}{2.238417in}}%
\pgfpathlineto{\pgfqpoint{7.444466in}{2.135112in}}%
\pgfpathlineto{\pgfqpoint{7.445342in}{1.977851in}}%
\pgfpathlineto{\pgfqpoint{7.447092in}{1.785701in}}%
\pgfpathlineto{\pgfqpoint{7.447968in}{1.718492in}}%
\pgfpathlineto{\pgfqpoint{7.448843in}{1.717624in}}%
\pgfpathlineto{\pgfqpoint{7.449718in}{1.783738in}}%
\pgfpathlineto{\pgfqpoint{7.453220in}{3.013133in}}%
\pgfpathlineto{\pgfqpoint{7.454095in}{2.974129in}}%
\pgfpathlineto{\pgfqpoint{7.454971in}{3.067655in}}%
\pgfpathlineto{\pgfqpoint{7.455846in}{3.038770in}}%
\pgfpathlineto{\pgfqpoint{7.456721in}{3.223935in}}%
\pgfpathlineto{\pgfqpoint{7.457597in}{3.312968in}}%
\pgfpathlineto{\pgfqpoint{7.458472in}{3.261579in}}%
\pgfpathlineto{\pgfqpoint{7.459348in}{3.164466in}}%
\pgfpathlineto{\pgfqpoint{7.461974in}{2.649751in}}%
\pgfpathlineto{\pgfqpoint{7.462849in}{2.563097in}}%
\pgfpathlineto{\pgfqpoint{7.463724in}{2.436193in}}%
\pgfpathlineto{\pgfqpoint{7.464600in}{2.205757in}}%
\pgfpathlineto{\pgfqpoint{7.465475in}{2.098185in}}%
\pgfpathlineto{\pgfqpoint{7.466351in}{1.950136in}}%
\pgfpathlineto{\pgfqpoint{7.468977in}{1.807638in}}%
\pgfpathlineto{\pgfqpoint{7.469852in}{1.829651in}}%
\pgfpathlineto{\pgfqpoint{7.470727in}{1.943529in}}%
\pgfpathlineto{\pgfqpoint{7.471603in}{2.110834in}}%
\pgfpathlineto{\pgfqpoint{7.472478in}{2.355051in}}%
\pgfpathlineto{\pgfqpoint{7.473354in}{2.503100in}}%
\pgfpathlineto{\pgfqpoint{7.475104in}{2.867425in}}%
\pgfpathlineto{\pgfqpoint{7.475980in}{2.958648in}}%
\pgfpathlineto{\pgfqpoint{7.476855in}{2.988250in}}%
\pgfpathlineto{\pgfqpoint{7.478606in}{3.263014in}}%
\pgfpathlineto{\pgfqpoint{7.479481in}{3.243342in}}%
\pgfpathlineto{\pgfqpoint{7.480357in}{3.284914in}}%
\pgfpathlineto{\pgfqpoint{7.482107in}{2.812752in}}%
\pgfpathlineto{\pgfqpoint{7.482983in}{2.690945in}}%
\pgfpathlineto{\pgfqpoint{7.483858in}{2.618450in}}%
\pgfpathlineto{\pgfqpoint{7.485609in}{2.267038in}}%
\pgfpathlineto{\pgfqpoint{7.488235in}{1.928501in}}%
\pgfpathlineto{\pgfqpoint{7.489110in}{1.871638in}}%
\pgfpathlineto{\pgfqpoint{7.489986in}{1.939451in}}%
\pgfpathlineto{\pgfqpoint{7.490861in}{1.855251in}}%
\pgfpathlineto{\pgfqpoint{7.492612in}{2.229620in}}%
\pgfpathlineto{\pgfqpoint{7.493487in}{2.291278in}}%
\pgfpathlineto{\pgfqpoint{7.495238in}{2.616109in}}%
\pgfpathlineto{\pgfqpoint{7.496113in}{2.732177in}}%
\pgfpathlineto{\pgfqpoint{7.496989in}{2.772464in}}%
\pgfpathlineto{\pgfqpoint{7.497864in}{2.762043in}}%
\pgfpathlineto{\pgfqpoint{7.498739in}{2.803463in}}%
\pgfpathlineto{\pgfqpoint{7.499615in}{2.782130in}}%
\pgfpathlineto{\pgfqpoint{7.500490in}{2.795006in}}%
\pgfpathlineto{\pgfqpoint{7.501366in}{2.731497in}}%
\pgfpathlineto{\pgfqpoint{7.502241in}{2.427660in}}%
\pgfpathlineto{\pgfqpoint{7.503992in}{2.091955in}}%
\pgfpathlineto{\pgfqpoint{7.504867in}{2.050799in}}%
\pgfpathlineto{\pgfqpoint{7.505742in}{1.936883in}}%
\pgfpathlineto{\pgfqpoint{7.507493in}{1.661666in}}%
\pgfpathlineto{\pgfqpoint{7.510119in}{1.391056in}}%
\pgfpathlineto{\pgfqpoint{7.510995in}{1.349447in}}%
\pgfpathlineto{\pgfqpoint{7.511870in}{1.277745in}}%
\pgfpathlineto{\pgfqpoint{7.512745in}{1.405706in}}%
\pgfpathlineto{\pgfqpoint{7.515372in}{1.950891in}}%
\pgfpathlineto{\pgfqpoint{7.516247in}{1.971016in}}%
\pgfpathlineto{\pgfqpoint{7.517998in}{2.453524in}}%
\pgfpathlineto{\pgfqpoint{7.518873in}{2.527718in}}%
\pgfpathlineto{\pgfqpoint{7.519748in}{2.475612in}}%
\pgfpathlineto{\pgfqpoint{7.520624in}{2.726362in}}%
\pgfpathlineto{\pgfqpoint{7.521499in}{2.376649in}}%
\pgfpathlineto{\pgfqpoint{7.522375in}{2.551581in}}%
\pgfpathlineto{\pgfqpoint{7.524125in}{1.896143in}}%
\pgfpathlineto{\pgfqpoint{7.525001in}{1.853967in}}%
\pgfpathlineto{\pgfqpoint{7.526752in}{1.629950in}}%
\pgfpathlineto{\pgfqpoint{7.527627in}{1.563420in}}%
\pgfpathlineto{\pgfqpoint{7.528502in}{1.404045in}}%
\pgfpathlineto{\pgfqpoint{7.529378in}{1.303269in}}%
\pgfpathlineto{\pgfqpoint{7.530253in}{1.284994in}}%
\pgfpathlineto{\pgfqpoint{7.531128in}{1.285712in}}%
\pgfpathlineto{\pgfqpoint{7.532004in}{1.260489in}}%
\pgfpathlineto{\pgfqpoint{7.532879in}{1.263170in}}%
\pgfpathlineto{\pgfqpoint{7.533755in}{1.352279in}}%
\pgfpathlineto{\pgfqpoint{7.536381in}{1.731707in}}%
\pgfpathlineto{\pgfqpoint{7.537256in}{1.746961in}}%
\pgfpathlineto{\pgfqpoint{7.538131in}{1.798539in}}%
\pgfpathlineto{\pgfqpoint{7.539007in}{1.869146in}}%
\pgfpathlineto{\pgfqpoint{7.540758in}{1.910302in}}%
\pgfpathlineto{\pgfqpoint{7.541633in}{1.986195in}}%
\pgfpathlineto{\pgfqpoint{7.542508in}{1.966372in}}%
\pgfpathlineto{\pgfqpoint{7.543384in}{1.870580in}}%
\pgfpathlineto{\pgfqpoint{7.544259in}{1.857479in}}%
\pgfpathlineto{\pgfqpoint{7.546010in}{1.670049in}}%
\pgfpathlineto{\pgfqpoint{7.546885in}{1.692628in}}%
\pgfpathlineto{\pgfqpoint{7.548636in}{1.488660in}}%
\pgfpathlineto{\pgfqpoint{7.549511in}{1.420847in}}%
\pgfpathlineto{\pgfqpoint{7.550387in}{1.383504in}}%
\pgfpathlineto{\pgfqpoint{7.551262in}{1.369761in}}%
\pgfpathlineto{\pgfqpoint{7.552137in}{1.281823in}}%
\pgfpathlineto{\pgfqpoint{7.553013in}{1.311727in}}%
\pgfpathlineto{\pgfqpoint{7.553888in}{1.314936in}}%
\pgfpathlineto{\pgfqpoint{7.554764in}{1.344387in}}%
\pgfpathlineto{\pgfqpoint{7.555639in}{1.407858in}}%
\pgfpathlineto{\pgfqpoint{7.556514in}{1.392793in}}%
\pgfpathlineto{\pgfqpoint{7.557390in}{1.495494in}}%
\pgfpathlineto{\pgfqpoint{7.558265in}{1.468082in}}%
\pgfpathlineto{\pgfqpoint{7.559140in}{1.553037in}}%
\pgfpathlineto{\pgfqpoint{7.560016in}{1.529552in}}%
\pgfpathlineto{\pgfqpoint{7.560891in}{1.555265in}}%
\pgfpathlineto{\pgfqpoint{7.561767in}{1.660269in}}%
\pgfpathlineto{\pgfqpoint{7.562642in}{1.709732in}}%
\pgfpathlineto{\pgfqpoint{7.563517in}{1.706523in}}%
\pgfpathlineto{\pgfqpoint{7.564393in}{1.710940in}}%
\pgfpathlineto{\pgfqpoint{7.565268in}{1.554736in}}%
\pgfpathlineto{\pgfqpoint{7.566143in}{1.530609in}}%
\pgfpathlineto{\pgfqpoint{7.567019in}{1.489075in}}%
\pgfpathlineto{\pgfqpoint{7.567894in}{1.535517in}}%
\pgfpathlineto{\pgfqpoint{7.568770in}{1.448108in}}%
\pgfpathlineto{\pgfqpoint{7.569645in}{1.407896in}}%
\pgfpathlineto{\pgfqpoint{7.570520in}{1.258564in}}%
\pgfpathlineto{\pgfqpoint{7.571396in}{1.223562in}}%
\pgfpathlineto{\pgfqpoint{7.572271in}{1.239458in}}%
\pgfpathlineto{\pgfqpoint{7.573146in}{1.338422in}}%
\pgfpathlineto{\pgfqpoint{7.574022in}{1.378709in}}%
\pgfpathlineto{\pgfqpoint{7.574897in}{1.396908in}}%
\pgfpathlineto{\pgfqpoint{7.575773in}{1.425000in}}%
\pgfpathlineto{\pgfqpoint{7.577523in}{1.772146in}}%
\pgfpathlineto{\pgfqpoint{7.578399in}{1.821042in}}%
\pgfpathlineto{\pgfqpoint{7.579274in}{1.821269in}}%
\pgfpathlineto{\pgfqpoint{7.580149in}{1.826404in}}%
\pgfpathlineto{\pgfqpoint{7.581025in}{1.904072in}}%
\pgfpathlineto{\pgfqpoint{7.581900in}{1.938658in}}%
\pgfpathlineto{\pgfqpoint{7.582776in}{1.985704in}}%
\pgfpathlineto{\pgfqpoint{7.583651in}{2.093012in}}%
\pgfpathlineto{\pgfqpoint{7.584526in}{2.077078in}}%
\pgfpathlineto{\pgfqpoint{7.585402in}{2.109437in}}%
\pgfpathlineto{\pgfqpoint{7.586277in}{1.983363in}}%
\pgfpathlineto{\pgfqpoint{7.587152in}{1.934693in}}%
\pgfpathlineto{\pgfqpoint{7.588028in}{1.843886in}}%
\pgfpathlineto{\pgfqpoint{7.588903in}{1.818777in}}%
\pgfpathlineto{\pgfqpoint{7.589779in}{1.738919in}}%
\pgfpathlineto{\pgfqpoint{7.590654in}{1.685114in}}%
\pgfpathlineto{\pgfqpoint{7.591529in}{1.614092in}}%
\pgfpathlineto{\pgfqpoint{7.592405in}{1.504669in}}%
\pgfpathlineto{\pgfqpoint{7.593280in}{1.580714in}}%
\pgfpathlineto{\pgfqpoint{7.594155in}{1.503914in}}%
\pgfpathlineto{\pgfqpoint{7.595031in}{1.556964in}}%
\pgfpathlineto{\pgfqpoint{7.595906in}{1.504820in}}%
\pgfpathlineto{\pgfqpoint{7.596782in}{1.596157in}}%
\pgfpathlineto{\pgfqpoint{7.598532in}{1.871487in}}%
\pgfpathlineto{\pgfqpoint{7.599408in}{1.929747in}}%
\pgfpathlineto{\pgfqpoint{7.600283in}{1.948701in}}%
\pgfpathlineto{\pgfqpoint{7.601158in}{1.956782in}}%
\pgfpathlineto{\pgfqpoint{7.602034in}{2.075606in}}%
\pgfpathlineto{\pgfqpoint{7.603785in}{1.966863in}}%
\pgfpathlineto{\pgfqpoint{7.604660in}{2.137075in}}%
\pgfpathlineto{\pgfqpoint{7.605535in}{2.205077in}}%
\pgfpathlineto{\pgfqpoint{7.606411in}{2.207569in}}%
\pgfpathlineto{\pgfqpoint{7.607286in}{1.993973in}}%
\pgfpathlineto{\pgfqpoint{7.609912in}{1.783586in}}%
\pgfpathlineto{\pgfqpoint{7.610788in}{1.638861in}}%
\pgfpathlineto{\pgfqpoint{7.611663in}{1.600763in}}%
\pgfpathlineto{\pgfqpoint{7.612538in}{1.525096in}}%
\pgfpathlineto{\pgfqpoint{7.613414in}{1.478579in}}%
\pgfpathlineto{\pgfqpoint{7.614289in}{1.461625in}}%
\pgfpathlineto{\pgfqpoint{7.615164in}{1.429607in}}%
\pgfpathlineto{\pgfqpoint{7.616915in}{1.551716in}}%
\pgfpathlineto{\pgfqpoint{7.617791in}{1.588718in}}%
\pgfpathlineto{\pgfqpoint{7.618666in}{1.753493in}}%
\pgfpathlineto{\pgfqpoint{7.619541in}{1.997334in}}%
\pgfpathlineto{\pgfqpoint{7.620417in}{2.039018in}}%
\pgfpathlineto{\pgfqpoint{7.621292in}{2.050195in}}%
\pgfpathlineto{\pgfqpoint{7.622167in}{2.017119in}}%
\pgfpathlineto{\pgfqpoint{7.623043in}{2.058161in}}%
\pgfpathlineto{\pgfqpoint{7.624794in}{2.344366in}}%
\pgfpathlineto{\pgfqpoint{7.625669in}{2.384351in}}%
\pgfpathlineto{\pgfqpoint{7.626544in}{2.370457in}}%
\pgfpathlineto{\pgfqpoint{7.627420in}{2.315557in}}%
\pgfpathlineto{\pgfqpoint{7.630046in}{1.900258in}}%
\pgfpathlineto{\pgfqpoint{7.630921in}{1.862425in}}%
\pgfpathlineto{\pgfqpoint{7.631797in}{1.873941in}}%
\pgfpathlineto{\pgfqpoint{7.632672in}{1.734312in}}%
\pgfpathlineto{\pgfqpoint{7.633547in}{1.642901in}}%
\pgfpathlineto{\pgfqpoint{7.634423in}{1.610014in}}%
\pgfpathlineto{\pgfqpoint{7.635298in}{1.551149in}}%
\pgfpathlineto{\pgfqpoint{7.636173in}{1.537103in}}%
\pgfpathlineto{\pgfqpoint{7.637049in}{1.514373in}}%
\pgfpathlineto{\pgfqpoint{7.637924in}{1.503839in}}%
\pgfpathlineto{\pgfqpoint{7.638800in}{1.563911in}}%
\pgfpathlineto{\pgfqpoint{7.639675in}{1.686209in}}%
\pgfpathlineto{\pgfqpoint{7.641426in}{1.786456in}}%
\pgfpathlineto{\pgfqpoint{7.642301in}{1.770409in}}%
\pgfpathlineto{\pgfqpoint{7.644052in}{1.917816in}}%
\pgfpathlineto{\pgfqpoint{7.644927in}{2.037546in}}%
\pgfpathlineto{\pgfqpoint{7.645803in}{2.069753in}}%
\pgfpathlineto{\pgfqpoint{7.646678in}{2.056613in}}%
\pgfpathlineto{\pgfqpoint{7.647553in}{2.123596in}}%
\pgfpathlineto{\pgfqpoint{7.648429in}{2.106303in}}%
\pgfpathlineto{\pgfqpoint{7.649304in}{2.001147in}}%
\pgfpathlineto{\pgfqpoint{7.650179in}{1.840714in}}%
\pgfpathlineto{\pgfqpoint{7.651055in}{1.791478in}}%
\pgfpathlineto{\pgfqpoint{7.651930in}{1.666839in}}%
\pgfpathlineto{\pgfqpoint{7.652806in}{1.723854in}}%
\pgfpathlineto{\pgfqpoint{7.653681in}{1.619264in}}%
\pgfpathlineto{\pgfqpoint{7.654556in}{1.566064in}}%
\pgfpathlineto{\pgfqpoint{7.655432in}{1.529929in}}%
\pgfpathlineto{\pgfqpoint{7.656307in}{1.564251in}}%
\pgfpathlineto{\pgfqpoint{7.657183in}{1.486583in}}%
\pgfpathlineto{\pgfqpoint{7.658058in}{1.525172in}}%
\pgfpathlineto{\pgfqpoint{7.658933in}{1.530609in}}%
\pgfpathlineto{\pgfqpoint{7.659809in}{1.628364in}}%
\pgfpathlineto{\pgfqpoint{7.660684in}{1.779848in}}%
\pgfpathlineto{\pgfqpoint{7.661559in}{1.819532in}}%
\pgfpathlineto{\pgfqpoint{7.662435in}{1.885419in}}%
\pgfpathlineto{\pgfqpoint{7.663310in}{1.845094in}}%
\pgfpathlineto{\pgfqpoint{7.664186in}{1.862878in}}%
\pgfpathlineto{\pgfqpoint{7.665061in}{1.947946in}}%
\pgfpathlineto{\pgfqpoint{7.665936in}{1.997900in}}%
\pgfpathlineto{\pgfqpoint{7.666812in}{2.109286in}}%
\pgfpathlineto{\pgfqpoint{7.667687in}{2.153500in}}%
\pgfpathlineto{\pgfqpoint{7.668562in}{2.162411in}}%
\pgfpathlineto{\pgfqpoint{7.669438in}{2.192957in}}%
\pgfpathlineto{\pgfqpoint{7.670313in}{2.143909in}}%
\pgfpathlineto{\pgfqpoint{7.671189in}{2.019988in}}%
\pgfpathlineto{\pgfqpoint{7.672064in}{1.985402in}}%
\pgfpathlineto{\pgfqpoint{7.672939in}{1.878623in}}%
\pgfpathlineto{\pgfqpoint{7.673815in}{1.902184in}}%
\pgfpathlineto{\pgfqpoint{7.674690in}{1.886892in}}%
\pgfpathlineto{\pgfqpoint{7.675565in}{1.780943in}}%
\pgfpathlineto{\pgfqpoint{7.676441in}{1.732840in}}%
\pgfpathlineto{\pgfqpoint{7.677316in}{1.701086in}}%
\pgfpathlineto{\pgfqpoint{7.679067in}{1.624513in}}%
\pgfpathlineto{\pgfqpoint{7.679942in}{1.592116in}}%
\pgfpathlineto{\pgfqpoint{7.680818in}{1.626854in}}%
\pgfpathlineto{\pgfqpoint{7.681693in}{1.686020in}}%
\pgfpathlineto{\pgfqpoint{7.682568in}{1.982608in}}%
\pgfpathlineto{\pgfqpoint{7.683444in}{2.073038in}}%
\pgfpathlineto{\pgfqpoint{7.684319in}{2.077003in}}%
\pgfpathlineto{\pgfqpoint{7.685195in}{2.153727in}}%
\pgfpathlineto{\pgfqpoint{7.686070in}{2.274023in}}%
\pgfpathlineto{\pgfqpoint{7.686945in}{2.256881in}}%
\pgfpathlineto{\pgfqpoint{7.687821in}{2.380727in}}%
\pgfpathlineto{\pgfqpoint{7.688696in}{2.462170in}}%
\pgfpathlineto{\pgfqpoint{7.689571in}{2.445444in}}%
\pgfpathlineto{\pgfqpoint{7.690447in}{2.392809in}}%
\pgfpathlineto{\pgfqpoint{7.692198in}{2.142701in}}%
\pgfpathlineto{\pgfqpoint{7.693073in}{2.064165in}}%
\pgfpathlineto{\pgfqpoint{7.693948in}{2.101319in}}%
\pgfpathlineto{\pgfqpoint{7.694824in}{2.045286in}}%
\pgfpathlineto{\pgfqpoint{7.697450in}{1.760403in}}%
\pgfpathlineto{\pgfqpoint{7.698325in}{1.731330in}}%
\pgfpathlineto{\pgfqpoint{7.699201in}{1.634141in}}%
\pgfpathlineto{\pgfqpoint{7.700076in}{1.610769in}}%
\pgfpathlineto{\pgfqpoint{7.700951in}{1.636029in}}%
\pgfpathlineto{\pgfqpoint{7.701827in}{1.772939in}}%
\pgfpathlineto{\pgfqpoint{7.702702in}{1.856950in}}%
\pgfpathlineto{\pgfqpoint{7.703577in}{1.857176in}}%
\pgfpathlineto{\pgfqpoint{7.705328in}{2.250122in}}%
\pgfpathlineto{\pgfqpoint{7.706204in}{2.420108in}}%
\pgfpathlineto{\pgfqpoint{7.707079in}{2.533155in}}%
\pgfpathlineto{\pgfqpoint{7.707954in}{2.603385in}}%
\pgfpathlineto{\pgfqpoint{7.708830in}{2.615958in}}%
\pgfpathlineto{\pgfqpoint{7.709705in}{2.642426in}}%
\pgfpathlineto{\pgfqpoint{7.710580in}{2.699894in}}%
\pgfpathlineto{\pgfqpoint{7.711456in}{2.723077in}}%
\pgfpathlineto{\pgfqpoint{7.713207in}{2.506536in}}%
\pgfpathlineto{\pgfqpoint{7.714082in}{2.418409in}}%
\pgfpathlineto{\pgfqpoint{7.714957in}{2.254087in}}%
\pgfpathlineto{\pgfqpoint{7.715833in}{2.213950in}}%
\pgfpathlineto{\pgfqpoint{7.719334in}{1.729706in}}%
\pgfpathlineto{\pgfqpoint{7.721085in}{1.652793in}}%
\pgfpathlineto{\pgfqpoint{7.721960in}{1.652302in}}%
\pgfpathlineto{\pgfqpoint{7.723711in}{1.816700in}}%
\pgfpathlineto{\pgfqpoint{7.726337in}{2.411046in}}%
\pgfpathlineto{\pgfqpoint{7.727213in}{2.432757in}}%
\pgfpathlineto{\pgfqpoint{7.728088in}{2.526736in}}%
\pgfpathlineto{\pgfqpoint{7.728963in}{2.476669in}}%
\pgfpathlineto{\pgfqpoint{7.730714in}{2.702008in}}%
\pgfpathlineto{\pgfqpoint{7.731589in}{2.691172in}}%
\pgfpathlineto{\pgfqpoint{7.732465in}{2.649713in}}%
\pgfpathlineto{\pgfqpoint{7.734216in}{2.282594in}}%
\pgfpathlineto{\pgfqpoint{7.735091in}{2.264244in}}%
\pgfpathlineto{\pgfqpoint{7.736842in}{2.182838in}}%
\pgfpathlineto{\pgfqpoint{7.737717in}{2.029843in}}%
\pgfpathlineto{\pgfqpoint{7.741219in}{1.598120in}}%
\pgfpathlineto{\pgfqpoint{7.742094in}{1.574484in}}%
\pgfpathlineto{\pgfqpoint{7.742969in}{1.569084in}}%
\pgfpathlineto{\pgfqpoint{7.743845in}{1.690967in}}%
\pgfpathlineto{\pgfqpoint{7.745595in}{2.176797in}}%
\pgfpathlineto{\pgfqpoint{7.746471in}{2.531229in}}%
\pgfpathlineto{\pgfqpoint{7.747346in}{2.437212in}}%
\pgfpathlineto{\pgfqpoint{7.749097in}{2.788511in}}%
\pgfpathlineto{\pgfqpoint{7.749972in}{2.776391in}}%
\pgfpathlineto{\pgfqpoint{7.751723in}{2.998407in}}%
\pgfpathlineto{\pgfqpoint{7.752598in}{3.044962in}}%
\pgfpathlineto{\pgfqpoint{7.753474in}{3.005052in}}%
\pgfpathlineto{\pgfqpoint{7.754349in}{2.664439in}}%
\pgfpathlineto{\pgfqpoint{7.755225in}{2.537384in}}%
\pgfpathlineto{\pgfqpoint{7.756100in}{2.575104in}}%
\pgfpathlineto{\pgfqpoint{7.756975in}{2.527227in}}%
\pgfpathlineto{\pgfqpoint{7.757851in}{2.504723in}}%
\pgfpathlineto{\pgfqpoint{7.760477in}{1.978304in}}%
\pgfpathlineto{\pgfqpoint{7.762228in}{1.836296in}}%
\pgfpathlineto{\pgfqpoint{7.763978in}{1.744696in}}%
\pgfpathlineto{\pgfqpoint{7.764854in}{1.755985in}}%
\pgfpathlineto{\pgfqpoint{7.766604in}{2.219312in}}%
\pgfpathlineto{\pgfqpoint{7.767480in}{2.445972in}}%
\pgfpathlineto{\pgfqpoint{7.768355in}{2.545842in}}%
\pgfpathlineto{\pgfqpoint{7.769231in}{2.502571in}}%
\pgfpathlineto{\pgfqpoint{7.770106in}{2.663759in}}%
\pgfpathlineto{\pgfqpoint{7.770981in}{2.717489in}}%
\pgfpathlineto{\pgfqpoint{7.771857in}{2.845412in}}%
\pgfpathlineto{\pgfqpoint{7.772732in}{2.927887in}}%
\pgfpathlineto{\pgfqpoint{7.773607in}{2.890067in}}%
\pgfpathlineto{\pgfqpoint{7.774483in}{2.817623in}}%
\pgfpathlineto{\pgfqpoint{7.775358in}{2.608746in}}%
\pgfpathlineto{\pgfqpoint{7.777984in}{2.366001in}}%
\pgfpathlineto{\pgfqpoint{7.778860in}{2.195902in}}%
\pgfpathlineto{\pgfqpoint{7.779735in}{2.100979in}}%
\pgfpathlineto{\pgfqpoint{7.780610in}{1.903166in}}%
\pgfpathlineto{\pgfqpoint{7.781486in}{1.782869in}}%
\pgfpathlineto{\pgfqpoint{7.783237in}{1.676543in}}%
\pgfpathlineto{\pgfqpoint{7.784112in}{1.638974in}}%
\pgfpathlineto{\pgfqpoint{7.784987in}{1.648376in}}%
\pgfpathlineto{\pgfqpoint{7.785863in}{1.710147in}}%
\pgfpathlineto{\pgfqpoint{7.787614in}{2.050976in}}%
\pgfpathlineto{\pgfqpoint{7.788489in}{2.281329in}}%
\pgfpathlineto{\pgfqpoint{7.789364in}{2.374818in}}%
\pgfpathlineto{\pgfqpoint{7.791115in}{2.696311in}}%
\pgfpathlineto{\pgfqpoint{7.791990in}{2.679697in}}%
\pgfpathlineto{\pgfqpoint{7.792866in}{2.807732in}}%
\pgfpathlineto{\pgfqpoint{7.794617in}{2.852264in}}%
\pgfpathlineto{\pgfqpoint{7.795492in}{2.781186in}}%
\pgfpathlineto{\pgfqpoint{7.797243in}{2.354145in}}%
\pgfpathlineto{\pgfqpoint{7.798118in}{2.302039in}}%
\pgfpathlineto{\pgfqpoint{7.798993in}{2.273230in}}%
\pgfpathlineto{\pgfqpoint{7.799869in}{2.346405in}}%
\pgfpathlineto{\pgfqpoint{7.802495in}{1.878245in}}%
\pgfpathlineto{\pgfqpoint{7.803370in}{1.780490in}}%
\pgfpathlineto{\pgfqpoint{7.804246in}{1.644373in}}%
\pgfpathlineto{\pgfqpoint{7.805121in}{1.570519in}}%
\pgfpathlineto{\pgfqpoint{7.805996in}{1.714301in}}%
\pgfpathlineto{\pgfqpoint{7.806872in}{1.687493in}}%
\pgfpathlineto{\pgfqpoint{7.807747in}{1.989555in}}%
\pgfpathlineto{\pgfqpoint{7.808623in}{2.105450in}}%
\pgfpathlineto{\pgfqpoint{7.809498in}{2.389841in}}%
\pgfpathlineto{\pgfqpoint{7.810373in}{2.497482in}}%
\pgfpathlineto{\pgfqpoint{7.811249in}{2.655143in}}%
\pgfpathlineto{\pgfqpoint{7.812124in}{2.697332in}}%
\pgfpathlineto{\pgfqpoint{7.812999in}{2.722559in}}%
\pgfpathlineto{\pgfqpoint{7.813875in}{2.773337in}}%
\pgfpathlineto{\pgfqpoint{7.814750in}{2.704582in}}%
\pgfpathlineto{\pgfqpoint{7.815626in}{2.701564in}}%
\pgfpathlineto{\pgfqpoint{7.816501in}{2.631816in}}%
\pgfpathlineto{\pgfqpoint{7.817376in}{2.601572in}}%
\pgfpathlineto{\pgfqpoint{7.818252in}{2.488563in}}%
\pgfpathlineto{\pgfqpoint{7.819127in}{2.461830in}}%
\pgfpathlineto{\pgfqpoint{7.820878in}{2.363094in}}%
\pgfpathlineto{\pgfqpoint{7.821753in}{2.224107in}}%
\pgfpathlineto{\pgfqpoint{7.822629in}{2.201150in}}%
\pgfpathlineto{\pgfqpoint{7.823504in}{2.205681in}}%
\pgfpathlineto{\pgfqpoint{7.824379in}{2.247895in}}%
\pgfpathlineto{\pgfqpoint{7.825255in}{2.163883in}}%
\pgfpathlineto{\pgfqpoint{7.826130in}{2.129222in}}%
\pgfpathlineto{\pgfqpoint{7.827005in}{2.163695in}}%
\pgfpathlineto{\pgfqpoint{7.827881in}{2.176646in}}%
\pgfpathlineto{\pgfqpoint{7.828756in}{2.254955in}}%
\pgfpathlineto{\pgfqpoint{7.829632in}{2.427303in}}%
\pgfpathlineto{\pgfqpoint{7.830507in}{2.754819in}}%
\pgfpathlineto{\pgfqpoint{7.831382in}{2.779514in}}%
\pgfpathlineto{\pgfqpoint{7.832258in}{2.878119in}}%
\pgfpathlineto{\pgfqpoint{7.833133in}{2.858773in}}%
\pgfpathlineto{\pgfqpoint{7.834008in}{2.979041in}}%
\pgfpathlineto{\pgfqpoint{7.834884in}{2.990364in}}%
\pgfpathlineto{\pgfqpoint{7.835759in}{3.059299in}}%
\pgfpathlineto{\pgfqpoint{7.836635in}{3.013862in}}%
\pgfpathlineto{\pgfqpoint{7.838385in}{2.802104in}}%
\pgfpathlineto{\pgfqpoint{7.839261in}{2.786925in}}%
\pgfpathlineto{\pgfqpoint{7.840136in}{2.689623in}}%
\pgfpathlineto{\pgfqpoint{7.841011in}{2.832348in}}%
\pgfpathlineto{\pgfqpoint{7.842762in}{2.588923in}}%
\pgfpathlineto{\pgfqpoint{7.843638in}{2.339155in}}%
\pgfpathlineto{\pgfqpoint{7.844513in}{2.244987in}}%
\pgfpathlineto{\pgfqpoint{7.845388in}{2.201528in}}%
\pgfpathlineto{\pgfqpoint{7.846264in}{2.184159in}}%
\pgfpathlineto{\pgfqpoint{7.847139in}{2.197450in}}%
\pgfpathlineto{\pgfqpoint{7.848014in}{2.276477in}}%
\pgfpathlineto{\pgfqpoint{7.848890in}{2.268661in}}%
\pgfpathlineto{\pgfqpoint{7.849765in}{2.365397in}}%
\pgfpathlineto{\pgfqpoint{7.851516in}{2.718334in}}%
\pgfpathlineto{\pgfqpoint{7.852391in}{2.878859in}}%
\pgfpathlineto{\pgfqpoint{7.853267in}{2.983421in}}%
\pgfpathlineto{\pgfqpoint{7.854142in}{2.943003in}}%
\pgfpathlineto{\pgfqpoint{7.855017in}{3.013882in}}%
\pgfpathlineto{\pgfqpoint{7.855893in}{2.918754in}}%
\pgfpathlineto{\pgfqpoint{7.856768in}{3.006536in}}%
\pgfpathlineto{\pgfqpoint{7.857644in}{3.170050in}}%
\pgfpathlineto{\pgfqpoint{7.858519in}{3.034692in}}%
\pgfpathlineto{\pgfqpoint{7.859394in}{2.977640in}}%
\pgfpathlineto{\pgfqpoint{7.860270in}{2.940298in}}%
\pgfpathlineto{\pgfqpoint{7.861145in}{2.928140in}}%
\pgfpathlineto{\pgfqpoint{7.862020in}{2.955401in}}%
\pgfpathlineto{\pgfqpoint{7.862896in}{2.783867in}}%
\pgfpathlineto{\pgfqpoint{7.863771in}{2.744825in}}%
\pgfpathlineto{\pgfqpoint{7.864647in}{2.652319in}}%
\pgfpathlineto{\pgfqpoint{7.865522in}{2.487846in}}%
\pgfpathlineto{\pgfqpoint{7.866397in}{2.494982in}}%
\pgfpathlineto{\pgfqpoint{7.867273in}{2.438081in}}%
\pgfpathlineto{\pgfqpoint{7.868148in}{2.343648in}}%
\pgfpathlineto{\pgfqpoint{7.869023in}{2.324807in}}%
\pgfpathlineto{\pgfqpoint{7.869899in}{2.366605in}}%
\pgfpathlineto{\pgfqpoint{7.870774in}{2.582731in}}%
\pgfpathlineto{\pgfqpoint{7.871650in}{2.554667in}}%
\pgfpathlineto{\pgfqpoint{7.872525in}{2.760338in}}%
\pgfpathlineto{\pgfqpoint{7.873400in}{2.657739in}}%
\pgfpathlineto{\pgfqpoint{7.874276in}{2.609776in}}%
\pgfpathlineto{\pgfqpoint{7.875151in}{2.784919in}}%
\pgfpathlineto{\pgfqpoint{7.876026in}{2.897543in}}%
\pgfpathlineto{\pgfqpoint{7.876902in}{2.933175in}}%
\pgfpathlineto{\pgfqpoint{7.877777in}{2.928509in}}%
\pgfpathlineto{\pgfqpoint{7.878653in}{3.112625in}}%
\pgfpathlineto{\pgfqpoint{7.879528in}{3.006638in}}%
\pgfpathlineto{\pgfqpoint{7.880403in}{2.815999in}}%
\pgfpathlineto{\pgfqpoint{7.881279in}{2.690190in}}%
\pgfpathlineto{\pgfqpoint{7.882154in}{2.638613in}}%
\pgfpathlineto{\pgfqpoint{7.883029in}{2.637065in}}%
\pgfpathlineto{\pgfqpoint{7.883905in}{2.546030in}}%
\pgfpathlineto{\pgfqpoint{7.884780in}{2.598401in}}%
\pgfpathlineto{\pgfqpoint{7.885656in}{2.676975in}}%
\pgfpathlineto{\pgfqpoint{7.887406in}{2.556829in}}%
\pgfpathlineto{\pgfqpoint{7.889157in}{2.368304in}}%
\pgfpathlineto{\pgfqpoint{7.890032in}{2.358752in}}%
\pgfpathlineto{\pgfqpoint{7.891783in}{2.442838in}}%
\pgfpathlineto{\pgfqpoint{7.892659in}{2.627285in}}%
\pgfpathlineto{\pgfqpoint{7.893534in}{2.891401in}}%
\pgfpathlineto{\pgfqpoint{7.894409in}{3.038695in}}%
\pgfpathlineto{\pgfqpoint{7.895285in}{3.069203in}}%
\pgfpathlineto{\pgfqpoint{7.896160in}{3.146002in}}%
\pgfpathlineto{\pgfqpoint{7.897035in}{3.074489in}}%
\pgfpathlineto{\pgfqpoint{7.897911in}{3.311759in}}%
\pgfpathlineto{\pgfqpoint{7.898786in}{3.350801in}}%
\pgfpathlineto{\pgfqpoint{7.899662in}{2.934710in}}%
\pgfpathlineto{\pgfqpoint{7.902288in}{2.556791in}}%
\pgfpathlineto{\pgfqpoint{7.903163in}{2.520846in}}%
\pgfpathlineto{\pgfqpoint{7.904038in}{2.887739in}}%
\pgfpathlineto{\pgfqpoint{7.904914in}{2.813696in}}%
\pgfpathlineto{\pgfqpoint{7.905789in}{2.770123in}}%
\pgfpathlineto{\pgfqpoint{7.906665in}{2.599005in}}%
\pgfpathlineto{\pgfqpoint{7.907540in}{2.533948in}}%
\pgfpathlineto{\pgfqpoint{7.908415in}{2.521186in}}%
\pgfpathlineto{\pgfqpoint{7.909291in}{2.236680in}}%
\pgfpathlineto{\pgfqpoint{7.910166in}{2.147157in}}%
\pgfpathlineto{\pgfqpoint{7.911041in}{2.105925in}}%
\pgfpathlineto{\pgfqpoint{7.911917in}{2.188955in}}%
\pgfpathlineto{\pgfqpoint{7.912792in}{2.368040in}}%
\pgfpathlineto{\pgfqpoint{7.913668in}{2.631325in}}%
\pgfpathlineto{\pgfqpoint{7.914543in}{2.501967in}}%
\pgfpathlineto{\pgfqpoint{7.915418in}{2.604631in}}%
\pgfpathlineto{\pgfqpoint{7.916294in}{2.499286in}}%
\pgfpathlineto{\pgfqpoint{7.917169in}{2.572046in}}%
\pgfpathlineto{\pgfqpoint{7.918044in}{2.684639in}}%
\pgfpathlineto{\pgfqpoint{7.918920in}{2.898802in}}%
\pgfpathlineto{\pgfqpoint{7.919795in}{2.879508in}}%
\pgfpathlineto{\pgfqpoint{7.920671in}{2.952569in}}%
\pgfpathlineto{\pgfqpoint{7.921546in}{2.947132in}}%
\pgfpathlineto{\pgfqpoint{7.922421in}{2.706463in}}%
\pgfpathlineto{\pgfqpoint{7.924172in}{2.481993in}}%
\pgfpathlineto{\pgfqpoint{7.925048in}{2.490300in}}%
\pgfpathlineto{\pgfqpoint{7.929424in}{1.959689in}}%
\pgfpathlineto{\pgfqpoint{7.930300in}{1.924990in}}%
\pgfpathlineto{\pgfqpoint{7.931175in}{1.854194in}}%
\pgfpathlineto{\pgfqpoint{7.932051in}{1.826781in}}%
\pgfpathlineto{\pgfqpoint{7.932926in}{1.872053in}}%
\pgfpathlineto{\pgfqpoint{7.933801in}{1.940772in}}%
\pgfpathlineto{\pgfqpoint{7.935552in}{2.329112in}}%
\pgfpathlineto{\pgfqpoint{7.937303in}{2.376838in}}%
\pgfpathlineto{\pgfqpoint{7.938178in}{2.410971in}}%
\pgfpathlineto{\pgfqpoint{7.941680in}{2.834387in}}%
\pgfpathlineto{\pgfqpoint{7.942555in}{2.817925in}}%
\pgfpathlineto{\pgfqpoint{7.943430in}{2.655113in}}%
\pgfpathlineto{\pgfqpoint{7.944306in}{2.553695in}}%
\pgfpathlineto{\pgfqpoint{7.945181in}{2.527416in}}%
\pgfpathlineto{\pgfqpoint{7.946057in}{2.543614in}}%
\pgfpathlineto{\pgfqpoint{7.946932in}{2.345725in}}%
\pgfpathlineto{\pgfqpoint{7.950433in}{1.861179in}}%
\pgfpathlineto{\pgfqpoint{7.951309in}{1.786834in}}%
\pgfpathlineto{\pgfqpoint{7.952184in}{1.779093in}}%
\pgfpathlineto{\pgfqpoint{7.953060in}{1.751530in}}%
\pgfpathlineto{\pgfqpoint{7.953935in}{1.771844in}}%
\pgfpathlineto{\pgfqpoint{7.957436in}{2.376536in}}%
\pgfpathlineto{\pgfqpoint{7.958312in}{2.430831in}}%
\pgfpathlineto{\pgfqpoint{7.959187in}{2.511180in}}%
\pgfpathlineto{\pgfqpoint{7.960063in}{2.617959in}}%
\pgfpathlineto{\pgfqpoint{7.960938in}{2.668819in}}%
\pgfpathlineto{\pgfqpoint{7.962689in}{2.811619in}}%
\pgfpathlineto{\pgfqpoint{7.963564in}{2.876034in}}%
\pgfpathlineto{\pgfqpoint{7.964439in}{2.698799in}}%
\pgfpathlineto{\pgfqpoint{7.965315in}{2.646995in}}%
\pgfpathlineto{\pgfqpoint{7.966190in}{2.523678in}}%
\pgfpathlineto{\pgfqpoint{7.967066in}{2.471308in}}%
\pgfpathlineto{\pgfqpoint{7.968816in}{2.275458in}}%
\pgfpathlineto{\pgfqpoint{7.969692in}{2.117932in}}%
\pgfpathlineto{\pgfqpoint{7.970567in}{2.056991in}}%
\pgfpathlineto{\pgfqpoint{7.972318in}{1.851551in}}%
\pgfpathlineto{\pgfqpoint{7.973193in}{1.817191in}}%
\pgfpathlineto{\pgfqpoint{7.974069in}{1.837693in}}%
\pgfpathlineto{\pgfqpoint{7.974944in}{1.888138in}}%
\pgfpathlineto{\pgfqpoint{7.975819in}{2.000014in}}%
\pgfpathlineto{\pgfqpoint{7.977570in}{2.381369in}}%
\pgfpathlineto{\pgfqpoint{7.978445in}{2.513030in}}%
\pgfpathlineto{\pgfqpoint{7.979321in}{2.572763in}}%
\pgfpathlineto{\pgfqpoint{7.980196in}{2.588969in}}%
\pgfpathlineto{\pgfqpoint{7.981072in}{2.633486in}}%
\pgfpathlineto{\pgfqpoint{7.981947in}{2.803400in}}%
\pgfpathlineto{\pgfqpoint{7.982822in}{2.770236in}}%
\pgfpathlineto{\pgfqpoint{7.983698in}{2.819402in}}%
\pgfpathlineto{\pgfqpoint{7.984573in}{2.808938in}}%
\pgfpathlineto{\pgfqpoint{7.985448in}{2.627587in}}%
\pgfpathlineto{\pgfqpoint{7.986324in}{2.574424in}}%
\pgfpathlineto{\pgfqpoint{7.987199in}{2.474857in}}%
\pgfpathlineto{\pgfqpoint{7.988075in}{2.427773in}}%
\pgfpathlineto{\pgfqpoint{7.989825in}{2.208664in}}%
\pgfpathlineto{\pgfqpoint{7.990701in}{1.972678in}}%
\pgfpathlineto{\pgfqpoint{7.993327in}{1.735370in}}%
\pgfpathlineto{\pgfqpoint{7.994202in}{1.897577in}}%
\pgfpathlineto{\pgfqpoint{7.995078in}{1.789401in}}%
\pgfpathlineto{\pgfqpoint{7.996828in}{2.035847in}}%
\pgfpathlineto{\pgfqpoint{7.997704in}{2.092386in}}%
\pgfpathlineto{\pgfqpoint{7.998579in}{2.504931in}}%
\pgfpathlineto{\pgfqpoint{7.999454in}{2.660574in}}%
\pgfpathlineto{\pgfqpoint{8.001205in}{2.820629in}}%
\pgfpathlineto{\pgfqpoint{8.002081in}{2.952222in}}%
\pgfpathlineto{\pgfqpoint{8.002956in}{2.637196in}}%
\pgfpathlineto{\pgfqpoint{8.003831in}{2.682501in}}%
\pgfpathlineto{\pgfqpoint{8.004707in}{2.637776in}}%
\pgfpathlineto{\pgfqpoint{8.005582in}{2.709748in}}%
\pgfpathlineto{\pgfqpoint{8.006457in}{2.473762in}}%
\pgfpathlineto{\pgfqpoint{8.007333in}{2.343271in}}%
\pgfpathlineto{\pgfqpoint{8.008208in}{2.399870in}}%
\pgfpathlineto{\pgfqpoint{8.009084in}{2.346820in}}%
\pgfpathlineto{\pgfqpoint{8.013460in}{1.644600in}}%
\pgfpathlineto{\pgfqpoint{8.014336in}{1.610316in}}%
\pgfpathlineto{\pgfqpoint{8.016087in}{1.510635in}}%
\pgfpathlineto{\pgfqpoint{8.016962in}{1.480315in}}%
\pgfpathlineto{\pgfqpoint{8.017837in}{1.674164in}}%
\pgfpathlineto{\pgfqpoint{8.019588in}{2.200667in}}%
\pgfpathlineto{\pgfqpoint{8.020463in}{2.267054in}}%
\pgfpathlineto{\pgfqpoint{8.021339in}{2.410131in}}%
\pgfpathlineto{\pgfqpoint{8.022214in}{2.348068in}}%
\pgfpathlineto{\pgfqpoint{8.023090in}{2.431649in}}%
\pgfpathlineto{\pgfqpoint{8.023965in}{2.635960in}}%
\pgfpathlineto{\pgfqpoint{8.024840in}{2.647491in}}%
\pgfpathlineto{\pgfqpoint{8.025716in}{2.605197in}}%
\pgfpathlineto{\pgfqpoint{8.026591in}{2.593265in}}%
\pgfpathlineto{\pgfqpoint{8.027466in}{2.596022in}}%
\pgfpathlineto{\pgfqpoint{8.029217in}{2.364529in}}%
\pgfpathlineto{\pgfqpoint{8.030093in}{2.272852in}}%
\pgfpathlineto{\pgfqpoint{8.030968in}{2.237624in}}%
\pgfpathlineto{\pgfqpoint{8.033594in}{1.815076in}}%
\pgfpathlineto{\pgfqpoint{8.035345in}{1.571048in}}%
\pgfpathlineto{\pgfqpoint{8.037096in}{1.501535in}}%
\pgfpathlineto{\pgfqpoint{8.037971in}{1.618358in}}%
\pgfpathlineto{\pgfqpoint{8.039722in}{2.186465in}}%
\pgfpathlineto{\pgfqpoint{8.041472in}{2.829076in}}%
\pgfpathlineto{\pgfqpoint{8.042348in}{2.861975in}}%
\pgfpathlineto{\pgfqpoint{8.043223in}{2.951585in}}%
\pgfpathlineto{\pgfqpoint{8.044099in}{2.901337in}}%
\pgfpathlineto{\pgfqpoint{8.044974in}{2.947403in}}%
\pgfpathlineto{\pgfqpoint{8.045849in}{2.938437in}}%
\pgfpathlineto{\pgfqpoint{8.046725in}{2.820086in}}%
\pgfpathlineto{\pgfqpoint{8.050226in}{2.121632in}}%
\pgfpathlineto{\pgfqpoint{8.051102in}{1.972225in}}%
\pgfpathlineto{\pgfqpoint{8.051977in}{1.952893in}}%
\pgfpathlineto{\pgfqpoint{8.054603in}{1.492096in}}%
\pgfpathlineto{\pgfqpoint{8.055479in}{1.444748in}}%
\pgfpathlineto{\pgfqpoint{8.056354in}{1.378558in}}%
\pgfpathlineto{\pgfqpoint{8.058105in}{1.292508in}}%
\pgfpathlineto{\pgfqpoint{8.058980in}{1.367231in}}%
\pgfpathlineto{\pgfqpoint{8.061606in}{2.114050in}}%
\pgfpathlineto{\pgfqpoint{8.062482in}{2.183368in}}%
\pgfpathlineto{\pgfqpoint{8.063357in}{2.293151in}}%
\pgfpathlineto{\pgfqpoint{8.064232in}{2.290775in}}%
\pgfpathlineto{\pgfqpoint{8.065108in}{2.313000in}}%
\pgfpathlineto{\pgfqpoint{8.065983in}{2.321447in}}%
\pgfpathlineto{\pgfqpoint{8.066858in}{2.387540in}}%
\pgfpathlineto{\pgfqpoint{8.067734in}{2.421092in}}%
\pgfpathlineto{\pgfqpoint{8.068609in}{2.328583in}}%
\pgfpathlineto{\pgfqpoint{8.069485in}{2.095202in}}%
\pgfpathlineto{\pgfqpoint{8.070360in}{1.984269in}}%
\pgfpathlineto{\pgfqpoint{8.071235in}{1.955158in}}%
\pgfpathlineto{\pgfqpoint{8.072111in}{1.959198in}}%
\pgfpathlineto{\pgfqpoint{8.072986in}{1.785701in}}%
\pgfpathlineto{\pgfqpoint{8.073861in}{1.667594in}}%
\pgfpathlineto{\pgfqpoint{8.075612in}{1.366740in}}%
\pgfpathlineto{\pgfqpoint{8.076488in}{1.324942in}}%
\pgfpathlineto{\pgfqpoint{8.077363in}{1.249200in}}%
\pgfpathlineto{\pgfqpoint{8.078238in}{1.147631in}}%
\pgfpathlineto{\pgfqpoint{8.079114in}{1.202569in}}%
\pgfpathlineto{\pgfqpoint{8.079989in}{1.283031in}}%
\pgfpathlineto{\pgfqpoint{8.082615in}{1.812989in}}%
\pgfpathlineto{\pgfqpoint{8.083491in}{1.840330in}}%
\pgfpathlineto{\pgfqpoint{8.085241in}{2.002768in}}%
\pgfpathlineto{\pgfqpoint{8.086117in}{2.034842in}}%
\pgfpathlineto{\pgfqpoint{8.086992in}{2.289754in}}%
\pgfpathlineto{\pgfqpoint{8.087867in}{2.354821in}}%
\pgfpathlineto{\pgfqpoint{8.088743in}{2.318683in}}%
\pgfpathlineto{\pgfqpoint{8.089618in}{2.260355in}}%
\pgfpathlineto{\pgfqpoint{8.091369in}{1.948248in}}%
\pgfpathlineto{\pgfqpoint{8.093995in}{1.752096in}}%
\pgfpathlineto{\pgfqpoint{8.094870in}{1.649282in}}%
\pgfpathlineto{\pgfqpoint{8.095746in}{1.500742in}}%
\pgfpathlineto{\pgfqpoint{8.098372in}{1.266833in}}%
\pgfpathlineto{\pgfqpoint{8.099247in}{1.214651in}}%
\pgfpathlineto{\pgfqpoint{8.100123in}{1.220957in}}%
\pgfpathlineto{\pgfqpoint{8.100998in}{1.282238in}}%
\pgfpathlineto{\pgfqpoint{8.102749in}{1.831069in}}%
\pgfpathlineto{\pgfqpoint{8.104500in}{2.173719in}}%
\pgfpathlineto{\pgfqpoint{8.105375in}{2.252081in}}%
\pgfpathlineto{\pgfqpoint{8.106250in}{2.260781in}}%
\pgfpathlineto{\pgfqpoint{8.107126in}{2.258157in}}%
\pgfpathlineto{\pgfqpoint{8.108001in}{2.393751in}}%
\pgfpathlineto{\pgfqpoint{8.109752in}{2.434723in}}%
\pgfpathlineto{\pgfqpoint{8.110627in}{2.368682in}}%
\pgfpathlineto{\pgfqpoint{8.111503in}{2.161769in}}%
\pgfpathlineto{\pgfqpoint{8.115004in}{1.714867in}}%
\pgfpathlineto{\pgfqpoint{8.115879in}{1.560513in}}%
\pgfpathlineto{\pgfqpoint{8.117630in}{1.358962in}}%
\pgfpathlineto{\pgfqpoint{8.118506in}{1.322714in}}%
\pgfpathlineto{\pgfqpoint{8.119381in}{1.253655in}}%
\pgfpathlineto{\pgfqpoint{8.120256in}{1.232360in}}%
\pgfpathlineto{\pgfqpoint{8.121132in}{1.272308in}}%
\pgfpathlineto{\pgfqpoint{8.122007in}{1.413031in}}%
\pgfpathlineto{\pgfqpoint{8.124633in}{2.128809in}}%
\pgfpathlineto{\pgfqpoint{8.125509in}{2.196227in}}%
\pgfpathlineto{\pgfqpoint{8.126384in}{2.211843in}}%
\pgfpathlineto{\pgfqpoint{8.127259in}{2.209364in}}%
\pgfpathlineto{\pgfqpoint{8.128135in}{2.224018in}}%
\pgfpathlineto{\pgfqpoint{8.129885in}{2.317230in}}%
\pgfpathlineto{\pgfqpoint{8.130761in}{2.286251in}}%
\pgfpathlineto{\pgfqpoint{8.131636in}{2.244874in}}%
\pgfpathlineto{\pgfqpoint{8.132512in}{2.017081in}}%
\pgfpathlineto{\pgfqpoint{8.134262in}{1.778338in}}%
\pgfpathlineto{\pgfqpoint{8.135138in}{1.734539in}}%
\pgfpathlineto{\pgfqpoint{8.136013in}{1.676316in}}%
\pgfpathlineto{\pgfqpoint{8.136888in}{1.582904in}}%
\pgfpathlineto{\pgfqpoint{8.137764in}{1.431985in}}%
\pgfpathlineto{\pgfqpoint{8.138639in}{1.381956in}}%
\pgfpathlineto{\pgfqpoint{8.139515in}{1.298021in}}%
\pgfpathlineto{\pgfqpoint{8.140390in}{1.287260in}}%
\pgfpathlineto{\pgfqpoint{8.141265in}{1.271137in}}%
\pgfpathlineto{\pgfqpoint{8.142141in}{1.278424in}}%
\pgfpathlineto{\pgfqpoint{8.143016in}{1.371271in}}%
\pgfpathlineto{\pgfqpoint{8.143891in}{1.510295in}}%
\pgfpathlineto{\pgfqpoint{8.144767in}{1.696911in}}%
\pgfpathlineto{\pgfqpoint{8.145642in}{1.800969in}}%
\pgfpathlineto{\pgfqpoint{8.146518in}{1.718070in}}%
\pgfpathlineto{\pgfqpoint{8.147393in}{1.827424in}}%
\pgfpathlineto{\pgfqpoint{8.148268in}{1.817742in}}%
\pgfpathlineto{\pgfqpoint{8.149144in}{1.859593in}}%
\pgfpathlineto{\pgfqpoint{8.150019in}{1.966242in}}%
\pgfpathlineto{\pgfqpoint{8.150894in}{1.999903in}}%
\pgfpathlineto{\pgfqpoint{8.152645in}{1.913851in}}%
\pgfpathlineto{\pgfqpoint{8.153521in}{1.839808in}}%
\pgfpathlineto{\pgfqpoint{8.155271in}{1.551262in}}%
\pgfpathlineto{\pgfqpoint{8.156147in}{1.624135in}}%
\pgfpathlineto{\pgfqpoint{8.157022in}{1.496816in}}%
\pgfpathlineto{\pgfqpoint{8.157897in}{1.461399in}}%
\pgfpathlineto{\pgfqpoint{8.158773in}{1.306780in}}%
\pgfpathlineto{\pgfqpoint{8.160524in}{1.149670in}}%
\pgfpathlineto{\pgfqpoint{8.161399in}{1.164056in}}%
\pgfpathlineto{\pgfqpoint{8.162274in}{1.130867in}}%
\pgfpathlineto{\pgfqpoint{8.163150in}{1.156164in}}%
\pgfpathlineto{\pgfqpoint{8.164025in}{1.235758in}}%
\pgfpathlineto{\pgfqpoint{8.164900in}{1.386525in}}%
\pgfpathlineto{\pgfqpoint{8.166651in}{1.740753in}}%
\pgfpathlineto{\pgfqpoint{8.167527in}{1.801278in}}%
\pgfpathlineto{\pgfqpoint{8.168402in}{1.971537in}}%
\pgfpathlineto{\pgfqpoint{8.169277in}{1.959511in}}%
\pgfpathlineto{\pgfqpoint{8.170153in}{1.984646in}}%
\pgfpathlineto{\pgfqpoint{8.171903in}{2.205456in}}%
\pgfpathlineto{\pgfqpoint{8.172779in}{2.136862in}}%
\pgfpathlineto{\pgfqpoint{8.173654in}{1.972791in}}%
\pgfpathlineto{\pgfqpoint{8.175405in}{1.388337in}}%
\pgfpathlineto{\pgfqpoint{8.176280in}{1.150803in}}%
\pgfpathlineto{\pgfqpoint{8.177156in}{1.203928in}}%
\pgfpathlineto{\pgfqpoint{8.179782in}{0.876794in}}%
\pgfpathlineto{\pgfqpoint{8.182408in}{0.746756in}}%
\pgfpathlineto{\pgfqpoint{8.184159in}{1.158732in}}%
\pgfpathlineto{\pgfqpoint{8.185034in}{1.178404in}}%
\pgfpathlineto{\pgfqpoint{8.187660in}{2.080224in}}%
\pgfpathlineto{\pgfqpoint{8.188536in}{2.107525in}}%
\pgfpathlineto{\pgfqpoint{8.189411in}{2.124130in}}%
\pgfpathlineto{\pgfqpoint{8.190286in}{2.171529in}}%
\pgfpathlineto{\pgfqpoint{8.191162in}{2.195731in}}%
\pgfpathlineto{\pgfqpoint{8.192037in}{2.247867in}}%
\pgfpathlineto{\pgfqpoint{8.192913in}{2.254634in}}%
\pgfpathlineto{\pgfqpoint{8.193788in}{2.209660in}}%
\pgfpathlineto{\pgfqpoint{8.194663in}{2.087952in}}%
\pgfpathlineto{\pgfqpoint{8.195539in}{1.790458in}}%
\pgfpathlineto{\pgfqpoint{8.197289in}{1.509162in}}%
\pgfpathlineto{\pgfqpoint{8.198165in}{1.384901in}}%
\pgfpathlineto{\pgfqpoint{8.200791in}{1.157901in}}%
\pgfpathlineto{\pgfqpoint{8.202542in}{1.041381in}}%
\pgfpathlineto{\pgfqpoint{8.203417in}{1.017367in}}%
\pgfpathlineto{\pgfqpoint{8.205168in}{1.106022in}}%
\pgfpathlineto{\pgfqpoint{8.206043in}{1.138947in}}%
\pgfpathlineto{\pgfqpoint{8.207794in}{1.522039in}}%
\pgfpathlineto{\pgfqpoint{8.208669in}{1.591297in}}%
\pgfpathlineto{\pgfqpoint{8.209545in}{1.730928in}}%
\pgfpathlineto{\pgfqpoint{8.210420in}{1.699511in}}%
\pgfpathlineto{\pgfqpoint{8.211295in}{1.606213in}}%
\pgfpathlineto{\pgfqpoint{8.212171in}{1.717672in}}%
\pgfpathlineto{\pgfqpoint{8.213046in}{1.792112in}}%
\pgfpathlineto{\pgfqpoint{8.213922in}{1.907210in}}%
\pgfpathlineto{\pgfqpoint{8.214797in}{1.821607in}}%
\pgfpathlineto{\pgfqpoint{8.215672in}{1.768181in}}%
\pgfpathlineto{\pgfqpoint{8.216548in}{1.580865in}}%
\pgfpathlineto{\pgfqpoint{8.218298in}{1.410350in}}%
\pgfpathlineto{\pgfqpoint{8.219174in}{1.448184in}}%
\pgfpathlineto{\pgfqpoint{8.220049in}{1.377312in}}%
\pgfpathlineto{\pgfqpoint{8.220925in}{1.358849in}}%
\pgfpathlineto{\pgfqpoint{8.221800in}{1.270080in}}%
\pgfpathlineto{\pgfqpoint{8.223551in}{1.201889in}}%
\pgfpathlineto{\pgfqpoint{8.224426in}{1.202418in}}%
\pgfpathlineto{\pgfqpoint{8.225301in}{1.189656in}}%
\pgfpathlineto{\pgfqpoint{8.226177in}{1.210271in}}%
\pgfpathlineto{\pgfqpoint{8.227928in}{1.354016in}}%
\pgfpathlineto{\pgfqpoint{8.228803in}{1.569943in}}%
\pgfpathlineto{\pgfqpoint{8.229678in}{1.656005in}}%
\pgfpathlineto{\pgfqpoint{8.230554in}{1.701036in}}%
\pgfpathlineto{\pgfqpoint{8.231429in}{1.701596in}}%
\pgfpathlineto{\pgfqpoint{8.232304in}{1.597385in}}%
\pgfpathlineto{\pgfqpoint{8.233180in}{1.576400in}}%
\pgfpathlineto{\pgfqpoint{8.234055in}{1.515978in}}%
\pgfpathlineto{\pgfqpoint{8.234931in}{1.520404in}}%
\pgfpathlineto{\pgfqpoint{8.235806in}{1.478029in}}%
\pgfpathlineto{\pgfqpoint{8.236681in}{1.604275in}}%
\pgfpathlineto{\pgfqpoint{8.238432in}{1.406990in}}%
\pgfpathlineto{\pgfqpoint{8.239307in}{1.350202in}}%
\pgfpathlineto{\pgfqpoint{8.241058in}{1.321128in}}%
\pgfpathlineto{\pgfqpoint{8.241934in}{1.218880in}}%
\pgfpathlineto{\pgfqpoint{8.244560in}{1.059726in}}%
\pgfpathlineto{\pgfqpoint{8.246310in}{1.016234in}}%
\pgfpathlineto{\pgfqpoint{8.248061in}{1.118822in}}%
\pgfpathlineto{\pgfqpoint{8.248937in}{1.221297in}}%
\pgfpathlineto{\pgfqpoint{8.249812in}{1.230850in}}%
\pgfpathlineto{\pgfqpoint{8.250687in}{1.176208in}}%
\pgfpathlineto{\pgfqpoint{8.251563in}{1.163316in}}%
\pgfpathlineto{\pgfqpoint{8.252438in}{1.156664in}}%
\pgfpathlineto{\pgfqpoint{8.253313in}{1.111279in}}%
\pgfpathlineto{\pgfqpoint{8.254189in}{1.177298in}}%
\pgfpathlineto{\pgfqpoint{8.256815in}{1.296862in}}%
\pgfpathlineto{\pgfqpoint{8.257690in}{1.428889in}}%
\pgfpathlineto{\pgfqpoint{8.258566in}{1.405215in}}%
\pgfpathlineto{\pgfqpoint{8.260316in}{1.199624in}}%
\pgfpathlineto{\pgfqpoint{8.262067in}{1.136002in}}%
\pgfpathlineto{\pgfqpoint{8.263818in}{0.950195in}}%
\pgfpathlineto{\pgfqpoint{8.264693in}{0.907454in}}%
\pgfpathlineto{\pgfqpoint{8.266444in}{0.961749in}}%
\pgfpathlineto{\pgfqpoint{8.268195in}{0.946457in}}%
\pgfpathlineto{\pgfqpoint{8.269070in}{1.005473in}}%
\pgfpathlineto{\pgfqpoint{8.270821in}{1.207204in}}%
\pgfpathlineto{\pgfqpoint{8.271696in}{1.168570in}}%
\pgfpathlineto{\pgfqpoint{8.272572in}{1.264327in}}%
\pgfpathlineto{\pgfqpoint{8.273447in}{1.137746in}}%
\pgfpathlineto{\pgfqpoint{8.274322in}{1.136466in}}%
\pgfpathlineto{\pgfqpoint{8.275198in}{1.184134in}}%
\pgfpathlineto{\pgfqpoint{8.276073in}{1.184611in}}%
\pgfpathlineto{\pgfqpoint{8.276949in}{1.183935in}}%
\pgfpathlineto{\pgfqpoint{8.278699in}{1.336609in}}%
\pgfpathlineto{\pgfqpoint{8.279575in}{1.284994in}}%
\pgfpathlineto{\pgfqpoint{8.281325in}{1.154730in}}%
\pgfpathlineto{\pgfqpoint{8.282201in}{1.145743in}}%
\pgfpathlineto{\pgfqpoint{8.284827in}{0.984140in}}%
\pgfpathlineto{\pgfqpoint{8.285702in}{1.011854in}}%
\pgfpathlineto{\pgfqpoint{8.286578in}{1.015307in}}%
\pgfpathlineto{\pgfqpoint{8.287453in}{0.952661in}}%
\pgfpathlineto{\pgfqpoint{8.288328in}{0.915415in}}%
\pgfpathlineto{\pgfqpoint{8.289204in}{0.949717in}}%
\pgfpathlineto{\pgfqpoint{8.290079in}{0.999369in}}%
\pgfpathlineto{\pgfqpoint{8.290955in}{1.119728in}}%
\pgfpathlineto{\pgfqpoint{8.291830in}{1.141203in}}%
\pgfpathlineto{\pgfqpoint{8.292705in}{1.054128in}}%
\pgfpathlineto{\pgfqpoint{8.294456in}{1.202324in}}%
\pgfpathlineto{\pgfqpoint{8.295331in}{1.257946in}}%
\pgfpathlineto{\pgfqpoint{8.296207in}{1.255954in}}%
\pgfpathlineto{\pgfqpoint{8.297082in}{1.282685in}}%
\pgfpathlineto{\pgfqpoint{8.297958in}{1.359975in}}%
\pgfpathlineto{\pgfqpoint{8.298833in}{1.332645in}}%
\pgfpathlineto{\pgfqpoint{8.299708in}{1.418015in}}%
\pgfpathlineto{\pgfqpoint{8.300584in}{1.357603in}}%
\pgfpathlineto{\pgfqpoint{8.301459in}{1.361001in}}%
\pgfpathlineto{\pgfqpoint{8.303210in}{1.284277in}}%
\pgfpathlineto{\pgfqpoint{8.304085in}{1.211895in}}%
\pgfpathlineto{\pgfqpoint{8.304961in}{1.185918in}}%
\pgfpathlineto{\pgfqpoint{8.305836in}{1.101227in}}%
\pgfpathlineto{\pgfqpoint{8.306711in}{1.053690in}}%
\pgfpathlineto{\pgfqpoint{8.307587in}{1.082943in}}%
\pgfpathlineto{\pgfqpoint{8.308462in}{1.086728in}}%
\pgfpathlineto{\pgfqpoint{8.309337in}{1.096507in}}%
\pgfpathlineto{\pgfqpoint{8.310213in}{1.034773in}}%
\pgfpathlineto{\pgfqpoint{8.312839in}{1.341498in}}%
\pgfpathlineto{\pgfqpoint{8.313714in}{1.346027in}}%
\pgfpathlineto{\pgfqpoint{8.315465in}{1.310953in}}%
\pgfpathlineto{\pgfqpoint{8.316340in}{1.354670in}}%
\pgfpathlineto{\pgfqpoint{8.317216in}{1.309474in}}%
\pgfpathlineto{\pgfqpoint{8.318091in}{1.222118in}}%
\pgfpathlineto{\pgfqpoint{8.318967in}{1.260306in}}%
\pgfpathlineto{\pgfqpoint{8.319842in}{1.157318in}}%
\pgfpathlineto{\pgfqpoint{8.320717in}{1.132188in}}%
\pgfpathlineto{\pgfqpoint{8.321593in}{1.262717in}}%
\pgfpathlineto{\pgfqpoint{8.322468in}{1.266644in}}%
\pgfpathlineto{\pgfqpoint{8.323344in}{1.252220in}}%
\pgfpathlineto{\pgfqpoint{8.324219in}{1.206798in}}%
\pgfpathlineto{\pgfqpoint{8.325094in}{1.179272in}}%
\pgfpathlineto{\pgfqpoint{8.325970in}{1.126147in}}%
\pgfpathlineto{\pgfqpoint{8.326845in}{1.172551in}}%
\pgfpathlineto{\pgfqpoint{8.327720in}{1.187806in}}%
\pgfpathlineto{\pgfqpoint{8.328596in}{1.160054in}}%
\pgfpathlineto{\pgfqpoint{8.329471in}{1.185993in}}%
\pgfpathlineto{\pgfqpoint{8.330347in}{1.126494in}}%
\pgfpathlineto{\pgfqpoint{8.331222in}{1.102473in}}%
\pgfpathlineto{\pgfqpoint{8.332097in}{1.128714in}}%
\pgfpathlineto{\pgfqpoint{8.332973in}{1.318674in}}%
\pgfpathlineto{\pgfqpoint{8.334723in}{1.470285in}}%
\pgfpathlineto{\pgfqpoint{8.335599in}{1.475158in}}%
\pgfpathlineto{\pgfqpoint{8.336474in}{1.501520in}}%
\pgfpathlineto{\pgfqpoint{8.337350in}{1.551236in}}%
\pgfpathlineto{\pgfqpoint{8.338225in}{1.504933in}}%
\pgfpathlineto{\pgfqpoint{8.339100in}{1.499808in}}%
\pgfpathlineto{\pgfqpoint{8.339976in}{1.523669in}}%
\pgfpathlineto{\pgfqpoint{8.340851in}{1.497138in}}%
\pgfpathlineto{\pgfqpoint{8.341726in}{1.530307in}}%
\pgfpathlineto{\pgfqpoint{8.344353in}{1.267210in}}%
\pgfpathlineto{\pgfqpoint{8.345228in}{1.277141in}}%
\pgfpathlineto{\pgfqpoint{8.346103in}{1.262188in}}%
\pgfpathlineto{\pgfqpoint{8.346979in}{1.231756in}}%
\pgfpathlineto{\pgfqpoint{8.347854in}{1.161941in}}%
\pgfpathlineto{\pgfqpoint{8.348729in}{1.171909in}}%
\pgfpathlineto{\pgfqpoint{8.349605in}{1.127808in}}%
\pgfpathlineto{\pgfqpoint{8.350480in}{1.137437in}}%
\pgfpathlineto{\pgfqpoint{8.351356in}{1.068906in}}%
\pgfpathlineto{\pgfqpoint{8.352231in}{1.029676in}}%
\pgfpathlineto{\pgfqpoint{8.353106in}{1.044639in}}%
\pgfpathlineto{\pgfqpoint{8.353982in}{1.170399in}}%
\pgfpathlineto{\pgfqpoint{8.354857in}{1.250432in}}%
\pgfpathlineto{\pgfqpoint{8.355732in}{1.242246in}}%
\pgfpathlineto{\pgfqpoint{8.356608in}{1.244824in}}%
\pgfpathlineto{\pgfqpoint{8.357483in}{1.259565in}}%
\pgfpathlineto{\pgfqpoint{8.358359in}{1.257112in}}%
\pgfpathlineto{\pgfqpoint{8.359234in}{1.302081in}}%
\pgfpathlineto{\pgfqpoint{8.360109in}{1.569119in}}%
\pgfpathlineto{\pgfqpoint{8.360985in}{1.417817in}}%
\pgfpathlineto{\pgfqpoint{8.361860in}{1.493793in}}%
\pgfpathlineto{\pgfqpoint{8.362735in}{1.502895in}}%
\pgfpathlineto{\pgfqpoint{8.363611in}{1.419110in}}%
\pgfpathlineto{\pgfqpoint{8.364486in}{1.274875in}}%
\pgfpathlineto{\pgfqpoint{8.365362in}{1.221712in}}%
\pgfpathlineto{\pgfqpoint{8.366237in}{1.150161in}}%
\pgfpathlineto{\pgfqpoint{8.367112in}{1.128752in}}%
\pgfpathlineto{\pgfqpoint{8.367988in}{1.114820in}}%
\pgfpathlineto{\pgfqpoint{8.368863in}{1.115008in}}%
\pgfpathlineto{\pgfqpoint{8.369738in}{1.065772in}}%
\pgfpathlineto{\pgfqpoint{8.370614in}{0.981383in}}%
\pgfpathlineto{\pgfqpoint{8.371489in}{0.981383in}}%
\pgfpathlineto{\pgfqpoint{8.372365in}{1.006228in}}%
\pgfpathlineto{\pgfqpoint{8.374115in}{1.079290in}}%
\pgfpathlineto{\pgfqpoint{8.374991in}{1.173420in}}%
\pgfpathlineto{\pgfqpoint{8.375866in}{1.235334in}}%
\pgfpathlineto{\pgfqpoint{8.376741in}{1.266615in}}%
\pgfpathlineto{\pgfqpoint{8.377617in}{1.229582in}}%
\pgfpathlineto{\pgfqpoint{8.378492in}{1.277424in}}%
\pgfpathlineto{\pgfqpoint{8.379368in}{1.215108in}}%
\pgfpathlineto{\pgfqpoint{8.381994in}{1.463947in}}%
\pgfpathlineto{\pgfqpoint{8.383744in}{1.533176in}}%
\pgfpathlineto{\pgfqpoint{8.385495in}{1.345067in}}%
\pgfpathlineto{\pgfqpoint{8.387246in}{1.202795in}}%
\pgfpathlineto{\pgfqpoint{8.388121in}{1.165944in}}%
\pgfpathlineto{\pgfqpoint{8.388997in}{1.102322in}}%
\pgfpathlineto{\pgfqpoint{8.389872in}{0.997128in}}%
\pgfpathlineto{\pgfqpoint{8.390747in}{1.004718in}}%
\pgfpathlineto{\pgfqpoint{8.391623in}{0.997166in}}%
\pgfpathlineto{\pgfqpoint{8.392498in}{1.019141in}}%
\pgfpathlineto{\pgfqpoint{8.393374in}{0.950611in}}%
\pgfpathlineto{\pgfqpoint{8.394249in}{0.953367in}}%
\pgfpathlineto{\pgfqpoint{8.395124in}{0.976777in}}%
\pgfpathlineto{\pgfqpoint{8.396875in}{1.238648in}}%
\pgfpathlineto{\pgfqpoint{8.397750in}{1.237114in}}%
\pgfpathlineto{\pgfqpoint{8.398626in}{1.232433in}}%
\pgfpathlineto{\pgfqpoint{8.399501in}{1.358153in}}%
\pgfpathlineto{\pgfqpoint{8.400377in}{1.255675in}}%
\pgfpathlineto{\pgfqpoint{8.401252in}{1.348313in}}%
\pgfpathlineto{\pgfqpoint{8.402127in}{1.368120in}}%
\pgfpathlineto{\pgfqpoint{8.403003in}{1.342558in}}%
\pgfpathlineto{\pgfqpoint{8.403878in}{1.336031in}}%
\pgfpathlineto{\pgfqpoint{8.404753in}{1.466345in}}%
\pgfpathlineto{\pgfqpoint{8.405629in}{1.425038in}}%
\pgfpathlineto{\pgfqpoint{8.406504in}{1.322714in}}%
\pgfpathlineto{\pgfqpoint{8.407380in}{1.270231in}}%
\pgfpathlineto{\pgfqpoint{8.408255in}{1.104361in}}%
\pgfpathlineto{\pgfqpoint{8.409130in}{1.149255in}}%
\pgfpathlineto{\pgfqpoint{8.410006in}{1.058636in}}%
\pgfpathlineto{\pgfqpoint{8.410881in}{1.220164in}}%
\pgfpathlineto{\pgfqpoint{8.411756in}{1.159751in}}%
\pgfpathlineto{\pgfqpoint{8.411756in}{1.159751in}}%
\pgfusepath{stroke}%
\end{pgfscope}%
\begin{pgfscope}%
\pgfsetrectcap%
\pgfsetmiterjoin%
\pgfsetlinewidth{0.803000pt}%
\definecolor{currentstroke}{rgb}{0.000000,0.000000,0.000000}%
\pgfsetstrokecolor{currentstroke}%
\pgfsetdash{}{0pt}%
\pgfpathmoveto{\pgfqpoint{0.742589in}{0.670138in}}%
\pgfpathlineto{\pgfqpoint{0.742589in}{5.516628in}}%
\pgfusepath{stroke}%
\end{pgfscope}%
\begin{pgfscope}%
\pgfsetrectcap%
\pgfsetmiterjoin%
\pgfsetlinewidth{0.803000pt}%
\definecolor{currentstroke}{rgb}{0.000000,0.000000,0.000000}%
\pgfsetstrokecolor{currentstroke}%
\pgfsetdash{}{0pt}%
\pgfpathmoveto{\pgfqpoint{8.410881in}{0.670138in}}%
\pgfpathlineto{\pgfqpoint{8.410881in}{5.516628in}}%
\pgfusepath{stroke}%
\end{pgfscope}%
\begin{pgfscope}%
\pgfsetrectcap%
\pgfsetmiterjoin%
\pgfsetlinewidth{0.803000pt}%
\definecolor{currentstroke}{rgb}{0.000000,0.000000,0.000000}%
\pgfsetstrokecolor{currentstroke}%
\pgfsetdash{}{0pt}%
\pgfpathmoveto{\pgfqpoint{0.742589in}{0.670138in}}%
\pgfpathlineto{\pgfqpoint{8.410881in}{0.670138in}}%
\pgfusepath{stroke}%
\end{pgfscope}%
\begin{pgfscope}%
\pgfsetrectcap%
\pgfsetmiterjoin%
\pgfsetlinewidth{0.803000pt}%
\definecolor{currentstroke}{rgb}{0.000000,0.000000,0.000000}%
\pgfsetstrokecolor{currentstroke}%
\pgfsetdash{}{0pt}%
\pgfpathmoveto{\pgfqpoint{0.742589in}{5.516628in}}%
\pgfpathlineto{\pgfqpoint{8.410881in}{5.516628in}}%
\pgfusepath{stroke}%
\end{pgfscope}%
\begin{pgfscope}%
\definecolor{textcolor}{rgb}{0.000000,0.000000,0.000000}%
\pgfsetstrokecolor{textcolor}%
\pgfsetfillcolor{textcolor}%
\pgftext[x=4.576735in,y=5.599962in,,base]{\color{textcolor}{\rmfamily\fontsize{20.000000}{24.000000}\selectfont\catcode`\^=\active\def^{\ifmmode\sp\else\^{}\fi}\catcode`\%=\active\def%{\%}Normalized Demand Curve}}%
\end{pgfscope}%
\begin{pgfscope}%
\pgfsetbuttcap%
\pgfsetmiterjoin%
\definecolor{currentfill}{rgb}{1.000000,1.000000,1.000000}%
\pgfsetfillcolor{currentfill}%
\pgfsetlinewidth{0.000000pt}%
\definecolor{currentstroke}{rgb}{0.000000,0.000000,0.000000}%
\pgfsetstrokecolor{currentstroke}%
\pgfsetstrokeopacity{0.000000}%
\pgfsetdash{}{0pt}%
\pgfpathmoveto{\pgfqpoint{8.609492in}{0.670138in}}%
\pgfpathlineto{\pgfqpoint{11.676809in}{0.670138in}}%
\pgfpathlineto{\pgfqpoint{11.676809in}{5.516628in}}%
\pgfpathlineto{\pgfqpoint{8.609492in}{5.516628in}}%
\pgfpathlineto{\pgfqpoint{8.609492in}{0.670138in}}%
\pgfpathclose%
\pgfusepath{fill}%
\end{pgfscope}%
\begin{pgfscope}%
\pgfpathrectangle{\pgfqpoint{8.609492in}{0.670138in}}{\pgfqpoint{3.067317in}{4.846490in}}%
\pgfusepath{clip}%
\pgfsetrectcap%
\pgfsetroundjoin%
\pgfsetlinewidth{0.803000pt}%
\definecolor{currentstroke}{rgb}{0.690196,0.690196,0.690196}%
\pgfsetstrokecolor{currentstroke}%
\pgfsetdash{}{0pt}%
\pgfpathmoveto{\pgfqpoint{8.639564in}{0.670138in}}%
\pgfpathlineto{\pgfqpoint{8.639564in}{5.516628in}}%
\pgfusepath{stroke}%
\end{pgfscope}%
\begin{pgfscope}%
\pgfsetbuttcap%
\pgfsetroundjoin%
\definecolor{currentfill}{rgb}{0.000000,0.000000,0.000000}%
\pgfsetfillcolor{currentfill}%
\pgfsetlinewidth{0.803000pt}%
\definecolor{currentstroke}{rgb}{0.000000,0.000000,0.000000}%
\pgfsetstrokecolor{currentstroke}%
\pgfsetdash{}{0pt}%
\pgfsys@defobject{currentmarker}{\pgfqpoint{0.000000in}{-0.048611in}}{\pgfqpoint{0.000000in}{0.000000in}}{%
\pgfpathmoveto{\pgfqpoint{0.000000in}{0.000000in}}%
\pgfpathlineto{\pgfqpoint{0.000000in}{-0.048611in}}%
\pgfusepath{stroke,fill}%
}%
\begin{pgfscope}%
\pgfsys@transformshift{8.639564in}{0.670138in}%
\pgfsys@useobject{currentmarker}{}%
\end{pgfscope}%
\end{pgfscope}%
\begin{pgfscope}%
\definecolor{textcolor}{rgb}{0.000000,0.000000,0.000000}%
\pgfsetstrokecolor{textcolor}%
\pgfsetfillcolor{textcolor}%
\pgftext[x=8.639564in,y=0.572916in,,top]{\color{textcolor}{\rmfamily\fontsize{14.000000}{16.800000}\selectfont\catcode`\^=\active\def^{\ifmmode\sp\else\^{}\fi}\catcode`\%=\active\def%{\%}$\mathdefault{0}$}}%
\end{pgfscope}%
\begin{pgfscope}%
\pgfpathrectangle{\pgfqpoint{8.609492in}{0.670138in}}{\pgfqpoint{3.067317in}{4.846490in}}%
\pgfusepath{clip}%
\pgfsetrectcap%
\pgfsetroundjoin%
\pgfsetlinewidth{0.803000pt}%
\definecolor{currentstroke}{rgb}{0.690196,0.690196,0.690196}%
\pgfsetstrokecolor{currentstroke}%
\pgfsetdash{}{0pt}%
\pgfpathmoveto{\pgfqpoint{9.391357in}{0.670138in}}%
\pgfpathlineto{\pgfqpoint{9.391357in}{5.516628in}}%
\pgfusepath{stroke}%
\end{pgfscope}%
\begin{pgfscope}%
\pgfsetbuttcap%
\pgfsetroundjoin%
\definecolor{currentfill}{rgb}{0.000000,0.000000,0.000000}%
\pgfsetfillcolor{currentfill}%
\pgfsetlinewidth{0.803000pt}%
\definecolor{currentstroke}{rgb}{0.000000,0.000000,0.000000}%
\pgfsetstrokecolor{currentstroke}%
\pgfsetdash{}{0pt}%
\pgfsys@defobject{currentmarker}{\pgfqpoint{0.000000in}{-0.048611in}}{\pgfqpoint{0.000000in}{0.000000in}}{%
\pgfpathmoveto{\pgfqpoint{0.000000in}{0.000000in}}%
\pgfpathlineto{\pgfqpoint{0.000000in}{-0.048611in}}%
\pgfusepath{stroke,fill}%
}%
\begin{pgfscope}%
\pgfsys@transformshift{9.391357in}{0.670138in}%
\pgfsys@useobject{currentmarker}{}%
\end{pgfscope}%
\end{pgfscope}%
\begin{pgfscope}%
\definecolor{textcolor}{rgb}{0.000000,0.000000,0.000000}%
\pgfsetstrokecolor{textcolor}%
\pgfsetfillcolor{textcolor}%
\pgftext[x=9.391357in,y=0.572916in,,top]{\color{textcolor}{\rmfamily\fontsize{14.000000}{16.800000}\selectfont\catcode`\^=\active\def^{\ifmmode\sp\else\^{}\fi}\catcode`\%=\active\def%{\%}$\mathdefault{25}$}}%
\end{pgfscope}%
\begin{pgfscope}%
\pgfpathrectangle{\pgfqpoint{8.609492in}{0.670138in}}{\pgfqpoint{3.067317in}{4.846490in}}%
\pgfusepath{clip}%
\pgfsetrectcap%
\pgfsetroundjoin%
\pgfsetlinewidth{0.803000pt}%
\definecolor{currentstroke}{rgb}{0.690196,0.690196,0.690196}%
\pgfsetstrokecolor{currentstroke}%
\pgfsetdash{}{0pt}%
\pgfpathmoveto{\pgfqpoint{10.143151in}{0.670138in}}%
\pgfpathlineto{\pgfqpoint{10.143151in}{5.516628in}}%
\pgfusepath{stroke}%
\end{pgfscope}%
\begin{pgfscope}%
\pgfsetbuttcap%
\pgfsetroundjoin%
\definecolor{currentfill}{rgb}{0.000000,0.000000,0.000000}%
\pgfsetfillcolor{currentfill}%
\pgfsetlinewidth{0.803000pt}%
\definecolor{currentstroke}{rgb}{0.000000,0.000000,0.000000}%
\pgfsetstrokecolor{currentstroke}%
\pgfsetdash{}{0pt}%
\pgfsys@defobject{currentmarker}{\pgfqpoint{0.000000in}{-0.048611in}}{\pgfqpoint{0.000000in}{0.000000in}}{%
\pgfpathmoveto{\pgfqpoint{0.000000in}{0.000000in}}%
\pgfpathlineto{\pgfqpoint{0.000000in}{-0.048611in}}%
\pgfusepath{stroke,fill}%
}%
\begin{pgfscope}%
\pgfsys@transformshift{10.143151in}{0.670138in}%
\pgfsys@useobject{currentmarker}{}%
\end{pgfscope}%
\end{pgfscope}%
\begin{pgfscope}%
\definecolor{textcolor}{rgb}{0.000000,0.000000,0.000000}%
\pgfsetstrokecolor{textcolor}%
\pgfsetfillcolor{textcolor}%
\pgftext[x=10.143151in,y=0.572916in,,top]{\color{textcolor}{\rmfamily\fontsize{14.000000}{16.800000}\selectfont\catcode`\^=\active\def^{\ifmmode\sp\else\^{}\fi}\catcode`\%=\active\def%{\%}$\mathdefault{50}$}}%
\end{pgfscope}%
\begin{pgfscope}%
\pgfpathrectangle{\pgfqpoint{8.609492in}{0.670138in}}{\pgfqpoint{3.067317in}{4.846490in}}%
\pgfusepath{clip}%
\pgfsetrectcap%
\pgfsetroundjoin%
\pgfsetlinewidth{0.803000pt}%
\definecolor{currentstroke}{rgb}{0.690196,0.690196,0.690196}%
\pgfsetstrokecolor{currentstroke}%
\pgfsetdash{}{0pt}%
\pgfpathmoveto{\pgfqpoint{10.894944in}{0.670138in}}%
\pgfpathlineto{\pgfqpoint{10.894944in}{5.516628in}}%
\pgfusepath{stroke}%
\end{pgfscope}%
\begin{pgfscope}%
\pgfsetbuttcap%
\pgfsetroundjoin%
\definecolor{currentfill}{rgb}{0.000000,0.000000,0.000000}%
\pgfsetfillcolor{currentfill}%
\pgfsetlinewidth{0.803000pt}%
\definecolor{currentstroke}{rgb}{0.000000,0.000000,0.000000}%
\pgfsetstrokecolor{currentstroke}%
\pgfsetdash{}{0pt}%
\pgfsys@defobject{currentmarker}{\pgfqpoint{0.000000in}{-0.048611in}}{\pgfqpoint{0.000000in}{0.000000in}}{%
\pgfpathmoveto{\pgfqpoint{0.000000in}{0.000000in}}%
\pgfpathlineto{\pgfqpoint{0.000000in}{-0.048611in}}%
\pgfusepath{stroke,fill}%
}%
\begin{pgfscope}%
\pgfsys@transformshift{10.894944in}{0.670138in}%
\pgfsys@useobject{currentmarker}{}%
\end{pgfscope}%
\end{pgfscope}%
\begin{pgfscope}%
\definecolor{textcolor}{rgb}{0.000000,0.000000,0.000000}%
\pgfsetstrokecolor{textcolor}%
\pgfsetfillcolor{textcolor}%
\pgftext[x=10.894944in,y=0.572916in,,top]{\color{textcolor}{\rmfamily\fontsize{14.000000}{16.800000}\selectfont\catcode`\^=\active\def^{\ifmmode\sp\else\^{}\fi}\catcode`\%=\active\def%{\%}$\mathdefault{75}$}}%
\end{pgfscope}%
\begin{pgfscope}%
\pgfpathrectangle{\pgfqpoint{8.609492in}{0.670138in}}{\pgfqpoint{3.067317in}{4.846490in}}%
\pgfusepath{clip}%
\pgfsetrectcap%
\pgfsetroundjoin%
\pgfsetlinewidth{0.803000pt}%
\definecolor{currentstroke}{rgb}{0.690196,0.690196,0.690196}%
\pgfsetstrokecolor{currentstroke}%
\pgfsetdash{}{0pt}%
\pgfpathmoveto{\pgfqpoint{11.646737in}{0.670138in}}%
\pgfpathlineto{\pgfqpoint{11.646737in}{5.516628in}}%
\pgfusepath{stroke}%
\end{pgfscope}%
\begin{pgfscope}%
\pgfsetbuttcap%
\pgfsetroundjoin%
\definecolor{currentfill}{rgb}{0.000000,0.000000,0.000000}%
\pgfsetfillcolor{currentfill}%
\pgfsetlinewidth{0.803000pt}%
\definecolor{currentstroke}{rgb}{0.000000,0.000000,0.000000}%
\pgfsetstrokecolor{currentstroke}%
\pgfsetdash{}{0pt}%
\pgfsys@defobject{currentmarker}{\pgfqpoint{0.000000in}{-0.048611in}}{\pgfqpoint{0.000000in}{0.000000in}}{%
\pgfpathmoveto{\pgfqpoint{0.000000in}{0.000000in}}%
\pgfpathlineto{\pgfqpoint{0.000000in}{-0.048611in}}%
\pgfusepath{stroke,fill}%
}%
\begin{pgfscope}%
\pgfsys@transformshift{11.646737in}{0.670138in}%
\pgfsys@useobject{currentmarker}{}%
\end{pgfscope}%
\end{pgfscope}%
\begin{pgfscope}%
\definecolor{textcolor}{rgb}{0.000000,0.000000,0.000000}%
\pgfsetstrokecolor{textcolor}%
\pgfsetfillcolor{textcolor}%
\pgftext[x=11.646737in,y=0.572916in,,top]{\color{textcolor}{\rmfamily\fontsize{14.000000}{16.800000}\selectfont\catcode`\^=\active\def^{\ifmmode\sp\else\^{}\fi}\catcode`\%=\active\def%{\%}$\mathdefault{100}$}}%
\end{pgfscope}%
\begin{pgfscope}%
\pgfsetbuttcap%
\pgfsetroundjoin%
\definecolor{currentfill}{rgb}{0.000000,0.000000,0.000000}%
\pgfsetfillcolor{currentfill}%
\pgfsetlinewidth{0.602250pt}%
\definecolor{currentstroke}{rgb}{0.000000,0.000000,0.000000}%
\pgfsetstrokecolor{currentstroke}%
\pgfsetdash{}{0pt}%
\pgfsys@defobject{currentmarker}{\pgfqpoint{0.000000in}{-0.027778in}}{\pgfqpoint{0.000000in}{0.000000in}}{%
\pgfpathmoveto{\pgfqpoint{0.000000in}{0.000000in}}%
\pgfpathlineto{\pgfqpoint{0.000000in}{-0.027778in}}%
\pgfusepath{stroke,fill}%
}%
\begin{pgfscope}%
\pgfsys@transformshift{8.789923in}{0.670138in}%
\pgfsys@useobject{currentmarker}{}%
\end{pgfscope}%
\end{pgfscope}%
\begin{pgfscope}%
\pgfsetbuttcap%
\pgfsetroundjoin%
\definecolor{currentfill}{rgb}{0.000000,0.000000,0.000000}%
\pgfsetfillcolor{currentfill}%
\pgfsetlinewidth{0.602250pt}%
\definecolor{currentstroke}{rgb}{0.000000,0.000000,0.000000}%
\pgfsetstrokecolor{currentstroke}%
\pgfsetdash{}{0pt}%
\pgfsys@defobject{currentmarker}{\pgfqpoint{0.000000in}{-0.027778in}}{\pgfqpoint{0.000000in}{0.000000in}}{%
\pgfpathmoveto{\pgfqpoint{0.000000in}{0.000000in}}%
\pgfpathlineto{\pgfqpoint{0.000000in}{-0.027778in}}%
\pgfusepath{stroke,fill}%
}%
\begin{pgfscope}%
\pgfsys@transformshift{8.940281in}{0.670138in}%
\pgfsys@useobject{currentmarker}{}%
\end{pgfscope}%
\end{pgfscope}%
\begin{pgfscope}%
\pgfsetbuttcap%
\pgfsetroundjoin%
\definecolor{currentfill}{rgb}{0.000000,0.000000,0.000000}%
\pgfsetfillcolor{currentfill}%
\pgfsetlinewidth{0.602250pt}%
\definecolor{currentstroke}{rgb}{0.000000,0.000000,0.000000}%
\pgfsetstrokecolor{currentstroke}%
\pgfsetdash{}{0pt}%
\pgfsys@defobject{currentmarker}{\pgfqpoint{0.000000in}{-0.027778in}}{\pgfqpoint{0.000000in}{0.000000in}}{%
\pgfpathmoveto{\pgfqpoint{0.000000in}{0.000000in}}%
\pgfpathlineto{\pgfqpoint{0.000000in}{-0.027778in}}%
\pgfusepath{stroke,fill}%
}%
\begin{pgfscope}%
\pgfsys@transformshift{9.090640in}{0.670138in}%
\pgfsys@useobject{currentmarker}{}%
\end{pgfscope}%
\end{pgfscope}%
\begin{pgfscope}%
\pgfsetbuttcap%
\pgfsetroundjoin%
\definecolor{currentfill}{rgb}{0.000000,0.000000,0.000000}%
\pgfsetfillcolor{currentfill}%
\pgfsetlinewidth{0.602250pt}%
\definecolor{currentstroke}{rgb}{0.000000,0.000000,0.000000}%
\pgfsetstrokecolor{currentstroke}%
\pgfsetdash{}{0pt}%
\pgfsys@defobject{currentmarker}{\pgfqpoint{0.000000in}{-0.027778in}}{\pgfqpoint{0.000000in}{0.000000in}}{%
\pgfpathmoveto{\pgfqpoint{0.000000in}{0.000000in}}%
\pgfpathlineto{\pgfqpoint{0.000000in}{-0.027778in}}%
\pgfusepath{stroke,fill}%
}%
\begin{pgfscope}%
\pgfsys@transformshift{9.240999in}{0.670138in}%
\pgfsys@useobject{currentmarker}{}%
\end{pgfscope}%
\end{pgfscope}%
\begin{pgfscope}%
\pgfsetbuttcap%
\pgfsetroundjoin%
\definecolor{currentfill}{rgb}{0.000000,0.000000,0.000000}%
\pgfsetfillcolor{currentfill}%
\pgfsetlinewidth{0.602250pt}%
\definecolor{currentstroke}{rgb}{0.000000,0.000000,0.000000}%
\pgfsetstrokecolor{currentstroke}%
\pgfsetdash{}{0pt}%
\pgfsys@defobject{currentmarker}{\pgfqpoint{0.000000in}{-0.027778in}}{\pgfqpoint{0.000000in}{0.000000in}}{%
\pgfpathmoveto{\pgfqpoint{0.000000in}{0.000000in}}%
\pgfpathlineto{\pgfqpoint{0.000000in}{-0.027778in}}%
\pgfusepath{stroke,fill}%
}%
\begin{pgfscope}%
\pgfsys@transformshift{9.541716in}{0.670138in}%
\pgfsys@useobject{currentmarker}{}%
\end{pgfscope}%
\end{pgfscope}%
\begin{pgfscope}%
\pgfsetbuttcap%
\pgfsetroundjoin%
\definecolor{currentfill}{rgb}{0.000000,0.000000,0.000000}%
\pgfsetfillcolor{currentfill}%
\pgfsetlinewidth{0.602250pt}%
\definecolor{currentstroke}{rgb}{0.000000,0.000000,0.000000}%
\pgfsetstrokecolor{currentstroke}%
\pgfsetdash{}{0pt}%
\pgfsys@defobject{currentmarker}{\pgfqpoint{0.000000in}{-0.027778in}}{\pgfqpoint{0.000000in}{0.000000in}}{%
\pgfpathmoveto{\pgfqpoint{0.000000in}{0.000000in}}%
\pgfpathlineto{\pgfqpoint{0.000000in}{-0.027778in}}%
\pgfusepath{stroke,fill}%
}%
\begin{pgfscope}%
\pgfsys@transformshift{9.692075in}{0.670138in}%
\pgfsys@useobject{currentmarker}{}%
\end{pgfscope}%
\end{pgfscope}%
\begin{pgfscope}%
\pgfsetbuttcap%
\pgfsetroundjoin%
\definecolor{currentfill}{rgb}{0.000000,0.000000,0.000000}%
\pgfsetfillcolor{currentfill}%
\pgfsetlinewidth{0.602250pt}%
\definecolor{currentstroke}{rgb}{0.000000,0.000000,0.000000}%
\pgfsetstrokecolor{currentstroke}%
\pgfsetdash{}{0pt}%
\pgfsys@defobject{currentmarker}{\pgfqpoint{0.000000in}{-0.027778in}}{\pgfqpoint{0.000000in}{0.000000in}}{%
\pgfpathmoveto{\pgfqpoint{0.000000in}{0.000000in}}%
\pgfpathlineto{\pgfqpoint{0.000000in}{-0.027778in}}%
\pgfusepath{stroke,fill}%
}%
\begin{pgfscope}%
\pgfsys@transformshift{9.842433in}{0.670138in}%
\pgfsys@useobject{currentmarker}{}%
\end{pgfscope}%
\end{pgfscope}%
\begin{pgfscope}%
\pgfsetbuttcap%
\pgfsetroundjoin%
\definecolor{currentfill}{rgb}{0.000000,0.000000,0.000000}%
\pgfsetfillcolor{currentfill}%
\pgfsetlinewidth{0.602250pt}%
\definecolor{currentstroke}{rgb}{0.000000,0.000000,0.000000}%
\pgfsetstrokecolor{currentstroke}%
\pgfsetdash{}{0pt}%
\pgfsys@defobject{currentmarker}{\pgfqpoint{0.000000in}{-0.027778in}}{\pgfqpoint{0.000000in}{0.000000in}}{%
\pgfpathmoveto{\pgfqpoint{0.000000in}{0.000000in}}%
\pgfpathlineto{\pgfqpoint{0.000000in}{-0.027778in}}%
\pgfusepath{stroke,fill}%
}%
\begin{pgfscope}%
\pgfsys@transformshift{9.992792in}{0.670138in}%
\pgfsys@useobject{currentmarker}{}%
\end{pgfscope}%
\end{pgfscope}%
\begin{pgfscope}%
\pgfsetbuttcap%
\pgfsetroundjoin%
\definecolor{currentfill}{rgb}{0.000000,0.000000,0.000000}%
\pgfsetfillcolor{currentfill}%
\pgfsetlinewidth{0.602250pt}%
\definecolor{currentstroke}{rgb}{0.000000,0.000000,0.000000}%
\pgfsetstrokecolor{currentstroke}%
\pgfsetdash{}{0pt}%
\pgfsys@defobject{currentmarker}{\pgfqpoint{0.000000in}{-0.027778in}}{\pgfqpoint{0.000000in}{0.000000in}}{%
\pgfpathmoveto{\pgfqpoint{0.000000in}{0.000000in}}%
\pgfpathlineto{\pgfqpoint{0.000000in}{-0.027778in}}%
\pgfusepath{stroke,fill}%
}%
\begin{pgfscope}%
\pgfsys@transformshift{10.293509in}{0.670138in}%
\pgfsys@useobject{currentmarker}{}%
\end{pgfscope}%
\end{pgfscope}%
\begin{pgfscope}%
\pgfsetbuttcap%
\pgfsetroundjoin%
\definecolor{currentfill}{rgb}{0.000000,0.000000,0.000000}%
\pgfsetfillcolor{currentfill}%
\pgfsetlinewidth{0.602250pt}%
\definecolor{currentstroke}{rgb}{0.000000,0.000000,0.000000}%
\pgfsetstrokecolor{currentstroke}%
\pgfsetdash{}{0pt}%
\pgfsys@defobject{currentmarker}{\pgfqpoint{0.000000in}{-0.027778in}}{\pgfqpoint{0.000000in}{0.000000in}}{%
\pgfpathmoveto{\pgfqpoint{0.000000in}{0.000000in}}%
\pgfpathlineto{\pgfqpoint{0.000000in}{-0.027778in}}%
\pgfusepath{stroke,fill}%
}%
\begin{pgfscope}%
\pgfsys@transformshift{10.443868in}{0.670138in}%
\pgfsys@useobject{currentmarker}{}%
\end{pgfscope}%
\end{pgfscope}%
\begin{pgfscope}%
\pgfsetbuttcap%
\pgfsetroundjoin%
\definecolor{currentfill}{rgb}{0.000000,0.000000,0.000000}%
\pgfsetfillcolor{currentfill}%
\pgfsetlinewidth{0.602250pt}%
\definecolor{currentstroke}{rgb}{0.000000,0.000000,0.000000}%
\pgfsetstrokecolor{currentstroke}%
\pgfsetdash{}{0pt}%
\pgfsys@defobject{currentmarker}{\pgfqpoint{0.000000in}{-0.027778in}}{\pgfqpoint{0.000000in}{0.000000in}}{%
\pgfpathmoveto{\pgfqpoint{0.000000in}{0.000000in}}%
\pgfpathlineto{\pgfqpoint{0.000000in}{-0.027778in}}%
\pgfusepath{stroke,fill}%
}%
\begin{pgfscope}%
\pgfsys@transformshift{10.594227in}{0.670138in}%
\pgfsys@useobject{currentmarker}{}%
\end{pgfscope}%
\end{pgfscope}%
\begin{pgfscope}%
\pgfsetbuttcap%
\pgfsetroundjoin%
\definecolor{currentfill}{rgb}{0.000000,0.000000,0.000000}%
\pgfsetfillcolor{currentfill}%
\pgfsetlinewidth{0.602250pt}%
\definecolor{currentstroke}{rgb}{0.000000,0.000000,0.000000}%
\pgfsetstrokecolor{currentstroke}%
\pgfsetdash{}{0pt}%
\pgfsys@defobject{currentmarker}{\pgfqpoint{0.000000in}{-0.027778in}}{\pgfqpoint{0.000000in}{0.000000in}}{%
\pgfpathmoveto{\pgfqpoint{0.000000in}{0.000000in}}%
\pgfpathlineto{\pgfqpoint{0.000000in}{-0.027778in}}%
\pgfusepath{stroke,fill}%
}%
\begin{pgfscope}%
\pgfsys@transformshift{10.744585in}{0.670138in}%
\pgfsys@useobject{currentmarker}{}%
\end{pgfscope}%
\end{pgfscope}%
\begin{pgfscope}%
\pgfsetbuttcap%
\pgfsetroundjoin%
\definecolor{currentfill}{rgb}{0.000000,0.000000,0.000000}%
\pgfsetfillcolor{currentfill}%
\pgfsetlinewidth{0.602250pt}%
\definecolor{currentstroke}{rgb}{0.000000,0.000000,0.000000}%
\pgfsetstrokecolor{currentstroke}%
\pgfsetdash{}{0pt}%
\pgfsys@defobject{currentmarker}{\pgfqpoint{0.000000in}{-0.027778in}}{\pgfqpoint{0.000000in}{0.000000in}}{%
\pgfpathmoveto{\pgfqpoint{0.000000in}{0.000000in}}%
\pgfpathlineto{\pgfqpoint{0.000000in}{-0.027778in}}%
\pgfusepath{stroke,fill}%
}%
\begin{pgfscope}%
\pgfsys@transformshift{11.045303in}{0.670138in}%
\pgfsys@useobject{currentmarker}{}%
\end{pgfscope}%
\end{pgfscope}%
\begin{pgfscope}%
\pgfsetbuttcap%
\pgfsetroundjoin%
\definecolor{currentfill}{rgb}{0.000000,0.000000,0.000000}%
\pgfsetfillcolor{currentfill}%
\pgfsetlinewidth{0.602250pt}%
\definecolor{currentstroke}{rgb}{0.000000,0.000000,0.000000}%
\pgfsetstrokecolor{currentstroke}%
\pgfsetdash{}{0pt}%
\pgfsys@defobject{currentmarker}{\pgfqpoint{0.000000in}{-0.027778in}}{\pgfqpoint{0.000000in}{0.000000in}}{%
\pgfpathmoveto{\pgfqpoint{0.000000in}{0.000000in}}%
\pgfpathlineto{\pgfqpoint{0.000000in}{-0.027778in}}%
\pgfusepath{stroke,fill}%
}%
\begin{pgfscope}%
\pgfsys@transformshift{11.195661in}{0.670138in}%
\pgfsys@useobject{currentmarker}{}%
\end{pgfscope}%
\end{pgfscope}%
\begin{pgfscope}%
\pgfsetbuttcap%
\pgfsetroundjoin%
\definecolor{currentfill}{rgb}{0.000000,0.000000,0.000000}%
\pgfsetfillcolor{currentfill}%
\pgfsetlinewidth{0.602250pt}%
\definecolor{currentstroke}{rgb}{0.000000,0.000000,0.000000}%
\pgfsetstrokecolor{currentstroke}%
\pgfsetdash{}{0pt}%
\pgfsys@defobject{currentmarker}{\pgfqpoint{0.000000in}{-0.027778in}}{\pgfqpoint{0.000000in}{0.000000in}}{%
\pgfpathmoveto{\pgfqpoint{0.000000in}{0.000000in}}%
\pgfpathlineto{\pgfqpoint{0.000000in}{-0.027778in}}%
\pgfusepath{stroke,fill}%
}%
\begin{pgfscope}%
\pgfsys@transformshift{11.346020in}{0.670138in}%
\pgfsys@useobject{currentmarker}{}%
\end{pgfscope}%
\end{pgfscope}%
\begin{pgfscope}%
\pgfsetbuttcap%
\pgfsetroundjoin%
\definecolor{currentfill}{rgb}{0.000000,0.000000,0.000000}%
\pgfsetfillcolor{currentfill}%
\pgfsetlinewidth{0.602250pt}%
\definecolor{currentstroke}{rgb}{0.000000,0.000000,0.000000}%
\pgfsetstrokecolor{currentstroke}%
\pgfsetdash{}{0pt}%
\pgfsys@defobject{currentmarker}{\pgfqpoint{0.000000in}{-0.027778in}}{\pgfqpoint{0.000000in}{0.000000in}}{%
\pgfpathmoveto{\pgfqpoint{0.000000in}{0.000000in}}%
\pgfpathlineto{\pgfqpoint{0.000000in}{-0.027778in}}%
\pgfusepath{stroke,fill}%
}%
\begin{pgfscope}%
\pgfsys@transformshift{11.496379in}{0.670138in}%
\pgfsys@useobject{currentmarker}{}%
\end{pgfscope}%
\end{pgfscope}%
\begin{pgfscope}%
\definecolor{textcolor}{rgb}{0.000000,0.000000,0.000000}%
\pgfsetstrokecolor{textcolor}%
\pgfsetfillcolor{textcolor}%
\pgftext[x=10.143151in,y=0.339583in,,top]{\color{textcolor}{\rmfamily\fontsize{18.000000}{21.600000}\selectfont\catcode`\^=\active\def^{\ifmmode\sp\else\^{}\fi}\catcode`\%=\active\def%{\%}Time [\%]}}%
\end{pgfscope}%
\begin{pgfscope}%
\pgfpathrectangle{\pgfqpoint{8.609492in}{0.670138in}}{\pgfqpoint{3.067317in}{4.846490in}}%
\pgfusepath{clip}%
\pgfsetrectcap%
\pgfsetroundjoin%
\pgfsetlinewidth{0.803000pt}%
\definecolor{currentstroke}{rgb}{0.690196,0.690196,0.690196}%
\pgfsetstrokecolor{currentstroke}%
\pgfsetdash{}{0pt}%
\pgfpathmoveto{\pgfqpoint{8.609492in}{1.314483in}}%
\pgfpathlineto{\pgfqpoint{11.676809in}{1.314483in}}%
\pgfusepath{stroke}%
\end{pgfscope}%
\begin{pgfscope}%
\pgfsetbuttcap%
\pgfsetroundjoin%
\definecolor{currentfill}{rgb}{0.000000,0.000000,0.000000}%
\pgfsetfillcolor{currentfill}%
\pgfsetlinewidth{0.803000pt}%
\definecolor{currentstroke}{rgb}{0.000000,0.000000,0.000000}%
\pgfsetstrokecolor{currentstroke}%
\pgfsetdash{}{0pt}%
\pgfsys@defobject{currentmarker}{\pgfqpoint{-0.048611in}{0.000000in}}{\pgfqpoint{-0.000000in}{0.000000in}}{%
\pgfpathmoveto{\pgfqpoint{-0.000000in}{0.000000in}}%
\pgfpathlineto{\pgfqpoint{-0.048611in}{0.000000in}}%
\pgfusepath{stroke,fill}%
}%
\begin{pgfscope}%
\pgfsys@transformshift{8.609492in}{1.314483in}%
\pgfsys@useobject{currentmarker}{}%
\end{pgfscope}%
\end{pgfscope}%
\begin{pgfscope}%
\pgfpathrectangle{\pgfqpoint{8.609492in}{0.670138in}}{\pgfqpoint{3.067317in}{4.846490in}}%
\pgfusepath{clip}%
\pgfsetrectcap%
\pgfsetroundjoin%
\pgfsetlinewidth{0.803000pt}%
\definecolor{currentstroke}{rgb}{0.690196,0.690196,0.690196}%
\pgfsetstrokecolor{currentstroke}%
\pgfsetdash{}{0pt}%
\pgfpathmoveto{\pgfqpoint{8.609492in}{2.365019in}}%
\pgfpathlineto{\pgfqpoint{11.676809in}{2.365019in}}%
\pgfusepath{stroke}%
\end{pgfscope}%
\begin{pgfscope}%
\pgfsetbuttcap%
\pgfsetroundjoin%
\definecolor{currentfill}{rgb}{0.000000,0.000000,0.000000}%
\pgfsetfillcolor{currentfill}%
\pgfsetlinewidth{0.803000pt}%
\definecolor{currentstroke}{rgb}{0.000000,0.000000,0.000000}%
\pgfsetstrokecolor{currentstroke}%
\pgfsetdash{}{0pt}%
\pgfsys@defobject{currentmarker}{\pgfqpoint{-0.048611in}{0.000000in}}{\pgfqpoint{-0.000000in}{0.000000in}}{%
\pgfpathmoveto{\pgfqpoint{-0.000000in}{0.000000in}}%
\pgfpathlineto{\pgfqpoint{-0.048611in}{0.000000in}}%
\pgfusepath{stroke,fill}%
}%
\begin{pgfscope}%
\pgfsys@transformshift{8.609492in}{2.365019in}%
\pgfsys@useobject{currentmarker}{}%
\end{pgfscope}%
\end{pgfscope}%
\begin{pgfscope}%
\pgfpathrectangle{\pgfqpoint{8.609492in}{0.670138in}}{\pgfqpoint{3.067317in}{4.846490in}}%
\pgfusepath{clip}%
\pgfsetrectcap%
\pgfsetroundjoin%
\pgfsetlinewidth{0.803000pt}%
\definecolor{currentstroke}{rgb}{0.690196,0.690196,0.690196}%
\pgfsetstrokecolor{currentstroke}%
\pgfsetdash{}{0pt}%
\pgfpathmoveto{\pgfqpoint{8.609492in}{3.415556in}}%
\pgfpathlineto{\pgfqpoint{11.676809in}{3.415556in}}%
\pgfusepath{stroke}%
\end{pgfscope}%
\begin{pgfscope}%
\pgfsetbuttcap%
\pgfsetroundjoin%
\definecolor{currentfill}{rgb}{0.000000,0.000000,0.000000}%
\pgfsetfillcolor{currentfill}%
\pgfsetlinewidth{0.803000pt}%
\definecolor{currentstroke}{rgb}{0.000000,0.000000,0.000000}%
\pgfsetstrokecolor{currentstroke}%
\pgfsetdash{}{0pt}%
\pgfsys@defobject{currentmarker}{\pgfqpoint{-0.048611in}{0.000000in}}{\pgfqpoint{-0.000000in}{0.000000in}}{%
\pgfpathmoveto{\pgfqpoint{-0.000000in}{0.000000in}}%
\pgfpathlineto{\pgfqpoint{-0.048611in}{0.000000in}}%
\pgfusepath{stroke,fill}%
}%
\begin{pgfscope}%
\pgfsys@transformshift{8.609492in}{3.415556in}%
\pgfsys@useobject{currentmarker}{}%
\end{pgfscope}%
\end{pgfscope}%
\begin{pgfscope}%
\pgfpathrectangle{\pgfqpoint{8.609492in}{0.670138in}}{\pgfqpoint{3.067317in}{4.846490in}}%
\pgfusepath{clip}%
\pgfsetrectcap%
\pgfsetroundjoin%
\pgfsetlinewidth{0.803000pt}%
\definecolor{currentstroke}{rgb}{0.690196,0.690196,0.690196}%
\pgfsetstrokecolor{currentstroke}%
\pgfsetdash{}{0pt}%
\pgfpathmoveto{\pgfqpoint{8.609492in}{4.466092in}}%
\pgfpathlineto{\pgfqpoint{11.676809in}{4.466092in}}%
\pgfusepath{stroke}%
\end{pgfscope}%
\begin{pgfscope}%
\pgfsetbuttcap%
\pgfsetroundjoin%
\definecolor{currentfill}{rgb}{0.000000,0.000000,0.000000}%
\pgfsetfillcolor{currentfill}%
\pgfsetlinewidth{0.803000pt}%
\definecolor{currentstroke}{rgb}{0.000000,0.000000,0.000000}%
\pgfsetstrokecolor{currentstroke}%
\pgfsetdash{}{0pt}%
\pgfsys@defobject{currentmarker}{\pgfqpoint{-0.048611in}{0.000000in}}{\pgfqpoint{-0.000000in}{0.000000in}}{%
\pgfpathmoveto{\pgfqpoint{-0.000000in}{0.000000in}}%
\pgfpathlineto{\pgfqpoint{-0.048611in}{0.000000in}}%
\pgfusepath{stroke,fill}%
}%
\begin{pgfscope}%
\pgfsys@transformshift{8.609492in}{4.466092in}%
\pgfsys@useobject{currentmarker}{}%
\end{pgfscope}%
\end{pgfscope}%
\begin{pgfscope}%
\pgfpathrectangle{\pgfqpoint{8.609492in}{0.670138in}}{\pgfqpoint{3.067317in}{4.846490in}}%
\pgfusepath{clip}%
\pgfsetrectcap%
\pgfsetroundjoin%
\pgfsetlinewidth{0.803000pt}%
\definecolor{currentstroke}{rgb}{0.690196,0.690196,0.690196}%
\pgfsetstrokecolor{currentstroke}%
\pgfsetdash{}{0pt}%
\pgfpathmoveto{\pgfqpoint{8.609492in}{5.516628in}}%
\pgfpathlineto{\pgfqpoint{11.676809in}{5.516628in}}%
\pgfusepath{stroke}%
\end{pgfscope}%
\begin{pgfscope}%
\pgfsetbuttcap%
\pgfsetroundjoin%
\definecolor{currentfill}{rgb}{0.000000,0.000000,0.000000}%
\pgfsetfillcolor{currentfill}%
\pgfsetlinewidth{0.803000pt}%
\definecolor{currentstroke}{rgb}{0.000000,0.000000,0.000000}%
\pgfsetstrokecolor{currentstroke}%
\pgfsetdash{}{0pt}%
\pgfsys@defobject{currentmarker}{\pgfqpoint{-0.048611in}{0.000000in}}{\pgfqpoint{-0.000000in}{0.000000in}}{%
\pgfpathmoveto{\pgfqpoint{-0.000000in}{0.000000in}}%
\pgfpathlineto{\pgfqpoint{-0.048611in}{0.000000in}}%
\pgfusepath{stroke,fill}%
}%
\begin{pgfscope}%
\pgfsys@transformshift{8.609492in}{5.516628in}%
\pgfsys@useobject{currentmarker}{}%
\end{pgfscope}%
\end{pgfscope}%
\begin{pgfscope}%
\pgfsetbuttcap%
\pgfsetroundjoin%
\definecolor{currentfill}{rgb}{0.000000,0.000000,0.000000}%
\pgfsetfillcolor{currentfill}%
\pgfsetlinewidth{0.602250pt}%
\definecolor{currentstroke}{rgb}{0.000000,0.000000,0.000000}%
\pgfsetstrokecolor{currentstroke}%
\pgfsetdash{}{0pt}%
\pgfsys@defobject{currentmarker}{\pgfqpoint{-0.027778in}{0.000000in}}{\pgfqpoint{-0.000000in}{0.000000in}}{%
\pgfpathmoveto{\pgfqpoint{-0.000000in}{0.000000in}}%
\pgfpathlineto{\pgfqpoint{-0.027778in}{0.000000in}}%
\pgfusepath{stroke,fill}%
}%
\begin{pgfscope}%
\pgfsys@transformshift{8.609492in}{0.684161in}%
\pgfsys@useobject{currentmarker}{}%
\end{pgfscope}%
\end{pgfscope}%
\begin{pgfscope}%
\pgfsetbuttcap%
\pgfsetroundjoin%
\definecolor{currentfill}{rgb}{0.000000,0.000000,0.000000}%
\pgfsetfillcolor{currentfill}%
\pgfsetlinewidth{0.602250pt}%
\definecolor{currentstroke}{rgb}{0.000000,0.000000,0.000000}%
\pgfsetstrokecolor{currentstroke}%
\pgfsetdash{}{0pt}%
\pgfsys@defobject{currentmarker}{\pgfqpoint{-0.027778in}{0.000000in}}{\pgfqpoint{-0.000000in}{0.000000in}}{%
\pgfpathmoveto{\pgfqpoint{-0.000000in}{0.000000in}}%
\pgfpathlineto{\pgfqpoint{-0.027778in}{0.000000in}}%
\pgfusepath{stroke,fill}%
}%
\begin{pgfscope}%
\pgfsys@transformshift{8.609492in}{0.894269in}%
\pgfsys@useobject{currentmarker}{}%
\end{pgfscope}%
\end{pgfscope}%
\begin{pgfscope}%
\pgfsetbuttcap%
\pgfsetroundjoin%
\definecolor{currentfill}{rgb}{0.000000,0.000000,0.000000}%
\pgfsetfillcolor{currentfill}%
\pgfsetlinewidth{0.602250pt}%
\definecolor{currentstroke}{rgb}{0.000000,0.000000,0.000000}%
\pgfsetstrokecolor{currentstroke}%
\pgfsetdash{}{0pt}%
\pgfsys@defobject{currentmarker}{\pgfqpoint{-0.027778in}{0.000000in}}{\pgfqpoint{-0.000000in}{0.000000in}}{%
\pgfpathmoveto{\pgfqpoint{-0.000000in}{0.000000in}}%
\pgfpathlineto{\pgfqpoint{-0.027778in}{0.000000in}}%
\pgfusepath{stroke,fill}%
}%
\begin{pgfscope}%
\pgfsys@transformshift{8.609492in}{1.104376in}%
\pgfsys@useobject{currentmarker}{}%
\end{pgfscope}%
\end{pgfscope}%
\begin{pgfscope}%
\pgfsetbuttcap%
\pgfsetroundjoin%
\definecolor{currentfill}{rgb}{0.000000,0.000000,0.000000}%
\pgfsetfillcolor{currentfill}%
\pgfsetlinewidth{0.602250pt}%
\definecolor{currentstroke}{rgb}{0.000000,0.000000,0.000000}%
\pgfsetstrokecolor{currentstroke}%
\pgfsetdash{}{0pt}%
\pgfsys@defobject{currentmarker}{\pgfqpoint{-0.027778in}{0.000000in}}{\pgfqpoint{-0.000000in}{0.000000in}}{%
\pgfpathmoveto{\pgfqpoint{-0.000000in}{0.000000in}}%
\pgfpathlineto{\pgfqpoint{-0.027778in}{0.000000in}}%
\pgfusepath{stroke,fill}%
}%
\begin{pgfscope}%
\pgfsys@transformshift{8.609492in}{1.524590in}%
\pgfsys@useobject{currentmarker}{}%
\end{pgfscope}%
\end{pgfscope}%
\begin{pgfscope}%
\pgfsetbuttcap%
\pgfsetroundjoin%
\definecolor{currentfill}{rgb}{0.000000,0.000000,0.000000}%
\pgfsetfillcolor{currentfill}%
\pgfsetlinewidth{0.602250pt}%
\definecolor{currentstroke}{rgb}{0.000000,0.000000,0.000000}%
\pgfsetstrokecolor{currentstroke}%
\pgfsetdash{}{0pt}%
\pgfsys@defobject{currentmarker}{\pgfqpoint{-0.027778in}{0.000000in}}{\pgfqpoint{-0.000000in}{0.000000in}}{%
\pgfpathmoveto{\pgfqpoint{-0.000000in}{0.000000in}}%
\pgfpathlineto{\pgfqpoint{-0.027778in}{0.000000in}}%
\pgfusepath{stroke,fill}%
}%
\begin{pgfscope}%
\pgfsys@transformshift{8.609492in}{1.734698in}%
\pgfsys@useobject{currentmarker}{}%
\end{pgfscope}%
\end{pgfscope}%
\begin{pgfscope}%
\pgfsetbuttcap%
\pgfsetroundjoin%
\definecolor{currentfill}{rgb}{0.000000,0.000000,0.000000}%
\pgfsetfillcolor{currentfill}%
\pgfsetlinewidth{0.602250pt}%
\definecolor{currentstroke}{rgb}{0.000000,0.000000,0.000000}%
\pgfsetstrokecolor{currentstroke}%
\pgfsetdash{}{0pt}%
\pgfsys@defobject{currentmarker}{\pgfqpoint{-0.027778in}{0.000000in}}{\pgfqpoint{-0.000000in}{0.000000in}}{%
\pgfpathmoveto{\pgfqpoint{-0.000000in}{0.000000in}}%
\pgfpathlineto{\pgfqpoint{-0.027778in}{0.000000in}}%
\pgfusepath{stroke,fill}%
}%
\begin{pgfscope}%
\pgfsys@transformshift{8.609492in}{1.944805in}%
\pgfsys@useobject{currentmarker}{}%
\end{pgfscope}%
\end{pgfscope}%
\begin{pgfscope}%
\pgfsetbuttcap%
\pgfsetroundjoin%
\definecolor{currentfill}{rgb}{0.000000,0.000000,0.000000}%
\pgfsetfillcolor{currentfill}%
\pgfsetlinewidth{0.602250pt}%
\definecolor{currentstroke}{rgb}{0.000000,0.000000,0.000000}%
\pgfsetstrokecolor{currentstroke}%
\pgfsetdash{}{0pt}%
\pgfsys@defobject{currentmarker}{\pgfqpoint{-0.027778in}{0.000000in}}{\pgfqpoint{-0.000000in}{0.000000in}}{%
\pgfpathmoveto{\pgfqpoint{-0.000000in}{0.000000in}}%
\pgfpathlineto{\pgfqpoint{-0.027778in}{0.000000in}}%
\pgfusepath{stroke,fill}%
}%
\begin{pgfscope}%
\pgfsys@transformshift{8.609492in}{2.154912in}%
\pgfsys@useobject{currentmarker}{}%
\end{pgfscope}%
\end{pgfscope}%
\begin{pgfscope}%
\pgfsetbuttcap%
\pgfsetroundjoin%
\definecolor{currentfill}{rgb}{0.000000,0.000000,0.000000}%
\pgfsetfillcolor{currentfill}%
\pgfsetlinewidth{0.602250pt}%
\definecolor{currentstroke}{rgb}{0.000000,0.000000,0.000000}%
\pgfsetstrokecolor{currentstroke}%
\pgfsetdash{}{0pt}%
\pgfsys@defobject{currentmarker}{\pgfqpoint{-0.027778in}{0.000000in}}{\pgfqpoint{-0.000000in}{0.000000in}}{%
\pgfpathmoveto{\pgfqpoint{-0.000000in}{0.000000in}}%
\pgfpathlineto{\pgfqpoint{-0.027778in}{0.000000in}}%
\pgfusepath{stroke,fill}%
}%
\begin{pgfscope}%
\pgfsys@transformshift{8.609492in}{2.575127in}%
\pgfsys@useobject{currentmarker}{}%
\end{pgfscope}%
\end{pgfscope}%
\begin{pgfscope}%
\pgfsetbuttcap%
\pgfsetroundjoin%
\definecolor{currentfill}{rgb}{0.000000,0.000000,0.000000}%
\pgfsetfillcolor{currentfill}%
\pgfsetlinewidth{0.602250pt}%
\definecolor{currentstroke}{rgb}{0.000000,0.000000,0.000000}%
\pgfsetstrokecolor{currentstroke}%
\pgfsetdash{}{0pt}%
\pgfsys@defobject{currentmarker}{\pgfqpoint{-0.027778in}{0.000000in}}{\pgfqpoint{-0.000000in}{0.000000in}}{%
\pgfpathmoveto{\pgfqpoint{-0.000000in}{0.000000in}}%
\pgfpathlineto{\pgfqpoint{-0.027778in}{0.000000in}}%
\pgfusepath{stroke,fill}%
}%
\begin{pgfscope}%
\pgfsys@transformshift{8.609492in}{2.785234in}%
\pgfsys@useobject{currentmarker}{}%
\end{pgfscope}%
\end{pgfscope}%
\begin{pgfscope}%
\pgfsetbuttcap%
\pgfsetroundjoin%
\definecolor{currentfill}{rgb}{0.000000,0.000000,0.000000}%
\pgfsetfillcolor{currentfill}%
\pgfsetlinewidth{0.602250pt}%
\definecolor{currentstroke}{rgb}{0.000000,0.000000,0.000000}%
\pgfsetstrokecolor{currentstroke}%
\pgfsetdash{}{0pt}%
\pgfsys@defobject{currentmarker}{\pgfqpoint{-0.027778in}{0.000000in}}{\pgfqpoint{-0.000000in}{0.000000in}}{%
\pgfpathmoveto{\pgfqpoint{-0.000000in}{0.000000in}}%
\pgfpathlineto{\pgfqpoint{-0.027778in}{0.000000in}}%
\pgfusepath{stroke,fill}%
}%
\begin{pgfscope}%
\pgfsys@transformshift{8.609492in}{2.995341in}%
\pgfsys@useobject{currentmarker}{}%
\end{pgfscope}%
\end{pgfscope}%
\begin{pgfscope}%
\pgfsetbuttcap%
\pgfsetroundjoin%
\definecolor{currentfill}{rgb}{0.000000,0.000000,0.000000}%
\pgfsetfillcolor{currentfill}%
\pgfsetlinewidth{0.602250pt}%
\definecolor{currentstroke}{rgb}{0.000000,0.000000,0.000000}%
\pgfsetstrokecolor{currentstroke}%
\pgfsetdash{}{0pt}%
\pgfsys@defobject{currentmarker}{\pgfqpoint{-0.027778in}{0.000000in}}{\pgfqpoint{-0.000000in}{0.000000in}}{%
\pgfpathmoveto{\pgfqpoint{-0.000000in}{0.000000in}}%
\pgfpathlineto{\pgfqpoint{-0.027778in}{0.000000in}}%
\pgfusepath{stroke,fill}%
}%
\begin{pgfscope}%
\pgfsys@transformshift{8.609492in}{3.205448in}%
\pgfsys@useobject{currentmarker}{}%
\end{pgfscope}%
\end{pgfscope}%
\begin{pgfscope}%
\pgfsetbuttcap%
\pgfsetroundjoin%
\definecolor{currentfill}{rgb}{0.000000,0.000000,0.000000}%
\pgfsetfillcolor{currentfill}%
\pgfsetlinewidth{0.602250pt}%
\definecolor{currentstroke}{rgb}{0.000000,0.000000,0.000000}%
\pgfsetstrokecolor{currentstroke}%
\pgfsetdash{}{0pt}%
\pgfsys@defobject{currentmarker}{\pgfqpoint{-0.027778in}{0.000000in}}{\pgfqpoint{-0.000000in}{0.000000in}}{%
\pgfpathmoveto{\pgfqpoint{-0.000000in}{0.000000in}}%
\pgfpathlineto{\pgfqpoint{-0.027778in}{0.000000in}}%
\pgfusepath{stroke,fill}%
}%
\begin{pgfscope}%
\pgfsys@transformshift{8.609492in}{3.625663in}%
\pgfsys@useobject{currentmarker}{}%
\end{pgfscope}%
\end{pgfscope}%
\begin{pgfscope}%
\pgfsetbuttcap%
\pgfsetroundjoin%
\definecolor{currentfill}{rgb}{0.000000,0.000000,0.000000}%
\pgfsetfillcolor{currentfill}%
\pgfsetlinewidth{0.602250pt}%
\definecolor{currentstroke}{rgb}{0.000000,0.000000,0.000000}%
\pgfsetstrokecolor{currentstroke}%
\pgfsetdash{}{0pt}%
\pgfsys@defobject{currentmarker}{\pgfqpoint{-0.027778in}{0.000000in}}{\pgfqpoint{-0.000000in}{0.000000in}}{%
\pgfpathmoveto{\pgfqpoint{-0.000000in}{0.000000in}}%
\pgfpathlineto{\pgfqpoint{-0.027778in}{0.000000in}}%
\pgfusepath{stroke,fill}%
}%
\begin{pgfscope}%
\pgfsys@transformshift{8.609492in}{3.835770in}%
\pgfsys@useobject{currentmarker}{}%
\end{pgfscope}%
\end{pgfscope}%
\begin{pgfscope}%
\pgfsetbuttcap%
\pgfsetroundjoin%
\definecolor{currentfill}{rgb}{0.000000,0.000000,0.000000}%
\pgfsetfillcolor{currentfill}%
\pgfsetlinewidth{0.602250pt}%
\definecolor{currentstroke}{rgb}{0.000000,0.000000,0.000000}%
\pgfsetstrokecolor{currentstroke}%
\pgfsetdash{}{0pt}%
\pgfsys@defobject{currentmarker}{\pgfqpoint{-0.027778in}{0.000000in}}{\pgfqpoint{-0.000000in}{0.000000in}}{%
\pgfpathmoveto{\pgfqpoint{-0.000000in}{0.000000in}}%
\pgfpathlineto{\pgfqpoint{-0.027778in}{0.000000in}}%
\pgfusepath{stroke,fill}%
}%
\begin{pgfscope}%
\pgfsys@transformshift{8.609492in}{4.045877in}%
\pgfsys@useobject{currentmarker}{}%
\end{pgfscope}%
\end{pgfscope}%
\begin{pgfscope}%
\pgfsetbuttcap%
\pgfsetroundjoin%
\definecolor{currentfill}{rgb}{0.000000,0.000000,0.000000}%
\pgfsetfillcolor{currentfill}%
\pgfsetlinewidth{0.602250pt}%
\definecolor{currentstroke}{rgb}{0.000000,0.000000,0.000000}%
\pgfsetstrokecolor{currentstroke}%
\pgfsetdash{}{0pt}%
\pgfsys@defobject{currentmarker}{\pgfqpoint{-0.027778in}{0.000000in}}{\pgfqpoint{-0.000000in}{0.000000in}}{%
\pgfpathmoveto{\pgfqpoint{-0.000000in}{0.000000in}}%
\pgfpathlineto{\pgfqpoint{-0.027778in}{0.000000in}}%
\pgfusepath{stroke,fill}%
}%
\begin{pgfscope}%
\pgfsys@transformshift{8.609492in}{4.255985in}%
\pgfsys@useobject{currentmarker}{}%
\end{pgfscope}%
\end{pgfscope}%
\begin{pgfscope}%
\pgfsetbuttcap%
\pgfsetroundjoin%
\definecolor{currentfill}{rgb}{0.000000,0.000000,0.000000}%
\pgfsetfillcolor{currentfill}%
\pgfsetlinewidth{0.602250pt}%
\definecolor{currentstroke}{rgb}{0.000000,0.000000,0.000000}%
\pgfsetstrokecolor{currentstroke}%
\pgfsetdash{}{0pt}%
\pgfsys@defobject{currentmarker}{\pgfqpoint{-0.027778in}{0.000000in}}{\pgfqpoint{-0.000000in}{0.000000in}}{%
\pgfpathmoveto{\pgfqpoint{-0.000000in}{0.000000in}}%
\pgfpathlineto{\pgfqpoint{-0.027778in}{0.000000in}}%
\pgfusepath{stroke,fill}%
}%
\begin{pgfscope}%
\pgfsys@transformshift{8.609492in}{4.676199in}%
\pgfsys@useobject{currentmarker}{}%
\end{pgfscope}%
\end{pgfscope}%
\begin{pgfscope}%
\pgfsetbuttcap%
\pgfsetroundjoin%
\definecolor{currentfill}{rgb}{0.000000,0.000000,0.000000}%
\pgfsetfillcolor{currentfill}%
\pgfsetlinewidth{0.602250pt}%
\definecolor{currentstroke}{rgb}{0.000000,0.000000,0.000000}%
\pgfsetstrokecolor{currentstroke}%
\pgfsetdash{}{0pt}%
\pgfsys@defobject{currentmarker}{\pgfqpoint{-0.027778in}{0.000000in}}{\pgfqpoint{-0.000000in}{0.000000in}}{%
\pgfpathmoveto{\pgfqpoint{-0.000000in}{0.000000in}}%
\pgfpathlineto{\pgfqpoint{-0.027778in}{0.000000in}}%
\pgfusepath{stroke,fill}%
}%
\begin{pgfscope}%
\pgfsys@transformshift{8.609492in}{4.886306in}%
\pgfsys@useobject{currentmarker}{}%
\end{pgfscope}%
\end{pgfscope}%
\begin{pgfscope}%
\pgfsetbuttcap%
\pgfsetroundjoin%
\definecolor{currentfill}{rgb}{0.000000,0.000000,0.000000}%
\pgfsetfillcolor{currentfill}%
\pgfsetlinewidth{0.602250pt}%
\definecolor{currentstroke}{rgb}{0.000000,0.000000,0.000000}%
\pgfsetstrokecolor{currentstroke}%
\pgfsetdash{}{0pt}%
\pgfsys@defobject{currentmarker}{\pgfqpoint{-0.027778in}{0.000000in}}{\pgfqpoint{-0.000000in}{0.000000in}}{%
\pgfpathmoveto{\pgfqpoint{-0.000000in}{0.000000in}}%
\pgfpathlineto{\pgfqpoint{-0.027778in}{0.000000in}}%
\pgfusepath{stroke,fill}%
}%
\begin{pgfscope}%
\pgfsys@transformshift{8.609492in}{5.096414in}%
\pgfsys@useobject{currentmarker}{}%
\end{pgfscope}%
\end{pgfscope}%
\begin{pgfscope}%
\pgfsetbuttcap%
\pgfsetroundjoin%
\definecolor{currentfill}{rgb}{0.000000,0.000000,0.000000}%
\pgfsetfillcolor{currentfill}%
\pgfsetlinewidth{0.602250pt}%
\definecolor{currentstroke}{rgb}{0.000000,0.000000,0.000000}%
\pgfsetstrokecolor{currentstroke}%
\pgfsetdash{}{0pt}%
\pgfsys@defobject{currentmarker}{\pgfqpoint{-0.027778in}{0.000000in}}{\pgfqpoint{-0.000000in}{0.000000in}}{%
\pgfpathmoveto{\pgfqpoint{-0.000000in}{0.000000in}}%
\pgfpathlineto{\pgfqpoint{-0.027778in}{0.000000in}}%
\pgfusepath{stroke,fill}%
}%
\begin{pgfscope}%
\pgfsys@transformshift{8.609492in}{5.306521in}%
\pgfsys@useobject{currentmarker}{}%
\end{pgfscope}%
\end{pgfscope}%
\begin{pgfscope}%
\pgfpathrectangle{\pgfqpoint{8.609492in}{0.670138in}}{\pgfqpoint{3.067317in}{4.846490in}}%
\pgfusepath{clip}%
\pgfsetrectcap%
\pgfsetroundjoin%
\pgfsetlinewidth{1.505625pt}%
\definecolor{currentstroke}{rgb}{0.839216,0.152941,0.156863}%
\pgfsetstrokecolor{currentstroke}%
\pgfsetdash{}{0pt}%
\pgfpathmoveto{\pgfqpoint{8.639906in}{5.516628in}}%
\pgfpathlineto{\pgfqpoint{8.640591in}{5.452515in}}%
\pgfpathlineto{\pgfqpoint{8.642645in}{5.224722in}}%
\pgfpathlineto{\pgfqpoint{8.643330in}{5.195573in}}%
\pgfpathlineto{\pgfqpoint{8.643672in}{5.095553in}}%
\pgfpathlineto{\pgfqpoint{8.644699in}{5.074292in}}%
\pgfpathlineto{\pgfqpoint{8.645041in}{5.073804in}}%
\pgfpathlineto{\pgfqpoint{8.645384in}{5.072376in}}%
\pgfpathlineto{\pgfqpoint{8.645726in}{5.056253in}}%
\pgfpathlineto{\pgfqpoint{8.646069in}{5.055301in}}%
\pgfpathlineto{\pgfqpoint{8.646411in}{5.055190in}}%
\pgfpathlineto{\pgfqpoint{8.647438in}{4.995910in}}%
\pgfpathlineto{\pgfqpoint{8.648123in}{4.982463in}}%
\pgfpathlineto{\pgfqpoint{8.648807in}{4.971745in}}%
\pgfpathlineto{\pgfqpoint{8.649150in}{4.952413in}}%
\pgfpathlineto{\pgfqpoint{8.649492in}{4.951695in}}%
\pgfpathlineto{\pgfqpoint{8.650177in}{4.946237in}}%
\pgfpathlineto{\pgfqpoint{8.650861in}{4.943502in}}%
\pgfpathlineto{\pgfqpoint{8.651204in}{4.931759in}}%
\pgfpathlineto{\pgfqpoint{8.651546in}{4.931722in}}%
\pgfpathlineto{\pgfqpoint{8.652231in}{4.921753in}}%
\pgfpathlineto{\pgfqpoint{8.652573in}{4.919848in}}%
\pgfpathlineto{\pgfqpoint{8.653258in}{4.913145in}}%
\pgfpathlineto{\pgfqpoint{8.653600in}{4.912125in}}%
\pgfpathlineto{\pgfqpoint{8.653942in}{4.908236in}}%
\pgfpathlineto{\pgfqpoint{8.654285in}{4.907883in}}%
\pgfpathlineto{\pgfqpoint{8.654970in}{4.898646in}}%
\pgfpathlineto{\pgfqpoint{8.655312in}{4.897815in}}%
\pgfpathlineto{\pgfqpoint{8.655997in}{4.887167in}}%
\pgfpathlineto{\pgfqpoint{8.657024in}{4.878596in}}%
\pgfpathlineto{\pgfqpoint{8.657366in}{4.878445in}}%
\pgfpathlineto{\pgfqpoint{8.658051in}{4.867609in}}%
\pgfpathlineto{\pgfqpoint{8.658735in}{4.865268in}}%
\pgfpathlineto{\pgfqpoint{8.659420in}{4.864437in}}%
\pgfpathlineto{\pgfqpoint{8.659762in}{4.858547in}}%
\pgfpathlineto{\pgfqpoint{8.660105in}{4.858434in}}%
\pgfpathlineto{\pgfqpoint{8.660789in}{4.853903in}}%
\pgfpathlineto{\pgfqpoint{8.661474in}{4.852166in}}%
\pgfpathlineto{\pgfqpoint{8.661816in}{4.850127in}}%
\pgfpathlineto{\pgfqpoint{8.662159in}{4.850089in}}%
\pgfpathlineto{\pgfqpoint{8.662844in}{4.846430in}}%
\pgfpathlineto{\pgfqpoint{8.663871in}{4.837025in}}%
\pgfpathlineto{\pgfqpoint{8.664213in}{4.836156in}}%
\pgfpathlineto{\pgfqpoint{8.664898in}{4.817353in}}%
\pgfpathlineto{\pgfqpoint{8.665240in}{4.816900in}}%
\pgfpathlineto{\pgfqpoint{8.666267in}{4.811161in}}%
\pgfpathlineto{\pgfqpoint{8.666609in}{4.786807in}}%
\pgfpathlineto{\pgfqpoint{8.666952in}{4.784579in}}%
\pgfpathlineto{\pgfqpoint{8.668321in}{4.758979in}}%
\pgfpathlineto{\pgfqpoint{8.668663in}{4.752938in}}%
\pgfpathlineto{\pgfqpoint{8.669348in}{4.751579in}}%
\pgfpathlineto{\pgfqpoint{8.670033in}{4.737231in}}%
\pgfpathlineto{\pgfqpoint{8.671060in}{4.735230in}}%
\pgfpathlineto{\pgfqpoint{8.671402in}{4.735192in}}%
\pgfpathlineto{\pgfqpoint{8.672087in}{4.733040in}}%
\pgfpathlineto{\pgfqpoint{8.672429in}{4.729981in}}%
\pgfpathlineto{\pgfqpoint{8.672772in}{4.729755in}}%
\pgfpathlineto{\pgfqpoint{8.673456in}{4.719107in}}%
\pgfpathlineto{\pgfqpoint{8.673799in}{4.718277in}}%
\pgfpathlineto{\pgfqpoint{8.674141in}{4.715829in}}%
\pgfpathlineto{\pgfqpoint{8.674483in}{4.715558in}}%
\pgfpathlineto{\pgfqpoint{8.674826in}{4.712537in}}%
\pgfpathlineto{\pgfqpoint{8.675168in}{4.712462in}}%
\pgfpathlineto{\pgfqpoint{8.676537in}{4.697887in}}%
\pgfpathlineto{\pgfqpoint{8.676880in}{4.696415in}}%
\pgfpathlineto{\pgfqpoint{8.677564in}{4.689392in}}%
\pgfpathlineto{\pgfqpoint{8.677907in}{4.689052in}}%
\pgfpathlineto{\pgfqpoint{8.678591in}{4.687428in}}%
\pgfpathlineto{\pgfqpoint{8.679961in}{4.674138in}}%
\pgfpathlineto{\pgfqpoint{8.682015in}{4.670966in}}%
\pgfpathlineto{\pgfqpoint{8.682357in}{4.669191in}}%
\pgfpathlineto{\pgfqpoint{8.682700in}{4.669154in}}%
\pgfpathlineto{\pgfqpoint{8.683384in}{4.668398in}}%
\pgfpathlineto{\pgfqpoint{8.684069in}{4.661690in}}%
\pgfpathlineto{\pgfqpoint{8.684411in}{4.661375in}}%
\pgfpathlineto{\pgfqpoint{8.685781in}{4.653862in}}%
\pgfpathlineto{\pgfqpoint{8.686123in}{4.653409in}}%
\pgfpathlineto{\pgfqpoint{8.686465in}{4.652314in}}%
\pgfpathlineto{\pgfqpoint{8.687492in}{4.647500in}}%
\pgfpathlineto{\pgfqpoint{8.687835in}{4.647405in}}%
\pgfpathlineto{\pgfqpoint{8.688177in}{4.646650in}}%
\pgfpathlineto{\pgfqpoint{8.688520in}{4.642761in}}%
\pgfpathlineto{\pgfqpoint{8.688862in}{4.642081in}}%
\pgfpathlineto{\pgfqpoint{8.691258in}{4.630641in}}%
\pgfpathlineto{\pgfqpoint{8.691601in}{4.627998in}}%
\pgfpathlineto{\pgfqpoint{8.691943in}{4.627801in}}%
\pgfpathlineto{\pgfqpoint{8.692285in}{4.627129in}}%
\pgfpathlineto{\pgfqpoint{8.692970in}{4.619502in}}%
\pgfpathlineto{\pgfqpoint{8.693655in}{4.616519in}}%
\pgfpathlineto{\pgfqpoint{8.694339in}{4.614073in}}%
\pgfpathlineto{\pgfqpoint{8.695366in}{4.610402in}}%
\pgfpathlineto{\pgfqpoint{8.695709in}{4.609043in}}%
\pgfpathlineto{\pgfqpoint{8.696051in}{4.609005in}}%
\pgfpathlineto{\pgfqpoint{8.697078in}{4.601492in}}%
\pgfpathlineto{\pgfqpoint{8.698105in}{4.600321in}}%
\pgfpathlineto{\pgfqpoint{8.699132in}{4.595866in}}%
\pgfpathlineto{\pgfqpoint{8.699475in}{4.595375in}}%
\pgfpathlineto{\pgfqpoint{8.700159in}{4.592467in}}%
\pgfpathlineto{\pgfqpoint{8.700844in}{4.591259in}}%
\pgfpathlineto{\pgfqpoint{8.701529in}{4.590693in}}%
\pgfpathlineto{\pgfqpoint{8.701871in}{4.590391in}}%
\pgfpathlineto{\pgfqpoint{8.702898in}{4.587257in}}%
\pgfpathlineto{\pgfqpoint{8.703240in}{4.583594in}}%
\pgfpathlineto{\pgfqpoint{8.703925in}{4.574155in}}%
\pgfpathlineto{\pgfqpoint{8.704610in}{4.573400in}}%
\pgfpathlineto{\pgfqpoint{8.704952in}{4.572964in}}%
\pgfpathlineto{\pgfqpoint{8.705637in}{4.571210in}}%
\pgfpathlineto{\pgfqpoint{8.706322in}{4.568982in}}%
\pgfpathlineto{\pgfqpoint{8.707006in}{4.565471in}}%
\pgfpathlineto{\pgfqpoint{8.707349in}{4.565093in}}%
\pgfpathlineto{\pgfqpoint{8.708033in}{4.561884in}}%
\pgfpathlineto{\pgfqpoint{8.708718in}{4.561204in}}%
\pgfpathlineto{\pgfqpoint{8.709745in}{4.557957in}}%
\pgfpathlineto{\pgfqpoint{8.710772in}{4.553615in}}%
\pgfpathlineto{\pgfqpoint{8.711114in}{4.546176in}}%
\pgfpathlineto{\pgfqpoint{8.712484in}{4.542929in}}%
\pgfpathlineto{\pgfqpoint{8.713511in}{4.534887in}}%
\pgfpathlineto{\pgfqpoint{8.714538in}{4.531904in}}%
\pgfpathlineto{\pgfqpoint{8.714880in}{4.531230in}}%
\pgfpathlineto{\pgfqpoint{8.716250in}{4.525825in}}%
\pgfpathlineto{\pgfqpoint{8.716592in}{4.525477in}}%
\pgfpathlineto{\pgfqpoint{8.716934in}{4.522427in}}%
\pgfpathlineto{\pgfqpoint{8.717961in}{4.520916in}}%
\pgfpathlineto{\pgfqpoint{8.718304in}{4.519467in}}%
\pgfpathlineto{\pgfqpoint{8.718988in}{4.513214in}}%
\pgfpathlineto{\pgfqpoint{8.719673in}{4.512723in}}%
\pgfpathlineto{\pgfqpoint{8.720015in}{4.510343in}}%
\pgfpathlineto{\pgfqpoint{8.720700in}{4.509778in}}%
\pgfpathlineto{\pgfqpoint{8.721042in}{4.506266in}}%
\pgfpathlineto{\pgfqpoint{8.721727in}{4.505700in}}%
\pgfpathlineto{\pgfqpoint{8.722412in}{4.502755in}}%
\pgfpathlineto{\pgfqpoint{8.723097in}{4.502303in}}%
\pgfpathlineto{\pgfqpoint{8.723781in}{4.500225in}}%
\pgfpathlineto{\pgfqpoint{8.724124in}{4.500074in}}%
\pgfpathlineto{\pgfqpoint{8.724808in}{4.498979in}}%
\pgfpathlineto{\pgfqpoint{8.725151in}{4.498903in}}%
\pgfpathlineto{\pgfqpoint{8.726862in}{4.493466in}}%
\pgfpathlineto{\pgfqpoint{8.727889in}{4.492409in}}%
\pgfpathlineto{\pgfqpoint{8.728574in}{4.490144in}}%
\pgfpathlineto{\pgfqpoint{8.728916in}{4.487954in}}%
\pgfpathlineto{\pgfqpoint{8.729601in}{4.487576in}}%
\pgfpathlineto{\pgfqpoint{8.730286in}{4.483763in}}%
\pgfpathlineto{\pgfqpoint{8.730628in}{4.483385in}}%
\pgfpathlineto{\pgfqpoint{8.731313in}{4.480591in}}%
\pgfpathlineto{\pgfqpoint{8.731655in}{4.479987in}}%
\pgfpathlineto{\pgfqpoint{8.731998in}{4.476046in}}%
\pgfpathlineto{\pgfqpoint{8.732340in}{4.475909in}}%
\pgfpathlineto{\pgfqpoint{8.733025in}{4.474436in}}%
\pgfpathlineto{\pgfqpoint{8.733367in}{4.474399in}}%
\pgfpathlineto{\pgfqpoint{8.736106in}{4.470019in}}%
\pgfpathlineto{\pgfqpoint{8.736448in}{4.466734in}}%
\pgfpathlineto{\pgfqpoint{8.737133in}{4.465563in}}%
\pgfpathlineto{\pgfqpoint{8.738160in}{4.459749in}}%
\pgfpathlineto{\pgfqpoint{8.738502in}{4.455167in}}%
\pgfpathlineto{\pgfqpoint{8.738844in}{4.455142in}}%
\pgfpathlineto{\pgfqpoint{8.740214in}{4.449705in}}%
\pgfpathlineto{\pgfqpoint{8.740899in}{4.448912in}}%
\pgfpathlineto{\pgfqpoint{8.741583in}{4.445740in}}%
\pgfpathlineto{\pgfqpoint{8.743295in}{4.441921in}}%
\pgfpathlineto{\pgfqpoint{8.743637in}{4.441776in}}%
\pgfpathlineto{\pgfqpoint{8.744322in}{4.440001in}}%
\pgfpathlineto{\pgfqpoint{8.744664in}{4.439812in}}%
\pgfpathlineto{\pgfqpoint{8.745007in}{4.439171in}}%
\pgfpathlineto{\pgfqpoint{8.745349in}{4.435115in}}%
\pgfpathlineto{\pgfqpoint{8.746034in}{4.434338in}}%
\pgfpathlineto{\pgfqpoint{8.746718in}{4.430297in}}%
\pgfpathlineto{\pgfqpoint{8.747061in}{4.430044in}}%
\pgfpathlineto{\pgfqpoint{8.748088in}{4.427390in}}%
\pgfpathlineto{\pgfqpoint{8.748773in}{4.424558in}}%
\pgfpathlineto{\pgfqpoint{8.749115in}{4.424407in}}%
\pgfpathlineto{\pgfqpoint{8.749457in}{4.421802in}}%
\pgfpathlineto{\pgfqpoint{8.750142in}{4.421387in}}%
\pgfpathlineto{\pgfqpoint{8.750827in}{4.415761in}}%
\pgfpathlineto{\pgfqpoint{8.751169in}{4.415572in}}%
\pgfpathlineto{\pgfqpoint{8.751511in}{4.414704in}}%
\pgfpathlineto{\pgfqpoint{8.752196in}{4.414439in}}%
\pgfpathlineto{\pgfqpoint{8.753908in}{4.406132in}}%
\pgfpathlineto{\pgfqpoint{8.754250in}{4.405149in}}%
\pgfpathlineto{\pgfqpoint{8.754592in}{4.405113in}}%
\pgfpathlineto{\pgfqpoint{8.755962in}{4.403489in}}%
\pgfpathlineto{\pgfqpoint{8.756304in}{4.403489in}}%
\pgfpathlineto{\pgfqpoint{8.758016in}{4.396731in}}%
\pgfpathlineto{\pgfqpoint{8.758701in}{4.392313in}}%
\pgfpathlineto{\pgfqpoint{8.759043in}{4.390501in}}%
\pgfpathlineto{\pgfqpoint{8.759385in}{4.390425in}}%
\pgfpathlineto{\pgfqpoint{8.760070in}{4.389066in}}%
\pgfpathlineto{\pgfqpoint{8.760755in}{4.385821in}}%
\pgfpathlineto{\pgfqpoint{8.764178in}{4.374982in}}%
\pgfpathlineto{\pgfqpoint{8.764863in}{4.372728in}}%
\pgfpathlineto{\pgfqpoint{8.765205in}{4.372150in}}%
\pgfpathlineto{\pgfqpoint{8.766232in}{4.367166in}}%
\pgfpathlineto{\pgfqpoint{8.767602in}{4.364070in}}%
\pgfpathlineto{\pgfqpoint{8.768286in}{4.363315in}}%
\pgfpathlineto{\pgfqpoint{8.769313in}{4.362938in}}%
\pgfpathlineto{\pgfqpoint{8.769998in}{4.360106in}}%
\pgfpathlineto{\pgfqpoint{8.770340in}{4.358034in}}%
\pgfpathlineto{\pgfqpoint{8.771367in}{4.357161in}}%
\pgfpathlineto{\pgfqpoint{8.772394in}{4.353611in}}%
\pgfpathlineto{\pgfqpoint{8.773421in}{4.352856in}}%
\pgfpathlineto{\pgfqpoint{8.773764in}{4.351233in}}%
\pgfpathlineto{\pgfqpoint{8.774791in}{4.350364in}}%
\pgfpathlineto{\pgfqpoint{8.775476in}{4.349571in}}%
\pgfpathlineto{\pgfqpoint{8.776503in}{4.348379in}}%
\pgfpathlineto{\pgfqpoint{8.776845in}{4.346437in}}%
\pgfpathlineto{\pgfqpoint{8.777530in}{4.346173in}}%
\pgfpathlineto{\pgfqpoint{8.778899in}{4.339688in}}%
\pgfpathlineto{\pgfqpoint{8.779241in}{4.338206in}}%
\pgfpathlineto{\pgfqpoint{8.779584in}{4.338173in}}%
\pgfpathlineto{\pgfqpoint{8.780953in}{4.334732in}}%
\pgfpathlineto{\pgfqpoint{8.781980in}{4.331170in}}%
\pgfpathlineto{\pgfqpoint{8.782665in}{4.330806in}}%
\pgfpathlineto{\pgfqpoint{8.783350in}{4.330315in}}%
\pgfpathlineto{\pgfqpoint{8.783692in}{4.330315in}}%
\pgfpathlineto{\pgfqpoint{8.785061in}{4.328238in}}%
\pgfpathlineto{\pgfqpoint{8.785746in}{4.324462in}}%
\pgfpathlineto{\pgfqpoint{8.786088in}{4.323558in}}%
\pgfpathlineto{\pgfqpoint{8.786431in}{4.321781in}}%
\pgfpathlineto{\pgfqpoint{8.786773in}{4.321622in}}%
\pgfpathlineto{\pgfqpoint{8.787458in}{4.318572in}}%
\pgfpathlineto{\pgfqpoint{8.788485in}{4.317326in}}%
\pgfpathlineto{\pgfqpoint{8.788827in}{4.315627in}}%
\pgfpathlineto{\pgfqpoint{8.789169in}{4.315438in}}%
\pgfpathlineto{\pgfqpoint{8.789512in}{4.314456in}}%
\pgfpathlineto{\pgfqpoint{8.789854in}{4.310756in}}%
\pgfpathlineto{\pgfqpoint{8.790539in}{4.310492in}}%
\pgfpathlineto{\pgfqpoint{8.791224in}{4.310190in}}%
\pgfpathlineto{\pgfqpoint{8.792251in}{4.307622in}}%
\pgfpathlineto{\pgfqpoint{8.792935in}{4.305659in}}%
\pgfpathlineto{\pgfqpoint{8.793278in}{4.305432in}}%
\pgfpathlineto{\pgfqpoint{8.793620in}{4.304224in}}%
\pgfpathlineto{\pgfqpoint{8.793962in}{4.299400in}}%
\pgfpathlineto{\pgfqpoint{8.794647in}{4.299165in}}%
\pgfpathlineto{\pgfqpoint{8.795332in}{4.298825in}}%
\pgfpathlineto{\pgfqpoint{8.796016in}{4.296786in}}%
\pgfpathlineto{\pgfqpoint{8.797043in}{4.296295in}}%
\pgfpathlineto{\pgfqpoint{8.797386in}{4.295794in}}%
\pgfpathlineto{\pgfqpoint{8.797728in}{4.294218in}}%
\pgfpathlineto{\pgfqpoint{8.798070in}{4.294029in}}%
\pgfpathlineto{\pgfqpoint{8.799097in}{4.289914in}}%
\pgfpathlineto{\pgfqpoint{8.799440in}{4.289876in}}%
\pgfpathlineto{\pgfqpoint{8.799782in}{4.288979in}}%
\pgfpathlineto{\pgfqpoint{8.800125in}{4.285874in}}%
\pgfpathlineto{\pgfqpoint{8.800467in}{4.285676in}}%
\pgfpathlineto{\pgfqpoint{8.800809in}{4.283117in}}%
\pgfpathlineto{\pgfqpoint{8.801152in}{4.282966in}}%
\pgfpathlineto{\pgfqpoint{8.801494in}{4.282359in}}%
\pgfpathlineto{\pgfqpoint{8.801836in}{4.280739in}}%
\pgfpathlineto{\pgfqpoint{8.802179in}{4.280550in}}%
\pgfpathlineto{\pgfqpoint{8.803206in}{4.277529in}}%
\pgfpathlineto{\pgfqpoint{8.803890in}{4.272599in}}%
\pgfpathlineto{\pgfqpoint{8.804233in}{4.272205in}}%
\pgfpathlineto{\pgfqpoint{8.805260in}{4.269638in}}%
\pgfpathlineto{\pgfqpoint{8.805602in}{4.269492in}}%
\pgfpathlineto{\pgfqpoint{8.806629in}{4.266731in}}%
\pgfpathlineto{\pgfqpoint{8.807999in}{4.266202in}}%
\pgfpathlineto{\pgfqpoint{8.808341in}{4.265673in}}%
\pgfpathlineto{\pgfqpoint{8.808683in}{4.264390in}}%
\pgfpathlineto{\pgfqpoint{8.809026in}{4.264276in}}%
\pgfpathlineto{\pgfqpoint{8.809710in}{4.261935in}}%
\pgfpathlineto{\pgfqpoint{8.810053in}{4.261633in}}%
\pgfpathlineto{\pgfqpoint{8.811764in}{4.256862in}}%
\pgfpathlineto{\pgfqpoint{8.812449in}{4.256574in}}%
\pgfpathlineto{\pgfqpoint{8.813818in}{4.255441in}}%
\pgfpathlineto{\pgfqpoint{8.814161in}{4.253677in}}%
\pgfpathlineto{\pgfqpoint{8.814503in}{4.253440in}}%
\pgfpathlineto{\pgfqpoint{8.815188in}{4.251741in}}%
\pgfpathlineto{\pgfqpoint{8.816900in}{4.250669in}}%
\pgfpathlineto{\pgfqpoint{8.817242in}{4.250117in}}%
\pgfpathlineto{\pgfqpoint{8.817584in}{4.245662in}}%
\pgfpathlineto{\pgfqpoint{8.818269in}{4.245281in}}%
\pgfpathlineto{\pgfqpoint{8.818611in}{4.245246in}}%
\pgfpathlineto{\pgfqpoint{8.820323in}{4.243396in}}%
\pgfpathlineto{\pgfqpoint{8.820665in}{4.240602in}}%
\pgfpathlineto{\pgfqpoint{8.823062in}{4.239092in}}%
\pgfpathlineto{\pgfqpoint{8.823746in}{4.239016in}}%
\pgfpathlineto{\pgfqpoint{8.824089in}{4.238563in}}%
\pgfpathlineto{\pgfqpoint{8.824431in}{4.237279in}}%
\pgfpathlineto{\pgfqpoint{8.824773in}{4.237253in}}%
\pgfpathlineto{\pgfqpoint{8.825801in}{4.235127in}}%
\pgfpathlineto{\pgfqpoint{8.826143in}{4.231842in}}%
\pgfpathlineto{\pgfqpoint{8.827170in}{4.231238in}}%
\pgfpathlineto{\pgfqpoint{8.827855in}{4.229161in}}%
\pgfpathlineto{\pgfqpoint{8.828197in}{4.229086in}}%
\pgfpathlineto{\pgfqpoint{8.828882in}{4.223385in}}%
\pgfpathlineto{\pgfqpoint{8.829224in}{4.223120in}}%
\pgfpathlineto{\pgfqpoint{8.830593in}{4.220517in}}%
\pgfpathlineto{\pgfqpoint{8.830936in}{4.220213in}}%
\pgfpathlineto{\pgfqpoint{8.832990in}{4.216173in}}%
\pgfpathlineto{\pgfqpoint{8.834017in}{4.215871in}}%
\pgfpathlineto{\pgfqpoint{8.835044in}{4.214285in}}%
\pgfpathlineto{\pgfqpoint{8.836071in}{4.213492in}}%
\pgfpathlineto{\pgfqpoint{8.837098in}{4.212737in}}%
\pgfpathlineto{\pgfqpoint{8.837440in}{4.211642in}}%
\pgfpathlineto{\pgfqpoint{8.838125in}{4.211453in}}%
\pgfpathlineto{\pgfqpoint{8.838810in}{4.209587in}}%
\pgfpathlineto{\pgfqpoint{8.839494in}{4.209339in}}%
\pgfpathlineto{\pgfqpoint{8.840179in}{4.206401in}}%
\pgfpathlineto{\pgfqpoint{8.840521in}{4.205940in}}%
\pgfpathlineto{\pgfqpoint{8.840864in}{4.204657in}}%
\pgfpathlineto{\pgfqpoint{8.841548in}{4.201409in}}%
\pgfpathlineto{\pgfqpoint{8.842233in}{4.200994in}}%
\pgfpathlineto{\pgfqpoint{8.843603in}{4.198578in}}%
\pgfpathlineto{\pgfqpoint{8.843945in}{4.198487in}}%
\pgfpathlineto{\pgfqpoint{8.844972in}{4.194840in}}%
\pgfpathlineto{\pgfqpoint{8.845314in}{4.193966in}}%
\pgfpathlineto{\pgfqpoint{8.845657in}{4.191550in}}%
\pgfpathlineto{\pgfqpoint{8.848395in}{4.189063in}}%
\pgfpathlineto{\pgfqpoint{8.848738in}{4.186118in}}%
\pgfpathlineto{\pgfqpoint{8.850449in}{4.185136in}}%
\pgfpathlineto{\pgfqpoint{8.851134in}{4.183890in}}%
\pgfpathlineto{\pgfqpoint{8.851819in}{4.183248in}}%
\pgfpathlineto{\pgfqpoint{8.852504in}{4.180378in}}%
\pgfpathlineto{\pgfqpoint{8.852846in}{4.180265in}}%
\pgfpathlineto{\pgfqpoint{8.853188in}{4.178931in}}%
\pgfpathlineto{\pgfqpoint{8.853531in}{4.178793in}}%
\pgfpathlineto{\pgfqpoint{8.853873in}{4.176944in}}%
\pgfpathlineto{\pgfqpoint{8.854215in}{4.176866in}}%
\pgfpathlineto{\pgfqpoint{8.854900in}{4.175810in}}%
\pgfpathlineto{\pgfqpoint{8.855585in}{4.174035in}}%
\pgfpathlineto{\pgfqpoint{8.855927in}{4.172978in}}%
\pgfpathlineto{\pgfqpoint{8.856612in}{4.172789in}}%
\pgfpathlineto{\pgfqpoint{8.857296in}{4.172147in}}%
\pgfpathlineto{\pgfqpoint{8.857981in}{4.171619in}}%
\pgfpathlineto{\pgfqpoint{8.858323in}{4.171430in}}%
\pgfpathlineto{\pgfqpoint{8.858666in}{4.170561in}}%
\pgfpathlineto{\pgfqpoint{8.859693in}{4.163803in}}%
\pgfpathlineto{\pgfqpoint{8.860720in}{4.163538in}}%
\pgfpathlineto{\pgfqpoint{8.861747in}{4.159951in}}%
\pgfpathlineto{\pgfqpoint{8.862089in}{4.159838in}}%
\pgfpathlineto{\pgfqpoint{8.862432in}{4.159423in}}%
\pgfpathlineto{\pgfqpoint{8.862774in}{4.158215in}}%
\pgfpathlineto{\pgfqpoint{8.863116in}{4.158101in}}%
\pgfpathlineto{\pgfqpoint{8.863801in}{4.155464in}}%
\pgfpathlineto{\pgfqpoint{8.864828in}{4.154363in}}%
\pgfpathlineto{\pgfqpoint{8.865513in}{4.152249in}}%
\pgfpathlineto{\pgfqpoint{8.866540in}{4.151041in}}%
\pgfpathlineto{\pgfqpoint{8.867567in}{4.147625in}}%
\pgfpathlineto{\pgfqpoint{8.867909in}{4.147151in}}%
\pgfpathlineto{\pgfqpoint{8.868252in}{4.145603in}}%
\pgfpathlineto{\pgfqpoint{8.868936in}{4.145377in}}%
\pgfpathlineto{\pgfqpoint{8.869621in}{4.144774in}}%
\pgfpathlineto{\pgfqpoint{8.869963in}{4.144395in}}%
\pgfpathlineto{\pgfqpoint{8.870306in}{4.142092in}}%
\pgfpathlineto{\pgfqpoint{8.871675in}{4.141261in}}%
\pgfpathlineto{\pgfqpoint{8.872017in}{4.139129in}}%
\pgfpathlineto{\pgfqpoint{8.872360in}{4.138883in}}%
\pgfpathlineto{\pgfqpoint{8.873044in}{4.135730in}}%
\pgfpathlineto{\pgfqpoint{8.875098in}{4.133294in}}%
\pgfpathlineto{\pgfqpoint{8.875783in}{4.130916in}}%
\pgfpathlineto{\pgfqpoint{8.876126in}{4.130878in}}%
\pgfpathlineto{\pgfqpoint{8.876468in}{4.130463in}}%
\pgfpathlineto{\pgfqpoint{8.876810in}{4.129368in}}%
\pgfpathlineto{\pgfqpoint{8.877153in}{4.129217in}}%
\pgfpathlineto{\pgfqpoint{8.878864in}{4.126271in}}%
\pgfpathlineto{\pgfqpoint{8.879891in}{4.125051in}}%
\pgfpathlineto{\pgfqpoint{8.881603in}{4.119971in}}%
\pgfpathlineto{\pgfqpoint{8.882630in}{4.119362in}}%
\pgfpathlineto{\pgfqpoint{8.882972in}{4.114906in}}%
\pgfpathlineto{\pgfqpoint{8.883657in}{4.113318in}}%
\pgfpathlineto{\pgfqpoint{8.884684in}{4.110866in}}%
\pgfpathlineto{\pgfqpoint{8.885369in}{4.107727in}}%
\pgfpathlineto{\pgfqpoint{8.885711in}{4.107581in}}%
\pgfpathlineto{\pgfqpoint{8.886396in}{4.105844in}}%
\pgfpathlineto{\pgfqpoint{8.886738in}{4.105693in}}%
\pgfpathlineto{\pgfqpoint{8.888450in}{4.102333in}}%
\pgfpathlineto{\pgfqpoint{8.889477in}{4.097689in}}%
\pgfpathlineto{\pgfqpoint{8.889819in}{4.096254in}}%
\pgfpathlineto{\pgfqpoint{8.890162in}{4.096178in}}%
\pgfpathlineto{\pgfqpoint{8.890504in}{4.095046in}}%
\pgfpathlineto{\pgfqpoint{8.891531in}{4.094706in}}%
\pgfpathlineto{\pgfqpoint{8.892216in}{4.091888in}}%
\pgfpathlineto{\pgfqpoint{8.892900in}{4.091836in}}%
\pgfpathlineto{\pgfqpoint{8.894612in}{4.089760in}}%
\pgfpathlineto{\pgfqpoint{8.895982in}{4.083605in}}%
\pgfpathlineto{\pgfqpoint{8.896324in}{4.083265in}}%
\pgfpathlineto{\pgfqpoint{8.897009in}{4.081264in}}%
\pgfpathlineto{\pgfqpoint{8.897693in}{4.080831in}}%
\pgfpathlineto{\pgfqpoint{8.898036in}{4.080502in}}%
\pgfpathlineto{\pgfqpoint{8.898720in}{4.078621in}}%
\pgfpathlineto{\pgfqpoint{8.899063in}{4.078432in}}%
\pgfpathlineto{\pgfqpoint{8.899747in}{4.077753in}}%
\pgfpathlineto{\pgfqpoint{8.900090in}{4.077660in}}%
\pgfpathlineto{\pgfqpoint{8.901117in}{4.075449in}}%
\pgfpathlineto{\pgfqpoint{8.902486in}{4.074958in}}%
\pgfpathlineto{\pgfqpoint{8.903171in}{4.073826in}}%
\pgfpathlineto{\pgfqpoint{8.903513in}{4.073637in}}%
\pgfpathlineto{\pgfqpoint{8.904198in}{4.071825in}}%
\pgfpathlineto{\pgfqpoint{8.905225in}{4.069521in}}%
\pgfpathlineto{\pgfqpoint{8.905567in}{4.069333in}}%
\pgfpathlineto{\pgfqpoint{8.906252in}{4.067785in}}%
\pgfpathlineto{\pgfqpoint{8.907279in}{4.066878in}}%
\pgfpathlineto{\pgfqpoint{8.909333in}{4.065783in}}%
\pgfpathlineto{\pgfqpoint{8.910018in}{4.064839in}}%
\pgfpathlineto{\pgfqpoint{8.911045in}{4.063631in}}%
\pgfpathlineto{\pgfqpoint{8.911387in}{4.062725in}}%
\pgfpathlineto{\pgfqpoint{8.912414in}{4.062574in}}%
\pgfpathlineto{\pgfqpoint{8.912757in}{4.062272in}}%
\pgfpathlineto{\pgfqpoint{8.913099in}{4.061441in}}%
\pgfpathlineto{\pgfqpoint{8.913441in}{4.061360in}}%
\pgfpathlineto{\pgfqpoint{8.914126in}{4.060515in}}%
\pgfpathlineto{\pgfqpoint{8.914468in}{4.058194in}}%
\pgfpathlineto{\pgfqpoint{8.915153in}{4.058005in}}%
\pgfpathlineto{\pgfqpoint{8.915495in}{4.057439in}}%
\pgfpathlineto{\pgfqpoint{8.916180in}{4.052681in}}%
\pgfpathlineto{\pgfqpoint{8.917207in}{4.051705in}}%
\pgfpathlineto{\pgfqpoint{8.917549in}{4.051662in}}%
\pgfpathlineto{\pgfqpoint{8.918234in}{4.050605in}}%
\pgfpathlineto{\pgfqpoint{8.918576in}{4.049812in}}%
\pgfpathlineto{\pgfqpoint{8.919261in}{4.047546in}}%
\pgfpathlineto{\pgfqpoint{8.920288in}{4.046829in}}%
\pgfpathlineto{\pgfqpoint{8.920973in}{4.045867in}}%
\pgfpathlineto{\pgfqpoint{8.922685in}{4.044790in}}%
\pgfpathlineto{\pgfqpoint{8.923027in}{4.042525in}}%
\pgfpathlineto{\pgfqpoint{8.923369in}{4.042487in}}%
\pgfpathlineto{\pgfqpoint{8.924054in}{4.041279in}}%
\pgfpathlineto{\pgfqpoint{8.924396in}{4.038522in}}%
\pgfpathlineto{\pgfqpoint{8.925081in}{4.038296in}}%
\pgfpathlineto{\pgfqpoint{8.925766in}{4.037050in}}%
\pgfpathlineto{\pgfqpoint{8.926793in}{4.035955in}}%
\pgfpathlineto{\pgfqpoint{8.927820in}{4.032443in}}%
\pgfpathlineto{\pgfqpoint{8.928162in}{4.032405in}}%
\pgfpathlineto{\pgfqpoint{8.928847in}{4.031046in}}%
\pgfpathlineto{\pgfqpoint{8.929189in}{4.030712in}}%
\pgfpathlineto{\pgfqpoint{8.929874in}{4.027970in}}%
\pgfpathlineto{\pgfqpoint{8.930559in}{4.027446in}}%
\pgfpathlineto{\pgfqpoint{8.931243in}{4.027195in}}%
\pgfpathlineto{\pgfqpoint{8.931586in}{4.025345in}}%
\pgfpathlineto{\pgfqpoint{8.931928in}{4.025295in}}%
\pgfpathlineto{\pgfqpoint{8.932613in}{4.023598in}}%
\pgfpathlineto{\pgfqpoint{8.934667in}{4.021569in}}%
\pgfpathlineto{\pgfqpoint{8.935009in}{4.019741in}}%
\pgfpathlineto{\pgfqpoint{8.935351in}{4.019681in}}%
\pgfpathlineto{\pgfqpoint{8.936036in}{4.018473in}}%
\pgfpathlineto{\pgfqpoint{8.936721in}{4.018397in}}%
\pgfpathlineto{\pgfqpoint{8.937063in}{4.017074in}}%
\pgfpathlineto{\pgfqpoint{8.938090in}{4.016396in}}%
\pgfpathlineto{\pgfqpoint{8.938775in}{4.013602in}}%
\pgfpathlineto{\pgfqpoint{8.939117in}{4.013375in}}%
\pgfpathlineto{\pgfqpoint{8.939802in}{4.010921in}}%
\pgfpathlineto{\pgfqpoint{8.940487in}{4.010675in}}%
\pgfpathlineto{\pgfqpoint{8.941856in}{4.009184in}}%
\pgfpathlineto{\pgfqpoint{8.943225in}{4.005748in}}%
\pgfpathlineto{\pgfqpoint{8.943568in}{4.004955in}}%
\pgfpathlineto{\pgfqpoint{8.943910in}{4.004880in}}%
\pgfpathlineto{\pgfqpoint{8.944595in}{4.004276in}}%
\pgfpathlineto{\pgfqpoint{8.944937in}{4.004087in}}%
\pgfpathlineto{\pgfqpoint{8.945622in}{4.002463in}}%
\pgfpathlineto{\pgfqpoint{8.945964in}{4.001767in}}%
\pgfpathlineto{\pgfqpoint{8.946649in}{4.001557in}}%
\pgfpathlineto{\pgfqpoint{8.948018in}{3.999858in}}%
\pgfpathlineto{\pgfqpoint{8.948361in}{3.999707in}}%
\pgfpathlineto{\pgfqpoint{8.949388in}{3.998461in}}%
\pgfpathlineto{\pgfqpoint{8.950072in}{3.997970in}}%
\pgfpathlineto{\pgfqpoint{8.950415in}{3.996533in}}%
\pgfpathlineto{\pgfqpoint{8.952469in}{3.995289in}}%
\pgfpathlineto{\pgfqpoint{8.954523in}{3.990645in}}%
\pgfpathlineto{\pgfqpoint{8.954865in}{3.990456in}}%
\pgfpathlineto{\pgfqpoint{8.955550in}{3.988418in}}%
\pgfpathlineto{\pgfqpoint{8.955892in}{3.988304in}}%
\pgfpathlineto{\pgfqpoint{8.956577in}{3.987058in}}%
\pgfpathlineto{\pgfqpoint{8.956919in}{3.987020in}}%
\pgfpathlineto{\pgfqpoint{8.957262in}{3.983962in}}%
\pgfpathlineto{\pgfqpoint{8.958631in}{3.983358in}}%
\pgfpathlineto{\pgfqpoint{8.959316in}{3.980337in}}%
\pgfpathlineto{\pgfqpoint{8.960000in}{3.979167in}}%
\pgfpathlineto{\pgfqpoint{8.960685in}{3.975937in}}%
\pgfpathlineto{\pgfqpoint{8.961027in}{3.975806in}}%
\pgfpathlineto{\pgfqpoint{8.962055in}{3.974447in}}%
\pgfpathlineto{\pgfqpoint{8.963082in}{3.970860in}}%
\pgfpathlineto{\pgfqpoint{8.963766in}{3.970369in}}%
\pgfpathlineto{\pgfqpoint{8.964109in}{3.970294in}}%
\pgfpathlineto{\pgfqpoint{8.964793in}{3.969274in}}%
\pgfpathlineto{\pgfqpoint{8.966163in}{3.966896in}}%
\pgfpathlineto{\pgfqpoint{8.966505in}{3.966745in}}%
\pgfpathlineto{\pgfqpoint{8.967532in}{3.965385in}}%
\pgfpathlineto{\pgfqpoint{8.969244in}{3.959164in}}%
\pgfpathlineto{\pgfqpoint{8.970271in}{3.957783in}}%
\pgfpathlineto{\pgfqpoint{8.971640in}{3.954284in}}%
\pgfpathlineto{\pgfqpoint{8.971983in}{3.954096in}}%
\pgfpathlineto{\pgfqpoint{8.972667in}{3.952736in}}%
\pgfpathlineto{\pgfqpoint{8.973352in}{3.952359in}}%
\pgfpathlineto{\pgfqpoint{8.974379in}{3.951415in}}%
\pgfpathlineto{\pgfqpoint{8.974721in}{3.950917in}}%
\pgfpathlineto{\pgfqpoint{8.975064in}{3.950914in}}%
\pgfpathlineto{\pgfqpoint{8.975748in}{3.950207in}}%
\pgfpathlineto{\pgfqpoint{8.976091in}{3.948432in}}%
\pgfpathlineto{\pgfqpoint{8.976775in}{3.948092in}}%
\pgfpathlineto{\pgfqpoint{8.977460in}{3.945411in}}%
\pgfpathlineto{\pgfqpoint{8.979514in}{3.942678in}}%
\pgfpathlineto{\pgfqpoint{8.980199in}{3.939181in}}%
\pgfpathlineto{\pgfqpoint{8.980541in}{3.938200in}}%
\pgfpathlineto{\pgfqpoint{8.980884in}{3.938200in}}%
\pgfpathlineto{\pgfqpoint{8.981911in}{3.936576in}}%
\pgfpathlineto{\pgfqpoint{8.982595in}{3.936425in}}%
\pgfpathlineto{\pgfqpoint{8.983280in}{3.934575in}}%
\pgfpathlineto{\pgfqpoint{8.983622in}{3.933631in}}%
\pgfpathlineto{\pgfqpoint{8.984307in}{3.933518in}}%
\pgfpathlineto{\pgfqpoint{8.984649in}{3.931970in}}%
\pgfpathlineto{\pgfqpoint{8.986019in}{3.931101in}}%
\pgfpathlineto{\pgfqpoint{8.987046in}{3.930351in}}%
\pgfpathlineto{\pgfqpoint{8.987731in}{3.928798in}}%
\pgfpathlineto{\pgfqpoint{8.988415in}{3.928115in}}%
\pgfpathlineto{\pgfqpoint{8.990127in}{3.925981in}}%
\pgfpathlineto{\pgfqpoint{8.990469in}{3.925887in}}%
\pgfpathlineto{\pgfqpoint{8.991496in}{3.923587in}}%
\pgfpathlineto{\pgfqpoint{8.991839in}{3.921888in}}%
\pgfpathlineto{\pgfqpoint{8.992523in}{3.921709in}}%
\pgfpathlineto{\pgfqpoint{8.993208in}{3.920695in}}%
\pgfpathlineto{\pgfqpoint{8.993893in}{3.920387in}}%
\pgfpathlineto{\pgfqpoint{8.994577in}{3.919321in}}%
\pgfpathlineto{\pgfqpoint{8.995262in}{3.917400in}}%
\pgfpathlineto{\pgfqpoint{8.995604in}{3.917282in}}%
\pgfpathlineto{\pgfqpoint{8.996289in}{3.916362in}}%
\pgfpathlineto{\pgfqpoint{8.996632in}{3.914189in}}%
\pgfpathlineto{\pgfqpoint{8.996974in}{3.913959in}}%
\pgfpathlineto{\pgfqpoint{8.997659in}{3.912260in}}%
\pgfpathlineto{\pgfqpoint{8.998001in}{3.912147in}}%
\pgfpathlineto{\pgfqpoint{9.000055in}{3.908258in}}%
\pgfpathlineto{\pgfqpoint{9.000397in}{3.905917in}}%
\pgfpathlineto{\pgfqpoint{9.001424in}{3.905388in}}%
\pgfpathlineto{\pgfqpoint{9.002451in}{3.903236in}}%
\pgfpathlineto{\pgfqpoint{9.003136in}{3.902896in}}%
\pgfpathlineto{\pgfqpoint{9.003478in}{3.902821in}}%
\pgfpathlineto{\pgfqpoint{9.005875in}{3.899045in}}%
\pgfpathlineto{\pgfqpoint{9.006560in}{3.897459in}}%
\pgfpathlineto{\pgfqpoint{9.006902in}{3.897421in}}%
\pgfpathlineto{\pgfqpoint{9.007587in}{3.895201in}}%
\pgfpathlineto{\pgfqpoint{9.008271in}{3.894627in}}%
\pgfpathlineto{\pgfqpoint{9.008614in}{3.894325in}}%
\pgfpathlineto{\pgfqpoint{9.008956in}{3.893276in}}%
\pgfpathlineto{\pgfqpoint{9.009298in}{3.893192in}}%
\pgfpathlineto{\pgfqpoint{9.009983in}{3.891871in}}%
\pgfpathlineto{\pgfqpoint{9.010325in}{3.891418in}}%
\pgfpathlineto{\pgfqpoint{9.010668in}{3.889605in}}%
\pgfpathlineto{\pgfqpoint{9.011352in}{3.888963in}}%
\pgfpathlineto{\pgfqpoint{9.012037in}{3.887642in}}%
\pgfpathlineto{\pgfqpoint{9.013407in}{3.887415in}}%
\pgfpathlineto{\pgfqpoint{9.014091in}{3.886207in}}%
\pgfpathlineto{\pgfqpoint{9.014776in}{3.883677in}}%
\pgfpathlineto{\pgfqpoint{9.015803in}{3.882696in}}%
\pgfpathlineto{\pgfqpoint{9.016145in}{3.881603in}}%
\pgfpathlineto{\pgfqpoint{9.016488in}{3.881601in}}%
\pgfpathlineto{\pgfqpoint{9.017515in}{3.880586in}}%
\pgfpathlineto{\pgfqpoint{9.017857in}{3.880468in}}%
\pgfpathlineto{\pgfqpoint{9.018542in}{3.879222in}}%
\pgfpathlineto{\pgfqpoint{9.019569in}{3.877221in}}%
\pgfpathlineto{\pgfqpoint{9.020253in}{3.876829in}}%
\pgfpathlineto{\pgfqpoint{9.020596in}{3.876654in}}%
\pgfpathlineto{\pgfqpoint{9.021281in}{3.876088in}}%
\pgfpathlineto{\pgfqpoint{9.021965in}{3.875523in}}%
\pgfpathlineto{\pgfqpoint{9.022308in}{3.874087in}}%
\pgfpathlineto{\pgfqpoint{9.022992in}{3.873483in}}%
\pgfpathlineto{\pgfqpoint{9.024019in}{3.872237in}}%
\pgfpathlineto{\pgfqpoint{9.024362in}{3.872237in}}%
\pgfpathlineto{\pgfqpoint{9.025389in}{3.869254in}}%
\pgfpathlineto{\pgfqpoint{9.026073in}{3.868440in}}%
\pgfpathlineto{\pgfqpoint{9.027785in}{3.866762in}}%
\pgfpathlineto{\pgfqpoint{9.028127in}{3.866611in}}%
\pgfpathlineto{\pgfqpoint{9.028470in}{3.866007in}}%
\pgfpathlineto{\pgfqpoint{9.029154in}{3.864119in}}%
\pgfpathlineto{\pgfqpoint{9.029839in}{3.863273in}}%
\pgfpathlineto{\pgfqpoint{9.030524in}{3.862049in}}%
\pgfpathlineto{\pgfqpoint{9.030866in}{3.861872in}}%
\pgfpathlineto{\pgfqpoint{9.031209in}{3.861340in}}%
\pgfpathlineto{\pgfqpoint{9.031893in}{3.861287in}}%
\pgfpathlineto{\pgfqpoint{9.032578in}{3.859210in}}%
\pgfpathlineto{\pgfqpoint{9.033263in}{3.857662in}}%
\pgfpathlineto{\pgfqpoint{9.034290in}{3.857398in}}%
\pgfpathlineto{\pgfqpoint{9.035317in}{3.855925in}}%
\pgfpathlineto{\pgfqpoint{9.035659in}{3.854226in}}%
\pgfpathlineto{\pgfqpoint{9.036001in}{3.854129in}}%
\pgfpathlineto{\pgfqpoint{9.036686in}{3.852754in}}%
\pgfpathlineto{\pgfqpoint{9.037028in}{3.852248in}}%
\pgfpathlineto{\pgfqpoint{9.038740in}{3.846108in}}%
\pgfpathlineto{\pgfqpoint{9.039767in}{3.845506in}}%
\pgfpathlineto{\pgfqpoint{9.040452in}{3.845051in}}%
\pgfpathlineto{\pgfqpoint{9.041479in}{3.842446in}}%
\pgfpathlineto{\pgfqpoint{9.041821in}{3.842398in}}%
\pgfpathlineto{\pgfqpoint{9.042506in}{3.841011in}}%
\pgfpathlineto{\pgfqpoint{9.042848in}{3.840029in}}%
\pgfpathlineto{\pgfqpoint{9.044218in}{3.839727in}}%
\pgfpathlineto{\pgfqpoint{9.044560in}{3.837500in}}%
\pgfpathlineto{\pgfqpoint{9.046957in}{3.836140in}}%
\pgfpathlineto{\pgfqpoint{9.047984in}{3.833761in}}%
\pgfpathlineto{\pgfqpoint{9.048668in}{3.833459in}}%
\pgfpathlineto{\pgfqpoint{9.049353in}{3.831269in}}%
\pgfpathlineto{\pgfqpoint{9.049695in}{3.830967in}}%
\pgfpathlineto{\pgfqpoint{9.050380in}{3.830008in}}%
\pgfpathlineto{\pgfqpoint{9.053119in}{3.825077in}}%
\pgfpathlineto{\pgfqpoint{9.053803in}{3.824926in}}%
\pgfpathlineto{\pgfqpoint{9.054488in}{3.824360in}}%
\pgfpathlineto{\pgfqpoint{9.054830in}{3.821113in}}%
\pgfpathlineto{\pgfqpoint{9.055515in}{3.820508in}}%
\pgfpathlineto{\pgfqpoint{9.056200in}{3.820018in}}%
\pgfpathlineto{\pgfqpoint{9.056885in}{3.819699in}}%
\pgfpathlineto{\pgfqpoint{9.057569in}{3.819036in}}%
\pgfpathlineto{\pgfqpoint{9.058596in}{3.818281in}}%
\pgfpathlineto{\pgfqpoint{9.058939in}{3.816959in}}%
\pgfpathlineto{\pgfqpoint{9.059281in}{3.816846in}}%
\pgfpathlineto{\pgfqpoint{9.060650in}{3.814505in}}%
\pgfpathlineto{\pgfqpoint{9.061335in}{3.814241in}}%
\pgfpathlineto{\pgfqpoint{9.063731in}{3.806765in}}%
\pgfpathlineto{\pgfqpoint{9.064759in}{3.806349in}}%
\pgfpathlineto{\pgfqpoint{9.066470in}{3.804839in}}%
\pgfpathlineto{\pgfqpoint{9.066813in}{3.803607in}}%
\pgfpathlineto{\pgfqpoint{9.067497in}{3.803404in}}%
\pgfpathlineto{\pgfqpoint{9.068182in}{3.802271in}}%
\pgfpathlineto{\pgfqpoint{9.070578in}{3.800421in}}%
\pgfpathlineto{\pgfqpoint{9.071263in}{3.799272in}}%
\pgfpathlineto{\pgfqpoint{9.071948in}{3.798684in}}%
\pgfpathlineto{\pgfqpoint{9.072290in}{3.796880in}}%
\pgfpathlineto{\pgfqpoint{9.072633in}{3.796797in}}%
\pgfpathlineto{\pgfqpoint{9.074687in}{3.794909in}}%
\pgfpathlineto{\pgfqpoint{9.075371in}{3.794603in}}%
\pgfpathlineto{\pgfqpoint{9.075714in}{3.794291in}}%
\pgfpathlineto{\pgfqpoint{9.076741in}{3.790848in}}%
\pgfpathlineto{\pgfqpoint{9.077425in}{3.790148in}}%
\pgfpathlineto{\pgfqpoint{9.078110in}{3.788867in}}%
\pgfpathlineto{\pgfqpoint{9.078452in}{3.788867in}}%
\pgfpathlineto{\pgfqpoint{9.078795in}{3.788565in}}%
\pgfpathlineto{\pgfqpoint{9.079137in}{3.786677in}}%
\pgfpathlineto{\pgfqpoint{9.080164in}{3.786305in}}%
\pgfpathlineto{\pgfqpoint{9.080506in}{3.785507in}}%
\pgfpathlineto{\pgfqpoint{9.082218in}{3.784843in}}%
\pgfpathlineto{\pgfqpoint{9.084615in}{3.783544in}}%
\pgfpathlineto{\pgfqpoint{9.084957in}{3.782856in}}%
\pgfpathlineto{\pgfqpoint{9.086326in}{3.782667in}}%
\pgfpathlineto{\pgfqpoint{9.087353in}{3.781468in}}%
\pgfpathlineto{\pgfqpoint{9.088038in}{3.780900in}}%
\pgfpathlineto{\pgfqpoint{9.090435in}{3.779013in}}%
\pgfpathlineto{\pgfqpoint{9.090777in}{3.778469in}}%
\pgfpathlineto{\pgfqpoint{9.091119in}{3.778454in}}%
\pgfpathlineto{\pgfqpoint{9.092489in}{3.777422in}}%
\pgfpathlineto{\pgfqpoint{9.093173in}{3.777162in}}%
\pgfpathlineto{\pgfqpoint{9.094200in}{3.776634in}}%
\pgfpathlineto{\pgfqpoint{9.094885in}{3.776105in}}%
\pgfpathlineto{\pgfqpoint{9.095227in}{3.775803in}}%
\pgfpathlineto{\pgfqpoint{9.096254in}{3.772711in}}%
\pgfpathlineto{\pgfqpoint{9.097624in}{3.770643in}}%
\pgfpathlineto{\pgfqpoint{9.098309in}{3.769649in}}%
\pgfpathlineto{\pgfqpoint{9.098651in}{3.768724in}}%
\pgfpathlineto{\pgfqpoint{9.099336in}{3.768252in}}%
\pgfpathlineto{\pgfqpoint{9.099678in}{3.768138in}}%
\pgfpathlineto{\pgfqpoint{9.100020in}{3.766288in}}%
\pgfpathlineto{\pgfqpoint{9.101047in}{3.765986in}}%
\pgfpathlineto{\pgfqpoint{9.101732in}{3.764502in}}%
\pgfpathlineto{\pgfqpoint{9.102759in}{3.764174in}}%
\pgfpathlineto{\pgfqpoint{9.103444in}{3.763526in}}%
\pgfpathlineto{\pgfqpoint{9.109264in}{3.758892in}}%
\pgfpathlineto{\pgfqpoint{9.109606in}{3.757701in}}%
\pgfpathlineto{\pgfqpoint{9.109948in}{3.757566in}}%
\pgfpathlineto{\pgfqpoint{9.110975in}{3.756053in}}%
\pgfpathlineto{\pgfqpoint{9.111318in}{3.756019in}}%
\pgfpathlineto{\pgfqpoint{9.114056in}{3.753292in}}%
\pgfpathlineto{\pgfqpoint{9.115426in}{3.752023in}}%
\pgfpathlineto{\pgfqpoint{9.115768in}{3.751374in}}%
\pgfpathlineto{\pgfqpoint{9.116111in}{3.749561in}}%
\pgfpathlineto{\pgfqpoint{9.116795in}{3.749395in}}%
\pgfpathlineto{\pgfqpoint{9.117480in}{3.748426in}}%
\pgfpathlineto{\pgfqpoint{9.118165in}{3.747598in}}%
\pgfpathlineto{\pgfqpoint{9.119192in}{3.746616in}}%
\pgfpathlineto{\pgfqpoint{9.119534in}{3.746314in}}%
\pgfpathlineto{\pgfqpoint{9.121246in}{3.742123in}}%
\pgfpathlineto{\pgfqpoint{9.121588in}{3.741481in}}%
\pgfpathlineto{\pgfqpoint{9.122957in}{3.741330in}}%
\pgfpathlineto{\pgfqpoint{9.123642in}{3.739820in}}%
\pgfpathlineto{\pgfqpoint{9.124669in}{3.738763in}}%
\pgfpathlineto{\pgfqpoint{9.125354in}{3.738083in}}%
\pgfpathlineto{\pgfqpoint{9.126381in}{3.736799in}}%
\pgfpathlineto{\pgfqpoint{9.126723in}{3.736799in}}%
\pgfpathlineto{\pgfqpoint{9.127408in}{3.736130in}}%
\pgfpathlineto{\pgfqpoint{9.127750in}{3.736006in}}%
\pgfpathlineto{\pgfqpoint{9.128435in}{3.734534in}}%
\pgfpathlineto{\pgfqpoint{9.129462in}{3.734153in}}%
\pgfpathlineto{\pgfqpoint{9.129804in}{3.733061in}}%
\pgfpathlineto{\pgfqpoint{9.130831in}{3.732872in}}%
\pgfpathlineto{\pgfqpoint{9.131516in}{3.731815in}}%
\pgfpathlineto{\pgfqpoint{9.132201in}{3.731625in}}%
\pgfpathlineto{\pgfqpoint{9.132543in}{3.730720in}}%
\pgfpathlineto{\pgfqpoint{9.132886in}{3.730607in}}%
\pgfpathlineto{\pgfqpoint{9.133570in}{3.729690in}}%
\pgfpathlineto{\pgfqpoint{9.133913in}{3.729550in}}%
\pgfpathlineto{\pgfqpoint{9.135282in}{3.727322in}}%
\pgfpathlineto{\pgfqpoint{9.135624in}{3.727133in}}%
\pgfpathlineto{\pgfqpoint{9.136309in}{3.724792in}}%
\pgfpathlineto{\pgfqpoint{9.136994in}{3.724717in}}%
\pgfpathlineto{\pgfqpoint{9.137678in}{3.723924in}}%
\pgfpathlineto{\pgfqpoint{9.138363in}{3.723379in}}%
\pgfpathlineto{\pgfqpoint{9.138705in}{3.722867in}}%
\pgfpathlineto{\pgfqpoint{9.139390in}{3.720812in}}%
\pgfpathlineto{\pgfqpoint{9.140759in}{3.719922in}}%
\pgfpathlineto{\pgfqpoint{9.142129in}{3.717256in}}%
\pgfpathlineto{\pgfqpoint{9.144183in}{3.715192in}}%
\pgfpathlineto{\pgfqpoint{9.144525in}{3.712769in}}%
\pgfpathlineto{\pgfqpoint{9.145895in}{3.711615in}}%
\pgfpathlineto{\pgfqpoint{9.146579in}{3.710369in}}%
\pgfpathlineto{\pgfqpoint{9.146922in}{3.710188in}}%
\pgfpathlineto{\pgfqpoint{9.147949in}{3.708103in}}%
\pgfpathlineto{\pgfqpoint{9.148633in}{3.707348in}}%
\pgfpathlineto{\pgfqpoint{9.149661in}{3.707159in}}%
\pgfpathlineto{\pgfqpoint{9.150003in}{3.706555in}}%
\pgfpathlineto{\pgfqpoint{9.150688in}{3.706268in}}%
\pgfpathlineto{\pgfqpoint{9.152057in}{3.705635in}}%
\pgfpathlineto{\pgfqpoint{9.152742in}{3.704683in}}%
\pgfpathlineto{\pgfqpoint{9.154453in}{3.703874in}}%
\pgfpathlineto{\pgfqpoint{9.154796in}{3.702591in}}%
\pgfpathlineto{\pgfqpoint{9.155480in}{3.702311in}}%
\pgfpathlineto{\pgfqpoint{9.157192in}{3.698665in}}%
\pgfpathlineto{\pgfqpoint{9.157534in}{3.698664in}}%
\pgfpathlineto{\pgfqpoint{9.158219in}{3.698135in}}%
\pgfpathlineto{\pgfqpoint{9.158904in}{3.697963in}}%
\pgfpathlineto{\pgfqpoint{9.159589in}{3.695606in}}%
\pgfpathlineto{\pgfqpoint{9.160616in}{3.691981in}}%
\pgfpathlineto{\pgfqpoint{9.160958in}{3.691674in}}%
\pgfpathlineto{\pgfqpoint{9.161643in}{3.690168in}}%
\pgfpathlineto{\pgfqpoint{9.162327in}{3.690093in}}%
\pgfpathlineto{\pgfqpoint{9.163012in}{3.689338in}}%
\pgfpathlineto{\pgfqpoint{9.163697in}{3.688243in}}%
\pgfpathlineto{\pgfqpoint{9.164381in}{3.686883in}}%
\pgfpathlineto{\pgfqpoint{9.164724in}{3.686846in}}%
\pgfpathlineto{\pgfqpoint{9.165066in}{3.686512in}}%
\pgfpathlineto{\pgfqpoint{9.165751in}{3.685058in}}%
\pgfpathlineto{\pgfqpoint{9.167120in}{3.683485in}}%
\pgfpathlineto{\pgfqpoint{9.167805in}{3.682919in}}%
\pgfpathlineto{\pgfqpoint{9.168490in}{3.682617in}}%
\pgfpathlineto{\pgfqpoint{9.168832in}{3.681711in}}%
\pgfpathlineto{\pgfqpoint{9.169174in}{3.679407in}}%
\pgfpathlineto{\pgfqpoint{9.169859in}{3.679347in}}%
\pgfpathlineto{\pgfqpoint{9.172940in}{3.675692in}}%
\pgfpathlineto{\pgfqpoint{9.173625in}{3.675481in}}%
\pgfpathlineto{\pgfqpoint{9.174309in}{3.674763in}}%
\pgfpathlineto{\pgfqpoint{9.174994in}{3.672913in}}%
\pgfpathlineto{\pgfqpoint{9.177048in}{3.672161in}}%
\pgfpathlineto{\pgfqpoint{9.177733in}{3.671063in}}%
\pgfpathlineto{\pgfqpoint{9.178075in}{3.670950in}}%
\pgfpathlineto{\pgfqpoint{9.179102in}{3.669681in}}%
\pgfpathlineto{\pgfqpoint{9.179445in}{3.669590in}}%
\pgfpathlineto{\pgfqpoint{9.179787in}{3.669137in}}%
\pgfpathlineto{\pgfqpoint{9.180472in}{3.667023in}}%
\pgfpathlineto{\pgfqpoint{9.180814in}{3.666910in}}%
\pgfpathlineto{\pgfqpoint{9.182183in}{3.665513in}}%
\pgfpathlineto{\pgfqpoint{9.183210in}{3.665150in}}%
\pgfpathlineto{\pgfqpoint{9.183553in}{3.664884in}}%
\pgfpathlineto{\pgfqpoint{9.183895in}{3.663738in}}%
\pgfpathlineto{\pgfqpoint{9.184922in}{3.663436in}}%
\pgfpathlineto{\pgfqpoint{9.185265in}{3.663209in}}%
\pgfpathlineto{\pgfqpoint{9.185607in}{3.661718in}}%
\pgfpathlineto{\pgfqpoint{9.186292in}{3.661420in}}%
\pgfpathlineto{\pgfqpoint{9.186976in}{3.660438in}}%
\pgfpathlineto{\pgfqpoint{9.188003in}{3.659773in}}%
\pgfpathlineto{\pgfqpoint{9.188346in}{3.659095in}}%
\pgfpathlineto{\pgfqpoint{9.189030in}{3.658829in}}%
\pgfpathlineto{\pgfqpoint{9.189373in}{3.658746in}}%
\pgfpathlineto{\pgfqpoint{9.189715in}{3.658301in}}%
\pgfpathlineto{\pgfqpoint{9.190742in}{3.654298in}}%
\pgfpathlineto{\pgfqpoint{9.191769in}{3.653921in}}%
\pgfpathlineto{\pgfqpoint{9.192112in}{3.653732in}}%
\pgfpathlineto{\pgfqpoint{9.193481in}{3.651240in}}%
\pgfpathlineto{\pgfqpoint{9.193823in}{3.651014in}}%
\pgfpathlineto{\pgfqpoint{9.194850in}{3.648106in}}%
\pgfpathlineto{\pgfqpoint{9.195193in}{3.647880in}}%
\pgfpathlineto{\pgfqpoint{9.195877in}{3.646822in}}%
\pgfpathlineto{\pgfqpoint{9.196904in}{3.645312in}}%
\pgfpathlineto{\pgfqpoint{9.197589in}{3.644351in}}%
\pgfpathlineto{\pgfqpoint{9.197931in}{3.643651in}}%
\pgfpathlineto{\pgfqpoint{9.198958in}{3.643424in}}%
\pgfpathlineto{\pgfqpoint{9.199301in}{3.643349in}}%
\pgfpathlineto{\pgfqpoint{9.199643in}{3.642707in}}%
\pgfpathlineto{\pgfqpoint{9.201013in}{3.642442in}}%
\pgfpathlineto{\pgfqpoint{9.201355in}{3.642291in}}%
\pgfpathlineto{\pgfqpoint{9.201697in}{3.641121in}}%
\pgfpathlineto{\pgfqpoint{9.202724in}{3.640743in}}%
\pgfpathlineto{\pgfqpoint{9.203067in}{3.640479in}}%
\pgfpathlineto{\pgfqpoint{9.203751in}{3.638942in}}%
\pgfpathlineto{\pgfqpoint{9.204436in}{3.638608in}}%
\pgfpathlineto{\pgfqpoint{9.205121in}{3.637761in}}%
\pgfpathlineto{\pgfqpoint{9.206490in}{3.637380in}}%
\pgfpathlineto{\pgfqpoint{9.207175in}{3.636918in}}%
\pgfpathlineto{\pgfqpoint{9.207859in}{3.636817in}}%
\pgfpathlineto{\pgfqpoint{9.208544in}{3.635533in}}%
\pgfpathlineto{\pgfqpoint{9.208886in}{3.635042in}}%
\pgfpathlineto{\pgfqpoint{9.209229in}{3.632618in}}%
\pgfpathlineto{\pgfqpoint{9.209571in}{3.632437in}}%
\pgfpathlineto{\pgfqpoint{9.210256in}{3.631153in}}%
\pgfpathlineto{\pgfqpoint{9.210598in}{3.631053in}}%
\pgfpathlineto{\pgfqpoint{9.210941in}{3.630398in}}%
\pgfpathlineto{\pgfqpoint{9.211283in}{3.630383in}}%
\pgfpathlineto{\pgfqpoint{9.212995in}{3.627339in}}%
\pgfpathlineto{\pgfqpoint{9.213679in}{3.626772in}}%
\pgfpathlineto{\pgfqpoint{9.214706in}{3.626486in}}%
\pgfpathlineto{\pgfqpoint{9.215049in}{3.625584in}}%
\pgfpathlineto{\pgfqpoint{9.215391in}{3.625527in}}%
\pgfpathlineto{\pgfqpoint{9.217445in}{3.620543in}}%
\pgfpathlineto{\pgfqpoint{9.220526in}{3.617787in}}%
\pgfpathlineto{\pgfqpoint{9.221553in}{3.615792in}}%
\pgfpathlineto{\pgfqpoint{9.222923in}{3.614955in}}%
\pgfpathlineto{\pgfqpoint{9.223607in}{3.614162in}}%
\pgfpathlineto{\pgfqpoint{9.226004in}{3.613105in}}%
\pgfpathlineto{\pgfqpoint{9.226346in}{3.612614in}}%
\pgfpathlineto{\pgfqpoint{9.227031in}{3.610915in}}%
\pgfpathlineto{\pgfqpoint{9.228400in}{3.609899in}}%
\pgfpathlineto{\pgfqpoint{9.228743in}{3.609087in}}%
\pgfpathlineto{\pgfqpoint{9.229085in}{3.609012in}}%
\pgfpathlineto{\pgfqpoint{9.230112in}{3.607970in}}%
\pgfpathlineto{\pgfqpoint{9.231139in}{3.607819in}}%
\pgfpathlineto{\pgfqpoint{9.231824in}{3.605947in}}%
\pgfpathlineto{\pgfqpoint{9.232851in}{3.605083in}}%
\pgfpathlineto{\pgfqpoint{9.233535in}{3.604873in}}%
\pgfpathlineto{\pgfqpoint{9.234562in}{3.602986in}}%
\pgfpathlineto{\pgfqpoint{9.235247in}{3.602019in}}%
\pgfpathlineto{\pgfqpoint{9.235590in}{3.601012in}}%
\pgfpathlineto{\pgfqpoint{9.235932in}{3.600947in}}%
\pgfpathlineto{\pgfqpoint{9.236617in}{3.600380in}}%
\pgfpathlineto{\pgfqpoint{9.239013in}{3.599363in}}%
\pgfpathlineto{\pgfqpoint{9.240382in}{3.598507in}}%
\pgfpathlineto{\pgfqpoint{9.241067in}{3.597133in}}%
\pgfpathlineto{\pgfqpoint{9.241409in}{3.597117in}}%
\pgfpathlineto{\pgfqpoint{9.241752in}{3.596206in}}%
\pgfpathlineto{\pgfqpoint{9.242094in}{3.596131in}}%
\pgfpathlineto{\pgfqpoint{9.243121in}{3.595207in}}%
\pgfpathlineto{\pgfqpoint{9.245175in}{3.594566in}}%
\pgfpathlineto{\pgfqpoint{9.247229in}{3.592678in}}%
\pgfpathlineto{\pgfqpoint{9.247914in}{3.592602in}}%
\pgfpathlineto{\pgfqpoint{9.248599in}{3.591432in}}%
\pgfpathlineto{\pgfqpoint{9.248941in}{3.591243in}}%
\pgfpathlineto{\pgfqpoint{9.249626in}{3.589279in}}%
\pgfpathlineto{\pgfqpoint{9.252365in}{3.586485in}}%
\pgfpathlineto{\pgfqpoint{9.252707in}{3.586070in}}%
\pgfpathlineto{\pgfqpoint{9.253392in}{3.584673in}}%
\pgfpathlineto{\pgfqpoint{9.254076in}{3.583578in}}%
\pgfpathlineto{\pgfqpoint{9.254419in}{3.581917in}}%
\pgfpathlineto{\pgfqpoint{9.254761in}{3.581902in}}%
\pgfpathlineto{\pgfqpoint{9.257500in}{3.579047in}}%
\pgfpathlineto{\pgfqpoint{9.257842in}{3.579022in}}%
\pgfpathlineto{\pgfqpoint{9.258184in}{3.578179in}}%
\pgfpathlineto{\pgfqpoint{9.258869in}{3.577952in}}%
\pgfpathlineto{\pgfqpoint{9.259554in}{3.576676in}}%
\pgfpathlineto{\pgfqpoint{9.260238in}{3.576555in}}%
\pgfpathlineto{\pgfqpoint{9.262635in}{3.575045in}}%
\pgfpathlineto{\pgfqpoint{9.263320in}{3.574667in}}%
\pgfpathlineto{\pgfqpoint{9.263662in}{3.574589in}}%
\pgfpathlineto{\pgfqpoint{9.264689in}{3.573044in}}%
\pgfpathlineto{\pgfqpoint{9.265031in}{3.572930in}}%
\pgfpathlineto{\pgfqpoint{9.265716in}{3.572326in}}%
\pgfpathlineto{\pgfqpoint{9.266058in}{3.571193in}}%
\pgfpathlineto{\pgfqpoint{9.266743in}{3.570740in}}%
\pgfpathlineto{\pgfqpoint{9.267428in}{3.570121in}}%
\pgfpathlineto{\pgfqpoint{9.268455in}{3.569625in}}%
\pgfpathlineto{\pgfqpoint{9.269140in}{3.568298in}}%
\pgfpathlineto{\pgfqpoint{9.270167in}{3.568022in}}%
\pgfpathlineto{\pgfqpoint{9.270509in}{3.567040in}}%
\pgfpathlineto{\pgfqpoint{9.270851in}{3.566919in}}%
\pgfpathlineto{\pgfqpoint{9.271878in}{3.565265in}}%
\pgfpathlineto{\pgfqpoint{9.273590in}{3.564699in}}%
\pgfpathlineto{\pgfqpoint{9.274275in}{3.563415in}}%
\pgfpathlineto{\pgfqpoint{9.276329in}{3.562590in}}%
\pgfpathlineto{\pgfqpoint{9.278041in}{3.561008in}}%
\pgfpathlineto{\pgfqpoint{9.278383in}{3.560961in}}%
\pgfpathlineto{\pgfqpoint{9.279068in}{3.559881in}}%
\pgfpathlineto{\pgfqpoint{9.280437in}{3.558469in}}%
\pgfpathlineto{\pgfqpoint{9.281122in}{3.557525in}}%
\pgfpathlineto{\pgfqpoint{9.281464in}{3.556845in}}%
\pgfpathlineto{\pgfqpoint{9.281806in}{3.556808in}}%
\pgfpathlineto{\pgfqpoint{9.282833in}{3.556053in}}%
\pgfpathlineto{\pgfqpoint{9.287284in}{3.551861in}}%
\pgfpathlineto{\pgfqpoint{9.289338in}{3.549778in}}%
\pgfpathlineto{\pgfqpoint{9.290023in}{3.549369in}}%
\pgfpathlineto{\pgfqpoint{9.290365in}{3.548530in}}%
\pgfpathlineto{\pgfqpoint{9.290707in}{3.548526in}}%
\pgfpathlineto{\pgfqpoint{9.292761in}{3.545556in}}%
\pgfpathlineto{\pgfqpoint{9.293104in}{3.545329in}}%
\pgfpathlineto{\pgfqpoint{9.293788in}{3.544266in}}%
\pgfpathlineto{\pgfqpoint{9.294131in}{3.544113in}}%
\pgfpathlineto{\pgfqpoint{9.295158in}{3.542535in}}%
\pgfpathlineto{\pgfqpoint{9.295843in}{3.542190in}}%
\pgfpathlineto{\pgfqpoint{9.296527in}{3.540647in}}%
\pgfpathlineto{\pgfqpoint{9.299266in}{3.539129in}}%
\pgfpathlineto{\pgfqpoint{9.300293in}{3.537627in}}%
\pgfpathlineto{\pgfqpoint{9.300635in}{3.537627in}}%
\pgfpathlineto{\pgfqpoint{9.301320in}{3.537136in}}%
\pgfpathlineto{\pgfqpoint{9.301662in}{3.537048in}}%
\pgfpathlineto{\pgfqpoint{9.303032in}{3.534493in}}%
\pgfpathlineto{\pgfqpoint{9.304401in}{3.534113in}}%
\pgfpathlineto{\pgfqpoint{9.304744in}{3.533285in}}%
\pgfpathlineto{\pgfqpoint{9.305771in}{3.532907in}}%
\pgfpathlineto{\pgfqpoint{9.308167in}{3.531777in}}%
\pgfpathlineto{\pgfqpoint{9.309194in}{3.529244in}}%
\pgfpathlineto{\pgfqpoint{9.310221in}{3.528036in}}%
\pgfpathlineto{\pgfqpoint{9.310563in}{3.527112in}}%
\pgfpathlineto{\pgfqpoint{9.311248in}{3.526941in}}%
\pgfpathlineto{\pgfqpoint{9.311591in}{3.526564in}}%
\pgfpathlineto{\pgfqpoint{9.311933in}{3.525355in}}%
\pgfpathlineto{\pgfqpoint{9.312275in}{3.525242in}}%
\pgfpathlineto{\pgfqpoint{9.312618in}{3.524827in}}%
\pgfpathlineto{\pgfqpoint{9.312960in}{3.522977in}}%
\pgfpathlineto{\pgfqpoint{9.313302in}{3.522788in}}%
\pgfpathlineto{\pgfqpoint{9.313987in}{3.521542in}}%
\pgfpathlineto{\pgfqpoint{9.314329in}{3.521285in}}%
\pgfpathlineto{\pgfqpoint{9.315014in}{3.520258in}}%
\pgfpathlineto{\pgfqpoint{9.315356in}{3.520035in}}%
\pgfpathlineto{\pgfqpoint{9.316041in}{3.519125in}}%
\pgfpathlineto{\pgfqpoint{9.316383in}{3.519088in}}%
\pgfpathlineto{\pgfqpoint{9.317068in}{3.518348in}}%
\pgfpathlineto{\pgfqpoint{9.317410in}{3.518332in}}%
\pgfpathlineto{\pgfqpoint{9.318095in}{3.517540in}}%
\pgfpathlineto{\pgfqpoint{9.318437in}{3.517351in}}%
\pgfpathlineto{\pgfqpoint{9.318780in}{3.516520in}}%
\pgfpathlineto{\pgfqpoint{9.319122in}{3.516520in}}%
\pgfpathlineto{\pgfqpoint{9.319464in}{3.516120in}}%
\pgfpathlineto{\pgfqpoint{9.319807in}{3.514632in}}%
\pgfpathlineto{\pgfqpoint{9.320492in}{3.514481in}}%
\pgfpathlineto{\pgfqpoint{9.320834in}{3.513915in}}%
\pgfpathlineto{\pgfqpoint{9.321861in}{3.510161in}}%
\pgfpathlineto{\pgfqpoint{9.323230in}{3.509875in}}%
\pgfpathlineto{\pgfqpoint{9.324257in}{3.509233in}}%
\pgfpathlineto{\pgfqpoint{9.324600in}{3.507529in}}%
\pgfpathlineto{\pgfqpoint{9.324942in}{3.507394in}}%
\pgfpathlineto{\pgfqpoint{9.325627in}{3.505862in}}%
\pgfpathlineto{\pgfqpoint{9.328023in}{3.504629in}}%
\pgfpathlineto{\pgfqpoint{9.329393in}{3.504030in}}%
\pgfpathlineto{\pgfqpoint{9.331447in}{3.501077in}}%
\pgfpathlineto{\pgfqpoint{9.332474in}{3.500851in}}%
\pgfpathlineto{\pgfqpoint{9.333158in}{3.500322in}}%
\pgfpathlineto{\pgfqpoint{9.333843in}{3.500095in}}%
\pgfpathlineto{\pgfqpoint{9.334185in}{3.499265in}}%
\pgfpathlineto{\pgfqpoint{9.334528in}{3.499265in}}%
\pgfpathlineto{\pgfqpoint{9.335212in}{3.498351in}}%
\pgfpathlineto{\pgfqpoint{9.336239in}{3.497981in}}%
\pgfpathlineto{\pgfqpoint{9.336924in}{3.496697in}}%
\pgfpathlineto{\pgfqpoint{9.337609in}{3.496047in}}%
\pgfpathlineto{\pgfqpoint{9.338294in}{3.494939in}}%
\pgfpathlineto{\pgfqpoint{9.338978in}{3.493337in}}%
\pgfpathlineto{\pgfqpoint{9.340005in}{3.491826in}}%
\pgfpathlineto{\pgfqpoint{9.341375in}{3.491347in}}%
\pgfpathlineto{\pgfqpoint{9.341717in}{3.491336in}}%
\pgfpathlineto{\pgfqpoint{9.342059in}{3.489976in}}%
\pgfpathlineto{\pgfqpoint{9.342402in}{3.489825in}}%
\pgfpathlineto{\pgfqpoint{9.343086in}{3.488827in}}%
\pgfpathlineto{\pgfqpoint{9.344798in}{3.487409in}}%
\pgfpathlineto{\pgfqpoint{9.345825in}{3.486571in}}%
\pgfpathlineto{\pgfqpoint{9.346510in}{3.485974in}}%
\pgfpathlineto{\pgfqpoint{9.347195in}{3.484615in}}%
\pgfpathlineto{\pgfqpoint{9.347879in}{3.484385in}}%
\pgfpathlineto{\pgfqpoint{9.348564in}{3.483943in}}%
\pgfpathlineto{\pgfqpoint{9.348906in}{3.483860in}}%
\pgfpathlineto{\pgfqpoint{9.349591in}{3.482402in}}%
\pgfpathlineto{\pgfqpoint{9.351645in}{3.480862in}}%
\pgfpathlineto{\pgfqpoint{9.354726in}{3.476329in}}%
\pgfpathlineto{\pgfqpoint{9.355411in}{3.476006in}}%
\pgfpathlineto{\pgfqpoint{9.356438in}{3.475923in}}%
\pgfpathlineto{\pgfqpoint{9.357123in}{3.475100in}}%
\pgfpathlineto{\pgfqpoint{9.357807in}{3.475016in}}%
\pgfpathlineto{\pgfqpoint{9.358834in}{3.474458in}}%
\pgfpathlineto{\pgfqpoint{9.359519in}{3.474005in}}%
\pgfpathlineto{\pgfqpoint{9.360888in}{3.473099in}}%
\pgfpathlineto{\pgfqpoint{9.361915in}{3.471966in}}%
\pgfpathlineto{\pgfqpoint{9.362258in}{3.471928in}}%
\pgfpathlineto{\pgfqpoint{9.363285in}{3.471211in}}%
\pgfpathlineto{\pgfqpoint{9.363970in}{3.471031in}}%
\pgfpathlineto{\pgfqpoint{9.366366in}{3.467171in}}%
\pgfpathlineto{\pgfqpoint{9.366708in}{3.466982in}}%
\pgfpathlineto{\pgfqpoint{9.367393in}{3.465547in}}%
\pgfpathlineto{\pgfqpoint{9.368078in}{3.465336in}}%
\pgfpathlineto{\pgfqpoint{9.368762in}{3.464768in}}%
\pgfpathlineto{\pgfqpoint{9.369447in}{3.463763in}}%
\pgfpathlineto{\pgfqpoint{9.371501in}{3.459468in}}%
\pgfpathlineto{\pgfqpoint{9.372528in}{3.459022in}}%
\pgfpathlineto{\pgfqpoint{9.374240in}{3.458071in}}%
\pgfpathlineto{\pgfqpoint{9.374925in}{3.456515in}}%
\pgfpathlineto{\pgfqpoint{9.375952in}{3.455503in}}%
\pgfpathlineto{\pgfqpoint{9.376294in}{3.455428in}}%
\pgfpathlineto{\pgfqpoint{9.377321in}{3.454220in}}%
\pgfpathlineto{\pgfqpoint{9.378006in}{3.454144in}}%
\pgfpathlineto{\pgfqpoint{9.378690in}{3.452558in}}%
\pgfpathlineto{\pgfqpoint{9.380402in}{3.451501in}}%
\pgfpathlineto{\pgfqpoint{9.381429in}{3.450821in}}%
\pgfpathlineto{\pgfqpoint{9.381772in}{3.449953in}}%
\pgfpathlineto{\pgfqpoint{9.382456in}{3.449878in}}%
\pgfpathlineto{\pgfqpoint{9.383141in}{3.449122in}}%
\pgfpathlineto{\pgfqpoint{9.383483in}{3.449011in}}%
\pgfpathlineto{\pgfqpoint{9.385195in}{3.447234in}}%
\pgfpathlineto{\pgfqpoint{9.385537in}{3.446960in}}%
\pgfpathlineto{\pgfqpoint{9.386222in}{3.445611in}}%
\pgfpathlineto{\pgfqpoint{9.386564in}{3.445535in}}%
\pgfpathlineto{\pgfqpoint{9.386907in}{3.444289in}}%
\pgfpathlineto{\pgfqpoint{9.387249in}{3.444196in}}%
\pgfpathlineto{\pgfqpoint{9.387934in}{3.443375in}}%
\pgfpathlineto{\pgfqpoint{9.388619in}{3.442930in}}%
\pgfpathlineto{\pgfqpoint{9.388961in}{3.442855in}}%
\pgfpathlineto{\pgfqpoint{9.389646in}{3.441835in}}%
\pgfpathlineto{\pgfqpoint{9.389988in}{3.441722in}}%
\pgfpathlineto{\pgfqpoint{9.390330in}{3.441307in}}%
\pgfpathlineto{\pgfqpoint{9.390673in}{3.439872in}}%
\pgfpathlineto{\pgfqpoint{9.391357in}{3.439645in}}%
\pgfpathlineto{\pgfqpoint{9.392042in}{3.437984in}}%
\pgfpathlineto{\pgfqpoint{9.392727in}{3.437946in}}%
\pgfpathlineto{\pgfqpoint{9.395123in}{3.435303in}}%
\pgfpathlineto{\pgfqpoint{9.395465in}{3.435227in}}%
\pgfpathlineto{\pgfqpoint{9.396492in}{3.434050in}}%
\pgfpathlineto{\pgfqpoint{9.397177in}{3.433727in}}%
\pgfpathlineto{\pgfqpoint{9.397862in}{3.433491in}}%
\pgfpathlineto{\pgfqpoint{9.398889in}{3.433000in}}%
\pgfpathlineto{\pgfqpoint{9.399231in}{3.431678in}}%
\pgfpathlineto{\pgfqpoint{9.399916in}{3.431399in}}%
\pgfpathlineto{\pgfqpoint{9.401285in}{3.429035in}}%
\pgfpathlineto{\pgfqpoint{9.402312in}{3.428366in}}%
\pgfpathlineto{\pgfqpoint{9.402997in}{3.427261in}}%
\pgfpathlineto{\pgfqpoint{9.404024in}{3.426279in}}%
\pgfpathlineto{\pgfqpoint{9.405736in}{3.424391in}}%
\pgfpathlineto{\pgfqpoint{9.409159in}{3.423114in}}%
\pgfpathlineto{\pgfqpoint{9.410871in}{3.418727in}}%
\pgfpathlineto{\pgfqpoint{9.411213in}{3.418623in}}%
\pgfpathlineto{\pgfqpoint{9.411898in}{3.417838in}}%
\pgfpathlineto{\pgfqpoint{9.412240in}{3.417821in}}%
\pgfpathlineto{\pgfqpoint{9.412583in}{3.417459in}}%
\pgfpathlineto{\pgfqpoint{9.413610in}{3.415556in}}%
\pgfpathlineto{\pgfqpoint{9.413952in}{3.415465in}}%
\pgfpathlineto{\pgfqpoint{9.414295in}{3.413781in}}%
\pgfpathlineto{\pgfqpoint{9.415664in}{3.412724in}}%
\pgfpathlineto{\pgfqpoint{9.416006in}{3.412120in}}%
\pgfpathlineto{\pgfqpoint{9.416349in}{3.412120in}}%
\pgfpathlineto{\pgfqpoint{9.417033in}{3.411123in}}%
\pgfpathlineto{\pgfqpoint{9.417376in}{3.410949in}}%
\pgfpathlineto{\pgfqpoint{9.418060in}{3.409605in}}%
\pgfpathlineto{\pgfqpoint{9.418403in}{3.409439in}}%
\pgfpathlineto{\pgfqpoint{9.419087in}{3.407551in}}%
\pgfpathlineto{\pgfqpoint{9.420799in}{3.406184in}}%
\pgfpathlineto{\pgfqpoint{9.421484in}{3.405776in}}%
\pgfpathlineto{\pgfqpoint{9.421826in}{3.405547in}}%
\pgfpathlineto{\pgfqpoint{9.423196in}{3.401963in}}%
\pgfpathlineto{\pgfqpoint{9.423538in}{3.401661in}}%
\pgfpathlineto{\pgfqpoint{9.423880in}{3.400943in}}%
\pgfpathlineto{\pgfqpoint{9.424907in}{3.400648in}}%
\pgfpathlineto{\pgfqpoint{9.427646in}{3.399244in}}%
\pgfpathlineto{\pgfqpoint{9.430042in}{3.397003in}}%
\pgfpathlineto{\pgfqpoint{9.430727in}{3.396728in}}%
\pgfpathlineto{\pgfqpoint{9.432097in}{3.395619in}}%
\pgfpathlineto{\pgfqpoint{9.432781in}{3.394321in}}%
\pgfpathlineto{\pgfqpoint{9.433124in}{3.394109in}}%
\pgfpathlineto{\pgfqpoint{9.433808in}{3.392710in}}%
\pgfpathlineto{\pgfqpoint{9.435178in}{3.392184in}}%
\pgfpathlineto{\pgfqpoint{9.436547in}{3.390371in}}%
\pgfpathlineto{\pgfqpoint{9.436889in}{3.390182in}}%
\pgfpathlineto{\pgfqpoint{9.437232in}{3.389427in}}%
\pgfpathlineto{\pgfqpoint{9.438259in}{3.389125in}}%
\pgfpathlineto{\pgfqpoint{9.439286in}{3.388219in}}%
\pgfpathlineto{\pgfqpoint{9.439971in}{3.387351in}}%
\pgfpathlineto{\pgfqpoint{9.440655in}{3.386618in}}%
\pgfpathlineto{\pgfqpoint{9.441340in}{3.385614in}}%
\pgfpathlineto{\pgfqpoint{9.441682in}{3.385521in}}%
\pgfpathlineto{\pgfqpoint{9.442025in}{3.384489in}}%
\pgfpathlineto{\pgfqpoint{9.442367in}{3.384368in}}%
\pgfpathlineto{\pgfqpoint{9.443394in}{3.382895in}}%
\pgfpathlineto{\pgfqpoint{9.444079in}{3.382178in}}%
\pgfpathlineto{\pgfqpoint{9.444421in}{3.381982in}}%
\pgfpathlineto{\pgfqpoint{9.445106in}{3.380757in}}%
\pgfpathlineto{\pgfqpoint{9.445790in}{3.380252in}}%
\pgfpathlineto{\pgfqpoint{9.446133in}{3.379137in}}%
\pgfpathlineto{\pgfqpoint{9.446475in}{3.379066in}}%
\pgfpathlineto{\pgfqpoint{9.447160in}{3.377987in}}%
\pgfpathlineto{\pgfqpoint{9.447844in}{3.375948in}}%
\pgfpathlineto{\pgfqpoint{9.448529in}{3.375344in}}%
\pgfpathlineto{\pgfqpoint{9.449556in}{3.375041in}}%
\pgfpathlineto{\pgfqpoint{9.452295in}{3.373758in}}%
\pgfpathlineto{\pgfqpoint{9.452637in}{3.372247in}}%
\pgfpathlineto{\pgfqpoint{9.454691in}{3.370926in}}%
\pgfpathlineto{\pgfqpoint{9.457088in}{3.368698in}}%
\pgfpathlineto{\pgfqpoint{9.457430in}{3.368660in}}%
\pgfpathlineto{\pgfqpoint{9.458115in}{3.367943in}}%
\pgfpathlineto{\pgfqpoint{9.458800in}{3.367716in}}%
\pgfpathlineto{\pgfqpoint{9.459142in}{3.367679in}}%
\pgfpathlineto{\pgfqpoint{9.460169in}{3.365954in}}%
\pgfpathlineto{\pgfqpoint{9.461881in}{3.364167in}}%
\pgfpathlineto{\pgfqpoint{9.462565in}{3.363827in}}%
\pgfpathlineto{\pgfqpoint{9.468385in}{3.356389in}}%
\pgfpathlineto{\pgfqpoint{9.470097in}{3.355596in}}%
\pgfpathlineto{\pgfqpoint{9.470782in}{3.354237in}}%
\pgfpathlineto{\pgfqpoint{9.471466in}{3.354010in}}%
\pgfpathlineto{\pgfqpoint{9.471809in}{3.353897in}}%
\pgfpathlineto{\pgfqpoint{9.472493in}{3.353331in}}%
\pgfpathlineto{\pgfqpoint{9.473520in}{3.352394in}}%
\pgfpathlineto{\pgfqpoint{9.474205in}{3.351216in}}%
\pgfpathlineto{\pgfqpoint{9.474890in}{3.350801in}}%
\pgfpathlineto{\pgfqpoint{9.475232in}{3.349752in}}%
\pgfpathlineto{\pgfqpoint{9.476259in}{3.349291in}}%
\pgfpathlineto{\pgfqpoint{9.477971in}{3.346685in}}%
\pgfpathlineto{\pgfqpoint{9.478656in}{3.346588in}}%
\pgfpathlineto{\pgfqpoint{9.479340in}{3.345689in}}%
\pgfpathlineto{\pgfqpoint{9.481052in}{3.342336in}}%
\pgfpathlineto{\pgfqpoint{9.481394in}{3.342126in}}%
\pgfpathlineto{\pgfqpoint{9.482764in}{3.340002in}}%
\pgfpathlineto{\pgfqpoint{9.483106in}{3.339964in}}%
\pgfpathlineto{\pgfqpoint{9.483449in}{3.338228in}}%
\pgfpathlineto{\pgfqpoint{9.484133in}{3.337884in}}%
\pgfpathlineto{\pgfqpoint{9.489268in}{3.328139in}}%
\pgfpathlineto{\pgfqpoint{9.490295in}{3.327550in}}%
\pgfpathlineto{\pgfqpoint{9.490980in}{3.325805in}}%
\pgfpathlineto{\pgfqpoint{9.491323in}{3.325805in}}%
\pgfpathlineto{\pgfqpoint{9.492692in}{3.324272in}}%
\pgfpathlineto{\pgfqpoint{9.493377in}{3.324068in}}%
\pgfpathlineto{\pgfqpoint{9.494061in}{3.323064in}}%
\pgfpathlineto{\pgfqpoint{9.494404in}{3.322762in}}%
\pgfpathlineto{\pgfqpoint{9.494746in}{3.322046in}}%
\pgfpathlineto{\pgfqpoint{9.495088in}{3.322029in}}%
\pgfpathlineto{\pgfqpoint{9.496115in}{3.320390in}}%
\pgfpathlineto{\pgfqpoint{9.497142in}{3.319235in}}%
\pgfpathlineto{\pgfqpoint{9.497827in}{3.317264in}}%
\pgfpathlineto{\pgfqpoint{9.498169in}{3.317159in}}%
\pgfpathlineto{\pgfqpoint{9.498512in}{3.315865in}}%
\pgfpathlineto{\pgfqpoint{9.498854in}{3.315837in}}%
\pgfpathlineto{\pgfqpoint{9.499539in}{3.313232in}}%
\pgfpathlineto{\pgfqpoint{9.500566in}{3.312779in}}%
\pgfpathlineto{\pgfqpoint{9.500908in}{3.311873in}}%
\pgfpathlineto{\pgfqpoint{9.501935in}{3.311382in}}%
\pgfpathlineto{\pgfqpoint{9.502620in}{3.310977in}}%
\pgfpathlineto{\pgfqpoint{9.503647in}{3.310575in}}%
\pgfpathlineto{\pgfqpoint{9.504674in}{3.309871in}}%
\pgfpathlineto{\pgfqpoint{9.505016in}{3.309683in}}%
\pgfpathlineto{\pgfqpoint{9.505701in}{3.307719in}}%
\pgfpathlineto{\pgfqpoint{9.506386in}{3.307190in}}%
\pgfpathlineto{\pgfqpoint{9.507070in}{3.306537in}}%
\pgfpathlineto{\pgfqpoint{9.507755in}{3.304963in}}%
\pgfpathlineto{\pgfqpoint{9.508440in}{3.303264in}}%
\pgfpathlineto{\pgfqpoint{9.509125in}{3.302645in}}%
\pgfpathlineto{\pgfqpoint{9.510836in}{3.300885in}}%
\pgfpathlineto{\pgfqpoint{9.511863in}{3.300711in}}%
\pgfpathlineto{\pgfqpoint{9.512548in}{3.300409in}}%
\pgfpathlineto{\pgfqpoint{9.513575in}{3.300319in}}%
\pgfpathlineto{\pgfqpoint{9.513917in}{3.299450in}}%
\pgfpathlineto{\pgfqpoint{9.514602in}{3.299133in}}%
\pgfpathlineto{\pgfqpoint{9.515287in}{3.297597in}}%
\pgfpathlineto{\pgfqpoint{9.515629in}{3.297336in}}%
\pgfpathlineto{\pgfqpoint{9.517341in}{3.293787in}}%
\pgfpathlineto{\pgfqpoint{9.518026in}{3.293583in}}%
\pgfpathlineto{\pgfqpoint{9.518710in}{3.292314in}}%
\pgfpathlineto{\pgfqpoint{9.519053in}{3.292314in}}%
\pgfpathlineto{\pgfqpoint{9.519737in}{3.290993in}}%
\pgfpathlineto{\pgfqpoint{9.521107in}{3.288236in}}%
\pgfpathlineto{\pgfqpoint{9.521449in}{3.288010in}}%
\pgfpathlineto{\pgfqpoint{9.522134in}{3.287005in}}%
\pgfpathlineto{\pgfqpoint{9.522476in}{3.286792in}}%
\pgfpathlineto{\pgfqpoint{9.523161in}{3.285178in}}%
\pgfpathlineto{\pgfqpoint{9.523845in}{3.284876in}}%
\pgfpathlineto{\pgfqpoint{9.524530in}{3.283798in}}%
\pgfpathlineto{\pgfqpoint{9.525215in}{3.283554in}}%
\pgfpathlineto{\pgfqpoint{9.525900in}{3.282799in}}%
\pgfpathlineto{\pgfqpoint{9.526584in}{3.281620in}}%
\pgfpathlineto{\pgfqpoint{9.527269in}{3.281553in}}%
\pgfpathlineto{\pgfqpoint{9.527611in}{3.280805in}}%
\pgfpathlineto{\pgfqpoint{9.527954in}{3.280790in}}%
\pgfpathlineto{\pgfqpoint{9.528638in}{3.279401in}}%
\pgfpathlineto{\pgfqpoint{9.530692in}{3.277475in}}%
\pgfpathlineto{\pgfqpoint{9.531035in}{3.276078in}}%
\pgfpathlineto{\pgfqpoint{9.533089in}{3.274837in}}%
\pgfpathlineto{\pgfqpoint{9.533774in}{3.273734in}}%
\pgfpathlineto{\pgfqpoint{9.534458in}{3.273406in}}%
\pgfpathlineto{\pgfqpoint{9.535143in}{3.273017in}}%
\pgfpathlineto{\pgfqpoint{9.535485in}{3.272869in}}%
\pgfpathlineto{\pgfqpoint{9.537197in}{3.270324in}}%
\pgfpathlineto{\pgfqpoint{9.538224in}{3.269992in}}%
\pgfpathlineto{\pgfqpoint{9.539251in}{3.269169in}}%
\pgfpathlineto{\pgfqpoint{9.539593in}{3.267771in}}%
\pgfpathlineto{\pgfqpoint{9.540278in}{3.267620in}}%
\pgfpathlineto{\pgfqpoint{9.540963in}{3.267055in}}%
\pgfpathlineto{\pgfqpoint{9.542332in}{3.265808in}}%
\pgfpathlineto{\pgfqpoint{9.543017in}{3.264222in}}%
\pgfpathlineto{\pgfqpoint{9.544044in}{3.263618in}}%
\pgfpathlineto{\pgfqpoint{9.544386in}{3.263014in}}%
\pgfpathlineto{\pgfqpoint{9.545071in}{3.262874in}}%
\pgfpathlineto{\pgfqpoint{9.546098in}{3.261579in}}%
\pgfpathlineto{\pgfqpoint{9.546440in}{3.261544in}}%
\pgfpathlineto{\pgfqpoint{9.546783in}{3.259903in}}%
\pgfpathlineto{\pgfqpoint{9.547125in}{3.259691in}}%
\pgfpathlineto{\pgfqpoint{9.547810in}{3.258867in}}%
\pgfpathlineto{\pgfqpoint{9.548494in}{3.258483in}}%
\pgfpathlineto{\pgfqpoint{9.548837in}{3.257237in}}%
\pgfpathlineto{\pgfqpoint{9.549179in}{3.257199in}}%
\pgfpathlineto{\pgfqpoint{9.550548in}{3.255527in}}%
\pgfpathlineto{\pgfqpoint{9.550891in}{3.254254in}}%
\pgfpathlineto{\pgfqpoint{9.551233in}{3.254216in}}%
\pgfpathlineto{\pgfqpoint{9.551918in}{3.253726in}}%
\pgfpathlineto{\pgfqpoint{9.552260in}{3.253669in}}%
\pgfpathlineto{\pgfqpoint{9.552945in}{3.253197in}}%
\pgfpathlineto{\pgfqpoint{9.553287in}{3.253046in}}%
\pgfpathlineto{\pgfqpoint{9.553972in}{3.250911in}}%
\pgfpathlineto{\pgfqpoint{9.554314in}{3.250232in}}%
\pgfpathlineto{\pgfqpoint{9.554657in}{3.250176in}}%
\pgfpathlineto{\pgfqpoint{9.555341in}{3.249723in}}%
\pgfpathlineto{\pgfqpoint{9.559450in}{3.247269in}}%
\pgfpathlineto{\pgfqpoint{9.561846in}{3.243516in}}%
\pgfpathlineto{\pgfqpoint{9.565269in}{3.241658in}}%
\pgfpathlineto{\pgfqpoint{9.566296in}{3.239491in}}%
\pgfpathlineto{\pgfqpoint{9.567323in}{3.239113in}}%
\pgfpathlineto{\pgfqpoint{9.568008in}{3.238509in}}%
\pgfpathlineto{\pgfqpoint{9.568693in}{3.237867in}}%
\pgfpathlineto{\pgfqpoint{9.569378in}{3.236795in}}%
\pgfpathlineto{\pgfqpoint{9.570062in}{3.236513in}}%
\pgfpathlineto{\pgfqpoint{9.570747in}{3.235904in}}%
\pgfpathlineto{\pgfqpoint{9.571432in}{3.233261in}}%
\pgfpathlineto{\pgfqpoint{9.572459in}{3.231855in}}%
\pgfpathlineto{\pgfqpoint{9.572801in}{3.230769in}}%
\pgfpathlineto{\pgfqpoint{9.573143in}{3.230618in}}%
\pgfpathlineto{\pgfqpoint{9.574170in}{3.228201in}}%
\pgfpathlineto{\pgfqpoint{9.574513in}{3.228088in}}%
\pgfpathlineto{\pgfqpoint{9.574855in}{3.227069in}}%
\pgfpathlineto{\pgfqpoint{9.576567in}{3.226578in}}%
\pgfpathlineto{\pgfqpoint{9.577252in}{3.225582in}}%
\pgfpathlineto{\pgfqpoint{9.577594in}{3.225445in}}%
\pgfpathlineto{\pgfqpoint{9.578621in}{3.224274in}}%
\pgfpathlineto{\pgfqpoint{9.579990in}{3.223935in}}%
\pgfpathlineto{\pgfqpoint{9.580675in}{3.222991in}}%
\pgfpathlineto{\pgfqpoint{9.581702in}{3.219819in}}%
\pgfpathlineto{\pgfqpoint{9.582387in}{3.219479in}}%
\pgfpathlineto{\pgfqpoint{9.582729in}{3.218762in}}%
\pgfpathlineto{\pgfqpoint{9.583414in}{3.218732in}}%
\pgfpathlineto{\pgfqpoint{9.585126in}{3.216708in}}%
\pgfpathlineto{\pgfqpoint{9.586153in}{3.215288in}}%
\pgfpathlineto{\pgfqpoint{9.586495in}{3.215024in}}%
\pgfpathlineto{\pgfqpoint{9.586837in}{3.214042in}}%
\pgfpathlineto{\pgfqpoint{9.587180in}{3.211021in}}%
\pgfpathlineto{\pgfqpoint{9.587864in}{3.210830in}}%
\pgfpathlineto{\pgfqpoint{9.588549in}{3.210028in}}%
\pgfpathlineto{\pgfqpoint{9.589234in}{3.209473in}}%
\pgfpathlineto{\pgfqpoint{9.589918in}{3.208879in}}%
\pgfpathlineto{\pgfqpoint{9.590603in}{3.206075in}}%
\pgfpathlineto{\pgfqpoint{9.590945in}{3.205849in}}%
\pgfpathlineto{\pgfqpoint{9.591630in}{3.204905in}}%
\pgfpathlineto{\pgfqpoint{9.592315in}{3.204301in}}%
\pgfpathlineto{\pgfqpoint{9.592657in}{3.204142in}}%
\pgfpathlineto{\pgfqpoint{9.593342in}{3.203545in}}%
\pgfpathlineto{\pgfqpoint{9.594027in}{3.203470in}}%
\pgfpathlineto{\pgfqpoint{9.595054in}{3.202429in}}%
\pgfpathlineto{\pgfqpoint{9.596765in}{3.201393in}}%
\pgfpathlineto{\pgfqpoint{9.597108in}{3.199770in}}%
\pgfpathlineto{\pgfqpoint{9.597450in}{3.199688in}}%
\pgfpathlineto{\pgfqpoint{9.598135in}{3.197472in}}%
\pgfpathlineto{\pgfqpoint{9.598819in}{3.197247in}}%
\pgfpathlineto{\pgfqpoint{9.599504in}{3.196598in}}%
\pgfpathlineto{\pgfqpoint{9.599846in}{3.196490in}}%
\pgfpathlineto{\pgfqpoint{9.600873in}{3.195465in}}%
\pgfpathlineto{\pgfqpoint{9.601216in}{3.195427in}}%
\pgfpathlineto{\pgfqpoint{9.601558in}{3.194257in}}%
\pgfpathlineto{\pgfqpoint{9.602243in}{3.194083in}}%
\pgfpathlineto{\pgfqpoint{9.602928in}{3.191123in}}%
\pgfpathlineto{\pgfqpoint{9.603955in}{3.190783in}}%
\pgfpathlineto{\pgfqpoint{9.604639in}{3.189726in}}%
\pgfpathlineto{\pgfqpoint{9.604982in}{3.188163in}}%
\pgfpathlineto{\pgfqpoint{9.607036in}{3.187385in}}%
\pgfpathlineto{\pgfqpoint{9.609432in}{3.182892in}}%
\pgfpathlineto{\pgfqpoint{9.609774in}{3.182854in}}%
\pgfpathlineto{\pgfqpoint{9.610459in}{3.181864in}}%
\pgfpathlineto{\pgfqpoint{9.616964in}{3.175235in}}%
\pgfpathlineto{\pgfqpoint{9.617648in}{3.175151in}}%
\pgfpathlineto{\pgfqpoint{9.621072in}{3.170009in}}%
\pgfpathlineto{\pgfqpoint{9.621414in}{3.169545in}}%
\pgfpathlineto{\pgfqpoint{9.622099in}{3.165962in}}%
\pgfpathlineto{\pgfqpoint{9.623126in}{3.165599in}}%
\pgfpathlineto{\pgfqpoint{9.623468in}{3.164919in}}%
\pgfpathlineto{\pgfqpoint{9.624495in}{3.164662in}}%
\pgfpathlineto{\pgfqpoint{9.625522in}{3.164315in}}%
\pgfpathlineto{\pgfqpoint{9.626207in}{3.164164in}}%
\pgfpathlineto{\pgfqpoint{9.626892in}{3.163598in}}%
\pgfpathlineto{\pgfqpoint{9.627234in}{3.163527in}}%
\pgfpathlineto{\pgfqpoint{9.627577in}{3.163145in}}%
\pgfpathlineto{\pgfqpoint{9.628261in}{3.162058in}}%
\pgfpathlineto{\pgfqpoint{9.628946in}{3.160728in}}%
\pgfpathlineto{\pgfqpoint{9.629288in}{3.160652in}}%
\pgfpathlineto{\pgfqpoint{9.629973in}{3.159938in}}%
\pgfpathlineto{\pgfqpoint{9.630658in}{3.157745in}}%
\pgfpathlineto{\pgfqpoint{9.632027in}{3.157323in}}%
\pgfpathlineto{\pgfqpoint{9.632712in}{3.156801in}}%
\pgfpathlineto{\pgfqpoint{9.633396in}{3.155178in}}%
\pgfpathlineto{\pgfqpoint{9.634081in}{3.154996in}}%
\pgfpathlineto{\pgfqpoint{9.634423in}{3.154309in}}%
\pgfpathlineto{\pgfqpoint{9.634766in}{3.152265in}}%
\pgfpathlineto{\pgfqpoint{9.635793in}{3.151692in}}%
\pgfpathlineto{\pgfqpoint{9.637505in}{3.148155in}}%
\pgfpathlineto{\pgfqpoint{9.638189in}{3.146536in}}%
\pgfpathlineto{\pgfqpoint{9.639216in}{3.146055in}}%
\pgfpathlineto{\pgfqpoint{9.639559in}{3.146002in}}%
\pgfpathlineto{\pgfqpoint{9.640243in}{3.145625in}}%
\pgfpathlineto{\pgfqpoint{9.640586in}{3.145501in}}%
\pgfpathlineto{\pgfqpoint{9.641270in}{3.144763in}}%
\pgfpathlineto{\pgfqpoint{9.641613in}{3.144605in}}%
\pgfpathlineto{\pgfqpoint{9.643324in}{3.142004in}}%
\pgfpathlineto{\pgfqpoint{9.643667in}{3.141962in}}%
\pgfpathlineto{\pgfqpoint{9.644009in}{3.141132in}}%
\pgfpathlineto{\pgfqpoint{9.644694in}{3.140830in}}%
\pgfpathlineto{\pgfqpoint{9.645036in}{3.140739in}}%
\pgfpathlineto{\pgfqpoint{9.646406in}{3.138791in}}%
\pgfpathlineto{\pgfqpoint{9.647090in}{3.138300in}}%
\pgfpathlineto{\pgfqpoint{9.647775in}{3.136110in}}%
\pgfpathlineto{\pgfqpoint{9.648460in}{3.135845in}}%
\pgfpathlineto{\pgfqpoint{9.649144in}{3.134436in}}%
\pgfpathlineto{\pgfqpoint{9.649829in}{3.133657in}}%
\pgfpathlineto{\pgfqpoint{9.650856in}{3.133315in}}%
\pgfpathlineto{\pgfqpoint{9.651198in}{3.132976in}}%
\pgfpathlineto{\pgfqpoint{9.651541in}{3.131713in}}%
\pgfpathlineto{\pgfqpoint{9.652225in}{3.131587in}}%
\pgfpathlineto{\pgfqpoint{9.652910in}{3.130635in}}%
\pgfpathlineto{\pgfqpoint{9.654280in}{3.129729in}}%
\pgfpathlineto{\pgfqpoint{9.655649in}{3.128636in}}%
\pgfpathlineto{\pgfqpoint{9.656334in}{3.127599in}}%
\pgfpathlineto{\pgfqpoint{9.656676in}{3.127388in}}%
\pgfpathlineto{\pgfqpoint{9.657361in}{3.126029in}}%
\pgfpathlineto{\pgfqpoint{9.657703in}{3.125991in}}%
\pgfpathlineto{\pgfqpoint{9.658045in}{3.123083in}}%
\pgfpathlineto{\pgfqpoint{9.658388in}{3.123072in}}%
\pgfpathlineto{\pgfqpoint{9.658730in}{3.122632in}}%
\pgfpathlineto{\pgfqpoint{9.659072in}{3.121686in}}%
\pgfpathlineto{\pgfqpoint{9.659415in}{3.121686in}}%
\pgfpathlineto{\pgfqpoint{9.660099in}{3.120063in}}%
\pgfpathlineto{\pgfqpoint{9.660784in}{3.119762in}}%
\pgfpathlineto{\pgfqpoint{9.661126in}{3.119194in}}%
\pgfpathlineto{\pgfqpoint{9.661469in}{3.119172in}}%
\pgfpathlineto{\pgfqpoint{9.661811in}{3.118666in}}%
\pgfpathlineto{\pgfqpoint{9.662154in}{3.118666in}}%
\pgfpathlineto{\pgfqpoint{9.663181in}{3.117517in}}%
\pgfpathlineto{\pgfqpoint{9.663523in}{3.117359in}}%
\pgfpathlineto{\pgfqpoint{9.664208in}{3.116174in}}%
\pgfpathlineto{\pgfqpoint{9.665577in}{3.114966in}}%
\pgfpathlineto{\pgfqpoint{9.665919in}{3.114042in}}%
\pgfpathlineto{\pgfqpoint{9.666604in}{3.113908in}}%
\pgfpathlineto{\pgfqpoint{9.667289in}{3.112342in}}%
\pgfpathlineto{\pgfqpoint{9.667631in}{3.111983in}}%
\pgfpathlineto{\pgfqpoint{9.668316in}{3.109743in}}%
\pgfpathlineto{\pgfqpoint{9.669000in}{3.109189in}}%
\pgfpathlineto{\pgfqpoint{9.670370in}{3.108169in}}%
\pgfpathlineto{\pgfqpoint{9.670712in}{3.107225in}}%
\pgfpathlineto{\pgfqpoint{9.671397in}{3.107119in}}%
\pgfpathlineto{\pgfqpoint{9.675163in}{3.103253in}}%
\pgfpathlineto{\pgfqpoint{9.677901in}{3.101675in}}%
\pgfpathlineto{\pgfqpoint{9.678929in}{3.097484in}}%
\pgfpathlineto{\pgfqpoint{9.679613in}{3.095641in}}%
\pgfpathlineto{\pgfqpoint{9.681325in}{3.094425in}}%
\pgfpathlineto{\pgfqpoint{9.682010in}{3.092468in}}%
\pgfpathlineto{\pgfqpoint{9.683037in}{3.091858in}}%
\pgfpathlineto{\pgfqpoint{9.683721in}{3.090199in}}%
\pgfpathlineto{\pgfqpoint{9.685091in}{3.089026in}}%
\pgfpathlineto{\pgfqpoint{9.685775in}{3.088120in}}%
\pgfpathlineto{\pgfqpoint{9.686118in}{3.088044in}}%
\pgfpathlineto{\pgfqpoint{9.686802in}{3.087440in}}%
\pgfpathlineto{\pgfqpoint{9.687145in}{3.087402in}}%
\pgfpathlineto{\pgfqpoint{9.687830in}{3.086723in}}%
\pgfpathlineto{\pgfqpoint{9.688172in}{3.085286in}}%
\pgfpathlineto{\pgfqpoint{9.689541in}{3.084495in}}%
\pgfpathlineto{\pgfqpoint{9.690226in}{3.083891in}}%
\pgfpathlineto{\pgfqpoint{9.690568in}{3.083853in}}%
\pgfpathlineto{\pgfqpoint{9.690911in}{3.081588in}}%
\pgfpathlineto{\pgfqpoint{9.691253in}{3.081383in}}%
\pgfpathlineto{\pgfqpoint{9.691938in}{3.079390in}}%
\pgfpathlineto{\pgfqpoint{9.692280in}{3.079096in}}%
\pgfpathlineto{\pgfqpoint{9.692622in}{3.077321in}}%
\pgfpathlineto{\pgfqpoint{9.692965in}{3.077167in}}%
\pgfpathlineto{\pgfqpoint{9.693307in}{3.076320in}}%
\pgfpathlineto{\pgfqpoint{9.693649in}{3.076218in}}%
\pgfpathlineto{\pgfqpoint{9.695361in}{3.072261in}}%
\pgfpathlineto{\pgfqpoint{9.695703in}{3.072224in}}%
\pgfpathlineto{\pgfqpoint{9.696388in}{3.070449in}}%
\pgfpathlineto{\pgfqpoint{9.697073in}{3.070017in}}%
\pgfpathlineto{\pgfqpoint{9.698100in}{3.068002in}}%
\pgfpathlineto{\pgfqpoint{9.698785in}{3.067315in}}%
\pgfpathlineto{\pgfqpoint{9.699469in}{3.066409in}}%
\pgfpathlineto{\pgfqpoint{9.699812in}{3.065676in}}%
\pgfpathlineto{\pgfqpoint{9.700154in}{3.065616in}}%
\pgfpathlineto{\pgfqpoint{9.700839in}{3.064559in}}%
\pgfpathlineto{\pgfqpoint{9.701866in}{3.064294in}}%
\pgfpathlineto{\pgfqpoint{9.702550in}{3.063381in}}%
\pgfpathlineto{\pgfqpoint{9.703235in}{3.062709in}}%
\pgfpathlineto{\pgfqpoint{9.703577in}{3.061463in}}%
\pgfpathlineto{\pgfqpoint{9.704947in}{3.060821in}}%
\pgfpathlineto{\pgfqpoint{9.707343in}{3.057581in}}%
\pgfpathlineto{\pgfqpoint{9.708713in}{3.056214in}}%
\pgfpathlineto{\pgfqpoint{9.709055in}{3.056177in}}%
\pgfpathlineto{\pgfqpoint{9.709740in}{3.055542in}}%
\pgfpathlineto{\pgfqpoint{9.710767in}{3.054440in}}%
\pgfpathlineto{\pgfqpoint{9.711451in}{3.054364in}}%
\pgfpathlineto{\pgfqpoint{9.712136in}{3.053005in}}%
\pgfpathlineto{\pgfqpoint{9.713506in}{3.051812in}}%
\pgfpathlineto{\pgfqpoint{9.713848in}{3.051747in}}%
\pgfpathlineto{\pgfqpoint{9.715902in}{3.048972in}}%
\pgfpathlineto{\pgfqpoint{9.716587in}{3.046616in}}%
\pgfpathlineto{\pgfqpoint{9.717271in}{3.044962in}}%
\pgfpathlineto{\pgfqpoint{9.717614in}{3.044925in}}%
\pgfpathlineto{\pgfqpoint{9.718298in}{3.043528in}}%
\pgfpathlineto{\pgfqpoint{9.719325in}{3.043037in}}%
\pgfpathlineto{\pgfqpoint{9.720010in}{3.041317in}}%
\pgfpathlineto{\pgfqpoint{9.721037in}{3.040545in}}%
\pgfpathlineto{\pgfqpoint{9.721722in}{3.038846in}}%
\pgfpathlineto{\pgfqpoint{9.722407in}{3.038695in}}%
\pgfpathlineto{\pgfqpoint{9.722749in}{3.038331in}}%
\pgfpathlineto{\pgfqpoint{9.723776in}{3.035825in}}%
\pgfpathlineto{\pgfqpoint{9.724461in}{3.035372in}}%
\pgfpathlineto{\pgfqpoint{9.725830in}{3.034258in}}%
\pgfpathlineto{\pgfqpoint{9.726515in}{3.032087in}}%
\pgfpathlineto{\pgfqpoint{9.727199in}{3.029912in}}%
\pgfpathlineto{\pgfqpoint{9.728226in}{3.029406in}}%
\pgfpathlineto{\pgfqpoint{9.728569in}{3.029368in}}%
\pgfpathlineto{\pgfqpoint{9.729253in}{3.028840in}}%
\pgfpathlineto{\pgfqpoint{9.729938in}{3.028500in}}%
\pgfpathlineto{\pgfqpoint{9.731992in}{3.027224in}}%
\pgfpathlineto{\pgfqpoint{9.732677in}{3.027076in}}%
\pgfpathlineto{\pgfqpoint{9.733704in}{3.025291in}}%
\pgfpathlineto{\pgfqpoint{9.734046in}{3.025185in}}%
\pgfpathlineto{\pgfqpoint{9.734731in}{3.024519in}}%
\pgfpathlineto{\pgfqpoint{9.735073in}{3.024254in}}%
\pgfpathlineto{\pgfqpoint{9.736100in}{3.021553in}}%
\pgfpathlineto{\pgfqpoint{9.736785in}{3.021024in}}%
\pgfpathlineto{\pgfqpoint{9.737127in}{3.019601in}}%
\pgfpathlineto{\pgfqpoint{9.737812in}{3.019394in}}%
\pgfpathlineto{\pgfqpoint{9.738497in}{3.018985in}}%
\pgfpathlineto{\pgfqpoint{9.738839in}{3.018910in}}%
\pgfpathlineto{\pgfqpoint{9.739182in}{3.018292in}}%
\pgfpathlineto{\pgfqpoint{9.739866in}{3.018278in}}%
\pgfpathlineto{\pgfqpoint{9.740551in}{3.015927in}}%
\pgfpathlineto{\pgfqpoint{9.740893in}{3.015866in}}%
\pgfpathlineto{\pgfqpoint{9.741578in}{3.014978in}}%
\pgfpathlineto{\pgfqpoint{9.742605in}{3.014357in}}%
\pgfpathlineto{\pgfqpoint{9.746028in}{3.009848in}}%
\pgfpathlineto{\pgfqpoint{9.747056in}{3.009032in}}%
\pgfpathlineto{\pgfqpoint{9.748083in}{3.008413in}}%
\pgfpathlineto{\pgfqpoint{9.748425in}{3.007529in}}%
\pgfpathlineto{\pgfqpoint{9.749452in}{3.007299in}}%
\pgfpathlineto{\pgfqpoint{9.750479in}{3.005898in}}%
\pgfpathlineto{\pgfqpoint{9.750821in}{3.005767in}}%
\pgfpathlineto{\pgfqpoint{9.751506in}{3.005181in}}%
\pgfpathlineto{\pgfqpoint{9.752875in}{3.004675in}}%
\pgfpathlineto{\pgfqpoint{9.753560in}{3.002032in}}%
\pgfpathlineto{\pgfqpoint{9.754245in}{3.001692in}}%
\pgfpathlineto{\pgfqpoint{9.755614in}{3.000378in}}%
\pgfpathlineto{\pgfqpoint{9.755957in}{3.000333in}}%
\pgfpathlineto{\pgfqpoint{9.756984in}{2.999578in}}%
\pgfpathlineto{\pgfqpoint{9.757326in}{2.999511in}}%
\pgfpathlineto{\pgfqpoint{9.758011in}{2.998143in}}%
\pgfpathlineto{\pgfqpoint{9.759722in}{2.994904in}}%
\pgfpathlineto{\pgfqpoint{9.760749in}{2.994896in}}%
\pgfpathlineto{\pgfqpoint{9.762803in}{2.990010in}}%
\pgfpathlineto{\pgfqpoint{9.764173in}{2.988379in}}%
\pgfpathlineto{\pgfqpoint{9.764858in}{2.988250in}}%
\pgfpathlineto{\pgfqpoint{9.765542in}{2.987346in}}%
\pgfpathlineto{\pgfqpoint{9.765885in}{2.986853in}}%
\pgfpathlineto{\pgfqpoint{9.766569in}{2.984416in}}%
\pgfpathlineto{\pgfqpoint{9.767254in}{2.984210in}}%
\pgfpathlineto{\pgfqpoint{9.767939in}{2.983647in}}%
\pgfpathlineto{\pgfqpoint{9.768281in}{2.983421in}}%
\pgfpathlineto{\pgfqpoint{9.769308in}{2.981401in}}%
\pgfpathlineto{\pgfqpoint{9.770335in}{2.980797in}}%
\pgfpathlineto{\pgfqpoint{9.771020in}{2.979633in}}%
\pgfpathlineto{\pgfqpoint{9.772389in}{2.978936in}}%
\pgfpathlineto{\pgfqpoint{9.773074in}{2.977740in}}%
\pgfpathlineto{\pgfqpoint{9.773759in}{2.977527in}}%
\pgfpathlineto{\pgfqpoint{9.774443in}{2.975903in}}%
\pgfpathlineto{\pgfqpoint{9.775128in}{2.975292in}}%
\pgfpathlineto{\pgfqpoint{9.775813in}{2.975027in}}%
\pgfpathlineto{\pgfqpoint{9.777524in}{2.973018in}}%
\pgfpathlineto{\pgfqpoint{9.778209in}{2.972958in}}%
\pgfpathlineto{\pgfqpoint{9.779236in}{2.970806in}}%
\pgfpathlineto{\pgfqpoint{9.779921in}{2.969832in}}%
\pgfpathlineto{\pgfqpoint{9.780605in}{2.969145in}}%
\pgfpathlineto{\pgfqpoint{9.781290in}{2.968261in}}%
\pgfpathlineto{\pgfqpoint{9.782660in}{2.966315in}}%
\pgfpathlineto{\pgfqpoint{9.783002in}{2.966275in}}%
\pgfpathlineto{\pgfqpoint{9.784029in}{2.965348in}}%
\pgfpathlineto{\pgfqpoint{9.784371in}{2.965234in}}%
\pgfpathlineto{\pgfqpoint{9.786083in}{2.962169in}}%
\pgfpathlineto{\pgfqpoint{9.788479in}{2.960347in}}%
\pgfpathlineto{\pgfqpoint{9.788822in}{2.958648in}}%
\pgfpathlineto{\pgfqpoint{9.789164in}{2.958426in}}%
\pgfpathlineto{\pgfqpoint{9.789506in}{2.957819in}}%
\pgfpathlineto{\pgfqpoint{9.789849in}{2.956571in}}%
\pgfpathlineto{\pgfqpoint{9.790191in}{2.956560in}}%
\pgfpathlineto{\pgfqpoint{9.790876in}{2.955401in}}%
\pgfpathlineto{\pgfqpoint{9.791218in}{2.955363in}}%
\pgfpathlineto{\pgfqpoint{9.791903in}{2.954457in}}%
\pgfpathlineto{\pgfqpoint{9.793272in}{2.953279in}}%
\pgfpathlineto{\pgfqpoint{9.793957in}{2.952222in}}%
\pgfpathlineto{\pgfqpoint{9.797038in}{2.950455in}}%
\pgfpathlineto{\pgfqpoint{9.797723in}{2.949624in}}%
\pgfpathlineto{\pgfqpoint{9.798408in}{2.949284in}}%
\pgfpathlineto{\pgfqpoint{9.799092in}{2.947510in}}%
\pgfpathlineto{\pgfqpoint{9.800119in}{2.947132in}}%
\pgfpathlineto{\pgfqpoint{9.801146in}{2.945393in}}%
\pgfpathlineto{\pgfqpoint{9.801489in}{2.945357in}}%
\pgfpathlineto{\pgfqpoint{9.802516in}{2.943641in}}%
\pgfpathlineto{\pgfqpoint{9.803543in}{2.943311in}}%
\pgfpathlineto{\pgfqpoint{9.804570in}{2.942103in}}%
\pgfpathlineto{\pgfqpoint{9.805254in}{2.940557in}}%
\pgfpathlineto{\pgfqpoint{9.805597in}{2.940298in}}%
\pgfpathlineto{\pgfqpoint{9.807309in}{2.937209in}}%
\pgfpathlineto{\pgfqpoint{9.807993in}{2.937051in}}%
\pgfpathlineto{\pgfqpoint{9.809363in}{2.935643in}}%
\pgfpathlineto{\pgfqpoint{9.810047in}{2.934521in}}%
\pgfpathlineto{\pgfqpoint{9.812101in}{2.932784in}}%
\pgfpathlineto{\pgfqpoint{9.812786in}{2.930738in}}%
\pgfpathlineto{\pgfqpoint{9.813813in}{2.929929in}}%
\pgfpathlineto{\pgfqpoint{9.814498in}{2.929416in}}%
\pgfpathlineto{\pgfqpoint{9.815182in}{2.928970in}}%
\pgfpathlineto{\pgfqpoint{9.815867in}{2.928509in}}%
\pgfpathlineto{\pgfqpoint{9.818264in}{2.927887in}}%
\pgfpathlineto{\pgfqpoint{9.820660in}{2.923911in}}%
\pgfpathlineto{\pgfqpoint{9.821687in}{2.923560in}}%
\pgfpathlineto{\pgfqpoint{9.822029in}{2.923344in}}%
\pgfpathlineto{\pgfqpoint{9.822714in}{2.922363in}}%
\pgfpathlineto{\pgfqpoint{9.823399in}{2.921778in}}%
\pgfpathlineto{\pgfqpoint{9.823741in}{2.920717in}}%
\pgfpathlineto{\pgfqpoint{9.824426in}{2.920513in}}%
\pgfpathlineto{\pgfqpoint{9.825111in}{2.919795in}}%
\pgfpathlineto{\pgfqpoint{9.826138in}{2.919078in}}%
\pgfpathlineto{\pgfqpoint{9.826480in}{2.918754in}}%
\pgfpathlineto{\pgfqpoint{9.828192in}{2.914773in}}%
\pgfpathlineto{\pgfqpoint{9.828876in}{2.914509in}}%
\pgfpathlineto{\pgfqpoint{9.829219in}{2.913677in}}%
\pgfpathlineto{\pgfqpoint{9.829561in}{2.913641in}}%
\pgfpathlineto{\pgfqpoint{9.829903in}{2.912508in}}%
\pgfpathlineto{\pgfqpoint{9.830246in}{2.912414in}}%
\pgfpathlineto{\pgfqpoint{9.830930in}{2.911081in}}%
\pgfpathlineto{\pgfqpoint{9.831273in}{2.910658in}}%
\pgfpathlineto{\pgfqpoint{9.831615in}{2.909676in}}%
\pgfpathlineto{\pgfqpoint{9.832300in}{2.909450in}}%
\pgfpathlineto{\pgfqpoint{9.832642in}{2.908166in}}%
\pgfpathlineto{\pgfqpoint{9.832985in}{2.908120in}}%
\pgfpathlineto{\pgfqpoint{9.834354in}{2.906580in}}%
\pgfpathlineto{\pgfqpoint{9.834696in}{2.903952in}}%
\pgfpathlineto{\pgfqpoint{9.835381in}{2.903439in}}%
\pgfpathlineto{\pgfqpoint{9.836066in}{2.902079in}}%
\pgfpathlineto{\pgfqpoint{9.837093in}{2.900629in}}%
\pgfpathlineto{\pgfqpoint{9.837435in}{2.900551in}}%
\pgfpathlineto{\pgfqpoint{9.838120in}{2.898802in}}%
\pgfpathlineto{\pgfqpoint{9.839147in}{2.898394in}}%
\pgfpathlineto{\pgfqpoint{9.839831in}{2.897543in}}%
\pgfpathlineto{\pgfqpoint{9.840174in}{2.897484in}}%
\pgfpathlineto{\pgfqpoint{9.840516in}{2.896889in}}%
\pgfpathlineto{\pgfqpoint{9.840858in}{2.896854in}}%
\pgfpathlineto{\pgfqpoint{9.841201in}{2.895064in}}%
\pgfpathlineto{\pgfqpoint{9.842228in}{2.894800in}}%
\pgfpathlineto{\pgfqpoint{9.842913in}{2.894316in}}%
\pgfpathlineto{\pgfqpoint{9.843597in}{2.892836in}}%
\pgfpathlineto{\pgfqpoint{9.844282in}{2.892761in}}%
\pgfpathlineto{\pgfqpoint{9.845651in}{2.892043in}}%
\pgfpathlineto{\pgfqpoint{9.845994in}{2.892043in}}%
\pgfpathlineto{\pgfqpoint{9.847021in}{2.890457in}}%
\pgfpathlineto{\pgfqpoint{9.848048in}{2.890004in}}%
\pgfpathlineto{\pgfqpoint{9.850102in}{2.886455in}}%
\pgfpathlineto{\pgfqpoint{9.850444in}{2.886417in}}%
\pgfpathlineto{\pgfqpoint{9.851129in}{2.884794in}}%
\pgfpathlineto{\pgfqpoint{9.851471in}{2.883208in}}%
\pgfpathlineto{\pgfqpoint{9.852498in}{2.882446in}}%
\pgfpathlineto{\pgfqpoint{9.853183in}{2.881086in}}%
\pgfpathlineto{\pgfqpoint{9.853525in}{2.880980in}}%
\pgfpathlineto{\pgfqpoint{9.854210in}{2.880278in}}%
\pgfpathlineto{\pgfqpoint{9.855579in}{2.878941in}}%
\pgfpathlineto{\pgfqpoint{9.855922in}{2.878859in}}%
\pgfpathlineto{\pgfqpoint{9.856606in}{2.877969in}}%
\pgfpathlineto{\pgfqpoint{9.856949in}{2.877642in}}%
\pgfpathlineto{\pgfqpoint{9.857633in}{2.876449in}}%
\pgfpathlineto{\pgfqpoint{9.860030in}{2.874901in}}%
\pgfpathlineto{\pgfqpoint{9.861742in}{2.874486in}}%
\pgfpathlineto{\pgfqpoint{9.862769in}{2.872754in}}%
\pgfpathlineto{\pgfqpoint{9.864823in}{2.870597in}}%
\pgfpathlineto{\pgfqpoint{9.865850in}{2.870151in}}%
\pgfpathlineto{\pgfqpoint{9.868246in}{2.867199in}}%
\pgfpathlineto{\pgfqpoint{9.868931in}{2.866617in}}%
\pgfpathlineto{\pgfqpoint{9.869616in}{2.865530in}}%
\pgfpathlineto{\pgfqpoint{9.870300in}{2.865386in}}%
\pgfpathlineto{\pgfqpoint{9.870985in}{2.864524in}}%
\pgfpathlineto{\pgfqpoint{9.871670in}{2.863800in}}%
\pgfpathlineto{\pgfqpoint{9.872012in}{2.862932in}}%
\pgfpathlineto{\pgfqpoint{9.872354in}{2.862901in}}%
\pgfpathlineto{\pgfqpoint{9.873381in}{2.861975in}}%
\pgfpathlineto{\pgfqpoint{9.873724in}{2.861663in}}%
\pgfpathlineto{\pgfqpoint{9.874408in}{2.859939in}}%
\pgfpathlineto{\pgfqpoint{9.874751in}{2.859836in}}%
\pgfpathlineto{\pgfqpoint{9.875436in}{2.858779in}}%
\pgfpathlineto{\pgfqpoint{9.876120in}{2.858514in}}%
\pgfpathlineto{\pgfqpoint{9.877490in}{2.858137in}}%
\pgfpathlineto{\pgfqpoint{9.878859in}{2.856777in}}%
\pgfpathlineto{\pgfqpoint{9.879544in}{2.855411in}}%
\pgfpathlineto{\pgfqpoint{9.879886in}{2.855224in}}%
\pgfpathlineto{\pgfqpoint{9.880571in}{2.854535in}}%
\pgfpathlineto{\pgfqpoint{9.881255in}{2.854384in}}%
\pgfpathlineto{\pgfqpoint{9.881598in}{2.852416in}}%
\pgfpathlineto{\pgfqpoint{9.885021in}{2.851147in}}%
\pgfpathlineto{\pgfqpoint{9.885364in}{2.851121in}}%
\pgfpathlineto{\pgfqpoint{9.886733in}{2.850126in}}%
\pgfpathlineto{\pgfqpoint{9.888102in}{2.849724in}}%
\pgfpathlineto{\pgfqpoint{9.889129in}{2.848524in}}%
\pgfpathlineto{\pgfqpoint{9.889814in}{2.848469in}}%
\pgfpathlineto{\pgfqpoint{9.890156in}{2.845946in}}%
\pgfpathlineto{\pgfqpoint{9.890841in}{2.845412in}}%
\pgfpathlineto{\pgfqpoint{9.891526in}{2.844767in}}%
\pgfpathlineto{\pgfqpoint{9.892211in}{2.843638in}}%
\pgfpathlineto{\pgfqpoint{9.893580in}{2.842958in}}%
\pgfpathlineto{\pgfqpoint{9.894949in}{2.839673in}}%
\pgfpathlineto{\pgfqpoint{9.895634in}{2.839107in}}%
\pgfpathlineto{\pgfqpoint{9.895976in}{2.838858in}}%
\pgfpathlineto{\pgfqpoint{9.898373in}{2.834500in}}%
\pgfpathlineto{\pgfqpoint{9.900427in}{2.833775in}}%
\pgfpathlineto{\pgfqpoint{9.901454in}{2.832348in}}%
\pgfpathlineto{\pgfqpoint{9.903508in}{2.830824in}}%
\pgfpathlineto{\pgfqpoint{9.904193in}{2.830725in}}%
\pgfpathlineto{\pgfqpoint{9.904877in}{2.829922in}}%
\pgfpathlineto{\pgfqpoint{9.907274in}{2.828436in}}%
\pgfpathlineto{\pgfqpoint{9.907958in}{2.827431in}}%
\pgfpathlineto{\pgfqpoint{9.908301in}{2.827412in}}%
\pgfpathlineto{\pgfqpoint{9.910355in}{2.824457in}}%
\pgfpathlineto{\pgfqpoint{9.910697in}{2.824328in}}%
\pgfpathlineto{\pgfqpoint{9.911382in}{2.823173in}}%
\pgfpathlineto{\pgfqpoint{9.914463in}{2.820629in}}%
\pgfpathlineto{\pgfqpoint{9.915148in}{2.819943in}}%
\pgfpathlineto{\pgfqpoint{9.915832in}{2.819812in}}%
\pgfpathlineto{\pgfqpoint{9.916859in}{2.819012in}}%
\pgfpathlineto{\pgfqpoint{9.917202in}{2.818991in}}%
\pgfpathlineto{\pgfqpoint{9.917887in}{2.817962in}}%
\pgfpathlineto{\pgfqpoint{9.918914in}{2.817620in}}%
\pgfpathlineto{\pgfqpoint{9.919256in}{2.817283in}}%
\pgfpathlineto{\pgfqpoint{9.919941in}{2.816037in}}%
\pgfpathlineto{\pgfqpoint{9.921652in}{2.815761in}}%
\pgfpathlineto{\pgfqpoint{9.922337in}{2.814887in}}%
\pgfpathlineto{\pgfqpoint{9.923022in}{2.814564in}}%
\pgfpathlineto{\pgfqpoint{9.923364in}{2.813733in}}%
\pgfpathlineto{\pgfqpoint{9.923706in}{2.813696in}}%
\pgfpathlineto{\pgfqpoint{9.924391in}{2.812903in}}%
\pgfpathlineto{\pgfqpoint{9.925760in}{2.812065in}}%
\pgfpathlineto{\pgfqpoint{9.926445in}{2.811368in}}%
\pgfpathlineto{\pgfqpoint{9.926788in}{2.810115in}}%
\pgfpathlineto{\pgfqpoint{9.927472in}{2.809860in}}%
\pgfpathlineto{\pgfqpoint{9.928157in}{2.809039in}}%
\pgfpathlineto{\pgfqpoint{9.929526in}{2.808447in}}%
\pgfpathlineto{\pgfqpoint{9.930211in}{2.807654in}}%
\pgfpathlineto{\pgfqpoint{9.931923in}{2.805011in}}%
\pgfpathlineto{\pgfqpoint{9.932607in}{2.804370in}}%
\pgfpathlineto{\pgfqpoint{9.933292in}{2.803463in}}%
\pgfpathlineto{\pgfqpoint{9.933977in}{2.803244in}}%
\pgfpathlineto{\pgfqpoint{9.934661in}{2.802434in}}%
\pgfpathlineto{\pgfqpoint{9.937400in}{2.800569in}}%
\pgfpathlineto{\pgfqpoint{9.937743in}{2.799680in}}%
\pgfpathlineto{\pgfqpoint{9.938085in}{2.797498in}}%
\pgfpathlineto{\pgfqpoint{9.939454in}{2.797120in}}%
\pgfpathlineto{\pgfqpoint{9.940139in}{2.795524in}}%
\pgfpathlineto{\pgfqpoint{9.940481in}{2.795512in}}%
\pgfpathlineto{\pgfqpoint{9.942535in}{2.792924in}}%
\pgfpathlineto{\pgfqpoint{9.943905in}{2.791791in}}%
\pgfpathlineto{\pgfqpoint{9.944932in}{2.790899in}}%
\pgfpathlineto{\pgfqpoint{9.945617in}{2.789493in}}%
\pgfpathlineto{\pgfqpoint{9.945959in}{2.789191in}}%
\pgfpathlineto{\pgfqpoint{9.946644in}{2.788322in}}%
\pgfpathlineto{\pgfqpoint{9.947671in}{2.787605in}}%
\pgfpathlineto{\pgfqpoint{9.949725in}{2.786117in}}%
\pgfpathlineto{\pgfqpoint{9.950067in}{2.785423in}}%
\pgfpathlineto{\pgfqpoint{9.950409in}{2.785377in}}%
\pgfpathlineto{\pgfqpoint{9.953148in}{2.782046in}}%
\pgfpathlineto{\pgfqpoint{9.954518in}{2.780886in}}%
\pgfpathlineto{\pgfqpoint{9.955545in}{2.778536in}}%
\pgfpathlineto{\pgfqpoint{9.956572in}{2.778203in}}%
\pgfpathlineto{\pgfqpoint{9.956914in}{2.777448in}}%
\pgfpathlineto{\pgfqpoint{9.957256in}{2.777426in}}%
\pgfpathlineto{\pgfqpoint{9.957941in}{2.776939in}}%
\pgfpathlineto{\pgfqpoint{9.958968in}{2.776580in}}%
\pgfpathlineto{\pgfqpoint{9.959995in}{2.776391in}}%
\pgfpathlineto{\pgfqpoint{9.963076in}{2.772464in}}%
\pgfpathlineto{\pgfqpoint{9.963419in}{2.770539in}}%
\pgfpathlineto{\pgfqpoint{9.964446in}{2.770289in}}%
\pgfpathlineto{\pgfqpoint{9.965130in}{2.770123in}}%
\pgfpathlineto{\pgfqpoint{9.965815in}{2.769708in}}%
\pgfpathlineto{\pgfqpoint{9.966157in}{2.768462in}}%
\pgfpathlineto{\pgfqpoint{9.967184in}{2.768315in}}%
\pgfpathlineto{\pgfqpoint{9.967869in}{2.767791in}}%
\pgfpathlineto{\pgfqpoint{9.968211in}{2.767678in}}%
\pgfpathlineto{\pgfqpoint{9.968896in}{2.766606in}}%
\pgfpathlineto{\pgfqpoint{9.969581in}{2.766347in}}%
\pgfpathlineto{\pgfqpoint{9.970266in}{2.765909in}}%
\pgfpathlineto{\pgfqpoint{9.970608in}{2.765774in}}%
\pgfpathlineto{\pgfqpoint{9.971635in}{2.764540in}}%
\pgfpathlineto{\pgfqpoint{9.972320in}{2.764201in}}%
\pgfpathlineto{\pgfqpoint{9.974374in}{2.762949in}}%
\pgfpathlineto{\pgfqpoint{9.975058in}{2.762194in}}%
\pgfpathlineto{\pgfqpoint{9.975401in}{2.762043in}}%
\pgfpathlineto{\pgfqpoint{9.976428in}{2.760541in}}%
\pgfpathlineto{\pgfqpoint{9.977112in}{2.760338in}}%
\pgfpathlineto{\pgfqpoint{9.977455in}{2.759264in}}%
\pgfpathlineto{\pgfqpoint{9.978482in}{2.758947in}}%
\pgfpathlineto{\pgfqpoint{9.979167in}{2.758293in}}%
\pgfpathlineto{\pgfqpoint{9.979509in}{2.758267in}}%
\pgfpathlineto{\pgfqpoint{9.980194in}{2.757134in}}%
\pgfpathlineto{\pgfqpoint{9.980536in}{2.756243in}}%
\pgfpathlineto{\pgfqpoint{9.981563in}{2.755881in}}%
\pgfpathlineto{\pgfqpoint{9.982248in}{2.755133in}}%
\pgfpathlineto{\pgfqpoint{9.983275in}{2.754884in}}%
\pgfpathlineto{\pgfqpoint{9.984302in}{2.754567in}}%
\pgfpathlineto{\pgfqpoint{9.984986in}{2.753740in}}%
\pgfpathlineto{\pgfqpoint{9.986014in}{2.752749in}}%
\pgfpathlineto{\pgfqpoint{9.986356in}{2.752679in}}%
\pgfpathlineto{\pgfqpoint{9.987041in}{2.751042in}}%
\pgfpathlineto{\pgfqpoint{9.988068in}{2.750697in}}%
\pgfpathlineto{\pgfqpoint{9.988752in}{2.750383in}}%
\pgfpathlineto{\pgfqpoint{9.990806in}{2.747242in}}%
\pgfpathlineto{\pgfqpoint{9.991491in}{2.745611in}}%
\pgfpathlineto{\pgfqpoint{9.991833in}{2.745430in}}%
\pgfpathlineto{\pgfqpoint{9.992518in}{2.744750in}}%
\pgfpathlineto{\pgfqpoint{9.993203in}{2.743844in}}%
\pgfpathlineto{\pgfqpoint{9.994915in}{2.743284in}}%
\pgfpathlineto{\pgfqpoint{9.995599in}{2.741578in}}%
\pgfpathlineto{\pgfqpoint{9.995942in}{2.741389in}}%
\pgfpathlineto{\pgfqpoint{9.996626in}{2.740325in}}%
\pgfpathlineto{\pgfqpoint{10.001419in}{2.735650in}}%
\pgfpathlineto{\pgfqpoint{10.002104in}{2.734669in}}%
\pgfpathlineto{\pgfqpoint{10.002446in}{2.734088in}}%
\pgfpathlineto{\pgfqpoint{10.003131in}{2.733838in}}%
\pgfpathlineto{\pgfqpoint{10.003816in}{2.733368in}}%
\pgfpathlineto{\pgfqpoint{10.004500in}{2.732554in}}%
\pgfpathlineto{\pgfqpoint{10.005527in}{2.731746in}}%
\pgfpathlineto{\pgfqpoint{10.006554in}{2.731421in}}%
\pgfpathlineto{\pgfqpoint{10.007581in}{2.729760in}}%
\pgfpathlineto{\pgfqpoint{10.008266in}{2.727586in}}%
\pgfpathlineto{\pgfqpoint{10.008951in}{2.726098in}}%
\pgfpathlineto{\pgfqpoint{10.009293in}{2.726098in}}%
\pgfpathlineto{\pgfqpoint{10.009978in}{2.725419in}}%
\pgfpathlineto{\pgfqpoint{10.010320in}{2.725380in}}%
\pgfpathlineto{\pgfqpoint{10.011005in}{2.724549in}}%
\pgfpathlineto{\pgfqpoint{10.011690in}{2.723233in}}%
\pgfpathlineto{\pgfqpoint{10.013744in}{2.722412in}}%
\pgfpathlineto{\pgfqpoint{10.015113in}{2.718952in}}%
\pgfpathlineto{\pgfqpoint{10.015798in}{2.718196in}}%
\pgfpathlineto{\pgfqpoint{10.016140in}{2.718032in}}%
\pgfpathlineto{\pgfqpoint{10.017509in}{2.716069in}}%
\pgfpathlineto{\pgfqpoint{10.018194in}{2.715903in}}%
\pgfpathlineto{\pgfqpoint{10.018536in}{2.714906in}}%
\pgfpathlineto{\pgfqpoint{10.019221in}{2.714529in}}%
\pgfpathlineto{\pgfqpoint{10.019906in}{2.713466in}}%
\pgfpathlineto{\pgfqpoint{10.020248in}{2.713222in}}%
\pgfpathlineto{\pgfqpoint{10.020933in}{2.712052in}}%
\pgfpathlineto{\pgfqpoint{10.021618in}{2.711475in}}%
\pgfpathlineto{\pgfqpoint{10.022302in}{2.710297in}}%
\pgfpathlineto{\pgfqpoint{10.022987in}{2.709416in}}%
\pgfpathlineto{\pgfqpoint{10.024699in}{2.708800in}}%
\pgfpathlineto{\pgfqpoint{10.026410in}{2.707400in}}%
\pgfpathlineto{\pgfqpoint{10.027095in}{2.706330in}}%
\pgfpathlineto{\pgfqpoint{10.029149in}{2.705074in}}%
\pgfpathlineto{\pgfqpoint{10.029834in}{2.704613in}}%
\pgfpathlineto{\pgfqpoint{10.030176in}{2.704582in}}%
\pgfpathlineto{\pgfqpoint{10.031203in}{2.703556in}}%
\pgfpathlineto{\pgfqpoint{10.031546in}{2.703443in}}%
\pgfpathlineto{\pgfqpoint{10.032230in}{2.702590in}}%
\pgfpathlineto{\pgfqpoint{10.032915in}{2.701819in}}%
\pgfpathlineto{\pgfqpoint{10.033942in}{2.701481in}}%
\pgfpathlineto{\pgfqpoint{10.034627in}{2.701177in}}%
\pgfpathlineto{\pgfqpoint{10.034969in}{2.701117in}}%
\pgfpathlineto{\pgfqpoint{10.035654in}{2.699894in}}%
\pgfpathlineto{\pgfqpoint{10.036338in}{2.699856in}}%
\pgfpathlineto{\pgfqpoint{10.037023in}{2.699138in}}%
\pgfpathlineto{\pgfqpoint{10.038050in}{2.698799in}}%
\pgfpathlineto{\pgfqpoint{10.038393in}{2.697900in}}%
\pgfpathlineto{\pgfqpoint{10.038735in}{2.697855in}}%
\pgfpathlineto{\pgfqpoint{10.040104in}{2.696080in}}%
\pgfpathlineto{\pgfqpoint{10.041131in}{2.695619in}}%
\pgfpathlineto{\pgfqpoint{10.041474in}{2.695529in}}%
\pgfpathlineto{\pgfqpoint{10.041816in}{2.694162in}}%
\pgfpathlineto{\pgfqpoint{10.043185in}{2.693602in}}%
\pgfpathlineto{\pgfqpoint{10.044897in}{2.690983in}}%
\pgfpathlineto{\pgfqpoint{10.045582in}{2.690699in}}%
\pgfpathlineto{\pgfqpoint{10.046951in}{2.689963in}}%
\pgfpathlineto{\pgfqpoint{10.047978in}{2.689623in}}%
\pgfpathlineto{\pgfqpoint{10.049005in}{2.688233in}}%
\pgfpathlineto{\pgfqpoint{10.049348in}{2.688113in}}%
\pgfpathlineto{\pgfqpoint{10.050032in}{2.687320in}}%
\pgfpathlineto{\pgfqpoint{10.050717in}{2.687094in}}%
\pgfpathlineto{\pgfqpoint{10.051402in}{2.686301in}}%
\pgfpathlineto{\pgfqpoint{10.051744in}{2.685963in}}%
\pgfpathlineto{\pgfqpoint{10.052771in}{2.683942in}}%
\pgfpathlineto{\pgfqpoint{10.054483in}{2.683021in}}%
\pgfpathlineto{\pgfqpoint{10.055852in}{2.682501in}}%
\pgfpathlineto{\pgfqpoint{10.056537in}{2.681550in}}%
\pgfpathlineto{\pgfqpoint{10.057222in}{2.681251in}}%
\pgfpathlineto{\pgfqpoint{10.057906in}{2.679697in}}%
\pgfpathlineto{\pgfqpoint{10.058591in}{2.679610in}}%
\pgfpathlineto{\pgfqpoint{10.058933in}{2.677815in}}%
\pgfpathlineto{\pgfqpoint{10.059618in}{2.677541in}}%
\pgfpathlineto{\pgfqpoint{10.060645in}{2.675963in}}%
\pgfpathlineto{\pgfqpoint{10.061330in}{2.675921in}}%
\pgfpathlineto{\pgfqpoint{10.063726in}{2.673649in}}%
\pgfpathlineto{\pgfqpoint{10.064411in}{2.671899in}}%
\pgfpathlineto{\pgfqpoint{10.065096in}{2.671198in}}%
\pgfpathlineto{\pgfqpoint{10.065780in}{2.670782in}}%
\pgfpathlineto{\pgfqpoint{10.066123in}{2.670639in}}%
\pgfpathlineto{\pgfqpoint{10.067150in}{2.669267in}}%
\pgfpathlineto{\pgfqpoint{10.068177in}{2.668819in}}%
\pgfpathlineto{\pgfqpoint{10.068519in}{2.667664in}}%
\pgfpathlineto{\pgfqpoint{10.068861in}{2.667574in}}%
\pgfpathlineto{\pgfqpoint{10.069546in}{2.666365in}}%
\pgfpathlineto{\pgfqpoint{10.069888in}{2.666289in}}%
\pgfpathlineto{\pgfqpoint{10.070573in}{2.665836in}}%
\pgfpathlineto{\pgfqpoint{10.071258in}{2.665776in}}%
\pgfpathlineto{\pgfqpoint{10.071943in}{2.665081in}}%
\pgfpathlineto{\pgfqpoint{10.072627in}{2.664898in}}%
\pgfpathlineto{\pgfqpoint{10.073654in}{2.664439in}}%
\pgfpathlineto{\pgfqpoint{10.074339in}{2.663457in}}%
\pgfpathlineto{\pgfqpoint{10.075024in}{2.663118in}}%
\pgfpathlineto{\pgfqpoint{10.078105in}{2.659063in}}%
\pgfpathlineto{\pgfqpoint{10.078789in}{2.657378in}}%
\pgfpathlineto{\pgfqpoint{10.080159in}{2.655566in}}%
\pgfpathlineto{\pgfqpoint{10.081871in}{2.654948in}}%
\pgfpathlineto{\pgfqpoint{10.082213in}{2.652933in}}%
\pgfpathlineto{\pgfqpoint{10.083925in}{2.652260in}}%
\pgfpathlineto{\pgfqpoint{10.084267in}{2.651337in}}%
\pgfpathlineto{\pgfqpoint{10.084952in}{2.651246in}}%
\pgfpathlineto{\pgfqpoint{10.086321in}{2.649736in}}%
\pgfpathlineto{\pgfqpoint{10.086663in}{2.649713in}}%
\pgfpathlineto{\pgfqpoint{10.087690in}{2.648958in}}%
\pgfpathlineto{\pgfqpoint{10.088375in}{2.647491in}}%
\pgfpathlineto{\pgfqpoint{10.088718in}{2.646995in}}%
\pgfpathlineto{\pgfqpoint{10.089402in}{2.644276in}}%
\pgfpathlineto{\pgfqpoint{10.089745in}{2.644239in}}%
\pgfpathlineto{\pgfqpoint{10.090087in}{2.643514in}}%
\pgfpathlineto{\pgfqpoint{10.090429in}{2.643483in}}%
\pgfpathlineto{\pgfqpoint{10.091114in}{2.642145in}}%
\pgfpathlineto{\pgfqpoint{10.093853in}{2.640954in}}%
\pgfpathlineto{\pgfqpoint{10.095564in}{2.639594in}}%
\pgfpathlineto{\pgfqpoint{10.096249in}{2.639368in}}%
\pgfpathlineto{\pgfqpoint{10.096934in}{2.638867in}}%
\pgfpathlineto{\pgfqpoint{10.097619in}{2.638613in}}%
\pgfpathlineto{\pgfqpoint{10.098303in}{2.637669in}}%
\pgfpathlineto{\pgfqpoint{10.101384in}{2.636792in}}%
\pgfpathlineto{\pgfqpoint{10.102411in}{2.636234in}}%
\pgfpathlineto{\pgfqpoint{10.103438in}{2.635630in}}%
\pgfpathlineto{\pgfqpoint{10.104123in}{2.634535in}}%
\pgfpathlineto{\pgfqpoint{10.104808in}{2.634384in}}%
\pgfpathlineto{\pgfqpoint{10.107547in}{2.631439in}}%
\pgfpathlineto{\pgfqpoint{10.108574in}{2.631325in}}%
\pgfpathlineto{\pgfqpoint{10.109601in}{2.629906in}}%
\pgfpathlineto{\pgfqpoint{10.109943in}{2.628502in}}%
\pgfpathlineto{\pgfqpoint{10.110285in}{2.628350in}}%
\pgfpathlineto{\pgfqpoint{10.110970in}{2.627587in}}%
\pgfpathlineto{\pgfqpoint{10.111655in}{2.627210in}}%
\pgfpathlineto{\pgfqpoint{10.111997in}{2.627165in}}%
\pgfpathlineto{\pgfqpoint{10.115421in}{2.623395in}}%
\pgfpathlineto{\pgfqpoint{10.116105in}{2.621372in}}%
\pgfpathlineto{\pgfqpoint{10.116448in}{2.621330in}}%
\pgfpathlineto{\pgfqpoint{10.118159in}{2.619222in}}%
\pgfpathlineto{\pgfqpoint{10.119186in}{2.617959in}}%
\pgfpathlineto{\pgfqpoint{10.119871in}{2.616677in}}%
\pgfpathlineto{\pgfqpoint{10.120898in}{2.615958in}}%
\pgfpathlineto{\pgfqpoint{10.121583in}{2.614519in}}%
\pgfpathlineto{\pgfqpoint{10.121925in}{2.614485in}}%
\pgfpathlineto{\pgfqpoint{10.122610in}{2.614123in}}%
\pgfpathlineto{\pgfqpoint{10.124664in}{2.612976in}}%
\pgfpathlineto{\pgfqpoint{10.125349in}{2.612311in}}%
\pgfpathlineto{\pgfqpoint{10.127060in}{2.610172in}}%
\pgfpathlineto{\pgfqpoint{10.127403in}{2.610143in}}%
\pgfpathlineto{\pgfqpoint{10.129799in}{2.607666in}}%
\pgfpathlineto{\pgfqpoint{10.130826in}{2.607462in}}%
\pgfpathlineto{\pgfqpoint{10.133565in}{2.606028in}}%
\pgfpathlineto{\pgfqpoint{10.136988in}{2.604055in}}%
\pgfpathlineto{\pgfqpoint{10.138700in}{2.603805in}}%
\pgfpathlineto{\pgfqpoint{10.140754in}{2.601630in}}%
\pgfpathlineto{\pgfqpoint{10.142124in}{2.601394in}}%
\pgfpathlineto{\pgfqpoint{10.143151in}{2.599005in}}%
\pgfpathlineto{\pgfqpoint{10.144178in}{2.598401in}}%
\pgfpathlineto{\pgfqpoint{10.144862in}{2.597108in}}%
\pgfpathlineto{\pgfqpoint{10.145205in}{2.596097in}}%
\pgfpathlineto{\pgfqpoint{10.145547in}{2.596022in}}%
\pgfpathlineto{\pgfqpoint{10.146916in}{2.593796in}}%
\pgfpathlineto{\pgfqpoint{10.148286in}{2.592019in}}%
\pgfpathlineto{\pgfqpoint{10.148628in}{2.591868in}}%
\pgfpathlineto{\pgfqpoint{10.151025in}{2.588969in}}%
\pgfpathlineto{\pgfqpoint{10.151709in}{2.588751in}}%
\pgfpathlineto{\pgfqpoint{10.152394in}{2.588281in}}%
\pgfpathlineto{\pgfqpoint{10.153079in}{2.587476in}}%
\pgfpathlineto{\pgfqpoint{10.154106in}{2.587028in}}%
\pgfpathlineto{\pgfqpoint{10.154448in}{2.586167in}}%
\pgfpathlineto{\pgfqpoint{10.155475in}{2.585903in}}%
\pgfpathlineto{\pgfqpoint{10.155817in}{2.585810in}}%
\pgfpathlineto{\pgfqpoint{10.156845in}{2.584909in}}%
\pgfpathlineto{\pgfqpoint{10.157187in}{2.582950in}}%
\pgfpathlineto{\pgfqpoint{10.158556in}{2.582542in}}%
\pgfpathlineto{\pgfqpoint{10.159241in}{2.582278in}}%
\pgfpathlineto{\pgfqpoint{10.159926in}{2.580956in}}%
\pgfpathlineto{\pgfqpoint{10.160610in}{2.580541in}}%
\pgfpathlineto{\pgfqpoint{10.161295in}{2.580239in}}%
\pgfpathlineto{\pgfqpoint{10.161980in}{2.580195in}}%
\pgfpathlineto{\pgfqpoint{10.162664in}{2.579899in}}%
\pgfpathlineto{\pgfqpoint{10.163691in}{2.579537in}}%
\pgfpathlineto{\pgfqpoint{10.164034in}{2.579465in}}%
\pgfpathlineto{\pgfqpoint{10.164376in}{2.578653in}}%
\pgfpathlineto{\pgfqpoint{10.164718in}{2.578600in}}%
\pgfpathlineto{\pgfqpoint{10.165746in}{2.576807in}}%
\pgfpathlineto{\pgfqpoint{10.166088in}{2.576717in}}%
\pgfpathlineto{\pgfqpoint{10.166773in}{2.576181in}}%
\pgfpathlineto{\pgfqpoint{10.167115in}{2.575983in}}%
\pgfpathlineto{\pgfqpoint{10.168827in}{2.572890in}}%
\pgfpathlineto{\pgfqpoint{10.169511in}{2.572385in}}%
\pgfpathlineto{\pgfqpoint{10.170538in}{2.571230in}}%
\pgfpathlineto{\pgfqpoint{10.170881in}{2.569818in}}%
\pgfpathlineto{\pgfqpoint{10.171223in}{2.569755in}}%
\pgfpathlineto{\pgfqpoint{10.172250in}{2.568362in}}%
\pgfpathlineto{\pgfqpoint{10.175331in}{2.567545in}}%
\pgfpathlineto{\pgfqpoint{10.176701in}{2.565665in}}%
\pgfpathlineto{\pgfqpoint{10.177043in}{2.565627in}}%
\pgfpathlineto{\pgfqpoint{10.177728in}{2.564399in}}%
\pgfpathlineto{\pgfqpoint{10.178412in}{2.563960in}}%
\pgfpathlineto{\pgfqpoint{10.179097in}{2.563105in}}%
\pgfpathlineto{\pgfqpoint{10.179782in}{2.563056in}}%
\pgfpathlineto{\pgfqpoint{10.180466in}{2.562016in}}%
\pgfpathlineto{\pgfqpoint{10.181836in}{2.560289in}}%
\pgfpathlineto{\pgfqpoint{10.182521in}{2.558944in}}%
\pgfpathlineto{\pgfqpoint{10.183548in}{2.558542in}}%
\pgfpathlineto{\pgfqpoint{10.184232in}{2.557924in}}%
\pgfpathlineto{\pgfqpoint{10.186971in}{2.555490in}}%
\pgfpathlineto{\pgfqpoint{10.187998in}{2.554667in}}%
\pgfpathlineto{\pgfqpoint{10.188683in}{2.553695in}}%
\pgfpathlineto{\pgfqpoint{10.190052in}{2.552883in}}%
\pgfpathlineto{\pgfqpoint{10.191079in}{2.551203in}}%
\pgfpathlineto{\pgfqpoint{10.191422in}{2.551082in}}%
\pgfpathlineto{\pgfqpoint{10.194845in}{2.546974in}}%
\pgfpathlineto{\pgfqpoint{10.196214in}{2.546030in}}%
\pgfpathlineto{\pgfqpoint{10.196899in}{2.545842in}}%
\pgfpathlineto{\pgfqpoint{10.197241in}{2.545110in}}%
\pgfpathlineto{\pgfqpoint{10.197584in}{2.545086in}}%
\pgfpathlineto{\pgfqpoint{10.198268in}{2.544226in}}%
\pgfpathlineto{\pgfqpoint{10.199980in}{2.542474in}}%
\pgfpathlineto{\pgfqpoint{10.200665in}{2.541543in}}%
\pgfpathlineto{\pgfqpoint{10.202377in}{2.540693in}}%
\pgfpathlineto{\pgfqpoint{10.203061in}{2.540601in}}%
\pgfpathlineto{\pgfqpoint{10.203404in}{2.540065in}}%
\pgfpathlineto{\pgfqpoint{10.204088in}{2.538145in}}%
\pgfpathlineto{\pgfqpoint{10.205115in}{2.537384in}}%
\pgfpathlineto{\pgfqpoint{10.205800in}{2.537042in}}%
\pgfpathlineto{\pgfqpoint{10.207512in}{2.535269in}}%
\pgfpathlineto{\pgfqpoint{10.208197in}{2.534378in}}%
\pgfpathlineto{\pgfqpoint{10.209224in}{2.533948in}}%
\pgfpathlineto{\pgfqpoint{10.209908in}{2.533457in}}%
\pgfpathlineto{\pgfqpoint{10.211278in}{2.532596in}}%
\pgfpathlineto{\pgfqpoint{10.211620in}{2.531229in}}%
\pgfpathlineto{\pgfqpoint{10.212647in}{2.530844in}}%
\pgfpathlineto{\pgfqpoint{10.212989in}{2.529183in}}%
\pgfpathlineto{\pgfqpoint{10.214359in}{2.528126in}}%
\pgfpathlineto{\pgfqpoint{10.215043in}{2.527722in}}%
\pgfpathlineto{\pgfqpoint{10.215728in}{2.527527in}}%
\pgfpathlineto{\pgfqpoint{10.217782in}{2.526736in}}%
\pgfpathlineto{\pgfqpoint{10.218467in}{2.525498in}}%
\pgfpathlineto{\pgfqpoint{10.219152in}{2.525165in}}%
\pgfpathlineto{\pgfqpoint{10.219836in}{2.525129in}}%
\pgfpathlineto{\pgfqpoint{10.220179in}{2.524206in}}%
\pgfpathlineto{\pgfqpoint{10.220521in}{2.524129in}}%
\pgfpathlineto{\pgfqpoint{10.221206in}{2.523678in}}%
\pgfpathlineto{\pgfqpoint{10.223260in}{2.522725in}}%
\pgfpathlineto{\pgfqpoint{10.223944in}{2.521299in}}%
\pgfpathlineto{\pgfqpoint{10.224971in}{2.520846in}}%
\pgfpathlineto{\pgfqpoint{10.225656in}{2.520100in}}%
\pgfpathlineto{\pgfqpoint{10.227710in}{2.518460in}}%
\pgfpathlineto{\pgfqpoint{10.229080in}{2.516358in}}%
\pgfpathlineto{\pgfqpoint{10.231134in}{2.515258in}}%
\pgfpathlineto{\pgfqpoint{10.231818in}{2.513181in}}%
\pgfpathlineto{\pgfqpoint{10.233188in}{2.513030in}}%
\pgfpathlineto{\pgfqpoint{10.233873in}{2.512388in}}%
\pgfpathlineto{\pgfqpoint{10.234215in}{2.511180in}}%
\pgfpathlineto{\pgfqpoint{10.234557in}{2.511150in}}%
\pgfpathlineto{\pgfqpoint{10.235242in}{2.510453in}}%
\pgfpathlineto{\pgfqpoint{10.237296in}{2.509364in}}%
\pgfpathlineto{\pgfqpoint{10.237981in}{2.507770in}}%
\pgfpathlineto{\pgfqpoint{10.240035in}{2.506143in}}%
\pgfpathlineto{\pgfqpoint{10.242774in}{2.504595in}}%
\pgfpathlineto{\pgfqpoint{10.243116in}{2.504044in}}%
\pgfpathlineto{\pgfqpoint{10.244143in}{2.503810in}}%
\pgfpathlineto{\pgfqpoint{10.245170in}{2.503100in}}%
\pgfpathlineto{\pgfqpoint{10.247566in}{2.500592in}}%
\pgfpathlineto{\pgfqpoint{10.248251in}{2.499713in}}%
\pgfpathlineto{\pgfqpoint{10.249620in}{2.499286in}}%
\pgfpathlineto{\pgfqpoint{10.250647in}{2.498540in}}%
\pgfpathlineto{\pgfqpoint{10.250990in}{2.498348in}}%
\pgfpathlineto{\pgfqpoint{10.252359in}{2.495322in}}%
\pgfpathlineto{\pgfqpoint{10.254071in}{2.494038in}}%
\pgfpathlineto{\pgfqpoint{10.255098in}{2.492947in}}%
\pgfpathlineto{\pgfqpoint{10.255440in}{2.492688in}}%
\pgfpathlineto{\pgfqpoint{10.256125in}{2.491636in}}%
\pgfpathlineto{\pgfqpoint{10.256810in}{2.490640in}}%
\pgfpathlineto{\pgfqpoint{10.258521in}{2.489922in}}%
\pgfpathlineto{\pgfqpoint{10.259549in}{2.488563in}}%
\pgfpathlineto{\pgfqpoint{10.260233in}{2.488136in}}%
\pgfpathlineto{\pgfqpoint{10.260576in}{2.487846in}}%
\pgfpathlineto{\pgfqpoint{10.260918in}{2.486204in}}%
\pgfpathlineto{\pgfqpoint{10.261260in}{2.486139in}}%
\pgfpathlineto{\pgfqpoint{10.261945in}{2.485731in}}%
\pgfpathlineto{\pgfqpoint{10.262287in}{2.485531in}}%
\pgfpathlineto{\pgfqpoint{10.262972in}{2.484689in}}%
\pgfpathlineto{\pgfqpoint{10.263314in}{2.484598in}}%
\pgfpathlineto{\pgfqpoint{10.263999in}{2.483243in}}%
\pgfpathlineto{\pgfqpoint{10.264684in}{2.482933in}}%
\pgfpathlineto{\pgfqpoint{10.265368in}{2.482144in}}%
\pgfpathlineto{\pgfqpoint{10.265711in}{2.481993in}}%
\pgfpathlineto{\pgfqpoint{10.266395in}{2.480709in}}%
\pgfpathlineto{\pgfqpoint{10.266738in}{2.480181in}}%
\pgfpathlineto{\pgfqpoint{10.267422in}{2.479916in}}%
\pgfpathlineto{\pgfqpoint{10.268450in}{2.478715in}}%
\pgfpathlineto{\pgfqpoint{10.272215in}{2.476669in}}%
\pgfpathlineto{\pgfqpoint{10.273242in}{2.475929in}}%
\pgfpathlineto{\pgfqpoint{10.273927in}{2.475612in}}%
\pgfpathlineto{\pgfqpoint{10.274612in}{2.474857in}}%
\pgfpathlineto{\pgfqpoint{10.275296in}{2.474479in}}%
\pgfpathlineto{\pgfqpoint{10.275639in}{2.474442in}}%
\pgfpathlineto{\pgfqpoint{10.276666in}{2.472989in}}%
\pgfpathlineto{\pgfqpoint{10.277351in}{2.472758in}}%
\pgfpathlineto{\pgfqpoint{10.277693in}{2.471610in}}%
\pgfpathlineto{\pgfqpoint{10.279405in}{2.470944in}}%
\pgfpathlineto{\pgfqpoint{10.280089in}{2.470099in}}%
\pgfpathlineto{\pgfqpoint{10.281116in}{2.469405in}}%
\pgfpathlineto{\pgfqpoint{10.281459in}{2.469374in}}%
\pgfpathlineto{\pgfqpoint{10.282486in}{2.468227in}}%
\pgfpathlineto{\pgfqpoint{10.282828in}{2.467260in}}%
\pgfpathlineto{\pgfqpoint{10.283170in}{2.467191in}}%
\pgfpathlineto{\pgfqpoint{10.283513in}{2.466444in}}%
\pgfpathlineto{\pgfqpoint{10.283855in}{2.466387in}}%
\pgfpathlineto{\pgfqpoint{10.284540in}{2.464654in}}%
\pgfpathlineto{\pgfqpoint{10.285225in}{2.464390in}}%
\pgfpathlineto{\pgfqpoint{10.285567in}{2.464360in}}%
\pgfpathlineto{\pgfqpoint{10.286252in}{2.463907in}}%
\pgfpathlineto{\pgfqpoint{10.287279in}{2.463681in}}%
\pgfpathlineto{\pgfqpoint{10.287963in}{2.463076in}}%
\pgfpathlineto{\pgfqpoint{10.289333in}{2.461830in}}%
\pgfpathlineto{\pgfqpoint{10.289675in}{2.461793in}}%
\pgfpathlineto{\pgfqpoint{10.290360in}{2.461264in}}%
\pgfpathlineto{\pgfqpoint{10.291387in}{2.460962in}}%
\pgfpathlineto{\pgfqpoint{10.292756in}{2.459201in}}%
\pgfpathlineto{\pgfqpoint{10.293441in}{2.458746in}}%
\pgfpathlineto{\pgfqpoint{10.293783in}{2.457204in}}%
\pgfpathlineto{\pgfqpoint{10.294468in}{2.456537in}}%
\pgfpathlineto{\pgfqpoint{10.295153in}{2.455751in}}%
\pgfpathlineto{\pgfqpoint{10.295495in}{2.455676in}}%
\pgfpathlineto{\pgfqpoint{10.296180in}{2.453524in}}%
\pgfpathlineto{\pgfqpoint{10.297549in}{2.452909in}}%
\pgfpathlineto{\pgfqpoint{10.298576in}{2.450526in}}%
\pgfpathlineto{\pgfqpoint{10.298918in}{2.450390in}}%
\pgfpathlineto{\pgfqpoint{10.299603in}{2.449660in}}%
\pgfpathlineto{\pgfqpoint{10.300972in}{2.447800in}}%
\pgfpathlineto{\pgfqpoint{10.302342in}{2.447048in}}%
\pgfpathlineto{\pgfqpoint{10.303027in}{2.446387in}}%
\pgfpathlineto{\pgfqpoint{10.303369in}{2.446387in}}%
\pgfpathlineto{\pgfqpoint{10.304054in}{2.445897in}}%
\pgfpathlineto{\pgfqpoint{10.305081in}{2.445280in}}%
\pgfpathlineto{\pgfqpoint{10.306108in}{2.443782in}}%
\pgfpathlineto{\pgfqpoint{10.309189in}{2.441132in}}%
\pgfpathlineto{\pgfqpoint{10.309873in}{2.439360in}}%
\pgfpathlineto{\pgfqpoint{10.310558in}{2.438927in}}%
\pgfpathlineto{\pgfqpoint{10.312270in}{2.436873in}}%
\pgfpathlineto{\pgfqpoint{10.312955in}{2.436608in}}%
\pgfpathlineto{\pgfqpoint{10.313297in}{2.436193in}}%
\pgfpathlineto{\pgfqpoint{10.313639in}{2.435316in}}%
\pgfpathlineto{\pgfqpoint{10.313982in}{2.435249in}}%
\pgfpathlineto{\pgfqpoint{10.315009in}{2.434063in}}%
\pgfpathlineto{\pgfqpoint{10.315693in}{2.433984in}}%
\pgfpathlineto{\pgfqpoint{10.316378in}{2.432784in}}%
\pgfpathlineto{\pgfqpoint{10.317063in}{2.432455in}}%
\pgfpathlineto{\pgfqpoint{10.317747in}{2.432141in}}%
\pgfpathlineto{\pgfqpoint{10.318432in}{2.431398in}}%
\pgfpathlineto{\pgfqpoint{10.318774in}{2.431330in}}%
\pgfpathlineto{\pgfqpoint{10.319459in}{2.430658in}}%
\pgfpathlineto{\pgfqpoint{10.319802in}{2.430499in}}%
\pgfpathlineto{\pgfqpoint{10.321171in}{2.428400in}}%
\pgfpathlineto{\pgfqpoint{10.321513in}{2.428339in}}%
\pgfpathlineto{\pgfqpoint{10.322198in}{2.427660in}}%
\pgfpathlineto{\pgfqpoint{10.323225in}{2.427303in}}%
\pgfpathlineto{\pgfqpoint{10.325279in}{2.426451in}}%
\pgfpathlineto{\pgfqpoint{10.325621in}{2.426149in}}%
\pgfpathlineto{\pgfqpoint{10.325964in}{2.424979in}}%
\pgfpathlineto{\pgfqpoint{10.326306in}{2.424903in}}%
\pgfpathlineto{\pgfqpoint{10.327676in}{2.423166in}}%
\pgfpathlineto{\pgfqpoint{10.328018in}{2.423043in}}%
\pgfpathlineto{\pgfqpoint{10.329045in}{2.421092in}}%
\pgfpathlineto{\pgfqpoint{10.331441in}{2.417578in}}%
\pgfpathlineto{\pgfqpoint{10.332811in}{2.416415in}}%
\pgfpathlineto{\pgfqpoint{10.333495in}{2.415658in}}%
\pgfpathlineto{\pgfqpoint{10.333838in}{2.415493in}}%
\pgfpathlineto{\pgfqpoint{10.334865in}{2.414037in}}%
\pgfpathlineto{\pgfqpoint{10.335549in}{2.413650in}}%
\pgfpathlineto{\pgfqpoint{10.335892in}{2.413644in}}%
\pgfpathlineto{\pgfqpoint{10.338973in}{2.410971in}}%
\pgfpathlineto{\pgfqpoint{10.339658in}{2.410706in}}%
\pgfpathlineto{\pgfqpoint{10.340685in}{2.409158in}}%
\pgfpathlineto{\pgfqpoint{10.343081in}{2.407895in}}%
\pgfpathlineto{\pgfqpoint{10.343766in}{2.405867in}}%
\pgfpathlineto{\pgfqpoint{10.344450in}{2.405307in}}%
\pgfpathlineto{\pgfqpoint{10.345820in}{2.403878in}}%
\pgfpathlineto{\pgfqpoint{10.346505in}{2.402129in}}%
\pgfpathlineto{\pgfqpoint{10.347532in}{2.401639in}}%
\pgfpathlineto{\pgfqpoint{10.347874in}{2.401341in}}%
\pgfpathlineto{\pgfqpoint{10.348559in}{2.400172in}}%
\pgfpathlineto{\pgfqpoint{10.351640in}{2.397288in}}%
\pgfpathlineto{\pgfqpoint{10.352667in}{2.396094in}}%
\pgfpathlineto{\pgfqpoint{10.353352in}{2.395679in}}%
\pgfpathlineto{\pgfqpoint{10.354379in}{2.394622in}}%
\pgfpathlineto{\pgfqpoint{10.354721in}{2.393806in}}%
\pgfpathlineto{\pgfqpoint{10.355063in}{2.393751in}}%
\pgfpathlineto{\pgfqpoint{10.355406in}{2.392809in}}%
\pgfpathlineto{\pgfqpoint{10.356775in}{2.392609in}}%
\pgfpathlineto{\pgfqpoint{10.358144in}{2.392085in}}%
\pgfpathlineto{\pgfqpoint{10.358487in}{2.391533in}}%
\pgfpathlineto{\pgfqpoint{10.358829in}{2.391527in}}%
\pgfpathlineto{\pgfqpoint{10.359514in}{2.390997in}}%
\pgfpathlineto{\pgfqpoint{10.360198in}{2.390899in}}%
\pgfpathlineto{\pgfqpoint{10.360883in}{2.390299in}}%
\pgfpathlineto{\pgfqpoint{10.362595in}{2.389369in}}%
\pgfpathlineto{\pgfqpoint{10.362937in}{2.388758in}}%
\pgfpathlineto{\pgfqpoint{10.363622in}{2.388452in}}%
\pgfpathlineto{\pgfqpoint{10.364307in}{2.388046in}}%
\pgfpathlineto{\pgfqpoint{10.364649in}{2.388029in}}%
\pgfpathlineto{\pgfqpoint{10.365334in}{2.387485in}}%
\pgfpathlineto{\pgfqpoint{10.366361in}{2.386768in}}%
\pgfpathlineto{\pgfqpoint{10.368415in}{2.385635in}}%
\pgfpathlineto{\pgfqpoint{10.369099in}{2.384706in}}%
\pgfpathlineto{\pgfqpoint{10.369784in}{2.384351in}}%
\pgfpathlineto{\pgfqpoint{10.371154in}{2.381947in}}%
\pgfpathlineto{\pgfqpoint{10.372181in}{2.381180in}}%
\pgfpathlineto{\pgfqpoint{10.372865in}{2.380727in}}%
\pgfpathlineto{\pgfqpoint{10.373550in}{2.380115in}}%
\pgfpathlineto{\pgfqpoint{10.374577in}{2.379896in}}%
\pgfpathlineto{\pgfqpoint{10.375262in}{2.376838in}}%
\pgfpathlineto{\pgfqpoint{10.376631in}{2.376536in}}%
\pgfpathlineto{\pgfqpoint{10.376973in}{2.376058in}}%
\pgfpathlineto{\pgfqpoint{10.377316in}{2.374979in}}%
\pgfpathlineto{\pgfqpoint{10.378000in}{2.374818in}}%
\pgfpathlineto{\pgfqpoint{10.379028in}{2.373073in}}%
\pgfpathlineto{\pgfqpoint{10.381082in}{2.371249in}}%
\pgfpathlineto{\pgfqpoint{10.382793in}{2.369475in}}%
\pgfpathlineto{\pgfqpoint{10.384505in}{2.368521in}}%
\pgfpathlineto{\pgfqpoint{10.386217in}{2.368040in}}%
\pgfpathlineto{\pgfqpoint{10.386901in}{2.367152in}}%
\pgfpathlineto{\pgfqpoint{10.387244in}{2.367093in}}%
\pgfpathlineto{\pgfqpoint{10.388271in}{2.366278in}}%
\pgfpathlineto{\pgfqpoint{10.389298in}{2.365940in}}%
\pgfpathlineto{\pgfqpoint{10.389983in}{2.365163in}}%
\pgfpathlineto{\pgfqpoint{10.390667in}{2.364815in}}%
\pgfpathlineto{\pgfqpoint{10.392379in}{2.363638in}}%
\pgfpathlineto{\pgfqpoint{10.393406in}{2.363094in}}%
\pgfpathlineto{\pgfqpoint{10.394091in}{2.361767in}}%
\pgfpathlineto{\pgfqpoint{10.394433in}{2.361659in}}%
\pgfpathlineto{\pgfqpoint{10.395118in}{2.360451in}}%
\pgfpathlineto{\pgfqpoint{10.395803in}{2.359847in}}%
\pgfpathlineto{\pgfqpoint{10.397172in}{2.355921in}}%
\pgfpathlineto{\pgfqpoint{10.399568in}{2.354821in}}%
\pgfpathlineto{\pgfqpoint{10.399911in}{2.354145in}}%
\pgfpathlineto{\pgfqpoint{10.400253in}{2.354145in}}%
\pgfpathlineto{\pgfqpoint{10.400938in}{2.353696in}}%
\pgfpathlineto{\pgfqpoint{10.401622in}{2.353314in}}%
\pgfpathlineto{\pgfqpoint{10.402307in}{2.352925in}}%
\pgfpathlineto{\pgfqpoint{10.403334in}{2.352204in}}%
\pgfpathlineto{\pgfqpoint{10.403676in}{2.350083in}}%
\pgfpathlineto{\pgfqpoint{10.404704in}{2.349481in}}%
\pgfpathlineto{\pgfqpoint{10.409839in}{2.346042in}}%
\pgfpathlineto{\pgfqpoint{10.410181in}{2.346027in}}%
\pgfpathlineto{\pgfqpoint{10.411208in}{2.345234in}}%
\pgfpathlineto{\pgfqpoint{10.411893in}{2.344985in}}%
\pgfpathlineto{\pgfqpoint{10.412235in}{2.344386in}}%
\pgfpathlineto{\pgfqpoint{10.413262in}{2.344109in}}%
\pgfpathlineto{\pgfqpoint{10.413947in}{2.343648in}}%
\pgfpathlineto{\pgfqpoint{10.414632in}{2.343271in}}%
\pgfpathlineto{\pgfqpoint{10.414974in}{2.342418in}}%
\pgfpathlineto{\pgfqpoint{10.416686in}{2.341638in}}%
\pgfpathlineto{\pgfqpoint{10.418397in}{2.340628in}}%
\pgfpathlineto{\pgfqpoint{10.418740in}{2.340552in}}%
\pgfpathlineto{\pgfqpoint{10.419424in}{2.339220in}}%
\pgfpathlineto{\pgfqpoint{10.419767in}{2.339155in}}%
\pgfpathlineto{\pgfqpoint{10.420451in}{2.338459in}}%
\pgfpathlineto{\pgfqpoint{10.421136in}{2.338174in}}%
\pgfpathlineto{\pgfqpoint{10.422506in}{2.337116in}}%
\pgfpathlineto{\pgfqpoint{10.423190in}{2.335772in}}%
\pgfpathlineto{\pgfqpoint{10.424560in}{2.335534in}}%
\pgfpathlineto{\pgfqpoint{10.425587in}{2.334897in}}%
\pgfpathlineto{\pgfqpoint{10.425929in}{2.334767in}}%
\pgfpathlineto{\pgfqpoint{10.426614in}{2.333492in}}%
\pgfpathlineto{\pgfqpoint{10.429010in}{2.331302in}}%
\pgfpathlineto{\pgfqpoint{10.429352in}{2.329403in}}%
\pgfpathlineto{\pgfqpoint{10.430380in}{2.328871in}}%
\pgfpathlineto{\pgfqpoint{10.432776in}{2.327715in}}%
\pgfpathlineto{\pgfqpoint{10.433118in}{2.327043in}}%
\pgfpathlineto{\pgfqpoint{10.433461in}{2.327043in}}%
\pgfpathlineto{\pgfqpoint{10.434488in}{2.325170in}}%
\pgfpathlineto{\pgfqpoint{10.435172in}{2.324535in}}%
\pgfpathlineto{\pgfqpoint{10.436199in}{2.322026in}}%
\pgfpathlineto{\pgfqpoint{10.437911in}{2.321183in}}%
\pgfpathlineto{\pgfqpoint{10.438596in}{2.320585in}}%
\pgfpathlineto{\pgfqpoint{10.440308in}{2.319755in}}%
\pgfpathlineto{\pgfqpoint{10.440992in}{2.318426in}}%
\pgfpathlineto{\pgfqpoint{10.441335in}{2.318426in}}%
\pgfpathlineto{\pgfqpoint{10.441677in}{2.317926in}}%
\pgfpathlineto{\pgfqpoint{10.442019in}{2.317920in}}%
\pgfpathlineto{\pgfqpoint{10.442704in}{2.316501in}}%
\pgfpathlineto{\pgfqpoint{10.443389in}{2.316411in}}%
\pgfpathlineto{\pgfqpoint{10.445785in}{2.314952in}}%
\pgfpathlineto{\pgfqpoint{10.446812in}{2.313147in}}%
\pgfpathlineto{\pgfqpoint{10.449551in}{2.312380in}}%
\pgfpathlineto{\pgfqpoint{10.451263in}{2.309704in}}%
\pgfpathlineto{\pgfqpoint{10.451947in}{2.308307in}}%
\pgfpathlineto{\pgfqpoint{10.452974in}{2.307499in}}%
\pgfpathlineto{\pgfqpoint{10.453317in}{2.307439in}}%
\pgfpathlineto{\pgfqpoint{10.454001in}{2.305576in}}%
\pgfpathlineto{\pgfqpoint{10.454344in}{2.305051in}}%
\pgfpathlineto{\pgfqpoint{10.455028in}{2.302806in}}%
\pgfpathlineto{\pgfqpoint{10.457425in}{2.301608in}}%
\pgfpathlineto{\pgfqpoint{10.459137in}{2.299508in}}%
\pgfpathlineto{\pgfqpoint{10.459479in}{2.299399in}}%
\pgfpathlineto{\pgfqpoint{10.461875in}{2.295728in}}%
\pgfpathlineto{\pgfqpoint{10.463245in}{2.295130in}}%
\pgfpathlineto{\pgfqpoint{10.464272in}{2.293151in}}%
\pgfpathlineto{\pgfqpoint{10.465984in}{2.290883in}}%
\pgfpathlineto{\pgfqpoint{10.466326in}{2.290775in}}%
\pgfpathlineto{\pgfqpoint{10.467011in}{2.289833in}}%
\pgfpathlineto{\pgfqpoint{10.467353in}{2.289754in}}%
\pgfpathlineto{\pgfqpoint{10.468038in}{2.289194in}}%
\pgfpathlineto{\pgfqpoint{10.468380in}{2.287744in}}%
\pgfpathlineto{\pgfqpoint{10.469065in}{2.287472in}}%
\pgfpathlineto{\pgfqpoint{10.470776in}{2.286294in}}%
\pgfpathlineto{\pgfqpoint{10.471803in}{2.286020in}}%
\pgfpathlineto{\pgfqpoint{10.472146in}{2.285479in}}%
\pgfpathlineto{\pgfqpoint{10.472488in}{2.285464in}}%
\pgfpathlineto{\pgfqpoint{10.473515in}{2.284520in}}%
\pgfpathlineto{\pgfqpoint{10.474200in}{2.284391in}}%
\pgfpathlineto{\pgfqpoint{10.474542in}{2.283508in}}%
\pgfpathlineto{\pgfqpoint{10.474885in}{2.283462in}}%
\pgfpathlineto{\pgfqpoint{10.475569in}{2.282594in}}%
\pgfpathlineto{\pgfqpoint{10.476254in}{2.281884in}}%
\pgfpathlineto{\pgfqpoint{10.477281in}{2.281329in}}%
\pgfpathlineto{\pgfqpoint{10.477966in}{2.280480in}}%
\pgfpathlineto{\pgfqpoint{10.478993in}{2.280253in}}%
\pgfpathlineto{\pgfqpoint{10.479335in}{2.279679in}}%
\pgfpathlineto{\pgfqpoint{10.479677in}{2.279679in}}%
\pgfpathlineto{\pgfqpoint{10.481047in}{2.276477in}}%
\pgfpathlineto{\pgfqpoint{10.482074in}{2.275458in}}%
\pgfpathlineto{\pgfqpoint{10.482759in}{2.274023in}}%
\pgfpathlineto{\pgfqpoint{10.486182in}{2.271977in}}%
\pgfpathlineto{\pgfqpoint{10.486867in}{2.271463in}}%
\pgfpathlineto{\pgfqpoint{10.487551in}{2.270772in}}%
\pgfpathlineto{\pgfqpoint{10.488236in}{2.270290in}}%
\pgfpathlineto{\pgfqpoint{10.489263in}{2.269288in}}%
\pgfpathlineto{\pgfqpoint{10.490290in}{2.268683in}}%
\pgfpathlineto{\pgfqpoint{10.490975in}{2.268533in}}%
\pgfpathlineto{\pgfqpoint{10.491660in}{2.267234in}}%
\pgfpathlineto{\pgfqpoint{10.493714in}{2.266056in}}%
\pgfpathlineto{\pgfqpoint{10.496110in}{2.264574in}}%
\pgfpathlineto{\pgfqpoint{10.497822in}{2.263549in}}%
\pgfpathlineto{\pgfqpoint{10.498849in}{2.263025in}}%
\pgfpathlineto{\pgfqpoint{10.499534in}{2.262545in}}%
\pgfpathlineto{\pgfqpoint{10.500903in}{2.260791in}}%
\pgfpathlineto{\pgfqpoint{10.501588in}{2.260770in}}%
\pgfpathlineto{\pgfqpoint{10.503984in}{2.258233in}}%
\pgfpathlineto{\pgfqpoint{10.504669in}{2.258157in}}%
\pgfpathlineto{\pgfqpoint{10.505011in}{2.257778in}}%
\pgfpathlineto{\pgfqpoint{10.505353in}{2.256902in}}%
\pgfpathlineto{\pgfqpoint{10.506038in}{2.256873in}}%
\pgfpathlineto{\pgfqpoint{10.506380in}{2.255605in}}%
\pgfpathlineto{\pgfqpoint{10.508777in}{2.254087in}}%
\pgfpathlineto{\pgfqpoint{10.509119in}{2.253177in}}%
\pgfpathlineto{\pgfqpoint{10.510146in}{2.252614in}}%
\pgfpathlineto{\pgfqpoint{10.511516in}{2.250991in}}%
\pgfpathlineto{\pgfqpoint{10.512200in}{2.250915in}}%
\pgfpathlineto{\pgfqpoint{10.514254in}{2.249167in}}%
\pgfpathlineto{\pgfqpoint{10.515966in}{2.248594in}}%
\pgfpathlineto{\pgfqpoint{10.516309in}{2.247895in}}%
\pgfpathlineto{\pgfqpoint{10.516993in}{2.247751in}}%
\pgfpathlineto{\pgfqpoint{10.517336in}{2.247509in}}%
\pgfpathlineto{\pgfqpoint{10.518020in}{2.244987in}}%
\pgfpathlineto{\pgfqpoint{10.518363in}{2.244874in}}%
\pgfpathlineto{\pgfqpoint{10.519047in}{2.244430in}}%
\pgfpathlineto{\pgfqpoint{10.519732in}{2.244006in}}%
\pgfpathlineto{\pgfqpoint{10.521101in}{2.241468in}}%
\pgfpathlineto{\pgfqpoint{10.522128in}{2.240865in}}%
\pgfpathlineto{\pgfqpoint{10.522813in}{2.238429in}}%
\pgfpathlineto{\pgfqpoint{10.523498in}{2.238266in}}%
\pgfpathlineto{\pgfqpoint{10.524867in}{2.237617in}}%
\pgfpathlineto{\pgfqpoint{10.527264in}{2.236665in}}%
\pgfpathlineto{\pgfqpoint{10.528291in}{2.235623in}}%
\pgfpathlineto{\pgfqpoint{10.528633in}{2.235501in}}%
\pgfpathlineto{\pgfqpoint{10.529660in}{2.234075in}}%
\pgfpathlineto{\pgfqpoint{10.530345in}{2.233282in}}%
\pgfpathlineto{\pgfqpoint{10.530687in}{2.232980in}}%
\pgfpathlineto{\pgfqpoint{10.531029in}{2.232044in}}%
\pgfpathlineto{\pgfqpoint{10.531714in}{2.231847in}}%
\pgfpathlineto{\pgfqpoint{10.532399in}{2.231168in}}%
\pgfpathlineto{\pgfqpoint{10.533426in}{2.230885in}}%
\pgfpathlineto{\pgfqpoint{10.533768in}{2.229620in}}%
\pgfpathlineto{\pgfqpoint{10.534453in}{2.229506in}}%
\pgfpathlineto{\pgfqpoint{10.537534in}{2.227672in}}%
\pgfpathlineto{\pgfqpoint{10.538219in}{2.227409in}}%
\pgfpathlineto{\pgfqpoint{10.538903in}{2.227241in}}%
\pgfpathlineto{\pgfqpoint{10.539588in}{2.226305in}}%
\pgfpathlineto{\pgfqpoint{10.539930in}{2.224795in}}%
\pgfpathlineto{\pgfqpoint{10.540615in}{2.224664in}}%
\pgfpathlineto{\pgfqpoint{10.541642in}{2.224018in}}%
\pgfpathlineto{\pgfqpoint{10.542669in}{2.223541in}}%
\pgfpathlineto{\pgfqpoint{10.543012in}{2.222982in}}%
\pgfpathlineto{\pgfqpoint{10.543696in}{2.222710in}}%
\pgfpathlineto{\pgfqpoint{10.544723in}{2.222332in}}%
\pgfpathlineto{\pgfqpoint{10.545750in}{2.221019in}}%
\pgfpathlineto{\pgfqpoint{10.546093in}{2.220938in}}%
\pgfpathlineto{\pgfqpoint{10.546777in}{2.219614in}}%
\pgfpathlineto{\pgfqpoint{10.547120in}{2.219312in}}%
\pgfpathlineto{\pgfqpoint{10.547804in}{2.218435in}}%
\pgfpathlineto{\pgfqpoint{10.548147in}{2.218240in}}%
\pgfpathlineto{\pgfqpoint{10.549174in}{2.216741in}}%
\pgfpathlineto{\pgfqpoint{10.549859in}{2.216500in}}%
\pgfpathlineto{\pgfqpoint{10.550543in}{2.214983in}}%
\pgfpathlineto{\pgfqpoint{10.550886in}{2.214977in}}%
\pgfpathlineto{\pgfqpoint{10.551228in}{2.213950in}}%
\pgfpathlineto{\pgfqpoint{10.551570in}{2.213888in}}%
\pgfpathlineto{\pgfqpoint{10.551913in}{2.213081in}}%
\pgfpathlineto{\pgfqpoint{10.553282in}{2.212802in}}%
\pgfpathlineto{\pgfqpoint{10.553967in}{2.211991in}}%
\pgfpathlineto{\pgfqpoint{10.554309in}{2.211843in}}%
\pgfpathlineto{\pgfqpoint{10.554651in}{2.210809in}}%
\pgfpathlineto{\pgfqpoint{10.555678in}{2.210570in}}%
\pgfpathlineto{\pgfqpoint{10.557390in}{2.208664in}}%
\pgfpathlineto{\pgfqpoint{10.558075in}{2.207892in}}%
\pgfpathlineto{\pgfqpoint{10.558760in}{2.207577in}}%
\pgfpathlineto{\pgfqpoint{10.559102in}{2.207569in}}%
\pgfpathlineto{\pgfqpoint{10.559444in}{2.206145in}}%
\pgfpathlineto{\pgfqpoint{10.559787in}{2.206134in}}%
\pgfpathlineto{\pgfqpoint{10.560471in}{2.205757in}}%
\pgfpathlineto{\pgfqpoint{10.561498in}{2.205456in}}%
\pgfpathlineto{\pgfqpoint{10.562183in}{2.205077in}}%
\pgfpathlineto{\pgfqpoint{10.562868in}{2.203831in}}%
\pgfpathlineto{\pgfqpoint{10.563895in}{2.203567in}}%
\pgfpathlineto{\pgfqpoint{10.564579in}{2.203318in}}%
\pgfpathlineto{\pgfqpoint{10.565264in}{2.202067in}}%
\pgfpathlineto{\pgfqpoint{10.566976in}{2.201150in}}%
\pgfpathlineto{\pgfqpoint{10.568003in}{2.200018in}}%
\pgfpathlineto{\pgfqpoint{10.569715in}{2.199021in}}%
\pgfpathlineto{\pgfqpoint{10.570399in}{2.198303in}}%
\pgfpathlineto{\pgfqpoint{10.570742in}{2.198111in}}%
\pgfpathlineto{\pgfqpoint{10.571426in}{2.197004in}}%
\pgfpathlineto{\pgfqpoint{10.574850in}{2.194996in}}%
\pgfpathlineto{\pgfqpoint{10.575535in}{2.193863in}}%
\pgfpathlineto{\pgfqpoint{10.575877in}{2.193712in}}%
\pgfpathlineto{\pgfqpoint{10.576562in}{2.192643in}}%
\pgfpathlineto{\pgfqpoint{10.577246in}{2.192164in}}%
\pgfpathlineto{\pgfqpoint{10.578273in}{2.191371in}}%
\pgfpathlineto{\pgfqpoint{10.578958in}{2.191107in}}%
\pgfpathlineto{\pgfqpoint{10.579300in}{2.190841in}}%
\pgfpathlineto{\pgfqpoint{10.579643in}{2.189537in}}%
\pgfpathlineto{\pgfqpoint{10.581697in}{2.188577in}}%
\pgfpathlineto{\pgfqpoint{10.582381in}{2.188003in}}%
\pgfpathlineto{\pgfqpoint{10.583751in}{2.186553in}}%
\pgfpathlineto{\pgfqpoint{10.585805in}{2.185828in}}%
\pgfpathlineto{\pgfqpoint{10.586832in}{2.185401in}}%
\pgfpathlineto{\pgfqpoint{10.587517in}{2.184575in}}%
\pgfpathlineto{\pgfqpoint{10.589228in}{2.183843in}}%
\pgfpathlineto{\pgfqpoint{10.590255in}{2.182282in}}%
\pgfpathlineto{\pgfqpoint{10.590940in}{2.181063in}}%
\pgfpathlineto{\pgfqpoint{10.591625in}{2.180421in}}%
\pgfpathlineto{\pgfqpoint{10.592310in}{2.180006in}}%
\pgfpathlineto{\pgfqpoint{10.592994in}{2.178533in}}%
\pgfpathlineto{\pgfqpoint{10.595733in}{2.176110in}}%
\pgfpathlineto{\pgfqpoint{10.597102in}{2.175365in}}%
\pgfpathlineto{\pgfqpoint{10.597787in}{2.174708in}}%
\pgfpathlineto{\pgfqpoint{10.598472in}{2.174566in}}%
\pgfpathlineto{\pgfqpoint{10.598814in}{2.173801in}}%
\pgfpathlineto{\pgfqpoint{10.599499in}{2.173570in}}%
\pgfpathlineto{\pgfqpoint{10.600183in}{2.171529in}}%
\pgfpathlineto{\pgfqpoint{10.602580in}{2.170363in}}%
\pgfpathlineto{\pgfqpoint{10.603607in}{2.169623in}}%
\pgfpathlineto{\pgfqpoint{10.604634in}{2.169245in}}%
\pgfpathlineto{\pgfqpoint{10.606003in}{2.168339in}}%
\pgfpathlineto{\pgfqpoint{10.606346in}{2.168274in}}%
\pgfpathlineto{\pgfqpoint{10.607030in}{2.167704in}}%
\pgfpathlineto{\pgfqpoint{10.609769in}{2.165205in}}%
\pgfpathlineto{\pgfqpoint{10.610112in}{2.164412in}}%
\pgfpathlineto{\pgfqpoint{10.610454in}{2.164363in}}%
\pgfpathlineto{\pgfqpoint{10.611139in}{2.163838in}}%
\pgfpathlineto{\pgfqpoint{10.614220in}{2.162411in}}%
\pgfpathlineto{\pgfqpoint{10.614904in}{2.161635in}}%
\pgfpathlineto{\pgfqpoint{10.616274in}{2.160002in}}%
\pgfpathlineto{\pgfqpoint{10.617643in}{2.158484in}}%
\pgfpathlineto{\pgfqpoint{10.617986in}{2.157125in}}%
\pgfpathlineto{\pgfqpoint{10.619697in}{2.156427in}}%
\pgfpathlineto{\pgfqpoint{10.621409in}{2.154600in}}%
\pgfpathlineto{\pgfqpoint{10.622094in}{2.154232in}}%
\pgfpathlineto{\pgfqpoint{10.622778in}{2.153538in}}%
\pgfpathlineto{\pgfqpoint{10.623805in}{2.153273in}}%
\pgfpathlineto{\pgfqpoint{10.624148in}{2.153036in}}%
\pgfpathlineto{\pgfqpoint{10.624490in}{2.151031in}}%
\pgfpathlineto{\pgfqpoint{10.625175in}{2.150895in}}%
\pgfpathlineto{\pgfqpoint{10.625859in}{2.150094in}}%
\pgfpathlineto{\pgfqpoint{10.627914in}{2.148298in}}%
\pgfpathlineto{\pgfqpoint{10.628598in}{2.147157in}}%
\pgfpathlineto{\pgfqpoint{10.629625in}{2.146815in}}%
\pgfpathlineto{\pgfqpoint{10.629968in}{2.145956in}}%
\pgfpathlineto{\pgfqpoint{10.630652in}{2.145558in}}%
\pgfpathlineto{\pgfqpoint{10.631337in}{2.145382in}}%
\pgfpathlineto{\pgfqpoint{10.632022in}{2.144514in}}%
\pgfpathlineto{\pgfqpoint{10.635103in}{2.143117in}}%
\pgfpathlineto{\pgfqpoint{10.636472in}{2.141795in}}%
\pgfpathlineto{\pgfqpoint{10.637157in}{2.140579in}}%
\pgfpathlineto{\pgfqpoint{10.637842in}{2.140303in}}%
\pgfpathlineto{\pgfqpoint{10.638869in}{2.138621in}}%
\pgfpathlineto{\pgfqpoint{10.639553in}{2.138064in}}%
\pgfpathlineto{\pgfqpoint{10.639896in}{2.137906in}}%
\pgfpathlineto{\pgfqpoint{10.640580in}{2.137075in}}%
\pgfpathlineto{\pgfqpoint{10.641265in}{2.136862in}}%
\pgfpathlineto{\pgfqpoint{10.641607in}{2.136834in}}%
\pgfpathlineto{\pgfqpoint{10.642292in}{2.136420in}}%
\pgfpathlineto{\pgfqpoint{10.644346in}{2.134201in}}%
\pgfpathlineto{\pgfqpoint{10.644689in}{2.133451in}}%
\pgfpathlineto{\pgfqpoint{10.645031in}{2.133399in}}%
\pgfpathlineto{\pgfqpoint{10.646058in}{2.131245in}}%
\pgfpathlineto{\pgfqpoint{10.647085in}{2.131072in}}%
\pgfpathlineto{\pgfqpoint{10.647770in}{2.129331in}}%
\pgfpathlineto{\pgfqpoint{10.649481in}{2.128631in}}%
\pgfpathlineto{\pgfqpoint{10.650166in}{2.128074in}}%
\pgfpathlineto{\pgfqpoint{10.650851in}{2.127828in}}%
\pgfpathlineto{\pgfqpoint{10.651878in}{2.126503in}}%
\pgfpathlineto{\pgfqpoint{10.653590in}{2.126141in}}%
\pgfpathlineto{\pgfqpoint{10.654617in}{2.125258in}}%
\pgfpathlineto{\pgfqpoint{10.657355in}{2.123890in}}%
\pgfpathlineto{\pgfqpoint{10.658040in}{2.123596in}}%
\pgfpathlineto{\pgfqpoint{10.658725in}{2.121987in}}%
\pgfpathlineto{\pgfqpoint{10.659409in}{2.121632in}}%
\pgfpathlineto{\pgfqpoint{10.660779in}{2.119756in}}%
\pgfpathlineto{\pgfqpoint{10.661121in}{2.119699in}}%
\pgfpathlineto{\pgfqpoint{10.661464in}{2.118952in}}%
\pgfpathlineto{\pgfqpoint{10.661806in}{2.118933in}}%
\pgfpathlineto{\pgfqpoint{10.662491in}{2.118574in}}%
\pgfpathlineto{\pgfqpoint{10.662833in}{2.118559in}}%
\pgfpathlineto{\pgfqpoint{10.663518in}{2.117932in}}%
\pgfpathlineto{\pgfqpoint{10.663860in}{2.117809in}}%
\pgfpathlineto{\pgfqpoint{10.664202in}{2.116189in}}%
\pgfpathlineto{\pgfqpoint{10.664887in}{2.116044in}}%
\pgfpathlineto{\pgfqpoint{10.665229in}{2.114995in}}%
\pgfpathlineto{\pgfqpoint{10.667626in}{2.114050in}}%
\pgfpathlineto{\pgfqpoint{10.668995in}{2.112117in}}%
\pgfpathlineto{\pgfqpoint{10.669680in}{2.111823in}}%
\pgfpathlineto{\pgfqpoint{10.672761in}{2.108742in}}%
\pgfpathlineto{\pgfqpoint{10.673446in}{2.107322in}}%
\pgfpathlineto{\pgfqpoint{10.675500in}{2.106054in}}%
\pgfpathlineto{\pgfqpoint{10.676184in}{2.105925in}}%
\pgfpathlineto{\pgfqpoint{10.676527in}{2.105450in}}%
\pgfpathlineto{\pgfqpoint{10.677211in}{2.103478in}}%
\pgfpathlineto{\pgfqpoint{10.678923in}{2.101583in}}%
\pgfpathlineto{\pgfqpoint{10.680293in}{2.101036in}}%
\pgfpathlineto{\pgfqpoint{10.681320in}{2.100828in}}%
\pgfpathlineto{\pgfqpoint{10.682004in}{2.100148in}}%
\pgfpathlineto{\pgfqpoint{10.682689in}{2.099884in}}%
\pgfpathlineto{\pgfqpoint{10.683716in}{2.096961in}}%
\pgfpathlineto{\pgfqpoint{10.685428in}{2.096138in}}%
\pgfpathlineto{\pgfqpoint{10.686455in}{2.093639in}}%
\pgfpathlineto{\pgfqpoint{10.686797in}{2.093585in}}%
\pgfpathlineto{\pgfqpoint{10.687482in}{2.092589in}}%
\pgfpathlineto{\pgfqpoint{10.688851in}{2.091930in}}%
\pgfpathlineto{\pgfqpoint{10.690221in}{2.090482in}}%
\pgfpathlineto{\pgfqpoint{10.690563in}{2.090452in}}%
\pgfpathlineto{\pgfqpoint{10.691590in}{2.089380in}}%
\pgfpathlineto{\pgfqpoint{10.692617in}{2.089156in}}%
\pgfpathlineto{\pgfqpoint{10.693986in}{2.087779in}}%
\pgfpathlineto{\pgfqpoint{10.695698in}{2.087356in}}%
\pgfpathlineto{\pgfqpoint{10.696041in}{2.084969in}}%
\pgfpathlineto{\pgfqpoint{10.696383in}{2.084639in}}%
\pgfpathlineto{\pgfqpoint{10.697068in}{2.083621in}}%
\pgfpathlineto{\pgfqpoint{10.698095in}{2.083500in}}%
\pgfpathlineto{\pgfqpoint{10.699806in}{2.080590in}}%
\pgfpathlineto{\pgfqpoint{10.700491in}{2.080138in}}%
\pgfpathlineto{\pgfqpoint{10.701176in}{2.079925in}}%
\pgfpathlineto{\pgfqpoint{10.701518in}{2.079426in}}%
\pgfpathlineto{\pgfqpoint{10.702203in}{2.077266in}}%
\pgfpathlineto{\pgfqpoint{10.703230in}{2.077003in}}%
\pgfpathlineto{\pgfqpoint{10.703915in}{2.075606in}}%
\pgfpathlineto{\pgfqpoint{10.704599in}{2.075266in}}%
\pgfpathlineto{\pgfqpoint{10.705969in}{2.073068in}}%
\pgfpathlineto{\pgfqpoint{10.706996in}{2.072706in}}%
\pgfpathlineto{\pgfqpoint{10.708023in}{2.072162in}}%
\pgfpathlineto{\pgfqpoint{10.708365in}{2.071900in}}%
\pgfpathlineto{\pgfqpoint{10.709734in}{2.068825in}}%
\pgfpathlineto{\pgfqpoint{10.710761in}{2.068557in}}%
\pgfpathlineto{\pgfqpoint{10.711446in}{2.067812in}}%
\pgfpathlineto{\pgfqpoint{10.711789in}{2.067746in}}%
\pgfpathlineto{\pgfqpoint{10.712816in}{2.065909in}}%
\pgfpathlineto{\pgfqpoint{10.713158in}{2.065879in}}%
\pgfpathlineto{\pgfqpoint{10.713500in}{2.065216in}}%
\pgfpathlineto{\pgfqpoint{10.713843in}{2.065171in}}%
\pgfpathlineto{\pgfqpoint{10.714870in}{2.063484in}}%
\pgfpathlineto{\pgfqpoint{10.715897in}{2.063273in}}%
\pgfpathlineto{\pgfqpoint{10.716581in}{2.062390in}}%
\pgfpathlineto{\pgfqpoint{10.717608in}{2.062258in}}%
\pgfpathlineto{\pgfqpoint{10.719662in}{2.060653in}}%
\pgfpathlineto{\pgfqpoint{10.720005in}{2.060648in}}%
\pgfpathlineto{\pgfqpoint{10.720347in}{2.060314in}}%
\pgfpathlineto{\pgfqpoint{10.721032in}{2.058448in}}%
\pgfpathlineto{\pgfqpoint{10.721717in}{2.057689in}}%
\pgfpathlineto{\pgfqpoint{10.722059in}{2.056991in}}%
\pgfpathlineto{\pgfqpoint{10.722401in}{2.056953in}}%
\pgfpathlineto{\pgfqpoint{10.723771in}{2.055971in}}%
\pgfpathlineto{\pgfqpoint{10.724455in}{2.055934in}}%
\pgfpathlineto{\pgfqpoint{10.724798in}{2.053998in}}%
\pgfpathlineto{\pgfqpoint{10.725140in}{2.053797in}}%
\pgfpathlineto{\pgfqpoint{10.725825in}{2.053223in}}%
\pgfpathlineto{\pgfqpoint{10.727879in}{2.050873in}}%
\pgfpathlineto{\pgfqpoint{10.728906in}{2.050410in}}%
\pgfpathlineto{\pgfqpoint{10.729933in}{2.050172in}}%
\pgfpathlineto{\pgfqpoint{10.730960in}{2.049658in}}%
\pgfpathlineto{\pgfqpoint{10.731987in}{2.048994in}}%
\pgfpathlineto{\pgfqpoint{10.732329in}{2.048994in}}%
\pgfpathlineto{\pgfqpoint{10.733014in}{2.048673in}}%
\pgfpathlineto{\pgfqpoint{10.733699in}{2.047212in}}%
\pgfpathlineto{\pgfqpoint{10.734726in}{2.046683in}}%
\pgfpathlineto{\pgfqpoint{10.735410in}{2.045973in}}%
\pgfpathlineto{\pgfqpoint{10.737122in}{2.045103in}}%
\pgfpathlineto{\pgfqpoint{10.738149in}{2.044366in}}%
\pgfpathlineto{\pgfqpoint{10.738834in}{2.043545in}}%
\pgfpathlineto{\pgfqpoint{10.739519in}{2.041737in}}%
\pgfpathlineto{\pgfqpoint{10.740888in}{2.040506in}}%
\pgfpathlineto{\pgfqpoint{10.741230in}{2.040453in}}%
\pgfpathlineto{\pgfqpoint{10.742600in}{2.038965in}}%
\pgfpathlineto{\pgfqpoint{10.742942in}{2.038875in}}%
\pgfpathlineto{\pgfqpoint{10.743284in}{2.038460in}}%
\pgfpathlineto{\pgfqpoint{10.743969in}{2.037285in}}%
\pgfpathlineto{\pgfqpoint{10.748420in}{2.034329in}}%
\pgfpathlineto{\pgfqpoint{10.749104in}{2.032723in}}%
\pgfpathlineto{\pgfqpoint{10.751501in}{2.030323in}}%
\pgfpathlineto{\pgfqpoint{10.753212in}{2.029632in}}%
\pgfpathlineto{\pgfqpoint{10.753897in}{2.028786in}}%
\pgfpathlineto{\pgfqpoint{10.755267in}{2.027985in}}%
\pgfpathlineto{\pgfqpoint{10.756294in}{2.026732in}}%
\pgfpathlineto{\pgfqpoint{10.756978in}{2.026563in}}%
\pgfpathlineto{\pgfqpoint{10.757663in}{2.025471in}}%
\pgfpathlineto{\pgfqpoint{10.758348in}{2.024708in}}%
\pgfpathlineto{\pgfqpoint{10.759032in}{2.024497in}}%
\pgfpathlineto{\pgfqpoint{10.761086in}{2.022302in}}%
\pgfpathlineto{\pgfqpoint{10.762113in}{2.021509in}}%
\pgfpathlineto{\pgfqpoint{10.762798in}{2.019452in}}%
\pgfpathlineto{\pgfqpoint{10.763141in}{2.019393in}}%
\pgfpathlineto{\pgfqpoint{10.766564in}{2.012952in}}%
\pgfpathlineto{\pgfqpoint{10.767249in}{2.012928in}}%
\pgfpathlineto{\pgfqpoint{10.768618in}{2.011040in}}%
\pgfpathlineto{\pgfqpoint{10.769303in}{2.010436in}}%
\pgfpathlineto{\pgfqpoint{10.769645in}{2.010118in}}%
\pgfpathlineto{\pgfqpoint{10.770330in}{2.008880in}}%
\pgfpathlineto{\pgfqpoint{10.770672in}{2.008819in}}%
\pgfpathlineto{\pgfqpoint{10.771014in}{2.008123in}}%
\pgfpathlineto{\pgfqpoint{10.772042in}{2.007827in}}%
\pgfpathlineto{\pgfqpoint{10.773069in}{2.006886in}}%
\pgfpathlineto{\pgfqpoint{10.773411in}{2.006768in}}%
\pgfpathlineto{\pgfqpoint{10.774096in}{2.005708in}}%
\pgfpathlineto{\pgfqpoint{10.775123in}{2.005086in}}%
\pgfpathlineto{\pgfqpoint{10.775807in}{2.004591in}}%
\pgfpathlineto{\pgfqpoint{10.777177in}{2.003886in}}%
\pgfpathlineto{\pgfqpoint{10.777519in}{2.002869in}}%
\pgfpathlineto{\pgfqpoint{10.778204in}{2.002431in}}%
\pgfpathlineto{\pgfqpoint{10.779573in}{2.001479in}}%
\pgfpathlineto{\pgfqpoint{10.781627in}{1.999903in}}%
\pgfpathlineto{\pgfqpoint{10.781970in}{1.999217in}}%
\pgfpathlineto{\pgfqpoint{10.783681in}{1.998858in}}%
\pgfpathlineto{\pgfqpoint{10.784366in}{1.998504in}}%
\pgfpathlineto{\pgfqpoint{10.784708in}{1.998398in}}%
\pgfpathlineto{\pgfqpoint{10.785393in}{1.997824in}}%
\pgfpathlineto{\pgfqpoint{10.786078in}{1.997643in}}%
\pgfpathlineto{\pgfqpoint{10.787447in}{1.996270in}}%
\pgfpathlineto{\pgfqpoint{10.788132in}{1.994502in}}%
\pgfpathlineto{\pgfqpoint{10.789159in}{1.993841in}}%
\pgfpathlineto{\pgfqpoint{10.789844in}{1.993444in}}%
\pgfpathlineto{\pgfqpoint{10.790528in}{1.992535in}}%
\pgfpathlineto{\pgfqpoint{10.792582in}{1.990876in}}%
\pgfpathlineto{\pgfqpoint{10.793609in}{1.989555in}}%
\pgfpathlineto{\pgfqpoint{10.794979in}{1.988128in}}%
\pgfpathlineto{\pgfqpoint{10.795321in}{1.987856in}}%
\pgfpathlineto{\pgfqpoint{10.796006in}{1.986437in}}%
\pgfpathlineto{\pgfqpoint{10.797718in}{1.985440in}}%
\pgfpathlineto{\pgfqpoint{10.798745in}{1.985307in}}%
\pgfpathlineto{\pgfqpoint{10.799429in}{1.984269in}}%
\pgfpathlineto{\pgfqpoint{10.800114in}{1.984232in}}%
\pgfpathlineto{\pgfqpoint{10.800799in}{1.983425in}}%
\pgfpathlineto{\pgfqpoint{10.802168in}{1.982933in}}%
\pgfpathlineto{\pgfqpoint{10.802853in}{1.982540in}}%
\pgfpathlineto{\pgfqpoint{10.803195in}{1.982532in}}%
\pgfpathlineto{\pgfqpoint{10.804564in}{1.981241in}}%
\pgfpathlineto{\pgfqpoint{10.805249in}{1.979701in}}%
\pgfpathlineto{\pgfqpoint{10.806276in}{1.977851in}}%
\pgfpathlineto{\pgfqpoint{10.806619in}{1.977851in}}%
\pgfpathlineto{\pgfqpoint{10.806961in}{1.976771in}}%
\pgfpathlineto{\pgfqpoint{10.807646in}{1.976604in}}%
\pgfpathlineto{\pgfqpoint{10.807988in}{1.975193in}}%
\pgfpathlineto{\pgfqpoint{10.808673in}{1.975019in}}%
\pgfpathlineto{\pgfqpoint{10.809015in}{1.973961in}}%
\pgfpathlineto{\pgfqpoint{10.809700in}{1.973933in}}%
\pgfpathlineto{\pgfqpoint{10.811069in}{1.972791in}}%
\pgfpathlineto{\pgfqpoint{10.812438in}{1.972225in}}%
\pgfpathlineto{\pgfqpoint{10.813465in}{1.970412in}}%
\pgfpathlineto{\pgfqpoint{10.813808in}{1.970397in}}%
\pgfpathlineto{\pgfqpoint{10.816547in}{1.967618in}}%
\pgfpathlineto{\pgfqpoint{10.817231in}{1.967165in}}%
\pgfpathlineto{\pgfqpoint{10.818258in}{1.966440in}}%
\pgfpathlineto{\pgfqpoint{10.819970in}{1.965919in}}%
\pgfpathlineto{\pgfqpoint{10.820655in}{1.965360in}}%
\pgfpathlineto{\pgfqpoint{10.821339in}{1.964628in}}%
\pgfpathlineto{\pgfqpoint{10.822024in}{1.964170in}}%
\pgfpathlineto{\pgfqpoint{10.822709in}{1.963087in}}%
\pgfpathlineto{\pgfqpoint{10.823394in}{1.962894in}}%
\pgfpathlineto{\pgfqpoint{10.824078in}{1.961577in}}%
\pgfpathlineto{\pgfqpoint{10.824421in}{1.961488in}}%
\pgfpathlineto{\pgfqpoint{10.825105in}{1.960248in}}%
\pgfpathlineto{\pgfqpoint{10.825790in}{1.959511in}}%
\pgfpathlineto{\pgfqpoint{10.826475in}{1.959198in}}%
\pgfpathlineto{\pgfqpoint{10.827159in}{1.958224in}}%
\pgfpathlineto{\pgfqpoint{10.827502in}{1.958141in}}%
\pgfpathlineto{\pgfqpoint{10.828871in}{1.956442in}}%
\pgfpathlineto{\pgfqpoint{10.829213in}{1.956381in}}%
\pgfpathlineto{\pgfqpoint{10.830583in}{1.954554in}}%
\pgfpathlineto{\pgfqpoint{10.833664in}{1.952893in}}%
\pgfpathlineto{\pgfqpoint{10.836745in}{1.950310in}}%
\pgfpathlineto{\pgfqpoint{10.837430in}{1.950068in}}%
\pgfpathlineto{\pgfqpoint{10.838114in}{1.948437in}}%
\pgfpathlineto{\pgfqpoint{10.839484in}{1.948019in}}%
\pgfpathlineto{\pgfqpoint{10.840169in}{1.947893in}}%
\pgfpathlineto{\pgfqpoint{10.841538in}{1.945687in}}%
\pgfpathlineto{\pgfqpoint{10.842223in}{1.945507in}}%
\pgfpathlineto{\pgfqpoint{10.842907in}{1.944571in}}%
\pgfpathlineto{\pgfqpoint{10.843934in}{1.944118in}}%
\pgfpathlineto{\pgfqpoint{10.844961in}{1.942925in}}%
\pgfpathlineto{\pgfqpoint{10.845304in}{1.942909in}}%
\pgfpathlineto{\pgfqpoint{10.845988in}{1.941263in}}%
\pgfpathlineto{\pgfqpoint{10.849070in}{1.939451in}}%
\pgfpathlineto{\pgfqpoint{10.849754in}{1.939436in}}%
\pgfpathlineto{\pgfqpoint{10.850439in}{1.938809in}}%
\pgfpathlineto{\pgfqpoint{10.852151in}{1.937774in}}%
\pgfpathlineto{\pgfqpoint{10.852835in}{1.936561in}}%
\pgfpathlineto{\pgfqpoint{10.853178in}{1.936455in}}%
\pgfpathlineto{\pgfqpoint{10.853520in}{1.935826in}}%
\pgfpathlineto{\pgfqpoint{10.854547in}{1.935605in}}%
\pgfpathlineto{\pgfqpoint{10.854889in}{1.934850in}}%
\pgfpathlineto{\pgfqpoint{10.855916in}{1.934686in}}%
\pgfpathlineto{\pgfqpoint{10.857286in}{1.931986in}}%
\pgfpathlineto{\pgfqpoint{10.857628in}{1.931793in}}%
\pgfpathlineto{\pgfqpoint{10.857971in}{1.929747in}}%
\pgfpathlineto{\pgfqpoint{10.860025in}{1.928018in}}%
\pgfpathlineto{\pgfqpoint{10.861736in}{1.926870in}}%
\pgfpathlineto{\pgfqpoint{10.862763in}{1.925692in}}%
\pgfpathlineto{\pgfqpoint{10.863790in}{1.925364in}}%
\pgfpathlineto{\pgfqpoint{10.864133in}{1.925360in}}%
\pgfpathlineto{\pgfqpoint{10.864817in}{1.924935in}}%
\pgfpathlineto{\pgfqpoint{10.865160in}{1.924846in}}%
\pgfpathlineto{\pgfqpoint{10.865845in}{1.924121in}}%
\pgfpathlineto{\pgfqpoint{10.866187in}{1.924114in}}%
\pgfpathlineto{\pgfqpoint{10.866529in}{1.922792in}}%
\pgfpathlineto{\pgfqpoint{10.867214in}{1.922581in}}%
\pgfpathlineto{\pgfqpoint{10.868926in}{1.920043in}}%
\pgfpathlineto{\pgfqpoint{10.869268in}{1.919763in}}%
\pgfpathlineto{\pgfqpoint{10.869953in}{1.918835in}}%
\pgfpathlineto{\pgfqpoint{10.871664in}{1.917816in}}%
\pgfpathlineto{\pgfqpoint{10.872349in}{1.916798in}}%
\pgfpathlineto{\pgfqpoint{10.874061in}{1.916388in}}%
\pgfpathlineto{\pgfqpoint{10.874746in}{1.915890in}}%
\pgfpathlineto{\pgfqpoint{10.875430in}{1.915633in}}%
\pgfpathlineto{\pgfqpoint{10.875773in}{1.914289in}}%
\pgfpathlineto{\pgfqpoint{10.877484in}{1.913851in}}%
\pgfpathlineto{\pgfqpoint{10.878169in}{1.913609in}}%
\pgfpathlineto{\pgfqpoint{10.879196in}{1.911847in}}%
\pgfpathlineto{\pgfqpoint{10.879538in}{1.911699in}}%
\pgfpathlineto{\pgfqpoint{10.880223in}{1.910770in}}%
\pgfpathlineto{\pgfqpoint{10.880565in}{1.910302in}}%
\pgfpathlineto{\pgfqpoint{10.880908in}{1.908958in}}%
\pgfpathlineto{\pgfqpoint{10.881592in}{1.908897in}}%
\pgfpathlineto{\pgfqpoint{10.882620in}{1.907848in}}%
\pgfpathlineto{\pgfqpoint{10.886043in}{1.906209in}}%
\pgfpathlineto{\pgfqpoint{10.887755in}{1.905960in}}%
\pgfpathlineto{\pgfqpoint{10.889124in}{1.904215in}}%
\pgfpathlineto{\pgfqpoint{10.889809in}{1.903996in}}%
\pgfpathlineto{\pgfqpoint{10.890493in}{1.903166in}}%
\pgfpathlineto{\pgfqpoint{10.890836in}{1.903143in}}%
\pgfpathlineto{\pgfqpoint{10.891178in}{1.902184in}}%
\pgfpathlineto{\pgfqpoint{10.891521in}{1.902108in}}%
\pgfpathlineto{\pgfqpoint{10.891863in}{1.901134in}}%
\pgfpathlineto{\pgfqpoint{10.893232in}{1.900590in}}%
\pgfpathlineto{\pgfqpoint{10.894259in}{1.900145in}}%
\pgfpathlineto{\pgfqpoint{10.894944in}{1.898606in}}%
\pgfpathlineto{\pgfqpoint{10.896998in}{1.896294in}}%
\pgfpathlineto{\pgfqpoint{10.897340in}{1.896143in}}%
\pgfpathlineto{\pgfqpoint{10.898025in}{1.895076in}}%
\pgfpathlineto{\pgfqpoint{10.900422in}{1.892465in}}%
\pgfpathlineto{\pgfqpoint{10.902133in}{1.890608in}}%
\pgfpathlineto{\pgfqpoint{10.902818in}{1.890479in}}%
\pgfpathlineto{\pgfqpoint{10.903160in}{1.889354in}}%
\pgfpathlineto{\pgfqpoint{10.903503in}{1.889203in}}%
\pgfpathlineto{\pgfqpoint{10.904187in}{1.887964in}}%
\pgfpathlineto{\pgfqpoint{10.905899in}{1.887481in}}%
\pgfpathlineto{\pgfqpoint{10.906241in}{1.887383in}}%
\pgfpathlineto{\pgfqpoint{10.907611in}{1.885669in}}%
\pgfpathlineto{\pgfqpoint{10.908296in}{1.885379in}}%
\pgfpathlineto{\pgfqpoint{10.909323in}{1.885064in}}%
\pgfpathlineto{\pgfqpoint{10.909665in}{1.884022in}}%
\pgfpathlineto{\pgfqpoint{10.910350in}{1.883890in}}%
\pgfpathlineto{\pgfqpoint{10.911034in}{1.882852in}}%
\pgfpathlineto{\pgfqpoint{10.913088in}{1.881493in}}%
\pgfpathlineto{\pgfqpoint{10.913773in}{1.880684in}}%
\pgfpathlineto{\pgfqpoint{10.914800in}{1.880128in}}%
\pgfpathlineto{\pgfqpoint{10.915142in}{1.879152in}}%
\pgfpathlineto{\pgfqpoint{10.917539in}{1.878170in}}%
\pgfpathlineto{\pgfqpoint{10.918224in}{1.877633in}}%
\pgfpathlineto{\pgfqpoint{10.918908in}{1.877392in}}%
\pgfpathlineto{\pgfqpoint{10.919593in}{1.877150in}}%
\pgfpathlineto{\pgfqpoint{10.919935in}{1.876546in}}%
\pgfpathlineto{\pgfqpoint{10.920278in}{1.876508in}}%
\pgfpathlineto{\pgfqpoint{10.920962in}{1.875187in}}%
\pgfpathlineto{\pgfqpoint{10.921305in}{1.875096in}}%
\pgfpathlineto{\pgfqpoint{10.921647in}{1.874234in}}%
\pgfpathlineto{\pgfqpoint{10.923701in}{1.873465in}}%
\pgfpathlineto{\pgfqpoint{10.924386in}{1.872982in}}%
\pgfpathlineto{\pgfqpoint{10.926440in}{1.871638in}}%
\pgfpathlineto{\pgfqpoint{10.926782in}{1.871487in}}%
\pgfpathlineto{\pgfqpoint{10.927467in}{1.870682in}}%
\pgfpathlineto{\pgfqpoint{10.931233in}{1.869659in}}%
\pgfpathlineto{\pgfqpoint{10.931917in}{1.869057in}}%
\pgfpathlineto{\pgfqpoint{10.932602in}{1.868633in}}%
\pgfpathlineto{\pgfqpoint{10.933972in}{1.866654in}}%
\pgfpathlineto{\pgfqpoint{10.934314in}{1.866578in}}%
\pgfpathlineto{\pgfqpoint{10.936026in}{1.864291in}}%
\pgfpathlineto{\pgfqpoint{10.938422in}{1.862425in}}%
\pgfpathlineto{\pgfqpoint{10.942873in}{1.857176in}}%
\pgfpathlineto{\pgfqpoint{10.946638in}{1.854888in}}%
\pgfpathlineto{\pgfqpoint{10.947665in}{1.854410in}}%
\pgfpathlineto{\pgfqpoint{10.950747in}{1.852419in}}%
\pgfpathlineto{\pgfqpoint{10.952116in}{1.850750in}}%
\pgfpathlineto{\pgfqpoint{10.952458in}{1.850629in}}%
\pgfpathlineto{\pgfqpoint{10.953485in}{1.848968in}}%
\pgfpathlineto{\pgfqpoint{10.954512in}{1.848636in}}%
\pgfpathlineto{\pgfqpoint{10.955197in}{1.848303in}}%
\pgfpathlineto{\pgfqpoint{10.955882in}{1.846869in}}%
\pgfpathlineto{\pgfqpoint{10.956566in}{1.846085in}}%
\pgfpathlineto{\pgfqpoint{10.957936in}{1.844742in}}%
\pgfpathlineto{\pgfqpoint{10.958620in}{1.843621in}}%
\pgfpathlineto{\pgfqpoint{10.959648in}{1.842300in}}%
\pgfpathlineto{\pgfqpoint{10.960332in}{1.841265in}}%
\pgfpathlineto{\pgfqpoint{10.961017in}{1.840330in}}%
\pgfpathlineto{\pgfqpoint{10.962386in}{1.839808in}}%
\pgfpathlineto{\pgfqpoint{10.962729in}{1.839100in}}%
\pgfpathlineto{\pgfqpoint{10.963071in}{1.839090in}}%
\pgfpathlineto{\pgfqpoint{10.963756in}{1.837693in}}%
\pgfpathlineto{\pgfqpoint{10.964440in}{1.836598in}}%
\pgfpathlineto{\pgfqpoint{10.965125in}{1.836296in}}%
\pgfpathlineto{\pgfqpoint{10.966152in}{1.834439in}}%
\pgfpathlineto{\pgfqpoint{10.966837in}{1.834033in}}%
\pgfpathlineto{\pgfqpoint{10.967179in}{1.833644in}}%
\pgfpathlineto{\pgfqpoint{10.967864in}{1.831954in}}%
\pgfpathlineto{\pgfqpoint{10.968206in}{1.831841in}}%
\pgfpathlineto{\pgfqpoint{10.968891in}{1.830754in}}%
\pgfpathlineto{\pgfqpoint{10.969576in}{1.830089in}}%
\pgfpathlineto{\pgfqpoint{10.970603in}{1.828767in}}%
\pgfpathlineto{\pgfqpoint{10.971287in}{1.827424in}}%
\pgfpathlineto{\pgfqpoint{10.972314in}{1.826464in}}%
\pgfpathlineto{\pgfqpoint{10.972657in}{1.826404in}}%
\pgfpathlineto{\pgfqpoint{10.974711in}{1.824350in}}%
\pgfpathlineto{\pgfqpoint{10.975738in}{1.824018in}}%
\pgfpathlineto{\pgfqpoint{10.976423in}{1.823496in}}%
\pgfpathlineto{\pgfqpoint{10.977450in}{1.823111in}}%
\pgfpathlineto{\pgfqpoint{10.978134in}{1.821722in}}%
\pgfpathlineto{\pgfqpoint{10.980188in}{1.821042in}}%
\pgfpathlineto{\pgfqpoint{10.980873in}{1.819796in}}%
\pgfpathlineto{\pgfqpoint{10.981558in}{1.819532in}}%
\pgfpathlineto{\pgfqpoint{10.981900in}{1.818792in}}%
\pgfpathlineto{\pgfqpoint{10.982242in}{1.818777in}}%
\pgfpathlineto{\pgfqpoint{10.982585in}{1.817742in}}%
\pgfpathlineto{\pgfqpoint{10.983269in}{1.817523in}}%
\pgfpathlineto{\pgfqpoint{10.984639in}{1.816518in}}%
\pgfpathlineto{\pgfqpoint{10.984981in}{1.816348in}}%
\pgfpathlineto{\pgfqpoint{10.985666in}{1.815235in}}%
\pgfpathlineto{\pgfqpoint{10.987720in}{1.814441in}}%
\pgfpathlineto{\pgfqpoint{10.990801in}{1.809277in}}%
\pgfpathlineto{\pgfqpoint{10.991486in}{1.809115in}}%
\pgfpathlineto{\pgfqpoint{10.991828in}{1.807751in}}%
\pgfpathlineto{\pgfqpoint{10.992170in}{1.807638in}}%
\pgfpathlineto{\pgfqpoint{10.992855in}{1.806357in}}%
\pgfpathlineto{\pgfqpoint{10.994225in}{1.805750in}}%
\pgfpathlineto{\pgfqpoint{10.995936in}{1.803115in}}%
\pgfpathlineto{\pgfqpoint{10.996621in}{1.802901in}}%
\pgfpathlineto{\pgfqpoint{10.996963in}{1.801846in}}%
\pgfpathlineto{\pgfqpoint{10.997990in}{1.801521in}}%
\pgfpathlineto{\pgfqpoint{11.000387in}{1.799581in}}%
\pgfpathlineto{\pgfqpoint{11.001071in}{1.799134in}}%
\pgfpathlineto{\pgfqpoint{11.001414in}{1.799007in}}%
\pgfpathlineto{\pgfqpoint{11.002441in}{1.798070in}}%
\pgfpathlineto{\pgfqpoint{11.003810in}{1.797708in}}%
\pgfpathlineto{\pgfqpoint{11.004153in}{1.797406in}}%
\pgfpathlineto{\pgfqpoint{11.005522in}{1.794899in}}%
\pgfpathlineto{\pgfqpoint{11.006207in}{1.794763in}}%
\pgfpathlineto{\pgfqpoint{11.006549in}{1.793781in}}%
\pgfpathlineto{\pgfqpoint{11.006891in}{1.793690in}}%
\pgfpathlineto{\pgfqpoint{11.007576in}{1.792112in}}%
\pgfpathlineto{\pgfqpoint{11.009288in}{1.791478in}}%
\pgfpathlineto{\pgfqpoint{11.014423in}{1.787770in}}%
\pgfpathlineto{\pgfqpoint{11.014765in}{1.787679in}}%
\pgfpathlineto{\pgfqpoint{11.015450in}{1.786834in}}%
\pgfpathlineto{\pgfqpoint{11.016135in}{1.786456in}}%
\pgfpathlineto{\pgfqpoint{11.016819in}{1.786018in}}%
\pgfpathlineto{\pgfqpoint{11.017504in}{1.785701in}}%
\pgfpathlineto{\pgfqpoint{11.017846in}{1.785595in}}%
\pgfpathlineto{\pgfqpoint{11.018531in}{1.784291in}}%
\pgfpathlineto{\pgfqpoint{11.019216in}{1.784105in}}%
\pgfpathlineto{\pgfqpoint{11.020928in}{1.782670in}}%
\pgfpathlineto{\pgfqpoint{11.021270in}{1.782514in}}%
\pgfpathlineto{\pgfqpoint{11.021955in}{1.780943in}}%
\pgfpathlineto{\pgfqpoint{11.022982in}{1.780037in}}%
\pgfpathlineto{\pgfqpoint{11.023666in}{1.779630in}}%
\pgfpathlineto{\pgfqpoint{11.024351in}{1.778753in}}%
\pgfpathlineto{\pgfqpoint{11.026405in}{1.777658in}}%
\pgfpathlineto{\pgfqpoint{11.027090in}{1.776633in}}%
\pgfpathlineto{\pgfqpoint{11.028117in}{1.775265in}}%
\pgfpathlineto{\pgfqpoint{11.028459in}{1.775174in}}%
\pgfpathlineto{\pgfqpoint{11.029144in}{1.774112in}}%
\pgfpathlineto{\pgfqpoint{11.029486in}{1.773029in}}%
\pgfpathlineto{\pgfqpoint{11.029829in}{1.772939in}}%
\pgfpathlineto{\pgfqpoint{11.030513in}{1.772002in}}%
\pgfpathlineto{\pgfqpoint{11.031883in}{1.771126in}}%
\pgfpathlineto{\pgfqpoint{11.032567in}{1.770703in}}%
\pgfpathlineto{\pgfqpoint{11.034279in}{1.770220in}}%
\pgfpathlineto{\pgfqpoint{11.035306in}{1.769004in}}%
\pgfpathlineto{\pgfqpoint{11.037018in}{1.768529in}}%
\pgfpathlineto{\pgfqpoint{11.037703in}{1.768079in}}%
\pgfpathlineto{\pgfqpoint{11.038045in}{1.766511in}}%
\pgfpathlineto{\pgfqpoint{11.038730in}{1.766293in}}%
\pgfpathlineto{\pgfqpoint{11.039414in}{1.766112in}}%
\pgfpathlineto{\pgfqpoint{11.040441in}{1.765177in}}%
\pgfpathlineto{\pgfqpoint{11.042838in}{1.762834in}}%
\pgfpathlineto{\pgfqpoint{11.043180in}{1.762820in}}%
\pgfpathlineto{\pgfqpoint{11.043865in}{1.762331in}}%
\pgfpathlineto{\pgfqpoint{11.044207in}{1.762246in}}%
\pgfpathlineto{\pgfqpoint{11.046261in}{1.759818in}}%
\pgfpathlineto{\pgfqpoint{11.046604in}{1.759746in}}%
\pgfpathlineto{\pgfqpoint{11.047973in}{1.756174in}}%
\pgfpathlineto{\pgfqpoint{11.048658in}{1.755985in}}%
\pgfpathlineto{\pgfqpoint{11.049000in}{1.754109in}}%
\pgfpathlineto{\pgfqpoint{11.050027in}{1.753525in}}%
\pgfpathlineto{\pgfqpoint{11.050369in}{1.753493in}}%
\pgfpathlineto{\pgfqpoint{11.051054in}{1.752532in}}%
\pgfpathlineto{\pgfqpoint{11.052081in}{1.751530in}}%
\pgfpathlineto{\pgfqpoint{11.053108in}{1.750538in}}%
\pgfpathlineto{\pgfqpoint{11.053451in}{1.749537in}}%
\pgfpathlineto{\pgfqpoint{11.053793in}{1.749529in}}%
\pgfpathlineto{\pgfqpoint{11.054135in}{1.748049in}}%
\pgfpathlineto{\pgfqpoint{11.054478in}{1.747837in}}%
\pgfpathlineto{\pgfqpoint{11.055847in}{1.744968in}}%
\pgfpathlineto{\pgfqpoint{11.056874in}{1.744394in}}%
\pgfpathlineto{\pgfqpoint{11.058243in}{1.743978in}}%
\pgfpathlineto{\pgfqpoint{11.058928in}{1.743155in}}%
\pgfpathlineto{\pgfqpoint{11.060297in}{1.741524in}}%
\pgfpathlineto{\pgfqpoint{11.061667in}{1.740753in}}%
\pgfpathlineto{\pgfqpoint{11.062009in}{1.740694in}}%
\pgfpathlineto{\pgfqpoint{11.063721in}{1.738919in}}%
\pgfpathlineto{\pgfqpoint{11.064406in}{1.738775in}}%
\pgfpathlineto{\pgfqpoint{11.065775in}{1.736200in}}%
\pgfpathlineto{\pgfqpoint{11.066802in}{1.735370in}}%
\pgfpathlineto{\pgfqpoint{11.067487in}{1.734426in}}%
\pgfpathlineto{\pgfqpoint{11.068514in}{1.734086in}}%
\pgfpathlineto{\pgfqpoint{11.069198in}{1.733376in}}%
\pgfpathlineto{\pgfqpoint{11.070568in}{1.732553in}}%
\pgfpathlineto{\pgfqpoint{11.071253in}{1.731375in}}%
\pgfpathlineto{\pgfqpoint{11.071937in}{1.731330in}}%
\pgfpathlineto{\pgfqpoint{11.074334in}{1.728868in}}%
\pgfpathlineto{\pgfqpoint{11.074676in}{1.728838in}}%
\pgfpathlineto{\pgfqpoint{11.075361in}{1.727905in}}%
\pgfpathlineto{\pgfqpoint{11.075703in}{1.727888in}}%
\pgfpathlineto{\pgfqpoint{11.076388in}{1.726648in}}%
\pgfpathlineto{\pgfqpoint{11.077757in}{1.726119in}}%
\pgfpathlineto{\pgfqpoint{11.078442in}{1.725696in}}%
\pgfpathlineto{\pgfqpoint{11.078784in}{1.724276in}}%
\pgfpathlineto{\pgfqpoint{11.080154in}{1.723521in}}%
\pgfpathlineto{\pgfqpoint{11.081181in}{1.721920in}}%
\pgfpathlineto{\pgfqpoint{11.081865in}{1.720771in}}%
\pgfpathlineto{\pgfqpoint{11.082892in}{1.720470in}}%
\pgfpathlineto{\pgfqpoint{11.085289in}{1.718054in}}%
\pgfpathlineto{\pgfqpoint{11.086316in}{1.717624in}}%
\pgfpathlineto{\pgfqpoint{11.086658in}{1.717397in}}%
\pgfpathlineto{\pgfqpoint{11.087343in}{1.716534in}}%
\pgfpathlineto{\pgfqpoint{11.087685in}{1.716491in}}%
\pgfpathlineto{\pgfqpoint{11.088370in}{1.715728in}}%
\pgfpathlineto{\pgfqpoint{11.088712in}{1.715728in}}%
\pgfpathlineto{\pgfqpoint{11.089397in}{1.714867in}}%
\pgfpathlineto{\pgfqpoint{11.090082in}{1.714301in}}%
\pgfpathlineto{\pgfqpoint{11.091793in}{1.713704in}}%
\pgfpathlineto{\pgfqpoint{11.092820in}{1.712710in}}%
\pgfpathlineto{\pgfqpoint{11.093163in}{1.712677in}}%
\pgfpathlineto{\pgfqpoint{11.093505in}{1.712164in}}%
\pgfpathlineto{\pgfqpoint{11.094190in}{1.710533in}}%
\pgfpathlineto{\pgfqpoint{11.096244in}{1.709052in}}%
\pgfpathlineto{\pgfqpoint{11.096586in}{1.706787in}}%
\pgfpathlineto{\pgfqpoint{11.101379in}{1.703313in}}%
\pgfpathlineto{\pgfqpoint{11.103091in}{1.702868in}}%
\pgfpathlineto{\pgfqpoint{11.104118in}{1.701596in}}%
\pgfpathlineto{\pgfqpoint{11.104803in}{1.701230in}}%
\pgfpathlineto{\pgfqpoint{11.106514in}{1.700504in}}%
\pgfpathlineto{\pgfqpoint{11.106857in}{1.699511in}}%
\pgfpathlineto{\pgfqpoint{11.107199in}{1.699477in}}%
\pgfpathlineto{\pgfqpoint{11.108226in}{1.698192in}}%
\pgfpathlineto{\pgfqpoint{11.110965in}{1.696034in}}%
\pgfpathlineto{\pgfqpoint{11.111649in}{1.695875in}}%
\pgfpathlineto{\pgfqpoint{11.112334in}{1.695158in}}%
\pgfpathlineto{\pgfqpoint{11.114046in}{1.693708in}}%
\pgfpathlineto{\pgfqpoint{11.114388in}{1.692801in}}%
\pgfpathlineto{\pgfqpoint{11.115758in}{1.692560in}}%
\pgfpathlineto{\pgfqpoint{11.118496in}{1.691382in}}%
\pgfpathlineto{\pgfqpoint{11.119523in}{1.690355in}}%
\pgfpathlineto{\pgfqpoint{11.119866in}{1.690325in}}%
\pgfpathlineto{\pgfqpoint{11.120550in}{1.689834in}}%
\pgfpathlineto{\pgfqpoint{11.121578in}{1.689449in}}%
\pgfpathlineto{\pgfqpoint{11.121920in}{1.688814in}}%
\pgfpathlineto{\pgfqpoint{11.122605in}{1.686964in}}%
\pgfpathlineto{\pgfqpoint{11.123632in}{1.686209in}}%
\pgfpathlineto{\pgfqpoint{11.124659in}{1.685718in}}%
\pgfpathlineto{\pgfqpoint{11.125343in}{1.685114in}}%
\pgfpathlineto{\pgfqpoint{11.126028in}{1.683135in}}%
\pgfpathlineto{\pgfqpoint{11.126370in}{1.683037in}}%
\pgfpathlineto{\pgfqpoint{11.127055in}{1.681353in}}%
\pgfpathlineto{\pgfqpoint{11.127740in}{1.680958in}}%
\pgfpathlineto{\pgfqpoint{11.129794in}{1.680357in}}%
\pgfpathlineto{\pgfqpoint{11.130821in}{1.679111in}}%
\pgfpathlineto{\pgfqpoint{11.131506in}{1.677819in}}%
\pgfpathlineto{\pgfqpoint{11.133217in}{1.677366in}}%
\pgfpathlineto{\pgfqpoint{11.133902in}{1.677185in}}%
\pgfpathlineto{\pgfqpoint{11.134587in}{1.676316in}}%
\pgfpathlineto{\pgfqpoint{11.134929in}{1.676279in}}%
\pgfpathlineto{\pgfqpoint{11.135614in}{1.675739in}}%
\pgfpathlineto{\pgfqpoint{11.135956in}{1.674345in}}%
\pgfpathlineto{\pgfqpoint{11.137325in}{1.673530in}}%
\pgfpathlineto{\pgfqpoint{11.138695in}{1.670237in}}%
\pgfpathlineto{\pgfqpoint{11.139722in}{1.669815in}}%
\pgfpathlineto{\pgfqpoint{11.140064in}{1.669815in}}%
\pgfpathlineto{\pgfqpoint{11.140749in}{1.668917in}}%
\pgfpathlineto{\pgfqpoint{11.141776in}{1.667428in}}%
\pgfpathlineto{\pgfqpoint{11.142118in}{1.667406in}}%
\pgfpathlineto{\pgfqpoint{11.142461in}{1.666839in}}%
\pgfpathlineto{\pgfqpoint{11.143145in}{1.664740in}}%
\pgfpathlineto{\pgfqpoint{11.143488in}{1.664528in}}%
\pgfpathlineto{\pgfqpoint{11.144172in}{1.663411in}}%
\pgfpathlineto{\pgfqpoint{11.144515in}{1.663328in}}%
\pgfpathlineto{\pgfqpoint{11.145884in}{1.661206in}}%
\pgfpathlineto{\pgfqpoint{11.146226in}{1.661145in}}%
\pgfpathlineto{\pgfqpoint{11.146911in}{1.660605in}}%
\pgfpathlineto{\pgfqpoint{11.147938in}{1.659846in}}%
\pgfpathlineto{\pgfqpoint{11.148623in}{1.659025in}}%
\pgfpathlineto{\pgfqpoint{11.149308in}{1.658850in}}%
\pgfpathlineto{\pgfqpoint{11.149650in}{1.658155in}}%
\pgfpathlineto{\pgfqpoint{11.150677in}{1.657732in}}%
\pgfpathlineto{\pgfqpoint{11.151704in}{1.657188in}}%
\pgfpathlineto{\pgfqpoint{11.152046in}{1.657067in}}%
\pgfpathlineto{\pgfqpoint{11.152731in}{1.655829in}}%
\pgfpathlineto{\pgfqpoint{11.153758in}{1.655164in}}%
\pgfpathlineto{\pgfqpoint{11.154785in}{1.654618in}}%
\pgfpathlineto{\pgfqpoint{11.155127in}{1.654473in}}%
\pgfpathlineto{\pgfqpoint{11.155470in}{1.653715in}}%
\pgfpathlineto{\pgfqpoint{11.156155in}{1.653503in}}%
\pgfpathlineto{\pgfqpoint{11.157182in}{1.652302in}}%
\pgfpathlineto{\pgfqpoint{11.157866in}{1.651993in}}%
\pgfpathlineto{\pgfqpoint{11.158551in}{1.650724in}}%
\pgfpathlineto{\pgfqpoint{11.158893in}{1.650483in}}%
\pgfpathlineto{\pgfqpoint{11.159578in}{1.648376in}}%
\pgfpathlineto{\pgfqpoint{11.160263in}{1.648111in}}%
\pgfpathlineto{\pgfqpoint{11.160947in}{1.647704in}}%
\pgfpathlineto{\pgfqpoint{11.161290in}{1.647696in}}%
\pgfpathlineto{\pgfqpoint{11.161974in}{1.647016in}}%
\pgfpathlineto{\pgfqpoint{11.162317in}{1.645015in}}%
\pgfpathlineto{\pgfqpoint{11.163344in}{1.644441in}}%
\pgfpathlineto{\pgfqpoint{11.164028in}{1.644109in}}%
\pgfpathlineto{\pgfqpoint{11.165740in}{1.641662in}}%
\pgfpathlineto{\pgfqpoint{11.166767in}{1.639314in}}%
\pgfpathlineto{\pgfqpoint{11.167452in}{1.638861in}}%
\pgfpathlineto{\pgfqpoint{11.167794in}{1.638823in}}%
\pgfpathlineto{\pgfqpoint{11.168479in}{1.636497in}}%
\pgfpathlineto{\pgfqpoint{11.169848in}{1.636029in}}%
\pgfpathlineto{\pgfqpoint{11.172930in}{1.634005in}}%
\pgfpathlineto{\pgfqpoint{11.173614in}{1.632510in}}%
\pgfpathlineto{\pgfqpoint{11.173957in}{1.632291in}}%
\pgfpathlineto{\pgfqpoint{11.174299in}{1.631422in}}%
\pgfpathlineto{\pgfqpoint{11.174641in}{1.631385in}}%
\pgfpathlineto{\pgfqpoint{11.175326in}{1.629534in}}%
\pgfpathlineto{\pgfqpoint{11.175668in}{1.628364in}}%
\pgfpathlineto{\pgfqpoint{11.176011in}{1.628276in}}%
\pgfpathlineto{\pgfqpoint{11.176353in}{1.627888in}}%
\pgfpathlineto{\pgfqpoint{11.177038in}{1.626854in}}%
\pgfpathlineto{\pgfqpoint{11.177722in}{1.626197in}}%
\pgfpathlineto{\pgfqpoint{11.178065in}{1.626190in}}%
\pgfpathlineto{\pgfqpoint{11.178749in}{1.625532in}}%
\pgfpathlineto{\pgfqpoint{11.179092in}{1.625260in}}%
\pgfpathlineto{\pgfqpoint{11.179434in}{1.624566in}}%
\pgfpathlineto{\pgfqpoint{11.181146in}{1.624082in}}%
\pgfpathlineto{\pgfqpoint{11.181488in}{1.623153in}}%
\pgfpathlineto{\pgfqpoint{11.181831in}{1.623085in}}%
\pgfpathlineto{\pgfqpoint{11.182515in}{1.621484in}}%
\pgfpathlineto{\pgfqpoint{11.182858in}{1.619612in}}%
\pgfpathlineto{\pgfqpoint{11.185254in}{1.617709in}}%
\pgfpathlineto{\pgfqpoint{11.185939in}{1.616500in}}%
\pgfpathlineto{\pgfqpoint{11.186966in}{1.615685in}}%
\pgfpathlineto{\pgfqpoint{11.187650in}{1.614092in}}%
\pgfpathlineto{\pgfqpoint{11.188677in}{1.613661in}}%
\pgfpathlineto{\pgfqpoint{11.189704in}{1.612151in}}%
\pgfpathlineto{\pgfqpoint{11.190389in}{1.611949in}}%
\pgfpathlineto{\pgfqpoint{11.191074in}{1.611396in}}%
\pgfpathlineto{\pgfqpoint{11.191416in}{1.611335in}}%
\pgfpathlineto{\pgfqpoint{11.192101in}{1.610852in}}%
\pgfpathlineto{\pgfqpoint{11.192443in}{1.610769in}}%
\pgfpathlineto{\pgfqpoint{11.193128in}{1.610157in}}%
\pgfpathlineto{\pgfqpoint{11.193813in}{1.609987in}}%
\pgfpathlineto{\pgfqpoint{11.194155in}{1.608435in}}%
\pgfpathlineto{\pgfqpoint{11.194497in}{1.608345in}}%
\pgfpathlineto{\pgfqpoint{11.194840in}{1.606407in}}%
\pgfpathlineto{\pgfqpoint{11.196209in}{1.605656in}}%
\pgfpathlineto{\pgfqpoint{11.196894in}{1.605214in}}%
\pgfpathlineto{\pgfqpoint{11.197236in}{1.605113in}}%
\pgfpathlineto{\pgfqpoint{11.197578in}{1.604297in}}%
\pgfpathlineto{\pgfqpoint{11.198263in}{1.604207in}}%
\pgfpathlineto{\pgfqpoint{11.198606in}{1.603935in}}%
\pgfpathlineto{\pgfqpoint{11.199290in}{1.602485in}}%
\pgfpathlineto{\pgfqpoint{11.200317in}{1.601095in}}%
\pgfpathlineto{\pgfqpoint{11.202371in}{1.600491in}}%
\pgfpathlineto{\pgfqpoint{11.203398in}{1.599464in}}%
\pgfpathlineto{\pgfqpoint{11.203741in}{1.598120in}}%
\pgfpathlineto{\pgfqpoint{11.204425in}{1.597907in}}%
\pgfpathlineto{\pgfqpoint{11.205110in}{1.597410in}}%
\pgfpathlineto{\pgfqpoint{11.205452in}{1.597385in}}%
\pgfpathlineto{\pgfqpoint{11.205795in}{1.596157in}}%
\pgfpathlineto{\pgfqpoint{11.206822in}{1.595779in}}%
\pgfpathlineto{\pgfqpoint{11.207164in}{1.595672in}}%
\pgfpathlineto{\pgfqpoint{11.208191in}{1.594669in}}%
\pgfpathlineto{\pgfqpoint{11.208876in}{1.594118in}}%
\pgfpathlineto{\pgfqpoint{11.209561in}{1.592698in}}%
\pgfpathlineto{\pgfqpoint{11.210588in}{1.592094in}}%
\pgfpathlineto{\pgfqpoint{11.211272in}{1.591475in}}%
\pgfpathlineto{\pgfqpoint{11.211615in}{1.591297in}}%
\pgfpathlineto{\pgfqpoint{11.211957in}{1.590327in}}%
\pgfpathlineto{\pgfqpoint{11.212299in}{1.590229in}}%
\pgfpathlineto{\pgfqpoint{11.213669in}{1.588718in}}%
\pgfpathlineto{\pgfqpoint{11.215723in}{1.587291in}}%
\pgfpathlineto{\pgfqpoint{11.216065in}{1.586929in}}%
\pgfpathlineto{\pgfqpoint{11.217092in}{1.583348in}}%
\pgfpathlineto{\pgfqpoint{11.219831in}{1.582035in}}%
\pgfpathlineto{\pgfqpoint{11.220858in}{1.581008in}}%
\pgfpathlineto{\pgfqpoint{11.221543in}{1.580714in}}%
\pgfpathlineto{\pgfqpoint{11.221885in}{1.580260in}}%
\pgfpathlineto{\pgfqpoint{11.222227in}{1.579322in}}%
\pgfpathlineto{\pgfqpoint{11.223254in}{1.579045in}}%
\pgfpathlineto{\pgfqpoint{11.223939in}{1.578335in}}%
\pgfpathlineto{\pgfqpoint{11.225651in}{1.577867in}}%
\pgfpathlineto{\pgfqpoint{11.226336in}{1.576400in}}%
\pgfpathlineto{\pgfqpoint{11.227705in}{1.574755in}}%
\pgfpathlineto{\pgfqpoint{11.229759in}{1.573911in}}%
\pgfpathlineto{\pgfqpoint{11.230444in}{1.572792in}}%
\pgfpathlineto{\pgfqpoint{11.231128in}{1.572399in}}%
\pgfpathlineto{\pgfqpoint{11.231471in}{1.572097in}}%
\pgfpathlineto{\pgfqpoint{11.232155in}{1.571048in}}%
\pgfpathlineto{\pgfqpoint{11.233183in}{1.569943in}}%
\pgfpathlineto{\pgfqpoint{11.234552in}{1.569084in}}%
\pgfpathlineto{\pgfqpoint{11.234894in}{1.568926in}}%
\pgfpathlineto{\pgfqpoint{11.235579in}{1.567234in}}%
\pgfpathlineto{\pgfqpoint{11.236264in}{1.566418in}}%
\pgfpathlineto{\pgfqpoint{11.239002in}{1.563639in}}%
\pgfpathlineto{\pgfqpoint{11.239687in}{1.563420in}}%
\pgfpathlineto{\pgfqpoint{11.240029in}{1.562107in}}%
\pgfpathlineto{\pgfqpoint{11.240372in}{1.561948in}}%
\pgfpathlineto{\pgfqpoint{11.241741in}{1.559834in}}%
\pgfpathlineto{\pgfqpoint{11.242426in}{1.559531in}}%
\pgfpathlineto{\pgfqpoint{11.242768in}{1.557115in}}%
\pgfpathlineto{\pgfqpoint{11.243453in}{1.556511in}}%
\pgfpathlineto{\pgfqpoint{11.244480in}{1.554736in}}%
\pgfpathlineto{\pgfqpoint{11.245165in}{1.554427in}}%
\pgfpathlineto{\pgfqpoint{11.245507in}{1.554245in}}%
\pgfpathlineto{\pgfqpoint{11.246192in}{1.553037in}}%
\pgfpathlineto{\pgfqpoint{11.247219in}{1.552191in}}%
\pgfpathlineto{\pgfqpoint{11.248246in}{1.551716in}}%
\pgfpathlineto{\pgfqpoint{11.248930in}{1.551236in}}%
\pgfpathlineto{\pgfqpoint{11.249273in}{1.551149in}}%
\pgfpathlineto{\pgfqpoint{11.249958in}{1.550681in}}%
\pgfpathlineto{\pgfqpoint{11.250300in}{1.550503in}}%
\pgfpathlineto{\pgfqpoint{11.250985in}{1.548446in}}%
\pgfpathlineto{\pgfqpoint{11.251669in}{1.547611in}}%
\pgfpathlineto{\pgfqpoint{11.252696in}{1.547185in}}%
\pgfpathlineto{\pgfqpoint{11.253039in}{1.546513in}}%
\pgfpathlineto{\pgfqpoint{11.253723in}{1.546301in}}%
\pgfpathlineto{\pgfqpoint{11.254066in}{1.546150in}}%
\pgfpathlineto{\pgfqpoint{11.255093in}{1.543764in}}%
\pgfpathlineto{\pgfqpoint{11.255777in}{1.543613in}}%
\pgfpathlineto{\pgfqpoint{11.256804in}{1.542525in}}%
\pgfpathlineto{\pgfqpoint{11.257147in}{1.542495in}}%
\pgfpathlineto{\pgfqpoint{11.258174in}{1.541136in}}%
\pgfpathlineto{\pgfqpoint{11.258859in}{1.540683in}}%
\pgfpathlineto{\pgfqpoint{11.259201in}{1.540350in}}%
\pgfpathlineto{\pgfqpoint{11.259886in}{1.538296in}}%
\pgfpathlineto{\pgfqpoint{11.260913in}{1.536242in}}%
\pgfpathlineto{\pgfqpoint{11.261255in}{1.536046in}}%
\pgfpathlineto{\pgfqpoint{11.262282in}{1.534762in}}%
\pgfpathlineto{\pgfqpoint{11.262967in}{1.534521in}}%
\pgfpathlineto{\pgfqpoint{11.263651in}{1.532859in}}%
\pgfpathlineto{\pgfqpoint{11.264336in}{1.532346in}}%
\pgfpathlineto{\pgfqpoint{11.265021in}{1.531051in}}%
\pgfpathlineto{\pgfqpoint{11.265363in}{1.530979in}}%
\pgfpathlineto{\pgfqpoint{11.266048in}{1.530609in}}%
\pgfpathlineto{\pgfqpoint{11.267417in}{1.529552in}}%
\pgfpathlineto{\pgfqpoint{11.267760in}{1.529537in}}%
\pgfpathlineto{\pgfqpoint{11.268787in}{1.527211in}}%
\pgfpathlineto{\pgfqpoint{11.269471in}{1.527135in}}%
\pgfpathlineto{\pgfqpoint{11.270498in}{1.526365in}}%
\pgfpathlineto{\pgfqpoint{11.270841in}{1.525172in}}%
\pgfpathlineto{\pgfqpoint{11.271183in}{1.525096in}}%
\pgfpathlineto{\pgfqpoint{11.272210in}{1.523858in}}%
\pgfpathlineto{\pgfqpoint{11.273579in}{1.523375in}}%
\pgfpathlineto{\pgfqpoint{11.273922in}{1.523284in}}%
\pgfpathlineto{\pgfqpoint{11.274949in}{1.522287in}}%
\pgfpathlineto{\pgfqpoint{11.278030in}{1.520052in}}%
\pgfpathlineto{\pgfqpoint{11.278715in}{1.517061in}}%
\pgfpathlineto{\pgfqpoint{11.279057in}{1.517031in}}%
\pgfpathlineto{\pgfqpoint{11.280084in}{1.515978in}}%
\pgfpathlineto{\pgfqpoint{11.280769in}{1.515732in}}%
\pgfpathlineto{\pgfqpoint{11.281453in}{1.514524in}}%
\pgfpathlineto{\pgfqpoint{11.282138in}{1.514373in}}%
\pgfpathlineto{\pgfqpoint{11.282823in}{1.512833in}}%
\pgfpathlineto{\pgfqpoint{11.283507in}{1.512138in}}%
\pgfpathlineto{\pgfqpoint{11.284877in}{1.510688in}}%
\pgfpathlineto{\pgfqpoint{11.285219in}{1.510635in}}%
\pgfpathlineto{\pgfqpoint{11.286931in}{1.509203in}}%
\pgfpathlineto{\pgfqpoint{11.287273in}{1.509162in}}%
\pgfpathlineto{\pgfqpoint{11.287616in}{1.508332in}}%
\pgfpathlineto{\pgfqpoint{11.287958in}{1.508241in}}%
\pgfpathlineto{\pgfqpoint{11.289327in}{1.506217in}}%
\pgfpathlineto{\pgfqpoint{11.289670in}{1.506127in}}%
\pgfpathlineto{\pgfqpoint{11.291039in}{1.504669in}}%
\pgfpathlineto{\pgfqpoint{11.292066in}{1.504043in}}%
\pgfpathlineto{\pgfqpoint{11.292751in}{1.503839in}}%
\pgfpathlineto{\pgfqpoint{11.293436in}{1.503272in}}%
\pgfpathlineto{\pgfqpoint{11.293778in}{1.502895in}}%
\pgfpathlineto{\pgfqpoint{11.294120in}{1.501535in}}%
\pgfpathlineto{\pgfqpoint{11.294805in}{1.501445in}}%
\pgfpathlineto{\pgfqpoint{11.296174in}{1.499111in}}%
\pgfpathlineto{\pgfqpoint{11.296517in}{1.499089in}}%
\pgfpathlineto{\pgfqpoint{11.297886in}{1.497138in}}%
\pgfpathlineto{\pgfqpoint{11.298571in}{1.496891in}}%
\pgfpathlineto{\pgfqpoint{11.299255in}{1.496733in}}%
\pgfpathlineto{\pgfqpoint{11.300282in}{1.495494in}}%
\pgfpathlineto{\pgfqpoint{11.301310in}{1.494849in}}%
\pgfpathlineto{\pgfqpoint{11.301994in}{1.493493in}}%
\pgfpathlineto{\pgfqpoint{11.302679in}{1.493108in}}%
\pgfpathlineto{\pgfqpoint{11.303364in}{1.492323in}}%
\pgfpathlineto{\pgfqpoint{11.304391in}{1.491507in}}%
\pgfpathlineto{\pgfqpoint{11.305760in}{1.489634in}}%
\pgfpathlineto{\pgfqpoint{11.308156in}{1.488399in}}%
\pgfpathlineto{\pgfqpoint{11.308499in}{1.487731in}}%
\pgfpathlineto{\pgfqpoint{11.309526in}{1.487405in}}%
\pgfpathlineto{\pgfqpoint{11.310211in}{1.486100in}}%
\pgfpathlineto{\pgfqpoint{11.310553in}{1.483865in}}%
\pgfpathlineto{\pgfqpoint{11.310895in}{1.483683in}}%
\pgfpathlineto{\pgfqpoint{11.311238in}{1.482596in}}%
\pgfpathlineto{\pgfqpoint{11.311922in}{1.482505in}}%
\pgfpathlineto{\pgfqpoint{11.312265in}{1.481932in}}%
\pgfpathlineto{\pgfqpoint{11.312607in}{1.480633in}}%
\pgfpathlineto{\pgfqpoint{11.313634in}{1.480315in}}%
\pgfpathlineto{\pgfqpoint{11.314319in}{1.479243in}}%
\pgfpathlineto{\pgfqpoint{11.316030in}{1.478029in}}%
\pgfpathlineto{\pgfqpoint{11.317057in}{1.476525in}}%
\pgfpathlineto{\pgfqpoint{11.318085in}{1.475709in}}%
\pgfpathlineto{\pgfqpoint{11.318769in}{1.475158in}}%
\pgfpathlineto{\pgfqpoint{11.319454in}{1.473987in}}%
\pgfpathlineto{\pgfqpoint{11.320823in}{1.471631in}}%
\pgfpathlineto{\pgfqpoint{11.322193in}{1.470688in}}%
\pgfpathlineto{\pgfqpoint{11.324589in}{1.468082in}}%
\pgfpathlineto{\pgfqpoint{11.324931in}{1.468054in}}%
\pgfpathlineto{\pgfqpoint{11.325958in}{1.466224in}}%
\pgfpathlineto{\pgfqpoint{11.327328in}{1.464175in}}%
\pgfpathlineto{\pgfqpoint{11.328013in}{1.463947in}}%
\pgfpathlineto{\pgfqpoint{11.328355in}{1.463204in}}%
\pgfpathlineto{\pgfqpoint{11.328697in}{1.463143in}}%
\pgfpathlineto{\pgfqpoint{11.329382in}{1.461875in}}%
\pgfpathlineto{\pgfqpoint{11.330409in}{1.460787in}}%
\pgfpathlineto{\pgfqpoint{11.331094in}{1.460334in}}%
\pgfpathlineto{\pgfqpoint{11.331778in}{1.460183in}}%
\pgfpathlineto{\pgfqpoint{11.332805in}{1.459126in}}%
\pgfpathlineto{\pgfqpoint{11.333490in}{1.457827in}}%
\pgfpathlineto{\pgfqpoint{11.335544in}{1.456981in}}%
\pgfpathlineto{\pgfqpoint{11.336229in}{1.454527in}}%
\pgfpathlineto{\pgfqpoint{11.337598in}{1.452299in}}%
\pgfpathlineto{\pgfqpoint{11.338283in}{1.451997in}}%
\pgfpathlineto{\pgfqpoint{11.339652in}{1.449248in}}%
\pgfpathlineto{\pgfqpoint{11.339995in}{1.449188in}}%
\pgfpathlineto{\pgfqpoint{11.340679in}{1.448184in}}%
\pgfpathlineto{\pgfqpoint{11.342049in}{1.447949in}}%
\pgfpathlineto{\pgfqpoint{11.342391in}{1.445926in}}%
\pgfpathlineto{\pgfqpoint{11.343076in}{1.445654in}}%
\pgfpathlineto{\pgfqpoint{11.343418in}{1.444838in}}%
\pgfpathlineto{\pgfqpoint{11.343761in}{1.444748in}}%
\pgfpathlineto{\pgfqpoint{11.345130in}{1.442663in}}%
\pgfpathlineto{\pgfqpoint{11.346157in}{1.442029in}}%
\pgfpathlineto{\pgfqpoint{11.348211in}{1.438818in}}%
\pgfpathlineto{\pgfqpoint{11.348896in}{1.438442in}}%
\pgfpathlineto{\pgfqpoint{11.349580in}{1.437518in}}%
\pgfpathlineto{\pgfqpoint{11.350265in}{1.433541in}}%
\pgfpathlineto{\pgfqpoint{11.350950in}{1.433183in}}%
\pgfpathlineto{\pgfqpoint{11.351977in}{1.431155in}}%
\pgfpathlineto{\pgfqpoint{11.352319in}{1.431085in}}%
\pgfpathlineto{\pgfqpoint{11.353004in}{1.429949in}}%
\pgfpathlineto{\pgfqpoint{11.354716in}{1.428557in}}%
\pgfpathlineto{\pgfqpoint{11.355058in}{1.428497in}}%
\pgfpathlineto{\pgfqpoint{11.356427in}{1.426322in}}%
\pgfpathlineto{\pgfqpoint{11.357112in}{1.425038in}}%
\pgfpathlineto{\pgfqpoint{11.357797in}{1.424963in}}%
\pgfpathlineto{\pgfqpoint{11.358824in}{1.423543in}}%
\pgfpathlineto{\pgfqpoint{11.359166in}{1.421247in}}%
\pgfpathlineto{\pgfqpoint{11.362932in}{1.417109in}}%
\pgfpathlineto{\pgfqpoint{11.363617in}{1.416886in}}%
\pgfpathlineto{\pgfqpoint{11.363959in}{1.415689in}}%
\pgfpathlineto{\pgfqpoint{11.364301in}{1.415599in}}%
\pgfpathlineto{\pgfqpoint{11.364986in}{1.413605in}}%
\pgfpathlineto{\pgfqpoint{11.367040in}{1.412215in}}%
\pgfpathlineto{\pgfqpoint{11.368067in}{1.410350in}}%
\pgfpathlineto{\pgfqpoint{11.368409in}{1.408538in}}%
\pgfpathlineto{\pgfqpoint{11.370806in}{1.406990in}}%
\pgfpathlineto{\pgfqpoint{11.371491in}{1.406174in}}%
\pgfpathlineto{\pgfqpoint{11.375599in}{1.403486in}}%
\pgfpathlineto{\pgfqpoint{11.377310in}{1.399408in}}%
\pgfpathlineto{\pgfqpoint{11.380392in}{1.391056in}}%
\pgfpathlineto{\pgfqpoint{11.381761in}{1.390618in}}%
\pgfpathlineto{\pgfqpoint{11.383130in}{1.388337in}}%
\pgfpathlineto{\pgfqpoint{11.384157in}{1.387406in}}%
\pgfpathlineto{\pgfqpoint{11.384842in}{1.386525in}}%
\pgfpathlineto{\pgfqpoint{11.385527in}{1.385423in}}%
\pgfpathlineto{\pgfqpoint{11.387581in}{1.384667in}}%
\pgfpathlineto{\pgfqpoint{11.387923in}{1.383504in}}%
\pgfpathlineto{\pgfqpoint{11.389977in}{1.382674in}}%
\pgfpathlineto{\pgfqpoint{11.390662in}{1.381345in}}%
\pgfpathlineto{\pgfqpoint{11.391004in}{1.381224in}}%
\pgfpathlineto{\pgfqpoint{11.391689in}{1.379599in}}%
\pgfpathlineto{\pgfqpoint{11.394428in}{1.377539in}}%
\pgfpathlineto{\pgfqpoint{11.395113in}{1.376904in}}%
\pgfpathlineto{\pgfqpoint{11.395797in}{1.376240in}}%
\pgfpathlineto{\pgfqpoint{11.396140in}{1.376089in}}%
\pgfpathlineto{\pgfqpoint{11.396482in}{1.373733in}}%
\pgfpathlineto{\pgfqpoint{11.397851in}{1.373249in}}%
\pgfpathlineto{\pgfqpoint{11.398194in}{1.371951in}}%
\pgfpathlineto{\pgfqpoint{11.398878in}{1.371618in}}%
\pgfpathlineto{\pgfqpoint{11.399221in}{1.371271in}}%
\pgfpathlineto{\pgfqpoint{11.400932in}{1.367933in}}%
\pgfpathlineto{\pgfqpoint{11.401275in}{1.367873in}}%
\pgfpathlineto{\pgfqpoint{11.402302in}{1.366483in}}%
\pgfpathlineto{\pgfqpoint{11.402644in}{1.364248in}}%
\pgfpathlineto{\pgfqpoint{11.403329in}{1.363583in}}%
\pgfpathlineto{\pgfqpoint{11.403671in}{1.362103in}}%
\pgfpathlineto{\pgfqpoint{11.404014in}{1.362013in}}%
\pgfpathlineto{\pgfqpoint{11.404698in}{1.359975in}}%
\pgfpathlineto{\pgfqpoint{11.406068in}{1.359475in}}%
\pgfpathlineto{\pgfqpoint{11.407095in}{1.358962in}}%
\pgfpathlineto{\pgfqpoint{11.408122in}{1.358841in}}%
\pgfpathlineto{\pgfqpoint{11.409149in}{1.357935in}}%
\pgfpathlineto{\pgfqpoint{11.410860in}{1.356606in}}%
\pgfpathlineto{\pgfqpoint{11.411203in}{1.356424in}}%
\pgfpathlineto{\pgfqpoint{11.411888in}{1.355186in}}%
\pgfpathlineto{\pgfqpoint{11.413942in}{1.352279in}}%
\pgfpathlineto{\pgfqpoint{11.414284in}{1.352150in}}%
\pgfpathlineto{\pgfqpoint{11.414626in}{1.351508in}}%
\pgfpathlineto{\pgfqpoint{11.414969in}{1.351501in}}%
\pgfpathlineto{\pgfqpoint{11.415311in}{1.350274in}}%
\pgfpathlineto{\pgfqpoint{11.417023in}{1.349809in}}%
\pgfpathlineto{\pgfqpoint{11.417365in}{1.349447in}}%
\pgfpathlineto{\pgfqpoint{11.418392in}{1.346487in}}%
\pgfpathlineto{\pgfqpoint{11.418734in}{1.346482in}}%
\pgfpathlineto{\pgfqpoint{11.419419in}{1.345852in}}%
\pgfpathlineto{\pgfqpoint{11.419761in}{1.345822in}}%
\pgfpathlineto{\pgfqpoint{11.420446in}{1.344387in}}%
\pgfpathlineto{\pgfqpoint{11.421131in}{1.344100in}}%
\pgfpathlineto{\pgfqpoint{11.422500in}{1.342558in}}%
\pgfpathlineto{\pgfqpoint{11.423185in}{1.341498in}}%
\pgfpathlineto{\pgfqpoint{11.423870in}{1.340627in}}%
\pgfpathlineto{\pgfqpoint{11.424212in}{1.338610in}}%
\pgfpathlineto{\pgfqpoint{11.425924in}{1.337455in}}%
\pgfpathlineto{\pgfqpoint{11.426608in}{1.336313in}}%
\pgfpathlineto{\pgfqpoint{11.427293in}{1.336005in}}%
\pgfpathlineto{\pgfqpoint{11.427978in}{1.335854in}}%
\pgfpathlineto{\pgfqpoint{11.429347in}{1.332645in}}%
\pgfpathlineto{\pgfqpoint{11.430032in}{1.330646in}}%
\pgfpathlineto{\pgfqpoint{11.431059in}{1.330417in}}%
\pgfpathlineto{\pgfqpoint{11.432086in}{1.327396in}}%
\pgfpathlineto{\pgfqpoint{11.432771in}{1.326943in}}%
\pgfpathlineto{\pgfqpoint{11.433455in}{1.326128in}}%
\pgfpathlineto{\pgfqpoint{11.434140in}{1.325675in}}%
\pgfpathlineto{\pgfqpoint{11.434482in}{1.325648in}}%
\pgfpathlineto{\pgfqpoint{11.435509in}{1.323772in}}%
\pgfpathlineto{\pgfqpoint{11.435852in}{1.323711in}}%
\pgfpathlineto{\pgfqpoint{11.436194in}{1.322714in}}%
\pgfpathlineto{\pgfqpoint{11.436879in}{1.322684in}}%
\pgfpathlineto{\pgfqpoint{11.442699in}{1.314287in}}%
\pgfpathlineto{\pgfqpoint{11.443726in}{1.313954in}}%
\pgfpathlineto{\pgfqpoint{11.444753in}{1.313622in}}%
\pgfpathlineto{\pgfqpoint{11.445437in}{1.310953in}}%
\pgfpathlineto{\pgfqpoint{11.446122in}{1.310420in}}%
\pgfpathlineto{\pgfqpoint{11.447149in}{1.308397in}}%
\pgfpathlineto{\pgfqpoint{11.447834in}{1.308034in}}%
\pgfpathlineto{\pgfqpoint{11.448176in}{1.306856in}}%
\pgfpathlineto{\pgfqpoint{11.448861in}{1.306509in}}%
\pgfpathlineto{\pgfqpoint{11.449546in}{1.305338in}}%
\pgfpathlineto{\pgfqpoint{11.450230in}{1.304288in}}%
\pgfpathlineto{\pgfqpoint{11.450915in}{1.304024in}}%
\pgfpathlineto{\pgfqpoint{11.451257in}{1.303995in}}%
\pgfpathlineto{\pgfqpoint{11.452284in}{1.302959in}}%
\pgfpathlineto{\pgfqpoint{11.452627in}{1.302899in}}%
\pgfpathlineto{\pgfqpoint{11.453311in}{1.302128in}}%
\pgfpathlineto{\pgfqpoint{11.453654in}{1.302081in}}%
\pgfpathlineto{\pgfqpoint{11.455366in}{1.298912in}}%
\pgfpathlineto{\pgfqpoint{11.458104in}{1.297551in}}%
\pgfpathlineto{\pgfqpoint{11.458789in}{1.296862in}}%
\pgfpathlineto{\pgfqpoint{11.460843in}{1.292523in}}%
\pgfpathlineto{\pgfqpoint{11.461185in}{1.292508in}}%
\pgfpathlineto{\pgfqpoint{11.462555in}{1.291270in}}%
\pgfpathlineto{\pgfqpoint{11.462897in}{1.291270in}}%
\pgfpathlineto{\pgfqpoint{11.463582in}{1.289336in}}%
\pgfpathlineto{\pgfqpoint{11.464267in}{1.287260in}}%
\pgfpathlineto{\pgfqpoint{11.464609in}{1.286950in}}%
\pgfpathlineto{\pgfqpoint{11.465294in}{1.285712in}}%
\pgfpathlineto{\pgfqpoint{11.465978in}{1.284994in}}%
\pgfpathlineto{\pgfqpoint{11.466663in}{1.284896in}}%
\pgfpathlineto{\pgfqpoint{11.467348in}{1.284292in}}%
\pgfpathlineto{\pgfqpoint{11.467690in}{1.284277in}}%
\pgfpathlineto{\pgfqpoint{11.468375in}{1.283204in}}%
\pgfpathlineto{\pgfqpoint{11.469059in}{1.282685in}}%
\pgfpathlineto{\pgfqpoint{11.470086in}{1.281301in}}%
\pgfpathlineto{\pgfqpoint{11.470429in}{1.279519in}}%
\pgfpathlineto{\pgfqpoint{11.471113in}{1.278825in}}%
\pgfpathlineto{\pgfqpoint{11.472825in}{1.277141in}}%
\pgfpathlineto{\pgfqpoint{11.473168in}{1.277073in}}%
\pgfpathlineto{\pgfqpoint{11.475222in}{1.273690in}}%
\pgfpathlineto{\pgfqpoint{11.475906in}{1.272663in}}%
\pgfpathlineto{\pgfqpoint{11.476933in}{1.271787in}}%
\pgfpathlineto{\pgfqpoint{11.477276in}{1.271726in}}%
\pgfpathlineto{\pgfqpoint{11.477618in}{1.271137in}}%
\pgfpathlineto{\pgfqpoint{11.477960in}{1.271122in}}%
\pgfpathlineto{\pgfqpoint{11.479330in}{1.269581in}}%
\pgfpathlineto{\pgfqpoint{11.480699in}{1.267467in}}%
\pgfpathlineto{\pgfqpoint{11.482069in}{1.266615in}}%
\pgfpathlineto{\pgfqpoint{11.482411in}{1.266564in}}%
\pgfpathlineto{\pgfqpoint{11.483438in}{1.264990in}}%
\pgfpathlineto{\pgfqpoint{11.484123in}{1.264327in}}%
\pgfpathlineto{\pgfqpoint{11.484807in}{1.263570in}}%
\pgfpathlineto{\pgfqpoint{11.485834in}{1.262574in}}%
\pgfpathlineto{\pgfqpoint{11.486177in}{1.262543in}}%
\pgfpathlineto{\pgfqpoint{11.486519in}{1.262188in}}%
\pgfpathlineto{\pgfqpoint{11.487546in}{1.260489in}}%
\pgfpathlineto{\pgfqpoint{11.487888in}{1.260306in}}%
\pgfpathlineto{\pgfqpoint{11.488573in}{1.259402in}}%
\pgfpathlineto{\pgfqpoint{11.489258in}{1.259130in}}%
\pgfpathlineto{\pgfqpoint{11.489943in}{1.258466in}}%
\pgfpathlineto{\pgfqpoint{11.490970in}{1.257112in}}%
\pgfpathlineto{\pgfqpoint{11.491312in}{1.255954in}}%
\pgfpathlineto{\pgfqpoint{11.492339in}{1.255675in}}%
\pgfpathlineto{\pgfqpoint{11.493024in}{1.253965in}}%
\pgfpathlineto{\pgfqpoint{11.493708in}{1.253572in}}%
\pgfpathlineto{\pgfqpoint{11.494735in}{1.250793in}}%
\pgfpathlineto{\pgfqpoint{11.495420in}{1.250432in}}%
\pgfpathlineto{\pgfqpoint{11.495762in}{1.249464in}}%
\pgfpathlineto{\pgfqpoint{11.496789in}{1.249200in}}%
\pgfpathlineto{\pgfqpoint{11.497474in}{1.247803in}}%
\pgfpathlineto{\pgfqpoint{11.498159in}{1.247709in}}%
\pgfpathlineto{\pgfqpoint{11.498844in}{1.247259in}}%
\pgfpathlineto{\pgfqpoint{11.499871in}{1.246897in}}%
\pgfpathlineto{\pgfqpoint{11.500555in}{1.245779in}}%
\pgfpathlineto{\pgfqpoint{11.501240in}{1.244824in}}%
\pgfpathlineto{\pgfqpoint{11.501582in}{1.244571in}}%
\pgfpathlineto{\pgfqpoint{11.502609in}{1.242246in}}%
\pgfpathlineto{\pgfqpoint{11.502952in}{1.242245in}}%
\pgfpathlineto{\pgfqpoint{11.503636in}{1.241369in}}%
\pgfpathlineto{\pgfqpoint{11.503979in}{1.239478in}}%
\pgfpathlineto{\pgfqpoint{11.504321in}{1.239458in}}%
\pgfpathlineto{\pgfqpoint{11.505348in}{1.237774in}}%
\pgfpathlineto{\pgfqpoint{11.506375in}{1.237114in}}%
\pgfpathlineto{\pgfqpoint{11.507060in}{1.235758in}}%
\pgfpathlineto{\pgfqpoint{11.507402in}{1.235720in}}%
\pgfpathlineto{\pgfqpoint{11.508429in}{1.234874in}}%
\pgfpathlineto{\pgfqpoint{11.508772in}{1.234703in}}%
\pgfpathlineto{\pgfqpoint{11.509114in}{1.233727in}}%
\pgfpathlineto{\pgfqpoint{11.509456in}{1.233727in}}%
\pgfpathlineto{\pgfqpoint{11.509799in}{1.232730in}}%
\pgfpathlineto{\pgfqpoint{11.510141in}{1.232725in}}%
\pgfpathlineto{\pgfqpoint{11.512195in}{1.230850in}}%
\pgfpathlineto{\pgfqpoint{11.514249in}{1.223547in}}%
\pgfpathlineto{\pgfqpoint{11.514592in}{1.223305in}}%
\pgfpathlineto{\pgfqpoint{11.514934in}{1.222118in}}%
\pgfpathlineto{\pgfqpoint{11.515619in}{1.221712in}}%
\pgfpathlineto{\pgfqpoint{11.516303in}{1.221251in}}%
\pgfpathlineto{\pgfqpoint{11.517330in}{1.220164in}}%
\pgfpathlineto{\pgfqpoint{11.517673in}{1.218895in}}%
\pgfpathlineto{\pgfqpoint{11.518015in}{1.218880in}}%
\pgfpathlineto{\pgfqpoint{11.518700in}{1.216338in}}%
\pgfpathlineto{\pgfqpoint{11.519042in}{1.216199in}}%
\pgfpathlineto{\pgfqpoint{11.519384in}{1.215108in}}%
\pgfpathlineto{\pgfqpoint{11.520411in}{1.214651in}}%
\pgfpathlineto{\pgfqpoint{11.521438in}{1.212069in}}%
\pgfpathlineto{\pgfqpoint{11.521781in}{1.211895in}}%
\pgfpathlineto{\pgfqpoint{11.522123in}{1.210347in}}%
\pgfpathlineto{\pgfqpoint{11.522465in}{1.210271in}}%
\pgfpathlineto{\pgfqpoint{11.523150in}{1.208716in}}%
\pgfpathlineto{\pgfqpoint{11.523493in}{1.208686in}}%
\pgfpathlineto{\pgfqpoint{11.523835in}{1.207406in}}%
\pgfpathlineto{\pgfqpoint{11.524862in}{1.206798in}}%
\pgfpathlineto{\pgfqpoint{11.525204in}{1.203928in}}%
\pgfpathlineto{\pgfqpoint{11.527601in}{1.202569in}}%
\pgfpathlineto{\pgfqpoint{11.528970in}{1.202324in}}%
\pgfpathlineto{\pgfqpoint{11.529312in}{1.201889in}}%
\pgfpathlineto{\pgfqpoint{11.531367in}{1.196694in}}%
\pgfpathlineto{\pgfqpoint{11.531709in}{1.196331in}}%
\pgfpathlineto{\pgfqpoint{11.532736in}{1.193613in}}%
\pgfpathlineto{\pgfqpoint{11.533078in}{1.193492in}}%
\pgfpathlineto{\pgfqpoint{11.536159in}{1.185993in}}%
\pgfpathlineto{\pgfqpoint{11.536502in}{1.185918in}}%
\pgfpathlineto{\pgfqpoint{11.537186in}{1.184611in}}%
\pgfpathlineto{\pgfqpoint{11.537871in}{1.184460in}}%
\pgfpathlineto{\pgfqpoint{11.538556in}{1.183935in}}%
\pgfpathlineto{\pgfqpoint{11.540268in}{1.180231in}}%
\pgfpathlineto{\pgfqpoint{11.540952in}{1.179272in}}%
\pgfpathlineto{\pgfqpoint{11.541637in}{1.178661in}}%
\pgfpathlineto{\pgfqpoint{11.542322in}{1.178404in}}%
\pgfpathlineto{\pgfqpoint{11.543006in}{1.176425in}}%
\pgfpathlineto{\pgfqpoint{11.544376in}{1.175459in}}%
\pgfpathlineto{\pgfqpoint{11.545060in}{1.173979in}}%
\pgfpathlineto{\pgfqpoint{11.545745in}{1.173420in}}%
\pgfpathlineto{\pgfqpoint{11.546430in}{1.172378in}}%
\pgfpathlineto{\pgfqpoint{11.546772in}{1.171909in}}%
\pgfpathlineto{\pgfqpoint{11.547799in}{1.168570in}}%
\pgfpathlineto{\pgfqpoint{11.548826in}{1.165944in}}%
\pgfpathlineto{\pgfqpoint{11.549511in}{1.165521in}}%
\pgfpathlineto{\pgfqpoint{11.552250in}{1.160898in}}%
\pgfpathlineto{\pgfqpoint{11.552592in}{1.160756in}}%
\pgfpathlineto{\pgfqpoint{11.553277in}{1.159751in}}%
\pgfpathlineto{\pgfqpoint{11.554646in}{1.158422in}}%
\pgfpathlineto{\pgfqpoint{11.554988in}{1.158362in}}%
\pgfpathlineto{\pgfqpoint{11.557043in}{1.156164in}}%
\pgfpathlineto{\pgfqpoint{11.557385in}{1.156099in}}%
\pgfpathlineto{\pgfqpoint{11.558754in}{1.153589in}}%
\pgfpathlineto{\pgfqpoint{11.559097in}{1.151535in}}%
\pgfpathlineto{\pgfqpoint{11.561151in}{1.149670in}}%
\pgfpathlineto{\pgfqpoint{11.561493in}{1.149255in}}%
\pgfpathlineto{\pgfqpoint{11.562178in}{1.147631in}}%
\pgfpathlineto{\pgfqpoint{11.562520in}{1.147548in}}%
\pgfpathlineto{\pgfqpoint{11.562862in}{1.145917in}}%
\pgfpathlineto{\pgfqpoint{11.563547in}{1.145434in}}%
\pgfpathlineto{\pgfqpoint{11.563889in}{1.141203in}}%
\pgfpathlineto{\pgfqpoint{11.564232in}{1.140812in}}%
\pgfpathlineto{\pgfqpoint{11.564574in}{1.139876in}}%
\pgfpathlineto{\pgfqpoint{11.564916in}{1.139815in}}%
\pgfpathlineto{\pgfqpoint{11.565259in}{1.138970in}}%
\pgfpathlineto{\pgfqpoint{11.565601in}{1.138947in}}%
\pgfpathlineto{\pgfqpoint{11.566628in}{1.137746in}}%
\pgfpathlineto{\pgfqpoint{11.568682in}{1.136466in}}%
\pgfpathlineto{\pgfqpoint{11.569709in}{1.134982in}}%
\pgfpathlineto{\pgfqpoint{11.570052in}{1.133925in}}%
\pgfpathlineto{\pgfqpoint{11.570736in}{1.133895in}}%
\pgfpathlineto{\pgfqpoint{11.571079in}{1.132717in}}%
\pgfpathlineto{\pgfqpoint{11.571763in}{1.132536in}}%
\pgfpathlineto{\pgfqpoint{11.572448in}{1.132052in}}%
\pgfpathlineto{\pgfqpoint{11.573133in}{1.131690in}}%
\pgfpathlineto{\pgfqpoint{11.574160in}{1.130867in}}%
\pgfpathlineto{\pgfqpoint{11.574502in}{1.128752in}}%
\pgfpathlineto{\pgfqpoint{11.574845in}{1.128714in}}%
\pgfpathlineto{\pgfqpoint{11.575872in}{1.127219in}}%
\pgfpathlineto{\pgfqpoint{11.576214in}{1.127219in}}%
\pgfpathlineto{\pgfqpoint{11.576899in}{1.126147in}}%
\pgfpathlineto{\pgfqpoint{11.577583in}{1.126102in}}%
\pgfpathlineto{\pgfqpoint{11.578268in}{1.125256in}}%
\pgfpathlineto{\pgfqpoint{11.578610in}{1.124308in}}%
\pgfpathlineto{\pgfqpoint{11.578953in}{1.124259in}}%
\pgfpathlineto{\pgfqpoint{11.579295in}{1.123504in}}%
\pgfpathlineto{\pgfqpoint{11.580322in}{1.118822in}}%
\pgfpathlineto{\pgfqpoint{11.580664in}{1.118731in}}%
\pgfpathlineto{\pgfqpoint{11.581007in}{1.116194in}}%
\pgfpathlineto{\pgfqpoint{11.581349in}{1.116103in}}%
\pgfpathlineto{\pgfqpoint{11.581691in}{1.115076in}}%
\pgfpathlineto{\pgfqpoint{11.582376in}{1.114820in}}%
\pgfpathlineto{\pgfqpoint{11.582719in}{1.111279in}}%
\pgfpathlineto{\pgfqpoint{11.583061in}{1.110983in}}%
\pgfpathlineto{\pgfqpoint{11.583746in}{1.109231in}}%
\pgfpathlineto{\pgfqpoint{11.584088in}{1.109005in}}%
\pgfpathlineto{\pgfqpoint{11.584773in}{1.104565in}}%
\pgfpathlineto{\pgfqpoint{11.585115in}{1.104361in}}%
\pgfpathlineto{\pgfqpoint{11.585800in}{1.102631in}}%
\pgfpathlineto{\pgfqpoint{11.586484in}{1.102322in}}%
\pgfpathlineto{\pgfqpoint{11.587511in}{1.101227in}}%
\pgfpathlineto{\pgfqpoint{11.587854in}{1.100940in}}%
\pgfpathlineto{\pgfqpoint{11.588538in}{1.099025in}}%
\pgfpathlineto{\pgfqpoint{11.588881in}{1.098735in}}%
\pgfpathlineto{\pgfqpoint{11.589565in}{1.097527in}}%
\pgfpathlineto{\pgfqpoint{11.590250in}{1.097073in}}%
\pgfpathlineto{\pgfqpoint{11.591277in}{1.095624in}}%
\pgfpathlineto{\pgfqpoint{11.591620in}{1.094355in}}%
\pgfpathlineto{\pgfqpoint{11.591962in}{1.091667in}}%
\pgfpathlineto{\pgfqpoint{11.592304in}{1.091523in}}%
\pgfpathlineto{\pgfqpoint{11.595385in}{1.086426in}}%
\pgfpathlineto{\pgfqpoint{11.595728in}{1.085082in}}%
\pgfpathlineto{\pgfqpoint{11.596412in}{1.085029in}}%
\pgfpathlineto{\pgfqpoint{11.597439in}{1.079290in}}%
\pgfpathlineto{\pgfqpoint{11.598124in}{1.075204in}}%
\pgfpathlineto{\pgfqpoint{11.598466in}{1.075204in}}%
\pgfpathlineto{\pgfqpoint{11.599151in}{1.073164in}}%
\pgfpathlineto{\pgfqpoint{11.600178in}{1.068906in}}%
\pgfpathlineto{\pgfqpoint{11.600863in}{1.067358in}}%
\pgfpathlineto{\pgfqpoint{11.602575in}{1.063129in}}%
\pgfpathlineto{\pgfqpoint{11.603602in}{1.061611in}}%
\pgfpathlineto{\pgfqpoint{11.604286in}{1.060010in}}%
\pgfpathlineto{\pgfqpoint{11.604971in}{1.059527in}}%
\pgfpathlineto{\pgfqpoint{11.605313in}{1.058636in}}%
\pgfpathlineto{\pgfqpoint{11.605998in}{1.054392in}}%
\pgfpathlineto{\pgfqpoint{11.606683in}{1.053690in}}%
\pgfpathlineto{\pgfqpoint{11.607367in}{1.052485in}}%
\pgfpathlineto{\pgfqpoint{11.607710in}{1.052278in}}%
\pgfpathlineto{\pgfqpoint{11.608052in}{1.049166in}}%
\pgfpathlineto{\pgfqpoint{11.608737in}{1.048647in}}%
\pgfpathlineto{\pgfqpoint{11.609079in}{1.045662in}}%
\pgfpathlineto{\pgfqpoint{11.609764in}{1.044786in}}%
\pgfpathlineto{\pgfqpoint{11.610106in}{1.044639in}}%
\pgfpathlineto{\pgfqpoint{11.611476in}{1.041313in}}%
\pgfpathlineto{\pgfqpoint{11.612160in}{1.040095in}}%
\pgfpathlineto{\pgfqpoint{11.612503in}{1.037718in}}%
\pgfpathlineto{\pgfqpoint{11.612845in}{1.037416in}}%
\pgfpathlineto{\pgfqpoint{11.613187in}{1.035332in}}%
\pgfpathlineto{\pgfqpoint{11.613872in}{1.034773in}}%
\pgfpathlineto{\pgfqpoint{11.615241in}{1.030287in}}%
\pgfpathlineto{\pgfqpoint{11.615584in}{1.029676in}}%
\pgfpathlineto{\pgfqpoint{11.616268in}{1.026844in}}%
\pgfpathlineto{\pgfqpoint{11.616611in}{1.026149in}}%
\pgfpathlineto{\pgfqpoint{11.617296in}{1.022373in}}%
\pgfpathlineto{\pgfqpoint{11.617980in}{1.017367in}}%
\pgfpathlineto{\pgfqpoint{11.618323in}{1.016272in}}%
\pgfpathlineto{\pgfqpoint{11.618665in}{1.016234in}}%
\pgfpathlineto{\pgfqpoint{11.621404in}{1.010442in}}%
\pgfpathlineto{\pgfqpoint{11.621746in}{1.009083in}}%
\pgfpathlineto{\pgfqpoint{11.622431in}{1.005473in}}%
\pgfpathlineto{\pgfqpoint{11.622773in}{1.004718in}}%
\pgfpathlineto{\pgfqpoint{11.623115in}{1.001229in}}%
\pgfpathlineto{\pgfqpoint{11.623458in}{1.000967in}}%
\pgfpathlineto{\pgfqpoint{11.624827in}{0.995822in}}%
\pgfpathlineto{\pgfqpoint{11.625169in}{0.995309in}}%
\pgfpathlineto{\pgfqpoint{11.625854in}{0.988270in}}%
\pgfpathlineto{\pgfqpoint{11.626197in}{0.987727in}}%
\pgfpathlineto{\pgfqpoint{11.626881in}{0.981383in}}%
\pgfpathlineto{\pgfqpoint{11.627224in}{0.981383in}}%
\pgfpathlineto{\pgfqpoint{11.627566in}{0.980507in}}%
\pgfpathlineto{\pgfqpoint{11.628251in}{0.975916in}}%
\pgfpathlineto{\pgfqpoint{11.628593in}{0.975282in}}%
\pgfpathlineto{\pgfqpoint{11.628935in}{0.973318in}}%
\pgfpathlineto{\pgfqpoint{11.629278in}{0.973254in}}%
\pgfpathlineto{\pgfqpoint{11.629962in}{0.969301in}}%
\pgfpathlineto{\pgfqpoint{11.630305in}{0.968425in}}%
\pgfpathlineto{\pgfqpoint{11.630989in}{0.961749in}}%
\pgfpathlineto{\pgfqpoint{11.631332in}{0.961387in}}%
\pgfpathlineto{\pgfqpoint{11.632701in}{0.956282in}}%
\pgfpathlineto{\pgfqpoint{11.634413in}{0.952661in}}%
\pgfpathlineto{\pgfqpoint{11.635098in}{0.950195in}}%
\pgfpathlineto{\pgfqpoint{11.635440in}{0.949717in}}%
\pgfpathlineto{\pgfqpoint{11.635782in}{0.946457in}}%
\pgfpathlineto{\pgfqpoint{11.636467in}{0.936285in}}%
\pgfpathlineto{\pgfqpoint{11.636809in}{0.934318in}}%
\pgfpathlineto{\pgfqpoint{11.637152in}{0.934111in}}%
\pgfpathlineto{\pgfqpoint{11.637494in}{0.933204in}}%
\pgfpathlineto{\pgfqpoint{11.638521in}{0.920654in}}%
\pgfpathlineto{\pgfqpoint{11.638863in}{0.919798in}}%
\pgfpathlineto{\pgfqpoint{11.639890in}{0.912876in}}%
\pgfpathlineto{\pgfqpoint{11.640575in}{0.907454in}}%
\pgfpathlineto{\pgfqpoint{11.641260in}{0.907076in}}%
\pgfpathlineto{\pgfqpoint{11.642972in}{0.772020in}}%
\pgfpathlineto{\pgfqpoint{11.643656in}{0.757638in}}%
\pgfpathlineto{\pgfqpoint{11.643999in}{0.757102in}}%
\pgfpathlineto{\pgfqpoint{11.644683in}{0.746756in}}%
\pgfpathlineto{\pgfqpoint{11.645026in}{0.724517in}}%
\pgfpathlineto{\pgfqpoint{11.645368in}{0.722859in}}%
\pgfpathlineto{\pgfqpoint{11.646053in}{0.690467in}}%
\pgfpathlineto{\pgfqpoint{11.646395in}{0.685876in}}%
\pgfpathlineto{\pgfqpoint{11.646737in}{0.670138in}}%
\pgfpathlineto{\pgfqpoint{11.646737in}{0.670138in}}%
\pgfusepath{stroke}%
\end{pgfscope}%
\begin{pgfscope}%
\pgfsetrectcap%
\pgfsetmiterjoin%
\pgfsetlinewidth{0.803000pt}%
\definecolor{currentstroke}{rgb}{0.000000,0.000000,0.000000}%
\pgfsetstrokecolor{currentstroke}%
\pgfsetdash{}{0pt}%
\pgfpathmoveto{\pgfqpoint{8.609492in}{0.670138in}}%
\pgfpathlineto{\pgfqpoint{8.609492in}{5.516628in}}%
\pgfusepath{stroke}%
\end{pgfscope}%
\begin{pgfscope}%
\pgfsetrectcap%
\pgfsetmiterjoin%
\pgfsetlinewidth{0.803000pt}%
\definecolor{currentstroke}{rgb}{0.000000,0.000000,0.000000}%
\pgfsetstrokecolor{currentstroke}%
\pgfsetdash{}{0pt}%
\pgfpathmoveto{\pgfqpoint{11.676809in}{0.670138in}}%
\pgfpathlineto{\pgfqpoint{11.676809in}{5.516628in}}%
\pgfusepath{stroke}%
\end{pgfscope}%
\begin{pgfscope}%
\pgfsetrectcap%
\pgfsetmiterjoin%
\pgfsetlinewidth{0.803000pt}%
\definecolor{currentstroke}{rgb}{0.000000,0.000000,0.000000}%
\pgfsetstrokecolor{currentstroke}%
\pgfsetdash{}{0pt}%
\pgfpathmoveto{\pgfqpoint{8.609492in}{0.670138in}}%
\pgfpathlineto{\pgfqpoint{11.676809in}{0.670138in}}%
\pgfusepath{stroke}%
\end{pgfscope}%
\begin{pgfscope}%
\pgfsetrectcap%
\pgfsetmiterjoin%
\pgfsetlinewidth{0.803000pt}%
\definecolor{currentstroke}{rgb}{0.000000,0.000000,0.000000}%
\pgfsetstrokecolor{currentstroke}%
\pgfsetdash{}{0pt}%
\pgfpathmoveto{\pgfqpoint{8.609492in}{5.516628in}}%
\pgfpathlineto{\pgfqpoint{11.676809in}{5.516628in}}%
\pgfusepath{stroke}%
\end{pgfscope}%
\begin{pgfscope}%
\definecolor{textcolor}{rgb}{0.000000,0.000000,0.000000}%
\pgfsetstrokecolor{textcolor}%
\pgfsetfillcolor{textcolor}%
\pgftext[x=10.143151in,y=5.599962in,,base]{\color{textcolor}{\rmfamily\fontsize{20.000000}{24.000000}\selectfont\catcode`\^=\active\def^{\ifmmode\sp\else\^{}\fi}\catcode`\%=\active\def%{\%}Load Duration Curve}}%
\end{pgfscope}%
\end{pgfpicture}%
\makeatother%
\endgroup%
}
  \caption{The normalized demand and load duration curves that are used in this thesis.}
  \label{fig:normalized_ldc}
\end{figure}

\FloatBarrier
\subsection{Exercise 1: Deciding Among Evolutionary Algorithms}

\ac{osier} allows users to choose among a variety of \ac{moo} methods. This is
motivated by the desire for flexibility. However, Exercises 1 and 2 use just one
algorithm, \ac{unsga3} as implemented by \ac{pymoo}, which should be justified
by comparing the results against different algorithms. As an important aside,
although I used the \ac{deap} implementations of \ac{nsga2} and \ac{nsga3},
these algorithms are not exclusive to \ac{deap}. \ac{pymoo} also implements
them, I simply wanted to show the breadth of support for different tools in
\ac{osier}. Figure \ref{fig:algorithm-comparison} provides the justification for
choosing \ac{unsga3} by comparing the results of three \ac{moo} algorithms by
showing the respective scatter plots and a density plot of the points on each
axis. Since it took \ac{unsga3} 128 generations to reach its convergence
criterion, the other two algorithms were also stopped after 128 generations,
before converging. The density plot above the scatter plot shows the density of
points along the ``total cost'' objective. Similarly, the density plot to the
right shows the distribution of points for the ``emissions'' objective.

There are a few notable features of Figure \ref{fig:algorithm-comparison}.
First, all three algorithms identified very similar Pareto fronts, the main
differences involve the distribution of points and the extent of their
respective solution sets. Second, the two \ac{deap} algorithms have a greater
number of points along the bottom part of the Pareto front, indicating a greater
sampling over the cost objective. This is further supported by the higher
concentration of points along the lower half of the emission objective's range.
Third, the algorithms implemented by \ac{deap} both have more extreme values
along both axes. All of these features can be attributed to the fact that
neither \ac{nsga2} nor \ac{nsga3} fully converged. Thus, choosing \ac{unsga3}
will be used for the remaining exercises for its faster convergence.

\begin{figure}[ht]
  \centering
  \resizebox{0.75\columnwidth}{!}{\input{figures/04_benchmark_chapter/algorithm_comparison_kde.pgf}}
  \caption{Compares the \ac{moo} algorithms.}
  \label{fig:algorithm-comparison}
\end{figure}

\FloatBarrier

\subsection{Exercise 2: Exploring objective space}
Since structural uncertainty persists regardless of the number of objectives
used, it's important to check the near-optimal objective space for alternative
solutions. In the first benchmark exercise, I used \ac{temoa} to calculate the
least-cost solution. Then I generated 30 alternative solutions with \ac{mga} as
described in Section \ref{section:mga} with a 10\% slack variable added to
\ac{temoa}'s objective function. Figure \ref{fig:temoa-benchmark-01} shows the
points from \ac{temoa} in red and \ac{osier}'s Pareto-front for the same problem
in black. The red- and gray-shaded regions are the sub-optimal spaces (i.e.,
within 10\% of any objective) for \ac{temoa} and \ac{osier}, respectively.
The solid black points indicates points along the Pareto-front, while the open
black points are points tested in early generations of the \ac{osier} simulation.
\footnote{There are many more tested points than shown in Figure \ref{fig:temoa-benchmark-01}. 
For simplicity and clarity, Figure \ref{fig:temoa-benchmark-01} shows a random subset
of points.}

\begin{figure}[h]
  \centering
  % \resizebox{0.6\columnwidth}{!}{%% Creator: Matplotlib, PGF backend
%%
%% To include the figure in your LaTeX document, write
%%   \input{<filename>.pgf}
%%
%% Make sure the required packages are loaded in your preamble
%%   \usepackage{pgf}
%%
%% Also ensure that all the required font packages are loaded; for instance,
%% the lmodern package is sometimes necessary when using math font.
%%   \usepackage{lmodern}
%%
%% Figures using additional raster images can only be included by \input if
%% they are in the same directory as the main LaTeX file. For loading figures
%% from other directories you can use the `import` package
%%   \usepackage{import}
%%
%% and then include the figures with
%%   \import{<path to file>}{<filename>.pgf}
%%
%% Matplotlib used the following preamble
%%   \def\mathdefault#1{#1}
%%   \everymath=\expandafter{\the\everymath\displaystyle}
%%   \IfFileExists{scrextend.sty}{
%%     \usepackage[fontsize=10.000000pt]{scrextend}
%%   }{
%%     \renewcommand{\normalsize}{\fontsize{10.000000}{12.000000}\selectfont}
%%     \normalsize
%%   }
%%   
%%   \makeatletter\@ifpackageloaded{underscore}{}{\usepackage[strings]{underscore}}\makeatother
%%
\begingroup%
\makeatletter%
\begin{pgfpicture}%
\pgfpathrectangle{\pgfpointorigin}{\pgfqpoint{6.988192in}{5.458470in}}%
\pgfusepath{use as bounding box, clip}%
\begin{pgfscope}%
\pgfsetbuttcap%
\pgfsetmiterjoin%
\definecolor{currentfill}{rgb}{1.000000,1.000000,1.000000}%
\pgfsetfillcolor{currentfill}%
\pgfsetlinewidth{0.000000pt}%
\definecolor{currentstroke}{rgb}{0.000000,0.000000,0.000000}%
\pgfsetstrokecolor{currentstroke}%
\pgfsetdash{}{0pt}%
\pgfpathmoveto{\pgfqpoint{0.000000in}{0.000000in}}%
\pgfpathlineto{\pgfqpoint{6.988192in}{0.000000in}}%
\pgfpathlineto{\pgfqpoint{6.988192in}{5.458470in}}%
\pgfpathlineto{\pgfqpoint{0.000000in}{5.458470in}}%
\pgfpathlineto{\pgfqpoint{0.000000in}{0.000000in}}%
\pgfpathclose%
\pgfusepath{fill}%
\end{pgfscope}%
\begin{pgfscope}%
\pgfsetbuttcap%
\pgfsetmiterjoin%
\definecolor{currentfill}{rgb}{1.000000,1.000000,1.000000}%
\pgfsetfillcolor{currentfill}%
\pgfsetlinewidth{0.000000pt}%
\definecolor{currentstroke}{rgb}{0.000000,0.000000,0.000000}%
\pgfsetstrokecolor{currentstroke}%
\pgfsetstrokeopacity{0.000000}%
\pgfsetdash{}{0pt}%
\pgfpathmoveto{\pgfqpoint{0.688192in}{0.670138in}}%
\pgfpathlineto{\pgfqpoint{6.888192in}{0.670138in}}%
\pgfpathlineto{\pgfqpoint{6.888192in}{5.290138in}}%
\pgfpathlineto{\pgfqpoint{0.688192in}{5.290138in}}%
\pgfpathlineto{\pgfqpoint{0.688192in}{0.670138in}}%
\pgfpathclose%
\pgfusepath{fill}%
\end{pgfscope}%
\begin{pgfscope}%
\pgfpathrectangle{\pgfqpoint{0.688192in}{0.670138in}}{\pgfqpoint{6.200000in}{4.620000in}}%
\pgfusepath{clip}%
\pgfsetbuttcap%
\pgfsetmiterjoin%
\definecolor{currentfill}{rgb}{0.121569,0.466667,0.705882}%
\pgfsetfillcolor{currentfill}%
\pgfsetfillopacity{0.500000}%
\pgfsetlinewidth{1.003750pt}%
\definecolor{currentstroke}{rgb}{0.121569,0.466667,0.705882}%
\pgfsetstrokecolor{currentstroke}%
\pgfsetstrokeopacity{0.500000}%
\pgfsetdash{}{0pt}%
\pgfpathmoveto{\pgfqpoint{0.741425in}{1.377543in}}%
\pgfpathlineto{\pgfqpoint{0.758703in}{0.955032in}}%
\pgfpathlineto{\pgfqpoint{0.768198in}{0.875033in}}%
\pgfpathlineto{\pgfqpoint{0.774746in}{0.828781in}}%
\pgfpathlineto{\pgfqpoint{0.778243in}{0.822495in}}%
\pgfpathlineto{\pgfqpoint{0.782159in}{0.789611in}}%
\pgfpathlineto{\pgfqpoint{0.786516in}{0.779881in}}%
\pgfpathlineto{\pgfqpoint{0.792538in}{0.779145in}}%
\pgfpathlineto{\pgfqpoint{0.794668in}{0.758056in}}%
\pgfpathlineto{\pgfqpoint{0.799837in}{0.752930in}}%
\pgfpathlineto{\pgfqpoint{0.809370in}{0.751978in}}%
\pgfpathlineto{\pgfqpoint{0.812629in}{0.743975in}}%
\pgfpathlineto{\pgfqpoint{0.815972in}{0.742575in}}%
\pgfpathlineto{\pgfqpoint{0.822987in}{0.738477in}}%
\pgfpathlineto{\pgfqpoint{0.828825in}{0.734937in}}%
\pgfpathlineto{\pgfqpoint{0.829214in}{0.733319in}}%
\pgfpathlineto{\pgfqpoint{0.833044in}{0.730858in}}%
\pgfpathlineto{\pgfqpoint{0.848459in}{0.726329in}}%
\pgfpathlineto{\pgfqpoint{0.864854in}{0.720019in}}%
\pgfpathlineto{\pgfqpoint{0.887104in}{0.715517in}}%
\pgfpathlineto{\pgfqpoint{0.907479in}{0.714004in}}%
\pgfpathlineto{\pgfqpoint{0.908310in}{0.712008in}}%
\pgfpathlineto{\pgfqpoint{0.909513in}{0.708525in}}%
\pgfpathlineto{\pgfqpoint{0.912740in}{0.707284in}}%
\pgfpathlineto{\pgfqpoint{0.920440in}{0.706723in}}%
\pgfpathlineto{\pgfqpoint{0.925670in}{0.705238in}}%
\pgfpathlineto{\pgfqpoint{0.948903in}{0.702931in}}%
\pgfpathlineto{\pgfqpoint{0.951945in}{0.701707in}}%
\pgfpathlineto{\pgfqpoint{0.952035in}{0.700391in}}%
\pgfpathlineto{\pgfqpoint{0.957029in}{0.700173in}}%
\pgfpathlineto{\pgfqpoint{0.968828in}{0.697963in}}%
\pgfpathlineto{\pgfqpoint{0.974412in}{0.697738in}}%
\pgfpathlineto{\pgfqpoint{0.975275in}{0.696914in}}%
\pgfpathlineto{\pgfqpoint{1.021767in}{0.694795in}}%
\pgfpathlineto{\pgfqpoint{1.025407in}{0.690657in}}%
\pgfpathlineto{\pgfqpoint{1.027475in}{0.690338in}}%
\pgfpathlineto{\pgfqpoint{1.034837in}{0.689784in}}%
\pgfpathlineto{\pgfqpoint{1.049406in}{0.687676in}}%
\pgfpathlineto{\pgfqpoint{1.054714in}{0.687138in}}%
\pgfpathlineto{\pgfqpoint{1.059617in}{0.686467in}}%
\pgfpathlineto{\pgfqpoint{1.072141in}{0.685078in}}%
\pgfpathlineto{\pgfqpoint{1.092208in}{0.684413in}}%
\pgfpathlineto{\pgfqpoint{1.115209in}{0.684111in}}%
\pgfpathlineto{\pgfqpoint{1.131834in}{0.684071in}}%
\pgfpathlineto{\pgfqpoint{1.152628in}{0.684059in}}%
\pgfpathlineto{\pgfqpoint{1.251312in}{0.683263in}}%
\pgfpathlineto{\pgfqpoint{1.277476in}{0.683159in}}%
\pgfpathlineto{\pgfqpoint{1.314870in}{0.682855in}}%
\pgfpathlineto{\pgfqpoint{1.369253in}{0.682756in}}%
\pgfpathlineto{\pgfqpoint{1.398687in}{0.682288in}}%
\pgfpathlineto{\pgfqpoint{1.467852in}{0.682134in}}%
\pgfpathlineto{\pgfqpoint{1.557026in}{0.681680in}}%
\pgfpathlineto{\pgfqpoint{1.627242in}{0.680913in}}%
\pgfpathlineto{\pgfqpoint{1.737728in}{0.680478in}}%
\pgfpathlineto{\pgfqpoint{1.887036in}{0.679610in}}%
\pgfpathlineto{\pgfqpoint{2.037481in}{0.678826in}}%
\pgfpathlineto{\pgfqpoint{2.258348in}{0.677741in}}%
\pgfpathlineto{\pgfqpoint{2.626338in}{0.676361in}}%
\pgfpathlineto{\pgfqpoint{3.263784in}{0.674352in}}%
\pgfpathlineto{\pgfqpoint{5.322800in}{0.670138in}}%
\pgfpathlineto{\pgfqpoint{6.888192in}{0.683471in}}%
\pgfpathlineto{\pgfqpoint{4.623274in}{0.688107in}}%
\pgfpathlineto{\pgfqpoint{3.922083in}{0.690316in}}%
\pgfpathlineto{\pgfqpoint{3.517294in}{0.691835in}}%
\pgfpathlineto{\pgfqpoint{3.274341in}{0.693028in}}%
\pgfpathlineto{\pgfqpoint{3.108851in}{0.693891in}}%
\pgfpathlineto{\pgfqpoint{2.944612in}{0.694845in}}%
\pgfpathlineto{\pgfqpoint{2.823078in}{0.695324in}}%
\pgfpathlineto{\pgfqpoint{2.745840in}{0.696168in}}%
\pgfpathlineto{\pgfqpoint{2.647748in}{0.696667in}}%
\pgfpathlineto{\pgfqpoint{2.571667in}{0.696836in}}%
\pgfpathlineto{\pgfqpoint{2.539290in}{0.697351in}}%
\pgfpathlineto{\pgfqpoint{2.479468in}{0.697460in}}%
\pgfpathlineto{\pgfqpoint{2.438334in}{0.697795in}}%
\pgfpathlineto{\pgfqpoint{2.409554in}{0.697909in}}%
\pgfpathlineto{\pgfqpoint{2.301002in}{0.698784in}}%
\pgfpathlineto{\pgfqpoint{2.278129in}{0.698798in}}%
\pgfpathlineto{\pgfqpoint{2.259841in}{0.698842in}}%
\pgfpathlineto{\pgfqpoint{2.234540in}{0.699174in}}%
\pgfpathlineto{\pgfqpoint{2.212467in}{0.699905in}}%
\pgfpathlineto{\pgfqpoint{2.198690in}{0.701434in}}%
\pgfpathlineto{\pgfqpoint{2.193297in}{0.702172in}}%
\pgfpathlineto{\pgfqpoint{2.187458in}{0.702763in}}%
\pgfpathlineto{\pgfqpoint{2.171432in}{0.705082in}}%
\pgfpathlineto{\pgfqpoint{2.163334in}{0.705692in}}%
\pgfpathlineto{\pgfqpoint{2.161059in}{0.706043in}}%
\pgfpathlineto{\pgfqpoint{2.157055in}{0.710594in}}%
\pgfpathlineto{\pgfqpoint{2.105914in}{0.712924in}}%
\pgfpathlineto{\pgfqpoint{2.104964in}{0.713831in}}%
\pgfpathlineto{\pgfqpoint{2.098822in}{0.714079in}}%
\pgfpathlineto{\pgfqpoint{2.085843in}{0.716510in}}%
\pgfpathlineto{\pgfqpoint{2.080349in}{0.716749in}}%
\pgfpathlineto{\pgfqpoint{2.080251in}{0.718197in}}%
\pgfpathlineto{\pgfqpoint{2.076904in}{0.719544in}}%
\pgfpathlineto{\pgfqpoint{2.051349in}{0.722081in}}%
\pgfpathlineto{\pgfqpoint{2.045595in}{0.723715in}}%
\pgfpathlineto{\pgfqpoint{2.037125in}{0.724332in}}%
\pgfpathlineto{\pgfqpoint{2.033576in}{0.725697in}}%
\pgfpathlineto{\pgfqpoint{2.032252in}{0.729528in}}%
\pgfpathlineto{\pgfqpoint{2.031338in}{0.731724in}}%
\pgfpathlineto{\pgfqpoint{2.008926in}{0.733388in}}%
\pgfpathlineto{\pgfqpoint{1.984450in}{0.738340in}}%
\pgfpathlineto{\pgfqpoint{1.966417in}{0.745282in}}%
\pgfpathlineto{\pgfqpoint{1.949459in}{0.750264in}}%
\pgfpathlineto{\pgfqpoint{1.945246in}{0.752971in}}%
\pgfpathlineto{\pgfqpoint{1.944819in}{0.754750in}}%
\pgfpathlineto{\pgfqpoint{1.938397in}{0.758644in}}%
\pgfpathlineto{\pgfqpoint{1.930681in}{0.763152in}}%
\pgfpathlineto{\pgfqpoint{1.927003in}{0.764692in}}%
\pgfpathlineto{\pgfqpoint{1.923418in}{0.773495in}}%
\pgfpathlineto{\pgfqpoint{1.912932in}{0.774543in}}%
\pgfpathlineto{\pgfqpoint{1.907246in}{0.780181in}}%
\pgfpathlineto{\pgfqpoint{1.904902in}{0.803379in}}%
\pgfpathlineto{\pgfqpoint{1.898279in}{0.804188in}}%
\pgfpathlineto{\pgfqpoint{1.893486in}{0.814892in}}%
\pgfpathlineto{\pgfqpoint{1.889178in}{0.851064in}}%
\pgfpathlineto{\pgfqpoint{1.885331in}{0.857978in}}%
\pgfpathlineto{\pgfqpoint{1.878129in}{0.908856in}}%
\pgfpathlineto{\pgfqpoint{1.867684in}{0.996854in}}%
\pgfpathlineto{\pgfqpoint{1.848679in}{1.461617in}}%
\pgfpathlineto{\pgfqpoint{0.741425in}{1.377543in}}%
\pgfpathclose%
\pgfusepath{stroke,fill}%
\end{pgfscope}%
\begin{pgfscope}%
\pgfpathrectangle{\pgfqpoint{0.688192in}{0.670138in}}{\pgfqpoint{6.200000in}{4.620000in}}%
\pgfusepath{clip}%
\pgfsetbuttcap%
\pgfsetroundjoin%
\pgfsetlinewidth{1.003750pt}%
\definecolor{currentstroke}{rgb}{1.000000,0.000000,0.000000}%
\pgfsetstrokecolor{currentstroke}%
\pgfsetdash{}{0pt}%
\pgfpathmoveto{\pgfqpoint{1.380312in}{4.826960in}}%
\pgfpathcurveto{\pgfqpoint{1.388548in}{4.826960in}}{\pgfqpoint{1.396449in}{4.830233in}}{\pgfqpoint{1.402272in}{4.836057in}}%
\pgfpathcurveto{\pgfqpoint{1.408096in}{4.841881in}}{\pgfqpoint{1.411369in}{4.849781in}}{\pgfqpoint{1.411369in}{4.858017in}}%
\pgfpathcurveto{\pgfqpoint{1.411369in}{4.866253in}}{\pgfqpoint{1.408096in}{4.874153in}}{\pgfqpoint{1.402272in}{4.879977in}}%
\pgfpathcurveto{\pgfqpoint{1.396449in}{4.885801in}}{\pgfqpoint{1.388548in}{4.889073in}}{\pgfqpoint{1.380312in}{4.889073in}}%
\pgfpathcurveto{\pgfqpoint{1.372076in}{4.889073in}}{\pgfqpoint{1.364176in}{4.885801in}}{\pgfqpoint{1.358352in}{4.879977in}}%
\pgfpathcurveto{\pgfqpoint{1.352528in}{4.874153in}}{\pgfqpoint{1.349256in}{4.866253in}}{\pgfqpoint{1.349256in}{4.858017in}}%
\pgfpathcurveto{\pgfqpoint{1.349256in}{4.849781in}}{\pgfqpoint{1.352528in}{4.841881in}}{\pgfqpoint{1.358352in}{4.836057in}}%
\pgfpathcurveto{\pgfqpoint{1.364176in}{4.830233in}}{\pgfqpoint{1.372076in}{4.826960in}}{\pgfqpoint{1.380312in}{4.826960in}}%
\pgfpathlineto{\pgfqpoint{1.380312in}{4.826960in}}%
\pgfpathclose%
\pgfusepath{stroke}%
\end{pgfscope}%
\begin{pgfscope}%
\pgfpathrectangle{\pgfqpoint{0.688192in}{0.670138in}}{\pgfqpoint{6.200000in}{4.620000in}}%
\pgfusepath{clip}%
\pgfsetbuttcap%
\pgfsetroundjoin%
\pgfsetlinewidth{1.003750pt}%
\definecolor{currentstroke}{rgb}{1.000000,0.000000,0.000000}%
\pgfsetstrokecolor{currentstroke}%
\pgfsetdash{}{0pt}%
\pgfpathmoveto{\pgfqpoint{1.103776in}{2.239473in}}%
\pgfpathcurveto{\pgfqpoint{1.112013in}{2.239473in}}{\pgfqpoint{1.119913in}{2.242745in}}{\pgfqpoint{1.125737in}{2.248569in}}%
\pgfpathcurveto{\pgfqpoint{1.131560in}{2.254393in}}{\pgfqpoint{1.134833in}{2.262293in}}{\pgfqpoint{1.134833in}{2.270529in}}%
\pgfpathcurveto{\pgfqpoint{1.134833in}{2.278765in}}{\pgfqpoint{1.131560in}{2.286665in}}{\pgfqpoint{1.125737in}{2.292489in}}%
\pgfpathcurveto{\pgfqpoint{1.119913in}{2.298313in}}{\pgfqpoint{1.112013in}{2.301586in}}{\pgfqpoint{1.103776in}{2.301586in}}%
\pgfpathcurveto{\pgfqpoint{1.095540in}{2.301586in}}{\pgfqpoint{1.087640in}{2.298313in}}{\pgfqpoint{1.081816in}{2.292489in}}%
\pgfpathcurveto{\pgfqpoint{1.075992in}{2.286665in}}{\pgfqpoint{1.072720in}{2.278765in}}{\pgfqpoint{1.072720in}{2.270529in}}%
\pgfpathcurveto{\pgfqpoint{1.072720in}{2.262293in}}{\pgfqpoint{1.075992in}{2.254393in}}{\pgfqpoint{1.081816in}{2.248569in}}%
\pgfpathcurveto{\pgfqpoint{1.087640in}{2.242745in}}{\pgfqpoint{1.095540in}{2.239473in}}{\pgfqpoint{1.103776in}{2.239473in}}%
\pgfpathlineto{\pgfqpoint{1.103776in}{2.239473in}}%
\pgfpathclose%
\pgfusepath{stroke}%
\end{pgfscope}%
\begin{pgfscope}%
\pgfpathrectangle{\pgfqpoint{0.688192in}{0.670138in}}{\pgfqpoint{6.200000in}{4.620000in}}%
\pgfusepath{clip}%
\pgfsetbuttcap%
\pgfsetroundjoin%
\pgfsetlinewidth{1.003750pt}%
\definecolor{currentstroke}{rgb}{1.000000,0.000000,0.000000}%
\pgfsetstrokecolor{currentstroke}%
\pgfsetdash{}{0pt}%
\pgfpathmoveto{\pgfqpoint{1.146295in}{2.309659in}}%
\pgfpathcurveto{\pgfqpoint{1.154531in}{2.309659in}}{\pgfqpoint{1.162431in}{2.312932in}}{\pgfqpoint{1.168255in}{2.318756in}}%
\pgfpathcurveto{\pgfqpoint{1.174079in}{2.324579in}}{\pgfqpoint{1.177352in}{2.332480in}}{\pgfqpoint{1.177352in}{2.340716in}}%
\pgfpathcurveto{\pgfqpoint{1.177352in}{2.348952in}}{\pgfqpoint{1.174079in}{2.356852in}}{\pgfqpoint{1.168255in}{2.362676in}}%
\pgfpathcurveto{\pgfqpoint{1.162431in}{2.368500in}}{\pgfqpoint{1.154531in}{2.371772in}}{\pgfqpoint{1.146295in}{2.371772in}}%
\pgfpathcurveto{\pgfqpoint{1.138059in}{2.371772in}}{\pgfqpoint{1.130159in}{2.368500in}}{\pgfqpoint{1.124335in}{2.362676in}}%
\pgfpathcurveto{\pgfqpoint{1.118511in}{2.356852in}}{\pgfqpoint{1.115239in}{2.348952in}}{\pgfqpoint{1.115239in}{2.340716in}}%
\pgfpathcurveto{\pgfqpoint{1.115239in}{2.332480in}}{\pgfqpoint{1.118511in}{2.324579in}}{\pgfqpoint{1.124335in}{2.318756in}}%
\pgfpathcurveto{\pgfqpoint{1.130159in}{2.312932in}}{\pgfqpoint{1.138059in}{2.309659in}}{\pgfqpoint{1.146295in}{2.309659in}}%
\pgfpathlineto{\pgfqpoint{1.146295in}{2.309659in}}%
\pgfpathclose%
\pgfusepath{stroke}%
\end{pgfscope}%
\begin{pgfscope}%
\pgfpathrectangle{\pgfqpoint{0.688192in}{0.670138in}}{\pgfqpoint{6.200000in}{4.620000in}}%
\pgfusepath{clip}%
\pgfsetbuttcap%
\pgfsetroundjoin%
\pgfsetlinewidth{1.003750pt}%
\definecolor{currentstroke}{rgb}{1.000000,0.000000,0.000000}%
\pgfsetstrokecolor{currentstroke}%
\pgfsetdash{}{0pt}%
\pgfpathmoveto{\pgfqpoint{1.283972in}{2.337286in}}%
\pgfpathcurveto{\pgfqpoint{1.292208in}{2.337286in}}{\pgfqpoint{1.300108in}{2.340558in}}{\pgfqpoint{1.305932in}{2.346382in}}%
\pgfpathcurveto{\pgfqpoint{1.311756in}{2.352206in}}{\pgfqpoint{1.315028in}{2.360106in}}{\pgfqpoint{1.315028in}{2.368343in}}%
\pgfpathcurveto{\pgfqpoint{1.315028in}{2.376579in}}{\pgfqpoint{1.311756in}{2.384479in}}{\pgfqpoint{1.305932in}{2.390303in}}%
\pgfpathcurveto{\pgfqpoint{1.300108in}{2.396127in}}{\pgfqpoint{1.292208in}{2.399399in}}{\pgfqpoint{1.283972in}{2.399399in}}%
\pgfpathcurveto{\pgfqpoint{1.275735in}{2.399399in}}{\pgfqpoint{1.267835in}{2.396127in}}{\pgfqpoint{1.262011in}{2.390303in}}%
\pgfpathcurveto{\pgfqpoint{1.256187in}{2.384479in}}{\pgfqpoint{1.252915in}{2.376579in}}{\pgfqpoint{1.252915in}{2.368343in}}%
\pgfpathcurveto{\pgfqpoint{1.252915in}{2.360106in}}{\pgfqpoint{1.256187in}{2.352206in}}{\pgfqpoint{1.262011in}{2.346382in}}%
\pgfpathcurveto{\pgfqpoint{1.267835in}{2.340558in}}{\pgfqpoint{1.275735in}{2.337286in}}{\pgfqpoint{1.283972in}{2.337286in}}%
\pgfpathlineto{\pgfqpoint{1.283972in}{2.337286in}}%
\pgfpathclose%
\pgfusepath{stroke}%
\end{pgfscope}%
\begin{pgfscope}%
\pgfpathrectangle{\pgfqpoint{0.688192in}{0.670138in}}{\pgfqpoint{6.200000in}{4.620000in}}%
\pgfusepath{clip}%
\pgfsetbuttcap%
\pgfsetroundjoin%
\pgfsetlinewidth{1.003750pt}%
\definecolor{currentstroke}{rgb}{1.000000,0.000000,0.000000}%
\pgfsetstrokecolor{currentstroke}%
\pgfsetdash{}{0pt}%
\pgfpathmoveto{\pgfqpoint{1.194044in}{2.459126in}}%
\pgfpathcurveto{\pgfqpoint{1.202281in}{2.459126in}}{\pgfqpoint{1.210181in}{2.462398in}}{\pgfqpoint{1.216005in}{2.468222in}}%
\pgfpathcurveto{\pgfqpoint{1.221828in}{2.474046in}}{\pgfqpoint{1.225101in}{2.481946in}}{\pgfqpoint{1.225101in}{2.490182in}}%
\pgfpathcurveto{\pgfqpoint{1.225101in}{2.498419in}}{\pgfqpoint{1.221828in}{2.506319in}}{\pgfqpoint{1.216005in}{2.512143in}}%
\pgfpathcurveto{\pgfqpoint{1.210181in}{2.517967in}}{\pgfqpoint{1.202281in}{2.521239in}}{\pgfqpoint{1.194044in}{2.521239in}}%
\pgfpathcurveto{\pgfqpoint{1.185808in}{2.521239in}}{\pgfqpoint{1.177908in}{2.517967in}}{\pgfqpoint{1.172084in}{2.512143in}}%
\pgfpathcurveto{\pgfqpoint{1.166260in}{2.506319in}}{\pgfqpoint{1.162988in}{2.498419in}}{\pgfqpoint{1.162988in}{2.490182in}}%
\pgfpathcurveto{\pgfqpoint{1.162988in}{2.481946in}}{\pgfqpoint{1.166260in}{2.474046in}}{\pgfqpoint{1.172084in}{2.468222in}}%
\pgfpathcurveto{\pgfqpoint{1.177908in}{2.462398in}}{\pgfqpoint{1.185808in}{2.459126in}}{\pgfqpoint{1.194044in}{2.459126in}}%
\pgfpathlineto{\pgfqpoint{1.194044in}{2.459126in}}%
\pgfpathclose%
\pgfusepath{stroke}%
\end{pgfscope}%
\begin{pgfscope}%
\pgfpathrectangle{\pgfqpoint{0.688192in}{0.670138in}}{\pgfqpoint{6.200000in}{4.620000in}}%
\pgfusepath{clip}%
\pgfsetbuttcap%
\pgfsetroundjoin%
\pgfsetlinewidth{1.003750pt}%
\definecolor{currentstroke}{rgb}{1.000000,0.000000,0.000000}%
\pgfsetstrokecolor{currentstroke}%
\pgfsetdash{}{0pt}%
\pgfpathmoveto{\pgfqpoint{1.126156in}{2.039611in}}%
\pgfpathcurveto{\pgfqpoint{1.134392in}{2.039611in}}{\pgfqpoint{1.142292in}{2.042884in}}{\pgfqpoint{1.148116in}{2.048708in}}%
\pgfpathcurveto{\pgfqpoint{1.153940in}{2.054532in}}{\pgfqpoint{1.157212in}{2.062432in}}{\pgfqpoint{1.157212in}{2.070668in}}%
\pgfpathcurveto{\pgfqpoint{1.157212in}{2.078904in}}{\pgfqpoint{1.153940in}{2.086804in}}{\pgfqpoint{1.148116in}{2.092628in}}%
\pgfpathcurveto{\pgfqpoint{1.142292in}{2.098452in}}{\pgfqpoint{1.134392in}{2.101724in}}{\pgfqpoint{1.126156in}{2.101724in}}%
\pgfpathcurveto{\pgfqpoint{1.117920in}{2.101724in}}{\pgfqpoint{1.110020in}{2.098452in}}{\pgfqpoint{1.104196in}{2.092628in}}%
\pgfpathcurveto{\pgfqpoint{1.098372in}{2.086804in}}{\pgfqpoint{1.095099in}{2.078904in}}{\pgfqpoint{1.095099in}{2.070668in}}%
\pgfpathcurveto{\pgfqpoint{1.095099in}{2.062432in}}{\pgfqpoint{1.098372in}{2.054532in}}{\pgfqpoint{1.104196in}{2.048708in}}%
\pgfpathcurveto{\pgfqpoint{1.110020in}{2.042884in}}{\pgfqpoint{1.117920in}{2.039611in}}{\pgfqpoint{1.126156in}{2.039611in}}%
\pgfpathlineto{\pgfqpoint{1.126156in}{2.039611in}}%
\pgfpathclose%
\pgfusepath{stroke}%
\end{pgfscope}%
\begin{pgfscope}%
\pgfpathrectangle{\pgfqpoint{0.688192in}{0.670138in}}{\pgfqpoint{6.200000in}{4.620000in}}%
\pgfusepath{clip}%
\pgfsetbuttcap%
\pgfsetroundjoin%
\pgfsetlinewidth{1.003750pt}%
\definecolor{currentstroke}{rgb}{1.000000,0.000000,0.000000}%
\pgfsetstrokecolor{currentstroke}%
\pgfsetdash{}{0pt}%
\pgfpathmoveto{\pgfqpoint{1.073783in}{1.736453in}}%
\pgfpathcurveto{\pgfqpoint{1.082020in}{1.736453in}}{\pgfqpoint{1.089920in}{1.739725in}}{\pgfqpoint{1.095744in}{1.745549in}}%
\pgfpathcurveto{\pgfqpoint{1.101568in}{1.751373in}}{\pgfqpoint{1.104840in}{1.759273in}}{\pgfqpoint{1.104840in}{1.767510in}}%
\pgfpathcurveto{\pgfqpoint{1.104840in}{1.775746in}}{\pgfqpoint{1.101568in}{1.783646in}}{\pgfqpoint{1.095744in}{1.789470in}}%
\pgfpathcurveto{\pgfqpoint{1.089920in}{1.795294in}}{\pgfqpoint{1.082020in}{1.798566in}}{\pgfqpoint{1.073783in}{1.798566in}}%
\pgfpathcurveto{\pgfqpoint{1.065547in}{1.798566in}}{\pgfqpoint{1.057647in}{1.795294in}}{\pgfqpoint{1.051823in}{1.789470in}}%
\pgfpathcurveto{\pgfqpoint{1.045999in}{1.783646in}}{\pgfqpoint{1.042727in}{1.775746in}}{\pgfqpoint{1.042727in}{1.767510in}}%
\pgfpathcurveto{\pgfqpoint{1.042727in}{1.759273in}}{\pgfqpoint{1.045999in}{1.751373in}}{\pgfqpoint{1.051823in}{1.745549in}}%
\pgfpathcurveto{\pgfqpoint{1.057647in}{1.739725in}}{\pgfqpoint{1.065547in}{1.736453in}}{\pgfqpoint{1.073783in}{1.736453in}}%
\pgfpathlineto{\pgfqpoint{1.073783in}{1.736453in}}%
\pgfpathclose%
\pgfusepath{stroke}%
\end{pgfscope}%
\begin{pgfscope}%
\pgfpathrectangle{\pgfqpoint{0.688192in}{0.670138in}}{\pgfqpoint{6.200000in}{4.620000in}}%
\pgfusepath{clip}%
\pgfsetbuttcap%
\pgfsetroundjoin%
\pgfsetlinewidth{1.003750pt}%
\definecolor{currentstroke}{rgb}{1.000000,0.000000,0.000000}%
\pgfsetstrokecolor{currentstroke}%
\pgfsetdash{}{0pt}%
\pgfpathmoveto{\pgfqpoint{1.071350in}{1.723563in}}%
\pgfpathcurveto{\pgfqpoint{1.079586in}{1.723563in}}{\pgfqpoint{1.087486in}{1.726835in}}{\pgfqpoint{1.093310in}{1.732659in}}%
\pgfpathcurveto{\pgfqpoint{1.099134in}{1.738483in}}{\pgfqpoint{1.102407in}{1.746383in}}{\pgfqpoint{1.102407in}{1.754620in}}%
\pgfpathcurveto{\pgfqpoint{1.102407in}{1.762856in}}{\pgfqpoint{1.099134in}{1.770756in}}{\pgfqpoint{1.093310in}{1.776580in}}%
\pgfpathcurveto{\pgfqpoint{1.087486in}{1.782404in}}{\pgfqpoint{1.079586in}{1.785676in}}{\pgfqpoint{1.071350in}{1.785676in}}%
\pgfpathcurveto{\pgfqpoint{1.063114in}{1.785676in}}{\pgfqpoint{1.055214in}{1.782404in}}{\pgfqpoint{1.049390in}{1.776580in}}%
\pgfpathcurveto{\pgfqpoint{1.043566in}{1.770756in}}{\pgfqpoint{1.040294in}{1.762856in}}{\pgfqpoint{1.040294in}{1.754620in}}%
\pgfpathcurveto{\pgfqpoint{1.040294in}{1.746383in}}{\pgfqpoint{1.043566in}{1.738483in}}{\pgfqpoint{1.049390in}{1.732659in}}%
\pgfpathcurveto{\pgfqpoint{1.055214in}{1.726835in}}{\pgfqpoint{1.063114in}{1.723563in}}{\pgfqpoint{1.071350in}{1.723563in}}%
\pgfpathlineto{\pgfqpoint{1.071350in}{1.723563in}}%
\pgfpathclose%
\pgfusepath{stroke}%
\end{pgfscope}%
\begin{pgfscope}%
\pgfpathrectangle{\pgfqpoint{0.688192in}{0.670138in}}{\pgfqpoint{6.200000in}{4.620000in}}%
\pgfusepath{clip}%
\pgfsetbuttcap%
\pgfsetroundjoin%
\pgfsetlinewidth{1.003750pt}%
\definecolor{currentstroke}{rgb}{1.000000,0.000000,0.000000}%
\pgfsetstrokecolor{currentstroke}%
\pgfsetdash{}{0pt}%
\pgfpathmoveto{\pgfqpoint{1.176638in}{2.187985in}}%
\pgfpathcurveto{\pgfqpoint{1.184874in}{2.187985in}}{\pgfqpoint{1.192774in}{2.191257in}}{\pgfqpoint{1.198598in}{2.197081in}}%
\pgfpathcurveto{\pgfqpoint{1.204422in}{2.202905in}}{\pgfqpoint{1.207694in}{2.210805in}}{\pgfqpoint{1.207694in}{2.219041in}}%
\pgfpathcurveto{\pgfqpoint{1.207694in}{2.227277in}}{\pgfqpoint{1.204422in}{2.235178in}}{\pgfqpoint{1.198598in}{2.241001in}}%
\pgfpathcurveto{\pgfqpoint{1.192774in}{2.246825in}}{\pgfqpoint{1.184874in}{2.250098in}}{\pgfqpoint{1.176638in}{2.250098in}}%
\pgfpathcurveto{\pgfqpoint{1.168401in}{2.250098in}}{\pgfqpoint{1.160501in}{2.246825in}}{\pgfqpoint{1.154677in}{2.241001in}}%
\pgfpathcurveto{\pgfqpoint{1.148853in}{2.235178in}}{\pgfqpoint{1.145581in}{2.227277in}}{\pgfqpoint{1.145581in}{2.219041in}}%
\pgfpathcurveto{\pgfqpoint{1.145581in}{2.210805in}}{\pgfqpoint{1.148853in}{2.202905in}}{\pgfqpoint{1.154677in}{2.197081in}}%
\pgfpathcurveto{\pgfqpoint{1.160501in}{2.191257in}}{\pgfqpoint{1.168401in}{2.187985in}}{\pgfqpoint{1.176638in}{2.187985in}}%
\pgfpathlineto{\pgfqpoint{1.176638in}{2.187985in}}%
\pgfpathclose%
\pgfusepath{stroke}%
\end{pgfscope}%
\begin{pgfscope}%
\pgfpathrectangle{\pgfqpoint{0.688192in}{0.670138in}}{\pgfqpoint{6.200000in}{4.620000in}}%
\pgfusepath{clip}%
\pgfsetbuttcap%
\pgfsetroundjoin%
\pgfsetlinewidth{1.003750pt}%
\definecolor{currentstroke}{rgb}{1.000000,0.000000,0.000000}%
\pgfsetstrokecolor{currentstroke}%
\pgfsetdash{}{0pt}%
\pgfpathmoveto{\pgfqpoint{1.160265in}{1.939369in}}%
\pgfpathcurveto{\pgfqpoint{1.168501in}{1.939369in}}{\pgfqpoint{1.176401in}{1.942641in}}{\pgfqpoint{1.182225in}{1.948465in}}%
\pgfpathcurveto{\pgfqpoint{1.188049in}{1.954289in}}{\pgfqpoint{1.191321in}{1.962189in}}{\pgfqpoint{1.191321in}{1.970426in}}%
\pgfpathcurveto{\pgfqpoint{1.191321in}{1.978662in}}{\pgfqpoint{1.188049in}{1.986562in}}{\pgfqpoint{1.182225in}{1.992386in}}%
\pgfpathcurveto{\pgfqpoint{1.176401in}{1.998210in}}{\pgfqpoint{1.168501in}{2.001482in}}{\pgfqpoint{1.160265in}{2.001482in}}%
\pgfpathcurveto{\pgfqpoint{1.152029in}{2.001482in}}{\pgfqpoint{1.144129in}{1.998210in}}{\pgfqpoint{1.138305in}{1.992386in}}%
\pgfpathcurveto{\pgfqpoint{1.132481in}{1.986562in}}{\pgfqpoint{1.129208in}{1.978662in}}{\pgfqpoint{1.129208in}{1.970426in}}%
\pgfpathcurveto{\pgfqpoint{1.129208in}{1.962189in}}{\pgfqpoint{1.132481in}{1.954289in}}{\pgfqpoint{1.138305in}{1.948465in}}%
\pgfpathcurveto{\pgfqpoint{1.144129in}{1.942641in}}{\pgfqpoint{1.152029in}{1.939369in}}{\pgfqpoint{1.160265in}{1.939369in}}%
\pgfpathlineto{\pgfqpoint{1.160265in}{1.939369in}}%
\pgfpathclose%
\pgfusepath{stroke}%
\end{pgfscope}%
\begin{pgfscope}%
\pgfpathrectangle{\pgfqpoint{0.688192in}{0.670138in}}{\pgfqpoint{6.200000in}{4.620000in}}%
\pgfusepath{clip}%
\pgfsetbuttcap%
\pgfsetroundjoin%
\pgfsetlinewidth{1.003750pt}%
\definecolor{currentstroke}{rgb}{1.000000,0.000000,0.000000}%
\pgfsetstrokecolor{currentstroke}%
\pgfsetdash{}{0pt}%
\pgfpathmoveto{\pgfqpoint{1.160053in}{1.935902in}}%
\pgfpathcurveto{\pgfqpoint{1.168289in}{1.935902in}}{\pgfqpoint{1.176189in}{1.939174in}}{\pgfqpoint{1.182013in}{1.944998in}}%
\pgfpathcurveto{\pgfqpoint{1.187837in}{1.950822in}}{\pgfqpoint{1.191109in}{1.958722in}}{\pgfqpoint{1.191109in}{1.966959in}}%
\pgfpathcurveto{\pgfqpoint{1.191109in}{1.975195in}}{\pgfqpoint{1.187837in}{1.983095in}}{\pgfqpoint{1.182013in}{1.988919in}}%
\pgfpathcurveto{\pgfqpoint{1.176189in}{1.994743in}}{\pgfqpoint{1.168289in}{1.998015in}}{\pgfqpoint{1.160053in}{1.998015in}}%
\pgfpathcurveto{\pgfqpoint{1.151817in}{1.998015in}}{\pgfqpoint{1.143916in}{1.994743in}}{\pgfqpoint{1.138093in}{1.988919in}}%
\pgfpathcurveto{\pgfqpoint{1.132269in}{1.983095in}}{\pgfqpoint{1.128996in}{1.975195in}}{\pgfqpoint{1.128996in}{1.966959in}}%
\pgfpathcurveto{\pgfqpoint{1.128996in}{1.958722in}}{\pgfqpoint{1.132269in}{1.950822in}}{\pgfqpoint{1.138093in}{1.944998in}}%
\pgfpathcurveto{\pgfqpoint{1.143916in}{1.939174in}}{\pgfqpoint{1.151817in}{1.935902in}}{\pgfqpoint{1.160053in}{1.935902in}}%
\pgfpathlineto{\pgfqpoint{1.160053in}{1.935902in}}%
\pgfpathclose%
\pgfusepath{stroke}%
\end{pgfscope}%
\begin{pgfscope}%
\pgfpathrectangle{\pgfqpoint{0.688192in}{0.670138in}}{\pgfqpoint{6.200000in}{4.620000in}}%
\pgfusepath{clip}%
\pgfsetbuttcap%
\pgfsetroundjoin%
\pgfsetlinewidth{1.003750pt}%
\definecolor{currentstroke}{rgb}{1.000000,0.000000,0.000000}%
\pgfsetstrokecolor{currentstroke}%
\pgfsetdash{}{0pt}%
\pgfpathmoveto{\pgfqpoint{1.157986in}{1.610198in}}%
\pgfpathcurveto{\pgfqpoint{1.166223in}{1.610198in}}{\pgfqpoint{1.174123in}{1.613470in}}{\pgfqpoint{1.179947in}{1.619294in}}%
\pgfpathcurveto{\pgfqpoint{1.185771in}{1.625118in}}{\pgfqpoint{1.189043in}{1.633018in}}{\pgfqpoint{1.189043in}{1.641254in}}%
\pgfpathcurveto{\pgfqpoint{1.189043in}{1.649491in}}{\pgfqpoint{1.185771in}{1.657391in}}{\pgfqpoint{1.179947in}{1.663215in}}%
\pgfpathcurveto{\pgfqpoint{1.174123in}{1.669038in}}{\pgfqpoint{1.166223in}{1.672311in}}{\pgfqpoint{1.157986in}{1.672311in}}%
\pgfpathcurveto{\pgfqpoint{1.149750in}{1.672311in}}{\pgfqpoint{1.141850in}{1.669038in}}{\pgfqpoint{1.136026in}{1.663215in}}%
\pgfpathcurveto{\pgfqpoint{1.130202in}{1.657391in}}{\pgfqpoint{1.126930in}{1.649491in}}{\pgfqpoint{1.126930in}{1.641254in}}%
\pgfpathcurveto{\pgfqpoint{1.126930in}{1.633018in}}{\pgfqpoint{1.130202in}{1.625118in}}{\pgfqpoint{1.136026in}{1.619294in}}%
\pgfpathcurveto{\pgfqpoint{1.141850in}{1.613470in}}{\pgfqpoint{1.149750in}{1.610198in}}{\pgfqpoint{1.157986in}{1.610198in}}%
\pgfpathlineto{\pgfqpoint{1.157986in}{1.610198in}}%
\pgfpathclose%
\pgfusepath{stroke}%
\end{pgfscope}%
\begin{pgfscope}%
\pgfpathrectangle{\pgfqpoint{0.688192in}{0.670138in}}{\pgfqpoint{6.200000in}{4.620000in}}%
\pgfusepath{clip}%
\pgfsetbuttcap%
\pgfsetroundjoin%
\pgfsetlinewidth{1.003750pt}%
\definecolor{currentstroke}{rgb}{1.000000,0.000000,0.000000}%
\pgfsetstrokecolor{currentstroke}%
\pgfsetdash{}{0pt}%
\pgfpathmoveto{\pgfqpoint{1.155798in}{1.612752in}}%
\pgfpathcurveto{\pgfqpoint{1.164035in}{1.612752in}}{\pgfqpoint{1.171935in}{1.616024in}}{\pgfqpoint{1.177759in}{1.621848in}}%
\pgfpathcurveto{\pgfqpoint{1.183582in}{1.627672in}}{\pgfqpoint{1.186855in}{1.635572in}}{\pgfqpoint{1.186855in}{1.643808in}}%
\pgfpathcurveto{\pgfqpoint{1.186855in}{1.652044in}}{\pgfqpoint{1.183582in}{1.659945in}}{\pgfqpoint{1.177759in}{1.665768in}}%
\pgfpathcurveto{\pgfqpoint{1.171935in}{1.671592in}}{\pgfqpoint{1.164035in}{1.674865in}}{\pgfqpoint{1.155798in}{1.674865in}}%
\pgfpathcurveto{\pgfqpoint{1.147562in}{1.674865in}}{\pgfqpoint{1.139662in}{1.671592in}}{\pgfqpoint{1.133838in}{1.665768in}}%
\pgfpathcurveto{\pgfqpoint{1.128014in}{1.659945in}}{\pgfqpoint{1.124742in}{1.652044in}}{\pgfqpoint{1.124742in}{1.643808in}}%
\pgfpathcurveto{\pgfqpoint{1.124742in}{1.635572in}}{\pgfqpoint{1.128014in}{1.627672in}}{\pgfqpoint{1.133838in}{1.621848in}}%
\pgfpathcurveto{\pgfqpoint{1.139662in}{1.616024in}}{\pgfqpoint{1.147562in}{1.612752in}}{\pgfqpoint{1.155798in}{1.612752in}}%
\pgfpathlineto{\pgfqpoint{1.155798in}{1.612752in}}%
\pgfpathclose%
\pgfusepath{stroke}%
\end{pgfscope}%
\begin{pgfscope}%
\pgfpathrectangle{\pgfqpoint{0.688192in}{0.670138in}}{\pgfqpoint{6.200000in}{4.620000in}}%
\pgfusepath{clip}%
\pgfsetbuttcap%
\pgfsetroundjoin%
\pgfsetlinewidth{1.003750pt}%
\definecolor{currentstroke}{rgb}{1.000000,0.000000,0.000000}%
\pgfsetstrokecolor{currentstroke}%
\pgfsetdash{}{0pt}%
\pgfpathmoveto{\pgfqpoint{1.295347in}{1.895177in}}%
\pgfpathcurveto{\pgfqpoint{1.303583in}{1.895177in}}{\pgfqpoint{1.311483in}{1.898449in}}{\pgfqpoint{1.317307in}{1.904273in}}%
\pgfpathcurveto{\pgfqpoint{1.323131in}{1.910097in}}{\pgfqpoint{1.326403in}{1.917997in}}{\pgfqpoint{1.326403in}{1.926233in}}%
\pgfpathcurveto{\pgfqpoint{1.326403in}{1.934470in}}{\pgfqpoint{1.323131in}{1.942370in}}{\pgfqpoint{1.317307in}{1.948193in}}%
\pgfpathcurveto{\pgfqpoint{1.311483in}{1.954017in}}{\pgfqpoint{1.303583in}{1.957290in}}{\pgfqpoint{1.295347in}{1.957290in}}%
\pgfpathcurveto{\pgfqpoint{1.287111in}{1.957290in}}{\pgfqpoint{1.279211in}{1.954017in}}{\pgfqpoint{1.273387in}{1.948193in}}%
\pgfpathcurveto{\pgfqpoint{1.267563in}{1.942370in}}{\pgfqpoint{1.264290in}{1.934470in}}{\pgfqpoint{1.264290in}{1.926233in}}%
\pgfpathcurveto{\pgfqpoint{1.264290in}{1.917997in}}{\pgfqpoint{1.267563in}{1.910097in}}{\pgfqpoint{1.273387in}{1.904273in}}%
\pgfpathcurveto{\pgfqpoint{1.279211in}{1.898449in}}{\pgfqpoint{1.287111in}{1.895177in}}{\pgfqpoint{1.295347in}{1.895177in}}%
\pgfpathlineto{\pgfqpoint{1.295347in}{1.895177in}}%
\pgfpathclose%
\pgfusepath{stroke}%
\end{pgfscope}%
\begin{pgfscope}%
\pgfpathrectangle{\pgfqpoint{0.688192in}{0.670138in}}{\pgfqpoint{6.200000in}{4.620000in}}%
\pgfusepath{clip}%
\pgfsetbuttcap%
\pgfsetroundjoin%
\pgfsetlinewidth{1.003750pt}%
\definecolor{currentstroke}{rgb}{1.000000,0.000000,0.000000}%
\pgfsetstrokecolor{currentstroke}%
\pgfsetdash{}{0pt}%
\pgfpathmoveto{\pgfqpoint{1.160535in}{1.595409in}}%
\pgfpathcurveto{\pgfqpoint{1.168772in}{1.595409in}}{\pgfqpoint{1.176672in}{1.598682in}}{\pgfqpoint{1.182496in}{1.604506in}}%
\pgfpathcurveto{\pgfqpoint{1.188320in}{1.610330in}}{\pgfqpoint{1.191592in}{1.618230in}}{\pgfqpoint{1.191592in}{1.626466in}}%
\pgfpathcurveto{\pgfqpoint{1.191592in}{1.634702in}}{\pgfqpoint{1.188320in}{1.642602in}}{\pgfqpoint{1.182496in}{1.648426in}}%
\pgfpathcurveto{\pgfqpoint{1.176672in}{1.654250in}}{\pgfqpoint{1.168772in}{1.657522in}}{\pgfqpoint{1.160535in}{1.657522in}}%
\pgfpathcurveto{\pgfqpoint{1.152299in}{1.657522in}}{\pgfqpoint{1.144399in}{1.654250in}}{\pgfqpoint{1.138575in}{1.648426in}}%
\pgfpathcurveto{\pgfqpoint{1.132751in}{1.642602in}}{\pgfqpoint{1.129479in}{1.634702in}}{\pgfqpoint{1.129479in}{1.626466in}}%
\pgfpathcurveto{\pgfqpoint{1.129479in}{1.618230in}}{\pgfqpoint{1.132751in}{1.610330in}}{\pgfqpoint{1.138575in}{1.604506in}}%
\pgfpathcurveto{\pgfqpoint{1.144399in}{1.598682in}}{\pgfqpoint{1.152299in}{1.595409in}}{\pgfqpoint{1.160535in}{1.595409in}}%
\pgfpathlineto{\pgfqpoint{1.160535in}{1.595409in}}%
\pgfpathclose%
\pgfusepath{stroke}%
\end{pgfscope}%
\begin{pgfscope}%
\pgfpathrectangle{\pgfqpoint{0.688192in}{0.670138in}}{\pgfqpoint{6.200000in}{4.620000in}}%
\pgfusepath{clip}%
\pgfsetbuttcap%
\pgfsetroundjoin%
\pgfsetlinewidth{1.003750pt}%
\definecolor{currentstroke}{rgb}{1.000000,0.000000,0.000000}%
\pgfsetstrokecolor{currentstroke}%
\pgfsetdash{}{0pt}%
\pgfpathmoveto{\pgfqpoint{1.265571in}{1.691299in}}%
\pgfpathcurveto{\pgfqpoint{1.273807in}{1.691299in}}{\pgfqpoint{1.281707in}{1.694572in}}{\pgfqpoint{1.287531in}{1.700396in}}%
\pgfpathcurveto{\pgfqpoint{1.293355in}{1.706220in}}{\pgfqpoint{1.296627in}{1.714120in}}{\pgfqpoint{1.296627in}{1.722356in}}%
\pgfpathcurveto{\pgfqpoint{1.296627in}{1.730592in}}{\pgfqpoint{1.293355in}{1.738492in}}{\pgfqpoint{1.287531in}{1.744316in}}%
\pgfpathcurveto{\pgfqpoint{1.281707in}{1.750140in}}{\pgfqpoint{1.273807in}{1.753412in}}{\pgfqpoint{1.265571in}{1.753412in}}%
\pgfpathcurveto{\pgfqpoint{1.257335in}{1.753412in}}{\pgfqpoint{1.249435in}{1.750140in}}{\pgfqpoint{1.243611in}{1.744316in}}%
\pgfpathcurveto{\pgfqpoint{1.237787in}{1.738492in}}{\pgfqpoint{1.234514in}{1.730592in}}{\pgfqpoint{1.234514in}{1.722356in}}%
\pgfpathcurveto{\pgfqpoint{1.234514in}{1.714120in}}{\pgfqpoint{1.237787in}{1.706220in}}{\pgfqpoint{1.243611in}{1.700396in}}%
\pgfpathcurveto{\pgfqpoint{1.249435in}{1.694572in}}{\pgfqpoint{1.257335in}{1.691299in}}{\pgfqpoint{1.265571in}{1.691299in}}%
\pgfpathlineto{\pgfqpoint{1.265571in}{1.691299in}}%
\pgfpathclose%
\pgfusepath{stroke}%
\end{pgfscope}%
\begin{pgfscope}%
\pgfpathrectangle{\pgfqpoint{0.688192in}{0.670138in}}{\pgfqpoint{6.200000in}{4.620000in}}%
\pgfusepath{clip}%
\pgfsetbuttcap%
\pgfsetroundjoin%
\pgfsetlinewidth{1.003750pt}%
\definecolor{currentstroke}{rgb}{1.000000,0.000000,0.000000}%
\pgfsetstrokecolor{currentstroke}%
\pgfsetdash{}{0pt}%
\pgfpathmoveto{\pgfqpoint{1.297728in}{1.636949in}}%
\pgfpathcurveto{\pgfqpoint{1.305964in}{1.636949in}}{\pgfqpoint{1.313864in}{1.640222in}}{\pgfqpoint{1.319688in}{1.646045in}}%
\pgfpathcurveto{\pgfqpoint{1.325512in}{1.651869in}}{\pgfqpoint{1.328784in}{1.659769in}}{\pgfqpoint{1.328784in}{1.668006in}}%
\pgfpathcurveto{\pgfqpoint{1.328784in}{1.676242in}}{\pgfqpoint{1.325512in}{1.684142in}}{\pgfqpoint{1.319688in}{1.689966in}}%
\pgfpathcurveto{\pgfqpoint{1.313864in}{1.695790in}}{\pgfqpoint{1.305964in}{1.699062in}}{\pgfqpoint{1.297728in}{1.699062in}}%
\pgfpathcurveto{\pgfqpoint{1.289491in}{1.699062in}}{\pgfqpoint{1.281591in}{1.695790in}}{\pgfqpoint{1.275767in}{1.689966in}}%
\pgfpathcurveto{\pgfqpoint{1.269943in}{1.684142in}}{\pgfqpoint{1.266671in}{1.676242in}}{\pgfqpoint{1.266671in}{1.668006in}}%
\pgfpathcurveto{\pgfqpoint{1.266671in}{1.659769in}}{\pgfqpoint{1.269943in}{1.651869in}}{\pgfqpoint{1.275767in}{1.646045in}}%
\pgfpathcurveto{\pgfqpoint{1.281591in}{1.640222in}}{\pgfqpoint{1.289491in}{1.636949in}}{\pgfqpoint{1.297728in}{1.636949in}}%
\pgfpathlineto{\pgfqpoint{1.297728in}{1.636949in}}%
\pgfpathclose%
\pgfusepath{stroke}%
\end{pgfscope}%
\begin{pgfscope}%
\pgfpathrectangle{\pgfqpoint{0.688192in}{0.670138in}}{\pgfqpoint{6.200000in}{4.620000in}}%
\pgfusepath{clip}%
\pgfsetbuttcap%
\pgfsetroundjoin%
\pgfsetlinewidth{1.003750pt}%
\definecolor{currentstroke}{rgb}{1.000000,0.000000,0.000000}%
\pgfsetstrokecolor{currentstroke}%
\pgfsetdash{}{0pt}%
\pgfpathmoveto{\pgfqpoint{1.147178in}{1.734556in}}%
\pgfpathcurveto{\pgfqpoint{1.155415in}{1.734556in}}{\pgfqpoint{1.163315in}{1.737828in}}{\pgfqpoint{1.169139in}{1.743652in}}%
\pgfpathcurveto{\pgfqpoint{1.174962in}{1.749476in}}{\pgfqpoint{1.178235in}{1.757376in}}{\pgfqpoint{1.178235in}{1.765612in}}%
\pgfpathcurveto{\pgfqpoint{1.178235in}{1.773848in}}{\pgfqpoint{1.174962in}{1.781749in}}{\pgfqpoint{1.169139in}{1.787572in}}%
\pgfpathcurveto{\pgfqpoint{1.163315in}{1.793396in}}{\pgfqpoint{1.155415in}{1.796669in}}{\pgfqpoint{1.147178in}{1.796669in}}%
\pgfpathcurveto{\pgfqpoint{1.138942in}{1.796669in}}{\pgfqpoint{1.131042in}{1.793396in}}{\pgfqpoint{1.125218in}{1.787572in}}%
\pgfpathcurveto{\pgfqpoint{1.119394in}{1.781749in}}{\pgfqpoint{1.116122in}{1.773848in}}{\pgfqpoint{1.116122in}{1.765612in}}%
\pgfpathcurveto{\pgfqpoint{1.116122in}{1.757376in}}{\pgfqpoint{1.119394in}{1.749476in}}{\pgfqpoint{1.125218in}{1.743652in}}%
\pgfpathcurveto{\pgfqpoint{1.131042in}{1.737828in}}{\pgfqpoint{1.138942in}{1.734556in}}{\pgfqpoint{1.147178in}{1.734556in}}%
\pgfpathlineto{\pgfqpoint{1.147178in}{1.734556in}}%
\pgfpathclose%
\pgfusepath{stroke}%
\end{pgfscope}%
\begin{pgfscope}%
\pgfpathrectangle{\pgfqpoint{0.688192in}{0.670138in}}{\pgfqpoint{6.200000in}{4.620000in}}%
\pgfusepath{clip}%
\pgfsetbuttcap%
\pgfsetroundjoin%
\pgfsetlinewidth{1.003750pt}%
\definecolor{currentstroke}{rgb}{1.000000,0.000000,0.000000}%
\pgfsetstrokecolor{currentstroke}%
\pgfsetdash{}{0pt}%
\pgfpathmoveto{\pgfqpoint{1.240097in}{1.549585in}}%
\pgfpathcurveto{\pgfqpoint{1.248334in}{1.549585in}}{\pgfqpoint{1.256234in}{1.552857in}}{\pgfqpoint{1.262058in}{1.558681in}}%
\pgfpathcurveto{\pgfqpoint{1.267881in}{1.564505in}}{\pgfqpoint{1.271154in}{1.572405in}}{\pgfqpoint{1.271154in}{1.580642in}}%
\pgfpathcurveto{\pgfqpoint{1.271154in}{1.588878in}}{\pgfqpoint{1.267881in}{1.596778in}}{\pgfqpoint{1.262058in}{1.602602in}}%
\pgfpathcurveto{\pgfqpoint{1.256234in}{1.608426in}}{\pgfqpoint{1.248334in}{1.611698in}}{\pgfqpoint{1.240097in}{1.611698in}}%
\pgfpathcurveto{\pgfqpoint{1.231861in}{1.611698in}}{\pgfqpoint{1.223961in}{1.608426in}}{\pgfqpoint{1.218137in}{1.602602in}}%
\pgfpathcurveto{\pgfqpoint{1.212313in}{1.596778in}}{\pgfqpoint{1.209041in}{1.588878in}}{\pgfqpoint{1.209041in}{1.580642in}}%
\pgfpathcurveto{\pgfqpoint{1.209041in}{1.572405in}}{\pgfqpoint{1.212313in}{1.564505in}}{\pgfqpoint{1.218137in}{1.558681in}}%
\pgfpathcurveto{\pgfqpoint{1.223961in}{1.552857in}}{\pgfqpoint{1.231861in}{1.549585in}}{\pgfqpoint{1.240097in}{1.549585in}}%
\pgfpathlineto{\pgfqpoint{1.240097in}{1.549585in}}%
\pgfpathclose%
\pgfusepath{stroke}%
\end{pgfscope}%
\begin{pgfscope}%
\pgfpathrectangle{\pgfqpoint{0.688192in}{0.670138in}}{\pgfqpoint{6.200000in}{4.620000in}}%
\pgfusepath{clip}%
\pgfsetbuttcap%
\pgfsetroundjoin%
\pgfsetlinewidth{1.003750pt}%
\definecolor{currentstroke}{rgb}{1.000000,0.000000,0.000000}%
\pgfsetstrokecolor{currentstroke}%
\pgfsetdash{}{0pt}%
\pgfpathmoveto{\pgfqpoint{1.275017in}{1.474367in}}%
\pgfpathcurveto{\pgfqpoint{1.283253in}{1.474367in}}{\pgfqpoint{1.291153in}{1.477639in}}{\pgfqpoint{1.296977in}{1.483463in}}%
\pgfpathcurveto{\pgfqpoint{1.302801in}{1.489287in}}{\pgfqpoint{1.306073in}{1.497187in}}{\pgfqpoint{1.306073in}{1.505423in}}%
\pgfpathcurveto{\pgfqpoint{1.306073in}{1.513660in}}{\pgfqpoint{1.302801in}{1.521560in}}{\pgfqpoint{1.296977in}{1.527384in}}%
\pgfpathcurveto{\pgfqpoint{1.291153in}{1.533208in}}{\pgfqpoint{1.283253in}{1.536480in}}{\pgfqpoint{1.275017in}{1.536480in}}%
\pgfpathcurveto{\pgfqpoint{1.266780in}{1.536480in}}{\pgfqpoint{1.258880in}{1.533208in}}{\pgfqpoint{1.253056in}{1.527384in}}%
\pgfpathcurveto{\pgfqpoint{1.247232in}{1.521560in}}{\pgfqpoint{1.243960in}{1.513660in}}{\pgfqpoint{1.243960in}{1.505423in}}%
\pgfpathcurveto{\pgfqpoint{1.243960in}{1.497187in}}{\pgfqpoint{1.247232in}{1.489287in}}{\pgfqpoint{1.253056in}{1.483463in}}%
\pgfpathcurveto{\pgfqpoint{1.258880in}{1.477639in}}{\pgfqpoint{1.266780in}{1.474367in}}{\pgfqpoint{1.275017in}{1.474367in}}%
\pgfpathlineto{\pgfqpoint{1.275017in}{1.474367in}}%
\pgfpathclose%
\pgfusepath{stroke}%
\end{pgfscope}%
\begin{pgfscope}%
\pgfpathrectangle{\pgfqpoint{0.688192in}{0.670138in}}{\pgfqpoint{6.200000in}{4.620000in}}%
\pgfusepath{clip}%
\pgfsetbuttcap%
\pgfsetroundjoin%
\pgfsetlinewidth{1.003750pt}%
\definecolor{currentstroke}{rgb}{1.000000,0.000000,0.000000}%
\pgfsetstrokecolor{currentstroke}%
\pgfsetdash{}{0pt}%
\pgfpathmoveto{\pgfqpoint{1.274070in}{1.506532in}}%
\pgfpathcurveto{\pgfqpoint{1.282306in}{1.506532in}}{\pgfqpoint{1.290206in}{1.509804in}}{\pgfqpoint{1.296030in}{1.515628in}}%
\pgfpathcurveto{\pgfqpoint{1.301854in}{1.521452in}}{\pgfqpoint{1.305126in}{1.529352in}}{\pgfqpoint{1.305126in}{1.537589in}}%
\pgfpathcurveto{\pgfqpoint{1.305126in}{1.545825in}}{\pgfqpoint{1.301854in}{1.553725in}}{\pgfqpoint{1.296030in}{1.559549in}}%
\pgfpathcurveto{\pgfqpoint{1.290206in}{1.565373in}}{\pgfqpoint{1.282306in}{1.568645in}}{\pgfqpoint{1.274070in}{1.568645in}}%
\pgfpathcurveto{\pgfqpoint{1.265833in}{1.568645in}}{\pgfqpoint{1.257933in}{1.565373in}}{\pgfqpoint{1.252109in}{1.559549in}}%
\pgfpathcurveto{\pgfqpoint{1.246285in}{1.553725in}}{\pgfqpoint{1.243013in}{1.545825in}}{\pgfqpoint{1.243013in}{1.537589in}}%
\pgfpathcurveto{\pgfqpoint{1.243013in}{1.529352in}}{\pgfqpoint{1.246285in}{1.521452in}}{\pgfqpoint{1.252109in}{1.515628in}}%
\pgfpathcurveto{\pgfqpoint{1.257933in}{1.509804in}}{\pgfqpoint{1.265833in}{1.506532in}}{\pgfqpoint{1.274070in}{1.506532in}}%
\pgfpathlineto{\pgfqpoint{1.274070in}{1.506532in}}%
\pgfpathclose%
\pgfusepath{stroke}%
\end{pgfscope}%
\begin{pgfscope}%
\pgfpathrectangle{\pgfqpoint{0.688192in}{0.670138in}}{\pgfqpoint{6.200000in}{4.620000in}}%
\pgfusepath{clip}%
\pgfsetbuttcap%
\pgfsetroundjoin%
\pgfsetlinewidth{1.003750pt}%
\definecolor{currentstroke}{rgb}{1.000000,0.000000,0.000000}%
\pgfsetstrokecolor{currentstroke}%
\pgfsetdash{}{0pt}%
\pgfpathmoveto{\pgfqpoint{1.147538in}{1.722969in}}%
\pgfpathcurveto{\pgfqpoint{1.155775in}{1.722969in}}{\pgfqpoint{1.163675in}{1.726242in}}{\pgfqpoint{1.169499in}{1.732066in}}%
\pgfpathcurveto{\pgfqpoint{1.175323in}{1.737890in}}{\pgfqpoint{1.178595in}{1.745790in}}{\pgfqpoint{1.178595in}{1.754026in}}%
\pgfpathcurveto{\pgfqpoint{1.178595in}{1.762262in}}{\pgfqpoint{1.175323in}{1.770162in}}{\pgfqpoint{1.169499in}{1.775986in}}%
\pgfpathcurveto{\pgfqpoint{1.163675in}{1.781810in}}{\pgfqpoint{1.155775in}{1.785082in}}{\pgfqpoint{1.147538in}{1.785082in}}%
\pgfpathcurveto{\pgfqpoint{1.139302in}{1.785082in}}{\pgfqpoint{1.131402in}{1.781810in}}{\pgfqpoint{1.125578in}{1.775986in}}%
\pgfpathcurveto{\pgfqpoint{1.119754in}{1.770162in}}{\pgfqpoint{1.116482in}{1.762262in}}{\pgfqpoint{1.116482in}{1.754026in}}%
\pgfpathcurveto{\pgfqpoint{1.116482in}{1.745790in}}{\pgfqpoint{1.119754in}{1.737890in}}{\pgfqpoint{1.125578in}{1.732066in}}%
\pgfpathcurveto{\pgfqpoint{1.131402in}{1.726242in}}{\pgfqpoint{1.139302in}{1.722969in}}{\pgfqpoint{1.147538in}{1.722969in}}%
\pgfpathlineto{\pgfqpoint{1.147538in}{1.722969in}}%
\pgfpathclose%
\pgfusepath{stroke}%
\end{pgfscope}%
\begin{pgfscope}%
\pgfpathrectangle{\pgfqpoint{0.688192in}{0.670138in}}{\pgfqpoint{6.200000in}{4.620000in}}%
\pgfusepath{clip}%
\pgfsetbuttcap%
\pgfsetroundjoin%
\pgfsetlinewidth{1.003750pt}%
\definecolor{currentstroke}{rgb}{1.000000,0.000000,0.000000}%
\pgfsetstrokecolor{currentstroke}%
\pgfsetdash{}{0pt}%
\pgfpathmoveto{\pgfqpoint{1.194960in}{1.895058in}}%
\pgfpathcurveto{\pgfqpoint{1.203197in}{1.895058in}}{\pgfqpoint{1.211097in}{1.898330in}}{\pgfqpoint{1.216921in}{1.904154in}}%
\pgfpathcurveto{\pgfqpoint{1.222745in}{1.909978in}}{\pgfqpoint{1.226017in}{1.917878in}}{\pgfqpoint{1.226017in}{1.926114in}}%
\pgfpathcurveto{\pgfqpoint{1.226017in}{1.934351in}}{\pgfqpoint{1.222745in}{1.942251in}}{\pgfqpoint{1.216921in}{1.948075in}}%
\pgfpathcurveto{\pgfqpoint{1.211097in}{1.953898in}}{\pgfqpoint{1.203197in}{1.957171in}}{\pgfqpoint{1.194960in}{1.957171in}}%
\pgfpathcurveto{\pgfqpoint{1.186724in}{1.957171in}}{\pgfqpoint{1.178824in}{1.953898in}}{\pgfqpoint{1.173000in}{1.948075in}}%
\pgfpathcurveto{\pgfqpoint{1.167176in}{1.942251in}}{\pgfqpoint{1.163904in}{1.934351in}}{\pgfqpoint{1.163904in}{1.926114in}}%
\pgfpathcurveto{\pgfqpoint{1.163904in}{1.917878in}}{\pgfqpoint{1.167176in}{1.909978in}}{\pgfqpoint{1.173000in}{1.904154in}}%
\pgfpathcurveto{\pgfqpoint{1.178824in}{1.898330in}}{\pgfqpoint{1.186724in}{1.895058in}}{\pgfqpoint{1.194960in}{1.895058in}}%
\pgfpathlineto{\pgfqpoint{1.194960in}{1.895058in}}%
\pgfpathclose%
\pgfusepath{stroke}%
\end{pgfscope}%
\begin{pgfscope}%
\pgfpathrectangle{\pgfqpoint{0.688192in}{0.670138in}}{\pgfqpoint{6.200000in}{4.620000in}}%
\pgfusepath{clip}%
\pgfsetbuttcap%
\pgfsetroundjoin%
\pgfsetlinewidth{1.003750pt}%
\definecolor{currentstroke}{rgb}{1.000000,0.000000,0.000000}%
\pgfsetstrokecolor{currentstroke}%
\pgfsetdash{}{0pt}%
\pgfpathmoveto{\pgfqpoint{1.164778in}{1.905975in}}%
\pgfpathcurveto{\pgfqpoint{1.173015in}{1.905975in}}{\pgfqpoint{1.180915in}{1.909248in}}{\pgfqpoint{1.186738in}{1.915072in}}%
\pgfpathcurveto{\pgfqpoint{1.192562in}{1.920896in}}{\pgfqpoint{1.195835in}{1.928796in}}{\pgfqpoint{1.195835in}{1.937032in}}%
\pgfpathcurveto{\pgfqpoint{1.195835in}{1.945268in}}{\pgfqpoint{1.192562in}{1.953168in}}{\pgfqpoint{1.186738in}{1.958992in}}%
\pgfpathcurveto{\pgfqpoint{1.180915in}{1.964816in}}{\pgfqpoint{1.173015in}{1.968088in}}{\pgfqpoint{1.164778in}{1.968088in}}%
\pgfpathcurveto{\pgfqpoint{1.156542in}{1.968088in}}{\pgfqpoint{1.148642in}{1.964816in}}{\pgfqpoint{1.142818in}{1.958992in}}%
\pgfpathcurveto{\pgfqpoint{1.136994in}{1.953168in}}{\pgfqpoint{1.133722in}{1.945268in}}{\pgfqpoint{1.133722in}{1.937032in}}%
\pgfpathcurveto{\pgfqpoint{1.133722in}{1.928796in}}{\pgfqpoint{1.136994in}{1.920896in}}{\pgfqpoint{1.142818in}{1.915072in}}%
\pgfpathcurveto{\pgfqpoint{1.148642in}{1.909248in}}{\pgfqpoint{1.156542in}{1.905975in}}{\pgfqpoint{1.164778in}{1.905975in}}%
\pgfpathlineto{\pgfqpoint{1.164778in}{1.905975in}}%
\pgfpathclose%
\pgfusepath{stroke}%
\end{pgfscope}%
\begin{pgfscope}%
\pgfpathrectangle{\pgfqpoint{0.688192in}{0.670138in}}{\pgfqpoint{6.200000in}{4.620000in}}%
\pgfusepath{clip}%
\pgfsetbuttcap%
\pgfsetroundjoin%
\pgfsetlinewidth{1.003750pt}%
\definecolor{currentstroke}{rgb}{1.000000,0.000000,0.000000}%
\pgfsetstrokecolor{currentstroke}%
\pgfsetdash{}{0pt}%
\pgfpathmoveto{\pgfqpoint{1.143899in}{1.731794in}}%
\pgfpathcurveto{\pgfqpoint{1.152135in}{1.731794in}}{\pgfqpoint{1.160035in}{1.735067in}}{\pgfqpoint{1.165859in}{1.740890in}}%
\pgfpathcurveto{\pgfqpoint{1.171683in}{1.746714in}}{\pgfqpoint{1.174956in}{1.754614in}}{\pgfqpoint{1.174956in}{1.762851in}}%
\pgfpathcurveto{\pgfqpoint{1.174956in}{1.771087in}}{\pgfqpoint{1.171683in}{1.778987in}}{\pgfqpoint{1.165859in}{1.784811in}}%
\pgfpathcurveto{\pgfqpoint{1.160035in}{1.790635in}}{\pgfqpoint{1.152135in}{1.793907in}}{\pgfqpoint{1.143899in}{1.793907in}}%
\pgfpathcurveto{\pgfqpoint{1.135663in}{1.793907in}}{\pgfqpoint{1.127763in}{1.790635in}}{\pgfqpoint{1.121939in}{1.784811in}}%
\pgfpathcurveto{\pgfqpoint{1.116115in}{1.778987in}}{\pgfqpoint{1.112843in}{1.771087in}}{\pgfqpoint{1.112843in}{1.762851in}}%
\pgfpathcurveto{\pgfqpoint{1.112843in}{1.754614in}}{\pgfqpoint{1.116115in}{1.746714in}}{\pgfqpoint{1.121939in}{1.740890in}}%
\pgfpathcurveto{\pgfqpoint{1.127763in}{1.735067in}}{\pgfqpoint{1.135663in}{1.731794in}}{\pgfqpoint{1.143899in}{1.731794in}}%
\pgfpathlineto{\pgfqpoint{1.143899in}{1.731794in}}%
\pgfpathclose%
\pgfusepath{stroke}%
\end{pgfscope}%
\begin{pgfscope}%
\pgfpathrectangle{\pgfqpoint{0.688192in}{0.670138in}}{\pgfqpoint{6.200000in}{4.620000in}}%
\pgfusepath{clip}%
\pgfsetbuttcap%
\pgfsetroundjoin%
\pgfsetlinewidth{1.003750pt}%
\definecolor{currentstroke}{rgb}{1.000000,0.000000,0.000000}%
\pgfsetstrokecolor{currentstroke}%
\pgfsetdash{}{0pt}%
\pgfpathmoveto{\pgfqpoint{0.688192in}{1.225706in}}%
\pgfpathcurveto{\pgfqpoint{0.696428in}{1.225706in}}{\pgfqpoint{0.704328in}{1.228978in}}{\pgfqpoint{0.710152in}{1.234802in}}%
\pgfpathcurveto{\pgfqpoint{0.715976in}{1.240626in}}{\pgfqpoint{0.719248in}{1.248526in}}{\pgfqpoint{0.719248in}{1.256762in}}%
\pgfpathcurveto{\pgfqpoint{0.719248in}{1.264998in}}{\pgfqpoint{0.715976in}{1.272898in}}{\pgfqpoint{0.710152in}{1.278722in}}%
\pgfpathcurveto{\pgfqpoint{0.704328in}{1.284546in}}{\pgfqpoint{0.696428in}{1.287819in}}{\pgfqpoint{0.688192in}{1.287819in}}%
\pgfpathcurveto{\pgfqpoint{0.679955in}{1.287819in}}{\pgfqpoint{0.672055in}{1.284546in}}{\pgfqpoint{0.666231in}{1.278722in}}%
\pgfpathcurveto{\pgfqpoint{0.660407in}{1.272898in}}{\pgfqpoint{0.657135in}{1.264998in}}{\pgfqpoint{0.657135in}{1.256762in}}%
\pgfpathcurveto{\pgfqpoint{0.657135in}{1.248526in}}{\pgfqpoint{0.660407in}{1.240626in}}{\pgfqpoint{0.666231in}{1.234802in}}%
\pgfpathcurveto{\pgfqpoint{0.672055in}{1.228978in}}{\pgfqpoint{0.679955in}{1.225706in}}{\pgfqpoint{0.688192in}{1.225706in}}%
\pgfpathlineto{\pgfqpoint{0.688192in}{1.225706in}}%
\pgfpathclose%
\pgfusepath{stroke}%
\end{pgfscope}%
\begin{pgfscope}%
\pgfpathrectangle{\pgfqpoint{0.688192in}{0.670138in}}{\pgfqpoint{6.200000in}{4.620000in}}%
\pgfusepath{clip}%
\pgfsetbuttcap%
\pgfsetmiterjoin%
\definecolor{currentfill}{rgb}{0.839216,0.152941,0.156863}%
\pgfsetfillcolor{currentfill}%
\pgfsetfillopacity{0.200000}%
\pgfsetlinewidth{1.003750pt}%
\definecolor{currentstroke}{rgb}{0.839216,0.152941,0.156863}%
\pgfsetstrokecolor{currentstroke}%
\pgfsetstrokeopacity{0.200000}%
\pgfsetdash{}{0pt}%
\pgfpathmoveto{\pgfqpoint{0.688192in}{0.670138in}}%
\pgfpathlineto{\pgfqpoint{1.790122in}{0.670138in}}%
\pgfpathlineto{\pgfqpoint{1.790122in}{5.290138in}}%
\pgfpathlineto{\pgfqpoint{0.688192in}{5.290138in}}%
\pgfpathlineto{\pgfqpoint{0.688192in}{0.670138in}}%
\pgfpathclose%
\pgfusepath{stroke,fill}%
\end{pgfscope}%
\begin{pgfscope}%
\pgfsetbuttcap%
\pgfsetmiterjoin%
\definecolor{currentfill}{rgb}{0.839216,0.152941,0.156863}%
\pgfsetfillcolor{currentfill}%
\pgfsetfillopacity{0.200000}%
\pgfsetlinewidth{1.003750pt}%
\definecolor{currentstroke}{rgb}{0.839216,0.152941,0.156863}%
\pgfsetstrokecolor{currentstroke}%
\pgfsetstrokeopacity{0.200000}%
\pgfsetdash{}{0pt}%
\pgfpathrectangle{\pgfqpoint{0.688192in}{0.670138in}}{\pgfqpoint{6.200000in}{4.620000in}}%
\pgfusepath{clip}%
\pgfpathmoveto{\pgfqpoint{0.688192in}{0.670138in}}%
\pgfpathlineto{\pgfqpoint{1.790122in}{0.670138in}}%
\pgfpathlineto{\pgfqpoint{1.790122in}{5.290138in}}%
\pgfpathlineto{\pgfqpoint{0.688192in}{5.290138in}}%
\pgfpathlineto{\pgfqpoint{0.688192in}{0.670138in}}%
\pgfpathclose%
\pgfusepath{clip}%
\pgfsys@defobject{currentpattern}{\pgfqpoint{0in}{0in}}{\pgfqpoint{1in}{1in}}{%
\begin{pgfscope}%
\pgfpathrectangle{\pgfqpoint{0in}{0in}}{\pgfqpoint{1in}{1in}}%
\pgfusepath{clip}%
\pgfpathmoveto{\pgfqpoint{-0.500000in}{0.500000in}}%
\pgfpathlineto{\pgfqpoint{0.500000in}{1.500000in}}%
\pgfpathmoveto{\pgfqpoint{-0.333333in}{0.333333in}}%
\pgfpathlineto{\pgfqpoint{0.666667in}{1.333333in}}%
\pgfpathmoveto{\pgfqpoint{-0.166667in}{0.166667in}}%
\pgfpathlineto{\pgfqpoint{0.833333in}{1.166667in}}%
\pgfpathmoveto{\pgfqpoint{0.000000in}{0.000000in}}%
\pgfpathlineto{\pgfqpoint{1.000000in}{1.000000in}}%
\pgfpathmoveto{\pgfqpoint{0.166667in}{-0.166667in}}%
\pgfpathlineto{\pgfqpoint{1.166667in}{0.833333in}}%
\pgfpathmoveto{\pgfqpoint{0.333333in}{-0.333333in}}%
\pgfpathlineto{\pgfqpoint{1.333333in}{0.666667in}}%
\pgfpathmoveto{\pgfqpoint{0.500000in}{-0.500000in}}%
\pgfpathlineto{\pgfqpoint{1.500000in}{0.500000in}}%
\pgfusepath{stroke}%
\end{pgfscope}%
}%
\pgfsys@transformshift{0.688192in}{0.670138in}%
\pgfsys@useobject{currentpattern}{}%
\pgfsys@transformshift{1in}{0in}%
\pgfsys@useobject{currentpattern}{}%
\pgfsys@transformshift{1in}{0in}%
\pgfsys@transformshift{-2in}{0in}%
\pgfsys@transformshift{0in}{1in}%
\pgfsys@useobject{currentpattern}{}%
\pgfsys@transformshift{1in}{0in}%
\pgfsys@useobject{currentpattern}{}%
\pgfsys@transformshift{1in}{0in}%
\pgfsys@transformshift{-2in}{0in}%
\pgfsys@transformshift{0in}{1in}%
\pgfsys@useobject{currentpattern}{}%
\pgfsys@transformshift{1in}{0in}%
\pgfsys@useobject{currentpattern}{}%
\pgfsys@transformshift{1in}{0in}%
\pgfsys@transformshift{-2in}{0in}%
\pgfsys@transformshift{0in}{1in}%
\pgfsys@useobject{currentpattern}{}%
\pgfsys@transformshift{1in}{0in}%
\pgfsys@useobject{currentpattern}{}%
\pgfsys@transformshift{1in}{0in}%
\pgfsys@transformshift{-2in}{0in}%
\pgfsys@transformshift{0in}{1in}%
\pgfsys@useobject{currentpattern}{}%
\pgfsys@transformshift{1in}{0in}%
\pgfsys@useobject{currentpattern}{}%
\pgfsys@transformshift{1in}{0in}%
\pgfsys@transformshift{-2in}{0in}%
\pgfsys@transformshift{0in}{1in}%
\end{pgfscope}%
\begin{pgfscope}%
\pgfpathrectangle{\pgfqpoint{0.688192in}{0.670138in}}{\pgfqpoint{6.200000in}{4.620000in}}%
\pgfusepath{clip}%
\pgfsetrectcap%
\pgfsetroundjoin%
\pgfsetlinewidth{0.803000pt}%
\definecolor{currentstroke}{rgb}{0.690196,0.690196,0.690196}%
\pgfsetstrokecolor{currentstroke}%
\pgfsetdash{}{0pt}%
\pgfpathmoveto{\pgfqpoint{1.122474in}{0.670138in}}%
\pgfpathlineto{\pgfqpoint{1.122474in}{5.290138in}}%
\pgfusepath{stroke}%
\end{pgfscope}%
\begin{pgfscope}%
\pgfsetbuttcap%
\pgfsetroundjoin%
\definecolor{currentfill}{rgb}{0.000000,0.000000,0.000000}%
\pgfsetfillcolor{currentfill}%
\pgfsetlinewidth{0.803000pt}%
\definecolor{currentstroke}{rgb}{0.000000,0.000000,0.000000}%
\pgfsetstrokecolor{currentstroke}%
\pgfsetdash{}{0pt}%
\pgfsys@defobject{currentmarker}{\pgfqpoint{0.000000in}{-0.048611in}}{\pgfqpoint{0.000000in}{0.000000in}}{%
\pgfpathmoveto{\pgfqpoint{0.000000in}{0.000000in}}%
\pgfpathlineto{\pgfqpoint{0.000000in}{-0.048611in}}%
\pgfusepath{stroke,fill}%
}%
\begin{pgfscope}%
\pgfsys@transformshift{1.122474in}{0.670138in}%
\pgfsys@useobject{currentmarker}{}%
\end{pgfscope}%
\end{pgfscope}%
\begin{pgfscope}%
\definecolor{textcolor}{rgb}{0.000000,0.000000,0.000000}%
\pgfsetstrokecolor{textcolor}%
\pgfsetfillcolor{textcolor}%
\pgftext[x=1.122474in,y=0.572916in,,top]{\color{textcolor}{\rmfamily\fontsize{14.000000}{16.800000}\selectfont\catcode`\^=\active\def^{\ifmmode\sp\else\^{}\fi}\catcode`\%=\active\def%{\%}$\mathdefault{5500}$}}%
\end{pgfscope}%
\begin{pgfscope}%
\pgfpathrectangle{\pgfqpoint{0.688192in}{0.670138in}}{\pgfqpoint{6.200000in}{4.620000in}}%
\pgfusepath{clip}%
\pgfsetrectcap%
\pgfsetroundjoin%
\pgfsetlinewidth{0.803000pt}%
\definecolor{currentstroke}{rgb}{0.690196,0.690196,0.690196}%
\pgfsetstrokecolor{currentstroke}%
\pgfsetdash{}{0pt}%
\pgfpathmoveto{\pgfqpoint{2.163709in}{0.670138in}}%
\pgfpathlineto{\pgfqpoint{2.163709in}{5.290138in}}%
\pgfusepath{stroke}%
\end{pgfscope}%
\begin{pgfscope}%
\pgfsetbuttcap%
\pgfsetroundjoin%
\definecolor{currentfill}{rgb}{0.000000,0.000000,0.000000}%
\pgfsetfillcolor{currentfill}%
\pgfsetlinewidth{0.803000pt}%
\definecolor{currentstroke}{rgb}{0.000000,0.000000,0.000000}%
\pgfsetstrokecolor{currentstroke}%
\pgfsetdash{}{0pt}%
\pgfsys@defobject{currentmarker}{\pgfqpoint{0.000000in}{-0.048611in}}{\pgfqpoint{0.000000in}{0.000000in}}{%
\pgfpathmoveto{\pgfqpoint{0.000000in}{0.000000in}}%
\pgfpathlineto{\pgfqpoint{0.000000in}{-0.048611in}}%
\pgfusepath{stroke,fill}%
}%
\begin{pgfscope}%
\pgfsys@transformshift{2.163709in}{0.670138in}%
\pgfsys@useobject{currentmarker}{}%
\end{pgfscope}%
\end{pgfscope}%
\begin{pgfscope}%
\definecolor{textcolor}{rgb}{0.000000,0.000000,0.000000}%
\pgfsetstrokecolor{textcolor}%
\pgfsetfillcolor{textcolor}%
\pgftext[x=2.163709in,y=0.572916in,,top]{\color{textcolor}{\rmfamily\fontsize{14.000000}{16.800000}\selectfont\catcode`\^=\active\def^{\ifmmode\sp\else\^{}\fi}\catcode`\%=\active\def%{\%}$\mathdefault{6000}$}}%
\end{pgfscope}%
\begin{pgfscope}%
\pgfpathrectangle{\pgfqpoint{0.688192in}{0.670138in}}{\pgfqpoint{6.200000in}{4.620000in}}%
\pgfusepath{clip}%
\pgfsetrectcap%
\pgfsetroundjoin%
\pgfsetlinewidth{0.803000pt}%
\definecolor{currentstroke}{rgb}{0.690196,0.690196,0.690196}%
\pgfsetstrokecolor{currentstroke}%
\pgfsetdash{}{0pt}%
\pgfpathmoveto{\pgfqpoint{3.204944in}{0.670138in}}%
\pgfpathlineto{\pgfqpoint{3.204944in}{5.290138in}}%
\pgfusepath{stroke}%
\end{pgfscope}%
\begin{pgfscope}%
\pgfsetbuttcap%
\pgfsetroundjoin%
\definecolor{currentfill}{rgb}{0.000000,0.000000,0.000000}%
\pgfsetfillcolor{currentfill}%
\pgfsetlinewidth{0.803000pt}%
\definecolor{currentstroke}{rgb}{0.000000,0.000000,0.000000}%
\pgfsetstrokecolor{currentstroke}%
\pgfsetdash{}{0pt}%
\pgfsys@defobject{currentmarker}{\pgfqpoint{0.000000in}{-0.048611in}}{\pgfqpoint{0.000000in}{0.000000in}}{%
\pgfpathmoveto{\pgfqpoint{0.000000in}{0.000000in}}%
\pgfpathlineto{\pgfqpoint{0.000000in}{-0.048611in}}%
\pgfusepath{stroke,fill}%
}%
\begin{pgfscope}%
\pgfsys@transformshift{3.204944in}{0.670138in}%
\pgfsys@useobject{currentmarker}{}%
\end{pgfscope}%
\end{pgfscope}%
\begin{pgfscope}%
\definecolor{textcolor}{rgb}{0.000000,0.000000,0.000000}%
\pgfsetstrokecolor{textcolor}%
\pgfsetfillcolor{textcolor}%
\pgftext[x=3.204944in,y=0.572916in,,top]{\color{textcolor}{\rmfamily\fontsize{14.000000}{16.800000}\selectfont\catcode`\^=\active\def^{\ifmmode\sp\else\^{}\fi}\catcode`\%=\active\def%{\%}$\mathdefault{6500}$}}%
\end{pgfscope}%
\begin{pgfscope}%
\pgfpathrectangle{\pgfqpoint{0.688192in}{0.670138in}}{\pgfqpoint{6.200000in}{4.620000in}}%
\pgfusepath{clip}%
\pgfsetrectcap%
\pgfsetroundjoin%
\pgfsetlinewidth{0.803000pt}%
\definecolor{currentstroke}{rgb}{0.690196,0.690196,0.690196}%
\pgfsetstrokecolor{currentstroke}%
\pgfsetdash{}{0pt}%
\pgfpathmoveto{\pgfqpoint{4.246179in}{0.670138in}}%
\pgfpathlineto{\pgfqpoint{4.246179in}{5.290138in}}%
\pgfusepath{stroke}%
\end{pgfscope}%
\begin{pgfscope}%
\pgfsetbuttcap%
\pgfsetroundjoin%
\definecolor{currentfill}{rgb}{0.000000,0.000000,0.000000}%
\pgfsetfillcolor{currentfill}%
\pgfsetlinewidth{0.803000pt}%
\definecolor{currentstroke}{rgb}{0.000000,0.000000,0.000000}%
\pgfsetstrokecolor{currentstroke}%
\pgfsetdash{}{0pt}%
\pgfsys@defobject{currentmarker}{\pgfqpoint{0.000000in}{-0.048611in}}{\pgfqpoint{0.000000in}{0.000000in}}{%
\pgfpathmoveto{\pgfqpoint{0.000000in}{0.000000in}}%
\pgfpathlineto{\pgfqpoint{0.000000in}{-0.048611in}}%
\pgfusepath{stroke,fill}%
}%
\begin{pgfscope}%
\pgfsys@transformshift{4.246179in}{0.670138in}%
\pgfsys@useobject{currentmarker}{}%
\end{pgfscope}%
\end{pgfscope}%
\begin{pgfscope}%
\definecolor{textcolor}{rgb}{0.000000,0.000000,0.000000}%
\pgfsetstrokecolor{textcolor}%
\pgfsetfillcolor{textcolor}%
\pgftext[x=4.246179in,y=0.572916in,,top]{\color{textcolor}{\rmfamily\fontsize{14.000000}{16.800000}\selectfont\catcode`\^=\active\def^{\ifmmode\sp\else\^{}\fi}\catcode`\%=\active\def%{\%}$\mathdefault{7000}$}}%
\end{pgfscope}%
\begin{pgfscope}%
\pgfpathrectangle{\pgfqpoint{0.688192in}{0.670138in}}{\pgfqpoint{6.200000in}{4.620000in}}%
\pgfusepath{clip}%
\pgfsetrectcap%
\pgfsetroundjoin%
\pgfsetlinewidth{0.803000pt}%
\definecolor{currentstroke}{rgb}{0.690196,0.690196,0.690196}%
\pgfsetstrokecolor{currentstroke}%
\pgfsetdash{}{0pt}%
\pgfpathmoveto{\pgfqpoint{5.287414in}{0.670138in}}%
\pgfpathlineto{\pgfqpoint{5.287414in}{5.290138in}}%
\pgfusepath{stroke}%
\end{pgfscope}%
\begin{pgfscope}%
\pgfsetbuttcap%
\pgfsetroundjoin%
\definecolor{currentfill}{rgb}{0.000000,0.000000,0.000000}%
\pgfsetfillcolor{currentfill}%
\pgfsetlinewidth{0.803000pt}%
\definecolor{currentstroke}{rgb}{0.000000,0.000000,0.000000}%
\pgfsetstrokecolor{currentstroke}%
\pgfsetdash{}{0pt}%
\pgfsys@defobject{currentmarker}{\pgfqpoint{0.000000in}{-0.048611in}}{\pgfqpoint{0.000000in}{0.000000in}}{%
\pgfpathmoveto{\pgfqpoint{0.000000in}{0.000000in}}%
\pgfpathlineto{\pgfqpoint{0.000000in}{-0.048611in}}%
\pgfusepath{stroke,fill}%
}%
\begin{pgfscope}%
\pgfsys@transformshift{5.287414in}{0.670138in}%
\pgfsys@useobject{currentmarker}{}%
\end{pgfscope}%
\end{pgfscope}%
\begin{pgfscope}%
\definecolor{textcolor}{rgb}{0.000000,0.000000,0.000000}%
\pgfsetstrokecolor{textcolor}%
\pgfsetfillcolor{textcolor}%
\pgftext[x=5.287414in,y=0.572916in,,top]{\color{textcolor}{\rmfamily\fontsize{14.000000}{16.800000}\selectfont\catcode`\^=\active\def^{\ifmmode\sp\else\^{}\fi}\catcode`\%=\active\def%{\%}$\mathdefault{7500}$}}%
\end{pgfscope}%
\begin{pgfscope}%
\pgfpathrectangle{\pgfqpoint{0.688192in}{0.670138in}}{\pgfqpoint{6.200000in}{4.620000in}}%
\pgfusepath{clip}%
\pgfsetrectcap%
\pgfsetroundjoin%
\pgfsetlinewidth{0.803000pt}%
\definecolor{currentstroke}{rgb}{0.690196,0.690196,0.690196}%
\pgfsetstrokecolor{currentstroke}%
\pgfsetdash{}{0pt}%
\pgfpathmoveto{\pgfqpoint{6.328649in}{0.670138in}}%
\pgfpathlineto{\pgfqpoint{6.328649in}{5.290138in}}%
\pgfusepath{stroke}%
\end{pgfscope}%
\begin{pgfscope}%
\pgfsetbuttcap%
\pgfsetroundjoin%
\definecolor{currentfill}{rgb}{0.000000,0.000000,0.000000}%
\pgfsetfillcolor{currentfill}%
\pgfsetlinewidth{0.803000pt}%
\definecolor{currentstroke}{rgb}{0.000000,0.000000,0.000000}%
\pgfsetstrokecolor{currentstroke}%
\pgfsetdash{}{0pt}%
\pgfsys@defobject{currentmarker}{\pgfqpoint{0.000000in}{-0.048611in}}{\pgfqpoint{0.000000in}{0.000000in}}{%
\pgfpathmoveto{\pgfqpoint{0.000000in}{0.000000in}}%
\pgfpathlineto{\pgfqpoint{0.000000in}{-0.048611in}}%
\pgfusepath{stroke,fill}%
}%
\begin{pgfscope}%
\pgfsys@transformshift{6.328649in}{0.670138in}%
\pgfsys@useobject{currentmarker}{}%
\end{pgfscope}%
\end{pgfscope}%
\begin{pgfscope}%
\definecolor{textcolor}{rgb}{0.000000,0.000000,0.000000}%
\pgfsetstrokecolor{textcolor}%
\pgfsetfillcolor{textcolor}%
\pgftext[x=6.328649in,y=0.572916in,,top]{\color{textcolor}{\rmfamily\fontsize{14.000000}{16.800000}\selectfont\catcode`\^=\active\def^{\ifmmode\sp\else\^{}\fi}\catcode`\%=\active\def%{\%}$\mathdefault{8000}$}}%
\end{pgfscope}%
\begin{pgfscope}%
\definecolor{textcolor}{rgb}{0.000000,0.000000,0.000000}%
\pgfsetstrokecolor{textcolor}%
\pgfsetfillcolor{textcolor}%
\pgftext[x=3.788192in,y=0.339583in,,top]{\color{textcolor}{\rmfamily\fontsize{18.000000}{21.600000}\selectfont\catcode`\^=\active\def^{\ifmmode\sp\else\^{}\fi}\catcode`\%=\active\def%{\%}Total Cost (M\$)}}%
\end{pgfscope}%
\begin{pgfscope}%
\pgfpathrectangle{\pgfqpoint{0.688192in}{0.670138in}}{\pgfqpoint{6.200000in}{4.620000in}}%
\pgfusepath{clip}%
\pgfsetrectcap%
\pgfsetroundjoin%
\pgfsetlinewidth{0.803000pt}%
\definecolor{currentstroke}{rgb}{0.690196,0.690196,0.690196}%
\pgfsetstrokecolor{currentstroke}%
\pgfsetdash{}{0pt}%
\pgfpathmoveto{\pgfqpoint{0.688192in}{1.130833in}}%
\pgfpathlineto{\pgfqpoint{6.888192in}{1.130833in}}%
\pgfusepath{stroke}%
\end{pgfscope}%
\begin{pgfscope}%
\pgfsetbuttcap%
\pgfsetroundjoin%
\definecolor{currentfill}{rgb}{0.000000,0.000000,0.000000}%
\pgfsetfillcolor{currentfill}%
\pgfsetlinewidth{0.803000pt}%
\definecolor{currentstroke}{rgb}{0.000000,0.000000,0.000000}%
\pgfsetstrokecolor{currentstroke}%
\pgfsetdash{}{0pt}%
\pgfsys@defobject{currentmarker}{\pgfqpoint{-0.048611in}{0.000000in}}{\pgfqpoint{-0.000000in}{0.000000in}}{%
\pgfpathmoveto{\pgfqpoint{-0.000000in}{0.000000in}}%
\pgfpathlineto{\pgfqpoint{-0.048611in}{0.000000in}}%
\pgfusepath{stroke,fill}%
}%
\begin{pgfscope}%
\pgfsys@transformshift{0.688192in}{1.130833in}%
\pgfsys@useobject{currentmarker}{}%
\end{pgfscope}%
\end{pgfscope}%
\begin{pgfscope}%
\definecolor{textcolor}{rgb}{0.000000,0.000000,0.000000}%
\pgfsetstrokecolor{textcolor}%
\pgfsetfillcolor{textcolor}%
\pgftext[x=0.395138in, y=1.061389in, left, base]{\color{textcolor}{\rmfamily\fontsize{14.000000}{16.800000}\selectfont\catcode`\^=\active\def^{\ifmmode\sp\else\^{}\fi}\catcode`\%=\active\def%{\%}$\mathdefault{10}$}}%
\end{pgfscope}%
\begin{pgfscope}%
\pgfpathrectangle{\pgfqpoint{0.688192in}{0.670138in}}{\pgfqpoint{6.200000in}{4.620000in}}%
\pgfusepath{clip}%
\pgfsetrectcap%
\pgfsetroundjoin%
\pgfsetlinewidth{0.803000pt}%
\definecolor{currentstroke}{rgb}{0.690196,0.690196,0.690196}%
\pgfsetstrokecolor{currentstroke}%
\pgfsetdash{}{0pt}%
\pgfpathmoveto{\pgfqpoint{0.688192in}{1.724860in}}%
\pgfpathlineto{\pgfqpoint{6.888192in}{1.724860in}}%
\pgfusepath{stroke}%
\end{pgfscope}%
\begin{pgfscope}%
\pgfsetbuttcap%
\pgfsetroundjoin%
\definecolor{currentfill}{rgb}{0.000000,0.000000,0.000000}%
\pgfsetfillcolor{currentfill}%
\pgfsetlinewidth{0.803000pt}%
\definecolor{currentstroke}{rgb}{0.000000,0.000000,0.000000}%
\pgfsetstrokecolor{currentstroke}%
\pgfsetdash{}{0pt}%
\pgfsys@defobject{currentmarker}{\pgfqpoint{-0.048611in}{0.000000in}}{\pgfqpoint{-0.000000in}{0.000000in}}{%
\pgfpathmoveto{\pgfqpoint{-0.000000in}{0.000000in}}%
\pgfpathlineto{\pgfqpoint{-0.048611in}{0.000000in}}%
\pgfusepath{stroke,fill}%
}%
\begin{pgfscope}%
\pgfsys@transformshift{0.688192in}{1.724860in}%
\pgfsys@useobject{currentmarker}{}%
\end{pgfscope}%
\end{pgfscope}%
\begin{pgfscope}%
\definecolor{textcolor}{rgb}{0.000000,0.000000,0.000000}%
\pgfsetstrokecolor{textcolor}%
\pgfsetfillcolor{textcolor}%
\pgftext[x=0.395138in, y=1.655416in, left, base]{\color{textcolor}{\rmfamily\fontsize{14.000000}{16.800000}\selectfont\catcode`\^=\active\def^{\ifmmode\sp\else\^{}\fi}\catcode`\%=\active\def%{\%}$\mathdefault{20}$}}%
\end{pgfscope}%
\begin{pgfscope}%
\pgfpathrectangle{\pgfqpoint{0.688192in}{0.670138in}}{\pgfqpoint{6.200000in}{4.620000in}}%
\pgfusepath{clip}%
\pgfsetrectcap%
\pgfsetroundjoin%
\pgfsetlinewidth{0.803000pt}%
\definecolor{currentstroke}{rgb}{0.690196,0.690196,0.690196}%
\pgfsetstrokecolor{currentstroke}%
\pgfsetdash{}{0pt}%
\pgfpathmoveto{\pgfqpoint{0.688192in}{2.318888in}}%
\pgfpathlineto{\pgfqpoint{6.888192in}{2.318888in}}%
\pgfusepath{stroke}%
\end{pgfscope}%
\begin{pgfscope}%
\pgfsetbuttcap%
\pgfsetroundjoin%
\definecolor{currentfill}{rgb}{0.000000,0.000000,0.000000}%
\pgfsetfillcolor{currentfill}%
\pgfsetlinewidth{0.803000pt}%
\definecolor{currentstroke}{rgb}{0.000000,0.000000,0.000000}%
\pgfsetstrokecolor{currentstroke}%
\pgfsetdash{}{0pt}%
\pgfsys@defobject{currentmarker}{\pgfqpoint{-0.048611in}{0.000000in}}{\pgfqpoint{-0.000000in}{0.000000in}}{%
\pgfpathmoveto{\pgfqpoint{-0.000000in}{0.000000in}}%
\pgfpathlineto{\pgfqpoint{-0.048611in}{0.000000in}}%
\pgfusepath{stroke,fill}%
}%
\begin{pgfscope}%
\pgfsys@transformshift{0.688192in}{2.318888in}%
\pgfsys@useobject{currentmarker}{}%
\end{pgfscope}%
\end{pgfscope}%
\begin{pgfscope}%
\definecolor{textcolor}{rgb}{0.000000,0.000000,0.000000}%
\pgfsetstrokecolor{textcolor}%
\pgfsetfillcolor{textcolor}%
\pgftext[x=0.395138in, y=2.249444in, left, base]{\color{textcolor}{\rmfamily\fontsize{14.000000}{16.800000}\selectfont\catcode`\^=\active\def^{\ifmmode\sp\else\^{}\fi}\catcode`\%=\active\def%{\%}$\mathdefault{30}$}}%
\end{pgfscope}%
\begin{pgfscope}%
\pgfpathrectangle{\pgfqpoint{0.688192in}{0.670138in}}{\pgfqpoint{6.200000in}{4.620000in}}%
\pgfusepath{clip}%
\pgfsetrectcap%
\pgfsetroundjoin%
\pgfsetlinewidth{0.803000pt}%
\definecolor{currentstroke}{rgb}{0.690196,0.690196,0.690196}%
\pgfsetstrokecolor{currentstroke}%
\pgfsetdash{}{0pt}%
\pgfpathmoveto{\pgfqpoint{0.688192in}{2.912915in}}%
\pgfpathlineto{\pgfqpoint{6.888192in}{2.912915in}}%
\pgfusepath{stroke}%
\end{pgfscope}%
\begin{pgfscope}%
\pgfsetbuttcap%
\pgfsetroundjoin%
\definecolor{currentfill}{rgb}{0.000000,0.000000,0.000000}%
\pgfsetfillcolor{currentfill}%
\pgfsetlinewidth{0.803000pt}%
\definecolor{currentstroke}{rgb}{0.000000,0.000000,0.000000}%
\pgfsetstrokecolor{currentstroke}%
\pgfsetdash{}{0pt}%
\pgfsys@defobject{currentmarker}{\pgfqpoint{-0.048611in}{0.000000in}}{\pgfqpoint{-0.000000in}{0.000000in}}{%
\pgfpathmoveto{\pgfqpoint{-0.000000in}{0.000000in}}%
\pgfpathlineto{\pgfqpoint{-0.048611in}{0.000000in}}%
\pgfusepath{stroke,fill}%
}%
\begin{pgfscope}%
\pgfsys@transformshift{0.688192in}{2.912915in}%
\pgfsys@useobject{currentmarker}{}%
\end{pgfscope}%
\end{pgfscope}%
\begin{pgfscope}%
\definecolor{textcolor}{rgb}{0.000000,0.000000,0.000000}%
\pgfsetstrokecolor{textcolor}%
\pgfsetfillcolor{textcolor}%
\pgftext[x=0.395138in, y=2.843471in, left, base]{\color{textcolor}{\rmfamily\fontsize{14.000000}{16.800000}\selectfont\catcode`\^=\active\def^{\ifmmode\sp\else\^{}\fi}\catcode`\%=\active\def%{\%}$\mathdefault{40}$}}%
\end{pgfscope}%
\begin{pgfscope}%
\pgfpathrectangle{\pgfqpoint{0.688192in}{0.670138in}}{\pgfqpoint{6.200000in}{4.620000in}}%
\pgfusepath{clip}%
\pgfsetrectcap%
\pgfsetroundjoin%
\pgfsetlinewidth{0.803000pt}%
\definecolor{currentstroke}{rgb}{0.690196,0.690196,0.690196}%
\pgfsetstrokecolor{currentstroke}%
\pgfsetdash{}{0pt}%
\pgfpathmoveto{\pgfqpoint{0.688192in}{3.506943in}}%
\pgfpathlineto{\pgfqpoint{6.888192in}{3.506943in}}%
\pgfusepath{stroke}%
\end{pgfscope}%
\begin{pgfscope}%
\pgfsetbuttcap%
\pgfsetroundjoin%
\definecolor{currentfill}{rgb}{0.000000,0.000000,0.000000}%
\pgfsetfillcolor{currentfill}%
\pgfsetlinewidth{0.803000pt}%
\definecolor{currentstroke}{rgb}{0.000000,0.000000,0.000000}%
\pgfsetstrokecolor{currentstroke}%
\pgfsetdash{}{0pt}%
\pgfsys@defobject{currentmarker}{\pgfqpoint{-0.048611in}{0.000000in}}{\pgfqpoint{-0.000000in}{0.000000in}}{%
\pgfpathmoveto{\pgfqpoint{-0.000000in}{0.000000in}}%
\pgfpathlineto{\pgfqpoint{-0.048611in}{0.000000in}}%
\pgfusepath{stroke,fill}%
}%
\begin{pgfscope}%
\pgfsys@transformshift{0.688192in}{3.506943in}%
\pgfsys@useobject{currentmarker}{}%
\end{pgfscope}%
\end{pgfscope}%
\begin{pgfscope}%
\definecolor{textcolor}{rgb}{0.000000,0.000000,0.000000}%
\pgfsetstrokecolor{textcolor}%
\pgfsetfillcolor{textcolor}%
\pgftext[x=0.395138in, y=3.437499in, left, base]{\color{textcolor}{\rmfamily\fontsize{14.000000}{16.800000}\selectfont\catcode`\^=\active\def^{\ifmmode\sp\else\^{}\fi}\catcode`\%=\active\def%{\%}$\mathdefault{50}$}}%
\end{pgfscope}%
\begin{pgfscope}%
\pgfpathrectangle{\pgfqpoint{0.688192in}{0.670138in}}{\pgfqpoint{6.200000in}{4.620000in}}%
\pgfusepath{clip}%
\pgfsetrectcap%
\pgfsetroundjoin%
\pgfsetlinewidth{0.803000pt}%
\definecolor{currentstroke}{rgb}{0.690196,0.690196,0.690196}%
\pgfsetstrokecolor{currentstroke}%
\pgfsetdash{}{0pt}%
\pgfpathmoveto{\pgfqpoint{0.688192in}{4.100970in}}%
\pgfpathlineto{\pgfqpoint{6.888192in}{4.100970in}}%
\pgfusepath{stroke}%
\end{pgfscope}%
\begin{pgfscope}%
\pgfsetbuttcap%
\pgfsetroundjoin%
\definecolor{currentfill}{rgb}{0.000000,0.000000,0.000000}%
\pgfsetfillcolor{currentfill}%
\pgfsetlinewidth{0.803000pt}%
\definecolor{currentstroke}{rgb}{0.000000,0.000000,0.000000}%
\pgfsetstrokecolor{currentstroke}%
\pgfsetdash{}{0pt}%
\pgfsys@defobject{currentmarker}{\pgfqpoint{-0.048611in}{0.000000in}}{\pgfqpoint{-0.000000in}{0.000000in}}{%
\pgfpathmoveto{\pgfqpoint{-0.000000in}{0.000000in}}%
\pgfpathlineto{\pgfqpoint{-0.048611in}{0.000000in}}%
\pgfusepath{stroke,fill}%
}%
\begin{pgfscope}%
\pgfsys@transformshift{0.688192in}{4.100970in}%
\pgfsys@useobject{currentmarker}{}%
\end{pgfscope}%
\end{pgfscope}%
\begin{pgfscope}%
\definecolor{textcolor}{rgb}{0.000000,0.000000,0.000000}%
\pgfsetstrokecolor{textcolor}%
\pgfsetfillcolor{textcolor}%
\pgftext[x=0.395138in, y=4.031526in, left, base]{\color{textcolor}{\rmfamily\fontsize{14.000000}{16.800000}\selectfont\catcode`\^=\active\def^{\ifmmode\sp\else\^{}\fi}\catcode`\%=\active\def%{\%}$\mathdefault{60}$}}%
\end{pgfscope}%
\begin{pgfscope}%
\pgfpathrectangle{\pgfqpoint{0.688192in}{0.670138in}}{\pgfqpoint{6.200000in}{4.620000in}}%
\pgfusepath{clip}%
\pgfsetrectcap%
\pgfsetroundjoin%
\pgfsetlinewidth{0.803000pt}%
\definecolor{currentstroke}{rgb}{0.690196,0.690196,0.690196}%
\pgfsetstrokecolor{currentstroke}%
\pgfsetdash{}{0pt}%
\pgfpathmoveto{\pgfqpoint{0.688192in}{4.694998in}}%
\pgfpathlineto{\pgfqpoint{6.888192in}{4.694998in}}%
\pgfusepath{stroke}%
\end{pgfscope}%
\begin{pgfscope}%
\pgfsetbuttcap%
\pgfsetroundjoin%
\definecolor{currentfill}{rgb}{0.000000,0.000000,0.000000}%
\pgfsetfillcolor{currentfill}%
\pgfsetlinewidth{0.803000pt}%
\definecolor{currentstroke}{rgb}{0.000000,0.000000,0.000000}%
\pgfsetstrokecolor{currentstroke}%
\pgfsetdash{}{0pt}%
\pgfsys@defobject{currentmarker}{\pgfqpoint{-0.048611in}{0.000000in}}{\pgfqpoint{-0.000000in}{0.000000in}}{%
\pgfpathmoveto{\pgfqpoint{-0.000000in}{0.000000in}}%
\pgfpathlineto{\pgfqpoint{-0.048611in}{0.000000in}}%
\pgfusepath{stroke,fill}%
}%
\begin{pgfscope}%
\pgfsys@transformshift{0.688192in}{4.694998in}%
\pgfsys@useobject{currentmarker}{}%
\end{pgfscope}%
\end{pgfscope}%
\begin{pgfscope}%
\definecolor{textcolor}{rgb}{0.000000,0.000000,0.000000}%
\pgfsetstrokecolor{textcolor}%
\pgfsetfillcolor{textcolor}%
\pgftext[x=0.395138in, y=4.625553in, left, base]{\color{textcolor}{\rmfamily\fontsize{14.000000}{16.800000}\selectfont\catcode`\^=\active\def^{\ifmmode\sp\else\^{}\fi}\catcode`\%=\active\def%{\%}$\mathdefault{70}$}}%
\end{pgfscope}%
\begin{pgfscope}%
\pgfpathrectangle{\pgfqpoint{0.688192in}{0.670138in}}{\pgfqpoint{6.200000in}{4.620000in}}%
\pgfusepath{clip}%
\pgfsetrectcap%
\pgfsetroundjoin%
\pgfsetlinewidth{0.803000pt}%
\definecolor{currentstroke}{rgb}{0.690196,0.690196,0.690196}%
\pgfsetstrokecolor{currentstroke}%
\pgfsetdash{}{0pt}%
\pgfpathmoveto{\pgfqpoint{0.688192in}{5.289025in}}%
\pgfpathlineto{\pgfqpoint{6.888192in}{5.289025in}}%
\pgfusepath{stroke}%
\end{pgfscope}%
\begin{pgfscope}%
\pgfsetbuttcap%
\pgfsetroundjoin%
\definecolor{currentfill}{rgb}{0.000000,0.000000,0.000000}%
\pgfsetfillcolor{currentfill}%
\pgfsetlinewidth{0.803000pt}%
\definecolor{currentstroke}{rgb}{0.000000,0.000000,0.000000}%
\pgfsetstrokecolor{currentstroke}%
\pgfsetdash{}{0pt}%
\pgfsys@defobject{currentmarker}{\pgfqpoint{-0.048611in}{0.000000in}}{\pgfqpoint{-0.000000in}{0.000000in}}{%
\pgfpathmoveto{\pgfqpoint{-0.000000in}{0.000000in}}%
\pgfpathlineto{\pgfqpoint{-0.048611in}{0.000000in}}%
\pgfusepath{stroke,fill}%
}%
\begin{pgfscope}%
\pgfsys@transformshift{0.688192in}{5.289025in}%
\pgfsys@useobject{currentmarker}{}%
\end{pgfscope}%
\end{pgfscope}%
\begin{pgfscope}%
\definecolor{textcolor}{rgb}{0.000000,0.000000,0.000000}%
\pgfsetstrokecolor{textcolor}%
\pgfsetfillcolor{textcolor}%
\pgftext[x=0.395138in, y=5.219581in, left, base]{\color{textcolor}{\rmfamily\fontsize{14.000000}{16.800000}\selectfont\catcode`\^=\active\def^{\ifmmode\sp\else\^{}\fi}\catcode`\%=\active\def%{\%}$\mathdefault{80}$}}%
\end{pgfscope}%
\begin{pgfscope}%
\definecolor{textcolor}{rgb}{0.000000,0.000000,0.000000}%
\pgfsetstrokecolor{textcolor}%
\pgfsetfillcolor{textcolor}%
\pgftext[x=0.339583in,y=2.980138in,,bottom,rotate=90.000000]{\color{textcolor}{\rmfamily\fontsize{18.000000}{21.600000}\selectfont\catcode`\^=\active\def^{\ifmmode\sp\else\^{}\fi}\catcode`\%=\active\def%{\%}CO2 emissions (MT CO2)}}%
\end{pgfscope}%
\begin{pgfscope}%
\pgfpathrectangle{\pgfqpoint{0.688192in}{0.670138in}}{\pgfqpoint{6.200000in}{4.620000in}}%
\pgfusepath{clip}%
\pgfsetrectcap%
\pgfsetroundjoin%
\pgfsetlinewidth{1.505625pt}%
\definecolor{currentstroke}{rgb}{0.000000,0.000000,1.000000}%
\pgfsetstrokecolor{currentstroke}%
\pgfsetdash{}{0pt}%
\pgfpathmoveto{\pgfqpoint{0.741425in}{1.377543in}}%
\pgfpathlineto{\pgfqpoint{0.758703in}{0.955032in}}%
\pgfpathlineto{\pgfqpoint{0.768198in}{0.875033in}}%
\pgfpathlineto{\pgfqpoint{0.774746in}{0.828781in}}%
\pgfpathlineto{\pgfqpoint{0.778243in}{0.822495in}}%
\pgfpathlineto{\pgfqpoint{0.782159in}{0.789611in}}%
\pgfpathlineto{\pgfqpoint{0.786516in}{0.779881in}}%
\pgfpathlineto{\pgfqpoint{0.792538in}{0.779145in}}%
\pgfpathlineto{\pgfqpoint{0.794668in}{0.758056in}}%
\pgfpathlineto{\pgfqpoint{0.799837in}{0.752930in}}%
\pgfpathlineto{\pgfqpoint{0.809370in}{0.751978in}}%
\pgfpathlineto{\pgfqpoint{0.812629in}{0.743975in}}%
\pgfpathlineto{\pgfqpoint{0.815972in}{0.742575in}}%
\pgfpathlineto{\pgfqpoint{0.822987in}{0.738477in}}%
\pgfpathlineto{\pgfqpoint{0.828825in}{0.734937in}}%
\pgfpathlineto{\pgfqpoint{0.829214in}{0.733319in}}%
\pgfpathlineto{\pgfqpoint{0.833044in}{0.730858in}}%
\pgfpathlineto{\pgfqpoint{0.848459in}{0.726329in}}%
\pgfpathlineto{\pgfqpoint{0.864854in}{0.720019in}}%
\pgfpathlineto{\pgfqpoint{0.887104in}{0.715517in}}%
\pgfpathlineto{\pgfqpoint{0.907479in}{0.714004in}}%
\pgfpathlineto{\pgfqpoint{0.908310in}{0.712008in}}%
\pgfpathlineto{\pgfqpoint{0.909513in}{0.708525in}}%
\pgfpathlineto{\pgfqpoint{0.912740in}{0.707284in}}%
\pgfpathlineto{\pgfqpoint{0.920440in}{0.706723in}}%
\pgfpathlineto{\pgfqpoint{0.925670in}{0.705238in}}%
\pgfpathlineto{\pgfqpoint{0.948903in}{0.702931in}}%
\pgfpathlineto{\pgfqpoint{0.951945in}{0.701707in}}%
\pgfpathlineto{\pgfqpoint{0.952035in}{0.700391in}}%
\pgfpathlineto{\pgfqpoint{0.957029in}{0.700173in}}%
\pgfpathlineto{\pgfqpoint{0.968828in}{0.697963in}}%
\pgfpathlineto{\pgfqpoint{0.974412in}{0.697738in}}%
\pgfpathlineto{\pgfqpoint{0.975275in}{0.696914in}}%
\pgfpathlineto{\pgfqpoint{1.021767in}{0.694795in}}%
\pgfpathlineto{\pgfqpoint{1.025407in}{0.690657in}}%
\pgfpathlineto{\pgfqpoint{1.027475in}{0.690338in}}%
\pgfpathlineto{\pgfqpoint{1.034837in}{0.689784in}}%
\pgfpathlineto{\pgfqpoint{1.049406in}{0.687676in}}%
\pgfpathlineto{\pgfqpoint{1.054714in}{0.687138in}}%
\pgfpathlineto{\pgfqpoint{1.059617in}{0.686467in}}%
\pgfpathlineto{\pgfqpoint{1.072141in}{0.685078in}}%
\pgfpathlineto{\pgfqpoint{1.092208in}{0.684413in}}%
\pgfpathlineto{\pgfqpoint{1.115209in}{0.684111in}}%
\pgfpathlineto{\pgfqpoint{1.131834in}{0.684071in}}%
\pgfpathlineto{\pgfqpoint{1.152628in}{0.684059in}}%
\pgfpathlineto{\pgfqpoint{1.251312in}{0.683263in}}%
\pgfpathlineto{\pgfqpoint{1.277476in}{0.683159in}}%
\pgfpathlineto{\pgfqpoint{1.314870in}{0.682855in}}%
\pgfpathlineto{\pgfqpoint{1.369253in}{0.682756in}}%
\pgfpathlineto{\pgfqpoint{1.398687in}{0.682288in}}%
\pgfpathlineto{\pgfqpoint{1.467852in}{0.682134in}}%
\pgfpathlineto{\pgfqpoint{1.557026in}{0.681680in}}%
\pgfpathlineto{\pgfqpoint{1.627242in}{0.680913in}}%
\pgfpathlineto{\pgfqpoint{1.737728in}{0.680478in}}%
\pgfpathlineto{\pgfqpoint{1.887036in}{0.679610in}}%
\pgfpathlineto{\pgfqpoint{2.037481in}{0.678826in}}%
\pgfpathlineto{\pgfqpoint{2.258348in}{0.677741in}}%
\pgfpathlineto{\pgfqpoint{2.626338in}{0.676361in}}%
\pgfpathlineto{\pgfqpoint{3.263784in}{0.674352in}}%
\pgfpathlineto{\pgfqpoint{5.322800in}{0.670138in}}%
\pgfusepath{stroke}%
\end{pgfscope}%
\begin{pgfscope}%
\pgfpathrectangle{\pgfqpoint{0.688192in}{0.670138in}}{\pgfqpoint{6.200000in}{4.620000in}}%
\pgfusepath{clip}%
\pgfsetbuttcap%
\pgfsetroundjoin%
\definecolor{currentfill}{rgb}{0.000000,0.000000,1.000000}%
\pgfsetfillcolor{currentfill}%
\pgfsetlinewidth{1.003750pt}%
\definecolor{currentstroke}{rgb}{0.000000,0.000000,1.000000}%
\pgfsetstrokecolor{currentstroke}%
\pgfsetdash{}{0pt}%
\pgfsys@defobject{currentmarker}{\pgfqpoint{-0.006944in}{-0.006944in}}{\pgfqpoint{0.006944in}{0.006944in}}{%
\pgfpathmoveto{\pgfqpoint{0.000000in}{-0.006944in}}%
\pgfpathcurveto{\pgfqpoint{0.001842in}{-0.006944in}}{\pgfqpoint{0.003608in}{-0.006213in}}{\pgfqpoint{0.004910in}{-0.004910in}}%
\pgfpathcurveto{\pgfqpoint{0.006213in}{-0.003608in}}{\pgfqpoint{0.006944in}{-0.001842in}}{\pgfqpoint{0.006944in}{0.000000in}}%
\pgfpathcurveto{\pgfqpoint{0.006944in}{0.001842in}}{\pgfqpoint{0.006213in}{0.003608in}}{\pgfqpoint{0.004910in}{0.004910in}}%
\pgfpathcurveto{\pgfqpoint{0.003608in}{0.006213in}}{\pgfqpoint{0.001842in}{0.006944in}}{\pgfqpoint{0.000000in}{0.006944in}}%
\pgfpathcurveto{\pgfqpoint{-0.001842in}{0.006944in}}{\pgfqpoint{-0.003608in}{0.006213in}}{\pgfqpoint{-0.004910in}{0.004910in}}%
\pgfpathcurveto{\pgfqpoint{-0.006213in}{0.003608in}}{\pgfqpoint{-0.006944in}{0.001842in}}{\pgfqpoint{-0.006944in}{0.000000in}}%
\pgfpathcurveto{\pgfqpoint{-0.006944in}{-0.001842in}}{\pgfqpoint{-0.006213in}{-0.003608in}}{\pgfqpoint{-0.004910in}{-0.004910in}}%
\pgfpathcurveto{\pgfqpoint{-0.003608in}{-0.006213in}}{\pgfqpoint{-0.001842in}{-0.006944in}}{\pgfqpoint{0.000000in}{-0.006944in}}%
\pgfpathlineto{\pgfqpoint{0.000000in}{-0.006944in}}%
\pgfpathclose%
\pgfusepath{stroke,fill}%
}%
\begin{pgfscope}%
\pgfsys@transformshift{0.741425in}{1.377543in}%
\pgfsys@useobject{currentmarker}{}%
\end{pgfscope}%
\begin{pgfscope}%
\pgfsys@transformshift{0.758703in}{0.955032in}%
\pgfsys@useobject{currentmarker}{}%
\end{pgfscope}%
\begin{pgfscope}%
\pgfsys@transformshift{0.768198in}{0.875033in}%
\pgfsys@useobject{currentmarker}{}%
\end{pgfscope}%
\begin{pgfscope}%
\pgfsys@transformshift{0.774746in}{0.828781in}%
\pgfsys@useobject{currentmarker}{}%
\end{pgfscope}%
\begin{pgfscope}%
\pgfsys@transformshift{0.778243in}{0.822495in}%
\pgfsys@useobject{currentmarker}{}%
\end{pgfscope}%
\begin{pgfscope}%
\pgfsys@transformshift{0.782159in}{0.789611in}%
\pgfsys@useobject{currentmarker}{}%
\end{pgfscope}%
\begin{pgfscope}%
\pgfsys@transformshift{0.786516in}{0.779881in}%
\pgfsys@useobject{currentmarker}{}%
\end{pgfscope}%
\begin{pgfscope}%
\pgfsys@transformshift{0.792538in}{0.779145in}%
\pgfsys@useobject{currentmarker}{}%
\end{pgfscope}%
\begin{pgfscope}%
\pgfsys@transformshift{0.794668in}{0.758056in}%
\pgfsys@useobject{currentmarker}{}%
\end{pgfscope}%
\begin{pgfscope}%
\pgfsys@transformshift{0.799837in}{0.752930in}%
\pgfsys@useobject{currentmarker}{}%
\end{pgfscope}%
\begin{pgfscope}%
\pgfsys@transformshift{0.809370in}{0.751978in}%
\pgfsys@useobject{currentmarker}{}%
\end{pgfscope}%
\begin{pgfscope}%
\pgfsys@transformshift{0.812629in}{0.743975in}%
\pgfsys@useobject{currentmarker}{}%
\end{pgfscope}%
\begin{pgfscope}%
\pgfsys@transformshift{0.815972in}{0.742575in}%
\pgfsys@useobject{currentmarker}{}%
\end{pgfscope}%
\begin{pgfscope}%
\pgfsys@transformshift{0.822987in}{0.738477in}%
\pgfsys@useobject{currentmarker}{}%
\end{pgfscope}%
\begin{pgfscope}%
\pgfsys@transformshift{0.828825in}{0.734937in}%
\pgfsys@useobject{currentmarker}{}%
\end{pgfscope}%
\begin{pgfscope}%
\pgfsys@transformshift{0.829214in}{0.733319in}%
\pgfsys@useobject{currentmarker}{}%
\end{pgfscope}%
\begin{pgfscope}%
\pgfsys@transformshift{0.833044in}{0.730858in}%
\pgfsys@useobject{currentmarker}{}%
\end{pgfscope}%
\begin{pgfscope}%
\pgfsys@transformshift{0.848459in}{0.726329in}%
\pgfsys@useobject{currentmarker}{}%
\end{pgfscope}%
\begin{pgfscope}%
\pgfsys@transformshift{0.864854in}{0.720019in}%
\pgfsys@useobject{currentmarker}{}%
\end{pgfscope}%
\begin{pgfscope}%
\pgfsys@transformshift{0.887104in}{0.715517in}%
\pgfsys@useobject{currentmarker}{}%
\end{pgfscope}%
\begin{pgfscope}%
\pgfsys@transformshift{0.907479in}{0.714004in}%
\pgfsys@useobject{currentmarker}{}%
\end{pgfscope}%
\begin{pgfscope}%
\pgfsys@transformshift{0.908310in}{0.712008in}%
\pgfsys@useobject{currentmarker}{}%
\end{pgfscope}%
\begin{pgfscope}%
\pgfsys@transformshift{0.909513in}{0.708525in}%
\pgfsys@useobject{currentmarker}{}%
\end{pgfscope}%
\begin{pgfscope}%
\pgfsys@transformshift{0.912740in}{0.707284in}%
\pgfsys@useobject{currentmarker}{}%
\end{pgfscope}%
\begin{pgfscope}%
\pgfsys@transformshift{0.920440in}{0.706723in}%
\pgfsys@useobject{currentmarker}{}%
\end{pgfscope}%
\begin{pgfscope}%
\pgfsys@transformshift{0.925670in}{0.705238in}%
\pgfsys@useobject{currentmarker}{}%
\end{pgfscope}%
\begin{pgfscope}%
\pgfsys@transformshift{0.948903in}{0.702931in}%
\pgfsys@useobject{currentmarker}{}%
\end{pgfscope}%
\begin{pgfscope}%
\pgfsys@transformshift{0.951945in}{0.701707in}%
\pgfsys@useobject{currentmarker}{}%
\end{pgfscope}%
\begin{pgfscope}%
\pgfsys@transformshift{0.952035in}{0.700391in}%
\pgfsys@useobject{currentmarker}{}%
\end{pgfscope}%
\begin{pgfscope}%
\pgfsys@transformshift{0.957029in}{0.700173in}%
\pgfsys@useobject{currentmarker}{}%
\end{pgfscope}%
\begin{pgfscope}%
\pgfsys@transformshift{0.968828in}{0.697963in}%
\pgfsys@useobject{currentmarker}{}%
\end{pgfscope}%
\begin{pgfscope}%
\pgfsys@transformshift{0.974412in}{0.697738in}%
\pgfsys@useobject{currentmarker}{}%
\end{pgfscope}%
\begin{pgfscope}%
\pgfsys@transformshift{0.975275in}{0.696914in}%
\pgfsys@useobject{currentmarker}{}%
\end{pgfscope}%
\begin{pgfscope}%
\pgfsys@transformshift{1.021767in}{0.694795in}%
\pgfsys@useobject{currentmarker}{}%
\end{pgfscope}%
\begin{pgfscope}%
\pgfsys@transformshift{1.025407in}{0.690657in}%
\pgfsys@useobject{currentmarker}{}%
\end{pgfscope}%
\begin{pgfscope}%
\pgfsys@transformshift{1.027475in}{0.690338in}%
\pgfsys@useobject{currentmarker}{}%
\end{pgfscope}%
\begin{pgfscope}%
\pgfsys@transformshift{1.034837in}{0.689784in}%
\pgfsys@useobject{currentmarker}{}%
\end{pgfscope}%
\begin{pgfscope}%
\pgfsys@transformshift{1.049406in}{0.687676in}%
\pgfsys@useobject{currentmarker}{}%
\end{pgfscope}%
\begin{pgfscope}%
\pgfsys@transformshift{1.054714in}{0.687138in}%
\pgfsys@useobject{currentmarker}{}%
\end{pgfscope}%
\begin{pgfscope}%
\pgfsys@transformshift{1.059617in}{0.686467in}%
\pgfsys@useobject{currentmarker}{}%
\end{pgfscope}%
\begin{pgfscope}%
\pgfsys@transformshift{1.072141in}{0.685078in}%
\pgfsys@useobject{currentmarker}{}%
\end{pgfscope}%
\begin{pgfscope}%
\pgfsys@transformshift{1.092208in}{0.684413in}%
\pgfsys@useobject{currentmarker}{}%
\end{pgfscope}%
\begin{pgfscope}%
\pgfsys@transformshift{1.115209in}{0.684111in}%
\pgfsys@useobject{currentmarker}{}%
\end{pgfscope}%
\begin{pgfscope}%
\pgfsys@transformshift{1.131834in}{0.684071in}%
\pgfsys@useobject{currentmarker}{}%
\end{pgfscope}%
\begin{pgfscope}%
\pgfsys@transformshift{1.152628in}{0.684059in}%
\pgfsys@useobject{currentmarker}{}%
\end{pgfscope}%
\begin{pgfscope}%
\pgfsys@transformshift{1.251312in}{0.683263in}%
\pgfsys@useobject{currentmarker}{}%
\end{pgfscope}%
\begin{pgfscope}%
\pgfsys@transformshift{1.277476in}{0.683159in}%
\pgfsys@useobject{currentmarker}{}%
\end{pgfscope}%
\begin{pgfscope}%
\pgfsys@transformshift{1.314870in}{0.682855in}%
\pgfsys@useobject{currentmarker}{}%
\end{pgfscope}%
\begin{pgfscope}%
\pgfsys@transformshift{1.369253in}{0.682756in}%
\pgfsys@useobject{currentmarker}{}%
\end{pgfscope}%
\begin{pgfscope}%
\pgfsys@transformshift{1.398687in}{0.682288in}%
\pgfsys@useobject{currentmarker}{}%
\end{pgfscope}%
\begin{pgfscope}%
\pgfsys@transformshift{1.467852in}{0.682134in}%
\pgfsys@useobject{currentmarker}{}%
\end{pgfscope}%
\begin{pgfscope}%
\pgfsys@transformshift{1.557026in}{0.681680in}%
\pgfsys@useobject{currentmarker}{}%
\end{pgfscope}%
\begin{pgfscope}%
\pgfsys@transformshift{1.627242in}{0.680913in}%
\pgfsys@useobject{currentmarker}{}%
\end{pgfscope}%
\begin{pgfscope}%
\pgfsys@transformshift{1.737728in}{0.680478in}%
\pgfsys@useobject{currentmarker}{}%
\end{pgfscope}%
\begin{pgfscope}%
\pgfsys@transformshift{1.887036in}{0.679610in}%
\pgfsys@useobject{currentmarker}{}%
\end{pgfscope}%
\begin{pgfscope}%
\pgfsys@transformshift{2.037481in}{0.678826in}%
\pgfsys@useobject{currentmarker}{}%
\end{pgfscope}%
\begin{pgfscope}%
\pgfsys@transformshift{2.258348in}{0.677741in}%
\pgfsys@useobject{currentmarker}{}%
\end{pgfscope}%
\begin{pgfscope}%
\pgfsys@transformshift{2.626338in}{0.676361in}%
\pgfsys@useobject{currentmarker}{}%
\end{pgfscope}%
\begin{pgfscope}%
\pgfsys@transformshift{3.263784in}{0.674352in}%
\pgfsys@useobject{currentmarker}{}%
\end{pgfscope}%
\begin{pgfscope}%
\pgfsys@transformshift{5.322800in}{0.670138in}%
\pgfsys@useobject{currentmarker}{}%
\end{pgfscope}%
\end{pgfscope}%
\begin{pgfscope}%
\pgfpathrectangle{\pgfqpoint{0.688192in}{0.670138in}}{\pgfqpoint{6.200000in}{4.620000in}}%
\pgfusepath{clip}%
\pgfsetrectcap%
\pgfsetroundjoin%
\pgfsetlinewidth{1.505625pt}%
\definecolor{currentstroke}{rgb}{0.121569,0.466667,0.705882}%
\pgfsetstrokecolor{currentstroke}%
\pgfsetstrokeopacity{0.500000}%
\pgfsetdash{}{0pt}%
\pgfpathmoveto{\pgfqpoint{1.848679in}{1.461617in}}%
\pgfpathlineto{\pgfqpoint{1.867684in}{0.996854in}}%
\pgfpathlineto{\pgfqpoint{1.878129in}{0.908856in}}%
\pgfpathlineto{\pgfqpoint{1.885331in}{0.857978in}}%
\pgfpathlineto{\pgfqpoint{1.889178in}{0.851064in}}%
\pgfpathlineto{\pgfqpoint{1.893486in}{0.814892in}}%
\pgfpathlineto{\pgfqpoint{1.898279in}{0.804188in}}%
\pgfpathlineto{\pgfqpoint{1.904902in}{0.803379in}}%
\pgfpathlineto{\pgfqpoint{1.907246in}{0.780181in}}%
\pgfpathlineto{\pgfqpoint{1.912932in}{0.774543in}}%
\pgfpathlineto{\pgfqpoint{1.923418in}{0.773495in}}%
\pgfpathlineto{\pgfqpoint{1.927003in}{0.764692in}}%
\pgfpathlineto{\pgfqpoint{1.930681in}{0.763152in}}%
\pgfpathlineto{\pgfqpoint{1.938397in}{0.758644in}}%
\pgfpathlineto{\pgfqpoint{1.944819in}{0.754750in}}%
\pgfpathlineto{\pgfqpoint{1.945246in}{0.752971in}}%
\pgfpathlineto{\pgfqpoint{1.949459in}{0.750264in}}%
\pgfpathlineto{\pgfqpoint{1.966417in}{0.745282in}}%
\pgfpathlineto{\pgfqpoint{1.984450in}{0.738340in}}%
\pgfpathlineto{\pgfqpoint{2.008926in}{0.733388in}}%
\pgfpathlineto{\pgfqpoint{2.031338in}{0.731724in}}%
\pgfpathlineto{\pgfqpoint{2.032252in}{0.729528in}}%
\pgfpathlineto{\pgfqpoint{2.033576in}{0.725697in}}%
\pgfpathlineto{\pgfqpoint{2.037125in}{0.724332in}}%
\pgfpathlineto{\pgfqpoint{2.045595in}{0.723715in}}%
\pgfpathlineto{\pgfqpoint{2.051349in}{0.722081in}}%
\pgfpathlineto{\pgfqpoint{2.076904in}{0.719544in}}%
\pgfpathlineto{\pgfqpoint{2.080251in}{0.718197in}}%
\pgfpathlineto{\pgfqpoint{2.080349in}{0.716749in}}%
\pgfpathlineto{\pgfqpoint{2.085843in}{0.716510in}}%
\pgfpathlineto{\pgfqpoint{2.098822in}{0.714079in}}%
\pgfpathlineto{\pgfqpoint{2.104964in}{0.713831in}}%
\pgfpathlineto{\pgfqpoint{2.105914in}{0.712924in}}%
\pgfpathlineto{\pgfqpoint{2.157055in}{0.710594in}}%
\pgfpathlineto{\pgfqpoint{2.161059in}{0.706043in}}%
\pgfpathlineto{\pgfqpoint{2.163334in}{0.705692in}}%
\pgfpathlineto{\pgfqpoint{2.171432in}{0.705082in}}%
\pgfpathlineto{\pgfqpoint{2.187458in}{0.702763in}}%
\pgfpathlineto{\pgfqpoint{2.193297in}{0.702172in}}%
\pgfpathlineto{\pgfqpoint{2.198690in}{0.701434in}}%
\pgfpathlineto{\pgfqpoint{2.212467in}{0.699905in}}%
\pgfpathlineto{\pgfqpoint{2.234540in}{0.699174in}}%
\pgfpathlineto{\pgfqpoint{2.259841in}{0.698842in}}%
\pgfpathlineto{\pgfqpoint{2.278129in}{0.698798in}}%
\pgfpathlineto{\pgfqpoint{2.301002in}{0.698784in}}%
\pgfpathlineto{\pgfqpoint{2.409554in}{0.697909in}}%
\pgfpathlineto{\pgfqpoint{2.438334in}{0.697795in}}%
\pgfpathlineto{\pgfqpoint{2.479468in}{0.697460in}}%
\pgfpathlineto{\pgfqpoint{2.539290in}{0.697351in}}%
\pgfpathlineto{\pgfqpoint{2.571667in}{0.696836in}}%
\pgfpathlineto{\pgfqpoint{2.647748in}{0.696667in}}%
\pgfpathlineto{\pgfqpoint{2.745840in}{0.696168in}}%
\pgfpathlineto{\pgfqpoint{2.823078in}{0.695324in}}%
\pgfpathlineto{\pgfqpoint{2.944612in}{0.694845in}}%
\pgfpathlineto{\pgfqpoint{3.108851in}{0.693891in}}%
\pgfpathlineto{\pgfqpoint{3.274341in}{0.693028in}}%
\pgfpathlineto{\pgfqpoint{3.517294in}{0.691835in}}%
\pgfpathlineto{\pgfqpoint{3.922083in}{0.690316in}}%
\pgfpathlineto{\pgfqpoint{4.623274in}{0.688107in}}%
\pgfpathlineto{\pgfqpoint{6.888192in}{0.683471in}}%
\pgfusepath{stroke}%
\end{pgfscope}%
\begin{pgfscope}%
\pgfsetrectcap%
\pgfsetmiterjoin%
\pgfsetlinewidth{0.803000pt}%
\definecolor{currentstroke}{rgb}{0.000000,0.000000,0.000000}%
\pgfsetstrokecolor{currentstroke}%
\pgfsetdash{}{0pt}%
\pgfpathmoveto{\pgfqpoint{0.688192in}{0.670138in}}%
\pgfpathlineto{\pgfqpoint{0.688192in}{5.290138in}}%
\pgfusepath{stroke}%
\end{pgfscope}%
\begin{pgfscope}%
\pgfsetrectcap%
\pgfsetmiterjoin%
\pgfsetlinewidth{0.803000pt}%
\definecolor{currentstroke}{rgb}{0.000000,0.000000,0.000000}%
\pgfsetstrokecolor{currentstroke}%
\pgfsetdash{}{0pt}%
\pgfpathmoveto{\pgfqpoint{6.888192in}{0.670138in}}%
\pgfpathlineto{\pgfqpoint{6.888192in}{5.290138in}}%
\pgfusepath{stroke}%
\end{pgfscope}%
\begin{pgfscope}%
\pgfsetrectcap%
\pgfsetmiterjoin%
\pgfsetlinewidth{0.803000pt}%
\definecolor{currentstroke}{rgb}{0.000000,0.000000,0.000000}%
\pgfsetstrokecolor{currentstroke}%
\pgfsetdash{}{0pt}%
\pgfpathmoveto{\pgfqpoint{0.688192in}{0.670138in}}%
\pgfpathlineto{\pgfqpoint{6.888192in}{0.670138in}}%
\pgfusepath{stroke}%
\end{pgfscope}%
\begin{pgfscope}%
\pgfsetrectcap%
\pgfsetmiterjoin%
\pgfsetlinewidth{0.803000pt}%
\definecolor{currentstroke}{rgb}{0.000000,0.000000,0.000000}%
\pgfsetstrokecolor{currentstroke}%
\pgfsetdash{}{0pt}%
\pgfpathmoveto{\pgfqpoint{0.688192in}{5.290138in}}%
\pgfpathlineto{\pgfqpoint{6.888192in}{5.290138in}}%
\pgfusepath{stroke}%
\end{pgfscope}%
\begin{pgfscope}%
\pgfsetbuttcap%
\pgfsetmiterjoin%
\pgfsetlinewidth{1.003750pt}%
\definecolor{currentstroke}{rgb}{0.000000,0.000000,0.000000}%
\pgfsetstrokecolor{currentstroke}%
\pgfsetstrokeopacity{0.500000}%
\pgfsetdash{}{0pt}%
\pgfpathmoveto{\pgfqpoint{0.646542in}{1.071430in}}%
\pgfpathlineto{\pgfqpoint{0.810103in}{1.071430in}}%
\pgfpathlineto{\pgfqpoint{0.810103in}{1.434970in}}%
\pgfpathlineto{\pgfqpoint{0.646542in}{1.434970in}}%
\pgfpathlineto{\pgfqpoint{0.646542in}{1.071430in}}%
\pgfpathclose%
\pgfpathmoveto{\pgfqpoint{3.788192in}{5.151538in}}%
\pgfpathquadraticcurveto{\pgfqpoint{2.217367in}{3.293254in}}{\pgfqpoint{0.646542in}{1.434970in}}%
\pgfpathmoveto{\pgfqpoint{6.702192in}{2.980138in}}%
\pgfpathquadraticcurveto{\pgfqpoint{3.756147in}{2.025784in}}{\pgfqpoint{0.810103in}{1.071430in}}%
\pgfusepath{stroke}%
\end{pgfscope}%
\begin{pgfscope}%
\pgfsetbuttcap%
\pgfsetmiterjoin%
\definecolor{currentfill}{rgb}{1.000000,1.000000,1.000000}%
\pgfsetfillcolor{currentfill}%
\pgfsetlinewidth{0.000000pt}%
\definecolor{currentstroke}{rgb}{0.000000,0.000000,0.000000}%
\pgfsetstrokecolor{currentstroke}%
\pgfsetstrokeopacity{0.000000}%
\pgfsetdash{}{0pt}%
\pgfpathmoveto{\pgfqpoint{3.788192in}{2.980138in}}%
\pgfpathlineto{\pgfqpoint{6.702192in}{2.980138in}}%
\pgfpathlineto{\pgfqpoint{6.702192in}{5.151538in}}%
\pgfpathlineto{\pgfqpoint{3.788192in}{5.151538in}}%
\pgfpathlineto{\pgfqpoint{3.788192in}{2.980138in}}%
\pgfpathclose%
\pgfusepath{fill}%
\end{pgfscope}%
\begin{pgfscope}%
\pgfpathrectangle{\pgfqpoint{3.788192in}{2.980138in}}{\pgfqpoint{2.914000in}{2.171400in}}%
\pgfusepath{clip}%
\pgfsetbuttcap%
\pgfsetmiterjoin%
\definecolor{currentfill}{rgb}{0.121569,0.466667,0.705882}%
\pgfsetfillcolor{currentfill}%
\pgfsetfillopacity{0.500000}%
\pgfsetlinewidth{1.003750pt}%
\definecolor{currentstroke}{rgb}{0.121569,0.466667,0.705882}%
\pgfsetstrokecolor{currentstroke}%
\pgfsetstrokeopacity{0.500000}%
\pgfsetdash{}{0pt}%
\pgfpathmoveto{\pgfqpoint{5.478622in}{4.808529in}}%
\pgfpathlineto{\pgfqpoint{5.786444in}{2.284897in}}%
\pgfpathlineto{\pgfqpoint{5.955602in}{1.807070in}}%
\pgfpathlineto{\pgfqpoint{6.072257in}{1.530809in}}%
\pgfpathlineto{\pgfqpoint{6.134561in}{1.493263in}}%
\pgfpathlineto{\pgfqpoint{6.204334in}{1.296852in}}%
\pgfpathlineto{\pgfqpoint{6.281957in}{1.238734in}}%
\pgfpathlineto{\pgfqpoint{6.389240in}{1.234341in}}%
\pgfpathlineto{\pgfqpoint{6.427201in}{1.108377in}}%
\pgfpathlineto{\pgfqpoint{6.519285in}{1.077759in}}%
\pgfpathlineto{\pgfqpoint{6.689124in}{1.072073in}}%
\pgfpathlineto{\pgfqpoint{6.747190in}{1.024269in}}%
\pgfpathlineto{\pgfqpoint{6.806750in}{1.015907in}}%
\pgfpathlineto{\pgfqpoint{6.931730in}{0.991431in}}%
\pgfpathlineto{\pgfqpoint{7.035734in}{0.970288in}}%
\pgfpathlineto{\pgfqpoint{7.042663in}{0.960624in}}%
\pgfpathlineto{\pgfqpoint{7.110896in}{0.945926in}}%
\pgfpathlineto{\pgfqpoint{7.385541in}{0.918874in}}%
\pgfpathlineto{\pgfqpoint{7.677622in}{0.881182in}}%
\pgfpathlineto{\pgfqpoint{8.074033in}{0.854291in}}%
\pgfpathlineto{\pgfqpoint{8.437037in}{0.845259in}}%
\pgfpathlineto{\pgfqpoint{8.451833in}{0.833333in}}%
\pgfpathlineto{\pgfqpoint{8.473270in}{0.812531in}}%
\pgfpathlineto{\pgfqpoint{8.530759in}{0.805118in}}%
\pgfpathlineto{\pgfqpoint{8.667946in}{0.801766in}}%
\pgfpathlineto{\pgfqpoint{8.761129in}{0.792896in}}%
\pgfpathlineto{\pgfqpoint{9.175032in}{0.779118in}}%
\pgfpathlineto{\pgfqpoint{9.229242in}{0.771806in}}%
\pgfpathlineto{\pgfqpoint{9.230836in}{0.763945in}}%
\pgfpathlineto{\pgfqpoint{9.319803in}{0.762647in}}%
\pgfpathlineto{\pgfqpoint{9.530028in}{0.749444in}}%
\pgfpathlineto{\pgfqpoint{9.629502in}{0.748101in}}%
\pgfpathlineto{\pgfqpoint{9.644888in}{0.743176in}}%
\pgfpathlineto{\pgfqpoint{10.473184in}{0.730520in}}%
\pgfpathlineto{\pgfqpoint{10.538033in}{0.705808in}}%
\pgfpathlineto{\pgfqpoint{10.574876in}{0.703903in}}%
\pgfpathlineto{\pgfqpoint{10.706030in}{0.700593in}}%
\pgfpathlineto{\pgfqpoint{10.965602in}{0.687998in}}%
\pgfpathlineto{\pgfqpoint{11.060169in}{0.684789in}}%
\pgfpathlineto{\pgfqpoint{11.147519in}{0.680781in}}%
\pgfpathlineto{\pgfqpoint{11.370645in}{0.672484in}}%
\pgfpathlineto{\pgfqpoint{11.728159in}{0.668514in}}%
\pgfpathlineto{\pgfqpoint{12.137942in}{0.666709in}}%
\pgfpathlineto{\pgfqpoint{12.434130in}{0.666468in}}%
\pgfpathlineto{\pgfqpoint{12.804602in}{0.666397in}}%
\pgfpathlineto{\pgfqpoint{14.562739in}{0.661643in}}%
\pgfpathlineto{\pgfqpoint{15.028873in}{0.661022in}}%
\pgfpathlineto{\pgfqpoint{15.695083in}{0.659204in}}%
\pgfpathlineto{\pgfqpoint{16.663977in}{0.658612in}}%
\pgfpathlineto{\pgfqpoint{17.188373in}{0.655818in}}%
\pgfpathlineto{\pgfqpoint{18.420605in}{0.654898in}}%
\pgfpathlineto{\pgfqpoint{20.009331in}{0.652189in}}%
\pgfpathlineto{\pgfqpoint{21.260300in}{0.647607in}}%
\pgfpathlineto{\pgfqpoint{23.228701in}{0.645006in}}%
\pgfpathlineto{\pgfqpoint{25.888770in}{0.639824in}}%
\pgfpathlineto{\pgfqpoint{28.569097in}{0.635141in}}%
\pgfpathlineto{\pgfqpoint{32.504050in}{0.628662in}}%
\pgfpathlineto{\pgfqpoint{39.060135in}{0.620415in}}%
\pgfpathlineto{\pgfqpoint{50.416853in}{0.608419in}}%
\pgfpathlineto{\pgfqpoint{87.100182in}{0.583248in}}%
\pgfpathlineto{\pgfqpoint{114.989116in}{0.662886in}}%
\pgfpathlineto{\pgfqpoint{74.637454in}{0.690574in}}%
\pgfpathlineto{\pgfqpoint{62.145064in}{0.703770in}}%
\pgfpathlineto{\pgfqpoint{54.933371in}{0.712842in}}%
\pgfpathlineto{\pgfqpoint{50.604922in}{0.719969in}}%
\pgfpathlineto{\pgfqpoint{47.656562in}{0.725121in}}%
\pgfpathlineto{\pgfqpoint{44.730486in}{0.730820in}}%
\pgfpathlineto{\pgfqpoint{42.565245in}{0.733682in}}%
\pgfpathlineto{\pgfqpoint{41.189180in}{0.738722in}}%
\pgfpathlineto{\pgfqpoint{39.441581in}{0.741702in}}%
\pgfpathlineto{\pgfqpoint{38.086125in}{0.742714in}}%
\pgfpathlineto{\pgfqpoint{37.509290in}{0.745787in}}%
\pgfpathlineto{\pgfqpoint{36.443507in}{0.746439in}}%
\pgfpathlineto{\pgfqpoint{35.710675in}{0.748438in}}%
\pgfpathlineto{\pgfqpoint{35.197928in}{0.749121in}}%
\pgfpathlineto{\pgfqpoint{33.263977in}{0.754350in}}%
\pgfpathlineto{\pgfqpoint{32.856458in}{0.754429in}}%
\pgfpathlineto{\pgfqpoint{32.530652in}{0.754694in}}%
\pgfpathlineto{\pgfqpoint{32.079890in}{0.756679in}}%
\pgfpathlineto{\pgfqpoint{31.686625in}{0.761046in}}%
\pgfpathlineto{\pgfqpoint{31.441186in}{0.770173in}}%
\pgfpathlineto{\pgfqpoint{31.345101in}{0.774582in}}%
\pgfpathlineto{\pgfqpoint{31.241077in}{0.778112in}}%
\pgfpathlineto{\pgfqpoint{30.955549in}{0.791966in}}%
\pgfpathlineto{\pgfqpoint{30.811278in}{0.795608in}}%
\pgfpathlineto{\pgfqpoint{30.770752in}{0.797703in}}%
\pgfpathlineto{\pgfqpoint{30.699417in}{0.824886in}}%
\pgfpathlineto{\pgfqpoint{29.788292in}{0.838808in}}%
\pgfpathlineto{\pgfqpoint{29.771368in}{0.844225in}}%
\pgfpathlineto{\pgfqpoint{29.661946in}{0.845702in}}%
\pgfpathlineto{\pgfqpoint{29.430699in}{0.860225in}}%
\pgfpathlineto{\pgfqpoint{29.332834in}{0.861654in}}%
\pgfpathlineto{\pgfqpoint{29.331081in}{0.870301in}}%
\pgfpathlineto{\pgfqpoint{29.271451in}{0.878344in}}%
\pgfpathlineto{\pgfqpoint{28.816157in}{0.893499in}}%
\pgfpathlineto{\pgfqpoint{28.713656in}{0.903257in}}%
\pgfpathlineto{\pgfqpoint{28.562750in}{0.906943in}}%
\pgfpathlineto{\pgfqpoint{28.499512in}{0.915098in}}%
\pgfpathlineto{\pgfqpoint{28.475931in}{0.937980in}}%
\pgfpathlineto{\pgfqpoint{28.459655in}{0.951098in}}%
\pgfpathlineto{\pgfqpoint{28.060351in}{0.961034in}}%
\pgfpathlineto{\pgfqpoint{27.624299in}{0.990614in}}%
\pgfpathlineto{\pgfqpoint{27.303010in}{1.032076in}}%
\pgfpathlineto{\pgfqpoint{27.000901in}{1.061832in}}%
\pgfpathlineto{\pgfqpoint{26.925844in}{1.078000in}}%
\pgfpathlineto{\pgfqpoint{26.918222in}{1.088630in}}%
\pgfpathlineto{\pgfqpoint{26.803818in}{1.111888in}}%
\pgfpathlineto{\pgfqpoint{26.666341in}{1.138811in}}%
\pgfpathlineto{\pgfqpoint{26.600824in}{1.148010in}}%
\pgfpathlineto{\pgfqpoint{26.536952in}{1.200594in}}%
\pgfpathlineto{\pgfqpoint{26.350128in}{1.206849in}}%
\pgfpathlineto{\pgfqpoint{26.248836in}{1.240528in}}%
\pgfpathlineto{\pgfqpoint{26.207079in}{1.379089in}}%
\pgfpathlineto{\pgfqpoint{26.089068in}{1.383921in}}%
\pgfpathlineto{\pgfqpoint{26.003683in}{1.447851in}}%
\pgfpathlineto{\pgfqpoint{25.926932in}{1.663903in}}%
\pgfpathlineto{\pgfqpoint{25.858398in}{1.705204in}}%
\pgfpathlineto{\pgfqpoint{25.730077in}{2.009091in}}%
\pgfpathlineto{\pgfqpoint{25.544004in}{2.534701in}}%
\pgfpathlineto{\pgfqpoint{25.205400in}{5.310696in}}%
\pgfpathlineto{\pgfqpoint{5.478622in}{4.808529in}}%
\pgfpathclose%
\pgfusepath{stroke,fill}%
\end{pgfscope}%
\begin{pgfscope}%
\pgfpathrectangle{\pgfqpoint{3.788192in}{2.980138in}}{\pgfqpoint{2.914000in}{2.171400in}}%
\pgfusepath{clip}%
\pgfsetbuttcap%
\pgfsetroundjoin%
\pgfsetlinewidth{1.003750pt}%
\definecolor{currentstroke}{rgb}{1.000000,0.000000,0.000000}%
\pgfsetstrokecolor{currentstroke}%
\pgfsetdash{}{0pt}%
\pgfpathmoveto{\pgfqpoint{16.861003in}{25.566105in}}%
\pgfpathcurveto{\pgfqpoint{16.869240in}{25.566105in}}{\pgfqpoint{16.877140in}{25.569377in}}{\pgfqpoint{16.882964in}{25.575201in}}%
\pgfpathcurveto{\pgfqpoint{16.888788in}{25.581025in}}{\pgfqpoint{16.892060in}{25.588925in}}{\pgfqpoint{16.892060in}{25.597161in}}%
\pgfpathcurveto{\pgfqpoint{16.892060in}{25.605397in}}{\pgfqpoint{16.888788in}{25.613297in}}{\pgfqpoint{16.882964in}{25.619121in}}%
\pgfpathcurveto{\pgfqpoint{16.877140in}{25.624945in}}{\pgfqpoint{16.869240in}{25.628218in}}{\pgfqpoint{16.861003in}{25.628218in}}%
\pgfpathcurveto{\pgfqpoint{16.852767in}{25.628218in}}{\pgfqpoint{16.844867in}{25.624945in}}{\pgfqpoint{16.839043in}{25.619121in}}%
\pgfpathcurveto{\pgfqpoint{16.833219in}{25.613297in}}{\pgfqpoint{16.829947in}{25.605397in}}{\pgfqpoint{16.829947in}{25.597161in}}%
\pgfpathcurveto{\pgfqpoint{16.829947in}{25.588925in}}{\pgfqpoint{16.833219in}{25.581025in}}{\pgfqpoint{16.839043in}{25.575201in}}%
\pgfpathcurveto{\pgfqpoint{16.844867in}{25.569377in}}{\pgfqpoint{16.852767in}{25.566105in}}{\pgfqpoint{16.861003in}{25.566105in}}%
\pgfusepath{stroke}%
\end{pgfscope}%
\begin{pgfscope}%
\pgfpathrectangle{\pgfqpoint{3.788192in}{2.980138in}}{\pgfqpoint{2.914000in}{2.171400in}}%
\pgfusepath{clip}%
\pgfsetbuttcap%
\pgfsetroundjoin%
\pgfsetlinewidth{1.003750pt}%
\definecolor{currentstroke}{rgb}{1.000000,0.000000,0.000000}%
\pgfsetstrokecolor{currentstroke}%
\pgfsetdash{}{0pt}%
\pgfpathmoveto{\pgfqpoint{11.934253in}{10.111217in}}%
\pgfpathcurveto{\pgfqpoint{11.942489in}{10.111217in}}{\pgfqpoint{11.950389in}{10.114489in}}{\pgfqpoint{11.956213in}{10.120313in}}%
\pgfpathcurveto{\pgfqpoint{11.962037in}{10.126137in}}{\pgfqpoint{11.965309in}{10.134037in}}{\pgfqpoint{11.965309in}{10.142274in}}%
\pgfpathcurveto{\pgfqpoint{11.965309in}{10.150510in}}{\pgfqpoint{11.962037in}{10.158410in}}{\pgfqpoint{11.956213in}{10.164234in}}%
\pgfpathcurveto{\pgfqpoint{11.950389in}{10.170058in}}{\pgfqpoint{11.942489in}{10.173330in}}{\pgfqpoint{11.934253in}{10.173330in}}%
\pgfpathcurveto{\pgfqpoint{11.926016in}{10.173330in}}{\pgfqpoint{11.918116in}{10.170058in}}{\pgfqpoint{11.912292in}{10.164234in}}%
\pgfpathcurveto{\pgfqpoint{11.906469in}{10.158410in}}{\pgfqpoint{11.903196in}{10.150510in}}{\pgfqpoint{11.903196in}{10.142274in}}%
\pgfpathcurveto{\pgfqpoint{11.903196in}{10.134037in}}{\pgfqpoint{11.906469in}{10.126137in}}{\pgfqpoint{11.912292in}{10.120313in}}%
\pgfpathcurveto{\pgfqpoint{11.918116in}{10.114489in}}{\pgfqpoint{11.926016in}{10.111217in}}{\pgfqpoint{11.934253in}{10.111217in}}%
\pgfusepath{stroke}%
\end{pgfscope}%
\begin{pgfscope}%
\pgfpathrectangle{\pgfqpoint{3.788192in}{2.980138in}}{\pgfqpoint{2.914000in}{2.171400in}}%
\pgfusepath{clip}%
\pgfsetbuttcap%
\pgfsetroundjoin%
\pgfsetlinewidth{1.003750pt}%
\definecolor{currentstroke}{rgb}{1.000000,0.000000,0.000000}%
\pgfsetstrokecolor{currentstroke}%
\pgfsetdash{}{0pt}%
\pgfpathmoveto{\pgfqpoint{12.691767in}{10.530438in}}%
\pgfpathcurveto{\pgfqpoint{12.700003in}{10.530438in}}{\pgfqpoint{12.707903in}{10.533710in}}{\pgfqpoint{12.713727in}{10.539534in}}%
\pgfpathcurveto{\pgfqpoint{12.719551in}{10.545358in}}{\pgfqpoint{12.722823in}{10.553258in}}{\pgfqpoint{12.722823in}{10.561494in}}%
\pgfpathcurveto{\pgfqpoint{12.722823in}{10.569730in}}{\pgfqpoint{12.719551in}{10.577630in}}{\pgfqpoint{12.713727in}{10.583454in}}%
\pgfpathcurveto{\pgfqpoint{12.707903in}{10.589278in}}{\pgfqpoint{12.700003in}{10.592551in}}{\pgfqpoint{12.691767in}{10.592551in}}%
\pgfpathcurveto{\pgfqpoint{12.683530in}{10.592551in}}{\pgfqpoint{12.675630in}{10.589278in}}{\pgfqpoint{12.669806in}{10.583454in}}%
\pgfpathcurveto{\pgfqpoint{12.663983in}{10.577630in}}{\pgfqpoint{12.660710in}{10.569730in}}{\pgfqpoint{12.660710in}{10.561494in}}%
\pgfpathcurveto{\pgfqpoint{12.660710in}{10.553258in}}{\pgfqpoint{12.663983in}{10.545358in}}{\pgfqpoint{12.669806in}{10.539534in}}%
\pgfpathcurveto{\pgfqpoint{12.675630in}{10.533710in}}{\pgfqpoint{12.683530in}{10.530438in}}{\pgfqpoint{12.691767in}{10.530438in}}%
\pgfusepath{stroke}%
\end{pgfscope}%
\begin{pgfscope}%
\pgfpathrectangle{\pgfqpoint{3.788192in}{2.980138in}}{\pgfqpoint{2.914000in}{2.171400in}}%
\pgfusepath{clip}%
\pgfsetbuttcap%
\pgfsetroundjoin%
\pgfsetlinewidth{1.003750pt}%
\definecolor{currentstroke}{rgb}{1.000000,0.000000,0.000000}%
\pgfsetstrokecolor{currentstroke}%
\pgfsetdash{}{0pt}%
\pgfpathmoveto{\pgfqpoint{15.144604in}{10.695451in}}%
\pgfpathcurveto{\pgfqpoint{15.152841in}{10.695451in}}{\pgfqpoint{15.160741in}{10.698723in}}{\pgfqpoint{15.166565in}{10.704547in}}%
\pgfpathcurveto{\pgfqpoint{15.172389in}{10.710371in}}{\pgfqpoint{15.175661in}{10.718271in}}{\pgfqpoint{15.175661in}{10.726507in}}%
\pgfpathcurveto{\pgfqpoint{15.175661in}{10.734743in}}{\pgfqpoint{15.172389in}{10.742643in}}{\pgfqpoint{15.166565in}{10.748467in}}%
\pgfpathcurveto{\pgfqpoint{15.160741in}{10.754291in}}{\pgfqpoint{15.152841in}{10.757564in}}{\pgfqpoint{15.144604in}{10.757564in}}%
\pgfpathcurveto{\pgfqpoint{15.136368in}{10.757564in}}{\pgfqpoint{15.128468in}{10.754291in}}{\pgfqpoint{15.122644in}{10.748467in}}%
\pgfpathcurveto{\pgfqpoint{15.116820in}{10.742643in}}{\pgfqpoint{15.113548in}{10.734743in}}{\pgfqpoint{15.113548in}{10.726507in}}%
\pgfpathcurveto{\pgfqpoint{15.113548in}{10.718271in}}{\pgfqpoint{15.116820in}{10.710371in}}{\pgfqpoint{15.122644in}{10.704547in}}%
\pgfpathcurveto{\pgfqpoint{15.128468in}{10.698723in}}{\pgfqpoint{15.136368in}{10.695451in}}{\pgfqpoint{15.144604in}{10.695451in}}%
\pgfusepath{stroke}%
\end{pgfscope}%
\begin{pgfscope}%
\pgfpathrectangle{\pgfqpoint{3.788192in}{2.980138in}}{\pgfqpoint{2.914000in}{2.171400in}}%
\pgfusepath{clip}%
\pgfsetbuttcap%
\pgfsetroundjoin%
\pgfsetlinewidth{1.003750pt}%
\definecolor{currentstroke}{rgb}{1.000000,0.000000,0.000000}%
\pgfsetstrokecolor{currentstroke}%
\pgfsetdash{}{0pt}%
\pgfpathmoveto{\pgfqpoint{13.542463in}{11.423191in}}%
\pgfpathcurveto{\pgfqpoint{13.550699in}{11.423191in}}{\pgfqpoint{13.558599in}{11.426464in}}{\pgfqpoint{13.564423in}{11.432288in}}%
\pgfpathcurveto{\pgfqpoint{13.570247in}{11.438112in}}{\pgfqpoint{13.573519in}{11.446012in}}{\pgfqpoint{13.573519in}{11.454248in}}%
\pgfpathcurveto{\pgfqpoint{13.573519in}{11.462484in}}{\pgfqpoint{13.570247in}{11.470384in}}{\pgfqpoint{13.564423in}{11.476208in}}%
\pgfpathcurveto{\pgfqpoint{13.558599in}{11.482032in}}{\pgfqpoint{13.550699in}{11.485304in}}{\pgfqpoint{13.542463in}{11.485304in}}%
\pgfpathcurveto{\pgfqpoint{13.534227in}{11.485304in}}{\pgfqpoint{13.526327in}{11.482032in}}{\pgfqpoint{13.520503in}{11.476208in}}%
\pgfpathcurveto{\pgfqpoint{13.514679in}{11.470384in}}{\pgfqpoint{13.511406in}{11.462484in}}{\pgfqpoint{13.511406in}{11.454248in}}%
\pgfpathcurveto{\pgfqpoint{13.511406in}{11.446012in}}{\pgfqpoint{13.514679in}{11.438112in}}{\pgfqpoint{13.520503in}{11.432288in}}%
\pgfpathcurveto{\pgfqpoint{13.526327in}{11.426464in}}{\pgfqpoint{13.534227in}{11.423191in}}{\pgfqpoint{13.542463in}{11.423191in}}%
\pgfusepath{stroke}%
\end{pgfscope}%
\begin{pgfscope}%
\pgfpathrectangle{\pgfqpoint{3.788192in}{2.980138in}}{\pgfqpoint{2.914000in}{2.171400in}}%
\pgfusepath{clip}%
\pgfsetbuttcap%
\pgfsetroundjoin%
\pgfsetlinewidth{1.003750pt}%
\definecolor{currentstroke}{rgb}{1.000000,0.000000,0.000000}%
\pgfsetstrokecolor{currentstroke}%
\pgfsetdash{}{0pt}%
\pgfpathmoveto{\pgfqpoint{12.332968in}{8.917460in}}%
\pgfpathcurveto{\pgfqpoint{12.341204in}{8.917460in}}{\pgfqpoint{12.349104in}{8.920732in}}{\pgfqpoint{12.354928in}{8.926556in}}%
\pgfpathcurveto{\pgfqpoint{12.360752in}{8.932380in}}{\pgfqpoint{12.364025in}{8.940280in}}{\pgfqpoint{12.364025in}{8.948517in}}%
\pgfpathcurveto{\pgfqpoint{12.364025in}{8.956753in}}{\pgfqpoint{12.360752in}{8.964653in}}{\pgfqpoint{12.354928in}{8.970477in}}%
\pgfpathcurveto{\pgfqpoint{12.349104in}{8.976301in}}{\pgfqpoint{12.341204in}{8.979573in}}{\pgfqpoint{12.332968in}{8.979573in}}%
\pgfpathcurveto{\pgfqpoint{12.324732in}{8.979573in}}{\pgfqpoint{12.316832in}{8.976301in}}{\pgfqpoint{12.311008in}{8.970477in}}%
\pgfpathcurveto{\pgfqpoint{12.305184in}{8.964653in}}{\pgfqpoint{12.301912in}{8.956753in}}{\pgfqpoint{12.301912in}{8.948517in}}%
\pgfpathcurveto{\pgfqpoint{12.301912in}{8.940280in}}{\pgfqpoint{12.305184in}{8.932380in}}{\pgfqpoint{12.311008in}{8.926556in}}%
\pgfpathcurveto{\pgfqpoint{12.316832in}{8.920732in}}{\pgfqpoint{12.324732in}{8.917460in}}{\pgfqpoint{12.332968in}{8.917460in}}%
\pgfusepath{stroke}%
\end{pgfscope}%
\begin{pgfscope}%
\pgfpathrectangle{\pgfqpoint{3.788192in}{2.980138in}}{\pgfqpoint{2.914000in}{2.171400in}}%
\pgfusepath{clip}%
\pgfsetbuttcap%
\pgfsetroundjoin%
\pgfsetlinewidth{1.003750pt}%
\definecolor{currentstroke}{rgb}{1.000000,0.000000,0.000000}%
\pgfsetstrokecolor{currentstroke}%
\pgfsetdash{}{0pt}%
\pgfpathmoveto{\pgfqpoint{11.399902in}{7.106716in}}%
\pgfpathcurveto{\pgfqpoint{11.408139in}{7.106716in}}{\pgfqpoint{11.416039in}{7.109988in}}{\pgfqpoint{11.421863in}{7.115812in}}%
\pgfpathcurveto{\pgfqpoint{11.427686in}{7.121636in}}{\pgfqpoint{11.430959in}{7.129536in}}{\pgfqpoint{11.430959in}{7.137773in}}%
\pgfpathcurveto{\pgfqpoint{11.430959in}{7.146009in}}{\pgfqpoint{11.427686in}{7.153909in}}{\pgfqpoint{11.421863in}{7.159733in}}%
\pgfpathcurveto{\pgfqpoint{11.416039in}{7.165557in}}{\pgfqpoint{11.408139in}{7.168829in}}{\pgfqpoint{11.399902in}{7.168829in}}%
\pgfpathcurveto{\pgfqpoint{11.391666in}{7.168829in}}{\pgfqpoint{11.383766in}{7.165557in}}{\pgfqpoint{11.377942in}{7.159733in}}%
\pgfpathcurveto{\pgfqpoint{11.372118in}{7.153909in}}{\pgfqpoint{11.368846in}{7.146009in}}{\pgfqpoint{11.368846in}{7.137773in}}%
\pgfpathcurveto{\pgfqpoint{11.368846in}{7.129536in}}{\pgfqpoint{11.372118in}{7.121636in}}{\pgfqpoint{11.377942in}{7.115812in}}%
\pgfpathcurveto{\pgfqpoint{11.383766in}{7.109988in}}{\pgfqpoint{11.391666in}{7.106716in}}{\pgfqpoint{11.399902in}{7.106716in}}%
\pgfusepath{stroke}%
\end{pgfscope}%
\begin{pgfscope}%
\pgfpathrectangle{\pgfqpoint{3.788192in}{2.980138in}}{\pgfqpoint{2.914000in}{2.171400in}}%
\pgfusepath{clip}%
\pgfsetbuttcap%
\pgfsetroundjoin%
\pgfsetlinewidth{1.003750pt}%
\definecolor{currentstroke}{rgb}{1.000000,0.000000,0.000000}%
\pgfsetstrokecolor{currentstroke}%
\pgfsetdash{}{0pt}%
\pgfpathmoveto{\pgfqpoint{11.356549in}{7.029726in}}%
\pgfpathcurveto{\pgfqpoint{11.364785in}{7.029726in}}{\pgfqpoint{11.372685in}{7.032998in}}{\pgfqpoint{11.378509in}{7.038822in}}%
\pgfpathcurveto{\pgfqpoint{11.384333in}{7.044646in}}{\pgfqpoint{11.387605in}{7.052546in}}{\pgfqpoint{11.387605in}{7.060782in}}%
\pgfpathcurveto{\pgfqpoint{11.387605in}{7.069018in}}{\pgfqpoint{11.384333in}{7.076918in}}{\pgfqpoint{11.378509in}{7.082742in}}%
\pgfpathcurveto{\pgfqpoint{11.372685in}{7.088566in}}{\pgfqpoint{11.364785in}{7.091839in}}{\pgfqpoint{11.356549in}{7.091839in}}%
\pgfpathcurveto{\pgfqpoint{11.348313in}{7.091839in}}{\pgfqpoint{11.340412in}{7.088566in}}{\pgfqpoint{11.334589in}{7.082742in}}%
\pgfpathcurveto{\pgfqpoint{11.328765in}{7.076918in}}{\pgfqpoint{11.325492in}{7.069018in}}{\pgfqpoint{11.325492in}{7.060782in}}%
\pgfpathcurveto{\pgfqpoint{11.325492in}{7.052546in}}{\pgfqpoint{11.328765in}{7.044646in}}{\pgfqpoint{11.334589in}{7.038822in}}%
\pgfpathcurveto{\pgfqpoint{11.340412in}{7.032998in}}{\pgfqpoint{11.348313in}{7.029726in}}{\pgfqpoint{11.356549in}{7.029726in}}%
\pgfusepath{stroke}%
\end{pgfscope}%
\begin{pgfscope}%
\pgfpathrectangle{\pgfqpoint{3.788192in}{2.980138in}}{\pgfqpoint{2.914000in}{2.171400in}}%
\pgfusepath{clip}%
\pgfsetbuttcap%
\pgfsetroundjoin%
\pgfsetlinewidth{1.003750pt}%
\definecolor{currentstroke}{rgb}{1.000000,0.000000,0.000000}%
\pgfsetstrokecolor{currentstroke}%
\pgfsetdash{}{0pt}%
\pgfpathmoveto{\pgfqpoint{13.232346in}{9.803683in}}%
\pgfpathcurveto{\pgfqpoint{13.240582in}{9.803683in}}{\pgfqpoint{13.248482in}{9.806956in}}{\pgfqpoint{13.254306in}{9.812780in}}%
\pgfpathcurveto{\pgfqpoint{13.260130in}{9.818604in}}{\pgfqpoint{13.263402in}{9.826504in}}{\pgfqpoint{13.263402in}{9.834740in}}%
\pgfpathcurveto{\pgfqpoint{13.263402in}{9.842976in}}{\pgfqpoint{13.260130in}{9.850876in}}{\pgfqpoint{13.254306in}{9.856700in}}%
\pgfpathcurveto{\pgfqpoint{13.248482in}{9.862524in}}{\pgfqpoint{13.240582in}{9.865796in}}{\pgfqpoint{13.232346in}{9.865796in}}%
\pgfpathcurveto{\pgfqpoint{13.224110in}{9.865796in}}{\pgfqpoint{13.216210in}{9.862524in}}{\pgfqpoint{13.210386in}{9.856700in}}%
\pgfpathcurveto{\pgfqpoint{13.204562in}{9.850876in}}{\pgfqpoint{13.201289in}{9.842976in}}{\pgfqpoint{13.201289in}{9.834740in}}%
\pgfpathcurveto{\pgfqpoint{13.201289in}{9.826504in}}{\pgfqpoint{13.204562in}{9.818604in}}{\pgfqpoint{13.210386in}{9.812780in}}%
\pgfpathcurveto{\pgfqpoint{13.216210in}{9.806956in}}{\pgfqpoint{13.224110in}{9.803683in}}{\pgfqpoint{13.232346in}{9.803683in}}%
\pgfusepath{stroke}%
\end{pgfscope}%
\begin{pgfscope}%
\pgfpathrectangle{\pgfqpoint{3.788192in}{2.980138in}}{\pgfqpoint{2.914000in}{2.171400in}}%
\pgfusepath{clip}%
\pgfsetbuttcap%
\pgfsetroundjoin%
\pgfsetlinewidth{1.003750pt}%
\definecolor{currentstroke}{rgb}{1.000000,0.000000,0.000000}%
\pgfsetstrokecolor{currentstroke}%
\pgfsetdash{}{0pt}%
\pgfpathmoveto{\pgfqpoint{12.940652in}{8.318719in}}%
\pgfpathcurveto{\pgfqpoint{12.948888in}{8.318719in}}{\pgfqpoint{12.956788in}{8.321992in}}{\pgfqpoint{12.962612in}{8.327816in}}%
\pgfpathcurveto{\pgfqpoint{12.968436in}{8.333640in}}{\pgfqpoint{12.971708in}{8.341540in}}{\pgfqpoint{12.971708in}{8.349776in}}%
\pgfpathcurveto{\pgfqpoint{12.971708in}{8.358012in}}{\pgfqpoint{12.968436in}{8.365912in}}{\pgfqpoint{12.962612in}{8.371736in}}%
\pgfpathcurveto{\pgfqpoint{12.956788in}{8.377560in}}{\pgfqpoint{12.948888in}{8.380832in}}{\pgfqpoint{12.940652in}{8.380832in}}%
\pgfpathcurveto{\pgfqpoint{12.932415in}{8.380832in}}{\pgfqpoint{12.924515in}{8.377560in}}{\pgfqpoint{12.918691in}{8.371736in}}%
\pgfpathcurveto{\pgfqpoint{12.912867in}{8.365912in}}{\pgfqpoint{12.909595in}{8.358012in}}{\pgfqpoint{12.909595in}{8.349776in}}%
\pgfpathcurveto{\pgfqpoint{12.909595in}{8.341540in}}{\pgfqpoint{12.912867in}{8.333640in}}{\pgfqpoint{12.918691in}{8.327816in}}%
\pgfpathcurveto{\pgfqpoint{12.924515in}{8.321992in}}{\pgfqpoint{12.932415in}{8.318719in}}{\pgfqpoint{12.940652in}{8.318719in}}%
\pgfusepath{stroke}%
\end{pgfscope}%
\begin{pgfscope}%
\pgfpathrectangle{\pgfqpoint{3.788192in}{2.980138in}}{\pgfqpoint{2.914000in}{2.171400in}}%
\pgfusepath{clip}%
\pgfsetbuttcap%
\pgfsetroundjoin%
\pgfsetlinewidth{1.003750pt}%
\definecolor{currentstroke}{rgb}{1.000000,0.000000,0.000000}%
\pgfsetstrokecolor{currentstroke}%
\pgfsetdash{}{0pt}%
\pgfpathmoveto{\pgfqpoint{12.936873in}{8.298012in}}%
\pgfpathcurveto{\pgfqpoint{12.945109in}{8.298012in}}{\pgfqpoint{12.953009in}{8.301284in}}{\pgfqpoint{12.958833in}{8.307108in}}%
\pgfpathcurveto{\pgfqpoint{12.964657in}{8.312932in}}{\pgfqpoint{12.967929in}{8.320832in}}{\pgfqpoint{12.967929in}{8.329068in}}%
\pgfpathcurveto{\pgfqpoint{12.967929in}{8.337305in}}{\pgfqpoint{12.964657in}{8.345205in}}{\pgfqpoint{12.958833in}{8.351028in}}%
\pgfpathcurveto{\pgfqpoint{12.953009in}{8.356852in}}{\pgfqpoint{12.945109in}{8.360125in}}{\pgfqpoint{12.936873in}{8.360125in}}%
\pgfpathcurveto{\pgfqpoint{12.928637in}{8.360125in}}{\pgfqpoint{12.920737in}{8.356852in}}{\pgfqpoint{12.914913in}{8.351028in}}%
\pgfpathcurveto{\pgfqpoint{12.909089in}{8.345205in}}{\pgfqpoint{12.905816in}{8.337305in}}{\pgfqpoint{12.905816in}{8.329068in}}%
\pgfpathcurveto{\pgfqpoint{12.905816in}{8.320832in}}{\pgfqpoint{12.909089in}{8.312932in}}{\pgfqpoint{12.914913in}{8.307108in}}%
\pgfpathcurveto{\pgfqpoint{12.920737in}{8.301284in}}{\pgfqpoint{12.928637in}{8.298012in}}{\pgfqpoint{12.936873in}{8.298012in}}%
\pgfusepath{stroke}%
\end{pgfscope}%
\begin{pgfscope}%
\pgfpathrectangle{\pgfqpoint{3.788192in}{2.980138in}}{\pgfqpoint{2.914000in}{2.171400in}}%
\pgfusepath{clip}%
\pgfsetbuttcap%
\pgfsetroundjoin%
\pgfsetlinewidth{1.003750pt}%
\definecolor{currentstroke}{rgb}{1.000000,0.000000,0.000000}%
\pgfsetstrokecolor{currentstroke}%
\pgfsetdash{}{0pt}%
\pgfpathmoveto{\pgfqpoint{12.900060in}{6.352602in}}%
\pgfpathcurveto{\pgfqpoint{12.908296in}{6.352602in}}{\pgfqpoint{12.916196in}{6.355874in}}{\pgfqpoint{12.922020in}{6.361698in}}%
\pgfpathcurveto{\pgfqpoint{12.927844in}{6.367522in}}{\pgfqpoint{12.931116in}{6.375422in}}{\pgfqpoint{12.931116in}{6.383658in}}%
\pgfpathcurveto{\pgfqpoint{12.931116in}{6.391894in}}{\pgfqpoint{12.927844in}{6.399794in}}{\pgfqpoint{12.922020in}{6.405618in}}%
\pgfpathcurveto{\pgfqpoint{12.916196in}{6.411442in}}{\pgfqpoint{12.908296in}{6.414715in}}{\pgfqpoint{12.900060in}{6.414715in}}%
\pgfpathcurveto{\pgfqpoint{12.891823in}{6.414715in}}{\pgfqpoint{12.883923in}{6.411442in}}{\pgfqpoint{12.878099in}{6.405618in}}%
\pgfpathcurveto{\pgfqpoint{12.872275in}{6.399794in}}{\pgfqpoint{12.869003in}{6.391894in}}{\pgfqpoint{12.869003in}{6.383658in}}%
\pgfpathcurveto{\pgfqpoint{12.869003in}{6.375422in}}{\pgfqpoint{12.872275in}{6.367522in}}{\pgfqpoint{12.878099in}{6.361698in}}%
\pgfpathcurveto{\pgfqpoint{12.883923in}{6.355874in}}{\pgfqpoint{12.891823in}{6.352602in}}{\pgfqpoint{12.900060in}{6.352602in}}%
\pgfusepath{stroke}%
\end{pgfscope}%
\begin{pgfscope}%
\pgfpathrectangle{\pgfqpoint{3.788192in}{2.980138in}}{\pgfqpoint{2.914000in}{2.171400in}}%
\pgfusepath{clip}%
\pgfsetbuttcap%
\pgfsetroundjoin%
\pgfsetlinewidth{1.003750pt}%
\definecolor{currentstroke}{rgb}{1.000000,0.000000,0.000000}%
\pgfsetstrokecolor{currentstroke}%
\pgfsetdash{}{0pt}%
\pgfpathmoveto{\pgfqpoint{12.861075in}{6.367856in}}%
\pgfpathcurveto{\pgfqpoint{12.869311in}{6.367856in}}{\pgfqpoint{12.877211in}{6.371128in}}{\pgfqpoint{12.883035in}{6.376952in}}%
\pgfpathcurveto{\pgfqpoint{12.888859in}{6.382776in}}{\pgfqpoint{12.892131in}{6.390676in}}{\pgfqpoint{12.892131in}{6.398913in}}%
\pgfpathcurveto{\pgfqpoint{12.892131in}{6.407149in}}{\pgfqpoint{12.888859in}{6.415049in}}{\pgfqpoint{12.883035in}{6.420873in}}%
\pgfpathcurveto{\pgfqpoint{12.877211in}{6.426697in}}{\pgfqpoint{12.869311in}{6.429969in}}{\pgfqpoint{12.861075in}{6.429969in}}%
\pgfpathcurveto{\pgfqpoint{12.852838in}{6.429969in}}{\pgfqpoint{12.844938in}{6.426697in}}{\pgfqpoint{12.839114in}{6.420873in}}%
\pgfpathcurveto{\pgfqpoint{12.833291in}{6.415049in}}{\pgfqpoint{12.830018in}{6.407149in}}{\pgfqpoint{12.830018in}{6.398913in}}%
\pgfpathcurveto{\pgfqpoint{12.830018in}{6.390676in}}{\pgfqpoint{12.833291in}{6.382776in}}{\pgfqpoint{12.839114in}{6.376952in}}%
\pgfpathcurveto{\pgfqpoint{12.844938in}{6.371128in}}{\pgfqpoint{12.852838in}{6.367856in}}{\pgfqpoint{12.861075in}{6.367856in}}%
\pgfusepath{stroke}%
\end{pgfscope}%
\begin{pgfscope}%
\pgfpathrectangle{\pgfqpoint{3.788192in}{2.980138in}}{\pgfqpoint{2.914000in}{2.171400in}}%
\pgfusepath{clip}%
\pgfsetbuttcap%
\pgfsetroundjoin%
\pgfsetlinewidth{1.003750pt}%
\definecolor{currentstroke}{rgb}{1.000000,0.000000,0.000000}%
\pgfsetstrokecolor{currentstroke}%
\pgfsetdash{}{0pt}%
\pgfpathmoveto{\pgfqpoint{15.347266in}{8.054762in}}%
\pgfpathcurveto{\pgfqpoint{15.355502in}{8.054762in}}{\pgfqpoint{15.363402in}{8.058034in}}{\pgfqpoint{15.369226in}{8.063858in}}%
\pgfpathcurveto{\pgfqpoint{15.375050in}{8.069682in}}{\pgfqpoint{15.378323in}{8.077582in}}{\pgfqpoint{15.378323in}{8.085818in}}%
\pgfpathcurveto{\pgfqpoint{15.378323in}{8.094054in}}{\pgfqpoint{15.375050in}{8.101954in}}{\pgfqpoint{15.369226in}{8.107778in}}%
\pgfpathcurveto{\pgfqpoint{15.363402in}{8.113602in}}{\pgfqpoint{15.355502in}{8.116875in}}{\pgfqpoint{15.347266in}{8.116875in}}%
\pgfpathcurveto{\pgfqpoint{15.339030in}{8.116875in}}{\pgfqpoint{15.331130in}{8.113602in}}{\pgfqpoint{15.325306in}{8.107778in}}%
\pgfpathcurveto{\pgfqpoint{15.319482in}{8.101954in}}{\pgfqpoint{15.316210in}{8.094054in}}{\pgfqpoint{15.316210in}{8.085818in}}%
\pgfpathcurveto{\pgfqpoint{15.316210in}{8.077582in}}{\pgfqpoint{15.319482in}{8.069682in}}{\pgfqpoint{15.325306in}{8.063858in}}%
\pgfpathcurveto{\pgfqpoint{15.331130in}{8.058034in}}{\pgfqpoint{15.339030in}{8.054762in}}{\pgfqpoint{15.347266in}{8.054762in}}%
\pgfusepath{stroke}%
\end{pgfscope}%
\begin{pgfscope}%
\pgfpathrectangle{\pgfqpoint{3.788192in}{2.980138in}}{\pgfqpoint{2.914000in}{2.171400in}}%
\pgfusepath{clip}%
\pgfsetbuttcap%
\pgfsetroundjoin%
\pgfsetlinewidth{1.003750pt}%
\definecolor{currentstroke}{rgb}{1.000000,0.000000,0.000000}%
\pgfsetstrokecolor{currentstroke}%
\pgfsetdash{}{0pt}%
\pgfpathmoveto{\pgfqpoint{12.945472in}{6.264272in}}%
\pgfpathcurveto{\pgfqpoint{12.953708in}{6.264272in}}{\pgfqpoint{12.961608in}{6.267545in}}{\pgfqpoint{12.967432in}{6.273369in}}%
\pgfpathcurveto{\pgfqpoint{12.973256in}{6.279193in}}{\pgfqpoint{12.976528in}{6.287093in}}{\pgfqpoint{12.976528in}{6.295329in}}%
\pgfpathcurveto{\pgfqpoint{12.976528in}{6.303565in}}{\pgfqpoint{12.973256in}{6.311465in}}{\pgfqpoint{12.967432in}{6.317289in}}%
\pgfpathcurveto{\pgfqpoint{12.961608in}{6.323113in}}{\pgfqpoint{12.953708in}{6.326385in}}{\pgfqpoint{12.945472in}{6.326385in}}%
\pgfpathcurveto{\pgfqpoint{12.937236in}{6.326385in}}{\pgfqpoint{12.929336in}{6.323113in}}{\pgfqpoint{12.923512in}{6.317289in}}%
\pgfpathcurveto{\pgfqpoint{12.917688in}{6.311465in}}{\pgfqpoint{12.914415in}{6.303565in}}{\pgfqpoint{12.914415in}{6.295329in}}%
\pgfpathcurveto{\pgfqpoint{12.914415in}{6.287093in}}{\pgfqpoint{12.917688in}{6.279193in}}{\pgfqpoint{12.923512in}{6.273369in}}%
\pgfpathcurveto{\pgfqpoint{12.929336in}{6.267545in}}{\pgfqpoint{12.937236in}{6.264272in}}{\pgfqpoint{12.945472in}{6.264272in}}%
\pgfusepath{stroke}%
\end{pgfscope}%
\begin{pgfscope}%
\pgfpathrectangle{\pgfqpoint{3.788192in}{2.980138in}}{\pgfqpoint{2.914000in}{2.171400in}}%
\pgfusepath{clip}%
\pgfsetbuttcap%
\pgfsetroundjoin%
\pgfsetlinewidth{1.003750pt}%
\definecolor{currentstroke}{rgb}{1.000000,0.000000,0.000000}%
\pgfsetstrokecolor{currentstroke}%
\pgfsetdash{}{0pt}%
\pgfpathmoveto{\pgfqpoint{14.816780in}{6.837016in}}%
\pgfpathcurveto{\pgfqpoint{14.825016in}{6.837016in}}{\pgfqpoint{14.832916in}{6.840288in}}{\pgfqpoint{14.838740in}{6.846112in}}%
\pgfpathcurveto{\pgfqpoint{14.844564in}{6.851936in}}{\pgfqpoint{14.847836in}{6.859836in}}{\pgfqpoint{14.847836in}{6.868073in}}%
\pgfpathcurveto{\pgfqpoint{14.847836in}{6.876309in}}{\pgfqpoint{14.844564in}{6.884209in}}{\pgfqpoint{14.838740in}{6.890033in}}%
\pgfpathcurveto{\pgfqpoint{14.832916in}{6.895857in}}{\pgfqpoint{14.825016in}{6.899129in}}{\pgfqpoint{14.816780in}{6.899129in}}%
\pgfpathcurveto{\pgfqpoint{14.808543in}{6.899129in}}{\pgfqpoint{14.800643in}{6.895857in}}{\pgfqpoint{14.794819in}{6.890033in}}%
\pgfpathcurveto{\pgfqpoint{14.788995in}{6.884209in}}{\pgfqpoint{14.785723in}{6.876309in}}{\pgfqpoint{14.785723in}{6.868073in}}%
\pgfpathcurveto{\pgfqpoint{14.785723in}{6.859836in}}{\pgfqpoint{14.788995in}{6.851936in}}{\pgfqpoint{14.794819in}{6.846112in}}%
\pgfpathcurveto{\pgfqpoint{14.800643in}{6.840288in}}{\pgfqpoint{14.808543in}{6.837016in}}{\pgfqpoint{14.816780in}{6.837016in}}%
\pgfusepath{stroke}%
\end{pgfscope}%
\begin{pgfscope}%
\pgfpathrectangle{\pgfqpoint{3.788192in}{2.980138in}}{\pgfqpoint{2.914000in}{2.171400in}}%
\pgfusepath{clip}%
\pgfsetbuttcap%
\pgfsetroundjoin%
\pgfsetlinewidth{1.003750pt}%
\definecolor{currentstroke}{rgb}{1.000000,0.000000,0.000000}%
\pgfsetstrokecolor{currentstroke}%
\pgfsetdash{}{0pt}%
\pgfpathmoveto{\pgfqpoint{15.389681in}{6.512387in}}%
\pgfpathcurveto{\pgfqpoint{15.397917in}{6.512387in}}{\pgfqpoint{15.405817in}{6.515659in}}{\pgfqpoint{15.411641in}{6.521483in}}%
\pgfpathcurveto{\pgfqpoint{15.417465in}{6.527307in}}{\pgfqpoint{15.420738in}{6.535207in}}{\pgfqpoint{15.420738in}{6.543443in}}%
\pgfpathcurveto{\pgfqpoint{15.420738in}{6.551679in}}{\pgfqpoint{15.417465in}{6.559579in}}{\pgfqpoint{15.411641in}{6.565403in}}%
\pgfpathcurveto{\pgfqpoint{15.405817in}{6.571227in}}{\pgfqpoint{15.397917in}{6.574500in}}{\pgfqpoint{15.389681in}{6.574500in}}%
\pgfpathcurveto{\pgfqpoint{15.381445in}{6.574500in}}{\pgfqpoint{15.373545in}{6.571227in}}{\pgfqpoint{15.367721in}{6.565403in}}%
\pgfpathcurveto{\pgfqpoint{15.361897in}{6.559579in}}{\pgfqpoint{15.358625in}{6.551679in}}{\pgfqpoint{15.358625in}{6.543443in}}%
\pgfpathcurveto{\pgfqpoint{15.358625in}{6.535207in}}{\pgfqpoint{15.361897in}{6.527307in}}{\pgfqpoint{15.367721in}{6.521483in}}%
\pgfpathcurveto{\pgfqpoint{15.373545in}{6.515659in}}{\pgfqpoint{15.381445in}{6.512387in}}{\pgfqpoint{15.389681in}{6.512387in}}%
\pgfusepath{stroke}%
\end{pgfscope}%
\begin{pgfscope}%
\pgfpathrectangle{\pgfqpoint{3.788192in}{2.980138in}}{\pgfqpoint{2.914000in}{2.171400in}}%
\pgfusepath{clip}%
\pgfsetbuttcap%
\pgfsetroundjoin%
\pgfsetlinewidth{1.003750pt}%
\definecolor{currentstroke}{rgb}{1.000000,0.000000,0.000000}%
\pgfsetstrokecolor{currentstroke}%
\pgfsetdash{}{0pt}%
\pgfpathmoveto{\pgfqpoint{12.707501in}{7.095383in}}%
\pgfpathcurveto{\pgfqpoint{12.715737in}{7.095383in}}{\pgfqpoint{12.723637in}{7.098656in}}{\pgfqpoint{12.729461in}{7.104479in}}%
\pgfpathcurveto{\pgfqpoint{12.735285in}{7.110303in}}{\pgfqpoint{12.738557in}{7.118203in}}{\pgfqpoint{12.738557in}{7.126440in}}%
\pgfpathcurveto{\pgfqpoint{12.738557in}{7.134676in}}{\pgfqpoint{12.735285in}{7.142576in}}{\pgfqpoint{12.729461in}{7.148400in}}%
\pgfpathcurveto{\pgfqpoint{12.723637in}{7.154224in}}{\pgfqpoint{12.715737in}{7.157496in}}{\pgfqpoint{12.707501in}{7.157496in}}%
\pgfpathcurveto{\pgfqpoint{12.699265in}{7.157496in}}{\pgfqpoint{12.691365in}{7.154224in}}{\pgfqpoint{12.685541in}{7.148400in}}%
\pgfpathcurveto{\pgfqpoint{12.679717in}{7.142576in}}{\pgfqpoint{12.676444in}{7.134676in}}{\pgfqpoint{12.676444in}{7.126440in}}%
\pgfpathcurveto{\pgfqpoint{12.676444in}{7.118203in}}{\pgfqpoint{12.679717in}{7.110303in}}{\pgfqpoint{12.685541in}{7.104479in}}%
\pgfpathcurveto{\pgfqpoint{12.691365in}{7.098656in}}{\pgfqpoint{12.699265in}{7.095383in}}{\pgfqpoint{12.707501in}{7.095383in}}%
\pgfusepath{stroke}%
\end{pgfscope}%
\begin{pgfscope}%
\pgfpathrectangle{\pgfqpoint{3.788192in}{2.980138in}}{\pgfqpoint{2.914000in}{2.171400in}}%
\pgfusepath{clip}%
\pgfsetbuttcap%
\pgfsetroundjoin%
\pgfsetlinewidth{1.003750pt}%
\definecolor{currentstroke}{rgb}{1.000000,0.000000,0.000000}%
\pgfsetstrokecolor{currentstroke}%
\pgfsetdash{}{0pt}%
\pgfpathmoveto{\pgfqpoint{14.362942in}{5.990566in}}%
\pgfpathcurveto{\pgfqpoint{14.371178in}{5.990566in}}{\pgfqpoint{14.379078in}{5.993838in}}{\pgfqpoint{14.384902in}{5.999662in}}%
\pgfpathcurveto{\pgfqpoint{14.390726in}{6.005486in}}{\pgfqpoint{14.393998in}{6.013386in}}{\pgfqpoint{14.393998in}{6.021622in}}%
\pgfpathcurveto{\pgfqpoint{14.393998in}{6.029859in}}{\pgfqpoint{14.390726in}{6.037759in}}{\pgfqpoint{14.384902in}{6.043583in}}%
\pgfpathcurveto{\pgfqpoint{14.379078in}{6.049407in}}{\pgfqpoint{14.371178in}{6.052679in}}{\pgfqpoint{14.362942in}{6.052679in}}%
\pgfpathcurveto{\pgfqpoint{14.354705in}{6.052679in}}{\pgfqpoint{14.346805in}{6.049407in}}{\pgfqpoint{14.340982in}{6.043583in}}%
\pgfpathcurveto{\pgfqpoint{14.335158in}{6.037759in}}{\pgfqpoint{14.331885in}{6.029859in}}{\pgfqpoint{14.331885in}{6.021622in}}%
\pgfpathcurveto{\pgfqpoint{14.331885in}{6.013386in}}{\pgfqpoint{14.335158in}{6.005486in}}{\pgfqpoint{14.340982in}{5.999662in}}%
\pgfpathcurveto{\pgfqpoint{14.346805in}{5.993838in}}{\pgfqpoint{14.354705in}{5.990566in}}{\pgfqpoint{14.362942in}{5.990566in}}%
\pgfusepath{stroke}%
\end{pgfscope}%
\begin{pgfscope}%
\pgfpathrectangle{\pgfqpoint{3.788192in}{2.980138in}}{\pgfqpoint{2.914000in}{2.171400in}}%
\pgfusepath{clip}%
\pgfsetbuttcap%
\pgfsetroundjoin%
\pgfsetlinewidth{1.003750pt}%
\definecolor{currentstroke}{rgb}{1.000000,0.000000,0.000000}%
\pgfsetstrokecolor{currentstroke}%
\pgfsetdash{}{0pt}%
\pgfpathmoveto{\pgfqpoint{14.985061in}{5.541293in}}%
\pgfpathcurveto{\pgfqpoint{14.993297in}{5.541293in}}{\pgfqpoint{15.001197in}{5.544566in}}{\pgfqpoint{15.007021in}{5.550390in}}%
\pgfpathcurveto{\pgfqpoint{15.012845in}{5.556213in}}{\pgfqpoint{15.016118in}{5.564113in}}{\pgfqpoint{15.016118in}{5.572350in}}%
\pgfpathcurveto{\pgfqpoint{15.016118in}{5.580586in}}{\pgfqpoint{15.012845in}{5.588486in}}{\pgfqpoint{15.007021in}{5.594310in}}%
\pgfpathcurveto{\pgfqpoint{15.001197in}{5.600134in}}{\pgfqpoint{14.993297in}{5.603406in}}{\pgfqpoint{14.985061in}{5.603406in}}%
\pgfpathcurveto{\pgfqpoint{14.976825in}{5.603406in}}{\pgfqpoint{14.968925in}{5.600134in}}{\pgfqpoint{14.963101in}{5.594310in}}%
\pgfpathcurveto{\pgfqpoint{14.957277in}{5.588486in}}{\pgfqpoint{14.954005in}{5.580586in}}{\pgfqpoint{14.954005in}{5.572350in}}%
\pgfpathcurveto{\pgfqpoint{14.954005in}{5.564113in}}{\pgfqpoint{14.957277in}{5.556213in}}{\pgfqpoint{14.963101in}{5.550390in}}%
\pgfpathcurveto{\pgfqpoint{14.968925in}{5.544566in}}{\pgfqpoint{14.976825in}{5.541293in}}{\pgfqpoint{14.985061in}{5.541293in}}%
\pgfusepath{stroke}%
\end{pgfscope}%
\begin{pgfscope}%
\pgfpathrectangle{\pgfqpoint{3.788192in}{2.980138in}}{\pgfqpoint{2.914000in}{2.171400in}}%
\pgfusepath{clip}%
\pgfsetbuttcap%
\pgfsetroundjoin%
\pgfsetlinewidth{1.003750pt}%
\definecolor{currentstroke}{rgb}{1.000000,0.000000,0.000000}%
\pgfsetstrokecolor{currentstroke}%
\pgfsetdash{}{0pt}%
\pgfpathmoveto{\pgfqpoint{14.968190in}{5.733414in}}%
\pgfpathcurveto{\pgfqpoint{14.976426in}{5.733414in}}{\pgfqpoint{14.984326in}{5.736686in}}{\pgfqpoint{14.990150in}{5.742510in}}%
\pgfpathcurveto{\pgfqpoint{14.995974in}{5.748334in}}{\pgfqpoint{14.999246in}{5.756234in}}{\pgfqpoint{14.999246in}{5.764470in}}%
\pgfpathcurveto{\pgfqpoint{14.999246in}{5.772706in}}{\pgfqpoint{14.995974in}{5.780606in}}{\pgfqpoint{14.990150in}{5.786430in}}%
\pgfpathcurveto{\pgfqpoint{14.984326in}{5.792254in}}{\pgfqpoint{14.976426in}{5.795527in}}{\pgfqpoint{14.968190in}{5.795527in}}%
\pgfpathcurveto{\pgfqpoint{14.959954in}{5.795527in}}{\pgfqpoint{14.952053in}{5.792254in}}{\pgfqpoint{14.946230in}{5.786430in}}%
\pgfpathcurveto{\pgfqpoint{14.940406in}{5.780606in}}{\pgfqpoint{14.937133in}{5.772706in}}{\pgfqpoint{14.937133in}{5.764470in}}%
\pgfpathcurveto{\pgfqpoint{14.937133in}{5.756234in}}{\pgfqpoint{14.940406in}{5.748334in}}{\pgfqpoint{14.946230in}{5.742510in}}%
\pgfpathcurveto{\pgfqpoint{14.952053in}{5.736686in}}{\pgfqpoint{14.959954in}{5.733414in}}{\pgfqpoint{14.968190in}{5.733414in}}%
\pgfusepath{stroke}%
\end{pgfscope}%
\begin{pgfscope}%
\pgfpathrectangle{\pgfqpoint{3.788192in}{2.980138in}}{\pgfqpoint{2.914000in}{2.171400in}}%
\pgfusepath{clip}%
\pgfsetbuttcap%
\pgfsetroundjoin%
\pgfsetlinewidth{1.003750pt}%
\definecolor{currentstroke}{rgb}{1.000000,0.000000,0.000000}%
\pgfsetstrokecolor{currentstroke}%
\pgfsetdash{}{0pt}%
\pgfpathmoveto{\pgfqpoint{12.713917in}{7.026179in}}%
\pgfpathcurveto{\pgfqpoint{12.722154in}{7.026179in}}{\pgfqpoint{12.730054in}{7.029451in}}{\pgfqpoint{12.735878in}{7.035275in}}%
\pgfpathcurveto{\pgfqpoint{12.741701in}{7.041099in}}{\pgfqpoint{12.744974in}{7.048999in}}{\pgfqpoint{12.744974in}{7.057235in}}%
\pgfpathcurveto{\pgfqpoint{12.744974in}{7.065472in}}{\pgfqpoint{12.741701in}{7.073372in}}{\pgfqpoint{12.735878in}{7.079196in}}%
\pgfpathcurveto{\pgfqpoint{12.730054in}{7.085019in}}{\pgfqpoint{12.722154in}{7.088292in}}{\pgfqpoint{12.713917in}{7.088292in}}%
\pgfpathcurveto{\pgfqpoint{12.705681in}{7.088292in}}{\pgfqpoint{12.697781in}{7.085019in}}{\pgfqpoint{12.691957in}{7.079196in}}%
\pgfpathcurveto{\pgfqpoint{12.686133in}{7.073372in}}{\pgfqpoint{12.682861in}{7.065472in}}{\pgfqpoint{12.682861in}{7.057235in}}%
\pgfpathcurveto{\pgfqpoint{12.682861in}{7.048999in}}{\pgfqpoint{12.686133in}{7.041099in}}{\pgfqpoint{12.691957in}{7.035275in}}%
\pgfpathcurveto{\pgfqpoint{12.697781in}{7.029451in}}{\pgfqpoint{12.705681in}{7.026179in}}{\pgfqpoint{12.713917in}{7.026179in}}%
\pgfusepath{stroke}%
\end{pgfscope}%
\begin{pgfscope}%
\pgfpathrectangle{\pgfqpoint{3.788192in}{2.980138in}}{\pgfqpoint{2.914000in}{2.171400in}}%
\pgfusepath{clip}%
\pgfsetbuttcap%
\pgfsetroundjoin%
\pgfsetlinewidth{1.003750pt}%
\definecolor{currentstroke}{rgb}{1.000000,0.000000,0.000000}%
\pgfsetstrokecolor{currentstroke}%
\pgfsetdash{}{0pt}%
\pgfpathmoveto{\pgfqpoint{13.558784in}{8.054051in}}%
\pgfpathcurveto{\pgfqpoint{13.567020in}{8.054051in}}{\pgfqpoint{13.574920in}{8.057323in}}{\pgfqpoint{13.580744in}{8.063147in}}%
\pgfpathcurveto{\pgfqpoint{13.586568in}{8.068971in}}{\pgfqpoint{13.589840in}{8.076871in}}{\pgfqpoint{13.589840in}{8.085108in}}%
\pgfpathcurveto{\pgfqpoint{13.589840in}{8.093344in}}{\pgfqpoint{13.586568in}{8.101244in}}{\pgfqpoint{13.580744in}{8.107068in}}%
\pgfpathcurveto{\pgfqpoint{13.574920in}{8.112892in}}{\pgfqpoint{13.567020in}{8.116164in}}{\pgfqpoint{13.558784in}{8.116164in}}%
\pgfpathcurveto{\pgfqpoint{13.550548in}{8.116164in}}{\pgfqpoint{13.542648in}{8.112892in}}{\pgfqpoint{13.536824in}{8.107068in}}%
\pgfpathcurveto{\pgfqpoint{13.531000in}{8.101244in}}{\pgfqpoint{13.527727in}{8.093344in}}{\pgfqpoint{13.527727in}{8.085108in}}%
\pgfpathcurveto{\pgfqpoint{13.527727in}{8.076871in}}{\pgfqpoint{13.531000in}{8.068971in}}{\pgfqpoint{13.536824in}{8.063147in}}%
\pgfpathcurveto{\pgfqpoint{13.542648in}{8.057323in}}{\pgfqpoint{13.550548in}{8.054051in}}{\pgfqpoint{13.558784in}{8.054051in}}%
\pgfusepath{stroke}%
\end{pgfscope}%
\begin{pgfscope}%
\pgfpathrectangle{\pgfqpoint{3.788192in}{2.980138in}}{\pgfqpoint{2.914000in}{2.171400in}}%
\pgfusepath{clip}%
\pgfsetbuttcap%
\pgfsetroundjoin%
\pgfsetlinewidth{1.003750pt}%
\definecolor{currentstroke}{rgb}{1.000000,0.000000,0.000000}%
\pgfsetstrokecolor{currentstroke}%
\pgfsetdash{}{0pt}%
\pgfpathmoveto{\pgfqpoint{13.021061in}{8.119261in}}%
\pgfpathcurveto{\pgfqpoint{13.029297in}{8.119261in}}{\pgfqpoint{13.037197in}{8.122533in}}{\pgfqpoint{13.043021in}{8.128357in}}%
\pgfpathcurveto{\pgfqpoint{13.048845in}{8.134181in}}{\pgfqpoint{13.052117in}{8.142081in}}{\pgfqpoint{13.052117in}{8.150317in}}%
\pgfpathcurveto{\pgfqpoint{13.052117in}{8.158554in}}{\pgfqpoint{13.048845in}{8.166454in}}{\pgfqpoint{13.043021in}{8.172278in}}%
\pgfpathcurveto{\pgfqpoint{13.037197in}{8.178102in}}{\pgfqpoint{13.029297in}{8.181374in}}{\pgfqpoint{13.021061in}{8.181374in}}%
\pgfpathcurveto{\pgfqpoint{13.012825in}{8.181374in}}{\pgfqpoint{13.004925in}{8.178102in}}{\pgfqpoint{12.999101in}{8.172278in}}%
\pgfpathcurveto{\pgfqpoint{12.993277in}{8.166454in}}{\pgfqpoint{12.990004in}{8.158554in}}{\pgfqpoint{12.990004in}{8.150317in}}%
\pgfpathcurveto{\pgfqpoint{12.990004in}{8.142081in}}{\pgfqpoint{12.993277in}{8.134181in}}{\pgfqpoint{12.999101in}{8.128357in}}%
\pgfpathcurveto{\pgfqpoint{13.004925in}{8.122533in}}{\pgfqpoint{13.012825in}{8.119261in}}{\pgfqpoint{13.021061in}{8.119261in}}%
\pgfusepath{stroke}%
\end{pgfscope}%
\begin{pgfscope}%
\pgfpathrectangle{\pgfqpoint{3.788192in}{2.980138in}}{\pgfqpoint{2.914000in}{2.171400in}}%
\pgfusepath{clip}%
\pgfsetbuttcap%
\pgfsetroundjoin%
\pgfsetlinewidth{1.003750pt}%
\definecolor{currentstroke}{rgb}{1.000000,0.000000,0.000000}%
\pgfsetstrokecolor{currentstroke}%
\pgfsetdash{}{0pt}%
\pgfpathmoveto{\pgfqpoint{12.649080in}{7.078889in}}%
\pgfpathcurveto{\pgfqpoint{12.657316in}{7.078889in}}{\pgfqpoint{12.665216in}{7.082161in}}{\pgfqpoint{12.671040in}{7.087985in}}%
\pgfpathcurveto{\pgfqpoint{12.676864in}{7.093809in}}{\pgfqpoint{12.680136in}{7.101709in}}{\pgfqpoint{12.680136in}{7.109946in}}%
\pgfpathcurveto{\pgfqpoint{12.680136in}{7.118182in}}{\pgfqpoint{12.676864in}{7.126082in}}{\pgfqpoint{12.671040in}{7.131906in}}%
\pgfpathcurveto{\pgfqpoint{12.665216in}{7.137730in}}{\pgfqpoint{12.657316in}{7.141002in}}{\pgfqpoint{12.649080in}{7.141002in}}%
\pgfpathcurveto{\pgfqpoint{12.640843in}{7.141002in}}{\pgfqpoint{12.632943in}{7.137730in}}{\pgfqpoint{12.627119in}{7.131906in}}%
\pgfpathcurveto{\pgfqpoint{12.621295in}{7.126082in}}{\pgfqpoint{12.618023in}{7.118182in}}{\pgfqpoint{12.618023in}{7.109946in}}%
\pgfpathcurveto{\pgfqpoint{12.618023in}{7.101709in}}{\pgfqpoint{12.621295in}{7.093809in}}{\pgfqpoint{12.627119in}{7.087985in}}%
\pgfpathcurveto{\pgfqpoint{12.632943in}{7.082161in}}{\pgfqpoint{12.640843in}{7.078889in}}{\pgfqpoint{12.649080in}{7.078889in}}%
\pgfusepath{stroke}%
\end{pgfscope}%
\begin{pgfscope}%
\pgfpathrectangle{\pgfqpoint{3.788192in}{2.980138in}}{\pgfqpoint{2.914000in}{2.171400in}}%
\pgfusepath{clip}%
\pgfsetbuttcap%
\pgfsetroundjoin%
\pgfsetlinewidth{1.003750pt}%
\definecolor{currentstroke}{rgb}{1.000000,0.000000,0.000000}%
\pgfsetstrokecolor{currentstroke}%
\pgfsetdash{}{0pt}%
\pgfpathmoveto{\pgfqpoint{4.530215in}{4.056056in}}%
\pgfpathcurveto{\pgfqpoint{4.538452in}{4.056056in}}{\pgfqpoint{4.546352in}{4.059328in}}{\pgfqpoint{4.552176in}{4.065152in}}%
\pgfpathcurveto{\pgfqpoint{4.557999in}{4.070976in}}{\pgfqpoint{4.561272in}{4.078876in}}{\pgfqpoint{4.561272in}{4.087112in}}%
\pgfpathcurveto{\pgfqpoint{4.561272in}{4.095349in}}{\pgfqpoint{4.557999in}{4.103249in}}{\pgfqpoint{4.552176in}{4.109073in}}%
\pgfpathcurveto{\pgfqpoint{4.546352in}{4.114897in}}{\pgfqpoint{4.538452in}{4.118169in}}{\pgfqpoint{4.530215in}{4.118169in}}%
\pgfpathcurveto{\pgfqpoint{4.521979in}{4.118169in}}{\pgfqpoint{4.514079in}{4.114897in}}{\pgfqpoint{4.508255in}{4.109073in}}%
\pgfpathcurveto{\pgfqpoint{4.502431in}{4.103249in}}{\pgfqpoint{4.499159in}{4.095349in}}{\pgfqpoint{4.499159in}{4.087112in}}%
\pgfpathcurveto{\pgfqpoint{4.499159in}{4.078876in}}{\pgfqpoint{4.502431in}{4.070976in}}{\pgfqpoint{4.508255in}{4.065152in}}%
\pgfpathcurveto{\pgfqpoint{4.514079in}{4.059328in}}{\pgfqpoint{4.521979in}{4.056056in}}{\pgfqpoint{4.530215in}{4.056056in}}%
\pgfpathlineto{\pgfqpoint{4.530215in}{4.056056in}}%
\pgfpathclose%
\pgfusepath{stroke}%
\end{pgfscope}%
\begin{pgfscope}%
\pgfpathrectangle{\pgfqpoint{3.788192in}{2.980138in}}{\pgfqpoint{2.914000in}{2.171400in}}%
\pgfusepath{clip}%
\pgfsetbuttcap%
\pgfsetmiterjoin%
\definecolor{currentfill}{rgb}{0.839216,0.152941,0.156863}%
\pgfsetfillcolor{currentfill}%
\pgfsetfillopacity{0.200000}%
\pgfsetlinewidth{1.003750pt}%
\definecolor{currentstroke}{rgb}{0.839216,0.152941,0.156863}%
\pgfsetstrokecolor{currentstroke}%
\pgfsetstrokeopacity{0.200000}%
\pgfsetdash{}{0pt}%
\pgfpathmoveto{\pgfqpoint{4.530215in}{2.980138in}}%
\pgfpathlineto{\pgfqpoint{24.162152in}{2.980138in}}%
\pgfpathlineto{\pgfqpoint{24.162152in}{5.151538in}}%
\pgfpathlineto{\pgfqpoint{4.530215in}{5.151538in}}%
\pgfpathlineto{\pgfqpoint{4.530215in}{2.980138in}}%
\pgfpathclose%
\pgfusepath{stroke,fill}%
\end{pgfscope}%
\begin{pgfscope}%
\pgfsetbuttcap%
\pgfsetmiterjoin%
\definecolor{currentfill}{rgb}{0.839216,0.152941,0.156863}%
\pgfsetfillcolor{currentfill}%
\pgfsetfillopacity{0.200000}%
\pgfsetlinewidth{1.003750pt}%
\definecolor{currentstroke}{rgb}{0.839216,0.152941,0.156863}%
\pgfsetstrokecolor{currentstroke}%
\pgfsetstrokeopacity{0.200000}%
\pgfsetdash{}{0pt}%
\pgfpathrectangle{\pgfqpoint{3.788192in}{2.980138in}}{\pgfqpoint{2.914000in}{2.171400in}}%
\pgfusepath{clip}%
\pgfpathmoveto{\pgfqpoint{4.530215in}{2.980138in}}%
\pgfpathlineto{\pgfqpoint{24.162152in}{2.980138in}}%
\pgfpathlineto{\pgfqpoint{24.162152in}{5.151538in}}%
\pgfpathlineto{\pgfqpoint{4.530215in}{5.151538in}}%
\pgfpathlineto{\pgfqpoint{4.530215in}{2.980138in}}%
\pgfpathclose%
\pgfusepath{clip}%
\pgfsys@defobject{currentpattern}{\pgfqpoint{0in}{0in}}{\pgfqpoint{1in}{1in}}{%
\begin{pgfscope}%
\pgfpathrectangle{\pgfqpoint{0in}{0in}}{\pgfqpoint{1in}{1in}}%
\pgfusepath{clip}%
\pgfpathmoveto{\pgfqpoint{-0.500000in}{0.500000in}}%
\pgfpathlineto{\pgfqpoint{0.500000in}{1.500000in}}%
\pgfpathmoveto{\pgfqpoint{-0.333333in}{0.333333in}}%
\pgfpathlineto{\pgfqpoint{0.666667in}{1.333333in}}%
\pgfpathmoveto{\pgfqpoint{-0.166667in}{0.166667in}}%
\pgfpathlineto{\pgfqpoint{0.833333in}{1.166667in}}%
\pgfpathmoveto{\pgfqpoint{0.000000in}{0.000000in}}%
\pgfpathlineto{\pgfqpoint{1.000000in}{1.000000in}}%
\pgfpathmoveto{\pgfqpoint{0.166667in}{-0.166667in}}%
\pgfpathlineto{\pgfqpoint{1.166667in}{0.833333in}}%
\pgfpathmoveto{\pgfqpoint{0.333333in}{-0.333333in}}%
\pgfpathlineto{\pgfqpoint{1.333333in}{0.666667in}}%
\pgfpathmoveto{\pgfqpoint{0.500000in}{-0.500000in}}%
\pgfpathlineto{\pgfqpoint{1.500000in}{0.500000in}}%
\pgfusepath{stroke}%
\end{pgfscope}%
}%
\pgfsys@transformshift{4.530215in}{2.980138in}%
\pgfsys@useobject{currentpattern}{}%
\pgfsys@transformshift{1in}{0in}%
\pgfsys@useobject{currentpattern}{}%
\pgfsys@transformshift{1in}{0in}%
\pgfsys@useobject{currentpattern}{}%
\pgfsys@transformshift{1in}{0in}%
\pgfsys@useobject{currentpattern}{}%
\pgfsys@transformshift{1in}{0in}%
\pgfsys@useobject{currentpattern}{}%
\pgfsys@transformshift{1in}{0in}%
\pgfsys@useobject{currentpattern}{}%
\pgfsys@transformshift{1in}{0in}%
\pgfsys@useobject{currentpattern}{}%
\pgfsys@transformshift{1in}{0in}%
\pgfsys@useobject{currentpattern}{}%
\pgfsys@transformshift{1in}{0in}%
\pgfsys@useobject{currentpattern}{}%
\pgfsys@transformshift{1in}{0in}%
\pgfsys@useobject{currentpattern}{}%
\pgfsys@transformshift{1in}{0in}%
\pgfsys@useobject{currentpattern}{}%
\pgfsys@transformshift{1in}{0in}%
\pgfsys@useobject{currentpattern}{}%
\pgfsys@transformshift{1in}{0in}%
\pgfsys@useobject{currentpattern}{}%
\pgfsys@transformshift{1in}{0in}%
\pgfsys@useobject{currentpattern}{}%
\pgfsys@transformshift{1in}{0in}%
\pgfsys@useobject{currentpattern}{}%
\pgfsys@transformshift{1in}{0in}%
\pgfsys@useobject{currentpattern}{}%
\pgfsys@transformshift{1in}{0in}%
\pgfsys@useobject{currentpattern}{}%
\pgfsys@transformshift{1in}{0in}%
\pgfsys@useobject{currentpattern}{}%
\pgfsys@transformshift{1in}{0in}%
\pgfsys@useobject{currentpattern}{}%
\pgfsys@transformshift{1in}{0in}%
\pgfsys@useobject{currentpattern}{}%
\pgfsys@transformshift{1in}{0in}%
\pgfsys@transformshift{-20in}{0in}%
\pgfsys@transformshift{0in}{1in}%
\pgfsys@useobject{currentpattern}{}%
\pgfsys@transformshift{1in}{0in}%
\pgfsys@useobject{currentpattern}{}%
\pgfsys@transformshift{1in}{0in}%
\pgfsys@useobject{currentpattern}{}%
\pgfsys@transformshift{1in}{0in}%
\pgfsys@useobject{currentpattern}{}%
\pgfsys@transformshift{1in}{0in}%
\pgfsys@useobject{currentpattern}{}%
\pgfsys@transformshift{1in}{0in}%
\pgfsys@useobject{currentpattern}{}%
\pgfsys@transformshift{1in}{0in}%
\pgfsys@useobject{currentpattern}{}%
\pgfsys@transformshift{1in}{0in}%
\pgfsys@useobject{currentpattern}{}%
\pgfsys@transformshift{1in}{0in}%
\pgfsys@useobject{currentpattern}{}%
\pgfsys@transformshift{1in}{0in}%
\pgfsys@useobject{currentpattern}{}%
\pgfsys@transformshift{1in}{0in}%
\pgfsys@useobject{currentpattern}{}%
\pgfsys@transformshift{1in}{0in}%
\pgfsys@useobject{currentpattern}{}%
\pgfsys@transformshift{1in}{0in}%
\pgfsys@useobject{currentpattern}{}%
\pgfsys@transformshift{1in}{0in}%
\pgfsys@useobject{currentpattern}{}%
\pgfsys@transformshift{1in}{0in}%
\pgfsys@useobject{currentpattern}{}%
\pgfsys@transformshift{1in}{0in}%
\pgfsys@useobject{currentpattern}{}%
\pgfsys@transformshift{1in}{0in}%
\pgfsys@useobject{currentpattern}{}%
\pgfsys@transformshift{1in}{0in}%
\pgfsys@useobject{currentpattern}{}%
\pgfsys@transformshift{1in}{0in}%
\pgfsys@useobject{currentpattern}{}%
\pgfsys@transformshift{1in}{0in}%
\pgfsys@useobject{currentpattern}{}%
\pgfsys@transformshift{1in}{0in}%
\pgfsys@transformshift{-20in}{0in}%
\pgfsys@transformshift{0in}{1in}%
\pgfsys@useobject{currentpattern}{}%
\pgfsys@transformshift{1in}{0in}%
\pgfsys@useobject{currentpattern}{}%
\pgfsys@transformshift{1in}{0in}%
\pgfsys@useobject{currentpattern}{}%
\pgfsys@transformshift{1in}{0in}%
\pgfsys@useobject{currentpattern}{}%
\pgfsys@transformshift{1in}{0in}%
\pgfsys@useobject{currentpattern}{}%
\pgfsys@transformshift{1in}{0in}%
\pgfsys@useobject{currentpattern}{}%
\pgfsys@transformshift{1in}{0in}%
\pgfsys@useobject{currentpattern}{}%
\pgfsys@transformshift{1in}{0in}%
\pgfsys@useobject{currentpattern}{}%
\pgfsys@transformshift{1in}{0in}%
\pgfsys@useobject{currentpattern}{}%
\pgfsys@transformshift{1in}{0in}%
\pgfsys@useobject{currentpattern}{}%
\pgfsys@transformshift{1in}{0in}%
\pgfsys@useobject{currentpattern}{}%
\pgfsys@transformshift{1in}{0in}%
\pgfsys@useobject{currentpattern}{}%
\pgfsys@transformshift{1in}{0in}%
\pgfsys@useobject{currentpattern}{}%
\pgfsys@transformshift{1in}{0in}%
\pgfsys@useobject{currentpattern}{}%
\pgfsys@transformshift{1in}{0in}%
\pgfsys@useobject{currentpattern}{}%
\pgfsys@transformshift{1in}{0in}%
\pgfsys@useobject{currentpattern}{}%
\pgfsys@transformshift{1in}{0in}%
\pgfsys@useobject{currentpattern}{}%
\pgfsys@transformshift{1in}{0in}%
\pgfsys@useobject{currentpattern}{}%
\pgfsys@transformshift{1in}{0in}%
\pgfsys@useobject{currentpattern}{}%
\pgfsys@transformshift{1in}{0in}%
\pgfsys@useobject{currentpattern}{}%
\pgfsys@transformshift{1in}{0in}%
\pgfsys@transformshift{-20in}{0in}%
\pgfsys@transformshift{0in}{1in}%
\end{pgfscope}%
\begin{pgfscope}%
\pgfpathrectangle{\pgfqpoint{3.788192in}{2.980138in}}{\pgfqpoint{2.914000in}{2.171400in}}%
\pgfusepath{clip}%
\pgfsetrectcap%
\pgfsetroundjoin%
\pgfsetlinewidth{0.803000pt}%
\definecolor{currentstroke}{rgb}{0.690196,0.690196,0.690196}%
\pgfsetstrokecolor{currentstroke}%
\pgfsetdash{}{0pt}%
\pgfpathmoveto{\pgfqpoint{4.105109in}{2.980138in}}%
\pgfpathlineto{\pgfqpoint{4.105109in}{5.151538in}}%
\pgfusepath{stroke}%
\end{pgfscope}%
\begin{pgfscope}%
\pgfsetbuttcap%
\pgfsetroundjoin%
\definecolor{currentfill}{rgb}{0.000000,0.000000,0.000000}%
\pgfsetfillcolor{currentfill}%
\pgfsetlinewidth{0.803000pt}%
\definecolor{currentstroke}{rgb}{0.000000,0.000000,0.000000}%
\pgfsetstrokecolor{currentstroke}%
\pgfsetdash{}{0pt}%
\pgfsys@defobject{currentmarker}{\pgfqpoint{0.000000in}{-0.048611in}}{\pgfqpoint{0.000000in}{0.000000in}}{%
\pgfpathmoveto{\pgfqpoint{0.000000in}{0.000000in}}%
\pgfpathlineto{\pgfqpoint{0.000000in}{-0.048611in}}%
\pgfusepath{stroke,fill}%
}%
\begin{pgfscope}%
\pgfsys@transformshift{4.105109in}{2.980138in}%
\pgfsys@useobject{currentmarker}{}%
\end{pgfscope}%
\end{pgfscope}%
\begin{pgfscope}%
\definecolor{textcolor}{rgb}{0.000000,0.000000,0.000000}%
\pgfsetstrokecolor{textcolor}%
\pgfsetfillcolor{textcolor}%
\pgftext[x=4.105109in,y=2.882916in,,top]{\color{textcolor}{\rmfamily\fontsize{14.000000}{16.800000}\selectfont\catcode`\^=\active\def^{\ifmmode\sp\else\^{}\fi}\catcode`\%=\active\def%{\%}$\mathdefault{5280}$}}%
\end{pgfscope}%
\begin{pgfscope}%
\pgfpathrectangle{\pgfqpoint{3.788192in}{2.980138in}}{\pgfqpoint{2.914000in}{2.171400in}}%
\pgfusepath{clip}%
\pgfsetrectcap%
\pgfsetroundjoin%
\pgfsetlinewidth{0.803000pt}%
\definecolor{currentstroke}{rgb}{0.690196,0.690196,0.690196}%
\pgfsetstrokecolor{currentstroke}%
\pgfsetdash{}{0pt}%
\pgfpathmoveto{\pgfqpoint{4.847132in}{2.980138in}}%
\pgfpathlineto{\pgfqpoint{4.847132in}{5.151538in}}%
\pgfusepath{stroke}%
\end{pgfscope}%
\begin{pgfscope}%
\pgfsetbuttcap%
\pgfsetroundjoin%
\definecolor{currentfill}{rgb}{0.000000,0.000000,0.000000}%
\pgfsetfillcolor{currentfill}%
\pgfsetlinewidth{0.803000pt}%
\definecolor{currentstroke}{rgb}{0.000000,0.000000,0.000000}%
\pgfsetstrokecolor{currentstroke}%
\pgfsetdash{}{0pt}%
\pgfsys@defobject{currentmarker}{\pgfqpoint{0.000000in}{-0.048611in}}{\pgfqpoint{0.000000in}{0.000000in}}{%
\pgfpathmoveto{\pgfqpoint{0.000000in}{0.000000in}}%
\pgfpathlineto{\pgfqpoint{0.000000in}{-0.048611in}}%
\pgfusepath{stroke,fill}%
}%
\begin{pgfscope}%
\pgfsys@transformshift{4.847132in}{2.980138in}%
\pgfsys@useobject{currentmarker}{}%
\end{pgfscope}%
\end{pgfscope}%
\begin{pgfscope}%
\definecolor{textcolor}{rgb}{0.000000,0.000000,0.000000}%
\pgfsetstrokecolor{textcolor}%
\pgfsetfillcolor{textcolor}%
\pgftext[x=4.847132in,y=2.882916in,,top]{\color{textcolor}{\rmfamily\fontsize{14.000000}{16.800000}\selectfont\catcode`\^=\active\def^{\ifmmode\sp\else\^{}\fi}\catcode`\%=\active\def%{\%}$\mathdefault{5300}$}}%
\end{pgfscope}%
\begin{pgfscope}%
\pgfpathrectangle{\pgfqpoint{3.788192in}{2.980138in}}{\pgfqpoint{2.914000in}{2.171400in}}%
\pgfusepath{clip}%
\pgfsetrectcap%
\pgfsetroundjoin%
\pgfsetlinewidth{0.803000pt}%
\definecolor{currentstroke}{rgb}{0.690196,0.690196,0.690196}%
\pgfsetstrokecolor{currentstroke}%
\pgfsetdash{}{0pt}%
\pgfpathmoveto{\pgfqpoint{5.589156in}{2.980138in}}%
\pgfpathlineto{\pgfqpoint{5.589156in}{5.151538in}}%
\pgfusepath{stroke}%
\end{pgfscope}%
\begin{pgfscope}%
\pgfsetbuttcap%
\pgfsetroundjoin%
\definecolor{currentfill}{rgb}{0.000000,0.000000,0.000000}%
\pgfsetfillcolor{currentfill}%
\pgfsetlinewidth{0.803000pt}%
\definecolor{currentstroke}{rgb}{0.000000,0.000000,0.000000}%
\pgfsetstrokecolor{currentstroke}%
\pgfsetdash{}{0pt}%
\pgfsys@defobject{currentmarker}{\pgfqpoint{0.000000in}{-0.048611in}}{\pgfqpoint{0.000000in}{0.000000in}}{%
\pgfpathmoveto{\pgfqpoint{0.000000in}{0.000000in}}%
\pgfpathlineto{\pgfqpoint{0.000000in}{-0.048611in}}%
\pgfusepath{stroke,fill}%
}%
\begin{pgfscope}%
\pgfsys@transformshift{5.589156in}{2.980138in}%
\pgfsys@useobject{currentmarker}{}%
\end{pgfscope}%
\end{pgfscope}%
\begin{pgfscope}%
\definecolor{textcolor}{rgb}{0.000000,0.000000,0.000000}%
\pgfsetstrokecolor{textcolor}%
\pgfsetfillcolor{textcolor}%
\pgftext[x=5.589156in,y=2.882916in,,top]{\color{textcolor}{\rmfamily\fontsize{14.000000}{16.800000}\selectfont\catcode`\^=\active\def^{\ifmmode\sp\else\^{}\fi}\catcode`\%=\active\def%{\%}$\mathdefault{5320}$}}%
\end{pgfscope}%
\begin{pgfscope}%
\pgfpathrectangle{\pgfqpoint{3.788192in}{2.980138in}}{\pgfqpoint{2.914000in}{2.171400in}}%
\pgfusepath{clip}%
\pgfsetrectcap%
\pgfsetroundjoin%
\pgfsetlinewidth{0.803000pt}%
\definecolor{currentstroke}{rgb}{0.690196,0.690196,0.690196}%
\pgfsetstrokecolor{currentstroke}%
\pgfsetdash{}{0pt}%
\pgfpathmoveto{\pgfqpoint{6.331180in}{2.980138in}}%
\pgfpathlineto{\pgfqpoint{6.331180in}{5.151538in}}%
\pgfusepath{stroke}%
\end{pgfscope}%
\begin{pgfscope}%
\pgfsetbuttcap%
\pgfsetroundjoin%
\definecolor{currentfill}{rgb}{0.000000,0.000000,0.000000}%
\pgfsetfillcolor{currentfill}%
\pgfsetlinewidth{0.803000pt}%
\definecolor{currentstroke}{rgb}{0.000000,0.000000,0.000000}%
\pgfsetstrokecolor{currentstroke}%
\pgfsetdash{}{0pt}%
\pgfsys@defobject{currentmarker}{\pgfqpoint{0.000000in}{-0.048611in}}{\pgfqpoint{0.000000in}{0.000000in}}{%
\pgfpathmoveto{\pgfqpoint{0.000000in}{0.000000in}}%
\pgfpathlineto{\pgfqpoint{0.000000in}{-0.048611in}}%
\pgfusepath{stroke,fill}%
}%
\begin{pgfscope}%
\pgfsys@transformshift{6.331180in}{2.980138in}%
\pgfsys@useobject{currentmarker}{}%
\end{pgfscope}%
\end{pgfscope}%
\begin{pgfscope}%
\definecolor{textcolor}{rgb}{0.000000,0.000000,0.000000}%
\pgfsetstrokecolor{textcolor}%
\pgfsetfillcolor{textcolor}%
\pgftext[x=6.331180in,y=2.882916in,,top]{\color{textcolor}{\rmfamily\fontsize{14.000000}{16.800000}\selectfont\catcode`\^=\active\def^{\ifmmode\sp\else\^{}\fi}\catcode`\%=\active\def%{\%}$\mathdefault{5340}$}}%
\end{pgfscope}%
\begin{pgfscope}%
\pgfpathrectangle{\pgfqpoint{3.788192in}{2.980138in}}{\pgfqpoint{2.914000in}{2.171400in}}%
\pgfusepath{clip}%
\pgfsetrectcap%
\pgfsetroundjoin%
\pgfsetlinewidth{0.803000pt}%
\definecolor{currentstroke}{rgb}{0.690196,0.690196,0.690196}%
\pgfsetstrokecolor{currentstroke}%
\pgfsetdash{}{0pt}%
\pgfpathmoveto{\pgfqpoint{3.788192in}{3.334947in}}%
\pgfpathlineto{\pgfqpoint{6.702192in}{3.334947in}}%
\pgfusepath{stroke}%
\end{pgfscope}%
\begin{pgfscope}%
\pgfsetbuttcap%
\pgfsetroundjoin%
\definecolor{currentfill}{rgb}{0.000000,0.000000,0.000000}%
\pgfsetfillcolor{currentfill}%
\pgfsetlinewidth{0.803000pt}%
\definecolor{currentstroke}{rgb}{0.000000,0.000000,0.000000}%
\pgfsetstrokecolor{currentstroke}%
\pgfsetdash{}{0pt}%
\pgfsys@defobject{currentmarker}{\pgfqpoint{-0.048611in}{0.000000in}}{\pgfqpoint{-0.000000in}{0.000000in}}{%
\pgfpathmoveto{\pgfqpoint{-0.000000in}{0.000000in}}%
\pgfpathlineto{\pgfqpoint{-0.048611in}{0.000000in}}%
\pgfusepath{stroke,fill}%
}%
\begin{pgfscope}%
\pgfsys@transformshift{3.788192in}{3.334947in}%
\pgfsys@useobject{currentmarker}{}%
\end{pgfscope}%
\end{pgfscope}%
\begin{pgfscope}%
\definecolor{textcolor}{rgb}{0.000000,0.000000,0.000000}%
\pgfsetstrokecolor{textcolor}%
\pgfsetfillcolor{textcolor}%
\pgftext[x=3.495138in, y=3.265502in, left, base]{\color{textcolor}{\rmfamily\fontsize{14.000000}{16.800000}\selectfont\catcode`\^=\active\def^{\ifmmode\sp\else\^{}\fi}\catcode`\%=\active\def%{\%}$\mathdefault{10}$}}%
\end{pgfscope}%
\begin{pgfscope}%
\pgfpathrectangle{\pgfqpoint{3.788192in}{2.980138in}}{\pgfqpoint{2.914000in}{2.171400in}}%
\pgfusepath{clip}%
\pgfsetrectcap%
\pgfsetroundjoin%
\pgfsetlinewidth{0.803000pt}%
\definecolor{currentstroke}{rgb}{0.690196,0.690196,0.690196}%
\pgfsetstrokecolor{currentstroke}%
\pgfsetdash{}{0pt}%
\pgfpathmoveto{\pgfqpoint{3.788192in}{4.044564in}}%
\pgfpathlineto{\pgfqpoint{6.702192in}{4.044564in}}%
\pgfusepath{stroke}%
\end{pgfscope}%
\begin{pgfscope}%
\pgfsetbuttcap%
\pgfsetroundjoin%
\definecolor{currentfill}{rgb}{0.000000,0.000000,0.000000}%
\pgfsetfillcolor{currentfill}%
\pgfsetlinewidth{0.803000pt}%
\definecolor{currentstroke}{rgb}{0.000000,0.000000,0.000000}%
\pgfsetstrokecolor{currentstroke}%
\pgfsetdash{}{0pt}%
\pgfsys@defobject{currentmarker}{\pgfqpoint{-0.048611in}{0.000000in}}{\pgfqpoint{-0.000000in}{0.000000in}}{%
\pgfpathmoveto{\pgfqpoint{-0.000000in}{0.000000in}}%
\pgfpathlineto{\pgfqpoint{-0.048611in}{0.000000in}}%
\pgfusepath{stroke,fill}%
}%
\begin{pgfscope}%
\pgfsys@transformshift{3.788192in}{4.044564in}%
\pgfsys@useobject{currentmarker}{}%
\end{pgfscope}%
\end{pgfscope}%
\begin{pgfscope}%
\definecolor{textcolor}{rgb}{0.000000,0.000000,0.000000}%
\pgfsetstrokecolor{textcolor}%
\pgfsetfillcolor{textcolor}%
\pgftext[x=3.495138in, y=3.975119in, left, base]{\color{textcolor}{\rmfamily\fontsize{14.000000}{16.800000}\selectfont\catcode`\^=\active\def^{\ifmmode\sp\else\^{}\fi}\catcode`\%=\active\def%{\%}$\mathdefault{12}$}}%
\end{pgfscope}%
\begin{pgfscope}%
\pgfpathrectangle{\pgfqpoint{3.788192in}{2.980138in}}{\pgfqpoint{2.914000in}{2.171400in}}%
\pgfusepath{clip}%
\pgfsetrectcap%
\pgfsetroundjoin%
\pgfsetlinewidth{0.803000pt}%
\definecolor{currentstroke}{rgb}{0.690196,0.690196,0.690196}%
\pgfsetstrokecolor{currentstroke}%
\pgfsetdash{}{0pt}%
\pgfpathmoveto{\pgfqpoint{3.788192in}{4.754181in}}%
\pgfpathlineto{\pgfqpoint{6.702192in}{4.754181in}}%
\pgfusepath{stroke}%
\end{pgfscope}%
\begin{pgfscope}%
\pgfsetbuttcap%
\pgfsetroundjoin%
\definecolor{currentfill}{rgb}{0.000000,0.000000,0.000000}%
\pgfsetfillcolor{currentfill}%
\pgfsetlinewidth{0.803000pt}%
\definecolor{currentstroke}{rgb}{0.000000,0.000000,0.000000}%
\pgfsetstrokecolor{currentstroke}%
\pgfsetdash{}{0pt}%
\pgfsys@defobject{currentmarker}{\pgfqpoint{-0.048611in}{0.000000in}}{\pgfqpoint{-0.000000in}{0.000000in}}{%
\pgfpathmoveto{\pgfqpoint{-0.000000in}{0.000000in}}%
\pgfpathlineto{\pgfqpoint{-0.048611in}{0.000000in}}%
\pgfusepath{stroke,fill}%
}%
\begin{pgfscope}%
\pgfsys@transformshift{3.788192in}{4.754181in}%
\pgfsys@useobject{currentmarker}{}%
\end{pgfscope}%
\end{pgfscope}%
\begin{pgfscope}%
\definecolor{textcolor}{rgb}{0.000000,0.000000,0.000000}%
\pgfsetstrokecolor{textcolor}%
\pgfsetfillcolor{textcolor}%
\pgftext[x=3.495138in, y=4.684736in, left, base]{\color{textcolor}{\rmfamily\fontsize{14.000000}{16.800000}\selectfont\catcode`\^=\active\def^{\ifmmode\sp\else\^{}\fi}\catcode`\%=\active\def%{\%}$\mathdefault{14}$}}%
\end{pgfscope}%
\begin{pgfscope}%
\pgfpathrectangle{\pgfqpoint{3.788192in}{2.980138in}}{\pgfqpoint{2.914000in}{2.171400in}}%
\pgfusepath{clip}%
\pgfsetrectcap%
\pgfsetroundjoin%
\pgfsetlinewidth{1.505625pt}%
\definecolor{currentstroke}{rgb}{0.000000,0.000000,1.000000}%
\pgfsetstrokecolor{currentstroke}%
\pgfsetdash{}{0pt}%
\pgfpathmoveto{\pgfqpoint{5.478622in}{4.808529in}}%
\pgfpathlineto{\pgfqpoint{5.701947in}{2.977638in}}%
\pgfusepath{stroke}%
\end{pgfscope}%
\begin{pgfscope}%
\pgfpathrectangle{\pgfqpoint{3.788192in}{2.980138in}}{\pgfqpoint{2.914000in}{2.171400in}}%
\pgfusepath{clip}%
\pgfsetbuttcap%
\pgfsetroundjoin%
\definecolor{currentfill}{rgb}{0.000000,0.000000,1.000000}%
\pgfsetfillcolor{currentfill}%
\pgfsetlinewidth{1.003750pt}%
\definecolor{currentstroke}{rgb}{0.000000,0.000000,1.000000}%
\pgfsetstrokecolor{currentstroke}%
\pgfsetdash{}{0pt}%
\pgfsys@defobject{currentmarker}{\pgfqpoint{-0.041667in}{-0.041667in}}{\pgfqpoint{0.041667in}{0.041667in}}{%
\pgfpathmoveto{\pgfqpoint{0.000000in}{-0.041667in}}%
\pgfpathcurveto{\pgfqpoint{0.011050in}{-0.041667in}}{\pgfqpoint{0.021649in}{-0.037276in}}{\pgfqpoint{0.029463in}{-0.029463in}}%
\pgfpathcurveto{\pgfqpoint{0.037276in}{-0.021649in}}{\pgfqpoint{0.041667in}{-0.011050in}}{\pgfqpoint{0.041667in}{0.000000in}}%
\pgfpathcurveto{\pgfqpoint{0.041667in}{0.011050in}}{\pgfqpoint{0.037276in}{0.021649in}}{\pgfqpoint{0.029463in}{0.029463in}}%
\pgfpathcurveto{\pgfqpoint{0.021649in}{0.037276in}}{\pgfqpoint{0.011050in}{0.041667in}}{\pgfqpoint{0.000000in}{0.041667in}}%
\pgfpathcurveto{\pgfqpoint{-0.011050in}{0.041667in}}{\pgfqpoint{-0.021649in}{0.037276in}}{\pgfqpoint{-0.029463in}{0.029463in}}%
\pgfpathcurveto{\pgfqpoint{-0.037276in}{0.021649in}}{\pgfqpoint{-0.041667in}{0.011050in}}{\pgfqpoint{-0.041667in}{0.000000in}}%
\pgfpathcurveto{\pgfqpoint{-0.041667in}{-0.011050in}}{\pgfqpoint{-0.037276in}{-0.021649in}}{\pgfqpoint{-0.029463in}{-0.029463in}}%
\pgfpathcurveto{\pgfqpoint{-0.021649in}{-0.037276in}}{\pgfqpoint{-0.011050in}{-0.041667in}}{\pgfqpoint{0.000000in}{-0.041667in}}%
\pgfpathlineto{\pgfqpoint{0.000000in}{-0.041667in}}%
\pgfpathclose%
\pgfusepath{stroke,fill}%
}%
\begin{pgfscope}%
\pgfsys@transformshift{5.478622in}{4.808529in}%
\pgfsys@useobject{currentmarker}{}%
\end{pgfscope}%
\begin{pgfscope}%
\pgfsys@transformshift{5.786444in}{2.284897in}%
\pgfsys@useobject{currentmarker}{}%
\end{pgfscope}%
\begin{pgfscope}%
\pgfsys@transformshift{5.955602in}{1.807070in}%
\pgfsys@useobject{currentmarker}{}%
\end{pgfscope}%
\begin{pgfscope}%
\pgfsys@transformshift{6.072257in}{1.530809in}%
\pgfsys@useobject{currentmarker}{}%
\end{pgfscope}%
\begin{pgfscope}%
\pgfsys@transformshift{6.134561in}{1.493263in}%
\pgfsys@useobject{currentmarker}{}%
\end{pgfscope}%
\begin{pgfscope}%
\pgfsys@transformshift{6.204334in}{1.296852in}%
\pgfsys@useobject{currentmarker}{}%
\end{pgfscope}%
\begin{pgfscope}%
\pgfsys@transformshift{6.281957in}{1.238734in}%
\pgfsys@useobject{currentmarker}{}%
\end{pgfscope}%
\begin{pgfscope}%
\pgfsys@transformshift{6.389240in}{1.234341in}%
\pgfsys@useobject{currentmarker}{}%
\end{pgfscope}%
\begin{pgfscope}%
\pgfsys@transformshift{6.427201in}{1.108377in}%
\pgfsys@useobject{currentmarker}{}%
\end{pgfscope}%
\begin{pgfscope}%
\pgfsys@transformshift{6.519285in}{1.077759in}%
\pgfsys@useobject{currentmarker}{}%
\end{pgfscope}%
\begin{pgfscope}%
\pgfsys@transformshift{6.689124in}{1.072073in}%
\pgfsys@useobject{currentmarker}{}%
\end{pgfscope}%
\begin{pgfscope}%
\pgfsys@transformshift{6.747190in}{1.024269in}%
\pgfsys@useobject{currentmarker}{}%
\end{pgfscope}%
\begin{pgfscope}%
\pgfsys@transformshift{6.806750in}{1.015907in}%
\pgfsys@useobject{currentmarker}{}%
\end{pgfscope}%
\begin{pgfscope}%
\pgfsys@transformshift{6.931730in}{0.991431in}%
\pgfsys@useobject{currentmarker}{}%
\end{pgfscope}%
\begin{pgfscope}%
\pgfsys@transformshift{7.035734in}{0.970288in}%
\pgfsys@useobject{currentmarker}{}%
\end{pgfscope}%
\begin{pgfscope}%
\pgfsys@transformshift{7.042663in}{0.960624in}%
\pgfsys@useobject{currentmarker}{}%
\end{pgfscope}%
\begin{pgfscope}%
\pgfsys@transformshift{7.110896in}{0.945926in}%
\pgfsys@useobject{currentmarker}{}%
\end{pgfscope}%
\begin{pgfscope}%
\pgfsys@transformshift{7.385541in}{0.918874in}%
\pgfsys@useobject{currentmarker}{}%
\end{pgfscope}%
\begin{pgfscope}%
\pgfsys@transformshift{7.677622in}{0.881182in}%
\pgfsys@useobject{currentmarker}{}%
\end{pgfscope}%
\begin{pgfscope}%
\pgfsys@transformshift{8.074033in}{0.854291in}%
\pgfsys@useobject{currentmarker}{}%
\end{pgfscope}%
\begin{pgfscope}%
\pgfsys@transformshift{8.437037in}{0.845259in}%
\pgfsys@useobject{currentmarker}{}%
\end{pgfscope}%
\begin{pgfscope}%
\pgfsys@transformshift{8.451833in}{0.833333in}%
\pgfsys@useobject{currentmarker}{}%
\end{pgfscope}%
\begin{pgfscope}%
\pgfsys@transformshift{8.473270in}{0.812531in}%
\pgfsys@useobject{currentmarker}{}%
\end{pgfscope}%
\begin{pgfscope}%
\pgfsys@transformshift{8.530759in}{0.805118in}%
\pgfsys@useobject{currentmarker}{}%
\end{pgfscope}%
\begin{pgfscope}%
\pgfsys@transformshift{8.667946in}{0.801766in}%
\pgfsys@useobject{currentmarker}{}%
\end{pgfscope}%
\begin{pgfscope}%
\pgfsys@transformshift{8.761129in}{0.792896in}%
\pgfsys@useobject{currentmarker}{}%
\end{pgfscope}%
\begin{pgfscope}%
\pgfsys@transformshift{9.175032in}{0.779118in}%
\pgfsys@useobject{currentmarker}{}%
\end{pgfscope}%
\begin{pgfscope}%
\pgfsys@transformshift{9.229242in}{0.771806in}%
\pgfsys@useobject{currentmarker}{}%
\end{pgfscope}%
\begin{pgfscope}%
\pgfsys@transformshift{9.230836in}{0.763945in}%
\pgfsys@useobject{currentmarker}{}%
\end{pgfscope}%
\begin{pgfscope}%
\pgfsys@transformshift{9.319803in}{0.762647in}%
\pgfsys@useobject{currentmarker}{}%
\end{pgfscope}%
\begin{pgfscope}%
\pgfsys@transformshift{9.530028in}{0.749444in}%
\pgfsys@useobject{currentmarker}{}%
\end{pgfscope}%
\begin{pgfscope}%
\pgfsys@transformshift{9.629502in}{0.748101in}%
\pgfsys@useobject{currentmarker}{}%
\end{pgfscope}%
\begin{pgfscope}%
\pgfsys@transformshift{9.644888in}{0.743176in}%
\pgfsys@useobject{currentmarker}{}%
\end{pgfscope}%
\begin{pgfscope}%
\pgfsys@transformshift{10.473184in}{0.730520in}%
\pgfsys@useobject{currentmarker}{}%
\end{pgfscope}%
\begin{pgfscope}%
\pgfsys@transformshift{10.538033in}{0.705808in}%
\pgfsys@useobject{currentmarker}{}%
\end{pgfscope}%
\begin{pgfscope}%
\pgfsys@transformshift{10.574876in}{0.703903in}%
\pgfsys@useobject{currentmarker}{}%
\end{pgfscope}%
\begin{pgfscope}%
\pgfsys@transformshift{10.706030in}{0.700593in}%
\pgfsys@useobject{currentmarker}{}%
\end{pgfscope}%
\begin{pgfscope}%
\pgfsys@transformshift{10.965602in}{0.687998in}%
\pgfsys@useobject{currentmarker}{}%
\end{pgfscope}%
\begin{pgfscope}%
\pgfsys@transformshift{11.060169in}{0.684789in}%
\pgfsys@useobject{currentmarker}{}%
\end{pgfscope}%
\begin{pgfscope}%
\pgfsys@transformshift{11.147519in}{0.680781in}%
\pgfsys@useobject{currentmarker}{}%
\end{pgfscope}%
\begin{pgfscope}%
\pgfsys@transformshift{11.370645in}{0.672484in}%
\pgfsys@useobject{currentmarker}{}%
\end{pgfscope}%
\begin{pgfscope}%
\pgfsys@transformshift{11.728159in}{0.668514in}%
\pgfsys@useobject{currentmarker}{}%
\end{pgfscope}%
\begin{pgfscope}%
\pgfsys@transformshift{12.137942in}{0.666709in}%
\pgfsys@useobject{currentmarker}{}%
\end{pgfscope}%
\begin{pgfscope}%
\pgfsys@transformshift{12.434130in}{0.666468in}%
\pgfsys@useobject{currentmarker}{}%
\end{pgfscope}%
\begin{pgfscope}%
\pgfsys@transformshift{12.804602in}{0.666397in}%
\pgfsys@useobject{currentmarker}{}%
\end{pgfscope}%
\begin{pgfscope}%
\pgfsys@transformshift{14.562739in}{0.661643in}%
\pgfsys@useobject{currentmarker}{}%
\end{pgfscope}%
\begin{pgfscope}%
\pgfsys@transformshift{15.028873in}{0.661022in}%
\pgfsys@useobject{currentmarker}{}%
\end{pgfscope}%
\begin{pgfscope}%
\pgfsys@transformshift{15.695083in}{0.659204in}%
\pgfsys@useobject{currentmarker}{}%
\end{pgfscope}%
\begin{pgfscope}%
\pgfsys@transformshift{16.663977in}{0.658612in}%
\pgfsys@useobject{currentmarker}{}%
\end{pgfscope}%
\begin{pgfscope}%
\pgfsys@transformshift{17.188373in}{0.655818in}%
\pgfsys@useobject{currentmarker}{}%
\end{pgfscope}%
\begin{pgfscope}%
\pgfsys@transformshift{18.420605in}{0.654898in}%
\pgfsys@useobject{currentmarker}{}%
\end{pgfscope}%
\begin{pgfscope}%
\pgfsys@transformshift{20.009331in}{0.652189in}%
\pgfsys@useobject{currentmarker}{}%
\end{pgfscope}%
\begin{pgfscope}%
\pgfsys@transformshift{21.260300in}{0.647607in}%
\pgfsys@useobject{currentmarker}{}%
\end{pgfscope}%
\begin{pgfscope}%
\pgfsys@transformshift{23.228701in}{0.645006in}%
\pgfsys@useobject{currentmarker}{}%
\end{pgfscope}%
\begin{pgfscope}%
\pgfsys@transformshift{25.888770in}{0.639824in}%
\pgfsys@useobject{currentmarker}{}%
\end{pgfscope}%
\begin{pgfscope}%
\pgfsys@transformshift{28.569097in}{0.635141in}%
\pgfsys@useobject{currentmarker}{}%
\end{pgfscope}%
\begin{pgfscope}%
\pgfsys@transformshift{32.504050in}{0.628662in}%
\pgfsys@useobject{currentmarker}{}%
\end{pgfscope}%
\begin{pgfscope}%
\pgfsys@transformshift{39.060135in}{0.620415in}%
\pgfsys@useobject{currentmarker}{}%
\end{pgfscope}%
\begin{pgfscope}%
\pgfsys@transformshift{50.416853in}{0.608419in}%
\pgfsys@useobject{currentmarker}{}%
\end{pgfscope}%
\begin{pgfscope}%
\pgfsys@transformshift{87.100182in}{0.583248in}%
\pgfsys@useobject{currentmarker}{}%
\end{pgfscope}%
\end{pgfscope}%
\begin{pgfscope}%
\pgfpathrectangle{\pgfqpoint{3.788192in}{2.980138in}}{\pgfqpoint{2.914000in}{2.171400in}}%
\pgfusepath{clip}%
\pgfsetrectcap%
\pgfsetroundjoin%
\pgfsetlinewidth{1.505625pt}%
\definecolor{currentstroke}{rgb}{0.121569,0.466667,0.705882}%
\pgfsetstrokecolor{currentstroke}%
\pgfsetstrokeopacity{0.500000}%
\pgfsetdash{}{0pt}%
\pgfusepath{stroke}%
\end{pgfscope}%
\begin{pgfscope}%
\pgfsetrectcap%
\pgfsetmiterjoin%
\pgfsetlinewidth{0.803000pt}%
\definecolor{currentstroke}{rgb}{0.000000,0.000000,0.000000}%
\pgfsetstrokecolor{currentstroke}%
\pgfsetdash{}{0pt}%
\pgfpathmoveto{\pgfqpoint{3.788192in}{2.980138in}}%
\pgfpathlineto{\pgfqpoint{3.788192in}{5.151538in}}%
\pgfusepath{stroke}%
\end{pgfscope}%
\begin{pgfscope}%
\pgfsetrectcap%
\pgfsetmiterjoin%
\pgfsetlinewidth{0.803000pt}%
\definecolor{currentstroke}{rgb}{0.000000,0.000000,0.000000}%
\pgfsetstrokecolor{currentstroke}%
\pgfsetdash{}{0pt}%
\pgfpathmoveto{\pgfqpoint{6.702192in}{2.980138in}}%
\pgfpathlineto{\pgfqpoint{6.702192in}{5.151538in}}%
\pgfusepath{stroke}%
\end{pgfscope}%
\begin{pgfscope}%
\pgfsetrectcap%
\pgfsetmiterjoin%
\pgfsetlinewidth{0.803000pt}%
\definecolor{currentstroke}{rgb}{0.000000,0.000000,0.000000}%
\pgfsetstrokecolor{currentstroke}%
\pgfsetdash{}{0pt}%
\pgfpathmoveto{\pgfqpoint{3.788192in}{2.980138in}}%
\pgfpathlineto{\pgfqpoint{6.702192in}{2.980138in}}%
\pgfusepath{stroke}%
\end{pgfscope}%
\begin{pgfscope}%
\pgfsetrectcap%
\pgfsetmiterjoin%
\pgfsetlinewidth{0.803000pt}%
\definecolor{currentstroke}{rgb}{0.000000,0.000000,0.000000}%
\pgfsetstrokecolor{currentstroke}%
\pgfsetdash{}{0pt}%
\pgfpathmoveto{\pgfqpoint{3.788192in}{5.151538in}}%
\pgfpathlineto{\pgfqpoint{6.702192in}{5.151538in}}%
\pgfusepath{stroke}%
\end{pgfscope}%
\begin{pgfscope}%
\pgfsetbuttcap%
\pgfsetmiterjoin%
\definecolor{currentfill}{rgb}{1.000000,1.000000,1.000000}%
\pgfsetfillcolor{currentfill}%
\pgfsetfillopacity{0.800000}%
\pgfsetlinewidth{1.003750pt}%
\definecolor{currentstroke}{rgb}{0.800000,0.800000,0.800000}%
\pgfsetstrokecolor{currentstroke}%
\pgfsetstrokeopacity{0.800000}%
\pgfsetdash{}{0pt}%
\pgfpathmoveto{\pgfqpoint{3.327765in}{0.781249in}}%
\pgfpathlineto{\pgfqpoint{6.732636in}{0.781249in}}%
\pgfpathquadraticcurveto{\pgfqpoint{6.777080in}{0.781249in}}{\pgfqpoint{6.777080in}{0.825694in}}%
\pgfpathlineto{\pgfqpoint{6.777080in}{2.153779in}}%
\pgfpathquadraticcurveto{\pgfqpoint{6.777080in}{2.198223in}}{\pgfqpoint{6.732636in}{2.198223in}}%
\pgfpathlineto{\pgfqpoint{3.327765in}{2.198223in}}%
\pgfpathquadraticcurveto{\pgfqpoint{3.283320in}{2.198223in}}{\pgfqpoint{3.283320in}{2.153779in}}%
\pgfpathlineto{\pgfqpoint{3.283320in}{0.825694in}}%
\pgfpathquadraticcurveto{\pgfqpoint{3.283320in}{0.781249in}}{\pgfqpoint{3.327765in}{0.781249in}}%
\pgfpathlineto{\pgfqpoint{3.327765in}{0.781249in}}%
\pgfpathclose%
\pgfusepath{stroke,fill}%
\end{pgfscope}%
\begin{pgfscope}%
\pgfsetrectcap%
\pgfsetroundjoin%
\pgfsetlinewidth{1.505625pt}%
\definecolor{currentstroke}{rgb}{0.000000,0.000000,1.000000}%
\pgfsetstrokecolor{currentstroke}%
\pgfsetdash{}{0pt}%
\pgfpathmoveto{\pgfqpoint{3.372209in}{2.020446in}}%
\pgfpathlineto{\pgfqpoint{3.594431in}{2.020446in}}%
\pgfpathlineto{\pgfqpoint{3.816654in}{2.020446in}}%
\pgfusepath{stroke}%
\end{pgfscope}%
\begin{pgfscope}%
\pgfsetbuttcap%
\pgfsetroundjoin%
\definecolor{currentfill}{rgb}{0.000000,0.000000,1.000000}%
\pgfsetfillcolor{currentfill}%
\pgfsetlinewidth{1.003750pt}%
\definecolor{currentstroke}{rgb}{0.000000,0.000000,1.000000}%
\pgfsetstrokecolor{currentstroke}%
\pgfsetdash{}{0pt}%
\pgfsys@defobject{currentmarker}{\pgfqpoint{-0.006944in}{-0.006944in}}{\pgfqpoint{0.006944in}{0.006944in}}{%
\pgfpathmoveto{\pgfqpoint{0.000000in}{-0.006944in}}%
\pgfpathcurveto{\pgfqpoint{0.001842in}{-0.006944in}}{\pgfqpoint{0.003608in}{-0.006213in}}{\pgfqpoint{0.004910in}{-0.004910in}}%
\pgfpathcurveto{\pgfqpoint{0.006213in}{-0.003608in}}{\pgfqpoint{0.006944in}{-0.001842in}}{\pgfqpoint{0.006944in}{0.000000in}}%
\pgfpathcurveto{\pgfqpoint{0.006944in}{0.001842in}}{\pgfqpoint{0.006213in}{0.003608in}}{\pgfqpoint{0.004910in}{0.004910in}}%
\pgfpathcurveto{\pgfqpoint{0.003608in}{0.006213in}}{\pgfqpoint{0.001842in}{0.006944in}}{\pgfqpoint{0.000000in}{0.006944in}}%
\pgfpathcurveto{\pgfqpoint{-0.001842in}{0.006944in}}{\pgfqpoint{-0.003608in}{0.006213in}}{\pgfqpoint{-0.004910in}{0.004910in}}%
\pgfpathcurveto{\pgfqpoint{-0.006213in}{0.003608in}}{\pgfqpoint{-0.006944in}{0.001842in}}{\pgfqpoint{-0.006944in}{0.000000in}}%
\pgfpathcurveto{\pgfqpoint{-0.006944in}{-0.001842in}}{\pgfqpoint{-0.006213in}{-0.003608in}}{\pgfqpoint{-0.004910in}{-0.004910in}}%
\pgfpathcurveto{\pgfqpoint{-0.003608in}{-0.006213in}}{\pgfqpoint{-0.001842in}{-0.006944in}}{\pgfqpoint{0.000000in}{-0.006944in}}%
\pgfpathlineto{\pgfqpoint{0.000000in}{-0.006944in}}%
\pgfpathclose%
\pgfusepath{stroke,fill}%
}%
\begin{pgfscope}%
\pgfsys@transformshift{3.594431in}{2.020446in}%
\pgfsys@useobject{currentmarker}{}%
\end{pgfscope}%
\end{pgfscope}%
\begin{pgfscope}%
\definecolor{textcolor}{rgb}{0.000000,0.000000,0.000000}%
\pgfsetstrokecolor{textcolor}%
\pgfsetfillcolor{textcolor}%
\pgftext[x=3.994431in,y=1.942668in,left,base]{\color{textcolor}{\rmfamily\fontsize{16.000000}{19.200000}\selectfont\catcode`\^=\active\def^{\ifmmode\sp\else\^{}\fi}\catcode`\%=\active\def%{\%}osier}}%
\end{pgfscope}%
\begin{pgfscope}%
\pgfsetrectcap%
\pgfsetroundjoin%
\pgfsetlinewidth{1.505625pt}%
\definecolor{currentstroke}{rgb}{0.121569,0.466667,0.705882}%
\pgfsetstrokecolor{currentstroke}%
\pgfsetstrokeopacity{0.500000}%
\pgfsetdash{}{0pt}%
\pgfpathmoveto{\pgfqpoint{3.372209in}{1.682869in}}%
\pgfpathlineto{\pgfqpoint{3.594431in}{1.682869in}}%
\pgfpathlineto{\pgfqpoint{3.816654in}{1.682869in}}%
\pgfusepath{stroke}%
\end{pgfscope}%
\begin{pgfscope}%
\definecolor{textcolor}{rgb}{0.000000,0.000000,0.000000}%
\pgfsetstrokecolor{textcolor}%
\pgfsetfillcolor{textcolor}%
\pgftext[x=3.994431in,y=1.605091in,left,base]{\color{textcolor}{\rmfamily\fontsize{16.000000}{19.200000}\selectfont\catcode`\^=\active\def^{\ifmmode\sp\else\^{}\fi}\catcode`\%=\active\def%{\%}near-optimal space (osier)}}%
\end{pgfscope}%
\begin{pgfscope}%
\pgfsetbuttcap%
\pgfsetroundjoin%
\pgfsetlinewidth{1.003750pt}%
\definecolor{currentstroke}{rgb}{1.000000,0.000000,0.000000}%
\pgfsetstrokecolor{currentstroke}%
\pgfsetdash{}{0pt}%
\pgfpathmoveto{\pgfqpoint{3.594431in}{1.294791in}}%
\pgfpathcurveto{\pgfqpoint{3.602668in}{1.294791in}}{\pgfqpoint{3.610568in}{1.298063in}}{\pgfqpoint{3.616392in}{1.303887in}}%
\pgfpathcurveto{\pgfqpoint{3.622216in}{1.309711in}}{\pgfqpoint{3.625488in}{1.317611in}}{\pgfqpoint{3.625488in}{1.325847in}}%
\pgfpathcurveto{\pgfqpoint{3.625488in}{1.334084in}}{\pgfqpoint{3.622216in}{1.341984in}}{\pgfqpoint{3.616392in}{1.347808in}}%
\pgfpathcurveto{\pgfqpoint{3.610568in}{1.353632in}}{\pgfqpoint{3.602668in}{1.356904in}}{\pgfqpoint{3.594431in}{1.356904in}}%
\pgfpathcurveto{\pgfqpoint{3.586195in}{1.356904in}}{\pgfqpoint{3.578295in}{1.353632in}}{\pgfqpoint{3.572471in}{1.347808in}}%
\pgfpathcurveto{\pgfqpoint{3.566647in}{1.341984in}}{\pgfqpoint{3.563375in}{1.334084in}}{\pgfqpoint{3.563375in}{1.325847in}}%
\pgfpathcurveto{\pgfqpoint{3.563375in}{1.317611in}}{\pgfqpoint{3.566647in}{1.309711in}}{\pgfqpoint{3.572471in}{1.303887in}}%
\pgfpathcurveto{\pgfqpoint{3.578295in}{1.298063in}}{\pgfqpoint{3.586195in}{1.294791in}}{\pgfqpoint{3.594431in}{1.294791in}}%
\pgfpathlineto{\pgfqpoint{3.594431in}{1.294791in}}%
\pgfpathclose%
\pgfusepath{stroke}%
\end{pgfscope}%
\begin{pgfscope}%
\definecolor{textcolor}{rgb}{0.000000,0.000000,0.000000}%
\pgfsetstrokecolor{textcolor}%
\pgfsetfillcolor{textcolor}%
\pgftext[x=3.994431in,y=1.267514in,left,base]{\color{textcolor}{\rmfamily\fontsize{16.000000}{19.200000}\selectfont\catcode`\^=\active\def^{\ifmmode\sp\else\^{}\fi}\catcode`\%=\active\def%{\%}temoa+mga}}%
\end{pgfscope}%
\begin{pgfscope}%
\pgfsetbuttcap%
\pgfsetmiterjoin%
\definecolor{currentfill}{rgb}{0.839216,0.152941,0.156863}%
\pgfsetfillcolor{currentfill}%
\pgfsetfillopacity{0.200000}%
\pgfsetlinewidth{1.003750pt}%
\definecolor{currentstroke}{rgb}{0.839216,0.152941,0.156863}%
\pgfsetstrokecolor{currentstroke}%
\pgfsetstrokeopacity{0.200000}%
\pgfsetdash{}{0pt}%
\pgfpathmoveto{\pgfqpoint{3.372209in}{0.929937in}}%
\pgfpathlineto{\pgfqpoint{3.816654in}{0.929937in}}%
\pgfpathlineto{\pgfqpoint{3.816654in}{1.085493in}}%
\pgfpathlineto{\pgfqpoint{3.372209in}{1.085493in}}%
\pgfpathlineto{\pgfqpoint{3.372209in}{0.929937in}}%
\pgfpathclose%
\pgfusepath{stroke,fill}%
\end{pgfscope}%
\begin{pgfscope}%
\pgfsetbuttcap%
\pgfsetmiterjoin%
\definecolor{currentfill}{rgb}{0.839216,0.152941,0.156863}%
\pgfsetfillcolor{currentfill}%
\pgfsetfillopacity{0.200000}%
\pgfsetlinewidth{1.003750pt}%
\definecolor{currentstroke}{rgb}{0.839216,0.152941,0.156863}%
\pgfsetstrokecolor{currentstroke}%
\pgfsetstrokeopacity{0.200000}%
\pgfsetdash{}{0pt}%
\pgfpathmoveto{\pgfqpoint{3.372209in}{0.929937in}}%
\pgfpathlineto{\pgfqpoint{3.816654in}{0.929937in}}%
\pgfpathlineto{\pgfqpoint{3.816654in}{1.085493in}}%
\pgfpathlineto{\pgfqpoint{3.372209in}{1.085493in}}%
\pgfpathlineto{\pgfqpoint{3.372209in}{0.929937in}}%
\pgfpathclose%
\pgfusepath{clip}%
\pgfsys@defobject{currentpattern}{\pgfqpoint{0in}{0in}}{\pgfqpoint{1in}{1in}}{%
\begin{pgfscope}%
\pgfpathrectangle{\pgfqpoint{0in}{0in}}{\pgfqpoint{1in}{1in}}%
\pgfusepath{clip}%
\pgfpathmoveto{\pgfqpoint{-0.500000in}{0.500000in}}%
\pgfpathlineto{\pgfqpoint{0.500000in}{1.500000in}}%
\pgfpathmoveto{\pgfqpoint{-0.333333in}{0.333333in}}%
\pgfpathlineto{\pgfqpoint{0.666667in}{1.333333in}}%
\pgfpathmoveto{\pgfqpoint{-0.166667in}{0.166667in}}%
\pgfpathlineto{\pgfqpoint{0.833333in}{1.166667in}}%
\pgfpathmoveto{\pgfqpoint{0.000000in}{0.000000in}}%
\pgfpathlineto{\pgfqpoint{1.000000in}{1.000000in}}%
\pgfpathmoveto{\pgfqpoint{0.166667in}{-0.166667in}}%
\pgfpathlineto{\pgfqpoint{1.166667in}{0.833333in}}%
\pgfpathmoveto{\pgfqpoint{0.333333in}{-0.333333in}}%
\pgfpathlineto{\pgfqpoint{1.333333in}{0.666667in}}%
\pgfpathmoveto{\pgfqpoint{0.500000in}{-0.500000in}}%
\pgfpathlineto{\pgfqpoint{1.500000in}{0.500000in}}%
\pgfusepath{stroke}%
\end{pgfscope}%
}%
\pgfsys@transformshift{3.372209in}{0.929937in}%
\pgfsys@useobject{currentpattern}{}%
\pgfsys@transformshift{1in}{0in}%
\pgfsys@transformshift{-1in}{0in}%
\pgfsys@transformshift{0in}{1in}%
\end{pgfscope}%
\begin{pgfscope}%
\definecolor{textcolor}{rgb}{0.000000,0.000000,0.000000}%
\pgfsetstrokecolor{textcolor}%
\pgfsetfillcolor{textcolor}%
\pgftext[x=3.994431in,y=0.929937in,left,base]{\color{textcolor}{\rmfamily\fontsize{16.000000}{19.200000}\selectfont\catcode`\^=\active\def^{\ifmmode\sp\else\^{}\fi}\catcode`\%=\active\def%{\%}near-optimal space (Temoa)}}%
\end{pgfscope}%
\end{pgfpicture}%
\makeatother%
\endgroup%
}
  \resizebox{0.75\columnwidth}{!}{%% Creator: Matplotlib, PGF backend
%%
%% To include the figure in your LaTeX document, write
%%   \input{<filename>.pgf}
%%
%% Make sure the required packages are loaded in your preamble
%%   \usepackage{pgf}
%%
%% Also ensure that all the required font packages are loaded; for instance,
%% the lmodern package is sometimes necessary when using math font.
%%   \usepackage{lmodern}
%%
%% Figures using additional raster images can only be included by \input if
%% they are in the same directory as the main LaTeX file. For loading figures
%% from other directories you can use the `import` package
%%   \usepackage{import}
%%
%% and then include the figures with
%%   \import{<path to file>}{<filename>.pgf}
%%
%% Matplotlib used the following preamble
%%   \def\mathdefault#1{#1}
%%   \everymath=\expandafter{\the\everymath\displaystyle}
%%   \IfFileExists{scrextend.sty}{
%%     \usepackage[fontsize=10.000000pt]{scrextend}
%%   }{
%%     \renewcommand{\normalsize}{\fontsize{10.000000}{12.000000}\selectfont}
%%     \normalsize
%%   }
%%   
%%   \makeatletter\@ifpackageloaded{underscore}{}{\usepackage[strings]{underscore}}\makeatother
%%
\begingroup%
\makeatletter%
\begin{pgfpicture}%
\pgfpathrectangle{\pgfpointorigin}{\pgfqpoint{6.988192in}{5.458470in}}%
\pgfusepath{use as bounding box, clip}%
\begin{pgfscope}%
\pgfsetbuttcap%
\pgfsetmiterjoin%
\definecolor{currentfill}{rgb}{1.000000,1.000000,1.000000}%
\pgfsetfillcolor{currentfill}%
\pgfsetlinewidth{0.000000pt}%
\definecolor{currentstroke}{rgb}{0.000000,0.000000,0.000000}%
\pgfsetstrokecolor{currentstroke}%
\pgfsetdash{}{0pt}%
\pgfpathmoveto{\pgfqpoint{0.000000in}{0.000000in}}%
\pgfpathlineto{\pgfqpoint{6.988192in}{0.000000in}}%
\pgfpathlineto{\pgfqpoint{6.988192in}{5.458470in}}%
\pgfpathlineto{\pgfqpoint{0.000000in}{5.458470in}}%
\pgfpathlineto{\pgfqpoint{0.000000in}{0.000000in}}%
\pgfpathclose%
\pgfusepath{fill}%
\end{pgfscope}%
\begin{pgfscope}%
\pgfsetbuttcap%
\pgfsetmiterjoin%
\definecolor{currentfill}{rgb}{1.000000,1.000000,1.000000}%
\pgfsetfillcolor{currentfill}%
\pgfsetlinewidth{0.000000pt}%
\definecolor{currentstroke}{rgb}{0.000000,0.000000,0.000000}%
\pgfsetstrokecolor{currentstroke}%
\pgfsetstrokeopacity{0.000000}%
\pgfsetdash{}{0pt}%
\pgfpathmoveto{\pgfqpoint{0.688192in}{0.670138in}}%
\pgfpathlineto{\pgfqpoint{6.888192in}{0.670138in}}%
\pgfpathlineto{\pgfqpoint{6.888192in}{5.290138in}}%
\pgfpathlineto{\pgfqpoint{0.688192in}{5.290138in}}%
\pgfpathlineto{\pgfqpoint{0.688192in}{0.670138in}}%
\pgfpathclose%
\pgfusepath{fill}%
\end{pgfscope}%
\begin{pgfscope}%
\pgfpathrectangle{\pgfqpoint{0.688192in}{0.670138in}}{\pgfqpoint{6.200000in}{4.620000in}}%
\pgfusepath{clip}%
\pgfsetbuttcap%
\pgfsetmiterjoin%
\definecolor{currentfill}{rgb}{0.121569,0.466667,0.705882}%
\pgfsetfillcolor{currentfill}%
\pgfsetfillopacity{0.500000}%
\pgfsetlinewidth{1.003750pt}%
\definecolor{currentstroke}{rgb}{0.121569,0.466667,0.705882}%
\pgfsetstrokecolor{currentstroke}%
\pgfsetstrokeopacity{0.500000}%
\pgfsetdash{}{0pt}%
\pgfpathmoveto{\pgfqpoint{0.741425in}{1.377543in}}%
\pgfpathlineto{\pgfqpoint{0.758703in}{0.955032in}}%
\pgfpathlineto{\pgfqpoint{0.768198in}{0.875033in}}%
\pgfpathlineto{\pgfqpoint{0.774746in}{0.828781in}}%
\pgfpathlineto{\pgfqpoint{0.778243in}{0.822495in}}%
\pgfpathlineto{\pgfqpoint{0.782159in}{0.789611in}}%
\pgfpathlineto{\pgfqpoint{0.786516in}{0.779881in}}%
\pgfpathlineto{\pgfqpoint{0.792538in}{0.779145in}}%
\pgfpathlineto{\pgfqpoint{0.794668in}{0.758056in}}%
\pgfpathlineto{\pgfqpoint{0.799837in}{0.752930in}}%
\pgfpathlineto{\pgfqpoint{0.809370in}{0.751978in}}%
\pgfpathlineto{\pgfqpoint{0.812629in}{0.743975in}}%
\pgfpathlineto{\pgfqpoint{0.815972in}{0.742575in}}%
\pgfpathlineto{\pgfqpoint{0.822987in}{0.738477in}}%
\pgfpathlineto{\pgfqpoint{0.828825in}{0.734937in}}%
\pgfpathlineto{\pgfqpoint{0.829214in}{0.733319in}}%
\pgfpathlineto{\pgfqpoint{0.833044in}{0.730858in}}%
\pgfpathlineto{\pgfqpoint{0.848459in}{0.726329in}}%
\pgfpathlineto{\pgfqpoint{0.864854in}{0.720019in}}%
\pgfpathlineto{\pgfqpoint{0.887104in}{0.715517in}}%
\pgfpathlineto{\pgfqpoint{0.907479in}{0.714004in}}%
\pgfpathlineto{\pgfqpoint{0.908310in}{0.712008in}}%
\pgfpathlineto{\pgfqpoint{0.909513in}{0.708525in}}%
\pgfpathlineto{\pgfqpoint{0.912740in}{0.707284in}}%
\pgfpathlineto{\pgfqpoint{0.920440in}{0.706723in}}%
\pgfpathlineto{\pgfqpoint{0.925670in}{0.705238in}}%
\pgfpathlineto{\pgfqpoint{0.948903in}{0.702931in}}%
\pgfpathlineto{\pgfqpoint{0.951945in}{0.701707in}}%
\pgfpathlineto{\pgfqpoint{0.952035in}{0.700391in}}%
\pgfpathlineto{\pgfqpoint{0.957029in}{0.700173in}}%
\pgfpathlineto{\pgfqpoint{0.968828in}{0.697963in}}%
\pgfpathlineto{\pgfqpoint{0.974412in}{0.697738in}}%
\pgfpathlineto{\pgfqpoint{0.975275in}{0.696914in}}%
\pgfpathlineto{\pgfqpoint{1.021767in}{0.694795in}}%
\pgfpathlineto{\pgfqpoint{1.025407in}{0.690657in}}%
\pgfpathlineto{\pgfqpoint{1.027475in}{0.690338in}}%
\pgfpathlineto{\pgfqpoint{1.034837in}{0.689784in}}%
\pgfpathlineto{\pgfqpoint{1.049406in}{0.687676in}}%
\pgfpathlineto{\pgfqpoint{1.054714in}{0.687138in}}%
\pgfpathlineto{\pgfqpoint{1.059617in}{0.686467in}}%
\pgfpathlineto{\pgfqpoint{1.072141in}{0.685078in}}%
\pgfpathlineto{\pgfqpoint{1.092208in}{0.684413in}}%
\pgfpathlineto{\pgfqpoint{1.115209in}{0.684111in}}%
\pgfpathlineto{\pgfqpoint{1.131834in}{0.684071in}}%
\pgfpathlineto{\pgfqpoint{1.152628in}{0.684059in}}%
\pgfpathlineto{\pgfqpoint{1.251312in}{0.683263in}}%
\pgfpathlineto{\pgfqpoint{1.277476in}{0.683159in}}%
\pgfpathlineto{\pgfqpoint{1.314870in}{0.682855in}}%
\pgfpathlineto{\pgfqpoint{1.369253in}{0.682756in}}%
\pgfpathlineto{\pgfqpoint{1.398687in}{0.682288in}}%
\pgfpathlineto{\pgfqpoint{1.467852in}{0.682134in}}%
\pgfpathlineto{\pgfqpoint{1.557026in}{0.681680in}}%
\pgfpathlineto{\pgfqpoint{1.627242in}{0.680913in}}%
\pgfpathlineto{\pgfqpoint{1.737728in}{0.680478in}}%
\pgfpathlineto{\pgfqpoint{1.887036in}{0.679610in}}%
\pgfpathlineto{\pgfqpoint{2.037481in}{0.678826in}}%
\pgfpathlineto{\pgfqpoint{2.258348in}{0.677741in}}%
\pgfpathlineto{\pgfqpoint{2.626338in}{0.676361in}}%
\pgfpathlineto{\pgfqpoint{3.263784in}{0.674352in}}%
\pgfpathlineto{\pgfqpoint{5.322800in}{0.670138in}}%
\pgfpathlineto{\pgfqpoint{6.888192in}{0.683471in}}%
\pgfpathlineto{\pgfqpoint{4.623274in}{0.688107in}}%
\pgfpathlineto{\pgfqpoint{3.922083in}{0.690316in}}%
\pgfpathlineto{\pgfqpoint{3.517294in}{0.691835in}}%
\pgfpathlineto{\pgfqpoint{3.274341in}{0.693028in}}%
\pgfpathlineto{\pgfqpoint{3.108851in}{0.693891in}}%
\pgfpathlineto{\pgfqpoint{2.944612in}{0.694845in}}%
\pgfpathlineto{\pgfqpoint{2.823078in}{0.695324in}}%
\pgfpathlineto{\pgfqpoint{2.745840in}{0.696168in}}%
\pgfpathlineto{\pgfqpoint{2.647748in}{0.696667in}}%
\pgfpathlineto{\pgfqpoint{2.571667in}{0.696836in}}%
\pgfpathlineto{\pgfqpoint{2.539290in}{0.697351in}}%
\pgfpathlineto{\pgfqpoint{2.479468in}{0.697460in}}%
\pgfpathlineto{\pgfqpoint{2.438334in}{0.697795in}}%
\pgfpathlineto{\pgfqpoint{2.409554in}{0.697909in}}%
\pgfpathlineto{\pgfqpoint{2.301002in}{0.698784in}}%
\pgfpathlineto{\pgfqpoint{2.278129in}{0.698798in}}%
\pgfpathlineto{\pgfqpoint{2.259841in}{0.698842in}}%
\pgfpathlineto{\pgfqpoint{2.234540in}{0.699174in}}%
\pgfpathlineto{\pgfqpoint{2.212467in}{0.699905in}}%
\pgfpathlineto{\pgfqpoint{2.198690in}{0.701434in}}%
\pgfpathlineto{\pgfqpoint{2.193297in}{0.702172in}}%
\pgfpathlineto{\pgfqpoint{2.187458in}{0.702763in}}%
\pgfpathlineto{\pgfqpoint{2.171432in}{0.705082in}}%
\pgfpathlineto{\pgfqpoint{2.163334in}{0.705692in}}%
\pgfpathlineto{\pgfqpoint{2.161059in}{0.706043in}}%
\pgfpathlineto{\pgfqpoint{2.157055in}{0.710594in}}%
\pgfpathlineto{\pgfqpoint{2.105914in}{0.712924in}}%
\pgfpathlineto{\pgfqpoint{2.104964in}{0.713831in}}%
\pgfpathlineto{\pgfqpoint{2.098822in}{0.714079in}}%
\pgfpathlineto{\pgfqpoint{2.085843in}{0.716510in}}%
\pgfpathlineto{\pgfqpoint{2.080349in}{0.716749in}}%
\pgfpathlineto{\pgfqpoint{2.080251in}{0.718197in}}%
\pgfpathlineto{\pgfqpoint{2.076904in}{0.719544in}}%
\pgfpathlineto{\pgfqpoint{2.051349in}{0.722081in}}%
\pgfpathlineto{\pgfqpoint{2.045595in}{0.723715in}}%
\pgfpathlineto{\pgfqpoint{2.037125in}{0.724332in}}%
\pgfpathlineto{\pgfqpoint{2.033576in}{0.725697in}}%
\pgfpathlineto{\pgfqpoint{2.032252in}{0.729528in}}%
\pgfpathlineto{\pgfqpoint{2.031338in}{0.731724in}}%
\pgfpathlineto{\pgfqpoint{2.008926in}{0.733388in}}%
\pgfpathlineto{\pgfqpoint{1.984450in}{0.738340in}}%
\pgfpathlineto{\pgfqpoint{1.966417in}{0.745282in}}%
\pgfpathlineto{\pgfqpoint{1.949459in}{0.750264in}}%
\pgfpathlineto{\pgfqpoint{1.945246in}{0.752971in}}%
\pgfpathlineto{\pgfqpoint{1.944819in}{0.754750in}}%
\pgfpathlineto{\pgfqpoint{1.938397in}{0.758644in}}%
\pgfpathlineto{\pgfqpoint{1.930681in}{0.763152in}}%
\pgfpathlineto{\pgfqpoint{1.927003in}{0.764692in}}%
\pgfpathlineto{\pgfqpoint{1.923418in}{0.773495in}}%
\pgfpathlineto{\pgfqpoint{1.912932in}{0.774543in}}%
\pgfpathlineto{\pgfqpoint{1.907246in}{0.780181in}}%
\pgfpathlineto{\pgfqpoint{1.904902in}{0.803379in}}%
\pgfpathlineto{\pgfqpoint{1.898279in}{0.804188in}}%
\pgfpathlineto{\pgfqpoint{1.893486in}{0.814892in}}%
\pgfpathlineto{\pgfqpoint{1.889178in}{0.851064in}}%
\pgfpathlineto{\pgfqpoint{1.885331in}{0.857978in}}%
\pgfpathlineto{\pgfqpoint{1.878129in}{0.908856in}}%
\pgfpathlineto{\pgfqpoint{1.867684in}{0.996854in}}%
\pgfpathlineto{\pgfqpoint{1.848679in}{1.461617in}}%
\pgfpathlineto{\pgfqpoint{0.741425in}{1.377543in}}%
\pgfpathclose%
\pgfusepath{stroke,fill}%
\end{pgfscope}%
\begin{pgfscope}%
\pgfpathrectangle{\pgfqpoint{0.688192in}{0.670138in}}{\pgfqpoint{6.200000in}{4.620000in}}%
\pgfusepath{clip}%
\pgfsetbuttcap%
\pgfsetroundjoin%
\pgfsetlinewidth{1.003750pt}%
\definecolor{currentstroke}{rgb}{1.000000,0.000000,0.000000}%
\pgfsetstrokecolor{currentstroke}%
\pgfsetdash{}{0pt}%
\pgfpathmoveto{\pgfqpoint{1.380312in}{4.826960in}}%
\pgfpathcurveto{\pgfqpoint{1.388548in}{4.826960in}}{\pgfqpoint{1.396449in}{4.830233in}}{\pgfqpoint{1.402272in}{4.836057in}}%
\pgfpathcurveto{\pgfqpoint{1.408096in}{4.841881in}}{\pgfqpoint{1.411369in}{4.849781in}}{\pgfqpoint{1.411369in}{4.858017in}}%
\pgfpathcurveto{\pgfqpoint{1.411369in}{4.866253in}}{\pgfqpoint{1.408096in}{4.874153in}}{\pgfqpoint{1.402272in}{4.879977in}}%
\pgfpathcurveto{\pgfqpoint{1.396449in}{4.885801in}}{\pgfqpoint{1.388548in}{4.889073in}}{\pgfqpoint{1.380312in}{4.889073in}}%
\pgfpathcurveto{\pgfqpoint{1.372076in}{4.889073in}}{\pgfqpoint{1.364176in}{4.885801in}}{\pgfqpoint{1.358352in}{4.879977in}}%
\pgfpathcurveto{\pgfqpoint{1.352528in}{4.874153in}}{\pgfqpoint{1.349256in}{4.866253in}}{\pgfqpoint{1.349256in}{4.858017in}}%
\pgfpathcurveto{\pgfqpoint{1.349256in}{4.849781in}}{\pgfqpoint{1.352528in}{4.841881in}}{\pgfqpoint{1.358352in}{4.836057in}}%
\pgfpathcurveto{\pgfqpoint{1.364176in}{4.830233in}}{\pgfqpoint{1.372076in}{4.826960in}}{\pgfqpoint{1.380312in}{4.826960in}}%
\pgfpathlineto{\pgfqpoint{1.380312in}{4.826960in}}%
\pgfpathclose%
\pgfusepath{stroke}%
\end{pgfscope}%
\begin{pgfscope}%
\pgfpathrectangle{\pgfqpoint{0.688192in}{0.670138in}}{\pgfqpoint{6.200000in}{4.620000in}}%
\pgfusepath{clip}%
\pgfsetbuttcap%
\pgfsetroundjoin%
\pgfsetlinewidth{1.003750pt}%
\definecolor{currentstroke}{rgb}{1.000000,0.000000,0.000000}%
\pgfsetstrokecolor{currentstroke}%
\pgfsetdash{}{0pt}%
\pgfpathmoveto{\pgfqpoint{1.103776in}{2.239473in}}%
\pgfpathcurveto{\pgfqpoint{1.112013in}{2.239473in}}{\pgfqpoint{1.119913in}{2.242745in}}{\pgfqpoint{1.125737in}{2.248569in}}%
\pgfpathcurveto{\pgfqpoint{1.131560in}{2.254393in}}{\pgfqpoint{1.134833in}{2.262293in}}{\pgfqpoint{1.134833in}{2.270529in}}%
\pgfpathcurveto{\pgfqpoint{1.134833in}{2.278765in}}{\pgfqpoint{1.131560in}{2.286665in}}{\pgfqpoint{1.125737in}{2.292489in}}%
\pgfpathcurveto{\pgfqpoint{1.119913in}{2.298313in}}{\pgfqpoint{1.112013in}{2.301586in}}{\pgfqpoint{1.103776in}{2.301586in}}%
\pgfpathcurveto{\pgfqpoint{1.095540in}{2.301586in}}{\pgfqpoint{1.087640in}{2.298313in}}{\pgfqpoint{1.081816in}{2.292489in}}%
\pgfpathcurveto{\pgfqpoint{1.075992in}{2.286665in}}{\pgfqpoint{1.072720in}{2.278765in}}{\pgfqpoint{1.072720in}{2.270529in}}%
\pgfpathcurveto{\pgfqpoint{1.072720in}{2.262293in}}{\pgfqpoint{1.075992in}{2.254393in}}{\pgfqpoint{1.081816in}{2.248569in}}%
\pgfpathcurveto{\pgfqpoint{1.087640in}{2.242745in}}{\pgfqpoint{1.095540in}{2.239473in}}{\pgfqpoint{1.103776in}{2.239473in}}%
\pgfpathlineto{\pgfqpoint{1.103776in}{2.239473in}}%
\pgfpathclose%
\pgfusepath{stroke}%
\end{pgfscope}%
\begin{pgfscope}%
\pgfpathrectangle{\pgfqpoint{0.688192in}{0.670138in}}{\pgfqpoint{6.200000in}{4.620000in}}%
\pgfusepath{clip}%
\pgfsetbuttcap%
\pgfsetroundjoin%
\pgfsetlinewidth{1.003750pt}%
\definecolor{currentstroke}{rgb}{1.000000,0.000000,0.000000}%
\pgfsetstrokecolor{currentstroke}%
\pgfsetdash{}{0pt}%
\pgfpathmoveto{\pgfqpoint{1.146295in}{2.309659in}}%
\pgfpathcurveto{\pgfqpoint{1.154531in}{2.309659in}}{\pgfqpoint{1.162431in}{2.312932in}}{\pgfqpoint{1.168255in}{2.318756in}}%
\pgfpathcurveto{\pgfqpoint{1.174079in}{2.324579in}}{\pgfqpoint{1.177352in}{2.332480in}}{\pgfqpoint{1.177352in}{2.340716in}}%
\pgfpathcurveto{\pgfqpoint{1.177352in}{2.348952in}}{\pgfqpoint{1.174079in}{2.356852in}}{\pgfqpoint{1.168255in}{2.362676in}}%
\pgfpathcurveto{\pgfqpoint{1.162431in}{2.368500in}}{\pgfqpoint{1.154531in}{2.371772in}}{\pgfqpoint{1.146295in}{2.371772in}}%
\pgfpathcurveto{\pgfqpoint{1.138059in}{2.371772in}}{\pgfqpoint{1.130159in}{2.368500in}}{\pgfqpoint{1.124335in}{2.362676in}}%
\pgfpathcurveto{\pgfqpoint{1.118511in}{2.356852in}}{\pgfqpoint{1.115239in}{2.348952in}}{\pgfqpoint{1.115239in}{2.340716in}}%
\pgfpathcurveto{\pgfqpoint{1.115239in}{2.332480in}}{\pgfqpoint{1.118511in}{2.324579in}}{\pgfqpoint{1.124335in}{2.318756in}}%
\pgfpathcurveto{\pgfqpoint{1.130159in}{2.312932in}}{\pgfqpoint{1.138059in}{2.309659in}}{\pgfqpoint{1.146295in}{2.309659in}}%
\pgfpathlineto{\pgfqpoint{1.146295in}{2.309659in}}%
\pgfpathclose%
\pgfusepath{stroke}%
\end{pgfscope}%
\begin{pgfscope}%
\pgfpathrectangle{\pgfqpoint{0.688192in}{0.670138in}}{\pgfqpoint{6.200000in}{4.620000in}}%
\pgfusepath{clip}%
\pgfsetbuttcap%
\pgfsetroundjoin%
\pgfsetlinewidth{1.003750pt}%
\definecolor{currentstroke}{rgb}{1.000000,0.000000,0.000000}%
\pgfsetstrokecolor{currentstroke}%
\pgfsetdash{}{0pt}%
\pgfpathmoveto{\pgfqpoint{1.283972in}{2.337286in}}%
\pgfpathcurveto{\pgfqpoint{1.292208in}{2.337286in}}{\pgfqpoint{1.300108in}{2.340558in}}{\pgfqpoint{1.305932in}{2.346382in}}%
\pgfpathcurveto{\pgfqpoint{1.311756in}{2.352206in}}{\pgfqpoint{1.315028in}{2.360106in}}{\pgfqpoint{1.315028in}{2.368343in}}%
\pgfpathcurveto{\pgfqpoint{1.315028in}{2.376579in}}{\pgfqpoint{1.311756in}{2.384479in}}{\pgfqpoint{1.305932in}{2.390303in}}%
\pgfpathcurveto{\pgfqpoint{1.300108in}{2.396127in}}{\pgfqpoint{1.292208in}{2.399399in}}{\pgfqpoint{1.283972in}{2.399399in}}%
\pgfpathcurveto{\pgfqpoint{1.275735in}{2.399399in}}{\pgfqpoint{1.267835in}{2.396127in}}{\pgfqpoint{1.262011in}{2.390303in}}%
\pgfpathcurveto{\pgfqpoint{1.256187in}{2.384479in}}{\pgfqpoint{1.252915in}{2.376579in}}{\pgfqpoint{1.252915in}{2.368343in}}%
\pgfpathcurveto{\pgfqpoint{1.252915in}{2.360106in}}{\pgfqpoint{1.256187in}{2.352206in}}{\pgfqpoint{1.262011in}{2.346382in}}%
\pgfpathcurveto{\pgfqpoint{1.267835in}{2.340558in}}{\pgfqpoint{1.275735in}{2.337286in}}{\pgfqpoint{1.283972in}{2.337286in}}%
\pgfpathlineto{\pgfqpoint{1.283972in}{2.337286in}}%
\pgfpathclose%
\pgfusepath{stroke}%
\end{pgfscope}%
\begin{pgfscope}%
\pgfpathrectangle{\pgfqpoint{0.688192in}{0.670138in}}{\pgfqpoint{6.200000in}{4.620000in}}%
\pgfusepath{clip}%
\pgfsetbuttcap%
\pgfsetroundjoin%
\pgfsetlinewidth{1.003750pt}%
\definecolor{currentstroke}{rgb}{1.000000,0.000000,0.000000}%
\pgfsetstrokecolor{currentstroke}%
\pgfsetdash{}{0pt}%
\pgfpathmoveto{\pgfqpoint{1.194044in}{2.459126in}}%
\pgfpathcurveto{\pgfqpoint{1.202281in}{2.459126in}}{\pgfqpoint{1.210181in}{2.462398in}}{\pgfqpoint{1.216005in}{2.468222in}}%
\pgfpathcurveto{\pgfqpoint{1.221828in}{2.474046in}}{\pgfqpoint{1.225101in}{2.481946in}}{\pgfqpoint{1.225101in}{2.490182in}}%
\pgfpathcurveto{\pgfqpoint{1.225101in}{2.498419in}}{\pgfqpoint{1.221828in}{2.506319in}}{\pgfqpoint{1.216005in}{2.512143in}}%
\pgfpathcurveto{\pgfqpoint{1.210181in}{2.517967in}}{\pgfqpoint{1.202281in}{2.521239in}}{\pgfqpoint{1.194044in}{2.521239in}}%
\pgfpathcurveto{\pgfqpoint{1.185808in}{2.521239in}}{\pgfqpoint{1.177908in}{2.517967in}}{\pgfqpoint{1.172084in}{2.512143in}}%
\pgfpathcurveto{\pgfqpoint{1.166260in}{2.506319in}}{\pgfqpoint{1.162988in}{2.498419in}}{\pgfqpoint{1.162988in}{2.490182in}}%
\pgfpathcurveto{\pgfqpoint{1.162988in}{2.481946in}}{\pgfqpoint{1.166260in}{2.474046in}}{\pgfqpoint{1.172084in}{2.468222in}}%
\pgfpathcurveto{\pgfqpoint{1.177908in}{2.462398in}}{\pgfqpoint{1.185808in}{2.459126in}}{\pgfqpoint{1.194044in}{2.459126in}}%
\pgfpathlineto{\pgfqpoint{1.194044in}{2.459126in}}%
\pgfpathclose%
\pgfusepath{stroke}%
\end{pgfscope}%
\begin{pgfscope}%
\pgfpathrectangle{\pgfqpoint{0.688192in}{0.670138in}}{\pgfqpoint{6.200000in}{4.620000in}}%
\pgfusepath{clip}%
\pgfsetbuttcap%
\pgfsetroundjoin%
\pgfsetlinewidth{1.003750pt}%
\definecolor{currentstroke}{rgb}{1.000000,0.000000,0.000000}%
\pgfsetstrokecolor{currentstroke}%
\pgfsetdash{}{0pt}%
\pgfpathmoveto{\pgfqpoint{1.126156in}{2.039611in}}%
\pgfpathcurveto{\pgfqpoint{1.134392in}{2.039611in}}{\pgfqpoint{1.142292in}{2.042884in}}{\pgfqpoint{1.148116in}{2.048708in}}%
\pgfpathcurveto{\pgfqpoint{1.153940in}{2.054532in}}{\pgfqpoint{1.157212in}{2.062432in}}{\pgfqpoint{1.157212in}{2.070668in}}%
\pgfpathcurveto{\pgfqpoint{1.157212in}{2.078904in}}{\pgfqpoint{1.153940in}{2.086804in}}{\pgfqpoint{1.148116in}{2.092628in}}%
\pgfpathcurveto{\pgfqpoint{1.142292in}{2.098452in}}{\pgfqpoint{1.134392in}{2.101724in}}{\pgfqpoint{1.126156in}{2.101724in}}%
\pgfpathcurveto{\pgfqpoint{1.117920in}{2.101724in}}{\pgfqpoint{1.110020in}{2.098452in}}{\pgfqpoint{1.104196in}{2.092628in}}%
\pgfpathcurveto{\pgfqpoint{1.098372in}{2.086804in}}{\pgfqpoint{1.095099in}{2.078904in}}{\pgfqpoint{1.095099in}{2.070668in}}%
\pgfpathcurveto{\pgfqpoint{1.095099in}{2.062432in}}{\pgfqpoint{1.098372in}{2.054532in}}{\pgfqpoint{1.104196in}{2.048708in}}%
\pgfpathcurveto{\pgfqpoint{1.110020in}{2.042884in}}{\pgfqpoint{1.117920in}{2.039611in}}{\pgfqpoint{1.126156in}{2.039611in}}%
\pgfpathlineto{\pgfqpoint{1.126156in}{2.039611in}}%
\pgfpathclose%
\pgfusepath{stroke}%
\end{pgfscope}%
\begin{pgfscope}%
\pgfpathrectangle{\pgfqpoint{0.688192in}{0.670138in}}{\pgfqpoint{6.200000in}{4.620000in}}%
\pgfusepath{clip}%
\pgfsetbuttcap%
\pgfsetroundjoin%
\pgfsetlinewidth{1.003750pt}%
\definecolor{currentstroke}{rgb}{1.000000,0.000000,0.000000}%
\pgfsetstrokecolor{currentstroke}%
\pgfsetdash{}{0pt}%
\pgfpathmoveto{\pgfqpoint{1.073783in}{1.736453in}}%
\pgfpathcurveto{\pgfqpoint{1.082020in}{1.736453in}}{\pgfqpoint{1.089920in}{1.739725in}}{\pgfqpoint{1.095744in}{1.745549in}}%
\pgfpathcurveto{\pgfqpoint{1.101568in}{1.751373in}}{\pgfqpoint{1.104840in}{1.759273in}}{\pgfqpoint{1.104840in}{1.767510in}}%
\pgfpathcurveto{\pgfqpoint{1.104840in}{1.775746in}}{\pgfqpoint{1.101568in}{1.783646in}}{\pgfqpoint{1.095744in}{1.789470in}}%
\pgfpathcurveto{\pgfqpoint{1.089920in}{1.795294in}}{\pgfqpoint{1.082020in}{1.798566in}}{\pgfqpoint{1.073783in}{1.798566in}}%
\pgfpathcurveto{\pgfqpoint{1.065547in}{1.798566in}}{\pgfqpoint{1.057647in}{1.795294in}}{\pgfqpoint{1.051823in}{1.789470in}}%
\pgfpathcurveto{\pgfqpoint{1.045999in}{1.783646in}}{\pgfqpoint{1.042727in}{1.775746in}}{\pgfqpoint{1.042727in}{1.767510in}}%
\pgfpathcurveto{\pgfqpoint{1.042727in}{1.759273in}}{\pgfqpoint{1.045999in}{1.751373in}}{\pgfqpoint{1.051823in}{1.745549in}}%
\pgfpathcurveto{\pgfqpoint{1.057647in}{1.739725in}}{\pgfqpoint{1.065547in}{1.736453in}}{\pgfqpoint{1.073783in}{1.736453in}}%
\pgfpathlineto{\pgfqpoint{1.073783in}{1.736453in}}%
\pgfpathclose%
\pgfusepath{stroke}%
\end{pgfscope}%
\begin{pgfscope}%
\pgfpathrectangle{\pgfqpoint{0.688192in}{0.670138in}}{\pgfqpoint{6.200000in}{4.620000in}}%
\pgfusepath{clip}%
\pgfsetbuttcap%
\pgfsetroundjoin%
\pgfsetlinewidth{1.003750pt}%
\definecolor{currentstroke}{rgb}{1.000000,0.000000,0.000000}%
\pgfsetstrokecolor{currentstroke}%
\pgfsetdash{}{0pt}%
\pgfpathmoveto{\pgfqpoint{1.071350in}{1.723563in}}%
\pgfpathcurveto{\pgfqpoint{1.079586in}{1.723563in}}{\pgfqpoint{1.087486in}{1.726835in}}{\pgfqpoint{1.093310in}{1.732659in}}%
\pgfpathcurveto{\pgfqpoint{1.099134in}{1.738483in}}{\pgfqpoint{1.102407in}{1.746383in}}{\pgfqpoint{1.102407in}{1.754620in}}%
\pgfpathcurveto{\pgfqpoint{1.102407in}{1.762856in}}{\pgfqpoint{1.099134in}{1.770756in}}{\pgfqpoint{1.093310in}{1.776580in}}%
\pgfpathcurveto{\pgfqpoint{1.087486in}{1.782404in}}{\pgfqpoint{1.079586in}{1.785676in}}{\pgfqpoint{1.071350in}{1.785676in}}%
\pgfpathcurveto{\pgfqpoint{1.063114in}{1.785676in}}{\pgfqpoint{1.055214in}{1.782404in}}{\pgfqpoint{1.049390in}{1.776580in}}%
\pgfpathcurveto{\pgfqpoint{1.043566in}{1.770756in}}{\pgfqpoint{1.040294in}{1.762856in}}{\pgfqpoint{1.040294in}{1.754620in}}%
\pgfpathcurveto{\pgfqpoint{1.040294in}{1.746383in}}{\pgfqpoint{1.043566in}{1.738483in}}{\pgfqpoint{1.049390in}{1.732659in}}%
\pgfpathcurveto{\pgfqpoint{1.055214in}{1.726835in}}{\pgfqpoint{1.063114in}{1.723563in}}{\pgfqpoint{1.071350in}{1.723563in}}%
\pgfpathlineto{\pgfqpoint{1.071350in}{1.723563in}}%
\pgfpathclose%
\pgfusepath{stroke}%
\end{pgfscope}%
\begin{pgfscope}%
\pgfpathrectangle{\pgfqpoint{0.688192in}{0.670138in}}{\pgfqpoint{6.200000in}{4.620000in}}%
\pgfusepath{clip}%
\pgfsetbuttcap%
\pgfsetroundjoin%
\pgfsetlinewidth{1.003750pt}%
\definecolor{currentstroke}{rgb}{1.000000,0.000000,0.000000}%
\pgfsetstrokecolor{currentstroke}%
\pgfsetdash{}{0pt}%
\pgfpathmoveto{\pgfqpoint{1.176638in}{2.187985in}}%
\pgfpathcurveto{\pgfqpoint{1.184874in}{2.187985in}}{\pgfqpoint{1.192774in}{2.191257in}}{\pgfqpoint{1.198598in}{2.197081in}}%
\pgfpathcurveto{\pgfqpoint{1.204422in}{2.202905in}}{\pgfqpoint{1.207694in}{2.210805in}}{\pgfqpoint{1.207694in}{2.219041in}}%
\pgfpathcurveto{\pgfqpoint{1.207694in}{2.227277in}}{\pgfqpoint{1.204422in}{2.235178in}}{\pgfqpoint{1.198598in}{2.241001in}}%
\pgfpathcurveto{\pgfqpoint{1.192774in}{2.246825in}}{\pgfqpoint{1.184874in}{2.250098in}}{\pgfqpoint{1.176638in}{2.250098in}}%
\pgfpathcurveto{\pgfqpoint{1.168401in}{2.250098in}}{\pgfqpoint{1.160501in}{2.246825in}}{\pgfqpoint{1.154677in}{2.241001in}}%
\pgfpathcurveto{\pgfqpoint{1.148853in}{2.235178in}}{\pgfqpoint{1.145581in}{2.227277in}}{\pgfqpoint{1.145581in}{2.219041in}}%
\pgfpathcurveto{\pgfqpoint{1.145581in}{2.210805in}}{\pgfqpoint{1.148853in}{2.202905in}}{\pgfqpoint{1.154677in}{2.197081in}}%
\pgfpathcurveto{\pgfqpoint{1.160501in}{2.191257in}}{\pgfqpoint{1.168401in}{2.187985in}}{\pgfqpoint{1.176638in}{2.187985in}}%
\pgfpathlineto{\pgfqpoint{1.176638in}{2.187985in}}%
\pgfpathclose%
\pgfusepath{stroke}%
\end{pgfscope}%
\begin{pgfscope}%
\pgfpathrectangle{\pgfqpoint{0.688192in}{0.670138in}}{\pgfqpoint{6.200000in}{4.620000in}}%
\pgfusepath{clip}%
\pgfsetbuttcap%
\pgfsetroundjoin%
\pgfsetlinewidth{1.003750pt}%
\definecolor{currentstroke}{rgb}{1.000000,0.000000,0.000000}%
\pgfsetstrokecolor{currentstroke}%
\pgfsetdash{}{0pt}%
\pgfpathmoveto{\pgfqpoint{1.160265in}{1.939369in}}%
\pgfpathcurveto{\pgfqpoint{1.168501in}{1.939369in}}{\pgfqpoint{1.176401in}{1.942641in}}{\pgfqpoint{1.182225in}{1.948465in}}%
\pgfpathcurveto{\pgfqpoint{1.188049in}{1.954289in}}{\pgfqpoint{1.191321in}{1.962189in}}{\pgfqpoint{1.191321in}{1.970426in}}%
\pgfpathcurveto{\pgfqpoint{1.191321in}{1.978662in}}{\pgfqpoint{1.188049in}{1.986562in}}{\pgfqpoint{1.182225in}{1.992386in}}%
\pgfpathcurveto{\pgfqpoint{1.176401in}{1.998210in}}{\pgfqpoint{1.168501in}{2.001482in}}{\pgfqpoint{1.160265in}{2.001482in}}%
\pgfpathcurveto{\pgfqpoint{1.152029in}{2.001482in}}{\pgfqpoint{1.144129in}{1.998210in}}{\pgfqpoint{1.138305in}{1.992386in}}%
\pgfpathcurveto{\pgfqpoint{1.132481in}{1.986562in}}{\pgfqpoint{1.129208in}{1.978662in}}{\pgfqpoint{1.129208in}{1.970426in}}%
\pgfpathcurveto{\pgfqpoint{1.129208in}{1.962189in}}{\pgfqpoint{1.132481in}{1.954289in}}{\pgfqpoint{1.138305in}{1.948465in}}%
\pgfpathcurveto{\pgfqpoint{1.144129in}{1.942641in}}{\pgfqpoint{1.152029in}{1.939369in}}{\pgfqpoint{1.160265in}{1.939369in}}%
\pgfpathlineto{\pgfqpoint{1.160265in}{1.939369in}}%
\pgfpathclose%
\pgfusepath{stroke}%
\end{pgfscope}%
\begin{pgfscope}%
\pgfpathrectangle{\pgfqpoint{0.688192in}{0.670138in}}{\pgfqpoint{6.200000in}{4.620000in}}%
\pgfusepath{clip}%
\pgfsetbuttcap%
\pgfsetroundjoin%
\pgfsetlinewidth{1.003750pt}%
\definecolor{currentstroke}{rgb}{1.000000,0.000000,0.000000}%
\pgfsetstrokecolor{currentstroke}%
\pgfsetdash{}{0pt}%
\pgfpathmoveto{\pgfqpoint{1.160053in}{1.935902in}}%
\pgfpathcurveto{\pgfqpoint{1.168289in}{1.935902in}}{\pgfqpoint{1.176189in}{1.939174in}}{\pgfqpoint{1.182013in}{1.944998in}}%
\pgfpathcurveto{\pgfqpoint{1.187837in}{1.950822in}}{\pgfqpoint{1.191109in}{1.958722in}}{\pgfqpoint{1.191109in}{1.966959in}}%
\pgfpathcurveto{\pgfqpoint{1.191109in}{1.975195in}}{\pgfqpoint{1.187837in}{1.983095in}}{\pgfqpoint{1.182013in}{1.988919in}}%
\pgfpathcurveto{\pgfqpoint{1.176189in}{1.994743in}}{\pgfqpoint{1.168289in}{1.998015in}}{\pgfqpoint{1.160053in}{1.998015in}}%
\pgfpathcurveto{\pgfqpoint{1.151817in}{1.998015in}}{\pgfqpoint{1.143916in}{1.994743in}}{\pgfqpoint{1.138093in}{1.988919in}}%
\pgfpathcurveto{\pgfqpoint{1.132269in}{1.983095in}}{\pgfqpoint{1.128996in}{1.975195in}}{\pgfqpoint{1.128996in}{1.966959in}}%
\pgfpathcurveto{\pgfqpoint{1.128996in}{1.958722in}}{\pgfqpoint{1.132269in}{1.950822in}}{\pgfqpoint{1.138093in}{1.944998in}}%
\pgfpathcurveto{\pgfqpoint{1.143916in}{1.939174in}}{\pgfqpoint{1.151817in}{1.935902in}}{\pgfqpoint{1.160053in}{1.935902in}}%
\pgfpathlineto{\pgfqpoint{1.160053in}{1.935902in}}%
\pgfpathclose%
\pgfusepath{stroke}%
\end{pgfscope}%
\begin{pgfscope}%
\pgfpathrectangle{\pgfqpoint{0.688192in}{0.670138in}}{\pgfqpoint{6.200000in}{4.620000in}}%
\pgfusepath{clip}%
\pgfsetbuttcap%
\pgfsetroundjoin%
\pgfsetlinewidth{1.003750pt}%
\definecolor{currentstroke}{rgb}{1.000000,0.000000,0.000000}%
\pgfsetstrokecolor{currentstroke}%
\pgfsetdash{}{0pt}%
\pgfpathmoveto{\pgfqpoint{1.157986in}{1.610198in}}%
\pgfpathcurveto{\pgfqpoint{1.166223in}{1.610198in}}{\pgfqpoint{1.174123in}{1.613470in}}{\pgfqpoint{1.179947in}{1.619294in}}%
\pgfpathcurveto{\pgfqpoint{1.185771in}{1.625118in}}{\pgfqpoint{1.189043in}{1.633018in}}{\pgfqpoint{1.189043in}{1.641254in}}%
\pgfpathcurveto{\pgfqpoint{1.189043in}{1.649491in}}{\pgfqpoint{1.185771in}{1.657391in}}{\pgfqpoint{1.179947in}{1.663215in}}%
\pgfpathcurveto{\pgfqpoint{1.174123in}{1.669038in}}{\pgfqpoint{1.166223in}{1.672311in}}{\pgfqpoint{1.157986in}{1.672311in}}%
\pgfpathcurveto{\pgfqpoint{1.149750in}{1.672311in}}{\pgfqpoint{1.141850in}{1.669038in}}{\pgfqpoint{1.136026in}{1.663215in}}%
\pgfpathcurveto{\pgfqpoint{1.130202in}{1.657391in}}{\pgfqpoint{1.126930in}{1.649491in}}{\pgfqpoint{1.126930in}{1.641254in}}%
\pgfpathcurveto{\pgfqpoint{1.126930in}{1.633018in}}{\pgfqpoint{1.130202in}{1.625118in}}{\pgfqpoint{1.136026in}{1.619294in}}%
\pgfpathcurveto{\pgfqpoint{1.141850in}{1.613470in}}{\pgfqpoint{1.149750in}{1.610198in}}{\pgfqpoint{1.157986in}{1.610198in}}%
\pgfpathlineto{\pgfqpoint{1.157986in}{1.610198in}}%
\pgfpathclose%
\pgfusepath{stroke}%
\end{pgfscope}%
\begin{pgfscope}%
\pgfpathrectangle{\pgfqpoint{0.688192in}{0.670138in}}{\pgfqpoint{6.200000in}{4.620000in}}%
\pgfusepath{clip}%
\pgfsetbuttcap%
\pgfsetroundjoin%
\pgfsetlinewidth{1.003750pt}%
\definecolor{currentstroke}{rgb}{1.000000,0.000000,0.000000}%
\pgfsetstrokecolor{currentstroke}%
\pgfsetdash{}{0pt}%
\pgfpathmoveto{\pgfqpoint{1.155798in}{1.612752in}}%
\pgfpathcurveto{\pgfqpoint{1.164035in}{1.612752in}}{\pgfqpoint{1.171935in}{1.616024in}}{\pgfqpoint{1.177759in}{1.621848in}}%
\pgfpathcurveto{\pgfqpoint{1.183582in}{1.627672in}}{\pgfqpoint{1.186855in}{1.635572in}}{\pgfqpoint{1.186855in}{1.643808in}}%
\pgfpathcurveto{\pgfqpoint{1.186855in}{1.652044in}}{\pgfqpoint{1.183582in}{1.659945in}}{\pgfqpoint{1.177759in}{1.665768in}}%
\pgfpathcurveto{\pgfqpoint{1.171935in}{1.671592in}}{\pgfqpoint{1.164035in}{1.674865in}}{\pgfqpoint{1.155798in}{1.674865in}}%
\pgfpathcurveto{\pgfqpoint{1.147562in}{1.674865in}}{\pgfqpoint{1.139662in}{1.671592in}}{\pgfqpoint{1.133838in}{1.665768in}}%
\pgfpathcurveto{\pgfqpoint{1.128014in}{1.659945in}}{\pgfqpoint{1.124742in}{1.652044in}}{\pgfqpoint{1.124742in}{1.643808in}}%
\pgfpathcurveto{\pgfqpoint{1.124742in}{1.635572in}}{\pgfqpoint{1.128014in}{1.627672in}}{\pgfqpoint{1.133838in}{1.621848in}}%
\pgfpathcurveto{\pgfqpoint{1.139662in}{1.616024in}}{\pgfqpoint{1.147562in}{1.612752in}}{\pgfqpoint{1.155798in}{1.612752in}}%
\pgfpathlineto{\pgfqpoint{1.155798in}{1.612752in}}%
\pgfpathclose%
\pgfusepath{stroke}%
\end{pgfscope}%
\begin{pgfscope}%
\pgfpathrectangle{\pgfqpoint{0.688192in}{0.670138in}}{\pgfqpoint{6.200000in}{4.620000in}}%
\pgfusepath{clip}%
\pgfsetbuttcap%
\pgfsetroundjoin%
\pgfsetlinewidth{1.003750pt}%
\definecolor{currentstroke}{rgb}{1.000000,0.000000,0.000000}%
\pgfsetstrokecolor{currentstroke}%
\pgfsetdash{}{0pt}%
\pgfpathmoveto{\pgfqpoint{1.295347in}{1.895177in}}%
\pgfpathcurveto{\pgfqpoint{1.303583in}{1.895177in}}{\pgfqpoint{1.311483in}{1.898449in}}{\pgfqpoint{1.317307in}{1.904273in}}%
\pgfpathcurveto{\pgfqpoint{1.323131in}{1.910097in}}{\pgfqpoint{1.326403in}{1.917997in}}{\pgfqpoint{1.326403in}{1.926233in}}%
\pgfpathcurveto{\pgfqpoint{1.326403in}{1.934470in}}{\pgfqpoint{1.323131in}{1.942370in}}{\pgfqpoint{1.317307in}{1.948193in}}%
\pgfpathcurveto{\pgfqpoint{1.311483in}{1.954017in}}{\pgfqpoint{1.303583in}{1.957290in}}{\pgfqpoint{1.295347in}{1.957290in}}%
\pgfpathcurveto{\pgfqpoint{1.287111in}{1.957290in}}{\pgfqpoint{1.279211in}{1.954017in}}{\pgfqpoint{1.273387in}{1.948193in}}%
\pgfpathcurveto{\pgfqpoint{1.267563in}{1.942370in}}{\pgfqpoint{1.264290in}{1.934470in}}{\pgfqpoint{1.264290in}{1.926233in}}%
\pgfpathcurveto{\pgfqpoint{1.264290in}{1.917997in}}{\pgfqpoint{1.267563in}{1.910097in}}{\pgfqpoint{1.273387in}{1.904273in}}%
\pgfpathcurveto{\pgfqpoint{1.279211in}{1.898449in}}{\pgfqpoint{1.287111in}{1.895177in}}{\pgfqpoint{1.295347in}{1.895177in}}%
\pgfpathlineto{\pgfqpoint{1.295347in}{1.895177in}}%
\pgfpathclose%
\pgfusepath{stroke}%
\end{pgfscope}%
\begin{pgfscope}%
\pgfpathrectangle{\pgfqpoint{0.688192in}{0.670138in}}{\pgfqpoint{6.200000in}{4.620000in}}%
\pgfusepath{clip}%
\pgfsetbuttcap%
\pgfsetroundjoin%
\pgfsetlinewidth{1.003750pt}%
\definecolor{currentstroke}{rgb}{1.000000,0.000000,0.000000}%
\pgfsetstrokecolor{currentstroke}%
\pgfsetdash{}{0pt}%
\pgfpathmoveto{\pgfqpoint{1.160535in}{1.595409in}}%
\pgfpathcurveto{\pgfqpoint{1.168772in}{1.595409in}}{\pgfqpoint{1.176672in}{1.598682in}}{\pgfqpoint{1.182496in}{1.604506in}}%
\pgfpathcurveto{\pgfqpoint{1.188320in}{1.610330in}}{\pgfqpoint{1.191592in}{1.618230in}}{\pgfqpoint{1.191592in}{1.626466in}}%
\pgfpathcurveto{\pgfqpoint{1.191592in}{1.634702in}}{\pgfqpoint{1.188320in}{1.642602in}}{\pgfqpoint{1.182496in}{1.648426in}}%
\pgfpathcurveto{\pgfqpoint{1.176672in}{1.654250in}}{\pgfqpoint{1.168772in}{1.657522in}}{\pgfqpoint{1.160535in}{1.657522in}}%
\pgfpathcurveto{\pgfqpoint{1.152299in}{1.657522in}}{\pgfqpoint{1.144399in}{1.654250in}}{\pgfqpoint{1.138575in}{1.648426in}}%
\pgfpathcurveto{\pgfqpoint{1.132751in}{1.642602in}}{\pgfqpoint{1.129479in}{1.634702in}}{\pgfqpoint{1.129479in}{1.626466in}}%
\pgfpathcurveto{\pgfqpoint{1.129479in}{1.618230in}}{\pgfqpoint{1.132751in}{1.610330in}}{\pgfqpoint{1.138575in}{1.604506in}}%
\pgfpathcurveto{\pgfqpoint{1.144399in}{1.598682in}}{\pgfqpoint{1.152299in}{1.595409in}}{\pgfqpoint{1.160535in}{1.595409in}}%
\pgfpathlineto{\pgfqpoint{1.160535in}{1.595409in}}%
\pgfpathclose%
\pgfusepath{stroke}%
\end{pgfscope}%
\begin{pgfscope}%
\pgfpathrectangle{\pgfqpoint{0.688192in}{0.670138in}}{\pgfqpoint{6.200000in}{4.620000in}}%
\pgfusepath{clip}%
\pgfsetbuttcap%
\pgfsetroundjoin%
\pgfsetlinewidth{1.003750pt}%
\definecolor{currentstroke}{rgb}{1.000000,0.000000,0.000000}%
\pgfsetstrokecolor{currentstroke}%
\pgfsetdash{}{0pt}%
\pgfpathmoveto{\pgfqpoint{1.265571in}{1.691299in}}%
\pgfpathcurveto{\pgfqpoint{1.273807in}{1.691299in}}{\pgfqpoint{1.281707in}{1.694572in}}{\pgfqpoint{1.287531in}{1.700396in}}%
\pgfpathcurveto{\pgfqpoint{1.293355in}{1.706220in}}{\pgfqpoint{1.296627in}{1.714120in}}{\pgfqpoint{1.296627in}{1.722356in}}%
\pgfpathcurveto{\pgfqpoint{1.296627in}{1.730592in}}{\pgfqpoint{1.293355in}{1.738492in}}{\pgfqpoint{1.287531in}{1.744316in}}%
\pgfpathcurveto{\pgfqpoint{1.281707in}{1.750140in}}{\pgfqpoint{1.273807in}{1.753412in}}{\pgfqpoint{1.265571in}{1.753412in}}%
\pgfpathcurveto{\pgfqpoint{1.257335in}{1.753412in}}{\pgfqpoint{1.249435in}{1.750140in}}{\pgfqpoint{1.243611in}{1.744316in}}%
\pgfpathcurveto{\pgfqpoint{1.237787in}{1.738492in}}{\pgfqpoint{1.234514in}{1.730592in}}{\pgfqpoint{1.234514in}{1.722356in}}%
\pgfpathcurveto{\pgfqpoint{1.234514in}{1.714120in}}{\pgfqpoint{1.237787in}{1.706220in}}{\pgfqpoint{1.243611in}{1.700396in}}%
\pgfpathcurveto{\pgfqpoint{1.249435in}{1.694572in}}{\pgfqpoint{1.257335in}{1.691299in}}{\pgfqpoint{1.265571in}{1.691299in}}%
\pgfpathlineto{\pgfqpoint{1.265571in}{1.691299in}}%
\pgfpathclose%
\pgfusepath{stroke}%
\end{pgfscope}%
\begin{pgfscope}%
\pgfpathrectangle{\pgfqpoint{0.688192in}{0.670138in}}{\pgfqpoint{6.200000in}{4.620000in}}%
\pgfusepath{clip}%
\pgfsetbuttcap%
\pgfsetroundjoin%
\pgfsetlinewidth{1.003750pt}%
\definecolor{currentstroke}{rgb}{1.000000,0.000000,0.000000}%
\pgfsetstrokecolor{currentstroke}%
\pgfsetdash{}{0pt}%
\pgfpathmoveto{\pgfqpoint{1.297728in}{1.636949in}}%
\pgfpathcurveto{\pgfqpoint{1.305964in}{1.636949in}}{\pgfqpoint{1.313864in}{1.640222in}}{\pgfqpoint{1.319688in}{1.646045in}}%
\pgfpathcurveto{\pgfqpoint{1.325512in}{1.651869in}}{\pgfqpoint{1.328784in}{1.659769in}}{\pgfqpoint{1.328784in}{1.668006in}}%
\pgfpathcurveto{\pgfqpoint{1.328784in}{1.676242in}}{\pgfqpoint{1.325512in}{1.684142in}}{\pgfqpoint{1.319688in}{1.689966in}}%
\pgfpathcurveto{\pgfqpoint{1.313864in}{1.695790in}}{\pgfqpoint{1.305964in}{1.699062in}}{\pgfqpoint{1.297728in}{1.699062in}}%
\pgfpathcurveto{\pgfqpoint{1.289491in}{1.699062in}}{\pgfqpoint{1.281591in}{1.695790in}}{\pgfqpoint{1.275767in}{1.689966in}}%
\pgfpathcurveto{\pgfqpoint{1.269943in}{1.684142in}}{\pgfqpoint{1.266671in}{1.676242in}}{\pgfqpoint{1.266671in}{1.668006in}}%
\pgfpathcurveto{\pgfqpoint{1.266671in}{1.659769in}}{\pgfqpoint{1.269943in}{1.651869in}}{\pgfqpoint{1.275767in}{1.646045in}}%
\pgfpathcurveto{\pgfqpoint{1.281591in}{1.640222in}}{\pgfqpoint{1.289491in}{1.636949in}}{\pgfqpoint{1.297728in}{1.636949in}}%
\pgfpathlineto{\pgfqpoint{1.297728in}{1.636949in}}%
\pgfpathclose%
\pgfusepath{stroke}%
\end{pgfscope}%
\begin{pgfscope}%
\pgfpathrectangle{\pgfqpoint{0.688192in}{0.670138in}}{\pgfqpoint{6.200000in}{4.620000in}}%
\pgfusepath{clip}%
\pgfsetbuttcap%
\pgfsetroundjoin%
\pgfsetlinewidth{1.003750pt}%
\definecolor{currentstroke}{rgb}{1.000000,0.000000,0.000000}%
\pgfsetstrokecolor{currentstroke}%
\pgfsetdash{}{0pt}%
\pgfpathmoveto{\pgfqpoint{1.147178in}{1.734556in}}%
\pgfpathcurveto{\pgfqpoint{1.155415in}{1.734556in}}{\pgfqpoint{1.163315in}{1.737828in}}{\pgfqpoint{1.169139in}{1.743652in}}%
\pgfpathcurveto{\pgfqpoint{1.174962in}{1.749476in}}{\pgfqpoint{1.178235in}{1.757376in}}{\pgfqpoint{1.178235in}{1.765612in}}%
\pgfpathcurveto{\pgfqpoint{1.178235in}{1.773848in}}{\pgfqpoint{1.174962in}{1.781749in}}{\pgfqpoint{1.169139in}{1.787572in}}%
\pgfpathcurveto{\pgfqpoint{1.163315in}{1.793396in}}{\pgfqpoint{1.155415in}{1.796669in}}{\pgfqpoint{1.147178in}{1.796669in}}%
\pgfpathcurveto{\pgfqpoint{1.138942in}{1.796669in}}{\pgfqpoint{1.131042in}{1.793396in}}{\pgfqpoint{1.125218in}{1.787572in}}%
\pgfpathcurveto{\pgfqpoint{1.119394in}{1.781749in}}{\pgfqpoint{1.116122in}{1.773848in}}{\pgfqpoint{1.116122in}{1.765612in}}%
\pgfpathcurveto{\pgfqpoint{1.116122in}{1.757376in}}{\pgfqpoint{1.119394in}{1.749476in}}{\pgfqpoint{1.125218in}{1.743652in}}%
\pgfpathcurveto{\pgfqpoint{1.131042in}{1.737828in}}{\pgfqpoint{1.138942in}{1.734556in}}{\pgfqpoint{1.147178in}{1.734556in}}%
\pgfpathlineto{\pgfqpoint{1.147178in}{1.734556in}}%
\pgfpathclose%
\pgfusepath{stroke}%
\end{pgfscope}%
\begin{pgfscope}%
\pgfpathrectangle{\pgfqpoint{0.688192in}{0.670138in}}{\pgfqpoint{6.200000in}{4.620000in}}%
\pgfusepath{clip}%
\pgfsetbuttcap%
\pgfsetroundjoin%
\pgfsetlinewidth{1.003750pt}%
\definecolor{currentstroke}{rgb}{1.000000,0.000000,0.000000}%
\pgfsetstrokecolor{currentstroke}%
\pgfsetdash{}{0pt}%
\pgfpathmoveto{\pgfqpoint{1.240097in}{1.549585in}}%
\pgfpathcurveto{\pgfqpoint{1.248334in}{1.549585in}}{\pgfqpoint{1.256234in}{1.552857in}}{\pgfqpoint{1.262058in}{1.558681in}}%
\pgfpathcurveto{\pgfqpoint{1.267881in}{1.564505in}}{\pgfqpoint{1.271154in}{1.572405in}}{\pgfqpoint{1.271154in}{1.580642in}}%
\pgfpathcurveto{\pgfqpoint{1.271154in}{1.588878in}}{\pgfqpoint{1.267881in}{1.596778in}}{\pgfqpoint{1.262058in}{1.602602in}}%
\pgfpathcurveto{\pgfqpoint{1.256234in}{1.608426in}}{\pgfqpoint{1.248334in}{1.611698in}}{\pgfqpoint{1.240097in}{1.611698in}}%
\pgfpathcurveto{\pgfqpoint{1.231861in}{1.611698in}}{\pgfqpoint{1.223961in}{1.608426in}}{\pgfqpoint{1.218137in}{1.602602in}}%
\pgfpathcurveto{\pgfqpoint{1.212313in}{1.596778in}}{\pgfqpoint{1.209041in}{1.588878in}}{\pgfqpoint{1.209041in}{1.580642in}}%
\pgfpathcurveto{\pgfqpoint{1.209041in}{1.572405in}}{\pgfqpoint{1.212313in}{1.564505in}}{\pgfqpoint{1.218137in}{1.558681in}}%
\pgfpathcurveto{\pgfqpoint{1.223961in}{1.552857in}}{\pgfqpoint{1.231861in}{1.549585in}}{\pgfqpoint{1.240097in}{1.549585in}}%
\pgfpathlineto{\pgfqpoint{1.240097in}{1.549585in}}%
\pgfpathclose%
\pgfusepath{stroke}%
\end{pgfscope}%
\begin{pgfscope}%
\pgfpathrectangle{\pgfqpoint{0.688192in}{0.670138in}}{\pgfqpoint{6.200000in}{4.620000in}}%
\pgfusepath{clip}%
\pgfsetbuttcap%
\pgfsetroundjoin%
\pgfsetlinewidth{1.003750pt}%
\definecolor{currentstroke}{rgb}{1.000000,0.000000,0.000000}%
\pgfsetstrokecolor{currentstroke}%
\pgfsetdash{}{0pt}%
\pgfpathmoveto{\pgfqpoint{1.275017in}{1.474367in}}%
\pgfpathcurveto{\pgfqpoint{1.283253in}{1.474367in}}{\pgfqpoint{1.291153in}{1.477639in}}{\pgfqpoint{1.296977in}{1.483463in}}%
\pgfpathcurveto{\pgfqpoint{1.302801in}{1.489287in}}{\pgfqpoint{1.306073in}{1.497187in}}{\pgfqpoint{1.306073in}{1.505423in}}%
\pgfpathcurveto{\pgfqpoint{1.306073in}{1.513660in}}{\pgfqpoint{1.302801in}{1.521560in}}{\pgfqpoint{1.296977in}{1.527384in}}%
\pgfpathcurveto{\pgfqpoint{1.291153in}{1.533208in}}{\pgfqpoint{1.283253in}{1.536480in}}{\pgfqpoint{1.275017in}{1.536480in}}%
\pgfpathcurveto{\pgfqpoint{1.266780in}{1.536480in}}{\pgfqpoint{1.258880in}{1.533208in}}{\pgfqpoint{1.253056in}{1.527384in}}%
\pgfpathcurveto{\pgfqpoint{1.247232in}{1.521560in}}{\pgfqpoint{1.243960in}{1.513660in}}{\pgfqpoint{1.243960in}{1.505423in}}%
\pgfpathcurveto{\pgfqpoint{1.243960in}{1.497187in}}{\pgfqpoint{1.247232in}{1.489287in}}{\pgfqpoint{1.253056in}{1.483463in}}%
\pgfpathcurveto{\pgfqpoint{1.258880in}{1.477639in}}{\pgfqpoint{1.266780in}{1.474367in}}{\pgfqpoint{1.275017in}{1.474367in}}%
\pgfpathlineto{\pgfqpoint{1.275017in}{1.474367in}}%
\pgfpathclose%
\pgfusepath{stroke}%
\end{pgfscope}%
\begin{pgfscope}%
\pgfpathrectangle{\pgfqpoint{0.688192in}{0.670138in}}{\pgfqpoint{6.200000in}{4.620000in}}%
\pgfusepath{clip}%
\pgfsetbuttcap%
\pgfsetroundjoin%
\pgfsetlinewidth{1.003750pt}%
\definecolor{currentstroke}{rgb}{1.000000,0.000000,0.000000}%
\pgfsetstrokecolor{currentstroke}%
\pgfsetdash{}{0pt}%
\pgfpathmoveto{\pgfqpoint{1.274070in}{1.506532in}}%
\pgfpathcurveto{\pgfqpoint{1.282306in}{1.506532in}}{\pgfqpoint{1.290206in}{1.509804in}}{\pgfqpoint{1.296030in}{1.515628in}}%
\pgfpathcurveto{\pgfqpoint{1.301854in}{1.521452in}}{\pgfqpoint{1.305126in}{1.529352in}}{\pgfqpoint{1.305126in}{1.537589in}}%
\pgfpathcurveto{\pgfqpoint{1.305126in}{1.545825in}}{\pgfqpoint{1.301854in}{1.553725in}}{\pgfqpoint{1.296030in}{1.559549in}}%
\pgfpathcurveto{\pgfqpoint{1.290206in}{1.565373in}}{\pgfqpoint{1.282306in}{1.568645in}}{\pgfqpoint{1.274070in}{1.568645in}}%
\pgfpathcurveto{\pgfqpoint{1.265833in}{1.568645in}}{\pgfqpoint{1.257933in}{1.565373in}}{\pgfqpoint{1.252109in}{1.559549in}}%
\pgfpathcurveto{\pgfqpoint{1.246285in}{1.553725in}}{\pgfqpoint{1.243013in}{1.545825in}}{\pgfqpoint{1.243013in}{1.537589in}}%
\pgfpathcurveto{\pgfqpoint{1.243013in}{1.529352in}}{\pgfqpoint{1.246285in}{1.521452in}}{\pgfqpoint{1.252109in}{1.515628in}}%
\pgfpathcurveto{\pgfqpoint{1.257933in}{1.509804in}}{\pgfqpoint{1.265833in}{1.506532in}}{\pgfqpoint{1.274070in}{1.506532in}}%
\pgfpathlineto{\pgfqpoint{1.274070in}{1.506532in}}%
\pgfpathclose%
\pgfusepath{stroke}%
\end{pgfscope}%
\begin{pgfscope}%
\pgfpathrectangle{\pgfqpoint{0.688192in}{0.670138in}}{\pgfqpoint{6.200000in}{4.620000in}}%
\pgfusepath{clip}%
\pgfsetbuttcap%
\pgfsetroundjoin%
\pgfsetlinewidth{1.003750pt}%
\definecolor{currentstroke}{rgb}{1.000000,0.000000,0.000000}%
\pgfsetstrokecolor{currentstroke}%
\pgfsetdash{}{0pt}%
\pgfpathmoveto{\pgfqpoint{1.147538in}{1.722969in}}%
\pgfpathcurveto{\pgfqpoint{1.155775in}{1.722969in}}{\pgfqpoint{1.163675in}{1.726242in}}{\pgfqpoint{1.169499in}{1.732066in}}%
\pgfpathcurveto{\pgfqpoint{1.175323in}{1.737890in}}{\pgfqpoint{1.178595in}{1.745790in}}{\pgfqpoint{1.178595in}{1.754026in}}%
\pgfpathcurveto{\pgfqpoint{1.178595in}{1.762262in}}{\pgfqpoint{1.175323in}{1.770162in}}{\pgfqpoint{1.169499in}{1.775986in}}%
\pgfpathcurveto{\pgfqpoint{1.163675in}{1.781810in}}{\pgfqpoint{1.155775in}{1.785082in}}{\pgfqpoint{1.147538in}{1.785082in}}%
\pgfpathcurveto{\pgfqpoint{1.139302in}{1.785082in}}{\pgfqpoint{1.131402in}{1.781810in}}{\pgfqpoint{1.125578in}{1.775986in}}%
\pgfpathcurveto{\pgfqpoint{1.119754in}{1.770162in}}{\pgfqpoint{1.116482in}{1.762262in}}{\pgfqpoint{1.116482in}{1.754026in}}%
\pgfpathcurveto{\pgfqpoint{1.116482in}{1.745790in}}{\pgfqpoint{1.119754in}{1.737890in}}{\pgfqpoint{1.125578in}{1.732066in}}%
\pgfpathcurveto{\pgfqpoint{1.131402in}{1.726242in}}{\pgfqpoint{1.139302in}{1.722969in}}{\pgfqpoint{1.147538in}{1.722969in}}%
\pgfpathlineto{\pgfqpoint{1.147538in}{1.722969in}}%
\pgfpathclose%
\pgfusepath{stroke}%
\end{pgfscope}%
\begin{pgfscope}%
\pgfpathrectangle{\pgfqpoint{0.688192in}{0.670138in}}{\pgfqpoint{6.200000in}{4.620000in}}%
\pgfusepath{clip}%
\pgfsetbuttcap%
\pgfsetroundjoin%
\pgfsetlinewidth{1.003750pt}%
\definecolor{currentstroke}{rgb}{1.000000,0.000000,0.000000}%
\pgfsetstrokecolor{currentstroke}%
\pgfsetdash{}{0pt}%
\pgfpathmoveto{\pgfqpoint{1.194960in}{1.895058in}}%
\pgfpathcurveto{\pgfqpoint{1.203197in}{1.895058in}}{\pgfqpoint{1.211097in}{1.898330in}}{\pgfqpoint{1.216921in}{1.904154in}}%
\pgfpathcurveto{\pgfqpoint{1.222745in}{1.909978in}}{\pgfqpoint{1.226017in}{1.917878in}}{\pgfqpoint{1.226017in}{1.926114in}}%
\pgfpathcurveto{\pgfqpoint{1.226017in}{1.934351in}}{\pgfqpoint{1.222745in}{1.942251in}}{\pgfqpoint{1.216921in}{1.948075in}}%
\pgfpathcurveto{\pgfqpoint{1.211097in}{1.953898in}}{\pgfqpoint{1.203197in}{1.957171in}}{\pgfqpoint{1.194960in}{1.957171in}}%
\pgfpathcurveto{\pgfqpoint{1.186724in}{1.957171in}}{\pgfqpoint{1.178824in}{1.953898in}}{\pgfqpoint{1.173000in}{1.948075in}}%
\pgfpathcurveto{\pgfqpoint{1.167176in}{1.942251in}}{\pgfqpoint{1.163904in}{1.934351in}}{\pgfqpoint{1.163904in}{1.926114in}}%
\pgfpathcurveto{\pgfqpoint{1.163904in}{1.917878in}}{\pgfqpoint{1.167176in}{1.909978in}}{\pgfqpoint{1.173000in}{1.904154in}}%
\pgfpathcurveto{\pgfqpoint{1.178824in}{1.898330in}}{\pgfqpoint{1.186724in}{1.895058in}}{\pgfqpoint{1.194960in}{1.895058in}}%
\pgfpathlineto{\pgfqpoint{1.194960in}{1.895058in}}%
\pgfpathclose%
\pgfusepath{stroke}%
\end{pgfscope}%
\begin{pgfscope}%
\pgfpathrectangle{\pgfqpoint{0.688192in}{0.670138in}}{\pgfqpoint{6.200000in}{4.620000in}}%
\pgfusepath{clip}%
\pgfsetbuttcap%
\pgfsetroundjoin%
\pgfsetlinewidth{1.003750pt}%
\definecolor{currentstroke}{rgb}{1.000000,0.000000,0.000000}%
\pgfsetstrokecolor{currentstroke}%
\pgfsetdash{}{0pt}%
\pgfpathmoveto{\pgfqpoint{1.164778in}{1.905975in}}%
\pgfpathcurveto{\pgfqpoint{1.173015in}{1.905975in}}{\pgfqpoint{1.180915in}{1.909248in}}{\pgfqpoint{1.186738in}{1.915072in}}%
\pgfpathcurveto{\pgfqpoint{1.192562in}{1.920896in}}{\pgfqpoint{1.195835in}{1.928796in}}{\pgfqpoint{1.195835in}{1.937032in}}%
\pgfpathcurveto{\pgfqpoint{1.195835in}{1.945268in}}{\pgfqpoint{1.192562in}{1.953168in}}{\pgfqpoint{1.186738in}{1.958992in}}%
\pgfpathcurveto{\pgfqpoint{1.180915in}{1.964816in}}{\pgfqpoint{1.173015in}{1.968088in}}{\pgfqpoint{1.164778in}{1.968088in}}%
\pgfpathcurveto{\pgfqpoint{1.156542in}{1.968088in}}{\pgfqpoint{1.148642in}{1.964816in}}{\pgfqpoint{1.142818in}{1.958992in}}%
\pgfpathcurveto{\pgfqpoint{1.136994in}{1.953168in}}{\pgfqpoint{1.133722in}{1.945268in}}{\pgfqpoint{1.133722in}{1.937032in}}%
\pgfpathcurveto{\pgfqpoint{1.133722in}{1.928796in}}{\pgfqpoint{1.136994in}{1.920896in}}{\pgfqpoint{1.142818in}{1.915072in}}%
\pgfpathcurveto{\pgfqpoint{1.148642in}{1.909248in}}{\pgfqpoint{1.156542in}{1.905975in}}{\pgfqpoint{1.164778in}{1.905975in}}%
\pgfpathlineto{\pgfqpoint{1.164778in}{1.905975in}}%
\pgfpathclose%
\pgfusepath{stroke}%
\end{pgfscope}%
\begin{pgfscope}%
\pgfpathrectangle{\pgfqpoint{0.688192in}{0.670138in}}{\pgfqpoint{6.200000in}{4.620000in}}%
\pgfusepath{clip}%
\pgfsetbuttcap%
\pgfsetroundjoin%
\pgfsetlinewidth{1.003750pt}%
\definecolor{currentstroke}{rgb}{1.000000,0.000000,0.000000}%
\pgfsetstrokecolor{currentstroke}%
\pgfsetdash{}{0pt}%
\pgfpathmoveto{\pgfqpoint{1.143899in}{1.731794in}}%
\pgfpathcurveto{\pgfqpoint{1.152135in}{1.731794in}}{\pgfqpoint{1.160035in}{1.735067in}}{\pgfqpoint{1.165859in}{1.740890in}}%
\pgfpathcurveto{\pgfqpoint{1.171683in}{1.746714in}}{\pgfqpoint{1.174956in}{1.754614in}}{\pgfqpoint{1.174956in}{1.762851in}}%
\pgfpathcurveto{\pgfqpoint{1.174956in}{1.771087in}}{\pgfqpoint{1.171683in}{1.778987in}}{\pgfqpoint{1.165859in}{1.784811in}}%
\pgfpathcurveto{\pgfqpoint{1.160035in}{1.790635in}}{\pgfqpoint{1.152135in}{1.793907in}}{\pgfqpoint{1.143899in}{1.793907in}}%
\pgfpathcurveto{\pgfqpoint{1.135663in}{1.793907in}}{\pgfqpoint{1.127763in}{1.790635in}}{\pgfqpoint{1.121939in}{1.784811in}}%
\pgfpathcurveto{\pgfqpoint{1.116115in}{1.778987in}}{\pgfqpoint{1.112843in}{1.771087in}}{\pgfqpoint{1.112843in}{1.762851in}}%
\pgfpathcurveto{\pgfqpoint{1.112843in}{1.754614in}}{\pgfqpoint{1.116115in}{1.746714in}}{\pgfqpoint{1.121939in}{1.740890in}}%
\pgfpathcurveto{\pgfqpoint{1.127763in}{1.735067in}}{\pgfqpoint{1.135663in}{1.731794in}}{\pgfqpoint{1.143899in}{1.731794in}}%
\pgfpathlineto{\pgfqpoint{1.143899in}{1.731794in}}%
\pgfpathclose%
\pgfusepath{stroke}%
\end{pgfscope}%
\begin{pgfscope}%
\pgfpathrectangle{\pgfqpoint{0.688192in}{0.670138in}}{\pgfqpoint{6.200000in}{4.620000in}}%
\pgfusepath{clip}%
\pgfsetbuttcap%
\pgfsetroundjoin%
\pgfsetlinewidth{1.003750pt}%
\definecolor{currentstroke}{rgb}{1.000000,0.000000,0.000000}%
\pgfsetstrokecolor{currentstroke}%
\pgfsetdash{}{0pt}%
\pgfpathmoveto{\pgfqpoint{0.688192in}{1.225706in}}%
\pgfpathcurveto{\pgfqpoint{0.696428in}{1.225706in}}{\pgfqpoint{0.704328in}{1.228978in}}{\pgfqpoint{0.710152in}{1.234802in}}%
\pgfpathcurveto{\pgfqpoint{0.715976in}{1.240626in}}{\pgfqpoint{0.719248in}{1.248526in}}{\pgfqpoint{0.719248in}{1.256762in}}%
\pgfpathcurveto{\pgfqpoint{0.719248in}{1.264998in}}{\pgfqpoint{0.715976in}{1.272898in}}{\pgfqpoint{0.710152in}{1.278722in}}%
\pgfpathcurveto{\pgfqpoint{0.704328in}{1.284546in}}{\pgfqpoint{0.696428in}{1.287819in}}{\pgfqpoint{0.688192in}{1.287819in}}%
\pgfpathcurveto{\pgfqpoint{0.679955in}{1.287819in}}{\pgfqpoint{0.672055in}{1.284546in}}{\pgfqpoint{0.666231in}{1.278722in}}%
\pgfpathcurveto{\pgfqpoint{0.660407in}{1.272898in}}{\pgfqpoint{0.657135in}{1.264998in}}{\pgfqpoint{0.657135in}{1.256762in}}%
\pgfpathcurveto{\pgfqpoint{0.657135in}{1.248526in}}{\pgfqpoint{0.660407in}{1.240626in}}{\pgfqpoint{0.666231in}{1.234802in}}%
\pgfpathcurveto{\pgfqpoint{0.672055in}{1.228978in}}{\pgfqpoint{0.679955in}{1.225706in}}{\pgfqpoint{0.688192in}{1.225706in}}%
\pgfpathlineto{\pgfqpoint{0.688192in}{1.225706in}}%
\pgfpathclose%
\pgfusepath{stroke}%
\end{pgfscope}%
\begin{pgfscope}%
\pgfpathrectangle{\pgfqpoint{0.688192in}{0.670138in}}{\pgfqpoint{6.200000in}{4.620000in}}%
\pgfusepath{clip}%
\pgfsetbuttcap%
\pgfsetmiterjoin%
\definecolor{currentfill}{rgb}{0.839216,0.152941,0.156863}%
\pgfsetfillcolor{currentfill}%
\pgfsetfillopacity{0.200000}%
\pgfsetlinewidth{1.003750pt}%
\definecolor{currentstroke}{rgb}{0.839216,0.152941,0.156863}%
\pgfsetstrokecolor{currentstroke}%
\pgfsetstrokeopacity{0.200000}%
\pgfsetdash{}{0pt}%
\pgfpathmoveto{\pgfqpoint{0.688192in}{0.670138in}}%
\pgfpathlineto{\pgfqpoint{1.790122in}{0.670138in}}%
\pgfpathlineto{\pgfqpoint{1.790122in}{5.290138in}}%
\pgfpathlineto{\pgfqpoint{0.688192in}{5.290138in}}%
\pgfpathlineto{\pgfqpoint{0.688192in}{0.670138in}}%
\pgfpathclose%
\pgfusepath{stroke,fill}%
\end{pgfscope}%
\begin{pgfscope}%
\pgfsetbuttcap%
\pgfsetmiterjoin%
\definecolor{currentfill}{rgb}{0.839216,0.152941,0.156863}%
\pgfsetfillcolor{currentfill}%
\pgfsetfillopacity{0.200000}%
\pgfsetlinewidth{1.003750pt}%
\definecolor{currentstroke}{rgb}{0.839216,0.152941,0.156863}%
\pgfsetstrokecolor{currentstroke}%
\pgfsetstrokeopacity{0.200000}%
\pgfsetdash{}{0pt}%
\pgfpathrectangle{\pgfqpoint{0.688192in}{0.670138in}}{\pgfqpoint{6.200000in}{4.620000in}}%
\pgfusepath{clip}%
\pgfpathmoveto{\pgfqpoint{0.688192in}{0.670138in}}%
\pgfpathlineto{\pgfqpoint{1.790122in}{0.670138in}}%
\pgfpathlineto{\pgfqpoint{1.790122in}{5.290138in}}%
\pgfpathlineto{\pgfqpoint{0.688192in}{5.290138in}}%
\pgfpathlineto{\pgfqpoint{0.688192in}{0.670138in}}%
\pgfpathclose%
\pgfusepath{clip}%
\pgfsys@defobject{currentpattern}{\pgfqpoint{0in}{0in}}{\pgfqpoint{1in}{1in}}{%
\begin{pgfscope}%
\pgfpathrectangle{\pgfqpoint{0in}{0in}}{\pgfqpoint{1in}{1in}}%
\pgfusepath{clip}%
\pgfpathmoveto{\pgfqpoint{-0.500000in}{0.500000in}}%
\pgfpathlineto{\pgfqpoint{0.500000in}{1.500000in}}%
\pgfpathmoveto{\pgfqpoint{-0.333333in}{0.333333in}}%
\pgfpathlineto{\pgfqpoint{0.666667in}{1.333333in}}%
\pgfpathmoveto{\pgfqpoint{-0.166667in}{0.166667in}}%
\pgfpathlineto{\pgfqpoint{0.833333in}{1.166667in}}%
\pgfpathmoveto{\pgfqpoint{0.000000in}{0.000000in}}%
\pgfpathlineto{\pgfqpoint{1.000000in}{1.000000in}}%
\pgfpathmoveto{\pgfqpoint{0.166667in}{-0.166667in}}%
\pgfpathlineto{\pgfqpoint{1.166667in}{0.833333in}}%
\pgfpathmoveto{\pgfqpoint{0.333333in}{-0.333333in}}%
\pgfpathlineto{\pgfqpoint{1.333333in}{0.666667in}}%
\pgfpathmoveto{\pgfqpoint{0.500000in}{-0.500000in}}%
\pgfpathlineto{\pgfqpoint{1.500000in}{0.500000in}}%
\pgfusepath{stroke}%
\end{pgfscope}%
}%
\pgfsys@transformshift{0.688192in}{0.670138in}%
\pgfsys@useobject{currentpattern}{}%
\pgfsys@transformshift{1in}{0in}%
\pgfsys@useobject{currentpattern}{}%
\pgfsys@transformshift{1in}{0in}%
\pgfsys@transformshift{-2in}{0in}%
\pgfsys@transformshift{0in}{1in}%
\pgfsys@useobject{currentpattern}{}%
\pgfsys@transformshift{1in}{0in}%
\pgfsys@useobject{currentpattern}{}%
\pgfsys@transformshift{1in}{0in}%
\pgfsys@transformshift{-2in}{0in}%
\pgfsys@transformshift{0in}{1in}%
\pgfsys@useobject{currentpattern}{}%
\pgfsys@transformshift{1in}{0in}%
\pgfsys@useobject{currentpattern}{}%
\pgfsys@transformshift{1in}{0in}%
\pgfsys@transformshift{-2in}{0in}%
\pgfsys@transformshift{0in}{1in}%
\pgfsys@useobject{currentpattern}{}%
\pgfsys@transformshift{1in}{0in}%
\pgfsys@useobject{currentpattern}{}%
\pgfsys@transformshift{1in}{0in}%
\pgfsys@transformshift{-2in}{0in}%
\pgfsys@transformshift{0in}{1in}%
\pgfsys@useobject{currentpattern}{}%
\pgfsys@transformshift{1in}{0in}%
\pgfsys@useobject{currentpattern}{}%
\pgfsys@transformshift{1in}{0in}%
\pgfsys@transformshift{-2in}{0in}%
\pgfsys@transformshift{0in}{1in}%
\end{pgfscope}%
\begin{pgfscope}%
\pgfpathrectangle{\pgfqpoint{0.688192in}{0.670138in}}{\pgfqpoint{6.200000in}{4.620000in}}%
\pgfusepath{clip}%
\pgfsetrectcap%
\pgfsetroundjoin%
\pgfsetlinewidth{0.803000pt}%
\definecolor{currentstroke}{rgb}{0.690196,0.690196,0.690196}%
\pgfsetstrokecolor{currentstroke}%
\pgfsetdash{}{0pt}%
\pgfpathmoveto{\pgfqpoint{1.122474in}{0.670138in}}%
\pgfpathlineto{\pgfqpoint{1.122474in}{5.290138in}}%
\pgfusepath{stroke}%
\end{pgfscope}%
\begin{pgfscope}%
\pgfsetbuttcap%
\pgfsetroundjoin%
\definecolor{currentfill}{rgb}{0.000000,0.000000,0.000000}%
\pgfsetfillcolor{currentfill}%
\pgfsetlinewidth{0.803000pt}%
\definecolor{currentstroke}{rgb}{0.000000,0.000000,0.000000}%
\pgfsetstrokecolor{currentstroke}%
\pgfsetdash{}{0pt}%
\pgfsys@defobject{currentmarker}{\pgfqpoint{0.000000in}{-0.048611in}}{\pgfqpoint{0.000000in}{0.000000in}}{%
\pgfpathmoveto{\pgfqpoint{0.000000in}{0.000000in}}%
\pgfpathlineto{\pgfqpoint{0.000000in}{-0.048611in}}%
\pgfusepath{stroke,fill}%
}%
\begin{pgfscope}%
\pgfsys@transformshift{1.122474in}{0.670138in}%
\pgfsys@useobject{currentmarker}{}%
\end{pgfscope}%
\end{pgfscope}%
\begin{pgfscope}%
\definecolor{textcolor}{rgb}{0.000000,0.000000,0.000000}%
\pgfsetstrokecolor{textcolor}%
\pgfsetfillcolor{textcolor}%
\pgftext[x=1.122474in,y=0.572916in,,top]{\color{textcolor}{\rmfamily\fontsize{14.000000}{16.800000}\selectfont\catcode`\^=\active\def^{\ifmmode\sp\else\^{}\fi}\catcode`\%=\active\def%{\%}$\mathdefault{5500}$}}%
\end{pgfscope}%
\begin{pgfscope}%
\pgfpathrectangle{\pgfqpoint{0.688192in}{0.670138in}}{\pgfqpoint{6.200000in}{4.620000in}}%
\pgfusepath{clip}%
\pgfsetrectcap%
\pgfsetroundjoin%
\pgfsetlinewidth{0.803000pt}%
\definecolor{currentstroke}{rgb}{0.690196,0.690196,0.690196}%
\pgfsetstrokecolor{currentstroke}%
\pgfsetdash{}{0pt}%
\pgfpathmoveto{\pgfqpoint{2.163709in}{0.670138in}}%
\pgfpathlineto{\pgfqpoint{2.163709in}{5.290138in}}%
\pgfusepath{stroke}%
\end{pgfscope}%
\begin{pgfscope}%
\pgfsetbuttcap%
\pgfsetroundjoin%
\definecolor{currentfill}{rgb}{0.000000,0.000000,0.000000}%
\pgfsetfillcolor{currentfill}%
\pgfsetlinewidth{0.803000pt}%
\definecolor{currentstroke}{rgb}{0.000000,0.000000,0.000000}%
\pgfsetstrokecolor{currentstroke}%
\pgfsetdash{}{0pt}%
\pgfsys@defobject{currentmarker}{\pgfqpoint{0.000000in}{-0.048611in}}{\pgfqpoint{0.000000in}{0.000000in}}{%
\pgfpathmoveto{\pgfqpoint{0.000000in}{0.000000in}}%
\pgfpathlineto{\pgfqpoint{0.000000in}{-0.048611in}}%
\pgfusepath{stroke,fill}%
}%
\begin{pgfscope}%
\pgfsys@transformshift{2.163709in}{0.670138in}%
\pgfsys@useobject{currentmarker}{}%
\end{pgfscope}%
\end{pgfscope}%
\begin{pgfscope}%
\definecolor{textcolor}{rgb}{0.000000,0.000000,0.000000}%
\pgfsetstrokecolor{textcolor}%
\pgfsetfillcolor{textcolor}%
\pgftext[x=2.163709in,y=0.572916in,,top]{\color{textcolor}{\rmfamily\fontsize{14.000000}{16.800000}\selectfont\catcode`\^=\active\def^{\ifmmode\sp\else\^{}\fi}\catcode`\%=\active\def%{\%}$\mathdefault{6000}$}}%
\end{pgfscope}%
\begin{pgfscope}%
\pgfpathrectangle{\pgfqpoint{0.688192in}{0.670138in}}{\pgfqpoint{6.200000in}{4.620000in}}%
\pgfusepath{clip}%
\pgfsetrectcap%
\pgfsetroundjoin%
\pgfsetlinewidth{0.803000pt}%
\definecolor{currentstroke}{rgb}{0.690196,0.690196,0.690196}%
\pgfsetstrokecolor{currentstroke}%
\pgfsetdash{}{0pt}%
\pgfpathmoveto{\pgfqpoint{3.204944in}{0.670138in}}%
\pgfpathlineto{\pgfqpoint{3.204944in}{5.290138in}}%
\pgfusepath{stroke}%
\end{pgfscope}%
\begin{pgfscope}%
\pgfsetbuttcap%
\pgfsetroundjoin%
\definecolor{currentfill}{rgb}{0.000000,0.000000,0.000000}%
\pgfsetfillcolor{currentfill}%
\pgfsetlinewidth{0.803000pt}%
\definecolor{currentstroke}{rgb}{0.000000,0.000000,0.000000}%
\pgfsetstrokecolor{currentstroke}%
\pgfsetdash{}{0pt}%
\pgfsys@defobject{currentmarker}{\pgfqpoint{0.000000in}{-0.048611in}}{\pgfqpoint{0.000000in}{0.000000in}}{%
\pgfpathmoveto{\pgfqpoint{0.000000in}{0.000000in}}%
\pgfpathlineto{\pgfqpoint{0.000000in}{-0.048611in}}%
\pgfusepath{stroke,fill}%
}%
\begin{pgfscope}%
\pgfsys@transformshift{3.204944in}{0.670138in}%
\pgfsys@useobject{currentmarker}{}%
\end{pgfscope}%
\end{pgfscope}%
\begin{pgfscope}%
\definecolor{textcolor}{rgb}{0.000000,0.000000,0.000000}%
\pgfsetstrokecolor{textcolor}%
\pgfsetfillcolor{textcolor}%
\pgftext[x=3.204944in,y=0.572916in,,top]{\color{textcolor}{\rmfamily\fontsize{14.000000}{16.800000}\selectfont\catcode`\^=\active\def^{\ifmmode\sp\else\^{}\fi}\catcode`\%=\active\def%{\%}$\mathdefault{6500}$}}%
\end{pgfscope}%
\begin{pgfscope}%
\pgfpathrectangle{\pgfqpoint{0.688192in}{0.670138in}}{\pgfqpoint{6.200000in}{4.620000in}}%
\pgfusepath{clip}%
\pgfsetrectcap%
\pgfsetroundjoin%
\pgfsetlinewidth{0.803000pt}%
\definecolor{currentstroke}{rgb}{0.690196,0.690196,0.690196}%
\pgfsetstrokecolor{currentstroke}%
\pgfsetdash{}{0pt}%
\pgfpathmoveto{\pgfqpoint{4.246179in}{0.670138in}}%
\pgfpathlineto{\pgfqpoint{4.246179in}{5.290138in}}%
\pgfusepath{stroke}%
\end{pgfscope}%
\begin{pgfscope}%
\pgfsetbuttcap%
\pgfsetroundjoin%
\definecolor{currentfill}{rgb}{0.000000,0.000000,0.000000}%
\pgfsetfillcolor{currentfill}%
\pgfsetlinewidth{0.803000pt}%
\definecolor{currentstroke}{rgb}{0.000000,0.000000,0.000000}%
\pgfsetstrokecolor{currentstroke}%
\pgfsetdash{}{0pt}%
\pgfsys@defobject{currentmarker}{\pgfqpoint{0.000000in}{-0.048611in}}{\pgfqpoint{0.000000in}{0.000000in}}{%
\pgfpathmoveto{\pgfqpoint{0.000000in}{0.000000in}}%
\pgfpathlineto{\pgfqpoint{0.000000in}{-0.048611in}}%
\pgfusepath{stroke,fill}%
}%
\begin{pgfscope}%
\pgfsys@transformshift{4.246179in}{0.670138in}%
\pgfsys@useobject{currentmarker}{}%
\end{pgfscope}%
\end{pgfscope}%
\begin{pgfscope}%
\definecolor{textcolor}{rgb}{0.000000,0.000000,0.000000}%
\pgfsetstrokecolor{textcolor}%
\pgfsetfillcolor{textcolor}%
\pgftext[x=4.246179in,y=0.572916in,,top]{\color{textcolor}{\rmfamily\fontsize{14.000000}{16.800000}\selectfont\catcode`\^=\active\def^{\ifmmode\sp\else\^{}\fi}\catcode`\%=\active\def%{\%}$\mathdefault{7000}$}}%
\end{pgfscope}%
\begin{pgfscope}%
\pgfpathrectangle{\pgfqpoint{0.688192in}{0.670138in}}{\pgfqpoint{6.200000in}{4.620000in}}%
\pgfusepath{clip}%
\pgfsetrectcap%
\pgfsetroundjoin%
\pgfsetlinewidth{0.803000pt}%
\definecolor{currentstroke}{rgb}{0.690196,0.690196,0.690196}%
\pgfsetstrokecolor{currentstroke}%
\pgfsetdash{}{0pt}%
\pgfpathmoveto{\pgfqpoint{5.287414in}{0.670138in}}%
\pgfpathlineto{\pgfqpoint{5.287414in}{5.290138in}}%
\pgfusepath{stroke}%
\end{pgfscope}%
\begin{pgfscope}%
\pgfsetbuttcap%
\pgfsetroundjoin%
\definecolor{currentfill}{rgb}{0.000000,0.000000,0.000000}%
\pgfsetfillcolor{currentfill}%
\pgfsetlinewidth{0.803000pt}%
\definecolor{currentstroke}{rgb}{0.000000,0.000000,0.000000}%
\pgfsetstrokecolor{currentstroke}%
\pgfsetdash{}{0pt}%
\pgfsys@defobject{currentmarker}{\pgfqpoint{0.000000in}{-0.048611in}}{\pgfqpoint{0.000000in}{0.000000in}}{%
\pgfpathmoveto{\pgfqpoint{0.000000in}{0.000000in}}%
\pgfpathlineto{\pgfqpoint{0.000000in}{-0.048611in}}%
\pgfusepath{stroke,fill}%
}%
\begin{pgfscope}%
\pgfsys@transformshift{5.287414in}{0.670138in}%
\pgfsys@useobject{currentmarker}{}%
\end{pgfscope}%
\end{pgfscope}%
\begin{pgfscope}%
\definecolor{textcolor}{rgb}{0.000000,0.000000,0.000000}%
\pgfsetstrokecolor{textcolor}%
\pgfsetfillcolor{textcolor}%
\pgftext[x=5.287414in,y=0.572916in,,top]{\color{textcolor}{\rmfamily\fontsize{14.000000}{16.800000}\selectfont\catcode`\^=\active\def^{\ifmmode\sp\else\^{}\fi}\catcode`\%=\active\def%{\%}$\mathdefault{7500}$}}%
\end{pgfscope}%
\begin{pgfscope}%
\pgfpathrectangle{\pgfqpoint{0.688192in}{0.670138in}}{\pgfqpoint{6.200000in}{4.620000in}}%
\pgfusepath{clip}%
\pgfsetrectcap%
\pgfsetroundjoin%
\pgfsetlinewidth{0.803000pt}%
\definecolor{currentstroke}{rgb}{0.690196,0.690196,0.690196}%
\pgfsetstrokecolor{currentstroke}%
\pgfsetdash{}{0pt}%
\pgfpathmoveto{\pgfqpoint{6.328649in}{0.670138in}}%
\pgfpathlineto{\pgfqpoint{6.328649in}{5.290138in}}%
\pgfusepath{stroke}%
\end{pgfscope}%
\begin{pgfscope}%
\pgfsetbuttcap%
\pgfsetroundjoin%
\definecolor{currentfill}{rgb}{0.000000,0.000000,0.000000}%
\pgfsetfillcolor{currentfill}%
\pgfsetlinewidth{0.803000pt}%
\definecolor{currentstroke}{rgb}{0.000000,0.000000,0.000000}%
\pgfsetstrokecolor{currentstroke}%
\pgfsetdash{}{0pt}%
\pgfsys@defobject{currentmarker}{\pgfqpoint{0.000000in}{-0.048611in}}{\pgfqpoint{0.000000in}{0.000000in}}{%
\pgfpathmoveto{\pgfqpoint{0.000000in}{0.000000in}}%
\pgfpathlineto{\pgfqpoint{0.000000in}{-0.048611in}}%
\pgfusepath{stroke,fill}%
}%
\begin{pgfscope}%
\pgfsys@transformshift{6.328649in}{0.670138in}%
\pgfsys@useobject{currentmarker}{}%
\end{pgfscope}%
\end{pgfscope}%
\begin{pgfscope}%
\definecolor{textcolor}{rgb}{0.000000,0.000000,0.000000}%
\pgfsetstrokecolor{textcolor}%
\pgfsetfillcolor{textcolor}%
\pgftext[x=6.328649in,y=0.572916in,,top]{\color{textcolor}{\rmfamily\fontsize{14.000000}{16.800000}\selectfont\catcode`\^=\active\def^{\ifmmode\sp\else\^{}\fi}\catcode`\%=\active\def%{\%}$\mathdefault{8000}$}}%
\end{pgfscope}%
\begin{pgfscope}%
\definecolor{textcolor}{rgb}{0.000000,0.000000,0.000000}%
\pgfsetstrokecolor{textcolor}%
\pgfsetfillcolor{textcolor}%
\pgftext[x=3.788192in,y=0.339583in,,top]{\color{textcolor}{\rmfamily\fontsize{18.000000}{21.600000}\selectfont\catcode`\^=\active\def^{\ifmmode\sp\else\^{}\fi}\catcode`\%=\active\def%{\%}Total Cost (M\$)}}%
\end{pgfscope}%
\begin{pgfscope}%
\pgfpathrectangle{\pgfqpoint{0.688192in}{0.670138in}}{\pgfqpoint{6.200000in}{4.620000in}}%
\pgfusepath{clip}%
\pgfsetrectcap%
\pgfsetroundjoin%
\pgfsetlinewidth{0.803000pt}%
\definecolor{currentstroke}{rgb}{0.690196,0.690196,0.690196}%
\pgfsetstrokecolor{currentstroke}%
\pgfsetdash{}{0pt}%
\pgfpathmoveto{\pgfqpoint{0.688192in}{1.130833in}}%
\pgfpathlineto{\pgfqpoint{6.888192in}{1.130833in}}%
\pgfusepath{stroke}%
\end{pgfscope}%
\begin{pgfscope}%
\pgfsetbuttcap%
\pgfsetroundjoin%
\definecolor{currentfill}{rgb}{0.000000,0.000000,0.000000}%
\pgfsetfillcolor{currentfill}%
\pgfsetlinewidth{0.803000pt}%
\definecolor{currentstroke}{rgb}{0.000000,0.000000,0.000000}%
\pgfsetstrokecolor{currentstroke}%
\pgfsetdash{}{0pt}%
\pgfsys@defobject{currentmarker}{\pgfqpoint{-0.048611in}{0.000000in}}{\pgfqpoint{-0.000000in}{0.000000in}}{%
\pgfpathmoveto{\pgfqpoint{-0.000000in}{0.000000in}}%
\pgfpathlineto{\pgfqpoint{-0.048611in}{0.000000in}}%
\pgfusepath{stroke,fill}%
}%
\begin{pgfscope}%
\pgfsys@transformshift{0.688192in}{1.130833in}%
\pgfsys@useobject{currentmarker}{}%
\end{pgfscope}%
\end{pgfscope}%
\begin{pgfscope}%
\definecolor{textcolor}{rgb}{0.000000,0.000000,0.000000}%
\pgfsetstrokecolor{textcolor}%
\pgfsetfillcolor{textcolor}%
\pgftext[x=0.395138in, y=1.061389in, left, base]{\color{textcolor}{\rmfamily\fontsize{14.000000}{16.800000}\selectfont\catcode`\^=\active\def^{\ifmmode\sp\else\^{}\fi}\catcode`\%=\active\def%{\%}$\mathdefault{10}$}}%
\end{pgfscope}%
\begin{pgfscope}%
\pgfpathrectangle{\pgfqpoint{0.688192in}{0.670138in}}{\pgfqpoint{6.200000in}{4.620000in}}%
\pgfusepath{clip}%
\pgfsetrectcap%
\pgfsetroundjoin%
\pgfsetlinewidth{0.803000pt}%
\definecolor{currentstroke}{rgb}{0.690196,0.690196,0.690196}%
\pgfsetstrokecolor{currentstroke}%
\pgfsetdash{}{0pt}%
\pgfpathmoveto{\pgfqpoint{0.688192in}{1.724860in}}%
\pgfpathlineto{\pgfqpoint{6.888192in}{1.724860in}}%
\pgfusepath{stroke}%
\end{pgfscope}%
\begin{pgfscope}%
\pgfsetbuttcap%
\pgfsetroundjoin%
\definecolor{currentfill}{rgb}{0.000000,0.000000,0.000000}%
\pgfsetfillcolor{currentfill}%
\pgfsetlinewidth{0.803000pt}%
\definecolor{currentstroke}{rgb}{0.000000,0.000000,0.000000}%
\pgfsetstrokecolor{currentstroke}%
\pgfsetdash{}{0pt}%
\pgfsys@defobject{currentmarker}{\pgfqpoint{-0.048611in}{0.000000in}}{\pgfqpoint{-0.000000in}{0.000000in}}{%
\pgfpathmoveto{\pgfqpoint{-0.000000in}{0.000000in}}%
\pgfpathlineto{\pgfqpoint{-0.048611in}{0.000000in}}%
\pgfusepath{stroke,fill}%
}%
\begin{pgfscope}%
\pgfsys@transformshift{0.688192in}{1.724860in}%
\pgfsys@useobject{currentmarker}{}%
\end{pgfscope}%
\end{pgfscope}%
\begin{pgfscope}%
\definecolor{textcolor}{rgb}{0.000000,0.000000,0.000000}%
\pgfsetstrokecolor{textcolor}%
\pgfsetfillcolor{textcolor}%
\pgftext[x=0.395138in, y=1.655416in, left, base]{\color{textcolor}{\rmfamily\fontsize{14.000000}{16.800000}\selectfont\catcode`\^=\active\def^{\ifmmode\sp\else\^{}\fi}\catcode`\%=\active\def%{\%}$\mathdefault{20}$}}%
\end{pgfscope}%
\begin{pgfscope}%
\pgfpathrectangle{\pgfqpoint{0.688192in}{0.670138in}}{\pgfqpoint{6.200000in}{4.620000in}}%
\pgfusepath{clip}%
\pgfsetrectcap%
\pgfsetroundjoin%
\pgfsetlinewidth{0.803000pt}%
\definecolor{currentstroke}{rgb}{0.690196,0.690196,0.690196}%
\pgfsetstrokecolor{currentstroke}%
\pgfsetdash{}{0pt}%
\pgfpathmoveto{\pgfqpoint{0.688192in}{2.318888in}}%
\pgfpathlineto{\pgfqpoint{6.888192in}{2.318888in}}%
\pgfusepath{stroke}%
\end{pgfscope}%
\begin{pgfscope}%
\pgfsetbuttcap%
\pgfsetroundjoin%
\definecolor{currentfill}{rgb}{0.000000,0.000000,0.000000}%
\pgfsetfillcolor{currentfill}%
\pgfsetlinewidth{0.803000pt}%
\definecolor{currentstroke}{rgb}{0.000000,0.000000,0.000000}%
\pgfsetstrokecolor{currentstroke}%
\pgfsetdash{}{0pt}%
\pgfsys@defobject{currentmarker}{\pgfqpoint{-0.048611in}{0.000000in}}{\pgfqpoint{-0.000000in}{0.000000in}}{%
\pgfpathmoveto{\pgfqpoint{-0.000000in}{0.000000in}}%
\pgfpathlineto{\pgfqpoint{-0.048611in}{0.000000in}}%
\pgfusepath{stroke,fill}%
}%
\begin{pgfscope}%
\pgfsys@transformshift{0.688192in}{2.318888in}%
\pgfsys@useobject{currentmarker}{}%
\end{pgfscope}%
\end{pgfscope}%
\begin{pgfscope}%
\definecolor{textcolor}{rgb}{0.000000,0.000000,0.000000}%
\pgfsetstrokecolor{textcolor}%
\pgfsetfillcolor{textcolor}%
\pgftext[x=0.395138in, y=2.249444in, left, base]{\color{textcolor}{\rmfamily\fontsize{14.000000}{16.800000}\selectfont\catcode`\^=\active\def^{\ifmmode\sp\else\^{}\fi}\catcode`\%=\active\def%{\%}$\mathdefault{30}$}}%
\end{pgfscope}%
\begin{pgfscope}%
\pgfpathrectangle{\pgfqpoint{0.688192in}{0.670138in}}{\pgfqpoint{6.200000in}{4.620000in}}%
\pgfusepath{clip}%
\pgfsetrectcap%
\pgfsetroundjoin%
\pgfsetlinewidth{0.803000pt}%
\definecolor{currentstroke}{rgb}{0.690196,0.690196,0.690196}%
\pgfsetstrokecolor{currentstroke}%
\pgfsetdash{}{0pt}%
\pgfpathmoveto{\pgfqpoint{0.688192in}{2.912915in}}%
\pgfpathlineto{\pgfqpoint{6.888192in}{2.912915in}}%
\pgfusepath{stroke}%
\end{pgfscope}%
\begin{pgfscope}%
\pgfsetbuttcap%
\pgfsetroundjoin%
\definecolor{currentfill}{rgb}{0.000000,0.000000,0.000000}%
\pgfsetfillcolor{currentfill}%
\pgfsetlinewidth{0.803000pt}%
\definecolor{currentstroke}{rgb}{0.000000,0.000000,0.000000}%
\pgfsetstrokecolor{currentstroke}%
\pgfsetdash{}{0pt}%
\pgfsys@defobject{currentmarker}{\pgfqpoint{-0.048611in}{0.000000in}}{\pgfqpoint{-0.000000in}{0.000000in}}{%
\pgfpathmoveto{\pgfqpoint{-0.000000in}{0.000000in}}%
\pgfpathlineto{\pgfqpoint{-0.048611in}{0.000000in}}%
\pgfusepath{stroke,fill}%
}%
\begin{pgfscope}%
\pgfsys@transformshift{0.688192in}{2.912915in}%
\pgfsys@useobject{currentmarker}{}%
\end{pgfscope}%
\end{pgfscope}%
\begin{pgfscope}%
\definecolor{textcolor}{rgb}{0.000000,0.000000,0.000000}%
\pgfsetstrokecolor{textcolor}%
\pgfsetfillcolor{textcolor}%
\pgftext[x=0.395138in, y=2.843471in, left, base]{\color{textcolor}{\rmfamily\fontsize{14.000000}{16.800000}\selectfont\catcode`\^=\active\def^{\ifmmode\sp\else\^{}\fi}\catcode`\%=\active\def%{\%}$\mathdefault{40}$}}%
\end{pgfscope}%
\begin{pgfscope}%
\pgfpathrectangle{\pgfqpoint{0.688192in}{0.670138in}}{\pgfqpoint{6.200000in}{4.620000in}}%
\pgfusepath{clip}%
\pgfsetrectcap%
\pgfsetroundjoin%
\pgfsetlinewidth{0.803000pt}%
\definecolor{currentstroke}{rgb}{0.690196,0.690196,0.690196}%
\pgfsetstrokecolor{currentstroke}%
\pgfsetdash{}{0pt}%
\pgfpathmoveto{\pgfqpoint{0.688192in}{3.506943in}}%
\pgfpathlineto{\pgfqpoint{6.888192in}{3.506943in}}%
\pgfusepath{stroke}%
\end{pgfscope}%
\begin{pgfscope}%
\pgfsetbuttcap%
\pgfsetroundjoin%
\definecolor{currentfill}{rgb}{0.000000,0.000000,0.000000}%
\pgfsetfillcolor{currentfill}%
\pgfsetlinewidth{0.803000pt}%
\definecolor{currentstroke}{rgb}{0.000000,0.000000,0.000000}%
\pgfsetstrokecolor{currentstroke}%
\pgfsetdash{}{0pt}%
\pgfsys@defobject{currentmarker}{\pgfqpoint{-0.048611in}{0.000000in}}{\pgfqpoint{-0.000000in}{0.000000in}}{%
\pgfpathmoveto{\pgfqpoint{-0.000000in}{0.000000in}}%
\pgfpathlineto{\pgfqpoint{-0.048611in}{0.000000in}}%
\pgfusepath{stroke,fill}%
}%
\begin{pgfscope}%
\pgfsys@transformshift{0.688192in}{3.506943in}%
\pgfsys@useobject{currentmarker}{}%
\end{pgfscope}%
\end{pgfscope}%
\begin{pgfscope}%
\definecolor{textcolor}{rgb}{0.000000,0.000000,0.000000}%
\pgfsetstrokecolor{textcolor}%
\pgfsetfillcolor{textcolor}%
\pgftext[x=0.395138in, y=3.437499in, left, base]{\color{textcolor}{\rmfamily\fontsize{14.000000}{16.800000}\selectfont\catcode`\^=\active\def^{\ifmmode\sp\else\^{}\fi}\catcode`\%=\active\def%{\%}$\mathdefault{50}$}}%
\end{pgfscope}%
\begin{pgfscope}%
\pgfpathrectangle{\pgfqpoint{0.688192in}{0.670138in}}{\pgfqpoint{6.200000in}{4.620000in}}%
\pgfusepath{clip}%
\pgfsetrectcap%
\pgfsetroundjoin%
\pgfsetlinewidth{0.803000pt}%
\definecolor{currentstroke}{rgb}{0.690196,0.690196,0.690196}%
\pgfsetstrokecolor{currentstroke}%
\pgfsetdash{}{0pt}%
\pgfpathmoveto{\pgfqpoint{0.688192in}{4.100970in}}%
\pgfpathlineto{\pgfqpoint{6.888192in}{4.100970in}}%
\pgfusepath{stroke}%
\end{pgfscope}%
\begin{pgfscope}%
\pgfsetbuttcap%
\pgfsetroundjoin%
\definecolor{currentfill}{rgb}{0.000000,0.000000,0.000000}%
\pgfsetfillcolor{currentfill}%
\pgfsetlinewidth{0.803000pt}%
\definecolor{currentstroke}{rgb}{0.000000,0.000000,0.000000}%
\pgfsetstrokecolor{currentstroke}%
\pgfsetdash{}{0pt}%
\pgfsys@defobject{currentmarker}{\pgfqpoint{-0.048611in}{0.000000in}}{\pgfqpoint{-0.000000in}{0.000000in}}{%
\pgfpathmoveto{\pgfqpoint{-0.000000in}{0.000000in}}%
\pgfpathlineto{\pgfqpoint{-0.048611in}{0.000000in}}%
\pgfusepath{stroke,fill}%
}%
\begin{pgfscope}%
\pgfsys@transformshift{0.688192in}{4.100970in}%
\pgfsys@useobject{currentmarker}{}%
\end{pgfscope}%
\end{pgfscope}%
\begin{pgfscope}%
\definecolor{textcolor}{rgb}{0.000000,0.000000,0.000000}%
\pgfsetstrokecolor{textcolor}%
\pgfsetfillcolor{textcolor}%
\pgftext[x=0.395138in, y=4.031526in, left, base]{\color{textcolor}{\rmfamily\fontsize{14.000000}{16.800000}\selectfont\catcode`\^=\active\def^{\ifmmode\sp\else\^{}\fi}\catcode`\%=\active\def%{\%}$\mathdefault{60}$}}%
\end{pgfscope}%
\begin{pgfscope}%
\pgfpathrectangle{\pgfqpoint{0.688192in}{0.670138in}}{\pgfqpoint{6.200000in}{4.620000in}}%
\pgfusepath{clip}%
\pgfsetrectcap%
\pgfsetroundjoin%
\pgfsetlinewidth{0.803000pt}%
\definecolor{currentstroke}{rgb}{0.690196,0.690196,0.690196}%
\pgfsetstrokecolor{currentstroke}%
\pgfsetdash{}{0pt}%
\pgfpathmoveto{\pgfqpoint{0.688192in}{4.694998in}}%
\pgfpathlineto{\pgfqpoint{6.888192in}{4.694998in}}%
\pgfusepath{stroke}%
\end{pgfscope}%
\begin{pgfscope}%
\pgfsetbuttcap%
\pgfsetroundjoin%
\definecolor{currentfill}{rgb}{0.000000,0.000000,0.000000}%
\pgfsetfillcolor{currentfill}%
\pgfsetlinewidth{0.803000pt}%
\definecolor{currentstroke}{rgb}{0.000000,0.000000,0.000000}%
\pgfsetstrokecolor{currentstroke}%
\pgfsetdash{}{0pt}%
\pgfsys@defobject{currentmarker}{\pgfqpoint{-0.048611in}{0.000000in}}{\pgfqpoint{-0.000000in}{0.000000in}}{%
\pgfpathmoveto{\pgfqpoint{-0.000000in}{0.000000in}}%
\pgfpathlineto{\pgfqpoint{-0.048611in}{0.000000in}}%
\pgfusepath{stroke,fill}%
}%
\begin{pgfscope}%
\pgfsys@transformshift{0.688192in}{4.694998in}%
\pgfsys@useobject{currentmarker}{}%
\end{pgfscope}%
\end{pgfscope}%
\begin{pgfscope}%
\definecolor{textcolor}{rgb}{0.000000,0.000000,0.000000}%
\pgfsetstrokecolor{textcolor}%
\pgfsetfillcolor{textcolor}%
\pgftext[x=0.395138in, y=4.625553in, left, base]{\color{textcolor}{\rmfamily\fontsize{14.000000}{16.800000}\selectfont\catcode`\^=\active\def^{\ifmmode\sp\else\^{}\fi}\catcode`\%=\active\def%{\%}$\mathdefault{70}$}}%
\end{pgfscope}%
\begin{pgfscope}%
\pgfpathrectangle{\pgfqpoint{0.688192in}{0.670138in}}{\pgfqpoint{6.200000in}{4.620000in}}%
\pgfusepath{clip}%
\pgfsetrectcap%
\pgfsetroundjoin%
\pgfsetlinewidth{0.803000pt}%
\definecolor{currentstroke}{rgb}{0.690196,0.690196,0.690196}%
\pgfsetstrokecolor{currentstroke}%
\pgfsetdash{}{0pt}%
\pgfpathmoveto{\pgfqpoint{0.688192in}{5.289025in}}%
\pgfpathlineto{\pgfqpoint{6.888192in}{5.289025in}}%
\pgfusepath{stroke}%
\end{pgfscope}%
\begin{pgfscope}%
\pgfsetbuttcap%
\pgfsetroundjoin%
\definecolor{currentfill}{rgb}{0.000000,0.000000,0.000000}%
\pgfsetfillcolor{currentfill}%
\pgfsetlinewidth{0.803000pt}%
\definecolor{currentstroke}{rgb}{0.000000,0.000000,0.000000}%
\pgfsetstrokecolor{currentstroke}%
\pgfsetdash{}{0pt}%
\pgfsys@defobject{currentmarker}{\pgfqpoint{-0.048611in}{0.000000in}}{\pgfqpoint{-0.000000in}{0.000000in}}{%
\pgfpathmoveto{\pgfqpoint{-0.000000in}{0.000000in}}%
\pgfpathlineto{\pgfqpoint{-0.048611in}{0.000000in}}%
\pgfusepath{stroke,fill}%
}%
\begin{pgfscope}%
\pgfsys@transformshift{0.688192in}{5.289025in}%
\pgfsys@useobject{currentmarker}{}%
\end{pgfscope}%
\end{pgfscope}%
\begin{pgfscope}%
\definecolor{textcolor}{rgb}{0.000000,0.000000,0.000000}%
\pgfsetstrokecolor{textcolor}%
\pgfsetfillcolor{textcolor}%
\pgftext[x=0.395138in, y=5.219581in, left, base]{\color{textcolor}{\rmfamily\fontsize{14.000000}{16.800000}\selectfont\catcode`\^=\active\def^{\ifmmode\sp\else\^{}\fi}\catcode`\%=\active\def%{\%}$\mathdefault{80}$}}%
\end{pgfscope}%
\begin{pgfscope}%
\definecolor{textcolor}{rgb}{0.000000,0.000000,0.000000}%
\pgfsetstrokecolor{textcolor}%
\pgfsetfillcolor{textcolor}%
\pgftext[x=0.339583in,y=2.980138in,,bottom,rotate=90.000000]{\color{textcolor}{\rmfamily\fontsize{18.000000}{21.600000}\selectfont\catcode`\^=\active\def^{\ifmmode\sp\else\^{}\fi}\catcode`\%=\active\def%{\%}CO2 emissions (MT CO2)}}%
\end{pgfscope}%
\begin{pgfscope}%
\pgfpathrectangle{\pgfqpoint{0.688192in}{0.670138in}}{\pgfqpoint{6.200000in}{4.620000in}}%
\pgfusepath{clip}%
\pgfsetrectcap%
\pgfsetroundjoin%
\pgfsetlinewidth{1.505625pt}%
\definecolor{currentstroke}{rgb}{0.000000,0.000000,1.000000}%
\pgfsetstrokecolor{currentstroke}%
\pgfsetdash{}{0pt}%
\pgfpathmoveto{\pgfqpoint{0.741425in}{1.377543in}}%
\pgfpathlineto{\pgfqpoint{0.758703in}{0.955032in}}%
\pgfpathlineto{\pgfqpoint{0.768198in}{0.875033in}}%
\pgfpathlineto{\pgfqpoint{0.774746in}{0.828781in}}%
\pgfpathlineto{\pgfqpoint{0.778243in}{0.822495in}}%
\pgfpathlineto{\pgfqpoint{0.782159in}{0.789611in}}%
\pgfpathlineto{\pgfqpoint{0.786516in}{0.779881in}}%
\pgfpathlineto{\pgfqpoint{0.792538in}{0.779145in}}%
\pgfpathlineto{\pgfqpoint{0.794668in}{0.758056in}}%
\pgfpathlineto{\pgfqpoint{0.799837in}{0.752930in}}%
\pgfpathlineto{\pgfqpoint{0.809370in}{0.751978in}}%
\pgfpathlineto{\pgfqpoint{0.812629in}{0.743975in}}%
\pgfpathlineto{\pgfqpoint{0.815972in}{0.742575in}}%
\pgfpathlineto{\pgfqpoint{0.822987in}{0.738477in}}%
\pgfpathlineto{\pgfqpoint{0.828825in}{0.734937in}}%
\pgfpathlineto{\pgfqpoint{0.829214in}{0.733319in}}%
\pgfpathlineto{\pgfqpoint{0.833044in}{0.730858in}}%
\pgfpathlineto{\pgfqpoint{0.848459in}{0.726329in}}%
\pgfpathlineto{\pgfqpoint{0.864854in}{0.720019in}}%
\pgfpathlineto{\pgfqpoint{0.887104in}{0.715517in}}%
\pgfpathlineto{\pgfqpoint{0.907479in}{0.714004in}}%
\pgfpathlineto{\pgfqpoint{0.908310in}{0.712008in}}%
\pgfpathlineto{\pgfqpoint{0.909513in}{0.708525in}}%
\pgfpathlineto{\pgfqpoint{0.912740in}{0.707284in}}%
\pgfpathlineto{\pgfqpoint{0.920440in}{0.706723in}}%
\pgfpathlineto{\pgfqpoint{0.925670in}{0.705238in}}%
\pgfpathlineto{\pgfqpoint{0.948903in}{0.702931in}}%
\pgfpathlineto{\pgfqpoint{0.951945in}{0.701707in}}%
\pgfpathlineto{\pgfqpoint{0.952035in}{0.700391in}}%
\pgfpathlineto{\pgfqpoint{0.957029in}{0.700173in}}%
\pgfpathlineto{\pgfqpoint{0.968828in}{0.697963in}}%
\pgfpathlineto{\pgfqpoint{0.974412in}{0.697738in}}%
\pgfpathlineto{\pgfqpoint{0.975275in}{0.696914in}}%
\pgfpathlineto{\pgfqpoint{1.021767in}{0.694795in}}%
\pgfpathlineto{\pgfqpoint{1.025407in}{0.690657in}}%
\pgfpathlineto{\pgfqpoint{1.027475in}{0.690338in}}%
\pgfpathlineto{\pgfqpoint{1.034837in}{0.689784in}}%
\pgfpathlineto{\pgfqpoint{1.049406in}{0.687676in}}%
\pgfpathlineto{\pgfqpoint{1.054714in}{0.687138in}}%
\pgfpathlineto{\pgfqpoint{1.059617in}{0.686467in}}%
\pgfpathlineto{\pgfqpoint{1.072141in}{0.685078in}}%
\pgfpathlineto{\pgfqpoint{1.092208in}{0.684413in}}%
\pgfpathlineto{\pgfqpoint{1.115209in}{0.684111in}}%
\pgfpathlineto{\pgfqpoint{1.131834in}{0.684071in}}%
\pgfpathlineto{\pgfqpoint{1.152628in}{0.684059in}}%
\pgfpathlineto{\pgfqpoint{1.251312in}{0.683263in}}%
\pgfpathlineto{\pgfqpoint{1.277476in}{0.683159in}}%
\pgfpathlineto{\pgfqpoint{1.314870in}{0.682855in}}%
\pgfpathlineto{\pgfqpoint{1.369253in}{0.682756in}}%
\pgfpathlineto{\pgfqpoint{1.398687in}{0.682288in}}%
\pgfpathlineto{\pgfqpoint{1.467852in}{0.682134in}}%
\pgfpathlineto{\pgfqpoint{1.557026in}{0.681680in}}%
\pgfpathlineto{\pgfqpoint{1.627242in}{0.680913in}}%
\pgfpathlineto{\pgfqpoint{1.737728in}{0.680478in}}%
\pgfpathlineto{\pgfqpoint{1.887036in}{0.679610in}}%
\pgfpathlineto{\pgfqpoint{2.037481in}{0.678826in}}%
\pgfpathlineto{\pgfqpoint{2.258348in}{0.677741in}}%
\pgfpathlineto{\pgfqpoint{2.626338in}{0.676361in}}%
\pgfpathlineto{\pgfqpoint{3.263784in}{0.674352in}}%
\pgfpathlineto{\pgfqpoint{5.322800in}{0.670138in}}%
\pgfusepath{stroke}%
\end{pgfscope}%
\begin{pgfscope}%
\pgfpathrectangle{\pgfqpoint{0.688192in}{0.670138in}}{\pgfqpoint{6.200000in}{4.620000in}}%
\pgfusepath{clip}%
\pgfsetbuttcap%
\pgfsetroundjoin%
\definecolor{currentfill}{rgb}{0.000000,0.000000,1.000000}%
\pgfsetfillcolor{currentfill}%
\pgfsetlinewidth{1.003750pt}%
\definecolor{currentstroke}{rgb}{0.000000,0.000000,1.000000}%
\pgfsetstrokecolor{currentstroke}%
\pgfsetdash{}{0pt}%
\pgfsys@defobject{currentmarker}{\pgfqpoint{-0.006944in}{-0.006944in}}{\pgfqpoint{0.006944in}{0.006944in}}{%
\pgfpathmoveto{\pgfqpoint{0.000000in}{-0.006944in}}%
\pgfpathcurveto{\pgfqpoint{0.001842in}{-0.006944in}}{\pgfqpoint{0.003608in}{-0.006213in}}{\pgfqpoint{0.004910in}{-0.004910in}}%
\pgfpathcurveto{\pgfqpoint{0.006213in}{-0.003608in}}{\pgfqpoint{0.006944in}{-0.001842in}}{\pgfqpoint{0.006944in}{0.000000in}}%
\pgfpathcurveto{\pgfqpoint{0.006944in}{0.001842in}}{\pgfqpoint{0.006213in}{0.003608in}}{\pgfqpoint{0.004910in}{0.004910in}}%
\pgfpathcurveto{\pgfqpoint{0.003608in}{0.006213in}}{\pgfqpoint{0.001842in}{0.006944in}}{\pgfqpoint{0.000000in}{0.006944in}}%
\pgfpathcurveto{\pgfqpoint{-0.001842in}{0.006944in}}{\pgfqpoint{-0.003608in}{0.006213in}}{\pgfqpoint{-0.004910in}{0.004910in}}%
\pgfpathcurveto{\pgfqpoint{-0.006213in}{0.003608in}}{\pgfqpoint{-0.006944in}{0.001842in}}{\pgfqpoint{-0.006944in}{0.000000in}}%
\pgfpathcurveto{\pgfqpoint{-0.006944in}{-0.001842in}}{\pgfqpoint{-0.006213in}{-0.003608in}}{\pgfqpoint{-0.004910in}{-0.004910in}}%
\pgfpathcurveto{\pgfqpoint{-0.003608in}{-0.006213in}}{\pgfqpoint{-0.001842in}{-0.006944in}}{\pgfqpoint{0.000000in}{-0.006944in}}%
\pgfpathlineto{\pgfqpoint{0.000000in}{-0.006944in}}%
\pgfpathclose%
\pgfusepath{stroke,fill}%
}%
\begin{pgfscope}%
\pgfsys@transformshift{0.741425in}{1.377543in}%
\pgfsys@useobject{currentmarker}{}%
\end{pgfscope}%
\begin{pgfscope}%
\pgfsys@transformshift{0.758703in}{0.955032in}%
\pgfsys@useobject{currentmarker}{}%
\end{pgfscope}%
\begin{pgfscope}%
\pgfsys@transformshift{0.768198in}{0.875033in}%
\pgfsys@useobject{currentmarker}{}%
\end{pgfscope}%
\begin{pgfscope}%
\pgfsys@transformshift{0.774746in}{0.828781in}%
\pgfsys@useobject{currentmarker}{}%
\end{pgfscope}%
\begin{pgfscope}%
\pgfsys@transformshift{0.778243in}{0.822495in}%
\pgfsys@useobject{currentmarker}{}%
\end{pgfscope}%
\begin{pgfscope}%
\pgfsys@transformshift{0.782159in}{0.789611in}%
\pgfsys@useobject{currentmarker}{}%
\end{pgfscope}%
\begin{pgfscope}%
\pgfsys@transformshift{0.786516in}{0.779881in}%
\pgfsys@useobject{currentmarker}{}%
\end{pgfscope}%
\begin{pgfscope}%
\pgfsys@transformshift{0.792538in}{0.779145in}%
\pgfsys@useobject{currentmarker}{}%
\end{pgfscope}%
\begin{pgfscope}%
\pgfsys@transformshift{0.794668in}{0.758056in}%
\pgfsys@useobject{currentmarker}{}%
\end{pgfscope}%
\begin{pgfscope}%
\pgfsys@transformshift{0.799837in}{0.752930in}%
\pgfsys@useobject{currentmarker}{}%
\end{pgfscope}%
\begin{pgfscope}%
\pgfsys@transformshift{0.809370in}{0.751978in}%
\pgfsys@useobject{currentmarker}{}%
\end{pgfscope}%
\begin{pgfscope}%
\pgfsys@transformshift{0.812629in}{0.743975in}%
\pgfsys@useobject{currentmarker}{}%
\end{pgfscope}%
\begin{pgfscope}%
\pgfsys@transformshift{0.815972in}{0.742575in}%
\pgfsys@useobject{currentmarker}{}%
\end{pgfscope}%
\begin{pgfscope}%
\pgfsys@transformshift{0.822987in}{0.738477in}%
\pgfsys@useobject{currentmarker}{}%
\end{pgfscope}%
\begin{pgfscope}%
\pgfsys@transformshift{0.828825in}{0.734937in}%
\pgfsys@useobject{currentmarker}{}%
\end{pgfscope}%
\begin{pgfscope}%
\pgfsys@transformshift{0.829214in}{0.733319in}%
\pgfsys@useobject{currentmarker}{}%
\end{pgfscope}%
\begin{pgfscope}%
\pgfsys@transformshift{0.833044in}{0.730858in}%
\pgfsys@useobject{currentmarker}{}%
\end{pgfscope}%
\begin{pgfscope}%
\pgfsys@transformshift{0.848459in}{0.726329in}%
\pgfsys@useobject{currentmarker}{}%
\end{pgfscope}%
\begin{pgfscope}%
\pgfsys@transformshift{0.864854in}{0.720019in}%
\pgfsys@useobject{currentmarker}{}%
\end{pgfscope}%
\begin{pgfscope}%
\pgfsys@transformshift{0.887104in}{0.715517in}%
\pgfsys@useobject{currentmarker}{}%
\end{pgfscope}%
\begin{pgfscope}%
\pgfsys@transformshift{0.907479in}{0.714004in}%
\pgfsys@useobject{currentmarker}{}%
\end{pgfscope}%
\begin{pgfscope}%
\pgfsys@transformshift{0.908310in}{0.712008in}%
\pgfsys@useobject{currentmarker}{}%
\end{pgfscope}%
\begin{pgfscope}%
\pgfsys@transformshift{0.909513in}{0.708525in}%
\pgfsys@useobject{currentmarker}{}%
\end{pgfscope}%
\begin{pgfscope}%
\pgfsys@transformshift{0.912740in}{0.707284in}%
\pgfsys@useobject{currentmarker}{}%
\end{pgfscope}%
\begin{pgfscope}%
\pgfsys@transformshift{0.920440in}{0.706723in}%
\pgfsys@useobject{currentmarker}{}%
\end{pgfscope}%
\begin{pgfscope}%
\pgfsys@transformshift{0.925670in}{0.705238in}%
\pgfsys@useobject{currentmarker}{}%
\end{pgfscope}%
\begin{pgfscope}%
\pgfsys@transformshift{0.948903in}{0.702931in}%
\pgfsys@useobject{currentmarker}{}%
\end{pgfscope}%
\begin{pgfscope}%
\pgfsys@transformshift{0.951945in}{0.701707in}%
\pgfsys@useobject{currentmarker}{}%
\end{pgfscope}%
\begin{pgfscope}%
\pgfsys@transformshift{0.952035in}{0.700391in}%
\pgfsys@useobject{currentmarker}{}%
\end{pgfscope}%
\begin{pgfscope}%
\pgfsys@transformshift{0.957029in}{0.700173in}%
\pgfsys@useobject{currentmarker}{}%
\end{pgfscope}%
\begin{pgfscope}%
\pgfsys@transformshift{0.968828in}{0.697963in}%
\pgfsys@useobject{currentmarker}{}%
\end{pgfscope}%
\begin{pgfscope}%
\pgfsys@transformshift{0.974412in}{0.697738in}%
\pgfsys@useobject{currentmarker}{}%
\end{pgfscope}%
\begin{pgfscope}%
\pgfsys@transformshift{0.975275in}{0.696914in}%
\pgfsys@useobject{currentmarker}{}%
\end{pgfscope}%
\begin{pgfscope}%
\pgfsys@transformshift{1.021767in}{0.694795in}%
\pgfsys@useobject{currentmarker}{}%
\end{pgfscope}%
\begin{pgfscope}%
\pgfsys@transformshift{1.025407in}{0.690657in}%
\pgfsys@useobject{currentmarker}{}%
\end{pgfscope}%
\begin{pgfscope}%
\pgfsys@transformshift{1.027475in}{0.690338in}%
\pgfsys@useobject{currentmarker}{}%
\end{pgfscope}%
\begin{pgfscope}%
\pgfsys@transformshift{1.034837in}{0.689784in}%
\pgfsys@useobject{currentmarker}{}%
\end{pgfscope}%
\begin{pgfscope}%
\pgfsys@transformshift{1.049406in}{0.687676in}%
\pgfsys@useobject{currentmarker}{}%
\end{pgfscope}%
\begin{pgfscope}%
\pgfsys@transformshift{1.054714in}{0.687138in}%
\pgfsys@useobject{currentmarker}{}%
\end{pgfscope}%
\begin{pgfscope}%
\pgfsys@transformshift{1.059617in}{0.686467in}%
\pgfsys@useobject{currentmarker}{}%
\end{pgfscope}%
\begin{pgfscope}%
\pgfsys@transformshift{1.072141in}{0.685078in}%
\pgfsys@useobject{currentmarker}{}%
\end{pgfscope}%
\begin{pgfscope}%
\pgfsys@transformshift{1.092208in}{0.684413in}%
\pgfsys@useobject{currentmarker}{}%
\end{pgfscope}%
\begin{pgfscope}%
\pgfsys@transformshift{1.115209in}{0.684111in}%
\pgfsys@useobject{currentmarker}{}%
\end{pgfscope}%
\begin{pgfscope}%
\pgfsys@transformshift{1.131834in}{0.684071in}%
\pgfsys@useobject{currentmarker}{}%
\end{pgfscope}%
\begin{pgfscope}%
\pgfsys@transformshift{1.152628in}{0.684059in}%
\pgfsys@useobject{currentmarker}{}%
\end{pgfscope}%
\begin{pgfscope}%
\pgfsys@transformshift{1.251312in}{0.683263in}%
\pgfsys@useobject{currentmarker}{}%
\end{pgfscope}%
\begin{pgfscope}%
\pgfsys@transformshift{1.277476in}{0.683159in}%
\pgfsys@useobject{currentmarker}{}%
\end{pgfscope}%
\begin{pgfscope}%
\pgfsys@transformshift{1.314870in}{0.682855in}%
\pgfsys@useobject{currentmarker}{}%
\end{pgfscope}%
\begin{pgfscope}%
\pgfsys@transformshift{1.369253in}{0.682756in}%
\pgfsys@useobject{currentmarker}{}%
\end{pgfscope}%
\begin{pgfscope}%
\pgfsys@transformshift{1.398687in}{0.682288in}%
\pgfsys@useobject{currentmarker}{}%
\end{pgfscope}%
\begin{pgfscope}%
\pgfsys@transformshift{1.467852in}{0.682134in}%
\pgfsys@useobject{currentmarker}{}%
\end{pgfscope}%
\begin{pgfscope}%
\pgfsys@transformshift{1.557026in}{0.681680in}%
\pgfsys@useobject{currentmarker}{}%
\end{pgfscope}%
\begin{pgfscope}%
\pgfsys@transformshift{1.627242in}{0.680913in}%
\pgfsys@useobject{currentmarker}{}%
\end{pgfscope}%
\begin{pgfscope}%
\pgfsys@transformshift{1.737728in}{0.680478in}%
\pgfsys@useobject{currentmarker}{}%
\end{pgfscope}%
\begin{pgfscope}%
\pgfsys@transformshift{1.887036in}{0.679610in}%
\pgfsys@useobject{currentmarker}{}%
\end{pgfscope}%
\begin{pgfscope}%
\pgfsys@transformshift{2.037481in}{0.678826in}%
\pgfsys@useobject{currentmarker}{}%
\end{pgfscope}%
\begin{pgfscope}%
\pgfsys@transformshift{2.258348in}{0.677741in}%
\pgfsys@useobject{currentmarker}{}%
\end{pgfscope}%
\begin{pgfscope}%
\pgfsys@transformshift{2.626338in}{0.676361in}%
\pgfsys@useobject{currentmarker}{}%
\end{pgfscope}%
\begin{pgfscope}%
\pgfsys@transformshift{3.263784in}{0.674352in}%
\pgfsys@useobject{currentmarker}{}%
\end{pgfscope}%
\begin{pgfscope}%
\pgfsys@transformshift{5.322800in}{0.670138in}%
\pgfsys@useobject{currentmarker}{}%
\end{pgfscope}%
\end{pgfscope}%
\begin{pgfscope}%
\pgfpathrectangle{\pgfqpoint{0.688192in}{0.670138in}}{\pgfqpoint{6.200000in}{4.620000in}}%
\pgfusepath{clip}%
\pgfsetrectcap%
\pgfsetroundjoin%
\pgfsetlinewidth{1.505625pt}%
\definecolor{currentstroke}{rgb}{0.121569,0.466667,0.705882}%
\pgfsetstrokecolor{currentstroke}%
\pgfsetstrokeopacity{0.500000}%
\pgfsetdash{}{0pt}%
\pgfpathmoveto{\pgfqpoint{1.848679in}{1.461617in}}%
\pgfpathlineto{\pgfqpoint{1.867684in}{0.996854in}}%
\pgfpathlineto{\pgfqpoint{1.878129in}{0.908856in}}%
\pgfpathlineto{\pgfqpoint{1.885331in}{0.857978in}}%
\pgfpathlineto{\pgfqpoint{1.889178in}{0.851064in}}%
\pgfpathlineto{\pgfqpoint{1.893486in}{0.814892in}}%
\pgfpathlineto{\pgfqpoint{1.898279in}{0.804188in}}%
\pgfpathlineto{\pgfqpoint{1.904902in}{0.803379in}}%
\pgfpathlineto{\pgfqpoint{1.907246in}{0.780181in}}%
\pgfpathlineto{\pgfqpoint{1.912932in}{0.774543in}}%
\pgfpathlineto{\pgfqpoint{1.923418in}{0.773495in}}%
\pgfpathlineto{\pgfqpoint{1.927003in}{0.764692in}}%
\pgfpathlineto{\pgfqpoint{1.930681in}{0.763152in}}%
\pgfpathlineto{\pgfqpoint{1.938397in}{0.758644in}}%
\pgfpathlineto{\pgfqpoint{1.944819in}{0.754750in}}%
\pgfpathlineto{\pgfqpoint{1.945246in}{0.752971in}}%
\pgfpathlineto{\pgfqpoint{1.949459in}{0.750264in}}%
\pgfpathlineto{\pgfqpoint{1.966417in}{0.745282in}}%
\pgfpathlineto{\pgfqpoint{1.984450in}{0.738340in}}%
\pgfpathlineto{\pgfqpoint{2.008926in}{0.733388in}}%
\pgfpathlineto{\pgfqpoint{2.031338in}{0.731724in}}%
\pgfpathlineto{\pgfqpoint{2.032252in}{0.729528in}}%
\pgfpathlineto{\pgfqpoint{2.033576in}{0.725697in}}%
\pgfpathlineto{\pgfqpoint{2.037125in}{0.724332in}}%
\pgfpathlineto{\pgfqpoint{2.045595in}{0.723715in}}%
\pgfpathlineto{\pgfqpoint{2.051349in}{0.722081in}}%
\pgfpathlineto{\pgfqpoint{2.076904in}{0.719544in}}%
\pgfpathlineto{\pgfqpoint{2.080251in}{0.718197in}}%
\pgfpathlineto{\pgfqpoint{2.080349in}{0.716749in}}%
\pgfpathlineto{\pgfqpoint{2.085843in}{0.716510in}}%
\pgfpathlineto{\pgfqpoint{2.098822in}{0.714079in}}%
\pgfpathlineto{\pgfqpoint{2.104964in}{0.713831in}}%
\pgfpathlineto{\pgfqpoint{2.105914in}{0.712924in}}%
\pgfpathlineto{\pgfqpoint{2.157055in}{0.710594in}}%
\pgfpathlineto{\pgfqpoint{2.161059in}{0.706043in}}%
\pgfpathlineto{\pgfqpoint{2.163334in}{0.705692in}}%
\pgfpathlineto{\pgfqpoint{2.171432in}{0.705082in}}%
\pgfpathlineto{\pgfqpoint{2.187458in}{0.702763in}}%
\pgfpathlineto{\pgfqpoint{2.193297in}{0.702172in}}%
\pgfpathlineto{\pgfqpoint{2.198690in}{0.701434in}}%
\pgfpathlineto{\pgfqpoint{2.212467in}{0.699905in}}%
\pgfpathlineto{\pgfqpoint{2.234540in}{0.699174in}}%
\pgfpathlineto{\pgfqpoint{2.259841in}{0.698842in}}%
\pgfpathlineto{\pgfqpoint{2.278129in}{0.698798in}}%
\pgfpathlineto{\pgfqpoint{2.301002in}{0.698784in}}%
\pgfpathlineto{\pgfqpoint{2.409554in}{0.697909in}}%
\pgfpathlineto{\pgfqpoint{2.438334in}{0.697795in}}%
\pgfpathlineto{\pgfqpoint{2.479468in}{0.697460in}}%
\pgfpathlineto{\pgfqpoint{2.539290in}{0.697351in}}%
\pgfpathlineto{\pgfqpoint{2.571667in}{0.696836in}}%
\pgfpathlineto{\pgfqpoint{2.647748in}{0.696667in}}%
\pgfpathlineto{\pgfqpoint{2.745840in}{0.696168in}}%
\pgfpathlineto{\pgfqpoint{2.823078in}{0.695324in}}%
\pgfpathlineto{\pgfqpoint{2.944612in}{0.694845in}}%
\pgfpathlineto{\pgfqpoint{3.108851in}{0.693891in}}%
\pgfpathlineto{\pgfqpoint{3.274341in}{0.693028in}}%
\pgfpathlineto{\pgfqpoint{3.517294in}{0.691835in}}%
\pgfpathlineto{\pgfqpoint{3.922083in}{0.690316in}}%
\pgfpathlineto{\pgfqpoint{4.623274in}{0.688107in}}%
\pgfpathlineto{\pgfqpoint{6.888192in}{0.683471in}}%
\pgfusepath{stroke}%
\end{pgfscope}%
\begin{pgfscope}%
\pgfsetrectcap%
\pgfsetmiterjoin%
\pgfsetlinewidth{0.803000pt}%
\definecolor{currentstroke}{rgb}{0.000000,0.000000,0.000000}%
\pgfsetstrokecolor{currentstroke}%
\pgfsetdash{}{0pt}%
\pgfpathmoveto{\pgfqpoint{0.688192in}{0.670138in}}%
\pgfpathlineto{\pgfqpoint{0.688192in}{5.290138in}}%
\pgfusepath{stroke}%
\end{pgfscope}%
\begin{pgfscope}%
\pgfsetrectcap%
\pgfsetmiterjoin%
\pgfsetlinewidth{0.803000pt}%
\definecolor{currentstroke}{rgb}{0.000000,0.000000,0.000000}%
\pgfsetstrokecolor{currentstroke}%
\pgfsetdash{}{0pt}%
\pgfpathmoveto{\pgfqpoint{6.888192in}{0.670138in}}%
\pgfpathlineto{\pgfqpoint{6.888192in}{5.290138in}}%
\pgfusepath{stroke}%
\end{pgfscope}%
\begin{pgfscope}%
\pgfsetrectcap%
\pgfsetmiterjoin%
\pgfsetlinewidth{0.803000pt}%
\definecolor{currentstroke}{rgb}{0.000000,0.000000,0.000000}%
\pgfsetstrokecolor{currentstroke}%
\pgfsetdash{}{0pt}%
\pgfpathmoveto{\pgfqpoint{0.688192in}{0.670138in}}%
\pgfpathlineto{\pgfqpoint{6.888192in}{0.670138in}}%
\pgfusepath{stroke}%
\end{pgfscope}%
\begin{pgfscope}%
\pgfsetrectcap%
\pgfsetmiterjoin%
\pgfsetlinewidth{0.803000pt}%
\definecolor{currentstroke}{rgb}{0.000000,0.000000,0.000000}%
\pgfsetstrokecolor{currentstroke}%
\pgfsetdash{}{0pt}%
\pgfpathmoveto{\pgfqpoint{0.688192in}{5.290138in}}%
\pgfpathlineto{\pgfqpoint{6.888192in}{5.290138in}}%
\pgfusepath{stroke}%
\end{pgfscope}%
\begin{pgfscope}%
\pgfsetbuttcap%
\pgfsetmiterjoin%
\pgfsetlinewidth{1.003750pt}%
\definecolor{currentstroke}{rgb}{0.000000,0.000000,0.000000}%
\pgfsetstrokecolor{currentstroke}%
\pgfsetstrokeopacity{0.500000}%
\pgfsetdash{}{0pt}%
\pgfpathmoveto{\pgfqpoint{0.646542in}{1.071430in}}%
\pgfpathlineto{\pgfqpoint{0.810103in}{1.071430in}}%
\pgfpathlineto{\pgfqpoint{0.810103in}{1.434970in}}%
\pgfpathlineto{\pgfqpoint{0.646542in}{1.434970in}}%
\pgfpathlineto{\pgfqpoint{0.646542in}{1.071430in}}%
\pgfpathclose%
\pgfpathmoveto{\pgfqpoint{3.788192in}{5.151538in}}%
\pgfpathquadraticcurveto{\pgfqpoint{2.217367in}{3.293254in}}{\pgfqpoint{0.646542in}{1.434970in}}%
\pgfpathmoveto{\pgfqpoint{6.702192in}{2.980138in}}%
\pgfpathquadraticcurveto{\pgfqpoint{3.756147in}{2.025784in}}{\pgfqpoint{0.810103in}{1.071430in}}%
\pgfusepath{stroke}%
\end{pgfscope}%
\begin{pgfscope}%
\pgfsetbuttcap%
\pgfsetmiterjoin%
\definecolor{currentfill}{rgb}{1.000000,1.000000,1.000000}%
\pgfsetfillcolor{currentfill}%
\pgfsetlinewidth{0.000000pt}%
\definecolor{currentstroke}{rgb}{0.000000,0.000000,0.000000}%
\pgfsetstrokecolor{currentstroke}%
\pgfsetstrokeopacity{0.000000}%
\pgfsetdash{}{0pt}%
\pgfpathmoveto{\pgfqpoint{3.788192in}{2.980138in}}%
\pgfpathlineto{\pgfqpoint{6.702192in}{2.980138in}}%
\pgfpathlineto{\pgfqpoint{6.702192in}{5.151538in}}%
\pgfpathlineto{\pgfqpoint{3.788192in}{5.151538in}}%
\pgfpathlineto{\pgfqpoint{3.788192in}{2.980138in}}%
\pgfpathclose%
\pgfusepath{fill}%
\end{pgfscope}%
\begin{pgfscope}%
\pgfpathrectangle{\pgfqpoint{3.788192in}{2.980138in}}{\pgfqpoint{2.914000in}{2.171400in}}%
\pgfusepath{clip}%
\pgfsetbuttcap%
\pgfsetmiterjoin%
\definecolor{currentfill}{rgb}{0.121569,0.466667,0.705882}%
\pgfsetfillcolor{currentfill}%
\pgfsetfillopacity{0.500000}%
\pgfsetlinewidth{1.003750pt}%
\definecolor{currentstroke}{rgb}{0.121569,0.466667,0.705882}%
\pgfsetstrokecolor{currentstroke}%
\pgfsetstrokeopacity{0.500000}%
\pgfsetdash{}{0pt}%
\pgfpathmoveto{\pgfqpoint{5.478622in}{4.808529in}}%
\pgfpathlineto{\pgfqpoint{5.786444in}{2.284897in}}%
\pgfpathlineto{\pgfqpoint{5.955602in}{1.807070in}}%
\pgfpathlineto{\pgfqpoint{6.072257in}{1.530809in}}%
\pgfpathlineto{\pgfqpoint{6.134561in}{1.493263in}}%
\pgfpathlineto{\pgfqpoint{6.204334in}{1.296852in}}%
\pgfpathlineto{\pgfqpoint{6.281957in}{1.238734in}}%
\pgfpathlineto{\pgfqpoint{6.389240in}{1.234341in}}%
\pgfpathlineto{\pgfqpoint{6.427201in}{1.108377in}}%
\pgfpathlineto{\pgfqpoint{6.519285in}{1.077759in}}%
\pgfpathlineto{\pgfqpoint{6.689124in}{1.072073in}}%
\pgfpathlineto{\pgfqpoint{6.747190in}{1.024269in}}%
\pgfpathlineto{\pgfqpoint{6.806750in}{1.015907in}}%
\pgfpathlineto{\pgfqpoint{6.931730in}{0.991431in}}%
\pgfpathlineto{\pgfqpoint{7.035734in}{0.970288in}}%
\pgfpathlineto{\pgfqpoint{7.042663in}{0.960624in}}%
\pgfpathlineto{\pgfqpoint{7.110896in}{0.945926in}}%
\pgfpathlineto{\pgfqpoint{7.385541in}{0.918874in}}%
\pgfpathlineto{\pgfqpoint{7.677622in}{0.881182in}}%
\pgfpathlineto{\pgfqpoint{8.074033in}{0.854291in}}%
\pgfpathlineto{\pgfqpoint{8.437037in}{0.845259in}}%
\pgfpathlineto{\pgfqpoint{8.451833in}{0.833333in}}%
\pgfpathlineto{\pgfqpoint{8.473270in}{0.812531in}}%
\pgfpathlineto{\pgfqpoint{8.530759in}{0.805118in}}%
\pgfpathlineto{\pgfqpoint{8.667946in}{0.801766in}}%
\pgfpathlineto{\pgfqpoint{8.761129in}{0.792896in}}%
\pgfpathlineto{\pgfqpoint{9.175032in}{0.779118in}}%
\pgfpathlineto{\pgfqpoint{9.229242in}{0.771806in}}%
\pgfpathlineto{\pgfqpoint{9.230836in}{0.763945in}}%
\pgfpathlineto{\pgfqpoint{9.319803in}{0.762647in}}%
\pgfpathlineto{\pgfqpoint{9.530028in}{0.749444in}}%
\pgfpathlineto{\pgfqpoint{9.629502in}{0.748101in}}%
\pgfpathlineto{\pgfqpoint{9.644888in}{0.743176in}}%
\pgfpathlineto{\pgfqpoint{10.473184in}{0.730520in}}%
\pgfpathlineto{\pgfqpoint{10.538033in}{0.705808in}}%
\pgfpathlineto{\pgfqpoint{10.574876in}{0.703903in}}%
\pgfpathlineto{\pgfqpoint{10.706030in}{0.700593in}}%
\pgfpathlineto{\pgfqpoint{10.965602in}{0.687998in}}%
\pgfpathlineto{\pgfqpoint{11.060169in}{0.684789in}}%
\pgfpathlineto{\pgfqpoint{11.147519in}{0.680781in}}%
\pgfpathlineto{\pgfqpoint{11.370645in}{0.672484in}}%
\pgfpathlineto{\pgfqpoint{11.728159in}{0.668514in}}%
\pgfpathlineto{\pgfqpoint{12.137942in}{0.666709in}}%
\pgfpathlineto{\pgfqpoint{12.434130in}{0.666468in}}%
\pgfpathlineto{\pgfqpoint{12.804602in}{0.666397in}}%
\pgfpathlineto{\pgfqpoint{14.562739in}{0.661643in}}%
\pgfpathlineto{\pgfqpoint{15.028873in}{0.661022in}}%
\pgfpathlineto{\pgfqpoint{15.695083in}{0.659204in}}%
\pgfpathlineto{\pgfqpoint{16.663977in}{0.658612in}}%
\pgfpathlineto{\pgfqpoint{17.188373in}{0.655818in}}%
\pgfpathlineto{\pgfqpoint{18.420605in}{0.654898in}}%
\pgfpathlineto{\pgfqpoint{20.009331in}{0.652189in}}%
\pgfpathlineto{\pgfqpoint{21.260300in}{0.647607in}}%
\pgfpathlineto{\pgfqpoint{23.228701in}{0.645006in}}%
\pgfpathlineto{\pgfqpoint{25.888770in}{0.639824in}}%
\pgfpathlineto{\pgfqpoint{28.569097in}{0.635141in}}%
\pgfpathlineto{\pgfqpoint{32.504050in}{0.628662in}}%
\pgfpathlineto{\pgfqpoint{39.060135in}{0.620415in}}%
\pgfpathlineto{\pgfqpoint{50.416853in}{0.608419in}}%
\pgfpathlineto{\pgfqpoint{87.100182in}{0.583248in}}%
\pgfpathlineto{\pgfqpoint{114.989116in}{0.662886in}}%
\pgfpathlineto{\pgfqpoint{74.637454in}{0.690574in}}%
\pgfpathlineto{\pgfqpoint{62.145064in}{0.703770in}}%
\pgfpathlineto{\pgfqpoint{54.933371in}{0.712842in}}%
\pgfpathlineto{\pgfqpoint{50.604922in}{0.719969in}}%
\pgfpathlineto{\pgfqpoint{47.656562in}{0.725121in}}%
\pgfpathlineto{\pgfqpoint{44.730486in}{0.730820in}}%
\pgfpathlineto{\pgfqpoint{42.565245in}{0.733682in}}%
\pgfpathlineto{\pgfqpoint{41.189180in}{0.738722in}}%
\pgfpathlineto{\pgfqpoint{39.441581in}{0.741702in}}%
\pgfpathlineto{\pgfqpoint{38.086125in}{0.742714in}}%
\pgfpathlineto{\pgfqpoint{37.509290in}{0.745787in}}%
\pgfpathlineto{\pgfqpoint{36.443507in}{0.746439in}}%
\pgfpathlineto{\pgfqpoint{35.710675in}{0.748438in}}%
\pgfpathlineto{\pgfqpoint{35.197928in}{0.749121in}}%
\pgfpathlineto{\pgfqpoint{33.263977in}{0.754350in}}%
\pgfpathlineto{\pgfqpoint{32.856458in}{0.754429in}}%
\pgfpathlineto{\pgfqpoint{32.530652in}{0.754694in}}%
\pgfpathlineto{\pgfqpoint{32.079890in}{0.756679in}}%
\pgfpathlineto{\pgfqpoint{31.686625in}{0.761046in}}%
\pgfpathlineto{\pgfqpoint{31.441186in}{0.770173in}}%
\pgfpathlineto{\pgfqpoint{31.345101in}{0.774582in}}%
\pgfpathlineto{\pgfqpoint{31.241077in}{0.778112in}}%
\pgfpathlineto{\pgfqpoint{30.955549in}{0.791966in}}%
\pgfpathlineto{\pgfqpoint{30.811278in}{0.795608in}}%
\pgfpathlineto{\pgfqpoint{30.770752in}{0.797703in}}%
\pgfpathlineto{\pgfqpoint{30.699417in}{0.824886in}}%
\pgfpathlineto{\pgfqpoint{29.788292in}{0.838808in}}%
\pgfpathlineto{\pgfqpoint{29.771368in}{0.844225in}}%
\pgfpathlineto{\pgfqpoint{29.661946in}{0.845702in}}%
\pgfpathlineto{\pgfqpoint{29.430699in}{0.860225in}}%
\pgfpathlineto{\pgfqpoint{29.332834in}{0.861654in}}%
\pgfpathlineto{\pgfqpoint{29.331081in}{0.870301in}}%
\pgfpathlineto{\pgfqpoint{29.271451in}{0.878344in}}%
\pgfpathlineto{\pgfqpoint{28.816157in}{0.893499in}}%
\pgfpathlineto{\pgfqpoint{28.713656in}{0.903257in}}%
\pgfpathlineto{\pgfqpoint{28.562750in}{0.906943in}}%
\pgfpathlineto{\pgfqpoint{28.499512in}{0.915098in}}%
\pgfpathlineto{\pgfqpoint{28.475931in}{0.937980in}}%
\pgfpathlineto{\pgfqpoint{28.459655in}{0.951098in}}%
\pgfpathlineto{\pgfqpoint{28.060351in}{0.961034in}}%
\pgfpathlineto{\pgfqpoint{27.624299in}{0.990614in}}%
\pgfpathlineto{\pgfqpoint{27.303010in}{1.032076in}}%
\pgfpathlineto{\pgfqpoint{27.000901in}{1.061832in}}%
\pgfpathlineto{\pgfqpoint{26.925844in}{1.078000in}}%
\pgfpathlineto{\pgfqpoint{26.918222in}{1.088630in}}%
\pgfpathlineto{\pgfqpoint{26.803818in}{1.111888in}}%
\pgfpathlineto{\pgfqpoint{26.666341in}{1.138811in}}%
\pgfpathlineto{\pgfqpoint{26.600824in}{1.148010in}}%
\pgfpathlineto{\pgfqpoint{26.536952in}{1.200594in}}%
\pgfpathlineto{\pgfqpoint{26.350128in}{1.206849in}}%
\pgfpathlineto{\pgfqpoint{26.248836in}{1.240528in}}%
\pgfpathlineto{\pgfqpoint{26.207079in}{1.379089in}}%
\pgfpathlineto{\pgfqpoint{26.089068in}{1.383921in}}%
\pgfpathlineto{\pgfqpoint{26.003683in}{1.447851in}}%
\pgfpathlineto{\pgfqpoint{25.926932in}{1.663903in}}%
\pgfpathlineto{\pgfqpoint{25.858398in}{1.705204in}}%
\pgfpathlineto{\pgfqpoint{25.730077in}{2.009091in}}%
\pgfpathlineto{\pgfqpoint{25.544004in}{2.534701in}}%
\pgfpathlineto{\pgfqpoint{25.205400in}{5.310696in}}%
\pgfpathlineto{\pgfqpoint{5.478622in}{4.808529in}}%
\pgfpathclose%
\pgfusepath{stroke,fill}%
\end{pgfscope}%
\begin{pgfscope}%
\pgfpathrectangle{\pgfqpoint{3.788192in}{2.980138in}}{\pgfqpoint{2.914000in}{2.171400in}}%
\pgfusepath{clip}%
\pgfsetbuttcap%
\pgfsetroundjoin%
\pgfsetlinewidth{1.003750pt}%
\definecolor{currentstroke}{rgb}{1.000000,0.000000,0.000000}%
\pgfsetstrokecolor{currentstroke}%
\pgfsetdash{}{0pt}%
\pgfpathmoveto{\pgfqpoint{16.861003in}{25.566105in}}%
\pgfpathcurveto{\pgfqpoint{16.869240in}{25.566105in}}{\pgfqpoint{16.877140in}{25.569377in}}{\pgfqpoint{16.882964in}{25.575201in}}%
\pgfpathcurveto{\pgfqpoint{16.888788in}{25.581025in}}{\pgfqpoint{16.892060in}{25.588925in}}{\pgfqpoint{16.892060in}{25.597161in}}%
\pgfpathcurveto{\pgfqpoint{16.892060in}{25.605397in}}{\pgfqpoint{16.888788in}{25.613297in}}{\pgfqpoint{16.882964in}{25.619121in}}%
\pgfpathcurveto{\pgfqpoint{16.877140in}{25.624945in}}{\pgfqpoint{16.869240in}{25.628218in}}{\pgfqpoint{16.861003in}{25.628218in}}%
\pgfpathcurveto{\pgfqpoint{16.852767in}{25.628218in}}{\pgfqpoint{16.844867in}{25.624945in}}{\pgfqpoint{16.839043in}{25.619121in}}%
\pgfpathcurveto{\pgfqpoint{16.833219in}{25.613297in}}{\pgfqpoint{16.829947in}{25.605397in}}{\pgfqpoint{16.829947in}{25.597161in}}%
\pgfpathcurveto{\pgfqpoint{16.829947in}{25.588925in}}{\pgfqpoint{16.833219in}{25.581025in}}{\pgfqpoint{16.839043in}{25.575201in}}%
\pgfpathcurveto{\pgfqpoint{16.844867in}{25.569377in}}{\pgfqpoint{16.852767in}{25.566105in}}{\pgfqpoint{16.861003in}{25.566105in}}%
\pgfusepath{stroke}%
\end{pgfscope}%
\begin{pgfscope}%
\pgfpathrectangle{\pgfqpoint{3.788192in}{2.980138in}}{\pgfqpoint{2.914000in}{2.171400in}}%
\pgfusepath{clip}%
\pgfsetbuttcap%
\pgfsetroundjoin%
\pgfsetlinewidth{1.003750pt}%
\definecolor{currentstroke}{rgb}{1.000000,0.000000,0.000000}%
\pgfsetstrokecolor{currentstroke}%
\pgfsetdash{}{0pt}%
\pgfpathmoveto{\pgfqpoint{11.934253in}{10.111217in}}%
\pgfpathcurveto{\pgfqpoint{11.942489in}{10.111217in}}{\pgfqpoint{11.950389in}{10.114489in}}{\pgfqpoint{11.956213in}{10.120313in}}%
\pgfpathcurveto{\pgfqpoint{11.962037in}{10.126137in}}{\pgfqpoint{11.965309in}{10.134037in}}{\pgfqpoint{11.965309in}{10.142274in}}%
\pgfpathcurveto{\pgfqpoint{11.965309in}{10.150510in}}{\pgfqpoint{11.962037in}{10.158410in}}{\pgfqpoint{11.956213in}{10.164234in}}%
\pgfpathcurveto{\pgfqpoint{11.950389in}{10.170058in}}{\pgfqpoint{11.942489in}{10.173330in}}{\pgfqpoint{11.934253in}{10.173330in}}%
\pgfpathcurveto{\pgfqpoint{11.926016in}{10.173330in}}{\pgfqpoint{11.918116in}{10.170058in}}{\pgfqpoint{11.912292in}{10.164234in}}%
\pgfpathcurveto{\pgfqpoint{11.906469in}{10.158410in}}{\pgfqpoint{11.903196in}{10.150510in}}{\pgfqpoint{11.903196in}{10.142274in}}%
\pgfpathcurveto{\pgfqpoint{11.903196in}{10.134037in}}{\pgfqpoint{11.906469in}{10.126137in}}{\pgfqpoint{11.912292in}{10.120313in}}%
\pgfpathcurveto{\pgfqpoint{11.918116in}{10.114489in}}{\pgfqpoint{11.926016in}{10.111217in}}{\pgfqpoint{11.934253in}{10.111217in}}%
\pgfusepath{stroke}%
\end{pgfscope}%
\begin{pgfscope}%
\pgfpathrectangle{\pgfqpoint{3.788192in}{2.980138in}}{\pgfqpoint{2.914000in}{2.171400in}}%
\pgfusepath{clip}%
\pgfsetbuttcap%
\pgfsetroundjoin%
\pgfsetlinewidth{1.003750pt}%
\definecolor{currentstroke}{rgb}{1.000000,0.000000,0.000000}%
\pgfsetstrokecolor{currentstroke}%
\pgfsetdash{}{0pt}%
\pgfpathmoveto{\pgfqpoint{12.691767in}{10.530438in}}%
\pgfpathcurveto{\pgfqpoint{12.700003in}{10.530438in}}{\pgfqpoint{12.707903in}{10.533710in}}{\pgfqpoint{12.713727in}{10.539534in}}%
\pgfpathcurveto{\pgfqpoint{12.719551in}{10.545358in}}{\pgfqpoint{12.722823in}{10.553258in}}{\pgfqpoint{12.722823in}{10.561494in}}%
\pgfpathcurveto{\pgfqpoint{12.722823in}{10.569730in}}{\pgfqpoint{12.719551in}{10.577630in}}{\pgfqpoint{12.713727in}{10.583454in}}%
\pgfpathcurveto{\pgfqpoint{12.707903in}{10.589278in}}{\pgfqpoint{12.700003in}{10.592551in}}{\pgfqpoint{12.691767in}{10.592551in}}%
\pgfpathcurveto{\pgfqpoint{12.683530in}{10.592551in}}{\pgfqpoint{12.675630in}{10.589278in}}{\pgfqpoint{12.669806in}{10.583454in}}%
\pgfpathcurveto{\pgfqpoint{12.663983in}{10.577630in}}{\pgfqpoint{12.660710in}{10.569730in}}{\pgfqpoint{12.660710in}{10.561494in}}%
\pgfpathcurveto{\pgfqpoint{12.660710in}{10.553258in}}{\pgfqpoint{12.663983in}{10.545358in}}{\pgfqpoint{12.669806in}{10.539534in}}%
\pgfpathcurveto{\pgfqpoint{12.675630in}{10.533710in}}{\pgfqpoint{12.683530in}{10.530438in}}{\pgfqpoint{12.691767in}{10.530438in}}%
\pgfusepath{stroke}%
\end{pgfscope}%
\begin{pgfscope}%
\pgfpathrectangle{\pgfqpoint{3.788192in}{2.980138in}}{\pgfqpoint{2.914000in}{2.171400in}}%
\pgfusepath{clip}%
\pgfsetbuttcap%
\pgfsetroundjoin%
\pgfsetlinewidth{1.003750pt}%
\definecolor{currentstroke}{rgb}{1.000000,0.000000,0.000000}%
\pgfsetstrokecolor{currentstroke}%
\pgfsetdash{}{0pt}%
\pgfpathmoveto{\pgfqpoint{15.144604in}{10.695451in}}%
\pgfpathcurveto{\pgfqpoint{15.152841in}{10.695451in}}{\pgfqpoint{15.160741in}{10.698723in}}{\pgfqpoint{15.166565in}{10.704547in}}%
\pgfpathcurveto{\pgfqpoint{15.172389in}{10.710371in}}{\pgfqpoint{15.175661in}{10.718271in}}{\pgfqpoint{15.175661in}{10.726507in}}%
\pgfpathcurveto{\pgfqpoint{15.175661in}{10.734743in}}{\pgfqpoint{15.172389in}{10.742643in}}{\pgfqpoint{15.166565in}{10.748467in}}%
\pgfpathcurveto{\pgfqpoint{15.160741in}{10.754291in}}{\pgfqpoint{15.152841in}{10.757564in}}{\pgfqpoint{15.144604in}{10.757564in}}%
\pgfpathcurveto{\pgfqpoint{15.136368in}{10.757564in}}{\pgfqpoint{15.128468in}{10.754291in}}{\pgfqpoint{15.122644in}{10.748467in}}%
\pgfpathcurveto{\pgfqpoint{15.116820in}{10.742643in}}{\pgfqpoint{15.113548in}{10.734743in}}{\pgfqpoint{15.113548in}{10.726507in}}%
\pgfpathcurveto{\pgfqpoint{15.113548in}{10.718271in}}{\pgfqpoint{15.116820in}{10.710371in}}{\pgfqpoint{15.122644in}{10.704547in}}%
\pgfpathcurveto{\pgfqpoint{15.128468in}{10.698723in}}{\pgfqpoint{15.136368in}{10.695451in}}{\pgfqpoint{15.144604in}{10.695451in}}%
\pgfusepath{stroke}%
\end{pgfscope}%
\begin{pgfscope}%
\pgfpathrectangle{\pgfqpoint{3.788192in}{2.980138in}}{\pgfqpoint{2.914000in}{2.171400in}}%
\pgfusepath{clip}%
\pgfsetbuttcap%
\pgfsetroundjoin%
\pgfsetlinewidth{1.003750pt}%
\definecolor{currentstroke}{rgb}{1.000000,0.000000,0.000000}%
\pgfsetstrokecolor{currentstroke}%
\pgfsetdash{}{0pt}%
\pgfpathmoveto{\pgfqpoint{13.542463in}{11.423191in}}%
\pgfpathcurveto{\pgfqpoint{13.550699in}{11.423191in}}{\pgfqpoint{13.558599in}{11.426464in}}{\pgfqpoint{13.564423in}{11.432288in}}%
\pgfpathcurveto{\pgfqpoint{13.570247in}{11.438112in}}{\pgfqpoint{13.573519in}{11.446012in}}{\pgfqpoint{13.573519in}{11.454248in}}%
\pgfpathcurveto{\pgfqpoint{13.573519in}{11.462484in}}{\pgfqpoint{13.570247in}{11.470384in}}{\pgfqpoint{13.564423in}{11.476208in}}%
\pgfpathcurveto{\pgfqpoint{13.558599in}{11.482032in}}{\pgfqpoint{13.550699in}{11.485304in}}{\pgfqpoint{13.542463in}{11.485304in}}%
\pgfpathcurveto{\pgfqpoint{13.534227in}{11.485304in}}{\pgfqpoint{13.526327in}{11.482032in}}{\pgfqpoint{13.520503in}{11.476208in}}%
\pgfpathcurveto{\pgfqpoint{13.514679in}{11.470384in}}{\pgfqpoint{13.511406in}{11.462484in}}{\pgfqpoint{13.511406in}{11.454248in}}%
\pgfpathcurveto{\pgfqpoint{13.511406in}{11.446012in}}{\pgfqpoint{13.514679in}{11.438112in}}{\pgfqpoint{13.520503in}{11.432288in}}%
\pgfpathcurveto{\pgfqpoint{13.526327in}{11.426464in}}{\pgfqpoint{13.534227in}{11.423191in}}{\pgfqpoint{13.542463in}{11.423191in}}%
\pgfusepath{stroke}%
\end{pgfscope}%
\begin{pgfscope}%
\pgfpathrectangle{\pgfqpoint{3.788192in}{2.980138in}}{\pgfqpoint{2.914000in}{2.171400in}}%
\pgfusepath{clip}%
\pgfsetbuttcap%
\pgfsetroundjoin%
\pgfsetlinewidth{1.003750pt}%
\definecolor{currentstroke}{rgb}{1.000000,0.000000,0.000000}%
\pgfsetstrokecolor{currentstroke}%
\pgfsetdash{}{0pt}%
\pgfpathmoveto{\pgfqpoint{12.332968in}{8.917460in}}%
\pgfpathcurveto{\pgfqpoint{12.341204in}{8.917460in}}{\pgfqpoint{12.349104in}{8.920732in}}{\pgfqpoint{12.354928in}{8.926556in}}%
\pgfpathcurveto{\pgfqpoint{12.360752in}{8.932380in}}{\pgfqpoint{12.364025in}{8.940280in}}{\pgfqpoint{12.364025in}{8.948517in}}%
\pgfpathcurveto{\pgfqpoint{12.364025in}{8.956753in}}{\pgfqpoint{12.360752in}{8.964653in}}{\pgfqpoint{12.354928in}{8.970477in}}%
\pgfpathcurveto{\pgfqpoint{12.349104in}{8.976301in}}{\pgfqpoint{12.341204in}{8.979573in}}{\pgfqpoint{12.332968in}{8.979573in}}%
\pgfpathcurveto{\pgfqpoint{12.324732in}{8.979573in}}{\pgfqpoint{12.316832in}{8.976301in}}{\pgfqpoint{12.311008in}{8.970477in}}%
\pgfpathcurveto{\pgfqpoint{12.305184in}{8.964653in}}{\pgfqpoint{12.301912in}{8.956753in}}{\pgfqpoint{12.301912in}{8.948517in}}%
\pgfpathcurveto{\pgfqpoint{12.301912in}{8.940280in}}{\pgfqpoint{12.305184in}{8.932380in}}{\pgfqpoint{12.311008in}{8.926556in}}%
\pgfpathcurveto{\pgfqpoint{12.316832in}{8.920732in}}{\pgfqpoint{12.324732in}{8.917460in}}{\pgfqpoint{12.332968in}{8.917460in}}%
\pgfusepath{stroke}%
\end{pgfscope}%
\begin{pgfscope}%
\pgfpathrectangle{\pgfqpoint{3.788192in}{2.980138in}}{\pgfqpoint{2.914000in}{2.171400in}}%
\pgfusepath{clip}%
\pgfsetbuttcap%
\pgfsetroundjoin%
\pgfsetlinewidth{1.003750pt}%
\definecolor{currentstroke}{rgb}{1.000000,0.000000,0.000000}%
\pgfsetstrokecolor{currentstroke}%
\pgfsetdash{}{0pt}%
\pgfpathmoveto{\pgfqpoint{11.399902in}{7.106716in}}%
\pgfpathcurveto{\pgfqpoint{11.408139in}{7.106716in}}{\pgfqpoint{11.416039in}{7.109988in}}{\pgfqpoint{11.421863in}{7.115812in}}%
\pgfpathcurveto{\pgfqpoint{11.427686in}{7.121636in}}{\pgfqpoint{11.430959in}{7.129536in}}{\pgfqpoint{11.430959in}{7.137773in}}%
\pgfpathcurveto{\pgfqpoint{11.430959in}{7.146009in}}{\pgfqpoint{11.427686in}{7.153909in}}{\pgfqpoint{11.421863in}{7.159733in}}%
\pgfpathcurveto{\pgfqpoint{11.416039in}{7.165557in}}{\pgfqpoint{11.408139in}{7.168829in}}{\pgfqpoint{11.399902in}{7.168829in}}%
\pgfpathcurveto{\pgfqpoint{11.391666in}{7.168829in}}{\pgfqpoint{11.383766in}{7.165557in}}{\pgfqpoint{11.377942in}{7.159733in}}%
\pgfpathcurveto{\pgfqpoint{11.372118in}{7.153909in}}{\pgfqpoint{11.368846in}{7.146009in}}{\pgfqpoint{11.368846in}{7.137773in}}%
\pgfpathcurveto{\pgfqpoint{11.368846in}{7.129536in}}{\pgfqpoint{11.372118in}{7.121636in}}{\pgfqpoint{11.377942in}{7.115812in}}%
\pgfpathcurveto{\pgfqpoint{11.383766in}{7.109988in}}{\pgfqpoint{11.391666in}{7.106716in}}{\pgfqpoint{11.399902in}{7.106716in}}%
\pgfusepath{stroke}%
\end{pgfscope}%
\begin{pgfscope}%
\pgfpathrectangle{\pgfqpoint{3.788192in}{2.980138in}}{\pgfqpoint{2.914000in}{2.171400in}}%
\pgfusepath{clip}%
\pgfsetbuttcap%
\pgfsetroundjoin%
\pgfsetlinewidth{1.003750pt}%
\definecolor{currentstroke}{rgb}{1.000000,0.000000,0.000000}%
\pgfsetstrokecolor{currentstroke}%
\pgfsetdash{}{0pt}%
\pgfpathmoveto{\pgfqpoint{11.356549in}{7.029726in}}%
\pgfpathcurveto{\pgfqpoint{11.364785in}{7.029726in}}{\pgfqpoint{11.372685in}{7.032998in}}{\pgfqpoint{11.378509in}{7.038822in}}%
\pgfpathcurveto{\pgfqpoint{11.384333in}{7.044646in}}{\pgfqpoint{11.387605in}{7.052546in}}{\pgfqpoint{11.387605in}{7.060782in}}%
\pgfpathcurveto{\pgfqpoint{11.387605in}{7.069018in}}{\pgfqpoint{11.384333in}{7.076918in}}{\pgfqpoint{11.378509in}{7.082742in}}%
\pgfpathcurveto{\pgfqpoint{11.372685in}{7.088566in}}{\pgfqpoint{11.364785in}{7.091839in}}{\pgfqpoint{11.356549in}{7.091839in}}%
\pgfpathcurveto{\pgfqpoint{11.348313in}{7.091839in}}{\pgfqpoint{11.340412in}{7.088566in}}{\pgfqpoint{11.334589in}{7.082742in}}%
\pgfpathcurveto{\pgfqpoint{11.328765in}{7.076918in}}{\pgfqpoint{11.325492in}{7.069018in}}{\pgfqpoint{11.325492in}{7.060782in}}%
\pgfpathcurveto{\pgfqpoint{11.325492in}{7.052546in}}{\pgfqpoint{11.328765in}{7.044646in}}{\pgfqpoint{11.334589in}{7.038822in}}%
\pgfpathcurveto{\pgfqpoint{11.340412in}{7.032998in}}{\pgfqpoint{11.348313in}{7.029726in}}{\pgfqpoint{11.356549in}{7.029726in}}%
\pgfusepath{stroke}%
\end{pgfscope}%
\begin{pgfscope}%
\pgfpathrectangle{\pgfqpoint{3.788192in}{2.980138in}}{\pgfqpoint{2.914000in}{2.171400in}}%
\pgfusepath{clip}%
\pgfsetbuttcap%
\pgfsetroundjoin%
\pgfsetlinewidth{1.003750pt}%
\definecolor{currentstroke}{rgb}{1.000000,0.000000,0.000000}%
\pgfsetstrokecolor{currentstroke}%
\pgfsetdash{}{0pt}%
\pgfpathmoveto{\pgfqpoint{13.232346in}{9.803683in}}%
\pgfpathcurveto{\pgfqpoint{13.240582in}{9.803683in}}{\pgfqpoint{13.248482in}{9.806956in}}{\pgfqpoint{13.254306in}{9.812780in}}%
\pgfpathcurveto{\pgfqpoint{13.260130in}{9.818604in}}{\pgfqpoint{13.263402in}{9.826504in}}{\pgfqpoint{13.263402in}{9.834740in}}%
\pgfpathcurveto{\pgfqpoint{13.263402in}{9.842976in}}{\pgfqpoint{13.260130in}{9.850876in}}{\pgfqpoint{13.254306in}{9.856700in}}%
\pgfpathcurveto{\pgfqpoint{13.248482in}{9.862524in}}{\pgfqpoint{13.240582in}{9.865796in}}{\pgfqpoint{13.232346in}{9.865796in}}%
\pgfpathcurveto{\pgfqpoint{13.224110in}{9.865796in}}{\pgfqpoint{13.216210in}{9.862524in}}{\pgfqpoint{13.210386in}{9.856700in}}%
\pgfpathcurveto{\pgfqpoint{13.204562in}{9.850876in}}{\pgfqpoint{13.201289in}{9.842976in}}{\pgfqpoint{13.201289in}{9.834740in}}%
\pgfpathcurveto{\pgfqpoint{13.201289in}{9.826504in}}{\pgfqpoint{13.204562in}{9.818604in}}{\pgfqpoint{13.210386in}{9.812780in}}%
\pgfpathcurveto{\pgfqpoint{13.216210in}{9.806956in}}{\pgfqpoint{13.224110in}{9.803683in}}{\pgfqpoint{13.232346in}{9.803683in}}%
\pgfusepath{stroke}%
\end{pgfscope}%
\begin{pgfscope}%
\pgfpathrectangle{\pgfqpoint{3.788192in}{2.980138in}}{\pgfqpoint{2.914000in}{2.171400in}}%
\pgfusepath{clip}%
\pgfsetbuttcap%
\pgfsetroundjoin%
\pgfsetlinewidth{1.003750pt}%
\definecolor{currentstroke}{rgb}{1.000000,0.000000,0.000000}%
\pgfsetstrokecolor{currentstroke}%
\pgfsetdash{}{0pt}%
\pgfpathmoveto{\pgfqpoint{12.940652in}{8.318719in}}%
\pgfpathcurveto{\pgfqpoint{12.948888in}{8.318719in}}{\pgfqpoint{12.956788in}{8.321992in}}{\pgfqpoint{12.962612in}{8.327816in}}%
\pgfpathcurveto{\pgfqpoint{12.968436in}{8.333640in}}{\pgfqpoint{12.971708in}{8.341540in}}{\pgfqpoint{12.971708in}{8.349776in}}%
\pgfpathcurveto{\pgfqpoint{12.971708in}{8.358012in}}{\pgfqpoint{12.968436in}{8.365912in}}{\pgfqpoint{12.962612in}{8.371736in}}%
\pgfpathcurveto{\pgfqpoint{12.956788in}{8.377560in}}{\pgfqpoint{12.948888in}{8.380832in}}{\pgfqpoint{12.940652in}{8.380832in}}%
\pgfpathcurveto{\pgfqpoint{12.932415in}{8.380832in}}{\pgfqpoint{12.924515in}{8.377560in}}{\pgfqpoint{12.918691in}{8.371736in}}%
\pgfpathcurveto{\pgfqpoint{12.912867in}{8.365912in}}{\pgfqpoint{12.909595in}{8.358012in}}{\pgfqpoint{12.909595in}{8.349776in}}%
\pgfpathcurveto{\pgfqpoint{12.909595in}{8.341540in}}{\pgfqpoint{12.912867in}{8.333640in}}{\pgfqpoint{12.918691in}{8.327816in}}%
\pgfpathcurveto{\pgfqpoint{12.924515in}{8.321992in}}{\pgfqpoint{12.932415in}{8.318719in}}{\pgfqpoint{12.940652in}{8.318719in}}%
\pgfusepath{stroke}%
\end{pgfscope}%
\begin{pgfscope}%
\pgfpathrectangle{\pgfqpoint{3.788192in}{2.980138in}}{\pgfqpoint{2.914000in}{2.171400in}}%
\pgfusepath{clip}%
\pgfsetbuttcap%
\pgfsetroundjoin%
\pgfsetlinewidth{1.003750pt}%
\definecolor{currentstroke}{rgb}{1.000000,0.000000,0.000000}%
\pgfsetstrokecolor{currentstroke}%
\pgfsetdash{}{0pt}%
\pgfpathmoveto{\pgfqpoint{12.936873in}{8.298012in}}%
\pgfpathcurveto{\pgfqpoint{12.945109in}{8.298012in}}{\pgfqpoint{12.953009in}{8.301284in}}{\pgfqpoint{12.958833in}{8.307108in}}%
\pgfpathcurveto{\pgfqpoint{12.964657in}{8.312932in}}{\pgfqpoint{12.967929in}{8.320832in}}{\pgfqpoint{12.967929in}{8.329068in}}%
\pgfpathcurveto{\pgfqpoint{12.967929in}{8.337305in}}{\pgfqpoint{12.964657in}{8.345205in}}{\pgfqpoint{12.958833in}{8.351028in}}%
\pgfpathcurveto{\pgfqpoint{12.953009in}{8.356852in}}{\pgfqpoint{12.945109in}{8.360125in}}{\pgfqpoint{12.936873in}{8.360125in}}%
\pgfpathcurveto{\pgfqpoint{12.928637in}{8.360125in}}{\pgfqpoint{12.920737in}{8.356852in}}{\pgfqpoint{12.914913in}{8.351028in}}%
\pgfpathcurveto{\pgfqpoint{12.909089in}{8.345205in}}{\pgfqpoint{12.905816in}{8.337305in}}{\pgfqpoint{12.905816in}{8.329068in}}%
\pgfpathcurveto{\pgfqpoint{12.905816in}{8.320832in}}{\pgfqpoint{12.909089in}{8.312932in}}{\pgfqpoint{12.914913in}{8.307108in}}%
\pgfpathcurveto{\pgfqpoint{12.920737in}{8.301284in}}{\pgfqpoint{12.928637in}{8.298012in}}{\pgfqpoint{12.936873in}{8.298012in}}%
\pgfusepath{stroke}%
\end{pgfscope}%
\begin{pgfscope}%
\pgfpathrectangle{\pgfqpoint{3.788192in}{2.980138in}}{\pgfqpoint{2.914000in}{2.171400in}}%
\pgfusepath{clip}%
\pgfsetbuttcap%
\pgfsetroundjoin%
\pgfsetlinewidth{1.003750pt}%
\definecolor{currentstroke}{rgb}{1.000000,0.000000,0.000000}%
\pgfsetstrokecolor{currentstroke}%
\pgfsetdash{}{0pt}%
\pgfpathmoveto{\pgfqpoint{12.900060in}{6.352602in}}%
\pgfpathcurveto{\pgfqpoint{12.908296in}{6.352602in}}{\pgfqpoint{12.916196in}{6.355874in}}{\pgfqpoint{12.922020in}{6.361698in}}%
\pgfpathcurveto{\pgfqpoint{12.927844in}{6.367522in}}{\pgfqpoint{12.931116in}{6.375422in}}{\pgfqpoint{12.931116in}{6.383658in}}%
\pgfpathcurveto{\pgfqpoint{12.931116in}{6.391894in}}{\pgfqpoint{12.927844in}{6.399794in}}{\pgfqpoint{12.922020in}{6.405618in}}%
\pgfpathcurveto{\pgfqpoint{12.916196in}{6.411442in}}{\pgfqpoint{12.908296in}{6.414715in}}{\pgfqpoint{12.900060in}{6.414715in}}%
\pgfpathcurveto{\pgfqpoint{12.891823in}{6.414715in}}{\pgfqpoint{12.883923in}{6.411442in}}{\pgfqpoint{12.878099in}{6.405618in}}%
\pgfpathcurveto{\pgfqpoint{12.872275in}{6.399794in}}{\pgfqpoint{12.869003in}{6.391894in}}{\pgfqpoint{12.869003in}{6.383658in}}%
\pgfpathcurveto{\pgfqpoint{12.869003in}{6.375422in}}{\pgfqpoint{12.872275in}{6.367522in}}{\pgfqpoint{12.878099in}{6.361698in}}%
\pgfpathcurveto{\pgfqpoint{12.883923in}{6.355874in}}{\pgfqpoint{12.891823in}{6.352602in}}{\pgfqpoint{12.900060in}{6.352602in}}%
\pgfusepath{stroke}%
\end{pgfscope}%
\begin{pgfscope}%
\pgfpathrectangle{\pgfqpoint{3.788192in}{2.980138in}}{\pgfqpoint{2.914000in}{2.171400in}}%
\pgfusepath{clip}%
\pgfsetbuttcap%
\pgfsetroundjoin%
\pgfsetlinewidth{1.003750pt}%
\definecolor{currentstroke}{rgb}{1.000000,0.000000,0.000000}%
\pgfsetstrokecolor{currentstroke}%
\pgfsetdash{}{0pt}%
\pgfpathmoveto{\pgfqpoint{12.861075in}{6.367856in}}%
\pgfpathcurveto{\pgfqpoint{12.869311in}{6.367856in}}{\pgfqpoint{12.877211in}{6.371128in}}{\pgfqpoint{12.883035in}{6.376952in}}%
\pgfpathcurveto{\pgfqpoint{12.888859in}{6.382776in}}{\pgfqpoint{12.892131in}{6.390676in}}{\pgfqpoint{12.892131in}{6.398913in}}%
\pgfpathcurveto{\pgfqpoint{12.892131in}{6.407149in}}{\pgfqpoint{12.888859in}{6.415049in}}{\pgfqpoint{12.883035in}{6.420873in}}%
\pgfpathcurveto{\pgfqpoint{12.877211in}{6.426697in}}{\pgfqpoint{12.869311in}{6.429969in}}{\pgfqpoint{12.861075in}{6.429969in}}%
\pgfpathcurveto{\pgfqpoint{12.852838in}{6.429969in}}{\pgfqpoint{12.844938in}{6.426697in}}{\pgfqpoint{12.839114in}{6.420873in}}%
\pgfpathcurveto{\pgfqpoint{12.833291in}{6.415049in}}{\pgfqpoint{12.830018in}{6.407149in}}{\pgfqpoint{12.830018in}{6.398913in}}%
\pgfpathcurveto{\pgfqpoint{12.830018in}{6.390676in}}{\pgfqpoint{12.833291in}{6.382776in}}{\pgfqpoint{12.839114in}{6.376952in}}%
\pgfpathcurveto{\pgfqpoint{12.844938in}{6.371128in}}{\pgfqpoint{12.852838in}{6.367856in}}{\pgfqpoint{12.861075in}{6.367856in}}%
\pgfusepath{stroke}%
\end{pgfscope}%
\begin{pgfscope}%
\pgfpathrectangle{\pgfqpoint{3.788192in}{2.980138in}}{\pgfqpoint{2.914000in}{2.171400in}}%
\pgfusepath{clip}%
\pgfsetbuttcap%
\pgfsetroundjoin%
\pgfsetlinewidth{1.003750pt}%
\definecolor{currentstroke}{rgb}{1.000000,0.000000,0.000000}%
\pgfsetstrokecolor{currentstroke}%
\pgfsetdash{}{0pt}%
\pgfpathmoveto{\pgfqpoint{15.347266in}{8.054762in}}%
\pgfpathcurveto{\pgfqpoint{15.355502in}{8.054762in}}{\pgfqpoint{15.363402in}{8.058034in}}{\pgfqpoint{15.369226in}{8.063858in}}%
\pgfpathcurveto{\pgfqpoint{15.375050in}{8.069682in}}{\pgfqpoint{15.378323in}{8.077582in}}{\pgfqpoint{15.378323in}{8.085818in}}%
\pgfpathcurveto{\pgfqpoint{15.378323in}{8.094054in}}{\pgfqpoint{15.375050in}{8.101954in}}{\pgfqpoint{15.369226in}{8.107778in}}%
\pgfpathcurveto{\pgfqpoint{15.363402in}{8.113602in}}{\pgfqpoint{15.355502in}{8.116875in}}{\pgfqpoint{15.347266in}{8.116875in}}%
\pgfpathcurveto{\pgfqpoint{15.339030in}{8.116875in}}{\pgfqpoint{15.331130in}{8.113602in}}{\pgfqpoint{15.325306in}{8.107778in}}%
\pgfpathcurveto{\pgfqpoint{15.319482in}{8.101954in}}{\pgfqpoint{15.316210in}{8.094054in}}{\pgfqpoint{15.316210in}{8.085818in}}%
\pgfpathcurveto{\pgfqpoint{15.316210in}{8.077582in}}{\pgfqpoint{15.319482in}{8.069682in}}{\pgfqpoint{15.325306in}{8.063858in}}%
\pgfpathcurveto{\pgfqpoint{15.331130in}{8.058034in}}{\pgfqpoint{15.339030in}{8.054762in}}{\pgfqpoint{15.347266in}{8.054762in}}%
\pgfusepath{stroke}%
\end{pgfscope}%
\begin{pgfscope}%
\pgfpathrectangle{\pgfqpoint{3.788192in}{2.980138in}}{\pgfqpoint{2.914000in}{2.171400in}}%
\pgfusepath{clip}%
\pgfsetbuttcap%
\pgfsetroundjoin%
\pgfsetlinewidth{1.003750pt}%
\definecolor{currentstroke}{rgb}{1.000000,0.000000,0.000000}%
\pgfsetstrokecolor{currentstroke}%
\pgfsetdash{}{0pt}%
\pgfpathmoveto{\pgfqpoint{12.945472in}{6.264272in}}%
\pgfpathcurveto{\pgfqpoint{12.953708in}{6.264272in}}{\pgfqpoint{12.961608in}{6.267545in}}{\pgfqpoint{12.967432in}{6.273369in}}%
\pgfpathcurveto{\pgfqpoint{12.973256in}{6.279193in}}{\pgfqpoint{12.976528in}{6.287093in}}{\pgfqpoint{12.976528in}{6.295329in}}%
\pgfpathcurveto{\pgfqpoint{12.976528in}{6.303565in}}{\pgfqpoint{12.973256in}{6.311465in}}{\pgfqpoint{12.967432in}{6.317289in}}%
\pgfpathcurveto{\pgfqpoint{12.961608in}{6.323113in}}{\pgfqpoint{12.953708in}{6.326385in}}{\pgfqpoint{12.945472in}{6.326385in}}%
\pgfpathcurveto{\pgfqpoint{12.937236in}{6.326385in}}{\pgfqpoint{12.929336in}{6.323113in}}{\pgfqpoint{12.923512in}{6.317289in}}%
\pgfpathcurveto{\pgfqpoint{12.917688in}{6.311465in}}{\pgfqpoint{12.914415in}{6.303565in}}{\pgfqpoint{12.914415in}{6.295329in}}%
\pgfpathcurveto{\pgfqpoint{12.914415in}{6.287093in}}{\pgfqpoint{12.917688in}{6.279193in}}{\pgfqpoint{12.923512in}{6.273369in}}%
\pgfpathcurveto{\pgfqpoint{12.929336in}{6.267545in}}{\pgfqpoint{12.937236in}{6.264272in}}{\pgfqpoint{12.945472in}{6.264272in}}%
\pgfusepath{stroke}%
\end{pgfscope}%
\begin{pgfscope}%
\pgfpathrectangle{\pgfqpoint{3.788192in}{2.980138in}}{\pgfqpoint{2.914000in}{2.171400in}}%
\pgfusepath{clip}%
\pgfsetbuttcap%
\pgfsetroundjoin%
\pgfsetlinewidth{1.003750pt}%
\definecolor{currentstroke}{rgb}{1.000000,0.000000,0.000000}%
\pgfsetstrokecolor{currentstroke}%
\pgfsetdash{}{0pt}%
\pgfpathmoveto{\pgfqpoint{14.816780in}{6.837016in}}%
\pgfpathcurveto{\pgfqpoint{14.825016in}{6.837016in}}{\pgfqpoint{14.832916in}{6.840288in}}{\pgfqpoint{14.838740in}{6.846112in}}%
\pgfpathcurveto{\pgfqpoint{14.844564in}{6.851936in}}{\pgfqpoint{14.847836in}{6.859836in}}{\pgfqpoint{14.847836in}{6.868073in}}%
\pgfpathcurveto{\pgfqpoint{14.847836in}{6.876309in}}{\pgfqpoint{14.844564in}{6.884209in}}{\pgfqpoint{14.838740in}{6.890033in}}%
\pgfpathcurveto{\pgfqpoint{14.832916in}{6.895857in}}{\pgfqpoint{14.825016in}{6.899129in}}{\pgfqpoint{14.816780in}{6.899129in}}%
\pgfpathcurveto{\pgfqpoint{14.808543in}{6.899129in}}{\pgfqpoint{14.800643in}{6.895857in}}{\pgfqpoint{14.794819in}{6.890033in}}%
\pgfpathcurveto{\pgfqpoint{14.788995in}{6.884209in}}{\pgfqpoint{14.785723in}{6.876309in}}{\pgfqpoint{14.785723in}{6.868073in}}%
\pgfpathcurveto{\pgfqpoint{14.785723in}{6.859836in}}{\pgfqpoint{14.788995in}{6.851936in}}{\pgfqpoint{14.794819in}{6.846112in}}%
\pgfpathcurveto{\pgfqpoint{14.800643in}{6.840288in}}{\pgfqpoint{14.808543in}{6.837016in}}{\pgfqpoint{14.816780in}{6.837016in}}%
\pgfusepath{stroke}%
\end{pgfscope}%
\begin{pgfscope}%
\pgfpathrectangle{\pgfqpoint{3.788192in}{2.980138in}}{\pgfqpoint{2.914000in}{2.171400in}}%
\pgfusepath{clip}%
\pgfsetbuttcap%
\pgfsetroundjoin%
\pgfsetlinewidth{1.003750pt}%
\definecolor{currentstroke}{rgb}{1.000000,0.000000,0.000000}%
\pgfsetstrokecolor{currentstroke}%
\pgfsetdash{}{0pt}%
\pgfpathmoveto{\pgfqpoint{15.389681in}{6.512387in}}%
\pgfpathcurveto{\pgfqpoint{15.397917in}{6.512387in}}{\pgfqpoint{15.405817in}{6.515659in}}{\pgfqpoint{15.411641in}{6.521483in}}%
\pgfpathcurveto{\pgfqpoint{15.417465in}{6.527307in}}{\pgfqpoint{15.420738in}{6.535207in}}{\pgfqpoint{15.420738in}{6.543443in}}%
\pgfpathcurveto{\pgfqpoint{15.420738in}{6.551679in}}{\pgfqpoint{15.417465in}{6.559579in}}{\pgfqpoint{15.411641in}{6.565403in}}%
\pgfpathcurveto{\pgfqpoint{15.405817in}{6.571227in}}{\pgfqpoint{15.397917in}{6.574500in}}{\pgfqpoint{15.389681in}{6.574500in}}%
\pgfpathcurveto{\pgfqpoint{15.381445in}{6.574500in}}{\pgfqpoint{15.373545in}{6.571227in}}{\pgfqpoint{15.367721in}{6.565403in}}%
\pgfpathcurveto{\pgfqpoint{15.361897in}{6.559579in}}{\pgfqpoint{15.358625in}{6.551679in}}{\pgfqpoint{15.358625in}{6.543443in}}%
\pgfpathcurveto{\pgfqpoint{15.358625in}{6.535207in}}{\pgfqpoint{15.361897in}{6.527307in}}{\pgfqpoint{15.367721in}{6.521483in}}%
\pgfpathcurveto{\pgfqpoint{15.373545in}{6.515659in}}{\pgfqpoint{15.381445in}{6.512387in}}{\pgfqpoint{15.389681in}{6.512387in}}%
\pgfusepath{stroke}%
\end{pgfscope}%
\begin{pgfscope}%
\pgfpathrectangle{\pgfqpoint{3.788192in}{2.980138in}}{\pgfqpoint{2.914000in}{2.171400in}}%
\pgfusepath{clip}%
\pgfsetbuttcap%
\pgfsetroundjoin%
\pgfsetlinewidth{1.003750pt}%
\definecolor{currentstroke}{rgb}{1.000000,0.000000,0.000000}%
\pgfsetstrokecolor{currentstroke}%
\pgfsetdash{}{0pt}%
\pgfpathmoveto{\pgfqpoint{12.707501in}{7.095383in}}%
\pgfpathcurveto{\pgfqpoint{12.715737in}{7.095383in}}{\pgfqpoint{12.723637in}{7.098656in}}{\pgfqpoint{12.729461in}{7.104479in}}%
\pgfpathcurveto{\pgfqpoint{12.735285in}{7.110303in}}{\pgfqpoint{12.738557in}{7.118203in}}{\pgfqpoint{12.738557in}{7.126440in}}%
\pgfpathcurveto{\pgfqpoint{12.738557in}{7.134676in}}{\pgfqpoint{12.735285in}{7.142576in}}{\pgfqpoint{12.729461in}{7.148400in}}%
\pgfpathcurveto{\pgfqpoint{12.723637in}{7.154224in}}{\pgfqpoint{12.715737in}{7.157496in}}{\pgfqpoint{12.707501in}{7.157496in}}%
\pgfpathcurveto{\pgfqpoint{12.699265in}{7.157496in}}{\pgfqpoint{12.691365in}{7.154224in}}{\pgfqpoint{12.685541in}{7.148400in}}%
\pgfpathcurveto{\pgfqpoint{12.679717in}{7.142576in}}{\pgfqpoint{12.676444in}{7.134676in}}{\pgfqpoint{12.676444in}{7.126440in}}%
\pgfpathcurveto{\pgfqpoint{12.676444in}{7.118203in}}{\pgfqpoint{12.679717in}{7.110303in}}{\pgfqpoint{12.685541in}{7.104479in}}%
\pgfpathcurveto{\pgfqpoint{12.691365in}{7.098656in}}{\pgfqpoint{12.699265in}{7.095383in}}{\pgfqpoint{12.707501in}{7.095383in}}%
\pgfusepath{stroke}%
\end{pgfscope}%
\begin{pgfscope}%
\pgfpathrectangle{\pgfqpoint{3.788192in}{2.980138in}}{\pgfqpoint{2.914000in}{2.171400in}}%
\pgfusepath{clip}%
\pgfsetbuttcap%
\pgfsetroundjoin%
\pgfsetlinewidth{1.003750pt}%
\definecolor{currentstroke}{rgb}{1.000000,0.000000,0.000000}%
\pgfsetstrokecolor{currentstroke}%
\pgfsetdash{}{0pt}%
\pgfpathmoveto{\pgfqpoint{14.362942in}{5.990566in}}%
\pgfpathcurveto{\pgfqpoint{14.371178in}{5.990566in}}{\pgfqpoint{14.379078in}{5.993838in}}{\pgfqpoint{14.384902in}{5.999662in}}%
\pgfpathcurveto{\pgfqpoint{14.390726in}{6.005486in}}{\pgfqpoint{14.393998in}{6.013386in}}{\pgfqpoint{14.393998in}{6.021622in}}%
\pgfpathcurveto{\pgfqpoint{14.393998in}{6.029859in}}{\pgfqpoint{14.390726in}{6.037759in}}{\pgfqpoint{14.384902in}{6.043583in}}%
\pgfpathcurveto{\pgfqpoint{14.379078in}{6.049407in}}{\pgfqpoint{14.371178in}{6.052679in}}{\pgfqpoint{14.362942in}{6.052679in}}%
\pgfpathcurveto{\pgfqpoint{14.354705in}{6.052679in}}{\pgfqpoint{14.346805in}{6.049407in}}{\pgfqpoint{14.340982in}{6.043583in}}%
\pgfpathcurveto{\pgfqpoint{14.335158in}{6.037759in}}{\pgfqpoint{14.331885in}{6.029859in}}{\pgfqpoint{14.331885in}{6.021622in}}%
\pgfpathcurveto{\pgfqpoint{14.331885in}{6.013386in}}{\pgfqpoint{14.335158in}{6.005486in}}{\pgfqpoint{14.340982in}{5.999662in}}%
\pgfpathcurveto{\pgfqpoint{14.346805in}{5.993838in}}{\pgfqpoint{14.354705in}{5.990566in}}{\pgfqpoint{14.362942in}{5.990566in}}%
\pgfusepath{stroke}%
\end{pgfscope}%
\begin{pgfscope}%
\pgfpathrectangle{\pgfqpoint{3.788192in}{2.980138in}}{\pgfqpoint{2.914000in}{2.171400in}}%
\pgfusepath{clip}%
\pgfsetbuttcap%
\pgfsetroundjoin%
\pgfsetlinewidth{1.003750pt}%
\definecolor{currentstroke}{rgb}{1.000000,0.000000,0.000000}%
\pgfsetstrokecolor{currentstroke}%
\pgfsetdash{}{0pt}%
\pgfpathmoveto{\pgfqpoint{14.985061in}{5.541293in}}%
\pgfpathcurveto{\pgfqpoint{14.993297in}{5.541293in}}{\pgfqpoint{15.001197in}{5.544566in}}{\pgfqpoint{15.007021in}{5.550390in}}%
\pgfpathcurveto{\pgfqpoint{15.012845in}{5.556213in}}{\pgfqpoint{15.016118in}{5.564113in}}{\pgfqpoint{15.016118in}{5.572350in}}%
\pgfpathcurveto{\pgfqpoint{15.016118in}{5.580586in}}{\pgfqpoint{15.012845in}{5.588486in}}{\pgfqpoint{15.007021in}{5.594310in}}%
\pgfpathcurveto{\pgfqpoint{15.001197in}{5.600134in}}{\pgfqpoint{14.993297in}{5.603406in}}{\pgfqpoint{14.985061in}{5.603406in}}%
\pgfpathcurveto{\pgfqpoint{14.976825in}{5.603406in}}{\pgfqpoint{14.968925in}{5.600134in}}{\pgfqpoint{14.963101in}{5.594310in}}%
\pgfpathcurveto{\pgfqpoint{14.957277in}{5.588486in}}{\pgfqpoint{14.954005in}{5.580586in}}{\pgfqpoint{14.954005in}{5.572350in}}%
\pgfpathcurveto{\pgfqpoint{14.954005in}{5.564113in}}{\pgfqpoint{14.957277in}{5.556213in}}{\pgfqpoint{14.963101in}{5.550390in}}%
\pgfpathcurveto{\pgfqpoint{14.968925in}{5.544566in}}{\pgfqpoint{14.976825in}{5.541293in}}{\pgfqpoint{14.985061in}{5.541293in}}%
\pgfusepath{stroke}%
\end{pgfscope}%
\begin{pgfscope}%
\pgfpathrectangle{\pgfqpoint{3.788192in}{2.980138in}}{\pgfqpoint{2.914000in}{2.171400in}}%
\pgfusepath{clip}%
\pgfsetbuttcap%
\pgfsetroundjoin%
\pgfsetlinewidth{1.003750pt}%
\definecolor{currentstroke}{rgb}{1.000000,0.000000,0.000000}%
\pgfsetstrokecolor{currentstroke}%
\pgfsetdash{}{0pt}%
\pgfpathmoveto{\pgfqpoint{14.968190in}{5.733414in}}%
\pgfpathcurveto{\pgfqpoint{14.976426in}{5.733414in}}{\pgfqpoint{14.984326in}{5.736686in}}{\pgfqpoint{14.990150in}{5.742510in}}%
\pgfpathcurveto{\pgfqpoint{14.995974in}{5.748334in}}{\pgfqpoint{14.999246in}{5.756234in}}{\pgfqpoint{14.999246in}{5.764470in}}%
\pgfpathcurveto{\pgfqpoint{14.999246in}{5.772706in}}{\pgfqpoint{14.995974in}{5.780606in}}{\pgfqpoint{14.990150in}{5.786430in}}%
\pgfpathcurveto{\pgfqpoint{14.984326in}{5.792254in}}{\pgfqpoint{14.976426in}{5.795527in}}{\pgfqpoint{14.968190in}{5.795527in}}%
\pgfpathcurveto{\pgfqpoint{14.959954in}{5.795527in}}{\pgfqpoint{14.952053in}{5.792254in}}{\pgfqpoint{14.946230in}{5.786430in}}%
\pgfpathcurveto{\pgfqpoint{14.940406in}{5.780606in}}{\pgfqpoint{14.937133in}{5.772706in}}{\pgfqpoint{14.937133in}{5.764470in}}%
\pgfpathcurveto{\pgfqpoint{14.937133in}{5.756234in}}{\pgfqpoint{14.940406in}{5.748334in}}{\pgfqpoint{14.946230in}{5.742510in}}%
\pgfpathcurveto{\pgfqpoint{14.952053in}{5.736686in}}{\pgfqpoint{14.959954in}{5.733414in}}{\pgfqpoint{14.968190in}{5.733414in}}%
\pgfusepath{stroke}%
\end{pgfscope}%
\begin{pgfscope}%
\pgfpathrectangle{\pgfqpoint{3.788192in}{2.980138in}}{\pgfqpoint{2.914000in}{2.171400in}}%
\pgfusepath{clip}%
\pgfsetbuttcap%
\pgfsetroundjoin%
\pgfsetlinewidth{1.003750pt}%
\definecolor{currentstroke}{rgb}{1.000000,0.000000,0.000000}%
\pgfsetstrokecolor{currentstroke}%
\pgfsetdash{}{0pt}%
\pgfpathmoveto{\pgfqpoint{12.713917in}{7.026179in}}%
\pgfpathcurveto{\pgfqpoint{12.722154in}{7.026179in}}{\pgfqpoint{12.730054in}{7.029451in}}{\pgfqpoint{12.735878in}{7.035275in}}%
\pgfpathcurveto{\pgfqpoint{12.741701in}{7.041099in}}{\pgfqpoint{12.744974in}{7.048999in}}{\pgfqpoint{12.744974in}{7.057235in}}%
\pgfpathcurveto{\pgfqpoint{12.744974in}{7.065472in}}{\pgfqpoint{12.741701in}{7.073372in}}{\pgfqpoint{12.735878in}{7.079196in}}%
\pgfpathcurveto{\pgfqpoint{12.730054in}{7.085019in}}{\pgfqpoint{12.722154in}{7.088292in}}{\pgfqpoint{12.713917in}{7.088292in}}%
\pgfpathcurveto{\pgfqpoint{12.705681in}{7.088292in}}{\pgfqpoint{12.697781in}{7.085019in}}{\pgfqpoint{12.691957in}{7.079196in}}%
\pgfpathcurveto{\pgfqpoint{12.686133in}{7.073372in}}{\pgfqpoint{12.682861in}{7.065472in}}{\pgfqpoint{12.682861in}{7.057235in}}%
\pgfpathcurveto{\pgfqpoint{12.682861in}{7.048999in}}{\pgfqpoint{12.686133in}{7.041099in}}{\pgfqpoint{12.691957in}{7.035275in}}%
\pgfpathcurveto{\pgfqpoint{12.697781in}{7.029451in}}{\pgfqpoint{12.705681in}{7.026179in}}{\pgfqpoint{12.713917in}{7.026179in}}%
\pgfusepath{stroke}%
\end{pgfscope}%
\begin{pgfscope}%
\pgfpathrectangle{\pgfqpoint{3.788192in}{2.980138in}}{\pgfqpoint{2.914000in}{2.171400in}}%
\pgfusepath{clip}%
\pgfsetbuttcap%
\pgfsetroundjoin%
\pgfsetlinewidth{1.003750pt}%
\definecolor{currentstroke}{rgb}{1.000000,0.000000,0.000000}%
\pgfsetstrokecolor{currentstroke}%
\pgfsetdash{}{0pt}%
\pgfpathmoveto{\pgfqpoint{13.558784in}{8.054051in}}%
\pgfpathcurveto{\pgfqpoint{13.567020in}{8.054051in}}{\pgfqpoint{13.574920in}{8.057323in}}{\pgfqpoint{13.580744in}{8.063147in}}%
\pgfpathcurveto{\pgfqpoint{13.586568in}{8.068971in}}{\pgfqpoint{13.589840in}{8.076871in}}{\pgfqpoint{13.589840in}{8.085108in}}%
\pgfpathcurveto{\pgfqpoint{13.589840in}{8.093344in}}{\pgfqpoint{13.586568in}{8.101244in}}{\pgfqpoint{13.580744in}{8.107068in}}%
\pgfpathcurveto{\pgfqpoint{13.574920in}{8.112892in}}{\pgfqpoint{13.567020in}{8.116164in}}{\pgfqpoint{13.558784in}{8.116164in}}%
\pgfpathcurveto{\pgfqpoint{13.550548in}{8.116164in}}{\pgfqpoint{13.542648in}{8.112892in}}{\pgfqpoint{13.536824in}{8.107068in}}%
\pgfpathcurveto{\pgfqpoint{13.531000in}{8.101244in}}{\pgfqpoint{13.527727in}{8.093344in}}{\pgfqpoint{13.527727in}{8.085108in}}%
\pgfpathcurveto{\pgfqpoint{13.527727in}{8.076871in}}{\pgfqpoint{13.531000in}{8.068971in}}{\pgfqpoint{13.536824in}{8.063147in}}%
\pgfpathcurveto{\pgfqpoint{13.542648in}{8.057323in}}{\pgfqpoint{13.550548in}{8.054051in}}{\pgfqpoint{13.558784in}{8.054051in}}%
\pgfusepath{stroke}%
\end{pgfscope}%
\begin{pgfscope}%
\pgfpathrectangle{\pgfqpoint{3.788192in}{2.980138in}}{\pgfqpoint{2.914000in}{2.171400in}}%
\pgfusepath{clip}%
\pgfsetbuttcap%
\pgfsetroundjoin%
\pgfsetlinewidth{1.003750pt}%
\definecolor{currentstroke}{rgb}{1.000000,0.000000,0.000000}%
\pgfsetstrokecolor{currentstroke}%
\pgfsetdash{}{0pt}%
\pgfpathmoveto{\pgfqpoint{13.021061in}{8.119261in}}%
\pgfpathcurveto{\pgfqpoint{13.029297in}{8.119261in}}{\pgfqpoint{13.037197in}{8.122533in}}{\pgfqpoint{13.043021in}{8.128357in}}%
\pgfpathcurveto{\pgfqpoint{13.048845in}{8.134181in}}{\pgfqpoint{13.052117in}{8.142081in}}{\pgfqpoint{13.052117in}{8.150317in}}%
\pgfpathcurveto{\pgfqpoint{13.052117in}{8.158554in}}{\pgfqpoint{13.048845in}{8.166454in}}{\pgfqpoint{13.043021in}{8.172278in}}%
\pgfpathcurveto{\pgfqpoint{13.037197in}{8.178102in}}{\pgfqpoint{13.029297in}{8.181374in}}{\pgfqpoint{13.021061in}{8.181374in}}%
\pgfpathcurveto{\pgfqpoint{13.012825in}{8.181374in}}{\pgfqpoint{13.004925in}{8.178102in}}{\pgfqpoint{12.999101in}{8.172278in}}%
\pgfpathcurveto{\pgfqpoint{12.993277in}{8.166454in}}{\pgfqpoint{12.990004in}{8.158554in}}{\pgfqpoint{12.990004in}{8.150317in}}%
\pgfpathcurveto{\pgfqpoint{12.990004in}{8.142081in}}{\pgfqpoint{12.993277in}{8.134181in}}{\pgfqpoint{12.999101in}{8.128357in}}%
\pgfpathcurveto{\pgfqpoint{13.004925in}{8.122533in}}{\pgfqpoint{13.012825in}{8.119261in}}{\pgfqpoint{13.021061in}{8.119261in}}%
\pgfusepath{stroke}%
\end{pgfscope}%
\begin{pgfscope}%
\pgfpathrectangle{\pgfqpoint{3.788192in}{2.980138in}}{\pgfqpoint{2.914000in}{2.171400in}}%
\pgfusepath{clip}%
\pgfsetbuttcap%
\pgfsetroundjoin%
\pgfsetlinewidth{1.003750pt}%
\definecolor{currentstroke}{rgb}{1.000000,0.000000,0.000000}%
\pgfsetstrokecolor{currentstroke}%
\pgfsetdash{}{0pt}%
\pgfpathmoveto{\pgfqpoint{12.649080in}{7.078889in}}%
\pgfpathcurveto{\pgfqpoint{12.657316in}{7.078889in}}{\pgfqpoint{12.665216in}{7.082161in}}{\pgfqpoint{12.671040in}{7.087985in}}%
\pgfpathcurveto{\pgfqpoint{12.676864in}{7.093809in}}{\pgfqpoint{12.680136in}{7.101709in}}{\pgfqpoint{12.680136in}{7.109946in}}%
\pgfpathcurveto{\pgfqpoint{12.680136in}{7.118182in}}{\pgfqpoint{12.676864in}{7.126082in}}{\pgfqpoint{12.671040in}{7.131906in}}%
\pgfpathcurveto{\pgfqpoint{12.665216in}{7.137730in}}{\pgfqpoint{12.657316in}{7.141002in}}{\pgfqpoint{12.649080in}{7.141002in}}%
\pgfpathcurveto{\pgfqpoint{12.640843in}{7.141002in}}{\pgfqpoint{12.632943in}{7.137730in}}{\pgfqpoint{12.627119in}{7.131906in}}%
\pgfpathcurveto{\pgfqpoint{12.621295in}{7.126082in}}{\pgfqpoint{12.618023in}{7.118182in}}{\pgfqpoint{12.618023in}{7.109946in}}%
\pgfpathcurveto{\pgfqpoint{12.618023in}{7.101709in}}{\pgfqpoint{12.621295in}{7.093809in}}{\pgfqpoint{12.627119in}{7.087985in}}%
\pgfpathcurveto{\pgfqpoint{12.632943in}{7.082161in}}{\pgfqpoint{12.640843in}{7.078889in}}{\pgfqpoint{12.649080in}{7.078889in}}%
\pgfusepath{stroke}%
\end{pgfscope}%
\begin{pgfscope}%
\pgfpathrectangle{\pgfqpoint{3.788192in}{2.980138in}}{\pgfqpoint{2.914000in}{2.171400in}}%
\pgfusepath{clip}%
\pgfsetbuttcap%
\pgfsetroundjoin%
\pgfsetlinewidth{1.003750pt}%
\definecolor{currentstroke}{rgb}{1.000000,0.000000,0.000000}%
\pgfsetstrokecolor{currentstroke}%
\pgfsetdash{}{0pt}%
\pgfpathmoveto{\pgfqpoint{4.530215in}{4.056056in}}%
\pgfpathcurveto{\pgfqpoint{4.538452in}{4.056056in}}{\pgfqpoint{4.546352in}{4.059328in}}{\pgfqpoint{4.552176in}{4.065152in}}%
\pgfpathcurveto{\pgfqpoint{4.557999in}{4.070976in}}{\pgfqpoint{4.561272in}{4.078876in}}{\pgfqpoint{4.561272in}{4.087112in}}%
\pgfpathcurveto{\pgfqpoint{4.561272in}{4.095349in}}{\pgfqpoint{4.557999in}{4.103249in}}{\pgfqpoint{4.552176in}{4.109073in}}%
\pgfpathcurveto{\pgfqpoint{4.546352in}{4.114897in}}{\pgfqpoint{4.538452in}{4.118169in}}{\pgfqpoint{4.530215in}{4.118169in}}%
\pgfpathcurveto{\pgfqpoint{4.521979in}{4.118169in}}{\pgfqpoint{4.514079in}{4.114897in}}{\pgfqpoint{4.508255in}{4.109073in}}%
\pgfpathcurveto{\pgfqpoint{4.502431in}{4.103249in}}{\pgfqpoint{4.499159in}{4.095349in}}{\pgfqpoint{4.499159in}{4.087112in}}%
\pgfpathcurveto{\pgfqpoint{4.499159in}{4.078876in}}{\pgfqpoint{4.502431in}{4.070976in}}{\pgfqpoint{4.508255in}{4.065152in}}%
\pgfpathcurveto{\pgfqpoint{4.514079in}{4.059328in}}{\pgfqpoint{4.521979in}{4.056056in}}{\pgfqpoint{4.530215in}{4.056056in}}%
\pgfpathlineto{\pgfqpoint{4.530215in}{4.056056in}}%
\pgfpathclose%
\pgfusepath{stroke}%
\end{pgfscope}%
\begin{pgfscope}%
\pgfpathrectangle{\pgfqpoint{3.788192in}{2.980138in}}{\pgfqpoint{2.914000in}{2.171400in}}%
\pgfusepath{clip}%
\pgfsetbuttcap%
\pgfsetmiterjoin%
\definecolor{currentfill}{rgb}{0.839216,0.152941,0.156863}%
\pgfsetfillcolor{currentfill}%
\pgfsetfillopacity{0.200000}%
\pgfsetlinewidth{1.003750pt}%
\definecolor{currentstroke}{rgb}{0.839216,0.152941,0.156863}%
\pgfsetstrokecolor{currentstroke}%
\pgfsetstrokeopacity{0.200000}%
\pgfsetdash{}{0pt}%
\pgfpathmoveto{\pgfqpoint{4.530215in}{2.980138in}}%
\pgfpathlineto{\pgfqpoint{24.162152in}{2.980138in}}%
\pgfpathlineto{\pgfqpoint{24.162152in}{5.151538in}}%
\pgfpathlineto{\pgfqpoint{4.530215in}{5.151538in}}%
\pgfpathlineto{\pgfqpoint{4.530215in}{2.980138in}}%
\pgfpathclose%
\pgfusepath{stroke,fill}%
\end{pgfscope}%
\begin{pgfscope}%
\pgfsetbuttcap%
\pgfsetmiterjoin%
\definecolor{currentfill}{rgb}{0.839216,0.152941,0.156863}%
\pgfsetfillcolor{currentfill}%
\pgfsetfillopacity{0.200000}%
\pgfsetlinewidth{1.003750pt}%
\definecolor{currentstroke}{rgb}{0.839216,0.152941,0.156863}%
\pgfsetstrokecolor{currentstroke}%
\pgfsetstrokeopacity{0.200000}%
\pgfsetdash{}{0pt}%
\pgfpathrectangle{\pgfqpoint{3.788192in}{2.980138in}}{\pgfqpoint{2.914000in}{2.171400in}}%
\pgfusepath{clip}%
\pgfpathmoveto{\pgfqpoint{4.530215in}{2.980138in}}%
\pgfpathlineto{\pgfqpoint{24.162152in}{2.980138in}}%
\pgfpathlineto{\pgfqpoint{24.162152in}{5.151538in}}%
\pgfpathlineto{\pgfqpoint{4.530215in}{5.151538in}}%
\pgfpathlineto{\pgfqpoint{4.530215in}{2.980138in}}%
\pgfpathclose%
\pgfusepath{clip}%
\pgfsys@defobject{currentpattern}{\pgfqpoint{0in}{0in}}{\pgfqpoint{1in}{1in}}{%
\begin{pgfscope}%
\pgfpathrectangle{\pgfqpoint{0in}{0in}}{\pgfqpoint{1in}{1in}}%
\pgfusepath{clip}%
\pgfpathmoveto{\pgfqpoint{-0.500000in}{0.500000in}}%
\pgfpathlineto{\pgfqpoint{0.500000in}{1.500000in}}%
\pgfpathmoveto{\pgfqpoint{-0.333333in}{0.333333in}}%
\pgfpathlineto{\pgfqpoint{0.666667in}{1.333333in}}%
\pgfpathmoveto{\pgfqpoint{-0.166667in}{0.166667in}}%
\pgfpathlineto{\pgfqpoint{0.833333in}{1.166667in}}%
\pgfpathmoveto{\pgfqpoint{0.000000in}{0.000000in}}%
\pgfpathlineto{\pgfqpoint{1.000000in}{1.000000in}}%
\pgfpathmoveto{\pgfqpoint{0.166667in}{-0.166667in}}%
\pgfpathlineto{\pgfqpoint{1.166667in}{0.833333in}}%
\pgfpathmoveto{\pgfqpoint{0.333333in}{-0.333333in}}%
\pgfpathlineto{\pgfqpoint{1.333333in}{0.666667in}}%
\pgfpathmoveto{\pgfqpoint{0.500000in}{-0.500000in}}%
\pgfpathlineto{\pgfqpoint{1.500000in}{0.500000in}}%
\pgfusepath{stroke}%
\end{pgfscope}%
}%
\pgfsys@transformshift{4.530215in}{2.980138in}%
\pgfsys@useobject{currentpattern}{}%
\pgfsys@transformshift{1in}{0in}%
\pgfsys@useobject{currentpattern}{}%
\pgfsys@transformshift{1in}{0in}%
\pgfsys@useobject{currentpattern}{}%
\pgfsys@transformshift{1in}{0in}%
\pgfsys@useobject{currentpattern}{}%
\pgfsys@transformshift{1in}{0in}%
\pgfsys@useobject{currentpattern}{}%
\pgfsys@transformshift{1in}{0in}%
\pgfsys@useobject{currentpattern}{}%
\pgfsys@transformshift{1in}{0in}%
\pgfsys@useobject{currentpattern}{}%
\pgfsys@transformshift{1in}{0in}%
\pgfsys@useobject{currentpattern}{}%
\pgfsys@transformshift{1in}{0in}%
\pgfsys@useobject{currentpattern}{}%
\pgfsys@transformshift{1in}{0in}%
\pgfsys@useobject{currentpattern}{}%
\pgfsys@transformshift{1in}{0in}%
\pgfsys@useobject{currentpattern}{}%
\pgfsys@transformshift{1in}{0in}%
\pgfsys@useobject{currentpattern}{}%
\pgfsys@transformshift{1in}{0in}%
\pgfsys@useobject{currentpattern}{}%
\pgfsys@transformshift{1in}{0in}%
\pgfsys@useobject{currentpattern}{}%
\pgfsys@transformshift{1in}{0in}%
\pgfsys@useobject{currentpattern}{}%
\pgfsys@transformshift{1in}{0in}%
\pgfsys@useobject{currentpattern}{}%
\pgfsys@transformshift{1in}{0in}%
\pgfsys@useobject{currentpattern}{}%
\pgfsys@transformshift{1in}{0in}%
\pgfsys@useobject{currentpattern}{}%
\pgfsys@transformshift{1in}{0in}%
\pgfsys@useobject{currentpattern}{}%
\pgfsys@transformshift{1in}{0in}%
\pgfsys@useobject{currentpattern}{}%
\pgfsys@transformshift{1in}{0in}%
\pgfsys@transformshift{-20in}{0in}%
\pgfsys@transformshift{0in}{1in}%
\pgfsys@useobject{currentpattern}{}%
\pgfsys@transformshift{1in}{0in}%
\pgfsys@useobject{currentpattern}{}%
\pgfsys@transformshift{1in}{0in}%
\pgfsys@useobject{currentpattern}{}%
\pgfsys@transformshift{1in}{0in}%
\pgfsys@useobject{currentpattern}{}%
\pgfsys@transformshift{1in}{0in}%
\pgfsys@useobject{currentpattern}{}%
\pgfsys@transformshift{1in}{0in}%
\pgfsys@useobject{currentpattern}{}%
\pgfsys@transformshift{1in}{0in}%
\pgfsys@useobject{currentpattern}{}%
\pgfsys@transformshift{1in}{0in}%
\pgfsys@useobject{currentpattern}{}%
\pgfsys@transformshift{1in}{0in}%
\pgfsys@useobject{currentpattern}{}%
\pgfsys@transformshift{1in}{0in}%
\pgfsys@useobject{currentpattern}{}%
\pgfsys@transformshift{1in}{0in}%
\pgfsys@useobject{currentpattern}{}%
\pgfsys@transformshift{1in}{0in}%
\pgfsys@useobject{currentpattern}{}%
\pgfsys@transformshift{1in}{0in}%
\pgfsys@useobject{currentpattern}{}%
\pgfsys@transformshift{1in}{0in}%
\pgfsys@useobject{currentpattern}{}%
\pgfsys@transformshift{1in}{0in}%
\pgfsys@useobject{currentpattern}{}%
\pgfsys@transformshift{1in}{0in}%
\pgfsys@useobject{currentpattern}{}%
\pgfsys@transformshift{1in}{0in}%
\pgfsys@useobject{currentpattern}{}%
\pgfsys@transformshift{1in}{0in}%
\pgfsys@useobject{currentpattern}{}%
\pgfsys@transformshift{1in}{0in}%
\pgfsys@useobject{currentpattern}{}%
\pgfsys@transformshift{1in}{0in}%
\pgfsys@useobject{currentpattern}{}%
\pgfsys@transformshift{1in}{0in}%
\pgfsys@transformshift{-20in}{0in}%
\pgfsys@transformshift{0in}{1in}%
\pgfsys@useobject{currentpattern}{}%
\pgfsys@transformshift{1in}{0in}%
\pgfsys@useobject{currentpattern}{}%
\pgfsys@transformshift{1in}{0in}%
\pgfsys@useobject{currentpattern}{}%
\pgfsys@transformshift{1in}{0in}%
\pgfsys@useobject{currentpattern}{}%
\pgfsys@transformshift{1in}{0in}%
\pgfsys@useobject{currentpattern}{}%
\pgfsys@transformshift{1in}{0in}%
\pgfsys@useobject{currentpattern}{}%
\pgfsys@transformshift{1in}{0in}%
\pgfsys@useobject{currentpattern}{}%
\pgfsys@transformshift{1in}{0in}%
\pgfsys@useobject{currentpattern}{}%
\pgfsys@transformshift{1in}{0in}%
\pgfsys@useobject{currentpattern}{}%
\pgfsys@transformshift{1in}{0in}%
\pgfsys@useobject{currentpattern}{}%
\pgfsys@transformshift{1in}{0in}%
\pgfsys@useobject{currentpattern}{}%
\pgfsys@transformshift{1in}{0in}%
\pgfsys@useobject{currentpattern}{}%
\pgfsys@transformshift{1in}{0in}%
\pgfsys@useobject{currentpattern}{}%
\pgfsys@transformshift{1in}{0in}%
\pgfsys@useobject{currentpattern}{}%
\pgfsys@transformshift{1in}{0in}%
\pgfsys@useobject{currentpattern}{}%
\pgfsys@transformshift{1in}{0in}%
\pgfsys@useobject{currentpattern}{}%
\pgfsys@transformshift{1in}{0in}%
\pgfsys@useobject{currentpattern}{}%
\pgfsys@transformshift{1in}{0in}%
\pgfsys@useobject{currentpattern}{}%
\pgfsys@transformshift{1in}{0in}%
\pgfsys@useobject{currentpattern}{}%
\pgfsys@transformshift{1in}{0in}%
\pgfsys@useobject{currentpattern}{}%
\pgfsys@transformshift{1in}{0in}%
\pgfsys@transformshift{-20in}{0in}%
\pgfsys@transformshift{0in}{1in}%
\end{pgfscope}%
\begin{pgfscope}%
\pgfpathrectangle{\pgfqpoint{3.788192in}{2.980138in}}{\pgfqpoint{2.914000in}{2.171400in}}%
\pgfusepath{clip}%
\pgfsetrectcap%
\pgfsetroundjoin%
\pgfsetlinewidth{0.803000pt}%
\definecolor{currentstroke}{rgb}{0.690196,0.690196,0.690196}%
\pgfsetstrokecolor{currentstroke}%
\pgfsetdash{}{0pt}%
\pgfpathmoveto{\pgfqpoint{4.105109in}{2.980138in}}%
\pgfpathlineto{\pgfqpoint{4.105109in}{5.151538in}}%
\pgfusepath{stroke}%
\end{pgfscope}%
\begin{pgfscope}%
\pgfsetbuttcap%
\pgfsetroundjoin%
\definecolor{currentfill}{rgb}{0.000000,0.000000,0.000000}%
\pgfsetfillcolor{currentfill}%
\pgfsetlinewidth{0.803000pt}%
\definecolor{currentstroke}{rgb}{0.000000,0.000000,0.000000}%
\pgfsetstrokecolor{currentstroke}%
\pgfsetdash{}{0pt}%
\pgfsys@defobject{currentmarker}{\pgfqpoint{0.000000in}{-0.048611in}}{\pgfqpoint{0.000000in}{0.000000in}}{%
\pgfpathmoveto{\pgfqpoint{0.000000in}{0.000000in}}%
\pgfpathlineto{\pgfqpoint{0.000000in}{-0.048611in}}%
\pgfusepath{stroke,fill}%
}%
\begin{pgfscope}%
\pgfsys@transformshift{4.105109in}{2.980138in}%
\pgfsys@useobject{currentmarker}{}%
\end{pgfscope}%
\end{pgfscope}%
\begin{pgfscope}%
\definecolor{textcolor}{rgb}{0.000000,0.000000,0.000000}%
\pgfsetstrokecolor{textcolor}%
\pgfsetfillcolor{textcolor}%
\pgftext[x=4.105109in,y=2.882916in,,top]{\color{textcolor}{\rmfamily\fontsize{14.000000}{16.800000}\selectfont\catcode`\^=\active\def^{\ifmmode\sp\else\^{}\fi}\catcode`\%=\active\def%{\%}$\mathdefault{5280}$}}%
\end{pgfscope}%
\begin{pgfscope}%
\pgfpathrectangle{\pgfqpoint{3.788192in}{2.980138in}}{\pgfqpoint{2.914000in}{2.171400in}}%
\pgfusepath{clip}%
\pgfsetrectcap%
\pgfsetroundjoin%
\pgfsetlinewidth{0.803000pt}%
\definecolor{currentstroke}{rgb}{0.690196,0.690196,0.690196}%
\pgfsetstrokecolor{currentstroke}%
\pgfsetdash{}{0pt}%
\pgfpathmoveto{\pgfqpoint{4.847132in}{2.980138in}}%
\pgfpathlineto{\pgfqpoint{4.847132in}{5.151538in}}%
\pgfusepath{stroke}%
\end{pgfscope}%
\begin{pgfscope}%
\pgfsetbuttcap%
\pgfsetroundjoin%
\definecolor{currentfill}{rgb}{0.000000,0.000000,0.000000}%
\pgfsetfillcolor{currentfill}%
\pgfsetlinewidth{0.803000pt}%
\definecolor{currentstroke}{rgb}{0.000000,0.000000,0.000000}%
\pgfsetstrokecolor{currentstroke}%
\pgfsetdash{}{0pt}%
\pgfsys@defobject{currentmarker}{\pgfqpoint{0.000000in}{-0.048611in}}{\pgfqpoint{0.000000in}{0.000000in}}{%
\pgfpathmoveto{\pgfqpoint{0.000000in}{0.000000in}}%
\pgfpathlineto{\pgfqpoint{0.000000in}{-0.048611in}}%
\pgfusepath{stroke,fill}%
}%
\begin{pgfscope}%
\pgfsys@transformshift{4.847132in}{2.980138in}%
\pgfsys@useobject{currentmarker}{}%
\end{pgfscope}%
\end{pgfscope}%
\begin{pgfscope}%
\definecolor{textcolor}{rgb}{0.000000,0.000000,0.000000}%
\pgfsetstrokecolor{textcolor}%
\pgfsetfillcolor{textcolor}%
\pgftext[x=4.847132in,y=2.882916in,,top]{\color{textcolor}{\rmfamily\fontsize{14.000000}{16.800000}\selectfont\catcode`\^=\active\def^{\ifmmode\sp\else\^{}\fi}\catcode`\%=\active\def%{\%}$\mathdefault{5300}$}}%
\end{pgfscope}%
\begin{pgfscope}%
\pgfpathrectangle{\pgfqpoint{3.788192in}{2.980138in}}{\pgfqpoint{2.914000in}{2.171400in}}%
\pgfusepath{clip}%
\pgfsetrectcap%
\pgfsetroundjoin%
\pgfsetlinewidth{0.803000pt}%
\definecolor{currentstroke}{rgb}{0.690196,0.690196,0.690196}%
\pgfsetstrokecolor{currentstroke}%
\pgfsetdash{}{0pt}%
\pgfpathmoveto{\pgfqpoint{5.589156in}{2.980138in}}%
\pgfpathlineto{\pgfqpoint{5.589156in}{5.151538in}}%
\pgfusepath{stroke}%
\end{pgfscope}%
\begin{pgfscope}%
\pgfsetbuttcap%
\pgfsetroundjoin%
\definecolor{currentfill}{rgb}{0.000000,0.000000,0.000000}%
\pgfsetfillcolor{currentfill}%
\pgfsetlinewidth{0.803000pt}%
\definecolor{currentstroke}{rgb}{0.000000,0.000000,0.000000}%
\pgfsetstrokecolor{currentstroke}%
\pgfsetdash{}{0pt}%
\pgfsys@defobject{currentmarker}{\pgfqpoint{0.000000in}{-0.048611in}}{\pgfqpoint{0.000000in}{0.000000in}}{%
\pgfpathmoveto{\pgfqpoint{0.000000in}{0.000000in}}%
\pgfpathlineto{\pgfqpoint{0.000000in}{-0.048611in}}%
\pgfusepath{stroke,fill}%
}%
\begin{pgfscope}%
\pgfsys@transformshift{5.589156in}{2.980138in}%
\pgfsys@useobject{currentmarker}{}%
\end{pgfscope}%
\end{pgfscope}%
\begin{pgfscope}%
\definecolor{textcolor}{rgb}{0.000000,0.000000,0.000000}%
\pgfsetstrokecolor{textcolor}%
\pgfsetfillcolor{textcolor}%
\pgftext[x=5.589156in,y=2.882916in,,top]{\color{textcolor}{\rmfamily\fontsize{14.000000}{16.800000}\selectfont\catcode`\^=\active\def^{\ifmmode\sp\else\^{}\fi}\catcode`\%=\active\def%{\%}$\mathdefault{5320}$}}%
\end{pgfscope}%
\begin{pgfscope}%
\pgfpathrectangle{\pgfqpoint{3.788192in}{2.980138in}}{\pgfqpoint{2.914000in}{2.171400in}}%
\pgfusepath{clip}%
\pgfsetrectcap%
\pgfsetroundjoin%
\pgfsetlinewidth{0.803000pt}%
\definecolor{currentstroke}{rgb}{0.690196,0.690196,0.690196}%
\pgfsetstrokecolor{currentstroke}%
\pgfsetdash{}{0pt}%
\pgfpathmoveto{\pgfqpoint{6.331180in}{2.980138in}}%
\pgfpathlineto{\pgfqpoint{6.331180in}{5.151538in}}%
\pgfusepath{stroke}%
\end{pgfscope}%
\begin{pgfscope}%
\pgfsetbuttcap%
\pgfsetroundjoin%
\definecolor{currentfill}{rgb}{0.000000,0.000000,0.000000}%
\pgfsetfillcolor{currentfill}%
\pgfsetlinewidth{0.803000pt}%
\definecolor{currentstroke}{rgb}{0.000000,0.000000,0.000000}%
\pgfsetstrokecolor{currentstroke}%
\pgfsetdash{}{0pt}%
\pgfsys@defobject{currentmarker}{\pgfqpoint{0.000000in}{-0.048611in}}{\pgfqpoint{0.000000in}{0.000000in}}{%
\pgfpathmoveto{\pgfqpoint{0.000000in}{0.000000in}}%
\pgfpathlineto{\pgfqpoint{0.000000in}{-0.048611in}}%
\pgfusepath{stroke,fill}%
}%
\begin{pgfscope}%
\pgfsys@transformshift{6.331180in}{2.980138in}%
\pgfsys@useobject{currentmarker}{}%
\end{pgfscope}%
\end{pgfscope}%
\begin{pgfscope}%
\definecolor{textcolor}{rgb}{0.000000,0.000000,0.000000}%
\pgfsetstrokecolor{textcolor}%
\pgfsetfillcolor{textcolor}%
\pgftext[x=6.331180in,y=2.882916in,,top]{\color{textcolor}{\rmfamily\fontsize{14.000000}{16.800000}\selectfont\catcode`\^=\active\def^{\ifmmode\sp\else\^{}\fi}\catcode`\%=\active\def%{\%}$\mathdefault{5340}$}}%
\end{pgfscope}%
\begin{pgfscope}%
\pgfpathrectangle{\pgfqpoint{3.788192in}{2.980138in}}{\pgfqpoint{2.914000in}{2.171400in}}%
\pgfusepath{clip}%
\pgfsetrectcap%
\pgfsetroundjoin%
\pgfsetlinewidth{0.803000pt}%
\definecolor{currentstroke}{rgb}{0.690196,0.690196,0.690196}%
\pgfsetstrokecolor{currentstroke}%
\pgfsetdash{}{0pt}%
\pgfpathmoveto{\pgfqpoint{3.788192in}{3.334947in}}%
\pgfpathlineto{\pgfqpoint{6.702192in}{3.334947in}}%
\pgfusepath{stroke}%
\end{pgfscope}%
\begin{pgfscope}%
\pgfsetbuttcap%
\pgfsetroundjoin%
\definecolor{currentfill}{rgb}{0.000000,0.000000,0.000000}%
\pgfsetfillcolor{currentfill}%
\pgfsetlinewidth{0.803000pt}%
\definecolor{currentstroke}{rgb}{0.000000,0.000000,0.000000}%
\pgfsetstrokecolor{currentstroke}%
\pgfsetdash{}{0pt}%
\pgfsys@defobject{currentmarker}{\pgfqpoint{-0.048611in}{0.000000in}}{\pgfqpoint{-0.000000in}{0.000000in}}{%
\pgfpathmoveto{\pgfqpoint{-0.000000in}{0.000000in}}%
\pgfpathlineto{\pgfqpoint{-0.048611in}{0.000000in}}%
\pgfusepath{stroke,fill}%
}%
\begin{pgfscope}%
\pgfsys@transformshift{3.788192in}{3.334947in}%
\pgfsys@useobject{currentmarker}{}%
\end{pgfscope}%
\end{pgfscope}%
\begin{pgfscope}%
\definecolor{textcolor}{rgb}{0.000000,0.000000,0.000000}%
\pgfsetstrokecolor{textcolor}%
\pgfsetfillcolor{textcolor}%
\pgftext[x=3.495138in, y=3.265502in, left, base]{\color{textcolor}{\rmfamily\fontsize{14.000000}{16.800000}\selectfont\catcode`\^=\active\def^{\ifmmode\sp\else\^{}\fi}\catcode`\%=\active\def%{\%}$\mathdefault{10}$}}%
\end{pgfscope}%
\begin{pgfscope}%
\pgfpathrectangle{\pgfqpoint{3.788192in}{2.980138in}}{\pgfqpoint{2.914000in}{2.171400in}}%
\pgfusepath{clip}%
\pgfsetrectcap%
\pgfsetroundjoin%
\pgfsetlinewidth{0.803000pt}%
\definecolor{currentstroke}{rgb}{0.690196,0.690196,0.690196}%
\pgfsetstrokecolor{currentstroke}%
\pgfsetdash{}{0pt}%
\pgfpathmoveto{\pgfqpoint{3.788192in}{4.044564in}}%
\pgfpathlineto{\pgfqpoint{6.702192in}{4.044564in}}%
\pgfusepath{stroke}%
\end{pgfscope}%
\begin{pgfscope}%
\pgfsetbuttcap%
\pgfsetroundjoin%
\definecolor{currentfill}{rgb}{0.000000,0.000000,0.000000}%
\pgfsetfillcolor{currentfill}%
\pgfsetlinewidth{0.803000pt}%
\definecolor{currentstroke}{rgb}{0.000000,0.000000,0.000000}%
\pgfsetstrokecolor{currentstroke}%
\pgfsetdash{}{0pt}%
\pgfsys@defobject{currentmarker}{\pgfqpoint{-0.048611in}{0.000000in}}{\pgfqpoint{-0.000000in}{0.000000in}}{%
\pgfpathmoveto{\pgfqpoint{-0.000000in}{0.000000in}}%
\pgfpathlineto{\pgfqpoint{-0.048611in}{0.000000in}}%
\pgfusepath{stroke,fill}%
}%
\begin{pgfscope}%
\pgfsys@transformshift{3.788192in}{4.044564in}%
\pgfsys@useobject{currentmarker}{}%
\end{pgfscope}%
\end{pgfscope}%
\begin{pgfscope}%
\definecolor{textcolor}{rgb}{0.000000,0.000000,0.000000}%
\pgfsetstrokecolor{textcolor}%
\pgfsetfillcolor{textcolor}%
\pgftext[x=3.495138in, y=3.975119in, left, base]{\color{textcolor}{\rmfamily\fontsize{14.000000}{16.800000}\selectfont\catcode`\^=\active\def^{\ifmmode\sp\else\^{}\fi}\catcode`\%=\active\def%{\%}$\mathdefault{12}$}}%
\end{pgfscope}%
\begin{pgfscope}%
\pgfpathrectangle{\pgfqpoint{3.788192in}{2.980138in}}{\pgfqpoint{2.914000in}{2.171400in}}%
\pgfusepath{clip}%
\pgfsetrectcap%
\pgfsetroundjoin%
\pgfsetlinewidth{0.803000pt}%
\definecolor{currentstroke}{rgb}{0.690196,0.690196,0.690196}%
\pgfsetstrokecolor{currentstroke}%
\pgfsetdash{}{0pt}%
\pgfpathmoveto{\pgfqpoint{3.788192in}{4.754181in}}%
\pgfpathlineto{\pgfqpoint{6.702192in}{4.754181in}}%
\pgfusepath{stroke}%
\end{pgfscope}%
\begin{pgfscope}%
\pgfsetbuttcap%
\pgfsetroundjoin%
\definecolor{currentfill}{rgb}{0.000000,0.000000,0.000000}%
\pgfsetfillcolor{currentfill}%
\pgfsetlinewidth{0.803000pt}%
\definecolor{currentstroke}{rgb}{0.000000,0.000000,0.000000}%
\pgfsetstrokecolor{currentstroke}%
\pgfsetdash{}{0pt}%
\pgfsys@defobject{currentmarker}{\pgfqpoint{-0.048611in}{0.000000in}}{\pgfqpoint{-0.000000in}{0.000000in}}{%
\pgfpathmoveto{\pgfqpoint{-0.000000in}{0.000000in}}%
\pgfpathlineto{\pgfqpoint{-0.048611in}{0.000000in}}%
\pgfusepath{stroke,fill}%
}%
\begin{pgfscope}%
\pgfsys@transformshift{3.788192in}{4.754181in}%
\pgfsys@useobject{currentmarker}{}%
\end{pgfscope}%
\end{pgfscope}%
\begin{pgfscope}%
\definecolor{textcolor}{rgb}{0.000000,0.000000,0.000000}%
\pgfsetstrokecolor{textcolor}%
\pgfsetfillcolor{textcolor}%
\pgftext[x=3.495138in, y=4.684736in, left, base]{\color{textcolor}{\rmfamily\fontsize{14.000000}{16.800000}\selectfont\catcode`\^=\active\def^{\ifmmode\sp\else\^{}\fi}\catcode`\%=\active\def%{\%}$\mathdefault{14}$}}%
\end{pgfscope}%
\begin{pgfscope}%
\pgfpathrectangle{\pgfqpoint{3.788192in}{2.980138in}}{\pgfqpoint{2.914000in}{2.171400in}}%
\pgfusepath{clip}%
\pgfsetrectcap%
\pgfsetroundjoin%
\pgfsetlinewidth{1.505625pt}%
\definecolor{currentstroke}{rgb}{0.000000,0.000000,1.000000}%
\pgfsetstrokecolor{currentstroke}%
\pgfsetdash{}{0pt}%
\pgfpathmoveto{\pgfqpoint{5.478622in}{4.808529in}}%
\pgfpathlineto{\pgfqpoint{5.701947in}{2.977638in}}%
\pgfusepath{stroke}%
\end{pgfscope}%
\begin{pgfscope}%
\pgfpathrectangle{\pgfqpoint{3.788192in}{2.980138in}}{\pgfqpoint{2.914000in}{2.171400in}}%
\pgfusepath{clip}%
\pgfsetbuttcap%
\pgfsetroundjoin%
\definecolor{currentfill}{rgb}{0.000000,0.000000,1.000000}%
\pgfsetfillcolor{currentfill}%
\pgfsetlinewidth{1.003750pt}%
\definecolor{currentstroke}{rgb}{0.000000,0.000000,1.000000}%
\pgfsetstrokecolor{currentstroke}%
\pgfsetdash{}{0pt}%
\pgfsys@defobject{currentmarker}{\pgfqpoint{-0.041667in}{-0.041667in}}{\pgfqpoint{0.041667in}{0.041667in}}{%
\pgfpathmoveto{\pgfqpoint{0.000000in}{-0.041667in}}%
\pgfpathcurveto{\pgfqpoint{0.011050in}{-0.041667in}}{\pgfqpoint{0.021649in}{-0.037276in}}{\pgfqpoint{0.029463in}{-0.029463in}}%
\pgfpathcurveto{\pgfqpoint{0.037276in}{-0.021649in}}{\pgfqpoint{0.041667in}{-0.011050in}}{\pgfqpoint{0.041667in}{0.000000in}}%
\pgfpathcurveto{\pgfqpoint{0.041667in}{0.011050in}}{\pgfqpoint{0.037276in}{0.021649in}}{\pgfqpoint{0.029463in}{0.029463in}}%
\pgfpathcurveto{\pgfqpoint{0.021649in}{0.037276in}}{\pgfqpoint{0.011050in}{0.041667in}}{\pgfqpoint{0.000000in}{0.041667in}}%
\pgfpathcurveto{\pgfqpoint{-0.011050in}{0.041667in}}{\pgfqpoint{-0.021649in}{0.037276in}}{\pgfqpoint{-0.029463in}{0.029463in}}%
\pgfpathcurveto{\pgfqpoint{-0.037276in}{0.021649in}}{\pgfqpoint{-0.041667in}{0.011050in}}{\pgfqpoint{-0.041667in}{0.000000in}}%
\pgfpathcurveto{\pgfqpoint{-0.041667in}{-0.011050in}}{\pgfqpoint{-0.037276in}{-0.021649in}}{\pgfqpoint{-0.029463in}{-0.029463in}}%
\pgfpathcurveto{\pgfqpoint{-0.021649in}{-0.037276in}}{\pgfqpoint{-0.011050in}{-0.041667in}}{\pgfqpoint{0.000000in}{-0.041667in}}%
\pgfpathlineto{\pgfqpoint{0.000000in}{-0.041667in}}%
\pgfpathclose%
\pgfusepath{stroke,fill}%
}%
\begin{pgfscope}%
\pgfsys@transformshift{5.478622in}{4.808529in}%
\pgfsys@useobject{currentmarker}{}%
\end{pgfscope}%
\begin{pgfscope}%
\pgfsys@transformshift{5.786444in}{2.284897in}%
\pgfsys@useobject{currentmarker}{}%
\end{pgfscope}%
\begin{pgfscope}%
\pgfsys@transformshift{5.955602in}{1.807070in}%
\pgfsys@useobject{currentmarker}{}%
\end{pgfscope}%
\begin{pgfscope}%
\pgfsys@transformshift{6.072257in}{1.530809in}%
\pgfsys@useobject{currentmarker}{}%
\end{pgfscope}%
\begin{pgfscope}%
\pgfsys@transformshift{6.134561in}{1.493263in}%
\pgfsys@useobject{currentmarker}{}%
\end{pgfscope}%
\begin{pgfscope}%
\pgfsys@transformshift{6.204334in}{1.296852in}%
\pgfsys@useobject{currentmarker}{}%
\end{pgfscope}%
\begin{pgfscope}%
\pgfsys@transformshift{6.281957in}{1.238734in}%
\pgfsys@useobject{currentmarker}{}%
\end{pgfscope}%
\begin{pgfscope}%
\pgfsys@transformshift{6.389240in}{1.234341in}%
\pgfsys@useobject{currentmarker}{}%
\end{pgfscope}%
\begin{pgfscope}%
\pgfsys@transformshift{6.427201in}{1.108377in}%
\pgfsys@useobject{currentmarker}{}%
\end{pgfscope}%
\begin{pgfscope}%
\pgfsys@transformshift{6.519285in}{1.077759in}%
\pgfsys@useobject{currentmarker}{}%
\end{pgfscope}%
\begin{pgfscope}%
\pgfsys@transformshift{6.689124in}{1.072073in}%
\pgfsys@useobject{currentmarker}{}%
\end{pgfscope}%
\begin{pgfscope}%
\pgfsys@transformshift{6.747190in}{1.024269in}%
\pgfsys@useobject{currentmarker}{}%
\end{pgfscope}%
\begin{pgfscope}%
\pgfsys@transformshift{6.806750in}{1.015907in}%
\pgfsys@useobject{currentmarker}{}%
\end{pgfscope}%
\begin{pgfscope}%
\pgfsys@transformshift{6.931730in}{0.991431in}%
\pgfsys@useobject{currentmarker}{}%
\end{pgfscope}%
\begin{pgfscope}%
\pgfsys@transformshift{7.035734in}{0.970288in}%
\pgfsys@useobject{currentmarker}{}%
\end{pgfscope}%
\begin{pgfscope}%
\pgfsys@transformshift{7.042663in}{0.960624in}%
\pgfsys@useobject{currentmarker}{}%
\end{pgfscope}%
\begin{pgfscope}%
\pgfsys@transformshift{7.110896in}{0.945926in}%
\pgfsys@useobject{currentmarker}{}%
\end{pgfscope}%
\begin{pgfscope}%
\pgfsys@transformshift{7.385541in}{0.918874in}%
\pgfsys@useobject{currentmarker}{}%
\end{pgfscope}%
\begin{pgfscope}%
\pgfsys@transformshift{7.677622in}{0.881182in}%
\pgfsys@useobject{currentmarker}{}%
\end{pgfscope}%
\begin{pgfscope}%
\pgfsys@transformshift{8.074033in}{0.854291in}%
\pgfsys@useobject{currentmarker}{}%
\end{pgfscope}%
\begin{pgfscope}%
\pgfsys@transformshift{8.437037in}{0.845259in}%
\pgfsys@useobject{currentmarker}{}%
\end{pgfscope}%
\begin{pgfscope}%
\pgfsys@transformshift{8.451833in}{0.833333in}%
\pgfsys@useobject{currentmarker}{}%
\end{pgfscope}%
\begin{pgfscope}%
\pgfsys@transformshift{8.473270in}{0.812531in}%
\pgfsys@useobject{currentmarker}{}%
\end{pgfscope}%
\begin{pgfscope}%
\pgfsys@transformshift{8.530759in}{0.805118in}%
\pgfsys@useobject{currentmarker}{}%
\end{pgfscope}%
\begin{pgfscope}%
\pgfsys@transformshift{8.667946in}{0.801766in}%
\pgfsys@useobject{currentmarker}{}%
\end{pgfscope}%
\begin{pgfscope}%
\pgfsys@transformshift{8.761129in}{0.792896in}%
\pgfsys@useobject{currentmarker}{}%
\end{pgfscope}%
\begin{pgfscope}%
\pgfsys@transformshift{9.175032in}{0.779118in}%
\pgfsys@useobject{currentmarker}{}%
\end{pgfscope}%
\begin{pgfscope}%
\pgfsys@transformshift{9.229242in}{0.771806in}%
\pgfsys@useobject{currentmarker}{}%
\end{pgfscope}%
\begin{pgfscope}%
\pgfsys@transformshift{9.230836in}{0.763945in}%
\pgfsys@useobject{currentmarker}{}%
\end{pgfscope}%
\begin{pgfscope}%
\pgfsys@transformshift{9.319803in}{0.762647in}%
\pgfsys@useobject{currentmarker}{}%
\end{pgfscope}%
\begin{pgfscope}%
\pgfsys@transformshift{9.530028in}{0.749444in}%
\pgfsys@useobject{currentmarker}{}%
\end{pgfscope}%
\begin{pgfscope}%
\pgfsys@transformshift{9.629502in}{0.748101in}%
\pgfsys@useobject{currentmarker}{}%
\end{pgfscope}%
\begin{pgfscope}%
\pgfsys@transformshift{9.644888in}{0.743176in}%
\pgfsys@useobject{currentmarker}{}%
\end{pgfscope}%
\begin{pgfscope}%
\pgfsys@transformshift{10.473184in}{0.730520in}%
\pgfsys@useobject{currentmarker}{}%
\end{pgfscope}%
\begin{pgfscope}%
\pgfsys@transformshift{10.538033in}{0.705808in}%
\pgfsys@useobject{currentmarker}{}%
\end{pgfscope}%
\begin{pgfscope}%
\pgfsys@transformshift{10.574876in}{0.703903in}%
\pgfsys@useobject{currentmarker}{}%
\end{pgfscope}%
\begin{pgfscope}%
\pgfsys@transformshift{10.706030in}{0.700593in}%
\pgfsys@useobject{currentmarker}{}%
\end{pgfscope}%
\begin{pgfscope}%
\pgfsys@transformshift{10.965602in}{0.687998in}%
\pgfsys@useobject{currentmarker}{}%
\end{pgfscope}%
\begin{pgfscope}%
\pgfsys@transformshift{11.060169in}{0.684789in}%
\pgfsys@useobject{currentmarker}{}%
\end{pgfscope}%
\begin{pgfscope}%
\pgfsys@transformshift{11.147519in}{0.680781in}%
\pgfsys@useobject{currentmarker}{}%
\end{pgfscope}%
\begin{pgfscope}%
\pgfsys@transformshift{11.370645in}{0.672484in}%
\pgfsys@useobject{currentmarker}{}%
\end{pgfscope}%
\begin{pgfscope}%
\pgfsys@transformshift{11.728159in}{0.668514in}%
\pgfsys@useobject{currentmarker}{}%
\end{pgfscope}%
\begin{pgfscope}%
\pgfsys@transformshift{12.137942in}{0.666709in}%
\pgfsys@useobject{currentmarker}{}%
\end{pgfscope}%
\begin{pgfscope}%
\pgfsys@transformshift{12.434130in}{0.666468in}%
\pgfsys@useobject{currentmarker}{}%
\end{pgfscope}%
\begin{pgfscope}%
\pgfsys@transformshift{12.804602in}{0.666397in}%
\pgfsys@useobject{currentmarker}{}%
\end{pgfscope}%
\begin{pgfscope}%
\pgfsys@transformshift{14.562739in}{0.661643in}%
\pgfsys@useobject{currentmarker}{}%
\end{pgfscope}%
\begin{pgfscope}%
\pgfsys@transformshift{15.028873in}{0.661022in}%
\pgfsys@useobject{currentmarker}{}%
\end{pgfscope}%
\begin{pgfscope}%
\pgfsys@transformshift{15.695083in}{0.659204in}%
\pgfsys@useobject{currentmarker}{}%
\end{pgfscope}%
\begin{pgfscope}%
\pgfsys@transformshift{16.663977in}{0.658612in}%
\pgfsys@useobject{currentmarker}{}%
\end{pgfscope}%
\begin{pgfscope}%
\pgfsys@transformshift{17.188373in}{0.655818in}%
\pgfsys@useobject{currentmarker}{}%
\end{pgfscope}%
\begin{pgfscope}%
\pgfsys@transformshift{18.420605in}{0.654898in}%
\pgfsys@useobject{currentmarker}{}%
\end{pgfscope}%
\begin{pgfscope}%
\pgfsys@transformshift{20.009331in}{0.652189in}%
\pgfsys@useobject{currentmarker}{}%
\end{pgfscope}%
\begin{pgfscope}%
\pgfsys@transformshift{21.260300in}{0.647607in}%
\pgfsys@useobject{currentmarker}{}%
\end{pgfscope}%
\begin{pgfscope}%
\pgfsys@transformshift{23.228701in}{0.645006in}%
\pgfsys@useobject{currentmarker}{}%
\end{pgfscope}%
\begin{pgfscope}%
\pgfsys@transformshift{25.888770in}{0.639824in}%
\pgfsys@useobject{currentmarker}{}%
\end{pgfscope}%
\begin{pgfscope}%
\pgfsys@transformshift{28.569097in}{0.635141in}%
\pgfsys@useobject{currentmarker}{}%
\end{pgfscope}%
\begin{pgfscope}%
\pgfsys@transformshift{32.504050in}{0.628662in}%
\pgfsys@useobject{currentmarker}{}%
\end{pgfscope}%
\begin{pgfscope}%
\pgfsys@transformshift{39.060135in}{0.620415in}%
\pgfsys@useobject{currentmarker}{}%
\end{pgfscope}%
\begin{pgfscope}%
\pgfsys@transformshift{50.416853in}{0.608419in}%
\pgfsys@useobject{currentmarker}{}%
\end{pgfscope}%
\begin{pgfscope}%
\pgfsys@transformshift{87.100182in}{0.583248in}%
\pgfsys@useobject{currentmarker}{}%
\end{pgfscope}%
\end{pgfscope}%
\begin{pgfscope}%
\pgfpathrectangle{\pgfqpoint{3.788192in}{2.980138in}}{\pgfqpoint{2.914000in}{2.171400in}}%
\pgfusepath{clip}%
\pgfsetrectcap%
\pgfsetroundjoin%
\pgfsetlinewidth{1.505625pt}%
\definecolor{currentstroke}{rgb}{0.121569,0.466667,0.705882}%
\pgfsetstrokecolor{currentstroke}%
\pgfsetstrokeopacity{0.500000}%
\pgfsetdash{}{0pt}%
\pgfusepath{stroke}%
\end{pgfscope}%
\begin{pgfscope}%
\pgfsetrectcap%
\pgfsetmiterjoin%
\pgfsetlinewidth{0.803000pt}%
\definecolor{currentstroke}{rgb}{0.000000,0.000000,0.000000}%
\pgfsetstrokecolor{currentstroke}%
\pgfsetdash{}{0pt}%
\pgfpathmoveto{\pgfqpoint{3.788192in}{2.980138in}}%
\pgfpathlineto{\pgfqpoint{3.788192in}{5.151538in}}%
\pgfusepath{stroke}%
\end{pgfscope}%
\begin{pgfscope}%
\pgfsetrectcap%
\pgfsetmiterjoin%
\pgfsetlinewidth{0.803000pt}%
\definecolor{currentstroke}{rgb}{0.000000,0.000000,0.000000}%
\pgfsetstrokecolor{currentstroke}%
\pgfsetdash{}{0pt}%
\pgfpathmoveto{\pgfqpoint{6.702192in}{2.980138in}}%
\pgfpathlineto{\pgfqpoint{6.702192in}{5.151538in}}%
\pgfusepath{stroke}%
\end{pgfscope}%
\begin{pgfscope}%
\pgfsetrectcap%
\pgfsetmiterjoin%
\pgfsetlinewidth{0.803000pt}%
\definecolor{currentstroke}{rgb}{0.000000,0.000000,0.000000}%
\pgfsetstrokecolor{currentstroke}%
\pgfsetdash{}{0pt}%
\pgfpathmoveto{\pgfqpoint{3.788192in}{2.980138in}}%
\pgfpathlineto{\pgfqpoint{6.702192in}{2.980138in}}%
\pgfusepath{stroke}%
\end{pgfscope}%
\begin{pgfscope}%
\pgfsetrectcap%
\pgfsetmiterjoin%
\pgfsetlinewidth{0.803000pt}%
\definecolor{currentstroke}{rgb}{0.000000,0.000000,0.000000}%
\pgfsetstrokecolor{currentstroke}%
\pgfsetdash{}{0pt}%
\pgfpathmoveto{\pgfqpoint{3.788192in}{5.151538in}}%
\pgfpathlineto{\pgfqpoint{6.702192in}{5.151538in}}%
\pgfusepath{stroke}%
\end{pgfscope}%
\begin{pgfscope}%
\pgfsetbuttcap%
\pgfsetmiterjoin%
\definecolor{currentfill}{rgb}{1.000000,1.000000,1.000000}%
\pgfsetfillcolor{currentfill}%
\pgfsetfillopacity{0.800000}%
\pgfsetlinewidth{1.003750pt}%
\definecolor{currentstroke}{rgb}{0.800000,0.800000,0.800000}%
\pgfsetstrokecolor{currentstroke}%
\pgfsetstrokeopacity{0.800000}%
\pgfsetdash{}{0pt}%
\pgfpathmoveto{\pgfqpoint{3.327765in}{0.781249in}}%
\pgfpathlineto{\pgfqpoint{6.732636in}{0.781249in}}%
\pgfpathquadraticcurveto{\pgfqpoint{6.777080in}{0.781249in}}{\pgfqpoint{6.777080in}{0.825694in}}%
\pgfpathlineto{\pgfqpoint{6.777080in}{2.153779in}}%
\pgfpathquadraticcurveto{\pgfqpoint{6.777080in}{2.198223in}}{\pgfqpoint{6.732636in}{2.198223in}}%
\pgfpathlineto{\pgfqpoint{3.327765in}{2.198223in}}%
\pgfpathquadraticcurveto{\pgfqpoint{3.283320in}{2.198223in}}{\pgfqpoint{3.283320in}{2.153779in}}%
\pgfpathlineto{\pgfqpoint{3.283320in}{0.825694in}}%
\pgfpathquadraticcurveto{\pgfqpoint{3.283320in}{0.781249in}}{\pgfqpoint{3.327765in}{0.781249in}}%
\pgfpathlineto{\pgfqpoint{3.327765in}{0.781249in}}%
\pgfpathclose%
\pgfusepath{stroke,fill}%
\end{pgfscope}%
\begin{pgfscope}%
\pgfsetrectcap%
\pgfsetroundjoin%
\pgfsetlinewidth{1.505625pt}%
\definecolor{currentstroke}{rgb}{0.000000,0.000000,1.000000}%
\pgfsetstrokecolor{currentstroke}%
\pgfsetdash{}{0pt}%
\pgfpathmoveto{\pgfqpoint{3.372209in}{2.020446in}}%
\pgfpathlineto{\pgfqpoint{3.594431in}{2.020446in}}%
\pgfpathlineto{\pgfqpoint{3.816654in}{2.020446in}}%
\pgfusepath{stroke}%
\end{pgfscope}%
\begin{pgfscope}%
\pgfsetbuttcap%
\pgfsetroundjoin%
\definecolor{currentfill}{rgb}{0.000000,0.000000,1.000000}%
\pgfsetfillcolor{currentfill}%
\pgfsetlinewidth{1.003750pt}%
\definecolor{currentstroke}{rgb}{0.000000,0.000000,1.000000}%
\pgfsetstrokecolor{currentstroke}%
\pgfsetdash{}{0pt}%
\pgfsys@defobject{currentmarker}{\pgfqpoint{-0.006944in}{-0.006944in}}{\pgfqpoint{0.006944in}{0.006944in}}{%
\pgfpathmoveto{\pgfqpoint{0.000000in}{-0.006944in}}%
\pgfpathcurveto{\pgfqpoint{0.001842in}{-0.006944in}}{\pgfqpoint{0.003608in}{-0.006213in}}{\pgfqpoint{0.004910in}{-0.004910in}}%
\pgfpathcurveto{\pgfqpoint{0.006213in}{-0.003608in}}{\pgfqpoint{0.006944in}{-0.001842in}}{\pgfqpoint{0.006944in}{0.000000in}}%
\pgfpathcurveto{\pgfqpoint{0.006944in}{0.001842in}}{\pgfqpoint{0.006213in}{0.003608in}}{\pgfqpoint{0.004910in}{0.004910in}}%
\pgfpathcurveto{\pgfqpoint{0.003608in}{0.006213in}}{\pgfqpoint{0.001842in}{0.006944in}}{\pgfqpoint{0.000000in}{0.006944in}}%
\pgfpathcurveto{\pgfqpoint{-0.001842in}{0.006944in}}{\pgfqpoint{-0.003608in}{0.006213in}}{\pgfqpoint{-0.004910in}{0.004910in}}%
\pgfpathcurveto{\pgfqpoint{-0.006213in}{0.003608in}}{\pgfqpoint{-0.006944in}{0.001842in}}{\pgfqpoint{-0.006944in}{0.000000in}}%
\pgfpathcurveto{\pgfqpoint{-0.006944in}{-0.001842in}}{\pgfqpoint{-0.006213in}{-0.003608in}}{\pgfqpoint{-0.004910in}{-0.004910in}}%
\pgfpathcurveto{\pgfqpoint{-0.003608in}{-0.006213in}}{\pgfqpoint{-0.001842in}{-0.006944in}}{\pgfqpoint{0.000000in}{-0.006944in}}%
\pgfpathlineto{\pgfqpoint{0.000000in}{-0.006944in}}%
\pgfpathclose%
\pgfusepath{stroke,fill}%
}%
\begin{pgfscope}%
\pgfsys@transformshift{3.594431in}{2.020446in}%
\pgfsys@useobject{currentmarker}{}%
\end{pgfscope}%
\end{pgfscope}%
\begin{pgfscope}%
\definecolor{textcolor}{rgb}{0.000000,0.000000,0.000000}%
\pgfsetstrokecolor{textcolor}%
\pgfsetfillcolor{textcolor}%
\pgftext[x=3.994431in,y=1.942668in,left,base]{\color{textcolor}{\rmfamily\fontsize{16.000000}{19.200000}\selectfont\catcode`\^=\active\def^{\ifmmode\sp\else\^{}\fi}\catcode`\%=\active\def%{\%}osier}}%
\end{pgfscope}%
\begin{pgfscope}%
\pgfsetrectcap%
\pgfsetroundjoin%
\pgfsetlinewidth{1.505625pt}%
\definecolor{currentstroke}{rgb}{0.121569,0.466667,0.705882}%
\pgfsetstrokecolor{currentstroke}%
\pgfsetstrokeopacity{0.500000}%
\pgfsetdash{}{0pt}%
\pgfpathmoveto{\pgfqpoint{3.372209in}{1.682869in}}%
\pgfpathlineto{\pgfqpoint{3.594431in}{1.682869in}}%
\pgfpathlineto{\pgfqpoint{3.816654in}{1.682869in}}%
\pgfusepath{stroke}%
\end{pgfscope}%
\begin{pgfscope}%
\definecolor{textcolor}{rgb}{0.000000,0.000000,0.000000}%
\pgfsetstrokecolor{textcolor}%
\pgfsetfillcolor{textcolor}%
\pgftext[x=3.994431in,y=1.605091in,left,base]{\color{textcolor}{\rmfamily\fontsize{16.000000}{19.200000}\selectfont\catcode`\^=\active\def^{\ifmmode\sp\else\^{}\fi}\catcode`\%=\active\def%{\%}near-optimal space (osier)}}%
\end{pgfscope}%
\begin{pgfscope}%
\pgfsetbuttcap%
\pgfsetroundjoin%
\pgfsetlinewidth{1.003750pt}%
\definecolor{currentstroke}{rgb}{1.000000,0.000000,0.000000}%
\pgfsetstrokecolor{currentstroke}%
\pgfsetdash{}{0pt}%
\pgfpathmoveto{\pgfqpoint{3.594431in}{1.294791in}}%
\pgfpathcurveto{\pgfqpoint{3.602668in}{1.294791in}}{\pgfqpoint{3.610568in}{1.298063in}}{\pgfqpoint{3.616392in}{1.303887in}}%
\pgfpathcurveto{\pgfqpoint{3.622216in}{1.309711in}}{\pgfqpoint{3.625488in}{1.317611in}}{\pgfqpoint{3.625488in}{1.325847in}}%
\pgfpathcurveto{\pgfqpoint{3.625488in}{1.334084in}}{\pgfqpoint{3.622216in}{1.341984in}}{\pgfqpoint{3.616392in}{1.347808in}}%
\pgfpathcurveto{\pgfqpoint{3.610568in}{1.353632in}}{\pgfqpoint{3.602668in}{1.356904in}}{\pgfqpoint{3.594431in}{1.356904in}}%
\pgfpathcurveto{\pgfqpoint{3.586195in}{1.356904in}}{\pgfqpoint{3.578295in}{1.353632in}}{\pgfqpoint{3.572471in}{1.347808in}}%
\pgfpathcurveto{\pgfqpoint{3.566647in}{1.341984in}}{\pgfqpoint{3.563375in}{1.334084in}}{\pgfqpoint{3.563375in}{1.325847in}}%
\pgfpathcurveto{\pgfqpoint{3.563375in}{1.317611in}}{\pgfqpoint{3.566647in}{1.309711in}}{\pgfqpoint{3.572471in}{1.303887in}}%
\pgfpathcurveto{\pgfqpoint{3.578295in}{1.298063in}}{\pgfqpoint{3.586195in}{1.294791in}}{\pgfqpoint{3.594431in}{1.294791in}}%
\pgfpathlineto{\pgfqpoint{3.594431in}{1.294791in}}%
\pgfpathclose%
\pgfusepath{stroke}%
\end{pgfscope}%
\begin{pgfscope}%
\definecolor{textcolor}{rgb}{0.000000,0.000000,0.000000}%
\pgfsetstrokecolor{textcolor}%
\pgfsetfillcolor{textcolor}%
\pgftext[x=3.994431in,y=1.267514in,left,base]{\color{textcolor}{\rmfamily\fontsize{16.000000}{19.200000}\selectfont\catcode`\^=\active\def^{\ifmmode\sp\else\^{}\fi}\catcode`\%=\active\def%{\%}temoa+mga}}%
\end{pgfscope}%
\begin{pgfscope}%
\pgfsetbuttcap%
\pgfsetmiterjoin%
\definecolor{currentfill}{rgb}{0.839216,0.152941,0.156863}%
\pgfsetfillcolor{currentfill}%
\pgfsetfillopacity{0.200000}%
\pgfsetlinewidth{1.003750pt}%
\definecolor{currentstroke}{rgb}{0.839216,0.152941,0.156863}%
\pgfsetstrokecolor{currentstroke}%
\pgfsetstrokeopacity{0.200000}%
\pgfsetdash{}{0pt}%
\pgfpathmoveto{\pgfqpoint{3.372209in}{0.929937in}}%
\pgfpathlineto{\pgfqpoint{3.816654in}{0.929937in}}%
\pgfpathlineto{\pgfqpoint{3.816654in}{1.085493in}}%
\pgfpathlineto{\pgfqpoint{3.372209in}{1.085493in}}%
\pgfpathlineto{\pgfqpoint{3.372209in}{0.929937in}}%
\pgfpathclose%
\pgfusepath{stroke,fill}%
\end{pgfscope}%
\begin{pgfscope}%
\pgfsetbuttcap%
\pgfsetmiterjoin%
\definecolor{currentfill}{rgb}{0.839216,0.152941,0.156863}%
\pgfsetfillcolor{currentfill}%
\pgfsetfillopacity{0.200000}%
\pgfsetlinewidth{1.003750pt}%
\definecolor{currentstroke}{rgb}{0.839216,0.152941,0.156863}%
\pgfsetstrokecolor{currentstroke}%
\pgfsetstrokeopacity{0.200000}%
\pgfsetdash{}{0pt}%
\pgfpathmoveto{\pgfqpoint{3.372209in}{0.929937in}}%
\pgfpathlineto{\pgfqpoint{3.816654in}{0.929937in}}%
\pgfpathlineto{\pgfqpoint{3.816654in}{1.085493in}}%
\pgfpathlineto{\pgfqpoint{3.372209in}{1.085493in}}%
\pgfpathlineto{\pgfqpoint{3.372209in}{0.929937in}}%
\pgfpathclose%
\pgfusepath{clip}%
\pgfsys@defobject{currentpattern}{\pgfqpoint{0in}{0in}}{\pgfqpoint{1in}{1in}}{%
\begin{pgfscope}%
\pgfpathrectangle{\pgfqpoint{0in}{0in}}{\pgfqpoint{1in}{1in}}%
\pgfusepath{clip}%
\pgfpathmoveto{\pgfqpoint{-0.500000in}{0.500000in}}%
\pgfpathlineto{\pgfqpoint{0.500000in}{1.500000in}}%
\pgfpathmoveto{\pgfqpoint{-0.333333in}{0.333333in}}%
\pgfpathlineto{\pgfqpoint{0.666667in}{1.333333in}}%
\pgfpathmoveto{\pgfqpoint{-0.166667in}{0.166667in}}%
\pgfpathlineto{\pgfqpoint{0.833333in}{1.166667in}}%
\pgfpathmoveto{\pgfqpoint{0.000000in}{0.000000in}}%
\pgfpathlineto{\pgfqpoint{1.000000in}{1.000000in}}%
\pgfpathmoveto{\pgfqpoint{0.166667in}{-0.166667in}}%
\pgfpathlineto{\pgfqpoint{1.166667in}{0.833333in}}%
\pgfpathmoveto{\pgfqpoint{0.333333in}{-0.333333in}}%
\pgfpathlineto{\pgfqpoint{1.333333in}{0.666667in}}%
\pgfpathmoveto{\pgfqpoint{0.500000in}{-0.500000in}}%
\pgfpathlineto{\pgfqpoint{1.500000in}{0.500000in}}%
\pgfusepath{stroke}%
\end{pgfscope}%
}%
\pgfsys@transformshift{3.372209in}{0.929937in}%
\pgfsys@useobject{currentpattern}{}%
\pgfsys@transformshift{1in}{0in}%
\pgfsys@transformshift{-1in}{0in}%
\pgfsys@transformshift{0in}{1in}%
\end{pgfscope}%
\begin{pgfscope}%
\definecolor{textcolor}{rgb}{0.000000,0.000000,0.000000}%
\pgfsetstrokecolor{textcolor}%
\pgfsetfillcolor{textcolor}%
\pgftext[x=3.994431in,y=0.929937in,left,base]{\color{textcolor}{\rmfamily\fontsize{16.000000}{19.200000}\selectfont\catcode`\^=\active\def^{\ifmmode\sp\else\^{}\fi}\catcode`\%=\active\def%{\%}near-optimal space (Temoa)}}%
\end{pgfscope}%
\end{pgfpicture}%
\makeatother%
\endgroup%
}
  \caption{Compares the least-cost solutions between \acs{temoa}
  and \acs{osier} as well as their sub-optimal spaces. The least-cost solutions
  for \ac{osier} and \ac{temoa} are within 0.5\% of each other.}
  \label{fig:temoa-benchmark-01}
\end{figure}

First, \ac{temoa}'s least-cost solution is slightly better (within 0.5\%) than
\ac{osier}'s in terms of both cost and emissions. This happens because
\ac{temoa} optimizes energy dispatch slightly differently than \ac{osier}. In
particular, the initial storage value for energy storage technologies is a
decision variable in \ac{temoa} and not in \ac{osier}. A second reason for this
discrepancy has to do with convergence. \ac{osier}'s Pareto-front could likely
be improved with a lower convergence tolerance, but this would use additional
computational resources. Although, \ac{temoa} calculated an optimal solution
with slightly lower cost than \ac{osier}, modelers should not place too much
importance on this fact because \acp{esom} should be used to generate insight
rather than answers, due to the nature of the systems being modeled
\cite{decarolis_using_2011}.

Next, the sub-optimal spaces mostly overlap, indicating that \ac{temoa} could
find a solution with lower carbon emissions after sufficient iterations.
However, none of \ac{temoa}'s \ac{mga} solutions fall within \ac{osier}'s
sub-optimal space. This point highlights the necessity for \acl{moo}. The
objective of \ac{mga} is to produce a \textit{diverse subset} of points in the
sub-optimal region. \ac{mga} may capture appealing alternatives for some
unmodeled objective in the original problem, but it cannot guarantee that those
solutions will be an improvement along any other objective axis. This is
especially apparent here, where the least-cost solution happens also to be the
lowest carbon solution, for \ac{temoa}. The relatively small area where the two
\acp{esom} do not overlap is fully explained by the difference in their
least-cost solutions.

Even though \ac{moo} reduces structural uncertainty, it will always exist, as
discussed in Section \ref{section:uncertainty}. Therefore, identifying
alternative solutions by sampling points in the inferior region is still useful.
Figure \ref{fig:temoa-benchmark-02} focuses on the near-optimal space presented
in \ref{fig:temoa-benchmark-01} and shows both the complete set of near-optimal
solutions (green) and some randomly selected points, highlighted in red.

\begin{figure}[h]
  \centering
  \resizebox{0.75\columnwidth}{!}{\input{figures/04_benchmark_chapter/osier_mga_subset_01.pgf}}
  % \includegraphics[width=0.6\columnwidth]{figures/results/osier_mga_subset_01.png}
  % \resizebox{0.6\columnwidth}{!}{\input{figures/results/osier_mga_subset_01.png}}
  \caption{Points within \ac{osier}'s sub-optimal space.}
  \label{fig:temoa-benchmark-02}
\end{figure}

Both Figure \ref{fig:temoa-benchmark-01} and Figure \ref{fig:temoa-benchmark-02}
present solutions in the objective space. However, in order to be prescriptive,
the policy solutions must be formulated according to the decision space. In
other words, described according to the mix of technologies that produced a
solution. Figure \ref{fig:temoa-benchmark-03} presents the spread of results in
the decision space for each model. Figure \ref{fig:temoa-benchmark-03}a shows
the spread of each technology present in \ac{osier}'s Pareto front. Figure
\ref{fig:temoa-benchmark-03}b shows the same, but also includes the randomly
selected points from \ac{osier}'s near-optimal space. Lastly, Figure
\ref{fig:temoa-benchmark-03}c shows the same kind of distribution for
\ac{temoa}'s \ac{mga} solutions. Presented in this way, the design space results
indicate which technologies are always or usually present. Technologies that are
absent in all cases, including the near-optimal solutions, may be safely
ignored. For \ac{osier}, these technologies include both types of coal, biomass,
and largely ignores wind energy. In \ac{temoa}'s results, there are no
technologies that are totally absent. This result is due to the imperative built
into standard \ac{mga} to identify solutions that are maximally different in
design space, whereas \ac{osier} randomly selected points in its inferior
region. This suggests one avenue for improving \ac{osier}.

\newpage
\begin{figure}[ht!]
  \centering
  \resizebox{\columnwidth}{!}{\input{figures/04_benchmark_chapter/temoa_osier_mga_comparison1x3.pgf}}
  \caption{The design spaces for a) points on the Pareto-front in Figure
  \ref{fig:temoa-benchmark-01}, b) selected points in \ac{osier}'s sub-optimal
  space, identified in Figure \ref{fig:temoa-benchmark-02}, and c) points
  generated by \ac{temoa}'s \ac{mga} algorithm shown in Figure
  \ref{fig:temoa-benchmark-01}.}
  \label{fig:temoa-benchmark-03}
\end{figure}

% Natural gas with \ac{ccs} shows up in the randomly selected points in
% \ac{osier}'s sub-optimal region. A geo-political locus for energy
% infrastructure, described in Section \ref{section:energy-system-boundaries}
% offers one possible explanation for this technology since states with
% significant natural gas resources might seek to maintain their influence by
% developing low-carbon technology that still uses natural gas.
 
\FloatBarrier

\subsection{Exercise 3: Four Simultaneous Objectives}
Chapter \ref{chapter:lit-review} showed that conventional \acp{esom} virtually
always model a single objective and that objective is uniformly cost (or a
similar aggregated economic indicator). Further, Section
\ref{section:moo-in-energy} showed that the existing literature employing
\ac{moo} never model more than three objectives simultaneously. The purpose of
this final exercise is to demonstrate that \ac{osier} can optimize many
objectives, thereby providing more context and confidence for the tool. This
exercise minimized four objectives simultaneously: total system cost, lifecycle
carbon emissions, land-use change, and percentage of total energy from
non-renewable energy sources. Renewable energy sources include solar, wind, and
biomass. Although batteries are often used in conjunction with \acp{vre}, they
are not considered ``renewable'' (nor are they a true energy ``source'' since
they store energy from other sources rather than producing their own). For
clarity, the ``percent non-renewable'' objective refers to the penetration of
non-renewable sources as a percentage of the energy produced rather than as a
percentage of the systems total installed capacity. Figure
\ref{fig:4-obj-pareto} shows the objective-space Pareto front for this
4-dimensional problem.

\begin{noteBox}
\textbf{Reading \Aclp{pcp}:} Visualizing the Pareto front for this problem
presents a challenge due to its high dimensionality. Therefore, I present the
results with a novel plot, called a \ac{pcp}. This plot is helpful for
highlighting differences among a small set of solutions with a potentially large
number of dimensions. Figure \ref{fig:4-obj-pareto} and Figure
\ref{fig:4-obj-design} are both \acp{pcp}. Although \acp{pcp} show continuous
lines, they do not show a ``trend''. That is, for a given solution, each
objective takes on a single value that is plotted on its respective vertical
axis. The lines connecting these points simply emphasize that these points
belong to the same solution. Additionally, each objective axis has its own upper
and lower bound because each objective is scaled differently. The \ac{mga}
solutions presented in Figure \ref{fig:4-obj-design-mga} using a boxplot due to
the larger number of solutions included in \ac{mga}. 
\end{noteBox}


\begin{figure}[h]
  \centering
  \resizebox{\columnwidth}{!}{\input{figures/04_benchmark_chapter/4_obj_objective_space.pgf}}
  \caption{The Pareto front for a four objective problem. Extreme values for
  each objective are colored. The gray lines represent solutions on the Pareto
  front that are not extremum.}
  \label{fig:4-obj-pareto}
\end{figure}

Each of the colored lines in Figure \ref{fig:4-obj-pareto} belongs to a solution
with an `extreme' value on the Pareto-front. For instance, the blue line labeled
``Highest Renewable'' has the lowest percentage of non-renewable energy sources
of any solution. The gray lines are simply other points along the Pareto-front.
Figure \ref{fig:4-obj-pareto} shows that minimizing land-use change and
renewable energy maximization are strongly competing objectives, since the other
three extremum are grouped together on those two axes and diametrically opposed
to the ``highest renewable'' solution. Figure \ref{fig:4-obj-design} illustrates
the design space for each extreme solution. 


\begin{figure}[h]
  \centering
  \resizebox{\columnwidth}{!}{\input{figures/04_benchmark_chapter/4_obj_design_space.pgf}}
  \caption{The design space for a four objective problem.}
  \label{fig:4-obj-design}
\end{figure}

Figure \ref{fig:4-obj-design} shows that conventional coal and advanced coal
technologies are largely uninteresting because they make up at most 7\% and 4\%
of a solution's peak demand, respectively. The ``highest renewable'' solution
achieves its goal of reaching approximately 100\% renewable energy (by
percentage of energy produced) with a significant overbuild of wind energy and
batteries, with natural gas and a small amount of coal for reliability.
Interestingly, this solution uses no solar energy, even though solar and wind
are frequently assumed to complement each other.

Figure \ref{fig:4-obj-design-mga} extends the design space results to include
the \ac{mga} solutions. This plot indicates the design preferences for a
middling solution, but hides the relationship among energy technologies. The
most popular technologies in Figure \ref{fig:4-obj-design-mga} are conventional
nuclear, battery storage, and solar panels. The least popular technologies are
wind turbines, biomass, and ``advanced'' coal plants.

\begin{figure}[h]
  \centering
  \resizebox{\columnwidth}{!}{\input{figures/04_benchmark_chapter/4-obj-mga-design-space.pgf}}
  \caption{The design space for a four objective problem including alternative solutions suggested by MGA.}
  \label{fig:4-obj-design-mga}
\end{figure}


\chapter{Examples with \acs{osier}}
\label{chapter:examples}

% \textcolor{red}{This chapter is entirely new to this thesis!}

% \section{Example 1: Updating the \ac{set}}

% \subsection{Overview of the Nuclear Fuel Cycle}

% \subsection{What is the \ac{set}?}

% \subsection{What metrics are available?}

% \subsection{Data for the simulation}

% \subsection{Results}

% \subsection{Discussion}

% \section{Example 2: Powering a Data Center}

% \subsection{Why data centers?}

% \subsection{What technology options exist?}

% \subsection{Data for the simulation}

% \subsection{Results}

% \subsection{Discussion}


\chapter{Using modeling to enhance just outcomes}
\label{chapter:communities}
% \iffalse


% This chapter addresses my proposal to ``validate'' \ac{osier} by conducting a
% case study of energy planning processes in Champaign-Urbana through interviews
% with decision makers and planners about energy justice, energy modeling, and
% energy planning. As well as introduce interviewees to \ac{osier} itself and
% solicit feedback from them about its potential usefulness and what obstacles may
% interfere with its adoption. 

% Although I originally set out to conduct a limited case study of Chambana, I
% discovered that municipalities in Illinois do not, in general, have the decision
% making authority to make specific choices about their energy supply.
% Thus, I expanded my study to the state level; interviewing people from the
% \ac{ipa} and \ac{icc}. This expanded scope provided evidence for structural
% challenges blocking municipal influence from energy planning processes. Further,
% this evinces a tension among distributive, procedural, and recognition justice.

% \section{Progress on developing tools for local use}

% \begin{itemize}
%     \item The literature review chapter indicated that most energy modeling tools
%     and much of the related literature focus on state- or national-scale energy
%     systems.
%     \item The the hyper-local section highlighted some studies that integrated
%     local perspectives with energy modeling. Those papers include 
%     \cite{mckenna_combining_2018, johannsen_municipal_2023, fleischhacker_portfolio_2019}
%     \item There have also been studies investigating the development of models for, and
%     use by, city and urban planners.
% \end{itemize}

% \section{Municipal Levers: How municipalities make choices about their energy supply}
% There are generally four ways municipalities can make choices about their energy supply.

% \subsection{\ac{mca}}
% This is where municipalities participate in electricity markets on behalf of
% their residents. Although this allows municipalities to choose an energy supply
% besides the standard portfolio provided by an electric utility, municipalities
% do not have full control over their energy supply. Instead, municipalities
% negotiate for a few portfolios through a bidding process. While they can specify
% some criteria, such as a percentage of renewable energy, the specific generation
% mix depends on the company that developed the portfolio bid. Further, residents
% are still allowed to opt-out of \ac{mca} and elect the standard portfolio
% provided by their electric utility.

% \subsection{Utility ownership}
% Some municipalities in Illinois, such as Naperville, own their own distribution
% system. This gives a municipality greater control over the design of their
% utility system and allows a municipality to make decisions about the tradeoff
% between cost and resiliency. For example, Naperville undergounded most of the
% electric distribution system and new distribution is automatically undergounded.
% However, this model does not award control of electric supply to the
% municipality, which must procure electricity through another entity such as the
% \acf{imea}.

% \subsection{Municipal-owned generation}
% In very few cases, a municipality may own some of its own generation. \ac{uiuc}
% owns a coal and natural gas plant, a solar array, chilled water storage, and
% participates in a \ac{ppa} to purchase wind power. More commonly, municipalities
% will lease land cheaply to a solar company, for example, and purchase some or all
% of the rights to that electricity through a \ac{ppa}. 

% \subsection{Municipal-owned buildings}

% \fi

\chapter{Conclusions}
