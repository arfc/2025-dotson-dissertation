\chapter{Introduction}
\chapter{Literature Review}
\chapter{\acs{osier}}
\label{chapter:osier}

This chapter introduces \acf{osier}, a novel open-source energy system modeling
framework for \acl{moo} \cite{dotson_osier_nodate}. There are currently no
\acp{esom} that enable \ac{moo} and \ac{osier} fills that gap. Figure
\ref{fig:osier_flow} illustrates the flow of data into and within \ac{osier}.

\begin{figure}[H]
    \centering
    \includegraphics[width=\columnwidth]{figures/osier_flow}
    \caption{The flow of data into and within \ac{osier}}
    \label{fig:osier_flow}
\end{figure}

Technology data, objectives, constraints, and a dispatch model are all features
within \ac{osier}, while \ac{pymoo} drives the optimization of these objectives.
The dispatch model is independently executable for inspecting specific test
cases and mapping solutions from other solvers onto \ac{osier}'s objective
space. The next section elaborates on the dispatch model's formulation.



\chapter{Benchmarks and Examples with \acs{osier}}

\chapter{Using modeling to enhance just outcomes}
\chapter{Conclusions}
