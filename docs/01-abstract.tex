Solving climate change will require our globalized society to transition from
fossil fuel infrastructure to clean energy infrastructure. This transition must
also be done equitably and justly to prevent entrenching further injustices to
marginalized and vulnerable communities. To that end, this thesis develops the
first multi-objective energy system optimization framework, \texttt{Osier}, to
enhance the democratic engagement necessary for a just transition. Rather than
making projections about the future, \texttt{Osier} accomplishes this goal by
combining user-supplied energy demand data with technology-specific data (e.g.,
emissions, land-use, cost) and delivers a set of co-optimal energy systems
(i.e., capacities for different energy generating technologies) that balance
competing priorities (e.g., emissions, cost, or land-use). Further,
\texttt{Osier} acknowledges structural uncertainty --- the existence of
unmodeled or \textit{unmodelable} objectives --- by extending the conventional
modeling-to-generate-alternatives approach into N-dimensional objective space.
This approach does not promise to model the unmodelable, rather it recognizes
that any presentation of ``optimal'' solutions will necessarily miss these
unmodelable priorities and that interesting solutions may be contained in a
models sub-optimal space.

This thesis verified \texttt{Osier}'s suitability for energy modeling problems
with several \textit{in silico} experiments. The first set of experiments
establish \texttt{Osier}'s superiority at exploring decision space over the
mature \texttt{Temoa} framework. A second set of experiments demonstrates
\texttt{Osier} on relevant problems, such as deciding among many nuclear fuel
cycle options. Finally, this thesis presents the results of thirteen expert
interviews which support \texttt{Osier}'s utility for facilitating democratic
engagement between decision makers and their constituents, thereby attending
to issues related to procedural and recognition justice.