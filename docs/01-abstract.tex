Solving climate change will require our globalized society to transition from
fossil fuel infrastructure to clean energy infrastructure. This transition must
also be done equitably and justly to prevent entrenching further injustices to
marginalized and vulnerable communities. An economy-wide transition presents a
challenge unto itself due to the spatial, temporal, and topological complexities
of energy systems. Energy system optimization models (ESOMs) are a class of
tools designed to optimize this transition used by energy planners and
decision-makers to generate insights that inform policy. However, ESOMs have a
few critical gaps. First, current ESOMs exclusively optimize on cost, yet real
world decisions are also informed by non-financial priorities, such as
sustainability and social benefits. Since these objectives cannot be neatly
captured by a cost metric, ESOMs fail to optimize for these goals. Second, ESOMs
struggle to meaningfully incorporate concepts of justice. Some studies model
distributive justice --- related to the way benefits and burdens are shared
among society's members --- but do so in an \textit{ex post} fashion. Procedural
justice and recognition justice --- aspects of justice related to the
policymaking process and context in which decisions are made --- are frequently
sidelined. Indeed, ESOMs may be misused during a policymaking process to dismiss
public input for lack of rigor. This thesis attends to these flaws in ESOM tools
by developing the first multi-objective energy system optimization framework,
\texttt{Osier}.

Rather than returning a single optimal solution, multi-objective optimization
generates a set of co-optimal solutions called a \textit{Pareto front}, where no
objective can be improved without making another objective worse. The existence
of a Pareto front evinces tradeoffs which can only be resolved through dialogue
in a participatory process. \texttt{Osier} allows users to optimize arbitrarily
many objectives and define new objectives to create a bespoke, contextualized,
model. This thesis recognizes that some structural uncertainty will persist
regardless of the number of objectives. Thus, \texttt{Osier} leverages genetic
algorithms for their ability to sample complex Pareto fronts and because their
search methods automatically sample sub-optimal space. Further, this thesis
develops a novel algorithm to calculate a subset of maximally different
solutions within the sub-optimal space to address this uncertainty related to
unmodeled objectives. By producing multiple solutions, \texttt{Osier} gives
modelers and decision-makers the tools to meaningfully engage with public
stakeholders and learn their preferences, thereby attending to issues of
procedural and recognition justice.

This thesis verified \texttt{Osier}'s suitability for energy modeling problems
with several \textit{in silico} experiments. The first set of experiments demonstrate that
\texttt{Osier} produces results that are internally consistent within its suite
of available methods. The next set of experiments compare \texttt{Osier} to a
more mature ESOM, \texttt{Temoa}, to verify that \texttt{Osier} produces results
consistent with known methods. The results for a least-cost optimization with
\texttt{Osier} and \texttt{Temoa} show strong agreement, with 0.5\% of each
other. In addition to benchmarking exercises, this thesis applies \texttt{Osier}
to two timely examples. The first uses \texttt{Osier} to reanalyze a set of
nuclear fuel cycle options through the lens of Pareto optimality. The second
shows how \texttt{Osier} optimizes a novel objective,
energy-return-on-investment, for a hypothetical data center. Finally, this
thesis validates earlier claims of \texttt{Osier}'s usefulness for energy
planning through a qualitative study of municipal and state-level energy
planners in Illinois. The results of thirteen expert interviews demonstrate
enthusiasm for a new tool that can optimize objectives beyond cost. However,
this study surfaced structural barriers to ESOM usage at the municipal level
which must be addressed before ESOMs, like \texttt{Osier}, can be adopted.
Lastly, I recommend that the state of Illinois develop a participatory process
for its energy modeling exercises.