Solving climate change will require our globalized society to transition from
fossil fuel infrastructure to clean energy infrastructure. This transition must
also be done equitably and justly to prevent entrenching further injustices to
marginalized and vulnerable communities. Current energy planning tools optimize
for a singular cost objective which challenges decision-makers' ability to
balance competing priorities such as sustainability, employment, or land use.
This thesis develops the first multi-objective energy system optimization
framework, \texttt{Osier}, to enhance the democratic engagement necessary for a
just transition. 

\texttt{Osier} stores information about different energy technologies (e.g.,
wind, solar, nuclear), including their costs, emissions, and other data. Users
provide energy demand data and define one or more goals for their energy system,
such as minimizing cost, maximizing renewable energy, or minimizing land use.
\texttt{Osier} then presents a set of co-optimal energy portfolios that
prescribe how much of each technology should be built. Further, \texttt{Osier}
recognizes that some objectives resist quantification. Rather than claiming to
model the unmodelable, \texttt{Osier} samples near-optimal solution space using
a novel algorithm developed herein. Together, these solution sets expose
tradeoffs and allows communities, planners, and decision-makers to deliberate
over priorities. Beyond planning, \texttt{Osier} serves as an accountability
tool that allows communities or non-profit organizations to evaluate the
alignment between implemented policies and stated values. By producing multiple
solutions, \texttt{Osier} gives modelers and decision-makers the tools to
meaningfully engage with public stakeholders and learn their preferences,
thereby attending to issues of procedural and recognition justice.

This thesis verified \texttt{Osier} through a set of benchmarking experiments,
demonstrating comparable results for a least-cost optimization within 0.5\% of
the mature modeling framework, \texttt{Temoa}. In addition to benchmarking
exercises, this thesis applies \texttt{Osier} to two timely examples related to
nuclear fuel cycle options and powering new data centers. Finally, this
thesis presents results from thirteen expert interviews with Illinois energy planners which
support \texttt{Osier}'s utility for enhancing democratic engagement by enabling
stakeholders to explore tradeoffs and articulate their values, though structural
barriers to energy modeling adoption at the municipal level persist.