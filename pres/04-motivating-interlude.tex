\begin{frame}
    \frametitle{Recap \#2}
    \begin{itemize}
        \item We have seen methods to narrow the options to evaluate, \gls{mga} and ``high tradeoff'' points, 
        \item If \gls{osier} can generate a set of solutions, \boldorange{how should the 
        ultimate solution be chosen}?
    \end{itemize}
\end{frame}
    
\section{Interlude: Motivating the Case Study}

\begin{frame}
    \frametitle{Limitations to Modeling Justice}
    We cannot assume that a ``high tradeoff'' point will be the collective preference because of Arrow's Theorem.
    \pause
    \begin{block}{Arrow's Impossibility Theorem}
        It is impossible to construct a utility function
        that maps individual preferences onto a global preference
        order without imposition or dictating \cite{arrow_difficulty_1950,franssen_arrows_2005}.
    \end{block}
    \pause
    \begin{block}{Consequences of Arrow's Theorem}
        \begin{enumerate}[<+->]
            \item There is no one-size-fits-all method for public engagement or
            decision-making.
            \item Aggregated metrics can never adequately capture ``just outcomes'' ---
            this would imply a utility function.
            \item Participatory processes must include a deliberative element to evaluate
            tradeoffs.
        \end{enumerate}
    \end{block}
\end{frame}

\begin{frame}
    \frametitle{Addressing the ``Human Dimension''}
    In order for \gls{osier}, or any \gls{esom}, to address the ``human dimension'' of energy systems it must
    \begin{enumerate}[<+->]
        \item enable bespoke and contextualized modeling,
        \item generate many solutions,
        \item facilitate dialogue about the tradeoffs among solutions
    \end{enumerate}

\end{frame}

\begin{frame}
    \frametitle{Case study objectives}
    
    \begin{block}{Research Questions}
        \begin{enumerate}[<+->]
            \item How, and to what extent, do decision-makers use \glspl{esom} to inform policies?
            \item How do decision-makers consider equity and justice in policy design?
            \item How do practitioners perceive \gls{osier}'s usefulness?
        \end{enumerate}
    \end{block}
    
\end{frame}
