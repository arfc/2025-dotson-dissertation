\section{Introduction}

\subsection{Background}

\begin{frame}
    \frametitle{The Challenge at Hand}
    \begin{block}{Purpose of Energy System Modeling}
        Modeling allows us to make predictions, test hypotheses, and understand
        counterintuitive behavior.\\~\\
        \textit{Models inform energy policy with prescriptive analyses
        \cite{decarolis_using_2011}.}
    \end{block}
    \begin{block}{Problem} % But, models are limited. Quantitative models are
    % especially limited in their ability to describe the ``human dimension.''
        Policies affect people --- energy systems models cannot adequately
        capture the ``human dimension'' \cite{pfenninger_energy_2014}.
    \end{block}
    \begin{block}{What is the ``human dimension?''}
        \begin{enumerate}
            \item People have preferences about their sources of energy that are
            ignored.
            \item Models cannot describe policy outcomes related to fairness or
            justice.
        \end{enumerate}
    \end{block}
\end{frame}

\begin{frame}
    \frametitle{Three tenets of justice}
    \begin{figure}
        \centering
        % \resizebox{0.7\columnwidth}{!}{
            \begin{tikzpicture}
                \begin{scope}[blend group = soft light]
                    % \fill[red!30!white]   ( 90:1.2) circle (2);
                    \fill[illiniorange]   ( 90:1.2) circle (2);
                    % \fill[green!30!white] (210:1.2) circle (2);
                    \fill[illiniorange] (210:1.2) circle (2);
                    % \fill[blue!30!white]  (330:1.2) circle (2);
                    \fill[illiniorange]  (330:1.2) circle (2);
                \end{scope}
                \node at ( 90:2)    {Recognition}; 
                \node at ( 210:2) {Distributive}; 
                \node at ( 330:2)   {Procedural}; 
                \node[font=\Large] {\textcolor{black}{Justice}};
              \end{tikzpicture}

        % }
        \caption{Three aspects of justice \cite{schlosberg_1_2007}.}
    \end{figure}
\end{frame}

\begin{frame}
    \frametitle{Climate change highlights energy system injustices}
    Our \boldorange{changing climate} and additional demand drivers from data 
    centers and \gls{ai} \boldorange{disproportionately impact} marginalized groups and require a \boldorange{just transition} from fossil 
    fuels to clean energy.
    \begin{columns}
        \column[t]{5cm}
        \begin{figure}
            \centering
            \resizebox{\columnwidth}{!}{%% Creator: Matplotlib, PGF backend
%%
%% To include the figure in your LaTeX document, write
%%   \input{<filename>.pgf}
%%
%% Make sure the required packages are loaded in your preamble
%%   \usepackage{pgf}
%%
%% Also ensure that all the required font packages are loaded; for instance,
%% the lmodern package is sometimes necessary when using math font.
%%   \usepackage{lmodern}
%%
%% Figures using additional raster images can only be included by \input if
%% they are in the same directory as the main LaTeX file. For loading figures
%% from other directories you can use the `import` package
%%   \usepackage{import}
%%
%% and then include the figures with
%%   \import{<path to file>}{<filename>.pgf}
%%
%% Matplotlib used the following preamble
%%   \def\mathdefault#1{#1}
%%   \everymath=\expandafter{\the\everymath\displaystyle}
%%   \IfFileExists{scrextend.sty}{
%%     \usepackage[fontsize=10.000000pt]{scrextend}
%%   }{
%%     \renewcommand{\normalsize}{\fontsize{10.000000}{12.000000}\selectfont}
%%     \normalsize
%%   }
%%   
%%   \makeatletter\@ifpackageloaded{underscore}{}{\usepackage[strings]{underscore}}\makeatother
%%
\begingroup%
\makeatletter%
\begin{pgfpicture}%
\pgfpathrectangle{\pgfpointorigin}{\pgfqpoint{7.880511in}{5.900000in}}%
\pgfusepath{use as bounding box, clip}%
\begin{pgfscope}%
\pgfsetbuttcap%
\pgfsetmiterjoin%
\definecolor{currentfill}{rgb}{1.000000,1.000000,1.000000}%
\pgfsetfillcolor{currentfill}%
\pgfsetlinewidth{0.000000pt}%
\definecolor{currentstroke}{rgb}{0.000000,0.000000,0.000000}%
\pgfsetstrokecolor{currentstroke}%
\pgfsetdash{}{0pt}%
\pgfpathmoveto{\pgfqpoint{0.000000in}{0.000000in}}%
\pgfpathlineto{\pgfqpoint{7.880511in}{0.000000in}}%
\pgfpathlineto{\pgfqpoint{7.880511in}{5.900000in}}%
\pgfpathlineto{\pgfqpoint{0.000000in}{5.900000in}}%
\pgfpathlineto{\pgfqpoint{0.000000in}{0.000000in}}%
\pgfpathclose%
\pgfusepath{fill}%
\end{pgfscope}%
\begin{pgfscope}%
\pgfsetbuttcap%
\pgfsetmiterjoin%
\definecolor{currentfill}{rgb}{1.000000,1.000000,1.000000}%
\pgfsetfillcolor{currentfill}%
\pgfsetlinewidth{0.000000pt}%
\definecolor{currentstroke}{rgb}{0.000000,0.000000,0.000000}%
\pgfsetstrokecolor{currentstroke}%
\pgfsetstrokeopacity{0.000000}%
\pgfsetdash{}{0pt}%
\pgfpathmoveto{\pgfqpoint{0.697913in}{0.559721in}}%
\pgfpathlineto{\pgfqpoint{7.746549in}{0.559721in}}%
\pgfpathlineto{\pgfqpoint{7.746549in}{5.550000in}}%
\pgfpathlineto{\pgfqpoint{0.697913in}{5.550000in}}%
\pgfpathlineto{\pgfqpoint{0.697913in}{0.559721in}}%
\pgfpathclose%
\pgfusepath{fill}%
\end{pgfscope}%
\begin{pgfscope}%
\pgfpathrectangle{\pgfqpoint{0.697913in}{0.559721in}}{\pgfqpoint{7.048636in}{4.990279in}}%
\pgfusepath{clip}%
\pgfsetrectcap%
\pgfsetroundjoin%
\pgfsetlinewidth{0.803000pt}%
\definecolor{currentstroke}{rgb}{0.690196,0.690196,0.690196}%
\pgfsetstrokecolor{currentstroke}%
\pgfsetdash{}{0pt}%
\pgfpathmoveto{\pgfqpoint{1.404248in}{0.559721in}}%
\pgfpathlineto{\pgfqpoint{1.404248in}{5.550000in}}%
\pgfusepath{stroke}%
\end{pgfscope}%
\begin{pgfscope}%
\pgfsetbuttcap%
\pgfsetroundjoin%
\definecolor{currentfill}{rgb}{0.000000,0.000000,0.000000}%
\pgfsetfillcolor{currentfill}%
\pgfsetlinewidth{1.254687pt}%
\definecolor{currentstroke}{rgb}{0.000000,0.000000,0.000000}%
\pgfsetstrokecolor{currentstroke}%
\pgfsetdash{}{0pt}%
\pgfsys@defobject{currentmarker}{\pgfqpoint{0.000000in}{0.000000in}}{\pgfqpoint{0.000000in}{0.111111in}}{%
\pgfpathmoveto{\pgfqpoint{0.000000in}{0.000000in}}%
\pgfpathlineto{\pgfqpoint{0.000000in}{0.111111in}}%
\pgfusepath{stroke,fill}%
}%
\begin{pgfscope}%
\pgfsys@transformshift{1.404248in}{0.559721in}%
\pgfsys@useobject{currentmarker}{}%
\end{pgfscope}%
\end{pgfscope}%
\begin{pgfscope}%
\definecolor{textcolor}{rgb}{0.000000,0.000000,0.000000}%
\pgfsetstrokecolor{textcolor}%
\pgfsetfillcolor{textcolor}%
\pgftext[x=1.404248in,y=0.511110in,,top]{\color{textcolor}{\rmfamily\fontsize{14.000000}{16.800000}\selectfont\catcode`\^=\active\def^{\ifmmode\sp\else\^{}\fi}\catcode`\%=\active\def%{\%}1989}}%
\end{pgfscope}%
\begin{pgfscope}%
\pgfpathrectangle{\pgfqpoint{0.697913in}{0.559721in}}{\pgfqpoint{7.048636in}{4.990279in}}%
\pgfusepath{clip}%
\pgfsetrectcap%
\pgfsetroundjoin%
\pgfsetlinewidth{0.803000pt}%
\definecolor{currentstroke}{rgb}{0.690196,0.690196,0.690196}%
\pgfsetstrokecolor{currentstroke}%
\pgfsetdash{}{0pt}%
\pgfpathmoveto{\pgfqpoint{2.287167in}{0.559721in}}%
\pgfpathlineto{\pgfqpoint{2.287167in}{5.550000in}}%
\pgfusepath{stroke}%
\end{pgfscope}%
\begin{pgfscope}%
\pgfsetbuttcap%
\pgfsetroundjoin%
\definecolor{currentfill}{rgb}{0.000000,0.000000,0.000000}%
\pgfsetfillcolor{currentfill}%
\pgfsetlinewidth{1.254687pt}%
\definecolor{currentstroke}{rgb}{0.000000,0.000000,0.000000}%
\pgfsetstrokecolor{currentstroke}%
\pgfsetdash{}{0pt}%
\pgfsys@defobject{currentmarker}{\pgfqpoint{0.000000in}{0.000000in}}{\pgfqpoint{0.000000in}{0.111111in}}{%
\pgfpathmoveto{\pgfqpoint{0.000000in}{0.000000in}}%
\pgfpathlineto{\pgfqpoint{0.000000in}{0.111111in}}%
\pgfusepath{stroke,fill}%
}%
\begin{pgfscope}%
\pgfsys@transformshift{2.287167in}{0.559721in}%
\pgfsys@useobject{currentmarker}{}%
\end{pgfscope}%
\end{pgfscope}%
\begin{pgfscope}%
\definecolor{textcolor}{rgb}{0.000000,0.000000,0.000000}%
\pgfsetstrokecolor{textcolor}%
\pgfsetfillcolor{textcolor}%
\pgftext[x=2.287167in,y=0.511110in,,top]{\color{textcolor}{\rmfamily\fontsize{14.000000}{16.800000}\selectfont\catcode`\^=\active\def^{\ifmmode\sp\else\^{}\fi}\catcode`\%=\active\def%{\%}1994}}%
\end{pgfscope}%
\begin{pgfscope}%
\pgfpathrectangle{\pgfqpoint{0.697913in}{0.559721in}}{\pgfqpoint{7.048636in}{4.990279in}}%
\pgfusepath{clip}%
\pgfsetrectcap%
\pgfsetroundjoin%
\pgfsetlinewidth{0.803000pt}%
\definecolor{currentstroke}{rgb}{0.690196,0.690196,0.690196}%
\pgfsetstrokecolor{currentstroke}%
\pgfsetdash{}{0pt}%
\pgfpathmoveto{\pgfqpoint{3.170086in}{0.559721in}}%
\pgfpathlineto{\pgfqpoint{3.170086in}{5.550000in}}%
\pgfusepath{stroke}%
\end{pgfscope}%
\begin{pgfscope}%
\pgfsetbuttcap%
\pgfsetroundjoin%
\definecolor{currentfill}{rgb}{0.000000,0.000000,0.000000}%
\pgfsetfillcolor{currentfill}%
\pgfsetlinewidth{1.254687pt}%
\definecolor{currentstroke}{rgb}{0.000000,0.000000,0.000000}%
\pgfsetstrokecolor{currentstroke}%
\pgfsetdash{}{0pt}%
\pgfsys@defobject{currentmarker}{\pgfqpoint{0.000000in}{0.000000in}}{\pgfqpoint{0.000000in}{0.111111in}}{%
\pgfpathmoveto{\pgfqpoint{0.000000in}{0.000000in}}%
\pgfpathlineto{\pgfqpoint{0.000000in}{0.111111in}}%
\pgfusepath{stroke,fill}%
}%
\begin{pgfscope}%
\pgfsys@transformshift{3.170086in}{0.559721in}%
\pgfsys@useobject{currentmarker}{}%
\end{pgfscope}%
\end{pgfscope}%
\begin{pgfscope}%
\definecolor{textcolor}{rgb}{0.000000,0.000000,0.000000}%
\pgfsetstrokecolor{textcolor}%
\pgfsetfillcolor{textcolor}%
\pgftext[x=3.170086in,y=0.511110in,,top]{\color{textcolor}{\rmfamily\fontsize{14.000000}{16.800000}\selectfont\catcode`\^=\active\def^{\ifmmode\sp\else\^{}\fi}\catcode`\%=\active\def%{\%}1999}}%
\end{pgfscope}%
\begin{pgfscope}%
\pgfpathrectangle{\pgfqpoint{0.697913in}{0.559721in}}{\pgfqpoint{7.048636in}{4.990279in}}%
\pgfusepath{clip}%
\pgfsetrectcap%
\pgfsetroundjoin%
\pgfsetlinewidth{0.803000pt}%
\definecolor{currentstroke}{rgb}{0.690196,0.690196,0.690196}%
\pgfsetstrokecolor{currentstroke}%
\pgfsetdash{}{0pt}%
\pgfpathmoveto{\pgfqpoint{4.053005in}{0.559721in}}%
\pgfpathlineto{\pgfqpoint{4.053005in}{5.550000in}}%
\pgfusepath{stroke}%
\end{pgfscope}%
\begin{pgfscope}%
\pgfsetbuttcap%
\pgfsetroundjoin%
\definecolor{currentfill}{rgb}{0.000000,0.000000,0.000000}%
\pgfsetfillcolor{currentfill}%
\pgfsetlinewidth{1.254687pt}%
\definecolor{currentstroke}{rgb}{0.000000,0.000000,0.000000}%
\pgfsetstrokecolor{currentstroke}%
\pgfsetdash{}{0pt}%
\pgfsys@defobject{currentmarker}{\pgfqpoint{0.000000in}{0.000000in}}{\pgfqpoint{0.000000in}{0.111111in}}{%
\pgfpathmoveto{\pgfqpoint{0.000000in}{0.000000in}}%
\pgfpathlineto{\pgfqpoint{0.000000in}{0.111111in}}%
\pgfusepath{stroke,fill}%
}%
\begin{pgfscope}%
\pgfsys@transformshift{4.053005in}{0.559721in}%
\pgfsys@useobject{currentmarker}{}%
\end{pgfscope}%
\end{pgfscope}%
\begin{pgfscope}%
\definecolor{textcolor}{rgb}{0.000000,0.000000,0.000000}%
\pgfsetstrokecolor{textcolor}%
\pgfsetfillcolor{textcolor}%
\pgftext[x=4.053005in,y=0.511110in,,top]{\color{textcolor}{\rmfamily\fontsize{14.000000}{16.800000}\selectfont\catcode`\^=\active\def^{\ifmmode\sp\else\^{}\fi}\catcode`\%=\active\def%{\%}2004}}%
\end{pgfscope}%
\begin{pgfscope}%
\pgfpathrectangle{\pgfqpoint{0.697913in}{0.559721in}}{\pgfqpoint{7.048636in}{4.990279in}}%
\pgfusepath{clip}%
\pgfsetrectcap%
\pgfsetroundjoin%
\pgfsetlinewidth{0.803000pt}%
\definecolor{currentstroke}{rgb}{0.690196,0.690196,0.690196}%
\pgfsetstrokecolor{currentstroke}%
\pgfsetdash{}{0pt}%
\pgfpathmoveto{\pgfqpoint{4.935924in}{0.559721in}}%
\pgfpathlineto{\pgfqpoint{4.935924in}{5.550000in}}%
\pgfusepath{stroke}%
\end{pgfscope}%
\begin{pgfscope}%
\pgfsetbuttcap%
\pgfsetroundjoin%
\definecolor{currentfill}{rgb}{0.000000,0.000000,0.000000}%
\pgfsetfillcolor{currentfill}%
\pgfsetlinewidth{1.254687pt}%
\definecolor{currentstroke}{rgb}{0.000000,0.000000,0.000000}%
\pgfsetstrokecolor{currentstroke}%
\pgfsetdash{}{0pt}%
\pgfsys@defobject{currentmarker}{\pgfqpoint{0.000000in}{0.000000in}}{\pgfqpoint{0.000000in}{0.111111in}}{%
\pgfpathmoveto{\pgfqpoint{0.000000in}{0.000000in}}%
\pgfpathlineto{\pgfqpoint{0.000000in}{0.111111in}}%
\pgfusepath{stroke,fill}%
}%
\begin{pgfscope}%
\pgfsys@transformshift{4.935924in}{0.559721in}%
\pgfsys@useobject{currentmarker}{}%
\end{pgfscope}%
\end{pgfscope}%
\begin{pgfscope}%
\definecolor{textcolor}{rgb}{0.000000,0.000000,0.000000}%
\pgfsetstrokecolor{textcolor}%
\pgfsetfillcolor{textcolor}%
\pgftext[x=4.935924in,y=0.511110in,,top]{\color{textcolor}{\rmfamily\fontsize{14.000000}{16.800000}\selectfont\catcode`\^=\active\def^{\ifmmode\sp\else\^{}\fi}\catcode`\%=\active\def%{\%}2009}}%
\end{pgfscope}%
\begin{pgfscope}%
\pgfpathrectangle{\pgfqpoint{0.697913in}{0.559721in}}{\pgfqpoint{7.048636in}{4.990279in}}%
\pgfusepath{clip}%
\pgfsetrectcap%
\pgfsetroundjoin%
\pgfsetlinewidth{0.803000pt}%
\definecolor{currentstroke}{rgb}{0.690196,0.690196,0.690196}%
\pgfsetstrokecolor{currentstroke}%
\pgfsetdash{}{0pt}%
\pgfpathmoveto{\pgfqpoint{5.818843in}{0.559721in}}%
\pgfpathlineto{\pgfqpoint{5.818843in}{5.550000in}}%
\pgfusepath{stroke}%
\end{pgfscope}%
\begin{pgfscope}%
\pgfsetbuttcap%
\pgfsetroundjoin%
\definecolor{currentfill}{rgb}{0.000000,0.000000,0.000000}%
\pgfsetfillcolor{currentfill}%
\pgfsetlinewidth{1.254687pt}%
\definecolor{currentstroke}{rgb}{0.000000,0.000000,0.000000}%
\pgfsetstrokecolor{currentstroke}%
\pgfsetdash{}{0pt}%
\pgfsys@defobject{currentmarker}{\pgfqpoint{0.000000in}{0.000000in}}{\pgfqpoint{0.000000in}{0.111111in}}{%
\pgfpathmoveto{\pgfqpoint{0.000000in}{0.000000in}}%
\pgfpathlineto{\pgfqpoint{0.000000in}{0.111111in}}%
\pgfusepath{stroke,fill}%
}%
\begin{pgfscope}%
\pgfsys@transformshift{5.818843in}{0.559721in}%
\pgfsys@useobject{currentmarker}{}%
\end{pgfscope}%
\end{pgfscope}%
\begin{pgfscope}%
\definecolor{textcolor}{rgb}{0.000000,0.000000,0.000000}%
\pgfsetstrokecolor{textcolor}%
\pgfsetfillcolor{textcolor}%
\pgftext[x=5.818843in,y=0.511110in,,top]{\color{textcolor}{\rmfamily\fontsize{14.000000}{16.800000}\selectfont\catcode`\^=\active\def^{\ifmmode\sp\else\^{}\fi}\catcode`\%=\active\def%{\%}2014}}%
\end{pgfscope}%
\begin{pgfscope}%
\pgfpathrectangle{\pgfqpoint{0.697913in}{0.559721in}}{\pgfqpoint{7.048636in}{4.990279in}}%
\pgfusepath{clip}%
\pgfsetrectcap%
\pgfsetroundjoin%
\pgfsetlinewidth{0.803000pt}%
\definecolor{currentstroke}{rgb}{0.690196,0.690196,0.690196}%
\pgfsetstrokecolor{currentstroke}%
\pgfsetdash{}{0pt}%
\pgfpathmoveto{\pgfqpoint{6.701762in}{0.559721in}}%
\pgfpathlineto{\pgfqpoint{6.701762in}{5.550000in}}%
\pgfusepath{stroke}%
\end{pgfscope}%
\begin{pgfscope}%
\pgfsetbuttcap%
\pgfsetroundjoin%
\definecolor{currentfill}{rgb}{0.000000,0.000000,0.000000}%
\pgfsetfillcolor{currentfill}%
\pgfsetlinewidth{1.254687pt}%
\definecolor{currentstroke}{rgb}{0.000000,0.000000,0.000000}%
\pgfsetstrokecolor{currentstroke}%
\pgfsetdash{}{0pt}%
\pgfsys@defobject{currentmarker}{\pgfqpoint{0.000000in}{0.000000in}}{\pgfqpoint{0.000000in}{0.111111in}}{%
\pgfpathmoveto{\pgfqpoint{0.000000in}{0.000000in}}%
\pgfpathlineto{\pgfqpoint{0.000000in}{0.111111in}}%
\pgfusepath{stroke,fill}%
}%
\begin{pgfscope}%
\pgfsys@transformshift{6.701762in}{0.559721in}%
\pgfsys@useobject{currentmarker}{}%
\end{pgfscope}%
\end{pgfscope}%
\begin{pgfscope}%
\definecolor{textcolor}{rgb}{0.000000,0.000000,0.000000}%
\pgfsetstrokecolor{textcolor}%
\pgfsetfillcolor{textcolor}%
\pgftext[x=6.701762in,y=0.511110in,,top]{\color{textcolor}{\rmfamily\fontsize{14.000000}{16.800000}\selectfont\catcode`\^=\active\def^{\ifmmode\sp\else\^{}\fi}\catcode`\%=\active\def%{\%}2019}}%
\end{pgfscope}%
\begin{pgfscope}%
\pgfpathrectangle{\pgfqpoint{0.697913in}{0.559721in}}{\pgfqpoint{7.048636in}{4.990279in}}%
\pgfusepath{clip}%
\pgfsetrectcap%
\pgfsetroundjoin%
\pgfsetlinewidth{0.803000pt}%
\definecolor{currentstroke}{rgb}{0.690196,0.690196,0.690196}%
\pgfsetstrokecolor{currentstroke}%
\pgfsetdash{}{0pt}%
\pgfpathmoveto{\pgfqpoint{7.584681in}{0.559721in}}%
\pgfpathlineto{\pgfqpoint{7.584681in}{5.550000in}}%
\pgfusepath{stroke}%
\end{pgfscope}%
\begin{pgfscope}%
\pgfsetbuttcap%
\pgfsetroundjoin%
\definecolor{currentfill}{rgb}{0.000000,0.000000,0.000000}%
\pgfsetfillcolor{currentfill}%
\pgfsetlinewidth{1.254687pt}%
\definecolor{currentstroke}{rgb}{0.000000,0.000000,0.000000}%
\pgfsetstrokecolor{currentstroke}%
\pgfsetdash{}{0pt}%
\pgfsys@defobject{currentmarker}{\pgfqpoint{0.000000in}{0.000000in}}{\pgfqpoint{0.000000in}{0.111111in}}{%
\pgfpathmoveto{\pgfqpoint{0.000000in}{0.000000in}}%
\pgfpathlineto{\pgfqpoint{0.000000in}{0.111111in}}%
\pgfusepath{stroke,fill}%
}%
\begin{pgfscope}%
\pgfsys@transformshift{7.584681in}{0.559721in}%
\pgfsys@useobject{currentmarker}{}%
\end{pgfscope}%
\end{pgfscope}%
\begin{pgfscope}%
\definecolor{textcolor}{rgb}{0.000000,0.000000,0.000000}%
\pgfsetstrokecolor{textcolor}%
\pgfsetfillcolor{textcolor}%
\pgftext[x=7.584681in,y=0.511110in,,top]{\color{textcolor}{\rmfamily\fontsize{14.000000}{16.800000}\selectfont\catcode`\^=\active\def^{\ifmmode\sp\else\^{}\fi}\catcode`\%=\active\def%{\%}2024}}%
\end{pgfscope}%
\begin{pgfscope}%
\pgfsetbuttcap%
\pgfsetroundjoin%
\definecolor{currentfill}{rgb}{0.000000,0.000000,0.000000}%
\pgfsetfillcolor{currentfill}%
\pgfsetlinewidth{1.254687pt}%
\definecolor{currentstroke}{rgb}{0.000000,0.000000,0.000000}%
\pgfsetstrokecolor{currentstroke}%
\pgfsetdash{}{0pt}%
\pgfsys@defobject{currentmarker}{\pgfqpoint{0.000000in}{0.000000in}}{\pgfqpoint{0.000000in}{0.055556in}}{%
\pgfpathmoveto{\pgfqpoint{0.000000in}{0.000000in}}%
\pgfpathlineto{\pgfqpoint{0.000000in}{0.055556in}}%
\pgfusepath{stroke,fill}%
}%
\begin{pgfscope}%
\pgfsys@transformshift{0.742059in}{0.559721in}%
\pgfsys@useobject{currentmarker}{}%
\end{pgfscope}%
\end{pgfscope}%
\begin{pgfscope}%
\pgfsetbuttcap%
\pgfsetroundjoin%
\definecolor{currentfill}{rgb}{0.000000,0.000000,0.000000}%
\pgfsetfillcolor{currentfill}%
\pgfsetlinewidth{1.254687pt}%
\definecolor{currentstroke}{rgb}{0.000000,0.000000,0.000000}%
\pgfsetstrokecolor{currentstroke}%
\pgfsetdash{}{0pt}%
\pgfsys@defobject{currentmarker}{\pgfqpoint{0.000000in}{0.000000in}}{\pgfqpoint{0.000000in}{0.055556in}}{%
\pgfpathmoveto{\pgfqpoint{0.000000in}{0.000000in}}%
\pgfpathlineto{\pgfqpoint{0.000000in}{0.055556in}}%
\pgfusepath{stroke,fill}%
}%
\begin{pgfscope}%
\pgfsys@transformshift{0.962788in}{0.559721in}%
\pgfsys@useobject{currentmarker}{}%
\end{pgfscope}%
\end{pgfscope}%
\begin{pgfscope}%
\pgfsetbuttcap%
\pgfsetroundjoin%
\definecolor{currentfill}{rgb}{0.000000,0.000000,0.000000}%
\pgfsetfillcolor{currentfill}%
\pgfsetlinewidth{1.254687pt}%
\definecolor{currentstroke}{rgb}{0.000000,0.000000,0.000000}%
\pgfsetstrokecolor{currentstroke}%
\pgfsetdash{}{0pt}%
\pgfsys@defobject{currentmarker}{\pgfqpoint{0.000000in}{0.000000in}}{\pgfqpoint{0.000000in}{0.055556in}}{%
\pgfpathmoveto{\pgfqpoint{0.000000in}{0.000000in}}%
\pgfpathlineto{\pgfqpoint{0.000000in}{0.055556in}}%
\pgfusepath{stroke,fill}%
}%
\begin{pgfscope}%
\pgfsys@transformshift{1.183518in}{0.559721in}%
\pgfsys@useobject{currentmarker}{}%
\end{pgfscope}%
\end{pgfscope}%
\begin{pgfscope}%
\pgfsetbuttcap%
\pgfsetroundjoin%
\definecolor{currentfill}{rgb}{0.000000,0.000000,0.000000}%
\pgfsetfillcolor{currentfill}%
\pgfsetlinewidth{1.254687pt}%
\definecolor{currentstroke}{rgb}{0.000000,0.000000,0.000000}%
\pgfsetstrokecolor{currentstroke}%
\pgfsetdash{}{0pt}%
\pgfsys@defobject{currentmarker}{\pgfqpoint{0.000000in}{0.000000in}}{\pgfqpoint{0.000000in}{0.055556in}}{%
\pgfpathmoveto{\pgfqpoint{0.000000in}{0.000000in}}%
\pgfpathlineto{\pgfqpoint{0.000000in}{0.055556in}}%
\pgfusepath{stroke,fill}%
}%
\begin{pgfscope}%
\pgfsys@transformshift{1.624978in}{0.559721in}%
\pgfsys@useobject{currentmarker}{}%
\end{pgfscope}%
\end{pgfscope}%
\begin{pgfscope}%
\pgfsetbuttcap%
\pgfsetroundjoin%
\definecolor{currentfill}{rgb}{0.000000,0.000000,0.000000}%
\pgfsetfillcolor{currentfill}%
\pgfsetlinewidth{1.254687pt}%
\definecolor{currentstroke}{rgb}{0.000000,0.000000,0.000000}%
\pgfsetstrokecolor{currentstroke}%
\pgfsetdash{}{0pt}%
\pgfsys@defobject{currentmarker}{\pgfqpoint{0.000000in}{0.000000in}}{\pgfqpoint{0.000000in}{0.055556in}}{%
\pgfpathmoveto{\pgfqpoint{0.000000in}{0.000000in}}%
\pgfpathlineto{\pgfqpoint{0.000000in}{0.055556in}}%
\pgfusepath{stroke,fill}%
}%
\begin{pgfscope}%
\pgfsys@transformshift{1.845707in}{0.559721in}%
\pgfsys@useobject{currentmarker}{}%
\end{pgfscope}%
\end{pgfscope}%
\begin{pgfscope}%
\pgfsetbuttcap%
\pgfsetroundjoin%
\definecolor{currentfill}{rgb}{0.000000,0.000000,0.000000}%
\pgfsetfillcolor{currentfill}%
\pgfsetlinewidth{1.254687pt}%
\definecolor{currentstroke}{rgb}{0.000000,0.000000,0.000000}%
\pgfsetstrokecolor{currentstroke}%
\pgfsetdash{}{0pt}%
\pgfsys@defobject{currentmarker}{\pgfqpoint{0.000000in}{0.000000in}}{\pgfqpoint{0.000000in}{0.055556in}}{%
\pgfpathmoveto{\pgfqpoint{0.000000in}{0.000000in}}%
\pgfpathlineto{\pgfqpoint{0.000000in}{0.055556in}}%
\pgfusepath{stroke,fill}%
}%
\begin{pgfscope}%
\pgfsys@transformshift{2.066437in}{0.559721in}%
\pgfsys@useobject{currentmarker}{}%
\end{pgfscope}%
\end{pgfscope}%
\begin{pgfscope}%
\pgfsetbuttcap%
\pgfsetroundjoin%
\definecolor{currentfill}{rgb}{0.000000,0.000000,0.000000}%
\pgfsetfillcolor{currentfill}%
\pgfsetlinewidth{1.254687pt}%
\definecolor{currentstroke}{rgb}{0.000000,0.000000,0.000000}%
\pgfsetstrokecolor{currentstroke}%
\pgfsetdash{}{0pt}%
\pgfsys@defobject{currentmarker}{\pgfqpoint{0.000000in}{0.000000in}}{\pgfqpoint{0.000000in}{0.055556in}}{%
\pgfpathmoveto{\pgfqpoint{0.000000in}{0.000000in}}%
\pgfpathlineto{\pgfqpoint{0.000000in}{0.055556in}}%
\pgfusepath{stroke,fill}%
}%
\begin{pgfscope}%
\pgfsys@transformshift{2.507897in}{0.559721in}%
\pgfsys@useobject{currentmarker}{}%
\end{pgfscope}%
\end{pgfscope}%
\begin{pgfscope}%
\pgfsetbuttcap%
\pgfsetroundjoin%
\definecolor{currentfill}{rgb}{0.000000,0.000000,0.000000}%
\pgfsetfillcolor{currentfill}%
\pgfsetlinewidth{1.254687pt}%
\definecolor{currentstroke}{rgb}{0.000000,0.000000,0.000000}%
\pgfsetstrokecolor{currentstroke}%
\pgfsetdash{}{0pt}%
\pgfsys@defobject{currentmarker}{\pgfqpoint{0.000000in}{0.000000in}}{\pgfqpoint{0.000000in}{0.055556in}}{%
\pgfpathmoveto{\pgfqpoint{0.000000in}{0.000000in}}%
\pgfpathlineto{\pgfqpoint{0.000000in}{0.055556in}}%
\pgfusepath{stroke,fill}%
}%
\begin{pgfscope}%
\pgfsys@transformshift{2.728626in}{0.559721in}%
\pgfsys@useobject{currentmarker}{}%
\end{pgfscope}%
\end{pgfscope}%
\begin{pgfscope}%
\pgfsetbuttcap%
\pgfsetroundjoin%
\definecolor{currentfill}{rgb}{0.000000,0.000000,0.000000}%
\pgfsetfillcolor{currentfill}%
\pgfsetlinewidth{1.254687pt}%
\definecolor{currentstroke}{rgb}{0.000000,0.000000,0.000000}%
\pgfsetstrokecolor{currentstroke}%
\pgfsetdash{}{0pt}%
\pgfsys@defobject{currentmarker}{\pgfqpoint{0.000000in}{0.000000in}}{\pgfqpoint{0.000000in}{0.055556in}}{%
\pgfpathmoveto{\pgfqpoint{0.000000in}{0.000000in}}%
\pgfpathlineto{\pgfqpoint{0.000000in}{0.055556in}}%
\pgfusepath{stroke,fill}%
}%
\begin{pgfscope}%
\pgfsys@transformshift{2.949356in}{0.559721in}%
\pgfsys@useobject{currentmarker}{}%
\end{pgfscope}%
\end{pgfscope}%
\begin{pgfscope}%
\pgfsetbuttcap%
\pgfsetroundjoin%
\definecolor{currentfill}{rgb}{0.000000,0.000000,0.000000}%
\pgfsetfillcolor{currentfill}%
\pgfsetlinewidth{1.254687pt}%
\definecolor{currentstroke}{rgb}{0.000000,0.000000,0.000000}%
\pgfsetstrokecolor{currentstroke}%
\pgfsetdash{}{0pt}%
\pgfsys@defobject{currentmarker}{\pgfqpoint{0.000000in}{0.000000in}}{\pgfqpoint{0.000000in}{0.055556in}}{%
\pgfpathmoveto{\pgfqpoint{0.000000in}{0.000000in}}%
\pgfpathlineto{\pgfqpoint{0.000000in}{0.055556in}}%
\pgfusepath{stroke,fill}%
}%
\begin{pgfscope}%
\pgfsys@transformshift{3.390815in}{0.559721in}%
\pgfsys@useobject{currentmarker}{}%
\end{pgfscope}%
\end{pgfscope}%
\begin{pgfscope}%
\pgfsetbuttcap%
\pgfsetroundjoin%
\definecolor{currentfill}{rgb}{0.000000,0.000000,0.000000}%
\pgfsetfillcolor{currentfill}%
\pgfsetlinewidth{1.254687pt}%
\definecolor{currentstroke}{rgb}{0.000000,0.000000,0.000000}%
\pgfsetstrokecolor{currentstroke}%
\pgfsetdash{}{0pt}%
\pgfsys@defobject{currentmarker}{\pgfqpoint{0.000000in}{0.000000in}}{\pgfqpoint{0.000000in}{0.055556in}}{%
\pgfpathmoveto{\pgfqpoint{0.000000in}{0.000000in}}%
\pgfpathlineto{\pgfqpoint{0.000000in}{0.055556in}}%
\pgfusepath{stroke,fill}%
}%
\begin{pgfscope}%
\pgfsys@transformshift{3.611545in}{0.559721in}%
\pgfsys@useobject{currentmarker}{}%
\end{pgfscope}%
\end{pgfscope}%
\begin{pgfscope}%
\pgfsetbuttcap%
\pgfsetroundjoin%
\definecolor{currentfill}{rgb}{0.000000,0.000000,0.000000}%
\pgfsetfillcolor{currentfill}%
\pgfsetlinewidth{1.254687pt}%
\definecolor{currentstroke}{rgb}{0.000000,0.000000,0.000000}%
\pgfsetstrokecolor{currentstroke}%
\pgfsetdash{}{0pt}%
\pgfsys@defobject{currentmarker}{\pgfqpoint{0.000000in}{0.000000in}}{\pgfqpoint{0.000000in}{0.055556in}}{%
\pgfpathmoveto{\pgfqpoint{0.000000in}{0.000000in}}%
\pgfpathlineto{\pgfqpoint{0.000000in}{0.055556in}}%
\pgfusepath{stroke,fill}%
}%
\begin{pgfscope}%
\pgfsys@transformshift{3.832275in}{0.559721in}%
\pgfsys@useobject{currentmarker}{}%
\end{pgfscope}%
\end{pgfscope}%
\begin{pgfscope}%
\pgfsetbuttcap%
\pgfsetroundjoin%
\definecolor{currentfill}{rgb}{0.000000,0.000000,0.000000}%
\pgfsetfillcolor{currentfill}%
\pgfsetlinewidth{1.254687pt}%
\definecolor{currentstroke}{rgb}{0.000000,0.000000,0.000000}%
\pgfsetstrokecolor{currentstroke}%
\pgfsetdash{}{0pt}%
\pgfsys@defobject{currentmarker}{\pgfqpoint{0.000000in}{0.000000in}}{\pgfqpoint{0.000000in}{0.055556in}}{%
\pgfpathmoveto{\pgfqpoint{0.000000in}{0.000000in}}%
\pgfpathlineto{\pgfqpoint{0.000000in}{0.055556in}}%
\pgfusepath{stroke,fill}%
}%
\begin{pgfscope}%
\pgfsys@transformshift{4.273734in}{0.559721in}%
\pgfsys@useobject{currentmarker}{}%
\end{pgfscope}%
\end{pgfscope}%
\begin{pgfscope}%
\pgfsetbuttcap%
\pgfsetroundjoin%
\definecolor{currentfill}{rgb}{0.000000,0.000000,0.000000}%
\pgfsetfillcolor{currentfill}%
\pgfsetlinewidth{1.254687pt}%
\definecolor{currentstroke}{rgb}{0.000000,0.000000,0.000000}%
\pgfsetstrokecolor{currentstroke}%
\pgfsetdash{}{0pt}%
\pgfsys@defobject{currentmarker}{\pgfqpoint{0.000000in}{0.000000in}}{\pgfqpoint{0.000000in}{0.055556in}}{%
\pgfpathmoveto{\pgfqpoint{0.000000in}{0.000000in}}%
\pgfpathlineto{\pgfqpoint{0.000000in}{0.055556in}}%
\pgfusepath{stroke,fill}%
}%
\begin{pgfscope}%
\pgfsys@transformshift{4.494464in}{0.559721in}%
\pgfsys@useobject{currentmarker}{}%
\end{pgfscope}%
\end{pgfscope}%
\begin{pgfscope}%
\pgfsetbuttcap%
\pgfsetroundjoin%
\definecolor{currentfill}{rgb}{0.000000,0.000000,0.000000}%
\pgfsetfillcolor{currentfill}%
\pgfsetlinewidth{1.254687pt}%
\definecolor{currentstroke}{rgb}{0.000000,0.000000,0.000000}%
\pgfsetstrokecolor{currentstroke}%
\pgfsetdash{}{0pt}%
\pgfsys@defobject{currentmarker}{\pgfqpoint{0.000000in}{0.000000in}}{\pgfqpoint{0.000000in}{0.055556in}}{%
\pgfpathmoveto{\pgfqpoint{0.000000in}{0.000000in}}%
\pgfpathlineto{\pgfqpoint{0.000000in}{0.055556in}}%
\pgfusepath{stroke,fill}%
}%
\begin{pgfscope}%
\pgfsys@transformshift{4.715194in}{0.559721in}%
\pgfsys@useobject{currentmarker}{}%
\end{pgfscope}%
\end{pgfscope}%
\begin{pgfscope}%
\pgfsetbuttcap%
\pgfsetroundjoin%
\definecolor{currentfill}{rgb}{0.000000,0.000000,0.000000}%
\pgfsetfillcolor{currentfill}%
\pgfsetlinewidth{1.254687pt}%
\definecolor{currentstroke}{rgb}{0.000000,0.000000,0.000000}%
\pgfsetstrokecolor{currentstroke}%
\pgfsetdash{}{0pt}%
\pgfsys@defobject{currentmarker}{\pgfqpoint{0.000000in}{0.000000in}}{\pgfqpoint{0.000000in}{0.055556in}}{%
\pgfpathmoveto{\pgfqpoint{0.000000in}{0.000000in}}%
\pgfpathlineto{\pgfqpoint{0.000000in}{0.055556in}}%
\pgfusepath{stroke,fill}%
}%
\begin{pgfscope}%
\pgfsys@transformshift{5.156653in}{0.559721in}%
\pgfsys@useobject{currentmarker}{}%
\end{pgfscope}%
\end{pgfscope}%
\begin{pgfscope}%
\pgfsetbuttcap%
\pgfsetroundjoin%
\definecolor{currentfill}{rgb}{0.000000,0.000000,0.000000}%
\pgfsetfillcolor{currentfill}%
\pgfsetlinewidth{1.254687pt}%
\definecolor{currentstroke}{rgb}{0.000000,0.000000,0.000000}%
\pgfsetstrokecolor{currentstroke}%
\pgfsetdash{}{0pt}%
\pgfsys@defobject{currentmarker}{\pgfqpoint{0.000000in}{0.000000in}}{\pgfqpoint{0.000000in}{0.055556in}}{%
\pgfpathmoveto{\pgfqpoint{0.000000in}{0.000000in}}%
\pgfpathlineto{\pgfqpoint{0.000000in}{0.055556in}}%
\pgfusepath{stroke,fill}%
}%
\begin{pgfscope}%
\pgfsys@transformshift{5.377383in}{0.559721in}%
\pgfsys@useobject{currentmarker}{}%
\end{pgfscope}%
\end{pgfscope}%
\begin{pgfscope}%
\pgfsetbuttcap%
\pgfsetroundjoin%
\definecolor{currentfill}{rgb}{0.000000,0.000000,0.000000}%
\pgfsetfillcolor{currentfill}%
\pgfsetlinewidth{1.254687pt}%
\definecolor{currentstroke}{rgb}{0.000000,0.000000,0.000000}%
\pgfsetstrokecolor{currentstroke}%
\pgfsetdash{}{0pt}%
\pgfsys@defobject{currentmarker}{\pgfqpoint{0.000000in}{0.000000in}}{\pgfqpoint{0.000000in}{0.055556in}}{%
\pgfpathmoveto{\pgfqpoint{0.000000in}{0.000000in}}%
\pgfpathlineto{\pgfqpoint{0.000000in}{0.055556in}}%
\pgfusepath{stroke,fill}%
}%
\begin{pgfscope}%
\pgfsys@transformshift{5.598113in}{0.559721in}%
\pgfsys@useobject{currentmarker}{}%
\end{pgfscope}%
\end{pgfscope}%
\begin{pgfscope}%
\pgfsetbuttcap%
\pgfsetroundjoin%
\definecolor{currentfill}{rgb}{0.000000,0.000000,0.000000}%
\pgfsetfillcolor{currentfill}%
\pgfsetlinewidth{1.254687pt}%
\definecolor{currentstroke}{rgb}{0.000000,0.000000,0.000000}%
\pgfsetstrokecolor{currentstroke}%
\pgfsetdash{}{0pt}%
\pgfsys@defobject{currentmarker}{\pgfqpoint{0.000000in}{0.000000in}}{\pgfqpoint{0.000000in}{0.055556in}}{%
\pgfpathmoveto{\pgfqpoint{0.000000in}{0.000000in}}%
\pgfpathlineto{\pgfqpoint{0.000000in}{0.055556in}}%
\pgfusepath{stroke,fill}%
}%
\begin{pgfscope}%
\pgfsys@transformshift{6.039572in}{0.559721in}%
\pgfsys@useobject{currentmarker}{}%
\end{pgfscope}%
\end{pgfscope}%
\begin{pgfscope}%
\pgfsetbuttcap%
\pgfsetroundjoin%
\definecolor{currentfill}{rgb}{0.000000,0.000000,0.000000}%
\pgfsetfillcolor{currentfill}%
\pgfsetlinewidth{1.254687pt}%
\definecolor{currentstroke}{rgb}{0.000000,0.000000,0.000000}%
\pgfsetstrokecolor{currentstroke}%
\pgfsetdash{}{0pt}%
\pgfsys@defobject{currentmarker}{\pgfqpoint{0.000000in}{0.000000in}}{\pgfqpoint{0.000000in}{0.055556in}}{%
\pgfpathmoveto{\pgfqpoint{0.000000in}{0.000000in}}%
\pgfpathlineto{\pgfqpoint{0.000000in}{0.055556in}}%
\pgfusepath{stroke,fill}%
}%
\begin{pgfscope}%
\pgfsys@transformshift{6.260302in}{0.559721in}%
\pgfsys@useobject{currentmarker}{}%
\end{pgfscope}%
\end{pgfscope}%
\begin{pgfscope}%
\pgfsetbuttcap%
\pgfsetroundjoin%
\definecolor{currentfill}{rgb}{0.000000,0.000000,0.000000}%
\pgfsetfillcolor{currentfill}%
\pgfsetlinewidth{1.254687pt}%
\definecolor{currentstroke}{rgb}{0.000000,0.000000,0.000000}%
\pgfsetstrokecolor{currentstroke}%
\pgfsetdash{}{0pt}%
\pgfsys@defobject{currentmarker}{\pgfqpoint{0.000000in}{0.000000in}}{\pgfqpoint{0.000000in}{0.055556in}}{%
\pgfpathmoveto{\pgfqpoint{0.000000in}{0.000000in}}%
\pgfpathlineto{\pgfqpoint{0.000000in}{0.055556in}}%
\pgfusepath{stroke,fill}%
}%
\begin{pgfscope}%
\pgfsys@transformshift{6.481032in}{0.559721in}%
\pgfsys@useobject{currentmarker}{}%
\end{pgfscope}%
\end{pgfscope}%
\begin{pgfscope}%
\pgfsetbuttcap%
\pgfsetroundjoin%
\definecolor{currentfill}{rgb}{0.000000,0.000000,0.000000}%
\pgfsetfillcolor{currentfill}%
\pgfsetlinewidth{1.254687pt}%
\definecolor{currentstroke}{rgb}{0.000000,0.000000,0.000000}%
\pgfsetstrokecolor{currentstroke}%
\pgfsetdash{}{0pt}%
\pgfsys@defobject{currentmarker}{\pgfqpoint{0.000000in}{0.000000in}}{\pgfqpoint{0.000000in}{0.055556in}}{%
\pgfpathmoveto{\pgfqpoint{0.000000in}{0.000000in}}%
\pgfpathlineto{\pgfqpoint{0.000000in}{0.055556in}}%
\pgfusepath{stroke,fill}%
}%
\begin{pgfscope}%
\pgfsys@transformshift{6.922491in}{0.559721in}%
\pgfsys@useobject{currentmarker}{}%
\end{pgfscope}%
\end{pgfscope}%
\begin{pgfscope}%
\pgfsetbuttcap%
\pgfsetroundjoin%
\definecolor{currentfill}{rgb}{0.000000,0.000000,0.000000}%
\pgfsetfillcolor{currentfill}%
\pgfsetlinewidth{1.254687pt}%
\definecolor{currentstroke}{rgb}{0.000000,0.000000,0.000000}%
\pgfsetstrokecolor{currentstroke}%
\pgfsetdash{}{0pt}%
\pgfsys@defobject{currentmarker}{\pgfqpoint{0.000000in}{0.000000in}}{\pgfqpoint{0.000000in}{0.055556in}}{%
\pgfpathmoveto{\pgfqpoint{0.000000in}{0.000000in}}%
\pgfpathlineto{\pgfqpoint{0.000000in}{0.055556in}}%
\pgfusepath{stroke,fill}%
}%
\begin{pgfscope}%
\pgfsys@transformshift{7.143221in}{0.559721in}%
\pgfsys@useobject{currentmarker}{}%
\end{pgfscope}%
\end{pgfscope}%
\begin{pgfscope}%
\pgfsetbuttcap%
\pgfsetroundjoin%
\definecolor{currentfill}{rgb}{0.000000,0.000000,0.000000}%
\pgfsetfillcolor{currentfill}%
\pgfsetlinewidth{1.254687pt}%
\definecolor{currentstroke}{rgb}{0.000000,0.000000,0.000000}%
\pgfsetstrokecolor{currentstroke}%
\pgfsetdash{}{0pt}%
\pgfsys@defobject{currentmarker}{\pgfqpoint{0.000000in}{0.000000in}}{\pgfqpoint{0.000000in}{0.055556in}}{%
\pgfpathmoveto{\pgfqpoint{0.000000in}{0.000000in}}%
\pgfpathlineto{\pgfqpoint{0.000000in}{0.055556in}}%
\pgfusepath{stroke,fill}%
}%
\begin{pgfscope}%
\pgfsys@transformshift{7.363951in}{0.559721in}%
\pgfsys@useobject{currentmarker}{}%
\end{pgfscope}%
\end{pgfscope}%
\begin{pgfscope}%
\definecolor{textcolor}{rgb}{0.000000,0.000000,0.000000}%
\pgfsetstrokecolor{textcolor}%
\pgfsetfillcolor{textcolor}%
\pgftext[x=4.222231in,y=0.277777in,,top]{\color{textcolor}{\rmfamily\fontsize{14.000000}{16.800000}\selectfont\catcode`\^=\active\def^{\ifmmode\sp\else\^{}\fi}\catcode`\%=\active\def%{\%}Year}}%
\end{pgfscope}%
\begin{pgfscope}%
\pgfpathrectangle{\pgfqpoint{0.697913in}{0.559721in}}{\pgfqpoint{7.048636in}{4.990279in}}%
\pgfusepath{clip}%
\pgfsetrectcap%
\pgfsetroundjoin%
\pgfsetlinewidth{0.803000pt}%
\definecolor{currentstroke}{rgb}{0.690196,0.690196,0.690196}%
\pgfsetstrokecolor{currentstroke}%
\pgfsetdash{}{0pt}%
\pgfpathmoveto{\pgfqpoint{0.697913in}{1.056039in}}%
\pgfpathlineto{\pgfqpoint{7.746549in}{1.056039in}}%
\pgfusepath{stroke}%
\end{pgfscope}%
\begin{pgfscope}%
\pgfsetbuttcap%
\pgfsetroundjoin%
\definecolor{currentfill}{rgb}{0.000000,0.000000,0.000000}%
\pgfsetfillcolor{currentfill}%
\pgfsetlinewidth{1.254687pt}%
\definecolor{currentstroke}{rgb}{0.000000,0.000000,0.000000}%
\pgfsetstrokecolor{currentstroke}%
\pgfsetdash{}{0pt}%
\pgfsys@defobject{currentmarker}{\pgfqpoint{0.000000in}{0.000000in}}{\pgfqpoint{0.111111in}{0.000000in}}{%
\pgfpathmoveto{\pgfqpoint{0.000000in}{0.000000in}}%
\pgfpathlineto{\pgfqpoint{0.111111in}{0.000000in}}%
\pgfusepath{stroke,fill}%
}%
\begin{pgfscope}%
\pgfsys@transformshift{0.697913in}{1.056039in}%
\pgfsys@useobject{currentmarker}{}%
\end{pgfscope}%
\end{pgfscope}%
\begin{pgfscope}%
\definecolor{textcolor}{rgb}{0.000000,0.000000,0.000000}%
\pgfsetstrokecolor{textcolor}%
\pgfsetfillcolor{textcolor}%
\pgftext[x=0.355555in, y=0.986595in, left, base]{\color{textcolor}{\rmfamily\fontsize{14.000000}{16.800000}\selectfont\catcode`\^=\active\def^{\ifmmode\sp\else\^{}\fi}\catcode`\%=\active\def%{\%}$\mathdefault{350}$}}%
\end{pgfscope}%
\begin{pgfscope}%
\pgfpathrectangle{\pgfqpoint{0.697913in}{0.559721in}}{\pgfqpoint{7.048636in}{4.990279in}}%
\pgfusepath{clip}%
\pgfsetrectcap%
\pgfsetroundjoin%
\pgfsetlinewidth{0.803000pt}%
\definecolor{currentstroke}{rgb}{0.690196,0.690196,0.690196}%
\pgfsetstrokecolor{currentstroke}%
\pgfsetdash{}{0pt}%
\pgfpathmoveto{\pgfqpoint{0.697913in}{1.620678in}}%
\pgfpathlineto{\pgfqpoint{7.746549in}{1.620678in}}%
\pgfusepath{stroke}%
\end{pgfscope}%
\begin{pgfscope}%
\pgfsetbuttcap%
\pgfsetroundjoin%
\definecolor{currentfill}{rgb}{0.000000,0.000000,0.000000}%
\pgfsetfillcolor{currentfill}%
\pgfsetlinewidth{1.254687pt}%
\definecolor{currentstroke}{rgb}{0.000000,0.000000,0.000000}%
\pgfsetstrokecolor{currentstroke}%
\pgfsetdash{}{0pt}%
\pgfsys@defobject{currentmarker}{\pgfqpoint{0.000000in}{0.000000in}}{\pgfqpoint{0.111111in}{0.000000in}}{%
\pgfpathmoveto{\pgfqpoint{0.000000in}{0.000000in}}%
\pgfpathlineto{\pgfqpoint{0.111111in}{0.000000in}}%
\pgfusepath{stroke,fill}%
}%
\begin{pgfscope}%
\pgfsys@transformshift{0.697913in}{1.620678in}%
\pgfsys@useobject{currentmarker}{}%
\end{pgfscope}%
\end{pgfscope}%
\begin{pgfscope}%
\definecolor{textcolor}{rgb}{0.000000,0.000000,0.000000}%
\pgfsetstrokecolor{textcolor}%
\pgfsetfillcolor{textcolor}%
\pgftext[x=0.355555in, y=1.551234in, left, base]{\color{textcolor}{\rmfamily\fontsize{14.000000}{16.800000}\selectfont\catcode`\^=\active\def^{\ifmmode\sp\else\^{}\fi}\catcode`\%=\active\def%{\%}$\mathdefault{360}$}}%
\end{pgfscope}%
\begin{pgfscope}%
\pgfpathrectangle{\pgfqpoint{0.697913in}{0.559721in}}{\pgfqpoint{7.048636in}{4.990279in}}%
\pgfusepath{clip}%
\pgfsetrectcap%
\pgfsetroundjoin%
\pgfsetlinewidth{0.803000pt}%
\definecolor{currentstroke}{rgb}{0.690196,0.690196,0.690196}%
\pgfsetstrokecolor{currentstroke}%
\pgfsetdash{}{0pt}%
\pgfpathmoveto{\pgfqpoint{0.697913in}{2.185317in}}%
\pgfpathlineto{\pgfqpoint{7.746549in}{2.185317in}}%
\pgfusepath{stroke}%
\end{pgfscope}%
\begin{pgfscope}%
\pgfsetbuttcap%
\pgfsetroundjoin%
\definecolor{currentfill}{rgb}{0.000000,0.000000,0.000000}%
\pgfsetfillcolor{currentfill}%
\pgfsetlinewidth{1.254687pt}%
\definecolor{currentstroke}{rgb}{0.000000,0.000000,0.000000}%
\pgfsetstrokecolor{currentstroke}%
\pgfsetdash{}{0pt}%
\pgfsys@defobject{currentmarker}{\pgfqpoint{0.000000in}{0.000000in}}{\pgfqpoint{0.111111in}{0.000000in}}{%
\pgfpathmoveto{\pgfqpoint{0.000000in}{0.000000in}}%
\pgfpathlineto{\pgfqpoint{0.111111in}{0.000000in}}%
\pgfusepath{stroke,fill}%
}%
\begin{pgfscope}%
\pgfsys@transformshift{0.697913in}{2.185317in}%
\pgfsys@useobject{currentmarker}{}%
\end{pgfscope}%
\end{pgfscope}%
\begin{pgfscope}%
\definecolor{textcolor}{rgb}{0.000000,0.000000,0.000000}%
\pgfsetstrokecolor{textcolor}%
\pgfsetfillcolor{textcolor}%
\pgftext[x=0.355555in, y=2.115872in, left, base]{\color{textcolor}{\rmfamily\fontsize{14.000000}{16.800000}\selectfont\catcode`\^=\active\def^{\ifmmode\sp\else\^{}\fi}\catcode`\%=\active\def%{\%}$\mathdefault{370}$}}%
\end{pgfscope}%
\begin{pgfscope}%
\pgfpathrectangle{\pgfqpoint{0.697913in}{0.559721in}}{\pgfqpoint{7.048636in}{4.990279in}}%
\pgfusepath{clip}%
\pgfsetrectcap%
\pgfsetroundjoin%
\pgfsetlinewidth{0.803000pt}%
\definecolor{currentstroke}{rgb}{0.690196,0.690196,0.690196}%
\pgfsetstrokecolor{currentstroke}%
\pgfsetdash{}{0pt}%
\pgfpathmoveto{\pgfqpoint{0.697913in}{2.749956in}}%
\pgfpathlineto{\pgfqpoint{7.746549in}{2.749956in}}%
\pgfusepath{stroke}%
\end{pgfscope}%
\begin{pgfscope}%
\pgfsetbuttcap%
\pgfsetroundjoin%
\definecolor{currentfill}{rgb}{0.000000,0.000000,0.000000}%
\pgfsetfillcolor{currentfill}%
\pgfsetlinewidth{1.254687pt}%
\definecolor{currentstroke}{rgb}{0.000000,0.000000,0.000000}%
\pgfsetstrokecolor{currentstroke}%
\pgfsetdash{}{0pt}%
\pgfsys@defobject{currentmarker}{\pgfqpoint{0.000000in}{0.000000in}}{\pgfqpoint{0.111111in}{0.000000in}}{%
\pgfpathmoveto{\pgfqpoint{0.000000in}{0.000000in}}%
\pgfpathlineto{\pgfqpoint{0.111111in}{0.000000in}}%
\pgfusepath{stroke,fill}%
}%
\begin{pgfscope}%
\pgfsys@transformshift{0.697913in}{2.749956in}%
\pgfsys@useobject{currentmarker}{}%
\end{pgfscope}%
\end{pgfscope}%
\begin{pgfscope}%
\definecolor{textcolor}{rgb}{0.000000,0.000000,0.000000}%
\pgfsetstrokecolor{textcolor}%
\pgfsetfillcolor{textcolor}%
\pgftext[x=0.355555in, y=2.680511in, left, base]{\color{textcolor}{\rmfamily\fontsize{14.000000}{16.800000}\selectfont\catcode`\^=\active\def^{\ifmmode\sp\else\^{}\fi}\catcode`\%=\active\def%{\%}$\mathdefault{380}$}}%
\end{pgfscope}%
\begin{pgfscope}%
\pgfpathrectangle{\pgfqpoint{0.697913in}{0.559721in}}{\pgfqpoint{7.048636in}{4.990279in}}%
\pgfusepath{clip}%
\pgfsetrectcap%
\pgfsetroundjoin%
\pgfsetlinewidth{0.803000pt}%
\definecolor{currentstroke}{rgb}{0.690196,0.690196,0.690196}%
\pgfsetstrokecolor{currentstroke}%
\pgfsetdash{}{0pt}%
\pgfpathmoveto{\pgfqpoint{0.697913in}{3.314595in}}%
\pgfpathlineto{\pgfqpoint{7.746549in}{3.314595in}}%
\pgfusepath{stroke}%
\end{pgfscope}%
\begin{pgfscope}%
\pgfsetbuttcap%
\pgfsetroundjoin%
\definecolor{currentfill}{rgb}{0.000000,0.000000,0.000000}%
\pgfsetfillcolor{currentfill}%
\pgfsetlinewidth{1.254687pt}%
\definecolor{currentstroke}{rgb}{0.000000,0.000000,0.000000}%
\pgfsetstrokecolor{currentstroke}%
\pgfsetdash{}{0pt}%
\pgfsys@defobject{currentmarker}{\pgfqpoint{0.000000in}{0.000000in}}{\pgfqpoint{0.111111in}{0.000000in}}{%
\pgfpathmoveto{\pgfqpoint{0.000000in}{0.000000in}}%
\pgfpathlineto{\pgfqpoint{0.111111in}{0.000000in}}%
\pgfusepath{stroke,fill}%
}%
\begin{pgfscope}%
\pgfsys@transformshift{0.697913in}{3.314595in}%
\pgfsys@useobject{currentmarker}{}%
\end{pgfscope}%
\end{pgfscope}%
\begin{pgfscope}%
\definecolor{textcolor}{rgb}{0.000000,0.000000,0.000000}%
\pgfsetstrokecolor{textcolor}%
\pgfsetfillcolor{textcolor}%
\pgftext[x=0.355555in, y=3.245150in, left, base]{\color{textcolor}{\rmfamily\fontsize{14.000000}{16.800000}\selectfont\catcode`\^=\active\def^{\ifmmode\sp\else\^{}\fi}\catcode`\%=\active\def%{\%}$\mathdefault{390}$}}%
\end{pgfscope}%
\begin{pgfscope}%
\pgfpathrectangle{\pgfqpoint{0.697913in}{0.559721in}}{\pgfqpoint{7.048636in}{4.990279in}}%
\pgfusepath{clip}%
\pgfsetrectcap%
\pgfsetroundjoin%
\pgfsetlinewidth{0.803000pt}%
\definecolor{currentstroke}{rgb}{0.690196,0.690196,0.690196}%
\pgfsetstrokecolor{currentstroke}%
\pgfsetdash{}{0pt}%
\pgfpathmoveto{\pgfqpoint{0.697913in}{3.879233in}}%
\pgfpathlineto{\pgfqpoint{7.746549in}{3.879233in}}%
\pgfusepath{stroke}%
\end{pgfscope}%
\begin{pgfscope}%
\pgfsetbuttcap%
\pgfsetroundjoin%
\definecolor{currentfill}{rgb}{0.000000,0.000000,0.000000}%
\pgfsetfillcolor{currentfill}%
\pgfsetlinewidth{1.254687pt}%
\definecolor{currentstroke}{rgb}{0.000000,0.000000,0.000000}%
\pgfsetstrokecolor{currentstroke}%
\pgfsetdash{}{0pt}%
\pgfsys@defobject{currentmarker}{\pgfqpoint{0.000000in}{0.000000in}}{\pgfqpoint{0.111111in}{0.000000in}}{%
\pgfpathmoveto{\pgfqpoint{0.000000in}{0.000000in}}%
\pgfpathlineto{\pgfqpoint{0.111111in}{0.000000in}}%
\pgfusepath{stroke,fill}%
}%
\begin{pgfscope}%
\pgfsys@transformshift{0.697913in}{3.879233in}%
\pgfsys@useobject{currentmarker}{}%
\end{pgfscope}%
\end{pgfscope}%
\begin{pgfscope}%
\definecolor{textcolor}{rgb}{0.000000,0.000000,0.000000}%
\pgfsetstrokecolor{textcolor}%
\pgfsetfillcolor{textcolor}%
\pgftext[x=0.355555in, y=3.809789in, left, base]{\color{textcolor}{\rmfamily\fontsize{14.000000}{16.800000}\selectfont\catcode`\^=\active\def^{\ifmmode\sp\else\^{}\fi}\catcode`\%=\active\def%{\%}$\mathdefault{400}$}}%
\end{pgfscope}%
\begin{pgfscope}%
\pgfpathrectangle{\pgfqpoint{0.697913in}{0.559721in}}{\pgfqpoint{7.048636in}{4.990279in}}%
\pgfusepath{clip}%
\pgfsetrectcap%
\pgfsetroundjoin%
\pgfsetlinewidth{0.803000pt}%
\definecolor{currentstroke}{rgb}{0.690196,0.690196,0.690196}%
\pgfsetstrokecolor{currentstroke}%
\pgfsetdash{}{0pt}%
\pgfpathmoveto{\pgfqpoint{0.697913in}{4.443872in}}%
\pgfpathlineto{\pgfqpoint{7.746549in}{4.443872in}}%
\pgfusepath{stroke}%
\end{pgfscope}%
\begin{pgfscope}%
\pgfsetbuttcap%
\pgfsetroundjoin%
\definecolor{currentfill}{rgb}{0.000000,0.000000,0.000000}%
\pgfsetfillcolor{currentfill}%
\pgfsetlinewidth{1.254687pt}%
\definecolor{currentstroke}{rgb}{0.000000,0.000000,0.000000}%
\pgfsetstrokecolor{currentstroke}%
\pgfsetdash{}{0pt}%
\pgfsys@defobject{currentmarker}{\pgfqpoint{0.000000in}{0.000000in}}{\pgfqpoint{0.111111in}{0.000000in}}{%
\pgfpathmoveto{\pgfqpoint{0.000000in}{0.000000in}}%
\pgfpathlineto{\pgfqpoint{0.111111in}{0.000000in}}%
\pgfusepath{stroke,fill}%
}%
\begin{pgfscope}%
\pgfsys@transformshift{0.697913in}{4.443872in}%
\pgfsys@useobject{currentmarker}{}%
\end{pgfscope}%
\end{pgfscope}%
\begin{pgfscope}%
\definecolor{textcolor}{rgb}{0.000000,0.000000,0.000000}%
\pgfsetstrokecolor{textcolor}%
\pgfsetfillcolor{textcolor}%
\pgftext[x=0.355555in, y=4.374428in, left, base]{\color{textcolor}{\rmfamily\fontsize{14.000000}{16.800000}\selectfont\catcode`\^=\active\def^{\ifmmode\sp\else\^{}\fi}\catcode`\%=\active\def%{\%}$\mathdefault{410}$}}%
\end{pgfscope}%
\begin{pgfscope}%
\pgfpathrectangle{\pgfqpoint{0.697913in}{0.559721in}}{\pgfqpoint{7.048636in}{4.990279in}}%
\pgfusepath{clip}%
\pgfsetrectcap%
\pgfsetroundjoin%
\pgfsetlinewidth{0.803000pt}%
\definecolor{currentstroke}{rgb}{0.690196,0.690196,0.690196}%
\pgfsetstrokecolor{currentstroke}%
\pgfsetdash{}{0pt}%
\pgfpathmoveto{\pgfqpoint{0.697913in}{5.008511in}}%
\pgfpathlineto{\pgfqpoint{7.746549in}{5.008511in}}%
\pgfusepath{stroke}%
\end{pgfscope}%
\begin{pgfscope}%
\pgfsetbuttcap%
\pgfsetroundjoin%
\definecolor{currentfill}{rgb}{0.000000,0.000000,0.000000}%
\pgfsetfillcolor{currentfill}%
\pgfsetlinewidth{1.254687pt}%
\definecolor{currentstroke}{rgb}{0.000000,0.000000,0.000000}%
\pgfsetstrokecolor{currentstroke}%
\pgfsetdash{}{0pt}%
\pgfsys@defobject{currentmarker}{\pgfqpoint{0.000000in}{0.000000in}}{\pgfqpoint{0.111111in}{0.000000in}}{%
\pgfpathmoveto{\pgfqpoint{0.000000in}{0.000000in}}%
\pgfpathlineto{\pgfqpoint{0.111111in}{0.000000in}}%
\pgfusepath{stroke,fill}%
}%
\begin{pgfscope}%
\pgfsys@transformshift{0.697913in}{5.008511in}%
\pgfsys@useobject{currentmarker}{}%
\end{pgfscope}%
\end{pgfscope}%
\begin{pgfscope}%
\definecolor{textcolor}{rgb}{0.000000,0.000000,0.000000}%
\pgfsetstrokecolor{textcolor}%
\pgfsetfillcolor{textcolor}%
\pgftext[x=0.355555in, y=4.939067in, left, base]{\color{textcolor}{\rmfamily\fontsize{14.000000}{16.800000}\selectfont\catcode`\^=\active\def^{\ifmmode\sp\else\^{}\fi}\catcode`\%=\active\def%{\%}$\mathdefault{420}$}}%
\end{pgfscope}%
\begin{pgfscope}%
\pgfsetbuttcap%
\pgfsetroundjoin%
\definecolor{currentfill}{rgb}{0.000000,0.000000,0.000000}%
\pgfsetfillcolor{currentfill}%
\pgfsetlinewidth{1.254687pt}%
\definecolor{currentstroke}{rgb}{0.000000,0.000000,0.000000}%
\pgfsetstrokecolor{currentstroke}%
\pgfsetdash{}{0pt}%
\pgfsys@defobject{currentmarker}{\pgfqpoint{0.000000in}{0.000000in}}{\pgfqpoint{0.055556in}{0.000000in}}{%
\pgfpathmoveto{\pgfqpoint{0.000000in}{0.000000in}}%
\pgfpathlineto{\pgfqpoint{0.055556in}{0.000000in}}%
\pgfusepath{stroke,fill}%
}%
\begin{pgfscope}%
\pgfsys@transformshift{0.697913in}{0.604328in}%
\pgfsys@useobject{currentmarker}{}%
\end{pgfscope}%
\end{pgfscope}%
\begin{pgfscope}%
\pgfsetbuttcap%
\pgfsetroundjoin%
\definecolor{currentfill}{rgb}{0.000000,0.000000,0.000000}%
\pgfsetfillcolor{currentfill}%
\pgfsetlinewidth{1.254687pt}%
\definecolor{currentstroke}{rgb}{0.000000,0.000000,0.000000}%
\pgfsetstrokecolor{currentstroke}%
\pgfsetdash{}{0pt}%
\pgfsys@defobject{currentmarker}{\pgfqpoint{0.000000in}{0.000000in}}{\pgfqpoint{0.055556in}{0.000000in}}{%
\pgfpathmoveto{\pgfqpoint{0.000000in}{0.000000in}}%
\pgfpathlineto{\pgfqpoint{0.055556in}{0.000000in}}%
\pgfusepath{stroke,fill}%
}%
\begin{pgfscope}%
\pgfsys@transformshift{0.697913in}{0.717256in}%
\pgfsys@useobject{currentmarker}{}%
\end{pgfscope}%
\end{pgfscope}%
\begin{pgfscope}%
\pgfsetbuttcap%
\pgfsetroundjoin%
\definecolor{currentfill}{rgb}{0.000000,0.000000,0.000000}%
\pgfsetfillcolor{currentfill}%
\pgfsetlinewidth{1.254687pt}%
\definecolor{currentstroke}{rgb}{0.000000,0.000000,0.000000}%
\pgfsetstrokecolor{currentstroke}%
\pgfsetdash{}{0pt}%
\pgfsys@defobject{currentmarker}{\pgfqpoint{0.000000in}{0.000000in}}{\pgfqpoint{0.055556in}{0.000000in}}{%
\pgfpathmoveto{\pgfqpoint{0.000000in}{0.000000in}}%
\pgfpathlineto{\pgfqpoint{0.055556in}{0.000000in}}%
\pgfusepath{stroke,fill}%
}%
\begin{pgfscope}%
\pgfsys@transformshift{0.697913in}{0.830183in}%
\pgfsys@useobject{currentmarker}{}%
\end{pgfscope}%
\end{pgfscope}%
\begin{pgfscope}%
\pgfsetbuttcap%
\pgfsetroundjoin%
\definecolor{currentfill}{rgb}{0.000000,0.000000,0.000000}%
\pgfsetfillcolor{currentfill}%
\pgfsetlinewidth{1.254687pt}%
\definecolor{currentstroke}{rgb}{0.000000,0.000000,0.000000}%
\pgfsetstrokecolor{currentstroke}%
\pgfsetdash{}{0pt}%
\pgfsys@defobject{currentmarker}{\pgfqpoint{0.000000in}{0.000000in}}{\pgfqpoint{0.055556in}{0.000000in}}{%
\pgfpathmoveto{\pgfqpoint{0.000000in}{0.000000in}}%
\pgfpathlineto{\pgfqpoint{0.055556in}{0.000000in}}%
\pgfusepath{stroke,fill}%
}%
\begin{pgfscope}%
\pgfsys@transformshift{0.697913in}{0.943111in}%
\pgfsys@useobject{currentmarker}{}%
\end{pgfscope}%
\end{pgfscope}%
\begin{pgfscope}%
\pgfsetbuttcap%
\pgfsetroundjoin%
\definecolor{currentfill}{rgb}{0.000000,0.000000,0.000000}%
\pgfsetfillcolor{currentfill}%
\pgfsetlinewidth{1.254687pt}%
\definecolor{currentstroke}{rgb}{0.000000,0.000000,0.000000}%
\pgfsetstrokecolor{currentstroke}%
\pgfsetdash{}{0pt}%
\pgfsys@defobject{currentmarker}{\pgfqpoint{0.000000in}{0.000000in}}{\pgfqpoint{0.055556in}{0.000000in}}{%
\pgfpathmoveto{\pgfqpoint{0.000000in}{0.000000in}}%
\pgfpathlineto{\pgfqpoint{0.055556in}{0.000000in}}%
\pgfusepath{stroke,fill}%
}%
\begin{pgfscope}%
\pgfsys@transformshift{0.697913in}{1.168967in}%
\pgfsys@useobject{currentmarker}{}%
\end{pgfscope}%
\end{pgfscope}%
\begin{pgfscope}%
\pgfsetbuttcap%
\pgfsetroundjoin%
\definecolor{currentfill}{rgb}{0.000000,0.000000,0.000000}%
\pgfsetfillcolor{currentfill}%
\pgfsetlinewidth{1.254687pt}%
\definecolor{currentstroke}{rgb}{0.000000,0.000000,0.000000}%
\pgfsetstrokecolor{currentstroke}%
\pgfsetdash{}{0pt}%
\pgfsys@defobject{currentmarker}{\pgfqpoint{0.000000in}{0.000000in}}{\pgfqpoint{0.055556in}{0.000000in}}{%
\pgfpathmoveto{\pgfqpoint{0.000000in}{0.000000in}}%
\pgfpathlineto{\pgfqpoint{0.055556in}{0.000000in}}%
\pgfusepath{stroke,fill}%
}%
\begin{pgfscope}%
\pgfsys@transformshift{0.697913in}{1.281895in}%
\pgfsys@useobject{currentmarker}{}%
\end{pgfscope}%
\end{pgfscope}%
\begin{pgfscope}%
\pgfsetbuttcap%
\pgfsetroundjoin%
\definecolor{currentfill}{rgb}{0.000000,0.000000,0.000000}%
\pgfsetfillcolor{currentfill}%
\pgfsetlinewidth{1.254687pt}%
\definecolor{currentstroke}{rgb}{0.000000,0.000000,0.000000}%
\pgfsetstrokecolor{currentstroke}%
\pgfsetdash{}{0pt}%
\pgfsys@defobject{currentmarker}{\pgfqpoint{0.000000in}{0.000000in}}{\pgfqpoint{0.055556in}{0.000000in}}{%
\pgfpathmoveto{\pgfqpoint{0.000000in}{0.000000in}}%
\pgfpathlineto{\pgfqpoint{0.055556in}{0.000000in}}%
\pgfusepath{stroke,fill}%
}%
\begin{pgfscope}%
\pgfsys@transformshift{0.697913in}{1.394822in}%
\pgfsys@useobject{currentmarker}{}%
\end{pgfscope}%
\end{pgfscope}%
\begin{pgfscope}%
\pgfsetbuttcap%
\pgfsetroundjoin%
\definecolor{currentfill}{rgb}{0.000000,0.000000,0.000000}%
\pgfsetfillcolor{currentfill}%
\pgfsetlinewidth{1.254687pt}%
\definecolor{currentstroke}{rgb}{0.000000,0.000000,0.000000}%
\pgfsetstrokecolor{currentstroke}%
\pgfsetdash{}{0pt}%
\pgfsys@defobject{currentmarker}{\pgfqpoint{0.000000in}{0.000000in}}{\pgfqpoint{0.055556in}{0.000000in}}{%
\pgfpathmoveto{\pgfqpoint{0.000000in}{0.000000in}}%
\pgfpathlineto{\pgfqpoint{0.055556in}{0.000000in}}%
\pgfusepath{stroke,fill}%
}%
\begin{pgfscope}%
\pgfsys@transformshift{0.697913in}{1.507750in}%
\pgfsys@useobject{currentmarker}{}%
\end{pgfscope}%
\end{pgfscope}%
\begin{pgfscope}%
\pgfsetbuttcap%
\pgfsetroundjoin%
\definecolor{currentfill}{rgb}{0.000000,0.000000,0.000000}%
\pgfsetfillcolor{currentfill}%
\pgfsetlinewidth{1.254687pt}%
\definecolor{currentstroke}{rgb}{0.000000,0.000000,0.000000}%
\pgfsetstrokecolor{currentstroke}%
\pgfsetdash{}{0pt}%
\pgfsys@defobject{currentmarker}{\pgfqpoint{0.000000in}{0.000000in}}{\pgfqpoint{0.055556in}{0.000000in}}{%
\pgfpathmoveto{\pgfqpoint{0.000000in}{0.000000in}}%
\pgfpathlineto{\pgfqpoint{0.055556in}{0.000000in}}%
\pgfusepath{stroke,fill}%
}%
\begin{pgfscope}%
\pgfsys@transformshift{0.697913in}{1.733606in}%
\pgfsys@useobject{currentmarker}{}%
\end{pgfscope}%
\end{pgfscope}%
\begin{pgfscope}%
\pgfsetbuttcap%
\pgfsetroundjoin%
\definecolor{currentfill}{rgb}{0.000000,0.000000,0.000000}%
\pgfsetfillcolor{currentfill}%
\pgfsetlinewidth{1.254687pt}%
\definecolor{currentstroke}{rgb}{0.000000,0.000000,0.000000}%
\pgfsetstrokecolor{currentstroke}%
\pgfsetdash{}{0pt}%
\pgfsys@defobject{currentmarker}{\pgfqpoint{0.000000in}{0.000000in}}{\pgfqpoint{0.055556in}{0.000000in}}{%
\pgfpathmoveto{\pgfqpoint{0.000000in}{0.000000in}}%
\pgfpathlineto{\pgfqpoint{0.055556in}{0.000000in}}%
\pgfusepath{stroke,fill}%
}%
\begin{pgfscope}%
\pgfsys@transformshift{0.697913in}{1.846533in}%
\pgfsys@useobject{currentmarker}{}%
\end{pgfscope}%
\end{pgfscope}%
\begin{pgfscope}%
\pgfsetbuttcap%
\pgfsetroundjoin%
\definecolor{currentfill}{rgb}{0.000000,0.000000,0.000000}%
\pgfsetfillcolor{currentfill}%
\pgfsetlinewidth{1.254687pt}%
\definecolor{currentstroke}{rgb}{0.000000,0.000000,0.000000}%
\pgfsetstrokecolor{currentstroke}%
\pgfsetdash{}{0pt}%
\pgfsys@defobject{currentmarker}{\pgfqpoint{0.000000in}{0.000000in}}{\pgfqpoint{0.055556in}{0.000000in}}{%
\pgfpathmoveto{\pgfqpoint{0.000000in}{0.000000in}}%
\pgfpathlineto{\pgfqpoint{0.055556in}{0.000000in}}%
\pgfusepath{stroke,fill}%
}%
\begin{pgfscope}%
\pgfsys@transformshift{0.697913in}{1.959461in}%
\pgfsys@useobject{currentmarker}{}%
\end{pgfscope}%
\end{pgfscope}%
\begin{pgfscope}%
\pgfsetbuttcap%
\pgfsetroundjoin%
\definecolor{currentfill}{rgb}{0.000000,0.000000,0.000000}%
\pgfsetfillcolor{currentfill}%
\pgfsetlinewidth{1.254687pt}%
\definecolor{currentstroke}{rgb}{0.000000,0.000000,0.000000}%
\pgfsetstrokecolor{currentstroke}%
\pgfsetdash{}{0pt}%
\pgfsys@defobject{currentmarker}{\pgfqpoint{0.000000in}{0.000000in}}{\pgfqpoint{0.055556in}{0.000000in}}{%
\pgfpathmoveto{\pgfqpoint{0.000000in}{0.000000in}}%
\pgfpathlineto{\pgfqpoint{0.055556in}{0.000000in}}%
\pgfusepath{stroke,fill}%
}%
\begin{pgfscope}%
\pgfsys@transformshift{0.697913in}{2.072389in}%
\pgfsys@useobject{currentmarker}{}%
\end{pgfscope}%
\end{pgfscope}%
\begin{pgfscope}%
\pgfsetbuttcap%
\pgfsetroundjoin%
\definecolor{currentfill}{rgb}{0.000000,0.000000,0.000000}%
\pgfsetfillcolor{currentfill}%
\pgfsetlinewidth{1.254687pt}%
\definecolor{currentstroke}{rgb}{0.000000,0.000000,0.000000}%
\pgfsetstrokecolor{currentstroke}%
\pgfsetdash{}{0pt}%
\pgfsys@defobject{currentmarker}{\pgfqpoint{0.000000in}{0.000000in}}{\pgfqpoint{0.055556in}{0.000000in}}{%
\pgfpathmoveto{\pgfqpoint{0.000000in}{0.000000in}}%
\pgfpathlineto{\pgfqpoint{0.055556in}{0.000000in}}%
\pgfusepath{stroke,fill}%
}%
\begin{pgfscope}%
\pgfsys@transformshift{0.697913in}{2.298245in}%
\pgfsys@useobject{currentmarker}{}%
\end{pgfscope}%
\end{pgfscope}%
\begin{pgfscope}%
\pgfsetbuttcap%
\pgfsetroundjoin%
\definecolor{currentfill}{rgb}{0.000000,0.000000,0.000000}%
\pgfsetfillcolor{currentfill}%
\pgfsetlinewidth{1.254687pt}%
\definecolor{currentstroke}{rgb}{0.000000,0.000000,0.000000}%
\pgfsetstrokecolor{currentstroke}%
\pgfsetdash{}{0pt}%
\pgfsys@defobject{currentmarker}{\pgfqpoint{0.000000in}{0.000000in}}{\pgfqpoint{0.055556in}{0.000000in}}{%
\pgfpathmoveto{\pgfqpoint{0.000000in}{0.000000in}}%
\pgfpathlineto{\pgfqpoint{0.055556in}{0.000000in}}%
\pgfusepath{stroke,fill}%
}%
\begin{pgfscope}%
\pgfsys@transformshift{0.697913in}{2.411172in}%
\pgfsys@useobject{currentmarker}{}%
\end{pgfscope}%
\end{pgfscope}%
\begin{pgfscope}%
\pgfsetbuttcap%
\pgfsetroundjoin%
\definecolor{currentfill}{rgb}{0.000000,0.000000,0.000000}%
\pgfsetfillcolor{currentfill}%
\pgfsetlinewidth{1.254687pt}%
\definecolor{currentstroke}{rgb}{0.000000,0.000000,0.000000}%
\pgfsetstrokecolor{currentstroke}%
\pgfsetdash{}{0pt}%
\pgfsys@defobject{currentmarker}{\pgfqpoint{0.000000in}{0.000000in}}{\pgfqpoint{0.055556in}{0.000000in}}{%
\pgfpathmoveto{\pgfqpoint{0.000000in}{0.000000in}}%
\pgfpathlineto{\pgfqpoint{0.055556in}{0.000000in}}%
\pgfusepath{stroke,fill}%
}%
\begin{pgfscope}%
\pgfsys@transformshift{0.697913in}{2.524100in}%
\pgfsys@useobject{currentmarker}{}%
\end{pgfscope}%
\end{pgfscope}%
\begin{pgfscope}%
\pgfsetbuttcap%
\pgfsetroundjoin%
\definecolor{currentfill}{rgb}{0.000000,0.000000,0.000000}%
\pgfsetfillcolor{currentfill}%
\pgfsetlinewidth{1.254687pt}%
\definecolor{currentstroke}{rgb}{0.000000,0.000000,0.000000}%
\pgfsetstrokecolor{currentstroke}%
\pgfsetdash{}{0pt}%
\pgfsys@defobject{currentmarker}{\pgfqpoint{0.000000in}{0.000000in}}{\pgfqpoint{0.055556in}{0.000000in}}{%
\pgfpathmoveto{\pgfqpoint{0.000000in}{0.000000in}}%
\pgfpathlineto{\pgfqpoint{0.055556in}{0.000000in}}%
\pgfusepath{stroke,fill}%
}%
\begin{pgfscope}%
\pgfsys@transformshift{0.697913in}{2.637028in}%
\pgfsys@useobject{currentmarker}{}%
\end{pgfscope}%
\end{pgfscope}%
\begin{pgfscope}%
\pgfsetbuttcap%
\pgfsetroundjoin%
\definecolor{currentfill}{rgb}{0.000000,0.000000,0.000000}%
\pgfsetfillcolor{currentfill}%
\pgfsetlinewidth{1.254687pt}%
\definecolor{currentstroke}{rgb}{0.000000,0.000000,0.000000}%
\pgfsetstrokecolor{currentstroke}%
\pgfsetdash{}{0pt}%
\pgfsys@defobject{currentmarker}{\pgfqpoint{0.000000in}{0.000000in}}{\pgfqpoint{0.055556in}{0.000000in}}{%
\pgfpathmoveto{\pgfqpoint{0.000000in}{0.000000in}}%
\pgfpathlineto{\pgfqpoint{0.055556in}{0.000000in}}%
\pgfusepath{stroke,fill}%
}%
\begin{pgfscope}%
\pgfsys@transformshift{0.697913in}{2.862883in}%
\pgfsys@useobject{currentmarker}{}%
\end{pgfscope}%
\end{pgfscope}%
\begin{pgfscope}%
\pgfsetbuttcap%
\pgfsetroundjoin%
\definecolor{currentfill}{rgb}{0.000000,0.000000,0.000000}%
\pgfsetfillcolor{currentfill}%
\pgfsetlinewidth{1.254687pt}%
\definecolor{currentstroke}{rgb}{0.000000,0.000000,0.000000}%
\pgfsetstrokecolor{currentstroke}%
\pgfsetdash{}{0pt}%
\pgfsys@defobject{currentmarker}{\pgfqpoint{0.000000in}{0.000000in}}{\pgfqpoint{0.055556in}{0.000000in}}{%
\pgfpathmoveto{\pgfqpoint{0.000000in}{0.000000in}}%
\pgfpathlineto{\pgfqpoint{0.055556in}{0.000000in}}%
\pgfusepath{stroke,fill}%
}%
\begin{pgfscope}%
\pgfsys@transformshift{0.697913in}{2.975811in}%
\pgfsys@useobject{currentmarker}{}%
\end{pgfscope}%
\end{pgfscope}%
\begin{pgfscope}%
\pgfsetbuttcap%
\pgfsetroundjoin%
\definecolor{currentfill}{rgb}{0.000000,0.000000,0.000000}%
\pgfsetfillcolor{currentfill}%
\pgfsetlinewidth{1.254687pt}%
\definecolor{currentstroke}{rgb}{0.000000,0.000000,0.000000}%
\pgfsetstrokecolor{currentstroke}%
\pgfsetdash{}{0pt}%
\pgfsys@defobject{currentmarker}{\pgfqpoint{0.000000in}{0.000000in}}{\pgfqpoint{0.055556in}{0.000000in}}{%
\pgfpathmoveto{\pgfqpoint{0.000000in}{0.000000in}}%
\pgfpathlineto{\pgfqpoint{0.055556in}{0.000000in}}%
\pgfusepath{stroke,fill}%
}%
\begin{pgfscope}%
\pgfsys@transformshift{0.697913in}{3.088739in}%
\pgfsys@useobject{currentmarker}{}%
\end{pgfscope}%
\end{pgfscope}%
\begin{pgfscope}%
\pgfsetbuttcap%
\pgfsetroundjoin%
\definecolor{currentfill}{rgb}{0.000000,0.000000,0.000000}%
\pgfsetfillcolor{currentfill}%
\pgfsetlinewidth{1.254687pt}%
\definecolor{currentstroke}{rgb}{0.000000,0.000000,0.000000}%
\pgfsetstrokecolor{currentstroke}%
\pgfsetdash{}{0pt}%
\pgfsys@defobject{currentmarker}{\pgfqpoint{0.000000in}{0.000000in}}{\pgfqpoint{0.055556in}{0.000000in}}{%
\pgfpathmoveto{\pgfqpoint{0.000000in}{0.000000in}}%
\pgfpathlineto{\pgfqpoint{0.055556in}{0.000000in}}%
\pgfusepath{stroke,fill}%
}%
\begin{pgfscope}%
\pgfsys@transformshift{0.697913in}{3.201667in}%
\pgfsys@useobject{currentmarker}{}%
\end{pgfscope}%
\end{pgfscope}%
\begin{pgfscope}%
\pgfsetbuttcap%
\pgfsetroundjoin%
\definecolor{currentfill}{rgb}{0.000000,0.000000,0.000000}%
\pgfsetfillcolor{currentfill}%
\pgfsetlinewidth{1.254687pt}%
\definecolor{currentstroke}{rgb}{0.000000,0.000000,0.000000}%
\pgfsetstrokecolor{currentstroke}%
\pgfsetdash{}{0pt}%
\pgfsys@defobject{currentmarker}{\pgfqpoint{0.000000in}{0.000000in}}{\pgfqpoint{0.055556in}{0.000000in}}{%
\pgfpathmoveto{\pgfqpoint{0.000000in}{0.000000in}}%
\pgfpathlineto{\pgfqpoint{0.055556in}{0.000000in}}%
\pgfusepath{stroke,fill}%
}%
\begin{pgfscope}%
\pgfsys@transformshift{0.697913in}{3.427522in}%
\pgfsys@useobject{currentmarker}{}%
\end{pgfscope}%
\end{pgfscope}%
\begin{pgfscope}%
\pgfsetbuttcap%
\pgfsetroundjoin%
\definecolor{currentfill}{rgb}{0.000000,0.000000,0.000000}%
\pgfsetfillcolor{currentfill}%
\pgfsetlinewidth{1.254687pt}%
\definecolor{currentstroke}{rgb}{0.000000,0.000000,0.000000}%
\pgfsetstrokecolor{currentstroke}%
\pgfsetdash{}{0pt}%
\pgfsys@defobject{currentmarker}{\pgfqpoint{0.000000in}{0.000000in}}{\pgfqpoint{0.055556in}{0.000000in}}{%
\pgfpathmoveto{\pgfqpoint{0.000000in}{0.000000in}}%
\pgfpathlineto{\pgfqpoint{0.055556in}{0.000000in}}%
\pgfusepath{stroke,fill}%
}%
\begin{pgfscope}%
\pgfsys@transformshift{0.697913in}{3.540450in}%
\pgfsys@useobject{currentmarker}{}%
\end{pgfscope}%
\end{pgfscope}%
\begin{pgfscope}%
\pgfsetbuttcap%
\pgfsetroundjoin%
\definecolor{currentfill}{rgb}{0.000000,0.000000,0.000000}%
\pgfsetfillcolor{currentfill}%
\pgfsetlinewidth{1.254687pt}%
\definecolor{currentstroke}{rgb}{0.000000,0.000000,0.000000}%
\pgfsetstrokecolor{currentstroke}%
\pgfsetdash{}{0pt}%
\pgfsys@defobject{currentmarker}{\pgfqpoint{0.000000in}{0.000000in}}{\pgfqpoint{0.055556in}{0.000000in}}{%
\pgfpathmoveto{\pgfqpoint{0.000000in}{0.000000in}}%
\pgfpathlineto{\pgfqpoint{0.055556in}{0.000000in}}%
\pgfusepath{stroke,fill}%
}%
\begin{pgfscope}%
\pgfsys@transformshift{0.697913in}{3.653378in}%
\pgfsys@useobject{currentmarker}{}%
\end{pgfscope}%
\end{pgfscope}%
\begin{pgfscope}%
\pgfsetbuttcap%
\pgfsetroundjoin%
\definecolor{currentfill}{rgb}{0.000000,0.000000,0.000000}%
\pgfsetfillcolor{currentfill}%
\pgfsetlinewidth{1.254687pt}%
\definecolor{currentstroke}{rgb}{0.000000,0.000000,0.000000}%
\pgfsetstrokecolor{currentstroke}%
\pgfsetdash{}{0pt}%
\pgfsys@defobject{currentmarker}{\pgfqpoint{0.000000in}{0.000000in}}{\pgfqpoint{0.055556in}{0.000000in}}{%
\pgfpathmoveto{\pgfqpoint{0.000000in}{0.000000in}}%
\pgfpathlineto{\pgfqpoint{0.055556in}{0.000000in}}%
\pgfusepath{stroke,fill}%
}%
\begin{pgfscope}%
\pgfsys@transformshift{0.697913in}{3.766306in}%
\pgfsys@useobject{currentmarker}{}%
\end{pgfscope}%
\end{pgfscope}%
\begin{pgfscope}%
\pgfsetbuttcap%
\pgfsetroundjoin%
\definecolor{currentfill}{rgb}{0.000000,0.000000,0.000000}%
\pgfsetfillcolor{currentfill}%
\pgfsetlinewidth{1.254687pt}%
\definecolor{currentstroke}{rgb}{0.000000,0.000000,0.000000}%
\pgfsetstrokecolor{currentstroke}%
\pgfsetdash{}{0pt}%
\pgfsys@defobject{currentmarker}{\pgfqpoint{0.000000in}{0.000000in}}{\pgfqpoint{0.055556in}{0.000000in}}{%
\pgfpathmoveto{\pgfqpoint{0.000000in}{0.000000in}}%
\pgfpathlineto{\pgfqpoint{0.055556in}{0.000000in}}%
\pgfusepath{stroke,fill}%
}%
\begin{pgfscope}%
\pgfsys@transformshift{0.697913in}{3.992161in}%
\pgfsys@useobject{currentmarker}{}%
\end{pgfscope}%
\end{pgfscope}%
\begin{pgfscope}%
\pgfsetbuttcap%
\pgfsetroundjoin%
\definecolor{currentfill}{rgb}{0.000000,0.000000,0.000000}%
\pgfsetfillcolor{currentfill}%
\pgfsetlinewidth{1.254687pt}%
\definecolor{currentstroke}{rgb}{0.000000,0.000000,0.000000}%
\pgfsetstrokecolor{currentstroke}%
\pgfsetdash{}{0pt}%
\pgfsys@defobject{currentmarker}{\pgfqpoint{0.000000in}{0.000000in}}{\pgfqpoint{0.055556in}{0.000000in}}{%
\pgfpathmoveto{\pgfqpoint{0.000000in}{0.000000in}}%
\pgfpathlineto{\pgfqpoint{0.055556in}{0.000000in}}%
\pgfusepath{stroke,fill}%
}%
\begin{pgfscope}%
\pgfsys@transformshift{0.697913in}{4.105089in}%
\pgfsys@useobject{currentmarker}{}%
\end{pgfscope}%
\end{pgfscope}%
\begin{pgfscope}%
\pgfsetbuttcap%
\pgfsetroundjoin%
\definecolor{currentfill}{rgb}{0.000000,0.000000,0.000000}%
\pgfsetfillcolor{currentfill}%
\pgfsetlinewidth{1.254687pt}%
\definecolor{currentstroke}{rgb}{0.000000,0.000000,0.000000}%
\pgfsetstrokecolor{currentstroke}%
\pgfsetdash{}{0pt}%
\pgfsys@defobject{currentmarker}{\pgfqpoint{0.000000in}{0.000000in}}{\pgfqpoint{0.055556in}{0.000000in}}{%
\pgfpathmoveto{\pgfqpoint{0.000000in}{0.000000in}}%
\pgfpathlineto{\pgfqpoint{0.055556in}{0.000000in}}%
\pgfusepath{stroke,fill}%
}%
\begin{pgfscope}%
\pgfsys@transformshift{0.697913in}{4.218017in}%
\pgfsys@useobject{currentmarker}{}%
\end{pgfscope}%
\end{pgfscope}%
\begin{pgfscope}%
\pgfsetbuttcap%
\pgfsetroundjoin%
\definecolor{currentfill}{rgb}{0.000000,0.000000,0.000000}%
\pgfsetfillcolor{currentfill}%
\pgfsetlinewidth{1.254687pt}%
\definecolor{currentstroke}{rgb}{0.000000,0.000000,0.000000}%
\pgfsetstrokecolor{currentstroke}%
\pgfsetdash{}{0pt}%
\pgfsys@defobject{currentmarker}{\pgfqpoint{0.000000in}{0.000000in}}{\pgfqpoint{0.055556in}{0.000000in}}{%
\pgfpathmoveto{\pgfqpoint{0.000000in}{0.000000in}}%
\pgfpathlineto{\pgfqpoint{0.055556in}{0.000000in}}%
\pgfusepath{stroke,fill}%
}%
\begin{pgfscope}%
\pgfsys@transformshift{0.697913in}{4.330945in}%
\pgfsys@useobject{currentmarker}{}%
\end{pgfscope}%
\end{pgfscope}%
\begin{pgfscope}%
\pgfsetbuttcap%
\pgfsetroundjoin%
\definecolor{currentfill}{rgb}{0.000000,0.000000,0.000000}%
\pgfsetfillcolor{currentfill}%
\pgfsetlinewidth{1.254687pt}%
\definecolor{currentstroke}{rgb}{0.000000,0.000000,0.000000}%
\pgfsetstrokecolor{currentstroke}%
\pgfsetdash{}{0pt}%
\pgfsys@defobject{currentmarker}{\pgfqpoint{0.000000in}{0.000000in}}{\pgfqpoint{0.055556in}{0.000000in}}{%
\pgfpathmoveto{\pgfqpoint{0.000000in}{0.000000in}}%
\pgfpathlineto{\pgfqpoint{0.055556in}{0.000000in}}%
\pgfusepath{stroke,fill}%
}%
\begin{pgfscope}%
\pgfsys@transformshift{0.697913in}{4.556800in}%
\pgfsys@useobject{currentmarker}{}%
\end{pgfscope}%
\end{pgfscope}%
\begin{pgfscope}%
\pgfsetbuttcap%
\pgfsetroundjoin%
\definecolor{currentfill}{rgb}{0.000000,0.000000,0.000000}%
\pgfsetfillcolor{currentfill}%
\pgfsetlinewidth{1.254687pt}%
\definecolor{currentstroke}{rgb}{0.000000,0.000000,0.000000}%
\pgfsetstrokecolor{currentstroke}%
\pgfsetdash{}{0pt}%
\pgfsys@defobject{currentmarker}{\pgfqpoint{0.000000in}{0.000000in}}{\pgfqpoint{0.055556in}{0.000000in}}{%
\pgfpathmoveto{\pgfqpoint{0.000000in}{0.000000in}}%
\pgfpathlineto{\pgfqpoint{0.055556in}{0.000000in}}%
\pgfusepath{stroke,fill}%
}%
\begin{pgfscope}%
\pgfsys@transformshift{0.697913in}{4.669728in}%
\pgfsys@useobject{currentmarker}{}%
\end{pgfscope}%
\end{pgfscope}%
\begin{pgfscope}%
\pgfsetbuttcap%
\pgfsetroundjoin%
\definecolor{currentfill}{rgb}{0.000000,0.000000,0.000000}%
\pgfsetfillcolor{currentfill}%
\pgfsetlinewidth{1.254687pt}%
\definecolor{currentstroke}{rgb}{0.000000,0.000000,0.000000}%
\pgfsetstrokecolor{currentstroke}%
\pgfsetdash{}{0pt}%
\pgfsys@defobject{currentmarker}{\pgfqpoint{0.000000in}{0.000000in}}{\pgfqpoint{0.055556in}{0.000000in}}{%
\pgfpathmoveto{\pgfqpoint{0.000000in}{0.000000in}}%
\pgfpathlineto{\pgfqpoint{0.055556in}{0.000000in}}%
\pgfusepath{stroke,fill}%
}%
\begin{pgfscope}%
\pgfsys@transformshift{0.697913in}{4.782656in}%
\pgfsys@useobject{currentmarker}{}%
\end{pgfscope}%
\end{pgfscope}%
\begin{pgfscope}%
\pgfsetbuttcap%
\pgfsetroundjoin%
\definecolor{currentfill}{rgb}{0.000000,0.000000,0.000000}%
\pgfsetfillcolor{currentfill}%
\pgfsetlinewidth{1.254687pt}%
\definecolor{currentstroke}{rgb}{0.000000,0.000000,0.000000}%
\pgfsetstrokecolor{currentstroke}%
\pgfsetdash{}{0pt}%
\pgfsys@defobject{currentmarker}{\pgfqpoint{0.000000in}{0.000000in}}{\pgfqpoint{0.055556in}{0.000000in}}{%
\pgfpathmoveto{\pgfqpoint{0.000000in}{0.000000in}}%
\pgfpathlineto{\pgfqpoint{0.055556in}{0.000000in}}%
\pgfusepath{stroke,fill}%
}%
\begin{pgfscope}%
\pgfsys@transformshift{0.697913in}{4.895584in}%
\pgfsys@useobject{currentmarker}{}%
\end{pgfscope}%
\end{pgfscope}%
\begin{pgfscope}%
\pgfsetbuttcap%
\pgfsetroundjoin%
\definecolor{currentfill}{rgb}{0.000000,0.000000,0.000000}%
\pgfsetfillcolor{currentfill}%
\pgfsetlinewidth{1.254687pt}%
\definecolor{currentstroke}{rgb}{0.000000,0.000000,0.000000}%
\pgfsetstrokecolor{currentstroke}%
\pgfsetdash{}{0pt}%
\pgfsys@defobject{currentmarker}{\pgfqpoint{0.000000in}{0.000000in}}{\pgfqpoint{0.055556in}{0.000000in}}{%
\pgfpathmoveto{\pgfqpoint{0.000000in}{0.000000in}}%
\pgfpathlineto{\pgfqpoint{0.055556in}{0.000000in}}%
\pgfusepath{stroke,fill}%
}%
\begin{pgfscope}%
\pgfsys@transformshift{0.697913in}{5.121439in}%
\pgfsys@useobject{currentmarker}{}%
\end{pgfscope}%
\end{pgfscope}%
\begin{pgfscope}%
\pgfsetbuttcap%
\pgfsetroundjoin%
\definecolor{currentfill}{rgb}{0.000000,0.000000,0.000000}%
\pgfsetfillcolor{currentfill}%
\pgfsetlinewidth{1.254687pt}%
\definecolor{currentstroke}{rgb}{0.000000,0.000000,0.000000}%
\pgfsetstrokecolor{currentstroke}%
\pgfsetdash{}{0pt}%
\pgfsys@defobject{currentmarker}{\pgfqpoint{0.000000in}{0.000000in}}{\pgfqpoint{0.055556in}{0.000000in}}{%
\pgfpathmoveto{\pgfqpoint{0.000000in}{0.000000in}}%
\pgfpathlineto{\pgfqpoint{0.055556in}{0.000000in}}%
\pgfusepath{stroke,fill}%
}%
\begin{pgfscope}%
\pgfsys@transformshift{0.697913in}{5.234367in}%
\pgfsys@useobject{currentmarker}{}%
\end{pgfscope}%
\end{pgfscope}%
\begin{pgfscope}%
\pgfsetbuttcap%
\pgfsetroundjoin%
\definecolor{currentfill}{rgb}{0.000000,0.000000,0.000000}%
\pgfsetfillcolor{currentfill}%
\pgfsetlinewidth{1.254687pt}%
\definecolor{currentstroke}{rgb}{0.000000,0.000000,0.000000}%
\pgfsetstrokecolor{currentstroke}%
\pgfsetdash{}{0pt}%
\pgfsys@defobject{currentmarker}{\pgfqpoint{0.000000in}{0.000000in}}{\pgfqpoint{0.055556in}{0.000000in}}{%
\pgfpathmoveto{\pgfqpoint{0.000000in}{0.000000in}}%
\pgfpathlineto{\pgfqpoint{0.055556in}{0.000000in}}%
\pgfusepath{stroke,fill}%
}%
\begin{pgfscope}%
\pgfsys@transformshift{0.697913in}{5.347295in}%
\pgfsys@useobject{currentmarker}{}%
\end{pgfscope}%
\end{pgfscope}%
\begin{pgfscope}%
\pgfsetbuttcap%
\pgfsetroundjoin%
\definecolor{currentfill}{rgb}{0.000000,0.000000,0.000000}%
\pgfsetfillcolor{currentfill}%
\pgfsetlinewidth{1.254687pt}%
\definecolor{currentstroke}{rgb}{0.000000,0.000000,0.000000}%
\pgfsetstrokecolor{currentstroke}%
\pgfsetdash{}{0pt}%
\pgfsys@defobject{currentmarker}{\pgfqpoint{0.000000in}{0.000000in}}{\pgfqpoint{0.055556in}{0.000000in}}{%
\pgfpathmoveto{\pgfqpoint{0.000000in}{0.000000in}}%
\pgfpathlineto{\pgfqpoint{0.055556in}{0.000000in}}%
\pgfusepath{stroke,fill}%
}%
\begin{pgfscope}%
\pgfsys@transformshift{0.697913in}{5.460222in}%
\pgfsys@useobject{currentmarker}{}%
\end{pgfscope}%
\end{pgfscope}%
\begin{pgfscope}%
\definecolor{textcolor}{rgb}{0.000000,0.000000,0.000000}%
\pgfsetstrokecolor{textcolor}%
\pgfsetfillcolor{textcolor}%
\pgftext[x=0.300000in,y=3.054861in,,bottom,rotate=90.000000]{\color{textcolor}{\rmfamily\fontsize{14.000000}{16.800000}\selectfont\catcode`\^=\active\def^{\ifmmode\sp\else\^{}\fi}\catcode`\%=\active\def%{\%}parts per million (ppm)}}%
\end{pgfscope}%
\begin{pgfscope}%
\pgfpathrectangle{\pgfqpoint{0.697913in}{0.559721in}}{\pgfqpoint{7.048636in}{4.990279in}}%
\pgfusepath{clip}%
\pgfsetrectcap%
\pgfsetroundjoin%
\pgfsetlinewidth{1.003750pt}%
\definecolor{currentstroke}{rgb}{1.000000,0.000000,0.000000}%
\pgfsetstrokecolor{currentstroke}%
\pgfsetdash{}{0pt}%
\pgfpathmoveto{\pgfqpoint{0.697913in}{0.800800in}}%
\pgfpathlineto{\pgfqpoint{0.712628in}{0.854764in}}%
\pgfpathlineto{\pgfqpoint{0.727343in}{0.933745in}}%
\pgfpathlineto{\pgfqpoint{0.742059in}{0.977716in}}%
\pgfpathlineto{\pgfqpoint{0.756774in}{1.015272in}}%
\pgfpathlineto{\pgfqpoint{0.771489in}{0.976371in}}%
\pgfpathlineto{\pgfqpoint{0.786205in}{0.886780in}}%
\pgfpathlineto{\pgfqpoint{0.800920in}{0.775096in}}%
\pgfpathlineto{\pgfqpoint{0.815635in}{0.687494in}}%
\pgfpathlineto{\pgfqpoint{0.830350in}{0.682107in}}%
\pgfpathlineto{\pgfqpoint{0.859781in}{0.838848in}}%
\pgfpathlineto{\pgfqpoint{0.874496in}{0.875106in}}%
\pgfpathlineto{\pgfqpoint{0.889212in}{0.913637in}}%
\pgfpathlineto{\pgfqpoint{0.903927in}{0.955498in}}%
\pgfpathlineto{\pgfqpoint{0.918642in}{1.048312in}}%
\pgfpathlineto{\pgfqpoint{0.933358in}{1.085369in}}%
\pgfpathlineto{\pgfqpoint{0.948073in}{1.055640in}}%
\pgfpathlineto{\pgfqpoint{0.962788in}{0.959881in}}%
\pgfpathlineto{\pgfqpoint{0.977504in}{0.841100in}}%
\pgfpathlineto{\pgfqpoint{0.992219in}{0.795475in}}%
\pgfpathlineto{\pgfqpoint{1.006934in}{0.760407in}}%
\pgfpathlineto{\pgfqpoint{1.021650in}{0.844761in}}%
\pgfpathlineto{\pgfqpoint{1.036365in}{0.909468in}}%
\pgfpathlineto{\pgfqpoint{1.051080in}{0.982320in}}%
\pgfpathlineto{\pgfqpoint{1.065796in}{0.994584in}}%
\pgfpathlineto{\pgfqpoint{1.080511in}{1.044934in}}%
\pgfpathlineto{\pgfqpoint{1.095226in}{1.136847in}}%
\pgfpathlineto{\pgfqpoint{1.109941in}{1.177436in}}%
\pgfpathlineto{\pgfqpoint{1.124657in}{1.150975in}}%
\pgfpathlineto{\pgfqpoint{1.139372in}{1.075536in}}%
\pgfpathlineto{\pgfqpoint{1.154087in}{0.954253in}}%
\pgfpathlineto{\pgfqpoint{1.168803in}{0.864921in}}%
\pgfpathlineto{\pgfqpoint{1.183518in}{0.873943in}}%
\pgfpathlineto{\pgfqpoint{1.198233in}{0.947407in}}%
\pgfpathlineto{\pgfqpoint{1.212949in}{1.016117in}}%
\pgfpathlineto{\pgfqpoint{1.227664in}{1.081354in}}%
\pgfpathlineto{\pgfqpoint{1.242379in}{1.154139in}}%
\pgfpathlineto{\pgfqpoint{1.257095in}{1.194737in}}%
\pgfpathlineto{\pgfqpoint{1.271810in}{1.262801in}}%
\pgfpathlineto{\pgfqpoint{1.286525in}{1.301798in}}%
\pgfpathlineto{\pgfqpoint{1.301241in}{1.271334in}}%
\pgfpathlineto{\pgfqpoint{1.315956in}{1.212318in}}%
\pgfpathlineto{\pgfqpoint{1.330671in}{1.080731in}}%
\pgfpathlineto{\pgfqpoint{1.345387in}{0.998402in}}%
\pgfpathlineto{\pgfqpoint{1.360102in}{1.021379in}}%
\pgfpathlineto{\pgfqpoint{1.374817in}{1.082667in}}%
\pgfpathlineto{\pgfqpoint{1.389532in}{1.146865in}}%
\pgfpathlineto{\pgfqpoint{1.404248in}{1.228919in}}%
\pgfpathlineto{\pgfqpoint{1.418963in}{1.249078in}}%
\pgfpathlineto{\pgfqpoint{1.433678in}{1.285964in}}%
\pgfpathlineto{\pgfqpoint{1.448394in}{1.379033in}}%
\pgfpathlineto{\pgfqpoint{1.463109in}{1.392104in}}%
\pgfpathlineto{\pgfqpoint{1.477824in}{1.362008in}}%
\pgfpathlineto{\pgfqpoint{1.492540in}{1.284978in}}%
\pgfpathlineto{\pgfqpoint{1.507255in}{1.162372in}}%
\pgfpathlineto{\pgfqpoint{1.521970in}{1.062532in}}%
\pgfpathlineto{\pgfqpoint{1.536686in}{1.076569in}}%
\pgfpathlineto{\pgfqpoint{1.551401in}{1.145336in}}%
\pgfpathlineto{\pgfqpoint{1.566116in}{1.217965in}}%
\pgfpathlineto{\pgfqpoint{1.580832in}{1.274509in}}%
\pgfpathlineto{\pgfqpoint{1.595547in}{1.343743in}}%
\pgfpathlineto{\pgfqpoint{1.610262in}{1.381731in}}%
\pgfpathlineto{\pgfqpoint{1.624978in}{1.415855in}}%
\pgfpathlineto{\pgfqpoint{1.639693in}{1.472501in}}%
\pgfpathlineto{\pgfqpoint{1.654408in}{1.417875in}}%
\pgfpathlineto{\pgfqpoint{1.669124in}{1.332844in}}%
\pgfpathlineto{\pgfqpoint{1.683839in}{1.227407in}}%
\pgfpathlineto{\pgfqpoint{1.698554in}{1.132997in}}%
\pgfpathlineto{\pgfqpoint{1.713269in}{1.152108in}}%
\pgfpathlineto{\pgfqpoint{1.727985in}{1.230512in}}%
\pgfpathlineto{\pgfqpoint{1.742700in}{1.306255in}}%
\pgfpathlineto{\pgfqpoint{1.757415in}{1.334366in}}%
\pgfpathlineto{\pgfqpoint{1.772131in}{1.384833in}}%
\pgfpathlineto{\pgfqpoint{1.786846in}{1.470691in}}%
\pgfpathlineto{\pgfqpoint{1.801561in}{1.551077in}}%
\pgfpathlineto{\pgfqpoint{1.816277in}{1.577512in}}%
\pgfpathlineto{\pgfqpoint{1.830992in}{1.520951in}}%
\pgfpathlineto{\pgfqpoint{1.845707in}{1.412191in}}%
\pgfpathlineto{\pgfqpoint{1.860423in}{1.272178in}}%
\pgfpathlineto{\pgfqpoint{1.875138in}{1.187642in}}%
\pgfpathlineto{\pgfqpoint{1.889853in}{1.192263in}}%
\pgfpathlineto{\pgfqpoint{1.904569in}{1.275825in}}%
\pgfpathlineto{\pgfqpoint{1.919284in}{1.351213in}}%
\pgfpathlineto{\pgfqpoint{1.933999in}{1.414220in}}%
\pgfpathlineto{\pgfqpoint{1.948715in}{1.463436in}}%
\pgfpathlineto{\pgfqpoint{1.963430in}{1.504903in}}%
\pgfpathlineto{\pgfqpoint{1.978145in}{1.576992in}}%
\pgfpathlineto{\pgfqpoint{1.992860in}{1.604477in}}%
\pgfpathlineto{\pgfqpoint{2.007576in}{1.588776in}}%
\pgfpathlineto{\pgfqpoint{2.022291in}{1.459711in}}%
\pgfpathlineto{\pgfqpoint{2.051722in}{1.224934in}}%
\pgfpathlineto{\pgfqpoint{2.066437in}{1.249769in}}%
\pgfpathlineto{\pgfqpoint{2.081152in}{1.304480in}}%
\pgfpathlineto{\pgfqpoint{2.095868in}{1.376535in}}%
\pgfpathlineto{\pgfqpoint{2.110583in}{1.457719in}}%
\pgfpathlineto{\pgfqpoint{2.125298in}{1.474739in}}%
\pgfpathlineto{\pgfqpoint{2.140014in}{1.541798in}}%
\pgfpathlineto{\pgfqpoint{2.154729in}{1.586777in}}%
\pgfpathlineto{\pgfqpoint{2.169444in}{1.637899in}}%
\pgfpathlineto{\pgfqpoint{2.184160in}{1.600492in}}%
\pgfpathlineto{\pgfqpoint{2.198875in}{1.479179in}}%
\pgfpathlineto{\pgfqpoint{2.213590in}{1.382317in}}%
\pgfpathlineto{\pgfqpoint{2.228306in}{1.288327in}}%
\pgfpathlineto{\pgfqpoint{2.243021in}{1.296898in}}%
\pgfpathlineto{\pgfqpoint{2.257736in}{1.367934in}}%
\pgfpathlineto{\pgfqpoint{2.272451in}{1.453039in}}%
\pgfpathlineto{\pgfqpoint{2.287167in}{1.525275in}}%
\pgfpathlineto{\pgfqpoint{2.301882in}{1.567715in}}%
\pgfpathlineto{\pgfqpoint{2.316597in}{1.626149in}}%
\pgfpathlineto{\pgfqpoint{2.331313in}{1.699022in}}%
\pgfpathlineto{\pgfqpoint{2.346028in}{1.721353in}}%
\pgfpathlineto{\pgfqpoint{2.360743in}{1.675260in}}%
\pgfpathlineto{\pgfqpoint{2.375459in}{1.593211in}}%
\pgfpathlineto{\pgfqpoint{2.390174in}{1.487188in}}%
\pgfpathlineto{\pgfqpoint{2.404889in}{1.387600in}}%
\pgfpathlineto{\pgfqpoint{2.419605in}{1.404099in}}%
\pgfpathlineto{\pgfqpoint{2.434320in}{1.489077in}}%
\pgfpathlineto{\pgfqpoint{2.449035in}{1.568045in}}%
\pgfpathlineto{\pgfqpoint{2.463751in}{1.623407in}}%
\pgfpathlineto{\pgfqpoint{2.493181in}{1.731250in}}%
\pgfpathlineto{\pgfqpoint{2.507897in}{1.813823in}}%
\pgfpathlineto{\pgfqpoint{2.522612in}{1.836876in}}%
\pgfpathlineto{\pgfqpoint{2.537327in}{1.809246in}}%
\pgfpathlineto{\pgfqpoint{2.552042in}{1.717090in}}%
\pgfpathlineto{\pgfqpoint{2.566758in}{1.583482in}}%
\pgfpathlineto{\pgfqpoint{2.581473in}{1.523889in}}%
\pgfpathlineto{\pgfqpoint{2.596188in}{1.515421in}}%
\pgfpathlineto{\pgfqpoint{2.610904in}{1.599895in}}%
\pgfpathlineto{\pgfqpoint{2.625619in}{1.668202in}}%
\pgfpathlineto{\pgfqpoint{2.640334in}{1.744451in}}%
\pgfpathlineto{\pgfqpoint{2.655050in}{1.811203in}}%
\pgfpathlineto{\pgfqpoint{2.669765in}{1.861214in}}%
\pgfpathlineto{\pgfqpoint{2.684480in}{1.886467in}}%
\pgfpathlineto{\pgfqpoint{2.699196in}{1.917320in}}%
\pgfpathlineto{\pgfqpoint{2.713911in}{1.906216in}}%
\pgfpathlineto{\pgfqpoint{2.728626in}{1.828829in}}%
\pgfpathlineto{\pgfqpoint{2.743342in}{1.708741in}}%
\pgfpathlineto{\pgfqpoint{2.758057in}{1.605004in}}%
\pgfpathlineto{\pgfqpoint{2.772772in}{1.606250in}}%
\pgfpathlineto{\pgfqpoint{2.787488in}{1.680121in}}%
\pgfpathlineto{\pgfqpoint{2.802203in}{1.757359in}}%
\pgfpathlineto{\pgfqpoint{2.816918in}{1.803530in}}%
\pgfpathlineto{\pgfqpoint{2.831633in}{1.858169in}}%
\pgfpathlineto{\pgfqpoint{2.846349in}{1.882998in}}%
\pgfpathlineto{\pgfqpoint{2.861064in}{1.988554in}}%
\pgfpathlineto{\pgfqpoint{2.875779in}{2.003172in}}%
\pgfpathlineto{\pgfqpoint{2.890495in}{1.940849in}}%
\pgfpathlineto{\pgfqpoint{2.905210in}{1.874601in}}%
\pgfpathlineto{\pgfqpoint{2.919925in}{1.757569in}}%
\pgfpathlineto{\pgfqpoint{2.934641in}{1.646651in}}%
\pgfpathlineto{\pgfqpoint{2.949356in}{1.676619in}}%
\pgfpathlineto{\pgfqpoint{2.964071in}{1.770213in}}%
\pgfpathlineto{\pgfqpoint{2.978787in}{1.875706in}}%
\pgfpathlineto{\pgfqpoint{2.993502in}{1.924661in}}%
\pgfpathlineto{\pgfqpoint{3.008217in}{1.965370in}}%
\pgfpathlineto{\pgfqpoint{3.022933in}{2.036452in}}%
\pgfpathlineto{\pgfqpoint{3.037648in}{2.116879in}}%
\pgfpathlineto{\pgfqpoint{3.052363in}{2.160322in}}%
\pgfpathlineto{\pgfqpoint{3.067079in}{2.135205in}}%
\pgfpathlineto{\pgfqpoint{3.081794in}{2.071972in}}%
\pgfpathlineto{\pgfqpoint{3.096509in}{1.966858in}}%
\pgfpathlineto{\pgfqpoint{3.111224in}{1.856132in}}%
\pgfpathlineto{\pgfqpoint{3.125940in}{1.876873in}}%
\pgfpathlineto{\pgfqpoint{3.140655in}{1.939208in}}%
\pgfpathlineto{\pgfqpoint{3.155370in}{2.035861in}}%
\pgfpathlineto{\pgfqpoint{3.170086in}{2.090478in}}%
\pgfpathlineto{\pgfqpoint{3.184801in}{2.140813in}}%
\pgfpathlineto{\pgfqpoint{3.199516in}{2.177751in}}%
\pgfpathlineto{\pgfqpoint{3.214232in}{2.251224in}}%
\pgfpathlineto{\pgfqpoint{3.228947in}{2.248600in}}%
\pgfpathlineto{\pgfqpoint{3.243662in}{2.213266in}}%
\pgfpathlineto{\pgfqpoint{3.258378in}{2.160034in}}%
\pgfpathlineto{\pgfqpoint{3.273093in}{2.018952in}}%
\pgfpathlineto{\pgfqpoint{3.287808in}{1.899710in}}%
\pgfpathlineto{\pgfqpoint{3.302524in}{1.932267in}}%
\pgfpathlineto{\pgfqpoint{3.317239in}{2.008121in}}%
\pgfpathlineto{\pgfqpoint{3.331954in}{2.087569in}}%
\pgfpathlineto{\pgfqpoint{3.346670in}{2.152937in}}%
\pgfpathlineto{\pgfqpoint{3.361385in}{2.171914in}}%
\pgfpathlineto{\pgfqpoint{3.376100in}{2.227382in}}%
\pgfpathlineto{\pgfqpoint{3.390815in}{2.296237in}}%
\pgfpathlineto{\pgfqpoint{3.405531in}{2.284008in}}%
\pgfpathlineto{\pgfqpoint{3.420246in}{2.290340in}}%
\pgfpathlineto{\pgfqpoint{3.434961in}{2.189405in}}%
\pgfpathlineto{\pgfqpoint{3.449677in}{2.092988in}}%
\pgfpathlineto{\pgfqpoint{3.464392in}{2.024112in}}%
\pgfpathlineto{\pgfqpoint{3.479107in}{2.025919in}}%
\pgfpathlineto{\pgfqpoint{3.493823in}{2.102225in}}%
\pgfpathlineto{\pgfqpoint{3.508538in}{2.176941in}}%
\pgfpathlineto{\pgfqpoint{3.523253in}{2.227966in}}%
\pgfpathlineto{\pgfqpoint{3.537969in}{2.281327in}}%
\pgfpathlineto{\pgfqpoint{3.552684in}{2.332186in}}%
\pgfpathlineto{\pgfqpoint{3.567399in}{2.385082in}}%
\pgfpathlineto{\pgfqpoint{3.582115in}{2.412631in}}%
\pgfpathlineto{\pgfqpoint{3.596830in}{2.374666in}}%
\pgfpathlineto{\pgfqpoint{3.611545in}{2.282164in}}%
\pgfpathlineto{\pgfqpoint{3.626261in}{2.166161in}}%
\pgfpathlineto{\pgfqpoint{3.640976in}{2.092151in}}%
\pgfpathlineto{\pgfqpoint{3.655691in}{2.106868in}}%
\pgfpathlineto{\pgfqpoint{3.670406in}{2.183364in}}%
\pgfpathlineto{\pgfqpoint{3.685122in}{2.264853in}}%
\pgfpathlineto{\pgfqpoint{3.699837in}{2.336196in}}%
\pgfpathlineto{\pgfqpoint{3.714552in}{2.375640in}}%
\pgfpathlineto{\pgfqpoint{3.729268in}{2.424982in}}%
\pgfpathlineto{\pgfqpoint{3.743983in}{2.478150in}}%
\pgfpathlineto{\pgfqpoint{3.758698in}{2.520284in}}%
\pgfpathlineto{\pgfqpoint{3.773414in}{2.507524in}}%
\pgfpathlineto{\pgfqpoint{3.788129in}{2.424904in}}%
\pgfpathlineto{\pgfqpoint{3.802844in}{2.299576in}}%
\pgfpathlineto{\pgfqpoint{3.817560in}{2.236134in}}%
\pgfpathlineto{\pgfqpoint{3.832275in}{2.226608in}}%
\pgfpathlineto{\pgfqpoint{3.846990in}{2.321823in}}%
\pgfpathlineto{\pgfqpoint{3.861706in}{2.409897in}}%
\pgfpathlineto{\pgfqpoint{3.876421in}{2.471099in}}%
\pgfpathlineto{\pgfqpoint{3.891136in}{2.513560in}}%
\pgfpathlineto{\pgfqpoint{3.905852in}{2.559572in}}%
\pgfpathlineto{\pgfqpoint{3.920567in}{2.632302in}}%
\pgfpathlineto{\pgfqpoint{3.935282in}{2.681032in}}%
\pgfpathlineto{\pgfqpoint{3.949997in}{2.660969in}}%
\pgfpathlineto{\pgfqpoint{3.964713in}{2.572562in}}%
\pgfpathlineto{\pgfqpoint{3.979428in}{2.453201in}}%
\pgfpathlineto{\pgfqpoint{3.994143in}{2.372235in}}%
\pgfpathlineto{\pgfqpoint{4.008859in}{2.372099in}}%
\pgfpathlineto{\pgfqpoint{4.023574in}{2.463749in}}%
\pgfpathlineto{\pgfqpoint{4.038289in}{2.530729in}}%
\pgfpathlineto{\pgfqpoint{4.053005in}{2.589786in}}%
\pgfpathlineto{\pgfqpoint{4.067720in}{2.640163in}}%
\pgfpathlineto{\pgfqpoint{4.082435in}{2.697827in}}%
\pgfpathlineto{\pgfqpoint{4.097151in}{2.780815in}}%
\pgfpathlineto{\pgfqpoint{4.111866in}{2.795046in}}%
\pgfpathlineto{\pgfqpoint{4.126581in}{2.746756in}}%
\pgfpathlineto{\pgfqpoint{4.141297in}{2.614488in}}%
\pgfpathlineto{\pgfqpoint{4.156012in}{2.546756in}}%
\pgfpathlineto{\pgfqpoint{4.170727in}{2.438426in}}%
\pgfpathlineto{\pgfqpoint{4.185443in}{2.447835in}}%
\pgfpathlineto{\pgfqpoint{4.200158in}{2.543726in}}%
\pgfpathlineto{\pgfqpoint{4.214873in}{2.618263in}}%
\pgfpathlineto{\pgfqpoint{4.229589in}{2.672363in}}%
\pgfpathlineto{\pgfqpoint{4.244304in}{2.745674in}}%
\pgfpathlineto{\pgfqpoint{4.259019in}{2.806202in}}%
\pgfpathlineto{\pgfqpoint{4.273734in}{2.889964in}}%
\pgfpathlineto{\pgfqpoint{4.288450in}{2.898893in}}%
\pgfpathlineto{\pgfqpoint{4.303165in}{2.886417in}}%
\pgfpathlineto{\pgfqpoint{4.317880in}{2.800754in}}%
\pgfpathlineto{\pgfqpoint{4.332596in}{2.692167in}}%
\pgfpathlineto{\pgfqpoint{4.347311in}{2.573537in}}%
\pgfpathlineto{\pgfqpoint{4.362026in}{2.581048in}}%
\pgfpathlineto{\pgfqpoint{4.376742in}{2.669507in}}%
\pgfpathlineto{\pgfqpoint{4.391457in}{2.765983in}}%
\pgfpathlineto{\pgfqpoint{4.406172in}{2.841474in}}%
\pgfpathlineto{\pgfqpoint{4.420888in}{2.886395in}}%
\pgfpathlineto{\pgfqpoint{4.435603in}{2.911579in}}%
\pgfpathlineto{\pgfqpoint{4.450318in}{3.025251in}}%
\pgfpathlineto{\pgfqpoint{4.465034in}{3.044627in}}%
\pgfpathlineto{\pgfqpoint{4.479749in}{2.989403in}}%
\pgfpathlineto{\pgfqpoint{4.494464in}{2.898926in}}%
\pgfpathlineto{\pgfqpoint{4.509180in}{2.782098in}}%
\pgfpathlineto{\pgfqpoint{4.523895in}{2.693424in}}%
\pgfpathlineto{\pgfqpoint{4.538610in}{2.711437in}}%
\pgfpathlineto{\pgfqpoint{4.553325in}{2.768121in}}%
\pgfpathlineto{\pgfqpoint{4.568041in}{2.861524in}}%
\pgfpathlineto{\pgfqpoint{4.582756in}{2.925864in}}%
\pgfpathlineto{\pgfqpoint{4.597471in}{2.981027in}}%
\pgfpathlineto{\pgfqpoint{4.612187in}{3.019895in}}%
\pgfpathlineto{\pgfqpoint{4.626902in}{3.130071in}}%
\pgfpathlineto{\pgfqpoint{4.641617in}{3.132820in}}%
\pgfpathlineto{\pgfqpoint{4.656333in}{3.109392in}}%
\pgfpathlineto{\pgfqpoint{4.671048in}{3.021191in}}%
\pgfpathlineto{\pgfqpoint{4.685763in}{2.879566in}}%
\pgfpathlineto{\pgfqpoint{4.700479in}{2.815185in}}%
\pgfpathlineto{\pgfqpoint{4.715194in}{2.827623in}}%
\pgfpathlineto{\pgfqpoint{4.729909in}{2.902540in}}%
\pgfpathlineto{\pgfqpoint{4.759340in}{3.076353in}}%
\pgfpathlineto{\pgfqpoint{4.774055in}{3.091714in}}%
\pgfpathlineto{\pgfqpoint{4.788771in}{3.105264in}}%
\pgfpathlineto{\pgfqpoint{4.803486in}{3.168330in}}%
\pgfpathlineto{\pgfqpoint{4.818201in}{3.245822in}}%
\pgfpathlineto{\pgfqpoint{4.832916in}{3.207878in}}%
\pgfpathlineto{\pgfqpoint{4.847632in}{3.081794in}}%
\pgfpathlineto{\pgfqpoint{4.862347in}{2.994873in}}%
\pgfpathlineto{\pgfqpoint{4.877062in}{2.942665in}}%
\pgfpathlineto{\pgfqpoint{4.891778in}{2.927964in}}%
\pgfpathlineto{\pgfqpoint{4.906493in}{2.999466in}}%
\pgfpathlineto{\pgfqpoint{4.921208in}{3.074662in}}%
\pgfpathlineto{\pgfqpoint{4.935924in}{3.154312in}}%
\pgfpathlineto{\pgfqpoint{4.950639in}{3.185292in}}%
\pgfpathlineto{\pgfqpoint{4.965354in}{3.260631in}}%
\pgfpathlineto{\pgfqpoint{4.980070in}{3.301160in}}%
\pgfpathlineto{\pgfqpoint{4.994785in}{3.335016in}}%
\pgfpathlineto{\pgfqpoint{5.009500in}{3.298668in}}%
\pgfpathlineto{\pgfqpoint{5.024216in}{3.215500in}}%
\pgfpathlineto{\pgfqpoint{5.038931in}{3.104630in}}%
\pgfpathlineto{\pgfqpoint{5.053646in}{3.028827in}}%
\pgfpathlineto{\pgfqpoint{5.068362in}{3.012287in}}%
\pgfpathlineto{\pgfqpoint{5.083077in}{3.101575in}}%
\pgfpathlineto{\pgfqpoint{5.097792in}{3.178545in}}%
\pgfpathlineto{\pgfqpoint{5.112507in}{3.253105in}}%
\pgfpathlineto{\pgfqpoint{5.127223in}{3.337095in}}%
\pgfpathlineto{\pgfqpoint{5.141938in}{3.392402in}}%
\pgfpathlineto{\pgfqpoint{5.156653in}{3.465701in}}%
\pgfpathlineto{\pgfqpoint{5.171369in}{3.495727in}}%
\pgfpathlineto{\pgfqpoint{5.186084in}{3.450047in}}%
\pgfpathlineto{\pgfqpoint{5.200799in}{3.338679in}}%
\pgfpathlineto{\pgfqpoint{5.215515in}{3.233330in}}%
\pgfpathlineto{\pgfqpoint{5.230230in}{3.146527in}}%
\pgfpathlineto{\pgfqpoint{5.244945in}{3.169610in}}%
\pgfpathlineto{\pgfqpoint{5.259661in}{3.249291in}}%
\pgfpathlineto{\pgfqpoint{5.274376in}{3.313777in}}%
\pgfpathlineto{\pgfqpoint{5.289091in}{3.399271in}}%
\pgfpathlineto{\pgfqpoint{5.303807in}{3.430567in}}%
\pgfpathlineto{\pgfqpoint{5.318522in}{3.471233in}}%
\pgfpathlineto{\pgfqpoint{5.333237in}{3.507600in}}%
\pgfpathlineto{\pgfqpoint{5.347953in}{3.563484in}}%
\pgfpathlineto{\pgfqpoint{5.362668in}{3.535711in}}%
\pgfpathlineto{\pgfqpoint{5.377383in}{3.468350in}}%
\pgfpathlineto{\pgfqpoint{5.392098in}{3.336385in}}%
\pgfpathlineto{\pgfqpoint{5.406814in}{3.272073in}}%
\pgfpathlineto{\pgfqpoint{5.421529in}{3.269405in}}%
\pgfpathlineto{\pgfqpoint{5.436244in}{3.342383in}}%
\pgfpathlineto{\pgfqpoint{5.450960in}{3.429823in}}%
\pgfpathlineto{\pgfqpoint{5.465675in}{3.501866in}}%
\pgfpathlineto{\pgfqpoint{5.480390in}{3.543338in}}%
\pgfpathlineto{\pgfqpoint{5.495106in}{3.573237in}}%
\pgfpathlineto{\pgfqpoint{5.509821in}{3.675126in}}%
\pgfpathlineto{\pgfqpoint{5.524536in}{3.705927in}}%
\pgfpathlineto{\pgfqpoint{5.539252in}{3.648982in}}%
\pgfpathlineto{\pgfqpoint{5.553967in}{3.572721in}}%
\pgfpathlineto{\pgfqpoint{5.568682in}{3.459462in}}%
\pgfpathlineto{\pgfqpoint{5.583398in}{3.388437in}}%
\pgfpathlineto{\pgfqpoint{5.598113in}{3.386364in}}%
\pgfpathlineto{\pgfqpoint{5.612828in}{3.496506in}}%
\pgfpathlineto{\pgfqpoint{5.627544in}{3.572323in}}%
\pgfpathlineto{\pgfqpoint{5.656974in}{3.711491in}}%
\pgfpathlineto{\pgfqpoint{5.671689in}{3.747861in}}%
\pgfpathlineto{\pgfqpoint{5.686405in}{3.801211in}}%
\pgfpathlineto{\pgfqpoint{5.701120in}{3.880383in}}%
\pgfpathlineto{\pgfqpoint{5.715835in}{3.814365in}}%
\pgfpathlineto{\pgfqpoint{5.730551in}{3.735358in}}%
\pgfpathlineto{\pgfqpoint{5.745266in}{3.619144in}}%
\pgfpathlineto{\pgfqpoint{5.759981in}{3.525009in}}%
\pgfpathlineto{\pgfqpoint{5.774697in}{3.535046in}}%
\pgfpathlineto{\pgfqpoint{5.789412in}{3.617279in}}%
\pgfpathlineto{\pgfqpoint{5.804127in}{3.711197in}}%
\pgfpathlineto{\pgfqpoint{5.818843in}{3.768746in}}%
\pgfpathlineto{\pgfqpoint{5.833558in}{3.781488in}}%
\pgfpathlineto{\pgfqpoint{5.848273in}{3.873330in}}%
\pgfpathlineto{\pgfqpoint{5.862989in}{3.963302in}}%
\pgfpathlineto{\pgfqpoint{5.877704in}{3.989235in}}%
\pgfpathlineto{\pgfqpoint{5.892419in}{3.960763in}}%
\pgfpathlineto{\pgfqpoint{5.907135in}{3.843413in}}%
\pgfpathlineto{\pgfqpoint{5.921850in}{3.718927in}}%
\pgfpathlineto{\pgfqpoint{5.936565in}{3.625038in}}%
\pgfpathlineto{\pgfqpoint{5.951280in}{3.660836in}}%
\pgfpathlineto{\pgfqpoint{5.965996in}{3.731089in}}%
\pgfpathlineto{\pgfqpoint{5.980711in}{3.828319in}}%
\pgfpathlineto{\pgfqpoint{5.995426in}{3.889209in}}%
\pgfpathlineto{\pgfqpoint{6.010142in}{3.910510in}}%
\pgfpathlineto{\pgfqpoint{6.024857in}{3.979245in}}%
\pgfpathlineto{\pgfqpoint{6.039572in}{4.069973in}}%
\pgfpathlineto{\pgfqpoint{6.054288in}{4.113220in}}%
\pgfpathlineto{\pgfqpoint{6.069003in}{4.045685in}}%
\pgfpathlineto{\pgfqpoint{6.083718in}{3.962188in}}%
\pgfpathlineto{\pgfqpoint{6.098434in}{3.827146in}}%
\pgfpathlineto{\pgfqpoint{6.113149in}{3.756187in}}%
\pgfpathlineto{\pgfqpoint{6.127864in}{3.794558in}}%
\pgfpathlineto{\pgfqpoint{6.157295in}{3.995888in}}%
\pgfpathlineto{\pgfqpoint{6.172010in}{4.034614in}}%
\pgfpathlineto{\pgfqpoint{6.186726in}{4.121554in}}%
\pgfpathlineto{\pgfqpoint{6.201441in}{4.164638in}}%
\pgfpathlineto{\pgfqpoint{6.216156in}{4.309818in}}%
\pgfpathlineto{\pgfqpoint{6.230871in}{4.325629in}}%
\pgfpathlineto{\pgfqpoint{6.245587in}{4.274806in}}%
\pgfpathlineto{\pgfqpoint{6.260302in}{4.134470in}}%
\pgfpathlineto{\pgfqpoint{6.275017in}{4.016864in}}%
\pgfpathlineto{\pgfqpoint{6.289733in}{3.948910in}}%
\pgfpathlineto{\pgfqpoint{6.304448in}{3.980245in}}%
\pgfpathlineto{\pgfqpoint{6.319163in}{4.089049in}}%
\pgfpathlineto{\pgfqpoint{6.333879in}{4.142277in}}%
\pgfpathlineto{\pgfqpoint{6.348594in}{4.236713in}}%
\pgfpathlineto{\pgfqpoint{6.363309in}{4.254219in}}%
\pgfpathlineto{\pgfqpoint{6.378025in}{4.307159in}}%
\pgfpathlineto{\pgfqpoint{6.392740in}{4.400504in}}%
\pgfpathlineto{\pgfqpoint{6.407455in}{4.437431in}}%
\pgfpathlineto{\pgfqpoint{6.422171in}{4.392230in}}%
\pgfpathlineto{\pgfqpoint{6.436886in}{4.292025in}}%
\pgfpathlineto{\pgfqpoint{6.451601in}{4.177460in}}%
\pgfpathlineto{\pgfqpoint{6.466317in}{4.081765in}}%
\pgfpathlineto{\pgfqpoint{6.481032in}{4.094528in}}%
\pgfpathlineto{\pgfqpoint{6.495747in}{4.175039in}}%
\pgfpathlineto{\pgfqpoint{6.510463in}{4.274353in}}%
\pgfpathlineto{\pgfqpoint{6.525178in}{4.339667in}}%
\pgfpathlineto{\pgfqpoint{6.539893in}{4.360346in}}%
\pgfpathlineto{\pgfqpoint{6.554608in}{4.419924in}}%
\pgfpathlineto{\pgfqpoint{6.569324in}{4.468421in}}%
\pgfpathlineto{\pgfqpoint{6.584039in}{4.524851in}}%
\pgfpathlineto{\pgfqpoint{6.598754in}{4.500336in}}%
\pgfpathlineto{\pgfqpoint{6.613470in}{4.379127in}}%
\pgfpathlineto{\pgfqpoint{6.628185in}{4.283351in}}%
\pgfpathlineto{\pgfqpoint{6.642900in}{4.202149in}}%
\pgfpathlineto{\pgfqpoint{6.657616in}{4.228745in}}%
\pgfpathlineto{\pgfqpoint{6.672331in}{4.346402in}}%
\pgfpathlineto{\pgfqpoint{6.687046in}{4.403463in}}%
\pgfpathlineto{\pgfqpoint{6.701762in}{4.502812in}}%
\pgfpathlineto{\pgfqpoint{6.716477in}{4.554563in}}%
\pgfpathlineto{\pgfqpoint{6.731192in}{4.564685in}}%
\pgfpathlineto{\pgfqpoint{6.745908in}{4.644779in}}%
\pgfpathlineto{\pgfqpoint{6.760623in}{4.718489in}}%
\pgfpathlineto{\pgfqpoint{6.775338in}{4.677779in}}%
\pgfpathlineto{\pgfqpoint{6.790054in}{4.557252in}}%
\pgfpathlineto{\pgfqpoint{6.804769in}{4.455691in}}%
\pgfpathlineto{\pgfqpoint{6.819484in}{4.373935in}}%
\pgfpathlineto{\pgfqpoint{6.834199in}{4.371754in}}%
\pgfpathlineto{\pgfqpoint{6.848915in}{4.472495in}}%
\pgfpathlineto{\pgfqpoint{6.863630in}{4.554833in}}%
\pgfpathlineto{\pgfqpoint{6.878345in}{4.647259in}}%
\pgfpathlineto{\pgfqpoint{6.893061in}{4.688422in}}%
\pgfpathlineto{\pgfqpoint{6.907776in}{4.708840in}}%
\pgfpathlineto{\pgfqpoint{6.922491in}{4.806492in}}%
\pgfpathlineto{\pgfqpoint{6.937207in}{4.855076in}}%
\pgfpathlineto{\pgfqpoint{6.951922in}{4.812916in}}%
\pgfpathlineto{\pgfqpoint{6.981353in}{4.594315in}}%
\pgfpathlineto{\pgfqpoint{6.996068in}{4.528666in}}%
\pgfpathlineto{\pgfqpoint{7.010783in}{4.527288in}}%
\pgfpathlineto{\pgfqpoint{7.025499in}{4.618743in}}%
\pgfpathlineto{\pgfqpoint{7.040214in}{4.681999in}}%
\pgfpathlineto{\pgfqpoint{7.054929in}{4.753762in}}%
\pgfpathlineto{\pgfqpoint{7.069645in}{4.823290in}}%
\pgfpathlineto{\pgfqpoint{7.084360in}{4.871264in}}%
\pgfpathlineto{\pgfqpoint{7.099075in}{4.952236in}}%
\pgfpathlineto{\pgfqpoint{7.113790in}{4.956988in}}%
\pgfpathlineto{\pgfqpoint{7.128506in}{4.948991in}}%
\pgfpathlineto{\pgfqpoint{7.143221in}{4.833328in}}%
\pgfpathlineto{\pgfqpoint{7.157936in}{4.699157in}}%
\pgfpathlineto{\pgfqpoint{7.172652in}{4.628028in}}%
\pgfpathlineto{\pgfqpoint{7.187367in}{4.662816in}}%
\pgfpathlineto{\pgfqpoint{7.202082in}{4.724724in}}%
\pgfpathlineto{\pgfqpoint{7.216798in}{4.821273in}}%
\pgfpathlineto{\pgfqpoint{7.231513in}{4.902077in}}%
\pgfpathlineto{\pgfqpoint{7.246228in}{4.965599in}}%
\pgfpathlineto{\pgfqpoint{7.260944in}{4.938496in}}%
\pgfpathlineto{\pgfqpoint{7.275659in}{5.020046in}}%
\pgfpathlineto{\pgfqpoint{7.290374in}{5.063044in}}%
\pgfpathlineto{\pgfqpoint{7.305090in}{5.061003in}}%
\pgfpathlineto{\pgfqpoint{7.319805in}{4.943641in}}%
\pgfpathlineto{\pgfqpoint{7.334520in}{4.849158in}}%
\pgfpathlineto{\pgfqpoint{7.349236in}{4.776324in}}%
\pgfpathlineto{\pgfqpoint{7.363951in}{4.767053in}}%
\pgfpathlineto{\pgfqpoint{7.378666in}{4.865884in}}%
\pgfpathlineto{\pgfqpoint{7.393381in}{4.980627in}}%
\pgfpathlineto{\pgfqpoint{7.408097in}{5.077119in}}%
\pgfpathlineto{\pgfqpoint{7.422812in}{5.145777in}}%
\pgfpathlineto{\pgfqpoint{7.437527in}{5.198205in}}%
\pgfpathlineto{\pgfqpoint{7.452243in}{5.222266in}}%
\pgfpathlineto{\pgfqpoint{7.466958in}{5.216280in}}%
\pgfpathlineto{\pgfqpoint{7.481673in}{5.177487in}}%
\pgfpathlineto{\pgfqpoint{7.496389in}{5.100016in}}%
\pgfpathlineto{\pgfqpoint{7.511104in}{4.992755in}}%
\pgfpathlineto{\pgfqpoint{7.525819in}{4.918483in}}%
\pgfpathlineto{\pgfqpoint{7.540535in}{4.942741in}}%
\pgfpathlineto{\pgfqpoint{7.555250in}{5.028274in}}%
\pgfpathlineto{\pgfqpoint{7.569965in}{5.115708in}}%
\pgfpathlineto{\pgfqpoint{7.584681in}{5.168722in}}%
\pgfpathlineto{\pgfqpoint{7.599396in}{5.266526in}}%
\pgfpathlineto{\pgfqpoint{7.614111in}{5.311312in}}%
\pgfpathlineto{\pgfqpoint{7.628827in}{5.379659in}}%
\pgfpathlineto{\pgfqpoint{7.643542in}{5.397878in}}%
\pgfpathlineto{\pgfqpoint{7.658257in}{5.396051in}}%
\pgfpathlineto{\pgfqpoint{7.672972in}{5.327932in}}%
\pgfpathlineto{\pgfqpoint{7.687688in}{5.176902in}}%
\pgfpathlineto{\pgfqpoint{7.702403in}{5.124796in}}%
\pgfpathlineto{\pgfqpoint{7.717118in}{5.145000in}}%
\pgfpathlineto{\pgfqpoint{7.731834in}{5.227074in}}%
\pgfpathlineto{\pgfqpoint{7.746549in}{5.314122in}}%
\pgfpathlineto{\pgfqpoint{7.746549in}{5.314122in}}%
\pgfusepath{stroke}%
\end{pgfscope}%
\begin{pgfscope}%
\pgfpathrectangle{\pgfqpoint{0.697913in}{0.559721in}}{\pgfqpoint{7.048636in}{4.990279in}}%
\pgfusepath{clip}%
\pgfsetbuttcap%
\pgfsetroundjoin%
\definecolor{currentfill}{rgb}{1.000000,0.000000,0.000000}%
\pgfsetfillcolor{currentfill}%
\pgfsetlinewidth{1.003750pt}%
\definecolor{currentstroke}{rgb}{1.000000,0.000000,0.000000}%
\pgfsetstrokecolor{currentstroke}%
\pgfsetdash{}{0pt}%
\pgfsys@defobject{currentmarker}{\pgfqpoint{-0.020833in}{-0.020833in}}{\pgfqpoint{0.020833in}{0.020833in}}{%
\pgfpathmoveto{\pgfqpoint{0.000000in}{-0.020833in}}%
\pgfpathcurveto{\pgfqpoint{0.005525in}{-0.020833in}}{\pgfqpoint{0.010825in}{-0.018638in}}{\pgfqpoint{0.014731in}{-0.014731in}}%
\pgfpathcurveto{\pgfqpoint{0.018638in}{-0.010825in}}{\pgfqpoint{0.020833in}{-0.005525in}}{\pgfqpoint{0.020833in}{0.000000in}}%
\pgfpathcurveto{\pgfqpoint{0.020833in}{0.005525in}}{\pgfqpoint{0.018638in}{0.010825in}}{\pgfqpoint{0.014731in}{0.014731in}}%
\pgfpathcurveto{\pgfqpoint{0.010825in}{0.018638in}}{\pgfqpoint{0.005525in}{0.020833in}}{\pgfqpoint{0.000000in}{0.020833in}}%
\pgfpathcurveto{\pgfqpoint{-0.005525in}{0.020833in}}{\pgfqpoint{-0.010825in}{0.018638in}}{\pgfqpoint{-0.014731in}{0.014731in}}%
\pgfpathcurveto{\pgfqpoint{-0.018638in}{0.010825in}}{\pgfqpoint{-0.020833in}{0.005525in}}{\pgfqpoint{-0.020833in}{0.000000in}}%
\pgfpathcurveto{\pgfqpoint{-0.020833in}{-0.005525in}}{\pgfqpoint{-0.018638in}{-0.010825in}}{\pgfqpoint{-0.014731in}{-0.014731in}}%
\pgfpathcurveto{\pgfqpoint{-0.010825in}{-0.018638in}}{\pgfqpoint{-0.005525in}{-0.020833in}}{\pgfqpoint{0.000000in}{-0.020833in}}%
\pgfpathlineto{\pgfqpoint{0.000000in}{-0.020833in}}%
\pgfpathclose%
\pgfusepath{stroke,fill}%
}%
\begin{pgfscope}%
\pgfsys@transformshift{0.697913in}{0.800800in}%
\pgfsys@useobject{currentmarker}{}%
\end{pgfscope}%
\begin{pgfscope}%
\pgfsys@transformshift{0.712628in}{0.854764in}%
\pgfsys@useobject{currentmarker}{}%
\end{pgfscope}%
\begin{pgfscope}%
\pgfsys@transformshift{0.727343in}{0.933745in}%
\pgfsys@useobject{currentmarker}{}%
\end{pgfscope}%
\begin{pgfscope}%
\pgfsys@transformshift{0.742059in}{0.977716in}%
\pgfsys@useobject{currentmarker}{}%
\end{pgfscope}%
\begin{pgfscope}%
\pgfsys@transformshift{0.756774in}{1.015272in}%
\pgfsys@useobject{currentmarker}{}%
\end{pgfscope}%
\begin{pgfscope}%
\pgfsys@transformshift{0.771489in}{0.976371in}%
\pgfsys@useobject{currentmarker}{}%
\end{pgfscope}%
\begin{pgfscope}%
\pgfsys@transformshift{0.786205in}{0.886780in}%
\pgfsys@useobject{currentmarker}{}%
\end{pgfscope}%
\begin{pgfscope}%
\pgfsys@transformshift{0.800920in}{0.775096in}%
\pgfsys@useobject{currentmarker}{}%
\end{pgfscope}%
\begin{pgfscope}%
\pgfsys@transformshift{0.815635in}{0.687494in}%
\pgfsys@useobject{currentmarker}{}%
\end{pgfscope}%
\begin{pgfscope}%
\pgfsys@transformshift{0.830350in}{0.682107in}%
\pgfsys@useobject{currentmarker}{}%
\end{pgfscope}%
\begin{pgfscope}%
\pgfsys@transformshift{0.845066in}{0.760034in}%
\pgfsys@useobject{currentmarker}{}%
\end{pgfscope}%
\begin{pgfscope}%
\pgfsys@transformshift{0.859781in}{0.838848in}%
\pgfsys@useobject{currentmarker}{}%
\end{pgfscope}%
\begin{pgfscope}%
\pgfsys@transformshift{0.874496in}{0.875106in}%
\pgfsys@useobject{currentmarker}{}%
\end{pgfscope}%
\begin{pgfscope}%
\pgfsys@transformshift{0.889212in}{0.913637in}%
\pgfsys@useobject{currentmarker}{}%
\end{pgfscope}%
\begin{pgfscope}%
\pgfsys@transformshift{0.903927in}{0.955498in}%
\pgfsys@useobject{currentmarker}{}%
\end{pgfscope}%
\begin{pgfscope}%
\pgfsys@transformshift{0.918642in}{1.048312in}%
\pgfsys@useobject{currentmarker}{}%
\end{pgfscope}%
\begin{pgfscope}%
\pgfsys@transformshift{0.933358in}{1.085369in}%
\pgfsys@useobject{currentmarker}{}%
\end{pgfscope}%
\begin{pgfscope}%
\pgfsys@transformshift{0.948073in}{1.055640in}%
\pgfsys@useobject{currentmarker}{}%
\end{pgfscope}%
\begin{pgfscope}%
\pgfsys@transformshift{0.962788in}{0.959881in}%
\pgfsys@useobject{currentmarker}{}%
\end{pgfscope}%
\begin{pgfscope}%
\pgfsys@transformshift{0.977504in}{0.841100in}%
\pgfsys@useobject{currentmarker}{}%
\end{pgfscope}%
\begin{pgfscope}%
\pgfsys@transformshift{0.992219in}{0.795475in}%
\pgfsys@useobject{currentmarker}{}%
\end{pgfscope}%
\begin{pgfscope}%
\pgfsys@transformshift{1.006934in}{0.760407in}%
\pgfsys@useobject{currentmarker}{}%
\end{pgfscope}%
\begin{pgfscope}%
\pgfsys@transformshift{1.021650in}{0.844761in}%
\pgfsys@useobject{currentmarker}{}%
\end{pgfscope}%
\begin{pgfscope}%
\pgfsys@transformshift{1.036365in}{0.909468in}%
\pgfsys@useobject{currentmarker}{}%
\end{pgfscope}%
\begin{pgfscope}%
\pgfsys@transformshift{1.051080in}{0.982320in}%
\pgfsys@useobject{currentmarker}{}%
\end{pgfscope}%
\begin{pgfscope}%
\pgfsys@transformshift{1.065796in}{0.994584in}%
\pgfsys@useobject{currentmarker}{}%
\end{pgfscope}%
\begin{pgfscope}%
\pgfsys@transformshift{1.080511in}{1.044934in}%
\pgfsys@useobject{currentmarker}{}%
\end{pgfscope}%
\begin{pgfscope}%
\pgfsys@transformshift{1.095226in}{1.136847in}%
\pgfsys@useobject{currentmarker}{}%
\end{pgfscope}%
\begin{pgfscope}%
\pgfsys@transformshift{1.109941in}{1.177436in}%
\pgfsys@useobject{currentmarker}{}%
\end{pgfscope}%
\begin{pgfscope}%
\pgfsys@transformshift{1.124657in}{1.150975in}%
\pgfsys@useobject{currentmarker}{}%
\end{pgfscope}%
\begin{pgfscope}%
\pgfsys@transformshift{1.139372in}{1.075536in}%
\pgfsys@useobject{currentmarker}{}%
\end{pgfscope}%
\begin{pgfscope}%
\pgfsys@transformshift{1.154087in}{0.954253in}%
\pgfsys@useobject{currentmarker}{}%
\end{pgfscope}%
\begin{pgfscope}%
\pgfsys@transformshift{1.168803in}{0.864921in}%
\pgfsys@useobject{currentmarker}{}%
\end{pgfscope}%
\begin{pgfscope}%
\pgfsys@transformshift{1.183518in}{0.873943in}%
\pgfsys@useobject{currentmarker}{}%
\end{pgfscope}%
\begin{pgfscope}%
\pgfsys@transformshift{1.198233in}{0.947407in}%
\pgfsys@useobject{currentmarker}{}%
\end{pgfscope}%
\begin{pgfscope}%
\pgfsys@transformshift{1.212949in}{1.016117in}%
\pgfsys@useobject{currentmarker}{}%
\end{pgfscope}%
\begin{pgfscope}%
\pgfsys@transformshift{1.227664in}{1.081354in}%
\pgfsys@useobject{currentmarker}{}%
\end{pgfscope}%
\begin{pgfscope}%
\pgfsys@transformshift{1.242379in}{1.154139in}%
\pgfsys@useobject{currentmarker}{}%
\end{pgfscope}%
\begin{pgfscope}%
\pgfsys@transformshift{1.257095in}{1.194737in}%
\pgfsys@useobject{currentmarker}{}%
\end{pgfscope}%
\begin{pgfscope}%
\pgfsys@transformshift{1.271810in}{1.262801in}%
\pgfsys@useobject{currentmarker}{}%
\end{pgfscope}%
\begin{pgfscope}%
\pgfsys@transformshift{1.286525in}{1.301798in}%
\pgfsys@useobject{currentmarker}{}%
\end{pgfscope}%
\begin{pgfscope}%
\pgfsys@transformshift{1.301241in}{1.271334in}%
\pgfsys@useobject{currentmarker}{}%
\end{pgfscope}%
\begin{pgfscope}%
\pgfsys@transformshift{1.315956in}{1.212318in}%
\pgfsys@useobject{currentmarker}{}%
\end{pgfscope}%
\begin{pgfscope}%
\pgfsys@transformshift{1.330671in}{1.080731in}%
\pgfsys@useobject{currentmarker}{}%
\end{pgfscope}%
\begin{pgfscope}%
\pgfsys@transformshift{1.345387in}{0.998402in}%
\pgfsys@useobject{currentmarker}{}%
\end{pgfscope}%
\begin{pgfscope}%
\pgfsys@transformshift{1.360102in}{1.021379in}%
\pgfsys@useobject{currentmarker}{}%
\end{pgfscope}%
\begin{pgfscope}%
\pgfsys@transformshift{1.374817in}{1.082667in}%
\pgfsys@useobject{currentmarker}{}%
\end{pgfscope}%
\begin{pgfscope}%
\pgfsys@transformshift{1.389532in}{1.146865in}%
\pgfsys@useobject{currentmarker}{}%
\end{pgfscope}%
\begin{pgfscope}%
\pgfsys@transformshift{1.404248in}{1.228919in}%
\pgfsys@useobject{currentmarker}{}%
\end{pgfscope}%
\begin{pgfscope}%
\pgfsys@transformshift{1.418963in}{1.249078in}%
\pgfsys@useobject{currentmarker}{}%
\end{pgfscope}%
\begin{pgfscope}%
\pgfsys@transformshift{1.433678in}{1.285964in}%
\pgfsys@useobject{currentmarker}{}%
\end{pgfscope}%
\begin{pgfscope}%
\pgfsys@transformshift{1.448394in}{1.379033in}%
\pgfsys@useobject{currentmarker}{}%
\end{pgfscope}%
\begin{pgfscope}%
\pgfsys@transformshift{1.463109in}{1.392104in}%
\pgfsys@useobject{currentmarker}{}%
\end{pgfscope}%
\begin{pgfscope}%
\pgfsys@transformshift{1.477824in}{1.362008in}%
\pgfsys@useobject{currentmarker}{}%
\end{pgfscope}%
\begin{pgfscope}%
\pgfsys@transformshift{1.492540in}{1.284978in}%
\pgfsys@useobject{currentmarker}{}%
\end{pgfscope}%
\begin{pgfscope}%
\pgfsys@transformshift{1.507255in}{1.162372in}%
\pgfsys@useobject{currentmarker}{}%
\end{pgfscope}%
\begin{pgfscope}%
\pgfsys@transformshift{1.521970in}{1.062532in}%
\pgfsys@useobject{currentmarker}{}%
\end{pgfscope}%
\begin{pgfscope}%
\pgfsys@transformshift{1.536686in}{1.076569in}%
\pgfsys@useobject{currentmarker}{}%
\end{pgfscope}%
\begin{pgfscope}%
\pgfsys@transformshift{1.551401in}{1.145336in}%
\pgfsys@useobject{currentmarker}{}%
\end{pgfscope}%
\begin{pgfscope}%
\pgfsys@transformshift{1.566116in}{1.217965in}%
\pgfsys@useobject{currentmarker}{}%
\end{pgfscope}%
\begin{pgfscope}%
\pgfsys@transformshift{1.580832in}{1.274509in}%
\pgfsys@useobject{currentmarker}{}%
\end{pgfscope}%
\begin{pgfscope}%
\pgfsys@transformshift{1.595547in}{1.343743in}%
\pgfsys@useobject{currentmarker}{}%
\end{pgfscope}%
\begin{pgfscope}%
\pgfsys@transformshift{1.610262in}{1.381731in}%
\pgfsys@useobject{currentmarker}{}%
\end{pgfscope}%
\begin{pgfscope}%
\pgfsys@transformshift{1.624978in}{1.415855in}%
\pgfsys@useobject{currentmarker}{}%
\end{pgfscope}%
\begin{pgfscope}%
\pgfsys@transformshift{1.639693in}{1.472501in}%
\pgfsys@useobject{currentmarker}{}%
\end{pgfscope}%
\begin{pgfscope}%
\pgfsys@transformshift{1.654408in}{1.417875in}%
\pgfsys@useobject{currentmarker}{}%
\end{pgfscope}%
\begin{pgfscope}%
\pgfsys@transformshift{1.669124in}{1.332844in}%
\pgfsys@useobject{currentmarker}{}%
\end{pgfscope}%
\begin{pgfscope}%
\pgfsys@transformshift{1.683839in}{1.227407in}%
\pgfsys@useobject{currentmarker}{}%
\end{pgfscope}%
\begin{pgfscope}%
\pgfsys@transformshift{1.698554in}{1.132997in}%
\pgfsys@useobject{currentmarker}{}%
\end{pgfscope}%
\begin{pgfscope}%
\pgfsys@transformshift{1.713269in}{1.152108in}%
\pgfsys@useobject{currentmarker}{}%
\end{pgfscope}%
\begin{pgfscope}%
\pgfsys@transformshift{1.727985in}{1.230512in}%
\pgfsys@useobject{currentmarker}{}%
\end{pgfscope}%
\begin{pgfscope}%
\pgfsys@transformshift{1.742700in}{1.306255in}%
\pgfsys@useobject{currentmarker}{}%
\end{pgfscope}%
\begin{pgfscope}%
\pgfsys@transformshift{1.757415in}{1.334366in}%
\pgfsys@useobject{currentmarker}{}%
\end{pgfscope}%
\begin{pgfscope}%
\pgfsys@transformshift{1.772131in}{1.384833in}%
\pgfsys@useobject{currentmarker}{}%
\end{pgfscope}%
\begin{pgfscope}%
\pgfsys@transformshift{1.786846in}{1.470691in}%
\pgfsys@useobject{currentmarker}{}%
\end{pgfscope}%
\begin{pgfscope}%
\pgfsys@transformshift{1.801561in}{1.551077in}%
\pgfsys@useobject{currentmarker}{}%
\end{pgfscope}%
\begin{pgfscope}%
\pgfsys@transformshift{1.816277in}{1.577512in}%
\pgfsys@useobject{currentmarker}{}%
\end{pgfscope}%
\begin{pgfscope}%
\pgfsys@transformshift{1.830992in}{1.520951in}%
\pgfsys@useobject{currentmarker}{}%
\end{pgfscope}%
\begin{pgfscope}%
\pgfsys@transformshift{1.845707in}{1.412191in}%
\pgfsys@useobject{currentmarker}{}%
\end{pgfscope}%
\begin{pgfscope}%
\pgfsys@transformshift{1.860423in}{1.272178in}%
\pgfsys@useobject{currentmarker}{}%
\end{pgfscope}%
\begin{pgfscope}%
\pgfsys@transformshift{1.875138in}{1.187642in}%
\pgfsys@useobject{currentmarker}{}%
\end{pgfscope}%
\begin{pgfscope}%
\pgfsys@transformshift{1.889853in}{1.192263in}%
\pgfsys@useobject{currentmarker}{}%
\end{pgfscope}%
\begin{pgfscope}%
\pgfsys@transformshift{1.904569in}{1.275825in}%
\pgfsys@useobject{currentmarker}{}%
\end{pgfscope}%
\begin{pgfscope}%
\pgfsys@transformshift{1.919284in}{1.351213in}%
\pgfsys@useobject{currentmarker}{}%
\end{pgfscope}%
\begin{pgfscope}%
\pgfsys@transformshift{1.933999in}{1.414220in}%
\pgfsys@useobject{currentmarker}{}%
\end{pgfscope}%
\begin{pgfscope}%
\pgfsys@transformshift{1.948715in}{1.463436in}%
\pgfsys@useobject{currentmarker}{}%
\end{pgfscope}%
\begin{pgfscope}%
\pgfsys@transformshift{1.963430in}{1.504903in}%
\pgfsys@useobject{currentmarker}{}%
\end{pgfscope}%
\begin{pgfscope}%
\pgfsys@transformshift{1.978145in}{1.576992in}%
\pgfsys@useobject{currentmarker}{}%
\end{pgfscope}%
\begin{pgfscope}%
\pgfsys@transformshift{1.992860in}{1.604477in}%
\pgfsys@useobject{currentmarker}{}%
\end{pgfscope}%
\begin{pgfscope}%
\pgfsys@transformshift{2.007576in}{1.588776in}%
\pgfsys@useobject{currentmarker}{}%
\end{pgfscope}%
\begin{pgfscope}%
\pgfsys@transformshift{2.022291in}{1.459711in}%
\pgfsys@useobject{currentmarker}{}%
\end{pgfscope}%
\begin{pgfscope}%
\pgfsys@transformshift{2.037006in}{1.342687in}%
\pgfsys@useobject{currentmarker}{}%
\end{pgfscope}%
\begin{pgfscope}%
\pgfsys@transformshift{2.051722in}{1.224934in}%
\pgfsys@useobject{currentmarker}{}%
\end{pgfscope}%
\begin{pgfscope}%
\pgfsys@transformshift{2.066437in}{1.249769in}%
\pgfsys@useobject{currentmarker}{}%
\end{pgfscope}%
\begin{pgfscope}%
\pgfsys@transformshift{2.081152in}{1.304480in}%
\pgfsys@useobject{currentmarker}{}%
\end{pgfscope}%
\begin{pgfscope}%
\pgfsys@transformshift{2.095868in}{1.376535in}%
\pgfsys@useobject{currentmarker}{}%
\end{pgfscope}%
\begin{pgfscope}%
\pgfsys@transformshift{2.110583in}{1.457719in}%
\pgfsys@useobject{currentmarker}{}%
\end{pgfscope}%
\begin{pgfscope}%
\pgfsys@transformshift{2.125298in}{1.474739in}%
\pgfsys@useobject{currentmarker}{}%
\end{pgfscope}%
\begin{pgfscope}%
\pgfsys@transformshift{2.140014in}{1.541798in}%
\pgfsys@useobject{currentmarker}{}%
\end{pgfscope}%
\begin{pgfscope}%
\pgfsys@transformshift{2.154729in}{1.586777in}%
\pgfsys@useobject{currentmarker}{}%
\end{pgfscope}%
\begin{pgfscope}%
\pgfsys@transformshift{2.169444in}{1.637899in}%
\pgfsys@useobject{currentmarker}{}%
\end{pgfscope}%
\begin{pgfscope}%
\pgfsys@transformshift{2.184160in}{1.600492in}%
\pgfsys@useobject{currentmarker}{}%
\end{pgfscope}%
\begin{pgfscope}%
\pgfsys@transformshift{2.198875in}{1.479179in}%
\pgfsys@useobject{currentmarker}{}%
\end{pgfscope}%
\begin{pgfscope}%
\pgfsys@transformshift{2.213590in}{1.382317in}%
\pgfsys@useobject{currentmarker}{}%
\end{pgfscope}%
\begin{pgfscope}%
\pgfsys@transformshift{2.228306in}{1.288327in}%
\pgfsys@useobject{currentmarker}{}%
\end{pgfscope}%
\begin{pgfscope}%
\pgfsys@transformshift{2.243021in}{1.296898in}%
\pgfsys@useobject{currentmarker}{}%
\end{pgfscope}%
\begin{pgfscope}%
\pgfsys@transformshift{2.257736in}{1.367934in}%
\pgfsys@useobject{currentmarker}{}%
\end{pgfscope}%
\begin{pgfscope}%
\pgfsys@transformshift{2.272451in}{1.453039in}%
\pgfsys@useobject{currentmarker}{}%
\end{pgfscope}%
\begin{pgfscope}%
\pgfsys@transformshift{2.287167in}{1.525275in}%
\pgfsys@useobject{currentmarker}{}%
\end{pgfscope}%
\begin{pgfscope}%
\pgfsys@transformshift{2.301882in}{1.567715in}%
\pgfsys@useobject{currentmarker}{}%
\end{pgfscope}%
\begin{pgfscope}%
\pgfsys@transformshift{2.316597in}{1.626149in}%
\pgfsys@useobject{currentmarker}{}%
\end{pgfscope}%
\begin{pgfscope}%
\pgfsys@transformshift{2.331313in}{1.699022in}%
\pgfsys@useobject{currentmarker}{}%
\end{pgfscope}%
\begin{pgfscope}%
\pgfsys@transformshift{2.346028in}{1.721353in}%
\pgfsys@useobject{currentmarker}{}%
\end{pgfscope}%
\begin{pgfscope}%
\pgfsys@transformshift{2.360743in}{1.675260in}%
\pgfsys@useobject{currentmarker}{}%
\end{pgfscope}%
\begin{pgfscope}%
\pgfsys@transformshift{2.375459in}{1.593211in}%
\pgfsys@useobject{currentmarker}{}%
\end{pgfscope}%
\begin{pgfscope}%
\pgfsys@transformshift{2.390174in}{1.487188in}%
\pgfsys@useobject{currentmarker}{}%
\end{pgfscope}%
\begin{pgfscope}%
\pgfsys@transformshift{2.404889in}{1.387600in}%
\pgfsys@useobject{currentmarker}{}%
\end{pgfscope}%
\begin{pgfscope}%
\pgfsys@transformshift{2.419605in}{1.404099in}%
\pgfsys@useobject{currentmarker}{}%
\end{pgfscope}%
\begin{pgfscope}%
\pgfsys@transformshift{2.434320in}{1.489077in}%
\pgfsys@useobject{currentmarker}{}%
\end{pgfscope}%
\begin{pgfscope}%
\pgfsys@transformshift{2.449035in}{1.568045in}%
\pgfsys@useobject{currentmarker}{}%
\end{pgfscope}%
\begin{pgfscope}%
\pgfsys@transformshift{2.463751in}{1.623407in}%
\pgfsys@useobject{currentmarker}{}%
\end{pgfscope}%
\begin{pgfscope}%
\pgfsys@transformshift{2.478466in}{1.676960in}%
\pgfsys@useobject{currentmarker}{}%
\end{pgfscope}%
\begin{pgfscope}%
\pgfsys@transformshift{2.493181in}{1.731250in}%
\pgfsys@useobject{currentmarker}{}%
\end{pgfscope}%
\begin{pgfscope}%
\pgfsys@transformshift{2.507897in}{1.813823in}%
\pgfsys@useobject{currentmarker}{}%
\end{pgfscope}%
\begin{pgfscope}%
\pgfsys@transformshift{2.522612in}{1.836876in}%
\pgfsys@useobject{currentmarker}{}%
\end{pgfscope}%
\begin{pgfscope}%
\pgfsys@transformshift{2.537327in}{1.809246in}%
\pgfsys@useobject{currentmarker}{}%
\end{pgfscope}%
\begin{pgfscope}%
\pgfsys@transformshift{2.552042in}{1.717090in}%
\pgfsys@useobject{currentmarker}{}%
\end{pgfscope}%
\begin{pgfscope}%
\pgfsys@transformshift{2.566758in}{1.583482in}%
\pgfsys@useobject{currentmarker}{}%
\end{pgfscope}%
\begin{pgfscope}%
\pgfsys@transformshift{2.581473in}{1.523889in}%
\pgfsys@useobject{currentmarker}{}%
\end{pgfscope}%
\begin{pgfscope}%
\pgfsys@transformshift{2.596188in}{1.515421in}%
\pgfsys@useobject{currentmarker}{}%
\end{pgfscope}%
\begin{pgfscope}%
\pgfsys@transformshift{2.610904in}{1.599895in}%
\pgfsys@useobject{currentmarker}{}%
\end{pgfscope}%
\begin{pgfscope}%
\pgfsys@transformshift{2.625619in}{1.668202in}%
\pgfsys@useobject{currentmarker}{}%
\end{pgfscope}%
\begin{pgfscope}%
\pgfsys@transformshift{2.640334in}{1.744451in}%
\pgfsys@useobject{currentmarker}{}%
\end{pgfscope}%
\begin{pgfscope}%
\pgfsys@transformshift{2.655050in}{1.811203in}%
\pgfsys@useobject{currentmarker}{}%
\end{pgfscope}%
\begin{pgfscope}%
\pgfsys@transformshift{2.669765in}{1.861214in}%
\pgfsys@useobject{currentmarker}{}%
\end{pgfscope}%
\begin{pgfscope}%
\pgfsys@transformshift{2.684480in}{1.886467in}%
\pgfsys@useobject{currentmarker}{}%
\end{pgfscope}%
\begin{pgfscope}%
\pgfsys@transformshift{2.699196in}{1.917320in}%
\pgfsys@useobject{currentmarker}{}%
\end{pgfscope}%
\begin{pgfscope}%
\pgfsys@transformshift{2.713911in}{1.906216in}%
\pgfsys@useobject{currentmarker}{}%
\end{pgfscope}%
\begin{pgfscope}%
\pgfsys@transformshift{2.728626in}{1.828829in}%
\pgfsys@useobject{currentmarker}{}%
\end{pgfscope}%
\begin{pgfscope}%
\pgfsys@transformshift{2.743342in}{1.708741in}%
\pgfsys@useobject{currentmarker}{}%
\end{pgfscope}%
\begin{pgfscope}%
\pgfsys@transformshift{2.758057in}{1.605004in}%
\pgfsys@useobject{currentmarker}{}%
\end{pgfscope}%
\begin{pgfscope}%
\pgfsys@transformshift{2.772772in}{1.606250in}%
\pgfsys@useobject{currentmarker}{}%
\end{pgfscope}%
\begin{pgfscope}%
\pgfsys@transformshift{2.787488in}{1.680121in}%
\pgfsys@useobject{currentmarker}{}%
\end{pgfscope}%
\begin{pgfscope}%
\pgfsys@transformshift{2.802203in}{1.757359in}%
\pgfsys@useobject{currentmarker}{}%
\end{pgfscope}%
\begin{pgfscope}%
\pgfsys@transformshift{2.816918in}{1.803530in}%
\pgfsys@useobject{currentmarker}{}%
\end{pgfscope}%
\begin{pgfscope}%
\pgfsys@transformshift{2.831633in}{1.858169in}%
\pgfsys@useobject{currentmarker}{}%
\end{pgfscope}%
\begin{pgfscope}%
\pgfsys@transformshift{2.846349in}{1.882998in}%
\pgfsys@useobject{currentmarker}{}%
\end{pgfscope}%
\begin{pgfscope}%
\pgfsys@transformshift{2.861064in}{1.988554in}%
\pgfsys@useobject{currentmarker}{}%
\end{pgfscope}%
\begin{pgfscope}%
\pgfsys@transformshift{2.875779in}{2.003172in}%
\pgfsys@useobject{currentmarker}{}%
\end{pgfscope}%
\begin{pgfscope}%
\pgfsys@transformshift{2.890495in}{1.940849in}%
\pgfsys@useobject{currentmarker}{}%
\end{pgfscope}%
\begin{pgfscope}%
\pgfsys@transformshift{2.905210in}{1.874601in}%
\pgfsys@useobject{currentmarker}{}%
\end{pgfscope}%
\begin{pgfscope}%
\pgfsys@transformshift{2.919925in}{1.757569in}%
\pgfsys@useobject{currentmarker}{}%
\end{pgfscope}%
\begin{pgfscope}%
\pgfsys@transformshift{2.934641in}{1.646651in}%
\pgfsys@useobject{currentmarker}{}%
\end{pgfscope}%
\begin{pgfscope}%
\pgfsys@transformshift{2.949356in}{1.676619in}%
\pgfsys@useobject{currentmarker}{}%
\end{pgfscope}%
\begin{pgfscope}%
\pgfsys@transformshift{2.964071in}{1.770213in}%
\pgfsys@useobject{currentmarker}{}%
\end{pgfscope}%
\begin{pgfscope}%
\pgfsys@transformshift{2.978787in}{1.875706in}%
\pgfsys@useobject{currentmarker}{}%
\end{pgfscope}%
\begin{pgfscope}%
\pgfsys@transformshift{2.993502in}{1.924661in}%
\pgfsys@useobject{currentmarker}{}%
\end{pgfscope}%
\begin{pgfscope}%
\pgfsys@transformshift{3.008217in}{1.965370in}%
\pgfsys@useobject{currentmarker}{}%
\end{pgfscope}%
\begin{pgfscope}%
\pgfsys@transformshift{3.022933in}{2.036452in}%
\pgfsys@useobject{currentmarker}{}%
\end{pgfscope}%
\begin{pgfscope}%
\pgfsys@transformshift{3.037648in}{2.116879in}%
\pgfsys@useobject{currentmarker}{}%
\end{pgfscope}%
\begin{pgfscope}%
\pgfsys@transformshift{3.052363in}{2.160322in}%
\pgfsys@useobject{currentmarker}{}%
\end{pgfscope}%
\begin{pgfscope}%
\pgfsys@transformshift{3.067079in}{2.135205in}%
\pgfsys@useobject{currentmarker}{}%
\end{pgfscope}%
\begin{pgfscope}%
\pgfsys@transformshift{3.081794in}{2.071972in}%
\pgfsys@useobject{currentmarker}{}%
\end{pgfscope}%
\begin{pgfscope}%
\pgfsys@transformshift{3.096509in}{1.966858in}%
\pgfsys@useobject{currentmarker}{}%
\end{pgfscope}%
\begin{pgfscope}%
\pgfsys@transformshift{3.111224in}{1.856132in}%
\pgfsys@useobject{currentmarker}{}%
\end{pgfscope}%
\begin{pgfscope}%
\pgfsys@transformshift{3.125940in}{1.876873in}%
\pgfsys@useobject{currentmarker}{}%
\end{pgfscope}%
\begin{pgfscope}%
\pgfsys@transformshift{3.140655in}{1.939208in}%
\pgfsys@useobject{currentmarker}{}%
\end{pgfscope}%
\begin{pgfscope}%
\pgfsys@transformshift{3.155370in}{2.035861in}%
\pgfsys@useobject{currentmarker}{}%
\end{pgfscope}%
\begin{pgfscope}%
\pgfsys@transformshift{3.170086in}{2.090478in}%
\pgfsys@useobject{currentmarker}{}%
\end{pgfscope}%
\begin{pgfscope}%
\pgfsys@transformshift{3.184801in}{2.140813in}%
\pgfsys@useobject{currentmarker}{}%
\end{pgfscope}%
\begin{pgfscope}%
\pgfsys@transformshift{3.199516in}{2.177751in}%
\pgfsys@useobject{currentmarker}{}%
\end{pgfscope}%
\begin{pgfscope}%
\pgfsys@transformshift{3.214232in}{2.251224in}%
\pgfsys@useobject{currentmarker}{}%
\end{pgfscope}%
\begin{pgfscope}%
\pgfsys@transformshift{3.228947in}{2.248600in}%
\pgfsys@useobject{currentmarker}{}%
\end{pgfscope}%
\begin{pgfscope}%
\pgfsys@transformshift{3.243662in}{2.213266in}%
\pgfsys@useobject{currentmarker}{}%
\end{pgfscope}%
\begin{pgfscope}%
\pgfsys@transformshift{3.258378in}{2.160034in}%
\pgfsys@useobject{currentmarker}{}%
\end{pgfscope}%
\begin{pgfscope}%
\pgfsys@transformshift{3.273093in}{2.018952in}%
\pgfsys@useobject{currentmarker}{}%
\end{pgfscope}%
\begin{pgfscope}%
\pgfsys@transformshift{3.287808in}{1.899710in}%
\pgfsys@useobject{currentmarker}{}%
\end{pgfscope}%
\begin{pgfscope}%
\pgfsys@transformshift{3.302524in}{1.932267in}%
\pgfsys@useobject{currentmarker}{}%
\end{pgfscope}%
\begin{pgfscope}%
\pgfsys@transformshift{3.317239in}{2.008121in}%
\pgfsys@useobject{currentmarker}{}%
\end{pgfscope}%
\begin{pgfscope}%
\pgfsys@transformshift{3.331954in}{2.087569in}%
\pgfsys@useobject{currentmarker}{}%
\end{pgfscope}%
\begin{pgfscope}%
\pgfsys@transformshift{3.346670in}{2.152937in}%
\pgfsys@useobject{currentmarker}{}%
\end{pgfscope}%
\begin{pgfscope}%
\pgfsys@transformshift{3.361385in}{2.171914in}%
\pgfsys@useobject{currentmarker}{}%
\end{pgfscope}%
\begin{pgfscope}%
\pgfsys@transformshift{3.376100in}{2.227382in}%
\pgfsys@useobject{currentmarker}{}%
\end{pgfscope}%
\begin{pgfscope}%
\pgfsys@transformshift{3.390815in}{2.296237in}%
\pgfsys@useobject{currentmarker}{}%
\end{pgfscope}%
\begin{pgfscope}%
\pgfsys@transformshift{3.405531in}{2.284008in}%
\pgfsys@useobject{currentmarker}{}%
\end{pgfscope}%
\begin{pgfscope}%
\pgfsys@transformshift{3.420246in}{2.290340in}%
\pgfsys@useobject{currentmarker}{}%
\end{pgfscope}%
\begin{pgfscope}%
\pgfsys@transformshift{3.434961in}{2.189405in}%
\pgfsys@useobject{currentmarker}{}%
\end{pgfscope}%
\begin{pgfscope}%
\pgfsys@transformshift{3.449677in}{2.092988in}%
\pgfsys@useobject{currentmarker}{}%
\end{pgfscope}%
\begin{pgfscope}%
\pgfsys@transformshift{3.464392in}{2.024112in}%
\pgfsys@useobject{currentmarker}{}%
\end{pgfscope}%
\begin{pgfscope}%
\pgfsys@transformshift{3.479107in}{2.025919in}%
\pgfsys@useobject{currentmarker}{}%
\end{pgfscope}%
\begin{pgfscope}%
\pgfsys@transformshift{3.493823in}{2.102225in}%
\pgfsys@useobject{currentmarker}{}%
\end{pgfscope}%
\begin{pgfscope}%
\pgfsys@transformshift{3.508538in}{2.176941in}%
\pgfsys@useobject{currentmarker}{}%
\end{pgfscope}%
\begin{pgfscope}%
\pgfsys@transformshift{3.523253in}{2.227966in}%
\pgfsys@useobject{currentmarker}{}%
\end{pgfscope}%
\begin{pgfscope}%
\pgfsys@transformshift{3.537969in}{2.281327in}%
\pgfsys@useobject{currentmarker}{}%
\end{pgfscope}%
\begin{pgfscope}%
\pgfsys@transformshift{3.552684in}{2.332186in}%
\pgfsys@useobject{currentmarker}{}%
\end{pgfscope}%
\begin{pgfscope}%
\pgfsys@transformshift{3.567399in}{2.385082in}%
\pgfsys@useobject{currentmarker}{}%
\end{pgfscope}%
\begin{pgfscope}%
\pgfsys@transformshift{3.582115in}{2.412631in}%
\pgfsys@useobject{currentmarker}{}%
\end{pgfscope}%
\begin{pgfscope}%
\pgfsys@transformshift{3.596830in}{2.374666in}%
\pgfsys@useobject{currentmarker}{}%
\end{pgfscope}%
\begin{pgfscope}%
\pgfsys@transformshift{3.611545in}{2.282164in}%
\pgfsys@useobject{currentmarker}{}%
\end{pgfscope}%
\begin{pgfscope}%
\pgfsys@transformshift{3.626261in}{2.166161in}%
\pgfsys@useobject{currentmarker}{}%
\end{pgfscope}%
\begin{pgfscope}%
\pgfsys@transformshift{3.640976in}{2.092151in}%
\pgfsys@useobject{currentmarker}{}%
\end{pgfscope}%
\begin{pgfscope}%
\pgfsys@transformshift{3.655691in}{2.106868in}%
\pgfsys@useobject{currentmarker}{}%
\end{pgfscope}%
\begin{pgfscope}%
\pgfsys@transformshift{3.670406in}{2.183364in}%
\pgfsys@useobject{currentmarker}{}%
\end{pgfscope}%
\begin{pgfscope}%
\pgfsys@transformshift{3.685122in}{2.264853in}%
\pgfsys@useobject{currentmarker}{}%
\end{pgfscope}%
\begin{pgfscope}%
\pgfsys@transformshift{3.699837in}{2.336196in}%
\pgfsys@useobject{currentmarker}{}%
\end{pgfscope}%
\begin{pgfscope}%
\pgfsys@transformshift{3.714552in}{2.375640in}%
\pgfsys@useobject{currentmarker}{}%
\end{pgfscope}%
\begin{pgfscope}%
\pgfsys@transformshift{3.729268in}{2.424982in}%
\pgfsys@useobject{currentmarker}{}%
\end{pgfscope}%
\begin{pgfscope}%
\pgfsys@transformshift{3.743983in}{2.478150in}%
\pgfsys@useobject{currentmarker}{}%
\end{pgfscope}%
\begin{pgfscope}%
\pgfsys@transformshift{3.758698in}{2.520284in}%
\pgfsys@useobject{currentmarker}{}%
\end{pgfscope}%
\begin{pgfscope}%
\pgfsys@transformshift{3.773414in}{2.507524in}%
\pgfsys@useobject{currentmarker}{}%
\end{pgfscope}%
\begin{pgfscope}%
\pgfsys@transformshift{3.788129in}{2.424904in}%
\pgfsys@useobject{currentmarker}{}%
\end{pgfscope}%
\begin{pgfscope}%
\pgfsys@transformshift{3.802844in}{2.299576in}%
\pgfsys@useobject{currentmarker}{}%
\end{pgfscope}%
\begin{pgfscope}%
\pgfsys@transformshift{3.817560in}{2.236134in}%
\pgfsys@useobject{currentmarker}{}%
\end{pgfscope}%
\begin{pgfscope}%
\pgfsys@transformshift{3.832275in}{2.226608in}%
\pgfsys@useobject{currentmarker}{}%
\end{pgfscope}%
\begin{pgfscope}%
\pgfsys@transformshift{3.846990in}{2.321823in}%
\pgfsys@useobject{currentmarker}{}%
\end{pgfscope}%
\begin{pgfscope}%
\pgfsys@transformshift{3.861706in}{2.409897in}%
\pgfsys@useobject{currentmarker}{}%
\end{pgfscope}%
\begin{pgfscope}%
\pgfsys@transformshift{3.876421in}{2.471099in}%
\pgfsys@useobject{currentmarker}{}%
\end{pgfscope}%
\begin{pgfscope}%
\pgfsys@transformshift{3.891136in}{2.513560in}%
\pgfsys@useobject{currentmarker}{}%
\end{pgfscope}%
\begin{pgfscope}%
\pgfsys@transformshift{3.905852in}{2.559572in}%
\pgfsys@useobject{currentmarker}{}%
\end{pgfscope}%
\begin{pgfscope}%
\pgfsys@transformshift{3.920567in}{2.632302in}%
\pgfsys@useobject{currentmarker}{}%
\end{pgfscope}%
\begin{pgfscope}%
\pgfsys@transformshift{3.935282in}{2.681032in}%
\pgfsys@useobject{currentmarker}{}%
\end{pgfscope}%
\begin{pgfscope}%
\pgfsys@transformshift{3.949997in}{2.660969in}%
\pgfsys@useobject{currentmarker}{}%
\end{pgfscope}%
\begin{pgfscope}%
\pgfsys@transformshift{3.964713in}{2.572562in}%
\pgfsys@useobject{currentmarker}{}%
\end{pgfscope}%
\begin{pgfscope}%
\pgfsys@transformshift{3.979428in}{2.453201in}%
\pgfsys@useobject{currentmarker}{}%
\end{pgfscope}%
\begin{pgfscope}%
\pgfsys@transformshift{3.994143in}{2.372235in}%
\pgfsys@useobject{currentmarker}{}%
\end{pgfscope}%
\begin{pgfscope}%
\pgfsys@transformshift{4.008859in}{2.372099in}%
\pgfsys@useobject{currentmarker}{}%
\end{pgfscope}%
\begin{pgfscope}%
\pgfsys@transformshift{4.023574in}{2.463749in}%
\pgfsys@useobject{currentmarker}{}%
\end{pgfscope}%
\begin{pgfscope}%
\pgfsys@transformshift{4.038289in}{2.530729in}%
\pgfsys@useobject{currentmarker}{}%
\end{pgfscope}%
\begin{pgfscope}%
\pgfsys@transformshift{4.053005in}{2.589786in}%
\pgfsys@useobject{currentmarker}{}%
\end{pgfscope}%
\begin{pgfscope}%
\pgfsys@transformshift{4.067720in}{2.640163in}%
\pgfsys@useobject{currentmarker}{}%
\end{pgfscope}%
\begin{pgfscope}%
\pgfsys@transformshift{4.082435in}{2.697827in}%
\pgfsys@useobject{currentmarker}{}%
\end{pgfscope}%
\begin{pgfscope}%
\pgfsys@transformshift{4.097151in}{2.780815in}%
\pgfsys@useobject{currentmarker}{}%
\end{pgfscope}%
\begin{pgfscope}%
\pgfsys@transformshift{4.111866in}{2.795046in}%
\pgfsys@useobject{currentmarker}{}%
\end{pgfscope}%
\begin{pgfscope}%
\pgfsys@transformshift{4.126581in}{2.746756in}%
\pgfsys@useobject{currentmarker}{}%
\end{pgfscope}%
\begin{pgfscope}%
\pgfsys@transformshift{4.141297in}{2.614488in}%
\pgfsys@useobject{currentmarker}{}%
\end{pgfscope}%
\begin{pgfscope}%
\pgfsys@transformshift{4.156012in}{2.546756in}%
\pgfsys@useobject{currentmarker}{}%
\end{pgfscope}%
\begin{pgfscope}%
\pgfsys@transformshift{4.170727in}{2.438426in}%
\pgfsys@useobject{currentmarker}{}%
\end{pgfscope}%
\begin{pgfscope}%
\pgfsys@transformshift{4.185443in}{2.447835in}%
\pgfsys@useobject{currentmarker}{}%
\end{pgfscope}%
\begin{pgfscope}%
\pgfsys@transformshift{4.200158in}{2.543726in}%
\pgfsys@useobject{currentmarker}{}%
\end{pgfscope}%
\begin{pgfscope}%
\pgfsys@transformshift{4.214873in}{2.618263in}%
\pgfsys@useobject{currentmarker}{}%
\end{pgfscope}%
\begin{pgfscope}%
\pgfsys@transformshift{4.229589in}{2.672363in}%
\pgfsys@useobject{currentmarker}{}%
\end{pgfscope}%
\begin{pgfscope}%
\pgfsys@transformshift{4.244304in}{2.745674in}%
\pgfsys@useobject{currentmarker}{}%
\end{pgfscope}%
\begin{pgfscope}%
\pgfsys@transformshift{4.259019in}{2.806202in}%
\pgfsys@useobject{currentmarker}{}%
\end{pgfscope}%
\begin{pgfscope}%
\pgfsys@transformshift{4.273734in}{2.889964in}%
\pgfsys@useobject{currentmarker}{}%
\end{pgfscope}%
\begin{pgfscope}%
\pgfsys@transformshift{4.288450in}{2.898893in}%
\pgfsys@useobject{currentmarker}{}%
\end{pgfscope}%
\begin{pgfscope}%
\pgfsys@transformshift{4.303165in}{2.886417in}%
\pgfsys@useobject{currentmarker}{}%
\end{pgfscope}%
\begin{pgfscope}%
\pgfsys@transformshift{4.317880in}{2.800754in}%
\pgfsys@useobject{currentmarker}{}%
\end{pgfscope}%
\begin{pgfscope}%
\pgfsys@transformshift{4.332596in}{2.692167in}%
\pgfsys@useobject{currentmarker}{}%
\end{pgfscope}%
\begin{pgfscope}%
\pgfsys@transformshift{4.347311in}{2.573537in}%
\pgfsys@useobject{currentmarker}{}%
\end{pgfscope}%
\begin{pgfscope}%
\pgfsys@transformshift{4.362026in}{2.581048in}%
\pgfsys@useobject{currentmarker}{}%
\end{pgfscope}%
\begin{pgfscope}%
\pgfsys@transformshift{4.376742in}{2.669507in}%
\pgfsys@useobject{currentmarker}{}%
\end{pgfscope}%
\begin{pgfscope}%
\pgfsys@transformshift{4.391457in}{2.765983in}%
\pgfsys@useobject{currentmarker}{}%
\end{pgfscope}%
\begin{pgfscope}%
\pgfsys@transformshift{4.406172in}{2.841474in}%
\pgfsys@useobject{currentmarker}{}%
\end{pgfscope}%
\begin{pgfscope}%
\pgfsys@transformshift{4.420888in}{2.886395in}%
\pgfsys@useobject{currentmarker}{}%
\end{pgfscope}%
\begin{pgfscope}%
\pgfsys@transformshift{4.435603in}{2.911579in}%
\pgfsys@useobject{currentmarker}{}%
\end{pgfscope}%
\begin{pgfscope}%
\pgfsys@transformshift{4.450318in}{3.025251in}%
\pgfsys@useobject{currentmarker}{}%
\end{pgfscope}%
\begin{pgfscope}%
\pgfsys@transformshift{4.465034in}{3.044627in}%
\pgfsys@useobject{currentmarker}{}%
\end{pgfscope}%
\begin{pgfscope}%
\pgfsys@transformshift{4.479749in}{2.989403in}%
\pgfsys@useobject{currentmarker}{}%
\end{pgfscope}%
\begin{pgfscope}%
\pgfsys@transformshift{4.494464in}{2.898926in}%
\pgfsys@useobject{currentmarker}{}%
\end{pgfscope}%
\begin{pgfscope}%
\pgfsys@transformshift{4.509180in}{2.782098in}%
\pgfsys@useobject{currentmarker}{}%
\end{pgfscope}%
\begin{pgfscope}%
\pgfsys@transformshift{4.523895in}{2.693424in}%
\pgfsys@useobject{currentmarker}{}%
\end{pgfscope}%
\begin{pgfscope}%
\pgfsys@transformshift{4.538610in}{2.711437in}%
\pgfsys@useobject{currentmarker}{}%
\end{pgfscope}%
\begin{pgfscope}%
\pgfsys@transformshift{4.553325in}{2.768121in}%
\pgfsys@useobject{currentmarker}{}%
\end{pgfscope}%
\begin{pgfscope}%
\pgfsys@transformshift{4.568041in}{2.861524in}%
\pgfsys@useobject{currentmarker}{}%
\end{pgfscope}%
\begin{pgfscope}%
\pgfsys@transformshift{4.582756in}{2.925864in}%
\pgfsys@useobject{currentmarker}{}%
\end{pgfscope}%
\begin{pgfscope}%
\pgfsys@transformshift{4.597471in}{2.981027in}%
\pgfsys@useobject{currentmarker}{}%
\end{pgfscope}%
\begin{pgfscope}%
\pgfsys@transformshift{4.612187in}{3.019895in}%
\pgfsys@useobject{currentmarker}{}%
\end{pgfscope}%
\begin{pgfscope}%
\pgfsys@transformshift{4.626902in}{3.130071in}%
\pgfsys@useobject{currentmarker}{}%
\end{pgfscope}%
\begin{pgfscope}%
\pgfsys@transformshift{4.641617in}{3.132820in}%
\pgfsys@useobject{currentmarker}{}%
\end{pgfscope}%
\begin{pgfscope}%
\pgfsys@transformshift{4.656333in}{3.109392in}%
\pgfsys@useobject{currentmarker}{}%
\end{pgfscope}%
\begin{pgfscope}%
\pgfsys@transformshift{4.671048in}{3.021191in}%
\pgfsys@useobject{currentmarker}{}%
\end{pgfscope}%
\begin{pgfscope}%
\pgfsys@transformshift{4.685763in}{2.879566in}%
\pgfsys@useobject{currentmarker}{}%
\end{pgfscope}%
\begin{pgfscope}%
\pgfsys@transformshift{4.700479in}{2.815185in}%
\pgfsys@useobject{currentmarker}{}%
\end{pgfscope}%
\begin{pgfscope}%
\pgfsys@transformshift{4.715194in}{2.827623in}%
\pgfsys@useobject{currentmarker}{}%
\end{pgfscope}%
\begin{pgfscope}%
\pgfsys@transformshift{4.729909in}{2.902540in}%
\pgfsys@useobject{currentmarker}{}%
\end{pgfscope}%
\begin{pgfscope}%
\pgfsys@transformshift{4.744625in}{2.989517in}%
\pgfsys@useobject{currentmarker}{}%
\end{pgfscope}%
\begin{pgfscope}%
\pgfsys@transformshift{4.759340in}{3.076353in}%
\pgfsys@useobject{currentmarker}{}%
\end{pgfscope}%
\begin{pgfscope}%
\pgfsys@transformshift{4.774055in}{3.091714in}%
\pgfsys@useobject{currentmarker}{}%
\end{pgfscope}%
\begin{pgfscope}%
\pgfsys@transformshift{4.788771in}{3.105264in}%
\pgfsys@useobject{currentmarker}{}%
\end{pgfscope}%
\begin{pgfscope}%
\pgfsys@transformshift{4.803486in}{3.168330in}%
\pgfsys@useobject{currentmarker}{}%
\end{pgfscope}%
\begin{pgfscope}%
\pgfsys@transformshift{4.818201in}{3.245822in}%
\pgfsys@useobject{currentmarker}{}%
\end{pgfscope}%
\begin{pgfscope}%
\pgfsys@transformshift{4.832916in}{3.207878in}%
\pgfsys@useobject{currentmarker}{}%
\end{pgfscope}%
\begin{pgfscope}%
\pgfsys@transformshift{4.847632in}{3.081794in}%
\pgfsys@useobject{currentmarker}{}%
\end{pgfscope}%
\begin{pgfscope}%
\pgfsys@transformshift{4.862347in}{2.994873in}%
\pgfsys@useobject{currentmarker}{}%
\end{pgfscope}%
\begin{pgfscope}%
\pgfsys@transformshift{4.877062in}{2.942665in}%
\pgfsys@useobject{currentmarker}{}%
\end{pgfscope}%
\begin{pgfscope}%
\pgfsys@transformshift{4.891778in}{2.927964in}%
\pgfsys@useobject{currentmarker}{}%
\end{pgfscope}%
\begin{pgfscope}%
\pgfsys@transformshift{4.906493in}{2.999466in}%
\pgfsys@useobject{currentmarker}{}%
\end{pgfscope}%
\begin{pgfscope}%
\pgfsys@transformshift{4.921208in}{3.074662in}%
\pgfsys@useobject{currentmarker}{}%
\end{pgfscope}%
\begin{pgfscope}%
\pgfsys@transformshift{4.935924in}{3.154312in}%
\pgfsys@useobject{currentmarker}{}%
\end{pgfscope}%
\begin{pgfscope}%
\pgfsys@transformshift{4.950639in}{3.185292in}%
\pgfsys@useobject{currentmarker}{}%
\end{pgfscope}%
\begin{pgfscope}%
\pgfsys@transformshift{4.965354in}{3.260631in}%
\pgfsys@useobject{currentmarker}{}%
\end{pgfscope}%
\begin{pgfscope}%
\pgfsys@transformshift{4.980070in}{3.301160in}%
\pgfsys@useobject{currentmarker}{}%
\end{pgfscope}%
\begin{pgfscope}%
\pgfsys@transformshift{4.994785in}{3.335016in}%
\pgfsys@useobject{currentmarker}{}%
\end{pgfscope}%
\begin{pgfscope}%
\pgfsys@transformshift{5.009500in}{3.298668in}%
\pgfsys@useobject{currentmarker}{}%
\end{pgfscope}%
\begin{pgfscope}%
\pgfsys@transformshift{5.024216in}{3.215500in}%
\pgfsys@useobject{currentmarker}{}%
\end{pgfscope}%
\begin{pgfscope}%
\pgfsys@transformshift{5.038931in}{3.104630in}%
\pgfsys@useobject{currentmarker}{}%
\end{pgfscope}%
\begin{pgfscope}%
\pgfsys@transformshift{5.053646in}{3.028827in}%
\pgfsys@useobject{currentmarker}{}%
\end{pgfscope}%
\begin{pgfscope}%
\pgfsys@transformshift{5.068362in}{3.012287in}%
\pgfsys@useobject{currentmarker}{}%
\end{pgfscope}%
\begin{pgfscope}%
\pgfsys@transformshift{5.083077in}{3.101575in}%
\pgfsys@useobject{currentmarker}{}%
\end{pgfscope}%
\begin{pgfscope}%
\pgfsys@transformshift{5.097792in}{3.178545in}%
\pgfsys@useobject{currentmarker}{}%
\end{pgfscope}%
\begin{pgfscope}%
\pgfsys@transformshift{5.112507in}{3.253105in}%
\pgfsys@useobject{currentmarker}{}%
\end{pgfscope}%
\begin{pgfscope}%
\pgfsys@transformshift{5.127223in}{3.337095in}%
\pgfsys@useobject{currentmarker}{}%
\end{pgfscope}%
\begin{pgfscope}%
\pgfsys@transformshift{5.141938in}{3.392402in}%
\pgfsys@useobject{currentmarker}{}%
\end{pgfscope}%
\begin{pgfscope}%
\pgfsys@transformshift{5.156653in}{3.465701in}%
\pgfsys@useobject{currentmarker}{}%
\end{pgfscope}%
\begin{pgfscope}%
\pgfsys@transformshift{5.171369in}{3.495727in}%
\pgfsys@useobject{currentmarker}{}%
\end{pgfscope}%
\begin{pgfscope}%
\pgfsys@transformshift{5.186084in}{3.450047in}%
\pgfsys@useobject{currentmarker}{}%
\end{pgfscope}%
\begin{pgfscope}%
\pgfsys@transformshift{5.200799in}{3.338679in}%
\pgfsys@useobject{currentmarker}{}%
\end{pgfscope}%
\begin{pgfscope}%
\pgfsys@transformshift{5.215515in}{3.233330in}%
\pgfsys@useobject{currentmarker}{}%
\end{pgfscope}%
\begin{pgfscope}%
\pgfsys@transformshift{5.230230in}{3.146527in}%
\pgfsys@useobject{currentmarker}{}%
\end{pgfscope}%
\begin{pgfscope}%
\pgfsys@transformshift{5.244945in}{3.169610in}%
\pgfsys@useobject{currentmarker}{}%
\end{pgfscope}%
\begin{pgfscope}%
\pgfsys@transformshift{5.259661in}{3.249291in}%
\pgfsys@useobject{currentmarker}{}%
\end{pgfscope}%
\begin{pgfscope}%
\pgfsys@transformshift{5.274376in}{3.313777in}%
\pgfsys@useobject{currentmarker}{}%
\end{pgfscope}%
\begin{pgfscope}%
\pgfsys@transformshift{5.289091in}{3.399271in}%
\pgfsys@useobject{currentmarker}{}%
\end{pgfscope}%
\begin{pgfscope}%
\pgfsys@transformshift{5.303807in}{3.430567in}%
\pgfsys@useobject{currentmarker}{}%
\end{pgfscope}%
\begin{pgfscope}%
\pgfsys@transformshift{5.318522in}{3.471233in}%
\pgfsys@useobject{currentmarker}{}%
\end{pgfscope}%
\begin{pgfscope}%
\pgfsys@transformshift{5.333237in}{3.507600in}%
\pgfsys@useobject{currentmarker}{}%
\end{pgfscope}%
\begin{pgfscope}%
\pgfsys@transformshift{5.347953in}{3.563484in}%
\pgfsys@useobject{currentmarker}{}%
\end{pgfscope}%
\begin{pgfscope}%
\pgfsys@transformshift{5.362668in}{3.535711in}%
\pgfsys@useobject{currentmarker}{}%
\end{pgfscope}%
\begin{pgfscope}%
\pgfsys@transformshift{5.377383in}{3.468350in}%
\pgfsys@useobject{currentmarker}{}%
\end{pgfscope}%
\begin{pgfscope}%
\pgfsys@transformshift{5.392098in}{3.336385in}%
\pgfsys@useobject{currentmarker}{}%
\end{pgfscope}%
\begin{pgfscope}%
\pgfsys@transformshift{5.406814in}{3.272073in}%
\pgfsys@useobject{currentmarker}{}%
\end{pgfscope}%
\begin{pgfscope}%
\pgfsys@transformshift{5.421529in}{3.269405in}%
\pgfsys@useobject{currentmarker}{}%
\end{pgfscope}%
\begin{pgfscope}%
\pgfsys@transformshift{5.436244in}{3.342383in}%
\pgfsys@useobject{currentmarker}{}%
\end{pgfscope}%
\begin{pgfscope}%
\pgfsys@transformshift{5.450960in}{3.429823in}%
\pgfsys@useobject{currentmarker}{}%
\end{pgfscope}%
\begin{pgfscope}%
\pgfsys@transformshift{5.465675in}{3.501866in}%
\pgfsys@useobject{currentmarker}{}%
\end{pgfscope}%
\begin{pgfscope}%
\pgfsys@transformshift{5.480390in}{3.543338in}%
\pgfsys@useobject{currentmarker}{}%
\end{pgfscope}%
\begin{pgfscope}%
\pgfsys@transformshift{5.495106in}{3.573237in}%
\pgfsys@useobject{currentmarker}{}%
\end{pgfscope}%
\begin{pgfscope}%
\pgfsys@transformshift{5.509821in}{3.675126in}%
\pgfsys@useobject{currentmarker}{}%
\end{pgfscope}%
\begin{pgfscope}%
\pgfsys@transformshift{5.524536in}{3.705927in}%
\pgfsys@useobject{currentmarker}{}%
\end{pgfscope}%
\begin{pgfscope}%
\pgfsys@transformshift{5.539252in}{3.648982in}%
\pgfsys@useobject{currentmarker}{}%
\end{pgfscope}%
\begin{pgfscope}%
\pgfsys@transformshift{5.553967in}{3.572721in}%
\pgfsys@useobject{currentmarker}{}%
\end{pgfscope}%
\begin{pgfscope}%
\pgfsys@transformshift{5.568682in}{3.459462in}%
\pgfsys@useobject{currentmarker}{}%
\end{pgfscope}%
\begin{pgfscope}%
\pgfsys@transformshift{5.583398in}{3.388437in}%
\pgfsys@useobject{currentmarker}{}%
\end{pgfscope}%
\begin{pgfscope}%
\pgfsys@transformshift{5.598113in}{3.386364in}%
\pgfsys@useobject{currentmarker}{}%
\end{pgfscope}%
\begin{pgfscope}%
\pgfsys@transformshift{5.612828in}{3.496506in}%
\pgfsys@useobject{currentmarker}{}%
\end{pgfscope}%
\begin{pgfscope}%
\pgfsys@transformshift{5.627544in}{3.572323in}%
\pgfsys@useobject{currentmarker}{}%
\end{pgfscope}%
\begin{pgfscope}%
\pgfsys@transformshift{5.642259in}{3.641984in}%
\pgfsys@useobject{currentmarker}{}%
\end{pgfscope}%
\begin{pgfscope}%
\pgfsys@transformshift{5.656974in}{3.711491in}%
\pgfsys@useobject{currentmarker}{}%
\end{pgfscope}%
\begin{pgfscope}%
\pgfsys@transformshift{5.671689in}{3.747861in}%
\pgfsys@useobject{currentmarker}{}%
\end{pgfscope}%
\begin{pgfscope}%
\pgfsys@transformshift{5.686405in}{3.801211in}%
\pgfsys@useobject{currentmarker}{}%
\end{pgfscope}%
\begin{pgfscope}%
\pgfsys@transformshift{5.701120in}{3.880383in}%
\pgfsys@useobject{currentmarker}{}%
\end{pgfscope}%
\begin{pgfscope}%
\pgfsys@transformshift{5.715835in}{3.814365in}%
\pgfsys@useobject{currentmarker}{}%
\end{pgfscope}%
\begin{pgfscope}%
\pgfsys@transformshift{5.730551in}{3.735358in}%
\pgfsys@useobject{currentmarker}{}%
\end{pgfscope}%
\begin{pgfscope}%
\pgfsys@transformshift{5.745266in}{3.619144in}%
\pgfsys@useobject{currentmarker}{}%
\end{pgfscope}%
\begin{pgfscope}%
\pgfsys@transformshift{5.759981in}{3.525009in}%
\pgfsys@useobject{currentmarker}{}%
\end{pgfscope}%
\begin{pgfscope}%
\pgfsys@transformshift{5.774697in}{3.535046in}%
\pgfsys@useobject{currentmarker}{}%
\end{pgfscope}%
\begin{pgfscope}%
\pgfsys@transformshift{5.789412in}{3.617279in}%
\pgfsys@useobject{currentmarker}{}%
\end{pgfscope}%
\begin{pgfscope}%
\pgfsys@transformshift{5.804127in}{3.711197in}%
\pgfsys@useobject{currentmarker}{}%
\end{pgfscope}%
\begin{pgfscope}%
\pgfsys@transformshift{5.818843in}{3.768746in}%
\pgfsys@useobject{currentmarker}{}%
\end{pgfscope}%
\begin{pgfscope}%
\pgfsys@transformshift{5.833558in}{3.781488in}%
\pgfsys@useobject{currentmarker}{}%
\end{pgfscope}%
\begin{pgfscope}%
\pgfsys@transformshift{5.848273in}{3.873330in}%
\pgfsys@useobject{currentmarker}{}%
\end{pgfscope}%
\begin{pgfscope}%
\pgfsys@transformshift{5.862989in}{3.963302in}%
\pgfsys@useobject{currentmarker}{}%
\end{pgfscope}%
\begin{pgfscope}%
\pgfsys@transformshift{5.877704in}{3.989235in}%
\pgfsys@useobject{currentmarker}{}%
\end{pgfscope}%
\begin{pgfscope}%
\pgfsys@transformshift{5.892419in}{3.960763in}%
\pgfsys@useobject{currentmarker}{}%
\end{pgfscope}%
\begin{pgfscope}%
\pgfsys@transformshift{5.907135in}{3.843413in}%
\pgfsys@useobject{currentmarker}{}%
\end{pgfscope}%
\begin{pgfscope}%
\pgfsys@transformshift{5.921850in}{3.718927in}%
\pgfsys@useobject{currentmarker}{}%
\end{pgfscope}%
\begin{pgfscope}%
\pgfsys@transformshift{5.936565in}{3.625038in}%
\pgfsys@useobject{currentmarker}{}%
\end{pgfscope}%
\begin{pgfscope}%
\pgfsys@transformshift{5.951280in}{3.660836in}%
\pgfsys@useobject{currentmarker}{}%
\end{pgfscope}%
\begin{pgfscope}%
\pgfsys@transformshift{5.965996in}{3.731089in}%
\pgfsys@useobject{currentmarker}{}%
\end{pgfscope}%
\begin{pgfscope}%
\pgfsys@transformshift{5.980711in}{3.828319in}%
\pgfsys@useobject{currentmarker}{}%
\end{pgfscope}%
\begin{pgfscope}%
\pgfsys@transformshift{5.995426in}{3.889209in}%
\pgfsys@useobject{currentmarker}{}%
\end{pgfscope}%
\begin{pgfscope}%
\pgfsys@transformshift{6.010142in}{3.910510in}%
\pgfsys@useobject{currentmarker}{}%
\end{pgfscope}%
\begin{pgfscope}%
\pgfsys@transformshift{6.024857in}{3.979245in}%
\pgfsys@useobject{currentmarker}{}%
\end{pgfscope}%
\begin{pgfscope}%
\pgfsys@transformshift{6.039572in}{4.069973in}%
\pgfsys@useobject{currentmarker}{}%
\end{pgfscope}%
\begin{pgfscope}%
\pgfsys@transformshift{6.054288in}{4.113220in}%
\pgfsys@useobject{currentmarker}{}%
\end{pgfscope}%
\begin{pgfscope}%
\pgfsys@transformshift{6.069003in}{4.045685in}%
\pgfsys@useobject{currentmarker}{}%
\end{pgfscope}%
\begin{pgfscope}%
\pgfsys@transformshift{6.083718in}{3.962188in}%
\pgfsys@useobject{currentmarker}{}%
\end{pgfscope}%
\begin{pgfscope}%
\pgfsys@transformshift{6.098434in}{3.827146in}%
\pgfsys@useobject{currentmarker}{}%
\end{pgfscope}%
\begin{pgfscope}%
\pgfsys@transformshift{6.113149in}{3.756187in}%
\pgfsys@useobject{currentmarker}{}%
\end{pgfscope}%
\begin{pgfscope}%
\pgfsys@transformshift{6.127864in}{3.794558in}%
\pgfsys@useobject{currentmarker}{}%
\end{pgfscope}%
\begin{pgfscope}%
\pgfsys@transformshift{6.142580in}{3.895673in}%
\pgfsys@useobject{currentmarker}{}%
\end{pgfscope}%
\begin{pgfscope}%
\pgfsys@transformshift{6.157295in}{3.995888in}%
\pgfsys@useobject{currentmarker}{}%
\end{pgfscope}%
\begin{pgfscope}%
\pgfsys@transformshift{6.172010in}{4.034614in}%
\pgfsys@useobject{currentmarker}{}%
\end{pgfscope}%
\begin{pgfscope}%
\pgfsys@transformshift{6.186726in}{4.121554in}%
\pgfsys@useobject{currentmarker}{}%
\end{pgfscope}%
\begin{pgfscope}%
\pgfsys@transformshift{6.201441in}{4.164638in}%
\pgfsys@useobject{currentmarker}{}%
\end{pgfscope}%
\begin{pgfscope}%
\pgfsys@transformshift{6.216156in}{4.309818in}%
\pgfsys@useobject{currentmarker}{}%
\end{pgfscope}%
\begin{pgfscope}%
\pgfsys@transformshift{6.230871in}{4.325629in}%
\pgfsys@useobject{currentmarker}{}%
\end{pgfscope}%
\begin{pgfscope}%
\pgfsys@transformshift{6.245587in}{4.274806in}%
\pgfsys@useobject{currentmarker}{}%
\end{pgfscope}%
\begin{pgfscope}%
\pgfsys@transformshift{6.260302in}{4.134470in}%
\pgfsys@useobject{currentmarker}{}%
\end{pgfscope}%
\begin{pgfscope}%
\pgfsys@transformshift{6.275017in}{4.016864in}%
\pgfsys@useobject{currentmarker}{}%
\end{pgfscope}%
\begin{pgfscope}%
\pgfsys@transformshift{6.289733in}{3.948910in}%
\pgfsys@useobject{currentmarker}{}%
\end{pgfscope}%
\begin{pgfscope}%
\pgfsys@transformshift{6.304448in}{3.980245in}%
\pgfsys@useobject{currentmarker}{}%
\end{pgfscope}%
\begin{pgfscope}%
\pgfsys@transformshift{6.319163in}{4.089049in}%
\pgfsys@useobject{currentmarker}{}%
\end{pgfscope}%
\begin{pgfscope}%
\pgfsys@transformshift{6.333879in}{4.142277in}%
\pgfsys@useobject{currentmarker}{}%
\end{pgfscope}%
\begin{pgfscope}%
\pgfsys@transformshift{6.348594in}{4.236713in}%
\pgfsys@useobject{currentmarker}{}%
\end{pgfscope}%
\begin{pgfscope}%
\pgfsys@transformshift{6.363309in}{4.254219in}%
\pgfsys@useobject{currentmarker}{}%
\end{pgfscope}%
\begin{pgfscope}%
\pgfsys@transformshift{6.378025in}{4.307159in}%
\pgfsys@useobject{currentmarker}{}%
\end{pgfscope}%
\begin{pgfscope}%
\pgfsys@transformshift{6.392740in}{4.400504in}%
\pgfsys@useobject{currentmarker}{}%
\end{pgfscope}%
\begin{pgfscope}%
\pgfsys@transformshift{6.407455in}{4.437431in}%
\pgfsys@useobject{currentmarker}{}%
\end{pgfscope}%
\begin{pgfscope}%
\pgfsys@transformshift{6.422171in}{4.392230in}%
\pgfsys@useobject{currentmarker}{}%
\end{pgfscope}%
\begin{pgfscope}%
\pgfsys@transformshift{6.436886in}{4.292025in}%
\pgfsys@useobject{currentmarker}{}%
\end{pgfscope}%
\begin{pgfscope}%
\pgfsys@transformshift{6.451601in}{4.177460in}%
\pgfsys@useobject{currentmarker}{}%
\end{pgfscope}%
\begin{pgfscope}%
\pgfsys@transformshift{6.466317in}{4.081765in}%
\pgfsys@useobject{currentmarker}{}%
\end{pgfscope}%
\begin{pgfscope}%
\pgfsys@transformshift{6.481032in}{4.094528in}%
\pgfsys@useobject{currentmarker}{}%
\end{pgfscope}%
\begin{pgfscope}%
\pgfsys@transformshift{6.495747in}{4.175039in}%
\pgfsys@useobject{currentmarker}{}%
\end{pgfscope}%
\begin{pgfscope}%
\pgfsys@transformshift{6.510463in}{4.274353in}%
\pgfsys@useobject{currentmarker}{}%
\end{pgfscope}%
\begin{pgfscope}%
\pgfsys@transformshift{6.525178in}{4.339667in}%
\pgfsys@useobject{currentmarker}{}%
\end{pgfscope}%
\begin{pgfscope}%
\pgfsys@transformshift{6.539893in}{4.360346in}%
\pgfsys@useobject{currentmarker}{}%
\end{pgfscope}%
\begin{pgfscope}%
\pgfsys@transformshift{6.554608in}{4.419924in}%
\pgfsys@useobject{currentmarker}{}%
\end{pgfscope}%
\begin{pgfscope}%
\pgfsys@transformshift{6.569324in}{4.468421in}%
\pgfsys@useobject{currentmarker}{}%
\end{pgfscope}%
\begin{pgfscope}%
\pgfsys@transformshift{6.584039in}{4.524851in}%
\pgfsys@useobject{currentmarker}{}%
\end{pgfscope}%
\begin{pgfscope}%
\pgfsys@transformshift{6.598754in}{4.500336in}%
\pgfsys@useobject{currentmarker}{}%
\end{pgfscope}%
\begin{pgfscope}%
\pgfsys@transformshift{6.613470in}{4.379127in}%
\pgfsys@useobject{currentmarker}{}%
\end{pgfscope}%
\begin{pgfscope}%
\pgfsys@transformshift{6.628185in}{4.283351in}%
\pgfsys@useobject{currentmarker}{}%
\end{pgfscope}%
\begin{pgfscope}%
\pgfsys@transformshift{6.642900in}{4.202149in}%
\pgfsys@useobject{currentmarker}{}%
\end{pgfscope}%
\begin{pgfscope}%
\pgfsys@transformshift{6.657616in}{4.228745in}%
\pgfsys@useobject{currentmarker}{}%
\end{pgfscope}%
\begin{pgfscope}%
\pgfsys@transformshift{6.672331in}{4.346402in}%
\pgfsys@useobject{currentmarker}{}%
\end{pgfscope}%
\begin{pgfscope}%
\pgfsys@transformshift{6.687046in}{4.403463in}%
\pgfsys@useobject{currentmarker}{}%
\end{pgfscope}%
\begin{pgfscope}%
\pgfsys@transformshift{6.701762in}{4.502812in}%
\pgfsys@useobject{currentmarker}{}%
\end{pgfscope}%
\begin{pgfscope}%
\pgfsys@transformshift{6.716477in}{4.554563in}%
\pgfsys@useobject{currentmarker}{}%
\end{pgfscope}%
\begin{pgfscope}%
\pgfsys@transformshift{6.731192in}{4.564685in}%
\pgfsys@useobject{currentmarker}{}%
\end{pgfscope}%
\begin{pgfscope}%
\pgfsys@transformshift{6.745908in}{4.644779in}%
\pgfsys@useobject{currentmarker}{}%
\end{pgfscope}%
\begin{pgfscope}%
\pgfsys@transformshift{6.760623in}{4.718489in}%
\pgfsys@useobject{currentmarker}{}%
\end{pgfscope}%
\begin{pgfscope}%
\pgfsys@transformshift{6.775338in}{4.677779in}%
\pgfsys@useobject{currentmarker}{}%
\end{pgfscope}%
\begin{pgfscope}%
\pgfsys@transformshift{6.790054in}{4.557252in}%
\pgfsys@useobject{currentmarker}{}%
\end{pgfscope}%
\begin{pgfscope}%
\pgfsys@transformshift{6.804769in}{4.455691in}%
\pgfsys@useobject{currentmarker}{}%
\end{pgfscope}%
\begin{pgfscope}%
\pgfsys@transformshift{6.819484in}{4.373935in}%
\pgfsys@useobject{currentmarker}{}%
\end{pgfscope}%
\begin{pgfscope}%
\pgfsys@transformshift{6.834199in}{4.371754in}%
\pgfsys@useobject{currentmarker}{}%
\end{pgfscope}%
\begin{pgfscope}%
\pgfsys@transformshift{6.848915in}{4.472495in}%
\pgfsys@useobject{currentmarker}{}%
\end{pgfscope}%
\begin{pgfscope}%
\pgfsys@transformshift{6.863630in}{4.554833in}%
\pgfsys@useobject{currentmarker}{}%
\end{pgfscope}%
\begin{pgfscope}%
\pgfsys@transformshift{6.878345in}{4.647259in}%
\pgfsys@useobject{currentmarker}{}%
\end{pgfscope}%
\begin{pgfscope}%
\pgfsys@transformshift{6.893061in}{4.688422in}%
\pgfsys@useobject{currentmarker}{}%
\end{pgfscope}%
\begin{pgfscope}%
\pgfsys@transformshift{6.907776in}{4.708840in}%
\pgfsys@useobject{currentmarker}{}%
\end{pgfscope}%
\begin{pgfscope}%
\pgfsys@transformshift{6.922491in}{4.806492in}%
\pgfsys@useobject{currentmarker}{}%
\end{pgfscope}%
\begin{pgfscope}%
\pgfsys@transformshift{6.937207in}{4.855076in}%
\pgfsys@useobject{currentmarker}{}%
\end{pgfscope}%
\begin{pgfscope}%
\pgfsys@transformshift{6.951922in}{4.812916in}%
\pgfsys@useobject{currentmarker}{}%
\end{pgfscope}%
\begin{pgfscope}%
\pgfsys@transformshift{6.966637in}{4.703776in}%
\pgfsys@useobject{currentmarker}{}%
\end{pgfscope}%
\begin{pgfscope}%
\pgfsys@transformshift{6.981353in}{4.594315in}%
\pgfsys@useobject{currentmarker}{}%
\end{pgfscope}%
\begin{pgfscope}%
\pgfsys@transformshift{6.996068in}{4.528666in}%
\pgfsys@useobject{currentmarker}{}%
\end{pgfscope}%
\begin{pgfscope}%
\pgfsys@transformshift{7.010783in}{4.527288in}%
\pgfsys@useobject{currentmarker}{}%
\end{pgfscope}%
\begin{pgfscope}%
\pgfsys@transformshift{7.025499in}{4.618743in}%
\pgfsys@useobject{currentmarker}{}%
\end{pgfscope}%
\begin{pgfscope}%
\pgfsys@transformshift{7.040214in}{4.681999in}%
\pgfsys@useobject{currentmarker}{}%
\end{pgfscope}%
\begin{pgfscope}%
\pgfsys@transformshift{7.054929in}{4.753762in}%
\pgfsys@useobject{currentmarker}{}%
\end{pgfscope}%
\begin{pgfscope}%
\pgfsys@transformshift{7.069645in}{4.823290in}%
\pgfsys@useobject{currentmarker}{}%
\end{pgfscope}%
\begin{pgfscope}%
\pgfsys@transformshift{7.084360in}{4.871264in}%
\pgfsys@useobject{currentmarker}{}%
\end{pgfscope}%
\begin{pgfscope}%
\pgfsys@transformshift{7.099075in}{4.952236in}%
\pgfsys@useobject{currentmarker}{}%
\end{pgfscope}%
\begin{pgfscope}%
\pgfsys@transformshift{7.113790in}{4.956988in}%
\pgfsys@useobject{currentmarker}{}%
\end{pgfscope}%
\begin{pgfscope}%
\pgfsys@transformshift{7.128506in}{4.948991in}%
\pgfsys@useobject{currentmarker}{}%
\end{pgfscope}%
\begin{pgfscope}%
\pgfsys@transformshift{7.143221in}{4.833328in}%
\pgfsys@useobject{currentmarker}{}%
\end{pgfscope}%
\begin{pgfscope}%
\pgfsys@transformshift{7.157936in}{4.699157in}%
\pgfsys@useobject{currentmarker}{}%
\end{pgfscope}%
\begin{pgfscope}%
\pgfsys@transformshift{7.172652in}{4.628028in}%
\pgfsys@useobject{currentmarker}{}%
\end{pgfscope}%
\begin{pgfscope}%
\pgfsys@transformshift{7.187367in}{4.662816in}%
\pgfsys@useobject{currentmarker}{}%
\end{pgfscope}%
\begin{pgfscope}%
\pgfsys@transformshift{7.202082in}{4.724724in}%
\pgfsys@useobject{currentmarker}{}%
\end{pgfscope}%
\begin{pgfscope}%
\pgfsys@transformshift{7.216798in}{4.821273in}%
\pgfsys@useobject{currentmarker}{}%
\end{pgfscope}%
\begin{pgfscope}%
\pgfsys@transformshift{7.231513in}{4.902077in}%
\pgfsys@useobject{currentmarker}{}%
\end{pgfscope}%
\begin{pgfscope}%
\pgfsys@transformshift{7.246228in}{4.965599in}%
\pgfsys@useobject{currentmarker}{}%
\end{pgfscope}%
\begin{pgfscope}%
\pgfsys@transformshift{7.260944in}{4.938496in}%
\pgfsys@useobject{currentmarker}{}%
\end{pgfscope}%
\begin{pgfscope}%
\pgfsys@transformshift{7.275659in}{5.020046in}%
\pgfsys@useobject{currentmarker}{}%
\end{pgfscope}%
\begin{pgfscope}%
\pgfsys@transformshift{7.290374in}{5.063044in}%
\pgfsys@useobject{currentmarker}{}%
\end{pgfscope}%
\begin{pgfscope}%
\pgfsys@transformshift{7.305090in}{5.061003in}%
\pgfsys@useobject{currentmarker}{}%
\end{pgfscope}%
\begin{pgfscope}%
\pgfsys@transformshift{7.319805in}{4.943641in}%
\pgfsys@useobject{currentmarker}{}%
\end{pgfscope}%
\begin{pgfscope}%
\pgfsys@transformshift{7.334520in}{4.849158in}%
\pgfsys@useobject{currentmarker}{}%
\end{pgfscope}%
\begin{pgfscope}%
\pgfsys@transformshift{7.349236in}{4.776324in}%
\pgfsys@useobject{currentmarker}{}%
\end{pgfscope}%
\begin{pgfscope}%
\pgfsys@transformshift{7.363951in}{4.767053in}%
\pgfsys@useobject{currentmarker}{}%
\end{pgfscope}%
\begin{pgfscope}%
\pgfsys@transformshift{7.378666in}{4.865884in}%
\pgfsys@useobject{currentmarker}{}%
\end{pgfscope}%
\begin{pgfscope}%
\pgfsys@transformshift{7.393381in}{4.980627in}%
\pgfsys@useobject{currentmarker}{}%
\end{pgfscope}%
\begin{pgfscope}%
\pgfsys@transformshift{7.408097in}{5.077119in}%
\pgfsys@useobject{currentmarker}{}%
\end{pgfscope}%
\begin{pgfscope}%
\pgfsys@transformshift{7.422812in}{5.145777in}%
\pgfsys@useobject{currentmarker}{}%
\end{pgfscope}%
\begin{pgfscope}%
\pgfsys@transformshift{7.437527in}{5.198205in}%
\pgfsys@useobject{currentmarker}{}%
\end{pgfscope}%
\begin{pgfscope}%
\pgfsys@transformshift{7.452243in}{5.222266in}%
\pgfsys@useobject{currentmarker}{}%
\end{pgfscope}%
\begin{pgfscope}%
\pgfsys@transformshift{7.466958in}{5.216280in}%
\pgfsys@useobject{currentmarker}{}%
\end{pgfscope}%
\begin{pgfscope}%
\pgfsys@transformshift{7.481673in}{5.177487in}%
\pgfsys@useobject{currentmarker}{}%
\end{pgfscope}%
\begin{pgfscope}%
\pgfsys@transformshift{7.496389in}{5.100016in}%
\pgfsys@useobject{currentmarker}{}%
\end{pgfscope}%
\begin{pgfscope}%
\pgfsys@transformshift{7.511104in}{4.992755in}%
\pgfsys@useobject{currentmarker}{}%
\end{pgfscope}%
\begin{pgfscope}%
\pgfsys@transformshift{7.525819in}{4.918483in}%
\pgfsys@useobject{currentmarker}{}%
\end{pgfscope}%
\begin{pgfscope}%
\pgfsys@transformshift{7.540535in}{4.942741in}%
\pgfsys@useobject{currentmarker}{}%
\end{pgfscope}%
\begin{pgfscope}%
\pgfsys@transformshift{7.555250in}{5.028274in}%
\pgfsys@useobject{currentmarker}{}%
\end{pgfscope}%
\begin{pgfscope}%
\pgfsys@transformshift{7.569965in}{5.115708in}%
\pgfsys@useobject{currentmarker}{}%
\end{pgfscope}%
\begin{pgfscope}%
\pgfsys@transformshift{7.584681in}{5.168722in}%
\pgfsys@useobject{currentmarker}{}%
\end{pgfscope}%
\begin{pgfscope}%
\pgfsys@transformshift{7.599396in}{5.266526in}%
\pgfsys@useobject{currentmarker}{}%
\end{pgfscope}%
\begin{pgfscope}%
\pgfsys@transformshift{7.614111in}{5.311312in}%
\pgfsys@useobject{currentmarker}{}%
\end{pgfscope}%
\begin{pgfscope}%
\pgfsys@transformshift{7.628827in}{5.379659in}%
\pgfsys@useobject{currentmarker}{}%
\end{pgfscope}%
\begin{pgfscope}%
\pgfsys@transformshift{7.643542in}{5.397878in}%
\pgfsys@useobject{currentmarker}{}%
\end{pgfscope}%
\begin{pgfscope}%
\pgfsys@transformshift{7.658257in}{5.396051in}%
\pgfsys@useobject{currentmarker}{}%
\end{pgfscope}%
\begin{pgfscope}%
\pgfsys@transformshift{7.672972in}{5.327932in}%
\pgfsys@useobject{currentmarker}{}%
\end{pgfscope}%
\begin{pgfscope}%
\pgfsys@transformshift{7.687688in}{5.176902in}%
\pgfsys@useobject{currentmarker}{}%
\end{pgfscope}%
\begin{pgfscope}%
\pgfsys@transformshift{7.702403in}{5.124796in}%
\pgfsys@useobject{currentmarker}{}%
\end{pgfscope}%
\begin{pgfscope}%
\pgfsys@transformshift{7.717118in}{5.145000in}%
\pgfsys@useobject{currentmarker}{}%
\end{pgfscope}%
\begin{pgfscope}%
\pgfsys@transformshift{7.731834in}{5.227074in}%
\pgfsys@useobject{currentmarker}{}%
\end{pgfscope}%
\begin{pgfscope}%
\pgfsys@transformshift{7.746549in}{5.314122in}%
\pgfsys@useobject{currentmarker}{}%
\end{pgfscope}%
\end{pgfscope}%
\begin{pgfscope}%
\pgfpathrectangle{\pgfqpoint{0.697913in}{0.559721in}}{\pgfqpoint{7.048636in}{4.990279in}}%
\pgfusepath{clip}%
\pgfsetrectcap%
\pgfsetroundjoin%
\pgfsetlinewidth{1.003750pt}%
\definecolor{currentstroke}{rgb}{0.000000,0.000000,0.000000}%
\pgfsetstrokecolor{currentstroke}%
\pgfsetdash{}{0pt}%
\pgfpathmoveto{\pgfqpoint{0.697913in}{0.772482in}}%
\pgfpathlineto{\pgfqpoint{0.712628in}{0.776708in}}%
\pgfpathlineto{\pgfqpoint{0.727343in}{0.778526in}}%
\pgfpathlineto{\pgfqpoint{0.742059in}{0.782814in}}%
\pgfpathlineto{\pgfqpoint{0.756774in}{0.793740in}}%
\pgfpathlineto{\pgfqpoint{0.771489in}{0.807925in}}%
\pgfpathlineto{\pgfqpoint{0.786205in}{0.824367in}}%
\pgfpathlineto{\pgfqpoint{0.800920in}{0.842143in}}%
\pgfpathlineto{\pgfqpoint{0.815635in}{0.850456in}}%
\pgfpathlineto{\pgfqpoint{0.830350in}{0.851458in}}%
\pgfpathlineto{\pgfqpoint{0.845066in}{0.850016in}}%
\pgfpathlineto{\pgfqpoint{0.859781in}{0.853475in}}%
\pgfpathlineto{\pgfqpoint{0.874496in}{0.855324in}}%
\pgfpathlineto{\pgfqpoint{0.903927in}{0.851477in}}%
\pgfpathlineto{\pgfqpoint{0.918642in}{0.854480in}}%
\pgfpathlineto{\pgfqpoint{0.933358in}{0.864389in}}%
\pgfpathlineto{\pgfqpoint{0.948073in}{0.879740in}}%
\pgfpathlineto{\pgfqpoint{0.962788in}{0.896539in}}%
\pgfpathlineto{\pgfqpoint{0.977504in}{0.910503in}}%
\pgfpathlineto{\pgfqpoint{0.992219in}{0.920809in}}%
\pgfpathlineto{\pgfqpoint{1.006934in}{0.920843in}}%
\pgfpathlineto{\pgfqpoint{1.021650in}{0.921381in}}%
\pgfpathlineto{\pgfqpoint{1.036365in}{0.924504in}}%
\pgfpathlineto{\pgfqpoint{1.051080in}{0.930748in}}%
\pgfpathlineto{\pgfqpoint{1.065796in}{0.934302in}}%
\pgfpathlineto{\pgfqpoint{1.080511in}{0.933994in}}%
\pgfpathlineto{\pgfqpoint{1.095226in}{0.938674in}}%
\pgfpathlineto{\pgfqpoint{1.109941in}{0.949747in}}%
\pgfpathlineto{\pgfqpoint{1.124657in}{0.967119in}}%
\pgfpathlineto{\pgfqpoint{1.139372in}{0.988431in}}%
\pgfpathlineto{\pgfqpoint{1.154087in}{1.002866in}}%
\pgfpathlineto{\pgfqpoint{1.168803in}{1.012367in}}%
\pgfpathlineto{\pgfqpoint{1.183518in}{1.015020in}}%
\pgfpathlineto{\pgfqpoint{1.198233in}{1.018469in}}%
\pgfpathlineto{\pgfqpoint{1.212949in}{1.021541in}}%
\pgfpathlineto{\pgfqpoint{1.227664in}{1.029429in}}%
\pgfpathlineto{\pgfqpoint{1.242379in}{1.039357in}}%
\pgfpathlineto{\pgfqpoint{1.257095in}{1.044620in}}%
\pgfpathlineto{\pgfqpoint{1.271810in}{1.052380in}}%
\pgfpathlineto{\pgfqpoint{1.286525in}{1.066091in}}%
\pgfpathlineto{\pgfqpoint{1.301241in}{1.083891in}}%
\pgfpathlineto{\pgfqpoint{1.315956in}{1.107352in}}%
\pgfpathlineto{\pgfqpoint{1.330671in}{1.126971in}}%
\pgfpathlineto{\pgfqpoint{1.345387in}{1.138285in}}%
\pgfpathlineto{\pgfqpoint{1.360102in}{1.145010in}}%
\pgfpathlineto{\pgfqpoint{1.389532in}{1.157016in}}%
\pgfpathlineto{\pgfqpoint{1.404248in}{1.163814in}}%
\pgfpathlineto{\pgfqpoint{1.418963in}{1.168754in}}%
\pgfpathlineto{\pgfqpoint{1.433678in}{1.170860in}}%
\pgfpathlineto{\pgfqpoint{1.448394in}{1.177881in}}%
\pgfpathlineto{\pgfqpoint{1.463109in}{1.188860in}}%
\pgfpathlineto{\pgfqpoint{1.477824in}{1.202468in}}%
\pgfpathlineto{\pgfqpoint{1.492540in}{1.221036in}}%
\pgfpathlineto{\pgfqpoint{1.507255in}{1.235942in}}%
\pgfpathlineto{\pgfqpoint{1.521970in}{1.239684in}}%
\pgfpathlineto{\pgfqpoint{1.536686in}{1.239129in}}%
\pgfpathlineto{\pgfqpoint{1.551401in}{1.238990in}}%
\pgfpathlineto{\pgfqpoint{1.566116in}{1.237994in}}%
\pgfpathlineto{\pgfqpoint{1.580832in}{1.240306in}}%
\pgfpathlineto{\pgfqpoint{1.595547in}{1.245559in}}%
\pgfpathlineto{\pgfqpoint{1.610262in}{1.245804in}}%
\pgfpathlineto{\pgfqpoint{1.624978in}{1.247963in}}%
\pgfpathlineto{\pgfqpoint{1.639693in}{1.258008in}}%
\pgfpathlineto{\pgfqpoint{1.654408in}{1.270090in}}%
\pgfpathlineto{\pgfqpoint{1.669124in}{1.285587in}}%
\pgfpathlineto{\pgfqpoint{1.683839in}{1.300576in}}%
\pgfpathlineto{\pgfqpoint{1.698554in}{1.305706in}}%
\pgfpathlineto{\pgfqpoint{1.713269in}{1.306321in}}%
\pgfpathlineto{\pgfqpoint{1.727985in}{1.307462in}}%
\pgfpathlineto{\pgfqpoint{1.742700in}{1.310348in}}%
\pgfpathlineto{\pgfqpoint{1.757415in}{1.309496in}}%
\pgfpathlineto{\pgfqpoint{1.772131in}{1.309777in}}%
\pgfpathlineto{\pgfqpoint{1.786846in}{1.314763in}}%
\pgfpathlineto{\pgfqpoint{1.801561in}{1.321906in}}%
\pgfpathlineto{\pgfqpoint{1.816277in}{1.336418in}}%
\pgfpathlineto{\pgfqpoint{1.845707in}{1.370317in}}%
\pgfpathlineto{\pgfqpoint{1.860423in}{1.382970in}}%
\pgfpathlineto{\pgfqpoint{1.875138in}{1.386201in}}%
\pgfpathlineto{\pgfqpoint{1.889853in}{1.382723in}}%
\pgfpathlineto{\pgfqpoint{1.904569in}{1.379957in}}%
\pgfpathlineto{\pgfqpoint{1.919284in}{1.381489in}}%
\pgfpathlineto{\pgfqpoint{1.933999in}{1.384160in}}%
\pgfpathlineto{\pgfqpoint{1.948715in}{1.383501in}}%
\pgfpathlineto{\pgfqpoint{1.963430in}{1.379303in}}%
\pgfpathlineto{\pgfqpoint{1.978145in}{1.379256in}}%
\pgfpathlineto{\pgfqpoint{1.992860in}{1.386849in}}%
\pgfpathlineto{\pgfqpoint{2.007576in}{1.402902in}}%
\pgfpathlineto{\pgfqpoint{2.022291in}{1.419951in}}%
\pgfpathlineto{\pgfqpoint{2.037006in}{1.434046in}}%
\pgfpathlineto{\pgfqpoint{2.051722in}{1.437016in}}%
\pgfpathlineto{\pgfqpoint{2.066437in}{1.434647in}}%
\pgfpathlineto{\pgfqpoint{2.081152in}{1.430399in}}%
\pgfpathlineto{\pgfqpoint{2.095868in}{1.426973in}}%
\pgfpathlineto{\pgfqpoint{2.110583in}{1.426453in}}%
\pgfpathlineto{\pgfqpoint{2.125298in}{1.423711in}}%
\pgfpathlineto{\pgfqpoint{2.140014in}{1.420511in}}%
\pgfpathlineto{\pgfqpoint{2.154729in}{1.418902in}}%
\pgfpathlineto{\pgfqpoint{2.169444in}{1.423368in}}%
\pgfpathlineto{\pgfqpoint{2.184160in}{1.436166in}}%
\pgfpathlineto{\pgfqpoint{2.198875in}{1.448575in}}%
\pgfpathlineto{\pgfqpoint{2.213590in}{1.462882in}}%
\pgfpathlineto{\pgfqpoint{2.228306in}{1.466387in}}%
\pgfpathlineto{\pgfqpoint{2.272451in}{1.464491in}}%
\pgfpathlineto{\pgfqpoint{2.287167in}{1.469085in}}%
\pgfpathlineto{\pgfqpoint{2.301882in}{1.471441in}}%
\pgfpathlineto{\pgfqpoint{2.316597in}{1.475020in}}%
\pgfpathlineto{\pgfqpoint{2.331313in}{1.480577in}}%
\pgfpathlineto{\pgfqpoint{2.346028in}{1.491564in}}%
\pgfpathlineto{\pgfqpoint{2.360743in}{1.509390in}}%
\pgfpathlineto{\pgfqpoint{2.375459in}{1.528562in}}%
\pgfpathlineto{\pgfqpoint{2.390174in}{1.546640in}}%
\pgfpathlineto{\pgfqpoint{2.404889in}{1.554886in}}%
\pgfpathlineto{\pgfqpoint{2.434320in}{1.561450in}}%
\pgfpathlineto{\pgfqpoint{2.449035in}{1.565338in}}%
\pgfpathlineto{\pgfqpoint{2.463751in}{1.570401in}}%
\pgfpathlineto{\pgfqpoint{2.478466in}{1.575020in}}%
\pgfpathlineto{\pgfqpoint{2.493181in}{1.577950in}}%
\pgfpathlineto{\pgfqpoint{2.507897in}{1.586356in}}%
\pgfpathlineto{\pgfqpoint{2.522612in}{1.601049in}}%
\pgfpathlineto{\pgfqpoint{2.537327in}{1.620688in}}%
\pgfpathlineto{\pgfqpoint{2.552042in}{1.641588in}}%
\pgfpathlineto{\pgfqpoint{2.566758in}{1.659396in}}%
\pgfpathlineto{\pgfqpoint{2.581473in}{1.670286in}}%
\pgfpathlineto{\pgfqpoint{2.596188in}{1.672681in}}%
\pgfpathlineto{\pgfqpoint{2.610904in}{1.675576in}}%
\pgfpathlineto{\pgfqpoint{2.625619in}{1.679649in}}%
\pgfpathlineto{\pgfqpoint{2.640334in}{1.685784in}}%
\pgfpathlineto{\pgfqpoint{2.655050in}{1.693053in}}%
\pgfpathlineto{\pgfqpoint{2.684480in}{1.701869in}}%
\pgfpathlineto{\pgfqpoint{2.699196in}{1.711694in}}%
\pgfpathlineto{\pgfqpoint{2.713911in}{1.728887in}}%
\pgfpathlineto{\pgfqpoint{2.728626in}{1.751192in}}%
\pgfpathlineto{\pgfqpoint{2.743342in}{1.767996in}}%
\pgfpathlineto{\pgfqpoint{2.758057in}{1.776140in}}%
\pgfpathlineto{\pgfqpoint{2.772772in}{1.776718in}}%
\pgfpathlineto{\pgfqpoint{2.802203in}{1.778975in}}%
\pgfpathlineto{\pgfqpoint{2.816918in}{1.778277in}}%
\pgfpathlineto{\pgfqpoint{2.846349in}{1.777685in}}%
\pgfpathlineto{\pgfqpoint{2.861064in}{1.784161in}}%
\pgfpathlineto{\pgfqpoint{2.875779in}{1.792975in}}%
\pgfpathlineto{\pgfqpoint{2.890495in}{1.803159in}}%
\pgfpathlineto{\pgfqpoint{2.905210in}{1.818237in}}%
\pgfpathlineto{\pgfqpoint{2.919925in}{1.832107in}}%
\pgfpathlineto{\pgfqpoint{2.934641in}{1.835779in}}%
\pgfpathlineto{\pgfqpoint{2.949356in}{1.835461in}}%
\pgfpathlineto{\pgfqpoint{2.964071in}{1.836630in}}%
\pgfpathlineto{\pgfqpoint{2.978787in}{1.843191in}}%
\pgfpathlineto{\pgfqpoint{2.993502in}{1.849236in}}%
\pgfpathlineto{\pgfqpoint{3.008217in}{1.856724in}}%
\pgfpathlineto{\pgfqpoint{3.022933in}{1.861078in}}%
\pgfpathlineto{\pgfqpoint{3.037648in}{1.871415in}}%
\pgfpathlineto{\pgfqpoint{3.052363in}{1.891368in}}%
\pgfpathlineto{\pgfqpoint{3.067079in}{1.915059in}}%
\pgfpathlineto{\pgfqpoint{3.096509in}{1.972751in}}%
\pgfpathlineto{\pgfqpoint{3.111224in}{1.989070in}}%
\pgfpathlineto{\pgfqpoint{3.125940in}{1.998766in}}%
\pgfpathlineto{\pgfqpoint{3.140655in}{2.004539in}}%
\pgfpathlineto{\pgfqpoint{3.155370in}{2.014648in}}%
\pgfpathlineto{\pgfqpoint{3.170086in}{2.026022in}}%
\pgfpathlineto{\pgfqpoint{3.184801in}{2.035509in}}%
\pgfpathlineto{\pgfqpoint{3.199516in}{2.041043in}}%
\pgfpathlineto{\pgfqpoint{3.214232in}{2.049307in}}%
\pgfpathlineto{\pgfqpoint{3.228947in}{2.059615in}}%
\pgfpathlineto{\pgfqpoint{3.243662in}{2.072460in}}%
\pgfpathlineto{\pgfqpoint{3.258378in}{2.090022in}}%
\pgfpathlineto{\pgfqpoint{3.273093in}{2.104824in}}%
\pgfpathlineto{\pgfqpoint{3.287808in}{2.106900in}}%
\pgfpathlineto{\pgfqpoint{3.302524in}{2.106269in}}%
\pgfpathlineto{\pgfqpoint{3.317239in}{2.103747in}}%
\pgfpathlineto{\pgfqpoint{3.331954in}{2.103482in}}%
\pgfpathlineto{\pgfqpoint{3.346670in}{2.104585in}}%
\pgfpathlineto{\pgfqpoint{3.361385in}{2.104054in}}%
\pgfpathlineto{\pgfqpoint{3.376100in}{2.101887in}}%
\pgfpathlineto{\pgfqpoint{3.390815in}{2.106217in}}%
\pgfpathlineto{\pgfqpoint{3.405531in}{2.112648in}}%
\pgfpathlineto{\pgfqpoint{3.420246in}{2.124494in}}%
\pgfpathlineto{\pgfqpoint{3.434961in}{2.139990in}}%
\pgfpathlineto{\pgfqpoint{3.449677in}{2.157561in}}%
\pgfpathlineto{\pgfqpoint{3.464392in}{2.165910in}}%
\pgfpathlineto{\pgfqpoint{3.479107in}{2.167528in}}%
\pgfpathlineto{\pgfqpoint{3.493823in}{2.168861in}}%
\pgfpathlineto{\pgfqpoint{3.508538in}{2.171043in}}%
\pgfpathlineto{\pgfqpoint{3.537969in}{2.181042in}}%
\pgfpathlineto{\pgfqpoint{3.552684in}{2.184311in}}%
\pgfpathlineto{\pgfqpoint{3.567399in}{2.193499in}}%
\pgfpathlineto{\pgfqpoint{3.582115in}{2.204617in}}%
\pgfpathlineto{\pgfqpoint{3.611545in}{2.238656in}}%
\pgfpathlineto{\pgfqpoint{3.626261in}{2.251570in}}%
\pgfpathlineto{\pgfqpoint{3.640976in}{2.257591in}}%
\pgfpathlineto{\pgfqpoint{3.670406in}{2.258597in}}%
\pgfpathlineto{\pgfqpoint{3.685122in}{2.261950in}}%
\pgfpathlineto{\pgfqpoint{3.699837in}{2.266938in}}%
\pgfpathlineto{\pgfqpoint{3.714552in}{2.270889in}}%
\pgfpathlineto{\pgfqpoint{3.729268in}{2.274516in}}%
\pgfpathlineto{\pgfqpoint{3.743983in}{2.280472in}}%
\pgfpathlineto{\pgfqpoint{3.758698in}{2.293710in}}%
\pgfpathlineto{\pgfqpoint{3.773414in}{2.314198in}}%
\pgfpathlineto{\pgfqpoint{3.788129in}{2.337720in}}%
\pgfpathlineto{\pgfqpoint{3.802844in}{2.356577in}}%
\pgfpathlineto{\pgfqpoint{3.817560in}{2.368328in}}%
\pgfpathlineto{\pgfqpoint{3.832275in}{2.372259in}}%
\pgfpathlineto{\pgfqpoint{3.846990in}{2.377438in}}%
\pgfpathlineto{\pgfqpoint{3.861706in}{2.384139in}}%
\pgfpathlineto{\pgfqpoint{3.876421in}{2.392817in}}%
\pgfpathlineto{\pgfqpoint{3.891136in}{2.400869in}}%
\pgfpathlineto{\pgfqpoint{3.905852in}{2.408271in}}%
\pgfpathlineto{\pgfqpoint{3.920567in}{2.418455in}}%
\pgfpathlineto{\pgfqpoint{3.935282in}{2.434228in}}%
\pgfpathlineto{\pgfqpoint{3.949997in}{2.455688in}}%
\pgfpathlineto{\pgfqpoint{3.964713in}{2.480505in}}%
\pgfpathlineto{\pgfqpoint{3.979428in}{2.500239in}}%
\pgfpathlineto{\pgfqpoint{3.994143in}{2.513477in}}%
\pgfpathlineto{\pgfqpoint{4.008859in}{2.518048in}}%
\pgfpathlineto{\pgfqpoint{4.023574in}{2.522944in}}%
\pgfpathlineto{\pgfqpoint{4.038289in}{2.528364in}}%
\pgfpathlineto{\pgfqpoint{4.053005in}{2.535294in}}%
\pgfpathlineto{\pgfqpoint{4.067720in}{2.542621in}}%
\pgfpathlineto{\pgfqpoint{4.082435in}{2.548578in}}%
\pgfpathlineto{\pgfqpoint{4.097151in}{2.557649in}}%
\pgfpathlineto{\pgfqpoint{4.111866in}{2.569838in}}%
\pgfpathlineto{\pgfqpoint{4.126581in}{2.585673in}}%
\pgfpathlineto{\pgfqpoint{4.141297in}{2.600336in}}%
\pgfpathlineto{\pgfqpoint{4.156012in}{2.616201in}}%
\pgfpathlineto{\pgfqpoint{4.170727in}{2.622231in}}%
\pgfpathlineto{\pgfqpoint{4.185443in}{2.620784in}}%
\pgfpathlineto{\pgfqpoint{4.200158in}{2.621966in}}%
\pgfpathlineto{\pgfqpoint{4.214873in}{2.624555in}}%
\pgfpathlineto{\pgfqpoint{4.229589in}{2.627482in}}%
\pgfpathlineto{\pgfqpoint{4.244304in}{2.631832in}}%
\pgfpathlineto{\pgfqpoint{4.259019in}{2.634140in}}%
\pgfpathlineto{\pgfqpoint{4.273734in}{2.642769in}}%
\pgfpathlineto{\pgfqpoint{4.288450in}{2.656599in}}%
\pgfpathlineto{\pgfqpoint{4.303165in}{2.681320in}}%
\pgfpathlineto{\pgfqpoint{4.332596in}{2.727478in}}%
\pgfpathlineto{\pgfqpoint{4.347311in}{2.738906in}}%
\pgfpathlineto{\pgfqpoint{4.362026in}{2.742298in}}%
\pgfpathlineto{\pgfqpoint{4.376742in}{2.746957in}}%
\pgfpathlineto{\pgfqpoint{4.406172in}{2.764177in}}%
\pgfpathlineto{\pgfqpoint{4.420888in}{2.771467in}}%
\pgfpathlineto{\pgfqpoint{4.435603in}{2.773432in}}%
\pgfpathlineto{\pgfqpoint{4.450318in}{2.784919in}}%
\pgfpathlineto{\pgfqpoint{4.465034in}{2.799302in}}%
\pgfpathlineto{\pgfqpoint{4.479749in}{2.816452in}}%
\pgfpathlineto{\pgfqpoint{4.509180in}{2.854208in}}%
\pgfpathlineto{\pgfqpoint{4.523895in}{2.864424in}}%
\pgfpathlineto{\pgfqpoint{4.538610in}{2.868236in}}%
\pgfpathlineto{\pgfqpoint{4.553325in}{2.868431in}}%
\pgfpathlineto{\pgfqpoint{4.568041in}{2.870253in}}%
\pgfpathlineto{\pgfqpoint{4.582756in}{2.873841in}}%
\pgfpathlineto{\pgfqpoint{4.597471in}{2.880155in}}%
\pgfpathlineto{\pgfqpoint{4.612187in}{2.879668in}}%
\pgfpathlineto{\pgfqpoint{4.626902in}{2.887436in}}%
\pgfpathlineto{\pgfqpoint{4.641617in}{2.900474in}}%
\pgfpathlineto{\pgfqpoint{4.656333in}{2.919607in}}%
\pgfpathlineto{\pgfqpoint{4.671048in}{2.941342in}}%
\pgfpathlineto{\pgfqpoint{4.685763in}{2.958264in}}%
\pgfpathlineto{\pgfqpoint{4.700479in}{2.967696in}}%
\pgfpathlineto{\pgfqpoint{4.715194in}{2.973105in}}%
\pgfpathlineto{\pgfqpoint{4.729909in}{2.976834in}}%
\pgfpathlineto{\pgfqpoint{4.744625in}{2.982621in}}%
\pgfpathlineto{\pgfqpoint{4.759340in}{2.991287in}}%
\pgfpathlineto{\pgfqpoint{4.774055in}{2.997816in}}%
\pgfpathlineto{\pgfqpoint{4.788771in}{2.995560in}}%
\pgfpathlineto{\pgfqpoint{4.803486in}{2.998789in}}%
\pgfpathlineto{\pgfqpoint{4.818201in}{3.011191in}}%
\pgfpathlineto{\pgfqpoint{4.832916in}{3.028163in}}%
\pgfpathlineto{\pgfqpoint{4.847632in}{3.046547in}}%
\pgfpathlineto{\pgfqpoint{4.862347in}{3.062882in}}%
\pgfpathlineto{\pgfqpoint{4.877062in}{3.073341in}}%
\pgfpathlineto{\pgfqpoint{4.891778in}{3.075652in}}%
\pgfpathlineto{\pgfqpoint{4.906493in}{3.076557in}}%
\pgfpathlineto{\pgfqpoint{4.921208in}{3.076403in}}%
\pgfpathlineto{\pgfqpoint{4.935924in}{3.082094in}}%
\pgfpathlineto{\pgfqpoint{4.950639in}{3.089369in}}%
\pgfpathlineto{\pgfqpoint{4.965354in}{3.097760in}}%
\pgfpathlineto{\pgfqpoint{4.980070in}{3.102791in}}%
\pgfpathlineto{\pgfqpoint{4.994785in}{3.114349in}}%
\pgfpathlineto{\pgfqpoint{5.024216in}{3.154122in}}%
\pgfpathlineto{\pgfqpoint{5.038931in}{3.168846in}}%
\pgfpathlineto{\pgfqpoint{5.053646in}{3.178015in}}%
\pgfpathlineto{\pgfqpoint{5.068362in}{3.179180in}}%
\pgfpathlineto{\pgfqpoint{5.097792in}{3.183830in}}%
\pgfpathlineto{\pgfqpoint{5.112507in}{3.189995in}}%
\pgfpathlineto{\pgfqpoint{5.127223in}{3.196946in}}%
\pgfpathlineto{\pgfqpoint{5.141938in}{3.205241in}}%
\pgfpathlineto{\pgfqpoint{5.156653in}{3.217121in}}%
\pgfpathlineto{\pgfqpoint{5.171369in}{3.235036in}}%
\pgfpathlineto{\pgfqpoint{5.200799in}{3.277636in}}%
\pgfpathlineto{\pgfqpoint{5.215515in}{3.296227in}}%
\pgfpathlineto{\pgfqpoint{5.230230in}{3.308430in}}%
\pgfpathlineto{\pgfqpoint{5.259661in}{3.321047in}}%
\pgfpathlineto{\pgfqpoint{5.289091in}{3.332215in}}%
\pgfpathlineto{\pgfqpoint{5.303807in}{3.335684in}}%
\pgfpathlineto{\pgfqpoint{5.318522in}{3.336187in}}%
\pgfpathlineto{\pgfqpoint{5.333237in}{3.337267in}}%
\pgfpathlineto{\pgfqpoint{5.347953in}{3.347579in}}%
\pgfpathlineto{\pgfqpoint{5.362668in}{3.365491in}}%
\pgfpathlineto{\pgfqpoint{5.377383in}{3.386856in}}%
\pgfpathlineto{\pgfqpoint{5.392098in}{3.404116in}}%
\pgfpathlineto{\pgfqpoint{5.406814in}{3.413431in}}%
\pgfpathlineto{\pgfqpoint{5.421529in}{3.415260in}}%
\pgfpathlineto{\pgfqpoint{5.450960in}{3.420638in}}%
\pgfpathlineto{\pgfqpoint{5.480390in}{3.433674in}}%
\pgfpathlineto{\pgfqpoint{5.495106in}{3.439641in}}%
\pgfpathlineto{\pgfqpoint{5.509821in}{3.449791in}}%
\pgfpathlineto{\pgfqpoint{5.524536in}{3.465265in}}%
\pgfpathlineto{\pgfqpoint{5.539252in}{3.481686in}}%
\pgfpathlineto{\pgfqpoint{5.553967in}{3.503171in}}%
\pgfpathlineto{\pgfqpoint{5.568682in}{3.520206in}}%
\pgfpathlineto{\pgfqpoint{5.583398in}{3.531028in}}%
\pgfpathlineto{\pgfqpoint{5.598113in}{3.535026in}}%
\pgfpathlineto{\pgfqpoint{5.612828in}{3.541088in}}%
\pgfpathlineto{\pgfqpoint{5.627544in}{3.547493in}}%
\pgfpathlineto{\pgfqpoint{5.642259in}{3.556461in}}%
\pgfpathlineto{\pgfqpoint{5.656974in}{3.569029in}}%
\pgfpathlineto{\pgfqpoint{5.671689in}{3.575642in}}%
\pgfpathlineto{\pgfqpoint{5.686405in}{3.584304in}}%
\pgfpathlineto{\pgfqpoint{5.701120in}{3.605340in}}%
\pgfpathlineto{\pgfqpoint{5.715835in}{3.627308in}}%
\pgfpathlineto{\pgfqpoint{5.730551in}{3.652389in}}%
\pgfpathlineto{\pgfqpoint{5.745266in}{3.673363in}}%
\pgfpathlineto{\pgfqpoint{5.759981in}{3.685967in}}%
\pgfpathlineto{\pgfqpoint{5.774697in}{3.689470in}}%
\pgfpathlineto{\pgfqpoint{5.789412in}{3.693557in}}%
\pgfpathlineto{\pgfqpoint{5.804127in}{3.699849in}}%
\pgfpathlineto{\pgfqpoint{5.818843in}{3.705054in}}%
\pgfpathlineto{\pgfqpoint{5.833558in}{3.708111in}}%
\pgfpathlineto{\pgfqpoint{5.848273in}{3.714668in}}%
\pgfpathlineto{\pgfqpoint{5.862989in}{3.722206in}}%
\pgfpathlineto{\pgfqpoint{5.877704in}{3.738103in}}%
\pgfpathlineto{\pgfqpoint{5.907135in}{3.778983in}}%
\pgfpathlineto{\pgfqpoint{5.921850in}{3.796612in}}%
\pgfpathlineto{\pgfqpoint{5.936565in}{3.804793in}}%
\pgfpathlineto{\pgfqpoint{5.951280in}{3.808752in}}%
\pgfpathlineto{\pgfqpoint{5.965996in}{3.810561in}}%
\pgfpathlineto{\pgfqpoint{5.980711in}{3.815976in}}%
\pgfpathlineto{\pgfqpoint{5.995426in}{3.825769in}}%
\pgfpathlineto{\pgfqpoint{6.010142in}{3.829149in}}%
\pgfpathlineto{\pgfqpoint{6.024857in}{3.830599in}}%
\pgfpathlineto{\pgfqpoint{6.039572in}{3.837938in}}%
\pgfpathlineto{\pgfqpoint{6.054288in}{3.851798in}}%
\pgfpathlineto{\pgfqpoint{6.069003in}{3.870187in}}%
\pgfpathlineto{\pgfqpoint{6.083718in}{3.892301in}}%
\pgfpathlineto{\pgfqpoint{6.098434in}{3.910675in}}%
\pgfpathlineto{\pgfqpoint{6.113149in}{3.919343in}}%
\pgfpathlineto{\pgfqpoint{6.127864in}{3.925113in}}%
\pgfpathlineto{\pgfqpoint{6.142580in}{3.931236in}}%
\pgfpathlineto{\pgfqpoint{6.157295in}{3.940934in}}%
\pgfpathlineto{\pgfqpoint{6.172010in}{3.952216in}}%
\pgfpathlineto{\pgfqpoint{6.186726in}{3.965153in}}%
\pgfpathlineto{\pgfqpoint{6.201441in}{3.973759in}}%
\pgfpathlineto{\pgfqpoint{6.216156in}{3.991632in}}%
\pgfpathlineto{\pgfqpoint{6.230871in}{4.017081in}}%
\pgfpathlineto{\pgfqpoint{6.260302in}{4.073440in}}%
\pgfpathlineto{\pgfqpoint{6.275017in}{4.097138in}}%
\pgfpathlineto{\pgfqpoint{6.289733in}{4.111170in}}%
\pgfpathlineto{\pgfqpoint{6.304448in}{4.118858in}}%
\pgfpathlineto{\pgfqpoint{6.319163in}{4.127327in}}%
\pgfpathlineto{\pgfqpoint{6.333879in}{4.137115in}}%
\pgfpathlineto{\pgfqpoint{6.348594in}{4.147584in}}%
\pgfpathlineto{\pgfqpoint{6.363309in}{4.155727in}}%
\pgfpathlineto{\pgfqpoint{6.378025in}{4.155486in}}%
\pgfpathlineto{\pgfqpoint{6.392740in}{4.162292in}}%
\pgfpathlineto{\pgfqpoint{6.407455in}{4.177077in}}%
\pgfpathlineto{\pgfqpoint{6.422171in}{4.200509in}}%
\pgfpathlineto{\pgfqpoint{6.436886in}{4.225524in}}%
\pgfpathlineto{\pgfqpoint{6.451601in}{4.246301in}}%
\pgfpathlineto{\pgfqpoint{6.466317in}{4.255530in}}%
\pgfpathlineto{\pgfqpoint{6.481032in}{4.256028in}}%
\pgfpathlineto{\pgfqpoint{6.495747in}{4.259007in}}%
\pgfpathlineto{\pgfqpoint{6.510463in}{4.262429in}}%
\pgfpathlineto{\pgfqpoint{6.525178in}{4.270197in}}%
\pgfpathlineto{\pgfqpoint{6.539893in}{4.275032in}}%
\pgfpathlineto{\pgfqpoint{6.554608in}{4.276797in}}%
\pgfpathlineto{\pgfqpoint{6.569324in}{4.279614in}}%
\pgfpathlineto{\pgfqpoint{6.584039in}{4.291671in}}%
\pgfpathlineto{\pgfqpoint{6.598754in}{4.310608in}}%
\pgfpathlineto{\pgfqpoint{6.628185in}{4.347268in}}%
\pgfpathlineto{\pgfqpoint{6.642900in}{4.357051in}}%
\pgfpathlineto{\pgfqpoint{6.657616in}{4.361934in}}%
\pgfpathlineto{\pgfqpoint{6.672331in}{4.368484in}}%
\pgfpathlineto{\pgfqpoint{6.687046in}{4.374283in}}%
\pgfpathlineto{\pgfqpoint{6.701762in}{4.387235in}}%
\pgfpathlineto{\pgfqpoint{6.716477in}{4.399474in}}%
\pgfpathlineto{\pgfqpoint{6.731192in}{4.408226in}}%
\pgfpathlineto{\pgfqpoint{6.745908in}{4.419128in}}%
\pgfpathlineto{\pgfqpoint{6.760623in}{4.438960in}}%
\pgfpathlineto{\pgfqpoint{6.775338in}{4.466111in}}%
\pgfpathlineto{\pgfqpoint{6.790054in}{4.491011in}}%
\pgfpathlineto{\pgfqpoint{6.804769in}{4.514060in}}%
\pgfpathlineto{\pgfqpoint{6.819484in}{4.527259in}}%
\pgfpathlineto{\pgfqpoint{6.834199in}{4.529564in}}%
\pgfpathlineto{\pgfqpoint{6.848915in}{4.535839in}}%
\pgfpathlineto{\pgfqpoint{6.863630in}{4.540569in}}%
\pgfpathlineto{\pgfqpoint{6.878345in}{4.548996in}}%
\pgfpathlineto{\pgfqpoint{6.893061in}{4.560244in}}%
\pgfpathlineto{\pgfqpoint{6.907776in}{4.566068in}}%
\pgfpathlineto{\pgfqpoint{6.922491in}{4.574068in}}%
\pgfpathlineto{\pgfqpoint{6.937207in}{4.590186in}}%
\pgfpathlineto{\pgfqpoint{6.951922in}{4.613428in}}%
\pgfpathlineto{\pgfqpoint{6.966637in}{4.635982in}}%
\pgfpathlineto{\pgfqpoint{6.981353in}{4.656016in}}%
\pgfpathlineto{\pgfqpoint{6.996068in}{4.670281in}}%
\pgfpathlineto{\pgfqpoint{7.010783in}{4.675262in}}%
\pgfpathlineto{\pgfqpoint{7.025499in}{4.681072in}}%
\pgfpathlineto{\pgfqpoint{7.040214in}{4.684230in}}%
\pgfpathlineto{\pgfqpoint{7.054929in}{4.690170in}}%
\pgfpathlineto{\pgfqpoint{7.069645in}{4.700575in}}%
\pgfpathlineto{\pgfqpoint{7.084360in}{4.706463in}}%
\pgfpathlineto{\pgfqpoint{7.099075in}{4.715296in}}%
\pgfpathlineto{\pgfqpoint{7.113790in}{4.728393in}}%
\pgfpathlineto{\pgfqpoint{7.128506in}{4.750686in}}%
\pgfpathlineto{\pgfqpoint{7.143221in}{4.772414in}}%
\pgfpathlineto{\pgfqpoint{7.157936in}{4.787913in}}%
\pgfpathlineto{\pgfqpoint{7.172652in}{4.797071in}}%
\pgfpathlineto{\pgfqpoint{7.202082in}{4.804962in}}%
\pgfpathlineto{\pgfqpoint{7.216798in}{4.811099in}}%
\pgfpathlineto{\pgfqpoint{7.231513in}{4.818262in}}%
\pgfpathlineto{\pgfqpoint{7.246228in}{4.826838in}}%
\pgfpathlineto{\pgfqpoint{7.260944in}{4.825589in}}%
\pgfpathlineto{\pgfqpoint{7.275659in}{4.831321in}}%
\pgfpathlineto{\pgfqpoint{7.290374in}{4.841690in}}%
\pgfpathlineto{\pgfqpoint{7.305090in}{4.862388in}}%
\pgfpathlineto{\pgfqpoint{7.319805in}{4.884613in}}%
\pgfpathlineto{\pgfqpoint{7.334520in}{4.904716in}}%
\pgfpathlineto{\pgfqpoint{7.349236in}{4.915035in}}%
\pgfpathlineto{\pgfqpoint{7.378666in}{4.922938in}}%
\pgfpathlineto{\pgfqpoint{7.393381in}{4.930079in}}%
\pgfpathlineto{\pgfqpoint{7.408097in}{4.940218in}}%
\pgfpathlineto{\pgfqpoint{7.422812in}{4.959061in}}%
\pgfpathlineto{\pgfqpoint{7.437527in}{4.975258in}}%
\pgfpathlineto{\pgfqpoint{7.466958in}{5.003848in}}%
\pgfpathlineto{\pgfqpoint{7.481673in}{5.025107in}}%
\pgfpathlineto{\pgfqpoint{7.496389in}{5.047913in}}%
\pgfpathlineto{\pgfqpoint{7.511104in}{5.067588in}}%
\pgfpathlineto{\pgfqpoint{7.525819in}{5.081354in}}%
\pgfpathlineto{\pgfqpoint{7.540535in}{5.088342in}}%
\pgfpathlineto{\pgfqpoint{7.555250in}{5.092673in}}%
\pgfpathlineto{\pgfqpoint{7.569965in}{5.096181in}}%
\pgfpathlineto{\pgfqpoint{7.584681in}{5.098267in}}%
\pgfpathlineto{\pgfqpoint{7.599396in}{5.104478in}}%
\pgfpathlineto{\pgfqpoint{7.614111in}{5.112573in}}%
\pgfpathlineto{\pgfqpoint{7.628827in}{5.127426in}}%
\pgfpathlineto{\pgfqpoint{7.643542in}{5.147461in}}%
\pgfpathlineto{\pgfqpoint{7.658257in}{5.174374in}}%
\pgfpathlineto{\pgfqpoint{7.672972in}{5.204844in}}%
\pgfpathlineto{\pgfqpoint{7.687688in}{5.228337in}}%
\pgfpathlineto{\pgfqpoint{7.702403in}{5.244887in}}%
\pgfpathlineto{\pgfqpoint{7.717118in}{5.255499in}}%
\pgfpathlineto{\pgfqpoint{7.731834in}{5.265623in}}%
\pgfpathlineto{\pgfqpoint{7.746549in}{5.278841in}}%
\pgfpathlineto{\pgfqpoint{7.746549in}{5.278841in}}%
\pgfusepath{stroke}%
\end{pgfscope}%
\begin{pgfscope}%
\pgfpathrectangle{\pgfqpoint{0.697913in}{0.559721in}}{\pgfqpoint{7.048636in}{4.990279in}}%
\pgfusepath{clip}%
\pgfsetbuttcap%
\pgfsetroundjoin%
\definecolor{currentfill}{rgb}{0.000000,0.000000,0.000000}%
\pgfsetfillcolor{currentfill}%
\pgfsetlinewidth{1.003750pt}%
\definecolor{currentstroke}{rgb}{0.000000,0.000000,0.000000}%
\pgfsetstrokecolor{currentstroke}%
\pgfsetdash{}{0pt}%
\pgfsys@defobject{currentmarker}{\pgfqpoint{-0.020833in}{-0.020833in}}{\pgfqpoint{0.020833in}{0.020833in}}{%
\pgfpathmoveto{\pgfqpoint{0.000000in}{-0.020833in}}%
\pgfpathcurveto{\pgfqpoint{0.005525in}{-0.020833in}}{\pgfqpoint{0.010825in}{-0.018638in}}{\pgfqpoint{0.014731in}{-0.014731in}}%
\pgfpathcurveto{\pgfqpoint{0.018638in}{-0.010825in}}{\pgfqpoint{0.020833in}{-0.005525in}}{\pgfqpoint{0.020833in}{0.000000in}}%
\pgfpathcurveto{\pgfqpoint{0.020833in}{0.005525in}}{\pgfqpoint{0.018638in}{0.010825in}}{\pgfqpoint{0.014731in}{0.014731in}}%
\pgfpathcurveto{\pgfqpoint{0.010825in}{0.018638in}}{\pgfqpoint{0.005525in}{0.020833in}}{\pgfqpoint{0.000000in}{0.020833in}}%
\pgfpathcurveto{\pgfqpoint{-0.005525in}{0.020833in}}{\pgfqpoint{-0.010825in}{0.018638in}}{\pgfqpoint{-0.014731in}{0.014731in}}%
\pgfpathcurveto{\pgfqpoint{-0.018638in}{0.010825in}}{\pgfqpoint{-0.020833in}{0.005525in}}{\pgfqpoint{-0.020833in}{0.000000in}}%
\pgfpathcurveto{\pgfqpoint{-0.020833in}{-0.005525in}}{\pgfqpoint{-0.018638in}{-0.010825in}}{\pgfqpoint{-0.014731in}{-0.014731in}}%
\pgfpathcurveto{\pgfqpoint{-0.010825in}{-0.018638in}}{\pgfqpoint{-0.005525in}{-0.020833in}}{\pgfqpoint{0.000000in}{-0.020833in}}%
\pgfpathlineto{\pgfqpoint{0.000000in}{-0.020833in}}%
\pgfpathclose%
\pgfusepath{stroke,fill}%
}%
\begin{pgfscope}%
\pgfsys@transformshift{0.697913in}{0.772482in}%
\pgfsys@useobject{currentmarker}{}%
\end{pgfscope}%
\begin{pgfscope}%
\pgfsys@transformshift{0.712628in}{0.776708in}%
\pgfsys@useobject{currentmarker}{}%
\end{pgfscope}%
\begin{pgfscope}%
\pgfsys@transformshift{0.727343in}{0.778526in}%
\pgfsys@useobject{currentmarker}{}%
\end{pgfscope}%
\begin{pgfscope}%
\pgfsys@transformshift{0.742059in}{0.782814in}%
\pgfsys@useobject{currentmarker}{}%
\end{pgfscope}%
\begin{pgfscope}%
\pgfsys@transformshift{0.756774in}{0.793740in}%
\pgfsys@useobject{currentmarker}{}%
\end{pgfscope}%
\begin{pgfscope}%
\pgfsys@transformshift{0.771489in}{0.807925in}%
\pgfsys@useobject{currentmarker}{}%
\end{pgfscope}%
\begin{pgfscope}%
\pgfsys@transformshift{0.786205in}{0.824367in}%
\pgfsys@useobject{currentmarker}{}%
\end{pgfscope}%
\begin{pgfscope}%
\pgfsys@transformshift{0.800920in}{0.842143in}%
\pgfsys@useobject{currentmarker}{}%
\end{pgfscope}%
\begin{pgfscope}%
\pgfsys@transformshift{0.815635in}{0.850456in}%
\pgfsys@useobject{currentmarker}{}%
\end{pgfscope}%
\begin{pgfscope}%
\pgfsys@transformshift{0.830350in}{0.851458in}%
\pgfsys@useobject{currentmarker}{}%
\end{pgfscope}%
\begin{pgfscope}%
\pgfsys@transformshift{0.845066in}{0.850016in}%
\pgfsys@useobject{currentmarker}{}%
\end{pgfscope}%
\begin{pgfscope}%
\pgfsys@transformshift{0.859781in}{0.853475in}%
\pgfsys@useobject{currentmarker}{}%
\end{pgfscope}%
\begin{pgfscope}%
\pgfsys@transformshift{0.874496in}{0.855324in}%
\pgfsys@useobject{currentmarker}{}%
\end{pgfscope}%
\begin{pgfscope}%
\pgfsys@transformshift{0.889212in}{0.853496in}%
\pgfsys@useobject{currentmarker}{}%
\end{pgfscope}%
\begin{pgfscope}%
\pgfsys@transformshift{0.903927in}{0.851477in}%
\pgfsys@useobject{currentmarker}{}%
\end{pgfscope}%
\begin{pgfscope}%
\pgfsys@transformshift{0.918642in}{0.854480in}%
\pgfsys@useobject{currentmarker}{}%
\end{pgfscope}%
\begin{pgfscope}%
\pgfsys@transformshift{0.933358in}{0.864389in}%
\pgfsys@useobject{currentmarker}{}%
\end{pgfscope}%
\begin{pgfscope}%
\pgfsys@transformshift{0.948073in}{0.879740in}%
\pgfsys@useobject{currentmarker}{}%
\end{pgfscope}%
\begin{pgfscope}%
\pgfsys@transformshift{0.962788in}{0.896539in}%
\pgfsys@useobject{currentmarker}{}%
\end{pgfscope}%
\begin{pgfscope}%
\pgfsys@transformshift{0.977504in}{0.910503in}%
\pgfsys@useobject{currentmarker}{}%
\end{pgfscope}%
\begin{pgfscope}%
\pgfsys@transformshift{0.992219in}{0.920809in}%
\pgfsys@useobject{currentmarker}{}%
\end{pgfscope}%
\begin{pgfscope}%
\pgfsys@transformshift{1.006934in}{0.920843in}%
\pgfsys@useobject{currentmarker}{}%
\end{pgfscope}%
\begin{pgfscope}%
\pgfsys@transformshift{1.021650in}{0.921381in}%
\pgfsys@useobject{currentmarker}{}%
\end{pgfscope}%
\begin{pgfscope}%
\pgfsys@transformshift{1.036365in}{0.924504in}%
\pgfsys@useobject{currentmarker}{}%
\end{pgfscope}%
\begin{pgfscope}%
\pgfsys@transformshift{1.051080in}{0.930748in}%
\pgfsys@useobject{currentmarker}{}%
\end{pgfscope}%
\begin{pgfscope}%
\pgfsys@transformshift{1.065796in}{0.934302in}%
\pgfsys@useobject{currentmarker}{}%
\end{pgfscope}%
\begin{pgfscope}%
\pgfsys@transformshift{1.080511in}{0.933994in}%
\pgfsys@useobject{currentmarker}{}%
\end{pgfscope}%
\begin{pgfscope}%
\pgfsys@transformshift{1.095226in}{0.938674in}%
\pgfsys@useobject{currentmarker}{}%
\end{pgfscope}%
\begin{pgfscope}%
\pgfsys@transformshift{1.109941in}{0.949747in}%
\pgfsys@useobject{currentmarker}{}%
\end{pgfscope}%
\begin{pgfscope}%
\pgfsys@transformshift{1.124657in}{0.967119in}%
\pgfsys@useobject{currentmarker}{}%
\end{pgfscope}%
\begin{pgfscope}%
\pgfsys@transformshift{1.139372in}{0.988431in}%
\pgfsys@useobject{currentmarker}{}%
\end{pgfscope}%
\begin{pgfscope}%
\pgfsys@transformshift{1.154087in}{1.002866in}%
\pgfsys@useobject{currentmarker}{}%
\end{pgfscope}%
\begin{pgfscope}%
\pgfsys@transformshift{1.168803in}{1.012367in}%
\pgfsys@useobject{currentmarker}{}%
\end{pgfscope}%
\begin{pgfscope}%
\pgfsys@transformshift{1.183518in}{1.015020in}%
\pgfsys@useobject{currentmarker}{}%
\end{pgfscope}%
\begin{pgfscope}%
\pgfsys@transformshift{1.198233in}{1.018469in}%
\pgfsys@useobject{currentmarker}{}%
\end{pgfscope}%
\begin{pgfscope}%
\pgfsys@transformshift{1.212949in}{1.021541in}%
\pgfsys@useobject{currentmarker}{}%
\end{pgfscope}%
\begin{pgfscope}%
\pgfsys@transformshift{1.227664in}{1.029429in}%
\pgfsys@useobject{currentmarker}{}%
\end{pgfscope}%
\begin{pgfscope}%
\pgfsys@transformshift{1.242379in}{1.039357in}%
\pgfsys@useobject{currentmarker}{}%
\end{pgfscope}%
\begin{pgfscope}%
\pgfsys@transformshift{1.257095in}{1.044620in}%
\pgfsys@useobject{currentmarker}{}%
\end{pgfscope}%
\begin{pgfscope}%
\pgfsys@transformshift{1.271810in}{1.052380in}%
\pgfsys@useobject{currentmarker}{}%
\end{pgfscope}%
\begin{pgfscope}%
\pgfsys@transformshift{1.286525in}{1.066091in}%
\pgfsys@useobject{currentmarker}{}%
\end{pgfscope}%
\begin{pgfscope}%
\pgfsys@transformshift{1.301241in}{1.083891in}%
\pgfsys@useobject{currentmarker}{}%
\end{pgfscope}%
\begin{pgfscope}%
\pgfsys@transformshift{1.315956in}{1.107352in}%
\pgfsys@useobject{currentmarker}{}%
\end{pgfscope}%
\begin{pgfscope}%
\pgfsys@transformshift{1.330671in}{1.126971in}%
\pgfsys@useobject{currentmarker}{}%
\end{pgfscope}%
\begin{pgfscope}%
\pgfsys@transformshift{1.345387in}{1.138285in}%
\pgfsys@useobject{currentmarker}{}%
\end{pgfscope}%
\begin{pgfscope}%
\pgfsys@transformshift{1.360102in}{1.145010in}%
\pgfsys@useobject{currentmarker}{}%
\end{pgfscope}%
\begin{pgfscope}%
\pgfsys@transformshift{1.374817in}{1.151060in}%
\pgfsys@useobject{currentmarker}{}%
\end{pgfscope}%
\begin{pgfscope}%
\pgfsys@transformshift{1.389532in}{1.157016in}%
\pgfsys@useobject{currentmarker}{}%
\end{pgfscope}%
\begin{pgfscope}%
\pgfsys@transformshift{1.404248in}{1.163814in}%
\pgfsys@useobject{currentmarker}{}%
\end{pgfscope}%
\begin{pgfscope}%
\pgfsys@transformshift{1.418963in}{1.168754in}%
\pgfsys@useobject{currentmarker}{}%
\end{pgfscope}%
\begin{pgfscope}%
\pgfsys@transformshift{1.433678in}{1.170860in}%
\pgfsys@useobject{currentmarker}{}%
\end{pgfscope}%
\begin{pgfscope}%
\pgfsys@transformshift{1.448394in}{1.177881in}%
\pgfsys@useobject{currentmarker}{}%
\end{pgfscope}%
\begin{pgfscope}%
\pgfsys@transformshift{1.463109in}{1.188860in}%
\pgfsys@useobject{currentmarker}{}%
\end{pgfscope}%
\begin{pgfscope}%
\pgfsys@transformshift{1.477824in}{1.202468in}%
\pgfsys@useobject{currentmarker}{}%
\end{pgfscope}%
\begin{pgfscope}%
\pgfsys@transformshift{1.492540in}{1.221036in}%
\pgfsys@useobject{currentmarker}{}%
\end{pgfscope}%
\begin{pgfscope}%
\pgfsys@transformshift{1.507255in}{1.235942in}%
\pgfsys@useobject{currentmarker}{}%
\end{pgfscope}%
\begin{pgfscope}%
\pgfsys@transformshift{1.521970in}{1.239684in}%
\pgfsys@useobject{currentmarker}{}%
\end{pgfscope}%
\begin{pgfscope}%
\pgfsys@transformshift{1.536686in}{1.239129in}%
\pgfsys@useobject{currentmarker}{}%
\end{pgfscope}%
\begin{pgfscope}%
\pgfsys@transformshift{1.551401in}{1.238990in}%
\pgfsys@useobject{currentmarker}{}%
\end{pgfscope}%
\begin{pgfscope}%
\pgfsys@transformshift{1.566116in}{1.237994in}%
\pgfsys@useobject{currentmarker}{}%
\end{pgfscope}%
\begin{pgfscope}%
\pgfsys@transformshift{1.580832in}{1.240306in}%
\pgfsys@useobject{currentmarker}{}%
\end{pgfscope}%
\begin{pgfscope}%
\pgfsys@transformshift{1.595547in}{1.245559in}%
\pgfsys@useobject{currentmarker}{}%
\end{pgfscope}%
\begin{pgfscope}%
\pgfsys@transformshift{1.610262in}{1.245804in}%
\pgfsys@useobject{currentmarker}{}%
\end{pgfscope}%
\begin{pgfscope}%
\pgfsys@transformshift{1.624978in}{1.247963in}%
\pgfsys@useobject{currentmarker}{}%
\end{pgfscope}%
\begin{pgfscope}%
\pgfsys@transformshift{1.639693in}{1.258008in}%
\pgfsys@useobject{currentmarker}{}%
\end{pgfscope}%
\begin{pgfscope}%
\pgfsys@transformshift{1.654408in}{1.270090in}%
\pgfsys@useobject{currentmarker}{}%
\end{pgfscope}%
\begin{pgfscope}%
\pgfsys@transformshift{1.669124in}{1.285587in}%
\pgfsys@useobject{currentmarker}{}%
\end{pgfscope}%
\begin{pgfscope}%
\pgfsys@transformshift{1.683839in}{1.300576in}%
\pgfsys@useobject{currentmarker}{}%
\end{pgfscope}%
\begin{pgfscope}%
\pgfsys@transformshift{1.698554in}{1.305706in}%
\pgfsys@useobject{currentmarker}{}%
\end{pgfscope}%
\begin{pgfscope}%
\pgfsys@transformshift{1.713269in}{1.306321in}%
\pgfsys@useobject{currentmarker}{}%
\end{pgfscope}%
\begin{pgfscope}%
\pgfsys@transformshift{1.727985in}{1.307462in}%
\pgfsys@useobject{currentmarker}{}%
\end{pgfscope}%
\begin{pgfscope}%
\pgfsys@transformshift{1.742700in}{1.310348in}%
\pgfsys@useobject{currentmarker}{}%
\end{pgfscope}%
\begin{pgfscope}%
\pgfsys@transformshift{1.757415in}{1.309496in}%
\pgfsys@useobject{currentmarker}{}%
\end{pgfscope}%
\begin{pgfscope}%
\pgfsys@transformshift{1.772131in}{1.309777in}%
\pgfsys@useobject{currentmarker}{}%
\end{pgfscope}%
\begin{pgfscope}%
\pgfsys@transformshift{1.786846in}{1.314763in}%
\pgfsys@useobject{currentmarker}{}%
\end{pgfscope}%
\begin{pgfscope}%
\pgfsys@transformshift{1.801561in}{1.321906in}%
\pgfsys@useobject{currentmarker}{}%
\end{pgfscope}%
\begin{pgfscope}%
\pgfsys@transformshift{1.816277in}{1.336418in}%
\pgfsys@useobject{currentmarker}{}%
\end{pgfscope}%
\begin{pgfscope}%
\pgfsys@transformshift{1.830992in}{1.353519in}%
\pgfsys@useobject{currentmarker}{}%
\end{pgfscope}%
\begin{pgfscope}%
\pgfsys@transformshift{1.845707in}{1.370317in}%
\pgfsys@useobject{currentmarker}{}%
\end{pgfscope}%
\begin{pgfscope}%
\pgfsys@transformshift{1.860423in}{1.382970in}%
\pgfsys@useobject{currentmarker}{}%
\end{pgfscope}%
\begin{pgfscope}%
\pgfsys@transformshift{1.875138in}{1.386201in}%
\pgfsys@useobject{currentmarker}{}%
\end{pgfscope}%
\begin{pgfscope}%
\pgfsys@transformshift{1.889853in}{1.382723in}%
\pgfsys@useobject{currentmarker}{}%
\end{pgfscope}%
\begin{pgfscope}%
\pgfsys@transformshift{1.904569in}{1.379957in}%
\pgfsys@useobject{currentmarker}{}%
\end{pgfscope}%
\begin{pgfscope}%
\pgfsys@transformshift{1.919284in}{1.381489in}%
\pgfsys@useobject{currentmarker}{}%
\end{pgfscope}%
\begin{pgfscope}%
\pgfsys@transformshift{1.933999in}{1.384160in}%
\pgfsys@useobject{currentmarker}{}%
\end{pgfscope}%
\begin{pgfscope}%
\pgfsys@transformshift{1.948715in}{1.383501in}%
\pgfsys@useobject{currentmarker}{}%
\end{pgfscope}%
\begin{pgfscope}%
\pgfsys@transformshift{1.963430in}{1.379303in}%
\pgfsys@useobject{currentmarker}{}%
\end{pgfscope}%
\begin{pgfscope}%
\pgfsys@transformshift{1.978145in}{1.379256in}%
\pgfsys@useobject{currentmarker}{}%
\end{pgfscope}%
\begin{pgfscope}%
\pgfsys@transformshift{1.992860in}{1.386849in}%
\pgfsys@useobject{currentmarker}{}%
\end{pgfscope}%
\begin{pgfscope}%
\pgfsys@transformshift{2.007576in}{1.402902in}%
\pgfsys@useobject{currentmarker}{}%
\end{pgfscope}%
\begin{pgfscope}%
\pgfsys@transformshift{2.022291in}{1.419951in}%
\pgfsys@useobject{currentmarker}{}%
\end{pgfscope}%
\begin{pgfscope}%
\pgfsys@transformshift{2.037006in}{1.434046in}%
\pgfsys@useobject{currentmarker}{}%
\end{pgfscope}%
\begin{pgfscope}%
\pgfsys@transformshift{2.051722in}{1.437016in}%
\pgfsys@useobject{currentmarker}{}%
\end{pgfscope}%
\begin{pgfscope}%
\pgfsys@transformshift{2.066437in}{1.434647in}%
\pgfsys@useobject{currentmarker}{}%
\end{pgfscope}%
\begin{pgfscope}%
\pgfsys@transformshift{2.081152in}{1.430399in}%
\pgfsys@useobject{currentmarker}{}%
\end{pgfscope}%
\begin{pgfscope}%
\pgfsys@transformshift{2.095868in}{1.426973in}%
\pgfsys@useobject{currentmarker}{}%
\end{pgfscope}%
\begin{pgfscope}%
\pgfsys@transformshift{2.110583in}{1.426453in}%
\pgfsys@useobject{currentmarker}{}%
\end{pgfscope}%
\begin{pgfscope}%
\pgfsys@transformshift{2.125298in}{1.423711in}%
\pgfsys@useobject{currentmarker}{}%
\end{pgfscope}%
\begin{pgfscope}%
\pgfsys@transformshift{2.140014in}{1.420511in}%
\pgfsys@useobject{currentmarker}{}%
\end{pgfscope}%
\begin{pgfscope}%
\pgfsys@transformshift{2.154729in}{1.418902in}%
\pgfsys@useobject{currentmarker}{}%
\end{pgfscope}%
\begin{pgfscope}%
\pgfsys@transformshift{2.169444in}{1.423368in}%
\pgfsys@useobject{currentmarker}{}%
\end{pgfscope}%
\begin{pgfscope}%
\pgfsys@transformshift{2.184160in}{1.436166in}%
\pgfsys@useobject{currentmarker}{}%
\end{pgfscope}%
\begin{pgfscope}%
\pgfsys@transformshift{2.198875in}{1.448575in}%
\pgfsys@useobject{currentmarker}{}%
\end{pgfscope}%
\begin{pgfscope}%
\pgfsys@transformshift{2.213590in}{1.462882in}%
\pgfsys@useobject{currentmarker}{}%
\end{pgfscope}%
\begin{pgfscope}%
\pgfsys@transformshift{2.228306in}{1.466387in}%
\pgfsys@useobject{currentmarker}{}%
\end{pgfscope}%
\begin{pgfscope}%
\pgfsys@transformshift{2.243021in}{1.465698in}%
\pgfsys@useobject{currentmarker}{}%
\end{pgfscope}%
\begin{pgfscope}%
\pgfsys@transformshift{2.257736in}{1.464916in}%
\pgfsys@useobject{currentmarker}{}%
\end{pgfscope}%
\begin{pgfscope}%
\pgfsys@transformshift{2.272451in}{1.464491in}%
\pgfsys@useobject{currentmarker}{}%
\end{pgfscope}%
\begin{pgfscope}%
\pgfsys@transformshift{2.287167in}{1.469085in}%
\pgfsys@useobject{currentmarker}{}%
\end{pgfscope}%
\begin{pgfscope}%
\pgfsys@transformshift{2.301882in}{1.471441in}%
\pgfsys@useobject{currentmarker}{}%
\end{pgfscope}%
\begin{pgfscope}%
\pgfsys@transformshift{2.316597in}{1.475020in}%
\pgfsys@useobject{currentmarker}{}%
\end{pgfscope}%
\begin{pgfscope}%
\pgfsys@transformshift{2.331313in}{1.480577in}%
\pgfsys@useobject{currentmarker}{}%
\end{pgfscope}%
\begin{pgfscope}%
\pgfsys@transformshift{2.346028in}{1.491564in}%
\pgfsys@useobject{currentmarker}{}%
\end{pgfscope}%
\begin{pgfscope}%
\pgfsys@transformshift{2.360743in}{1.509390in}%
\pgfsys@useobject{currentmarker}{}%
\end{pgfscope}%
\begin{pgfscope}%
\pgfsys@transformshift{2.375459in}{1.528562in}%
\pgfsys@useobject{currentmarker}{}%
\end{pgfscope}%
\begin{pgfscope}%
\pgfsys@transformshift{2.390174in}{1.546640in}%
\pgfsys@useobject{currentmarker}{}%
\end{pgfscope}%
\begin{pgfscope}%
\pgfsys@transformshift{2.404889in}{1.554886in}%
\pgfsys@useobject{currentmarker}{}%
\end{pgfscope}%
\begin{pgfscope}%
\pgfsys@transformshift{2.419605in}{1.558173in}%
\pgfsys@useobject{currentmarker}{}%
\end{pgfscope}%
\begin{pgfscope}%
\pgfsys@transformshift{2.434320in}{1.561450in}%
\pgfsys@useobject{currentmarker}{}%
\end{pgfscope}%
\begin{pgfscope}%
\pgfsys@transformshift{2.449035in}{1.565338in}%
\pgfsys@useobject{currentmarker}{}%
\end{pgfscope}%
\begin{pgfscope}%
\pgfsys@transformshift{2.463751in}{1.570401in}%
\pgfsys@useobject{currentmarker}{}%
\end{pgfscope}%
\begin{pgfscope}%
\pgfsys@transformshift{2.478466in}{1.575020in}%
\pgfsys@useobject{currentmarker}{}%
\end{pgfscope}%
\begin{pgfscope}%
\pgfsys@transformshift{2.493181in}{1.577950in}%
\pgfsys@useobject{currentmarker}{}%
\end{pgfscope}%
\begin{pgfscope}%
\pgfsys@transformshift{2.507897in}{1.586356in}%
\pgfsys@useobject{currentmarker}{}%
\end{pgfscope}%
\begin{pgfscope}%
\pgfsys@transformshift{2.522612in}{1.601049in}%
\pgfsys@useobject{currentmarker}{}%
\end{pgfscope}%
\begin{pgfscope}%
\pgfsys@transformshift{2.537327in}{1.620688in}%
\pgfsys@useobject{currentmarker}{}%
\end{pgfscope}%
\begin{pgfscope}%
\pgfsys@transformshift{2.552042in}{1.641588in}%
\pgfsys@useobject{currentmarker}{}%
\end{pgfscope}%
\begin{pgfscope}%
\pgfsys@transformshift{2.566758in}{1.659396in}%
\pgfsys@useobject{currentmarker}{}%
\end{pgfscope}%
\begin{pgfscope}%
\pgfsys@transformshift{2.581473in}{1.670286in}%
\pgfsys@useobject{currentmarker}{}%
\end{pgfscope}%
\begin{pgfscope}%
\pgfsys@transformshift{2.596188in}{1.672681in}%
\pgfsys@useobject{currentmarker}{}%
\end{pgfscope}%
\begin{pgfscope}%
\pgfsys@transformshift{2.610904in}{1.675576in}%
\pgfsys@useobject{currentmarker}{}%
\end{pgfscope}%
\begin{pgfscope}%
\pgfsys@transformshift{2.625619in}{1.679649in}%
\pgfsys@useobject{currentmarker}{}%
\end{pgfscope}%
\begin{pgfscope}%
\pgfsys@transformshift{2.640334in}{1.685784in}%
\pgfsys@useobject{currentmarker}{}%
\end{pgfscope}%
\begin{pgfscope}%
\pgfsys@transformshift{2.655050in}{1.693053in}%
\pgfsys@useobject{currentmarker}{}%
\end{pgfscope}%
\begin{pgfscope}%
\pgfsys@transformshift{2.669765in}{1.697361in}%
\pgfsys@useobject{currentmarker}{}%
\end{pgfscope}%
\begin{pgfscope}%
\pgfsys@transformshift{2.684480in}{1.701869in}%
\pgfsys@useobject{currentmarker}{}%
\end{pgfscope}%
\begin{pgfscope}%
\pgfsys@transformshift{2.699196in}{1.711694in}%
\pgfsys@useobject{currentmarker}{}%
\end{pgfscope}%
\begin{pgfscope}%
\pgfsys@transformshift{2.713911in}{1.728887in}%
\pgfsys@useobject{currentmarker}{}%
\end{pgfscope}%
\begin{pgfscope}%
\pgfsys@transformshift{2.728626in}{1.751192in}%
\pgfsys@useobject{currentmarker}{}%
\end{pgfscope}%
\begin{pgfscope}%
\pgfsys@transformshift{2.743342in}{1.767996in}%
\pgfsys@useobject{currentmarker}{}%
\end{pgfscope}%
\begin{pgfscope}%
\pgfsys@transformshift{2.758057in}{1.776140in}%
\pgfsys@useobject{currentmarker}{}%
\end{pgfscope}%
\begin{pgfscope}%
\pgfsys@transformshift{2.772772in}{1.776718in}%
\pgfsys@useobject{currentmarker}{}%
\end{pgfscope}%
\begin{pgfscope}%
\pgfsys@transformshift{2.787488in}{1.777801in}%
\pgfsys@useobject{currentmarker}{}%
\end{pgfscope}%
\begin{pgfscope}%
\pgfsys@transformshift{2.802203in}{1.778975in}%
\pgfsys@useobject{currentmarker}{}%
\end{pgfscope}%
\begin{pgfscope}%
\pgfsys@transformshift{2.816918in}{1.778277in}%
\pgfsys@useobject{currentmarker}{}%
\end{pgfscope}%
\begin{pgfscope}%
\pgfsys@transformshift{2.831633in}{1.778001in}%
\pgfsys@useobject{currentmarker}{}%
\end{pgfscope}%
\begin{pgfscope}%
\pgfsys@transformshift{2.846349in}{1.777685in}%
\pgfsys@useobject{currentmarker}{}%
\end{pgfscope}%
\begin{pgfscope}%
\pgfsys@transformshift{2.861064in}{1.784161in}%
\pgfsys@useobject{currentmarker}{}%
\end{pgfscope}%
\begin{pgfscope}%
\pgfsys@transformshift{2.875779in}{1.792975in}%
\pgfsys@useobject{currentmarker}{}%
\end{pgfscope}%
\begin{pgfscope}%
\pgfsys@transformshift{2.890495in}{1.803159in}%
\pgfsys@useobject{currentmarker}{}%
\end{pgfscope}%
\begin{pgfscope}%
\pgfsys@transformshift{2.905210in}{1.818237in}%
\pgfsys@useobject{currentmarker}{}%
\end{pgfscope}%
\begin{pgfscope}%
\pgfsys@transformshift{2.919925in}{1.832107in}%
\pgfsys@useobject{currentmarker}{}%
\end{pgfscope}%
\begin{pgfscope}%
\pgfsys@transformshift{2.934641in}{1.835779in}%
\pgfsys@useobject{currentmarker}{}%
\end{pgfscope}%
\begin{pgfscope}%
\pgfsys@transformshift{2.949356in}{1.835461in}%
\pgfsys@useobject{currentmarker}{}%
\end{pgfscope}%
\begin{pgfscope}%
\pgfsys@transformshift{2.964071in}{1.836630in}%
\pgfsys@useobject{currentmarker}{}%
\end{pgfscope}%
\begin{pgfscope}%
\pgfsys@transformshift{2.978787in}{1.843191in}%
\pgfsys@useobject{currentmarker}{}%
\end{pgfscope}%
\begin{pgfscope}%
\pgfsys@transformshift{2.993502in}{1.849236in}%
\pgfsys@useobject{currentmarker}{}%
\end{pgfscope}%
\begin{pgfscope}%
\pgfsys@transformshift{3.008217in}{1.856724in}%
\pgfsys@useobject{currentmarker}{}%
\end{pgfscope}%
\begin{pgfscope}%
\pgfsys@transformshift{3.022933in}{1.861078in}%
\pgfsys@useobject{currentmarker}{}%
\end{pgfscope}%
\begin{pgfscope}%
\pgfsys@transformshift{3.037648in}{1.871415in}%
\pgfsys@useobject{currentmarker}{}%
\end{pgfscope}%
\begin{pgfscope}%
\pgfsys@transformshift{3.052363in}{1.891368in}%
\pgfsys@useobject{currentmarker}{}%
\end{pgfscope}%
\begin{pgfscope}%
\pgfsys@transformshift{3.067079in}{1.915059in}%
\pgfsys@useobject{currentmarker}{}%
\end{pgfscope}%
\begin{pgfscope}%
\pgfsys@transformshift{3.081794in}{1.943641in}%
\pgfsys@useobject{currentmarker}{}%
\end{pgfscope}%
\begin{pgfscope}%
\pgfsys@transformshift{3.096509in}{1.972751in}%
\pgfsys@useobject{currentmarker}{}%
\end{pgfscope}%
\begin{pgfscope}%
\pgfsys@transformshift{3.111224in}{1.989070in}%
\pgfsys@useobject{currentmarker}{}%
\end{pgfscope}%
\begin{pgfscope}%
\pgfsys@transformshift{3.125940in}{1.998766in}%
\pgfsys@useobject{currentmarker}{}%
\end{pgfscope}%
\begin{pgfscope}%
\pgfsys@transformshift{3.140655in}{2.004539in}%
\pgfsys@useobject{currentmarker}{}%
\end{pgfscope}%
\begin{pgfscope}%
\pgfsys@transformshift{3.155370in}{2.014648in}%
\pgfsys@useobject{currentmarker}{}%
\end{pgfscope}%
\begin{pgfscope}%
\pgfsys@transformshift{3.170086in}{2.026022in}%
\pgfsys@useobject{currentmarker}{}%
\end{pgfscope}%
\begin{pgfscope}%
\pgfsys@transformshift{3.184801in}{2.035509in}%
\pgfsys@useobject{currentmarker}{}%
\end{pgfscope}%
\begin{pgfscope}%
\pgfsys@transformshift{3.199516in}{2.041043in}%
\pgfsys@useobject{currentmarker}{}%
\end{pgfscope}%
\begin{pgfscope}%
\pgfsys@transformshift{3.214232in}{2.049307in}%
\pgfsys@useobject{currentmarker}{}%
\end{pgfscope}%
\begin{pgfscope}%
\pgfsys@transformshift{3.228947in}{2.059615in}%
\pgfsys@useobject{currentmarker}{}%
\end{pgfscope}%
\begin{pgfscope}%
\pgfsys@transformshift{3.243662in}{2.072460in}%
\pgfsys@useobject{currentmarker}{}%
\end{pgfscope}%
\begin{pgfscope}%
\pgfsys@transformshift{3.258378in}{2.090022in}%
\pgfsys@useobject{currentmarker}{}%
\end{pgfscope}%
\begin{pgfscope}%
\pgfsys@transformshift{3.273093in}{2.104824in}%
\pgfsys@useobject{currentmarker}{}%
\end{pgfscope}%
\begin{pgfscope}%
\pgfsys@transformshift{3.287808in}{2.106900in}%
\pgfsys@useobject{currentmarker}{}%
\end{pgfscope}%
\begin{pgfscope}%
\pgfsys@transformshift{3.302524in}{2.106269in}%
\pgfsys@useobject{currentmarker}{}%
\end{pgfscope}%
\begin{pgfscope}%
\pgfsys@transformshift{3.317239in}{2.103747in}%
\pgfsys@useobject{currentmarker}{}%
\end{pgfscope}%
\begin{pgfscope}%
\pgfsys@transformshift{3.331954in}{2.103482in}%
\pgfsys@useobject{currentmarker}{}%
\end{pgfscope}%
\begin{pgfscope}%
\pgfsys@transformshift{3.346670in}{2.104585in}%
\pgfsys@useobject{currentmarker}{}%
\end{pgfscope}%
\begin{pgfscope}%
\pgfsys@transformshift{3.361385in}{2.104054in}%
\pgfsys@useobject{currentmarker}{}%
\end{pgfscope}%
\begin{pgfscope}%
\pgfsys@transformshift{3.376100in}{2.101887in}%
\pgfsys@useobject{currentmarker}{}%
\end{pgfscope}%
\begin{pgfscope}%
\pgfsys@transformshift{3.390815in}{2.106217in}%
\pgfsys@useobject{currentmarker}{}%
\end{pgfscope}%
\begin{pgfscope}%
\pgfsys@transformshift{3.405531in}{2.112648in}%
\pgfsys@useobject{currentmarker}{}%
\end{pgfscope}%
\begin{pgfscope}%
\pgfsys@transformshift{3.420246in}{2.124494in}%
\pgfsys@useobject{currentmarker}{}%
\end{pgfscope}%
\begin{pgfscope}%
\pgfsys@transformshift{3.434961in}{2.139990in}%
\pgfsys@useobject{currentmarker}{}%
\end{pgfscope}%
\begin{pgfscope}%
\pgfsys@transformshift{3.449677in}{2.157561in}%
\pgfsys@useobject{currentmarker}{}%
\end{pgfscope}%
\begin{pgfscope}%
\pgfsys@transformshift{3.464392in}{2.165910in}%
\pgfsys@useobject{currentmarker}{}%
\end{pgfscope}%
\begin{pgfscope}%
\pgfsys@transformshift{3.479107in}{2.167528in}%
\pgfsys@useobject{currentmarker}{}%
\end{pgfscope}%
\begin{pgfscope}%
\pgfsys@transformshift{3.493823in}{2.168861in}%
\pgfsys@useobject{currentmarker}{}%
\end{pgfscope}%
\begin{pgfscope}%
\pgfsys@transformshift{3.508538in}{2.171043in}%
\pgfsys@useobject{currentmarker}{}%
\end{pgfscope}%
\begin{pgfscope}%
\pgfsys@transformshift{3.523253in}{2.176138in}%
\pgfsys@useobject{currentmarker}{}%
\end{pgfscope}%
\begin{pgfscope}%
\pgfsys@transformshift{3.537969in}{2.181042in}%
\pgfsys@useobject{currentmarker}{}%
\end{pgfscope}%
\begin{pgfscope}%
\pgfsys@transformshift{3.552684in}{2.184311in}%
\pgfsys@useobject{currentmarker}{}%
\end{pgfscope}%
\begin{pgfscope}%
\pgfsys@transformshift{3.567399in}{2.193499in}%
\pgfsys@useobject{currentmarker}{}%
\end{pgfscope}%
\begin{pgfscope}%
\pgfsys@transformshift{3.582115in}{2.204617in}%
\pgfsys@useobject{currentmarker}{}%
\end{pgfscope}%
\begin{pgfscope}%
\pgfsys@transformshift{3.596830in}{2.221458in}%
\pgfsys@useobject{currentmarker}{}%
\end{pgfscope}%
\begin{pgfscope}%
\pgfsys@transformshift{3.611545in}{2.238656in}%
\pgfsys@useobject{currentmarker}{}%
\end{pgfscope}%
\begin{pgfscope}%
\pgfsys@transformshift{3.626261in}{2.251570in}%
\pgfsys@useobject{currentmarker}{}%
\end{pgfscope}%
\begin{pgfscope}%
\pgfsys@transformshift{3.640976in}{2.257591in}%
\pgfsys@useobject{currentmarker}{}%
\end{pgfscope}%
\begin{pgfscope}%
\pgfsys@transformshift{3.655691in}{2.258013in}%
\pgfsys@useobject{currentmarker}{}%
\end{pgfscope}%
\begin{pgfscope}%
\pgfsys@transformshift{3.670406in}{2.258597in}%
\pgfsys@useobject{currentmarker}{}%
\end{pgfscope}%
\begin{pgfscope}%
\pgfsys@transformshift{3.685122in}{2.261950in}%
\pgfsys@useobject{currentmarker}{}%
\end{pgfscope}%
\begin{pgfscope}%
\pgfsys@transformshift{3.699837in}{2.266938in}%
\pgfsys@useobject{currentmarker}{}%
\end{pgfscope}%
\begin{pgfscope}%
\pgfsys@transformshift{3.714552in}{2.270889in}%
\pgfsys@useobject{currentmarker}{}%
\end{pgfscope}%
\begin{pgfscope}%
\pgfsys@transformshift{3.729268in}{2.274516in}%
\pgfsys@useobject{currentmarker}{}%
\end{pgfscope}%
\begin{pgfscope}%
\pgfsys@transformshift{3.743983in}{2.280472in}%
\pgfsys@useobject{currentmarker}{}%
\end{pgfscope}%
\begin{pgfscope}%
\pgfsys@transformshift{3.758698in}{2.293710in}%
\pgfsys@useobject{currentmarker}{}%
\end{pgfscope}%
\begin{pgfscope}%
\pgfsys@transformshift{3.773414in}{2.314198in}%
\pgfsys@useobject{currentmarker}{}%
\end{pgfscope}%
\begin{pgfscope}%
\pgfsys@transformshift{3.788129in}{2.337720in}%
\pgfsys@useobject{currentmarker}{}%
\end{pgfscope}%
\begin{pgfscope}%
\pgfsys@transformshift{3.802844in}{2.356577in}%
\pgfsys@useobject{currentmarker}{}%
\end{pgfscope}%
\begin{pgfscope}%
\pgfsys@transformshift{3.817560in}{2.368328in}%
\pgfsys@useobject{currentmarker}{}%
\end{pgfscope}%
\begin{pgfscope}%
\pgfsys@transformshift{3.832275in}{2.372259in}%
\pgfsys@useobject{currentmarker}{}%
\end{pgfscope}%
\begin{pgfscope}%
\pgfsys@transformshift{3.846990in}{2.377438in}%
\pgfsys@useobject{currentmarker}{}%
\end{pgfscope}%
\begin{pgfscope}%
\pgfsys@transformshift{3.861706in}{2.384139in}%
\pgfsys@useobject{currentmarker}{}%
\end{pgfscope}%
\begin{pgfscope}%
\pgfsys@transformshift{3.876421in}{2.392817in}%
\pgfsys@useobject{currentmarker}{}%
\end{pgfscope}%
\begin{pgfscope}%
\pgfsys@transformshift{3.891136in}{2.400869in}%
\pgfsys@useobject{currentmarker}{}%
\end{pgfscope}%
\begin{pgfscope}%
\pgfsys@transformshift{3.905852in}{2.408271in}%
\pgfsys@useobject{currentmarker}{}%
\end{pgfscope}%
\begin{pgfscope}%
\pgfsys@transformshift{3.920567in}{2.418455in}%
\pgfsys@useobject{currentmarker}{}%
\end{pgfscope}%
\begin{pgfscope}%
\pgfsys@transformshift{3.935282in}{2.434228in}%
\pgfsys@useobject{currentmarker}{}%
\end{pgfscope}%
\begin{pgfscope}%
\pgfsys@transformshift{3.949997in}{2.455688in}%
\pgfsys@useobject{currentmarker}{}%
\end{pgfscope}%
\begin{pgfscope}%
\pgfsys@transformshift{3.964713in}{2.480505in}%
\pgfsys@useobject{currentmarker}{}%
\end{pgfscope}%
\begin{pgfscope}%
\pgfsys@transformshift{3.979428in}{2.500239in}%
\pgfsys@useobject{currentmarker}{}%
\end{pgfscope}%
\begin{pgfscope}%
\pgfsys@transformshift{3.994143in}{2.513477in}%
\pgfsys@useobject{currentmarker}{}%
\end{pgfscope}%
\begin{pgfscope}%
\pgfsys@transformshift{4.008859in}{2.518048in}%
\pgfsys@useobject{currentmarker}{}%
\end{pgfscope}%
\begin{pgfscope}%
\pgfsys@transformshift{4.023574in}{2.522944in}%
\pgfsys@useobject{currentmarker}{}%
\end{pgfscope}%
\begin{pgfscope}%
\pgfsys@transformshift{4.038289in}{2.528364in}%
\pgfsys@useobject{currentmarker}{}%
\end{pgfscope}%
\begin{pgfscope}%
\pgfsys@transformshift{4.053005in}{2.535294in}%
\pgfsys@useobject{currentmarker}{}%
\end{pgfscope}%
\begin{pgfscope}%
\pgfsys@transformshift{4.067720in}{2.542621in}%
\pgfsys@useobject{currentmarker}{}%
\end{pgfscope}%
\begin{pgfscope}%
\pgfsys@transformshift{4.082435in}{2.548578in}%
\pgfsys@useobject{currentmarker}{}%
\end{pgfscope}%
\begin{pgfscope}%
\pgfsys@transformshift{4.097151in}{2.557649in}%
\pgfsys@useobject{currentmarker}{}%
\end{pgfscope}%
\begin{pgfscope}%
\pgfsys@transformshift{4.111866in}{2.569838in}%
\pgfsys@useobject{currentmarker}{}%
\end{pgfscope}%
\begin{pgfscope}%
\pgfsys@transformshift{4.126581in}{2.585673in}%
\pgfsys@useobject{currentmarker}{}%
\end{pgfscope}%
\begin{pgfscope}%
\pgfsys@transformshift{4.141297in}{2.600336in}%
\pgfsys@useobject{currentmarker}{}%
\end{pgfscope}%
\begin{pgfscope}%
\pgfsys@transformshift{4.156012in}{2.616201in}%
\pgfsys@useobject{currentmarker}{}%
\end{pgfscope}%
\begin{pgfscope}%
\pgfsys@transformshift{4.170727in}{2.622231in}%
\pgfsys@useobject{currentmarker}{}%
\end{pgfscope}%
\begin{pgfscope}%
\pgfsys@transformshift{4.185443in}{2.620784in}%
\pgfsys@useobject{currentmarker}{}%
\end{pgfscope}%
\begin{pgfscope}%
\pgfsys@transformshift{4.200158in}{2.621966in}%
\pgfsys@useobject{currentmarker}{}%
\end{pgfscope}%
\begin{pgfscope}%
\pgfsys@transformshift{4.214873in}{2.624555in}%
\pgfsys@useobject{currentmarker}{}%
\end{pgfscope}%
\begin{pgfscope}%
\pgfsys@transformshift{4.229589in}{2.627482in}%
\pgfsys@useobject{currentmarker}{}%
\end{pgfscope}%
\begin{pgfscope}%
\pgfsys@transformshift{4.244304in}{2.631832in}%
\pgfsys@useobject{currentmarker}{}%
\end{pgfscope}%
\begin{pgfscope}%
\pgfsys@transformshift{4.259019in}{2.634140in}%
\pgfsys@useobject{currentmarker}{}%
\end{pgfscope}%
\begin{pgfscope}%
\pgfsys@transformshift{4.273734in}{2.642769in}%
\pgfsys@useobject{currentmarker}{}%
\end{pgfscope}%
\begin{pgfscope}%
\pgfsys@transformshift{4.288450in}{2.656599in}%
\pgfsys@useobject{currentmarker}{}%
\end{pgfscope}%
\begin{pgfscope}%
\pgfsys@transformshift{4.303165in}{2.681320in}%
\pgfsys@useobject{currentmarker}{}%
\end{pgfscope}%
\begin{pgfscope}%
\pgfsys@transformshift{4.317880in}{2.704411in}%
\pgfsys@useobject{currentmarker}{}%
\end{pgfscope}%
\begin{pgfscope}%
\pgfsys@transformshift{4.332596in}{2.727478in}%
\pgfsys@useobject{currentmarker}{}%
\end{pgfscope}%
\begin{pgfscope}%
\pgfsys@transformshift{4.347311in}{2.738906in}%
\pgfsys@useobject{currentmarker}{}%
\end{pgfscope}%
\begin{pgfscope}%
\pgfsys@transformshift{4.362026in}{2.742298in}%
\pgfsys@useobject{currentmarker}{}%
\end{pgfscope}%
\begin{pgfscope}%
\pgfsys@transformshift{4.376742in}{2.746957in}%
\pgfsys@useobject{currentmarker}{}%
\end{pgfscope}%
\begin{pgfscope}%
\pgfsys@transformshift{4.391457in}{2.755468in}%
\pgfsys@useobject{currentmarker}{}%
\end{pgfscope}%
\begin{pgfscope}%
\pgfsys@transformshift{4.406172in}{2.764177in}%
\pgfsys@useobject{currentmarker}{}%
\end{pgfscope}%
\begin{pgfscope}%
\pgfsys@transformshift{4.420888in}{2.771467in}%
\pgfsys@useobject{currentmarker}{}%
\end{pgfscope}%
\begin{pgfscope}%
\pgfsys@transformshift{4.435603in}{2.773432in}%
\pgfsys@useobject{currentmarker}{}%
\end{pgfscope}%
\begin{pgfscope}%
\pgfsys@transformshift{4.450318in}{2.784919in}%
\pgfsys@useobject{currentmarker}{}%
\end{pgfscope}%
\begin{pgfscope}%
\pgfsys@transformshift{4.465034in}{2.799302in}%
\pgfsys@useobject{currentmarker}{}%
\end{pgfscope}%
\begin{pgfscope}%
\pgfsys@transformshift{4.479749in}{2.816452in}%
\pgfsys@useobject{currentmarker}{}%
\end{pgfscope}%
\begin{pgfscope}%
\pgfsys@transformshift{4.494464in}{2.835248in}%
\pgfsys@useobject{currentmarker}{}%
\end{pgfscope}%
\begin{pgfscope}%
\pgfsys@transformshift{4.509180in}{2.854208in}%
\pgfsys@useobject{currentmarker}{}%
\end{pgfscope}%
\begin{pgfscope}%
\pgfsys@transformshift{4.523895in}{2.864424in}%
\pgfsys@useobject{currentmarker}{}%
\end{pgfscope}%
\begin{pgfscope}%
\pgfsys@transformshift{4.538610in}{2.868236in}%
\pgfsys@useobject{currentmarker}{}%
\end{pgfscope}%
\begin{pgfscope}%
\pgfsys@transformshift{4.553325in}{2.868431in}%
\pgfsys@useobject{currentmarker}{}%
\end{pgfscope}%
\begin{pgfscope}%
\pgfsys@transformshift{4.568041in}{2.870253in}%
\pgfsys@useobject{currentmarker}{}%
\end{pgfscope}%
\begin{pgfscope}%
\pgfsys@transformshift{4.582756in}{2.873841in}%
\pgfsys@useobject{currentmarker}{}%
\end{pgfscope}%
\begin{pgfscope}%
\pgfsys@transformshift{4.597471in}{2.880155in}%
\pgfsys@useobject{currentmarker}{}%
\end{pgfscope}%
\begin{pgfscope}%
\pgfsys@transformshift{4.612187in}{2.879668in}%
\pgfsys@useobject{currentmarker}{}%
\end{pgfscope}%
\begin{pgfscope}%
\pgfsys@transformshift{4.626902in}{2.887436in}%
\pgfsys@useobject{currentmarker}{}%
\end{pgfscope}%
\begin{pgfscope}%
\pgfsys@transformshift{4.641617in}{2.900474in}%
\pgfsys@useobject{currentmarker}{}%
\end{pgfscope}%
\begin{pgfscope}%
\pgfsys@transformshift{4.656333in}{2.919607in}%
\pgfsys@useobject{currentmarker}{}%
\end{pgfscope}%
\begin{pgfscope}%
\pgfsys@transformshift{4.671048in}{2.941342in}%
\pgfsys@useobject{currentmarker}{}%
\end{pgfscope}%
\begin{pgfscope}%
\pgfsys@transformshift{4.685763in}{2.958264in}%
\pgfsys@useobject{currentmarker}{}%
\end{pgfscope}%
\begin{pgfscope}%
\pgfsys@transformshift{4.700479in}{2.967696in}%
\pgfsys@useobject{currentmarker}{}%
\end{pgfscope}%
\begin{pgfscope}%
\pgfsys@transformshift{4.715194in}{2.973105in}%
\pgfsys@useobject{currentmarker}{}%
\end{pgfscope}%
\begin{pgfscope}%
\pgfsys@transformshift{4.729909in}{2.976834in}%
\pgfsys@useobject{currentmarker}{}%
\end{pgfscope}%
\begin{pgfscope}%
\pgfsys@transformshift{4.744625in}{2.982621in}%
\pgfsys@useobject{currentmarker}{}%
\end{pgfscope}%
\begin{pgfscope}%
\pgfsys@transformshift{4.759340in}{2.991287in}%
\pgfsys@useobject{currentmarker}{}%
\end{pgfscope}%
\begin{pgfscope}%
\pgfsys@transformshift{4.774055in}{2.997816in}%
\pgfsys@useobject{currentmarker}{}%
\end{pgfscope}%
\begin{pgfscope}%
\pgfsys@transformshift{4.788771in}{2.995560in}%
\pgfsys@useobject{currentmarker}{}%
\end{pgfscope}%
\begin{pgfscope}%
\pgfsys@transformshift{4.803486in}{2.998789in}%
\pgfsys@useobject{currentmarker}{}%
\end{pgfscope}%
\begin{pgfscope}%
\pgfsys@transformshift{4.818201in}{3.011191in}%
\pgfsys@useobject{currentmarker}{}%
\end{pgfscope}%
\begin{pgfscope}%
\pgfsys@transformshift{4.832916in}{3.028163in}%
\pgfsys@useobject{currentmarker}{}%
\end{pgfscope}%
\begin{pgfscope}%
\pgfsys@transformshift{4.847632in}{3.046547in}%
\pgfsys@useobject{currentmarker}{}%
\end{pgfscope}%
\begin{pgfscope}%
\pgfsys@transformshift{4.862347in}{3.062882in}%
\pgfsys@useobject{currentmarker}{}%
\end{pgfscope}%
\begin{pgfscope}%
\pgfsys@transformshift{4.877062in}{3.073341in}%
\pgfsys@useobject{currentmarker}{}%
\end{pgfscope}%
\begin{pgfscope}%
\pgfsys@transformshift{4.891778in}{3.075652in}%
\pgfsys@useobject{currentmarker}{}%
\end{pgfscope}%
\begin{pgfscope}%
\pgfsys@transformshift{4.906493in}{3.076557in}%
\pgfsys@useobject{currentmarker}{}%
\end{pgfscope}%
\begin{pgfscope}%
\pgfsys@transformshift{4.921208in}{3.076403in}%
\pgfsys@useobject{currentmarker}{}%
\end{pgfscope}%
\begin{pgfscope}%
\pgfsys@transformshift{4.935924in}{3.082094in}%
\pgfsys@useobject{currentmarker}{}%
\end{pgfscope}%
\begin{pgfscope}%
\pgfsys@transformshift{4.950639in}{3.089369in}%
\pgfsys@useobject{currentmarker}{}%
\end{pgfscope}%
\begin{pgfscope}%
\pgfsys@transformshift{4.965354in}{3.097760in}%
\pgfsys@useobject{currentmarker}{}%
\end{pgfscope}%
\begin{pgfscope}%
\pgfsys@transformshift{4.980070in}{3.102791in}%
\pgfsys@useobject{currentmarker}{}%
\end{pgfscope}%
\begin{pgfscope}%
\pgfsys@transformshift{4.994785in}{3.114349in}%
\pgfsys@useobject{currentmarker}{}%
\end{pgfscope}%
\begin{pgfscope}%
\pgfsys@transformshift{5.009500in}{3.134065in}%
\pgfsys@useobject{currentmarker}{}%
\end{pgfscope}%
\begin{pgfscope}%
\pgfsys@transformshift{5.024216in}{3.154122in}%
\pgfsys@useobject{currentmarker}{}%
\end{pgfscope}%
\begin{pgfscope}%
\pgfsys@transformshift{5.038931in}{3.168846in}%
\pgfsys@useobject{currentmarker}{}%
\end{pgfscope}%
\begin{pgfscope}%
\pgfsys@transformshift{5.053646in}{3.178015in}%
\pgfsys@useobject{currentmarker}{}%
\end{pgfscope}%
\begin{pgfscope}%
\pgfsys@transformshift{5.068362in}{3.179180in}%
\pgfsys@useobject{currentmarker}{}%
\end{pgfscope}%
\begin{pgfscope}%
\pgfsys@transformshift{5.083077in}{3.181627in}%
\pgfsys@useobject{currentmarker}{}%
\end{pgfscope}%
\begin{pgfscope}%
\pgfsys@transformshift{5.097792in}{3.183830in}%
\pgfsys@useobject{currentmarker}{}%
\end{pgfscope}%
\begin{pgfscope}%
\pgfsys@transformshift{5.112507in}{3.189995in}%
\pgfsys@useobject{currentmarker}{}%
\end{pgfscope}%
\begin{pgfscope}%
\pgfsys@transformshift{5.127223in}{3.196946in}%
\pgfsys@useobject{currentmarker}{}%
\end{pgfscope}%
\begin{pgfscope}%
\pgfsys@transformshift{5.141938in}{3.205241in}%
\pgfsys@useobject{currentmarker}{}%
\end{pgfscope}%
\begin{pgfscope}%
\pgfsys@transformshift{5.156653in}{3.217121in}%
\pgfsys@useobject{currentmarker}{}%
\end{pgfscope}%
\begin{pgfscope}%
\pgfsys@transformshift{5.171369in}{3.235036in}%
\pgfsys@useobject{currentmarker}{}%
\end{pgfscope}%
\begin{pgfscope}%
\pgfsys@transformshift{5.186084in}{3.256358in}%
\pgfsys@useobject{currentmarker}{}%
\end{pgfscope}%
\begin{pgfscope}%
\pgfsys@transformshift{5.200799in}{3.277636in}%
\pgfsys@useobject{currentmarker}{}%
\end{pgfscope}%
\begin{pgfscope}%
\pgfsys@transformshift{5.215515in}{3.296227in}%
\pgfsys@useobject{currentmarker}{}%
\end{pgfscope}%
\begin{pgfscope}%
\pgfsys@transformshift{5.230230in}{3.308430in}%
\pgfsys@useobject{currentmarker}{}%
\end{pgfscope}%
\begin{pgfscope}%
\pgfsys@transformshift{5.244945in}{3.314615in}%
\pgfsys@useobject{currentmarker}{}%
\end{pgfscope}%
\begin{pgfscope}%
\pgfsys@transformshift{5.259661in}{3.321047in}%
\pgfsys@useobject{currentmarker}{}%
\end{pgfscope}%
\begin{pgfscope}%
\pgfsys@transformshift{5.274376in}{3.326562in}%
\pgfsys@useobject{currentmarker}{}%
\end{pgfscope}%
\begin{pgfscope}%
\pgfsys@transformshift{5.289091in}{3.332215in}%
\pgfsys@useobject{currentmarker}{}%
\end{pgfscope}%
\begin{pgfscope}%
\pgfsys@transformshift{5.303807in}{3.335684in}%
\pgfsys@useobject{currentmarker}{}%
\end{pgfscope}%
\begin{pgfscope}%
\pgfsys@transformshift{5.318522in}{3.336187in}%
\pgfsys@useobject{currentmarker}{}%
\end{pgfscope}%
\begin{pgfscope}%
\pgfsys@transformshift{5.333237in}{3.337267in}%
\pgfsys@useobject{currentmarker}{}%
\end{pgfscope}%
\begin{pgfscope}%
\pgfsys@transformshift{5.347953in}{3.347579in}%
\pgfsys@useobject{currentmarker}{}%
\end{pgfscope}%
\begin{pgfscope}%
\pgfsys@transformshift{5.362668in}{3.365491in}%
\pgfsys@useobject{currentmarker}{}%
\end{pgfscope}%
\begin{pgfscope}%
\pgfsys@transformshift{5.377383in}{3.386856in}%
\pgfsys@useobject{currentmarker}{}%
\end{pgfscope}%
\begin{pgfscope}%
\pgfsys@transformshift{5.392098in}{3.404116in}%
\pgfsys@useobject{currentmarker}{}%
\end{pgfscope}%
\begin{pgfscope}%
\pgfsys@transformshift{5.406814in}{3.413431in}%
\pgfsys@useobject{currentmarker}{}%
\end{pgfscope}%
\begin{pgfscope}%
\pgfsys@transformshift{5.421529in}{3.415260in}%
\pgfsys@useobject{currentmarker}{}%
\end{pgfscope}%
\begin{pgfscope}%
\pgfsys@transformshift{5.436244in}{3.417860in}%
\pgfsys@useobject{currentmarker}{}%
\end{pgfscope}%
\begin{pgfscope}%
\pgfsys@transformshift{5.450960in}{3.420638in}%
\pgfsys@useobject{currentmarker}{}%
\end{pgfscope}%
\begin{pgfscope}%
\pgfsys@transformshift{5.465675in}{3.427119in}%
\pgfsys@useobject{currentmarker}{}%
\end{pgfscope}%
\begin{pgfscope}%
\pgfsys@transformshift{5.480390in}{3.433674in}%
\pgfsys@useobject{currentmarker}{}%
\end{pgfscope}%
\begin{pgfscope}%
\pgfsys@transformshift{5.495106in}{3.439641in}%
\pgfsys@useobject{currentmarker}{}%
\end{pgfscope}%
\begin{pgfscope}%
\pgfsys@transformshift{5.509821in}{3.449791in}%
\pgfsys@useobject{currentmarker}{}%
\end{pgfscope}%
\begin{pgfscope}%
\pgfsys@transformshift{5.524536in}{3.465265in}%
\pgfsys@useobject{currentmarker}{}%
\end{pgfscope}%
\begin{pgfscope}%
\pgfsys@transformshift{5.539252in}{3.481686in}%
\pgfsys@useobject{currentmarker}{}%
\end{pgfscope}%
\begin{pgfscope}%
\pgfsys@transformshift{5.553967in}{3.503171in}%
\pgfsys@useobject{currentmarker}{}%
\end{pgfscope}%
\begin{pgfscope}%
\pgfsys@transformshift{5.568682in}{3.520206in}%
\pgfsys@useobject{currentmarker}{}%
\end{pgfscope}%
\begin{pgfscope}%
\pgfsys@transformshift{5.583398in}{3.531028in}%
\pgfsys@useobject{currentmarker}{}%
\end{pgfscope}%
\begin{pgfscope}%
\pgfsys@transformshift{5.598113in}{3.535026in}%
\pgfsys@useobject{currentmarker}{}%
\end{pgfscope}%
\begin{pgfscope}%
\pgfsys@transformshift{5.612828in}{3.541088in}%
\pgfsys@useobject{currentmarker}{}%
\end{pgfscope}%
\begin{pgfscope}%
\pgfsys@transformshift{5.627544in}{3.547493in}%
\pgfsys@useobject{currentmarker}{}%
\end{pgfscope}%
\begin{pgfscope}%
\pgfsys@transformshift{5.642259in}{3.556461in}%
\pgfsys@useobject{currentmarker}{}%
\end{pgfscope}%
\begin{pgfscope}%
\pgfsys@transformshift{5.656974in}{3.569029in}%
\pgfsys@useobject{currentmarker}{}%
\end{pgfscope}%
\begin{pgfscope}%
\pgfsys@transformshift{5.671689in}{3.575642in}%
\pgfsys@useobject{currentmarker}{}%
\end{pgfscope}%
\begin{pgfscope}%
\pgfsys@transformshift{5.686405in}{3.584304in}%
\pgfsys@useobject{currentmarker}{}%
\end{pgfscope}%
\begin{pgfscope}%
\pgfsys@transformshift{5.701120in}{3.605340in}%
\pgfsys@useobject{currentmarker}{}%
\end{pgfscope}%
\begin{pgfscope}%
\pgfsys@transformshift{5.715835in}{3.627308in}%
\pgfsys@useobject{currentmarker}{}%
\end{pgfscope}%
\begin{pgfscope}%
\pgfsys@transformshift{5.730551in}{3.652389in}%
\pgfsys@useobject{currentmarker}{}%
\end{pgfscope}%
\begin{pgfscope}%
\pgfsys@transformshift{5.745266in}{3.673363in}%
\pgfsys@useobject{currentmarker}{}%
\end{pgfscope}%
\begin{pgfscope}%
\pgfsys@transformshift{5.759981in}{3.685967in}%
\pgfsys@useobject{currentmarker}{}%
\end{pgfscope}%
\begin{pgfscope}%
\pgfsys@transformshift{5.774697in}{3.689470in}%
\pgfsys@useobject{currentmarker}{}%
\end{pgfscope}%
\begin{pgfscope}%
\pgfsys@transformshift{5.789412in}{3.693557in}%
\pgfsys@useobject{currentmarker}{}%
\end{pgfscope}%
\begin{pgfscope}%
\pgfsys@transformshift{5.804127in}{3.699849in}%
\pgfsys@useobject{currentmarker}{}%
\end{pgfscope}%
\begin{pgfscope}%
\pgfsys@transformshift{5.818843in}{3.705054in}%
\pgfsys@useobject{currentmarker}{}%
\end{pgfscope}%
\begin{pgfscope}%
\pgfsys@transformshift{5.833558in}{3.708111in}%
\pgfsys@useobject{currentmarker}{}%
\end{pgfscope}%
\begin{pgfscope}%
\pgfsys@transformshift{5.848273in}{3.714668in}%
\pgfsys@useobject{currentmarker}{}%
\end{pgfscope}%
\begin{pgfscope}%
\pgfsys@transformshift{5.862989in}{3.722206in}%
\pgfsys@useobject{currentmarker}{}%
\end{pgfscope}%
\begin{pgfscope}%
\pgfsys@transformshift{5.877704in}{3.738103in}%
\pgfsys@useobject{currentmarker}{}%
\end{pgfscope}%
\begin{pgfscope}%
\pgfsys@transformshift{5.892419in}{3.758595in}%
\pgfsys@useobject{currentmarker}{}%
\end{pgfscope}%
\begin{pgfscope}%
\pgfsys@transformshift{5.907135in}{3.778983in}%
\pgfsys@useobject{currentmarker}{}%
\end{pgfscope}%
\begin{pgfscope}%
\pgfsys@transformshift{5.921850in}{3.796612in}%
\pgfsys@useobject{currentmarker}{}%
\end{pgfscope}%
\begin{pgfscope}%
\pgfsys@transformshift{5.936565in}{3.804793in}%
\pgfsys@useobject{currentmarker}{}%
\end{pgfscope}%
\begin{pgfscope}%
\pgfsys@transformshift{5.951280in}{3.808752in}%
\pgfsys@useobject{currentmarker}{}%
\end{pgfscope}%
\begin{pgfscope}%
\pgfsys@transformshift{5.965996in}{3.810561in}%
\pgfsys@useobject{currentmarker}{}%
\end{pgfscope}%
\begin{pgfscope}%
\pgfsys@transformshift{5.980711in}{3.815976in}%
\pgfsys@useobject{currentmarker}{}%
\end{pgfscope}%
\begin{pgfscope}%
\pgfsys@transformshift{5.995426in}{3.825769in}%
\pgfsys@useobject{currentmarker}{}%
\end{pgfscope}%
\begin{pgfscope}%
\pgfsys@transformshift{6.010142in}{3.829149in}%
\pgfsys@useobject{currentmarker}{}%
\end{pgfscope}%
\begin{pgfscope}%
\pgfsys@transformshift{6.024857in}{3.830599in}%
\pgfsys@useobject{currentmarker}{}%
\end{pgfscope}%
\begin{pgfscope}%
\pgfsys@transformshift{6.039572in}{3.837938in}%
\pgfsys@useobject{currentmarker}{}%
\end{pgfscope}%
\begin{pgfscope}%
\pgfsys@transformshift{6.054288in}{3.851798in}%
\pgfsys@useobject{currentmarker}{}%
\end{pgfscope}%
\begin{pgfscope}%
\pgfsys@transformshift{6.069003in}{3.870187in}%
\pgfsys@useobject{currentmarker}{}%
\end{pgfscope}%
\begin{pgfscope}%
\pgfsys@transformshift{6.083718in}{3.892301in}%
\pgfsys@useobject{currentmarker}{}%
\end{pgfscope}%
\begin{pgfscope}%
\pgfsys@transformshift{6.098434in}{3.910675in}%
\pgfsys@useobject{currentmarker}{}%
\end{pgfscope}%
\begin{pgfscope}%
\pgfsys@transformshift{6.113149in}{3.919343in}%
\pgfsys@useobject{currentmarker}{}%
\end{pgfscope}%
\begin{pgfscope}%
\pgfsys@transformshift{6.127864in}{3.925113in}%
\pgfsys@useobject{currentmarker}{}%
\end{pgfscope}%
\begin{pgfscope}%
\pgfsys@transformshift{6.142580in}{3.931236in}%
\pgfsys@useobject{currentmarker}{}%
\end{pgfscope}%
\begin{pgfscope}%
\pgfsys@transformshift{6.157295in}{3.940934in}%
\pgfsys@useobject{currentmarker}{}%
\end{pgfscope}%
\begin{pgfscope}%
\pgfsys@transformshift{6.172010in}{3.952216in}%
\pgfsys@useobject{currentmarker}{}%
\end{pgfscope}%
\begin{pgfscope}%
\pgfsys@transformshift{6.186726in}{3.965153in}%
\pgfsys@useobject{currentmarker}{}%
\end{pgfscope}%
\begin{pgfscope}%
\pgfsys@transformshift{6.201441in}{3.973759in}%
\pgfsys@useobject{currentmarker}{}%
\end{pgfscope}%
\begin{pgfscope}%
\pgfsys@transformshift{6.216156in}{3.991632in}%
\pgfsys@useobject{currentmarker}{}%
\end{pgfscope}%
\begin{pgfscope}%
\pgfsys@transformshift{6.230871in}{4.017081in}%
\pgfsys@useobject{currentmarker}{}%
\end{pgfscope}%
\begin{pgfscope}%
\pgfsys@transformshift{6.245587in}{4.045501in}%
\pgfsys@useobject{currentmarker}{}%
\end{pgfscope}%
\begin{pgfscope}%
\pgfsys@transformshift{6.260302in}{4.073440in}%
\pgfsys@useobject{currentmarker}{}%
\end{pgfscope}%
\begin{pgfscope}%
\pgfsys@transformshift{6.275017in}{4.097138in}%
\pgfsys@useobject{currentmarker}{}%
\end{pgfscope}%
\begin{pgfscope}%
\pgfsys@transformshift{6.289733in}{4.111170in}%
\pgfsys@useobject{currentmarker}{}%
\end{pgfscope}%
\begin{pgfscope}%
\pgfsys@transformshift{6.304448in}{4.118858in}%
\pgfsys@useobject{currentmarker}{}%
\end{pgfscope}%
\begin{pgfscope}%
\pgfsys@transformshift{6.319163in}{4.127327in}%
\pgfsys@useobject{currentmarker}{}%
\end{pgfscope}%
\begin{pgfscope}%
\pgfsys@transformshift{6.333879in}{4.137115in}%
\pgfsys@useobject{currentmarker}{}%
\end{pgfscope}%
\begin{pgfscope}%
\pgfsys@transformshift{6.348594in}{4.147584in}%
\pgfsys@useobject{currentmarker}{}%
\end{pgfscope}%
\begin{pgfscope}%
\pgfsys@transformshift{6.363309in}{4.155727in}%
\pgfsys@useobject{currentmarker}{}%
\end{pgfscope}%
\begin{pgfscope}%
\pgfsys@transformshift{6.378025in}{4.155486in}%
\pgfsys@useobject{currentmarker}{}%
\end{pgfscope}%
\begin{pgfscope}%
\pgfsys@transformshift{6.392740in}{4.162292in}%
\pgfsys@useobject{currentmarker}{}%
\end{pgfscope}%
\begin{pgfscope}%
\pgfsys@transformshift{6.407455in}{4.177077in}%
\pgfsys@useobject{currentmarker}{}%
\end{pgfscope}%
\begin{pgfscope}%
\pgfsys@transformshift{6.422171in}{4.200509in}%
\pgfsys@useobject{currentmarker}{}%
\end{pgfscope}%
\begin{pgfscope}%
\pgfsys@transformshift{6.436886in}{4.225524in}%
\pgfsys@useobject{currentmarker}{}%
\end{pgfscope}%
\begin{pgfscope}%
\pgfsys@transformshift{6.451601in}{4.246301in}%
\pgfsys@useobject{currentmarker}{}%
\end{pgfscope}%
\begin{pgfscope}%
\pgfsys@transformshift{6.466317in}{4.255530in}%
\pgfsys@useobject{currentmarker}{}%
\end{pgfscope}%
\begin{pgfscope}%
\pgfsys@transformshift{6.481032in}{4.256028in}%
\pgfsys@useobject{currentmarker}{}%
\end{pgfscope}%
\begin{pgfscope}%
\pgfsys@transformshift{6.495747in}{4.259007in}%
\pgfsys@useobject{currentmarker}{}%
\end{pgfscope}%
\begin{pgfscope}%
\pgfsys@transformshift{6.510463in}{4.262429in}%
\pgfsys@useobject{currentmarker}{}%
\end{pgfscope}%
\begin{pgfscope}%
\pgfsys@transformshift{6.525178in}{4.270197in}%
\pgfsys@useobject{currentmarker}{}%
\end{pgfscope}%
\begin{pgfscope}%
\pgfsys@transformshift{6.539893in}{4.275032in}%
\pgfsys@useobject{currentmarker}{}%
\end{pgfscope}%
\begin{pgfscope}%
\pgfsys@transformshift{6.554608in}{4.276797in}%
\pgfsys@useobject{currentmarker}{}%
\end{pgfscope}%
\begin{pgfscope}%
\pgfsys@transformshift{6.569324in}{4.279614in}%
\pgfsys@useobject{currentmarker}{}%
\end{pgfscope}%
\begin{pgfscope}%
\pgfsys@transformshift{6.584039in}{4.291671in}%
\pgfsys@useobject{currentmarker}{}%
\end{pgfscope}%
\begin{pgfscope}%
\pgfsys@transformshift{6.598754in}{4.310608in}%
\pgfsys@useobject{currentmarker}{}%
\end{pgfscope}%
\begin{pgfscope}%
\pgfsys@transformshift{6.613470in}{4.328942in}%
\pgfsys@useobject{currentmarker}{}%
\end{pgfscope}%
\begin{pgfscope}%
\pgfsys@transformshift{6.628185in}{4.347268in}%
\pgfsys@useobject{currentmarker}{}%
\end{pgfscope}%
\begin{pgfscope}%
\pgfsys@transformshift{6.642900in}{4.357051in}%
\pgfsys@useobject{currentmarker}{}%
\end{pgfscope}%
\begin{pgfscope}%
\pgfsys@transformshift{6.657616in}{4.361934in}%
\pgfsys@useobject{currentmarker}{}%
\end{pgfscope}%
\begin{pgfscope}%
\pgfsys@transformshift{6.672331in}{4.368484in}%
\pgfsys@useobject{currentmarker}{}%
\end{pgfscope}%
\begin{pgfscope}%
\pgfsys@transformshift{6.687046in}{4.374283in}%
\pgfsys@useobject{currentmarker}{}%
\end{pgfscope}%
\begin{pgfscope}%
\pgfsys@transformshift{6.701762in}{4.387235in}%
\pgfsys@useobject{currentmarker}{}%
\end{pgfscope}%
\begin{pgfscope}%
\pgfsys@transformshift{6.716477in}{4.399474in}%
\pgfsys@useobject{currentmarker}{}%
\end{pgfscope}%
\begin{pgfscope}%
\pgfsys@transformshift{6.731192in}{4.408226in}%
\pgfsys@useobject{currentmarker}{}%
\end{pgfscope}%
\begin{pgfscope}%
\pgfsys@transformshift{6.745908in}{4.419128in}%
\pgfsys@useobject{currentmarker}{}%
\end{pgfscope}%
\begin{pgfscope}%
\pgfsys@transformshift{6.760623in}{4.438960in}%
\pgfsys@useobject{currentmarker}{}%
\end{pgfscope}%
\begin{pgfscope}%
\pgfsys@transformshift{6.775338in}{4.466111in}%
\pgfsys@useobject{currentmarker}{}%
\end{pgfscope}%
\begin{pgfscope}%
\pgfsys@transformshift{6.790054in}{4.491011in}%
\pgfsys@useobject{currentmarker}{}%
\end{pgfscope}%
\begin{pgfscope}%
\pgfsys@transformshift{6.804769in}{4.514060in}%
\pgfsys@useobject{currentmarker}{}%
\end{pgfscope}%
\begin{pgfscope}%
\pgfsys@transformshift{6.819484in}{4.527259in}%
\pgfsys@useobject{currentmarker}{}%
\end{pgfscope}%
\begin{pgfscope}%
\pgfsys@transformshift{6.834199in}{4.529564in}%
\pgfsys@useobject{currentmarker}{}%
\end{pgfscope}%
\begin{pgfscope}%
\pgfsys@transformshift{6.848915in}{4.535839in}%
\pgfsys@useobject{currentmarker}{}%
\end{pgfscope}%
\begin{pgfscope}%
\pgfsys@transformshift{6.863630in}{4.540569in}%
\pgfsys@useobject{currentmarker}{}%
\end{pgfscope}%
\begin{pgfscope}%
\pgfsys@transformshift{6.878345in}{4.548996in}%
\pgfsys@useobject{currentmarker}{}%
\end{pgfscope}%
\begin{pgfscope}%
\pgfsys@transformshift{6.893061in}{4.560244in}%
\pgfsys@useobject{currentmarker}{}%
\end{pgfscope}%
\begin{pgfscope}%
\pgfsys@transformshift{6.907776in}{4.566068in}%
\pgfsys@useobject{currentmarker}{}%
\end{pgfscope}%
\begin{pgfscope}%
\pgfsys@transformshift{6.922491in}{4.574068in}%
\pgfsys@useobject{currentmarker}{}%
\end{pgfscope}%
\begin{pgfscope}%
\pgfsys@transformshift{6.937207in}{4.590186in}%
\pgfsys@useobject{currentmarker}{}%
\end{pgfscope}%
\begin{pgfscope}%
\pgfsys@transformshift{6.951922in}{4.613428in}%
\pgfsys@useobject{currentmarker}{}%
\end{pgfscope}%
\begin{pgfscope}%
\pgfsys@transformshift{6.966637in}{4.635982in}%
\pgfsys@useobject{currentmarker}{}%
\end{pgfscope}%
\begin{pgfscope}%
\pgfsys@transformshift{6.981353in}{4.656016in}%
\pgfsys@useobject{currentmarker}{}%
\end{pgfscope}%
\begin{pgfscope}%
\pgfsys@transformshift{6.996068in}{4.670281in}%
\pgfsys@useobject{currentmarker}{}%
\end{pgfscope}%
\begin{pgfscope}%
\pgfsys@transformshift{7.010783in}{4.675262in}%
\pgfsys@useobject{currentmarker}{}%
\end{pgfscope}%
\begin{pgfscope}%
\pgfsys@transformshift{7.025499in}{4.681072in}%
\pgfsys@useobject{currentmarker}{}%
\end{pgfscope}%
\begin{pgfscope}%
\pgfsys@transformshift{7.040214in}{4.684230in}%
\pgfsys@useobject{currentmarker}{}%
\end{pgfscope}%
\begin{pgfscope}%
\pgfsys@transformshift{7.054929in}{4.690170in}%
\pgfsys@useobject{currentmarker}{}%
\end{pgfscope}%
\begin{pgfscope}%
\pgfsys@transformshift{7.069645in}{4.700575in}%
\pgfsys@useobject{currentmarker}{}%
\end{pgfscope}%
\begin{pgfscope}%
\pgfsys@transformshift{7.084360in}{4.706463in}%
\pgfsys@useobject{currentmarker}{}%
\end{pgfscope}%
\begin{pgfscope}%
\pgfsys@transformshift{7.099075in}{4.715296in}%
\pgfsys@useobject{currentmarker}{}%
\end{pgfscope}%
\begin{pgfscope}%
\pgfsys@transformshift{7.113790in}{4.728393in}%
\pgfsys@useobject{currentmarker}{}%
\end{pgfscope}%
\begin{pgfscope}%
\pgfsys@transformshift{7.128506in}{4.750686in}%
\pgfsys@useobject{currentmarker}{}%
\end{pgfscope}%
\begin{pgfscope}%
\pgfsys@transformshift{7.143221in}{4.772414in}%
\pgfsys@useobject{currentmarker}{}%
\end{pgfscope}%
\begin{pgfscope}%
\pgfsys@transformshift{7.157936in}{4.787913in}%
\pgfsys@useobject{currentmarker}{}%
\end{pgfscope}%
\begin{pgfscope}%
\pgfsys@transformshift{7.172652in}{4.797071in}%
\pgfsys@useobject{currentmarker}{}%
\end{pgfscope}%
\begin{pgfscope}%
\pgfsys@transformshift{7.187367in}{4.801078in}%
\pgfsys@useobject{currentmarker}{}%
\end{pgfscope}%
\begin{pgfscope}%
\pgfsys@transformshift{7.202082in}{4.804962in}%
\pgfsys@useobject{currentmarker}{}%
\end{pgfscope}%
\begin{pgfscope}%
\pgfsys@transformshift{7.216798in}{4.811099in}%
\pgfsys@useobject{currentmarker}{}%
\end{pgfscope}%
\begin{pgfscope}%
\pgfsys@transformshift{7.231513in}{4.818262in}%
\pgfsys@useobject{currentmarker}{}%
\end{pgfscope}%
\begin{pgfscope}%
\pgfsys@transformshift{7.246228in}{4.826838in}%
\pgfsys@useobject{currentmarker}{}%
\end{pgfscope}%
\begin{pgfscope}%
\pgfsys@transformshift{7.260944in}{4.825589in}%
\pgfsys@useobject{currentmarker}{}%
\end{pgfscope}%
\begin{pgfscope}%
\pgfsys@transformshift{7.275659in}{4.831321in}%
\pgfsys@useobject{currentmarker}{}%
\end{pgfscope}%
\begin{pgfscope}%
\pgfsys@transformshift{7.290374in}{4.841690in}%
\pgfsys@useobject{currentmarker}{}%
\end{pgfscope}%
\begin{pgfscope}%
\pgfsys@transformshift{7.305090in}{4.862388in}%
\pgfsys@useobject{currentmarker}{}%
\end{pgfscope}%
\begin{pgfscope}%
\pgfsys@transformshift{7.319805in}{4.884613in}%
\pgfsys@useobject{currentmarker}{}%
\end{pgfscope}%
\begin{pgfscope}%
\pgfsys@transformshift{7.334520in}{4.904716in}%
\pgfsys@useobject{currentmarker}{}%
\end{pgfscope}%
\begin{pgfscope}%
\pgfsys@transformshift{7.349236in}{4.915035in}%
\pgfsys@useobject{currentmarker}{}%
\end{pgfscope}%
\begin{pgfscope}%
\pgfsys@transformshift{7.363951in}{4.918883in}%
\pgfsys@useobject{currentmarker}{}%
\end{pgfscope}%
\begin{pgfscope}%
\pgfsys@transformshift{7.378666in}{4.922938in}%
\pgfsys@useobject{currentmarker}{}%
\end{pgfscope}%
\begin{pgfscope}%
\pgfsys@transformshift{7.393381in}{4.930079in}%
\pgfsys@useobject{currentmarker}{}%
\end{pgfscope}%
\begin{pgfscope}%
\pgfsys@transformshift{7.408097in}{4.940218in}%
\pgfsys@useobject{currentmarker}{}%
\end{pgfscope}%
\begin{pgfscope}%
\pgfsys@transformshift{7.422812in}{4.959061in}%
\pgfsys@useobject{currentmarker}{}%
\end{pgfscope}%
\begin{pgfscope}%
\pgfsys@transformshift{7.437527in}{4.975258in}%
\pgfsys@useobject{currentmarker}{}%
\end{pgfscope}%
\begin{pgfscope}%
\pgfsys@transformshift{7.452243in}{4.989732in}%
\pgfsys@useobject{currentmarker}{}%
\end{pgfscope}%
\begin{pgfscope}%
\pgfsys@transformshift{7.466958in}{5.003848in}%
\pgfsys@useobject{currentmarker}{}%
\end{pgfscope}%
\begin{pgfscope}%
\pgfsys@transformshift{7.481673in}{5.025107in}%
\pgfsys@useobject{currentmarker}{}%
\end{pgfscope}%
\begin{pgfscope}%
\pgfsys@transformshift{7.496389in}{5.047913in}%
\pgfsys@useobject{currentmarker}{}%
\end{pgfscope}%
\begin{pgfscope}%
\pgfsys@transformshift{7.511104in}{5.067588in}%
\pgfsys@useobject{currentmarker}{}%
\end{pgfscope}%
\begin{pgfscope}%
\pgfsys@transformshift{7.525819in}{5.081354in}%
\pgfsys@useobject{currentmarker}{}%
\end{pgfscope}%
\begin{pgfscope}%
\pgfsys@transformshift{7.540535in}{5.088342in}%
\pgfsys@useobject{currentmarker}{}%
\end{pgfscope}%
\begin{pgfscope}%
\pgfsys@transformshift{7.555250in}{5.092673in}%
\pgfsys@useobject{currentmarker}{}%
\end{pgfscope}%
\begin{pgfscope}%
\pgfsys@transformshift{7.569965in}{5.096181in}%
\pgfsys@useobject{currentmarker}{}%
\end{pgfscope}%
\begin{pgfscope}%
\pgfsys@transformshift{7.584681in}{5.098267in}%
\pgfsys@useobject{currentmarker}{}%
\end{pgfscope}%
\begin{pgfscope}%
\pgfsys@transformshift{7.599396in}{5.104478in}%
\pgfsys@useobject{currentmarker}{}%
\end{pgfscope}%
\begin{pgfscope}%
\pgfsys@transformshift{7.614111in}{5.112573in}%
\pgfsys@useobject{currentmarker}{}%
\end{pgfscope}%
\begin{pgfscope}%
\pgfsys@transformshift{7.628827in}{5.127426in}%
\pgfsys@useobject{currentmarker}{}%
\end{pgfscope}%
\begin{pgfscope}%
\pgfsys@transformshift{7.643542in}{5.147461in}%
\pgfsys@useobject{currentmarker}{}%
\end{pgfscope}%
\begin{pgfscope}%
\pgfsys@transformshift{7.658257in}{5.174374in}%
\pgfsys@useobject{currentmarker}{}%
\end{pgfscope}%
\begin{pgfscope}%
\pgfsys@transformshift{7.672972in}{5.204844in}%
\pgfsys@useobject{currentmarker}{}%
\end{pgfscope}%
\begin{pgfscope}%
\pgfsys@transformshift{7.687688in}{5.228337in}%
\pgfsys@useobject{currentmarker}{}%
\end{pgfscope}%
\begin{pgfscope}%
\pgfsys@transformshift{7.702403in}{5.244887in}%
\pgfsys@useobject{currentmarker}{}%
\end{pgfscope}%
\begin{pgfscope}%
\pgfsys@transformshift{7.717118in}{5.255499in}%
\pgfsys@useobject{currentmarker}{}%
\end{pgfscope}%
\begin{pgfscope}%
\pgfsys@transformshift{7.731834in}{5.265623in}%
\pgfsys@useobject{currentmarker}{}%
\end{pgfscope}%
\begin{pgfscope}%
\pgfsys@transformshift{7.746549in}{5.278841in}%
\pgfsys@useobject{currentmarker}{}%
\end{pgfscope}%
\end{pgfscope}%
\begin{pgfscope}%
\pgfsetrectcap%
\pgfsetmiterjoin%
\pgfsetlinewidth{0.803000pt}%
\definecolor{currentstroke}{rgb}{0.000000,0.000000,0.000000}%
\pgfsetstrokecolor{currentstroke}%
\pgfsetdash{}{0pt}%
\pgfpathmoveto{\pgfqpoint{0.697913in}{0.559721in}}%
\pgfpathlineto{\pgfqpoint{0.697913in}{5.550000in}}%
\pgfusepath{stroke}%
\end{pgfscope}%
\begin{pgfscope}%
\pgfsetrectcap%
\pgfsetmiterjoin%
\pgfsetlinewidth{0.803000pt}%
\definecolor{currentstroke}{rgb}{0.000000,0.000000,0.000000}%
\pgfsetstrokecolor{currentstroke}%
\pgfsetdash{}{0pt}%
\pgfpathmoveto{\pgfqpoint{7.746549in}{0.559721in}}%
\pgfpathlineto{\pgfqpoint{7.746549in}{5.550000in}}%
\pgfusepath{stroke}%
\end{pgfscope}%
\begin{pgfscope}%
\pgfsetrectcap%
\pgfsetmiterjoin%
\pgfsetlinewidth{0.803000pt}%
\definecolor{currentstroke}{rgb}{0.000000,0.000000,0.000000}%
\pgfsetstrokecolor{currentstroke}%
\pgfsetdash{}{0pt}%
\pgfpathmoveto{\pgfqpoint{0.697913in}{0.559721in}}%
\pgfpathlineto{\pgfqpoint{7.746549in}{0.559721in}}%
\pgfusepath{stroke}%
\end{pgfscope}%
\begin{pgfscope}%
\pgfsetrectcap%
\pgfsetmiterjoin%
\pgfsetlinewidth{0.803000pt}%
\definecolor{currentstroke}{rgb}{0.000000,0.000000,0.000000}%
\pgfsetstrokecolor{currentstroke}%
\pgfsetdash{}{0pt}%
\pgfpathmoveto{\pgfqpoint{0.697913in}{5.550000in}}%
\pgfpathlineto{\pgfqpoint{7.746549in}{5.550000in}}%
\pgfusepath{stroke}%
\end{pgfscope}%
\begin{pgfscope}%
\definecolor{textcolor}{rgb}{0.000000,0.000000,0.000000}%
\pgfsetstrokecolor{textcolor}%
\pgfsetfillcolor{textcolor}%
\pgftext[x=4.222231in,y=5.633333in,,base]{\color{textcolor}{\rmfamily\fontsize{18.000000}{21.600000}\selectfont\catcode`\^=\active\def^{\ifmmode\sp\else\^{}\fi}\catcode`\%=\active\def%{\%}Atmospheric CO$_2$ at Mauna Loa Observatory}}%
\end{pgfscope}%
\begin{pgfscope}%
\pgfsetbuttcap%
\pgfsetmiterjoin%
\definecolor{currentfill}{rgb}{1.000000,1.000000,1.000000}%
\pgfsetfillcolor{currentfill}%
\pgfsetfillopacity{0.800000}%
\pgfsetlinewidth{1.003750pt}%
\definecolor{currentstroke}{rgb}{0.800000,0.800000,0.800000}%
\pgfsetstrokecolor{currentstroke}%
\pgfsetstrokeopacity{0.800000}%
\pgfsetdash{}{0pt}%
\pgfpathmoveto{\pgfqpoint{0.834024in}{4.844445in}}%
\pgfpathlineto{\pgfqpoint{2.666486in}{4.844445in}}%
\pgfpathquadraticcurveto{\pgfqpoint{2.705375in}{4.844445in}}{\pgfqpoint{2.705375in}{4.883334in}}%
\pgfpathlineto{\pgfqpoint{2.705375in}{5.413889in}}%
\pgfpathquadraticcurveto{\pgfqpoint{2.705375in}{5.452778in}}{\pgfqpoint{2.666486in}{5.452778in}}%
\pgfpathlineto{\pgfqpoint{0.834024in}{5.452778in}}%
\pgfpathquadraticcurveto{\pgfqpoint{0.795135in}{5.452778in}}{\pgfqpoint{0.795135in}{5.413889in}}%
\pgfpathlineto{\pgfqpoint{0.795135in}{4.883334in}}%
\pgfpathquadraticcurveto{\pgfqpoint{0.795135in}{4.844445in}}{\pgfqpoint{0.834024in}{4.844445in}}%
\pgfpathlineto{\pgfqpoint{0.834024in}{4.844445in}}%
\pgfpathclose%
\pgfusepath{stroke,fill}%
\end{pgfscope}%
\begin{pgfscope}%
\pgfsetrectcap%
\pgfsetroundjoin%
\pgfsetlinewidth{1.003750pt}%
\definecolor{currentstroke}{rgb}{1.000000,0.000000,0.000000}%
\pgfsetstrokecolor{currentstroke}%
\pgfsetdash{}{0pt}%
\pgfpathmoveto{\pgfqpoint{0.872913in}{5.304167in}}%
\pgfpathlineto{\pgfqpoint{1.067357in}{5.304167in}}%
\pgfpathlineto{\pgfqpoint{1.261802in}{5.304167in}}%
\pgfusepath{stroke}%
\end{pgfscope}%
\begin{pgfscope}%
\pgfsetbuttcap%
\pgfsetroundjoin%
\definecolor{currentfill}{rgb}{1.000000,0.000000,0.000000}%
\pgfsetfillcolor{currentfill}%
\pgfsetlinewidth{1.003750pt}%
\definecolor{currentstroke}{rgb}{1.000000,0.000000,0.000000}%
\pgfsetstrokecolor{currentstroke}%
\pgfsetdash{}{0pt}%
\pgfsys@defobject{currentmarker}{\pgfqpoint{-0.020833in}{-0.020833in}}{\pgfqpoint{0.020833in}{0.020833in}}{%
\pgfpathmoveto{\pgfqpoint{0.000000in}{-0.020833in}}%
\pgfpathcurveto{\pgfqpoint{0.005525in}{-0.020833in}}{\pgfqpoint{0.010825in}{-0.018638in}}{\pgfqpoint{0.014731in}{-0.014731in}}%
\pgfpathcurveto{\pgfqpoint{0.018638in}{-0.010825in}}{\pgfqpoint{0.020833in}{-0.005525in}}{\pgfqpoint{0.020833in}{0.000000in}}%
\pgfpathcurveto{\pgfqpoint{0.020833in}{0.005525in}}{\pgfqpoint{0.018638in}{0.010825in}}{\pgfqpoint{0.014731in}{0.014731in}}%
\pgfpathcurveto{\pgfqpoint{0.010825in}{0.018638in}}{\pgfqpoint{0.005525in}{0.020833in}}{\pgfqpoint{0.000000in}{0.020833in}}%
\pgfpathcurveto{\pgfqpoint{-0.005525in}{0.020833in}}{\pgfqpoint{-0.010825in}{0.018638in}}{\pgfqpoint{-0.014731in}{0.014731in}}%
\pgfpathcurveto{\pgfqpoint{-0.018638in}{0.010825in}}{\pgfqpoint{-0.020833in}{0.005525in}}{\pgfqpoint{-0.020833in}{0.000000in}}%
\pgfpathcurveto{\pgfqpoint{-0.020833in}{-0.005525in}}{\pgfqpoint{-0.018638in}{-0.010825in}}{\pgfqpoint{-0.014731in}{-0.014731in}}%
\pgfpathcurveto{\pgfqpoint{-0.010825in}{-0.018638in}}{\pgfqpoint{-0.005525in}{-0.020833in}}{\pgfqpoint{0.000000in}{-0.020833in}}%
\pgfpathlineto{\pgfqpoint{0.000000in}{-0.020833in}}%
\pgfpathclose%
\pgfusepath{stroke,fill}%
}%
\begin{pgfscope}%
\pgfsys@transformshift{1.067357in}{5.304167in}%
\pgfsys@useobject{currentmarker}{}%
\end{pgfscope}%
\end{pgfscope}%
\begin{pgfscope}%
\definecolor{textcolor}{rgb}{0.000000,0.000000,0.000000}%
\pgfsetstrokecolor{textcolor}%
\pgfsetfillcolor{textcolor}%
\pgftext[x=1.417357in,y=5.236111in,left,base]{\color{textcolor}{\rmfamily\fontsize{14.000000}{16.800000}\selectfont\catcode`\^=\active\def^{\ifmmode\sp\else\^{}\fi}\catcode`\%=\active\def%{\%}Monthly Data}}%
\end{pgfscope}%
\begin{pgfscope}%
\pgfsetrectcap%
\pgfsetroundjoin%
\pgfsetlinewidth{1.003750pt}%
\definecolor{currentstroke}{rgb}{0.000000,0.000000,0.000000}%
\pgfsetstrokecolor{currentstroke}%
\pgfsetdash{}{0pt}%
\pgfpathmoveto{\pgfqpoint{0.872913in}{5.029167in}}%
\pgfpathlineto{\pgfqpoint{1.067357in}{5.029167in}}%
\pgfpathlineto{\pgfqpoint{1.261802in}{5.029167in}}%
\pgfusepath{stroke}%
\end{pgfscope}%
\begin{pgfscope}%
\pgfsetbuttcap%
\pgfsetroundjoin%
\definecolor{currentfill}{rgb}{0.000000,0.000000,0.000000}%
\pgfsetfillcolor{currentfill}%
\pgfsetlinewidth{1.003750pt}%
\definecolor{currentstroke}{rgb}{0.000000,0.000000,0.000000}%
\pgfsetstrokecolor{currentstroke}%
\pgfsetdash{}{0pt}%
\pgfsys@defobject{currentmarker}{\pgfqpoint{-0.020833in}{-0.020833in}}{\pgfqpoint{0.020833in}{0.020833in}}{%
\pgfpathmoveto{\pgfqpoint{0.000000in}{-0.020833in}}%
\pgfpathcurveto{\pgfqpoint{0.005525in}{-0.020833in}}{\pgfqpoint{0.010825in}{-0.018638in}}{\pgfqpoint{0.014731in}{-0.014731in}}%
\pgfpathcurveto{\pgfqpoint{0.018638in}{-0.010825in}}{\pgfqpoint{0.020833in}{-0.005525in}}{\pgfqpoint{0.020833in}{0.000000in}}%
\pgfpathcurveto{\pgfqpoint{0.020833in}{0.005525in}}{\pgfqpoint{0.018638in}{0.010825in}}{\pgfqpoint{0.014731in}{0.014731in}}%
\pgfpathcurveto{\pgfqpoint{0.010825in}{0.018638in}}{\pgfqpoint{0.005525in}{0.020833in}}{\pgfqpoint{0.000000in}{0.020833in}}%
\pgfpathcurveto{\pgfqpoint{-0.005525in}{0.020833in}}{\pgfqpoint{-0.010825in}{0.018638in}}{\pgfqpoint{-0.014731in}{0.014731in}}%
\pgfpathcurveto{\pgfqpoint{-0.018638in}{0.010825in}}{\pgfqpoint{-0.020833in}{0.005525in}}{\pgfqpoint{-0.020833in}{0.000000in}}%
\pgfpathcurveto{\pgfqpoint{-0.020833in}{-0.005525in}}{\pgfqpoint{-0.018638in}{-0.010825in}}{\pgfqpoint{-0.014731in}{-0.014731in}}%
\pgfpathcurveto{\pgfqpoint{-0.010825in}{-0.018638in}}{\pgfqpoint{-0.005525in}{-0.020833in}}{\pgfqpoint{0.000000in}{-0.020833in}}%
\pgfpathlineto{\pgfqpoint{0.000000in}{-0.020833in}}%
\pgfpathclose%
\pgfusepath{stroke,fill}%
}%
\begin{pgfscope}%
\pgfsys@transformshift{1.067357in}{5.029167in}%
\pgfsys@useobject{currentmarker}{}%
\end{pgfscope}%
\end{pgfscope}%
\begin{pgfscope}%
\definecolor{textcolor}{rgb}{0.000000,0.000000,0.000000}%
\pgfsetstrokecolor{textcolor}%
\pgfsetfillcolor{textcolor}%
\pgftext[x=1.417357in,y=4.961112in,left,base]{\color{textcolor}{\rmfamily\fontsize{14.000000}{16.800000}\selectfont\catcode`\^=\active\def^{\ifmmode\sp\else\^{}\fi}\catcode`\%=\active\def%{\%}Smoothed}}%
\end{pgfscope}%
\end{pgfpicture}%
\makeatother%
\endgroup%
}
            % \resizebox{\columnwidth}{!}{%% Creator: Matplotlib, PGF backend
%%
%% To include the figure in your LaTeX document, write
%%   \input{<filename>.pgf}
%%
%% Make sure the required packages are loaded in your preamble
%%   \usepackage{pgf}
%%
%% Also ensure that all the required font packages are loaded; for instance,
%% the lmodern package is sometimes necessary when using math font.
%%   \usepackage{lmodern}
%%
%% Figures using additional raster images can only be included by \input if
%% they are in the same directory as the main LaTeX file. For loading figures
%% from other directories you can use the `import` package
%%   \usepackage{import}
%%
%% and then include the figures with
%%   \import{<path to file>}{<filename>.pgf}
%%
%% Matplotlib used the following preamble
%%   \def\mathdefault#1{#1}
%%   \everymath=\expandafter{\the\everymath\displaystyle}
%%   \IfFileExists{scrextend.sty}{
%%     \usepackage[fontsize=10.000000pt]{scrextend}
%%   }{
%%     \renewcommand{\normalsize}{\fontsize{10.000000}{12.000000}\selectfont}
%%     \normalsize
%%   }
%%   
%%   \makeatletter\@ifpackageloaded{underscore}{}{\usepackage[strings]{underscore}}\makeatother
%%
\begingroup%
\makeatletter%
\begin{pgfpicture}%
\pgfpathrectangle{\pgfpointorigin}{\pgfqpoint{7.880511in}{5.900000in}}%
\pgfusepath{use as bounding box, clip}%
\begin{pgfscope}%
\pgfsetbuttcap%
\pgfsetmiterjoin%
\definecolor{currentfill}{rgb}{1.000000,1.000000,1.000000}%
\pgfsetfillcolor{currentfill}%
\pgfsetlinewidth{0.000000pt}%
\definecolor{currentstroke}{rgb}{0.000000,0.000000,0.000000}%
\pgfsetstrokecolor{currentstroke}%
\pgfsetdash{}{0pt}%
\pgfpathmoveto{\pgfqpoint{0.000000in}{0.000000in}}%
\pgfpathlineto{\pgfqpoint{7.880511in}{0.000000in}}%
\pgfpathlineto{\pgfqpoint{7.880511in}{5.900000in}}%
\pgfpathlineto{\pgfqpoint{0.000000in}{5.900000in}}%
\pgfpathlineto{\pgfqpoint{0.000000in}{0.000000in}}%
\pgfpathclose%
\pgfusepath{fill}%
\end{pgfscope}%
\begin{pgfscope}%
\pgfsetbuttcap%
\pgfsetmiterjoin%
\definecolor{currentfill}{rgb}{1.000000,1.000000,1.000000}%
\pgfsetfillcolor{currentfill}%
\pgfsetlinewidth{0.000000pt}%
\definecolor{currentstroke}{rgb}{0.000000,0.000000,0.000000}%
\pgfsetstrokecolor{currentstroke}%
\pgfsetstrokeopacity{0.000000}%
\pgfsetdash{}{0pt}%
\pgfpathmoveto{\pgfqpoint{0.697913in}{0.559721in}}%
\pgfpathlineto{\pgfqpoint{7.746549in}{0.559721in}}%
\pgfpathlineto{\pgfqpoint{7.746549in}{5.550000in}}%
\pgfpathlineto{\pgfqpoint{0.697913in}{5.550000in}}%
\pgfpathlineto{\pgfqpoint{0.697913in}{0.559721in}}%
\pgfpathclose%
\pgfusepath{fill}%
\end{pgfscope}%
\begin{pgfscope}%
\pgfpathrectangle{\pgfqpoint{0.697913in}{0.559721in}}{\pgfqpoint{7.048636in}{4.990279in}}%
\pgfusepath{clip}%
\pgfsetrectcap%
\pgfsetroundjoin%
\pgfsetlinewidth{0.803000pt}%
\definecolor{currentstroke}{rgb}{0.690196,0.690196,0.690196}%
\pgfsetstrokecolor{currentstroke}%
\pgfsetdash{}{0pt}%
\pgfpathmoveto{\pgfqpoint{1.404248in}{0.559721in}}%
\pgfpathlineto{\pgfqpoint{1.404248in}{5.550000in}}%
\pgfusepath{stroke}%
\end{pgfscope}%
\begin{pgfscope}%
\pgfsetbuttcap%
\pgfsetroundjoin%
\definecolor{currentfill}{rgb}{0.000000,0.000000,0.000000}%
\pgfsetfillcolor{currentfill}%
\pgfsetlinewidth{1.254687pt}%
\definecolor{currentstroke}{rgb}{0.000000,0.000000,0.000000}%
\pgfsetstrokecolor{currentstroke}%
\pgfsetdash{}{0pt}%
\pgfsys@defobject{currentmarker}{\pgfqpoint{0.000000in}{0.000000in}}{\pgfqpoint{0.000000in}{0.111111in}}{%
\pgfpathmoveto{\pgfqpoint{0.000000in}{0.000000in}}%
\pgfpathlineto{\pgfqpoint{0.000000in}{0.111111in}}%
\pgfusepath{stroke,fill}%
}%
\begin{pgfscope}%
\pgfsys@transformshift{1.404248in}{0.559721in}%
\pgfsys@useobject{currentmarker}{}%
\end{pgfscope}%
\end{pgfscope}%
\begin{pgfscope}%
\definecolor{textcolor}{rgb}{0.000000,0.000000,0.000000}%
\pgfsetstrokecolor{textcolor}%
\pgfsetfillcolor{textcolor}%
\pgftext[x=1.404248in,y=0.511110in,,top]{\color{textcolor}{\rmfamily\fontsize{14.000000}{16.800000}\selectfont\catcode`\^=\active\def^{\ifmmode\sp\else\^{}\fi}\catcode`\%=\active\def%{\%}1989}}%
\end{pgfscope}%
\begin{pgfscope}%
\pgfpathrectangle{\pgfqpoint{0.697913in}{0.559721in}}{\pgfqpoint{7.048636in}{4.990279in}}%
\pgfusepath{clip}%
\pgfsetrectcap%
\pgfsetroundjoin%
\pgfsetlinewidth{0.803000pt}%
\definecolor{currentstroke}{rgb}{0.690196,0.690196,0.690196}%
\pgfsetstrokecolor{currentstroke}%
\pgfsetdash{}{0pt}%
\pgfpathmoveto{\pgfqpoint{2.287167in}{0.559721in}}%
\pgfpathlineto{\pgfqpoint{2.287167in}{5.550000in}}%
\pgfusepath{stroke}%
\end{pgfscope}%
\begin{pgfscope}%
\pgfsetbuttcap%
\pgfsetroundjoin%
\definecolor{currentfill}{rgb}{0.000000,0.000000,0.000000}%
\pgfsetfillcolor{currentfill}%
\pgfsetlinewidth{1.254687pt}%
\definecolor{currentstroke}{rgb}{0.000000,0.000000,0.000000}%
\pgfsetstrokecolor{currentstroke}%
\pgfsetdash{}{0pt}%
\pgfsys@defobject{currentmarker}{\pgfqpoint{0.000000in}{0.000000in}}{\pgfqpoint{0.000000in}{0.111111in}}{%
\pgfpathmoveto{\pgfqpoint{0.000000in}{0.000000in}}%
\pgfpathlineto{\pgfqpoint{0.000000in}{0.111111in}}%
\pgfusepath{stroke,fill}%
}%
\begin{pgfscope}%
\pgfsys@transformshift{2.287167in}{0.559721in}%
\pgfsys@useobject{currentmarker}{}%
\end{pgfscope}%
\end{pgfscope}%
\begin{pgfscope}%
\definecolor{textcolor}{rgb}{0.000000,0.000000,0.000000}%
\pgfsetstrokecolor{textcolor}%
\pgfsetfillcolor{textcolor}%
\pgftext[x=2.287167in,y=0.511110in,,top]{\color{textcolor}{\rmfamily\fontsize{14.000000}{16.800000}\selectfont\catcode`\^=\active\def^{\ifmmode\sp\else\^{}\fi}\catcode`\%=\active\def%{\%}1994}}%
\end{pgfscope}%
\begin{pgfscope}%
\pgfpathrectangle{\pgfqpoint{0.697913in}{0.559721in}}{\pgfqpoint{7.048636in}{4.990279in}}%
\pgfusepath{clip}%
\pgfsetrectcap%
\pgfsetroundjoin%
\pgfsetlinewidth{0.803000pt}%
\definecolor{currentstroke}{rgb}{0.690196,0.690196,0.690196}%
\pgfsetstrokecolor{currentstroke}%
\pgfsetdash{}{0pt}%
\pgfpathmoveto{\pgfqpoint{3.170086in}{0.559721in}}%
\pgfpathlineto{\pgfqpoint{3.170086in}{5.550000in}}%
\pgfusepath{stroke}%
\end{pgfscope}%
\begin{pgfscope}%
\pgfsetbuttcap%
\pgfsetroundjoin%
\definecolor{currentfill}{rgb}{0.000000,0.000000,0.000000}%
\pgfsetfillcolor{currentfill}%
\pgfsetlinewidth{1.254687pt}%
\definecolor{currentstroke}{rgb}{0.000000,0.000000,0.000000}%
\pgfsetstrokecolor{currentstroke}%
\pgfsetdash{}{0pt}%
\pgfsys@defobject{currentmarker}{\pgfqpoint{0.000000in}{0.000000in}}{\pgfqpoint{0.000000in}{0.111111in}}{%
\pgfpathmoveto{\pgfqpoint{0.000000in}{0.000000in}}%
\pgfpathlineto{\pgfqpoint{0.000000in}{0.111111in}}%
\pgfusepath{stroke,fill}%
}%
\begin{pgfscope}%
\pgfsys@transformshift{3.170086in}{0.559721in}%
\pgfsys@useobject{currentmarker}{}%
\end{pgfscope}%
\end{pgfscope}%
\begin{pgfscope}%
\definecolor{textcolor}{rgb}{0.000000,0.000000,0.000000}%
\pgfsetstrokecolor{textcolor}%
\pgfsetfillcolor{textcolor}%
\pgftext[x=3.170086in,y=0.511110in,,top]{\color{textcolor}{\rmfamily\fontsize{14.000000}{16.800000}\selectfont\catcode`\^=\active\def^{\ifmmode\sp\else\^{}\fi}\catcode`\%=\active\def%{\%}1999}}%
\end{pgfscope}%
\begin{pgfscope}%
\pgfpathrectangle{\pgfqpoint{0.697913in}{0.559721in}}{\pgfqpoint{7.048636in}{4.990279in}}%
\pgfusepath{clip}%
\pgfsetrectcap%
\pgfsetroundjoin%
\pgfsetlinewidth{0.803000pt}%
\definecolor{currentstroke}{rgb}{0.690196,0.690196,0.690196}%
\pgfsetstrokecolor{currentstroke}%
\pgfsetdash{}{0pt}%
\pgfpathmoveto{\pgfqpoint{4.053005in}{0.559721in}}%
\pgfpathlineto{\pgfqpoint{4.053005in}{5.550000in}}%
\pgfusepath{stroke}%
\end{pgfscope}%
\begin{pgfscope}%
\pgfsetbuttcap%
\pgfsetroundjoin%
\definecolor{currentfill}{rgb}{0.000000,0.000000,0.000000}%
\pgfsetfillcolor{currentfill}%
\pgfsetlinewidth{1.254687pt}%
\definecolor{currentstroke}{rgb}{0.000000,0.000000,0.000000}%
\pgfsetstrokecolor{currentstroke}%
\pgfsetdash{}{0pt}%
\pgfsys@defobject{currentmarker}{\pgfqpoint{0.000000in}{0.000000in}}{\pgfqpoint{0.000000in}{0.111111in}}{%
\pgfpathmoveto{\pgfqpoint{0.000000in}{0.000000in}}%
\pgfpathlineto{\pgfqpoint{0.000000in}{0.111111in}}%
\pgfusepath{stroke,fill}%
}%
\begin{pgfscope}%
\pgfsys@transformshift{4.053005in}{0.559721in}%
\pgfsys@useobject{currentmarker}{}%
\end{pgfscope}%
\end{pgfscope}%
\begin{pgfscope}%
\definecolor{textcolor}{rgb}{0.000000,0.000000,0.000000}%
\pgfsetstrokecolor{textcolor}%
\pgfsetfillcolor{textcolor}%
\pgftext[x=4.053005in,y=0.511110in,,top]{\color{textcolor}{\rmfamily\fontsize{14.000000}{16.800000}\selectfont\catcode`\^=\active\def^{\ifmmode\sp\else\^{}\fi}\catcode`\%=\active\def%{\%}2004}}%
\end{pgfscope}%
\begin{pgfscope}%
\pgfpathrectangle{\pgfqpoint{0.697913in}{0.559721in}}{\pgfqpoint{7.048636in}{4.990279in}}%
\pgfusepath{clip}%
\pgfsetrectcap%
\pgfsetroundjoin%
\pgfsetlinewidth{0.803000pt}%
\definecolor{currentstroke}{rgb}{0.690196,0.690196,0.690196}%
\pgfsetstrokecolor{currentstroke}%
\pgfsetdash{}{0pt}%
\pgfpathmoveto{\pgfqpoint{4.935924in}{0.559721in}}%
\pgfpathlineto{\pgfqpoint{4.935924in}{5.550000in}}%
\pgfusepath{stroke}%
\end{pgfscope}%
\begin{pgfscope}%
\pgfsetbuttcap%
\pgfsetroundjoin%
\definecolor{currentfill}{rgb}{0.000000,0.000000,0.000000}%
\pgfsetfillcolor{currentfill}%
\pgfsetlinewidth{1.254687pt}%
\definecolor{currentstroke}{rgb}{0.000000,0.000000,0.000000}%
\pgfsetstrokecolor{currentstroke}%
\pgfsetdash{}{0pt}%
\pgfsys@defobject{currentmarker}{\pgfqpoint{0.000000in}{0.000000in}}{\pgfqpoint{0.000000in}{0.111111in}}{%
\pgfpathmoveto{\pgfqpoint{0.000000in}{0.000000in}}%
\pgfpathlineto{\pgfqpoint{0.000000in}{0.111111in}}%
\pgfusepath{stroke,fill}%
}%
\begin{pgfscope}%
\pgfsys@transformshift{4.935924in}{0.559721in}%
\pgfsys@useobject{currentmarker}{}%
\end{pgfscope}%
\end{pgfscope}%
\begin{pgfscope}%
\definecolor{textcolor}{rgb}{0.000000,0.000000,0.000000}%
\pgfsetstrokecolor{textcolor}%
\pgfsetfillcolor{textcolor}%
\pgftext[x=4.935924in,y=0.511110in,,top]{\color{textcolor}{\rmfamily\fontsize{14.000000}{16.800000}\selectfont\catcode`\^=\active\def^{\ifmmode\sp\else\^{}\fi}\catcode`\%=\active\def%{\%}2009}}%
\end{pgfscope}%
\begin{pgfscope}%
\pgfpathrectangle{\pgfqpoint{0.697913in}{0.559721in}}{\pgfqpoint{7.048636in}{4.990279in}}%
\pgfusepath{clip}%
\pgfsetrectcap%
\pgfsetroundjoin%
\pgfsetlinewidth{0.803000pt}%
\definecolor{currentstroke}{rgb}{0.690196,0.690196,0.690196}%
\pgfsetstrokecolor{currentstroke}%
\pgfsetdash{}{0pt}%
\pgfpathmoveto{\pgfqpoint{5.818843in}{0.559721in}}%
\pgfpathlineto{\pgfqpoint{5.818843in}{5.550000in}}%
\pgfusepath{stroke}%
\end{pgfscope}%
\begin{pgfscope}%
\pgfsetbuttcap%
\pgfsetroundjoin%
\definecolor{currentfill}{rgb}{0.000000,0.000000,0.000000}%
\pgfsetfillcolor{currentfill}%
\pgfsetlinewidth{1.254687pt}%
\definecolor{currentstroke}{rgb}{0.000000,0.000000,0.000000}%
\pgfsetstrokecolor{currentstroke}%
\pgfsetdash{}{0pt}%
\pgfsys@defobject{currentmarker}{\pgfqpoint{0.000000in}{0.000000in}}{\pgfqpoint{0.000000in}{0.111111in}}{%
\pgfpathmoveto{\pgfqpoint{0.000000in}{0.000000in}}%
\pgfpathlineto{\pgfqpoint{0.000000in}{0.111111in}}%
\pgfusepath{stroke,fill}%
}%
\begin{pgfscope}%
\pgfsys@transformshift{5.818843in}{0.559721in}%
\pgfsys@useobject{currentmarker}{}%
\end{pgfscope}%
\end{pgfscope}%
\begin{pgfscope}%
\definecolor{textcolor}{rgb}{0.000000,0.000000,0.000000}%
\pgfsetstrokecolor{textcolor}%
\pgfsetfillcolor{textcolor}%
\pgftext[x=5.818843in,y=0.511110in,,top]{\color{textcolor}{\rmfamily\fontsize{14.000000}{16.800000}\selectfont\catcode`\^=\active\def^{\ifmmode\sp\else\^{}\fi}\catcode`\%=\active\def%{\%}2014}}%
\end{pgfscope}%
\begin{pgfscope}%
\pgfpathrectangle{\pgfqpoint{0.697913in}{0.559721in}}{\pgfqpoint{7.048636in}{4.990279in}}%
\pgfusepath{clip}%
\pgfsetrectcap%
\pgfsetroundjoin%
\pgfsetlinewidth{0.803000pt}%
\definecolor{currentstroke}{rgb}{0.690196,0.690196,0.690196}%
\pgfsetstrokecolor{currentstroke}%
\pgfsetdash{}{0pt}%
\pgfpathmoveto{\pgfqpoint{6.701762in}{0.559721in}}%
\pgfpathlineto{\pgfqpoint{6.701762in}{5.550000in}}%
\pgfusepath{stroke}%
\end{pgfscope}%
\begin{pgfscope}%
\pgfsetbuttcap%
\pgfsetroundjoin%
\definecolor{currentfill}{rgb}{0.000000,0.000000,0.000000}%
\pgfsetfillcolor{currentfill}%
\pgfsetlinewidth{1.254687pt}%
\definecolor{currentstroke}{rgb}{0.000000,0.000000,0.000000}%
\pgfsetstrokecolor{currentstroke}%
\pgfsetdash{}{0pt}%
\pgfsys@defobject{currentmarker}{\pgfqpoint{0.000000in}{0.000000in}}{\pgfqpoint{0.000000in}{0.111111in}}{%
\pgfpathmoveto{\pgfqpoint{0.000000in}{0.000000in}}%
\pgfpathlineto{\pgfqpoint{0.000000in}{0.111111in}}%
\pgfusepath{stroke,fill}%
}%
\begin{pgfscope}%
\pgfsys@transformshift{6.701762in}{0.559721in}%
\pgfsys@useobject{currentmarker}{}%
\end{pgfscope}%
\end{pgfscope}%
\begin{pgfscope}%
\definecolor{textcolor}{rgb}{0.000000,0.000000,0.000000}%
\pgfsetstrokecolor{textcolor}%
\pgfsetfillcolor{textcolor}%
\pgftext[x=6.701762in,y=0.511110in,,top]{\color{textcolor}{\rmfamily\fontsize{14.000000}{16.800000}\selectfont\catcode`\^=\active\def^{\ifmmode\sp\else\^{}\fi}\catcode`\%=\active\def%{\%}2019}}%
\end{pgfscope}%
\begin{pgfscope}%
\pgfpathrectangle{\pgfqpoint{0.697913in}{0.559721in}}{\pgfqpoint{7.048636in}{4.990279in}}%
\pgfusepath{clip}%
\pgfsetrectcap%
\pgfsetroundjoin%
\pgfsetlinewidth{0.803000pt}%
\definecolor{currentstroke}{rgb}{0.690196,0.690196,0.690196}%
\pgfsetstrokecolor{currentstroke}%
\pgfsetdash{}{0pt}%
\pgfpathmoveto{\pgfqpoint{7.584681in}{0.559721in}}%
\pgfpathlineto{\pgfqpoint{7.584681in}{5.550000in}}%
\pgfusepath{stroke}%
\end{pgfscope}%
\begin{pgfscope}%
\pgfsetbuttcap%
\pgfsetroundjoin%
\definecolor{currentfill}{rgb}{0.000000,0.000000,0.000000}%
\pgfsetfillcolor{currentfill}%
\pgfsetlinewidth{1.254687pt}%
\definecolor{currentstroke}{rgb}{0.000000,0.000000,0.000000}%
\pgfsetstrokecolor{currentstroke}%
\pgfsetdash{}{0pt}%
\pgfsys@defobject{currentmarker}{\pgfqpoint{0.000000in}{0.000000in}}{\pgfqpoint{0.000000in}{0.111111in}}{%
\pgfpathmoveto{\pgfqpoint{0.000000in}{0.000000in}}%
\pgfpathlineto{\pgfqpoint{0.000000in}{0.111111in}}%
\pgfusepath{stroke,fill}%
}%
\begin{pgfscope}%
\pgfsys@transformshift{7.584681in}{0.559721in}%
\pgfsys@useobject{currentmarker}{}%
\end{pgfscope}%
\end{pgfscope}%
\begin{pgfscope}%
\definecolor{textcolor}{rgb}{0.000000,0.000000,0.000000}%
\pgfsetstrokecolor{textcolor}%
\pgfsetfillcolor{textcolor}%
\pgftext[x=7.584681in,y=0.511110in,,top]{\color{textcolor}{\rmfamily\fontsize{14.000000}{16.800000}\selectfont\catcode`\^=\active\def^{\ifmmode\sp\else\^{}\fi}\catcode`\%=\active\def%{\%}2024}}%
\end{pgfscope}%
\begin{pgfscope}%
\pgfsetbuttcap%
\pgfsetroundjoin%
\definecolor{currentfill}{rgb}{0.000000,0.000000,0.000000}%
\pgfsetfillcolor{currentfill}%
\pgfsetlinewidth{1.254687pt}%
\definecolor{currentstroke}{rgb}{0.000000,0.000000,0.000000}%
\pgfsetstrokecolor{currentstroke}%
\pgfsetdash{}{0pt}%
\pgfsys@defobject{currentmarker}{\pgfqpoint{0.000000in}{0.000000in}}{\pgfqpoint{0.000000in}{0.055556in}}{%
\pgfpathmoveto{\pgfqpoint{0.000000in}{0.000000in}}%
\pgfpathlineto{\pgfqpoint{0.000000in}{0.055556in}}%
\pgfusepath{stroke,fill}%
}%
\begin{pgfscope}%
\pgfsys@transformshift{0.742059in}{0.559721in}%
\pgfsys@useobject{currentmarker}{}%
\end{pgfscope}%
\end{pgfscope}%
\begin{pgfscope}%
\pgfsetbuttcap%
\pgfsetroundjoin%
\definecolor{currentfill}{rgb}{0.000000,0.000000,0.000000}%
\pgfsetfillcolor{currentfill}%
\pgfsetlinewidth{1.254687pt}%
\definecolor{currentstroke}{rgb}{0.000000,0.000000,0.000000}%
\pgfsetstrokecolor{currentstroke}%
\pgfsetdash{}{0pt}%
\pgfsys@defobject{currentmarker}{\pgfqpoint{0.000000in}{0.000000in}}{\pgfqpoint{0.000000in}{0.055556in}}{%
\pgfpathmoveto{\pgfqpoint{0.000000in}{0.000000in}}%
\pgfpathlineto{\pgfqpoint{0.000000in}{0.055556in}}%
\pgfusepath{stroke,fill}%
}%
\begin{pgfscope}%
\pgfsys@transformshift{0.962788in}{0.559721in}%
\pgfsys@useobject{currentmarker}{}%
\end{pgfscope}%
\end{pgfscope}%
\begin{pgfscope}%
\pgfsetbuttcap%
\pgfsetroundjoin%
\definecolor{currentfill}{rgb}{0.000000,0.000000,0.000000}%
\pgfsetfillcolor{currentfill}%
\pgfsetlinewidth{1.254687pt}%
\definecolor{currentstroke}{rgb}{0.000000,0.000000,0.000000}%
\pgfsetstrokecolor{currentstroke}%
\pgfsetdash{}{0pt}%
\pgfsys@defobject{currentmarker}{\pgfqpoint{0.000000in}{0.000000in}}{\pgfqpoint{0.000000in}{0.055556in}}{%
\pgfpathmoveto{\pgfqpoint{0.000000in}{0.000000in}}%
\pgfpathlineto{\pgfqpoint{0.000000in}{0.055556in}}%
\pgfusepath{stroke,fill}%
}%
\begin{pgfscope}%
\pgfsys@transformshift{1.183518in}{0.559721in}%
\pgfsys@useobject{currentmarker}{}%
\end{pgfscope}%
\end{pgfscope}%
\begin{pgfscope}%
\pgfsetbuttcap%
\pgfsetroundjoin%
\definecolor{currentfill}{rgb}{0.000000,0.000000,0.000000}%
\pgfsetfillcolor{currentfill}%
\pgfsetlinewidth{1.254687pt}%
\definecolor{currentstroke}{rgb}{0.000000,0.000000,0.000000}%
\pgfsetstrokecolor{currentstroke}%
\pgfsetdash{}{0pt}%
\pgfsys@defobject{currentmarker}{\pgfqpoint{0.000000in}{0.000000in}}{\pgfqpoint{0.000000in}{0.055556in}}{%
\pgfpathmoveto{\pgfqpoint{0.000000in}{0.000000in}}%
\pgfpathlineto{\pgfqpoint{0.000000in}{0.055556in}}%
\pgfusepath{stroke,fill}%
}%
\begin{pgfscope}%
\pgfsys@transformshift{1.624978in}{0.559721in}%
\pgfsys@useobject{currentmarker}{}%
\end{pgfscope}%
\end{pgfscope}%
\begin{pgfscope}%
\pgfsetbuttcap%
\pgfsetroundjoin%
\definecolor{currentfill}{rgb}{0.000000,0.000000,0.000000}%
\pgfsetfillcolor{currentfill}%
\pgfsetlinewidth{1.254687pt}%
\definecolor{currentstroke}{rgb}{0.000000,0.000000,0.000000}%
\pgfsetstrokecolor{currentstroke}%
\pgfsetdash{}{0pt}%
\pgfsys@defobject{currentmarker}{\pgfqpoint{0.000000in}{0.000000in}}{\pgfqpoint{0.000000in}{0.055556in}}{%
\pgfpathmoveto{\pgfqpoint{0.000000in}{0.000000in}}%
\pgfpathlineto{\pgfqpoint{0.000000in}{0.055556in}}%
\pgfusepath{stroke,fill}%
}%
\begin{pgfscope}%
\pgfsys@transformshift{1.845707in}{0.559721in}%
\pgfsys@useobject{currentmarker}{}%
\end{pgfscope}%
\end{pgfscope}%
\begin{pgfscope}%
\pgfsetbuttcap%
\pgfsetroundjoin%
\definecolor{currentfill}{rgb}{0.000000,0.000000,0.000000}%
\pgfsetfillcolor{currentfill}%
\pgfsetlinewidth{1.254687pt}%
\definecolor{currentstroke}{rgb}{0.000000,0.000000,0.000000}%
\pgfsetstrokecolor{currentstroke}%
\pgfsetdash{}{0pt}%
\pgfsys@defobject{currentmarker}{\pgfqpoint{0.000000in}{0.000000in}}{\pgfqpoint{0.000000in}{0.055556in}}{%
\pgfpathmoveto{\pgfqpoint{0.000000in}{0.000000in}}%
\pgfpathlineto{\pgfqpoint{0.000000in}{0.055556in}}%
\pgfusepath{stroke,fill}%
}%
\begin{pgfscope}%
\pgfsys@transformshift{2.066437in}{0.559721in}%
\pgfsys@useobject{currentmarker}{}%
\end{pgfscope}%
\end{pgfscope}%
\begin{pgfscope}%
\pgfsetbuttcap%
\pgfsetroundjoin%
\definecolor{currentfill}{rgb}{0.000000,0.000000,0.000000}%
\pgfsetfillcolor{currentfill}%
\pgfsetlinewidth{1.254687pt}%
\definecolor{currentstroke}{rgb}{0.000000,0.000000,0.000000}%
\pgfsetstrokecolor{currentstroke}%
\pgfsetdash{}{0pt}%
\pgfsys@defobject{currentmarker}{\pgfqpoint{0.000000in}{0.000000in}}{\pgfqpoint{0.000000in}{0.055556in}}{%
\pgfpathmoveto{\pgfqpoint{0.000000in}{0.000000in}}%
\pgfpathlineto{\pgfqpoint{0.000000in}{0.055556in}}%
\pgfusepath{stroke,fill}%
}%
\begin{pgfscope}%
\pgfsys@transformshift{2.507897in}{0.559721in}%
\pgfsys@useobject{currentmarker}{}%
\end{pgfscope}%
\end{pgfscope}%
\begin{pgfscope}%
\pgfsetbuttcap%
\pgfsetroundjoin%
\definecolor{currentfill}{rgb}{0.000000,0.000000,0.000000}%
\pgfsetfillcolor{currentfill}%
\pgfsetlinewidth{1.254687pt}%
\definecolor{currentstroke}{rgb}{0.000000,0.000000,0.000000}%
\pgfsetstrokecolor{currentstroke}%
\pgfsetdash{}{0pt}%
\pgfsys@defobject{currentmarker}{\pgfqpoint{0.000000in}{0.000000in}}{\pgfqpoint{0.000000in}{0.055556in}}{%
\pgfpathmoveto{\pgfqpoint{0.000000in}{0.000000in}}%
\pgfpathlineto{\pgfqpoint{0.000000in}{0.055556in}}%
\pgfusepath{stroke,fill}%
}%
\begin{pgfscope}%
\pgfsys@transformshift{2.728626in}{0.559721in}%
\pgfsys@useobject{currentmarker}{}%
\end{pgfscope}%
\end{pgfscope}%
\begin{pgfscope}%
\pgfsetbuttcap%
\pgfsetroundjoin%
\definecolor{currentfill}{rgb}{0.000000,0.000000,0.000000}%
\pgfsetfillcolor{currentfill}%
\pgfsetlinewidth{1.254687pt}%
\definecolor{currentstroke}{rgb}{0.000000,0.000000,0.000000}%
\pgfsetstrokecolor{currentstroke}%
\pgfsetdash{}{0pt}%
\pgfsys@defobject{currentmarker}{\pgfqpoint{0.000000in}{0.000000in}}{\pgfqpoint{0.000000in}{0.055556in}}{%
\pgfpathmoveto{\pgfqpoint{0.000000in}{0.000000in}}%
\pgfpathlineto{\pgfqpoint{0.000000in}{0.055556in}}%
\pgfusepath{stroke,fill}%
}%
\begin{pgfscope}%
\pgfsys@transformshift{2.949356in}{0.559721in}%
\pgfsys@useobject{currentmarker}{}%
\end{pgfscope}%
\end{pgfscope}%
\begin{pgfscope}%
\pgfsetbuttcap%
\pgfsetroundjoin%
\definecolor{currentfill}{rgb}{0.000000,0.000000,0.000000}%
\pgfsetfillcolor{currentfill}%
\pgfsetlinewidth{1.254687pt}%
\definecolor{currentstroke}{rgb}{0.000000,0.000000,0.000000}%
\pgfsetstrokecolor{currentstroke}%
\pgfsetdash{}{0pt}%
\pgfsys@defobject{currentmarker}{\pgfqpoint{0.000000in}{0.000000in}}{\pgfqpoint{0.000000in}{0.055556in}}{%
\pgfpathmoveto{\pgfqpoint{0.000000in}{0.000000in}}%
\pgfpathlineto{\pgfqpoint{0.000000in}{0.055556in}}%
\pgfusepath{stroke,fill}%
}%
\begin{pgfscope}%
\pgfsys@transformshift{3.390815in}{0.559721in}%
\pgfsys@useobject{currentmarker}{}%
\end{pgfscope}%
\end{pgfscope}%
\begin{pgfscope}%
\pgfsetbuttcap%
\pgfsetroundjoin%
\definecolor{currentfill}{rgb}{0.000000,0.000000,0.000000}%
\pgfsetfillcolor{currentfill}%
\pgfsetlinewidth{1.254687pt}%
\definecolor{currentstroke}{rgb}{0.000000,0.000000,0.000000}%
\pgfsetstrokecolor{currentstroke}%
\pgfsetdash{}{0pt}%
\pgfsys@defobject{currentmarker}{\pgfqpoint{0.000000in}{0.000000in}}{\pgfqpoint{0.000000in}{0.055556in}}{%
\pgfpathmoveto{\pgfqpoint{0.000000in}{0.000000in}}%
\pgfpathlineto{\pgfqpoint{0.000000in}{0.055556in}}%
\pgfusepath{stroke,fill}%
}%
\begin{pgfscope}%
\pgfsys@transformshift{3.611545in}{0.559721in}%
\pgfsys@useobject{currentmarker}{}%
\end{pgfscope}%
\end{pgfscope}%
\begin{pgfscope}%
\pgfsetbuttcap%
\pgfsetroundjoin%
\definecolor{currentfill}{rgb}{0.000000,0.000000,0.000000}%
\pgfsetfillcolor{currentfill}%
\pgfsetlinewidth{1.254687pt}%
\definecolor{currentstroke}{rgb}{0.000000,0.000000,0.000000}%
\pgfsetstrokecolor{currentstroke}%
\pgfsetdash{}{0pt}%
\pgfsys@defobject{currentmarker}{\pgfqpoint{0.000000in}{0.000000in}}{\pgfqpoint{0.000000in}{0.055556in}}{%
\pgfpathmoveto{\pgfqpoint{0.000000in}{0.000000in}}%
\pgfpathlineto{\pgfqpoint{0.000000in}{0.055556in}}%
\pgfusepath{stroke,fill}%
}%
\begin{pgfscope}%
\pgfsys@transformshift{3.832275in}{0.559721in}%
\pgfsys@useobject{currentmarker}{}%
\end{pgfscope}%
\end{pgfscope}%
\begin{pgfscope}%
\pgfsetbuttcap%
\pgfsetroundjoin%
\definecolor{currentfill}{rgb}{0.000000,0.000000,0.000000}%
\pgfsetfillcolor{currentfill}%
\pgfsetlinewidth{1.254687pt}%
\definecolor{currentstroke}{rgb}{0.000000,0.000000,0.000000}%
\pgfsetstrokecolor{currentstroke}%
\pgfsetdash{}{0pt}%
\pgfsys@defobject{currentmarker}{\pgfqpoint{0.000000in}{0.000000in}}{\pgfqpoint{0.000000in}{0.055556in}}{%
\pgfpathmoveto{\pgfqpoint{0.000000in}{0.000000in}}%
\pgfpathlineto{\pgfqpoint{0.000000in}{0.055556in}}%
\pgfusepath{stroke,fill}%
}%
\begin{pgfscope}%
\pgfsys@transformshift{4.273734in}{0.559721in}%
\pgfsys@useobject{currentmarker}{}%
\end{pgfscope}%
\end{pgfscope}%
\begin{pgfscope}%
\pgfsetbuttcap%
\pgfsetroundjoin%
\definecolor{currentfill}{rgb}{0.000000,0.000000,0.000000}%
\pgfsetfillcolor{currentfill}%
\pgfsetlinewidth{1.254687pt}%
\definecolor{currentstroke}{rgb}{0.000000,0.000000,0.000000}%
\pgfsetstrokecolor{currentstroke}%
\pgfsetdash{}{0pt}%
\pgfsys@defobject{currentmarker}{\pgfqpoint{0.000000in}{0.000000in}}{\pgfqpoint{0.000000in}{0.055556in}}{%
\pgfpathmoveto{\pgfqpoint{0.000000in}{0.000000in}}%
\pgfpathlineto{\pgfqpoint{0.000000in}{0.055556in}}%
\pgfusepath{stroke,fill}%
}%
\begin{pgfscope}%
\pgfsys@transformshift{4.494464in}{0.559721in}%
\pgfsys@useobject{currentmarker}{}%
\end{pgfscope}%
\end{pgfscope}%
\begin{pgfscope}%
\pgfsetbuttcap%
\pgfsetroundjoin%
\definecolor{currentfill}{rgb}{0.000000,0.000000,0.000000}%
\pgfsetfillcolor{currentfill}%
\pgfsetlinewidth{1.254687pt}%
\definecolor{currentstroke}{rgb}{0.000000,0.000000,0.000000}%
\pgfsetstrokecolor{currentstroke}%
\pgfsetdash{}{0pt}%
\pgfsys@defobject{currentmarker}{\pgfqpoint{0.000000in}{0.000000in}}{\pgfqpoint{0.000000in}{0.055556in}}{%
\pgfpathmoveto{\pgfqpoint{0.000000in}{0.000000in}}%
\pgfpathlineto{\pgfqpoint{0.000000in}{0.055556in}}%
\pgfusepath{stroke,fill}%
}%
\begin{pgfscope}%
\pgfsys@transformshift{4.715194in}{0.559721in}%
\pgfsys@useobject{currentmarker}{}%
\end{pgfscope}%
\end{pgfscope}%
\begin{pgfscope}%
\pgfsetbuttcap%
\pgfsetroundjoin%
\definecolor{currentfill}{rgb}{0.000000,0.000000,0.000000}%
\pgfsetfillcolor{currentfill}%
\pgfsetlinewidth{1.254687pt}%
\definecolor{currentstroke}{rgb}{0.000000,0.000000,0.000000}%
\pgfsetstrokecolor{currentstroke}%
\pgfsetdash{}{0pt}%
\pgfsys@defobject{currentmarker}{\pgfqpoint{0.000000in}{0.000000in}}{\pgfqpoint{0.000000in}{0.055556in}}{%
\pgfpathmoveto{\pgfqpoint{0.000000in}{0.000000in}}%
\pgfpathlineto{\pgfqpoint{0.000000in}{0.055556in}}%
\pgfusepath{stroke,fill}%
}%
\begin{pgfscope}%
\pgfsys@transformshift{5.156653in}{0.559721in}%
\pgfsys@useobject{currentmarker}{}%
\end{pgfscope}%
\end{pgfscope}%
\begin{pgfscope}%
\pgfsetbuttcap%
\pgfsetroundjoin%
\definecolor{currentfill}{rgb}{0.000000,0.000000,0.000000}%
\pgfsetfillcolor{currentfill}%
\pgfsetlinewidth{1.254687pt}%
\definecolor{currentstroke}{rgb}{0.000000,0.000000,0.000000}%
\pgfsetstrokecolor{currentstroke}%
\pgfsetdash{}{0pt}%
\pgfsys@defobject{currentmarker}{\pgfqpoint{0.000000in}{0.000000in}}{\pgfqpoint{0.000000in}{0.055556in}}{%
\pgfpathmoveto{\pgfqpoint{0.000000in}{0.000000in}}%
\pgfpathlineto{\pgfqpoint{0.000000in}{0.055556in}}%
\pgfusepath{stroke,fill}%
}%
\begin{pgfscope}%
\pgfsys@transformshift{5.377383in}{0.559721in}%
\pgfsys@useobject{currentmarker}{}%
\end{pgfscope}%
\end{pgfscope}%
\begin{pgfscope}%
\pgfsetbuttcap%
\pgfsetroundjoin%
\definecolor{currentfill}{rgb}{0.000000,0.000000,0.000000}%
\pgfsetfillcolor{currentfill}%
\pgfsetlinewidth{1.254687pt}%
\definecolor{currentstroke}{rgb}{0.000000,0.000000,0.000000}%
\pgfsetstrokecolor{currentstroke}%
\pgfsetdash{}{0pt}%
\pgfsys@defobject{currentmarker}{\pgfqpoint{0.000000in}{0.000000in}}{\pgfqpoint{0.000000in}{0.055556in}}{%
\pgfpathmoveto{\pgfqpoint{0.000000in}{0.000000in}}%
\pgfpathlineto{\pgfqpoint{0.000000in}{0.055556in}}%
\pgfusepath{stroke,fill}%
}%
\begin{pgfscope}%
\pgfsys@transformshift{5.598113in}{0.559721in}%
\pgfsys@useobject{currentmarker}{}%
\end{pgfscope}%
\end{pgfscope}%
\begin{pgfscope}%
\pgfsetbuttcap%
\pgfsetroundjoin%
\definecolor{currentfill}{rgb}{0.000000,0.000000,0.000000}%
\pgfsetfillcolor{currentfill}%
\pgfsetlinewidth{1.254687pt}%
\definecolor{currentstroke}{rgb}{0.000000,0.000000,0.000000}%
\pgfsetstrokecolor{currentstroke}%
\pgfsetdash{}{0pt}%
\pgfsys@defobject{currentmarker}{\pgfqpoint{0.000000in}{0.000000in}}{\pgfqpoint{0.000000in}{0.055556in}}{%
\pgfpathmoveto{\pgfqpoint{0.000000in}{0.000000in}}%
\pgfpathlineto{\pgfqpoint{0.000000in}{0.055556in}}%
\pgfusepath{stroke,fill}%
}%
\begin{pgfscope}%
\pgfsys@transformshift{6.039572in}{0.559721in}%
\pgfsys@useobject{currentmarker}{}%
\end{pgfscope}%
\end{pgfscope}%
\begin{pgfscope}%
\pgfsetbuttcap%
\pgfsetroundjoin%
\definecolor{currentfill}{rgb}{0.000000,0.000000,0.000000}%
\pgfsetfillcolor{currentfill}%
\pgfsetlinewidth{1.254687pt}%
\definecolor{currentstroke}{rgb}{0.000000,0.000000,0.000000}%
\pgfsetstrokecolor{currentstroke}%
\pgfsetdash{}{0pt}%
\pgfsys@defobject{currentmarker}{\pgfqpoint{0.000000in}{0.000000in}}{\pgfqpoint{0.000000in}{0.055556in}}{%
\pgfpathmoveto{\pgfqpoint{0.000000in}{0.000000in}}%
\pgfpathlineto{\pgfqpoint{0.000000in}{0.055556in}}%
\pgfusepath{stroke,fill}%
}%
\begin{pgfscope}%
\pgfsys@transformshift{6.260302in}{0.559721in}%
\pgfsys@useobject{currentmarker}{}%
\end{pgfscope}%
\end{pgfscope}%
\begin{pgfscope}%
\pgfsetbuttcap%
\pgfsetroundjoin%
\definecolor{currentfill}{rgb}{0.000000,0.000000,0.000000}%
\pgfsetfillcolor{currentfill}%
\pgfsetlinewidth{1.254687pt}%
\definecolor{currentstroke}{rgb}{0.000000,0.000000,0.000000}%
\pgfsetstrokecolor{currentstroke}%
\pgfsetdash{}{0pt}%
\pgfsys@defobject{currentmarker}{\pgfqpoint{0.000000in}{0.000000in}}{\pgfqpoint{0.000000in}{0.055556in}}{%
\pgfpathmoveto{\pgfqpoint{0.000000in}{0.000000in}}%
\pgfpathlineto{\pgfqpoint{0.000000in}{0.055556in}}%
\pgfusepath{stroke,fill}%
}%
\begin{pgfscope}%
\pgfsys@transformshift{6.481032in}{0.559721in}%
\pgfsys@useobject{currentmarker}{}%
\end{pgfscope}%
\end{pgfscope}%
\begin{pgfscope}%
\pgfsetbuttcap%
\pgfsetroundjoin%
\definecolor{currentfill}{rgb}{0.000000,0.000000,0.000000}%
\pgfsetfillcolor{currentfill}%
\pgfsetlinewidth{1.254687pt}%
\definecolor{currentstroke}{rgb}{0.000000,0.000000,0.000000}%
\pgfsetstrokecolor{currentstroke}%
\pgfsetdash{}{0pt}%
\pgfsys@defobject{currentmarker}{\pgfqpoint{0.000000in}{0.000000in}}{\pgfqpoint{0.000000in}{0.055556in}}{%
\pgfpathmoveto{\pgfqpoint{0.000000in}{0.000000in}}%
\pgfpathlineto{\pgfqpoint{0.000000in}{0.055556in}}%
\pgfusepath{stroke,fill}%
}%
\begin{pgfscope}%
\pgfsys@transformshift{6.922491in}{0.559721in}%
\pgfsys@useobject{currentmarker}{}%
\end{pgfscope}%
\end{pgfscope}%
\begin{pgfscope}%
\pgfsetbuttcap%
\pgfsetroundjoin%
\definecolor{currentfill}{rgb}{0.000000,0.000000,0.000000}%
\pgfsetfillcolor{currentfill}%
\pgfsetlinewidth{1.254687pt}%
\definecolor{currentstroke}{rgb}{0.000000,0.000000,0.000000}%
\pgfsetstrokecolor{currentstroke}%
\pgfsetdash{}{0pt}%
\pgfsys@defobject{currentmarker}{\pgfqpoint{0.000000in}{0.000000in}}{\pgfqpoint{0.000000in}{0.055556in}}{%
\pgfpathmoveto{\pgfqpoint{0.000000in}{0.000000in}}%
\pgfpathlineto{\pgfqpoint{0.000000in}{0.055556in}}%
\pgfusepath{stroke,fill}%
}%
\begin{pgfscope}%
\pgfsys@transformshift{7.143221in}{0.559721in}%
\pgfsys@useobject{currentmarker}{}%
\end{pgfscope}%
\end{pgfscope}%
\begin{pgfscope}%
\pgfsetbuttcap%
\pgfsetroundjoin%
\definecolor{currentfill}{rgb}{0.000000,0.000000,0.000000}%
\pgfsetfillcolor{currentfill}%
\pgfsetlinewidth{1.254687pt}%
\definecolor{currentstroke}{rgb}{0.000000,0.000000,0.000000}%
\pgfsetstrokecolor{currentstroke}%
\pgfsetdash{}{0pt}%
\pgfsys@defobject{currentmarker}{\pgfqpoint{0.000000in}{0.000000in}}{\pgfqpoint{0.000000in}{0.055556in}}{%
\pgfpathmoveto{\pgfqpoint{0.000000in}{0.000000in}}%
\pgfpathlineto{\pgfqpoint{0.000000in}{0.055556in}}%
\pgfusepath{stroke,fill}%
}%
\begin{pgfscope}%
\pgfsys@transformshift{7.363951in}{0.559721in}%
\pgfsys@useobject{currentmarker}{}%
\end{pgfscope}%
\end{pgfscope}%
\begin{pgfscope}%
\definecolor{textcolor}{rgb}{0.000000,0.000000,0.000000}%
\pgfsetstrokecolor{textcolor}%
\pgfsetfillcolor{textcolor}%
\pgftext[x=4.222231in,y=0.277777in,,top]{\color{textcolor}{\rmfamily\fontsize{14.000000}{16.800000}\selectfont\catcode`\^=\active\def^{\ifmmode\sp\else\^{}\fi}\catcode`\%=\active\def%{\%}Year}}%
\end{pgfscope}%
\begin{pgfscope}%
\pgfpathrectangle{\pgfqpoint{0.697913in}{0.559721in}}{\pgfqpoint{7.048636in}{4.990279in}}%
\pgfusepath{clip}%
\pgfsetrectcap%
\pgfsetroundjoin%
\pgfsetlinewidth{0.803000pt}%
\definecolor{currentstroke}{rgb}{0.690196,0.690196,0.690196}%
\pgfsetstrokecolor{currentstroke}%
\pgfsetdash{}{0pt}%
\pgfpathmoveto{\pgfqpoint{0.697913in}{1.056039in}}%
\pgfpathlineto{\pgfqpoint{7.746549in}{1.056039in}}%
\pgfusepath{stroke}%
\end{pgfscope}%
\begin{pgfscope}%
\pgfsetbuttcap%
\pgfsetroundjoin%
\definecolor{currentfill}{rgb}{0.000000,0.000000,0.000000}%
\pgfsetfillcolor{currentfill}%
\pgfsetlinewidth{1.254687pt}%
\definecolor{currentstroke}{rgb}{0.000000,0.000000,0.000000}%
\pgfsetstrokecolor{currentstroke}%
\pgfsetdash{}{0pt}%
\pgfsys@defobject{currentmarker}{\pgfqpoint{0.000000in}{0.000000in}}{\pgfqpoint{0.111111in}{0.000000in}}{%
\pgfpathmoveto{\pgfqpoint{0.000000in}{0.000000in}}%
\pgfpathlineto{\pgfqpoint{0.111111in}{0.000000in}}%
\pgfusepath{stroke,fill}%
}%
\begin{pgfscope}%
\pgfsys@transformshift{0.697913in}{1.056039in}%
\pgfsys@useobject{currentmarker}{}%
\end{pgfscope}%
\end{pgfscope}%
\begin{pgfscope}%
\definecolor{textcolor}{rgb}{0.000000,0.000000,0.000000}%
\pgfsetstrokecolor{textcolor}%
\pgfsetfillcolor{textcolor}%
\pgftext[x=0.355555in, y=0.986595in, left, base]{\color{textcolor}{\rmfamily\fontsize{14.000000}{16.800000}\selectfont\catcode`\^=\active\def^{\ifmmode\sp\else\^{}\fi}\catcode`\%=\active\def%{\%}$\mathdefault{350}$}}%
\end{pgfscope}%
\begin{pgfscope}%
\pgfpathrectangle{\pgfqpoint{0.697913in}{0.559721in}}{\pgfqpoint{7.048636in}{4.990279in}}%
\pgfusepath{clip}%
\pgfsetrectcap%
\pgfsetroundjoin%
\pgfsetlinewidth{0.803000pt}%
\definecolor{currentstroke}{rgb}{0.690196,0.690196,0.690196}%
\pgfsetstrokecolor{currentstroke}%
\pgfsetdash{}{0pt}%
\pgfpathmoveto{\pgfqpoint{0.697913in}{1.620678in}}%
\pgfpathlineto{\pgfqpoint{7.746549in}{1.620678in}}%
\pgfusepath{stroke}%
\end{pgfscope}%
\begin{pgfscope}%
\pgfsetbuttcap%
\pgfsetroundjoin%
\definecolor{currentfill}{rgb}{0.000000,0.000000,0.000000}%
\pgfsetfillcolor{currentfill}%
\pgfsetlinewidth{1.254687pt}%
\definecolor{currentstroke}{rgb}{0.000000,0.000000,0.000000}%
\pgfsetstrokecolor{currentstroke}%
\pgfsetdash{}{0pt}%
\pgfsys@defobject{currentmarker}{\pgfqpoint{0.000000in}{0.000000in}}{\pgfqpoint{0.111111in}{0.000000in}}{%
\pgfpathmoveto{\pgfqpoint{0.000000in}{0.000000in}}%
\pgfpathlineto{\pgfqpoint{0.111111in}{0.000000in}}%
\pgfusepath{stroke,fill}%
}%
\begin{pgfscope}%
\pgfsys@transformshift{0.697913in}{1.620678in}%
\pgfsys@useobject{currentmarker}{}%
\end{pgfscope}%
\end{pgfscope}%
\begin{pgfscope}%
\definecolor{textcolor}{rgb}{0.000000,0.000000,0.000000}%
\pgfsetstrokecolor{textcolor}%
\pgfsetfillcolor{textcolor}%
\pgftext[x=0.355555in, y=1.551234in, left, base]{\color{textcolor}{\rmfamily\fontsize{14.000000}{16.800000}\selectfont\catcode`\^=\active\def^{\ifmmode\sp\else\^{}\fi}\catcode`\%=\active\def%{\%}$\mathdefault{360}$}}%
\end{pgfscope}%
\begin{pgfscope}%
\pgfpathrectangle{\pgfqpoint{0.697913in}{0.559721in}}{\pgfqpoint{7.048636in}{4.990279in}}%
\pgfusepath{clip}%
\pgfsetrectcap%
\pgfsetroundjoin%
\pgfsetlinewidth{0.803000pt}%
\definecolor{currentstroke}{rgb}{0.690196,0.690196,0.690196}%
\pgfsetstrokecolor{currentstroke}%
\pgfsetdash{}{0pt}%
\pgfpathmoveto{\pgfqpoint{0.697913in}{2.185317in}}%
\pgfpathlineto{\pgfqpoint{7.746549in}{2.185317in}}%
\pgfusepath{stroke}%
\end{pgfscope}%
\begin{pgfscope}%
\pgfsetbuttcap%
\pgfsetroundjoin%
\definecolor{currentfill}{rgb}{0.000000,0.000000,0.000000}%
\pgfsetfillcolor{currentfill}%
\pgfsetlinewidth{1.254687pt}%
\definecolor{currentstroke}{rgb}{0.000000,0.000000,0.000000}%
\pgfsetstrokecolor{currentstroke}%
\pgfsetdash{}{0pt}%
\pgfsys@defobject{currentmarker}{\pgfqpoint{0.000000in}{0.000000in}}{\pgfqpoint{0.111111in}{0.000000in}}{%
\pgfpathmoveto{\pgfqpoint{0.000000in}{0.000000in}}%
\pgfpathlineto{\pgfqpoint{0.111111in}{0.000000in}}%
\pgfusepath{stroke,fill}%
}%
\begin{pgfscope}%
\pgfsys@transformshift{0.697913in}{2.185317in}%
\pgfsys@useobject{currentmarker}{}%
\end{pgfscope}%
\end{pgfscope}%
\begin{pgfscope}%
\definecolor{textcolor}{rgb}{0.000000,0.000000,0.000000}%
\pgfsetstrokecolor{textcolor}%
\pgfsetfillcolor{textcolor}%
\pgftext[x=0.355555in, y=2.115872in, left, base]{\color{textcolor}{\rmfamily\fontsize{14.000000}{16.800000}\selectfont\catcode`\^=\active\def^{\ifmmode\sp\else\^{}\fi}\catcode`\%=\active\def%{\%}$\mathdefault{370}$}}%
\end{pgfscope}%
\begin{pgfscope}%
\pgfpathrectangle{\pgfqpoint{0.697913in}{0.559721in}}{\pgfqpoint{7.048636in}{4.990279in}}%
\pgfusepath{clip}%
\pgfsetrectcap%
\pgfsetroundjoin%
\pgfsetlinewidth{0.803000pt}%
\definecolor{currentstroke}{rgb}{0.690196,0.690196,0.690196}%
\pgfsetstrokecolor{currentstroke}%
\pgfsetdash{}{0pt}%
\pgfpathmoveto{\pgfqpoint{0.697913in}{2.749956in}}%
\pgfpathlineto{\pgfqpoint{7.746549in}{2.749956in}}%
\pgfusepath{stroke}%
\end{pgfscope}%
\begin{pgfscope}%
\pgfsetbuttcap%
\pgfsetroundjoin%
\definecolor{currentfill}{rgb}{0.000000,0.000000,0.000000}%
\pgfsetfillcolor{currentfill}%
\pgfsetlinewidth{1.254687pt}%
\definecolor{currentstroke}{rgb}{0.000000,0.000000,0.000000}%
\pgfsetstrokecolor{currentstroke}%
\pgfsetdash{}{0pt}%
\pgfsys@defobject{currentmarker}{\pgfqpoint{0.000000in}{0.000000in}}{\pgfqpoint{0.111111in}{0.000000in}}{%
\pgfpathmoveto{\pgfqpoint{0.000000in}{0.000000in}}%
\pgfpathlineto{\pgfqpoint{0.111111in}{0.000000in}}%
\pgfusepath{stroke,fill}%
}%
\begin{pgfscope}%
\pgfsys@transformshift{0.697913in}{2.749956in}%
\pgfsys@useobject{currentmarker}{}%
\end{pgfscope}%
\end{pgfscope}%
\begin{pgfscope}%
\definecolor{textcolor}{rgb}{0.000000,0.000000,0.000000}%
\pgfsetstrokecolor{textcolor}%
\pgfsetfillcolor{textcolor}%
\pgftext[x=0.355555in, y=2.680511in, left, base]{\color{textcolor}{\rmfamily\fontsize{14.000000}{16.800000}\selectfont\catcode`\^=\active\def^{\ifmmode\sp\else\^{}\fi}\catcode`\%=\active\def%{\%}$\mathdefault{380}$}}%
\end{pgfscope}%
\begin{pgfscope}%
\pgfpathrectangle{\pgfqpoint{0.697913in}{0.559721in}}{\pgfqpoint{7.048636in}{4.990279in}}%
\pgfusepath{clip}%
\pgfsetrectcap%
\pgfsetroundjoin%
\pgfsetlinewidth{0.803000pt}%
\definecolor{currentstroke}{rgb}{0.690196,0.690196,0.690196}%
\pgfsetstrokecolor{currentstroke}%
\pgfsetdash{}{0pt}%
\pgfpathmoveto{\pgfqpoint{0.697913in}{3.314595in}}%
\pgfpathlineto{\pgfqpoint{7.746549in}{3.314595in}}%
\pgfusepath{stroke}%
\end{pgfscope}%
\begin{pgfscope}%
\pgfsetbuttcap%
\pgfsetroundjoin%
\definecolor{currentfill}{rgb}{0.000000,0.000000,0.000000}%
\pgfsetfillcolor{currentfill}%
\pgfsetlinewidth{1.254687pt}%
\definecolor{currentstroke}{rgb}{0.000000,0.000000,0.000000}%
\pgfsetstrokecolor{currentstroke}%
\pgfsetdash{}{0pt}%
\pgfsys@defobject{currentmarker}{\pgfqpoint{0.000000in}{0.000000in}}{\pgfqpoint{0.111111in}{0.000000in}}{%
\pgfpathmoveto{\pgfqpoint{0.000000in}{0.000000in}}%
\pgfpathlineto{\pgfqpoint{0.111111in}{0.000000in}}%
\pgfusepath{stroke,fill}%
}%
\begin{pgfscope}%
\pgfsys@transformshift{0.697913in}{3.314595in}%
\pgfsys@useobject{currentmarker}{}%
\end{pgfscope}%
\end{pgfscope}%
\begin{pgfscope}%
\definecolor{textcolor}{rgb}{0.000000,0.000000,0.000000}%
\pgfsetstrokecolor{textcolor}%
\pgfsetfillcolor{textcolor}%
\pgftext[x=0.355555in, y=3.245150in, left, base]{\color{textcolor}{\rmfamily\fontsize{14.000000}{16.800000}\selectfont\catcode`\^=\active\def^{\ifmmode\sp\else\^{}\fi}\catcode`\%=\active\def%{\%}$\mathdefault{390}$}}%
\end{pgfscope}%
\begin{pgfscope}%
\pgfpathrectangle{\pgfqpoint{0.697913in}{0.559721in}}{\pgfqpoint{7.048636in}{4.990279in}}%
\pgfusepath{clip}%
\pgfsetrectcap%
\pgfsetroundjoin%
\pgfsetlinewidth{0.803000pt}%
\definecolor{currentstroke}{rgb}{0.690196,0.690196,0.690196}%
\pgfsetstrokecolor{currentstroke}%
\pgfsetdash{}{0pt}%
\pgfpathmoveto{\pgfqpoint{0.697913in}{3.879233in}}%
\pgfpathlineto{\pgfqpoint{7.746549in}{3.879233in}}%
\pgfusepath{stroke}%
\end{pgfscope}%
\begin{pgfscope}%
\pgfsetbuttcap%
\pgfsetroundjoin%
\definecolor{currentfill}{rgb}{0.000000,0.000000,0.000000}%
\pgfsetfillcolor{currentfill}%
\pgfsetlinewidth{1.254687pt}%
\definecolor{currentstroke}{rgb}{0.000000,0.000000,0.000000}%
\pgfsetstrokecolor{currentstroke}%
\pgfsetdash{}{0pt}%
\pgfsys@defobject{currentmarker}{\pgfqpoint{0.000000in}{0.000000in}}{\pgfqpoint{0.111111in}{0.000000in}}{%
\pgfpathmoveto{\pgfqpoint{0.000000in}{0.000000in}}%
\pgfpathlineto{\pgfqpoint{0.111111in}{0.000000in}}%
\pgfusepath{stroke,fill}%
}%
\begin{pgfscope}%
\pgfsys@transformshift{0.697913in}{3.879233in}%
\pgfsys@useobject{currentmarker}{}%
\end{pgfscope}%
\end{pgfscope}%
\begin{pgfscope}%
\definecolor{textcolor}{rgb}{0.000000,0.000000,0.000000}%
\pgfsetstrokecolor{textcolor}%
\pgfsetfillcolor{textcolor}%
\pgftext[x=0.355555in, y=3.809789in, left, base]{\color{textcolor}{\rmfamily\fontsize{14.000000}{16.800000}\selectfont\catcode`\^=\active\def^{\ifmmode\sp\else\^{}\fi}\catcode`\%=\active\def%{\%}$\mathdefault{400}$}}%
\end{pgfscope}%
\begin{pgfscope}%
\pgfpathrectangle{\pgfqpoint{0.697913in}{0.559721in}}{\pgfqpoint{7.048636in}{4.990279in}}%
\pgfusepath{clip}%
\pgfsetrectcap%
\pgfsetroundjoin%
\pgfsetlinewidth{0.803000pt}%
\definecolor{currentstroke}{rgb}{0.690196,0.690196,0.690196}%
\pgfsetstrokecolor{currentstroke}%
\pgfsetdash{}{0pt}%
\pgfpathmoveto{\pgfqpoint{0.697913in}{4.443872in}}%
\pgfpathlineto{\pgfqpoint{7.746549in}{4.443872in}}%
\pgfusepath{stroke}%
\end{pgfscope}%
\begin{pgfscope}%
\pgfsetbuttcap%
\pgfsetroundjoin%
\definecolor{currentfill}{rgb}{0.000000,0.000000,0.000000}%
\pgfsetfillcolor{currentfill}%
\pgfsetlinewidth{1.254687pt}%
\definecolor{currentstroke}{rgb}{0.000000,0.000000,0.000000}%
\pgfsetstrokecolor{currentstroke}%
\pgfsetdash{}{0pt}%
\pgfsys@defobject{currentmarker}{\pgfqpoint{0.000000in}{0.000000in}}{\pgfqpoint{0.111111in}{0.000000in}}{%
\pgfpathmoveto{\pgfqpoint{0.000000in}{0.000000in}}%
\pgfpathlineto{\pgfqpoint{0.111111in}{0.000000in}}%
\pgfusepath{stroke,fill}%
}%
\begin{pgfscope}%
\pgfsys@transformshift{0.697913in}{4.443872in}%
\pgfsys@useobject{currentmarker}{}%
\end{pgfscope}%
\end{pgfscope}%
\begin{pgfscope}%
\definecolor{textcolor}{rgb}{0.000000,0.000000,0.000000}%
\pgfsetstrokecolor{textcolor}%
\pgfsetfillcolor{textcolor}%
\pgftext[x=0.355555in, y=4.374428in, left, base]{\color{textcolor}{\rmfamily\fontsize{14.000000}{16.800000}\selectfont\catcode`\^=\active\def^{\ifmmode\sp\else\^{}\fi}\catcode`\%=\active\def%{\%}$\mathdefault{410}$}}%
\end{pgfscope}%
\begin{pgfscope}%
\pgfpathrectangle{\pgfqpoint{0.697913in}{0.559721in}}{\pgfqpoint{7.048636in}{4.990279in}}%
\pgfusepath{clip}%
\pgfsetrectcap%
\pgfsetroundjoin%
\pgfsetlinewidth{0.803000pt}%
\definecolor{currentstroke}{rgb}{0.690196,0.690196,0.690196}%
\pgfsetstrokecolor{currentstroke}%
\pgfsetdash{}{0pt}%
\pgfpathmoveto{\pgfqpoint{0.697913in}{5.008511in}}%
\pgfpathlineto{\pgfqpoint{7.746549in}{5.008511in}}%
\pgfusepath{stroke}%
\end{pgfscope}%
\begin{pgfscope}%
\pgfsetbuttcap%
\pgfsetroundjoin%
\definecolor{currentfill}{rgb}{0.000000,0.000000,0.000000}%
\pgfsetfillcolor{currentfill}%
\pgfsetlinewidth{1.254687pt}%
\definecolor{currentstroke}{rgb}{0.000000,0.000000,0.000000}%
\pgfsetstrokecolor{currentstroke}%
\pgfsetdash{}{0pt}%
\pgfsys@defobject{currentmarker}{\pgfqpoint{0.000000in}{0.000000in}}{\pgfqpoint{0.111111in}{0.000000in}}{%
\pgfpathmoveto{\pgfqpoint{0.000000in}{0.000000in}}%
\pgfpathlineto{\pgfqpoint{0.111111in}{0.000000in}}%
\pgfusepath{stroke,fill}%
}%
\begin{pgfscope}%
\pgfsys@transformshift{0.697913in}{5.008511in}%
\pgfsys@useobject{currentmarker}{}%
\end{pgfscope}%
\end{pgfscope}%
\begin{pgfscope}%
\definecolor{textcolor}{rgb}{0.000000,0.000000,0.000000}%
\pgfsetstrokecolor{textcolor}%
\pgfsetfillcolor{textcolor}%
\pgftext[x=0.355555in, y=4.939067in, left, base]{\color{textcolor}{\rmfamily\fontsize{14.000000}{16.800000}\selectfont\catcode`\^=\active\def^{\ifmmode\sp\else\^{}\fi}\catcode`\%=\active\def%{\%}$\mathdefault{420}$}}%
\end{pgfscope}%
\begin{pgfscope}%
\pgfsetbuttcap%
\pgfsetroundjoin%
\definecolor{currentfill}{rgb}{0.000000,0.000000,0.000000}%
\pgfsetfillcolor{currentfill}%
\pgfsetlinewidth{1.254687pt}%
\definecolor{currentstroke}{rgb}{0.000000,0.000000,0.000000}%
\pgfsetstrokecolor{currentstroke}%
\pgfsetdash{}{0pt}%
\pgfsys@defobject{currentmarker}{\pgfqpoint{0.000000in}{0.000000in}}{\pgfqpoint{0.055556in}{0.000000in}}{%
\pgfpathmoveto{\pgfqpoint{0.000000in}{0.000000in}}%
\pgfpathlineto{\pgfqpoint{0.055556in}{0.000000in}}%
\pgfusepath{stroke,fill}%
}%
\begin{pgfscope}%
\pgfsys@transformshift{0.697913in}{0.604328in}%
\pgfsys@useobject{currentmarker}{}%
\end{pgfscope}%
\end{pgfscope}%
\begin{pgfscope}%
\pgfsetbuttcap%
\pgfsetroundjoin%
\definecolor{currentfill}{rgb}{0.000000,0.000000,0.000000}%
\pgfsetfillcolor{currentfill}%
\pgfsetlinewidth{1.254687pt}%
\definecolor{currentstroke}{rgb}{0.000000,0.000000,0.000000}%
\pgfsetstrokecolor{currentstroke}%
\pgfsetdash{}{0pt}%
\pgfsys@defobject{currentmarker}{\pgfqpoint{0.000000in}{0.000000in}}{\pgfqpoint{0.055556in}{0.000000in}}{%
\pgfpathmoveto{\pgfqpoint{0.000000in}{0.000000in}}%
\pgfpathlineto{\pgfqpoint{0.055556in}{0.000000in}}%
\pgfusepath{stroke,fill}%
}%
\begin{pgfscope}%
\pgfsys@transformshift{0.697913in}{0.717256in}%
\pgfsys@useobject{currentmarker}{}%
\end{pgfscope}%
\end{pgfscope}%
\begin{pgfscope}%
\pgfsetbuttcap%
\pgfsetroundjoin%
\definecolor{currentfill}{rgb}{0.000000,0.000000,0.000000}%
\pgfsetfillcolor{currentfill}%
\pgfsetlinewidth{1.254687pt}%
\definecolor{currentstroke}{rgb}{0.000000,0.000000,0.000000}%
\pgfsetstrokecolor{currentstroke}%
\pgfsetdash{}{0pt}%
\pgfsys@defobject{currentmarker}{\pgfqpoint{0.000000in}{0.000000in}}{\pgfqpoint{0.055556in}{0.000000in}}{%
\pgfpathmoveto{\pgfqpoint{0.000000in}{0.000000in}}%
\pgfpathlineto{\pgfqpoint{0.055556in}{0.000000in}}%
\pgfusepath{stroke,fill}%
}%
\begin{pgfscope}%
\pgfsys@transformshift{0.697913in}{0.830183in}%
\pgfsys@useobject{currentmarker}{}%
\end{pgfscope}%
\end{pgfscope}%
\begin{pgfscope}%
\pgfsetbuttcap%
\pgfsetroundjoin%
\definecolor{currentfill}{rgb}{0.000000,0.000000,0.000000}%
\pgfsetfillcolor{currentfill}%
\pgfsetlinewidth{1.254687pt}%
\definecolor{currentstroke}{rgb}{0.000000,0.000000,0.000000}%
\pgfsetstrokecolor{currentstroke}%
\pgfsetdash{}{0pt}%
\pgfsys@defobject{currentmarker}{\pgfqpoint{0.000000in}{0.000000in}}{\pgfqpoint{0.055556in}{0.000000in}}{%
\pgfpathmoveto{\pgfqpoint{0.000000in}{0.000000in}}%
\pgfpathlineto{\pgfqpoint{0.055556in}{0.000000in}}%
\pgfusepath{stroke,fill}%
}%
\begin{pgfscope}%
\pgfsys@transformshift{0.697913in}{0.943111in}%
\pgfsys@useobject{currentmarker}{}%
\end{pgfscope}%
\end{pgfscope}%
\begin{pgfscope}%
\pgfsetbuttcap%
\pgfsetroundjoin%
\definecolor{currentfill}{rgb}{0.000000,0.000000,0.000000}%
\pgfsetfillcolor{currentfill}%
\pgfsetlinewidth{1.254687pt}%
\definecolor{currentstroke}{rgb}{0.000000,0.000000,0.000000}%
\pgfsetstrokecolor{currentstroke}%
\pgfsetdash{}{0pt}%
\pgfsys@defobject{currentmarker}{\pgfqpoint{0.000000in}{0.000000in}}{\pgfqpoint{0.055556in}{0.000000in}}{%
\pgfpathmoveto{\pgfqpoint{0.000000in}{0.000000in}}%
\pgfpathlineto{\pgfqpoint{0.055556in}{0.000000in}}%
\pgfusepath{stroke,fill}%
}%
\begin{pgfscope}%
\pgfsys@transformshift{0.697913in}{1.168967in}%
\pgfsys@useobject{currentmarker}{}%
\end{pgfscope}%
\end{pgfscope}%
\begin{pgfscope}%
\pgfsetbuttcap%
\pgfsetroundjoin%
\definecolor{currentfill}{rgb}{0.000000,0.000000,0.000000}%
\pgfsetfillcolor{currentfill}%
\pgfsetlinewidth{1.254687pt}%
\definecolor{currentstroke}{rgb}{0.000000,0.000000,0.000000}%
\pgfsetstrokecolor{currentstroke}%
\pgfsetdash{}{0pt}%
\pgfsys@defobject{currentmarker}{\pgfqpoint{0.000000in}{0.000000in}}{\pgfqpoint{0.055556in}{0.000000in}}{%
\pgfpathmoveto{\pgfqpoint{0.000000in}{0.000000in}}%
\pgfpathlineto{\pgfqpoint{0.055556in}{0.000000in}}%
\pgfusepath{stroke,fill}%
}%
\begin{pgfscope}%
\pgfsys@transformshift{0.697913in}{1.281895in}%
\pgfsys@useobject{currentmarker}{}%
\end{pgfscope}%
\end{pgfscope}%
\begin{pgfscope}%
\pgfsetbuttcap%
\pgfsetroundjoin%
\definecolor{currentfill}{rgb}{0.000000,0.000000,0.000000}%
\pgfsetfillcolor{currentfill}%
\pgfsetlinewidth{1.254687pt}%
\definecolor{currentstroke}{rgb}{0.000000,0.000000,0.000000}%
\pgfsetstrokecolor{currentstroke}%
\pgfsetdash{}{0pt}%
\pgfsys@defobject{currentmarker}{\pgfqpoint{0.000000in}{0.000000in}}{\pgfqpoint{0.055556in}{0.000000in}}{%
\pgfpathmoveto{\pgfqpoint{0.000000in}{0.000000in}}%
\pgfpathlineto{\pgfqpoint{0.055556in}{0.000000in}}%
\pgfusepath{stroke,fill}%
}%
\begin{pgfscope}%
\pgfsys@transformshift{0.697913in}{1.394822in}%
\pgfsys@useobject{currentmarker}{}%
\end{pgfscope}%
\end{pgfscope}%
\begin{pgfscope}%
\pgfsetbuttcap%
\pgfsetroundjoin%
\definecolor{currentfill}{rgb}{0.000000,0.000000,0.000000}%
\pgfsetfillcolor{currentfill}%
\pgfsetlinewidth{1.254687pt}%
\definecolor{currentstroke}{rgb}{0.000000,0.000000,0.000000}%
\pgfsetstrokecolor{currentstroke}%
\pgfsetdash{}{0pt}%
\pgfsys@defobject{currentmarker}{\pgfqpoint{0.000000in}{0.000000in}}{\pgfqpoint{0.055556in}{0.000000in}}{%
\pgfpathmoveto{\pgfqpoint{0.000000in}{0.000000in}}%
\pgfpathlineto{\pgfqpoint{0.055556in}{0.000000in}}%
\pgfusepath{stroke,fill}%
}%
\begin{pgfscope}%
\pgfsys@transformshift{0.697913in}{1.507750in}%
\pgfsys@useobject{currentmarker}{}%
\end{pgfscope}%
\end{pgfscope}%
\begin{pgfscope}%
\pgfsetbuttcap%
\pgfsetroundjoin%
\definecolor{currentfill}{rgb}{0.000000,0.000000,0.000000}%
\pgfsetfillcolor{currentfill}%
\pgfsetlinewidth{1.254687pt}%
\definecolor{currentstroke}{rgb}{0.000000,0.000000,0.000000}%
\pgfsetstrokecolor{currentstroke}%
\pgfsetdash{}{0pt}%
\pgfsys@defobject{currentmarker}{\pgfqpoint{0.000000in}{0.000000in}}{\pgfqpoint{0.055556in}{0.000000in}}{%
\pgfpathmoveto{\pgfqpoint{0.000000in}{0.000000in}}%
\pgfpathlineto{\pgfqpoint{0.055556in}{0.000000in}}%
\pgfusepath{stroke,fill}%
}%
\begin{pgfscope}%
\pgfsys@transformshift{0.697913in}{1.733606in}%
\pgfsys@useobject{currentmarker}{}%
\end{pgfscope}%
\end{pgfscope}%
\begin{pgfscope}%
\pgfsetbuttcap%
\pgfsetroundjoin%
\definecolor{currentfill}{rgb}{0.000000,0.000000,0.000000}%
\pgfsetfillcolor{currentfill}%
\pgfsetlinewidth{1.254687pt}%
\definecolor{currentstroke}{rgb}{0.000000,0.000000,0.000000}%
\pgfsetstrokecolor{currentstroke}%
\pgfsetdash{}{0pt}%
\pgfsys@defobject{currentmarker}{\pgfqpoint{0.000000in}{0.000000in}}{\pgfqpoint{0.055556in}{0.000000in}}{%
\pgfpathmoveto{\pgfqpoint{0.000000in}{0.000000in}}%
\pgfpathlineto{\pgfqpoint{0.055556in}{0.000000in}}%
\pgfusepath{stroke,fill}%
}%
\begin{pgfscope}%
\pgfsys@transformshift{0.697913in}{1.846533in}%
\pgfsys@useobject{currentmarker}{}%
\end{pgfscope}%
\end{pgfscope}%
\begin{pgfscope}%
\pgfsetbuttcap%
\pgfsetroundjoin%
\definecolor{currentfill}{rgb}{0.000000,0.000000,0.000000}%
\pgfsetfillcolor{currentfill}%
\pgfsetlinewidth{1.254687pt}%
\definecolor{currentstroke}{rgb}{0.000000,0.000000,0.000000}%
\pgfsetstrokecolor{currentstroke}%
\pgfsetdash{}{0pt}%
\pgfsys@defobject{currentmarker}{\pgfqpoint{0.000000in}{0.000000in}}{\pgfqpoint{0.055556in}{0.000000in}}{%
\pgfpathmoveto{\pgfqpoint{0.000000in}{0.000000in}}%
\pgfpathlineto{\pgfqpoint{0.055556in}{0.000000in}}%
\pgfusepath{stroke,fill}%
}%
\begin{pgfscope}%
\pgfsys@transformshift{0.697913in}{1.959461in}%
\pgfsys@useobject{currentmarker}{}%
\end{pgfscope}%
\end{pgfscope}%
\begin{pgfscope}%
\pgfsetbuttcap%
\pgfsetroundjoin%
\definecolor{currentfill}{rgb}{0.000000,0.000000,0.000000}%
\pgfsetfillcolor{currentfill}%
\pgfsetlinewidth{1.254687pt}%
\definecolor{currentstroke}{rgb}{0.000000,0.000000,0.000000}%
\pgfsetstrokecolor{currentstroke}%
\pgfsetdash{}{0pt}%
\pgfsys@defobject{currentmarker}{\pgfqpoint{0.000000in}{0.000000in}}{\pgfqpoint{0.055556in}{0.000000in}}{%
\pgfpathmoveto{\pgfqpoint{0.000000in}{0.000000in}}%
\pgfpathlineto{\pgfqpoint{0.055556in}{0.000000in}}%
\pgfusepath{stroke,fill}%
}%
\begin{pgfscope}%
\pgfsys@transformshift{0.697913in}{2.072389in}%
\pgfsys@useobject{currentmarker}{}%
\end{pgfscope}%
\end{pgfscope}%
\begin{pgfscope}%
\pgfsetbuttcap%
\pgfsetroundjoin%
\definecolor{currentfill}{rgb}{0.000000,0.000000,0.000000}%
\pgfsetfillcolor{currentfill}%
\pgfsetlinewidth{1.254687pt}%
\definecolor{currentstroke}{rgb}{0.000000,0.000000,0.000000}%
\pgfsetstrokecolor{currentstroke}%
\pgfsetdash{}{0pt}%
\pgfsys@defobject{currentmarker}{\pgfqpoint{0.000000in}{0.000000in}}{\pgfqpoint{0.055556in}{0.000000in}}{%
\pgfpathmoveto{\pgfqpoint{0.000000in}{0.000000in}}%
\pgfpathlineto{\pgfqpoint{0.055556in}{0.000000in}}%
\pgfusepath{stroke,fill}%
}%
\begin{pgfscope}%
\pgfsys@transformshift{0.697913in}{2.298245in}%
\pgfsys@useobject{currentmarker}{}%
\end{pgfscope}%
\end{pgfscope}%
\begin{pgfscope}%
\pgfsetbuttcap%
\pgfsetroundjoin%
\definecolor{currentfill}{rgb}{0.000000,0.000000,0.000000}%
\pgfsetfillcolor{currentfill}%
\pgfsetlinewidth{1.254687pt}%
\definecolor{currentstroke}{rgb}{0.000000,0.000000,0.000000}%
\pgfsetstrokecolor{currentstroke}%
\pgfsetdash{}{0pt}%
\pgfsys@defobject{currentmarker}{\pgfqpoint{0.000000in}{0.000000in}}{\pgfqpoint{0.055556in}{0.000000in}}{%
\pgfpathmoveto{\pgfqpoint{0.000000in}{0.000000in}}%
\pgfpathlineto{\pgfqpoint{0.055556in}{0.000000in}}%
\pgfusepath{stroke,fill}%
}%
\begin{pgfscope}%
\pgfsys@transformshift{0.697913in}{2.411172in}%
\pgfsys@useobject{currentmarker}{}%
\end{pgfscope}%
\end{pgfscope}%
\begin{pgfscope}%
\pgfsetbuttcap%
\pgfsetroundjoin%
\definecolor{currentfill}{rgb}{0.000000,0.000000,0.000000}%
\pgfsetfillcolor{currentfill}%
\pgfsetlinewidth{1.254687pt}%
\definecolor{currentstroke}{rgb}{0.000000,0.000000,0.000000}%
\pgfsetstrokecolor{currentstroke}%
\pgfsetdash{}{0pt}%
\pgfsys@defobject{currentmarker}{\pgfqpoint{0.000000in}{0.000000in}}{\pgfqpoint{0.055556in}{0.000000in}}{%
\pgfpathmoveto{\pgfqpoint{0.000000in}{0.000000in}}%
\pgfpathlineto{\pgfqpoint{0.055556in}{0.000000in}}%
\pgfusepath{stroke,fill}%
}%
\begin{pgfscope}%
\pgfsys@transformshift{0.697913in}{2.524100in}%
\pgfsys@useobject{currentmarker}{}%
\end{pgfscope}%
\end{pgfscope}%
\begin{pgfscope}%
\pgfsetbuttcap%
\pgfsetroundjoin%
\definecolor{currentfill}{rgb}{0.000000,0.000000,0.000000}%
\pgfsetfillcolor{currentfill}%
\pgfsetlinewidth{1.254687pt}%
\definecolor{currentstroke}{rgb}{0.000000,0.000000,0.000000}%
\pgfsetstrokecolor{currentstroke}%
\pgfsetdash{}{0pt}%
\pgfsys@defobject{currentmarker}{\pgfqpoint{0.000000in}{0.000000in}}{\pgfqpoint{0.055556in}{0.000000in}}{%
\pgfpathmoveto{\pgfqpoint{0.000000in}{0.000000in}}%
\pgfpathlineto{\pgfqpoint{0.055556in}{0.000000in}}%
\pgfusepath{stroke,fill}%
}%
\begin{pgfscope}%
\pgfsys@transformshift{0.697913in}{2.637028in}%
\pgfsys@useobject{currentmarker}{}%
\end{pgfscope}%
\end{pgfscope}%
\begin{pgfscope}%
\pgfsetbuttcap%
\pgfsetroundjoin%
\definecolor{currentfill}{rgb}{0.000000,0.000000,0.000000}%
\pgfsetfillcolor{currentfill}%
\pgfsetlinewidth{1.254687pt}%
\definecolor{currentstroke}{rgb}{0.000000,0.000000,0.000000}%
\pgfsetstrokecolor{currentstroke}%
\pgfsetdash{}{0pt}%
\pgfsys@defobject{currentmarker}{\pgfqpoint{0.000000in}{0.000000in}}{\pgfqpoint{0.055556in}{0.000000in}}{%
\pgfpathmoveto{\pgfqpoint{0.000000in}{0.000000in}}%
\pgfpathlineto{\pgfqpoint{0.055556in}{0.000000in}}%
\pgfusepath{stroke,fill}%
}%
\begin{pgfscope}%
\pgfsys@transformshift{0.697913in}{2.862883in}%
\pgfsys@useobject{currentmarker}{}%
\end{pgfscope}%
\end{pgfscope}%
\begin{pgfscope}%
\pgfsetbuttcap%
\pgfsetroundjoin%
\definecolor{currentfill}{rgb}{0.000000,0.000000,0.000000}%
\pgfsetfillcolor{currentfill}%
\pgfsetlinewidth{1.254687pt}%
\definecolor{currentstroke}{rgb}{0.000000,0.000000,0.000000}%
\pgfsetstrokecolor{currentstroke}%
\pgfsetdash{}{0pt}%
\pgfsys@defobject{currentmarker}{\pgfqpoint{0.000000in}{0.000000in}}{\pgfqpoint{0.055556in}{0.000000in}}{%
\pgfpathmoveto{\pgfqpoint{0.000000in}{0.000000in}}%
\pgfpathlineto{\pgfqpoint{0.055556in}{0.000000in}}%
\pgfusepath{stroke,fill}%
}%
\begin{pgfscope}%
\pgfsys@transformshift{0.697913in}{2.975811in}%
\pgfsys@useobject{currentmarker}{}%
\end{pgfscope}%
\end{pgfscope}%
\begin{pgfscope}%
\pgfsetbuttcap%
\pgfsetroundjoin%
\definecolor{currentfill}{rgb}{0.000000,0.000000,0.000000}%
\pgfsetfillcolor{currentfill}%
\pgfsetlinewidth{1.254687pt}%
\definecolor{currentstroke}{rgb}{0.000000,0.000000,0.000000}%
\pgfsetstrokecolor{currentstroke}%
\pgfsetdash{}{0pt}%
\pgfsys@defobject{currentmarker}{\pgfqpoint{0.000000in}{0.000000in}}{\pgfqpoint{0.055556in}{0.000000in}}{%
\pgfpathmoveto{\pgfqpoint{0.000000in}{0.000000in}}%
\pgfpathlineto{\pgfqpoint{0.055556in}{0.000000in}}%
\pgfusepath{stroke,fill}%
}%
\begin{pgfscope}%
\pgfsys@transformshift{0.697913in}{3.088739in}%
\pgfsys@useobject{currentmarker}{}%
\end{pgfscope}%
\end{pgfscope}%
\begin{pgfscope}%
\pgfsetbuttcap%
\pgfsetroundjoin%
\definecolor{currentfill}{rgb}{0.000000,0.000000,0.000000}%
\pgfsetfillcolor{currentfill}%
\pgfsetlinewidth{1.254687pt}%
\definecolor{currentstroke}{rgb}{0.000000,0.000000,0.000000}%
\pgfsetstrokecolor{currentstroke}%
\pgfsetdash{}{0pt}%
\pgfsys@defobject{currentmarker}{\pgfqpoint{0.000000in}{0.000000in}}{\pgfqpoint{0.055556in}{0.000000in}}{%
\pgfpathmoveto{\pgfqpoint{0.000000in}{0.000000in}}%
\pgfpathlineto{\pgfqpoint{0.055556in}{0.000000in}}%
\pgfusepath{stroke,fill}%
}%
\begin{pgfscope}%
\pgfsys@transformshift{0.697913in}{3.201667in}%
\pgfsys@useobject{currentmarker}{}%
\end{pgfscope}%
\end{pgfscope}%
\begin{pgfscope}%
\pgfsetbuttcap%
\pgfsetroundjoin%
\definecolor{currentfill}{rgb}{0.000000,0.000000,0.000000}%
\pgfsetfillcolor{currentfill}%
\pgfsetlinewidth{1.254687pt}%
\definecolor{currentstroke}{rgb}{0.000000,0.000000,0.000000}%
\pgfsetstrokecolor{currentstroke}%
\pgfsetdash{}{0pt}%
\pgfsys@defobject{currentmarker}{\pgfqpoint{0.000000in}{0.000000in}}{\pgfqpoint{0.055556in}{0.000000in}}{%
\pgfpathmoveto{\pgfqpoint{0.000000in}{0.000000in}}%
\pgfpathlineto{\pgfqpoint{0.055556in}{0.000000in}}%
\pgfusepath{stroke,fill}%
}%
\begin{pgfscope}%
\pgfsys@transformshift{0.697913in}{3.427522in}%
\pgfsys@useobject{currentmarker}{}%
\end{pgfscope}%
\end{pgfscope}%
\begin{pgfscope}%
\pgfsetbuttcap%
\pgfsetroundjoin%
\definecolor{currentfill}{rgb}{0.000000,0.000000,0.000000}%
\pgfsetfillcolor{currentfill}%
\pgfsetlinewidth{1.254687pt}%
\definecolor{currentstroke}{rgb}{0.000000,0.000000,0.000000}%
\pgfsetstrokecolor{currentstroke}%
\pgfsetdash{}{0pt}%
\pgfsys@defobject{currentmarker}{\pgfqpoint{0.000000in}{0.000000in}}{\pgfqpoint{0.055556in}{0.000000in}}{%
\pgfpathmoveto{\pgfqpoint{0.000000in}{0.000000in}}%
\pgfpathlineto{\pgfqpoint{0.055556in}{0.000000in}}%
\pgfusepath{stroke,fill}%
}%
\begin{pgfscope}%
\pgfsys@transformshift{0.697913in}{3.540450in}%
\pgfsys@useobject{currentmarker}{}%
\end{pgfscope}%
\end{pgfscope}%
\begin{pgfscope}%
\pgfsetbuttcap%
\pgfsetroundjoin%
\definecolor{currentfill}{rgb}{0.000000,0.000000,0.000000}%
\pgfsetfillcolor{currentfill}%
\pgfsetlinewidth{1.254687pt}%
\definecolor{currentstroke}{rgb}{0.000000,0.000000,0.000000}%
\pgfsetstrokecolor{currentstroke}%
\pgfsetdash{}{0pt}%
\pgfsys@defobject{currentmarker}{\pgfqpoint{0.000000in}{0.000000in}}{\pgfqpoint{0.055556in}{0.000000in}}{%
\pgfpathmoveto{\pgfqpoint{0.000000in}{0.000000in}}%
\pgfpathlineto{\pgfqpoint{0.055556in}{0.000000in}}%
\pgfusepath{stroke,fill}%
}%
\begin{pgfscope}%
\pgfsys@transformshift{0.697913in}{3.653378in}%
\pgfsys@useobject{currentmarker}{}%
\end{pgfscope}%
\end{pgfscope}%
\begin{pgfscope}%
\pgfsetbuttcap%
\pgfsetroundjoin%
\definecolor{currentfill}{rgb}{0.000000,0.000000,0.000000}%
\pgfsetfillcolor{currentfill}%
\pgfsetlinewidth{1.254687pt}%
\definecolor{currentstroke}{rgb}{0.000000,0.000000,0.000000}%
\pgfsetstrokecolor{currentstroke}%
\pgfsetdash{}{0pt}%
\pgfsys@defobject{currentmarker}{\pgfqpoint{0.000000in}{0.000000in}}{\pgfqpoint{0.055556in}{0.000000in}}{%
\pgfpathmoveto{\pgfqpoint{0.000000in}{0.000000in}}%
\pgfpathlineto{\pgfqpoint{0.055556in}{0.000000in}}%
\pgfusepath{stroke,fill}%
}%
\begin{pgfscope}%
\pgfsys@transformshift{0.697913in}{3.766306in}%
\pgfsys@useobject{currentmarker}{}%
\end{pgfscope}%
\end{pgfscope}%
\begin{pgfscope}%
\pgfsetbuttcap%
\pgfsetroundjoin%
\definecolor{currentfill}{rgb}{0.000000,0.000000,0.000000}%
\pgfsetfillcolor{currentfill}%
\pgfsetlinewidth{1.254687pt}%
\definecolor{currentstroke}{rgb}{0.000000,0.000000,0.000000}%
\pgfsetstrokecolor{currentstroke}%
\pgfsetdash{}{0pt}%
\pgfsys@defobject{currentmarker}{\pgfqpoint{0.000000in}{0.000000in}}{\pgfqpoint{0.055556in}{0.000000in}}{%
\pgfpathmoveto{\pgfqpoint{0.000000in}{0.000000in}}%
\pgfpathlineto{\pgfqpoint{0.055556in}{0.000000in}}%
\pgfusepath{stroke,fill}%
}%
\begin{pgfscope}%
\pgfsys@transformshift{0.697913in}{3.992161in}%
\pgfsys@useobject{currentmarker}{}%
\end{pgfscope}%
\end{pgfscope}%
\begin{pgfscope}%
\pgfsetbuttcap%
\pgfsetroundjoin%
\definecolor{currentfill}{rgb}{0.000000,0.000000,0.000000}%
\pgfsetfillcolor{currentfill}%
\pgfsetlinewidth{1.254687pt}%
\definecolor{currentstroke}{rgb}{0.000000,0.000000,0.000000}%
\pgfsetstrokecolor{currentstroke}%
\pgfsetdash{}{0pt}%
\pgfsys@defobject{currentmarker}{\pgfqpoint{0.000000in}{0.000000in}}{\pgfqpoint{0.055556in}{0.000000in}}{%
\pgfpathmoveto{\pgfqpoint{0.000000in}{0.000000in}}%
\pgfpathlineto{\pgfqpoint{0.055556in}{0.000000in}}%
\pgfusepath{stroke,fill}%
}%
\begin{pgfscope}%
\pgfsys@transformshift{0.697913in}{4.105089in}%
\pgfsys@useobject{currentmarker}{}%
\end{pgfscope}%
\end{pgfscope}%
\begin{pgfscope}%
\pgfsetbuttcap%
\pgfsetroundjoin%
\definecolor{currentfill}{rgb}{0.000000,0.000000,0.000000}%
\pgfsetfillcolor{currentfill}%
\pgfsetlinewidth{1.254687pt}%
\definecolor{currentstroke}{rgb}{0.000000,0.000000,0.000000}%
\pgfsetstrokecolor{currentstroke}%
\pgfsetdash{}{0pt}%
\pgfsys@defobject{currentmarker}{\pgfqpoint{0.000000in}{0.000000in}}{\pgfqpoint{0.055556in}{0.000000in}}{%
\pgfpathmoveto{\pgfqpoint{0.000000in}{0.000000in}}%
\pgfpathlineto{\pgfqpoint{0.055556in}{0.000000in}}%
\pgfusepath{stroke,fill}%
}%
\begin{pgfscope}%
\pgfsys@transformshift{0.697913in}{4.218017in}%
\pgfsys@useobject{currentmarker}{}%
\end{pgfscope}%
\end{pgfscope}%
\begin{pgfscope}%
\pgfsetbuttcap%
\pgfsetroundjoin%
\definecolor{currentfill}{rgb}{0.000000,0.000000,0.000000}%
\pgfsetfillcolor{currentfill}%
\pgfsetlinewidth{1.254687pt}%
\definecolor{currentstroke}{rgb}{0.000000,0.000000,0.000000}%
\pgfsetstrokecolor{currentstroke}%
\pgfsetdash{}{0pt}%
\pgfsys@defobject{currentmarker}{\pgfqpoint{0.000000in}{0.000000in}}{\pgfqpoint{0.055556in}{0.000000in}}{%
\pgfpathmoveto{\pgfqpoint{0.000000in}{0.000000in}}%
\pgfpathlineto{\pgfqpoint{0.055556in}{0.000000in}}%
\pgfusepath{stroke,fill}%
}%
\begin{pgfscope}%
\pgfsys@transformshift{0.697913in}{4.330945in}%
\pgfsys@useobject{currentmarker}{}%
\end{pgfscope}%
\end{pgfscope}%
\begin{pgfscope}%
\pgfsetbuttcap%
\pgfsetroundjoin%
\definecolor{currentfill}{rgb}{0.000000,0.000000,0.000000}%
\pgfsetfillcolor{currentfill}%
\pgfsetlinewidth{1.254687pt}%
\definecolor{currentstroke}{rgb}{0.000000,0.000000,0.000000}%
\pgfsetstrokecolor{currentstroke}%
\pgfsetdash{}{0pt}%
\pgfsys@defobject{currentmarker}{\pgfqpoint{0.000000in}{0.000000in}}{\pgfqpoint{0.055556in}{0.000000in}}{%
\pgfpathmoveto{\pgfqpoint{0.000000in}{0.000000in}}%
\pgfpathlineto{\pgfqpoint{0.055556in}{0.000000in}}%
\pgfusepath{stroke,fill}%
}%
\begin{pgfscope}%
\pgfsys@transformshift{0.697913in}{4.556800in}%
\pgfsys@useobject{currentmarker}{}%
\end{pgfscope}%
\end{pgfscope}%
\begin{pgfscope}%
\pgfsetbuttcap%
\pgfsetroundjoin%
\definecolor{currentfill}{rgb}{0.000000,0.000000,0.000000}%
\pgfsetfillcolor{currentfill}%
\pgfsetlinewidth{1.254687pt}%
\definecolor{currentstroke}{rgb}{0.000000,0.000000,0.000000}%
\pgfsetstrokecolor{currentstroke}%
\pgfsetdash{}{0pt}%
\pgfsys@defobject{currentmarker}{\pgfqpoint{0.000000in}{0.000000in}}{\pgfqpoint{0.055556in}{0.000000in}}{%
\pgfpathmoveto{\pgfqpoint{0.000000in}{0.000000in}}%
\pgfpathlineto{\pgfqpoint{0.055556in}{0.000000in}}%
\pgfusepath{stroke,fill}%
}%
\begin{pgfscope}%
\pgfsys@transformshift{0.697913in}{4.669728in}%
\pgfsys@useobject{currentmarker}{}%
\end{pgfscope}%
\end{pgfscope}%
\begin{pgfscope}%
\pgfsetbuttcap%
\pgfsetroundjoin%
\definecolor{currentfill}{rgb}{0.000000,0.000000,0.000000}%
\pgfsetfillcolor{currentfill}%
\pgfsetlinewidth{1.254687pt}%
\definecolor{currentstroke}{rgb}{0.000000,0.000000,0.000000}%
\pgfsetstrokecolor{currentstroke}%
\pgfsetdash{}{0pt}%
\pgfsys@defobject{currentmarker}{\pgfqpoint{0.000000in}{0.000000in}}{\pgfqpoint{0.055556in}{0.000000in}}{%
\pgfpathmoveto{\pgfqpoint{0.000000in}{0.000000in}}%
\pgfpathlineto{\pgfqpoint{0.055556in}{0.000000in}}%
\pgfusepath{stroke,fill}%
}%
\begin{pgfscope}%
\pgfsys@transformshift{0.697913in}{4.782656in}%
\pgfsys@useobject{currentmarker}{}%
\end{pgfscope}%
\end{pgfscope}%
\begin{pgfscope}%
\pgfsetbuttcap%
\pgfsetroundjoin%
\definecolor{currentfill}{rgb}{0.000000,0.000000,0.000000}%
\pgfsetfillcolor{currentfill}%
\pgfsetlinewidth{1.254687pt}%
\definecolor{currentstroke}{rgb}{0.000000,0.000000,0.000000}%
\pgfsetstrokecolor{currentstroke}%
\pgfsetdash{}{0pt}%
\pgfsys@defobject{currentmarker}{\pgfqpoint{0.000000in}{0.000000in}}{\pgfqpoint{0.055556in}{0.000000in}}{%
\pgfpathmoveto{\pgfqpoint{0.000000in}{0.000000in}}%
\pgfpathlineto{\pgfqpoint{0.055556in}{0.000000in}}%
\pgfusepath{stroke,fill}%
}%
\begin{pgfscope}%
\pgfsys@transformshift{0.697913in}{4.895584in}%
\pgfsys@useobject{currentmarker}{}%
\end{pgfscope}%
\end{pgfscope}%
\begin{pgfscope}%
\pgfsetbuttcap%
\pgfsetroundjoin%
\definecolor{currentfill}{rgb}{0.000000,0.000000,0.000000}%
\pgfsetfillcolor{currentfill}%
\pgfsetlinewidth{1.254687pt}%
\definecolor{currentstroke}{rgb}{0.000000,0.000000,0.000000}%
\pgfsetstrokecolor{currentstroke}%
\pgfsetdash{}{0pt}%
\pgfsys@defobject{currentmarker}{\pgfqpoint{0.000000in}{0.000000in}}{\pgfqpoint{0.055556in}{0.000000in}}{%
\pgfpathmoveto{\pgfqpoint{0.000000in}{0.000000in}}%
\pgfpathlineto{\pgfqpoint{0.055556in}{0.000000in}}%
\pgfusepath{stroke,fill}%
}%
\begin{pgfscope}%
\pgfsys@transformshift{0.697913in}{5.121439in}%
\pgfsys@useobject{currentmarker}{}%
\end{pgfscope}%
\end{pgfscope}%
\begin{pgfscope}%
\pgfsetbuttcap%
\pgfsetroundjoin%
\definecolor{currentfill}{rgb}{0.000000,0.000000,0.000000}%
\pgfsetfillcolor{currentfill}%
\pgfsetlinewidth{1.254687pt}%
\definecolor{currentstroke}{rgb}{0.000000,0.000000,0.000000}%
\pgfsetstrokecolor{currentstroke}%
\pgfsetdash{}{0pt}%
\pgfsys@defobject{currentmarker}{\pgfqpoint{0.000000in}{0.000000in}}{\pgfqpoint{0.055556in}{0.000000in}}{%
\pgfpathmoveto{\pgfqpoint{0.000000in}{0.000000in}}%
\pgfpathlineto{\pgfqpoint{0.055556in}{0.000000in}}%
\pgfusepath{stroke,fill}%
}%
\begin{pgfscope}%
\pgfsys@transformshift{0.697913in}{5.234367in}%
\pgfsys@useobject{currentmarker}{}%
\end{pgfscope}%
\end{pgfscope}%
\begin{pgfscope}%
\pgfsetbuttcap%
\pgfsetroundjoin%
\definecolor{currentfill}{rgb}{0.000000,0.000000,0.000000}%
\pgfsetfillcolor{currentfill}%
\pgfsetlinewidth{1.254687pt}%
\definecolor{currentstroke}{rgb}{0.000000,0.000000,0.000000}%
\pgfsetstrokecolor{currentstroke}%
\pgfsetdash{}{0pt}%
\pgfsys@defobject{currentmarker}{\pgfqpoint{0.000000in}{0.000000in}}{\pgfqpoint{0.055556in}{0.000000in}}{%
\pgfpathmoveto{\pgfqpoint{0.000000in}{0.000000in}}%
\pgfpathlineto{\pgfqpoint{0.055556in}{0.000000in}}%
\pgfusepath{stroke,fill}%
}%
\begin{pgfscope}%
\pgfsys@transformshift{0.697913in}{5.347295in}%
\pgfsys@useobject{currentmarker}{}%
\end{pgfscope}%
\end{pgfscope}%
\begin{pgfscope}%
\pgfsetbuttcap%
\pgfsetroundjoin%
\definecolor{currentfill}{rgb}{0.000000,0.000000,0.000000}%
\pgfsetfillcolor{currentfill}%
\pgfsetlinewidth{1.254687pt}%
\definecolor{currentstroke}{rgb}{0.000000,0.000000,0.000000}%
\pgfsetstrokecolor{currentstroke}%
\pgfsetdash{}{0pt}%
\pgfsys@defobject{currentmarker}{\pgfqpoint{0.000000in}{0.000000in}}{\pgfqpoint{0.055556in}{0.000000in}}{%
\pgfpathmoveto{\pgfqpoint{0.000000in}{0.000000in}}%
\pgfpathlineto{\pgfqpoint{0.055556in}{0.000000in}}%
\pgfusepath{stroke,fill}%
}%
\begin{pgfscope}%
\pgfsys@transformshift{0.697913in}{5.460222in}%
\pgfsys@useobject{currentmarker}{}%
\end{pgfscope}%
\end{pgfscope}%
\begin{pgfscope}%
\definecolor{textcolor}{rgb}{0.000000,0.000000,0.000000}%
\pgfsetstrokecolor{textcolor}%
\pgfsetfillcolor{textcolor}%
\pgftext[x=0.300000in,y=3.054861in,,bottom,rotate=90.000000]{\color{textcolor}{\rmfamily\fontsize{14.000000}{16.800000}\selectfont\catcode`\^=\active\def^{\ifmmode\sp\else\^{}\fi}\catcode`\%=\active\def%{\%}parts per million (ppm)}}%
\end{pgfscope}%
\begin{pgfscope}%
\pgfpathrectangle{\pgfqpoint{0.697913in}{0.559721in}}{\pgfqpoint{7.048636in}{4.990279in}}%
\pgfusepath{clip}%
\pgfsetrectcap%
\pgfsetroundjoin%
\pgfsetlinewidth{1.003750pt}%
\definecolor{currentstroke}{rgb}{1.000000,0.000000,0.000000}%
\pgfsetstrokecolor{currentstroke}%
\pgfsetdash{}{0pt}%
\pgfpathmoveto{\pgfqpoint{0.697913in}{0.800800in}}%
\pgfpathlineto{\pgfqpoint{0.712628in}{0.854764in}}%
\pgfpathlineto{\pgfqpoint{0.727343in}{0.933745in}}%
\pgfpathlineto{\pgfqpoint{0.742059in}{0.977716in}}%
\pgfpathlineto{\pgfqpoint{0.756774in}{1.015272in}}%
\pgfpathlineto{\pgfqpoint{0.771489in}{0.976371in}}%
\pgfpathlineto{\pgfqpoint{0.786205in}{0.886780in}}%
\pgfpathlineto{\pgfqpoint{0.800920in}{0.775096in}}%
\pgfpathlineto{\pgfqpoint{0.815635in}{0.687494in}}%
\pgfpathlineto{\pgfqpoint{0.830350in}{0.682107in}}%
\pgfpathlineto{\pgfqpoint{0.859781in}{0.838848in}}%
\pgfpathlineto{\pgfqpoint{0.874496in}{0.875106in}}%
\pgfpathlineto{\pgfqpoint{0.889212in}{0.913637in}}%
\pgfpathlineto{\pgfqpoint{0.903927in}{0.955498in}}%
\pgfpathlineto{\pgfqpoint{0.918642in}{1.048312in}}%
\pgfpathlineto{\pgfqpoint{0.933358in}{1.085369in}}%
\pgfpathlineto{\pgfqpoint{0.948073in}{1.055640in}}%
\pgfpathlineto{\pgfqpoint{0.962788in}{0.959881in}}%
\pgfpathlineto{\pgfqpoint{0.977504in}{0.841100in}}%
\pgfpathlineto{\pgfqpoint{0.992219in}{0.795475in}}%
\pgfpathlineto{\pgfqpoint{1.006934in}{0.760407in}}%
\pgfpathlineto{\pgfqpoint{1.021650in}{0.844761in}}%
\pgfpathlineto{\pgfqpoint{1.036365in}{0.909468in}}%
\pgfpathlineto{\pgfqpoint{1.051080in}{0.982320in}}%
\pgfpathlineto{\pgfqpoint{1.065796in}{0.994584in}}%
\pgfpathlineto{\pgfqpoint{1.080511in}{1.044934in}}%
\pgfpathlineto{\pgfqpoint{1.095226in}{1.136847in}}%
\pgfpathlineto{\pgfqpoint{1.109941in}{1.177436in}}%
\pgfpathlineto{\pgfqpoint{1.124657in}{1.150975in}}%
\pgfpathlineto{\pgfqpoint{1.139372in}{1.075536in}}%
\pgfpathlineto{\pgfqpoint{1.154087in}{0.954253in}}%
\pgfpathlineto{\pgfqpoint{1.168803in}{0.864921in}}%
\pgfpathlineto{\pgfqpoint{1.183518in}{0.873943in}}%
\pgfpathlineto{\pgfqpoint{1.198233in}{0.947407in}}%
\pgfpathlineto{\pgfqpoint{1.212949in}{1.016117in}}%
\pgfpathlineto{\pgfqpoint{1.227664in}{1.081354in}}%
\pgfpathlineto{\pgfqpoint{1.242379in}{1.154139in}}%
\pgfpathlineto{\pgfqpoint{1.257095in}{1.194737in}}%
\pgfpathlineto{\pgfqpoint{1.271810in}{1.262801in}}%
\pgfpathlineto{\pgfqpoint{1.286525in}{1.301798in}}%
\pgfpathlineto{\pgfqpoint{1.301241in}{1.271334in}}%
\pgfpathlineto{\pgfqpoint{1.315956in}{1.212318in}}%
\pgfpathlineto{\pgfqpoint{1.330671in}{1.080731in}}%
\pgfpathlineto{\pgfqpoint{1.345387in}{0.998402in}}%
\pgfpathlineto{\pgfqpoint{1.360102in}{1.021379in}}%
\pgfpathlineto{\pgfqpoint{1.374817in}{1.082667in}}%
\pgfpathlineto{\pgfqpoint{1.389532in}{1.146865in}}%
\pgfpathlineto{\pgfqpoint{1.404248in}{1.228919in}}%
\pgfpathlineto{\pgfqpoint{1.418963in}{1.249078in}}%
\pgfpathlineto{\pgfqpoint{1.433678in}{1.285964in}}%
\pgfpathlineto{\pgfqpoint{1.448394in}{1.379033in}}%
\pgfpathlineto{\pgfqpoint{1.463109in}{1.392104in}}%
\pgfpathlineto{\pgfqpoint{1.477824in}{1.362008in}}%
\pgfpathlineto{\pgfqpoint{1.492540in}{1.284978in}}%
\pgfpathlineto{\pgfqpoint{1.507255in}{1.162372in}}%
\pgfpathlineto{\pgfqpoint{1.521970in}{1.062532in}}%
\pgfpathlineto{\pgfqpoint{1.536686in}{1.076569in}}%
\pgfpathlineto{\pgfqpoint{1.551401in}{1.145336in}}%
\pgfpathlineto{\pgfqpoint{1.566116in}{1.217965in}}%
\pgfpathlineto{\pgfqpoint{1.580832in}{1.274509in}}%
\pgfpathlineto{\pgfqpoint{1.595547in}{1.343743in}}%
\pgfpathlineto{\pgfqpoint{1.610262in}{1.381731in}}%
\pgfpathlineto{\pgfqpoint{1.624978in}{1.415855in}}%
\pgfpathlineto{\pgfqpoint{1.639693in}{1.472501in}}%
\pgfpathlineto{\pgfqpoint{1.654408in}{1.417875in}}%
\pgfpathlineto{\pgfqpoint{1.669124in}{1.332844in}}%
\pgfpathlineto{\pgfqpoint{1.683839in}{1.227407in}}%
\pgfpathlineto{\pgfqpoint{1.698554in}{1.132997in}}%
\pgfpathlineto{\pgfqpoint{1.713269in}{1.152108in}}%
\pgfpathlineto{\pgfqpoint{1.727985in}{1.230512in}}%
\pgfpathlineto{\pgfqpoint{1.742700in}{1.306255in}}%
\pgfpathlineto{\pgfqpoint{1.757415in}{1.334366in}}%
\pgfpathlineto{\pgfqpoint{1.772131in}{1.384833in}}%
\pgfpathlineto{\pgfqpoint{1.786846in}{1.470691in}}%
\pgfpathlineto{\pgfqpoint{1.801561in}{1.551077in}}%
\pgfpathlineto{\pgfqpoint{1.816277in}{1.577512in}}%
\pgfpathlineto{\pgfqpoint{1.830992in}{1.520951in}}%
\pgfpathlineto{\pgfqpoint{1.845707in}{1.412191in}}%
\pgfpathlineto{\pgfqpoint{1.860423in}{1.272178in}}%
\pgfpathlineto{\pgfqpoint{1.875138in}{1.187642in}}%
\pgfpathlineto{\pgfqpoint{1.889853in}{1.192263in}}%
\pgfpathlineto{\pgfqpoint{1.904569in}{1.275825in}}%
\pgfpathlineto{\pgfqpoint{1.919284in}{1.351213in}}%
\pgfpathlineto{\pgfqpoint{1.933999in}{1.414220in}}%
\pgfpathlineto{\pgfqpoint{1.948715in}{1.463436in}}%
\pgfpathlineto{\pgfqpoint{1.963430in}{1.504903in}}%
\pgfpathlineto{\pgfqpoint{1.978145in}{1.576992in}}%
\pgfpathlineto{\pgfqpoint{1.992860in}{1.604477in}}%
\pgfpathlineto{\pgfqpoint{2.007576in}{1.588776in}}%
\pgfpathlineto{\pgfqpoint{2.022291in}{1.459711in}}%
\pgfpathlineto{\pgfqpoint{2.051722in}{1.224934in}}%
\pgfpathlineto{\pgfqpoint{2.066437in}{1.249769in}}%
\pgfpathlineto{\pgfqpoint{2.081152in}{1.304480in}}%
\pgfpathlineto{\pgfqpoint{2.095868in}{1.376535in}}%
\pgfpathlineto{\pgfqpoint{2.110583in}{1.457719in}}%
\pgfpathlineto{\pgfqpoint{2.125298in}{1.474739in}}%
\pgfpathlineto{\pgfqpoint{2.140014in}{1.541798in}}%
\pgfpathlineto{\pgfqpoint{2.154729in}{1.586777in}}%
\pgfpathlineto{\pgfqpoint{2.169444in}{1.637899in}}%
\pgfpathlineto{\pgfqpoint{2.184160in}{1.600492in}}%
\pgfpathlineto{\pgfqpoint{2.198875in}{1.479179in}}%
\pgfpathlineto{\pgfqpoint{2.213590in}{1.382317in}}%
\pgfpathlineto{\pgfqpoint{2.228306in}{1.288327in}}%
\pgfpathlineto{\pgfqpoint{2.243021in}{1.296898in}}%
\pgfpathlineto{\pgfqpoint{2.257736in}{1.367934in}}%
\pgfpathlineto{\pgfqpoint{2.272451in}{1.453039in}}%
\pgfpathlineto{\pgfqpoint{2.287167in}{1.525275in}}%
\pgfpathlineto{\pgfqpoint{2.301882in}{1.567715in}}%
\pgfpathlineto{\pgfqpoint{2.316597in}{1.626149in}}%
\pgfpathlineto{\pgfqpoint{2.331313in}{1.699022in}}%
\pgfpathlineto{\pgfqpoint{2.346028in}{1.721353in}}%
\pgfpathlineto{\pgfqpoint{2.360743in}{1.675260in}}%
\pgfpathlineto{\pgfqpoint{2.375459in}{1.593211in}}%
\pgfpathlineto{\pgfqpoint{2.390174in}{1.487188in}}%
\pgfpathlineto{\pgfqpoint{2.404889in}{1.387600in}}%
\pgfpathlineto{\pgfqpoint{2.419605in}{1.404099in}}%
\pgfpathlineto{\pgfqpoint{2.434320in}{1.489077in}}%
\pgfpathlineto{\pgfqpoint{2.449035in}{1.568045in}}%
\pgfpathlineto{\pgfqpoint{2.463751in}{1.623407in}}%
\pgfpathlineto{\pgfqpoint{2.493181in}{1.731250in}}%
\pgfpathlineto{\pgfqpoint{2.507897in}{1.813823in}}%
\pgfpathlineto{\pgfqpoint{2.522612in}{1.836876in}}%
\pgfpathlineto{\pgfqpoint{2.537327in}{1.809246in}}%
\pgfpathlineto{\pgfqpoint{2.552042in}{1.717090in}}%
\pgfpathlineto{\pgfqpoint{2.566758in}{1.583482in}}%
\pgfpathlineto{\pgfqpoint{2.581473in}{1.523889in}}%
\pgfpathlineto{\pgfqpoint{2.596188in}{1.515421in}}%
\pgfpathlineto{\pgfqpoint{2.610904in}{1.599895in}}%
\pgfpathlineto{\pgfqpoint{2.625619in}{1.668202in}}%
\pgfpathlineto{\pgfqpoint{2.640334in}{1.744451in}}%
\pgfpathlineto{\pgfqpoint{2.655050in}{1.811203in}}%
\pgfpathlineto{\pgfqpoint{2.669765in}{1.861214in}}%
\pgfpathlineto{\pgfqpoint{2.684480in}{1.886467in}}%
\pgfpathlineto{\pgfqpoint{2.699196in}{1.917320in}}%
\pgfpathlineto{\pgfqpoint{2.713911in}{1.906216in}}%
\pgfpathlineto{\pgfqpoint{2.728626in}{1.828829in}}%
\pgfpathlineto{\pgfqpoint{2.743342in}{1.708741in}}%
\pgfpathlineto{\pgfqpoint{2.758057in}{1.605004in}}%
\pgfpathlineto{\pgfqpoint{2.772772in}{1.606250in}}%
\pgfpathlineto{\pgfqpoint{2.787488in}{1.680121in}}%
\pgfpathlineto{\pgfqpoint{2.802203in}{1.757359in}}%
\pgfpathlineto{\pgfqpoint{2.816918in}{1.803530in}}%
\pgfpathlineto{\pgfqpoint{2.831633in}{1.858169in}}%
\pgfpathlineto{\pgfqpoint{2.846349in}{1.882998in}}%
\pgfpathlineto{\pgfqpoint{2.861064in}{1.988554in}}%
\pgfpathlineto{\pgfqpoint{2.875779in}{2.003172in}}%
\pgfpathlineto{\pgfqpoint{2.890495in}{1.940849in}}%
\pgfpathlineto{\pgfqpoint{2.905210in}{1.874601in}}%
\pgfpathlineto{\pgfqpoint{2.919925in}{1.757569in}}%
\pgfpathlineto{\pgfqpoint{2.934641in}{1.646651in}}%
\pgfpathlineto{\pgfqpoint{2.949356in}{1.676619in}}%
\pgfpathlineto{\pgfqpoint{2.964071in}{1.770213in}}%
\pgfpathlineto{\pgfqpoint{2.978787in}{1.875706in}}%
\pgfpathlineto{\pgfqpoint{2.993502in}{1.924661in}}%
\pgfpathlineto{\pgfqpoint{3.008217in}{1.965370in}}%
\pgfpathlineto{\pgfqpoint{3.022933in}{2.036452in}}%
\pgfpathlineto{\pgfqpoint{3.037648in}{2.116879in}}%
\pgfpathlineto{\pgfqpoint{3.052363in}{2.160322in}}%
\pgfpathlineto{\pgfqpoint{3.067079in}{2.135205in}}%
\pgfpathlineto{\pgfqpoint{3.081794in}{2.071972in}}%
\pgfpathlineto{\pgfqpoint{3.096509in}{1.966858in}}%
\pgfpathlineto{\pgfqpoint{3.111224in}{1.856132in}}%
\pgfpathlineto{\pgfqpoint{3.125940in}{1.876873in}}%
\pgfpathlineto{\pgfqpoint{3.140655in}{1.939208in}}%
\pgfpathlineto{\pgfqpoint{3.155370in}{2.035861in}}%
\pgfpathlineto{\pgfqpoint{3.170086in}{2.090478in}}%
\pgfpathlineto{\pgfqpoint{3.184801in}{2.140813in}}%
\pgfpathlineto{\pgfqpoint{3.199516in}{2.177751in}}%
\pgfpathlineto{\pgfqpoint{3.214232in}{2.251224in}}%
\pgfpathlineto{\pgfqpoint{3.228947in}{2.248600in}}%
\pgfpathlineto{\pgfqpoint{3.243662in}{2.213266in}}%
\pgfpathlineto{\pgfqpoint{3.258378in}{2.160034in}}%
\pgfpathlineto{\pgfqpoint{3.273093in}{2.018952in}}%
\pgfpathlineto{\pgfqpoint{3.287808in}{1.899710in}}%
\pgfpathlineto{\pgfqpoint{3.302524in}{1.932267in}}%
\pgfpathlineto{\pgfqpoint{3.317239in}{2.008121in}}%
\pgfpathlineto{\pgfqpoint{3.331954in}{2.087569in}}%
\pgfpathlineto{\pgfqpoint{3.346670in}{2.152937in}}%
\pgfpathlineto{\pgfqpoint{3.361385in}{2.171914in}}%
\pgfpathlineto{\pgfqpoint{3.376100in}{2.227382in}}%
\pgfpathlineto{\pgfqpoint{3.390815in}{2.296237in}}%
\pgfpathlineto{\pgfqpoint{3.405531in}{2.284008in}}%
\pgfpathlineto{\pgfqpoint{3.420246in}{2.290340in}}%
\pgfpathlineto{\pgfqpoint{3.434961in}{2.189405in}}%
\pgfpathlineto{\pgfqpoint{3.449677in}{2.092988in}}%
\pgfpathlineto{\pgfqpoint{3.464392in}{2.024112in}}%
\pgfpathlineto{\pgfqpoint{3.479107in}{2.025919in}}%
\pgfpathlineto{\pgfqpoint{3.493823in}{2.102225in}}%
\pgfpathlineto{\pgfqpoint{3.508538in}{2.176941in}}%
\pgfpathlineto{\pgfqpoint{3.523253in}{2.227966in}}%
\pgfpathlineto{\pgfqpoint{3.537969in}{2.281327in}}%
\pgfpathlineto{\pgfqpoint{3.552684in}{2.332186in}}%
\pgfpathlineto{\pgfqpoint{3.567399in}{2.385082in}}%
\pgfpathlineto{\pgfqpoint{3.582115in}{2.412631in}}%
\pgfpathlineto{\pgfqpoint{3.596830in}{2.374666in}}%
\pgfpathlineto{\pgfqpoint{3.611545in}{2.282164in}}%
\pgfpathlineto{\pgfqpoint{3.626261in}{2.166161in}}%
\pgfpathlineto{\pgfqpoint{3.640976in}{2.092151in}}%
\pgfpathlineto{\pgfqpoint{3.655691in}{2.106868in}}%
\pgfpathlineto{\pgfqpoint{3.670406in}{2.183364in}}%
\pgfpathlineto{\pgfqpoint{3.685122in}{2.264853in}}%
\pgfpathlineto{\pgfqpoint{3.699837in}{2.336196in}}%
\pgfpathlineto{\pgfqpoint{3.714552in}{2.375640in}}%
\pgfpathlineto{\pgfqpoint{3.729268in}{2.424982in}}%
\pgfpathlineto{\pgfqpoint{3.743983in}{2.478150in}}%
\pgfpathlineto{\pgfqpoint{3.758698in}{2.520284in}}%
\pgfpathlineto{\pgfqpoint{3.773414in}{2.507524in}}%
\pgfpathlineto{\pgfqpoint{3.788129in}{2.424904in}}%
\pgfpathlineto{\pgfqpoint{3.802844in}{2.299576in}}%
\pgfpathlineto{\pgfqpoint{3.817560in}{2.236134in}}%
\pgfpathlineto{\pgfqpoint{3.832275in}{2.226608in}}%
\pgfpathlineto{\pgfqpoint{3.846990in}{2.321823in}}%
\pgfpathlineto{\pgfqpoint{3.861706in}{2.409897in}}%
\pgfpathlineto{\pgfqpoint{3.876421in}{2.471099in}}%
\pgfpathlineto{\pgfqpoint{3.891136in}{2.513560in}}%
\pgfpathlineto{\pgfqpoint{3.905852in}{2.559572in}}%
\pgfpathlineto{\pgfqpoint{3.920567in}{2.632302in}}%
\pgfpathlineto{\pgfqpoint{3.935282in}{2.681032in}}%
\pgfpathlineto{\pgfqpoint{3.949997in}{2.660969in}}%
\pgfpathlineto{\pgfqpoint{3.964713in}{2.572562in}}%
\pgfpathlineto{\pgfqpoint{3.979428in}{2.453201in}}%
\pgfpathlineto{\pgfqpoint{3.994143in}{2.372235in}}%
\pgfpathlineto{\pgfqpoint{4.008859in}{2.372099in}}%
\pgfpathlineto{\pgfqpoint{4.023574in}{2.463749in}}%
\pgfpathlineto{\pgfqpoint{4.038289in}{2.530729in}}%
\pgfpathlineto{\pgfqpoint{4.053005in}{2.589786in}}%
\pgfpathlineto{\pgfqpoint{4.067720in}{2.640163in}}%
\pgfpathlineto{\pgfqpoint{4.082435in}{2.697827in}}%
\pgfpathlineto{\pgfqpoint{4.097151in}{2.780815in}}%
\pgfpathlineto{\pgfqpoint{4.111866in}{2.795046in}}%
\pgfpathlineto{\pgfqpoint{4.126581in}{2.746756in}}%
\pgfpathlineto{\pgfqpoint{4.141297in}{2.614488in}}%
\pgfpathlineto{\pgfqpoint{4.156012in}{2.546756in}}%
\pgfpathlineto{\pgfqpoint{4.170727in}{2.438426in}}%
\pgfpathlineto{\pgfqpoint{4.185443in}{2.447835in}}%
\pgfpathlineto{\pgfqpoint{4.200158in}{2.543726in}}%
\pgfpathlineto{\pgfqpoint{4.214873in}{2.618263in}}%
\pgfpathlineto{\pgfqpoint{4.229589in}{2.672363in}}%
\pgfpathlineto{\pgfqpoint{4.244304in}{2.745674in}}%
\pgfpathlineto{\pgfqpoint{4.259019in}{2.806202in}}%
\pgfpathlineto{\pgfqpoint{4.273734in}{2.889964in}}%
\pgfpathlineto{\pgfqpoint{4.288450in}{2.898893in}}%
\pgfpathlineto{\pgfqpoint{4.303165in}{2.886417in}}%
\pgfpathlineto{\pgfqpoint{4.317880in}{2.800754in}}%
\pgfpathlineto{\pgfqpoint{4.332596in}{2.692167in}}%
\pgfpathlineto{\pgfqpoint{4.347311in}{2.573537in}}%
\pgfpathlineto{\pgfqpoint{4.362026in}{2.581048in}}%
\pgfpathlineto{\pgfqpoint{4.376742in}{2.669507in}}%
\pgfpathlineto{\pgfqpoint{4.391457in}{2.765983in}}%
\pgfpathlineto{\pgfqpoint{4.406172in}{2.841474in}}%
\pgfpathlineto{\pgfqpoint{4.420888in}{2.886395in}}%
\pgfpathlineto{\pgfqpoint{4.435603in}{2.911579in}}%
\pgfpathlineto{\pgfqpoint{4.450318in}{3.025251in}}%
\pgfpathlineto{\pgfqpoint{4.465034in}{3.044627in}}%
\pgfpathlineto{\pgfqpoint{4.479749in}{2.989403in}}%
\pgfpathlineto{\pgfqpoint{4.494464in}{2.898926in}}%
\pgfpathlineto{\pgfqpoint{4.509180in}{2.782098in}}%
\pgfpathlineto{\pgfqpoint{4.523895in}{2.693424in}}%
\pgfpathlineto{\pgfqpoint{4.538610in}{2.711437in}}%
\pgfpathlineto{\pgfqpoint{4.553325in}{2.768121in}}%
\pgfpathlineto{\pgfqpoint{4.568041in}{2.861524in}}%
\pgfpathlineto{\pgfqpoint{4.582756in}{2.925864in}}%
\pgfpathlineto{\pgfqpoint{4.597471in}{2.981027in}}%
\pgfpathlineto{\pgfqpoint{4.612187in}{3.019895in}}%
\pgfpathlineto{\pgfqpoint{4.626902in}{3.130071in}}%
\pgfpathlineto{\pgfqpoint{4.641617in}{3.132820in}}%
\pgfpathlineto{\pgfqpoint{4.656333in}{3.109392in}}%
\pgfpathlineto{\pgfqpoint{4.671048in}{3.021191in}}%
\pgfpathlineto{\pgfqpoint{4.685763in}{2.879566in}}%
\pgfpathlineto{\pgfqpoint{4.700479in}{2.815185in}}%
\pgfpathlineto{\pgfqpoint{4.715194in}{2.827623in}}%
\pgfpathlineto{\pgfqpoint{4.729909in}{2.902540in}}%
\pgfpathlineto{\pgfqpoint{4.759340in}{3.076353in}}%
\pgfpathlineto{\pgfqpoint{4.774055in}{3.091714in}}%
\pgfpathlineto{\pgfqpoint{4.788771in}{3.105264in}}%
\pgfpathlineto{\pgfqpoint{4.803486in}{3.168330in}}%
\pgfpathlineto{\pgfqpoint{4.818201in}{3.245822in}}%
\pgfpathlineto{\pgfqpoint{4.832916in}{3.207878in}}%
\pgfpathlineto{\pgfqpoint{4.847632in}{3.081794in}}%
\pgfpathlineto{\pgfqpoint{4.862347in}{2.994873in}}%
\pgfpathlineto{\pgfqpoint{4.877062in}{2.942665in}}%
\pgfpathlineto{\pgfqpoint{4.891778in}{2.927964in}}%
\pgfpathlineto{\pgfqpoint{4.906493in}{2.999466in}}%
\pgfpathlineto{\pgfqpoint{4.921208in}{3.074662in}}%
\pgfpathlineto{\pgfqpoint{4.935924in}{3.154312in}}%
\pgfpathlineto{\pgfqpoint{4.950639in}{3.185292in}}%
\pgfpathlineto{\pgfqpoint{4.965354in}{3.260631in}}%
\pgfpathlineto{\pgfqpoint{4.980070in}{3.301160in}}%
\pgfpathlineto{\pgfqpoint{4.994785in}{3.335016in}}%
\pgfpathlineto{\pgfqpoint{5.009500in}{3.298668in}}%
\pgfpathlineto{\pgfqpoint{5.024216in}{3.215500in}}%
\pgfpathlineto{\pgfqpoint{5.038931in}{3.104630in}}%
\pgfpathlineto{\pgfqpoint{5.053646in}{3.028827in}}%
\pgfpathlineto{\pgfqpoint{5.068362in}{3.012287in}}%
\pgfpathlineto{\pgfqpoint{5.083077in}{3.101575in}}%
\pgfpathlineto{\pgfqpoint{5.097792in}{3.178545in}}%
\pgfpathlineto{\pgfqpoint{5.112507in}{3.253105in}}%
\pgfpathlineto{\pgfqpoint{5.127223in}{3.337095in}}%
\pgfpathlineto{\pgfqpoint{5.141938in}{3.392402in}}%
\pgfpathlineto{\pgfqpoint{5.156653in}{3.465701in}}%
\pgfpathlineto{\pgfqpoint{5.171369in}{3.495727in}}%
\pgfpathlineto{\pgfqpoint{5.186084in}{3.450047in}}%
\pgfpathlineto{\pgfqpoint{5.200799in}{3.338679in}}%
\pgfpathlineto{\pgfqpoint{5.215515in}{3.233330in}}%
\pgfpathlineto{\pgfqpoint{5.230230in}{3.146527in}}%
\pgfpathlineto{\pgfqpoint{5.244945in}{3.169610in}}%
\pgfpathlineto{\pgfqpoint{5.259661in}{3.249291in}}%
\pgfpathlineto{\pgfqpoint{5.274376in}{3.313777in}}%
\pgfpathlineto{\pgfqpoint{5.289091in}{3.399271in}}%
\pgfpathlineto{\pgfqpoint{5.303807in}{3.430567in}}%
\pgfpathlineto{\pgfqpoint{5.318522in}{3.471233in}}%
\pgfpathlineto{\pgfqpoint{5.333237in}{3.507600in}}%
\pgfpathlineto{\pgfqpoint{5.347953in}{3.563484in}}%
\pgfpathlineto{\pgfqpoint{5.362668in}{3.535711in}}%
\pgfpathlineto{\pgfqpoint{5.377383in}{3.468350in}}%
\pgfpathlineto{\pgfqpoint{5.392098in}{3.336385in}}%
\pgfpathlineto{\pgfqpoint{5.406814in}{3.272073in}}%
\pgfpathlineto{\pgfqpoint{5.421529in}{3.269405in}}%
\pgfpathlineto{\pgfqpoint{5.436244in}{3.342383in}}%
\pgfpathlineto{\pgfqpoint{5.450960in}{3.429823in}}%
\pgfpathlineto{\pgfqpoint{5.465675in}{3.501866in}}%
\pgfpathlineto{\pgfqpoint{5.480390in}{3.543338in}}%
\pgfpathlineto{\pgfqpoint{5.495106in}{3.573237in}}%
\pgfpathlineto{\pgfqpoint{5.509821in}{3.675126in}}%
\pgfpathlineto{\pgfqpoint{5.524536in}{3.705927in}}%
\pgfpathlineto{\pgfqpoint{5.539252in}{3.648982in}}%
\pgfpathlineto{\pgfqpoint{5.553967in}{3.572721in}}%
\pgfpathlineto{\pgfqpoint{5.568682in}{3.459462in}}%
\pgfpathlineto{\pgfqpoint{5.583398in}{3.388437in}}%
\pgfpathlineto{\pgfqpoint{5.598113in}{3.386364in}}%
\pgfpathlineto{\pgfqpoint{5.612828in}{3.496506in}}%
\pgfpathlineto{\pgfqpoint{5.627544in}{3.572323in}}%
\pgfpathlineto{\pgfqpoint{5.656974in}{3.711491in}}%
\pgfpathlineto{\pgfqpoint{5.671689in}{3.747861in}}%
\pgfpathlineto{\pgfqpoint{5.686405in}{3.801211in}}%
\pgfpathlineto{\pgfqpoint{5.701120in}{3.880383in}}%
\pgfpathlineto{\pgfqpoint{5.715835in}{3.814365in}}%
\pgfpathlineto{\pgfqpoint{5.730551in}{3.735358in}}%
\pgfpathlineto{\pgfqpoint{5.745266in}{3.619144in}}%
\pgfpathlineto{\pgfqpoint{5.759981in}{3.525009in}}%
\pgfpathlineto{\pgfqpoint{5.774697in}{3.535046in}}%
\pgfpathlineto{\pgfqpoint{5.789412in}{3.617279in}}%
\pgfpathlineto{\pgfqpoint{5.804127in}{3.711197in}}%
\pgfpathlineto{\pgfqpoint{5.818843in}{3.768746in}}%
\pgfpathlineto{\pgfqpoint{5.833558in}{3.781488in}}%
\pgfpathlineto{\pgfqpoint{5.848273in}{3.873330in}}%
\pgfpathlineto{\pgfqpoint{5.862989in}{3.963302in}}%
\pgfpathlineto{\pgfqpoint{5.877704in}{3.989235in}}%
\pgfpathlineto{\pgfqpoint{5.892419in}{3.960763in}}%
\pgfpathlineto{\pgfqpoint{5.907135in}{3.843413in}}%
\pgfpathlineto{\pgfqpoint{5.921850in}{3.718927in}}%
\pgfpathlineto{\pgfqpoint{5.936565in}{3.625038in}}%
\pgfpathlineto{\pgfqpoint{5.951280in}{3.660836in}}%
\pgfpathlineto{\pgfqpoint{5.965996in}{3.731089in}}%
\pgfpathlineto{\pgfqpoint{5.980711in}{3.828319in}}%
\pgfpathlineto{\pgfqpoint{5.995426in}{3.889209in}}%
\pgfpathlineto{\pgfqpoint{6.010142in}{3.910510in}}%
\pgfpathlineto{\pgfqpoint{6.024857in}{3.979245in}}%
\pgfpathlineto{\pgfqpoint{6.039572in}{4.069973in}}%
\pgfpathlineto{\pgfqpoint{6.054288in}{4.113220in}}%
\pgfpathlineto{\pgfqpoint{6.069003in}{4.045685in}}%
\pgfpathlineto{\pgfqpoint{6.083718in}{3.962188in}}%
\pgfpathlineto{\pgfqpoint{6.098434in}{3.827146in}}%
\pgfpathlineto{\pgfqpoint{6.113149in}{3.756187in}}%
\pgfpathlineto{\pgfqpoint{6.127864in}{3.794558in}}%
\pgfpathlineto{\pgfqpoint{6.157295in}{3.995888in}}%
\pgfpathlineto{\pgfqpoint{6.172010in}{4.034614in}}%
\pgfpathlineto{\pgfqpoint{6.186726in}{4.121554in}}%
\pgfpathlineto{\pgfqpoint{6.201441in}{4.164638in}}%
\pgfpathlineto{\pgfqpoint{6.216156in}{4.309818in}}%
\pgfpathlineto{\pgfqpoint{6.230871in}{4.325629in}}%
\pgfpathlineto{\pgfqpoint{6.245587in}{4.274806in}}%
\pgfpathlineto{\pgfqpoint{6.260302in}{4.134470in}}%
\pgfpathlineto{\pgfqpoint{6.275017in}{4.016864in}}%
\pgfpathlineto{\pgfqpoint{6.289733in}{3.948910in}}%
\pgfpathlineto{\pgfqpoint{6.304448in}{3.980245in}}%
\pgfpathlineto{\pgfqpoint{6.319163in}{4.089049in}}%
\pgfpathlineto{\pgfqpoint{6.333879in}{4.142277in}}%
\pgfpathlineto{\pgfqpoint{6.348594in}{4.236713in}}%
\pgfpathlineto{\pgfqpoint{6.363309in}{4.254219in}}%
\pgfpathlineto{\pgfqpoint{6.378025in}{4.307159in}}%
\pgfpathlineto{\pgfqpoint{6.392740in}{4.400504in}}%
\pgfpathlineto{\pgfqpoint{6.407455in}{4.437431in}}%
\pgfpathlineto{\pgfqpoint{6.422171in}{4.392230in}}%
\pgfpathlineto{\pgfqpoint{6.436886in}{4.292025in}}%
\pgfpathlineto{\pgfqpoint{6.451601in}{4.177460in}}%
\pgfpathlineto{\pgfqpoint{6.466317in}{4.081765in}}%
\pgfpathlineto{\pgfqpoint{6.481032in}{4.094528in}}%
\pgfpathlineto{\pgfqpoint{6.495747in}{4.175039in}}%
\pgfpathlineto{\pgfqpoint{6.510463in}{4.274353in}}%
\pgfpathlineto{\pgfqpoint{6.525178in}{4.339667in}}%
\pgfpathlineto{\pgfqpoint{6.539893in}{4.360346in}}%
\pgfpathlineto{\pgfqpoint{6.554608in}{4.419924in}}%
\pgfpathlineto{\pgfqpoint{6.569324in}{4.468421in}}%
\pgfpathlineto{\pgfqpoint{6.584039in}{4.524851in}}%
\pgfpathlineto{\pgfqpoint{6.598754in}{4.500336in}}%
\pgfpathlineto{\pgfqpoint{6.613470in}{4.379127in}}%
\pgfpathlineto{\pgfqpoint{6.628185in}{4.283351in}}%
\pgfpathlineto{\pgfqpoint{6.642900in}{4.202149in}}%
\pgfpathlineto{\pgfqpoint{6.657616in}{4.228745in}}%
\pgfpathlineto{\pgfqpoint{6.672331in}{4.346402in}}%
\pgfpathlineto{\pgfqpoint{6.687046in}{4.403463in}}%
\pgfpathlineto{\pgfqpoint{6.701762in}{4.502812in}}%
\pgfpathlineto{\pgfqpoint{6.716477in}{4.554563in}}%
\pgfpathlineto{\pgfqpoint{6.731192in}{4.564685in}}%
\pgfpathlineto{\pgfqpoint{6.745908in}{4.644779in}}%
\pgfpathlineto{\pgfqpoint{6.760623in}{4.718489in}}%
\pgfpathlineto{\pgfqpoint{6.775338in}{4.677779in}}%
\pgfpathlineto{\pgfqpoint{6.790054in}{4.557252in}}%
\pgfpathlineto{\pgfqpoint{6.804769in}{4.455691in}}%
\pgfpathlineto{\pgfqpoint{6.819484in}{4.373935in}}%
\pgfpathlineto{\pgfqpoint{6.834199in}{4.371754in}}%
\pgfpathlineto{\pgfqpoint{6.848915in}{4.472495in}}%
\pgfpathlineto{\pgfqpoint{6.863630in}{4.554833in}}%
\pgfpathlineto{\pgfqpoint{6.878345in}{4.647259in}}%
\pgfpathlineto{\pgfqpoint{6.893061in}{4.688422in}}%
\pgfpathlineto{\pgfqpoint{6.907776in}{4.708840in}}%
\pgfpathlineto{\pgfqpoint{6.922491in}{4.806492in}}%
\pgfpathlineto{\pgfqpoint{6.937207in}{4.855076in}}%
\pgfpathlineto{\pgfqpoint{6.951922in}{4.812916in}}%
\pgfpathlineto{\pgfqpoint{6.981353in}{4.594315in}}%
\pgfpathlineto{\pgfqpoint{6.996068in}{4.528666in}}%
\pgfpathlineto{\pgfqpoint{7.010783in}{4.527288in}}%
\pgfpathlineto{\pgfqpoint{7.025499in}{4.618743in}}%
\pgfpathlineto{\pgfqpoint{7.040214in}{4.681999in}}%
\pgfpathlineto{\pgfqpoint{7.054929in}{4.753762in}}%
\pgfpathlineto{\pgfqpoint{7.069645in}{4.823290in}}%
\pgfpathlineto{\pgfqpoint{7.084360in}{4.871264in}}%
\pgfpathlineto{\pgfqpoint{7.099075in}{4.952236in}}%
\pgfpathlineto{\pgfqpoint{7.113790in}{4.956988in}}%
\pgfpathlineto{\pgfqpoint{7.128506in}{4.948991in}}%
\pgfpathlineto{\pgfqpoint{7.143221in}{4.833328in}}%
\pgfpathlineto{\pgfqpoint{7.157936in}{4.699157in}}%
\pgfpathlineto{\pgfqpoint{7.172652in}{4.628028in}}%
\pgfpathlineto{\pgfqpoint{7.187367in}{4.662816in}}%
\pgfpathlineto{\pgfqpoint{7.202082in}{4.724724in}}%
\pgfpathlineto{\pgfqpoint{7.216798in}{4.821273in}}%
\pgfpathlineto{\pgfqpoint{7.231513in}{4.902077in}}%
\pgfpathlineto{\pgfqpoint{7.246228in}{4.965599in}}%
\pgfpathlineto{\pgfqpoint{7.260944in}{4.938496in}}%
\pgfpathlineto{\pgfqpoint{7.275659in}{5.020046in}}%
\pgfpathlineto{\pgfqpoint{7.290374in}{5.063044in}}%
\pgfpathlineto{\pgfqpoint{7.305090in}{5.061003in}}%
\pgfpathlineto{\pgfqpoint{7.319805in}{4.943641in}}%
\pgfpathlineto{\pgfqpoint{7.334520in}{4.849158in}}%
\pgfpathlineto{\pgfqpoint{7.349236in}{4.776324in}}%
\pgfpathlineto{\pgfqpoint{7.363951in}{4.767053in}}%
\pgfpathlineto{\pgfqpoint{7.378666in}{4.865884in}}%
\pgfpathlineto{\pgfqpoint{7.393381in}{4.980627in}}%
\pgfpathlineto{\pgfqpoint{7.408097in}{5.077119in}}%
\pgfpathlineto{\pgfqpoint{7.422812in}{5.145777in}}%
\pgfpathlineto{\pgfqpoint{7.437527in}{5.198205in}}%
\pgfpathlineto{\pgfqpoint{7.452243in}{5.222266in}}%
\pgfpathlineto{\pgfqpoint{7.466958in}{5.216280in}}%
\pgfpathlineto{\pgfqpoint{7.481673in}{5.177487in}}%
\pgfpathlineto{\pgfqpoint{7.496389in}{5.100016in}}%
\pgfpathlineto{\pgfqpoint{7.511104in}{4.992755in}}%
\pgfpathlineto{\pgfqpoint{7.525819in}{4.918483in}}%
\pgfpathlineto{\pgfqpoint{7.540535in}{4.942741in}}%
\pgfpathlineto{\pgfqpoint{7.555250in}{5.028274in}}%
\pgfpathlineto{\pgfqpoint{7.569965in}{5.115708in}}%
\pgfpathlineto{\pgfqpoint{7.584681in}{5.168722in}}%
\pgfpathlineto{\pgfqpoint{7.599396in}{5.266526in}}%
\pgfpathlineto{\pgfqpoint{7.614111in}{5.311312in}}%
\pgfpathlineto{\pgfqpoint{7.628827in}{5.379659in}}%
\pgfpathlineto{\pgfqpoint{7.643542in}{5.397878in}}%
\pgfpathlineto{\pgfqpoint{7.658257in}{5.396051in}}%
\pgfpathlineto{\pgfqpoint{7.672972in}{5.327932in}}%
\pgfpathlineto{\pgfqpoint{7.687688in}{5.176902in}}%
\pgfpathlineto{\pgfqpoint{7.702403in}{5.124796in}}%
\pgfpathlineto{\pgfqpoint{7.717118in}{5.145000in}}%
\pgfpathlineto{\pgfqpoint{7.731834in}{5.227074in}}%
\pgfpathlineto{\pgfqpoint{7.746549in}{5.314122in}}%
\pgfpathlineto{\pgfqpoint{7.746549in}{5.314122in}}%
\pgfusepath{stroke}%
\end{pgfscope}%
\begin{pgfscope}%
\pgfpathrectangle{\pgfqpoint{0.697913in}{0.559721in}}{\pgfqpoint{7.048636in}{4.990279in}}%
\pgfusepath{clip}%
\pgfsetbuttcap%
\pgfsetroundjoin%
\definecolor{currentfill}{rgb}{1.000000,0.000000,0.000000}%
\pgfsetfillcolor{currentfill}%
\pgfsetlinewidth{1.003750pt}%
\definecolor{currentstroke}{rgb}{1.000000,0.000000,0.000000}%
\pgfsetstrokecolor{currentstroke}%
\pgfsetdash{}{0pt}%
\pgfsys@defobject{currentmarker}{\pgfqpoint{-0.020833in}{-0.020833in}}{\pgfqpoint{0.020833in}{0.020833in}}{%
\pgfpathmoveto{\pgfqpoint{0.000000in}{-0.020833in}}%
\pgfpathcurveto{\pgfqpoint{0.005525in}{-0.020833in}}{\pgfqpoint{0.010825in}{-0.018638in}}{\pgfqpoint{0.014731in}{-0.014731in}}%
\pgfpathcurveto{\pgfqpoint{0.018638in}{-0.010825in}}{\pgfqpoint{0.020833in}{-0.005525in}}{\pgfqpoint{0.020833in}{0.000000in}}%
\pgfpathcurveto{\pgfqpoint{0.020833in}{0.005525in}}{\pgfqpoint{0.018638in}{0.010825in}}{\pgfqpoint{0.014731in}{0.014731in}}%
\pgfpathcurveto{\pgfqpoint{0.010825in}{0.018638in}}{\pgfqpoint{0.005525in}{0.020833in}}{\pgfqpoint{0.000000in}{0.020833in}}%
\pgfpathcurveto{\pgfqpoint{-0.005525in}{0.020833in}}{\pgfqpoint{-0.010825in}{0.018638in}}{\pgfqpoint{-0.014731in}{0.014731in}}%
\pgfpathcurveto{\pgfqpoint{-0.018638in}{0.010825in}}{\pgfqpoint{-0.020833in}{0.005525in}}{\pgfqpoint{-0.020833in}{0.000000in}}%
\pgfpathcurveto{\pgfqpoint{-0.020833in}{-0.005525in}}{\pgfqpoint{-0.018638in}{-0.010825in}}{\pgfqpoint{-0.014731in}{-0.014731in}}%
\pgfpathcurveto{\pgfqpoint{-0.010825in}{-0.018638in}}{\pgfqpoint{-0.005525in}{-0.020833in}}{\pgfqpoint{0.000000in}{-0.020833in}}%
\pgfpathlineto{\pgfqpoint{0.000000in}{-0.020833in}}%
\pgfpathclose%
\pgfusepath{stroke,fill}%
}%
\begin{pgfscope}%
\pgfsys@transformshift{0.697913in}{0.800800in}%
\pgfsys@useobject{currentmarker}{}%
\end{pgfscope}%
\begin{pgfscope}%
\pgfsys@transformshift{0.712628in}{0.854764in}%
\pgfsys@useobject{currentmarker}{}%
\end{pgfscope}%
\begin{pgfscope}%
\pgfsys@transformshift{0.727343in}{0.933745in}%
\pgfsys@useobject{currentmarker}{}%
\end{pgfscope}%
\begin{pgfscope}%
\pgfsys@transformshift{0.742059in}{0.977716in}%
\pgfsys@useobject{currentmarker}{}%
\end{pgfscope}%
\begin{pgfscope}%
\pgfsys@transformshift{0.756774in}{1.015272in}%
\pgfsys@useobject{currentmarker}{}%
\end{pgfscope}%
\begin{pgfscope}%
\pgfsys@transformshift{0.771489in}{0.976371in}%
\pgfsys@useobject{currentmarker}{}%
\end{pgfscope}%
\begin{pgfscope}%
\pgfsys@transformshift{0.786205in}{0.886780in}%
\pgfsys@useobject{currentmarker}{}%
\end{pgfscope}%
\begin{pgfscope}%
\pgfsys@transformshift{0.800920in}{0.775096in}%
\pgfsys@useobject{currentmarker}{}%
\end{pgfscope}%
\begin{pgfscope}%
\pgfsys@transformshift{0.815635in}{0.687494in}%
\pgfsys@useobject{currentmarker}{}%
\end{pgfscope}%
\begin{pgfscope}%
\pgfsys@transformshift{0.830350in}{0.682107in}%
\pgfsys@useobject{currentmarker}{}%
\end{pgfscope}%
\begin{pgfscope}%
\pgfsys@transformshift{0.845066in}{0.760034in}%
\pgfsys@useobject{currentmarker}{}%
\end{pgfscope}%
\begin{pgfscope}%
\pgfsys@transformshift{0.859781in}{0.838848in}%
\pgfsys@useobject{currentmarker}{}%
\end{pgfscope}%
\begin{pgfscope}%
\pgfsys@transformshift{0.874496in}{0.875106in}%
\pgfsys@useobject{currentmarker}{}%
\end{pgfscope}%
\begin{pgfscope}%
\pgfsys@transformshift{0.889212in}{0.913637in}%
\pgfsys@useobject{currentmarker}{}%
\end{pgfscope}%
\begin{pgfscope}%
\pgfsys@transformshift{0.903927in}{0.955498in}%
\pgfsys@useobject{currentmarker}{}%
\end{pgfscope}%
\begin{pgfscope}%
\pgfsys@transformshift{0.918642in}{1.048312in}%
\pgfsys@useobject{currentmarker}{}%
\end{pgfscope}%
\begin{pgfscope}%
\pgfsys@transformshift{0.933358in}{1.085369in}%
\pgfsys@useobject{currentmarker}{}%
\end{pgfscope}%
\begin{pgfscope}%
\pgfsys@transformshift{0.948073in}{1.055640in}%
\pgfsys@useobject{currentmarker}{}%
\end{pgfscope}%
\begin{pgfscope}%
\pgfsys@transformshift{0.962788in}{0.959881in}%
\pgfsys@useobject{currentmarker}{}%
\end{pgfscope}%
\begin{pgfscope}%
\pgfsys@transformshift{0.977504in}{0.841100in}%
\pgfsys@useobject{currentmarker}{}%
\end{pgfscope}%
\begin{pgfscope}%
\pgfsys@transformshift{0.992219in}{0.795475in}%
\pgfsys@useobject{currentmarker}{}%
\end{pgfscope}%
\begin{pgfscope}%
\pgfsys@transformshift{1.006934in}{0.760407in}%
\pgfsys@useobject{currentmarker}{}%
\end{pgfscope}%
\begin{pgfscope}%
\pgfsys@transformshift{1.021650in}{0.844761in}%
\pgfsys@useobject{currentmarker}{}%
\end{pgfscope}%
\begin{pgfscope}%
\pgfsys@transformshift{1.036365in}{0.909468in}%
\pgfsys@useobject{currentmarker}{}%
\end{pgfscope}%
\begin{pgfscope}%
\pgfsys@transformshift{1.051080in}{0.982320in}%
\pgfsys@useobject{currentmarker}{}%
\end{pgfscope}%
\begin{pgfscope}%
\pgfsys@transformshift{1.065796in}{0.994584in}%
\pgfsys@useobject{currentmarker}{}%
\end{pgfscope}%
\begin{pgfscope}%
\pgfsys@transformshift{1.080511in}{1.044934in}%
\pgfsys@useobject{currentmarker}{}%
\end{pgfscope}%
\begin{pgfscope}%
\pgfsys@transformshift{1.095226in}{1.136847in}%
\pgfsys@useobject{currentmarker}{}%
\end{pgfscope}%
\begin{pgfscope}%
\pgfsys@transformshift{1.109941in}{1.177436in}%
\pgfsys@useobject{currentmarker}{}%
\end{pgfscope}%
\begin{pgfscope}%
\pgfsys@transformshift{1.124657in}{1.150975in}%
\pgfsys@useobject{currentmarker}{}%
\end{pgfscope}%
\begin{pgfscope}%
\pgfsys@transformshift{1.139372in}{1.075536in}%
\pgfsys@useobject{currentmarker}{}%
\end{pgfscope}%
\begin{pgfscope}%
\pgfsys@transformshift{1.154087in}{0.954253in}%
\pgfsys@useobject{currentmarker}{}%
\end{pgfscope}%
\begin{pgfscope}%
\pgfsys@transformshift{1.168803in}{0.864921in}%
\pgfsys@useobject{currentmarker}{}%
\end{pgfscope}%
\begin{pgfscope}%
\pgfsys@transformshift{1.183518in}{0.873943in}%
\pgfsys@useobject{currentmarker}{}%
\end{pgfscope}%
\begin{pgfscope}%
\pgfsys@transformshift{1.198233in}{0.947407in}%
\pgfsys@useobject{currentmarker}{}%
\end{pgfscope}%
\begin{pgfscope}%
\pgfsys@transformshift{1.212949in}{1.016117in}%
\pgfsys@useobject{currentmarker}{}%
\end{pgfscope}%
\begin{pgfscope}%
\pgfsys@transformshift{1.227664in}{1.081354in}%
\pgfsys@useobject{currentmarker}{}%
\end{pgfscope}%
\begin{pgfscope}%
\pgfsys@transformshift{1.242379in}{1.154139in}%
\pgfsys@useobject{currentmarker}{}%
\end{pgfscope}%
\begin{pgfscope}%
\pgfsys@transformshift{1.257095in}{1.194737in}%
\pgfsys@useobject{currentmarker}{}%
\end{pgfscope}%
\begin{pgfscope}%
\pgfsys@transformshift{1.271810in}{1.262801in}%
\pgfsys@useobject{currentmarker}{}%
\end{pgfscope}%
\begin{pgfscope}%
\pgfsys@transformshift{1.286525in}{1.301798in}%
\pgfsys@useobject{currentmarker}{}%
\end{pgfscope}%
\begin{pgfscope}%
\pgfsys@transformshift{1.301241in}{1.271334in}%
\pgfsys@useobject{currentmarker}{}%
\end{pgfscope}%
\begin{pgfscope}%
\pgfsys@transformshift{1.315956in}{1.212318in}%
\pgfsys@useobject{currentmarker}{}%
\end{pgfscope}%
\begin{pgfscope}%
\pgfsys@transformshift{1.330671in}{1.080731in}%
\pgfsys@useobject{currentmarker}{}%
\end{pgfscope}%
\begin{pgfscope}%
\pgfsys@transformshift{1.345387in}{0.998402in}%
\pgfsys@useobject{currentmarker}{}%
\end{pgfscope}%
\begin{pgfscope}%
\pgfsys@transformshift{1.360102in}{1.021379in}%
\pgfsys@useobject{currentmarker}{}%
\end{pgfscope}%
\begin{pgfscope}%
\pgfsys@transformshift{1.374817in}{1.082667in}%
\pgfsys@useobject{currentmarker}{}%
\end{pgfscope}%
\begin{pgfscope}%
\pgfsys@transformshift{1.389532in}{1.146865in}%
\pgfsys@useobject{currentmarker}{}%
\end{pgfscope}%
\begin{pgfscope}%
\pgfsys@transformshift{1.404248in}{1.228919in}%
\pgfsys@useobject{currentmarker}{}%
\end{pgfscope}%
\begin{pgfscope}%
\pgfsys@transformshift{1.418963in}{1.249078in}%
\pgfsys@useobject{currentmarker}{}%
\end{pgfscope}%
\begin{pgfscope}%
\pgfsys@transformshift{1.433678in}{1.285964in}%
\pgfsys@useobject{currentmarker}{}%
\end{pgfscope}%
\begin{pgfscope}%
\pgfsys@transformshift{1.448394in}{1.379033in}%
\pgfsys@useobject{currentmarker}{}%
\end{pgfscope}%
\begin{pgfscope}%
\pgfsys@transformshift{1.463109in}{1.392104in}%
\pgfsys@useobject{currentmarker}{}%
\end{pgfscope}%
\begin{pgfscope}%
\pgfsys@transformshift{1.477824in}{1.362008in}%
\pgfsys@useobject{currentmarker}{}%
\end{pgfscope}%
\begin{pgfscope}%
\pgfsys@transformshift{1.492540in}{1.284978in}%
\pgfsys@useobject{currentmarker}{}%
\end{pgfscope}%
\begin{pgfscope}%
\pgfsys@transformshift{1.507255in}{1.162372in}%
\pgfsys@useobject{currentmarker}{}%
\end{pgfscope}%
\begin{pgfscope}%
\pgfsys@transformshift{1.521970in}{1.062532in}%
\pgfsys@useobject{currentmarker}{}%
\end{pgfscope}%
\begin{pgfscope}%
\pgfsys@transformshift{1.536686in}{1.076569in}%
\pgfsys@useobject{currentmarker}{}%
\end{pgfscope}%
\begin{pgfscope}%
\pgfsys@transformshift{1.551401in}{1.145336in}%
\pgfsys@useobject{currentmarker}{}%
\end{pgfscope}%
\begin{pgfscope}%
\pgfsys@transformshift{1.566116in}{1.217965in}%
\pgfsys@useobject{currentmarker}{}%
\end{pgfscope}%
\begin{pgfscope}%
\pgfsys@transformshift{1.580832in}{1.274509in}%
\pgfsys@useobject{currentmarker}{}%
\end{pgfscope}%
\begin{pgfscope}%
\pgfsys@transformshift{1.595547in}{1.343743in}%
\pgfsys@useobject{currentmarker}{}%
\end{pgfscope}%
\begin{pgfscope}%
\pgfsys@transformshift{1.610262in}{1.381731in}%
\pgfsys@useobject{currentmarker}{}%
\end{pgfscope}%
\begin{pgfscope}%
\pgfsys@transformshift{1.624978in}{1.415855in}%
\pgfsys@useobject{currentmarker}{}%
\end{pgfscope}%
\begin{pgfscope}%
\pgfsys@transformshift{1.639693in}{1.472501in}%
\pgfsys@useobject{currentmarker}{}%
\end{pgfscope}%
\begin{pgfscope}%
\pgfsys@transformshift{1.654408in}{1.417875in}%
\pgfsys@useobject{currentmarker}{}%
\end{pgfscope}%
\begin{pgfscope}%
\pgfsys@transformshift{1.669124in}{1.332844in}%
\pgfsys@useobject{currentmarker}{}%
\end{pgfscope}%
\begin{pgfscope}%
\pgfsys@transformshift{1.683839in}{1.227407in}%
\pgfsys@useobject{currentmarker}{}%
\end{pgfscope}%
\begin{pgfscope}%
\pgfsys@transformshift{1.698554in}{1.132997in}%
\pgfsys@useobject{currentmarker}{}%
\end{pgfscope}%
\begin{pgfscope}%
\pgfsys@transformshift{1.713269in}{1.152108in}%
\pgfsys@useobject{currentmarker}{}%
\end{pgfscope}%
\begin{pgfscope}%
\pgfsys@transformshift{1.727985in}{1.230512in}%
\pgfsys@useobject{currentmarker}{}%
\end{pgfscope}%
\begin{pgfscope}%
\pgfsys@transformshift{1.742700in}{1.306255in}%
\pgfsys@useobject{currentmarker}{}%
\end{pgfscope}%
\begin{pgfscope}%
\pgfsys@transformshift{1.757415in}{1.334366in}%
\pgfsys@useobject{currentmarker}{}%
\end{pgfscope}%
\begin{pgfscope}%
\pgfsys@transformshift{1.772131in}{1.384833in}%
\pgfsys@useobject{currentmarker}{}%
\end{pgfscope}%
\begin{pgfscope}%
\pgfsys@transformshift{1.786846in}{1.470691in}%
\pgfsys@useobject{currentmarker}{}%
\end{pgfscope}%
\begin{pgfscope}%
\pgfsys@transformshift{1.801561in}{1.551077in}%
\pgfsys@useobject{currentmarker}{}%
\end{pgfscope}%
\begin{pgfscope}%
\pgfsys@transformshift{1.816277in}{1.577512in}%
\pgfsys@useobject{currentmarker}{}%
\end{pgfscope}%
\begin{pgfscope}%
\pgfsys@transformshift{1.830992in}{1.520951in}%
\pgfsys@useobject{currentmarker}{}%
\end{pgfscope}%
\begin{pgfscope}%
\pgfsys@transformshift{1.845707in}{1.412191in}%
\pgfsys@useobject{currentmarker}{}%
\end{pgfscope}%
\begin{pgfscope}%
\pgfsys@transformshift{1.860423in}{1.272178in}%
\pgfsys@useobject{currentmarker}{}%
\end{pgfscope}%
\begin{pgfscope}%
\pgfsys@transformshift{1.875138in}{1.187642in}%
\pgfsys@useobject{currentmarker}{}%
\end{pgfscope}%
\begin{pgfscope}%
\pgfsys@transformshift{1.889853in}{1.192263in}%
\pgfsys@useobject{currentmarker}{}%
\end{pgfscope}%
\begin{pgfscope}%
\pgfsys@transformshift{1.904569in}{1.275825in}%
\pgfsys@useobject{currentmarker}{}%
\end{pgfscope}%
\begin{pgfscope}%
\pgfsys@transformshift{1.919284in}{1.351213in}%
\pgfsys@useobject{currentmarker}{}%
\end{pgfscope}%
\begin{pgfscope}%
\pgfsys@transformshift{1.933999in}{1.414220in}%
\pgfsys@useobject{currentmarker}{}%
\end{pgfscope}%
\begin{pgfscope}%
\pgfsys@transformshift{1.948715in}{1.463436in}%
\pgfsys@useobject{currentmarker}{}%
\end{pgfscope}%
\begin{pgfscope}%
\pgfsys@transformshift{1.963430in}{1.504903in}%
\pgfsys@useobject{currentmarker}{}%
\end{pgfscope}%
\begin{pgfscope}%
\pgfsys@transformshift{1.978145in}{1.576992in}%
\pgfsys@useobject{currentmarker}{}%
\end{pgfscope}%
\begin{pgfscope}%
\pgfsys@transformshift{1.992860in}{1.604477in}%
\pgfsys@useobject{currentmarker}{}%
\end{pgfscope}%
\begin{pgfscope}%
\pgfsys@transformshift{2.007576in}{1.588776in}%
\pgfsys@useobject{currentmarker}{}%
\end{pgfscope}%
\begin{pgfscope}%
\pgfsys@transformshift{2.022291in}{1.459711in}%
\pgfsys@useobject{currentmarker}{}%
\end{pgfscope}%
\begin{pgfscope}%
\pgfsys@transformshift{2.037006in}{1.342687in}%
\pgfsys@useobject{currentmarker}{}%
\end{pgfscope}%
\begin{pgfscope}%
\pgfsys@transformshift{2.051722in}{1.224934in}%
\pgfsys@useobject{currentmarker}{}%
\end{pgfscope}%
\begin{pgfscope}%
\pgfsys@transformshift{2.066437in}{1.249769in}%
\pgfsys@useobject{currentmarker}{}%
\end{pgfscope}%
\begin{pgfscope}%
\pgfsys@transformshift{2.081152in}{1.304480in}%
\pgfsys@useobject{currentmarker}{}%
\end{pgfscope}%
\begin{pgfscope}%
\pgfsys@transformshift{2.095868in}{1.376535in}%
\pgfsys@useobject{currentmarker}{}%
\end{pgfscope}%
\begin{pgfscope}%
\pgfsys@transformshift{2.110583in}{1.457719in}%
\pgfsys@useobject{currentmarker}{}%
\end{pgfscope}%
\begin{pgfscope}%
\pgfsys@transformshift{2.125298in}{1.474739in}%
\pgfsys@useobject{currentmarker}{}%
\end{pgfscope}%
\begin{pgfscope}%
\pgfsys@transformshift{2.140014in}{1.541798in}%
\pgfsys@useobject{currentmarker}{}%
\end{pgfscope}%
\begin{pgfscope}%
\pgfsys@transformshift{2.154729in}{1.586777in}%
\pgfsys@useobject{currentmarker}{}%
\end{pgfscope}%
\begin{pgfscope}%
\pgfsys@transformshift{2.169444in}{1.637899in}%
\pgfsys@useobject{currentmarker}{}%
\end{pgfscope}%
\begin{pgfscope}%
\pgfsys@transformshift{2.184160in}{1.600492in}%
\pgfsys@useobject{currentmarker}{}%
\end{pgfscope}%
\begin{pgfscope}%
\pgfsys@transformshift{2.198875in}{1.479179in}%
\pgfsys@useobject{currentmarker}{}%
\end{pgfscope}%
\begin{pgfscope}%
\pgfsys@transformshift{2.213590in}{1.382317in}%
\pgfsys@useobject{currentmarker}{}%
\end{pgfscope}%
\begin{pgfscope}%
\pgfsys@transformshift{2.228306in}{1.288327in}%
\pgfsys@useobject{currentmarker}{}%
\end{pgfscope}%
\begin{pgfscope}%
\pgfsys@transformshift{2.243021in}{1.296898in}%
\pgfsys@useobject{currentmarker}{}%
\end{pgfscope}%
\begin{pgfscope}%
\pgfsys@transformshift{2.257736in}{1.367934in}%
\pgfsys@useobject{currentmarker}{}%
\end{pgfscope}%
\begin{pgfscope}%
\pgfsys@transformshift{2.272451in}{1.453039in}%
\pgfsys@useobject{currentmarker}{}%
\end{pgfscope}%
\begin{pgfscope}%
\pgfsys@transformshift{2.287167in}{1.525275in}%
\pgfsys@useobject{currentmarker}{}%
\end{pgfscope}%
\begin{pgfscope}%
\pgfsys@transformshift{2.301882in}{1.567715in}%
\pgfsys@useobject{currentmarker}{}%
\end{pgfscope}%
\begin{pgfscope}%
\pgfsys@transformshift{2.316597in}{1.626149in}%
\pgfsys@useobject{currentmarker}{}%
\end{pgfscope}%
\begin{pgfscope}%
\pgfsys@transformshift{2.331313in}{1.699022in}%
\pgfsys@useobject{currentmarker}{}%
\end{pgfscope}%
\begin{pgfscope}%
\pgfsys@transformshift{2.346028in}{1.721353in}%
\pgfsys@useobject{currentmarker}{}%
\end{pgfscope}%
\begin{pgfscope}%
\pgfsys@transformshift{2.360743in}{1.675260in}%
\pgfsys@useobject{currentmarker}{}%
\end{pgfscope}%
\begin{pgfscope}%
\pgfsys@transformshift{2.375459in}{1.593211in}%
\pgfsys@useobject{currentmarker}{}%
\end{pgfscope}%
\begin{pgfscope}%
\pgfsys@transformshift{2.390174in}{1.487188in}%
\pgfsys@useobject{currentmarker}{}%
\end{pgfscope}%
\begin{pgfscope}%
\pgfsys@transformshift{2.404889in}{1.387600in}%
\pgfsys@useobject{currentmarker}{}%
\end{pgfscope}%
\begin{pgfscope}%
\pgfsys@transformshift{2.419605in}{1.404099in}%
\pgfsys@useobject{currentmarker}{}%
\end{pgfscope}%
\begin{pgfscope}%
\pgfsys@transformshift{2.434320in}{1.489077in}%
\pgfsys@useobject{currentmarker}{}%
\end{pgfscope}%
\begin{pgfscope}%
\pgfsys@transformshift{2.449035in}{1.568045in}%
\pgfsys@useobject{currentmarker}{}%
\end{pgfscope}%
\begin{pgfscope}%
\pgfsys@transformshift{2.463751in}{1.623407in}%
\pgfsys@useobject{currentmarker}{}%
\end{pgfscope}%
\begin{pgfscope}%
\pgfsys@transformshift{2.478466in}{1.676960in}%
\pgfsys@useobject{currentmarker}{}%
\end{pgfscope}%
\begin{pgfscope}%
\pgfsys@transformshift{2.493181in}{1.731250in}%
\pgfsys@useobject{currentmarker}{}%
\end{pgfscope}%
\begin{pgfscope}%
\pgfsys@transformshift{2.507897in}{1.813823in}%
\pgfsys@useobject{currentmarker}{}%
\end{pgfscope}%
\begin{pgfscope}%
\pgfsys@transformshift{2.522612in}{1.836876in}%
\pgfsys@useobject{currentmarker}{}%
\end{pgfscope}%
\begin{pgfscope}%
\pgfsys@transformshift{2.537327in}{1.809246in}%
\pgfsys@useobject{currentmarker}{}%
\end{pgfscope}%
\begin{pgfscope}%
\pgfsys@transformshift{2.552042in}{1.717090in}%
\pgfsys@useobject{currentmarker}{}%
\end{pgfscope}%
\begin{pgfscope}%
\pgfsys@transformshift{2.566758in}{1.583482in}%
\pgfsys@useobject{currentmarker}{}%
\end{pgfscope}%
\begin{pgfscope}%
\pgfsys@transformshift{2.581473in}{1.523889in}%
\pgfsys@useobject{currentmarker}{}%
\end{pgfscope}%
\begin{pgfscope}%
\pgfsys@transformshift{2.596188in}{1.515421in}%
\pgfsys@useobject{currentmarker}{}%
\end{pgfscope}%
\begin{pgfscope}%
\pgfsys@transformshift{2.610904in}{1.599895in}%
\pgfsys@useobject{currentmarker}{}%
\end{pgfscope}%
\begin{pgfscope}%
\pgfsys@transformshift{2.625619in}{1.668202in}%
\pgfsys@useobject{currentmarker}{}%
\end{pgfscope}%
\begin{pgfscope}%
\pgfsys@transformshift{2.640334in}{1.744451in}%
\pgfsys@useobject{currentmarker}{}%
\end{pgfscope}%
\begin{pgfscope}%
\pgfsys@transformshift{2.655050in}{1.811203in}%
\pgfsys@useobject{currentmarker}{}%
\end{pgfscope}%
\begin{pgfscope}%
\pgfsys@transformshift{2.669765in}{1.861214in}%
\pgfsys@useobject{currentmarker}{}%
\end{pgfscope}%
\begin{pgfscope}%
\pgfsys@transformshift{2.684480in}{1.886467in}%
\pgfsys@useobject{currentmarker}{}%
\end{pgfscope}%
\begin{pgfscope}%
\pgfsys@transformshift{2.699196in}{1.917320in}%
\pgfsys@useobject{currentmarker}{}%
\end{pgfscope}%
\begin{pgfscope}%
\pgfsys@transformshift{2.713911in}{1.906216in}%
\pgfsys@useobject{currentmarker}{}%
\end{pgfscope}%
\begin{pgfscope}%
\pgfsys@transformshift{2.728626in}{1.828829in}%
\pgfsys@useobject{currentmarker}{}%
\end{pgfscope}%
\begin{pgfscope}%
\pgfsys@transformshift{2.743342in}{1.708741in}%
\pgfsys@useobject{currentmarker}{}%
\end{pgfscope}%
\begin{pgfscope}%
\pgfsys@transformshift{2.758057in}{1.605004in}%
\pgfsys@useobject{currentmarker}{}%
\end{pgfscope}%
\begin{pgfscope}%
\pgfsys@transformshift{2.772772in}{1.606250in}%
\pgfsys@useobject{currentmarker}{}%
\end{pgfscope}%
\begin{pgfscope}%
\pgfsys@transformshift{2.787488in}{1.680121in}%
\pgfsys@useobject{currentmarker}{}%
\end{pgfscope}%
\begin{pgfscope}%
\pgfsys@transformshift{2.802203in}{1.757359in}%
\pgfsys@useobject{currentmarker}{}%
\end{pgfscope}%
\begin{pgfscope}%
\pgfsys@transformshift{2.816918in}{1.803530in}%
\pgfsys@useobject{currentmarker}{}%
\end{pgfscope}%
\begin{pgfscope}%
\pgfsys@transformshift{2.831633in}{1.858169in}%
\pgfsys@useobject{currentmarker}{}%
\end{pgfscope}%
\begin{pgfscope}%
\pgfsys@transformshift{2.846349in}{1.882998in}%
\pgfsys@useobject{currentmarker}{}%
\end{pgfscope}%
\begin{pgfscope}%
\pgfsys@transformshift{2.861064in}{1.988554in}%
\pgfsys@useobject{currentmarker}{}%
\end{pgfscope}%
\begin{pgfscope}%
\pgfsys@transformshift{2.875779in}{2.003172in}%
\pgfsys@useobject{currentmarker}{}%
\end{pgfscope}%
\begin{pgfscope}%
\pgfsys@transformshift{2.890495in}{1.940849in}%
\pgfsys@useobject{currentmarker}{}%
\end{pgfscope}%
\begin{pgfscope}%
\pgfsys@transformshift{2.905210in}{1.874601in}%
\pgfsys@useobject{currentmarker}{}%
\end{pgfscope}%
\begin{pgfscope}%
\pgfsys@transformshift{2.919925in}{1.757569in}%
\pgfsys@useobject{currentmarker}{}%
\end{pgfscope}%
\begin{pgfscope}%
\pgfsys@transformshift{2.934641in}{1.646651in}%
\pgfsys@useobject{currentmarker}{}%
\end{pgfscope}%
\begin{pgfscope}%
\pgfsys@transformshift{2.949356in}{1.676619in}%
\pgfsys@useobject{currentmarker}{}%
\end{pgfscope}%
\begin{pgfscope}%
\pgfsys@transformshift{2.964071in}{1.770213in}%
\pgfsys@useobject{currentmarker}{}%
\end{pgfscope}%
\begin{pgfscope}%
\pgfsys@transformshift{2.978787in}{1.875706in}%
\pgfsys@useobject{currentmarker}{}%
\end{pgfscope}%
\begin{pgfscope}%
\pgfsys@transformshift{2.993502in}{1.924661in}%
\pgfsys@useobject{currentmarker}{}%
\end{pgfscope}%
\begin{pgfscope}%
\pgfsys@transformshift{3.008217in}{1.965370in}%
\pgfsys@useobject{currentmarker}{}%
\end{pgfscope}%
\begin{pgfscope}%
\pgfsys@transformshift{3.022933in}{2.036452in}%
\pgfsys@useobject{currentmarker}{}%
\end{pgfscope}%
\begin{pgfscope}%
\pgfsys@transformshift{3.037648in}{2.116879in}%
\pgfsys@useobject{currentmarker}{}%
\end{pgfscope}%
\begin{pgfscope}%
\pgfsys@transformshift{3.052363in}{2.160322in}%
\pgfsys@useobject{currentmarker}{}%
\end{pgfscope}%
\begin{pgfscope}%
\pgfsys@transformshift{3.067079in}{2.135205in}%
\pgfsys@useobject{currentmarker}{}%
\end{pgfscope}%
\begin{pgfscope}%
\pgfsys@transformshift{3.081794in}{2.071972in}%
\pgfsys@useobject{currentmarker}{}%
\end{pgfscope}%
\begin{pgfscope}%
\pgfsys@transformshift{3.096509in}{1.966858in}%
\pgfsys@useobject{currentmarker}{}%
\end{pgfscope}%
\begin{pgfscope}%
\pgfsys@transformshift{3.111224in}{1.856132in}%
\pgfsys@useobject{currentmarker}{}%
\end{pgfscope}%
\begin{pgfscope}%
\pgfsys@transformshift{3.125940in}{1.876873in}%
\pgfsys@useobject{currentmarker}{}%
\end{pgfscope}%
\begin{pgfscope}%
\pgfsys@transformshift{3.140655in}{1.939208in}%
\pgfsys@useobject{currentmarker}{}%
\end{pgfscope}%
\begin{pgfscope}%
\pgfsys@transformshift{3.155370in}{2.035861in}%
\pgfsys@useobject{currentmarker}{}%
\end{pgfscope}%
\begin{pgfscope}%
\pgfsys@transformshift{3.170086in}{2.090478in}%
\pgfsys@useobject{currentmarker}{}%
\end{pgfscope}%
\begin{pgfscope}%
\pgfsys@transformshift{3.184801in}{2.140813in}%
\pgfsys@useobject{currentmarker}{}%
\end{pgfscope}%
\begin{pgfscope}%
\pgfsys@transformshift{3.199516in}{2.177751in}%
\pgfsys@useobject{currentmarker}{}%
\end{pgfscope}%
\begin{pgfscope}%
\pgfsys@transformshift{3.214232in}{2.251224in}%
\pgfsys@useobject{currentmarker}{}%
\end{pgfscope}%
\begin{pgfscope}%
\pgfsys@transformshift{3.228947in}{2.248600in}%
\pgfsys@useobject{currentmarker}{}%
\end{pgfscope}%
\begin{pgfscope}%
\pgfsys@transformshift{3.243662in}{2.213266in}%
\pgfsys@useobject{currentmarker}{}%
\end{pgfscope}%
\begin{pgfscope}%
\pgfsys@transformshift{3.258378in}{2.160034in}%
\pgfsys@useobject{currentmarker}{}%
\end{pgfscope}%
\begin{pgfscope}%
\pgfsys@transformshift{3.273093in}{2.018952in}%
\pgfsys@useobject{currentmarker}{}%
\end{pgfscope}%
\begin{pgfscope}%
\pgfsys@transformshift{3.287808in}{1.899710in}%
\pgfsys@useobject{currentmarker}{}%
\end{pgfscope}%
\begin{pgfscope}%
\pgfsys@transformshift{3.302524in}{1.932267in}%
\pgfsys@useobject{currentmarker}{}%
\end{pgfscope}%
\begin{pgfscope}%
\pgfsys@transformshift{3.317239in}{2.008121in}%
\pgfsys@useobject{currentmarker}{}%
\end{pgfscope}%
\begin{pgfscope}%
\pgfsys@transformshift{3.331954in}{2.087569in}%
\pgfsys@useobject{currentmarker}{}%
\end{pgfscope}%
\begin{pgfscope}%
\pgfsys@transformshift{3.346670in}{2.152937in}%
\pgfsys@useobject{currentmarker}{}%
\end{pgfscope}%
\begin{pgfscope}%
\pgfsys@transformshift{3.361385in}{2.171914in}%
\pgfsys@useobject{currentmarker}{}%
\end{pgfscope}%
\begin{pgfscope}%
\pgfsys@transformshift{3.376100in}{2.227382in}%
\pgfsys@useobject{currentmarker}{}%
\end{pgfscope}%
\begin{pgfscope}%
\pgfsys@transformshift{3.390815in}{2.296237in}%
\pgfsys@useobject{currentmarker}{}%
\end{pgfscope}%
\begin{pgfscope}%
\pgfsys@transformshift{3.405531in}{2.284008in}%
\pgfsys@useobject{currentmarker}{}%
\end{pgfscope}%
\begin{pgfscope}%
\pgfsys@transformshift{3.420246in}{2.290340in}%
\pgfsys@useobject{currentmarker}{}%
\end{pgfscope}%
\begin{pgfscope}%
\pgfsys@transformshift{3.434961in}{2.189405in}%
\pgfsys@useobject{currentmarker}{}%
\end{pgfscope}%
\begin{pgfscope}%
\pgfsys@transformshift{3.449677in}{2.092988in}%
\pgfsys@useobject{currentmarker}{}%
\end{pgfscope}%
\begin{pgfscope}%
\pgfsys@transformshift{3.464392in}{2.024112in}%
\pgfsys@useobject{currentmarker}{}%
\end{pgfscope}%
\begin{pgfscope}%
\pgfsys@transformshift{3.479107in}{2.025919in}%
\pgfsys@useobject{currentmarker}{}%
\end{pgfscope}%
\begin{pgfscope}%
\pgfsys@transformshift{3.493823in}{2.102225in}%
\pgfsys@useobject{currentmarker}{}%
\end{pgfscope}%
\begin{pgfscope}%
\pgfsys@transformshift{3.508538in}{2.176941in}%
\pgfsys@useobject{currentmarker}{}%
\end{pgfscope}%
\begin{pgfscope}%
\pgfsys@transformshift{3.523253in}{2.227966in}%
\pgfsys@useobject{currentmarker}{}%
\end{pgfscope}%
\begin{pgfscope}%
\pgfsys@transformshift{3.537969in}{2.281327in}%
\pgfsys@useobject{currentmarker}{}%
\end{pgfscope}%
\begin{pgfscope}%
\pgfsys@transformshift{3.552684in}{2.332186in}%
\pgfsys@useobject{currentmarker}{}%
\end{pgfscope}%
\begin{pgfscope}%
\pgfsys@transformshift{3.567399in}{2.385082in}%
\pgfsys@useobject{currentmarker}{}%
\end{pgfscope}%
\begin{pgfscope}%
\pgfsys@transformshift{3.582115in}{2.412631in}%
\pgfsys@useobject{currentmarker}{}%
\end{pgfscope}%
\begin{pgfscope}%
\pgfsys@transformshift{3.596830in}{2.374666in}%
\pgfsys@useobject{currentmarker}{}%
\end{pgfscope}%
\begin{pgfscope}%
\pgfsys@transformshift{3.611545in}{2.282164in}%
\pgfsys@useobject{currentmarker}{}%
\end{pgfscope}%
\begin{pgfscope}%
\pgfsys@transformshift{3.626261in}{2.166161in}%
\pgfsys@useobject{currentmarker}{}%
\end{pgfscope}%
\begin{pgfscope}%
\pgfsys@transformshift{3.640976in}{2.092151in}%
\pgfsys@useobject{currentmarker}{}%
\end{pgfscope}%
\begin{pgfscope}%
\pgfsys@transformshift{3.655691in}{2.106868in}%
\pgfsys@useobject{currentmarker}{}%
\end{pgfscope}%
\begin{pgfscope}%
\pgfsys@transformshift{3.670406in}{2.183364in}%
\pgfsys@useobject{currentmarker}{}%
\end{pgfscope}%
\begin{pgfscope}%
\pgfsys@transformshift{3.685122in}{2.264853in}%
\pgfsys@useobject{currentmarker}{}%
\end{pgfscope}%
\begin{pgfscope}%
\pgfsys@transformshift{3.699837in}{2.336196in}%
\pgfsys@useobject{currentmarker}{}%
\end{pgfscope}%
\begin{pgfscope}%
\pgfsys@transformshift{3.714552in}{2.375640in}%
\pgfsys@useobject{currentmarker}{}%
\end{pgfscope}%
\begin{pgfscope}%
\pgfsys@transformshift{3.729268in}{2.424982in}%
\pgfsys@useobject{currentmarker}{}%
\end{pgfscope}%
\begin{pgfscope}%
\pgfsys@transformshift{3.743983in}{2.478150in}%
\pgfsys@useobject{currentmarker}{}%
\end{pgfscope}%
\begin{pgfscope}%
\pgfsys@transformshift{3.758698in}{2.520284in}%
\pgfsys@useobject{currentmarker}{}%
\end{pgfscope}%
\begin{pgfscope}%
\pgfsys@transformshift{3.773414in}{2.507524in}%
\pgfsys@useobject{currentmarker}{}%
\end{pgfscope}%
\begin{pgfscope}%
\pgfsys@transformshift{3.788129in}{2.424904in}%
\pgfsys@useobject{currentmarker}{}%
\end{pgfscope}%
\begin{pgfscope}%
\pgfsys@transformshift{3.802844in}{2.299576in}%
\pgfsys@useobject{currentmarker}{}%
\end{pgfscope}%
\begin{pgfscope}%
\pgfsys@transformshift{3.817560in}{2.236134in}%
\pgfsys@useobject{currentmarker}{}%
\end{pgfscope}%
\begin{pgfscope}%
\pgfsys@transformshift{3.832275in}{2.226608in}%
\pgfsys@useobject{currentmarker}{}%
\end{pgfscope}%
\begin{pgfscope}%
\pgfsys@transformshift{3.846990in}{2.321823in}%
\pgfsys@useobject{currentmarker}{}%
\end{pgfscope}%
\begin{pgfscope}%
\pgfsys@transformshift{3.861706in}{2.409897in}%
\pgfsys@useobject{currentmarker}{}%
\end{pgfscope}%
\begin{pgfscope}%
\pgfsys@transformshift{3.876421in}{2.471099in}%
\pgfsys@useobject{currentmarker}{}%
\end{pgfscope}%
\begin{pgfscope}%
\pgfsys@transformshift{3.891136in}{2.513560in}%
\pgfsys@useobject{currentmarker}{}%
\end{pgfscope}%
\begin{pgfscope}%
\pgfsys@transformshift{3.905852in}{2.559572in}%
\pgfsys@useobject{currentmarker}{}%
\end{pgfscope}%
\begin{pgfscope}%
\pgfsys@transformshift{3.920567in}{2.632302in}%
\pgfsys@useobject{currentmarker}{}%
\end{pgfscope}%
\begin{pgfscope}%
\pgfsys@transformshift{3.935282in}{2.681032in}%
\pgfsys@useobject{currentmarker}{}%
\end{pgfscope}%
\begin{pgfscope}%
\pgfsys@transformshift{3.949997in}{2.660969in}%
\pgfsys@useobject{currentmarker}{}%
\end{pgfscope}%
\begin{pgfscope}%
\pgfsys@transformshift{3.964713in}{2.572562in}%
\pgfsys@useobject{currentmarker}{}%
\end{pgfscope}%
\begin{pgfscope}%
\pgfsys@transformshift{3.979428in}{2.453201in}%
\pgfsys@useobject{currentmarker}{}%
\end{pgfscope}%
\begin{pgfscope}%
\pgfsys@transformshift{3.994143in}{2.372235in}%
\pgfsys@useobject{currentmarker}{}%
\end{pgfscope}%
\begin{pgfscope}%
\pgfsys@transformshift{4.008859in}{2.372099in}%
\pgfsys@useobject{currentmarker}{}%
\end{pgfscope}%
\begin{pgfscope}%
\pgfsys@transformshift{4.023574in}{2.463749in}%
\pgfsys@useobject{currentmarker}{}%
\end{pgfscope}%
\begin{pgfscope}%
\pgfsys@transformshift{4.038289in}{2.530729in}%
\pgfsys@useobject{currentmarker}{}%
\end{pgfscope}%
\begin{pgfscope}%
\pgfsys@transformshift{4.053005in}{2.589786in}%
\pgfsys@useobject{currentmarker}{}%
\end{pgfscope}%
\begin{pgfscope}%
\pgfsys@transformshift{4.067720in}{2.640163in}%
\pgfsys@useobject{currentmarker}{}%
\end{pgfscope}%
\begin{pgfscope}%
\pgfsys@transformshift{4.082435in}{2.697827in}%
\pgfsys@useobject{currentmarker}{}%
\end{pgfscope}%
\begin{pgfscope}%
\pgfsys@transformshift{4.097151in}{2.780815in}%
\pgfsys@useobject{currentmarker}{}%
\end{pgfscope}%
\begin{pgfscope}%
\pgfsys@transformshift{4.111866in}{2.795046in}%
\pgfsys@useobject{currentmarker}{}%
\end{pgfscope}%
\begin{pgfscope}%
\pgfsys@transformshift{4.126581in}{2.746756in}%
\pgfsys@useobject{currentmarker}{}%
\end{pgfscope}%
\begin{pgfscope}%
\pgfsys@transformshift{4.141297in}{2.614488in}%
\pgfsys@useobject{currentmarker}{}%
\end{pgfscope}%
\begin{pgfscope}%
\pgfsys@transformshift{4.156012in}{2.546756in}%
\pgfsys@useobject{currentmarker}{}%
\end{pgfscope}%
\begin{pgfscope}%
\pgfsys@transformshift{4.170727in}{2.438426in}%
\pgfsys@useobject{currentmarker}{}%
\end{pgfscope}%
\begin{pgfscope}%
\pgfsys@transformshift{4.185443in}{2.447835in}%
\pgfsys@useobject{currentmarker}{}%
\end{pgfscope}%
\begin{pgfscope}%
\pgfsys@transformshift{4.200158in}{2.543726in}%
\pgfsys@useobject{currentmarker}{}%
\end{pgfscope}%
\begin{pgfscope}%
\pgfsys@transformshift{4.214873in}{2.618263in}%
\pgfsys@useobject{currentmarker}{}%
\end{pgfscope}%
\begin{pgfscope}%
\pgfsys@transformshift{4.229589in}{2.672363in}%
\pgfsys@useobject{currentmarker}{}%
\end{pgfscope}%
\begin{pgfscope}%
\pgfsys@transformshift{4.244304in}{2.745674in}%
\pgfsys@useobject{currentmarker}{}%
\end{pgfscope}%
\begin{pgfscope}%
\pgfsys@transformshift{4.259019in}{2.806202in}%
\pgfsys@useobject{currentmarker}{}%
\end{pgfscope}%
\begin{pgfscope}%
\pgfsys@transformshift{4.273734in}{2.889964in}%
\pgfsys@useobject{currentmarker}{}%
\end{pgfscope}%
\begin{pgfscope}%
\pgfsys@transformshift{4.288450in}{2.898893in}%
\pgfsys@useobject{currentmarker}{}%
\end{pgfscope}%
\begin{pgfscope}%
\pgfsys@transformshift{4.303165in}{2.886417in}%
\pgfsys@useobject{currentmarker}{}%
\end{pgfscope}%
\begin{pgfscope}%
\pgfsys@transformshift{4.317880in}{2.800754in}%
\pgfsys@useobject{currentmarker}{}%
\end{pgfscope}%
\begin{pgfscope}%
\pgfsys@transformshift{4.332596in}{2.692167in}%
\pgfsys@useobject{currentmarker}{}%
\end{pgfscope}%
\begin{pgfscope}%
\pgfsys@transformshift{4.347311in}{2.573537in}%
\pgfsys@useobject{currentmarker}{}%
\end{pgfscope}%
\begin{pgfscope}%
\pgfsys@transformshift{4.362026in}{2.581048in}%
\pgfsys@useobject{currentmarker}{}%
\end{pgfscope}%
\begin{pgfscope}%
\pgfsys@transformshift{4.376742in}{2.669507in}%
\pgfsys@useobject{currentmarker}{}%
\end{pgfscope}%
\begin{pgfscope}%
\pgfsys@transformshift{4.391457in}{2.765983in}%
\pgfsys@useobject{currentmarker}{}%
\end{pgfscope}%
\begin{pgfscope}%
\pgfsys@transformshift{4.406172in}{2.841474in}%
\pgfsys@useobject{currentmarker}{}%
\end{pgfscope}%
\begin{pgfscope}%
\pgfsys@transformshift{4.420888in}{2.886395in}%
\pgfsys@useobject{currentmarker}{}%
\end{pgfscope}%
\begin{pgfscope}%
\pgfsys@transformshift{4.435603in}{2.911579in}%
\pgfsys@useobject{currentmarker}{}%
\end{pgfscope}%
\begin{pgfscope}%
\pgfsys@transformshift{4.450318in}{3.025251in}%
\pgfsys@useobject{currentmarker}{}%
\end{pgfscope}%
\begin{pgfscope}%
\pgfsys@transformshift{4.465034in}{3.044627in}%
\pgfsys@useobject{currentmarker}{}%
\end{pgfscope}%
\begin{pgfscope}%
\pgfsys@transformshift{4.479749in}{2.989403in}%
\pgfsys@useobject{currentmarker}{}%
\end{pgfscope}%
\begin{pgfscope}%
\pgfsys@transformshift{4.494464in}{2.898926in}%
\pgfsys@useobject{currentmarker}{}%
\end{pgfscope}%
\begin{pgfscope}%
\pgfsys@transformshift{4.509180in}{2.782098in}%
\pgfsys@useobject{currentmarker}{}%
\end{pgfscope}%
\begin{pgfscope}%
\pgfsys@transformshift{4.523895in}{2.693424in}%
\pgfsys@useobject{currentmarker}{}%
\end{pgfscope}%
\begin{pgfscope}%
\pgfsys@transformshift{4.538610in}{2.711437in}%
\pgfsys@useobject{currentmarker}{}%
\end{pgfscope}%
\begin{pgfscope}%
\pgfsys@transformshift{4.553325in}{2.768121in}%
\pgfsys@useobject{currentmarker}{}%
\end{pgfscope}%
\begin{pgfscope}%
\pgfsys@transformshift{4.568041in}{2.861524in}%
\pgfsys@useobject{currentmarker}{}%
\end{pgfscope}%
\begin{pgfscope}%
\pgfsys@transformshift{4.582756in}{2.925864in}%
\pgfsys@useobject{currentmarker}{}%
\end{pgfscope}%
\begin{pgfscope}%
\pgfsys@transformshift{4.597471in}{2.981027in}%
\pgfsys@useobject{currentmarker}{}%
\end{pgfscope}%
\begin{pgfscope}%
\pgfsys@transformshift{4.612187in}{3.019895in}%
\pgfsys@useobject{currentmarker}{}%
\end{pgfscope}%
\begin{pgfscope}%
\pgfsys@transformshift{4.626902in}{3.130071in}%
\pgfsys@useobject{currentmarker}{}%
\end{pgfscope}%
\begin{pgfscope}%
\pgfsys@transformshift{4.641617in}{3.132820in}%
\pgfsys@useobject{currentmarker}{}%
\end{pgfscope}%
\begin{pgfscope}%
\pgfsys@transformshift{4.656333in}{3.109392in}%
\pgfsys@useobject{currentmarker}{}%
\end{pgfscope}%
\begin{pgfscope}%
\pgfsys@transformshift{4.671048in}{3.021191in}%
\pgfsys@useobject{currentmarker}{}%
\end{pgfscope}%
\begin{pgfscope}%
\pgfsys@transformshift{4.685763in}{2.879566in}%
\pgfsys@useobject{currentmarker}{}%
\end{pgfscope}%
\begin{pgfscope}%
\pgfsys@transformshift{4.700479in}{2.815185in}%
\pgfsys@useobject{currentmarker}{}%
\end{pgfscope}%
\begin{pgfscope}%
\pgfsys@transformshift{4.715194in}{2.827623in}%
\pgfsys@useobject{currentmarker}{}%
\end{pgfscope}%
\begin{pgfscope}%
\pgfsys@transformshift{4.729909in}{2.902540in}%
\pgfsys@useobject{currentmarker}{}%
\end{pgfscope}%
\begin{pgfscope}%
\pgfsys@transformshift{4.744625in}{2.989517in}%
\pgfsys@useobject{currentmarker}{}%
\end{pgfscope}%
\begin{pgfscope}%
\pgfsys@transformshift{4.759340in}{3.076353in}%
\pgfsys@useobject{currentmarker}{}%
\end{pgfscope}%
\begin{pgfscope}%
\pgfsys@transformshift{4.774055in}{3.091714in}%
\pgfsys@useobject{currentmarker}{}%
\end{pgfscope}%
\begin{pgfscope}%
\pgfsys@transformshift{4.788771in}{3.105264in}%
\pgfsys@useobject{currentmarker}{}%
\end{pgfscope}%
\begin{pgfscope}%
\pgfsys@transformshift{4.803486in}{3.168330in}%
\pgfsys@useobject{currentmarker}{}%
\end{pgfscope}%
\begin{pgfscope}%
\pgfsys@transformshift{4.818201in}{3.245822in}%
\pgfsys@useobject{currentmarker}{}%
\end{pgfscope}%
\begin{pgfscope}%
\pgfsys@transformshift{4.832916in}{3.207878in}%
\pgfsys@useobject{currentmarker}{}%
\end{pgfscope}%
\begin{pgfscope}%
\pgfsys@transformshift{4.847632in}{3.081794in}%
\pgfsys@useobject{currentmarker}{}%
\end{pgfscope}%
\begin{pgfscope}%
\pgfsys@transformshift{4.862347in}{2.994873in}%
\pgfsys@useobject{currentmarker}{}%
\end{pgfscope}%
\begin{pgfscope}%
\pgfsys@transformshift{4.877062in}{2.942665in}%
\pgfsys@useobject{currentmarker}{}%
\end{pgfscope}%
\begin{pgfscope}%
\pgfsys@transformshift{4.891778in}{2.927964in}%
\pgfsys@useobject{currentmarker}{}%
\end{pgfscope}%
\begin{pgfscope}%
\pgfsys@transformshift{4.906493in}{2.999466in}%
\pgfsys@useobject{currentmarker}{}%
\end{pgfscope}%
\begin{pgfscope}%
\pgfsys@transformshift{4.921208in}{3.074662in}%
\pgfsys@useobject{currentmarker}{}%
\end{pgfscope}%
\begin{pgfscope}%
\pgfsys@transformshift{4.935924in}{3.154312in}%
\pgfsys@useobject{currentmarker}{}%
\end{pgfscope}%
\begin{pgfscope}%
\pgfsys@transformshift{4.950639in}{3.185292in}%
\pgfsys@useobject{currentmarker}{}%
\end{pgfscope}%
\begin{pgfscope}%
\pgfsys@transformshift{4.965354in}{3.260631in}%
\pgfsys@useobject{currentmarker}{}%
\end{pgfscope}%
\begin{pgfscope}%
\pgfsys@transformshift{4.980070in}{3.301160in}%
\pgfsys@useobject{currentmarker}{}%
\end{pgfscope}%
\begin{pgfscope}%
\pgfsys@transformshift{4.994785in}{3.335016in}%
\pgfsys@useobject{currentmarker}{}%
\end{pgfscope}%
\begin{pgfscope}%
\pgfsys@transformshift{5.009500in}{3.298668in}%
\pgfsys@useobject{currentmarker}{}%
\end{pgfscope}%
\begin{pgfscope}%
\pgfsys@transformshift{5.024216in}{3.215500in}%
\pgfsys@useobject{currentmarker}{}%
\end{pgfscope}%
\begin{pgfscope}%
\pgfsys@transformshift{5.038931in}{3.104630in}%
\pgfsys@useobject{currentmarker}{}%
\end{pgfscope}%
\begin{pgfscope}%
\pgfsys@transformshift{5.053646in}{3.028827in}%
\pgfsys@useobject{currentmarker}{}%
\end{pgfscope}%
\begin{pgfscope}%
\pgfsys@transformshift{5.068362in}{3.012287in}%
\pgfsys@useobject{currentmarker}{}%
\end{pgfscope}%
\begin{pgfscope}%
\pgfsys@transformshift{5.083077in}{3.101575in}%
\pgfsys@useobject{currentmarker}{}%
\end{pgfscope}%
\begin{pgfscope}%
\pgfsys@transformshift{5.097792in}{3.178545in}%
\pgfsys@useobject{currentmarker}{}%
\end{pgfscope}%
\begin{pgfscope}%
\pgfsys@transformshift{5.112507in}{3.253105in}%
\pgfsys@useobject{currentmarker}{}%
\end{pgfscope}%
\begin{pgfscope}%
\pgfsys@transformshift{5.127223in}{3.337095in}%
\pgfsys@useobject{currentmarker}{}%
\end{pgfscope}%
\begin{pgfscope}%
\pgfsys@transformshift{5.141938in}{3.392402in}%
\pgfsys@useobject{currentmarker}{}%
\end{pgfscope}%
\begin{pgfscope}%
\pgfsys@transformshift{5.156653in}{3.465701in}%
\pgfsys@useobject{currentmarker}{}%
\end{pgfscope}%
\begin{pgfscope}%
\pgfsys@transformshift{5.171369in}{3.495727in}%
\pgfsys@useobject{currentmarker}{}%
\end{pgfscope}%
\begin{pgfscope}%
\pgfsys@transformshift{5.186084in}{3.450047in}%
\pgfsys@useobject{currentmarker}{}%
\end{pgfscope}%
\begin{pgfscope}%
\pgfsys@transformshift{5.200799in}{3.338679in}%
\pgfsys@useobject{currentmarker}{}%
\end{pgfscope}%
\begin{pgfscope}%
\pgfsys@transformshift{5.215515in}{3.233330in}%
\pgfsys@useobject{currentmarker}{}%
\end{pgfscope}%
\begin{pgfscope}%
\pgfsys@transformshift{5.230230in}{3.146527in}%
\pgfsys@useobject{currentmarker}{}%
\end{pgfscope}%
\begin{pgfscope}%
\pgfsys@transformshift{5.244945in}{3.169610in}%
\pgfsys@useobject{currentmarker}{}%
\end{pgfscope}%
\begin{pgfscope}%
\pgfsys@transformshift{5.259661in}{3.249291in}%
\pgfsys@useobject{currentmarker}{}%
\end{pgfscope}%
\begin{pgfscope}%
\pgfsys@transformshift{5.274376in}{3.313777in}%
\pgfsys@useobject{currentmarker}{}%
\end{pgfscope}%
\begin{pgfscope}%
\pgfsys@transformshift{5.289091in}{3.399271in}%
\pgfsys@useobject{currentmarker}{}%
\end{pgfscope}%
\begin{pgfscope}%
\pgfsys@transformshift{5.303807in}{3.430567in}%
\pgfsys@useobject{currentmarker}{}%
\end{pgfscope}%
\begin{pgfscope}%
\pgfsys@transformshift{5.318522in}{3.471233in}%
\pgfsys@useobject{currentmarker}{}%
\end{pgfscope}%
\begin{pgfscope}%
\pgfsys@transformshift{5.333237in}{3.507600in}%
\pgfsys@useobject{currentmarker}{}%
\end{pgfscope}%
\begin{pgfscope}%
\pgfsys@transformshift{5.347953in}{3.563484in}%
\pgfsys@useobject{currentmarker}{}%
\end{pgfscope}%
\begin{pgfscope}%
\pgfsys@transformshift{5.362668in}{3.535711in}%
\pgfsys@useobject{currentmarker}{}%
\end{pgfscope}%
\begin{pgfscope}%
\pgfsys@transformshift{5.377383in}{3.468350in}%
\pgfsys@useobject{currentmarker}{}%
\end{pgfscope}%
\begin{pgfscope}%
\pgfsys@transformshift{5.392098in}{3.336385in}%
\pgfsys@useobject{currentmarker}{}%
\end{pgfscope}%
\begin{pgfscope}%
\pgfsys@transformshift{5.406814in}{3.272073in}%
\pgfsys@useobject{currentmarker}{}%
\end{pgfscope}%
\begin{pgfscope}%
\pgfsys@transformshift{5.421529in}{3.269405in}%
\pgfsys@useobject{currentmarker}{}%
\end{pgfscope}%
\begin{pgfscope}%
\pgfsys@transformshift{5.436244in}{3.342383in}%
\pgfsys@useobject{currentmarker}{}%
\end{pgfscope}%
\begin{pgfscope}%
\pgfsys@transformshift{5.450960in}{3.429823in}%
\pgfsys@useobject{currentmarker}{}%
\end{pgfscope}%
\begin{pgfscope}%
\pgfsys@transformshift{5.465675in}{3.501866in}%
\pgfsys@useobject{currentmarker}{}%
\end{pgfscope}%
\begin{pgfscope}%
\pgfsys@transformshift{5.480390in}{3.543338in}%
\pgfsys@useobject{currentmarker}{}%
\end{pgfscope}%
\begin{pgfscope}%
\pgfsys@transformshift{5.495106in}{3.573237in}%
\pgfsys@useobject{currentmarker}{}%
\end{pgfscope}%
\begin{pgfscope}%
\pgfsys@transformshift{5.509821in}{3.675126in}%
\pgfsys@useobject{currentmarker}{}%
\end{pgfscope}%
\begin{pgfscope}%
\pgfsys@transformshift{5.524536in}{3.705927in}%
\pgfsys@useobject{currentmarker}{}%
\end{pgfscope}%
\begin{pgfscope}%
\pgfsys@transformshift{5.539252in}{3.648982in}%
\pgfsys@useobject{currentmarker}{}%
\end{pgfscope}%
\begin{pgfscope}%
\pgfsys@transformshift{5.553967in}{3.572721in}%
\pgfsys@useobject{currentmarker}{}%
\end{pgfscope}%
\begin{pgfscope}%
\pgfsys@transformshift{5.568682in}{3.459462in}%
\pgfsys@useobject{currentmarker}{}%
\end{pgfscope}%
\begin{pgfscope}%
\pgfsys@transformshift{5.583398in}{3.388437in}%
\pgfsys@useobject{currentmarker}{}%
\end{pgfscope}%
\begin{pgfscope}%
\pgfsys@transformshift{5.598113in}{3.386364in}%
\pgfsys@useobject{currentmarker}{}%
\end{pgfscope}%
\begin{pgfscope}%
\pgfsys@transformshift{5.612828in}{3.496506in}%
\pgfsys@useobject{currentmarker}{}%
\end{pgfscope}%
\begin{pgfscope}%
\pgfsys@transformshift{5.627544in}{3.572323in}%
\pgfsys@useobject{currentmarker}{}%
\end{pgfscope}%
\begin{pgfscope}%
\pgfsys@transformshift{5.642259in}{3.641984in}%
\pgfsys@useobject{currentmarker}{}%
\end{pgfscope}%
\begin{pgfscope}%
\pgfsys@transformshift{5.656974in}{3.711491in}%
\pgfsys@useobject{currentmarker}{}%
\end{pgfscope}%
\begin{pgfscope}%
\pgfsys@transformshift{5.671689in}{3.747861in}%
\pgfsys@useobject{currentmarker}{}%
\end{pgfscope}%
\begin{pgfscope}%
\pgfsys@transformshift{5.686405in}{3.801211in}%
\pgfsys@useobject{currentmarker}{}%
\end{pgfscope}%
\begin{pgfscope}%
\pgfsys@transformshift{5.701120in}{3.880383in}%
\pgfsys@useobject{currentmarker}{}%
\end{pgfscope}%
\begin{pgfscope}%
\pgfsys@transformshift{5.715835in}{3.814365in}%
\pgfsys@useobject{currentmarker}{}%
\end{pgfscope}%
\begin{pgfscope}%
\pgfsys@transformshift{5.730551in}{3.735358in}%
\pgfsys@useobject{currentmarker}{}%
\end{pgfscope}%
\begin{pgfscope}%
\pgfsys@transformshift{5.745266in}{3.619144in}%
\pgfsys@useobject{currentmarker}{}%
\end{pgfscope}%
\begin{pgfscope}%
\pgfsys@transformshift{5.759981in}{3.525009in}%
\pgfsys@useobject{currentmarker}{}%
\end{pgfscope}%
\begin{pgfscope}%
\pgfsys@transformshift{5.774697in}{3.535046in}%
\pgfsys@useobject{currentmarker}{}%
\end{pgfscope}%
\begin{pgfscope}%
\pgfsys@transformshift{5.789412in}{3.617279in}%
\pgfsys@useobject{currentmarker}{}%
\end{pgfscope}%
\begin{pgfscope}%
\pgfsys@transformshift{5.804127in}{3.711197in}%
\pgfsys@useobject{currentmarker}{}%
\end{pgfscope}%
\begin{pgfscope}%
\pgfsys@transformshift{5.818843in}{3.768746in}%
\pgfsys@useobject{currentmarker}{}%
\end{pgfscope}%
\begin{pgfscope}%
\pgfsys@transformshift{5.833558in}{3.781488in}%
\pgfsys@useobject{currentmarker}{}%
\end{pgfscope}%
\begin{pgfscope}%
\pgfsys@transformshift{5.848273in}{3.873330in}%
\pgfsys@useobject{currentmarker}{}%
\end{pgfscope}%
\begin{pgfscope}%
\pgfsys@transformshift{5.862989in}{3.963302in}%
\pgfsys@useobject{currentmarker}{}%
\end{pgfscope}%
\begin{pgfscope}%
\pgfsys@transformshift{5.877704in}{3.989235in}%
\pgfsys@useobject{currentmarker}{}%
\end{pgfscope}%
\begin{pgfscope}%
\pgfsys@transformshift{5.892419in}{3.960763in}%
\pgfsys@useobject{currentmarker}{}%
\end{pgfscope}%
\begin{pgfscope}%
\pgfsys@transformshift{5.907135in}{3.843413in}%
\pgfsys@useobject{currentmarker}{}%
\end{pgfscope}%
\begin{pgfscope}%
\pgfsys@transformshift{5.921850in}{3.718927in}%
\pgfsys@useobject{currentmarker}{}%
\end{pgfscope}%
\begin{pgfscope}%
\pgfsys@transformshift{5.936565in}{3.625038in}%
\pgfsys@useobject{currentmarker}{}%
\end{pgfscope}%
\begin{pgfscope}%
\pgfsys@transformshift{5.951280in}{3.660836in}%
\pgfsys@useobject{currentmarker}{}%
\end{pgfscope}%
\begin{pgfscope}%
\pgfsys@transformshift{5.965996in}{3.731089in}%
\pgfsys@useobject{currentmarker}{}%
\end{pgfscope}%
\begin{pgfscope}%
\pgfsys@transformshift{5.980711in}{3.828319in}%
\pgfsys@useobject{currentmarker}{}%
\end{pgfscope}%
\begin{pgfscope}%
\pgfsys@transformshift{5.995426in}{3.889209in}%
\pgfsys@useobject{currentmarker}{}%
\end{pgfscope}%
\begin{pgfscope}%
\pgfsys@transformshift{6.010142in}{3.910510in}%
\pgfsys@useobject{currentmarker}{}%
\end{pgfscope}%
\begin{pgfscope}%
\pgfsys@transformshift{6.024857in}{3.979245in}%
\pgfsys@useobject{currentmarker}{}%
\end{pgfscope}%
\begin{pgfscope}%
\pgfsys@transformshift{6.039572in}{4.069973in}%
\pgfsys@useobject{currentmarker}{}%
\end{pgfscope}%
\begin{pgfscope}%
\pgfsys@transformshift{6.054288in}{4.113220in}%
\pgfsys@useobject{currentmarker}{}%
\end{pgfscope}%
\begin{pgfscope}%
\pgfsys@transformshift{6.069003in}{4.045685in}%
\pgfsys@useobject{currentmarker}{}%
\end{pgfscope}%
\begin{pgfscope}%
\pgfsys@transformshift{6.083718in}{3.962188in}%
\pgfsys@useobject{currentmarker}{}%
\end{pgfscope}%
\begin{pgfscope}%
\pgfsys@transformshift{6.098434in}{3.827146in}%
\pgfsys@useobject{currentmarker}{}%
\end{pgfscope}%
\begin{pgfscope}%
\pgfsys@transformshift{6.113149in}{3.756187in}%
\pgfsys@useobject{currentmarker}{}%
\end{pgfscope}%
\begin{pgfscope}%
\pgfsys@transformshift{6.127864in}{3.794558in}%
\pgfsys@useobject{currentmarker}{}%
\end{pgfscope}%
\begin{pgfscope}%
\pgfsys@transformshift{6.142580in}{3.895673in}%
\pgfsys@useobject{currentmarker}{}%
\end{pgfscope}%
\begin{pgfscope}%
\pgfsys@transformshift{6.157295in}{3.995888in}%
\pgfsys@useobject{currentmarker}{}%
\end{pgfscope}%
\begin{pgfscope}%
\pgfsys@transformshift{6.172010in}{4.034614in}%
\pgfsys@useobject{currentmarker}{}%
\end{pgfscope}%
\begin{pgfscope}%
\pgfsys@transformshift{6.186726in}{4.121554in}%
\pgfsys@useobject{currentmarker}{}%
\end{pgfscope}%
\begin{pgfscope}%
\pgfsys@transformshift{6.201441in}{4.164638in}%
\pgfsys@useobject{currentmarker}{}%
\end{pgfscope}%
\begin{pgfscope}%
\pgfsys@transformshift{6.216156in}{4.309818in}%
\pgfsys@useobject{currentmarker}{}%
\end{pgfscope}%
\begin{pgfscope}%
\pgfsys@transformshift{6.230871in}{4.325629in}%
\pgfsys@useobject{currentmarker}{}%
\end{pgfscope}%
\begin{pgfscope}%
\pgfsys@transformshift{6.245587in}{4.274806in}%
\pgfsys@useobject{currentmarker}{}%
\end{pgfscope}%
\begin{pgfscope}%
\pgfsys@transformshift{6.260302in}{4.134470in}%
\pgfsys@useobject{currentmarker}{}%
\end{pgfscope}%
\begin{pgfscope}%
\pgfsys@transformshift{6.275017in}{4.016864in}%
\pgfsys@useobject{currentmarker}{}%
\end{pgfscope}%
\begin{pgfscope}%
\pgfsys@transformshift{6.289733in}{3.948910in}%
\pgfsys@useobject{currentmarker}{}%
\end{pgfscope}%
\begin{pgfscope}%
\pgfsys@transformshift{6.304448in}{3.980245in}%
\pgfsys@useobject{currentmarker}{}%
\end{pgfscope}%
\begin{pgfscope}%
\pgfsys@transformshift{6.319163in}{4.089049in}%
\pgfsys@useobject{currentmarker}{}%
\end{pgfscope}%
\begin{pgfscope}%
\pgfsys@transformshift{6.333879in}{4.142277in}%
\pgfsys@useobject{currentmarker}{}%
\end{pgfscope}%
\begin{pgfscope}%
\pgfsys@transformshift{6.348594in}{4.236713in}%
\pgfsys@useobject{currentmarker}{}%
\end{pgfscope}%
\begin{pgfscope}%
\pgfsys@transformshift{6.363309in}{4.254219in}%
\pgfsys@useobject{currentmarker}{}%
\end{pgfscope}%
\begin{pgfscope}%
\pgfsys@transformshift{6.378025in}{4.307159in}%
\pgfsys@useobject{currentmarker}{}%
\end{pgfscope}%
\begin{pgfscope}%
\pgfsys@transformshift{6.392740in}{4.400504in}%
\pgfsys@useobject{currentmarker}{}%
\end{pgfscope}%
\begin{pgfscope}%
\pgfsys@transformshift{6.407455in}{4.437431in}%
\pgfsys@useobject{currentmarker}{}%
\end{pgfscope}%
\begin{pgfscope}%
\pgfsys@transformshift{6.422171in}{4.392230in}%
\pgfsys@useobject{currentmarker}{}%
\end{pgfscope}%
\begin{pgfscope}%
\pgfsys@transformshift{6.436886in}{4.292025in}%
\pgfsys@useobject{currentmarker}{}%
\end{pgfscope}%
\begin{pgfscope}%
\pgfsys@transformshift{6.451601in}{4.177460in}%
\pgfsys@useobject{currentmarker}{}%
\end{pgfscope}%
\begin{pgfscope}%
\pgfsys@transformshift{6.466317in}{4.081765in}%
\pgfsys@useobject{currentmarker}{}%
\end{pgfscope}%
\begin{pgfscope}%
\pgfsys@transformshift{6.481032in}{4.094528in}%
\pgfsys@useobject{currentmarker}{}%
\end{pgfscope}%
\begin{pgfscope}%
\pgfsys@transformshift{6.495747in}{4.175039in}%
\pgfsys@useobject{currentmarker}{}%
\end{pgfscope}%
\begin{pgfscope}%
\pgfsys@transformshift{6.510463in}{4.274353in}%
\pgfsys@useobject{currentmarker}{}%
\end{pgfscope}%
\begin{pgfscope}%
\pgfsys@transformshift{6.525178in}{4.339667in}%
\pgfsys@useobject{currentmarker}{}%
\end{pgfscope}%
\begin{pgfscope}%
\pgfsys@transformshift{6.539893in}{4.360346in}%
\pgfsys@useobject{currentmarker}{}%
\end{pgfscope}%
\begin{pgfscope}%
\pgfsys@transformshift{6.554608in}{4.419924in}%
\pgfsys@useobject{currentmarker}{}%
\end{pgfscope}%
\begin{pgfscope}%
\pgfsys@transformshift{6.569324in}{4.468421in}%
\pgfsys@useobject{currentmarker}{}%
\end{pgfscope}%
\begin{pgfscope}%
\pgfsys@transformshift{6.584039in}{4.524851in}%
\pgfsys@useobject{currentmarker}{}%
\end{pgfscope}%
\begin{pgfscope}%
\pgfsys@transformshift{6.598754in}{4.500336in}%
\pgfsys@useobject{currentmarker}{}%
\end{pgfscope}%
\begin{pgfscope}%
\pgfsys@transformshift{6.613470in}{4.379127in}%
\pgfsys@useobject{currentmarker}{}%
\end{pgfscope}%
\begin{pgfscope}%
\pgfsys@transformshift{6.628185in}{4.283351in}%
\pgfsys@useobject{currentmarker}{}%
\end{pgfscope}%
\begin{pgfscope}%
\pgfsys@transformshift{6.642900in}{4.202149in}%
\pgfsys@useobject{currentmarker}{}%
\end{pgfscope}%
\begin{pgfscope}%
\pgfsys@transformshift{6.657616in}{4.228745in}%
\pgfsys@useobject{currentmarker}{}%
\end{pgfscope}%
\begin{pgfscope}%
\pgfsys@transformshift{6.672331in}{4.346402in}%
\pgfsys@useobject{currentmarker}{}%
\end{pgfscope}%
\begin{pgfscope}%
\pgfsys@transformshift{6.687046in}{4.403463in}%
\pgfsys@useobject{currentmarker}{}%
\end{pgfscope}%
\begin{pgfscope}%
\pgfsys@transformshift{6.701762in}{4.502812in}%
\pgfsys@useobject{currentmarker}{}%
\end{pgfscope}%
\begin{pgfscope}%
\pgfsys@transformshift{6.716477in}{4.554563in}%
\pgfsys@useobject{currentmarker}{}%
\end{pgfscope}%
\begin{pgfscope}%
\pgfsys@transformshift{6.731192in}{4.564685in}%
\pgfsys@useobject{currentmarker}{}%
\end{pgfscope}%
\begin{pgfscope}%
\pgfsys@transformshift{6.745908in}{4.644779in}%
\pgfsys@useobject{currentmarker}{}%
\end{pgfscope}%
\begin{pgfscope}%
\pgfsys@transformshift{6.760623in}{4.718489in}%
\pgfsys@useobject{currentmarker}{}%
\end{pgfscope}%
\begin{pgfscope}%
\pgfsys@transformshift{6.775338in}{4.677779in}%
\pgfsys@useobject{currentmarker}{}%
\end{pgfscope}%
\begin{pgfscope}%
\pgfsys@transformshift{6.790054in}{4.557252in}%
\pgfsys@useobject{currentmarker}{}%
\end{pgfscope}%
\begin{pgfscope}%
\pgfsys@transformshift{6.804769in}{4.455691in}%
\pgfsys@useobject{currentmarker}{}%
\end{pgfscope}%
\begin{pgfscope}%
\pgfsys@transformshift{6.819484in}{4.373935in}%
\pgfsys@useobject{currentmarker}{}%
\end{pgfscope}%
\begin{pgfscope}%
\pgfsys@transformshift{6.834199in}{4.371754in}%
\pgfsys@useobject{currentmarker}{}%
\end{pgfscope}%
\begin{pgfscope}%
\pgfsys@transformshift{6.848915in}{4.472495in}%
\pgfsys@useobject{currentmarker}{}%
\end{pgfscope}%
\begin{pgfscope}%
\pgfsys@transformshift{6.863630in}{4.554833in}%
\pgfsys@useobject{currentmarker}{}%
\end{pgfscope}%
\begin{pgfscope}%
\pgfsys@transformshift{6.878345in}{4.647259in}%
\pgfsys@useobject{currentmarker}{}%
\end{pgfscope}%
\begin{pgfscope}%
\pgfsys@transformshift{6.893061in}{4.688422in}%
\pgfsys@useobject{currentmarker}{}%
\end{pgfscope}%
\begin{pgfscope}%
\pgfsys@transformshift{6.907776in}{4.708840in}%
\pgfsys@useobject{currentmarker}{}%
\end{pgfscope}%
\begin{pgfscope}%
\pgfsys@transformshift{6.922491in}{4.806492in}%
\pgfsys@useobject{currentmarker}{}%
\end{pgfscope}%
\begin{pgfscope}%
\pgfsys@transformshift{6.937207in}{4.855076in}%
\pgfsys@useobject{currentmarker}{}%
\end{pgfscope}%
\begin{pgfscope}%
\pgfsys@transformshift{6.951922in}{4.812916in}%
\pgfsys@useobject{currentmarker}{}%
\end{pgfscope}%
\begin{pgfscope}%
\pgfsys@transformshift{6.966637in}{4.703776in}%
\pgfsys@useobject{currentmarker}{}%
\end{pgfscope}%
\begin{pgfscope}%
\pgfsys@transformshift{6.981353in}{4.594315in}%
\pgfsys@useobject{currentmarker}{}%
\end{pgfscope}%
\begin{pgfscope}%
\pgfsys@transformshift{6.996068in}{4.528666in}%
\pgfsys@useobject{currentmarker}{}%
\end{pgfscope}%
\begin{pgfscope}%
\pgfsys@transformshift{7.010783in}{4.527288in}%
\pgfsys@useobject{currentmarker}{}%
\end{pgfscope}%
\begin{pgfscope}%
\pgfsys@transformshift{7.025499in}{4.618743in}%
\pgfsys@useobject{currentmarker}{}%
\end{pgfscope}%
\begin{pgfscope}%
\pgfsys@transformshift{7.040214in}{4.681999in}%
\pgfsys@useobject{currentmarker}{}%
\end{pgfscope}%
\begin{pgfscope}%
\pgfsys@transformshift{7.054929in}{4.753762in}%
\pgfsys@useobject{currentmarker}{}%
\end{pgfscope}%
\begin{pgfscope}%
\pgfsys@transformshift{7.069645in}{4.823290in}%
\pgfsys@useobject{currentmarker}{}%
\end{pgfscope}%
\begin{pgfscope}%
\pgfsys@transformshift{7.084360in}{4.871264in}%
\pgfsys@useobject{currentmarker}{}%
\end{pgfscope}%
\begin{pgfscope}%
\pgfsys@transformshift{7.099075in}{4.952236in}%
\pgfsys@useobject{currentmarker}{}%
\end{pgfscope}%
\begin{pgfscope}%
\pgfsys@transformshift{7.113790in}{4.956988in}%
\pgfsys@useobject{currentmarker}{}%
\end{pgfscope}%
\begin{pgfscope}%
\pgfsys@transformshift{7.128506in}{4.948991in}%
\pgfsys@useobject{currentmarker}{}%
\end{pgfscope}%
\begin{pgfscope}%
\pgfsys@transformshift{7.143221in}{4.833328in}%
\pgfsys@useobject{currentmarker}{}%
\end{pgfscope}%
\begin{pgfscope}%
\pgfsys@transformshift{7.157936in}{4.699157in}%
\pgfsys@useobject{currentmarker}{}%
\end{pgfscope}%
\begin{pgfscope}%
\pgfsys@transformshift{7.172652in}{4.628028in}%
\pgfsys@useobject{currentmarker}{}%
\end{pgfscope}%
\begin{pgfscope}%
\pgfsys@transformshift{7.187367in}{4.662816in}%
\pgfsys@useobject{currentmarker}{}%
\end{pgfscope}%
\begin{pgfscope}%
\pgfsys@transformshift{7.202082in}{4.724724in}%
\pgfsys@useobject{currentmarker}{}%
\end{pgfscope}%
\begin{pgfscope}%
\pgfsys@transformshift{7.216798in}{4.821273in}%
\pgfsys@useobject{currentmarker}{}%
\end{pgfscope}%
\begin{pgfscope}%
\pgfsys@transformshift{7.231513in}{4.902077in}%
\pgfsys@useobject{currentmarker}{}%
\end{pgfscope}%
\begin{pgfscope}%
\pgfsys@transformshift{7.246228in}{4.965599in}%
\pgfsys@useobject{currentmarker}{}%
\end{pgfscope}%
\begin{pgfscope}%
\pgfsys@transformshift{7.260944in}{4.938496in}%
\pgfsys@useobject{currentmarker}{}%
\end{pgfscope}%
\begin{pgfscope}%
\pgfsys@transformshift{7.275659in}{5.020046in}%
\pgfsys@useobject{currentmarker}{}%
\end{pgfscope}%
\begin{pgfscope}%
\pgfsys@transformshift{7.290374in}{5.063044in}%
\pgfsys@useobject{currentmarker}{}%
\end{pgfscope}%
\begin{pgfscope}%
\pgfsys@transformshift{7.305090in}{5.061003in}%
\pgfsys@useobject{currentmarker}{}%
\end{pgfscope}%
\begin{pgfscope}%
\pgfsys@transformshift{7.319805in}{4.943641in}%
\pgfsys@useobject{currentmarker}{}%
\end{pgfscope}%
\begin{pgfscope}%
\pgfsys@transformshift{7.334520in}{4.849158in}%
\pgfsys@useobject{currentmarker}{}%
\end{pgfscope}%
\begin{pgfscope}%
\pgfsys@transformshift{7.349236in}{4.776324in}%
\pgfsys@useobject{currentmarker}{}%
\end{pgfscope}%
\begin{pgfscope}%
\pgfsys@transformshift{7.363951in}{4.767053in}%
\pgfsys@useobject{currentmarker}{}%
\end{pgfscope}%
\begin{pgfscope}%
\pgfsys@transformshift{7.378666in}{4.865884in}%
\pgfsys@useobject{currentmarker}{}%
\end{pgfscope}%
\begin{pgfscope}%
\pgfsys@transformshift{7.393381in}{4.980627in}%
\pgfsys@useobject{currentmarker}{}%
\end{pgfscope}%
\begin{pgfscope}%
\pgfsys@transformshift{7.408097in}{5.077119in}%
\pgfsys@useobject{currentmarker}{}%
\end{pgfscope}%
\begin{pgfscope}%
\pgfsys@transformshift{7.422812in}{5.145777in}%
\pgfsys@useobject{currentmarker}{}%
\end{pgfscope}%
\begin{pgfscope}%
\pgfsys@transformshift{7.437527in}{5.198205in}%
\pgfsys@useobject{currentmarker}{}%
\end{pgfscope}%
\begin{pgfscope}%
\pgfsys@transformshift{7.452243in}{5.222266in}%
\pgfsys@useobject{currentmarker}{}%
\end{pgfscope}%
\begin{pgfscope}%
\pgfsys@transformshift{7.466958in}{5.216280in}%
\pgfsys@useobject{currentmarker}{}%
\end{pgfscope}%
\begin{pgfscope}%
\pgfsys@transformshift{7.481673in}{5.177487in}%
\pgfsys@useobject{currentmarker}{}%
\end{pgfscope}%
\begin{pgfscope}%
\pgfsys@transformshift{7.496389in}{5.100016in}%
\pgfsys@useobject{currentmarker}{}%
\end{pgfscope}%
\begin{pgfscope}%
\pgfsys@transformshift{7.511104in}{4.992755in}%
\pgfsys@useobject{currentmarker}{}%
\end{pgfscope}%
\begin{pgfscope}%
\pgfsys@transformshift{7.525819in}{4.918483in}%
\pgfsys@useobject{currentmarker}{}%
\end{pgfscope}%
\begin{pgfscope}%
\pgfsys@transformshift{7.540535in}{4.942741in}%
\pgfsys@useobject{currentmarker}{}%
\end{pgfscope}%
\begin{pgfscope}%
\pgfsys@transformshift{7.555250in}{5.028274in}%
\pgfsys@useobject{currentmarker}{}%
\end{pgfscope}%
\begin{pgfscope}%
\pgfsys@transformshift{7.569965in}{5.115708in}%
\pgfsys@useobject{currentmarker}{}%
\end{pgfscope}%
\begin{pgfscope}%
\pgfsys@transformshift{7.584681in}{5.168722in}%
\pgfsys@useobject{currentmarker}{}%
\end{pgfscope}%
\begin{pgfscope}%
\pgfsys@transformshift{7.599396in}{5.266526in}%
\pgfsys@useobject{currentmarker}{}%
\end{pgfscope}%
\begin{pgfscope}%
\pgfsys@transformshift{7.614111in}{5.311312in}%
\pgfsys@useobject{currentmarker}{}%
\end{pgfscope}%
\begin{pgfscope}%
\pgfsys@transformshift{7.628827in}{5.379659in}%
\pgfsys@useobject{currentmarker}{}%
\end{pgfscope}%
\begin{pgfscope}%
\pgfsys@transformshift{7.643542in}{5.397878in}%
\pgfsys@useobject{currentmarker}{}%
\end{pgfscope}%
\begin{pgfscope}%
\pgfsys@transformshift{7.658257in}{5.396051in}%
\pgfsys@useobject{currentmarker}{}%
\end{pgfscope}%
\begin{pgfscope}%
\pgfsys@transformshift{7.672972in}{5.327932in}%
\pgfsys@useobject{currentmarker}{}%
\end{pgfscope}%
\begin{pgfscope}%
\pgfsys@transformshift{7.687688in}{5.176902in}%
\pgfsys@useobject{currentmarker}{}%
\end{pgfscope}%
\begin{pgfscope}%
\pgfsys@transformshift{7.702403in}{5.124796in}%
\pgfsys@useobject{currentmarker}{}%
\end{pgfscope}%
\begin{pgfscope}%
\pgfsys@transformshift{7.717118in}{5.145000in}%
\pgfsys@useobject{currentmarker}{}%
\end{pgfscope}%
\begin{pgfscope}%
\pgfsys@transformshift{7.731834in}{5.227074in}%
\pgfsys@useobject{currentmarker}{}%
\end{pgfscope}%
\begin{pgfscope}%
\pgfsys@transformshift{7.746549in}{5.314122in}%
\pgfsys@useobject{currentmarker}{}%
\end{pgfscope}%
\end{pgfscope}%
\begin{pgfscope}%
\pgfpathrectangle{\pgfqpoint{0.697913in}{0.559721in}}{\pgfqpoint{7.048636in}{4.990279in}}%
\pgfusepath{clip}%
\pgfsetrectcap%
\pgfsetroundjoin%
\pgfsetlinewidth{1.003750pt}%
\definecolor{currentstroke}{rgb}{0.000000,0.000000,0.000000}%
\pgfsetstrokecolor{currentstroke}%
\pgfsetdash{}{0pt}%
\pgfpathmoveto{\pgfqpoint{0.697913in}{0.772482in}}%
\pgfpathlineto{\pgfqpoint{0.712628in}{0.776708in}}%
\pgfpathlineto{\pgfqpoint{0.727343in}{0.778526in}}%
\pgfpathlineto{\pgfqpoint{0.742059in}{0.782814in}}%
\pgfpathlineto{\pgfqpoint{0.756774in}{0.793740in}}%
\pgfpathlineto{\pgfqpoint{0.771489in}{0.807925in}}%
\pgfpathlineto{\pgfqpoint{0.786205in}{0.824367in}}%
\pgfpathlineto{\pgfqpoint{0.800920in}{0.842143in}}%
\pgfpathlineto{\pgfqpoint{0.815635in}{0.850456in}}%
\pgfpathlineto{\pgfqpoint{0.830350in}{0.851458in}}%
\pgfpathlineto{\pgfqpoint{0.845066in}{0.850016in}}%
\pgfpathlineto{\pgfqpoint{0.859781in}{0.853475in}}%
\pgfpathlineto{\pgfqpoint{0.874496in}{0.855324in}}%
\pgfpathlineto{\pgfqpoint{0.903927in}{0.851477in}}%
\pgfpathlineto{\pgfqpoint{0.918642in}{0.854480in}}%
\pgfpathlineto{\pgfqpoint{0.933358in}{0.864389in}}%
\pgfpathlineto{\pgfqpoint{0.948073in}{0.879740in}}%
\pgfpathlineto{\pgfqpoint{0.962788in}{0.896539in}}%
\pgfpathlineto{\pgfqpoint{0.977504in}{0.910503in}}%
\pgfpathlineto{\pgfqpoint{0.992219in}{0.920809in}}%
\pgfpathlineto{\pgfqpoint{1.006934in}{0.920843in}}%
\pgfpathlineto{\pgfqpoint{1.021650in}{0.921381in}}%
\pgfpathlineto{\pgfqpoint{1.036365in}{0.924504in}}%
\pgfpathlineto{\pgfqpoint{1.051080in}{0.930748in}}%
\pgfpathlineto{\pgfqpoint{1.065796in}{0.934302in}}%
\pgfpathlineto{\pgfqpoint{1.080511in}{0.933994in}}%
\pgfpathlineto{\pgfqpoint{1.095226in}{0.938674in}}%
\pgfpathlineto{\pgfqpoint{1.109941in}{0.949747in}}%
\pgfpathlineto{\pgfqpoint{1.124657in}{0.967119in}}%
\pgfpathlineto{\pgfqpoint{1.139372in}{0.988431in}}%
\pgfpathlineto{\pgfqpoint{1.154087in}{1.002866in}}%
\pgfpathlineto{\pgfqpoint{1.168803in}{1.012367in}}%
\pgfpathlineto{\pgfqpoint{1.183518in}{1.015020in}}%
\pgfpathlineto{\pgfqpoint{1.198233in}{1.018469in}}%
\pgfpathlineto{\pgfqpoint{1.212949in}{1.021541in}}%
\pgfpathlineto{\pgfqpoint{1.227664in}{1.029429in}}%
\pgfpathlineto{\pgfqpoint{1.242379in}{1.039357in}}%
\pgfpathlineto{\pgfqpoint{1.257095in}{1.044620in}}%
\pgfpathlineto{\pgfqpoint{1.271810in}{1.052380in}}%
\pgfpathlineto{\pgfqpoint{1.286525in}{1.066091in}}%
\pgfpathlineto{\pgfqpoint{1.301241in}{1.083891in}}%
\pgfpathlineto{\pgfqpoint{1.315956in}{1.107352in}}%
\pgfpathlineto{\pgfqpoint{1.330671in}{1.126971in}}%
\pgfpathlineto{\pgfqpoint{1.345387in}{1.138285in}}%
\pgfpathlineto{\pgfqpoint{1.360102in}{1.145010in}}%
\pgfpathlineto{\pgfqpoint{1.389532in}{1.157016in}}%
\pgfpathlineto{\pgfqpoint{1.404248in}{1.163814in}}%
\pgfpathlineto{\pgfqpoint{1.418963in}{1.168754in}}%
\pgfpathlineto{\pgfqpoint{1.433678in}{1.170860in}}%
\pgfpathlineto{\pgfqpoint{1.448394in}{1.177881in}}%
\pgfpathlineto{\pgfqpoint{1.463109in}{1.188860in}}%
\pgfpathlineto{\pgfqpoint{1.477824in}{1.202468in}}%
\pgfpathlineto{\pgfqpoint{1.492540in}{1.221036in}}%
\pgfpathlineto{\pgfqpoint{1.507255in}{1.235942in}}%
\pgfpathlineto{\pgfqpoint{1.521970in}{1.239684in}}%
\pgfpathlineto{\pgfqpoint{1.536686in}{1.239129in}}%
\pgfpathlineto{\pgfqpoint{1.551401in}{1.238990in}}%
\pgfpathlineto{\pgfqpoint{1.566116in}{1.237994in}}%
\pgfpathlineto{\pgfqpoint{1.580832in}{1.240306in}}%
\pgfpathlineto{\pgfqpoint{1.595547in}{1.245559in}}%
\pgfpathlineto{\pgfqpoint{1.610262in}{1.245804in}}%
\pgfpathlineto{\pgfqpoint{1.624978in}{1.247963in}}%
\pgfpathlineto{\pgfqpoint{1.639693in}{1.258008in}}%
\pgfpathlineto{\pgfqpoint{1.654408in}{1.270090in}}%
\pgfpathlineto{\pgfqpoint{1.669124in}{1.285587in}}%
\pgfpathlineto{\pgfqpoint{1.683839in}{1.300576in}}%
\pgfpathlineto{\pgfqpoint{1.698554in}{1.305706in}}%
\pgfpathlineto{\pgfqpoint{1.713269in}{1.306321in}}%
\pgfpathlineto{\pgfqpoint{1.727985in}{1.307462in}}%
\pgfpathlineto{\pgfqpoint{1.742700in}{1.310348in}}%
\pgfpathlineto{\pgfqpoint{1.757415in}{1.309496in}}%
\pgfpathlineto{\pgfqpoint{1.772131in}{1.309777in}}%
\pgfpathlineto{\pgfqpoint{1.786846in}{1.314763in}}%
\pgfpathlineto{\pgfqpoint{1.801561in}{1.321906in}}%
\pgfpathlineto{\pgfqpoint{1.816277in}{1.336418in}}%
\pgfpathlineto{\pgfqpoint{1.845707in}{1.370317in}}%
\pgfpathlineto{\pgfqpoint{1.860423in}{1.382970in}}%
\pgfpathlineto{\pgfqpoint{1.875138in}{1.386201in}}%
\pgfpathlineto{\pgfqpoint{1.889853in}{1.382723in}}%
\pgfpathlineto{\pgfqpoint{1.904569in}{1.379957in}}%
\pgfpathlineto{\pgfqpoint{1.919284in}{1.381489in}}%
\pgfpathlineto{\pgfqpoint{1.933999in}{1.384160in}}%
\pgfpathlineto{\pgfqpoint{1.948715in}{1.383501in}}%
\pgfpathlineto{\pgfqpoint{1.963430in}{1.379303in}}%
\pgfpathlineto{\pgfqpoint{1.978145in}{1.379256in}}%
\pgfpathlineto{\pgfqpoint{1.992860in}{1.386849in}}%
\pgfpathlineto{\pgfqpoint{2.007576in}{1.402902in}}%
\pgfpathlineto{\pgfqpoint{2.022291in}{1.419951in}}%
\pgfpathlineto{\pgfqpoint{2.037006in}{1.434046in}}%
\pgfpathlineto{\pgfqpoint{2.051722in}{1.437016in}}%
\pgfpathlineto{\pgfqpoint{2.066437in}{1.434647in}}%
\pgfpathlineto{\pgfqpoint{2.081152in}{1.430399in}}%
\pgfpathlineto{\pgfqpoint{2.095868in}{1.426973in}}%
\pgfpathlineto{\pgfqpoint{2.110583in}{1.426453in}}%
\pgfpathlineto{\pgfqpoint{2.125298in}{1.423711in}}%
\pgfpathlineto{\pgfqpoint{2.140014in}{1.420511in}}%
\pgfpathlineto{\pgfqpoint{2.154729in}{1.418902in}}%
\pgfpathlineto{\pgfqpoint{2.169444in}{1.423368in}}%
\pgfpathlineto{\pgfqpoint{2.184160in}{1.436166in}}%
\pgfpathlineto{\pgfqpoint{2.198875in}{1.448575in}}%
\pgfpathlineto{\pgfqpoint{2.213590in}{1.462882in}}%
\pgfpathlineto{\pgfqpoint{2.228306in}{1.466387in}}%
\pgfpathlineto{\pgfqpoint{2.272451in}{1.464491in}}%
\pgfpathlineto{\pgfqpoint{2.287167in}{1.469085in}}%
\pgfpathlineto{\pgfqpoint{2.301882in}{1.471441in}}%
\pgfpathlineto{\pgfqpoint{2.316597in}{1.475020in}}%
\pgfpathlineto{\pgfqpoint{2.331313in}{1.480577in}}%
\pgfpathlineto{\pgfqpoint{2.346028in}{1.491564in}}%
\pgfpathlineto{\pgfqpoint{2.360743in}{1.509390in}}%
\pgfpathlineto{\pgfqpoint{2.375459in}{1.528562in}}%
\pgfpathlineto{\pgfqpoint{2.390174in}{1.546640in}}%
\pgfpathlineto{\pgfqpoint{2.404889in}{1.554886in}}%
\pgfpathlineto{\pgfqpoint{2.434320in}{1.561450in}}%
\pgfpathlineto{\pgfqpoint{2.449035in}{1.565338in}}%
\pgfpathlineto{\pgfqpoint{2.463751in}{1.570401in}}%
\pgfpathlineto{\pgfqpoint{2.478466in}{1.575020in}}%
\pgfpathlineto{\pgfqpoint{2.493181in}{1.577950in}}%
\pgfpathlineto{\pgfqpoint{2.507897in}{1.586356in}}%
\pgfpathlineto{\pgfqpoint{2.522612in}{1.601049in}}%
\pgfpathlineto{\pgfqpoint{2.537327in}{1.620688in}}%
\pgfpathlineto{\pgfqpoint{2.552042in}{1.641588in}}%
\pgfpathlineto{\pgfqpoint{2.566758in}{1.659396in}}%
\pgfpathlineto{\pgfqpoint{2.581473in}{1.670286in}}%
\pgfpathlineto{\pgfqpoint{2.596188in}{1.672681in}}%
\pgfpathlineto{\pgfqpoint{2.610904in}{1.675576in}}%
\pgfpathlineto{\pgfqpoint{2.625619in}{1.679649in}}%
\pgfpathlineto{\pgfqpoint{2.640334in}{1.685784in}}%
\pgfpathlineto{\pgfqpoint{2.655050in}{1.693053in}}%
\pgfpathlineto{\pgfqpoint{2.684480in}{1.701869in}}%
\pgfpathlineto{\pgfqpoint{2.699196in}{1.711694in}}%
\pgfpathlineto{\pgfqpoint{2.713911in}{1.728887in}}%
\pgfpathlineto{\pgfqpoint{2.728626in}{1.751192in}}%
\pgfpathlineto{\pgfqpoint{2.743342in}{1.767996in}}%
\pgfpathlineto{\pgfqpoint{2.758057in}{1.776140in}}%
\pgfpathlineto{\pgfqpoint{2.772772in}{1.776718in}}%
\pgfpathlineto{\pgfqpoint{2.802203in}{1.778975in}}%
\pgfpathlineto{\pgfqpoint{2.816918in}{1.778277in}}%
\pgfpathlineto{\pgfqpoint{2.846349in}{1.777685in}}%
\pgfpathlineto{\pgfqpoint{2.861064in}{1.784161in}}%
\pgfpathlineto{\pgfqpoint{2.875779in}{1.792975in}}%
\pgfpathlineto{\pgfqpoint{2.890495in}{1.803159in}}%
\pgfpathlineto{\pgfqpoint{2.905210in}{1.818237in}}%
\pgfpathlineto{\pgfqpoint{2.919925in}{1.832107in}}%
\pgfpathlineto{\pgfqpoint{2.934641in}{1.835779in}}%
\pgfpathlineto{\pgfqpoint{2.949356in}{1.835461in}}%
\pgfpathlineto{\pgfqpoint{2.964071in}{1.836630in}}%
\pgfpathlineto{\pgfqpoint{2.978787in}{1.843191in}}%
\pgfpathlineto{\pgfqpoint{2.993502in}{1.849236in}}%
\pgfpathlineto{\pgfqpoint{3.008217in}{1.856724in}}%
\pgfpathlineto{\pgfqpoint{3.022933in}{1.861078in}}%
\pgfpathlineto{\pgfqpoint{3.037648in}{1.871415in}}%
\pgfpathlineto{\pgfqpoint{3.052363in}{1.891368in}}%
\pgfpathlineto{\pgfqpoint{3.067079in}{1.915059in}}%
\pgfpathlineto{\pgfqpoint{3.096509in}{1.972751in}}%
\pgfpathlineto{\pgfqpoint{3.111224in}{1.989070in}}%
\pgfpathlineto{\pgfqpoint{3.125940in}{1.998766in}}%
\pgfpathlineto{\pgfqpoint{3.140655in}{2.004539in}}%
\pgfpathlineto{\pgfqpoint{3.155370in}{2.014648in}}%
\pgfpathlineto{\pgfqpoint{3.170086in}{2.026022in}}%
\pgfpathlineto{\pgfqpoint{3.184801in}{2.035509in}}%
\pgfpathlineto{\pgfqpoint{3.199516in}{2.041043in}}%
\pgfpathlineto{\pgfqpoint{3.214232in}{2.049307in}}%
\pgfpathlineto{\pgfqpoint{3.228947in}{2.059615in}}%
\pgfpathlineto{\pgfqpoint{3.243662in}{2.072460in}}%
\pgfpathlineto{\pgfqpoint{3.258378in}{2.090022in}}%
\pgfpathlineto{\pgfqpoint{3.273093in}{2.104824in}}%
\pgfpathlineto{\pgfqpoint{3.287808in}{2.106900in}}%
\pgfpathlineto{\pgfqpoint{3.302524in}{2.106269in}}%
\pgfpathlineto{\pgfqpoint{3.317239in}{2.103747in}}%
\pgfpathlineto{\pgfqpoint{3.331954in}{2.103482in}}%
\pgfpathlineto{\pgfqpoint{3.346670in}{2.104585in}}%
\pgfpathlineto{\pgfqpoint{3.361385in}{2.104054in}}%
\pgfpathlineto{\pgfqpoint{3.376100in}{2.101887in}}%
\pgfpathlineto{\pgfqpoint{3.390815in}{2.106217in}}%
\pgfpathlineto{\pgfqpoint{3.405531in}{2.112648in}}%
\pgfpathlineto{\pgfqpoint{3.420246in}{2.124494in}}%
\pgfpathlineto{\pgfqpoint{3.434961in}{2.139990in}}%
\pgfpathlineto{\pgfqpoint{3.449677in}{2.157561in}}%
\pgfpathlineto{\pgfqpoint{3.464392in}{2.165910in}}%
\pgfpathlineto{\pgfqpoint{3.479107in}{2.167528in}}%
\pgfpathlineto{\pgfqpoint{3.493823in}{2.168861in}}%
\pgfpathlineto{\pgfqpoint{3.508538in}{2.171043in}}%
\pgfpathlineto{\pgfqpoint{3.537969in}{2.181042in}}%
\pgfpathlineto{\pgfqpoint{3.552684in}{2.184311in}}%
\pgfpathlineto{\pgfqpoint{3.567399in}{2.193499in}}%
\pgfpathlineto{\pgfqpoint{3.582115in}{2.204617in}}%
\pgfpathlineto{\pgfqpoint{3.611545in}{2.238656in}}%
\pgfpathlineto{\pgfqpoint{3.626261in}{2.251570in}}%
\pgfpathlineto{\pgfqpoint{3.640976in}{2.257591in}}%
\pgfpathlineto{\pgfqpoint{3.670406in}{2.258597in}}%
\pgfpathlineto{\pgfqpoint{3.685122in}{2.261950in}}%
\pgfpathlineto{\pgfqpoint{3.699837in}{2.266938in}}%
\pgfpathlineto{\pgfqpoint{3.714552in}{2.270889in}}%
\pgfpathlineto{\pgfqpoint{3.729268in}{2.274516in}}%
\pgfpathlineto{\pgfqpoint{3.743983in}{2.280472in}}%
\pgfpathlineto{\pgfqpoint{3.758698in}{2.293710in}}%
\pgfpathlineto{\pgfqpoint{3.773414in}{2.314198in}}%
\pgfpathlineto{\pgfqpoint{3.788129in}{2.337720in}}%
\pgfpathlineto{\pgfqpoint{3.802844in}{2.356577in}}%
\pgfpathlineto{\pgfqpoint{3.817560in}{2.368328in}}%
\pgfpathlineto{\pgfqpoint{3.832275in}{2.372259in}}%
\pgfpathlineto{\pgfqpoint{3.846990in}{2.377438in}}%
\pgfpathlineto{\pgfqpoint{3.861706in}{2.384139in}}%
\pgfpathlineto{\pgfqpoint{3.876421in}{2.392817in}}%
\pgfpathlineto{\pgfqpoint{3.891136in}{2.400869in}}%
\pgfpathlineto{\pgfqpoint{3.905852in}{2.408271in}}%
\pgfpathlineto{\pgfqpoint{3.920567in}{2.418455in}}%
\pgfpathlineto{\pgfqpoint{3.935282in}{2.434228in}}%
\pgfpathlineto{\pgfqpoint{3.949997in}{2.455688in}}%
\pgfpathlineto{\pgfqpoint{3.964713in}{2.480505in}}%
\pgfpathlineto{\pgfqpoint{3.979428in}{2.500239in}}%
\pgfpathlineto{\pgfqpoint{3.994143in}{2.513477in}}%
\pgfpathlineto{\pgfqpoint{4.008859in}{2.518048in}}%
\pgfpathlineto{\pgfqpoint{4.023574in}{2.522944in}}%
\pgfpathlineto{\pgfqpoint{4.038289in}{2.528364in}}%
\pgfpathlineto{\pgfqpoint{4.053005in}{2.535294in}}%
\pgfpathlineto{\pgfqpoint{4.067720in}{2.542621in}}%
\pgfpathlineto{\pgfqpoint{4.082435in}{2.548578in}}%
\pgfpathlineto{\pgfqpoint{4.097151in}{2.557649in}}%
\pgfpathlineto{\pgfqpoint{4.111866in}{2.569838in}}%
\pgfpathlineto{\pgfqpoint{4.126581in}{2.585673in}}%
\pgfpathlineto{\pgfqpoint{4.141297in}{2.600336in}}%
\pgfpathlineto{\pgfqpoint{4.156012in}{2.616201in}}%
\pgfpathlineto{\pgfqpoint{4.170727in}{2.622231in}}%
\pgfpathlineto{\pgfqpoint{4.185443in}{2.620784in}}%
\pgfpathlineto{\pgfqpoint{4.200158in}{2.621966in}}%
\pgfpathlineto{\pgfqpoint{4.214873in}{2.624555in}}%
\pgfpathlineto{\pgfqpoint{4.229589in}{2.627482in}}%
\pgfpathlineto{\pgfqpoint{4.244304in}{2.631832in}}%
\pgfpathlineto{\pgfqpoint{4.259019in}{2.634140in}}%
\pgfpathlineto{\pgfqpoint{4.273734in}{2.642769in}}%
\pgfpathlineto{\pgfqpoint{4.288450in}{2.656599in}}%
\pgfpathlineto{\pgfqpoint{4.303165in}{2.681320in}}%
\pgfpathlineto{\pgfqpoint{4.332596in}{2.727478in}}%
\pgfpathlineto{\pgfqpoint{4.347311in}{2.738906in}}%
\pgfpathlineto{\pgfqpoint{4.362026in}{2.742298in}}%
\pgfpathlineto{\pgfqpoint{4.376742in}{2.746957in}}%
\pgfpathlineto{\pgfqpoint{4.406172in}{2.764177in}}%
\pgfpathlineto{\pgfqpoint{4.420888in}{2.771467in}}%
\pgfpathlineto{\pgfqpoint{4.435603in}{2.773432in}}%
\pgfpathlineto{\pgfqpoint{4.450318in}{2.784919in}}%
\pgfpathlineto{\pgfqpoint{4.465034in}{2.799302in}}%
\pgfpathlineto{\pgfqpoint{4.479749in}{2.816452in}}%
\pgfpathlineto{\pgfqpoint{4.509180in}{2.854208in}}%
\pgfpathlineto{\pgfqpoint{4.523895in}{2.864424in}}%
\pgfpathlineto{\pgfqpoint{4.538610in}{2.868236in}}%
\pgfpathlineto{\pgfqpoint{4.553325in}{2.868431in}}%
\pgfpathlineto{\pgfqpoint{4.568041in}{2.870253in}}%
\pgfpathlineto{\pgfqpoint{4.582756in}{2.873841in}}%
\pgfpathlineto{\pgfqpoint{4.597471in}{2.880155in}}%
\pgfpathlineto{\pgfqpoint{4.612187in}{2.879668in}}%
\pgfpathlineto{\pgfqpoint{4.626902in}{2.887436in}}%
\pgfpathlineto{\pgfqpoint{4.641617in}{2.900474in}}%
\pgfpathlineto{\pgfqpoint{4.656333in}{2.919607in}}%
\pgfpathlineto{\pgfqpoint{4.671048in}{2.941342in}}%
\pgfpathlineto{\pgfqpoint{4.685763in}{2.958264in}}%
\pgfpathlineto{\pgfqpoint{4.700479in}{2.967696in}}%
\pgfpathlineto{\pgfqpoint{4.715194in}{2.973105in}}%
\pgfpathlineto{\pgfqpoint{4.729909in}{2.976834in}}%
\pgfpathlineto{\pgfqpoint{4.744625in}{2.982621in}}%
\pgfpathlineto{\pgfqpoint{4.759340in}{2.991287in}}%
\pgfpathlineto{\pgfqpoint{4.774055in}{2.997816in}}%
\pgfpathlineto{\pgfqpoint{4.788771in}{2.995560in}}%
\pgfpathlineto{\pgfqpoint{4.803486in}{2.998789in}}%
\pgfpathlineto{\pgfqpoint{4.818201in}{3.011191in}}%
\pgfpathlineto{\pgfqpoint{4.832916in}{3.028163in}}%
\pgfpathlineto{\pgfqpoint{4.847632in}{3.046547in}}%
\pgfpathlineto{\pgfqpoint{4.862347in}{3.062882in}}%
\pgfpathlineto{\pgfqpoint{4.877062in}{3.073341in}}%
\pgfpathlineto{\pgfqpoint{4.891778in}{3.075652in}}%
\pgfpathlineto{\pgfqpoint{4.906493in}{3.076557in}}%
\pgfpathlineto{\pgfqpoint{4.921208in}{3.076403in}}%
\pgfpathlineto{\pgfqpoint{4.935924in}{3.082094in}}%
\pgfpathlineto{\pgfqpoint{4.950639in}{3.089369in}}%
\pgfpathlineto{\pgfqpoint{4.965354in}{3.097760in}}%
\pgfpathlineto{\pgfqpoint{4.980070in}{3.102791in}}%
\pgfpathlineto{\pgfqpoint{4.994785in}{3.114349in}}%
\pgfpathlineto{\pgfqpoint{5.024216in}{3.154122in}}%
\pgfpathlineto{\pgfqpoint{5.038931in}{3.168846in}}%
\pgfpathlineto{\pgfqpoint{5.053646in}{3.178015in}}%
\pgfpathlineto{\pgfqpoint{5.068362in}{3.179180in}}%
\pgfpathlineto{\pgfqpoint{5.097792in}{3.183830in}}%
\pgfpathlineto{\pgfqpoint{5.112507in}{3.189995in}}%
\pgfpathlineto{\pgfqpoint{5.127223in}{3.196946in}}%
\pgfpathlineto{\pgfqpoint{5.141938in}{3.205241in}}%
\pgfpathlineto{\pgfqpoint{5.156653in}{3.217121in}}%
\pgfpathlineto{\pgfqpoint{5.171369in}{3.235036in}}%
\pgfpathlineto{\pgfqpoint{5.200799in}{3.277636in}}%
\pgfpathlineto{\pgfqpoint{5.215515in}{3.296227in}}%
\pgfpathlineto{\pgfqpoint{5.230230in}{3.308430in}}%
\pgfpathlineto{\pgfqpoint{5.259661in}{3.321047in}}%
\pgfpathlineto{\pgfqpoint{5.289091in}{3.332215in}}%
\pgfpathlineto{\pgfqpoint{5.303807in}{3.335684in}}%
\pgfpathlineto{\pgfqpoint{5.318522in}{3.336187in}}%
\pgfpathlineto{\pgfqpoint{5.333237in}{3.337267in}}%
\pgfpathlineto{\pgfqpoint{5.347953in}{3.347579in}}%
\pgfpathlineto{\pgfqpoint{5.362668in}{3.365491in}}%
\pgfpathlineto{\pgfqpoint{5.377383in}{3.386856in}}%
\pgfpathlineto{\pgfqpoint{5.392098in}{3.404116in}}%
\pgfpathlineto{\pgfqpoint{5.406814in}{3.413431in}}%
\pgfpathlineto{\pgfqpoint{5.421529in}{3.415260in}}%
\pgfpathlineto{\pgfqpoint{5.450960in}{3.420638in}}%
\pgfpathlineto{\pgfqpoint{5.480390in}{3.433674in}}%
\pgfpathlineto{\pgfqpoint{5.495106in}{3.439641in}}%
\pgfpathlineto{\pgfqpoint{5.509821in}{3.449791in}}%
\pgfpathlineto{\pgfqpoint{5.524536in}{3.465265in}}%
\pgfpathlineto{\pgfqpoint{5.539252in}{3.481686in}}%
\pgfpathlineto{\pgfqpoint{5.553967in}{3.503171in}}%
\pgfpathlineto{\pgfqpoint{5.568682in}{3.520206in}}%
\pgfpathlineto{\pgfqpoint{5.583398in}{3.531028in}}%
\pgfpathlineto{\pgfqpoint{5.598113in}{3.535026in}}%
\pgfpathlineto{\pgfqpoint{5.612828in}{3.541088in}}%
\pgfpathlineto{\pgfqpoint{5.627544in}{3.547493in}}%
\pgfpathlineto{\pgfqpoint{5.642259in}{3.556461in}}%
\pgfpathlineto{\pgfqpoint{5.656974in}{3.569029in}}%
\pgfpathlineto{\pgfqpoint{5.671689in}{3.575642in}}%
\pgfpathlineto{\pgfqpoint{5.686405in}{3.584304in}}%
\pgfpathlineto{\pgfqpoint{5.701120in}{3.605340in}}%
\pgfpathlineto{\pgfqpoint{5.715835in}{3.627308in}}%
\pgfpathlineto{\pgfqpoint{5.730551in}{3.652389in}}%
\pgfpathlineto{\pgfqpoint{5.745266in}{3.673363in}}%
\pgfpathlineto{\pgfqpoint{5.759981in}{3.685967in}}%
\pgfpathlineto{\pgfqpoint{5.774697in}{3.689470in}}%
\pgfpathlineto{\pgfqpoint{5.789412in}{3.693557in}}%
\pgfpathlineto{\pgfqpoint{5.804127in}{3.699849in}}%
\pgfpathlineto{\pgfqpoint{5.818843in}{3.705054in}}%
\pgfpathlineto{\pgfqpoint{5.833558in}{3.708111in}}%
\pgfpathlineto{\pgfqpoint{5.848273in}{3.714668in}}%
\pgfpathlineto{\pgfqpoint{5.862989in}{3.722206in}}%
\pgfpathlineto{\pgfqpoint{5.877704in}{3.738103in}}%
\pgfpathlineto{\pgfqpoint{5.907135in}{3.778983in}}%
\pgfpathlineto{\pgfqpoint{5.921850in}{3.796612in}}%
\pgfpathlineto{\pgfqpoint{5.936565in}{3.804793in}}%
\pgfpathlineto{\pgfqpoint{5.951280in}{3.808752in}}%
\pgfpathlineto{\pgfqpoint{5.965996in}{3.810561in}}%
\pgfpathlineto{\pgfqpoint{5.980711in}{3.815976in}}%
\pgfpathlineto{\pgfqpoint{5.995426in}{3.825769in}}%
\pgfpathlineto{\pgfqpoint{6.010142in}{3.829149in}}%
\pgfpathlineto{\pgfqpoint{6.024857in}{3.830599in}}%
\pgfpathlineto{\pgfqpoint{6.039572in}{3.837938in}}%
\pgfpathlineto{\pgfqpoint{6.054288in}{3.851798in}}%
\pgfpathlineto{\pgfqpoint{6.069003in}{3.870187in}}%
\pgfpathlineto{\pgfqpoint{6.083718in}{3.892301in}}%
\pgfpathlineto{\pgfqpoint{6.098434in}{3.910675in}}%
\pgfpathlineto{\pgfqpoint{6.113149in}{3.919343in}}%
\pgfpathlineto{\pgfqpoint{6.127864in}{3.925113in}}%
\pgfpathlineto{\pgfqpoint{6.142580in}{3.931236in}}%
\pgfpathlineto{\pgfqpoint{6.157295in}{3.940934in}}%
\pgfpathlineto{\pgfqpoint{6.172010in}{3.952216in}}%
\pgfpathlineto{\pgfqpoint{6.186726in}{3.965153in}}%
\pgfpathlineto{\pgfqpoint{6.201441in}{3.973759in}}%
\pgfpathlineto{\pgfqpoint{6.216156in}{3.991632in}}%
\pgfpathlineto{\pgfqpoint{6.230871in}{4.017081in}}%
\pgfpathlineto{\pgfqpoint{6.260302in}{4.073440in}}%
\pgfpathlineto{\pgfqpoint{6.275017in}{4.097138in}}%
\pgfpathlineto{\pgfqpoint{6.289733in}{4.111170in}}%
\pgfpathlineto{\pgfqpoint{6.304448in}{4.118858in}}%
\pgfpathlineto{\pgfqpoint{6.319163in}{4.127327in}}%
\pgfpathlineto{\pgfqpoint{6.333879in}{4.137115in}}%
\pgfpathlineto{\pgfqpoint{6.348594in}{4.147584in}}%
\pgfpathlineto{\pgfqpoint{6.363309in}{4.155727in}}%
\pgfpathlineto{\pgfqpoint{6.378025in}{4.155486in}}%
\pgfpathlineto{\pgfqpoint{6.392740in}{4.162292in}}%
\pgfpathlineto{\pgfqpoint{6.407455in}{4.177077in}}%
\pgfpathlineto{\pgfqpoint{6.422171in}{4.200509in}}%
\pgfpathlineto{\pgfqpoint{6.436886in}{4.225524in}}%
\pgfpathlineto{\pgfqpoint{6.451601in}{4.246301in}}%
\pgfpathlineto{\pgfqpoint{6.466317in}{4.255530in}}%
\pgfpathlineto{\pgfqpoint{6.481032in}{4.256028in}}%
\pgfpathlineto{\pgfqpoint{6.495747in}{4.259007in}}%
\pgfpathlineto{\pgfqpoint{6.510463in}{4.262429in}}%
\pgfpathlineto{\pgfqpoint{6.525178in}{4.270197in}}%
\pgfpathlineto{\pgfqpoint{6.539893in}{4.275032in}}%
\pgfpathlineto{\pgfqpoint{6.554608in}{4.276797in}}%
\pgfpathlineto{\pgfqpoint{6.569324in}{4.279614in}}%
\pgfpathlineto{\pgfqpoint{6.584039in}{4.291671in}}%
\pgfpathlineto{\pgfqpoint{6.598754in}{4.310608in}}%
\pgfpathlineto{\pgfqpoint{6.628185in}{4.347268in}}%
\pgfpathlineto{\pgfqpoint{6.642900in}{4.357051in}}%
\pgfpathlineto{\pgfqpoint{6.657616in}{4.361934in}}%
\pgfpathlineto{\pgfqpoint{6.672331in}{4.368484in}}%
\pgfpathlineto{\pgfqpoint{6.687046in}{4.374283in}}%
\pgfpathlineto{\pgfqpoint{6.701762in}{4.387235in}}%
\pgfpathlineto{\pgfqpoint{6.716477in}{4.399474in}}%
\pgfpathlineto{\pgfqpoint{6.731192in}{4.408226in}}%
\pgfpathlineto{\pgfqpoint{6.745908in}{4.419128in}}%
\pgfpathlineto{\pgfqpoint{6.760623in}{4.438960in}}%
\pgfpathlineto{\pgfqpoint{6.775338in}{4.466111in}}%
\pgfpathlineto{\pgfqpoint{6.790054in}{4.491011in}}%
\pgfpathlineto{\pgfqpoint{6.804769in}{4.514060in}}%
\pgfpathlineto{\pgfqpoint{6.819484in}{4.527259in}}%
\pgfpathlineto{\pgfqpoint{6.834199in}{4.529564in}}%
\pgfpathlineto{\pgfqpoint{6.848915in}{4.535839in}}%
\pgfpathlineto{\pgfqpoint{6.863630in}{4.540569in}}%
\pgfpathlineto{\pgfqpoint{6.878345in}{4.548996in}}%
\pgfpathlineto{\pgfqpoint{6.893061in}{4.560244in}}%
\pgfpathlineto{\pgfqpoint{6.907776in}{4.566068in}}%
\pgfpathlineto{\pgfqpoint{6.922491in}{4.574068in}}%
\pgfpathlineto{\pgfqpoint{6.937207in}{4.590186in}}%
\pgfpathlineto{\pgfqpoint{6.951922in}{4.613428in}}%
\pgfpathlineto{\pgfqpoint{6.966637in}{4.635982in}}%
\pgfpathlineto{\pgfqpoint{6.981353in}{4.656016in}}%
\pgfpathlineto{\pgfqpoint{6.996068in}{4.670281in}}%
\pgfpathlineto{\pgfqpoint{7.010783in}{4.675262in}}%
\pgfpathlineto{\pgfqpoint{7.025499in}{4.681072in}}%
\pgfpathlineto{\pgfqpoint{7.040214in}{4.684230in}}%
\pgfpathlineto{\pgfqpoint{7.054929in}{4.690170in}}%
\pgfpathlineto{\pgfqpoint{7.069645in}{4.700575in}}%
\pgfpathlineto{\pgfqpoint{7.084360in}{4.706463in}}%
\pgfpathlineto{\pgfqpoint{7.099075in}{4.715296in}}%
\pgfpathlineto{\pgfqpoint{7.113790in}{4.728393in}}%
\pgfpathlineto{\pgfqpoint{7.128506in}{4.750686in}}%
\pgfpathlineto{\pgfqpoint{7.143221in}{4.772414in}}%
\pgfpathlineto{\pgfqpoint{7.157936in}{4.787913in}}%
\pgfpathlineto{\pgfqpoint{7.172652in}{4.797071in}}%
\pgfpathlineto{\pgfqpoint{7.202082in}{4.804962in}}%
\pgfpathlineto{\pgfqpoint{7.216798in}{4.811099in}}%
\pgfpathlineto{\pgfqpoint{7.231513in}{4.818262in}}%
\pgfpathlineto{\pgfqpoint{7.246228in}{4.826838in}}%
\pgfpathlineto{\pgfqpoint{7.260944in}{4.825589in}}%
\pgfpathlineto{\pgfqpoint{7.275659in}{4.831321in}}%
\pgfpathlineto{\pgfqpoint{7.290374in}{4.841690in}}%
\pgfpathlineto{\pgfqpoint{7.305090in}{4.862388in}}%
\pgfpathlineto{\pgfqpoint{7.319805in}{4.884613in}}%
\pgfpathlineto{\pgfqpoint{7.334520in}{4.904716in}}%
\pgfpathlineto{\pgfqpoint{7.349236in}{4.915035in}}%
\pgfpathlineto{\pgfqpoint{7.378666in}{4.922938in}}%
\pgfpathlineto{\pgfqpoint{7.393381in}{4.930079in}}%
\pgfpathlineto{\pgfqpoint{7.408097in}{4.940218in}}%
\pgfpathlineto{\pgfqpoint{7.422812in}{4.959061in}}%
\pgfpathlineto{\pgfqpoint{7.437527in}{4.975258in}}%
\pgfpathlineto{\pgfqpoint{7.466958in}{5.003848in}}%
\pgfpathlineto{\pgfqpoint{7.481673in}{5.025107in}}%
\pgfpathlineto{\pgfqpoint{7.496389in}{5.047913in}}%
\pgfpathlineto{\pgfqpoint{7.511104in}{5.067588in}}%
\pgfpathlineto{\pgfqpoint{7.525819in}{5.081354in}}%
\pgfpathlineto{\pgfqpoint{7.540535in}{5.088342in}}%
\pgfpathlineto{\pgfqpoint{7.555250in}{5.092673in}}%
\pgfpathlineto{\pgfqpoint{7.569965in}{5.096181in}}%
\pgfpathlineto{\pgfqpoint{7.584681in}{5.098267in}}%
\pgfpathlineto{\pgfqpoint{7.599396in}{5.104478in}}%
\pgfpathlineto{\pgfqpoint{7.614111in}{5.112573in}}%
\pgfpathlineto{\pgfqpoint{7.628827in}{5.127426in}}%
\pgfpathlineto{\pgfqpoint{7.643542in}{5.147461in}}%
\pgfpathlineto{\pgfqpoint{7.658257in}{5.174374in}}%
\pgfpathlineto{\pgfqpoint{7.672972in}{5.204844in}}%
\pgfpathlineto{\pgfqpoint{7.687688in}{5.228337in}}%
\pgfpathlineto{\pgfqpoint{7.702403in}{5.244887in}}%
\pgfpathlineto{\pgfqpoint{7.717118in}{5.255499in}}%
\pgfpathlineto{\pgfqpoint{7.731834in}{5.265623in}}%
\pgfpathlineto{\pgfqpoint{7.746549in}{5.278841in}}%
\pgfpathlineto{\pgfqpoint{7.746549in}{5.278841in}}%
\pgfusepath{stroke}%
\end{pgfscope}%
\begin{pgfscope}%
\pgfpathrectangle{\pgfqpoint{0.697913in}{0.559721in}}{\pgfqpoint{7.048636in}{4.990279in}}%
\pgfusepath{clip}%
\pgfsetbuttcap%
\pgfsetroundjoin%
\definecolor{currentfill}{rgb}{0.000000,0.000000,0.000000}%
\pgfsetfillcolor{currentfill}%
\pgfsetlinewidth{1.003750pt}%
\definecolor{currentstroke}{rgb}{0.000000,0.000000,0.000000}%
\pgfsetstrokecolor{currentstroke}%
\pgfsetdash{}{0pt}%
\pgfsys@defobject{currentmarker}{\pgfqpoint{-0.020833in}{-0.020833in}}{\pgfqpoint{0.020833in}{0.020833in}}{%
\pgfpathmoveto{\pgfqpoint{0.000000in}{-0.020833in}}%
\pgfpathcurveto{\pgfqpoint{0.005525in}{-0.020833in}}{\pgfqpoint{0.010825in}{-0.018638in}}{\pgfqpoint{0.014731in}{-0.014731in}}%
\pgfpathcurveto{\pgfqpoint{0.018638in}{-0.010825in}}{\pgfqpoint{0.020833in}{-0.005525in}}{\pgfqpoint{0.020833in}{0.000000in}}%
\pgfpathcurveto{\pgfqpoint{0.020833in}{0.005525in}}{\pgfqpoint{0.018638in}{0.010825in}}{\pgfqpoint{0.014731in}{0.014731in}}%
\pgfpathcurveto{\pgfqpoint{0.010825in}{0.018638in}}{\pgfqpoint{0.005525in}{0.020833in}}{\pgfqpoint{0.000000in}{0.020833in}}%
\pgfpathcurveto{\pgfqpoint{-0.005525in}{0.020833in}}{\pgfqpoint{-0.010825in}{0.018638in}}{\pgfqpoint{-0.014731in}{0.014731in}}%
\pgfpathcurveto{\pgfqpoint{-0.018638in}{0.010825in}}{\pgfqpoint{-0.020833in}{0.005525in}}{\pgfqpoint{-0.020833in}{0.000000in}}%
\pgfpathcurveto{\pgfqpoint{-0.020833in}{-0.005525in}}{\pgfqpoint{-0.018638in}{-0.010825in}}{\pgfqpoint{-0.014731in}{-0.014731in}}%
\pgfpathcurveto{\pgfqpoint{-0.010825in}{-0.018638in}}{\pgfqpoint{-0.005525in}{-0.020833in}}{\pgfqpoint{0.000000in}{-0.020833in}}%
\pgfpathlineto{\pgfqpoint{0.000000in}{-0.020833in}}%
\pgfpathclose%
\pgfusepath{stroke,fill}%
}%
\begin{pgfscope}%
\pgfsys@transformshift{0.697913in}{0.772482in}%
\pgfsys@useobject{currentmarker}{}%
\end{pgfscope}%
\begin{pgfscope}%
\pgfsys@transformshift{0.712628in}{0.776708in}%
\pgfsys@useobject{currentmarker}{}%
\end{pgfscope}%
\begin{pgfscope}%
\pgfsys@transformshift{0.727343in}{0.778526in}%
\pgfsys@useobject{currentmarker}{}%
\end{pgfscope}%
\begin{pgfscope}%
\pgfsys@transformshift{0.742059in}{0.782814in}%
\pgfsys@useobject{currentmarker}{}%
\end{pgfscope}%
\begin{pgfscope}%
\pgfsys@transformshift{0.756774in}{0.793740in}%
\pgfsys@useobject{currentmarker}{}%
\end{pgfscope}%
\begin{pgfscope}%
\pgfsys@transformshift{0.771489in}{0.807925in}%
\pgfsys@useobject{currentmarker}{}%
\end{pgfscope}%
\begin{pgfscope}%
\pgfsys@transformshift{0.786205in}{0.824367in}%
\pgfsys@useobject{currentmarker}{}%
\end{pgfscope}%
\begin{pgfscope}%
\pgfsys@transformshift{0.800920in}{0.842143in}%
\pgfsys@useobject{currentmarker}{}%
\end{pgfscope}%
\begin{pgfscope}%
\pgfsys@transformshift{0.815635in}{0.850456in}%
\pgfsys@useobject{currentmarker}{}%
\end{pgfscope}%
\begin{pgfscope}%
\pgfsys@transformshift{0.830350in}{0.851458in}%
\pgfsys@useobject{currentmarker}{}%
\end{pgfscope}%
\begin{pgfscope}%
\pgfsys@transformshift{0.845066in}{0.850016in}%
\pgfsys@useobject{currentmarker}{}%
\end{pgfscope}%
\begin{pgfscope}%
\pgfsys@transformshift{0.859781in}{0.853475in}%
\pgfsys@useobject{currentmarker}{}%
\end{pgfscope}%
\begin{pgfscope}%
\pgfsys@transformshift{0.874496in}{0.855324in}%
\pgfsys@useobject{currentmarker}{}%
\end{pgfscope}%
\begin{pgfscope}%
\pgfsys@transformshift{0.889212in}{0.853496in}%
\pgfsys@useobject{currentmarker}{}%
\end{pgfscope}%
\begin{pgfscope}%
\pgfsys@transformshift{0.903927in}{0.851477in}%
\pgfsys@useobject{currentmarker}{}%
\end{pgfscope}%
\begin{pgfscope}%
\pgfsys@transformshift{0.918642in}{0.854480in}%
\pgfsys@useobject{currentmarker}{}%
\end{pgfscope}%
\begin{pgfscope}%
\pgfsys@transformshift{0.933358in}{0.864389in}%
\pgfsys@useobject{currentmarker}{}%
\end{pgfscope}%
\begin{pgfscope}%
\pgfsys@transformshift{0.948073in}{0.879740in}%
\pgfsys@useobject{currentmarker}{}%
\end{pgfscope}%
\begin{pgfscope}%
\pgfsys@transformshift{0.962788in}{0.896539in}%
\pgfsys@useobject{currentmarker}{}%
\end{pgfscope}%
\begin{pgfscope}%
\pgfsys@transformshift{0.977504in}{0.910503in}%
\pgfsys@useobject{currentmarker}{}%
\end{pgfscope}%
\begin{pgfscope}%
\pgfsys@transformshift{0.992219in}{0.920809in}%
\pgfsys@useobject{currentmarker}{}%
\end{pgfscope}%
\begin{pgfscope}%
\pgfsys@transformshift{1.006934in}{0.920843in}%
\pgfsys@useobject{currentmarker}{}%
\end{pgfscope}%
\begin{pgfscope}%
\pgfsys@transformshift{1.021650in}{0.921381in}%
\pgfsys@useobject{currentmarker}{}%
\end{pgfscope}%
\begin{pgfscope}%
\pgfsys@transformshift{1.036365in}{0.924504in}%
\pgfsys@useobject{currentmarker}{}%
\end{pgfscope}%
\begin{pgfscope}%
\pgfsys@transformshift{1.051080in}{0.930748in}%
\pgfsys@useobject{currentmarker}{}%
\end{pgfscope}%
\begin{pgfscope}%
\pgfsys@transformshift{1.065796in}{0.934302in}%
\pgfsys@useobject{currentmarker}{}%
\end{pgfscope}%
\begin{pgfscope}%
\pgfsys@transformshift{1.080511in}{0.933994in}%
\pgfsys@useobject{currentmarker}{}%
\end{pgfscope}%
\begin{pgfscope}%
\pgfsys@transformshift{1.095226in}{0.938674in}%
\pgfsys@useobject{currentmarker}{}%
\end{pgfscope}%
\begin{pgfscope}%
\pgfsys@transformshift{1.109941in}{0.949747in}%
\pgfsys@useobject{currentmarker}{}%
\end{pgfscope}%
\begin{pgfscope}%
\pgfsys@transformshift{1.124657in}{0.967119in}%
\pgfsys@useobject{currentmarker}{}%
\end{pgfscope}%
\begin{pgfscope}%
\pgfsys@transformshift{1.139372in}{0.988431in}%
\pgfsys@useobject{currentmarker}{}%
\end{pgfscope}%
\begin{pgfscope}%
\pgfsys@transformshift{1.154087in}{1.002866in}%
\pgfsys@useobject{currentmarker}{}%
\end{pgfscope}%
\begin{pgfscope}%
\pgfsys@transformshift{1.168803in}{1.012367in}%
\pgfsys@useobject{currentmarker}{}%
\end{pgfscope}%
\begin{pgfscope}%
\pgfsys@transformshift{1.183518in}{1.015020in}%
\pgfsys@useobject{currentmarker}{}%
\end{pgfscope}%
\begin{pgfscope}%
\pgfsys@transformshift{1.198233in}{1.018469in}%
\pgfsys@useobject{currentmarker}{}%
\end{pgfscope}%
\begin{pgfscope}%
\pgfsys@transformshift{1.212949in}{1.021541in}%
\pgfsys@useobject{currentmarker}{}%
\end{pgfscope}%
\begin{pgfscope}%
\pgfsys@transformshift{1.227664in}{1.029429in}%
\pgfsys@useobject{currentmarker}{}%
\end{pgfscope}%
\begin{pgfscope}%
\pgfsys@transformshift{1.242379in}{1.039357in}%
\pgfsys@useobject{currentmarker}{}%
\end{pgfscope}%
\begin{pgfscope}%
\pgfsys@transformshift{1.257095in}{1.044620in}%
\pgfsys@useobject{currentmarker}{}%
\end{pgfscope}%
\begin{pgfscope}%
\pgfsys@transformshift{1.271810in}{1.052380in}%
\pgfsys@useobject{currentmarker}{}%
\end{pgfscope}%
\begin{pgfscope}%
\pgfsys@transformshift{1.286525in}{1.066091in}%
\pgfsys@useobject{currentmarker}{}%
\end{pgfscope}%
\begin{pgfscope}%
\pgfsys@transformshift{1.301241in}{1.083891in}%
\pgfsys@useobject{currentmarker}{}%
\end{pgfscope}%
\begin{pgfscope}%
\pgfsys@transformshift{1.315956in}{1.107352in}%
\pgfsys@useobject{currentmarker}{}%
\end{pgfscope}%
\begin{pgfscope}%
\pgfsys@transformshift{1.330671in}{1.126971in}%
\pgfsys@useobject{currentmarker}{}%
\end{pgfscope}%
\begin{pgfscope}%
\pgfsys@transformshift{1.345387in}{1.138285in}%
\pgfsys@useobject{currentmarker}{}%
\end{pgfscope}%
\begin{pgfscope}%
\pgfsys@transformshift{1.360102in}{1.145010in}%
\pgfsys@useobject{currentmarker}{}%
\end{pgfscope}%
\begin{pgfscope}%
\pgfsys@transformshift{1.374817in}{1.151060in}%
\pgfsys@useobject{currentmarker}{}%
\end{pgfscope}%
\begin{pgfscope}%
\pgfsys@transformshift{1.389532in}{1.157016in}%
\pgfsys@useobject{currentmarker}{}%
\end{pgfscope}%
\begin{pgfscope}%
\pgfsys@transformshift{1.404248in}{1.163814in}%
\pgfsys@useobject{currentmarker}{}%
\end{pgfscope}%
\begin{pgfscope}%
\pgfsys@transformshift{1.418963in}{1.168754in}%
\pgfsys@useobject{currentmarker}{}%
\end{pgfscope}%
\begin{pgfscope}%
\pgfsys@transformshift{1.433678in}{1.170860in}%
\pgfsys@useobject{currentmarker}{}%
\end{pgfscope}%
\begin{pgfscope}%
\pgfsys@transformshift{1.448394in}{1.177881in}%
\pgfsys@useobject{currentmarker}{}%
\end{pgfscope}%
\begin{pgfscope}%
\pgfsys@transformshift{1.463109in}{1.188860in}%
\pgfsys@useobject{currentmarker}{}%
\end{pgfscope}%
\begin{pgfscope}%
\pgfsys@transformshift{1.477824in}{1.202468in}%
\pgfsys@useobject{currentmarker}{}%
\end{pgfscope}%
\begin{pgfscope}%
\pgfsys@transformshift{1.492540in}{1.221036in}%
\pgfsys@useobject{currentmarker}{}%
\end{pgfscope}%
\begin{pgfscope}%
\pgfsys@transformshift{1.507255in}{1.235942in}%
\pgfsys@useobject{currentmarker}{}%
\end{pgfscope}%
\begin{pgfscope}%
\pgfsys@transformshift{1.521970in}{1.239684in}%
\pgfsys@useobject{currentmarker}{}%
\end{pgfscope}%
\begin{pgfscope}%
\pgfsys@transformshift{1.536686in}{1.239129in}%
\pgfsys@useobject{currentmarker}{}%
\end{pgfscope}%
\begin{pgfscope}%
\pgfsys@transformshift{1.551401in}{1.238990in}%
\pgfsys@useobject{currentmarker}{}%
\end{pgfscope}%
\begin{pgfscope}%
\pgfsys@transformshift{1.566116in}{1.237994in}%
\pgfsys@useobject{currentmarker}{}%
\end{pgfscope}%
\begin{pgfscope}%
\pgfsys@transformshift{1.580832in}{1.240306in}%
\pgfsys@useobject{currentmarker}{}%
\end{pgfscope}%
\begin{pgfscope}%
\pgfsys@transformshift{1.595547in}{1.245559in}%
\pgfsys@useobject{currentmarker}{}%
\end{pgfscope}%
\begin{pgfscope}%
\pgfsys@transformshift{1.610262in}{1.245804in}%
\pgfsys@useobject{currentmarker}{}%
\end{pgfscope}%
\begin{pgfscope}%
\pgfsys@transformshift{1.624978in}{1.247963in}%
\pgfsys@useobject{currentmarker}{}%
\end{pgfscope}%
\begin{pgfscope}%
\pgfsys@transformshift{1.639693in}{1.258008in}%
\pgfsys@useobject{currentmarker}{}%
\end{pgfscope}%
\begin{pgfscope}%
\pgfsys@transformshift{1.654408in}{1.270090in}%
\pgfsys@useobject{currentmarker}{}%
\end{pgfscope}%
\begin{pgfscope}%
\pgfsys@transformshift{1.669124in}{1.285587in}%
\pgfsys@useobject{currentmarker}{}%
\end{pgfscope}%
\begin{pgfscope}%
\pgfsys@transformshift{1.683839in}{1.300576in}%
\pgfsys@useobject{currentmarker}{}%
\end{pgfscope}%
\begin{pgfscope}%
\pgfsys@transformshift{1.698554in}{1.305706in}%
\pgfsys@useobject{currentmarker}{}%
\end{pgfscope}%
\begin{pgfscope}%
\pgfsys@transformshift{1.713269in}{1.306321in}%
\pgfsys@useobject{currentmarker}{}%
\end{pgfscope}%
\begin{pgfscope}%
\pgfsys@transformshift{1.727985in}{1.307462in}%
\pgfsys@useobject{currentmarker}{}%
\end{pgfscope}%
\begin{pgfscope}%
\pgfsys@transformshift{1.742700in}{1.310348in}%
\pgfsys@useobject{currentmarker}{}%
\end{pgfscope}%
\begin{pgfscope}%
\pgfsys@transformshift{1.757415in}{1.309496in}%
\pgfsys@useobject{currentmarker}{}%
\end{pgfscope}%
\begin{pgfscope}%
\pgfsys@transformshift{1.772131in}{1.309777in}%
\pgfsys@useobject{currentmarker}{}%
\end{pgfscope}%
\begin{pgfscope}%
\pgfsys@transformshift{1.786846in}{1.314763in}%
\pgfsys@useobject{currentmarker}{}%
\end{pgfscope}%
\begin{pgfscope}%
\pgfsys@transformshift{1.801561in}{1.321906in}%
\pgfsys@useobject{currentmarker}{}%
\end{pgfscope}%
\begin{pgfscope}%
\pgfsys@transformshift{1.816277in}{1.336418in}%
\pgfsys@useobject{currentmarker}{}%
\end{pgfscope}%
\begin{pgfscope}%
\pgfsys@transformshift{1.830992in}{1.353519in}%
\pgfsys@useobject{currentmarker}{}%
\end{pgfscope}%
\begin{pgfscope}%
\pgfsys@transformshift{1.845707in}{1.370317in}%
\pgfsys@useobject{currentmarker}{}%
\end{pgfscope}%
\begin{pgfscope}%
\pgfsys@transformshift{1.860423in}{1.382970in}%
\pgfsys@useobject{currentmarker}{}%
\end{pgfscope}%
\begin{pgfscope}%
\pgfsys@transformshift{1.875138in}{1.386201in}%
\pgfsys@useobject{currentmarker}{}%
\end{pgfscope}%
\begin{pgfscope}%
\pgfsys@transformshift{1.889853in}{1.382723in}%
\pgfsys@useobject{currentmarker}{}%
\end{pgfscope}%
\begin{pgfscope}%
\pgfsys@transformshift{1.904569in}{1.379957in}%
\pgfsys@useobject{currentmarker}{}%
\end{pgfscope}%
\begin{pgfscope}%
\pgfsys@transformshift{1.919284in}{1.381489in}%
\pgfsys@useobject{currentmarker}{}%
\end{pgfscope}%
\begin{pgfscope}%
\pgfsys@transformshift{1.933999in}{1.384160in}%
\pgfsys@useobject{currentmarker}{}%
\end{pgfscope}%
\begin{pgfscope}%
\pgfsys@transformshift{1.948715in}{1.383501in}%
\pgfsys@useobject{currentmarker}{}%
\end{pgfscope}%
\begin{pgfscope}%
\pgfsys@transformshift{1.963430in}{1.379303in}%
\pgfsys@useobject{currentmarker}{}%
\end{pgfscope}%
\begin{pgfscope}%
\pgfsys@transformshift{1.978145in}{1.379256in}%
\pgfsys@useobject{currentmarker}{}%
\end{pgfscope}%
\begin{pgfscope}%
\pgfsys@transformshift{1.992860in}{1.386849in}%
\pgfsys@useobject{currentmarker}{}%
\end{pgfscope}%
\begin{pgfscope}%
\pgfsys@transformshift{2.007576in}{1.402902in}%
\pgfsys@useobject{currentmarker}{}%
\end{pgfscope}%
\begin{pgfscope}%
\pgfsys@transformshift{2.022291in}{1.419951in}%
\pgfsys@useobject{currentmarker}{}%
\end{pgfscope}%
\begin{pgfscope}%
\pgfsys@transformshift{2.037006in}{1.434046in}%
\pgfsys@useobject{currentmarker}{}%
\end{pgfscope}%
\begin{pgfscope}%
\pgfsys@transformshift{2.051722in}{1.437016in}%
\pgfsys@useobject{currentmarker}{}%
\end{pgfscope}%
\begin{pgfscope}%
\pgfsys@transformshift{2.066437in}{1.434647in}%
\pgfsys@useobject{currentmarker}{}%
\end{pgfscope}%
\begin{pgfscope}%
\pgfsys@transformshift{2.081152in}{1.430399in}%
\pgfsys@useobject{currentmarker}{}%
\end{pgfscope}%
\begin{pgfscope}%
\pgfsys@transformshift{2.095868in}{1.426973in}%
\pgfsys@useobject{currentmarker}{}%
\end{pgfscope}%
\begin{pgfscope}%
\pgfsys@transformshift{2.110583in}{1.426453in}%
\pgfsys@useobject{currentmarker}{}%
\end{pgfscope}%
\begin{pgfscope}%
\pgfsys@transformshift{2.125298in}{1.423711in}%
\pgfsys@useobject{currentmarker}{}%
\end{pgfscope}%
\begin{pgfscope}%
\pgfsys@transformshift{2.140014in}{1.420511in}%
\pgfsys@useobject{currentmarker}{}%
\end{pgfscope}%
\begin{pgfscope}%
\pgfsys@transformshift{2.154729in}{1.418902in}%
\pgfsys@useobject{currentmarker}{}%
\end{pgfscope}%
\begin{pgfscope}%
\pgfsys@transformshift{2.169444in}{1.423368in}%
\pgfsys@useobject{currentmarker}{}%
\end{pgfscope}%
\begin{pgfscope}%
\pgfsys@transformshift{2.184160in}{1.436166in}%
\pgfsys@useobject{currentmarker}{}%
\end{pgfscope}%
\begin{pgfscope}%
\pgfsys@transformshift{2.198875in}{1.448575in}%
\pgfsys@useobject{currentmarker}{}%
\end{pgfscope}%
\begin{pgfscope}%
\pgfsys@transformshift{2.213590in}{1.462882in}%
\pgfsys@useobject{currentmarker}{}%
\end{pgfscope}%
\begin{pgfscope}%
\pgfsys@transformshift{2.228306in}{1.466387in}%
\pgfsys@useobject{currentmarker}{}%
\end{pgfscope}%
\begin{pgfscope}%
\pgfsys@transformshift{2.243021in}{1.465698in}%
\pgfsys@useobject{currentmarker}{}%
\end{pgfscope}%
\begin{pgfscope}%
\pgfsys@transformshift{2.257736in}{1.464916in}%
\pgfsys@useobject{currentmarker}{}%
\end{pgfscope}%
\begin{pgfscope}%
\pgfsys@transformshift{2.272451in}{1.464491in}%
\pgfsys@useobject{currentmarker}{}%
\end{pgfscope}%
\begin{pgfscope}%
\pgfsys@transformshift{2.287167in}{1.469085in}%
\pgfsys@useobject{currentmarker}{}%
\end{pgfscope}%
\begin{pgfscope}%
\pgfsys@transformshift{2.301882in}{1.471441in}%
\pgfsys@useobject{currentmarker}{}%
\end{pgfscope}%
\begin{pgfscope}%
\pgfsys@transformshift{2.316597in}{1.475020in}%
\pgfsys@useobject{currentmarker}{}%
\end{pgfscope}%
\begin{pgfscope}%
\pgfsys@transformshift{2.331313in}{1.480577in}%
\pgfsys@useobject{currentmarker}{}%
\end{pgfscope}%
\begin{pgfscope}%
\pgfsys@transformshift{2.346028in}{1.491564in}%
\pgfsys@useobject{currentmarker}{}%
\end{pgfscope}%
\begin{pgfscope}%
\pgfsys@transformshift{2.360743in}{1.509390in}%
\pgfsys@useobject{currentmarker}{}%
\end{pgfscope}%
\begin{pgfscope}%
\pgfsys@transformshift{2.375459in}{1.528562in}%
\pgfsys@useobject{currentmarker}{}%
\end{pgfscope}%
\begin{pgfscope}%
\pgfsys@transformshift{2.390174in}{1.546640in}%
\pgfsys@useobject{currentmarker}{}%
\end{pgfscope}%
\begin{pgfscope}%
\pgfsys@transformshift{2.404889in}{1.554886in}%
\pgfsys@useobject{currentmarker}{}%
\end{pgfscope}%
\begin{pgfscope}%
\pgfsys@transformshift{2.419605in}{1.558173in}%
\pgfsys@useobject{currentmarker}{}%
\end{pgfscope}%
\begin{pgfscope}%
\pgfsys@transformshift{2.434320in}{1.561450in}%
\pgfsys@useobject{currentmarker}{}%
\end{pgfscope}%
\begin{pgfscope}%
\pgfsys@transformshift{2.449035in}{1.565338in}%
\pgfsys@useobject{currentmarker}{}%
\end{pgfscope}%
\begin{pgfscope}%
\pgfsys@transformshift{2.463751in}{1.570401in}%
\pgfsys@useobject{currentmarker}{}%
\end{pgfscope}%
\begin{pgfscope}%
\pgfsys@transformshift{2.478466in}{1.575020in}%
\pgfsys@useobject{currentmarker}{}%
\end{pgfscope}%
\begin{pgfscope}%
\pgfsys@transformshift{2.493181in}{1.577950in}%
\pgfsys@useobject{currentmarker}{}%
\end{pgfscope}%
\begin{pgfscope}%
\pgfsys@transformshift{2.507897in}{1.586356in}%
\pgfsys@useobject{currentmarker}{}%
\end{pgfscope}%
\begin{pgfscope}%
\pgfsys@transformshift{2.522612in}{1.601049in}%
\pgfsys@useobject{currentmarker}{}%
\end{pgfscope}%
\begin{pgfscope}%
\pgfsys@transformshift{2.537327in}{1.620688in}%
\pgfsys@useobject{currentmarker}{}%
\end{pgfscope}%
\begin{pgfscope}%
\pgfsys@transformshift{2.552042in}{1.641588in}%
\pgfsys@useobject{currentmarker}{}%
\end{pgfscope}%
\begin{pgfscope}%
\pgfsys@transformshift{2.566758in}{1.659396in}%
\pgfsys@useobject{currentmarker}{}%
\end{pgfscope}%
\begin{pgfscope}%
\pgfsys@transformshift{2.581473in}{1.670286in}%
\pgfsys@useobject{currentmarker}{}%
\end{pgfscope}%
\begin{pgfscope}%
\pgfsys@transformshift{2.596188in}{1.672681in}%
\pgfsys@useobject{currentmarker}{}%
\end{pgfscope}%
\begin{pgfscope}%
\pgfsys@transformshift{2.610904in}{1.675576in}%
\pgfsys@useobject{currentmarker}{}%
\end{pgfscope}%
\begin{pgfscope}%
\pgfsys@transformshift{2.625619in}{1.679649in}%
\pgfsys@useobject{currentmarker}{}%
\end{pgfscope}%
\begin{pgfscope}%
\pgfsys@transformshift{2.640334in}{1.685784in}%
\pgfsys@useobject{currentmarker}{}%
\end{pgfscope}%
\begin{pgfscope}%
\pgfsys@transformshift{2.655050in}{1.693053in}%
\pgfsys@useobject{currentmarker}{}%
\end{pgfscope}%
\begin{pgfscope}%
\pgfsys@transformshift{2.669765in}{1.697361in}%
\pgfsys@useobject{currentmarker}{}%
\end{pgfscope}%
\begin{pgfscope}%
\pgfsys@transformshift{2.684480in}{1.701869in}%
\pgfsys@useobject{currentmarker}{}%
\end{pgfscope}%
\begin{pgfscope}%
\pgfsys@transformshift{2.699196in}{1.711694in}%
\pgfsys@useobject{currentmarker}{}%
\end{pgfscope}%
\begin{pgfscope}%
\pgfsys@transformshift{2.713911in}{1.728887in}%
\pgfsys@useobject{currentmarker}{}%
\end{pgfscope}%
\begin{pgfscope}%
\pgfsys@transformshift{2.728626in}{1.751192in}%
\pgfsys@useobject{currentmarker}{}%
\end{pgfscope}%
\begin{pgfscope}%
\pgfsys@transformshift{2.743342in}{1.767996in}%
\pgfsys@useobject{currentmarker}{}%
\end{pgfscope}%
\begin{pgfscope}%
\pgfsys@transformshift{2.758057in}{1.776140in}%
\pgfsys@useobject{currentmarker}{}%
\end{pgfscope}%
\begin{pgfscope}%
\pgfsys@transformshift{2.772772in}{1.776718in}%
\pgfsys@useobject{currentmarker}{}%
\end{pgfscope}%
\begin{pgfscope}%
\pgfsys@transformshift{2.787488in}{1.777801in}%
\pgfsys@useobject{currentmarker}{}%
\end{pgfscope}%
\begin{pgfscope}%
\pgfsys@transformshift{2.802203in}{1.778975in}%
\pgfsys@useobject{currentmarker}{}%
\end{pgfscope}%
\begin{pgfscope}%
\pgfsys@transformshift{2.816918in}{1.778277in}%
\pgfsys@useobject{currentmarker}{}%
\end{pgfscope}%
\begin{pgfscope}%
\pgfsys@transformshift{2.831633in}{1.778001in}%
\pgfsys@useobject{currentmarker}{}%
\end{pgfscope}%
\begin{pgfscope}%
\pgfsys@transformshift{2.846349in}{1.777685in}%
\pgfsys@useobject{currentmarker}{}%
\end{pgfscope}%
\begin{pgfscope}%
\pgfsys@transformshift{2.861064in}{1.784161in}%
\pgfsys@useobject{currentmarker}{}%
\end{pgfscope}%
\begin{pgfscope}%
\pgfsys@transformshift{2.875779in}{1.792975in}%
\pgfsys@useobject{currentmarker}{}%
\end{pgfscope}%
\begin{pgfscope}%
\pgfsys@transformshift{2.890495in}{1.803159in}%
\pgfsys@useobject{currentmarker}{}%
\end{pgfscope}%
\begin{pgfscope}%
\pgfsys@transformshift{2.905210in}{1.818237in}%
\pgfsys@useobject{currentmarker}{}%
\end{pgfscope}%
\begin{pgfscope}%
\pgfsys@transformshift{2.919925in}{1.832107in}%
\pgfsys@useobject{currentmarker}{}%
\end{pgfscope}%
\begin{pgfscope}%
\pgfsys@transformshift{2.934641in}{1.835779in}%
\pgfsys@useobject{currentmarker}{}%
\end{pgfscope}%
\begin{pgfscope}%
\pgfsys@transformshift{2.949356in}{1.835461in}%
\pgfsys@useobject{currentmarker}{}%
\end{pgfscope}%
\begin{pgfscope}%
\pgfsys@transformshift{2.964071in}{1.836630in}%
\pgfsys@useobject{currentmarker}{}%
\end{pgfscope}%
\begin{pgfscope}%
\pgfsys@transformshift{2.978787in}{1.843191in}%
\pgfsys@useobject{currentmarker}{}%
\end{pgfscope}%
\begin{pgfscope}%
\pgfsys@transformshift{2.993502in}{1.849236in}%
\pgfsys@useobject{currentmarker}{}%
\end{pgfscope}%
\begin{pgfscope}%
\pgfsys@transformshift{3.008217in}{1.856724in}%
\pgfsys@useobject{currentmarker}{}%
\end{pgfscope}%
\begin{pgfscope}%
\pgfsys@transformshift{3.022933in}{1.861078in}%
\pgfsys@useobject{currentmarker}{}%
\end{pgfscope}%
\begin{pgfscope}%
\pgfsys@transformshift{3.037648in}{1.871415in}%
\pgfsys@useobject{currentmarker}{}%
\end{pgfscope}%
\begin{pgfscope}%
\pgfsys@transformshift{3.052363in}{1.891368in}%
\pgfsys@useobject{currentmarker}{}%
\end{pgfscope}%
\begin{pgfscope}%
\pgfsys@transformshift{3.067079in}{1.915059in}%
\pgfsys@useobject{currentmarker}{}%
\end{pgfscope}%
\begin{pgfscope}%
\pgfsys@transformshift{3.081794in}{1.943641in}%
\pgfsys@useobject{currentmarker}{}%
\end{pgfscope}%
\begin{pgfscope}%
\pgfsys@transformshift{3.096509in}{1.972751in}%
\pgfsys@useobject{currentmarker}{}%
\end{pgfscope}%
\begin{pgfscope}%
\pgfsys@transformshift{3.111224in}{1.989070in}%
\pgfsys@useobject{currentmarker}{}%
\end{pgfscope}%
\begin{pgfscope}%
\pgfsys@transformshift{3.125940in}{1.998766in}%
\pgfsys@useobject{currentmarker}{}%
\end{pgfscope}%
\begin{pgfscope}%
\pgfsys@transformshift{3.140655in}{2.004539in}%
\pgfsys@useobject{currentmarker}{}%
\end{pgfscope}%
\begin{pgfscope}%
\pgfsys@transformshift{3.155370in}{2.014648in}%
\pgfsys@useobject{currentmarker}{}%
\end{pgfscope}%
\begin{pgfscope}%
\pgfsys@transformshift{3.170086in}{2.026022in}%
\pgfsys@useobject{currentmarker}{}%
\end{pgfscope}%
\begin{pgfscope}%
\pgfsys@transformshift{3.184801in}{2.035509in}%
\pgfsys@useobject{currentmarker}{}%
\end{pgfscope}%
\begin{pgfscope}%
\pgfsys@transformshift{3.199516in}{2.041043in}%
\pgfsys@useobject{currentmarker}{}%
\end{pgfscope}%
\begin{pgfscope}%
\pgfsys@transformshift{3.214232in}{2.049307in}%
\pgfsys@useobject{currentmarker}{}%
\end{pgfscope}%
\begin{pgfscope}%
\pgfsys@transformshift{3.228947in}{2.059615in}%
\pgfsys@useobject{currentmarker}{}%
\end{pgfscope}%
\begin{pgfscope}%
\pgfsys@transformshift{3.243662in}{2.072460in}%
\pgfsys@useobject{currentmarker}{}%
\end{pgfscope}%
\begin{pgfscope}%
\pgfsys@transformshift{3.258378in}{2.090022in}%
\pgfsys@useobject{currentmarker}{}%
\end{pgfscope}%
\begin{pgfscope}%
\pgfsys@transformshift{3.273093in}{2.104824in}%
\pgfsys@useobject{currentmarker}{}%
\end{pgfscope}%
\begin{pgfscope}%
\pgfsys@transformshift{3.287808in}{2.106900in}%
\pgfsys@useobject{currentmarker}{}%
\end{pgfscope}%
\begin{pgfscope}%
\pgfsys@transformshift{3.302524in}{2.106269in}%
\pgfsys@useobject{currentmarker}{}%
\end{pgfscope}%
\begin{pgfscope}%
\pgfsys@transformshift{3.317239in}{2.103747in}%
\pgfsys@useobject{currentmarker}{}%
\end{pgfscope}%
\begin{pgfscope}%
\pgfsys@transformshift{3.331954in}{2.103482in}%
\pgfsys@useobject{currentmarker}{}%
\end{pgfscope}%
\begin{pgfscope}%
\pgfsys@transformshift{3.346670in}{2.104585in}%
\pgfsys@useobject{currentmarker}{}%
\end{pgfscope}%
\begin{pgfscope}%
\pgfsys@transformshift{3.361385in}{2.104054in}%
\pgfsys@useobject{currentmarker}{}%
\end{pgfscope}%
\begin{pgfscope}%
\pgfsys@transformshift{3.376100in}{2.101887in}%
\pgfsys@useobject{currentmarker}{}%
\end{pgfscope}%
\begin{pgfscope}%
\pgfsys@transformshift{3.390815in}{2.106217in}%
\pgfsys@useobject{currentmarker}{}%
\end{pgfscope}%
\begin{pgfscope}%
\pgfsys@transformshift{3.405531in}{2.112648in}%
\pgfsys@useobject{currentmarker}{}%
\end{pgfscope}%
\begin{pgfscope}%
\pgfsys@transformshift{3.420246in}{2.124494in}%
\pgfsys@useobject{currentmarker}{}%
\end{pgfscope}%
\begin{pgfscope}%
\pgfsys@transformshift{3.434961in}{2.139990in}%
\pgfsys@useobject{currentmarker}{}%
\end{pgfscope}%
\begin{pgfscope}%
\pgfsys@transformshift{3.449677in}{2.157561in}%
\pgfsys@useobject{currentmarker}{}%
\end{pgfscope}%
\begin{pgfscope}%
\pgfsys@transformshift{3.464392in}{2.165910in}%
\pgfsys@useobject{currentmarker}{}%
\end{pgfscope}%
\begin{pgfscope}%
\pgfsys@transformshift{3.479107in}{2.167528in}%
\pgfsys@useobject{currentmarker}{}%
\end{pgfscope}%
\begin{pgfscope}%
\pgfsys@transformshift{3.493823in}{2.168861in}%
\pgfsys@useobject{currentmarker}{}%
\end{pgfscope}%
\begin{pgfscope}%
\pgfsys@transformshift{3.508538in}{2.171043in}%
\pgfsys@useobject{currentmarker}{}%
\end{pgfscope}%
\begin{pgfscope}%
\pgfsys@transformshift{3.523253in}{2.176138in}%
\pgfsys@useobject{currentmarker}{}%
\end{pgfscope}%
\begin{pgfscope}%
\pgfsys@transformshift{3.537969in}{2.181042in}%
\pgfsys@useobject{currentmarker}{}%
\end{pgfscope}%
\begin{pgfscope}%
\pgfsys@transformshift{3.552684in}{2.184311in}%
\pgfsys@useobject{currentmarker}{}%
\end{pgfscope}%
\begin{pgfscope}%
\pgfsys@transformshift{3.567399in}{2.193499in}%
\pgfsys@useobject{currentmarker}{}%
\end{pgfscope}%
\begin{pgfscope}%
\pgfsys@transformshift{3.582115in}{2.204617in}%
\pgfsys@useobject{currentmarker}{}%
\end{pgfscope}%
\begin{pgfscope}%
\pgfsys@transformshift{3.596830in}{2.221458in}%
\pgfsys@useobject{currentmarker}{}%
\end{pgfscope}%
\begin{pgfscope}%
\pgfsys@transformshift{3.611545in}{2.238656in}%
\pgfsys@useobject{currentmarker}{}%
\end{pgfscope}%
\begin{pgfscope}%
\pgfsys@transformshift{3.626261in}{2.251570in}%
\pgfsys@useobject{currentmarker}{}%
\end{pgfscope}%
\begin{pgfscope}%
\pgfsys@transformshift{3.640976in}{2.257591in}%
\pgfsys@useobject{currentmarker}{}%
\end{pgfscope}%
\begin{pgfscope}%
\pgfsys@transformshift{3.655691in}{2.258013in}%
\pgfsys@useobject{currentmarker}{}%
\end{pgfscope}%
\begin{pgfscope}%
\pgfsys@transformshift{3.670406in}{2.258597in}%
\pgfsys@useobject{currentmarker}{}%
\end{pgfscope}%
\begin{pgfscope}%
\pgfsys@transformshift{3.685122in}{2.261950in}%
\pgfsys@useobject{currentmarker}{}%
\end{pgfscope}%
\begin{pgfscope}%
\pgfsys@transformshift{3.699837in}{2.266938in}%
\pgfsys@useobject{currentmarker}{}%
\end{pgfscope}%
\begin{pgfscope}%
\pgfsys@transformshift{3.714552in}{2.270889in}%
\pgfsys@useobject{currentmarker}{}%
\end{pgfscope}%
\begin{pgfscope}%
\pgfsys@transformshift{3.729268in}{2.274516in}%
\pgfsys@useobject{currentmarker}{}%
\end{pgfscope}%
\begin{pgfscope}%
\pgfsys@transformshift{3.743983in}{2.280472in}%
\pgfsys@useobject{currentmarker}{}%
\end{pgfscope}%
\begin{pgfscope}%
\pgfsys@transformshift{3.758698in}{2.293710in}%
\pgfsys@useobject{currentmarker}{}%
\end{pgfscope}%
\begin{pgfscope}%
\pgfsys@transformshift{3.773414in}{2.314198in}%
\pgfsys@useobject{currentmarker}{}%
\end{pgfscope}%
\begin{pgfscope}%
\pgfsys@transformshift{3.788129in}{2.337720in}%
\pgfsys@useobject{currentmarker}{}%
\end{pgfscope}%
\begin{pgfscope}%
\pgfsys@transformshift{3.802844in}{2.356577in}%
\pgfsys@useobject{currentmarker}{}%
\end{pgfscope}%
\begin{pgfscope}%
\pgfsys@transformshift{3.817560in}{2.368328in}%
\pgfsys@useobject{currentmarker}{}%
\end{pgfscope}%
\begin{pgfscope}%
\pgfsys@transformshift{3.832275in}{2.372259in}%
\pgfsys@useobject{currentmarker}{}%
\end{pgfscope}%
\begin{pgfscope}%
\pgfsys@transformshift{3.846990in}{2.377438in}%
\pgfsys@useobject{currentmarker}{}%
\end{pgfscope}%
\begin{pgfscope}%
\pgfsys@transformshift{3.861706in}{2.384139in}%
\pgfsys@useobject{currentmarker}{}%
\end{pgfscope}%
\begin{pgfscope}%
\pgfsys@transformshift{3.876421in}{2.392817in}%
\pgfsys@useobject{currentmarker}{}%
\end{pgfscope}%
\begin{pgfscope}%
\pgfsys@transformshift{3.891136in}{2.400869in}%
\pgfsys@useobject{currentmarker}{}%
\end{pgfscope}%
\begin{pgfscope}%
\pgfsys@transformshift{3.905852in}{2.408271in}%
\pgfsys@useobject{currentmarker}{}%
\end{pgfscope}%
\begin{pgfscope}%
\pgfsys@transformshift{3.920567in}{2.418455in}%
\pgfsys@useobject{currentmarker}{}%
\end{pgfscope}%
\begin{pgfscope}%
\pgfsys@transformshift{3.935282in}{2.434228in}%
\pgfsys@useobject{currentmarker}{}%
\end{pgfscope}%
\begin{pgfscope}%
\pgfsys@transformshift{3.949997in}{2.455688in}%
\pgfsys@useobject{currentmarker}{}%
\end{pgfscope}%
\begin{pgfscope}%
\pgfsys@transformshift{3.964713in}{2.480505in}%
\pgfsys@useobject{currentmarker}{}%
\end{pgfscope}%
\begin{pgfscope}%
\pgfsys@transformshift{3.979428in}{2.500239in}%
\pgfsys@useobject{currentmarker}{}%
\end{pgfscope}%
\begin{pgfscope}%
\pgfsys@transformshift{3.994143in}{2.513477in}%
\pgfsys@useobject{currentmarker}{}%
\end{pgfscope}%
\begin{pgfscope}%
\pgfsys@transformshift{4.008859in}{2.518048in}%
\pgfsys@useobject{currentmarker}{}%
\end{pgfscope}%
\begin{pgfscope}%
\pgfsys@transformshift{4.023574in}{2.522944in}%
\pgfsys@useobject{currentmarker}{}%
\end{pgfscope}%
\begin{pgfscope}%
\pgfsys@transformshift{4.038289in}{2.528364in}%
\pgfsys@useobject{currentmarker}{}%
\end{pgfscope}%
\begin{pgfscope}%
\pgfsys@transformshift{4.053005in}{2.535294in}%
\pgfsys@useobject{currentmarker}{}%
\end{pgfscope}%
\begin{pgfscope}%
\pgfsys@transformshift{4.067720in}{2.542621in}%
\pgfsys@useobject{currentmarker}{}%
\end{pgfscope}%
\begin{pgfscope}%
\pgfsys@transformshift{4.082435in}{2.548578in}%
\pgfsys@useobject{currentmarker}{}%
\end{pgfscope}%
\begin{pgfscope}%
\pgfsys@transformshift{4.097151in}{2.557649in}%
\pgfsys@useobject{currentmarker}{}%
\end{pgfscope}%
\begin{pgfscope}%
\pgfsys@transformshift{4.111866in}{2.569838in}%
\pgfsys@useobject{currentmarker}{}%
\end{pgfscope}%
\begin{pgfscope}%
\pgfsys@transformshift{4.126581in}{2.585673in}%
\pgfsys@useobject{currentmarker}{}%
\end{pgfscope}%
\begin{pgfscope}%
\pgfsys@transformshift{4.141297in}{2.600336in}%
\pgfsys@useobject{currentmarker}{}%
\end{pgfscope}%
\begin{pgfscope}%
\pgfsys@transformshift{4.156012in}{2.616201in}%
\pgfsys@useobject{currentmarker}{}%
\end{pgfscope}%
\begin{pgfscope}%
\pgfsys@transformshift{4.170727in}{2.622231in}%
\pgfsys@useobject{currentmarker}{}%
\end{pgfscope}%
\begin{pgfscope}%
\pgfsys@transformshift{4.185443in}{2.620784in}%
\pgfsys@useobject{currentmarker}{}%
\end{pgfscope}%
\begin{pgfscope}%
\pgfsys@transformshift{4.200158in}{2.621966in}%
\pgfsys@useobject{currentmarker}{}%
\end{pgfscope}%
\begin{pgfscope}%
\pgfsys@transformshift{4.214873in}{2.624555in}%
\pgfsys@useobject{currentmarker}{}%
\end{pgfscope}%
\begin{pgfscope}%
\pgfsys@transformshift{4.229589in}{2.627482in}%
\pgfsys@useobject{currentmarker}{}%
\end{pgfscope}%
\begin{pgfscope}%
\pgfsys@transformshift{4.244304in}{2.631832in}%
\pgfsys@useobject{currentmarker}{}%
\end{pgfscope}%
\begin{pgfscope}%
\pgfsys@transformshift{4.259019in}{2.634140in}%
\pgfsys@useobject{currentmarker}{}%
\end{pgfscope}%
\begin{pgfscope}%
\pgfsys@transformshift{4.273734in}{2.642769in}%
\pgfsys@useobject{currentmarker}{}%
\end{pgfscope}%
\begin{pgfscope}%
\pgfsys@transformshift{4.288450in}{2.656599in}%
\pgfsys@useobject{currentmarker}{}%
\end{pgfscope}%
\begin{pgfscope}%
\pgfsys@transformshift{4.303165in}{2.681320in}%
\pgfsys@useobject{currentmarker}{}%
\end{pgfscope}%
\begin{pgfscope}%
\pgfsys@transformshift{4.317880in}{2.704411in}%
\pgfsys@useobject{currentmarker}{}%
\end{pgfscope}%
\begin{pgfscope}%
\pgfsys@transformshift{4.332596in}{2.727478in}%
\pgfsys@useobject{currentmarker}{}%
\end{pgfscope}%
\begin{pgfscope}%
\pgfsys@transformshift{4.347311in}{2.738906in}%
\pgfsys@useobject{currentmarker}{}%
\end{pgfscope}%
\begin{pgfscope}%
\pgfsys@transformshift{4.362026in}{2.742298in}%
\pgfsys@useobject{currentmarker}{}%
\end{pgfscope}%
\begin{pgfscope}%
\pgfsys@transformshift{4.376742in}{2.746957in}%
\pgfsys@useobject{currentmarker}{}%
\end{pgfscope}%
\begin{pgfscope}%
\pgfsys@transformshift{4.391457in}{2.755468in}%
\pgfsys@useobject{currentmarker}{}%
\end{pgfscope}%
\begin{pgfscope}%
\pgfsys@transformshift{4.406172in}{2.764177in}%
\pgfsys@useobject{currentmarker}{}%
\end{pgfscope}%
\begin{pgfscope}%
\pgfsys@transformshift{4.420888in}{2.771467in}%
\pgfsys@useobject{currentmarker}{}%
\end{pgfscope}%
\begin{pgfscope}%
\pgfsys@transformshift{4.435603in}{2.773432in}%
\pgfsys@useobject{currentmarker}{}%
\end{pgfscope}%
\begin{pgfscope}%
\pgfsys@transformshift{4.450318in}{2.784919in}%
\pgfsys@useobject{currentmarker}{}%
\end{pgfscope}%
\begin{pgfscope}%
\pgfsys@transformshift{4.465034in}{2.799302in}%
\pgfsys@useobject{currentmarker}{}%
\end{pgfscope}%
\begin{pgfscope}%
\pgfsys@transformshift{4.479749in}{2.816452in}%
\pgfsys@useobject{currentmarker}{}%
\end{pgfscope}%
\begin{pgfscope}%
\pgfsys@transformshift{4.494464in}{2.835248in}%
\pgfsys@useobject{currentmarker}{}%
\end{pgfscope}%
\begin{pgfscope}%
\pgfsys@transformshift{4.509180in}{2.854208in}%
\pgfsys@useobject{currentmarker}{}%
\end{pgfscope}%
\begin{pgfscope}%
\pgfsys@transformshift{4.523895in}{2.864424in}%
\pgfsys@useobject{currentmarker}{}%
\end{pgfscope}%
\begin{pgfscope}%
\pgfsys@transformshift{4.538610in}{2.868236in}%
\pgfsys@useobject{currentmarker}{}%
\end{pgfscope}%
\begin{pgfscope}%
\pgfsys@transformshift{4.553325in}{2.868431in}%
\pgfsys@useobject{currentmarker}{}%
\end{pgfscope}%
\begin{pgfscope}%
\pgfsys@transformshift{4.568041in}{2.870253in}%
\pgfsys@useobject{currentmarker}{}%
\end{pgfscope}%
\begin{pgfscope}%
\pgfsys@transformshift{4.582756in}{2.873841in}%
\pgfsys@useobject{currentmarker}{}%
\end{pgfscope}%
\begin{pgfscope}%
\pgfsys@transformshift{4.597471in}{2.880155in}%
\pgfsys@useobject{currentmarker}{}%
\end{pgfscope}%
\begin{pgfscope}%
\pgfsys@transformshift{4.612187in}{2.879668in}%
\pgfsys@useobject{currentmarker}{}%
\end{pgfscope}%
\begin{pgfscope}%
\pgfsys@transformshift{4.626902in}{2.887436in}%
\pgfsys@useobject{currentmarker}{}%
\end{pgfscope}%
\begin{pgfscope}%
\pgfsys@transformshift{4.641617in}{2.900474in}%
\pgfsys@useobject{currentmarker}{}%
\end{pgfscope}%
\begin{pgfscope}%
\pgfsys@transformshift{4.656333in}{2.919607in}%
\pgfsys@useobject{currentmarker}{}%
\end{pgfscope}%
\begin{pgfscope}%
\pgfsys@transformshift{4.671048in}{2.941342in}%
\pgfsys@useobject{currentmarker}{}%
\end{pgfscope}%
\begin{pgfscope}%
\pgfsys@transformshift{4.685763in}{2.958264in}%
\pgfsys@useobject{currentmarker}{}%
\end{pgfscope}%
\begin{pgfscope}%
\pgfsys@transformshift{4.700479in}{2.967696in}%
\pgfsys@useobject{currentmarker}{}%
\end{pgfscope}%
\begin{pgfscope}%
\pgfsys@transformshift{4.715194in}{2.973105in}%
\pgfsys@useobject{currentmarker}{}%
\end{pgfscope}%
\begin{pgfscope}%
\pgfsys@transformshift{4.729909in}{2.976834in}%
\pgfsys@useobject{currentmarker}{}%
\end{pgfscope}%
\begin{pgfscope}%
\pgfsys@transformshift{4.744625in}{2.982621in}%
\pgfsys@useobject{currentmarker}{}%
\end{pgfscope}%
\begin{pgfscope}%
\pgfsys@transformshift{4.759340in}{2.991287in}%
\pgfsys@useobject{currentmarker}{}%
\end{pgfscope}%
\begin{pgfscope}%
\pgfsys@transformshift{4.774055in}{2.997816in}%
\pgfsys@useobject{currentmarker}{}%
\end{pgfscope}%
\begin{pgfscope}%
\pgfsys@transformshift{4.788771in}{2.995560in}%
\pgfsys@useobject{currentmarker}{}%
\end{pgfscope}%
\begin{pgfscope}%
\pgfsys@transformshift{4.803486in}{2.998789in}%
\pgfsys@useobject{currentmarker}{}%
\end{pgfscope}%
\begin{pgfscope}%
\pgfsys@transformshift{4.818201in}{3.011191in}%
\pgfsys@useobject{currentmarker}{}%
\end{pgfscope}%
\begin{pgfscope}%
\pgfsys@transformshift{4.832916in}{3.028163in}%
\pgfsys@useobject{currentmarker}{}%
\end{pgfscope}%
\begin{pgfscope}%
\pgfsys@transformshift{4.847632in}{3.046547in}%
\pgfsys@useobject{currentmarker}{}%
\end{pgfscope}%
\begin{pgfscope}%
\pgfsys@transformshift{4.862347in}{3.062882in}%
\pgfsys@useobject{currentmarker}{}%
\end{pgfscope}%
\begin{pgfscope}%
\pgfsys@transformshift{4.877062in}{3.073341in}%
\pgfsys@useobject{currentmarker}{}%
\end{pgfscope}%
\begin{pgfscope}%
\pgfsys@transformshift{4.891778in}{3.075652in}%
\pgfsys@useobject{currentmarker}{}%
\end{pgfscope}%
\begin{pgfscope}%
\pgfsys@transformshift{4.906493in}{3.076557in}%
\pgfsys@useobject{currentmarker}{}%
\end{pgfscope}%
\begin{pgfscope}%
\pgfsys@transformshift{4.921208in}{3.076403in}%
\pgfsys@useobject{currentmarker}{}%
\end{pgfscope}%
\begin{pgfscope}%
\pgfsys@transformshift{4.935924in}{3.082094in}%
\pgfsys@useobject{currentmarker}{}%
\end{pgfscope}%
\begin{pgfscope}%
\pgfsys@transformshift{4.950639in}{3.089369in}%
\pgfsys@useobject{currentmarker}{}%
\end{pgfscope}%
\begin{pgfscope}%
\pgfsys@transformshift{4.965354in}{3.097760in}%
\pgfsys@useobject{currentmarker}{}%
\end{pgfscope}%
\begin{pgfscope}%
\pgfsys@transformshift{4.980070in}{3.102791in}%
\pgfsys@useobject{currentmarker}{}%
\end{pgfscope}%
\begin{pgfscope}%
\pgfsys@transformshift{4.994785in}{3.114349in}%
\pgfsys@useobject{currentmarker}{}%
\end{pgfscope}%
\begin{pgfscope}%
\pgfsys@transformshift{5.009500in}{3.134065in}%
\pgfsys@useobject{currentmarker}{}%
\end{pgfscope}%
\begin{pgfscope}%
\pgfsys@transformshift{5.024216in}{3.154122in}%
\pgfsys@useobject{currentmarker}{}%
\end{pgfscope}%
\begin{pgfscope}%
\pgfsys@transformshift{5.038931in}{3.168846in}%
\pgfsys@useobject{currentmarker}{}%
\end{pgfscope}%
\begin{pgfscope}%
\pgfsys@transformshift{5.053646in}{3.178015in}%
\pgfsys@useobject{currentmarker}{}%
\end{pgfscope}%
\begin{pgfscope}%
\pgfsys@transformshift{5.068362in}{3.179180in}%
\pgfsys@useobject{currentmarker}{}%
\end{pgfscope}%
\begin{pgfscope}%
\pgfsys@transformshift{5.083077in}{3.181627in}%
\pgfsys@useobject{currentmarker}{}%
\end{pgfscope}%
\begin{pgfscope}%
\pgfsys@transformshift{5.097792in}{3.183830in}%
\pgfsys@useobject{currentmarker}{}%
\end{pgfscope}%
\begin{pgfscope}%
\pgfsys@transformshift{5.112507in}{3.189995in}%
\pgfsys@useobject{currentmarker}{}%
\end{pgfscope}%
\begin{pgfscope}%
\pgfsys@transformshift{5.127223in}{3.196946in}%
\pgfsys@useobject{currentmarker}{}%
\end{pgfscope}%
\begin{pgfscope}%
\pgfsys@transformshift{5.141938in}{3.205241in}%
\pgfsys@useobject{currentmarker}{}%
\end{pgfscope}%
\begin{pgfscope}%
\pgfsys@transformshift{5.156653in}{3.217121in}%
\pgfsys@useobject{currentmarker}{}%
\end{pgfscope}%
\begin{pgfscope}%
\pgfsys@transformshift{5.171369in}{3.235036in}%
\pgfsys@useobject{currentmarker}{}%
\end{pgfscope}%
\begin{pgfscope}%
\pgfsys@transformshift{5.186084in}{3.256358in}%
\pgfsys@useobject{currentmarker}{}%
\end{pgfscope}%
\begin{pgfscope}%
\pgfsys@transformshift{5.200799in}{3.277636in}%
\pgfsys@useobject{currentmarker}{}%
\end{pgfscope}%
\begin{pgfscope}%
\pgfsys@transformshift{5.215515in}{3.296227in}%
\pgfsys@useobject{currentmarker}{}%
\end{pgfscope}%
\begin{pgfscope}%
\pgfsys@transformshift{5.230230in}{3.308430in}%
\pgfsys@useobject{currentmarker}{}%
\end{pgfscope}%
\begin{pgfscope}%
\pgfsys@transformshift{5.244945in}{3.314615in}%
\pgfsys@useobject{currentmarker}{}%
\end{pgfscope}%
\begin{pgfscope}%
\pgfsys@transformshift{5.259661in}{3.321047in}%
\pgfsys@useobject{currentmarker}{}%
\end{pgfscope}%
\begin{pgfscope}%
\pgfsys@transformshift{5.274376in}{3.326562in}%
\pgfsys@useobject{currentmarker}{}%
\end{pgfscope}%
\begin{pgfscope}%
\pgfsys@transformshift{5.289091in}{3.332215in}%
\pgfsys@useobject{currentmarker}{}%
\end{pgfscope}%
\begin{pgfscope}%
\pgfsys@transformshift{5.303807in}{3.335684in}%
\pgfsys@useobject{currentmarker}{}%
\end{pgfscope}%
\begin{pgfscope}%
\pgfsys@transformshift{5.318522in}{3.336187in}%
\pgfsys@useobject{currentmarker}{}%
\end{pgfscope}%
\begin{pgfscope}%
\pgfsys@transformshift{5.333237in}{3.337267in}%
\pgfsys@useobject{currentmarker}{}%
\end{pgfscope}%
\begin{pgfscope}%
\pgfsys@transformshift{5.347953in}{3.347579in}%
\pgfsys@useobject{currentmarker}{}%
\end{pgfscope}%
\begin{pgfscope}%
\pgfsys@transformshift{5.362668in}{3.365491in}%
\pgfsys@useobject{currentmarker}{}%
\end{pgfscope}%
\begin{pgfscope}%
\pgfsys@transformshift{5.377383in}{3.386856in}%
\pgfsys@useobject{currentmarker}{}%
\end{pgfscope}%
\begin{pgfscope}%
\pgfsys@transformshift{5.392098in}{3.404116in}%
\pgfsys@useobject{currentmarker}{}%
\end{pgfscope}%
\begin{pgfscope}%
\pgfsys@transformshift{5.406814in}{3.413431in}%
\pgfsys@useobject{currentmarker}{}%
\end{pgfscope}%
\begin{pgfscope}%
\pgfsys@transformshift{5.421529in}{3.415260in}%
\pgfsys@useobject{currentmarker}{}%
\end{pgfscope}%
\begin{pgfscope}%
\pgfsys@transformshift{5.436244in}{3.417860in}%
\pgfsys@useobject{currentmarker}{}%
\end{pgfscope}%
\begin{pgfscope}%
\pgfsys@transformshift{5.450960in}{3.420638in}%
\pgfsys@useobject{currentmarker}{}%
\end{pgfscope}%
\begin{pgfscope}%
\pgfsys@transformshift{5.465675in}{3.427119in}%
\pgfsys@useobject{currentmarker}{}%
\end{pgfscope}%
\begin{pgfscope}%
\pgfsys@transformshift{5.480390in}{3.433674in}%
\pgfsys@useobject{currentmarker}{}%
\end{pgfscope}%
\begin{pgfscope}%
\pgfsys@transformshift{5.495106in}{3.439641in}%
\pgfsys@useobject{currentmarker}{}%
\end{pgfscope}%
\begin{pgfscope}%
\pgfsys@transformshift{5.509821in}{3.449791in}%
\pgfsys@useobject{currentmarker}{}%
\end{pgfscope}%
\begin{pgfscope}%
\pgfsys@transformshift{5.524536in}{3.465265in}%
\pgfsys@useobject{currentmarker}{}%
\end{pgfscope}%
\begin{pgfscope}%
\pgfsys@transformshift{5.539252in}{3.481686in}%
\pgfsys@useobject{currentmarker}{}%
\end{pgfscope}%
\begin{pgfscope}%
\pgfsys@transformshift{5.553967in}{3.503171in}%
\pgfsys@useobject{currentmarker}{}%
\end{pgfscope}%
\begin{pgfscope}%
\pgfsys@transformshift{5.568682in}{3.520206in}%
\pgfsys@useobject{currentmarker}{}%
\end{pgfscope}%
\begin{pgfscope}%
\pgfsys@transformshift{5.583398in}{3.531028in}%
\pgfsys@useobject{currentmarker}{}%
\end{pgfscope}%
\begin{pgfscope}%
\pgfsys@transformshift{5.598113in}{3.535026in}%
\pgfsys@useobject{currentmarker}{}%
\end{pgfscope}%
\begin{pgfscope}%
\pgfsys@transformshift{5.612828in}{3.541088in}%
\pgfsys@useobject{currentmarker}{}%
\end{pgfscope}%
\begin{pgfscope}%
\pgfsys@transformshift{5.627544in}{3.547493in}%
\pgfsys@useobject{currentmarker}{}%
\end{pgfscope}%
\begin{pgfscope}%
\pgfsys@transformshift{5.642259in}{3.556461in}%
\pgfsys@useobject{currentmarker}{}%
\end{pgfscope}%
\begin{pgfscope}%
\pgfsys@transformshift{5.656974in}{3.569029in}%
\pgfsys@useobject{currentmarker}{}%
\end{pgfscope}%
\begin{pgfscope}%
\pgfsys@transformshift{5.671689in}{3.575642in}%
\pgfsys@useobject{currentmarker}{}%
\end{pgfscope}%
\begin{pgfscope}%
\pgfsys@transformshift{5.686405in}{3.584304in}%
\pgfsys@useobject{currentmarker}{}%
\end{pgfscope}%
\begin{pgfscope}%
\pgfsys@transformshift{5.701120in}{3.605340in}%
\pgfsys@useobject{currentmarker}{}%
\end{pgfscope}%
\begin{pgfscope}%
\pgfsys@transformshift{5.715835in}{3.627308in}%
\pgfsys@useobject{currentmarker}{}%
\end{pgfscope}%
\begin{pgfscope}%
\pgfsys@transformshift{5.730551in}{3.652389in}%
\pgfsys@useobject{currentmarker}{}%
\end{pgfscope}%
\begin{pgfscope}%
\pgfsys@transformshift{5.745266in}{3.673363in}%
\pgfsys@useobject{currentmarker}{}%
\end{pgfscope}%
\begin{pgfscope}%
\pgfsys@transformshift{5.759981in}{3.685967in}%
\pgfsys@useobject{currentmarker}{}%
\end{pgfscope}%
\begin{pgfscope}%
\pgfsys@transformshift{5.774697in}{3.689470in}%
\pgfsys@useobject{currentmarker}{}%
\end{pgfscope}%
\begin{pgfscope}%
\pgfsys@transformshift{5.789412in}{3.693557in}%
\pgfsys@useobject{currentmarker}{}%
\end{pgfscope}%
\begin{pgfscope}%
\pgfsys@transformshift{5.804127in}{3.699849in}%
\pgfsys@useobject{currentmarker}{}%
\end{pgfscope}%
\begin{pgfscope}%
\pgfsys@transformshift{5.818843in}{3.705054in}%
\pgfsys@useobject{currentmarker}{}%
\end{pgfscope}%
\begin{pgfscope}%
\pgfsys@transformshift{5.833558in}{3.708111in}%
\pgfsys@useobject{currentmarker}{}%
\end{pgfscope}%
\begin{pgfscope}%
\pgfsys@transformshift{5.848273in}{3.714668in}%
\pgfsys@useobject{currentmarker}{}%
\end{pgfscope}%
\begin{pgfscope}%
\pgfsys@transformshift{5.862989in}{3.722206in}%
\pgfsys@useobject{currentmarker}{}%
\end{pgfscope}%
\begin{pgfscope}%
\pgfsys@transformshift{5.877704in}{3.738103in}%
\pgfsys@useobject{currentmarker}{}%
\end{pgfscope}%
\begin{pgfscope}%
\pgfsys@transformshift{5.892419in}{3.758595in}%
\pgfsys@useobject{currentmarker}{}%
\end{pgfscope}%
\begin{pgfscope}%
\pgfsys@transformshift{5.907135in}{3.778983in}%
\pgfsys@useobject{currentmarker}{}%
\end{pgfscope}%
\begin{pgfscope}%
\pgfsys@transformshift{5.921850in}{3.796612in}%
\pgfsys@useobject{currentmarker}{}%
\end{pgfscope}%
\begin{pgfscope}%
\pgfsys@transformshift{5.936565in}{3.804793in}%
\pgfsys@useobject{currentmarker}{}%
\end{pgfscope}%
\begin{pgfscope}%
\pgfsys@transformshift{5.951280in}{3.808752in}%
\pgfsys@useobject{currentmarker}{}%
\end{pgfscope}%
\begin{pgfscope}%
\pgfsys@transformshift{5.965996in}{3.810561in}%
\pgfsys@useobject{currentmarker}{}%
\end{pgfscope}%
\begin{pgfscope}%
\pgfsys@transformshift{5.980711in}{3.815976in}%
\pgfsys@useobject{currentmarker}{}%
\end{pgfscope}%
\begin{pgfscope}%
\pgfsys@transformshift{5.995426in}{3.825769in}%
\pgfsys@useobject{currentmarker}{}%
\end{pgfscope}%
\begin{pgfscope}%
\pgfsys@transformshift{6.010142in}{3.829149in}%
\pgfsys@useobject{currentmarker}{}%
\end{pgfscope}%
\begin{pgfscope}%
\pgfsys@transformshift{6.024857in}{3.830599in}%
\pgfsys@useobject{currentmarker}{}%
\end{pgfscope}%
\begin{pgfscope}%
\pgfsys@transformshift{6.039572in}{3.837938in}%
\pgfsys@useobject{currentmarker}{}%
\end{pgfscope}%
\begin{pgfscope}%
\pgfsys@transformshift{6.054288in}{3.851798in}%
\pgfsys@useobject{currentmarker}{}%
\end{pgfscope}%
\begin{pgfscope}%
\pgfsys@transformshift{6.069003in}{3.870187in}%
\pgfsys@useobject{currentmarker}{}%
\end{pgfscope}%
\begin{pgfscope}%
\pgfsys@transformshift{6.083718in}{3.892301in}%
\pgfsys@useobject{currentmarker}{}%
\end{pgfscope}%
\begin{pgfscope}%
\pgfsys@transformshift{6.098434in}{3.910675in}%
\pgfsys@useobject{currentmarker}{}%
\end{pgfscope}%
\begin{pgfscope}%
\pgfsys@transformshift{6.113149in}{3.919343in}%
\pgfsys@useobject{currentmarker}{}%
\end{pgfscope}%
\begin{pgfscope}%
\pgfsys@transformshift{6.127864in}{3.925113in}%
\pgfsys@useobject{currentmarker}{}%
\end{pgfscope}%
\begin{pgfscope}%
\pgfsys@transformshift{6.142580in}{3.931236in}%
\pgfsys@useobject{currentmarker}{}%
\end{pgfscope}%
\begin{pgfscope}%
\pgfsys@transformshift{6.157295in}{3.940934in}%
\pgfsys@useobject{currentmarker}{}%
\end{pgfscope}%
\begin{pgfscope}%
\pgfsys@transformshift{6.172010in}{3.952216in}%
\pgfsys@useobject{currentmarker}{}%
\end{pgfscope}%
\begin{pgfscope}%
\pgfsys@transformshift{6.186726in}{3.965153in}%
\pgfsys@useobject{currentmarker}{}%
\end{pgfscope}%
\begin{pgfscope}%
\pgfsys@transformshift{6.201441in}{3.973759in}%
\pgfsys@useobject{currentmarker}{}%
\end{pgfscope}%
\begin{pgfscope}%
\pgfsys@transformshift{6.216156in}{3.991632in}%
\pgfsys@useobject{currentmarker}{}%
\end{pgfscope}%
\begin{pgfscope}%
\pgfsys@transformshift{6.230871in}{4.017081in}%
\pgfsys@useobject{currentmarker}{}%
\end{pgfscope}%
\begin{pgfscope}%
\pgfsys@transformshift{6.245587in}{4.045501in}%
\pgfsys@useobject{currentmarker}{}%
\end{pgfscope}%
\begin{pgfscope}%
\pgfsys@transformshift{6.260302in}{4.073440in}%
\pgfsys@useobject{currentmarker}{}%
\end{pgfscope}%
\begin{pgfscope}%
\pgfsys@transformshift{6.275017in}{4.097138in}%
\pgfsys@useobject{currentmarker}{}%
\end{pgfscope}%
\begin{pgfscope}%
\pgfsys@transformshift{6.289733in}{4.111170in}%
\pgfsys@useobject{currentmarker}{}%
\end{pgfscope}%
\begin{pgfscope}%
\pgfsys@transformshift{6.304448in}{4.118858in}%
\pgfsys@useobject{currentmarker}{}%
\end{pgfscope}%
\begin{pgfscope}%
\pgfsys@transformshift{6.319163in}{4.127327in}%
\pgfsys@useobject{currentmarker}{}%
\end{pgfscope}%
\begin{pgfscope}%
\pgfsys@transformshift{6.333879in}{4.137115in}%
\pgfsys@useobject{currentmarker}{}%
\end{pgfscope}%
\begin{pgfscope}%
\pgfsys@transformshift{6.348594in}{4.147584in}%
\pgfsys@useobject{currentmarker}{}%
\end{pgfscope}%
\begin{pgfscope}%
\pgfsys@transformshift{6.363309in}{4.155727in}%
\pgfsys@useobject{currentmarker}{}%
\end{pgfscope}%
\begin{pgfscope}%
\pgfsys@transformshift{6.378025in}{4.155486in}%
\pgfsys@useobject{currentmarker}{}%
\end{pgfscope}%
\begin{pgfscope}%
\pgfsys@transformshift{6.392740in}{4.162292in}%
\pgfsys@useobject{currentmarker}{}%
\end{pgfscope}%
\begin{pgfscope}%
\pgfsys@transformshift{6.407455in}{4.177077in}%
\pgfsys@useobject{currentmarker}{}%
\end{pgfscope}%
\begin{pgfscope}%
\pgfsys@transformshift{6.422171in}{4.200509in}%
\pgfsys@useobject{currentmarker}{}%
\end{pgfscope}%
\begin{pgfscope}%
\pgfsys@transformshift{6.436886in}{4.225524in}%
\pgfsys@useobject{currentmarker}{}%
\end{pgfscope}%
\begin{pgfscope}%
\pgfsys@transformshift{6.451601in}{4.246301in}%
\pgfsys@useobject{currentmarker}{}%
\end{pgfscope}%
\begin{pgfscope}%
\pgfsys@transformshift{6.466317in}{4.255530in}%
\pgfsys@useobject{currentmarker}{}%
\end{pgfscope}%
\begin{pgfscope}%
\pgfsys@transformshift{6.481032in}{4.256028in}%
\pgfsys@useobject{currentmarker}{}%
\end{pgfscope}%
\begin{pgfscope}%
\pgfsys@transformshift{6.495747in}{4.259007in}%
\pgfsys@useobject{currentmarker}{}%
\end{pgfscope}%
\begin{pgfscope}%
\pgfsys@transformshift{6.510463in}{4.262429in}%
\pgfsys@useobject{currentmarker}{}%
\end{pgfscope}%
\begin{pgfscope}%
\pgfsys@transformshift{6.525178in}{4.270197in}%
\pgfsys@useobject{currentmarker}{}%
\end{pgfscope}%
\begin{pgfscope}%
\pgfsys@transformshift{6.539893in}{4.275032in}%
\pgfsys@useobject{currentmarker}{}%
\end{pgfscope}%
\begin{pgfscope}%
\pgfsys@transformshift{6.554608in}{4.276797in}%
\pgfsys@useobject{currentmarker}{}%
\end{pgfscope}%
\begin{pgfscope}%
\pgfsys@transformshift{6.569324in}{4.279614in}%
\pgfsys@useobject{currentmarker}{}%
\end{pgfscope}%
\begin{pgfscope}%
\pgfsys@transformshift{6.584039in}{4.291671in}%
\pgfsys@useobject{currentmarker}{}%
\end{pgfscope}%
\begin{pgfscope}%
\pgfsys@transformshift{6.598754in}{4.310608in}%
\pgfsys@useobject{currentmarker}{}%
\end{pgfscope}%
\begin{pgfscope}%
\pgfsys@transformshift{6.613470in}{4.328942in}%
\pgfsys@useobject{currentmarker}{}%
\end{pgfscope}%
\begin{pgfscope}%
\pgfsys@transformshift{6.628185in}{4.347268in}%
\pgfsys@useobject{currentmarker}{}%
\end{pgfscope}%
\begin{pgfscope}%
\pgfsys@transformshift{6.642900in}{4.357051in}%
\pgfsys@useobject{currentmarker}{}%
\end{pgfscope}%
\begin{pgfscope}%
\pgfsys@transformshift{6.657616in}{4.361934in}%
\pgfsys@useobject{currentmarker}{}%
\end{pgfscope}%
\begin{pgfscope}%
\pgfsys@transformshift{6.672331in}{4.368484in}%
\pgfsys@useobject{currentmarker}{}%
\end{pgfscope}%
\begin{pgfscope}%
\pgfsys@transformshift{6.687046in}{4.374283in}%
\pgfsys@useobject{currentmarker}{}%
\end{pgfscope}%
\begin{pgfscope}%
\pgfsys@transformshift{6.701762in}{4.387235in}%
\pgfsys@useobject{currentmarker}{}%
\end{pgfscope}%
\begin{pgfscope}%
\pgfsys@transformshift{6.716477in}{4.399474in}%
\pgfsys@useobject{currentmarker}{}%
\end{pgfscope}%
\begin{pgfscope}%
\pgfsys@transformshift{6.731192in}{4.408226in}%
\pgfsys@useobject{currentmarker}{}%
\end{pgfscope}%
\begin{pgfscope}%
\pgfsys@transformshift{6.745908in}{4.419128in}%
\pgfsys@useobject{currentmarker}{}%
\end{pgfscope}%
\begin{pgfscope}%
\pgfsys@transformshift{6.760623in}{4.438960in}%
\pgfsys@useobject{currentmarker}{}%
\end{pgfscope}%
\begin{pgfscope}%
\pgfsys@transformshift{6.775338in}{4.466111in}%
\pgfsys@useobject{currentmarker}{}%
\end{pgfscope}%
\begin{pgfscope}%
\pgfsys@transformshift{6.790054in}{4.491011in}%
\pgfsys@useobject{currentmarker}{}%
\end{pgfscope}%
\begin{pgfscope}%
\pgfsys@transformshift{6.804769in}{4.514060in}%
\pgfsys@useobject{currentmarker}{}%
\end{pgfscope}%
\begin{pgfscope}%
\pgfsys@transformshift{6.819484in}{4.527259in}%
\pgfsys@useobject{currentmarker}{}%
\end{pgfscope}%
\begin{pgfscope}%
\pgfsys@transformshift{6.834199in}{4.529564in}%
\pgfsys@useobject{currentmarker}{}%
\end{pgfscope}%
\begin{pgfscope}%
\pgfsys@transformshift{6.848915in}{4.535839in}%
\pgfsys@useobject{currentmarker}{}%
\end{pgfscope}%
\begin{pgfscope}%
\pgfsys@transformshift{6.863630in}{4.540569in}%
\pgfsys@useobject{currentmarker}{}%
\end{pgfscope}%
\begin{pgfscope}%
\pgfsys@transformshift{6.878345in}{4.548996in}%
\pgfsys@useobject{currentmarker}{}%
\end{pgfscope}%
\begin{pgfscope}%
\pgfsys@transformshift{6.893061in}{4.560244in}%
\pgfsys@useobject{currentmarker}{}%
\end{pgfscope}%
\begin{pgfscope}%
\pgfsys@transformshift{6.907776in}{4.566068in}%
\pgfsys@useobject{currentmarker}{}%
\end{pgfscope}%
\begin{pgfscope}%
\pgfsys@transformshift{6.922491in}{4.574068in}%
\pgfsys@useobject{currentmarker}{}%
\end{pgfscope}%
\begin{pgfscope}%
\pgfsys@transformshift{6.937207in}{4.590186in}%
\pgfsys@useobject{currentmarker}{}%
\end{pgfscope}%
\begin{pgfscope}%
\pgfsys@transformshift{6.951922in}{4.613428in}%
\pgfsys@useobject{currentmarker}{}%
\end{pgfscope}%
\begin{pgfscope}%
\pgfsys@transformshift{6.966637in}{4.635982in}%
\pgfsys@useobject{currentmarker}{}%
\end{pgfscope}%
\begin{pgfscope}%
\pgfsys@transformshift{6.981353in}{4.656016in}%
\pgfsys@useobject{currentmarker}{}%
\end{pgfscope}%
\begin{pgfscope}%
\pgfsys@transformshift{6.996068in}{4.670281in}%
\pgfsys@useobject{currentmarker}{}%
\end{pgfscope}%
\begin{pgfscope}%
\pgfsys@transformshift{7.010783in}{4.675262in}%
\pgfsys@useobject{currentmarker}{}%
\end{pgfscope}%
\begin{pgfscope}%
\pgfsys@transformshift{7.025499in}{4.681072in}%
\pgfsys@useobject{currentmarker}{}%
\end{pgfscope}%
\begin{pgfscope}%
\pgfsys@transformshift{7.040214in}{4.684230in}%
\pgfsys@useobject{currentmarker}{}%
\end{pgfscope}%
\begin{pgfscope}%
\pgfsys@transformshift{7.054929in}{4.690170in}%
\pgfsys@useobject{currentmarker}{}%
\end{pgfscope}%
\begin{pgfscope}%
\pgfsys@transformshift{7.069645in}{4.700575in}%
\pgfsys@useobject{currentmarker}{}%
\end{pgfscope}%
\begin{pgfscope}%
\pgfsys@transformshift{7.084360in}{4.706463in}%
\pgfsys@useobject{currentmarker}{}%
\end{pgfscope}%
\begin{pgfscope}%
\pgfsys@transformshift{7.099075in}{4.715296in}%
\pgfsys@useobject{currentmarker}{}%
\end{pgfscope}%
\begin{pgfscope}%
\pgfsys@transformshift{7.113790in}{4.728393in}%
\pgfsys@useobject{currentmarker}{}%
\end{pgfscope}%
\begin{pgfscope}%
\pgfsys@transformshift{7.128506in}{4.750686in}%
\pgfsys@useobject{currentmarker}{}%
\end{pgfscope}%
\begin{pgfscope}%
\pgfsys@transformshift{7.143221in}{4.772414in}%
\pgfsys@useobject{currentmarker}{}%
\end{pgfscope}%
\begin{pgfscope}%
\pgfsys@transformshift{7.157936in}{4.787913in}%
\pgfsys@useobject{currentmarker}{}%
\end{pgfscope}%
\begin{pgfscope}%
\pgfsys@transformshift{7.172652in}{4.797071in}%
\pgfsys@useobject{currentmarker}{}%
\end{pgfscope}%
\begin{pgfscope}%
\pgfsys@transformshift{7.187367in}{4.801078in}%
\pgfsys@useobject{currentmarker}{}%
\end{pgfscope}%
\begin{pgfscope}%
\pgfsys@transformshift{7.202082in}{4.804962in}%
\pgfsys@useobject{currentmarker}{}%
\end{pgfscope}%
\begin{pgfscope}%
\pgfsys@transformshift{7.216798in}{4.811099in}%
\pgfsys@useobject{currentmarker}{}%
\end{pgfscope}%
\begin{pgfscope}%
\pgfsys@transformshift{7.231513in}{4.818262in}%
\pgfsys@useobject{currentmarker}{}%
\end{pgfscope}%
\begin{pgfscope}%
\pgfsys@transformshift{7.246228in}{4.826838in}%
\pgfsys@useobject{currentmarker}{}%
\end{pgfscope}%
\begin{pgfscope}%
\pgfsys@transformshift{7.260944in}{4.825589in}%
\pgfsys@useobject{currentmarker}{}%
\end{pgfscope}%
\begin{pgfscope}%
\pgfsys@transformshift{7.275659in}{4.831321in}%
\pgfsys@useobject{currentmarker}{}%
\end{pgfscope}%
\begin{pgfscope}%
\pgfsys@transformshift{7.290374in}{4.841690in}%
\pgfsys@useobject{currentmarker}{}%
\end{pgfscope}%
\begin{pgfscope}%
\pgfsys@transformshift{7.305090in}{4.862388in}%
\pgfsys@useobject{currentmarker}{}%
\end{pgfscope}%
\begin{pgfscope}%
\pgfsys@transformshift{7.319805in}{4.884613in}%
\pgfsys@useobject{currentmarker}{}%
\end{pgfscope}%
\begin{pgfscope}%
\pgfsys@transformshift{7.334520in}{4.904716in}%
\pgfsys@useobject{currentmarker}{}%
\end{pgfscope}%
\begin{pgfscope}%
\pgfsys@transformshift{7.349236in}{4.915035in}%
\pgfsys@useobject{currentmarker}{}%
\end{pgfscope}%
\begin{pgfscope}%
\pgfsys@transformshift{7.363951in}{4.918883in}%
\pgfsys@useobject{currentmarker}{}%
\end{pgfscope}%
\begin{pgfscope}%
\pgfsys@transformshift{7.378666in}{4.922938in}%
\pgfsys@useobject{currentmarker}{}%
\end{pgfscope}%
\begin{pgfscope}%
\pgfsys@transformshift{7.393381in}{4.930079in}%
\pgfsys@useobject{currentmarker}{}%
\end{pgfscope}%
\begin{pgfscope}%
\pgfsys@transformshift{7.408097in}{4.940218in}%
\pgfsys@useobject{currentmarker}{}%
\end{pgfscope}%
\begin{pgfscope}%
\pgfsys@transformshift{7.422812in}{4.959061in}%
\pgfsys@useobject{currentmarker}{}%
\end{pgfscope}%
\begin{pgfscope}%
\pgfsys@transformshift{7.437527in}{4.975258in}%
\pgfsys@useobject{currentmarker}{}%
\end{pgfscope}%
\begin{pgfscope}%
\pgfsys@transformshift{7.452243in}{4.989732in}%
\pgfsys@useobject{currentmarker}{}%
\end{pgfscope}%
\begin{pgfscope}%
\pgfsys@transformshift{7.466958in}{5.003848in}%
\pgfsys@useobject{currentmarker}{}%
\end{pgfscope}%
\begin{pgfscope}%
\pgfsys@transformshift{7.481673in}{5.025107in}%
\pgfsys@useobject{currentmarker}{}%
\end{pgfscope}%
\begin{pgfscope}%
\pgfsys@transformshift{7.496389in}{5.047913in}%
\pgfsys@useobject{currentmarker}{}%
\end{pgfscope}%
\begin{pgfscope}%
\pgfsys@transformshift{7.511104in}{5.067588in}%
\pgfsys@useobject{currentmarker}{}%
\end{pgfscope}%
\begin{pgfscope}%
\pgfsys@transformshift{7.525819in}{5.081354in}%
\pgfsys@useobject{currentmarker}{}%
\end{pgfscope}%
\begin{pgfscope}%
\pgfsys@transformshift{7.540535in}{5.088342in}%
\pgfsys@useobject{currentmarker}{}%
\end{pgfscope}%
\begin{pgfscope}%
\pgfsys@transformshift{7.555250in}{5.092673in}%
\pgfsys@useobject{currentmarker}{}%
\end{pgfscope}%
\begin{pgfscope}%
\pgfsys@transformshift{7.569965in}{5.096181in}%
\pgfsys@useobject{currentmarker}{}%
\end{pgfscope}%
\begin{pgfscope}%
\pgfsys@transformshift{7.584681in}{5.098267in}%
\pgfsys@useobject{currentmarker}{}%
\end{pgfscope}%
\begin{pgfscope}%
\pgfsys@transformshift{7.599396in}{5.104478in}%
\pgfsys@useobject{currentmarker}{}%
\end{pgfscope}%
\begin{pgfscope}%
\pgfsys@transformshift{7.614111in}{5.112573in}%
\pgfsys@useobject{currentmarker}{}%
\end{pgfscope}%
\begin{pgfscope}%
\pgfsys@transformshift{7.628827in}{5.127426in}%
\pgfsys@useobject{currentmarker}{}%
\end{pgfscope}%
\begin{pgfscope}%
\pgfsys@transformshift{7.643542in}{5.147461in}%
\pgfsys@useobject{currentmarker}{}%
\end{pgfscope}%
\begin{pgfscope}%
\pgfsys@transformshift{7.658257in}{5.174374in}%
\pgfsys@useobject{currentmarker}{}%
\end{pgfscope}%
\begin{pgfscope}%
\pgfsys@transformshift{7.672972in}{5.204844in}%
\pgfsys@useobject{currentmarker}{}%
\end{pgfscope}%
\begin{pgfscope}%
\pgfsys@transformshift{7.687688in}{5.228337in}%
\pgfsys@useobject{currentmarker}{}%
\end{pgfscope}%
\begin{pgfscope}%
\pgfsys@transformshift{7.702403in}{5.244887in}%
\pgfsys@useobject{currentmarker}{}%
\end{pgfscope}%
\begin{pgfscope}%
\pgfsys@transformshift{7.717118in}{5.255499in}%
\pgfsys@useobject{currentmarker}{}%
\end{pgfscope}%
\begin{pgfscope}%
\pgfsys@transformshift{7.731834in}{5.265623in}%
\pgfsys@useobject{currentmarker}{}%
\end{pgfscope}%
\begin{pgfscope}%
\pgfsys@transformshift{7.746549in}{5.278841in}%
\pgfsys@useobject{currentmarker}{}%
\end{pgfscope}%
\end{pgfscope}%
\begin{pgfscope}%
\pgfsetrectcap%
\pgfsetmiterjoin%
\pgfsetlinewidth{0.803000pt}%
\definecolor{currentstroke}{rgb}{0.000000,0.000000,0.000000}%
\pgfsetstrokecolor{currentstroke}%
\pgfsetdash{}{0pt}%
\pgfpathmoveto{\pgfqpoint{0.697913in}{0.559721in}}%
\pgfpathlineto{\pgfqpoint{0.697913in}{5.550000in}}%
\pgfusepath{stroke}%
\end{pgfscope}%
\begin{pgfscope}%
\pgfsetrectcap%
\pgfsetmiterjoin%
\pgfsetlinewidth{0.803000pt}%
\definecolor{currentstroke}{rgb}{0.000000,0.000000,0.000000}%
\pgfsetstrokecolor{currentstroke}%
\pgfsetdash{}{0pt}%
\pgfpathmoveto{\pgfqpoint{7.746549in}{0.559721in}}%
\pgfpathlineto{\pgfqpoint{7.746549in}{5.550000in}}%
\pgfusepath{stroke}%
\end{pgfscope}%
\begin{pgfscope}%
\pgfsetrectcap%
\pgfsetmiterjoin%
\pgfsetlinewidth{0.803000pt}%
\definecolor{currentstroke}{rgb}{0.000000,0.000000,0.000000}%
\pgfsetstrokecolor{currentstroke}%
\pgfsetdash{}{0pt}%
\pgfpathmoveto{\pgfqpoint{0.697913in}{0.559721in}}%
\pgfpathlineto{\pgfqpoint{7.746549in}{0.559721in}}%
\pgfusepath{stroke}%
\end{pgfscope}%
\begin{pgfscope}%
\pgfsetrectcap%
\pgfsetmiterjoin%
\pgfsetlinewidth{0.803000pt}%
\definecolor{currentstroke}{rgb}{0.000000,0.000000,0.000000}%
\pgfsetstrokecolor{currentstroke}%
\pgfsetdash{}{0pt}%
\pgfpathmoveto{\pgfqpoint{0.697913in}{5.550000in}}%
\pgfpathlineto{\pgfqpoint{7.746549in}{5.550000in}}%
\pgfusepath{stroke}%
\end{pgfscope}%
\begin{pgfscope}%
\definecolor{textcolor}{rgb}{0.000000,0.000000,0.000000}%
\pgfsetstrokecolor{textcolor}%
\pgfsetfillcolor{textcolor}%
\pgftext[x=4.222231in,y=5.633333in,,base]{\color{textcolor}{\rmfamily\fontsize{18.000000}{21.600000}\selectfont\catcode`\^=\active\def^{\ifmmode\sp\else\^{}\fi}\catcode`\%=\active\def%{\%}Atmospheric CO$_2$ at Mauna Loa Observatory}}%
\end{pgfscope}%
\begin{pgfscope}%
\pgfsetbuttcap%
\pgfsetmiterjoin%
\definecolor{currentfill}{rgb}{1.000000,1.000000,1.000000}%
\pgfsetfillcolor{currentfill}%
\pgfsetfillopacity{0.800000}%
\pgfsetlinewidth{1.003750pt}%
\definecolor{currentstroke}{rgb}{0.800000,0.800000,0.800000}%
\pgfsetstrokecolor{currentstroke}%
\pgfsetstrokeopacity{0.800000}%
\pgfsetdash{}{0pt}%
\pgfpathmoveto{\pgfqpoint{0.834024in}{4.844445in}}%
\pgfpathlineto{\pgfqpoint{2.666486in}{4.844445in}}%
\pgfpathquadraticcurveto{\pgfqpoint{2.705375in}{4.844445in}}{\pgfqpoint{2.705375in}{4.883334in}}%
\pgfpathlineto{\pgfqpoint{2.705375in}{5.413889in}}%
\pgfpathquadraticcurveto{\pgfqpoint{2.705375in}{5.452778in}}{\pgfqpoint{2.666486in}{5.452778in}}%
\pgfpathlineto{\pgfqpoint{0.834024in}{5.452778in}}%
\pgfpathquadraticcurveto{\pgfqpoint{0.795135in}{5.452778in}}{\pgfqpoint{0.795135in}{5.413889in}}%
\pgfpathlineto{\pgfqpoint{0.795135in}{4.883334in}}%
\pgfpathquadraticcurveto{\pgfqpoint{0.795135in}{4.844445in}}{\pgfqpoint{0.834024in}{4.844445in}}%
\pgfpathlineto{\pgfqpoint{0.834024in}{4.844445in}}%
\pgfpathclose%
\pgfusepath{stroke,fill}%
\end{pgfscope}%
\begin{pgfscope}%
\pgfsetrectcap%
\pgfsetroundjoin%
\pgfsetlinewidth{1.003750pt}%
\definecolor{currentstroke}{rgb}{1.000000,0.000000,0.000000}%
\pgfsetstrokecolor{currentstroke}%
\pgfsetdash{}{0pt}%
\pgfpathmoveto{\pgfqpoint{0.872913in}{5.304167in}}%
\pgfpathlineto{\pgfqpoint{1.067357in}{5.304167in}}%
\pgfpathlineto{\pgfqpoint{1.261802in}{5.304167in}}%
\pgfusepath{stroke}%
\end{pgfscope}%
\begin{pgfscope}%
\pgfsetbuttcap%
\pgfsetroundjoin%
\definecolor{currentfill}{rgb}{1.000000,0.000000,0.000000}%
\pgfsetfillcolor{currentfill}%
\pgfsetlinewidth{1.003750pt}%
\definecolor{currentstroke}{rgb}{1.000000,0.000000,0.000000}%
\pgfsetstrokecolor{currentstroke}%
\pgfsetdash{}{0pt}%
\pgfsys@defobject{currentmarker}{\pgfqpoint{-0.020833in}{-0.020833in}}{\pgfqpoint{0.020833in}{0.020833in}}{%
\pgfpathmoveto{\pgfqpoint{0.000000in}{-0.020833in}}%
\pgfpathcurveto{\pgfqpoint{0.005525in}{-0.020833in}}{\pgfqpoint{0.010825in}{-0.018638in}}{\pgfqpoint{0.014731in}{-0.014731in}}%
\pgfpathcurveto{\pgfqpoint{0.018638in}{-0.010825in}}{\pgfqpoint{0.020833in}{-0.005525in}}{\pgfqpoint{0.020833in}{0.000000in}}%
\pgfpathcurveto{\pgfqpoint{0.020833in}{0.005525in}}{\pgfqpoint{0.018638in}{0.010825in}}{\pgfqpoint{0.014731in}{0.014731in}}%
\pgfpathcurveto{\pgfqpoint{0.010825in}{0.018638in}}{\pgfqpoint{0.005525in}{0.020833in}}{\pgfqpoint{0.000000in}{0.020833in}}%
\pgfpathcurveto{\pgfqpoint{-0.005525in}{0.020833in}}{\pgfqpoint{-0.010825in}{0.018638in}}{\pgfqpoint{-0.014731in}{0.014731in}}%
\pgfpathcurveto{\pgfqpoint{-0.018638in}{0.010825in}}{\pgfqpoint{-0.020833in}{0.005525in}}{\pgfqpoint{-0.020833in}{0.000000in}}%
\pgfpathcurveto{\pgfqpoint{-0.020833in}{-0.005525in}}{\pgfqpoint{-0.018638in}{-0.010825in}}{\pgfqpoint{-0.014731in}{-0.014731in}}%
\pgfpathcurveto{\pgfqpoint{-0.010825in}{-0.018638in}}{\pgfqpoint{-0.005525in}{-0.020833in}}{\pgfqpoint{0.000000in}{-0.020833in}}%
\pgfpathlineto{\pgfqpoint{0.000000in}{-0.020833in}}%
\pgfpathclose%
\pgfusepath{stroke,fill}%
}%
\begin{pgfscope}%
\pgfsys@transformshift{1.067357in}{5.304167in}%
\pgfsys@useobject{currentmarker}{}%
\end{pgfscope}%
\end{pgfscope}%
\begin{pgfscope}%
\definecolor{textcolor}{rgb}{0.000000,0.000000,0.000000}%
\pgfsetstrokecolor{textcolor}%
\pgfsetfillcolor{textcolor}%
\pgftext[x=1.417357in,y=5.236111in,left,base]{\color{textcolor}{\rmfamily\fontsize{14.000000}{16.800000}\selectfont\catcode`\^=\active\def^{\ifmmode\sp\else\^{}\fi}\catcode`\%=\active\def%{\%}Monthly Data}}%
\end{pgfscope}%
\begin{pgfscope}%
\pgfsetrectcap%
\pgfsetroundjoin%
\pgfsetlinewidth{1.003750pt}%
\definecolor{currentstroke}{rgb}{0.000000,0.000000,0.000000}%
\pgfsetstrokecolor{currentstroke}%
\pgfsetdash{}{0pt}%
\pgfpathmoveto{\pgfqpoint{0.872913in}{5.029167in}}%
\pgfpathlineto{\pgfqpoint{1.067357in}{5.029167in}}%
\pgfpathlineto{\pgfqpoint{1.261802in}{5.029167in}}%
\pgfusepath{stroke}%
\end{pgfscope}%
\begin{pgfscope}%
\pgfsetbuttcap%
\pgfsetroundjoin%
\definecolor{currentfill}{rgb}{0.000000,0.000000,0.000000}%
\pgfsetfillcolor{currentfill}%
\pgfsetlinewidth{1.003750pt}%
\definecolor{currentstroke}{rgb}{0.000000,0.000000,0.000000}%
\pgfsetstrokecolor{currentstroke}%
\pgfsetdash{}{0pt}%
\pgfsys@defobject{currentmarker}{\pgfqpoint{-0.020833in}{-0.020833in}}{\pgfqpoint{0.020833in}{0.020833in}}{%
\pgfpathmoveto{\pgfqpoint{0.000000in}{-0.020833in}}%
\pgfpathcurveto{\pgfqpoint{0.005525in}{-0.020833in}}{\pgfqpoint{0.010825in}{-0.018638in}}{\pgfqpoint{0.014731in}{-0.014731in}}%
\pgfpathcurveto{\pgfqpoint{0.018638in}{-0.010825in}}{\pgfqpoint{0.020833in}{-0.005525in}}{\pgfqpoint{0.020833in}{0.000000in}}%
\pgfpathcurveto{\pgfqpoint{0.020833in}{0.005525in}}{\pgfqpoint{0.018638in}{0.010825in}}{\pgfqpoint{0.014731in}{0.014731in}}%
\pgfpathcurveto{\pgfqpoint{0.010825in}{0.018638in}}{\pgfqpoint{0.005525in}{0.020833in}}{\pgfqpoint{0.000000in}{0.020833in}}%
\pgfpathcurveto{\pgfqpoint{-0.005525in}{0.020833in}}{\pgfqpoint{-0.010825in}{0.018638in}}{\pgfqpoint{-0.014731in}{0.014731in}}%
\pgfpathcurveto{\pgfqpoint{-0.018638in}{0.010825in}}{\pgfqpoint{-0.020833in}{0.005525in}}{\pgfqpoint{-0.020833in}{0.000000in}}%
\pgfpathcurveto{\pgfqpoint{-0.020833in}{-0.005525in}}{\pgfqpoint{-0.018638in}{-0.010825in}}{\pgfqpoint{-0.014731in}{-0.014731in}}%
\pgfpathcurveto{\pgfqpoint{-0.010825in}{-0.018638in}}{\pgfqpoint{-0.005525in}{-0.020833in}}{\pgfqpoint{0.000000in}{-0.020833in}}%
\pgfpathlineto{\pgfqpoint{0.000000in}{-0.020833in}}%
\pgfpathclose%
\pgfusepath{stroke,fill}%
}%
\begin{pgfscope}%
\pgfsys@transformshift{1.067357in}{5.029167in}%
\pgfsys@useobject{currentmarker}{}%
\end{pgfscope}%
\end{pgfscope}%
\begin{pgfscope}%
\definecolor{textcolor}{rgb}{0.000000,0.000000,0.000000}%
\pgfsetstrokecolor{textcolor}%
\pgfsetfillcolor{textcolor}%
\pgftext[x=1.417357in,y=4.961112in,left,base]{\color{textcolor}{\rmfamily\fontsize{14.000000}{16.800000}\selectfont\catcode`\^=\active\def^{\ifmmode\sp\else\^{}\fi}\catcode`\%=\active\def%{\%}Smoothed}}%
\end{pgfscope}%
\end{pgfpicture}%
\makeatother%
\endgroup%
}            
            \caption{Observed increase in CO$_2$ levels at Mauna Loa Observatory
            \cite{kane_atmospheric_1996}.}
            \label{figure:mauna-loa}
        \end{figure}
        
        \column[t]{5cm}
        \begin{figure}
            \centering
            \resizebox{\columnwidth}{!}{%% Creator: Matplotlib, PGF backend
%%
%% To include the figure in your LaTeX document, write
%%   \input{<filename>.pgf}
%%
%% Make sure the required packages are loaded in your preamble
%%   \usepackage{pgf}
%%
%% Also ensure that all the required font packages are loaded; for instance,
%% the lmodern package is sometimes necessary when using math font.
%%   \usepackage{lmodern}
%%
%% Figures using additional raster images can only be included by \input if
%% they are in the same directory as the main LaTeX file. For loading figures
%% from other directories you can use the `import` package
%%   \usepackage{import}
%%
%% and then include the figures with
%%   \import{<path to file>}{<filename>.pgf}
%%
%% Matplotlib used the following preamble
%%   \def\mathdefault#1{#1}
%%   \everymath=\expandafter{\the\everymath\displaystyle}
%%   \IfFileExists{scrextend.sty}{
%%     \usepackage[fontsize=10.000000pt]{scrextend}
%%   }{
%%     \renewcommand{\normalsize}{\fontsize{10.000000}{12.000000}\selectfont}
%%     \normalsize
%%   }
%%   
%%   \ifdefined\pdftexversion\else  % non-pdftex case.
%%     \usepackage{fontspec}
%%     \setmainfont{DejaVuSerif.ttf}[Path=\detokenize{/Users/samdotson/miniforge3/envs/2025-dotson-thesis-2/lib/python3.12/site-packages/matplotlib/mpl-data/fonts/ttf/}]
%%     \setsansfont{DejaVuSans.ttf}[Path=\detokenize{/Users/samdotson/miniforge3/envs/2025-dotson-thesis-2/lib/python3.12/site-packages/matplotlib/mpl-data/fonts/ttf/}]
%%     \setmonofont{DejaVuSansMono.ttf}[Path=\detokenize{/Users/samdotson/miniforge3/envs/2025-dotson-thesis-2/lib/python3.12/site-packages/matplotlib/mpl-data/fonts/ttf/}]
%%   \fi
%%   \makeatletter\@ifpackageloaded{underscore}{}{\usepackage[strings]{underscore}}\makeatother
%%
\begingroup%
\makeatletter%
\begin{pgfpicture}%
\pgfpathrectangle{\pgfpointorigin}{\pgfqpoint{7.878053in}{5.899207in}}%
\pgfusepath{use as bounding box, clip}%
\begin{pgfscope}%
\pgfsetbuttcap%
\pgfsetmiterjoin%
\definecolor{currentfill}{rgb}{1.000000,1.000000,1.000000}%
\pgfsetfillcolor{currentfill}%
\pgfsetlinewidth{0.000000pt}%
\definecolor{currentstroke}{rgb}{1.000000,1.000000,1.000000}%
\pgfsetstrokecolor{currentstroke}%
\pgfsetdash{}{0pt}%
\pgfpathmoveto{\pgfqpoint{0.000000in}{0.000000in}}%
\pgfpathlineto{\pgfqpoint{7.878053in}{0.000000in}}%
\pgfpathlineto{\pgfqpoint{7.878053in}{5.899207in}}%
\pgfpathlineto{\pgfqpoint{0.000000in}{5.899207in}}%
\pgfpathlineto{\pgfqpoint{0.000000in}{0.000000in}}%
\pgfpathclose%
\pgfusepath{fill}%
\end{pgfscope}%
\begin{pgfscope}%
\pgfsetbuttcap%
\pgfsetmiterjoin%
\definecolor{currentfill}{rgb}{1.000000,1.000000,1.000000}%
\pgfsetfillcolor{currentfill}%
\pgfsetlinewidth{0.000000pt}%
\definecolor{currentstroke}{rgb}{0.000000,0.000000,0.000000}%
\pgfsetstrokecolor{currentstroke}%
\pgfsetstrokeopacity{0.000000}%
\pgfsetdash{}{0pt}%
\pgfpathmoveto{\pgfqpoint{0.965831in}{0.388154in}}%
\pgfpathlineto{\pgfqpoint{7.778053in}{0.388154in}}%
\pgfpathlineto{\pgfqpoint{7.778053in}{5.799207in}}%
\pgfpathlineto{\pgfqpoint{0.965831in}{5.799207in}}%
\pgfpathlineto{\pgfqpoint{0.965831in}{0.388154in}}%
\pgfpathclose%
\pgfusepath{fill}%
\end{pgfscope}%
\begin{pgfscope}%
\pgfpathrectangle{\pgfqpoint{0.965831in}{0.388154in}}{\pgfqpoint{6.812222in}{5.411054in}}%
\pgfusepath{clip}%
\pgfsetrectcap%
\pgfsetroundjoin%
\pgfsetlinewidth{0.803000pt}%
\definecolor{currentstroke}{rgb}{0.690196,0.690196,0.690196}%
\pgfsetstrokecolor{currentstroke}%
\pgfsetdash{}{0pt}%
\pgfpathmoveto{\pgfqpoint{1.452418in}{0.388154in}}%
\pgfpathlineto{\pgfqpoint{1.452418in}{5.799207in}}%
\pgfusepath{stroke}%
\end{pgfscope}%
\begin{pgfscope}%
\pgfsetbuttcap%
\pgfsetroundjoin%
\definecolor{currentfill}{rgb}{0.000000,0.000000,0.000000}%
\pgfsetfillcolor{currentfill}%
\pgfsetlinewidth{0.803000pt}%
\definecolor{currentstroke}{rgb}{0.000000,0.000000,0.000000}%
\pgfsetstrokecolor{currentstroke}%
\pgfsetdash{}{0pt}%
\pgfsys@defobject{currentmarker}{\pgfqpoint{0.000000in}{-0.048611in}}{\pgfqpoint{0.000000in}{0.000000in}}{%
\pgfpathmoveto{\pgfqpoint{0.000000in}{0.000000in}}%
\pgfpathlineto{\pgfqpoint{0.000000in}{-0.048611in}}%
\pgfusepath{stroke,fill}%
}%
\begin{pgfscope}%
\pgfsys@transformshift{1.452418in}{0.388154in}%
\pgfsys@useobject{currentmarker}{}%
\end{pgfscope}%
\end{pgfscope}%
\begin{pgfscope}%
\definecolor{textcolor}{rgb}{0.000000,0.000000,0.000000}%
\pgfsetstrokecolor{textcolor}%
\pgfsetfillcolor{textcolor}%
\pgftext[x=1.452418in,y=0.290931in,,top]{\color{textcolor}{\rmfamily\fontsize{14.000000}{16.800000}\selectfont\catcode`\^=\active\def^{\ifmmode\sp\else\^{}\fi}\catcode`\%=\active\def%{\%}Hard coal}}%
\end{pgfscope}%
\begin{pgfscope}%
\pgfpathrectangle{\pgfqpoint{0.965831in}{0.388154in}}{\pgfqpoint{6.812222in}{5.411054in}}%
\pgfusepath{clip}%
\pgfsetrectcap%
\pgfsetroundjoin%
\pgfsetlinewidth{0.803000pt}%
\definecolor{currentstroke}{rgb}{0.690196,0.690196,0.690196}%
\pgfsetstrokecolor{currentstroke}%
\pgfsetdash{}{0pt}%
\pgfpathmoveto{\pgfqpoint{2.425593in}{0.388154in}}%
\pgfpathlineto{\pgfqpoint{2.425593in}{5.799207in}}%
\pgfusepath{stroke}%
\end{pgfscope}%
\begin{pgfscope}%
\pgfsetbuttcap%
\pgfsetroundjoin%
\definecolor{currentfill}{rgb}{0.000000,0.000000,0.000000}%
\pgfsetfillcolor{currentfill}%
\pgfsetlinewidth{0.803000pt}%
\definecolor{currentstroke}{rgb}{0.000000,0.000000,0.000000}%
\pgfsetstrokecolor{currentstroke}%
\pgfsetdash{}{0pt}%
\pgfsys@defobject{currentmarker}{\pgfqpoint{0.000000in}{-0.048611in}}{\pgfqpoint{0.000000in}{0.000000in}}{%
\pgfpathmoveto{\pgfqpoint{0.000000in}{0.000000in}}%
\pgfpathlineto{\pgfqpoint{0.000000in}{-0.048611in}}%
\pgfusepath{stroke,fill}%
}%
\begin{pgfscope}%
\pgfsys@transformshift{2.425593in}{0.388154in}%
\pgfsys@useobject{currentmarker}{}%
\end{pgfscope}%
\end{pgfscope}%
\begin{pgfscope}%
\definecolor{textcolor}{rgb}{0.000000,0.000000,0.000000}%
\pgfsetstrokecolor{textcolor}%
\pgfsetfillcolor{textcolor}%
\pgftext[x=2.425593in,y=0.290931in,,top]{\color{textcolor}{\rmfamily\fontsize{14.000000}{16.800000}\selectfont\catcode`\^=\active\def^{\ifmmode\sp\else\^{}\fi}\catcode`\%=\active\def%{\%}Natural gas}}%
\end{pgfscope}%
\begin{pgfscope}%
\pgfpathrectangle{\pgfqpoint{0.965831in}{0.388154in}}{\pgfqpoint{6.812222in}{5.411054in}}%
\pgfusepath{clip}%
\pgfsetrectcap%
\pgfsetroundjoin%
\pgfsetlinewidth{0.803000pt}%
\definecolor{currentstroke}{rgb}{0.690196,0.690196,0.690196}%
\pgfsetstrokecolor{currentstroke}%
\pgfsetdash{}{0pt}%
\pgfpathmoveto{\pgfqpoint{3.398767in}{0.388154in}}%
\pgfpathlineto{\pgfqpoint{3.398767in}{5.799207in}}%
\pgfusepath{stroke}%
\end{pgfscope}%
\begin{pgfscope}%
\pgfsetbuttcap%
\pgfsetroundjoin%
\definecolor{currentfill}{rgb}{0.000000,0.000000,0.000000}%
\pgfsetfillcolor{currentfill}%
\pgfsetlinewidth{0.803000pt}%
\definecolor{currentstroke}{rgb}{0.000000,0.000000,0.000000}%
\pgfsetstrokecolor{currentstroke}%
\pgfsetdash{}{0pt}%
\pgfsys@defobject{currentmarker}{\pgfqpoint{0.000000in}{-0.048611in}}{\pgfqpoint{0.000000in}{0.000000in}}{%
\pgfpathmoveto{\pgfqpoint{0.000000in}{0.000000in}}%
\pgfpathlineto{\pgfqpoint{0.000000in}{-0.048611in}}%
\pgfusepath{stroke,fill}%
}%
\begin{pgfscope}%
\pgfsys@transformshift{3.398767in}{0.388154in}%
\pgfsys@useobject{currentmarker}{}%
\end{pgfscope}%
\end{pgfscope}%
\begin{pgfscope}%
\definecolor{textcolor}{rgb}{0.000000,0.000000,0.000000}%
\pgfsetstrokecolor{textcolor}%
\pgfsetfillcolor{textcolor}%
\pgftext[x=3.398767in,y=0.290931in,,top]{\color{textcolor}{\rmfamily\fontsize{14.000000}{16.800000}\selectfont\catcode`\^=\active\def^{\ifmmode\sp\else\^{}\fi}\catcode`\%=\active\def%{\%}Hydro}}%
\end{pgfscope}%
\begin{pgfscope}%
\pgfpathrectangle{\pgfqpoint{0.965831in}{0.388154in}}{\pgfqpoint{6.812222in}{5.411054in}}%
\pgfusepath{clip}%
\pgfsetrectcap%
\pgfsetroundjoin%
\pgfsetlinewidth{0.803000pt}%
\definecolor{currentstroke}{rgb}{0.690196,0.690196,0.690196}%
\pgfsetstrokecolor{currentstroke}%
\pgfsetdash{}{0pt}%
\pgfpathmoveto{\pgfqpoint{4.371942in}{0.388154in}}%
\pgfpathlineto{\pgfqpoint{4.371942in}{5.799207in}}%
\pgfusepath{stroke}%
\end{pgfscope}%
\begin{pgfscope}%
\pgfsetbuttcap%
\pgfsetroundjoin%
\definecolor{currentfill}{rgb}{0.000000,0.000000,0.000000}%
\pgfsetfillcolor{currentfill}%
\pgfsetlinewidth{0.803000pt}%
\definecolor{currentstroke}{rgb}{0.000000,0.000000,0.000000}%
\pgfsetstrokecolor{currentstroke}%
\pgfsetdash{}{0pt}%
\pgfsys@defobject{currentmarker}{\pgfqpoint{0.000000in}{-0.048611in}}{\pgfqpoint{0.000000in}{0.000000in}}{%
\pgfpathmoveto{\pgfqpoint{0.000000in}{0.000000in}}%
\pgfpathlineto{\pgfqpoint{0.000000in}{-0.048611in}}%
\pgfusepath{stroke,fill}%
}%
\begin{pgfscope}%
\pgfsys@transformshift{4.371942in}{0.388154in}%
\pgfsys@useobject{currentmarker}{}%
\end{pgfscope}%
\end{pgfscope}%
\begin{pgfscope}%
\definecolor{textcolor}{rgb}{0.000000,0.000000,0.000000}%
\pgfsetstrokecolor{textcolor}%
\pgfsetfillcolor{textcolor}%
\pgftext[x=4.371942in,y=0.290931in,,top]{\color{textcolor}{\rmfamily\fontsize{14.000000}{16.800000}\selectfont\catcode`\^=\active\def^{\ifmmode\sp\else\^{}\fi}\catcode`\%=\active\def%{\%}Nuclear}}%
\end{pgfscope}%
\begin{pgfscope}%
\pgfpathrectangle{\pgfqpoint{0.965831in}{0.388154in}}{\pgfqpoint{6.812222in}{5.411054in}}%
\pgfusepath{clip}%
\pgfsetrectcap%
\pgfsetroundjoin%
\pgfsetlinewidth{0.803000pt}%
\definecolor{currentstroke}{rgb}{0.690196,0.690196,0.690196}%
\pgfsetstrokecolor{currentstroke}%
\pgfsetdash{}{0pt}%
\pgfpathmoveto{\pgfqpoint{5.345117in}{0.388154in}}%
\pgfpathlineto{\pgfqpoint{5.345117in}{5.799207in}}%
\pgfusepath{stroke}%
\end{pgfscope}%
\begin{pgfscope}%
\pgfsetbuttcap%
\pgfsetroundjoin%
\definecolor{currentfill}{rgb}{0.000000,0.000000,0.000000}%
\pgfsetfillcolor{currentfill}%
\pgfsetlinewidth{0.803000pt}%
\definecolor{currentstroke}{rgb}{0.000000,0.000000,0.000000}%
\pgfsetstrokecolor{currentstroke}%
\pgfsetdash{}{0pt}%
\pgfsys@defobject{currentmarker}{\pgfqpoint{0.000000in}{-0.048611in}}{\pgfqpoint{0.000000in}{0.000000in}}{%
\pgfpathmoveto{\pgfqpoint{0.000000in}{0.000000in}}%
\pgfpathlineto{\pgfqpoint{0.000000in}{-0.048611in}}%
\pgfusepath{stroke,fill}%
}%
\begin{pgfscope}%
\pgfsys@transformshift{5.345117in}{0.388154in}%
\pgfsys@useobject{currentmarker}{}%
\end{pgfscope}%
\end{pgfscope}%
\begin{pgfscope}%
\definecolor{textcolor}{rgb}{0.000000,0.000000,0.000000}%
\pgfsetstrokecolor{textcolor}%
\pgfsetfillcolor{textcolor}%
\pgftext[x=5.345117in,y=0.290931in,,top]{\color{textcolor}{\rmfamily\fontsize{14.000000}{16.800000}\selectfont\catcode`\^=\active\def^{\ifmmode\sp\else\^{}\fi}\catcode`\%=\active\def%{\%}CSP}}%
\end{pgfscope}%
\begin{pgfscope}%
\pgfpathrectangle{\pgfqpoint{0.965831in}{0.388154in}}{\pgfqpoint{6.812222in}{5.411054in}}%
\pgfusepath{clip}%
\pgfsetrectcap%
\pgfsetroundjoin%
\pgfsetlinewidth{0.803000pt}%
\definecolor{currentstroke}{rgb}{0.690196,0.690196,0.690196}%
\pgfsetstrokecolor{currentstroke}%
\pgfsetdash{}{0pt}%
\pgfpathmoveto{\pgfqpoint{6.318291in}{0.388154in}}%
\pgfpathlineto{\pgfqpoint{6.318291in}{5.799207in}}%
\pgfusepath{stroke}%
\end{pgfscope}%
\begin{pgfscope}%
\pgfsetbuttcap%
\pgfsetroundjoin%
\definecolor{currentfill}{rgb}{0.000000,0.000000,0.000000}%
\pgfsetfillcolor{currentfill}%
\pgfsetlinewidth{0.803000pt}%
\definecolor{currentstroke}{rgb}{0.000000,0.000000,0.000000}%
\pgfsetstrokecolor{currentstroke}%
\pgfsetdash{}{0pt}%
\pgfsys@defobject{currentmarker}{\pgfqpoint{0.000000in}{-0.048611in}}{\pgfqpoint{0.000000in}{0.000000in}}{%
\pgfpathmoveto{\pgfqpoint{0.000000in}{0.000000in}}%
\pgfpathlineto{\pgfqpoint{0.000000in}{-0.048611in}}%
\pgfusepath{stroke,fill}%
}%
\begin{pgfscope}%
\pgfsys@transformshift{6.318291in}{0.388154in}%
\pgfsys@useobject{currentmarker}{}%
\end{pgfscope}%
\end{pgfscope}%
\begin{pgfscope}%
\definecolor{textcolor}{rgb}{0.000000,0.000000,0.000000}%
\pgfsetstrokecolor{textcolor}%
\pgfsetfillcolor{textcolor}%
\pgftext[x=6.318291in,y=0.290931in,,top]{\color{textcolor}{\rmfamily\fontsize{14.000000}{16.800000}\selectfont\catcode`\^=\active\def^{\ifmmode\sp\else\^{}\fi}\catcode`\%=\active\def%{\%}PV}}%
\end{pgfscope}%
\begin{pgfscope}%
\pgfpathrectangle{\pgfqpoint{0.965831in}{0.388154in}}{\pgfqpoint{6.812222in}{5.411054in}}%
\pgfusepath{clip}%
\pgfsetrectcap%
\pgfsetroundjoin%
\pgfsetlinewidth{0.803000pt}%
\definecolor{currentstroke}{rgb}{0.690196,0.690196,0.690196}%
\pgfsetstrokecolor{currentstroke}%
\pgfsetdash{}{0pt}%
\pgfpathmoveto{\pgfqpoint{7.291466in}{0.388154in}}%
\pgfpathlineto{\pgfqpoint{7.291466in}{5.799207in}}%
\pgfusepath{stroke}%
\end{pgfscope}%
\begin{pgfscope}%
\pgfsetbuttcap%
\pgfsetroundjoin%
\definecolor{currentfill}{rgb}{0.000000,0.000000,0.000000}%
\pgfsetfillcolor{currentfill}%
\pgfsetlinewidth{0.803000pt}%
\definecolor{currentstroke}{rgb}{0.000000,0.000000,0.000000}%
\pgfsetstrokecolor{currentstroke}%
\pgfsetdash{}{0pt}%
\pgfsys@defobject{currentmarker}{\pgfqpoint{0.000000in}{-0.048611in}}{\pgfqpoint{0.000000in}{0.000000in}}{%
\pgfpathmoveto{\pgfqpoint{0.000000in}{0.000000in}}%
\pgfpathlineto{\pgfqpoint{0.000000in}{-0.048611in}}%
\pgfusepath{stroke,fill}%
}%
\begin{pgfscope}%
\pgfsys@transformshift{7.291466in}{0.388154in}%
\pgfsys@useobject{currentmarker}{}%
\end{pgfscope}%
\end{pgfscope}%
\begin{pgfscope}%
\definecolor{textcolor}{rgb}{0.000000,0.000000,0.000000}%
\pgfsetstrokecolor{textcolor}%
\pgfsetfillcolor{textcolor}%
\pgftext[x=7.291466in,y=0.290931in,,top]{\color{textcolor}{\rmfamily\fontsize{14.000000}{16.800000}\selectfont\catcode`\^=\active\def^{\ifmmode\sp\else\^{}\fi}\catcode`\%=\active\def%{\%}Wind}}%
\end{pgfscope}%
\begin{pgfscope}%
\pgfpathrectangle{\pgfqpoint{0.965831in}{0.388154in}}{\pgfqpoint{6.812222in}{5.411054in}}%
\pgfusepath{clip}%
\pgfsetrectcap%
\pgfsetroundjoin%
\pgfsetlinewidth{0.803000pt}%
\definecolor{currentstroke}{rgb}{0.690196,0.690196,0.690196}%
\pgfsetstrokecolor{currentstroke}%
\pgfsetdash{}{0pt}%
\pgfpathmoveto{\pgfqpoint{0.965831in}{0.388154in}}%
\pgfpathlineto{\pgfqpoint{7.778053in}{0.388154in}}%
\pgfusepath{stroke}%
\end{pgfscope}%
\begin{pgfscope}%
\pgfsetbuttcap%
\pgfsetroundjoin%
\definecolor{currentfill}{rgb}{0.000000,0.000000,0.000000}%
\pgfsetfillcolor{currentfill}%
\pgfsetlinewidth{0.803000pt}%
\definecolor{currentstroke}{rgb}{0.000000,0.000000,0.000000}%
\pgfsetstrokecolor{currentstroke}%
\pgfsetdash{}{0pt}%
\pgfsys@defobject{currentmarker}{\pgfqpoint{-0.048611in}{0.000000in}}{\pgfqpoint{-0.000000in}{0.000000in}}{%
\pgfpathmoveto{\pgfqpoint{-0.000000in}{0.000000in}}%
\pgfpathlineto{\pgfqpoint{-0.048611in}{0.000000in}}%
\pgfusepath{stroke,fill}%
}%
\begin{pgfscope}%
\pgfsys@transformshift{0.965831in}{0.388154in}%
\pgfsys@useobject{currentmarker}{}%
\end{pgfscope}%
\end{pgfscope}%
\begin{pgfscope}%
\definecolor{textcolor}{rgb}{0.000000,0.000000,0.000000}%
\pgfsetstrokecolor{textcolor}%
\pgfsetfillcolor{textcolor}%
\pgftext[x=0.744897in, y=0.314288in, left, base]{\color{textcolor}{\rmfamily\fontsize{14.000000}{16.800000}\selectfont\catcode`\^=\active\def^{\ifmmode\sp\else\^{}\fi}\catcode`\%=\active\def%{\%}0}}%
\end{pgfscope}%
\begin{pgfscope}%
\pgfpathrectangle{\pgfqpoint{0.965831in}{0.388154in}}{\pgfqpoint{6.812222in}{5.411054in}}%
\pgfusepath{clip}%
\pgfsetrectcap%
\pgfsetroundjoin%
\pgfsetlinewidth{0.803000pt}%
\definecolor{currentstroke}{rgb}{0.690196,0.690196,0.690196}%
\pgfsetstrokecolor{currentstroke}%
\pgfsetdash{}{0pt}%
\pgfpathmoveto{\pgfqpoint{0.965831in}{1.454371in}}%
\pgfpathlineto{\pgfqpoint{7.778053in}{1.454371in}}%
\pgfusepath{stroke}%
\end{pgfscope}%
\begin{pgfscope}%
\pgfsetbuttcap%
\pgfsetroundjoin%
\definecolor{currentfill}{rgb}{0.000000,0.000000,0.000000}%
\pgfsetfillcolor{currentfill}%
\pgfsetlinewidth{0.803000pt}%
\definecolor{currentstroke}{rgb}{0.000000,0.000000,0.000000}%
\pgfsetstrokecolor{currentstroke}%
\pgfsetdash{}{0pt}%
\pgfsys@defobject{currentmarker}{\pgfqpoint{-0.048611in}{0.000000in}}{\pgfqpoint{-0.000000in}{0.000000in}}{%
\pgfpathmoveto{\pgfqpoint{-0.000000in}{0.000000in}}%
\pgfpathlineto{\pgfqpoint{-0.048611in}{0.000000in}}%
\pgfusepath{stroke,fill}%
}%
\begin{pgfscope}%
\pgfsys@transformshift{0.965831in}{1.454371in}%
\pgfsys@useobject{currentmarker}{}%
\end{pgfscope}%
\end{pgfscope}%
\begin{pgfscope}%
\definecolor{textcolor}{rgb}{0.000000,0.000000,0.000000}%
\pgfsetstrokecolor{textcolor}%
\pgfsetfillcolor{textcolor}%
\pgftext[x=0.497474in, y=1.380505in, left, base]{\color{textcolor}{\rmfamily\fontsize{14.000000}{16.800000}\selectfont\catcode`\^=\active\def^{\ifmmode\sp\else\^{}\fi}\catcode`\%=\active\def%{\%}200}}%
\end{pgfscope}%
\begin{pgfscope}%
\pgfpathrectangle{\pgfqpoint{0.965831in}{0.388154in}}{\pgfqpoint{6.812222in}{5.411054in}}%
\pgfusepath{clip}%
\pgfsetrectcap%
\pgfsetroundjoin%
\pgfsetlinewidth{0.803000pt}%
\definecolor{currentstroke}{rgb}{0.690196,0.690196,0.690196}%
\pgfsetstrokecolor{currentstroke}%
\pgfsetdash{}{0pt}%
\pgfpathmoveto{\pgfqpoint{0.965831in}{2.520589in}}%
\pgfpathlineto{\pgfqpoint{7.778053in}{2.520589in}}%
\pgfusepath{stroke}%
\end{pgfscope}%
\begin{pgfscope}%
\pgfsetbuttcap%
\pgfsetroundjoin%
\definecolor{currentfill}{rgb}{0.000000,0.000000,0.000000}%
\pgfsetfillcolor{currentfill}%
\pgfsetlinewidth{0.803000pt}%
\definecolor{currentstroke}{rgb}{0.000000,0.000000,0.000000}%
\pgfsetstrokecolor{currentstroke}%
\pgfsetdash{}{0pt}%
\pgfsys@defobject{currentmarker}{\pgfqpoint{-0.048611in}{0.000000in}}{\pgfqpoint{-0.000000in}{0.000000in}}{%
\pgfpathmoveto{\pgfqpoint{-0.000000in}{0.000000in}}%
\pgfpathlineto{\pgfqpoint{-0.048611in}{0.000000in}}%
\pgfusepath{stroke,fill}%
}%
\begin{pgfscope}%
\pgfsys@transformshift{0.965831in}{2.520589in}%
\pgfsys@useobject{currentmarker}{}%
\end{pgfscope}%
\end{pgfscope}%
\begin{pgfscope}%
\definecolor{textcolor}{rgb}{0.000000,0.000000,0.000000}%
\pgfsetstrokecolor{textcolor}%
\pgfsetfillcolor{textcolor}%
\pgftext[x=0.497474in, y=2.446722in, left, base]{\color{textcolor}{\rmfamily\fontsize{14.000000}{16.800000}\selectfont\catcode`\^=\active\def^{\ifmmode\sp\else\^{}\fi}\catcode`\%=\active\def%{\%}400}}%
\end{pgfscope}%
\begin{pgfscope}%
\pgfpathrectangle{\pgfqpoint{0.965831in}{0.388154in}}{\pgfqpoint{6.812222in}{5.411054in}}%
\pgfusepath{clip}%
\pgfsetrectcap%
\pgfsetroundjoin%
\pgfsetlinewidth{0.803000pt}%
\definecolor{currentstroke}{rgb}{0.690196,0.690196,0.690196}%
\pgfsetstrokecolor{currentstroke}%
\pgfsetdash{}{0pt}%
\pgfpathmoveto{\pgfqpoint{0.965831in}{3.586806in}}%
\pgfpathlineto{\pgfqpoint{7.778053in}{3.586806in}}%
\pgfusepath{stroke}%
\end{pgfscope}%
\begin{pgfscope}%
\pgfsetbuttcap%
\pgfsetroundjoin%
\definecolor{currentfill}{rgb}{0.000000,0.000000,0.000000}%
\pgfsetfillcolor{currentfill}%
\pgfsetlinewidth{0.803000pt}%
\definecolor{currentstroke}{rgb}{0.000000,0.000000,0.000000}%
\pgfsetstrokecolor{currentstroke}%
\pgfsetdash{}{0pt}%
\pgfsys@defobject{currentmarker}{\pgfqpoint{-0.048611in}{0.000000in}}{\pgfqpoint{-0.000000in}{0.000000in}}{%
\pgfpathmoveto{\pgfqpoint{-0.000000in}{0.000000in}}%
\pgfpathlineto{\pgfqpoint{-0.048611in}{0.000000in}}%
\pgfusepath{stroke,fill}%
}%
\begin{pgfscope}%
\pgfsys@transformshift{0.965831in}{3.586806in}%
\pgfsys@useobject{currentmarker}{}%
\end{pgfscope}%
\end{pgfscope}%
\begin{pgfscope}%
\definecolor{textcolor}{rgb}{0.000000,0.000000,0.000000}%
\pgfsetstrokecolor{textcolor}%
\pgfsetfillcolor{textcolor}%
\pgftext[x=0.497474in, y=3.512940in, left, base]{\color{textcolor}{\rmfamily\fontsize{14.000000}{16.800000}\selectfont\catcode`\^=\active\def^{\ifmmode\sp\else\^{}\fi}\catcode`\%=\active\def%{\%}600}}%
\end{pgfscope}%
\begin{pgfscope}%
\pgfpathrectangle{\pgfqpoint{0.965831in}{0.388154in}}{\pgfqpoint{6.812222in}{5.411054in}}%
\pgfusepath{clip}%
\pgfsetrectcap%
\pgfsetroundjoin%
\pgfsetlinewidth{0.803000pt}%
\definecolor{currentstroke}{rgb}{0.690196,0.690196,0.690196}%
\pgfsetstrokecolor{currentstroke}%
\pgfsetdash{}{0pt}%
\pgfpathmoveto{\pgfqpoint{0.965831in}{4.653023in}}%
\pgfpathlineto{\pgfqpoint{7.778053in}{4.653023in}}%
\pgfusepath{stroke}%
\end{pgfscope}%
\begin{pgfscope}%
\pgfsetbuttcap%
\pgfsetroundjoin%
\definecolor{currentfill}{rgb}{0.000000,0.000000,0.000000}%
\pgfsetfillcolor{currentfill}%
\pgfsetlinewidth{0.803000pt}%
\definecolor{currentstroke}{rgb}{0.000000,0.000000,0.000000}%
\pgfsetstrokecolor{currentstroke}%
\pgfsetdash{}{0pt}%
\pgfsys@defobject{currentmarker}{\pgfqpoint{-0.048611in}{0.000000in}}{\pgfqpoint{-0.000000in}{0.000000in}}{%
\pgfpathmoveto{\pgfqpoint{-0.000000in}{0.000000in}}%
\pgfpathlineto{\pgfqpoint{-0.048611in}{0.000000in}}%
\pgfusepath{stroke,fill}%
}%
\begin{pgfscope}%
\pgfsys@transformshift{0.965831in}{4.653023in}%
\pgfsys@useobject{currentmarker}{}%
\end{pgfscope}%
\end{pgfscope}%
\begin{pgfscope}%
\definecolor{textcolor}{rgb}{0.000000,0.000000,0.000000}%
\pgfsetstrokecolor{textcolor}%
\pgfsetfillcolor{textcolor}%
\pgftext[x=0.497474in, y=4.579157in, left, base]{\color{textcolor}{\rmfamily\fontsize{14.000000}{16.800000}\selectfont\catcode`\^=\active\def^{\ifmmode\sp\else\^{}\fi}\catcode`\%=\active\def%{\%}800}}%
\end{pgfscope}%
\begin{pgfscope}%
\pgfpathrectangle{\pgfqpoint{0.965831in}{0.388154in}}{\pgfqpoint{6.812222in}{5.411054in}}%
\pgfusepath{clip}%
\pgfsetrectcap%
\pgfsetroundjoin%
\pgfsetlinewidth{0.803000pt}%
\definecolor{currentstroke}{rgb}{0.690196,0.690196,0.690196}%
\pgfsetstrokecolor{currentstroke}%
\pgfsetdash{}{0pt}%
\pgfpathmoveto{\pgfqpoint{0.965831in}{5.719241in}}%
\pgfpathlineto{\pgfqpoint{7.778053in}{5.719241in}}%
\pgfusepath{stroke}%
\end{pgfscope}%
\begin{pgfscope}%
\pgfsetbuttcap%
\pgfsetroundjoin%
\definecolor{currentfill}{rgb}{0.000000,0.000000,0.000000}%
\pgfsetfillcolor{currentfill}%
\pgfsetlinewidth{0.803000pt}%
\definecolor{currentstroke}{rgb}{0.000000,0.000000,0.000000}%
\pgfsetstrokecolor{currentstroke}%
\pgfsetdash{}{0pt}%
\pgfsys@defobject{currentmarker}{\pgfqpoint{-0.048611in}{0.000000in}}{\pgfqpoint{-0.000000in}{0.000000in}}{%
\pgfpathmoveto{\pgfqpoint{-0.000000in}{0.000000in}}%
\pgfpathlineto{\pgfqpoint{-0.048611in}{0.000000in}}%
\pgfusepath{stroke,fill}%
}%
\begin{pgfscope}%
\pgfsys@transformshift{0.965831in}{5.719241in}%
\pgfsys@useobject{currentmarker}{}%
\end{pgfscope}%
\end{pgfscope}%
\begin{pgfscope}%
\definecolor{textcolor}{rgb}{0.000000,0.000000,0.000000}%
\pgfsetstrokecolor{textcolor}%
\pgfsetfillcolor{textcolor}%
\pgftext[x=0.373763in, y=5.645375in, left, base]{\color{textcolor}{\rmfamily\fontsize{14.000000}{16.800000}\selectfont\catcode`\^=\active\def^{\ifmmode\sp\else\^{}\fi}\catcode`\%=\active\def%{\%}1000}}%
\end{pgfscope}%
\begin{pgfscope}%
\definecolor{textcolor}{rgb}{0.000000,0.000000,0.000000}%
\pgfsetstrokecolor{textcolor}%
\pgfsetfillcolor{textcolor}%
\pgftext[x=0.318207in,y=3.093680in,,bottom,rotate=90.000000]{\color{textcolor}{\rmfamily\fontsize{16.000000}{19.200000}\selectfont\catcode`\^=\active\def^{\ifmmode\sp\else\^{}\fi}\catcode`\%=\active\def%{\%}Lifecycle Emissions [gCO$_2$eq/kWh]}}%
\end{pgfscope}%
\begin{pgfscope}%
\pgfpathrectangle{\pgfqpoint{0.965831in}{0.388154in}}{\pgfqpoint{6.812222in}{5.411054in}}%
\pgfusepath{clip}%
\pgfsetbuttcap%
\pgfsetmiterjoin%
\definecolor{currentfill}{rgb}{0.706373,0.832843,0.874020}%
\pgfsetfillcolor{currentfill}%
\pgfsetlinewidth{0.000000pt}%
\definecolor{currentstroke}{rgb}{0.000000,0.000000,0.000000}%
\pgfsetstrokecolor{currentstroke}%
\pgfsetstrokeopacity{0.000000}%
\pgfsetdash{}{0pt}%
\pgfpathmoveto{\pgfqpoint{1.063148in}{0.388154in}}%
\pgfpathlineto{\pgfqpoint{1.841688in}{0.388154in}}%
\pgfpathlineto{\pgfqpoint{1.841688in}{5.363835in}}%
\pgfpathlineto{\pgfqpoint{1.063148in}{5.363835in}}%
\pgfpathlineto{\pgfqpoint{1.063148in}{0.388154in}}%
\pgfpathclose%
\pgfusepath{fill}%
\end{pgfscope}%
\begin{pgfscope}%
\pgfpathrectangle{\pgfqpoint{0.965831in}{0.388154in}}{\pgfqpoint{6.812222in}{5.411054in}}%
\pgfusepath{clip}%
\pgfsetbuttcap%
\pgfsetmiterjoin%
\definecolor{currentfill}{rgb}{0.706373,0.832843,0.874020}%
\pgfsetfillcolor{currentfill}%
\pgfsetlinewidth{0.000000pt}%
\definecolor{currentstroke}{rgb}{0.000000,0.000000,0.000000}%
\pgfsetstrokecolor{currentstroke}%
\pgfsetstrokeopacity{0.000000}%
\pgfsetdash{}{0pt}%
\pgfpathmoveto{\pgfqpoint{2.036323in}{0.388154in}}%
\pgfpathlineto{\pgfqpoint{2.814863in}{0.388154in}}%
\pgfpathlineto{\pgfqpoint{2.814863in}{2.680521in}}%
\pgfpathlineto{\pgfqpoint{2.036323in}{2.680521in}}%
\pgfpathlineto{\pgfqpoint{2.036323in}{0.388154in}}%
\pgfpathclose%
\pgfusepath{fill}%
\end{pgfscope}%
\begin{pgfscope}%
\pgfpathrectangle{\pgfqpoint{0.965831in}{0.388154in}}{\pgfqpoint{6.812222in}{5.411054in}}%
\pgfusepath{clip}%
\pgfsetbuttcap%
\pgfsetmiterjoin%
\definecolor{currentfill}{rgb}{0.706373,0.832843,0.874020}%
\pgfsetfillcolor{currentfill}%
\pgfsetlinewidth{0.000000pt}%
\definecolor{currentstroke}{rgb}{0.000000,0.000000,0.000000}%
\pgfsetstrokecolor{currentstroke}%
\pgfsetstrokeopacity{0.000000}%
\pgfsetdash{}{0pt}%
\pgfpathmoveto{\pgfqpoint{3.009498in}{0.388154in}}%
\pgfpathlineto{\pgfqpoint{3.788037in}{0.388154in}}%
\pgfpathlineto{\pgfqpoint{3.788037in}{0.817306in}}%
\pgfpathlineto{\pgfqpoint{3.009498in}{0.817306in}}%
\pgfpathlineto{\pgfqpoint{3.009498in}{0.388154in}}%
\pgfpathclose%
\pgfusepath{fill}%
\end{pgfscope}%
\begin{pgfscope}%
\pgfpathrectangle{\pgfqpoint{0.965831in}{0.388154in}}{\pgfqpoint{6.812222in}{5.411054in}}%
\pgfusepath{clip}%
\pgfsetbuttcap%
\pgfsetmiterjoin%
\definecolor{currentfill}{rgb}{0.706373,0.832843,0.874020}%
\pgfsetfillcolor{currentfill}%
\pgfsetlinewidth{0.000000pt}%
\definecolor{currentstroke}{rgb}{0.000000,0.000000,0.000000}%
\pgfsetstrokecolor{currentstroke}%
\pgfsetstrokeopacity{0.000000}%
\pgfsetdash{}{0pt}%
\pgfpathmoveto{\pgfqpoint{3.982672in}{0.388154in}}%
\pgfpathlineto{\pgfqpoint{4.761212in}{0.388154in}}%
\pgfpathlineto{\pgfqpoint{4.761212in}{0.415342in}}%
\pgfpathlineto{\pgfqpoint{3.982672in}{0.415342in}}%
\pgfpathlineto{\pgfqpoint{3.982672in}{0.388154in}}%
\pgfpathclose%
\pgfusepath{fill}%
\end{pgfscope}%
\begin{pgfscope}%
\pgfpathrectangle{\pgfqpoint{0.965831in}{0.388154in}}{\pgfqpoint{6.812222in}{5.411054in}}%
\pgfusepath{clip}%
\pgfsetbuttcap%
\pgfsetmiterjoin%
\definecolor{currentfill}{rgb}{0.706373,0.832843,0.874020}%
\pgfsetfillcolor{currentfill}%
\pgfsetlinewidth{0.000000pt}%
\definecolor{currentstroke}{rgb}{0.000000,0.000000,0.000000}%
\pgfsetstrokecolor{currentstroke}%
\pgfsetstrokeopacity{0.000000}%
\pgfsetdash{}{0pt}%
\pgfpathmoveto{\pgfqpoint{4.955847in}{0.388154in}}%
\pgfpathlineto{\pgfqpoint{5.734386in}{0.388154in}}%
\pgfpathlineto{\pgfqpoint{5.734386in}{0.558748in}}%
\pgfpathlineto{\pgfqpoint{4.955847in}{0.558748in}}%
\pgfpathlineto{\pgfqpoint{4.955847in}{0.388154in}}%
\pgfpathclose%
\pgfusepath{fill}%
\end{pgfscope}%
\begin{pgfscope}%
\pgfpathrectangle{\pgfqpoint{0.965831in}{0.388154in}}{\pgfqpoint{6.812222in}{5.411054in}}%
\pgfusepath{clip}%
\pgfsetbuttcap%
\pgfsetmiterjoin%
\definecolor{currentfill}{rgb}{0.706373,0.832843,0.874020}%
\pgfsetfillcolor{currentfill}%
\pgfsetlinewidth{0.000000pt}%
\definecolor{currentstroke}{rgb}{0.000000,0.000000,0.000000}%
\pgfsetstrokecolor{currentstroke}%
\pgfsetstrokeopacity{0.000000}%
\pgfsetdash{}{0pt}%
\pgfpathmoveto{\pgfqpoint{5.929021in}{0.388154in}}%
\pgfpathlineto{\pgfqpoint{6.707561in}{0.388154in}}%
\pgfpathlineto{\pgfqpoint{6.707561in}{0.500106in}}%
\pgfpathlineto{\pgfqpoint{5.929021in}{0.500106in}}%
\pgfpathlineto{\pgfqpoint{5.929021in}{0.388154in}}%
\pgfpathclose%
\pgfusepath{fill}%
\end{pgfscope}%
\begin{pgfscope}%
\pgfpathrectangle{\pgfqpoint{0.965831in}{0.388154in}}{\pgfqpoint{6.812222in}{5.411054in}}%
\pgfusepath{clip}%
\pgfsetbuttcap%
\pgfsetmiterjoin%
\definecolor{currentfill}{rgb}{0.706373,0.832843,0.874020}%
\pgfsetfillcolor{currentfill}%
\pgfsetlinewidth{0.000000pt}%
\definecolor{currentstroke}{rgb}{0.000000,0.000000,0.000000}%
\pgfsetstrokecolor{currentstroke}%
\pgfsetstrokeopacity{0.000000}%
\pgfsetdash{}{0pt}%
\pgfpathmoveto{\pgfqpoint{6.902196in}{0.388154in}}%
\pgfpathlineto{\pgfqpoint{7.680736in}{0.388154in}}%
\pgfpathlineto{\pgfqpoint{7.680736in}{0.457458in}}%
\pgfpathlineto{\pgfqpoint{6.902196in}{0.457458in}}%
\pgfpathlineto{\pgfqpoint{6.902196in}{0.388154in}}%
\pgfpathclose%
\pgfusepath{fill}%
\end{pgfscope}%
\begin{pgfscope}%
\pgfpathrectangle{\pgfqpoint{0.965831in}{0.388154in}}{\pgfqpoint{6.812222in}{5.411054in}}%
\pgfusepath{clip}%
\pgfsetrectcap%
\pgfsetroundjoin%
\pgfsetlinewidth{2.258437pt}%
\definecolor{currentstroke}{rgb}{0.260000,0.260000,0.260000}%
\pgfsetstrokecolor{currentstroke}%
\pgfsetdash{}{0pt}%
\pgfpathmoveto{\pgfqpoint{1.355101in}{5.186132in}}%
\pgfpathlineto{\pgfqpoint{1.549736in}{5.186132in}}%
\pgfpathmoveto{\pgfqpoint{1.452418in}{5.186132in}}%
\pgfpathlineto{\pgfqpoint{1.452418in}{5.541538in}}%
\pgfpathmoveto{\pgfqpoint{1.355101in}{5.541538in}}%
\pgfpathlineto{\pgfqpoint{1.549736in}{5.541538in}}%
\pgfusepath{stroke}%
\end{pgfscope}%
\begin{pgfscope}%
\pgfpathrectangle{\pgfqpoint{0.965831in}{0.388154in}}{\pgfqpoint{6.812222in}{5.411054in}}%
\pgfusepath{clip}%
\pgfsetrectcap%
\pgfsetroundjoin%
\pgfsetlinewidth{2.258437pt}%
\definecolor{currentstroke}{rgb}{0.260000,0.260000,0.260000}%
\pgfsetstrokecolor{currentstroke}%
\pgfsetdash{}{0pt}%
\pgfusepath{stroke}%
\end{pgfscope}%
\begin{pgfscope}%
\pgfpathrectangle{\pgfqpoint{0.965831in}{0.388154in}}{\pgfqpoint{6.812222in}{5.411054in}}%
\pgfusepath{clip}%
\pgfsetrectcap%
\pgfsetroundjoin%
\pgfsetlinewidth{2.258437pt}%
\definecolor{currentstroke}{rgb}{0.260000,0.260000,0.260000}%
\pgfsetstrokecolor{currentstroke}%
\pgfsetdash{}{0pt}%
\pgfpathmoveto{\pgfqpoint{3.301450in}{0.446796in}}%
\pgfpathlineto{\pgfqpoint{3.496085in}{0.446796in}}%
\pgfpathmoveto{\pgfqpoint{3.398767in}{0.446796in}}%
\pgfpathlineto{\pgfqpoint{3.398767in}{1.187817in}}%
\pgfpathmoveto{\pgfqpoint{3.301450in}{1.187817in}}%
\pgfpathlineto{\pgfqpoint{3.496085in}{1.187817in}}%
\pgfusepath{stroke}%
\end{pgfscope}%
\begin{pgfscope}%
\pgfpathrectangle{\pgfqpoint{0.965831in}{0.388154in}}{\pgfqpoint{6.812222in}{5.411054in}}%
\pgfusepath{clip}%
\pgfsetrectcap%
\pgfsetroundjoin%
\pgfsetlinewidth{2.258437pt}%
\definecolor{currentstroke}{rgb}{0.260000,0.260000,0.260000}%
\pgfsetstrokecolor{currentstroke}%
\pgfsetdash{}{0pt}%
\pgfusepath{stroke}%
\end{pgfscope}%
\begin{pgfscope}%
\pgfpathrectangle{\pgfqpoint{0.965831in}{0.388154in}}{\pgfqpoint{6.812222in}{5.411054in}}%
\pgfusepath{clip}%
\pgfsetrectcap%
\pgfsetroundjoin%
\pgfsetlinewidth{2.258437pt}%
\definecolor{currentstroke}{rgb}{0.260000,0.260000,0.260000}%
\pgfsetstrokecolor{currentstroke}%
\pgfsetdash{}{0pt}%
\pgfpathmoveto{\pgfqpoint{5.247799in}{0.505438in}}%
\pgfpathlineto{\pgfqpoint{5.442434in}{0.505438in}}%
\pgfpathmoveto{\pgfqpoint{5.345117in}{0.505438in}}%
\pgfpathlineto{\pgfqpoint{5.345117in}{0.612059in}}%
\pgfpathmoveto{\pgfqpoint{5.247799in}{0.612059in}}%
\pgfpathlineto{\pgfqpoint{5.442434in}{0.612059in}}%
\pgfusepath{stroke}%
\end{pgfscope}%
\begin{pgfscope}%
\pgfpathrectangle{\pgfqpoint{0.965831in}{0.388154in}}{\pgfqpoint{6.812222in}{5.411054in}}%
\pgfusepath{clip}%
\pgfsetrectcap%
\pgfsetroundjoin%
\pgfsetlinewidth{2.258437pt}%
\definecolor{currentstroke}{rgb}{0.260000,0.260000,0.260000}%
\pgfsetstrokecolor{currentstroke}%
\pgfsetdash{}{0pt}%
\pgfpathmoveto{\pgfqpoint{6.220974in}{0.477005in}}%
\pgfpathlineto{\pgfqpoint{6.415609in}{0.477005in}}%
\pgfpathmoveto{\pgfqpoint{6.318291in}{0.477005in}}%
\pgfpathlineto{\pgfqpoint{6.318291in}{0.522319in}}%
\pgfpathmoveto{\pgfqpoint{6.220974in}{0.522319in}}%
\pgfpathlineto{\pgfqpoint{6.415609in}{0.522319in}}%
\pgfusepath{stroke}%
\end{pgfscope}%
\begin{pgfscope}%
\pgfpathrectangle{\pgfqpoint{0.965831in}{0.388154in}}{\pgfqpoint{6.812222in}{5.411054in}}%
\pgfusepath{clip}%
\pgfsetrectcap%
\pgfsetroundjoin%
\pgfsetlinewidth{2.258437pt}%
\definecolor{currentstroke}{rgb}{0.260000,0.260000,0.260000}%
\pgfsetstrokecolor{currentstroke}%
\pgfsetdash{}{0pt}%
\pgfpathmoveto{\pgfqpoint{7.194148in}{0.455681in}}%
\pgfpathlineto{\pgfqpoint{7.388783in}{0.455681in}}%
\pgfpathmoveto{\pgfqpoint{7.291466in}{0.455681in}}%
\pgfpathlineto{\pgfqpoint{7.291466in}{0.459235in}}%
\pgfpathmoveto{\pgfqpoint{7.194148in}{0.459235in}}%
\pgfpathlineto{\pgfqpoint{7.388783in}{0.459235in}}%
\pgfusepath{stroke}%
\end{pgfscope}%
\begin{pgfscope}%
\pgfsetrectcap%
\pgfsetmiterjoin%
\pgfsetlinewidth{0.803000pt}%
\definecolor{currentstroke}{rgb}{0.000000,0.000000,0.000000}%
\pgfsetstrokecolor{currentstroke}%
\pgfsetdash{}{0pt}%
\pgfpathmoveto{\pgfqpoint{0.965831in}{0.388154in}}%
\pgfpathlineto{\pgfqpoint{0.965831in}{5.799207in}}%
\pgfusepath{stroke}%
\end{pgfscope}%
\begin{pgfscope}%
\pgfsetrectcap%
\pgfsetmiterjoin%
\pgfsetlinewidth{0.803000pt}%
\definecolor{currentstroke}{rgb}{0.000000,0.000000,0.000000}%
\pgfsetstrokecolor{currentstroke}%
\pgfsetdash{}{0pt}%
\pgfpathmoveto{\pgfqpoint{7.778053in}{0.388154in}}%
\pgfpathlineto{\pgfqpoint{7.778053in}{5.799207in}}%
\pgfusepath{stroke}%
\end{pgfscope}%
\begin{pgfscope}%
\pgfsetrectcap%
\pgfsetmiterjoin%
\pgfsetlinewidth{0.803000pt}%
\definecolor{currentstroke}{rgb}{0.000000,0.000000,0.000000}%
\pgfsetstrokecolor{currentstroke}%
\pgfsetdash{}{0pt}%
\pgfpathmoveto{\pgfqpoint{0.965831in}{0.388154in}}%
\pgfpathlineto{\pgfqpoint{7.778053in}{0.388154in}}%
\pgfusepath{stroke}%
\end{pgfscope}%
\begin{pgfscope}%
\pgfsetrectcap%
\pgfsetmiterjoin%
\pgfsetlinewidth{0.803000pt}%
\definecolor{currentstroke}{rgb}{0.000000,0.000000,0.000000}%
\pgfsetstrokecolor{currentstroke}%
\pgfsetdash{}{0pt}%
\pgfpathmoveto{\pgfqpoint{0.965831in}{5.799207in}}%
\pgfpathlineto{\pgfqpoint{7.778053in}{5.799207in}}%
\pgfusepath{stroke}%
\end{pgfscope}%
\end{pgfpicture}%
\makeatother%
\endgroup%
}
            % \resizebox{\columnwidth}{!}{%% Creator: Matplotlib, PGF backend
%%
%% To include the figure in your LaTeX document, write
%%   \input{<filename>.pgf}
%%
%% Make sure the required packages are loaded in your preamble
%%   \usepackage{pgf}
%%
%% Also ensure that all the required font packages are loaded; for instance,
%% the lmodern package is sometimes necessary when using math font.
%%   \usepackage{lmodern}
%%
%% Figures using additional raster images can only be included by \input if
%% they are in the same directory as the main LaTeX file. For loading figures
%% from other directories you can use the `import` package
%%   \usepackage{import}
%%
%% and then include the figures with
%%   \import{<path to file>}{<filename>.pgf}
%%
%% Matplotlib used the following preamble
%%   \def\mathdefault#1{#1}
%%   \everymath=\expandafter{\the\everymath\displaystyle}
%%   \IfFileExists{scrextend.sty}{
%%     \usepackage[fontsize=10.000000pt]{scrextend}
%%   }{
%%     \renewcommand{\normalsize}{\fontsize{10.000000}{12.000000}\selectfont}
%%     \normalsize
%%   }
%%   
%%   \ifdefined\pdftexversion\else  % non-pdftex case.
%%     \usepackage{fontspec}
%%     \setmainfont{DejaVuSerif.ttf}[Path=\detokenize{/Users/samdotson/miniforge3/envs/2025-dotson-thesis-2/lib/python3.12/site-packages/matplotlib/mpl-data/fonts/ttf/}]
%%     \setsansfont{DejaVuSans.ttf}[Path=\detokenize{/Users/samdotson/miniforge3/envs/2025-dotson-thesis-2/lib/python3.12/site-packages/matplotlib/mpl-data/fonts/ttf/}]
%%     \setmonofont{DejaVuSansMono.ttf}[Path=\detokenize{/Users/samdotson/miniforge3/envs/2025-dotson-thesis-2/lib/python3.12/site-packages/matplotlib/mpl-data/fonts/ttf/}]
%%   \fi
%%   \makeatletter\@ifpackageloaded{underscore}{}{\usepackage[strings]{underscore}}\makeatother
%%
\begingroup%
\makeatletter%
\begin{pgfpicture}%
\pgfpathrectangle{\pgfpointorigin}{\pgfqpoint{7.878053in}{5.899207in}}%
\pgfusepath{use as bounding box, clip}%
\begin{pgfscope}%
\pgfsetbuttcap%
\pgfsetmiterjoin%
\definecolor{currentfill}{rgb}{1.000000,1.000000,1.000000}%
\pgfsetfillcolor{currentfill}%
\pgfsetlinewidth{0.000000pt}%
\definecolor{currentstroke}{rgb}{1.000000,1.000000,1.000000}%
\pgfsetstrokecolor{currentstroke}%
\pgfsetdash{}{0pt}%
\pgfpathmoveto{\pgfqpoint{0.000000in}{0.000000in}}%
\pgfpathlineto{\pgfqpoint{7.878053in}{0.000000in}}%
\pgfpathlineto{\pgfqpoint{7.878053in}{5.899207in}}%
\pgfpathlineto{\pgfqpoint{0.000000in}{5.899207in}}%
\pgfpathlineto{\pgfqpoint{0.000000in}{0.000000in}}%
\pgfpathclose%
\pgfusepath{fill}%
\end{pgfscope}%
\begin{pgfscope}%
\pgfsetbuttcap%
\pgfsetmiterjoin%
\definecolor{currentfill}{rgb}{1.000000,1.000000,1.000000}%
\pgfsetfillcolor{currentfill}%
\pgfsetlinewidth{0.000000pt}%
\definecolor{currentstroke}{rgb}{0.000000,0.000000,0.000000}%
\pgfsetstrokecolor{currentstroke}%
\pgfsetstrokeopacity{0.000000}%
\pgfsetdash{}{0pt}%
\pgfpathmoveto{\pgfqpoint{0.965831in}{0.388154in}}%
\pgfpathlineto{\pgfqpoint{7.778053in}{0.388154in}}%
\pgfpathlineto{\pgfqpoint{7.778053in}{5.799207in}}%
\pgfpathlineto{\pgfqpoint{0.965831in}{5.799207in}}%
\pgfpathlineto{\pgfqpoint{0.965831in}{0.388154in}}%
\pgfpathclose%
\pgfusepath{fill}%
\end{pgfscope}%
\begin{pgfscope}%
\pgfpathrectangle{\pgfqpoint{0.965831in}{0.388154in}}{\pgfqpoint{6.812222in}{5.411054in}}%
\pgfusepath{clip}%
\pgfsetrectcap%
\pgfsetroundjoin%
\pgfsetlinewidth{0.803000pt}%
\definecolor{currentstroke}{rgb}{0.690196,0.690196,0.690196}%
\pgfsetstrokecolor{currentstroke}%
\pgfsetdash{}{0pt}%
\pgfpathmoveto{\pgfqpoint{1.452418in}{0.388154in}}%
\pgfpathlineto{\pgfqpoint{1.452418in}{5.799207in}}%
\pgfusepath{stroke}%
\end{pgfscope}%
\begin{pgfscope}%
\pgfsetbuttcap%
\pgfsetroundjoin%
\definecolor{currentfill}{rgb}{0.000000,0.000000,0.000000}%
\pgfsetfillcolor{currentfill}%
\pgfsetlinewidth{0.803000pt}%
\definecolor{currentstroke}{rgb}{0.000000,0.000000,0.000000}%
\pgfsetstrokecolor{currentstroke}%
\pgfsetdash{}{0pt}%
\pgfsys@defobject{currentmarker}{\pgfqpoint{0.000000in}{-0.048611in}}{\pgfqpoint{0.000000in}{0.000000in}}{%
\pgfpathmoveto{\pgfqpoint{0.000000in}{0.000000in}}%
\pgfpathlineto{\pgfqpoint{0.000000in}{-0.048611in}}%
\pgfusepath{stroke,fill}%
}%
\begin{pgfscope}%
\pgfsys@transformshift{1.452418in}{0.388154in}%
\pgfsys@useobject{currentmarker}{}%
\end{pgfscope}%
\end{pgfscope}%
\begin{pgfscope}%
\definecolor{textcolor}{rgb}{0.000000,0.000000,0.000000}%
\pgfsetstrokecolor{textcolor}%
\pgfsetfillcolor{textcolor}%
\pgftext[x=1.452418in,y=0.290931in,,top]{\color{textcolor}{\rmfamily\fontsize{14.000000}{16.800000}\selectfont\catcode`\^=\active\def^{\ifmmode\sp\else\^{}\fi}\catcode`\%=\active\def%{\%}Hard coal}}%
\end{pgfscope}%
\begin{pgfscope}%
\pgfpathrectangle{\pgfqpoint{0.965831in}{0.388154in}}{\pgfqpoint{6.812222in}{5.411054in}}%
\pgfusepath{clip}%
\pgfsetrectcap%
\pgfsetroundjoin%
\pgfsetlinewidth{0.803000pt}%
\definecolor{currentstroke}{rgb}{0.690196,0.690196,0.690196}%
\pgfsetstrokecolor{currentstroke}%
\pgfsetdash{}{0pt}%
\pgfpathmoveto{\pgfqpoint{2.425593in}{0.388154in}}%
\pgfpathlineto{\pgfqpoint{2.425593in}{5.799207in}}%
\pgfusepath{stroke}%
\end{pgfscope}%
\begin{pgfscope}%
\pgfsetbuttcap%
\pgfsetroundjoin%
\definecolor{currentfill}{rgb}{0.000000,0.000000,0.000000}%
\pgfsetfillcolor{currentfill}%
\pgfsetlinewidth{0.803000pt}%
\definecolor{currentstroke}{rgb}{0.000000,0.000000,0.000000}%
\pgfsetstrokecolor{currentstroke}%
\pgfsetdash{}{0pt}%
\pgfsys@defobject{currentmarker}{\pgfqpoint{0.000000in}{-0.048611in}}{\pgfqpoint{0.000000in}{0.000000in}}{%
\pgfpathmoveto{\pgfqpoint{0.000000in}{0.000000in}}%
\pgfpathlineto{\pgfqpoint{0.000000in}{-0.048611in}}%
\pgfusepath{stroke,fill}%
}%
\begin{pgfscope}%
\pgfsys@transformshift{2.425593in}{0.388154in}%
\pgfsys@useobject{currentmarker}{}%
\end{pgfscope}%
\end{pgfscope}%
\begin{pgfscope}%
\definecolor{textcolor}{rgb}{0.000000,0.000000,0.000000}%
\pgfsetstrokecolor{textcolor}%
\pgfsetfillcolor{textcolor}%
\pgftext[x=2.425593in,y=0.290931in,,top]{\color{textcolor}{\rmfamily\fontsize{14.000000}{16.800000}\selectfont\catcode`\^=\active\def^{\ifmmode\sp\else\^{}\fi}\catcode`\%=\active\def%{\%}Natural gas}}%
\end{pgfscope}%
\begin{pgfscope}%
\pgfpathrectangle{\pgfqpoint{0.965831in}{0.388154in}}{\pgfqpoint{6.812222in}{5.411054in}}%
\pgfusepath{clip}%
\pgfsetrectcap%
\pgfsetroundjoin%
\pgfsetlinewidth{0.803000pt}%
\definecolor{currentstroke}{rgb}{0.690196,0.690196,0.690196}%
\pgfsetstrokecolor{currentstroke}%
\pgfsetdash{}{0pt}%
\pgfpathmoveto{\pgfqpoint{3.398767in}{0.388154in}}%
\pgfpathlineto{\pgfqpoint{3.398767in}{5.799207in}}%
\pgfusepath{stroke}%
\end{pgfscope}%
\begin{pgfscope}%
\pgfsetbuttcap%
\pgfsetroundjoin%
\definecolor{currentfill}{rgb}{0.000000,0.000000,0.000000}%
\pgfsetfillcolor{currentfill}%
\pgfsetlinewidth{0.803000pt}%
\definecolor{currentstroke}{rgb}{0.000000,0.000000,0.000000}%
\pgfsetstrokecolor{currentstroke}%
\pgfsetdash{}{0pt}%
\pgfsys@defobject{currentmarker}{\pgfqpoint{0.000000in}{-0.048611in}}{\pgfqpoint{0.000000in}{0.000000in}}{%
\pgfpathmoveto{\pgfqpoint{0.000000in}{0.000000in}}%
\pgfpathlineto{\pgfqpoint{0.000000in}{-0.048611in}}%
\pgfusepath{stroke,fill}%
}%
\begin{pgfscope}%
\pgfsys@transformshift{3.398767in}{0.388154in}%
\pgfsys@useobject{currentmarker}{}%
\end{pgfscope}%
\end{pgfscope}%
\begin{pgfscope}%
\definecolor{textcolor}{rgb}{0.000000,0.000000,0.000000}%
\pgfsetstrokecolor{textcolor}%
\pgfsetfillcolor{textcolor}%
\pgftext[x=3.398767in,y=0.290931in,,top]{\color{textcolor}{\rmfamily\fontsize{14.000000}{16.800000}\selectfont\catcode`\^=\active\def^{\ifmmode\sp\else\^{}\fi}\catcode`\%=\active\def%{\%}Hydro}}%
\end{pgfscope}%
\begin{pgfscope}%
\pgfpathrectangle{\pgfqpoint{0.965831in}{0.388154in}}{\pgfqpoint{6.812222in}{5.411054in}}%
\pgfusepath{clip}%
\pgfsetrectcap%
\pgfsetroundjoin%
\pgfsetlinewidth{0.803000pt}%
\definecolor{currentstroke}{rgb}{0.690196,0.690196,0.690196}%
\pgfsetstrokecolor{currentstroke}%
\pgfsetdash{}{0pt}%
\pgfpathmoveto{\pgfqpoint{4.371942in}{0.388154in}}%
\pgfpathlineto{\pgfqpoint{4.371942in}{5.799207in}}%
\pgfusepath{stroke}%
\end{pgfscope}%
\begin{pgfscope}%
\pgfsetbuttcap%
\pgfsetroundjoin%
\definecolor{currentfill}{rgb}{0.000000,0.000000,0.000000}%
\pgfsetfillcolor{currentfill}%
\pgfsetlinewidth{0.803000pt}%
\definecolor{currentstroke}{rgb}{0.000000,0.000000,0.000000}%
\pgfsetstrokecolor{currentstroke}%
\pgfsetdash{}{0pt}%
\pgfsys@defobject{currentmarker}{\pgfqpoint{0.000000in}{-0.048611in}}{\pgfqpoint{0.000000in}{0.000000in}}{%
\pgfpathmoveto{\pgfqpoint{0.000000in}{0.000000in}}%
\pgfpathlineto{\pgfqpoint{0.000000in}{-0.048611in}}%
\pgfusepath{stroke,fill}%
}%
\begin{pgfscope}%
\pgfsys@transformshift{4.371942in}{0.388154in}%
\pgfsys@useobject{currentmarker}{}%
\end{pgfscope}%
\end{pgfscope}%
\begin{pgfscope}%
\definecolor{textcolor}{rgb}{0.000000,0.000000,0.000000}%
\pgfsetstrokecolor{textcolor}%
\pgfsetfillcolor{textcolor}%
\pgftext[x=4.371942in,y=0.290931in,,top]{\color{textcolor}{\rmfamily\fontsize{14.000000}{16.800000}\selectfont\catcode`\^=\active\def^{\ifmmode\sp\else\^{}\fi}\catcode`\%=\active\def%{\%}Nuclear}}%
\end{pgfscope}%
\begin{pgfscope}%
\pgfpathrectangle{\pgfqpoint{0.965831in}{0.388154in}}{\pgfqpoint{6.812222in}{5.411054in}}%
\pgfusepath{clip}%
\pgfsetrectcap%
\pgfsetroundjoin%
\pgfsetlinewidth{0.803000pt}%
\definecolor{currentstroke}{rgb}{0.690196,0.690196,0.690196}%
\pgfsetstrokecolor{currentstroke}%
\pgfsetdash{}{0pt}%
\pgfpathmoveto{\pgfqpoint{5.345117in}{0.388154in}}%
\pgfpathlineto{\pgfqpoint{5.345117in}{5.799207in}}%
\pgfusepath{stroke}%
\end{pgfscope}%
\begin{pgfscope}%
\pgfsetbuttcap%
\pgfsetroundjoin%
\definecolor{currentfill}{rgb}{0.000000,0.000000,0.000000}%
\pgfsetfillcolor{currentfill}%
\pgfsetlinewidth{0.803000pt}%
\definecolor{currentstroke}{rgb}{0.000000,0.000000,0.000000}%
\pgfsetstrokecolor{currentstroke}%
\pgfsetdash{}{0pt}%
\pgfsys@defobject{currentmarker}{\pgfqpoint{0.000000in}{-0.048611in}}{\pgfqpoint{0.000000in}{0.000000in}}{%
\pgfpathmoveto{\pgfqpoint{0.000000in}{0.000000in}}%
\pgfpathlineto{\pgfqpoint{0.000000in}{-0.048611in}}%
\pgfusepath{stroke,fill}%
}%
\begin{pgfscope}%
\pgfsys@transformshift{5.345117in}{0.388154in}%
\pgfsys@useobject{currentmarker}{}%
\end{pgfscope}%
\end{pgfscope}%
\begin{pgfscope}%
\definecolor{textcolor}{rgb}{0.000000,0.000000,0.000000}%
\pgfsetstrokecolor{textcolor}%
\pgfsetfillcolor{textcolor}%
\pgftext[x=5.345117in,y=0.290931in,,top]{\color{textcolor}{\rmfamily\fontsize{14.000000}{16.800000}\selectfont\catcode`\^=\active\def^{\ifmmode\sp\else\^{}\fi}\catcode`\%=\active\def%{\%}CSP}}%
\end{pgfscope}%
\begin{pgfscope}%
\pgfpathrectangle{\pgfqpoint{0.965831in}{0.388154in}}{\pgfqpoint{6.812222in}{5.411054in}}%
\pgfusepath{clip}%
\pgfsetrectcap%
\pgfsetroundjoin%
\pgfsetlinewidth{0.803000pt}%
\definecolor{currentstroke}{rgb}{0.690196,0.690196,0.690196}%
\pgfsetstrokecolor{currentstroke}%
\pgfsetdash{}{0pt}%
\pgfpathmoveto{\pgfqpoint{6.318291in}{0.388154in}}%
\pgfpathlineto{\pgfqpoint{6.318291in}{5.799207in}}%
\pgfusepath{stroke}%
\end{pgfscope}%
\begin{pgfscope}%
\pgfsetbuttcap%
\pgfsetroundjoin%
\definecolor{currentfill}{rgb}{0.000000,0.000000,0.000000}%
\pgfsetfillcolor{currentfill}%
\pgfsetlinewidth{0.803000pt}%
\definecolor{currentstroke}{rgb}{0.000000,0.000000,0.000000}%
\pgfsetstrokecolor{currentstroke}%
\pgfsetdash{}{0pt}%
\pgfsys@defobject{currentmarker}{\pgfqpoint{0.000000in}{-0.048611in}}{\pgfqpoint{0.000000in}{0.000000in}}{%
\pgfpathmoveto{\pgfqpoint{0.000000in}{0.000000in}}%
\pgfpathlineto{\pgfqpoint{0.000000in}{-0.048611in}}%
\pgfusepath{stroke,fill}%
}%
\begin{pgfscope}%
\pgfsys@transformshift{6.318291in}{0.388154in}%
\pgfsys@useobject{currentmarker}{}%
\end{pgfscope}%
\end{pgfscope}%
\begin{pgfscope}%
\definecolor{textcolor}{rgb}{0.000000,0.000000,0.000000}%
\pgfsetstrokecolor{textcolor}%
\pgfsetfillcolor{textcolor}%
\pgftext[x=6.318291in,y=0.290931in,,top]{\color{textcolor}{\rmfamily\fontsize{14.000000}{16.800000}\selectfont\catcode`\^=\active\def^{\ifmmode\sp\else\^{}\fi}\catcode`\%=\active\def%{\%}PV}}%
\end{pgfscope}%
\begin{pgfscope}%
\pgfpathrectangle{\pgfqpoint{0.965831in}{0.388154in}}{\pgfqpoint{6.812222in}{5.411054in}}%
\pgfusepath{clip}%
\pgfsetrectcap%
\pgfsetroundjoin%
\pgfsetlinewidth{0.803000pt}%
\definecolor{currentstroke}{rgb}{0.690196,0.690196,0.690196}%
\pgfsetstrokecolor{currentstroke}%
\pgfsetdash{}{0pt}%
\pgfpathmoveto{\pgfqpoint{7.291466in}{0.388154in}}%
\pgfpathlineto{\pgfqpoint{7.291466in}{5.799207in}}%
\pgfusepath{stroke}%
\end{pgfscope}%
\begin{pgfscope}%
\pgfsetbuttcap%
\pgfsetroundjoin%
\definecolor{currentfill}{rgb}{0.000000,0.000000,0.000000}%
\pgfsetfillcolor{currentfill}%
\pgfsetlinewidth{0.803000pt}%
\definecolor{currentstroke}{rgb}{0.000000,0.000000,0.000000}%
\pgfsetstrokecolor{currentstroke}%
\pgfsetdash{}{0pt}%
\pgfsys@defobject{currentmarker}{\pgfqpoint{0.000000in}{-0.048611in}}{\pgfqpoint{0.000000in}{0.000000in}}{%
\pgfpathmoveto{\pgfqpoint{0.000000in}{0.000000in}}%
\pgfpathlineto{\pgfqpoint{0.000000in}{-0.048611in}}%
\pgfusepath{stroke,fill}%
}%
\begin{pgfscope}%
\pgfsys@transformshift{7.291466in}{0.388154in}%
\pgfsys@useobject{currentmarker}{}%
\end{pgfscope}%
\end{pgfscope}%
\begin{pgfscope}%
\definecolor{textcolor}{rgb}{0.000000,0.000000,0.000000}%
\pgfsetstrokecolor{textcolor}%
\pgfsetfillcolor{textcolor}%
\pgftext[x=7.291466in,y=0.290931in,,top]{\color{textcolor}{\rmfamily\fontsize{14.000000}{16.800000}\selectfont\catcode`\^=\active\def^{\ifmmode\sp\else\^{}\fi}\catcode`\%=\active\def%{\%}Wind}}%
\end{pgfscope}%
\begin{pgfscope}%
\pgfpathrectangle{\pgfqpoint{0.965831in}{0.388154in}}{\pgfqpoint{6.812222in}{5.411054in}}%
\pgfusepath{clip}%
\pgfsetrectcap%
\pgfsetroundjoin%
\pgfsetlinewidth{0.803000pt}%
\definecolor{currentstroke}{rgb}{0.690196,0.690196,0.690196}%
\pgfsetstrokecolor{currentstroke}%
\pgfsetdash{}{0pt}%
\pgfpathmoveto{\pgfqpoint{0.965831in}{0.388154in}}%
\pgfpathlineto{\pgfqpoint{7.778053in}{0.388154in}}%
\pgfusepath{stroke}%
\end{pgfscope}%
\begin{pgfscope}%
\pgfsetbuttcap%
\pgfsetroundjoin%
\definecolor{currentfill}{rgb}{0.000000,0.000000,0.000000}%
\pgfsetfillcolor{currentfill}%
\pgfsetlinewidth{0.803000pt}%
\definecolor{currentstroke}{rgb}{0.000000,0.000000,0.000000}%
\pgfsetstrokecolor{currentstroke}%
\pgfsetdash{}{0pt}%
\pgfsys@defobject{currentmarker}{\pgfqpoint{-0.048611in}{0.000000in}}{\pgfqpoint{-0.000000in}{0.000000in}}{%
\pgfpathmoveto{\pgfqpoint{-0.000000in}{0.000000in}}%
\pgfpathlineto{\pgfqpoint{-0.048611in}{0.000000in}}%
\pgfusepath{stroke,fill}%
}%
\begin{pgfscope}%
\pgfsys@transformshift{0.965831in}{0.388154in}%
\pgfsys@useobject{currentmarker}{}%
\end{pgfscope}%
\end{pgfscope}%
\begin{pgfscope}%
\definecolor{textcolor}{rgb}{0.000000,0.000000,0.000000}%
\pgfsetstrokecolor{textcolor}%
\pgfsetfillcolor{textcolor}%
\pgftext[x=0.744897in, y=0.314288in, left, base]{\color{textcolor}{\rmfamily\fontsize{14.000000}{16.800000}\selectfont\catcode`\^=\active\def^{\ifmmode\sp\else\^{}\fi}\catcode`\%=\active\def%{\%}0}}%
\end{pgfscope}%
\begin{pgfscope}%
\pgfpathrectangle{\pgfqpoint{0.965831in}{0.388154in}}{\pgfqpoint{6.812222in}{5.411054in}}%
\pgfusepath{clip}%
\pgfsetrectcap%
\pgfsetroundjoin%
\pgfsetlinewidth{0.803000pt}%
\definecolor{currentstroke}{rgb}{0.690196,0.690196,0.690196}%
\pgfsetstrokecolor{currentstroke}%
\pgfsetdash{}{0pt}%
\pgfpathmoveto{\pgfqpoint{0.965831in}{1.454371in}}%
\pgfpathlineto{\pgfqpoint{7.778053in}{1.454371in}}%
\pgfusepath{stroke}%
\end{pgfscope}%
\begin{pgfscope}%
\pgfsetbuttcap%
\pgfsetroundjoin%
\definecolor{currentfill}{rgb}{0.000000,0.000000,0.000000}%
\pgfsetfillcolor{currentfill}%
\pgfsetlinewidth{0.803000pt}%
\definecolor{currentstroke}{rgb}{0.000000,0.000000,0.000000}%
\pgfsetstrokecolor{currentstroke}%
\pgfsetdash{}{0pt}%
\pgfsys@defobject{currentmarker}{\pgfqpoint{-0.048611in}{0.000000in}}{\pgfqpoint{-0.000000in}{0.000000in}}{%
\pgfpathmoveto{\pgfqpoint{-0.000000in}{0.000000in}}%
\pgfpathlineto{\pgfqpoint{-0.048611in}{0.000000in}}%
\pgfusepath{stroke,fill}%
}%
\begin{pgfscope}%
\pgfsys@transformshift{0.965831in}{1.454371in}%
\pgfsys@useobject{currentmarker}{}%
\end{pgfscope}%
\end{pgfscope}%
\begin{pgfscope}%
\definecolor{textcolor}{rgb}{0.000000,0.000000,0.000000}%
\pgfsetstrokecolor{textcolor}%
\pgfsetfillcolor{textcolor}%
\pgftext[x=0.497474in, y=1.380505in, left, base]{\color{textcolor}{\rmfamily\fontsize{14.000000}{16.800000}\selectfont\catcode`\^=\active\def^{\ifmmode\sp\else\^{}\fi}\catcode`\%=\active\def%{\%}200}}%
\end{pgfscope}%
\begin{pgfscope}%
\pgfpathrectangle{\pgfqpoint{0.965831in}{0.388154in}}{\pgfqpoint{6.812222in}{5.411054in}}%
\pgfusepath{clip}%
\pgfsetrectcap%
\pgfsetroundjoin%
\pgfsetlinewidth{0.803000pt}%
\definecolor{currentstroke}{rgb}{0.690196,0.690196,0.690196}%
\pgfsetstrokecolor{currentstroke}%
\pgfsetdash{}{0pt}%
\pgfpathmoveto{\pgfqpoint{0.965831in}{2.520589in}}%
\pgfpathlineto{\pgfqpoint{7.778053in}{2.520589in}}%
\pgfusepath{stroke}%
\end{pgfscope}%
\begin{pgfscope}%
\pgfsetbuttcap%
\pgfsetroundjoin%
\definecolor{currentfill}{rgb}{0.000000,0.000000,0.000000}%
\pgfsetfillcolor{currentfill}%
\pgfsetlinewidth{0.803000pt}%
\definecolor{currentstroke}{rgb}{0.000000,0.000000,0.000000}%
\pgfsetstrokecolor{currentstroke}%
\pgfsetdash{}{0pt}%
\pgfsys@defobject{currentmarker}{\pgfqpoint{-0.048611in}{0.000000in}}{\pgfqpoint{-0.000000in}{0.000000in}}{%
\pgfpathmoveto{\pgfqpoint{-0.000000in}{0.000000in}}%
\pgfpathlineto{\pgfqpoint{-0.048611in}{0.000000in}}%
\pgfusepath{stroke,fill}%
}%
\begin{pgfscope}%
\pgfsys@transformshift{0.965831in}{2.520589in}%
\pgfsys@useobject{currentmarker}{}%
\end{pgfscope}%
\end{pgfscope}%
\begin{pgfscope}%
\definecolor{textcolor}{rgb}{0.000000,0.000000,0.000000}%
\pgfsetstrokecolor{textcolor}%
\pgfsetfillcolor{textcolor}%
\pgftext[x=0.497474in, y=2.446722in, left, base]{\color{textcolor}{\rmfamily\fontsize{14.000000}{16.800000}\selectfont\catcode`\^=\active\def^{\ifmmode\sp\else\^{}\fi}\catcode`\%=\active\def%{\%}400}}%
\end{pgfscope}%
\begin{pgfscope}%
\pgfpathrectangle{\pgfqpoint{0.965831in}{0.388154in}}{\pgfqpoint{6.812222in}{5.411054in}}%
\pgfusepath{clip}%
\pgfsetrectcap%
\pgfsetroundjoin%
\pgfsetlinewidth{0.803000pt}%
\definecolor{currentstroke}{rgb}{0.690196,0.690196,0.690196}%
\pgfsetstrokecolor{currentstroke}%
\pgfsetdash{}{0pt}%
\pgfpathmoveto{\pgfqpoint{0.965831in}{3.586806in}}%
\pgfpathlineto{\pgfqpoint{7.778053in}{3.586806in}}%
\pgfusepath{stroke}%
\end{pgfscope}%
\begin{pgfscope}%
\pgfsetbuttcap%
\pgfsetroundjoin%
\definecolor{currentfill}{rgb}{0.000000,0.000000,0.000000}%
\pgfsetfillcolor{currentfill}%
\pgfsetlinewidth{0.803000pt}%
\definecolor{currentstroke}{rgb}{0.000000,0.000000,0.000000}%
\pgfsetstrokecolor{currentstroke}%
\pgfsetdash{}{0pt}%
\pgfsys@defobject{currentmarker}{\pgfqpoint{-0.048611in}{0.000000in}}{\pgfqpoint{-0.000000in}{0.000000in}}{%
\pgfpathmoveto{\pgfqpoint{-0.000000in}{0.000000in}}%
\pgfpathlineto{\pgfqpoint{-0.048611in}{0.000000in}}%
\pgfusepath{stroke,fill}%
}%
\begin{pgfscope}%
\pgfsys@transformshift{0.965831in}{3.586806in}%
\pgfsys@useobject{currentmarker}{}%
\end{pgfscope}%
\end{pgfscope}%
\begin{pgfscope}%
\definecolor{textcolor}{rgb}{0.000000,0.000000,0.000000}%
\pgfsetstrokecolor{textcolor}%
\pgfsetfillcolor{textcolor}%
\pgftext[x=0.497474in, y=3.512940in, left, base]{\color{textcolor}{\rmfamily\fontsize{14.000000}{16.800000}\selectfont\catcode`\^=\active\def^{\ifmmode\sp\else\^{}\fi}\catcode`\%=\active\def%{\%}600}}%
\end{pgfscope}%
\begin{pgfscope}%
\pgfpathrectangle{\pgfqpoint{0.965831in}{0.388154in}}{\pgfqpoint{6.812222in}{5.411054in}}%
\pgfusepath{clip}%
\pgfsetrectcap%
\pgfsetroundjoin%
\pgfsetlinewidth{0.803000pt}%
\definecolor{currentstroke}{rgb}{0.690196,0.690196,0.690196}%
\pgfsetstrokecolor{currentstroke}%
\pgfsetdash{}{0pt}%
\pgfpathmoveto{\pgfqpoint{0.965831in}{4.653023in}}%
\pgfpathlineto{\pgfqpoint{7.778053in}{4.653023in}}%
\pgfusepath{stroke}%
\end{pgfscope}%
\begin{pgfscope}%
\pgfsetbuttcap%
\pgfsetroundjoin%
\definecolor{currentfill}{rgb}{0.000000,0.000000,0.000000}%
\pgfsetfillcolor{currentfill}%
\pgfsetlinewidth{0.803000pt}%
\definecolor{currentstroke}{rgb}{0.000000,0.000000,0.000000}%
\pgfsetstrokecolor{currentstroke}%
\pgfsetdash{}{0pt}%
\pgfsys@defobject{currentmarker}{\pgfqpoint{-0.048611in}{0.000000in}}{\pgfqpoint{-0.000000in}{0.000000in}}{%
\pgfpathmoveto{\pgfqpoint{-0.000000in}{0.000000in}}%
\pgfpathlineto{\pgfqpoint{-0.048611in}{0.000000in}}%
\pgfusepath{stroke,fill}%
}%
\begin{pgfscope}%
\pgfsys@transformshift{0.965831in}{4.653023in}%
\pgfsys@useobject{currentmarker}{}%
\end{pgfscope}%
\end{pgfscope}%
\begin{pgfscope}%
\definecolor{textcolor}{rgb}{0.000000,0.000000,0.000000}%
\pgfsetstrokecolor{textcolor}%
\pgfsetfillcolor{textcolor}%
\pgftext[x=0.497474in, y=4.579157in, left, base]{\color{textcolor}{\rmfamily\fontsize{14.000000}{16.800000}\selectfont\catcode`\^=\active\def^{\ifmmode\sp\else\^{}\fi}\catcode`\%=\active\def%{\%}800}}%
\end{pgfscope}%
\begin{pgfscope}%
\pgfpathrectangle{\pgfqpoint{0.965831in}{0.388154in}}{\pgfqpoint{6.812222in}{5.411054in}}%
\pgfusepath{clip}%
\pgfsetrectcap%
\pgfsetroundjoin%
\pgfsetlinewidth{0.803000pt}%
\definecolor{currentstroke}{rgb}{0.690196,0.690196,0.690196}%
\pgfsetstrokecolor{currentstroke}%
\pgfsetdash{}{0pt}%
\pgfpathmoveto{\pgfqpoint{0.965831in}{5.719241in}}%
\pgfpathlineto{\pgfqpoint{7.778053in}{5.719241in}}%
\pgfusepath{stroke}%
\end{pgfscope}%
\begin{pgfscope}%
\pgfsetbuttcap%
\pgfsetroundjoin%
\definecolor{currentfill}{rgb}{0.000000,0.000000,0.000000}%
\pgfsetfillcolor{currentfill}%
\pgfsetlinewidth{0.803000pt}%
\definecolor{currentstroke}{rgb}{0.000000,0.000000,0.000000}%
\pgfsetstrokecolor{currentstroke}%
\pgfsetdash{}{0pt}%
\pgfsys@defobject{currentmarker}{\pgfqpoint{-0.048611in}{0.000000in}}{\pgfqpoint{-0.000000in}{0.000000in}}{%
\pgfpathmoveto{\pgfqpoint{-0.000000in}{0.000000in}}%
\pgfpathlineto{\pgfqpoint{-0.048611in}{0.000000in}}%
\pgfusepath{stroke,fill}%
}%
\begin{pgfscope}%
\pgfsys@transformshift{0.965831in}{5.719241in}%
\pgfsys@useobject{currentmarker}{}%
\end{pgfscope}%
\end{pgfscope}%
\begin{pgfscope}%
\definecolor{textcolor}{rgb}{0.000000,0.000000,0.000000}%
\pgfsetstrokecolor{textcolor}%
\pgfsetfillcolor{textcolor}%
\pgftext[x=0.373763in, y=5.645375in, left, base]{\color{textcolor}{\rmfamily\fontsize{14.000000}{16.800000}\selectfont\catcode`\^=\active\def^{\ifmmode\sp\else\^{}\fi}\catcode`\%=\active\def%{\%}1000}}%
\end{pgfscope}%
\begin{pgfscope}%
\definecolor{textcolor}{rgb}{0.000000,0.000000,0.000000}%
\pgfsetstrokecolor{textcolor}%
\pgfsetfillcolor{textcolor}%
\pgftext[x=0.318207in,y=3.093680in,,bottom,rotate=90.000000]{\color{textcolor}{\rmfamily\fontsize{16.000000}{19.200000}\selectfont\catcode`\^=\active\def^{\ifmmode\sp\else\^{}\fi}\catcode`\%=\active\def%{\%}Lifecycle Emissions [gCO$_2$eq/kWh]}}%
\end{pgfscope}%
\begin{pgfscope}%
\pgfpathrectangle{\pgfqpoint{0.965831in}{0.388154in}}{\pgfqpoint{6.812222in}{5.411054in}}%
\pgfusepath{clip}%
\pgfsetbuttcap%
\pgfsetmiterjoin%
\definecolor{currentfill}{rgb}{0.706373,0.832843,0.874020}%
\pgfsetfillcolor{currentfill}%
\pgfsetlinewidth{0.000000pt}%
\definecolor{currentstroke}{rgb}{0.000000,0.000000,0.000000}%
\pgfsetstrokecolor{currentstroke}%
\pgfsetstrokeopacity{0.000000}%
\pgfsetdash{}{0pt}%
\pgfpathmoveto{\pgfqpoint{1.063148in}{0.388154in}}%
\pgfpathlineto{\pgfqpoint{1.841688in}{0.388154in}}%
\pgfpathlineto{\pgfqpoint{1.841688in}{5.363835in}}%
\pgfpathlineto{\pgfqpoint{1.063148in}{5.363835in}}%
\pgfpathlineto{\pgfqpoint{1.063148in}{0.388154in}}%
\pgfpathclose%
\pgfusepath{fill}%
\end{pgfscope}%
\begin{pgfscope}%
\pgfpathrectangle{\pgfqpoint{0.965831in}{0.388154in}}{\pgfqpoint{6.812222in}{5.411054in}}%
\pgfusepath{clip}%
\pgfsetbuttcap%
\pgfsetmiterjoin%
\definecolor{currentfill}{rgb}{0.706373,0.832843,0.874020}%
\pgfsetfillcolor{currentfill}%
\pgfsetlinewidth{0.000000pt}%
\definecolor{currentstroke}{rgb}{0.000000,0.000000,0.000000}%
\pgfsetstrokecolor{currentstroke}%
\pgfsetstrokeopacity{0.000000}%
\pgfsetdash{}{0pt}%
\pgfpathmoveto{\pgfqpoint{2.036323in}{0.388154in}}%
\pgfpathlineto{\pgfqpoint{2.814863in}{0.388154in}}%
\pgfpathlineto{\pgfqpoint{2.814863in}{2.680521in}}%
\pgfpathlineto{\pgfqpoint{2.036323in}{2.680521in}}%
\pgfpathlineto{\pgfqpoint{2.036323in}{0.388154in}}%
\pgfpathclose%
\pgfusepath{fill}%
\end{pgfscope}%
\begin{pgfscope}%
\pgfpathrectangle{\pgfqpoint{0.965831in}{0.388154in}}{\pgfqpoint{6.812222in}{5.411054in}}%
\pgfusepath{clip}%
\pgfsetbuttcap%
\pgfsetmiterjoin%
\definecolor{currentfill}{rgb}{0.706373,0.832843,0.874020}%
\pgfsetfillcolor{currentfill}%
\pgfsetlinewidth{0.000000pt}%
\definecolor{currentstroke}{rgb}{0.000000,0.000000,0.000000}%
\pgfsetstrokecolor{currentstroke}%
\pgfsetstrokeopacity{0.000000}%
\pgfsetdash{}{0pt}%
\pgfpathmoveto{\pgfqpoint{3.009498in}{0.388154in}}%
\pgfpathlineto{\pgfqpoint{3.788037in}{0.388154in}}%
\pgfpathlineto{\pgfqpoint{3.788037in}{0.817306in}}%
\pgfpathlineto{\pgfqpoint{3.009498in}{0.817306in}}%
\pgfpathlineto{\pgfqpoint{3.009498in}{0.388154in}}%
\pgfpathclose%
\pgfusepath{fill}%
\end{pgfscope}%
\begin{pgfscope}%
\pgfpathrectangle{\pgfqpoint{0.965831in}{0.388154in}}{\pgfqpoint{6.812222in}{5.411054in}}%
\pgfusepath{clip}%
\pgfsetbuttcap%
\pgfsetmiterjoin%
\definecolor{currentfill}{rgb}{0.706373,0.832843,0.874020}%
\pgfsetfillcolor{currentfill}%
\pgfsetlinewidth{0.000000pt}%
\definecolor{currentstroke}{rgb}{0.000000,0.000000,0.000000}%
\pgfsetstrokecolor{currentstroke}%
\pgfsetstrokeopacity{0.000000}%
\pgfsetdash{}{0pt}%
\pgfpathmoveto{\pgfqpoint{3.982672in}{0.388154in}}%
\pgfpathlineto{\pgfqpoint{4.761212in}{0.388154in}}%
\pgfpathlineto{\pgfqpoint{4.761212in}{0.415342in}}%
\pgfpathlineto{\pgfqpoint{3.982672in}{0.415342in}}%
\pgfpathlineto{\pgfqpoint{3.982672in}{0.388154in}}%
\pgfpathclose%
\pgfusepath{fill}%
\end{pgfscope}%
\begin{pgfscope}%
\pgfpathrectangle{\pgfqpoint{0.965831in}{0.388154in}}{\pgfqpoint{6.812222in}{5.411054in}}%
\pgfusepath{clip}%
\pgfsetbuttcap%
\pgfsetmiterjoin%
\definecolor{currentfill}{rgb}{0.706373,0.832843,0.874020}%
\pgfsetfillcolor{currentfill}%
\pgfsetlinewidth{0.000000pt}%
\definecolor{currentstroke}{rgb}{0.000000,0.000000,0.000000}%
\pgfsetstrokecolor{currentstroke}%
\pgfsetstrokeopacity{0.000000}%
\pgfsetdash{}{0pt}%
\pgfpathmoveto{\pgfqpoint{4.955847in}{0.388154in}}%
\pgfpathlineto{\pgfqpoint{5.734386in}{0.388154in}}%
\pgfpathlineto{\pgfqpoint{5.734386in}{0.558748in}}%
\pgfpathlineto{\pgfqpoint{4.955847in}{0.558748in}}%
\pgfpathlineto{\pgfqpoint{4.955847in}{0.388154in}}%
\pgfpathclose%
\pgfusepath{fill}%
\end{pgfscope}%
\begin{pgfscope}%
\pgfpathrectangle{\pgfqpoint{0.965831in}{0.388154in}}{\pgfqpoint{6.812222in}{5.411054in}}%
\pgfusepath{clip}%
\pgfsetbuttcap%
\pgfsetmiterjoin%
\definecolor{currentfill}{rgb}{0.706373,0.832843,0.874020}%
\pgfsetfillcolor{currentfill}%
\pgfsetlinewidth{0.000000pt}%
\definecolor{currentstroke}{rgb}{0.000000,0.000000,0.000000}%
\pgfsetstrokecolor{currentstroke}%
\pgfsetstrokeopacity{0.000000}%
\pgfsetdash{}{0pt}%
\pgfpathmoveto{\pgfqpoint{5.929021in}{0.388154in}}%
\pgfpathlineto{\pgfqpoint{6.707561in}{0.388154in}}%
\pgfpathlineto{\pgfqpoint{6.707561in}{0.500106in}}%
\pgfpathlineto{\pgfqpoint{5.929021in}{0.500106in}}%
\pgfpathlineto{\pgfqpoint{5.929021in}{0.388154in}}%
\pgfpathclose%
\pgfusepath{fill}%
\end{pgfscope}%
\begin{pgfscope}%
\pgfpathrectangle{\pgfqpoint{0.965831in}{0.388154in}}{\pgfqpoint{6.812222in}{5.411054in}}%
\pgfusepath{clip}%
\pgfsetbuttcap%
\pgfsetmiterjoin%
\definecolor{currentfill}{rgb}{0.706373,0.832843,0.874020}%
\pgfsetfillcolor{currentfill}%
\pgfsetlinewidth{0.000000pt}%
\definecolor{currentstroke}{rgb}{0.000000,0.000000,0.000000}%
\pgfsetstrokecolor{currentstroke}%
\pgfsetstrokeopacity{0.000000}%
\pgfsetdash{}{0pt}%
\pgfpathmoveto{\pgfqpoint{6.902196in}{0.388154in}}%
\pgfpathlineto{\pgfqpoint{7.680736in}{0.388154in}}%
\pgfpathlineto{\pgfqpoint{7.680736in}{0.457458in}}%
\pgfpathlineto{\pgfqpoint{6.902196in}{0.457458in}}%
\pgfpathlineto{\pgfqpoint{6.902196in}{0.388154in}}%
\pgfpathclose%
\pgfusepath{fill}%
\end{pgfscope}%
\begin{pgfscope}%
\pgfpathrectangle{\pgfqpoint{0.965831in}{0.388154in}}{\pgfqpoint{6.812222in}{5.411054in}}%
\pgfusepath{clip}%
\pgfsetrectcap%
\pgfsetroundjoin%
\pgfsetlinewidth{2.258437pt}%
\definecolor{currentstroke}{rgb}{0.260000,0.260000,0.260000}%
\pgfsetstrokecolor{currentstroke}%
\pgfsetdash{}{0pt}%
\pgfpathmoveto{\pgfqpoint{1.355101in}{5.186132in}}%
\pgfpathlineto{\pgfqpoint{1.549736in}{5.186132in}}%
\pgfpathmoveto{\pgfqpoint{1.452418in}{5.186132in}}%
\pgfpathlineto{\pgfqpoint{1.452418in}{5.541538in}}%
\pgfpathmoveto{\pgfqpoint{1.355101in}{5.541538in}}%
\pgfpathlineto{\pgfqpoint{1.549736in}{5.541538in}}%
\pgfusepath{stroke}%
\end{pgfscope}%
\begin{pgfscope}%
\pgfpathrectangle{\pgfqpoint{0.965831in}{0.388154in}}{\pgfqpoint{6.812222in}{5.411054in}}%
\pgfusepath{clip}%
\pgfsetrectcap%
\pgfsetroundjoin%
\pgfsetlinewidth{2.258437pt}%
\definecolor{currentstroke}{rgb}{0.260000,0.260000,0.260000}%
\pgfsetstrokecolor{currentstroke}%
\pgfsetdash{}{0pt}%
\pgfusepath{stroke}%
\end{pgfscope}%
\begin{pgfscope}%
\pgfpathrectangle{\pgfqpoint{0.965831in}{0.388154in}}{\pgfqpoint{6.812222in}{5.411054in}}%
\pgfusepath{clip}%
\pgfsetrectcap%
\pgfsetroundjoin%
\pgfsetlinewidth{2.258437pt}%
\definecolor{currentstroke}{rgb}{0.260000,0.260000,0.260000}%
\pgfsetstrokecolor{currentstroke}%
\pgfsetdash{}{0pt}%
\pgfpathmoveto{\pgfqpoint{3.301450in}{0.446796in}}%
\pgfpathlineto{\pgfqpoint{3.496085in}{0.446796in}}%
\pgfpathmoveto{\pgfqpoint{3.398767in}{0.446796in}}%
\pgfpathlineto{\pgfqpoint{3.398767in}{1.187817in}}%
\pgfpathmoveto{\pgfqpoint{3.301450in}{1.187817in}}%
\pgfpathlineto{\pgfqpoint{3.496085in}{1.187817in}}%
\pgfusepath{stroke}%
\end{pgfscope}%
\begin{pgfscope}%
\pgfpathrectangle{\pgfqpoint{0.965831in}{0.388154in}}{\pgfqpoint{6.812222in}{5.411054in}}%
\pgfusepath{clip}%
\pgfsetrectcap%
\pgfsetroundjoin%
\pgfsetlinewidth{2.258437pt}%
\definecolor{currentstroke}{rgb}{0.260000,0.260000,0.260000}%
\pgfsetstrokecolor{currentstroke}%
\pgfsetdash{}{0pt}%
\pgfusepath{stroke}%
\end{pgfscope}%
\begin{pgfscope}%
\pgfpathrectangle{\pgfqpoint{0.965831in}{0.388154in}}{\pgfqpoint{6.812222in}{5.411054in}}%
\pgfusepath{clip}%
\pgfsetrectcap%
\pgfsetroundjoin%
\pgfsetlinewidth{2.258437pt}%
\definecolor{currentstroke}{rgb}{0.260000,0.260000,0.260000}%
\pgfsetstrokecolor{currentstroke}%
\pgfsetdash{}{0pt}%
\pgfpathmoveto{\pgfqpoint{5.247799in}{0.505438in}}%
\pgfpathlineto{\pgfqpoint{5.442434in}{0.505438in}}%
\pgfpathmoveto{\pgfqpoint{5.345117in}{0.505438in}}%
\pgfpathlineto{\pgfqpoint{5.345117in}{0.612059in}}%
\pgfpathmoveto{\pgfqpoint{5.247799in}{0.612059in}}%
\pgfpathlineto{\pgfqpoint{5.442434in}{0.612059in}}%
\pgfusepath{stroke}%
\end{pgfscope}%
\begin{pgfscope}%
\pgfpathrectangle{\pgfqpoint{0.965831in}{0.388154in}}{\pgfqpoint{6.812222in}{5.411054in}}%
\pgfusepath{clip}%
\pgfsetrectcap%
\pgfsetroundjoin%
\pgfsetlinewidth{2.258437pt}%
\definecolor{currentstroke}{rgb}{0.260000,0.260000,0.260000}%
\pgfsetstrokecolor{currentstroke}%
\pgfsetdash{}{0pt}%
\pgfpathmoveto{\pgfqpoint{6.220974in}{0.477005in}}%
\pgfpathlineto{\pgfqpoint{6.415609in}{0.477005in}}%
\pgfpathmoveto{\pgfqpoint{6.318291in}{0.477005in}}%
\pgfpathlineto{\pgfqpoint{6.318291in}{0.522319in}}%
\pgfpathmoveto{\pgfqpoint{6.220974in}{0.522319in}}%
\pgfpathlineto{\pgfqpoint{6.415609in}{0.522319in}}%
\pgfusepath{stroke}%
\end{pgfscope}%
\begin{pgfscope}%
\pgfpathrectangle{\pgfqpoint{0.965831in}{0.388154in}}{\pgfqpoint{6.812222in}{5.411054in}}%
\pgfusepath{clip}%
\pgfsetrectcap%
\pgfsetroundjoin%
\pgfsetlinewidth{2.258437pt}%
\definecolor{currentstroke}{rgb}{0.260000,0.260000,0.260000}%
\pgfsetstrokecolor{currentstroke}%
\pgfsetdash{}{0pt}%
\pgfpathmoveto{\pgfqpoint{7.194148in}{0.455681in}}%
\pgfpathlineto{\pgfqpoint{7.388783in}{0.455681in}}%
\pgfpathmoveto{\pgfqpoint{7.291466in}{0.455681in}}%
\pgfpathlineto{\pgfqpoint{7.291466in}{0.459235in}}%
\pgfpathmoveto{\pgfqpoint{7.194148in}{0.459235in}}%
\pgfpathlineto{\pgfqpoint{7.388783in}{0.459235in}}%
\pgfusepath{stroke}%
\end{pgfscope}%
\begin{pgfscope}%
\pgfsetrectcap%
\pgfsetmiterjoin%
\pgfsetlinewidth{0.803000pt}%
\definecolor{currentstroke}{rgb}{0.000000,0.000000,0.000000}%
\pgfsetstrokecolor{currentstroke}%
\pgfsetdash{}{0pt}%
\pgfpathmoveto{\pgfqpoint{0.965831in}{0.388154in}}%
\pgfpathlineto{\pgfqpoint{0.965831in}{5.799207in}}%
\pgfusepath{stroke}%
\end{pgfscope}%
\begin{pgfscope}%
\pgfsetrectcap%
\pgfsetmiterjoin%
\pgfsetlinewidth{0.803000pt}%
\definecolor{currentstroke}{rgb}{0.000000,0.000000,0.000000}%
\pgfsetstrokecolor{currentstroke}%
\pgfsetdash{}{0pt}%
\pgfpathmoveto{\pgfqpoint{7.778053in}{0.388154in}}%
\pgfpathlineto{\pgfqpoint{7.778053in}{5.799207in}}%
\pgfusepath{stroke}%
\end{pgfscope}%
\begin{pgfscope}%
\pgfsetrectcap%
\pgfsetmiterjoin%
\pgfsetlinewidth{0.803000pt}%
\definecolor{currentstroke}{rgb}{0.000000,0.000000,0.000000}%
\pgfsetstrokecolor{currentstroke}%
\pgfsetdash{}{0pt}%
\pgfpathmoveto{\pgfqpoint{0.965831in}{0.388154in}}%
\pgfpathlineto{\pgfqpoint{7.778053in}{0.388154in}}%
\pgfusepath{stroke}%
\end{pgfscope}%
\begin{pgfscope}%
\pgfsetrectcap%
\pgfsetmiterjoin%
\pgfsetlinewidth{0.803000pt}%
\definecolor{currentstroke}{rgb}{0.000000,0.000000,0.000000}%
\pgfsetstrokecolor{currentstroke}%
\pgfsetdash{}{0pt}%
\pgfpathmoveto{\pgfqpoint{0.965831in}{5.799207in}}%
\pgfpathlineto{\pgfqpoint{7.778053in}{5.799207in}}%
\pgfusepath{stroke}%
\end{pgfscope}%
\end{pgfpicture}%
\makeatother%
\endgroup%
}
            \caption{Lifecycle carbon emissions by energy source
            \cite{united_nations_economic_commission_for_europe_carbon_2022}.}
            \label{figure:energy-emissions}
        \end{figure}
    \end{columns}

\end{frame}

\begin{frame}
    \frametitle{Addressing climate change?}

        \begin{block}{Just Energy Transition}
            \begin{enumerate}
                \item Requires new, low carbon, energy projects.
                \item Adhering to values of democracy necessitates local support
                for these projects.
            \end{enumerate}
        \end{block}
        \begin{block}{Public Opposition --- it's not NIMBY}
            Perceptions of fairness and inclusion, rather than NIMBY attitudes,
            condition local support
            \cite{konisky_proximity_2021,aitken_why_2010,stokes_prevalence_2023,firestone_public_2012-1}.
        \end{block}
        \begin{block}{}
            Public testimony can be dismissed for being non-technical
            \cite{johnson_dakota_2021}. Existing energy planning processes and
            new energy projects (even ``clean energy'' projects) reproduce
            existing sociopolitical structures that violate principles of
            justice.
        \end{block}
\end{frame}

\begin{frame}
    \frametitle{\glspl{esom}}
    \Glspl{esom} are a class of tools designed to 
    optimize this transition.
    \\
    % (Provide examples?)

    They will become more important with more \gls{vre}
    and more volatile weather.
    \\\\
    \textit{But they have at least two big flaws}.
    \begin{enumerate}[<+->]
        \item All current \glspl{esom} optimize a single objective --- cost.
        \item \glspl{esom} struggle to model the ``human dimension'' thereby
        limiting their ability to address justice \cite{pfenninger_energy_2014}.
    \end{enumerate}
\end{frame}

\begin{frame}
    \frametitle{Energy Modeling and Distributional Justice}
    \begin{columns}
        \column[t]{3cm}
        \begin{figure}
            \centering
            \resizebox{0.7\columnwidth}{!}{
                \begin{tikzpicture}[nodes={text depth=0.25ex,text height=1.25ex distance=1.7cm}]
                    \tikzstyle{every node}=[font=\small] \tikzstyle{vertex} =
                    [circle, draw=black, fill=trueilliniorange]
                    \tikzstyle{unfocus} = [circle, draw=gray, fill=illiniorange]
                    \tikzstyle{hidden} = [draw=none] \tikzstyle{edge} = [<->,
                    very thick]
                    
                    \node[vertex](v3) at (0,2)
                    {\textcolor{black}{\textbf{Distribution}}};
                    \node[unfocus](v2) at (0,0)
                    {\textcolor{gray}{\textbf{Procedural}}}; \node[unfocus](v1)
                    at (0,-2) {\textcolor{gray}{\textbf{Recognition}}};
        
            \end{tikzpicture}}
        \end{figure}
        \column[t]{7cm}
        \begin{block}{ESOMs and Distributional Justice}
            ESOM literature has begun considering distributional justice
            \cite{neumann_near-optimal_2021,sasse_distributional_2019,obrecht_integrating_2020}.
        \end{block}
        \begin{block}{}
            \begin{itemize}
                \item Quantifiable
                \item ``Objective'' --- research questions can be purely
                descriptive.
            \end{itemize}
        \end{block}
    \end{columns}
    
\end{frame}

\begin{frame}
    \frametitle{Energy Modeling and Procedural/Recognition Justice}
    \begin{columns}
        \column[t]{3cm}
        \begin{figure}
            \centering
            \resizebox{0.7\columnwidth}{!}{
                \begin{tikzpicture}[nodes={text depth=0.25ex,text height=1.25ex distance=1.7cm}]
                    \tikzstyle{every node}=[font=\small] \tikzstyle{vertex} =
                    [circle, draw=black, fill=trueilliniorange]
                    \tikzstyle{unfocus} = [circle, draw=gray, fill=illiniorange]
                    \tikzstyle{hidden} = [draw=none] \tikzstyle{edge} = [<->,
                    very thick]
                    
                    \node[unfocus](v3) at (0,2){\textcolor{gray}{\textbf{Distribution}}}; 
                    \node[vertex](v2) at (0,0) {\textcolor{black}{\textbf{Procedural}}};
                    \node[vertex](v1) at (0,-2){\textcolor{black}{\textbf{Recognition}}};
        
            \end{tikzpicture}}
        \end{figure}
        \column[t]{7cm}
        \begin{block}{Procedural Justice}
            ESOM literature now emphasizes code and data transparency
            \cite{decarolis_formalizing_2017} and highlights the importance of
            producing \textit{insight} rather than \textit{answers}
            \cite{decarolis_using_2011}.
        \end{block}
        \pause
        \begin{block}{}
            However, the literature does not consider the ways its methods
            inform policies. Do energy system models make this more transparent or less?
        \end{block}
        \pause
        \begin{block}{Recognition Justice}
            As a corollary of its lack of self-awareness, the ESOM literature
            does not address recognition justice at all --- modeling is
            independent from public influence.
        \end{block}
        \begin{block}{}
            How inclusive is energy modeling of community preferences?
        \end{block}
    \end{columns}
    
\end{frame}

% \begin{frame}
%     \frametitle{Three tenets of justice}
%     \begin{figure}
%         \centering
%         % \resizebox{0.7\columnwidth}{!}{
%             \begin{tikzpicture}
%                 \begin{scope}[blend group = soft light]
%                     % \fill[red!30!white]   ( 90:1.2) circle (2);
%                     \fill[illiniorange]   ( 90:1.2) circle (2);
%                     % \fill[green!30!white] (210:1.2) circle (2);
%                     \fill[illiniorange] (210:1.2) circle (2);
%                     % \fill[blue!30!white]  (330:1.2) circle (2);
%                     \fill[illiniorange]  (330:1.2) circle (2);
%                 \end{scope}
%                 \node at ( 90:2)    {Recognition}; 
%                 \node at ( 210:2) {Distribution}; 
%                 \node at ( 330:2)   {Procedural}; 
%                 \node[font=\Large] {\textcolor{black}{Justice}};
%               \end{tikzpicture}

%         % }
%         \caption{Three aspects of justice \cite{schlosberg_1_2007}.}
%     \end{figure}
% \end{frame}

\subsection{Introducing \gls{osier}}
\begin{frame}
    \frametitle{\gls{osier}}
    I developed \gls{osier} to fill these gaps by \cite{dotson_osier_2024}
    \begin{enumerate}[<+->]
        \item optimizing multi- and many-objective problems,
        \item allowing user-defined objectives,
        \item facilitating multi-criteria decision making,
        \item being accessible and usable.
    \end{enumerate}
\end{frame}

\begin{frame}
    \frametitle{How \gls{osier} Works}

    \begin{figure}
        \centering
        \includegraphics[width=\columnwidth]{../docs/figures/03_osier_chapter/osier_flow.png}
        \caption{Flow of data into and within \gls{osier}}
        \label{fig:osier-flow-1}
    \end{figure}
    % [Show the data flow diagram]

    % Osier works by leveraging genetic algorithms... 
\end{frame}

\begin{frame}
    \frametitle{Evolutionary Algorithms}

    \begin{columns}
        \column[t]{6cm}
        \begin{block}{Evolutionary Algorithms for Energy System Optimization}
            \begin{itemize}
            \item Inspired by natural selection
            \item Parallelizable
            \item Superior to pure linear programming methods for
                \begin{itemize}
                    \item independence from problem convexity
                    \item good sampling/spacing of points along solution set.
                \end{itemize}
            \end{itemize}
            
            Right: Evolutionary algorithm flow \cite{deb_evolutionary_2014}.
        \end{block}
            \column[t]{4cm}
            \centering
            \begin{figure}
            \includegraphics[width=0.75\linewidth]{images/ea-flow.png}  
            \end{figure}
        \end{columns}
\end{frame}


\begin{frame}
    \frametitle{Pareto Fronts}
    \begin{figure}
        \centering
        \resizebox{0.75\columnwidth}{!}{%% Creator: Matplotlib, PGF backend
%%
%% To include the figure in your LaTeX document, write
%%   \input{<filename>.pgf}
%%
%% Make sure the required packages are loaded in your preamble
%%   \usepackage{pgf}
%%
%% Also ensure that all the required font packages are loaded; for instance,
%% the lmodern package is sometimes necessary when using math font.
%%   \usepackage{lmodern}
%%
%% Figures using additional raster images can only be included by \input if
%% they are in the same directory as the main LaTeX file. For loading figures
%% from other directories you can use the `import` package
%%   \usepackage{import}
%%
%% and then include the figures with
%%   \import{<path to file>}{<filename>.pgf}
%%
%% Matplotlib used the following preamble
%%   \def\mathdefault#1{#1}
%%   \everymath=\expandafter{\the\everymath\displaystyle}
%%   
%%   \ifdefined\pdftexversion\else  % non-pdftex case.
%%     \usepackage{fontspec}
%%   \fi
%%   \makeatletter\@ifpackageloaded{underscore}{}{\usepackage[strings]{underscore}}\makeatother
%%
\begingroup%
\makeatletter%
\begin{pgfpicture}%
\pgfpathrectangle{\pgfpointorigin}{\pgfqpoint{7.147223in}{5.322237in}}%
\pgfusepath{use as bounding box, clip}%
\begin{pgfscope}%
\pgfsetbuttcap%
\pgfsetmiterjoin%
\definecolor{currentfill}{rgb}{1.000000,1.000000,1.000000}%
\pgfsetfillcolor{currentfill}%
\pgfsetlinewidth{0.000000pt}%
\definecolor{currentstroke}{rgb}{0.000000,0.000000,0.000000}%
\pgfsetstrokecolor{currentstroke}%
\pgfsetdash{}{0pt}%
\pgfpathmoveto{\pgfqpoint{0.000000in}{0.000000in}}%
\pgfpathlineto{\pgfqpoint{7.147223in}{0.000000in}}%
\pgfpathlineto{\pgfqpoint{7.147223in}{5.322237in}}%
\pgfpathlineto{\pgfqpoint{0.000000in}{5.322237in}}%
\pgfpathlineto{\pgfqpoint{0.000000in}{0.000000in}}%
\pgfpathclose%
\pgfusepath{fill}%
\end{pgfscope}%
\begin{pgfscope}%
\pgfsetbuttcap%
\pgfsetmiterjoin%
\definecolor{currentfill}{rgb}{1.000000,1.000000,1.000000}%
\pgfsetfillcolor{currentfill}%
\pgfsetlinewidth{0.000000pt}%
\definecolor{currentstroke}{rgb}{0.000000,0.000000,0.000000}%
\pgfsetstrokecolor{currentstroke}%
\pgfsetstrokeopacity{0.000000}%
\pgfsetdash{}{0pt}%
\pgfpathmoveto{\pgfqpoint{0.847223in}{0.554012in}}%
\pgfpathlineto{\pgfqpoint{7.047223in}{0.554012in}}%
\pgfpathlineto{\pgfqpoint{7.047223in}{5.174012in}}%
\pgfpathlineto{\pgfqpoint{0.847223in}{5.174012in}}%
\pgfpathlineto{\pgfqpoint{0.847223in}{0.554012in}}%
\pgfpathclose%
\pgfusepath{fill}%
\end{pgfscope}%
\begin{pgfscope}%
\pgfpathrectangle{\pgfqpoint{0.847223in}{0.554012in}}{\pgfqpoint{6.200000in}{4.620000in}}%
\pgfusepath{clip}%
\pgfsetbuttcap%
\pgfsetroundjoin%
\pgfsetlinewidth{1.003750pt}%
\definecolor{currentstroke}{rgb}{1.000000,0.000000,0.000000}%
\pgfsetstrokecolor{currentstroke}%
\pgfsetdash{}{0pt}%
\pgfpathmoveto{\pgfqpoint{0.847223in}{5.132345in}}%
\pgfpathcurveto{\pgfqpoint{0.858273in}{5.132345in}}{\pgfqpoint{0.868872in}{5.136735in}}{\pgfqpoint{0.876686in}{5.144549in}}%
\pgfpathcurveto{\pgfqpoint{0.884499in}{5.152363in}}{\pgfqpoint{0.888890in}{5.162962in}}{\pgfqpoint{0.888890in}{5.174012in}}%
\pgfpathcurveto{\pgfqpoint{0.888890in}{5.185062in}}{\pgfqpoint{0.884499in}{5.195661in}}{\pgfqpoint{0.876686in}{5.203475in}}%
\pgfpathcurveto{\pgfqpoint{0.868872in}{5.211288in}}{\pgfqpoint{0.858273in}{5.215678in}}{\pgfqpoint{0.847223in}{5.215678in}}%
\pgfpathcurveto{\pgfqpoint{0.836173in}{5.215678in}}{\pgfqpoint{0.825574in}{5.211288in}}{\pgfqpoint{0.817760in}{5.203475in}}%
\pgfpathcurveto{\pgfqpoint{0.809947in}{5.195661in}}{\pgfqpoint{0.805556in}{5.185062in}}{\pgfqpoint{0.805556in}{5.174012in}}%
\pgfpathcurveto{\pgfqpoint{0.805556in}{5.162962in}}{\pgfqpoint{0.809947in}{5.152363in}}{\pgfqpoint{0.817760in}{5.144549in}}%
\pgfpathcurveto{\pgfqpoint{0.825574in}{5.136735in}}{\pgfqpoint{0.836173in}{5.132345in}}{\pgfqpoint{0.847223in}{5.132345in}}%
\pgfpathlineto{\pgfqpoint{0.847223in}{5.132345in}}%
\pgfpathclose%
\pgfusepath{stroke}%
\end{pgfscope}%
\begin{pgfscope}%
\pgfpathrectangle{\pgfqpoint{0.847223in}{0.554012in}}{\pgfqpoint{6.200000in}{4.620000in}}%
\pgfusepath{clip}%
\pgfsetbuttcap%
\pgfsetroundjoin%
\pgfsetlinewidth{1.003750pt}%
\definecolor{currentstroke}{rgb}{1.000000,0.000000,0.000000}%
\pgfsetstrokecolor{currentstroke}%
\pgfsetdash{}{0pt}%
\pgfpathmoveto{\pgfqpoint{0.852556in}{5.081448in}}%
\pgfpathcurveto{\pgfqpoint{0.863606in}{5.081448in}}{\pgfqpoint{0.874205in}{5.085838in}}{\pgfqpoint{0.882019in}{5.093652in}}%
\pgfpathcurveto{\pgfqpoint{0.889833in}{5.101465in}}{\pgfqpoint{0.894223in}{5.112064in}}{\pgfqpoint{0.894223in}{5.123114in}}%
\pgfpathcurveto{\pgfqpoint{0.894223in}{5.134165in}}{\pgfqpoint{0.889833in}{5.144764in}}{\pgfqpoint{0.882019in}{5.152577in}}%
\pgfpathcurveto{\pgfqpoint{0.874205in}{5.160391in}}{\pgfqpoint{0.863606in}{5.164781in}}{\pgfqpoint{0.852556in}{5.164781in}}%
\pgfpathcurveto{\pgfqpoint{0.841506in}{5.164781in}}{\pgfqpoint{0.830907in}{5.160391in}}{\pgfqpoint{0.823093in}{5.152577in}}%
\pgfpathcurveto{\pgfqpoint{0.815280in}{5.144764in}}{\pgfqpoint{0.810890in}{5.134165in}}{\pgfqpoint{0.810890in}{5.123114in}}%
\pgfpathcurveto{\pgfqpoint{0.810890in}{5.112064in}}{\pgfqpoint{0.815280in}{5.101465in}}{\pgfqpoint{0.823093in}{5.093652in}}%
\pgfpathcurveto{\pgfqpoint{0.830907in}{5.085838in}}{\pgfqpoint{0.841506in}{5.081448in}}{\pgfqpoint{0.852556in}{5.081448in}}%
\pgfpathlineto{\pgfqpoint{0.852556in}{5.081448in}}%
\pgfpathclose%
\pgfusepath{stroke}%
\end{pgfscope}%
\begin{pgfscope}%
\pgfpathrectangle{\pgfqpoint{0.847223in}{0.554012in}}{\pgfqpoint{6.200000in}{4.620000in}}%
\pgfusepath{clip}%
\pgfsetbuttcap%
\pgfsetroundjoin%
\pgfsetlinewidth{1.003750pt}%
\definecolor{currentstroke}{rgb}{1.000000,0.000000,0.000000}%
\pgfsetstrokecolor{currentstroke}%
\pgfsetdash{}{0pt}%
\pgfpathmoveto{\pgfqpoint{0.857889in}{5.031567in}}%
\pgfpathcurveto{\pgfqpoint{0.868940in}{5.031567in}}{\pgfqpoint{0.879539in}{5.035957in}}{\pgfqpoint{0.887352in}{5.043771in}}%
\pgfpathcurveto{\pgfqpoint{0.895166in}{5.051585in}}{\pgfqpoint{0.899556in}{5.062184in}}{\pgfqpoint{0.899556in}{5.073234in}}%
\pgfpathcurveto{\pgfqpoint{0.899556in}{5.084284in}}{\pgfqpoint{0.895166in}{5.094883in}}{\pgfqpoint{0.887352in}{5.102696in}}%
\pgfpathcurveto{\pgfqpoint{0.879539in}{5.110510in}}{\pgfqpoint{0.868940in}{5.114900in}}{\pgfqpoint{0.857889in}{5.114900in}}%
\pgfpathcurveto{\pgfqpoint{0.846839in}{5.114900in}}{\pgfqpoint{0.836240in}{5.110510in}}{\pgfqpoint{0.828427in}{5.102696in}}%
\pgfpathcurveto{\pgfqpoint{0.820613in}{5.094883in}}{\pgfqpoint{0.816223in}{5.084284in}}{\pgfqpoint{0.816223in}{5.073234in}}%
\pgfpathcurveto{\pgfqpoint{0.816223in}{5.062184in}}{\pgfqpoint{0.820613in}{5.051585in}}{\pgfqpoint{0.828427in}{5.043771in}}%
\pgfpathcurveto{\pgfqpoint{0.836240in}{5.035957in}}{\pgfqpoint{0.846839in}{5.031567in}}{\pgfqpoint{0.857889in}{5.031567in}}%
\pgfpathlineto{\pgfqpoint{0.857889in}{5.031567in}}%
\pgfpathclose%
\pgfusepath{stroke}%
\end{pgfscope}%
\begin{pgfscope}%
\pgfpathrectangle{\pgfqpoint{0.847223in}{0.554012in}}{\pgfqpoint{6.200000in}{4.620000in}}%
\pgfusepath{clip}%
\pgfsetbuttcap%
\pgfsetroundjoin%
\pgfsetlinewidth{1.003750pt}%
\definecolor{currentstroke}{rgb}{1.000000,0.000000,0.000000}%
\pgfsetstrokecolor{currentstroke}%
\pgfsetdash{}{0pt}%
\pgfpathmoveto{\pgfqpoint{0.863223in}{4.982673in}}%
\pgfpathcurveto{\pgfqpoint{0.874273in}{4.982673in}}{\pgfqpoint{0.884872in}{4.987063in}}{\pgfqpoint{0.892685in}{4.994877in}}%
\pgfpathcurveto{\pgfqpoint{0.900499in}{5.002690in}}{\pgfqpoint{0.904889in}{5.013289in}}{\pgfqpoint{0.904889in}{5.024339in}}%
\pgfpathcurveto{\pgfqpoint{0.904889in}{5.035390in}}{\pgfqpoint{0.900499in}{5.045989in}}{\pgfqpoint{0.892685in}{5.053802in}}%
\pgfpathcurveto{\pgfqpoint{0.884872in}{5.061616in}}{\pgfqpoint{0.874273in}{5.066006in}}{\pgfqpoint{0.863223in}{5.066006in}}%
\pgfpathcurveto{\pgfqpoint{0.852173in}{5.066006in}}{\pgfqpoint{0.841574in}{5.061616in}}{\pgfqpoint{0.833760in}{5.053802in}}%
\pgfpathcurveto{\pgfqpoint{0.825946in}{5.045989in}}{\pgfqpoint{0.821556in}{5.035390in}}{\pgfqpoint{0.821556in}{5.024339in}}%
\pgfpathcurveto{\pgfqpoint{0.821556in}{5.013289in}}{\pgfqpoint{0.825946in}{5.002690in}}{\pgfqpoint{0.833760in}{4.994877in}}%
\pgfpathcurveto{\pgfqpoint{0.841574in}{4.987063in}}{\pgfqpoint{0.852173in}{4.982673in}}{\pgfqpoint{0.863223in}{4.982673in}}%
\pgfpathlineto{\pgfqpoint{0.863223in}{4.982673in}}%
\pgfpathclose%
\pgfusepath{stroke}%
\end{pgfscope}%
\begin{pgfscope}%
\pgfpathrectangle{\pgfqpoint{0.847223in}{0.554012in}}{\pgfqpoint{6.200000in}{4.620000in}}%
\pgfusepath{clip}%
\pgfsetbuttcap%
\pgfsetroundjoin%
\pgfsetlinewidth{1.003750pt}%
\definecolor{currentstroke}{rgb}{1.000000,0.000000,0.000000}%
\pgfsetstrokecolor{currentstroke}%
\pgfsetdash{}{0pt}%
\pgfpathmoveto{\pgfqpoint{0.868556in}{4.934736in}}%
\pgfpathcurveto{\pgfqpoint{0.879606in}{4.934736in}}{\pgfqpoint{0.890205in}{4.939126in}}{\pgfqpoint{0.898019in}{4.946940in}}%
\pgfpathcurveto{\pgfqpoint{0.905832in}{4.954753in}}{\pgfqpoint{0.910223in}{4.965353in}}{\pgfqpoint{0.910223in}{4.976403in}}%
\pgfpathcurveto{\pgfqpoint{0.910223in}{4.987453in}}{\pgfqpoint{0.905832in}{4.998052in}}{\pgfqpoint{0.898019in}{5.005865in}}%
\pgfpathcurveto{\pgfqpoint{0.890205in}{5.013679in}}{\pgfqpoint{0.879606in}{5.018069in}}{\pgfqpoint{0.868556in}{5.018069in}}%
\pgfpathcurveto{\pgfqpoint{0.857506in}{5.018069in}}{\pgfqpoint{0.846907in}{5.013679in}}{\pgfqpoint{0.839093in}{5.005865in}}%
\pgfpathcurveto{\pgfqpoint{0.831280in}{4.998052in}}{\pgfqpoint{0.826889in}{4.987453in}}{\pgfqpoint{0.826889in}{4.976403in}}%
\pgfpathcurveto{\pgfqpoint{0.826889in}{4.965353in}}{\pgfqpoint{0.831280in}{4.954753in}}{\pgfqpoint{0.839093in}{4.946940in}}%
\pgfpathcurveto{\pgfqpoint{0.846907in}{4.939126in}}{\pgfqpoint{0.857506in}{4.934736in}}{\pgfqpoint{0.868556in}{4.934736in}}%
\pgfpathlineto{\pgfqpoint{0.868556in}{4.934736in}}%
\pgfpathclose%
\pgfusepath{stroke}%
\end{pgfscope}%
\begin{pgfscope}%
\pgfpathrectangle{\pgfqpoint{0.847223in}{0.554012in}}{\pgfqpoint{6.200000in}{4.620000in}}%
\pgfusepath{clip}%
\pgfsetbuttcap%
\pgfsetroundjoin%
\pgfsetlinewidth{1.003750pt}%
\definecolor{currentstroke}{rgb}{1.000000,0.000000,0.000000}%
\pgfsetstrokecolor{currentstroke}%
\pgfsetdash{}{0pt}%
\pgfpathmoveto{\pgfqpoint{0.873889in}{4.887729in}}%
\pgfpathcurveto{\pgfqpoint{0.884939in}{4.887729in}}{\pgfqpoint{0.895538in}{4.892119in}}{\pgfqpoint{0.903352in}{4.899933in}}%
\pgfpathcurveto{\pgfqpoint{0.911166in}{4.907746in}}{\pgfqpoint{0.915556in}{4.918345in}}{\pgfqpoint{0.915556in}{4.929396in}}%
\pgfpathcurveto{\pgfqpoint{0.915556in}{4.940446in}}{\pgfqpoint{0.911166in}{4.951045in}}{\pgfqpoint{0.903352in}{4.958858in}}%
\pgfpathcurveto{\pgfqpoint{0.895538in}{4.966672in}}{\pgfqpoint{0.884939in}{4.971062in}}{\pgfqpoint{0.873889in}{4.971062in}}%
\pgfpathcurveto{\pgfqpoint{0.862839in}{4.971062in}}{\pgfqpoint{0.852240in}{4.966672in}}{\pgfqpoint{0.844426in}{4.958858in}}%
\pgfpathcurveto{\pgfqpoint{0.836613in}{4.951045in}}{\pgfqpoint{0.832222in}{4.940446in}}{\pgfqpoint{0.832222in}{4.929396in}}%
\pgfpathcurveto{\pgfqpoint{0.832222in}{4.918345in}}{\pgfqpoint{0.836613in}{4.907746in}}{\pgfqpoint{0.844426in}{4.899933in}}%
\pgfpathcurveto{\pgfqpoint{0.852240in}{4.892119in}}{\pgfqpoint{0.862839in}{4.887729in}}{\pgfqpoint{0.873889in}{4.887729in}}%
\pgfpathlineto{\pgfqpoint{0.873889in}{4.887729in}}%
\pgfpathclose%
\pgfusepath{stroke}%
\end{pgfscope}%
\begin{pgfscope}%
\pgfpathrectangle{\pgfqpoint{0.847223in}{0.554012in}}{\pgfqpoint{6.200000in}{4.620000in}}%
\pgfusepath{clip}%
\pgfsetbuttcap%
\pgfsetroundjoin%
\pgfsetlinewidth{1.003750pt}%
\definecolor{currentstroke}{rgb}{1.000000,0.000000,0.000000}%
\pgfsetstrokecolor{currentstroke}%
\pgfsetdash{}{0pt}%
\pgfpathmoveto{\pgfqpoint{0.879222in}{4.841625in}}%
\pgfpathcurveto{\pgfqpoint{0.890272in}{4.841625in}}{\pgfqpoint{0.900872in}{4.846015in}}{\pgfqpoint{0.908685in}{4.853828in}}%
\pgfpathcurveto{\pgfqpoint{0.916499in}{4.861642in}}{\pgfqpoint{0.920889in}{4.872241in}}{\pgfqpoint{0.920889in}{4.883291in}}%
\pgfpathcurveto{\pgfqpoint{0.920889in}{4.894341in}}{\pgfqpoint{0.916499in}{4.904940in}}{\pgfqpoint{0.908685in}{4.912754in}}%
\pgfpathcurveto{\pgfqpoint{0.900872in}{4.920568in}}{\pgfqpoint{0.890272in}{4.924958in}}{\pgfqpoint{0.879222in}{4.924958in}}%
\pgfpathcurveto{\pgfqpoint{0.868172in}{4.924958in}}{\pgfqpoint{0.857573in}{4.920568in}}{\pgfqpoint{0.849760in}{4.912754in}}%
\pgfpathcurveto{\pgfqpoint{0.841946in}{4.904940in}}{\pgfqpoint{0.837556in}{4.894341in}}{\pgfqpoint{0.837556in}{4.883291in}}%
\pgfpathcurveto{\pgfqpoint{0.837556in}{4.872241in}}{\pgfqpoint{0.841946in}{4.861642in}}{\pgfqpoint{0.849760in}{4.853828in}}%
\pgfpathcurveto{\pgfqpoint{0.857573in}{4.846015in}}{\pgfqpoint{0.868172in}{4.841625in}}{\pgfqpoint{0.879222in}{4.841625in}}%
\pgfpathlineto{\pgfqpoint{0.879222in}{4.841625in}}%
\pgfpathclose%
\pgfusepath{stroke}%
\end{pgfscope}%
\begin{pgfscope}%
\pgfpathrectangle{\pgfqpoint{0.847223in}{0.554012in}}{\pgfqpoint{6.200000in}{4.620000in}}%
\pgfusepath{clip}%
\pgfsetbuttcap%
\pgfsetroundjoin%
\pgfsetlinewidth{1.003750pt}%
\definecolor{currentstroke}{rgb}{1.000000,0.000000,0.000000}%
\pgfsetstrokecolor{currentstroke}%
\pgfsetdash{}{0pt}%
\pgfpathmoveto{\pgfqpoint{0.884556in}{4.796397in}}%
\pgfpathcurveto{\pgfqpoint{0.895606in}{4.796397in}}{\pgfqpoint{0.906205in}{4.800788in}}{\pgfqpoint{0.914018in}{4.808601in}}%
\pgfpathcurveto{\pgfqpoint{0.921832in}{4.816415in}}{\pgfqpoint{0.926222in}{4.827014in}}{\pgfqpoint{0.926222in}{4.838064in}}%
\pgfpathcurveto{\pgfqpoint{0.926222in}{4.849114in}}{\pgfqpoint{0.921832in}{4.859713in}}{\pgfqpoint{0.914018in}{4.867527in}}%
\pgfpathcurveto{\pgfqpoint{0.906205in}{4.875340in}}{\pgfqpoint{0.895606in}{4.879731in}}{\pgfqpoint{0.884556in}{4.879731in}}%
\pgfpathcurveto{\pgfqpoint{0.873505in}{4.879731in}}{\pgfqpoint{0.862906in}{4.875340in}}{\pgfqpoint{0.855093in}{4.867527in}}%
\pgfpathcurveto{\pgfqpoint{0.847279in}{4.859713in}}{\pgfqpoint{0.842889in}{4.849114in}}{\pgfqpoint{0.842889in}{4.838064in}}%
\pgfpathcurveto{\pgfqpoint{0.842889in}{4.827014in}}{\pgfqpoint{0.847279in}{4.816415in}}{\pgfqpoint{0.855093in}{4.808601in}}%
\pgfpathcurveto{\pgfqpoint{0.862906in}{4.800788in}}{\pgfqpoint{0.873505in}{4.796397in}}{\pgfqpoint{0.884556in}{4.796397in}}%
\pgfpathlineto{\pgfqpoint{0.884556in}{4.796397in}}%
\pgfpathclose%
\pgfusepath{stroke}%
\end{pgfscope}%
\begin{pgfscope}%
\pgfpathrectangle{\pgfqpoint{0.847223in}{0.554012in}}{\pgfqpoint{6.200000in}{4.620000in}}%
\pgfusepath{clip}%
\pgfsetbuttcap%
\pgfsetroundjoin%
\pgfsetlinewidth{1.003750pt}%
\definecolor{currentstroke}{rgb}{1.000000,0.000000,0.000000}%
\pgfsetstrokecolor{currentstroke}%
\pgfsetdash{}{0pt}%
\pgfpathmoveto{\pgfqpoint{0.889889in}{4.752022in}}%
\pgfpathcurveto{\pgfqpoint{0.900939in}{4.752022in}}{\pgfqpoint{0.911538in}{4.756413in}}{\pgfqpoint{0.919352in}{4.764226in}}%
\pgfpathcurveto{\pgfqpoint{0.927165in}{4.772040in}}{\pgfqpoint{0.931555in}{4.782639in}}{\pgfqpoint{0.931555in}{4.793689in}}%
\pgfpathcurveto{\pgfqpoint{0.931555in}{4.804739in}}{\pgfqpoint{0.927165in}{4.815338in}}{\pgfqpoint{0.919352in}{4.823152in}}%
\pgfpathcurveto{\pgfqpoint{0.911538in}{4.830965in}}{\pgfqpoint{0.900939in}{4.835356in}}{\pgfqpoint{0.889889in}{4.835356in}}%
\pgfpathcurveto{\pgfqpoint{0.878839in}{4.835356in}}{\pgfqpoint{0.868240in}{4.830965in}}{\pgfqpoint{0.860426in}{4.823152in}}%
\pgfpathcurveto{\pgfqpoint{0.852612in}{4.815338in}}{\pgfqpoint{0.848222in}{4.804739in}}{\pgfqpoint{0.848222in}{4.793689in}}%
\pgfpathcurveto{\pgfqpoint{0.848222in}{4.782639in}}{\pgfqpoint{0.852612in}{4.772040in}}{\pgfqpoint{0.860426in}{4.764226in}}%
\pgfpathcurveto{\pgfqpoint{0.868240in}{4.756413in}}{\pgfqpoint{0.878839in}{4.752022in}}{\pgfqpoint{0.889889in}{4.752022in}}%
\pgfpathlineto{\pgfqpoint{0.889889in}{4.752022in}}%
\pgfpathclose%
\pgfusepath{stroke}%
\end{pgfscope}%
\begin{pgfscope}%
\pgfpathrectangle{\pgfqpoint{0.847223in}{0.554012in}}{\pgfqpoint{6.200000in}{4.620000in}}%
\pgfusepath{clip}%
\pgfsetbuttcap%
\pgfsetroundjoin%
\pgfsetlinewidth{1.003750pt}%
\definecolor{currentstroke}{rgb}{1.000000,0.000000,0.000000}%
\pgfsetstrokecolor{currentstroke}%
\pgfsetdash{}{0pt}%
\pgfpathmoveto{\pgfqpoint{0.895222in}{4.708476in}}%
\pgfpathcurveto{\pgfqpoint{0.906272in}{4.708476in}}{\pgfqpoint{0.916871in}{4.712866in}}{\pgfqpoint{0.924685in}{4.720680in}}%
\pgfpathcurveto{\pgfqpoint{0.932498in}{4.728493in}}{\pgfqpoint{0.936889in}{4.739092in}}{\pgfqpoint{0.936889in}{4.750143in}}%
\pgfpathcurveto{\pgfqpoint{0.936889in}{4.761193in}}{\pgfqpoint{0.932498in}{4.771792in}}{\pgfqpoint{0.924685in}{4.779605in}}%
\pgfpathcurveto{\pgfqpoint{0.916871in}{4.787419in}}{\pgfqpoint{0.906272in}{4.791809in}}{\pgfqpoint{0.895222in}{4.791809in}}%
\pgfpathcurveto{\pgfqpoint{0.884172in}{4.791809in}}{\pgfqpoint{0.873573in}{4.787419in}}{\pgfqpoint{0.865759in}{4.779605in}}%
\pgfpathcurveto{\pgfqpoint{0.857946in}{4.771792in}}{\pgfqpoint{0.853555in}{4.761193in}}{\pgfqpoint{0.853555in}{4.750143in}}%
\pgfpathcurveto{\pgfqpoint{0.853555in}{4.739092in}}{\pgfqpoint{0.857946in}{4.728493in}}{\pgfqpoint{0.865759in}{4.720680in}}%
\pgfpathcurveto{\pgfqpoint{0.873573in}{4.712866in}}{\pgfqpoint{0.884172in}{4.708476in}}{\pgfqpoint{0.895222in}{4.708476in}}%
\pgfpathlineto{\pgfqpoint{0.895222in}{4.708476in}}%
\pgfpathclose%
\pgfusepath{stroke}%
\end{pgfscope}%
\begin{pgfscope}%
\pgfpathrectangle{\pgfqpoint{0.847223in}{0.554012in}}{\pgfqpoint{6.200000in}{4.620000in}}%
\pgfusepath{clip}%
\pgfsetbuttcap%
\pgfsetroundjoin%
\pgfsetlinewidth{1.003750pt}%
\definecolor{currentstroke}{rgb}{1.000000,0.000000,0.000000}%
\pgfsetstrokecolor{currentstroke}%
\pgfsetdash{}{0pt}%
\pgfpathmoveto{\pgfqpoint{0.900555in}{4.665735in}}%
\pgfpathcurveto{\pgfqpoint{0.911605in}{4.665735in}}{\pgfqpoint{0.922204in}{4.670125in}}{\pgfqpoint{0.930018in}{4.677939in}}%
\pgfpathcurveto{\pgfqpoint{0.937832in}{4.685752in}}{\pgfqpoint{0.942222in}{4.696351in}}{\pgfqpoint{0.942222in}{4.707402in}}%
\pgfpathcurveto{\pgfqpoint{0.942222in}{4.718452in}}{\pgfqpoint{0.937832in}{4.729051in}}{\pgfqpoint{0.930018in}{4.736864in}}%
\pgfpathcurveto{\pgfqpoint{0.922204in}{4.744678in}}{\pgfqpoint{0.911605in}{4.749068in}}{\pgfqpoint{0.900555in}{4.749068in}}%
\pgfpathcurveto{\pgfqpoint{0.889505in}{4.749068in}}{\pgfqpoint{0.878906in}{4.744678in}}{\pgfqpoint{0.871092in}{4.736864in}}%
\pgfpathcurveto{\pgfqpoint{0.863279in}{4.729051in}}{\pgfqpoint{0.858889in}{4.718452in}}{\pgfqpoint{0.858889in}{4.707402in}}%
\pgfpathcurveto{\pgfqpoint{0.858889in}{4.696351in}}{\pgfqpoint{0.863279in}{4.685752in}}{\pgfqpoint{0.871092in}{4.677939in}}%
\pgfpathcurveto{\pgfqpoint{0.878906in}{4.670125in}}{\pgfqpoint{0.889505in}{4.665735in}}{\pgfqpoint{0.900555in}{4.665735in}}%
\pgfpathlineto{\pgfqpoint{0.900555in}{4.665735in}}%
\pgfpathclose%
\pgfusepath{stroke}%
\end{pgfscope}%
\begin{pgfscope}%
\pgfpathrectangle{\pgfqpoint{0.847223in}{0.554012in}}{\pgfqpoint{6.200000in}{4.620000in}}%
\pgfusepath{clip}%
\pgfsetbuttcap%
\pgfsetroundjoin%
\pgfsetlinewidth{1.003750pt}%
\definecolor{currentstroke}{rgb}{1.000000,0.000000,0.000000}%
\pgfsetstrokecolor{currentstroke}%
\pgfsetdash{}{0pt}%
\pgfpathmoveto{\pgfqpoint{0.905888in}{4.623777in}}%
\pgfpathcurveto{\pgfqpoint{0.916939in}{4.623777in}}{\pgfqpoint{0.927538in}{4.628167in}}{\pgfqpoint{0.935351in}{4.635981in}}%
\pgfpathcurveto{\pgfqpoint{0.943165in}{4.643795in}}{\pgfqpoint{0.947555in}{4.654394in}}{\pgfqpoint{0.947555in}{4.665444in}}%
\pgfpathcurveto{\pgfqpoint{0.947555in}{4.676494in}}{\pgfqpoint{0.943165in}{4.687093in}}{\pgfqpoint{0.935351in}{4.694907in}}%
\pgfpathcurveto{\pgfqpoint{0.927538in}{4.702720in}}{\pgfqpoint{0.916939in}{4.707110in}}{\pgfqpoint{0.905888in}{4.707110in}}%
\pgfpathcurveto{\pgfqpoint{0.894838in}{4.707110in}}{\pgfqpoint{0.884239in}{4.702720in}}{\pgfqpoint{0.876426in}{4.694907in}}%
\pgfpathcurveto{\pgfqpoint{0.868612in}{4.687093in}}{\pgfqpoint{0.864222in}{4.676494in}}{\pgfqpoint{0.864222in}{4.665444in}}%
\pgfpathcurveto{\pgfqpoint{0.864222in}{4.654394in}}{\pgfqpoint{0.868612in}{4.643795in}}{\pgfqpoint{0.876426in}{4.635981in}}%
\pgfpathcurveto{\pgfqpoint{0.884239in}{4.628167in}}{\pgfqpoint{0.894838in}{4.623777in}}{\pgfqpoint{0.905888in}{4.623777in}}%
\pgfpathlineto{\pgfqpoint{0.905888in}{4.623777in}}%
\pgfpathclose%
\pgfusepath{stroke}%
\end{pgfscope}%
\begin{pgfscope}%
\pgfpathrectangle{\pgfqpoint{0.847223in}{0.554012in}}{\pgfqpoint{6.200000in}{4.620000in}}%
\pgfusepath{clip}%
\pgfsetbuttcap%
\pgfsetroundjoin%
\pgfsetlinewidth{1.003750pt}%
\definecolor{currentstroke}{rgb}{1.000000,0.000000,0.000000}%
\pgfsetstrokecolor{currentstroke}%
\pgfsetdash{}{0pt}%
\pgfpathmoveto{\pgfqpoint{0.911222in}{4.582581in}}%
\pgfpathcurveto{\pgfqpoint{0.922272in}{4.582581in}}{\pgfqpoint{0.932871in}{4.586972in}}{\pgfqpoint{0.940684in}{4.594785in}}%
\pgfpathcurveto{\pgfqpoint{0.948498in}{4.602599in}}{\pgfqpoint{0.952888in}{4.613198in}}{\pgfqpoint{0.952888in}{4.624248in}}%
\pgfpathcurveto{\pgfqpoint{0.952888in}{4.635298in}}{\pgfqpoint{0.948498in}{4.645897in}}{\pgfqpoint{0.940684in}{4.653711in}}%
\pgfpathcurveto{\pgfqpoint{0.932871in}{4.661524in}}{\pgfqpoint{0.922272in}{4.665915in}}{\pgfqpoint{0.911222in}{4.665915in}}%
\pgfpathcurveto{\pgfqpoint{0.900172in}{4.665915in}}{\pgfqpoint{0.889572in}{4.661524in}}{\pgfqpoint{0.881759in}{4.653711in}}%
\pgfpathcurveto{\pgfqpoint{0.873945in}{4.645897in}}{\pgfqpoint{0.869555in}{4.635298in}}{\pgfqpoint{0.869555in}{4.624248in}}%
\pgfpathcurveto{\pgfqpoint{0.869555in}{4.613198in}}{\pgfqpoint{0.873945in}{4.602599in}}{\pgfqpoint{0.881759in}{4.594785in}}%
\pgfpathcurveto{\pgfqpoint{0.889572in}{4.586972in}}{\pgfqpoint{0.900172in}{4.582581in}}{\pgfqpoint{0.911222in}{4.582581in}}%
\pgfpathlineto{\pgfqpoint{0.911222in}{4.582581in}}%
\pgfpathclose%
\pgfusepath{stroke}%
\end{pgfscope}%
\begin{pgfscope}%
\pgfpathrectangle{\pgfqpoint{0.847223in}{0.554012in}}{\pgfqpoint{6.200000in}{4.620000in}}%
\pgfusepath{clip}%
\pgfsetbuttcap%
\pgfsetroundjoin%
\pgfsetlinewidth{1.003750pt}%
\definecolor{currentstroke}{rgb}{1.000000,0.000000,0.000000}%
\pgfsetstrokecolor{currentstroke}%
\pgfsetdash{}{0pt}%
\pgfpathmoveto{\pgfqpoint{0.916555in}{4.542127in}}%
\pgfpathcurveto{\pgfqpoint{0.927605in}{4.542127in}}{\pgfqpoint{0.938204in}{4.546517in}}{\pgfqpoint{0.946018in}{4.554331in}}%
\pgfpathcurveto{\pgfqpoint{0.953831in}{4.562145in}}{\pgfqpoint{0.958222in}{4.572744in}}{\pgfqpoint{0.958222in}{4.583794in}}%
\pgfpathcurveto{\pgfqpoint{0.958222in}{4.594844in}}{\pgfqpoint{0.953831in}{4.605443in}}{\pgfqpoint{0.946018in}{4.613256in}}%
\pgfpathcurveto{\pgfqpoint{0.938204in}{4.621070in}}{\pgfqpoint{0.927605in}{4.625460in}}{\pgfqpoint{0.916555in}{4.625460in}}%
\pgfpathcurveto{\pgfqpoint{0.905505in}{4.625460in}}{\pgfqpoint{0.894906in}{4.621070in}}{\pgfqpoint{0.887092in}{4.613256in}}%
\pgfpathcurveto{\pgfqpoint{0.879278in}{4.605443in}}{\pgfqpoint{0.874888in}{4.594844in}}{\pgfqpoint{0.874888in}{4.583794in}}%
\pgfpathcurveto{\pgfqpoint{0.874888in}{4.572744in}}{\pgfqpoint{0.879278in}{4.562145in}}{\pgfqpoint{0.887092in}{4.554331in}}%
\pgfpathcurveto{\pgfqpoint{0.894906in}{4.546517in}}{\pgfqpoint{0.905505in}{4.542127in}}{\pgfqpoint{0.916555in}{4.542127in}}%
\pgfpathlineto{\pgfqpoint{0.916555in}{4.542127in}}%
\pgfpathclose%
\pgfusepath{stroke}%
\end{pgfscope}%
\begin{pgfscope}%
\pgfpathrectangle{\pgfqpoint{0.847223in}{0.554012in}}{\pgfqpoint{6.200000in}{4.620000in}}%
\pgfusepath{clip}%
\pgfsetbuttcap%
\pgfsetroundjoin%
\pgfsetlinewidth{1.003750pt}%
\definecolor{currentstroke}{rgb}{1.000000,0.000000,0.000000}%
\pgfsetstrokecolor{currentstroke}%
\pgfsetdash{}{0pt}%
\pgfpathmoveto{\pgfqpoint{0.921888in}{4.502394in}}%
\pgfpathcurveto{\pgfqpoint{0.932938in}{4.502394in}}{\pgfqpoint{0.943537in}{4.506784in}}{\pgfqpoint{0.951351in}{4.514598in}}%
\pgfpathcurveto{\pgfqpoint{0.959164in}{4.522412in}}{\pgfqpoint{0.963555in}{4.533011in}}{\pgfqpoint{0.963555in}{4.544061in}}%
\pgfpathcurveto{\pgfqpoint{0.963555in}{4.555111in}}{\pgfqpoint{0.959164in}{4.565710in}}{\pgfqpoint{0.951351in}{4.573524in}}%
\pgfpathcurveto{\pgfqpoint{0.943537in}{4.581337in}}{\pgfqpoint{0.932938in}{4.585728in}}{\pgfqpoint{0.921888in}{4.585728in}}%
\pgfpathcurveto{\pgfqpoint{0.910838in}{4.585728in}}{\pgfqpoint{0.900239in}{4.581337in}}{\pgfqpoint{0.892425in}{4.573524in}}%
\pgfpathcurveto{\pgfqpoint{0.884612in}{4.565710in}}{\pgfqpoint{0.880221in}{4.555111in}}{\pgfqpoint{0.880221in}{4.544061in}}%
\pgfpathcurveto{\pgfqpoint{0.880221in}{4.533011in}}{\pgfqpoint{0.884612in}{4.522412in}}{\pgfqpoint{0.892425in}{4.514598in}}%
\pgfpathcurveto{\pgfqpoint{0.900239in}{4.506784in}}{\pgfqpoint{0.910838in}{4.502394in}}{\pgfqpoint{0.921888in}{4.502394in}}%
\pgfpathlineto{\pgfqpoint{0.921888in}{4.502394in}}%
\pgfpathclose%
\pgfusepath{stroke}%
\end{pgfscope}%
\begin{pgfscope}%
\pgfpathrectangle{\pgfqpoint{0.847223in}{0.554012in}}{\pgfqpoint{6.200000in}{4.620000in}}%
\pgfusepath{clip}%
\pgfsetbuttcap%
\pgfsetroundjoin%
\pgfsetlinewidth{1.003750pt}%
\definecolor{currentstroke}{rgb}{1.000000,0.000000,0.000000}%
\pgfsetstrokecolor{currentstroke}%
\pgfsetdash{}{0pt}%
\pgfpathmoveto{\pgfqpoint{0.927221in}{4.463364in}}%
\pgfpathcurveto{\pgfqpoint{0.938271in}{4.463364in}}{\pgfqpoint{0.948870in}{4.467754in}}{\pgfqpoint{0.956684in}{4.475568in}}%
\pgfpathcurveto{\pgfqpoint{0.964498in}{4.483381in}}{\pgfqpoint{0.968888in}{4.493980in}}{\pgfqpoint{0.968888in}{4.505030in}}%
\pgfpathcurveto{\pgfqpoint{0.968888in}{4.516081in}}{\pgfqpoint{0.964498in}{4.526680in}}{\pgfqpoint{0.956684in}{4.534493in}}%
\pgfpathcurveto{\pgfqpoint{0.948870in}{4.542307in}}{\pgfqpoint{0.938271in}{4.546697in}}{\pgfqpoint{0.927221in}{4.546697in}}%
\pgfpathcurveto{\pgfqpoint{0.916171in}{4.546697in}}{\pgfqpoint{0.905572in}{4.542307in}}{\pgfqpoint{0.897759in}{4.534493in}}%
\pgfpathcurveto{\pgfqpoint{0.889945in}{4.526680in}}{\pgfqpoint{0.885555in}{4.516081in}}{\pgfqpoint{0.885555in}{4.505030in}}%
\pgfpathcurveto{\pgfqpoint{0.885555in}{4.493980in}}{\pgfqpoint{0.889945in}{4.483381in}}{\pgfqpoint{0.897759in}{4.475568in}}%
\pgfpathcurveto{\pgfqpoint{0.905572in}{4.467754in}}{\pgfqpoint{0.916171in}{4.463364in}}{\pgfqpoint{0.927221in}{4.463364in}}%
\pgfpathlineto{\pgfqpoint{0.927221in}{4.463364in}}%
\pgfpathclose%
\pgfusepath{stroke}%
\end{pgfscope}%
\begin{pgfscope}%
\pgfpathrectangle{\pgfqpoint{0.847223in}{0.554012in}}{\pgfqpoint{6.200000in}{4.620000in}}%
\pgfusepath{clip}%
\pgfsetbuttcap%
\pgfsetroundjoin%
\pgfsetlinewidth{1.003750pt}%
\definecolor{currentstroke}{rgb}{1.000000,0.000000,0.000000}%
\pgfsetstrokecolor{currentstroke}%
\pgfsetdash{}{0pt}%
\pgfpathmoveto{\pgfqpoint{0.932555in}{4.425017in}}%
\pgfpathcurveto{\pgfqpoint{0.943605in}{4.425017in}}{\pgfqpoint{0.954204in}{4.429408in}}{\pgfqpoint{0.962017in}{4.437221in}}%
\pgfpathcurveto{\pgfqpoint{0.969831in}{4.445035in}}{\pgfqpoint{0.974221in}{4.455634in}}{\pgfqpoint{0.974221in}{4.466684in}}%
\pgfpathcurveto{\pgfqpoint{0.974221in}{4.477734in}}{\pgfqpoint{0.969831in}{4.488333in}}{\pgfqpoint{0.962017in}{4.496147in}}%
\pgfpathcurveto{\pgfqpoint{0.954204in}{4.503960in}}{\pgfqpoint{0.943605in}{4.508351in}}{\pgfqpoint{0.932555in}{4.508351in}}%
\pgfpathcurveto{\pgfqpoint{0.921504in}{4.508351in}}{\pgfqpoint{0.910905in}{4.503960in}}{\pgfqpoint{0.903092in}{4.496147in}}%
\pgfpathcurveto{\pgfqpoint{0.895278in}{4.488333in}}{\pgfqpoint{0.890888in}{4.477734in}}{\pgfqpoint{0.890888in}{4.466684in}}%
\pgfpathcurveto{\pgfqpoint{0.890888in}{4.455634in}}{\pgfqpoint{0.895278in}{4.445035in}}{\pgfqpoint{0.903092in}{4.437221in}}%
\pgfpathcurveto{\pgfqpoint{0.910905in}{4.429408in}}{\pgfqpoint{0.921504in}{4.425017in}}{\pgfqpoint{0.932555in}{4.425017in}}%
\pgfpathlineto{\pgfqpoint{0.932555in}{4.425017in}}%
\pgfpathclose%
\pgfusepath{stroke}%
\end{pgfscope}%
\begin{pgfscope}%
\pgfpathrectangle{\pgfqpoint{0.847223in}{0.554012in}}{\pgfqpoint{6.200000in}{4.620000in}}%
\pgfusepath{clip}%
\pgfsetbuttcap%
\pgfsetroundjoin%
\pgfsetlinewidth{1.003750pt}%
\definecolor{currentstroke}{rgb}{1.000000,0.000000,0.000000}%
\pgfsetstrokecolor{currentstroke}%
\pgfsetdash{}{0pt}%
\pgfpathmoveto{\pgfqpoint{0.937888in}{4.387337in}}%
\pgfpathcurveto{\pgfqpoint{0.948938in}{4.387337in}}{\pgfqpoint{0.959537in}{4.391727in}}{\pgfqpoint{0.967351in}{4.399541in}}%
\pgfpathcurveto{\pgfqpoint{0.975164in}{4.407355in}}{\pgfqpoint{0.979554in}{4.417954in}}{\pgfqpoint{0.979554in}{4.429004in}}%
\pgfpathcurveto{\pgfqpoint{0.979554in}{4.440054in}}{\pgfqpoint{0.975164in}{4.450653in}}{\pgfqpoint{0.967351in}{4.458467in}}%
\pgfpathcurveto{\pgfqpoint{0.959537in}{4.466280in}}{\pgfqpoint{0.948938in}{4.470670in}}{\pgfqpoint{0.937888in}{4.470670in}}%
\pgfpathcurveto{\pgfqpoint{0.926838in}{4.470670in}}{\pgfqpoint{0.916239in}{4.466280in}}{\pgfqpoint{0.908425in}{4.458467in}}%
\pgfpathcurveto{\pgfqpoint{0.900611in}{4.450653in}}{\pgfqpoint{0.896221in}{4.440054in}}{\pgfqpoint{0.896221in}{4.429004in}}%
\pgfpathcurveto{\pgfqpoint{0.896221in}{4.417954in}}{\pgfqpoint{0.900611in}{4.407355in}}{\pgfqpoint{0.908425in}{4.399541in}}%
\pgfpathcurveto{\pgfqpoint{0.916239in}{4.391727in}}{\pgfqpoint{0.926838in}{4.387337in}}{\pgfqpoint{0.937888in}{4.387337in}}%
\pgfpathlineto{\pgfqpoint{0.937888in}{4.387337in}}%
\pgfpathclose%
\pgfusepath{stroke}%
\end{pgfscope}%
\begin{pgfscope}%
\pgfpathrectangle{\pgfqpoint{0.847223in}{0.554012in}}{\pgfqpoint{6.200000in}{4.620000in}}%
\pgfusepath{clip}%
\pgfsetbuttcap%
\pgfsetroundjoin%
\pgfsetlinewidth{1.003750pt}%
\definecolor{currentstroke}{rgb}{1.000000,0.000000,0.000000}%
\pgfsetstrokecolor{currentstroke}%
\pgfsetdash{}{0pt}%
\pgfpathmoveto{\pgfqpoint{0.943221in}{4.350306in}}%
\pgfpathcurveto{\pgfqpoint{0.954271in}{4.350306in}}{\pgfqpoint{0.964870in}{4.354696in}}{\pgfqpoint{0.972684in}{4.362510in}}%
\pgfpathcurveto{\pgfqpoint{0.980497in}{4.370323in}}{\pgfqpoint{0.984888in}{4.380922in}}{\pgfqpoint{0.984888in}{4.391972in}}%
\pgfpathcurveto{\pgfqpoint{0.984888in}{4.403023in}}{\pgfqpoint{0.980497in}{4.413622in}}{\pgfqpoint{0.972684in}{4.421435in}}%
\pgfpathcurveto{\pgfqpoint{0.964870in}{4.429249in}}{\pgfqpoint{0.954271in}{4.433639in}}{\pgfqpoint{0.943221in}{4.433639in}}%
\pgfpathcurveto{\pgfqpoint{0.932171in}{4.433639in}}{\pgfqpoint{0.921572in}{4.429249in}}{\pgfqpoint{0.913758in}{4.421435in}}%
\pgfpathcurveto{\pgfqpoint{0.905945in}{4.413622in}}{\pgfqpoint{0.901554in}{4.403023in}}{\pgfqpoint{0.901554in}{4.391972in}}%
\pgfpathcurveto{\pgfqpoint{0.901554in}{4.380922in}}{\pgfqpoint{0.905945in}{4.370323in}}{\pgfqpoint{0.913758in}{4.362510in}}%
\pgfpathcurveto{\pgfqpoint{0.921572in}{4.354696in}}{\pgfqpoint{0.932171in}{4.350306in}}{\pgfqpoint{0.943221in}{4.350306in}}%
\pgfpathlineto{\pgfqpoint{0.943221in}{4.350306in}}%
\pgfpathclose%
\pgfusepath{stroke}%
\end{pgfscope}%
\begin{pgfscope}%
\pgfpathrectangle{\pgfqpoint{0.847223in}{0.554012in}}{\pgfqpoint{6.200000in}{4.620000in}}%
\pgfusepath{clip}%
\pgfsetbuttcap%
\pgfsetroundjoin%
\pgfsetlinewidth{1.003750pt}%
\definecolor{currentstroke}{rgb}{1.000000,0.000000,0.000000}%
\pgfsetstrokecolor{currentstroke}%
\pgfsetdash{}{0pt}%
\pgfpathmoveto{\pgfqpoint{0.948554in}{4.313907in}}%
\pgfpathcurveto{\pgfqpoint{0.959604in}{4.313907in}}{\pgfqpoint{0.970203in}{4.318297in}}{\pgfqpoint{0.978017in}{4.326111in}}%
\pgfpathcurveto{\pgfqpoint{0.985831in}{4.333924in}}{\pgfqpoint{0.990221in}{4.344523in}}{\pgfqpoint{0.990221in}{4.355573in}}%
\pgfpathcurveto{\pgfqpoint{0.990221in}{4.366623in}}{\pgfqpoint{0.985831in}{4.377222in}}{\pgfqpoint{0.978017in}{4.385036in}}%
\pgfpathcurveto{\pgfqpoint{0.970203in}{4.392850in}}{\pgfqpoint{0.959604in}{4.397240in}}{\pgfqpoint{0.948554in}{4.397240in}}%
\pgfpathcurveto{\pgfqpoint{0.937504in}{4.397240in}}{\pgfqpoint{0.926905in}{4.392850in}}{\pgfqpoint{0.919091in}{4.385036in}}%
\pgfpathcurveto{\pgfqpoint{0.911278in}{4.377222in}}{\pgfqpoint{0.906887in}{4.366623in}}{\pgfqpoint{0.906887in}{4.355573in}}%
\pgfpathcurveto{\pgfqpoint{0.906887in}{4.344523in}}{\pgfqpoint{0.911278in}{4.333924in}}{\pgfqpoint{0.919091in}{4.326111in}}%
\pgfpathcurveto{\pgfqpoint{0.926905in}{4.318297in}}{\pgfqpoint{0.937504in}{4.313907in}}{\pgfqpoint{0.948554in}{4.313907in}}%
\pgfpathlineto{\pgfqpoint{0.948554in}{4.313907in}}%
\pgfpathclose%
\pgfusepath{stroke}%
\end{pgfscope}%
\begin{pgfscope}%
\pgfpathrectangle{\pgfqpoint{0.847223in}{0.554012in}}{\pgfqpoint{6.200000in}{4.620000in}}%
\pgfusepath{clip}%
\pgfsetbuttcap%
\pgfsetroundjoin%
\pgfsetlinewidth{1.003750pt}%
\definecolor{currentstroke}{rgb}{1.000000,0.000000,0.000000}%
\pgfsetstrokecolor{currentstroke}%
\pgfsetdash{}{0pt}%
\pgfpathmoveto{\pgfqpoint{0.953887in}{4.278124in}}%
\pgfpathcurveto{\pgfqpoint{0.964937in}{4.278124in}}{\pgfqpoint{0.975537in}{4.282514in}}{\pgfqpoint{0.983350in}{4.290328in}}%
\pgfpathcurveto{\pgfqpoint{0.991164in}{4.298141in}}{\pgfqpoint{0.995554in}{4.308740in}}{\pgfqpoint{0.995554in}{4.319791in}}%
\pgfpathcurveto{\pgfqpoint{0.995554in}{4.330841in}}{\pgfqpoint{0.991164in}{4.341440in}}{\pgfqpoint{0.983350in}{4.349253in}}%
\pgfpathcurveto{\pgfqpoint{0.975537in}{4.357067in}}{\pgfqpoint{0.964937in}{4.361457in}}{\pgfqpoint{0.953887in}{4.361457in}}%
\pgfpathcurveto{\pgfqpoint{0.942837in}{4.361457in}}{\pgfqpoint{0.932238in}{4.357067in}}{\pgfqpoint{0.924425in}{4.349253in}}%
\pgfpathcurveto{\pgfqpoint{0.916611in}{4.341440in}}{\pgfqpoint{0.912221in}{4.330841in}}{\pgfqpoint{0.912221in}{4.319791in}}%
\pgfpathcurveto{\pgfqpoint{0.912221in}{4.308740in}}{\pgfqpoint{0.916611in}{4.298141in}}{\pgfqpoint{0.924425in}{4.290328in}}%
\pgfpathcurveto{\pgfqpoint{0.932238in}{4.282514in}}{\pgfqpoint{0.942837in}{4.278124in}}{\pgfqpoint{0.953887in}{4.278124in}}%
\pgfpathlineto{\pgfqpoint{0.953887in}{4.278124in}}%
\pgfpathclose%
\pgfusepath{stroke}%
\end{pgfscope}%
\begin{pgfscope}%
\pgfpathrectangle{\pgfqpoint{0.847223in}{0.554012in}}{\pgfqpoint{6.200000in}{4.620000in}}%
\pgfusepath{clip}%
\pgfsetbuttcap%
\pgfsetroundjoin%
\pgfsetlinewidth{1.003750pt}%
\definecolor{currentstroke}{rgb}{1.000000,0.000000,0.000000}%
\pgfsetstrokecolor{currentstroke}%
\pgfsetdash{}{0pt}%
\pgfpathmoveto{\pgfqpoint{0.959221in}{4.242942in}}%
\pgfpathcurveto{\pgfqpoint{0.970271in}{4.242942in}}{\pgfqpoint{0.980870in}{4.247332in}}{\pgfqpoint{0.988683in}{4.255146in}}%
\pgfpathcurveto{\pgfqpoint{0.996497in}{4.262959in}}{\pgfqpoint{1.000887in}{4.273558in}}{\pgfqpoint{1.000887in}{4.284608in}}%
\pgfpathcurveto{\pgfqpoint{1.000887in}{4.295659in}}{\pgfqpoint{0.996497in}{4.306258in}}{\pgfqpoint{0.988683in}{4.314071in}}%
\pgfpathcurveto{\pgfqpoint{0.980870in}{4.321885in}}{\pgfqpoint{0.970271in}{4.326275in}}{\pgfqpoint{0.959221in}{4.326275in}}%
\pgfpathcurveto{\pgfqpoint{0.948170in}{4.326275in}}{\pgfqpoint{0.937571in}{4.321885in}}{\pgfqpoint{0.929758in}{4.314071in}}%
\pgfpathcurveto{\pgfqpoint{0.921944in}{4.306258in}}{\pgfqpoint{0.917554in}{4.295659in}}{\pgfqpoint{0.917554in}{4.284608in}}%
\pgfpathcurveto{\pgfqpoint{0.917554in}{4.273558in}}{\pgfqpoint{0.921944in}{4.262959in}}{\pgfqpoint{0.929758in}{4.255146in}}%
\pgfpathcurveto{\pgfqpoint{0.937571in}{4.247332in}}{\pgfqpoint{0.948170in}{4.242942in}}{\pgfqpoint{0.959221in}{4.242942in}}%
\pgfpathlineto{\pgfqpoint{0.959221in}{4.242942in}}%
\pgfpathclose%
\pgfusepath{stroke}%
\end{pgfscope}%
\begin{pgfscope}%
\pgfpathrectangle{\pgfqpoint{0.847223in}{0.554012in}}{\pgfqpoint{6.200000in}{4.620000in}}%
\pgfusepath{clip}%
\pgfsetbuttcap%
\pgfsetroundjoin%
\pgfsetlinewidth{1.003750pt}%
\definecolor{currentstroke}{rgb}{1.000000,0.000000,0.000000}%
\pgfsetstrokecolor{currentstroke}%
\pgfsetdash{}{0pt}%
\pgfpathmoveto{\pgfqpoint{0.964554in}{4.208345in}}%
\pgfpathcurveto{\pgfqpoint{0.975604in}{4.208345in}}{\pgfqpoint{0.986203in}{4.212736in}}{\pgfqpoint{0.994017in}{4.220549in}}%
\pgfpathcurveto{\pgfqpoint{1.001830in}{4.228363in}}{\pgfqpoint{1.006220in}{4.238962in}}{\pgfqpoint{1.006220in}{4.250012in}}%
\pgfpathcurveto{\pgfqpoint{1.006220in}{4.261062in}}{\pgfqpoint{1.001830in}{4.271661in}}{\pgfqpoint{0.994017in}{4.279475in}}%
\pgfpathcurveto{\pgfqpoint{0.986203in}{4.287289in}}{\pgfqpoint{0.975604in}{4.291679in}}{\pgfqpoint{0.964554in}{4.291679in}}%
\pgfpathcurveto{\pgfqpoint{0.953504in}{4.291679in}}{\pgfqpoint{0.942905in}{4.287289in}}{\pgfqpoint{0.935091in}{4.279475in}}%
\pgfpathcurveto{\pgfqpoint{0.927277in}{4.271661in}}{\pgfqpoint{0.922887in}{4.261062in}}{\pgfqpoint{0.922887in}{4.250012in}}%
\pgfpathcurveto{\pgfqpoint{0.922887in}{4.238962in}}{\pgfqpoint{0.927277in}{4.228363in}}{\pgfqpoint{0.935091in}{4.220549in}}%
\pgfpathcurveto{\pgfqpoint{0.942905in}{4.212736in}}{\pgfqpoint{0.953504in}{4.208345in}}{\pgfqpoint{0.964554in}{4.208345in}}%
\pgfpathlineto{\pgfqpoint{0.964554in}{4.208345in}}%
\pgfpathclose%
\pgfusepath{stroke}%
\end{pgfscope}%
\begin{pgfscope}%
\pgfpathrectangle{\pgfqpoint{0.847223in}{0.554012in}}{\pgfqpoint{6.200000in}{4.620000in}}%
\pgfusepath{clip}%
\pgfsetbuttcap%
\pgfsetroundjoin%
\pgfsetlinewidth{1.003750pt}%
\definecolor{currentstroke}{rgb}{1.000000,0.000000,0.000000}%
\pgfsetstrokecolor{currentstroke}%
\pgfsetdash{}{0pt}%
\pgfpathmoveto{\pgfqpoint{0.969887in}{4.174320in}}%
\pgfpathcurveto{\pgfqpoint{0.980937in}{4.174320in}}{\pgfqpoint{0.991536in}{4.178711in}}{\pgfqpoint{0.999350in}{4.186524in}}%
\pgfpathcurveto{\pgfqpoint{1.007163in}{4.194338in}}{\pgfqpoint{1.011554in}{4.204937in}}{\pgfqpoint{1.011554in}{4.215987in}}%
\pgfpathcurveto{\pgfqpoint{1.011554in}{4.227037in}}{\pgfqpoint{1.007163in}{4.237636in}}{\pgfqpoint{0.999350in}{4.245450in}}%
\pgfpathcurveto{\pgfqpoint{0.991536in}{4.253263in}}{\pgfqpoint{0.980937in}{4.257654in}}{\pgfqpoint{0.969887in}{4.257654in}}%
\pgfpathcurveto{\pgfqpoint{0.958837in}{4.257654in}}{\pgfqpoint{0.948238in}{4.253263in}}{\pgfqpoint{0.940424in}{4.245450in}}%
\pgfpathcurveto{\pgfqpoint{0.932611in}{4.237636in}}{\pgfqpoint{0.928220in}{4.227037in}}{\pgfqpoint{0.928220in}{4.215987in}}%
\pgfpathcurveto{\pgfqpoint{0.928220in}{4.204937in}}{\pgfqpoint{0.932611in}{4.194338in}}{\pgfqpoint{0.940424in}{4.186524in}}%
\pgfpathcurveto{\pgfqpoint{0.948238in}{4.178711in}}{\pgfqpoint{0.958837in}{4.174320in}}{\pgfqpoint{0.969887in}{4.174320in}}%
\pgfpathlineto{\pgfqpoint{0.969887in}{4.174320in}}%
\pgfpathclose%
\pgfusepath{stroke}%
\end{pgfscope}%
\begin{pgfscope}%
\pgfpathrectangle{\pgfqpoint{0.847223in}{0.554012in}}{\pgfqpoint{6.200000in}{4.620000in}}%
\pgfusepath{clip}%
\pgfsetbuttcap%
\pgfsetroundjoin%
\pgfsetlinewidth{1.003750pt}%
\definecolor{currentstroke}{rgb}{1.000000,0.000000,0.000000}%
\pgfsetstrokecolor{currentstroke}%
\pgfsetdash{}{0pt}%
\pgfpathmoveto{\pgfqpoint{0.975220in}{4.140852in}}%
\pgfpathcurveto{\pgfqpoint{0.986270in}{4.140852in}}{\pgfqpoint{0.996869in}{4.145243in}}{\pgfqpoint{1.004683in}{4.153056in}}%
\pgfpathcurveto{\pgfqpoint{1.012497in}{4.160870in}}{\pgfqpoint{1.016887in}{4.171469in}}{\pgfqpoint{1.016887in}{4.182519in}}%
\pgfpathcurveto{\pgfqpoint{1.016887in}{4.193569in}}{\pgfqpoint{1.012497in}{4.204168in}}{\pgfqpoint{1.004683in}{4.211982in}}%
\pgfpathcurveto{\pgfqpoint{0.996869in}{4.219796in}}{\pgfqpoint{0.986270in}{4.224186in}}{\pgfqpoint{0.975220in}{4.224186in}}%
\pgfpathcurveto{\pgfqpoint{0.964170in}{4.224186in}}{\pgfqpoint{0.953571in}{4.219796in}}{\pgfqpoint{0.945757in}{4.211982in}}%
\pgfpathcurveto{\pgfqpoint{0.937944in}{4.204168in}}{\pgfqpoint{0.933554in}{4.193569in}}{\pgfqpoint{0.933554in}{4.182519in}}%
\pgfpathcurveto{\pgfqpoint{0.933554in}{4.171469in}}{\pgfqpoint{0.937944in}{4.160870in}}{\pgfqpoint{0.945757in}{4.153056in}}%
\pgfpathcurveto{\pgfqpoint{0.953571in}{4.145243in}}{\pgfqpoint{0.964170in}{4.140852in}}{\pgfqpoint{0.975220in}{4.140852in}}%
\pgfpathlineto{\pgfqpoint{0.975220in}{4.140852in}}%
\pgfpathclose%
\pgfusepath{stroke}%
\end{pgfscope}%
\begin{pgfscope}%
\pgfpathrectangle{\pgfqpoint{0.847223in}{0.554012in}}{\pgfqpoint{6.200000in}{4.620000in}}%
\pgfusepath{clip}%
\pgfsetbuttcap%
\pgfsetroundjoin%
\pgfsetlinewidth{1.003750pt}%
\definecolor{currentstroke}{rgb}{1.000000,0.000000,0.000000}%
\pgfsetstrokecolor{currentstroke}%
\pgfsetdash{}{0pt}%
\pgfpathmoveto{\pgfqpoint{0.980553in}{4.107928in}}%
\pgfpathcurveto{\pgfqpoint{0.991604in}{4.107928in}}{\pgfqpoint{1.002203in}{4.112318in}}{\pgfqpoint{1.010016in}{4.120132in}}%
\pgfpathcurveto{\pgfqpoint{1.017830in}{4.127946in}}{\pgfqpoint{1.022220in}{4.138545in}}{\pgfqpoint{1.022220in}{4.149595in}}%
\pgfpathcurveto{\pgfqpoint{1.022220in}{4.160645in}}{\pgfqpoint{1.017830in}{4.171244in}}{\pgfqpoint{1.010016in}{4.179058in}}%
\pgfpathcurveto{\pgfqpoint{1.002203in}{4.186871in}}{\pgfqpoint{0.991604in}{4.191262in}}{\pgfqpoint{0.980553in}{4.191262in}}%
\pgfpathcurveto{\pgfqpoint{0.969503in}{4.191262in}}{\pgfqpoint{0.958904in}{4.186871in}}{\pgfqpoint{0.951091in}{4.179058in}}%
\pgfpathcurveto{\pgfqpoint{0.943277in}{4.171244in}}{\pgfqpoint{0.938887in}{4.160645in}}{\pgfqpoint{0.938887in}{4.149595in}}%
\pgfpathcurveto{\pgfqpoint{0.938887in}{4.138545in}}{\pgfqpoint{0.943277in}{4.127946in}}{\pgfqpoint{0.951091in}{4.120132in}}%
\pgfpathcurveto{\pgfqpoint{0.958904in}{4.112318in}}{\pgfqpoint{0.969503in}{4.107928in}}{\pgfqpoint{0.980553in}{4.107928in}}%
\pgfpathlineto{\pgfqpoint{0.980553in}{4.107928in}}%
\pgfpathclose%
\pgfusepath{stroke}%
\end{pgfscope}%
\begin{pgfscope}%
\pgfpathrectangle{\pgfqpoint{0.847223in}{0.554012in}}{\pgfqpoint{6.200000in}{4.620000in}}%
\pgfusepath{clip}%
\pgfsetbuttcap%
\pgfsetroundjoin%
\pgfsetlinewidth{1.003750pt}%
\definecolor{currentstroke}{rgb}{1.000000,0.000000,0.000000}%
\pgfsetstrokecolor{currentstroke}%
\pgfsetdash{}{0pt}%
\pgfpathmoveto{\pgfqpoint{0.985887in}{4.075534in}}%
\pgfpathcurveto{\pgfqpoint{0.996937in}{4.075534in}}{\pgfqpoint{1.007536in}{4.079925in}}{\pgfqpoint{1.015349in}{4.087738in}}%
\pgfpathcurveto{\pgfqpoint{1.023163in}{4.095552in}}{\pgfqpoint{1.027553in}{4.106151in}}{\pgfqpoint{1.027553in}{4.117201in}}%
\pgfpathcurveto{\pgfqpoint{1.027553in}{4.128251in}}{\pgfqpoint{1.023163in}{4.138850in}}{\pgfqpoint{1.015349in}{4.146664in}}%
\pgfpathcurveto{\pgfqpoint{1.007536in}{4.154477in}}{\pgfqpoint{0.996937in}{4.158868in}}{\pgfqpoint{0.985887in}{4.158868in}}%
\pgfpathcurveto{\pgfqpoint{0.974837in}{4.158868in}}{\pgfqpoint{0.964238in}{4.154477in}}{\pgfqpoint{0.956424in}{4.146664in}}%
\pgfpathcurveto{\pgfqpoint{0.948610in}{4.138850in}}{\pgfqpoint{0.944220in}{4.128251in}}{\pgfqpoint{0.944220in}{4.117201in}}%
\pgfpathcurveto{\pgfqpoint{0.944220in}{4.106151in}}{\pgfqpoint{0.948610in}{4.095552in}}{\pgfqpoint{0.956424in}{4.087738in}}%
\pgfpathcurveto{\pgfqpoint{0.964238in}{4.079925in}}{\pgfqpoint{0.974837in}{4.075534in}}{\pgfqpoint{0.985887in}{4.075534in}}%
\pgfpathlineto{\pgfqpoint{0.985887in}{4.075534in}}%
\pgfpathclose%
\pgfusepath{stroke}%
\end{pgfscope}%
\begin{pgfscope}%
\pgfpathrectangle{\pgfqpoint{0.847223in}{0.554012in}}{\pgfqpoint{6.200000in}{4.620000in}}%
\pgfusepath{clip}%
\pgfsetbuttcap%
\pgfsetroundjoin%
\pgfsetlinewidth{1.003750pt}%
\definecolor{currentstroke}{rgb}{1.000000,0.000000,0.000000}%
\pgfsetstrokecolor{currentstroke}%
\pgfsetdash{}{0pt}%
\pgfpathmoveto{\pgfqpoint{0.991220in}{4.043658in}}%
\pgfpathcurveto{\pgfqpoint{1.002270in}{4.043658in}}{\pgfqpoint{1.012869in}{4.048049in}}{\pgfqpoint{1.020683in}{4.055862in}}%
\pgfpathcurveto{\pgfqpoint{1.028496in}{4.063676in}}{\pgfqpoint{1.032887in}{4.074275in}}{\pgfqpoint{1.032887in}{4.085325in}}%
\pgfpathcurveto{\pgfqpoint{1.032887in}{4.096375in}}{\pgfqpoint{1.028496in}{4.106974in}}{\pgfqpoint{1.020683in}{4.114788in}}%
\pgfpathcurveto{\pgfqpoint{1.012869in}{4.122601in}}{\pgfqpoint{1.002270in}{4.126992in}}{\pgfqpoint{0.991220in}{4.126992in}}%
\pgfpathcurveto{\pgfqpoint{0.980170in}{4.126992in}}{\pgfqpoint{0.969571in}{4.122601in}}{\pgfqpoint{0.961757in}{4.114788in}}%
\pgfpathcurveto{\pgfqpoint{0.953943in}{4.106974in}}{\pgfqpoint{0.949553in}{4.096375in}}{\pgfqpoint{0.949553in}{4.085325in}}%
\pgfpathcurveto{\pgfqpoint{0.949553in}{4.074275in}}{\pgfqpoint{0.953943in}{4.063676in}}{\pgfqpoint{0.961757in}{4.055862in}}%
\pgfpathcurveto{\pgfqpoint{0.969571in}{4.048049in}}{\pgfqpoint{0.980170in}{4.043658in}}{\pgfqpoint{0.991220in}{4.043658in}}%
\pgfpathlineto{\pgfqpoint{0.991220in}{4.043658in}}%
\pgfpathclose%
\pgfusepath{stroke}%
\end{pgfscope}%
\begin{pgfscope}%
\pgfpathrectangle{\pgfqpoint{0.847223in}{0.554012in}}{\pgfqpoint{6.200000in}{4.620000in}}%
\pgfusepath{clip}%
\pgfsetbuttcap%
\pgfsetroundjoin%
\pgfsetlinewidth{1.003750pt}%
\definecolor{currentstroke}{rgb}{1.000000,0.000000,0.000000}%
\pgfsetstrokecolor{currentstroke}%
\pgfsetdash{}{0pt}%
\pgfpathmoveto{\pgfqpoint{0.996553in}{4.012288in}}%
\pgfpathcurveto{\pgfqpoint{1.007603in}{4.012288in}}{\pgfqpoint{1.018202in}{4.016678in}}{\pgfqpoint{1.026016in}{4.024492in}}%
\pgfpathcurveto{\pgfqpoint{1.033829in}{4.032305in}}{\pgfqpoint{1.038220in}{4.042904in}}{\pgfqpoint{1.038220in}{4.053954in}}%
\pgfpathcurveto{\pgfqpoint{1.038220in}{4.065004in}}{\pgfqpoint{1.033829in}{4.075604in}}{\pgfqpoint{1.026016in}{4.083417in}}%
\pgfpathcurveto{\pgfqpoint{1.018202in}{4.091231in}}{\pgfqpoint{1.007603in}{4.095621in}}{\pgfqpoint{0.996553in}{4.095621in}}%
\pgfpathcurveto{\pgfqpoint{0.985503in}{4.095621in}}{\pgfqpoint{0.974904in}{4.091231in}}{\pgfqpoint{0.967090in}{4.083417in}}%
\pgfpathcurveto{\pgfqpoint{0.959277in}{4.075604in}}{\pgfqpoint{0.954886in}{4.065004in}}{\pgfqpoint{0.954886in}{4.053954in}}%
\pgfpathcurveto{\pgfqpoint{0.954886in}{4.042904in}}{\pgfqpoint{0.959277in}{4.032305in}}{\pgfqpoint{0.967090in}{4.024492in}}%
\pgfpathcurveto{\pgfqpoint{0.974904in}{4.016678in}}{\pgfqpoint{0.985503in}{4.012288in}}{\pgfqpoint{0.996553in}{4.012288in}}%
\pgfpathlineto{\pgfqpoint{0.996553in}{4.012288in}}%
\pgfpathclose%
\pgfusepath{stroke}%
\end{pgfscope}%
\begin{pgfscope}%
\pgfpathrectangle{\pgfqpoint{0.847223in}{0.554012in}}{\pgfqpoint{6.200000in}{4.620000in}}%
\pgfusepath{clip}%
\pgfsetbuttcap%
\pgfsetroundjoin%
\pgfsetlinewidth{1.003750pt}%
\definecolor{currentstroke}{rgb}{1.000000,0.000000,0.000000}%
\pgfsetstrokecolor{currentstroke}%
\pgfsetdash{}{0pt}%
\pgfpathmoveto{\pgfqpoint{1.001886in}{3.981411in}}%
\pgfpathcurveto{\pgfqpoint{1.012936in}{3.981411in}}{\pgfqpoint{1.023535in}{3.985801in}}{\pgfqpoint{1.031349in}{3.993614in}}%
\pgfpathcurveto{\pgfqpoint{1.039163in}{4.001428in}}{\pgfqpoint{1.043553in}{4.012027in}}{\pgfqpoint{1.043553in}{4.023077in}}%
\pgfpathcurveto{\pgfqpoint{1.043553in}{4.034127in}}{\pgfqpoint{1.039163in}{4.044726in}}{\pgfqpoint{1.031349in}{4.052540in}}%
\pgfpathcurveto{\pgfqpoint{1.023535in}{4.060354in}}{\pgfqpoint{1.012936in}{4.064744in}}{\pgfqpoint{1.001886in}{4.064744in}}%
\pgfpathcurveto{\pgfqpoint{0.990836in}{4.064744in}}{\pgfqpoint{0.980237in}{4.060354in}}{\pgfqpoint{0.972424in}{4.052540in}}%
\pgfpathcurveto{\pgfqpoint{0.964610in}{4.044726in}}{\pgfqpoint{0.960220in}{4.034127in}}{\pgfqpoint{0.960220in}{4.023077in}}%
\pgfpathcurveto{\pgfqpoint{0.960220in}{4.012027in}}{\pgfqpoint{0.964610in}{4.001428in}}{\pgfqpoint{0.972424in}{3.993614in}}%
\pgfpathcurveto{\pgfqpoint{0.980237in}{3.985801in}}{\pgfqpoint{0.990836in}{3.981411in}}{\pgfqpoint{1.001886in}{3.981411in}}%
\pgfpathlineto{\pgfqpoint{1.001886in}{3.981411in}}%
\pgfpathclose%
\pgfusepath{stroke}%
\end{pgfscope}%
\begin{pgfscope}%
\pgfpathrectangle{\pgfqpoint{0.847223in}{0.554012in}}{\pgfqpoint{6.200000in}{4.620000in}}%
\pgfusepath{clip}%
\pgfsetbuttcap%
\pgfsetroundjoin%
\pgfsetlinewidth{1.003750pt}%
\definecolor{currentstroke}{rgb}{1.000000,0.000000,0.000000}%
\pgfsetstrokecolor{currentstroke}%
\pgfsetdash{}{0pt}%
\pgfpathmoveto{\pgfqpoint{1.007220in}{3.951015in}}%
\pgfpathcurveto{\pgfqpoint{1.018270in}{3.951015in}}{\pgfqpoint{1.028869in}{3.955406in}}{\pgfqpoint{1.036682in}{3.963219in}}%
\pgfpathcurveto{\pgfqpoint{1.044496in}{3.971033in}}{\pgfqpoint{1.048886in}{3.981632in}}{\pgfqpoint{1.048886in}{3.992682in}}%
\pgfpathcurveto{\pgfqpoint{1.048886in}{4.003732in}}{\pgfqpoint{1.044496in}{4.014331in}}{\pgfqpoint{1.036682in}{4.022145in}}%
\pgfpathcurveto{\pgfqpoint{1.028869in}{4.029959in}}{\pgfqpoint{1.018270in}{4.034349in}}{\pgfqpoint{1.007220in}{4.034349in}}%
\pgfpathcurveto{\pgfqpoint{0.996169in}{4.034349in}}{\pgfqpoint{0.985570in}{4.029959in}}{\pgfqpoint{0.977757in}{4.022145in}}%
\pgfpathcurveto{\pgfqpoint{0.969943in}{4.014331in}}{\pgfqpoint{0.965553in}{4.003732in}}{\pgfqpoint{0.965553in}{3.992682in}}%
\pgfpathcurveto{\pgfqpoint{0.965553in}{3.981632in}}{\pgfqpoint{0.969943in}{3.971033in}}{\pgfqpoint{0.977757in}{3.963219in}}%
\pgfpathcurveto{\pgfqpoint{0.985570in}{3.955406in}}{\pgfqpoint{0.996169in}{3.951015in}}{\pgfqpoint{1.007220in}{3.951015in}}%
\pgfpathlineto{\pgfqpoint{1.007220in}{3.951015in}}%
\pgfpathclose%
\pgfusepath{stroke}%
\end{pgfscope}%
\begin{pgfscope}%
\pgfpathrectangle{\pgfqpoint{0.847223in}{0.554012in}}{\pgfqpoint{6.200000in}{4.620000in}}%
\pgfusepath{clip}%
\pgfsetbuttcap%
\pgfsetroundjoin%
\pgfsetlinewidth{1.003750pt}%
\definecolor{currentstroke}{rgb}{1.000000,0.000000,0.000000}%
\pgfsetstrokecolor{currentstroke}%
\pgfsetdash{}{0pt}%
\pgfpathmoveto{\pgfqpoint{1.012553in}{3.921091in}}%
\pgfpathcurveto{\pgfqpoint{1.023603in}{3.921091in}}{\pgfqpoint{1.034202in}{3.925481in}}{\pgfqpoint{1.042016in}{3.933295in}}%
\pgfpathcurveto{\pgfqpoint{1.049829in}{3.941109in}}{\pgfqpoint{1.054219in}{3.951708in}}{\pgfqpoint{1.054219in}{3.962758in}}%
\pgfpathcurveto{\pgfqpoint{1.054219in}{3.973808in}}{\pgfqpoint{1.049829in}{3.984407in}}{\pgfqpoint{1.042016in}{3.992221in}}%
\pgfpathcurveto{\pgfqpoint{1.034202in}{4.000034in}}{\pgfqpoint{1.023603in}{4.004424in}}{\pgfqpoint{1.012553in}{4.004424in}}%
\pgfpathcurveto{\pgfqpoint{1.001503in}{4.004424in}}{\pgfqpoint{0.990904in}{4.000034in}}{\pgfqpoint{0.983090in}{3.992221in}}%
\pgfpathcurveto{\pgfqpoint{0.975276in}{3.984407in}}{\pgfqpoint{0.970886in}{3.973808in}}{\pgfqpoint{0.970886in}{3.962758in}}%
\pgfpathcurveto{\pgfqpoint{0.970886in}{3.951708in}}{\pgfqpoint{0.975276in}{3.941109in}}{\pgfqpoint{0.983090in}{3.933295in}}%
\pgfpathcurveto{\pgfqpoint{0.990904in}{3.925481in}}{\pgfqpoint{1.001503in}{3.921091in}}{\pgfqpoint{1.012553in}{3.921091in}}%
\pgfpathlineto{\pgfqpoint{1.012553in}{3.921091in}}%
\pgfpathclose%
\pgfusepath{stroke}%
\end{pgfscope}%
\begin{pgfscope}%
\pgfpathrectangle{\pgfqpoint{0.847223in}{0.554012in}}{\pgfqpoint{6.200000in}{4.620000in}}%
\pgfusepath{clip}%
\pgfsetbuttcap%
\pgfsetroundjoin%
\pgfsetlinewidth{1.003750pt}%
\definecolor{currentstroke}{rgb}{1.000000,0.000000,0.000000}%
\pgfsetstrokecolor{currentstroke}%
\pgfsetdash{}{0pt}%
\pgfpathmoveto{\pgfqpoint{1.017886in}{3.891627in}}%
\pgfpathcurveto{\pgfqpoint{1.028936in}{3.891627in}}{\pgfqpoint{1.039535in}{3.896017in}}{\pgfqpoint{1.047349in}{3.903831in}}%
\pgfpathcurveto{\pgfqpoint{1.055162in}{3.911644in}}{\pgfqpoint{1.059553in}{3.922243in}}{\pgfqpoint{1.059553in}{3.933293in}}%
\pgfpathcurveto{\pgfqpoint{1.059553in}{3.944343in}}{\pgfqpoint{1.055162in}{3.954943in}}{\pgfqpoint{1.047349in}{3.962756in}}%
\pgfpathcurveto{\pgfqpoint{1.039535in}{3.970570in}}{\pgfqpoint{1.028936in}{3.974960in}}{\pgfqpoint{1.017886in}{3.974960in}}%
\pgfpathcurveto{\pgfqpoint{1.006836in}{3.974960in}}{\pgfqpoint{0.996237in}{3.970570in}}{\pgfqpoint{0.988423in}{3.962756in}}%
\pgfpathcurveto{\pgfqpoint{0.980610in}{3.954943in}}{\pgfqpoint{0.976219in}{3.944343in}}{\pgfqpoint{0.976219in}{3.933293in}}%
\pgfpathcurveto{\pgfqpoint{0.976219in}{3.922243in}}{\pgfqpoint{0.980610in}{3.911644in}}{\pgfqpoint{0.988423in}{3.903831in}}%
\pgfpathcurveto{\pgfqpoint{0.996237in}{3.896017in}}{\pgfqpoint{1.006836in}{3.891627in}}{\pgfqpoint{1.017886in}{3.891627in}}%
\pgfpathlineto{\pgfqpoint{1.017886in}{3.891627in}}%
\pgfpathclose%
\pgfusepath{stroke}%
\end{pgfscope}%
\begin{pgfscope}%
\pgfpathrectangle{\pgfqpoint{0.847223in}{0.554012in}}{\pgfqpoint{6.200000in}{4.620000in}}%
\pgfusepath{clip}%
\pgfsetbuttcap%
\pgfsetroundjoin%
\pgfsetlinewidth{1.003750pt}%
\definecolor{currentstroke}{rgb}{1.000000,0.000000,0.000000}%
\pgfsetstrokecolor{currentstroke}%
\pgfsetdash{}{0pt}%
\pgfpathmoveto{\pgfqpoint{1.023219in}{3.862612in}}%
\pgfpathcurveto{\pgfqpoint{1.034269in}{3.862612in}}{\pgfqpoint{1.044868in}{3.867002in}}{\pgfqpoint{1.052682in}{3.874816in}}%
\pgfpathcurveto{\pgfqpoint{1.060496in}{3.882629in}}{\pgfqpoint{1.064886in}{3.893228in}}{\pgfqpoint{1.064886in}{3.904278in}}%
\pgfpathcurveto{\pgfqpoint{1.064886in}{3.915328in}}{\pgfqpoint{1.060496in}{3.925927in}}{\pgfqpoint{1.052682in}{3.933741in}}%
\pgfpathcurveto{\pgfqpoint{1.044868in}{3.941555in}}{\pgfqpoint{1.034269in}{3.945945in}}{\pgfqpoint{1.023219in}{3.945945in}}%
\pgfpathcurveto{\pgfqpoint{1.012169in}{3.945945in}}{\pgfqpoint{1.001570in}{3.941555in}}{\pgfqpoint{0.993756in}{3.933741in}}%
\pgfpathcurveto{\pgfqpoint{0.985943in}{3.925927in}}{\pgfqpoint{0.981553in}{3.915328in}}{\pgfqpoint{0.981553in}{3.904278in}}%
\pgfpathcurveto{\pgfqpoint{0.981553in}{3.893228in}}{\pgfqpoint{0.985943in}{3.882629in}}{\pgfqpoint{0.993756in}{3.874816in}}%
\pgfpathcurveto{\pgfqpoint{1.001570in}{3.867002in}}{\pgfqpoint{1.012169in}{3.862612in}}{\pgfqpoint{1.023219in}{3.862612in}}%
\pgfpathlineto{\pgfqpoint{1.023219in}{3.862612in}}%
\pgfpathclose%
\pgfusepath{stroke}%
\end{pgfscope}%
\begin{pgfscope}%
\pgfpathrectangle{\pgfqpoint{0.847223in}{0.554012in}}{\pgfqpoint{6.200000in}{4.620000in}}%
\pgfusepath{clip}%
\pgfsetbuttcap%
\pgfsetroundjoin%
\pgfsetlinewidth{1.003750pt}%
\definecolor{currentstroke}{rgb}{1.000000,0.000000,0.000000}%
\pgfsetstrokecolor{currentstroke}%
\pgfsetdash{}{0pt}%
\pgfpathmoveto{\pgfqpoint{1.028552in}{3.834036in}}%
\pgfpathcurveto{\pgfqpoint{1.039603in}{3.834036in}}{\pgfqpoint{1.050202in}{3.838426in}}{\pgfqpoint{1.058015in}{3.846240in}}%
\pgfpathcurveto{\pgfqpoint{1.065829in}{3.854053in}}{\pgfqpoint{1.070219in}{3.864652in}}{\pgfqpoint{1.070219in}{3.875702in}}%
\pgfpathcurveto{\pgfqpoint{1.070219in}{3.886753in}}{\pgfqpoint{1.065829in}{3.897352in}}{\pgfqpoint{1.058015in}{3.905165in}}%
\pgfpathcurveto{\pgfqpoint{1.050202in}{3.912979in}}{\pgfqpoint{1.039603in}{3.917369in}}{\pgfqpoint{1.028552in}{3.917369in}}%
\pgfpathcurveto{\pgfqpoint{1.017502in}{3.917369in}}{\pgfqpoint{1.006903in}{3.912979in}}{\pgfqpoint{0.999090in}{3.905165in}}%
\pgfpathcurveto{\pgfqpoint{0.991276in}{3.897352in}}{\pgfqpoint{0.986886in}{3.886753in}}{\pgfqpoint{0.986886in}{3.875702in}}%
\pgfpathcurveto{\pgfqpoint{0.986886in}{3.864652in}}{\pgfqpoint{0.991276in}{3.854053in}}{\pgfqpoint{0.999090in}{3.846240in}}%
\pgfpathcurveto{\pgfqpoint{1.006903in}{3.838426in}}{\pgfqpoint{1.017502in}{3.834036in}}{\pgfqpoint{1.028552in}{3.834036in}}%
\pgfpathlineto{\pgfqpoint{1.028552in}{3.834036in}}%
\pgfpathclose%
\pgfusepath{stroke}%
\end{pgfscope}%
\begin{pgfscope}%
\pgfpathrectangle{\pgfqpoint{0.847223in}{0.554012in}}{\pgfqpoint{6.200000in}{4.620000in}}%
\pgfusepath{clip}%
\pgfsetbuttcap%
\pgfsetroundjoin%
\pgfsetlinewidth{1.003750pt}%
\definecolor{currentstroke}{rgb}{1.000000,0.000000,0.000000}%
\pgfsetstrokecolor{currentstroke}%
\pgfsetdash{}{0pt}%
\pgfpathmoveto{\pgfqpoint{1.033886in}{3.805889in}}%
\pgfpathcurveto{\pgfqpoint{1.044936in}{3.805889in}}{\pgfqpoint{1.055535in}{3.810279in}}{\pgfqpoint{1.063348in}{3.818093in}}%
\pgfpathcurveto{\pgfqpoint{1.071162in}{3.825907in}}{\pgfqpoint{1.075552in}{3.836506in}}{\pgfqpoint{1.075552in}{3.847556in}}%
\pgfpathcurveto{\pgfqpoint{1.075552in}{3.858606in}}{\pgfqpoint{1.071162in}{3.869205in}}{\pgfqpoint{1.063348in}{3.877019in}}%
\pgfpathcurveto{\pgfqpoint{1.055535in}{3.884832in}}{\pgfqpoint{1.044936in}{3.889223in}}{\pgfqpoint{1.033886in}{3.889223in}}%
\pgfpathcurveto{\pgfqpoint{1.022835in}{3.889223in}}{\pgfqpoint{1.012236in}{3.884832in}}{\pgfqpoint{1.004423in}{3.877019in}}%
\pgfpathcurveto{\pgfqpoint{0.996609in}{3.869205in}}{\pgfqpoint{0.992219in}{3.858606in}}{\pgfqpoint{0.992219in}{3.847556in}}%
\pgfpathcurveto{\pgfqpoint{0.992219in}{3.836506in}}{\pgfqpoint{0.996609in}{3.825907in}}{\pgfqpoint{1.004423in}{3.818093in}}%
\pgfpathcurveto{\pgfqpoint{1.012236in}{3.810279in}}{\pgfqpoint{1.022835in}{3.805889in}}{\pgfqpoint{1.033886in}{3.805889in}}%
\pgfpathlineto{\pgfqpoint{1.033886in}{3.805889in}}%
\pgfpathclose%
\pgfusepath{stroke}%
\end{pgfscope}%
\begin{pgfscope}%
\pgfpathrectangle{\pgfqpoint{0.847223in}{0.554012in}}{\pgfqpoint{6.200000in}{4.620000in}}%
\pgfusepath{clip}%
\pgfsetbuttcap%
\pgfsetroundjoin%
\pgfsetlinewidth{1.003750pt}%
\definecolor{currentstroke}{rgb}{1.000000,0.000000,0.000000}%
\pgfsetstrokecolor{currentstroke}%
\pgfsetdash{}{0pt}%
\pgfpathmoveto{\pgfqpoint{1.039219in}{3.778162in}}%
\pgfpathcurveto{\pgfqpoint{1.050269in}{3.778162in}}{\pgfqpoint{1.060868in}{3.782553in}}{\pgfqpoint{1.068682in}{3.790366in}}%
\pgfpathcurveto{\pgfqpoint{1.076495in}{3.798180in}}{\pgfqpoint{1.080885in}{3.808779in}}{\pgfqpoint{1.080885in}{3.819829in}}%
\pgfpathcurveto{\pgfqpoint{1.080885in}{3.830879in}}{\pgfqpoint{1.076495in}{3.841478in}}{\pgfqpoint{1.068682in}{3.849292in}}%
\pgfpathcurveto{\pgfqpoint{1.060868in}{3.857105in}}{\pgfqpoint{1.050269in}{3.861496in}}{\pgfqpoint{1.039219in}{3.861496in}}%
\pgfpathcurveto{\pgfqpoint{1.028169in}{3.861496in}}{\pgfqpoint{1.017570in}{3.857105in}}{\pgfqpoint{1.009756in}{3.849292in}}%
\pgfpathcurveto{\pgfqpoint{1.001942in}{3.841478in}}{\pgfqpoint{0.997552in}{3.830879in}}{\pgfqpoint{0.997552in}{3.819829in}}%
\pgfpathcurveto{\pgfqpoint{0.997552in}{3.808779in}}{\pgfqpoint{1.001942in}{3.798180in}}{\pgfqpoint{1.009756in}{3.790366in}}%
\pgfpathcurveto{\pgfqpoint{1.017570in}{3.782553in}}{\pgfqpoint{1.028169in}{3.778162in}}{\pgfqpoint{1.039219in}{3.778162in}}%
\pgfpathlineto{\pgfqpoint{1.039219in}{3.778162in}}%
\pgfpathclose%
\pgfusepath{stroke}%
\end{pgfscope}%
\begin{pgfscope}%
\pgfpathrectangle{\pgfqpoint{0.847223in}{0.554012in}}{\pgfqpoint{6.200000in}{4.620000in}}%
\pgfusepath{clip}%
\pgfsetbuttcap%
\pgfsetroundjoin%
\pgfsetlinewidth{1.003750pt}%
\definecolor{currentstroke}{rgb}{1.000000,0.000000,0.000000}%
\pgfsetstrokecolor{currentstroke}%
\pgfsetdash{}{0pt}%
\pgfpathmoveto{\pgfqpoint{1.044552in}{3.750846in}}%
\pgfpathcurveto{\pgfqpoint{1.055602in}{3.750846in}}{\pgfqpoint{1.066201in}{3.755236in}}{\pgfqpoint{1.074015in}{3.763050in}}%
\pgfpathcurveto{\pgfqpoint{1.081828in}{3.770863in}}{\pgfqpoint{1.086219in}{3.781462in}}{\pgfqpoint{1.086219in}{3.792512in}}%
\pgfpathcurveto{\pgfqpoint{1.086219in}{3.803563in}}{\pgfqpoint{1.081828in}{3.814162in}}{\pgfqpoint{1.074015in}{3.821975in}}%
\pgfpathcurveto{\pgfqpoint{1.066201in}{3.829789in}}{\pgfqpoint{1.055602in}{3.834179in}}{\pgfqpoint{1.044552in}{3.834179in}}%
\pgfpathcurveto{\pgfqpoint{1.033502in}{3.834179in}}{\pgfqpoint{1.022903in}{3.829789in}}{\pgfqpoint{1.015089in}{3.821975in}}%
\pgfpathcurveto{\pgfqpoint{1.007276in}{3.814162in}}{\pgfqpoint{1.002885in}{3.803563in}}{\pgfqpoint{1.002885in}{3.792512in}}%
\pgfpathcurveto{\pgfqpoint{1.002885in}{3.781462in}}{\pgfqpoint{1.007276in}{3.770863in}}{\pgfqpoint{1.015089in}{3.763050in}}%
\pgfpathcurveto{\pgfqpoint{1.022903in}{3.755236in}}{\pgfqpoint{1.033502in}{3.750846in}}{\pgfqpoint{1.044552in}{3.750846in}}%
\pgfpathlineto{\pgfqpoint{1.044552in}{3.750846in}}%
\pgfpathclose%
\pgfusepath{stroke}%
\end{pgfscope}%
\begin{pgfscope}%
\pgfpathrectangle{\pgfqpoint{0.847223in}{0.554012in}}{\pgfqpoint{6.200000in}{4.620000in}}%
\pgfusepath{clip}%
\pgfsetbuttcap%
\pgfsetroundjoin%
\pgfsetlinewidth{1.003750pt}%
\definecolor{currentstroke}{rgb}{1.000000,0.000000,0.000000}%
\pgfsetstrokecolor{currentstroke}%
\pgfsetdash{}{0pt}%
\pgfpathmoveto{\pgfqpoint{1.049885in}{3.723931in}}%
\pgfpathcurveto{\pgfqpoint{1.060935in}{3.723931in}}{\pgfqpoint{1.071534in}{3.728321in}}{\pgfqpoint{1.079348in}{3.736135in}}%
\pgfpathcurveto{\pgfqpoint{1.087162in}{3.743948in}}{\pgfqpoint{1.091552in}{3.754547in}}{\pgfqpoint{1.091552in}{3.765597in}}%
\pgfpathcurveto{\pgfqpoint{1.091552in}{3.776647in}}{\pgfqpoint{1.087162in}{3.787246in}}{\pgfqpoint{1.079348in}{3.795060in}}%
\pgfpathcurveto{\pgfqpoint{1.071534in}{3.802874in}}{\pgfqpoint{1.060935in}{3.807264in}}{\pgfqpoint{1.049885in}{3.807264in}}%
\pgfpathcurveto{\pgfqpoint{1.038835in}{3.807264in}}{\pgfqpoint{1.028236in}{3.802874in}}{\pgfqpoint{1.020422in}{3.795060in}}%
\pgfpathcurveto{\pgfqpoint{1.012609in}{3.787246in}}{\pgfqpoint{1.008219in}{3.776647in}}{\pgfqpoint{1.008219in}{3.765597in}}%
\pgfpathcurveto{\pgfqpoint{1.008219in}{3.754547in}}{\pgfqpoint{1.012609in}{3.743948in}}{\pgfqpoint{1.020422in}{3.736135in}}%
\pgfpathcurveto{\pgfqpoint{1.028236in}{3.728321in}}{\pgfqpoint{1.038835in}{3.723931in}}{\pgfqpoint{1.049885in}{3.723931in}}%
\pgfpathlineto{\pgfqpoint{1.049885in}{3.723931in}}%
\pgfpathclose%
\pgfusepath{stroke}%
\end{pgfscope}%
\begin{pgfscope}%
\pgfpathrectangle{\pgfqpoint{0.847223in}{0.554012in}}{\pgfqpoint{6.200000in}{4.620000in}}%
\pgfusepath{clip}%
\pgfsetbuttcap%
\pgfsetroundjoin%
\pgfsetlinewidth{1.003750pt}%
\definecolor{currentstroke}{rgb}{1.000000,0.000000,0.000000}%
\pgfsetstrokecolor{currentstroke}%
\pgfsetdash{}{0pt}%
\pgfpathmoveto{\pgfqpoint{1.055218in}{3.697408in}}%
\pgfpathcurveto{\pgfqpoint{1.066269in}{3.697408in}}{\pgfqpoint{1.076868in}{3.701798in}}{\pgfqpoint{1.084681in}{3.709612in}}%
\pgfpathcurveto{\pgfqpoint{1.092495in}{3.717425in}}{\pgfqpoint{1.096885in}{3.728025in}}{\pgfqpoint{1.096885in}{3.739075in}}%
\pgfpathcurveto{\pgfqpoint{1.096885in}{3.750125in}}{\pgfqpoint{1.092495in}{3.760724in}}{\pgfqpoint{1.084681in}{3.768537in}}%
\pgfpathcurveto{\pgfqpoint{1.076868in}{3.776351in}}{\pgfqpoint{1.066269in}{3.780741in}}{\pgfqpoint{1.055218in}{3.780741in}}%
\pgfpathcurveto{\pgfqpoint{1.044168in}{3.780741in}}{\pgfqpoint{1.033569in}{3.776351in}}{\pgfqpoint{1.025756in}{3.768537in}}%
\pgfpathcurveto{\pgfqpoint{1.017942in}{3.760724in}}{\pgfqpoint{1.013552in}{3.750125in}}{\pgfqpoint{1.013552in}{3.739075in}}%
\pgfpathcurveto{\pgfqpoint{1.013552in}{3.728025in}}{\pgfqpoint{1.017942in}{3.717425in}}{\pgfqpoint{1.025756in}{3.709612in}}%
\pgfpathcurveto{\pgfqpoint{1.033569in}{3.701798in}}{\pgfqpoint{1.044168in}{3.697408in}}{\pgfqpoint{1.055218in}{3.697408in}}%
\pgfpathlineto{\pgfqpoint{1.055218in}{3.697408in}}%
\pgfpathclose%
\pgfusepath{stroke}%
\end{pgfscope}%
\begin{pgfscope}%
\pgfpathrectangle{\pgfqpoint{0.847223in}{0.554012in}}{\pgfqpoint{6.200000in}{4.620000in}}%
\pgfusepath{clip}%
\pgfsetbuttcap%
\pgfsetroundjoin%
\pgfsetlinewidth{1.003750pt}%
\definecolor{currentstroke}{rgb}{1.000000,0.000000,0.000000}%
\pgfsetstrokecolor{currentstroke}%
\pgfsetdash{}{0pt}%
\pgfpathmoveto{\pgfqpoint{1.060552in}{3.671269in}}%
\pgfpathcurveto{\pgfqpoint{1.071602in}{3.671269in}}{\pgfqpoint{1.082201in}{3.675660in}}{\pgfqpoint{1.090014in}{3.683473in}}%
\pgfpathcurveto{\pgfqpoint{1.097828in}{3.691287in}}{\pgfqpoint{1.102218in}{3.701886in}}{\pgfqpoint{1.102218in}{3.712936in}}%
\pgfpathcurveto{\pgfqpoint{1.102218in}{3.723986in}}{\pgfqpoint{1.097828in}{3.734585in}}{\pgfqpoint{1.090014in}{3.742399in}}%
\pgfpathcurveto{\pgfqpoint{1.082201in}{3.750212in}}{\pgfqpoint{1.071602in}{3.754603in}}{\pgfqpoint{1.060552in}{3.754603in}}%
\pgfpathcurveto{\pgfqpoint{1.049502in}{3.754603in}}{\pgfqpoint{1.038903in}{3.750212in}}{\pgfqpoint{1.031089in}{3.742399in}}%
\pgfpathcurveto{\pgfqpoint{1.023275in}{3.734585in}}{\pgfqpoint{1.018885in}{3.723986in}}{\pgfqpoint{1.018885in}{3.712936in}}%
\pgfpathcurveto{\pgfqpoint{1.018885in}{3.701886in}}{\pgfqpoint{1.023275in}{3.691287in}}{\pgfqpoint{1.031089in}{3.683473in}}%
\pgfpathcurveto{\pgfqpoint{1.038903in}{3.675660in}}{\pgfqpoint{1.049502in}{3.671269in}}{\pgfqpoint{1.060552in}{3.671269in}}%
\pgfpathlineto{\pgfqpoint{1.060552in}{3.671269in}}%
\pgfpathclose%
\pgfusepath{stroke}%
\end{pgfscope}%
\begin{pgfscope}%
\pgfpathrectangle{\pgfqpoint{0.847223in}{0.554012in}}{\pgfqpoint{6.200000in}{4.620000in}}%
\pgfusepath{clip}%
\pgfsetbuttcap%
\pgfsetroundjoin%
\pgfsetlinewidth{1.003750pt}%
\definecolor{currentstroke}{rgb}{1.000000,0.000000,0.000000}%
\pgfsetstrokecolor{currentstroke}%
\pgfsetdash{}{0pt}%
\pgfpathmoveto{\pgfqpoint{1.065885in}{3.645507in}}%
\pgfpathcurveto{\pgfqpoint{1.076935in}{3.645507in}}{\pgfqpoint{1.087534in}{3.649897in}}{\pgfqpoint{1.095348in}{3.657710in}}%
\pgfpathcurveto{\pgfqpoint{1.103161in}{3.665524in}}{\pgfqpoint{1.107552in}{3.676123in}}{\pgfqpoint{1.107552in}{3.687173in}}%
\pgfpathcurveto{\pgfqpoint{1.107552in}{3.698223in}}{\pgfqpoint{1.103161in}{3.708822in}}{\pgfqpoint{1.095348in}{3.716636in}}%
\pgfpathcurveto{\pgfqpoint{1.087534in}{3.724450in}}{\pgfqpoint{1.076935in}{3.728840in}}{\pgfqpoint{1.065885in}{3.728840in}}%
\pgfpathcurveto{\pgfqpoint{1.054835in}{3.728840in}}{\pgfqpoint{1.044236in}{3.724450in}}{\pgfqpoint{1.036422in}{3.716636in}}%
\pgfpathcurveto{\pgfqpoint{1.028608in}{3.708822in}}{\pgfqpoint{1.024218in}{3.698223in}}{\pgfqpoint{1.024218in}{3.687173in}}%
\pgfpathcurveto{\pgfqpoint{1.024218in}{3.676123in}}{\pgfqpoint{1.028608in}{3.665524in}}{\pgfqpoint{1.036422in}{3.657710in}}%
\pgfpathcurveto{\pgfqpoint{1.044236in}{3.649897in}}{\pgfqpoint{1.054835in}{3.645507in}}{\pgfqpoint{1.065885in}{3.645507in}}%
\pgfpathlineto{\pgfqpoint{1.065885in}{3.645507in}}%
\pgfpathclose%
\pgfusepath{stroke}%
\end{pgfscope}%
\begin{pgfscope}%
\pgfpathrectangle{\pgfqpoint{0.847223in}{0.554012in}}{\pgfqpoint{6.200000in}{4.620000in}}%
\pgfusepath{clip}%
\pgfsetbuttcap%
\pgfsetroundjoin%
\pgfsetlinewidth{1.003750pt}%
\definecolor{currentstroke}{rgb}{1.000000,0.000000,0.000000}%
\pgfsetstrokecolor{currentstroke}%
\pgfsetdash{}{0pt}%
\pgfpathmoveto{\pgfqpoint{1.071218in}{3.620111in}}%
\pgfpathcurveto{\pgfqpoint{1.082268in}{3.620111in}}{\pgfqpoint{1.092867in}{3.624502in}}{\pgfqpoint{1.100681in}{3.632315in}}%
\pgfpathcurveto{\pgfqpoint{1.108495in}{3.640129in}}{\pgfqpoint{1.112885in}{3.650728in}}{\pgfqpoint{1.112885in}{3.661778in}}%
\pgfpathcurveto{\pgfqpoint{1.112885in}{3.672828in}}{\pgfqpoint{1.108495in}{3.683427in}}{\pgfqpoint{1.100681in}{3.691241in}}%
\pgfpathcurveto{\pgfqpoint{1.092867in}{3.699054in}}{\pgfqpoint{1.082268in}{3.703445in}}{\pgfqpoint{1.071218in}{3.703445in}}%
\pgfpathcurveto{\pgfqpoint{1.060168in}{3.703445in}}{\pgfqpoint{1.049569in}{3.699054in}}{\pgfqpoint{1.041755in}{3.691241in}}%
\pgfpathcurveto{\pgfqpoint{1.033942in}{3.683427in}}{\pgfqpoint{1.029551in}{3.672828in}}{\pgfqpoint{1.029551in}{3.661778in}}%
\pgfpathcurveto{\pgfqpoint{1.029551in}{3.650728in}}{\pgfqpoint{1.033942in}{3.640129in}}{\pgfqpoint{1.041755in}{3.632315in}}%
\pgfpathcurveto{\pgfqpoint{1.049569in}{3.624502in}}{\pgfqpoint{1.060168in}{3.620111in}}{\pgfqpoint{1.071218in}{3.620111in}}%
\pgfpathlineto{\pgfqpoint{1.071218in}{3.620111in}}%
\pgfpathclose%
\pgfusepath{stroke}%
\end{pgfscope}%
\begin{pgfscope}%
\pgfpathrectangle{\pgfqpoint{0.847223in}{0.554012in}}{\pgfqpoint{6.200000in}{4.620000in}}%
\pgfusepath{clip}%
\pgfsetbuttcap%
\pgfsetroundjoin%
\pgfsetlinewidth{1.003750pt}%
\definecolor{currentstroke}{rgb}{1.000000,0.000000,0.000000}%
\pgfsetstrokecolor{currentstroke}%
\pgfsetdash{}{0pt}%
\pgfpathmoveto{\pgfqpoint{1.076551in}{3.595076in}}%
\pgfpathcurveto{\pgfqpoint{1.087601in}{3.595076in}}{\pgfqpoint{1.098200in}{3.599466in}}{\pgfqpoint{1.106014in}{3.607280in}}%
\pgfpathcurveto{\pgfqpoint{1.113828in}{3.615094in}}{\pgfqpoint{1.118218in}{3.625693in}}{\pgfqpoint{1.118218in}{3.636743in}}%
\pgfpathcurveto{\pgfqpoint{1.118218in}{3.647793in}}{\pgfqpoint{1.113828in}{3.658392in}}{\pgfqpoint{1.106014in}{3.666206in}}%
\pgfpathcurveto{\pgfqpoint{1.098200in}{3.674019in}}{\pgfqpoint{1.087601in}{3.678409in}}{\pgfqpoint{1.076551in}{3.678409in}}%
\pgfpathcurveto{\pgfqpoint{1.065501in}{3.678409in}}{\pgfqpoint{1.054902in}{3.674019in}}{\pgfqpoint{1.047089in}{3.666206in}}%
\pgfpathcurveto{\pgfqpoint{1.039275in}{3.658392in}}{\pgfqpoint{1.034885in}{3.647793in}}{\pgfqpoint{1.034885in}{3.636743in}}%
\pgfpathcurveto{\pgfqpoint{1.034885in}{3.625693in}}{\pgfqpoint{1.039275in}{3.615094in}}{\pgfqpoint{1.047089in}{3.607280in}}%
\pgfpathcurveto{\pgfqpoint{1.054902in}{3.599466in}}{\pgfqpoint{1.065501in}{3.595076in}}{\pgfqpoint{1.076551in}{3.595076in}}%
\pgfpathlineto{\pgfqpoint{1.076551in}{3.595076in}}%
\pgfpathclose%
\pgfusepath{stroke}%
\end{pgfscope}%
\begin{pgfscope}%
\pgfpathrectangle{\pgfqpoint{0.847223in}{0.554012in}}{\pgfqpoint{6.200000in}{4.620000in}}%
\pgfusepath{clip}%
\pgfsetbuttcap%
\pgfsetroundjoin%
\pgfsetlinewidth{1.003750pt}%
\definecolor{currentstroke}{rgb}{1.000000,0.000000,0.000000}%
\pgfsetstrokecolor{currentstroke}%
\pgfsetdash{}{0pt}%
\pgfpathmoveto{\pgfqpoint{1.081885in}{3.570393in}}%
\pgfpathcurveto{\pgfqpoint{1.092935in}{3.570393in}}{\pgfqpoint{1.103534in}{3.574783in}}{\pgfqpoint{1.111347in}{3.582597in}}%
\pgfpathcurveto{\pgfqpoint{1.119161in}{3.590411in}}{\pgfqpoint{1.123551in}{3.601010in}}{\pgfqpoint{1.123551in}{3.612060in}}%
\pgfpathcurveto{\pgfqpoint{1.123551in}{3.623110in}}{\pgfqpoint{1.119161in}{3.633709in}}{\pgfqpoint{1.111347in}{3.641523in}}%
\pgfpathcurveto{\pgfqpoint{1.103534in}{3.649336in}}{\pgfqpoint{1.092935in}{3.653726in}}{\pgfqpoint{1.081885in}{3.653726in}}%
\pgfpathcurveto{\pgfqpoint{1.070834in}{3.653726in}}{\pgfqpoint{1.060235in}{3.649336in}}{\pgfqpoint{1.052422in}{3.641523in}}%
\pgfpathcurveto{\pgfqpoint{1.044608in}{3.633709in}}{\pgfqpoint{1.040218in}{3.623110in}}{\pgfqpoint{1.040218in}{3.612060in}}%
\pgfpathcurveto{\pgfqpoint{1.040218in}{3.601010in}}{\pgfqpoint{1.044608in}{3.590411in}}{\pgfqpoint{1.052422in}{3.582597in}}%
\pgfpathcurveto{\pgfqpoint{1.060235in}{3.574783in}}{\pgfqpoint{1.070834in}{3.570393in}}{\pgfqpoint{1.081885in}{3.570393in}}%
\pgfpathlineto{\pgfqpoint{1.081885in}{3.570393in}}%
\pgfpathclose%
\pgfusepath{stroke}%
\end{pgfscope}%
\begin{pgfscope}%
\pgfpathrectangle{\pgfqpoint{0.847223in}{0.554012in}}{\pgfqpoint{6.200000in}{4.620000in}}%
\pgfusepath{clip}%
\pgfsetbuttcap%
\pgfsetroundjoin%
\pgfsetlinewidth{1.003750pt}%
\definecolor{currentstroke}{rgb}{1.000000,0.000000,0.000000}%
\pgfsetstrokecolor{currentstroke}%
\pgfsetdash{}{0pt}%
\pgfpathmoveto{\pgfqpoint{1.087218in}{3.546055in}}%
\pgfpathcurveto{\pgfqpoint{1.098268in}{3.546055in}}{\pgfqpoint{1.108867in}{3.550445in}}{\pgfqpoint{1.116681in}{3.558259in}}%
\pgfpathcurveto{\pgfqpoint{1.124494in}{3.566072in}}{\pgfqpoint{1.128884in}{3.576672in}}{\pgfqpoint{1.128884in}{3.587722in}}%
\pgfpathcurveto{\pgfqpoint{1.128884in}{3.598772in}}{\pgfqpoint{1.124494in}{3.609371in}}{\pgfqpoint{1.116681in}{3.617184in}}%
\pgfpathcurveto{\pgfqpoint{1.108867in}{3.624998in}}{\pgfqpoint{1.098268in}{3.629388in}}{\pgfqpoint{1.087218in}{3.629388in}}%
\pgfpathcurveto{\pgfqpoint{1.076168in}{3.629388in}}{\pgfqpoint{1.065569in}{3.624998in}}{\pgfqpoint{1.057755in}{3.617184in}}%
\pgfpathcurveto{\pgfqpoint{1.049941in}{3.609371in}}{\pgfqpoint{1.045551in}{3.598772in}}{\pgfqpoint{1.045551in}{3.587722in}}%
\pgfpathcurveto{\pgfqpoint{1.045551in}{3.576672in}}{\pgfqpoint{1.049941in}{3.566072in}}{\pgfqpoint{1.057755in}{3.558259in}}%
\pgfpathcurveto{\pgfqpoint{1.065569in}{3.550445in}}{\pgfqpoint{1.076168in}{3.546055in}}{\pgfqpoint{1.087218in}{3.546055in}}%
\pgfpathlineto{\pgfqpoint{1.087218in}{3.546055in}}%
\pgfpathclose%
\pgfusepath{stroke}%
\end{pgfscope}%
\begin{pgfscope}%
\pgfpathrectangle{\pgfqpoint{0.847223in}{0.554012in}}{\pgfqpoint{6.200000in}{4.620000in}}%
\pgfusepath{clip}%
\pgfsetbuttcap%
\pgfsetroundjoin%
\pgfsetlinewidth{1.003750pt}%
\definecolor{currentstroke}{rgb}{1.000000,0.000000,0.000000}%
\pgfsetstrokecolor{currentstroke}%
\pgfsetdash{}{0pt}%
\pgfpathmoveto{\pgfqpoint{1.092551in}{3.522055in}}%
\pgfpathcurveto{\pgfqpoint{1.103601in}{3.522055in}}{\pgfqpoint{1.114200in}{3.526445in}}{\pgfqpoint{1.122014in}{3.534258in}}%
\pgfpathcurveto{\pgfqpoint{1.129827in}{3.542072in}}{\pgfqpoint{1.134218in}{3.552671in}}{\pgfqpoint{1.134218in}{3.563721in}}%
\pgfpathcurveto{\pgfqpoint{1.134218in}{3.574771in}}{\pgfqpoint{1.129827in}{3.585370in}}{\pgfqpoint{1.122014in}{3.593184in}}%
\pgfpathcurveto{\pgfqpoint{1.114200in}{3.600998in}}{\pgfqpoint{1.103601in}{3.605388in}}{\pgfqpoint{1.092551in}{3.605388in}}%
\pgfpathcurveto{\pgfqpoint{1.081501in}{3.605388in}}{\pgfqpoint{1.070902in}{3.600998in}}{\pgfqpoint{1.063088in}{3.593184in}}%
\pgfpathcurveto{\pgfqpoint{1.055275in}{3.585370in}}{\pgfqpoint{1.050884in}{3.574771in}}{\pgfqpoint{1.050884in}{3.563721in}}%
\pgfpathcurveto{\pgfqpoint{1.050884in}{3.552671in}}{\pgfqpoint{1.055275in}{3.542072in}}{\pgfqpoint{1.063088in}{3.534258in}}%
\pgfpathcurveto{\pgfqpoint{1.070902in}{3.526445in}}{\pgfqpoint{1.081501in}{3.522055in}}{\pgfqpoint{1.092551in}{3.522055in}}%
\pgfpathlineto{\pgfqpoint{1.092551in}{3.522055in}}%
\pgfpathclose%
\pgfusepath{stroke}%
\end{pgfscope}%
\begin{pgfscope}%
\pgfpathrectangle{\pgfqpoint{0.847223in}{0.554012in}}{\pgfqpoint{6.200000in}{4.620000in}}%
\pgfusepath{clip}%
\pgfsetbuttcap%
\pgfsetroundjoin%
\pgfsetlinewidth{1.003750pt}%
\definecolor{currentstroke}{rgb}{1.000000,0.000000,0.000000}%
\pgfsetstrokecolor{currentstroke}%
\pgfsetdash{}{0pt}%
\pgfpathmoveto{\pgfqpoint{1.097884in}{3.498385in}}%
\pgfpathcurveto{\pgfqpoint{1.108934in}{3.498385in}}{\pgfqpoint{1.119533in}{3.502775in}}{\pgfqpoint{1.127347in}{3.510589in}}%
\pgfpathcurveto{\pgfqpoint{1.135161in}{3.518402in}}{\pgfqpoint{1.139551in}{3.529002in}}{\pgfqpoint{1.139551in}{3.540052in}}%
\pgfpathcurveto{\pgfqpoint{1.139551in}{3.551102in}}{\pgfqpoint{1.135161in}{3.561701in}}{\pgfqpoint{1.127347in}{3.569514in}}%
\pgfpathcurveto{\pgfqpoint{1.119533in}{3.577328in}}{\pgfqpoint{1.108934in}{3.581718in}}{\pgfqpoint{1.097884in}{3.581718in}}%
\pgfpathcurveto{\pgfqpoint{1.086834in}{3.581718in}}{\pgfqpoint{1.076235in}{3.577328in}}{\pgfqpoint{1.068421in}{3.569514in}}%
\pgfpathcurveto{\pgfqpoint{1.060608in}{3.561701in}}{\pgfqpoint{1.056218in}{3.551102in}}{\pgfqpoint{1.056218in}{3.540052in}}%
\pgfpathcurveto{\pgfqpoint{1.056218in}{3.529002in}}{\pgfqpoint{1.060608in}{3.518402in}}{\pgfqpoint{1.068421in}{3.510589in}}%
\pgfpathcurveto{\pgfqpoint{1.076235in}{3.502775in}}{\pgfqpoint{1.086834in}{3.498385in}}{\pgfqpoint{1.097884in}{3.498385in}}%
\pgfpathlineto{\pgfqpoint{1.097884in}{3.498385in}}%
\pgfpathclose%
\pgfusepath{stroke}%
\end{pgfscope}%
\begin{pgfscope}%
\pgfpathrectangle{\pgfqpoint{0.847223in}{0.554012in}}{\pgfqpoint{6.200000in}{4.620000in}}%
\pgfusepath{clip}%
\pgfsetbuttcap%
\pgfsetroundjoin%
\pgfsetlinewidth{1.003750pt}%
\definecolor{currentstroke}{rgb}{1.000000,0.000000,0.000000}%
\pgfsetstrokecolor{currentstroke}%
\pgfsetdash{}{0pt}%
\pgfpathmoveto{\pgfqpoint{1.103217in}{3.475039in}}%
\pgfpathcurveto{\pgfqpoint{1.114268in}{3.475039in}}{\pgfqpoint{1.124867in}{3.479430in}}{\pgfqpoint{1.132680in}{3.487243in}}%
\pgfpathcurveto{\pgfqpoint{1.140494in}{3.495057in}}{\pgfqpoint{1.144884in}{3.505656in}}{\pgfqpoint{1.144884in}{3.516706in}}%
\pgfpathcurveto{\pgfqpoint{1.144884in}{3.527756in}}{\pgfqpoint{1.140494in}{3.538355in}}{\pgfqpoint{1.132680in}{3.546169in}}%
\pgfpathcurveto{\pgfqpoint{1.124867in}{3.553982in}}{\pgfqpoint{1.114268in}{3.558373in}}{\pgfqpoint{1.103217in}{3.558373in}}%
\pgfpathcurveto{\pgfqpoint{1.092167in}{3.558373in}}{\pgfqpoint{1.081568in}{3.553982in}}{\pgfqpoint{1.073755in}{3.546169in}}%
\pgfpathcurveto{\pgfqpoint{1.065941in}{3.538355in}}{\pgfqpoint{1.061551in}{3.527756in}}{\pgfqpoint{1.061551in}{3.516706in}}%
\pgfpathcurveto{\pgfqpoint{1.061551in}{3.505656in}}{\pgfqpoint{1.065941in}{3.495057in}}{\pgfqpoint{1.073755in}{3.487243in}}%
\pgfpathcurveto{\pgfqpoint{1.081568in}{3.479430in}}{\pgfqpoint{1.092167in}{3.475039in}}{\pgfqpoint{1.103217in}{3.475039in}}%
\pgfpathlineto{\pgfqpoint{1.103217in}{3.475039in}}%
\pgfpathclose%
\pgfusepath{stroke}%
\end{pgfscope}%
\begin{pgfscope}%
\pgfpathrectangle{\pgfqpoint{0.847223in}{0.554012in}}{\pgfqpoint{6.200000in}{4.620000in}}%
\pgfusepath{clip}%
\pgfsetbuttcap%
\pgfsetroundjoin%
\pgfsetlinewidth{1.003750pt}%
\definecolor{currentstroke}{rgb}{1.000000,0.000000,0.000000}%
\pgfsetstrokecolor{currentstroke}%
\pgfsetdash{}{0pt}%
\pgfpathmoveto{\pgfqpoint{1.108551in}{3.452011in}}%
\pgfpathcurveto{\pgfqpoint{1.119601in}{3.452011in}}{\pgfqpoint{1.130200in}{3.456401in}}{\pgfqpoint{1.138013in}{3.464215in}}%
\pgfpathcurveto{\pgfqpoint{1.145827in}{3.472028in}}{\pgfqpoint{1.150217in}{3.482627in}}{\pgfqpoint{1.150217in}{3.493678in}}%
\pgfpathcurveto{\pgfqpoint{1.150217in}{3.504728in}}{\pgfqpoint{1.145827in}{3.515327in}}{\pgfqpoint{1.138013in}{3.523140in}}%
\pgfpathcurveto{\pgfqpoint{1.130200in}{3.530954in}}{\pgfqpoint{1.119601in}{3.535344in}}{\pgfqpoint{1.108551in}{3.535344in}}%
\pgfpathcurveto{\pgfqpoint{1.097500in}{3.535344in}}{\pgfqpoint{1.086901in}{3.530954in}}{\pgfqpoint{1.079088in}{3.523140in}}%
\pgfpathcurveto{\pgfqpoint{1.071274in}{3.515327in}}{\pgfqpoint{1.066884in}{3.504728in}}{\pgfqpoint{1.066884in}{3.493678in}}%
\pgfpathcurveto{\pgfqpoint{1.066884in}{3.482627in}}{\pgfqpoint{1.071274in}{3.472028in}}{\pgfqpoint{1.079088in}{3.464215in}}%
\pgfpathcurveto{\pgfqpoint{1.086901in}{3.456401in}}{\pgfqpoint{1.097500in}{3.452011in}}{\pgfqpoint{1.108551in}{3.452011in}}%
\pgfpathlineto{\pgfqpoint{1.108551in}{3.452011in}}%
\pgfpathclose%
\pgfusepath{stroke}%
\end{pgfscope}%
\begin{pgfscope}%
\pgfpathrectangle{\pgfqpoint{0.847223in}{0.554012in}}{\pgfqpoint{6.200000in}{4.620000in}}%
\pgfusepath{clip}%
\pgfsetbuttcap%
\pgfsetroundjoin%
\pgfsetlinewidth{1.003750pt}%
\definecolor{currentstroke}{rgb}{1.000000,0.000000,0.000000}%
\pgfsetstrokecolor{currentstroke}%
\pgfsetdash{}{0pt}%
\pgfpathmoveto{\pgfqpoint{1.113884in}{3.429294in}}%
\pgfpathcurveto{\pgfqpoint{1.124934in}{3.429294in}}{\pgfqpoint{1.135533in}{3.433684in}}{\pgfqpoint{1.143347in}{3.441497in}}%
\pgfpathcurveto{\pgfqpoint{1.151160in}{3.449311in}}{\pgfqpoint{1.155551in}{3.459910in}}{\pgfqpoint{1.155551in}{3.470960in}}%
\pgfpathcurveto{\pgfqpoint{1.155551in}{3.482010in}}{\pgfqpoint{1.151160in}{3.492609in}}{\pgfqpoint{1.143347in}{3.500423in}}%
\pgfpathcurveto{\pgfqpoint{1.135533in}{3.508237in}}{\pgfqpoint{1.124934in}{3.512627in}}{\pgfqpoint{1.113884in}{3.512627in}}%
\pgfpathcurveto{\pgfqpoint{1.102834in}{3.512627in}}{\pgfqpoint{1.092235in}{3.508237in}}{\pgfqpoint{1.084421in}{3.500423in}}%
\pgfpathcurveto{\pgfqpoint{1.076607in}{3.492609in}}{\pgfqpoint{1.072217in}{3.482010in}}{\pgfqpoint{1.072217in}{3.470960in}}%
\pgfpathcurveto{\pgfqpoint{1.072217in}{3.459910in}}{\pgfqpoint{1.076607in}{3.449311in}}{\pgfqpoint{1.084421in}{3.441497in}}%
\pgfpathcurveto{\pgfqpoint{1.092235in}{3.433684in}}{\pgfqpoint{1.102834in}{3.429294in}}{\pgfqpoint{1.113884in}{3.429294in}}%
\pgfpathlineto{\pgfqpoint{1.113884in}{3.429294in}}%
\pgfpathclose%
\pgfusepath{stroke}%
\end{pgfscope}%
\begin{pgfscope}%
\pgfpathrectangle{\pgfqpoint{0.847223in}{0.554012in}}{\pgfqpoint{6.200000in}{4.620000in}}%
\pgfusepath{clip}%
\pgfsetbuttcap%
\pgfsetroundjoin%
\pgfsetlinewidth{1.003750pt}%
\definecolor{currentstroke}{rgb}{1.000000,0.000000,0.000000}%
\pgfsetstrokecolor{currentstroke}%
\pgfsetdash{}{0pt}%
\pgfpathmoveto{\pgfqpoint{1.119217in}{3.406881in}}%
\pgfpathcurveto{\pgfqpoint{1.130267in}{3.406881in}}{\pgfqpoint{1.140866in}{3.411271in}}{\pgfqpoint{1.148680in}{3.419085in}}%
\pgfpathcurveto{\pgfqpoint{1.156493in}{3.426898in}}{\pgfqpoint{1.160884in}{3.437497in}}{\pgfqpoint{1.160884in}{3.448547in}}%
\pgfpathcurveto{\pgfqpoint{1.160884in}{3.459598in}}{\pgfqpoint{1.156493in}{3.470197in}}{\pgfqpoint{1.148680in}{3.478010in}}%
\pgfpathcurveto{\pgfqpoint{1.140866in}{3.485824in}}{\pgfqpoint{1.130267in}{3.490214in}}{\pgfqpoint{1.119217in}{3.490214in}}%
\pgfpathcurveto{\pgfqpoint{1.108167in}{3.490214in}}{\pgfqpoint{1.097568in}{3.485824in}}{\pgfqpoint{1.089754in}{3.478010in}}%
\pgfpathcurveto{\pgfqpoint{1.081941in}{3.470197in}}{\pgfqpoint{1.077550in}{3.459598in}}{\pgfqpoint{1.077550in}{3.448547in}}%
\pgfpathcurveto{\pgfqpoint{1.077550in}{3.437497in}}{\pgfqpoint{1.081941in}{3.426898in}}{\pgfqpoint{1.089754in}{3.419085in}}%
\pgfpathcurveto{\pgfqpoint{1.097568in}{3.411271in}}{\pgfqpoint{1.108167in}{3.406881in}}{\pgfqpoint{1.119217in}{3.406881in}}%
\pgfpathlineto{\pgfqpoint{1.119217in}{3.406881in}}%
\pgfpathclose%
\pgfusepath{stroke}%
\end{pgfscope}%
\begin{pgfscope}%
\pgfpathrectangle{\pgfqpoint{0.847223in}{0.554012in}}{\pgfqpoint{6.200000in}{4.620000in}}%
\pgfusepath{clip}%
\pgfsetbuttcap%
\pgfsetroundjoin%
\pgfsetlinewidth{1.003750pt}%
\definecolor{currentstroke}{rgb}{1.000000,0.000000,0.000000}%
\pgfsetstrokecolor{currentstroke}%
\pgfsetdash{}{0pt}%
\pgfpathmoveto{\pgfqpoint{1.124550in}{3.384767in}}%
\pgfpathcurveto{\pgfqpoint{1.135600in}{3.384767in}}{\pgfqpoint{1.146199in}{3.389157in}}{\pgfqpoint{1.154013in}{3.396971in}}%
\pgfpathcurveto{\pgfqpoint{1.161827in}{3.404784in}}{\pgfqpoint{1.166217in}{3.415383in}}{\pgfqpoint{1.166217in}{3.426433in}}%
\pgfpathcurveto{\pgfqpoint{1.166217in}{3.437483in}}{\pgfqpoint{1.161827in}{3.448082in}}{\pgfqpoint{1.154013in}{3.455896in}}%
\pgfpathcurveto{\pgfqpoint{1.146199in}{3.463710in}}{\pgfqpoint{1.135600in}{3.468100in}}{\pgfqpoint{1.124550in}{3.468100in}}%
\pgfpathcurveto{\pgfqpoint{1.113500in}{3.468100in}}{\pgfqpoint{1.102901in}{3.463710in}}{\pgfqpoint{1.095087in}{3.455896in}}%
\pgfpathcurveto{\pgfqpoint{1.087274in}{3.448082in}}{\pgfqpoint{1.082884in}{3.437483in}}{\pgfqpoint{1.082884in}{3.426433in}}%
\pgfpathcurveto{\pgfqpoint{1.082884in}{3.415383in}}{\pgfqpoint{1.087274in}{3.404784in}}{\pgfqpoint{1.095087in}{3.396971in}}%
\pgfpathcurveto{\pgfqpoint{1.102901in}{3.389157in}}{\pgfqpoint{1.113500in}{3.384767in}}{\pgfqpoint{1.124550in}{3.384767in}}%
\pgfpathlineto{\pgfqpoint{1.124550in}{3.384767in}}%
\pgfpathclose%
\pgfusepath{stroke}%
\end{pgfscope}%
\begin{pgfscope}%
\pgfpathrectangle{\pgfqpoint{0.847223in}{0.554012in}}{\pgfqpoint{6.200000in}{4.620000in}}%
\pgfusepath{clip}%
\pgfsetbuttcap%
\pgfsetroundjoin%
\pgfsetlinewidth{1.003750pt}%
\definecolor{currentstroke}{rgb}{1.000000,0.000000,0.000000}%
\pgfsetstrokecolor{currentstroke}%
\pgfsetdash{}{0pt}%
\pgfpathmoveto{\pgfqpoint{1.129883in}{3.362945in}}%
\pgfpathcurveto{\pgfqpoint{1.140934in}{3.362945in}}{\pgfqpoint{1.151533in}{3.367335in}}{\pgfqpoint{1.159346in}{3.375149in}}%
\pgfpathcurveto{\pgfqpoint{1.167160in}{3.382963in}}{\pgfqpoint{1.171550in}{3.393562in}}{\pgfqpoint{1.171550in}{3.404612in}}%
\pgfpathcurveto{\pgfqpoint{1.171550in}{3.415662in}}{\pgfqpoint{1.167160in}{3.426261in}}{\pgfqpoint{1.159346in}{3.434075in}}%
\pgfpathcurveto{\pgfqpoint{1.151533in}{3.441888in}}{\pgfqpoint{1.140934in}{3.446278in}}{\pgfqpoint{1.129883in}{3.446278in}}%
\pgfpathcurveto{\pgfqpoint{1.118833in}{3.446278in}}{\pgfqpoint{1.108234in}{3.441888in}}{\pgfqpoint{1.100421in}{3.434075in}}%
\pgfpathcurveto{\pgfqpoint{1.092607in}{3.426261in}}{\pgfqpoint{1.088217in}{3.415662in}}{\pgfqpoint{1.088217in}{3.404612in}}%
\pgfpathcurveto{\pgfqpoint{1.088217in}{3.393562in}}{\pgfqpoint{1.092607in}{3.382963in}}{\pgfqpoint{1.100421in}{3.375149in}}%
\pgfpathcurveto{\pgfqpoint{1.108234in}{3.367335in}}{\pgfqpoint{1.118833in}{3.362945in}}{\pgfqpoint{1.129883in}{3.362945in}}%
\pgfpathlineto{\pgfqpoint{1.129883in}{3.362945in}}%
\pgfpathclose%
\pgfusepath{stroke}%
\end{pgfscope}%
\begin{pgfscope}%
\pgfpathrectangle{\pgfqpoint{0.847223in}{0.554012in}}{\pgfqpoint{6.200000in}{4.620000in}}%
\pgfusepath{clip}%
\pgfsetbuttcap%
\pgfsetroundjoin%
\pgfsetlinewidth{1.003750pt}%
\definecolor{currentstroke}{rgb}{1.000000,0.000000,0.000000}%
\pgfsetstrokecolor{currentstroke}%
\pgfsetdash{}{0pt}%
\pgfpathmoveto{\pgfqpoint{1.135217in}{3.341410in}}%
\pgfpathcurveto{\pgfqpoint{1.146267in}{3.341410in}}{\pgfqpoint{1.156866in}{3.345801in}}{\pgfqpoint{1.164679in}{3.353614in}}%
\pgfpathcurveto{\pgfqpoint{1.172493in}{3.361428in}}{\pgfqpoint{1.176883in}{3.372027in}}{\pgfqpoint{1.176883in}{3.383077in}}%
\pgfpathcurveto{\pgfqpoint{1.176883in}{3.394127in}}{\pgfqpoint{1.172493in}{3.404726in}}{\pgfqpoint{1.164679in}{3.412540in}}%
\pgfpathcurveto{\pgfqpoint{1.156866in}{3.420354in}}{\pgfqpoint{1.146267in}{3.424744in}}{\pgfqpoint{1.135217in}{3.424744in}}%
\pgfpathcurveto{\pgfqpoint{1.124167in}{3.424744in}}{\pgfqpoint{1.113568in}{3.420354in}}{\pgfqpoint{1.105754in}{3.412540in}}%
\pgfpathcurveto{\pgfqpoint{1.097940in}{3.404726in}}{\pgfqpoint{1.093550in}{3.394127in}}{\pgfqpoint{1.093550in}{3.383077in}}%
\pgfpathcurveto{\pgfqpoint{1.093550in}{3.372027in}}{\pgfqpoint{1.097940in}{3.361428in}}{\pgfqpoint{1.105754in}{3.353614in}}%
\pgfpathcurveto{\pgfqpoint{1.113568in}{3.345801in}}{\pgfqpoint{1.124167in}{3.341410in}}{\pgfqpoint{1.135217in}{3.341410in}}%
\pgfpathlineto{\pgfqpoint{1.135217in}{3.341410in}}%
\pgfpathclose%
\pgfusepath{stroke}%
\end{pgfscope}%
\begin{pgfscope}%
\pgfpathrectangle{\pgfqpoint{0.847223in}{0.554012in}}{\pgfqpoint{6.200000in}{4.620000in}}%
\pgfusepath{clip}%
\pgfsetbuttcap%
\pgfsetroundjoin%
\pgfsetlinewidth{1.003750pt}%
\definecolor{currentstroke}{rgb}{1.000000,0.000000,0.000000}%
\pgfsetstrokecolor{currentstroke}%
\pgfsetdash{}{0pt}%
\pgfpathmoveto{\pgfqpoint{1.140550in}{3.320157in}}%
\pgfpathcurveto{\pgfqpoint{1.151600in}{3.320157in}}{\pgfqpoint{1.162199in}{3.324547in}}{\pgfqpoint{1.170013in}{3.332361in}}%
\pgfpathcurveto{\pgfqpoint{1.177826in}{3.340175in}}{\pgfqpoint{1.182217in}{3.350774in}}{\pgfqpoint{1.182217in}{3.361824in}}%
\pgfpathcurveto{\pgfqpoint{1.182217in}{3.372874in}}{\pgfqpoint{1.177826in}{3.383473in}}{\pgfqpoint{1.170013in}{3.391287in}}%
\pgfpathcurveto{\pgfqpoint{1.162199in}{3.399100in}}{\pgfqpoint{1.151600in}{3.403490in}}{\pgfqpoint{1.140550in}{3.403490in}}%
\pgfpathcurveto{\pgfqpoint{1.129500in}{3.403490in}}{\pgfqpoint{1.118901in}{3.399100in}}{\pgfqpoint{1.111087in}{3.391287in}}%
\pgfpathcurveto{\pgfqpoint{1.103274in}{3.383473in}}{\pgfqpoint{1.098883in}{3.372874in}}{\pgfqpoint{1.098883in}{3.361824in}}%
\pgfpathcurveto{\pgfqpoint{1.098883in}{3.350774in}}{\pgfqpoint{1.103274in}{3.340175in}}{\pgfqpoint{1.111087in}{3.332361in}}%
\pgfpathcurveto{\pgfqpoint{1.118901in}{3.324547in}}{\pgfqpoint{1.129500in}{3.320157in}}{\pgfqpoint{1.140550in}{3.320157in}}%
\pgfpathlineto{\pgfqpoint{1.140550in}{3.320157in}}%
\pgfpathclose%
\pgfusepath{stroke}%
\end{pgfscope}%
\begin{pgfscope}%
\pgfpathrectangle{\pgfqpoint{0.847223in}{0.554012in}}{\pgfqpoint{6.200000in}{4.620000in}}%
\pgfusepath{clip}%
\pgfsetbuttcap%
\pgfsetroundjoin%
\pgfsetlinewidth{1.003750pt}%
\definecolor{currentstroke}{rgb}{1.000000,0.000000,0.000000}%
\pgfsetstrokecolor{currentstroke}%
\pgfsetdash{}{0pt}%
\pgfpathmoveto{\pgfqpoint{1.145883in}{3.299179in}}%
\pgfpathcurveto{\pgfqpoint{1.156933in}{3.299179in}}{\pgfqpoint{1.167532in}{3.303570in}}{\pgfqpoint{1.175346in}{3.311383in}}%
\pgfpathcurveto{\pgfqpoint{1.183160in}{3.319197in}}{\pgfqpoint{1.187550in}{3.329796in}}{\pgfqpoint{1.187550in}{3.340846in}}%
\pgfpathcurveto{\pgfqpoint{1.187550in}{3.351896in}}{\pgfqpoint{1.183160in}{3.362495in}}{\pgfqpoint{1.175346in}{3.370309in}}%
\pgfpathcurveto{\pgfqpoint{1.167532in}{3.378123in}}{\pgfqpoint{1.156933in}{3.382513in}}{\pgfqpoint{1.145883in}{3.382513in}}%
\pgfpathcurveto{\pgfqpoint{1.134833in}{3.382513in}}{\pgfqpoint{1.124234in}{3.378123in}}{\pgfqpoint{1.116420in}{3.370309in}}%
\pgfpathcurveto{\pgfqpoint{1.108607in}{3.362495in}}{\pgfqpoint{1.104216in}{3.351896in}}{\pgfqpoint{1.104216in}{3.340846in}}%
\pgfpathcurveto{\pgfqpoint{1.104216in}{3.329796in}}{\pgfqpoint{1.108607in}{3.319197in}}{\pgfqpoint{1.116420in}{3.311383in}}%
\pgfpathcurveto{\pgfqpoint{1.124234in}{3.303570in}}{\pgfqpoint{1.134833in}{3.299179in}}{\pgfqpoint{1.145883in}{3.299179in}}%
\pgfpathlineto{\pgfqpoint{1.145883in}{3.299179in}}%
\pgfpathclose%
\pgfusepath{stroke}%
\end{pgfscope}%
\begin{pgfscope}%
\pgfpathrectangle{\pgfqpoint{0.847223in}{0.554012in}}{\pgfqpoint{6.200000in}{4.620000in}}%
\pgfusepath{clip}%
\pgfsetbuttcap%
\pgfsetroundjoin%
\pgfsetlinewidth{1.003750pt}%
\definecolor{currentstroke}{rgb}{1.000000,0.000000,0.000000}%
\pgfsetstrokecolor{currentstroke}%
\pgfsetdash{}{0pt}%
\pgfpathmoveto{\pgfqpoint{1.151216in}{3.278472in}}%
\pgfpathcurveto{\pgfqpoint{1.162266in}{3.278472in}}{\pgfqpoint{1.172866in}{3.282863in}}{\pgfqpoint{1.180679in}{3.290676in}}%
\pgfpathcurveto{\pgfqpoint{1.188493in}{3.298490in}}{\pgfqpoint{1.192883in}{3.309089in}}{\pgfqpoint{1.192883in}{3.320139in}}%
\pgfpathcurveto{\pgfqpoint{1.192883in}{3.331189in}}{\pgfqpoint{1.188493in}{3.341788in}}{\pgfqpoint{1.180679in}{3.349602in}}%
\pgfpathcurveto{\pgfqpoint{1.172866in}{3.357415in}}{\pgfqpoint{1.162266in}{3.361806in}}{\pgfqpoint{1.151216in}{3.361806in}}%
\pgfpathcurveto{\pgfqpoint{1.140166in}{3.361806in}}{\pgfqpoint{1.129567in}{3.357415in}}{\pgfqpoint{1.121754in}{3.349602in}}%
\pgfpathcurveto{\pgfqpoint{1.113940in}{3.341788in}}{\pgfqpoint{1.109550in}{3.331189in}}{\pgfqpoint{1.109550in}{3.320139in}}%
\pgfpathcurveto{\pgfqpoint{1.109550in}{3.309089in}}{\pgfqpoint{1.113940in}{3.298490in}}{\pgfqpoint{1.121754in}{3.290676in}}%
\pgfpathcurveto{\pgfqpoint{1.129567in}{3.282863in}}{\pgfqpoint{1.140166in}{3.278472in}}{\pgfqpoint{1.151216in}{3.278472in}}%
\pgfpathlineto{\pgfqpoint{1.151216in}{3.278472in}}%
\pgfpathclose%
\pgfusepath{stroke}%
\end{pgfscope}%
\begin{pgfscope}%
\pgfpathrectangle{\pgfqpoint{0.847223in}{0.554012in}}{\pgfqpoint{6.200000in}{4.620000in}}%
\pgfusepath{clip}%
\pgfsetbuttcap%
\pgfsetroundjoin%
\pgfsetlinewidth{1.003750pt}%
\definecolor{currentstroke}{rgb}{1.000000,0.000000,0.000000}%
\pgfsetstrokecolor{currentstroke}%
\pgfsetdash{}{0pt}%
\pgfpathmoveto{\pgfqpoint{1.156550in}{3.258030in}}%
\pgfpathcurveto{\pgfqpoint{1.167600in}{3.258030in}}{\pgfqpoint{1.178199in}{3.262421in}}{\pgfqpoint{1.186012in}{3.270234in}}%
\pgfpathcurveto{\pgfqpoint{1.193826in}{3.278048in}}{\pgfqpoint{1.198216in}{3.288647in}}{\pgfqpoint{1.198216in}{3.299697in}}%
\pgfpathcurveto{\pgfqpoint{1.198216in}{3.310747in}}{\pgfqpoint{1.193826in}{3.321346in}}{\pgfqpoint{1.186012in}{3.329160in}}%
\pgfpathcurveto{\pgfqpoint{1.178199in}{3.336974in}}{\pgfqpoint{1.167600in}{3.341364in}}{\pgfqpoint{1.156550in}{3.341364in}}%
\pgfpathcurveto{\pgfqpoint{1.145499in}{3.341364in}}{\pgfqpoint{1.134900in}{3.336974in}}{\pgfqpoint{1.127087in}{3.329160in}}%
\pgfpathcurveto{\pgfqpoint{1.119273in}{3.321346in}}{\pgfqpoint{1.114883in}{3.310747in}}{\pgfqpoint{1.114883in}{3.299697in}}%
\pgfpathcurveto{\pgfqpoint{1.114883in}{3.288647in}}{\pgfqpoint{1.119273in}{3.278048in}}{\pgfqpoint{1.127087in}{3.270234in}}%
\pgfpathcurveto{\pgfqpoint{1.134900in}{3.262421in}}{\pgfqpoint{1.145499in}{3.258030in}}{\pgfqpoint{1.156550in}{3.258030in}}%
\pgfpathlineto{\pgfqpoint{1.156550in}{3.258030in}}%
\pgfpathclose%
\pgfusepath{stroke}%
\end{pgfscope}%
\begin{pgfscope}%
\pgfpathrectangle{\pgfqpoint{0.847223in}{0.554012in}}{\pgfqpoint{6.200000in}{4.620000in}}%
\pgfusepath{clip}%
\pgfsetbuttcap%
\pgfsetroundjoin%
\pgfsetlinewidth{1.003750pt}%
\definecolor{currentstroke}{rgb}{1.000000,0.000000,0.000000}%
\pgfsetstrokecolor{currentstroke}%
\pgfsetdash{}{0pt}%
\pgfpathmoveto{\pgfqpoint{1.161883in}{3.237849in}}%
\pgfpathcurveto{\pgfqpoint{1.172933in}{3.237849in}}{\pgfqpoint{1.183532in}{3.242239in}}{\pgfqpoint{1.191346in}{3.250053in}}%
\pgfpathcurveto{\pgfqpoint{1.199159in}{3.257866in}}{\pgfqpoint{1.203549in}{3.268465in}}{\pgfqpoint{1.203549in}{3.279515in}}%
\pgfpathcurveto{\pgfqpoint{1.203549in}{3.290566in}}{\pgfqpoint{1.199159in}{3.301165in}}{\pgfqpoint{1.191346in}{3.308978in}}%
\pgfpathcurveto{\pgfqpoint{1.183532in}{3.316792in}}{\pgfqpoint{1.172933in}{3.321182in}}{\pgfqpoint{1.161883in}{3.321182in}}%
\pgfpathcurveto{\pgfqpoint{1.150833in}{3.321182in}}{\pgfqpoint{1.140234in}{3.316792in}}{\pgfqpoint{1.132420in}{3.308978in}}%
\pgfpathcurveto{\pgfqpoint{1.124606in}{3.301165in}}{\pgfqpoint{1.120216in}{3.290566in}}{\pgfqpoint{1.120216in}{3.279515in}}%
\pgfpathcurveto{\pgfqpoint{1.120216in}{3.268465in}}{\pgfqpoint{1.124606in}{3.257866in}}{\pgfqpoint{1.132420in}{3.250053in}}%
\pgfpathcurveto{\pgfqpoint{1.140234in}{3.242239in}}{\pgfqpoint{1.150833in}{3.237849in}}{\pgfqpoint{1.161883in}{3.237849in}}%
\pgfpathlineto{\pgfqpoint{1.161883in}{3.237849in}}%
\pgfpathclose%
\pgfusepath{stroke}%
\end{pgfscope}%
\begin{pgfscope}%
\pgfpathrectangle{\pgfqpoint{0.847223in}{0.554012in}}{\pgfqpoint{6.200000in}{4.620000in}}%
\pgfusepath{clip}%
\pgfsetbuttcap%
\pgfsetroundjoin%
\pgfsetlinewidth{1.003750pt}%
\definecolor{currentstroke}{rgb}{1.000000,0.000000,0.000000}%
\pgfsetstrokecolor{currentstroke}%
\pgfsetdash{}{0pt}%
\pgfpathmoveto{\pgfqpoint{1.167216in}{3.217922in}}%
\pgfpathcurveto{\pgfqpoint{1.178266in}{3.217922in}}{\pgfqpoint{1.188865in}{3.222313in}}{\pgfqpoint{1.196679in}{3.230126in}}%
\pgfpathcurveto{\pgfqpoint{1.204492in}{3.237940in}}{\pgfqpoint{1.208883in}{3.248539in}}{\pgfqpoint{1.208883in}{3.259589in}}%
\pgfpathcurveto{\pgfqpoint{1.208883in}{3.270639in}}{\pgfqpoint{1.204492in}{3.281238in}}{\pgfqpoint{1.196679in}{3.289052in}}%
\pgfpathcurveto{\pgfqpoint{1.188865in}{3.296865in}}{\pgfqpoint{1.178266in}{3.301256in}}{\pgfqpoint{1.167216in}{3.301256in}}%
\pgfpathcurveto{\pgfqpoint{1.156166in}{3.301256in}}{\pgfqpoint{1.145567in}{3.296865in}}{\pgfqpoint{1.137753in}{3.289052in}}%
\pgfpathcurveto{\pgfqpoint{1.129940in}{3.281238in}}{\pgfqpoint{1.125549in}{3.270639in}}{\pgfqpoint{1.125549in}{3.259589in}}%
\pgfpathcurveto{\pgfqpoint{1.125549in}{3.248539in}}{\pgfqpoint{1.129940in}{3.237940in}}{\pgfqpoint{1.137753in}{3.230126in}}%
\pgfpathcurveto{\pgfqpoint{1.145567in}{3.222313in}}{\pgfqpoint{1.156166in}{3.217922in}}{\pgfqpoint{1.167216in}{3.217922in}}%
\pgfpathlineto{\pgfqpoint{1.167216in}{3.217922in}}%
\pgfpathclose%
\pgfusepath{stroke}%
\end{pgfscope}%
\begin{pgfscope}%
\pgfpathrectangle{\pgfqpoint{0.847223in}{0.554012in}}{\pgfqpoint{6.200000in}{4.620000in}}%
\pgfusepath{clip}%
\pgfsetbuttcap%
\pgfsetroundjoin%
\pgfsetlinewidth{1.003750pt}%
\definecolor{currentstroke}{rgb}{1.000000,0.000000,0.000000}%
\pgfsetstrokecolor{currentstroke}%
\pgfsetdash{}{0pt}%
\pgfpathmoveto{\pgfqpoint{1.172549in}{3.198246in}}%
\pgfpathcurveto{\pgfqpoint{1.183599in}{3.198246in}}{\pgfqpoint{1.194198in}{3.202637in}}{\pgfqpoint{1.202012in}{3.210450in}}%
\pgfpathcurveto{\pgfqpoint{1.209826in}{3.218264in}}{\pgfqpoint{1.214216in}{3.228863in}}{\pgfqpoint{1.214216in}{3.239913in}}%
\pgfpathcurveto{\pgfqpoint{1.214216in}{3.250963in}}{\pgfqpoint{1.209826in}{3.261562in}}{\pgfqpoint{1.202012in}{3.269376in}}%
\pgfpathcurveto{\pgfqpoint{1.194198in}{3.277189in}}{\pgfqpoint{1.183599in}{3.281580in}}{\pgfqpoint{1.172549in}{3.281580in}}%
\pgfpathcurveto{\pgfqpoint{1.161499in}{3.281580in}}{\pgfqpoint{1.150900in}{3.277189in}}{\pgfqpoint{1.143086in}{3.269376in}}%
\pgfpathcurveto{\pgfqpoint{1.135273in}{3.261562in}}{\pgfqpoint{1.130883in}{3.250963in}}{\pgfqpoint{1.130883in}{3.239913in}}%
\pgfpathcurveto{\pgfqpoint{1.130883in}{3.228863in}}{\pgfqpoint{1.135273in}{3.218264in}}{\pgfqpoint{1.143086in}{3.210450in}}%
\pgfpathcurveto{\pgfqpoint{1.150900in}{3.202637in}}{\pgfqpoint{1.161499in}{3.198246in}}{\pgfqpoint{1.172549in}{3.198246in}}%
\pgfpathlineto{\pgfqpoint{1.172549in}{3.198246in}}%
\pgfpathclose%
\pgfusepath{stroke}%
\end{pgfscope}%
\begin{pgfscope}%
\pgfpathrectangle{\pgfqpoint{0.847223in}{0.554012in}}{\pgfqpoint{6.200000in}{4.620000in}}%
\pgfusepath{clip}%
\pgfsetbuttcap%
\pgfsetroundjoin%
\pgfsetlinewidth{1.003750pt}%
\definecolor{currentstroke}{rgb}{1.000000,0.000000,0.000000}%
\pgfsetstrokecolor{currentstroke}%
\pgfsetdash{}{0pt}%
\pgfpathmoveto{\pgfqpoint{1.177882in}{3.178816in}}%
\pgfpathcurveto{\pgfqpoint{1.188933in}{3.178816in}}{\pgfqpoint{1.199532in}{3.183206in}}{\pgfqpoint{1.207345in}{3.191020in}}%
\pgfpathcurveto{\pgfqpoint{1.215159in}{3.198834in}}{\pgfqpoint{1.219549in}{3.209433in}}{\pgfqpoint{1.219549in}{3.220483in}}%
\pgfpathcurveto{\pgfqpoint{1.219549in}{3.231533in}}{\pgfqpoint{1.215159in}{3.242132in}}{\pgfqpoint{1.207345in}{3.249946in}}%
\pgfpathcurveto{\pgfqpoint{1.199532in}{3.257759in}}{\pgfqpoint{1.188933in}{3.262149in}}{\pgfqpoint{1.177882in}{3.262149in}}%
\pgfpathcurveto{\pgfqpoint{1.166832in}{3.262149in}}{\pgfqpoint{1.156233in}{3.257759in}}{\pgfqpoint{1.148420in}{3.249946in}}%
\pgfpathcurveto{\pgfqpoint{1.140606in}{3.242132in}}{\pgfqpoint{1.136216in}{3.231533in}}{\pgfqpoint{1.136216in}{3.220483in}}%
\pgfpathcurveto{\pgfqpoint{1.136216in}{3.209433in}}{\pgfqpoint{1.140606in}{3.198834in}}{\pgfqpoint{1.148420in}{3.191020in}}%
\pgfpathcurveto{\pgfqpoint{1.156233in}{3.183206in}}{\pgfqpoint{1.166832in}{3.178816in}}{\pgfqpoint{1.177882in}{3.178816in}}%
\pgfpathlineto{\pgfqpoint{1.177882in}{3.178816in}}%
\pgfpathclose%
\pgfusepath{stroke}%
\end{pgfscope}%
\begin{pgfscope}%
\pgfpathrectangle{\pgfqpoint{0.847223in}{0.554012in}}{\pgfqpoint{6.200000in}{4.620000in}}%
\pgfusepath{clip}%
\pgfsetbuttcap%
\pgfsetroundjoin%
\pgfsetlinewidth{1.003750pt}%
\definecolor{currentstroke}{rgb}{1.000000,0.000000,0.000000}%
\pgfsetstrokecolor{currentstroke}%
\pgfsetdash{}{0pt}%
\pgfpathmoveto{\pgfqpoint{1.183216in}{3.159627in}}%
\pgfpathcurveto{\pgfqpoint{1.194266in}{3.159627in}}{\pgfqpoint{1.204865in}{3.164017in}}{\pgfqpoint{1.212678in}{3.171831in}}%
\pgfpathcurveto{\pgfqpoint{1.220492in}{3.179644in}}{\pgfqpoint{1.224882in}{3.190244in}}{\pgfqpoint{1.224882in}{3.201294in}}%
\pgfpathcurveto{\pgfqpoint{1.224882in}{3.212344in}}{\pgfqpoint{1.220492in}{3.222943in}}{\pgfqpoint{1.212678in}{3.230756in}}%
\pgfpathcurveto{\pgfqpoint{1.204865in}{3.238570in}}{\pgfqpoint{1.194266in}{3.242960in}}{\pgfqpoint{1.183216in}{3.242960in}}%
\pgfpathcurveto{\pgfqpoint{1.172166in}{3.242960in}}{\pgfqpoint{1.161566in}{3.238570in}}{\pgfqpoint{1.153753in}{3.230756in}}%
\pgfpathcurveto{\pgfqpoint{1.145939in}{3.222943in}}{\pgfqpoint{1.141549in}{3.212344in}}{\pgfqpoint{1.141549in}{3.201294in}}%
\pgfpathcurveto{\pgfqpoint{1.141549in}{3.190244in}}{\pgfqpoint{1.145939in}{3.179644in}}{\pgfqpoint{1.153753in}{3.171831in}}%
\pgfpathcurveto{\pgfqpoint{1.161566in}{3.164017in}}{\pgfqpoint{1.172166in}{3.159627in}}{\pgfqpoint{1.183216in}{3.159627in}}%
\pgfpathlineto{\pgfqpoint{1.183216in}{3.159627in}}%
\pgfpathclose%
\pgfusepath{stroke}%
\end{pgfscope}%
\begin{pgfscope}%
\pgfpathrectangle{\pgfqpoint{0.847223in}{0.554012in}}{\pgfqpoint{6.200000in}{4.620000in}}%
\pgfusepath{clip}%
\pgfsetbuttcap%
\pgfsetroundjoin%
\pgfsetlinewidth{1.003750pt}%
\definecolor{currentstroke}{rgb}{1.000000,0.000000,0.000000}%
\pgfsetstrokecolor{currentstroke}%
\pgfsetdash{}{0pt}%
\pgfpathmoveto{\pgfqpoint{1.188549in}{3.140675in}}%
\pgfpathcurveto{\pgfqpoint{1.199599in}{3.140675in}}{\pgfqpoint{1.210198in}{3.145065in}}{\pgfqpoint{1.218012in}{3.152879in}}%
\pgfpathcurveto{\pgfqpoint{1.225825in}{3.160692in}}{\pgfqpoint{1.230216in}{3.171291in}}{\pgfqpoint{1.230216in}{3.182341in}}%
\pgfpathcurveto{\pgfqpoint{1.230216in}{3.193391in}}{\pgfqpoint{1.225825in}{3.203990in}}{\pgfqpoint{1.218012in}{3.211804in}}%
\pgfpathcurveto{\pgfqpoint{1.210198in}{3.219618in}}{\pgfqpoint{1.199599in}{3.224008in}}{\pgfqpoint{1.188549in}{3.224008in}}%
\pgfpathcurveto{\pgfqpoint{1.177499in}{3.224008in}}{\pgfqpoint{1.166900in}{3.219618in}}{\pgfqpoint{1.159086in}{3.211804in}}%
\pgfpathcurveto{\pgfqpoint{1.151272in}{3.203990in}}{\pgfqpoint{1.146882in}{3.193391in}}{\pgfqpoint{1.146882in}{3.182341in}}%
\pgfpathcurveto{\pgfqpoint{1.146882in}{3.171291in}}{\pgfqpoint{1.151272in}{3.160692in}}{\pgfqpoint{1.159086in}{3.152879in}}%
\pgfpathcurveto{\pgfqpoint{1.166900in}{3.145065in}}{\pgfqpoint{1.177499in}{3.140675in}}{\pgfqpoint{1.188549in}{3.140675in}}%
\pgfpathlineto{\pgfqpoint{1.188549in}{3.140675in}}%
\pgfpathclose%
\pgfusepath{stroke}%
\end{pgfscope}%
\begin{pgfscope}%
\pgfpathrectangle{\pgfqpoint{0.847223in}{0.554012in}}{\pgfqpoint{6.200000in}{4.620000in}}%
\pgfusepath{clip}%
\pgfsetbuttcap%
\pgfsetroundjoin%
\pgfsetlinewidth{1.003750pt}%
\definecolor{currentstroke}{rgb}{1.000000,0.000000,0.000000}%
\pgfsetstrokecolor{currentstroke}%
\pgfsetdash{}{0pt}%
\pgfpathmoveto{\pgfqpoint{1.193882in}{3.121955in}}%
\pgfpathcurveto{\pgfqpoint{1.204932in}{3.121955in}}{\pgfqpoint{1.215531in}{3.126345in}}{\pgfqpoint{1.223345in}{3.134158in}}%
\pgfpathcurveto{\pgfqpoint{1.231158in}{3.141972in}}{\pgfqpoint{1.235549in}{3.152571in}}{\pgfqpoint{1.235549in}{3.163621in}}%
\pgfpathcurveto{\pgfqpoint{1.235549in}{3.174671in}}{\pgfqpoint{1.231158in}{3.185270in}}{\pgfqpoint{1.223345in}{3.193084in}}%
\pgfpathcurveto{\pgfqpoint{1.215531in}{3.200898in}}{\pgfqpoint{1.204932in}{3.205288in}}{\pgfqpoint{1.193882in}{3.205288in}}%
\pgfpathcurveto{\pgfqpoint{1.182832in}{3.205288in}}{\pgfqpoint{1.172233in}{3.200898in}}{\pgfqpoint{1.164419in}{3.193084in}}%
\pgfpathcurveto{\pgfqpoint{1.156606in}{3.185270in}}{\pgfqpoint{1.152215in}{3.174671in}}{\pgfqpoint{1.152215in}{3.163621in}}%
\pgfpathcurveto{\pgfqpoint{1.152215in}{3.152571in}}{\pgfqpoint{1.156606in}{3.141972in}}{\pgfqpoint{1.164419in}{3.134158in}}%
\pgfpathcurveto{\pgfqpoint{1.172233in}{3.126345in}}{\pgfqpoint{1.182832in}{3.121955in}}{\pgfqpoint{1.193882in}{3.121955in}}%
\pgfpathlineto{\pgfqpoint{1.193882in}{3.121955in}}%
\pgfpathclose%
\pgfusepath{stroke}%
\end{pgfscope}%
\begin{pgfscope}%
\pgfpathrectangle{\pgfqpoint{0.847223in}{0.554012in}}{\pgfqpoint{6.200000in}{4.620000in}}%
\pgfusepath{clip}%
\pgfsetbuttcap%
\pgfsetroundjoin%
\pgfsetlinewidth{1.003750pt}%
\definecolor{currentstroke}{rgb}{1.000000,0.000000,0.000000}%
\pgfsetstrokecolor{currentstroke}%
\pgfsetdash{}{0pt}%
\pgfpathmoveto{\pgfqpoint{1.199215in}{3.103463in}}%
\pgfpathcurveto{\pgfqpoint{1.210265in}{3.103463in}}{\pgfqpoint{1.220864in}{3.107853in}}{\pgfqpoint{1.228678in}{3.115667in}}%
\pgfpathcurveto{\pgfqpoint{1.236492in}{3.123480in}}{\pgfqpoint{1.240882in}{3.134079in}}{\pgfqpoint{1.240882in}{3.145129in}}%
\pgfpathcurveto{\pgfqpoint{1.240882in}{3.156180in}}{\pgfqpoint{1.236492in}{3.166779in}}{\pgfqpoint{1.228678in}{3.174592in}}%
\pgfpathcurveto{\pgfqpoint{1.220864in}{3.182406in}}{\pgfqpoint{1.210265in}{3.186796in}}{\pgfqpoint{1.199215in}{3.186796in}}%
\pgfpathcurveto{\pgfqpoint{1.188165in}{3.186796in}}{\pgfqpoint{1.177566in}{3.182406in}}{\pgfqpoint{1.169753in}{3.174592in}}%
\pgfpathcurveto{\pgfqpoint{1.161939in}{3.166779in}}{\pgfqpoint{1.157549in}{3.156180in}}{\pgfqpoint{1.157549in}{3.145129in}}%
\pgfpathcurveto{\pgfqpoint{1.157549in}{3.134079in}}{\pgfqpoint{1.161939in}{3.123480in}}{\pgfqpoint{1.169753in}{3.115667in}}%
\pgfpathcurveto{\pgfqpoint{1.177566in}{3.107853in}}{\pgfqpoint{1.188165in}{3.103463in}}{\pgfqpoint{1.199215in}{3.103463in}}%
\pgfpathlineto{\pgfqpoint{1.199215in}{3.103463in}}%
\pgfpathclose%
\pgfusepath{stroke}%
\end{pgfscope}%
\begin{pgfscope}%
\pgfpathrectangle{\pgfqpoint{0.847223in}{0.554012in}}{\pgfqpoint{6.200000in}{4.620000in}}%
\pgfusepath{clip}%
\pgfsetbuttcap%
\pgfsetroundjoin%
\pgfsetlinewidth{1.003750pt}%
\definecolor{currentstroke}{rgb}{1.000000,0.000000,0.000000}%
\pgfsetstrokecolor{currentstroke}%
\pgfsetdash{}{0pt}%
\pgfpathmoveto{\pgfqpoint{1.204549in}{3.085195in}}%
\pgfpathcurveto{\pgfqpoint{1.215599in}{3.085195in}}{\pgfqpoint{1.226198in}{3.089585in}}{\pgfqpoint{1.234011in}{3.097399in}}%
\pgfpathcurveto{\pgfqpoint{1.241825in}{3.105212in}}{\pgfqpoint{1.246215in}{3.115811in}}{\pgfqpoint{1.246215in}{3.126861in}}%
\pgfpathcurveto{\pgfqpoint{1.246215in}{3.137912in}}{\pgfqpoint{1.241825in}{3.148511in}}{\pgfqpoint{1.234011in}{3.156324in}}%
\pgfpathcurveto{\pgfqpoint{1.226198in}{3.164138in}}{\pgfqpoint{1.215599in}{3.168528in}}{\pgfqpoint{1.204549in}{3.168528in}}%
\pgfpathcurveto{\pgfqpoint{1.193498in}{3.168528in}}{\pgfqpoint{1.182899in}{3.164138in}}{\pgfqpoint{1.175086in}{3.156324in}}%
\pgfpathcurveto{\pgfqpoint{1.167272in}{3.148511in}}{\pgfqpoint{1.162882in}{3.137912in}}{\pgfqpoint{1.162882in}{3.126861in}}%
\pgfpathcurveto{\pgfqpoint{1.162882in}{3.115811in}}{\pgfqpoint{1.167272in}{3.105212in}}{\pgfqpoint{1.175086in}{3.097399in}}%
\pgfpathcurveto{\pgfqpoint{1.182899in}{3.089585in}}{\pgfqpoint{1.193498in}{3.085195in}}{\pgfqpoint{1.204549in}{3.085195in}}%
\pgfpathlineto{\pgfqpoint{1.204549in}{3.085195in}}%
\pgfpathclose%
\pgfusepath{stroke}%
\end{pgfscope}%
\begin{pgfscope}%
\pgfpathrectangle{\pgfqpoint{0.847223in}{0.554012in}}{\pgfqpoint{6.200000in}{4.620000in}}%
\pgfusepath{clip}%
\pgfsetbuttcap%
\pgfsetroundjoin%
\pgfsetlinewidth{1.003750pt}%
\definecolor{currentstroke}{rgb}{1.000000,0.000000,0.000000}%
\pgfsetstrokecolor{currentstroke}%
\pgfsetdash{}{0pt}%
\pgfpathmoveto{\pgfqpoint{1.209882in}{3.067147in}}%
\pgfpathcurveto{\pgfqpoint{1.220932in}{3.067147in}}{\pgfqpoint{1.231531in}{3.071537in}}{\pgfqpoint{1.239345in}{3.079351in}}%
\pgfpathcurveto{\pgfqpoint{1.247158in}{3.087164in}}{\pgfqpoint{1.251548in}{3.097763in}}{\pgfqpoint{1.251548in}{3.108813in}}%
\pgfpathcurveto{\pgfqpoint{1.251548in}{3.119864in}}{\pgfqpoint{1.247158in}{3.130463in}}{\pgfqpoint{1.239345in}{3.138276in}}%
\pgfpathcurveto{\pgfqpoint{1.231531in}{3.146090in}}{\pgfqpoint{1.220932in}{3.150480in}}{\pgfqpoint{1.209882in}{3.150480in}}%
\pgfpathcurveto{\pgfqpoint{1.198832in}{3.150480in}}{\pgfqpoint{1.188233in}{3.146090in}}{\pgfqpoint{1.180419in}{3.138276in}}%
\pgfpathcurveto{\pgfqpoint{1.172605in}{3.130463in}}{\pgfqpoint{1.168215in}{3.119864in}}{\pgfqpoint{1.168215in}{3.108813in}}%
\pgfpathcurveto{\pgfqpoint{1.168215in}{3.097763in}}{\pgfqpoint{1.172605in}{3.087164in}}{\pgfqpoint{1.180419in}{3.079351in}}%
\pgfpathcurveto{\pgfqpoint{1.188233in}{3.071537in}}{\pgfqpoint{1.198832in}{3.067147in}}{\pgfqpoint{1.209882in}{3.067147in}}%
\pgfpathlineto{\pgfqpoint{1.209882in}{3.067147in}}%
\pgfpathclose%
\pgfusepath{stroke}%
\end{pgfscope}%
\begin{pgfscope}%
\pgfpathrectangle{\pgfqpoint{0.847223in}{0.554012in}}{\pgfqpoint{6.200000in}{4.620000in}}%
\pgfusepath{clip}%
\pgfsetbuttcap%
\pgfsetroundjoin%
\pgfsetlinewidth{1.003750pt}%
\definecolor{currentstroke}{rgb}{1.000000,0.000000,0.000000}%
\pgfsetstrokecolor{currentstroke}%
\pgfsetdash{}{0pt}%
\pgfpathmoveto{\pgfqpoint{1.215215in}{3.049315in}}%
\pgfpathcurveto{\pgfqpoint{1.226265in}{3.049315in}}{\pgfqpoint{1.236864in}{3.053705in}}{\pgfqpoint{1.244678in}{3.061519in}}%
\pgfpathcurveto{\pgfqpoint{1.252491in}{3.069332in}}{\pgfqpoint{1.256882in}{3.079931in}}{\pgfqpoint{1.256882in}{3.090981in}}%
\pgfpathcurveto{\pgfqpoint{1.256882in}{3.102032in}}{\pgfqpoint{1.252491in}{3.112631in}}{\pgfqpoint{1.244678in}{3.120444in}}%
\pgfpathcurveto{\pgfqpoint{1.236864in}{3.128258in}}{\pgfqpoint{1.226265in}{3.132648in}}{\pgfqpoint{1.215215in}{3.132648in}}%
\pgfpathcurveto{\pgfqpoint{1.204165in}{3.132648in}}{\pgfqpoint{1.193566in}{3.128258in}}{\pgfqpoint{1.185752in}{3.120444in}}%
\pgfpathcurveto{\pgfqpoint{1.177939in}{3.112631in}}{\pgfqpoint{1.173548in}{3.102032in}}{\pgfqpoint{1.173548in}{3.090981in}}%
\pgfpathcurveto{\pgfqpoint{1.173548in}{3.079931in}}{\pgfqpoint{1.177939in}{3.069332in}}{\pgfqpoint{1.185752in}{3.061519in}}%
\pgfpathcurveto{\pgfqpoint{1.193566in}{3.053705in}}{\pgfqpoint{1.204165in}{3.049315in}}{\pgfqpoint{1.215215in}{3.049315in}}%
\pgfpathlineto{\pgfqpoint{1.215215in}{3.049315in}}%
\pgfpathclose%
\pgfusepath{stroke}%
\end{pgfscope}%
\begin{pgfscope}%
\pgfpathrectangle{\pgfqpoint{0.847223in}{0.554012in}}{\pgfqpoint{6.200000in}{4.620000in}}%
\pgfusepath{clip}%
\pgfsetbuttcap%
\pgfsetroundjoin%
\pgfsetlinewidth{1.003750pt}%
\definecolor{currentstroke}{rgb}{1.000000,0.000000,0.000000}%
\pgfsetstrokecolor{currentstroke}%
\pgfsetdash{}{0pt}%
\pgfpathmoveto{\pgfqpoint{1.220548in}{3.031695in}}%
\pgfpathcurveto{\pgfqpoint{1.231598in}{3.031695in}}{\pgfqpoint{1.242197in}{3.036085in}}{\pgfqpoint{1.250011in}{3.043899in}}%
\pgfpathcurveto{\pgfqpoint{1.257825in}{3.051712in}}{\pgfqpoint{1.262215in}{3.062311in}}{\pgfqpoint{1.262215in}{3.073361in}}%
\pgfpathcurveto{\pgfqpoint{1.262215in}{3.084412in}}{\pgfqpoint{1.257825in}{3.095011in}}{\pgfqpoint{1.250011in}{3.102824in}}%
\pgfpathcurveto{\pgfqpoint{1.242197in}{3.110638in}}{\pgfqpoint{1.231598in}{3.115028in}}{\pgfqpoint{1.220548in}{3.115028in}}%
\pgfpathcurveto{\pgfqpoint{1.209498in}{3.115028in}}{\pgfqpoint{1.198899in}{3.110638in}}{\pgfqpoint{1.191085in}{3.102824in}}%
\pgfpathcurveto{\pgfqpoint{1.183272in}{3.095011in}}{\pgfqpoint{1.178881in}{3.084412in}}{\pgfqpoint{1.178881in}{3.073361in}}%
\pgfpathcurveto{\pgfqpoint{1.178881in}{3.062311in}}{\pgfqpoint{1.183272in}{3.051712in}}{\pgfqpoint{1.191085in}{3.043899in}}%
\pgfpathcurveto{\pgfqpoint{1.198899in}{3.036085in}}{\pgfqpoint{1.209498in}{3.031695in}}{\pgfqpoint{1.220548in}{3.031695in}}%
\pgfpathlineto{\pgfqpoint{1.220548in}{3.031695in}}%
\pgfpathclose%
\pgfusepath{stroke}%
\end{pgfscope}%
\begin{pgfscope}%
\pgfpathrectangle{\pgfqpoint{0.847223in}{0.554012in}}{\pgfqpoint{6.200000in}{4.620000in}}%
\pgfusepath{clip}%
\pgfsetbuttcap%
\pgfsetroundjoin%
\pgfsetlinewidth{1.003750pt}%
\definecolor{currentstroke}{rgb}{1.000000,0.000000,0.000000}%
\pgfsetstrokecolor{currentstroke}%
\pgfsetdash{}{0pt}%
\pgfpathmoveto{\pgfqpoint{1.225881in}{3.014283in}}%
\pgfpathcurveto{\pgfqpoint{1.236931in}{3.014283in}}{\pgfqpoint{1.247531in}{3.018674in}}{\pgfqpoint{1.255344in}{3.026487in}}%
\pgfpathcurveto{\pgfqpoint{1.263158in}{3.034301in}}{\pgfqpoint{1.267548in}{3.044900in}}{\pgfqpoint{1.267548in}{3.055950in}}%
\pgfpathcurveto{\pgfqpoint{1.267548in}{3.067000in}}{\pgfqpoint{1.263158in}{3.077599in}}{\pgfqpoint{1.255344in}{3.085413in}}%
\pgfpathcurveto{\pgfqpoint{1.247531in}{3.093226in}}{\pgfqpoint{1.236931in}{3.097617in}}{\pgfqpoint{1.225881in}{3.097617in}}%
\pgfpathcurveto{\pgfqpoint{1.214831in}{3.097617in}}{\pgfqpoint{1.204232in}{3.093226in}}{\pgfqpoint{1.196419in}{3.085413in}}%
\pgfpathcurveto{\pgfqpoint{1.188605in}{3.077599in}}{\pgfqpoint{1.184215in}{3.067000in}}{\pgfqpoint{1.184215in}{3.055950in}}%
\pgfpathcurveto{\pgfqpoint{1.184215in}{3.044900in}}{\pgfqpoint{1.188605in}{3.034301in}}{\pgfqpoint{1.196419in}{3.026487in}}%
\pgfpathcurveto{\pgfqpoint{1.204232in}{3.018674in}}{\pgfqpoint{1.214831in}{3.014283in}}{\pgfqpoint{1.225881in}{3.014283in}}%
\pgfpathlineto{\pgfqpoint{1.225881in}{3.014283in}}%
\pgfpathclose%
\pgfusepath{stroke}%
\end{pgfscope}%
\begin{pgfscope}%
\pgfpathrectangle{\pgfqpoint{0.847223in}{0.554012in}}{\pgfqpoint{6.200000in}{4.620000in}}%
\pgfusepath{clip}%
\pgfsetbuttcap%
\pgfsetroundjoin%
\pgfsetlinewidth{1.003750pt}%
\definecolor{currentstroke}{rgb}{1.000000,0.000000,0.000000}%
\pgfsetstrokecolor{currentstroke}%
\pgfsetdash{}{0pt}%
\pgfpathmoveto{\pgfqpoint{1.231215in}{2.997076in}}%
\pgfpathcurveto{\pgfqpoint{1.242265in}{2.997076in}}{\pgfqpoint{1.252864in}{3.001467in}}{\pgfqpoint{1.260677in}{3.009280in}}%
\pgfpathcurveto{\pgfqpoint{1.268491in}{3.017094in}}{\pgfqpoint{1.272881in}{3.027693in}}{\pgfqpoint{1.272881in}{3.038743in}}%
\pgfpathcurveto{\pgfqpoint{1.272881in}{3.049793in}}{\pgfqpoint{1.268491in}{3.060392in}}{\pgfqpoint{1.260677in}{3.068206in}}%
\pgfpathcurveto{\pgfqpoint{1.252864in}{3.076019in}}{\pgfqpoint{1.242265in}{3.080410in}}{\pgfqpoint{1.231215in}{3.080410in}}%
\pgfpathcurveto{\pgfqpoint{1.220164in}{3.080410in}}{\pgfqpoint{1.209565in}{3.076019in}}{\pgfqpoint{1.201752in}{3.068206in}}%
\pgfpathcurveto{\pgfqpoint{1.193938in}{3.060392in}}{\pgfqpoint{1.189548in}{3.049793in}}{\pgfqpoint{1.189548in}{3.038743in}}%
\pgfpathcurveto{\pgfqpoint{1.189548in}{3.027693in}}{\pgfqpoint{1.193938in}{3.017094in}}{\pgfqpoint{1.201752in}{3.009280in}}%
\pgfpathcurveto{\pgfqpoint{1.209565in}{3.001467in}}{\pgfqpoint{1.220164in}{2.997076in}}{\pgfqpoint{1.231215in}{2.997076in}}%
\pgfpathlineto{\pgfqpoint{1.231215in}{2.997076in}}%
\pgfpathclose%
\pgfusepath{stroke}%
\end{pgfscope}%
\begin{pgfscope}%
\pgfpathrectangle{\pgfqpoint{0.847223in}{0.554012in}}{\pgfqpoint{6.200000in}{4.620000in}}%
\pgfusepath{clip}%
\pgfsetbuttcap%
\pgfsetroundjoin%
\pgfsetlinewidth{1.003750pt}%
\definecolor{currentstroke}{rgb}{1.000000,0.000000,0.000000}%
\pgfsetstrokecolor{currentstroke}%
\pgfsetdash{}{0pt}%
\pgfpathmoveto{\pgfqpoint{1.236548in}{2.980071in}}%
\pgfpathcurveto{\pgfqpoint{1.247598in}{2.980071in}}{\pgfqpoint{1.258197in}{2.984461in}}{\pgfqpoint{1.266011in}{2.992275in}}%
\pgfpathcurveto{\pgfqpoint{1.273824in}{3.000088in}}{\pgfqpoint{1.278214in}{3.010687in}}{\pgfqpoint{1.278214in}{3.021737in}}%
\pgfpathcurveto{\pgfqpoint{1.278214in}{3.032787in}}{\pgfqpoint{1.273824in}{3.043386in}}{\pgfqpoint{1.266011in}{3.051200in}}%
\pgfpathcurveto{\pgfqpoint{1.258197in}{3.059014in}}{\pgfqpoint{1.247598in}{3.063404in}}{\pgfqpoint{1.236548in}{3.063404in}}%
\pgfpathcurveto{\pgfqpoint{1.225498in}{3.063404in}}{\pgfqpoint{1.214899in}{3.059014in}}{\pgfqpoint{1.207085in}{3.051200in}}%
\pgfpathcurveto{\pgfqpoint{1.199271in}{3.043386in}}{\pgfqpoint{1.194881in}{3.032787in}}{\pgfqpoint{1.194881in}{3.021737in}}%
\pgfpathcurveto{\pgfqpoint{1.194881in}{3.010687in}}{\pgfqpoint{1.199271in}{3.000088in}}{\pgfqpoint{1.207085in}{2.992275in}}%
\pgfpathcurveto{\pgfqpoint{1.214899in}{2.984461in}}{\pgfqpoint{1.225498in}{2.980071in}}{\pgfqpoint{1.236548in}{2.980071in}}%
\pgfpathlineto{\pgfqpoint{1.236548in}{2.980071in}}%
\pgfpathclose%
\pgfusepath{stroke}%
\end{pgfscope}%
\begin{pgfscope}%
\pgfpathrectangle{\pgfqpoint{0.847223in}{0.554012in}}{\pgfqpoint{6.200000in}{4.620000in}}%
\pgfusepath{clip}%
\pgfsetbuttcap%
\pgfsetroundjoin%
\pgfsetlinewidth{1.003750pt}%
\definecolor{currentstroke}{rgb}{1.000000,0.000000,0.000000}%
\pgfsetstrokecolor{currentstroke}%
\pgfsetdash{}{0pt}%
\pgfpathmoveto{\pgfqpoint{1.241881in}{2.963262in}}%
\pgfpathcurveto{\pgfqpoint{1.252931in}{2.963262in}}{\pgfqpoint{1.263530in}{2.967653in}}{\pgfqpoint{1.271344in}{2.975466in}}%
\pgfpathcurveto{\pgfqpoint{1.279157in}{2.983280in}}{\pgfqpoint{1.283548in}{2.993879in}}{\pgfqpoint{1.283548in}{3.004929in}}%
\pgfpathcurveto{\pgfqpoint{1.283548in}{3.015979in}}{\pgfqpoint{1.279157in}{3.026578in}}{\pgfqpoint{1.271344in}{3.034392in}}%
\pgfpathcurveto{\pgfqpoint{1.263530in}{3.042206in}}{\pgfqpoint{1.252931in}{3.046596in}}{\pgfqpoint{1.241881in}{3.046596in}}%
\pgfpathcurveto{\pgfqpoint{1.230831in}{3.046596in}}{\pgfqpoint{1.220232in}{3.042206in}}{\pgfqpoint{1.212418in}{3.034392in}}%
\pgfpathcurveto{\pgfqpoint{1.204605in}{3.026578in}}{\pgfqpoint{1.200214in}{3.015979in}}{\pgfqpoint{1.200214in}{3.004929in}}%
\pgfpathcurveto{\pgfqpoint{1.200214in}{2.993879in}}{\pgfqpoint{1.204605in}{2.983280in}}{\pgfqpoint{1.212418in}{2.975466in}}%
\pgfpathcurveto{\pgfqpoint{1.220232in}{2.967653in}}{\pgfqpoint{1.230831in}{2.963262in}}{\pgfqpoint{1.241881in}{2.963262in}}%
\pgfpathlineto{\pgfqpoint{1.241881in}{2.963262in}}%
\pgfpathclose%
\pgfusepath{stroke}%
\end{pgfscope}%
\begin{pgfscope}%
\pgfpathrectangle{\pgfqpoint{0.847223in}{0.554012in}}{\pgfqpoint{6.200000in}{4.620000in}}%
\pgfusepath{clip}%
\pgfsetbuttcap%
\pgfsetroundjoin%
\pgfsetlinewidth{1.003750pt}%
\definecolor{currentstroke}{rgb}{1.000000,0.000000,0.000000}%
\pgfsetstrokecolor{currentstroke}%
\pgfsetdash{}{0pt}%
\pgfpathmoveto{\pgfqpoint{1.247214in}{2.946648in}}%
\pgfpathcurveto{\pgfqpoint{1.258264in}{2.946648in}}{\pgfqpoint{1.268863in}{2.951039in}}{\pgfqpoint{1.276677in}{2.958852in}}%
\pgfpathcurveto{\pgfqpoint{1.284491in}{2.966666in}}{\pgfqpoint{1.288881in}{2.977265in}}{\pgfqpoint{1.288881in}{2.988315in}}%
\pgfpathcurveto{\pgfqpoint{1.288881in}{2.999365in}}{\pgfqpoint{1.284491in}{3.009964in}}{\pgfqpoint{1.276677in}{3.017778in}}%
\pgfpathcurveto{\pgfqpoint{1.268863in}{3.025592in}}{\pgfqpoint{1.258264in}{3.029982in}}{\pgfqpoint{1.247214in}{3.029982in}}%
\pgfpathcurveto{\pgfqpoint{1.236164in}{3.029982in}}{\pgfqpoint{1.225565in}{3.025592in}}{\pgfqpoint{1.217751in}{3.017778in}}%
\pgfpathcurveto{\pgfqpoint{1.209938in}{3.009964in}}{\pgfqpoint{1.205548in}{2.999365in}}{\pgfqpoint{1.205548in}{2.988315in}}%
\pgfpathcurveto{\pgfqpoint{1.205548in}{2.977265in}}{\pgfqpoint{1.209938in}{2.966666in}}{\pgfqpoint{1.217751in}{2.958852in}}%
\pgfpathcurveto{\pgfqpoint{1.225565in}{2.951039in}}{\pgfqpoint{1.236164in}{2.946648in}}{\pgfqpoint{1.247214in}{2.946648in}}%
\pgfpathlineto{\pgfqpoint{1.247214in}{2.946648in}}%
\pgfpathclose%
\pgfusepath{stroke}%
\end{pgfscope}%
\begin{pgfscope}%
\pgfpathrectangle{\pgfqpoint{0.847223in}{0.554012in}}{\pgfqpoint{6.200000in}{4.620000in}}%
\pgfusepath{clip}%
\pgfsetbuttcap%
\pgfsetroundjoin%
\pgfsetlinewidth{1.003750pt}%
\definecolor{currentstroke}{rgb}{1.000000,0.000000,0.000000}%
\pgfsetstrokecolor{currentstroke}%
\pgfsetdash{}{0pt}%
\pgfpathmoveto{\pgfqpoint{1.252547in}{2.930225in}}%
\pgfpathcurveto{\pgfqpoint{1.263598in}{2.930225in}}{\pgfqpoint{1.274197in}{2.934616in}}{\pgfqpoint{1.282010in}{2.942429in}}%
\pgfpathcurveto{\pgfqpoint{1.289824in}{2.950243in}}{\pgfqpoint{1.294214in}{2.960842in}}{\pgfqpoint{1.294214in}{2.971892in}}%
\pgfpathcurveto{\pgfqpoint{1.294214in}{2.982942in}}{\pgfqpoint{1.289824in}{2.993541in}}{\pgfqpoint{1.282010in}{3.001355in}}%
\pgfpathcurveto{\pgfqpoint{1.274197in}{3.009168in}}{\pgfqpoint{1.263598in}{3.013559in}}{\pgfqpoint{1.252547in}{3.013559in}}%
\pgfpathcurveto{\pgfqpoint{1.241497in}{3.013559in}}{\pgfqpoint{1.230898in}{3.009168in}}{\pgfqpoint{1.223085in}{3.001355in}}%
\pgfpathcurveto{\pgfqpoint{1.215271in}{2.993541in}}{\pgfqpoint{1.210881in}{2.982942in}}{\pgfqpoint{1.210881in}{2.971892in}}%
\pgfpathcurveto{\pgfqpoint{1.210881in}{2.960842in}}{\pgfqpoint{1.215271in}{2.950243in}}{\pgfqpoint{1.223085in}{2.942429in}}%
\pgfpathcurveto{\pgfqpoint{1.230898in}{2.934616in}}{\pgfqpoint{1.241497in}{2.930225in}}{\pgfqpoint{1.252547in}{2.930225in}}%
\pgfpathlineto{\pgfqpoint{1.252547in}{2.930225in}}%
\pgfpathclose%
\pgfusepath{stroke}%
\end{pgfscope}%
\begin{pgfscope}%
\pgfpathrectangle{\pgfqpoint{0.847223in}{0.554012in}}{\pgfqpoint{6.200000in}{4.620000in}}%
\pgfusepath{clip}%
\pgfsetbuttcap%
\pgfsetroundjoin%
\pgfsetlinewidth{1.003750pt}%
\definecolor{currentstroke}{rgb}{1.000000,0.000000,0.000000}%
\pgfsetstrokecolor{currentstroke}%
\pgfsetdash{}{0pt}%
\pgfpathmoveto{\pgfqpoint{1.257881in}{2.913990in}}%
\pgfpathcurveto{\pgfqpoint{1.268931in}{2.913990in}}{\pgfqpoint{1.279530in}{2.918380in}}{\pgfqpoint{1.287343in}{2.926194in}}%
\pgfpathcurveto{\pgfqpoint{1.295157in}{2.934007in}}{\pgfqpoint{1.299547in}{2.944606in}}{\pgfqpoint{1.299547in}{2.955656in}}%
\pgfpathcurveto{\pgfqpoint{1.299547in}{2.966706in}}{\pgfqpoint{1.295157in}{2.977305in}}{\pgfqpoint{1.287343in}{2.985119in}}%
\pgfpathcurveto{\pgfqpoint{1.279530in}{2.992933in}}{\pgfqpoint{1.268931in}{2.997323in}}{\pgfqpoint{1.257881in}{2.997323in}}%
\pgfpathcurveto{\pgfqpoint{1.246831in}{2.997323in}}{\pgfqpoint{1.236232in}{2.992933in}}{\pgfqpoint{1.228418in}{2.985119in}}%
\pgfpathcurveto{\pgfqpoint{1.220604in}{2.977305in}}{\pgfqpoint{1.216214in}{2.966706in}}{\pgfqpoint{1.216214in}{2.955656in}}%
\pgfpathcurveto{\pgfqpoint{1.216214in}{2.944606in}}{\pgfqpoint{1.220604in}{2.934007in}}{\pgfqpoint{1.228418in}{2.926194in}}%
\pgfpathcurveto{\pgfqpoint{1.236232in}{2.918380in}}{\pgfqpoint{1.246831in}{2.913990in}}{\pgfqpoint{1.257881in}{2.913990in}}%
\pgfpathlineto{\pgfqpoint{1.257881in}{2.913990in}}%
\pgfpathclose%
\pgfusepath{stroke}%
\end{pgfscope}%
\begin{pgfscope}%
\pgfpathrectangle{\pgfqpoint{0.847223in}{0.554012in}}{\pgfqpoint{6.200000in}{4.620000in}}%
\pgfusepath{clip}%
\pgfsetbuttcap%
\pgfsetroundjoin%
\pgfsetlinewidth{1.003750pt}%
\definecolor{currentstroke}{rgb}{1.000000,0.000000,0.000000}%
\pgfsetstrokecolor{currentstroke}%
\pgfsetdash{}{0pt}%
\pgfpathmoveto{\pgfqpoint{1.263214in}{2.897938in}}%
\pgfpathcurveto{\pgfqpoint{1.274264in}{2.897938in}}{\pgfqpoint{1.284863in}{2.902329in}}{\pgfqpoint{1.292677in}{2.910142in}}%
\pgfpathcurveto{\pgfqpoint{1.300490in}{2.917956in}}{\pgfqpoint{1.304881in}{2.928555in}}{\pgfqpoint{1.304881in}{2.939605in}}%
\pgfpathcurveto{\pgfqpoint{1.304881in}{2.950655in}}{\pgfqpoint{1.300490in}{2.961254in}}{\pgfqpoint{1.292677in}{2.969068in}}%
\pgfpathcurveto{\pgfqpoint{1.284863in}{2.976881in}}{\pgfqpoint{1.274264in}{2.981272in}}{\pgfqpoint{1.263214in}{2.981272in}}%
\pgfpathcurveto{\pgfqpoint{1.252164in}{2.981272in}}{\pgfqpoint{1.241565in}{2.976881in}}{\pgfqpoint{1.233751in}{2.969068in}}%
\pgfpathcurveto{\pgfqpoint{1.225937in}{2.961254in}}{\pgfqpoint{1.221547in}{2.950655in}}{\pgfqpoint{1.221547in}{2.939605in}}%
\pgfpathcurveto{\pgfqpoint{1.221547in}{2.928555in}}{\pgfqpoint{1.225937in}{2.917956in}}{\pgfqpoint{1.233751in}{2.910142in}}%
\pgfpathcurveto{\pgfqpoint{1.241565in}{2.902329in}}{\pgfqpoint{1.252164in}{2.897938in}}{\pgfqpoint{1.263214in}{2.897938in}}%
\pgfpathlineto{\pgfqpoint{1.263214in}{2.897938in}}%
\pgfpathclose%
\pgfusepath{stroke}%
\end{pgfscope}%
\begin{pgfscope}%
\pgfpathrectangle{\pgfqpoint{0.847223in}{0.554012in}}{\pgfqpoint{6.200000in}{4.620000in}}%
\pgfusepath{clip}%
\pgfsetbuttcap%
\pgfsetroundjoin%
\pgfsetlinewidth{1.003750pt}%
\definecolor{currentstroke}{rgb}{1.000000,0.000000,0.000000}%
\pgfsetstrokecolor{currentstroke}%
\pgfsetdash{}{0pt}%
\pgfpathmoveto{\pgfqpoint{1.268547in}{2.882068in}}%
\pgfpathcurveto{\pgfqpoint{1.279597in}{2.882068in}}{\pgfqpoint{1.290196in}{2.886459in}}{\pgfqpoint{1.298010in}{2.894272in}}%
\pgfpathcurveto{\pgfqpoint{1.305823in}{2.902086in}}{\pgfqpoint{1.310214in}{2.912685in}}{\pgfqpoint{1.310214in}{2.923735in}}%
\pgfpathcurveto{\pgfqpoint{1.310214in}{2.934785in}}{\pgfqpoint{1.305823in}{2.945384in}}{\pgfqpoint{1.298010in}{2.953198in}}%
\pgfpathcurveto{\pgfqpoint{1.290196in}{2.961011in}}{\pgfqpoint{1.279597in}{2.965402in}}{\pgfqpoint{1.268547in}{2.965402in}}%
\pgfpathcurveto{\pgfqpoint{1.257497in}{2.965402in}}{\pgfqpoint{1.246898in}{2.961011in}}{\pgfqpoint{1.239084in}{2.953198in}}%
\pgfpathcurveto{\pgfqpoint{1.231271in}{2.945384in}}{\pgfqpoint{1.226880in}{2.934785in}}{\pgfqpoint{1.226880in}{2.923735in}}%
\pgfpathcurveto{\pgfqpoint{1.226880in}{2.912685in}}{\pgfqpoint{1.231271in}{2.902086in}}{\pgfqpoint{1.239084in}{2.894272in}}%
\pgfpathcurveto{\pgfqpoint{1.246898in}{2.886459in}}{\pgfqpoint{1.257497in}{2.882068in}}{\pgfqpoint{1.268547in}{2.882068in}}%
\pgfpathlineto{\pgfqpoint{1.268547in}{2.882068in}}%
\pgfpathclose%
\pgfusepath{stroke}%
\end{pgfscope}%
\begin{pgfscope}%
\pgfpathrectangle{\pgfqpoint{0.847223in}{0.554012in}}{\pgfqpoint{6.200000in}{4.620000in}}%
\pgfusepath{clip}%
\pgfsetbuttcap%
\pgfsetroundjoin%
\pgfsetlinewidth{1.003750pt}%
\definecolor{currentstroke}{rgb}{1.000000,0.000000,0.000000}%
\pgfsetstrokecolor{currentstroke}%
\pgfsetdash{}{0pt}%
\pgfpathmoveto{\pgfqpoint{1.273880in}{2.866376in}}%
\pgfpathcurveto{\pgfqpoint{1.284930in}{2.866376in}}{\pgfqpoint{1.295529in}{2.870767in}}{\pgfqpoint{1.303343in}{2.878580in}}%
\pgfpathcurveto{\pgfqpoint{1.311157in}{2.886394in}}{\pgfqpoint{1.315547in}{2.896993in}}{\pgfqpoint{1.315547in}{2.908043in}}%
\pgfpathcurveto{\pgfqpoint{1.315547in}{2.919093in}}{\pgfqpoint{1.311157in}{2.929692in}}{\pgfqpoint{1.303343in}{2.937506in}}%
\pgfpathcurveto{\pgfqpoint{1.295529in}{2.945320in}}{\pgfqpoint{1.284930in}{2.949710in}}{\pgfqpoint{1.273880in}{2.949710in}}%
\pgfpathcurveto{\pgfqpoint{1.262830in}{2.949710in}}{\pgfqpoint{1.252231in}{2.945320in}}{\pgfqpoint{1.244418in}{2.937506in}}%
\pgfpathcurveto{\pgfqpoint{1.236604in}{2.929692in}}{\pgfqpoint{1.232214in}{2.919093in}}{\pgfqpoint{1.232214in}{2.908043in}}%
\pgfpathcurveto{\pgfqpoint{1.232214in}{2.896993in}}{\pgfqpoint{1.236604in}{2.886394in}}{\pgfqpoint{1.244418in}{2.878580in}}%
\pgfpathcurveto{\pgfqpoint{1.252231in}{2.870767in}}{\pgfqpoint{1.262830in}{2.866376in}}{\pgfqpoint{1.273880in}{2.866376in}}%
\pgfpathlineto{\pgfqpoint{1.273880in}{2.866376in}}%
\pgfpathclose%
\pgfusepath{stroke}%
\end{pgfscope}%
\begin{pgfscope}%
\pgfpathrectangle{\pgfqpoint{0.847223in}{0.554012in}}{\pgfqpoint{6.200000in}{4.620000in}}%
\pgfusepath{clip}%
\pgfsetbuttcap%
\pgfsetroundjoin%
\pgfsetlinewidth{1.003750pt}%
\definecolor{currentstroke}{rgb}{1.000000,0.000000,0.000000}%
\pgfsetstrokecolor{currentstroke}%
\pgfsetdash{}{0pt}%
\pgfpathmoveto{\pgfqpoint{1.279214in}{2.850860in}}%
\pgfpathcurveto{\pgfqpoint{1.290264in}{2.850860in}}{\pgfqpoint{1.300863in}{2.855250in}}{\pgfqpoint{1.308676in}{2.863064in}}%
\pgfpathcurveto{\pgfqpoint{1.316490in}{2.870877in}}{\pgfqpoint{1.320880in}{2.881476in}}{\pgfqpoint{1.320880in}{2.892526in}}%
\pgfpathcurveto{\pgfqpoint{1.320880in}{2.903577in}}{\pgfqpoint{1.316490in}{2.914176in}}{\pgfqpoint{1.308676in}{2.921989in}}%
\pgfpathcurveto{\pgfqpoint{1.300863in}{2.929803in}}{\pgfqpoint{1.290264in}{2.934193in}}{\pgfqpoint{1.279214in}{2.934193in}}%
\pgfpathcurveto{\pgfqpoint{1.268163in}{2.934193in}}{\pgfqpoint{1.257564in}{2.929803in}}{\pgfqpoint{1.249751in}{2.921989in}}%
\pgfpathcurveto{\pgfqpoint{1.241937in}{2.914176in}}{\pgfqpoint{1.237547in}{2.903577in}}{\pgfqpoint{1.237547in}{2.892526in}}%
\pgfpathcurveto{\pgfqpoint{1.237547in}{2.881476in}}{\pgfqpoint{1.241937in}{2.870877in}}{\pgfqpoint{1.249751in}{2.863064in}}%
\pgfpathcurveto{\pgfqpoint{1.257564in}{2.855250in}}{\pgfqpoint{1.268163in}{2.850860in}}{\pgfqpoint{1.279214in}{2.850860in}}%
\pgfpathlineto{\pgfqpoint{1.279214in}{2.850860in}}%
\pgfpathclose%
\pgfusepath{stroke}%
\end{pgfscope}%
\begin{pgfscope}%
\pgfpathrectangle{\pgfqpoint{0.847223in}{0.554012in}}{\pgfqpoint{6.200000in}{4.620000in}}%
\pgfusepath{clip}%
\pgfsetbuttcap%
\pgfsetroundjoin%
\pgfsetlinewidth{1.003750pt}%
\definecolor{currentstroke}{rgb}{1.000000,0.000000,0.000000}%
\pgfsetstrokecolor{currentstroke}%
\pgfsetdash{}{0pt}%
\pgfpathmoveto{\pgfqpoint{1.284547in}{2.835515in}}%
\pgfpathcurveto{\pgfqpoint{1.295597in}{2.835515in}}{\pgfqpoint{1.306196in}{2.839906in}}{\pgfqpoint{1.314010in}{2.847719in}}%
\pgfpathcurveto{\pgfqpoint{1.321823in}{2.855533in}}{\pgfqpoint{1.326213in}{2.866132in}}{\pgfqpoint{1.326213in}{2.877182in}}%
\pgfpathcurveto{\pgfqpoint{1.326213in}{2.888232in}}{\pgfqpoint{1.321823in}{2.898831in}}{\pgfqpoint{1.314010in}{2.906645in}}%
\pgfpathcurveto{\pgfqpoint{1.306196in}{2.914459in}}{\pgfqpoint{1.295597in}{2.918849in}}{\pgfqpoint{1.284547in}{2.918849in}}%
\pgfpathcurveto{\pgfqpoint{1.273497in}{2.918849in}}{\pgfqpoint{1.262898in}{2.914459in}}{\pgfqpoint{1.255084in}{2.906645in}}%
\pgfpathcurveto{\pgfqpoint{1.247270in}{2.898831in}}{\pgfqpoint{1.242880in}{2.888232in}}{\pgfqpoint{1.242880in}{2.877182in}}%
\pgfpathcurveto{\pgfqpoint{1.242880in}{2.866132in}}{\pgfqpoint{1.247270in}{2.855533in}}{\pgfqpoint{1.255084in}{2.847719in}}%
\pgfpathcurveto{\pgfqpoint{1.262898in}{2.839906in}}{\pgfqpoint{1.273497in}{2.835515in}}{\pgfqpoint{1.284547in}{2.835515in}}%
\pgfpathlineto{\pgfqpoint{1.284547in}{2.835515in}}%
\pgfpathclose%
\pgfusepath{stroke}%
\end{pgfscope}%
\begin{pgfscope}%
\pgfpathrectangle{\pgfqpoint{0.847223in}{0.554012in}}{\pgfqpoint{6.200000in}{4.620000in}}%
\pgfusepath{clip}%
\pgfsetbuttcap%
\pgfsetroundjoin%
\pgfsetlinewidth{1.003750pt}%
\definecolor{currentstroke}{rgb}{1.000000,0.000000,0.000000}%
\pgfsetstrokecolor{currentstroke}%
\pgfsetdash{}{0pt}%
\pgfpathmoveto{\pgfqpoint{1.289880in}{2.820341in}}%
\pgfpathcurveto{\pgfqpoint{1.300930in}{2.820341in}}{\pgfqpoint{1.311529in}{2.824731in}}{\pgfqpoint{1.319343in}{2.832544in}}%
\pgfpathcurveto{\pgfqpoint{1.327156in}{2.840358in}}{\pgfqpoint{1.331547in}{2.850957in}}{\pgfqpoint{1.331547in}{2.862007in}}%
\pgfpathcurveto{\pgfqpoint{1.331547in}{2.873057in}}{\pgfqpoint{1.327156in}{2.883656in}}{\pgfqpoint{1.319343in}{2.891470in}}%
\pgfpathcurveto{\pgfqpoint{1.311529in}{2.899284in}}{\pgfqpoint{1.300930in}{2.903674in}}{\pgfqpoint{1.289880in}{2.903674in}}%
\pgfpathcurveto{\pgfqpoint{1.278830in}{2.903674in}}{\pgfqpoint{1.268231in}{2.899284in}}{\pgfqpoint{1.260417in}{2.891470in}}%
\pgfpathcurveto{\pgfqpoint{1.252604in}{2.883656in}}{\pgfqpoint{1.248213in}{2.873057in}}{\pgfqpoint{1.248213in}{2.862007in}}%
\pgfpathcurveto{\pgfqpoint{1.248213in}{2.850957in}}{\pgfqpoint{1.252604in}{2.840358in}}{\pgfqpoint{1.260417in}{2.832544in}}%
\pgfpathcurveto{\pgfqpoint{1.268231in}{2.824731in}}{\pgfqpoint{1.278830in}{2.820341in}}{\pgfqpoint{1.289880in}{2.820341in}}%
\pgfpathlineto{\pgfqpoint{1.289880in}{2.820341in}}%
\pgfpathclose%
\pgfusepath{stroke}%
\end{pgfscope}%
\begin{pgfscope}%
\pgfpathrectangle{\pgfqpoint{0.847223in}{0.554012in}}{\pgfqpoint{6.200000in}{4.620000in}}%
\pgfusepath{clip}%
\pgfsetbuttcap%
\pgfsetroundjoin%
\pgfsetlinewidth{1.003750pt}%
\definecolor{currentstroke}{rgb}{1.000000,0.000000,0.000000}%
\pgfsetstrokecolor{currentstroke}%
\pgfsetdash{}{0pt}%
\pgfpathmoveto{\pgfqpoint{1.295213in}{2.805332in}}%
\pgfpathcurveto{\pgfqpoint{1.306263in}{2.805332in}}{\pgfqpoint{1.316862in}{2.809722in}}{\pgfqpoint{1.324676in}{2.817536in}}%
\pgfpathcurveto{\pgfqpoint{1.332490in}{2.825350in}}{\pgfqpoint{1.336880in}{2.835949in}}{\pgfqpoint{1.336880in}{2.846999in}}%
\pgfpathcurveto{\pgfqpoint{1.336880in}{2.858049in}}{\pgfqpoint{1.332490in}{2.868648in}}{\pgfqpoint{1.324676in}{2.876462in}}%
\pgfpathcurveto{\pgfqpoint{1.316862in}{2.884275in}}{\pgfqpoint{1.306263in}{2.888666in}}{\pgfqpoint{1.295213in}{2.888666in}}%
\pgfpathcurveto{\pgfqpoint{1.284163in}{2.888666in}}{\pgfqpoint{1.273564in}{2.884275in}}{\pgfqpoint{1.265750in}{2.876462in}}%
\pgfpathcurveto{\pgfqpoint{1.257937in}{2.868648in}}{\pgfqpoint{1.253547in}{2.858049in}}{\pgfqpoint{1.253547in}{2.846999in}}%
\pgfpathcurveto{\pgfqpoint{1.253547in}{2.835949in}}{\pgfqpoint{1.257937in}{2.825350in}}{\pgfqpoint{1.265750in}{2.817536in}}%
\pgfpathcurveto{\pgfqpoint{1.273564in}{2.809722in}}{\pgfqpoint{1.284163in}{2.805332in}}{\pgfqpoint{1.295213in}{2.805332in}}%
\pgfpathlineto{\pgfqpoint{1.295213in}{2.805332in}}%
\pgfpathclose%
\pgfusepath{stroke}%
\end{pgfscope}%
\begin{pgfscope}%
\pgfpathrectangle{\pgfqpoint{0.847223in}{0.554012in}}{\pgfqpoint{6.200000in}{4.620000in}}%
\pgfusepath{clip}%
\pgfsetbuttcap%
\pgfsetroundjoin%
\pgfsetlinewidth{1.003750pt}%
\definecolor{currentstroke}{rgb}{1.000000,0.000000,0.000000}%
\pgfsetstrokecolor{currentstroke}%
\pgfsetdash{}{0pt}%
\pgfpathmoveto{\pgfqpoint{1.300546in}{2.790488in}}%
\pgfpathcurveto{\pgfqpoint{1.311597in}{2.790488in}}{\pgfqpoint{1.322196in}{2.794878in}}{\pgfqpoint{1.330009in}{2.802692in}}%
\pgfpathcurveto{\pgfqpoint{1.337823in}{2.810505in}}{\pgfqpoint{1.342213in}{2.821104in}}{\pgfqpoint{1.342213in}{2.832154in}}%
\pgfpathcurveto{\pgfqpoint{1.342213in}{2.843205in}}{\pgfqpoint{1.337823in}{2.853804in}}{\pgfqpoint{1.330009in}{2.861617in}}%
\pgfpathcurveto{\pgfqpoint{1.322196in}{2.869431in}}{\pgfqpoint{1.311597in}{2.873821in}}{\pgfqpoint{1.300546in}{2.873821in}}%
\pgfpathcurveto{\pgfqpoint{1.289496in}{2.873821in}}{\pgfqpoint{1.278897in}{2.869431in}}{\pgfqpoint{1.271084in}{2.861617in}}%
\pgfpathcurveto{\pgfqpoint{1.263270in}{2.853804in}}{\pgfqpoint{1.258880in}{2.843205in}}{\pgfqpoint{1.258880in}{2.832154in}}%
\pgfpathcurveto{\pgfqpoint{1.258880in}{2.821104in}}{\pgfqpoint{1.263270in}{2.810505in}}{\pgfqpoint{1.271084in}{2.802692in}}%
\pgfpathcurveto{\pgfqpoint{1.278897in}{2.794878in}}{\pgfqpoint{1.289496in}{2.790488in}}{\pgfqpoint{1.300546in}{2.790488in}}%
\pgfpathlineto{\pgfqpoint{1.300546in}{2.790488in}}%
\pgfpathclose%
\pgfusepath{stroke}%
\end{pgfscope}%
\begin{pgfscope}%
\pgfpathrectangle{\pgfqpoint{0.847223in}{0.554012in}}{\pgfqpoint{6.200000in}{4.620000in}}%
\pgfusepath{clip}%
\pgfsetbuttcap%
\pgfsetroundjoin%
\pgfsetlinewidth{1.003750pt}%
\definecolor{currentstroke}{rgb}{1.000000,0.000000,0.000000}%
\pgfsetstrokecolor{currentstroke}%
\pgfsetdash{}{0pt}%
\pgfpathmoveto{\pgfqpoint{1.305880in}{2.775805in}}%
\pgfpathcurveto{\pgfqpoint{1.316930in}{2.775805in}}{\pgfqpoint{1.327529in}{2.780195in}}{\pgfqpoint{1.335342in}{2.788009in}}%
\pgfpathcurveto{\pgfqpoint{1.343156in}{2.795822in}}{\pgfqpoint{1.347546in}{2.806421in}}{\pgfqpoint{1.347546in}{2.817471in}}%
\pgfpathcurveto{\pgfqpoint{1.347546in}{2.828521in}}{\pgfqpoint{1.343156in}{2.839120in}}{\pgfqpoint{1.335342in}{2.846934in}}%
\pgfpathcurveto{\pgfqpoint{1.327529in}{2.854748in}}{\pgfqpoint{1.316930in}{2.859138in}}{\pgfqpoint{1.305880in}{2.859138in}}%
\pgfpathcurveto{\pgfqpoint{1.294829in}{2.859138in}}{\pgfqpoint{1.284230in}{2.854748in}}{\pgfqpoint{1.276417in}{2.846934in}}%
\pgfpathcurveto{\pgfqpoint{1.268603in}{2.839120in}}{\pgfqpoint{1.264213in}{2.828521in}}{\pgfqpoint{1.264213in}{2.817471in}}%
\pgfpathcurveto{\pgfqpoint{1.264213in}{2.806421in}}{\pgfqpoint{1.268603in}{2.795822in}}{\pgfqpoint{1.276417in}{2.788009in}}%
\pgfpathcurveto{\pgfqpoint{1.284230in}{2.780195in}}{\pgfqpoint{1.294829in}{2.775805in}}{\pgfqpoint{1.305880in}{2.775805in}}%
\pgfpathlineto{\pgfqpoint{1.305880in}{2.775805in}}%
\pgfpathclose%
\pgfusepath{stroke}%
\end{pgfscope}%
\begin{pgfscope}%
\pgfpathrectangle{\pgfqpoint{0.847223in}{0.554012in}}{\pgfqpoint{6.200000in}{4.620000in}}%
\pgfusepath{clip}%
\pgfsetbuttcap%
\pgfsetroundjoin%
\pgfsetlinewidth{1.003750pt}%
\definecolor{currentstroke}{rgb}{1.000000,0.000000,0.000000}%
\pgfsetstrokecolor{currentstroke}%
\pgfsetdash{}{0pt}%
\pgfpathmoveto{\pgfqpoint{1.311213in}{2.761280in}}%
\pgfpathcurveto{\pgfqpoint{1.322263in}{2.761280in}}{\pgfqpoint{1.332862in}{2.765670in}}{\pgfqpoint{1.340676in}{2.773484in}}%
\pgfpathcurveto{\pgfqpoint{1.348489in}{2.781298in}}{\pgfqpoint{1.352879in}{2.791897in}}{\pgfqpoint{1.352879in}{2.802947in}}%
\pgfpathcurveto{\pgfqpoint{1.352879in}{2.813997in}}{\pgfqpoint{1.348489in}{2.824596in}}{\pgfqpoint{1.340676in}{2.832410in}}%
\pgfpathcurveto{\pgfqpoint{1.332862in}{2.840223in}}{\pgfqpoint{1.322263in}{2.844613in}}{\pgfqpoint{1.311213in}{2.844613in}}%
\pgfpathcurveto{\pgfqpoint{1.300163in}{2.844613in}}{\pgfqpoint{1.289564in}{2.840223in}}{\pgfqpoint{1.281750in}{2.832410in}}%
\pgfpathcurveto{\pgfqpoint{1.273936in}{2.824596in}}{\pgfqpoint{1.269546in}{2.813997in}}{\pgfqpoint{1.269546in}{2.802947in}}%
\pgfpathcurveto{\pgfqpoint{1.269546in}{2.791897in}}{\pgfqpoint{1.273936in}{2.781298in}}{\pgfqpoint{1.281750in}{2.773484in}}%
\pgfpathcurveto{\pgfqpoint{1.289564in}{2.765670in}}{\pgfqpoint{1.300163in}{2.761280in}}{\pgfqpoint{1.311213in}{2.761280in}}%
\pgfpathlineto{\pgfqpoint{1.311213in}{2.761280in}}%
\pgfpathclose%
\pgfusepath{stroke}%
\end{pgfscope}%
\begin{pgfscope}%
\pgfpathrectangle{\pgfqpoint{0.847223in}{0.554012in}}{\pgfqpoint{6.200000in}{4.620000in}}%
\pgfusepath{clip}%
\pgfsetbuttcap%
\pgfsetroundjoin%
\pgfsetlinewidth{1.003750pt}%
\definecolor{currentstroke}{rgb}{1.000000,0.000000,0.000000}%
\pgfsetstrokecolor{currentstroke}%
\pgfsetdash{}{0pt}%
\pgfpathmoveto{\pgfqpoint{1.316546in}{2.746912in}}%
\pgfpathcurveto{\pgfqpoint{1.327596in}{2.746912in}}{\pgfqpoint{1.338195in}{2.751302in}}{\pgfqpoint{1.346009in}{2.759116in}}%
\pgfpathcurveto{\pgfqpoint{1.353822in}{2.766929in}}{\pgfqpoint{1.358213in}{2.777528in}}{\pgfqpoint{1.358213in}{2.788578in}}%
\pgfpathcurveto{\pgfqpoint{1.358213in}{2.799628in}}{\pgfqpoint{1.353822in}{2.810227in}}{\pgfqpoint{1.346009in}{2.818041in}}%
\pgfpathcurveto{\pgfqpoint{1.338195in}{2.825855in}}{\pgfqpoint{1.327596in}{2.830245in}}{\pgfqpoint{1.316546in}{2.830245in}}%
\pgfpathcurveto{\pgfqpoint{1.305496in}{2.830245in}}{\pgfqpoint{1.294897in}{2.825855in}}{\pgfqpoint{1.287083in}{2.818041in}}%
\pgfpathcurveto{\pgfqpoint{1.279270in}{2.810227in}}{\pgfqpoint{1.274879in}{2.799628in}}{\pgfqpoint{1.274879in}{2.788578in}}%
\pgfpathcurveto{\pgfqpoint{1.274879in}{2.777528in}}{\pgfqpoint{1.279270in}{2.766929in}}{\pgfqpoint{1.287083in}{2.759116in}}%
\pgfpathcurveto{\pgfqpoint{1.294897in}{2.751302in}}{\pgfqpoint{1.305496in}{2.746912in}}{\pgfqpoint{1.316546in}{2.746912in}}%
\pgfpathlineto{\pgfqpoint{1.316546in}{2.746912in}}%
\pgfpathclose%
\pgfusepath{stroke}%
\end{pgfscope}%
\begin{pgfscope}%
\pgfpathrectangle{\pgfqpoint{0.847223in}{0.554012in}}{\pgfqpoint{6.200000in}{4.620000in}}%
\pgfusepath{clip}%
\pgfsetbuttcap%
\pgfsetroundjoin%
\pgfsetlinewidth{1.003750pt}%
\definecolor{currentstroke}{rgb}{1.000000,0.000000,0.000000}%
\pgfsetstrokecolor{currentstroke}%
\pgfsetdash{}{0pt}%
\pgfpathmoveto{\pgfqpoint{1.321879in}{2.732697in}}%
\pgfpathcurveto{\pgfqpoint{1.332929in}{2.732697in}}{\pgfqpoint{1.343528in}{2.737087in}}{\pgfqpoint{1.351342in}{2.744901in}}%
\pgfpathcurveto{\pgfqpoint{1.359156in}{2.752714in}}{\pgfqpoint{1.363546in}{2.763313in}}{\pgfqpoint{1.363546in}{2.774363in}}%
\pgfpathcurveto{\pgfqpoint{1.363546in}{2.785414in}}{\pgfqpoint{1.359156in}{2.796013in}}{\pgfqpoint{1.351342in}{2.803826in}}%
\pgfpathcurveto{\pgfqpoint{1.343528in}{2.811640in}}{\pgfqpoint{1.332929in}{2.816030in}}{\pgfqpoint{1.321879in}{2.816030in}}%
\pgfpathcurveto{\pgfqpoint{1.310829in}{2.816030in}}{\pgfqpoint{1.300230in}{2.811640in}}{\pgfqpoint{1.292416in}{2.803826in}}%
\pgfpathcurveto{\pgfqpoint{1.284603in}{2.796013in}}{\pgfqpoint{1.280213in}{2.785414in}}{\pgfqpoint{1.280213in}{2.774363in}}%
\pgfpathcurveto{\pgfqpoint{1.280213in}{2.763313in}}{\pgfqpoint{1.284603in}{2.752714in}}{\pgfqpoint{1.292416in}{2.744901in}}%
\pgfpathcurveto{\pgfqpoint{1.300230in}{2.737087in}}{\pgfqpoint{1.310829in}{2.732697in}}{\pgfqpoint{1.321879in}{2.732697in}}%
\pgfpathlineto{\pgfqpoint{1.321879in}{2.732697in}}%
\pgfpathclose%
\pgfusepath{stroke}%
\end{pgfscope}%
\begin{pgfscope}%
\pgfpathrectangle{\pgfqpoint{0.847223in}{0.554012in}}{\pgfqpoint{6.200000in}{4.620000in}}%
\pgfusepath{clip}%
\pgfsetbuttcap%
\pgfsetroundjoin%
\pgfsetlinewidth{1.003750pt}%
\definecolor{currentstroke}{rgb}{1.000000,0.000000,0.000000}%
\pgfsetstrokecolor{currentstroke}%
\pgfsetdash{}{0pt}%
\pgfpathmoveto{\pgfqpoint{1.327212in}{2.718633in}}%
\pgfpathcurveto{\pgfqpoint{1.338263in}{2.718633in}}{\pgfqpoint{1.348862in}{2.723023in}}{\pgfqpoint{1.356675in}{2.730837in}}%
\pgfpathcurveto{\pgfqpoint{1.364489in}{2.738650in}}{\pgfqpoint{1.368879in}{2.749250in}}{\pgfqpoint{1.368879in}{2.760300in}}%
\pgfpathcurveto{\pgfqpoint{1.368879in}{2.771350in}}{\pgfqpoint{1.364489in}{2.781949in}}{\pgfqpoint{1.356675in}{2.789762in}}%
\pgfpathcurveto{\pgfqpoint{1.348862in}{2.797576in}}{\pgfqpoint{1.338263in}{2.801966in}}{\pgfqpoint{1.327212in}{2.801966in}}%
\pgfpathcurveto{\pgfqpoint{1.316162in}{2.801966in}}{\pgfqpoint{1.305563in}{2.797576in}}{\pgfqpoint{1.297750in}{2.789762in}}%
\pgfpathcurveto{\pgfqpoint{1.289936in}{2.781949in}}{\pgfqpoint{1.285546in}{2.771350in}}{\pgfqpoint{1.285546in}{2.760300in}}%
\pgfpathcurveto{\pgfqpoint{1.285546in}{2.749250in}}{\pgfqpoint{1.289936in}{2.738650in}}{\pgfqpoint{1.297750in}{2.730837in}}%
\pgfpathcurveto{\pgfqpoint{1.305563in}{2.723023in}}{\pgfqpoint{1.316162in}{2.718633in}}{\pgfqpoint{1.327212in}{2.718633in}}%
\pgfpathlineto{\pgfqpoint{1.327212in}{2.718633in}}%
\pgfpathclose%
\pgfusepath{stroke}%
\end{pgfscope}%
\begin{pgfscope}%
\pgfpathrectangle{\pgfqpoint{0.847223in}{0.554012in}}{\pgfqpoint{6.200000in}{4.620000in}}%
\pgfusepath{clip}%
\pgfsetbuttcap%
\pgfsetroundjoin%
\pgfsetlinewidth{1.003750pt}%
\definecolor{currentstroke}{rgb}{1.000000,0.000000,0.000000}%
\pgfsetstrokecolor{currentstroke}%
\pgfsetdash{}{0pt}%
\pgfpathmoveto{\pgfqpoint{1.332546in}{2.704718in}}%
\pgfpathcurveto{\pgfqpoint{1.343596in}{2.704718in}}{\pgfqpoint{1.354195in}{2.709108in}}{\pgfqpoint{1.362008in}{2.716922in}}%
\pgfpathcurveto{\pgfqpoint{1.369822in}{2.724735in}}{\pgfqpoint{1.374212in}{2.735334in}}{\pgfqpoint{1.374212in}{2.746385in}}%
\pgfpathcurveto{\pgfqpoint{1.374212in}{2.757435in}}{\pgfqpoint{1.369822in}{2.768034in}}{\pgfqpoint{1.362008in}{2.775847in}}%
\pgfpathcurveto{\pgfqpoint{1.354195in}{2.783661in}}{\pgfqpoint{1.343596in}{2.788051in}}{\pgfqpoint{1.332546in}{2.788051in}}%
\pgfpathcurveto{\pgfqpoint{1.321496in}{2.788051in}}{\pgfqpoint{1.310897in}{2.783661in}}{\pgfqpoint{1.303083in}{2.775847in}}%
\pgfpathcurveto{\pgfqpoint{1.295269in}{2.768034in}}{\pgfqpoint{1.290879in}{2.757435in}}{\pgfqpoint{1.290879in}{2.746385in}}%
\pgfpathcurveto{\pgfqpoint{1.290879in}{2.735334in}}{\pgfqpoint{1.295269in}{2.724735in}}{\pgfqpoint{1.303083in}{2.716922in}}%
\pgfpathcurveto{\pgfqpoint{1.310897in}{2.709108in}}{\pgfqpoint{1.321496in}{2.704718in}}{\pgfqpoint{1.332546in}{2.704718in}}%
\pgfpathlineto{\pgfqpoint{1.332546in}{2.704718in}}%
\pgfpathclose%
\pgfusepath{stroke}%
\end{pgfscope}%
\begin{pgfscope}%
\pgfpathrectangle{\pgfqpoint{0.847223in}{0.554012in}}{\pgfqpoint{6.200000in}{4.620000in}}%
\pgfusepath{clip}%
\pgfsetbuttcap%
\pgfsetroundjoin%
\pgfsetlinewidth{1.003750pt}%
\definecolor{currentstroke}{rgb}{1.000000,0.000000,0.000000}%
\pgfsetstrokecolor{currentstroke}%
\pgfsetdash{}{0pt}%
\pgfpathmoveto{\pgfqpoint{1.337879in}{2.690949in}}%
\pgfpathcurveto{\pgfqpoint{1.348929in}{2.690949in}}{\pgfqpoint{1.359528in}{2.695340in}}{\pgfqpoint{1.367342in}{2.703153in}}%
\pgfpathcurveto{\pgfqpoint{1.375155in}{2.710967in}}{\pgfqpoint{1.379546in}{2.721566in}}{\pgfqpoint{1.379546in}{2.732616in}}%
\pgfpathcurveto{\pgfqpoint{1.379546in}{2.743666in}}{\pgfqpoint{1.375155in}{2.754265in}}{\pgfqpoint{1.367342in}{2.762079in}}%
\pgfpathcurveto{\pgfqpoint{1.359528in}{2.769892in}}{\pgfqpoint{1.348929in}{2.774283in}}{\pgfqpoint{1.337879in}{2.774283in}}%
\pgfpathcurveto{\pgfqpoint{1.326829in}{2.774283in}}{\pgfqpoint{1.316230in}{2.769892in}}{\pgfqpoint{1.308416in}{2.762079in}}%
\pgfpathcurveto{\pgfqpoint{1.300602in}{2.754265in}}{\pgfqpoint{1.296212in}{2.743666in}}{\pgfqpoint{1.296212in}{2.732616in}}%
\pgfpathcurveto{\pgfqpoint{1.296212in}{2.721566in}}{\pgfqpoint{1.300602in}{2.710967in}}{\pgfqpoint{1.308416in}{2.703153in}}%
\pgfpathcurveto{\pgfqpoint{1.316230in}{2.695340in}}{\pgfqpoint{1.326829in}{2.690949in}}{\pgfqpoint{1.337879in}{2.690949in}}%
\pgfpathlineto{\pgfqpoint{1.337879in}{2.690949in}}%
\pgfpathclose%
\pgfusepath{stroke}%
\end{pgfscope}%
\begin{pgfscope}%
\pgfpathrectangle{\pgfqpoint{0.847223in}{0.554012in}}{\pgfqpoint{6.200000in}{4.620000in}}%
\pgfusepath{clip}%
\pgfsetbuttcap%
\pgfsetroundjoin%
\pgfsetlinewidth{1.003750pt}%
\definecolor{currentstroke}{rgb}{1.000000,0.000000,0.000000}%
\pgfsetstrokecolor{currentstroke}%
\pgfsetdash{}{0pt}%
\pgfpathmoveto{\pgfqpoint{1.343212in}{2.677325in}}%
\pgfpathcurveto{\pgfqpoint{1.354262in}{2.677325in}}{\pgfqpoint{1.364861in}{2.681715in}}{\pgfqpoint{1.372675in}{2.689529in}}%
\pgfpathcurveto{\pgfqpoint{1.380489in}{2.697342in}}{\pgfqpoint{1.384879in}{2.707941in}}{\pgfqpoint{1.384879in}{2.718991in}}%
\pgfpathcurveto{\pgfqpoint{1.384879in}{2.730041in}}{\pgfqpoint{1.380489in}{2.740640in}}{\pgfqpoint{1.372675in}{2.748454in}}%
\pgfpathcurveto{\pgfqpoint{1.364861in}{2.756268in}}{\pgfqpoint{1.354262in}{2.760658in}}{\pgfqpoint{1.343212in}{2.760658in}}%
\pgfpathcurveto{\pgfqpoint{1.332162in}{2.760658in}}{\pgfqpoint{1.321563in}{2.756268in}}{\pgfqpoint{1.313749in}{2.748454in}}%
\pgfpathcurveto{\pgfqpoint{1.305936in}{2.740640in}}{\pgfqpoint{1.301545in}{2.730041in}}{\pgfqpoint{1.301545in}{2.718991in}}%
\pgfpathcurveto{\pgfqpoint{1.301545in}{2.707941in}}{\pgfqpoint{1.305936in}{2.697342in}}{\pgfqpoint{1.313749in}{2.689529in}}%
\pgfpathcurveto{\pgfqpoint{1.321563in}{2.681715in}}{\pgfqpoint{1.332162in}{2.677325in}}{\pgfqpoint{1.343212in}{2.677325in}}%
\pgfpathlineto{\pgfqpoint{1.343212in}{2.677325in}}%
\pgfpathclose%
\pgfusepath{stroke}%
\end{pgfscope}%
\begin{pgfscope}%
\pgfpathrectangle{\pgfqpoint{0.847223in}{0.554012in}}{\pgfqpoint{6.200000in}{4.620000in}}%
\pgfusepath{clip}%
\pgfsetbuttcap%
\pgfsetroundjoin%
\pgfsetlinewidth{1.003750pt}%
\definecolor{currentstroke}{rgb}{1.000000,0.000000,0.000000}%
\pgfsetstrokecolor{currentstroke}%
\pgfsetdash{}{0pt}%
\pgfpathmoveto{\pgfqpoint{1.348545in}{2.663842in}}%
\pgfpathcurveto{\pgfqpoint{1.359595in}{2.663842in}}{\pgfqpoint{1.370194in}{2.668232in}}{\pgfqpoint{1.378008in}{2.676046in}}%
\pgfpathcurveto{\pgfqpoint{1.385822in}{2.683859in}}{\pgfqpoint{1.390212in}{2.694458in}}{\pgfqpoint{1.390212in}{2.705509in}}%
\pgfpathcurveto{\pgfqpoint{1.390212in}{2.716559in}}{\pgfqpoint{1.385822in}{2.727158in}}{\pgfqpoint{1.378008in}{2.734971in}}%
\pgfpathcurveto{\pgfqpoint{1.370194in}{2.742785in}}{\pgfqpoint{1.359595in}{2.747175in}}{\pgfqpoint{1.348545in}{2.747175in}}%
\pgfpathcurveto{\pgfqpoint{1.337495in}{2.747175in}}{\pgfqpoint{1.326896in}{2.742785in}}{\pgfqpoint{1.319083in}{2.734971in}}%
\pgfpathcurveto{\pgfqpoint{1.311269in}{2.727158in}}{\pgfqpoint{1.306879in}{2.716559in}}{\pgfqpoint{1.306879in}{2.705509in}}%
\pgfpathcurveto{\pgfqpoint{1.306879in}{2.694458in}}{\pgfqpoint{1.311269in}{2.683859in}}{\pgfqpoint{1.319083in}{2.676046in}}%
\pgfpathcurveto{\pgfqpoint{1.326896in}{2.668232in}}{\pgfqpoint{1.337495in}{2.663842in}}{\pgfqpoint{1.348545in}{2.663842in}}%
\pgfpathlineto{\pgfqpoint{1.348545in}{2.663842in}}%
\pgfpathclose%
\pgfusepath{stroke}%
\end{pgfscope}%
\begin{pgfscope}%
\pgfpathrectangle{\pgfqpoint{0.847223in}{0.554012in}}{\pgfqpoint{6.200000in}{4.620000in}}%
\pgfusepath{clip}%
\pgfsetbuttcap%
\pgfsetroundjoin%
\pgfsetlinewidth{1.003750pt}%
\definecolor{currentstroke}{rgb}{1.000000,0.000000,0.000000}%
\pgfsetstrokecolor{currentstroke}%
\pgfsetdash{}{0pt}%
\pgfpathmoveto{\pgfqpoint{1.353879in}{2.650499in}}%
\pgfpathcurveto{\pgfqpoint{1.364929in}{2.650499in}}{\pgfqpoint{1.375528in}{2.654889in}}{\pgfqpoint{1.383341in}{2.662703in}}%
\pgfpathcurveto{\pgfqpoint{1.391155in}{2.670516in}}{\pgfqpoint{1.395545in}{2.681115in}}{\pgfqpoint{1.395545in}{2.692165in}}%
\pgfpathcurveto{\pgfqpoint{1.395545in}{2.703215in}}{\pgfqpoint{1.391155in}{2.713815in}}{\pgfqpoint{1.383341in}{2.721628in}}%
\pgfpathcurveto{\pgfqpoint{1.375528in}{2.729442in}}{\pgfqpoint{1.364929in}{2.733832in}}{\pgfqpoint{1.353879in}{2.733832in}}%
\pgfpathcurveto{\pgfqpoint{1.342828in}{2.733832in}}{\pgfqpoint{1.332229in}{2.729442in}}{\pgfqpoint{1.324416in}{2.721628in}}%
\pgfpathcurveto{\pgfqpoint{1.316602in}{2.713815in}}{\pgfqpoint{1.312212in}{2.703215in}}{\pgfqpoint{1.312212in}{2.692165in}}%
\pgfpathcurveto{\pgfqpoint{1.312212in}{2.681115in}}{\pgfqpoint{1.316602in}{2.670516in}}{\pgfqpoint{1.324416in}{2.662703in}}%
\pgfpathcurveto{\pgfqpoint{1.332229in}{2.654889in}}{\pgfqpoint{1.342828in}{2.650499in}}{\pgfqpoint{1.353879in}{2.650499in}}%
\pgfpathlineto{\pgfqpoint{1.353879in}{2.650499in}}%
\pgfpathclose%
\pgfusepath{stroke}%
\end{pgfscope}%
\begin{pgfscope}%
\pgfpathrectangle{\pgfqpoint{0.847223in}{0.554012in}}{\pgfqpoint{6.200000in}{4.620000in}}%
\pgfusepath{clip}%
\pgfsetbuttcap%
\pgfsetroundjoin%
\pgfsetlinewidth{1.003750pt}%
\definecolor{currentstroke}{rgb}{1.000000,0.000000,0.000000}%
\pgfsetstrokecolor{currentstroke}%
\pgfsetdash{}{0pt}%
\pgfpathmoveto{\pgfqpoint{1.359212in}{2.637293in}}%
\pgfpathcurveto{\pgfqpoint{1.370262in}{2.637293in}}{\pgfqpoint{1.380861in}{2.641683in}}{\pgfqpoint{1.388675in}{2.649497in}}%
\pgfpathcurveto{\pgfqpoint{1.396488in}{2.657311in}}{\pgfqpoint{1.400878in}{2.667910in}}{\pgfqpoint{1.400878in}{2.678960in}}%
\pgfpathcurveto{\pgfqpoint{1.400878in}{2.690010in}}{\pgfqpoint{1.396488in}{2.700609in}}{\pgfqpoint{1.388675in}{2.708422in}}%
\pgfpathcurveto{\pgfqpoint{1.380861in}{2.716236in}}{\pgfqpoint{1.370262in}{2.720626in}}{\pgfqpoint{1.359212in}{2.720626in}}%
\pgfpathcurveto{\pgfqpoint{1.348162in}{2.720626in}}{\pgfqpoint{1.337563in}{2.716236in}}{\pgfqpoint{1.329749in}{2.708422in}}%
\pgfpathcurveto{\pgfqpoint{1.321935in}{2.700609in}}{\pgfqpoint{1.317545in}{2.690010in}}{\pgfqpoint{1.317545in}{2.678960in}}%
\pgfpathcurveto{\pgfqpoint{1.317545in}{2.667910in}}{\pgfqpoint{1.321935in}{2.657311in}}{\pgfqpoint{1.329749in}{2.649497in}}%
\pgfpathcurveto{\pgfqpoint{1.337563in}{2.641683in}}{\pgfqpoint{1.348162in}{2.637293in}}{\pgfqpoint{1.359212in}{2.637293in}}%
\pgfpathlineto{\pgfqpoint{1.359212in}{2.637293in}}%
\pgfpathclose%
\pgfusepath{stroke}%
\end{pgfscope}%
\begin{pgfscope}%
\pgfpathrectangle{\pgfqpoint{0.847223in}{0.554012in}}{\pgfqpoint{6.200000in}{4.620000in}}%
\pgfusepath{clip}%
\pgfsetbuttcap%
\pgfsetroundjoin%
\pgfsetlinewidth{1.003750pt}%
\definecolor{currentstroke}{rgb}{1.000000,0.000000,0.000000}%
\pgfsetstrokecolor{currentstroke}%
\pgfsetdash{}{0pt}%
\pgfpathmoveto{\pgfqpoint{1.364545in}{2.624223in}}%
\pgfpathcurveto{\pgfqpoint{1.375595in}{2.624223in}}{\pgfqpoint{1.386194in}{2.628613in}}{\pgfqpoint{1.394008in}{2.636427in}}%
\pgfpathcurveto{\pgfqpoint{1.401821in}{2.644240in}}{\pgfqpoint{1.406212in}{2.654839in}}{\pgfqpoint{1.406212in}{2.665889in}}%
\pgfpathcurveto{\pgfqpoint{1.406212in}{2.676939in}}{\pgfqpoint{1.401821in}{2.687538in}}{\pgfqpoint{1.394008in}{2.695352in}}%
\pgfpathcurveto{\pgfqpoint{1.386194in}{2.703166in}}{\pgfqpoint{1.375595in}{2.707556in}}{\pgfqpoint{1.364545in}{2.707556in}}%
\pgfpathcurveto{\pgfqpoint{1.353495in}{2.707556in}}{\pgfqpoint{1.342896in}{2.703166in}}{\pgfqpoint{1.335082in}{2.695352in}}%
\pgfpathcurveto{\pgfqpoint{1.327269in}{2.687538in}}{\pgfqpoint{1.322878in}{2.676939in}}{\pgfqpoint{1.322878in}{2.665889in}}%
\pgfpathcurveto{\pgfqpoint{1.322878in}{2.654839in}}{\pgfqpoint{1.327269in}{2.644240in}}{\pgfqpoint{1.335082in}{2.636427in}}%
\pgfpathcurveto{\pgfqpoint{1.342896in}{2.628613in}}{\pgfqpoint{1.353495in}{2.624223in}}{\pgfqpoint{1.364545in}{2.624223in}}%
\pgfpathlineto{\pgfqpoint{1.364545in}{2.624223in}}%
\pgfpathclose%
\pgfusepath{stroke}%
\end{pgfscope}%
\begin{pgfscope}%
\pgfpathrectangle{\pgfqpoint{0.847223in}{0.554012in}}{\pgfqpoint{6.200000in}{4.620000in}}%
\pgfusepath{clip}%
\pgfsetbuttcap%
\pgfsetroundjoin%
\pgfsetlinewidth{1.003750pt}%
\definecolor{currentstroke}{rgb}{1.000000,0.000000,0.000000}%
\pgfsetstrokecolor{currentstroke}%
\pgfsetdash{}{0pt}%
\pgfpathmoveto{\pgfqpoint{1.369878in}{2.611286in}}%
\pgfpathcurveto{\pgfqpoint{1.380928in}{2.611286in}}{\pgfqpoint{1.391527in}{2.615676in}}{\pgfqpoint{1.399341in}{2.623489in}}%
\pgfpathcurveto{\pgfqpoint{1.407155in}{2.631303in}}{\pgfqpoint{1.411545in}{2.641902in}}{\pgfqpoint{1.411545in}{2.652952in}}%
\pgfpathcurveto{\pgfqpoint{1.411545in}{2.664002in}}{\pgfqpoint{1.407155in}{2.674601in}}{\pgfqpoint{1.399341in}{2.682415in}}%
\pgfpathcurveto{\pgfqpoint{1.391527in}{2.690229in}}{\pgfqpoint{1.380928in}{2.694619in}}{\pgfqpoint{1.369878in}{2.694619in}}%
\pgfpathcurveto{\pgfqpoint{1.358828in}{2.694619in}}{\pgfqpoint{1.348229in}{2.690229in}}{\pgfqpoint{1.340415in}{2.682415in}}%
\pgfpathcurveto{\pgfqpoint{1.332602in}{2.674601in}}{\pgfqpoint{1.328212in}{2.664002in}}{\pgfqpoint{1.328212in}{2.652952in}}%
\pgfpathcurveto{\pgfqpoint{1.328212in}{2.641902in}}{\pgfqpoint{1.332602in}{2.631303in}}{\pgfqpoint{1.340415in}{2.623489in}}%
\pgfpathcurveto{\pgfqpoint{1.348229in}{2.615676in}}{\pgfqpoint{1.358828in}{2.611286in}}{\pgfqpoint{1.369878in}{2.611286in}}%
\pgfpathlineto{\pgfqpoint{1.369878in}{2.611286in}}%
\pgfpathclose%
\pgfusepath{stroke}%
\end{pgfscope}%
\begin{pgfscope}%
\pgfpathrectangle{\pgfqpoint{0.847223in}{0.554012in}}{\pgfqpoint{6.200000in}{4.620000in}}%
\pgfusepath{clip}%
\pgfsetbuttcap%
\pgfsetroundjoin%
\pgfsetlinewidth{1.003750pt}%
\definecolor{currentstroke}{rgb}{1.000000,0.000000,0.000000}%
\pgfsetstrokecolor{currentstroke}%
\pgfsetdash{}{0pt}%
\pgfpathmoveto{\pgfqpoint{1.375211in}{2.598480in}}%
\pgfpathcurveto{\pgfqpoint{1.386262in}{2.598480in}}{\pgfqpoint{1.396861in}{2.602870in}}{\pgfqpoint{1.404674in}{2.610684in}}%
\pgfpathcurveto{\pgfqpoint{1.412488in}{2.618497in}}{\pgfqpoint{1.416878in}{2.629096in}}{\pgfqpoint{1.416878in}{2.640146in}}%
\pgfpathcurveto{\pgfqpoint{1.416878in}{2.651197in}}{\pgfqpoint{1.412488in}{2.661796in}}{\pgfqpoint{1.404674in}{2.669609in}}%
\pgfpathcurveto{\pgfqpoint{1.396861in}{2.677423in}}{\pgfqpoint{1.386262in}{2.681813in}}{\pgfqpoint{1.375211in}{2.681813in}}%
\pgfpathcurveto{\pgfqpoint{1.364161in}{2.681813in}}{\pgfqpoint{1.353562in}{2.677423in}}{\pgfqpoint{1.345749in}{2.669609in}}%
\pgfpathcurveto{\pgfqpoint{1.337935in}{2.661796in}}{\pgfqpoint{1.333545in}{2.651197in}}{\pgfqpoint{1.333545in}{2.640146in}}%
\pgfpathcurveto{\pgfqpoint{1.333545in}{2.629096in}}{\pgfqpoint{1.337935in}{2.618497in}}{\pgfqpoint{1.345749in}{2.610684in}}%
\pgfpathcurveto{\pgfqpoint{1.353562in}{2.602870in}}{\pgfqpoint{1.364161in}{2.598480in}}{\pgfqpoint{1.375211in}{2.598480in}}%
\pgfpathlineto{\pgfqpoint{1.375211in}{2.598480in}}%
\pgfpathclose%
\pgfusepath{stroke}%
\end{pgfscope}%
\begin{pgfscope}%
\pgfpathrectangle{\pgfqpoint{0.847223in}{0.554012in}}{\pgfqpoint{6.200000in}{4.620000in}}%
\pgfusepath{clip}%
\pgfsetbuttcap%
\pgfsetroundjoin%
\pgfsetlinewidth{1.003750pt}%
\definecolor{currentstroke}{rgb}{1.000000,0.000000,0.000000}%
\pgfsetstrokecolor{currentstroke}%
\pgfsetdash{}{0pt}%
\pgfpathmoveto{\pgfqpoint{1.380545in}{2.585803in}}%
\pgfpathcurveto{\pgfqpoint{1.391595in}{2.585803in}}{\pgfqpoint{1.402194in}{2.590194in}}{\pgfqpoint{1.410007in}{2.598007in}}%
\pgfpathcurveto{\pgfqpoint{1.417821in}{2.605821in}}{\pgfqpoint{1.422211in}{2.616420in}}{\pgfqpoint{1.422211in}{2.627470in}}%
\pgfpathcurveto{\pgfqpoint{1.422211in}{2.638520in}}{\pgfqpoint{1.417821in}{2.649119in}}{\pgfqpoint{1.410007in}{2.656933in}}%
\pgfpathcurveto{\pgfqpoint{1.402194in}{2.664746in}}{\pgfqpoint{1.391595in}{2.669137in}}{\pgfqpoint{1.380545in}{2.669137in}}%
\pgfpathcurveto{\pgfqpoint{1.369494in}{2.669137in}}{\pgfqpoint{1.358895in}{2.664746in}}{\pgfqpoint{1.351082in}{2.656933in}}%
\pgfpathcurveto{\pgfqpoint{1.343268in}{2.649119in}}{\pgfqpoint{1.338878in}{2.638520in}}{\pgfqpoint{1.338878in}{2.627470in}}%
\pgfpathcurveto{\pgfqpoint{1.338878in}{2.616420in}}{\pgfqpoint{1.343268in}{2.605821in}}{\pgfqpoint{1.351082in}{2.598007in}}%
\pgfpathcurveto{\pgfqpoint{1.358895in}{2.590194in}}{\pgfqpoint{1.369494in}{2.585803in}}{\pgfqpoint{1.380545in}{2.585803in}}%
\pgfpathlineto{\pgfqpoint{1.380545in}{2.585803in}}%
\pgfpathclose%
\pgfusepath{stroke}%
\end{pgfscope}%
\begin{pgfscope}%
\pgfpathrectangle{\pgfqpoint{0.847223in}{0.554012in}}{\pgfqpoint{6.200000in}{4.620000in}}%
\pgfusepath{clip}%
\pgfsetbuttcap%
\pgfsetroundjoin%
\pgfsetlinewidth{1.003750pt}%
\definecolor{currentstroke}{rgb}{1.000000,0.000000,0.000000}%
\pgfsetstrokecolor{currentstroke}%
\pgfsetdash{}{0pt}%
\pgfpathmoveto{\pgfqpoint{1.385878in}{2.573254in}}%
\pgfpathcurveto{\pgfqpoint{1.396928in}{2.573254in}}{\pgfqpoint{1.407527in}{2.577644in}}{\pgfqpoint{1.415341in}{2.585458in}}%
\pgfpathcurveto{\pgfqpoint{1.423154in}{2.593272in}}{\pgfqpoint{1.427545in}{2.603871in}}{\pgfqpoint{1.427545in}{2.614921in}}%
\pgfpathcurveto{\pgfqpoint{1.427545in}{2.625971in}}{\pgfqpoint{1.423154in}{2.636570in}}{\pgfqpoint{1.415341in}{2.644384in}}%
\pgfpathcurveto{\pgfqpoint{1.407527in}{2.652197in}}{\pgfqpoint{1.396928in}{2.656587in}}{\pgfqpoint{1.385878in}{2.656587in}}%
\pgfpathcurveto{\pgfqpoint{1.374828in}{2.656587in}}{\pgfqpoint{1.364229in}{2.652197in}}{\pgfqpoint{1.356415in}{2.644384in}}%
\pgfpathcurveto{\pgfqpoint{1.348601in}{2.636570in}}{\pgfqpoint{1.344211in}{2.625971in}}{\pgfqpoint{1.344211in}{2.614921in}}%
\pgfpathcurveto{\pgfqpoint{1.344211in}{2.603871in}}{\pgfqpoint{1.348601in}{2.593272in}}{\pgfqpoint{1.356415in}{2.585458in}}%
\pgfpathcurveto{\pgfqpoint{1.364229in}{2.577644in}}{\pgfqpoint{1.374828in}{2.573254in}}{\pgfqpoint{1.385878in}{2.573254in}}%
\pgfpathlineto{\pgfqpoint{1.385878in}{2.573254in}}%
\pgfpathclose%
\pgfusepath{stroke}%
\end{pgfscope}%
\begin{pgfscope}%
\pgfpathrectangle{\pgfqpoint{0.847223in}{0.554012in}}{\pgfqpoint{6.200000in}{4.620000in}}%
\pgfusepath{clip}%
\pgfsetbuttcap%
\pgfsetroundjoin%
\pgfsetlinewidth{1.003750pt}%
\definecolor{currentstroke}{rgb}{1.000000,0.000000,0.000000}%
\pgfsetstrokecolor{currentstroke}%
\pgfsetdash{}{0pt}%
\pgfpathmoveto{\pgfqpoint{1.391211in}{2.560830in}}%
\pgfpathcurveto{\pgfqpoint{1.402261in}{2.560830in}}{\pgfqpoint{1.412860in}{2.565221in}}{\pgfqpoint{1.420674in}{2.573034in}}%
\pgfpathcurveto{\pgfqpoint{1.428487in}{2.580848in}}{\pgfqpoint{1.432878in}{2.591447in}}{\pgfqpoint{1.432878in}{2.602497in}}%
\pgfpathcurveto{\pgfqpoint{1.432878in}{2.613547in}}{\pgfqpoint{1.428487in}{2.624146in}}{\pgfqpoint{1.420674in}{2.631960in}}%
\pgfpathcurveto{\pgfqpoint{1.412860in}{2.639773in}}{\pgfqpoint{1.402261in}{2.644164in}}{\pgfqpoint{1.391211in}{2.644164in}}%
\pgfpathcurveto{\pgfqpoint{1.380161in}{2.644164in}}{\pgfqpoint{1.369562in}{2.639773in}}{\pgfqpoint{1.361748in}{2.631960in}}%
\pgfpathcurveto{\pgfqpoint{1.353935in}{2.624146in}}{\pgfqpoint{1.349544in}{2.613547in}}{\pgfqpoint{1.349544in}{2.602497in}}%
\pgfpathcurveto{\pgfqpoint{1.349544in}{2.591447in}}{\pgfqpoint{1.353935in}{2.580848in}}{\pgfqpoint{1.361748in}{2.573034in}}%
\pgfpathcurveto{\pgfqpoint{1.369562in}{2.565221in}}{\pgfqpoint{1.380161in}{2.560830in}}{\pgfqpoint{1.391211in}{2.560830in}}%
\pgfpathlineto{\pgfqpoint{1.391211in}{2.560830in}}%
\pgfpathclose%
\pgfusepath{stroke}%
\end{pgfscope}%
\begin{pgfscope}%
\pgfpathrectangle{\pgfqpoint{0.847223in}{0.554012in}}{\pgfqpoint{6.200000in}{4.620000in}}%
\pgfusepath{clip}%
\pgfsetbuttcap%
\pgfsetroundjoin%
\pgfsetlinewidth{1.003750pt}%
\definecolor{currentstroke}{rgb}{1.000000,0.000000,0.000000}%
\pgfsetstrokecolor{currentstroke}%
\pgfsetdash{}{0pt}%
\pgfpathmoveto{\pgfqpoint{1.396544in}{2.548530in}}%
\pgfpathcurveto{\pgfqpoint{1.407594in}{2.548530in}}{\pgfqpoint{1.418193in}{2.552920in}}{\pgfqpoint{1.426007in}{2.560734in}}%
\pgfpathcurveto{\pgfqpoint{1.433821in}{2.568547in}}{\pgfqpoint{1.438211in}{2.579147in}}{\pgfqpoint{1.438211in}{2.590197in}}%
\pgfpathcurveto{\pgfqpoint{1.438211in}{2.601247in}}{\pgfqpoint{1.433821in}{2.611846in}}{\pgfqpoint{1.426007in}{2.619659in}}%
\pgfpathcurveto{\pgfqpoint{1.418193in}{2.627473in}}{\pgfqpoint{1.407594in}{2.631863in}}{\pgfqpoint{1.396544in}{2.631863in}}%
\pgfpathcurveto{\pgfqpoint{1.385494in}{2.631863in}}{\pgfqpoint{1.374895in}{2.627473in}}{\pgfqpoint{1.367081in}{2.619659in}}%
\pgfpathcurveto{\pgfqpoint{1.359268in}{2.611846in}}{\pgfqpoint{1.354878in}{2.601247in}}{\pgfqpoint{1.354878in}{2.590197in}}%
\pgfpathcurveto{\pgfqpoint{1.354878in}{2.579147in}}{\pgfqpoint{1.359268in}{2.568547in}}{\pgfqpoint{1.367081in}{2.560734in}}%
\pgfpathcurveto{\pgfqpoint{1.374895in}{2.552920in}}{\pgfqpoint{1.385494in}{2.548530in}}{\pgfqpoint{1.396544in}{2.548530in}}%
\pgfpathlineto{\pgfqpoint{1.396544in}{2.548530in}}%
\pgfpathclose%
\pgfusepath{stroke}%
\end{pgfscope}%
\begin{pgfscope}%
\pgfpathrectangle{\pgfqpoint{0.847223in}{0.554012in}}{\pgfqpoint{6.200000in}{4.620000in}}%
\pgfusepath{clip}%
\pgfsetbuttcap%
\pgfsetroundjoin%
\pgfsetlinewidth{1.003750pt}%
\definecolor{currentstroke}{rgb}{1.000000,0.000000,0.000000}%
\pgfsetstrokecolor{currentstroke}%
\pgfsetdash{}{0pt}%
\pgfpathmoveto{\pgfqpoint{1.401877in}{2.536351in}}%
\pgfpathcurveto{\pgfqpoint{1.412928in}{2.536351in}}{\pgfqpoint{1.423527in}{2.540742in}}{\pgfqpoint{1.431340in}{2.548555in}}%
\pgfpathcurveto{\pgfqpoint{1.439154in}{2.556369in}}{\pgfqpoint{1.443544in}{2.566968in}}{\pgfqpoint{1.443544in}{2.578018in}}%
\pgfpathcurveto{\pgfqpoint{1.443544in}{2.589068in}}{\pgfqpoint{1.439154in}{2.599667in}}{\pgfqpoint{1.431340in}{2.607481in}}%
\pgfpathcurveto{\pgfqpoint{1.423527in}{2.615294in}}{\pgfqpoint{1.412928in}{2.619685in}}{\pgfqpoint{1.401877in}{2.619685in}}%
\pgfpathcurveto{\pgfqpoint{1.390827in}{2.619685in}}{\pgfqpoint{1.380228in}{2.615294in}}{\pgfqpoint{1.372415in}{2.607481in}}%
\pgfpathcurveto{\pgfqpoint{1.364601in}{2.599667in}}{\pgfqpoint{1.360211in}{2.589068in}}{\pgfqpoint{1.360211in}{2.578018in}}%
\pgfpathcurveto{\pgfqpoint{1.360211in}{2.566968in}}{\pgfqpoint{1.364601in}{2.556369in}}{\pgfqpoint{1.372415in}{2.548555in}}%
\pgfpathcurveto{\pgfqpoint{1.380228in}{2.540742in}}{\pgfqpoint{1.390827in}{2.536351in}}{\pgfqpoint{1.401877in}{2.536351in}}%
\pgfpathlineto{\pgfqpoint{1.401877in}{2.536351in}}%
\pgfpathclose%
\pgfusepath{stroke}%
\end{pgfscope}%
\begin{pgfscope}%
\pgfpathrectangle{\pgfqpoint{0.847223in}{0.554012in}}{\pgfqpoint{6.200000in}{4.620000in}}%
\pgfusepath{clip}%
\pgfsetbuttcap%
\pgfsetroundjoin%
\pgfsetlinewidth{1.003750pt}%
\definecolor{currentstroke}{rgb}{1.000000,0.000000,0.000000}%
\pgfsetstrokecolor{currentstroke}%
\pgfsetdash{}{0pt}%
\pgfpathmoveto{\pgfqpoint{1.407211in}{2.524293in}}%
\pgfpathcurveto{\pgfqpoint{1.418261in}{2.524293in}}{\pgfqpoint{1.428860in}{2.528683in}}{\pgfqpoint{1.436673in}{2.536497in}}%
\pgfpathcurveto{\pgfqpoint{1.444487in}{2.544310in}}{\pgfqpoint{1.448877in}{2.554909in}}{\pgfqpoint{1.448877in}{2.565959in}}%
\pgfpathcurveto{\pgfqpoint{1.448877in}{2.577010in}}{\pgfqpoint{1.444487in}{2.587609in}}{\pgfqpoint{1.436673in}{2.595422in}}%
\pgfpathcurveto{\pgfqpoint{1.428860in}{2.603236in}}{\pgfqpoint{1.418261in}{2.607626in}}{\pgfqpoint{1.407211in}{2.607626in}}%
\pgfpathcurveto{\pgfqpoint{1.396161in}{2.607626in}}{\pgfqpoint{1.385562in}{2.603236in}}{\pgfqpoint{1.377748in}{2.595422in}}%
\pgfpathcurveto{\pgfqpoint{1.369934in}{2.587609in}}{\pgfqpoint{1.365544in}{2.577010in}}{\pgfqpoint{1.365544in}{2.565959in}}%
\pgfpathcurveto{\pgfqpoint{1.365544in}{2.554909in}}{\pgfqpoint{1.369934in}{2.544310in}}{\pgfqpoint{1.377748in}{2.536497in}}%
\pgfpathcurveto{\pgfqpoint{1.385562in}{2.528683in}}{\pgfqpoint{1.396161in}{2.524293in}}{\pgfqpoint{1.407211in}{2.524293in}}%
\pgfpathlineto{\pgfqpoint{1.407211in}{2.524293in}}%
\pgfpathclose%
\pgfusepath{stroke}%
\end{pgfscope}%
\begin{pgfscope}%
\pgfpathrectangle{\pgfqpoint{0.847223in}{0.554012in}}{\pgfqpoint{6.200000in}{4.620000in}}%
\pgfusepath{clip}%
\pgfsetbuttcap%
\pgfsetroundjoin%
\pgfsetlinewidth{1.003750pt}%
\definecolor{currentstroke}{rgb}{1.000000,0.000000,0.000000}%
\pgfsetstrokecolor{currentstroke}%
\pgfsetdash{}{0pt}%
\pgfpathmoveto{\pgfqpoint{1.412544in}{2.512352in}}%
\pgfpathcurveto{\pgfqpoint{1.423594in}{2.512352in}}{\pgfqpoint{1.434193in}{2.516743in}}{\pgfqpoint{1.442007in}{2.524556in}}%
\pgfpathcurveto{\pgfqpoint{1.449820in}{2.532370in}}{\pgfqpoint{1.454211in}{2.542969in}}{\pgfqpoint{1.454211in}{2.554019in}}%
\pgfpathcurveto{\pgfqpoint{1.454211in}{2.565069in}}{\pgfqpoint{1.449820in}{2.575668in}}{\pgfqpoint{1.442007in}{2.583482in}}%
\pgfpathcurveto{\pgfqpoint{1.434193in}{2.591295in}}{\pgfqpoint{1.423594in}{2.595686in}}{\pgfqpoint{1.412544in}{2.595686in}}%
\pgfpathcurveto{\pgfqpoint{1.401494in}{2.595686in}}{\pgfqpoint{1.390895in}{2.591295in}}{\pgfqpoint{1.383081in}{2.583482in}}%
\pgfpathcurveto{\pgfqpoint{1.375268in}{2.575668in}}{\pgfqpoint{1.370877in}{2.565069in}}{\pgfqpoint{1.370877in}{2.554019in}}%
\pgfpathcurveto{\pgfqpoint{1.370877in}{2.542969in}}{\pgfqpoint{1.375268in}{2.532370in}}{\pgfqpoint{1.383081in}{2.524556in}}%
\pgfpathcurveto{\pgfqpoint{1.390895in}{2.516743in}}{\pgfqpoint{1.401494in}{2.512352in}}{\pgfqpoint{1.412544in}{2.512352in}}%
\pgfpathlineto{\pgfqpoint{1.412544in}{2.512352in}}%
\pgfpathclose%
\pgfusepath{stroke}%
\end{pgfscope}%
\begin{pgfscope}%
\pgfpathrectangle{\pgfqpoint{0.847223in}{0.554012in}}{\pgfqpoint{6.200000in}{4.620000in}}%
\pgfusepath{clip}%
\pgfsetbuttcap%
\pgfsetroundjoin%
\pgfsetlinewidth{1.003750pt}%
\definecolor{currentstroke}{rgb}{1.000000,0.000000,0.000000}%
\pgfsetstrokecolor{currentstroke}%
\pgfsetdash{}{0pt}%
\pgfpathmoveto{\pgfqpoint{1.417877in}{2.500528in}}%
\pgfpathcurveto{\pgfqpoint{1.428927in}{2.500528in}}{\pgfqpoint{1.439526in}{2.504918in}}{\pgfqpoint{1.447340in}{2.512732in}}%
\pgfpathcurveto{\pgfqpoint{1.455154in}{2.520546in}}{\pgfqpoint{1.459544in}{2.531145in}}{\pgfqpoint{1.459544in}{2.542195in}}%
\pgfpathcurveto{\pgfqpoint{1.459544in}{2.553245in}}{\pgfqpoint{1.455154in}{2.563844in}}{\pgfqpoint{1.447340in}{2.571658in}}%
\pgfpathcurveto{\pgfqpoint{1.439526in}{2.579471in}}{\pgfqpoint{1.428927in}{2.583862in}}{\pgfqpoint{1.417877in}{2.583862in}}%
\pgfpathcurveto{\pgfqpoint{1.406827in}{2.583862in}}{\pgfqpoint{1.396228in}{2.579471in}}{\pgfqpoint{1.388414in}{2.571658in}}%
\pgfpathcurveto{\pgfqpoint{1.380601in}{2.563844in}}{\pgfqpoint{1.376210in}{2.553245in}}{\pgfqpoint{1.376210in}{2.542195in}}%
\pgfpathcurveto{\pgfqpoint{1.376210in}{2.531145in}}{\pgfqpoint{1.380601in}{2.520546in}}{\pgfqpoint{1.388414in}{2.512732in}}%
\pgfpathcurveto{\pgfqpoint{1.396228in}{2.504918in}}{\pgfqpoint{1.406827in}{2.500528in}}{\pgfqpoint{1.417877in}{2.500528in}}%
\pgfpathlineto{\pgfqpoint{1.417877in}{2.500528in}}%
\pgfpathclose%
\pgfusepath{stroke}%
\end{pgfscope}%
\begin{pgfscope}%
\pgfpathrectangle{\pgfqpoint{0.847223in}{0.554012in}}{\pgfqpoint{6.200000in}{4.620000in}}%
\pgfusepath{clip}%
\pgfsetbuttcap%
\pgfsetroundjoin%
\pgfsetlinewidth{1.003750pt}%
\definecolor{currentstroke}{rgb}{1.000000,0.000000,0.000000}%
\pgfsetstrokecolor{currentstroke}%
\pgfsetdash{}{0pt}%
\pgfpathmoveto{\pgfqpoint{1.423210in}{2.488819in}}%
\pgfpathcurveto{\pgfqpoint{1.434260in}{2.488819in}}{\pgfqpoint{1.444860in}{2.493209in}}{\pgfqpoint{1.452673in}{2.501023in}}%
\pgfpathcurveto{\pgfqpoint{1.460487in}{2.508836in}}{\pgfqpoint{1.464877in}{2.519435in}}{\pgfqpoint{1.464877in}{2.530485in}}%
\pgfpathcurveto{\pgfqpoint{1.464877in}{2.541536in}}{\pgfqpoint{1.460487in}{2.552135in}}{\pgfqpoint{1.452673in}{2.559948in}}%
\pgfpathcurveto{\pgfqpoint{1.444860in}{2.567762in}}{\pgfqpoint{1.434260in}{2.572152in}}{\pgfqpoint{1.423210in}{2.572152in}}%
\pgfpathcurveto{\pgfqpoint{1.412160in}{2.572152in}}{\pgfqpoint{1.401561in}{2.567762in}}{\pgfqpoint{1.393748in}{2.559948in}}%
\pgfpathcurveto{\pgfqpoint{1.385934in}{2.552135in}}{\pgfqpoint{1.381544in}{2.541536in}}{\pgfqpoint{1.381544in}{2.530485in}}%
\pgfpathcurveto{\pgfqpoint{1.381544in}{2.519435in}}{\pgfqpoint{1.385934in}{2.508836in}}{\pgfqpoint{1.393748in}{2.501023in}}%
\pgfpathcurveto{\pgfqpoint{1.401561in}{2.493209in}}{\pgfqpoint{1.412160in}{2.488819in}}{\pgfqpoint{1.423210in}{2.488819in}}%
\pgfpathlineto{\pgfqpoint{1.423210in}{2.488819in}}%
\pgfpathclose%
\pgfusepath{stroke}%
\end{pgfscope}%
\begin{pgfscope}%
\pgfpathrectangle{\pgfqpoint{0.847223in}{0.554012in}}{\pgfqpoint{6.200000in}{4.620000in}}%
\pgfusepath{clip}%
\pgfsetbuttcap%
\pgfsetroundjoin%
\pgfsetlinewidth{1.003750pt}%
\definecolor{currentstroke}{rgb}{1.000000,0.000000,0.000000}%
\pgfsetstrokecolor{currentstroke}%
\pgfsetdash{}{0pt}%
\pgfpathmoveto{\pgfqpoint{1.428544in}{2.477223in}}%
\pgfpathcurveto{\pgfqpoint{1.439594in}{2.477223in}}{\pgfqpoint{1.450193in}{2.481613in}}{\pgfqpoint{1.458006in}{2.489426in}}%
\pgfpathcurveto{\pgfqpoint{1.465820in}{2.497240in}}{\pgfqpoint{1.470210in}{2.507839in}}{\pgfqpoint{1.470210in}{2.518889in}}%
\pgfpathcurveto{\pgfqpoint{1.470210in}{2.529939in}}{\pgfqpoint{1.465820in}{2.540538in}}{\pgfqpoint{1.458006in}{2.548352in}}%
\pgfpathcurveto{\pgfqpoint{1.450193in}{2.556166in}}{\pgfqpoint{1.439594in}{2.560556in}}{\pgfqpoint{1.428544in}{2.560556in}}%
\pgfpathcurveto{\pgfqpoint{1.417493in}{2.560556in}}{\pgfqpoint{1.406894in}{2.556166in}}{\pgfqpoint{1.399081in}{2.548352in}}%
\pgfpathcurveto{\pgfqpoint{1.391267in}{2.540538in}}{\pgfqpoint{1.386877in}{2.529939in}}{\pgfqpoint{1.386877in}{2.518889in}}%
\pgfpathcurveto{\pgfqpoint{1.386877in}{2.507839in}}{\pgfqpoint{1.391267in}{2.497240in}}{\pgfqpoint{1.399081in}{2.489426in}}%
\pgfpathcurveto{\pgfqpoint{1.406894in}{2.481613in}}{\pgfqpoint{1.417493in}{2.477223in}}{\pgfqpoint{1.428544in}{2.477223in}}%
\pgfpathlineto{\pgfqpoint{1.428544in}{2.477223in}}%
\pgfpathclose%
\pgfusepath{stroke}%
\end{pgfscope}%
\begin{pgfscope}%
\pgfpathrectangle{\pgfqpoint{0.847223in}{0.554012in}}{\pgfqpoint{6.200000in}{4.620000in}}%
\pgfusepath{clip}%
\pgfsetbuttcap%
\pgfsetroundjoin%
\pgfsetlinewidth{1.003750pt}%
\definecolor{currentstroke}{rgb}{1.000000,0.000000,0.000000}%
\pgfsetstrokecolor{currentstroke}%
\pgfsetdash{}{0pt}%
\pgfpathmoveto{\pgfqpoint{1.433877in}{2.465738in}}%
\pgfpathcurveto{\pgfqpoint{1.444927in}{2.465738in}}{\pgfqpoint{1.455526in}{2.470128in}}{\pgfqpoint{1.463340in}{2.477942in}}%
\pgfpathcurveto{\pgfqpoint{1.471153in}{2.485755in}}{\pgfqpoint{1.475543in}{2.496354in}}{\pgfqpoint{1.475543in}{2.507404in}}%
\pgfpathcurveto{\pgfqpoint{1.475543in}{2.518454in}}{\pgfqpoint{1.471153in}{2.529053in}}{\pgfqpoint{1.463340in}{2.536867in}}%
\pgfpathcurveto{\pgfqpoint{1.455526in}{2.544681in}}{\pgfqpoint{1.444927in}{2.549071in}}{\pgfqpoint{1.433877in}{2.549071in}}%
\pgfpathcurveto{\pgfqpoint{1.422827in}{2.549071in}}{\pgfqpoint{1.412228in}{2.544681in}}{\pgfqpoint{1.404414in}{2.536867in}}%
\pgfpathcurveto{\pgfqpoint{1.396600in}{2.529053in}}{\pgfqpoint{1.392210in}{2.518454in}}{\pgfqpoint{1.392210in}{2.507404in}}%
\pgfpathcurveto{\pgfqpoint{1.392210in}{2.496354in}}{\pgfqpoint{1.396600in}{2.485755in}}{\pgfqpoint{1.404414in}{2.477942in}}%
\pgfpathcurveto{\pgfqpoint{1.412228in}{2.470128in}}{\pgfqpoint{1.422827in}{2.465738in}}{\pgfqpoint{1.433877in}{2.465738in}}%
\pgfpathlineto{\pgfqpoint{1.433877in}{2.465738in}}%
\pgfpathclose%
\pgfusepath{stroke}%
\end{pgfscope}%
\begin{pgfscope}%
\pgfpathrectangle{\pgfqpoint{0.847223in}{0.554012in}}{\pgfqpoint{6.200000in}{4.620000in}}%
\pgfusepath{clip}%
\pgfsetbuttcap%
\pgfsetroundjoin%
\pgfsetlinewidth{1.003750pt}%
\definecolor{currentstroke}{rgb}{1.000000,0.000000,0.000000}%
\pgfsetstrokecolor{currentstroke}%
\pgfsetdash{}{0pt}%
\pgfpathmoveto{\pgfqpoint{1.439210in}{2.454363in}}%
\pgfpathcurveto{\pgfqpoint{1.450260in}{2.454363in}}{\pgfqpoint{1.460859in}{2.458753in}}{\pgfqpoint{1.468673in}{2.466567in}}%
\pgfpathcurveto{\pgfqpoint{1.476486in}{2.474380in}}{\pgfqpoint{1.480877in}{2.484979in}}{\pgfqpoint{1.480877in}{2.496029in}}%
\pgfpathcurveto{\pgfqpoint{1.480877in}{2.507079in}}{\pgfqpoint{1.476486in}{2.517678in}}{\pgfqpoint{1.468673in}{2.525492in}}%
\pgfpathcurveto{\pgfqpoint{1.460859in}{2.533306in}}{\pgfqpoint{1.450260in}{2.537696in}}{\pgfqpoint{1.439210in}{2.537696in}}%
\pgfpathcurveto{\pgfqpoint{1.428160in}{2.537696in}}{\pgfqpoint{1.417561in}{2.533306in}}{\pgfqpoint{1.409747in}{2.525492in}}%
\pgfpathcurveto{\pgfqpoint{1.401934in}{2.517678in}}{\pgfqpoint{1.397543in}{2.507079in}}{\pgfqpoint{1.397543in}{2.496029in}}%
\pgfpathcurveto{\pgfqpoint{1.397543in}{2.484979in}}{\pgfqpoint{1.401934in}{2.474380in}}{\pgfqpoint{1.409747in}{2.466567in}}%
\pgfpathcurveto{\pgfqpoint{1.417561in}{2.458753in}}{\pgfqpoint{1.428160in}{2.454363in}}{\pgfqpoint{1.439210in}{2.454363in}}%
\pgfpathlineto{\pgfqpoint{1.439210in}{2.454363in}}%
\pgfpathclose%
\pgfusepath{stroke}%
\end{pgfscope}%
\begin{pgfscope}%
\pgfpathrectangle{\pgfqpoint{0.847223in}{0.554012in}}{\pgfqpoint{6.200000in}{4.620000in}}%
\pgfusepath{clip}%
\pgfsetbuttcap%
\pgfsetroundjoin%
\pgfsetlinewidth{1.003750pt}%
\definecolor{currentstroke}{rgb}{1.000000,0.000000,0.000000}%
\pgfsetstrokecolor{currentstroke}%
\pgfsetdash{}{0pt}%
\pgfpathmoveto{\pgfqpoint{1.444543in}{2.443096in}}%
\pgfpathcurveto{\pgfqpoint{1.455593in}{2.443096in}}{\pgfqpoint{1.466192in}{2.447486in}}{\pgfqpoint{1.474006in}{2.455300in}}%
\pgfpathcurveto{\pgfqpoint{1.481820in}{2.463113in}}{\pgfqpoint{1.486210in}{2.473712in}}{\pgfqpoint{1.486210in}{2.484763in}}%
\pgfpathcurveto{\pgfqpoint{1.486210in}{2.495813in}}{\pgfqpoint{1.481820in}{2.506412in}}{\pgfqpoint{1.474006in}{2.514225in}}%
\pgfpathcurveto{\pgfqpoint{1.466192in}{2.522039in}}{\pgfqpoint{1.455593in}{2.526429in}}{\pgfqpoint{1.444543in}{2.526429in}}%
\pgfpathcurveto{\pgfqpoint{1.433493in}{2.526429in}}{\pgfqpoint{1.422894in}{2.522039in}}{\pgfqpoint{1.415080in}{2.514225in}}%
\pgfpathcurveto{\pgfqpoint{1.407267in}{2.506412in}}{\pgfqpoint{1.402877in}{2.495813in}}{\pgfqpoint{1.402877in}{2.484763in}}%
\pgfpathcurveto{\pgfqpoint{1.402877in}{2.473712in}}{\pgfqpoint{1.407267in}{2.463113in}}{\pgfqpoint{1.415080in}{2.455300in}}%
\pgfpathcurveto{\pgfqpoint{1.422894in}{2.447486in}}{\pgfqpoint{1.433493in}{2.443096in}}{\pgfqpoint{1.444543in}{2.443096in}}%
\pgfpathlineto{\pgfqpoint{1.444543in}{2.443096in}}%
\pgfpathclose%
\pgfusepath{stroke}%
\end{pgfscope}%
\begin{pgfscope}%
\pgfpathrectangle{\pgfqpoint{0.847223in}{0.554012in}}{\pgfqpoint{6.200000in}{4.620000in}}%
\pgfusepath{clip}%
\pgfsetbuttcap%
\pgfsetroundjoin%
\pgfsetlinewidth{1.003750pt}%
\definecolor{currentstroke}{rgb}{1.000000,0.000000,0.000000}%
\pgfsetstrokecolor{currentstroke}%
\pgfsetdash{}{0pt}%
\pgfpathmoveto{\pgfqpoint{1.449876in}{2.431936in}}%
\pgfpathcurveto{\pgfqpoint{1.460927in}{2.431936in}}{\pgfqpoint{1.471526in}{2.436326in}}{\pgfqpoint{1.479339in}{2.444140in}}%
\pgfpathcurveto{\pgfqpoint{1.487153in}{2.451953in}}{\pgfqpoint{1.491543in}{2.462552in}}{\pgfqpoint{1.491543in}{2.473602in}}%
\pgfpathcurveto{\pgfqpoint{1.491543in}{2.484653in}}{\pgfqpoint{1.487153in}{2.495252in}}{\pgfqpoint{1.479339in}{2.503065in}}%
\pgfpathcurveto{\pgfqpoint{1.471526in}{2.510879in}}{\pgfqpoint{1.460927in}{2.515269in}}{\pgfqpoint{1.449876in}{2.515269in}}%
\pgfpathcurveto{\pgfqpoint{1.438826in}{2.515269in}}{\pgfqpoint{1.428227in}{2.510879in}}{\pgfqpoint{1.420414in}{2.503065in}}%
\pgfpathcurveto{\pgfqpoint{1.412600in}{2.495252in}}{\pgfqpoint{1.408210in}{2.484653in}}{\pgfqpoint{1.408210in}{2.473602in}}%
\pgfpathcurveto{\pgfqpoint{1.408210in}{2.462552in}}{\pgfqpoint{1.412600in}{2.451953in}}{\pgfqpoint{1.420414in}{2.444140in}}%
\pgfpathcurveto{\pgfqpoint{1.428227in}{2.436326in}}{\pgfqpoint{1.438826in}{2.431936in}}{\pgfqpoint{1.449876in}{2.431936in}}%
\pgfpathlineto{\pgfqpoint{1.449876in}{2.431936in}}%
\pgfpathclose%
\pgfusepath{stroke}%
\end{pgfscope}%
\begin{pgfscope}%
\pgfpathrectangle{\pgfqpoint{0.847223in}{0.554012in}}{\pgfqpoint{6.200000in}{4.620000in}}%
\pgfusepath{clip}%
\pgfsetbuttcap%
\pgfsetroundjoin%
\pgfsetlinewidth{1.003750pt}%
\definecolor{currentstroke}{rgb}{1.000000,0.000000,0.000000}%
\pgfsetstrokecolor{currentstroke}%
\pgfsetdash{}{0pt}%
\pgfpathmoveto{\pgfqpoint{1.455210in}{2.420881in}}%
\pgfpathcurveto{\pgfqpoint{1.466260in}{2.420881in}}{\pgfqpoint{1.476859in}{2.425271in}}{\pgfqpoint{1.484672in}{2.433085in}}%
\pgfpathcurveto{\pgfqpoint{1.492486in}{2.440899in}}{\pgfqpoint{1.496876in}{2.451498in}}{\pgfqpoint{1.496876in}{2.462548in}}%
\pgfpathcurveto{\pgfqpoint{1.496876in}{2.473598in}}{\pgfqpoint{1.492486in}{2.484197in}}{\pgfqpoint{1.484672in}{2.492010in}}%
\pgfpathcurveto{\pgfqpoint{1.476859in}{2.499824in}}{\pgfqpoint{1.466260in}{2.504214in}}{\pgfqpoint{1.455210in}{2.504214in}}%
\pgfpathcurveto{\pgfqpoint{1.444160in}{2.504214in}}{\pgfqpoint{1.433560in}{2.499824in}}{\pgfqpoint{1.425747in}{2.492010in}}%
\pgfpathcurveto{\pgfqpoint{1.417933in}{2.484197in}}{\pgfqpoint{1.413543in}{2.473598in}}{\pgfqpoint{1.413543in}{2.462548in}}%
\pgfpathcurveto{\pgfqpoint{1.413543in}{2.451498in}}{\pgfqpoint{1.417933in}{2.440899in}}{\pgfqpoint{1.425747in}{2.433085in}}%
\pgfpathcurveto{\pgfqpoint{1.433560in}{2.425271in}}{\pgfqpoint{1.444160in}{2.420881in}}{\pgfqpoint{1.455210in}{2.420881in}}%
\pgfpathlineto{\pgfqpoint{1.455210in}{2.420881in}}%
\pgfpathclose%
\pgfusepath{stroke}%
\end{pgfscope}%
\begin{pgfscope}%
\pgfpathrectangle{\pgfqpoint{0.847223in}{0.554012in}}{\pgfqpoint{6.200000in}{4.620000in}}%
\pgfusepath{clip}%
\pgfsetbuttcap%
\pgfsetroundjoin%
\pgfsetlinewidth{1.003750pt}%
\definecolor{currentstroke}{rgb}{1.000000,0.000000,0.000000}%
\pgfsetstrokecolor{currentstroke}%
\pgfsetdash{}{0pt}%
\pgfpathmoveto{\pgfqpoint{1.460543in}{2.409930in}}%
\pgfpathcurveto{\pgfqpoint{1.471593in}{2.409930in}}{\pgfqpoint{1.482192in}{2.414320in}}{\pgfqpoint{1.490006in}{2.422134in}}%
\pgfpathcurveto{\pgfqpoint{1.497819in}{2.429947in}}{\pgfqpoint{1.502210in}{2.440546in}}{\pgfqpoint{1.502210in}{2.451597in}}%
\pgfpathcurveto{\pgfqpoint{1.502210in}{2.462647in}}{\pgfqpoint{1.497819in}{2.473246in}}{\pgfqpoint{1.490006in}{2.481059in}}%
\pgfpathcurveto{\pgfqpoint{1.482192in}{2.488873in}}{\pgfqpoint{1.471593in}{2.493263in}}{\pgfqpoint{1.460543in}{2.493263in}}%
\pgfpathcurveto{\pgfqpoint{1.449493in}{2.493263in}}{\pgfqpoint{1.438894in}{2.488873in}}{\pgfqpoint{1.431080in}{2.481059in}}%
\pgfpathcurveto{\pgfqpoint{1.423266in}{2.473246in}}{\pgfqpoint{1.418876in}{2.462647in}}{\pgfqpoint{1.418876in}{2.451597in}}%
\pgfpathcurveto{\pgfqpoint{1.418876in}{2.440546in}}{\pgfqpoint{1.423266in}{2.429947in}}{\pgfqpoint{1.431080in}{2.422134in}}%
\pgfpathcurveto{\pgfqpoint{1.438894in}{2.414320in}}{\pgfqpoint{1.449493in}{2.409930in}}{\pgfqpoint{1.460543in}{2.409930in}}%
\pgfpathlineto{\pgfqpoint{1.460543in}{2.409930in}}%
\pgfpathclose%
\pgfusepath{stroke}%
\end{pgfscope}%
\begin{pgfscope}%
\pgfpathrectangle{\pgfqpoint{0.847223in}{0.554012in}}{\pgfqpoint{6.200000in}{4.620000in}}%
\pgfusepath{clip}%
\pgfsetbuttcap%
\pgfsetroundjoin%
\pgfsetlinewidth{1.003750pt}%
\definecolor{currentstroke}{rgb}{1.000000,0.000000,0.000000}%
\pgfsetstrokecolor{currentstroke}%
\pgfsetdash{}{0pt}%
\pgfpathmoveto{\pgfqpoint{1.465876in}{2.399081in}}%
\pgfpathcurveto{\pgfqpoint{1.476926in}{2.399081in}}{\pgfqpoint{1.487525in}{2.403471in}}{\pgfqpoint{1.495339in}{2.411285in}}%
\pgfpathcurveto{\pgfqpoint{1.503152in}{2.419099in}}{\pgfqpoint{1.507543in}{2.429698in}}{\pgfqpoint{1.507543in}{2.440748in}}%
\pgfpathcurveto{\pgfqpoint{1.507543in}{2.451798in}}{\pgfqpoint{1.503152in}{2.462397in}}{\pgfqpoint{1.495339in}{2.470211in}}%
\pgfpathcurveto{\pgfqpoint{1.487525in}{2.478024in}}{\pgfqpoint{1.476926in}{2.482414in}}{\pgfqpoint{1.465876in}{2.482414in}}%
\pgfpathcurveto{\pgfqpoint{1.454826in}{2.482414in}}{\pgfqpoint{1.444227in}{2.478024in}}{\pgfqpoint{1.436413in}{2.470211in}}%
\pgfpathcurveto{\pgfqpoint{1.428600in}{2.462397in}}{\pgfqpoint{1.424209in}{2.451798in}}{\pgfqpoint{1.424209in}{2.440748in}}%
\pgfpathcurveto{\pgfqpoint{1.424209in}{2.429698in}}{\pgfqpoint{1.428600in}{2.419099in}}{\pgfqpoint{1.436413in}{2.411285in}}%
\pgfpathcurveto{\pgfqpoint{1.444227in}{2.403471in}}{\pgfqpoint{1.454826in}{2.399081in}}{\pgfqpoint{1.465876in}{2.399081in}}%
\pgfpathlineto{\pgfqpoint{1.465876in}{2.399081in}}%
\pgfpathclose%
\pgfusepath{stroke}%
\end{pgfscope}%
\begin{pgfscope}%
\pgfpathrectangle{\pgfqpoint{0.847223in}{0.554012in}}{\pgfqpoint{6.200000in}{4.620000in}}%
\pgfusepath{clip}%
\pgfsetbuttcap%
\pgfsetroundjoin%
\pgfsetlinewidth{1.003750pt}%
\definecolor{currentstroke}{rgb}{1.000000,0.000000,0.000000}%
\pgfsetstrokecolor{currentstroke}%
\pgfsetdash{}{0pt}%
\pgfpathmoveto{\pgfqpoint{1.471209in}{2.388333in}}%
\pgfpathcurveto{\pgfqpoint{1.482259in}{2.388333in}}{\pgfqpoint{1.492858in}{2.392723in}}{\pgfqpoint{1.500672in}{2.400537in}}%
\pgfpathcurveto{\pgfqpoint{1.508486in}{2.408351in}}{\pgfqpoint{1.512876in}{2.418950in}}{\pgfqpoint{1.512876in}{2.430000in}}%
\pgfpathcurveto{\pgfqpoint{1.512876in}{2.441050in}}{\pgfqpoint{1.508486in}{2.451649in}}{\pgfqpoint{1.500672in}{2.459463in}}%
\pgfpathcurveto{\pgfqpoint{1.492858in}{2.467276in}}{\pgfqpoint{1.482259in}{2.471667in}}{\pgfqpoint{1.471209in}{2.471667in}}%
\pgfpathcurveto{\pgfqpoint{1.460159in}{2.471667in}}{\pgfqpoint{1.449560in}{2.467276in}}{\pgfqpoint{1.441747in}{2.459463in}}%
\pgfpathcurveto{\pgfqpoint{1.433933in}{2.451649in}}{\pgfqpoint{1.429543in}{2.441050in}}{\pgfqpoint{1.429543in}{2.430000in}}%
\pgfpathcurveto{\pgfqpoint{1.429543in}{2.418950in}}{\pgfqpoint{1.433933in}{2.408351in}}{\pgfqpoint{1.441747in}{2.400537in}}%
\pgfpathcurveto{\pgfqpoint{1.449560in}{2.392723in}}{\pgfqpoint{1.460159in}{2.388333in}}{\pgfqpoint{1.471209in}{2.388333in}}%
\pgfpathlineto{\pgfqpoint{1.471209in}{2.388333in}}%
\pgfpathclose%
\pgfusepath{stroke}%
\end{pgfscope}%
\begin{pgfscope}%
\pgfpathrectangle{\pgfqpoint{0.847223in}{0.554012in}}{\pgfqpoint{6.200000in}{4.620000in}}%
\pgfusepath{clip}%
\pgfsetbuttcap%
\pgfsetroundjoin%
\pgfsetlinewidth{1.003750pt}%
\definecolor{currentstroke}{rgb}{1.000000,0.000000,0.000000}%
\pgfsetstrokecolor{currentstroke}%
\pgfsetdash{}{0pt}%
\pgfpathmoveto{\pgfqpoint{1.476543in}{2.377685in}}%
\pgfpathcurveto{\pgfqpoint{1.487593in}{2.377685in}}{\pgfqpoint{1.498192in}{2.382075in}}{\pgfqpoint{1.506005in}{2.389889in}}%
\pgfpathcurveto{\pgfqpoint{1.513819in}{2.397702in}}{\pgfqpoint{1.518209in}{2.408301in}}{\pgfqpoint{1.518209in}{2.419351in}}%
\pgfpathcurveto{\pgfqpoint{1.518209in}{2.430401in}}{\pgfqpoint{1.513819in}{2.441001in}}{\pgfqpoint{1.506005in}{2.448814in}}%
\pgfpathcurveto{\pgfqpoint{1.498192in}{2.456628in}}{\pgfqpoint{1.487593in}{2.461018in}}{\pgfqpoint{1.476543in}{2.461018in}}%
\pgfpathcurveto{\pgfqpoint{1.465492in}{2.461018in}}{\pgfqpoint{1.454893in}{2.456628in}}{\pgfqpoint{1.447080in}{2.448814in}}%
\pgfpathcurveto{\pgfqpoint{1.439266in}{2.441001in}}{\pgfqpoint{1.434876in}{2.430401in}}{\pgfqpoint{1.434876in}{2.419351in}}%
\pgfpathcurveto{\pgfqpoint{1.434876in}{2.408301in}}{\pgfqpoint{1.439266in}{2.397702in}}{\pgfqpoint{1.447080in}{2.389889in}}%
\pgfpathcurveto{\pgfqpoint{1.454893in}{2.382075in}}{\pgfqpoint{1.465492in}{2.377685in}}{\pgfqpoint{1.476543in}{2.377685in}}%
\pgfpathlineto{\pgfqpoint{1.476543in}{2.377685in}}%
\pgfpathclose%
\pgfusepath{stroke}%
\end{pgfscope}%
\begin{pgfscope}%
\pgfpathrectangle{\pgfqpoint{0.847223in}{0.554012in}}{\pgfqpoint{6.200000in}{4.620000in}}%
\pgfusepath{clip}%
\pgfsetbuttcap%
\pgfsetroundjoin%
\pgfsetlinewidth{1.003750pt}%
\definecolor{currentstroke}{rgb}{1.000000,0.000000,0.000000}%
\pgfsetstrokecolor{currentstroke}%
\pgfsetdash{}{0pt}%
\pgfpathmoveto{\pgfqpoint{1.481876in}{2.367134in}}%
\pgfpathcurveto{\pgfqpoint{1.492926in}{2.367134in}}{\pgfqpoint{1.503525in}{2.371525in}}{\pgfqpoint{1.511339in}{2.379338in}}%
\pgfpathcurveto{\pgfqpoint{1.519152in}{2.387152in}}{\pgfqpoint{1.523542in}{2.397751in}}{\pgfqpoint{1.523542in}{2.408801in}}%
\pgfpathcurveto{\pgfqpoint{1.523542in}{2.419851in}}{\pgfqpoint{1.519152in}{2.430450in}}{\pgfqpoint{1.511339in}{2.438264in}}%
\pgfpathcurveto{\pgfqpoint{1.503525in}{2.446077in}}{\pgfqpoint{1.492926in}{2.450468in}}{\pgfqpoint{1.481876in}{2.450468in}}%
\pgfpathcurveto{\pgfqpoint{1.470826in}{2.450468in}}{\pgfqpoint{1.460227in}{2.446077in}}{\pgfqpoint{1.452413in}{2.438264in}}%
\pgfpathcurveto{\pgfqpoint{1.444599in}{2.430450in}}{\pgfqpoint{1.440209in}{2.419851in}}{\pgfqpoint{1.440209in}{2.408801in}}%
\pgfpathcurveto{\pgfqpoint{1.440209in}{2.397751in}}{\pgfqpoint{1.444599in}{2.387152in}}{\pgfqpoint{1.452413in}{2.379338in}}%
\pgfpathcurveto{\pgfqpoint{1.460227in}{2.371525in}}{\pgfqpoint{1.470826in}{2.367134in}}{\pgfqpoint{1.481876in}{2.367134in}}%
\pgfpathlineto{\pgfqpoint{1.481876in}{2.367134in}}%
\pgfpathclose%
\pgfusepath{stroke}%
\end{pgfscope}%
\begin{pgfscope}%
\pgfpathrectangle{\pgfqpoint{0.847223in}{0.554012in}}{\pgfqpoint{6.200000in}{4.620000in}}%
\pgfusepath{clip}%
\pgfsetbuttcap%
\pgfsetroundjoin%
\pgfsetlinewidth{1.003750pt}%
\definecolor{currentstroke}{rgb}{1.000000,0.000000,0.000000}%
\pgfsetstrokecolor{currentstroke}%
\pgfsetdash{}{0pt}%
\pgfpathmoveto{\pgfqpoint{1.487209in}{2.356681in}}%
\pgfpathcurveto{\pgfqpoint{1.498259in}{2.356681in}}{\pgfqpoint{1.508858in}{2.361071in}}{\pgfqpoint{1.516672in}{2.368885in}}%
\pgfpathcurveto{\pgfqpoint{1.524485in}{2.376698in}}{\pgfqpoint{1.528876in}{2.387297in}}{\pgfqpoint{1.528876in}{2.398347in}}%
\pgfpathcurveto{\pgfqpoint{1.528876in}{2.409397in}}{\pgfqpoint{1.524485in}{2.419996in}}{\pgfqpoint{1.516672in}{2.427810in}}%
\pgfpathcurveto{\pgfqpoint{1.508858in}{2.435624in}}{\pgfqpoint{1.498259in}{2.440014in}}{\pgfqpoint{1.487209in}{2.440014in}}%
\pgfpathcurveto{\pgfqpoint{1.476159in}{2.440014in}}{\pgfqpoint{1.465560in}{2.435624in}}{\pgfqpoint{1.457746in}{2.427810in}}%
\pgfpathcurveto{\pgfqpoint{1.449933in}{2.419996in}}{\pgfqpoint{1.445542in}{2.409397in}}{\pgfqpoint{1.445542in}{2.398347in}}%
\pgfpathcurveto{\pgfqpoint{1.445542in}{2.387297in}}{\pgfqpoint{1.449933in}{2.376698in}}{\pgfqpoint{1.457746in}{2.368885in}}%
\pgfpathcurveto{\pgfqpoint{1.465560in}{2.361071in}}{\pgfqpoint{1.476159in}{2.356681in}}{\pgfqpoint{1.487209in}{2.356681in}}%
\pgfpathlineto{\pgfqpoint{1.487209in}{2.356681in}}%
\pgfpathclose%
\pgfusepath{stroke}%
\end{pgfscope}%
\begin{pgfscope}%
\pgfpathrectangle{\pgfqpoint{0.847223in}{0.554012in}}{\pgfqpoint{6.200000in}{4.620000in}}%
\pgfusepath{clip}%
\pgfsetbuttcap%
\pgfsetroundjoin%
\pgfsetlinewidth{1.003750pt}%
\definecolor{currentstroke}{rgb}{1.000000,0.000000,0.000000}%
\pgfsetstrokecolor{currentstroke}%
\pgfsetdash{}{0pt}%
\pgfpathmoveto{\pgfqpoint{1.492542in}{2.346322in}}%
\pgfpathcurveto{\pgfqpoint{1.503592in}{2.346322in}}{\pgfqpoint{1.514191in}{2.350713in}}{\pgfqpoint{1.522005in}{2.358526in}}%
\pgfpathcurveto{\pgfqpoint{1.529819in}{2.366340in}}{\pgfqpoint{1.534209in}{2.376939in}}{\pgfqpoint{1.534209in}{2.387989in}}%
\pgfpathcurveto{\pgfqpoint{1.534209in}{2.399039in}}{\pgfqpoint{1.529819in}{2.409638in}}{\pgfqpoint{1.522005in}{2.417452in}}%
\pgfpathcurveto{\pgfqpoint{1.514191in}{2.425265in}}{\pgfqpoint{1.503592in}{2.429656in}}{\pgfqpoint{1.492542in}{2.429656in}}%
\pgfpathcurveto{\pgfqpoint{1.481492in}{2.429656in}}{\pgfqpoint{1.470893in}{2.425265in}}{\pgfqpoint{1.463079in}{2.417452in}}%
\pgfpathcurveto{\pgfqpoint{1.455266in}{2.409638in}}{\pgfqpoint{1.450875in}{2.399039in}}{\pgfqpoint{1.450875in}{2.387989in}}%
\pgfpathcurveto{\pgfqpoint{1.450875in}{2.376939in}}{\pgfqpoint{1.455266in}{2.366340in}}{\pgfqpoint{1.463079in}{2.358526in}}%
\pgfpathcurveto{\pgfqpoint{1.470893in}{2.350713in}}{\pgfqpoint{1.481492in}{2.346322in}}{\pgfqpoint{1.492542in}{2.346322in}}%
\pgfpathlineto{\pgfqpoint{1.492542in}{2.346322in}}%
\pgfpathclose%
\pgfusepath{stroke}%
\end{pgfscope}%
\begin{pgfscope}%
\pgfpathrectangle{\pgfqpoint{0.847223in}{0.554012in}}{\pgfqpoint{6.200000in}{4.620000in}}%
\pgfusepath{clip}%
\pgfsetbuttcap%
\pgfsetroundjoin%
\pgfsetlinewidth{1.003750pt}%
\definecolor{currentstroke}{rgb}{1.000000,0.000000,0.000000}%
\pgfsetstrokecolor{currentstroke}%
\pgfsetdash{}{0pt}%
\pgfpathmoveto{\pgfqpoint{1.497875in}{2.336058in}}%
\pgfpathcurveto{\pgfqpoint{1.508925in}{2.336058in}}{\pgfqpoint{1.519525in}{2.340449in}}{\pgfqpoint{1.527338in}{2.348262in}}%
\pgfpathcurveto{\pgfqpoint{1.535152in}{2.356076in}}{\pgfqpoint{1.539542in}{2.366675in}}{\pgfqpoint{1.539542in}{2.377725in}}%
\pgfpathcurveto{\pgfqpoint{1.539542in}{2.388775in}}{\pgfqpoint{1.535152in}{2.399374in}}{\pgfqpoint{1.527338in}{2.407188in}}%
\pgfpathcurveto{\pgfqpoint{1.519525in}{2.415001in}}{\pgfqpoint{1.508925in}{2.419392in}}{\pgfqpoint{1.497875in}{2.419392in}}%
\pgfpathcurveto{\pgfqpoint{1.486825in}{2.419392in}}{\pgfqpoint{1.476226in}{2.415001in}}{\pgfqpoint{1.468413in}{2.407188in}}%
\pgfpathcurveto{\pgfqpoint{1.460599in}{2.399374in}}{\pgfqpoint{1.456209in}{2.388775in}}{\pgfqpoint{1.456209in}{2.377725in}}%
\pgfpathcurveto{\pgfqpoint{1.456209in}{2.366675in}}{\pgfqpoint{1.460599in}{2.356076in}}{\pgfqpoint{1.468413in}{2.348262in}}%
\pgfpathcurveto{\pgfqpoint{1.476226in}{2.340449in}}{\pgfqpoint{1.486825in}{2.336058in}}{\pgfqpoint{1.497875in}{2.336058in}}%
\pgfpathlineto{\pgfqpoint{1.497875in}{2.336058in}}%
\pgfpathclose%
\pgfusepath{stroke}%
\end{pgfscope}%
\begin{pgfscope}%
\pgfpathrectangle{\pgfqpoint{0.847223in}{0.554012in}}{\pgfqpoint{6.200000in}{4.620000in}}%
\pgfusepath{clip}%
\pgfsetbuttcap%
\pgfsetroundjoin%
\pgfsetlinewidth{1.003750pt}%
\definecolor{currentstroke}{rgb}{1.000000,0.000000,0.000000}%
\pgfsetstrokecolor{currentstroke}%
\pgfsetdash{}{0pt}%
\pgfpathmoveto{\pgfqpoint{1.503209in}{2.325887in}}%
\pgfpathcurveto{\pgfqpoint{1.514259in}{2.325887in}}{\pgfqpoint{1.524858in}{2.330277in}}{\pgfqpoint{1.532671in}{2.338091in}}%
\pgfpathcurveto{\pgfqpoint{1.540485in}{2.345904in}}{\pgfqpoint{1.544875in}{2.356503in}}{\pgfqpoint{1.544875in}{2.367554in}}%
\pgfpathcurveto{\pgfqpoint{1.544875in}{2.378604in}}{\pgfqpoint{1.540485in}{2.389203in}}{\pgfqpoint{1.532671in}{2.397016in}}%
\pgfpathcurveto{\pgfqpoint{1.524858in}{2.404830in}}{\pgfqpoint{1.514259in}{2.409220in}}{\pgfqpoint{1.503209in}{2.409220in}}%
\pgfpathcurveto{\pgfqpoint{1.492158in}{2.409220in}}{\pgfqpoint{1.481559in}{2.404830in}}{\pgfqpoint{1.473746in}{2.397016in}}%
\pgfpathcurveto{\pgfqpoint{1.465932in}{2.389203in}}{\pgfqpoint{1.461542in}{2.378604in}}{\pgfqpoint{1.461542in}{2.367554in}}%
\pgfpathcurveto{\pgfqpoint{1.461542in}{2.356503in}}{\pgfqpoint{1.465932in}{2.345904in}}{\pgfqpoint{1.473746in}{2.338091in}}%
\pgfpathcurveto{\pgfqpoint{1.481559in}{2.330277in}}{\pgfqpoint{1.492158in}{2.325887in}}{\pgfqpoint{1.503209in}{2.325887in}}%
\pgfpathlineto{\pgfqpoint{1.503209in}{2.325887in}}%
\pgfpathclose%
\pgfusepath{stroke}%
\end{pgfscope}%
\begin{pgfscope}%
\pgfpathrectangle{\pgfqpoint{0.847223in}{0.554012in}}{\pgfqpoint{6.200000in}{4.620000in}}%
\pgfusepath{clip}%
\pgfsetbuttcap%
\pgfsetroundjoin%
\pgfsetlinewidth{1.003750pt}%
\definecolor{currentstroke}{rgb}{1.000000,0.000000,0.000000}%
\pgfsetstrokecolor{currentstroke}%
\pgfsetdash{}{0pt}%
\pgfpathmoveto{\pgfqpoint{1.508542in}{2.315807in}}%
\pgfpathcurveto{\pgfqpoint{1.519592in}{2.315807in}}{\pgfqpoint{1.530191in}{2.320197in}}{\pgfqpoint{1.538005in}{2.328011in}}%
\pgfpathcurveto{\pgfqpoint{1.545818in}{2.335825in}}{\pgfqpoint{1.550208in}{2.346424in}}{\pgfqpoint{1.550208in}{2.357474in}}%
\pgfpathcurveto{\pgfqpoint{1.550208in}{2.368524in}}{\pgfqpoint{1.545818in}{2.379123in}}{\pgfqpoint{1.538005in}{2.386937in}}%
\pgfpathcurveto{\pgfqpoint{1.530191in}{2.394750in}}{\pgfqpoint{1.519592in}{2.399141in}}{\pgfqpoint{1.508542in}{2.399141in}}%
\pgfpathcurveto{\pgfqpoint{1.497492in}{2.399141in}}{\pgfqpoint{1.486893in}{2.394750in}}{\pgfqpoint{1.479079in}{2.386937in}}%
\pgfpathcurveto{\pgfqpoint{1.471265in}{2.379123in}}{\pgfqpoint{1.466875in}{2.368524in}}{\pgfqpoint{1.466875in}{2.357474in}}%
\pgfpathcurveto{\pgfqpoint{1.466875in}{2.346424in}}{\pgfqpoint{1.471265in}{2.335825in}}{\pgfqpoint{1.479079in}{2.328011in}}%
\pgfpathcurveto{\pgfqpoint{1.486893in}{2.320197in}}{\pgfqpoint{1.497492in}{2.315807in}}{\pgfqpoint{1.508542in}{2.315807in}}%
\pgfpathlineto{\pgfqpoint{1.508542in}{2.315807in}}%
\pgfpathclose%
\pgfusepath{stroke}%
\end{pgfscope}%
\begin{pgfscope}%
\pgfpathrectangle{\pgfqpoint{0.847223in}{0.554012in}}{\pgfqpoint{6.200000in}{4.620000in}}%
\pgfusepath{clip}%
\pgfsetbuttcap%
\pgfsetroundjoin%
\pgfsetlinewidth{1.003750pt}%
\definecolor{currentstroke}{rgb}{1.000000,0.000000,0.000000}%
\pgfsetstrokecolor{currentstroke}%
\pgfsetdash{}{0pt}%
\pgfpathmoveto{\pgfqpoint{1.513875in}{2.305818in}}%
\pgfpathcurveto{\pgfqpoint{1.524925in}{2.305818in}}{\pgfqpoint{1.535524in}{2.310208in}}{\pgfqpoint{1.543338in}{2.318022in}}%
\pgfpathcurveto{\pgfqpoint{1.551151in}{2.325835in}}{\pgfqpoint{1.555542in}{2.336434in}}{\pgfqpoint{1.555542in}{2.347484in}}%
\pgfpathcurveto{\pgfqpoint{1.555542in}{2.358535in}}{\pgfqpoint{1.551151in}{2.369134in}}{\pgfqpoint{1.543338in}{2.376947in}}%
\pgfpathcurveto{\pgfqpoint{1.535524in}{2.384761in}}{\pgfqpoint{1.524925in}{2.389151in}}{\pgfqpoint{1.513875in}{2.389151in}}%
\pgfpathcurveto{\pgfqpoint{1.502825in}{2.389151in}}{\pgfqpoint{1.492226in}{2.384761in}}{\pgfqpoint{1.484412in}{2.376947in}}%
\pgfpathcurveto{\pgfqpoint{1.476599in}{2.369134in}}{\pgfqpoint{1.472208in}{2.358535in}}{\pgfqpoint{1.472208in}{2.347484in}}%
\pgfpathcurveto{\pgfqpoint{1.472208in}{2.336434in}}{\pgfqpoint{1.476599in}{2.325835in}}{\pgfqpoint{1.484412in}{2.318022in}}%
\pgfpathcurveto{\pgfqpoint{1.492226in}{2.310208in}}{\pgfqpoint{1.502825in}{2.305818in}}{\pgfqpoint{1.513875in}{2.305818in}}%
\pgfpathlineto{\pgfqpoint{1.513875in}{2.305818in}}%
\pgfpathclose%
\pgfusepath{stroke}%
\end{pgfscope}%
\begin{pgfscope}%
\pgfpathrectangle{\pgfqpoint{0.847223in}{0.554012in}}{\pgfqpoint{6.200000in}{4.620000in}}%
\pgfusepath{clip}%
\pgfsetbuttcap%
\pgfsetroundjoin%
\pgfsetlinewidth{1.003750pt}%
\definecolor{currentstroke}{rgb}{1.000000,0.000000,0.000000}%
\pgfsetstrokecolor{currentstroke}%
\pgfsetdash{}{0pt}%
\pgfpathmoveto{\pgfqpoint{1.519208in}{2.295918in}}%
\pgfpathcurveto{\pgfqpoint{1.530258in}{2.295918in}}{\pgfqpoint{1.540857in}{2.300308in}}{\pgfqpoint{1.548671in}{2.308121in}}%
\pgfpathcurveto{\pgfqpoint{1.556485in}{2.315935in}}{\pgfqpoint{1.560875in}{2.326534in}}{\pgfqpoint{1.560875in}{2.337584in}}%
\pgfpathcurveto{\pgfqpoint{1.560875in}{2.348634in}}{\pgfqpoint{1.556485in}{2.359233in}}{\pgfqpoint{1.548671in}{2.367047in}}%
\pgfpathcurveto{\pgfqpoint{1.540857in}{2.374861in}}{\pgfqpoint{1.530258in}{2.379251in}}{\pgfqpoint{1.519208in}{2.379251in}}%
\pgfpathcurveto{\pgfqpoint{1.508158in}{2.379251in}}{\pgfqpoint{1.497559in}{2.374861in}}{\pgfqpoint{1.489745in}{2.367047in}}%
\pgfpathcurveto{\pgfqpoint{1.481932in}{2.359233in}}{\pgfqpoint{1.477542in}{2.348634in}}{\pgfqpoint{1.477542in}{2.337584in}}%
\pgfpathcurveto{\pgfqpoint{1.477542in}{2.326534in}}{\pgfqpoint{1.481932in}{2.315935in}}{\pgfqpoint{1.489745in}{2.308121in}}%
\pgfpathcurveto{\pgfqpoint{1.497559in}{2.300308in}}{\pgfqpoint{1.508158in}{2.295918in}}{\pgfqpoint{1.519208in}{2.295918in}}%
\pgfpathlineto{\pgfqpoint{1.519208in}{2.295918in}}%
\pgfpathclose%
\pgfusepath{stroke}%
\end{pgfscope}%
\begin{pgfscope}%
\pgfpathrectangle{\pgfqpoint{0.847223in}{0.554012in}}{\pgfqpoint{6.200000in}{4.620000in}}%
\pgfusepath{clip}%
\pgfsetbuttcap%
\pgfsetroundjoin%
\pgfsetlinewidth{1.003750pt}%
\definecolor{currentstroke}{rgb}{1.000000,0.000000,0.000000}%
\pgfsetstrokecolor{currentstroke}%
\pgfsetdash{}{0pt}%
\pgfpathmoveto{\pgfqpoint{1.524541in}{2.286105in}}%
\pgfpathcurveto{\pgfqpoint{1.535592in}{2.286105in}}{\pgfqpoint{1.546191in}{2.290496in}}{\pgfqpoint{1.554004in}{2.298309in}}%
\pgfpathcurveto{\pgfqpoint{1.561818in}{2.306123in}}{\pgfqpoint{1.566208in}{2.316722in}}{\pgfqpoint{1.566208in}{2.327772in}}%
\pgfpathcurveto{\pgfqpoint{1.566208in}{2.338822in}}{\pgfqpoint{1.561818in}{2.349421in}}{\pgfqpoint{1.554004in}{2.357235in}}%
\pgfpathcurveto{\pgfqpoint{1.546191in}{2.365048in}}{\pgfqpoint{1.535592in}{2.369439in}}{\pgfqpoint{1.524541in}{2.369439in}}%
\pgfpathcurveto{\pgfqpoint{1.513491in}{2.369439in}}{\pgfqpoint{1.502892in}{2.365048in}}{\pgfqpoint{1.495079in}{2.357235in}}%
\pgfpathcurveto{\pgfqpoint{1.487265in}{2.349421in}}{\pgfqpoint{1.482875in}{2.338822in}}{\pgfqpoint{1.482875in}{2.327772in}}%
\pgfpathcurveto{\pgfqpoint{1.482875in}{2.316722in}}{\pgfqpoint{1.487265in}{2.306123in}}{\pgfqpoint{1.495079in}{2.298309in}}%
\pgfpathcurveto{\pgfqpoint{1.502892in}{2.290496in}}{\pgfqpoint{1.513491in}{2.286105in}}{\pgfqpoint{1.524541in}{2.286105in}}%
\pgfpathlineto{\pgfqpoint{1.524541in}{2.286105in}}%
\pgfpathclose%
\pgfusepath{stroke}%
\end{pgfscope}%
\begin{pgfscope}%
\pgfpathrectangle{\pgfqpoint{0.847223in}{0.554012in}}{\pgfqpoint{6.200000in}{4.620000in}}%
\pgfusepath{clip}%
\pgfsetbuttcap%
\pgfsetroundjoin%
\pgfsetlinewidth{1.003750pt}%
\definecolor{currentstroke}{rgb}{1.000000,0.000000,0.000000}%
\pgfsetstrokecolor{currentstroke}%
\pgfsetdash{}{0pt}%
\pgfpathmoveto{\pgfqpoint{1.529875in}{2.276380in}}%
\pgfpathcurveto{\pgfqpoint{1.540925in}{2.276380in}}{\pgfqpoint{1.551524in}{2.280770in}}{\pgfqpoint{1.559337in}{2.288584in}}%
\pgfpathcurveto{\pgfqpoint{1.567151in}{2.296397in}}{\pgfqpoint{1.571541in}{2.306996in}}{\pgfqpoint{1.571541in}{2.318046in}}%
\pgfpathcurveto{\pgfqpoint{1.571541in}{2.329097in}}{\pgfqpoint{1.567151in}{2.339696in}}{\pgfqpoint{1.559337in}{2.347509in}}%
\pgfpathcurveto{\pgfqpoint{1.551524in}{2.355323in}}{\pgfqpoint{1.540925in}{2.359713in}}{\pgfqpoint{1.529875in}{2.359713in}}%
\pgfpathcurveto{\pgfqpoint{1.518825in}{2.359713in}}{\pgfqpoint{1.508225in}{2.355323in}}{\pgfqpoint{1.500412in}{2.347509in}}%
\pgfpathcurveto{\pgfqpoint{1.492598in}{2.339696in}}{\pgfqpoint{1.488208in}{2.329097in}}{\pgfqpoint{1.488208in}{2.318046in}}%
\pgfpathcurveto{\pgfqpoint{1.488208in}{2.306996in}}{\pgfqpoint{1.492598in}{2.296397in}}{\pgfqpoint{1.500412in}{2.288584in}}%
\pgfpathcurveto{\pgfqpoint{1.508225in}{2.280770in}}{\pgfqpoint{1.518825in}{2.276380in}}{\pgfqpoint{1.529875in}{2.276380in}}%
\pgfpathlineto{\pgfqpoint{1.529875in}{2.276380in}}%
\pgfpathclose%
\pgfusepath{stroke}%
\end{pgfscope}%
\begin{pgfscope}%
\pgfpathrectangle{\pgfqpoint{0.847223in}{0.554012in}}{\pgfqpoint{6.200000in}{4.620000in}}%
\pgfusepath{clip}%
\pgfsetbuttcap%
\pgfsetroundjoin%
\pgfsetlinewidth{1.003750pt}%
\definecolor{currentstroke}{rgb}{1.000000,0.000000,0.000000}%
\pgfsetstrokecolor{currentstroke}%
\pgfsetdash{}{0pt}%
\pgfpathmoveto{\pgfqpoint{1.535208in}{2.266740in}}%
\pgfpathcurveto{\pgfqpoint{1.546258in}{2.266740in}}{\pgfqpoint{1.556857in}{2.271130in}}{\pgfqpoint{1.564671in}{2.278944in}}%
\pgfpathcurveto{\pgfqpoint{1.572484in}{2.286757in}}{\pgfqpoint{1.576875in}{2.297356in}}{\pgfqpoint{1.576875in}{2.308407in}}%
\pgfpathcurveto{\pgfqpoint{1.576875in}{2.319457in}}{\pgfqpoint{1.572484in}{2.330056in}}{\pgfqpoint{1.564671in}{2.337869in}}%
\pgfpathcurveto{\pgfqpoint{1.556857in}{2.345683in}}{\pgfqpoint{1.546258in}{2.350073in}}{\pgfqpoint{1.535208in}{2.350073in}}%
\pgfpathcurveto{\pgfqpoint{1.524158in}{2.350073in}}{\pgfqpoint{1.513559in}{2.345683in}}{\pgfqpoint{1.505745in}{2.337869in}}%
\pgfpathcurveto{\pgfqpoint{1.497931in}{2.330056in}}{\pgfqpoint{1.493541in}{2.319457in}}{\pgfqpoint{1.493541in}{2.308407in}}%
\pgfpathcurveto{\pgfqpoint{1.493541in}{2.297356in}}{\pgfqpoint{1.497931in}{2.286757in}}{\pgfqpoint{1.505745in}{2.278944in}}%
\pgfpathcurveto{\pgfqpoint{1.513559in}{2.271130in}}{\pgfqpoint{1.524158in}{2.266740in}}{\pgfqpoint{1.535208in}{2.266740in}}%
\pgfpathlineto{\pgfqpoint{1.535208in}{2.266740in}}%
\pgfpathclose%
\pgfusepath{stroke}%
\end{pgfscope}%
\begin{pgfscope}%
\pgfpathrectangle{\pgfqpoint{0.847223in}{0.554012in}}{\pgfqpoint{6.200000in}{4.620000in}}%
\pgfusepath{clip}%
\pgfsetbuttcap%
\pgfsetroundjoin%
\pgfsetlinewidth{1.003750pt}%
\definecolor{currentstroke}{rgb}{1.000000,0.000000,0.000000}%
\pgfsetstrokecolor{currentstroke}%
\pgfsetdash{}{0pt}%
\pgfpathmoveto{\pgfqpoint{1.540541in}{2.257185in}}%
\pgfpathcurveto{\pgfqpoint{1.551591in}{2.257185in}}{\pgfqpoint{1.562190in}{2.261575in}}{\pgfqpoint{1.570004in}{2.269388in}}%
\pgfpathcurveto{\pgfqpoint{1.577817in}{2.277202in}}{\pgfqpoint{1.582208in}{2.287801in}}{\pgfqpoint{1.582208in}{2.298851in}}%
\pgfpathcurveto{\pgfqpoint{1.582208in}{2.309901in}}{\pgfqpoint{1.577817in}{2.320500in}}{\pgfqpoint{1.570004in}{2.328314in}}%
\pgfpathcurveto{\pgfqpoint{1.562190in}{2.336128in}}{\pgfqpoint{1.551591in}{2.340518in}}{\pgfqpoint{1.540541in}{2.340518in}}%
\pgfpathcurveto{\pgfqpoint{1.529491in}{2.340518in}}{\pgfqpoint{1.518892in}{2.336128in}}{\pgfqpoint{1.511078in}{2.328314in}}%
\pgfpathcurveto{\pgfqpoint{1.503265in}{2.320500in}}{\pgfqpoint{1.498874in}{2.309901in}}{\pgfqpoint{1.498874in}{2.298851in}}%
\pgfpathcurveto{\pgfqpoint{1.498874in}{2.287801in}}{\pgfqpoint{1.503265in}{2.277202in}}{\pgfqpoint{1.511078in}{2.269388in}}%
\pgfpathcurveto{\pgfqpoint{1.518892in}{2.261575in}}{\pgfqpoint{1.529491in}{2.257185in}}{\pgfqpoint{1.540541in}{2.257185in}}%
\pgfpathlineto{\pgfqpoint{1.540541in}{2.257185in}}%
\pgfpathclose%
\pgfusepath{stroke}%
\end{pgfscope}%
\begin{pgfscope}%
\pgfpathrectangle{\pgfqpoint{0.847223in}{0.554012in}}{\pgfqpoint{6.200000in}{4.620000in}}%
\pgfusepath{clip}%
\pgfsetbuttcap%
\pgfsetroundjoin%
\pgfsetlinewidth{1.003750pt}%
\definecolor{currentstroke}{rgb}{1.000000,0.000000,0.000000}%
\pgfsetstrokecolor{currentstroke}%
\pgfsetdash{}{0pt}%
\pgfpathmoveto{\pgfqpoint{1.545874in}{2.247713in}}%
\pgfpathcurveto{\pgfqpoint{1.556924in}{2.247713in}}{\pgfqpoint{1.567523in}{2.252103in}}{\pgfqpoint{1.575337in}{2.259917in}}%
\pgfpathcurveto{\pgfqpoint{1.583151in}{2.267730in}}{\pgfqpoint{1.587541in}{2.278329in}}{\pgfqpoint{1.587541in}{2.289379in}}%
\pgfpathcurveto{\pgfqpoint{1.587541in}{2.300429in}}{\pgfqpoint{1.583151in}{2.311028in}}{\pgfqpoint{1.575337in}{2.318842in}}%
\pgfpathcurveto{\pgfqpoint{1.567523in}{2.326656in}}{\pgfqpoint{1.556924in}{2.331046in}}{\pgfqpoint{1.545874in}{2.331046in}}%
\pgfpathcurveto{\pgfqpoint{1.534824in}{2.331046in}}{\pgfqpoint{1.524225in}{2.326656in}}{\pgfqpoint{1.516412in}{2.318842in}}%
\pgfpathcurveto{\pgfqpoint{1.508598in}{2.311028in}}{\pgfqpoint{1.504208in}{2.300429in}}{\pgfqpoint{1.504208in}{2.289379in}}%
\pgfpathcurveto{\pgfqpoint{1.504208in}{2.278329in}}{\pgfqpoint{1.508598in}{2.267730in}}{\pgfqpoint{1.516412in}{2.259917in}}%
\pgfpathcurveto{\pgfqpoint{1.524225in}{2.252103in}}{\pgfqpoint{1.534824in}{2.247713in}}{\pgfqpoint{1.545874in}{2.247713in}}%
\pgfpathlineto{\pgfqpoint{1.545874in}{2.247713in}}%
\pgfpathclose%
\pgfusepath{stroke}%
\end{pgfscope}%
\begin{pgfscope}%
\pgfpathrectangle{\pgfqpoint{0.847223in}{0.554012in}}{\pgfqpoint{6.200000in}{4.620000in}}%
\pgfusepath{clip}%
\pgfsetbuttcap%
\pgfsetroundjoin%
\pgfsetlinewidth{1.003750pt}%
\definecolor{currentstroke}{rgb}{1.000000,0.000000,0.000000}%
\pgfsetstrokecolor{currentstroke}%
\pgfsetdash{}{0pt}%
\pgfpathmoveto{\pgfqpoint{1.551208in}{2.238323in}}%
\pgfpathcurveto{\pgfqpoint{1.562258in}{2.238323in}}{\pgfqpoint{1.572857in}{2.242713in}}{\pgfqpoint{1.580670in}{2.250527in}}%
\pgfpathcurveto{\pgfqpoint{1.588484in}{2.258340in}}{\pgfqpoint{1.592874in}{2.268940in}}{\pgfqpoint{1.592874in}{2.279990in}}%
\pgfpathcurveto{\pgfqpoint{1.592874in}{2.291040in}}{\pgfqpoint{1.588484in}{2.301639in}}{\pgfqpoint{1.580670in}{2.309452in}}%
\pgfpathcurveto{\pgfqpoint{1.572857in}{2.317266in}}{\pgfqpoint{1.562258in}{2.321656in}}{\pgfqpoint{1.551208in}{2.321656in}}%
\pgfpathcurveto{\pgfqpoint{1.540157in}{2.321656in}}{\pgfqpoint{1.529558in}{2.317266in}}{\pgfqpoint{1.521745in}{2.309452in}}%
\pgfpathcurveto{\pgfqpoint{1.513931in}{2.301639in}}{\pgfqpoint{1.509541in}{2.291040in}}{\pgfqpoint{1.509541in}{2.279990in}}%
\pgfpathcurveto{\pgfqpoint{1.509541in}{2.268940in}}{\pgfqpoint{1.513931in}{2.258340in}}{\pgfqpoint{1.521745in}{2.250527in}}%
\pgfpathcurveto{\pgfqpoint{1.529558in}{2.242713in}}{\pgfqpoint{1.540157in}{2.238323in}}{\pgfqpoint{1.551208in}{2.238323in}}%
\pgfpathlineto{\pgfqpoint{1.551208in}{2.238323in}}%
\pgfpathclose%
\pgfusepath{stroke}%
\end{pgfscope}%
\begin{pgfscope}%
\pgfpathrectangle{\pgfqpoint{0.847223in}{0.554012in}}{\pgfqpoint{6.200000in}{4.620000in}}%
\pgfusepath{clip}%
\pgfsetbuttcap%
\pgfsetroundjoin%
\pgfsetlinewidth{1.003750pt}%
\definecolor{currentstroke}{rgb}{1.000000,0.000000,0.000000}%
\pgfsetstrokecolor{currentstroke}%
\pgfsetdash{}{0pt}%
\pgfpathmoveto{\pgfqpoint{1.556541in}{2.229015in}}%
\pgfpathcurveto{\pgfqpoint{1.567591in}{2.229015in}}{\pgfqpoint{1.578190in}{2.233405in}}{\pgfqpoint{1.586004in}{2.241218in}}%
\pgfpathcurveto{\pgfqpoint{1.593817in}{2.249032in}}{\pgfqpoint{1.598207in}{2.259631in}}{\pgfqpoint{1.598207in}{2.270681in}}%
\pgfpathcurveto{\pgfqpoint{1.598207in}{2.281731in}}{\pgfqpoint{1.593817in}{2.292330in}}{\pgfqpoint{1.586004in}{2.300144in}}%
\pgfpathcurveto{\pgfqpoint{1.578190in}{2.307958in}}{\pgfqpoint{1.567591in}{2.312348in}}{\pgfqpoint{1.556541in}{2.312348in}}%
\pgfpathcurveto{\pgfqpoint{1.545491in}{2.312348in}}{\pgfqpoint{1.534892in}{2.307958in}}{\pgfqpoint{1.527078in}{2.300144in}}%
\pgfpathcurveto{\pgfqpoint{1.519264in}{2.292330in}}{\pgfqpoint{1.514874in}{2.281731in}}{\pgfqpoint{1.514874in}{2.270681in}}%
\pgfpathcurveto{\pgfqpoint{1.514874in}{2.259631in}}{\pgfqpoint{1.519264in}{2.249032in}}{\pgfqpoint{1.527078in}{2.241218in}}%
\pgfpathcurveto{\pgfqpoint{1.534892in}{2.233405in}}{\pgfqpoint{1.545491in}{2.229015in}}{\pgfqpoint{1.556541in}{2.229015in}}%
\pgfpathlineto{\pgfqpoint{1.556541in}{2.229015in}}%
\pgfpathclose%
\pgfusepath{stroke}%
\end{pgfscope}%
\begin{pgfscope}%
\pgfpathrectangle{\pgfqpoint{0.847223in}{0.554012in}}{\pgfqpoint{6.200000in}{4.620000in}}%
\pgfusepath{clip}%
\pgfsetbuttcap%
\pgfsetroundjoin%
\pgfsetlinewidth{1.003750pt}%
\definecolor{currentstroke}{rgb}{1.000000,0.000000,0.000000}%
\pgfsetstrokecolor{currentstroke}%
\pgfsetdash{}{0pt}%
\pgfpathmoveto{\pgfqpoint{1.561874in}{2.219786in}}%
\pgfpathcurveto{\pgfqpoint{1.572924in}{2.219786in}}{\pgfqpoint{1.583523in}{2.224177in}}{\pgfqpoint{1.591337in}{2.231990in}}%
\pgfpathcurveto{\pgfqpoint{1.599150in}{2.239804in}}{\pgfqpoint{1.603541in}{2.250403in}}{\pgfqpoint{1.603541in}{2.261453in}}%
\pgfpathcurveto{\pgfqpoint{1.603541in}{2.272503in}}{\pgfqpoint{1.599150in}{2.283102in}}{\pgfqpoint{1.591337in}{2.290916in}}%
\pgfpathcurveto{\pgfqpoint{1.583523in}{2.298729in}}{\pgfqpoint{1.572924in}{2.303120in}}{\pgfqpoint{1.561874in}{2.303120in}}%
\pgfpathcurveto{\pgfqpoint{1.550824in}{2.303120in}}{\pgfqpoint{1.540225in}{2.298729in}}{\pgfqpoint{1.532411in}{2.290916in}}%
\pgfpathcurveto{\pgfqpoint{1.524598in}{2.283102in}}{\pgfqpoint{1.520207in}{2.272503in}}{\pgfqpoint{1.520207in}{2.261453in}}%
\pgfpathcurveto{\pgfqpoint{1.520207in}{2.250403in}}{\pgfqpoint{1.524598in}{2.239804in}}{\pgfqpoint{1.532411in}{2.231990in}}%
\pgfpathcurveto{\pgfqpoint{1.540225in}{2.224177in}}{\pgfqpoint{1.550824in}{2.219786in}}{\pgfqpoint{1.561874in}{2.219786in}}%
\pgfpathlineto{\pgfqpoint{1.561874in}{2.219786in}}%
\pgfpathclose%
\pgfusepath{stroke}%
\end{pgfscope}%
\begin{pgfscope}%
\pgfpathrectangle{\pgfqpoint{0.847223in}{0.554012in}}{\pgfqpoint{6.200000in}{4.620000in}}%
\pgfusepath{clip}%
\pgfsetbuttcap%
\pgfsetroundjoin%
\pgfsetlinewidth{1.003750pt}%
\definecolor{currentstroke}{rgb}{1.000000,0.000000,0.000000}%
\pgfsetstrokecolor{currentstroke}%
\pgfsetdash{}{0pt}%
\pgfpathmoveto{\pgfqpoint{1.567207in}{2.210637in}}%
\pgfpathcurveto{\pgfqpoint{1.578257in}{2.210637in}}{\pgfqpoint{1.588856in}{2.215028in}}{\pgfqpoint{1.596670in}{2.222841in}}%
\pgfpathcurveto{\pgfqpoint{1.604484in}{2.230655in}}{\pgfqpoint{1.608874in}{2.241254in}}{\pgfqpoint{1.608874in}{2.252304in}}%
\pgfpathcurveto{\pgfqpoint{1.608874in}{2.263354in}}{\pgfqpoint{1.604484in}{2.273953in}}{\pgfqpoint{1.596670in}{2.281767in}}%
\pgfpathcurveto{\pgfqpoint{1.588856in}{2.289580in}}{\pgfqpoint{1.578257in}{2.293971in}}{\pgfqpoint{1.567207in}{2.293971in}}%
\pgfpathcurveto{\pgfqpoint{1.556157in}{2.293971in}}{\pgfqpoint{1.545558in}{2.289580in}}{\pgfqpoint{1.537744in}{2.281767in}}%
\pgfpathcurveto{\pgfqpoint{1.529931in}{2.273953in}}{\pgfqpoint{1.525541in}{2.263354in}}{\pgfqpoint{1.525541in}{2.252304in}}%
\pgfpathcurveto{\pgfqpoint{1.525541in}{2.241254in}}{\pgfqpoint{1.529931in}{2.230655in}}{\pgfqpoint{1.537744in}{2.222841in}}%
\pgfpathcurveto{\pgfqpoint{1.545558in}{2.215028in}}{\pgfqpoint{1.556157in}{2.210637in}}{\pgfqpoint{1.567207in}{2.210637in}}%
\pgfpathlineto{\pgfqpoint{1.567207in}{2.210637in}}%
\pgfpathclose%
\pgfusepath{stroke}%
\end{pgfscope}%
\begin{pgfscope}%
\pgfpathrectangle{\pgfqpoint{0.847223in}{0.554012in}}{\pgfqpoint{6.200000in}{4.620000in}}%
\pgfusepath{clip}%
\pgfsetbuttcap%
\pgfsetroundjoin%
\pgfsetlinewidth{1.003750pt}%
\definecolor{currentstroke}{rgb}{1.000000,0.000000,0.000000}%
\pgfsetstrokecolor{currentstroke}%
\pgfsetdash{}{0pt}%
\pgfpathmoveto{\pgfqpoint{1.572540in}{2.201567in}}%
\pgfpathcurveto{\pgfqpoint{1.583591in}{2.201567in}}{\pgfqpoint{1.594190in}{2.205957in}}{\pgfqpoint{1.602003in}{2.213770in}}%
\pgfpathcurveto{\pgfqpoint{1.609817in}{2.221584in}}{\pgfqpoint{1.614207in}{2.232183in}}{\pgfqpoint{1.614207in}{2.243233in}}%
\pgfpathcurveto{\pgfqpoint{1.614207in}{2.254283in}}{\pgfqpoint{1.609817in}{2.264882in}}{\pgfqpoint{1.602003in}{2.272696in}}%
\pgfpathcurveto{\pgfqpoint{1.594190in}{2.280510in}}{\pgfqpoint{1.583591in}{2.284900in}}{\pgfqpoint{1.572540in}{2.284900in}}%
\pgfpathcurveto{\pgfqpoint{1.561490in}{2.284900in}}{\pgfqpoint{1.550891in}{2.280510in}}{\pgfqpoint{1.543078in}{2.272696in}}%
\pgfpathcurveto{\pgfqpoint{1.535264in}{2.264882in}}{\pgfqpoint{1.530874in}{2.254283in}}{\pgfqpoint{1.530874in}{2.243233in}}%
\pgfpathcurveto{\pgfqpoint{1.530874in}{2.232183in}}{\pgfqpoint{1.535264in}{2.221584in}}{\pgfqpoint{1.543078in}{2.213770in}}%
\pgfpathcurveto{\pgfqpoint{1.550891in}{2.205957in}}{\pgfqpoint{1.561490in}{2.201567in}}{\pgfqpoint{1.572540in}{2.201567in}}%
\pgfpathlineto{\pgfqpoint{1.572540in}{2.201567in}}%
\pgfpathclose%
\pgfusepath{stroke}%
\end{pgfscope}%
\begin{pgfscope}%
\pgfpathrectangle{\pgfqpoint{0.847223in}{0.554012in}}{\pgfqpoint{6.200000in}{4.620000in}}%
\pgfusepath{clip}%
\pgfsetbuttcap%
\pgfsetroundjoin%
\pgfsetlinewidth{1.003750pt}%
\definecolor{currentstroke}{rgb}{1.000000,0.000000,0.000000}%
\pgfsetstrokecolor{currentstroke}%
\pgfsetdash{}{0pt}%
\pgfpathmoveto{\pgfqpoint{1.577874in}{2.192573in}}%
\pgfpathcurveto{\pgfqpoint{1.588924in}{2.192573in}}{\pgfqpoint{1.599523in}{2.196963in}}{\pgfqpoint{1.607336in}{2.204777in}}%
\pgfpathcurveto{\pgfqpoint{1.615150in}{2.212590in}}{\pgfqpoint{1.619540in}{2.223189in}}{\pgfqpoint{1.619540in}{2.234240in}}%
\pgfpathcurveto{\pgfqpoint{1.619540in}{2.245290in}}{\pgfqpoint{1.615150in}{2.255889in}}{\pgfqpoint{1.607336in}{2.263702in}}%
\pgfpathcurveto{\pgfqpoint{1.599523in}{2.271516in}}{\pgfqpoint{1.588924in}{2.275906in}}{\pgfqpoint{1.577874in}{2.275906in}}%
\pgfpathcurveto{\pgfqpoint{1.566823in}{2.275906in}}{\pgfqpoint{1.556224in}{2.271516in}}{\pgfqpoint{1.548411in}{2.263702in}}%
\pgfpathcurveto{\pgfqpoint{1.540597in}{2.255889in}}{\pgfqpoint{1.536207in}{2.245290in}}{\pgfqpoint{1.536207in}{2.234240in}}%
\pgfpathcurveto{\pgfqpoint{1.536207in}{2.223189in}}{\pgfqpoint{1.540597in}{2.212590in}}{\pgfqpoint{1.548411in}{2.204777in}}%
\pgfpathcurveto{\pgfqpoint{1.556224in}{2.196963in}}{\pgfqpoint{1.566823in}{2.192573in}}{\pgfqpoint{1.577874in}{2.192573in}}%
\pgfpathlineto{\pgfqpoint{1.577874in}{2.192573in}}%
\pgfpathclose%
\pgfusepath{stroke}%
\end{pgfscope}%
\begin{pgfscope}%
\pgfpathrectangle{\pgfqpoint{0.847223in}{0.554012in}}{\pgfqpoint{6.200000in}{4.620000in}}%
\pgfusepath{clip}%
\pgfsetbuttcap%
\pgfsetroundjoin%
\pgfsetlinewidth{1.003750pt}%
\definecolor{currentstroke}{rgb}{1.000000,0.000000,0.000000}%
\pgfsetstrokecolor{currentstroke}%
\pgfsetdash{}{0pt}%
\pgfpathmoveto{\pgfqpoint{1.583207in}{2.183655in}}%
\pgfpathcurveto{\pgfqpoint{1.594257in}{2.183655in}}{\pgfqpoint{1.604856in}{2.188046in}}{\pgfqpoint{1.612670in}{2.195859in}}%
\pgfpathcurveto{\pgfqpoint{1.620483in}{2.203673in}}{\pgfqpoint{1.624873in}{2.214272in}}{\pgfqpoint{1.624873in}{2.225322in}}%
\pgfpathcurveto{\pgfqpoint{1.624873in}{2.236372in}}{\pgfqpoint{1.620483in}{2.246971in}}{\pgfqpoint{1.612670in}{2.254785in}}%
\pgfpathcurveto{\pgfqpoint{1.604856in}{2.262598in}}{\pgfqpoint{1.594257in}{2.266989in}}{\pgfqpoint{1.583207in}{2.266989in}}%
\pgfpathcurveto{\pgfqpoint{1.572157in}{2.266989in}}{\pgfqpoint{1.561558in}{2.262598in}}{\pgfqpoint{1.553744in}{2.254785in}}%
\pgfpathcurveto{\pgfqpoint{1.545930in}{2.246971in}}{\pgfqpoint{1.541540in}{2.236372in}}{\pgfqpoint{1.541540in}{2.225322in}}%
\pgfpathcurveto{\pgfqpoint{1.541540in}{2.214272in}}{\pgfqpoint{1.545930in}{2.203673in}}{\pgfqpoint{1.553744in}{2.195859in}}%
\pgfpathcurveto{\pgfqpoint{1.561558in}{2.188046in}}{\pgfqpoint{1.572157in}{2.183655in}}{\pgfqpoint{1.583207in}{2.183655in}}%
\pgfpathlineto{\pgfqpoint{1.583207in}{2.183655in}}%
\pgfpathclose%
\pgfusepath{stroke}%
\end{pgfscope}%
\begin{pgfscope}%
\pgfpathrectangle{\pgfqpoint{0.847223in}{0.554012in}}{\pgfqpoint{6.200000in}{4.620000in}}%
\pgfusepath{clip}%
\pgfsetbuttcap%
\pgfsetroundjoin%
\pgfsetlinewidth{1.003750pt}%
\definecolor{currentstroke}{rgb}{1.000000,0.000000,0.000000}%
\pgfsetstrokecolor{currentstroke}%
\pgfsetdash{}{0pt}%
\pgfpathmoveto{\pgfqpoint{1.588540in}{2.174813in}}%
\pgfpathcurveto{\pgfqpoint{1.599590in}{2.174813in}}{\pgfqpoint{1.610189in}{2.179203in}}{\pgfqpoint{1.618003in}{2.187017in}}%
\pgfpathcurveto{\pgfqpoint{1.625816in}{2.194830in}}{\pgfqpoint{1.630207in}{2.205430in}}{\pgfqpoint{1.630207in}{2.216480in}}%
\pgfpathcurveto{\pgfqpoint{1.630207in}{2.227530in}}{\pgfqpoint{1.625816in}{2.238129in}}{\pgfqpoint{1.618003in}{2.245942in}}%
\pgfpathcurveto{\pgfqpoint{1.610189in}{2.253756in}}{\pgfqpoint{1.599590in}{2.258146in}}{\pgfqpoint{1.588540in}{2.258146in}}%
\pgfpathcurveto{\pgfqpoint{1.577490in}{2.258146in}}{\pgfqpoint{1.566891in}{2.253756in}}{\pgfqpoint{1.559077in}{2.245942in}}%
\pgfpathcurveto{\pgfqpoint{1.551264in}{2.238129in}}{\pgfqpoint{1.546873in}{2.227530in}}{\pgfqpoint{1.546873in}{2.216480in}}%
\pgfpathcurveto{\pgfqpoint{1.546873in}{2.205430in}}{\pgfqpoint{1.551264in}{2.194830in}}{\pgfqpoint{1.559077in}{2.187017in}}%
\pgfpathcurveto{\pgfqpoint{1.566891in}{2.179203in}}{\pgfqpoint{1.577490in}{2.174813in}}{\pgfqpoint{1.588540in}{2.174813in}}%
\pgfpathlineto{\pgfqpoint{1.588540in}{2.174813in}}%
\pgfpathclose%
\pgfusepath{stroke}%
\end{pgfscope}%
\begin{pgfscope}%
\pgfpathrectangle{\pgfqpoint{0.847223in}{0.554012in}}{\pgfqpoint{6.200000in}{4.620000in}}%
\pgfusepath{clip}%
\pgfsetbuttcap%
\pgfsetroundjoin%
\pgfsetlinewidth{1.003750pt}%
\definecolor{currentstroke}{rgb}{1.000000,0.000000,0.000000}%
\pgfsetstrokecolor{currentstroke}%
\pgfsetdash{}{0pt}%
\pgfpathmoveto{\pgfqpoint{1.593873in}{2.166045in}}%
\pgfpathcurveto{\pgfqpoint{1.604923in}{2.166045in}}{\pgfqpoint{1.615522in}{2.170435in}}{\pgfqpoint{1.623336in}{2.178249in}}%
\pgfpathcurveto{\pgfqpoint{1.631150in}{2.186062in}}{\pgfqpoint{1.635540in}{2.196661in}}{\pgfqpoint{1.635540in}{2.207712in}}%
\pgfpathcurveto{\pgfqpoint{1.635540in}{2.218762in}}{\pgfqpoint{1.631150in}{2.229361in}}{\pgfqpoint{1.623336in}{2.237174in}}%
\pgfpathcurveto{\pgfqpoint{1.615522in}{2.244988in}}{\pgfqpoint{1.604923in}{2.249378in}}{\pgfqpoint{1.593873in}{2.249378in}}%
\pgfpathcurveto{\pgfqpoint{1.582823in}{2.249378in}}{\pgfqpoint{1.572224in}{2.244988in}}{\pgfqpoint{1.564410in}{2.237174in}}%
\pgfpathcurveto{\pgfqpoint{1.556597in}{2.229361in}}{\pgfqpoint{1.552207in}{2.218762in}}{\pgfqpoint{1.552207in}{2.207712in}}%
\pgfpathcurveto{\pgfqpoint{1.552207in}{2.196661in}}{\pgfqpoint{1.556597in}{2.186062in}}{\pgfqpoint{1.564410in}{2.178249in}}%
\pgfpathcurveto{\pgfqpoint{1.572224in}{2.170435in}}{\pgfqpoint{1.582823in}{2.166045in}}{\pgfqpoint{1.593873in}{2.166045in}}%
\pgfpathlineto{\pgfqpoint{1.593873in}{2.166045in}}%
\pgfpathclose%
\pgfusepath{stroke}%
\end{pgfscope}%
\begin{pgfscope}%
\pgfpathrectangle{\pgfqpoint{0.847223in}{0.554012in}}{\pgfqpoint{6.200000in}{4.620000in}}%
\pgfusepath{clip}%
\pgfsetbuttcap%
\pgfsetroundjoin%
\pgfsetlinewidth{1.003750pt}%
\definecolor{currentstroke}{rgb}{1.000000,0.000000,0.000000}%
\pgfsetstrokecolor{currentstroke}%
\pgfsetdash{}{0pt}%
\pgfpathmoveto{\pgfqpoint{1.599206in}{2.157350in}}%
\pgfpathcurveto{\pgfqpoint{1.610257in}{2.157350in}}{\pgfqpoint{1.620856in}{2.161740in}}{\pgfqpoint{1.628669in}{2.169554in}}%
\pgfpathcurveto{\pgfqpoint{1.636483in}{2.177368in}}{\pgfqpoint{1.640873in}{2.187967in}}{\pgfqpoint{1.640873in}{2.199017in}}%
\pgfpathcurveto{\pgfqpoint{1.640873in}{2.210067in}}{\pgfqpoint{1.636483in}{2.220666in}}{\pgfqpoint{1.628669in}{2.228480in}}%
\pgfpathcurveto{\pgfqpoint{1.620856in}{2.236293in}}{\pgfqpoint{1.610257in}{2.240684in}}{\pgfqpoint{1.599206in}{2.240684in}}%
\pgfpathcurveto{\pgfqpoint{1.588156in}{2.240684in}}{\pgfqpoint{1.577557in}{2.236293in}}{\pgfqpoint{1.569744in}{2.228480in}}%
\pgfpathcurveto{\pgfqpoint{1.561930in}{2.220666in}}{\pgfqpoint{1.557540in}{2.210067in}}{\pgfqpoint{1.557540in}{2.199017in}}%
\pgfpathcurveto{\pgfqpoint{1.557540in}{2.187967in}}{\pgfqpoint{1.561930in}{2.177368in}}{\pgfqpoint{1.569744in}{2.169554in}}%
\pgfpathcurveto{\pgfqpoint{1.577557in}{2.161740in}}{\pgfqpoint{1.588156in}{2.157350in}}{\pgfqpoint{1.599206in}{2.157350in}}%
\pgfpathlineto{\pgfqpoint{1.599206in}{2.157350in}}%
\pgfpathclose%
\pgfusepath{stroke}%
\end{pgfscope}%
\begin{pgfscope}%
\pgfpathrectangle{\pgfqpoint{0.847223in}{0.554012in}}{\pgfqpoint{6.200000in}{4.620000in}}%
\pgfusepath{clip}%
\pgfsetbuttcap%
\pgfsetroundjoin%
\pgfsetlinewidth{1.003750pt}%
\definecolor{currentstroke}{rgb}{1.000000,0.000000,0.000000}%
\pgfsetstrokecolor{currentstroke}%
\pgfsetdash{}{0pt}%
\pgfpathmoveto{\pgfqpoint{1.604540in}{2.148728in}}%
\pgfpathcurveto{\pgfqpoint{1.615590in}{2.148728in}}{\pgfqpoint{1.626189in}{2.153118in}}{\pgfqpoint{1.634002in}{2.160932in}}%
\pgfpathcurveto{\pgfqpoint{1.641816in}{2.168745in}}{\pgfqpoint{1.646206in}{2.179344in}}{\pgfqpoint{1.646206in}{2.190395in}}%
\pgfpathcurveto{\pgfqpoint{1.646206in}{2.201445in}}{\pgfqpoint{1.641816in}{2.212044in}}{\pgfqpoint{1.634002in}{2.219857in}}%
\pgfpathcurveto{\pgfqpoint{1.626189in}{2.227671in}}{\pgfqpoint{1.615590in}{2.232061in}}{\pgfqpoint{1.604540in}{2.232061in}}%
\pgfpathcurveto{\pgfqpoint{1.593490in}{2.232061in}}{\pgfqpoint{1.582891in}{2.227671in}}{\pgfqpoint{1.575077in}{2.219857in}}%
\pgfpathcurveto{\pgfqpoint{1.567263in}{2.212044in}}{\pgfqpoint{1.562873in}{2.201445in}}{\pgfqpoint{1.562873in}{2.190395in}}%
\pgfpathcurveto{\pgfqpoint{1.562873in}{2.179344in}}{\pgfqpoint{1.567263in}{2.168745in}}{\pgfqpoint{1.575077in}{2.160932in}}%
\pgfpathcurveto{\pgfqpoint{1.582891in}{2.153118in}}{\pgfqpoint{1.593490in}{2.148728in}}{\pgfqpoint{1.604540in}{2.148728in}}%
\pgfpathlineto{\pgfqpoint{1.604540in}{2.148728in}}%
\pgfpathclose%
\pgfusepath{stroke}%
\end{pgfscope}%
\begin{pgfscope}%
\pgfpathrectangle{\pgfqpoint{0.847223in}{0.554012in}}{\pgfqpoint{6.200000in}{4.620000in}}%
\pgfusepath{clip}%
\pgfsetbuttcap%
\pgfsetroundjoin%
\pgfsetlinewidth{1.003750pt}%
\definecolor{currentstroke}{rgb}{1.000000,0.000000,0.000000}%
\pgfsetstrokecolor{currentstroke}%
\pgfsetdash{}{0pt}%
\pgfpathmoveto{\pgfqpoint{1.609873in}{2.140177in}}%
\pgfpathcurveto{\pgfqpoint{1.620923in}{2.140177in}}{\pgfqpoint{1.631522in}{2.144567in}}{\pgfqpoint{1.639336in}{2.152381in}}%
\pgfpathcurveto{\pgfqpoint{1.647149in}{2.160195in}}{\pgfqpoint{1.651540in}{2.170794in}}{\pgfqpoint{1.651540in}{2.181844in}}%
\pgfpathcurveto{\pgfqpoint{1.651540in}{2.192894in}}{\pgfqpoint{1.647149in}{2.203493in}}{\pgfqpoint{1.639336in}{2.211307in}}%
\pgfpathcurveto{\pgfqpoint{1.631522in}{2.219120in}}{\pgfqpoint{1.620923in}{2.223510in}}{\pgfqpoint{1.609873in}{2.223510in}}%
\pgfpathcurveto{\pgfqpoint{1.598823in}{2.223510in}}{\pgfqpoint{1.588224in}{2.219120in}}{\pgfqpoint{1.580410in}{2.211307in}}%
\pgfpathcurveto{\pgfqpoint{1.572596in}{2.203493in}}{\pgfqpoint{1.568206in}{2.192894in}}{\pgfqpoint{1.568206in}{2.181844in}}%
\pgfpathcurveto{\pgfqpoint{1.568206in}{2.170794in}}{\pgfqpoint{1.572596in}{2.160195in}}{\pgfqpoint{1.580410in}{2.152381in}}%
\pgfpathcurveto{\pgfqpoint{1.588224in}{2.144567in}}{\pgfqpoint{1.598823in}{2.140177in}}{\pgfqpoint{1.609873in}{2.140177in}}%
\pgfpathlineto{\pgfqpoint{1.609873in}{2.140177in}}%
\pgfpathclose%
\pgfusepath{stroke}%
\end{pgfscope}%
\begin{pgfscope}%
\pgfpathrectangle{\pgfqpoint{0.847223in}{0.554012in}}{\pgfqpoint{6.200000in}{4.620000in}}%
\pgfusepath{clip}%
\pgfsetbuttcap%
\pgfsetroundjoin%
\pgfsetlinewidth{1.003750pt}%
\definecolor{currentstroke}{rgb}{1.000000,0.000000,0.000000}%
\pgfsetstrokecolor{currentstroke}%
\pgfsetdash{}{0pt}%
\pgfpathmoveto{\pgfqpoint{1.615206in}{2.131697in}}%
\pgfpathcurveto{\pgfqpoint{1.626256in}{2.131697in}}{\pgfqpoint{1.636855in}{2.136087in}}{\pgfqpoint{1.644669in}{2.143901in}}%
\pgfpathcurveto{\pgfqpoint{1.652483in}{2.151714in}}{\pgfqpoint{1.656873in}{2.162314in}}{\pgfqpoint{1.656873in}{2.173364in}}%
\pgfpathcurveto{\pgfqpoint{1.656873in}{2.184414in}}{\pgfqpoint{1.652483in}{2.195013in}}{\pgfqpoint{1.644669in}{2.202826in}}%
\pgfpathcurveto{\pgfqpoint{1.636855in}{2.210640in}}{\pgfqpoint{1.626256in}{2.215030in}}{\pgfqpoint{1.615206in}{2.215030in}}%
\pgfpathcurveto{\pgfqpoint{1.604156in}{2.215030in}}{\pgfqpoint{1.593557in}{2.210640in}}{\pgfqpoint{1.585743in}{2.202826in}}%
\pgfpathcurveto{\pgfqpoint{1.577930in}{2.195013in}}{\pgfqpoint{1.573539in}{2.184414in}}{\pgfqpoint{1.573539in}{2.173364in}}%
\pgfpathcurveto{\pgfqpoint{1.573539in}{2.162314in}}{\pgfqpoint{1.577930in}{2.151714in}}{\pgfqpoint{1.585743in}{2.143901in}}%
\pgfpathcurveto{\pgfqpoint{1.593557in}{2.136087in}}{\pgfqpoint{1.604156in}{2.131697in}}{\pgfqpoint{1.615206in}{2.131697in}}%
\pgfpathlineto{\pgfqpoint{1.615206in}{2.131697in}}%
\pgfpathclose%
\pgfusepath{stroke}%
\end{pgfscope}%
\begin{pgfscope}%
\pgfpathrectangle{\pgfqpoint{0.847223in}{0.554012in}}{\pgfqpoint{6.200000in}{4.620000in}}%
\pgfusepath{clip}%
\pgfsetbuttcap%
\pgfsetroundjoin%
\pgfsetlinewidth{1.003750pt}%
\definecolor{currentstroke}{rgb}{1.000000,0.000000,0.000000}%
\pgfsetstrokecolor{currentstroke}%
\pgfsetdash{}{0pt}%
\pgfpathmoveto{\pgfqpoint{1.620539in}{2.123287in}}%
\pgfpathcurveto{\pgfqpoint{1.631589in}{2.123287in}}{\pgfqpoint{1.642188in}{2.127677in}}{\pgfqpoint{1.650002in}{2.135490in}}%
\pgfpathcurveto{\pgfqpoint{1.657816in}{2.143304in}}{\pgfqpoint{1.662206in}{2.153903in}}{\pgfqpoint{1.662206in}{2.164953in}}%
\pgfpathcurveto{\pgfqpoint{1.662206in}{2.176003in}}{\pgfqpoint{1.657816in}{2.186602in}}{\pgfqpoint{1.650002in}{2.194416in}}%
\pgfpathcurveto{\pgfqpoint{1.642188in}{2.202230in}}{\pgfqpoint{1.631589in}{2.206620in}}{\pgfqpoint{1.620539in}{2.206620in}}%
\pgfpathcurveto{\pgfqpoint{1.609489in}{2.206620in}}{\pgfqpoint{1.598890in}{2.202230in}}{\pgfqpoint{1.591077in}{2.194416in}}%
\pgfpathcurveto{\pgfqpoint{1.583263in}{2.186602in}}{\pgfqpoint{1.578873in}{2.176003in}}{\pgfqpoint{1.578873in}{2.164953in}}%
\pgfpathcurveto{\pgfqpoint{1.578873in}{2.153903in}}{\pgfqpoint{1.583263in}{2.143304in}}{\pgfqpoint{1.591077in}{2.135490in}}%
\pgfpathcurveto{\pgfqpoint{1.598890in}{2.127677in}}{\pgfqpoint{1.609489in}{2.123287in}}{\pgfqpoint{1.620539in}{2.123287in}}%
\pgfpathlineto{\pgfqpoint{1.620539in}{2.123287in}}%
\pgfpathclose%
\pgfusepath{stroke}%
\end{pgfscope}%
\begin{pgfscope}%
\pgfpathrectangle{\pgfqpoint{0.847223in}{0.554012in}}{\pgfqpoint{6.200000in}{4.620000in}}%
\pgfusepath{clip}%
\pgfsetbuttcap%
\pgfsetroundjoin%
\pgfsetlinewidth{1.003750pt}%
\definecolor{currentstroke}{rgb}{1.000000,0.000000,0.000000}%
\pgfsetstrokecolor{currentstroke}%
\pgfsetdash{}{0pt}%
\pgfpathmoveto{\pgfqpoint{1.625873in}{2.114945in}}%
\pgfpathcurveto{\pgfqpoint{1.636923in}{2.114945in}}{\pgfqpoint{1.647522in}{2.119335in}}{\pgfqpoint{1.655335in}{2.127149in}}%
\pgfpathcurveto{\pgfqpoint{1.663149in}{2.134963in}}{\pgfqpoint{1.667539in}{2.145562in}}{\pgfqpoint{1.667539in}{2.156612in}}%
\pgfpathcurveto{\pgfqpoint{1.667539in}{2.167662in}}{\pgfqpoint{1.663149in}{2.178261in}}{\pgfqpoint{1.655335in}{2.186075in}}%
\pgfpathcurveto{\pgfqpoint{1.647522in}{2.193888in}}{\pgfqpoint{1.636923in}{2.198278in}}{\pgfqpoint{1.625873in}{2.198278in}}%
\pgfpathcurveto{\pgfqpoint{1.614822in}{2.198278in}}{\pgfqpoint{1.604223in}{2.193888in}}{\pgfqpoint{1.596410in}{2.186075in}}%
\pgfpathcurveto{\pgfqpoint{1.588596in}{2.178261in}}{\pgfqpoint{1.584206in}{2.167662in}}{\pgfqpoint{1.584206in}{2.156612in}}%
\pgfpathcurveto{\pgfqpoint{1.584206in}{2.145562in}}{\pgfqpoint{1.588596in}{2.134963in}}{\pgfqpoint{1.596410in}{2.127149in}}%
\pgfpathcurveto{\pgfqpoint{1.604223in}{2.119335in}}{\pgfqpoint{1.614822in}{2.114945in}}{\pgfqpoint{1.625873in}{2.114945in}}%
\pgfpathlineto{\pgfqpoint{1.625873in}{2.114945in}}%
\pgfpathclose%
\pgfusepath{stroke}%
\end{pgfscope}%
\begin{pgfscope}%
\pgfpathrectangle{\pgfqpoint{0.847223in}{0.554012in}}{\pgfqpoint{6.200000in}{4.620000in}}%
\pgfusepath{clip}%
\pgfsetbuttcap%
\pgfsetroundjoin%
\pgfsetlinewidth{1.003750pt}%
\definecolor{currentstroke}{rgb}{1.000000,0.000000,0.000000}%
\pgfsetstrokecolor{currentstroke}%
\pgfsetdash{}{0pt}%
\pgfpathmoveto{\pgfqpoint{1.631206in}{2.106672in}}%
\pgfpathcurveto{\pgfqpoint{1.642256in}{2.106672in}}{\pgfqpoint{1.652855in}{2.111062in}}{\pgfqpoint{1.660669in}{2.118876in}}%
\pgfpathcurveto{\pgfqpoint{1.668482in}{2.126689in}}{\pgfqpoint{1.672872in}{2.137288in}}{\pgfqpoint{1.672872in}{2.148338in}}%
\pgfpathcurveto{\pgfqpoint{1.672872in}{2.159388in}}{\pgfqpoint{1.668482in}{2.169987in}}{\pgfqpoint{1.660669in}{2.177801in}}%
\pgfpathcurveto{\pgfqpoint{1.652855in}{2.185615in}}{\pgfqpoint{1.642256in}{2.190005in}}{\pgfqpoint{1.631206in}{2.190005in}}%
\pgfpathcurveto{\pgfqpoint{1.620156in}{2.190005in}}{\pgfqpoint{1.609557in}{2.185615in}}{\pgfqpoint{1.601743in}{2.177801in}}%
\pgfpathcurveto{\pgfqpoint{1.593929in}{2.169987in}}{\pgfqpoint{1.589539in}{2.159388in}}{\pgfqpoint{1.589539in}{2.148338in}}%
\pgfpathcurveto{\pgfqpoint{1.589539in}{2.137288in}}{\pgfqpoint{1.593929in}{2.126689in}}{\pgfqpoint{1.601743in}{2.118876in}}%
\pgfpathcurveto{\pgfqpoint{1.609557in}{2.111062in}}{\pgfqpoint{1.620156in}{2.106672in}}{\pgfqpoint{1.631206in}{2.106672in}}%
\pgfpathlineto{\pgfqpoint{1.631206in}{2.106672in}}%
\pgfpathclose%
\pgfusepath{stroke}%
\end{pgfscope}%
\begin{pgfscope}%
\pgfpathrectangle{\pgfqpoint{0.847223in}{0.554012in}}{\pgfqpoint{6.200000in}{4.620000in}}%
\pgfusepath{clip}%
\pgfsetbuttcap%
\pgfsetroundjoin%
\pgfsetlinewidth{1.003750pt}%
\definecolor{currentstroke}{rgb}{1.000000,0.000000,0.000000}%
\pgfsetstrokecolor{currentstroke}%
\pgfsetdash{}{0pt}%
\pgfpathmoveto{\pgfqpoint{1.636539in}{2.098465in}}%
\pgfpathcurveto{\pgfqpoint{1.647589in}{2.098465in}}{\pgfqpoint{1.658188in}{2.102856in}}{\pgfqpoint{1.666002in}{2.110669in}}%
\pgfpathcurveto{\pgfqpoint{1.673815in}{2.118483in}}{\pgfqpoint{1.678206in}{2.129082in}}{\pgfqpoint{1.678206in}{2.140132in}}%
\pgfpathcurveto{\pgfqpoint{1.678206in}{2.151182in}}{\pgfqpoint{1.673815in}{2.161781in}}{\pgfqpoint{1.666002in}{2.169595in}}%
\pgfpathcurveto{\pgfqpoint{1.658188in}{2.177408in}}{\pgfqpoint{1.647589in}{2.181799in}}{\pgfqpoint{1.636539in}{2.181799in}}%
\pgfpathcurveto{\pgfqpoint{1.625489in}{2.181799in}}{\pgfqpoint{1.614890in}{2.177408in}}{\pgfqpoint{1.607076in}{2.169595in}}%
\pgfpathcurveto{\pgfqpoint{1.599263in}{2.161781in}}{\pgfqpoint{1.594872in}{2.151182in}}{\pgfqpoint{1.594872in}{2.140132in}}%
\pgfpathcurveto{\pgfqpoint{1.594872in}{2.129082in}}{\pgfqpoint{1.599263in}{2.118483in}}{\pgfqpoint{1.607076in}{2.110669in}}%
\pgfpathcurveto{\pgfqpoint{1.614890in}{2.102856in}}{\pgfqpoint{1.625489in}{2.098465in}}{\pgfqpoint{1.636539in}{2.098465in}}%
\pgfpathlineto{\pgfqpoint{1.636539in}{2.098465in}}%
\pgfpathclose%
\pgfusepath{stroke}%
\end{pgfscope}%
\begin{pgfscope}%
\pgfpathrectangle{\pgfqpoint{0.847223in}{0.554012in}}{\pgfqpoint{6.200000in}{4.620000in}}%
\pgfusepath{clip}%
\pgfsetbuttcap%
\pgfsetroundjoin%
\pgfsetlinewidth{1.003750pt}%
\definecolor{currentstroke}{rgb}{1.000000,0.000000,0.000000}%
\pgfsetstrokecolor{currentstroke}%
\pgfsetdash{}{0pt}%
\pgfpathmoveto{\pgfqpoint{1.641872in}{2.090326in}}%
\pgfpathcurveto{\pgfqpoint{1.652922in}{2.090326in}}{\pgfqpoint{1.663521in}{2.094716in}}{\pgfqpoint{1.671335in}{2.102529in}}%
\pgfpathcurveto{\pgfqpoint{1.679149in}{2.110343in}}{\pgfqpoint{1.683539in}{2.120942in}}{\pgfqpoint{1.683539in}{2.131992in}}%
\pgfpathcurveto{\pgfqpoint{1.683539in}{2.143042in}}{\pgfqpoint{1.679149in}{2.153641in}}{\pgfqpoint{1.671335in}{2.161455in}}%
\pgfpathcurveto{\pgfqpoint{1.663521in}{2.169269in}}{\pgfqpoint{1.652922in}{2.173659in}}{\pgfqpoint{1.641872in}{2.173659in}}%
\pgfpathcurveto{\pgfqpoint{1.630822in}{2.173659in}}{\pgfqpoint{1.620223in}{2.169269in}}{\pgfqpoint{1.612409in}{2.161455in}}%
\pgfpathcurveto{\pgfqpoint{1.604596in}{2.153641in}}{\pgfqpoint{1.600206in}{2.143042in}}{\pgfqpoint{1.600206in}{2.131992in}}%
\pgfpathcurveto{\pgfqpoint{1.600206in}{2.120942in}}{\pgfqpoint{1.604596in}{2.110343in}}{\pgfqpoint{1.612409in}{2.102529in}}%
\pgfpathcurveto{\pgfqpoint{1.620223in}{2.094716in}}{\pgfqpoint{1.630822in}{2.090326in}}{\pgfqpoint{1.641872in}{2.090326in}}%
\pgfpathlineto{\pgfqpoint{1.641872in}{2.090326in}}%
\pgfpathclose%
\pgfusepath{stroke}%
\end{pgfscope}%
\begin{pgfscope}%
\pgfpathrectangle{\pgfqpoint{0.847223in}{0.554012in}}{\pgfqpoint{6.200000in}{4.620000in}}%
\pgfusepath{clip}%
\pgfsetbuttcap%
\pgfsetroundjoin%
\pgfsetlinewidth{1.003750pt}%
\definecolor{currentstroke}{rgb}{1.000000,0.000000,0.000000}%
\pgfsetstrokecolor{currentstroke}%
\pgfsetdash{}{0pt}%
\pgfpathmoveto{\pgfqpoint{1.647205in}{2.082251in}}%
\pgfpathcurveto{\pgfqpoint{1.658256in}{2.082251in}}{\pgfqpoint{1.668855in}{2.086642in}}{\pgfqpoint{1.676668in}{2.094455in}}%
\pgfpathcurveto{\pgfqpoint{1.684482in}{2.102269in}}{\pgfqpoint{1.688872in}{2.112868in}}{\pgfqpoint{1.688872in}{2.123918in}}%
\pgfpathcurveto{\pgfqpoint{1.688872in}{2.134968in}}{\pgfqpoint{1.684482in}{2.145567in}}{\pgfqpoint{1.676668in}{2.153381in}}%
\pgfpathcurveto{\pgfqpoint{1.668855in}{2.161194in}}{\pgfqpoint{1.658256in}{2.165585in}}{\pgfqpoint{1.647205in}{2.165585in}}%
\pgfpathcurveto{\pgfqpoint{1.636155in}{2.165585in}}{\pgfqpoint{1.625556in}{2.161194in}}{\pgfqpoint{1.617743in}{2.153381in}}%
\pgfpathcurveto{\pgfqpoint{1.609929in}{2.145567in}}{\pgfqpoint{1.605539in}{2.134968in}}{\pgfqpoint{1.605539in}{2.123918in}}%
\pgfpathcurveto{\pgfqpoint{1.605539in}{2.112868in}}{\pgfqpoint{1.609929in}{2.102269in}}{\pgfqpoint{1.617743in}{2.094455in}}%
\pgfpathcurveto{\pgfqpoint{1.625556in}{2.086642in}}{\pgfqpoint{1.636155in}{2.082251in}}{\pgfqpoint{1.647205in}{2.082251in}}%
\pgfpathlineto{\pgfqpoint{1.647205in}{2.082251in}}%
\pgfpathclose%
\pgfusepath{stroke}%
\end{pgfscope}%
\begin{pgfscope}%
\pgfpathrectangle{\pgfqpoint{0.847223in}{0.554012in}}{\pgfqpoint{6.200000in}{4.620000in}}%
\pgfusepath{clip}%
\pgfsetbuttcap%
\pgfsetroundjoin%
\pgfsetlinewidth{1.003750pt}%
\definecolor{currentstroke}{rgb}{1.000000,0.000000,0.000000}%
\pgfsetstrokecolor{currentstroke}%
\pgfsetdash{}{0pt}%
\pgfpathmoveto{\pgfqpoint{1.652539in}{2.074242in}}%
\pgfpathcurveto{\pgfqpoint{1.663589in}{2.074242in}}{\pgfqpoint{1.674188in}{2.078632in}}{\pgfqpoint{1.682001in}{2.086446in}}%
\pgfpathcurveto{\pgfqpoint{1.689815in}{2.094260in}}{\pgfqpoint{1.694205in}{2.104859in}}{\pgfqpoint{1.694205in}{2.115909in}}%
\pgfpathcurveto{\pgfqpoint{1.694205in}{2.126959in}}{\pgfqpoint{1.689815in}{2.137558in}}{\pgfqpoint{1.682001in}{2.145371in}}%
\pgfpathcurveto{\pgfqpoint{1.674188in}{2.153185in}}{\pgfqpoint{1.663589in}{2.157575in}}{\pgfqpoint{1.652539in}{2.157575in}}%
\pgfpathcurveto{\pgfqpoint{1.641488in}{2.157575in}}{\pgfqpoint{1.630889in}{2.153185in}}{\pgfqpoint{1.623076in}{2.145371in}}%
\pgfpathcurveto{\pgfqpoint{1.615262in}{2.137558in}}{\pgfqpoint{1.610872in}{2.126959in}}{\pgfqpoint{1.610872in}{2.115909in}}%
\pgfpathcurveto{\pgfqpoint{1.610872in}{2.104859in}}{\pgfqpoint{1.615262in}{2.094260in}}{\pgfqpoint{1.623076in}{2.086446in}}%
\pgfpathcurveto{\pgfqpoint{1.630889in}{2.078632in}}{\pgfqpoint{1.641488in}{2.074242in}}{\pgfqpoint{1.652539in}{2.074242in}}%
\pgfpathlineto{\pgfqpoint{1.652539in}{2.074242in}}%
\pgfpathclose%
\pgfusepath{stroke}%
\end{pgfscope}%
\begin{pgfscope}%
\pgfpathrectangle{\pgfqpoint{0.847223in}{0.554012in}}{\pgfqpoint{6.200000in}{4.620000in}}%
\pgfusepath{clip}%
\pgfsetbuttcap%
\pgfsetroundjoin%
\pgfsetlinewidth{1.003750pt}%
\definecolor{currentstroke}{rgb}{1.000000,0.000000,0.000000}%
\pgfsetstrokecolor{currentstroke}%
\pgfsetdash{}{0pt}%
\pgfpathmoveto{\pgfqpoint{1.657872in}{2.066297in}}%
\pgfpathcurveto{\pgfqpoint{1.668922in}{2.066297in}}{\pgfqpoint{1.679521in}{2.070687in}}{\pgfqpoint{1.687335in}{2.078501in}}%
\pgfpathcurveto{\pgfqpoint{1.695148in}{2.086314in}}{\pgfqpoint{1.699539in}{2.096913in}}{\pgfqpoint{1.699539in}{2.107963in}}%
\pgfpathcurveto{\pgfqpoint{1.699539in}{2.119014in}}{\pgfqpoint{1.695148in}{2.129613in}}{\pgfqpoint{1.687335in}{2.137426in}}%
\pgfpathcurveto{\pgfqpoint{1.679521in}{2.145240in}}{\pgfqpoint{1.668922in}{2.149630in}}{\pgfqpoint{1.657872in}{2.149630in}}%
\pgfpathcurveto{\pgfqpoint{1.646822in}{2.149630in}}{\pgfqpoint{1.636223in}{2.145240in}}{\pgfqpoint{1.628409in}{2.137426in}}%
\pgfpathcurveto{\pgfqpoint{1.620595in}{2.129613in}}{\pgfqpoint{1.616205in}{2.119014in}}{\pgfqpoint{1.616205in}{2.107963in}}%
\pgfpathcurveto{\pgfqpoint{1.616205in}{2.096913in}}{\pgfqpoint{1.620595in}{2.086314in}}{\pgfqpoint{1.628409in}{2.078501in}}%
\pgfpathcurveto{\pgfqpoint{1.636223in}{2.070687in}}{\pgfqpoint{1.646822in}{2.066297in}}{\pgfqpoint{1.657872in}{2.066297in}}%
\pgfpathlineto{\pgfqpoint{1.657872in}{2.066297in}}%
\pgfpathclose%
\pgfusepath{stroke}%
\end{pgfscope}%
\begin{pgfscope}%
\pgfpathrectangle{\pgfqpoint{0.847223in}{0.554012in}}{\pgfqpoint{6.200000in}{4.620000in}}%
\pgfusepath{clip}%
\pgfsetbuttcap%
\pgfsetroundjoin%
\pgfsetlinewidth{1.003750pt}%
\definecolor{currentstroke}{rgb}{1.000000,0.000000,0.000000}%
\pgfsetstrokecolor{currentstroke}%
\pgfsetdash{}{0pt}%
\pgfpathmoveto{\pgfqpoint{1.663205in}{2.058415in}}%
\pgfpathcurveto{\pgfqpoint{1.674255in}{2.058415in}}{\pgfqpoint{1.684854in}{2.062805in}}{\pgfqpoint{1.692668in}{2.070619in}}%
\pgfpathcurveto{\pgfqpoint{1.700481in}{2.078432in}}{\pgfqpoint{1.704872in}{2.089031in}}{\pgfqpoint{1.704872in}{2.100081in}}%
\pgfpathcurveto{\pgfqpoint{1.704872in}{2.111131in}}{\pgfqpoint{1.700481in}{2.121730in}}{\pgfqpoint{1.692668in}{2.129544in}}%
\pgfpathcurveto{\pgfqpoint{1.684854in}{2.137358in}}{\pgfqpoint{1.674255in}{2.141748in}}{\pgfqpoint{1.663205in}{2.141748in}}%
\pgfpathcurveto{\pgfqpoint{1.652155in}{2.141748in}}{\pgfqpoint{1.641556in}{2.137358in}}{\pgfqpoint{1.633742in}{2.129544in}}%
\pgfpathcurveto{\pgfqpoint{1.625929in}{2.121730in}}{\pgfqpoint{1.621538in}{2.111131in}}{\pgfqpoint{1.621538in}{2.100081in}}%
\pgfpathcurveto{\pgfqpoint{1.621538in}{2.089031in}}{\pgfqpoint{1.625929in}{2.078432in}}{\pgfqpoint{1.633742in}{2.070619in}}%
\pgfpathcurveto{\pgfqpoint{1.641556in}{2.062805in}}{\pgfqpoint{1.652155in}{2.058415in}}{\pgfqpoint{1.663205in}{2.058415in}}%
\pgfpathlineto{\pgfqpoint{1.663205in}{2.058415in}}%
\pgfpathclose%
\pgfusepath{stroke}%
\end{pgfscope}%
\begin{pgfscope}%
\pgfpathrectangle{\pgfqpoint{0.847223in}{0.554012in}}{\pgfqpoint{6.200000in}{4.620000in}}%
\pgfusepath{clip}%
\pgfsetbuttcap%
\pgfsetroundjoin%
\pgfsetlinewidth{1.003750pt}%
\definecolor{currentstroke}{rgb}{1.000000,0.000000,0.000000}%
\pgfsetstrokecolor{currentstroke}%
\pgfsetdash{}{0pt}%
\pgfpathmoveto{\pgfqpoint{1.668538in}{2.050595in}}%
\pgfpathcurveto{\pgfqpoint{1.679588in}{2.050595in}}{\pgfqpoint{1.690187in}{2.054985in}}{\pgfqpoint{1.698001in}{2.062799in}}%
\pgfpathcurveto{\pgfqpoint{1.705815in}{2.070613in}}{\pgfqpoint{1.710205in}{2.081212in}}{\pgfqpoint{1.710205in}{2.092262in}}%
\pgfpathcurveto{\pgfqpoint{1.710205in}{2.103312in}}{\pgfqpoint{1.705815in}{2.113911in}}{\pgfqpoint{1.698001in}{2.121725in}}%
\pgfpathcurveto{\pgfqpoint{1.690187in}{2.129538in}}{\pgfqpoint{1.679588in}{2.133928in}}{\pgfqpoint{1.668538in}{2.133928in}}%
\pgfpathcurveto{\pgfqpoint{1.657488in}{2.133928in}}{\pgfqpoint{1.646889in}{2.129538in}}{\pgfqpoint{1.639075in}{2.121725in}}%
\pgfpathcurveto{\pgfqpoint{1.631262in}{2.113911in}}{\pgfqpoint{1.626872in}{2.103312in}}{\pgfqpoint{1.626872in}{2.092262in}}%
\pgfpathcurveto{\pgfqpoint{1.626872in}{2.081212in}}{\pgfqpoint{1.631262in}{2.070613in}}{\pgfqpoint{1.639075in}{2.062799in}}%
\pgfpathcurveto{\pgfqpoint{1.646889in}{2.054985in}}{\pgfqpoint{1.657488in}{2.050595in}}{\pgfqpoint{1.668538in}{2.050595in}}%
\pgfpathlineto{\pgfqpoint{1.668538in}{2.050595in}}%
\pgfpathclose%
\pgfusepath{stroke}%
\end{pgfscope}%
\begin{pgfscope}%
\pgfpathrectangle{\pgfqpoint{0.847223in}{0.554012in}}{\pgfqpoint{6.200000in}{4.620000in}}%
\pgfusepath{clip}%
\pgfsetbuttcap%
\pgfsetroundjoin%
\pgfsetlinewidth{1.003750pt}%
\definecolor{currentstroke}{rgb}{1.000000,0.000000,0.000000}%
\pgfsetstrokecolor{currentstroke}%
\pgfsetdash{}{0pt}%
\pgfpathmoveto{\pgfqpoint{1.673871in}{2.042837in}}%
\pgfpathcurveto{\pgfqpoint{1.684922in}{2.042837in}}{\pgfqpoint{1.695521in}{2.047228in}}{\pgfqpoint{1.703334in}{2.055041in}}%
\pgfpathcurveto{\pgfqpoint{1.711148in}{2.062855in}}{\pgfqpoint{1.715538in}{2.073454in}}{\pgfqpoint{1.715538in}{2.084504in}}%
\pgfpathcurveto{\pgfqpoint{1.715538in}{2.095554in}}{\pgfqpoint{1.711148in}{2.106153in}}{\pgfqpoint{1.703334in}{2.113967in}}%
\pgfpathcurveto{\pgfqpoint{1.695521in}{2.121780in}}{\pgfqpoint{1.684922in}{2.126171in}}{\pgfqpoint{1.673871in}{2.126171in}}%
\pgfpathcurveto{\pgfqpoint{1.662821in}{2.126171in}}{\pgfqpoint{1.652222in}{2.121780in}}{\pgfqpoint{1.644409in}{2.113967in}}%
\pgfpathcurveto{\pgfqpoint{1.636595in}{2.106153in}}{\pgfqpoint{1.632205in}{2.095554in}}{\pgfqpoint{1.632205in}{2.084504in}}%
\pgfpathcurveto{\pgfqpoint{1.632205in}{2.073454in}}{\pgfqpoint{1.636595in}{2.062855in}}{\pgfqpoint{1.644409in}{2.055041in}}%
\pgfpathcurveto{\pgfqpoint{1.652222in}{2.047228in}}{\pgfqpoint{1.662821in}{2.042837in}}{\pgfqpoint{1.673871in}{2.042837in}}%
\pgfpathlineto{\pgfqpoint{1.673871in}{2.042837in}}%
\pgfpathclose%
\pgfusepath{stroke}%
\end{pgfscope}%
\begin{pgfscope}%
\pgfpathrectangle{\pgfqpoint{0.847223in}{0.554012in}}{\pgfqpoint{6.200000in}{4.620000in}}%
\pgfusepath{clip}%
\pgfsetbuttcap%
\pgfsetroundjoin%
\pgfsetlinewidth{1.003750pt}%
\definecolor{currentstroke}{rgb}{1.000000,0.000000,0.000000}%
\pgfsetstrokecolor{currentstroke}%
\pgfsetdash{}{0pt}%
\pgfpathmoveto{\pgfqpoint{1.679205in}{2.035141in}}%
\pgfpathcurveto{\pgfqpoint{1.690255in}{2.035141in}}{\pgfqpoint{1.700854in}{2.039531in}}{\pgfqpoint{1.708667in}{2.047345in}}%
\pgfpathcurveto{\pgfqpoint{1.716481in}{2.055158in}}{\pgfqpoint{1.720871in}{2.065757in}}{\pgfqpoint{1.720871in}{2.076807in}}%
\pgfpathcurveto{\pgfqpoint{1.720871in}{2.087857in}}{\pgfqpoint{1.716481in}{2.098457in}}{\pgfqpoint{1.708667in}{2.106270in}}%
\pgfpathcurveto{\pgfqpoint{1.700854in}{2.114084in}}{\pgfqpoint{1.690255in}{2.118474in}}{\pgfqpoint{1.679205in}{2.118474in}}%
\pgfpathcurveto{\pgfqpoint{1.668155in}{2.118474in}}{\pgfqpoint{1.657556in}{2.114084in}}{\pgfqpoint{1.649742in}{2.106270in}}%
\pgfpathcurveto{\pgfqpoint{1.641928in}{2.098457in}}{\pgfqpoint{1.637538in}{2.087857in}}{\pgfqpoint{1.637538in}{2.076807in}}%
\pgfpathcurveto{\pgfqpoint{1.637538in}{2.065757in}}{\pgfqpoint{1.641928in}{2.055158in}}{\pgfqpoint{1.649742in}{2.047345in}}%
\pgfpathcurveto{\pgfqpoint{1.657556in}{2.039531in}}{\pgfqpoint{1.668155in}{2.035141in}}{\pgfqpoint{1.679205in}{2.035141in}}%
\pgfpathlineto{\pgfqpoint{1.679205in}{2.035141in}}%
\pgfpathclose%
\pgfusepath{stroke}%
\end{pgfscope}%
\begin{pgfscope}%
\pgfpathrectangle{\pgfqpoint{0.847223in}{0.554012in}}{\pgfqpoint{6.200000in}{4.620000in}}%
\pgfusepath{clip}%
\pgfsetbuttcap%
\pgfsetroundjoin%
\pgfsetlinewidth{1.003750pt}%
\definecolor{currentstroke}{rgb}{1.000000,0.000000,0.000000}%
\pgfsetstrokecolor{currentstroke}%
\pgfsetdash{}{0pt}%
\pgfpathmoveto{\pgfqpoint{1.684538in}{2.027504in}}%
\pgfpathcurveto{\pgfqpoint{1.695588in}{2.027504in}}{\pgfqpoint{1.706187in}{2.031895in}}{\pgfqpoint{1.714001in}{2.039708in}}%
\pgfpathcurveto{\pgfqpoint{1.721814in}{2.047522in}}{\pgfqpoint{1.726205in}{2.058121in}}{\pgfqpoint{1.726205in}{2.069171in}}%
\pgfpathcurveto{\pgfqpoint{1.726205in}{2.080221in}}{\pgfqpoint{1.721814in}{2.090820in}}{\pgfqpoint{1.714001in}{2.098634in}}%
\pgfpathcurveto{\pgfqpoint{1.706187in}{2.106447in}}{\pgfqpoint{1.695588in}{2.110838in}}{\pgfqpoint{1.684538in}{2.110838in}}%
\pgfpathcurveto{\pgfqpoint{1.673488in}{2.110838in}}{\pgfqpoint{1.662889in}{2.106447in}}{\pgfqpoint{1.655075in}{2.098634in}}%
\pgfpathcurveto{\pgfqpoint{1.647262in}{2.090820in}}{\pgfqpoint{1.642871in}{2.080221in}}{\pgfqpoint{1.642871in}{2.069171in}}%
\pgfpathcurveto{\pgfqpoint{1.642871in}{2.058121in}}{\pgfqpoint{1.647262in}{2.047522in}}{\pgfqpoint{1.655075in}{2.039708in}}%
\pgfpathcurveto{\pgfqpoint{1.662889in}{2.031895in}}{\pgfqpoint{1.673488in}{2.027504in}}{\pgfqpoint{1.684538in}{2.027504in}}%
\pgfpathlineto{\pgfqpoint{1.684538in}{2.027504in}}%
\pgfpathclose%
\pgfusepath{stroke}%
\end{pgfscope}%
\begin{pgfscope}%
\pgfpathrectangle{\pgfqpoint{0.847223in}{0.554012in}}{\pgfqpoint{6.200000in}{4.620000in}}%
\pgfusepath{clip}%
\pgfsetbuttcap%
\pgfsetroundjoin%
\pgfsetlinewidth{1.003750pt}%
\definecolor{currentstroke}{rgb}{1.000000,0.000000,0.000000}%
\pgfsetstrokecolor{currentstroke}%
\pgfsetdash{}{0pt}%
\pgfpathmoveto{\pgfqpoint{1.689871in}{2.019928in}}%
\pgfpathcurveto{\pgfqpoint{1.700921in}{2.019928in}}{\pgfqpoint{1.711520in}{2.024318in}}{\pgfqpoint{1.719334in}{2.032131in}}%
\pgfpathcurveto{\pgfqpoint{1.727148in}{2.039945in}}{\pgfqpoint{1.731538in}{2.050544in}}{\pgfqpoint{1.731538in}{2.061594in}}%
\pgfpathcurveto{\pgfqpoint{1.731538in}{2.072644in}}{\pgfqpoint{1.727148in}{2.083243in}}{\pgfqpoint{1.719334in}{2.091057in}}%
\pgfpathcurveto{\pgfqpoint{1.711520in}{2.098871in}}{\pgfqpoint{1.700921in}{2.103261in}}{\pgfqpoint{1.689871in}{2.103261in}}%
\pgfpathcurveto{\pgfqpoint{1.678821in}{2.103261in}}{\pgfqpoint{1.668222in}{2.098871in}}{\pgfqpoint{1.660408in}{2.091057in}}%
\pgfpathcurveto{\pgfqpoint{1.652595in}{2.083243in}}{\pgfqpoint{1.648204in}{2.072644in}}{\pgfqpoint{1.648204in}{2.061594in}}%
\pgfpathcurveto{\pgfqpoint{1.648204in}{2.050544in}}{\pgfqpoint{1.652595in}{2.039945in}}{\pgfqpoint{1.660408in}{2.032131in}}%
\pgfpathcurveto{\pgfqpoint{1.668222in}{2.024318in}}{\pgfqpoint{1.678821in}{2.019928in}}{\pgfqpoint{1.689871in}{2.019928in}}%
\pgfpathlineto{\pgfqpoint{1.689871in}{2.019928in}}%
\pgfpathclose%
\pgfusepath{stroke}%
\end{pgfscope}%
\begin{pgfscope}%
\pgfpathrectangle{\pgfqpoint{0.847223in}{0.554012in}}{\pgfqpoint{6.200000in}{4.620000in}}%
\pgfusepath{clip}%
\pgfsetbuttcap%
\pgfsetroundjoin%
\pgfsetlinewidth{1.003750pt}%
\definecolor{currentstroke}{rgb}{1.000000,0.000000,0.000000}%
\pgfsetstrokecolor{currentstroke}%
\pgfsetdash{}{0pt}%
\pgfpathmoveto{\pgfqpoint{1.695204in}{2.012410in}}%
\pgfpathcurveto{\pgfqpoint{1.706254in}{2.012410in}}{\pgfqpoint{1.716854in}{2.016800in}}{\pgfqpoint{1.724667in}{2.024614in}}%
\pgfpathcurveto{\pgfqpoint{1.732481in}{2.032427in}}{\pgfqpoint{1.736871in}{2.043026in}}{\pgfqpoint{1.736871in}{2.054076in}}%
\pgfpathcurveto{\pgfqpoint{1.736871in}{2.065127in}}{\pgfqpoint{1.732481in}{2.075726in}}{\pgfqpoint{1.724667in}{2.083539in}}%
\pgfpathcurveto{\pgfqpoint{1.716854in}{2.091353in}}{\pgfqpoint{1.706254in}{2.095743in}}{\pgfqpoint{1.695204in}{2.095743in}}%
\pgfpathcurveto{\pgfqpoint{1.684154in}{2.095743in}}{\pgfqpoint{1.673555in}{2.091353in}}{\pgfqpoint{1.665742in}{2.083539in}}%
\pgfpathcurveto{\pgfqpoint{1.657928in}{2.075726in}}{\pgfqpoint{1.653538in}{2.065127in}}{\pgfqpoint{1.653538in}{2.054076in}}%
\pgfpathcurveto{\pgfqpoint{1.653538in}{2.043026in}}{\pgfqpoint{1.657928in}{2.032427in}}{\pgfqpoint{1.665742in}{2.024614in}}%
\pgfpathcurveto{\pgfqpoint{1.673555in}{2.016800in}}{\pgfqpoint{1.684154in}{2.012410in}}{\pgfqpoint{1.695204in}{2.012410in}}%
\pgfpathlineto{\pgfqpoint{1.695204in}{2.012410in}}%
\pgfpathclose%
\pgfusepath{stroke}%
\end{pgfscope}%
\begin{pgfscope}%
\pgfpathrectangle{\pgfqpoint{0.847223in}{0.554012in}}{\pgfqpoint{6.200000in}{4.620000in}}%
\pgfusepath{clip}%
\pgfsetbuttcap%
\pgfsetroundjoin%
\pgfsetlinewidth{1.003750pt}%
\definecolor{currentstroke}{rgb}{1.000000,0.000000,0.000000}%
\pgfsetstrokecolor{currentstroke}%
\pgfsetdash{}{0pt}%
\pgfpathmoveto{\pgfqpoint{1.700538in}{2.004950in}}%
\pgfpathcurveto{\pgfqpoint{1.711588in}{2.004950in}}{\pgfqpoint{1.722187in}{2.009341in}}{\pgfqpoint{1.730000in}{2.017154in}}%
\pgfpathcurveto{\pgfqpoint{1.737814in}{2.024968in}}{\pgfqpoint{1.742204in}{2.035567in}}{\pgfqpoint{1.742204in}{2.046617in}}%
\pgfpathcurveto{\pgfqpoint{1.742204in}{2.057667in}}{\pgfqpoint{1.737814in}{2.068266in}}{\pgfqpoint{1.730000in}{2.076080in}}%
\pgfpathcurveto{\pgfqpoint{1.722187in}{2.083893in}}{\pgfqpoint{1.711588in}{2.088284in}}{\pgfqpoint{1.700538in}{2.088284in}}%
\pgfpathcurveto{\pgfqpoint{1.689487in}{2.088284in}}{\pgfqpoint{1.678888in}{2.083893in}}{\pgfqpoint{1.671075in}{2.076080in}}%
\pgfpathcurveto{\pgfqpoint{1.663261in}{2.068266in}}{\pgfqpoint{1.658871in}{2.057667in}}{\pgfqpoint{1.658871in}{2.046617in}}%
\pgfpathcurveto{\pgfqpoint{1.658871in}{2.035567in}}{\pgfqpoint{1.663261in}{2.024968in}}{\pgfqpoint{1.671075in}{2.017154in}}%
\pgfpathcurveto{\pgfqpoint{1.678888in}{2.009341in}}{\pgfqpoint{1.689487in}{2.004950in}}{\pgfqpoint{1.700538in}{2.004950in}}%
\pgfpathlineto{\pgfqpoint{1.700538in}{2.004950in}}%
\pgfpathclose%
\pgfusepath{stroke}%
\end{pgfscope}%
\begin{pgfscope}%
\pgfpathrectangle{\pgfqpoint{0.847223in}{0.554012in}}{\pgfqpoint{6.200000in}{4.620000in}}%
\pgfusepath{clip}%
\pgfsetbuttcap%
\pgfsetroundjoin%
\pgfsetlinewidth{1.003750pt}%
\definecolor{currentstroke}{rgb}{1.000000,0.000000,0.000000}%
\pgfsetstrokecolor{currentstroke}%
\pgfsetdash{}{0pt}%
\pgfpathmoveto{\pgfqpoint{1.705871in}{1.997548in}}%
\pgfpathcurveto{\pgfqpoint{1.716921in}{1.997548in}}{\pgfqpoint{1.727520in}{2.001939in}}{\pgfqpoint{1.735334in}{2.009752in}}%
\pgfpathcurveto{\pgfqpoint{1.743147in}{2.017566in}}{\pgfqpoint{1.747537in}{2.028165in}}{\pgfqpoint{1.747537in}{2.039215in}}%
\pgfpathcurveto{\pgfqpoint{1.747537in}{2.050265in}}{\pgfqpoint{1.743147in}{2.060864in}}{\pgfqpoint{1.735334in}{2.068678in}}%
\pgfpathcurveto{\pgfqpoint{1.727520in}{2.076491in}}{\pgfqpoint{1.716921in}{2.080882in}}{\pgfqpoint{1.705871in}{2.080882in}}%
\pgfpathcurveto{\pgfqpoint{1.694821in}{2.080882in}}{\pgfqpoint{1.684222in}{2.076491in}}{\pgfqpoint{1.676408in}{2.068678in}}%
\pgfpathcurveto{\pgfqpoint{1.668594in}{2.060864in}}{\pgfqpoint{1.664204in}{2.050265in}}{\pgfqpoint{1.664204in}{2.039215in}}%
\pgfpathcurveto{\pgfqpoint{1.664204in}{2.028165in}}{\pgfqpoint{1.668594in}{2.017566in}}{\pgfqpoint{1.676408in}{2.009752in}}%
\pgfpathcurveto{\pgfqpoint{1.684222in}{2.001939in}}{\pgfqpoint{1.694821in}{1.997548in}}{\pgfqpoint{1.705871in}{1.997548in}}%
\pgfpathlineto{\pgfqpoint{1.705871in}{1.997548in}}%
\pgfpathclose%
\pgfusepath{stroke}%
\end{pgfscope}%
\begin{pgfscope}%
\pgfpathrectangle{\pgfqpoint{0.847223in}{0.554012in}}{\pgfqpoint{6.200000in}{4.620000in}}%
\pgfusepath{clip}%
\pgfsetbuttcap%
\pgfsetroundjoin%
\pgfsetlinewidth{1.003750pt}%
\definecolor{currentstroke}{rgb}{1.000000,0.000000,0.000000}%
\pgfsetstrokecolor{currentstroke}%
\pgfsetdash{}{0pt}%
\pgfpathmoveto{\pgfqpoint{1.711204in}{1.990203in}}%
\pgfpathcurveto{\pgfqpoint{1.722254in}{1.990203in}}{\pgfqpoint{1.732853in}{1.994593in}}{\pgfqpoint{1.740667in}{2.002407in}}%
\pgfpathcurveto{\pgfqpoint{1.748480in}{2.010221in}}{\pgfqpoint{1.752871in}{2.020820in}}{\pgfqpoint{1.752871in}{2.031870in}}%
\pgfpathcurveto{\pgfqpoint{1.752871in}{2.042920in}}{\pgfqpoint{1.748480in}{2.053519in}}{\pgfqpoint{1.740667in}{2.061333in}}%
\pgfpathcurveto{\pgfqpoint{1.732853in}{2.069146in}}{\pgfqpoint{1.722254in}{2.073537in}}{\pgfqpoint{1.711204in}{2.073537in}}%
\pgfpathcurveto{\pgfqpoint{1.700154in}{2.073537in}}{\pgfqpoint{1.689555in}{2.069146in}}{\pgfqpoint{1.681741in}{2.061333in}}%
\pgfpathcurveto{\pgfqpoint{1.673928in}{2.053519in}}{\pgfqpoint{1.669537in}{2.042920in}}{\pgfqpoint{1.669537in}{2.031870in}}%
\pgfpathcurveto{\pgfqpoint{1.669537in}{2.020820in}}{\pgfqpoint{1.673928in}{2.010221in}}{\pgfqpoint{1.681741in}{2.002407in}}%
\pgfpathcurveto{\pgfqpoint{1.689555in}{1.994593in}}{\pgfqpoint{1.700154in}{1.990203in}}{\pgfqpoint{1.711204in}{1.990203in}}%
\pgfpathlineto{\pgfqpoint{1.711204in}{1.990203in}}%
\pgfpathclose%
\pgfusepath{stroke}%
\end{pgfscope}%
\begin{pgfscope}%
\pgfpathrectangle{\pgfqpoint{0.847223in}{0.554012in}}{\pgfqpoint{6.200000in}{4.620000in}}%
\pgfusepath{clip}%
\pgfsetbuttcap%
\pgfsetroundjoin%
\pgfsetlinewidth{1.003750pt}%
\definecolor{currentstroke}{rgb}{1.000000,0.000000,0.000000}%
\pgfsetstrokecolor{currentstroke}%
\pgfsetdash{}{0pt}%
\pgfpathmoveto{\pgfqpoint{1.716537in}{1.982914in}}%
\pgfpathcurveto{\pgfqpoint{1.727587in}{1.982914in}}{\pgfqpoint{1.738186in}{1.987305in}}{\pgfqpoint{1.746000in}{1.995118in}}%
\pgfpathcurveto{\pgfqpoint{1.753814in}{2.002932in}}{\pgfqpoint{1.758204in}{2.013531in}}{\pgfqpoint{1.758204in}{2.024581in}}%
\pgfpathcurveto{\pgfqpoint{1.758204in}{2.035631in}}{\pgfqpoint{1.753814in}{2.046230in}}{\pgfqpoint{1.746000in}{2.054044in}}%
\pgfpathcurveto{\pgfqpoint{1.738186in}{2.061857in}}{\pgfqpoint{1.727587in}{2.066248in}}{\pgfqpoint{1.716537in}{2.066248in}}%
\pgfpathcurveto{\pgfqpoint{1.705487in}{2.066248in}}{\pgfqpoint{1.694888in}{2.061857in}}{\pgfqpoint{1.687074in}{2.054044in}}%
\pgfpathcurveto{\pgfqpoint{1.679261in}{2.046230in}}{\pgfqpoint{1.674871in}{2.035631in}}{\pgfqpoint{1.674871in}{2.024581in}}%
\pgfpathcurveto{\pgfqpoint{1.674871in}{2.013531in}}{\pgfqpoint{1.679261in}{2.002932in}}{\pgfqpoint{1.687074in}{1.995118in}}%
\pgfpathcurveto{\pgfqpoint{1.694888in}{1.987305in}}{\pgfqpoint{1.705487in}{1.982914in}}{\pgfqpoint{1.716537in}{1.982914in}}%
\pgfpathlineto{\pgfqpoint{1.716537in}{1.982914in}}%
\pgfpathclose%
\pgfusepath{stroke}%
\end{pgfscope}%
\begin{pgfscope}%
\pgfpathrectangle{\pgfqpoint{0.847223in}{0.554012in}}{\pgfqpoint{6.200000in}{4.620000in}}%
\pgfusepath{clip}%
\pgfsetbuttcap%
\pgfsetroundjoin%
\pgfsetlinewidth{1.003750pt}%
\definecolor{currentstroke}{rgb}{1.000000,0.000000,0.000000}%
\pgfsetstrokecolor{currentstroke}%
\pgfsetdash{}{0pt}%
\pgfpathmoveto{\pgfqpoint{1.721870in}{1.975681in}}%
\pgfpathcurveto{\pgfqpoint{1.732921in}{1.975681in}}{\pgfqpoint{1.743520in}{1.980071in}}{\pgfqpoint{1.751333in}{1.987885in}}%
\pgfpathcurveto{\pgfqpoint{1.759147in}{1.995699in}}{\pgfqpoint{1.763537in}{2.006298in}}{\pgfqpoint{1.763537in}{2.017348in}}%
\pgfpathcurveto{\pgfqpoint{1.763537in}{2.028398in}}{\pgfqpoint{1.759147in}{2.038997in}}{\pgfqpoint{1.751333in}{2.046811in}}%
\pgfpathcurveto{\pgfqpoint{1.743520in}{2.054624in}}{\pgfqpoint{1.732921in}{2.059015in}}{\pgfqpoint{1.721870in}{2.059015in}}%
\pgfpathcurveto{\pgfqpoint{1.710820in}{2.059015in}}{\pgfqpoint{1.700221in}{2.054624in}}{\pgfqpoint{1.692408in}{2.046811in}}%
\pgfpathcurveto{\pgfqpoint{1.684594in}{2.038997in}}{\pgfqpoint{1.680204in}{2.028398in}}{\pgfqpoint{1.680204in}{2.017348in}}%
\pgfpathcurveto{\pgfqpoint{1.680204in}{2.006298in}}{\pgfqpoint{1.684594in}{1.995699in}}{\pgfqpoint{1.692408in}{1.987885in}}%
\pgfpathcurveto{\pgfqpoint{1.700221in}{1.980071in}}{\pgfqpoint{1.710820in}{1.975681in}}{\pgfqpoint{1.721870in}{1.975681in}}%
\pgfpathlineto{\pgfqpoint{1.721870in}{1.975681in}}%
\pgfpathclose%
\pgfusepath{stroke}%
\end{pgfscope}%
\begin{pgfscope}%
\pgfpathrectangle{\pgfqpoint{0.847223in}{0.554012in}}{\pgfqpoint{6.200000in}{4.620000in}}%
\pgfusepath{clip}%
\pgfsetbuttcap%
\pgfsetroundjoin%
\pgfsetlinewidth{1.003750pt}%
\definecolor{currentstroke}{rgb}{1.000000,0.000000,0.000000}%
\pgfsetstrokecolor{currentstroke}%
\pgfsetdash{}{0pt}%
\pgfpathmoveto{\pgfqpoint{1.727204in}{1.968503in}}%
\pgfpathcurveto{\pgfqpoint{1.738254in}{1.968503in}}{\pgfqpoint{1.748853in}{1.972893in}}{\pgfqpoint{1.756666in}{1.980707in}}%
\pgfpathcurveto{\pgfqpoint{1.764480in}{1.988520in}}{\pgfqpoint{1.768870in}{1.999120in}}{\pgfqpoint{1.768870in}{2.010170in}}%
\pgfpathcurveto{\pgfqpoint{1.768870in}{2.021220in}}{\pgfqpoint{1.764480in}{2.031819in}}{\pgfqpoint{1.756666in}{2.039632in}}%
\pgfpathcurveto{\pgfqpoint{1.748853in}{2.047446in}}{\pgfqpoint{1.738254in}{2.051836in}}{\pgfqpoint{1.727204in}{2.051836in}}%
\pgfpathcurveto{\pgfqpoint{1.716154in}{2.051836in}}{\pgfqpoint{1.705554in}{2.047446in}}{\pgfqpoint{1.697741in}{2.039632in}}%
\pgfpathcurveto{\pgfqpoint{1.689927in}{2.031819in}}{\pgfqpoint{1.685537in}{2.021220in}}{\pgfqpoint{1.685537in}{2.010170in}}%
\pgfpathcurveto{\pgfqpoint{1.685537in}{1.999120in}}{\pgfqpoint{1.689927in}{1.988520in}}{\pgfqpoint{1.697741in}{1.980707in}}%
\pgfpathcurveto{\pgfqpoint{1.705554in}{1.972893in}}{\pgfqpoint{1.716154in}{1.968503in}}{\pgfqpoint{1.727204in}{1.968503in}}%
\pgfpathlineto{\pgfqpoint{1.727204in}{1.968503in}}%
\pgfpathclose%
\pgfusepath{stroke}%
\end{pgfscope}%
\begin{pgfscope}%
\pgfpathrectangle{\pgfqpoint{0.847223in}{0.554012in}}{\pgfqpoint{6.200000in}{4.620000in}}%
\pgfusepath{clip}%
\pgfsetbuttcap%
\pgfsetroundjoin%
\pgfsetlinewidth{1.003750pt}%
\definecolor{currentstroke}{rgb}{1.000000,0.000000,0.000000}%
\pgfsetstrokecolor{currentstroke}%
\pgfsetdash{}{0pt}%
\pgfpathmoveto{\pgfqpoint{1.732537in}{1.961379in}}%
\pgfpathcurveto{\pgfqpoint{1.743587in}{1.961379in}}{\pgfqpoint{1.754186in}{1.965769in}}{\pgfqpoint{1.762000in}{1.973583in}}%
\pgfpathcurveto{\pgfqpoint{1.769813in}{1.981397in}}{\pgfqpoint{1.774204in}{1.991996in}}{\pgfqpoint{1.774204in}{2.003046in}}%
\pgfpathcurveto{\pgfqpoint{1.774204in}{2.014096in}}{\pgfqpoint{1.769813in}{2.024695in}}{\pgfqpoint{1.762000in}{2.032509in}}%
\pgfpathcurveto{\pgfqpoint{1.754186in}{2.040322in}}{\pgfqpoint{1.743587in}{2.044712in}}{\pgfqpoint{1.732537in}{2.044712in}}%
\pgfpathcurveto{\pgfqpoint{1.721487in}{2.044712in}}{\pgfqpoint{1.710888in}{2.040322in}}{\pgfqpoint{1.703074in}{2.032509in}}%
\pgfpathcurveto{\pgfqpoint{1.695260in}{2.024695in}}{\pgfqpoint{1.690870in}{2.014096in}}{\pgfqpoint{1.690870in}{2.003046in}}%
\pgfpathcurveto{\pgfqpoint{1.690870in}{1.991996in}}{\pgfqpoint{1.695260in}{1.981397in}}{\pgfqpoint{1.703074in}{1.973583in}}%
\pgfpathcurveto{\pgfqpoint{1.710888in}{1.965769in}}{\pgfqpoint{1.721487in}{1.961379in}}{\pgfqpoint{1.732537in}{1.961379in}}%
\pgfpathlineto{\pgfqpoint{1.732537in}{1.961379in}}%
\pgfpathclose%
\pgfusepath{stroke}%
\end{pgfscope}%
\begin{pgfscope}%
\pgfpathrectangle{\pgfqpoint{0.847223in}{0.554012in}}{\pgfqpoint{6.200000in}{4.620000in}}%
\pgfusepath{clip}%
\pgfsetbuttcap%
\pgfsetroundjoin%
\pgfsetlinewidth{1.003750pt}%
\definecolor{currentstroke}{rgb}{1.000000,0.000000,0.000000}%
\pgfsetstrokecolor{currentstroke}%
\pgfsetdash{}{0pt}%
\pgfpathmoveto{\pgfqpoint{1.737870in}{1.954309in}}%
\pgfpathcurveto{\pgfqpoint{1.748920in}{1.954309in}}{\pgfqpoint{1.759519in}{1.958699in}}{\pgfqpoint{1.767333in}{1.966513in}}%
\pgfpathcurveto{\pgfqpoint{1.775146in}{1.974326in}}{\pgfqpoint{1.779537in}{1.984926in}}{\pgfqpoint{1.779537in}{1.995976in}}%
\pgfpathcurveto{\pgfqpoint{1.779537in}{2.007026in}}{\pgfqpoint{1.775146in}{2.017625in}}{\pgfqpoint{1.767333in}{2.025438in}}%
\pgfpathcurveto{\pgfqpoint{1.759519in}{2.033252in}}{\pgfqpoint{1.748920in}{2.037642in}}{\pgfqpoint{1.737870in}{2.037642in}}%
\pgfpathcurveto{\pgfqpoint{1.726820in}{2.037642in}}{\pgfqpoint{1.716221in}{2.033252in}}{\pgfqpoint{1.708407in}{2.025438in}}%
\pgfpathcurveto{\pgfqpoint{1.700594in}{2.017625in}}{\pgfqpoint{1.696203in}{2.007026in}}{\pgfqpoint{1.696203in}{1.995976in}}%
\pgfpathcurveto{\pgfqpoint{1.696203in}{1.984926in}}{\pgfqpoint{1.700594in}{1.974326in}}{\pgfqpoint{1.708407in}{1.966513in}}%
\pgfpathcurveto{\pgfqpoint{1.716221in}{1.958699in}}{\pgfqpoint{1.726820in}{1.954309in}}{\pgfqpoint{1.737870in}{1.954309in}}%
\pgfpathlineto{\pgfqpoint{1.737870in}{1.954309in}}%
\pgfpathclose%
\pgfusepath{stroke}%
\end{pgfscope}%
\begin{pgfscope}%
\pgfpathrectangle{\pgfqpoint{0.847223in}{0.554012in}}{\pgfqpoint{6.200000in}{4.620000in}}%
\pgfusepath{clip}%
\pgfsetbuttcap%
\pgfsetroundjoin%
\pgfsetlinewidth{1.003750pt}%
\definecolor{currentstroke}{rgb}{1.000000,0.000000,0.000000}%
\pgfsetstrokecolor{currentstroke}%
\pgfsetdash{}{0pt}%
\pgfpathmoveto{\pgfqpoint{1.743203in}{1.947292in}}%
\pgfpathcurveto{\pgfqpoint{1.754253in}{1.947292in}}{\pgfqpoint{1.764852in}{1.951682in}}{\pgfqpoint{1.772666in}{1.959496in}}%
\pgfpathcurveto{\pgfqpoint{1.780480in}{1.967309in}}{\pgfqpoint{1.784870in}{1.977909in}}{\pgfqpoint{1.784870in}{1.988959in}}%
\pgfpathcurveto{\pgfqpoint{1.784870in}{2.000009in}}{\pgfqpoint{1.780480in}{2.010608in}}{\pgfqpoint{1.772666in}{2.018421in}}%
\pgfpathcurveto{\pgfqpoint{1.764852in}{2.026235in}}{\pgfqpoint{1.754253in}{2.030625in}}{\pgfqpoint{1.743203in}{2.030625in}}%
\pgfpathcurveto{\pgfqpoint{1.732153in}{2.030625in}}{\pgfqpoint{1.721554in}{2.026235in}}{\pgfqpoint{1.713741in}{2.018421in}}%
\pgfpathcurveto{\pgfqpoint{1.705927in}{2.010608in}}{\pgfqpoint{1.701537in}{2.000009in}}{\pgfqpoint{1.701537in}{1.988959in}}%
\pgfpathcurveto{\pgfqpoint{1.701537in}{1.977909in}}{\pgfqpoint{1.705927in}{1.967309in}}{\pgfqpoint{1.713741in}{1.959496in}}%
\pgfpathcurveto{\pgfqpoint{1.721554in}{1.951682in}}{\pgfqpoint{1.732153in}{1.947292in}}{\pgfqpoint{1.743203in}{1.947292in}}%
\pgfpathlineto{\pgfqpoint{1.743203in}{1.947292in}}%
\pgfpathclose%
\pgfusepath{stroke}%
\end{pgfscope}%
\begin{pgfscope}%
\pgfpathrectangle{\pgfqpoint{0.847223in}{0.554012in}}{\pgfqpoint{6.200000in}{4.620000in}}%
\pgfusepath{clip}%
\pgfsetbuttcap%
\pgfsetroundjoin%
\pgfsetlinewidth{1.003750pt}%
\definecolor{currentstroke}{rgb}{1.000000,0.000000,0.000000}%
\pgfsetstrokecolor{currentstroke}%
\pgfsetdash{}{0pt}%
\pgfpathmoveto{\pgfqpoint{1.748537in}{1.940328in}}%
\pgfpathcurveto{\pgfqpoint{1.759587in}{1.940328in}}{\pgfqpoint{1.770186in}{1.944718in}}{\pgfqpoint{1.777999in}{1.952531in}}%
\pgfpathcurveto{\pgfqpoint{1.785813in}{1.960345in}}{\pgfqpoint{1.790203in}{1.970944in}}{\pgfqpoint{1.790203in}{1.981994in}}%
\pgfpathcurveto{\pgfqpoint{1.790203in}{1.993044in}}{\pgfqpoint{1.785813in}{2.003643in}}{\pgfqpoint{1.777999in}{2.011457in}}%
\pgfpathcurveto{\pgfqpoint{1.770186in}{2.019271in}}{\pgfqpoint{1.759587in}{2.023661in}}{\pgfqpoint{1.748537in}{2.023661in}}%
\pgfpathcurveto{\pgfqpoint{1.737486in}{2.023661in}}{\pgfqpoint{1.726887in}{2.019271in}}{\pgfqpoint{1.719074in}{2.011457in}}%
\pgfpathcurveto{\pgfqpoint{1.711260in}{2.003643in}}{\pgfqpoint{1.706870in}{1.993044in}}{\pgfqpoint{1.706870in}{1.981994in}}%
\pgfpathcurveto{\pgfqpoint{1.706870in}{1.970944in}}{\pgfqpoint{1.711260in}{1.960345in}}{\pgfqpoint{1.719074in}{1.952531in}}%
\pgfpathcurveto{\pgfqpoint{1.726887in}{1.944718in}}{\pgfqpoint{1.737486in}{1.940328in}}{\pgfqpoint{1.748537in}{1.940328in}}%
\pgfpathlineto{\pgfqpoint{1.748537in}{1.940328in}}%
\pgfpathclose%
\pgfusepath{stroke}%
\end{pgfscope}%
\begin{pgfscope}%
\pgfpathrectangle{\pgfqpoint{0.847223in}{0.554012in}}{\pgfqpoint{6.200000in}{4.620000in}}%
\pgfusepath{clip}%
\pgfsetbuttcap%
\pgfsetroundjoin%
\pgfsetlinewidth{1.003750pt}%
\definecolor{currentstroke}{rgb}{1.000000,0.000000,0.000000}%
\pgfsetstrokecolor{currentstroke}%
\pgfsetdash{}{0pt}%
\pgfpathmoveto{\pgfqpoint{1.753870in}{1.933415in}}%
\pgfpathcurveto{\pgfqpoint{1.764920in}{1.933415in}}{\pgfqpoint{1.775519in}{1.937805in}}{\pgfqpoint{1.783333in}{1.945619in}}%
\pgfpathcurveto{\pgfqpoint{1.791146in}{1.953433in}}{\pgfqpoint{1.795536in}{1.964032in}}{\pgfqpoint{1.795536in}{1.975082in}}%
\pgfpathcurveto{\pgfqpoint{1.795536in}{1.986132in}}{\pgfqpoint{1.791146in}{1.996731in}}{\pgfqpoint{1.783333in}{2.004544in}}%
\pgfpathcurveto{\pgfqpoint{1.775519in}{2.012358in}}{\pgfqpoint{1.764920in}{2.016748in}}{\pgfqpoint{1.753870in}{2.016748in}}%
\pgfpathcurveto{\pgfqpoint{1.742820in}{2.016748in}}{\pgfqpoint{1.732221in}{2.012358in}}{\pgfqpoint{1.724407in}{2.004544in}}%
\pgfpathcurveto{\pgfqpoint{1.716593in}{1.996731in}}{\pgfqpoint{1.712203in}{1.986132in}}{\pgfqpoint{1.712203in}{1.975082in}}%
\pgfpathcurveto{\pgfqpoint{1.712203in}{1.964032in}}{\pgfqpoint{1.716593in}{1.953433in}}{\pgfqpoint{1.724407in}{1.945619in}}%
\pgfpathcurveto{\pgfqpoint{1.732221in}{1.937805in}}{\pgfqpoint{1.742820in}{1.933415in}}{\pgfqpoint{1.753870in}{1.933415in}}%
\pgfpathlineto{\pgfqpoint{1.753870in}{1.933415in}}%
\pgfpathclose%
\pgfusepath{stroke}%
\end{pgfscope}%
\begin{pgfscope}%
\pgfpathrectangle{\pgfqpoint{0.847223in}{0.554012in}}{\pgfqpoint{6.200000in}{4.620000in}}%
\pgfusepath{clip}%
\pgfsetbuttcap%
\pgfsetroundjoin%
\pgfsetlinewidth{1.003750pt}%
\definecolor{currentstroke}{rgb}{1.000000,0.000000,0.000000}%
\pgfsetstrokecolor{currentstroke}%
\pgfsetdash{}{0pt}%
\pgfpathmoveto{\pgfqpoint{1.759203in}{1.926554in}}%
\pgfpathcurveto{\pgfqpoint{1.770253in}{1.926554in}}{\pgfqpoint{1.780852in}{1.930944in}}{\pgfqpoint{1.788666in}{1.938758in}}%
\pgfpathcurveto{\pgfqpoint{1.796479in}{1.946571in}}{\pgfqpoint{1.800870in}{1.957170in}}{\pgfqpoint{1.800870in}{1.968221in}}%
\pgfpathcurveto{\pgfqpoint{1.800870in}{1.979271in}}{\pgfqpoint{1.796479in}{1.989870in}}{\pgfqpoint{1.788666in}{1.997683in}}%
\pgfpathcurveto{\pgfqpoint{1.780852in}{2.005497in}}{\pgfqpoint{1.770253in}{2.009887in}}{\pgfqpoint{1.759203in}{2.009887in}}%
\pgfpathcurveto{\pgfqpoint{1.748153in}{2.009887in}}{\pgfqpoint{1.737554in}{2.005497in}}{\pgfqpoint{1.729740in}{1.997683in}}%
\pgfpathcurveto{\pgfqpoint{1.721927in}{1.989870in}}{\pgfqpoint{1.717536in}{1.979271in}}{\pgfqpoint{1.717536in}{1.968221in}}%
\pgfpathcurveto{\pgfqpoint{1.717536in}{1.957170in}}{\pgfqpoint{1.721927in}{1.946571in}}{\pgfqpoint{1.729740in}{1.938758in}}%
\pgfpathcurveto{\pgfqpoint{1.737554in}{1.930944in}}{\pgfqpoint{1.748153in}{1.926554in}}{\pgfqpoint{1.759203in}{1.926554in}}%
\pgfpathlineto{\pgfqpoint{1.759203in}{1.926554in}}%
\pgfpathclose%
\pgfusepath{stroke}%
\end{pgfscope}%
\begin{pgfscope}%
\pgfpathrectangle{\pgfqpoint{0.847223in}{0.554012in}}{\pgfqpoint{6.200000in}{4.620000in}}%
\pgfusepath{clip}%
\pgfsetbuttcap%
\pgfsetroundjoin%
\pgfsetlinewidth{1.003750pt}%
\definecolor{currentstroke}{rgb}{1.000000,0.000000,0.000000}%
\pgfsetstrokecolor{currentstroke}%
\pgfsetdash{}{0pt}%
\pgfpathmoveto{\pgfqpoint{1.764536in}{1.919744in}}%
\pgfpathcurveto{\pgfqpoint{1.775586in}{1.919744in}}{\pgfqpoint{1.786185in}{1.924134in}}{\pgfqpoint{1.793999in}{1.931947in}}%
\pgfpathcurveto{\pgfqpoint{1.801813in}{1.939761in}}{\pgfqpoint{1.806203in}{1.950360in}}{\pgfqpoint{1.806203in}{1.961410in}}%
\pgfpathcurveto{\pgfqpoint{1.806203in}{1.972460in}}{\pgfqpoint{1.801813in}{1.983059in}}{\pgfqpoint{1.793999in}{1.990873in}}%
\pgfpathcurveto{\pgfqpoint{1.786185in}{1.998687in}}{\pgfqpoint{1.775586in}{2.003077in}}{\pgfqpoint{1.764536in}{2.003077in}}%
\pgfpathcurveto{\pgfqpoint{1.753486in}{2.003077in}}{\pgfqpoint{1.742887in}{1.998687in}}{\pgfqpoint{1.735073in}{1.990873in}}%
\pgfpathcurveto{\pgfqpoint{1.727260in}{1.983059in}}{\pgfqpoint{1.722869in}{1.972460in}}{\pgfqpoint{1.722869in}{1.961410in}}%
\pgfpathcurveto{\pgfqpoint{1.722869in}{1.950360in}}{\pgfqpoint{1.727260in}{1.939761in}}{\pgfqpoint{1.735073in}{1.931947in}}%
\pgfpathcurveto{\pgfqpoint{1.742887in}{1.924134in}}{\pgfqpoint{1.753486in}{1.919744in}}{\pgfqpoint{1.764536in}{1.919744in}}%
\pgfpathlineto{\pgfqpoint{1.764536in}{1.919744in}}%
\pgfpathclose%
\pgfusepath{stroke}%
\end{pgfscope}%
\begin{pgfscope}%
\pgfpathrectangle{\pgfqpoint{0.847223in}{0.554012in}}{\pgfqpoint{6.200000in}{4.620000in}}%
\pgfusepath{clip}%
\pgfsetbuttcap%
\pgfsetroundjoin%
\pgfsetlinewidth{1.003750pt}%
\definecolor{currentstroke}{rgb}{1.000000,0.000000,0.000000}%
\pgfsetstrokecolor{currentstroke}%
\pgfsetdash{}{0pt}%
\pgfpathmoveto{\pgfqpoint{1.769869in}{1.912983in}}%
\pgfpathcurveto{\pgfqpoint{1.780919in}{1.912983in}}{\pgfqpoint{1.791519in}{1.917374in}}{\pgfqpoint{1.799332in}{1.925187in}}%
\pgfpathcurveto{\pgfqpoint{1.807146in}{1.933001in}}{\pgfqpoint{1.811536in}{1.943600in}}{\pgfqpoint{1.811536in}{1.954650in}}%
\pgfpathcurveto{\pgfqpoint{1.811536in}{1.965700in}}{\pgfqpoint{1.807146in}{1.976299in}}{\pgfqpoint{1.799332in}{1.984113in}}%
\pgfpathcurveto{\pgfqpoint{1.791519in}{1.991926in}}{\pgfqpoint{1.780919in}{1.996317in}}{\pgfqpoint{1.769869in}{1.996317in}}%
\pgfpathcurveto{\pgfqpoint{1.758819in}{1.996317in}}{\pgfqpoint{1.748220in}{1.991926in}}{\pgfqpoint{1.740407in}{1.984113in}}%
\pgfpathcurveto{\pgfqpoint{1.732593in}{1.976299in}}{\pgfqpoint{1.728203in}{1.965700in}}{\pgfqpoint{1.728203in}{1.954650in}}%
\pgfpathcurveto{\pgfqpoint{1.728203in}{1.943600in}}{\pgfqpoint{1.732593in}{1.933001in}}{\pgfqpoint{1.740407in}{1.925187in}}%
\pgfpathcurveto{\pgfqpoint{1.748220in}{1.917374in}}{\pgfqpoint{1.758819in}{1.912983in}}{\pgfqpoint{1.769869in}{1.912983in}}%
\pgfpathlineto{\pgfqpoint{1.769869in}{1.912983in}}%
\pgfpathclose%
\pgfusepath{stroke}%
\end{pgfscope}%
\begin{pgfscope}%
\pgfpathrectangle{\pgfqpoint{0.847223in}{0.554012in}}{\pgfqpoint{6.200000in}{4.620000in}}%
\pgfusepath{clip}%
\pgfsetbuttcap%
\pgfsetroundjoin%
\pgfsetlinewidth{1.003750pt}%
\definecolor{currentstroke}{rgb}{1.000000,0.000000,0.000000}%
\pgfsetstrokecolor{currentstroke}%
\pgfsetdash{}{0pt}%
\pgfpathmoveto{\pgfqpoint{1.775203in}{1.906273in}}%
\pgfpathcurveto{\pgfqpoint{1.786253in}{1.906273in}}{\pgfqpoint{1.796852in}{1.910663in}}{\pgfqpoint{1.804665in}{1.918477in}}%
\pgfpathcurveto{\pgfqpoint{1.812479in}{1.926290in}}{\pgfqpoint{1.816869in}{1.936890in}}{\pgfqpoint{1.816869in}{1.947940in}}%
\pgfpathcurveto{\pgfqpoint{1.816869in}{1.958990in}}{\pgfqpoint{1.812479in}{1.969589in}}{\pgfqpoint{1.804665in}{1.977402in}}%
\pgfpathcurveto{\pgfqpoint{1.796852in}{1.985216in}}{\pgfqpoint{1.786253in}{1.989606in}}{\pgfqpoint{1.775203in}{1.989606in}}%
\pgfpathcurveto{\pgfqpoint{1.764152in}{1.989606in}}{\pgfqpoint{1.753553in}{1.985216in}}{\pgfqpoint{1.745740in}{1.977402in}}%
\pgfpathcurveto{\pgfqpoint{1.737926in}{1.969589in}}{\pgfqpoint{1.733536in}{1.958990in}}{\pgfqpoint{1.733536in}{1.947940in}}%
\pgfpathcurveto{\pgfqpoint{1.733536in}{1.936890in}}{\pgfqpoint{1.737926in}{1.926290in}}{\pgfqpoint{1.745740in}{1.918477in}}%
\pgfpathcurveto{\pgfqpoint{1.753553in}{1.910663in}}{\pgfqpoint{1.764152in}{1.906273in}}{\pgfqpoint{1.775203in}{1.906273in}}%
\pgfpathlineto{\pgfqpoint{1.775203in}{1.906273in}}%
\pgfpathclose%
\pgfusepath{stroke}%
\end{pgfscope}%
\begin{pgfscope}%
\pgfpathrectangle{\pgfqpoint{0.847223in}{0.554012in}}{\pgfqpoint{6.200000in}{4.620000in}}%
\pgfusepath{clip}%
\pgfsetbuttcap%
\pgfsetroundjoin%
\pgfsetlinewidth{1.003750pt}%
\definecolor{currentstroke}{rgb}{1.000000,0.000000,0.000000}%
\pgfsetstrokecolor{currentstroke}%
\pgfsetdash{}{0pt}%
\pgfpathmoveto{\pgfqpoint{1.780536in}{1.899612in}}%
\pgfpathcurveto{\pgfqpoint{1.791586in}{1.899612in}}{\pgfqpoint{1.802185in}{1.904002in}}{\pgfqpoint{1.809999in}{1.911816in}}%
\pgfpathcurveto{\pgfqpoint{1.817812in}{1.919629in}}{\pgfqpoint{1.822202in}{1.930228in}}{\pgfqpoint{1.822202in}{1.941278in}}%
\pgfpathcurveto{\pgfqpoint{1.822202in}{1.952328in}}{\pgfqpoint{1.817812in}{1.962928in}}{\pgfqpoint{1.809999in}{1.970741in}}%
\pgfpathcurveto{\pgfqpoint{1.802185in}{1.978555in}}{\pgfqpoint{1.791586in}{1.982945in}}{\pgfqpoint{1.780536in}{1.982945in}}%
\pgfpathcurveto{\pgfqpoint{1.769486in}{1.982945in}}{\pgfqpoint{1.758887in}{1.978555in}}{\pgfqpoint{1.751073in}{1.970741in}}%
\pgfpathcurveto{\pgfqpoint{1.743259in}{1.962928in}}{\pgfqpoint{1.738869in}{1.952328in}}{\pgfqpoint{1.738869in}{1.941278in}}%
\pgfpathcurveto{\pgfqpoint{1.738869in}{1.930228in}}{\pgfqpoint{1.743259in}{1.919629in}}{\pgfqpoint{1.751073in}{1.911816in}}%
\pgfpathcurveto{\pgfqpoint{1.758887in}{1.904002in}}{\pgfqpoint{1.769486in}{1.899612in}}{\pgfqpoint{1.780536in}{1.899612in}}%
\pgfpathlineto{\pgfqpoint{1.780536in}{1.899612in}}%
\pgfpathclose%
\pgfusepath{stroke}%
\end{pgfscope}%
\begin{pgfscope}%
\pgfpathrectangle{\pgfqpoint{0.847223in}{0.554012in}}{\pgfqpoint{6.200000in}{4.620000in}}%
\pgfusepath{clip}%
\pgfsetbuttcap%
\pgfsetroundjoin%
\pgfsetlinewidth{1.003750pt}%
\definecolor{currentstroke}{rgb}{1.000000,0.000000,0.000000}%
\pgfsetstrokecolor{currentstroke}%
\pgfsetdash{}{0pt}%
\pgfpathmoveto{\pgfqpoint{1.785869in}{1.892999in}}%
\pgfpathcurveto{\pgfqpoint{1.796919in}{1.892999in}}{\pgfqpoint{1.807518in}{1.897389in}}{\pgfqpoint{1.815332in}{1.905203in}}%
\pgfpathcurveto{\pgfqpoint{1.823145in}{1.913017in}}{\pgfqpoint{1.827536in}{1.923616in}}{\pgfqpoint{1.827536in}{1.934666in}}%
\pgfpathcurveto{\pgfqpoint{1.827536in}{1.945716in}}{\pgfqpoint{1.823145in}{1.956315in}}{\pgfqpoint{1.815332in}{1.964128in}}%
\pgfpathcurveto{\pgfqpoint{1.807518in}{1.971942in}}{\pgfqpoint{1.796919in}{1.976332in}}{\pgfqpoint{1.785869in}{1.976332in}}%
\pgfpathcurveto{\pgfqpoint{1.774819in}{1.976332in}}{\pgfqpoint{1.764220in}{1.971942in}}{\pgfqpoint{1.756406in}{1.964128in}}%
\pgfpathcurveto{\pgfqpoint{1.748593in}{1.956315in}}{\pgfqpoint{1.744202in}{1.945716in}}{\pgfqpoint{1.744202in}{1.934666in}}%
\pgfpathcurveto{\pgfqpoint{1.744202in}{1.923616in}}{\pgfqpoint{1.748593in}{1.913017in}}{\pgfqpoint{1.756406in}{1.905203in}}%
\pgfpathcurveto{\pgfqpoint{1.764220in}{1.897389in}}{\pgfqpoint{1.774819in}{1.892999in}}{\pgfqpoint{1.785869in}{1.892999in}}%
\pgfpathlineto{\pgfqpoint{1.785869in}{1.892999in}}%
\pgfpathclose%
\pgfusepath{stroke}%
\end{pgfscope}%
\begin{pgfscope}%
\pgfpathrectangle{\pgfqpoint{0.847223in}{0.554012in}}{\pgfqpoint{6.200000in}{4.620000in}}%
\pgfusepath{clip}%
\pgfsetbuttcap%
\pgfsetroundjoin%
\pgfsetlinewidth{1.003750pt}%
\definecolor{currentstroke}{rgb}{1.000000,0.000000,0.000000}%
\pgfsetstrokecolor{currentstroke}%
\pgfsetdash{}{0pt}%
\pgfpathmoveto{\pgfqpoint{1.791202in}{1.886434in}}%
\pgfpathcurveto{\pgfqpoint{1.802252in}{1.886434in}}{\pgfqpoint{1.812851in}{1.890825in}}{\pgfqpoint{1.820665in}{1.898638in}}%
\pgfpathcurveto{\pgfqpoint{1.828479in}{1.906452in}}{\pgfqpoint{1.832869in}{1.917051in}}{\pgfqpoint{1.832869in}{1.928101in}}%
\pgfpathcurveto{\pgfqpoint{1.832869in}{1.939151in}}{\pgfqpoint{1.828479in}{1.949750in}}{\pgfqpoint{1.820665in}{1.957564in}}%
\pgfpathcurveto{\pgfqpoint{1.812851in}{1.965377in}}{\pgfqpoint{1.802252in}{1.969768in}}{\pgfqpoint{1.791202in}{1.969768in}}%
\pgfpathcurveto{\pgfqpoint{1.780152in}{1.969768in}}{\pgfqpoint{1.769553in}{1.965377in}}{\pgfqpoint{1.761739in}{1.957564in}}%
\pgfpathcurveto{\pgfqpoint{1.753926in}{1.949750in}}{\pgfqpoint{1.749536in}{1.939151in}}{\pgfqpoint{1.749536in}{1.928101in}}%
\pgfpathcurveto{\pgfqpoint{1.749536in}{1.917051in}}{\pgfqpoint{1.753926in}{1.906452in}}{\pgfqpoint{1.761739in}{1.898638in}}%
\pgfpathcurveto{\pgfqpoint{1.769553in}{1.890825in}}{\pgfqpoint{1.780152in}{1.886434in}}{\pgfqpoint{1.791202in}{1.886434in}}%
\pgfpathlineto{\pgfqpoint{1.791202in}{1.886434in}}%
\pgfpathclose%
\pgfusepath{stroke}%
\end{pgfscope}%
\begin{pgfscope}%
\pgfpathrectangle{\pgfqpoint{0.847223in}{0.554012in}}{\pgfqpoint{6.200000in}{4.620000in}}%
\pgfusepath{clip}%
\pgfsetbuttcap%
\pgfsetroundjoin%
\pgfsetlinewidth{1.003750pt}%
\definecolor{currentstroke}{rgb}{1.000000,0.000000,0.000000}%
\pgfsetstrokecolor{currentstroke}%
\pgfsetdash{}{0pt}%
\pgfpathmoveto{\pgfqpoint{1.796535in}{1.879917in}}%
\pgfpathcurveto{\pgfqpoint{1.807586in}{1.879917in}}{\pgfqpoint{1.818185in}{1.884308in}}{\pgfqpoint{1.825998in}{1.892121in}}%
\pgfpathcurveto{\pgfqpoint{1.833812in}{1.899935in}}{\pgfqpoint{1.838202in}{1.910534in}}{\pgfqpoint{1.838202in}{1.921584in}}%
\pgfpathcurveto{\pgfqpoint{1.838202in}{1.932634in}}{\pgfqpoint{1.833812in}{1.943233in}}{\pgfqpoint{1.825998in}{1.951047in}}%
\pgfpathcurveto{\pgfqpoint{1.818185in}{1.958860in}}{\pgfqpoint{1.807586in}{1.963251in}}{\pgfqpoint{1.796535in}{1.963251in}}%
\pgfpathcurveto{\pgfqpoint{1.785485in}{1.963251in}}{\pgfqpoint{1.774886in}{1.958860in}}{\pgfqpoint{1.767073in}{1.951047in}}%
\pgfpathcurveto{\pgfqpoint{1.759259in}{1.943233in}}{\pgfqpoint{1.754869in}{1.932634in}}{\pgfqpoint{1.754869in}{1.921584in}}%
\pgfpathcurveto{\pgfqpoint{1.754869in}{1.910534in}}{\pgfqpoint{1.759259in}{1.899935in}}{\pgfqpoint{1.767073in}{1.892121in}}%
\pgfpathcurveto{\pgfqpoint{1.774886in}{1.884308in}}{\pgfqpoint{1.785485in}{1.879917in}}{\pgfqpoint{1.796535in}{1.879917in}}%
\pgfpathlineto{\pgfqpoint{1.796535in}{1.879917in}}%
\pgfpathclose%
\pgfusepath{stroke}%
\end{pgfscope}%
\begin{pgfscope}%
\pgfpathrectangle{\pgfqpoint{0.847223in}{0.554012in}}{\pgfqpoint{6.200000in}{4.620000in}}%
\pgfusepath{clip}%
\pgfsetbuttcap%
\pgfsetroundjoin%
\pgfsetlinewidth{1.003750pt}%
\definecolor{currentstroke}{rgb}{1.000000,0.000000,0.000000}%
\pgfsetstrokecolor{currentstroke}%
\pgfsetdash{}{0pt}%
\pgfpathmoveto{\pgfqpoint{1.801869in}{1.873447in}}%
\pgfpathcurveto{\pgfqpoint{1.812919in}{1.873447in}}{\pgfqpoint{1.823518in}{1.877838in}}{\pgfqpoint{1.831331in}{1.885651in}}%
\pgfpathcurveto{\pgfqpoint{1.839145in}{1.893465in}}{\pgfqpoint{1.843535in}{1.904064in}}{\pgfqpoint{1.843535in}{1.915114in}}%
\pgfpathcurveto{\pgfqpoint{1.843535in}{1.926164in}}{\pgfqpoint{1.839145in}{1.936763in}}{\pgfqpoint{1.831331in}{1.944577in}}%
\pgfpathcurveto{\pgfqpoint{1.823518in}{1.952390in}}{\pgfqpoint{1.812919in}{1.956781in}}{\pgfqpoint{1.801869in}{1.956781in}}%
\pgfpathcurveto{\pgfqpoint{1.790819in}{1.956781in}}{\pgfqpoint{1.780219in}{1.952390in}}{\pgfqpoint{1.772406in}{1.944577in}}%
\pgfpathcurveto{\pgfqpoint{1.764592in}{1.936763in}}{\pgfqpoint{1.760202in}{1.926164in}}{\pgfqpoint{1.760202in}{1.915114in}}%
\pgfpathcurveto{\pgfqpoint{1.760202in}{1.904064in}}{\pgfqpoint{1.764592in}{1.893465in}}{\pgfqpoint{1.772406in}{1.885651in}}%
\pgfpathcurveto{\pgfqpoint{1.780219in}{1.877838in}}{\pgfqpoint{1.790819in}{1.873447in}}{\pgfqpoint{1.801869in}{1.873447in}}%
\pgfpathlineto{\pgfqpoint{1.801869in}{1.873447in}}%
\pgfpathclose%
\pgfusepath{stroke}%
\end{pgfscope}%
\begin{pgfscope}%
\pgfpathrectangle{\pgfqpoint{0.847223in}{0.554012in}}{\pgfqpoint{6.200000in}{4.620000in}}%
\pgfusepath{clip}%
\pgfsetbuttcap%
\pgfsetroundjoin%
\pgfsetlinewidth{1.003750pt}%
\definecolor{currentstroke}{rgb}{1.000000,0.000000,0.000000}%
\pgfsetstrokecolor{currentstroke}%
\pgfsetdash{}{0pt}%
\pgfpathmoveto{\pgfqpoint{1.807202in}{1.867024in}}%
\pgfpathcurveto{\pgfqpoint{1.818252in}{1.867024in}}{\pgfqpoint{1.828851in}{1.871414in}}{\pgfqpoint{1.836665in}{1.879228in}}%
\pgfpathcurveto{\pgfqpoint{1.844478in}{1.887041in}}{\pgfqpoint{1.848869in}{1.897640in}}{\pgfqpoint{1.848869in}{1.908690in}}%
\pgfpathcurveto{\pgfqpoint{1.848869in}{1.919741in}}{\pgfqpoint{1.844478in}{1.930340in}}{\pgfqpoint{1.836665in}{1.938153in}}%
\pgfpathcurveto{\pgfqpoint{1.828851in}{1.945967in}}{\pgfqpoint{1.818252in}{1.950357in}}{\pgfqpoint{1.807202in}{1.950357in}}%
\pgfpathcurveto{\pgfqpoint{1.796152in}{1.950357in}}{\pgfqpoint{1.785553in}{1.945967in}}{\pgfqpoint{1.777739in}{1.938153in}}%
\pgfpathcurveto{\pgfqpoint{1.769925in}{1.930340in}}{\pgfqpoint{1.765535in}{1.919741in}}{\pgfqpoint{1.765535in}{1.908690in}}%
\pgfpathcurveto{\pgfqpoint{1.765535in}{1.897640in}}{\pgfqpoint{1.769925in}{1.887041in}}{\pgfqpoint{1.777739in}{1.879228in}}%
\pgfpathcurveto{\pgfqpoint{1.785553in}{1.871414in}}{\pgfqpoint{1.796152in}{1.867024in}}{\pgfqpoint{1.807202in}{1.867024in}}%
\pgfpathlineto{\pgfqpoint{1.807202in}{1.867024in}}%
\pgfpathclose%
\pgfusepath{stroke}%
\end{pgfscope}%
\begin{pgfscope}%
\pgfpathrectangle{\pgfqpoint{0.847223in}{0.554012in}}{\pgfqpoint{6.200000in}{4.620000in}}%
\pgfusepath{clip}%
\pgfsetbuttcap%
\pgfsetroundjoin%
\pgfsetlinewidth{1.003750pt}%
\definecolor{currentstroke}{rgb}{1.000000,0.000000,0.000000}%
\pgfsetstrokecolor{currentstroke}%
\pgfsetdash{}{0pt}%
\pgfpathmoveto{\pgfqpoint{1.812535in}{1.860646in}}%
\pgfpathcurveto{\pgfqpoint{1.823585in}{1.860646in}}{\pgfqpoint{1.834184in}{1.865037in}}{\pgfqpoint{1.841998in}{1.872850in}}%
\pgfpathcurveto{\pgfqpoint{1.849811in}{1.880664in}}{\pgfqpoint{1.854202in}{1.891263in}}{\pgfqpoint{1.854202in}{1.902313in}}%
\pgfpathcurveto{\pgfqpoint{1.854202in}{1.913363in}}{\pgfqpoint{1.849811in}{1.923962in}}{\pgfqpoint{1.841998in}{1.931776in}}%
\pgfpathcurveto{\pgfqpoint{1.834184in}{1.939589in}}{\pgfqpoint{1.823585in}{1.943980in}}{\pgfqpoint{1.812535in}{1.943980in}}%
\pgfpathcurveto{\pgfqpoint{1.801485in}{1.943980in}}{\pgfqpoint{1.790886in}{1.939589in}}{\pgfqpoint{1.783072in}{1.931776in}}%
\pgfpathcurveto{\pgfqpoint{1.775259in}{1.923962in}}{\pgfqpoint{1.770868in}{1.913363in}}{\pgfqpoint{1.770868in}{1.902313in}}%
\pgfpathcurveto{\pgfqpoint{1.770868in}{1.891263in}}{\pgfqpoint{1.775259in}{1.880664in}}{\pgfqpoint{1.783072in}{1.872850in}}%
\pgfpathcurveto{\pgfqpoint{1.790886in}{1.865037in}}{\pgfqpoint{1.801485in}{1.860646in}}{\pgfqpoint{1.812535in}{1.860646in}}%
\pgfpathlineto{\pgfqpoint{1.812535in}{1.860646in}}%
\pgfpathclose%
\pgfusepath{stroke}%
\end{pgfscope}%
\begin{pgfscope}%
\pgfpathrectangle{\pgfqpoint{0.847223in}{0.554012in}}{\pgfqpoint{6.200000in}{4.620000in}}%
\pgfusepath{clip}%
\pgfsetbuttcap%
\pgfsetroundjoin%
\pgfsetlinewidth{1.003750pt}%
\definecolor{currentstroke}{rgb}{1.000000,0.000000,0.000000}%
\pgfsetstrokecolor{currentstroke}%
\pgfsetdash{}{0pt}%
\pgfpathmoveto{\pgfqpoint{1.817868in}{1.854314in}}%
\pgfpathcurveto{\pgfqpoint{1.828918in}{1.854314in}}{\pgfqpoint{1.839517in}{1.858705in}}{\pgfqpoint{1.847331in}{1.866518in}}%
\pgfpathcurveto{\pgfqpoint{1.855145in}{1.874332in}}{\pgfqpoint{1.859535in}{1.884931in}}{\pgfqpoint{1.859535in}{1.895981in}}%
\pgfpathcurveto{\pgfqpoint{1.859535in}{1.907031in}}{\pgfqpoint{1.855145in}{1.917630in}}{\pgfqpoint{1.847331in}{1.925444in}}%
\pgfpathcurveto{\pgfqpoint{1.839517in}{1.933257in}}{\pgfqpoint{1.828918in}{1.937648in}}{\pgfqpoint{1.817868in}{1.937648in}}%
\pgfpathcurveto{\pgfqpoint{1.806818in}{1.937648in}}{\pgfqpoint{1.796219in}{1.933257in}}{\pgfqpoint{1.788406in}{1.925444in}}%
\pgfpathcurveto{\pgfqpoint{1.780592in}{1.917630in}}{\pgfqpoint{1.776202in}{1.907031in}}{\pgfqpoint{1.776202in}{1.895981in}}%
\pgfpathcurveto{\pgfqpoint{1.776202in}{1.884931in}}{\pgfqpoint{1.780592in}{1.874332in}}{\pgfqpoint{1.788406in}{1.866518in}}%
\pgfpathcurveto{\pgfqpoint{1.796219in}{1.858705in}}{\pgfqpoint{1.806818in}{1.854314in}}{\pgfqpoint{1.817868in}{1.854314in}}%
\pgfpathlineto{\pgfqpoint{1.817868in}{1.854314in}}%
\pgfpathclose%
\pgfusepath{stroke}%
\end{pgfscope}%
\begin{pgfscope}%
\pgfpathrectangle{\pgfqpoint{0.847223in}{0.554012in}}{\pgfqpoint{6.200000in}{4.620000in}}%
\pgfusepath{clip}%
\pgfsetbuttcap%
\pgfsetroundjoin%
\pgfsetlinewidth{1.003750pt}%
\definecolor{currentstroke}{rgb}{1.000000,0.000000,0.000000}%
\pgfsetstrokecolor{currentstroke}%
\pgfsetdash{}{0pt}%
\pgfpathmoveto{\pgfqpoint{1.823202in}{1.848027in}}%
\pgfpathcurveto{\pgfqpoint{1.834252in}{1.848027in}}{\pgfqpoint{1.844851in}{1.852418in}}{\pgfqpoint{1.852664in}{1.860231in}}%
\pgfpathcurveto{\pgfqpoint{1.860478in}{1.868045in}}{\pgfqpoint{1.864868in}{1.878644in}}{\pgfqpoint{1.864868in}{1.889694in}}%
\pgfpathcurveto{\pgfqpoint{1.864868in}{1.900744in}}{\pgfqpoint{1.860478in}{1.911343in}}{\pgfqpoint{1.852664in}{1.919157in}}%
\pgfpathcurveto{\pgfqpoint{1.844851in}{1.926970in}}{\pgfqpoint{1.834252in}{1.931361in}}{\pgfqpoint{1.823202in}{1.931361in}}%
\pgfpathcurveto{\pgfqpoint{1.812151in}{1.931361in}}{\pgfqpoint{1.801552in}{1.926970in}}{\pgfqpoint{1.793739in}{1.919157in}}%
\pgfpathcurveto{\pgfqpoint{1.785925in}{1.911343in}}{\pgfqpoint{1.781535in}{1.900744in}}{\pgfqpoint{1.781535in}{1.889694in}}%
\pgfpathcurveto{\pgfqpoint{1.781535in}{1.878644in}}{\pgfqpoint{1.785925in}{1.868045in}}{\pgfqpoint{1.793739in}{1.860231in}}%
\pgfpathcurveto{\pgfqpoint{1.801552in}{1.852418in}}{\pgfqpoint{1.812151in}{1.848027in}}{\pgfqpoint{1.823202in}{1.848027in}}%
\pgfpathlineto{\pgfqpoint{1.823202in}{1.848027in}}%
\pgfpathclose%
\pgfusepath{stroke}%
\end{pgfscope}%
\begin{pgfscope}%
\pgfpathrectangle{\pgfqpoint{0.847223in}{0.554012in}}{\pgfqpoint{6.200000in}{4.620000in}}%
\pgfusepath{clip}%
\pgfsetbuttcap%
\pgfsetroundjoin%
\pgfsetlinewidth{1.003750pt}%
\definecolor{currentstroke}{rgb}{1.000000,0.000000,0.000000}%
\pgfsetstrokecolor{currentstroke}%
\pgfsetdash{}{0pt}%
\pgfpathmoveto{\pgfqpoint{1.828535in}{1.841785in}}%
\pgfpathcurveto{\pgfqpoint{1.839585in}{1.841785in}}{\pgfqpoint{1.850184in}{1.846175in}}{\pgfqpoint{1.857998in}{1.853989in}}%
\pgfpathcurveto{\pgfqpoint{1.865811in}{1.861803in}}{\pgfqpoint{1.870201in}{1.872402in}}{\pgfqpoint{1.870201in}{1.883452in}}%
\pgfpathcurveto{\pgfqpoint{1.870201in}{1.894502in}}{\pgfqpoint{1.865811in}{1.905101in}}{\pgfqpoint{1.857998in}{1.912915in}}%
\pgfpathcurveto{\pgfqpoint{1.850184in}{1.920728in}}{\pgfqpoint{1.839585in}{1.925118in}}{\pgfqpoint{1.828535in}{1.925118in}}%
\pgfpathcurveto{\pgfqpoint{1.817485in}{1.925118in}}{\pgfqpoint{1.806886in}{1.920728in}}{\pgfqpoint{1.799072in}{1.912915in}}%
\pgfpathcurveto{\pgfqpoint{1.791258in}{1.905101in}}{\pgfqpoint{1.786868in}{1.894502in}}{\pgfqpoint{1.786868in}{1.883452in}}%
\pgfpathcurveto{\pgfqpoint{1.786868in}{1.872402in}}{\pgfqpoint{1.791258in}{1.861803in}}{\pgfqpoint{1.799072in}{1.853989in}}%
\pgfpathcurveto{\pgfqpoint{1.806886in}{1.846175in}}{\pgfqpoint{1.817485in}{1.841785in}}{\pgfqpoint{1.828535in}{1.841785in}}%
\pgfpathlineto{\pgfqpoint{1.828535in}{1.841785in}}%
\pgfpathclose%
\pgfusepath{stroke}%
\end{pgfscope}%
\begin{pgfscope}%
\pgfpathrectangle{\pgfqpoint{0.847223in}{0.554012in}}{\pgfqpoint{6.200000in}{4.620000in}}%
\pgfusepath{clip}%
\pgfsetbuttcap%
\pgfsetroundjoin%
\pgfsetlinewidth{1.003750pt}%
\definecolor{currentstroke}{rgb}{1.000000,0.000000,0.000000}%
\pgfsetstrokecolor{currentstroke}%
\pgfsetdash{}{0pt}%
\pgfpathmoveto{\pgfqpoint{1.833868in}{1.835587in}}%
\pgfpathcurveto{\pgfqpoint{1.844918in}{1.835587in}}{\pgfqpoint{1.855517in}{1.839977in}}{\pgfqpoint{1.863331in}{1.847791in}}%
\pgfpathcurveto{\pgfqpoint{1.871144in}{1.855604in}}{\pgfqpoint{1.875535in}{1.866203in}}{\pgfqpoint{1.875535in}{1.877253in}}%
\pgfpathcurveto{\pgfqpoint{1.875535in}{1.888304in}}{\pgfqpoint{1.871144in}{1.898903in}}{\pgfqpoint{1.863331in}{1.906716in}}%
\pgfpathcurveto{\pgfqpoint{1.855517in}{1.914530in}}{\pgfqpoint{1.844918in}{1.918920in}}{\pgfqpoint{1.833868in}{1.918920in}}%
\pgfpathcurveto{\pgfqpoint{1.822818in}{1.918920in}}{\pgfqpoint{1.812219in}{1.914530in}}{\pgfqpoint{1.804405in}{1.906716in}}%
\pgfpathcurveto{\pgfqpoint{1.796592in}{1.898903in}}{\pgfqpoint{1.792201in}{1.888304in}}{\pgfqpoint{1.792201in}{1.877253in}}%
\pgfpathcurveto{\pgfqpoint{1.792201in}{1.866203in}}{\pgfqpoint{1.796592in}{1.855604in}}{\pgfqpoint{1.804405in}{1.847791in}}%
\pgfpathcurveto{\pgfqpoint{1.812219in}{1.839977in}}{\pgfqpoint{1.822818in}{1.835587in}}{\pgfqpoint{1.833868in}{1.835587in}}%
\pgfpathlineto{\pgfqpoint{1.833868in}{1.835587in}}%
\pgfpathclose%
\pgfusepath{stroke}%
\end{pgfscope}%
\begin{pgfscope}%
\pgfpathrectangle{\pgfqpoint{0.847223in}{0.554012in}}{\pgfqpoint{6.200000in}{4.620000in}}%
\pgfusepath{clip}%
\pgfsetbuttcap%
\pgfsetroundjoin%
\pgfsetlinewidth{1.003750pt}%
\definecolor{currentstroke}{rgb}{1.000000,0.000000,0.000000}%
\pgfsetstrokecolor{currentstroke}%
\pgfsetdash{}{0pt}%
\pgfpathmoveto{\pgfqpoint{1.839201in}{1.829432in}}%
\pgfpathcurveto{\pgfqpoint{1.850251in}{1.829432in}}{\pgfqpoint{1.860850in}{1.833822in}}{\pgfqpoint{1.868664in}{1.841636in}}%
\pgfpathcurveto{\pgfqpoint{1.876478in}{1.849450in}}{\pgfqpoint{1.880868in}{1.860049in}}{\pgfqpoint{1.880868in}{1.871099in}}%
\pgfpathcurveto{\pgfqpoint{1.880868in}{1.882149in}}{\pgfqpoint{1.876478in}{1.892748in}}{\pgfqpoint{1.868664in}{1.900562in}}%
\pgfpathcurveto{\pgfqpoint{1.860850in}{1.908375in}}{\pgfqpoint{1.850251in}{1.912766in}}{\pgfqpoint{1.839201in}{1.912766in}}%
\pgfpathcurveto{\pgfqpoint{1.828151in}{1.912766in}}{\pgfqpoint{1.817552in}{1.908375in}}{\pgfqpoint{1.809738in}{1.900562in}}%
\pgfpathcurveto{\pgfqpoint{1.801925in}{1.892748in}}{\pgfqpoint{1.797535in}{1.882149in}}{\pgfqpoint{1.797535in}{1.871099in}}%
\pgfpathcurveto{\pgfqpoint{1.797535in}{1.860049in}}{\pgfqpoint{1.801925in}{1.849450in}}{\pgfqpoint{1.809738in}{1.841636in}}%
\pgfpathcurveto{\pgfqpoint{1.817552in}{1.833822in}}{\pgfqpoint{1.828151in}{1.829432in}}{\pgfqpoint{1.839201in}{1.829432in}}%
\pgfpathlineto{\pgfqpoint{1.839201in}{1.829432in}}%
\pgfpathclose%
\pgfusepath{stroke}%
\end{pgfscope}%
\begin{pgfscope}%
\pgfpathrectangle{\pgfqpoint{0.847223in}{0.554012in}}{\pgfqpoint{6.200000in}{4.620000in}}%
\pgfusepath{clip}%
\pgfsetbuttcap%
\pgfsetroundjoin%
\pgfsetlinewidth{1.003750pt}%
\definecolor{currentstroke}{rgb}{1.000000,0.000000,0.000000}%
\pgfsetstrokecolor{currentstroke}%
\pgfsetdash{}{0pt}%
\pgfpathmoveto{\pgfqpoint{1.844534in}{1.823321in}}%
\pgfpathcurveto{\pgfqpoint{1.855585in}{1.823321in}}{\pgfqpoint{1.866184in}{1.827711in}}{\pgfqpoint{1.873997in}{1.835525in}}%
\pgfpathcurveto{\pgfqpoint{1.881811in}{1.843338in}}{\pgfqpoint{1.886201in}{1.853937in}}{\pgfqpoint{1.886201in}{1.864987in}}%
\pgfpathcurveto{\pgfqpoint{1.886201in}{1.876038in}}{\pgfqpoint{1.881811in}{1.886637in}}{\pgfqpoint{1.873997in}{1.894450in}}%
\pgfpathcurveto{\pgfqpoint{1.866184in}{1.902264in}}{\pgfqpoint{1.855585in}{1.906654in}}{\pgfqpoint{1.844534in}{1.906654in}}%
\pgfpathcurveto{\pgfqpoint{1.833484in}{1.906654in}}{\pgfqpoint{1.822885in}{1.902264in}}{\pgfqpoint{1.815072in}{1.894450in}}%
\pgfpathcurveto{\pgfqpoint{1.807258in}{1.886637in}}{\pgfqpoint{1.802868in}{1.876038in}}{\pgfqpoint{1.802868in}{1.864987in}}%
\pgfpathcurveto{\pgfqpoint{1.802868in}{1.853937in}}{\pgfqpoint{1.807258in}{1.843338in}}{\pgfqpoint{1.815072in}{1.835525in}}%
\pgfpathcurveto{\pgfqpoint{1.822885in}{1.827711in}}{\pgfqpoint{1.833484in}{1.823321in}}{\pgfqpoint{1.844534in}{1.823321in}}%
\pgfpathlineto{\pgfqpoint{1.844534in}{1.823321in}}%
\pgfpathclose%
\pgfusepath{stroke}%
\end{pgfscope}%
\begin{pgfscope}%
\pgfpathrectangle{\pgfqpoint{0.847223in}{0.554012in}}{\pgfqpoint{6.200000in}{4.620000in}}%
\pgfusepath{clip}%
\pgfsetbuttcap%
\pgfsetroundjoin%
\pgfsetlinewidth{1.003750pt}%
\definecolor{currentstroke}{rgb}{1.000000,0.000000,0.000000}%
\pgfsetstrokecolor{currentstroke}%
\pgfsetdash{}{0pt}%
\pgfpathmoveto{\pgfqpoint{1.849868in}{1.817252in}}%
\pgfpathcurveto{\pgfqpoint{1.860918in}{1.817252in}}{\pgfqpoint{1.871517in}{1.821642in}}{\pgfqpoint{1.879330in}{1.829456in}}%
\pgfpathcurveto{\pgfqpoint{1.887144in}{1.837270in}}{\pgfqpoint{1.891534in}{1.847869in}}{\pgfqpoint{1.891534in}{1.858919in}}%
\pgfpathcurveto{\pgfqpoint{1.891534in}{1.869969in}}{\pgfqpoint{1.887144in}{1.880568in}}{\pgfqpoint{1.879330in}{1.888381in}}%
\pgfpathcurveto{\pgfqpoint{1.871517in}{1.896195in}}{\pgfqpoint{1.860918in}{1.900585in}}{\pgfqpoint{1.849868in}{1.900585in}}%
\pgfpathcurveto{\pgfqpoint{1.838817in}{1.900585in}}{\pgfqpoint{1.828218in}{1.896195in}}{\pgfqpoint{1.820405in}{1.888381in}}%
\pgfpathcurveto{\pgfqpoint{1.812591in}{1.880568in}}{\pgfqpoint{1.808201in}{1.869969in}}{\pgfqpoint{1.808201in}{1.858919in}}%
\pgfpathcurveto{\pgfqpoint{1.808201in}{1.847869in}}{\pgfqpoint{1.812591in}{1.837270in}}{\pgfqpoint{1.820405in}{1.829456in}}%
\pgfpathcurveto{\pgfqpoint{1.828218in}{1.821642in}}{\pgfqpoint{1.838817in}{1.817252in}}{\pgfqpoint{1.849868in}{1.817252in}}%
\pgfpathlineto{\pgfqpoint{1.849868in}{1.817252in}}%
\pgfpathclose%
\pgfusepath{stroke}%
\end{pgfscope}%
\begin{pgfscope}%
\pgfpathrectangle{\pgfqpoint{0.847223in}{0.554012in}}{\pgfqpoint{6.200000in}{4.620000in}}%
\pgfusepath{clip}%
\pgfsetbuttcap%
\pgfsetroundjoin%
\pgfsetlinewidth{1.003750pt}%
\definecolor{currentstroke}{rgb}{1.000000,0.000000,0.000000}%
\pgfsetstrokecolor{currentstroke}%
\pgfsetdash{}{0pt}%
\pgfpathmoveto{\pgfqpoint{1.855201in}{1.811226in}}%
\pgfpathcurveto{\pgfqpoint{1.866251in}{1.811226in}}{\pgfqpoint{1.876850in}{1.815616in}}{\pgfqpoint{1.884664in}{1.823429in}}%
\pgfpathcurveto{\pgfqpoint{1.892477in}{1.831243in}}{\pgfqpoint{1.896867in}{1.841842in}}{\pgfqpoint{1.896867in}{1.852892in}}%
\pgfpathcurveto{\pgfqpoint{1.896867in}{1.863942in}}{\pgfqpoint{1.892477in}{1.874541in}}{\pgfqpoint{1.884664in}{1.882355in}}%
\pgfpathcurveto{\pgfqpoint{1.876850in}{1.890169in}}{\pgfqpoint{1.866251in}{1.894559in}}{\pgfqpoint{1.855201in}{1.894559in}}%
\pgfpathcurveto{\pgfqpoint{1.844151in}{1.894559in}}{\pgfqpoint{1.833552in}{1.890169in}}{\pgfqpoint{1.825738in}{1.882355in}}%
\pgfpathcurveto{\pgfqpoint{1.817924in}{1.874541in}}{\pgfqpoint{1.813534in}{1.863942in}}{\pgfqpoint{1.813534in}{1.852892in}}%
\pgfpathcurveto{\pgfqpoint{1.813534in}{1.841842in}}{\pgfqpoint{1.817924in}{1.831243in}}{\pgfqpoint{1.825738in}{1.823429in}}%
\pgfpathcurveto{\pgfqpoint{1.833552in}{1.815616in}}{\pgfqpoint{1.844151in}{1.811226in}}{\pgfqpoint{1.855201in}{1.811226in}}%
\pgfpathlineto{\pgfqpoint{1.855201in}{1.811226in}}%
\pgfpathclose%
\pgfusepath{stroke}%
\end{pgfscope}%
\begin{pgfscope}%
\pgfpathrectangle{\pgfqpoint{0.847223in}{0.554012in}}{\pgfqpoint{6.200000in}{4.620000in}}%
\pgfusepath{clip}%
\pgfsetbuttcap%
\pgfsetroundjoin%
\pgfsetlinewidth{1.003750pt}%
\definecolor{currentstroke}{rgb}{1.000000,0.000000,0.000000}%
\pgfsetstrokecolor{currentstroke}%
\pgfsetdash{}{0pt}%
\pgfpathmoveto{\pgfqpoint{1.860534in}{1.805241in}}%
\pgfpathcurveto{\pgfqpoint{1.871584in}{1.805241in}}{\pgfqpoint{1.882183in}{1.809631in}}{\pgfqpoint{1.889997in}{1.817445in}}%
\pgfpathcurveto{\pgfqpoint{1.897810in}{1.825258in}}{\pgfqpoint{1.902201in}{1.835857in}}{\pgfqpoint{1.902201in}{1.846908in}}%
\pgfpathcurveto{\pgfqpoint{1.902201in}{1.857958in}}{\pgfqpoint{1.897810in}{1.868557in}}{\pgfqpoint{1.889997in}{1.876370in}}%
\pgfpathcurveto{\pgfqpoint{1.882183in}{1.884184in}}{\pgfqpoint{1.871584in}{1.888574in}}{\pgfqpoint{1.860534in}{1.888574in}}%
\pgfpathcurveto{\pgfqpoint{1.849484in}{1.888574in}}{\pgfqpoint{1.838885in}{1.884184in}}{\pgfqpoint{1.831071in}{1.876370in}}%
\pgfpathcurveto{\pgfqpoint{1.823258in}{1.868557in}}{\pgfqpoint{1.818867in}{1.857958in}}{\pgfqpoint{1.818867in}{1.846908in}}%
\pgfpathcurveto{\pgfqpoint{1.818867in}{1.835857in}}{\pgfqpoint{1.823258in}{1.825258in}}{\pgfqpoint{1.831071in}{1.817445in}}%
\pgfpathcurveto{\pgfqpoint{1.838885in}{1.809631in}}{\pgfqpoint{1.849484in}{1.805241in}}{\pgfqpoint{1.860534in}{1.805241in}}%
\pgfpathlineto{\pgfqpoint{1.860534in}{1.805241in}}%
\pgfpathclose%
\pgfusepath{stroke}%
\end{pgfscope}%
\begin{pgfscope}%
\pgfpathrectangle{\pgfqpoint{0.847223in}{0.554012in}}{\pgfqpoint{6.200000in}{4.620000in}}%
\pgfusepath{clip}%
\pgfsetbuttcap%
\pgfsetroundjoin%
\pgfsetlinewidth{1.003750pt}%
\definecolor{currentstroke}{rgb}{1.000000,0.000000,0.000000}%
\pgfsetstrokecolor{currentstroke}%
\pgfsetdash{}{0pt}%
\pgfpathmoveto{\pgfqpoint{1.865867in}{1.799298in}}%
\pgfpathcurveto{\pgfqpoint{1.876917in}{1.799298in}}{\pgfqpoint{1.887516in}{1.803688in}}{\pgfqpoint{1.895330in}{1.811502in}}%
\pgfpathcurveto{\pgfqpoint{1.903144in}{1.819315in}}{\pgfqpoint{1.907534in}{1.829914in}}{\pgfqpoint{1.907534in}{1.840964in}}%
\pgfpathcurveto{\pgfqpoint{1.907534in}{1.852015in}}{\pgfqpoint{1.903144in}{1.862614in}}{\pgfqpoint{1.895330in}{1.870427in}}%
\pgfpathcurveto{\pgfqpoint{1.887516in}{1.878241in}}{\pgfqpoint{1.876917in}{1.882631in}}{\pgfqpoint{1.865867in}{1.882631in}}%
\pgfpathcurveto{\pgfqpoint{1.854817in}{1.882631in}}{\pgfqpoint{1.844218in}{1.878241in}}{\pgfqpoint{1.836404in}{1.870427in}}%
\pgfpathcurveto{\pgfqpoint{1.828591in}{1.862614in}}{\pgfqpoint{1.824201in}{1.852015in}}{\pgfqpoint{1.824201in}{1.840964in}}%
\pgfpathcurveto{\pgfqpoint{1.824201in}{1.829914in}}{\pgfqpoint{1.828591in}{1.819315in}}{\pgfqpoint{1.836404in}{1.811502in}}%
\pgfpathcurveto{\pgfqpoint{1.844218in}{1.803688in}}{\pgfqpoint{1.854817in}{1.799298in}}{\pgfqpoint{1.865867in}{1.799298in}}%
\pgfpathlineto{\pgfqpoint{1.865867in}{1.799298in}}%
\pgfpathclose%
\pgfusepath{stroke}%
\end{pgfscope}%
\begin{pgfscope}%
\pgfpathrectangle{\pgfqpoint{0.847223in}{0.554012in}}{\pgfqpoint{6.200000in}{4.620000in}}%
\pgfusepath{clip}%
\pgfsetbuttcap%
\pgfsetroundjoin%
\pgfsetlinewidth{1.003750pt}%
\definecolor{currentstroke}{rgb}{1.000000,0.000000,0.000000}%
\pgfsetstrokecolor{currentstroke}%
\pgfsetdash{}{0pt}%
\pgfpathmoveto{\pgfqpoint{1.871200in}{1.793395in}}%
\pgfpathcurveto{\pgfqpoint{1.882251in}{1.793395in}}{\pgfqpoint{1.892850in}{1.797786in}}{\pgfqpoint{1.900663in}{1.805599in}}%
\pgfpathcurveto{\pgfqpoint{1.908477in}{1.813413in}}{\pgfqpoint{1.912867in}{1.824012in}}{\pgfqpoint{1.912867in}{1.835062in}}%
\pgfpathcurveto{\pgfqpoint{1.912867in}{1.846112in}}{\pgfqpoint{1.908477in}{1.856711in}}{\pgfqpoint{1.900663in}{1.864525in}}%
\pgfpathcurveto{\pgfqpoint{1.892850in}{1.872339in}}{\pgfqpoint{1.882251in}{1.876729in}}{\pgfqpoint{1.871200in}{1.876729in}}%
\pgfpathcurveto{\pgfqpoint{1.860150in}{1.876729in}}{\pgfqpoint{1.849551in}{1.872339in}}{\pgfqpoint{1.841738in}{1.864525in}}%
\pgfpathcurveto{\pgfqpoint{1.833924in}{1.856711in}}{\pgfqpoint{1.829534in}{1.846112in}}{\pgfqpoint{1.829534in}{1.835062in}}%
\pgfpathcurveto{\pgfqpoint{1.829534in}{1.824012in}}{\pgfqpoint{1.833924in}{1.813413in}}{\pgfqpoint{1.841738in}{1.805599in}}%
\pgfpathcurveto{\pgfqpoint{1.849551in}{1.797786in}}{\pgfqpoint{1.860150in}{1.793395in}}{\pgfqpoint{1.871200in}{1.793395in}}%
\pgfpathlineto{\pgfqpoint{1.871200in}{1.793395in}}%
\pgfpathclose%
\pgfusepath{stroke}%
\end{pgfscope}%
\begin{pgfscope}%
\pgfpathrectangle{\pgfqpoint{0.847223in}{0.554012in}}{\pgfqpoint{6.200000in}{4.620000in}}%
\pgfusepath{clip}%
\pgfsetbuttcap%
\pgfsetroundjoin%
\pgfsetlinewidth{1.003750pt}%
\definecolor{currentstroke}{rgb}{1.000000,0.000000,0.000000}%
\pgfsetstrokecolor{currentstroke}%
\pgfsetdash{}{0pt}%
\pgfpathmoveto{\pgfqpoint{1.876534in}{1.787534in}}%
\pgfpathcurveto{\pgfqpoint{1.887584in}{1.787534in}}{\pgfqpoint{1.898183in}{1.791924in}}{\pgfqpoint{1.905996in}{1.799738in}}%
\pgfpathcurveto{\pgfqpoint{1.913810in}{1.807551in}}{\pgfqpoint{1.918200in}{1.818150in}}{\pgfqpoint{1.918200in}{1.829200in}}%
\pgfpathcurveto{\pgfqpoint{1.918200in}{1.840251in}}{\pgfqpoint{1.913810in}{1.850850in}}{\pgfqpoint{1.905996in}{1.858663in}}%
\pgfpathcurveto{\pgfqpoint{1.898183in}{1.866477in}}{\pgfqpoint{1.887584in}{1.870867in}}{\pgfqpoint{1.876534in}{1.870867in}}%
\pgfpathcurveto{\pgfqpoint{1.865484in}{1.870867in}}{\pgfqpoint{1.854885in}{1.866477in}}{\pgfqpoint{1.847071in}{1.858663in}}%
\pgfpathcurveto{\pgfqpoint{1.839257in}{1.850850in}}{\pgfqpoint{1.834867in}{1.840251in}}{\pgfqpoint{1.834867in}{1.829200in}}%
\pgfpathcurveto{\pgfqpoint{1.834867in}{1.818150in}}{\pgfqpoint{1.839257in}{1.807551in}}{\pgfqpoint{1.847071in}{1.799738in}}%
\pgfpathcurveto{\pgfqpoint{1.854885in}{1.791924in}}{\pgfqpoint{1.865484in}{1.787534in}}{\pgfqpoint{1.876534in}{1.787534in}}%
\pgfpathlineto{\pgfqpoint{1.876534in}{1.787534in}}%
\pgfpathclose%
\pgfusepath{stroke}%
\end{pgfscope}%
\begin{pgfscope}%
\pgfpathrectangle{\pgfqpoint{0.847223in}{0.554012in}}{\pgfqpoint{6.200000in}{4.620000in}}%
\pgfusepath{clip}%
\pgfsetbuttcap%
\pgfsetroundjoin%
\pgfsetlinewidth{1.003750pt}%
\definecolor{currentstroke}{rgb}{1.000000,0.000000,0.000000}%
\pgfsetstrokecolor{currentstroke}%
\pgfsetdash{}{0pt}%
\pgfpathmoveto{\pgfqpoint{1.881867in}{1.781712in}}%
\pgfpathcurveto{\pgfqpoint{1.892917in}{1.781712in}}{\pgfqpoint{1.903516in}{1.786102in}}{\pgfqpoint{1.911330in}{1.793916in}}%
\pgfpathcurveto{\pgfqpoint{1.919143in}{1.801730in}}{\pgfqpoint{1.923534in}{1.812329in}}{\pgfqpoint{1.923534in}{1.823379in}}%
\pgfpathcurveto{\pgfqpoint{1.923534in}{1.834429in}}{\pgfqpoint{1.919143in}{1.845028in}}{\pgfqpoint{1.911330in}{1.852842in}}%
\pgfpathcurveto{\pgfqpoint{1.903516in}{1.860655in}}{\pgfqpoint{1.892917in}{1.865046in}}{\pgfqpoint{1.881867in}{1.865046in}}%
\pgfpathcurveto{\pgfqpoint{1.870817in}{1.865046in}}{\pgfqpoint{1.860218in}{1.860655in}}{\pgfqpoint{1.852404in}{1.852842in}}%
\pgfpathcurveto{\pgfqpoint{1.844590in}{1.845028in}}{\pgfqpoint{1.840200in}{1.834429in}}{\pgfqpoint{1.840200in}{1.823379in}}%
\pgfpathcurveto{\pgfqpoint{1.840200in}{1.812329in}}{\pgfqpoint{1.844590in}{1.801730in}}{\pgfqpoint{1.852404in}{1.793916in}}%
\pgfpathcurveto{\pgfqpoint{1.860218in}{1.786102in}}{\pgfqpoint{1.870817in}{1.781712in}}{\pgfqpoint{1.881867in}{1.781712in}}%
\pgfpathlineto{\pgfqpoint{1.881867in}{1.781712in}}%
\pgfpathclose%
\pgfusepath{stroke}%
\end{pgfscope}%
\begin{pgfscope}%
\pgfpathrectangle{\pgfqpoint{0.847223in}{0.554012in}}{\pgfqpoint{6.200000in}{4.620000in}}%
\pgfusepath{clip}%
\pgfsetbuttcap%
\pgfsetroundjoin%
\pgfsetlinewidth{1.003750pt}%
\definecolor{currentstroke}{rgb}{1.000000,0.000000,0.000000}%
\pgfsetstrokecolor{currentstroke}%
\pgfsetdash{}{0pt}%
\pgfpathmoveto{\pgfqpoint{1.887200in}{1.775930in}}%
\pgfpathcurveto{\pgfqpoint{1.898250in}{1.775930in}}{\pgfqpoint{1.908849in}{1.780321in}}{\pgfqpoint{1.916663in}{1.788134in}}%
\pgfpathcurveto{\pgfqpoint{1.924477in}{1.795948in}}{\pgfqpoint{1.928867in}{1.806547in}}{\pgfqpoint{1.928867in}{1.817597in}}%
\pgfpathcurveto{\pgfqpoint{1.928867in}{1.828647in}}{\pgfqpoint{1.924477in}{1.839246in}}{\pgfqpoint{1.916663in}{1.847060in}}%
\pgfpathcurveto{\pgfqpoint{1.908849in}{1.854873in}}{\pgfqpoint{1.898250in}{1.859264in}}{\pgfqpoint{1.887200in}{1.859264in}}%
\pgfpathcurveto{\pgfqpoint{1.876150in}{1.859264in}}{\pgfqpoint{1.865551in}{1.854873in}}{\pgfqpoint{1.857737in}{1.847060in}}%
\pgfpathcurveto{\pgfqpoint{1.849924in}{1.839246in}}{\pgfqpoint{1.845533in}{1.828647in}}{\pgfqpoint{1.845533in}{1.817597in}}%
\pgfpathcurveto{\pgfqpoint{1.845533in}{1.806547in}}{\pgfqpoint{1.849924in}{1.795948in}}{\pgfqpoint{1.857737in}{1.788134in}}%
\pgfpathcurveto{\pgfqpoint{1.865551in}{1.780321in}}{\pgfqpoint{1.876150in}{1.775930in}}{\pgfqpoint{1.887200in}{1.775930in}}%
\pgfpathlineto{\pgfqpoint{1.887200in}{1.775930in}}%
\pgfpathclose%
\pgfusepath{stroke}%
\end{pgfscope}%
\begin{pgfscope}%
\pgfpathrectangle{\pgfqpoint{0.847223in}{0.554012in}}{\pgfqpoint{6.200000in}{4.620000in}}%
\pgfusepath{clip}%
\pgfsetbuttcap%
\pgfsetroundjoin%
\pgfsetlinewidth{1.003750pt}%
\definecolor{currentstroke}{rgb}{1.000000,0.000000,0.000000}%
\pgfsetstrokecolor{currentstroke}%
\pgfsetdash{}{0pt}%
\pgfpathmoveto{\pgfqpoint{1.892533in}{1.770188in}}%
\pgfpathcurveto{\pgfqpoint{1.903583in}{1.770188in}}{\pgfqpoint{1.914182in}{1.774578in}}{\pgfqpoint{1.921996in}{1.782392in}}%
\pgfpathcurveto{\pgfqpoint{1.929810in}{1.790205in}}{\pgfqpoint{1.934200in}{1.800804in}}{\pgfqpoint{1.934200in}{1.811854in}}%
\pgfpathcurveto{\pgfqpoint{1.934200in}{1.822905in}}{\pgfqpoint{1.929810in}{1.833504in}}{\pgfqpoint{1.921996in}{1.841317in}}%
\pgfpathcurveto{\pgfqpoint{1.914182in}{1.849131in}}{\pgfqpoint{1.903583in}{1.853521in}}{\pgfqpoint{1.892533in}{1.853521in}}%
\pgfpathcurveto{\pgfqpoint{1.881483in}{1.853521in}}{\pgfqpoint{1.870884in}{1.849131in}}{\pgfqpoint{1.863071in}{1.841317in}}%
\pgfpathcurveto{\pgfqpoint{1.855257in}{1.833504in}}{\pgfqpoint{1.850867in}{1.822905in}}{\pgfqpoint{1.850867in}{1.811854in}}%
\pgfpathcurveto{\pgfqpoint{1.850867in}{1.800804in}}{\pgfqpoint{1.855257in}{1.790205in}}{\pgfqpoint{1.863071in}{1.782392in}}%
\pgfpathcurveto{\pgfqpoint{1.870884in}{1.774578in}}{\pgfqpoint{1.881483in}{1.770188in}}{\pgfqpoint{1.892533in}{1.770188in}}%
\pgfpathlineto{\pgfqpoint{1.892533in}{1.770188in}}%
\pgfpathclose%
\pgfusepath{stroke}%
\end{pgfscope}%
\begin{pgfscope}%
\pgfpathrectangle{\pgfqpoint{0.847223in}{0.554012in}}{\pgfqpoint{6.200000in}{4.620000in}}%
\pgfusepath{clip}%
\pgfsetbuttcap%
\pgfsetroundjoin%
\pgfsetlinewidth{1.003750pt}%
\definecolor{currentstroke}{rgb}{1.000000,0.000000,0.000000}%
\pgfsetstrokecolor{currentstroke}%
\pgfsetdash{}{0pt}%
\pgfpathmoveto{\pgfqpoint{1.897867in}{1.764484in}}%
\pgfpathcurveto{\pgfqpoint{1.908917in}{1.764484in}}{\pgfqpoint{1.919516in}{1.768874in}}{\pgfqpoint{1.927329in}{1.776688in}}%
\pgfpathcurveto{\pgfqpoint{1.935143in}{1.784502in}}{\pgfqpoint{1.939533in}{1.795101in}}{\pgfqpoint{1.939533in}{1.806151in}}%
\pgfpathcurveto{\pgfqpoint{1.939533in}{1.817201in}}{\pgfqpoint{1.935143in}{1.827800in}}{\pgfqpoint{1.927329in}{1.835614in}}%
\pgfpathcurveto{\pgfqpoint{1.919516in}{1.843427in}}{\pgfqpoint{1.908917in}{1.847817in}}{\pgfqpoint{1.897867in}{1.847817in}}%
\pgfpathcurveto{\pgfqpoint{1.886816in}{1.847817in}}{\pgfqpoint{1.876217in}{1.843427in}}{\pgfqpoint{1.868404in}{1.835614in}}%
\pgfpathcurveto{\pgfqpoint{1.860590in}{1.827800in}}{\pgfqpoint{1.856200in}{1.817201in}}{\pgfqpoint{1.856200in}{1.806151in}}%
\pgfpathcurveto{\pgfqpoint{1.856200in}{1.795101in}}{\pgfqpoint{1.860590in}{1.784502in}}{\pgfqpoint{1.868404in}{1.776688in}}%
\pgfpathcurveto{\pgfqpoint{1.876217in}{1.768874in}}{\pgfqpoint{1.886816in}{1.764484in}}{\pgfqpoint{1.897867in}{1.764484in}}%
\pgfpathlineto{\pgfqpoint{1.897867in}{1.764484in}}%
\pgfpathclose%
\pgfusepath{stroke}%
\end{pgfscope}%
\begin{pgfscope}%
\pgfpathrectangle{\pgfqpoint{0.847223in}{0.554012in}}{\pgfqpoint{6.200000in}{4.620000in}}%
\pgfusepath{clip}%
\pgfsetbuttcap%
\pgfsetroundjoin%
\pgfsetlinewidth{1.003750pt}%
\definecolor{currentstroke}{rgb}{1.000000,0.000000,0.000000}%
\pgfsetstrokecolor{currentstroke}%
\pgfsetdash{}{0pt}%
\pgfpathmoveto{\pgfqpoint{1.903200in}{1.758819in}}%
\pgfpathcurveto{\pgfqpoint{1.914250in}{1.758819in}}{\pgfqpoint{1.924849in}{1.763209in}}{\pgfqpoint{1.932663in}{1.771023in}}%
\pgfpathcurveto{\pgfqpoint{1.940476in}{1.778837in}}{\pgfqpoint{1.944866in}{1.789436in}}{\pgfqpoint{1.944866in}{1.800486in}}%
\pgfpathcurveto{\pgfqpoint{1.944866in}{1.811536in}}{\pgfqpoint{1.940476in}{1.822135in}}{\pgfqpoint{1.932663in}{1.829948in}}%
\pgfpathcurveto{\pgfqpoint{1.924849in}{1.837762in}}{\pgfqpoint{1.914250in}{1.842152in}}{\pgfqpoint{1.903200in}{1.842152in}}%
\pgfpathcurveto{\pgfqpoint{1.892150in}{1.842152in}}{\pgfqpoint{1.881551in}{1.837762in}}{\pgfqpoint{1.873737in}{1.829948in}}%
\pgfpathcurveto{\pgfqpoint{1.865923in}{1.822135in}}{\pgfqpoint{1.861533in}{1.811536in}}{\pgfqpoint{1.861533in}{1.800486in}}%
\pgfpathcurveto{\pgfqpoint{1.861533in}{1.789436in}}{\pgfqpoint{1.865923in}{1.778837in}}{\pgfqpoint{1.873737in}{1.771023in}}%
\pgfpathcurveto{\pgfqpoint{1.881551in}{1.763209in}}{\pgfqpoint{1.892150in}{1.758819in}}{\pgfqpoint{1.903200in}{1.758819in}}%
\pgfpathlineto{\pgfqpoint{1.903200in}{1.758819in}}%
\pgfpathclose%
\pgfusepath{stroke}%
\end{pgfscope}%
\begin{pgfscope}%
\pgfpathrectangle{\pgfqpoint{0.847223in}{0.554012in}}{\pgfqpoint{6.200000in}{4.620000in}}%
\pgfusepath{clip}%
\pgfsetbuttcap%
\pgfsetroundjoin%
\pgfsetlinewidth{1.003750pt}%
\definecolor{currentstroke}{rgb}{1.000000,0.000000,0.000000}%
\pgfsetstrokecolor{currentstroke}%
\pgfsetdash{}{0pt}%
\pgfpathmoveto{\pgfqpoint{1.908533in}{1.753192in}}%
\pgfpathcurveto{\pgfqpoint{1.919583in}{1.753192in}}{\pgfqpoint{1.930182in}{1.757582in}}{\pgfqpoint{1.937996in}{1.765396in}}%
\pgfpathcurveto{\pgfqpoint{1.945809in}{1.773210in}}{\pgfqpoint{1.950200in}{1.783809in}}{\pgfqpoint{1.950200in}{1.794859in}}%
\pgfpathcurveto{\pgfqpoint{1.950200in}{1.805909in}}{\pgfqpoint{1.945809in}{1.816508in}}{\pgfqpoint{1.937996in}{1.824322in}}%
\pgfpathcurveto{\pgfqpoint{1.930182in}{1.832135in}}{\pgfqpoint{1.919583in}{1.836525in}}{\pgfqpoint{1.908533in}{1.836525in}}%
\pgfpathcurveto{\pgfqpoint{1.897483in}{1.836525in}}{\pgfqpoint{1.886884in}{1.832135in}}{\pgfqpoint{1.879070in}{1.824322in}}%
\pgfpathcurveto{\pgfqpoint{1.871257in}{1.816508in}}{\pgfqpoint{1.866866in}{1.805909in}}{\pgfqpoint{1.866866in}{1.794859in}}%
\pgfpathcurveto{\pgfqpoint{1.866866in}{1.783809in}}{\pgfqpoint{1.871257in}{1.773210in}}{\pgfqpoint{1.879070in}{1.765396in}}%
\pgfpathcurveto{\pgfqpoint{1.886884in}{1.757582in}}{\pgfqpoint{1.897483in}{1.753192in}}{\pgfqpoint{1.908533in}{1.753192in}}%
\pgfpathlineto{\pgfqpoint{1.908533in}{1.753192in}}%
\pgfpathclose%
\pgfusepath{stroke}%
\end{pgfscope}%
\begin{pgfscope}%
\pgfpathrectangle{\pgfqpoint{0.847223in}{0.554012in}}{\pgfqpoint{6.200000in}{4.620000in}}%
\pgfusepath{clip}%
\pgfsetbuttcap%
\pgfsetroundjoin%
\pgfsetlinewidth{1.003750pt}%
\definecolor{currentstroke}{rgb}{1.000000,0.000000,0.000000}%
\pgfsetstrokecolor{currentstroke}%
\pgfsetdash{}{0pt}%
\pgfpathmoveto{\pgfqpoint{1.913866in}{1.747603in}}%
\pgfpathcurveto{\pgfqpoint{1.924916in}{1.747603in}}{\pgfqpoint{1.935515in}{1.751993in}}{\pgfqpoint{1.943329in}{1.759807in}}%
\pgfpathcurveto{\pgfqpoint{1.951143in}{1.767620in}}{\pgfqpoint{1.955533in}{1.778219in}}{\pgfqpoint{1.955533in}{1.789269in}}%
\pgfpathcurveto{\pgfqpoint{1.955533in}{1.800320in}}{\pgfqpoint{1.951143in}{1.810919in}}{\pgfqpoint{1.943329in}{1.818732in}}%
\pgfpathcurveto{\pgfqpoint{1.935515in}{1.826546in}}{\pgfqpoint{1.924916in}{1.830936in}}{\pgfqpoint{1.913866in}{1.830936in}}%
\pgfpathcurveto{\pgfqpoint{1.902816in}{1.830936in}}{\pgfqpoint{1.892217in}{1.826546in}}{\pgfqpoint{1.884403in}{1.818732in}}%
\pgfpathcurveto{\pgfqpoint{1.876590in}{1.810919in}}{\pgfqpoint{1.872200in}{1.800320in}}{\pgfqpoint{1.872200in}{1.789269in}}%
\pgfpathcurveto{\pgfqpoint{1.872200in}{1.778219in}}{\pgfqpoint{1.876590in}{1.767620in}}{\pgfqpoint{1.884403in}{1.759807in}}%
\pgfpathcurveto{\pgfqpoint{1.892217in}{1.751993in}}{\pgfqpoint{1.902816in}{1.747603in}}{\pgfqpoint{1.913866in}{1.747603in}}%
\pgfpathlineto{\pgfqpoint{1.913866in}{1.747603in}}%
\pgfpathclose%
\pgfusepath{stroke}%
\end{pgfscope}%
\begin{pgfscope}%
\pgfpathrectangle{\pgfqpoint{0.847223in}{0.554012in}}{\pgfqpoint{6.200000in}{4.620000in}}%
\pgfusepath{clip}%
\pgfsetbuttcap%
\pgfsetroundjoin%
\pgfsetlinewidth{1.003750pt}%
\definecolor{currentstroke}{rgb}{1.000000,0.000000,0.000000}%
\pgfsetstrokecolor{currentstroke}%
\pgfsetdash{}{0pt}%
\pgfpathmoveto{\pgfqpoint{1.919199in}{1.742051in}}%
\pgfpathcurveto{\pgfqpoint{1.930250in}{1.742051in}}{\pgfqpoint{1.940849in}{1.746441in}}{\pgfqpoint{1.948662in}{1.754255in}}%
\pgfpathcurveto{\pgfqpoint{1.956476in}{1.762068in}}{\pgfqpoint{1.960866in}{1.772668in}}{\pgfqpoint{1.960866in}{1.783718in}}%
\pgfpathcurveto{\pgfqpoint{1.960866in}{1.794768in}}{\pgfqpoint{1.956476in}{1.805367in}}{\pgfqpoint{1.948662in}{1.813180in}}%
\pgfpathcurveto{\pgfqpoint{1.940849in}{1.820994in}}{\pgfqpoint{1.930250in}{1.825384in}}{\pgfqpoint{1.919199in}{1.825384in}}%
\pgfpathcurveto{\pgfqpoint{1.908149in}{1.825384in}}{\pgfqpoint{1.897550in}{1.820994in}}{\pgfqpoint{1.889737in}{1.813180in}}%
\pgfpathcurveto{\pgfqpoint{1.881923in}{1.805367in}}{\pgfqpoint{1.877533in}{1.794768in}}{\pgfqpoint{1.877533in}{1.783718in}}%
\pgfpathcurveto{\pgfqpoint{1.877533in}{1.772668in}}{\pgfqpoint{1.881923in}{1.762068in}}{\pgfqpoint{1.889737in}{1.754255in}}%
\pgfpathcurveto{\pgfqpoint{1.897550in}{1.746441in}}{\pgfqpoint{1.908149in}{1.742051in}}{\pgfqpoint{1.919199in}{1.742051in}}%
\pgfpathlineto{\pgfqpoint{1.919199in}{1.742051in}}%
\pgfpathclose%
\pgfusepath{stroke}%
\end{pgfscope}%
\begin{pgfscope}%
\pgfpathrectangle{\pgfqpoint{0.847223in}{0.554012in}}{\pgfqpoint{6.200000in}{4.620000in}}%
\pgfusepath{clip}%
\pgfsetbuttcap%
\pgfsetroundjoin%
\pgfsetlinewidth{1.003750pt}%
\definecolor{currentstroke}{rgb}{1.000000,0.000000,0.000000}%
\pgfsetstrokecolor{currentstroke}%
\pgfsetdash{}{0pt}%
\pgfpathmoveto{\pgfqpoint{1.924533in}{1.736536in}}%
\pgfpathcurveto{\pgfqpoint{1.935583in}{1.736536in}}{\pgfqpoint{1.946182in}{1.740926in}}{\pgfqpoint{1.953995in}{1.748740in}}%
\pgfpathcurveto{\pgfqpoint{1.961809in}{1.756554in}}{\pgfqpoint{1.966199in}{1.767153in}}{\pgfqpoint{1.966199in}{1.778203in}}%
\pgfpathcurveto{\pgfqpoint{1.966199in}{1.789253in}}{\pgfqpoint{1.961809in}{1.799852in}}{\pgfqpoint{1.953995in}{1.807666in}}%
\pgfpathcurveto{\pgfqpoint{1.946182in}{1.815479in}}{\pgfqpoint{1.935583in}{1.819869in}}{\pgfqpoint{1.924533in}{1.819869in}}%
\pgfpathcurveto{\pgfqpoint{1.913482in}{1.819869in}}{\pgfqpoint{1.902883in}{1.815479in}}{\pgfqpoint{1.895070in}{1.807666in}}%
\pgfpathcurveto{\pgfqpoint{1.887256in}{1.799852in}}{\pgfqpoint{1.882866in}{1.789253in}}{\pgfqpoint{1.882866in}{1.778203in}}%
\pgfpathcurveto{\pgfqpoint{1.882866in}{1.767153in}}{\pgfqpoint{1.887256in}{1.756554in}}{\pgfqpoint{1.895070in}{1.748740in}}%
\pgfpathcurveto{\pgfqpoint{1.902883in}{1.740926in}}{\pgfqpoint{1.913482in}{1.736536in}}{\pgfqpoint{1.924533in}{1.736536in}}%
\pgfpathlineto{\pgfqpoint{1.924533in}{1.736536in}}%
\pgfpathclose%
\pgfusepath{stroke}%
\end{pgfscope}%
\begin{pgfscope}%
\pgfpathrectangle{\pgfqpoint{0.847223in}{0.554012in}}{\pgfqpoint{6.200000in}{4.620000in}}%
\pgfusepath{clip}%
\pgfsetbuttcap%
\pgfsetroundjoin%
\pgfsetlinewidth{1.003750pt}%
\definecolor{currentstroke}{rgb}{1.000000,0.000000,0.000000}%
\pgfsetstrokecolor{currentstroke}%
\pgfsetdash{}{0pt}%
\pgfpathmoveto{\pgfqpoint{1.929866in}{1.731058in}}%
\pgfpathcurveto{\pgfqpoint{1.940916in}{1.731058in}}{\pgfqpoint{1.951515in}{1.735448in}}{\pgfqpoint{1.959329in}{1.743262in}}%
\pgfpathcurveto{\pgfqpoint{1.967142in}{1.751075in}}{\pgfqpoint{1.971533in}{1.761674in}}{\pgfqpoint{1.971533in}{1.772725in}}%
\pgfpathcurveto{\pgfqpoint{1.971533in}{1.783775in}}{\pgfqpoint{1.967142in}{1.794374in}}{\pgfqpoint{1.959329in}{1.802187in}}%
\pgfpathcurveto{\pgfqpoint{1.951515in}{1.810001in}}{\pgfqpoint{1.940916in}{1.814391in}}{\pgfqpoint{1.929866in}{1.814391in}}%
\pgfpathcurveto{\pgfqpoint{1.918816in}{1.814391in}}{\pgfqpoint{1.908217in}{1.810001in}}{\pgfqpoint{1.900403in}{1.802187in}}%
\pgfpathcurveto{\pgfqpoint{1.892589in}{1.794374in}}{\pgfqpoint{1.888199in}{1.783775in}}{\pgfqpoint{1.888199in}{1.772725in}}%
\pgfpathcurveto{\pgfqpoint{1.888199in}{1.761674in}}{\pgfqpoint{1.892589in}{1.751075in}}{\pgfqpoint{1.900403in}{1.743262in}}%
\pgfpathcurveto{\pgfqpoint{1.908217in}{1.735448in}}{\pgfqpoint{1.918816in}{1.731058in}}{\pgfqpoint{1.929866in}{1.731058in}}%
\pgfpathlineto{\pgfqpoint{1.929866in}{1.731058in}}%
\pgfpathclose%
\pgfusepath{stroke}%
\end{pgfscope}%
\begin{pgfscope}%
\pgfpathrectangle{\pgfqpoint{0.847223in}{0.554012in}}{\pgfqpoint{6.200000in}{4.620000in}}%
\pgfusepath{clip}%
\pgfsetbuttcap%
\pgfsetroundjoin%
\pgfsetlinewidth{1.003750pt}%
\definecolor{currentstroke}{rgb}{1.000000,0.000000,0.000000}%
\pgfsetstrokecolor{currentstroke}%
\pgfsetdash{}{0pt}%
\pgfpathmoveto{\pgfqpoint{1.935199in}{1.725616in}}%
\pgfpathcurveto{\pgfqpoint{1.946249in}{1.725616in}}{\pgfqpoint{1.956848in}{1.730006in}}{\pgfqpoint{1.964662in}{1.737820in}}%
\pgfpathcurveto{\pgfqpoint{1.972475in}{1.745633in}}{\pgfqpoint{1.976866in}{1.756232in}}{\pgfqpoint{1.976866in}{1.767283in}}%
\pgfpathcurveto{\pgfqpoint{1.976866in}{1.778333in}}{\pgfqpoint{1.972475in}{1.788932in}}{\pgfqpoint{1.964662in}{1.796745in}}%
\pgfpathcurveto{\pgfqpoint{1.956848in}{1.804559in}}{\pgfqpoint{1.946249in}{1.808949in}}{\pgfqpoint{1.935199in}{1.808949in}}%
\pgfpathcurveto{\pgfqpoint{1.924149in}{1.808949in}}{\pgfqpoint{1.913550in}{1.804559in}}{\pgfqpoint{1.905736in}{1.796745in}}%
\pgfpathcurveto{\pgfqpoint{1.897923in}{1.788932in}}{\pgfqpoint{1.893532in}{1.778333in}}{\pgfqpoint{1.893532in}{1.767283in}}%
\pgfpathcurveto{\pgfqpoint{1.893532in}{1.756232in}}{\pgfqpoint{1.897923in}{1.745633in}}{\pgfqpoint{1.905736in}{1.737820in}}%
\pgfpathcurveto{\pgfqpoint{1.913550in}{1.730006in}}{\pgfqpoint{1.924149in}{1.725616in}}{\pgfqpoint{1.935199in}{1.725616in}}%
\pgfpathlineto{\pgfqpoint{1.935199in}{1.725616in}}%
\pgfpathclose%
\pgfusepath{stroke}%
\end{pgfscope}%
\begin{pgfscope}%
\pgfpathrectangle{\pgfqpoint{0.847223in}{0.554012in}}{\pgfqpoint{6.200000in}{4.620000in}}%
\pgfusepath{clip}%
\pgfsetbuttcap%
\pgfsetroundjoin%
\pgfsetlinewidth{1.003750pt}%
\definecolor{currentstroke}{rgb}{1.000000,0.000000,0.000000}%
\pgfsetstrokecolor{currentstroke}%
\pgfsetdash{}{0pt}%
\pgfpathmoveto{\pgfqpoint{1.940532in}{1.720210in}}%
\pgfpathcurveto{\pgfqpoint{1.951582in}{1.720210in}}{\pgfqpoint{1.962181in}{1.724600in}}{\pgfqpoint{1.969995in}{1.732414in}}%
\pgfpathcurveto{\pgfqpoint{1.977809in}{1.740227in}}{\pgfqpoint{1.982199in}{1.750826in}}{\pgfqpoint{1.982199in}{1.761877in}}%
\pgfpathcurveto{\pgfqpoint{1.982199in}{1.772927in}}{\pgfqpoint{1.977809in}{1.783526in}}{\pgfqpoint{1.969995in}{1.791339in}}%
\pgfpathcurveto{\pgfqpoint{1.962181in}{1.799153in}}{\pgfqpoint{1.951582in}{1.803543in}}{\pgfqpoint{1.940532in}{1.803543in}}%
\pgfpathcurveto{\pgfqpoint{1.929482in}{1.803543in}}{\pgfqpoint{1.918883in}{1.799153in}}{\pgfqpoint{1.911069in}{1.791339in}}%
\pgfpathcurveto{\pgfqpoint{1.903256in}{1.783526in}}{\pgfqpoint{1.898866in}{1.772927in}}{\pgfqpoint{1.898866in}{1.761877in}}%
\pgfpathcurveto{\pgfqpoint{1.898866in}{1.750826in}}{\pgfqpoint{1.903256in}{1.740227in}}{\pgfqpoint{1.911069in}{1.732414in}}%
\pgfpathcurveto{\pgfqpoint{1.918883in}{1.724600in}}{\pgfqpoint{1.929482in}{1.720210in}}{\pgfqpoint{1.940532in}{1.720210in}}%
\pgfpathlineto{\pgfqpoint{1.940532in}{1.720210in}}%
\pgfpathclose%
\pgfusepath{stroke}%
\end{pgfscope}%
\begin{pgfscope}%
\pgfpathrectangle{\pgfqpoint{0.847223in}{0.554012in}}{\pgfqpoint{6.200000in}{4.620000in}}%
\pgfusepath{clip}%
\pgfsetbuttcap%
\pgfsetroundjoin%
\pgfsetlinewidth{1.003750pt}%
\definecolor{currentstroke}{rgb}{1.000000,0.000000,0.000000}%
\pgfsetstrokecolor{currentstroke}%
\pgfsetdash{}{0pt}%
\pgfpathmoveto{\pgfqpoint{1.945865in}{1.714839in}}%
\pgfpathcurveto{\pgfqpoint{1.956916in}{1.714839in}}{\pgfqpoint{1.967515in}{1.719230in}}{\pgfqpoint{1.975328in}{1.727043in}}%
\pgfpathcurveto{\pgfqpoint{1.983142in}{1.734857in}}{\pgfqpoint{1.987532in}{1.745456in}}{\pgfqpoint{1.987532in}{1.756506in}}%
\pgfpathcurveto{\pgfqpoint{1.987532in}{1.767556in}}{\pgfqpoint{1.983142in}{1.778155in}}{\pgfqpoint{1.975328in}{1.785969in}}%
\pgfpathcurveto{\pgfqpoint{1.967515in}{1.793782in}}{\pgfqpoint{1.956916in}{1.798173in}}{\pgfqpoint{1.945865in}{1.798173in}}%
\pgfpathcurveto{\pgfqpoint{1.934815in}{1.798173in}}{\pgfqpoint{1.924216in}{1.793782in}}{\pgfqpoint{1.916403in}{1.785969in}}%
\pgfpathcurveto{\pgfqpoint{1.908589in}{1.778155in}}{\pgfqpoint{1.904199in}{1.767556in}}{\pgfqpoint{1.904199in}{1.756506in}}%
\pgfpathcurveto{\pgfqpoint{1.904199in}{1.745456in}}{\pgfqpoint{1.908589in}{1.734857in}}{\pgfqpoint{1.916403in}{1.727043in}}%
\pgfpathcurveto{\pgfqpoint{1.924216in}{1.719230in}}{\pgfqpoint{1.934815in}{1.714839in}}{\pgfqpoint{1.945865in}{1.714839in}}%
\pgfpathlineto{\pgfqpoint{1.945865in}{1.714839in}}%
\pgfpathclose%
\pgfusepath{stroke}%
\end{pgfscope}%
\begin{pgfscope}%
\pgfpathrectangle{\pgfqpoint{0.847223in}{0.554012in}}{\pgfqpoint{6.200000in}{4.620000in}}%
\pgfusepath{clip}%
\pgfsetbuttcap%
\pgfsetroundjoin%
\pgfsetlinewidth{1.003750pt}%
\definecolor{currentstroke}{rgb}{1.000000,0.000000,0.000000}%
\pgfsetstrokecolor{currentstroke}%
\pgfsetdash{}{0pt}%
\pgfpathmoveto{\pgfqpoint{1.951199in}{1.709504in}}%
\pgfpathcurveto{\pgfqpoint{1.962249in}{1.709504in}}{\pgfqpoint{1.972848in}{1.713894in}}{\pgfqpoint{1.980661in}{1.721708in}}%
\pgfpathcurveto{\pgfqpoint{1.988475in}{1.729522in}}{\pgfqpoint{1.992865in}{1.740121in}}{\pgfqpoint{1.992865in}{1.751171in}}%
\pgfpathcurveto{\pgfqpoint{1.992865in}{1.762221in}}{\pgfqpoint{1.988475in}{1.772820in}}{\pgfqpoint{1.980661in}{1.780633in}}%
\pgfpathcurveto{\pgfqpoint{1.972848in}{1.788447in}}{\pgfqpoint{1.962249in}{1.792837in}}{\pgfqpoint{1.951199in}{1.792837in}}%
\pgfpathcurveto{\pgfqpoint{1.940149in}{1.792837in}}{\pgfqpoint{1.929550in}{1.788447in}}{\pgfqpoint{1.921736in}{1.780633in}}%
\pgfpathcurveto{\pgfqpoint{1.913922in}{1.772820in}}{\pgfqpoint{1.909532in}{1.762221in}}{\pgfqpoint{1.909532in}{1.751171in}}%
\pgfpathcurveto{\pgfqpoint{1.909532in}{1.740121in}}{\pgfqpoint{1.913922in}{1.729522in}}{\pgfqpoint{1.921736in}{1.721708in}}%
\pgfpathcurveto{\pgfqpoint{1.929550in}{1.713894in}}{\pgfqpoint{1.940149in}{1.709504in}}{\pgfqpoint{1.951199in}{1.709504in}}%
\pgfpathlineto{\pgfqpoint{1.951199in}{1.709504in}}%
\pgfpathclose%
\pgfusepath{stroke}%
\end{pgfscope}%
\begin{pgfscope}%
\pgfpathrectangle{\pgfqpoint{0.847223in}{0.554012in}}{\pgfqpoint{6.200000in}{4.620000in}}%
\pgfusepath{clip}%
\pgfsetbuttcap%
\pgfsetroundjoin%
\pgfsetlinewidth{1.003750pt}%
\definecolor{currentstroke}{rgb}{1.000000,0.000000,0.000000}%
\pgfsetstrokecolor{currentstroke}%
\pgfsetdash{}{0pt}%
\pgfpathmoveto{\pgfqpoint{1.956532in}{1.704204in}}%
\pgfpathcurveto{\pgfqpoint{1.967582in}{1.704204in}}{\pgfqpoint{1.978181in}{1.708594in}}{\pgfqpoint{1.985995in}{1.716407in}}%
\pgfpathcurveto{\pgfqpoint{1.993808in}{1.724221in}}{\pgfqpoint{1.998199in}{1.734820in}}{\pgfqpoint{1.998199in}{1.745870in}}%
\pgfpathcurveto{\pgfqpoint{1.998199in}{1.756920in}}{\pgfqpoint{1.993808in}{1.767519in}}{\pgfqpoint{1.985995in}{1.775333in}}%
\pgfpathcurveto{\pgfqpoint{1.978181in}{1.783147in}}{\pgfqpoint{1.967582in}{1.787537in}}{\pgfqpoint{1.956532in}{1.787537in}}%
\pgfpathcurveto{\pgfqpoint{1.945482in}{1.787537in}}{\pgfqpoint{1.934883in}{1.783147in}}{\pgfqpoint{1.927069in}{1.775333in}}%
\pgfpathcurveto{\pgfqpoint{1.919256in}{1.767519in}}{\pgfqpoint{1.914865in}{1.756920in}}{\pgfqpoint{1.914865in}{1.745870in}}%
\pgfpathcurveto{\pgfqpoint{1.914865in}{1.734820in}}{\pgfqpoint{1.919256in}{1.724221in}}{\pgfqpoint{1.927069in}{1.716407in}}%
\pgfpathcurveto{\pgfqpoint{1.934883in}{1.708594in}}{\pgfqpoint{1.945482in}{1.704204in}}{\pgfqpoint{1.956532in}{1.704204in}}%
\pgfpathlineto{\pgfqpoint{1.956532in}{1.704204in}}%
\pgfpathclose%
\pgfusepath{stroke}%
\end{pgfscope}%
\begin{pgfscope}%
\pgfpathrectangle{\pgfqpoint{0.847223in}{0.554012in}}{\pgfqpoint{6.200000in}{4.620000in}}%
\pgfusepath{clip}%
\pgfsetbuttcap%
\pgfsetroundjoin%
\pgfsetlinewidth{1.003750pt}%
\definecolor{currentstroke}{rgb}{1.000000,0.000000,0.000000}%
\pgfsetstrokecolor{currentstroke}%
\pgfsetdash{}{0pt}%
\pgfpathmoveto{\pgfqpoint{1.961865in}{1.698938in}}%
\pgfpathcurveto{\pgfqpoint{1.972915in}{1.698938in}}{\pgfqpoint{1.983514in}{1.703328in}}{\pgfqpoint{1.991328in}{1.711142in}}%
\pgfpathcurveto{\pgfqpoint{1.999142in}{1.718955in}}{\pgfqpoint{2.003532in}{1.729554in}}{\pgfqpoint{2.003532in}{1.740604in}}%
\pgfpathcurveto{\pgfqpoint{2.003532in}{1.751654in}}{\pgfqpoint{1.999142in}{1.762253in}}{\pgfqpoint{1.991328in}{1.770067in}}%
\pgfpathcurveto{\pgfqpoint{1.983514in}{1.777881in}}{\pgfqpoint{1.972915in}{1.782271in}}{\pgfqpoint{1.961865in}{1.782271in}}%
\pgfpathcurveto{\pgfqpoint{1.950815in}{1.782271in}}{\pgfqpoint{1.940216in}{1.777881in}}{\pgfqpoint{1.932402in}{1.770067in}}%
\pgfpathcurveto{\pgfqpoint{1.924589in}{1.762253in}}{\pgfqpoint{1.920198in}{1.751654in}}{\pgfqpoint{1.920198in}{1.740604in}}%
\pgfpathcurveto{\pgfqpoint{1.920198in}{1.729554in}}{\pgfqpoint{1.924589in}{1.718955in}}{\pgfqpoint{1.932402in}{1.711142in}}%
\pgfpathcurveto{\pgfqpoint{1.940216in}{1.703328in}}{\pgfqpoint{1.950815in}{1.698938in}}{\pgfqpoint{1.961865in}{1.698938in}}%
\pgfpathlineto{\pgfqpoint{1.961865in}{1.698938in}}%
\pgfpathclose%
\pgfusepath{stroke}%
\end{pgfscope}%
\begin{pgfscope}%
\pgfpathrectangle{\pgfqpoint{0.847223in}{0.554012in}}{\pgfqpoint{6.200000in}{4.620000in}}%
\pgfusepath{clip}%
\pgfsetbuttcap%
\pgfsetroundjoin%
\pgfsetlinewidth{1.003750pt}%
\definecolor{currentstroke}{rgb}{1.000000,0.000000,0.000000}%
\pgfsetstrokecolor{currentstroke}%
\pgfsetdash{}{0pt}%
\pgfpathmoveto{\pgfqpoint{1.967198in}{1.693706in}}%
\pgfpathcurveto{\pgfqpoint{1.978248in}{1.693706in}}{\pgfqpoint{1.988848in}{1.698096in}}{\pgfqpoint{1.996661in}{1.705910in}}%
\pgfpathcurveto{\pgfqpoint{2.004475in}{1.713723in}}{\pgfqpoint{2.008865in}{1.724322in}}{\pgfqpoint{2.008865in}{1.735373in}}%
\pgfpathcurveto{\pgfqpoint{2.008865in}{1.746423in}}{\pgfqpoint{2.004475in}{1.757022in}}{\pgfqpoint{1.996661in}{1.764835in}}%
\pgfpathcurveto{\pgfqpoint{1.988848in}{1.772649in}}{\pgfqpoint{1.978248in}{1.777039in}}{\pgfqpoint{1.967198in}{1.777039in}}%
\pgfpathcurveto{\pgfqpoint{1.956148in}{1.777039in}}{\pgfqpoint{1.945549in}{1.772649in}}{\pgfqpoint{1.937736in}{1.764835in}}%
\pgfpathcurveto{\pgfqpoint{1.929922in}{1.757022in}}{\pgfqpoint{1.925532in}{1.746423in}}{\pgfqpoint{1.925532in}{1.735373in}}%
\pgfpathcurveto{\pgfqpoint{1.925532in}{1.724322in}}{\pgfqpoint{1.929922in}{1.713723in}}{\pgfqpoint{1.937736in}{1.705910in}}%
\pgfpathcurveto{\pgfqpoint{1.945549in}{1.698096in}}{\pgfqpoint{1.956148in}{1.693706in}}{\pgfqpoint{1.967198in}{1.693706in}}%
\pgfpathlineto{\pgfqpoint{1.967198in}{1.693706in}}%
\pgfpathclose%
\pgfusepath{stroke}%
\end{pgfscope}%
\begin{pgfscope}%
\pgfpathrectangle{\pgfqpoint{0.847223in}{0.554012in}}{\pgfqpoint{6.200000in}{4.620000in}}%
\pgfusepath{clip}%
\pgfsetbuttcap%
\pgfsetroundjoin%
\pgfsetlinewidth{1.003750pt}%
\definecolor{currentstroke}{rgb}{1.000000,0.000000,0.000000}%
\pgfsetstrokecolor{currentstroke}%
\pgfsetdash{}{0pt}%
\pgfpathmoveto{\pgfqpoint{1.972532in}{1.688508in}}%
\pgfpathcurveto{\pgfqpoint{1.983582in}{1.688508in}}{\pgfqpoint{1.994181in}{1.692898in}}{\pgfqpoint{2.001994in}{1.700712in}}%
\pgfpathcurveto{\pgfqpoint{2.009808in}{1.708525in}}{\pgfqpoint{2.014198in}{1.719124in}}{\pgfqpoint{2.014198in}{1.730175in}}%
\pgfpathcurveto{\pgfqpoint{2.014198in}{1.741225in}}{\pgfqpoint{2.009808in}{1.751824in}}{\pgfqpoint{2.001994in}{1.759637in}}%
\pgfpathcurveto{\pgfqpoint{1.994181in}{1.767451in}}{\pgfqpoint{1.983582in}{1.771841in}}{\pgfqpoint{1.972532in}{1.771841in}}%
\pgfpathcurveto{\pgfqpoint{1.961481in}{1.771841in}}{\pgfqpoint{1.950882in}{1.767451in}}{\pgfqpoint{1.943069in}{1.759637in}}%
\pgfpathcurveto{\pgfqpoint{1.935255in}{1.751824in}}{\pgfqpoint{1.930865in}{1.741225in}}{\pgfqpoint{1.930865in}{1.730175in}}%
\pgfpathcurveto{\pgfqpoint{1.930865in}{1.719124in}}{\pgfqpoint{1.935255in}{1.708525in}}{\pgfqpoint{1.943069in}{1.700712in}}%
\pgfpathcurveto{\pgfqpoint{1.950882in}{1.692898in}}{\pgfqpoint{1.961481in}{1.688508in}}{\pgfqpoint{1.972532in}{1.688508in}}%
\pgfpathlineto{\pgfqpoint{1.972532in}{1.688508in}}%
\pgfpathclose%
\pgfusepath{stroke}%
\end{pgfscope}%
\begin{pgfscope}%
\pgfpathrectangle{\pgfqpoint{0.847223in}{0.554012in}}{\pgfqpoint{6.200000in}{4.620000in}}%
\pgfusepath{clip}%
\pgfsetbuttcap%
\pgfsetroundjoin%
\pgfsetlinewidth{1.003750pt}%
\definecolor{currentstroke}{rgb}{1.000000,0.000000,0.000000}%
\pgfsetstrokecolor{currentstroke}%
\pgfsetdash{}{0pt}%
\pgfpathmoveto{\pgfqpoint{1.977865in}{1.683344in}}%
\pgfpathcurveto{\pgfqpoint{1.988915in}{1.683344in}}{\pgfqpoint{1.999514in}{1.687734in}}{\pgfqpoint{2.007328in}{1.695547in}}%
\pgfpathcurveto{\pgfqpoint{2.015141in}{1.703361in}}{\pgfqpoint{2.019531in}{1.713960in}}{\pgfqpoint{2.019531in}{1.725010in}}%
\pgfpathcurveto{\pgfqpoint{2.019531in}{1.736060in}}{\pgfqpoint{2.015141in}{1.746659in}}{\pgfqpoint{2.007328in}{1.754473in}}%
\pgfpathcurveto{\pgfqpoint{1.999514in}{1.762287in}}{\pgfqpoint{1.988915in}{1.766677in}}{\pgfqpoint{1.977865in}{1.766677in}}%
\pgfpathcurveto{\pgfqpoint{1.966815in}{1.766677in}}{\pgfqpoint{1.956216in}{1.762287in}}{\pgfqpoint{1.948402in}{1.754473in}}%
\pgfpathcurveto{\pgfqpoint{1.940588in}{1.746659in}}{\pgfqpoint{1.936198in}{1.736060in}}{\pgfqpoint{1.936198in}{1.725010in}}%
\pgfpathcurveto{\pgfqpoint{1.936198in}{1.713960in}}{\pgfqpoint{1.940588in}{1.703361in}}{\pgfqpoint{1.948402in}{1.695547in}}%
\pgfpathcurveto{\pgfqpoint{1.956216in}{1.687734in}}{\pgfqpoint{1.966815in}{1.683344in}}{\pgfqpoint{1.977865in}{1.683344in}}%
\pgfpathlineto{\pgfqpoint{1.977865in}{1.683344in}}%
\pgfpathclose%
\pgfusepath{stroke}%
\end{pgfscope}%
\begin{pgfscope}%
\pgfpathrectangle{\pgfqpoint{0.847223in}{0.554012in}}{\pgfqpoint{6.200000in}{4.620000in}}%
\pgfusepath{clip}%
\pgfsetbuttcap%
\pgfsetroundjoin%
\pgfsetlinewidth{1.003750pt}%
\definecolor{currentstroke}{rgb}{1.000000,0.000000,0.000000}%
\pgfsetstrokecolor{currentstroke}%
\pgfsetdash{}{0pt}%
\pgfpathmoveto{\pgfqpoint{1.983198in}{1.678212in}}%
\pgfpathcurveto{\pgfqpoint{1.994248in}{1.678212in}}{\pgfqpoint{2.004847in}{1.682603in}}{\pgfqpoint{2.012661in}{1.690416in}}%
\pgfpathcurveto{\pgfqpoint{2.020474in}{1.698230in}}{\pgfqpoint{2.024865in}{1.708829in}}{\pgfqpoint{2.024865in}{1.719879in}}%
\pgfpathcurveto{\pgfqpoint{2.024865in}{1.730929in}}{\pgfqpoint{2.020474in}{1.741528in}}{\pgfqpoint{2.012661in}{1.749342in}}%
\pgfpathcurveto{\pgfqpoint{2.004847in}{1.757155in}}{\pgfqpoint{1.994248in}{1.761546in}}{\pgfqpoint{1.983198in}{1.761546in}}%
\pgfpathcurveto{\pgfqpoint{1.972148in}{1.761546in}}{\pgfqpoint{1.961549in}{1.757155in}}{\pgfqpoint{1.953735in}{1.749342in}}%
\pgfpathcurveto{\pgfqpoint{1.945922in}{1.741528in}}{\pgfqpoint{1.941531in}{1.730929in}}{\pgfqpoint{1.941531in}{1.719879in}}%
\pgfpathcurveto{\pgfqpoint{1.941531in}{1.708829in}}{\pgfqpoint{1.945922in}{1.698230in}}{\pgfqpoint{1.953735in}{1.690416in}}%
\pgfpathcurveto{\pgfqpoint{1.961549in}{1.682603in}}{\pgfqpoint{1.972148in}{1.678212in}}{\pgfqpoint{1.983198in}{1.678212in}}%
\pgfpathlineto{\pgfqpoint{1.983198in}{1.678212in}}%
\pgfpathclose%
\pgfusepath{stroke}%
\end{pgfscope}%
\begin{pgfscope}%
\pgfpathrectangle{\pgfqpoint{0.847223in}{0.554012in}}{\pgfqpoint{6.200000in}{4.620000in}}%
\pgfusepath{clip}%
\pgfsetbuttcap%
\pgfsetroundjoin%
\pgfsetlinewidth{1.003750pt}%
\definecolor{currentstroke}{rgb}{1.000000,0.000000,0.000000}%
\pgfsetstrokecolor{currentstroke}%
\pgfsetdash{}{0pt}%
\pgfpathmoveto{\pgfqpoint{1.988531in}{1.673114in}}%
\pgfpathcurveto{\pgfqpoint{1.999581in}{1.673114in}}{\pgfqpoint{2.010180in}{1.677504in}}{\pgfqpoint{2.017994in}{1.685318in}}%
\pgfpathcurveto{\pgfqpoint{2.025808in}{1.693132in}}{\pgfqpoint{2.030198in}{1.703731in}}{\pgfqpoint{2.030198in}{1.714781in}}%
\pgfpathcurveto{\pgfqpoint{2.030198in}{1.725831in}}{\pgfqpoint{2.025808in}{1.736430in}}{\pgfqpoint{2.017994in}{1.744243in}}%
\pgfpathcurveto{\pgfqpoint{2.010180in}{1.752057in}}{\pgfqpoint{1.999581in}{1.756447in}}{\pgfqpoint{1.988531in}{1.756447in}}%
\pgfpathcurveto{\pgfqpoint{1.977481in}{1.756447in}}{\pgfqpoint{1.966882in}{1.752057in}}{\pgfqpoint{1.959068in}{1.744243in}}%
\pgfpathcurveto{\pgfqpoint{1.951255in}{1.736430in}}{\pgfqpoint{1.946865in}{1.725831in}}{\pgfqpoint{1.946865in}{1.714781in}}%
\pgfpathcurveto{\pgfqpoint{1.946865in}{1.703731in}}{\pgfqpoint{1.951255in}{1.693132in}}{\pgfqpoint{1.959068in}{1.685318in}}%
\pgfpathcurveto{\pgfqpoint{1.966882in}{1.677504in}}{\pgfqpoint{1.977481in}{1.673114in}}{\pgfqpoint{1.988531in}{1.673114in}}%
\pgfpathlineto{\pgfqpoint{1.988531in}{1.673114in}}%
\pgfpathclose%
\pgfusepath{stroke}%
\end{pgfscope}%
\begin{pgfscope}%
\pgfpathrectangle{\pgfqpoint{0.847223in}{0.554012in}}{\pgfqpoint{6.200000in}{4.620000in}}%
\pgfusepath{clip}%
\pgfsetbuttcap%
\pgfsetroundjoin%
\pgfsetlinewidth{1.003750pt}%
\definecolor{currentstroke}{rgb}{1.000000,0.000000,0.000000}%
\pgfsetstrokecolor{currentstroke}%
\pgfsetdash{}{0pt}%
\pgfpathmoveto{\pgfqpoint{1.993864in}{1.668048in}}%
\pgfpathcurveto{\pgfqpoint{2.004915in}{1.668048in}}{\pgfqpoint{2.015514in}{1.672439in}}{\pgfqpoint{2.023327in}{1.680252in}}%
\pgfpathcurveto{\pgfqpoint{2.031141in}{1.688066in}}{\pgfqpoint{2.035531in}{1.698665in}}{\pgfqpoint{2.035531in}{1.709715in}}%
\pgfpathcurveto{\pgfqpoint{2.035531in}{1.720765in}}{\pgfqpoint{2.031141in}{1.731364in}}{\pgfqpoint{2.023327in}{1.739178in}}%
\pgfpathcurveto{\pgfqpoint{2.015514in}{1.746991in}}{\pgfqpoint{2.004915in}{1.751382in}}{\pgfqpoint{1.993864in}{1.751382in}}%
\pgfpathcurveto{\pgfqpoint{1.982814in}{1.751382in}}{\pgfqpoint{1.972215in}{1.746991in}}{\pgfqpoint{1.964402in}{1.739178in}}%
\pgfpathcurveto{\pgfqpoint{1.956588in}{1.731364in}}{\pgfqpoint{1.952198in}{1.720765in}}{\pgfqpoint{1.952198in}{1.709715in}}%
\pgfpathcurveto{\pgfqpoint{1.952198in}{1.698665in}}{\pgfqpoint{1.956588in}{1.688066in}}{\pgfqpoint{1.964402in}{1.680252in}}%
\pgfpathcurveto{\pgfqpoint{1.972215in}{1.672439in}}{\pgfqpoint{1.982814in}{1.668048in}}{\pgfqpoint{1.993864in}{1.668048in}}%
\pgfpathlineto{\pgfqpoint{1.993864in}{1.668048in}}%
\pgfpathclose%
\pgfusepath{stroke}%
\end{pgfscope}%
\begin{pgfscope}%
\pgfpathrectangle{\pgfqpoint{0.847223in}{0.554012in}}{\pgfqpoint{6.200000in}{4.620000in}}%
\pgfusepath{clip}%
\pgfsetbuttcap%
\pgfsetroundjoin%
\pgfsetlinewidth{1.003750pt}%
\definecolor{currentstroke}{rgb}{1.000000,0.000000,0.000000}%
\pgfsetstrokecolor{currentstroke}%
\pgfsetdash{}{0pt}%
\pgfpathmoveto{\pgfqpoint{1.999198in}{1.663015in}}%
\pgfpathcurveto{\pgfqpoint{2.010248in}{1.663015in}}{\pgfqpoint{2.020847in}{1.667405in}}{\pgfqpoint{2.028660in}{1.675219in}}%
\pgfpathcurveto{\pgfqpoint{2.036474in}{1.683032in}}{\pgfqpoint{2.040864in}{1.693631in}}{\pgfqpoint{2.040864in}{1.704681in}}%
\pgfpathcurveto{\pgfqpoint{2.040864in}{1.715732in}}{\pgfqpoint{2.036474in}{1.726331in}}{\pgfqpoint{2.028660in}{1.734144in}}%
\pgfpathcurveto{\pgfqpoint{2.020847in}{1.741958in}}{\pgfqpoint{2.010248in}{1.746348in}}{\pgfqpoint{1.999198in}{1.746348in}}%
\pgfpathcurveto{\pgfqpoint{1.988148in}{1.746348in}}{\pgfqpoint{1.977548in}{1.741958in}}{\pgfqpoint{1.969735in}{1.734144in}}%
\pgfpathcurveto{\pgfqpoint{1.961921in}{1.726331in}}{\pgfqpoint{1.957531in}{1.715732in}}{\pgfqpoint{1.957531in}{1.704681in}}%
\pgfpathcurveto{\pgfqpoint{1.957531in}{1.693631in}}{\pgfqpoint{1.961921in}{1.683032in}}{\pgfqpoint{1.969735in}{1.675219in}}%
\pgfpathcurveto{\pgfqpoint{1.977548in}{1.667405in}}{\pgfqpoint{1.988148in}{1.663015in}}{\pgfqpoint{1.999198in}{1.663015in}}%
\pgfpathlineto{\pgfqpoint{1.999198in}{1.663015in}}%
\pgfpathclose%
\pgfusepath{stroke}%
\end{pgfscope}%
\begin{pgfscope}%
\pgfpathrectangle{\pgfqpoint{0.847223in}{0.554012in}}{\pgfqpoint{6.200000in}{4.620000in}}%
\pgfusepath{clip}%
\pgfsetbuttcap%
\pgfsetroundjoin%
\pgfsetlinewidth{1.003750pt}%
\definecolor{currentstroke}{rgb}{1.000000,0.000000,0.000000}%
\pgfsetstrokecolor{currentstroke}%
\pgfsetdash{}{0pt}%
\pgfpathmoveto{\pgfqpoint{2.004531in}{1.658013in}}%
\pgfpathcurveto{\pgfqpoint{2.015581in}{1.658013in}}{\pgfqpoint{2.026180in}{1.662404in}}{\pgfqpoint{2.033994in}{1.670217in}}%
\pgfpathcurveto{\pgfqpoint{2.041807in}{1.678031in}}{\pgfqpoint{2.046198in}{1.688630in}}{\pgfqpoint{2.046198in}{1.699680in}}%
\pgfpathcurveto{\pgfqpoint{2.046198in}{1.710730in}}{\pgfqpoint{2.041807in}{1.721329in}}{\pgfqpoint{2.033994in}{1.729143in}}%
\pgfpathcurveto{\pgfqpoint{2.026180in}{1.736956in}}{\pgfqpoint{2.015581in}{1.741347in}}{\pgfqpoint{2.004531in}{1.741347in}}%
\pgfpathcurveto{\pgfqpoint{1.993481in}{1.741347in}}{\pgfqpoint{1.982882in}{1.736956in}}{\pgfqpoint{1.975068in}{1.729143in}}%
\pgfpathcurveto{\pgfqpoint{1.967254in}{1.721329in}}{\pgfqpoint{1.962864in}{1.710730in}}{\pgfqpoint{1.962864in}{1.699680in}}%
\pgfpathcurveto{\pgfqpoint{1.962864in}{1.688630in}}{\pgfqpoint{1.967254in}{1.678031in}}{\pgfqpoint{1.975068in}{1.670217in}}%
\pgfpathcurveto{\pgfqpoint{1.982882in}{1.662404in}}{\pgfqpoint{1.993481in}{1.658013in}}{\pgfqpoint{2.004531in}{1.658013in}}%
\pgfpathlineto{\pgfqpoint{2.004531in}{1.658013in}}%
\pgfpathclose%
\pgfusepath{stroke}%
\end{pgfscope}%
\begin{pgfscope}%
\pgfpathrectangle{\pgfqpoint{0.847223in}{0.554012in}}{\pgfqpoint{6.200000in}{4.620000in}}%
\pgfusepath{clip}%
\pgfsetbuttcap%
\pgfsetroundjoin%
\pgfsetlinewidth{1.003750pt}%
\definecolor{currentstroke}{rgb}{1.000000,0.000000,0.000000}%
\pgfsetstrokecolor{currentstroke}%
\pgfsetdash{}{0pt}%
\pgfpathmoveto{\pgfqpoint{2.009864in}{1.653043in}}%
\pgfpathcurveto{\pgfqpoint{2.020914in}{1.653043in}}{\pgfqpoint{2.031513in}{1.657434in}}{\pgfqpoint{2.039327in}{1.665247in}}%
\pgfpathcurveto{\pgfqpoint{2.047140in}{1.673061in}}{\pgfqpoint{2.051531in}{1.683660in}}{\pgfqpoint{2.051531in}{1.694710in}}%
\pgfpathcurveto{\pgfqpoint{2.051531in}{1.705760in}}{\pgfqpoint{2.047140in}{1.716359in}}{\pgfqpoint{2.039327in}{1.724173in}}%
\pgfpathcurveto{\pgfqpoint{2.031513in}{1.731986in}}{\pgfqpoint{2.020914in}{1.736377in}}{\pgfqpoint{2.009864in}{1.736377in}}%
\pgfpathcurveto{\pgfqpoint{1.998814in}{1.736377in}}{\pgfqpoint{1.988215in}{1.731986in}}{\pgfqpoint{1.980401in}{1.724173in}}%
\pgfpathcurveto{\pgfqpoint{1.972588in}{1.716359in}}{\pgfqpoint{1.968197in}{1.705760in}}{\pgfqpoint{1.968197in}{1.694710in}}%
\pgfpathcurveto{\pgfqpoint{1.968197in}{1.683660in}}{\pgfqpoint{1.972588in}{1.673061in}}{\pgfqpoint{1.980401in}{1.665247in}}%
\pgfpathcurveto{\pgfqpoint{1.988215in}{1.657434in}}{\pgfqpoint{1.998814in}{1.653043in}}{\pgfqpoint{2.009864in}{1.653043in}}%
\pgfpathlineto{\pgfqpoint{2.009864in}{1.653043in}}%
\pgfpathclose%
\pgfusepath{stroke}%
\end{pgfscope}%
\begin{pgfscope}%
\pgfpathrectangle{\pgfqpoint{0.847223in}{0.554012in}}{\pgfqpoint{6.200000in}{4.620000in}}%
\pgfusepath{clip}%
\pgfsetbuttcap%
\pgfsetroundjoin%
\pgfsetlinewidth{1.003750pt}%
\definecolor{currentstroke}{rgb}{1.000000,0.000000,0.000000}%
\pgfsetstrokecolor{currentstroke}%
\pgfsetdash{}{0pt}%
\pgfpathmoveto{\pgfqpoint{2.015197in}{1.648105in}}%
\pgfpathcurveto{\pgfqpoint{2.026247in}{1.648105in}}{\pgfqpoint{2.036846in}{1.652495in}}{\pgfqpoint{2.044660in}{1.660309in}}%
\pgfpathcurveto{\pgfqpoint{2.052474in}{1.668122in}}{\pgfqpoint{2.056864in}{1.678721in}}{\pgfqpoint{2.056864in}{1.689771in}}%
\pgfpathcurveto{\pgfqpoint{2.056864in}{1.700822in}}{\pgfqpoint{2.052474in}{1.711421in}}{\pgfqpoint{2.044660in}{1.719234in}}%
\pgfpathcurveto{\pgfqpoint{2.036846in}{1.727048in}}{\pgfqpoint{2.026247in}{1.731438in}}{\pgfqpoint{2.015197in}{1.731438in}}%
\pgfpathcurveto{\pgfqpoint{2.004147in}{1.731438in}}{\pgfqpoint{1.993548in}{1.727048in}}{\pgfqpoint{1.985735in}{1.719234in}}%
\pgfpathcurveto{\pgfqpoint{1.977921in}{1.711421in}}{\pgfqpoint{1.973531in}{1.700822in}}{\pgfqpoint{1.973531in}{1.689771in}}%
\pgfpathcurveto{\pgfqpoint{1.973531in}{1.678721in}}{\pgfqpoint{1.977921in}{1.668122in}}{\pgfqpoint{1.985735in}{1.660309in}}%
\pgfpathcurveto{\pgfqpoint{1.993548in}{1.652495in}}{\pgfqpoint{2.004147in}{1.648105in}}{\pgfqpoint{2.015197in}{1.648105in}}%
\pgfpathlineto{\pgfqpoint{2.015197in}{1.648105in}}%
\pgfpathclose%
\pgfusepath{stroke}%
\end{pgfscope}%
\begin{pgfscope}%
\pgfpathrectangle{\pgfqpoint{0.847223in}{0.554012in}}{\pgfqpoint{6.200000in}{4.620000in}}%
\pgfusepath{clip}%
\pgfsetbuttcap%
\pgfsetroundjoin%
\pgfsetlinewidth{1.003750pt}%
\definecolor{currentstroke}{rgb}{1.000000,0.000000,0.000000}%
\pgfsetstrokecolor{currentstroke}%
\pgfsetdash{}{0pt}%
\pgfpathmoveto{\pgfqpoint{2.020531in}{1.643197in}}%
\pgfpathcurveto{\pgfqpoint{2.031581in}{1.643197in}}{\pgfqpoint{2.042180in}{1.647588in}}{\pgfqpoint{2.049993in}{1.655401in}}%
\pgfpathcurveto{\pgfqpoint{2.057807in}{1.663215in}}{\pgfqpoint{2.062197in}{1.673814in}}{\pgfqpoint{2.062197in}{1.684864in}}%
\pgfpathcurveto{\pgfqpoint{2.062197in}{1.695914in}}{\pgfqpoint{2.057807in}{1.706513in}}{\pgfqpoint{2.049993in}{1.714327in}}%
\pgfpathcurveto{\pgfqpoint{2.042180in}{1.722140in}}{\pgfqpoint{2.031581in}{1.726531in}}{\pgfqpoint{2.020531in}{1.726531in}}%
\pgfpathcurveto{\pgfqpoint{2.009480in}{1.726531in}}{\pgfqpoint{1.998881in}{1.722140in}}{\pgfqpoint{1.991068in}{1.714327in}}%
\pgfpathcurveto{\pgfqpoint{1.983254in}{1.706513in}}{\pgfqpoint{1.978864in}{1.695914in}}{\pgfqpoint{1.978864in}{1.684864in}}%
\pgfpathcurveto{\pgfqpoint{1.978864in}{1.673814in}}{\pgfqpoint{1.983254in}{1.663215in}}{\pgfqpoint{1.991068in}{1.655401in}}%
\pgfpathcurveto{\pgfqpoint{1.998881in}{1.647588in}}{\pgfqpoint{2.009480in}{1.643197in}}{\pgfqpoint{2.020531in}{1.643197in}}%
\pgfpathlineto{\pgfqpoint{2.020531in}{1.643197in}}%
\pgfpathclose%
\pgfusepath{stroke}%
\end{pgfscope}%
\begin{pgfscope}%
\pgfpathrectangle{\pgfqpoint{0.847223in}{0.554012in}}{\pgfqpoint{6.200000in}{4.620000in}}%
\pgfusepath{clip}%
\pgfsetbuttcap%
\pgfsetroundjoin%
\pgfsetlinewidth{1.003750pt}%
\definecolor{currentstroke}{rgb}{1.000000,0.000000,0.000000}%
\pgfsetstrokecolor{currentstroke}%
\pgfsetdash{}{0pt}%
\pgfpathmoveto{\pgfqpoint{2.025864in}{1.638320in}}%
\pgfpathcurveto{\pgfqpoint{2.036914in}{1.638320in}}{\pgfqpoint{2.047513in}{1.642711in}}{\pgfqpoint{2.055326in}{1.650524in}}%
\pgfpathcurveto{\pgfqpoint{2.063140in}{1.658338in}}{\pgfqpoint{2.067530in}{1.668937in}}{\pgfqpoint{2.067530in}{1.679987in}}%
\pgfpathcurveto{\pgfqpoint{2.067530in}{1.691037in}}{\pgfqpoint{2.063140in}{1.701636in}}{\pgfqpoint{2.055326in}{1.709450in}}%
\pgfpathcurveto{\pgfqpoint{2.047513in}{1.717264in}}{\pgfqpoint{2.036914in}{1.721654in}}{\pgfqpoint{2.025864in}{1.721654in}}%
\pgfpathcurveto{\pgfqpoint{2.014814in}{1.721654in}}{\pgfqpoint{2.004215in}{1.717264in}}{\pgfqpoint{1.996401in}{1.709450in}}%
\pgfpathcurveto{\pgfqpoint{1.988587in}{1.701636in}}{\pgfqpoint{1.984197in}{1.691037in}}{\pgfqpoint{1.984197in}{1.679987in}}%
\pgfpathcurveto{\pgfqpoint{1.984197in}{1.668937in}}{\pgfqpoint{1.988587in}{1.658338in}}{\pgfqpoint{1.996401in}{1.650524in}}%
\pgfpathcurveto{\pgfqpoint{2.004215in}{1.642711in}}{\pgfqpoint{2.014814in}{1.638320in}}{\pgfqpoint{2.025864in}{1.638320in}}%
\pgfpathlineto{\pgfqpoint{2.025864in}{1.638320in}}%
\pgfpathclose%
\pgfusepath{stroke}%
\end{pgfscope}%
\begin{pgfscope}%
\pgfpathrectangle{\pgfqpoint{0.847223in}{0.554012in}}{\pgfqpoint{6.200000in}{4.620000in}}%
\pgfusepath{clip}%
\pgfsetbuttcap%
\pgfsetroundjoin%
\pgfsetlinewidth{1.003750pt}%
\definecolor{currentstroke}{rgb}{1.000000,0.000000,0.000000}%
\pgfsetstrokecolor{currentstroke}%
\pgfsetdash{}{0pt}%
\pgfpathmoveto{\pgfqpoint{2.031197in}{1.633474in}}%
\pgfpathcurveto{\pgfqpoint{2.042247in}{1.633474in}}{\pgfqpoint{2.052846in}{1.637864in}}{\pgfqpoint{2.060660in}{1.645678in}}%
\pgfpathcurveto{\pgfqpoint{2.068473in}{1.653492in}}{\pgfqpoint{2.072864in}{1.664091in}}{\pgfqpoint{2.072864in}{1.675141in}}%
\pgfpathcurveto{\pgfqpoint{2.072864in}{1.686191in}}{\pgfqpoint{2.068473in}{1.696790in}}{\pgfqpoint{2.060660in}{1.704604in}}%
\pgfpathcurveto{\pgfqpoint{2.052846in}{1.712417in}}{\pgfqpoint{2.042247in}{1.716808in}}{\pgfqpoint{2.031197in}{1.716808in}}%
\pgfpathcurveto{\pgfqpoint{2.020147in}{1.716808in}}{\pgfqpoint{2.009548in}{1.712417in}}{\pgfqpoint{2.001734in}{1.704604in}}%
\pgfpathcurveto{\pgfqpoint{1.993921in}{1.696790in}}{\pgfqpoint{1.989530in}{1.686191in}}{\pgfqpoint{1.989530in}{1.675141in}}%
\pgfpathcurveto{\pgfqpoint{1.989530in}{1.664091in}}{\pgfqpoint{1.993921in}{1.653492in}}{\pgfqpoint{2.001734in}{1.645678in}}%
\pgfpathcurveto{\pgfqpoint{2.009548in}{1.637864in}}{\pgfqpoint{2.020147in}{1.633474in}}{\pgfqpoint{2.031197in}{1.633474in}}%
\pgfpathlineto{\pgfqpoint{2.031197in}{1.633474in}}%
\pgfpathclose%
\pgfusepath{stroke}%
\end{pgfscope}%
\begin{pgfscope}%
\pgfpathrectangle{\pgfqpoint{0.847223in}{0.554012in}}{\pgfqpoint{6.200000in}{4.620000in}}%
\pgfusepath{clip}%
\pgfsetbuttcap%
\pgfsetroundjoin%
\pgfsetlinewidth{1.003750pt}%
\definecolor{currentstroke}{rgb}{1.000000,0.000000,0.000000}%
\pgfsetstrokecolor{currentstroke}%
\pgfsetdash{}{0pt}%
\pgfpathmoveto{\pgfqpoint{2.036530in}{1.628658in}}%
\pgfpathcurveto{\pgfqpoint{2.047580in}{1.628658in}}{\pgfqpoint{2.058179in}{1.633048in}}{\pgfqpoint{2.065993in}{1.640862in}}%
\pgfpathcurveto{\pgfqpoint{2.073807in}{1.648676in}}{\pgfqpoint{2.078197in}{1.659275in}}{\pgfqpoint{2.078197in}{1.670325in}}%
\pgfpathcurveto{\pgfqpoint{2.078197in}{1.681375in}}{\pgfqpoint{2.073807in}{1.691974in}}{\pgfqpoint{2.065993in}{1.699788in}}%
\pgfpathcurveto{\pgfqpoint{2.058179in}{1.707601in}}{\pgfqpoint{2.047580in}{1.711991in}}{\pgfqpoint{2.036530in}{1.711991in}}%
\pgfpathcurveto{\pgfqpoint{2.025480in}{1.711991in}}{\pgfqpoint{2.014881in}{1.707601in}}{\pgfqpoint{2.007067in}{1.699788in}}%
\pgfpathcurveto{\pgfqpoint{1.999254in}{1.691974in}}{\pgfqpoint{1.994863in}{1.681375in}}{\pgfqpoint{1.994863in}{1.670325in}}%
\pgfpathcurveto{\pgfqpoint{1.994863in}{1.659275in}}{\pgfqpoint{1.999254in}{1.648676in}}{\pgfqpoint{2.007067in}{1.640862in}}%
\pgfpathcurveto{\pgfqpoint{2.014881in}{1.633048in}}{\pgfqpoint{2.025480in}{1.628658in}}{\pgfqpoint{2.036530in}{1.628658in}}%
\pgfpathlineto{\pgfqpoint{2.036530in}{1.628658in}}%
\pgfpathclose%
\pgfusepath{stroke}%
\end{pgfscope}%
\begin{pgfscope}%
\pgfpathrectangle{\pgfqpoint{0.847223in}{0.554012in}}{\pgfqpoint{6.200000in}{4.620000in}}%
\pgfusepath{clip}%
\pgfsetbuttcap%
\pgfsetroundjoin%
\pgfsetlinewidth{1.003750pt}%
\definecolor{currentstroke}{rgb}{1.000000,0.000000,0.000000}%
\pgfsetstrokecolor{currentstroke}%
\pgfsetdash{}{0pt}%
\pgfpathmoveto{\pgfqpoint{2.041863in}{1.623872in}}%
\pgfpathcurveto{\pgfqpoint{2.052913in}{1.623872in}}{\pgfqpoint{2.063513in}{1.628262in}}{\pgfqpoint{2.071326in}{1.636076in}}%
\pgfpathcurveto{\pgfqpoint{2.079140in}{1.643889in}}{\pgfqpoint{2.083530in}{1.654488in}}{\pgfqpoint{2.083530in}{1.665539in}}%
\pgfpathcurveto{\pgfqpoint{2.083530in}{1.676589in}}{\pgfqpoint{2.079140in}{1.687188in}}{\pgfqpoint{2.071326in}{1.695001in}}%
\pgfpathcurveto{\pgfqpoint{2.063513in}{1.702815in}}{\pgfqpoint{2.052913in}{1.707205in}}{\pgfqpoint{2.041863in}{1.707205in}}%
\pgfpathcurveto{\pgfqpoint{2.030813in}{1.707205in}}{\pgfqpoint{2.020214in}{1.702815in}}{\pgfqpoint{2.012401in}{1.695001in}}%
\pgfpathcurveto{\pgfqpoint{2.004587in}{1.687188in}}{\pgfqpoint{2.000197in}{1.676589in}}{\pgfqpoint{2.000197in}{1.665539in}}%
\pgfpathcurveto{\pgfqpoint{2.000197in}{1.654488in}}{\pgfqpoint{2.004587in}{1.643889in}}{\pgfqpoint{2.012401in}{1.636076in}}%
\pgfpathcurveto{\pgfqpoint{2.020214in}{1.628262in}}{\pgfqpoint{2.030813in}{1.623872in}}{\pgfqpoint{2.041863in}{1.623872in}}%
\pgfpathlineto{\pgfqpoint{2.041863in}{1.623872in}}%
\pgfpathclose%
\pgfusepath{stroke}%
\end{pgfscope}%
\begin{pgfscope}%
\pgfpathrectangle{\pgfqpoint{0.847223in}{0.554012in}}{\pgfqpoint{6.200000in}{4.620000in}}%
\pgfusepath{clip}%
\pgfsetbuttcap%
\pgfsetroundjoin%
\pgfsetlinewidth{1.003750pt}%
\definecolor{currentstroke}{rgb}{1.000000,0.000000,0.000000}%
\pgfsetstrokecolor{currentstroke}%
\pgfsetdash{}{0pt}%
\pgfpathmoveto{\pgfqpoint{2.047197in}{1.619115in}}%
\pgfpathcurveto{\pgfqpoint{2.058247in}{1.619115in}}{\pgfqpoint{2.068846in}{1.623506in}}{\pgfqpoint{2.076659in}{1.631319in}}%
\pgfpathcurveto{\pgfqpoint{2.084473in}{1.639133in}}{\pgfqpoint{2.088863in}{1.649732in}}{\pgfqpoint{2.088863in}{1.660782in}}%
\pgfpathcurveto{\pgfqpoint{2.088863in}{1.671832in}}{\pgfqpoint{2.084473in}{1.682431in}}{\pgfqpoint{2.076659in}{1.690245in}}%
\pgfpathcurveto{\pgfqpoint{2.068846in}{1.698058in}}{\pgfqpoint{2.058247in}{1.702449in}}{\pgfqpoint{2.047197in}{1.702449in}}%
\pgfpathcurveto{\pgfqpoint{2.036146in}{1.702449in}}{\pgfqpoint{2.025547in}{1.698058in}}{\pgfqpoint{2.017734in}{1.690245in}}%
\pgfpathcurveto{\pgfqpoint{2.009920in}{1.682431in}}{\pgfqpoint{2.005530in}{1.671832in}}{\pgfqpoint{2.005530in}{1.660782in}}%
\pgfpathcurveto{\pgfqpoint{2.005530in}{1.649732in}}{\pgfqpoint{2.009920in}{1.639133in}}{\pgfqpoint{2.017734in}{1.631319in}}%
\pgfpathcurveto{\pgfqpoint{2.025547in}{1.623506in}}{\pgfqpoint{2.036146in}{1.619115in}}{\pgfqpoint{2.047197in}{1.619115in}}%
\pgfpathlineto{\pgfqpoint{2.047197in}{1.619115in}}%
\pgfpathclose%
\pgfusepath{stroke}%
\end{pgfscope}%
\begin{pgfscope}%
\pgfpathrectangle{\pgfqpoint{0.847223in}{0.554012in}}{\pgfqpoint{6.200000in}{4.620000in}}%
\pgfusepath{clip}%
\pgfsetbuttcap%
\pgfsetroundjoin%
\pgfsetlinewidth{1.003750pt}%
\definecolor{currentstroke}{rgb}{1.000000,0.000000,0.000000}%
\pgfsetstrokecolor{currentstroke}%
\pgfsetdash{}{0pt}%
\pgfpathmoveto{\pgfqpoint{2.052530in}{1.614388in}}%
\pgfpathcurveto{\pgfqpoint{2.063580in}{1.614388in}}{\pgfqpoint{2.074179in}{1.618778in}}{\pgfqpoint{2.081993in}{1.626592in}}%
\pgfpathcurveto{\pgfqpoint{2.089806in}{1.634406in}}{\pgfqpoint{2.094196in}{1.645005in}}{\pgfqpoint{2.094196in}{1.656055in}}%
\pgfpathcurveto{\pgfqpoint{2.094196in}{1.667105in}}{\pgfqpoint{2.089806in}{1.677704in}}{\pgfqpoint{2.081993in}{1.685517in}}%
\pgfpathcurveto{\pgfqpoint{2.074179in}{1.693331in}}{\pgfqpoint{2.063580in}{1.697721in}}{\pgfqpoint{2.052530in}{1.697721in}}%
\pgfpathcurveto{\pgfqpoint{2.041480in}{1.697721in}}{\pgfqpoint{2.030881in}{1.693331in}}{\pgfqpoint{2.023067in}{1.685517in}}%
\pgfpathcurveto{\pgfqpoint{2.015253in}{1.677704in}}{\pgfqpoint{2.010863in}{1.667105in}}{\pgfqpoint{2.010863in}{1.656055in}}%
\pgfpathcurveto{\pgfqpoint{2.010863in}{1.645005in}}{\pgfqpoint{2.015253in}{1.634406in}}{\pgfqpoint{2.023067in}{1.626592in}}%
\pgfpathcurveto{\pgfqpoint{2.030881in}{1.618778in}}{\pgfqpoint{2.041480in}{1.614388in}}{\pgfqpoint{2.052530in}{1.614388in}}%
\pgfpathlineto{\pgfqpoint{2.052530in}{1.614388in}}%
\pgfpathclose%
\pgfusepath{stroke}%
\end{pgfscope}%
\begin{pgfscope}%
\pgfpathrectangle{\pgfqpoint{0.847223in}{0.554012in}}{\pgfqpoint{6.200000in}{4.620000in}}%
\pgfusepath{clip}%
\pgfsetbuttcap%
\pgfsetroundjoin%
\pgfsetlinewidth{1.003750pt}%
\definecolor{currentstroke}{rgb}{1.000000,0.000000,0.000000}%
\pgfsetstrokecolor{currentstroke}%
\pgfsetdash{}{0pt}%
\pgfpathmoveto{\pgfqpoint{2.057863in}{1.609690in}}%
\pgfpathcurveto{\pgfqpoint{2.068913in}{1.609690in}}{\pgfqpoint{2.079512in}{1.614080in}}{\pgfqpoint{2.087326in}{1.621894in}}%
\pgfpathcurveto{\pgfqpoint{2.095139in}{1.629707in}}{\pgfqpoint{2.099530in}{1.640306in}}{\pgfqpoint{2.099530in}{1.651357in}}%
\pgfpathcurveto{\pgfqpoint{2.099530in}{1.662407in}}{\pgfqpoint{2.095139in}{1.673006in}}{\pgfqpoint{2.087326in}{1.680819in}}%
\pgfpathcurveto{\pgfqpoint{2.079512in}{1.688633in}}{\pgfqpoint{2.068913in}{1.693023in}}{\pgfqpoint{2.057863in}{1.693023in}}%
\pgfpathcurveto{\pgfqpoint{2.046813in}{1.693023in}}{\pgfqpoint{2.036214in}{1.688633in}}{\pgfqpoint{2.028400in}{1.680819in}}%
\pgfpathcurveto{\pgfqpoint{2.020587in}{1.673006in}}{\pgfqpoint{2.016196in}{1.662407in}}{\pgfqpoint{2.016196in}{1.651357in}}%
\pgfpathcurveto{\pgfqpoint{2.016196in}{1.640306in}}{\pgfqpoint{2.020587in}{1.629707in}}{\pgfqpoint{2.028400in}{1.621894in}}%
\pgfpathcurveto{\pgfqpoint{2.036214in}{1.614080in}}{\pgfqpoint{2.046813in}{1.609690in}}{\pgfqpoint{2.057863in}{1.609690in}}%
\pgfpathlineto{\pgfqpoint{2.057863in}{1.609690in}}%
\pgfpathclose%
\pgfusepath{stroke}%
\end{pgfscope}%
\begin{pgfscope}%
\pgfpathrectangle{\pgfqpoint{0.847223in}{0.554012in}}{\pgfqpoint{6.200000in}{4.620000in}}%
\pgfusepath{clip}%
\pgfsetbuttcap%
\pgfsetroundjoin%
\pgfsetlinewidth{1.003750pt}%
\definecolor{currentstroke}{rgb}{1.000000,0.000000,0.000000}%
\pgfsetstrokecolor{currentstroke}%
\pgfsetdash{}{0pt}%
\pgfpathmoveto{\pgfqpoint{2.063196in}{1.605021in}}%
\pgfpathcurveto{\pgfqpoint{2.074246in}{1.605021in}}{\pgfqpoint{2.084845in}{1.609411in}}{\pgfqpoint{2.092659in}{1.617224in}}%
\pgfpathcurveto{\pgfqpoint{2.100473in}{1.625038in}}{\pgfqpoint{2.104863in}{1.635637in}}{\pgfqpoint{2.104863in}{1.646687in}}%
\pgfpathcurveto{\pgfqpoint{2.104863in}{1.657737in}}{\pgfqpoint{2.100473in}{1.668336in}}{\pgfqpoint{2.092659in}{1.676150in}}%
\pgfpathcurveto{\pgfqpoint{2.084845in}{1.683964in}}{\pgfqpoint{2.074246in}{1.688354in}}{\pgfqpoint{2.063196in}{1.688354in}}%
\pgfpathcurveto{\pgfqpoint{2.052146in}{1.688354in}}{\pgfqpoint{2.041547in}{1.683964in}}{\pgfqpoint{2.033733in}{1.676150in}}%
\pgfpathcurveto{\pgfqpoint{2.025920in}{1.668336in}}{\pgfqpoint{2.021530in}{1.657737in}}{\pgfqpoint{2.021530in}{1.646687in}}%
\pgfpathcurveto{\pgfqpoint{2.021530in}{1.635637in}}{\pgfqpoint{2.025920in}{1.625038in}}{\pgfqpoint{2.033733in}{1.617224in}}%
\pgfpathcurveto{\pgfqpoint{2.041547in}{1.609411in}}{\pgfqpoint{2.052146in}{1.605021in}}{\pgfqpoint{2.063196in}{1.605021in}}%
\pgfpathlineto{\pgfqpoint{2.063196in}{1.605021in}}%
\pgfpathclose%
\pgfusepath{stroke}%
\end{pgfscope}%
\begin{pgfscope}%
\pgfpathrectangle{\pgfqpoint{0.847223in}{0.554012in}}{\pgfqpoint{6.200000in}{4.620000in}}%
\pgfusepath{clip}%
\pgfsetbuttcap%
\pgfsetroundjoin%
\pgfsetlinewidth{1.003750pt}%
\definecolor{currentstroke}{rgb}{1.000000,0.000000,0.000000}%
\pgfsetstrokecolor{currentstroke}%
\pgfsetdash{}{0pt}%
\pgfpathmoveto{\pgfqpoint{2.068529in}{1.600380in}}%
\pgfpathcurveto{\pgfqpoint{2.079580in}{1.600380in}}{\pgfqpoint{2.090179in}{1.604770in}}{\pgfqpoint{2.097992in}{1.612584in}}%
\pgfpathcurveto{\pgfqpoint{2.105806in}{1.620397in}}{\pgfqpoint{2.110196in}{1.630996in}}{\pgfqpoint{2.110196in}{1.642046in}}%
\pgfpathcurveto{\pgfqpoint{2.110196in}{1.653097in}}{\pgfqpoint{2.105806in}{1.663696in}}{\pgfqpoint{2.097992in}{1.671509in}}%
\pgfpathcurveto{\pgfqpoint{2.090179in}{1.679323in}}{\pgfqpoint{2.079580in}{1.683713in}}{\pgfqpoint{2.068529in}{1.683713in}}%
\pgfpathcurveto{\pgfqpoint{2.057479in}{1.683713in}}{\pgfqpoint{2.046880in}{1.679323in}}{\pgfqpoint{2.039067in}{1.671509in}}%
\pgfpathcurveto{\pgfqpoint{2.031253in}{1.663696in}}{\pgfqpoint{2.026863in}{1.653097in}}{\pgfqpoint{2.026863in}{1.642046in}}%
\pgfpathcurveto{\pgfqpoint{2.026863in}{1.630996in}}{\pgfqpoint{2.031253in}{1.620397in}}{\pgfqpoint{2.039067in}{1.612584in}}%
\pgfpathcurveto{\pgfqpoint{2.046880in}{1.604770in}}{\pgfqpoint{2.057479in}{1.600380in}}{\pgfqpoint{2.068529in}{1.600380in}}%
\pgfpathlineto{\pgfqpoint{2.068529in}{1.600380in}}%
\pgfpathclose%
\pgfusepath{stroke}%
\end{pgfscope}%
\begin{pgfscope}%
\pgfpathrectangle{\pgfqpoint{0.847223in}{0.554012in}}{\pgfqpoint{6.200000in}{4.620000in}}%
\pgfusepath{clip}%
\pgfsetbuttcap%
\pgfsetroundjoin%
\pgfsetlinewidth{1.003750pt}%
\definecolor{currentstroke}{rgb}{1.000000,0.000000,0.000000}%
\pgfsetstrokecolor{currentstroke}%
\pgfsetdash{}{0pt}%
\pgfpathmoveto{\pgfqpoint{2.073863in}{1.595767in}}%
\pgfpathcurveto{\pgfqpoint{2.084913in}{1.595767in}}{\pgfqpoint{2.095512in}{1.600157in}}{\pgfqpoint{2.103325in}{1.607971in}}%
\pgfpathcurveto{\pgfqpoint{2.111139in}{1.615785in}}{\pgfqpoint{2.115529in}{1.626384in}}{\pgfqpoint{2.115529in}{1.637434in}}%
\pgfpathcurveto{\pgfqpoint{2.115529in}{1.648484in}}{\pgfqpoint{2.111139in}{1.659083in}}{\pgfqpoint{2.103325in}{1.666897in}}%
\pgfpathcurveto{\pgfqpoint{2.095512in}{1.674710in}}{\pgfqpoint{2.084913in}{1.679101in}}{\pgfqpoint{2.073863in}{1.679101in}}%
\pgfpathcurveto{\pgfqpoint{2.062813in}{1.679101in}}{\pgfqpoint{2.052213in}{1.674710in}}{\pgfqpoint{2.044400in}{1.666897in}}%
\pgfpathcurveto{\pgfqpoint{2.036586in}{1.659083in}}{\pgfqpoint{2.032196in}{1.648484in}}{\pgfqpoint{2.032196in}{1.637434in}}%
\pgfpathcurveto{\pgfqpoint{2.032196in}{1.626384in}}{\pgfqpoint{2.036586in}{1.615785in}}{\pgfqpoint{2.044400in}{1.607971in}}%
\pgfpathcurveto{\pgfqpoint{2.052213in}{1.600157in}}{\pgfqpoint{2.062813in}{1.595767in}}{\pgfqpoint{2.073863in}{1.595767in}}%
\pgfpathlineto{\pgfqpoint{2.073863in}{1.595767in}}%
\pgfpathclose%
\pgfusepath{stroke}%
\end{pgfscope}%
\begin{pgfscope}%
\pgfpathrectangle{\pgfqpoint{0.847223in}{0.554012in}}{\pgfqpoint{6.200000in}{4.620000in}}%
\pgfusepath{clip}%
\pgfsetbuttcap%
\pgfsetroundjoin%
\pgfsetlinewidth{1.003750pt}%
\definecolor{currentstroke}{rgb}{1.000000,0.000000,0.000000}%
\pgfsetstrokecolor{currentstroke}%
\pgfsetdash{}{0pt}%
\pgfpathmoveto{\pgfqpoint{2.079196in}{1.591183in}}%
\pgfpathcurveto{\pgfqpoint{2.090246in}{1.591183in}}{\pgfqpoint{2.100845in}{1.595573in}}{\pgfqpoint{2.108659in}{1.603387in}}%
\pgfpathcurveto{\pgfqpoint{2.116472in}{1.611200in}}{\pgfqpoint{2.120863in}{1.621799in}}{\pgfqpoint{2.120863in}{1.632849in}}%
\pgfpathcurveto{\pgfqpoint{2.120863in}{1.643900in}}{\pgfqpoint{2.116472in}{1.654499in}}{\pgfqpoint{2.108659in}{1.662312in}}%
\pgfpathcurveto{\pgfqpoint{2.100845in}{1.670126in}}{\pgfqpoint{2.090246in}{1.674516in}}{\pgfqpoint{2.079196in}{1.674516in}}%
\pgfpathcurveto{\pgfqpoint{2.068146in}{1.674516in}}{\pgfqpoint{2.057547in}{1.670126in}}{\pgfqpoint{2.049733in}{1.662312in}}%
\pgfpathcurveto{\pgfqpoint{2.041919in}{1.654499in}}{\pgfqpoint{2.037529in}{1.643900in}}{\pgfqpoint{2.037529in}{1.632849in}}%
\pgfpathcurveto{\pgfqpoint{2.037529in}{1.621799in}}{\pgfqpoint{2.041919in}{1.611200in}}{\pgfqpoint{2.049733in}{1.603387in}}%
\pgfpathcurveto{\pgfqpoint{2.057547in}{1.595573in}}{\pgfqpoint{2.068146in}{1.591183in}}{\pgfqpoint{2.079196in}{1.591183in}}%
\pgfpathlineto{\pgfqpoint{2.079196in}{1.591183in}}%
\pgfpathclose%
\pgfusepath{stroke}%
\end{pgfscope}%
\begin{pgfscope}%
\pgfpathrectangle{\pgfqpoint{0.847223in}{0.554012in}}{\pgfqpoint{6.200000in}{4.620000in}}%
\pgfusepath{clip}%
\pgfsetbuttcap%
\pgfsetroundjoin%
\pgfsetlinewidth{1.003750pt}%
\definecolor{currentstroke}{rgb}{1.000000,0.000000,0.000000}%
\pgfsetstrokecolor{currentstroke}%
\pgfsetdash{}{0pt}%
\pgfpathmoveto{\pgfqpoint{2.084529in}{1.586626in}}%
\pgfpathcurveto{\pgfqpoint{2.095579in}{1.586626in}}{\pgfqpoint{2.106178in}{1.591016in}}{\pgfqpoint{2.113992in}{1.598830in}}%
\pgfpathcurveto{\pgfqpoint{2.121805in}{1.606644in}}{\pgfqpoint{2.126196in}{1.617243in}}{\pgfqpoint{2.126196in}{1.628293in}}%
\pgfpathcurveto{\pgfqpoint{2.126196in}{1.639343in}}{\pgfqpoint{2.121805in}{1.649942in}}{\pgfqpoint{2.113992in}{1.657755in}}%
\pgfpathcurveto{\pgfqpoint{2.106178in}{1.665569in}}{\pgfqpoint{2.095579in}{1.669959in}}{\pgfqpoint{2.084529in}{1.669959in}}%
\pgfpathcurveto{\pgfqpoint{2.073479in}{1.669959in}}{\pgfqpoint{2.062880in}{1.665569in}}{\pgfqpoint{2.055066in}{1.657755in}}%
\pgfpathcurveto{\pgfqpoint{2.047253in}{1.649942in}}{\pgfqpoint{2.042862in}{1.639343in}}{\pgfqpoint{2.042862in}{1.628293in}}%
\pgfpathcurveto{\pgfqpoint{2.042862in}{1.617243in}}{\pgfqpoint{2.047253in}{1.606644in}}{\pgfqpoint{2.055066in}{1.598830in}}%
\pgfpathcurveto{\pgfqpoint{2.062880in}{1.591016in}}{\pgfqpoint{2.073479in}{1.586626in}}{\pgfqpoint{2.084529in}{1.586626in}}%
\pgfpathlineto{\pgfqpoint{2.084529in}{1.586626in}}%
\pgfpathclose%
\pgfusepath{stroke}%
\end{pgfscope}%
\begin{pgfscope}%
\pgfpathrectangle{\pgfqpoint{0.847223in}{0.554012in}}{\pgfqpoint{6.200000in}{4.620000in}}%
\pgfusepath{clip}%
\pgfsetbuttcap%
\pgfsetroundjoin%
\pgfsetlinewidth{1.003750pt}%
\definecolor{currentstroke}{rgb}{1.000000,0.000000,0.000000}%
\pgfsetstrokecolor{currentstroke}%
\pgfsetdash{}{0pt}%
\pgfpathmoveto{\pgfqpoint{2.089862in}{1.582097in}}%
\pgfpathcurveto{\pgfqpoint{2.100912in}{1.582097in}}{\pgfqpoint{2.111511in}{1.586487in}}{\pgfqpoint{2.119325in}{1.594301in}}%
\pgfpathcurveto{\pgfqpoint{2.127139in}{1.602114in}}{\pgfqpoint{2.131529in}{1.612713in}}{\pgfqpoint{2.131529in}{1.623763in}}%
\pgfpathcurveto{\pgfqpoint{2.131529in}{1.634814in}}{\pgfqpoint{2.127139in}{1.645413in}}{\pgfqpoint{2.119325in}{1.653226in}}%
\pgfpathcurveto{\pgfqpoint{2.111511in}{1.661040in}}{\pgfqpoint{2.100912in}{1.665430in}}{\pgfqpoint{2.089862in}{1.665430in}}%
\pgfpathcurveto{\pgfqpoint{2.078812in}{1.665430in}}{\pgfqpoint{2.068213in}{1.661040in}}{\pgfqpoint{2.060400in}{1.653226in}}%
\pgfpathcurveto{\pgfqpoint{2.052586in}{1.645413in}}{\pgfqpoint{2.048196in}{1.634814in}}{\pgfqpoint{2.048196in}{1.623763in}}%
\pgfpathcurveto{\pgfqpoint{2.048196in}{1.612713in}}{\pgfqpoint{2.052586in}{1.602114in}}{\pgfqpoint{2.060400in}{1.594301in}}%
\pgfpathcurveto{\pgfqpoint{2.068213in}{1.586487in}}{\pgfqpoint{2.078812in}{1.582097in}}{\pgfqpoint{2.089862in}{1.582097in}}%
\pgfpathlineto{\pgfqpoint{2.089862in}{1.582097in}}%
\pgfpathclose%
\pgfusepath{stroke}%
\end{pgfscope}%
\begin{pgfscope}%
\pgfpathrectangle{\pgfqpoint{0.847223in}{0.554012in}}{\pgfqpoint{6.200000in}{4.620000in}}%
\pgfusepath{clip}%
\pgfsetbuttcap%
\pgfsetroundjoin%
\pgfsetlinewidth{1.003750pt}%
\definecolor{currentstroke}{rgb}{1.000000,0.000000,0.000000}%
\pgfsetstrokecolor{currentstroke}%
\pgfsetdash{}{0pt}%
\pgfpathmoveto{\pgfqpoint{2.095196in}{1.577595in}}%
\pgfpathcurveto{\pgfqpoint{2.106246in}{1.577595in}}{\pgfqpoint{2.116845in}{1.581985in}}{\pgfqpoint{2.124658in}{1.589799in}}%
\pgfpathcurveto{\pgfqpoint{2.132472in}{1.597612in}}{\pgfqpoint{2.136862in}{1.608211in}}{\pgfqpoint{2.136862in}{1.619262in}}%
\pgfpathcurveto{\pgfqpoint{2.136862in}{1.630312in}}{\pgfqpoint{2.132472in}{1.640911in}}{\pgfqpoint{2.124658in}{1.648724in}}%
\pgfpathcurveto{\pgfqpoint{2.116845in}{1.656538in}}{\pgfqpoint{2.106246in}{1.660928in}}{\pgfqpoint{2.095196in}{1.660928in}}%
\pgfpathcurveto{\pgfqpoint{2.084145in}{1.660928in}}{\pgfqpoint{2.073546in}{1.656538in}}{\pgfqpoint{2.065733in}{1.648724in}}%
\pgfpathcurveto{\pgfqpoint{2.057919in}{1.640911in}}{\pgfqpoint{2.053529in}{1.630312in}}{\pgfqpoint{2.053529in}{1.619262in}}%
\pgfpathcurveto{\pgfqpoint{2.053529in}{1.608211in}}{\pgfqpoint{2.057919in}{1.597612in}}{\pgfqpoint{2.065733in}{1.589799in}}%
\pgfpathcurveto{\pgfqpoint{2.073546in}{1.581985in}}{\pgfqpoint{2.084145in}{1.577595in}}{\pgfqpoint{2.095196in}{1.577595in}}%
\pgfpathlineto{\pgfqpoint{2.095196in}{1.577595in}}%
\pgfpathclose%
\pgfusepath{stroke}%
\end{pgfscope}%
\begin{pgfscope}%
\pgfpathrectangle{\pgfqpoint{0.847223in}{0.554012in}}{\pgfqpoint{6.200000in}{4.620000in}}%
\pgfusepath{clip}%
\pgfsetbuttcap%
\pgfsetroundjoin%
\pgfsetlinewidth{1.003750pt}%
\definecolor{currentstroke}{rgb}{1.000000,0.000000,0.000000}%
\pgfsetstrokecolor{currentstroke}%
\pgfsetdash{}{0pt}%
\pgfpathmoveto{\pgfqpoint{2.100529in}{1.573120in}}%
\pgfpathcurveto{\pgfqpoint{2.111579in}{1.573120in}}{\pgfqpoint{2.122178in}{1.577510in}}{\pgfqpoint{2.129992in}{1.585324in}}%
\pgfpathcurveto{\pgfqpoint{2.137805in}{1.593138in}}{\pgfqpoint{2.142195in}{1.603737in}}{\pgfqpoint{2.142195in}{1.614787in}}%
\pgfpathcurveto{\pgfqpoint{2.142195in}{1.625837in}}{\pgfqpoint{2.137805in}{1.636436in}}{\pgfqpoint{2.129992in}{1.644249in}}%
\pgfpathcurveto{\pgfqpoint{2.122178in}{1.652063in}}{\pgfqpoint{2.111579in}{1.656453in}}{\pgfqpoint{2.100529in}{1.656453in}}%
\pgfpathcurveto{\pgfqpoint{2.089479in}{1.656453in}}{\pgfqpoint{2.078880in}{1.652063in}}{\pgfqpoint{2.071066in}{1.644249in}}%
\pgfpathcurveto{\pgfqpoint{2.063252in}{1.636436in}}{\pgfqpoint{2.058862in}{1.625837in}}{\pgfqpoint{2.058862in}{1.614787in}}%
\pgfpathcurveto{\pgfqpoint{2.058862in}{1.603737in}}{\pgfqpoint{2.063252in}{1.593138in}}{\pgfqpoint{2.071066in}{1.585324in}}%
\pgfpathcurveto{\pgfqpoint{2.078880in}{1.577510in}}{\pgfqpoint{2.089479in}{1.573120in}}{\pgfqpoint{2.100529in}{1.573120in}}%
\pgfpathlineto{\pgfqpoint{2.100529in}{1.573120in}}%
\pgfpathclose%
\pgfusepath{stroke}%
\end{pgfscope}%
\begin{pgfscope}%
\pgfpathrectangle{\pgfqpoint{0.847223in}{0.554012in}}{\pgfqpoint{6.200000in}{4.620000in}}%
\pgfusepath{clip}%
\pgfsetbuttcap%
\pgfsetroundjoin%
\pgfsetlinewidth{1.003750pt}%
\definecolor{currentstroke}{rgb}{1.000000,0.000000,0.000000}%
\pgfsetstrokecolor{currentstroke}%
\pgfsetdash{}{0pt}%
\pgfpathmoveto{\pgfqpoint{2.105862in}{1.568672in}}%
\pgfpathcurveto{\pgfqpoint{2.116912in}{1.568672in}}{\pgfqpoint{2.127511in}{1.573062in}}{\pgfqpoint{2.135325in}{1.580876in}}%
\pgfpathcurveto{\pgfqpoint{2.143138in}{1.588689in}}{\pgfqpoint{2.147529in}{1.599288in}}{\pgfqpoint{2.147529in}{1.610339in}}%
\pgfpathcurveto{\pgfqpoint{2.147529in}{1.621389in}}{\pgfqpoint{2.143138in}{1.631988in}}{\pgfqpoint{2.135325in}{1.639801in}}%
\pgfpathcurveto{\pgfqpoint{2.127511in}{1.647615in}}{\pgfqpoint{2.116912in}{1.652005in}}{\pgfqpoint{2.105862in}{1.652005in}}%
\pgfpathcurveto{\pgfqpoint{2.094812in}{1.652005in}}{\pgfqpoint{2.084213in}{1.647615in}}{\pgfqpoint{2.076399in}{1.639801in}}%
\pgfpathcurveto{\pgfqpoint{2.068586in}{1.631988in}}{\pgfqpoint{2.064195in}{1.621389in}}{\pgfqpoint{2.064195in}{1.610339in}}%
\pgfpathcurveto{\pgfqpoint{2.064195in}{1.599288in}}{\pgfqpoint{2.068586in}{1.588689in}}{\pgfqpoint{2.076399in}{1.580876in}}%
\pgfpathcurveto{\pgfqpoint{2.084213in}{1.573062in}}{\pgfqpoint{2.094812in}{1.568672in}}{\pgfqpoint{2.105862in}{1.568672in}}%
\pgfpathlineto{\pgfqpoint{2.105862in}{1.568672in}}%
\pgfpathclose%
\pgfusepath{stroke}%
\end{pgfscope}%
\begin{pgfscope}%
\pgfpathrectangle{\pgfqpoint{0.847223in}{0.554012in}}{\pgfqpoint{6.200000in}{4.620000in}}%
\pgfusepath{clip}%
\pgfsetbuttcap%
\pgfsetroundjoin%
\pgfsetlinewidth{1.003750pt}%
\definecolor{currentstroke}{rgb}{1.000000,0.000000,0.000000}%
\pgfsetstrokecolor{currentstroke}%
\pgfsetdash{}{0pt}%
\pgfpathmoveto{\pgfqpoint{2.111195in}{1.564250in}}%
\pgfpathcurveto{\pgfqpoint{2.122245in}{1.564250in}}{\pgfqpoint{2.132844in}{1.568641in}}{\pgfqpoint{2.140658in}{1.576454in}}%
\pgfpathcurveto{\pgfqpoint{2.148472in}{1.584268in}}{\pgfqpoint{2.152862in}{1.594867in}}{\pgfqpoint{2.152862in}{1.605917in}}%
\pgfpathcurveto{\pgfqpoint{2.152862in}{1.616967in}}{\pgfqpoint{2.148472in}{1.627566in}}{\pgfqpoint{2.140658in}{1.635380in}}%
\pgfpathcurveto{\pgfqpoint{2.132844in}{1.643193in}}{\pgfqpoint{2.122245in}{1.647584in}}{\pgfqpoint{2.111195in}{1.647584in}}%
\pgfpathcurveto{\pgfqpoint{2.100145in}{1.647584in}}{\pgfqpoint{2.089546in}{1.643193in}}{\pgfqpoint{2.081732in}{1.635380in}}%
\pgfpathcurveto{\pgfqpoint{2.073919in}{1.627566in}}{\pgfqpoint{2.069529in}{1.616967in}}{\pgfqpoint{2.069529in}{1.605917in}}%
\pgfpathcurveto{\pgfqpoint{2.069529in}{1.594867in}}{\pgfqpoint{2.073919in}{1.584268in}}{\pgfqpoint{2.081732in}{1.576454in}}%
\pgfpathcurveto{\pgfqpoint{2.089546in}{1.568641in}}{\pgfqpoint{2.100145in}{1.564250in}}{\pgfqpoint{2.111195in}{1.564250in}}%
\pgfpathlineto{\pgfqpoint{2.111195in}{1.564250in}}%
\pgfpathclose%
\pgfusepath{stroke}%
\end{pgfscope}%
\begin{pgfscope}%
\pgfpathrectangle{\pgfqpoint{0.847223in}{0.554012in}}{\pgfqpoint{6.200000in}{4.620000in}}%
\pgfusepath{clip}%
\pgfsetbuttcap%
\pgfsetroundjoin%
\pgfsetlinewidth{1.003750pt}%
\definecolor{currentstroke}{rgb}{1.000000,0.000000,0.000000}%
\pgfsetstrokecolor{currentstroke}%
\pgfsetdash{}{0pt}%
\pgfpathmoveto{\pgfqpoint{2.116528in}{1.559855in}}%
\pgfpathcurveto{\pgfqpoint{2.127579in}{1.559855in}}{\pgfqpoint{2.138178in}{1.564245in}}{\pgfqpoint{2.145991in}{1.572059in}}%
\pgfpathcurveto{\pgfqpoint{2.153805in}{1.579873in}}{\pgfqpoint{2.158195in}{1.590472in}}{\pgfqpoint{2.158195in}{1.601522in}}%
\pgfpathcurveto{\pgfqpoint{2.158195in}{1.612572in}}{\pgfqpoint{2.153805in}{1.623171in}}{\pgfqpoint{2.145991in}{1.630985in}}%
\pgfpathcurveto{\pgfqpoint{2.138178in}{1.638798in}}{\pgfqpoint{2.127579in}{1.643188in}}{\pgfqpoint{2.116528in}{1.643188in}}%
\pgfpathcurveto{\pgfqpoint{2.105478in}{1.643188in}}{\pgfqpoint{2.094879in}{1.638798in}}{\pgfqpoint{2.087066in}{1.630985in}}%
\pgfpathcurveto{\pgfqpoint{2.079252in}{1.623171in}}{\pgfqpoint{2.074862in}{1.612572in}}{\pgfqpoint{2.074862in}{1.601522in}}%
\pgfpathcurveto{\pgfqpoint{2.074862in}{1.590472in}}{\pgfqpoint{2.079252in}{1.579873in}}{\pgfqpoint{2.087066in}{1.572059in}}%
\pgfpathcurveto{\pgfqpoint{2.094879in}{1.564245in}}{\pgfqpoint{2.105478in}{1.559855in}}{\pgfqpoint{2.116528in}{1.559855in}}%
\pgfpathlineto{\pgfqpoint{2.116528in}{1.559855in}}%
\pgfpathclose%
\pgfusepath{stroke}%
\end{pgfscope}%
\begin{pgfscope}%
\pgfpathrectangle{\pgfqpoint{0.847223in}{0.554012in}}{\pgfqpoint{6.200000in}{4.620000in}}%
\pgfusepath{clip}%
\pgfsetbuttcap%
\pgfsetroundjoin%
\pgfsetlinewidth{1.003750pt}%
\definecolor{currentstroke}{rgb}{1.000000,0.000000,0.000000}%
\pgfsetstrokecolor{currentstroke}%
\pgfsetdash{}{0pt}%
\pgfpathmoveto{\pgfqpoint{2.121862in}{1.555486in}}%
\pgfpathcurveto{\pgfqpoint{2.132912in}{1.555486in}}{\pgfqpoint{2.143511in}{1.559876in}}{\pgfqpoint{2.151324in}{1.567690in}}%
\pgfpathcurveto{\pgfqpoint{2.159138in}{1.575503in}}{\pgfqpoint{2.163528in}{1.586102in}}{\pgfqpoint{2.163528in}{1.597153in}}%
\pgfpathcurveto{\pgfqpoint{2.163528in}{1.608203in}}{\pgfqpoint{2.159138in}{1.618802in}}{\pgfqpoint{2.151324in}{1.626615in}}%
\pgfpathcurveto{\pgfqpoint{2.143511in}{1.634429in}}{\pgfqpoint{2.132912in}{1.638819in}}{\pgfqpoint{2.121862in}{1.638819in}}%
\pgfpathcurveto{\pgfqpoint{2.110811in}{1.638819in}}{\pgfqpoint{2.100212in}{1.634429in}}{\pgfqpoint{2.092399in}{1.626615in}}%
\pgfpathcurveto{\pgfqpoint{2.084585in}{1.618802in}}{\pgfqpoint{2.080195in}{1.608203in}}{\pgfqpoint{2.080195in}{1.597153in}}%
\pgfpathcurveto{\pgfqpoint{2.080195in}{1.586102in}}{\pgfqpoint{2.084585in}{1.575503in}}{\pgfqpoint{2.092399in}{1.567690in}}%
\pgfpathcurveto{\pgfqpoint{2.100212in}{1.559876in}}{\pgfqpoint{2.110811in}{1.555486in}}{\pgfqpoint{2.121862in}{1.555486in}}%
\pgfpathlineto{\pgfqpoint{2.121862in}{1.555486in}}%
\pgfpathclose%
\pgfusepath{stroke}%
\end{pgfscope}%
\begin{pgfscope}%
\pgfpathrectangle{\pgfqpoint{0.847223in}{0.554012in}}{\pgfqpoint{6.200000in}{4.620000in}}%
\pgfusepath{clip}%
\pgfsetbuttcap%
\pgfsetroundjoin%
\pgfsetlinewidth{1.003750pt}%
\definecolor{currentstroke}{rgb}{1.000000,0.000000,0.000000}%
\pgfsetstrokecolor{currentstroke}%
\pgfsetdash{}{0pt}%
\pgfpathmoveto{\pgfqpoint{2.127195in}{1.551143in}}%
\pgfpathcurveto{\pgfqpoint{2.138245in}{1.551143in}}{\pgfqpoint{2.148844in}{1.555533in}}{\pgfqpoint{2.156658in}{1.563346in}}%
\pgfpathcurveto{\pgfqpoint{2.164471in}{1.571160in}}{\pgfqpoint{2.168861in}{1.581759in}}{\pgfqpoint{2.168861in}{1.592809in}}%
\pgfpathcurveto{\pgfqpoint{2.168861in}{1.603859in}}{\pgfqpoint{2.164471in}{1.614458in}}{\pgfqpoint{2.156658in}{1.622272in}}%
\pgfpathcurveto{\pgfqpoint{2.148844in}{1.630086in}}{\pgfqpoint{2.138245in}{1.634476in}}{\pgfqpoint{2.127195in}{1.634476in}}%
\pgfpathcurveto{\pgfqpoint{2.116145in}{1.634476in}}{\pgfqpoint{2.105546in}{1.630086in}}{\pgfqpoint{2.097732in}{1.622272in}}%
\pgfpathcurveto{\pgfqpoint{2.089918in}{1.614458in}}{\pgfqpoint{2.085528in}{1.603859in}}{\pgfqpoint{2.085528in}{1.592809in}}%
\pgfpathcurveto{\pgfqpoint{2.085528in}{1.581759in}}{\pgfqpoint{2.089918in}{1.571160in}}{\pgfqpoint{2.097732in}{1.563346in}}%
\pgfpathcurveto{\pgfqpoint{2.105546in}{1.555533in}}{\pgfqpoint{2.116145in}{1.551143in}}{\pgfqpoint{2.127195in}{1.551143in}}%
\pgfpathlineto{\pgfqpoint{2.127195in}{1.551143in}}%
\pgfpathclose%
\pgfusepath{stroke}%
\end{pgfscope}%
\begin{pgfscope}%
\pgfpathrectangle{\pgfqpoint{0.847223in}{0.554012in}}{\pgfqpoint{6.200000in}{4.620000in}}%
\pgfusepath{clip}%
\pgfsetbuttcap%
\pgfsetroundjoin%
\pgfsetlinewidth{1.003750pt}%
\definecolor{currentstroke}{rgb}{1.000000,0.000000,0.000000}%
\pgfsetstrokecolor{currentstroke}%
\pgfsetdash{}{0pt}%
\pgfpathmoveto{\pgfqpoint{2.132528in}{1.546825in}}%
\pgfpathcurveto{\pgfqpoint{2.143578in}{1.546825in}}{\pgfqpoint{2.154177in}{1.551215in}}{\pgfqpoint{2.161991in}{1.559029in}}%
\pgfpathcurveto{\pgfqpoint{2.169804in}{1.566842in}}{\pgfqpoint{2.174195in}{1.577441in}}{\pgfqpoint{2.174195in}{1.588492in}}%
\pgfpathcurveto{\pgfqpoint{2.174195in}{1.599542in}}{\pgfqpoint{2.169804in}{1.610141in}}{\pgfqpoint{2.161991in}{1.617954in}}%
\pgfpathcurveto{\pgfqpoint{2.154177in}{1.625768in}}{\pgfqpoint{2.143578in}{1.630158in}}{\pgfqpoint{2.132528in}{1.630158in}}%
\pgfpathcurveto{\pgfqpoint{2.121478in}{1.630158in}}{\pgfqpoint{2.110879in}{1.625768in}}{\pgfqpoint{2.103065in}{1.617954in}}%
\pgfpathcurveto{\pgfqpoint{2.095252in}{1.610141in}}{\pgfqpoint{2.090861in}{1.599542in}}{\pgfqpoint{2.090861in}{1.588492in}}%
\pgfpathcurveto{\pgfqpoint{2.090861in}{1.577441in}}{\pgfqpoint{2.095252in}{1.566842in}}{\pgfqpoint{2.103065in}{1.559029in}}%
\pgfpathcurveto{\pgfqpoint{2.110879in}{1.551215in}}{\pgfqpoint{2.121478in}{1.546825in}}{\pgfqpoint{2.132528in}{1.546825in}}%
\pgfpathlineto{\pgfqpoint{2.132528in}{1.546825in}}%
\pgfpathclose%
\pgfusepath{stroke}%
\end{pgfscope}%
\begin{pgfscope}%
\pgfpathrectangle{\pgfqpoint{0.847223in}{0.554012in}}{\pgfqpoint{6.200000in}{4.620000in}}%
\pgfusepath{clip}%
\pgfsetbuttcap%
\pgfsetroundjoin%
\pgfsetlinewidth{1.003750pt}%
\definecolor{currentstroke}{rgb}{1.000000,0.000000,0.000000}%
\pgfsetstrokecolor{currentstroke}%
\pgfsetdash{}{0pt}%
\pgfpathmoveto{\pgfqpoint{2.137861in}{1.542533in}}%
\pgfpathcurveto{\pgfqpoint{2.148911in}{1.542533in}}{\pgfqpoint{2.159510in}{1.546923in}}{\pgfqpoint{2.167324in}{1.554736in}}%
\pgfpathcurveto{\pgfqpoint{2.175138in}{1.562550in}}{\pgfqpoint{2.179528in}{1.573149in}}{\pgfqpoint{2.179528in}{1.584199in}}%
\pgfpathcurveto{\pgfqpoint{2.179528in}{1.595249in}}{\pgfqpoint{2.175138in}{1.605848in}}{\pgfqpoint{2.167324in}{1.613662in}}%
\pgfpathcurveto{\pgfqpoint{2.159510in}{1.621476in}}{\pgfqpoint{2.148911in}{1.625866in}}{\pgfqpoint{2.137861in}{1.625866in}}%
\pgfpathcurveto{\pgfqpoint{2.126811in}{1.625866in}}{\pgfqpoint{2.116212in}{1.621476in}}{\pgfqpoint{2.108398in}{1.613662in}}%
\pgfpathcurveto{\pgfqpoint{2.100585in}{1.605848in}}{\pgfqpoint{2.096195in}{1.595249in}}{\pgfqpoint{2.096195in}{1.584199in}}%
\pgfpathcurveto{\pgfqpoint{2.096195in}{1.573149in}}{\pgfqpoint{2.100585in}{1.562550in}}{\pgfqpoint{2.108398in}{1.554736in}}%
\pgfpathcurveto{\pgfqpoint{2.116212in}{1.546923in}}{\pgfqpoint{2.126811in}{1.542533in}}{\pgfqpoint{2.137861in}{1.542533in}}%
\pgfpathlineto{\pgfqpoint{2.137861in}{1.542533in}}%
\pgfpathclose%
\pgfusepath{stroke}%
\end{pgfscope}%
\begin{pgfscope}%
\pgfpathrectangle{\pgfqpoint{0.847223in}{0.554012in}}{\pgfqpoint{6.200000in}{4.620000in}}%
\pgfusepath{clip}%
\pgfsetbuttcap%
\pgfsetroundjoin%
\pgfsetlinewidth{1.003750pt}%
\definecolor{currentstroke}{rgb}{1.000000,0.000000,0.000000}%
\pgfsetstrokecolor{currentstroke}%
\pgfsetdash{}{0pt}%
\pgfpathmoveto{\pgfqpoint{2.143194in}{1.538265in}}%
\pgfpathcurveto{\pgfqpoint{2.154245in}{1.538265in}}{\pgfqpoint{2.164844in}{1.542656in}}{\pgfqpoint{2.172657in}{1.550469in}}%
\pgfpathcurveto{\pgfqpoint{2.180471in}{1.558283in}}{\pgfqpoint{2.184861in}{1.568882in}}{\pgfqpoint{2.184861in}{1.579932in}}%
\pgfpathcurveto{\pgfqpoint{2.184861in}{1.590982in}}{\pgfqpoint{2.180471in}{1.601581in}}{\pgfqpoint{2.172657in}{1.609395in}}%
\pgfpathcurveto{\pgfqpoint{2.164844in}{1.617208in}}{\pgfqpoint{2.154245in}{1.621599in}}{\pgfqpoint{2.143194in}{1.621599in}}%
\pgfpathcurveto{\pgfqpoint{2.132144in}{1.621599in}}{\pgfqpoint{2.121545in}{1.617208in}}{\pgfqpoint{2.113732in}{1.609395in}}%
\pgfpathcurveto{\pgfqpoint{2.105918in}{1.601581in}}{\pgfqpoint{2.101528in}{1.590982in}}{\pgfqpoint{2.101528in}{1.579932in}}%
\pgfpathcurveto{\pgfqpoint{2.101528in}{1.568882in}}{\pgfqpoint{2.105918in}{1.558283in}}{\pgfqpoint{2.113732in}{1.550469in}}%
\pgfpathcurveto{\pgfqpoint{2.121545in}{1.542656in}}{\pgfqpoint{2.132144in}{1.538265in}}{\pgfqpoint{2.143194in}{1.538265in}}%
\pgfpathlineto{\pgfqpoint{2.143194in}{1.538265in}}%
\pgfpathclose%
\pgfusepath{stroke}%
\end{pgfscope}%
\begin{pgfscope}%
\pgfpathrectangle{\pgfqpoint{0.847223in}{0.554012in}}{\pgfqpoint{6.200000in}{4.620000in}}%
\pgfusepath{clip}%
\pgfsetbuttcap%
\pgfsetroundjoin%
\pgfsetlinewidth{1.003750pt}%
\definecolor{currentstroke}{rgb}{1.000000,0.000000,0.000000}%
\pgfsetstrokecolor{currentstroke}%
\pgfsetdash{}{0pt}%
\pgfpathmoveto{\pgfqpoint{2.148528in}{1.534023in}}%
\pgfpathcurveto{\pgfqpoint{2.159578in}{1.534023in}}{\pgfqpoint{2.170177in}{1.538413in}}{\pgfqpoint{2.177990in}{1.546227in}}%
\pgfpathcurveto{\pgfqpoint{2.185804in}{1.554041in}}{\pgfqpoint{2.190194in}{1.564640in}}{\pgfqpoint{2.190194in}{1.575690in}}%
\pgfpathcurveto{\pgfqpoint{2.190194in}{1.586740in}}{\pgfqpoint{2.185804in}{1.597339in}}{\pgfqpoint{2.177990in}{1.605153in}}%
\pgfpathcurveto{\pgfqpoint{2.170177in}{1.612966in}}{\pgfqpoint{2.159578in}{1.617356in}}{\pgfqpoint{2.148528in}{1.617356in}}%
\pgfpathcurveto{\pgfqpoint{2.137478in}{1.617356in}}{\pgfqpoint{2.126879in}{1.612966in}}{\pgfqpoint{2.119065in}{1.605153in}}%
\pgfpathcurveto{\pgfqpoint{2.111251in}{1.597339in}}{\pgfqpoint{2.106861in}{1.586740in}}{\pgfqpoint{2.106861in}{1.575690in}}%
\pgfpathcurveto{\pgfqpoint{2.106861in}{1.564640in}}{\pgfqpoint{2.111251in}{1.554041in}}{\pgfqpoint{2.119065in}{1.546227in}}%
\pgfpathcurveto{\pgfqpoint{2.126879in}{1.538413in}}{\pgfqpoint{2.137478in}{1.534023in}}{\pgfqpoint{2.148528in}{1.534023in}}%
\pgfpathlineto{\pgfqpoint{2.148528in}{1.534023in}}%
\pgfpathclose%
\pgfusepath{stroke}%
\end{pgfscope}%
\begin{pgfscope}%
\pgfpathrectangle{\pgfqpoint{0.847223in}{0.554012in}}{\pgfqpoint{6.200000in}{4.620000in}}%
\pgfusepath{clip}%
\pgfsetbuttcap%
\pgfsetroundjoin%
\pgfsetlinewidth{1.003750pt}%
\definecolor{currentstroke}{rgb}{1.000000,0.000000,0.000000}%
\pgfsetstrokecolor{currentstroke}%
\pgfsetdash{}{0pt}%
\pgfpathmoveto{\pgfqpoint{2.153861in}{1.529806in}}%
\pgfpathcurveto{\pgfqpoint{2.164911in}{1.529806in}}{\pgfqpoint{2.175510in}{1.534196in}}{\pgfqpoint{2.183324in}{1.542010in}}%
\pgfpathcurveto{\pgfqpoint{2.191137in}{1.549823in}}{\pgfqpoint{2.195528in}{1.560422in}}{\pgfqpoint{2.195528in}{1.571472in}}%
\pgfpathcurveto{\pgfqpoint{2.195528in}{1.582522in}}{\pgfqpoint{2.191137in}{1.593121in}}{\pgfqpoint{2.183324in}{1.600935in}}%
\pgfpathcurveto{\pgfqpoint{2.175510in}{1.608749in}}{\pgfqpoint{2.164911in}{1.613139in}}{\pgfqpoint{2.153861in}{1.613139in}}%
\pgfpathcurveto{\pgfqpoint{2.142811in}{1.613139in}}{\pgfqpoint{2.132212in}{1.608749in}}{\pgfqpoint{2.124398in}{1.600935in}}%
\pgfpathcurveto{\pgfqpoint{2.116584in}{1.593121in}}{\pgfqpoint{2.112194in}{1.582522in}}{\pgfqpoint{2.112194in}{1.571472in}}%
\pgfpathcurveto{\pgfqpoint{2.112194in}{1.560422in}}{\pgfqpoint{2.116584in}{1.549823in}}{\pgfqpoint{2.124398in}{1.542010in}}%
\pgfpathcurveto{\pgfqpoint{2.132212in}{1.534196in}}{\pgfqpoint{2.142811in}{1.529806in}}{\pgfqpoint{2.153861in}{1.529806in}}%
\pgfpathlineto{\pgfqpoint{2.153861in}{1.529806in}}%
\pgfpathclose%
\pgfusepath{stroke}%
\end{pgfscope}%
\begin{pgfscope}%
\pgfpathrectangle{\pgfqpoint{0.847223in}{0.554012in}}{\pgfqpoint{6.200000in}{4.620000in}}%
\pgfusepath{clip}%
\pgfsetbuttcap%
\pgfsetroundjoin%
\pgfsetlinewidth{1.003750pt}%
\definecolor{currentstroke}{rgb}{1.000000,0.000000,0.000000}%
\pgfsetstrokecolor{currentstroke}%
\pgfsetdash{}{0pt}%
\pgfpathmoveto{\pgfqpoint{2.159194in}{1.525613in}}%
\pgfpathcurveto{\pgfqpoint{2.170244in}{1.525613in}}{\pgfqpoint{2.180843in}{1.530003in}}{\pgfqpoint{2.188657in}{1.537817in}}%
\pgfpathcurveto{\pgfqpoint{2.196471in}{1.545630in}}{\pgfqpoint{2.200861in}{1.556229in}}{\pgfqpoint{2.200861in}{1.567279in}}%
\pgfpathcurveto{\pgfqpoint{2.200861in}{1.578329in}}{\pgfqpoint{2.196471in}{1.588929in}}{\pgfqpoint{2.188657in}{1.596742in}}%
\pgfpathcurveto{\pgfqpoint{2.180843in}{1.604556in}}{\pgfqpoint{2.170244in}{1.608946in}}{\pgfqpoint{2.159194in}{1.608946in}}%
\pgfpathcurveto{\pgfqpoint{2.148144in}{1.608946in}}{\pgfqpoint{2.137545in}{1.604556in}}{\pgfqpoint{2.129731in}{1.596742in}}%
\pgfpathcurveto{\pgfqpoint{2.121918in}{1.588929in}}{\pgfqpoint{2.117527in}{1.578329in}}{\pgfqpoint{2.117527in}{1.567279in}}%
\pgfpathcurveto{\pgfqpoint{2.117527in}{1.556229in}}{\pgfqpoint{2.121918in}{1.545630in}}{\pgfqpoint{2.129731in}{1.537817in}}%
\pgfpathcurveto{\pgfqpoint{2.137545in}{1.530003in}}{\pgfqpoint{2.148144in}{1.525613in}}{\pgfqpoint{2.159194in}{1.525613in}}%
\pgfpathlineto{\pgfqpoint{2.159194in}{1.525613in}}%
\pgfpathclose%
\pgfusepath{stroke}%
\end{pgfscope}%
\begin{pgfscope}%
\pgfpathrectangle{\pgfqpoint{0.847223in}{0.554012in}}{\pgfqpoint{6.200000in}{4.620000in}}%
\pgfusepath{clip}%
\pgfsetbuttcap%
\pgfsetroundjoin%
\pgfsetlinewidth{1.003750pt}%
\definecolor{currentstroke}{rgb}{1.000000,0.000000,0.000000}%
\pgfsetstrokecolor{currentstroke}%
\pgfsetdash{}{0pt}%
\pgfpathmoveto{\pgfqpoint{2.164527in}{1.521444in}}%
\pgfpathcurveto{\pgfqpoint{2.175577in}{1.521444in}}{\pgfqpoint{2.186176in}{1.525834in}}{\pgfqpoint{2.193990in}{1.533648in}}%
\pgfpathcurveto{\pgfqpoint{2.201804in}{1.541461in}}{\pgfqpoint{2.206194in}{1.552061in}}{\pgfqpoint{2.206194in}{1.563111in}}%
\pgfpathcurveto{\pgfqpoint{2.206194in}{1.574161in}}{\pgfqpoint{2.201804in}{1.584760in}}{\pgfqpoint{2.193990in}{1.592573in}}%
\pgfpathcurveto{\pgfqpoint{2.186176in}{1.600387in}}{\pgfqpoint{2.175577in}{1.604777in}}{\pgfqpoint{2.164527in}{1.604777in}}%
\pgfpathcurveto{\pgfqpoint{2.153477in}{1.604777in}}{\pgfqpoint{2.142878in}{1.600387in}}{\pgfqpoint{2.135065in}{1.592573in}}%
\pgfpathcurveto{\pgfqpoint{2.127251in}{1.584760in}}{\pgfqpoint{2.122861in}{1.574161in}}{\pgfqpoint{2.122861in}{1.563111in}}%
\pgfpathcurveto{\pgfqpoint{2.122861in}{1.552061in}}{\pgfqpoint{2.127251in}{1.541461in}}{\pgfqpoint{2.135065in}{1.533648in}}%
\pgfpathcurveto{\pgfqpoint{2.142878in}{1.525834in}}{\pgfqpoint{2.153477in}{1.521444in}}{\pgfqpoint{2.164527in}{1.521444in}}%
\pgfpathlineto{\pgfqpoint{2.164527in}{1.521444in}}%
\pgfpathclose%
\pgfusepath{stroke}%
\end{pgfscope}%
\begin{pgfscope}%
\pgfpathrectangle{\pgfqpoint{0.847223in}{0.554012in}}{\pgfqpoint{6.200000in}{4.620000in}}%
\pgfusepath{clip}%
\pgfsetbuttcap%
\pgfsetroundjoin%
\pgfsetlinewidth{1.003750pt}%
\definecolor{currentstroke}{rgb}{1.000000,0.000000,0.000000}%
\pgfsetstrokecolor{currentstroke}%
\pgfsetdash{}{0pt}%
\pgfpathmoveto{\pgfqpoint{2.169861in}{1.517299in}}%
\pgfpathcurveto{\pgfqpoint{2.180911in}{1.517299in}}{\pgfqpoint{2.191510in}{1.521690in}}{\pgfqpoint{2.199323in}{1.529503in}}%
\pgfpathcurveto{\pgfqpoint{2.207137in}{1.537317in}}{\pgfqpoint{2.211527in}{1.547916in}}{\pgfqpoint{2.211527in}{1.558966in}}%
\pgfpathcurveto{\pgfqpoint{2.211527in}{1.570016in}}{\pgfqpoint{2.207137in}{1.580615in}}{\pgfqpoint{2.199323in}{1.588429in}}%
\pgfpathcurveto{\pgfqpoint{2.191510in}{1.596242in}}{\pgfqpoint{2.180911in}{1.600633in}}{\pgfqpoint{2.169861in}{1.600633in}}%
\pgfpathcurveto{\pgfqpoint{2.158810in}{1.600633in}}{\pgfqpoint{2.148211in}{1.596242in}}{\pgfqpoint{2.140398in}{1.588429in}}%
\pgfpathcurveto{\pgfqpoint{2.132584in}{1.580615in}}{\pgfqpoint{2.128194in}{1.570016in}}{\pgfqpoint{2.128194in}{1.558966in}}%
\pgfpathcurveto{\pgfqpoint{2.128194in}{1.547916in}}{\pgfqpoint{2.132584in}{1.537317in}}{\pgfqpoint{2.140398in}{1.529503in}}%
\pgfpathcurveto{\pgfqpoint{2.148211in}{1.521690in}}{\pgfqpoint{2.158810in}{1.517299in}}{\pgfqpoint{2.169861in}{1.517299in}}%
\pgfpathlineto{\pgfqpoint{2.169861in}{1.517299in}}%
\pgfpathclose%
\pgfusepath{stroke}%
\end{pgfscope}%
\begin{pgfscope}%
\pgfpathrectangle{\pgfqpoint{0.847223in}{0.554012in}}{\pgfqpoint{6.200000in}{4.620000in}}%
\pgfusepath{clip}%
\pgfsetbuttcap%
\pgfsetroundjoin%
\pgfsetlinewidth{1.003750pt}%
\definecolor{currentstroke}{rgb}{1.000000,0.000000,0.000000}%
\pgfsetstrokecolor{currentstroke}%
\pgfsetdash{}{0pt}%
\pgfpathmoveto{\pgfqpoint{2.175194in}{1.513179in}}%
\pgfpathcurveto{\pgfqpoint{2.186244in}{1.513179in}}{\pgfqpoint{2.196843in}{1.517569in}}{\pgfqpoint{2.204657in}{1.525383in}}%
\pgfpathcurveto{\pgfqpoint{2.212470in}{1.533196in}}{\pgfqpoint{2.216860in}{1.543795in}}{\pgfqpoint{2.216860in}{1.554845in}}%
\pgfpathcurveto{\pgfqpoint{2.216860in}{1.565895in}}{\pgfqpoint{2.212470in}{1.576495in}}{\pgfqpoint{2.204657in}{1.584308in}}%
\pgfpathcurveto{\pgfqpoint{2.196843in}{1.592122in}}{\pgfqpoint{2.186244in}{1.596512in}}{\pgfqpoint{2.175194in}{1.596512in}}%
\pgfpathcurveto{\pgfqpoint{2.164144in}{1.596512in}}{\pgfqpoint{2.153545in}{1.592122in}}{\pgfqpoint{2.145731in}{1.584308in}}%
\pgfpathcurveto{\pgfqpoint{2.137917in}{1.576495in}}{\pgfqpoint{2.133527in}{1.565895in}}{\pgfqpoint{2.133527in}{1.554845in}}%
\pgfpathcurveto{\pgfqpoint{2.133527in}{1.543795in}}{\pgfqpoint{2.137917in}{1.533196in}}{\pgfqpoint{2.145731in}{1.525383in}}%
\pgfpathcurveto{\pgfqpoint{2.153545in}{1.517569in}}{\pgfqpoint{2.164144in}{1.513179in}}{\pgfqpoint{2.175194in}{1.513179in}}%
\pgfpathlineto{\pgfqpoint{2.175194in}{1.513179in}}%
\pgfpathclose%
\pgfusepath{stroke}%
\end{pgfscope}%
\begin{pgfscope}%
\pgfpathrectangle{\pgfqpoint{0.847223in}{0.554012in}}{\pgfqpoint{6.200000in}{4.620000in}}%
\pgfusepath{clip}%
\pgfsetbuttcap%
\pgfsetroundjoin%
\pgfsetlinewidth{1.003750pt}%
\definecolor{currentstroke}{rgb}{1.000000,0.000000,0.000000}%
\pgfsetstrokecolor{currentstroke}%
\pgfsetdash{}{0pt}%
\pgfpathmoveto{\pgfqpoint{2.180527in}{1.509082in}}%
\pgfpathcurveto{\pgfqpoint{2.191577in}{1.509082in}}{\pgfqpoint{2.202176in}{1.513472in}}{\pgfqpoint{2.209990in}{1.521286in}}%
\pgfpathcurveto{\pgfqpoint{2.217803in}{1.529099in}}{\pgfqpoint{2.222194in}{1.539698in}}{\pgfqpoint{2.222194in}{1.550748in}}%
\pgfpathcurveto{\pgfqpoint{2.222194in}{1.561798in}}{\pgfqpoint{2.217803in}{1.572397in}}{\pgfqpoint{2.209990in}{1.580211in}}%
\pgfpathcurveto{\pgfqpoint{2.202176in}{1.588025in}}{\pgfqpoint{2.191577in}{1.592415in}}{\pgfqpoint{2.180527in}{1.592415in}}%
\pgfpathcurveto{\pgfqpoint{2.169477in}{1.592415in}}{\pgfqpoint{2.158878in}{1.588025in}}{\pgfqpoint{2.151064in}{1.580211in}}%
\pgfpathcurveto{\pgfqpoint{2.143251in}{1.572397in}}{\pgfqpoint{2.138860in}{1.561798in}}{\pgfqpoint{2.138860in}{1.550748in}}%
\pgfpathcurveto{\pgfqpoint{2.138860in}{1.539698in}}{\pgfqpoint{2.143251in}{1.529099in}}{\pgfqpoint{2.151064in}{1.521286in}}%
\pgfpathcurveto{\pgfqpoint{2.158878in}{1.513472in}}{\pgfqpoint{2.169477in}{1.509082in}}{\pgfqpoint{2.180527in}{1.509082in}}%
\pgfpathlineto{\pgfqpoint{2.180527in}{1.509082in}}%
\pgfpathclose%
\pgfusepath{stroke}%
\end{pgfscope}%
\begin{pgfscope}%
\pgfpathrectangle{\pgfqpoint{0.847223in}{0.554012in}}{\pgfqpoint{6.200000in}{4.620000in}}%
\pgfusepath{clip}%
\pgfsetbuttcap%
\pgfsetroundjoin%
\pgfsetlinewidth{1.003750pt}%
\definecolor{currentstroke}{rgb}{1.000000,0.000000,0.000000}%
\pgfsetstrokecolor{currentstroke}%
\pgfsetdash{}{0pt}%
\pgfpathmoveto{\pgfqpoint{2.185860in}{1.505008in}}%
\pgfpathcurveto{\pgfqpoint{2.196910in}{1.505008in}}{\pgfqpoint{2.207509in}{1.509398in}}{\pgfqpoint{2.215323in}{1.517212in}}%
\pgfpathcurveto{\pgfqpoint{2.223137in}{1.525026in}}{\pgfqpoint{2.227527in}{1.535625in}}{\pgfqpoint{2.227527in}{1.546675in}}%
\pgfpathcurveto{\pgfqpoint{2.227527in}{1.557725in}}{\pgfqpoint{2.223137in}{1.568324in}}{\pgfqpoint{2.215323in}{1.576138in}}%
\pgfpathcurveto{\pgfqpoint{2.207509in}{1.583951in}}{\pgfqpoint{2.196910in}{1.588341in}}{\pgfqpoint{2.185860in}{1.588341in}}%
\pgfpathcurveto{\pgfqpoint{2.174810in}{1.588341in}}{\pgfqpoint{2.164211in}{1.583951in}}{\pgfqpoint{2.156397in}{1.576138in}}%
\pgfpathcurveto{\pgfqpoint{2.148584in}{1.568324in}}{\pgfqpoint{2.144194in}{1.557725in}}{\pgfqpoint{2.144194in}{1.546675in}}%
\pgfpathcurveto{\pgfqpoint{2.144194in}{1.535625in}}{\pgfqpoint{2.148584in}{1.525026in}}{\pgfqpoint{2.156397in}{1.517212in}}%
\pgfpathcurveto{\pgfqpoint{2.164211in}{1.509398in}}{\pgfqpoint{2.174810in}{1.505008in}}{\pgfqpoint{2.185860in}{1.505008in}}%
\pgfpathlineto{\pgfqpoint{2.185860in}{1.505008in}}%
\pgfpathclose%
\pgfusepath{stroke}%
\end{pgfscope}%
\begin{pgfscope}%
\pgfpathrectangle{\pgfqpoint{0.847223in}{0.554012in}}{\pgfqpoint{6.200000in}{4.620000in}}%
\pgfusepath{clip}%
\pgfsetbuttcap%
\pgfsetroundjoin%
\pgfsetlinewidth{1.003750pt}%
\definecolor{currentstroke}{rgb}{1.000000,0.000000,0.000000}%
\pgfsetstrokecolor{currentstroke}%
\pgfsetdash{}{0pt}%
\pgfpathmoveto{\pgfqpoint{2.191193in}{1.500958in}}%
\pgfpathcurveto{\pgfqpoint{2.202244in}{1.500958in}}{\pgfqpoint{2.212843in}{1.505348in}}{\pgfqpoint{2.220656in}{1.513162in}}%
\pgfpathcurveto{\pgfqpoint{2.228470in}{1.520975in}}{\pgfqpoint{2.232860in}{1.531574in}}{\pgfqpoint{2.232860in}{1.542624in}}%
\pgfpathcurveto{\pgfqpoint{2.232860in}{1.553675in}}{\pgfqpoint{2.228470in}{1.564274in}}{\pgfqpoint{2.220656in}{1.572087in}}%
\pgfpathcurveto{\pgfqpoint{2.212843in}{1.579901in}}{\pgfqpoint{2.202244in}{1.584291in}}{\pgfqpoint{2.191193in}{1.584291in}}%
\pgfpathcurveto{\pgfqpoint{2.180143in}{1.584291in}}{\pgfqpoint{2.169544in}{1.579901in}}{\pgfqpoint{2.161731in}{1.572087in}}%
\pgfpathcurveto{\pgfqpoint{2.153917in}{1.564274in}}{\pgfqpoint{2.149527in}{1.553675in}}{\pgfqpoint{2.149527in}{1.542624in}}%
\pgfpathcurveto{\pgfqpoint{2.149527in}{1.531574in}}{\pgfqpoint{2.153917in}{1.520975in}}{\pgfqpoint{2.161731in}{1.513162in}}%
\pgfpathcurveto{\pgfqpoint{2.169544in}{1.505348in}}{\pgfqpoint{2.180143in}{1.500958in}}{\pgfqpoint{2.191193in}{1.500958in}}%
\pgfpathlineto{\pgfqpoint{2.191193in}{1.500958in}}%
\pgfpathclose%
\pgfusepath{stroke}%
\end{pgfscope}%
\begin{pgfscope}%
\pgfpathrectangle{\pgfqpoint{0.847223in}{0.554012in}}{\pgfqpoint{6.200000in}{4.620000in}}%
\pgfusepath{clip}%
\pgfsetbuttcap%
\pgfsetroundjoin%
\pgfsetlinewidth{1.003750pt}%
\definecolor{currentstroke}{rgb}{1.000000,0.000000,0.000000}%
\pgfsetstrokecolor{currentstroke}%
\pgfsetdash{}{0pt}%
\pgfpathmoveto{\pgfqpoint{2.196527in}{1.496931in}}%
\pgfpathcurveto{\pgfqpoint{2.207577in}{1.496931in}}{\pgfqpoint{2.218176in}{1.501321in}}{\pgfqpoint{2.225989in}{1.509134in}}%
\pgfpathcurveto{\pgfqpoint{2.233803in}{1.516948in}}{\pgfqpoint{2.238193in}{1.527547in}}{\pgfqpoint{2.238193in}{1.538597in}}%
\pgfpathcurveto{\pgfqpoint{2.238193in}{1.549647in}}{\pgfqpoint{2.233803in}{1.560246in}}{\pgfqpoint{2.225989in}{1.568060in}}%
\pgfpathcurveto{\pgfqpoint{2.218176in}{1.575874in}}{\pgfqpoint{2.207577in}{1.580264in}}{\pgfqpoint{2.196527in}{1.580264in}}%
\pgfpathcurveto{\pgfqpoint{2.185476in}{1.580264in}}{\pgfqpoint{2.174877in}{1.575874in}}{\pgfqpoint{2.167064in}{1.568060in}}%
\pgfpathcurveto{\pgfqpoint{2.159250in}{1.560246in}}{\pgfqpoint{2.154860in}{1.549647in}}{\pgfqpoint{2.154860in}{1.538597in}}%
\pgfpathcurveto{\pgfqpoint{2.154860in}{1.527547in}}{\pgfqpoint{2.159250in}{1.516948in}}{\pgfqpoint{2.167064in}{1.509134in}}%
\pgfpathcurveto{\pgfqpoint{2.174877in}{1.501321in}}{\pgfqpoint{2.185476in}{1.496931in}}{\pgfqpoint{2.196527in}{1.496931in}}%
\pgfpathlineto{\pgfqpoint{2.196527in}{1.496931in}}%
\pgfpathclose%
\pgfusepath{stroke}%
\end{pgfscope}%
\begin{pgfscope}%
\pgfpathrectangle{\pgfqpoint{0.847223in}{0.554012in}}{\pgfqpoint{6.200000in}{4.620000in}}%
\pgfusepath{clip}%
\pgfsetbuttcap%
\pgfsetroundjoin%
\pgfsetlinewidth{1.003750pt}%
\definecolor{currentstroke}{rgb}{1.000000,0.000000,0.000000}%
\pgfsetstrokecolor{currentstroke}%
\pgfsetdash{}{0pt}%
\pgfpathmoveto{\pgfqpoint{2.201860in}{1.492926in}}%
\pgfpathcurveto{\pgfqpoint{2.212910in}{1.492926in}}{\pgfqpoint{2.223509in}{1.497316in}}{\pgfqpoint{2.231323in}{1.505130in}}%
\pgfpathcurveto{\pgfqpoint{2.239136in}{1.512944in}}{\pgfqpoint{2.243526in}{1.523543in}}{\pgfqpoint{2.243526in}{1.534593in}}%
\pgfpathcurveto{\pgfqpoint{2.243526in}{1.545643in}}{\pgfqpoint{2.239136in}{1.556242in}}{\pgfqpoint{2.231323in}{1.564056in}}%
\pgfpathcurveto{\pgfqpoint{2.223509in}{1.571869in}}{\pgfqpoint{2.212910in}{1.576259in}}{\pgfqpoint{2.201860in}{1.576259in}}%
\pgfpathcurveto{\pgfqpoint{2.190810in}{1.576259in}}{\pgfqpoint{2.180211in}{1.571869in}}{\pgfqpoint{2.172397in}{1.564056in}}%
\pgfpathcurveto{\pgfqpoint{2.164583in}{1.556242in}}{\pgfqpoint{2.160193in}{1.545643in}}{\pgfqpoint{2.160193in}{1.534593in}}%
\pgfpathcurveto{\pgfqpoint{2.160193in}{1.523543in}}{\pgfqpoint{2.164583in}{1.512944in}}{\pgfqpoint{2.172397in}{1.505130in}}%
\pgfpathcurveto{\pgfqpoint{2.180211in}{1.497316in}}{\pgfqpoint{2.190810in}{1.492926in}}{\pgfqpoint{2.201860in}{1.492926in}}%
\pgfpathlineto{\pgfqpoint{2.201860in}{1.492926in}}%
\pgfpathclose%
\pgfusepath{stroke}%
\end{pgfscope}%
\begin{pgfscope}%
\pgfpathrectangle{\pgfqpoint{0.847223in}{0.554012in}}{\pgfqpoint{6.200000in}{4.620000in}}%
\pgfusepath{clip}%
\pgfsetbuttcap%
\pgfsetroundjoin%
\pgfsetlinewidth{1.003750pt}%
\definecolor{currentstroke}{rgb}{1.000000,0.000000,0.000000}%
\pgfsetstrokecolor{currentstroke}%
\pgfsetdash{}{0pt}%
\pgfpathmoveto{\pgfqpoint{2.207193in}{1.488944in}}%
\pgfpathcurveto{\pgfqpoint{2.218243in}{1.488944in}}{\pgfqpoint{2.228842in}{1.493335in}}{\pgfqpoint{2.236656in}{1.501148in}}%
\pgfpathcurveto{\pgfqpoint{2.244469in}{1.508962in}}{\pgfqpoint{2.248860in}{1.519561in}}{\pgfqpoint{2.248860in}{1.530611in}}%
\pgfpathcurveto{\pgfqpoint{2.248860in}{1.541661in}}{\pgfqpoint{2.244469in}{1.552260in}}{\pgfqpoint{2.236656in}{1.560074in}}%
\pgfpathcurveto{\pgfqpoint{2.228842in}{1.567888in}}{\pgfqpoint{2.218243in}{1.572278in}}{\pgfqpoint{2.207193in}{1.572278in}}%
\pgfpathcurveto{\pgfqpoint{2.196143in}{1.572278in}}{\pgfqpoint{2.185544in}{1.567888in}}{\pgfqpoint{2.177730in}{1.560074in}}%
\pgfpathcurveto{\pgfqpoint{2.169917in}{1.552260in}}{\pgfqpoint{2.165526in}{1.541661in}}{\pgfqpoint{2.165526in}{1.530611in}}%
\pgfpathcurveto{\pgfqpoint{2.165526in}{1.519561in}}{\pgfqpoint{2.169917in}{1.508962in}}{\pgfqpoint{2.177730in}{1.501148in}}%
\pgfpathcurveto{\pgfqpoint{2.185544in}{1.493335in}}{\pgfqpoint{2.196143in}{1.488944in}}{\pgfqpoint{2.207193in}{1.488944in}}%
\pgfpathlineto{\pgfqpoint{2.207193in}{1.488944in}}%
\pgfpathclose%
\pgfusepath{stroke}%
\end{pgfscope}%
\begin{pgfscope}%
\pgfpathrectangle{\pgfqpoint{0.847223in}{0.554012in}}{\pgfqpoint{6.200000in}{4.620000in}}%
\pgfusepath{clip}%
\pgfsetbuttcap%
\pgfsetroundjoin%
\pgfsetlinewidth{1.003750pt}%
\definecolor{currentstroke}{rgb}{1.000000,0.000000,0.000000}%
\pgfsetstrokecolor{currentstroke}%
\pgfsetdash{}{0pt}%
\pgfpathmoveto{\pgfqpoint{2.212526in}{1.484985in}}%
\pgfpathcurveto{\pgfqpoint{2.223576in}{1.484985in}}{\pgfqpoint{2.234175in}{1.489376in}}{\pgfqpoint{2.241989in}{1.497189in}}%
\pgfpathcurveto{\pgfqpoint{2.249803in}{1.505003in}}{\pgfqpoint{2.254193in}{1.515602in}}{\pgfqpoint{2.254193in}{1.526652in}}%
\pgfpathcurveto{\pgfqpoint{2.254193in}{1.537702in}}{\pgfqpoint{2.249803in}{1.548301in}}{\pgfqpoint{2.241989in}{1.556115in}}%
\pgfpathcurveto{\pgfqpoint{2.234175in}{1.563928in}}{\pgfqpoint{2.223576in}{1.568319in}}{\pgfqpoint{2.212526in}{1.568319in}}%
\pgfpathcurveto{\pgfqpoint{2.201476in}{1.568319in}}{\pgfqpoint{2.190877in}{1.563928in}}{\pgfqpoint{2.183063in}{1.556115in}}%
\pgfpathcurveto{\pgfqpoint{2.175250in}{1.548301in}}{\pgfqpoint{2.170860in}{1.537702in}}{\pgfqpoint{2.170860in}{1.526652in}}%
\pgfpathcurveto{\pgfqpoint{2.170860in}{1.515602in}}{\pgfqpoint{2.175250in}{1.505003in}}{\pgfqpoint{2.183063in}{1.497189in}}%
\pgfpathcurveto{\pgfqpoint{2.190877in}{1.489376in}}{\pgfqpoint{2.201476in}{1.484985in}}{\pgfqpoint{2.212526in}{1.484985in}}%
\pgfpathlineto{\pgfqpoint{2.212526in}{1.484985in}}%
\pgfpathclose%
\pgfusepath{stroke}%
\end{pgfscope}%
\begin{pgfscope}%
\pgfpathrectangle{\pgfqpoint{0.847223in}{0.554012in}}{\pgfqpoint{6.200000in}{4.620000in}}%
\pgfusepath{clip}%
\pgfsetbuttcap%
\pgfsetroundjoin%
\pgfsetlinewidth{1.003750pt}%
\definecolor{currentstroke}{rgb}{1.000000,0.000000,0.000000}%
\pgfsetstrokecolor{currentstroke}%
\pgfsetdash{}{0pt}%
\pgfpathmoveto{\pgfqpoint{2.217859in}{1.481048in}}%
\pgfpathcurveto{\pgfqpoint{2.228910in}{1.481048in}}{\pgfqpoint{2.239509in}{1.485439in}}{\pgfqpoint{2.247322in}{1.493252in}}%
\pgfpathcurveto{\pgfqpoint{2.255136in}{1.501066in}}{\pgfqpoint{2.259526in}{1.511665in}}{\pgfqpoint{2.259526in}{1.522715in}}%
\pgfpathcurveto{\pgfqpoint{2.259526in}{1.533765in}}{\pgfqpoint{2.255136in}{1.544364in}}{\pgfqpoint{2.247322in}{1.552178in}}%
\pgfpathcurveto{\pgfqpoint{2.239509in}{1.559991in}}{\pgfqpoint{2.228910in}{1.564382in}}{\pgfqpoint{2.217859in}{1.564382in}}%
\pgfpathcurveto{\pgfqpoint{2.206809in}{1.564382in}}{\pgfqpoint{2.196210in}{1.559991in}}{\pgfqpoint{2.188397in}{1.552178in}}%
\pgfpathcurveto{\pgfqpoint{2.180583in}{1.544364in}}{\pgfqpoint{2.176193in}{1.533765in}}{\pgfqpoint{2.176193in}{1.522715in}}%
\pgfpathcurveto{\pgfqpoint{2.176193in}{1.511665in}}{\pgfqpoint{2.180583in}{1.501066in}}{\pgfqpoint{2.188397in}{1.493252in}}%
\pgfpathcurveto{\pgfqpoint{2.196210in}{1.485439in}}{\pgfqpoint{2.206809in}{1.481048in}}{\pgfqpoint{2.217859in}{1.481048in}}%
\pgfpathlineto{\pgfqpoint{2.217859in}{1.481048in}}%
\pgfpathclose%
\pgfusepath{stroke}%
\end{pgfscope}%
\begin{pgfscope}%
\pgfpathrectangle{\pgfqpoint{0.847223in}{0.554012in}}{\pgfqpoint{6.200000in}{4.620000in}}%
\pgfusepath{clip}%
\pgfsetbuttcap%
\pgfsetroundjoin%
\pgfsetlinewidth{1.003750pt}%
\definecolor{currentstroke}{rgb}{1.000000,0.000000,0.000000}%
\pgfsetstrokecolor{currentstroke}%
\pgfsetdash{}{0pt}%
\pgfpathmoveto{\pgfqpoint{2.223193in}{1.477134in}}%
\pgfpathcurveto{\pgfqpoint{2.234243in}{1.477134in}}{\pgfqpoint{2.244842in}{1.481524in}}{\pgfqpoint{2.252655in}{1.489337in}}%
\pgfpathcurveto{\pgfqpoint{2.260469in}{1.497151in}}{\pgfqpoint{2.264859in}{1.507750in}}{\pgfqpoint{2.264859in}{1.518800in}}%
\pgfpathcurveto{\pgfqpoint{2.264859in}{1.529850in}}{\pgfqpoint{2.260469in}{1.540449in}}{\pgfqpoint{2.252655in}{1.548263in}}%
\pgfpathcurveto{\pgfqpoint{2.244842in}{1.556077in}}{\pgfqpoint{2.234243in}{1.560467in}}{\pgfqpoint{2.223193in}{1.560467in}}%
\pgfpathcurveto{\pgfqpoint{2.212143in}{1.560467in}}{\pgfqpoint{2.201544in}{1.556077in}}{\pgfqpoint{2.193730in}{1.548263in}}%
\pgfpathcurveto{\pgfqpoint{2.185916in}{1.540449in}}{\pgfqpoint{2.181526in}{1.529850in}}{\pgfqpoint{2.181526in}{1.518800in}}%
\pgfpathcurveto{\pgfqpoint{2.181526in}{1.507750in}}{\pgfqpoint{2.185916in}{1.497151in}}{\pgfqpoint{2.193730in}{1.489337in}}%
\pgfpathcurveto{\pgfqpoint{2.201544in}{1.481524in}}{\pgfqpoint{2.212143in}{1.477134in}}{\pgfqpoint{2.223193in}{1.477134in}}%
\pgfpathlineto{\pgfqpoint{2.223193in}{1.477134in}}%
\pgfpathclose%
\pgfusepath{stroke}%
\end{pgfscope}%
\begin{pgfscope}%
\pgfpathrectangle{\pgfqpoint{0.847223in}{0.554012in}}{\pgfqpoint{6.200000in}{4.620000in}}%
\pgfusepath{clip}%
\pgfsetbuttcap%
\pgfsetroundjoin%
\pgfsetlinewidth{1.003750pt}%
\definecolor{currentstroke}{rgb}{1.000000,0.000000,0.000000}%
\pgfsetstrokecolor{currentstroke}%
\pgfsetdash{}{0pt}%
\pgfpathmoveto{\pgfqpoint{2.228526in}{1.473241in}}%
\pgfpathcurveto{\pgfqpoint{2.239576in}{1.473241in}}{\pgfqpoint{2.250175in}{1.477631in}}{\pgfqpoint{2.257989in}{1.485445in}}%
\pgfpathcurveto{\pgfqpoint{2.265802in}{1.493258in}}{\pgfqpoint{2.270193in}{1.503857in}}{\pgfqpoint{2.270193in}{1.514907in}}%
\pgfpathcurveto{\pgfqpoint{2.270193in}{1.525958in}}{\pgfqpoint{2.265802in}{1.536557in}}{\pgfqpoint{2.257989in}{1.544370in}}%
\pgfpathcurveto{\pgfqpoint{2.250175in}{1.552184in}}{\pgfqpoint{2.239576in}{1.556574in}}{\pgfqpoint{2.228526in}{1.556574in}}%
\pgfpathcurveto{\pgfqpoint{2.217476in}{1.556574in}}{\pgfqpoint{2.206877in}{1.552184in}}{\pgfqpoint{2.199063in}{1.544370in}}%
\pgfpathcurveto{\pgfqpoint{2.191250in}{1.536557in}}{\pgfqpoint{2.186859in}{1.525958in}}{\pgfqpoint{2.186859in}{1.514907in}}%
\pgfpathcurveto{\pgfqpoint{2.186859in}{1.503857in}}{\pgfqpoint{2.191250in}{1.493258in}}{\pgfqpoint{2.199063in}{1.485445in}}%
\pgfpathcurveto{\pgfqpoint{2.206877in}{1.477631in}}{\pgfqpoint{2.217476in}{1.473241in}}{\pgfqpoint{2.228526in}{1.473241in}}%
\pgfpathlineto{\pgfqpoint{2.228526in}{1.473241in}}%
\pgfpathclose%
\pgfusepath{stroke}%
\end{pgfscope}%
\begin{pgfscope}%
\pgfpathrectangle{\pgfqpoint{0.847223in}{0.554012in}}{\pgfqpoint{6.200000in}{4.620000in}}%
\pgfusepath{clip}%
\pgfsetbuttcap%
\pgfsetroundjoin%
\pgfsetlinewidth{1.003750pt}%
\definecolor{currentstroke}{rgb}{1.000000,0.000000,0.000000}%
\pgfsetstrokecolor{currentstroke}%
\pgfsetdash{}{0pt}%
\pgfpathmoveto{\pgfqpoint{2.233859in}{1.469370in}}%
\pgfpathcurveto{\pgfqpoint{2.244909in}{1.469370in}}{\pgfqpoint{2.255508in}{1.473760in}}{\pgfqpoint{2.263322in}{1.481573in}}%
\pgfpathcurveto{\pgfqpoint{2.271136in}{1.489387in}}{\pgfqpoint{2.275526in}{1.499986in}}{\pgfqpoint{2.275526in}{1.511036in}}%
\pgfpathcurveto{\pgfqpoint{2.275526in}{1.522086in}}{\pgfqpoint{2.271136in}{1.532685in}}{\pgfqpoint{2.263322in}{1.540499in}}%
\pgfpathcurveto{\pgfqpoint{2.255508in}{1.548313in}}{\pgfqpoint{2.244909in}{1.552703in}}{\pgfqpoint{2.233859in}{1.552703in}}%
\pgfpathcurveto{\pgfqpoint{2.222809in}{1.552703in}}{\pgfqpoint{2.212210in}{1.548313in}}{\pgfqpoint{2.204396in}{1.540499in}}%
\pgfpathcurveto{\pgfqpoint{2.196583in}{1.532685in}}{\pgfqpoint{2.192192in}{1.522086in}}{\pgfqpoint{2.192192in}{1.511036in}}%
\pgfpathcurveto{\pgfqpoint{2.192192in}{1.499986in}}{\pgfqpoint{2.196583in}{1.489387in}}{\pgfqpoint{2.204396in}{1.481573in}}%
\pgfpathcurveto{\pgfqpoint{2.212210in}{1.473760in}}{\pgfqpoint{2.222809in}{1.469370in}}{\pgfqpoint{2.233859in}{1.469370in}}%
\pgfpathlineto{\pgfqpoint{2.233859in}{1.469370in}}%
\pgfpathclose%
\pgfusepath{stroke}%
\end{pgfscope}%
\begin{pgfscope}%
\pgfpathrectangle{\pgfqpoint{0.847223in}{0.554012in}}{\pgfqpoint{6.200000in}{4.620000in}}%
\pgfusepath{clip}%
\pgfsetbuttcap%
\pgfsetroundjoin%
\pgfsetlinewidth{1.003750pt}%
\definecolor{currentstroke}{rgb}{1.000000,0.000000,0.000000}%
\pgfsetstrokecolor{currentstroke}%
\pgfsetdash{}{0pt}%
\pgfpathmoveto{\pgfqpoint{2.239192in}{1.465520in}}%
\pgfpathcurveto{\pgfqpoint{2.250242in}{1.465520in}}{\pgfqpoint{2.260842in}{1.469910in}}{\pgfqpoint{2.268655in}{1.477724in}}%
\pgfpathcurveto{\pgfqpoint{2.276469in}{1.485538in}}{\pgfqpoint{2.280859in}{1.496137in}}{\pgfqpoint{2.280859in}{1.507187in}}%
\pgfpathcurveto{\pgfqpoint{2.280859in}{1.518237in}}{\pgfqpoint{2.276469in}{1.528836in}}{\pgfqpoint{2.268655in}{1.536649in}}%
\pgfpathcurveto{\pgfqpoint{2.260842in}{1.544463in}}{\pgfqpoint{2.250242in}{1.548853in}}{\pgfqpoint{2.239192in}{1.548853in}}%
\pgfpathcurveto{\pgfqpoint{2.228142in}{1.548853in}}{\pgfqpoint{2.217543in}{1.544463in}}{\pgfqpoint{2.209730in}{1.536649in}}%
\pgfpathcurveto{\pgfqpoint{2.201916in}{1.528836in}}{\pgfqpoint{2.197526in}{1.518237in}}{\pgfqpoint{2.197526in}{1.507187in}}%
\pgfpathcurveto{\pgfqpoint{2.197526in}{1.496137in}}{\pgfqpoint{2.201916in}{1.485538in}}{\pgfqpoint{2.209730in}{1.477724in}}%
\pgfpathcurveto{\pgfqpoint{2.217543in}{1.469910in}}{\pgfqpoint{2.228142in}{1.465520in}}{\pgfqpoint{2.239192in}{1.465520in}}%
\pgfpathlineto{\pgfqpoint{2.239192in}{1.465520in}}%
\pgfpathclose%
\pgfusepath{stroke}%
\end{pgfscope}%
\begin{pgfscope}%
\pgfpathrectangle{\pgfqpoint{0.847223in}{0.554012in}}{\pgfqpoint{6.200000in}{4.620000in}}%
\pgfusepath{clip}%
\pgfsetbuttcap%
\pgfsetroundjoin%
\pgfsetlinewidth{1.003750pt}%
\definecolor{currentstroke}{rgb}{1.000000,0.000000,0.000000}%
\pgfsetstrokecolor{currentstroke}%
\pgfsetdash{}{0pt}%
\pgfpathmoveto{\pgfqpoint{2.244526in}{1.461692in}}%
\pgfpathcurveto{\pgfqpoint{2.255576in}{1.461692in}}{\pgfqpoint{2.266175in}{1.466082in}}{\pgfqpoint{2.273988in}{1.473896in}}%
\pgfpathcurveto{\pgfqpoint{2.281802in}{1.481709in}}{\pgfqpoint{2.286192in}{1.492308in}}{\pgfqpoint{2.286192in}{1.503359in}}%
\pgfpathcurveto{\pgfqpoint{2.286192in}{1.514409in}}{\pgfqpoint{2.281802in}{1.525008in}}{\pgfqpoint{2.273988in}{1.532821in}}%
\pgfpathcurveto{\pgfqpoint{2.266175in}{1.540635in}}{\pgfqpoint{2.255576in}{1.545025in}}{\pgfqpoint{2.244526in}{1.545025in}}%
\pgfpathcurveto{\pgfqpoint{2.233475in}{1.545025in}}{\pgfqpoint{2.222876in}{1.540635in}}{\pgfqpoint{2.215063in}{1.532821in}}%
\pgfpathcurveto{\pgfqpoint{2.207249in}{1.525008in}}{\pgfqpoint{2.202859in}{1.514409in}}{\pgfqpoint{2.202859in}{1.503359in}}%
\pgfpathcurveto{\pgfqpoint{2.202859in}{1.492308in}}{\pgfqpoint{2.207249in}{1.481709in}}{\pgfqpoint{2.215063in}{1.473896in}}%
\pgfpathcurveto{\pgfqpoint{2.222876in}{1.466082in}}{\pgfqpoint{2.233475in}{1.461692in}}{\pgfqpoint{2.244526in}{1.461692in}}%
\pgfpathlineto{\pgfqpoint{2.244526in}{1.461692in}}%
\pgfpathclose%
\pgfusepath{stroke}%
\end{pgfscope}%
\begin{pgfscope}%
\pgfpathrectangle{\pgfqpoint{0.847223in}{0.554012in}}{\pgfqpoint{6.200000in}{4.620000in}}%
\pgfusepath{clip}%
\pgfsetbuttcap%
\pgfsetroundjoin%
\pgfsetlinewidth{1.003750pt}%
\definecolor{currentstroke}{rgb}{1.000000,0.000000,0.000000}%
\pgfsetstrokecolor{currentstroke}%
\pgfsetdash{}{0pt}%
\pgfpathmoveto{\pgfqpoint{2.249859in}{1.457885in}}%
\pgfpathcurveto{\pgfqpoint{2.260909in}{1.457885in}}{\pgfqpoint{2.271508in}{1.462275in}}{\pgfqpoint{2.279322in}{1.470089in}}%
\pgfpathcurveto{\pgfqpoint{2.287135in}{1.477902in}}{\pgfqpoint{2.291525in}{1.488501in}}{\pgfqpoint{2.291525in}{1.499552in}}%
\pgfpathcurveto{\pgfqpoint{2.291525in}{1.510602in}}{\pgfqpoint{2.287135in}{1.521201in}}{\pgfqpoint{2.279322in}{1.529014in}}%
\pgfpathcurveto{\pgfqpoint{2.271508in}{1.536828in}}{\pgfqpoint{2.260909in}{1.541218in}}{\pgfqpoint{2.249859in}{1.541218in}}%
\pgfpathcurveto{\pgfqpoint{2.238809in}{1.541218in}}{\pgfqpoint{2.228210in}{1.536828in}}{\pgfqpoint{2.220396in}{1.529014in}}%
\pgfpathcurveto{\pgfqpoint{2.212582in}{1.521201in}}{\pgfqpoint{2.208192in}{1.510602in}}{\pgfqpoint{2.208192in}{1.499552in}}%
\pgfpathcurveto{\pgfqpoint{2.208192in}{1.488501in}}{\pgfqpoint{2.212582in}{1.477902in}}{\pgfqpoint{2.220396in}{1.470089in}}%
\pgfpathcurveto{\pgfqpoint{2.228210in}{1.462275in}}{\pgfqpoint{2.238809in}{1.457885in}}{\pgfqpoint{2.249859in}{1.457885in}}%
\pgfpathlineto{\pgfqpoint{2.249859in}{1.457885in}}%
\pgfpathclose%
\pgfusepath{stroke}%
\end{pgfscope}%
\begin{pgfscope}%
\pgfpathrectangle{\pgfqpoint{0.847223in}{0.554012in}}{\pgfqpoint{6.200000in}{4.620000in}}%
\pgfusepath{clip}%
\pgfsetbuttcap%
\pgfsetroundjoin%
\pgfsetlinewidth{1.003750pt}%
\definecolor{currentstroke}{rgb}{1.000000,0.000000,0.000000}%
\pgfsetstrokecolor{currentstroke}%
\pgfsetdash{}{0pt}%
\pgfpathmoveto{\pgfqpoint{2.255192in}{1.454099in}}%
\pgfpathcurveto{\pgfqpoint{2.266242in}{1.454099in}}{\pgfqpoint{2.276841in}{1.458489in}}{\pgfqpoint{2.284655in}{1.466303in}}%
\pgfpathcurveto{\pgfqpoint{2.292468in}{1.474116in}}{\pgfqpoint{2.296859in}{1.484715in}}{\pgfqpoint{2.296859in}{1.495766in}}%
\pgfpathcurveto{\pgfqpoint{2.296859in}{1.506816in}}{\pgfqpoint{2.292468in}{1.517415in}}{\pgfqpoint{2.284655in}{1.525228in}}%
\pgfpathcurveto{\pgfqpoint{2.276841in}{1.533042in}}{\pgfqpoint{2.266242in}{1.537432in}}{\pgfqpoint{2.255192in}{1.537432in}}%
\pgfpathcurveto{\pgfqpoint{2.244142in}{1.537432in}}{\pgfqpoint{2.233543in}{1.533042in}}{\pgfqpoint{2.225729in}{1.525228in}}%
\pgfpathcurveto{\pgfqpoint{2.217916in}{1.517415in}}{\pgfqpoint{2.213525in}{1.506816in}}{\pgfqpoint{2.213525in}{1.495766in}}%
\pgfpathcurveto{\pgfqpoint{2.213525in}{1.484715in}}{\pgfqpoint{2.217916in}{1.474116in}}{\pgfqpoint{2.225729in}{1.466303in}}%
\pgfpathcurveto{\pgfqpoint{2.233543in}{1.458489in}}{\pgfqpoint{2.244142in}{1.454099in}}{\pgfqpoint{2.255192in}{1.454099in}}%
\pgfpathlineto{\pgfqpoint{2.255192in}{1.454099in}}%
\pgfpathclose%
\pgfusepath{stroke}%
\end{pgfscope}%
\begin{pgfscope}%
\pgfpathrectangle{\pgfqpoint{0.847223in}{0.554012in}}{\pgfqpoint{6.200000in}{4.620000in}}%
\pgfusepath{clip}%
\pgfsetbuttcap%
\pgfsetroundjoin%
\pgfsetlinewidth{1.003750pt}%
\definecolor{currentstroke}{rgb}{1.000000,0.000000,0.000000}%
\pgfsetstrokecolor{currentstroke}%
\pgfsetdash{}{0pt}%
\pgfpathmoveto{\pgfqpoint{2.260525in}{1.450334in}}%
\pgfpathcurveto{\pgfqpoint{2.271575in}{1.450334in}}{\pgfqpoint{2.282174in}{1.454724in}}{\pgfqpoint{2.289988in}{1.462538in}}%
\pgfpathcurveto{\pgfqpoint{2.297802in}{1.470351in}}{\pgfqpoint{2.302192in}{1.480950in}}{\pgfqpoint{2.302192in}{1.492000in}}%
\pgfpathcurveto{\pgfqpoint{2.302192in}{1.503051in}}{\pgfqpoint{2.297802in}{1.513650in}}{\pgfqpoint{2.289988in}{1.521463in}}%
\pgfpathcurveto{\pgfqpoint{2.282174in}{1.529277in}}{\pgfqpoint{2.271575in}{1.533667in}}{\pgfqpoint{2.260525in}{1.533667in}}%
\pgfpathcurveto{\pgfqpoint{2.249475in}{1.533667in}}{\pgfqpoint{2.238876in}{1.529277in}}{\pgfqpoint{2.231062in}{1.521463in}}%
\pgfpathcurveto{\pgfqpoint{2.223249in}{1.513650in}}{\pgfqpoint{2.218859in}{1.503051in}}{\pgfqpoint{2.218859in}{1.492000in}}%
\pgfpathcurveto{\pgfqpoint{2.218859in}{1.480950in}}{\pgfqpoint{2.223249in}{1.470351in}}{\pgfqpoint{2.231062in}{1.462538in}}%
\pgfpathcurveto{\pgfqpoint{2.238876in}{1.454724in}}{\pgfqpoint{2.249475in}{1.450334in}}{\pgfqpoint{2.260525in}{1.450334in}}%
\pgfpathlineto{\pgfqpoint{2.260525in}{1.450334in}}%
\pgfpathclose%
\pgfusepath{stroke}%
\end{pgfscope}%
\begin{pgfscope}%
\pgfpathrectangle{\pgfqpoint{0.847223in}{0.554012in}}{\pgfqpoint{6.200000in}{4.620000in}}%
\pgfusepath{clip}%
\pgfsetbuttcap%
\pgfsetroundjoin%
\pgfsetlinewidth{1.003750pt}%
\definecolor{currentstroke}{rgb}{1.000000,0.000000,0.000000}%
\pgfsetstrokecolor{currentstroke}%
\pgfsetdash{}{0pt}%
\pgfpathmoveto{\pgfqpoint{2.265858in}{1.446589in}}%
\pgfpathcurveto{\pgfqpoint{2.276909in}{1.446589in}}{\pgfqpoint{2.287508in}{1.450980in}}{\pgfqpoint{2.295321in}{1.458793in}}%
\pgfpathcurveto{\pgfqpoint{2.303135in}{1.466607in}}{\pgfqpoint{2.307525in}{1.477206in}}{\pgfqpoint{2.307525in}{1.488256in}}%
\pgfpathcurveto{\pgfqpoint{2.307525in}{1.499306in}}{\pgfqpoint{2.303135in}{1.509905in}}{\pgfqpoint{2.295321in}{1.517719in}}%
\pgfpathcurveto{\pgfqpoint{2.287508in}{1.525532in}}{\pgfqpoint{2.276909in}{1.529923in}}{\pgfqpoint{2.265858in}{1.529923in}}%
\pgfpathcurveto{\pgfqpoint{2.254808in}{1.529923in}}{\pgfqpoint{2.244209in}{1.525532in}}{\pgfqpoint{2.236396in}{1.517719in}}%
\pgfpathcurveto{\pgfqpoint{2.228582in}{1.509905in}}{\pgfqpoint{2.224192in}{1.499306in}}{\pgfqpoint{2.224192in}{1.488256in}}%
\pgfpathcurveto{\pgfqpoint{2.224192in}{1.477206in}}{\pgfqpoint{2.228582in}{1.466607in}}{\pgfqpoint{2.236396in}{1.458793in}}%
\pgfpathcurveto{\pgfqpoint{2.244209in}{1.450980in}}{\pgfqpoint{2.254808in}{1.446589in}}{\pgfqpoint{2.265858in}{1.446589in}}%
\pgfpathlineto{\pgfqpoint{2.265858in}{1.446589in}}%
\pgfpathclose%
\pgfusepath{stroke}%
\end{pgfscope}%
\begin{pgfscope}%
\pgfpathrectangle{\pgfqpoint{0.847223in}{0.554012in}}{\pgfqpoint{6.200000in}{4.620000in}}%
\pgfusepath{clip}%
\pgfsetbuttcap%
\pgfsetroundjoin%
\pgfsetlinewidth{1.003750pt}%
\definecolor{currentstroke}{rgb}{1.000000,0.000000,0.000000}%
\pgfsetstrokecolor{currentstroke}%
\pgfsetdash{}{0pt}%
\pgfpathmoveto{\pgfqpoint{2.271192in}{1.442865in}}%
\pgfpathcurveto{\pgfqpoint{2.282242in}{1.442865in}}{\pgfqpoint{2.292841in}{1.447256in}}{\pgfqpoint{2.300654in}{1.455069in}}%
\pgfpathcurveto{\pgfqpoint{2.308468in}{1.462883in}}{\pgfqpoint{2.312858in}{1.473482in}}{\pgfqpoint{2.312858in}{1.484532in}}%
\pgfpathcurveto{\pgfqpoint{2.312858in}{1.495582in}}{\pgfqpoint{2.308468in}{1.506181in}}{\pgfqpoint{2.300654in}{1.513995in}}%
\pgfpathcurveto{\pgfqpoint{2.292841in}{1.521809in}}{\pgfqpoint{2.282242in}{1.526199in}}{\pgfqpoint{2.271192in}{1.526199in}}%
\pgfpathcurveto{\pgfqpoint{2.260142in}{1.526199in}}{\pgfqpoint{2.249542in}{1.521809in}}{\pgfqpoint{2.241729in}{1.513995in}}%
\pgfpathcurveto{\pgfqpoint{2.233915in}{1.506181in}}{\pgfqpoint{2.229525in}{1.495582in}}{\pgfqpoint{2.229525in}{1.484532in}}%
\pgfpathcurveto{\pgfqpoint{2.229525in}{1.473482in}}{\pgfqpoint{2.233915in}{1.462883in}}{\pgfqpoint{2.241729in}{1.455069in}}%
\pgfpathcurveto{\pgfqpoint{2.249542in}{1.447256in}}{\pgfqpoint{2.260142in}{1.442865in}}{\pgfqpoint{2.271192in}{1.442865in}}%
\pgfpathlineto{\pgfqpoint{2.271192in}{1.442865in}}%
\pgfpathclose%
\pgfusepath{stroke}%
\end{pgfscope}%
\begin{pgfscope}%
\pgfpathrectangle{\pgfqpoint{0.847223in}{0.554012in}}{\pgfqpoint{6.200000in}{4.620000in}}%
\pgfusepath{clip}%
\pgfsetbuttcap%
\pgfsetroundjoin%
\pgfsetlinewidth{1.003750pt}%
\definecolor{currentstroke}{rgb}{1.000000,0.000000,0.000000}%
\pgfsetstrokecolor{currentstroke}%
\pgfsetdash{}{0pt}%
\pgfpathmoveto{\pgfqpoint{2.276525in}{1.439162in}}%
\pgfpathcurveto{\pgfqpoint{2.287575in}{1.439162in}}{\pgfqpoint{2.298174in}{1.443552in}}{\pgfqpoint{2.305988in}{1.451366in}}%
\pgfpathcurveto{\pgfqpoint{2.313801in}{1.459179in}}{\pgfqpoint{2.318192in}{1.469778in}}{\pgfqpoint{2.318192in}{1.480829in}}%
\pgfpathcurveto{\pgfqpoint{2.318192in}{1.491879in}}{\pgfqpoint{2.313801in}{1.502478in}}{\pgfqpoint{2.305988in}{1.510291in}}%
\pgfpathcurveto{\pgfqpoint{2.298174in}{1.518105in}}{\pgfqpoint{2.287575in}{1.522495in}}{\pgfqpoint{2.276525in}{1.522495in}}%
\pgfpathcurveto{\pgfqpoint{2.265475in}{1.522495in}}{\pgfqpoint{2.254876in}{1.518105in}}{\pgfqpoint{2.247062in}{1.510291in}}%
\pgfpathcurveto{\pgfqpoint{2.239248in}{1.502478in}}{\pgfqpoint{2.234858in}{1.491879in}}{\pgfqpoint{2.234858in}{1.480829in}}%
\pgfpathcurveto{\pgfqpoint{2.234858in}{1.469778in}}{\pgfqpoint{2.239248in}{1.459179in}}{\pgfqpoint{2.247062in}{1.451366in}}%
\pgfpathcurveto{\pgfqpoint{2.254876in}{1.443552in}}{\pgfqpoint{2.265475in}{1.439162in}}{\pgfqpoint{2.276525in}{1.439162in}}%
\pgfpathlineto{\pgfqpoint{2.276525in}{1.439162in}}%
\pgfpathclose%
\pgfusepath{stroke}%
\end{pgfscope}%
\begin{pgfscope}%
\pgfpathrectangle{\pgfqpoint{0.847223in}{0.554012in}}{\pgfqpoint{6.200000in}{4.620000in}}%
\pgfusepath{clip}%
\pgfsetbuttcap%
\pgfsetroundjoin%
\pgfsetlinewidth{1.003750pt}%
\definecolor{currentstroke}{rgb}{1.000000,0.000000,0.000000}%
\pgfsetstrokecolor{currentstroke}%
\pgfsetdash{}{0pt}%
\pgfpathmoveto{\pgfqpoint{2.281858in}{1.435478in}}%
\pgfpathcurveto{\pgfqpoint{2.292908in}{1.435478in}}{\pgfqpoint{2.303507in}{1.439869in}}{\pgfqpoint{2.311321in}{1.447682in}}%
\pgfpathcurveto{\pgfqpoint{2.319134in}{1.455496in}}{\pgfqpoint{2.323525in}{1.466095in}}{\pgfqpoint{2.323525in}{1.477145in}}%
\pgfpathcurveto{\pgfqpoint{2.323525in}{1.488195in}}{\pgfqpoint{2.319134in}{1.498794in}}{\pgfqpoint{2.311321in}{1.506608in}}%
\pgfpathcurveto{\pgfqpoint{2.303507in}{1.514422in}}{\pgfqpoint{2.292908in}{1.518812in}}{\pgfqpoint{2.281858in}{1.518812in}}%
\pgfpathcurveto{\pgfqpoint{2.270808in}{1.518812in}}{\pgfqpoint{2.260209in}{1.514422in}}{\pgfqpoint{2.252395in}{1.506608in}}%
\pgfpathcurveto{\pgfqpoint{2.244582in}{1.498794in}}{\pgfqpoint{2.240191in}{1.488195in}}{\pgfqpoint{2.240191in}{1.477145in}}%
\pgfpathcurveto{\pgfqpoint{2.240191in}{1.466095in}}{\pgfqpoint{2.244582in}{1.455496in}}{\pgfqpoint{2.252395in}{1.447682in}}%
\pgfpathcurveto{\pgfqpoint{2.260209in}{1.439869in}}{\pgfqpoint{2.270808in}{1.435478in}}{\pgfqpoint{2.281858in}{1.435478in}}%
\pgfpathlineto{\pgfqpoint{2.281858in}{1.435478in}}%
\pgfpathclose%
\pgfusepath{stroke}%
\end{pgfscope}%
\begin{pgfscope}%
\pgfpathrectangle{\pgfqpoint{0.847223in}{0.554012in}}{\pgfqpoint{6.200000in}{4.620000in}}%
\pgfusepath{clip}%
\pgfsetbuttcap%
\pgfsetroundjoin%
\pgfsetlinewidth{1.003750pt}%
\definecolor{currentstroke}{rgb}{1.000000,0.000000,0.000000}%
\pgfsetstrokecolor{currentstroke}%
\pgfsetdash{}{0pt}%
\pgfpathmoveto{\pgfqpoint{2.287191in}{1.431815in}}%
\pgfpathcurveto{\pgfqpoint{2.298241in}{1.431815in}}{\pgfqpoint{2.308840in}{1.436205in}}{\pgfqpoint{2.316654in}{1.444019in}}%
\pgfpathcurveto{\pgfqpoint{2.324468in}{1.451833in}}{\pgfqpoint{2.328858in}{1.462432in}}{\pgfqpoint{2.328858in}{1.473482in}}%
\pgfpathcurveto{\pgfqpoint{2.328858in}{1.484532in}}{\pgfqpoint{2.324468in}{1.495131in}}{\pgfqpoint{2.316654in}{1.502945in}}%
\pgfpathcurveto{\pgfqpoint{2.308840in}{1.510758in}}{\pgfqpoint{2.298241in}{1.515148in}}{\pgfqpoint{2.287191in}{1.515148in}}%
\pgfpathcurveto{\pgfqpoint{2.276141in}{1.515148in}}{\pgfqpoint{2.265542in}{1.510758in}}{\pgfqpoint{2.257729in}{1.502945in}}%
\pgfpathcurveto{\pgfqpoint{2.249915in}{1.495131in}}{\pgfqpoint{2.245525in}{1.484532in}}{\pgfqpoint{2.245525in}{1.473482in}}%
\pgfpathcurveto{\pgfqpoint{2.245525in}{1.462432in}}{\pgfqpoint{2.249915in}{1.451833in}}{\pgfqpoint{2.257729in}{1.444019in}}%
\pgfpathcurveto{\pgfqpoint{2.265542in}{1.436205in}}{\pgfqpoint{2.276141in}{1.431815in}}{\pgfqpoint{2.287191in}{1.431815in}}%
\pgfpathlineto{\pgfqpoint{2.287191in}{1.431815in}}%
\pgfpathclose%
\pgfusepath{stroke}%
\end{pgfscope}%
\begin{pgfscope}%
\pgfpathrectangle{\pgfqpoint{0.847223in}{0.554012in}}{\pgfqpoint{6.200000in}{4.620000in}}%
\pgfusepath{clip}%
\pgfsetbuttcap%
\pgfsetroundjoin%
\pgfsetlinewidth{1.003750pt}%
\definecolor{currentstroke}{rgb}{1.000000,0.000000,0.000000}%
\pgfsetstrokecolor{currentstroke}%
\pgfsetdash{}{0pt}%
\pgfpathmoveto{\pgfqpoint{2.292524in}{1.428172in}}%
\pgfpathcurveto{\pgfqpoint{2.303575in}{1.428172in}}{\pgfqpoint{2.314174in}{1.432562in}}{\pgfqpoint{2.321987in}{1.440375in}}%
\pgfpathcurveto{\pgfqpoint{2.329801in}{1.448189in}}{\pgfqpoint{2.334191in}{1.458788in}}{\pgfqpoint{2.334191in}{1.469838in}}%
\pgfpathcurveto{\pgfqpoint{2.334191in}{1.480888in}}{\pgfqpoint{2.329801in}{1.491487in}}{\pgfqpoint{2.321987in}{1.499301in}}%
\pgfpathcurveto{\pgfqpoint{2.314174in}{1.507115in}}{\pgfqpoint{2.303575in}{1.511505in}}{\pgfqpoint{2.292524in}{1.511505in}}%
\pgfpathcurveto{\pgfqpoint{2.281474in}{1.511505in}}{\pgfqpoint{2.270875in}{1.507115in}}{\pgfqpoint{2.263062in}{1.499301in}}%
\pgfpathcurveto{\pgfqpoint{2.255248in}{1.491487in}}{\pgfqpoint{2.250858in}{1.480888in}}{\pgfqpoint{2.250858in}{1.469838in}}%
\pgfpathcurveto{\pgfqpoint{2.250858in}{1.458788in}}{\pgfqpoint{2.255248in}{1.448189in}}{\pgfqpoint{2.263062in}{1.440375in}}%
\pgfpathcurveto{\pgfqpoint{2.270875in}{1.432562in}}{\pgfqpoint{2.281474in}{1.428172in}}{\pgfqpoint{2.292524in}{1.428172in}}%
\pgfpathlineto{\pgfqpoint{2.292524in}{1.428172in}}%
\pgfpathclose%
\pgfusepath{stroke}%
\end{pgfscope}%
\begin{pgfscope}%
\pgfpathrectangle{\pgfqpoint{0.847223in}{0.554012in}}{\pgfqpoint{6.200000in}{4.620000in}}%
\pgfusepath{clip}%
\pgfsetbuttcap%
\pgfsetroundjoin%
\pgfsetlinewidth{1.003750pt}%
\definecolor{currentstroke}{rgb}{1.000000,0.000000,0.000000}%
\pgfsetstrokecolor{currentstroke}%
\pgfsetdash{}{0pt}%
\pgfpathmoveto{\pgfqpoint{2.297858in}{1.424548in}}%
\pgfpathcurveto{\pgfqpoint{2.308908in}{1.424548in}}{\pgfqpoint{2.319507in}{1.428938in}}{\pgfqpoint{2.327320in}{1.436752in}}%
\pgfpathcurveto{\pgfqpoint{2.335134in}{1.444565in}}{\pgfqpoint{2.339524in}{1.455164in}}{\pgfqpoint{2.339524in}{1.466214in}}%
\pgfpathcurveto{\pgfqpoint{2.339524in}{1.477264in}}{\pgfqpoint{2.335134in}{1.487864in}}{\pgfqpoint{2.327320in}{1.495677in}}%
\pgfpathcurveto{\pgfqpoint{2.319507in}{1.503491in}}{\pgfqpoint{2.308908in}{1.507881in}}{\pgfqpoint{2.297858in}{1.507881in}}%
\pgfpathcurveto{\pgfqpoint{2.286808in}{1.507881in}}{\pgfqpoint{2.276209in}{1.503491in}}{\pgfqpoint{2.268395in}{1.495677in}}%
\pgfpathcurveto{\pgfqpoint{2.260581in}{1.487864in}}{\pgfqpoint{2.256191in}{1.477264in}}{\pgfqpoint{2.256191in}{1.466214in}}%
\pgfpathcurveto{\pgfqpoint{2.256191in}{1.455164in}}{\pgfqpoint{2.260581in}{1.444565in}}{\pgfqpoint{2.268395in}{1.436752in}}%
\pgfpathcurveto{\pgfqpoint{2.276209in}{1.428938in}}{\pgfqpoint{2.286808in}{1.424548in}}{\pgfqpoint{2.297858in}{1.424548in}}%
\pgfpathlineto{\pgfqpoint{2.297858in}{1.424548in}}%
\pgfpathclose%
\pgfusepath{stroke}%
\end{pgfscope}%
\begin{pgfscope}%
\pgfpathrectangle{\pgfqpoint{0.847223in}{0.554012in}}{\pgfqpoint{6.200000in}{4.620000in}}%
\pgfusepath{clip}%
\pgfsetbuttcap%
\pgfsetroundjoin%
\pgfsetlinewidth{1.003750pt}%
\definecolor{currentstroke}{rgb}{1.000000,0.000000,0.000000}%
\pgfsetstrokecolor{currentstroke}%
\pgfsetdash{}{0pt}%
\pgfpathmoveto{\pgfqpoint{2.303191in}{1.420943in}}%
\pgfpathcurveto{\pgfqpoint{2.314241in}{1.420943in}}{\pgfqpoint{2.324840in}{1.425334in}}{\pgfqpoint{2.332654in}{1.433147in}}%
\pgfpathcurveto{\pgfqpoint{2.340467in}{1.440961in}}{\pgfqpoint{2.344858in}{1.451560in}}{\pgfqpoint{2.344858in}{1.462610in}}%
\pgfpathcurveto{\pgfqpoint{2.344858in}{1.473660in}}{\pgfqpoint{2.340467in}{1.484259in}}{\pgfqpoint{2.332654in}{1.492073in}}%
\pgfpathcurveto{\pgfqpoint{2.324840in}{1.499886in}}{\pgfqpoint{2.314241in}{1.504277in}}{\pgfqpoint{2.303191in}{1.504277in}}%
\pgfpathcurveto{\pgfqpoint{2.292141in}{1.504277in}}{\pgfqpoint{2.281542in}{1.499886in}}{\pgfqpoint{2.273728in}{1.492073in}}%
\pgfpathcurveto{\pgfqpoint{2.265915in}{1.484259in}}{\pgfqpoint{2.261524in}{1.473660in}}{\pgfqpoint{2.261524in}{1.462610in}}%
\pgfpathcurveto{\pgfqpoint{2.261524in}{1.451560in}}{\pgfqpoint{2.265915in}{1.440961in}}{\pgfqpoint{2.273728in}{1.433147in}}%
\pgfpathcurveto{\pgfqpoint{2.281542in}{1.425334in}}{\pgfqpoint{2.292141in}{1.420943in}}{\pgfqpoint{2.303191in}{1.420943in}}%
\pgfpathlineto{\pgfqpoint{2.303191in}{1.420943in}}%
\pgfpathclose%
\pgfusepath{stroke}%
\end{pgfscope}%
\begin{pgfscope}%
\pgfpathrectangle{\pgfqpoint{0.847223in}{0.554012in}}{\pgfqpoint{6.200000in}{4.620000in}}%
\pgfusepath{clip}%
\pgfsetbuttcap%
\pgfsetroundjoin%
\pgfsetlinewidth{1.003750pt}%
\definecolor{currentstroke}{rgb}{1.000000,0.000000,0.000000}%
\pgfsetstrokecolor{currentstroke}%
\pgfsetdash{}{0pt}%
\pgfpathmoveto{\pgfqpoint{2.308524in}{1.417358in}}%
\pgfpathcurveto{\pgfqpoint{2.319574in}{1.417358in}}{\pgfqpoint{2.330173in}{1.421749in}}{\pgfqpoint{2.337987in}{1.429562in}}%
\pgfpathcurveto{\pgfqpoint{2.345801in}{1.437376in}}{\pgfqpoint{2.350191in}{1.447975in}}{\pgfqpoint{2.350191in}{1.459025in}}%
\pgfpathcurveto{\pgfqpoint{2.350191in}{1.470075in}}{\pgfqpoint{2.345801in}{1.480674in}}{\pgfqpoint{2.337987in}{1.488488in}}%
\pgfpathcurveto{\pgfqpoint{2.330173in}{1.496301in}}{\pgfqpoint{2.319574in}{1.500692in}}{\pgfqpoint{2.308524in}{1.500692in}}%
\pgfpathcurveto{\pgfqpoint{2.297474in}{1.500692in}}{\pgfqpoint{2.286875in}{1.496301in}}{\pgfqpoint{2.279061in}{1.488488in}}%
\pgfpathcurveto{\pgfqpoint{2.271248in}{1.480674in}}{\pgfqpoint{2.266857in}{1.470075in}}{\pgfqpoint{2.266857in}{1.459025in}}%
\pgfpathcurveto{\pgfqpoint{2.266857in}{1.447975in}}{\pgfqpoint{2.271248in}{1.437376in}}{\pgfqpoint{2.279061in}{1.429562in}}%
\pgfpathcurveto{\pgfqpoint{2.286875in}{1.421749in}}{\pgfqpoint{2.297474in}{1.417358in}}{\pgfqpoint{2.308524in}{1.417358in}}%
\pgfpathlineto{\pgfqpoint{2.308524in}{1.417358in}}%
\pgfpathclose%
\pgfusepath{stroke}%
\end{pgfscope}%
\begin{pgfscope}%
\pgfpathrectangle{\pgfqpoint{0.847223in}{0.554012in}}{\pgfqpoint{6.200000in}{4.620000in}}%
\pgfusepath{clip}%
\pgfsetbuttcap%
\pgfsetroundjoin%
\pgfsetlinewidth{1.003750pt}%
\definecolor{currentstroke}{rgb}{1.000000,0.000000,0.000000}%
\pgfsetstrokecolor{currentstroke}%
\pgfsetdash{}{0pt}%
\pgfpathmoveto{\pgfqpoint{2.313857in}{1.413793in}}%
\pgfpathcurveto{\pgfqpoint{2.324907in}{1.413793in}}{\pgfqpoint{2.335507in}{1.418183in}}{\pgfqpoint{2.343320in}{1.425997in}}%
\pgfpathcurveto{\pgfqpoint{2.351134in}{1.433810in}}{\pgfqpoint{2.355524in}{1.444409in}}{\pgfqpoint{2.355524in}{1.455459in}}%
\pgfpathcurveto{\pgfqpoint{2.355524in}{1.466509in}}{\pgfqpoint{2.351134in}{1.477109in}}{\pgfqpoint{2.343320in}{1.484922in}}%
\pgfpathcurveto{\pgfqpoint{2.335507in}{1.492736in}}{\pgfqpoint{2.324907in}{1.497126in}}{\pgfqpoint{2.313857in}{1.497126in}}%
\pgfpathcurveto{\pgfqpoint{2.302807in}{1.497126in}}{\pgfqpoint{2.292208in}{1.492736in}}{\pgfqpoint{2.284395in}{1.484922in}}%
\pgfpathcurveto{\pgfqpoint{2.276581in}{1.477109in}}{\pgfqpoint{2.272191in}{1.466509in}}{\pgfqpoint{2.272191in}{1.455459in}}%
\pgfpathcurveto{\pgfqpoint{2.272191in}{1.444409in}}{\pgfqpoint{2.276581in}{1.433810in}}{\pgfqpoint{2.284395in}{1.425997in}}%
\pgfpathcurveto{\pgfqpoint{2.292208in}{1.418183in}}{\pgfqpoint{2.302807in}{1.413793in}}{\pgfqpoint{2.313857in}{1.413793in}}%
\pgfpathlineto{\pgfqpoint{2.313857in}{1.413793in}}%
\pgfpathclose%
\pgfusepath{stroke}%
\end{pgfscope}%
\begin{pgfscope}%
\pgfpathrectangle{\pgfqpoint{0.847223in}{0.554012in}}{\pgfqpoint{6.200000in}{4.620000in}}%
\pgfusepath{clip}%
\pgfsetbuttcap%
\pgfsetroundjoin%
\pgfsetlinewidth{1.003750pt}%
\definecolor{currentstroke}{rgb}{1.000000,0.000000,0.000000}%
\pgfsetstrokecolor{currentstroke}%
\pgfsetdash{}{0pt}%
\pgfpathmoveto{\pgfqpoint{2.319191in}{1.410246in}}%
\pgfpathcurveto{\pgfqpoint{2.330241in}{1.410246in}}{\pgfqpoint{2.340840in}{1.414636in}}{\pgfqpoint{2.348653in}{1.422450in}}%
\pgfpathcurveto{\pgfqpoint{2.356467in}{1.430264in}}{\pgfqpoint{2.360857in}{1.440863in}}{\pgfqpoint{2.360857in}{1.451913in}}%
\pgfpathcurveto{\pgfqpoint{2.360857in}{1.462963in}}{\pgfqpoint{2.356467in}{1.473562in}}{\pgfqpoint{2.348653in}{1.481375in}}%
\pgfpathcurveto{\pgfqpoint{2.340840in}{1.489189in}}{\pgfqpoint{2.330241in}{1.493579in}}{\pgfqpoint{2.319191in}{1.493579in}}%
\pgfpathcurveto{\pgfqpoint{2.308140in}{1.493579in}}{\pgfqpoint{2.297541in}{1.489189in}}{\pgfqpoint{2.289728in}{1.481375in}}%
\pgfpathcurveto{\pgfqpoint{2.281914in}{1.473562in}}{\pgfqpoint{2.277524in}{1.462963in}}{\pgfqpoint{2.277524in}{1.451913in}}%
\pgfpathcurveto{\pgfqpoint{2.277524in}{1.440863in}}{\pgfqpoint{2.281914in}{1.430264in}}{\pgfqpoint{2.289728in}{1.422450in}}%
\pgfpathcurveto{\pgfqpoint{2.297541in}{1.414636in}}{\pgfqpoint{2.308140in}{1.410246in}}{\pgfqpoint{2.319191in}{1.410246in}}%
\pgfpathlineto{\pgfqpoint{2.319191in}{1.410246in}}%
\pgfpathclose%
\pgfusepath{stroke}%
\end{pgfscope}%
\begin{pgfscope}%
\pgfpathrectangle{\pgfqpoint{0.847223in}{0.554012in}}{\pgfqpoint{6.200000in}{4.620000in}}%
\pgfusepath{clip}%
\pgfsetbuttcap%
\pgfsetroundjoin%
\pgfsetlinewidth{1.003750pt}%
\definecolor{currentstroke}{rgb}{1.000000,0.000000,0.000000}%
\pgfsetstrokecolor{currentstroke}%
\pgfsetdash{}{0pt}%
\pgfpathmoveto{\pgfqpoint{2.324524in}{1.406718in}}%
\pgfpathcurveto{\pgfqpoint{2.335574in}{1.406718in}}{\pgfqpoint{2.346173in}{1.411108in}}{\pgfqpoint{2.353987in}{1.418922in}}%
\pgfpathcurveto{\pgfqpoint{2.361800in}{1.426736in}}{\pgfqpoint{2.366190in}{1.437335in}}{\pgfqpoint{2.366190in}{1.448385in}}%
\pgfpathcurveto{\pgfqpoint{2.366190in}{1.459435in}}{\pgfqpoint{2.361800in}{1.470034in}}{\pgfqpoint{2.353987in}{1.477848in}}%
\pgfpathcurveto{\pgfqpoint{2.346173in}{1.485661in}}{\pgfqpoint{2.335574in}{1.490052in}}{\pgfqpoint{2.324524in}{1.490052in}}%
\pgfpathcurveto{\pgfqpoint{2.313474in}{1.490052in}}{\pgfqpoint{2.302875in}{1.485661in}}{\pgfqpoint{2.295061in}{1.477848in}}%
\pgfpathcurveto{\pgfqpoint{2.287247in}{1.470034in}}{\pgfqpoint{2.282857in}{1.459435in}}{\pgfqpoint{2.282857in}{1.448385in}}%
\pgfpathcurveto{\pgfqpoint{2.282857in}{1.437335in}}{\pgfqpoint{2.287247in}{1.426736in}}{\pgfqpoint{2.295061in}{1.418922in}}%
\pgfpathcurveto{\pgfqpoint{2.302875in}{1.411108in}}{\pgfqpoint{2.313474in}{1.406718in}}{\pgfqpoint{2.324524in}{1.406718in}}%
\pgfpathlineto{\pgfqpoint{2.324524in}{1.406718in}}%
\pgfpathclose%
\pgfusepath{stroke}%
\end{pgfscope}%
\begin{pgfscope}%
\pgfpathrectangle{\pgfqpoint{0.847223in}{0.554012in}}{\pgfqpoint{6.200000in}{4.620000in}}%
\pgfusepath{clip}%
\pgfsetbuttcap%
\pgfsetroundjoin%
\pgfsetlinewidth{1.003750pt}%
\definecolor{currentstroke}{rgb}{1.000000,0.000000,0.000000}%
\pgfsetstrokecolor{currentstroke}%
\pgfsetdash{}{0pt}%
\pgfpathmoveto{\pgfqpoint{2.329857in}{1.403209in}}%
\pgfpathcurveto{\pgfqpoint{2.340907in}{1.403209in}}{\pgfqpoint{2.351506in}{1.407599in}}{\pgfqpoint{2.359320in}{1.415413in}}%
\pgfpathcurveto{\pgfqpoint{2.367133in}{1.423227in}}{\pgfqpoint{2.371524in}{1.433826in}}{\pgfqpoint{2.371524in}{1.444876in}}%
\pgfpathcurveto{\pgfqpoint{2.371524in}{1.455926in}}{\pgfqpoint{2.367133in}{1.466525in}}{\pgfqpoint{2.359320in}{1.474339in}}%
\pgfpathcurveto{\pgfqpoint{2.351506in}{1.482152in}}{\pgfqpoint{2.340907in}{1.486543in}}{\pgfqpoint{2.329857in}{1.486543in}}%
\pgfpathcurveto{\pgfqpoint{2.318807in}{1.486543in}}{\pgfqpoint{2.308208in}{1.482152in}}{\pgfqpoint{2.300394in}{1.474339in}}%
\pgfpathcurveto{\pgfqpoint{2.292581in}{1.466525in}}{\pgfqpoint{2.288190in}{1.455926in}}{\pgfqpoint{2.288190in}{1.444876in}}%
\pgfpathcurveto{\pgfqpoint{2.288190in}{1.433826in}}{\pgfqpoint{2.292581in}{1.423227in}}{\pgfqpoint{2.300394in}{1.415413in}}%
\pgfpathcurveto{\pgfqpoint{2.308208in}{1.407599in}}{\pgfqpoint{2.318807in}{1.403209in}}{\pgfqpoint{2.329857in}{1.403209in}}%
\pgfpathlineto{\pgfqpoint{2.329857in}{1.403209in}}%
\pgfpathclose%
\pgfusepath{stroke}%
\end{pgfscope}%
\begin{pgfscope}%
\pgfpathrectangle{\pgfqpoint{0.847223in}{0.554012in}}{\pgfqpoint{6.200000in}{4.620000in}}%
\pgfusepath{clip}%
\pgfsetbuttcap%
\pgfsetroundjoin%
\pgfsetlinewidth{1.003750pt}%
\definecolor{currentstroke}{rgb}{1.000000,0.000000,0.000000}%
\pgfsetstrokecolor{currentstroke}%
\pgfsetdash{}{0pt}%
\pgfpathmoveto{\pgfqpoint{2.335190in}{1.399719in}}%
\pgfpathcurveto{\pgfqpoint{2.346240in}{1.399719in}}{\pgfqpoint{2.356839in}{1.404109in}}{\pgfqpoint{2.364653in}{1.411923in}}%
\pgfpathcurveto{\pgfqpoint{2.372467in}{1.419736in}}{\pgfqpoint{2.376857in}{1.430335in}}{\pgfqpoint{2.376857in}{1.441385in}}%
\pgfpathcurveto{\pgfqpoint{2.376857in}{1.452436in}}{\pgfqpoint{2.372467in}{1.463035in}}{\pgfqpoint{2.364653in}{1.470848in}}%
\pgfpathcurveto{\pgfqpoint{2.356839in}{1.478662in}}{\pgfqpoint{2.346240in}{1.483052in}}{\pgfqpoint{2.335190in}{1.483052in}}%
\pgfpathcurveto{\pgfqpoint{2.324140in}{1.483052in}}{\pgfqpoint{2.313541in}{1.478662in}}{\pgfqpoint{2.305727in}{1.470848in}}%
\pgfpathcurveto{\pgfqpoint{2.297914in}{1.463035in}}{\pgfqpoint{2.293524in}{1.452436in}}{\pgfqpoint{2.293524in}{1.441385in}}%
\pgfpathcurveto{\pgfqpoint{2.293524in}{1.430335in}}{\pgfqpoint{2.297914in}{1.419736in}}{\pgfqpoint{2.305727in}{1.411923in}}%
\pgfpathcurveto{\pgfqpoint{2.313541in}{1.404109in}}{\pgfqpoint{2.324140in}{1.399719in}}{\pgfqpoint{2.335190in}{1.399719in}}%
\pgfpathlineto{\pgfqpoint{2.335190in}{1.399719in}}%
\pgfpathclose%
\pgfusepath{stroke}%
\end{pgfscope}%
\begin{pgfscope}%
\pgfpathrectangle{\pgfqpoint{0.847223in}{0.554012in}}{\pgfqpoint{6.200000in}{4.620000in}}%
\pgfusepath{clip}%
\pgfsetbuttcap%
\pgfsetroundjoin%
\pgfsetlinewidth{1.003750pt}%
\definecolor{currentstroke}{rgb}{1.000000,0.000000,0.000000}%
\pgfsetstrokecolor{currentstroke}%
\pgfsetdash{}{0pt}%
\pgfpathmoveto{\pgfqpoint{2.340523in}{1.396247in}}%
\pgfpathcurveto{\pgfqpoint{2.351574in}{1.396247in}}{\pgfqpoint{2.362173in}{1.400637in}}{\pgfqpoint{2.369986in}{1.408451in}}%
\pgfpathcurveto{\pgfqpoint{2.377800in}{1.416264in}}{\pgfqpoint{2.382190in}{1.426863in}}{\pgfqpoint{2.382190in}{1.437914in}}%
\pgfpathcurveto{\pgfqpoint{2.382190in}{1.448964in}}{\pgfqpoint{2.377800in}{1.459563in}}{\pgfqpoint{2.369986in}{1.467376in}}%
\pgfpathcurveto{\pgfqpoint{2.362173in}{1.475190in}}{\pgfqpoint{2.351574in}{1.479580in}}{\pgfqpoint{2.340523in}{1.479580in}}%
\pgfpathcurveto{\pgfqpoint{2.329473in}{1.479580in}}{\pgfqpoint{2.318874in}{1.475190in}}{\pgfqpoint{2.311061in}{1.467376in}}%
\pgfpathcurveto{\pgfqpoint{2.303247in}{1.459563in}}{\pgfqpoint{2.298857in}{1.448964in}}{\pgfqpoint{2.298857in}{1.437914in}}%
\pgfpathcurveto{\pgfqpoint{2.298857in}{1.426863in}}{\pgfqpoint{2.303247in}{1.416264in}}{\pgfqpoint{2.311061in}{1.408451in}}%
\pgfpathcurveto{\pgfqpoint{2.318874in}{1.400637in}}{\pgfqpoint{2.329473in}{1.396247in}}{\pgfqpoint{2.340523in}{1.396247in}}%
\pgfpathlineto{\pgfqpoint{2.340523in}{1.396247in}}%
\pgfpathclose%
\pgfusepath{stroke}%
\end{pgfscope}%
\begin{pgfscope}%
\pgfpathrectangle{\pgfqpoint{0.847223in}{0.554012in}}{\pgfqpoint{6.200000in}{4.620000in}}%
\pgfusepath{clip}%
\pgfsetbuttcap%
\pgfsetroundjoin%
\pgfsetlinewidth{1.003750pt}%
\definecolor{currentstroke}{rgb}{1.000000,0.000000,0.000000}%
\pgfsetstrokecolor{currentstroke}%
\pgfsetdash{}{0pt}%
\pgfpathmoveto{\pgfqpoint{2.345857in}{1.392793in}}%
\pgfpathcurveto{\pgfqpoint{2.356907in}{1.392793in}}{\pgfqpoint{2.367506in}{1.397184in}}{\pgfqpoint{2.375319in}{1.404997in}}%
\pgfpathcurveto{\pgfqpoint{2.383133in}{1.412811in}}{\pgfqpoint{2.387523in}{1.423410in}}{\pgfqpoint{2.387523in}{1.434460in}}%
\pgfpathcurveto{\pgfqpoint{2.387523in}{1.445510in}}{\pgfqpoint{2.383133in}{1.456109in}}{\pgfqpoint{2.375319in}{1.463923in}}%
\pgfpathcurveto{\pgfqpoint{2.367506in}{1.471736in}}{\pgfqpoint{2.356907in}{1.476127in}}{\pgfqpoint{2.345857in}{1.476127in}}%
\pgfpathcurveto{\pgfqpoint{2.334807in}{1.476127in}}{\pgfqpoint{2.324207in}{1.471736in}}{\pgfqpoint{2.316394in}{1.463923in}}%
\pgfpathcurveto{\pgfqpoint{2.308580in}{1.456109in}}{\pgfqpoint{2.304190in}{1.445510in}}{\pgfqpoint{2.304190in}{1.434460in}}%
\pgfpathcurveto{\pgfqpoint{2.304190in}{1.423410in}}{\pgfqpoint{2.308580in}{1.412811in}}{\pgfqpoint{2.316394in}{1.404997in}}%
\pgfpathcurveto{\pgfqpoint{2.324207in}{1.397184in}}{\pgfqpoint{2.334807in}{1.392793in}}{\pgfqpoint{2.345857in}{1.392793in}}%
\pgfpathlineto{\pgfqpoint{2.345857in}{1.392793in}}%
\pgfpathclose%
\pgfusepath{stroke}%
\end{pgfscope}%
\begin{pgfscope}%
\pgfpathrectangle{\pgfqpoint{0.847223in}{0.554012in}}{\pgfqpoint{6.200000in}{4.620000in}}%
\pgfusepath{clip}%
\pgfsetbuttcap%
\pgfsetroundjoin%
\pgfsetlinewidth{1.003750pt}%
\definecolor{currentstroke}{rgb}{1.000000,0.000000,0.000000}%
\pgfsetstrokecolor{currentstroke}%
\pgfsetdash{}{0pt}%
\pgfpathmoveto{\pgfqpoint{2.351190in}{1.389358in}}%
\pgfpathcurveto{\pgfqpoint{2.362240in}{1.389358in}}{\pgfqpoint{2.372839in}{1.393748in}}{\pgfqpoint{2.380653in}{1.401562in}}%
\pgfpathcurveto{\pgfqpoint{2.388466in}{1.409375in}}{\pgfqpoint{2.392857in}{1.419974in}}{\pgfqpoint{2.392857in}{1.431024in}}%
\pgfpathcurveto{\pgfqpoint{2.392857in}{1.442075in}}{\pgfqpoint{2.388466in}{1.452674in}}{\pgfqpoint{2.380653in}{1.460487in}}%
\pgfpathcurveto{\pgfqpoint{2.372839in}{1.468301in}}{\pgfqpoint{2.362240in}{1.472691in}}{\pgfqpoint{2.351190in}{1.472691in}}%
\pgfpathcurveto{\pgfqpoint{2.340140in}{1.472691in}}{\pgfqpoint{2.329541in}{1.468301in}}{\pgfqpoint{2.321727in}{1.460487in}}%
\pgfpathcurveto{\pgfqpoint{2.313913in}{1.452674in}}{\pgfqpoint{2.309523in}{1.442075in}}{\pgfqpoint{2.309523in}{1.431024in}}%
\pgfpathcurveto{\pgfqpoint{2.309523in}{1.419974in}}{\pgfqpoint{2.313913in}{1.409375in}}{\pgfqpoint{2.321727in}{1.401562in}}%
\pgfpathcurveto{\pgfqpoint{2.329541in}{1.393748in}}{\pgfqpoint{2.340140in}{1.389358in}}{\pgfqpoint{2.351190in}{1.389358in}}%
\pgfpathlineto{\pgfqpoint{2.351190in}{1.389358in}}%
\pgfpathclose%
\pgfusepath{stroke}%
\end{pgfscope}%
\begin{pgfscope}%
\pgfpathrectangle{\pgfqpoint{0.847223in}{0.554012in}}{\pgfqpoint{6.200000in}{4.620000in}}%
\pgfusepath{clip}%
\pgfsetbuttcap%
\pgfsetroundjoin%
\pgfsetlinewidth{1.003750pt}%
\definecolor{currentstroke}{rgb}{1.000000,0.000000,0.000000}%
\pgfsetstrokecolor{currentstroke}%
\pgfsetdash{}{0pt}%
\pgfpathmoveto{\pgfqpoint{2.356523in}{1.385940in}}%
\pgfpathcurveto{\pgfqpoint{2.367573in}{1.385940in}}{\pgfqpoint{2.378172in}{1.390331in}}{\pgfqpoint{2.385986in}{1.398144in}}%
\pgfpathcurveto{\pgfqpoint{2.393799in}{1.405958in}}{\pgfqpoint{2.398190in}{1.416557in}}{\pgfqpoint{2.398190in}{1.427607in}}%
\pgfpathcurveto{\pgfqpoint{2.398190in}{1.438657in}}{\pgfqpoint{2.393799in}{1.449256in}}{\pgfqpoint{2.385986in}{1.457070in}}%
\pgfpathcurveto{\pgfqpoint{2.378172in}{1.464883in}}{\pgfqpoint{2.367573in}{1.469274in}}{\pgfqpoint{2.356523in}{1.469274in}}%
\pgfpathcurveto{\pgfqpoint{2.345473in}{1.469274in}}{\pgfqpoint{2.334874in}{1.464883in}}{\pgfqpoint{2.327060in}{1.457070in}}%
\pgfpathcurveto{\pgfqpoint{2.319247in}{1.449256in}}{\pgfqpoint{2.314856in}{1.438657in}}{\pgfqpoint{2.314856in}{1.427607in}}%
\pgfpathcurveto{\pgfqpoint{2.314856in}{1.416557in}}{\pgfqpoint{2.319247in}{1.405958in}}{\pgfqpoint{2.327060in}{1.398144in}}%
\pgfpathcurveto{\pgfqpoint{2.334874in}{1.390331in}}{\pgfqpoint{2.345473in}{1.385940in}}{\pgfqpoint{2.356523in}{1.385940in}}%
\pgfpathlineto{\pgfqpoint{2.356523in}{1.385940in}}%
\pgfpathclose%
\pgfusepath{stroke}%
\end{pgfscope}%
\begin{pgfscope}%
\pgfpathrectangle{\pgfqpoint{0.847223in}{0.554012in}}{\pgfqpoint{6.200000in}{4.620000in}}%
\pgfusepath{clip}%
\pgfsetbuttcap%
\pgfsetroundjoin%
\pgfsetlinewidth{1.003750pt}%
\definecolor{currentstroke}{rgb}{1.000000,0.000000,0.000000}%
\pgfsetstrokecolor{currentstroke}%
\pgfsetdash{}{0pt}%
\pgfpathmoveto{\pgfqpoint{2.361856in}{1.382541in}}%
\pgfpathcurveto{\pgfqpoint{2.372906in}{1.382541in}}{\pgfqpoint{2.383505in}{1.386931in}}{\pgfqpoint{2.391319in}{1.394745in}}%
\pgfpathcurveto{\pgfqpoint{2.399133in}{1.402558in}}{\pgfqpoint{2.403523in}{1.413157in}}{\pgfqpoint{2.403523in}{1.424208in}}%
\pgfpathcurveto{\pgfqpoint{2.403523in}{1.435258in}}{\pgfqpoint{2.399133in}{1.445857in}}{\pgfqpoint{2.391319in}{1.453670in}}%
\pgfpathcurveto{\pgfqpoint{2.383505in}{1.461484in}}{\pgfqpoint{2.372906in}{1.465874in}}{\pgfqpoint{2.361856in}{1.465874in}}%
\pgfpathcurveto{\pgfqpoint{2.350806in}{1.465874in}}{\pgfqpoint{2.340207in}{1.461484in}}{\pgfqpoint{2.332394in}{1.453670in}}%
\pgfpathcurveto{\pgfqpoint{2.324580in}{1.445857in}}{\pgfqpoint{2.320190in}{1.435258in}}{\pgfqpoint{2.320190in}{1.424208in}}%
\pgfpathcurveto{\pgfqpoint{2.320190in}{1.413157in}}{\pgfqpoint{2.324580in}{1.402558in}}{\pgfqpoint{2.332394in}{1.394745in}}%
\pgfpathcurveto{\pgfqpoint{2.340207in}{1.386931in}}{\pgfqpoint{2.350806in}{1.382541in}}{\pgfqpoint{2.361856in}{1.382541in}}%
\pgfpathlineto{\pgfqpoint{2.361856in}{1.382541in}}%
\pgfpathclose%
\pgfusepath{stroke}%
\end{pgfscope}%
\begin{pgfscope}%
\pgfpathrectangle{\pgfqpoint{0.847223in}{0.554012in}}{\pgfqpoint{6.200000in}{4.620000in}}%
\pgfusepath{clip}%
\pgfsetbuttcap%
\pgfsetroundjoin%
\pgfsetlinewidth{1.003750pt}%
\definecolor{currentstroke}{rgb}{1.000000,0.000000,0.000000}%
\pgfsetstrokecolor{currentstroke}%
\pgfsetdash{}{0pt}%
\pgfpathmoveto{\pgfqpoint{2.367190in}{1.379159in}}%
\pgfpathcurveto{\pgfqpoint{2.378240in}{1.379159in}}{\pgfqpoint{2.388839in}{1.383549in}}{\pgfqpoint{2.396652in}{1.391363in}}%
\pgfpathcurveto{\pgfqpoint{2.404466in}{1.399177in}}{\pgfqpoint{2.408856in}{1.409776in}}{\pgfqpoint{2.408856in}{1.420826in}}%
\pgfpathcurveto{\pgfqpoint{2.408856in}{1.431876in}}{\pgfqpoint{2.404466in}{1.442475in}}{\pgfqpoint{2.396652in}{1.450289in}}%
\pgfpathcurveto{\pgfqpoint{2.388839in}{1.458102in}}{\pgfqpoint{2.378240in}{1.462492in}}{\pgfqpoint{2.367190in}{1.462492in}}%
\pgfpathcurveto{\pgfqpoint{2.356139in}{1.462492in}}{\pgfqpoint{2.345540in}{1.458102in}}{\pgfqpoint{2.337727in}{1.450289in}}%
\pgfpathcurveto{\pgfqpoint{2.329913in}{1.442475in}}{\pgfqpoint{2.325523in}{1.431876in}}{\pgfqpoint{2.325523in}{1.420826in}}%
\pgfpathcurveto{\pgfqpoint{2.325523in}{1.409776in}}{\pgfqpoint{2.329913in}{1.399177in}}{\pgfqpoint{2.337727in}{1.391363in}}%
\pgfpathcurveto{\pgfqpoint{2.345540in}{1.383549in}}{\pgfqpoint{2.356139in}{1.379159in}}{\pgfqpoint{2.367190in}{1.379159in}}%
\pgfpathlineto{\pgfqpoint{2.367190in}{1.379159in}}%
\pgfpathclose%
\pgfusepath{stroke}%
\end{pgfscope}%
\begin{pgfscope}%
\pgfpathrectangle{\pgfqpoint{0.847223in}{0.554012in}}{\pgfqpoint{6.200000in}{4.620000in}}%
\pgfusepath{clip}%
\pgfsetbuttcap%
\pgfsetroundjoin%
\pgfsetlinewidth{1.003750pt}%
\definecolor{currentstroke}{rgb}{1.000000,0.000000,0.000000}%
\pgfsetstrokecolor{currentstroke}%
\pgfsetdash{}{0pt}%
\pgfpathmoveto{\pgfqpoint{2.372523in}{1.375795in}}%
\pgfpathcurveto{\pgfqpoint{2.383573in}{1.375795in}}{\pgfqpoint{2.394172in}{1.380185in}}{\pgfqpoint{2.401986in}{1.387999in}}%
\pgfpathcurveto{\pgfqpoint{2.409799in}{1.395812in}}{\pgfqpoint{2.414189in}{1.406411in}}{\pgfqpoint{2.414189in}{1.417462in}}%
\pgfpathcurveto{\pgfqpoint{2.414189in}{1.428512in}}{\pgfqpoint{2.409799in}{1.439111in}}{\pgfqpoint{2.401986in}{1.446924in}}%
\pgfpathcurveto{\pgfqpoint{2.394172in}{1.454738in}}{\pgfqpoint{2.383573in}{1.459128in}}{\pgfqpoint{2.372523in}{1.459128in}}%
\pgfpathcurveto{\pgfqpoint{2.361473in}{1.459128in}}{\pgfqpoint{2.350874in}{1.454738in}}{\pgfqpoint{2.343060in}{1.446924in}}%
\pgfpathcurveto{\pgfqpoint{2.335246in}{1.439111in}}{\pgfqpoint{2.330856in}{1.428512in}}{\pgfqpoint{2.330856in}{1.417462in}}%
\pgfpathcurveto{\pgfqpoint{2.330856in}{1.406411in}}{\pgfqpoint{2.335246in}{1.395812in}}{\pgfqpoint{2.343060in}{1.387999in}}%
\pgfpathcurveto{\pgfqpoint{2.350874in}{1.380185in}}{\pgfqpoint{2.361473in}{1.375795in}}{\pgfqpoint{2.372523in}{1.375795in}}%
\pgfpathlineto{\pgfqpoint{2.372523in}{1.375795in}}%
\pgfpathclose%
\pgfusepath{stroke}%
\end{pgfscope}%
\begin{pgfscope}%
\pgfpathrectangle{\pgfqpoint{0.847223in}{0.554012in}}{\pgfqpoint{6.200000in}{4.620000in}}%
\pgfusepath{clip}%
\pgfsetbuttcap%
\pgfsetroundjoin%
\pgfsetlinewidth{1.003750pt}%
\definecolor{currentstroke}{rgb}{1.000000,0.000000,0.000000}%
\pgfsetstrokecolor{currentstroke}%
\pgfsetdash{}{0pt}%
\pgfpathmoveto{\pgfqpoint{2.377856in}{1.372448in}}%
\pgfpathcurveto{\pgfqpoint{2.388906in}{1.372448in}}{\pgfqpoint{2.399505in}{1.376838in}}{\pgfqpoint{2.407319in}{1.384652in}}%
\pgfpathcurveto{\pgfqpoint{2.415132in}{1.392466in}}{\pgfqpoint{2.419523in}{1.403065in}}{\pgfqpoint{2.419523in}{1.414115in}}%
\pgfpathcurveto{\pgfqpoint{2.419523in}{1.425165in}}{\pgfqpoint{2.415132in}{1.435764in}}{\pgfqpoint{2.407319in}{1.443578in}}%
\pgfpathcurveto{\pgfqpoint{2.399505in}{1.451391in}}{\pgfqpoint{2.388906in}{1.455781in}}{\pgfqpoint{2.377856in}{1.455781in}}%
\pgfpathcurveto{\pgfqpoint{2.366806in}{1.455781in}}{\pgfqpoint{2.356207in}{1.451391in}}{\pgfqpoint{2.348393in}{1.443578in}}%
\pgfpathcurveto{\pgfqpoint{2.340580in}{1.435764in}}{\pgfqpoint{2.336189in}{1.425165in}}{\pgfqpoint{2.336189in}{1.414115in}}%
\pgfpathcurveto{\pgfqpoint{2.336189in}{1.403065in}}{\pgfqpoint{2.340580in}{1.392466in}}{\pgfqpoint{2.348393in}{1.384652in}}%
\pgfpathcurveto{\pgfqpoint{2.356207in}{1.376838in}}{\pgfqpoint{2.366806in}{1.372448in}}{\pgfqpoint{2.377856in}{1.372448in}}%
\pgfpathlineto{\pgfqpoint{2.377856in}{1.372448in}}%
\pgfpathclose%
\pgfusepath{stroke}%
\end{pgfscope}%
\begin{pgfscope}%
\pgfpathrectangle{\pgfqpoint{0.847223in}{0.554012in}}{\pgfqpoint{6.200000in}{4.620000in}}%
\pgfusepath{clip}%
\pgfsetbuttcap%
\pgfsetroundjoin%
\pgfsetlinewidth{1.003750pt}%
\definecolor{currentstroke}{rgb}{1.000000,0.000000,0.000000}%
\pgfsetstrokecolor{currentstroke}%
\pgfsetdash{}{0pt}%
\pgfpathmoveto{\pgfqpoint{2.383189in}{1.369119in}}%
\pgfpathcurveto{\pgfqpoint{2.394239in}{1.369119in}}{\pgfqpoint{2.404838in}{1.373509in}}{\pgfqpoint{2.412652in}{1.381323in}}%
\pgfpathcurveto{\pgfqpoint{2.420466in}{1.389136in}}{\pgfqpoint{2.424856in}{1.399735in}}{\pgfqpoint{2.424856in}{1.410785in}}%
\pgfpathcurveto{\pgfqpoint{2.424856in}{1.421836in}}{\pgfqpoint{2.420466in}{1.432435in}}{\pgfqpoint{2.412652in}{1.440248in}}%
\pgfpathcurveto{\pgfqpoint{2.404838in}{1.448062in}}{\pgfqpoint{2.394239in}{1.452452in}}{\pgfqpoint{2.383189in}{1.452452in}}%
\pgfpathcurveto{\pgfqpoint{2.372139in}{1.452452in}}{\pgfqpoint{2.361540in}{1.448062in}}{\pgfqpoint{2.353726in}{1.440248in}}%
\pgfpathcurveto{\pgfqpoint{2.345913in}{1.432435in}}{\pgfqpoint{2.341522in}{1.421836in}}{\pgfqpoint{2.341522in}{1.410785in}}%
\pgfpathcurveto{\pgfqpoint{2.341522in}{1.399735in}}{\pgfqpoint{2.345913in}{1.389136in}}{\pgfqpoint{2.353726in}{1.381323in}}%
\pgfpathcurveto{\pgfqpoint{2.361540in}{1.373509in}}{\pgfqpoint{2.372139in}{1.369119in}}{\pgfqpoint{2.383189in}{1.369119in}}%
\pgfpathlineto{\pgfqpoint{2.383189in}{1.369119in}}%
\pgfpathclose%
\pgfusepath{stroke}%
\end{pgfscope}%
\begin{pgfscope}%
\pgfpathrectangle{\pgfqpoint{0.847223in}{0.554012in}}{\pgfqpoint{6.200000in}{4.620000in}}%
\pgfusepath{clip}%
\pgfsetbuttcap%
\pgfsetroundjoin%
\pgfsetlinewidth{1.003750pt}%
\definecolor{currentstroke}{rgb}{1.000000,0.000000,0.000000}%
\pgfsetstrokecolor{currentstroke}%
\pgfsetdash{}{0pt}%
\pgfpathmoveto{\pgfqpoint{2.388522in}{1.365807in}}%
\pgfpathcurveto{\pgfqpoint{2.399573in}{1.365807in}}{\pgfqpoint{2.410172in}{1.370197in}}{\pgfqpoint{2.417985in}{1.378010in}}%
\pgfpathcurveto{\pgfqpoint{2.425799in}{1.385824in}}{\pgfqpoint{2.430189in}{1.396423in}}{\pgfqpoint{2.430189in}{1.407473in}}%
\pgfpathcurveto{\pgfqpoint{2.430189in}{1.418523in}}{\pgfqpoint{2.425799in}{1.429122in}}{\pgfqpoint{2.417985in}{1.436936in}}%
\pgfpathcurveto{\pgfqpoint{2.410172in}{1.444750in}}{\pgfqpoint{2.399573in}{1.449140in}}{\pgfqpoint{2.388522in}{1.449140in}}%
\pgfpathcurveto{\pgfqpoint{2.377472in}{1.449140in}}{\pgfqpoint{2.366873in}{1.444750in}}{\pgfqpoint{2.359060in}{1.436936in}}%
\pgfpathcurveto{\pgfqpoint{2.351246in}{1.429122in}}{\pgfqpoint{2.346856in}{1.418523in}}{\pgfqpoint{2.346856in}{1.407473in}}%
\pgfpathcurveto{\pgfqpoint{2.346856in}{1.396423in}}{\pgfqpoint{2.351246in}{1.385824in}}{\pgfqpoint{2.359060in}{1.378010in}}%
\pgfpathcurveto{\pgfqpoint{2.366873in}{1.370197in}}{\pgfqpoint{2.377472in}{1.365807in}}{\pgfqpoint{2.388522in}{1.365807in}}%
\pgfpathlineto{\pgfqpoint{2.388522in}{1.365807in}}%
\pgfpathclose%
\pgfusepath{stroke}%
\end{pgfscope}%
\begin{pgfscope}%
\pgfpathrectangle{\pgfqpoint{0.847223in}{0.554012in}}{\pgfqpoint{6.200000in}{4.620000in}}%
\pgfusepath{clip}%
\pgfsetbuttcap%
\pgfsetroundjoin%
\pgfsetlinewidth{1.003750pt}%
\definecolor{currentstroke}{rgb}{1.000000,0.000000,0.000000}%
\pgfsetstrokecolor{currentstroke}%
\pgfsetdash{}{0pt}%
\pgfpathmoveto{\pgfqpoint{2.393856in}{1.362511in}}%
\pgfpathcurveto{\pgfqpoint{2.404906in}{1.362511in}}{\pgfqpoint{2.415505in}{1.366902in}}{\pgfqpoint{2.423318in}{1.374715in}}%
\pgfpathcurveto{\pgfqpoint{2.431132in}{1.382529in}}{\pgfqpoint{2.435522in}{1.393128in}}{\pgfqpoint{2.435522in}{1.404178in}}%
\pgfpathcurveto{\pgfqpoint{2.435522in}{1.415228in}}{\pgfqpoint{2.431132in}{1.425827in}}{\pgfqpoint{2.423318in}{1.433641in}}%
\pgfpathcurveto{\pgfqpoint{2.415505in}{1.441455in}}{\pgfqpoint{2.404906in}{1.445845in}}{\pgfqpoint{2.393856in}{1.445845in}}%
\pgfpathcurveto{\pgfqpoint{2.382805in}{1.445845in}}{\pgfqpoint{2.372206in}{1.441455in}}{\pgfqpoint{2.364393in}{1.433641in}}%
\pgfpathcurveto{\pgfqpoint{2.356579in}{1.425827in}}{\pgfqpoint{2.352189in}{1.415228in}}{\pgfqpoint{2.352189in}{1.404178in}}%
\pgfpathcurveto{\pgfqpoint{2.352189in}{1.393128in}}{\pgfqpoint{2.356579in}{1.382529in}}{\pgfqpoint{2.364393in}{1.374715in}}%
\pgfpathcurveto{\pgfqpoint{2.372206in}{1.366902in}}{\pgfqpoint{2.382805in}{1.362511in}}{\pgfqpoint{2.393856in}{1.362511in}}%
\pgfpathlineto{\pgfqpoint{2.393856in}{1.362511in}}%
\pgfpathclose%
\pgfusepath{stroke}%
\end{pgfscope}%
\begin{pgfscope}%
\pgfpathrectangle{\pgfqpoint{0.847223in}{0.554012in}}{\pgfqpoint{6.200000in}{4.620000in}}%
\pgfusepath{clip}%
\pgfsetbuttcap%
\pgfsetroundjoin%
\pgfsetlinewidth{1.003750pt}%
\definecolor{currentstroke}{rgb}{1.000000,0.000000,0.000000}%
\pgfsetstrokecolor{currentstroke}%
\pgfsetdash{}{0pt}%
\pgfpathmoveto{\pgfqpoint{2.399189in}{1.359233in}}%
\pgfpathcurveto{\pgfqpoint{2.410239in}{1.359233in}}{\pgfqpoint{2.420838in}{1.363624in}}{\pgfqpoint{2.428652in}{1.371437in}}%
\pgfpathcurveto{\pgfqpoint{2.436465in}{1.379251in}}{\pgfqpoint{2.440855in}{1.389850in}}{\pgfqpoint{2.440855in}{1.400900in}}%
\pgfpathcurveto{\pgfqpoint{2.440855in}{1.411950in}}{\pgfqpoint{2.436465in}{1.422549in}}{\pgfqpoint{2.428652in}{1.430363in}}%
\pgfpathcurveto{\pgfqpoint{2.420838in}{1.438176in}}{\pgfqpoint{2.410239in}{1.442567in}}{\pgfqpoint{2.399189in}{1.442567in}}%
\pgfpathcurveto{\pgfqpoint{2.388139in}{1.442567in}}{\pgfqpoint{2.377540in}{1.438176in}}{\pgfqpoint{2.369726in}{1.430363in}}%
\pgfpathcurveto{\pgfqpoint{2.361912in}{1.422549in}}{\pgfqpoint{2.357522in}{1.411950in}}{\pgfqpoint{2.357522in}{1.400900in}}%
\pgfpathcurveto{\pgfqpoint{2.357522in}{1.389850in}}{\pgfqpoint{2.361912in}{1.379251in}}{\pgfqpoint{2.369726in}{1.371437in}}%
\pgfpathcurveto{\pgfqpoint{2.377540in}{1.363624in}}{\pgfqpoint{2.388139in}{1.359233in}}{\pgfqpoint{2.399189in}{1.359233in}}%
\pgfpathlineto{\pgfqpoint{2.399189in}{1.359233in}}%
\pgfpathclose%
\pgfusepath{stroke}%
\end{pgfscope}%
\begin{pgfscope}%
\pgfpathrectangle{\pgfqpoint{0.847223in}{0.554012in}}{\pgfqpoint{6.200000in}{4.620000in}}%
\pgfusepath{clip}%
\pgfsetbuttcap%
\pgfsetroundjoin%
\pgfsetlinewidth{1.003750pt}%
\definecolor{currentstroke}{rgb}{1.000000,0.000000,0.000000}%
\pgfsetstrokecolor{currentstroke}%
\pgfsetdash{}{0pt}%
\pgfpathmoveto{\pgfqpoint{2.404522in}{1.355972in}}%
\pgfpathcurveto{\pgfqpoint{2.415572in}{1.355972in}}{\pgfqpoint{2.426171in}{1.360362in}}{\pgfqpoint{2.433985in}{1.368176in}}%
\pgfpathcurveto{\pgfqpoint{2.441798in}{1.375989in}}{\pgfqpoint{2.446189in}{1.386588in}}{\pgfqpoint{2.446189in}{1.397639in}}%
\pgfpathcurveto{\pgfqpoint{2.446189in}{1.408689in}}{\pgfqpoint{2.441798in}{1.419288in}}{\pgfqpoint{2.433985in}{1.427101in}}%
\pgfpathcurveto{\pgfqpoint{2.426171in}{1.434915in}}{\pgfqpoint{2.415572in}{1.439305in}}{\pgfqpoint{2.404522in}{1.439305in}}%
\pgfpathcurveto{\pgfqpoint{2.393472in}{1.439305in}}{\pgfqpoint{2.382873in}{1.434915in}}{\pgfqpoint{2.375059in}{1.427101in}}%
\pgfpathcurveto{\pgfqpoint{2.367246in}{1.419288in}}{\pgfqpoint{2.362855in}{1.408689in}}{\pgfqpoint{2.362855in}{1.397639in}}%
\pgfpathcurveto{\pgfqpoint{2.362855in}{1.386588in}}{\pgfqpoint{2.367246in}{1.375989in}}{\pgfqpoint{2.375059in}{1.368176in}}%
\pgfpathcurveto{\pgfqpoint{2.382873in}{1.360362in}}{\pgfqpoint{2.393472in}{1.355972in}}{\pgfqpoint{2.404522in}{1.355972in}}%
\pgfpathlineto{\pgfqpoint{2.404522in}{1.355972in}}%
\pgfpathclose%
\pgfusepath{stroke}%
\end{pgfscope}%
\begin{pgfscope}%
\pgfpathrectangle{\pgfqpoint{0.847223in}{0.554012in}}{\pgfqpoint{6.200000in}{4.620000in}}%
\pgfusepath{clip}%
\pgfsetbuttcap%
\pgfsetroundjoin%
\pgfsetlinewidth{1.003750pt}%
\definecolor{currentstroke}{rgb}{1.000000,0.000000,0.000000}%
\pgfsetstrokecolor{currentstroke}%
\pgfsetdash{}{0pt}%
\pgfpathmoveto{\pgfqpoint{2.409855in}{1.352727in}}%
\pgfpathcurveto{\pgfqpoint{2.420905in}{1.352727in}}{\pgfqpoint{2.431504in}{1.357118in}}{\pgfqpoint{2.439318in}{1.364931in}}%
\pgfpathcurveto{\pgfqpoint{2.447132in}{1.372745in}}{\pgfqpoint{2.451522in}{1.383344in}}{\pgfqpoint{2.451522in}{1.394394in}}%
\pgfpathcurveto{\pgfqpoint{2.451522in}{1.405444in}}{\pgfqpoint{2.447132in}{1.416043in}}{\pgfqpoint{2.439318in}{1.423857in}}%
\pgfpathcurveto{\pgfqpoint{2.431504in}{1.431670in}}{\pgfqpoint{2.420905in}{1.436061in}}{\pgfqpoint{2.409855in}{1.436061in}}%
\pgfpathcurveto{\pgfqpoint{2.398805in}{1.436061in}}{\pgfqpoint{2.388206in}{1.431670in}}{\pgfqpoint{2.380392in}{1.423857in}}%
\pgfpathcurveto{\pgfqpoint{2.372579in}{1.416043in}}{\pgfqpoint{2.368189in}{1.405444in}}{\pgfqpoint{2.368189in}{1.394394in}}%
\pgfpathcurveto{\pgfqpoint{2.368189in}{1.383344in}}{\pgfqpoint{2.372579in}{1.372745in}}{\pgfqpoint{2.380392in}{1.364931in}}%
\pgfpathcurveto{\pgfqpoint{2.388206in}{1.357118in}}{\pgfqpoint{2.398805in}{1.352727in}}{\pgfqpoint{2.409855in}{1.352727in}}%
\pgfpathlineto{\pgfqpoint{2.409855in}{1.352727in}}%
\pgfpathclose%
\pgfusepath{stroke}%
\end{pgfscope}%
\begin{pgfscope}%
\pgfpathrectangle{\pgfqpoint{0.847223in}{0.554012in}}{\pgfqpoint{6.200000in}{4.620000in}}%
\pgfusepath{clip}%
\pgfsetbuttcap%
\pgfsetroundjoin%
\pgfsetlinewidth{1.003750pt}%
\definecolor{currentstroke}{rgb}{1.000000,0.000000,0.000000}%
\pgfsetstrokecolor{currentstroke}%
\pgfsetdash{}{0pt}%
\pgfpathmoveto{\pgfqpoint{2.415188in}{1.349499in}}%
\pgfpathcurveto{\pgfqpoint{2.426239in}{1.349499in}}{\pgfqpoint{2.436838in}{1.353889in}}{\pgfqpoint{2.444651in}{1.361703in}}%
\pgfpathcurveto{\pgfqpoint{2.452465in}{1.369517in}}{\pgfqpoint{2.456855in}{1.380116in}}{\pgfqpoint{2.456855in}{1.391166in}}%
\pgfpathcurveto{\pgfqpoint{2.456855in}{1.402216in}}{\pgfqpoint{2.452465in}{1.412815in}}{\pgfqpoint{2.444651in}{1.420629in}}%
\pgfpathcurveto{\pgfqpoint{2.436838in}{1.428442in}}{\pgfqpoint{2.426239in}{1.432832in}}{\pgfqpoint{2.415188in}{1.432832in}}%
\pgfpathcurveto{\pgfqpoint{2.404138in}{1.432832in}}{\pgfqpoint{2.393539in}{1.428442in}}{\pgfqpoint{2.385726in}{1.420629in}}%
\pgfpathcurveto{\pgfqpoint{2.377912in}{1.412815in}}{\pgfqpoint{2.373522in}{1.402216in}}{\pgfqpoint{2.373522in}{1.391166in}}%
\pgfpathcurveto{\pgfqpoint{2.373522in}{1.380116in}}{\pgfqpoint{2.377912in}{1.369517in}}{\pgfqpoint{2.385726in}{1.361703in}}%
\pgfpathcurveto{\pgfqpoint{2.393539in}{1.353889in}}{\pgfqpoint{2.404138in}{1.349499in}}{\pgfqpoint{2.415188in}{1.349499in}}%
\pgfpathlineto{\pgfqpoint{2.415188in}{1.349499in}}%
\pgfpathclose%
\pgfusepath{stroke}%
\end{pgfscope}%
\begin{pgfscope}%
\pgfpathrectangle{\pgfqpoint{0.847223in}{0.554012in}}{\pgfqpoint{6.200000in}{4.620000in}}%
\pgfusepath{clip}%
\pgfsetbuttcap%
\pgfsetroundjoin%
\pgfsetlinewidth{1.003750pt}%
\definecolor{currentstroke}{rgb}{1.000000,0.000000,0.000000}%
\pgfsetstrokecolor{currentstroke}%
\pgfsetdash{}{0pt}%
\pgfpathmoveto{\pgfqpoint{2.420522in}{1.346287in}}%
\pgfpathcurveto{\pgfqpoint{2.431572in}{1.346287in}}{\pgfqpoint{2.442171in}{1.350678in}}{\pgfqpoint{2.449984in}{1.358491in}}%
\pgfpathcurveto{\pgfqpoint{2.457798in}{1.366305in}}{\pgfqpoint{2.462188in}{1.376904in}}{\pgfqpoint{2.462188in}{1.387954in}}%
\pgfpathcurveto{\pgfqpoint{2.462188in}{1.399004in}}{\pgfqpoint{2.457798in}{1.409603in}}{\pgfqpoint{2.449984in}{1.417417in}}%
\pgfpathcurveto{\pgfqpoint{2.442171in}{1.425230in}}{\pgfqpoint{2.431572in}{1.429621in}}{\pgfqpoint{2.420522in}{1.429621in}}%
\pgfpathcurveto{\pgfqpoint{2.409472in}{1.429621in}}{\pgfqpoint{2.398873in}{1.425230in}}{\pgfqpoint{2.391059in}{1.417417in}}%
\pgfpathcurveto{\pgfqpoint{2.383245in}{1.409603in}}{\pgfqpoint{2.378855in}{1.399004in}}{\pgfqpoint{2.378855in}{1.387954in}}%
\pgfpathcurveto{\pgfqpoint{2.378855in}{1.376904in}}{\pgfqpoint{2.383245in}{1.366305in}}{\pgfqpoint{2.391059in}{1.358491in}}%
\pgfpathcurveto{\pgfqpoint{2.398873in}{1.350678in}}{\pgfqpoint{2.409472in}{1.346287in}}{\pgfqpoint{2.420522in}{1.346287in}}%
\pgfpathlineto{\pgfqpoint{2.420522in}{1.346287in}}%
\pgfpathclose%
\pgfusepath{stroke}%
\end{pgfscope}%
\begin{pgfscope}%
\pgfpathrectangle{\pgfqpoint{0.847223in}{0.554012in}}{\pgfqpoint{6.200000in}{4.620000in}}%
\pgfusepath{clip}%
\pgfsetbuttcap%
\pgfsetroundjoin%
\pgfsetlinewidth{1.003750pt}%
\definecolor{currentstroke}{rgb}{1.000000,0.000000,0.000000}%
\pgfsetstrokecolor{currentstroke}%
\pgfsetdash{}{0pt}%
\pgfpathmoveto{\pgfqpoint{2.425855in}{1.343092in}}%
\pgfpathcurveto{\pgfqpoint{2.436905in}{1.343092in}}{\pgfqpoint{2.447504in}{1.347482in}}{\pgfqpoint{2.455318in}{1.355296in}}%
\pgfpathcurveto{\pgfqpoint{2.463131in}{1.363109in}}{\pgfqpoint{2.467522in}{1.373708in}}{\pgfqpoint{2.467522in}{1.384759in}}%
\pgfpathcurveto{\pgfqpoint{2.467522in}{1.395809in}}{\pgfqpoint{2.463131in}{1.406408in}}{\pgfqpoint{2.455318in}{1.414221in}}%
\pgfpathcurveto{\pgfqpoint{2.447504in}{1.422035in}}{\pgfqpoint{2.436905in}{1.426425in}}{\pgfqpoint{2.425855in}{1.426425in}}%
\pgfpathcurveto{\pgfqpoint{2.414805in}{1.426425in}}{\pgfqpoint{2.404206in}{1.422035in}}{\pgfqpoint{2.396392in}{1.414221in}}%
\pgfpathcurveto{\pgfqpoint{2.388578in}{1.406408in}}{\pgfqpoint{2.384188in}{1.395809in}}{\pgfqpoint{2.384188in}{1.384759in}}%
\pgfpathcurveto{\pgfqpoint{2.384188in}{1.373708in}}{\pgfqpoint{2.388578in}{1.363109in}}{\pgfqpoint{2.396392in}{1.355296in}}%
\pgfpathcurveto{\pgfqpoint{2.404206in}{1.347482in}}{\pgfqpoint{2.414805in}{1.343092in}}{\pgfqpoint{2.425855in}{1.343092in}}%
\pgfpathlineto{\pgfqpoint{2.425855in}{1.343092in}}%
\pgfpathclose%
\pgfusepath{stroke}%
\end{pgfscope}%
\begin{pgfscope}%
\pgfpathrectangle{\pgfqpoint{0.847223in}{0.554012in}}{\pgfqpoint{6.200000in}{4.620000in}}%
\pgfusepath{clip}%
\pgfsetbuttcap%
\pgfsetroundjoin%
\pgfsetlinewidth{1.003750pt}%
\definecolor{currentstroke}{rgb}{1.000000,0.000000,0.000000}%
\pgfsetstrokecolor{currentstroke}%
\pgfsetdash{}{0pt}%
\pgfpathmoveto{\pgfqpoint{2.431188in}{1.339913in}}%
\pgfpathcurveto{\pgfqpoint{2.442238in}{1.339913in}}{\pgfqpoint{2.452837in}{1.344303in}}{\pgfqpoint{2.460651in}{1.352117in}}%
\pgfpathcurveto{\pgfqpoint{2.468465in}{1.359930in}}{\pgfqpoint{2.472855in}{1.370529in}}{\pgfqpoint{2.472855in}{1.381579in}}%
\pgfpathcurveto{\pgfqpoint{2.472855in}{1.392629in}}{\pgfqpoint{2.468465in}{1.403229in}}{\pgfqpoint{2.460651in}{1.411042in}}%
\pgfpathcurveto{\pgfqpoint{2.452837in}{1.418856in}}{\pgfqpoint{2.442238in}{1.423246in}}{\pgfqpoint{2.431188in}{1.423246in}}%
\pgfpathcurveto{\pgfqpoint{2.420138in}{1.423246in}}{\pgfqpoint{2.409539in}{1.418856in}}{\pgfqpoint{2.401725in}{1.411042in}}%
\pgfpathcurveto{\pgfqpoint{2.393912in}{1.403229in}}{\pgfqpoint{2.389521in}{1.392629in}}{\pgfqpoint{2.389521in}{1.381579in}}%
\pgfpathcurveto{\pgfqpoint{2.389521in}{1.370529in}}{\pgfqpoint{2.393912in}{1.359930in}}{\pgfqpoint{2.401725in}{1.352117in}}%
\pgfpathcurveto{\pgfqpoint{2.409539in}{1.344303in}}{\pgfqpoint{2.420138in}{1.339913in}}{\pgfqpoint{2.431188in}{1.339913in}}%
\pgfpathlineto{\pgfqpoint{2.431188in}{1.339913in}}%
\pgfpathclose%
\pgfusepath{stroke}%
\end{pgfscope}%
\begin{pgfscope}%
\pgfpathrectangle{\pgfqpoint{0.847223in}{0.554012in}}{\pgfqpoint{6.200000in}{4.620000in}}%
\pgfusepath{clip}%
\pgfsetbuttcap%
\pgfsetroundjoin%
\pgfsetlinewidth{1.003750pt}%
\definecolor{currentstroke}{rgb}{1.000000,0.000000,0.000000}%
\pgfsetstrokecolor{currentstroke}%
\pgfsetdash{}{0pt}%
\pgfpathmoveto{\pgfqpoint{2.436521in}{1.336749in}}%
\pgfpathcurveto{\pgfqpoint{2.447571in}{1.336749in}}{\pgfqpoint{2.458170in}{1.341140in}}{\pgfqpoint{2.465984in}{1.348953in}}%
\pgfpathcurveto{\pgfqpoint{2.473798in}{1.356767in}}{\pgfqpoint{2.478188in}{1.367366in}}{\pgfqpoint{2.478188in}{1.378416in}}%
\pgfpathcurveto{\pgfqpoint{2.478188in}{1.389466in}}{\pgfqpoint{2.473798in}{1.400065in}}{\pgfqpoint{2.465984in}{1.407879in}}%
\pgfpathcurveto{\pgfqpoint{2.458170in}{1.415693in}}{\pgfqpoint{2.447571in}{1.420083in}}{\pgfqpoint{2.436521in}{1.420083in}}%
\pgfpathcurveto{\pgfqpoint{2.425471in}{1.420083in}}{\pgfqpoint{2.414872in}{1.415693in}}{\pgfqpoint{2.407059in}{1.407879in}}%
\pgfpathcurveto{\pgfqpoint{2.399245in}{1.400065in}}{\pgfqpoint{2.394855in}{1.389466in}}{\pgfqpoint{2.394855in}{1.378416in}}%
\pgfpathcurveto{\pgfqpoint{2.394855in}{1.367366in}}{\pgfqpoint{2.399245in}{1.356767in}}{\pgfqpoint{2.407059in}{1.348953in}}%
\pgfpathcurveto{\pgfqpoint{2.414872in}{1.341140in}}{\pgfqpoint{2.425471in}{1.336749in}}{\pgfqpoint{2.436521in}{1.336749in}}%
\pgfpathlineto{\pgfqpoint{2.436521in}{1.336749in}}%
\pgfpathclose%
\pgfusepath{stroke}%
\end{pgfscope}%
\begin{pgfscope}%
\pgfpathrectangle{\pgfqpoint{0.847223in}{0.554012in}}{\pgfqpoint{6.200000in}{4.620000in}}%
\pgfusepath{clip}%
\pgfsetbuttcap%
\pgfsetroundjoin%
\pgfsetlinewidth{1.003750pt}%
\definecolor{currentstroke}{rgb}{1.000000,0.000000,0.000000}%
\pgfsetstrokecolor{currentstroke}%
\pgfsetdash{}{0pt}%
\pgfpathmoveto{\pgfqpoint{2.441855in}{1.333602in}}%
\pgfpathcurveto{\pgfqpoint{2.452905in}{1.333602in}}{\pgfqpoint{2.463504in}{1.337992in}}{\pgfqpoint{2.471317in}{1.345806in}}%
\pgfpathcurveto{\pgfqpoint{2.479131in}{1.353620in}}{\pgfqpoint{2.483521in}{1.364219in}}{\pgfqpoint{2.483521in}{1.375269in}}%
\pgfpathcurveto{\pgfqpoint{2.483521in}{1.386319in}}{\pgfqpoint{2.479131in}{1.396918in}}{\pgfqpoint{2.471317in}{1.404732in}}%
\pgfpathcurveto{\pgfqpoint{2.463504in}{1.412545in}}{\pgfqpoint{2.452905in}{1.416936in}}{\pgfqpoint{2.441855in}{1.416936in}}%
\pgfpathcurveto{\pgfqpoint{2.430804in}{1.416936in}}{\pgfqpoint{2.420205in}{1.412545in}}{\pgfqpoint{2.412392in}{1.404732in}}%
\pgfpathcurveto{\pgfqpoint{2.404578in}{1.396918in}}{\pgfqpoint{2.400188in}{1.386319in}}{\pgfqpoint{2.400188in}{1.375269in}}%
\pgfpathcurveto{\pgfqpoint{2.400188in}{1.364219in}}{\pgfqpoint{2.404578in}{1.353620in}}{\pgfqpoint{2.412392in}{1.345806in}}%
\pgfpathcurveto{\pgfqpoint{2.420205in}{1.337992in}}{\pgfqpoint{2.430804in}{1.333602in}}{\pgfqpoint{2.441855in}{1.333602in}}%
\pgfpathlineto{\pgfqpoint{2.441855in}{1.333602in}}%
\pgfpathclose%
\pgfusepath{stroke}%
\end{pgfscope}%
\begin{pgfscope}%
\pgfpathrectangle{\pgfqpoint{0.847223in}{0.554012in}}{\pgfqpoint{6.200000in}{4.620000in}}%
\pgfusepath{clip}%
\pgfsetbuttcap%
\pgfsetroundjoin%
\pgfsetlinewidth{1.003750pt}%
\definecolor{currentstroke}{rgb}{1.000000,0.000000,0.000000}%
\pgfsetstrokecolor{currentstroke}%
\pgfsetdash{}{0pt}%
\pgfpathmoveto{\pgfqpoint{2.447188in}{1.330471in}}%
\pgfpathcurveto{\pgfqpoint{2.458238in}{1.330471in}}{\pgfqpoint{2.468837in}{1.334861in}}{\pgfqpoint{2.476651in}{1.342675in}}%
\pgfpathcurveto{\pgfqpoint{2.484464in}{1.350488in}}{\pgfqpoint{2.488854in}{1.361087in}}{\pgfqpoint{2.488854in}{1.372137in}}%
\pgfpathcurveto{\pgfqpoint{2.488854in}{1.383188in}}{\pgfqpoint{2.484464in}{1.393787in}}{\pgfqpoint{2.476651in}{1.401600in}}%
\pgfpathcurveto{\pgfqpoint{2.468837in}{1.409414in}}{\pgfqpoint{2.458238in}{1.413804in}}{\pgfqpoint{2.447188in}{1.413804in}}%
\pgfpathcurveto{\pgfqpoint{2.436138in}{1.413804in}}{\pgfqpoint{2.425539in}{1.409414in}}{\pgfqpoint{2.417725in}{1.401600in}}%
\pgfpathcurveto{\pgfqpoint{2.409911in}{1.393787in}}{\pgfqpoint{2.405521in}{1.383188in}}{\pgfqpoint{2.405521in}{1.372137in}}%
\pgfpathcurveto{\pgfqpoint{2.405521in}{1.361087in}}{\pgfqpoint{2.409911in}{1.350488in}}{\pgfqpoint{2.417725in}{1.342675in}}%
\pgfpathcurveto{\pgfqpoint{2.425539in}{1.334861in}}{\pgfqpoint{2.436138in}{1.330471in}}{\pgfqpoint{2.447188in}{1.330471in}}%
\pgfpathlineto{\pgfqpoint{2.447188in}{1.330471in}}%
\pgfpathclose%
\pgfusepath{stroke}%
\end{pgfscope}%
\begin{pgfscope}%
\pgfpathrectangle{\pgfqpoint{0.847223in}{0.554012in}}{\pgfqpoint{6.200000in}{4.620000in}}%
\pgfusepath{clip}%
\pgfsetbuttcap%
\pgfsetroundjoin%
\pgfsetlinewidth{1.003750pt}%
\definecolor{currentstroke}{rgb}{1.000000,0.000000,0.000000}%
\pgfsetstrokecolor{currentstroke}%
\pgfsetdash{}{0pt}%
\pgfpathmoveto{\pgfqpoint{2.452521in}{1.327355in}}%
\pgfpathcurveto{\pgfqpoint{2.463571in}{1.327355in}}{\pgfqpoint{2.474170in}{1.331745in}}{\pgfqpoint{2.481984in}{1.339559in}}%
\pgfpathcurveto{\pgfqpoint{2.489797in}{1.347373in}}{\pgfqpoint{2.494188in}{1.357972in}}{\pgfqpoint{2.494188in}{1.369022in}}%
\pgfpathcurveto{\pgfqpoint{2.494188in}{1.380072in}}{\pgfqpoint{2.489797in}{1.390671in}}{\pgfqpoint{2.481984in}{1.398484in}}%
\pgfpathcurveto{\pgfqpoint{2.474170in}{1.406298in}}{\pgfqpoint{2.463571in}{1.410688in}}{\pgfqpoint{2.452521in}{1.410688in}}%
\pgfpathcurveto{\pgfqpoint{2.441471in}{1.410688in}}{\pgfqpoint{2.430872in}{1.406298in}}{\pgfqpoint{2.423058in}{1.398484in}}%
\pgfpathcurveto{\pgfqpoint{2.415245in}{1.390671in}}{\pgfqpoint{2.410854in}{1.380072in}}{\pgfqpoint{2.410854in}{1.369022in}}%
\pgfpathcurveto{\pgfqpoint{2.410854in}{1.357972in}}{\pgfqpoint{2.415245in}{1.347373in}}{\pgfqpoint{2.423058in}{1.339559in}}%
\pgfpathcurveto{\pgfqpoint{2.430872in}{1.331745in}}{\pgfqpoint{2.441471in}{1.327355in}}{\pgfqpoint{2.452521in}{1.327355in}}%
\pgfpathlineto{\pgfqpoint{2.452521in}{1.327355in}}%
\pgfpathclose%
\pgfusepath{stroke}%
\end{pgfscope}%
\begin{pgfscope}%
\pgfpathrectangle{\pgfqpoint{0.847223in}{0.554012in}}{\pgfqpoint{6.200000in}{4.620000in}}%
\pgfusepath{clip}%
\pgfsetbuttcap%
\pgfsetroundjoin%
\pgfsetlinewidth{1.003750pt}%
\definecolor{currentstroke}{rgb}{1.000000,0.000000,0.000000}%
\pgfsetstrokecolor{currentstroke}%
\pgfsetdash{}{0pt}%
\pgfpathmoveto{\pgfqpoint{2.457854in}{1.324255in}}%
\pgfpathcurveto{\pgfqpoint{2.468904in}{1.324255in}}{\pgfqpoint{2.479503in}{1.328645in}}{\pgfqpoint{2.487317in}{1.336459in}}%
\pgfpathcurveto{\pgfqpoint{2.495131in}{1.344272in}}{\pgfqpoint{2.499521in}{1.354871in}}{\pgfqpoint{2.499521in}{1.365921in}}%
\pgfpathcurveto{\pgfqpoint{2.499521in}{1.376972in}}{\pgfqpoint{2.495131in}{1.387571in}}{\pgfqpoint{2.487317in}{1.395384in}}%
\pgfpathcurveto{\pgfqpoint{2.479503in}{1.403198in}}{\pgfqpoint{2.468904in}{1.407588in}}{\pgfqpoint{2.457854in}{1.407588in}}%
\pgfpathcurveto{\pgfqpoint{2.446804in}{1.407588in}}{\pgfqpoint{2.436205in}{1.403198in}}{\pgfqpoint{2.428391in}{1.395384in}}%
\pgfpathcurveto{\pgfqpoint{2.420578in}{1.387571in}}{\pgfqpoint{2.416188in}{1.376972in}}{\pgfqpoint{2.416188in}{1.365921in}}%
\pgfpathcurveto{\pgfqpoint{2.416188in}{1.354871in}}{\pgfqpoint{2.420578in}{1.344272in}}{\pgfqpoint{2.428391in}{1.336459in}}%
\pgfpathcurveto{\pgfqpoint{2.436205in}{1.328645in}}{\pgfqpoint{2.446804in}{1.324255in}}{\pgfqpoint{2.457854in}{1.324255in}}%
\pgfpathlineto{\pgfqpoint{2.457854in}{1.324255in}}%
\pgfpathclose%
\pgfusepath{stroke}%
\end{pgfscope}%
\begin{pgfscope}%
\pgfpathrectangle{\pgfqpoint{0.847223in}{0.554012in}}{\pgfqpoint{6.200000in}{4.620000in}}%
\pgfusepath{clip}%
\pgfsetbuttcap%
\pgfsetroundjoin%
\pgfsetlinewidth{1.003750pt}%
\definecolor{currentstroke}{rgb}{1.000000,0.000000,0.000000}%
\pgfsetstrokecolor{currentstroke}%
\pgfsetdash{}{0pt}%
\pgfpathmoveto{\pgfqpoint{2.463187in}{1.321170in}}%
\pgfpathcurveto{\pgfqpoint{2.474238in}{1.321170in}}{\pgfqpoint{2.484837in}{1.325560in}}{\pgfqpoint{2.492650in}{1.333374in}}%
\pgfpathcurveto{\pgfqpoint{2.500464in}{1.341188in}}{\pgfqpoint{2.504854in}{1.351787in}}{\pgfqpoint{2.504854in}{1.362837in}}%
\pgfpathcurveto{\pgfqpoint{2.504854in}{1.373887in}}{\pgfqpoint{2.500464in}{1.384486in}}{\pgfqpoint{2.492650in}{1.392300in}}%
\pgfpathcurveto{\pgfqpoint{2.484837in}{1.400113in}}{\pgfqpoint{2.474238in}{1.404503in}}{\pgfqpoint{2.463187in}{1.404503in}}%
\pgfpathcurveto{\pgfqpoint{2.452137in}{1.404503in}}{\pgfqpoint{2.441538in}{1.400113in}}{\pgfqpoint{2.433725in}{1.392300in}}%
\pgfpathcurveto{\pgfqpoint{2.425911in}{1.384486in}}{\pgfqpoint{2.421521in}{1.373887in}}{\pgfqpoint{2.421521in}{1.362837in}}%
\pgfpathcurveto{\pgfqpoint{2.421521in}{1.351787in}}{\pgfqpoint{2.425911in}{1.341188in}}{\pgfqpoint{2.433725in}{1.333374in}}%
\pgfpathcurveto{\pgfqpoint{2.441538in}{1.325560in}}{\pgfqpoint{2.452137in}{1.321170in}}{\pgfqpoint{2.463187in}{1.321170in}}%
\pgfpathlineto{\pgfqpoint{2.463187in}{1.321170in}}%
\pgfpathclose%
\pgfusepath{stroke}%
\end{pgfscope}%
\begin{pgfscope}%
\pgfpathrectangle{\pgfqpoint{0.847223in}{0.554012in}}{\pgfqpoint{6.200000in}{4.620000in}}%
\pgfusepath{clip}%
\pgfsetbuttcap%
\pgfsetroundjoin%
\pgfsetlinewidth{1.003750pt}%
\definecolor{currentstroke}{rgb}{1.000000,0.000000,0.000000}%
\pgfsetstrokecolor{currentstroke}%
\pgfsetdash{}{0pt}%
\pgfpathmoveto{\pgfqpoint{2.468521in}{1.318101in}}%
\pgfpathcurveto{\pgfqpoint{2.479571in}{1.318101in}}{\pgfqpoint{2.490170in}{1.322491in}}{\pgfqpoint{2.497983in}{1.330305in}}%
\pgfpathcurveto{\pgfqpoint{2.505797in}{1.338118in}}{\pgfqpoint{2.510187in}{1.348717in}}{\pgfqpoint{2.510187in}{1.359767in}}%
\pgfpathcurveto{\pgfqpoint{2.510187in}{1.370817in}}{\pgfqpoint{2.505797in}{1.381417in}}{\pgfqpoint{2.497983in}{1.389230in}}%
\pgfpathcurveto{\pgfqpoint{2.490170in}{1.397044in}}{\pgfqpoint{2.479571in}{1.401434in}}{\pgfqpoint{2.468521in}{1.401434in}}%
\pgfpathcurveto{\pgfqpoint{2.457470in}{1.401434in}}{\pgfqpoint{2.446871in}{1.397044in}}{\pgfqpoint{2.439058in}{1.389230in}}%
\pgfpathcurveto{\pgfqpoint{2.431244in}{1.381417in}}{\pgfqpoint{2.426854in}{1.370817in}}{\pgfqpoint{2.426854in}{1.359767in}}%
\pgfpathcurveto{\pgfqpoint{2.426854in}{1.348717in}}{\pgfqpoint{2.431244in}{1.338118in}}{\pgfqpoint{2.439058in}{1.330305in}}%
\pgfpathcurveto{\pgfqpoint{2.446871in}{1.322491in}}{\pgfqpoint{2.457470in}{1.318101in}}{\pgfqpoint{2.468521in}{1.318101in}}%
\pgfpathlineto{\pgfqpoint{2.468521in}{1.318101in}}%
\pgfpathclose%
\pgfusepath{stroke}%
\end{pgfscope}%
\begin{pgfscope}%
\pgfpathrectangle{\pgfqpoint{0.847223in}{0.554012in}}{\pgfqpoint{6.200000in}{4.620000in}}%
\pgfusepath{clip}%
\pgfsetbuttcap%
\pgfsetroundjoin%
\pgfsetlinewidth{1.003750pt}%
\definecolor{currentstroke}{rgb}{1.000000,0.000000,0.000000}%
\pgfsetstrokecolor{currentstroke}%
\pgfsetdash{}{0pt}%
\pgfpathmoveto{\pgfqpoint{2.473854in}{1.315047in}}%
\pgfpathcurveto{\pgfqpoint{2.484904in}{1.315047in}}{\pgfqpoint{2.495503in}{1.319437in}}{\pgfqpoint{2.503317in}{1.327250in}}%
\pgfpathcurveto{\pgfqpoint{2.511130in}{1.335064in}}{\pgfqpoint{2.515520in}{1.345663in}}{\pgfqpoint{2.515520in}{1.356713in}}%
\pgfpathcurveto{\pgfqpoint{2.515520in}{1.367763in}}{\pgfqpoint{2.511130in}{1.378362in}}{\pgfqpoint{2.503317in}{1.386176in}}%
\pgfpathcurveto{\pgfqpoint{2.495503in}{1.393990in}}{\pgfqpoint{2.484904in}{1.398380in}}{\pgfqpoint{2.473854in}{1.398380in}}%
\pgfpathcurveto{\pgfqpoint{2.462804in}{1.398380in}}{\pgfqpoint{2.452205in}{1.393990in}}{\pgfqpoint{2.444391in}{1.386176in}}%
\pgfpathcurveto{\pgfqpoint{2.436577in}{1.378362in}}{\pgfqpoint{2.432187in}{1.367763in}}{\pgfqpoint{2.432187in}{1.356713in}}%
\pgfpathcurveto{\pgfqpoint{2.432187in}{1.345663in}}{\pgfqpoint{2.436577in}{1.335064in}}{\pgfqpoint{2.444391in}{1.327250in}}%
\pgfpathcurveto{\pgfqpoint{2.452205in}{1.319437in}}{\pgfqpoint{2.462804in}{1.315047in}}{\pgfqpoint{2.473854in}{1.315047in}}%
\pgfpathlineto{\pgfqpoint{2.473854in}{1.315047in}}%
\pgfpathclose%
\pgfusepath{stroke}%
\end{pgfscope}%
\begin{pgfscope}%
\pgfpathrectangle{\pgfqpoint{0.847223in}{0.554012in}}{\pgfqpoint{6.200000in}{4.620000in}}%
\pgfusepath{clip}%
\pgfsetbuttcap%
\pgfsetroundjoin%
\pgfsetlinewidth{1.003750pt}%
\definecolor{currentstroke}{rgb}{1.000000,0.000000,0.000000}%
\pgfsetstrokecolor{currentstroke}%
\pgfsetdash{}{0pt}%
\pgfpathmoveto{\pgfqpoint{2.479187in}{1.312007in}}%
\pgfpathcurveto{\pgfqpoint{2.490237in}{1.312007in}}{\pgfqpoint{2.500836in}{1.316398in}}{\pgfqpoint{2.508650in}{1.324211in}}%
\pgfpathcurveto{\pgfqpoint{2.516463in}{1.332025in}}{\pgfqpoint{2.520854in}{1.342624in}}{\pgfqpoint{2.520854in}{1.353674in}}%
\pgfpathcurveto{\pgfqpoint{2.520854in}{1.364724in}}{\pgfqpoint{2.516463in}{1.375323in}}{\pgfqpoint{2.508650in}{1.383137in}}%
\pgfpathcurveto{\pgfqpoint{2.500836in}{1.390951in}}{\pgfqpoint{2.490237in}{1.395341in}}{\pgfqpoint{2.479187in}{1.395341in}}%
\pgfpathcurveto{\pgfqpoint{2.468137in}{1.395341in}}{\pgfqpoint{2.457538in}{1.390951in}}{\pgfqpoint{2.449724in}{1.383137in}}%
\pgfpathcurveto{\pgfqpoint{2.441911in}{1.375323in}}{\pgfqpoint{2.437520in}{1.364724in}}{\pgfqpoint{2.437520in}{1.353674in}}%
\pgfpathcurveto{\pgfqpoint{2.437520in}{1.342624in}}{\pgfqpoint{2.441911in}{1.332025in}}{\pgfqpoint{2.449724in}{1.324211in}}%
\pgfpathcurveto{\pgfqpoint{2.457538in}{1.316398in}}{\pgfqpoint{2.468137in}{1.312007in}}{\pgfqpoint{2.479187in}{1.312007in}}%
\pgfpathlineto{\pgfqpoint{2.479187in}{1.312007in}}%
\pgfpathclose%
\pgfusepath{stroke}%
\end{pgfscope}%
\begin{pgfscope}%
\pgfpathrectangle{\pgfqpoint{0.847223in}{0.554012in}}{\pgfqpoint{6.200000in}{4.620000in}}%
\pgfusepath{clip}%
\pgfsetbuttcap%
\pgfsetroundjoin%
\pgfsetlinewidth{1.003750pt}%
\definecolor{currentstroke}{rgb}{1.000000,0.000000,0.000000}%
\pgfsetstrokecolor{currentstroke}%
\pgfsetdash{}{0pt}%
\pgfpathmoveto{\pgfqpoint{2.484520in}{1.308983in}}%
\pgfpathcurveto{\pgfqpoint{2.495570in}{1.308983in}}{\pgfqpoint{2.506169in}{1.313374in}}{\pgfqpoint{2.513983in}{1.321187in}}%
\pgfpathcurveto{\pgfqpoint{2.521797in}{1.329001in}}{\pgfqpoint{2.526187in}{1.339600in}}{\pgfqpoint{2.526187in}{1.350650in}}%
\pgfpathcurveto{\pgfqpoint{2.526187in}{1.361700in}}{\pgfqpoint{2.521797in}{1.372299in}}{\pgfqpoint{2.513983in}{1.380113in}}%
\pgfpathcurveto{\pgfqpoint{2.506169in}{1.387926in}}{\pgfqpoint{2.495570in}{1.392317in}}{\pgfqpoint{2.484520in}{1.392317in}}%
\pgfpathcurveto{\pgfqpoint{2.473470in}{1.392317in}}{\pgfqpoint{2.462871in}{1.387926in}}{\pgfqpoint{2.455057in}{1.380113in}}%
\pgfpathcurveto{\pgfqpoint{2.447244in}{1.372299in}}{\pgfqpoint{2.442854in}{1.361700in}}{\pgfqpoint{2.442854in}{1.350650in}}%
\pgfpathcurveto{\pgfqpoint{2.442854in}{1.339600in}}{\pgfqpoint{2.447244in}{1.329001in}}{\pgfqpoint{2.455057in}{1.321187in}}%
\pgfpathcurveto{\pgfqpoint{2.462871in}{1.313374in}}{\pgfqpoint{2.473470in}{1.308983in}}{\pgfqpoint{2.484520in}{1.308983in}}%
\pgfpathlineto{\pgfqpoint{2.484520in}{1.308983in}}%
\pgfpathclose%
\pgfusepath{stroke}%
\end{pgfscope}%
\begin{pgfscope}%
\pgfpathrectangle{\pgfqpoint{0.847223in}{0.554012in}}{\pgfqpoint{6.200000in}{4.620000in}}%
\pgfusepath{clip}%
\pgfsetbuttcap%
\pgfsetroundjoin%
\pgfsetlinewidth{1.003750pt}%
\definecolor{currentstroke}{rgb}{1.000000,0.000000,0.000000}%
\pgfsetstrokecolor{currentstroke}%
\pgfsetdash{}{0pt}%
\pgfpathmoveto{\pgfqpoint{2.489853in}{1.305974in}}%
\pgfpathcurveto{\pgfqpoint{2.500904in}{1.305974in}}{\pgfqpoint{2.511503in}{1.310365in}}{\pgfqpoint{2.519316in}{1.318178in}}%
\pgfpathcurveto{\pgfqpoint{2.527130in}{1.325992in}}{\pgfqpoint{2.531520in}{1.336591in}}{\pgfqpoint{2.531520in}{1.347641in}}%
\pgfpathcurveto{\pgfqpoint{2.531520in}{1.358691in}}{\pgfqpoint{2.527130in}{1.369290in}}{\pgfqpoint{2.519316in}{1.377104in}}%
\pgfpathcurveto{\pgfqpoint{2.511503in}{1.384917in}}{\pgfqpoint{2.500904in}{1.389308in}}{\pgfqpoint{2.489853in}{1.389308in}}%
\pgfpathcurveto{\pgfqpoint{2.478803in}{1.389308in}}{\pgfqpoint{2.468204in}{1.384917in}}{\pgfqpoint{2.460391in}{1.377104in}}%
\pgfpathcurveto{\pgfqpoint{2.452577in}{1.369290in}}{\pgfqpoint{2.448187in}{1.358691in}}{\pgfqpoint{2.448187in}{1.347641in}}%
\pgfpathcurveto{\pgfqpoint{2.448187in}{1.336591in}}{\pgfqpoint{2.452577in}{1.325992in}}{\pgfqpoint{2.460391in}{1.318178in}}%
\pgfpathcurveto{\pgfqpoint{2.468204in}{1.310365in}}{\pgfqpoint{2.478803in}{1.305974in}}{\pgfqpoint{2.489853in}{1.305974in}}%
\pgfpathlineto{\pgfqpoint{2.489853in}{1.305974in}}%
\pgfpathclose%
\pgfusepath{stroke}%
\end{pgfscope}%
\begin{pgfscope}%
\pgfpathrectangle{\pgfqpoint{0.847223in}{0.554012in}}{\pgfqpoint{6.200000in}{4.620000in}}%
\pgfusepath{clip}%
\pgfsetbuttcap%
\pgfsetroundjoin%
\pgfsetlinewidth{1.003750pt}%
\definecolor{currentstroke}{rgb}{1.000000,0.000000,0.000000}%
\pgfsetstrokecolor{currentstroke}%
\pgfsetdash{}{0pt}%
\pgfpathmoveto{\pgfqpoint{2.495187in}{1.302980in}}%
\pgfpathcurveto{\pgfqpoint{2.506237in}{1.302980in}}{\pgfqpoint{2.516836in}{1.307370in}}{\pgfqpoint{2.524649in}{1.315184in}}%
\pgfpathcurveto{\pgfqpoint{2.532463in}{1.322997in}}{\pgfqpoint{2.536853in}{1.333596in}}{\pgfqpoint{2.536853in}{1.344647in}}%
\pgfpathcurveto{\pgfqpoint{2.536853in}{1.355697in}}{\pgfqpoint{2.532463in}{1.366296in}}{\pgfqpoint{2.524649in}{1.374109in}}%
\pgfpathcurveto{\pgfqpoint{2.516836in}{1.381923in}}{\pgfqpoint{2.506237in}{1.386313in}}{\pgfqpoint{2.495187in}{1.386313in}}%
\pgfpathcurveto{\pgfqpoint{2.484137in}{1.386313in}}{\pgfqpoint{2.473538in}{1.381923in}}{\pgfqpoint{2.465724in}{1.374109in}}%
\pgfpathcurveto{\pgfqpoint{2.457910in}{1.366296in}}{\pgfqpoint{2.453520in}{1.355697in}}{\pgfqpoint{2.453520in}{1.344647in}}%
\pgfpathcurveto{\pgfqpoint{2.453520in}{1.333596in}}{\pgfqpoint{2.457910in}{1.322997in}}{\pgfqpoint{2.465724in}{1.315184in}}%
\pgfpathcurveto{\pgfqpoint{2.473538in}{1.307370in}}{\pgfqpoint{2.484137in}{1.302980in}}{\pgfqpoint{2.495187in}{1.302980in}}%
\pgfpathlineto{\pgfqpoint{2.495187in}{1.302980in}}%
\pgfpathclose%
\pgfusepath{stroke}%
\end{pgfscope}%
\begin{pgfscope}%
\pgfpathrectangle{\pgfqpoint{0.847223in}{0.554012in}}{\pgfqpoint{6.200000in}{4.620000in}}%
\pgfusepath{clip}%
\pgfsetbuttcap%
\pgfsetroundjoin%
\pgfsetlinewidth{1.003750pt}%
\definecolor{currentstroke}{rgb}{1.000000,0.000000,0.000000}%
\pgfsetstrokecolor{currentstroke}%
\pgfsetdash{}{0pt}%
\pgfpathmoveto{\pgfqpoint{2.500520in}{1.300000in}}%
\pgfpathcurveto{\pgfqpoint{2.511570in}{1.300000in}}{\pgfqpoint{2.522169in}{1.304390in}}{\pgfqpoint{2.529983in}{1.312204in}}%
\pgfpathcurveto{\pgfqpoint{2.537796in}{1.320018in}}{\pgfqpoint{2.542187in}{1.330617in}}{\pgfqpoint{2.542187in}{1.341667in}}%
\pgfpathcurveto{\pgfqpoint{2.542187in}{1.352717in}}{\pgfqpoint{2.537796in}{1.363316in}}{\pgfqpoint{2.529983in}{1.371130in}}%
\pgfpathcurveto{\pgfqpoint{2.522169in}{1.378943in}}{\pgfqpoint{2.511570in}{1.383334in}}{\pgfqpoint{2.500520in}{1.383334in}}%
\pgfpathcurveto{\pgfqpoint{2.489470in}{1.383334in}}{\pgfqpoint{2.478871in}{1.378943in}}{\pgfqpoint{2.471057in}{1.371130in}}%
\pgfpathcurveto{\pgfqpoint{2.463244in}{1.363316in}}{\pgfqpoint{2.458853in}{1.352717in}}{\pgfqpoint{2.458853in}{1.341667in}}%
\pgfpathcurveto{\pgfqpoint{2.458853in}{1.330617in}}{\pgfqpoint{2.463244in}{1.320018in}}{\pgfqpoint{2.471057in}{1.312204in}}%
\pgfpathcurveto{\pgfqpoint{2.478871in}{1.304390in}}{\pgfqpoint{2.489470in}{1.300000in}}{\pgfqpoint{2.500520in}{1.300000in}}%
\pgfpathlineto{\pgfqpoint{2.500520in}{1.300000in}}%
\pgfpathclose%
\pgfusepath{stroke}%
\end{pgfscope}%
\begin{pgfscope}%
\pgfpathrectangle{\pgfqpoint{0.847223in}{0.554012in}}{\pgfqpoint{6.200000in}{4.620000in}}%
\pgfusepath{clip}%
\pgfsetbuttcap%
\pgfsetroundjoin%
\pgfsetlinewidth{1.003750pt}%
\definecolor{currentstroke}{rgb}{1.000000,0.000000,0.000000}%
\pgfsetstrokecolor{currentstroke}%
\pgfsetdash{}{0pt}%
\pgfpathmoveto{\pgfqpoint{2.505853in}{1.297035in}}%
\pgfpathcurveto{\pgfqpoint{2.516903in}{1.297035in}}{\pgfqpoint{2.527502in}{1.301425in}}{\pgfqpoint{2.535316in}{1.309239in}}%
\pgfpathcurveto{\pgfqpoint{2.543130in}{1.317053in}}{\pgfqpoint{2.547520in}{1.327652in}}{\pgfqpoint{2.547520in}{1.338702in}}%
\pgfpathcurveto{\pgfqpoint{2.547520in}{1.349752in}}{\pgfqpoint{2.543130in}{1.360351in}}{\pgfqpoint{2.535316in}{1.368164in}}%
\pgfpathcurveto{\pgfqpoint{2.527502in}{1.375978in}}{\pgfqpoint{2.516903in}{1.380368in}}{\pgfqpoint{2.505853in}{1.380368in}}%
\pgfpathcurveto{\pgfqpoint{2.494803in}{1.380368in}}{\pgfqpoint{2.484204in}{1.375978in}}{\pgfqpoint{2.476390in}{1.368164in}}%
\pgfpathcurveto{\pgfqpoint{2.468577in}{1.360351in}}{\pgfqpoint{2.464186in}{1.349752in}}{\pgfqpoint{2.464186in}{1.338702in}}%
\pgfpathcurveto{\pgfqpoint{2.464186in}{1.327652in}}{\pgfqpoint{2.468577in}{1.317053in}}{\pgfqpoint{2.476390in}{1.309239in}}%
\pgfpathcurveto{\pgfqpoint{2.484204in}{1.301425in}}{\pgfqpoint{2.494803in}{1.297035in}}{\pgfqpoint{2.505853in}{1.297035in}}%
\pgfpathlineto{\pgfqpoint{2.505853in}{1.297035in}}%
\pgfpathclose%
\pgfusepath{stroke}%
\end{pgfscope}%
\begin{pgfscope}%
\pgfpathrectangle{\pgfqpoint{0.847223in}{0.554012in}}{\pgfqpoint{6.200000in}{4.620000in}}%
\pgfusepath{clip}%
\pgfsetbuttcap%
\pgfsetroundjoin%
\pgfsetlinewidth{1.003750pt}%
\definecolor{currentstroke}{rgb}{1.000000,0.000000,0.000000}%
\pgfsetstrokecolor{currentstroke}%
\pgfsetdash{}{0pt}%
\pgfpathmoveto{\pgfqpoint{2.511186in}{1.294084in}}%
\pgfpathcurveto{\pgfqpoint{2.522236in}{1.294084in}}{\pgfqpoint{2.532836in}{1.298475in}}{\pgfqpoint{2.540649in}{1.306288in}}%
\pgfpathcurveto{\pgfqpoint{2.548463in}{1.314102in}}{\pgfqpoint{2.552853in}{1.324701in}}{\pgfqpoint{2.552853in}{1.335751in}}%
\pgfpathcurveto{\pgfqpoint{2.552853in}{1.346801in}}{\pgfqpoint{2.548463in}{1.357400in}}{\pgfqpoint{2.540649in}{1.365214in}}%
\pgfpathcurveto{\pgfqpoint{2.532836in}{1.373027in}}{\pgfqpoint{2.522236in}{1.377418in}}{\pgfqpoint{2.511186in}{1.377418in}}%
\pgfpathcurveto{\pgfqpoint{2.500136in}{1.377418in}}{\pgfqpoint{2.489537in}{1.373027in}}{\pgfqpoint{2.481724in}{1.365214in}}%
\pgfpathcurveto{\pgfqpoint{2.473910in}{1.357400in}}{\pgfqpoint{2.469520in}{1.346801in}}{\pgfqpoint{2.469520in}{1.335751in}}%
\pgfpathcurveto{\pgfqpoint{2.469520in}{1.324701in}}{\pgfqpoint{2.473910in}{1.314102in}}{\pgfqpoint{2.481724in}{1.306288in}}%
\pgfpathcurveto{\pgfqpoint{2.489537in}{1.298475in}}{\pgfqpoint{2.500136in}{1.294084in}}{\pgfqpoint{2.511186in}{1.294084in}}%
\pgfpathlineto{\pgfqpoint{2.511186in}{1.294084in}}%
\pgfpathclose%
\pgfusepath{stroke}%
\end{pgfscope}%
\begin{pgfscope}%
\pgfpathrectangle{\pgfqpoint{0.847223in}{0.554012in}}{\pgfqpoint{6.200000in}{4.620000in}}%
\pgfusepath{clip}%
\pgfsetbuttcap%
\pgfsetroundjoin%
\pgfsetlinewidth{1.003750pt}%
\definecolor{currentstroke}{rgb}{1.000000,0.000000,0.000000}%
\pgfsetstrokecolor{currentstroke}%
\pgfsetdash{}{0pt}%
\pgfpathmoveto{\pgfqpoint{2.516520in}{1.291148in}}%
\pgfpathcurveto{\pgfqpoint{2.527570in}{1.291148in}}{\pgfqpoint{2.538169in}{1.295538in}}{\pgfqpoint{2.545982in}{1.303352in}}%
\pgfpathcurveto{\pgfqpoint{2.553796in}{1.311166in}}{\pgfqpoint{2.558186in}{1.321765in}}{\pgfqpoint{2.558186in}{1.332815in}}%
\pgfpathcurveto{\pgfqpoint{2.558186in}{1.343865in}}{\pgfqpoint{2.553796in}{1.354464in}}{\pgfqpoint{2.545982in}{1.362277in}}%
\pgfpathcurveto{\pgfqpoint{2.538169in}{1.370091in}}{\pgfqpoint{2.527570in}{1.374481in}}{\pgfqpoint{2.516520in}{1.374481in}}%
\pgfpathcurveto{\pgfqpoint{2.505469in}{1.374481in}}{\pgfqpoint{2.494870in}{1.370091in}}{\pgfqpoint{2.487057in}{1.362277in}}%
\pgfpathcurveto{\pgfqpoint{2.479243in}{1.354464in}}{\pgfqpoint{2.474853in}{1.343865in}}{\pgfqpoint{2.474853in}{1.332815in}}%
\pgfpathcurveto{\pgfqpoint{2.474853in}{1.321765in}}{\pgfqpoint{2.479243in}{1.311166in}}{\pgfqpoint{2.487057in}{1.303352in}}%
\pgfpathcurveto{\pgfqpoint{2.494870in}{1.295538in}}{\pgfqpoint{2.505469in}{1.291148in}}{\pgfqpoint{2.516520in}{1.291148in}}%
\pgfpathlineto{\pgfqpoint{2.516520in}{1.291148in}}%
\pgfpathclose%
\pgfusepath{stroke}%
\end{pgfscope}%
\begin{pgfscope}%
\pgfpathrectangle{\pgfqpoint{0.847223in}{0.554012in}}{\pgfqpoint{6.200000in}{4.620000in}}%
\pgfusepath{clip}%
\pgfsetbuttcap%
\pgfsetroundjoin%
\pgfsetlinewidth{1.003750pt}%
\definecolor{currentstroke}{rgb}{1.000000,0.000000,0.000000}%
\pgfsetstrokecolor{currentstroke}%
\pgfsetdash{}{0pt}%
\pgfpathmoveto{\pgfqpoint{2.521853in}{1.288226in}}%
\pgfpathcurveto{\pgfqpoint{2.532903in}{1.288226in}}{\pgfqpoint{2.543502in}{1.292616in}}{\pgfqpoint{2.551316in}{1.300430in}}%
\pgfpathcurveto{\pgfqpoint{2.559129in}{1.308243in}}{\pgfqpoint{2.563519in}{1.318843in}}{\pgfqpoint{2.563519in}{1.329893in}}%
\pgfpathcurveto{\pgfqpoint{2.563519in}{1.340943in}}{\pgfqpoint{2.559129in}{1.351542in}}{\pgfqpoint{2.551316in}{1.359355in}}%
\pgfpathcurveto{\pgfqpoint{2.543502in}{1.367169in}}{\pgfqpoint{2.532903in}{1.371559in}}{\pgfqpoint{2.521853in}{1.371559in}}%
\pgfpathcurveto{\pgfqpoint{2.510803in}{1.371559in}}{\pgfqpoint{2.500204in}{1.367169in}}{\pgfqpoint{2.492390in}{1.359355in}}%
\pgfpathcurveto{\pgfqpoint{2.484576in}{1.351542in}}{\pgfqpoint{2.480186in}{1.340943in}}{\pgfqpoint{2.480186in}{1.329893in}}%
\pgfpathcurveto{\pgfqpoint{2.480186in}{1.318843in}}{\pgfqpoint{2.484576in}{1.308243in}}{\pgfqpoint{2.492390in}{1.300430in}}%
\pgfpathcurveto{\pgfqpoint{2.500204in}{1.292616in}}{\pgfqpoint{2.510803in}{1.288226in}}{\pgfqpoint{2.521853in}{1.288226in}}%
\pgfpathlineto{\pgfqpoint{2.521853in}{1.288226in}}%
\pgfpathclose%
\pgfusepath{stroke}%
\end{pgfscope}%
\begin{pgfscope}%
\pgfpathrectangle{\pgfqpoint{0.847223in}{0.554012in}}{\pgfqpoint{6.200000in}{4.620000in}}%
\pgfusepath{clip}%
\pgfsetbuttcap%
\pgfsetroundjoin%
\pgfsetlinewidth{1.003750pt}%
\definecolor{currentstroke}{rgb}{1.000000,0.000000,0.000000}%
\pgfsetstrokecolor{currentstroke}%
\pgfsetdash{}{0pt}%
\pgfpathmoveto{\pgfqpoint{2.527186in}{1.285318in}}%
\pgfpathcurveto{\pgfqpoint{2.538236in}{1.285318in}}{\pgfqpoint{2.548835in}{1.289708in}}{\pgfqpoint{2.556649in}{1.297522in}}%
\pgfpathcurveto{\pgfqpoint{2.564462in}{1.305336in}}{\pgfqpoint{2.568853in}{1.315935in}}{\pgfqpoint{2.568853in}{1.326985in}}%
\pgfpathcurveto{\pgfqpoint{2.568853in}{1.338035in}}{\pgfqpoint{2.564462in}{1.348634in}}{\pgfqpoint{2.556649in}{1.356448in}}%
\pgfpathcurveto{\pgfqpoint{2.548835in}{1.364261in}}{\pgfqpoint{2.538236in}{1.368651in}}{\pgfqpoint{2.527186in}{1.368651in}}%
\pgfpathcurveto{\pgfqpoint{2.516136in}{1.368651in}}{\pgfqpoint{2.505537in}{1.364261in}}{\pgfqpoint{2.497723in}{1.356448in}}%
\pgfpathcurveto{\pgfqpoint{2.489910in}{1.348634in}}{\pgfqpoint{2.485519in}{1.338035in}}{\pgfqpoint{2.485519in}{1.326985in}}%
\pgfpathcurveto{\pgfqpoint{2.485519in}{1.315935in}}{\pgfqpoint{2.489910in}{1.305336in}}{\pgfqpoint{2.497723in}{1.297522in}}%
\pgfpathcurveto{\pgfqpoint{2.505537in}{1.289708in}}{\pgfqpoint{2.516136in}{1.285318in}}{\pgfqpoint{2.527186in}{1.285318in}}%
\pgfpathlineto{\pgfqpoint{2.527186in}{1.285318in}}%
\pgfpathclose%
\pgfusepath{stroke}%
\end{pgfscope}%
\begin{pgfscope}%
\pgfpathrectangle{\pgfqpoint{0.847223in}{0.554012in}}{\pgfqpoint{6.200000in}{4.620000in}}%
\pgfusepath{clip}%
\pgfsetbuttcap%
\pgfsetroundjoin%
\pgfsetlinewidth{1.003750pt}%
\definecolor{currentstroke}{rgb}{1.000000,0.000000,0.000000}%
\pgfsetstrokecolor{currentstroke}%
\pgfsetdash{}{0pt}%
\pgfpathmoveto{\pgfqpoint{2.532519in}{1.282424in}}%
\pgfpathcurveto{\pgfqpoint{2.543569in}{1.282424in}}{\pgfqpoint{2.554168in}{1.286814in}}{\pgfqpoint{2.561982in}{1.294628in}}%
\pgfpathcurveto{\pgfqpoint{2.569796in}{1.302442in}}{\pgfqpoint{2.574186in}{1.313041in}}{\pgfqpoint{2.574186in}{1.324091in}}%
\pgfpathcurveto{\pgfqpoint{2.574186in}{1.335141in}}{\pgfqpoint{2.569796in}{1.345740in}}{\pgfqpoint{2.561982in}{1.353554in}}%
\pgfpathcurveto{\pgfqpoint{2.554168in}{1.361367in}}{\pgfqpoint{2.543569in}{1.365758in}}{\pgfqpoint{2.532519in}{1.365758in}}%
\pgfpathcurveto{\pgfqpoint{2.521469in}{1.365758in}}{\pgfqpoint{2.510870in}{1.361367in}}{\pgfqpoint{2.503056in}{1.353554in}}%
\pgfpathcurveto{\pgfqpoint{2.495243in}{1.345740in}}{\pgfqpoint{2.490853in}{1.335141in}}{\pgfqpoint{2.490853in}{1.324091in}}%
\pgfpathcurveto{\pgfqpoint{2.490853in}{1.313041in}}{\pgfqpoint{2.495243in}{1.302442in}}{\pgfqpoint{2.503056in}{1.294628in}}%
\pgfpathcurveto{\pgfqpoint{2.510870in}{1.286814in}}{\pgfqpoint{2.521469in}{1.282424in}}{\pgfqpoint{2.532519in}{1.282424in}}%
\pgfpathlineto{\pgfqpoint{2.532519in}{1.282424in}}%
\pgfpathclose%
\pgfusepath{stroke}%
\end{pgfscope}%
\begin{pgfscope}%
\pgfpathrectangle{\pgfqpoint{0.847223in}{0.554012in}}{\pgfqpoint{6.200000in}{4.620000in}}%
\pgfusepath{clip}%
\pgfsetbuttcap%
\pgfsetroundjoin%
\pgfsetlinewidth{1.003750pt}%
\definecolor{currentstroke}{rgb}{1.000000,0.000000,0.000000}%
\pgfsetstrokecolor{currentstroke}%
\pgfsetdash{}{0pt}%
\pgfpathmoveto{\pgfqpoint{2.537852in}{1.279544in}}%
\pgfpathcurveto{\pgfqpoint{2.548903in}{1.279544in}}{\pgfqpoint{2.559502in}{1.283934in}}{\pgfqpoint{2.567315in}{1.291748in}}%
\pgfpathcurveto{\pgfqpoint{2.575129in}{1.299562in}}{\pgfqpoint{2.579519in}{1.310161in}}{\pgfqpoint{2.579519in}{1.321211in}}%
\pgfpathcurveto{\pgfqpoint{2.579519in}{1.332261in}}{\pgfqpoint{2.575129in}{1.342860in}}{\pgfqpoint{2.567315in}{1.350674in}}%
\pgfpathcurveto{\pgfqpoint{2.559502in}{1.358487in}}{\pgfqpoint{2.548903in}{1.362878in}}{\pgfqpoint{2.537852in}{1.362878in}}%
\pgfpathcurveto{\pgfqpoint{2.526802in}{1.362878in}}{\pgfqpoint{2.516203in}{1.358487in}}{\pgfqpoint{2.508390in}{1.350674in}}%
\pgfpathcurveto{\pgfqpoint{2.500576in}{1.342860in}}{\pgfqpoint{2.496186in}{1.332261in}}{\pgfqpoint{2.496186in}{1.321211in}}%
\pgfpathcurveto{\pgfqpoint{2.496186in}{1.310161in}}{\pgfqpoint{2.500576in}{1.299562in}}{\pgfqpoint{2.508390in}{1.291748in}}%
\pgfpathcurveto{\pgfqpoint{2.516203in}{1.283934in}}{\pgfqpoint{2.526802in}{1.279544in}}{\pgfqpoint{2.537852in}{1.279544in}}%
\pgfpathlineto{\pgfqpoint{2.537852in}{1.279544in}}%
\pgfpathclose%
\pgfusepath{stroke}%
\end{pgfscope}%
\begin{pgfscope}%
\pgfpathrectangle{\pgfqpoint{0.847223in}{0.554012in}}{\pgfqpoint{6.200000in}{4.620000in}}%
\pgfusepath{clip}%
\pgfsetbuttcap%
\pgfsetroundjoin%
\pgfsetlinewidth{1.003750pt}%
\definecolor{currentstroke}{rgb}{1.000000,0.000000,0.000000}%
\pgfsetstrokecolor{currentstroke}%
\pgfsetdash{}{0pt}%
\pgfpathmoveto{\pgfqpoint{2.543186in}{1.276678in}}%
\pgfpathcurveto{\pgfqpoint{2.554236in}{1.276678in}}{\pgfqpoint{2.564835in}{1.281068in}}{\pgfqpoint{2.572648in}{1.288882in}}%
\pgfpathcurveto{\pgfqpoint{2.580462in}{1.296696in}}{\pgfqpoint{2.584852in}{1.307295in}}{\pgfqpoint{2.584852in}{1.318345in}}%
\pgfpathcurveto{\pgfqpoint{2.584852in}{1.329395in}}{\pgfqpoint{2.580462in}{1.339994in}}{\pgfqpoint{2.572648in}{1.347808in}}%
\pgfpathcurveto{\pgfqpoint{2.564835in}{1.355621in}}{\pgfqpoint{2.554236in}{1.360011in}}{\pgfqpoint{2.543186in}{1.360011in}}%
\pgfpathcurveto{\pgfqpoint{2.532136in}{1.360011in}}{\pgfqpoint{2.521536in}{1.355621in}}{\pgfqpoint{2.513723in}{1.347808in}}%
\pgfpathcurveto{\pgfqpoint{2.505909in}{1.339994in}}{\pgfqpoint{2.501519in}{1.329395in}}{\pgfqpoint{2.501519in}{1.318345in}}%
\pgfpathcurveto{\pgfqpoint{2.501519in}{1.307295in}}{\pgfqpoint{2.505909in}{1.296696in}}{\pgfqpoint{2.513723in}{1.288882in}}%
\pgfpathcurveto{\pgfqpoint{2.521536in}{1.281068in}}{\pgfqpoint{2.532136in}{1.276678in}}{\pgfqpoint{2.543186in}{1.276678in}}%
\pgfpathlineto{\pgfqpoint{2.543186in}{1.276678in}}%
\pgfpathclose%
\pgfusepath{stroke}%
\end{pgfscope}%
\begin{pgfscope}%
\pgfpathrectangle{\pgfqpoint{0.847223in}{0.554012in}}{\pgfqpoint{6.200000in}{4.620000in}}%
\pgfusepath{clip}%
\pgfsetbuttcap%
\pgfsetroundjoin%
\pgfsetlinewidth{1.003750pt}%
\definecolor{currentstroke}{rgb}{1.000000,0.000000,0.000000}%
\pgfsetstrokecolor{currentstroke}%
\pgfsetdash{}{0pt}%
\pgfpathmoveto{\pgfqpoint{2.548519in}{1.273826in}}%
\pgfpathcurveto{\pgfqpoint{2.559569in}{1.273826in}}{\pgfqpoint{2.570168in}{1.278216in}}{\pgfqpoint{2.577982in}{1.286030in}}%
\pgfpathcurveto{\pgfqpoint{2.585795in}{1.293843in}}{\pgfqpoint{2.590186in}{1.304442in}}{\pgfqpoint{2.590186in}{1.315492in}}%
\pgfpathcurveto{\pgfqpoint{2.590186in}{1.326543in}}{\pgfqpoint{2.585795in}{1.337142in}}{\pgfqpoint{2.577982in}{1.344955in}}%
\pgfpathcurveto{\pgfqpoint{2.570168in}{1.352769in}}{\pgfqpoint{2.559569in}{1.357159in}}{\pgfqpoint{2.548519in}{1.357159in}}%
\pgfpathcurveto{\pgfqpoint{2.537469in}{1.357159in}}{\pgfqpoint{2.526870in}{1.352769in}}{\pgfqpoint{2.519056in}{1.344955in}}%
\pgfpathcurveto{\pgfqpoint{2.511242in}{1.337142in}}{\pgfqpoint{2.506852in}{1.326543in}}{\pgfqpoint{2.506852in}{1.315492in}}%
\pgfpathcurveto{\pgfqpoint{2.506852in}{1.304442in}}{\pgfqpoint{2.511242in}{1.293843in}}{\pgfqpoint{2.519056in}{1.286030in}}%
\pgfpathcurveto{\pgfqpoint{2.526870in}{1.278216in}}{\pgfqpoint{2.537469in}{1.273826in}}{\pgfqpoint{2.548519in}{1.273826in}}%
\pgfpathlineto{\pgfqpoint{2.548519in}{1.273826in}}%
\pgfpathclose%
\pgfusepath{stroke}%
\end{pgfscope}%
\begin{pgfscope}%
\pgfpathrectangle{\pgfqpoint{0.847223in}{0.554012in}}{\pgfqpoint{6.200000in}{4.620000in}}%
\pgfusepath{clip}%
\pgfsetbuttcap%
\pgfsetroundjoin%
\pgfsetlinewidth{1.003750pt}%
\definecolor{currentstroke}{rgb}{1.000000,0.000000,0.000000}%
\pgfsetstrokecolor{currentstroke}%
\pgfsetdash{}{0pt}%
\pgfpathmoveto{\pgfqpoint{2.553852in}{1.270987in}}%
\pgfpathcurveto{\pgfqpoint{2.564902in}{1.270987in}}{\pgfqpoint{2.575501in}{1.275377in}}{\pgfqpoint{2.583315in}{1.283191in}}%
\pgfpathcurveto{\pgfqpoint{2.591128in}{1.291005in}}{\pgfqpoint{2.595519in}{1.301604in}}{\pgfqpoint{2.595519in}{1.312654in}}%
\pgfpathcurveto{\pgfqpoint{2.595519in}{1.323704in}}{\pgfqpoint{2.591128in}{1.334303in}}{\pgfqpoint{2.583315in}{1.342117in}}%
\pgfpathcurveto{\pgfqpoint{2.575501in}{1.349930in}}{\pgfqpoint{2.564902in}{1.354320in}}{\pgfqpoint{2.553852in}{1.354320in}}%
\pgfpathcurveto{\pgfqpoint{2.542802in}{1.354320in}}{\pgfqpoint{2.532203in}{1.349930in}}{\pgfqpoint{2.524389in}{1.342117in}}%
\pgfpathcurveto{\pgfqpoint{2.516576in}{1.334303in}}{\pgfqpoint{2.512185in}{1.323704in}}{\pgfqpoint{2.512185in}{1.312654in}}%
\pgfpathcurveto{\pgfqpoint{2.512185in}{1.301604in}}{\pgfqpoint{2.516576in}{1.291005in}}{\pgfqpoint{2.524389in}{1.283191in}}%
\pgfpathcurveto{\pgfqpoint{2.532203in}{1.275377in}}{\pgfqpoint{2.542802in}{1.270987in}}{\pgfqpoint{2.553852in}{1.270987in}}%
\pgfpathlineto{\pgfqpoint{2.553852in}{1.270987in}}%
\pgfpathclose%
\pgfusepath{stroke}%
\end{pgfscope}%
\begin{pgfscope}%
\pgfpathrectangle{\pgfqpoint{0.847223in}{0.554012in}}{\pgfqpoint{6.200000in}{4.620000in}}%
\pgfusepath{clip}%
\pgfsetbuttcap%
\pgfsetroundjoin%
\pgfsetlinewidth{1.003750pt}%
\definecolor{currentstroke}{rgb}{1.000000,0.000000,0.000000}%
\pgfsetstrokecolor{currentstroke}%
\pgfsetdash{}{0pt}%
\pgfpathmoveto{\pgfqpoint{2.559185in}{1.268162in}}%
\pgfpathcurveto{\pgfqpoint{2.570235in}{1.268162in}}{\pgfqpoint{2.580834in}{1.272552in}}{\pgfqpoint{2.588648in}{1.280366in}}%
\pgfpathcurveto{\pgfqpoint{2.596462in}{1.288179in}}{\pgfqpoint{2.600852in}{1.298778in}}{\pgfqpoint{2.600852in}{1.309829in}}%
\pgfpathcurveto{\pgfqpoint{2.600852in}{1.320879in}}{\pgfqpoint{2.596462in}{1.331478in}}{\pgfqpoint{2.588648in}{1.339291in}}%
\pgfpathcurveto{\pgfqpoint{2.580834in}{1.347105in}}{\pgfqpoint{2.570235in}{1.351495in}}{\pgfqpoint{2.559185in}{1.351495in}}%
\pgfpathcurveto{\pgfqpoint{2.548135in}{1.351495in}}{\pgfqpoint{2.537536in}{1.347105in}}{\pgfqpoint{2.529722in}{1.339291in}}%
\pgfpathcurveto{\pgfqpoint{2.521909in}{1.331478in}}{\pgfqpoint{2.517519in}{1.320879in}}{\pgfqpoint{2.517519in}{1.309829in}}%
\pgfpathcurveto{\pgfqpoint{2.517519in}{1.298778in}}{\pgfqpoint{2.521909in}{1.288179in}}{\pgfqpoint{2.529722in}{1.280366in}}%
\pgfpathcurveto{\pgfqpoint{2.537536in}{1.272552in}}{\pgfqpoint{2.548135in}{1.268162in}}{\pgfqpoint{2.559185in}{1.268162in}}%
\pgfpathlineto{\pgfqpoint{2.559185in}{1.268162in}}%
\pgfpathclose%
\pgfusepath{stroke}%
\end{pgfscope}%
\begin{pgfscope}%
\pgfpathrectangle{\pgfqpoint{0.847223in}{0.554012in}}{\pgfqpoint{6.200000in}{4.620000in}}%
\pgfusepath{clip}%
\pgfsetbuttcap%
\pgfsetroundjoin%
\pgfsetlinewidth{1.003750pt}%
\definecolor{currentstroke}{rgb}{1.000000,0.000000,0.000000}%
\pgfsetstrokecolor{currentstroke}%
\pgfsetdash{}{0pt}%
\pgfpathmoveto{\pgfqpoint{2.564518in}{1.265350in}}%
\pgfpathcurveto{\pgfqpoint{2.575569in}{1.265350in}}{\pgfqpoint{2.586168in}{1.269740in}}{\pgfqpoint{2.593981in}{1.277554in}}%
\pgfpathcurveto{\pgfqpoint{2.601795in}{1.285368in}}{\pgfqpoint{2.606185in}{1.295967in}}{\pgfqpoint{2.606185in}{1.307017in}}%
\pgfpathcurveto{\pgfqpoint{2.606185in}{1.318067in}}{\pgfqpoint{2.601795in}{1.328666in}}{\pgfqpoint{2.593981in}{1.336480in}}%
\pgfpathcurveto{\pgfqpoint{2.586168in}{1.344293in}}{\pgfqpoint{2.575569in}{1.348684in}}{\pgfqpoint{2.564518in}{1.348684in}}%
\pgfpathcurveto{\pgfqpoint{2.553468in}{1.348684in}}{\pgfqpoint{2.542869in}{1.344293in}}{\pgfqpoint{2.535056in}{1.336480in}}%
\pgfpathcurveto{\pgfqpoint{2.527242in}{1.328666in}}{\pgfqpoint{2.522852in}{1.318067in}}{\pgfqpoint{2.522852in}{1.307017in}}%
\pgfpathcurveto{\pgfqpoint{2.522852in}{1.295967in}}{\pgfqpoint{2.527242in}{1.285368in}}{\pgfqpoint{2.535056in}{1.277554in}}%
\pgfpathcurveto{\pgfqpoint{2.542869in}{1.269740in}}{\pgfqpoint{2.553468in}{1.265350in}}{\pgfqpoint{2.564518in}{1.265350in}}%
\pgfpathlineto{\pgfqpoint{2.564518in}{1.265350in}}%
\pgfpathclose%
\pgfusepath{stroke}%
\end{pgfscope}%
\begin{pgfscope}%
\pgfpathrectangle{\pgfqpoint{0.847223in}{0.554012in}}{\pgfqpoint{6.200000in}{4.620000in}}%
\pgfusepath{clip}%
\pgfsetbuttcap%
\pgfsetroundjoin%
\pgfsetlinewidth{1.003750pt}%
\definecolor{currentstroke}{rgb}{1.000000,0.000000,0.000000}%
\pgfsetstrokecolor{currentstroke}%
\pgfsetdash{}{0pt}%
\pgfpathmoveto{\pgfqpoint{2.569852in}{1.262552in}}%
\pgfpathcurveto{\pgfqpoint{2.580902in}{1.262552in}}{\pgfqpoint{2.591501in}{1.266942in}}{\pgfqpoint{2.599314in}{1.274756in}}%
\pgfpathcurveto{\pgfqpoint{2.607128in}{1.282569in}}{\pgfqpoint{2.611518in}{1.293168in}}{\pgfqpoint{2.611518in}{1.304218in}}%
\pgfpathcurveto{\pgfqpoint{2.611518in}{1.315269in}}{\pgfqpoint{2.607128in}{1.325868in}}{\pgfqpoint{2.599314in}{1.333681in}}%
\pgfpathcurveto{\pgfqpoint{2.591501in}{1.341495in}}{\pgfqpoint{2.580902in}{1.345885in}}{\pgfqpoint{2.569852in}{1.345885in}}%
\pgfpathcurveto{\pgfqpoint{2.558802in}{1.345885in}}{\pgfqpoint{2.548203in}{1.341495in}}{\pgfqpoint{2.540389in}{1.333681in}}%
\pgfpathcurveto{\pgfqpoint{2.532575in}{1.325868in}}{\pgfqpoint{2.528185in}{1.315269in}}{\pgfqpoint{2.528185in}{1.304218in}}%
\pgfpathcurveto{\pgfqpoint{2.528185in}{1.293168in}}{\pgfqpoint{2.532575in}{1.282569in}}{\pgfqpoint{2.540389in}{1.274756in}}%
\pgfpathcurveto{\pgfqpoint{2.548203in}{1.266942in}}{\pgfqpoint{2.558802in}{1.262552in}}{\pgfqpoint{2.569852in}{1.262552in}}%
\pgfpathlineto{\pgfqpoint{2.569852in}{1.262552in}}%
\pgfpathclose%
\pgfusepath{stroke}%
\end{pgfscope}%
\begin{pgfscope}%
\pgfpathrectangle{\pgfqpoint{0.847223in}{0.554012in}}{\pgfqpoint{6.200000in}{4.620000in}}%
\pgfusepath{clip}%
\pgfsetbuttcap%
\pgfsetroundjoin%
\pgfsetlinewidth{1.003750pt}%
\definecolor{currentstroke}{rgb}{1.000000,0.000000,0.000000}%
\pgfsetstrokecolor{currentstroke}%
\pgfsetdash{}{0pt}%
\pgfpathmoveto{\pgfqpoint{2.575185in}{1.259767in}}%
\pgfpathcurveto{\pgfqpoint{2.586235in}{1.259767in}}{\pgfqpoint{2.596834in}{1.264157in}}{\pgfqpoint{2.604648in}{1.271971in}}%
\pgfpathcurveto{\pgfqpoint{2.612461in}{1.279784in}}{\pgfqpoint{2.616852in}{1.290383in}}{\pgfqpoint{2.616852in}{1.301433in}}%
\pgfpathcurveto{\pgfqpoint{2.616852in}{1.312483in}}{\pgfqpoint{2.612461in}{1.323083in}}{\pgfqpoint{2.604648in}{1.330896in}}%
\pgfpathcurveto{\pgfqpoint{2.596834in}{1.338710in}}{\pgfqpoint{2.586235in}{1.343100in}}{\pgfqpoint{2.575185in}{1.343100in}}%
\pgfpathcurveto{\pgfqpoint{2.564135in}{1.343100in}}{\pgfqpoint{2.553536in}{1.338710in}}{\pgfqpoint{2.545722in}{1.330896in}}%
\pgfpathcurveto{\pgfqpoint{2.537909in}{1.323083in}}{\pgfqpoint{2.533518in}{1.312483in}}{\pgfqpoint{2.533518in}{1.301433in}}%
\pgfpathcurveto{\pgfqpoint{2.533518in}{1.290383in}}{\pgfqpoint{2.537909in}{1.279784in}}{\pgfqpoint{2.545722in}{1.271971in}}%
\pgfpathcurveto{\pgfqpoint{2.553536in}{1.264157in}}{\pgfqpoint{2.564135in}{1.259767in}}{\pgfqpoint{2.575185in}{1.259767in}}%
\pgfpathlineto{\pgfqpoint{2.575185in}{1.259767in}}%
\pgfpathclose%
\pgfusepath{stroke}%
\end{pgfscope}%
\begin{pgfscope}%
\pgfpathrectangle{\pgfqpoint{0.847223in}{0.554012in}}{\pgfqpoint{6.200000in}{4.620000in}}%
\pgfusepath{clip}%
\pgfsetbuttcap%
\pgfsetroundjoin%
\pgfsetlinewidth{1.003750pt}%
\definecolor{currentstroke}{rgb}{1.000000,0.000000,0.000000}%
\pgfsetstrokecolor{currentstroke}%
\pgfsetdash{}{0pt}%
\pgfpathmoveto{\pgfqpoint{2.580518in}{1.256995in}}%
\pgfpathcurveto{\pgfqpoint{2.591568in}{1.256995in}}{\pgfqpoint{2.602167in}{1.261385in}}{\pgfqpoint{2.609981in}{1.269199in}}%
\pgfpathcurveto{\pgfqpoint{2.617795in}{1.277012in}}{\pgfqpoint{2.622185in}{1.287611in}}{\pgfqpoint{2.622185in}{1.298661in}}%
\pgfpathcurveto{\pgfqpoint{2.622185in}{1.309712in}}{\pgfqpoint{2.617795in}{1.320311in}}{\pgfqpoint{2.609981in}{1.328124in}}%
\pgfpathcurveto{\pgfqpoint{2.602167in}{1.335938in}}{\pgfqpoint{2.591568in}{1.340328in}}{\pgfqpoint{2.580518in}{1.340328in}}%
\pgfpathcurveto{\pgfqpoint{2.569468in}{1.340328in}}{\pgfqpoint{2.558869in}{1.335938in}}{\pgfqpoint{2.551055in}{1.328124in}}%
\pgfpathcurveto{\pgfqpoint{2.543242in}{1.320311in}}{\pgfqpoint{2.538851in}{1.309712in}}{\pgfqpoint{2.538851in}{1.298661in}}%
\pgfpathcurveto{\pgfqpoint{2.538851in}{1.287611in}}{\pgfqpoint{2.543242in}{1.277012in}}{\pgfqpoint{2.551055in}{1.269199in}}%
\pgfpathcurveto{\pgfqpoint{2.558869in}{1.261385in}}{\pgfqpoint{2.569468in}{1.256995in}}{\pgfqpoint{2.580518in}{1.256995in}}%
\pgfpathlineto{\pgfqpoint{2.580518in}{1.256995in}}%
\pgfpathclose%
\pgfusepath{stroke}%
\end{pgfscope}%
\begin{pgfscope}%
\pgfpathrectangle{\pgfqpoint{0.847223in}{0.554012in}}{\pgfqpoint{6.200000in}{4.620000in}}%
\pgfusepath{clip}%
\pgfsetbuttcap%
\pgfsetroundjoin%
\pgfsetlinewidth{1.003750pt}%
\definecolor{currentstroke}{rgb}{1.000000,0.000000,0.000000}%
\pgfsetstrokecolor{currentstroke}%
\pgfsetdash{}{0pt}%
\pgfpathmoveto{\pgfqpoint{2.585851in}{1.254236in}}%
\pgfpathcurveto{\pgfqpoint{2.596901in}{1.254236in}}{\pgfqpoint{2.607501in}{1.258626in}}{\pgfqpoint{2.615314in}{1.266440in}}%
\pgfpathcurveto{\pgfqpoint{2.623128in}{1.274253in}}{\pgfqpoint{2.627518in}{1.284852in}}{\pgfqpoint{2.627518in}{1.295903in}}%
\pgfpathcurveto{\pgfqpoint{2.627518in}{1.306953in}}{\pgfqpoint{2.623128in}{1.317552in}}{\pgfqpoint{2.615314in}{1.325365in}}%
\pgfpathcurveto{\pgfqpoint{2.607501in}{1.333179in}}{\pgfqpoint{2.596901in}{1.337569in}}{\pgfqpoint{2.585851in}{1.337569in}}%
\pgfpathcurveto{\pgfqpoint{2.574801in}{1.337569in}}{\pgfqpoint{2.564202in}{1.333179in}}{\pgfqpoint{2.556389in}{1.325365in}}%
\pgfpathcurveto{\pgfqpoint{2.548575in}{1.317552in}}{\pgfqpoint{2.544185in}{1.306953in}}{\pgfqpoint{2.544185in}{1.295903in}}%
\pgfpathcurveto{\pgfqpoint{2.544185in}{1.284852in}}{\pgfqpoint{2.548575in}{1.274253in}}{\pgfqpoint{2.556389in}{1.266440in}}%
\pgfpathcurveto{\pgfqpoint{2.564202in}{1.258626in}}{\pgfqpoint{2.574801in}{1.254236in}}{\pgfqpoint{2.585851in}{1.254236in}}%
\pgfpathlineto{\pgfqpoint{2.585851in}{1.254236in}}%
\pgfpathclose%
\pgfusepath{stroke}%
\end{pgfscope}%
\begin{pgfscope}%
\pgfpathrectangle{\pgfqpoint{0.847223in}{0.554012in}}{\pgfqpoint{6.200000in}{4.620000in}}%
\pgfusepath{clip}%
\pgfsetbuttcap%
\pgfsetroundjoin%
\pgfsetlinewidth{1.003750pt}%
\definecolor{currentstroke}{rgb}{1.000000,0.000000,0.000000}%
\pgfsetstrokecolor{currentstroke}%
\pgfsetdash{}{0pt}%
\pgfpathmoveto{\pgfqpoint{2.591185in}{1.251490in}}%
\pgfpathcurveto{\pgfqpoint{2.602235in}{1.251490in}}{\pgfqpoint{2.612834in}{1.255880in}}{\pgfqpoint{2.620647in}{1.263694in}}%
\pgfpathcurveto{\pgfqpoint{2.628461in}{1.271507in}}{\pgfqpoint{2.632851in}{1.282107in}}{\pgfqpoint{2.632851in}{1.293157in}}%
\pgfpathcurveto{\pgfqpoint{2.632851in}{1.304207in}}{\pgfqpoint{2.628461in}{1.314806in}}{\pgfqpoint{2.620647in}{1.322619in}}%
\pgfpathcurveto{\pgfqpoint{2.612834in}{1.330433in}}{\pgfqpoint{2.602235in}{1.334823in}}{\pgfqpoint{2.591185in}{1.334823in}}%
\pgfpathcurveto{\pgfqpoint{2.580134in}{1.334823in}}{\pgfqpoint{2.569535in}{1.330433in}}{\pgfqpoint{2.561722in}{1.322619in}}%
\pgfpathcurveto{\pgfqpoint{2.553908in}{1.314806in}}{\pgfqpoint{2.549518in}{1.304207in}}{\pgfqpoint{2.549518in}{1.293157in}}%
\pgfpathcurveto{\pgfqpoint{2.549518in}{1.282107in}}{\pgfqpoint{2.553908in}{1.271507in}}{\pgfqpoint{2.561722in}{1.263694in}}%
\pgfpathcurveto{\pgfqpoint{2.569535in}{1.255880in}}{\pgfqpoint{2.580134in}{1.251490in}}{\pgfqpoint{2.591185in}{1.251490in}}%
\pgfpathlineto{\pgfqpoint{2.591185in}{1.251490in}}%
\pgfpathclose%
\pgfusepath{stroke}%
\end{pgfscope}%
\begin{pgfscope}%
\pgfpathrectangle{\pgfqpoint{0.847223in}{0.554012in}}{\pgfqpoint{6.200000in}{4.620000in}}%
\pgfusepath{clip}%
\pgfsetbuttcap%
\pgfsetroundjoin%
\pgfsetlinewidth{1.003750pt}%
\definecolor{currentstroke}{rgb}{1.000000,0.000000,0.000000}%
\pgfsetstrokecolor{currentstroke}%
\pgfsetdash{}{0pt}%
\pgfpathmoveto{\pgfqpoint{2.596518in}{1.248757in}}%
\pgfpathcurveto{\pgfqpoint{2.607568in}{1.248757in}}{\pgfqpoint{2.618167in}{1.253147in}}{\pgfqpoint{2.625981in}{1.260961in}}%
\pgfpathcurveto{\pgfqpoint{2.633794in}{1.268774in}}{\pgfqpoint{2.638184in}{1.279374in}}{\pgfqpoint{2.638184in}{1.290424in}}%
\pgfpathcurveto{\pgfqpoint{2.638184in}{1.301474in}}{\pgfqpoint{2.633794in}{1.312073in}}{\pgfqpoint{2.625981in}{1.319886in}}%
\pgfpathcurveto{\pgfqpoint{2.618167in}{1.327700in}}{\pgfqpoint{2.607568in}{1.332090in}}{\pgfqpoint{2.596518in}{1.332090in}}%
\pgfpathcurveto{\pgfqpoint{2.585468in}{1.332090in}}{\pgfqpoint{2.574869in}{1.327700in}}{\pgfqpoint{2.567055in}{1.319886in}}%
\pgfpathcurveto{\pgfqpoint{2.559241in}{1.312073in}}{\pgfqpoint{2.554851in}{1.301474in}}{\pgfqpoint{2.554851in}{1.290424in}}%
\pgfpathcurveto{\pgfqpoint{2.554851in}{1.279374in}}{\pgfqpoint{2.559241in}{1.268774in}}{\pgfqpoint{2.567055in}{1.260961in}}%
\pgfpathcurveto{\pgfqpoint{2.574869in}{1.253147in}}{\pgfqpoint{2.585468in}{1.248757in}}{\pgfqpoint{2.596518in}{1.248757in}}%
\pgfpathlineto{\pgfqpoint{2.596518in}{1.248757in}}%
\pgfpathclose%
\pgfusepath{stroke}%
\end{pgfscope}%
\begin{pgfscope}%
\pgfpathrectangle{\pgfqpoint{0.847223in}{0.554012in}}{\pgfqpoint{6.200000in}{4.620000in}}%
\pgfusepath{clip}%
\pgfsetbuttcap%
\pgfsetroundjoin%
\pgfsetlinewidth{1.003750pt}%
\definecolor{currentstroke}{rgb}{1.000000,0.000000,0.000000}%
\pgfsetstrokecolor{currentstroke}%
\pgfsetdash{}{0pt}%
\pgfpathmoveto{\pgfqpoint{2.601851in}{1.246037in}}%
\pgfpathcurveto{\pgfqpoint{2.612901in}{1.246037in}}{\pgfqpoint{2.623500in}{1.250427in}}{\pgfqpoint{2.631314in}{1.258241in}}%
\pgfpathcurveto{\pgfqpoint{2.639127in}{1.266054in}}{\pgfqpoint{2.643518in}{1.276653in}}{\pgfqpoint{2.643518in}{1.287703in}}%
\pgfpathcurveto{\pgfqpoint{2.643518in}{1.298754in}}{\pgfqpoint{2.639127in}{1.309353in}}{\pgfqpoint{2.631314in}{1.317166in}}%
\pgfpathcurveto{\pgfqpoint{2.623500in}{1.324980in}}{\pgfqpoint{2.612901in}{1.329370in}}{\pgfqpoint{2.601851in}{1.329370in}}%
\pgfpathcurveto{\pgfqpoint{2.590801in}{1.329370in}}{\pgfqpoint{2.580202in}{1.324980in}}{\pgfqpoint{2.572388in}{1.317166in}}%
\pgfpathcurveto{\pgfqpoint{2.564575in}{1.309353in}}{\pgfqpoint{2.560184in}{1.298754in}}{\pgfqpoint{2.560184in}{1.287703in}}%
\pgfpathcurveto{\pgfqpoint{2.560184in}{1.276653in}}{\pgfqpoint{2.564575in}{1.266054in}}{\pgfqpoint{2.572388in}{1.258241in}}%
\pgfpathcurveto{\pgfqpoint{2.580202in}{1.250427in}}{\pgfqpoint{2.590801in}{1.246037in}}{\pgfqpoint{2.601851in}{1.246037in}}%
\pgfpathlineto{\pgfqpoint{2.601851in}{1.246037in}}%
\pgfpathclose%
\pgfusepath{stroke}%
\end{pgfscope}%
\begin{pgfscope}%
\pgfpathrectangle{\pgfqpoint{0.847223in}{0.554012in}}{\pgfqpoint{6.200000in}{4.620000in}}%
\pgfusepath{clip}%
\pgfsetbuttcap%
\pgfsetroundjoin%
\pgfsetlinewidth{1.003750pt}%
\definecolor{currentstroke}{rgb}{1.000000,0.000000,0.000000}%
\pgfsetstrokecolor{currentstroke}%
\pgfsetdash{}{0pt}%
\pgfpathmoveto{\pgfqpoint{2.607184in}{1.243329in}}%
\pgfpathcurveto{\pgfqpoint{2.618234in}{1.243329in}}{\pgfqpoint{2.628833in}{1.247720in}}{\pgfqpoint{2.636647in}{1.255533in}}%
\pgfpathcurveto{\pgfqpoint{2.644461in}{1.263347in}}{\pgfqpoint{2.648851in}{1.273946in}}{\pgfqpoint{2.648851in}{1.284996in}}%
\pgfpathcurveto{\pgfqpoint{2.648851in}{1.296046in}}{\pgfqpoint{2.644461in}{1.306645in}}{\pgfqpoint{2.636647in}{1.314459in}}%
\pgfpathcurveto{\pgfqpoint{2.628833in}{1.322272in}}{\pgfqpoint{2.618234in}{1.326663in}}{\pgfqpoint{2.607184in}{1.326663in}}%
\pgfpathcurveto{\pgfqpoint{2.596134in}{1.326663in}}{\pgfqpoint{2.585535in}{1.322272in}}{\pgfqpoint{2.577721in}{1.314459in}}%
\pgfpathcurveto{\pgfqpoint{2.569908in}{1.306645in}}{\pgfqpoint{2.565518in}{1.296046in}}{\pgfqpoint{2.565518in}{1.284996in}}%
\pgfpathcurveto{\pgfqpoint{2.565518in}{1.273946in}}{\pgfqpoint{2.569908in}{1.263347in}}{\pgfqpoint{2.577721in}{1.255533in}}%
\pgfpathcurveto{\pgfqpoint{2.585535in}{1.247720in}}{\pgfqpoint{2.596134in}{1.243329in}}{\pgfqpoint{2.607184in}{1.243329in}}%
\pgfpathlineto{\pgfqpoint{2.607184in}{1.243329in}}%
\pgfpathclose%
\pgfusepath{stroke}%
\end{pgfscope}%
\begin{pgfscope}%
\pgfpathrectangle{\pgfqpoint{0.847223in}{0.554012in}}{\pgfqpoint{6.200000in}{4.620000in}}%
\pgfusepath{clip}%
\pgfsetbuttcap%
\pgfsetroundjoin%
\pgfsetlinewidth{1.003750pt}%
\definecolor{currentstroke}{rgb}{1.000000,0.000000,0.000000}%
\pgfsetstrokecolor{currentstroke}%
\pgfsetdash{}{0pt}%
\pgfpathmoveto{\pgfqpoint{2.612517in}{1.240634in}}%
\pgfpathcurveto{\pgfqpoint{2.623568in}{1.240634in}}{\pgfqpoint{2.634167in}{1.245025in}}{\pgfqpoint{2.641980in}{1.252838in}}%
\pgfpathcurveto{\pgfqpoint{2.649794in}{1.260652in}}{\pgfqpoint{2.654184in}{1.271251in}}{\pgfqpoint{2.654184in}{1.282301in}}%
\pgfpathcurveto{\pgfqpoint{2.654184in}{1.293351in}}{\pgfqpoint{2.649794in}{1.303950in}}{\pgfqpoint{2.641980in}{1.311764in}}%
\pgfpathcurveto{\pgfqpoint{2.634167in}{1.319577in}}{\pgfqpoint{2.623568in}{1.323968in}}{\pgfqpoint{2.612517in}{1.323968in}}%
\pgfpathcurveto{\pgfqpoint{2.601467in}{1.323968in}}{\pgfqpoint{2.590868in}{1.319577in}}{\pgfqpoint{2.583055in}{1.311764in}}%
\pgfpathcurveto{\pgfqpoint{2.575241in}{1.303950in}}{\pgfqpoint{2.570851in}{1.293351in}}{\pgfqpoint{2.570851in}{1.282301in}}%
\pgfpathcurveto{\pgfqpoint{2.570851in}{1.271251in}}{\pgfqpoint{2.575241in}{1.260652in}}{\pgfqpoint{2.583055in}{1.252838in}}%
\pgfpathcurveto{\pgfqpoint{2.590868in}{1.245025in}}{\pgfqpoint{2.601467in}{1.240634in}}{\pgfqpoint{2.612517in}{1.240634in}}%
\pgfpathlineto{\pgfqpoint{2.612517in}{1.240634in}}%
\pgfpathclose%
\pgfusepath{stroke}%
\end{pgfscope}%
\begin{pgfscope}%
\pgfpathrectangle{\pgfqpoint{0.847223in}{0.554012in}}{\pgfqpoint{6.200000in}{4.620000in}}%
\pgfusepath{clip}%
\pgfsetbuttcap%
\pgfsetroundjoin%
\pgfsetlinewidth{1.003750pt}%
\definecolor{currentstroke}{rgb}{1.000000,0.000000,0.000000}%
\pgfsetstrokecolor{currentstroke}%
\pgfsetdash{}{0pt}%
\pgfpathmoveto{\pgfqpoint{2.617851in}{1.237952in}}%
\pgfpathcurveto{\pgfqpoint{2.628901in}{1.237952in}}{\pgfqpoint{2.639500in}{1.242342in}}{\pgfqpoint{2.647313in}{1.250156in}}%
\pgfpathcurveto{\pgfqpoint{2.655127in}{1.257970in}}{\pgfqpoint{2.659517in}{1.268569in}}{\pgfqpoint{2.659517in}{1.279619in}}%
\pgfpathcurveto{\pgfqpoint{2.659517in}{1.290669in}}{\pgfqpoint{2.655127in}{1.301268in}}{\pgfqpoint{2.647313in}{1.309081in}}%
\pgfpathcurveto{\pgfqpoint{2.639500in}{1.316895in}}{\pgfqpoint{2.628901in}{1.321285in}}{\pgfqpoint{2.617851in}{1.321285in}}%
\pgfpathcurveto{\pgfqpoint{2.606801in}{1.321285in}}{\pgfqpoint{2.596201in}{1.316895in}}{\pgfqpoint{2.588388in}{1.309081in}}%
\pgfpathcurveto{\pgfqpoint{2.580574in}{1.301268in}}{\pgfqpoint{2.576184in}{1.290669in}}{\pgfqpoint{2.576184in}{1.279619in}}%
\pgfpathcurveto{\pgfqpoint{2.576184in}{1.268569in}}{\pgfqpoint{2.580574in}{1.257970in}}{\pgfqpoint{2.588388in}{1.250156in}}%
\pgfpathcurveto{\pgfqpoint{2.596201in}{1.242342in}}{\pgfqpoint{2.606801in}{1.237952in}}{\pgfqpoint{2.617851in}{1.237952in}}%
\pgfpathlineto{\pgfqpoint{2.617851in}{1.237952in}}%
\pgfpathclose%
\pgfusepath{stroke}%
\end{pgfscope}%
\begin{pgfscope}%
\pgfpathrectangle{\pgfqpoint{0.847223in}{0.554012in}}{\pgfqpoint{6.200000in}{4.620000in}}%
\pgfusepath{clip}%
\pgfsetbuttcap%
\pgfsetroundjoin%
\pgfsetlinewidth{1.003750pt}%
\definecolor{currentstroke}{rgb}{1.000000,0.000000,0.000000}%
\pgfsetstrokecolor{currentstroke}%
\pgfsetdash{}{0pt}%
\pgfpathmoveto{\pgfqpoint{2.623184in}{1.235282in}}%
\pgfpathcurveto{\pgfqpoint{2.634234in}{1.235282in}}{\pgfqpoint{2.644833in}{1.239672in}}{\pgfqpoint{2.652647in}{1.247486in}}%
\pgfpathcurveto{\pgfqpoint{2.660460in}{1.255300in}}{\pgfqpoint{2.664851in}{1.265899in}}{\pgfqpoint{2.664851in}{1.276949in}}%
\pgfpathcurveto{\pgfqpoint{2.664851in}{1.287999in}}{\pgfqpoint{2.660460in}{1.298598in}}{\pgfqpoint{2.652647in}{1.306412in}}%
\pgfpathcurveto{\pgfqpoint{2.644833in}{1.314225in}}{\pgfqpoint{2.634234in}{1.318615in}}{\pgfqpoint{2.623184in}{1.318615in}}%
\pgfpathcurveto{\pgfqpoint{2.612134in}{1.318615in}}{\pgfqpoint{2.601535in}{1.314225in}}{\pgfqpoint{2.593721in}{1.306412in}}%
\pgfpathcurveto{\pgfqpoint{2.585907in}{1.298598in}}{\pgfqpoint{2.581517in}{1.287999in}}{\pgfqpoint{2.581517in}{1.276949in}}%
\pgfpathcurveto{\pgfqpoint{2.581517in}{1.265899in}}{\pgfqpoint{2.585907in}{1.255300in}}{\pgfqpoint{2.593721in}{1.247486in}}%
\pgfpathcurveto{\pgfqpoint{2.601535in}{1.239672in}}{\pgfqpoint{2.612134in}{1.235282in}}{\pgfqpoint{2.623184in}{1.235282in}}%
\pgfpathlineto{\pgfqpoint{2.623184in}{1.235282in}}%
\pgfpathclose%
\pgfusepath{stroke}%
\end{pgfscope}%
\begin{pgfscope}%
\pgfpathrectangle{\pgfqpoint{0.847223in}{0.554012in}}{\pgfqpoint{6.200000in}{4.620000in}}%
\pgfusepath{clip}%
\pgfsetbuttcap%
\pgfsetroundjoin%
\pgfsetlinewidth{1.003750pt}%
\definecolor{currentstroke}{rgb}{1.000000,0.000000,0.000000}%
\pgfsetstrokecolor{currentstroke}%
\pgfsetdash{}{0pt}%
\pgfpathmoveto{\pgfqpoint{2.628517in}{1.232625in}}%
\pgfpathcurveto{\pgfqpoint{2.639567in}{1.232625in}}{\pgfqpoint{2.650166in}{1.237015in}}{\pgfqpoint{2.657980in}{1.244828in}}%
\pgfpathcurveto{\pgfqpoint{2.665793in}{1.252642in}}{\pgfqpoint{2.670184in}{1.263241in}}{\pgfqpoint{2.670184in}{1.274291in}}%
\pgfpathcurveto{\pgfqpoint{2.670184in}{1.285341in}}{\pgfqpoint{2.665793in}{1.295940in}}{\pgfqpoint{2.657980in}{1.303754in}}%
\pgfpathcurveto{\pgfqpoint{2.650166in}{1.311568in}}{\pgfqpoint{2.639567in}{1.315958in}}{\pgfqpoint{2.628517in}{1.315958in}}%
\pgfpathcurveto{\pgfqpoint{2.617467in}{1.315958in}}{\pgfqpoint{2.606868in}{1.311568in}}{\pgfqpoint{2.599054in}{1.303754in}}%
\pgfpathcurveto{\pgfqpoint{2.591241in}{1.295940in}}{\pgfqpoint{2.586850in}{1.285341in}}{\pgfqpoint{2.586850in}{1.274291in}}%
\pgfpathcurveto{\pgfqpoint{2.586850in}{1.263241in}}{\pgfqpoint{2.591241in}{1.252642in}}{\pgfqpoint{2.599054in}{1.244828in}}%
\pgfpathcurveto{\pgfqpoint{2.606868in}{1.237015in}}{\pgfqpoint{2.617467in}{1.232625in}}{\pgfqpoint{2.628517in}{1.232625in}}%
\pgfpathlineto{\pgfqpoint{2.628517in}{1.232625in}}%
\pgfpathclose%
\pgfusepath{stroke}%
\end{pgfscope}%
\begin{pgfscope}%
\pgfpathrectangle{\pgfqpoint{0.847223in}{0.554012in}}{\pgfqpoint{6.200000in}{4.620000in}}%
\pgfusepath{clip}%
\pgfsetbuttcap%
\pgfsetroundjoin%
\pgfsetlinewidth{1.003750pt}%
\definecolor{currentstroke}{rgb}{1.000000,0.000000,0.000000}%
\pgfsetstrokecolor{currentstroke}%
\pgfsetdash{}{0pt}%
\pgfpathmoveto{\pgfqpoint{2.633850in}{1.229979in}}%
\pgfpathcurveto{\pgfqpoint{2.644900in}{1.229979in}}{\pgfqpoint{2.655499in}{1.234370in}}{\pgfqpoint{2.663313in}{1.242183in}}%
\pgfpathcurveto{\pgfqpoint{2.671127in}{1.249997in}}{\pgfqpoint{2.675517in}{1.260596in}}{\pgfqpoint{2.675517in}{1.271646in}}%
\pgfpathcurveto{\pgfqpoint{2.675517in}{1.282696in}}{\pgfqpoint{2.671127in}{1.293295in}}{\pgfqpoint{2.663313in}{1.301109in}}%
\pgfpathcurveto{\pgfqpoint{2.655499in}{1.308922in}}{\pgfqpoint{2.644900in}{1.313313in}}{\pgfqpoint{2.633850in}{1.313313in}}%
\pgfpathcurveto{\pgfqpoint{2.622800in}{1.313313in}}{\pgfqpoint{2.612201in}{1.308922in}}{\pgfqpoint{2.604388in}{1.301109in}}%
\pgfpathcurveto{\pgfqpoint{2.596574in}{1.293295in}}{\pgfqpoint{2.592184in}{1.282696in}}{\pgfqpoint{2.592184in}{1.271646in}}%
\pgfpathcurveto{\pgfqpoint{2.592184in}{1.260596in}}{\pgfqpoint{2.596574in}{1.249997in}}{\pgfqpoint{2.604388in}{1.242183in}}%
\pgfpathcurveto{\pgfqpoint{2.612201in}{1.234370in}}{\pgfqpoint{2.622800in}{1.229979in}}{\pgfqpoint{2.633850in}{1.229979in}}%
\pgfpathlineto{\pgfqpoint{2.633850in}{1.229979in}}%
\pgfpathclose%
\pgfusepath{stroke}%
\end{pgfscope}%
\begin{pgfscope}%
\pgfpathrectangle{\pgfqpoint{0.847223in}{0.554012in}}{\pgfqpoint{6.200000in}{4.620000in}}%
\pgfusepath{clip}%
\pgfsetbuttcap%
\pgfsetroundjoin%
\pgfsetlinewidth{1.003750pt}%
\definecolor{currentstroke}{rgb}{1.000000,0.000000,0.000000}%
\pgfsetstrokecolor{currentstroke}%
\pgfsetdash{}{0pt}%
\pgfpathmoveto{\pgfqpoint{2.639184in}{1.227346in}}%
\pgfpathcurveto{\pgfqpoint{2.650234in}{1.227346in}}{\pgfqpoint{2.660833in}{1.231736in}}{\pgfqpoint{2.668646in}{1.239550in}}%
\pgfpathcurveto{\pgfqpoint{2.676460in}{1.247364in}}{\pgfqpoint{2.680850in}{1.257963in}}{\pgfqpoint{2.680850in}{1.269013in}}%
\pgfpathcurveto{\pgfqpoint{2.680850in}{1.280063in}}{\pgfqpoint{2.676460in}{1.290662in}}{\pgfqpoint{2.668646in}{1.298476in}}%
\pgfpathcurveto{\pgfqpoint{2.660833in}{1.306289in}}{\pgfqpoint{2.650234in}{1.310680in}}{\pgfqpoint{2.639184in}{1.310680in}}%
\pgfpathcurveto{\pgfqpoint{2.628133in}{1.310680in}}{\pgfqpoint{2.617534in}{1.306289in}}{\pgfqpoint{2.609721in}{1.298476in}}%
\pgfpathcurveto{\pgfqpoint{2.601907in}{1.290662in}}{\pgfqpoint{2.597517in}{1.280063in}}{\pgfqpoint{2.597517in}{1.269013in}}%
\pgfpathcurveto{\pgfqpoint{2.597517in}{1.257963in}}{\pgfqpoint{2.601907in}{1.247364in}}{\pgfqpoint{2.609721in}{1.239550in}}%
\pgfpathcurveto{\pgfqpoint{2.617534in}{1.231736in}}{\pgfqpoint{2.628133in}{1.227346in}}{\pgfqpoint{2.639184in}{1.227346in}}%
\pgfpathlineto{\pgfqpoint{2.639184in}{1.227346in}}%
\pgfpathclose%
\pgfusepath{stroke}%
\end{pgfscope}%
\begin{pgfscope}%
\pgfpathrectangle{\pgfqpoint{0.847223in}{0.554012in}}{\pgfqpoint{6.200000in}{4.620000in}}%
\pgfusepath{clip}%
\pgfsetbuttcap%
\pgfsetroundjoin%
\pgfsetlinewidth{1.003750pt}%
\definecolor{currentstroke}{rgb}{1.000000,0.000000,0.000000}%
\pgfsetstrokecolor{currentstroke}%
\pgfsetdash{}{0pt}%
\pgfpathmoveto{\pgfqpoint{2.644517in}{1.224725in}}%
\pgfpathcurveto{\pgfqpoint{2.655567in}{1.224725in}}{\pgfqpoint{2.666166in}{1.229115in}}{\pgfqpoint{2.673980in}{1.236929in}}%
\pgfpathcurveto{\pgfqpoint{2.681793in}{1.244743in}}{\pgfqpoint{2.686183in}{1.255342in}}{\pgfqpoint{2.686183in}{1.266392in}}%
\pgfpathcurveto{\pgfqpoint{2.686183in}{1.277442in}}{\pgfqpoint{2.681793in}{1.288041in}}{\pgfqpoint{2.673980in}{1.295855in}}%
\pgfpathcurveto{\pgfqpoint{2.666166in}{1.303668in}}{\pgfqpoint{2.655567in}{1.308059in}}{\pgfqpoint{2.644517in}{1.308059in}}%
\pgfpathcurveto{\pgfqpoint{2.633467in}{1.308059in}}{\pgfqpoint{2.622868in}{1.303668in}}{\pgfqpoint{2.615054in}{1.295855in}}%
\pgfpathcurveto{\pgfqpoint{2.607240in}{1.288041in}}{\pgfqpoint{2.602850in}{1.277442in}}{\pgfqpoint{2.602850in}{1.266392in}}%
\pgfpathcurveto{\pgfqpoint{2.602850in}{1.255342in}}{\pgfqpoint{2.607240in}{1.244743in}}{\pgfqpoint{2.615054in}{1.236929in}}%
\pgfpathcurveto{\pgfqpoint{2.622868in}{1.229115in}}{\pgfqpoint{2.633467in}{1.224725in}}{\pgfqpoint{2.644517in}{1.224725in}}%
\pgfpathlineto{\pgfqpoint{2.644517in}{1.224725in}}%
\pgfpathclose%
\pgfusepath{stroke}%
\end{pgfscope}%
\begin{pgfscope}%
\pgfpathrectangle{\pgfqpoint{0.847223in}{0.554012in}}{\pgfqpoint{6.200000in}{4.620000in}}%
\pgfusepath{clip}%
\pgfsetbuttcap%
\pgfsetroundjoin%
\pgfsetlinewidth{1.003750pt}%
\definecolor{currentstroke}{rgb}{1.000000,0.000000,0.000000}%
\pgfsetstrokecolor{currentstroke}%
\pgfsetdash{}{0pt}%
\pgfpathmoveto{\pgfqpoint{2.649850in}{1.222116in}}%
\pgfpathcurveto{\pgfqpoint{2.660900in}{1.222116in}}{\pgfqpoint{2.671499in}{1.226507in}}{\pgfqpoint{2.679313in}{1.234320in}}%
\pgfpathcurveto{\pgfqpoint{2.687126in}{1.242134in}}{\pgfqpoint{2.691517in}{1.252733in}}{\pgfqpoint{2.691517in}{1.263783in}}%
\pgfpathcurveto{\pgfqpoint{2.691517in}{1.274833in}}{\pgfqpoint{2.687126in}{1.285432in}}{\pgfqpoint{2.679313in}{1.293246in}}%
\pgfpathcurveto{\pgfqpoint{2.671499in}{1.301059in}}{\pgfqpoint{2.660900in}{1.305450in}}{\pgfqpoint{2.649850in}{1.305450in}}%
\pgfpathcurveto{\pgfqpoint{2.638800in}{1.305450in}}{\pgfqpoint{2.628201in}{1.301059in}}{\pgfqpoint{2.620387in}{1.293246in}}%
\pgfpathcurveto{\pgfqpoint{2.612574in}{1.285432in}}{\pgfqpoint{2.608183in}{1.274833in}}{\pgfqpoint{2.608183in}{1.263783in}}%
\pgfpathcurveto{\pgfqpoint{2.608183in}{1.252733in}}{\pgfqpoint{2.612574in}{1.242134in}}{\pgfqpoint{2.620387in}{1.234320in}}%
\pgfpathcurveto{\pgfqpoint{2.628201in}{1.226507in}}{\pgfqpoint{2.638800in}{1.222116in}}{\pgfqpoint{2.649850in}{1.222116in}}%
\pgfpathlineto{\pgfqpoint{2.649850in}{1.222116in}}%
\pgfpathclose%
\pgfusepath{stroke}%
\end{pgfscope}%
\begin{pgfscope}%
\pgfpathrectangle{\pgfqpoint{0.847223in}{0.554012in}}{\pgfqpoint{6.200000in}{4.620000in}}%
\pgfusepath{clip}%
\pgfsetbuttcap%
\pgfsetroundjoin%
\pgfsetlinewidth{1.003750pt}%
\definecolor{currentstroke}{rgb}{1.000000,0.000000,0.000000}%
\pgfsetstrokecolor{currentstroke}%
\pgfsetdash{}{0pt}%
\pgfpathmoveto{\pgfqpoint{2.655183in}{1.219519in}}%
\pgfpathcurveto{\pgfqpoint{2.666233in}{1.219519in}}{\pgfqpoint{2.676832in}{1.223909in}}{\pgfqpoint{2.684646in}{1.231723in}}%
\pgfpathcurveto{\pgfqpoint{2.692460in}{1.239537in}}{\pgfqpoint{2.696850in}{1.250136in}}{\pgfqpoint{2.696850in}{1.261186in}}%
\pgfpathcurveto{\pgfqpoint{2.696850in}{1.272236in}}{\pgfqpoint{2.692460in}{1.282835in}}{\pgfqpoint{2.684646in}{1.290649in}}%
\pgfpathcurveto{\pgfqpoint{2.676832in}{1.298462in}}{\pgfqpoint{2.666233in}{1.302853in}}{\pgfqpoint{2.655183in}{1.302853in}}%
\pgfpathcurveto{\pgfqpoint{2.644133in}{1.302853in}}{\pgfqpoint{2.633534in}{1.298462in}}{\pgfqpoint{2.625720in}{1.290649in}}%
\pgfpathcurveto{\pgfqpoint{2.617907in}{1.282835in}}{\pgfqpoint{2.613516in}{1.272236in}}{\pgfqpoint{2.613516in}{1.261186in}}%
\pgfpathcurveto{\pgfqpoint{2.613516in}{1.250136in}}{\pgfqpoint{2.617907in}{1.239537in}}{\pgfqpoint{2.625720in}{1.231723in}}%
\pgfpathcurveto{\pgfqpoint{2.633534in}{1.223909in}}{\pgfqpoint{2.644133in}{1.219519in}}{\pgfqpoint{2.655183in}{1.219519in}}%
\pgfpathlineto{\pgfqpoint{2.655183in}{1.219519in}}%
\pgfpathclose%
\pgfusepath{stroke}%
\end{pgfscope}%
\begin{pgfscope}%
\pgfpathrectangle{\pgfqpoint{0.847223in}{0.554012in}}{\pgfqpoint{6.200000in}{4.620000in}}%
\pgfusepath{clip}%
\pgfsetbuttcap%
\pgfsetroundjoin%
\pgfsetlinewidth{1.003750pt}%
\definecolor{currentstroke}{rgb}{1.000000,0.000000,0.000000}%
\pgfsetstrokecolor{currentstroke}%
\pgfsetdash{}{0pt}%
\pgfpathmoveto{\pgfqpoint{2.660516in}{1.216934in}}%
\pgfpathcurveto{\pgfqpoint{2.671567in}{1.216934in}}{\pgfqpoint{2.682166in}{1.221324in}}{\pgfqpoint{2.689979in}{1.229138in}}%
\pgfpathcurveto{\pgfqpoint{2.697793in}{1.236952in}}{\pgfqpoint{2.702183in}{1.247551in}}{\pgfqpoint{2.702183in}{1.258601in}}%
\pgfpathcurveto{\pgfqpoint{2.702183in}{1.269651in}}{\pgfqpoint{2.697793in}{1.280250in}}{\pgfqpoint{2.689979in}{1.288063in}}%
\pgfpathcurveto{\pgfqpoint{2.682166in}{1.295877in}}{\pgfqpoint{2.671567in}{1.300267in}}{\pgfqpoint{2.660516in}{1.300267in}}%
\pgfpathcurveto{\pgfqpoint{2.649466in}{1.300267in}}{\pgfqpoint{2.638867in}{1.295877in}}{\pgfqpoint{2.631054in}{1.288063in}}%
\pgfpathcurveto{\pgfqpoint{2.623240in}{1.280250in}}{\pgfqpoint{2.618850in}{1.269651in}}{\pgfqpoint{2.618850in}{1.258601in}}%
\pgfpathcurveto{\pgfqpoint{2.618850in}{1.247551in}}{\pgfqpoint{2.623240in}{1.236952in}}{\pgfqpoint{2.631054in}{1.229138in}}%
\pgfpathcurveto{\pgfqpoint{2.638867in}{1.221324in}}{\pgfqpoint{2.649466in}{1.216934in}}{\pgfqpoint{2.660516in}{1.216934in}}%
\pgfpathlineto{\pgfqpoint{2.660516in}{1.216934in}}%
\pgfpathclose%
\pgfusepath{stroke}%
\end{pgfscope}%
\begin{pgfscope}%
\pgfpathrectangle{\pgfqpoint{0.847223in}{0.554012in}}{\pgfqpoint{6.200000in}{4.620000in}}%
\pgfusepath{clip}%
\pgfsetbuttcap%
\pgfsetroundjoin%
\pgfsetlinewidth{1.003750pt}%
\definecolor{currentstroke}{rgb}{1.000000,0.000000,0.000000}%
\pgfsetstrokecolor{currentstroke}%
\pgfsetdash{}{0pt}%
\pgfpathmoveto{\pgfqpoint{2.665850in}{1.214361in}}%
\pgfpathcurveto{\pgfqpoint{2.676900in}{1.214361in}}{\pgfqpoint{2.687499in}{1.218751in}}{\pgfqpoint{2.695312in}{1.226565in}}%
\pgfpathcurveto{\pgfqpoint{2.703126in}{1.234378in}}{\pgfqpoint{2.707516in}{1.244977in}}{\pgfqpoint{2.707516in}{1.256027in}}%
\pgfpathcurveto{\pgfqpoint{2.707516in}{1.267077in}}{\pgfqpoint{2.703126in}{1.277676in}}{\pgfqpoint{2.695312in}{1.285490in}}%
\pgfpathcurveto{\pgfqpoint{2.687499in}{1.293304in}}{\pgfqpoint{2.676900in}{1.297694in}}{\pgfqpoint{2.665850in}{1.297694in}}%
\pgfpathcurveto{\pgfqpoint{2.654799in}{1.297694in}}{\pgfqpoint{2.644200in}{1.293304in}}{\pgfqpoint{2.636387in}{1.285490in}}%
\pgfpathcurveto{\pgfqpoint{2.628573in}{1.277676in}}{\pgfqpoint{2.624183in}{1.267077in}}{\pgfqpoint{2.624183in}{1.256027in}}%
\pgfpathcurveto{\pgfqpoint{2.624183in}{1.244977in}}{\pgfqpoint{2.628573in}{1.234378in}}{\pgfqpoint{2.636387in}{1.226565in}}%
\pgfpathcurveto{\pgfqpoint{2.644200in}{1.218751in}}{\pgfqpoint{2.654799in}{1.214361in}}{\pgfqpoint{2.665850in}{1.214361in}}%
\pgfpathlineto{\pgfqpoint{2.665850in}{1.214361in}}%
\pgfpathclose%
\pgfusepath{stroke}%
\end{pgfscope}%
\begin{pgfscope}%
\pgfpathrectangle{\pgfqpoint{0.847223in}{0.554012in}}{\pgfqpoint{6.200000in}{4.620000in}}%
\pgfusepath{clip}%
\pgfsetbuttcap%
\pgfsetroundjoin%
\pgfsetlinewidth{1.003750pt}%
\definecolor{currentstroke}{rgb}{1.000000,0.000000,0.000000}%
\pgfsetstrokecolor{currentstroke}%
\pgfsetdash{}{0pt}%
\pgfpathmoveto{\pgfqpoint{2.671183in}{1.211799in}}%
\pgfpathcurveto{\pgfqpoint{2.682233in}{1.211799in}}{\pgfqpoint{2.692832in}{1.216189in}}{\pgfqpoint{2.700646in}{1.224003in}}%
\pgfpathcurveto{\pgfqpoint{2.708459in}{1.231816in}}{\pgfqpoint{2.712849in}{1.242415in}}{\pgfqpoint{2.712849in}{1.253466in}}%
\pgfpathcurveto{\pgfqpoint{2.712849in}{1.264516in}}{\pgfqpoint{2.708459in}{1.275115in}}{\pgfqpoint{2.700646in}{1.282928in}}%
\pgfpathcurveto{\pgfqpoint{2.692832in}{1.290742in}}{\pgfqpoint{2.682233in}{1.295132in}}{\pgfqpoint{2.671183in}{1.295132in}}%
\pgfpathcurveto{\pgfqpoint{2.660133in}{1.295132in}}{\pgfqpoint{2.649534in}{1.290742in}}{\pgfqpoint{2.641720in}{1.282928in}}%
\pgfpathcurveto{\pgfqpoint{2.633906in}{1.275115in}}{\pgfqpoint{2.629516in}{1.264516in}}{\pgfqpoint{2.629516in}{1.253466in}}%
\pgfpathcurveto{\pgfqpoint{2.629516in}{1.242415in}}{\pgfqpoint{2.633906in}{1.231816in}}{\pgfqpoint{2.641720in}{1.224003in}}%
\pgfpathcurveto{\pgfqpoint{2.649534in}{1.216189in}}{\pgfqpoint{2.660133in}{1.211799in}}{\pgfqpoint{2.671183in}{1.211799in}}%
\pgfpathlineto{\pgfqpoint{2.671183in}{1.211799in}}%
\pgfpathclose%
\pgfusepath{stroke}%
\end{pgfscope}%
\begin{pgfscope}%
\pgfpathrectangle{\pgfqpoint{0.847223in}{0.554012in}}{\pgfqpoint{6.200000in}{4.620000in}}%
\pgfusepath{clip}%
\pgfsetbuttcap%
\pgfsetroundjoin%
\pgfsetlinewidth{1.003750pt}%
\definecolor{currentstroke}{rgb}{1.000000,0.000000,0.000000}%
\pgfsetstrokecolor{currentstroke}%
\pgfsetdash{}{0pt}%
\pgfpathmoveto{\pgfqpoint{2.676516in}{1.209249in}}%
\pgfpathcurveto{\pgfqpoint{2.687566in}{1.209249in}}{\pgfqpoint{2.698165in}{1.213639in}}{\pgfqpoint{2.705979in}{1.221453in}}%
\pgfpathcurveto{\pgfqpoint{2.713792in}{1.229266in}}{\pgfqpoint{2.718183in}{1.239865in}}{\pgfqpoint{2.718183in}{1.250915in}}%
\pgfpathcurveto{\pgfqpoint{2.718183in}{1.261966in}}{\pgfqpoint{2.713792in}{1.272565in}}{\pgfqpoint{2.705979in}{1.280378in}}%
\pgfpathcurveto{\pgfqpoint{2.698165in}{1.288192in}}{\pgfqpoint{2.687566in}{1.292582in}}{\pgfqpoint{2.676516in}{1.292582in}}%
\pgfpathcurveto{\pgfqpoint{2.665466in}{1.292582in}}{\pgfqpoint{2.654867in}{1.288192in}}{\pgfqpoint{2.647053in}{1.280378in}}%
\pgfpathcurveto{\pgfqpoint{2.639240in}{1.272565in}}{\pgfqpoint{2.634849in}{1.261966in}}{\pgfqpoint{2.634849in}{1.250915in}}%
\pgfpathcurveto{\pgfqpoint{2.634849in}{1.239865in}}{\pgfqpoint{2.639240in}{1.229266in}}{\pgfqpoint{2.647053in}{1.221453in}}%
\pgfpathcurveto{\pgfqpoint{2.654867in}{1.213639in}}{\pgfqpoint{2.665466in}{1.209249in}}{\pgfqpoint{2.676516in}{1.209249in}}%
\pgfpathlineto{\pgfqpoint{2.676516in}{1.209249in}}%
\pgfpathclose%
\pgfusepath{stroke}%
\end{pgfscope}%
\begin{pgfscope}%
\pgfpathrectangle{\pgfqpoint{0.847223in}{0.554012in}}{\pgfqpoint{6.200000in}{4.620000in}}%
\pgfusepath{clip}%
\pgfsetbuttcap%
\pgfsetroundjoin%
\pgfsetlinewidth{1.003750pt}%
\definecolor{currentstroke}{rgb}{1.000000,0.000000,0.000000}%
\pgfsetstrokecolor{currentstroke}%
\pgfsetdash{}{0pt}%
\pgfpathmoveto{\pgfqpoint{2.681849in}{1.206710in}}%
\pgfpathcurveto{\pgfqpoint{2.692899in}{1.206710in}}{\pgfqpoint{2.703498in}{1.211101in}}{\pgfqpoint{2.711312in}{1.218914in}}%
\pgfpathcurveto{\pgfqpoint{2.719126in}{1.226728in}}{\pgfqpoint{2.723516in}{1.237327in}}{\pgfqpoint{2.723516in}{1.248377in}}%
\pgfpathcurveto{\pgfqpoint{2.723516in}{1.259427in}}{\pgfqpoint{2.719126in}{1.270026in}}{\pgfqpoint{2.711312in}{1.277840in}}%
\pgfpathcurveto{\pgfqpoint{2.703498in}{1.285653in}}{\pgfqpoint{2.692899in}{1.290044in}}{\pgfqpoint{2.681849in}{1.290044in}}%
\pgfpathcurveto{\pgfqpoint{2.670799in}{1.290044in}}{\pgfqpoint{2.660200in}{1.285653in}}{\pgfqpoint{2.652386in}{1.277840in}}%
\pgfpathcurveto{\pgfqpoint{2.644573in}{1.270026in}}{\pgfqpoint{2.640183in}{1.259427in}}{\pgfqpoint{2.640183in}{1.248377in}}%
\pgfpathcurveto{\pgfqpoint{2.640183in}{1.237327in}}{\pgfqpoint{2.644573in}{1.226728in}}{\pgfqpoint{2.652386in}{1.218914in}}%
\pgfpathcurveto{\pgfqpoint{2.660200in}{1.211101in}}{\pgfqpoint{2.670799in}{1.206710in}}{\pgfqpoint{2.681849in}{1.206710in}}%
\pgfpathlineto{\pgfqpoint{2.681849in}{1.206710in}}%
\pgfpathclose%
\pgfusepath{stroke}%
\end{pgfscope}%
\begin{pgfscope}%
\pgfpathrectangle{\pgfqpoint{0.847223in}{0.554012in}}{\pgfqpoint{6.200000in}{4.620000in}}%
\pgfusepath{clip}%
\pgfsetbuttcap%
\pgfsetroundjoin%
\pgfsetlinewidth{1.003750pt}%
\definecolor{currentstroke}{rgb}{1.000000,0.000000,0.000000}%
\pgfsetstrokecolor{currentstroke}%
\pgfsetdash{}{0pt}%
\pgfpathmoveto{\pgfqpoint{2.687182in}{1.204183in}}%
\pgfpathcurveto{\pgfqpoint{2.698233in}{1.204183in}}{\pgfqpoint{2.708832in}{1.208573in}}{\pgfqpoint{2.716645in}{1.216387in}}%
\pgfpathcurveto{\pgfqpoint{2.724459in}{1.224201in}}{\pgfqpoint{2.728849in}{1.234800in}}{\pgfqpoint{2.728849in}{1.245850in}}%
\pgfpathcurveto{\pgfqpoint{2.728849in}{1.256900in}}{\pgfqpoint{2.724459in}{1.267499in}}{\pgfqpoint{2.716645in}{1.275313in}}%
\pgfpathcurveto{\pgfqpoint{2.708832in}{1.283126in}}{\pgfqpoint{2.698233in}{1.287516in}}{\pgfqpoint{2.687182in}{1.287516in}}%
\pgfpathcurveto{\pgfqpoint{2.676132in}{1.287516in}}{\pgfqpoint{2.665533in}{1.283126in}}{\pgfqpoint{2.657720in}{1.275313in}}%
\pgfpathcurveto{\pgfqpoint{2.649906in}{1.267499in}}{\pgfqpoint{2.645516in}{1.256900in}}{\pgfqpoint{2.645516in}{1.245850in}}%
\pgfpathcurveto{\pgfqpoint{2.645516in}{1.234800in}}{\pgfqpoint{2.649906in}{1.224201in}}{\pgfqpoint{2.657720in}{1.216387in}}%
\pgfpathcurveto{\pgfqpoint{2.665533in}{1.208573in}}{\pgfqpoint{2.676132in}{1.204183in}}{\pgfqpoint{2.687182in}{1.204183in}}%
\pgfpathlineto{\pgfqpoint{2.687182in}{1.204183in}}%
\pgfpathclose%
\pgfusepath{stroke}%
\end{pgfscope}%
\begin{pgfscope}%
\pgfpathrectangle{\pgfqpoint{0.847223in}{0.554012in}}{\pgfqpoint{6.200000in}{4.620000in}}%
\pgfusepath{clip}%
\pgfsetbuttcap%
\pgfsetroundjoin%
\pgfsetlinewidth{1.003750pt}%
\definecolor{currentstroke}{rgb}{1.000000,0.000000,0.000000}%
\pgfsetstrokecolor{currentstroke}%
\pgfsetdash{}{0pt}%
\pgfpathmoveto{\pgfqpoint{2.692516in}{1.201667in}}%
\pgfpathcurveto{\pgfqpoint{2.703566in}{1.201667in}}{\pgfqpoint{2.714165in}{1.206058in}}{\pgfqpoint{2.721978in}{1.213871in}}%
\pgfpathcurveto{\pgfqpoint{2.729792in}{1.221685in}}{\pgfqpoint{2.734182in}{1.232284in}}{\pgfqpoint{2.734182in}{1.243334in}}%
\pgfpathcurveto{\pgfqpoint{2.734182in}{1.254384in}}{\pgfqpoint{2.729792in}{1.264983in}}{\pgfqpoint{2.721978in}{1.272797in}}%
\pgfpathcurveto{\pgfqpoint{2.714165in}{1.280611in}}{\pgfqpoint{2.703566in}{1.285001in}}{\pgfqpoint{2.692516in}{1.285001in}}%
\pgfpathcurveto{\pgfqpoint{2.681466in}{1.285001in}}{\pgfqpoint{2.670867in}{1.280611in}}{\pgfqpoint{2.663053in}{1.272797in}}%
\pgfpathcurveto{\pgfqpoint{2.655239in}{1.264983in}}{\pgfqpoint{2.650849in}{1.254384in}}{\pgfqpoint{2.650849in}{1.243334in}}%
\pgfpathcurveto{\pgfqpoint{2.650849in}{1.232284in}}{\pgfqpoint{2.655239in}{1.221685in}}{\pgfqpoint{2.663053in}{1.213871in}}%
\pgfpathcurveto{\pgfqpoint{2.670867in}{1.206058in}}{\pgfqpoint{2.681466in}{1.201667in}}{\pgfqpoint{2.692516in}{1.201667in}}%
\pgfpathlineto{\pgfqpoint{2.692516in}{1.201667in}}%
\pgfpathclose%
\pgfusepath{stroke}%
\end{pgfscope}%
\begin{pgfscope}%
\pgfpathrectangle{\pgfqpoint{0.847223in}{0.554012in}}{\pgfqpoint{6.200000in}{4.620000in}}%
\pgfusepath{clip}%
\pgfsetbuttcap%
\pgfsetroundjoin%
\pgfsetlinewidth{1.003750pt}%
\definecolor{currentstroke}{rgb}{1.000000,0.000000,0.000000}%
\pgfsetstrokecolor{currentstroke}%
\pgfsetdash{}{0pt}%
\pgfpathmoveto{\pgfqpoint{2.697849in}{1.199163in}}%
\pgfpathcurveto{\pgfqpoint{2.708899in}{1.199163in}}{\pgfqpoint{2.719498in}{1.203553in}}{\pgfqpoint{2.727312in}{1.211367in}}%
\pgfpathcurveto{\pgfqpoint{2.735125in}{1.219181in}}{\pgfqpoint{2.739516in}{1.229780in}}{\pgfqpoint{2.739516in}{1.240830in}}%
\pgfpathcurveto{\pgfqpoint{2.739516in}{1.251880in}}{\pgfqpoint{2.735125in}{1.262479in}}{\pgfqpoint{2.727312in}{1.270292in}}%
\pgfpathcurveto{\pgfqpoint{2.719498in}{1.278106in}}{\pgfqpoint{2.708899in}{1.282496in}}{\pgfqpoint{2.697849in}{1.282496in}}%
\pgfpathcurveto{\pgfqpoint{2.686799in}{1.282496in}}{\pgfqpoint{2.676200in}{1.278106in}}{\pgfqpoint{2.668386in}{1.270292in}}%
\pgfpathcurveto{\pgfqpoint{2.660572in}{1.262479in}}{\pgfqpoint{2.656182in}{1.251880in}}{\pgfqpoint{2.656182in}{1.240830in}}%
\pgfpathcurveto{\pgfqpoint{2.656182in}{1.229780in}}{\pgfqpoint{2.660572in}{1.219181in}}{\pgfqpoint{2.668386in}{1.211367in}}%
\pgfpathcurveto{\pgfqpoint{2.676200in}{1.203553in}}{\pgfqpoint{2.686799in}{1.199163in}}{\pgfqpoint{2.697849in}{1.199163in}}%
\pgfpathlineto{\pgfqpoint{2.697849in}{1.199163in}}%
\pgfpathclose%
\pgfusepath{stroke}%
\end{pgfscope}%
\begin{pgfscope}%
\pgfpathrectangle{\pgfqpoint{0.847223in}{0.554012in}}{\pgfqpoint{6.200000in}{4.620000in}}%
\pgfusepath{clip}%
\pgfsetbuttcap%
\pgfsetroundjoin%
\pgfsetlinewidth{1.003750pt}%
\definecolor{currentstroke}{rgb}{1.000000,0.000000,0.000000}%
\pgfsetstrokecolor{currentstroke}%
\pgfsetdash{}{0pt}%
\pgfpathmoveto{\pgfqpoint{2.703182in}{1.196670in}}%
\pgfpathcurveto{\pgfqpoint{2.714232in}{1.196670in}}{\pgfqpoint{2.724831in}{1.201060in}}{\pgfqpoint{2.732645in}{1.208874in}}%
\pgfpathcurveto{\pgfqpoint{2.740459in}{1.216687in}}{\pgfqpoint{2.744849in}{1.227286in}}{\pgfqpoint{2.744849in}{1.238336in}}%
\pgfpathcurveto{\pgfqpoint{2.744849in}{1.249387in}}{\pgfqpoint{2.740459in}{1.259986in}}{\pgfqpoint{2.732645in}{1.267799in}}%
\pgfpathcurveto{\pgfqpoint{2.724831in}{1.275613in}}{\pgfqpoint{2.714232in}{1.280003in}}{\pgfqpoint{2.703182in}{1.280003in}}%
\pgfpathcurveto{\pgfqpoint{2.692132in}{1.280003in}}{\pgfqpoint{2.681533in}{1.275613in}}{\pgfqpoint{2.673719in}{1.267799in}}%
\pgfpathcurveto{\pgfqpoint{2.665906in}{1.259986in}}{\pgfqpoint{2.661515in}{1.249387in}}{\pgfqpoint{2.661515in}{1.238336in}}%
\pgfpathcurveto{\pgfqpoint{2.661515in}{1.227286in}}{\pgfqpoint{2.665906in}{1.216687in}}{\pgfqpoint{2.673719in}{1.208874in}}%
\pgfpathcurveto{\pgfqpoint{2.681533in}{1.201060in}}{\pgfqpoint{2.692132in}{1.196670in}}{\pgfqpoint{2.703182in}{1.196670in}}%
\pgfpathlineto{\pgfqpoint{2.703182in}{1.196670in}}%
\pgfpathclose%
\pgfusepath{stroke}%
\end{pgfscope}%
\begin{pgfscope}%
\pgfpathrectangle{\pgfqpoint{0.847223in}{0.554012in}}{\pgfqpoint{6.200000in}{4.620000in}}%
\pgfusepath{clip}%
\pgfsetbuttcap%
\pgfsetroundjoin%
\pgfsetlinewidth{1.003750pt}%
\definecolor{currentstroke}{rgb}{1.000000,0.000000,0.000000}%
\pgfsetstrokecolor{currentstroke}%
\pgfsetdash{}{0pt}%
\pgfpathmoveto{\pgfqpoint{2.708515in}{1.194188in}}%
\pgfpathcurveto{\pgfqpoint{2.719565in}{1.194188in}}{\pgfqpoint{2.730164in}{1.198578in}}{\pgfqpoint{2.737978in}{1.206392in}}%
\pgfpathcurveto{\pgfqpoint{2.745792in}{1.214205in}}{\pgfqpoint{2.750182in}{1.224804in}}{\pgfqpoint{2.750182in}{1.235854in}}%
\pgfpathcurveto{\pgfqpoint{2.750182in}{1.246905in}}{\pgfqpoint{2.745792in}{1.257504in}}{\pgfqpoint{2.737978in}{1.265317in}}%
\pgfpathcurveto{\pgfqpoint{2.730164in}{1.273131in}}{\pgfqpoint{2.719565in}{1.277521in}}{\pgfqpoint{2.708515in}{1.277521in}}%
\pgfpathcurveto{\pgfqpoint{2.697465in}{1.277521in}}{\pgfqpoint{2.686866in}{1.273131in}}{\pgfqpoint{2.679053in}{1.265317in}}%
\pgfpathcurveto{\pgfqpoint{2.671239in}{1.257504in}}{\pgfqpoint{2.666849in}{1.246905in}}{\pgfqpoint{2.666849in}{1.235854in}}%
\pgfpathcurveto{\pgfqpoint{2.666849in}{1.224804in}}{\pgfqpoint{2.671239in}{1.214205in}}{\pgfqpoint{2.679053in}{1.206392in}}%
\pgfpathcurveto{\pgfqpoint{2.686866in}{1.198578in}}{\pgfqpoint{2.697465in}{1.194188in}}{\pgfqpoint{2.708515in}{1.194188in}}%
\pgfpathlineto{\pgfqpoint{2.708515in}{1.194188in}}%
\pgfpathclose%
\pgfusepath{stroke}%
\end{pgfscope}%
\begin{pgfscope}%
\pgfpathrectangle{\pgfqpoint{0.847223in}{0.554012in}}{\pgfqpoint{6.200000in}{4.620000in}}%
\pgfusepath{clip}%
\pgfsetbuttcap%
\pgfsetroundjoin%
\pgfsetlinewidth{1.003750pt}%
\definecolor{currentstroke}{rgb}{1.000000,0.000000,0.000000}%
\pgfsetstrokecolor{currentstroke}%
\pgfsetdash{}{0pt}%
\pgfpathmoveto{\pgfqpoint{2.713849in}{1.191717in}}%
\pgfpathcurveto{\pgfqpoint{2.724899in}{1.191717in}}{\pgfqpoint{2.735498in}{1.196107in}}{\pgfqpoint{2.743311in}{1.203921in}}%
\pgfpathcurveto{\pgfqpoint{2.751125in}{1.211734in}}{\pgfqpoint{2.755515in}{1.222333in}}{\pgfqpoint{2.755515in}{1.233383in}}%
\pgfpathcurveto{\pgfqpoint{2.755515in}{1.244434in}}{\pgfqpoint{2.751125in}{1.255033in}}{\pgfqpoint{2.743311in}{1.262846in}}%
\pgfpathcurveto{\pgfqpoint{2.735498in}{1.270660in}}{\pgfqpoint{2.724899in}{1.275050in}}{\pgfqpoint{2.713849in}{1.275050in}}%
\pgfpathcurveto{\pgfqpoint{2.702798in}{1.275050in}}{\pgfqpoint{2.692199in}{1.270660in}}{\pgfqpoint{2.684386in}{1.262846in}}%
\pgfpathcurveto{\pgfqpoint{2.676572in}{1.255033in}}{\pgfqpoint{2.672182in}{1.244434in}}{\pgfqpoint{2.672182in}{1.233383in}}%
\pgfpathcurveto{\pgfqpoint{2.672182in}{1.222333in}}{\pgfqpoint{2.676572in}{1.211734in}}{\pgfqpoint{2.684386in}{1.203921in}}%
\pgfpathcurveto{\pgfqpoint{2.692199in}{1.196107in}}{\pgfqpoint{2.702798in}{1.191717in}}{\pgfqpoint{2.713849in}{1.191717in}}%
\pgfpathlineto{\pgfqpoint{2.713849in}{1.191717in}}%
\pgfpathclose%
\pgfusepath{stroke}%
\end{pgfscope}%
\begin{pgfscope}%
\pgfpathrectangle{\pgfqpoint{0.847223in}{0.554012in}}{\pgfqpoint{6.200000in}{4.620000in}}%
\pgfusepath{clip}%
\pgfsetbuttcap%
\pgfsetroundjoin%
\pgfsetlinewidth{1.003750pt}%
\definecolor{currentstroke}{rgb}{1.000000,0.000000,0.000000}%
\pgfsetstrokecolor{currentstroke}%
\pgfsetdash{}{0pt}%
\pgfpathmoveto{\pgfqpoint{2.719182in}{1.189257in}}%
\pgfpathcurveto{\pgfqpoint{2.730232in}{1.189257in}}{\pgfqpoint{2.740831in}{1.193647in}}{\pgfqpoint{2.748645in}{1.201461in}}%
\pgfpathcurveto{\pgfqpoint{2.756458in}{1.209274in}}{\pgfqpoint{2.760848in}{1.219873in}}{\pgfqpoint{2.760848in}{1.230924in}}%
\pgfpathcurveto{\pgfqpoint{2.760848in}{1.241974in}}{\pgfqpoint{2.756458in}{1.252573in}}{\pgfqpoint{2.748645in}{1.260386in}}%
\pgfpathcurveto{\pgfqpoint{2.740831in}{1.268200in}}{\pgfqpoint{2.730232in}{1.272590in}}{\pgfqpoint{2.719182in}{1.272590in}}%
\pgfpathcurveto{\pgfqpoint{2.708132in}{1.272590in}}{\pgfqpoint{2.697533in}{1.268200in}}{\pgfqpoint{2.689719in}{1.260386in}}%
\pgfpathcurveto{\pgfqpoint{2.681905in}{1.252573in}}{\pgfqpoint{2.677515in}{1.241974in}}{\pgfqpoint{2.677515in}{1.230924in}}%
\pgfpathcurveto{\pgfqpoint{2.677515in}{1.219873in}}{\pgfqpoint{2.681905in}{1.209274in}}{\pgfqpoint{2.689719in}{1.201461in}}%
\pgfpathcurveto{\pgfqpoint{2.697533in}{1.193647in}}{\pgfqpoint{2.708132in}{1.189257in}}{\pgfqpoint{2.719182in}{1.189257in}}%
\pgfpathlineto{\pgfqpoint{2.719182in}{1.189257in}}%
\pgfpathclose%
\pgfusepath{stroke}%
\end{pgfscope}%
\begin{pgfscope}%
\pgfpathrectangle{\pgfqpoint{0.847223in}{0.554012in}}{\pgfqpoint{6.200000in}{4.620000in}}%
\pgfusepath{clip}%
\pgfsetbuttcap%
\pgfsetroundjoin%
\pgfsetlinewidth{1.003750pt}%
\definecolor{currentstroke}{rgb}{1.000000,0.000000,0.000000}%
\pgfsetstrokecolor{currentstroke}%
\pgfsetdash{}{0pt}%
\pgfpathmoveto{\pgfqpoint{2.724515in}{1.186808in}}%
\pgfpathcurveto{\pgfqpoint{2.735565in}{1.186808in}}{\pgfqpoint{2.746164in}{1.191198in}}{\pgfqpoint{2.753978in}{1.199012in}}%
\pgfpathcurveto{\pgfqpoint{2.761791in}{1.206825in}}{\pgfqpoint{2.766182in}{1.217424in}}{\pgfqpoint{2.766182in}{1.228474in}}%
\pgfpathcurveto{\pgfqpoint{2.766182in}{1.239525in}}{\pgfqpoint{2.761791in}{1.250124in}}{\pgfqpoint{2.753978in}{1.257937in}}%
\pgfpathcurveto{\pgfqpoint{2.746164in}{1.265751in}}{\pgfqpoint{2.735565in}{1.270141in}}{\pgfqpoint{2.724515in}{1.270141in}}%
\pgfpathcurveto{\pgfqpoint{2.713465in}{1.270141in}}{\pgfqpoint{2.702866in}{1.265751in}}{\pgfqpoint{2.695052in}{1.257937in}}%
\pgfpathcurveto{\pgfqpoint{2.687239in}{1.250124in}}{\pgfqpoint{2.682848in}{1.239525in}}{\pgfqpoint{2.682848in}{1.228474in}}%
\pgfpathcurveto{\pgfqpoint{2.682848in}{1.217424in}}{\pgfqpoint{2.687239in}{1.206825in}}{\pgfqpoint{2.695052in}{1.199012in}}%
\pgfpathcurveto{\pgfqpoint{2.702866in}{1.191198in}}{\pgfqpoint{2.713465in}{1.186808in}}{\pgfqpoint{2.724515in}{1.186808in}}%
\pgfpathlineto{\pgfqpoint{2.724515in}{1.186808in}}%
\pgfpathclose%
\pgfusepath{stroke}%
\end{pgfscope}%
\begin{pgfscope}%
\pgfpathrectangle{\pgfqpoint{0.847223in}{0.554012in}}{\pgfqpoint{6.200000in}{4.620000in}}%
\pgfusepath{clip}%
\pgfsetbuttcap%
\pgfsetroundjoin%
\pgfsetlinewidth{1.003750pt}%
\definecolor{currentstroke}{rgb}{1.000000,0.000000,0.000000}%
\pgfsetstrokecolor{currentstroke}%
\pgfsetdash{}{0pt}%
\pgfpathmoveto{\pgfqpoint{2.729848in}{1.184370in}}%
\pgfpathcurveto{\pgfqpoint{2.740898in}{1.184370in}}{\pgfqpoint{2.751497in}{1.188760in}}{\pgfqpoint{2.759311in}{1.196574in}}%
\pgfpathcurveto{\pgfqpoint{2.767125in}{1.204387in}}{\pgfqpoint{2.771515in}{1.214986in}}{\pgfqpoint{2.771515in}{1.226036in}}%
\pgfpathcurveto{\pgfqpoint{2.771515in}{1.237086in}}{\pgfqpoint{2.767125in}{1.247685in}}{\pgfqpoint{2.759311in}{1.255499in}}%
\pgfpathcurveto{\pgfqpoint{2.751497in}{1.263313in}}{\pgfqpoint{2.740898in}{1.267703in}}{\pgfqpoint{2.729848in}{1.267703in}}%
\pgfpathcurveto{\pgfqpoint{2.718798in}{1.267703in}}{\pgfqpoint{2.708199in}{1.263313in}}{\pgfqpoint{2.700385in}{1.255499in}}%
\pgfpathcurveto{\pgfqpoint{2.692572in}{1.247685in}}{\pgfqpoint{2.688182in}{1.237086in}}{\pgfqpoint{2.688182in}{1.226036in}}%
\pgfpathcurveto{\pgfqpoint{2.688182in}{1.214986in}}{\pgfqpoint{2.692572in}{1.204387in}}{\pgfqpoint{2.700385in}{1.196574in}}%
\pgfpathcurveto{\pgfqpoint{2.708199in}{1.188760in}}{\pgfqpoint{2.718798in}{1.184370in}}{\pgfqpoint{2.729848in}{1.184370in}}%
\pgfpathlineto{\pgfqpoint{2.729848in}{1.184370in}}%
\pgfpathclose%
\pgfusepath{stroke}%
\end{pgfscope}%
\begin{pgfscope}%
\pgfpathrectangle{\pgfqpoint{0.847223in}{0.554012in}}{\pgfqpoint{6.200000in}{4.620000in}}%
\pgfusepath{clip}%
\pgfsetbuttcap%
\pgfsetroundjoin%
\pgfsetlinewidth{1.003750pt}%
\definecolor{currentstroke}{rgb}{1.000000,0.000000,0.000000}%
\pgfsetstrokecolor{currentstroke}%
\pgfsetdash{}{0pt}%
\pgfpathmoveto{\pgfqpoint{2.735181in}{1.181942in}}%
\pgfpathcurveto{\pgfqpoint{2.746232in}{1.181942in}}{\pgfqpoint{2.756831in}{1.186333in}}{\pgfqpoint{2.764644in}{1.194146in}}%
\pgfpathcurveto{\pgfqpoint{2.772458in}{1.201960in}}{\pgfqpoint{2.776848in}{1.212559in}}{\pgfqpoint{2.776848in}{1.223609in}}%
\pgfpathcurveto{\pgfqpoint{2.776848in}{1.234659in}}{\pgfqpoint{2.772458in}{1.245258in}}{\pgfqpoint{2.764644in}{1.253072in}}%
\pgfpathcurveto{\pgfqpoint{2.756831in}{1.260885in}}{\pgfqpoint{2.746232in}{1.265276in}}{\pgfqpoint{2.735181in}{1.265276in}}%
\pgfpathcurveto{\pgfqpoint{2.724131in}{1.265276in}}{\pgfqpoint{2.713532in}{1.260885in}}{\pgfqpoint{2.705719in}{1.253072in}}%
\pgfpathcurveto{\pgfqpoint{2.697905in}{1.245258in}}{\pgfqpoint{2.693515in}{1.234659in}}{\pgfqpoint{2.693515in}{1.223609in}}%
\pgfpathcurveto{\pgfqpoint{2.693515in}{1.212559in}}{\pgfqpoint{2.697905in}{1.201960in}}{\pgfqpoint{2.705719in}{1.194146in}}%
\pgfpathcurveto{\pgfqpoint{2.713532in}{1.186333in}}{\pgfqpoint{2.724131in}{1.181942in}}{\pgfqpoint{2.735181in}{1.181942in}}%
\pgfpathlineto{\pgfqpoint{2.735181in}{1.181942in}}%
\pgfpathclose%
\pgfusepath{stroke}%
\end{pgfscope}%
\begin{pgfscope}%
\pgfpathrectangle{\pgfqpoint{0.847223in}{0.554012in}}{\pgfqpoint{6.200000in}{4.620000in}}%
\pgfusepath{clip}%
\pgfsetbuttcap%
\pgfsetroundjoin%
\pgfsetlinewidth{1.003750pt}%
\definecolor{currentstroke}{rgb}{1.000000,0.000000,0.000000}%
\pgfsetstrokecolor{currentstroke}%
\pgfsetdash{}{0pt}%
\pgfpathmoveto{\pgfqpoint{2.740515in}{1.179526in}}%
\pgfpathcurveto{\pgfqpoint{2.751565in}{1.179526in}}{\pgfqpoint{2.762164in}{1.183916in}}{\pgfqpoint{2.769977in}{1.191729in}}%
\pgfpathcurveto{\pgfqpoint{2.777791in}{1.199543in}}{\pgfqpoint{2.782181in}{1.210142in}}{\pgfqpoint{2.782181in}{1.221192in}}%
\pgfpathcurveto{\pgfqpoint{2.782181in}{1.232242in}}{\pgfqpoint{2.777791in}{1.242841in}}{\pgfqpoint{2.769977in}{1.250655in}}%
\pgfpathcurveto{\pgfqpoint{2.762164in}{1.258469in}}{\pgfqpoint{2.751565in}{1.262859in}}{\pgfqpoint{2.740515in}{1.262859in}}%
\pgfpathcurveto{\pgfqpoint{2.729464in}{1.262859in}}{\pgfqpoint{2.718865in}{1.258469in}}{\pgfqpoint{2.711052in}{1.250655in}}%
\pgfpathcurveto{\pgfqpoint{2.703238in}{1.242841in}}{\pgfqpoint{2.698848in}{1.232242in}}{\pgfqpoint{2.698848in}{1.221192in}}%
\pgfpathcurveto{\pgfqpoint{2.698848in}{1.210142in}}{\pgfqpoint{2.703238in}{1.199543in}}{\pgfqpoint{2.711052in}{1.191729in}}%
\pgfpathcurveto{\pgfqpoint{2.718865in}{1.183916in}}{\pgfqpoint{2.729464in}{1.179526in}}{\pgfqpoint{2.740515in}{1.179526in}}%
\pgfpathlineto{\pgfqpoint{2.740515in}{1.179526in}}%
\pgfpathclose%
\pgfusepath{stroke}%
\end{pgfscope}%
\begin{pgfscope}%
\pgfpathrectangle{\pgfqpoint{0.847223in}{0.554012in}}{\pgfqpoint{6.200000in}{4.620000in}}%
\pgfusepath{clip}%
\pgfsetbuttcap%
\pgfsetroundjoin%
\pgfsetlinewidth{1.003750pt}%
\definecolor{currentstroke}{rgb}{1.000000,0.000000,0.000000}%
\pgfsetstrokecolor{currentstroke}%
\pgfsetdash{}{0pt}%
\pgfpathmoveto{\pgfqpoint{2.745848in}{1.177120in}}%
\pgfpathcurveto{\pgfqpoint{2.756898in}{1.177120in}}{\pgfqpoint{2.767497in}{1.181510in}}{\pgfqpoint{2.775311in}{1.189323in}}%
\pgfpathcurveto{\pgfqpoint{2.783124in}{1.197137in}}{\pgfqpoint{2.787514in}{1.207736in}}{\pgfqpoint{2.787514in}{1.218786in}}%
\pgfpathcurveto{\pgfqpoint{2.787514in}{1.229836in}}{\pgfqpoint{2.783124in}{1.240435in}}{\pgfqpoint{2.775311in}{1.248249in}}%
\pgfpathcurveto{\pgfqpoint{2.767497in}{1.256063in}}{\pgfqpoint{2.756898in}{1.260453in}}{\pgfqpoint{2.745848in}{1.260453in}}%
\pgfpathcurveto{\pgfqpoint{2.734798in}{1.260453in}}{\pgfqpoint{2.724199in}{1.256063in}}{\pgfqpoint{2.716385in}{1.248249in}}%
\pgfpathcurveto{\pgfqpoint{2.708571in}{1.240435in}}{\pgfqpoint{2.704181in}{1.229836in}}{\pgfqpoint{2.704181in}{1.218786in}}%
\pgfpathcurveto{\pgfqpoint{2.704181in}{1.207736in}}{\pgfqpoint{2.708571in}{1.197137in}}{\pgfqpoint{2.716385in}{1.189323in}}%
\pgfpathcurveto{\pgfqpoint{2.724199in}{1.181510in}}{\pgfqpoint{2.734798in}{1.177120in}}{\pgfqpoint{2.745848in}{1.177120in}}%
\pgfpathlineto{\pgfqpoint{2.745848in}{1.177120in}}%
\pgfpathclose%
\pgfusepath{stroke}%
\end{pgfscope}%
\begin{pgfscope}%
\pgfpathrectangle{\pgfqpoint{0.847223in}{0.554012in}}{\pgfqpoint{6.200000in}{4.620000in}}%
\pgfusepath{clip}%
\pgfsetbuttcap%
\pgfsetroundjoin%
\pgfsetlinewidth{1.003750pt}%
\definecolor{currentstroke}{rgb}{1.000000,0.000000,0.000000}%
\pgfsetstrokecolor{currentstroke}%
\pgfsetdash{}{0pt}%
\pgfpathmoveto{\pgfqpoint{2.751181in}{1.174724in}}%
\pgfpathcurveto{\pgfqpoint{2.762231in}{1.174724in}}{\pgfqpoint{2.772830in}{1.179114in}}{\pgfqpoint{2.780644in}{1.186928in}}%
\pgfpathcurveto{\pgfqpoint{2.788457in}{1.194742in}}{\pgfqpoint{2.792848in}{1.205341in}}{\pgfqpoint{2.792848in}{1.216391in}}%
\pgfpathcurveto{\pgfqpoint{2.792848in}{1.227441in}}{\pgfqpoint{2.788457in}{1.238040in}}{\pgfqpoint{2.780644in}{1.245854in}}%
\pgfpathcurveto{\pgfqpoint{2.772830in}{1.253667in}}{\pgfqpoint{2.762231in}{1.258057in}}{\pgfqpoint{2.751181in}{1.258057in}}%
\pgfpathcurveto{\pgfqpoint{2.740131in}{1.258057in}}{\pgfqpoint{2.729532in}{1.253667in}}{\pgfqpoint{2.721718in}{1.245854in}}%
\pgfpathcurveto{\pgfqpoint{2.713905in}{1.238040in}}{\pgfqpoint{2.709514in}{1.227441in}}{\pgfqpoint{2.709514in}{1.216391in}}%
\pgfpathcurveto{\pgfqpoint{2.709514in}{1.205341in}}{\pgfqpoint{2.713905in}{1.194742in}}{\pgfqpoint{2.721718in}{1.186928in}}%
\pgfpathcurveto{\pgfqpoint{2.729532in}{1.179114in}}{\pgfqpoint{2.740131in}{1.174724in}}{\pgfqpoint{2.751181in}{1.174724in}}%
\pgfpathlineto{\pgfqpoint{2.751181in}{1.174724in}}%
\pgfpathclose%
\pgfusepath{stroke}%
\end{pgfscope}%
\begin{pgfscope}%
\pgfpathrectangle{\pgfqpoint{0.847223in}{0.554012in}}{\pgfqpoint{6.200000in}{4.620000in}}%
\pgfusepath{clip}%
\pgfsetbuttcap%
\pgfsetroundjoin%
\pgfsetlinewidth{1.003750pt}%
\definecolor{currentstroke}{rgb}{1.000000,0.000000,0.000000}%
\pgfsetstrokecolor{currentstroke}%
\pgfsetdash{}{0pt}%
\pgfpathmoveto{\pgfqpoint{2.756514in}{1.172339in}}%
\pgfpathcurveto{\pgfqpoint{2.767564in}{1.172339in}}{\pgfqpoint{2.778163in}{1.176729in}}{\pgfqpoint{2.785977in}{1.184543in}}%
\pgfpathcurveto{\pgfqpoint{2.793791in}{1.192357in}}{\pgfqpoint{2.798181in}{1.202956in}}{\pgfqpoint{2.798181in}{1.214006in}}%
\pgfpathcurveto{\pgfqpoint{2.798181in}{1.225056in}}{\pgfqpoint{2.793791in}{1.235655in}}{\pgfqpoint{2.785977in}{1.243469in}}%
\pgfpathcurveto{\pgfqpoint{2.778163in}{1.251282in}}{\pgfqpoint{2.767564in}{1.255672in}}{\pgfqpoint{2.756514in}{1.255672in}}%
\pgfpathcurveto{\pgfqpoint{2.745464in}{1.255672in}}{\pgfqpoint{2.734865in}{1.251282in}}{\pgfqpoint{2.727051in}{1.243469in}}%
\pgfpathcurveto{\pgfqpoint{2.719238in}{1.235655in}}{\pgfqpoint{2.714848in}{1.225056in}}{\pgfqpoint{2.714848in}{1.214006in}}%
\pgfpathcurveto{\pgfqpoint{2.714848in}{1.202956in}}{\pgfqpoint{2.719238in}{1.192357in}}{\pgfqpoint{2.727051in}{1.184543in}}%
\pgfpathcurveto{\pgfqpoint{2.734865in}{1.176729in}}{\pgfqpoint{2.745464in}{1.172339in}}{\pgfqpoint{2.756514in}{1.172339in}}%
\pgfpathlineto{\pgfqpoint{2.756514in}{1.172339in}}%
\pgfpathclose%
\pgfusepath{stroke}%
\end{pgfscope}%
\begin{pgfscope}%
\pgfpathrectangle{\pgfqpoint{0.847223in}{0.554012in}}{\pgfqpoint{6.200000in}{4.620000in}}%
\pgfusepath{clip}%
\pgfsetbuttcap%
\pgfsetroundjoin%
\pgfsetlinewidth{1.003750pt}%
\definecolor{currentstroke}{rgb}{1.000000,0.000000,0.000000}%
\pgfsetstrokecolor{currentstroke}%
\pgfsetdash{}{0pt}%
\pgfpathmoveto{\pgfqpoint{2.761847in}{1.169965in}}%
\pgfpathcurveto{\pgfqpoint{2.772898in}{1.169965in}}{\pgfqpoint{2.783497in}{1.174355in}}{\pgfqpoint{2.791310in}{1.182168in}}%
\pgfpathcurveto{\pgfqpoint{2.799124in}{1.189982in}}{\pgfqpoint{2.803514in}{1.200581in}}{\pgfqpoint{2.803514in}{1.211631in}}%
\pgfpathcurveto{\pgfqpoint{2.803514in}{1.222681in}}{\pgfqpoint{2.799124in}{1.233280in}}{\pgfqpoint{2.791310in}{1.241094in}}%
\pgfpathcurveto{\pgfqpoint{2.783497in}{1.248908in}}{\pgfqpoint{2.772898in}{1.253298in}}{\pgfqpoint{2.761847in}{1.253298in}}%
\pgfpathcurveto{\pgfqpoint{2.750797in}{1.253298in}}{\pgfqpoint{2.740198in}{1.248908in}}{\pgfqpoint{2.732385in}{1.241094in}}%
\pgfpathcurveto{\pgfqpoint{2.724571in}{1.233280in}}{\pgfqpoint{2.720181in}{1.222681in}}{\pgfqpoint{2.720181in}{1.211631in}}%
\pgfpathcurveto{\pgfqpoint{2.720181in}{1.200581in}}{\pgfqpoint{2.724571in}{1.189982in}}{\pgfqpoint{2.732385in}{1.182168in}}%
\pgfpathcurveto{\pgfqpoint{2.740198in}{1.174355in}}{\pgfqpoint{2.750797in}{1.169965in}}{\pgfqpoint{2.761847in}{1.169965in}}%
\pgfpathlineto{\pgfqpoint{2.761847in}{1.169965in}}%
\pgfpathclose%
\pgfusepath{stroke}%
\end{pgfscope}%
\begin{pgfscope}%
\pgfpathrectangle{\pgfqpoint{0.847223in}{0.554012in}}{\pgfqpoint{6.200000in}{4.620000in}}%
\pgfusepath{clip}%
\pgfsetbuttcap%
\pgfsetroundjoin%
\pgfsetlinewidth{1.003750pt}%
\definecolor{currentstroke}{rgb}{1.000000,0.000000,0.000000}%
\pgfsetstrokecolor{currentstroke}%
\pgfsetdash{}{0pt}%
\pgfpathmoveto{\pgfqpoint{2.767181in}{1.167600in}}%
\pgfpathcurveto{\pgfqpoint{2.778231in}{1.167600in}}{\pgfqpoint{2.788830in}{1.171991in}}{\pgfqpoint{2.796643in}{1.179804in}}%
\pgfpathcurveto{\pgfqpoint{2.804457in}{1.187618in}}{\pgfqpoint{2.808847in}{1.198217in}}{\pgfqpoint{2.808847in}{1.209267in}}%
\pgfpathcurveto{\pgfqpoint{2.808847in}{1.220317in}}{\pgfqpoint{2.804457in}{1.230916in}}{\pgfqpoint{2.796643in}{1.238730in}}%
\pgfpathcurveto{\pgfqpoint{2.788830in}{1.246544in}}{\pgfqpoint{2.778231in}{1.250934in}}{\pgfqpoint{2.767181in}{1.250934in}}%
\pgfpathcurveto{\pgfqpoint{2.756131in}{1.250934in}}{\pgfqpoint{2.745532in}{1.246544in}}{\pgfqpoint{2.737718in}{1.238730in}}%
\pgfpathcurveto{\pgfqpoint{2.729904in}{1.230916in}}{\pgfqpoint{2.725514in}{1.220317in}}{\pgfqpoint{2.725514in}{1.209267in}}%
\pgfpathcurveto{\pgfqpoint{2.725514in}{1.198217in}}{\pgfqpoint{2.729904in}{1.187618in}}{\pgfqpoint{2.737718in}{1.179804in}}%
\pgfpathcurveto{\pgfqpoint{2.745532in}{1.171991in}}{\pgfqpoint{2.756131in}{1.167600in}}{\pgfqpoint{2.767181in}{1.167600in}}%
\pgfpathlineto{\pgfqpoint{2.767181in}{1.167600in}}%
\pgfpathclose%
\pgfusepath{stroke}%
\end{pgfscope}%
\begin{pgfscope}%
\pgfpathrectangle{\pgfqpoint{0.847223in}{0.554012in}}{\pgfqpoint{6.200000in}{4.620000in}}%
\pgfusepath{clip}%
\pgfsetbuttcap%
\pgfsetroundjoin%
\pgfsetlinewidth{1.003750pt}%
\definecolor{currentstroke}{rgb}{1.000000,0.000000,0.000000}%
\pgfsetstrokecolor{currentstroke}%
\pgfsetdash{}{0pt}%
\pgfpathmoveto{\pgfqpoint{2.772514in}{1.165247in}}%
\pgfpathcurveto{\pgfqpoint{2.783564in}{1.165247in}}{\pgfqpoint{2.794163in}{1.169637in}}{\pgfqpoint{2.801977in}{1.177450in}}%
\pgfpathcurveto{\pgfqpoint{2.809790in}{1.185264in}}{\pgfqpoint{2.814181in}{1.195863in}}{\pgfqpoint{2.814181in}{1.206913in}}%
\pgfpathcurveto{\pgfqpoint{2.814181in}{1.217963in}}{\pgfqpoint{2.809790in}{1.228562in}}{\pgfqpoint{2.801977in}{1.236376in}}%
\pgfpathcurveto{\pgfqpoint{2.794163in}{1.244190in}}{\pgfqpoint{2.783564in}{1.248580in}}{\pgfqpoint{2.772514in}{1.248580in}}%
\pgfpathcurveto{\pgfqpoint{2.761464in}{1.248580in}}{\pgfqpoint{2.750865in}{1.244190in}}{\pgfqpoint{2.743051in}{1.236376in}}%
\pgfpathcurveto{\pgfqpoint{2.735238in}{1.228562in}}{\pgfqpoint{2.730847in}{1.217963in}}{\pgfqpoint{2.730847in}{1.206913in}}%
\pgfpathcurveto{\pgfqpoint{2.730847in}{1.195863in}}{\pgfqpoint{2.735238in}{1.185264in}}{\pgfqpoint{2.743051in}{1.177450in}}%
\pgfpathcurveto{\pgfqpoint{2.750865in}{1.169637in}}{\pgfqpoint{2.761464in}{1.165247in}}{\pgfqpoint{2.772514in}{1.165247in}}%
\pgfpathlineto{\pgfqpoint{2.772514in}{1.165247in}}%
\pgfpathclose%
\pgfusepath{stroke}%
\end{pgfscope}%
\begin{pgfscope}%
\pgfpathrectangle{\pgfqpoint{0.847223in}{0.554012in}}{\pgfqpoint{6.200000in}{4.620000in}}%
\pgfusepath{clip}%
\pgfsetbuttcap%
\pgfsetroundjoin%
\pgfsetlinewidth{1.003750pt}%
\definecolor{currentstroke}{rgb}{1.000000,0.000000,0.000000}%
\pgfsetstrokecolor{currentstroke}%
\pgfsetdash{}{0pt}%
\pgfpathmoveto{\pgfqpoint{2.777847in}{1.162903in}}%
\pgfpathcurveto{\pgfqpoint{2.788897in}{1.162903in}}{\pgfqpoint{2.799496in}{1.167293in}}{\pgfqpoint{2.807310in}{1.175107in}}%
\pgfpathcurveto{\pgfqpoint{2.815124in}{1.182920in}}{\pgfqpoint{2.819514in}{1.193520in}}{\pgfqpoint{2.819514in}{1.204570in}}%
\pgfpathcurveto{\pgfqpoint{2.819514in}{1.215620in}}{\pgfqpoint{2.815124in}{1.226219in}}{\pgfqpoint{2.807310in}{1.234032in}}%
\pgfpathcurveto{\pgfqpoint{2.799496in}{1.241846in}}{\pgfqpoint{2.788897in}{1.246236in}}{\pgfqpoint{2.777847in}{1.246236in}}%
\pgfpathcurveto{\pgfqpoint{2.766797in}{1.246236in}}{\pgfqpoint{2.756198in}{1.241846in}}{\pgfqpoint{2.748384in}{1.234032in}}%
\pgfpathcurveto{\pgfqpoint{2.740571in}{1.226219in}}{\pgfqpoint{2.736180in}{1.215620in}}{\pgfqpoint{2.736180in}{1.204570in}}%
\pgfpathcurveto{\pgfqpoint{2.736180in}{1.193520in}}{\pgfqpoint{2.740571in}{1.182920in}}{\pgfqpoint{2.748384in}{1.175107in}}%
\pgfpathcurveto{\pgfqpoint{2.756198in}{1.167293in}}{\pgfqpoint{2.766797in}{1.162903in}}{\pgfqpoint{2.777847in}{1.162903in}}%
\pgfpathlineto{\pgfqpoint{2.777847in}{1.162903in}}%
\pgfpathclose%
\pgfusepath{stroke}%
\end{pgfscope}%
\begin{pgfscope}%
\pgfpathrectangle{\pgfqpoint{0.847223in}{0.554012in}}{\pgfqpoint{6.200000in}{4.620000in}}%
\pgfusepath{clip}%
\pgfsetbuttcap%
\pgfsetroundjoin%
\pgfsetlinewidth{1.003750pt}%
\definecolor{currentstroke}{rgb}{1.000000,0.000000,0.000000}%
\pgfsetstrokecolor{currentstroke}%
\pgfsetdash{}{0pt}%
\pgfpathmoveto{\pgfqpoint{2.783180in}{1.160570in}}%
\pgfpathcurveto{\pgfqpoint{2.794230in}{1.160570in}}{\pgfqpoint{2.804830in}{1.164960in}}{\pgfqpoint{2.812643in}{1.172773in}}%
\pgfpathcurveto{\pgfqpoint{2.820457in}{1.180587in}}{\pgfqpoint{2.824847in}{1.191186in}}{\pgfqpoint{2.824847in}{1.202236in}}%
\pgfpathcurveto{\pgfqpoint{2.824847in}{1.213286in}}{\pgfqpoint{2.820457in}{1.223885in}}{\pgfqpoint{2.812643in}{1.231699in}}%
\pgfpathcurveto{\pgfqpoint{2.804830in}{1.239513in}}{\pgfqpoint{2.794230in}{1.243903in}}{\pgfqpoint{2.783180in}{1.243903in}}%
\pgfpathcurveto{\pgfqpoint{2.772130in}{1.243903in}}{\pgfqpoint{2.761531in}{1.239513in}}{\pgfqpoint{2.753718in}{1.231699in}}%
\pgfpathcurveto{\pgfqpoint{2.745904in}{1.223885in}}{\pgfqpoint{2.741514in}{1.213286in}}{\pgfqpoint{2.741514in}{1.202236in}}%
\pgfpathcurveto{\pgfqpoint{2.741514in}{1.191186in}}{\pgfqpoint{2.745904in}{1.180587in}}{\pgfqpoint{2.753718in}{1.172773in}}%
\pgfpathcurveto{\pgfqpoint{2.761531in}{1.164960in}}{\pgfqpoint{2.772130in}{1.160570in}}{\pgfqpoint{2.783180in}{1.160570in}}%
\pgfpathlineto{\pgfqpoint{2.783180in}{1.160570in}}%
\pgfpathclose%
\pgfusepath{stroke}%
\end{pgfscope}%
\begin{pgfscope}%
\pgfpathrectangle{\pgfqpoint{0.847223in}{0.554012in}}{\pgfqpoint{6.200000in}{4.620000in}}%
\pgfusepath{clip}%
\pgfsetbuttcap%
\pgfsetroundjoin%
\pgfsetlinewidth{1.003750pt}%
\definecolor{currentstroke}{rgb}{1.000000,0.000000,0.000000}%
\pgfsetstrokecolor{currentstroke}%
\pgfsetdash{}{0pt}%
\pgfpathmoveto{\pgfqpoint{2.788514in}{1.158246in}}%
\pgfpathcurveto{\pgfqpoint{2.799564in}{1.158246in}}{\pgfqpoint{2.810163in}{1.162636in}}{\pgfqpoint{2.817976in}{1.170450in}}%
\pgfpathcurveto{\pgfqpoint{2.825790in}{1.178264in}}{\pgfqpoint{2.830180in}{1.188863in}}{\pgfqpoint{2.830180in}{1.199913in}}%
\pgfpathcurveto{\pgfqpoint{2.830180in}{1.210963in}}{\pgfqpoint{2.825790in}{1.221562in}}{\pgfqpoint{2.817976in}{1.229376in}}%
\pgfpathcurveto{\pgfqpoint{2.810163in}{1.237189in}}{\pgfqpoint{2.799564in}{1.241579in}}{\pgfqpoint{2.788514in}{1.241579in}}%
\pgfpathcurveto{\pgfqpoint{2.777463in}{1.241579in}}{\pgfqpoint{2.766864in}{1.237189in}}{\pgfqpoint{2.759051in}{1.229376in}}%
\pgfpathcurveto{\pgfqpoint{2.751237in}{1.221562in}}{\pgfqpoint{2.746847in}{1.210963in}}{\pgfqpoint{2.746847in}{1.199913in}}%
\pgfpathcurveto{\pgfqpoint{2.746847in}{1.188863in}}{\pgfqpoint{2.751237in}{1.178264in}}{\pgfqpoint{2.759051in}{1.170450in}}%
\pgfpathcurveto{\pgfqpoint{2.766864in}{1.162636in}}{\pgfqpoint{2.777463in}{1.158246in}}{\pgfqpoint{2.788514in}{1.158246in}}%
\pgfpathlineto{\pgfqpoint{2.788514in}{1.158246in}}%
\pgfpathclose%
\pgfusepath{stroke}%
\end{pgfscope}%
\begin{pgfscope}%
\pgfpathrectangle{\pgfqpoint{0.847223in}{0.554012in}}{\pgfqpoint{6.200000in}{4.620000in}}%
\pgfusepath{clip}%
\pgfsetbuttcap%
\pgfsetroundjoin%
\pgfsetlinewidth{1.003750pt}%
\definecolor{currentstroke}{rgb}{1.000000,0.000000,0.000000}%
\pgfsetstrokecolor{currentstroke}%
\pgfsetdash{}{0pt}%
\pgfpathmoveto{\pgfqpoint{2.793847in}{1.155933in}}%
\pgfpathcurveto{\pgfqpoint{2.804897in}{1.155933in}}{\pgfqpoint{2.815496in}{1.160323in}}{\pgfqpoint{2.823310in}{1.168137in}}%
\pgfpathcurveto{\pgfqpoint{2.831123in}{1.175950in}}{\pgfqpoint{2.835513in}{1.186549in}}{\pgfqpoint{2.835513in}{1.197599in}}%
\pgfpathcurveto{\pgfqpoint{2.835513in}{1.208650in}}{\pgfqpoint{2.831123in}{1.219249in}}{\pgfqpoint{2.823310in}{1.227062in}}%
\pgfpathcurveto{\pgfqpoint{2.815496in}{1.234876in}}{\pgfqpoint{2.804897in}{1.239266in}}{\pgfqpoint{2.793847in}{1.239266in}}%
\pgfpathcurveto{\pgfqpoint{2.782797in}{1.239266in}}{\pgfqpoint{2.772198in}{1.234876in}}{\pgfqpoint{2.764384in}{1.227062in}}%
\pgfpathcurveto{\pgfqpoint{2.756570in}{1.219249in}}{\pgfqpoint{2.752180in}{1.208650in}}{\pgfqpoint{2.752180in}{1.197599in}}%
\pgfpathcurveto{\pgfqpoint{2.752180in}{1.186549in}}{\pgfqpoint{2.756570in}{1.175950in}}{\pgfqpoint{2.764384in}{1.168137in}}%
\pgfpathcurveto{\pgfqpoint{2.772198in}{1.160323in}}{\pgfqpoint{2.782797in}{1.155933in}}{\pgfqpoint{2.793847in}{1.155933in}}%
\pgfpathlineto{\pgfqpoint{2.793847in}{1.155933in}}%
\pgfpathclose%
\pgfusepath{stroke}%
\end{pgfscope}%
\begin{pgfscope}%
\pgfpathrectangle{\pgfqpoint{0.847223in}{0.554012in}}{\pgfqpoint{6.200000in}{4.620000in}}%
\pgfusepath{clip}%
\pgfsetbuttcap%
\pgfsetroundjoin%
\pgfsetlinewidth{1.003750pt}%
\definecolor{currentstroke}{rgb}{1.000000,0.000000,0.000000}%
\pgfsetstrokecolor{currentstroke}%
\pgfsetdash{}{0pt}%
\pgfpathmoveto{\pgfqpoint{2.799180in}{1.153629in}}%
\pgfpathcurveto{\pgfqpoint{2.810230in}{1.153629in}}{\pgfqpoint{2.820829in}{1.158020in}}{\pgfqpoint{2.828643in}{1.165833in}}%
\pgfpathcurveto{\pgfqpoint{2.836456in}{1.173647in}}{\pgfqpoint{2.840847in}{1.184246in}}{\pgfqpoint{2.840847in}{1.195296in}}%
\pgfpathcurveto{\pgfqpoint{2.840847in}{1.206346in}}{\pgfqpoint{2.836456in}{1.216945in}}{\pgfqpoint{2.828643in}{1.224759in}}%
\pgfpathcurveto{\pgfqpoint{2.820829in}{1.232573in}}{\pgfqpoint{2.810230in}{1.236963in}}{\pgfqpoint{2.799180in}{1.236963in}}%
\pgfpathcurveto{\pgfqpoint{2.788130in}{1.236963in}}{\pgfqpoint{2.777531in}{1.232573in}}{\pgfqpoint{2.769717in}{1.224759in}}%
\pgfpathcurveto{\pgfqpoint{2.761904in}{1.216945in}}{\pgfqpoint{2.757513in}{1.206346in}}{\pgfqpoint{2.757513in}{1.195296in}}%
\pgfpathcurveto{\pgfqpoint{2.757513in}{1.184246in}}{\pgfqpoint{2.761904in}{1.173647in}}{\pgfqpoint{2.769717in}{1.165833in}}%
\pgfpathcurveto{\pgfqpoint{2.777531in}{1.158020in}}{\pgfqpoint{2.788130in}{1.153629in}}{\pgfqpoint{2.799180in}{1.153629in}}%
\pgfpathlineto{\pgfqpoint{2.799180in}{1.153629in}}%
\pgfpathclose%
\pgfusepath{stroke}%
\end{pgfscope}%
\begin{pgfscope}%
\pgfpathrectangle{\pgfqpoint{0.847223in}{0.554012in}}{\pgfqpoint{6.200000in}{4.620000in}}%
\pgfusepath{clip}%
\pgfsetbuttcap%
\pgfsetroundjoin%
\pgfsetlinewidth{1.003750pt}%
\definecolor{currentstroke}{rgb}{1.000000,0.000000,0.000000}%
\pgfsetstrokecolor{currentstroke}%
\pgfsetdash{}{0pt}%
\pgfpathmoveto{\pgfqpoint{2.804513in}{1.151336in}}%
\pgfpathcurveto{\pgfqpoint{2.815563in}{1.151336in}}{\pgfqpoint{2.826162in}{1.155726in}}{\pgfqpoint{2.833976in}{1.163540in}}%
\pgfpathcurveto{\pgfqpoint{2.841790in}{1.171354in}}{\pgfqpoint{2.846180in}{1.181953in}}{\pgfqpoint{2.846180in}{1.193003in}}%
\pgfpathcurveto{\pgfqpoint{2.846180in}{1.204053in}}{\pgfqpoint{2.841790in}{1.214652in}}{\pgfqpoint{2.833976in}{1.222465in}}%
\pgfpathcurveto{\pgfqpoint{2.826162in}{1.230279in}}{\pgfqpoint{2.815563in}{1.234669in}}{\pgfqpoint{2.804513in}{1.234669in}}%
\pgfpathcurveto{\pgfqpoint{2.793463in}{1.234669in}}{\pgfqpoint{2.782864in}{1.230279in}}{\pgfqpoint{2.775050in}{1.222465in}}%
\pgfpathcurveto{\pgfqpoint{2.767237in}{1.214652in}}{\pgfqpoint{2.762847in}{1.204053in}}{\pgfqpoint{2.762847in}{1.193003in}}%
\pgfpathcurveto{\pgfqpoint{2.762847in}{1.181953in}}{\pgfqpoint{2.767237in}{1.171354in}}{\pgfqpoint{2.775050in}{1.163540in}}%
\pgfpathcurveto{\pgfqpoint{2.782864in}{1.155726in}}{\pgfqpoint{2.793463in}{1.151336in}}{\pgfqpoint{2.804513in}{1.151336in}}%
\pgfpathlineto{\pgfqpoint{2.804513in}{1.151336in}}%
\pgfpathclose%
\pgfusepath{stroke}%
\end{pgfscope}%
\begin{pgfscope}%
\pgfpathrectangle{\pgfqpoint{0.847223in}{0.554012in}}{\pgfqpoint{6.200000in}{4.620000in}}%
\pgfusepath{clip}%
\pgfsetbuttcap%
\pgfsetroundjoin%
\pgfsetlinewidth{1.003750pt}%
\definecolor{currentstroke}{rgb}{1.000000,0.000000,0.000000}%
\pgfsetstrokecolor{currentstroke}%
\pgfsetdash{}{0pt}%
\pgfpathmoveto{\pgfqpoint{2.809846in}{1.149052in}}%
\pgfpathcurveto{\pgfqpoint{2.820897in}{1.149052in}}{\pgfqpoint{2.831496in}{1.153443in}}{\pgfqpoint{2.839309in}{1.161256in}}%
\pgfpathcurveto{\pgfqpoint{2.847123in}{1.169070in}}{\pgfqpoint{2.851513in}{1.179669in}}{\pgfqpoint{2.851513in}{1.190719in}}%
\pgfpathcurveto{\pgfqpoint{2.851513in}{1.201769in}}{\pgfqpoint{2.847123in}{1.212368in}}{\pgfqpoint{2.839309in}{1.220182in}}%
\pgfpathcurveto{\pgfqpoint{2.831496in}{1.227995in}}{\pgfqpoint{2.820897in}{1.232386in}}{\pgfqpoint{2.809846in}{1.232386in}}%
\pgfpathcurveto{\pgfqpoint{2.798796in}{1.232386in}}{\pgfqpoint{2.788197in}{1.227995in}}{\pgfqpoint{2.780384in}{1.220182in}}%
\pgfpathcurveto{\pgfqpoint{2.772570in}{1.212368in}}{\pgfqpoint{2.768180in}{1.201769in}}{\pgfqpoint{2.768180in}{1.190719in}}%
\pgfpathcurveto{\pgfqpoint{2.768180in}{1.179669in}}{\pgfqpoint{2.772570in}{1.169070in}}{\pgfqpoint{2.780384in}{1.161256in}}%
\pgfpathcurveto{\pgfqpoint{2.788197in}{1.153443in}}{\pgfqpoint{2.798796in}{1.149052in}}{\pgfqpoint{2.809846in}{1.149052in}}%
\pgfpathlineto{\pgfqpoint{2.809846in}{1.149052in}}%
\pgfpathclose%
\pgfusepath{stroke}%
\end{pgfscope}%
\begin{pgfscope}%
\pgfpathrectangle{\pgfqpoint{0.847223in}{0.554012in}}{\pgfqpoint{6.200000in}{4.620000in}}%
\pgfusepath{clip}%
\pgfsetbuttcap%
\pgfsetroundjoin%
\pgfsetlinewidth{1.003750pt}%
\definecolor{currentstroke}{rgb}{1.000000,0.000000,0.000000}%
\pgfsetstrokecolor{currentstroke}%
\pgfsetdash{}{0pt}%
\pgfpathmoveto{\pgfqpoint{2.815180in}{1.146779in}}%
\pgfpathcurveto{\pgfqpoint{2.826230in}{1.146779in}}{\pgfqpoint{2.836829in}{1.151169in}}{\pgfqpoint{2.844642in}{1.158982in}}%
\pgfpathcurveto{\pgfqpoint{2.852456in}{1.166796in}}{\pgfqpoint{2.856846in}{1.177395in}}{\pgfqpoint{2.856846in}{1.188445in}}%
\pgfpathcurveto{\pgfqpoint{2.856846in}{1.199495in}}{\pgfqpoint{2.852456in}{1.210094in}}{\pgfqpoint{2.844642in}{1.217908in}}%
\pgfpathcurveto{\pgfqpoint{2.836829in}{1.225722in}}{\pgfqpoint{2.826230in}{1.230112in}}{\pgfqpoint{2.815180in}{1.230112in}}%
\pgfpathcurveto{\pgfqpoint{2.804130in}{1.230112in}}{\pgfqpoint{2.793530in}{1.225722in}}{\pgfqpoint{2.785717in}{1.217908in}}%
\pgfpathcurveto{\pgfqpoint{2.777903in}{1.210094in}}{\pgfqpoint{2.773513in}{1.199495in}}{\pgfqpoint{2.773513in}{1.188445in}}%
\pgfpathcurveto{\pgfqpoint{2.773513in}{1.177395in}}{\pgfqpoint{2.777903in}{1.166796in}}{\pgfqpoint{2.785717in}{1.158982in}}%
\pgfpathcurveto{\pgfqpoint{2.793530in}{1.151169in}}{\pgfqpoint{2.804130in}{1.146779in}}{\pgfqpoint{2.815180in}{1.146779in}}%
\pgfpathlineto{\pgfqpoint{2.815180in}{1.146779in}}%
\pgfpathclose%
\pgfusepath{stroke}%
\end{pgfscope}%
\begin{pgfscope}%
\pgfpathrectangle{\pgfqpoint{0.847223in}{0.554012in}}{\pgfqpoint{6.200000in}{4.620000in}}%
\pgfusepath{clip}%
\pgfsetbuttcap%
\pgfsetroundjoin%
\pgfsetlinewidth{1.003750pt}%
\definecolor{currentstroke}{rgb}{1.000000,0.000000,0.000000}%
\pgfsetstrokecolor{currentstroke}%
\pgfsetdash{}{0pt}%
\pgfpathmoveto{\pgfqpoint{2.820513in}{1.144514in}}%
\pgfpathcurveto{\pgfqpoint{2.831563in}{1.144514in}}{\pgfqpoint{2.842162in}{1.148905in}}{\pgfqpoint{2.849976in}{1.156718in}}%
\pgfpathcurveto{\pgfqpoint{2.857789in}{1.164532in}}{\pgfqpoint{2.862180in}{1.175131in}}{\pgfqpoint{2.862180in}{1.186181in}}%
\pgfpathcurveto{\pgfqpoint{2.862180in}{1.197231in}}{\pgfqpoint{2.857789in}{1.207830in}}{\pgfqpoint{2.849976in}{1.215644in}}%
\pgfpathcurveto{\pgfqpoint{2.842162in}{1.223458in}}{\pgfqpoint{2.831563in}{1.227848in}}{\pgfqpoint{2.820513in}{1.227848in}}%
\pgfpathcurveto{\pgfqpoint{2.809463in}{1.227848in}}{\pgfqpoint{2.798864in}{1.223458in}}{\pgfqpoint{2.791050in}{1.215644in}}%
\pgfpathcurveto{\pgfqpoint{2.783236in}{1.207830in}}{\pgfqpoint{2.778846in}{1.197231in}}{\pgfqpoint{2.778846in}{1.186181in}}%
\pgfpathcurveto{\pgfqpoint{2.778846in}{1.175131in}}{\pgfqpoint{2.783236in}{1.164532in}}{\pgfqpoint{2.791050in}{1.156718in}}%
\pgfpathcurveto{\pgfqpoint{2.798864in}{1.148905in}}{\pgfqpoint{2.809463in}{1.144514in}}{\pgfqpoint{2.820513in}{1.144514in}}%
\pgfpathlineto{\pgfqpoint{2.820513in}{1.144514in}}%
\pgfpathclose%
\pgfusepath{stroke}%
\end{pgfscope}%
\begin{pgfscope}%
\pgfpathrectangle{\pgfqpoint{0.847223in}{0.554012in}}{\pgfqpoint{6.200000in}{4.620000in}}%
\pgfusepath{clip}%
\pgfsetbuttcap%
\pgfsetroundjoin%
\pgfsetlinewidth{1.003750pt}%
\definecolor{currentstroke}{rgb}{1.000000,0.000000,0.000000}%
\pgfsetstrokecolor{currentstroke}%
\pgfsetdash{}{0pt}%
\pgfpathmoveto{\pgfqpoint{2.825846in}{1.142260in}}%
\pgfpathcurveto{\pgfqpoint{2.836896in}{1.142260in}}{\pgfqpoint{2.847495in}{1.146650in}}{\pgfqpoint{2.855309in}{1.154464in}}%
\pgfpathcurveto{\pgfqpoint{2.863122in}{1.162277in}}{\pgfqpoint{2.867513in}{1.172876in}}{\pgfqpoint{2.867513in}{1.183927in}}%
\pgfpathcurveto{\pgfqpoint{2.867513in}{1.194977in}}{\pgfqpoint{2.863122in}{1.205576in}}{\pgfqpoint{2.855309in}{1.213389in}}%
\pgfpathcurveto{\pgfqpoint{2.847495in}{1.221203in}}{\pgfqpoint{2.836896in}{1.225593in}}{\pgfqpoint{2.825846in}{1.225593in}}%
\pgfpathcurveto{\pgfqpoint{2.814796in}{1.225593in}}{\pgfqpoint{2.804197in}{1.221203in}}{\pgfqpoint{2.796383in}{1.213389in}}%
\pgfpathcurveto{\pgfqpoint{2.788570in}{1.205576in}}{\pgfqpoint{2.784179in}{1.194977in}}{\pgfqpoint{2.784179in}{1.183927in}}%
\pgfpathcurveto{\pgfqpoint{2.784179in}{1.172876in}}{\pgfqpoint{2.788570in}{1.162277in}}{\pgfqpoint{2.796383in}{1.154464in}}%
\pgfpathcurveto{\pgfqpoint{2.804197in}{1.146650in}}{\pgfqpoint{2.814796in}{1.142260in}}{\pgfqpoint{2.825846in}{1.142260in}}%
\pgfpathlineto{\pgfqpoint{2.825846in}{1.142260in}}%
\pgfpathclose%
\pgfusepath{stroke}%
\end{pgfscope}%
\begin{pgfscope}%
\pgfpathrectangle{\pgfqpoint{0.847223in}{0.554012in}}{\pgfqpoint{6.200000in}{4.620000in}}%
\pgfusepath{clip}%
\pgfsetbuttcap%
\pgfsetroundjoin%
\pgfsetlinewidth{1.003750pt}%
\definecolor{currentstroke}{rgb}{1.000000,0.000000,0.000000}%
\pgfsetstrokecolor{currentstroke}%
\pgfsetdash{}{0pt}%
\pgfpathmoveto{\pgfqpoint{2.831179in}{1.140015in}}%
\pgfpathcurveto{\pgfqpoint{2.842229in}{1.140015in}}{\pgfqpoint{2.852828in}{1.144405in}}{\pgfqpoint{2.860642in}{1.152219in}}%
\pgfpathcurveto{\pgfqpoint{2.868456in}{1.160033in}}{\pgfqpoint{2.872846in}{1.170632in}}{\pgfqpoint{2.872846in}{1.181682in}}%
\pgfpathcurveto{\pgfqpoint{2.872846in}{1.192732in}}{\pgfqpoint{2.868456in}{1.203331in}}{\pgfqpoint{2.860642in}{1.211145in}}%
\pgfpathcurveto{\pgfqpoint{2.852828in}{1.218958in}}{\pgfqpoint{2.842229in}{1.223348in}}{\pgfqpoint{2.831179in}{1.223348in}}%
\pgfpathcurveto{\pgfqpoint{2.820129in}{1.223348in}}{\pgfqpoint{2.809530in}{1.218958in}}{\pgfqpoint{2.801716in}{1.211145in}}%
\pgfpathcurveto{\pgfqpoint{2.793903in}{1.203331in}}{\pgfqpoint{2.789513in}{1.192732in}}{\pgfqpoint{2.789513in}{1.181682in}}%
\pgfpathcurveto{\pgfqpoint{2.789513in}{1.170632in}}{\pgfqpoint{2.793903in}{1.160033in}}{\pgfqpoint{2.801716in}{1.152219in}}%
\pgfpathcurveto{\pgfqpoint{2.809530in}{1.144405in}}{\pgfqpoint{2.820129in}{1.140015in}}{\pgfqpoint{2.831179in}{1.140015in}}%
\pgfpathlineto{\pgfqpoint{2.831179in}{1.140015in}}%
\pgfpathclose%
\pgfusepath{stroke}%
\end{pgfscope}%
\begin{pgfscope}%
\pgfpathrectangle{\pgfqpoint{0.847223in}{0.554012in}}{\pgfqpoint{6.200000in}{4.620000in}}%
\pgfusepath{clip}%
\pgfsetbuttcap%
\pgfsetroundjoin%
\pgfsetlinewidth{1.003750pt}%
\definecolor{currentstroke}{rgb}{1.000000,0.000000,0.000000}%
\pgfsetstrokecolor{currentstroke}%
\pgfsetdash{}{0pt}%
\pgfpathmoveto{\pgfqpoint{2.836512in}{1.137780in}}%
\pgfpathcurveto{\pgfqpoint{2.847563in}{1.137780in}}{\pgfqpoint{2.858162in}{1.142170in}}{\pgfqpoint{2.865975in}{1.149984in}}%
\pgfpathcurveto{\pgfqpoint{2.873789in}{1.157797in}}{\pgfqpoint{2.878179in}{1.168396in}}{\pgfqpoint{2.878179in}{1.179446in}}%
\pgfpathcurveto{\pgfqpoint{2.878179in}{1.190497in}}{\pgfqpoint{2.873789in}{1.201096in}}{\pgfqpoint{2.865975in}{1.208909in}}%
\pgfpathcurveto{\pgfqpoint{2.858162in}{1.216723in}}{\pgfqpoint{2.847563in}{1.221113in}}{\pgfqpoint{2.836512in}{1.221113in}}%
\pgfpathcurveto{\pgfqpoint{2.825462in}{1.221113in}}{\pgfqpoint{2.814863in}{1.216723in}}{\pgfqpoint{2.807050in}{1.208909in}}%
\pgfpathcurveto{\pgfqpoint{2.799236in}{1.201096in}}{\pgfqpoint{2.794846in}{1.190497in}}{\pgfqpoint{2.794846in}{1.179446in}}%
\pgfpathcurveto{\pgfqpoint{2.794846in}{1.168396in}}{\pgfqpoint{2.799236in}{1.157797in}}{\pgfqpoint{2.807050in}{1.149984in}}%
\pgfpathcurveto{\pgfqpoint{2.814863in}{1.142170in}}{\pgfqpoint{2.825462in}{1.137780in}}{\pgfqpoint{2.836512in}{1.137780in}}%
\pgfpathlineto{\pgfqpoint{2.836512in}{1.137780in}}%
\pgfpathclose%
\pgfusepath{stroke}%
\end{pgfscope}%
\begin{pgfscope}%
\pgfpathrectangle{\pgfqpoint{0.847223in}{0.554012in}}{\pgfqpoint{6.200000in}{4.620000in}}%
\pgfusepath{clip}%
\pgfsetbuttcap%
\pgfsetroundjoin%
\pgfsetlinewidth{1.003750pt}%
\definecolor{currentstroke}{rgb}{1.000000,0.000000,0.000000}%
\pgfsetstrokecolor{currentstroke}%
\pgfsetdash{}{0pt}%
\pgfpathmoveto{\pgfqpoint{2.841846in}{1.135554in}}%
\pgfpathcurveto{\pgfqpoint{2.852896in}{1.135554in}}{\pgfqpoint{2.863495in}{1.139944in}}{\pgfqpoint{2.871308in}{1.147758in}}%
\pgfpathcurveto{\pgfqpoint{2.879122in}{1.155571in}}{\pgfqpoint{2.883512in}{1.166170in}}{\pgfqpoint{2.883512in}{1.177221in}}%
\pgfpathcurveto{\pgfqpoint{2.883512in}{1.188271in}}{\pgfqpoint{2.879122in}{1.198870in}}{\pgfqpoint{2.871308in}{1.206683in}}%
\pgfpathcurveto{\pgfqpoint{2.863495in}{1.214497in}}{\pgfqpoint{2.852896in}{1.218887in}}{\pgfqpoint{2.841846in}{1.218887in}}%
\pgfpathcurveto{\pgfqpoint{2.830796in}{1.218887in}}{\pgfqpoint{2.820197in}{1.214497in}}{\pgfqpoint{2.812383in}{1.206683in}}%
\pgfpathcurveto{\pgfqpoint{2.804569in}{1.198870in}}{\pgfqpoint{2.800179in}{1.188271in}}{\pgfqpoint{2.800179in}{1.177221in}}%
\pgfpathcurveto{\pgfqpoint{2.800179in}{1.166170in}}{\pgfqpoint{2.804569in}{1.155571in}}{\pgfqpoint{2.812383in}{1.147758in}}%
\pgfpathcurveto{\pgfqpoint{2.820197in}{1.139944in}}{\pgfqpoint{2.830796in}{1.135554in}}{\pgfqpoint{2.841846in}{1.135554in}}%
\pgfpathlineto{\pgfqpoint{2.841846in}{1.135554in}}%
\pgfpathclose%
\pgfusepath{stroke}%
\end{pgfscope}%
\begin{pgfscope}%
\pgfpathrectangle{\pgfqpoint{0.847223in}{0.554012in}}{\pgfqpoint{6.200000in}{4.620000in}}%
\pgfusepath{clip}%
\pgfsetbuttcap%
\pgfsetroundjoin%
\pgfsetlinewidth{1.003750pt}%
\definecolor{currentstroke}{rgb}{1.000000,0.000000,0.000000}%
\pgfsetstrokecolor{currentstroke}%
\pgfsetdash{}{0pt}%
\pgfpathmoveto{\pgfqpoint{2.847179in}{1.133337in}}%
\pgfpathcurveto{\pgfqpoint{2.858229in}{1.133337in}}{\pgfqpoint{2.868828in}{1.137728in}}{\pgfqpoint{2.876642in}{1.145541in}}%
\pgfpathcurveto{\pgfqpoint{2.884455in}{1.153355in}}{\pgfqpoint{2.888846in}{1.163954in}}{\pgfqpoint{2.888846in}{1.175004in}}%
\pgfpathcurveto{\pgfqpoint{2.888846in}{1.186054in}}{\pgfqpoint{2.884455in}{1.196653in}}{\pgfqpoint{2.876642in}{1.204467in}}%
\pgfpathcurveto{\pgfqpoint{2.868828in}{1.212280in}}{\pgfqpoint{2.858229in}{1.216671in}}{\pgfqpoint{2.847179in}{1.216671in}}%
\pgfpathcurveto{\pgfqpoint{2.836129in}{1.216671in}}{\pgfqpoint{2.825530in}{1.212280in}}{\pgfqpoint{2.817716in}{1.204467in}}%
\pgfpathcurveto{\pgfqpoint{2.809903in}{1.196653in}}{\pgfqpoint{2.805512in}{1.186054in}}{\pgfqpoint{2.805512in}{1.175004in}}%
\pgfpathcurveto{\pgfqpoint{2.805512in}{1.163954in}}{\pgfqpoint{2.809903in}{1.153355in}}{\pgfqpoint{2.817716in}{1.145541in}}%
\pgfpathcurveto{\pgfqpoint{2.825530in}{1.137728in}}{\pgfqpoint{2.836129in}{1.133337in}}{\pgfqpoint{2.847179in}{1.133337in}}%
\pgfpathlineto{\pgfqpoint{2.847179in}{1.133337in}}%
\pgfpathclose%
\pgfusepath{stroke}%
\end{pgfscope}%
\begin{pgfscope}%
\pgfpathrectangle{\pgfqpoint{0.847223in}{0.554012in}}{\pgfqpoint{6.200000in}{4.620000in}}%
\pgfusepath{clip}%
\pgfsetbuttcap%
\pgfsetroundjoin%
\pgfsetlinewidth{1.003750pt}%
\definecolor{currentstroke}{rgb}{1.000000,0.000000,0.000000}%
\pgfsetstrokecolor{currentstroke}%
\pgfsetdash{}{0pt}%
\pgfpathmoveto{\pgfqpoint{2.852512in}{1.131130in}}%
\pgfpathcurveto{\pgfqpoint{2.863562in}{1.131130in}}{\pgfqpoint{2.874161in}{1.135521in}}{\pgfqpoint{2.881975in}{1.143334in}}%
\pgfpathcurveto{\pgfqpoint{2.889789in}{1.151148in}}{\pgfqpoint{2.894179in}{1.161747in}}{\pgfqpoint{2.894179in}{1.172797in}}%
\pgfpathcurveto{\pgfqpoint{2.894179in}{1.183847in}}{\pgfqpoint{2.889789in}{1.194446in}}{\pgfqpoint{2.881975in}{1.202260in}}%
\pgfpathcurveto{\pgfqpoint{2.874161in}{1.210073in}}{\pgfqpoint{2.863562in}{1.214464in}}{\pgfqpoint{2.852512in}{1.214464in}}%
\pgfpathcurveto{\pgfqpoint{2.841462in}{1.214464in}}{\pgfqpoint{2.830863in}{1.210073in}}{\pgfqpoint{2.823049in}{1.202260in}}%
\pgfpathcurveto{\pgfqpoint{2.815236in}{1.194446in}}{\pgfqpoint{2.810845in}{1.183847in}}{\pgfqpoint{2.810845in}{1.172797in}}%
\pgfpathcurveto{\pgfqpoint{2.810845in}{1.161747in}}{\pgfqpoint{2.815236in}{1.151148in}}{\pgfqpoint{2.823049in}{1.143334in}}%
\pgfpathcurveto{\pgfqpoint{2.830863in}{1.135521in}}{\pgfqpoint{2.841462in}{1.131130in}}{\pgfqpoint{2.852512in}{1.131130in}}%
\pgfpathlineto{\pgfqpoint{2.852512in}{1.131130in}}%
\pgfpathclose%
\pgfusepath{stroke}%
\end{pgfscope}%
\begin{pgfscope}%
\pgfpathrectangle{\pgfqpoint{0.847223in}{0.554012in}}{\pgfqpoint{6.200000in}{4.620000in}}%
\pgfusepath{clip}%
\pgfsetbuttcap%
\pgfsetroundjoin%
\pgfsetlinewidth{1.003750pt}%
\definecolor{currentstroke}{rgb}{1.000000,0.000000,0.000000}%
\pgfsetstrokecolor{currentstroke}%
\pgfsetdash{}{0pt}%
\pgfpathmoveto{\pgfqpoint{2.857845in}{1.128932in}}%
\pgfpathcurveto{\pgfqpoint{2.868895in}{1.128932in}}{\pgfqpoint{2.879495in}{1.133323in}}{\pgfqpoint{2.887308in}{1.141136in}}%
\pgfpathcurveto{\pgfqpoint{2.895122in}{1.148950in}}{\pgfqpoint{2.899512in}{1.159549in}}{\pgfqpoint{2.899512in}{1.170599in}}%
\pgfpathcurveto{\pgfqpoint{2.899512in}{1.181649in}}{\pgfqpoint{2.895122in}{1.192248in}}{\pgfqpoint{2.887308in}{1.200062in}}%
\pgfpathcurveto{\pgfqpoint{2.879495in}{1.207875in}}{\pgfqpoint{2.868895in}{1.212266in}}{\pgfqpoint{2.857845in}{1.212266in}}%
\pgfpathcurveto{\pgfqpoint{2.846795in}{1.212266in}}{\pgfqpoint{2.836196in}{1.207875in}}{\pgfqpoint{2.828383in}{1.200062in}}%
\pgfpathcurveto{\pgfqpoint{2.820569in}{1.192248in}}{\pgfqpoint{2.816179in}{1.181649in}}{\pgfqpoint{2.816179in}{1.170599in}}%
\pgfpathcurveto{\pgfqpoint{2.816179in}{1.159549in}}{\pgfqpoint{2.820569in}{1.148950in}}{\pgfqpoint{2.828383in}{1.141136in}}%
\pgfpathcurveto{\pgfqpoint{2.836196in}{1.133323in}}{\pgfqpoint{2.846795in}{1.128932in}}{\pgfqpoint{2.857845in}{1.128932in}}%
\pgfpathlineto{\pgfqpoint{2.857845in}{1.128932in}}%
\pgfpathclose%
\pgfusepath{stroke}%
\end{pgfscope}%
\begin{pgfscope}%
\pgfpathrectangle{\pgfqpoint{0.847223in}{0.554012in}}{\pgfqpoint{6.200000in}{4.620000in}}%
\pgfusepath{clip}%
\pgfsetbuttcap%
\pgfsetroundjoin%
\pgfsetlinewidth{1.003750pt}%
\definecolor{currentstroke}{rgb}{1.000000,0.000000,0.000000}%
\pgfsetstrokecolor{currentstroke}%
\pgfsetdash{}{0pt}%
\pgfpathmoveto{\pgfqpoint{2.863179in}{1.126744in}}%
\pgfpathcurveto{\pgfqpoint{2.874229in}{1.126744in}}{\pgfqpoint{2.884828in}{1.131134in}}{\pgfqpoint{2.892641in}{1.138948in}}%
\pgfpathcurveto{\pgfqpoint{2.900455in}{1.146761in}}{\pgfqpoint{2.904845in}{1.157360in}}{\pgfqpoint{2.904845in}{1.168410in}}%
\pgfpathcurveto{\pgfqpoint{2.904845in}{1.179461in}}{\pgfqpoint{2.900455in}{1.190060in}}{\pgfqpoint{2.892641in}{1.197873in}}%
\pgfpathcurveto{\pgfqpoint{2.884828in}{1.205687in}}{\pgfqpoint{2.874229in}{1.210077in}}{\pgfqpoint{2.863179in}{1.210077in}}%
\pgfpathcurveto{\pgfqpoint{2.852128in}{1.210077in}}{\pgfqpoint{2.841529in}{1.205687in}}{\pgfqpoint{2.833716in}{1.197873in}}%
\pgfpathcurveto{\pgfqpoint{2.825902in}{1.190060in}}{\pgfqpoint{2.821512in}{1.179461in}}{\pgfqpoint{2.821512in}{1.168410in}}%
\pgfpathcurveto{\pgfqpoint{2.821512in}{1.157360in}}{\pgfqpoint{2.825902in}{1.146761in}}{\pgfqpoint{2.833716in}{1.138948in}}%
\pgfpathcurveto{\pgfqpoint{2.841529in}{1.131134in}}{\pgfqpoint{2.852128in}{1.126744in}}{\pgfqpoint{2.863179in}{1.126744in}}%
\pgfpathlineto{\pgfqpoint{2.863179in}{1.126744in}}%
\pgfpathclose%
\pgfusepath{stroke}%
\end{pgfscope}%
\begin{pgfscope}%
\pgfpathrectangle{\pgfqpoint{0.847223in}{0.554012in}}{\pgfqpoint{6.200000in}{4.620000in}}%
\pgfusepath{clip}%
\pgfsetbuttcap%
\pgfsetroundjoin%
\pgfsetlinewidth{1.003750pt}%
\definecolor{currentstroke}{rgb}{1.000000,0.000000,0.000000}%
\pgfsetstrokecolor{currentstroke}%
\pgfsetdash{}{0pt}%
\pgfpathmoveto{\pgfqpoint{2.868512in}{1.124564in}}%
\pgfpathcurveto{\pgfqpoint{2.879562in}{1.124564in}}{\pgfqpoint{2.890161in}{1.128955in}}{\pgfqpoint{2.897975in}{1.136768in}}%
\pgfpathcurveto{\pgfqpoint{2.905788in}{1.144582in}}{\pgfqpoint{2.910178in}{1.155181in}}{\pgfqpoint{2.910178in}{1.166231in}}%
\pgfpathcurveto{\pgfqpoint{2.910178in}{1.177281in}}{\pgfqpoint{2.905788in}{1.187880in}}{\pgfqpoint{2.897975in}{1.195694in}}%
\pgfpathcurveto{\pgfqpoint{2.890161in}{1.203507in}}{\pgfqpoint{2.879562in}{1.207898in}}{\pgfqpoint{2.868512in}{1.207898in}}%
\pgfpathcurveto{\pgfqpoint{2.857462in}{1.207898in}}{\pgfqpoint{2.846863in}{1.203507in}}{\pgfqpoint{2.839049in}{1.195694in}}%
\pgfpathcurveto{\pgfqpoint{2.831235in}{1.187880in}}{\pgfqpoint{2.826845in}{1.177281in}}{\pgfqpoint{2.826845in}{1.166231in}}%
\pgfpathcurveto{\pgfqpoint{2.826845in}{1.155181in}}{\pgfqpoint{2.831235in}{1.144582in}}{\pgfqpoint{2.839049in}{1.136768in}}%
\pgfpathcurveto{\pgfqpoint{2.846863in}{1.128955in}}{\pgfqpoint{2.857462in}{1.124564in}}{\pgfqpoint{2.868512in}{1.124564in}}%
\pgfpathlineto{\pgfqpoint{2.868512in}{1.124564in}}%
\pgfpathclose%
\pgfusepath{stroke}%
\end{pgfscope}%
\begin{pgfscope}%
\pgfpathrectangle{\pgfqpoint{0.847223in}{0.554012in}}{\pgfqpoint{6.200000in}{4.620000in}}%
\pgfusepath{clip}%
\pgfsetbuttcap%
\pgfsetroundjoin%
\pgfsetlinewidth{1.003750pt}%
\definecolor{currentstroke}{rgb}{1.000000,0.000000,0.000000}%
\pgfsetstrokecolor{currentstroke}%
\pgfsetdash{}{0pt}%
\pgfpathmoveto{\pgfqpoint{2.873845in}{1.122394in}}%
\pgfpathcurveto{\pgfqpoint{2.884895in}{1.122394in}}{\pgfqpoint{2.895494in}{1.126784in}}{\pgfqpoint{2.903308in}{1.134598in}}%
\pgfpathcurveto{\pgfqpoint{2.911121in}{1.142412in}}{\pgfqpoint{2.915512in}{1.153011in}}{\pgfqpoint{2.915512in}{1.164061in}}%
\pgfpathcurveto{\pgfqpoint{2.915512in}{1.175111in}}{\pgfqpoint{2.911121in}{1.185710in}}{\pgfqpoint{2.903308in}{1.193524in}}%
\pgfpathcurveto{\pgfqpoint{2.895494in}{1.201337in}}{\pgfqpoint{2.884895in}{1.205727in}}{\pgfqpoint{2.873845in}{1.205727in}}%
\pgfpathcurveto{\pgfqpoint{2.862795in}{1.205727in}}{\pgfqpoint{2.852196in}{1.201337in}}{\pgfqpoint{2.844382in}{1.193524in}}%
\pgfpathcurveto{\pgfqpoint{2.836569in}{1.185710in}}{\pgfqpoint{2.832178in}{1.175111in}}{\pgfqpoint{2.832178in}{1.164061in}}%
\pgfpathcurveto{\pgfqpoint{2.832178in}{1.153011in}}{\pgfqpoint{2.836569in}{1.142412in}}{\pgfqpoint{2.844382in}{1.134598in}}%
\pgfpathcurveto{\pgfqpoint{2.852196in}{1.126784in}}{\pgfqpoint{2.862795in}{1.122394in}}{\pgfqpoint{2.873845in}{1.122394in}}%
\pgfpathlineto{\pgfqpoint{2.873845in}{1.122394in}}%
\pgfpathclose%
\pgfusepath{stroke}%
\end{pgfscope}%
\begin{pgfscope}%
\pgfpathrectangle{\pgfqpoint{0.847223in}{0.554012in}}{\pgfqpoint{6.200000in}{4.620000in}}%
\pgfusepath{clip}%
\pgfsetbuttcap%
\pgfsetroundjoin%
\pgfsetlinewidth{1.003750pt}%
\definecolor{currentstroke}{rgb}{1.000000,0.000000,0.000000}%
\pgfsetstrokecolor{currentstroke}%
\pgfsetdash{}{0pt}%
\pgfpathmoveto{\pgfqpoint{2.879178in}{1.120233in}}%
\pgfpathcurveto{\pgfqpoint{2.890228in}{1.120233in}}{\pgfqpoint{2.900827in}{1.124623in}}{\pgfqpoint{2.908641in}{1.132437in}}%
\pgfpathcurveto{\pgfqpoint{2.916455in}{1.140250in}}{\pgfqpoint{2.920845in}{1.150849in}}{\pgfqpoint{2.920845in}{1.161900in}}%
\pgfpathcurveto{\pgfqpoint{2.920845in}{1.172950in}}{\pgfqpoint{2.916455in}{1.183549in}}{\pgfqpoint{2.908641in}{1.191362in}}%
\pgfpathcurveto{\pgfqpoint{2.900827in}{1.199176in}}{\pgfqpoint{2.890228in}{1.203566in}}{\pgfqpoint{2.879178in}{1.203566in}}%
\pgfpathcurveto{\pgfqpoint{2.868128in}{1.203566in}}{\pgfqpoint{2.857529in}{1.199176in}}{\pgfqpoint{2.849715in}{1.191362in}}%
\pgfpathcurveto{\pgfqpoint{2.841902in}{1.183549in}}{\pgfqpoint{2.837512in}{1.172950in}}{\pgfqpoint{2.837512in}{1.161900in}}%
\pgfpathcurveto{\pgfqpoint{2.837512in}{1.150849in}}{\pgfqpoint{2.841902in}{1.140250in}}{\pgfqpoint{2.849715in}{1.132437in}}%
\pgfpathcurveto{\pgfqpoint{2.857529in}{1.124623in}}{\pgfqpoint{2.868128in}{1.120233in}}{\pgfqpoint{2.879178in}{1.120233in}}%
\pgfpathlineto{\pgfqpoint{2.879178in}{1.120233in}}%
\pgfpathclose%
\pgfusepath{stroke}%
\end{pgfscope}%
\begin{pgfscope}%
\pgfpathrectangle{\pgfqpoint{0.847223in}{0.554012in}}{\pgfqpoint{6.200000in}{4.620000in}}%
\pgfusepath{clip}%
\pgfsetbuttcap%
\pgfsetroundjoin%
\pgfsetlinewidth{1.003750pt}%
\definecolor{currentstroke}{rgb}{1.000000,0.000000,0.000000}%
\pgfsetstrokecolor{currentstroke}%
\pgfsetdash{}{0pt}%
\pgfpathmoveto{\pgfqpoint{2.884511in}{1.118081in}}%
\pgfpathcurveto{\pgfqpoint{2.895562in}{1.118081in}}{\pgfqpoint{2.906161in}{1.122471in}}{\pgfqpoint{2.913974in}{1.130284in}}%
\pgfpathcurveto{\pgfqpoint{2.921788in}{1.138098in}}{\pgfqpoint{2.926178in}{1.148697in}}{\pgfqpoint{2.926178in}{1.159747in}}%
\pgfpathcurveto{\pgfqpoint{2.926178in}{1.170797in}}{\pgfqpoint{2.921788in}{1.181396in}}{\pgfqpoint{2.913974in}{1.189210in}}%
\pgfpathcurveto{\pgfqpoint{2.906161in}{1.197024in}}{\pgfqpoint{2.895562in}{1.201414in}}{\pgfqpoint{2.884511in}{1.201414in}}%
\pgfpathcurveto{\pgfqpoint{2.873461in}{1.201414in}}{\pgfqpoint{2.862862in}{1.197024in}}{\pgfqpoint{2.855049in}{1.189210in}}%
\pgfpathcurveto{\pgfqpoint{2.847235in}{1.181396in}}{\pgfqpoint{2.842845in}{1.170797in}}{\pgfqpoint{2.842845in}{1.159747in}}%
\pgfpathcurveto{\pgfqpoint{2.842845in}{1.148697in}}{\pgfqpoint{2.847235in}{1.138098in}}{\pgfqpoint{2.855049in}{1.130284in}}%
\pgfpathcurveto{\pgfqpoint{2.862862in}{1.122471in}}{\pgfqpoint{2.873461in}{1.118081in}}{\pgfqpoint{2.884511in}{1.118081in}}%
\pgfpathlineto{\pgfqpoint{2.884511in}{1.118081in}}%
\pgfpathclose%
\pgfusepath{stroke}%
\end{pgfscope}%
\begin{pgfscope}%
\pgfpathrectangle{\pgfqpoint{0.847223in}{0.554012in}}{\pgfqpoint{6.200000in}{4.620000in}}%
\pgfusepath{clip}%
\pgfsetbuttcap%
\pgfsetroundjoin%
\pgfsetlinewidth{1.003750pt}%
\definecolor{currentstroke}{rgb}{1.000000,0.000000,0.000000}%
\pgfsetstrokecolor{currentstroke}%
\pgfsetdash{}{0pt}%
\pgfpathmoveto{\pgfqpoint{2.889845in}{1.115937in}}%
\pgfpathcurveto{\pgfqpoint{2.900895in}{1.115937in}}{\pgfqpoint{2.911494in}{1.120328in}}{\pgfqpoint{2.919307in}{1.128141in}}%
\pgfpathcurveto{\pgfqpoint{2.927121in}{1.135955in}}{\pgfqpoint{2.931511in}{1.146554in}}{\pgfqpoint{2.931511in}{1.157604in}}%
\pgfpathcurveto{\pgfqpoint{2.931511in}{1.168654in}}{\pgfqpoint{2.927121in}{1.179253in}}{\pgfqpoint{2.919307in}{1.187067in}}%
\pgfpathcurveto{\pgfqpoint{2.911494in}{1.194880in}}{\pgfqpoint{2.900895in}{1.199271in}}{\pgfqpoint{2.889845in}{1.199271in}}%
\pgfpathcurveto{\pgfqpoint{2.878795in}{1.199271in}}{\pgfqpoint{2.868195in}{1.194880in}}{\pgfqpoint{2.860382in}{1.187067in}}%
\pgfpathcurveto{\pgfqpoint{2.852568in}{1.179253in}}{\pgfqpoint{2.848178in}{1.168654in}}{\pgfqpoint{2.848178in}{1.157604in}}%
\pgfpathcurveto{\pgfqpoint{2.848178in}{1.146554in}}{\pgfqpoint{2.852568in}{1.135955in}}{\pgfqpoint{2.860382in}{1.128141in}}%
\pgfpathcurveto{\pgfqpoint{2.868195in}{1.120328in}}{\pgfqpoint{2.878795in}{1.115937in}}{\pgfqpoint{2.889845in}{1.115937in}}%
\pgfpathlineto{\pgfqpoint{2.889845in}{1.115937in}}%
\pgfpathclose%
\pgfusepath{stroke}%
\end{pgfscope}%
\begin{pgfscope}%
\pgfpathrectangle{\pgfqpoint{0.847223in}{0.554012in}}{\pgfqpoint{6.200000in}{4.620000in}}%
\pgfusepath{clip}%
\pgfsetbuttcap%
\pgfsetroundjoin%
\pgfsetlinewidth{1.003750pt}%
\definecolor{currentstroke}{rgb}{1.000000,0.000000,0.000000}%
\pgfsetstrokecolor{currentstroke}%
\pgfsetdash{}{0pt}%
\pgfpathmoveto{\pgfqpoint{2.895178in}{1.113803in}}%
\pgfpathcurveto{\pgfqpoint{2.906228in}{1.113803in}}{\pgfqpoint{2.916827in}{1.118193in}}{\pgfqpoint{2.924641in}{1.126007in}}%
\pgfpathcurveto{\pgfqpoint{2.932454in}{1.133820in}}{\pgfqpoint{2.936845in}{1.144419in}}{\pgfqpoint{2.936845in}{1.155470in}}%
\pgfpathcurveto{\pgfqpoint{2.936845in}{1.166520in}}{\pgfqpoint{2.932454in}{1.177119in}}{\pgfqpoint{2.924641in}{1.184932in}}%
\pgfpathcurveto{\pgfqpoint{2.916827in}{1.192746in}}{\pgfqpoint{2.906228in}{1.197136in}}{\pgfqpoint{2.895178in}{1.197136in}}%
\pgfpathcurveto{\pgfqpoint{2.884128in}{1.197136in}}{\pgfqpoint{2.873529in}{1.192746in}}{\pgfqpoint{2.865715in}{1.184932in}}%
\pgfpathcurveto{\pgfqpoint{2.857901in}{1.177119in}}{\pgfqpoint{2.853511in}{1.166520in}}{\pgfqpoint{2.853511in}{1.155470in}}%
\pgfpathcurveto{\pgfqpoint{2.853511in}{1.144419in}}{\pgfqpoint{2.857901in}{1.133820in}}{\pgfqpoint{2.865715in}{1.126007in}}%
\pgfpathcurveto{\pgfqpoint{2.873529in}{1.118193in}}{\pgfqpoint{2.884128in}{1.113803in}}{\pgfqpoint{2.895178in}{1.113803in}}%
\pgfpathlineto{\pgfqpoint{2.895178in}{1.113803in}}%
\pgfpathclose%
\pgfusepath{stroke}%
\end{pgfscope}%
\begin{pgfscope}%
\pgfpathrectangle{\pgfqpoint{0.847223in}{0.554012in}}{\pgfqpoint{6.200000in}{4.620000in}}%
\pgfusepath{clip}%
\pgfsetbuttcap%
\pgfsetroundjoin%
\pgfsetlinewidth{1.003750pt}%
\definecolor{currentstroke}{rgb}{1.000000,0.000000,0.000000}%
\pgfsetstrokecolor{currentstroke}%
\pgfsetdash{}{0pt}%
\pgfpathmoveto{\pgfqpoint{2.900511in}{1.111677in}}%
\pgfpathcurveto{\pgfqpoint{2.911561in}{1.111677in}}{\pgfqpoint{2.922160in}{1.116068in}}{\pgfqpoint{2.929974in}{1.123881in}}%
\pgfpathcurveto{\pgfqpoint{2.937787in}{1.131695in}}{\pgfqpoint{2.942178in}{1.142294in}}{\pgfqpoint{2.942178in}{1.153344in}}%
\pgfpathcurveto{\pgfqpoint{2.942178in}{1.164394in}}{\pgfqpoint{2.937787in}{1.174993in}}{\pgfqpoint{2.929974in}{1.182807in}}%
\pgfpathcurveto{\pgfqpoint{2.922160in}{1.190620in}}{\pgfqpoint{2.911561in}{1.195011in}}{\pgfqpoint{2.900511in}{1.195011in}}%
\pgfpathcurveto{\pgfqpoint{2.889461in}{1.195011in}}{\pgfqpoint{2.878862in}{1.190620in}}{\pgfqpoint{2.871048in}{1.182807in}}%
\pgfpathcurveto{\pgfqpoint{2.863235in}{1.174993in}}{\pgfqpoint{2.858844in}{1.164394in}}{\pgfqpoint{2.858844in}{1.153344in}}%
\pgfpathcurveto{\pgfqpoint{2.858844in}{1.142294in}}{\pgfqpoint{2.863235in}{1.131695in}}{\pgfqpoint{2.871048in}{1.123881in}}%
\pgfpathcurveto{\pgfqpoint{2.878862in}{1.116068in}}{\pgfqpoint{2.889461in}{1.111677in}}{\pgfqpoint{2.900511in}{1.111677in}}%
\pgfpathlineto{\pgfqpoint{2.900511in}{1.111677in}}%
\pgfpathclose%
\pgfusepath{stroke}%
\end{pgfscope}%
\begin{pgfscope}%
\pgfpathrectangle{\pgfqpoint{0.847223in}{0.554012in}}{\pgfqpoint{6.200000in}{4.620000in}}%
\pgfusepath{clip}%
\pgfsetbuttcap%
\pgfsetroundjoin%
\pgfsetlinewidth{1.003750pt}%
\definecolor{currentstroke}{rgb}{1.000000,0.000000,0.000000}%
\pgfsetstrokecolor{currentstroke}%
\pgfsetdash{}{0pt}%
\pgfpathmoveto{\pgfqpoint{2.905844in}{1.109561in}}%
\pgfpathcurveto{\pgfqpoint{2.916894in}{1.109561in}}{\pgfqpoint{2.927493in}{1.113951in}}{\pgfqpoint{2.935307in}{1.121764in}}%
\pgfpathcurveto{\pgfqpoint{2.943121in}{1.129578in}}{\pgfqpoint{2.947511in}{1.140177in}}{\pgfqpoint{2.947511in}{1.151227in}}%
\pgfpathcurveto{\pgfqpoint{2.947511in}{1.162277in}}{\pgfqpoint{2.943121in}{1.172876in}}{\pgfqpoint{2.935307in}{1.180690in}}%
\pgfpathcurveto{\pgfqpoint{2.927493in}{1.188504in}}{\pgfqpoint{2.916894in}{1.192894in}}{\pgfqpoint{2.905844in}{1.192894in}}%
\pgfpathcurveto{\pgfqpoint{2.894794in}{1.192894in}}{\pgfqpoint{2.884195in}{1.188504in}}{\pgfqpoint{2.876382in}{1.180690in}}%
\pgfpathcurveto{\pgfqpoint{2.868568in}{1.172876in}}{\pgfqpoint{2.864178in}{1.162277in}}{\pgfqpoint{2.864178in}{1.151227in}}%
\pgfpathcurveto{\pgfqpoint{2.864178in}{1.140177in}}{\pgfqpoint{2.868568in}{1.129578in}}{\pgfqpoint{2.876382in}{1.121764in}}%
\pgfpathcurveto{\pgfqpoint{2.884195in}{1.113951in}}{\pgfqpoint{2.894794in}{1.109561in}}{\pgfqpoint{2.905844in}{1.109561in}}%
\pgfpathlineto{\pgfqpoint{2.905844in}{1.109561in}}%
\pgfpathclose%
\pgfusepath{stroke}%
\end{pgfscope}%
\begin{pgfscope}%
\pgfpathrectangle{\pgfqpoint{0.847223in}{0.554012in}}{\pgfqpoint{6.200000in}{4.620000in}}%
\pgfusepath{clip}%
\pgfsetbuttcap%
\pgfsetroundjoin%
\pgfsetlinewidth{1.003750pt}%
\definecolor{currentstroke}{rgb}{1.000000,0.000000,0.000000}%
\pgfsetstrokecolor{currentstroke}%
\pgfsetdash{}{0pt}%
\pgfpathmoveto{\pgfqpoint{2.911178in}{1.107452in}}%
\pgfpathcurveto{\pgfqpoint{2.922228in}{1.107452in}}{\pgfqpoint{2.932827in}{1.111843in}}{\pgfqpoint{2.940640in}{1.119656in}}%
\pgfpathcurveto{\pgfqpoint{2.948454in}{1.127470in}}{\pgfqpoint{2.952844in}{1.138069in}}{\pgfqpoint{2.952844in}{1.149119in}}%
\pgfpathcurveto{\pgfqpoint{2.952844in}{1.160169in}}{\pgfqpoint{2.948454in}{1.170768in}}{\pgfqpoint{2.940640in}{1.178582in}}%
\pgfpathcurveto{\pgfqpoint{2.932827in}{1.186396in}}{\pgfqpoint{2.922228in}{1.190786in}}{\pgfqpoint{2.911178in}{1.190786in}}%
\pgfpathcurveto{\pgfqpoint{2.900127in}{1.190786in}}{\pgfqpoint{2.889528in}{1.186396in}}{\pgfqpoint{2.881715in}{1.178582in}}%
\pgfpathcurveto{\pgfqpoint{2.873901in}{1.170768in}}{\pgfqpoint{2.869511in}{1.160169in}}{\pgfqpoint{2.869511in}{1.149119in}}%
\pgfpathcurveto{\pgfqpoint{2.869511in}{1.138069in}}{\pgfqpoint{2.873901in}{1.127470in}}{\pgfqpoint{2.881715in}{1.119656in}}%
\pgfpathcurveto{\pgfqpoint{2.889528in}{1.111843in}}{\pgfqpoint{2.900127in}{1.107452in}}{\pgfqpoint{2.911178in}{1.107452in}}%
\pgfpathlineto{\pgfqpoint{2.911178in}{1.107452in}}%
\pgfpathclose%
\pgfusepath{stroke}%
\end{pgfscope}%
\begin{pgfscope}%
\pgfpathrectangle{\pgfqpoint{0.847223in}{0.554012in}}{\pgfqpoint{6.200000in}{4.620000in}}%
\pgfusepath{clip}%
\pgfsetbuttcap%
\pgfsetroundjoin%
\pgfsetlinewidth{1.003750pt}%
\definecolor{currentstroke}{rgb}{1.000000,0.000000,0.000000}%
\pgfsetstrokecolor{currentstroke}%
\pgfsetdash{}{0pt}%
\pgfpathmoveto{\pgfqpoint{2.916511in}{1.105353in}}%
\pgfpathcurveto{\pgfqpoint{2.927561in}{1.105353in}}{\pgfqpoint{2.938160in}{1.109743in}}{\pgfqpoint{2.945974in}{1.117557in}}%
\pgfpathcurveto{\pgfqpoint{2.953787in}{1.125371in}}{\pgfqpoint{2.958177in}{1.135970in}}{\pgfqpoint{2.958177in}{1.147020in}}%
\pgfpathcurveto{\pgfqpoint{2.958177in}{1.158070in}}{\pgfqpoint{2.953787in}{1.168669in}}{\pgfqpoint{2.945974in}{1.176482in}}%
\pgfpathcurveto{\pgfqpoint{2.938160in}{1.184296in}}{\pgfqpoint{2.927561in}{1.188686in}}{\pgfqpoint{2.916511in}{1.188686in}}%
\pgfpathcurveto{\pgfqpoint{2.905461in}{1.188686in}}{\pgfqpoint{2.894862in}{1.184296in}}{\pgfqpoint{2.887048in}{1.176482in}}%
\pgfpathcurveto{\pgfqpoint{2.879234in}{1.168669in}}{\pgfqpoint{2.874844in}{1.158070in}}{\pgfqpoint{2.874844in}{1.147020in}}%
\pgfpathcurveto{\pgfqpoint{2.874844in}{1.135970in}}{\pgfqpoint{2.879234in}{1.125371in}}{\pgfqpoint{2.887048in}{1.117557in}}%
\pgfpathcurveto{\pgfqpoint{2.894862in}{1.109743in}}{\pgfqpoint{2.905461in}{1.105353in}}{\pgfqpoint{2.916511in}{1.105353in}}%
\pgfpathlineto{\pgfqpoint{2.916511in}{1.105353in}}%
\pgfpathclose%
\pgfusepath{stroke}%
\end{pgfscope}%
\begin{pgfscope}%
\pgfpathrectangle{\pgfqpoint{0.847223in}{0.554012in}}{\pgfqpoint{6.200000in}{4.620000in}}%
\pgfusepath{clip}%
\pgfsetbuttcap%
\pgfsetroundjoin%
\pgfsetlinewidth{1.003750pt}%
\definecolor{currentstroke}{rgb}{1.000000,0.000000,0.000000}%
\pgfsetstrokecolor{currentstroke}%
\pgfsetdash{}{0pt}%
\pgfpathmoveto{\pgfqpoint{2.921844in}{1.103262in}}%
\pgfpathcurveto{\pgfqpoint{2.932894in}{1.103262in}}{\pgfqpoint{2.943493in}{1.107653in}}{\pgfqpoint{2.951307in}{1.115466in}}%
\pgfpathcurveto{\pgfqpoint{2.959120in}{1.123280in}}{\pgfqpoint{2.963511in}{1.133879in}}{\pgfqpoint{2.963511in}{1.144929in}}%
\pgfpathcurveto{\pgfqpoint{2.963511in}{1.155979in}}{\pgfqpoint{2.959120in}{1.166578in}}{\pgfqpoint{2.951307in}{1.174392in}}%
\pgfpathcurveto{\pgfqpoint{2.943493in}{1.182205in}}{\pgfqpoint{2.932894in}{1.186596in}}{\pgfqpoint{2.921844in}{1.186596in}}%
\pgfpathcurveto{\pgfqpoint{2.910794in}{1.186596in}}{\pgfqpoint{2.900195in}{1.182205in}}{\pgfqpoint{2.892381in}{1.174392in}}%
\pgfpathcurveto{\pgfqpoint{2.884568in}{1.166578in}}{\pgfqpoint{2.880177in}{1.155979in}}{\pgfqpoint{2.880177in}{1.144929in}}%
\pgfpathcurveto{\pgfqpoint{2.880177in}{1.133879in}}{\pgfqpoint{2.884568in}{1.123280in}}{\pgfqpoint{2.892381in}{1.115466in}}%
\pgfpathcurveto{\pgfqpoint{2.900195in}{1.107653in}}{\pgfqpoint{2.910794in}{1.103262in}}{\pgfqpoint{2.921844in}{1.103262in}}%
\pgfpathlineto{\pgfqpoint{2.921844in}{1.103262in}}%
\pgfpathclose%
\pgfusepath{stroke}%
\end{pgfscope}%
\begin{pgfscope}%
\pgfpathrectangle{\pgfqpoint{0.847223in}{0.554012in}}{\pgfqpoint{6.200000in}{4.620000in}}%
\pgfusepath{clip}%
\pgfsetbuttcap%
\pgfsetroundjoin%
\pgfsetlinewidth{1.003750pt}%
\definecolor{currentstroke}{rgb}{1.000000,0.000000,0.000000}%
\pgfsetstrokecolor{currentstroke}%
\pgfsetdash{}{0pt}%
\pgfpathmoveto{\pgfqpoint{2.927177in}{1.101180in}}%
\pgfpathcurveto{\pgfqpoint{2.938227in}{1.101180in}}{\pgfqpoint{2.948826in}{1.105570in}}{\pgfqpoint{2.956640in}{1.113384in}}%
\pgfpathcurveto{\pgfqpoint{2.964454in}{1.121198in}}{\pgfqpoint{2.968844in}{1.131797in}}{\pgfqpoint{2.968844in}{1.142847in}}%
\pgfpathcurveto{\pgfqpoint{2.968844in}{1.153897in}}{\pgfqpoint{2.964454in}{1.164496in}}{\pgfqpoint{2.956640in}{1.172310in}}%
\pgfpathcurveto{\pgfqpoint{2.948826in}{1.180123in}}{\pgfqpoint{2.938227in}{1.184513in}}{\pgfqpoint{2.927177in}{1.184513in}}%
\pgfpathcurveto{\pgfqpoint{2.916127in}{1.184513in}}{\pgfqpoint{2.905528in}{1.180123in}}{\pgfqpoint{2.897714in}{1.172310in}}%
\pgfpathcurveto{\pgfqpoint{2.889901in}{1.164496in}}{\pgfqpoint{2.885510in}{1.153897in}}{\pgfqpoint{2.885510in}{1.142847in}}%
\pgfpathcurveto{\pgfqpoint{2.885510in}{1.131797in}}{\pgfqpoint{2.889901in}{1.121198in}}{\pgfqpoint{2.897714in}{1.113384in}}%
\pgfpathcurveto{\pgfqpoint{2.905528in}{1.105570in}}{\pgfqpoint{2.916127in}{1.101180in}}{\pgfqpoint{2.927177in}{1.101180in}}%
\pgfpathlineto{\pgfqpoint{2.927177in}{1.101180in}}%
\pgfpathclose%
\pgfusepath{stroke}%
\end{pgfscope}%
\begin{pgfscope}%
\pgfpathrectangle{\pgfqpoint{0.847223in}{0.554012in}}{\pgfqpoint{6.200000in}{4.620000in}}%
\pgfusepath{clip}%
\pgfsetbuttcap%
\pgfsetroundjoin%
\pgfsetlinewidth{1.003750pt}%
\definecolor{currentstroke}{rgb}{1.000000,0.000000,0.000000}%
\pgfsetstrokecolor{currentstroke}%
\pgfsetdash{}{0pt}%
\pgfpathmoveto{\pgfqpoint{2.932510in}{1.099106in}}%
\pgfpathcurveto{\pgfqpoint{2.943561in}{1.099106in}}{\pgfqpoint{2.954160in}{1.103497in}}{\pgfqpoint{2.961973in}{1.111310in}}%
\pgfpathcurveto{\pgfqpoint{2.969787in}{1.119124in}}{\pgfqpoint{2.974177in}{1.129723in}}{\pgfqpoint{2.974177in}{1.140773in}}%
\pgfpathcurveto{\pgfqpoint{2.974177in}{1.151823in}}{\pgfqpoint{2.969787in}{1.162422in}}{\pgfqpoint{2.961973in}{1.170236in}}%
\pgfpathcurveto{\pgfqpoint{2.954160in}{1.178049in}}{\pgfqpoint{2.943561in}{1.182440in}}{\pgfqpoint{2.932510in}{1.182440in}}%
\pgfpathcurveto{\pgfqpoint{2.921460in}{1.182440in}}{\pgfqpoint{2.910861in}{1.178049in}}{\pgfqpoint{2.903048in}{1.170236in}}%
\pgfpathcurveto{\pgfqpoint{2.895234in}{1.162422in}}{\pgfqpoint{2.890844in}{1.151823in}}{\pgfqpoint{2.890844in}{1.140773in}}%
\pgfpathcurveto{\pgfqpoint{2.890844in}{1.129723in}}{\pgfqpoint{2.895234in}{1.119124in}}{\pgfqpoint{2.903048in}{1.111310in}}%
\pgfpathcurveto{\pgfqpoint{2.910861in}{1.103497in}}{\pgfqpoint{2.921460in}{1.099106in}}{\pgfqpoint{2.932510in}{1.099106in}}%
\pgfpathlineto{\pgfqpoint{2.932510in}{1.099106in}}%
\pgfpathclose%
\pgfusepath{stroke}%
\end{pgfscope}%
\begin{pgfscope}%
\pgfpathrectangle{\pgfqpoint{0.847223in}{0.554012in}}{\pgfqpoint{6.200000in}{4.620000in}}%
\pgfusepath{clip}%
\pgfsetbuttcap%
\pgfsetroundjoin%
\pgfsetlinewidth{1.003750pt}%
\definecolor{currentstroke}{rgb}{1.000000,0.000000,0.000000}%
\pgfsetstrokecolor{currentstroke}%
\pgfsetdash{}{0pt}%
\pgfpathmoveto{\pgfqpoint{2.937844in}{1.097041in}}%
\pgfpathcurveto{\pgfqpoint{2.948894in}{1.097041in}}{\pgfqpoint{2.959493in}{1.101431in}}{\pgfqpoint{2.967306in}{1.109245in}}%
\pgfpathcurveto{\pgfqpoint{2.975120in}{1.117059in}}{\pgfqpoint{2.979510in}{1.127658in}}{\pgfqpoint{2.979510in}{1.138708in}}%
\pgfpathcurveto{\pgfqpoint{2.979510in}{1.149758in}}{\pgfqpoint{2.975120in}{1.160357in}}{\pgfqpoint{2.967306in}{1.168171in}}%
\pgfpathcurveto{\pgfqpoint{2.959493in}{1.175984in}}{\pgfqpoint{2.948894in}{1.180374in}}{\pgfqpoint{2.937844in}{1.180374in}}%
\pgfpathcurveto{\pgfqpoint{2.926793in}{1.180374in}}{\pgfqpoint{2.916194in}{1.175984in}}{\pgfqpoint{2.908381in}{1.168171in}}%
\pgfpathcurveto{\pgfqpoint{2.900567in}{1.160357in}}{\pgfqpoint{2.896177in}{1.149758in}}{\pgfqpoint{2.896177in}{1.138708in}}%
\pgfpathcurveto{\pgfqpoint{2.896177in}{1.127658in}}{\pgfqpoint{2.900567in}{1.117059in}}{\pgfqpoint{2.908381in}{1.109245in}}%
\pgfpathcurveto{\pgfqpoint{2.916194in}{1.101431in}}{\pgfqpoint{2.926793in}{1.097041in}}{\pgfqpoint{2.937844in}{1.097041in}}%
\pgfpathlineto{\pgfqpoint{2.937844in}{1.097041in}}%
\pgfpathclose%
\pgfusepath{stroke}%
\end{pgfscope}%
\begin{pgfscope}%
\pgfpathrectangle{\pgfqpoint{0.847223in}{0.554012in}}{\pgfqpoint{6.200000in}{4.620000in}}%
\pgfusepath{clip}%
\pgfsetbuttcap%
\pgfsetroundjoin%
\pgfsetlinewidth{1.003750pt}%
\definecolor{currentstroke}{rgb}{1.000000,0.000000,0.000000}%
\pgfsetstrokecolor{currentstroke}%
\pgfsetdash{}{0pt}%
\pgfpathmoveto{\pgfqpoint{2.943177in}{1.094984in}}%
\pgfpathcurveto{\pgfqpoint{2.954227in}{1.094984in}}{\pgfqpoint{2.964826in}{1.099375in}}{\pgfqpoint{2.972640in}{1.107188in}}%
\pgfpathcurveto{\pgfqpoint{2.980453in}{1.115002in}}{\pgfqpoint{2.984843in}{1.125601in}}{\pgfqpoint{2.984843in}{1.136651in}}%
\pgfpathcurveto{\pgfqpoint{2.984843in}{1.147701in}}{\pgfqpoint{2.980453in}{1.158300in}}{\pgfqpoint{2.972640in}{1.166114in}}%
\pgfpathcurveto{\pgfqpoint{2.964826in}{1.173927in}}{\pgfqpoint{2.954227in}{1.178318in}}{\pgfqpoint{2.943177in}{1.178318in}}%
\pgfpathcurveto{\pgfqpoint{2.932127in}{1.178318in}}{\pgfqpoint{2.921528in}{1.173927in}}{\pgfqpoint{2.913714in}{1.166114in}}%
\pgfpathcurveto{\pgfqpoint{2.905900in}{1.158300in}}{\pgfqpoint{2.901510in}{1.147701in}}{\pgfqpoint{2.901510in}{1.136651in}}%
\pgfpathcurveto{\pgfqpoint{2.901510in}{1.125601in}}{\pgfqpoint{2.905900in}{1.115002in}}{\pgfqpoint{2.913714in}{1.107188in}}%
\pgfpathcurveto{\pgfqpoint{2.921528in}{1.099375in}}{\pgfqpoint{2.932127in}{1.094984in}}{\pgfqpoint{2.943177in}{1.094984in}}%
\pgfpathlineto{\pgfqpoint{2.943177in}{1.094984in}}%
\pgfpathclose%
\pgfusepath{stroke}%
\end{pgfscope}%
\begin{pgfscope}%
\pgfpathrectangle{\pgfqpoint{0.847223in}{0.554012in}}{\pgfqpoint{6.200000in}{4.620000in}}%
\pgfusepath{clip}%
\pgfsetbuttcap%
\pgfsetroundjoin%
\pgfsetlinewidth{1.003750pt}%
\definecolor{currentstroke}{rgb}{1.000000,0.000000,0.000000}%
\pgfsetstrokecolor{currentstroke}%
\pgfsetdash{}{0pt}%
\pgfpathmoveto{\pgfqpoint{2.948510in}{1.092936in}}%
\pgfpathcurveto{\pgfqpoint{2.959560in}{1.092936in}}{\pgfqpoint{2.970159in}{1.097326in}}{\pgfqpoint{2.977973in}{1.105140in}}%
\pgfpathcurveto{\pgfqpoint{2.985786in}{1.112953in}}{\pgfqpoint{2.990177in}{1.123552in}}{\pgfqpoint{2.990177in}{1.134603in}}%
\pgfpathcurveto{\pgfqpoint{2.990177in}{1.145653in}}{\pgfqpoint{2.985786in}{1.156252in}}{\pgfqpoint{2.977973in}{1.164065in}}%
\pgfpathcurveto{\pgfqpoint{2.970159in}{1.171879in}}{\pgfqpoint{2.959560in}{1.176269in}}{\pgfqpoint{2.948510in}{1.176269in}}%
\pgfpathcurveto{\pgfqpoint{2.937460in}{1.176269in}}{\pgfqpoint{2.926861in}{1.171879in}}{\pgfqpoint{2.919047in}{1.164065in}}%
\pgfpathcurveto{\pgfqpoint{2.911234in}{1.156252in}}{\pgfqpoint{2.906843in}{1.145653in}}{\pgfqpoint{2.906843in}{1.134603in}}%
\pgfpathcurveto{\pgfqpoint{2.906843in}{1.123552in}}{\pgfqpoint{2.911234in}{1.112953in}}{\pgfqpoint{2.919047in}{1.105140in}}%
\pgfpathcurveto{\pgfqpoint{2.926861in}{1.097326in}}{\pgfqpoint{2.937460in}{1.092936in}}{\pgfqpoint{2.948510in}{1.092936in}}%
\pgfpathlineto{\pgfqpoint{2.948510in}{1.092936in}}%
\pgfpathclose%
\pgfusepath{stroke}%
\end{pgfscope}%
\begin{pgfscope}%
\pgfpathrectangle{\pgfqpoint{0.847223in}{0.554012in}}{\pgfqpoint{6.200000in}{4.620000in}}%
\pgfusepath{clip}%
\pgfsetbuttcap%
\pgfsetroundjoin%
\pgfsetlinewidth{1.003750pt}%
\definecolor{currentstroke}{rgb}{1.000000,0.000000,0.000000}%
\pgfsetstrokecolor{currentstroke}%
\pgfsetdash{}{0pt}%
\pgfpathmoveto{\pgfqpoint{2.953843in}{1.090896in}}%
\pgfpathcurveto{\pgfqpoint{2.964893in}{1.090896in}}{\pgfqpoint{2.975492in}{1.095286in}}{\pgfqpoint{2.983306in}{1.103100in}}%
\pgfpathcurveto{\pgfqpoint{2.991120in}{1.110913in}}{\pgfqpoint{2.995510in}{1.121512in}}{\pgfqpoint{2.995510in}{1.132562in}}%
\pgfpathcurveto{\pgfqpoint{2.995510in}{1.143612in}}{\pgfqpoint{2.991120in}{1.154212in}}{\pgfqpoint{2.983306in}{1.162025in}}%
\pgfpathcurveto{\pgfqpoint{2.975492in}{1.169839in}}{\pgfqpoint{2.964893in}{1.174229in}}{\pgfqpoint{2.953843in}{1.174229in}}%
\pgfpathcurveto{\pgfqpoint{2.942793in}{1.174229in}}{\pgfqpoint{2.932194in}{1.169839in}}{\pgfqpoint{2.924380in}{1.162025in}}%
\pgfpathcurveto{\pgfqpoint{2.916567in}{1.154212in}}{\pgfqpoint{2.912177in}{1.143612in}}{\pgfqpoint{2.912177in}{1.132562in}}%
\pgfpathcurveto{\pgfqpoint{2.912177in}{1.121512in}}{\pgfqpoint{2.916567in}{1.110913in}}{\pgfqpoint{2.924380in}{1.103100in}}%
\pgfpathcurveto{\pgfqpoint{2.932194in}{1.095286in}}{\pgfqpoint{2.942793in}{1.090896in}}{\pgfqpoint{2.953843in}{1.090896in}}%
\pgfpathlineto{\pgfqpoint{2.953843in}{1.090896in}}%
\pgfpathclose%
\pgfusepath{stroke}%
\end{pgfscope}%
\begin{pgfscope}%
\pgfpathrectangle{\pgfqpoint{0.847223in}{0.554012in}}{\pgfqpoint{6.200000in}{4.620000in}}%
\pgfusepath{clip}%
\pgfsetbuttcap%
\pgfsetroundjoin%
\pgfsetlinewidth{1.003750pt}%
\definecolor{currentstroke}{rgb}{1.000000,0.000000,0.000000}%
\pgfsetstrokecolor{currentstroke}%
\pgfsetdash{}{0pt}%
\pgfpathmoveto{\pgfqpoint{2.959176in}{1.088864in}}%
\pgfpathcurveto{\pgfqpoint{2.970227in}{1.088864in}}{\pgfqpoint{2.980826in}{1.093254in}}{\pgfqpoint{2.988639in}{1.101068in}}%
\pgfpathcurveto{\pgfqpoint{2.996453in}{1.108881in}}{\pgfqpoint{3.000843in}{1.119480in}}{\pgfqpoint{3.000843in}{1.130530in}}%
\pgfpathcurveto{\pgfqpoint{3.000843in}{1.141581in}}{\pgfqpoint{2.996453in}{1.152180in}}{\pgfqpoint{2.988639in}{1.159993in}}%
\pgfpathcurveto{\pgfqpoint{2.980826in}{1.167807in}}{\pgfqpoint{2.970227in}{1.172197in}}{\pgfqpoint{2.959176in}{1.172197in}}%
\pgfpathcurveto{\pgfqpoint{2.948126in}{1.172197in}}{\pgfqpoint{2.937527in}{1.167807in}}{\pgfqpoint{2.929714in}{1.159993in}}%
\pgfpathcurveto{\pgfqpoint{2.921900in}{1.152180in}}{\pgfqpoint{2.917510in}{1.141581in}}{\pgfqpoint{2.917510in}{1.130530in}}%
\pgfpathcurveto{\pgfqpoint{2.917510in}{1.119480in}}{\pgfqpoint{2.921900in}{1.108881in}}{\pgfqpoint{2.929714in}{1.101068in}}%
\pgfpathcurveto{\pgfqpoint{2.937527in}{1.093254in}}{\pgfqpoint{2.948126in}{1.088864in}}{\pgfqpoint{2.959176in}{1.088864in}}%
\pgfpathlineto{\pgfqpoint{2.959176in}{1.088864in}}%
\pgfpathclose%
\pgfusepath{stroke}%
\end{pgfscope}%
\begin{pgfscope}%
\pgfpathrectangle{\pgfqpoint{0.847223in}{0.554012in}}{\pgfqpoint{6.200000in}{4.620000in}}%
\pgfusepath{clip}%
\pgfsetbuttcap%
\pgfsetroundjoin%
\pgfsetlinewidth{1.003750pt}%
\definecolor{currentstroke}{rgb}{1.000000,0.000000,0.000000}%
\pgfsetstrokecolor{currentstroke}%
\pgfsetdash{}{0pt}%
\pgfpathmoveto{\pgfqpoint{2.964510in}{1.086840in}}%
\pgfpathcurveto{\pgfqpoint{2.975560in}{1.086840in}}{\pgfqpoint{2.986159in}{1.091230in}}{\pgfqpoint{2.993972in}{1.099044in}}%
\pgfpathcurveto{\pgfqpoint{3.001786in}{1.106858in}}{\pgfqpoint{3.006176in}{1.117457in}}{\pgfqpoint{3.006176in}{1.128507in}}%
\pgfpathcurveto{\pgfqpoint{3.006176in}{1.139557in}}{\pgfqpoint{3.001786in}{1.150156in}}{\pgfqpoint{2.993972in}{1.157970in}}%
\pgfpathcurveto{\pgfqpoint{2.986159in}{1.165783in}}{\pgfqpoint{2.975560in}{1.170173in}}{\pgfqpoint{2.964510in}{1.170173in}}%
\pgfpathcurveto{\pgfqpoint{2.953460in}{1.170173in}}{\pgfqpoint{2.942861in}{1.165783in}}{\pgfqpoint{2.935047in}{1.157970in}}%
\pgfpathcurveto{\pgfqpoint{2.927233in}{1.150156in}}{\pgfqpoint{2.922843in}{1.139557in}}{\pgfqpoint{2.922843in}{1.128507in}}%
\pgfpathcurveto{\pgfqpoint{2.922843in}{1.117457in}}{\pgfqpoint{2.927233in}{1.106858in}}{\pgfqpoint{2.935047in}{1.099044in}}%
\pgfpathcurveto{\pgfqpoint{2.942861in}{1.091230in}}{\pgfqpoint{2.953460in}{1.086840in}}{\pgfqpoint{2.964510in}{1.086840in}}%
\pgfpathlineto{\pgfqpoint{2.964510in}{1.086840in}}%
\pgfpathclose%
\pgfusepath{stroke}%
\end{pgfscope}%
\begin{pgfscope}%
\pgfpathrectangle{\pgfqpoint{0.847223in}{0.554012in}}{\pgfqpoint{6.200000in}{4.620000in}}%
\pgfusepath{clip}%
\pgfsetbuttcap%
\pgfsetroundjoin%
\pgfsetlinewidth{1.003750pt}%
\definecolor{currentstroke}{rgb}{1.000000,0.000000,0.000000}%
\pgfsetstrokecolor{currentstroke}%
\pgfsetdash{}{0pt}%
\pgfpathmoveto{\pgfqpoint{2.969843in}{1.084825in}}%
\pgfpathcurveto{\pgfqpoint{2.980893in}{1.084825in}}{\pgfqpoint{2.991492in}{1.089215in}}{\pgfqpoint{2.999306in}{1.097028in}}%
\pgfpathcurveto{\pgfqpoint{3.007119in}{1.104842in}}{\pgfqpoint{3.011510in}{1.115441in}}{\pgfqpoint{3.011510in}{1.126491in}}%
\pgfpathcurveto{\pgfqpoint{3.011510in}{1.137541in}}{\pgfqpoint{3.007119in}{1.148140in}}{\pgfqpoint{2.999306in}{1.155954in}}%
\pgfpathcurveto{\pgfqpoint{2.991492in}{1.163768in}}{\pgfqpoint{2.980893in}{1.168158in}}{\pgfqpoint{2.969843in}{1.168158in}}%
\pgfpathcurveto{\pgfqpoint{2.958793in}{1.168158in}}{\pgfqpoint{2.948194in}{1.163768in}}{\pgfqpoint{2.940380in}{1.155954in}}%
\pgfpathcurveto{\pgfqpoint{2.932566in}{1.148140in}}{\pgfqpoint{2.928176in}{1.137541in}}{\pgfqpoint{2.928176in}{1.126491in}}%
\pgfpathcurveto{\pgfqpoint{2.928176in}{1.115441in}}{\pgfqpoint{2.932566in}{1.104842in}}{\pgfqpoint{2.940380in}{1.097028in}}%
\pgfpathcurveto{\pgfqpoint{2.948194in}{1.089215in}}{\pgfqpoint{2.958793in}{1.084825in}}{\pgfqpoint{2.969843in}{1.084825in}}%
\pgfpathlineto{\pgfqpoint{2.969843in}{1.084825in}}%
\pgfpathclose%
\pgfusepath{stroke}%
\end{pgfscope}%
\begin{pgfscope}%
\pgfpathrectangle{\pgfqpoint{0.847223in}{0.554012in}}{\pgfqpoint{6.200000in}{4.620000in}}%
\pgfusepath{clip}%
\pgfsetbuttcap%
\pgfsetroundjoin%
\pgfsetlinewidth{1.003750pt}%
\definecolor{currentstroke}{rgb}{1.000000,0.000000,0.000000}%
\pgfsetstrokecolor{currentstroke}%
\pgfsetdash{}{0pt}%
\pgfpathmoveto{\pgfqpoint{2.975176in}{1.082817in}}%
\pgfpathcurveto{\pgfqpoint{2.986226in}{1.082817in}}{\pgfqpoint{2.996825in}{1.087207in}}{\pgfqpoint{3.004639in}{1.095021in}}%
\pgfpathcurveto{\pgfqpoint{3.012453in}{1.102835in}}{\pgfqpoint{3.016843in}{1.113434in}}{\pgfqpoint{3.016843in}{1.124484in}}%
\pgfpathcurveto{\pgfqpoint{3.016843in}{1.135534in}}{\pgfqpoint{3.012453in}{1.146133in}}{\pgfqpoint{3.004639in}{1.153947in}}%
\pgfpathcurveto{\pgfqpoint{2.996825in}{1.161760in}}{\pgfqpoint{2.986226in}{1.166150in}}{\pgfqpoint{2.975176in}{1.166150in}}%
\pgfpathcurveto{\pgfqpoint{2.964126in}{1.166150in}}{\pgfqpoint{2.953527in}{1.161760in}}{\pgfqpoint{2.945713in}{1.153947in}}%
\pgfpathcurveto{\pgfqpoint{2.937900in}{1.146133in}}{\pgfqpoint{2.933509in}{1.135534in}}{\pgfqpoint{2.933509in}{1.124484in}}%
\pgfpathcurveto{\pgfqpoint{2.933509in}{1.113434in}}{\pgfqpoint{2.937900in}{1.102835in}}{\pgfqpoint{2.945713in}{1.095021in}}%
\pgfpathcurveto{\pgfqpoint{2.953527in}{1.087207in}}{\pgfqpoint{2.964126in}{1.082817in}}{\pgfqpoint{2.975176in}{1.082817in}}%
\pgfpathlineto{\pgfqpoint{2.975176in}{1.082817in}}%
\pgfpathclose%
\pgfusepath{stroke}%
\end{pgfscope}%
\begin{pgfscope}%
\pgfpathrectangle{\pgfqpoint{0.847223in}{0.554012in}}{\pgfqpoint{6.200000in}{4.620000in}}%
\pgfusepath{clip}%
\pgfsetbuttcap%
\pgfsetroundjoin%
\pgfsetlinewidth{1.003750pt}%
\definecolor{currentstroke}{rgb}{1.000000,0.000000,0.000000}%
\pgfsetstrokecolor{currentstroke}%
\pgfsetdash{}{0pt}%
\pgfpathmoveto{\pgfqpoint{2.980509in}{1.080818in}}%
\pgfpathcurveto{\pgfqpoint{2.991559in}{1.080818in}}{\pgfqpoint{3.002158in}{1.085208in}}{\pgfqpoint{3.009972in}{1.093022in}}%
\pgfpathcurveto{\pgfqpoint{3.017786in}{1.100835in}}{\pgfqpoint{3.022176in}{1.111434in}}{\pgfqpoint{3.022176in}{1.122484in}}%
\pgfpathcurveto{\pgfqpoint{3.022176in}{1.133535in}}{\pgfqpoint{3.017786in}{1.144134in}}{\pgfqpoint{3.009972in}{1.151947in}}%
\pgfpathcurveto{\pgfqpoint{3.002158in}{1.159761in}}{\pgfqpoint{2.991559in}{1.164151in}}{\pgfqpoint{2.980509in}{1.164151in}}%
\pgfpathcurveto{\pgfqpoint{2.969459in}{1.164151in}}{\pgfqpoint{2.958860in}{1.159761in}}{\pgfqpoint{2.951047in}{1.151947in}}%
\pgfpathcurveto{\pgfqpoint{2.943233in}{1.144134in}}{\pgfqpoint{2.938843in}{1.133535in}}{\pgfqpoint{2.938843in}{1.122484in}}%
\pgfpathcurveto{\pgfqpoint{2.938843in}{1.111434in}}{\pgfqpoint{2.943233in}{1.100835in}}{\pgfqpoint{2.951047in}{1.093022in}}%
\pgfpathcurveto{\pgfqpoint{2.958860in}{1.085208in}}{\pgfqpoint{2.969459in}{1.080818in}}{\pgfqpoint{2.980509in}{1.080818in}}%
\pgfpathlineto{\pgfqpoint{2.980509in}{1.080818in}}%
\pgfpathclose%
\pgfusepath{stroke}%
\end{pgfscope}%
\begin{pgfscope}%
\pgfpathrectangle{\pgfqpoint{0.847223in}{0.554012in}}{\pgfqpoint{6.200000in}{4.620000in}}%
\pgfusepath{clip}%
\pgfsetbuttcap%
\pgfsetroundjoin%
\pgfsetlinewidth{1.003750pt}%
\definecolor{currentstroke}{rgb}{1.000000,0.000000,0.000000}%
\pgfsetstrokecolor{currentstroke}%
\pgfsetdash{}{0pt}%
\pgfpathmoveto{\pgfqpoint{2.985843in}{1.078826in}}%
\pgfpathcurveto{\pgfqpoint{2.996893in}{1.078826in}}{\pgfqpoint{3.007492in}{1.083217in}}{\pgfqpoint{3.015305in}{1.091030in}}%
\pgfpathcurveto{\pgfqpoint{3.023119in}{1.098844in}}{\pgfqpoint{3.027509in}{1.109443in}}{\pgfqpoint{3.027509in}{1.120493in}}%
\pgfpathcurveto{\pgfqpoint{3.027509in}{1.131543in}}{\pgfqpoint{3.023119in}{1.142142in}}{\pgfqpoint{3.015305in}{1.149956in}}%
\pgfpathcurveto{\pgfqpoint{3.007492in}{1.157770in}}{\pgfqpoint{2.996893in}{1.162160in}}{\pgfqpoint{2.985843in}{1.162160in}}%
\pgfpathcurveto{\pgfqpoint{2.974792in}{1.162160in}}{\pgfqpoint{2.964193in}{1.157770in}}{\pgfqpoint{2.956380in}{1.149956in}}%
\pgfpathcurveto{\pgfqpoint{2.948566in}{1.142142in}}{\pgfqpoint{2.944176in}{1.131543in}}{\pgfqpoint{2.944176in}{1.120493in}}%
\pgfpathcurveto{\pgfqpoint{2.944176in}{1.109443in}}{\pgfqpoint{2.948566in}{1.098844in}}{\pgfqpoint{2.956380in}{1.091030in}}%
\pgfpathcurveto{\pgfqpoint{2.964193in}{1.083217in}}{\pgfqpoint{2.974792in}{1.078826in}}{\pgfqpoint{2.985843in}{1.078826in}}%
\pgfpathlineto{\pgfqpoint{2.985843in}{1.078826in}}%
\pgfpathclose%
\pgfusepath{stroke}%
\end{pgfscope}%
\begin{pgfscope}%
\pgfpathrectangle{\pgfqpoint{0.847223in}{0.554012in}}{\pgfqpoint{6.200000in}{4.620000in}}%
\pgfusepath{clip}%
\pgfsetbuttcap%
\pgfsetroundjoin%
\pgfsetlinewidth{1.003750pt}%
\definecolor{currentstroke}{rgb}{1.000000,0.000000,0.000000}%
\pgfsetstrokecolor{currentstroke}%
\pgfsetdash{}{0pt}%
\pgfpathmoveto{\pgfqpoint{2.991176in}{1.076843in}}%
\pgfpathcurveto{\pgfqpoint{3.002226in}{1.076843in}}{\pgfqpoint{3.012825in}{1.081233in}}{\pgfqpoint{3.020639in}{1.089047in}}%
\pgfpathcurveto{\pgfqpoint{3.028452in}{1.096861in}}{\pgfqpoint{3.032842in}{1.107460in}}{\pgfqpoint{3.032842in}{1.118510in}}%
\pgfpathcurveto{\pgfqpoint{3.032842in}{1.129560in}}{\pgfqpoint{3.028452in}{1.140159in}}{\pgfqpoint{3.020639in}{1.147973in}}%
\pgfpathcurveto{\pgfqpoint{3.012825in}{1.155786in}}{\pgfqpoint{3.002226in}{1.160176in}}{\pgfqpoint{2.991176in}{1.160176in}}%
\pgfpathcurveto{\pgfqpoint{2.980126in}{1.160176in}}{\pgfqpoint{2.969527in}{1.155786in}}{\pgfqpoint{2.961713in}{1.147973in}}%
\pgfpathcurveto{\pgfqpoint{2.953899in}{1.140159in}}{\pgfqpoint{2.949509in}{1.129560in}}{\pgfqpoint{2.949509in}{1.118510in}}%
\pgfpathcurveto{\pgfqpoint{2.949509in}{1.107460in}}{\pgfqpoint{2.953899in}{1.096861in}}{\pgfqpoint{2.961713in}{1.089047in}}%
\pgfpathcurveto{\pgfqpoint{2.969527in}{1.081233in}}{\pgfqpoint{2.980126in}{1.076843in}}{\pgfqpoint{2.991176in}{1.076843in}}%
\pgfpathlineto{\pgfqpoint{2.991176in}{1.076843in}}%
\pgfpathclose%
\pgfusepath{stroke}%
\end{pgfscope}%
\begin{pgfscope}%
\pgfpathrectangle{\pgfqpoint{0.847223in}{0.554012in}}{\pgfqpoint{6.200000in}{4.620000in}}%
\pgfusepath{clip}%
\pgfsetbuttcap%
\pgfsetroundjoin%
\pgfsetlinewidth{1.003750pt}%
\definecolor{currentstroke}{rgb}{1.000000,0.000000,0.000000}%
\pgfsetstrokecolor{currentstroke}%
\pgfsetdash{}{0pt}%
\pgfpathmoveto{\pgfqpoint{2.996509in}{1.074868in}}%
\pgfpathcurveto{\pgfqpoint{3.007559in}{1.074868in}}{\pgfqpoint{3.018158in}{1.079258in}}{\pgfqpoint{3.025972in}{1.087071in}}%
\pgfpathcurveto{\pgfqpoint{3.033785in}{1.094885in}}{\pgfqpoint{3.038176in}{1.105484in}}{\pgfqpoint{3.038176in}{1.116534in}}%
\pgfpathcurveto{\pgfqpoint{3.038176in}{1.127584in}}{\pgfqpoint{3.033785in}{1.138183in}}{\pgfqpoint{3.025972in}{1.145997in}}%
\pgfpathcurveto{\pgfqpoint{3.018158in}{1.153811in}}{\pgfqpoint{3.007559in}{1.158201in}}{\pgfqpoint{2.996509in}{1.158201in}}%
\pgfpathcurveto{\pgfqpoint{2.985459in}{1.158201in}}{\pgfqpoint{2.974860in}{1.153811in}}{\pgfqpoint{2.967046in}{1.145997in}}%
\pgfpathcurveto{\pgfqpoint{2.959233in}{1.138183in}}{\pgfqpoint{2.954842in}{1.127584in}}{\pgfqpoint{2.954842in}{1.116534in}}%
\pgfpathcurveto{\pgfqpoint{2.954842in}{1.105484in}}{\pgfqpoint{2.959233in}{1.094885in}}{\pgfqpoint{2.967046in}{1.087071in}}%
\pgfpathcurveto{\pgfqpoint{2.974860in}{1.079258in}}{\pgfqpoint{2.985459in}{1.074868in}}{\pgfqpoint{2.996509in}{1.074868in}}%
\pgfpathlineto{\pgfqpoint{2.996509in}{1.074868in}}%
\pgfpathclose%
\pgfusepath{stroke}%
\end{pgfscope}%
\begin{pgfscope}%
\pgfpathrectangle{\pgfqpoint{0.847223in}{0.554012in}}{\pgfqpoint{6.200000in}{4.620000in}}%
\pgfusepath{clip}%
\pgfsetbuttcap%
\pgfsetroundjoin%
\pgfsetlinewidth{1.003750pt}%
\definecolor{currentstroke}{rgb}{1.000000,0.000000,0.000000}%
\pgfsetstrokecolor{currentstroke}%
\pgfsetdash{}{0pt}%
\pgfpathmoveto{\pgfqpoint{3.001842in}{1.072900in}}%
\pgfpathcurveto{\pgfqpoint{3.012892in}{1.072900in}}{\pgfqpoint{3.023491in}{1.077290in}}{\pgfqpoint{3.031305in}{1.085104in}}%
\pgfpathcurveto{\pgfqpoint{3.039119in}{1.092917in}}{\pgfqpoint{3.043509in}{1.103517in}}{\pgfqpoint{3.043509in}{1.114567in}}%
\pgfpathcurveto{\pgfqpoint{3.043509in}{1.125617in}}{\pgfqpoint{3.039119in}{1.136216in}}{\pgfqpoint{3.031305in}{1.144029in}}%
\pgfpathcurveto{\pgfqpoint{3.023491in}{1.151843in}}{\pgfqpoint{3.012892in}{1.156233in}}{\pgfqpoint{3.001842in}{1.156233in}}%
\pgfpathcurveto{\pgfqpoint{2.990792in}{1.156233in}}{\pgfqpoint{2.980193in}{1.151843in}}{\pgfqpoint{2.972379in}{1.144029in}}%
\pgfpathcurveto{\pgfqpoint{2.964566in}{1.136216in}}{\pgfqpoint{2.960176in}{1.125617in}}{\pgfqpoint{2.960176in}{1.114567in}}%
\pgfpathcurveto{\pgfqpoint{2.960176in}{1.103517in}}{\pgfqpoint{2.964566in}{1.092917in}}{\pgfqpoint{2.972379in}{1.085104in}}%
\pgfpathcurveto{\pgfqpoint{2.980193in}{1.077290in}}{\pgfqpoint{2.990792in}{1.072900in}}{\pgfqpoint{3.001842in}{1.072900in}}%
\pgfpathlineto{\pgfqpoint{3.001842in}{1.072900in}}%
\pgfpathclose%
\pgfusepath{stroke}%
\end{pgfscope}%
\begin{pgfscope}%
\pgfpathrectangle{\pgfqpoint{0.847223in}{0.554012in}}{\pgfqpoint{6.200000in}{4.620000in}}%
\pgfusepath{clip}%
\pgfsetbuttcap%
\pgfsetroundjoin%
\pgfsetlinewidth{1.003750pt}%
\definecolor{currentstroke}{rgb}{1.000000,0.000000,0.000000}%
\pgfsetstrokecolor{currentstroke}%
\pgfsetdash{}{0pt}%
\pgfpathmoveto{\pgfqpoint{3.007175in}{1.070940in}}%
\pgfpathcurveto{\pgfqpoint{3.018226in}{1.070940in}}{\pgfqpoint{3.028825in}{1.075330in}}{\pgfqpoint{3.036638in}{1.083144in}}%
\pgfpathcurveto{\pgfqpoint{3.044452in}{1.090958in}}{\pgfqpoint{3.048842in}{1.101557in}}{\pgfqpoint{3.048842in}{1.112607in}}%
\pgfpathcurveto{\pgfqpoint{3.048842in}{1.123657in}}{\pgfqpoint{3.044452in}{1.134256in}}{\pgfqpoint{3.036638in}{1.142070in}}%
\pgfpathcurveto{\pgfqpoint{3.028825in}{1.149883in}}{\pgfqpoint{3.018226in}{1.154274in}}{\pgfqpoint{3.007175in}{1.154274in}}%
\pgfpathcurveto{\pgfqpoint{2.996125in}{1.154274in}}{\pgfqpoint{2.985526in}{1.149883in}}{\pgfqpoint{2.977713in}{1.142070in}}%
\pgfpathcurveto{\pgfqpoint{2.969899in}{1.134256in}}{\pgfqpoint{2.965509in}{1.123657in}}{\pgfqpoint{2.965509in}{1.112607in}}%
\pgfpathcurveto{\pgfqpoint{2.965509in}{1.101557in}}{\pgfqpoint{2.969899in}{1.090958in}}{\pgfqpoint{2.977713in}{1.083144in}}%
\pgfpathcurveto{\pgfqpoint{2.985526in}{1.075330in}}{\pgfqpoint{2.996125in}{1.070940in}}{\pgfqpoint{3.007175in}{1.070940in}}%
\pgfpathlineto{\pgfqpoint{3.007175in}{1.070940in}}%
\pgfpathclose%
\pgfusepath{stroke}%
\end{pgfscope}%
\begin{pgfscope}%
\pgfpathrectangle{\pgfqpoint{0.847223in}{0.554012in}}{\pgfqpoint{6.200000in}{4.620000in}}%
\pgfusepath{clip}%
\pgfsetbuttcap%
\pgfsetroundjoin%
\pgfsetlinewidth{1.003750pt}%
\definecolor{currentstroke}{rgb}{1.000000,0.000000,0.000000}%
\pgfsetstrokecolor{currentstroke}%
\pgfsetdash{}{0pt}%
\pgfpathmoveto{\pgfqpoint{3.012509in}{1.068988in}}%
\pgfpathcurveto{\pgfqpoint{3.023559in}{1.068988in}}{\pgfqpoint{3.034158in}{1.073378in}}{\pgfqpoint{3.041971in}{1.081192in}}%
\pgfpathcurveto{\pgfqpoint{3.049785in}{1.089006in}}{\pgfqpoint{3.054175in}{1.099605in}}{\pgfqpoint{3.054175in}{1.110655in}}%
\pgfpathcurveto{\pgfqpoint{3.054175in}{1.121705in}}{\pgfqpoint{3.049785in}{1.132304in}}{\pgfqpoint{3.041971in}{1.140118in}}%
\pgfpathcurveto{\pgfqpoint{3.034158in}{1.147931in}}{\pgfqpoint{3.023559in}{1.152322in}}{\pgfqpoint{3.012509in}{1.152322in}}%
\pgfpathcurveto{\pgfqpoint{3.001458in}{1.152322in}}{\pgfqpoint{2.990859in}{1.147931in}}{\pgfqpoint{2.983046in}{1.140118in}}%
\pgfpathcurveto{\pgfqpoint{2.975232in}{1.132304in}}{\pgfqpoint{2.970842in}{1.121705in}}{\pgfqpoint{2.970842in}{1.110655in}}%
\pgfpathcurveto{\pgfqpoint{2.970842in}{1.099605in}}{\pgfqpoint{2.975232in}{1.089006in}}{\pgfqpoint{2.983046in}{1.081192in}}%
\pgfpathcurveto{\pgfqpoint{2.990859in}{1.073378in}}{\pgfqpoint{3.001458in}{1.068988in}}{\pgfqpoint{3.012509in}{1.068988in}}%
\pgfpathlineto{\pgfqpoint{3.012509in}{1.068988in}}%
\pgfpathclose%
\pgfusepath{stroke}%
\end{pgfscope}%
\begin{pgfscope}%
\pgfpathrectangle{\pgfqpoint{0.847223in}{0.554012in}}{\pgfqpoint{6.200000in}{4.620000in}}%
\pgfusepath{clip}%
\pgfsetbuttcap%
\pgfsetroundjoin%
\pgfsetlinewidth{1.003750pt}%
\definecolor{currentstroke}{rgb}{1.000000,0.000000,0.000000}%
\pgfsetstrokecolor{currentstroke}%
\pgfsetdash{}{0pt}%
\pgfpathmoveto{\pgfqpoint{3.017842in}{1.067044in}}%
\pgfpathcurveto{\pgfqpoint{3.028892in}{1.067044in}}{\pgfqpoint{3.039491in}{1.071434in}}{\pgfqpoint{3.047305in}{1.079248in}}%
\pgfpathcurveto{\pgfqpoint{3.055118in}{1.087061in}}{\pgfqpoint{3.059508in}{1.097660in}}{\pgfqpoint{3.059508in}{1.108711in}}%
\pgfpathcurveto{\pgfqpoint{3.059508in}{1.119761in}}{\pgfqpoint{3.055118in}{1.130360in}}{\pgfqpoint{3.047305in}{1.138173in}}%
\pgfpathcurveto{\pgfqpoint{3.039491in}{1.145987in}}{\pgfqpoint{3.028892in}{1.150377in}}{\pgfqpoint{3.017842in}{1.150377in}}%
\pgfpathcurveto{\pgfqpoint{3.006792in}{1.150377in}}{\pgfqpoint{2.996193in}{1.145987in}}{\pgfqpoint{2.988379in}{1.138173in}}%
\pgfpathcurveto{\pgfqpoint{2.980565in}{1.130360in}}{\pgfqpoint{2.976175in}{1.119761in}}{\pgfqpoint{2.976175in}{1.108711in}}%
\pgfpathcurveto{\pgfqpoint{2.976175in}{1.097660in}}{\pgfqpoint{2.980565in}{1.087061in}}{\pgfqpoint{2.988379in}{1.079248in}}%
\pgfpathcurveto{\pgfqpoint{2.996193in}{1.071434in}}{\pgfqpoint{3.006792in}{1.067044in}}{\pgfqpoint{3.017842in}{1.067044in}}%
\pgfpathlineto{\pgfqpoint{3.017842in}{1.067044in}}%
\pgfpathclose%
\pgfusepath{stroke}%
\end{pgfscope}%
\begin{pgfscope}%
\pgfpathrectangle{\pgfqpoint{0.847223in}{0.554012in}}{\pgfqpoint{6.200000in}{4.620000in}}%
\pgfusepath{clip}%
\pgfsetbuttcap%
\pgfsetroundjoin%
\pgfsetlinewidth{1.003750pt}%
\definecolor{currentstroke}{rgb}{1.000000,0.000000,0.000000}%
\pgfsetstrokecolor{currentstroke}%
\pgfsetdash{}{0pt}%
\pgfpathmoveto{\pgfqpoint{3.023175in}{1.065107in}}%
\pgfpathcurveto{\pgfqpoint{3.034225in}{1.065107in}}{\pgfqpoint{3.044824in}{1.069498in}}{\pgfqpoint{3.052638in}{1.077311in}}%
\pgfpathcurveto{\pgfqpoint{3.060451in}{1.085125in}}{\pgfqpoint{3.064842in}{1.095724in}}{\pgfqpoint{3.064842in}{1.106774in}}%
\pgfpathcurveto{\pgfqpoint{3.064842in}{1.117824in}}{\pgfqpoint{3.060451in}{1.128423in}}{\pgfqpoint{3.052638in}{1.136237in}}%
\pgfpathcurveto{\pgfqpoint{3.044824in}{1.144050in}}{\pgfqpoint{3.034225in}{1.148441in}}{\pgfqpoint{3.023175in}{1.148441in}}%
\pgfpathcurveto{\pgfqpoint{3.012125in}{1.148441in}}{\pgfqpoint{3.001526in}{1.144050in}}{\pgfqpoint{2.993712in}{1.136237in}}%
\pgfpathcurveto{\pgfqpoint{2.985899in}{1.128423in}}{\pgfqpoint{2.981508in}{1.117824in}}{\pgfqpoint{2.981508in}{1.106774in}}%
\pgfpathcurveto{\pgfqpoint{2.981508in}{1.095724in}}{\pgfqpoint{2.985899in}{1.085125in}}{\pgfqpoint{2.993712in}{1.077311in}}%
\pgfpathcurveto{\pgfqpoint{3.001526in}{1.069498in}}{\pgfqpoint{3.012125in}{1.065107in}}{\pgfqpoint{3.023175in}{1.065107in}}%
\pgfpathlineto{\pgfqpoint{3.023175in}{1.065107in}}%
\pgfpathclose%
\pgfusepath{stroke}%
\end{pgfscope}%
\begin{pgfscope}%
\pgfpathrectangle{\pgfqpoint{0.847223in}{0.554012in}}{\pgfqpoint{6.200000in}{4.620000in}}%
\pgfusepath{clip}%
\pgfsetbuttcap%
\pgfsetroundjoin%
\pgfsetlinewidth{1.003750pt}%
\definecolor{currentstroke}{rgb}{1.000000,0.000000,0.000000}%
\pgfsetstrokecolor{currentstroke}%
\pgfsetdash{}{0pt}%
\pgfpathmoveto{\pgfqpoint{3.028508in}{1.063178in}}%
\pgfpathcurveto{\pgfqpoint{3.039558in}{1.063178in}}{\pgfqpoint{3.050157in}{1.067569in}}{\pgfqpoint{3.057971in}{1.075382in}}%
\pgfpathcurveto{\pgfqpoint{3.065785in}{1.083196in}}{\pgfqpoint{3.070175in}{1.093795in}}{\pgfqpoint{3.070175in}{1.104845in}}%
\pgfpathcurveto{\pgfqpoint{3.070175in}{1.115895in}}{\pgfqpoint{3.065785in}{1.126494in}}{\pgfqpoint{3.057971in}{1.134308in}}%
\pgfpathcurveto{\pgfqpoint{3.050157in}{1.142121in}}{\pgfqpoint{3.039558in}{1.146512in}}{\pgfqpoint{3.028508in}{1.146512in}}%
\pgfpathcurveto{\pgfqpoint{3.017458in}{1.146512in}}{\pgfqpoint{3.006859in}{1.142121in}}{\pgfqpoint{2.999045in}{1.134308in}}%
\pgfpathcurveto{\pgfqpoint{2.991232in}{1.126494in}}{\pgfqpoint{2.986842in}{1.115895in}}{\pgfqpoint{2.986842in}{1.104845in}}%
\pgfpathcurveto{\pgfqpoint{2.986842in}{1.093795in}}{\pgfqpoint{2.991232in}{1.083196in}}{\pgfqpoint{2.999045in}{1.075382in}}%
\pgfpathcurveto{\pgfqpoint{3.006859in}{1.067569in}}{\pgfqpoint{3.017458in}{1.063178in}}{\pgfqpoint{3.028508in}{1.063178in}}%
\pgfpathlineto{\pgfqpoint{3.028508in}{1.063178in}}%
\pgfpathclose%
\pgfusepath{stroke}%
\end{pgfscope}%
\begin{pgfscope}%
\pgfpathrectangle{\pgfqpoint{0.847223in}{0.554012in}}{\pgfqpoint{6.200000in}{4.620000in}}%
\pgfusepath{clip}%
\pgfsetbuttcap%
\pgfsetroundjoin%
\pgfsetlinewidth{1.003750pt}%
\definecolor{currentstroke}{rgb}{1.000000,0.000000,0.000000}%
\pgfsetstrokecolor{currentstroke}%
\pgfsetdash{}{0pt}%
\pgfpathmoveto{\pgfqpoint{3.033841in}{1.061257in}}%
\pgfpathcurveto{\pgfqpoint{3.044892in}{1.061257in}}{\pgfqpoint{3.055491in}{1.065647in}}{\pgfqpoint{3.063304in}{1.073461in}}%
\pgfpathcurveto{\pgfqpoint{3.071118in}{1.081274in}}{\pgfqpoint{3.075508in}{1.091874in}}{\pgfqpoint{3.075508in}{1.102924in}}%
\pgfpathcurveto{\pgfqpoint{3.075508in}{1.113974in}}{\pgfqpoint{3.071118in}{1.124573in}}{\pgfqpoint{3.063304in}{1.132386in}}%
\pgfpathcurveto{\pgfqpoint{3.055491in}{1.140200in}}{\pgfqpoint{3.044892in}{1.144590in}}{\pgfqpoint{3.033841in}{1.144590in}}%
\pgfpathcurveto{\pgfqpoint{3.022791in}{1.144590in}}{\pgfqpoint{3.012192in}{1.140200in}}{\pgfqpoint{3.004379in}{1.132386in}}%
\pgfpathcurveto{\pgfqpoint{2.996565in}{1.124573in}}{\pgfqpoint{2.992175in}{1.113974in}}{\pgfqpoint{2.992175in}{1.102924in}}%
\pgfpathcurveto{\pgfqpoint{2.992175in}{1.091874in}}{\pgfqpoint{2.996565in}{1.081274in}}{\pgfqpoint{3.004379in}{1.073461in}}%
\pgfpathcurveto{\pgfqpoint{3.012192in}{1.065647in}}{\pgfqpoint{3.022791in}{1.061257in}}{\pgfqpoint{3.033841in}{1.061257in}}%
\pgfpathlineto{\pgfqpoint{3.033841in}{1.061257in}}%
\pgfpathclose%
\pgfusepath{stroke}%
\end{pgfscope}%
\begin{pgfscope}%
\pgfpathrectangle{\pgfqpoint{0.847223in}{0.554012in}}{\pgfqpoint{6.200000in}{4.620000in}}%
\pgfusepath{clip}%
\pgfsetbuttcap%
\pgfsetroundjoin%
\pgfsetlinewidth{1.003750pt}%
\definecolor{currentstroke}{rgb}{1.000000,0.000000,0.000000}%
\pgfsetstrokecolor{currentstroke}%
\pgfsetdash{}{0pt}%
\pgfpathmoveto{\pgfqpoint{3.039175in}{1.059343in}}%
\pgfpathcurveto{\pgfqpoint{3.050225in}{1.059343in}}{\pgfqpoint{3.060824in}{1.063733in}}{\pgfqpoint{3.068637in}{1.071547in}}%
\pgfpathcurveto{\pgfqpoint{3.076451in}{1.079361in}}{\pgfqpoint{3.080841in}{1.089960in}}{\pgfqpoint{3.080841in}{1.101010in}}%
\pgfpathcurveto{\pgfqpoint{3.080841in}{1.112060in}}{\pgfqpoint{3.076451in}{1.122659in}}{\pgfqpoint{3.068637in}{1.130473in}}%
\pgfpathcurveto{\pgfqpoint{3.060824in}{1.138286in}}{\pgfqpoint{3.050225in}{1.142676in}}{\pgfqpoint{3.039175in}{1.142676in}}%
\pgfpathcurveto{\pgfqpoint{3.028125in}{1.142676in}}{\pgfqpoint{3.017526in}{1.138286in}}{\pgfqpoint{3.009712in}{1.130473in}}%
\pgfpathcurveto{\pgfqpoint{3.001898in}{1.122659in}}{\pgfqpoint{2.997508in}{1.112060in}}{\pgfqpoint{2.997508in}{1.101010in}}%
\pgfpathcurveto{\pgfqpoint{2.997508in}{1.089960in}}{\pgfqpoint{3.001898in}{1.079361in}}{\pgfqpoint{3.009712in}{1.071547in}}%
\pgfpathcurveto{\pgfqpoint{3.017526in}{1.063733in}}{\pgfqpoint{3.028125in}{1.059343in}}{\pgfqpoint{3.039175in}{1.059343in}}%
\pgfpathlineto{\pgfqpoint{3.039175in}{1.059343in}}%
\pgfpathclose%
\pgfusepath{stroke}%
\end{pgfscope}%
\begin{pgfscope}%
\pgfpathrectangle{\pgfqpoint{0.847223in}{0.554012in}}{\pgfqpoint{6.200000in}{4.620000in}}%
\pgfusepath{clip}%
\pgfsetbuttcap%
\pgfsetroundjoin%
\pgfsetlinewidth{1.003750pt}%
\definecolor{currentstroke}{rgb}{1.000000,0.000000,0.000000}%
\pgfsetstrokecolor{currentstroke}%
\pgfsetdash{}{0pt}%
\pgfpathmoveto{\pgfqpoint{3.044508in}{1.057437in}}%
\pgfpathcurveto{\pgfqpoint{3.055558in}{1.057437in}}{\pgfqpoint{3.066157in}{1.061827in}}{\pgfqpoint{3.073971in}{1.069641in}}%
\pgfpathcurveto{\pgfqpoint{3.081784in}{1.077454in}}{\pgfqpoint{3.086175in}{1.088053in}}{\pgfqpoint{3.086175in}{1.099104in}}%
\pgfpathcurveto{\pgfqpoint{3.086175in}{1.110154in}}{\pgfqpoint{3.081784in}{1.120753in}}{\pgfqpoint{3.073971in}{1.128566in}}%
\pgfpathcurveto{\pgfqpoint{3.066157in}{1.136380in}}{\pgfqpoint{3.055558in}{1.140770in}}{\pgfqpoint{3.044508in}{1.140770in}}%
\pgfpathcurveto{\pgfqpoint{3.033458in}{1.140770in}}{\pgfqpoint{3.022859in}{1.136380in}}{\pgfqpoint{3.015045in}{1.128566in}}%
\pgfpathcurveto{\pgfqpoint{3.007232in}{1.120753in}}{\pgfqpoint{3.002841in}{1.110154in}}{\pgfqpoint{3.002841in}{1.099104in}}%
\pgfpathcurveto{\pgfqpoint{3.002841in}{1.088053in}}{\pgfqpoint{3.007232in}{1.077454in}}{\pgfqpoint{3.015045in}{1.069641in}}%
\pgfpathcurveto{\pgfqpoint{3.022859in}{1.061827in}}{\pgfqpoint{3.033458in}{1.057437in}}{\pgfqpoint{3.044508in}{1.057437in}}%
\pgfpathlineto{\pgfqpoint{3.044508in}{1.057437in}}%
\pgfpathclose%
\pgfusepath{stroke}%
\end{pgfscope}%
\begin{pgfscope}%
\pgfpathrectangle{\pgfqpoint{0.847223in}{0.554012in}}{\pgfqpoint{6.200000in}{4.620000in}}%
\pgfusepath{clip}%
\pgfsetbuttcap%
\pgfsetroundjoin%
\pgfsetlinewidth{1.003750pt}%
\definecolor{currentstroke}{rgb}{1.000000,0.000000,0.000000}%
\pgfsetstrokecolor{currentstroke}%
\pgfsetdash{}{0pt}%
\pgfpathmoveto{\pgfqpoint{3.049841in}{1.055538in}}%
\pgfpathcurveto{\pgfqpoint{3.060891in}{1.055538in}}{\pgfqpoint{3.071490in}{1.059928in}}{\pgfqpoint{3.079304in}{1.067742in}}%
\pgfpathcurveto{\pgfqpoint{3.087118in}{1.075555in}}{\pgfqpoint{3.091508in}{1.086155in}}{\pgfqpoint{3.091508in}{1.097205in}}%
\pgfpathcurveto{\pgfqpoint{3.091508in}{1.108255in}}{\pgfqpoint{3.087118in}{1.118854in}}{\pgfqpoint{3.079304in}{1.126667in}}%
\pgfpathcurveto{\pgfqpoint{3.071490in}{1.134481in}}{\pgfqpoint{3.060891in}{1.138871in}}{\pgfqpoint{3.049841in}{1.138871in}}%
\pgfpathcurveto{\pgfqpoint{3.038791in}{1.138871in}}{\pgfqpoint{3.028192in}{1.134481in}}{\pgfqpoint{3.020378in}{1.126667in}}%
\pgfpathcurveto{\pgfqpoint{3.012565in}{1.118854in}}{\pgfqpoint{3.008174in}{1.108255in}}{\pgfqpoint{3.008174in}{1.097205in}}%
\pgfpathcurveto{\pgfqpoint{3.008174in}{1.086155in}}{\pgfqpoint{3.012565in}{1.075555in}}{\pgfqpoint{3.020378in}{1.067742in}}%
\pgfpathcurveto{\pgfqpoint{3.028192in}{1.059928in}}{\pgfqpoint{3.038791in}{1.055538in}}{\pgfqpoint{3.049841in}{1.055538in}}%
\pgfpathlineto{\pgfqpoint{3.049841in}{1.055538in}}%
\pgfpathclose%
\pgfusepath{stroke}%
\end{pgfscope}%
\begin{pgfscope}%
\pgfpathrectangle{\pgfqpoint{0.847223in}{0.554012in}}{\pgfqpoint{6.200000in}{4.620000in}}%
\pgfusepath{clip}%
\pgfsetbuttcap%
\pgfsetroundjoin%
\pgfsetlinewidth{1.003750pt}%
\definecolor{currentstroke}{rgb}{1.000000,0.000000,0.000000}%
\pgfsetstrokecolor{currentstroke}%
\pgfsetdash{}{0pt}%
\pgfpathmoveto{\pgfqpoint{3.055174in}{1.053647in}}%
\pgfpathcurveto{\pgfqpoint{3.066224in}{1.053647in}}{\pgfqpoint{3.076824in}{1.058037in}}{\pgfqpoint{3.084637in}{1.065850in}}%
\pgfpathcurveto{\pgfqpoint{3.092451in}{1.073664in}}{\pgfqpoint{3.096841in}{1.084263in}}{\pgfqpoint{3.096841in}{1.095313in}}%
\pgfpathcurveto{\pgfqpoint{3.096841in}{1.106363in}}{\pgfqpoint{3.092451in}{1.116962in}}{\pgfqpoint{3.084637in}{1.124776in}}%
\pgfpathcurveto{\pgfqpoint{3.076824in}{1.132590in}}{\pgfqpoint{3.066224in}{1.136980in}}{\pgfqpoint{3.055174in}{1.136980in}}%
\pgfpathcurveto{\pgfqpoint{3.044124in}{1.136980in}}{\pgfqpoint{3.033525in}{1.132590in}}{\pgfqpoint{3.025712in}{1.124776in}}%
\pgfpathcurveto{\pgfqpoint{3.017898in}{1.116962in}}{\pgfqpoint{3.013508in}{1.106363in}}{\pgfqpoint{3.013508in}{1.095313in}}%
\pgfpathcurveto{\pgfqpoint{3.013508in}{1.084263in}}{\pgfqpoint{3.017898in}{1.073664in}}{\pgfqpoint{3.025712in}{1.065850in}}%
\pgfpathcurveto{\pgfqpoint{3.033525in}{1.058037in}}{\pgfqpoint{3.044124in}{1.053647in}}{\pgfqpoint{3.055174in}{1.053647in}}%
\pgfpathlineto{\pgfqpoint{3.055174in}{1.053647in}}%
\pgfpathclose%
\pgfusepath{stroke}%
\end{pgfscope}%
\begin{pgfscope}%
\pgfpathrectangle{\pgfqpoint{0.847223in}{0.554012in}}{\pgfqpoint{6.200000in}{4.620000in}}%
\pgfusepath{clip}%
\pgfsetbuttcap%
\pgfsetroundjoin%
\pgfsetlinewidth{1.003750pt}%
\definecolor{currentstroke}{rgb}{1.000000,0.000000,0.000000}%
\pgfsetstrokecolor{currentstroke}%
\pgfsetdash{}{0pt}%
\pgfpathmoveto{\pgfqpoint{3.060508in}{1.051762in}}%
\pgfpathcurveto{\pgfqpoint{3.071558in}{1.051762in}}{\pgfqpoint{3.082157in}{1.056153in}}{\pgfqpoint{3.089970in}{1.063966in}}%
\pgfpathcurveto{\pgfqpoint{3.097784in}{1.071780in}}{\pgfqpoint{3.102174in}{1.082379in}}{\pgfqpoint{3.102174in}{1.093429in}}%
\pgfpathcurveto{\pgfqpoint{3.102174in}{1.104479in}}{\pgfqpoint{3.097784in}{1.115078in}}{\pgfqpoint{3.089970in}{1.122892in}}%
\pgfpathcurveto{\pgfqpoint{3.082157in}{1.130706in}}{\pgfqpoint{3.071558in}{1.135096in}}{\pgfqpoint{3.060508in}{1.135096in}}%
\pgfpathcurveto{\pgfqpoint{3.049457in}{1.135096in}}{\pgfqpoint{3.038858in}{1.130706in}}{\pgfqpoint{3.031045in}{1.122892in}}%
\pgfpathcurveto{\pgfqpoint{3.023231in}{1.115078in}}{\pgfqpoint{3.018841in}{1.104479in}}{\pgfqpoint{3.018841in}{1.093429in}}%
\pgfpathcurveto{\pgfqpoint{3.018841in}{1.082379in}}{\pgfqpoint{3.023231in}{1.071780in}}{\pgfqpoint{3.031045in}{1.063966in}}%
\pgfpathcurveto{\pgfqpoint{3.038858in}{1.056153in}}{\pgfqpoint{3.049457in}{1.051762in}}{\pgfqpoint{3.060508in}{1.051762in}}%
\pgfpathlineto{\pgfqpoint{3.060508in}{1.051762in}}%
\pgfpathclose%
\pgfusepath{stroke}%
\end{pgfscope}%
\begin{pgfscope}%
\pgfpathrectangle{\pgfqpoint{0.847223in}{0.554012in}}{\pgfqpoint{6.200000in}{4.620000in}}%
\pgfusepath{clip}%
\pgfsetbuttcap%
\pgfsetroundjoin%
\pgfsetlinewidth{1.003750pt}%
\definecolor{currentstroke}{rgb}{1.000000,0.000000,0.000000}%
\pgfsetstrokecolor{currentstroke}%
\pgfsetdash{}{0pt}%
\pgfpathmoveto{\pgfqpoint{3.065841in}{1.049886in}}%
\pgfpathcurveto{\pgfqpoint{3.076891in}{1.049886in}}{\pgfqpoint{3.087490in}{1.054276in}}{\pgfqpoint{3.095304in}{1.062090in}}%
\pgfpathcurveto{\pgfqpoint{3.103117in}{1.069903in}}{\pgfqpoint{3.107507in}{1.080502in}}{\pgfqpoint{3.107507in}{1.091552in}}%
\pgfpathcurveto{\pgfqpoint{3.107507in}{1.102603in}}{\pgfqpoint{3.103117in}{1.113202in}}{\pgfqpoint{3.095304in}{1.121015in}}%
\pgfpathcurveto{\pgfqpoint{3.087490in}{1.128829in}}{\pgfqpoint{3.076891in}{1.133219in}}{\pgfqpoint{3.065841in}{1.133219in}}%
\pgfpathcurveto{\pgfqpoint{3.054791in}{1.133219in}}{\pgfqpoint{3.044192in}{1.128829in}}{\pgfqpoint{3.036378in}{1.121015in}}%
\pgfpathcurveto{\pgfqpoint{3.028564in}{1.113202in}}{\pgfqpoint{3.024174in}{1.102603in}}{\pgfqpoint{3.024174in}{1.091552in}}%
\pgfpathcurveto{\pgfqpoint{3.024174in}{1.080502in}}{\pgfqpoint{3.028564in}{1.069903in}}{\pgfqpoint{3.036378in}{1.062090in}}%
\pgfpathcurveto{\pgfqpoint{3.044192in}{1.054276in}}{\pgfqpoint{3.054791in}{1.049886in}}{\pgfqpoint{3.065841in}{1.049886in}}%
\pgfpathlineto{\pgfqpoint{3.065841in}{1.049886in}}%
\pgfpathclose%
\pgfusepath{stroke}%
\end{pgfscope}%
\begin{pgfscope}%
\pgfpathrectangle{\pgfqpoint{0.847223in}{0.554012in}}{\pgfqpoint{6.200000in}{4.620000in}}%
\pgfusepath{clip}%
\pgfsetbuttcap%
\pgfsetroundjoin%
\pgfsetlinewidth{1.003750pt}%
\definecolor{currentstroke}{rgb}{1.000000,0.000000,0.000000}%
\pgfsetstrokecolor{currentstroke}%
\pgfsetdash{}{0pt}%
\pgfpathmoveto{\pgfqpoint{3.071174in}{1.048016in}}%
\pgfpathcurveto{\pgfqpoint{3.082224in}{1.048016in}}{\pgfqpoint{3.092823in}{1.052407in}}{\pgfqpoint{3.100637in}{1.060220in}}%
\pgfpathcurveto{\pgfqpoint{3.108450in}{1.068034in}}{\pgfqpoint{3.112841in}{1.078633in}}{\pgfqpoint{3.112841in}{1.089683in}}%
\pgfpathcurveto{\pgfqpoint{3.112841in}{1.100733in}}{\pgfqpoint{3.108450in}{1.111332in}}{\pgfqpoint{3.100637in}{1.119146in}}%
\pgfpathcurveto{\pgfqpoint{3.092823in}{1.126959in}}{\pgfqpoint{3.082224in}{1.131350in}}{\pgfqpoint{3.071174in}{1.131350in}}%
\pgfpathcurveto{\pgfqpoint{3.060124in}{1.131350in}}{\pgfqpoint{3.049525in}{1.126959in}}{\pgfqpoint{3.041711in}{1.119146in}}%
\pgfpathcurveto{\pgfqpoint{3.033898in}{1.111332in}}{\pgfqpoint{3.029507in}{1.100733in}}{\pgfqpoint{3.029507in}{1.089683in}}%
\pgfpathcurveto{\pgfqpoint{3.029507in}{1.078633in}}{\pgfqpoint{3.033898in}{1.068034in}}{\pgfqpoint{3.041711in}{1.060220in}}%
\pgfpathcurveto{\pgfqpoint{3.049525in}{1.052407in}}{\pgfqpoint{3.060124in}{1.048016in}}{\pgfqpoint{3.071174in}{1.048016in}}%
\pgfpathlineto{\pgfqpoint{3.071174in}{1.048016in}}%
\pgfpathclose%
\pgfusepath{stroke}%
\end{pgfscope}%
\begin{pgfscope}%
\pgfpathrectangle{\pgfqpoint{0.847223in}{0.554012in}}{\pgfqpoint{6.200000in}{4.620000in}}%
\pgfusepath{clip}%
\pgfsetbuttcap%
\pgfsetroundjoin%
\pgfsetlinewidth{1.003750pt}%
\definecolor{currentstroke}{rgb}{1.000000,0.000000,0.000000}%
\pgfsetstrokecolor{currentstroke}%
\pgfsetdash{}{0pt}%
\pgfpathmoveto{\pgfqpoint{3.076507in}{1.046154in}}%
\pgfpathcurveto{\pgfqpoint{3.087557in}{1.046154in}}{\pgfqpoint{3.098156in}{1.050544in}}{\pgfqpoint{3.105970in}{1.058358in}}%
\pgfpathcurveto{\pgfqpoint{3.113784in}{1.066172in}}{\pgfqpoint{3.118174in}{1.076771in}}{\pgfqpoint{3.118174in}{1.087821in}}%
\pgfpathcurveto{\pgfqpoint{3.118174in}{1.098871in}}{\pgfqpoint{3.113784in}{1.109470in}}{\pgfqpoint{3.105970in}{1.117284in}}%
\pgfpathcurveto{\pgfqpoint{3.098156in}{1.125097in}}{\pgfqpoint{3.087557in}{1.129487in}}{\pgfqpoint{3.076507in}{1.129487in}}%
\pgfpathcurveto{\pgfqpoint{3.065457in}{1.129487in}}{\pgfqpoint{3.054858in}{1.125097in}}{\pgfqpoint{3.047044in}{1.117284in}}%
\pgfpathcurveto{\pgfqpoint{3.039231in}{1.109470in}}{\pgfqpoint{3.034841in}{1.098871in}}{\pgfqpoint{3.034841in}{1.087821in}}%
\pgfpathcurveto{\pgfqpoint{3.034841in}{1.076771in}}{\pgfqpoint{3.039231in}{1.066172in}}{\pgfqpoint{3.047044in}{1.058358in}}%
\pgfpathcurveto{\pgfqpoint{3.054858in}{1.050544in}}{\pgfqpoint{3.065457in}{1.046154in}}{\pgfqpoint{3.076507in}{1.046154in}}%
\pgfpathlineto{\pgfqpoint{3.076507in}{1.046154in}}%
\pgfpathclose%
\pgfusepath{stroke}%
\end{pgfscope}%
\begin{pgfscope}%
\pgfpathrectangle{\pgfqpoint{0.847223in}{0.554012in}}{\pgfqpoint{6.200000in}{4.620000in}}%
\pgfusepath{clip}%
\pgfsetbuttcap%
\pgfsetroundjoin%
\pgfsetlinewidth{1.003750pt}%
\definecolor{currentstroke}{rgb}{1.000000,0.000000,0.000000}%
\pgfsetstrokecolor{currentstroke}%
\pgfsetdash{}{0pt}%
\pgfpathmoveto{\pgfqpoint{3.081840in}{1.044299in}}%
\pgfpathcurveto{\pgfqpoint{3.092891in}{1.044299in}}{\pgfqpoint{3.103490in}{1.048689in}}{\pgfqpoint{3.111303in}{1.056503in}}%
\pgfpathcurveto{\pgfqpoint{3.119117in}{1.064317in}}{\pgfqpoint{3.123507in}{1.074916in}}{\pgfqpoint{3.123507in}{1.085966in}}%
\pgfpathcurveto{\pgfqpoint{3.123507in}{1.097016in}}{\pgfqpoint{3.119117in}{1.107615in}}{\pgfqpoint{3.111303in}{1.115429in}}%
\pgfpathcurveto{\pgfqpoint{3.103490in}{1.123242in}}{\pgfqpoint{3.092891in}{1.127632in}}{\pgfqpoint{3.081840in}{1.127632in}}%
\pgfpathcurveto{\pgfqpoint{3.070790in}{1.127632in}}{\pgfqpoint{3.060191in}{1.123242in}}{\pgfqpoint{3.052378in}{1.115429in}}%
\pgfpathcurveto{\pgfqpoint{3.044564in}{1.107615in}}{\pgfqpoint{3.040174in}{1.097016in}}{\pgfqpoint{3.040174in}{1.085966in}}%
\pgfpathcurveto{\pgfqpoint{3.040174in}{1.074916in}}{\pgfqpoint{3.044564in}{1.064317in}}{\pgfqpoint{3.052378in}{1.056503in}}%
\pgfpathcurveto{\pgfqpoint{3.060191in}{1.048689in}}{\pgfqpoint{3.070790in}{1.044299in}}{\pgfqpoint{3.081840in}{1.044299in}}%
\pgfpathlineto{\pgfqpoint{3.081840in}{1.044299in}}%
\pgfpathclose%
\pgfusepath{stroke}%
\end{pgfscope}%
\begin{pgfscope}%
\pgfpathrectangle{\pgfqpoint{0.847223in}{0.554012in}}{\pgfqpoint{6.200000in}{4.620000in}}%
\pgfusepath{clip}%
\pgfsetbuttcap%
\pgfsetroundjoin%
\pgfsetlinewidth{1.003750pt}%
\definecolor{currentstroke}{rgb}{1.000000,0.000000,0.000000}%
\pgfsetstrokecolor{currentstroke}%
\pgfsetdash{}{0pt}%
\pgfpathmoveto{\pgfqpoint{3.087174in}{1.042451in}}%
\pgfpathcurveto{\pgfqpoint{3.098224in}{1.042451in}}{\pgfqpoint{3.108823in}{1.046841in}}{\pgfqpoint{3.116636in}{1.054655in}}%
\pgfpathcurveto{\pgfqpoint{3.124450in}{1.062469in}}{\pgfqpoint{3.128840in}{1.073068in}}{\pgfqpoint{3.128840in}{1.084118in}}%
\pgfpathcurveto{\pgfqpoint{3.128840in}{1.095168in}}{\pgfqpoint{3.124450in}{1.105767in}}{\pgfqpoint{3.116636in}{1.113581in}}%
\pgfpathcurveto{\pgfqpoint{3.108823in}{1.121394in}}{\pgfqpoint{3.098224in}{1.125785in}}{\pgfqpoint{3.087174in}{1.125785in}}%
\pgfpathcurveto{\pgfqpoint{3.076124in}{1.125785in}}{\pgfqpoint{3.065524in}{1.121394in}}{\pgfqpoint{3.057711in}{1.113581in}}%
\pgfpathcurveto{\pgfqpoint{3.049897in}{1.105767in}}{\pgfqpoint{3.045507in}{1.095168in}}{\pgfqpoint{3.045507in}{1.084118in}}%
\pgfpathcurveto{\pgfqpoint{3.045507in}{1.073068in}}{\pgfqpoint{3.049897in}{1.062469in}}{\pgfqpoint{3.057711in}{1.054655in}}%
\pgfpathcurveto{\pgfqpoint{3.065524in}{1.046841in}}{\pgfqpoint{3.076124in}{1.042451in}}{\pgfqpoint{3.087174in}{1.042451in}}%
\pgfpathlineto{\pgfqpoint{3.087174in}{1.042451in}}%
\pgfpathclose%
\pgfusepath{stroke}%
\end{pgfscope}%
\begin{pgfscope}%
\pgfpathrectangle{\pgfqpoint{0.847223in}{0.554012in}}{\pgfqpoint{6.200000in}{4.620000in}}%
\pgfusepath{clip}%
\pgfsetbuttcap%
\pgfsetroundjoin%
\pgfsetlinewidth{1.003750pt}%
\definecolor{currentstroke}{rgb}{1.000000,0.000000,0.000000}%
\pgfsetstrokecolor{currentstroke}%
\pgfsetdash{}{0pt}%
\pgfpathmoveto{\pgfqpoint{3.092507in}{1.040610in}}%
\pgfpathcurveto{\pgfqpoint{3.103557in}{1.040610in}}{\pgfqpoint{3.114156in}{1.045001in}}{\pgfqpoint{3.121970in}{1.052814in}}%
\pgfpathcurveto{\pgfqpoint{3.129783in}{1.060628in}}{\pgfqpoint{3.134174in}{1.071227in}}{\pgfqpoint{3.134174in}{1.082277in}}%
\pgfpathcurveto{\pgfqpoint{3.134174in}{1.093327in}}{\pgfqpoint{3.129783in}{1.103926in}}{\pgfqpoint{3.121970in}{1.111740in}}%
\pgfpathcurveto{\pgfqpoint{3.114156in}{1.119554in}}{\pgfqpoint{3.103557in}{1.123944in}}{\pgfqpoint{3.092507in}{1.123944in}}%
\pgfpathcurveto{\pgfqpoint{3.081457in}{1.123944in}}{\pgfqpoint{3.070858in}{1.119554in}}{\pgfqpoint{3.063044in}{1.111740in}}%
\pgfpathcurveto{\pgfqpoint{3.055230in}{1.103926in}}{\pgfqpoint{3.050840in}{1.093327in}}{\pgfqpoint{3.050840in}{1.082277in}}%
\pgfpathcurveto{\pgfqpoint{3.050840in}{1.071227in}}{\pgfqpoint{3.055230in}{1.060628in}}{\pgfqpoint{3.063044in}{1.052814in}}%
\pgfpathcurveto{\pgfqpoint{3.070858in}{1.045001in}}{\pgfqpoint{3.081457in}{1.040610in}}{\pgfqpoint{3.092507in}{1.040610in}}%
\pgfpathlineto{\pgfqpoint{3.092507in}{1.040610in}}%
\pgfpathclose%
\pgfusepath{stroke}%
\end{pgfscope}%
\begin{pgfscope}%
\pgfpathrectangle{\pgfqpoint{0.847223in}{0.554012in}}{\pgfqpoint{6.200000in}{4.620000in}}%
\pgfusepath{clip}%
\pgfsetbuttcap%
\pgfsetroundjoin%
\pgfsetlinewidth{1.003750pt}%
\definecolor{currentstroke}{rgb}{1.000000,0.000000,0.000000}%
\pgfsetstrokecolor{currentstroke}%
\pgfsetdash{}{0pt}%
\pgfpathmoveto{\pgfqpoint{3.097840in}{1.038777in}}%
\pgfpathcurveto{\pgfqpoint{3.108890in}{1.038777in}}{\pgfqpoint{3.119489in}{1.043167in}}{\pgfqpoint{3.127303in}{1.050981in}}%
\pgfpathcurveto{\pgfqpoint{3.135116in}{1.058794in}}{\pgfqpoint{3.139507in}{1.069393in}}{\pgfqpoint{3.139507in}{1.080444in}}%
\pgfpathcurveto{\pgfqpoint{3.139507in}{1.091494in}}{\pgfqpoint{3.135116in}{1.102093in}}{\pgfqpoint{3.127303in}{1.109906in}}%
\pgfpathcurveto{\pgfqpoint{3.119489in}{1.117720in}}{\pgfqpoint{3.108890in}{1.122110in}}{\pgfqpoint{3.097840in}{1.122110in}}%
\pgfpathcurveto{\pgfqpoint{3.086790in}{1.122110in}}{\pgfqpoint{3.076191in}{1.117720in}}{\pgfqpoint{3.068377in}{1.109906in}}%
\pgfpathcurveto{\pgfqpoint{3.060564in}{1.102093in}}{\pgfqpoint{3.056173in}{1.091494in}}{\pgfqpoint{3.056173in}{1.080444in}}%
\pgfpathcurveto{\pgfqpoint{3.056173in}{1.069393in}}{\pgfqpoint{3.060564in}{1.058794in}}{\pgfqpoint{3.068377in}{1.050981in}}%
\pgfpathcurveto{\pgfqpoint{3.076191in}{1.043167in}}{\pgfqpoint{3.086790in}{1.038777in}}{\pgfqpoint{3.097840in}{1.038777in}}%
\pgfpathlineto{\pgfqpoint{3.097840in}{1.038777in}}%
\pgfpathclose%
\pgfusepath{stroke}%
\end{pgfscope}%
\begin{pgfscope}%
\pgfpathrectangle{\pgfqpoint{0.847223in}{0.554012in}}{\pgfqpoint{6.200000in}{4.620000in}}%
\pgfusepath{clip}%
\pgfsetbuttcap%
\pgfsetroundjoin%
\pgfsetlinewidth{1.003750pt}%
\definecolor{currentstroke}{rgb}{1.000000,0.000000,0.000000}%
\pgfsetstrokecolor{currentstroke}%
\pgfsetdash{}{0pt}%
\pgfpathmoveto{\pgfqpoint{3.103173in}{1.036950in}}%
\pgfpathcurveto{\pgfqpoint{3.114223in}{1.036950in}}{\pgfqpoint{3.124822in}{1.041340in}}{\pgfqpoint{3.132636in}{1.049154in}}%
\pgfpathcurveto{\pgfqpoint{3.140450in}{1.056968in}}{\pgfqpoint{3.144840in}{1.067567in}}{\pgfqpoint{3.144840in}{1.078617in}}%
\pgfpathcurveto{\pgfqpoint{3.144840in}{1.089667in}}{\pgfqpoint{3.140450in}{1.100266in}}{\pgfqpoint{3.132636in}{1.108080in}}%
\pgfpathcurveto{\pgfqpoint{3.124822in}{1.115893in}}{\pgfqpoint{3.114223in}{1.120284in}}{\pgfqpoint{3.103173in}{1.120284in}}%
\pgfpathcurveto{\pgfqpoint{3.092123in}{1.120284in}}{\pgfqpoint{3.081524in}{1.115893in}}{\pgfqpoint{3.073710in}{1.108080in}}%
\pgfpathcurveto{\pgfqpoint{3.065897in}{1.100266in}}{\pgfqpoint{3.061507in}{1.089667in}}{\pgfqpoint{3.061507in}{1.078617in}}%
\pgfpathcurveto{\pgfqpoint{3.061507in}{1.067567in}}{\pgfqpoint{3.065897in}{1.056968in}}{\pgfqpoint{3.073710in}{1.049154in}}%
\pgfpathcurveto{\pgfqpoint{3.081524in}{1.041340in}}{\pgfqpoint{3.092123in}{1.036950in}}{\pgfqpoint{3.103173in}{1.036950in}}%
\pgfpathlineto{\pgfqpoint{3.103173in}{1.036950in}}%
\pgfpathclose%
\pgfusepath{stroke}%
\end{pgfscope}%
\begin{pgfscope}%
\pgfpathrectangle{\pgfqpoint{0.847223in}{0.554012in}}{\pgfqpoint{6.200000in}{4.620000in}}%
\pgfusepath{clip}%
\pgfsetbuttcap%
\pgfsetroundjoin%
\pgfsetlinewidth{1.003750pt}%
\definecolor{currentstroke}{rgb}{1.000000,0.000000,0.000000}%
\pgfsetstrokecolor{currentstroke}%
\pgfsetdash{}{0pt}%
\pgfpathmoveto{\pgfqpoint{3.108506in}{1.035131in}}%
\pgfpathcurveto{\pgfqpoint{3.119557in}{1.035131in}}{\pgfqpoint{3.130156in}{1.039521in}}{\pgfqpoint{3.137969in}{1.047335in}}%
\pgfpathcurveto{\pgfqpoint{3.145783in}{1.055148in}}{\pgfqpoint{3.150173in}{1.065747in}}{\pgfqpoint{3.150173in}{1.076797in}}%
\pgfpathcurveto{\pgfqpoint{3.150173in}{1.087847in}}{\pgfqpoint{3.145783in}{1.098446in}}{\pgfqpoint{3.137969in}{1.106260in}}%
\pgfpathcurveto{\pgfqpoint{3.130156in}{1.114074in}}{\pgfqpoint{3.119557in}{1.118464in}}{\pgfqpoint{3.108506in}{1.118464in}}%
\pgfpathcurveto{\pgfqpoint{3.097456in}{1.118464in}}{\pgfqpoint{3.086857in}{1.114074in}}{\pgfqpoint{3.079044in}{1.106260in}}%
\pgfpathcurveto{\pgfqpoint{3.071230in}{1.098446in}}{\pgfqpoint{3.066840in}{1.087847in}}{\pgfqpoint{3.066840in}{1.076797in}}%
\pgfpathcurveto{\pgfqpoint{3.066840in}{1.065747in}}{\pgfqpoint{3.071230in}{1.055148in}}{\pgfqpoint{3.079044in}{1.047335in}}%
\pgfpathcurveto{\pgfqpoint{3.086857in}{1.039521in}}{\pgfqpoint{3.097456in}{1.035131in}}{\pgfqpoint{3.108506in}{1.035131in}}%
\pgfpathlineto{\pgfqpoint{3.108506in}{1.035131in}}%
\pgfpathclose%
\pgfusepath{stroke}%
\end{pgfscope}%
\begin{pgfscope}%
\pgfpathrectangle{\pgfqpoint{0.847223in}{0.554012in}}{\pgfqpoint{6.200000in}{4.620000in}}%
\pgfusepath{clip}%
\pgfsetbuttcap%
\pgfsetroundjoin%
\pgfsetlinewidth{1.003750pt}%
\definecolor{currentstroke}{rgb}{1.000000,0.000000,0.000000}%
\pgfsetstrokecolor{currentstroke}%
\pgfsetdash{}{0pt}%
\pgfpathmoveto{\pgfqpoint{3.113840in}{1.033318in}}%
\pgfpathcurveto{\pgfqpoint{3.124890in}{1.033318in}}{\pgfqpoint{3.135489in}{1.037708in}}{\pgfqpoint{3.143302in}{1.045522in}}%
\pgfpathcurveto{\pgfqpoint{3.151116in}{1.053335in}}{\pgfqpoint{3.155506in}{1.063934in}}{\pgfqpoint{3.155506in}{1.074985in}}%
\pgfpathcurveto{\pgfqpoint{3.155506in}{1.086035in}}{\pgfqpoint{3.151116in}{1.096634in}}{\pgfqpoint{3.143302in}{1.104447in}}%
\pgfpathcurveto{\pgfqpoint{3.135489in}{1.112261in}}{\pgfqpoint{3.124890in}{1.116651in}}{\pgfqpoint{3.113840in}{1.116651in}}%
\pgfpathcurveto{\pgfqpoint{3.102790in}{1.116651in}}{\pgfqpoint{3.092191in}{1.112261in}}{\pgfqpoint{3.084377in}{1.104447in}}%
\pgfpathcurveto{\pgfqpoint{3.076563in}{1.096634in}}{\pgfqpoint{3.072173in}{1.086035in}}{\pgfqpoint{3.072173in}{1.074985in}}%
\pgfpathcurveto{\pgfqpoint{3.072173in}{1.063934in}}{\pgfqpoint{3.076563in}{1.053335in}}{\pgfqpoint{3.084377in}{1.045522in}}%
\pgfpathcurveto{\pgfqpoint{3.092191in}{1.037708in}}{\pgfqpoint{3.102790in}{1.033318in}}{\pgfqpoint{3.113840in}{1.033318in}}%
\pgfpathlineto{\pgfqpoint{3.113840in}{1.033318in}}%
\pgfpathclose%
\pgfusepath{stroke}%
\end{pgfscope}%
\begin{pgfscope}%
\pgfpathrectangle{\pgfqpoint{0.847223in}{0.554012in}}{\pgfqpoint{6.200000in}{4.620000in}}%
\pgfusepath{clip}%
\pgfsetbuttcap%
\pgfsetroundjoin%
\pgfsetlinewidth{1.003750pt}%
\definecolor{currentstroke}{rgb}{1.000000,0.000000,0.000000}%
\pgfsetstrokecolor{currentstroke}%
\pgfsetdash{}{0pt}%
\pgfpathmoveto{\pgfqpoint{3.119173in}{1.031512in}}%
\pgfpathcurveto{\pgfqpoint{3.130223in}{1.031512in}}{\pgfqpoint{3.140822in}{1.035902in}}{\pgfqpoint{3.148636in}{1.043716in}}%
\pgfpathcurveto{\pgfqpoint{3.156449in}{1.051530in}}{\pgfqpoint{3.160840in}{1.062129in}}{\pgfqpoint{3.160840in}{1.073179in}}%
\pgfpathcurveto{\pgfqpoint{3.160840in}{1.084229in}}{\pgfqpoint{3.156449in}{1.094828in}}{\pgfqpoint{3.148636in}{1.102642in}}%
\pgfpathcurveto{\pgfqpoint{3.140822in}{1.110455in}}{\pgfqpoint{3.130223in}{1.114846in}}{\pgfqpoint{3.119173in}{1.114846in}}%
\pgfpathcurveto{\pgfqpoint{3.108123in}{1.114846in}}{\pgfqpoint{3.097524in}{1.110455in}}{\pgfqpoint{3.089710in}{1.102642in}}%
\pgfpathcurveto{\pgfqpoint{3.081897in}{1.094828in}}{\pgfqpoint{3.077506in}{1.084229in}}{\pgfqpoint{3.077506in}{1.073179in}}%
\pgfpathcurveto{\pgfqpoint{3.077506in}{1.062129in}}{\pgfqpoint{3.081897in}{1.051530in}}{\pgfqpoint{3.089710in}{1.043716in}}%
\pgfpathcurveto{\pgfqpoint{3.097524in}{1.035902in}}{\pgfqpoint{3.108123in}{1.031512in}}{\pgfqpoint{3.119173in}{1.031512in}}%
\pgfpathlineto{\pgfqpoint{3.119173in}{1.031512in}}%
\pgfpathclose%
\pgfusepath{stroke}%
\end{pgfscope}%
\begin{pgfscope}%
\pgfpathrectangle{\pgfqpoint{0.847223in}{0.554012in}}{\pgfqpoint{6.200000in}{4.620000in}}%
\pgfusepath{clip}%
\pgfsetbuttcap%
\pgfsetroundjoin%
\pgfsetlinewidth{1.003750pt}%
\definecolor{currentstroke}{rgb}{1.000000,0.000000,0.000000}%
\pgfsetstrokecolor{currentstroke}%
\pgfsetdash{}{0pt}%
\pgfpathmoveto{\pgfqpoint{3.124506in}{1.029713in}}%
\pgfpathcurveto{\pgfqpoint{3.135556in}{1.029713in}}{\pgfqpoint{3.146155in}{1.034104in}}{\pgfqpoint{3.153969in}{1.041917in}}%
\pgfpathcurveto{\pgfqpoint{3.161783in}{1.049731in}}{\pgfqpoint{3.166173in}{1.060330in}}{\pgfqpoint{3.166173in}{1.071380in}}%
\pgfpathcurveto{\pgfqpoint{3.166173in}{1.082430in}}{\pgfqpoint{3.161783in}{1.093029in}}{\pgfqpoint{3.153969in}{1.100843in}}%
\pgfpathcurveto{\pgfqpoint{3.146155in}{1.108656in}}{\pgfqpoint{3.135556in}{1.113047in}}{\pgfqpoint{3.124506in}{1.113047in}}%
\pgfpathcurveto{\pgfqpoint{3.113456in}{1.113047in}}{\pgfqpoint{3.102857in}{1.108656in}}{\pgfqpoint{3.095043in}{1.100843in}}%
\pgfpathcurveto{\pgfqpoint{3.087230in}{1.093029in}}{\pgfqpoint{3.082839in}{1.082430in}}{\pgfqpoint{3.082839in}{1.071380in}}%
\pgfpathcurveto{\pgfqpoint{3.082839in}{1.060330in}}{\pgfqpoint{3.087230in}{1.049731in}}{\pgfqpoint{3.095043in}{1.041917in}}%
\pgfpathcurveto{\pgfqpoint{3.102857in}{1.034104in}}{\pgfqpoint{3.113456in}{1.029713in}}{\pgfqpoint{3.124506in}{1.029713in}}%
\pgfpathlineto{\pgfqpoint{3.124506in}{1.029713in}}%
\pgfpathclose%
\pgfusepath{stroke}%
\end{pgfscope}%
\begin{pgfscope}%
\pgfpathrectangle{\pgfqpoint{0.847223in}{0.554012in}}{\pgfqpoint{6.200000in}{4.620000in}}%
\pgfusepath{clip}%
\pgfsetbuttcap%
\pgfsetroundjoin%
\pgfsetlinewidth{1.003750pt}%
\definecolor{currentstroke}{rgb}{1.000000,0.000000,0.000000}%
\pgfsetstrokecolor{currentstroke}%
\pgfsetdash{}{0pt}%
\pgfpathmoveto{\pgfqpoint{3.129839in}{1.027921in}}%
\pgfpathcurveto{\pgfqpoint{3.140889in}{1.027921in}}{\pgfqpoint{3.151489in}{1.032312in}}{\pgfqpoint{3.159302in}{1.040125in}}%
\pgfpathcurveto{\pgfqpoint{3.167116in}{1.047939in}}{\pgfqpoint{3.171506in}{1.058538in}}{\pgfqpoint{3.171506in}{1.069588in}}%
\pgfpathcurveto{\pgfqpoint{3.171506in}{1.080638in}}{\pgfqpoint{3.167116in}{1.091237in}}{\pgfqpoint{3.159302in}{1.099051in}}%
\pgfpathcurveto{\pgfqpoint{3.151489in}{1.106864in}}{\pgfqpoint{3.140889in}{1.111255in}}{\pgfqpoint{3.129839in}{1.111255in}}%
\pgfpathcurveto{\pgfqpoint{3.118789in}{1.111255in}}{\pgfqpoint{3.108190in}{1.106864in}}{\pgfqpoint{3.100377in}{1.099051in}}%
\pgfpathcurveto{\pgfqpoint{3.092563in}{1.091237in}}{\pgfqpoint{3.088173in}{1.080638in}}{\pgfqpoint{3.088173in}{1.069588in}}%
\pgfpathcurveto{\pgfqpoint{3.088173in}{1.058538in}}{\pgfqpoint{3.092563in}{1.047939in}}{\pgfqpoint{3.100377in}{1.040125in}}%
\pgfpathcurveto{\pgfqpoint{3.108190in}{1.032312in}}{\pgfqpoint{3.118789in}{1.027921in}}{\pgfqpoint{3.129839in}{1.027921in}}%
\pgfpathlineto{\pgfqpoint{3.129839in}{1.027921in}}%
\pgfpathclose%
\pgfusepath{stroke}%
\end{pgfscope}%
\begin{pgfscope}%
\pgfpathrectangle{\pgfqpoint{0.847223in}{0.554012in}}{\pgfqpoint{6.200000in}{4.620000in}}%
\pgfusepath{clip}%
\pgfsetbuttcap%
\pgfsetroundjoin%
\pgfsetlinewidth{1.003750pt}%
\definecolor{currentstroke}{rgb}{1.000000,0.000000,0.000000}%
\pgfsetstrokecolor{currentstroke}%
\pgfsetdash{}{0pt}%
\pgfpathmoveto{\pgfqpoint{3.135173in}{1.026136in}}%
\pgfpathcurveto{\pgfqpoint{3.146223in}{1.026136in}}{\pgfqpoint{3.156822in}{1.030526in}}{\pgfqpoint{3.164635in}{1.038340in}}%
\pgfpathcurveto{\pgfqpoint{3.172449in}{1.046154in}}{\pgfqpoint{3.176839in}{1.056753in}}{\pgfqpoint{3.176839in}{1.067803in}}%
\pgfpathcurveto{\pgfqpoint{3.176839in}{1.078853in}}{\pgfqpoint{3.172449in}{1.089452in}}{\pgfqpoint{3.164635in}{1.097266in}}%
\pgfpathcurveto{\pgfqpoint{3.156822in}{1.105079in}}{\pgfqpoint{3.146223in}{1.109469in}}{\pgfqpoint{3.135173in}{1.109469in}}%
\pgfpathcurveto{\pgfqpoint{3.124122in}{1.109469in}}{\pgfqpoint{3.113523in}{1.105079in}}{\pgfqpoint{3.105710in}{1.097266in}}%
\pgfpathcurveto{\pgfqpoint{3.097896in}{1.089452in}}{\pgfqpoint{3.093506in}{1.078853in}}{\pgfqpoint{3.093506in}{1.067803in}}%
\pgfpathcurveto{\pgfqpoint{3.093506in}{1.056753in}}{\pgfqpoint{3.097896in}{1.046154in}}{\pgfqpoint{3.105710in}{1.038340in}}%
\pgfpathcurveto{\pgfqpoint{3.113523in}{1.030526in}}{\pgfqpoint{3.124122in}{1.026136in}}{\pgfqpoint{3.135173in}{1.026136in}}%
\pgfpathlineto{\pgfqpoint{3.135173in}{1.026136in}}%
\pgfpathclose%
\pgfusepath{stroke}%
\end{pgfscope}%
\begin{pgfscope}%
\pgfpathrectangle{\pgfqpoint{0.847223in}{0.554012in}}{\pgfqpoint{6.200000in}{4.620000in}}%
\pgfusepath{clip}%
\pgfsetbuttcap%
\pgfsetroundjoin%
\pgfsetlinewidth{1.003750pt}%
\definecolor{currentstroke}{rgb}{1.000000,0.000000,0.000000}%
\pgfsetstrokecolor{currentstroke}%
\pgfsetdash{}{0pt}%
\pgfpathmoveto{\pgfqpoint{3.140506in}{1.024358in}}%
\pgfpathcurveto{\pgfqpoint{3.151556in}{1.024358in}}{\pgfqpoint{3.162155in}{1.028748in}}{\pgfqpoint{3.169969in}{1.036562in}}%
\pgfpathcurveto{\pgfqpoint{3.177782in}{1.044375in}}{\pgfqpoint{3.182172in}{1.054974in}}{\pgfqpoint{3.182172in}{1.066024in}}%
\pgfpathcurveto{\pgfqpoint{3.182172in}{1.077074in}}{\pgfqpoint{3.177782in}{1.087673in}}{\pgfqpoint{3.169969in}{1.095487in}}%
\pgfpathcurveto{\pgfqpoint{3.162155in}{1.103301in}}{\pgfqpoint{3.151556in}{1.107691in}}{\pgfqpoint{3.140506in}{1.107691in}}%
\pgfpathcurveto{\pgfqpoint{3.129456in}{1.107691in}}{\pgfqpoint{3.118857in}{1.103301in}}{\pgfqpoint{3.111043in}{1.095487in}}%
\pgfpathcurveto{\pgfqpoint{3.103229in}{1.087673in}}{\pgfqpoint{3.098839in}{1.077074in}}{\pgfqpoint{3.098839in}{1.066024in}}%
\pgfpathcurveto{\pgfqpoint{3.098839in}{1.054974in}}{\pgfqpoint{3.103229in}{1.044375in}}{\pgfqpoint{3.111043in}{1.036562in}}%
\pgfpathcurveto{\pgfqpoint{3.118857in}{1.028748in}}{\pgfqpoint{3.129456in}{1.024358in}}{\pgfqpoint{3.140506in}{1.024358in}}%
\pgfpathlineto{\pgfqpoint{3.140506in}{1.024358in}}%
\pgfpathclose%
\pgfusepath{stroke}%
\end{pgfscope}%
\begin{pgfscope}%
\pgfpathrectangle{\pgfqpoint{0.847223in}{0.554012in}}{\pgfqpoint{6.200000in}{4.620000in}}%
\pgfusepath{clip}%
\pgfsetbuttcap%
\pgfsetroundjoin%
\pgfsetlinewidth{1.003750pt}%
\definecolor{currentstroke}{rgb}{1.000000,0.000000,0.000000}%
\pgfsetstrokecolor{currentstroke}%
\pgfsetdash{}{0pt}%
\pgfpathmoveto{\pgfqpoint{3.145839in}{1.022586in}}%
\pgfpathcurveto{\pgfqpoint{3.156889in}{1.022586in}}{\pgfqpoint{3.167488in}{1.026976in}}{\pgfqpoint{3.175302in}{1.034790in}}%
\pgfpathcurveto{\pgfqpoint{3.183115in}{1.042603in}}{\pgfqpoint{3.187506in}{1.053202in}}{\pgfqpoint{3.187506in}{1.064253in}}%
\pgfpathcurveto{\pgfqpoint{3.187506in}{1.075303in}}{\pgfqpoint{3.183115in}{1.085902in}}{\pgfqpoint{3.175302in}{1.093715in}}%
\pgfpathcurveto{\pgfqpoint{3.167488in}{1.101529in}}{\pgfqpoint{3.156889in}{1.105919in}}{\pgfqpoint{3.145839in}{1.105919in}}%
\pgfpathcurveto{\pgfqpoint{3.134789in}{1.105919in}}{\pgfqpoint{3.124190in}{1.101529in}}{\pgfqpoint{3.116376in}{1.093715in}}%
\pgfpathcurveto{\pgfqpoint{3.108563in}{1.085902in}}{\pgfqpoint{3.104172in}{1.075303in}}{\pgfqpoint{3.104172in}{1.064253in}}%
\pgfpathcurveto{\pgfqpoint{3.104172in}{1.053202in}}{\pgfqpoint{3.108563in}{1.042603in}}{\pgfqpoint{3.116376in}{1.034790in}}%
\pgfpathcurveto{\pgfqpoint{3.124190in}{1.026976in}}{\pgfqpoint{3.134789in}{1.022586in}}{\pgfqpoint{3.145839in}{1.022586in}}%
\pgfpathlineto{\pgfqpoint{3.145839in}{1.022586in}}%
\pgfpathclose%
\pgfusepath{stroke}%
\end{pgfscope}%
\begin{pgfscope}%
\pgfpathrectangle{\pgfqpoint{0.847223in}{0.554012in}}{\pgfqpoint{6.200000in}{4.620000in}}%
\pgfusepath{clip}%
\pgfsetbuttcap%
\pgfsetroundjoin%
\pgfsetlinewidth{1.003750pt}%
\definecolor{currentstroke}{rgb}{1.000000,0.000000,0.000000}%
\pgfsetstrokecolor{currentstroke}%
\pgfsetdash{}{0pt}%
\pgfpathmoveto{\pgfqpoint{3.151172in}{1.020821in}}%
\pgfpathcurveto{\pgfqpoint{3.162222in}{1.020821in}}{\pgfqpoint{3.172821in}{1.025211in}}{\pgfqpoint{3.180635in}{1.033025in}}%
\pgfpathcurveto{\pgfqpoint{3.188449in}{1.040838in}}{\pgfqpoint{3.192839in}{1.051437in}}{\pgfqpoint{3.192839in}{1.062487in}}%
\pgfpathcurveto{\pgfqpoint{3.192839in}{1.073538in}}{\pgfqpoint{3.188449in}{1.084137in}}{\pgfqpoint{3.180635in}{1.091950in}}%
\pgfpathcurveto{\pgfqpoint{3.172821in}{1.099764in}}{\pgfqpoint{3.162222in}{1.104154in}}{\pgfqpoint{3.151172in}{1.104154in}}%
\pgfpathcurveto{\pgfqpoint{3.140122in}{1.104154in}}{\pgfqpoint{3.129523in}{1.099764in}}{\pgfqpoint{3.121709in}{1.091950in}}%
\pgfpathcurveto{\pgfqpoint{3.113896in}{1.084137in}}{\pgfqpoint{3.109506in}{1.073538in}}{\pgfqpoint{3.109506in}{1.062487in}}%
\pgfpathcurveto{\pgfqpoint{3.109506in}{1.051437in}}{\pgfqpoint{3.113896in}{1.040838in}}{\pgfqpoint{3.121709in}{1.033025in}}%
\pgfpathcurveto{\pgfqpoint{3.129523in}{1.025211in}}{\pgfqpoint{3.140122in}{1.020821in}}{\pgfqpoint{3.151172in}{1.020821in}}%
\pgfpathlineto{\pgfqpoint{3.151172in}{1.020821in}}%
\pgfpathclose%
\pgfusepath{stroke}%
\end{pgfscope}%
\begin{pgfscope}%
\pgfpathrectangle{\pgfqpoint{0.847223in}{0.554012in}}{\pgfqpoint{6.200000in}{4.620000in}}%
\pgfusepath{clip}%
\pgfsetbuttcap%
\pgfsetroundjoin%
\pgfsetlinewidth{1.003750pt}%
\definecolor{currentstroke}{rgb}{1.000000,0.000000,0.000000}%
\pgfsetstrokecolor{currentstroke}%
\pgfsetdash{}{0pt}%
\pgfpathmoveto{\pgfqpoint{3.156505in}{1.019062in}}%
\pgfpathcurveto{\pgfqpoint{3.167556in}{1.019062in}}{\pgfqpoint{3.178155in}{1.023453in}}{\pgfqpoint{3.185968in}{1.031266in}}%
\pgfpathcurveto{\pgfqpoint{3.193782in}{1.039080in}}{\pgfqpoint{3.198172in}{1.049679in}}{\pgfqpoint{3.198172in}{1.060729in}}%
\pgfpathcurveto{\pgfqpoint{3.198172in}{1.071779in}}{\pgfqpoint{3.193782in}{1.082378in}}{\pgfqpoint{3.185968in}{1.090192in}}%
\pgfpathcurveto{\pgfqpoint{3.178155in}{1.098005in}}{\pgfqpoint{3.167556in}{1.102396in}}{\pgfqpoint{3.156505in}{1.102396in}}%
\pgfpathcurveto{\pgfqpoint{3.145455in}{1.102396in}}{\pgfqpoint{3.134856in}{1.098005in}}{\pgfqpoint{3.127043in}{1.090192in}}%
\pgfpathcurveto{\pgfqpoint{3.119229in}{1.082378in}}{\pgfqpoint{3.114839in}{1.071779in}}{\pgfqpoint{3.114839in}{1.060729in}}%
\pgfpathcurveto{\pgfqpoint{3.114839in}{1.049679in}}{\pgfqpoint{3.119229in}{1.039080in}}{\pgfqpoint{3.127043in}{1.031266in}}%
\pgfpathcurveto{\pgfqpoint{3.134856in}{1.023453in}}{\pgfqpoint{3.145455in}{1.019062in}}{\pgfqpoint{3.156505in}{1.019062in}}%
\pgfpathlineto{\pgfqpoint{3.156505in}{1.019062in}}%
\pgfpathclose%
\pgfusepath{stroke}%
\end{pgfscope}%
\begin{pgfscope}%
\pgfpathrectangle{\pgfqpoint{0.847223in}{0.554012in}}{\pgfqpoint{6.200000in}{4.620000in}}%
\pgfusepath{clip}%
\pgfsetbuttcap%
\pgfsetroundjoin%
\pgfsetlinewidth{1.003750pt}%
\definecolor{currentstroke}{rgb}{1.000000,0.000000,0.000000}%
\pgfsetstrokecolor{currentstroke}%
\pgfsetdash{}{0pt}%
\pgfpathmoveto{\pgfqpoint{3.161839in}{1.017311in}}%
\pgfpathcurveto{\pgfqpoint{3.172889in}{1.017311in}}{\pgfqpoint{3.183488in}{1.021701in}}{\pgfqpoint{3.191301in}{1.029514in}}%
\pgfpathcurveto{\pgfqpoint{3.199115in}{1.037328in}}{\pgfqpoint{3.203505in}{1.047927in}}{\pgfqpoint{3.203505in}{1.058977in}}%
\pgfpathcurveto{\pgfqpoint{3.203505in}{1.070027in}}{\pgfqpoint{3.199115in}{1.080626in}}{\pgfqpoint{3.191301in}{1.088440in}}%
\pgfpathcurveto{\pgfqpoint{3.183488in}{1.096254in}}{\pgfqpoint{3.172889in}{1.100644in}}{\pgfqpoint{3.161839in}{1.100644in}}%
\pgfpathcurveto{\pgfqpoint{3.150789in}{1.100644in}}{\pgfqpoint{3.140189in}{1.096254in}}{\pgfqpoint{3.132376in}{1.088440in}}%
\pgfpathcurveto{\pgfqpoint{3.124562in}{1.080626in}}{\pgfqpoint{3.120172in}{1.070027in}}{\pgfqpoint{3.120172in}{1.058977in}}%
\pgfpathcurveto{\pgfqpoint{3.120172in}{1.047927in}}{\pgfqpoint{3.124562in}{1.037328in}}{\pgfqpoint{3.132376in}{1.029514in}}%
\pgfpathcurveto{\pgfqpoint{3.140189in}{1.021701in}}{\pgfqpoint{3.150789in}{1.017311in}}{\pgfqpoint{3.161839in}{1.017311in}}%
\pgfpathlineto{\pgfqpoint{3.161839in}{1.017311in}}%
\pgfpathclose%
\pgfusepath{stroke}%
\end{pgfscope}%
\begin{pgfscope}%
\pgfpathrectangle{\pgfqpoint{0.847223in}{0.554012in}}{\pgfqpoint{6.200000in}{4.620000in}}%
\pgfusepath{clip}%
\pgfsetbuttcap%
\pgfsetroundjoin%
\pgfsetlinewidth{1.003750pt}%
\definecolor{currentstroke}{rgb}{1.000000,0.000000,0.000000}%
\pgfsetstrokecolor{currentstroke}%
\pgfsetdash{}{0pt}%
\pgfpathmoveto{\pgfqpoint{3.167172in}{1.015565in}}%
\pgfpathcurveto{\pgfqpoint{3.178222in}{1.015565in}}{\pgfqpoint{3.188821in}{1.019956in}}{\pgfqpoint{3.196635in}{1.027769in}}%
\pgfpathcurveto{\pgfqpoint{3.204448in}{1.035583in}}{\pgfqpoint{3.208839in}{1.046182in}}{\pgfqpoint{3.208839in}{1.057232in}}%
\pgfpathcurveto{\pgfqpoint{3.208839in}{1.068282in}}{\pgfqpoint{3.204448in}{1.078881in}}{\pgfqpoint{3.196635in}{1.086695in}}%
\pgfpathcurveto{\pgfqpoint{3.188821in}{1.094508in}}{\pgfqpoint{3.178222in}{1.098899in}}{\pgfqpoint{3.167172in}{1.098899in}}%
\pgfpathcurveto{\pgfqpoint{3.156122in}{1.098899in}}{\pgfqpoint{3.145523in}{1.094508in}}{\pgfqpoint{3.137709in}{1.086695in}}%
\pgfpathcurveto{\pgfqpoint{3.129895in}{1.078881in}}{\pgfqpoint{3.125505in}{1.068282in}}{\pgfqpoint{3.125505in}{1.057232in}}%
\pgfpathcurveto{\pgfqpoint{3.125505in}{1.046182in}}{\pgfqpoint{3.129895in}{1.035583in}}{\pgfqpoint{3.137709in}{1.027769in}}%
\pgfpathcurveto{\pgfqpoint{3.145523in}{1.019956in}}{\pgfqpoint{3.156122in}{1.015565in}}{\pgfqpoint{3.167172in}{1.015565in}}%
\pgfpathlineto{\pgfqpoint{3.167172in}{1.015565in}}%
\pgfpathclose%
\pgfusepath{stroke}%
\end{pgfscope}%
\begin{pgfscope}%
\pgfpathrectangle{\pgfqpoint{0.847223in}{0.554012in}}{\pgfqpoint{6.200000in}{4.620000in}}%
\pgfusepath{clip}%
\pgfsetbuttcap%
\pgfsetroundjoin%
\pgfsetlinewidth{1.003750pt}%
\definecolor{currentstroke}{rgb}{1.000000,0.000000,0.000000}%
\pgfsetstrokecolor{currentstroke}%
\pgfsetdash{}{0pt}%
\pgfpathmoveto{\pgfqpoint{3.172505in}{1.013827in}}%
\pgfpathcurveto{\pgfqpoint{3.183555in}{1.013827in}}{\pgfqpoint{3.194154in}{1.018217in}}{\pgfqpoint{3.201968in}{1.026031in}}%
\pgfpathcurveto{\pgfqpoint{3.209781in}{1.033844in}}{\pgfqpoint{3.214172in}{1.044443in}}{\pgfqpoint{3.214172in}{1.055493in}}%
\pgfpathcurveto{\pgfqpoint{3.214172in}{1.066543in}}{\pgfqpoint{3.209781in}{1.077143in}}{\pgfqpoint{3.201968in}{1.084956in}}%
\pgfpathcurveto{\pgfqpoint{3.194154in}{1.092770in}}{\pgfqpoint{3.183555in}{1.097160in}}{\pgfqpoint{3.172505in}{1.097160in}}%
\pgfpathcurveto{\pgfqpoint{3.161455in}{1.097160in}}{\pgfqpoint{3.150856in}{1.092770in}}{\pgfqpoint{3.143042in}{1.084956in}}%
\pgfpathcurveto{\pgfqpoint{3.135229in}{1.077143in}}{\pgfqpoint{3.130838in}{1.066543in}}{\pgfqpoint{3.130838in}{1.055493in}}%
\pgfpathcurveto{\pgfqpoint{3.130838in}{1.044443in}}{\pgfqpoint{3.135229in}{1.033844in}}{\pgfqpoint{3.143042in}{1.026031in}}%
\pgfpathcurveto{\pgfqpoint{3.150856in}{1.018217in}}{\pgfqpoint{3.161455in}{1.013827in}}{\pgfqpoint{3.172505in}{1.013827in}}%
\pgfpathlineto{\pgfqpoint{3.172505in}{1.013827in}}%
\pgfpathclose%
\pgfusepath{stroke}%
\end{pgfscope}%
\begin{pgfscope}%
\pgfpathrectangle{\pgfqpoint{0.847223in}{0.554012in}}{\pgfqpoint{6.200000in}{4.620000in}}%
\pgfusepath{clip}%
\pgfsetbuttcap%
\pgfsetroundjoin%
\pgfsetlinewidth{1.003750pt}%
\definecolor{currentstroke}{rgb}{1.000000,0.000000,0.000000}%
\pgfsetstrokecolor{currentstroke}%
\pgfsetdash{}{0pt}%
\pgfpathmoveto{\pgfqpoint{3.177838in}{1.012094in}}%
\pgfpathcurveto{\pgfqpoint{3.188888in}{1.012094in}}{\pgfqpoint{3.199487in}{1.016485in}}{\pgfqpoint{3.207301in}{1.024298in}}%
\pgfpathcurveto{\pgfqpoint{3.215115in}{1.032112in}}{\pgfqpoint{3.219505in}{1.042711in}}{\pgfqpoint{3.219505in}{1.053761in}}%
\pgfpathcurveto{\pgfqpoint{3.219505in}{1.064811in}}{\pgfqpoint{3.215115in}{1.075410in}}{\pgfqpoint{3.207301in}{1.083224in}}%
\pgfpathcurveto{\pgfqpoint{3.199487in}{1.091038in}}{\pgfqpoint{3.188888in}{1.095428in}}{\pgfqpoint{3.177838in}{1.095428in}}%
\pgfpathcurveto{\pgfqpoint{3.166788in}{1.095428in}}{\pgfqpoint{3.156189in}{1.091038in}}{\pgfqpoint{3.148376in}{1.083224in}}%
\pgfpathcurveto{\pgfqpoint{3.140562in}{1.075410in}}{\pgfqpoint{3.136172in}{1.064811in}}{\pgfqpoint{3.136172in}{1.053761in}}%
\pgfpathcurveto{\pgfqpoint{3.136172in}{1.042711in}}{\pgfqpoint{3.140562in}{1.032112in}}{\pgfqpoint{3.148376in}{1.024298in}}%
\pgfpathcurveto{\pgfqpoint{3.156189in}{1.016485in}}{\pgfqpoint{3.166788in}{1.012094in}}{\pgfqpoint{3.177838in}{1.012094in}}%
\pgfpathlineto{\pgfqpoint{3.177838in}{1.012094in}}%
\pgfpathclose%
\pgfusepath{stroke}%
\end{pgfscope}%
\begin{pgfscope}%
\pgfpathrectangle{\pgfqpoint{0.847223in}{0.554012in}}{\pgfqpoint{6.200000in}{4.620000in}}%
\pgfusepath{clip}%
\pgfsetbuttcap%
\pgfsetroundjoin%
\pgfsetlinewidth{1.003750pt}%
\definecolor{currentstroke}{rgb}{1.000000,0.000000,0.000000}%
\pgfsetstrokecolor{currentstroke}%
\pgfsetdash{}{0pt}%
\pgfpathmoveto{\pgfqpoint{3.183172in}{1.010369in}}%
\pgfpathcurveto{\pgfqpoint{3.194222in}{1.010369in}}{\pgfqpoint{3.204821in}{1.014759in}}{\pgfqpoint{3.212634in}{1.022573in}}%
\pgfpathcurveto{\pgfqpoint{3.220448in}{1.030386in}}{\pgfqpoint{3.224838in}{1.040985in}}{\pgfqpoint{3.224838in}{1.052035in}}%
\pgfpathcurveto{\pgfqpoint{3.224838in}{1.063086in}}{\pgfqpoint{3.220448in}{1.073685in}}{\pgfqpoint{3.212634in}{1.081498in}}%
\pgfpathcurveto{\pgfqpoint{3.204821in}{1.089312in}}{\pgfqpoint{3.194222in}{1.093702in}}{\pgfqpoint{3.183172in}{1.093702in}}%
\pgfpathcurveto{\pgfqpoint{3.172121in}{1.093702in}}{\pgfqpoint{3.161522in}{1.089312in}}{\pgfqpoint{3.153709in}{1.081498in}}%
\pgfpathcurveto{\pgfqpoint{3.145895in}{1.073685in}}{\pgfqpoint{3.141505in}{1.063086in}}{\pgfqpoint{3.141505in}{1.052035in}}%
\pgfpathcurveto{\pgfqpoint{3.141505in}{1.040985in}}{\pgfqpoint{3.145895in}{1.030386in}}{\pgfqpoint{3.153709in}{1.022573in}}%
\pgfpathcurveto{\pgfqpoint{3.161522in}{1.014759in}}{\pgfqpoint{3.172121in}{1.010369in}}{\pgfqpoint{3.183172in}{1.010369in}}%
\pgfpathlineto{\pgfqpoint{3.183172in}{1.010369in}}%
\pgfpathclose%
\pgfusepath{stroke}%
\end{pgfscope}%
\begin{pgfscope}%
\pgfpathrectangle{\pgfqpoint{0.847223in}{0.554012in}}{\pgfqpoint{6.200000in}{4.620000in}}%
\pgfusepath{clip}%
\pgfsetbuttcap%
\pgfsetroundjoin%
\pgfsetlinewidth{1.003750pt}%
\definecolor{currentstroke}{rgb}{1.000000,0.000000,0.000000}%
\pgfsetstrokecolor{currentstroke}%
\pgfsetdash{}{0pt}%
\pgfpathmoveto{\pgfqpoint{3.188505in}{1.008649in}}%
\pgfpathcurveto{\pgfqpoint{3.199555in}{1.008649in}}{\pgfqpoint{3.210154in}{1.013040in}}{\pgfqpoint{3.217968in}{1.020853in}}%
\pgfpathcurveto{\pgfqpoint{3.225781in}{1.028667in}}{\pgfqpoint{3.230171in}{1.039266in}}{\pgfqpoint{3.230171in}{1.050316in}}%
\pgfpathcurveto{\pgfqpoint{3.230171in}{1.061366in}}{\pgfqpoint{3.225781in}{1.071965in}}{\pgfqpoint{3.217968in}{1.079779in}}%
\pgfpathcurveto{\pgfqpoint{3.210154in}{1.087592in}}{\pgfqpoint{3.199555in}{1.091983in}}{\pgfqpoint{3.188505in}{1.091983in}}%
\pgfpathcurveto{\pgfqpoint{3.177455in}{1.091983in}}{\pgfqpoint{3.166856in}{1.087592in}}{\pgfqpoint{3.159042in}{1.079779in}}%
\pgfpathcurveto{\pgfqpoint{3.151228in}{1.071965in}}{\pgfqpoint{3.146838in}{1.061366in}}{\pgfqpoint{3.146838in}{1.050316in}}%
\pgfpathcurveto{\pgfqpoint{3.146838in}{1.039266in}}{\pgfqpoint{3.151228in}{1.028667in}}{\pgfqpoint{3.159042in}{1.020853in}}%
\pgfpathcurveto{\pgfqpoint{3.166856in}{1.013040in}}{\pgfqpoint{3.177455in}{1.008649in}}{\pgfqpoint{3.188505in}{1.008649in}}%
\pgfpathlineto{\pgfqpoint{3.188505in}{1.008649in}}%
\pgfpathclose%
\pgfusepath{stroke}%
\end{pgfscope}%
\begin{pgfscope}%
\pgfpathrectangle{\pgfqpoint{0.847223in}{0.554012in}}{\pgfqpoint{6.200000in}{4.620000in}}%
\pgfusepath{clip}%
\pgfsetbuttcap%
\pgfsetroundjoin%
\pgfsetlinewidth{1.003750pt}%
\definecolor{currentstroke}{rgb}{1.000000,0.000000,0.000000}%
\pgfsetstrokecolor{currentstroke}%
\pgfsetdash{}{0pt}%
\pgfpathmoveto{\pgfqpoint{3.193838in}{1.006936in}}%
\pgfpathcurveto{\pgfqpoint{3.204888in}{1.006936in}}{\pgfqpoint{3.215487in}{1.011327in}}{\pgfqpoint{3.223301in}{1.019140in}}%
\pgfpathcurveto{\pgfqpoint{3.231114in}{1.026954in}}{\pgfqpoint{3.235505in}{1.037553in}}{\pgfqpoint{3.235505in}{1.048603in}}%
\pgfpathcurveto{\pgfqpoint{3.235505in}{1.059653in}}{\pgfqpoint{3.231114in}{1.070252in}}{\pgfqpoint{3.223301in}{1.078066in}}%
\pgfpathcurveto{\pgfqpoint{3.215487in}{1.085880in}}{\pgfqpoint{3.204888in}{1.090270in}}{\pgfqpoint{3.193838in}{1.090270in}}%
\pgfpathcurveto{\pgfqpoint{3.182788in}{1.090270in}}{\pgfqpoint{3.172189in}{1.085880in}}{\pgfqpoint{3.164375in}{1.078066in}}%
\pgfpathcurveto{\pgfqpoint{3.156562in}{1.070252in}}{\pgfqpoint{3.152171in}{1.059653in}}{\pgfqpoint{3.152171in}{1.048603in}}%
\pgfpathcurveto{\pgfqpoint{3.152171in}{1.037553in}}{\pgfqpoint{3.156562in}{1.026954in}}{\pgfqpoint{3.164375in}{1.019140in}}%
\pgfpathcurveto{\pgfqpoint{3.172189in}{1.011327in}}{\pgfqpoint{3.182788in}{1.006936in}}{\pgfqpoint{3.193838in}{1.006936in}}%
\pgfpathlineto{\pgfqpoint{3.193838in}{1.006936in}}%
\pgfpathclose%
\pgfusepath{stroke}%
\end{pgfscope}%
\begin{pgfscope}%
\pgfpathrectangle{\pgfqpoint{0.847223in}{0.554012in}}{\pgfqpoint{6.200000in}{4.620000in}}%
\pgfusepath{clip}%
\pgfsetbuttcap%
\pgfsetroundjoin%
\pgfsetlinewidth{1.003750pt}%
\definecolor{currentstroke}{rgb}{1.000000,0.000000,0.000000}%
\pgfsetstrokecolor{currentstroke}%
\pgfsetdash{}{0pt}%
\pgfpathmoveto{\pgfqpoint{3.199171in}{1.005230in}}%
\pgfpathcurveto{\pgfqpoint{3.210221in}{1.005230in}}{\pgfqpoint{3.220820in}{1.009620in}}{\pgfqpoint{3.228634in}{1.017434in}}%
\pgfpathcurveto{\pgfqpoint{3.236448in}{1.025247in}}{\pgfqpoint{3.240838in}{1.035846in}}{\pgfqpoint{3.240838in}{1.046897in}}%
\pgfpathcurveto{\pgfqpoint{3.240838in}{1.057947in}}{\pgfqpoint{3.236448in}{1.068546in}}{\pgfqpoint{3.228634in}{1.076359in}}%
\pgfpathcurveto{\pgfqpoint{3.220820in}{1.084173in}}{\pgfqpoint{3.210221in}{1.088563in}}{\pgfqpoint{3.199171in}{1.088563in}}%
\pgfpathcurveto{\pgfqpoint{3.188121in}{1.088563in}}{\pgfqpoint{3.177522in}{1.084173in}}{\pgfqpoint{3.169708in}{1.076359in}}%
\pgfpathcurveto{\pgfqpoint{3.161895in}{1.068546in}}{\pgfqpoint{3.157504in}{1.057947in}}{\pgfqpoint{3.157504in}{1.046897in}}%
\pgfpathcurveto{\pgfqpoint{3.157504in}{1.035846in}}{\pgfqpoint{3.161895in}{1.025247in}}{\pgfqpoint{3.169708in}{1.017434in}}%
\pgfpathcurveto{\pgfqpoint{3.177522in}{1.009620in}}{\pgfqpoint{3.188121in}{1.005230in}}{\pgfqpoint{3.199171in}{1.005230in}}%
\pgfpathlineto{\pgfqpoint{3.199171in}{1.005230in}}%
\pgfpathclose%
\pgfusepath{stroke}%
\end{pgfscope}%
\begin{pgfscope}%
\pgfpathrectangle{\pgfqpoint{0.847223in}{0.554012in}}{\pgfqpoint{6.200000in}{4.620000in}}%
\pgfusepath{clip}%
\pgfsetbuttcap%
\pgfsetroundjoin%
\pgfsetlinewidth{1.003750pt}%
\definecolor{currentstroke}{rgb}{1.000000,0.000000,0.000000}%
\pgfsetstrokecolor{currentstroke}%
\pgfsetdash{}{0pt}%
\pgfpathmoveto{\pgfqpoint{3.204504in}{1.003530in}}%
\pgfpathcurveto{\pgfqpoint{3.215555in}{1.003530in}}{\pgfqpoint{3.226154in}{1.007920in}}{\pgfqpoint{3.233967in}{1.015734in}}%
\pgfpathcurveto{\pgfqpoint{3.241781in}{1.023547in}}{\pgfqpoint{3.246171in}{1.034146in}}{\pgfqpoint{3.246171in}{1.045196in}}%
\pgfpathcurveto{\pgfqpoint{3.246171in}{1.056246in}}{\pgfqpoint{3.241781in}{1.066846in}}{\pgfqpoint{3.233967in}{1.074659in}}%
\pgfpathcurveto{\pgfqpoint{3.226154in}{1.082473in}}{\pgfqpoint{3.215555in}{1.086863in}}{\pgfqpoint{3.204504in}{1.086863in}}%
\pgfpathcurveto{\pgfqpoint{3.193454in}{1.086863in}}{\pgfqpoint{3.182855in}{1.082473in}}{\pgfqpoint{3.175042in}{1.074659in}}%
\pgfpathcurveto{\pgfqpoint{3.167228in}{1.066846in}}{\pgfqpoint{3.162838in}{1.056246in}}{\pgfqpoint{3.162838in}{1.045196in}}%
\pgfpathcurveto{\pgfqpoint{3.162838in}{1.034146in}}{\pgfqpoint{3.167228in}{1.023547in}}{\pgfqpoint{3.175042in}{1.015734in}}%
\pgfpathcurveto{\pgfqpoint{3.182855in}{1.007920in}}{\pgfqpoint{3.193454in}{1.003530in}}{\pgfqpoint{3.204504in}{1.003530in}}%
\pgfpathlineto{\pgfqpoint{3.204504in}{1.003530in}}%
\pgfpathclose%
\pgfusepath{stroke}%
\end{pgfscope}%
\begin{pgfscope}%
\pgfpathrectangle{\pgfqpoint{0.847223in}{0.554012in}}{\pgfqpoint{6.200000in}{4.620000in}}%
\pgfusepath{clip}%
\pgfsetbuttcap%
\pgfsetroundjoin%
\pgfsetlinewidth{1.003750pt}%
\definecolor{currentstroke}{rgb}{1.000000,0.000000,0.000000}%
\pgfsetstrokecolor{currentstroke}%
\pgfsetdash{}{0pt}%
\pgfpathmoveto{\pgfqpoint{3.209838in}{1.001836in}}%
\pgfpathcurveto{\pgfqpoint{3.220888in}{1.001836in}}{\pgfqpoint{3.231487in}{1.006226in}}{\pgfqpoint{3.239300in}{1.014040in}}%
\pgfpathcurveto{\pgfqpoint{3.247114in}{1.021853in}}{\pgfqpoint{3.251504in}{1.032452in}}{\pgfqpoint{3.251504in}{1.043502in}}%
\pgfpathcurveto{\pgfqpoint{3.251504in}{1.054553in}}{\pgfqpoint{3.247114in}{1.065152in}}{\pgfqpoint{3.239300in}{1.072965in}}%
\pgfpathcurveto{\pgfqpoint{3.231487in}{1.080779in}}{\pgfqpoint{3.220888in}{1.085169in}}{\pgfqpoint{3.209838in}{1.085169in}}%
\pgfpathcurveto{\pgfqpoint{3.198787in}{1.085169in}}{\pgfqpoint{3.188188in}{1.080779in}}{\pgfqpoint{3.180375in}{1.072965in}}%
\pgfpathcurveto{\pgfqpoint{3.172561in}{1.065152in}}{\pgfqpoint{3.168171in}{1.054553in}}{\pgfqpoint{3.168171in}{1.043502in}}%
\pgfpathcurveto{\pgfqpoint{3.168171in}{1.032452in}}{\pgfqpoint{3.172561in}{1.021853in}}{\pgfqpoint{3.180375in}{1.014040in}}%
\pgfpathcurveto{\pgfqpoint{3.188188in}{1.006226in}}{\pgfqpoint{3.198787in}{1.001836in}}{\pgfqpoint{3.209838in}{1.001836in}}%
\pgfpathlineto{\pgfqpoint{3.209838in}{1.001836in}}%
\pgfpathclose%
\pgfusepath{stroke}%
\end{pgfscope}%
\begin{pgfscope}%
\pgfpathrectangle{\pgfqpoint{0.847223in}{0.554012in}}{\pgfqpoint{6.200000in}{4.620000in}}%
\pgfusepath{clip}%
\pgfsetbuttcap%
\pgfsetroundjoin%
\pgfsetlinewidth{1.003750pt}%
\definecolor{currentstroke}{rgb}{1.000000,0.000000,0.000000}%
\pgfsetstrokecolor{currentstroke}%
\pgfsetdash{}{0pt}%
\pgfpathmoveto{\pgfqpoint{3.215171in}{1.000148in}}%
\pgfpathcurveto{\pgfqpoint{3.226221in}{1.000148in}}{\pgfqpoint{3.236820in}{1.004538in}}{\pgfqpoint{3.244634in}{1.012352in}}%
\pgfpathcurveto{\pgfqpoint{3.252447in}{1.020166in}}{\pgfqpoint{3.256837in}{1.030765in}}{\pgfqpoint{3.256837in}{1.041815in}}%
\pgfpathcurveto{\pgfqpoint{3.256837in}{1.052865in}}{\pgfqpoint{3.252447in}{1.063464in}}{\pgfqpoint{3.244634in}{1.071277in}}%
\pgfpathcurveto{\pgfqpoint{3.236820in}{1.079091in}}{\pgfqpoint{3.226221in}{1.083481in}}{\pgfqpoint{3.215171in}{1.083481in}}%
\pgfpathcurveto{\pgfqpoint{3.204121in}{1.083481in}}{\pgfqpoint{3.193522in}{1.079091in}}{\pgfqpoint{3.185708in}{1.071277in}}%
\pgfpathcurveto{\pgfqpoint{3.177894in}{1.063464in}}{\pgfqpoint{3.173504in}{1.052865in}}{\pgfqpoint{3.173504in}{1.041815in}}%
\pgfpathcurveto{\pgfqpoint{3.173504in}{1.030765in}}{\pgfqpoint{3.177894in}{1.020166in}}{\pgfqpoint{3.185708in}{1.012352in}}%
\pgfpathcurveto{\pgfqpoint{3.193522in}{1.004538in}}{\pgfqpoint{3.204121in}{1.000148in}}{\pgfqpoint{3.215171in}{1.000148in}}%
\pgfpathlineto{\pgfqpoint{3.215171in}{1.000148in}}%
\pgfpathclose%
\pgfusepath{stroke}%
\end{pgfscope}%
\begin{pgfscope}%
\pgfpathrectangle{\pgfqpoint{0.847223in}{0.554012in}}{\pgfqpoint{6.200000in}{4.620000in}}%
\pgfusepath{clip}%
\pgfsetbuttcap%
\pgfsetroundjoin%
\pgfsetlinewidth{1.003750pt}%
\definecolor{currentstroke}{rgb}{1.000000,0.000000,0.000000}%
\pgfsetstrokecolor{currentstroke}%
\pgfsetdash{}{0pt}%
\pgfpathmoveto{\pgfqpoint{3.220504in}{0.998466in}}%
\pgfpathcurveto{\pgfqpoint{3.231554in}{0.998466in}}{\pgfqpoint{3.242153in}{1.002857in}}{\pgfqpoint{3.249967in}{1.010670in}}%
\pgfpathcurveto{\pgfqpoint{3.257780in}{1.018484in}}{\pgfqpoint{3.262171in}{1.029083in}}{\pgfqpoint{3.262171in}{1.040133in}}%
\pgfpathcurveto{\pgfqpoint{3.262171in}{1.051183in}}{\pgfqpoint{3.257780in}{1.061782in}}{\pgfqpoint{3.249967in}{1.069596in}}%
\pgfpathcurveto{\pgfqpoint{3.242153in}{1.077410in}}{\pgfqpoint{3.231554in}{1.081800in}}{\pgfqpoint{3.220504in}{1.081800in}}%
\pgfpathcurveto{\pgfqpoint{3.209454in}{1.081800in}}{\pgfqpoint{3.198855in}{1.077410in}}{\pgfqpoint{3.191041in}{1.069596in}}%
\pgfpathcurveto{\pgfqpoint{3.183228in}{1.061782in}}{\pgfqpoint{3.178837in}{1.051183in}}{\pgfqpoint{3.178837in}{1.040133in}}%
\pgfpathcurveto{\pgfqpoint{3.178837in}{1.029083in}}{\pgfqpoint{3.183228in}{1.018484in}}{\pgfqpoint{3.191041in}{1.010670in}}%
\pgfpathcurveto{\pgfqpoint{3.198855in}{1.002857in}}{\pgfqpoint{3.209454in}{0.998466in}}{\pgfqpoint{3.220504in}{0.998466in}}%
\pgfpathlineto{\pgfqpoint{3.220504in}{0.998466in}}%
\pgfpathclose%
\pgfusepath{stroke}%
\end{pgfscope}%
\begin{pgfscope}%
\pgfpathrectangle{\pgfqpoint{0.847223in}{0.554012in}}{\pgfqpoint{6.200000in}{4.620000in}}%
\pgfusepath{clip}%
\pgfsetbuttcap%
\pgfsetroundjoin%
\pgfsetlinewidth{1.003750pt}%
\definecolor{currentstroke}{rgb}{1.000000,0.000000,0.000000}%
\pgfsetstrokecolor{currentstroke}%
\pgfsetdash{}{0pt}%
\pgfpathmoveto{\pgfqpoint{3.225837in}{0.996791in}}%
\pgfpathcurveto{\pgfqpoint{3.236887in}{0.996791in}}{\pgfqpoint{3.247486in}{1.001181in}}{\pgfqpoint{3.255300in}{1.008995in}}%
\pgfpathcurveto{\pgfqpoint{3.263114in}{1.016809in}}{\pgfqpoint{3.267504in}{1.027408in}}{\pgfqpoint{3.267504in}{1.038458in}}%
\pgfpathcurveto{\pgfqpoint{3.267504in}{1.049508in}}{\pgfqpoint{3.263114in}{1.060107in}}{\pgfqpoint{3.255300in}{1.067921in}}%
\pgfpathcurveto{\pgfqpoint{3.247486in}{1.075734in}}{\pgfqpoint{3.236887in}{1.080125in}}{\pgfqpoint{3.225837in}{1.080125in}}%
\pgfpathcurveto{\pgfqpoint{3.214787in}{1.080125in}}{\pgfqpoint{3.204188in}{1.075734in}}{\pgfqpoint{3.196374in}{1.067921in}}%
\pgfpathcurveto{\pgfqpoint{3.188561in}{1.060107in}}{\pgfqpoint{3.184171in}{1.049508in}}{\pgfqpoint{3.184171in}{1.038458in}}%
\pgfpathcurveto{\pgfqpoint{3.184171in}{1.027408in}}{\pgfqpoint{3.188561in}{1.016809in}}{\pgfqpoint{3.196374in}{1.008995in}}%
\pgfpathcurveto{\pgfqpoint{3.204188in}{1.001181in}}{\pgfqpoint{3.214787in}{0.996791in}}{\pgfqpoint{3.225837in}{0.996791in}}%
\pgfpathlineto{\pgfqpoint{3.225837in}{0.996791in}}%
\pgfpathclose%
\pgfusepath{stroke}%
\end{pgfscope}%
\begin{pgfscope}%
\pgfpathrectangle{\pgfqpoint{0.847223in}{0.554012in}}{\pgfqpoint{6.200000in}{4.620000in}}%
\pgfusepath{clip}%
\pgfsetbuttcap%
\pgfsetroundjoin%
\pgfsetlinewidth{1.003750pt}%
\definecolor{currentstroke}{rgb}{1.000000,0.000000,0.000000}%
\pgfsetstrokecolor{currentstroke}%
\pgfsetdash{}{0pt}%
\pgfpathmoveto{\pgfqpoint{3.231170in}{0.995122in}}%
\pgfpathcurveto{\pgfqpoint{3.242221in}{0.995122in}}{\pgfqpoint{3.252820in}{0.999512in}}{\pgfqpoint{3.260633in}{1.007326in}}%
\pgfpathcurveto{\pgfqpoint{3.268447in}{1.015140in}}{\pgfqpoint{3.272837in}{1.025739in}}{\pgfqpoint{3.272837in}{1.036789in}}%
\pgfpathcurveto{\pgfqpoint{3.272837in}{1.047839in}}{\pgfqpoint{3.268447in}{1.058438in}}{\pgfqpoint{3.260633in}{1.066251in}}%
\pgfpathcurveto{\pgfqpoint{3.252820in}{1.074065in}}{\pgfqpoint{3.242221in}{1.078455in}}{\pgfqpoint{3.231170in}{1.078455in}}%
\pgfpathcurveto{\pgfqpoint{3.220120in}{1.078455in}}{\pgfqpoint{3.209521in}{1.074065in}}{\pgfqpoint{3.201708in}{1.066251in}}%
\pgfpathcurveto{\pgfqpoint{3.193894in}{1.058438in}}{\pgfqpoint{3.189504in}{1.047839in}}{\pgfqpoint{3.189504in}{1.036789in}}%
\pgfpathcurveto{\pgfqpoint{3.189504in}{1.025739in}}{\pgfqpoint{3.193894in}{1.015140in}}{\pgfqpoint{3.201708in}{1.007326in}}%
\pgfpathcurveto{\pgfqpoint{3.209521in}{0.999512in}}{\pgfqpoint{3.220120in}{0.995122in}}{\pgfqpoint{3.231170in}{0.995122in}}%
\pgfpathlineto{\pgfqpoint{3.231170in}{0.995122in}}%
\pgfpathclose%
\pgfusepath{stroke}%
\end{pgfscope}%
\begin{pgfscope}%
\pgfpathrectangle{\pgfqpoint{0.847223in}{0.554012in}}{\pgfqpoint{6.200000in}{4.620000in}}%
\pgfusepath{clip}%
\pgfsetbuttcap%
\pgfsetroundjoin%
\pgfsetlinewidth{1.003750pt}%
\definecolor{currentstroke}{rgb}{1.000000,0.000000,0.000000}%
\pgfsetstrokecolor{currentstroke}%
\pgfsetdash{}{0pt}%
\pgfpathmoveto{\pgfqpoint{3.236504in}{0.993459in}}%
\pgfpathcurveto{\pgfqpoint{3.247554in}{0.993459in}}{\pgfqpoint{3.258153in}{0.997849in}}{\pgfqpoint{3.265966in}{1.005663in}}%
\pgfpathcurveto{\pgfqpoint{3.273780in}{1.013476in}}{\pgfqpoint{3.278170in}{1.024075in}}{\pgfqpoint{3.278170in}{1.035126in}}%
\pgfpathcurveto{\pgfqpoint{3.278170in}{1.046176in}}{\pgfqpoint{3.273780in}{1.056775in}}{\pgfqpoint{3.265966in}{1.064588in}}%
\pgfpathcurveto{\pgfqpoint{3.258153in}{1.072402in}}{\pgfqpoint{3.247554in}{1.076792in}}{\pgfqpoint{3.236504in}{1.076792in}}%
\pgfpathcurveto{\pgfqpoint{3.225454in}{1.076792in}}{\pgfqpoint{3.214855in}{1.072402in}}{\pgfqpoint{3.207041in}{1.064588in}}%
\pgfpathcurveto{\pgfqpoint{3.199227in}{1.056775in}}{\pgfqpoint{3.194837in}{1.046176in}}{\pgfqpoint{3.194837in}{1.035126in}}%
\pgfpathcurveto{\pgfqpoint{3.194837in}{1.024075in}}{\pgfqpoint{3.199227in}{1.013476in}}{\pgfqpoint{3.207041in}{1.005663in}}%
\pgfpathcurveto{\pgfqpoint{3.214855in}{0.997849in}}{\pgfqpoint{3.225454in}{0.993459in}}{\pgfqpoint{3.236504in}{0.993459in}}%
\pgfpathlineto{\pgfqpoint{3.236504in}{0.993459in}}%
\pgfpathclose%
\pgfusepath{stroke}%
\end{pgfscope}%
\begin{pgfscope}%
\pgfpathrectangle{\pgfqpoint{0.847223in}{0.554012in}}{\pgfqpoint{6.200000in}{4.620000in}}%
\pgfusepath{clip}%
\pgfsetbuttcap%
\pgfsetroundjoin%
\pgfsetlinewidth{1.003750pt}%
\definecolor{currentstroke}{rgb}{1.000000,0.000000,0.000000}%
\pgfsetstrokecolor{currentstroke}%
\pgfsetdash{}{0pt}%
\pgfpathmoveto{\pgfqpoint{3.241837in}{0.991802in}}%
\pgfpathcurveto{\pgfqpoint{3.252887in}{0.991802in}}{\pgfqpoint{3.263486in}{0.996192in}}{\pgfqpoint{3.271300in}{1.004006in}}%
\pgfpathcurveto{\pgfqpoint{3.279113in}{1.011819in}}{\pgfqpoint{3.283504in}{1.022418in}}{\pgfqpoint{3.283504in}{1.033469in}}%
\pgfpathcurveto{\pgfqpoint{3.283504in}{1.044519in}}{\pgfqpoint{3.279113in}{1.055118in}}{\pgfqpoint{3.271300in}{1.062931in}}%
\pgfpathcurveto{\pgfqpoint{3.263486in}{1.070745in}}{\pgfqpoint{3.252887in}{1.075135in}}{\pgfqpoint{3.241837in}{1.075135in}}%
\pgfpathcurveto{\pgfqpoint{3.230787in}{1.075135in}}{\pgfqpoint{3.220188in}{1.070745in}}{\pgfqpoint{3.212374in}{1.062931in}}%
\pgfpathcurveto{\pgfqpoint{3.204560in}{1.055118in}}{\pgfqpoint{3.200170in}{1.044519in}}{\pgfqpoint{3.200170in}{1.033469in}}%
\pgfpathcurveto{\pgfqpoint{3.200170in}{1.022418in}}{\pgfqpoint{3.204560in}{1.011819in}}{\pgfqpoint{3.212374in}{1.004006in}}%
\pgfpathcurveto{\pgfqpoint{3.220188in}{0.996192in}}{\pgfqpoint{3.230787in}{0.991802in}}{\pgfqpoint{3.241837in}{0.991802in}}%
\pgfpathlineto{\pgfqpoint{3.241837in}{0.991802in}}%
\pgfpathclose%
\pgfusepath{stroke}%
\end{pgfscope}%
\begin{pgfscope}%
\pgfpathrectangle{\pgfqpoint{0.847223in}{0.554012in}}{\pgfqpoint{6.200000in}{4.620000in}}%
\pgfusepath{clip}%
\pgfsetbuttcap%
\pgfsetroundjoin%
\pgfsetlinewidth{1.003750pt}%
\definecolor{currentstroke}{rgb}{1.000000,0.000000,0.000000}%
\pgfsetstrokecolor{currentstroke}%
\pgfsetdash{}{0pt}%
\pgfpathmoveto{\pgfqpoint{3.247170in}{0.990151in}}%
\pgfpathcurveto{\pgfqpoint{3.258220in}{0.990151in}}{\pgfqpoint{3.268819in}{0.994541in}}{\pgfqpoint{3.276633in}{1.002355in}}%
\pgfpathcurveto{\pgfqpoint{3.284447in}{1.010169in}}{\pgfqpoint{3.288837in}{1.020768in}}{\pgfqpoint{3.288837in}{1.031818in}}%
\pgfpathcurveto{\pgfqpoint{3.288837in}{1.042868in}}{\pgfqpoint{3.284447in}{1.053467in}}{\pgfqpoint{3.276633in}{1.061280in}}%
\pgfpathcurveto{\pgfqpoint{3.268819in}{1.069094in}}{\pgfqpoint{3.258220in}{1.073484in}}{\pgfqpoint{3.247170in}{1.073484in}}%
\pgfpathcurveto{\pgfqpoint{3.236120in}{1.073484in}}{\pgfqpoint{3.225521in}{1.069094in}}{\pgfqpoint{3.217707in}{1.061280in}}%
\pgfpathcurveto{\pgfqpoint{3.209894in}{1.053467in}}{\pgfqpoint{3.205503in}{1.042868in}}{\pgfqpoint{3.205503in}{1.031818in}}%
\pgfpathcurveto{\pgfqpoint{3.205503in}{1.020768in}}{\pgfqpoint{3.209894in}{1.010169in}}{\pgfqpoint{3.217707in}{1.002355in}}%
\pgfpathcurveto{\pgfqpoint{3.225521in}{0.994541in}}{\pgfqpoint{3.236120in}{0.990151in}}{\pgfqpoint{3.247170in}{0.990151in}}%
\pgfpathlineto{\pgfqpoint{3.247170in}{0.990151in}}%
\pgfpathclose%
\pgfusepath{stroke}%
\end{pgfscope}%
\begin{pgfscope}%
\pgfpathrectangle{\pgfqpoint{0.847223in}{0.554012in}}{\pgfqpoint{6.200000in}{4.620000in}}%
\pgfusepath{clip}%
\pgfsetbuttcap%
\pgfsetroundjoin%
\pgfsetlinewidth{1.003750pt}%
\definecolor{currentstroke}{rgb}{1.000000,0.000000,0.000000}%
\pgfsetstrokecolor{currentstroke}%
\pgfsetdash{}{0pt}%
\pgfpathmoveto{\pgfqpoint{3.252503in}{0.988506in}}%
\pgfpathcurveto{\pgfqpoint{3.263553in}{0.988506in}}{\pgfqpoint{3.274152in}{0.992896in}}{\pgfqpoint{3.281966in}{1.000710in}}%
\pgfpathcurveto{\pgfqpoint{3.289780in}{1.008524in}}{\pgfqpoint{3.294170in}{1.019123in}}{\pgfqpoint{3.294170in}{1.030173in}}%
\pgfpathcurveto{\pgfqpoint{3.294170in}{1.041223in}}{\pgfqpoint{3.289780in}{1.051822in}}{\pgfqpoint{3.281966in}{1.059636in}}%
\pgfpathcurveto{\pgfqpoint{3.274152in}{1.067449in}}{\pgfqpoint{3.263553in}{1.071839in}}{\pgfqpoint{3.252503in}{1.071839in}}%
\pgfpathcurveto{\pgfqpoint{3.241453in}{1.071839in}}{\pgfqpoint{3.230854in}{1.067449in}}{\pgfqpoint{3.223041in}{1.059636in}}%
\pgfpathcurveto{\pgfqpoint{3.215227in}{1.051822in}}{\pgfqpoint{3.210837in}{1.041223in}}{\pgfqpoint{3.210837in}{1.030173in}}%
\pgfpathcurveto{\pgfqpoint{3.210837in}{1.019123in}}{\pgfqpoint{3.215227in}{1.008524in}}{\pgfqpoint{3.223041in}{1.000710in}}%
\pgfpathcurveto{\pgfqpoint{3.230854in}{0.992896in}}{\pgfqpoint{3.241453in}{0.988506in}}{\pgfqpoint{3.252503in}{0.988506in}}%
\pgfpathlineto{\pgfqpoint{3.252503in}{0.988506in}}%
\pgfpathclose%
\pgfusepath{stroke}%
\end{pgfscope}%
\begin{pgfscope}%
\pgfpathrectangle{\pgfqpoint{0.847223in}{0.554012in}}{\pgfqpoint{6.200000in}{4.620000in}}%
\pgfusepath{clip}%
\pgfsetbuttcap%
\pgfsetroundjoin%
\pgfsetlinewidth{1.003750pt}%
\definecolor{currentstroke}{rgb}{1.000000,0.000000,0.000000}%
\pgfsetstrokecolor{currentstroke}%
\pgfsetdash{}{0pt}%
\pgfpathmoveto{\pgfqpoint{3.257837in}{0.986867in}}%
\pgfpathcurveto{\pgfqpoint{3.268887in}{0.986867in}}{\pgfqpoint{3.279486in}{0.991257in}}{\pgfqpoint{3.287299in}{0.999071in}}%
\pgfpathcurveto{\pgfqpoint{3.295113in}{1.006885in}}{\pgfqpoint{3.299503in}{1.017484in}}{\pgfqpoint{3.299503in}{1.028534in}}%
\pgfpathcurveto{\pgfqpoint{3.299503in}{1.039584in}}{\pgfqpoint{3.295113in}{1.050183in}}{\pgfqpoint{3.287299in}{1.057997in}}%
\pgfpathcurveto{\pgfqpoint{3.279486in}{1.065810in}}{\pgfqpoint{3.268887in}{1.070200in}}{\pgfqpoint{3.257837in}{1.070200in}}%
\pgfpathcurveto{\pgfqpoint{3.246786in}{1.070200in}}{\pgfqpoint{3.236187in}{1.065810in}}{\pgfqpoint{3.228374in}{1.057997in}}%
\pgfpathcurveto{\pgfqpoint{3.220560in}{1.050183in}}{\pgfqpoint{3.216170in}{1.039584in}}{\pgfqpoint{3.216170in}{1.028534in}}%
\pgfpathcurveto{\pgfqpoint{3.216170in}{1.017484in}}{\pgfqpoint{3.220560in}{1.006885in}}{\pgfqpoint{3.228374in}{0.999071in}}%
\pgfpathcurveto{\pgfqpoint{3.236187in}{0.991257in}}{\pgfqpoint{3.246786in}{0.986867in}}{\pgfqpoint{3.257837in}{0.986867in}}%
\pgfpathlineto{\pgfqpoint{3.257837in}{0.986867in}}%
\pgfpathclose%
\pgfusepath{stroke}%
\end{pgfscope}%
\begin{pgfscope}%
\pgfpathrectangle{\pgfqpoint{0.847223in}{0.554012in}}{\pgfqpoint{6.200000in}{4.620000in}}%
\pgfusepath{clip}%
\pgfsetbuttcap%
\pgfsetroundjoin%
\pgfsetlinewidth{1.003750pt}%
\definecolor{currentstroke}{rgb}{1.000000,0.000000,0.000000}%
\pgfsetstrokecolor{currentstroke}%
\pgfsetdash{}{0pt}%
\pgfpathmoveto{\pgfqpoint{3.263170in}{0.985234in}}%
\pgfpathcurveto{\pgfqpoint{3.274220in}{0.985234in}}{\pgfqpoint{3.284819in}{0.989624in}}{\pgfqpoint{3.292633in}{0.997438in}}%
\pgfpathcurveto{\pgfqpoint{3.300446in}{1.005252in}}{\pgfqpoint{3.304836in}{1.015851in}}{\pgfqpoint{3.304836in}{1.026901in}}%
\pgfpathcurveto{\pgfqpoint{3.304836in}{1.037951in}}{\pgfqpoint{3.300446in}{1.048550in}}{\pgfqpoint{3.292633in}{1.056364in}}%
\pgfpathcurveto{\pgfqpoint{3.284819in}{1.064177in}}{\pgfqpoint{3.274220in}{1.068567in}}{\pgfqpoint{3.263170in}{1.068567in}}%
\pgfpathcurveto{\pgfqpoint{3.252120in}{1.068567in}}{\pgfqpoint{3.241521in}{1.064177in}}{\pgfqpoint{3.233707in}{1.056364in}}%
\pgfpathcurveto{\pgfqpoint{3.225893in}{1.048550in}}{\pgfqpoint{3.221503in}{1.037951in}}{\pgfqpoint{3.221503in}{1.026901in}}%
\pgfpathcurveto{\pgfqpoint{3.221503in}{1.015851in}}{\pgfqpoint{3.225893in}{1.005252in}}{\pgfqpoint{3.233707in}{0.997438in}}%
\pgfpathcurveto{\pgfqpoint{3.241521in}{0.989624in}}{\pgfqpoint{3.252120in}{0.985234in}}{\pgfqpoint{3.263170in}{0.985234in}}%
\pgfpathlineto{\pgfqpoint{3.263170in}{0.985234in}}%
\pgfpathclose%
\pgfusepath{stroke}%
\end{pgfscope}%
\begin{pgfscope}%
\pgfpathrectangle{\pgfqpoint{0.847223in}{0.554012in}}{\pgfqpoint{6.200000in}{4.620000in}}%
\pgfusepath{clip}%
\pgfsetbuttcap%
\pgfsetroundjoin%
\pgfsetlinewidth{1.003750pt}%
\definecolor{currentstroke}{rgb}{1.000000,0.000000,0.000000}%
\pgfsetstrokecolor{currentstroke}%
\pgfsetdash{}{0pt}%
\pgfpathmoveto{\pgfqpoint{3.268503in}{0.983607in}}%
\pgfpathcurveto{\pgfqpoint{3.279553in}{0.983607in}}{\pgfqpoint{3.290152in}{0.987997in}}{\pgfqpoint{3.297966in}{0.995811in}}%
\pgfpathcurveto{\pgfqpoint{3.305779in}{1.003625in}}{\pgfqpoint{3.310170in}{1.014224in}}{\pgfqpoint{3.310170in}{1.025274in}}%
\pgfpathcurveto{\pgfqpoint{3.310170in}{1.036324in}}{\pgfqpoint{3.305779in}{1.046923in}}{\pgfqpoint{3.297966in}{1.054737in}}%
\pgfpathcurveto{\pgfqpoint{3.290152in}{1.062550in}}{\pgfqpoint{3.279553in}{1.066940in}}{\pgfqpoint{3.268503in}{1.066940in}}%
\pgfpathcurveto{\pgfqpoint{3.257453in}{1.066940in}}{\pgfqpoint{3.246854in}{1.062550in}}{\pgfqpoint{3.239040in}{1.054737in}}%
\pgfpathcurveto{\pgfqpoint{3.231227in}{1.046923in}}{\pgfqpoint{3.226836in}{1.036324in}}{\pgfqpoint{3.226836in}{1.025274in}}%
\pgfpathcurveto{\pgfqpoint{3.226836in}{1.014224in}}{\pgfqpoint{3.231227in}{1.003625in}}{\pgfqpoint{3.239040in}{0.995811in}}%
\pgfpathcurveto{\pgfqpoint{3.246854in}{0.987997in}}{\pgfqpoint{3.257453in}{0.983607in}}{\pgfqpoint{3.268503in}{0.983607in}}%
\pgfpathlineto{\pgfqpoint{3.268503in}{0.983607in}}%
\pgfpathclose%
\pgfusepath{stroke}%
\end{pgfscope}%
\begin{pgfscope}%
\pgfpathrectangle{\pgfqpoint{0.847223in}{0.554012in}}{\pgfqpoint{6.200000in}{4.620000in}}%
\pgfusepath{clip}%
\pgfsetbuttcap%
\pgfsetroundjoin%
\pgfsetlinewidth{1.003750pt}%
\definecolor{currentstroke}{rgb}{1.000000,0.000000,0.000000}%
\pgfsetstrokecolor{currentstroke}%
\pgfsetdash{}{0pt}%
\pgfpathmoveto{\pgfqpoint{3.273836in}{0.981986in}}%
\pgfpathcurveto{\pgfqpoint{3.284886in}{0.981986in}}{\pgfqpoint{3.295485in}{0.986376in}}{\pgfqpoint{3.303299in}{0.994190in}}%
\pgfpathcurveto{\pgfqpoint{3.311113in}{1.002003in}}{\pgfqpoint{3.315503in}{1.012602in}}{\pgfqpoint{3.315503in}{1.023653in}}%
\pgfpathcurveto{\pgfqpoint{3.315503in}{1.034703in}}{\pgfqpoint{3.311113in}{1.045302in}}{\pgfqpoint{3.303299in}{1.053115in}}%
\pgfpathcurveto{\pgfqpoint{3.295485in}{1.060929in}}{\pgfqpoint{3.284886in}{1.065319in}}{\pgfqpoint{3.273836in}{1.065319in}}%
\pgfpathcurveto{\pgfqpoint{3.262786in}{1.065319in}}{\pgfqpoint{3.252187in}{1.060929in}}{\pgfqpoint{3.244373in}{1.053115in}}%
\pgfpathcurveto{\pgfqpoint{3.236560in}{1.045302in}}{\pgfqpoint{3.232170in}{1.034703in}}{\pgfqpoint{3.232170in}{1.023653in}}%
\pgfpathcurveto{\pgfqpoint{3.232170in}{1.012602in}}{\pgfqpoint{3.236560in}{1.002003in}}{\pgfqpoint{3.244373in}{0.994190in}}%
\pgfpathcurveto{\pgfqpoint{3.252187in}{0.986376in}}{\pgfqpoint{3.262786in}{0.981986in}}{\pgfqpoint{3.273836in}{0.981986in}}%
\pgfpathlineto{\pgfqpoint{3.273836in}{0.981986in}}%
\pgfpathclose%
\pgfusepath{stroke}%
\end{pgfscope}%
\begin{pgfscope}%
\pgfpathrectangle{\pgfqpoint{0.847223in}{0.554012in}}{\pgfqpoint{6.200000in}{4.620000in}}%
\pgfusepath{clip}%
\pgfsetbuttcap%
\pgfsetroundjoin%
\pgfsetlinewidth{1.003750pt}%
\definecolor{currentstroke}{rgb}{1.000000,0.000000,0.000000}%
\pgfsetstrokecolor{currentstroke}%
\pgfsetdash{}{0pt}%
\pgfpathmoveto{\pgfqpoint{3.279169in}{0.980371in}}%
\pgfpathcurveto{\pgfqpoint{3.290220in}{0.980371in}}{\pgfqpoint{3.300819in}{0.984761in}}{\pgfqpoint{3.308632in}{0.992574in}}%
\pgfpathcurveto{\pgfqpoint{3.316446in}{1.000388in}}{\pgfqpoint{3.320836in}{1.010987in}}{\pgfqpoint{3.320836in}{1.022037in}}%
\pgfpathcurveto{\pgfqpoint{3.320836in}{1.033087in}}{\pgfqpoint{3.316446in}{1.043686in}}{\pgfqpoint{3.308632in}{1.051500in}}%
\pgfpathcurveto{\pgfqpoint{3.300819in}{1.059314in}}{\pgfqpoint{3.290220in}{1.063704in}}{\pgfqpoint{3.279169in}{1.063704in}}%
\pgfpathcurveto{\pgfqpoint{3.268119in}{1.063704in}}{\pgfqpoint{3.257520in}{1.059314in}}{\pgfqpoint{3.249707in}{1.051500in}}%
\pgfpathcurveto{\pgfqpoint{3.241893in}{1.043686in}}{\pgfqpoint{3.237503in}{1.033087in}}{\pgfqpoint{3.237503in}{1.022037in}}%
\pgfpathcurveto{\pgfqpoint{3.237503in}{1.010987in}}{\pgfqpoint{3.241893in}{1.000388in}}{\pgfqpoint{3.249707in}{0.992574in}}%
\pgfpathcurveto{\pgfqpoint{3.257520in}{0.984761in}}{\pgfqpoint{3.268119in}{0.980371in}}{\pgfqpoint{3.279169in}{0.980371in}}%
\pgfpathlineto{\pgfqpoint{3.279169in}{0.980371in}}%
\pgfpathclose%
\pgfusepath{stroke}%
\end{pgfscope}%
\begin{pgfscope}%
\pgfpathrectangle{\pgfqpoint{0.847223in}{0.554012in}}{\pgfqpoint{6.200000in}{4.620000in}}%
\pgfusepath{clip}%
\pgfsetbuttcap%
\pgfsetroundjoin%
\pgfsetlinewidth{1.003750pt}%
\definecolor{currentstroke}{rgb}{1.000000,0.000000,0.000000}%
\pgfsetstrokecolor{currentstroke}%
\pgfsetdash{}{0pt}%
\pgfpathmoveto{\pgfqpoint{3.284503in}{0.978761in}}%
\pgfpathcurveto{\pgfqpoint{3.295553in}{0.978761in}}{\pgfqpoint{3.306152in}{0.983151in}}{\pgfqpoint{3.313965in}{0.990965in}}%
\pgfpathcurveto{\pgfqpoint{3.321779in}{0.998779in}}{\pgfqpoint{3.326169in}{1.009378in}}{\pgfqpoint{3.326169in}{1.020428in}}%
\pgfpathcurveto{\pgfqpoint{3.326169in}{1.031478in}}{\pgfqpoint{3.321779in}{1.042077in}}{\pgfqpoint{3.313965in}{1.049890in}}%
\pgfpathcurveto{\pgfqpoint{3.306152in}{1.057704in}}{\pgfqpoint{3.295553in}{1.062094in}}{\pgfqpoint{3.284503in}{1.062094in}}%
\pgfpathcurveto{\pgfqpoint{3.273452in}{1.062094in}}{\pgfqpoint{3.262853in}{1.057704in}}{\pgfqpoint{3.255040in}{1.049890in}}%
\pgfpathcurveto{\pgfqpoint{3.247226in}{1.042077in}}{\pgfqpoint{3.242836in}{1.031478in}}{\pgfqpoint{3.242836in}{1.020428in}}%
\pgfpathcurveto{\pgfqpoint{3.242836in}{1.009378in}}{\pgfqpoint{3.247226in}{0.998779in}}{\pgfqpoint{3.255040in}{0.990965in}}%
\pgfpathcurveto{\pgfqpoint{3.262853in}{0.983151in}}{\pgfqpoint{3.273452in}{0.978761in}}{\pgfqpoint{3.284503in}{0.978761in}}%
\pgfpathlineto{\pgfqpoint{3.284503in}{0.978761in}}%
\pgfpathclose%
\pgfusepath{stroke}%
\end{pgfscope}%
\begin{pgfscope}%
\pgfpathrectangle{\pgfqpoint{0.847223in}{0.554012in}}{\pgfqpoint{6.200000in}{4.620000in}}%
\pgfusepath{clip}%
\pgfsetbuttcap%
\pgfsetroundjoin%
\pgfsetlinewidth{1.003750pt}%
\definecolor{currentstroke}{rgb}{1.000000,0.000000,0.000000}%
\pgfsetstrokecolor{currentstroke}%
\pgfsetdash{}{0pt}%
\pgfpathmoveto{\pgfqpoint{3.289836in}{0.977157in}}%
\pgfpathcurveto{\pgfqpoint{3.300886in}{0.977157in}}{\pgfqpoint{3.311485in}{0.981548in}}{\pgfqpoint{3.319299in}{0.989361in}}%
\pgfpathcurveto{\pgfqpoint{3.327112in}{0.997175in}}{\pgfqpoint{3.331502in}{1.007774in}}{\pgfqpoint{3.331502in}{1.018824in}}%
\pgfpathcurveto{\pgfqpoint{3.331502in}{1.029874in}}{\pgfqpoint{3.327112in}{1.040473in}}{\pgfqpoint{3.319299in}{1.048287in}}%
\pgfpathcurveto{\pgfqpoint{3.311485in}{1.056100in}}{\pgfqpoint{3.300886in}{1.060491in}}{\pgfqpoint{3.289836in}{1.060491in}}%
\pgfpathcurveto{\pgfqpoint{3.278786in}{1.060491in}}{\pgfqpoint{3.268187in}{1.056100in}}{\pgfqpoint{3.260373in}{1.048287in}}%
\pgfpathcurveto{\pgfqpoint{3.252559in}{1.040473in}}{\pgfqpoint{3.248169in}{1.029874in}}{\pgfqpoint{3.248169in}{1.018824in}}%
\pgfpathcurveto{\pgfqpoint{3.248169in}{1.007774in}}{\pgfqpoint{3.252559in}{0.997175in}}{\pgfqpoint{3.260373in}{0.989361in}}%
\pgfpathcurveto{\pgfqpoint{3.268187in}{0.981548in}}{\pgfqpoint{3.278786in}{0.977157in}}{\pgfqpoint{3.289836in}{0.977157in}}%
\pgfpathlineto{\pgfqpoint{3.289836in}{0.977157in}}%
\pgfpathclose%
\pgfusepath{stroke}%
\end{pgfscope}%
\begin{pgfscope}%
\pgfpathrectangle{\pgfqpoint{0.847223in}{0.554012in}}{\pgfqpoint{6.200000in}{4.620000in}}%
\pgfusepath{clip}%
\pgfsetbuttcap%
\pgfsetroundjoin%
\pgfsetlinewidth{1.003750pt}%
\definecolor{currentstroke}{rgb}{1.000000,0.000000,0.000000}%
\pgfsetstrokecolor{currentstroke}%
\pgfsetdash{}{0pt}%
\pgfpathmoveto{\pgfqpoint{3.295169in}{0.975559in}}%
\pgfpathcurveto{\pgfqpoint{3.306219in}{0.975559in}}{\pgfqpoint{3.316818in}{0.979950in}}{\pgfqpoint{3.324632in}{0.987763in}}%
\pgfpathcurveto{\pgfqpoint{3.332445in}{0.995577in}}{\pgfqpoint{3.336836in}{1.006176in}}{\pgfqpoint{3.336836in}{1.017226in}}%
\pgfpathcurveto{\pgfqpoint{3.336836in}{1.028276in}}{\pgfqpoint{3.332445in}{1.038875in}}{\pgfqpoint{3.324632in}{1.046689in}}%
\pgfpathcurveto{\pgfqpoint{3.316818in}{1.054502in}}{\pgfqpoint{3.306219in}{1.058893in}}{\pgfqpoint{3.295169in}{1.058893in}}%
\pgfpathcurveto{\pgfqpoint{3.284119in}{1.058893in}}{\pgfqpoint{3.273520in}{1.054502in}}{\pgfqpoint{3.265706in}{1.046689in}}%
\pgfpathcurveto{\pgfqpoint{3.257893in}{1.038875in}}{\pgfqpoint{3.253502in}{1.028276in}}{\pgfqpoint{3.253502in}{1.017226in}}%
\pgfpathcurveto{\pgfqpoint{3.253502in}{1.006176in}}{\pgfqpoint{3.257893in}{0.995577in}}{\pgfqpoint{3.265706in}{0.987763in}}%
\pgfpathcurveto{\pgfqpoint{3.273520in}{0.979950in}}{\pgfqpoint{3.284119in}{0.975559in}}{\pgfqpoint{3.295169in}{0.975559in}}%
\pgfpathlineto{\pgfqpoint{3.295169in}{0.975559in}}%
\pgfpathclose%
\pgfusepath{stroke}%
\end{pgfscope}%
\begin{pgfscope}%
\pgfpathrectangle{\pgfqpoint{0.847223in}{0.554012in}}{\pgfqpoint{6.200000in}{4.620000in}}%
\pgfusepath{clip}%
\pgfsetbuttcap%
\pgfsetroundjoin%
\pgfsetlinewidth{1.003750pt}%
\definecolor{currentstroke}{rgb}{1.000000,0.000000,0.000000}%
\pgfsetstrokecolor{currentstroke}%
\pgfsetdash{}{0pt}%
\pgfpathmoveto{\pgfqpoint{3.300502in}{0.973967in}}%
\pgfpathcurveto{\pgfqpoint{3.311552in}{0.973967in}}{\pgfqpoint{3.322151in}{0.978357in}}{\pgfqpoint{3.329965in}{0.986171in}}%
\pgfpathcurveto{\pgfqpoint{3.337779in}{0.993985in}}{\pgfqpoint{3.342169in}{1.004584in}}{\pgfqpoint{3.342169in}{1.015634in}}%
\pgfpathcurveto{\pgfqpoint{3.342169in}{1.026684in}}{\pgfqpoint{3.337779in}{1.037283in}}{\pgfqpoint{3.329965in}{1.045096in}}%
\pgfpathcurveto{\pgfqpoint{3.322151in}{1.052910in}}{\pgfqpoint{3.311552in}{1.057300in}}{\pgfqpoint{3.300502in}{1.057300in}}%
\pgfpathcurveto{\pgfqpoint{3.289452in}{1.057300in}}{\pgfqpoint{3.278853in}{1.052910in}}{\pgfqpoint{3.271039in}{1.045096in}}%
\pgfpathcurveto{\pgfqpoint{3.263226in}{1.037283in}}{\pgfqpoint{3.258836in}{1.026684in}}{\pgfqpoint{3.258836in}{1.015634in}}%
\pgfpathcurveto{\pgfqpoint{3.258836in}{1.004584in}}{\pgfqpoint{3.263226in}{0.993985in}}{\pgfqpoint{3.271039in}{0.986171in}}%
\pgfpathcurveto{\pgfqpoint{3.278853in}{0.978357in}}{\pgfqpoint{3.289452in}{0.973967in}}{\pgfqpoint{3.300502in}{0.973967in}}%
\pgfpathlineto{\pgfqpoint{3.300502in}{0.973967in}}%
\pgfpathclose%
\pgfusepath{stroke}%
\end{pgfscope}%
\begin{pgfscope}%
\pgfpathrectangle{\pgfqpoint{0.847223in}{0.554012in}}{\pgfqpoint{6.200000in}{4.620000in}}%
\pgfusepath{clip}%
\pgfsetbuttcap%
\pgfsetroundjoin%
\pgfsetlinewidth{1.003750pt}%
\definecolor{currentstroke}{rgb}{1.000000,0.000000,0.000000}%
\pgfsetstrokecolor{currentstroke}%
\pgfsetdash{}{0pt}%
\pgfpathmoveto{\pgfqpoint{3.305835in}{0.972380in}}%
\pgfpathcurveto{\pgfqpoint{3.316886in}{0.972380in}}{\pgfqpoint{3.327485in}{0.976771in}}{\pgfqpoint{3.335298in}{0.984584in}}%
\pgfpathcurveto{\pgfqpoint{3.343112in}{0.992398in}}{\pgfqpoint{3.347502in}{1.002997in}}{\pgfqpoint{3.347502in}{1.014047in}}%
\pgfpathcurveto{\pgfqpoint{3.347502in}{1.025097in}}{\pgfqpoint{3.343112in}{1.035696in}}{\pgfqpoint{3.335298in}{1.043510in}}%
\pgfpathcurveto{\pgfqpoint{3.327485in}{1.051324in}}{\pgfqpoint{3.316886in}{1.055714in}}{\pgfqpoint{3.305835in}{1.055714in}}%
\pgfpathcurveto{\pgfqpoint{3.294785in}{1.055714in}}{\pgfqpoint{3.284186in}{1.051324in}}{\pgfqpoint{3.276373in}{1.043510in}}%
\pgfpathcurveto{\pgfqpoint{3.268559in}{1.035696in}}{\pgfqpoint{3.264169in}{1.025097in}}{\pgfqpoint{3.264169in}{1.014047in}}%
\pgfpathcurveto{\pgfqpoint{3.264169in}{1.002997in}}{\pgfqpoint{3.268559in}{0.992398in}}{\pgfqpoint{3.276373in}{0.984584in}}%
\pgfpathcurveto{\pgfqpoint{3.284186in}{0.976771in}}{\pgfqpoint{3.294785in}{0.972380in}}{\pgfqpoint{3.305835in}{0.972380in}}%
\pgfpathlineto{\pgfqpoint{3.305835in}{0.972380in}}%
\pgfpathclose%
\pgfusepath{stroke}%
\end{pgfscope}%
\begin{pgfscope}%
\pgfpathrectangle{\pgfqpoint{0.847223in}{0.554012in}}{\pgfqpoint{6.200000in}{4.620000in}}%
\pgfusepath{clip}%
\pgfsetbuttcap%
\pgfsetroundjoin%
\pgfsetlinewidth{1.003750pt}%
\definecolor{currentstroke}{rgb}{1.000000,0.000000,0.000000}%
\pgfsetstrokecolor{currentstroke}%
\pgfsetdash{}{0pt}%
\pgfpathmoveto{\pgfqpoint{3.311169in}{0.970800in}}%
\pgfpathcurveto{\pgfqpoint{3.322219in}{0.970800in}}{\pgfqpoint{3.332818in}{0.975190in}}{\pgfqpoint{3.340631in}{0.983003in}}%
\pgfpathcurveto{\pgfqpoint{3.348445in}{0.990817in}}{\pgfqpoint{3.352835in}{1.001416in}}{\pgfqpoint{3.352835in}{1.012466in}}%
\pgfpathcurveto{\pgfqpoint{3.352835in}{1.023516in}}{\pgfqpoint{3.348445in}{1.034115in}}{\pgfqpoint{3.340631in}{1.041929in}}%
\pgfpathcurveto{\pgfqpoint{3.332818in}{1.049743in}}{\pgfqpoint{3.322219in}{1.054133in}}{\pgfqpoint{3.311169in}{1.054133in}}%
\pgfpathcurveto{\pgfqpoint{3.300119in}{1.054133in}}{\pgfqpoint{3.289520in}{1.049743in}}{\pgfqpoint{3.281706in}{1.041929in}}%
\pgfpathcurveto{\pgfqpoint{3.273892in}{1.034115in}}{\pgfqpoint{3.269502in}{1.023516in}}{\pgfqpoint{3.269502in}{1.012466in}}%
\pgfpathcurveto{\pgfqpoint{3.269502in}{1.001416in}}{\pgfqpoint{3.273892in}{0.990817in}}{\pgfqpoint{3.281706in}{0.983003in}}%
\pgfpathcurveto{\pgfqpoint{3.289520in}{0.975190in}}{\pgfqpoint{3.300119in}{0.970800in}}{\pgfqpoint{3.311169in}{0.970800in}}%
\pgfpathlineto{\pgfqpoint{3.311169in}{0.970800in}}%
\pgfpathclose%
\pgfusepath{stroke}%
\end{pgfscope}%
\begin{pgfscope}%
\pgfpathrectangle{\pgfqpoint{0.847223in}{0.554012in}}{\pgfqpoint{6.200000in}{4.620000in}}%
\pgfusepath{clip}%
\pgfsetbuttcap%
\pgfsetroundjoin%
\pgfsetlinewidth{1.003750pt}%
\definecolor{currentstroke}{rgb}{1.000000,0.000000,0.000000}%
\pgfsetstrokecolor{currentstroke}%
\pgfsetdash{}{0pt}%
\pgfpathmoveto{\pgfqpoint{3.316502in}{0.969224in}}%
\pgfpathcurveto{\pgfqpoint{3.327552in}{0.969224in}}{\pgfqpoint{3.338151in}{0.973615in}}{\pgfqpoint{3.345965in}{0.981428in}}%
\pgfpathcurveto{\pgfqpoint{3.353778in}{0.989242in}}{\pgfqpoint{3.358169in}{0.999841in}}{\pgfqpoint{3.358169in}{1.010891in}}%
\pgfpathcurveto{\pgfqpoint{3.358169in}{1.021941in}}{\pgfqpoint{3.353778in}{1.032540in}}{\pgfqpoint{3.345965in}{1.040354in}}%
\pgfpathcurveto{\pgfqpoint{3.338151in}{1.048167in}}{\pgfqpoint{3.327552in}{1.052558in}}{\pgfqpoint{3.316502in}{1.052558in}}%
\pgfpathcurveto{\pgfqpoint{3.305452in}{1.052558in}}{\pgfqpoint{3.294853in}{1.048167in}}{\pgfqpoint{3.287039in}{1.040354in}}%
\pgfpathcurveto{\pgfqpoint{3.279226in}{1.032540in}}{\pgfqpoint{3.274835in}{1.021941in}}{\pgfqpoint{3.274835in}{1.010891in}}%
\pgfpathcurveto{\pgfqpoint{3.274835in}{0.999841in}}{\pgfqpoint{3.279226in}{0.989242in}}{\pgfqpoint{3.287039in}{0.981428in}}%
\pgfpathcurveto{\pgfqpoint{3.294853in}{0.973615in}}{\pgfqpoint{3.305452in}{0.969224in}}{\pgfqpoint{3.316502in}{0.969224in}}%
\pgfpathlineto{\pgfqpoint{3.316502in}{0.969224in}}%
\pgfpathclose%
\pgfusepath{stroke}%
\end{pgfscope}%
\begin{pgfscope}%
\pgfpathrectangle{\pgfqpoint{0.847223in}{0.554012in}}{\pgfqpoint{6.200000in}{4.620000in}}%
\pgfusepath{clip}%
\pgfsetbuttcap%
\pgfsetroundjoin%
\pgfsetlinewidth{1.003750pt}%
\definecolor{currentstroke}{rgb}{1.000000,0.000000,0.000000}%
\pgfsetstrokecolor{currentstroke}%
\pgfsetdash{}{0pt}%
\pgfpathmoveto{\pgfqpoint{3.321835in}{0.967655in}}%
\pgfpathcurveto{\pgfqpoint{3.332885in}{0.967655in}}{\pgfqpoint{3.343484in}{0.972045in}}{\pgfqpoint{3.351298in}{0.979859in}}%
\pgfpathcurveto{\pgfqpoint{3.359112in}{0.987672in}}{\pgfqpoint{3.363502in}{0.998271in}}{\pgfqpoint{3.363502in}{1.009321in}}%
\pgfpathcurveto{\pgfqpoint{3.363502in}{1.020371in}}{\pgfqpoint{3.359112in}{1.030970in}}{\pgfqpoint{3.351298in}{1.038784in}}%
\pgfpathcurveto{\pgfqpoint{3.343484in}{1.046598in}}{\pgfqpoint{3.332885in}{1.050988in}}{\pgfqpoint{3.321835in}{1.050988in}}%
\pgfpathcurveto{\pgfqpoint{3.310785in}{1.050988in}}{\pgfqpoint{3.300186in}{1.046598in}}{\pgfqpoint{3.292372in}{1.038784in}}%
\pgfpathcurveto{\pgfqpoint{3.284559in}{1.030970in}}{\pgfqpoint{3.280168in}{1.020371in}}{\pgfqpoint{3.280168in}{1.009321in}}%
\pgfpathcurveto{\pgfqpoint{3.280168in}{0.998271in}}{\pgfqpoint{3.284559in}{0.987672in}}{\pgfqpoint{3.292372in}{0.979859in}}%
\pgfpathcurveto{\pgfqpoint{3.300186in}{0.972045in}}{\pgfqpoint{3.310785in}{0.967655in}}{\pgfqpoint{3.321835in}{0.967655in}}%
\pgfpathlineto{\pgfqpoint{3.321835in}{0.967655in}}%
\pgfpathclose%
\pgfusepath{stroke}%
\end{pgfscope}%
\begin{pgfscope}%
\pgfpathrectangle{\pgfqpoint{0.847223in}{0.554012in}}{\pgfqpoint{6.200000in}{4.620000in}}%
\pgfusepath{clip}%
\pgfsetbuttcap%
\pgfsetroundjoin%
\pgfsetlinewidth{1.003750pt}%
\definecolor{currentstroke}{rgb}{1.000000,0.000000,0.000000}%
\pgfsetstrokecolor{currentstroke}%
\pgfsetdash{}{0pt}%
\pgfpathmoveto{\pgfqpoint{3.327168in}{0.966091in}}%
\pgfpathcurveto{\pgfqpoint{3.338218in}{0.966091in}}{\pgfqpoint{3.348817in}{0.970481in}}{\pgfqpoint{3.356631in}{0.978294in}}%
\pgfpathcurveto{\pgfqpoint{3.364445in}{0.986108in}}{\pgfqpoint{3.368835in}{0.996707in}}{\pgfqpoint{3.368835in}{1.007757in}}%
\pgfpathcurveto{\pgfqpoint{3.368835in}{1.018807in}}{\pgfqpoint{3.364445in}{1.029406in}}{\pgfqpoint{3.356631in}{1.037220in}}%
\pgfpathcurveto{\pgfqpoint{3.348817in}{1.045034in}}{\pgfqpoint{3.338218in}{1.049424in}}{\pgfqpoint{3.327168in}{1.049424in}}%
\pgfpathcurveto{\pgfqpoint{3.316118in}{1.049424in}}{\pgfqpoint{3.305519in}{1.045034in}}{\pgfqpoint{3.297706in}{1.037220in}}%
\pgfpathcurveto{\pgfqpoint{3.289892in}{1.029406in}}{\pgfqpoint{3.285502in}{1.018807in}}{\pgfqpoint{3.285502in}{1.007757in}}%
\pgfpathcurveto{\pgfqpoint{3.285502in}{0.996707in}}{\pgfqpoint{3.289892in}{0.986108in}}{\pgfqpoint{3.297706in}{0.978294in}}%
\pgfpathcurveto{\pgfqpoint{3.305519in}{0.970481in}}{\pgfqpoint{3.316118in}{0.966091in}}{\pgfqpoint{3.327168in}{0.966091in}}%
\pgfpathlineto{\pgfqpoint{3.327168in}{0.966091in}}%
\pgfpathclose%
\pgfusepath{stroke}%
\end{pgfscope}%
\begin{pgfscope}%
\pgfpathrectangle{\pgfqpoint{0.847223in}{0.554012in}}{\pgfqpoint{6.200000in}{4.620000in}}%
\pgfusepath{clip}%
\pgfsetbuttcap%
\pgfsetroundjoin%
\pgfsetlinewidth{1.003750pt}%
\definecolor{currentstroke}{rgb}{1.000000,0.000000,0.000000}%
\pgfsetstrokecolor{currentstroke}%
\pgfsetdash{}{0pt}%
\pgfpathmoveto{\pgfqpoint{3.332502in}{0.964532in}}%
\pgfpathcurveto{\pgfqpoint{3.343552in}{0.964532in}}{\pgfqpoint{3.354151in}{0.968922in}}{\pgfqpoint{3.361964in}{0.976736in}}%
\pgfpathcurveto{\pgfqpoint{3.369778in}{0.984550in}}{\pgfqpoint{3.374168in}{0.995149in}}{\pgfqpoint{3.374168in}{1.006199in}}%
\pgfpathcurveto{\pgfqpoint{3.374168in}{1.017249in}}{\pgfqpoint{3.369778in}{1.027848in}}{\pgfqpoint{3.361964in}{1.035662in}}%
\pgfpathcurveto{\pgfqpoint{3.354151in}{1.043475in}}{\pgfqpoint{3.343552in}{1.047865in}}{\pgfqpoint{3.332502in}{1.047865in}}%
\pgfpathcurveto{\pgfqpoint{3.321451in}{1.047865in}}{\pgfqpoint{3.310852in}{1.043475in}}{\pgfqpoint{3.303039in}{1.035662in}}%
\pgfpathcurveto{\pgfqpoint{3.295225in}{1.027848in}}{\pgfqpoint{3.290835in}{1.017249in}}{\pgfqpoint{3.290835in}{1.006199in}}%
\pgfpathcurveto{\pgfqpoint{3.290835in}{0.995149in}}{\pgfqpoint{3.295225in}{0.984550in}}{\pgfqpoint{3.303039in}{0.976736in}}%
\pgfpathcurveto{\pgfqpoint{3.310852in}{0.968922in}}{\pgfqpoint{3.321451in}{0.964532in}}{\pgfqpoint{3.332502in}{0.964532in}}%
\pgfpathlineto{\pgfqpoint{3.332502in}{0.964532in}}%
\pgfpathclose%
\pgfusepath{stroke}%
\end{pgfscope}%
\begin{pgfscope}%
\pgfpathrectangle{\pgfqpoint{0.847223in}{0.554012in}}{\pgfqpoint{6.200000in}{4.620000in}}%
\pgfusepath{clip}%
\pgfsetbuttcap%
\pgfsetroundjoin%
\pgfsetlinewidth{1.003750pt}%
\definecolor{currentstroke}{rgb}{1.000000,0.000000,0.000000}%
\pgfsetstrokecolor{currentstroke}%
\pgfsetdash{}{0pt}%
\pgfpathmoveto{\pgfqpoint{3.337835in}{0.962979in}}%
\pgfpathcurveto{\pgfqpoint{3.348885in}{0.962979in}}{\pgfqpoint{3.359484in}{0.967369in}}{\pgfqpoint{3.367298in}{0.975183in}}%
\pgfpathcurveto{\pgfqpoint{3.375111in}{0.982997in}}{\pgfqpoint{3.379501in}{0.993596in}}{\pgfqpoint{3.379501in}{1.004646in}}%
\pgfpathcurveto{\pgfqpoint{3.379501in}{1.015696in}}{\pgfqpoint{3.375111in}{1.026295in}}{\pgfqpoint{3.367298in}{1.034108in}}%
\pgfpathcurveto{\pgfqpoint{3.359484in}{1.041922in}}{\pgfqpoint{3.348885in}{1.046312in}}{\pgfqpoint{3.337835in}{1.046312in}}%
\pgfpathcurveto{\pgfqpoint{3.326785in}{1.046312in}}{\pgfqpoint{3.316186in}{1.041922in}}{\pgfqpoint{3.308372in}{1.034108in}}%
\pgfpathcurveto{\pgfqpoint{3.300558in}{1.026295in}}{\pgfqpoint{3.296168in}{1.015696in}}{\pgfqpoint{3.296168in}{1.004646in}}%
\pgfpathcurveto{\pgfqpoint{3.296168in}{0.993596in}}{\pgfqpoint{3.300558in}{0.982997in}}{\pgfqpoint{3.308372in}{0.975183in}}%
\pgfpathcurveto{\pgfqpoint{3.316186in}{0.967369in}}{\pgfqpoint{3.326785in}{0.962979in}}{\pgfqpoint{3.337835in}{0.962979in}}%
\pgfpathlineto{\pgfqpoint{3.337835in}{0.962979in}}%
\pgfpathclose%
\pgfusepath{stroke}%
\end{pgfscope}%
\begin{pgfscope}%
\pgfpathrectangle{\pgfqpoint{0.847223in}{0.554012in}}{\pgfqpoint{6.200000in}{4.620000in}}%
\pgfusepath{clip}%
\pgfsetbuttcap%
\pgfsetroundjoin%
\pgfsetlinewidth{1.003750pt}%
\definecolor{currentstroke}{rgb}{1.000000,0.000000,0.000000}%
\pgfsetstrokecolor{currentstroke}%
\pgfsetdash{}{0pt}%
\pgfpathmoveto{\pgfqpoint{3.343168in}{0.961431in}}%
\pgfpathcurveto{\pgfqpoint{3.354218in}{0.961431in}}{\pgfqpoint{3.364817in}{0.965822in}}{\pgfqpoint{3.372631in}{0.973635in}}%
\pgfpathcurveto{\pgfqpoint{3.380444in}{0.981449in}}{\pgfqpoint{3.384835in}{0.992048in}}{\pgfqpoint{3.384835in}{1.003098in}}%
\pgfpathcurveto{\pgfqpoint{3.384835in}{1.014148in}}{\pgfqpoint{3.380444in}{1.024747in}}{\pgfqpoint{3.372631in}{1.032561in}}%
\pgfpathcurveto{\pgfqpoint{3.364817in}{1.040375in}}{\pgfqpoint{3.354218in}{1.044765in}}{\pgfqpoint{3.343168in}{1.044765in}}%
\pgfpathcurveto{\pgfqpoint{3.332118in}{1.044765in}}{\pgfqpoint{3.321519in}{1.040375in}}{\pgfqpoint{3.313705in}{1.032561in}}%
\pgfpathcurveto{\pgfqpoint{3.305892in}{1.024747in}}{\pgfqpoint{3.301501in}{1.014148in}}{\pgfqpoint{3.301501in}{1.003098in}}%
\pgfpathcurveto{\pgfqpoint{3.301501in}{0.992048in}}{\pgfqpoint{3.305892in}{0.981449in}}{\pgfqpoint{3.313705in}{0.973635in}}%
\pgfpathcurveto{\pgfqpoint{3.321519in}{0.965822in}}{\pgfqpoint{3.332118in}{0.961431in}}{\pgfqpoint{3.343168in}{0.961431in}}%
\pgfpathlineto{\pgfqpoint{3.343168in}{0.961431in}}%
\pgfpathclose%
\pgfusepath{stroke}%
\end{pgfscope}%
\begin{pgfscope}%
\pgfpathrectangle{\pgfqpoint{0.847223in}{0.554012in}}{\pgfqpoint{6.200000in}{4.620000in}}%
\pgfusepath{clip}%
\pgfsetbuttcap%
\pgfsetroundjoin%
\pgfsetlinewidth{1.003750pt}%
\definecolor{currentstroke}{rgb}{1.000000,0.000000,0.000000}%
\pgfsetstrokecolor{currentstroke}%
\pgfsetdash{}{0pt}%
\pgfpathmoveto{\pgfqpoint{3.348501in}{0.959889in}}%
\pgfpathcurveto{\pgfqpoint{3.359551in}{0.959889in}}{\pgfqpoint{3.370150in}{0.964280in}}{\pgfqpoint{3.377964in}{0.972093in}}%
\pgfpathcurveto{\pgfqpoint{3.385778in}{0.979907in}}{\pgfqpoint{3.390168in}{0.990506in}}{\pgfqpoint{3.390168in}{1.001556in}}%
\pgfpathcurveto{\pgfqpoint{3.390168in}{1.012606in}}{\pgfqpoint{3.385778in}{1.023205in}}{\pgfqpoint{3.377964in}{1.031019in}}%
\pgfpathcurveto{\pgfqpoint{3.370150in}{1.038832in}}{\pgfqpoint{3.359551in}{1.043223in}}{\pgfqpoint{3.348501in}{1.043223in}}%
\pgfpathcurveto{\pgfqpoint{3.337451in}{1.043223in}}{\pgfqpoint{3.326852in}{1.038832in}}{\pgfqpoint{3.319038in}{1.031019in}}%
\pgfpathcurveto{\pgfqpoint{3.311225in}{1.023205in}}{\pgfqpoint{3.306835in}{1.012606in}}{\pgfqpoint{3.306835in}{1.001556in}}%
\pgfpathcurveto{\pgfqpoint{3.306835in}{0.990506in}}{\pgfqpoint{3.311225in}{0.979907in}}{\pgfqpoint{3.319038in}{0.972093in}}%
\pgfpathcurveto{\pgfqpoint{3.326852in}{0.964280in}}{\pgfqpoint{3.337451in}{0.959889in}}{\pgfqpoint{3.348501in}{0.959889in}}%
\pgfpathlineto{\pgfqpoint{3.348501in}{0.959889in}}%
\pgfpathclose%
\pgfusepath{stroke}%
\end{pgfscope}%
\begin{pgfscope}%
\pgfpathrectangle{\pgfqpoint{0.847223in}{0.554012in}}{\pgfqpoint{6.200000in}{4.620000in}}%
\pgfusepath{clip}%
\pgfsetbuttcap%
\pgfsetroundjoin%
\pgfsetlinewidth{1.003750pt}%
\definecolor{currentstroke}{rgb}{1.000000,0.000000,0.000000}%
\pgfsetstrokecolor{currentstroke}%
\pgfsetdash{}{0pt}%
\pgfpathmoveto{\pgfqpoint{3.353834in}{0.958353in}}%
\pgfpathcurveto{\pgfqpoint{3.364885in}{0.958353in}}{\pgfqpoint{3.375484in}{0.962743in}}{\pgfqpoint{3.383297in}{0.970557in}}%
\pgfpathcurveto{\pgfqpoint{3.391111in}{0.978370in}}{\pgfqpoint{3.395501in}{0.988969in}}{\pgfqpoint{3.395501in}{1.000019in}}%
\pgfpathcurveto{\pgfqpoint{3.395501in}{1.011070in}}{\pgfqpoint{3.391111in}{1.021669in}}{\pgfqpoint{3.383297in}{1.029482in}}%
\pgfpathcurveto{\pgfqpoint{3.375484in}{1.037296in}}{\pgfqpoint{3.364885in}{1.041686in}}{\pgfqpoint{3.353834in}{1.041686in}}%
\pgfpathcurveto{\pgfqpoint{3.342784in}{1.041686in}}{\pgfqpoint{3.332185in}{1.037296in}}{\pgfqpoint{3.324372in}{1.029482in}}%
\pgfpathcurveto{\pgfqpoint{3.316558in}{1.021669in}}{\pgfqpoint{3.312168in}{1.011070in}}{\pgfqpoint{3.312168in}{1.000019in}}%
\pgfpathcurveto{\pgfqpoint{3.312168in}{0.988969in}}{\pgfqpoint{3.316558in}{0.978370in}}{\pgfqpoint{3.324372in}{0.970557in}}%
\pgfpathcurveto{\pgfqpoint{3.332185in}{0.962743in}}{\pgfqpoint{3.342784in}{0.958353in}}{\pgfqpoint{3.353834in}{0.958353in}}%
\pgfpathlineto{\pgfqpoint{3.353834in}{0.958353in}}%
\pgfpathclose%
\pgfusepath{stroke}%
\end{pgfscope}%
\begin{pgfscope}%
\pgfpathrectangle{\pgfqpoint{0.847223in}{0.554012in}}{\pgfqpoint{6.200000in}{4.620000in}}%
\pgfusepath{clip}%
\pgfsetbuttcap%
\pgfsetroundjoin%
\pgfsetlinewidth{1.003750pt}%
\definecolor{currentstroke}{rgb}{1.000000,0.000000,0.000000}%
\pgfsetstrokecolor{currentstroke}%
\pgfsetdash{}{0pt}%
\pgfpathmoveto{\pgfqpoint{3.359168in}{0.956822in}}%
\pgfpathcurveto{\pgfqpoint{3.370218in}{0.956822in}}{\pgfqpoint{3.380817in}{0.961212in}}{\pgfqpoint{3.388630in}{0.969025in}}%
\pgfpathcurveto{\pgfqpoint{3.396444in}{0.976839in}}{\pgfqpoint{3.400834in}{0.987438in}}{\pgfqpoint{3.400834in}{0.998488in}}%
\pgfpathcurveto{\pgfqpoint{3.400834in}{1.009538in}}{\pgfqpoint{3.396444in}{1.020137in}}{\pgfqpoint{3.388630in}{1.027951in}}%
\pgfpathcurveto{\pgfqpoint{3.380817in}{1.035765in}}{\pgfqpoint{3.370218in}{1.040155in}}{\pgfqpoint{3.359168in}{1.040155in}}%
\pgfpathcurveto{\pgfqpoint{3.348118in}{1.040155in}}{\pgfqpoint{3.337518in}{1.035765in}}{\pgfqpoint{3.329705in}{1.027951in}}%
\pgfpathcurveto{\pgfqpoint{3.321891in}{1.020137in}}{\pgfqpoint{3.317501in}{1.009538in}}{\pgfqpoint{3.317501in}{0.998488in}}%
\pgfpathcurveto{\pgfqpoint{3.317501in}{0.987438in}}{\pgfqpoint{3.321891in}{0.976839in}}{\pgfqpoint{3.329705in}{0.969025in}}%
\pgfpathcurveto{\pgfqpoint{3.337518in}{0.961212in}}{\pgfqpoint{3.348118in}{0.956822in}}{\pgfqpoint{3.359168in}{0.956822in}}%
\pgfpathlineto{\pgfqpoint{3.359168in}{0.956822in}}%
\pgfpathclose%
\pgfusepath{stroke}%
\end{pgfscope}%
\begin{pgfscope}%
\pgfpathrectangle{\pgfqpoint{0.847223in}{0.554012in}}{\pgfqpoint{6.200000in}{4.620000in}}%
\pgfusepath{clip}%
\pgfsetbuttcap%
\pgfsetroundjoin%
\pgfsetlinewidth{1.003750pt}%
\definecolor{currentstroke}{rgb}{1.000000,0.000000,0.000000}%
\pgfsetstrokecolor{currentstroke}%
\pgfsetdash{}{0pt}%
\pgfpathmoveto{\pgfqpoint{3.364501in}{0.955296in}}%
\pgfpathcurveto{\pgfqpoint{3.375551in}{0.955296in}}{\pgfqpoint{3.386150in}{0.959686in}}{\pgfqpoint{3.393964in}{0.967500in}}%
\pgfpathcurveto{\pgfqpoint{3.401777in}{0.975313in}}{\pgfqpoint{3.406168in}{0.985912in}}{\pgfqpoint{3.406168in}{0.996962in}}%
\pgfpathcurveto{\pgfqpoint{3.406168in}{1.008012in}}{\pgfqpoint{3.401777in}{1.018611in}}{\pgfqpoint{3.393964in}{1.026425in}}%
\pgfpathcurveto{\pgfqpoint{3.386150in}{1.034239in}}{\pgfqpoint{3.375551in}{1.038629in}}{\pgfqpoint{3.364501in}{1.038629in}}%
\pgfpathcurveto{\pgfqpoint{3.353451in}{1.038629in}}{\pgfqpoint{3.342852in}{1.034239in}}{\pgfqpoint{3.335038in}{1.026425in}}%
\pgfpathcurveto{\pgfqpoint{3.327224in}{1.018611in}}{\pgfqpoint{3.322834in}{1.008012in}}{\pgfqpoint{3.322834in}{0.996962in}}%
\pgfpathcurveto{\pgfqpoint{3.322834in}{0.985912in}}{\pgfqpoint{3.327224in}{0.975313in}}{\pgfqpoint{3.335038in}{0.967500in}}%
\pgfpathcurveto{\pgfqpoint{3.342852in}{0.959686in}}{\pgfqpoint{3.353451in}{0.955296in}}{\pgfqpoint{3.364501in}{0.955296in}}%
\pgfpathlineto{\pgfqpoint{3.364501in}{0.955296in}}%
\pgfpathclose%
\pgfusepath{stroke}%
\end{pgfscope}%
\begin{pgfscope}%
\pgfpathrectangle{\pgfqpoint{0.847223in}{0.554012in}}{\pgfqpoint{6.200000in}{4.620000in}}%
\pgfusepath{clip}%
\pgfsetbuttcap%
\pgfsetroundjoin%
\pgfsetlinewidth{1.003750pt}%
\definecolor{currentstroke}{rgb}{1.000000,0.000000,0.000000}%
\pgfsetstrokecolor{currentstroke}%
\pgfsetdash{}{0pt}%
\pgfpathmoveto{\pgfqpoint{3.369834in}{0.953775in}}%
\pgfpathcurveto{\pgfqpoint{3.380884in}{0.953775in}}{\pgfqpoint{3.391483in}{0.958165in}}{\pgfqpoint{3.399297in}{0.965979in}}%
\pgfpathcurveto{\pgfqpoint{3.407110in}{0.973793in}}{\pgfqpoint{3.411501in}{0.984392in}}{\pgfqpoint{3.411501in}{0.995442in}}%
\pgfpathcurveto{\pgfqpoint{3.411501in}{1.006492in}}{\pgfqpoint{3.407110in}{1.017091in}}{\pgfqpoint{3.399297in}{1.024905in}}%
\pgfpathcurveto{\pgfqpoint{3.391483in}{1.032718in}}{\pgfqpoint{3.380884in}{1.037108in}}{\pgfqpoint{3.369834in}{1.037108in}}%
\pgfpathcurveto{\pgfqpoint{3.358784in}{1.037108in}}{\pgfqpoint{3.348185in}{1.032718in}}{\pgfqpoint{3.340371in}{1.024905in}}%
\pgfpathcurveto{\pgfqpoint{3.332558in}{1.017091in}}{\pgfqpoint{3.328167in}{1.006492in}}{\pgfqpoint{3.328167in}{0.995442in}}%
\pgfpathcurveto{\pgfqpoint{3.328167in}{0.984392in}}{\pgfqpoint{3.332558in}{0.973793in}}{\pgfqpoint{3.340371in}{0.965979in}}%
\pgfpathcurveto{\pgfqpoint{3.348185in}{0.958165in}}{\pgfqpoint{3.358784in}{0.953775in}}{\pgfqpoint{3.369834in}{0.953775in}}%
\pgfpathlineto{\pgfqpoint{3.369834in}{0.953775in}}%
\pgfpathclose%
\pgfusepath{stroke}%
\end{pgfscope}%
\begin{pgfscope}%
\pgfpathrectangle{\pgfqpoint{0.847223in}{0.554012in}}{\pgfqpoint{6.200000in}{4.620000in}}%
\pgfusepath{clip}%
\pgfsetbuttcap%
\pgfsetroundjoin%
\pgfsetlinewidth{1.003750pt}%
\definecolor{currentstroke}{rgb}{1.000000,0.000000,0.000000}%
\pgfsetstrokecolor{currentstroke}%
\pgfsetdash{}{0pt}%
\pgfpathmoveto{\pgfqpoint{3.375167in}{0.952260in}}%
\pgfpathcurveto{\pgfqpoint{3.386217in}{0.952260in}}{\pgfqpoint{3.396816in}{0.956650in}}{\pgfqpoint{3.404630in}{0.964464in}}%
\pgfpathcurveto{\pgfqpoint{3.412444in}{0.972277in}}{\pgfqpoint{3.416834in}{0.982876in}}{\pgfqpoint{3.416834in}{0.993927in}}%
\pgfpathcurveto{\pgfqpoint{3.416834in}{1.004977in}}{\pgfqpoint{3.412444in}{1.015576in}}{\pgfqpoint{3.404630in}{1.023389in}}%
\pgfpathcurveto{\pgfqpoint{3.396816in}{1.031203in}}{\pgfqpoint{3.386217in}{1.035593in}}{\pgfqpoint{3.375167in}{1.035593in}}%
\pgfpathcurveto{\pgfqpoint{3.364117in}{1.035593in}}{\pgfqpoint{3.353518in}{1.031203in}}{\pgfqpoint{3.345704in}{1.023389in}}%
\pgfpathcurveto{\pgfqpoint{3.337891in}{1.015576in}}{\pgfqpoint{3.333501in}{1.004977in}}{\pgfqpoint{3.333501in}{0.993927in}}%
\pgfpathcurveto{\pgfqpoint{3.333501in}{0.982876in}}{\pgfqpoint{3.337891in}{0.972277in}}{\pgfqpoint{3.345704in}{0.964464in}}%
\pgfpathcurveto{\pgfqpoint{3.353518in}{0.956650in}}{\pgfqpoint{3.364117in}{0.952260in}}{\pgfqpoint{3.375167in}{0.952260in}}%
\pgfpathlineto{\pgfqpoint{3.375167in}{0.952260in}}%
\pgfpathclose%
\pgfusepath{stroke}%
\end{pgfscope}%
\begin{pgfscope}%
\pgfpathrectangle{\pgfqpoint{0.847223in}{0.554012in}}{\pgfqpoint{6.200000in}{4.620000in}}%
\pgfusepath{clip}%
\pgfsetbuttcap%
\pgfsetroundjoin%
\pgfsetlinewidth{1.003750pt}%
\definecolor{currentstroke}{rgb}{1.000000,0.000000,0.000000}%
\pgfsetstrokecolor{currentstroke}%
\pgfsetdash{}{0pt}%
\pgfpathmoveto{\pgfqpoint{3.380500in}{0.950750in}}%
\pgfpathcurveto{\pgfqpoint{3.391551in}{0.950750in}}{\pgfqpoint{3.402150in}{0.955140in}}{\pgfqpoint{3.409963in}{0.962954in}}%
\pgfpathcurveto{\pgfqpoint{3.417777in}{0.970767in}}{\pgfqpoint{3.422167in}{0.981367in}}{\pgfqpoint{3.422167in}{0.992417in}}%
\pgfpathcurveto{\pgfqpoint{3.422167in}{1.003467in}}{\pgfqpoint{3.417777in}{1.014066in}}{\pgfqpoint{3.409963in}{1.021879in}}%
\pgfpathcurveto{\pgfqpoint{3.402150in}{1.029693in}}{\pgfqpoint{3.391551in}{1.034083in}}{\pgfqpoint{3.380500in}{1.034083in}}%
\pgfpathcurveto{\pgfqpoint{3.369450in}{1.034083in}}{\pgfqpoint{3.358851in}{1.029693in}}{\pgfqpoint{3.351038in}{1.021879in}}%
\pgfpathcurveto{\pgfqpoint{3.343224in}{1.014066in}}{\pgfqpoint{3.338834in}{1.003467in}}{\pgfqpoint{3.338834in}{0.992417in}}%
\pgfpathcurveto{\pgfqpoint{3.338834in}{0.981367in}}{\pgfqpoint{3.343224in}{0.970767in}}{\pgfqpoint{3.351038in}{0.962954in}}%
\pgfpathcurveto{\pgfqpoint{3.358851in}{0.955140in}}{\pgfqpoint{3.369450in}{0.950750in}}{\pgfqpoint{3.380500in}{0.950750in}}%
\pgfpathlineto{\pgfqpoint{3.380500in}{0.950750in}}%
\pgfpathclose%
\pgfusepath{stroke}%
\end{pgfscope}%
\begin{pgfscope}%
\pgfpathrectangle{\pgfqpoint{0.847223in}{0.554012in}}{\pgfqpoint{6.200000in}{4.620000in}}%
\pgfusepath{clip}%
\pgfsetbuttcap%
\pgfsetroundjoin%
\pgfsetlinewidth{1.003750pt}%
\definecolor{currentstroke}{rgb}{1.000000,0.000000,0.000000}%
\pgfsetstrokecolor{currentstroke}%
\pgfsetdash{}{0pt}%
\pgfpathmoveto{\pgfqpoint{3.385834in}{0.949245in}}%
\pgfpathcurveto{\pgfqpoint{3.396884in}{0.949245in}}{\pgfqpoint{3.407483in}{0.953636in}}{\pgfqpoint{3.415296in}{0.961449in}}%
\pgfpathcurveto{\pgfqpoint{3.423110in}{0.969263in}}{\pgfqpoint{3.427500in}{0.979862in}}{\pgfqpoint{3.427500in}{0.990912in}}%
\pgfpathcurveto{\pgfqpoint{3.427500in}{1.001962in}}{\pgfqpoint{3.423110in}{1.012561in}}{\pgfqpoint{3.415296in}{1.020375in}}%
\pgfpathcurveto{\pgfqpoint{3.407483in}{1.028188in}}{\pgfqpoint{3.396884in}{1.032579in}}{\pgfqpoint{3.385834in}{1.032579in}}%
\pgfpathcurveto{\pgfqpoint{3.374784in}{1.032579in}}{\pgfqpoint{3.364185in}{1.028188in}}{\pgfqpoint{3.356371in}{1.020375in}}%
\pgfpathcurveto{\pgfqpoint{3.348557in}{1.012561in}}{\pgfqpoint{3.344167in}{1.001962in}}{\pgfqpoint{3.344167in}{0.990912in}}%
\pgfpathcurveto{\pgfqpoint{3.344167in}{0.979862in}}{\pgfqpoint{3.348557in}{0.969263in}}{\pgfqpoint{3.356371in}{0.961449in}}%
\pgfpathcurveto{\pgfqpoint{3.364185in}{0.953636in}}{\pgfqpoint{3.374784in}{0.949245in}}{\pgfqpoint{3.385834in}{0.949245in}}%
\pgfpathlineto{\pgfqpoint{3.385834in}{0.949245in}}%
\pgfpathclose%
\pgfusepath{stroke}%
\end{pgfscope}%
\begin{pgfscope}%
\pgfpathrectangle{\pgfqpoint{0.847223in}{0.554012in}}{\pgfqpoint{6.200000in}{4.620000in}}%
\pgfusepath{clip}%
\pgfsetbuttcap%
\pgfsetroundjoin%
\pgfsetlinewidth{1.003750pt}%
\definecolor{currentstroke}{rgb}{1.000000,0.000000,0.000000}%
\pgfsetstrokecolor{currentstroke}%
\pgfsetdash{}{0pt}%
\pgfpathmoveto{\pgfqpoint{3.391167in}{0.947746in}}%
\pgfpathcurveto{\pgfqpoint{3.402217in}{0.947746in}}{\pgfqpoint{3.412816in}{0.952136in}}{\pgfqpoint{3.420630in}{0.959950in}}%
\pgfpathcurveto{\pgfqpoint{3.428443in}{0.967763in}}{\pgfqpoint{3.432834in}{0.978362in}}{\pgfqpoint{3.432834in}{0.989413in}}%
\pgfpathcurveto{\pgfqpoint{3.432834in}{1.000463in}}{\pgfqpoint{3.428443in}{1.011062in}}{\pgfqpoint{3.420630in}{1.018875in}}%
\pgfpathcurveto{\pgfqpoint{3.412816in}{1.026689in}}{\pgfqpoint{3.402217in}{1.031079in}}{\pgfqpoint{3.391167in}{1.031079in}}%
\pgfpathcurveto{\pgfqpoint{3.380117in}{1.031079in}}{\pgfqpoint{3.369518in}{1.026689in}}{\pgfqpoint{3.361704in}{1.018875in}}%
\pgfpathcurveto{\pgfqpoint{3.353891in}{1.011062in}}{\pgfqpoint{3.349500in}{1.000463in}}{\pgfqpoint{3.349500in}{0.989413in}}%
\pgfpathcurveto{\pgfqpoint{3.349500in}{0.978362in}}{\pgfqpoint{3.353891in}{0.967763in}}{\pgfqpoint{3.361704in}{0.959950in}}%
\pgfpathcurveto{\pgfqpoint{3.369518in}{0.952136in}}{\pgfqpoint{3.380117in}{0.947746in}}{\pgfqpoint{3.391167in}{0.947746in}}%
\pgfpathlineto{\pgfqpoint{3.391167in}{0.947746in}}%
\pgfpathclose%
\pgfusepath{stroke}%
\end{pgfscope}%
\begin{pgfscope}%
\pgfpathrectangle{\pgfqpoint{0.847223in}{0.554012in}}{\pgfqpoint{6.200000in}{4.620000in}}%
\pgfusepath{clip}%
\pgfsetbuttcap%
\pgfsetroundjoin%
\pgfsetlinewidth{1.003750pt}%
\definecolor{currentstroke}{rgb}{1.000000,0.000000,0.000000}%
\pgfsetstrokecolor{currentstroke}%
\pgfsetdash{}{0pt}%
\pgfpathmoveto{\pgfqpoint{3.396500in}{0.946252in}}%
\pgfpathcurveto{\pgfqpoint{3.407550in}{0.946252in}}{\pgfqpoint{3.418149in}{0.950642in}}{\pgfqpoint{3.425963in}{0.958456in}}%
\pgfpathcurveto{\pgfqpoint{3.433777in}{0.966269in}}{\pgfqpoint{3.438167in}{0.976868in}}{\pgfqpoint{3.438167in}{0.987918in}}%
\pgfpathcurveto{\pgfqpoint{3.438167in}{0.998968in}}{\pgfqpoint{3.433777in}{1.009567in}}{\pgfqpoint{3.425963in}{1.017381in}}%
\pgfpathcurveto{\pgfqpoint{3.418149in}{1.025195in}}{\pgfqpoint{3.407550in}{1.029585in}}{\pgfqpoint{3.396500in}{1.029585in}}%
\pgfpathcurveto{\pgfqpoint{3.385450in}{1.029585in}}{\pgfqpoint{3.374851in}{1.025195in}}{\pgfqpoint{3.367037in}{1.017381in}}%
\pgfpathcurveto{\pgfqpoint{3.359224in}{1.009567in}}{\pgfqpoint{3.354833in}{0.998968in}}{\pgfqpoint{3.354833in}{0.987918in}}%
\pgfpathcurveto{\pgfqpoint{3.354833in}{0.976868in}}{\pgfqpoint{3.359224in}{0.966269in}}{\pgfqpoint{3.367037in}{0.958456in}}%
\pgfpathcurveto{\pgfqpoint{3.374851in}{0.950642in}}{\pgfqpoint{3.385450in}{0.946252in}}{\pgfqpoint{3.396500in}{0.946252in}}%
\pgfpathlineto{\pgfqpoint{3.396500in}{0.946252in}}%
\pgfpathclose%
\pgfusepath{stroke}%
\end{pgfscope}%
\begin{pgfscope}%
\pgfpathrectangle{\pgfqpoint{0.847223in}{0.554012in}}{\pgfqpoint{6.200000in}{4.620000in}}%
\pgfusepath{clip}%
\pgfsetbuttcap%
\pgfsetroundjoin%
\pgfsetlinewidth{1.003750pt}%
\definecolor{currentstroke}{rgb}{1.000000,0.000000,0.000000}%
\pgfsetstrokecolor{currentstroke}%
\pgfsetdash{}{0pt}%
\pgfpathmoveto{\pgfqpoint{3.401833in}{0.944763in}}%
\pgfpathcurveto{\pgfqpoint{3.412883in}{0.944763in}}{\pgfqpoint{3.423483in}{0.949153in}}{\pgfqpoint{3.431296in}{0.956966in}}%
\pgfpathcurveto{\pgfqpoint{3.439110in}{0.964780in}}{\pgfqpoint{3.443500in}{0.975379in}}{\pgfqpoint{3.443500in}{0.986429in}}%
\pgfpathcurveto{\pgfqpoint{3.443500in}{0.997479in}}{\pgfqpoint{3.439110in}{1.008078in}}{\pgfqpoint{3.431296in}{1.015892in}}%
\pgfpathcurveto{\pgfqpoint{3.423483in}{1.023706in}}{\pgfqpoint{3.412883in}{1.028096in}}{\pgfqpoint{3.401833in}{1.028096in}}%
\pgfpathcurveto{\pgfqpoint{3.390783in}{1.028096in}}{\pgfqpoint{3.380184in}{1.023706in}}{\pgfqpoint{3.372371in}{1.015892in}}%
\pgfpathcurveto{\pgfqpoint{3.364557in}{1.008078in}}{\pgfqpoint{3.360167in}{0.997479in}}{\pgfqpoint{3.360167in}{0.986429in}}%
\pgfpathcurveto{\pgfqpoint{3.360167in}{0.975379in}}{\pgfqpoint{3.364557in}{0.964780in}}{\pgfqpoint{3.372371in}{0.956966in}}%
\pgfpathcurveto{\pgfqpoint{3.380184in}{0.949153in}}{\pgfqpoint{3.390783in}{0.944763in}}{\pgfqpoint{3.401833in}{0.944763in}}%
\pgfpathlineto{\pgfqpoint{3.401833in}{0.944763in}}%
\pgfpathclose%
\pgfusepath{stroke}%
\end{pgfscope}%
\begin{pgfscope}%
\pgfpathrectangle{\pgfqpoint{0.847223in}{0.554012in}}{\pgfqpoint{6.200000in}{4.620000in}}%
\pgfusepath{clip}%
\pgfsetbuttcap%
\pgfsetroundjoin%
\pgfsetlinewidth{1.003750pt}%
\definecolor{currentstroke}{rgb}{1.000000,0.000000,0.000000}%
\pgfsetstrokecolor{currentstroke}%
\pgfsetdash{}{0pt}%
\pgfpathmoveto{\pgfqpoint{3.407167in}{0.943279in}}%
\pgfpathcurveto{\pgfqpoint{3.418217in}{0.943279in}}{\pgfqpoint{3.428816in}{0.947669in}}{\pgfqpoint{3.436629in}{0.955483in}}%
\pgfpathcurveto{\pgfqpoint{3.444443in}{0.963296in}}{\pgfqpoint{3.448833in}{0.973895in}}{\pgfqpoint{3.448833in}{0.984945in}}%
\pgfpathcurveto{\pgfqpoint{3.448833in}{0.995995in}}{\pgfqpoint{3.444443in}{1.006595in}}{\pgfqpoint{3.436629in}{1.014408in}}%
\pgfpathcurveto{\pgfqpoint{3.428816in}{1.022222in}}{\pgfqpoint{3.418217in}{1.026612in}}{\pgfqpoint{3.407167in}{1.026612in}}%
\pgfpathcurveto{\pgfqpoint{3.396116in}{1.026612in}}{\pgfqpoint{3.385517in}{1.022222in}}{\pgfqpoint{3.377704in}{1.014408in}}%
\pgfpathcurveto{\pgfqpoint{3.369890in}{1.006595in}}{\pgfqpoint{3.365500in}{0.995995in}}{\pgfqpoint{3.365500in}{0.984945in}}%
\pgfpathcurveto{\pgfqpoint{3.365500in}{0.973895in}}{\pgfqpoint{3.369890in}{0.963296in}}{\pgfqpoint{3.377704in}{0.955483in}}%
\pgfpathcurveto{\pgfqpoint{3.385517in}{0.947669in}}{\pgfqpoint{3.396116in}{0.943279in}}{\pgfqpoint{3.407167in}{0.943279in}}%
\pgfpathlineto{\pgfqpoint{3.407167in}{0.943279in}}%
\pgfpathclose%
\pgfusepath{stroke}%
\end{pgfscope}%
\begin{pgfscope}%
\pgfpathrectangle{\pgfqpoint{0.847223in}{0.554012in}}{\pgfqpoint{6.200000in}{4.620000in}}%
\pgfusepath{clip}%
\pgfsetbuttcap%
\pgfsetroundjoin%
\pgfsetlinewidth{1.003750pt}%
\definecolor{currentstroke}{rgb}{1.000000,0.000000,0.000000}%
\pgfsetstrokecolor{currentstroke}%
\pgfsetdash{}{0pt}%
\pgfpathmoveto{\pgfqpoint{3.412500in}{0.941800in}}%
\pgfpathcurveto{\pgfqpoint{3.423550in}{0.941800in}}{\pgfqpoint{3.434149in}{0.946190in}}{\pgfqpoint{3.441963in}{0.954004in}}%
\pgfpathcurveto{\pgfqpoint{3.449776in}{0.961817in}}{\pgfqpoint{3.454166in}{0.972416in}}{\pgfqpoint{3.454166in}{0.983467in}}%
\pgfpathcurveto{\pgfqpoint{3.454166in}{0.994517in}}{\pgfqpoint{3.449776in}{1.005116in}}{\pgfqpoint{3.441963in}{1.012929in}}%
\pgfpathcurveto{\pgfqpoint{3.434149in}{1.020743in}}{\pgfqpoint{3.423550in}{1.025133in}}{\pgfqpoint{3.412500in}{1.025133in}}%
\pgfpathcurveto{\pgfqpoint{3.401450in}{1.025133in}}{\pgfqpoint{3.390851in}{1.020743in}}{\pgfqpoint{3.383037in}{1.012929in}}%
\pgfpathcurveto{\pgfqpoint{3.375223in}{1.005116in}}{\pgfqpoint{3.370833in}{0.994517in}}{\pgfqpoint{3.370833in}{0.983467in}}%
\pgfpathcurveto{\pgfqpoint{3.370833in}{0.972416in}}{\pgfqpoint{3.375223in}{0.961817in}}{\pgfqpoint{3.383037in}{0.954004in}}%
\pgfpathcurveto{\pgfqpoint{3.390851in}{0.946190in}}{\pgfqpoint{3.401450in}{0.941800in}}{\pgfqpoint{3.412500in}{0.941800in}}%
\pgfpathlineto{\pgfqpoint{3.412500in}{0.941800in}}%
\pgfpathclose%
\pgfusepath{stroke}%
\end{pgfscope}%
\begin{pgfscope}%
\pgfpathrectangle{\pgfqpoint{0.847223in}{0.554012in}}{\pgfqpoint{6.200000in}{4.620000in}}%
\pgfusepath{clip}%
\pgfsetbuttcap%
\pgfsetroundjoin%
\pgfsetlinewidth{1.003750pt}%
\definecolor{currentstroke}{rgb}{1.000000,0.000000,0.000000}%
\pgfsetstrokecolor{currentstroke}%
\pgfsetdash{}{0pt}%
\pgfpathmoveto{\pgfqpoint{3.417833in}{0.940326in}}%
\pgfpathcurveto{\pgfqpoint{3.428883in}{0.940326in}}{\pgfqpoint{3.439482in}{0.944716in}}{\pgfqpoint{3.447296in}{0.952530in}}%
\pgfpathcurveto{\pgfqpoint{3.455109in}{0.960344in}}{\pgfqpoint{3.459500in}{0.970943in}}{\pgfqpoint{3.459500in}{0.981993in}}%
\pgfpathcurveto{\pgfqpoint{3.459500in}{0.993043in}}{\pgfqpoint{3.455109in}{1.003642in}}{\pgfqpoint{3.447296in}{1.011456in}}%
\pgfpathcurveto{\pgfqpoint{3.439482in}{1.019269in}}{\pgfqpoint{3.428883in}{1.023660in}}{\pgfqpoint{3.417833in}{1.023660in}}%
\pgfpathcurveto{\pgfqpoint{3.406783in}{1.023660in}}{\pgfqpoint{3.396184in}{1.019269in}}{\pgfqpoint{3.388370in}{1.011456in}}%
\pgfpathcurveto{\pgfqpoint{3.380557in}{1.003642in}}{\pgfqpoint{3.376166in}{0.993043in}}{\pgfqpoint{3.376166in}{0.981993in}}%
\pgfpathcurveto{\pgfqpoint{3.376166in}{0.970943in}}{\pgfqpoint{3.380557in}{0.960344in}}{\pgfqpoint{3.388370in}{0.952530in}}%
\pgfpathcurveto{\pgfqpoint{3.396184in}{0.944716in}}{\pgfqpoint{3.406783in}{0.940326in}}{\pgfqpoint{3.417833in}{0.940326in}}%
\pgfpathlineto{\pgfqpoint{3.417833in}{0.940326in}}%
\pgfpathclose%
\pgfusepath{stroke}%
\end{pgfscope}%
\begin{pgfscope}%
\pgfpathrectangle{\pgfqpoint{0.847223in}{0.554012in}}{\pgfqpoint{6.200000in}{4.620000in}}%
\pgfusepath{clip}%
\pgfsetbuttcap%
\pgfsetroundjoin%
\pgfsetlinewidth{1.003750pt}%
\definecolor{currentstroke}{rgb}{1.000000,0.000000,0.000000}%
\pgfsetstrokecolor{currentstroke}%
\pgfsetdash{}{0pt}%
\pgfpathmoveto{\pgfqpoint{3.423166in}{0.938858in}}%
\pgfpathcurveto{\pgfqpoint{3.434216in}{0.938858in}}{\pgfqpoint{3.444815in}{0.943248in}}{\pgfqpoint{3.452629in}{0.951061in}}%
\pgfpathcurveto{\pgfqpoint{3.460443in}{0.958875in}}{\pgfqpoint{3.464833in}{0.969474in}}{\pgfqpoint{3.464833in}{0.980524in}}%
\pgfpathcurveto{\pgfqpoint{3.464833in}{0.991574in}}{\pgfqpoint{3.460443in}{1.002173in}}{\pgfqpoint{3.452629in}{1.009987in}}%
\pgfpathcurveto{\pgfqpoint{3.444815in}{1.017801in}}{\pgfqpoint{3.434216in}{1.022191in}}{\pgfqpoint{3.423166in}{1.022191in}}%
\pgfpathcurveto{\pgfqpoint{3.412116in}{1.022191in}}{\pgfqpoint{3.401517in}{1.017801in}}{\pgfqpoint{3.393703in}{1.009987in}}%
\pgfpathcurveto{\pgfqpoint{3.385890in}{1.002173in}}{\pgfqpoint{3.381500in}{0.991574in}}{\pgfqpoint{3.381500in}{0.980524in}}%
\pgfpathcurveto{\pgfqpoint{3.381500in}{0.969474in}}{\pgfqpoint{3.385890in}{0.958875in}}{\pgfqpoint{3.393703in}{0.951061in}}%
\pgfpathcurveto{\pgfqpoint{3.401517in}{0.943248in}}{\pgfqpoint{3.412116in}{0.938858in}}{\pgfqpoint{3.423166in}{0.938858in}}%
\pgfpathlineto{\pgfqpoint{3.423166in}{0.938858in}}%
\pgfpathclose%
\pgfusepath{stroke}%
\end{pgfscope}%
\begin{pgfscope}%
\pgfpathrectangle{\pgfqpoint{0.847223in}{0.554012in}}{\pgfqpoint{6.200000in}{4.620000in}}%
\pgfusepath{clip}%
\pgfsetbuttcap%
\pgfsetroundjoin%
\pgfsetlinewidth{1.003750pt}%
\definecolor{currentstroke}{rgb}{1.000000,0.000000,0.000000}%
\pgfsetstrokecolor{currentstroke}%
\pgfsetdash{}{0pt}%
\pgfpathmoveto{\pgfqpoint{3.428499in}{0.937394in}}%
\pgfpathcurveto{\pgfqpoint{3.439550in}{0.937394in}}{\pgfqpoint{3.450149in}{0.941784in}}{\pgfqpoint{3.457962in}{0.949598in}}%
\pgfpathcurveto{\pgfqpoint{3.465776in}{0.957412in}}{\pgfqpoint{3.470166in}{0.968011in}}{\pgfqpoint{3.470166in}{0.979061in}}%
\pgfpathcurveto{\pgfqpoint{3.470166in}{0.990111in}}{\pgfqpoint{3.465776in}{1.000710in}}{\pgfqpoint{3.457962in}{1.008524in}}%
\pgfpathcurveto{\pgfqpoint{3.450149in}{1.016337in}}{\pgfqpoint{3.439550in}{1.020727in}}{\pgfqpoint{3.428499in}{1.020727in}}%
\pgfpathcurveto{\pgfqpoint{3.417449in}{1.020727in}}{\pgfqpoint{3.406850in}{1.016337in}}{\pgfqpoint{3.399037in}{1.008524in}}%
\pgfpathcurveto{\pgfqpoint{3.391223in}{1.000710in}}{\pgfqpoint{3.386833in}{0.990111in}}{\pgfqpoint{3.386833in}{0.979061in}}%
\pgfpathcurveto{\pgfqpoint{3.386833in}{0.968011in}}{\pgfqpoint{3.391223in}{0.957412in}}{\pgfqpoint{3.399037in}{0.949598in}}%
\pgfpathcurveto{\pgfqpoint{3.406850in}{0.941784in}}{\pgfqpoint{3.417449in}{0.937394in}}{\pgfqpoint{3.428499in}{0.937394in}}%
\pgfpathlineto{\pgfqpoint{3.428499in}{0.937394in}}%
\pgfpathclose%
\pgfusepath{stroke}%
\end{pgfscope}%
\begin{pgfscope}%
\pgfpathrectangle{\pgfqpoint{0.847223in}{0.554012in}}{\pgfqpoint{6.200000in}{4.620000in}}%
\pgfusepath{clip}%
\pgfsetbuttcap%
\pgfsetroundjoin%
\pgfsetlinewidth{1.003750pt}%
\definecolor{currentstroke}{rgb}{1.000000,0.000000,0.000000}%
\pgfsetstrokecolor{currentstroke}%
\pgfsetdash{}{0pt}%
\pgfpathmoveto{\pgfqpoint{3.433833in}{0.935936in}}%
\pgfpathcurveto{\pgfqpoint{3.444883in}{0.935936in}}{\pgfqpoint{3.455482in}{0.940326in}}{\pgfqpoint{3.463295in}{0.948139in}}%
\pgfpathcurveto{\pgfqpoint{3.471109in}{0.955953in}}{\pgfqpoint{3.475499in}{0.966552in}}{\pgfqpoint{3.475499in}{0.977602in}}%
\pgfpathcurveto{\pgfqpoint{3.475499in}{0.988652in}}{\pgfqpoint{3.471109in}{0.999251in}}{\pgfqpoint{3.463295in}{1.007065in}}%
\pgfpathcurveto{\pgfqpoint{3.455482in}{1.014879in}}{\pgfqpoint{3.444883in}{1.019269in}}{\pgfqpoint{3.433833in}{1.019269in}}%
\pgfpathcurveto{\pgfqpoint{3.422783in}{1.019269in}}{\pgfqpoint{3.412183in}{1.014879in}}{\pgfqpoint{3.404370in}{1.007065in}}%
\pgfpathcurveto{\pgfqpoint{3.396556in}{0.999251in}}{\pgfqpoint{3.392166in}{0.988652in}}{\pgfqpoint{3.392166in}{0.977602in}}%
\pgfpathcurveto{\pgfqpoint{3.392166in}{0.966552in}}{\pgfqpoint{3.396556in}{0.955953in}}{\pgfqpoint{3.404370in}{0.948139in}}%
\pgfpathcurveto{\pgfqpoint{3.412183in}{0.940326in}}{\pgfqpoint{3.422783in}{0.935936in}}{\pgfqpoint{3.433833in}{0.935936in}}%
\pgfpathlineto{\pgfqpoint{3.433833in}{0.935936in}}%
\pgfpathclose%
\pgfusepath{stroke}%
\end{pgfscope}%
\begin{pgfscope}%
\pgfpathrectangle{\pgfqpoint{0.847223in}{0.554012in}}{\pgfqpoint{6.200000in}{4.620000in}}%
\pgfusepath{clip}%
\pgfsetbuttcap%
\pgfsetroundjoin%
\pgfsetlinewidth{1.003750pt}%
\definecolor{currentstroke}{rgb}{1.000000,0.000000,0.000000}%
\pgfsetstrokecolor{currentstroke}%
\pgfsetdash{}{0pt}%
\pgfpathmoveto{\pgfqpoint{3.439166in}{0.934482in}}%
\pgfpathcurveto{\pgfqpoint{3.450216in}{0.934482in}}{\pgfqpoint{3.460815in}{0.938872in}}{\pgfqpoint{3.468629in}{0.946686in}}%
\pgfpathcurveto{\pgfqpoint{3.476442in}{0.954499in}}{\pgfqpoint{3.480833in}{0.965098in}}{\pgfqpoint{3.480833in}{0.976149in}}%
\pgfpathcurveto{\pgfqpoint{3.480833in}{0.987199in}}{\pgfqpoint{3.476442in}{0.997798in}}{\pgfqpoint{3.468629in}{1.005611in}}%
\pgfpathcurveto{\pgfqpoint{3.460815in}{1.013425in}}{\pgfqpoint{3.450216in}{1.017815in}}{\pgfqpoint{3.439166in}{1.017815in}}%
\pgfpathcurveto{\pgfqpoint{3.428116in}{1.017815in}}{\pgfqpoint{3.417517in}{1.013425in}}{\pgfqpoint{3.409703in}{1.005611in}}%
\pgfpathcurveto{\pgfqpoint{3.401889in}{0.997798in}}{\pgfqpoint{3.397499in}{0.987199in}}{\pgfqpoint{3.397499in}{0.976149in}}%
\pgfpathcurveto{\pgfqpoint{3.397499in}{0.965098in}}{\pgfqpoint{3.401889in}{0.954499in}}{\pgfqpoint{3.409703in}{0.946686in}}%
\pgfpathcurveto{\pgfqpoint{3.417517in}{0.938872in}}{\pgfqpoint{3.428116in}{0.934482in}}{\pgfqpoint{3.439166in}{0.934482in}}%
\pgfpathlineto{\pgfqpoint{3.439166in}{0.934482in}}%
\pgfpathclose%
\pgfusepath{stroke}%
\end{pgfscope}%
\begin{pgfscope}%
\pgfpathrectangle{\pgfqpoint{0.847223in}{0.554012in}}{\pgfqpoint{6.200000in}{4.620000in}}%
\pgfusepath{clip}%
\pgfsetbuttcap%
\pgfsetroundjoin%
\pgfsetlinewidth{1.003750pt}%
\definecolor{currentstroke}{rgb}{1.000000,0.000000,0.000000}%
\pgfsetstrokecolor{currentstroke}%
\pgfsetdash{}{0pt}%
\pgfpathmoveto{\pgfqpoint{3.444499in}{0.933033in}}%
\pgfpathcurveto{\pgfqpoint{3.455549in}{0.933033in}}{\pgfqpoint{3.466148in}{0.937424in}}{\pgfqpoint{3.473962in}{0.945237in}}%
\pgfpathcurveto{\pgfqpoint{3.481775in}{0.953051in}}{\pgfqpoint{3.486166in}{0.963650in}}{\pgfqpoint{3.486166in}{0.974700in}}%
\pgfpathcurveto{\pgfqpoint{3.486166in}{0.985750in}}{\pgfqpoint{3.481775in}{0.996349in}}{\pgfqpoint{3.473962in}{1.004163in}}%
\pgfpathcurveto{\pgfqpoint{3.466148in}{1.011976in}}{\pgfqpoint{3.455549in}{1.016367in}}{\pgfqpoint{3.444499in}{1.016367in}}%
\pgfpathcurveto{\pgfqpoint{3.433449in}{1.016367in}}{\pgfqpoint{3.422850in}{1.011976in}}{\pgfqpoint{3.415036in}{1.004163in}}%
\pgfpathcurveto{\pgfqpoint{3.407223in}{0.996349in}}{\pgfqpoint{3.402832in}{0.985750in}}{\pgfqpoint{3.402832in}{0.974700in}}%
\pgfpathcurveto{\pgfqpoint{3.402832in}{0.963650in}}{\pgfqpoint{3.407223in}{0.953051in}}{\pgfqpoint{3.415036in}{0.945237in}}%
\pgfpathcurveto{\pgfqpoint{3.422850in}{0.937424in}}{\pgfqpoint{3.433449in}{0.933033in}}{\pgfqpoint{3.444499in}{0.933033in}}%
\pgfpathlineto{\pgfqpoint{3.444499in}{0.933033in}}%
\pgfpathclose%
\pgfusepath{stroke}%
\end{pgfscope}%
\begin{pgfscope}%
\pgfpathrectangle{\pgfqpoint{0.847223in}{0.554012in}}{\pgfqpoint{6.200000in}{4.620000in}}%
\pgfusepath{clip}%
\pgfsetbuttcap%
\pgfsetroundjoin%
\pgfsetlinewidth{1.003750pt}%
\definecolor{currentstroke}{rgb}{1.000000,0.000000,0.000000}%
\pgfsetstrokecolor{currentstroke}%
\pgfsetdash{}{0pt}%
\pgfpathmoveto{\pgfqpoint{3.449832in}{0.931590in}}%
\pgfpathcurveto{\pgfqpoint{3.460882in}{0.931590in}}{\pgfqpoint{3.471481in}{0.935980in}}{\pgfqpoint{3.479295in}{0.943794in}}%
\pgfpathcurveto{\pgfqpoint{3.487109in}{0.951607in}}{\pgfqpoint{3.491499in}{0.962206in}}{\pgfqpoint{3.491499in}{0.973256in}}%
\pgfpathcurveto{\pgfqpoint{3.491499in}{0.984307in}}{\pgfqpoint{3.487109in}{0.994906in}}{\pgfqpoint{3.479295in}{1.002719in}}%
\pgfpathcurveto{\pgfqpoint{3.471481in}{1.010533in}}{\pgfqpoint{3.460882in}{1.014923in}}{\pgfqpoint{3.449832in}{1.014923in}}%
\pgfpathcurveto{\pgfqpoint{3.438782in}{1.014923in}}{\pgfqpoint{3.428183in}{1.010533in}}{\pgfqpoint{3.420370in}{1.002719in}}%
\pgfpathcurveto{\pgfqpoint{3.412556in}{0.994906in}}{\pgfqpoint{3.408166in}{0.984307in}}{\pgfqpoint{3.408166in}{0.973256in}}%
\pgfpathcurveto{\pgfqpoint{3.408166in}{0.962206in}}{\pgfqpoint{3.412556in}{0.951607in}}{\pgfqpoint{3.420370in}{0.943794in}}%
\pgfpathcurveto{\pgfqpoint{3.428183in}{0.935980in}}{\pgfqpoint{3.438782in}{0.931590in}}{\pgfqpoint{3.449832in}{0.931590in}}%
\pgfpathlineto{\pgfqpoint{3.449832in}{0.931590in}}%
\pgfpathclose%
\pgfusepath{stroke}%
\end{pgfscope}%
\begin{pgfscope}%
\pgfpathrectangle{\pgfqpoint{0.847223in}{0.554012in}}{\pgfqpoint{6.200000in}{4.620000in}}%
\pgfusepath{clip}%
\pgfsetbuttcap%
\pgfsetroundjoin%
\pgfsetlinewidth{1.003750pt}%
\definecolor{currentstroke}{rgb}{1.000000,0.000000,0.000000}%
\pgfsetstrokecolor{currentstroke}%
\pgfsetdash{}{0pt}%
\pgfpathmoveto{\pgfqpoint{3.455166in}{0.930151in}}%
\pgfpathcurveto{\pgfqpoint{3.466216in}{0.930151in}}{\pgfqpoint{3.476815in}{0.934541in}}{\pgfqpoint{3.484628in}{0.942355in}}%
\pgfpathcurveto{\pgfqpoint{3.492442in}{0.950169in}}{\pgfqpoint{3.496832in}{0.960768in}}{\pgfqpoint{3.496832in}{0.971818in}}%
\pgfpathcurveto{\pgfqpoint{3.496832in}{0.982868in}}{\pgfqpoint{3.492442in}{0.993467in}}{\pgfqpoint{3.484628in}{1.001280in}}%
\pgfpathcurveto{\pgfqpoint{3.476815in}{1.009094in}}{\pgfqpoint{3.466216in}{1.013484in}}{\pgfqpoint{3.455166in}{1.013484in}}%
\pgfpathcurveto{\pgfqpoint{3.444115in}{1.013484in}}{\pgfqpoint{3.433516in}{1.009094in}}{\pgfqpoint{3.425703in}{1.001280in}}%
\pgfpathcurveto{\pgfqpoint{3.417889in}{0.993467in}}{\pgfqpoint{3.413499in}{0.982868in}}{\pgfqpoint{3.413499in}{0.971818in}}%
\pgfpathcurveto{\pgfqpoint{3.413499in}{0.960768in}}{\pgfqpoint{3.417889in}{0.950169in}}{\pgfqpoint{3.425703in}{0.942355in}}%
\pgfpathcurveto{\pgfqpoint{3.433516in}{0.934541in}}{\pgfqpoint{3.444115in}{0.930151in}}{\pgfqpoint{3.455166in}{0.930151in}}%
\pgfpathlineto{\pgfqpoint{3.455166in}{0.930151in}}%
\pgfpathclose%
\pgfusepath{stroke}%
\end{pgfscope}%
\begin{pgfscope}%
\pgfpathrectangle{\pgfqpoint{0.847223in}{0.554012in}}{\pgfqpoint{6.200000in}{4.620000in}}%
\pgfusepath{clip}%
\pgfsetbuttcap%
\pgfsetroundjoin%
\pgfsetlinewidth{1.003750pt}%
\definecolor{currentstroke}{rgb}{1.000000,0.000000,0.000000}%
\pgfsetstrokecolor{currentstroke}%
\pgfsetdash{}{0pt}%
\pgfpathmoveto{\pgfqpoint{3.460499in}{0.928717in}}%
\pgfpathcurveto{\pgfqpoint{3.471549in}{0.928717in}}{\pgfqpoint{3.482148in}{0.933107in}}{\pgfqpoint{3.489962in}{0.940921in}}%
\pgfpathcurveto{\pgfqpoint{3.497775in}{0.948735in}}{\pgfqpoint{3.502165in}{0.959334in}}{\pgfqpoint{3.502165in}{0.970384in}}%
\pgfpathcurveto{\pgfqpoint{3.502165in}{0.981434in}}{\pgfqpoint{3.497775in}{0.992033in}}{\pgfqpoint{3.489962in}{0.999847in}}%
\pgfpathcurveto{\pgfqpoint{3.482148in}{1.007660in}}{\pgfqpoint{3.471549in}{1.012051in}}{\pgfqpoint{3.460499in}{1.012051in}}%
\pgfpathcurveto{\pgfqpoint{3.449449in}{1.012051in}}{\pgfqpoint{3.438850in}{1.007660in}}{\pgfqpoint{3.431036in}{0.999847in}}%
\pgfpathcurveto{\pgfqpoint{3.423222in}{0.992033in}}{\pgfqpoint{3.418832in}{0.981434in}}{\pgfqpoint{3.418832in}{0.970384in}}%
\pgfpathcurveto{\pgfqpoint{3.418832in}{0.959334in}}{\pgfqpoint{3.423222in}{0.948735in}}{\pgfqpoint{3.431036in}{0.940921in}}%
\pgfpathcurveto{\pgfqpoint{3.438850in}{0.933107in}}{\pgfqpoint{3.449449in}{0.928717in}}{\pgfqpoint{3.460499in}{0.928717in}}%
\pgfpathlineto{\pgfqpoint{3.460499in}{0.928717in}}%
\pgfpathclose%
\pgfusepath{stroke}%
\end{pgfscope}%
\begin{pgfscope}%
\pgfpathrectangle{\pgfqpoint{0.847223in}{0.554012in}}{\pgfqpoint{6.200000in}{4.620000in}}%
\pgfusepath{clip}%
\pgfsetbuttcap%
\pgfsetroundjoin%
\pgfsetlinewidth{1.003750pt}%
\definecolor{currentstroke}{rgb}{1.000000,0.000000,0.000000}%
\pgfsetstrokecolor{currentstroke}%
\pgfsetdash{}{0pt}%
\pgfpathmoveto{\pgfqpoint{3.465832in}{0.927288in}}%
\pgfpathcurveto{\pgfqpoint{3.476882in}{0.927288in}}{\pgfqpoint{3.487481in}{0.931678in}}{\pgfqpoint{3.495295in}{0.939492in}}%
\pgfpathcurveto{\pgfqpoint{3.503108in}{0.947306in}}{\pgfqpoint{3.507499in}{0.957905in}}{\pgfqpoint{3.507499in}{0.968955in}}%
\pgfpathcurveto{\pgfqpoint{3.507499in}{0.980005in}}{\pgfqpoint{3.503108in}{0.990604in}}{\pgfqpoint{3.495295in}{0.998418in}}%
\pgfpathcurveto{\pgfqpoint{3.487481in}{1.006231in}}{\pgfqpoint{3.476882in}{1.010622in}}{\pgfqpoint{3.465832in}{1.010622in}}%
\pgfpathcurveto{\pgfqpoint{3.454782in}{1.010622in}}{\pgfqpoint{3.444183in}{1.006231in}}{\pgfqpoint{3.436369in}{0.998418in}}%
\pgfpathcurveto{\pgfqpoint{3.428556in}{0.990604in}}{\pgfqpoint{3.424165in}{0.980005in}}{\pgfqpoint{3.424165in}{0.968955in}}%
\pgfpathcurveto{\pgfqpoint{3.424165in}{0.957905in}}{\pgfqpoint{3.428556in}{0.947306in}}{\pgfqpoint{3.436369in}{0.939492in}}%
\pgfpathcurveto{\pgfqpoint{3.444183in}{0.931678in}}{\pgfqpoint{3.454782in}{0.927288in}}{\pgfqpoint{3.465832in}{0.927288in}}%
\pgfpathlineto{\pgfqpoint{3.465832in}{0.927288in}}%
\pgfpathclose%
\pgfusepath{stroke}%
\end{pgfscope}%
\begin{pgfscope}%
\pgfpathrectangle{\pgfqpoint{0.847223in}{0.554012in}}{\pgfqpoint{6.200000in}{4.620000in}}%
\pgfusepath{clip}%
\pgfsetbuttcap%
\pgfsetroundjoin%
\pgfsetlinewidth{1.003750pt}%
\definecolor{currentstroke}{rgb}{1.000000,0.000000,0.000000}%
\pgfsetstrokecolor{currentstroke}%
\pgfsetdash{}{0pt}%
\pgfpathmoveto{\pgfqpoint{3.471165in}{0.925864in}}%
\pgfpathcurveto{\pgfqpoint{3.482215in}{0.925864in}}{\pgfqpoint{3.492814in}{0.930254in}}{\pgfqpoint{3.500628in}{0.938068in}}%
\pgfpathcurveto{\pgfqpoint{3.508442in}{0.945882in}}{\pgfqpoint{3.512832in}{0.956481in}}{\pgfqpoint{3.512832in}{0.967531in}}%
\pgfpathcurveto{\pgfqpoint{3.512832in}{0.978581in}}{\pgfqpoint{3.508442in}{0.989180in}}{\pgfqpoint{3.500628in}{0.996994in}}%
\pgfpathcurveto{\pgfqpoint{3.492814in}{1.004807in}}{\pgfqpoint{3.482215in}{1.009197in}}{\pgfqpoint{3.471165in}{1.009197in}}%
\pgfpathcurveto{\pgfqpoint{3.460115in}{1.009197in}}{\pgfqpoint{3.449516in}{1.004807in}}{\pgfqpoint{3.441702in}{0.996994in}}%
\pgfpathcurveto{\pgfqpoint{3.433889in}{0.989180in}}{\pgfqpoint{3.429498in}{0.978581in}}{\pgfqpoint{3.429498in}{0.967531in}}%
\pgfpathcurveto{\pgfqpoint{3.429498in}{0.956481in}}{\pgfqpoint{3.433889in}{0.945882in}}{\pgfqpoint{3.441702in}{0.938068in}}%
\pgfpathcurveto{\pgfqpoint{3.449516in}{0.930254in}}{\pgfqpoint{3.460115in}{0.925864in}}{\pgfqpoint{3.471165in}{0.925864in}}%
\pgfpathlineto{\pgfqpoint{3.471165in}{0.925864in}}%
\pgfpathclose%
\pgfusepath{stroke}%
\end{pgfscope}%
\begin{pgfscope}%
\pgfpathrectangle{\pgfqpoint{0.847223in}{0.554012in}}{\pgfqpoint{6.200000in}{4.620000in}}%
\pgfusepath{clip}%
\pgfsetbuttcap%
\pgfsetroundjoin%
\pgfsetlinewidth{1.003750pt}%
\definecolor{currentstroke}{rgb}{1.000000,0.000000,0.000000}%
\pgfsetstrokecolor{currentstroke}%
\pgfsetdash{}{0pt}%
\pgfpathmoveto{\pgfqpoint{3.476498in}{0.924445in}}%
\pgfpathcurveto{\pgfqpoint{3.487549in}{0.924445in}}{\pgfqpoint{3.498148in}{0.928835in}}{\pgfqpoint{3.505961in}{0.936649in}}%
\pgfpathcurveto{\pgfqpoint{3.513775in}{0.944462in}}{\pgfqpoint{3.518165in}{0.955061in}}{\pgfqpoint{3.518165in}{0.966111in}}%
\pgfpathcurveto{\pgfqpoint{3.518165in}{0.977162in}}{\pgfqpoint{3.513775in}{0.987761in}}{\pgfqpoint{3.505961in}{0.995574in}}%
\pgfpathcurveto{\pgfqpoint{3.498148in}{1.003388in}}{\pgfqpoint{3.487549in}{1.007778in}}{\pgfqpoint{3.476498in}{1.007778in}}%
\pgfpathcurveto{\pgfqpoint{3.465448in}{1.007778in}}{\pgfqpoint{3.454849in}{1.003388in}}{\pgfqpoint{3.447036in}{0.995574in}}%
\pgfpathcurveto{\pgfqpoint{3.439222in}{0.987761in}}{\pgfqpoint{3.434832in}{0.977162in}}{\pgfqpoint{3.434832in}{0.966111in}}%
\pgfpathcurveto{\pgfqpoint{3.434832in}{0.955061in}}{\pgfqpoint{3.439222in}{0.944462in}}{\pgfqpoint{3.447036in}{0.936649in}}%
\pgfpathcurveto{\pgfqpoint{3.454849in}{0.928835in}}{\pgfqpoint{3.465448in}{0.924445in}}{\pgfqpoint{3.476498in}{0.924445in}}%
\pgfpathlineto{\pgfqpoint{3.476498in}{0.924445in}}%
\pgfpathclose%
\pgfusepath{stroke}%
\end{pgfscope}%
\begin{pgfscope}%
\pgfpathrectangle{\pgfqpoint{0.847223in}{0.554012in}}{\pgfqpoint{6.200000in}{4.620000in}}%
\pgfusepath{clip}%
\pgfsetbuttcap%
\pgfsetroundjoin%
\pgfsetlinewidth{1.003750pt}%
\definecolor{currentstroke}{rgb}{1.000000,0.000000,0.000000}%
\pgfsetstrokecolor{currentstroke}%
\pgfsetdash{}{0pt}%
\pgfpathmoveto{\pgfqpoint{3.481832in}{0.923030in}}%
\pgfpathcurveto{\pgfqpoint{3.492882in}{0.923030in}}{\pgfqpoint{3.503481in}{0.927421in}}{\pgfqpoint{3.511294in}{0.935234in}}%
\pgfpathcurveto{\pgfqpoint{3.519108in}{0.943048in}}{\pgfqpoint{3.523498in}{0.953647in}}{\pgfqpoint{3.523498in}{0.964697in}}%
\pgfpathcurveto{\pgfqpoint{3.523498in}{0.975747in}}{\pgfqpoint{3.519108in}{0.986346in}}{\pgfqpoint{3.511294in}{0.994160in}}%
\pgfpathcurveto{\pgfqpoint{3.503481in}{1.001973in}}{\pgfqpoint{3.492882in}{1.006364in}}{\pgfqpoint{3.481832in}{1.006364in}}%
\pgfpathcurveto{\pgfqpoint{3.470781in}{1.006364in}}{\pgfqpoint{3.460182in}{1.001973in}}{\pgfqpoint{3.452369in}{0.994160in}}%
\pgfpathcurveto{\pgfqpoint{3.444555in}{0.986346in}}{\pgfqpoint{3.440165in}{0.975747in}}{\pgfqpoint{3.440165in}{0.964697in}}%
\pgfpathcurveto{\pgfqpoint{3.440165in}{0.953647in}}{\pgfqpoint{3.444555in}{0.943048in}}{\pgfqpoint{3.452369in}{0.935234in}}%
\pgfpathcurveto{\pgfqpoint{3.460182in}{0.927421in}}{\pgfqpoint{3.470781in}{0.923030in}}{\pgfqpoint{3.481832in}{0.923030in}}%
\pgfpathlineto{\pgfqpoint{3.481832in}{0.923030in}}%
\pgfpathclose%
\pgfusepath{stroke}%
\end{pgfscope}%
\begin{pgfscope}%
\pgfpathrectangle{\pgfqpoint{0.847223in}{0.554012in}}{\pgfqpoint{6.200000in}{4.620000in}}%
\pgfusepath{clip}%
\pgfsetbuttcap%
\pgfsetroundjoin%
\pgfsetlinewidth{1.003750pt}%
\definecolor{currentstroke}{rgb}{1.000000,0.000000,0.000000}%
\pgfsetstrokecolor{currentstroke}%
\pgfsetdash{}{0pt}%
\pgfpathmoveto{\pgfqpoint{3.487165in}{0.921621in}}%
\pgfpathcurveto{\pgfqpoint{3.498215in}{0.921621in}}{\pgfqpoint{3.508814in}{0.926011in}}{\pgfqpoint{3.516628in}{0.933824in}}%
\pgfpathcurveto{\pgfqpoint{3.524441in}{0.941638in}}{\pgfqpoint{3.528831in}{0.952237in}}{\pgfqpoint{3.528831in}{0.963287in}}%
\pgfpathcurveto{\pgfqpoint{3.528831in}{0.974337in}}{\pgfqpoint{3.524441in}{0.984936in}}{\pgfqpoint{3.516628in}{0.992750in}}%
\pgfpathcurveto{\pgfqpoint{3.508814in}{1.000564in}}{\pgfqpoint{3.498215in}{1.004954in}}{\pgfqpoint{3.487165in}{1.004954in}}%
\pgfpathcurveto{\pgfqpoint{3.476115in}{1.004954in}}{\pgfqpoint{3.465516in}{1.000564in}}{\pgfqpoint{3.457702in}{0.992750in}}%
\pgfpathcurveto{\pgfqpoint{3.449888in}{0.984936in}}{\pgfqpoint{3.445498in}{0.974337in}}{\pgfqpoint{3.445498in}{0.963287in}}%
\pgfpathcurveto{\pgfqpoint{3.445498in}{0.952237in}}{\pgfqpoint{3.449888in}{0.941638in}}{\pgfqpoint{3.457702in}{0.933824in}}%
\pgfpathcurveto{\pgfqpoint{3.465516in}{0.926011in}}{\pgfqpoint{3.476115in}{0.921621in}}{\pgfqpoint{3.487165in}{0.921621in}}%
\pgfpathlineto{\pgfqpoint{3.487165in}{0.921621in}}%
\pgfpathclose%
\pgfusepath{stroke}%
\end{pgfscope}%
\begin{pgfscope}%
\pgfpathrectangle{\pgfqpoint{0.847223in}{0.554012in}}{\pgfqpoint{6.200000in}{4.620000in}}%
\pgfusepath{clip}%
\pgfsetbuttcap%
\pgfsetroundjoin%
\pgfsetlinewidth{1.003750pt}%
\definecolor{currentstroke}{rgb}{1.000000,0.000000,0.000000}%
\pgfsetstrokecolor{currentstroke}%
\pgfsetdash{}{0pt}%
\pgfpathmoveto{\pgfqpoint{3.492498in}{0.920216in}}%
\pgfpathcurveto{\pgfqpoint{3.503548in}{0.920216in}}{\pgfqpoint{3.514147in}{0.924606in}}{\pgfqpoint{3.521961in}{0.932419in}}%
\pgfpathcurveto{\pgfqpoint{3.529774in}{0.940233in}}{\pgfqpoint{3.534165in}{0.950832in}}{\pgfqpoint{3.534165in}{0.961882in}}%
\pgfpathcurveto{\pgfqpoint{3.534165in}{0.972932in}}{\pgfqpoint{3.529774in}{0.983531in}}{\pgfqpoint{3.521961in}{0.991345in}}%
\pgfpathcurveto{\pgfqpoint{3.514147in}{0.999159in}}{\pgfqpoint{3.503548in}{1.003549in}}{\pgfqpoint{3.492498in}{1.003549in}}%
\pgfpathcurveto{\pgfqpoint{3.481448in}{1.003549in}}{\pgfqpoint{3.470849in}{0.999159in}}{\pgfqpoint{3.463035in}{0.991345in}}%
\pgfpathcurveto{\pgfqpoint{3.455222in}{0.983531in}}{\pgfqpoint{3.450831in}{0.972932in}}{\pgfqpoint{3.450831in}{0.961882in}}%
\pgfpathcurveto{\pgfqpoint{3.450831in}{0.950832in}}{\pgfqpoint{3.455222in}{0.940233in}}{\pgfqpoint{3.463035in}{0.932419in}}%
\pgfpathcurveto{\pgfqpoint{3.470849in}{0.924606in}}{\pgfqpoint{3.481448in}{0.920216in}}{\pgfqpoint{3.492498in}{0.920216in}}%
\pgfpathlineto{\pgfqpoint{3.492498in}{0.920216in}}%
\pgfpathclose%
\pgfusepath{stroke}%
\end{pgfscope}%
\begin{pgfscope}%
\pgfpathrectangle{\pgfqpoint{0.847223in}{0.554012in}}{\pgfqpoint{6.200000in}{4.620000in}}%
\pgfusepath{clip}%
\pgfsetbuttcap%
\pgfsetroundjoin%
\pgfsetlinewidth{1.003750pt}%
\definecolor{currentstroke}{rgb}{1.000000,0.000000,0.000000}%
\pgfsetstrokecolor{currentstroke}%
\pgfsetdash{}{0pt}%
\pgfpathmoveto{\pgfqpoint{3.497831in}{0.918815in}}%
\pgfpathcurveto{\pgfqpoint{3.508881in}{0.918815in}}{\pgfqpoint{3.519480in}{0.923206in}}{\pgfqpoint{3.527294in}{0.931019in}}%
\pgfpathcurveto{\pgfqpoint{3.535108in}{0.938833in}}{\pgfqpoint{3.539498in}{0.949432in}}{\pgfqpoint{3.539498in}{0.960482in}}%
\pgfpathcurveto{\pgfqpoint{3.539498in}{0.971532in}}{\pgfqpoint{3.535108in}{0.982131in}}{\pgfqpoint{3.527294in}{0.989945in}}%
\pgfpathcurveto{\pgfqpoint{3.519480in}{0.997758in}}{\pgfqpoint{3.508881in}{1.002149in}}{\pgfqpoint{3.497831in}{1.002149in}}%
\pgfpathcurveto{\pgfqpoint{3.486781in}{1.002149in}}{\pgfqpoint{3.476182in}{0.997758in}}{\pgfqpoint{3.468368in}{0.989945in}}%
\pgfpathcurveto{\pgfqpoint{3.460555in}{0.982131in}}{\pgfqpoint{3.456165in}{0.971532in}}{\pgfqpoint{3.456165in}{0.960482in}}%
\pgfpathcurveto{\pgfqpoint{3.456165in}{0.949432in}}{\pgfqpoint{3.460555in}{0.938833in}}{\pgfqpoint{3.468368in}{0.931019in}}%
\pgfpathcurveto{\pgfqpoint{3.476182in}{0.923206in}}{\pgfqpoint{3.486781in}{0.918815in}}{\pgfqpoint{3.497831in}{0.918815in}}%
\pgfpathlineto{\pgfqpoint{3.497831in}{0.918815in}}%
\pgfpathclose%
\pgfusepath{stroke}%
\end{pgfscope}%
\begin{pgfscope}%
\pgfpathrectangle{\pgfqpoint{0.847223in}{0.554012in}}{\pgfqpoint{6.200000in}{4.620000in}}%
\pgfusepath{clip}%
\pgfsetbuttcap%
\pgfsetroundjoin%
\pgfsetlinewidth{1.003750pt}%
\definecolor{currentstroke}{rgb}{1.000000,0.000000,0.000000}%
\pgfsetstrokecolor{currentstroke}%
\pgfsetdash{}{0pt}%
\pgfpathmoveto{\pgfqpoint{3.503164in}{0.917420in}}%
\pgfpathcurveto{\pgfqpoint{3.514215in}{0.917420in}}{\pgfqpoint{3.524814in}{0.921810in}}{\pgfqpoint{3.532627in}{0.929624in}}%
\pgfpathcurveto{\pgfqpoint{3.540441in}{0.937437in}}{\pgfqpoint{3.544831in}{0.948036in}}{\pgfqpoint{3.544831in}{0.959086in}}%
\pgfpathcurveto{\pgfqpoint{3.544831in}{0.970137in}}{\pgfqpoint{3.540441in}{0.980736in}}{\pgfqpoint{3.532627in}{0.988549in}}%
\pgfpathcurveto{\pgfqpoint{3.524814in}{0.996363in}}{\pgfqpoint{3.514215in}{1.000753in}}{\pgfqpoint{3.503164in}{1.000753in}}%
\pgfpathcurveto{\pgfqpoint{3.492114in}{1.000753in}}{\pgfqpoint{3.481515in}{0.996363in}}{\pgfqpoint{3.473702in}{0.988549in}}%
\pgfpathcurveto{\pgfqpoint{3.465888in}{0.980736in}}{\pgfqpoint{3.461498in}{0.970137in}}{\pgfqpoint{3.461498in}{0.959086in}}%
\pgfpathcurveto{\pgfqpoint{3.461498in}{0.948036in}}{\pgfqpoint{3.465888in}{0.937437in}}{\pgfqpoint{3.473702in}{0.929624in}}%
\pgfpathcurveto{\pgfqpoint{3.481515in}{0.921810in}}{\pgfqpoint{3.492114in}{0.917420in}}{\pgfqpoint{3.503164in}{0.917420in}}%
\pgfpathlineto{\pgfqpoint{3.503164in}{0.917420in}}%
\pgfpathclose%
\pgfusepath{stroke}%
\end{pgfscope}%
\begin{pgfscope}%
\pgfpathrectangle{\pgfqpoint{0.847223in}{0.554012in}}{\pgfqpoint{6.200000in}{4.620000in}}%
\pgfusepath{clip}%
\pgfsetbuttcap%
\pgfsetroundjoin%
\pgfsetlinewidth{1.003750pt}%
\definecolor{currentstroke}{rgb}{1.000000,0.000000,0.000000}%
\pgfsetstrokecolor{currentstroke}%
\pgfsetdash{}{0pt}%
\pgfpathmoveto{\pgfqpoint{3.508498in}{0.916029in}}%
\pgfpathcurveto{\pgfqpoint{3.519548in}{0.916029in}}{\pgfqpoint{3.530147in}{0.920419in}}{\pgfqpoint{3.537960in}{0.928233in}}%
\pgfpathcurveto{\pgfqpoint{3.545774in}{0.936046in}}{\pgfqpoint{3.550164in}{0.946645in}}{\pgfqpoint{3.550164in}{0.957696in}}%
\pgfpathcurveto{\pgfqpoint{3.550164in}{0.968746in}}{\pgfqpoint{3.545774in}{0.979345in}}{\pgfqpoint{3.537960in}{0.987158in}}%
\pgfpathcurveto{\pgfqpoint{3.530147in}{0.994972in}}{\pgfqpoint{3.519548in}{0.999362in}}{\pgfqpoint{3.508498in}{0.999362in}}%
\pgfpathcurveto{\pgfqpoint{3.497448in}{0.999362in}}{\pgfqpoint{3.486849in}{0.994972in}}{\pgfqpoint{3.479035in}{0.987158in}}%
\pgfpathcurveto{\pgfqpoint{3.471221in}{0.979345in}}{\pgfqpoint{3.466831in}{0.968746in}}{\pgfqpoint{3.466831in}{0.957696in}}%
\pgfpathcurveto{\pgfqpoint{3.466831in}{0.946645in}}{\pgfqpoint{3.471221in}{0.936046in}}{\pgfqpoint{3.479035in}{0.928233in}}%
\pgfpathcurveto{\pgfqpoint{3.486849in}{0.920419in}}{\pgfqpoint{3.497448in}{0.916029in}}{\pgfqpoint{3.508498in}{0.916029in}}%
\pgfpathlineto{\pgfqpoint{3.508498in}{0.916029in}}%
\pgfpathclose%
\pgfusepath{stroke}%
\end{pgfscope}%
\begin{pgfscope}%
\pgfpathrectangle{\pgfqpoint{0.847223in}{0.554012in}}{\pgfqpoint{6.200000in}{4.620000in}}%
\pgfusepath{clip}%
\pgfsetbuttcap%
\pgfsetroundjoin%
\pgfsetlinewidth{1.003750pt}%
\definecolor{currentstroke}{rgb}{1.000000,0.000000,0.000000}%
\pgfsetstrokecolor{currentstroke}%
\pgfsetdash{}{0pt}%
\pgfpathmoveto{\pgfqpoint{3.513831in}{0.914643in}}%
\pgfpathcurveto{\pgfqpoint{3.524881in}{0.914643in}}{\pgfqpoint{3.535480in}{0.919033in}}{\pgfqpoint{3.543294in}{0.926846in}}%
\pgfpathcurveto{\pgfqpoint{3.551107in}{0.934660in}}{\pgfqpoint{3.555498in}{0.945259in}}{\pgfqpoint{3.555498in}{0.956309in}}%
\pgfpathcurveto{\pgfqpoint{3.555498in}{0.967359in}}{\pgfqpoint{3.551107in}{0.977958in}}{\pgfqpoint{3.543294in}{0.985772in}}%
\pgfpathcurveto{\pgfqpoint{3.535480in}{0.993586in}}{\pgfqpoint{3.524881in}{0.997976in}}{\pgfqpoint{3.513831in}{0.997976in}}%
\pgfpathcurveto{\pgfqpoint{3.502781in}{0.997976in}}{\pgfqpoint{3.492182in}{0.993586in}}{\pgfqpoint{3.484368in}{0.985772in}}%
\pgfpathcurveto{\pgfqpoint{3.476554in}{0.977958in}}{\pgfqpoint{3.472164in}{0.967359in}}{\pgfqpoint{3.472164in}{0.956309in}}%
\pgfpathcurveto{\pgfqpoint{3.472164in}{0.945259in}}{\pgfqpoint{3.476554in}{0.934660in}}{\pgfqpoint{3.484368in}{0.926846in}}%
\pgfpathcurveto{\pgfqpoint{3.492182in}{0.919033in}}{\pgfqpoint{3.502781in}{0.914643in}}{\pgfqpoint{3.513831in}{0.914643in}}%
\pgfpathlineto{\pgfqpoint{3.513831in}{0.914643in}}%
\pgfpathclose%
\pgfusepath{stroke}%
\end{pgfscope}%
\begin{pgfscope}%
\pgfpathrectangle{\pgfqpoint{0.847223in}{0.554012in}}{\pgfqpoint{6.200000in}{4.620000in}}%
\pgfusepath{clip}%
\pgfsetbuttcap%
\pgfsetroundjoin%
\pgfsetlinewidth{1.003750pt}%
\definecolor{currentstroke}{rgb}{1.000000,0.000000,0.000000}%
\pgfsetstrokecolor{currentstroke}%
\pgfsetdash{}{0pt}%
\pgfpathmoveto{\pgfqpoint{3.519164in}{0.913261in}}%
\pgfpathcurveto{\pgfqpoint{3.530214in}{0.913261in}}{\pgfqpoint{3.540813in}{0.917651in}}{\pgfqpoint{3.548627in}{0.925465in}}%
\pgfpathcurveto{\pgfqpoint{3.556441in}{0.933278in}}{\pgfqpoint{3.560831in}{0.943878in}}{\pgfqpoint{3.560831in}{0.954928in}}%
\pgfpathcurveto{\pgfqpoint{3.560831in}{0.965978in}}{\pgfqpoint{3.556441in}{0.976577in}}{\pgfqpoint{3.548627in}{0.984390in}}%
\pgfpathcurveto{\pgfqpoint{3.540813in}{0.992204in}}{\pgfqpoint{3.530214in}{0.996594in}}{\pgfqpoint{3.519164in}{0.996594in}}%
\pgfpathcurveto{\pgfqpoint{3.508114in}{0.996594in}}{\pgfqpoint{3.497515in}{0.992204in}}{\pgfqpoint{3.489701in}{0.984390in}}%
\pgfpathcurveto{\pgfqpoint{3.481888in}{0.976577in}}{\pgfqpoint{3.477497in}{0.965978in}}{\pgfqpoint{3.477497in}{0.954928in}}%
\pgfpathcurveto{\pgfqpoint{3.477497in}{0.943878in}}{\pgfqpoint{3.481888in}{0.933278in}}{\pgfqpoint{3.489701in}{0.925465in}}%
\pgfpathcurveto{\pgfqpoint{3.497515in}{0.917651in}}{\pgfqpoint{3.508114in}{0.913261in}}{\pgfqpoint{3.519164in}{0.913261in}}%
\pgfpathlineto{\pgfqpoint{3.519164in}{0.913261in}}%
\pgfpathclose%
\pgfusepath{stroke}%
\end{pgfscope}%
\begin{pgfscope}%
\pgfpathrectangle{\pgfqpoint{0.847223in}{0.554012in}}{\pgfqpoint{6.200000in}{4.620000in}}%
\pgfusepath{clip}%
\pgfsetbuttcap%
\pgfsetroundjoin%
\pgfsetlinewidth{1.003750pt}%
\definecolor{currentstroke}{rgb}{1.000000,0.000000,0.000000}%
\pgfsetstrokecolor{currentstroke}%
\pgfsetdash{}{0pt}%
\pgfpathmoveto{\pgfqpoint{3.524497in}{0.911884in}}%
\pgfpathcurveto{\pgfqpoint{3.535547in}{0.911884in}}{\pgfqpoint{3.546146in}{0.916274in}}{\pgfqpoint{3.553960in}{0.924088in}}%
\pgfpathcurveto{\pgfqpoint{3.561774in}{0.931901in}}{\pgfqpoint{3.566164in}{0.942501in}}{\pgfqpoint{3.566164in}{0.953551in}}%
\pgfpathcurveto{\pgfqpoint{3.566164in}{0.964601in}}{\pgfqpoint{3.561774in}{0.975200in}}{\pgfqpoint{3.553960in}{0.983013in}}%
\pgfpathcurveto{\pgfqpoint{3.546146in}{0.990827in}}{\pgfqpoint{3.535547in}{0.995217in}}{\pgfqpoint{3.524497in}{0.995217in}}%
\pgfpathcurveto{\pgfqpoint{3.513447in}{0.995217in}}{\pgfqpoint{3.502848in}{0.990827in}}{\pgfqpoint{3.495035in}{0.983013in}}%
\pgfpathcurveto{\pgfqpoint{3.487221in}{0.975200in}}{\pgfqpoint{3.482831in}{0.964601in}}{\pgfqpoint{3.482831in}{0.953551in}}%
\pgfpathcurveto{\pgfqpoint{3.482831in}{0.942501in}}{\pgfqpoint{3.487221in}{0.931901in}}{\pgfqpoint{3.495035in}{0.924088in}}%
\pgfpathcurveto{\pgfqpoint{3.502848in}{0.916274in}}{\pgfqpoint{3.513447in}{0.911884in}}{\pgfqpoint{3.524497in}{0.911884in}}%
\pgfpathlineto{\pgfqpoint{3.524497in}{0.911884in}}%
\pgfpathclose%
\pgfusepath{stroke}%
\end{pgfscope}%
\begin{pgfscope}%
\pgfpathrectangle{\pgfqpoint{0.847223in}{0.554012in}}{\pgfqpoint{6.200000in}{4.620000in}}%
\pgfusepath{clip}%
\pgfsetbuttcap%
\pgfsetroundjoin%
\pgfsetlinewidth{1.003750pt}%
\definecolor{currentstroke}{rgb}{1.000000,0.000000,0.000000}%
\pgfsetstrokecolor{currentstroke}%
\pgfsetdash{}{0pt}%
\pgfpathmoveto{\pgfqpoint{3.529831in}{0.910512in}}%
\pgfpathcurveto{\pgfqpoint{3.540881in}{0.910512in}}{\pgfqpoint{3.551480in}{0.914902in}}{\pgfqpoint{3.559293in}{0.922715in}}%
\pgfpathcurveto{\pgfqpoint{3.567107in}{0.930529in}}{\pgfqpoint{3.571497in}{0.941128in}}{\pgfqpoint{3.571497in}{0.952178in}}%
\pgfpathcurveto{\pgfqpoint{3.571497in}{0.963228in}}{\pgfqpoint{3.567107in}{0.973827in}}{\pgfqpoint{3.559293in}{0.981641in}}%
\pgfpathcurveto{\pgfqpoint{3.551480in}{0.989455in}}{\pgfqpoint{3.540881in}{0.993845in}}{\pgfqpoint{3.529831in}{0.993845in}}%
\pgfpathcurveto{\pgfqpoint{3.518780in}{0.993845in}}{\pgfqpoint{3.508181in}{0.989455in}}{\pgfqpoint{3.500368in}{0.981641in}}%
\pgfpathcurveto{\pgfqpoint{3.492554in}{0.973827in}}{\pgfqpoint{3.488164in}{0.963228in}}{\pgfqpoint{3.488164in}{0.952178in}}%
\pgfpathcurveto{\pgfqpoint{3.488164in}{0.941128in}}{\pgfqpoint{3.492554in}{0.930529in}}{\pgfqpoint{3.500368in}{0.922715in}}%
\pgfpathcurveto{\pgfqpoint{3.508181in}{0.914902in}}{\pgfqpoint{3.518780in}{0.910512in}}{\pgfqpoint{3.529831in}{0.910512in}}%
\pgfpathlineto{\pgfqpoint{3.529831in}{0.910512in}}%
\pgfpathclose%
\pgfusepath{stroke}%
\end{pgfscope}%
\begin{pgfscope}%
\pgfpathrectangle{\pgfqpoint{0.847223in}{0.554012in}}{\pgfqpoint{6.200000in}{4.620000in}}%
\pgfusepath{clip}%
\pgfsetbuttcap%
\pgfsetroundjoin%
\pgfsetlinewidth{1.003750pt}%
\definecolor{currentstroke}{rgb}{1.000000,0.000000,0.000000}%
\pgfsetstrokecolor{currentstroke}%
\pgfsetdash{}{0pt}%
\pgfpathmoveto{\pgfqpoint{3.535164in}{0.909144in}}%
\pgfpathcurveto{\pgfqpoint{3.546214in}{0.909144in}}{\pgfqpoint{3.556813in}{0.913534in}}{\pgfqpoint{3.564627in}{0.921348in}}%
\pgfpathcurveto{\pgfqpoint{3.572440in}{0.929161in}}{\pgfqpoint{3.576830in}{0.939760in}}{\pgfqpoint{3.576830in}{0.950810in}}%
\pgfpathcurveto{\pgfqpoint{3.576830in}{0.961860in}}{\pgfqpoint{3.572440in}{0.972460in}}{\pgfqpoint{3.564627in}{0.980273in}}%
\pgfpathcurveto{\pgfqpoint{3.556813in}{0.988087in}}{\pgfqpoint{3.546214in}{0.992477in}}{\pgfqpoint{3.535164in}{0.992477in}}%
\pgfpathcurveto{\pgfqpoint{3.524114in}{0.992477in}}{\pgfqpoint{3.513515in}{0.988087in}}{\pgfqpoint{3.505701in}{0.980273in}}%
\pgfpathcurveto{\pgfqpoint{3.497887in}{0.972460in}}{\pgfqpoint{3.493497in}{0.961860in}}{\pgfqpoint{3.493497in}{0.950810in}}%
\pgfpathcurveto{\pgfqpoint{3.493497in}{0.939760in}}{\pgfqpoint{3.497887in}{0.929161in}}{\pgfqpoint{3.505701in}{0.921348in}}%
\pgfpathcurveto{\pgfqpoint{3.513515in}{0.913534in}}{\pgfqpoint{3.524114in}{0.909144in}}{\pgfqpoint{3.535164in}{0.909144in}}%
\pgfpathlineto{\pgfqpoint{3.535164in}{0.909144in}}%
\pgfpathclose%
\pgfusepath{stroke}%
\end{pgfscope}%
\begin{pgfscope}%
\pgfpathrectangle{\pgfqpoint{0.847223in}{0.554012in}}{\pgfqpoint{6.200000in}{4.620000in}}%
\pgfusepath{clip}%
\pgfsetbuttcap%
\pgfsetroundjoin%
\pgfsetlinewidth{1.003750pt}%
\definecolor{currentstroke}{rgb}{1.000000,0.000000,0.000000}%
\pgfsetstrokecolor{currentstroke}%
\pgfsetdash{}{0pt}%
\pgfpathmoveto{\pgfqpoint{3.540497in}{0.907780in}}%
\pgfpathcurveto{\pgfqpoint{3.551547in}{0.907780in}}{\pgfqpoint{3.562146in}{0.912171in}}{\pgfqpoint{3.569960in}{0.919984in}}%
\pgfpathcurveto{\pgfqpoint{3.577773in}{0.927798in}}{\pgfqpoint{3.582164in}{0.938397in}}{\pgfqpoint{3.582164in}{0.949447in}}%
\pgfpathcurveto{\pgfqpoint{3.582164in}{0.960497in}}{\pgfqpoint{3.577773in}{0.971096in}}{\pgfqpoint{3.569960in}{0.978910in}}%
\pgfpathcurveto{\pgfqpoint{3.562146in}{0.986723in}}{\pgfqpoint{3.551547in}{0.991114in}}{\pgfqpoint{3.540497in}{0.991114in}}%
\pgfpathcurveto{\pgfqpoint{3.529447in}{0.991114in}}{\pgfqpoint{3.518848in}{0.986723in}}{\pgfqpoint{3.511034in}{0.978910in}}%
\pgfpathcurveto{\pgfqpoint{3.503221in}{0.971096in}}{\pgfqpoint{3.498830in}{0.960497in}}{\pgfqpoint{3.498830in}{0.949447in}}%
\pgfpathcurveto{\pgfqpoint{3.498830in}{0.938397in}}{\pgfqpoint{3.503221in}{0.927798in}}{\pgfqpoint{3.511034in}{0.919984in}}%
\pgfpathcurveto{\pgfqpoint{3.518848in}{0.912171in}}{\pgfqpoint{3.529447in}{0.907780in}}{\pgfqpoint{3.540497in}{0.907780in}}%
\pgfpathlineto{\pgfqpoint{3.540497in}{0.907780in}}%
\pgfpathclose%
\pgfusepath{stroke}%
\end{pgfscope}%
\begin{pgfscope}%
\pgfpathrectangle{\pgfqpoint{0.847223in}{0.554012in}}{\pgfqpoint{6.200000in}{4.620000in}}%
\pgfusepath{clip}%
\pgfsetbuttcap%
\pgfsetroundjoin%
\pgfsetlinewidth{1.003750pt}%
\definecolor{currentstroke}{rgb}{1.000000,0.000000,0.000000}%
\pgfsetstrokecolor{currentstroke}%
\pgfsetdash{}{0pt}%
\pgfpathmoveto{\pgfqpoint{3.545830in}{0.906422in}}%
\pgfpathcurveto{\pgfqpoint{3.556880in}{0.906422in}}{\pgfqpoint{3.567479in}{0.910812in}}{\pgfqpoint{3.575293in}{0.918625in}}%
\pgfpathcurveto{\pgfqpoint{3.583107in}{0.926439in}}{\pgfqpoint{3.587497in}{0.937038in}}{\pgfqpoint{3.587497in}{0.948088in}}%
\pgfpathcurveto{\pgfqpoint{3.587497in}{0.959138in}}{\pgfqpoint{3.583107in}{0.969737in}}{\pgfqpoint{3.575293in}{0.977551in}}%
\pgfpathcurveto{\pgfqpoint{3.567479in}{0.985365in}}{\pgfqpoint{3.556880in}{0.989755in}}{\pgfqpoint{3.545830in}{0.989755in}}%
\pgfpathcurveto{\pgfqpoint{3.534780in}{0.989755in}}{\pgfqpoint{3.524181in}{0.985365in}}{\pgfqpoint{3.516367in}{0.977551in}}%
\pgfpathcurveto{\pgfqpoint{3.508554in}{0.969737in}}{\pgfqpoint{3.504164in}{0.959138in}}{\pgfqpoint{3.504164in}{0.948088in}}%
\pgfpathcurveto{\pgfqpoint{3.504164in}{0.937038in}}{\pgfqpoint{3.508554in}{0.926439in}}{\pgfqpoint{3.516367in}{0.918625in}}%
\pgfpathcurveto{\pgfqpoint{3.524181in}{0.910812in}}{\pgfqpoint{3.534780in}{0.906422in}}{\pgfqpoint{3.545830in}{0.906422in}}%
\pgfpathlineto{\pgfqpoint{3.545830in}{0.906422in}}%
\pgfpathclose%
\pgfusepath{stroke}%
\end{pgfscope}%
\begin{pgfscope}%
\pgfpathrectangle{\pgfqpoint{0.847223in}{0.554012in}}{\pgfqpoint{6.200000in}{4.620000in}}%
\pgfusepath{clip}%
\pgfsetbuttcap%
\pgfsetroundjoin%
\pgfsetlinewidth{1.003750pt}%
\definecolor{currentstroke}{rgb}{1.000000,0.000000,0.000000}%
\pgfsetstrokecolor{currentstroke}%
\pgfsetdash{}{0pt}%
\pgfpathmoveto{\pgfqpoint{3.551163in}{0.905067in}}%
\pgfpathcurveto{\pgfqpoint{3.562214in}{0.905067in}}{\pgfqpoint{3.572813in}{0.909457in}}{\pgfqpoint{3.580626in}{0.917271in}}%
\pgfpathcurveto{\pgfqpoint{3.588440in}{0.925085in}}{\pgfqpoint{3.592830in}{0.935684in}}{\pgfqpoint{3.592830in}{0.946734in}}%
\pgfpathcurveto{\pgfqpoint{3.592830in}{0.957784in}}{\pgfqpoint{3.588440in}{0.968383in}}{\pgfqpoint{3.580626in}{0.976197in}}%
\pgfpathcurveto{\pgfqpoint{3.572813in}{0.984010in}}{\pgfqpoint{3.562214in}{0.988401in}}{\pgfqpoint{3.551163in}{0.988401in}}%
\pgfpathcurveto{\pgfqpoint{3.540113in}{0.988401in}}{\pgfqpoint{3.529514in}{0.984010in}}{\pgfqpoint{3.521701in}{0.976197in}}%
\pgfpathcurveto{\pgfqpoint{3.513887in}{0.968383in}}{\pgfqpoint{3.509497in}{0.957784in}}{\pgfqpoint{3.509497in}{0.946734in}}%
\pgfpathcurveto{\pgfqpoint{3.509497in}{0.935684in}}{\pgfqpoint{3.513887in}{0.925085in}}{\pgfqpoint{3.521701in}{0.917271in}}%
\pgfpathcurveto{\pgfqpoint{3.529514in}{0.909457in}}{\pgfqpoint{3.540113in}{0.905067in}}{\pgfqpoint{3.551163in}{0.905067in}}%
\pgfpathlineto{\pgfqpoint{3.551163in}{0.905067in}}%
\pgfpathclose%
\pgfusepath{stroke}%
\end{pgfscope}%
\begin{pgfscope}%
\pgfpathrectangle{\pgfqpoint{0.847223in}{0.554012in}}{\pgfqpoint{6.200000in}{4.620000in}}%
\pgfusepath{clip}%
\pgfsetbuttcap%
\pgfsetroundjoin%
\pgfsetlinewidth{1.003750pt}%
\definecolor{currentstroke}{rgb}{1.000000,0.000000,0.000000}%
\pgfsetstrokecolor{currentstroke}%
\pgfsetdash{}{0pt}%
\pgfpathmoveto{\pgfqpoint{3.556497in}{0.903717in}}%
\pgfpathcurveto{\pgfqpoint{3.567547in}{0.903717in}}{\pgfqpoint{3.578146in}{0.908108in}}{\pgfqpoint{3.585959in}{0.915921in}}%
\pgfpathcurveto{\pgfqpoint{3.593773in}{0.923735in}}{\pgfqpoint{3.598163in}{0.934334in}}{\pgfqpoint{3.598163in}{0.945384in}}%
\pgfpathcurveto{\pgfqpoint{3.598163in}{0.956434in}}{\pgfqpoint{3.593773in}{0.967033in}}{\pgfqpoint{3.585959in}{0.974847in}}%
\pgfpathcurveto{\pgfqpoint{3.578146in}{0.982660in}}{\pgfqpoint{3.567547in}{0.987051in}}{\pgfqpoint{3.556497in}{0.987051in}}%
\pgfpathcurveto{\pgfqpoint{3.545446in}{0.987051in}}{\pgfqpoint{3.534847in}{0.982660in}}{\pgfqpoint{3.527034in}{0.974847in}}%
\pgfpathcurveto{\pgfqpoint{3.519220in}{0.967033in}}{\pgfqpoint{3.514830in}{0.956434in}}{\pgfqpoint{3.514830in}{0.945384in}}%
\pgfpathcurveto{\pgfqpoint{3.514830in}{0.934334in}}{\pgfqpoint{3.519220in}{0.923735in}}{\pgfqpoint{3.527034in}{0.915921in}}%
\pgfpathcurveto{\pgfqpoint{3.534847in}{0.908108in}}{\pgfqpoint{3.545446in}{0.903717in}}{\pgfqpoint{3.556497in}{0.903717in}}%
\pgfpathlineto{\pgfqpoint{3.556497in}{0.903717in}}%
\pgfpathclose%
\pgfusepath{stroke}%
\end{pgfscope}%
\begin{pgfscope}%
\pgfpathrectangle{\pgfqpoint{0.847223in}{0.554012in}}{\pgfqpoint{6.200000in}{4.620000in}}%
\pgfusepath{clip}%
\pgfsetbuttcap%
\pgfsetroundjoin%
\pgfsetlinewidth{1.003750pt}%
\definecolor{currentstroke}{rgb}{1.000000,0.000000,0.000000}%
\pgfsetstrokecolor{currentstroke}%
\pgfsetdash{}{0pt}%
\pgfpathmoveto{\pgfqpoint{3.561830in}{0.902372in}}%
\pgfpathcurveto{\pgfqpoint{3.572880in}{0.902372in}}{\pgfqpoint{3.583479in}{0.906762in}}{\pgfqpoint{3.591293in}{0.914576in}}%
\pgfpathcurveto{\pgfqpoint{3.599106in}{0.922389in}}{\pgfqpoint{3.603496in}{0.932988in}}{\pgfqpoint{3.603496in}{0.944039in}}%
\pgfpathcurveto{\pgfqpoint{3.603496in}{0.955089in}}{\pgfqpoint{3.599106in}{0.965688in}}{\pgfqpoint{3.591293in}{0.973501in}}%
\pgfpathcurveto{\pgfqpoint{3.583479in}{0.981315in}}{\pgfqpoint{3.572880in}{0.985705in}}{\pgfqpoint{3.561830in}{0.985705in}}%
\pgfpathcurveto{\pgfqpoint{3.550780in}{0.985705in}}{\pgfqpoint{3.540181in}{0.981315in}}{\pgfqpoint{3.532367in}{0.973501in}}%
\pgfpathcurveto{\pgfqpoint{3.524553in}{0.965688in}}{\pgfqpoint{3.520163in}{0.955089in}}{\pgfqpoint{3.520163in}{0.944039in}}%
\pgfpathcurveto{\pgfqpoint{3.520163in}{0.932988in}}{\pgfqpoint{3.524553in}{0.922389in}}{\pgfqpoint{3.532367in}{0.914576in}}%
\pgfpathcurveto{\pgfqpoint{3.540181in}{0.906762in}}{\pgfqpoint{3.550780in}{0.902372in}}{\pgfqpoint{3.561830in}{0.902372in}}%
\pgfpathlineto{\pgfqpoint{3.561830in}{0.902372in}}%
\pgfpathclose%
\pgfusepath{stroke}%
\end{pgfscope}%
\begin{pgfscope}%
\pgfpathrectangle{\pgfqpoint{0.847223in}{0.554012in}}{\pgfqpoint{6.200000in}{4.620000in}}%
\pgfusepath{clip}%
\pgfsetbuttcap%
\pgfsetroundjoin%
\pgfsetlinewidth{1.003750pt}%
\definecolor{currentstroke}{rgb}{1.000000,0.000000,0.000000}%
\pgfsetstrokecolor{currentstroke}%
\pgfsetdash{}{0pt}%
\pgfpathmoveto{\pgfqpoint{3.567163in}{0.901031in}}%
\pgfpathcurveto{\pgfqpoint{3.578213in}{0.901031in}}{\pgfqpoint{3.588812in}{0.905421in}}{\pgfqpoint{3.596626in}{0.913235in}}%
\pgfpathcurveto{\pgfqpoint{3.604439in}{0.921048in}}{\pgfqpoint{3.608830in}{0.931647in}}{\pgfqpoint{3.608830in}{0.942698in}}%
\pgfpathcurveto{\pgfqpoint{3.608830in}{0.953748in}}{\pgfqpoint{3.604439in}{0.964347in}}{\pgfqpoint{3.596626in}{0.972160in}}%
\pgfpathcurveto{\pgfqpoint{3.588812in}{0.979974in}}{\pgfqpoint{3.578213in}{0.984364in}}{\pgfqpoint{3.567163in}{0.984364in}}%
\pgfpathcurveto{\pgfqpoint{3.556113in}{0.984364in}}{\pgfqpoint{3.545514in}{0.979974in}}{\pgfqpoint{3.537700in}{0.972160in}}%
\pgfpathcurveto{\pgfqpoint{3.529887in}{0.964347in}}{\pgfqpoint{3.525496in}{0.953748in}}{\pgfqpoint{3.525496in}{0.942698in}}%
\pgfpathcurveto{\pgfqpoint{3.525496in}{0.931647in}}{\pgfqpoint{3.529887in}{0.921048in}}{\pgfqpoint{3.537700in}{0.913235in}}%
\pgfpathcurveto{\pgfqpoint{3.545514in}{0.905421in}}{\pgfqpoint{3.556113in}{0.901031in}}{\pgfqpoint{3.567163in}{0.901031in}}%
\pgfpathlineto{\pgfqpoint{3.567163in}{0.901031in}}%
\pgfpathclose%
\pgfusepath{stroke}%
\end{pgfscope}%
\begin{pgfscope}%
\pgfpathrectangle{\pgfqpoint{0.847223in}{0.554012in}}{\pgfqpoint{6.200000in}{4.620000in}}%
\pgfusepath{clip}%
\pgfsetbuttcap%
\pgfsetroundjoin%
\pgfsetlinewidth{1.003750pt}%
\definecolor{currentstroke}{rgb}{1.000000,0.000000,0.000000}%
\pgfsetstrokecolor{currentstroke}%
\pgfsetdash{}{0pt}%
\pgfpathmoveto{\pgfqpoint{3.572496in}{0.899694in}}%
\pgfpathcurveto{\pgfqpoint{3.583546in}{0.899694in}}{\pgfqpoint{3.594145in}{0.904085in}}{\pgfqpoint{3.601959in}{0.911898in}}%
\pgfpathcurveto{\pgfqpoint{3.609773in}{0.919712in}}{\pgfqpoint{3.614163in}{0.930311in}}{\pgfqpoint{3.614163in}{0.941361in}}%
\pgfpathcurveto{\pgfqpoint{3.614163in}{0.952411in}}{\pgfqpoint{3.609773in}{0.963010in}}{\pgfqpoint{3.601959in}{0.970824in}}%
\pgfpathcurveto{\pgfqpoint{3.594145in}{0.978637in}}{\pgfqpoint{3.583546in}{0.983028in}}{\pgfqpoint{3.572496in}{0.983028in}}%
\pgfpathcurveto{\pgfqpoint{3.561446in}{0.983028in}}{\pgfqpoint{3.550847in}{0.978637in}}{\pgfqpoint{3.543033in}{0.970824in}}%
\pgfpathcurveto{\pgfqpoint{3.535220in}{0.963010in}}{\pgfqpoint{3.530830in}{0.952411in}}{\pgfqpoint{3.530830in}{0.941361in}}%
\pgfpathcurveto{\pgfqpoint{3.530830in}{0.930311in}}{\pgfqpoint{3.535220in}{0.919712in}}{\pgfqpoint{3.543033in}{0.911898in}}%
\pgfpathcurveto{\pgfqpoint{3.550847in}{0.904085in}}{\pgfqpoint{3.561446in}{0.899694in}}{\pgfqpoint{3.572496in}{0.899694in}}%
\pgfpathlineto{\pgfqpoint{3.572496in}{0.899694in}}%
\pgfpathclose%
\pgfusepath{stroke}%
\end{pgfscope}%
\begin{pgfscope}%
\pgfpathrectangle{\pgfqpoint{0.847223in}{0.554012in}}{\pgfqpoint{6.200000in}{4.620000in}}%
\pgfusepath{clip}%
\pgfsetbuttcap%
\pgfsetroundjoin%
\pgfsetlinewidth{1.003750pt}%
\definecolor{currentstroke}{rgb}{1.000000,0.000000,0.000000}%
\pgfsetstrokecolor{currentstroke}%
\pgfsetdash{}{0pt}%
\pgfpathmoveto{\pgfqpoint{3.577829in}{0.898362in}}%
\pgfpathcurveto{\pgfqpoint{3.588880in}{0.898362in}}{\pgfqpoint{3.599479in}{0.902752in}}{\pgfqpoint{3.607292in}{0.910566in}}%
\pgfpathcurveto{\pgfqpoint{3.615106in}{0.918380in}}{\pgfqpoint{3.619496in}{0.928979in}}{\pgfqpoint{3.619496in}{0.940029in}}%
\pgfpathcurveto{\pgfqpoint{3.619496in}{0.951079in}}{\pgfqpoint{3.615106in}{0.961678in}}{\pgfqpoint{3.607292in}{0.969492in}}%
\pgfpathcurveto{\pgfqpoint{3.599479in}{0.977305in}}{\pgfqpoint{3.588880in}{0.981695in}}{\pgfqpoint{3.577829in}{0.981695in}}%
\pgfpathcurveto{\pgfqpoint{3.566779in}{0.981695in}}{\pgfqpoint{3.556180in}{0.977305in}}{\pgfqpoint{3.548367in}{0.969492in}}%
\pgfpathcurveto{\pgfqpoint{3.540553in}{0.961678in}}{\pgfqpoint{3.536163in}{0.951079in}}{\pgfqpoint{3.536163in}{0.940029in}}%
\pgfpathcurveto{\pgfqpoint{3.536163in}{0.928979in}}{\pgfqpoint{3.540553in}{0.918380in}}{\pgfqpoint{3.548367in}{0.910566in}}%
\pgfpathcurveto{\pgfqpoint{3.556180in}{0.902752in}}{\pgfqpoint{3.566779in}{0.898362in}}{\pgfqpoint{3.577829in}{0.898362in}}%
\pgfpathlineto{\pgfqpoint{3.577829in}{0.898362in}}%
\pgfpathclose%
\pgfusepath{stroke}%
\end{pgfscope}%
\begin{pgfscope}%
\pgfpathrectangle{\pgfqpoint{0.847223in}{0.554012in}}{\pgfqpoint{6.200000in}{4.620000in}}%
\pgfusepath{clip}%
\pgfsetbuttcap%
\pgfsetroundjoin%
\pgfsetlinewidth{1.003750pt}%
\definecolor{currentstroke}{rgb}{1.000000,0.000000,0.000000}%
\pgfsetstrokecolor{currentstroke}%
\pgfsetdash{}{0pt}%
\pgfpathmoveto{\pgfqpoint{3.583163in}{0.897034in}}%
\pgfpathcurveto{\pgfqpoint{3.594213in}{0.897034in}}{\pgfqpoint{3.604812in}{0.901425in}}{\pgfqpoint{3.612625in}{0.909238in}}%
\pgfpathcurveto{\pgfqpoint{3.620439in}{0.917052in}}{\pgfqpoint{3.624829in}{0.927651in}}{\pgfqpoint{3.624829in}{0.938701in}}%
\pgfpathcurveto{\pgfqpoint{3.624829in}{0.949751in}}{\pgfqpoint{3.620439in}{0.960350in}}{\pgfqpoint{3.612625in}{0.968164in}}%
\pgfpathcurveto{\pgfqpoint{3.604812in}{0.975977in}}{\pgfqpoint{3.594213in}{0.980368in}}{\pgfqpoint{3.583163in}{0.980368in}}%
\pgfpathcurveto{\pgfqpoint{3.572113in}{0.980368in}}{\pgfqpoint{3.561514in}{0.975977in}}{\pgfqpoint{3.553700in}{0.968164in}}%
\pgfpathcurveto{\pgfqpoint{3.545886in}{0.960350in}}{\pgfqpoint{3.541496in}{0.949751in}}{\pgfqpoint{3.541496in}{0.938701in}}%
\pgfpathcurveto{\pgfqpoint{3.541496in}{0.927651in}}{\pgfqpoint{3.545886in}{0.917052in}}{\pgfqpoint{3.553700in}{0.909238in}}%
\pgfpathcurveto{\pgfqpoint{3.561514in}{0.901425in}}{\pgfqpoint{3.572113in}{0.897034in}}{\pgfqpoint{3.583163in}{0.897034in}}%
\pgfpathlineto{\pgfqpoint{3.583163in}{0.897034in}}%
\pgfpathclose%
\pgfusepath{stroke}%
\end{pgfscope}%
\begin{pgfscope}%
\pgfpathrectangle{\pgfqpoint{0.847223in}{0.554012in}}{\pgfqpoint{6.200000in}{4.620000in}}%
\pgfusepath{clip}%
\pgfsetbuttcap%
\pgfsetroundjoin%
\pgfsetlinewidth{1.003750pt}%
\definecolor{currentstroke}{rgb}{1.000000,0.000000,0.000000}%
\pgfsetstrokecolor{currentstroke}%
\pgfsetdash{}{0pt}%
\pgfpathmoveto{\pgfqpoint{3.588496in}{0.895711in}}%
\pgfpathcurveto{\pgfqpoint{3.599546in}{0.895711in}}{\pgfqpoint{3.610145in}{0.900101in}}{\pgfqpoint{3.617959in}{0.907915in}}%
\pgfpathcurveto{\pgfqpoint{3.625772in}{0.915728in}}{\pgfqpoint{3.630163in}{0.926327in}}{\pgfqpoint{3.630163in}{0.937377in}}%
\pgfpathcurveto{\pgfqpoint{3.630163in}{0.948428in}}{\pgfqpoint{3.625772in}{0.959027in}}{\pgfqpoint{3.617959in}{0.966840in}}%
\pgfpathcurveto{\pgfqpoint{3.610145in}{0.974654in}}{\pgfqpoint{3.599546in}{0.979044in}}{\pgfqpoint{3.588496in}{0.979044in}}%
\pgfpathcurveto{\pgfqpoint{3.577446in}{0.979044in}}{\pgfqpoint{3.566847in}{0.974654in}}{\pgfqpoint{3.559033in}{0.966840in}}%
\pgfpathcurveto{\pgfqpoint{3.551220in}{0.959027in}}{\pgfqpoint{3.546829in}{0.948428in}}{\pgfqpoint{3.546829in}{0.937377in}}%
\pgfpathcurveto{\pgfqpoint{3.546829in}{0.926327in}}{\pgfqpoint{3.551220in}{0.915728in}}{\pgfqpoint{3.559033in}{0.907915in}}%
\pgfpathcurveto{\pgfqpoint{3.566847in}{0.900101in}}{\pgfqpoint{3.577446in}{0.895711in}}{\pgfqpoint{3.588496in}{0.895711in}}%
\pgfpathlineto{\pgfqpoint{3.588496in}{0.895711in}}%
\pgfpathclose%
\pgfusepath{stroke}%
\end{pgfscope}%
\begin{pgfscope}%
\pgfpathrectangle{\pgfqpoint{0.847223in}{0.554012in}}{\pgfqpoint{6.200000in}{4.620000in}}%
\pgfusepath{clip}%
\pgfsetbuttcap%
\pgfsetroundjoin%
\pgfsetlinewidth{1.003750pt}%
\definecolor{currentstroke}{rgb}{1.000000,0.000000,0.000000}%
\pgfsetstrokecolor{currentstroke}%
\pgfsetdash{}{0pt}%
\pgfpathmoveto{\pgfqpoint{3.593829in}{0.894392in}}%
\pgfpathcurveto{\pgfqpoint{3.604879in}{0.894392in}}{\pgfqpoint{3.615478in}{0.898782in}}{\pgfqpoint{3.623292in}{0.906595in}}%
\pgfpathcurveto{\pgfqpoint{3.631106in}{0.914409in}}{\pgfqpoint{3.635496in}{0.925008in}}{\pgfqpoint{3.635496in}{0.936058in}}%
\pgfpathcurveto{\pgfqpoint{3.635496in}{0.947108in}}{\pgfqpoint{3.631106in}{0.957707in}}{\pgfqpoint{3.623292in}{0.965521in}}%
\pgfpathcurveto{\pgfqpoint{3.615478in}{0.973335in}}{\pgfqpoint{3.604879in}{0.977725in}}{\pgfqpoint{3.593829in}{0.977725in}}%
\pgfpathcurveto{\pgfqpoint{3.582779in}{0.977725in}}{\pgfqpoint{3.572180in}{0.973335in}}{\pgfqpoint{3.564366in}{0.965521in}}%
\pgfpathcurveto{\pgfqpoint{3.556553in}{0.957707in}}{\pgfqpoint{3.552162in}{0.947108in}}{\pgfqpoint{3.552162in}{0.936058in}}%
\pgfpathcurveto{\pgfqpoint{3.552162in}{0.925008in}}{\pgfqpoint{3.556553in}{0.914409in}}{\pgfqpoint{3.564366in}{0.906595in}}%
\pgfpathcurveto{\pgfqpoint{3.572180in}{0.898782in}}{\pgfqpoint{3.582779in}{0.894392in}}{\pgfqpoint{3.593829in}{0.894392in}}%
\pgfpathlineto{\pgfqpoint{3.593829in}{0.894392in}}%
\pgfpathclose%
\pgfusepath{stroke}%
\end{pgfscope}%
\begin{pgfscope}%
\pgfpathrectangle{\pgfqpoint{0.847223in}{0.554012in}}{\pgfqpoint{6.200000in}{4.620000in}}%
\pgfusepath{clip}%
\pgfsetbuttcap%
\pgfsetroundjoin%
\pgfsetlinewidth{1.003750pt}%
\definecolor{currentstroke}{rgb}{1.000000,0.000000,0.000000}%
\pgfsetstrokecolor{currentstroke}%
\pgfsetdash{}{0pt}%
\pgfpathmoveto{\pgfqpoint{3.599162in}{0.893077in}}%
\pgfpathcurveto{\pgfqpoint{3.610212in}{0.893077in}}{\pgfqpoint{3.620811in}{0.897467in}}{\pgfqpoint{3.628625in}{0.905281in}}%
\pgfpathcurveto{\pgfqpoint{3.636439in}{0.913094in}}{\pgfqpoint{3.640829in}{0.923693in}}{\pgfqpoint{3.640829in}{0.934743in}}%
\pgfpathcurveto{\pgfqpoint{3.640829in}{0.945793in}}{\pgfqpoint{3.636439in}{0.956392in}}{\pgfqpoint{3.628625in}{0.964206in}}%
\pgfpathcurveto{\pgfqpoint{3.620811in}{0.972020in}}{\pgfqpoint{3.610212in}{0.976410in}}{\pgfqpoint{3.599162in}{0.976410in}}%
\pgfpathcurveto{\pgfqpoint{3.588112in}{0.976410in}}{\pgfqpoint{3.577513in}{0.972020in}}{\pgfqpoint{3.569700in}{0.964206in}}%
\pgfpathcurveto{\pgfqpoint{3.561886in}{0.956392in}}{\pgfqpoint{3.557496in}{0.945793in}}{\pgfqpoint{3.557496in}{0.934743in}}%
\pgfpathcurveto{\pgfqpoint{3.557496in}{0.923693in}}{\pgfqpoint{3.561886in}{0.913094in}}{\pgfqpoint{3.569700in}{0.905281in}}%
\pgfpathcurveto{\pgfqpoint{3.577513in}{0.897467in}}{\pgfqpoint{3.588112in}{0.893077in}}{\pgfqpoint{3.599162in}{0.893077in}}%
\pgfpathlineto{\pgfqpoint{3.599162in}{0.893077in}}%
\pgfpathclose%
\pgfusepath{stroke}%
\end{pgfscope}%
\begin{pgfscope}%
\pgfpathrectangle{\pgfqpoint{0.847223in}{0.554012in}}{\pgfqpoint{6.200000in}{4.620000in}}%
\pgfusepath{clip}%
\pgfsetbuttcap%
\pgfsetroundjoin%
\pgfsetlinewidth{1.003750pt}%
\definecolor{currentstroke}{rgb}{1.000000,0.000000,0.000000}%
\pgfsetstrokecolor{currentstroke}%
\pgfsetdash{}{0pt}%
\pgfpathmoveto{\pgfqpoint{3.604496in}{0.891766in}}%
\pgfpathcurveto{\pgfqpoint{3.615546in}{0.891766in}}{\pgfqpoint{3.626145in}{0.896156in}}{\pgfqpoint{3.633958in}{0.903970in}}%
\pgfpathcurveto{\pgfqpoint{3.641772in}{0.911784in}}{\pgfqpoint{3.646162in}{0.922383in}}{\pgfqpoint{3.646162in}{0.933433in}}%
\pgfpathcurveto{\pgfqpoint{3.646162in}{0.944483in}}{\pgfqpoint{3.641772in}{0.955082in}}{\pgfqpoint{3.633958in}{0.962895in}}%
\pgfpathcurveto{\pgfqpoint{3.626145in}{0.970709in}}{\pgfqpoint{3.615546in}{0.975099in}}{\pgfqpoint{3.604496in}{0.975099in}}%
\pgfpathcurveto{\pgfqpoint{3.593445in}{0.975099in}}{\pgfqpoint{3.582846in}{0.970709in}}{\pgfqpoint{3.575033in}{0.962895in}}%
\pgfpathcurveto{\pgfqpoint{3.567219in}{0.955082in}}{\pgfqpoint{3.562829in}{0.944483in}}{\pgfqpoint{3.562829in}{0.933433in}}%
\pgfpathcurveto{\pgfqpoint{3.562829in}{0.922383in}}{\pgfqpoint{3.567219in}{0.911784in}}{\pgfqpoint{3.575033in}{0.903970in}}%
\pgfpathcurveto{\pgfqpoint{3.582846in}{0.896156in}}{\pgfqpoint{3.593445in}{0.891766in}}{\pgfqpoint{3.604496in}{0.891766in}}%
\pgfpathlineto{\pgfqpoint{3.604496in}{0.891766in}}%
\pgfpathclose%
\pgfusepath{stroke}%
\end{pgfscope}%
\begin{pgfscope}%
\pgfpathrectangle{\pgfqpoint{0.847223in}{0.554012in}}{\pgfqpoint{6.200000in}{4.620000in}}%
\pgfusepath{clip}%
\pgfsetbuttcap%
\pgfsetroundjoin%
\pgfsetlinewidth{1.003750pt}%
\definecolor{currentstroke}{rgb}{1.000000,0.000000,0.000000}%
\pgfsetstrokecolor{currentstroke}%
\pgfsetdash{}{0pt}%
\pgfpathmoveto{\pgfqpoint{3.609829in}{0.890460in}}%
\pgfpathcurveto{\pgfqpoint{3.620879in}{0.890460in}}{\pgfqpoint{3.631478in}{0.894850in}}{\pgfqpoint{3.639292in}{0.902664in}}%
\pgfpathcurveto{\pgfqpoint{3.647105in}{0.910477in}}{\pgfqpoint{3.651495in}{0.921076in}}{\pgfqpoint{3.651495in}{0.932126in}}%
\pgfpathcurveto{\pgfqpoint{3.651495in}{0.943176in}}{\pgfqpoint{3.647105in}{0.953776in}}{\pgfqpoint{3.639292in}{0.961589in}}%
\pgfpathcurveto{\pgfqpoint{3.631478in}{0.969403in}}{\pgfqpoint{3.620879in}{0.973793in}}{\pgfqpoint{3.609829in}{0.973793in}}%
\pgfpathcurveto{\pgfqpoint{3.598779in}{0.973793in}}{\pgfqpoint{3.588180in}{0.969403in}}{\pgfqpoint{3.580366in}{0.961589in}}%
\pgfpathcurveto{\pgfqpoint{3.572552in}{0.953776in}}{\pgfqpoint{3.568162in}{0.943176in}}{\pgfqpoint{3.568162in}{0.932126in}}%
\pgfpathcurveto{\pgfqpoint{3.568162in}{0.921076in}}{\pgfqpoint{3.572552in}{0.910477in}}{\pgfqpoint{3.580366in}{0.902664in}}%
\pgfpathcurveto{\pgfqpoint{3.588180in}{0.894850in}}{\pgfqpoint{3.598779in}{0.890460in}}{\pgfqpoint{3.609829in}{0.890460in}}%
\pgfpathlineto{\pgfqpoint{3.609829in}{0.890460in}}%
\pgfpathclose%
\pgfusepath{stroke}%
\end{pgfscope}%
\begin{pgfscope}%
\pgfpathrectangle{\pgfqpoint{0.847223in}{0.554012in}}{\pgfqpoint{6.200000in}{4.620000in}}%
\pgfusepath{clip}%
\pgfsetbuttcap%
\pgfsetroundjoin%
\pgfsetlinewidth{1.003750pt}%
\definecolor{currentstroke}{rgb}{1.000000,0.000000,0.000000}%
\pgfsetstrokecolor{currentstroke}%
\pgfsetdash{}{0pt}%
\pgfpathmoveto{\pgfqpoint{3.615162in}{0.889158in}}%
\pgfpathcurveto{\pgfqpoint{3.626212in}{0.889158in}}{\pgfqpoint{3.636811in}{0.893548in}}{\pgfqpoint{3.644625in}{0.901361in}}%
\pgfpathcurveto{\pgfqpoint{3.652438in}{0.909175in}}{\pgfqpoint{3.656829in}{0.919774in}}{\pgfqpoint{3.656829in}{0.930824in}}%
\pgfpathcurveto{\pgfqpoint{3.656829in}{0.941874in}}{\pgfqpoint{3.652438in}{0.952473in}}{\pgfqpoint{3.644625in}{0.960287in}}%
\pgfpathcurveto{\pgfqpoint{3.636811in}{0.968101in}}{\pgfqpoint{3.626212in}{0.972491in}}{\pgfqpoint{3.615162in}{0.972491in}}%
\pgfpathcurveto{\pgfqpoint{3.604112in}{0.972491in}}{\pgfqpoint{3.593513in}{0.968101in}}{\pgfqpoint{3.585699in}{0.960287in}}%
\pgfpathcurveto{\pgfqpoint{3.577886in}{0.952473in}}{\pgfqpoint{3.573495in}{0.941874in}}{\pgfqpoint{3.573495in}{0.930824in}}%
\pgfpathcurveto{\pgfqpoint{3.573495in}{0.919774in}}{\pgfqpoint{3.577886in}{0.909175in}}{\pgfqpoint{3.585699in}{0.901361in}}%
\pgfpathcurveto{\pgfqpoint{3.593513in}{0.893548in}}{\pgfqpoint{3.604112in}{0.889158in}}{\pgfqpoint{3.615162in}{0.889158in}}%
\pgfpathlineto{\pgfqpoint{3.615162in}{0.889158in}}%
\pgfpathclose%
\pgfusepath{stroke}%
\end{pgfscope}%
\begin{pgfscope}%
\pgfpathrectangle{\pgfqpoint{0.847223in}{0.554012in}}{\pgfqpoint{6.200000in}{4.620000in}}%
\pgfusepath{clip}%
\pgfsetbuttcap%
\pgfsetroundjoin%
\pgfsetlinewidth{1.003750pt}%
\definecolor{currentstroke}{rgb}{1.000000,0.000000,0.000000}%
\pgfsetstrokecolor{currentstroke}%
\pgfsetdash{}{0pt}%
\pgfpathmoveto{\pgfqpoint{3.620495in}{0.887860in}}%
\pgfpathcurveto{\pgfqpoint{3.631545in}{0.887860in}}{\pgfqpoint{3.642144in}{0.892250in}}{\pgfqpoint{3.649958in}{0.900064in}}%
\pgfpathcurveto{\pgfqpoint{3.657772in}{0.907877in}}{\pgfqpoint{3.662162in}{0.918476in}}{\pgfqpoint{3.662162in}{0.929526in}}%
\pgfpathcurveto{\pgfqpoint{3.662162in}{0.940576in}}{\pgfqpoint{3.657772in}{0.951175in}}{\pgfqpoint{3.649958in}{0.958989in}}%
\pgfpathcurveto{\pgfqpoint{3.642144in}{0.966803in}}{\pgfqpoint{3.631545in}{0.971193in}}{\pgfqpoint{3.620495in}{0.971193in}}%
\pgfpathcurveto{\pgfqpoint{3.609445in}{0.971193in}}{\pgfqpoint{3.598846in}{0.966803in}}{\pgfqpoint{3.591032in}{0.958989in}}%
\pgfpathcurveto{\pgfqpoint{3.583219in}{0.951175in}}{\pgfqpoint{3.578829in}{0.940576in}}{\pgfqpoint{3.578829in}{0.929526in}}%
\pgfpathcurveto{\pgfqpoint{3.578829in}{0.918476in}}{\pgfqpoint{3.583219in}{0.907877in}}{\pgfqpoint{3.591032in}{0.900064in}}%
\pgfpathcurveto{\pgfqpoint{3.598846in}{0.892250in}}{\pgfqpoint{3.609445in}{0.887860in}}{\pgfqpoint{3.620495in}{0.887860in}}%
\pgfpathlineto{\pgfqpoint{3.620495in}{0.887860in}}%
\pgfpathclose%
\pgfusepath{stroke}%
\end{pgfscope}%
\begin{pgfscope}%
\pgfpathrectangle{\pgfqpoint{0.847223in}{0.554012in}}{\pgfqpoint{6.200000in}{4.620000in}}%
\pgfusepath{clip}%
\pgfsetbuttcap%
\pgfsetroundjoin%
\pgfsetlinewidth{1.003750pt}%
\definecolor{currentstroke}{rgb}{1.000000,0.000000,0.000000}%
\pgfsetstrokecolor{currentstroke}%
\pgfsetdash{}{0pt}%
\pgfpathmoveto{\pgfqpoint{3.625828in}{0.886566in}}%
\pgfpathcurveto{\pgfqpoint{3.636879in}{0.886566in}}{\pgfqpoint{3.647478in}{0.890956in}}{\pgfqpoint{3.655291in}{0.898770in}}%
\pgfpathcurveto{\pgfqpoint{3.663105in}{0.906583in}}{\pgfqpoint{3.667495in}{0.917182in}}{\pgfqpoint{3.667495in}{0.928233in}}%
\pgfpathcurveto{\pgfqpoint{3.667495in}{0.939283in}}{\pgfqpoint{3.663105in}{0.949882in}}{\pgfqpoint{3.655291in}{0.957695in}}%
\pgfpathcurveto{\pgfqpoint{3.647478in}{0.965509in}}{\pgfqpoint{3.636879in}{0.969899in}}{\pgfqpoint{3.625828in}{0.969899in}}%
\pgfpathcurveto{\pgfqpoint{3.614778in}{0.969899in}}{\pgfqpoint{3.604179in}{0.965509in}}{\pgfqpoint{3.596366in}{0.957695in}}%
\pgfpathcurveto{\pgfqpoint{3.588552in}{0.949882in}}{\pgfqpoint{3.584162in}{0.939283in}}{\pgfqpoint{3.584162in}{0.928233in}}%
\pgfpathcurveto{\pgfqpoint{3.584162in}{0.917182in}}{\pgfqpoint{3.588552in}{0.906583in}}{\pgfqpoint{3.596366in}{0.898770in}}%
\pgfpathcurveto{\pgfqpoint{3.604179in}{0.890956in}}{\pgfqpoint{3.614778in}{0.886566in}}{\pgfqpoint{3.625828in}{0.886566in}}%
\pgfpathlineto{\pgfqpoint{3.625828in}{0.886566in}}%
\pgfpathclose%
\pgfusepath{stroke}%
\end{pgfscope}%
\begin{pgfscope}%
\pgfpathrectangle{\pgfqpoint{0.847223in}{0.554012in}}{\pgfqpoint{6.200000in}{4.620000in}}%
\pgfusepath{clip}%
\pgfsetbuttcap%
\pgfsetroundjoin%
\pgfsetlinewidth{1.003750pt}%
\definecolor{currentstroke}{rgb}{1.000000,0.000000,0.000000}%
\pgfsetstrokecolor{currentstroke}%
\pgfsetdash{}{0pt}%
\pgfpathmoveto{\pgfqpoint{3.631162in}{0.885276in}}%
\pgfpathcurveto{\pgfqpoint{3.642212in}{0.885276in}}{\pgfqpoint{3.652811in}{0.889667in}}{\pgfqpoint{3.660624in}{0.897480in}}%
\pgfpathcurveto{\pgfqpoint{3.668438in}{0.905294in}}{\pgfqpoint{3.672828in}{0.915893in}}{\pgfqpoint{3.672828in}{0.926943in}}%
\pgfpathcurveto{\pgfqpoint{3.672828in}{0.937993in}}{\pgfqpoint{3.668438in}{0.948592in}}{\pgfqpoint{3.660624in}{0.956406in}}%
\pgfpathcurveto{\pgfqpoint{3.652811in}{0.964219in}}{\pgfqpoint{3.642212in}{0.968610in}}{\pgfqpoint{3.631162in}{0.968610in}}%
\pgfpathcurveto{\pgfqpoint{3.620112in}{0.968610in}}{\pgfqpoint{3.609512in}{0.964219in}}{\pgfqpoint{3.601699in}{0.956406in}}%
\pgfpathcurveto{\pgfqpoint{3.593885in}{0.948592in}}{\pgfqpoint{3.589495in}{0.937993in}}{\pgfqpoint{3.589495in}{0.926943in}}%
\pgfpathcurveto{\pgfqpoint{3.589495in}{0.915893in}}{\pgfqpoint{3.593885in}{0.905294in}}{\pgfqpoint{3.601699in}{0.897480in}}%
\pgfpathcurveto{\pgfqpoint{3.609512in}{0.889667in}}{\pgfqpoint{3.620112in}{0.885276in}}{\pgfqpoint{3.631162in}{0.885276in}}%
\pgfpathlineto{\pgfqpoint{3.631162in}{0.885276in}}%
\pgfpathclose%
\pgfusepath{stroke}%
\end{pgfscope}%
\begin{pgfscope}%
\pgfpathrectangle{\pgfqpoint{0.847223in}{0.554012in}}{\pgfqpoint{6.200000in}{4.620000in}}%
\pgfusepath{clip}%
\pgfsetbuttcap%
\pgfsetroundjoin%
\pgfsetlinewidth{1.003750pt}%
\definecolor{currentstroke}{rgb}{1.000000,0.000000,0.000000}%
\pgfsetstrokecolor{currentstroke}%
\pgfsetdash{}{0pt}%
\pgfpathmoveto{\pgfqpoint{3.636495in}{0.883991in}}%
\pgfpathcurveto{\pgfqpoint{3.647545in}{0.883991in}}{\pgfqpoint{3.658144in}{0.888381in}}{\pgfqpoint{3.665958in}{0.896195in}}%
\pgfpathcurveto{\pgfqpoint{3.673771in}{0.904008in}}{\pgfqpoint{3.678162in}{0.914607in}}{\pgfqpoint{3.678162in}{0.925658in}}%
\pgfpathcurveto{\pgfqpoint{3.678162in}{0.936708in}}{\pgfqpoint{3.673771in}{0.947307in}}{\pgfqpoint{3.665958in}{0.955120in}}%
\pgfpathcurveto{\pgfqpoint{3.658144in}{0.962934in}}{\pgfqpoint{3.647545in}{0.967324in}}{\pgfqpoint{3.636495in}{0.967324in}}%
\pgfpathcurveto{\pgfqpoint{3.625445in}{0.967324in}}{\pgfqpoint{3.614846in}{0.962934in}}{\pgfqpoint{3.607032in}{0.955120in}}%
\pgfpathcurveto{\pgfqpoint{3.599218in}{0.947307in}}{\pgfqpoint{3.594828in}{0.936708in}}{\pgfqpoint{3.594828in}{0.925658in}}%
\pgfpathcurveto{\pgfqpoint{3.594828in}{0.914607in}}{\pgfqpoint{3.599218in}{0.904008in}}{\pgfqpoint{3.607032in}{0.896195in}}%
\pgfpathcurveto{\pgfqpoint{3.614846in}{0.888381in}}{\pgfqpoint{3.625445in}{0.883991in}}{\pgfqpoint{3.636495in}{0.883991in}}%
\pgfpathlineto{\pgfqpoint{3.636495in}{0.883991in}}%
\pgfpathclose%
\pgfusepath{stroke}%
\end{pgfscope}%
\begin{pgfscope}%
\pgfpathrectangle{\pgfqpoint{0.847223in}{0.554012in}}{\pgfqpoint{6.200000in}{4.620000in}}%
\pgfusepath{clip}%
\pgfsetbuttcap%
\pgfsetroundjoin%
\pgfsetlinewidth{1.003750pt}%
\definecolor{currentstroke}{rgb}{1.000000,0.000000,0.000000}%
\pgfsetstrokecolor{currentstroke}%
\pgfsetdash{}{0pt}%
\pgfpathmoveto{\pgfqpoint{3.641828in}{0.882710in}}%
\pgfpathcurveto{\pgfqpoint{3.652878in}{0.882710in}}{\pgfqpoint{3.663477in}{0.887100in}}{\pgfqpoint{3.671291in}{0.894914in}}%
\pgfpathcurveto{\pgfqpoint{3.679104in}{0.902727in}}{\pgfqpoint{3.683495in}{0.913326in}}{\pgfqpoint{3.683495in}{0.924376in}}%
\pgfpathcurveto{\pgfqpoint{3.683495in}{0.935426in}}{\pgfqpoint{3.679104in}{0.946025in}}{\pgfqpoint{3.671291in}{0.953839in}}%
\pgfpathcurveto{\pgfqpoint{3.663477in}{0.961653in}}{\pgfqpoint{3.652878in}{0.966043in}}{\pgfqpoint{3.641828in}{0.966043in}}%
\pgfpathcurveto{\pgfqpoint{3.630778in}{0.966043in}}{\pgfqpoint{3.620179in}{0.961653in}}{\pgfqpoint{3.612365in}{0.953839in}}%
\pgfpathcurveto{\pgfqpoint{3.604552in}{0.946025in}}{\pgfqpoint{3.600161in}{0.935426in}}{\pgfqpoint{3.600161in}{0.924376in}}%
\pgfpathcurveto{\pgfqpoint{3.600161in}{0.913326in}}{\pgfqpoint{3.604552in}{0.902727in}}{\pgfqpoint{3.612365in}{0.894914in}}%
\pgfpathcurveto{\pgfqpoint{3.620179in}{0.887100in}}{\pgfqpoint{3.630778in}{0.882710in}}{\pgfqpoint{3.641828in}{0.882710in}}%
\pgfpathlineto{\pgfqpoint{3.641828in}{0.882710in}}%
\pgfpathclose%
\pgfusepath{stroke}%
\end{pgfscope}%
\begin{pgfscope}%
\pgfpathrectangle{\pgfqpoint{0.847223in}{0.554012in}}{\pgfqpoint{6.200000in}{4.620000in}}%
\pgfusepath{clip}%
\pgfsetbuttcap%
\pgfsetroundjoin%
\pgfsetlinewidth{1.003750pt}%
\definecolor{currentstroke}{rgb}{1.000000,0.000000,0.000000}%
\pgfsetstrokecolor{currentstroke}%
\pgfsetdash{}{0pt}%
\pgfpathmoveto{\pgfqpoint{3.647161in}{0.881433in}}%
\pgfpathcurveto{\pgfqpoint{3.658211in}{0.881433in}}{\pgfqpoint{3.668810in}{0.885823in}}{\pgfqpoint{3.676624in}{0.893636in}}%
\pgfpathcurveto{\pgfqpoint{3.684438in}{0.901450in}}{\pgfqpoint{3.688828in}{0.912049in}}{\pgfqpoint{3.688828in}{0.923099in}}%
\pgfpathcurveto{\pgfqpoint{3.688828in}{0.934149in}}{\pgfqpoint{3.684438in}{0.944748in}}{\pgfqpoint{3.676624in}{0.952562in}}%
\pgfpathcurveto{\pgfqpoint{3.668810in}{0.960376in}}{\pgfqpoint{3.658211in}{0.964766in}}{\pgfqpoint{3.647161in}{0.964766in}}%
\pgfpathcurveto{\pgfqpoint{3.636111in}{0.964766in}}{\pgfqpoint{3.625512in}{0.960376in}}{\pgfqpoint{3.617698in}{0.952562in}}%
\pgfpathcurveto{\pgfqpoint{3.609885in}{0.944748in}}{\pgfqpoint{3.605495in}{0.934149in}}{\pgfqpoint{3.605495in}{0.923099in}}%
\pgfpathcurveto{\pgfqpoint{3.605495in}{0.912049in}}{\pgfqpoint{3.609885in}{0.901450in}}{\pgfqpoint{3.617698in}{0.893636in}}%
\pgfpathcurveto{\pgfqpoint{3.625512in}{0.885823in}}{\pgfqpoint{3.636111in}{0.881433in}}{\pgfqpoint{3.647161in}{0.881433in}}%
\pgfpathlineto{\pgfqpoint{3.647161in}{0.881433in}}%
\pgfpathclose%
\pgfusepath{stroke}%
\end{pgfscope}%
\begin{pgfscope}%
\pgfpathrectangle{\pgfqpoint{0.847223in}{0.554012in}}{\pgfqpoint{6.200000in}{4.620000in}}%
\pgfusepath{clip}%
\pgfsetbuttcap%
\pgfsetroundjoin%
\pgfsetlinewidth{1.003750pt}%
\definecolor{currentstroke}{rgb}{1.000000,0.000000,0.000000}%
\pgfsetstrokecolor{currentstroke}%
\pgfsetdash{}{0pt}%
\pgfpathmoveto{\pgfqpoint{3.652494in}{0.880159in}}%
\pgfpathcurveto{\pgfqpoint{3.663545in}{0.880159in}}{\pgfqpoint{3.674144in}{0.884550in}}{\pgfqpoint{3.681957in}{0.892363in}}%
\pgfpathcurveto{\pgfqpoint{3.689771in}{0.900177in}}{\pgfqpoint{3.694161in}{0.910776in}}{\pgfqpoint{3.694161in}{0.921826in}}%
\pgfpathcurveto{\pgfqpoint{3.694161in}{0.932876in}}{\pgfqpoint{3.689771in}{0.943475in}}{\pgfqpoint{3.681957in}{0.951289in}}%
\pgfpathcurveto{\pgfqpoint{3.674144in}{0.959103in}}{\pgfqpoint{3.663545in}{0.963493in}}{\pgfqpoint{3.652494in}{0.963493in}}%
\pgfpathcurveto{\pgfqpoint{3.641444in}{0.963493in}}{\pgfqpoint{3.630845in}{0.959103in}}{\pgfqpoint{3.623032in}{0.951289in}}%
\pgfpathcurveto{\pgfqpoint{3.615218in}{0.943475in}}{\pgfqpoint{3.610828in}{0.932876in}}{\pgfqpoint{3.610828in}{0.921826in}}%
\pgfpathcurveto{\pgfqpoint{3.610828in}{0.910776in}}{\pgfqpoint{3.615218in}{0.900177in}}{\pgfqpoint{3.623032in}{0.892363in}}%
\pgfpathcurveto{\pgfqpoint{3.630845in}{0.884550in}}{\pgfqpoint{3.641444in}{0.880159in}}{\pgfqpoint{3.652494in}{0.880159in}}%
\pgfpathlineto{\pgfqpoint{3.652494in}{0.880159in}}%
\pgfpathclose%
\pgfusepath{stroke}%
\end{pgfscope}%
\begin{pgfscope}%
\pgfpathrectangle{\pgfqpoint{0.847223in}{0.554012in}}{\pgfqpoint{6.200000in}{4.620000in}}%
\pgfusepath{clip}%
\pgfsetbuttcap%
\pgfsetroundjoin%
\pgfsetlinewidth{1.003750pt}%
\definecolor{currentstroke}{rgb}{1.000000,0.000000,0.000000}%
\pgfsetstrokecolor{currentstroke}%
\pgfsetdash{}{0pt}%
\pgfpathmoveto{\pgfqpoint{3.657828in}{0.878890in}}%
\pgfpathcurveto{\pgfqpoint{3.668878in}{0.878890in}}{\pgfqpoint{3.679477in}{0.883281in}}{\pgfqpoint{3.687290in}{0.891094in}}%
\pgfpathcurveto{\pgfqpoint{3.695104in}{0.898908in}}{\pgfqpoint{3.699494in}{0.909507in}}{\pgfqpoint{3.699494in}{0.920557in}}%
\pgfpathcurveto{\pgfqpoint{3.699494in}{0.931607in}}{\pgfqpoint{3.695104in}{0.942206in}}{\pgfqpoint{3.687290in}{0.950020in}}%
\pgfpathcurveto{\pgfqpoint{3.679477in}{0.957834in}}{\pgfqpoint{3.668878in}{0.962224in}}{\pgfqpoint{3.657828in}{0.962224in}}%
\pgfpathcurveto{\pgfqpoint{3.646778in}{0.962224in}}{\pgfqpoint{3.636179in}{0.957834in}}{\pgfqpoint{3.628365in}{0.950020in}}%
\pgfpathcurveto{\pgfqpoint{3.620551in}{0.942206in}}{\pgfqpoint{3.616161in}{0.931607in}}{\pgfqpoint{3.616161in}{0.920557in}}%
\pgfpathcurveto{\pgfqpoint{3.616161in}{0.909507in}}{\pgfqpoint{3.620551in}{0.898908in}}{\pgfqpoint{3.628365in}{0.891094in}}%
\pgfpathcurveto{\pgfqpoint{3.636179in}{0.883281in}}{\pgfqpoint{3.646778in}{0.878890in}}{\pgfqpoint{3.657828in}{0.878890in}}%
\pgfpathlineto{\pgfqpoint{3.657828in}{0.878890in}}%
\pgfpathclose%
\pgfusepath{stroke}%
\end{pgfscope}%
\begin{pgfscope}%
\pgfpathrectangle{\pgfqpoint{0.847223in}{0.554012in}}{\pgfqpoint{6.200000in}{4.620000in}}%
\pgfusepath{clip}%
\pgfsetbuttcap%
\pgfsetroundjoin%
\pgfsetlinewidth{1.003750pt}%
\definecolor{currentstroke}{rgb}{1.000000,0.000000,0.000000}%
\pgfsetstrokecolor{currentstroke}%
\pgfsetdash{}{0pt}%
\pgfpathmoveto{\pgfqpoint{3.663161in}{0.877626in}}%
\pgfpathcurveto{\pgfqpoint{3.674211in}{0.877626in}}{\pgfqpoint{3.684810in}{0.882016in}}{\pgfqpoint{3.692624in}{0.889829in}}%
\pgfpathcurveto{\pgfqpoint{3.700437in}{0.897643in}}{\pgfqpoint{3.704828in}{0.908242in}}{\pgfqpoint{3.704828in}{0.919292in}}%
\pgfpathcurveto{\pgfqpoint{3.704828in}{0.930342in}}{\pgfqpoint{3.700437in}{0.940941in}}{\pgfqpoint{3.692624in}{0.948755in}}%
\pgfpathcurveto{\pgfqpoint{3.684810in}{0.956569in}}{\pgfqpoint{3.674211in}{0.960959in}}{\pgfqpoint{3.663161in}{0.960959in}}%
\pgfpathcurveto{\pgfqpoint{3.652111in}{0.960959in}}{\pgfqpoint{3.641512in}{0.956569in}}{\pgfqpoint{3.633698in}{0.948755in}}%
\pgfpathcurveto{\pgfqpoint{3.625885in}{0.940941in}}{\pgfqpoint{3.621494in}{0.930342in}}{\pgfqpoint{3.621494in}{0.919292in}}%
\pgfpathcurveto{\pgfqpoint{3.621494in}{0.908242in}}{\pgfqpoint{3.625885in}{0.897643in}}{\pgfqpoint{3.633698in}{0.889829in}}%
\pgfpathcurveto{\pgfqpoint{3.641512in}{0.882016in}}{\pgfqpoint{3.652111in}{0.877626in}}{\pgfqpoint{3.663161in}{0.877626in}}%
\pgfpathlineto{\pgfqpoint{3.663161in}{0.877626in}}%
\pgfpathclose%
\pgfusepath{stroke}%
\end{pgfscope}%
\begin{pgfscope}%
\pgfpathrectangle{\pgfqpoint{0.847223in}{0.554012in}}{\pgfqpoint{6.200000in}{4.620000in}}%
\pgfusepath{clip}%
\pgfsetbuttcap%
\pgfsetroundjoin%
\pgfsetlinewidth{1.003750pt}%
\definecolor{currentstroke}{rgb}{1.000000,0.000000,0.000000}%
\pgfsetstrokecolor{currentstroke}%
\pgfsetdash{}{0pt}%
\pgfpathmoveto{\pgfqpoint{3.668494in}{0.876365in}}%
\pgfpathcurveto{\pgfqpoint{3.679544in}{0.876365in}}{\pgfqpoint{3.690143in}{0.880755in}}{\pgfqpoint{3.697957in}{0.888568in}}%
\pgfpathcurveto{\pgfqpoint{3.705771in}{0.896382in}}{\pgfqpoint{3.710161in}{0.906981in}}{\pgfqpoint{3.710161in}{0.918031in}}%
\pgfpathcurveto{\pgfqpoint{3.710161in}{0.929081in}}{\pgfqpoint{3.705771in}{0.939680in}}{\pgfqpoint{3.697957in}{0.947494in}}%
\pgfpathcurveto{\pgfqpoint{3.690143in}{0.955308in}}{\pgfqpoint{3.679544in}{0.959698in}}{\pgfqpoint{3.668494in}{0.959698in}}%
\pgfpathcurveto{\pgfqpoint{3.657444in}{0.959698in}}{\pgfqpoint{3.646845in}{0.955308in}}{\pgfqpoint{3.639031in}{0.947494in}}%
\pgfpathcurveto{\pgfqpoint{3.631218in}{0.939680in}}{\pgfqpoint{3.626827in}{0.929081in}}{\pgfqpoint{3.626827in}{0.918031in}}%
\pgfpathcurveto{\pgfqpoint{3.626827in}{0.906981in}}{\pgfqpoint{3.631218in}{0.896382in}}{\pgfqpoint{3.639031in}{0.888568in}}%
\pgfpathcurveto{\pgfqpoint{3.646845in}{0.880755in}}{\pgfqpoint{3.657444in}{0.876365in}}{\pgfqpoint{3.668494in}{0.876365in}}%
\pgfpathlineto{\pgfqpoint{3.668494in}{0.876365in}}%
\pgfpathclose%
\pgfusepath{stroke}%
\end{pgfscope}%
\begin{pgfscope}%
\pgfpathrectangle{\pgfqpoint{0.847223in}{0.554012in}}{\pgfqpoint{6.200000in}{4.620000in}}%
\pgfusepath{clip}%
\pgfsetbuttcap%
\pgfsetroundjoin%
\pgfsetlinewidth{1.003750pt}%
\definecolor{currentstroke}{rgb}{1.000000,0.000000,0.000000}%
\pgfsetstrokecolor{currentstroke}%
\pgfsetdash{}{0pt}%
\pgfpathmoveto{\pgfqpoint{3.673827in}{0.875108in}}%
\pgfpathcurveto{\pgfqpoint{3.684877in}{0.875108in}}{\pgfqpoint{3.695477in}{0.879498in}}{\pgfqpoint{3.703290in}{0.887312in}}%
\pgfpathcurveto{\pgfqpoint{3.711104in}{0.895125in}}{\pgfqpoint{3.715494in}{0.905724in}}{\pgfqpoint{3.715494in}{0.916774in}}%
\pgfpathcurveto{\pgfqpoint{3.715494in}{0.927825in}}{\pgfqpoint{3.711104in}{0.938424in}}{\pgfqpoint{3.703290in}{0.946237in}}%
\pgfpathcurveto{\pgfqpoint{3.695477in}{0.954051in}}{\pgfqpoint{3.684877in}{0.958441in}}{\pgfqpoint{3.673827in}{0.958441in}}%
\pgfpathcurveto{\pgfqpoint{3.662777in}{0.958441in}}{\pgfqpoint{3.652178in}{0.954051in}}{\pgfqpoint{3.644365in}{0.946237in}}%
\pgfpathcurveto{\pgfqpoint{3.636551in}{0.938424in}}{\pgfqpoint{3.632161in}{0.927825in}}{\pgfqpoint{3.632161in}{0.916774in}}%
\pgfpathcurveto{\pgfqpoint{3.632161in}{0.905724in}}{\pgfqpoint{3.636551in}{0.895125in}}{\pgfqpoint{3.644365in}{0.887312in}}%
\pgfpathcurveto{\pgfqpoint{3.652178in}{0.879498in}}{\pgfqpoint{3.662777in}{0.875108in}}{\pgfqpoint{3.673827in}{0.875108in}}%
\pgfpathlineto{\pgfqpoint{3.673827in}{0.875108in}}%
\pgfpathclose%
\pgfusepath{stroke}%
\end{pgfscope}%
\begin{pgfscope}%
\pgfpathrectangle{\pgfqpoint{0.847223in}{0.554012in}}{\pgfqpoint{6.200000in}{4.620000in}}%
\pgfusepath{clip}%
\pgfsetbuttcap%
\pgfsetroundjoin%
\pgfsetlinewidth{1.003750pt}%
\definecolor{currentstroke}{rgb}{1.000000,0.000000,0.000000}%
\pgfsetstrokecolor{currentstroke}%
\pgfsetdash{}{0pt}%
\pgfpathmoveto{\pgfqpoint{3.679161in}{0.873855in}}%
\pgfpathcurveto{\pgfqpoint{3.690211in}{0.873855in}}{\pgfqpoint{3.700810in}{0.878245in}}{\pgfqpoint{3.708623in}{0.886059in}}%
\pgfpathcurveto{\pgfqpoint{3.716437in}{0.893872in}}{\pgfqpoint{3.720827in}{0.904471in}}{\pgfqpoint{3.720827in}{0.915521in}}%
\pgfpathcurveto{\pgfqpoint{3.720827in}{0.926572in}}{\pgfqpoint{3.716437in}{0.937171in}}{\pgfqpoint{3.708623in}{0.944984in}}%
\pgfpathcurveto{\pgfqpoint{3.700810in}{0.952798in}}{\pgfqpoint{3.690211in}{0.957188in}}{\pgfqpoint{3.679161in}{0.957188in}}%
\pgfpathcurveto{\pgfqpoint{3.668110in}{0.957188in}}{\pgfqpoint{3.657511in}{0.952798in}}{\pgfqpoint{3.649698in}{0.944984in}}%
\pgfpathcurveto{\pgfqpoint{3.641884in}{0.937171in}}{\pgfqpoint{3.637494in}{0.926572in}}{\pgfqpoint{3.637494in}{0.915521in}}%
\pgfpathcurveto{\pgfqpoint{3.637494in}{0.904471in}}{\pgfqpoint{3.641884in}{0.893872in}}{\pgfqpoint{3.649698in}{0.886059in}}%
\pgfpathcurveto{\pgfqpoint{3.657511in}{0.878245in}}{\pgfqpoint{3.668110in}{0.873855in}}{\pgfqpoint{3.679161in}{0.873855in}}%
\pgfpathlineto{\pgfqpoint{3.679161in}{0.873855in}}%
\pgfpathclose%
\pgfusepath{stroke}%
\end{pgfscope}%
\begin{pgfscope}%
\pgfpathrectangle{\pgfqpoint{0.847223in}{0.554012in}}{\pgfqpoint{6.200000in}{4.620000in}}%
\pgfusepath{clip}%
\pgfsetbuttcap%
\pgfsetroundjoin%
\pgfsetlinewidth{1.003750pt}%
\definecolor{currentstroke}{rgb}{1.000000,0.000000,0.000000}%
\pgfsetstrokecolor{currentstroke}%
\pgfsetdash{}{0pt}%
\pgfpathmoveto{\pgfqpoint{3.684494in}{0.872606in}}%
\pgfpathcurveto{\pgfqpoint{3.695544in}{0.872606in}}{\pgfqpoint{3.706143in}{0.876996in}}{\pgfqpoint{3.713957in}{0.884810in}}%
\pgfpathcurveto{\pgfqpoint{3.721770in}{0.892623in}}{\pgfqpoint{3.726160in}{0.903222in}}{\pgfqpoint{3.726160in}{0.914273in}}%
\pgfpathcurveto{\pgfqpoint{3.726160in}{0.925323in}}{\pgfqpoint{3.721770in}{0.935922in}}{\pgfqpoint{3.713957in}{0.943735in}}%
\pgfpathcurveto{\pgfqpoint{3.706143in}{0.951549in}}{\pgfqpoint{3.695544in}{0.955939in}}{\pgfqpoint{3.684494in}{0.955939in}}%
\pgfpathcurveto{\pgfqpoint{3.673444in}{0.955939in}}{\pgfqpoint{3.662845in}{0.951549in}}{\pgfqpoint{3.655031in}{0.943735in}}%
\pgfpathcurveto{\pgfqpoint{3.647217in}{0.935922in}}{\pgfqpoint{3.642827in}{0.925323in}}{\pgfqpoint{3.642827in}{0.914273in}}%
\pgfpathcurveto{\pgfqpoint{3.642827in}{0.903222in}}{\pgfqpoint{3.647217in}{0.892623in}}{\pgfqpoint{3.655031in}{0.884810in}}%
\pgfpathcurveto{\pgfqpoint{3.662845in}{0.876996in}}{\pgfqpoint{3.673444in}{0.872606in}}{\pgfqpoint{3.684494in}{0.872606in}}%
\pgfpathlineto{\pgfqpoint{3.684494in}{0.872606in}}%
\pgfpathclose%
\pgfusepath{stroke}%
\end{pgfscope}%
\begin{pgfscope}%
\pgfpathrectangle{\pgfqpoint{0.847223in}{0.554012in}}{\pgfqpoint{6.200000in}{4.620000in}}%
\pgfusepath{clip}%
\pgfsetbuttcap%
\pgfsetroundjoin%
\pgfsetlinewidth{1.003750pt}%
\definecolor{currentstroke}{rgb}{1.000000,0.000000,0.000000}%
\pgfsetstrokecolor{currentstroke}%
\pgfsetdash{}{0pt}%
\pgfpathmoveto{\pgfqpoint{3.689827in}{0.871361in}}%
\pgfpathcurveto{\pgfqpoint{3.700877in}{0.871361in}}{\pgfqpoint{3.711476in}{0.875751in}}{\pgfqpoint{3.719290in}{0.883565in}}%
\pgfpathcurveto{\pgfqpoint{3.727103in}{0.891378in}}{\pgfqpoint{3.731494in}{0.901977in}}{\pgfqpoint{3.731494in}{0.913028in}}%
\pgfpathcurveto{\pgfqpoint{3.731494in}{0.924078in}}{\pgfqpoint{3.727103in}{0.934677in}}{\pgfqpoint{3.719290in}{0.942490in}}%
\pgfpathcurveto{\pgfqpoint{3.711476in}{0.950304in}}{\pgfqpoint{3.700877in}{0.954694in}}{\pgfqpoint{3.689827in}{0.954694in}}%
\pgfpathcurveto{\pgfqpoint{3.678777in}{0.954694in}}{\pgfqpoint{3.668178in}{0.950304in}}{\pgfqpoint{3.660364in}{0.942490in}}%
\pgfpathcurveto{\pgfqpoint{3.652551in}{0.934677in}}{\pgfqpoint{3.648160in}{0.924078in}}{\pgfqpoint{3.648160in}{0.913028in}}%
\pgfpathcurveto{\pgfqpoint{3.648160in}{0.901977in}}{\pgfqpoint{3.652551in}{0.891378in}}{\pgfqpoint{3.660364in}{0.883565in}}%
\pgfpathcurveto{\pgfqpoint{3.668178in}{0.875751in}}{\pgfqpoint{3.678777in}{0.871361in}}{\pgfqpoint{3.689827in}{0.871361in}}%
\pgfpathlineto{\pgfqpoint{3.689827in}{0.871361in}}%
\pgfpathclose%
\pgfusepath{stroke}%
\end{pgfscope}%
\begin{pgfscope}%
\pgfpathrectangle{\pgfqpoint{0.847223in}{0.554012in}}{\pgfqpoint{6.200000in}{4.620000in}}%
\pgfusepath{clip}%
\pgfsetbuttcap%
\pgfsetroundjoin%
\pgfsetlinewidth{1.003750pt}%
\definecolor{currentstroke}{rgb}{1.000000,0.000000,0.000000}%
\pgfsetstrokecolor{currentstroke}%
\pgfsetdash{}{0pt}%
\pgfpathmoveto{\pgfqpoint{3.695160in}{0.870120in}}%
\pgfpathcurveto{\pgfqpoint{3.706210in}{0.870120in}}{\pgfqpoint{3.716809in}{0.874510in}}{\pgfqpoint{3.724623in}{0.882324in}}%
\pgfpathcurveto{\pgfqpoint{3.732437in}{0.890137in}}{\pgfqpoint{3.736827in}{0.900736in}}{\pgfqpoint{3.736827in}{0.911787in}}%
\pgfpathcurveto{\pgfqpoint{3.736827in}{0.922837in}}{\pgfqpoint{3.732437in}{0.933436in}}{\pgfqpoint{3.724623in}{0.941249in}}%
\pgfpathcurveto{\pgfqpoint{3.716809in}{0.949063in}}{\pgfqpoint{3.706210in}{0.953453in}}{\pgfqpoint{3.695160in}{0.953453in}}%
\pgfpathcurveto{\pgfqpoint{3.684110in}{0.953453in}}{\pgfqpoint{3.673511in}{0.949063in}}{\pgfqpoint{3.665697in}{0.941249in}}%
\pgfpathcurveto{\pgfqpoint{3.657884in}{0.933436in}}{\pgfqpoint{3.653494in}{0.922837in}}{\pgfqpoint{3.653494in}{0.911787in}}%
\pgfpathcurveto{\pgfqpoint{3.653494in}{0.900736in}}{\pgfqpoint{3.657884in}{0.890137in}}{\pgfqpoint{3.665697in}{0.882324in}}%
\pgfpathcurveto{\pgfqpoint{3.673511in}{0.874510in}}{\pgfqpoint{3.684110in}{0.870120in}}{\pgfqpoint{3.695160in}{0.870120in}}%
\pgfpathlineto{\pgfqpoint{3.695160in}{0.870120in}}%
\pgfpathclose%
\pgfusepath{stroke}%
\end{pgfscope}%
\begin{pgfscope}%
\pgfpathrectangle{\pgfqpoint{0.847223in}{0.554012in}}{\pgfqpoint{6.200000in}{4.620000in}}%
\pgfusepath{clip}%
\pgfsetbuttcap%
\pgfsetroundjoin%
\pgfsetlinewidth{1.003750pt}%
\definecolor{currentstroke}{rgb}{1.000000,0.000000,0.000000}%
\pgfsetstrokecolor{currentstroke}%
\pgfsetdash{}{0pt}%
\pgfpathmoveto{\pgfqpoint{3.700493in}{0.868883in}}%
\pgfpathcurveto{\pgfqpoint{3.711544in}{0.868883in}}{\pgfqpoint{3.722143in}{0.873273in}}{\pgfqpoint{3.729956in}{0.881087in}}%
\pgfpathcurveto{\pgfqpoint{3.737770in}{0.888900in}}{\pgfqpoint{3.742160in}{0.899499in}}{\pgfqpoint{3.742160in}{0.910549in}}%
\pgfpathcurveto{\pgfqpoint{3.742160in}{0.921600in}}{\pgfqpoint{3.737770in}{0.932199in}}{\pgfqpoint{3.729956in}{0.940012in}}%
\pgfpathcurveto{\pgfqpoint{3.722143in}{0.947826in}}{\pgfqpoint{3.711544in}{0.952216in}}{\pgfqpoint{3.700493in}{0.952216in}}%
\pgfpathcurveto{\pgfqpoint{3.689443in}{0.952216in}}{\pgfqpoint{3.678844in}{0.947826in}}{\pgfqpoint{3.671031in}{0.940012in}}%
\pgfpathcurveto{\pgfqpoint{3.663217in}{0.932199in}}{\pgfqpoint{3.658827in}{0.921600in}}{\pgfqpoint{3.658827in}{0.910549in}}%
\pgfpathcurveto{\pgfqpoint{3.658827in}{0.899499in}}{\pgfqpoint{3.663217in}{0.888900in}}{\pgfqpoint{3.671031in}{0.881087in}}%
\pgfpathcurveto{\pgfqpoint{3.678844in}{0.873273in}}{\pgfqpoint{3.689443in}{0.868883in}}{\pgfqpoint{3.700493in}{0.868883in}}%
\pgfpathlineto{\pgfqpoint{3.700493in}{0.868883in}}%
\pgfpathclose%
\pgfusepath{stroke}%
\end{pgfscope}%
\begin{pgfscope}%
\pgfpathrectangle{\pgfqpoint{0.847223in}{0.554012in}}{\pgfqpoint{6.200000in}{4.620000in}}%
\pgfusepath{clip}%
\pgfsetbuttcap%
\pgfsetroundjoin%
\pgfsetlinewidth{1.003750pt}%
\definecolor{currentstroke}{rgb}{1.000000,0.000000,0.000000}%
\pgfsetstrokecolor{currentstroke}%
\pgfsetdash{}{0pt}%
\pgfpathmoveto{\pgfqpoint{3.705827in}{0.867650in}}%
\pgfpathcurveto{\pgfqpoint{3.716877in}{0.867650in}}{\pgfqpoint{3.727476in}{0.872040in}}{\pgfqpoint{3.735289in}{0.879853in}}%
\pgfpathcurveto{\pgfqpoint{3.743103in}{0.887667in}}{\pgfqpoint{3.747493in}{0.898266in}}{\pgfqpoint{3.747493in}{0.909316in}}%
\pgfpathcurveto{\pgfqpoint{3.747493in}{0.920366in}}{\pgfqpoint{3.743103in}{0.930965in}}{\pgfqpoint{3.735289in}{0.938779in}}%
\pgfpathcurveto{\pgfqpoint{3.727476in}{0.946593in}}{\pgfqpoint{3.716877in}{0.950983in}}{\pgfqpoint{3.705827in}{0.950983in}}%
\pgfpathcurveto{\pgfqpoint{3.694777in}{0.950983in}}{\pgfqpoint{3.684177in}{0.946593in}}{\pgfqpoint{3.676364in}{0.938779in}}%
\pgfpathcurveto{\pgfqpoint{3.668550in}{0.930965in}}{\pgfqpoint{3.664160in}{0.920366in}}{\pgfqpoint{3.664160in}{0.909316in}}%
\pgfpathcurveto{\pgfqpoint{3.664160in}{0.898266in}}{\pgfqpoint{3.668550in}{0.887667in}}{\pgfqpoint{3.676364in}{0.879853in}}%
\pgfpathcurveto{\pgfqpoint{3.684177in}{0.872040in}}{\pgfqpoint{3.694777in}{0.867650in}}{\pgfqpoint{3.705827in}{0.867650in}}%
\pgfpathlineto{\pgfqpoint{3.705827in}{0.867650in}}%
\pgfpathclose%
\pgfusepath{stroke}%
\end{pgfscope}%
\begin{pgfscope}%
\pgfpathrectangle{\pgfqpoint{0.847223in}{0.554012in}}{\pgfqpoint{6.200000in}{4.620000in}}%
\pgfusepath{clip}%
\pgfsetbuttcap%
\pgfsetroundjoin%
\pgfsetlinewidth{1.003750pt}%
\definecolor{currentstroke}{rgb}{1.000000,0.000000,0.000000}%
\pgfsetstrokecolor{currentstroke}%
\pgfsetdash{}{0pt}%
\pgfpathmoveto{\pgfqpoint{3.711160in}{0.866420in}}%
\pgfpathcurveto{\pgfqpoint{3.722210in}{0.866420in}}{\pgfqpoint{3.732809in}{0.870811in}}{\pgfqpoint{3.740623in}{0.878624in}}%
\pgfpathcurveto{\pgfqpoint{3.748436in}{0.886438in}}{\pgfqpoint{3.752827in}{0.897037in}}{\pgfqpoint{3.752827in}{0.908087in}}%
\pgfpathcurveto{\pgfqpoint{3.752827in}{0.919137in}}{\pgfqpoint{3.748436in}{0.929736in}}{\pgfqpoint{3.740623in}{0.937550in}}%
\pgfpathcurveto{\pgfqpoint{3.732809in}{0.945363in}}{\pgfqpoint{3.722210in}{0.949754in}}{\pgfqpoint{3.711160in}{0.949754in}}%
\pgfpathcurveto{\pgfqpoint{3.700110in}{0.949754in}}{\pgfqpoint{3.689511in}{0.945363in}}{\pgfqpoint{3.681697in}{0.937550in}}%
\pgfpathcurveto{\pgfqpoint{3.673883in}{0.929736in}}{\pgfqpoint{3.669493in}{0.919137in}}{\pgfqpoint{3.669493in}{0.908087in}}%
\pgfpathcurveto{\pgfqpoint{3.669493in}{0.897037in}}{\pgfqpoint{3.673883in}{0.886438in}}{\pgfqpoint{3.681697in}{0.878624in}}%
\pgfpathcurveto{\pgfqpoint{3.689511in}{0.870811in}}{\pgfqpoint{3.700110in}{0.866420in}}{\pgfqpoint{3.711160in}{0.866420in}}%
\pgfpathlineto{\pgfqpoint{3.711160in}{0.866420in}}%
\pgfpathclose%
\pgfusepath{stroke}%
\end{pgfscope}%
\begin{pgfscope}%
\pgfpathrectangle{\pgfqpoint{0.847223in}{0.554012in}}{\pgfqpoint{6.200000in}{4.620000in}}%
\pgfusepath{clip}%
\pgfsetbuttcap%
\pgfsetroundjoin%
\pgfsetlinewidth{1.003750pt}%
\definecolor{currentstroke}{rgb}{1.000000,0.000000,0.000000}%
\pgfsetstrokecolor{currentstroke}%
\pgfsetdash{}{0pt}%
\pgfpathmoveto{\pgfqpoint{3.716493in}{0.865195in}}%
\pgfpathcurveto{\pgfqpoint{3.727543in}{0.865195in}}{\pgfqpoint{3.738142in}{0.869585in}}{\pgfqpoint{3.745956in}{0.877399in}}%
\pgfpathcurveto{\pgfqpoint{3.753769in}{0.885212in}}{\pgfqpoint{3.758160in}{0.895811in}}{\pgfqpoint{3.758160in}{0.906862in}}%
\pgfpathcurveto{\pgfqpoint{3.758160in}{0.917912in}}{\pgfqpoint{3.753769in}{0.928511in}}{\pgfqpoint{3.745956in}{0.936324in}}%
\pgfpathcurveto{\pgfqpoint{3.738142in}{0.944138in}}{\pgfqpoint{3.727543in}{0.948528in}}{\pgfqpoint{3.716493in}{0.948528in}}%
\pgfpathcurveto{\pgfqpoint{3.705443in}{0.948528in}}{\pgfqpoint{3.694844in}{0.944138in}}{\pgfqpoint{3.687030in}{0.936324in}}%
\pgfpathcurveto{\pgfqpoint{3.679217in}{0.928511in}}{\pgfqpoint{3.674826in}{0.917912in}}{\pgfqpoint{3.674826in}{0.906862in}}%
\pgfpathcurveto{\pgfqpoint{3.674826in}{0.895811in}}{\pgfqpoint{3.679217in}{0.885212in}}{\pgfqpoint{3.687030in}{0.877399in}}%
\pgfpathcurveto{\pgfqpoint{3.694844in}{0.869585in}}{\pgfqpoint{3.705443in}{0.865195in}}{\pgfqpoint{3.716493in}{0.865195in}}%
\pgfpathlineto{\pgfqpoint{3.716493in}{0.865195in}}%
\pgfpathclose%
\pgfusepath{stroke}%
\end{pgfscope}%
\begin{pgfscope}%
\pgfpathrectangle{\pgfqpoint{0.847223in}{0.554012in}}{\pgfqpoint{6.200000in}{4.620000in}}%
\pgfusepath{clip}%
\pgfsetbuttcap%
\pgfsetroundjoin%
\pgfsetlinewidth{1.003750pt}%
\definecolor{currentstroke}{rgb}{1.000000,0.000000,0.000000}%
\pgfsetstrokecolor{currentstroke}%
\pgfsetdash{}{0pt}%
\pgfpathmoveto{\pgfqpoint{3.721826in}{0.863973in}}%
\pgfpathcurveto{\pgfqpoint{3.732876in}{0.863973in}}{\pgfqpoint{3.743475in}{0.868364in}}{\pgfqpoint{3.751289in}{0.876177in}}%
\pgfpathcurveto{\pgfqpoint{3.759103in}{0.883991in}}{\pgfqpoint{3.763493in}{0.894590in}}{\pgfqpoint{3.763493in}{0.905640in}}%
\pgfpathcurveto{\pgfqpoint{3.763493in}{0.916690in}}{\pgfqpoint{3.759103in}{0.927289in}}{\pgfqpoint{3.751289in}{0.935103in}}%
\pgfpathcurveto{\pgfqpoint{3.743475in}{0.942916in}}{\pgfqpoint{3.732876in}{0.947307in}}{\pgfqpoint{3.721826in}{0.947307in}}%
\pgfpathcurveto{\pgfqpoint{3.710776in}{0.947307in}}{\pgfqpoint{3.700177in}{0.942916in}}{\pgfqpoint{3.692364in}{0.935103in}}%
\pgfpathcurveto{\pgfqpoint{3.684550in}{0.927289in}}{\pgfqpoint{3.680160in}{0.916690in}}{\pgfqpoint{3.680160in}{0.905640in}}%
\pgfpathcurveto{\pgfqpoint{3.680160in}{0.894590in}}{\pgfqpoint{3.684550in}{0.883991in}}{\pgfqpoint{3.692364in}{0.876177in}}%
\pgfpathcurveto{\pgfqpoint{3.700177in}{0.868364in}}{\pgfqpoint{3.710776in}{0.863973in}}{\pgfqpoint{3.721826in}{0.863973in}}%
\pgfpathlineto{\pgfqpoint{3.721826in}{0.863973in}}%
\pgfpathclose%
\pgfusepath{stroke}%
\end{pgfscope}%
\begin{pgfscope}%
\pgfpathrectangle{\pgfqpoint{0.847223in}{0.554012in}}{\pgfqpoint{6.200000in}{4.620000in}}%
\pgfusepath{clip}%
\pgfsetbuttcap%
\pgfsetroundjoin%
\pgfsetlinewidth{1.003750pt}%
\definecolor{currentstroke}{rgb}{1.000000,0.000000,0.000000}%
\pgfsetstrokecolor{currentstroke}%
\pgfsetdash{}{0pt}%
\pgfpathmoveto{\pgfqpoint{3.727160in}{0.862755in}}%
\pgfpathcurveto{\pgfqpoint{3.738210in}{0.862755in}}{\pgfqpoint{3.748809in}{0.867146in}}{\pgfqpoint{3.756622in}{0.874959in}}%
\pgfpathcurveto{\pgfqpoint{3.764436in}{0.882773in}}{\pgfqpoint{3.768826in}{0.893372in}}{\pgfqpoint{3.768826in}{0.904422in}}%
\pgfpathcurveto{\pgfqpoint{3.768826in}{0.915472in}}{\pgfqpoint{3.764436in}{0.926071in}}{\pgfqpoint{3.756622in}{0.933885in}}%
\pgfpathcurveto{\pgfqpoint{3.748809in}{0.941699in}}{\pgfqpoint{3.738210in}{0.946089in}}{\pgfqpoint{3.727160in}{0.946089in}}%
\pgfpathcurveto{\pgfqpoint{3.716109in}{0.946089in}}{\pgfqpoint{3.705510in}{0.941699in}}{\pgfqpoint{3.697697in}{0.933885in}}%
\pgfpathcurveto{\pgfqpoint{3.689883in}{0.926071in}}{\pgfqpoint{3.685493in}{0.915472in}}{\pgfqpoint{3.685493in}{0.904422in}}%
\pgfpathcurveto{\pgfqpoint{3.685493in}{0.893372in}}{\pgfqpoint{3.689883in}{0.882773in}}{\pgfqpoint{3.697697in}{0.874959in}}%
\pgfpathcurveto{\pgfqpoint{3.705510in}{0.867146in}}{\pgfqpoint{3.716109in}{0.862755in}}{\pgfqpoint{3.727160in}{0.862755in}}%
\pgfpathlineto{\pgfqpoint{3.727160in}{0.862755in}}%
\pgfpathclose%
\pgfusepath{stroke}%
\end{pgfscope}%
\begin{pgfscope}%
\pgfpathrectangle{\pgfqpoint{0.847223in}{0.554012in}}{\pgfqpoint{6.200000in}{4.620000in}}%
\pgfusepath{clip}%
\pgfsetbuttcap%
\pgfsetroundjoin%
\pgfsetlinewidth{1.003750pt}%
\definecolor{currentstroke}{rgb}{1.000000,0.000000,0.000000}%
\pgfsetstrokecolor{currentstroke}%
\pgfsetdash{}{0pt}%
\pgfpathmoveto{\pgfqpoint{3.732493in}{0.861542in}}%
\pgfpathcurveto{\pgfqpoint{3.743543in}{0.861542in}}{\pgfqpoint{3.754142in}{0.865932in}}{\pgfqpoint{3.761956in}{0.873745in}}%
\pgfpathcurveto{\pgfqpoint{3.769769in}{0.881559in}}{\pgfqpoint{3.774159in}{0.892158in}}{\pgfqpoint{3.774159in}{0.903208in}}%
\pgfpathcurveto{\pgfqpoint{3.774159in}{0.914258in}}{\pgfqpoint{3.769769in}{0.924857in}}{\pgfqpoint{3.761956in}{0.932671in}}%
\pgfpathcurveto{\pgfqpoint{3.754142in}{0.940485in}}{\pgfqpoint{3.743543in}{0.944875in}}{\pgfqpoint{3.732493in}{0.944875in}}%
\pgfpathcurveto{\pgfqpoint{3.721443in}{0.944875in}}{\pgfqpoint{3.710844in}{0.940485in}}{\pgfqpoint{3.703030in}{0.932671in}}%
\pgfpathcurveto{\pgfqpoint{3.695216in}{0.924857in}}{\pgfqpoint{3.690826in}{0.914258in}}{\pgfqpoint{3.690826in}{0.903208in}}%
\pgfpathcurveto{\pgfqpoint{3.690826in}{0.892158in}}{\pgfqpoint{3.695216in}{0.881559in}}{\pgfqpoint{3.703030in}{0.873745in}}%
\pgfpathcurveto{\pgfqpoint{3.710844in}{0.865932in}}{\pgfqpoint{3.721443in}{0.861542in}}{\pgfqpoint{3.732493in}{0.861542in}}%
\pgfpathlineto{\pgfqpoint{3.732493in}{0.861542in}}%
\pgfpathclose%
\pgfusepath{stroke}%
\end{pgfscope}%
\begin{pgfscope}%
\pgfpathrectangle{\pgfqpoint{0.847223in}{0.554012in}}{\pgfqpoint{6.200000in}{4.620000in}}%
\pgfusepath{clip}%
\pgfsetbuttcap%
\pgfsetroundjoin%
\pgfsetlinewidth{1.003750pt}%
\definecolor{currentstroke}{rgb}{1.000000,0.000000,0.000000}%
\pgfsetstrokecolor{currentstroke}%
\pgfsetdash{}{0pt}%
\pgfpathmoveto{\pgfqpoint{3.737826in}{0.860331in}}%
\pgfpathcurveto{\pgfqpoint{3.748876in}{0.860331in}}{\pgfqpoint{3.759475in}{0.864722in}}{\pgfqpoint{3.767289in}{0.872535in}}%
\pgfpathcurveto{\pgfqpoint{3.775102in}{0.880349in}}{\pgfqpoint{3.779493in}{0.890948in}}{\pgfqpoint{3.779493in}{0.901998in}}%
\pgfpathcurveto{\pgfqpoint{3.779493in}{0.913048in}}{\pgfqpoint{3.775102in}{0.923647in}}{\pgfqpoint{3.767289in}{0.931461in}}%
\pgfpathcurveto{\pgfqpoint{3.759475in}{0.939274in}}{\pgfqpoint{3.748876in}{0.943665in}}{\pgfqpoint{3.737826in}{0.943665in}}%
\pgfpathcurveto{\pgfqpoint{3.726776in}{0.943665in}}{\pgfqpoint{3.716177in}{0.939274in}}{\pgfqpoint{3.708363in}{0.931461in}}%
\pgfpathcurveto{\pgfqpoint{3.700550in}{0.923647in}}{\pgfqpoint{3.696159in}{0.913048in}}{\pgfqpoint{3.696159in}{0.901998in}}%
\pgfpathcurveto{\pgfqpoint{3.696159in}{0.890948in}}{\pgfqpoint{3.700550in}{0.880349in}}{\pgfqpoint{3.708363in}{0.872535in}}%
\pgfpathcurveto{\pgfqpoint{3.716177in}{0.864722in}}{\pgfqpoint{3.726776in}{0.860331in}}{\pgfqpoint{3.737826in}{0.860331in}}%
\pgfpathlineto{\pgfqpoint{3.737826in}{0.860331in}}%
\pgfpathclose%
\pgfusepath{stroke}%
\end{pgfscope}%
\begin{pgfscope}%
\pgfpathrectangle{\pgfqpoint{0.847223in}{0.554012in}}{\pgfqpoint{6.200000in}{4.620000in}}%
\pgfusepath{clip}%
\pgfsetbuttcap%
\pgfsetroundjoin%
\pgfsetlinewidth{1.003750pt}%
\definecolor{currentstroke}{rgb}{1.000000,0.000000,0.000000}%
\pgfsetstrokecolor{currentstroke}%
\pgfsetdash{}{0pt}%
\pgfpathmoveto{\pgfqpoint{3.743159in}{0.859125in}}%
\pgfpathcurveto{\pgfqpoint{3.754209in}{0.859125in}}{\pgfqpoint{3.764808in}{0.863515in}}{\pgfqpoint{3.772622in}{0.871329in}}%
\pgfpathcurveto{\pgfqpoint{3.780436in}{0.879142in}}{\pgfqpoint{3.784826in}{0.889741in}}{\pgfqpoint{3.784826in}{0.900792in}}%
\pgfpathcurveto{\pgfqpoint{3.784826in}{0.911842in}}{\pgfqpoint{3.780436in}{0.922441in}}{\pgfqpoint{3.772622in}{0.930254in}}%
\pgfpathcurveto{\pgfqpoint{3.764808in}{0.938068in}}{\pgfqpoint{3.754209in}{0.942458in}}{\pgfqpoint{3.743159in}{0.942458in}}%
\pgfpathcurveto{\pgfqpoint{3.732109in}{0.942458in}}{\pgfqpoint{3.721510in}{0.938068in}}{\pgfqpoint{3.713696in}{0.930254in}}%
\pgfpathcurveto{\pgfqpoint{3.705883in}{0.922441in}}{\pgfqpoint{3.701492in}{0.911842in}}{\pgfqpoint{3.701492in}{0.900792in}}%
\pgfpathcurveto{\pgfqpoint{3.701492in}{0.889741in}}{\pgfqpoint{3.705883in}{0.879142in}}{\pgfqpoint{3.713696in}{0.871329in}}%
\pgfpathcurveto{\pgfqpoint{3.721510in}{0.863515in}}{\pgfqpoint{3.732109in}{0.859125in}}{\pgfqpoint{3.743159in}{0.859125in}}%
\pgfpathlineto{\pgfqpoint{3.743159in}{0.859125in}}%
\pgfpathclose%
\pgfusepath{stroke}%
\end{pgfscope}%
\begin{pgfscope}%
\pgfpathrectangle{\pgfqpoint{0.847223in}{0.554012in}}{\pgfqpoint{6.200000in}{4.620000in}}%
\pgfusepath{clip}%
\pgfsetbuttcap%
\pgfsetroundjoin%
\pgfsetlinewidth{1.003750pt}%
\definecolor{currentstroke}{rgb}{1.000000,0.000000,0.000000}%
\pgfsetstrokecolor{currentstroke}%
\pgfsetdash{}{0pt}%
\pgfpathmoveto{\pgfqpoint{3.748492in}{0.857922in}}%
\pgfpathcurveto{\pgfqpoint{3.759543in}{0.857922in}}{\pgfqpoint{3.770142in}{0.862313in}}{\pgfqpoint{3.777955in}{0.870126in}}%
\pgfpathcurveto{\pgfqpoint{3.785769in}{0.877940in}}{\pgfqpoint{3.790159in}{0.888539in}}{\pgfqpoint{3.790159in}{0.899589in}}%
\pgfpathcurveto{\pgfqpoint{3.790159in}{0.910639in}}{\pgfqpoint{3.785769in}{0.921238in}}{\pgfqpoint{3.777955in}{0.929052in}}%
\pgfpathcurveto{\pgfqpoint{3.770142in}{0.936865in}}{\pgfqpoint{3.759543in}{0.941256in}}{\pgfqpoint{3.748492in}{0.941256in}}%
\pgfpathcurveto{\pgfqpoint{3.737442in}{0.941256in}}{\pgfqpoint{3.726843in}{0.936865in}}{\pgfqpoint{3.719030in}{0.929052in}}%
\pgfpathcurveto{\pgfqpoint{3.711216in}{0.921238in}}{\pgfqpoint{3.706826in}{0.910639in}}{\pgfqpoint{3.706826in}{0.899589in}}%
\pgfpathcurveto{\pgfqpoint{3.706826in}{0.888539in}}{\pgfqpoint{3.711216in}{0.877940in}}{\pgfqpoint{3.719030in}{0.870126in}}%
\pgfpathcurveto{\pgfqpoint{3.726843in}{0.862313in}}{\pgfqpoint{3.737442in}{0.857922in}}{\pgfqpoint{3.748492in}{0.857922in}}%
\pgfpathlineto{\pgfqpoint{3.748492in}{0.857922in}}%
\pgfpathclose%
\pgfusepath{stroke}%
\end{pgfscope}%
\begin{pgfscope}%
\pgfpathrectangle{\pgfqpoint{0.847223in}{0.554012in}}{\pgfqpoint{6.200000in}{4.620000in}}%
\pgfusepath{clip}%
\pgfsetbuttcap%
\pgfsetroundjoin%
\pgfsetlinewidth{1.003750pt}%
\definecolor{currentstroke}{rgb}{1.000000,0.000000,0.000000}%
\pgfsetstrokecolor{currentstroke}%
\pgfsetdash{}{0pt}%
\pgfpathmoveto{\pgfqpoint{3.753826in}{0.856723in}}%
\pgfpathcurveto{\pgfqpoint{3.764876in}{0.856723in}}{\pgfqpoint{3.775475in}{0.861114in}}{\pgfqpoint{3.783288in}{0.868927in}}%
\pgfpathcurveto{\pgfqpoint{3.791102in}{0.876741in}}{\pgfqpoint{3.795492in}{0.887340in}}{\pgfqpoint{3.795492in}{0.898390in}}%
\pgfpathcurveto{\pgfqpoint{3.795492in}{0.909440in}}{\pgfqpoint{3.791102in}{0.920039in}}{\pgfqpoint{3.783288in}{0.927853in}}%
\pgfpathcurveto{\pgfqpoint{3.775475in}{0.935666in}}{\pgfqpoint{3.764876in}{0.940057in}}{\pgfqpoint{3.753826in}{0.940057in}}%
\pgfpathcurveto{\pgfqpoint{3.742775in}{0.940057in}}{\pgfqpoint{3.732176in}{0.935666in}}{\pgfqpoint{3.724363in}{0.927853in}}%
\pgfpathcurveto{\pgfqpoint{3.716549in}{0.920039in}}{\pgfqpoint{3.712159in}{0.909440in}}{\pgfqpoint{3.712159in}{0.898390in}}%
\pgfpathcurveto{\pgfqpoint{3.712159in}{0.887340in}}{\pgfqpoint{3.716549in}{0.876741in}}{\pgfqpoint{3.724363in}{0.868927in}}%
\pgfpathcurveto{\pgfqpoint{3.732176in}{0.861114in}}{\pgfqpoint{3.742775in}{0.856723in}}{\pgfqpoint{3.753826in}{0.856723in}}%
\pgfpathlineto{\pgfqpoint{3.753826in}{0.856723in}}%
\pgfpathclose%
\pgfusepath{stroke}%
\end{pgfscope}%
\begin{pgfscope}%
\pgfpathrectangle{\pgfqpoint{0.847223in}{0.554012in}}{\pgfqpoint{6.200000in}{4.620000in}}%
\pgfusepath{clip}%
\pgfsetbuttcap%
\pgfsetroundjoin%
\pgfsetlinewidth{1.003750pt}%
\definecolor{currentstroke}{rgb}{1.000000,0.000000,0.000000}%
\pgfsetstrokecolor{currentstroke}%
\pgfsetdash{}{0pt}%
\pgfpathmoveto{\pgfqpoint{3.759159in}{0.855528in}}%
\pgfpathcurveto{\pgfqpoint{3.770209in}{0.855528in}}{\pgfqpoint{3.780808in}{0.859918in}}{\pgfqpoint{3.788622in}{0.867732in}}%
\pgfpathcurveto{\pgfqpoint{3.796435in}{0.875546in}}{\pgfqpoint{3.800825in}{0.886145in}}{\pgfqpoint{3.800825in}{0.897195in}}%
\pgfpathcurveto{\pgfqpoint{3.800825in}{0.908245in}}{\pgfqpoint{3.796435in}{0.918844in}}{\pgfqpoint{3.788622in}{0.926658in}}%
\pgfpathcurveto{\pgfqpoint{3.780808in}{0.934471in}}{\pgfqpoint{3.770209in}{0.938862in}}{\pgfqpoint{3.759159in}{0.938862in}}%
\pgfpathcurveto{\pgfqpoint{3.748109in}{0.938862in}}{\pgfqpoint{3.737510in}{0.934471in}}{\pgfqpoint{3.729696in}{0.926658in}}%
\pgfpathcurveto{\pgfqpoint{3.721882in}{0.918844in}}{\pgfqpoint{3.717492in}{0.908245in}}{\pgfqpoint{3.717492in}{0.897195in}}%
\pgfpathcurveto{\pgfqpoint{3.717492in}{0.886145in}}{\pgfqpoint{3.721882in}{0.875546in}}{\pgfqpoint{3.729696in}{0.867732in}}%
\pgfpathcurveto{\pgfqpoint{3.737510in}{0.859918in}}{\pgfqpoint{3.748109in}{0.855528in}}{\pgfqpoint{3.759159in}{0.855528in}}%
\pgfpathlineto{\pgfqpoint{3.759159in}{0.855528in}}%
\pgfpathclose%
\pgfusepath{stroke}%
\end{pgfscope}%
\begin{pgfscope}%
\pgfpathrectangle{\pgfqpoint{0.847223in}{0.554012in}}{\pgfqpoint{6.200000in}{4.620000in}}%
\pgfusepath{clip}%
\pgfsetbuttcap%
\pgfsetroundjoin%
\pgfsetlinewidth{1.003750pt}%
\definecolor{currentstroke}{rgb}{1.000000,0.000000,0.000000}%
\pgfsetstrokecolor{currentstroke}%
\pgfsetdash{}{0pt}%
\pgfpathmoveto{\pgfqpoint{3.764492in}{0.854337in}}%
\pgfpathcurveto{\pgfqpoint{3.775542in}{0.854337in}}{\pgfqpoint{3.786141in}{0.858727in}}{\pgfqpoint{3.793955in}{0.866541in}}%
\pgfpathcurveto{\pgfqpoint{3.801768in}{0.874354in}}{\pgfqpoint{3.806159in}{0.884953in}}{\pgfqpoint{3.806159in}{0.896003in}}%
\pgfpathcurveto{\pgfqpoint{3.806159in}{0.907054in}}{\pgfqpoint{3.801768in}{0.917653in}}{\pgfqpoint{3.793955in}{0.925466in}}%
\pgfpathcurveto{\pgfqpoint{3.786141in}{0.933280in}}{\pgfqpoint{3.775542in}{0.937670in}}{\pgfqpoint{3.764492in}{0.937670in}}%
\pgfpathcurveto{\pgfqpoint{3.753442in}{0.937670in}}{\pgfqpoint{3.742843in}{0.933280in}}{\pgfqpoint{3.735029in}{0.925466in}}%
\pgfpathcurveto{\pgfqpoint{3.727216in}{0.917653in}}{\pgfqpoint{3.722825in}{0.907054in}}{\pgfqpoint{3.722825in}{0.896003in}}%
\pgfpathcurveto{\pgfqpoint{3.722825in}{0.884953in}}{\pgfqpoint{3.727216in}{0.874354in}}{\pgfqpoint{3.735029in}{0.866541in}}%
\pgfpathcurveto{\pgfqpoint{3.742843in}{0.858727in}}{\pgfqpoint{3.753442in}{0.854337in}}{\pgfqpoint{3.764492in}{0.854337in}}%
\pgfpathlineto{\pgfqpoint{3.764492in}{0.854337in}}%
\pgfpathclose%
\pgfusepath{stroke}%
\end{pgfscope}%
\begin{pgfscope}%
\pgfpathrectangle{\pgfqpoint{0.847223in}{0.554012in}}{\pgfqpoint{6.200000in}{4.620000in}}%
\pgfusepath{clip}%
\pgfsetbuttcap%
\pgfsetroundjoin%
\pgfsetlinewidth{1.003750pt}%
\definecolor{currentstroke}{rgb}{1.000000,0.000000,0.000000}%
\pgfsetstrokecolor{currentstroke}%
\pgfsetdash{}{0pt}%
\pgfpathmoveto{\pgfqpoint{3.769825in}{0.853149in}}%
\pgfpathcurveto{\pgfqpoint{3.780875in}{0.853149in}}{\pgfqpoint{3.791474in}{0.857539in}}{\pgfqpoint{3.799288in}{0.865353in}}%
\pgfpathcurveto{\pgfqpoint{3.807102in}{0.873166in}}{\pgfqpoint{3.811492in}{0.883765in}}{\pgfqpoint{3.811492in}{0.894816in}}%
\pgfpathcurveto{\pgfqpoint{3.811492in}{0.905866in}}{\pgfqpoint{3.807102in}{0.916465in}}{\pgfqpoint{3.799288in}{0.924278in}}%
\pgfpathcurveto{\pgfqpoint{3.791474in}{0.932092in}}{\pgfqpoint{3.780875in}{0.936482in}}{\pgfqpoint{3.769825in}{0.936482in}}%
\pgfpathcurveto{\pgfqpoint{3.758775in}{0.936482in}}{\pgfqpoint{3.748176in}{0.932092in}}{\pgfqpoint{3.740362in}{0.924278in}}%
\pgfpathcurveto{\pgfqpoint{3.732549in}{0.916465in}}{\pgfqpoint{3.728159in}{0.905866in}}{\pgfqpoint{3.728159in}{0.894816in}}%
\pgfpathcurveto{\pgfqpoint{3.728159in}{0.883765in}}{\pgfqpoint{3.732549in}{0.873166in}}{\pgfqpoint{3.740362in}{0.865353in}}%
\pgfpathcurveto{\pgfqpoint{3.748176in}{0.857539in}}{\pgfqpoint{3.758775in}{0.853149in}}{\pgfqpoint{3.769825in}{0.853149in}}%
\pgfpathlineto{\pgfqpoint{3.769825in}{0.853149in}}%
\pgfpathclose%
\pgfusepath{stroke}%
\end{pgfscope}%
\begin{pgfscope}%
\pgfpathrectangle{\pgfqpoint{0.847223in}{0.554012in}}{\pgfqpoint{6.200000in}{4.620000in}}%
\pgfusepath{clip}%
\pgfsetbuttcap%
\pgfsetroundjoin%
\pgfsetlinewidth{1.003750pt}%
\definecolor{currentstroke}{rgb}{1.000000,0.000000,0.000000}%
\pgfsetstrokecolor{currentstroke}%
\pgfsetdash{}{0pt}%
\pgfpathmoveto{\pgfqpoint{3.775158in}{0.851965in}}%
\pgfpathcurveto{\pgfqpoint{3.786209in}{0.851965in}}{\pgfqpoint{3.796808in}{0.856355in}}{\pgfqpoint{3.804621in}{0.864169in}}%
\pgfpathcurveto{\pgfqpoint{3.812435in}{0.871982in}}{\pgfqpoint{3.816825in}{0.882581in}}{\pgfqpoint{3.816825in}{0.893632in}}%
\pgfpathcurveto{\pgfqpoint{3.816825in}{0.904682in}}{\pgfqpoint{3.812435in}{0.915281in}}{\pgfqpoint{3.804621in}{0.923094in}}%
\pgfpathcurveto{\pgfqpoint{3.796808in}{0.930908in}}{\pgfqpoint{3.786209in}{0.935298in}}{\pgfqpoint{3.775158in}{0.935298in}}%
\pgfpathcurveto{\pgfqpoint{3.764108in}{0.935298in}}{\pgfqpoint{3.753509in}{0.930908in}}{\pgfqpoint{3.745696in}{0.923094in}}%
\pgfpathcurveto{\pgfqpoint{3.737882in}{0.915281in}}{\pgfqpoint{3.733492in}{0.904682in}}{\pgfqpoint{3.733492in}{0.893632in}}%
\pgfpathcurveto{\pgfqpoint{3.733492in}{0.882581in}}{\pgfqpoint{3.737882in}{0.871982in}}{\pgfqpoint{3.745696in}{0.864169in}}%
\pgfpathcurveto{\pgfqpoint{3.753509in}{0.856355in}}{\pgfqpoint{3.764108in}{0.851965in}}{\pgfqpoint{3.775158in}{0.851965in}}%
\pgfpathlineto{\pgfqpoint{3.775158in}{0.851965in}}%
\pgfpathclose%
\pgfusepath{stroke}%
\end{pgfscope}%
\begin{pgfscope}%
\pgfpathrectangle{\pgfqpoint{0.847223in}{0.554012in}}{\pgfqpoint{6.200000in}{4.620000in}}%
\pgfusepath{clip}%
\pgfsetbuttcap%
\pgfsetroundjoin%
\pgfsetlinewidth{1.003750pt}%
\definecolor{currentstroke}{rgb}{1.000000,0.000000,0.000000}%
\pgfsetstrokecolor{currentstroke}%
\pgfsetdash{}{0pt}%
\pgfpathmoveto{\pgfqpoint{3.780492in}{0.850784in}}%
\pgfpathcurveto{\pgfqpoint{3.791542in}{0.850784in}}{\pgfqpoint{3.802141in}{0.855175in}}{\pgfqpoint{3.809954in}{0.862988in}}%
\pgfpathcurveto{\pgfqpoint{3.817768in}{0.870802in}}{\pgfqpoint{3.822158in}{0.881401in}}{\pgfqpoint{3.822158in}{0.892451in}}%
\pgfpathcurveto{\pgfqpoint{3.822158in}{0.903501in}}{\pgfqpoint{3.817768in}{0.914100in}}{\pgfqpoint{3.809954in}{0.921914in}}%
\pgfpathcurveto{\pgfqpoint{3.802141in}{0.929727in}}{\pgfqpoint{3.791542in}{0.934118in}}{\pgfqpoint{3.780492in}{0.934118in}}%
\pgfpathcurveto{\pgfqpoint{3.769442in}{0.934118in}}{\pgfqpoint{3.758843in}{0.929727in}}{\pgfqpoint{3.751029in}{0.921914in}}%
\pgfpathcurveto{\pgfqpoint{3.743215in}{0.914100in}}{\pgfqpoint{3.738825in}{0.903501in}}{\pgfqpoint{3.738825in}{0.892451in}}%
\pgfpathcurveto{\pgfqpoint{3.738825in}{0.881401in}}{\pgfqpoint{3.743215in}{0.870802in}}{\pgfqpoint{3.751029in}{0.862988in}}%
\pgfpathcurveto{\pgfqpoint{3.758843in}{0.855175in}}{\pgfqpoint{3.769442in}{0.850784in}}{\pgfqpoint{3.780492in}{0.850784in}}%
\pgfpathlineto{\pgfqpoint{3.780492in}{0.850784in}}%
\pgfpathclose%
\pgfusepath{stroke}%
\end{pgfscope}%
\begin{pgfscope}%
\pgfpathrectangle{\pgfqpoint{0.847223in}{0.554012in}}{\pgfqpoint{6.200000in}{4.620000in}}%
\pgfusepath{clip}%
\pgfsetbuttcap%
\pgfsetroundjoin%
\pgfsetlinewidth{1.003750pt}%
\definecolor{currentstroke}{rgb}{1.000000,0.000000,0.000000}%
\pgfsetstrokecolor{currentstroke}%
\pgfsetdash{}{0pt}%
\pgfpathmoveto{\pgfqpoint{3.785825in}{0.849608in}}%
\pgfpathcurveto{\pgfqpoint{3.796875in}{0.849608in}}{\pgfqpoint{3.807474in}{0.853998in}}{\pgfqpoint{3.815288in}{0.861811in}}%
\pgfpathcurveto{\pgfqpoint{3.823101in}{0.869625in}}{\pgfqpoint{3.827492in}{0.880224in}}{\pgfqpoint{3.827492in}{0.891274in}}%
\pgfpathcurveto{\pgfqpoint{3.827492in}{0.902324in}}{\pgfqpoint{3.823101in}{0.912923in}}{\pgfqpoint{3.815288in}{0.920737in}}%
\pgfpathcurveto{\pgfqpoint{3.807474in}{0.928551in}}{\pgfqpoint{3.796875in}{0.932941in}}{\pgfqpoint{3.785825in}{0.932941in}}%
\pgfpathcurveto{\pgfqpoint{3.774775in}{0.932941in}}{\pgfqpoint{3.764176in}{0.928551in}}{\pgfqpoint{3.756362in}{0.920737in}}%
\pgfpathcurveto{\pgfqpoint{3.748548in}{0.912923in}}{\pgfqpoint{3.744158in}{0.902324in}}{\pgfqpoint{3.744158in}{0.891274in}}%
\pgfpathcurveto{\pgfqpoint{3.744158in}{0.880224in}}{\pgfqpoint{3.748548in}{0.869625in}}{\pgfqpoint{3.756362in}{0.861811in}}%
\pgfpathcurveto{\pgfqpoint{3.764176in}{0.853998in}}{\pgfqpoint{3.774775in}{0.849608in}}{\pgfqpoint{3.785825in}{0.849608in}}%
\pgfpathlineto{\pgfqpoint{3.785825in}{0.849608in}}%
\pgfpathclose%
\pgfusepath{stroke}%
\end{pgfscope}%
\begin{pgfscope}%
\pgfpathrectangle{\pgfqpoint{0.847223in}{0.554012in}}{\pgfqpoint{6.200000in}{4.620000in}}%
\pgfusepath{clip}%
\pgfsetbuttcap%
\pgfsetroundjoin%
\pgfsetlinewidth{1.003750pt}%
\definecolor{currentstroke}{rgb}{1.000000,0.000000,0.000000}%
\pgfsetstrokecolor{currentstroke}%
\pgfsetdash{}{0pt}%
\pgfpathmoveto{\pgfqpoint{3.791158in}{0.848434in}}%
\pgfpathcurveto{\pgfqpoint{3.802208in}{0.848434in}}{\pgfqpoint{3.812807in}{0.852825in}}{\pgfqpoint{3.820621in}{0.860638in}}%
\pgfpathcurveto{\pgfqpoint{3.828435in}{0.868452in}}{\pgfqpoint{3.832825in}{0.879051in}}{\pgfqpoint{3.832825in}{0.890101in}}%
\pgfpathcurveto{\pgfqpoint{3.832825in}{0.901151in}}{\pgfqpoint{3.828435in}{0.911750in}}{\pgfqpoint{3.820621in}{0.919564in}}%
\pgfpathcurveto{\pgfqpoint{3.812807in}{0.927377in}}{\pgfqpoint{3.802208in}{0.931768in}}{\pgfqpoint{3.791158in}{0.931768in}}%
\pgfpathcurveto{\pgfqpoint{3.780108in}{0.931768in}}{\pgfqpoint{3.769509in}{0.927377in}}{\pgfqpoint{3.761695in}{0.919564in}}%
\pgfpathcurveto{\pgfqpoint{3.753882in}{0.911750in}}{\pgfqpoint{3.749491in}{0.901151in}}{\pgfqpoint{3.749491in}{0.890101in}}%
\pgfpathcurveto{\pgfqpoint{3.749491in}{0.879051in}}{\pgfqpoint{3.753882in}{0.868452in}}{\pgfqpoint{3.761695in}{0.860638in}}%
\pgfpathcurveto{\pgfqpoint{3.769509in}{0.852825in}}{\pgfqpoint{3.780108in}{0.848434in}}{\pgfqpoint{3.791158in}{0.848434in}}%
\pgfpathlineto{\pgfqpoint{3.791158in}{0.848434in}}%
\pgfpathclose%
\pgfusepath{stroke}%
\end{pgfscope}%
\begin{pgfscope}%
\pgfpathrectangle{\pgfqpoint{0.847223in}{0.554012in}}{\pgfqpoint{6.200000in}{4.620000in}}%
\pgfusepath{clip}%
\pgfsetbuttcap%
\pgfsetroundjoin%
\pgfsetlinewidth{1.003750pt}%
\definecolor{currentstroke}{rgb}{1.000000,0.000000,0.000000}%
\pgfsetstrokecolor{currentstroke}%
\pgfsetdash{}{0pt}%
\pgfpathmoveto{\pgfqpoint{3.796491in}{0.847265in}}%
\pgfpathcurveto{\pgfqpoint{3.807541in}{0.847265in}}{\pgfqpoint{3.818140in}{0.851655in}}{\pgfqpoint{3.825954in}{0.859469in}}%
\pgfpathcurveto{\pgfqpoint{3.833768in}{0.867282in}}{\pgfqpoint{3.838158in}{0.877881in}}{\pgfqpoint{3.838158in}{0.888931in}}%
\pgfpathcurveto{\pgfqpoint{3.838158in}{0.899982in}}{\pgfqpoint{3.833768in}{0.910581in}}{\pgfqpoint{3.825954in}{0.918394in}}%
\pgfpathcurveto{\pgfqpoint{3.818140in}{0.926208in}}{\pgfqpoint{3.807541in}{0.930598in}}{\pgfqpoint{3.796491in}{0.930598in}}%
\pgfpathcurveto{\pgfqpoint{3.785441in}{0.930598in}}{\pgfqpoint{3.774842in}{0.926208in}}{\pgfqpoint{3.767029in}{0.918394in}}%
\pgfpathcurveto{\pgfqpoint{3.759215in}{0.910581in}}{\pgfqpoint{3.754825in}{0.899982in}}{\pgfqpoint{3.754825in}{0.888931in}}%
\pgfpathcurveto{\pgfqpoint{3.754825in}{0.877881in}}{\pgfqpoint{3.759215in}{0.867282in}}{\pgfqpoint{3.767029in}{0.859469in}}%
\pgfpathcurveto{\pgfqpoint{3.774842in}{0.851655in}}{\pgfqpoint{3.785441in}{0.847265in}}{\pgfqpoint{3.796491in}{0.847265in}}%
\pgfpathlineto{\pgfqpoint{3.796491in}{0.847265in}}%
\pgfpathclose%
\pgfusepath{stroke}%
\end{pgfscope}%
\begin{pgfscope}%
\pgfpathrectangle{\pgfqpoint{0.847223in}{0.554012in}}{\pgfqpoint{6.200000in}{4.620000in}}%
\pgfusepath{clip}%
\pgfsetbuttcap%
\pgfsetroundjoin%
\pgfsetlinewidth{1.003750pt}%
\definecolor{currentstroke}{rgb}{1.000000,0.000000,0.000000}%
\pgfsetstrokecolor{currentstroke}%
\pgfsetdash{}{0pt}%
\pgfpathmoveto{\pgfqpoint{3.801825in}{0.846099in}}%
\pgfpathcurveto{\pgfqpoint{3.812875in}{0.846099in}}{\pgfqpoint{3.823474in}{0.850489in}}{\pgfqpoint{3.831287in}{0.858303in}}%
\pgfpathcurveto{\pgfqpoint{3.839101in}{0.866116in}}{\pgfqpoint{3.843491in}{0.876715in}}{\pgfqpoint{3.843491in}{0.887765in}}%
\pgfpathcurveto{\pgfqpoint{3.843491in}{0.898816in}}{\pgfqpoint{3.839101in}{0.909415in}}{\pgfqpoint{3.831287in}{0.917228in}}%
\pgfpathcurveto{\pgfqpoint{3.823474in}{0.925042in}}{\pgfqpoint{3.812875in}{0.929432in}}{\pgfqpoint{3.801825in}{0.929432in}}%
\pgfpathcurveto{\pgfqpoint{3.790774in}{0.929432in}}{\pgfqpoint{3.780175in}{0.925042in}}{\pgfqpoint{3.772362in}{0.917228in}}%
\pgfpathcurveto{\pgfqpoint{3.764548in}{0.909415in}}{\pgfqpoint{3.760158in}{0.898816in}}{\pgfqpoint{3.760158in}{0.887765in}}%
\pgfpathcurveto{\pgfqpoint{3.760158in}{0.876715in}}{\pgfqpoint{3.764548in}{0.866116in}}{\pgfqpoint{3.772362in}{0.858303in}}%
\pgfpathcurveto{\pgfqpoint{3.780175in}{0.850489in}}{\pgfqpoint{3.790774in}{0.846099in}}{\pgfqpoint{3.801825in}{0.846099in}}%
\pgfpathlineto{\pgfqpoint{3.801825in}{0.846099in}}%
\pgfpathclose%
\pgfusepath{stroke}%
\end{pgfscope}%
\begin{pgfscope}%
\pgfpathrectangle{\pgfqpoint{0.847223in}{0.554012in}}{\pgfqpoint{6.200000in}{4.620000in}}%
\pgfusepath{clip}%
\pgfsetbuttcap%
\pgfsetroundjoin%
\pgfsetlinewidth{1.003750pt}%
\definecolor{currentstroke}{rgb}{1.000000,0.000000,0.000000}%
\pgfsetstrokecolor{currentstroke}%
\pgfsetdash{}{0pt}%
\pgfpathmoveto{\pgfqpoint{3.807158in}{0.844936in}}%
\pgfpathcurveto{\pgfqpoint{3.818208in}{0.844936in}}{\pgfqpoint{3.828807in}{0.849327in}}{\pgfqpoint{3.836621in}{0.857140in}}%
\pgfpathcurveto{\pgfqpoint{3.844434in}{0.864954in}}{\pgfqpoint{3.848824in}{0.875553in}}{\pgfqpoint{3.848824in}{0.886603in}}%
\pgfpathcurveto{\pgfqpoint{3.848824in}{0.897653in}}{\pgfqpoint{3.844434in}{0.908252in}}{\pgfqpoint{3.836621in}{0.916066in}}%
\pgfpathcurveto{\pgfqpoint{3.828807in}{0.923879in}}{\pgfqpoint{3.818208in}{0.928270in}}{\pgfqpoint{3.807158in}{0.928270in}}%
\pgfpathcurveto{\pgfqpoint{3.796108in}{0.928270in}}{\pgfqpoint{3.785509in}{0.923879in}}{\pgfqpoint{3.777695in}{0.916066in}}%
\pgfpathcurveto{\pgfqpoint{3.769881in}{0.908252in}}{\pgfqpoint{3.765491in}{0.897653in}}{\pgfqpoint{3.765491in}{0.886603in}}%
\pgfpathcurveto{\pgfqpoint{3.765491in}{0.875553in}}{\pgfqpoint{3.769881in}{0.864954in}}{\pgfqpoint{3.777695in}{0.857140in}}%
\pgfpathcurveto{\pgfqpoint{3.785509in}{0.849327in}}{\pgfqpoint{3.796108in}{0.844936in}}{\pgfqpoint{3.807158in}{0.844936in}}%
\pgfpathlineto{\pgfqpoint{3.807158in}{0.844936in}}%
\pgfpathclose%
\pgfusepath{stroke}%
\end{pgfscope}%
\begin{pgfscope}%
\pgfpathrectangle{\pgfqpoint{0.847223in}{0.554012in}}{\pgfqpoint{6.200000in}{4.620000in}}%
\pgfusepath{clip}%
\pgfsetbuttcap%
\pgfsetroundjoin%
\pgfsetlinewidth{1.003750pt}%
\definecolor{currentstroke}{rgb}{1.000000,0.000000,0.000000}%
\pgfsetstrokecolor{currentstroke}%
\pgfsetdash{}{0pt}%
\pgfpathmoveto{\pgfqpoint{3.812491in}{0.843777in}}%
\pgfpathcurveto{\pgfqpoint{3.823541in}{0.843777in}}{\pgfqpoint{3.834140in}{0.848168in}}{\pgfqpoint{3.841954in}{0.855981in}}%
\pgfpathcurveto{\pgfqpoint{3.849767in}{0.863795in}}{\pgfqpoint{3.854158in}{0.874394in}}{\pgfqpoint{3.854158in}{0.885444in}}%
\pgfpathcurveto{\pgfqpoint{3.854158in}{0.896494in}}{\pgfqpoint{3.849767in}{0.907093in}}{\pgfqpoint{3.841954in}{0.914907in}}%
\pgfpathcurveto{\pgfqpoint{3.834140in}{0.922721in}}{\pgfqpoint{3.823541in}{0.927111in}}{\pgfqpoint{3.812491in}{0.927111in}}%
\pgfpathcurveto{\pgfqpoint{3.801441in}{0.927111in}}{\pgfqpoint{3.790842in}{0.922721in}}{\pgfqpoint{3.783028in}{0.914907in}}%
\pgfpathcurveto{\pgfqpoint{3.775215in}{0.907093in}}{\pgfqpoint{3.770824in}{0.896494in}}{\pgfqpoint{3.770824in}{0.885444in}}%
\pgfpathcurveto{\pgfqpoint{3.770824in}{0.874394in}}{\pgfqpoint{3.775215in}{0.863795in}}{\pgfqpoint{3.783028in}{0.855981in}}%
\pgfpathcurveto{\pgfqpoint{3.790842in}{0.848168in}}{\pgfqpoint{3.801441in}{0.843777in}}{\pgfqpoint{3.812491in}{0.843777in}}%
\pgfpathlineto{\pgfqpoint{3.812491in}{0.843777in}}%
\pgfpathclose%
\pgfusepath{stroke}%
\end{pgfscope}%
\begin{pgfscope}%
\pgfpathrectangle{\pgfqpoint{0.847223in}{0.554012in}}{\pgfqpoint{6.200000in}{4.620000in}}%
\pgfusepath{clip}%
\pgfsetbuttcap%
\pgfsetroundjoin%
\pgfsetlinewidth{1.003750pt}%
\definecolor{currentstroke}{rgb}{1.000000,0.000000,0.000000}%
\pgfsetstrokecolor{currentstroke}%
\pgfsetdash{}{0pt}%
\pgfpathmoveto{\pgfqpoint{3.817824in}{0.842622in}}%
\pgfpathcurveto{\pgfqpoint{3.828874in}{0.842622in}}{\pgfqpoint{3.839473in}{0.847012in}}{\pgfqpoint{3.847287in}{0.854826in}}%
\pgfpathcurveto{\pgfqpoint{3.855101in}{0.862640in}}{\pgfqpoint{3.859491in}{0.873239in}}{\pgfqpoint{3.859491in}{0.884289in}}%
\pgfpathcurveto{\pgfqpoint{3.859491in}{0.895339in}}{\pgfqpoint{3.855101in}{0.905938in}}{\pgfqpoint{3.847287in}{0.913752in}}%
\pgfpathcurveto{\pgfqpoint{3.839473in}{0.921565in}}{\pgfqpoint{3.828874in}{0.925955in}}{\pgfqpoint{3.817824in}{0.925955in}}%
\pgfpathcurveto{\pgfqpoint{3.806774in}{0.925955in}}{\pgfqpoint{3.796175in}{0.921565in}}{\pgfqpoint{3.788361in}{0.913752in}}%
\pgfpathcurveto{\pgfqpoint{3.780548in}{0.905938in}}{\pgfqpoint{3.776158in}{0.895339in}}{\pgfqpoint{3.776158in}{0.884289in}}%
\pgfpathcurveto{\pgfqpoint{3.776158in}{0.873239in}}{\pgfqpoint{3.780548in}{0.862640in}}{\pgfqpoint{3.788361in}{0.854826in}}%
\pgfpathcurveto{\pgfqpoint{3.796175in}{0.847012in}}{\pgfqpoint{3.806774in}{0.842622in}}{\pgfqpoint{3.817824in}{0.842622in}}%
\pgfpathlineto{\pgfqpoint{3.817824in}{0.842622in}}%
\pgfpathclose%
\pgfusepath{stroke}%
\end{pgfscope}%
\begin{pgfscope}%
\pgfpathrectangle{\pgfqpoint{0.847223in}{0.554012in}}{\pgfqpoint{6.200000in}{4.620000in}}%
\pgfusepath{clip}%
\pgfsetbuttcap%
\pgfsetroundjoin%
\pgfsetlinewidth{1.003750pt}%
\definecolor{currentstroke}{rgb}{1.000000,0.000000,0.000000}%
\pgfsetstrokecolor{currentstroke}%
\pgfsetdash{}{0pt}%
\pgfpathmoveto{\pgfqpoint{3.823157in}{0.841470in}}%
\pgfpathcurveto{\pgfqpoint{3.834208in}{0.841470in}}{\pgfqpoint{3.844807in}{0.845861in}}{\pgfqpoint{3.852620in}{0.853674in}}%
\pgfpathcurveto{\pgfqpoint{3.860434in}{0.861488in}}{\pgfqpoint{3.864824in}{0.872087in}}{\pgfqpoint{3.864824in}{0.883137in}}%
\pgfpathcurveto{\pgfqpoint{3.864824in}{0.894187in}}{\pgfqpoint{3.860434in}{0.904786in}}{\pgfqpoint{3.852620in}{0.912600in}}%
\pgfpathcurveto{\pgfqpoint{3.844807in}{0.920413in}}{\pgfqpoint{3.834208in}{0.924804in}}{\pgfqpoint{3.823157in}{0.924804in}}%
\pgfpathcurveto{\pgfqpoint{3.812107in}{0.924804in}}{\pgfqpoint{3.801508in}{0.920413in}}{\pgfqpoint{3.793695in}{0.912600in}}%
\pgfpathcurveto{\pgfqpoint{3.785881in}{0.904786in}}{\pgfqpoint{3.781491in}{0.894187in}}{\pgfqpoint{3.781491in}{0.883137in}}%
\pgfpathcurveto{\pgfqpoint{3.781491in}{0.872087in}}{\pgfqpoint{3.785881in}{0.861488in}}{\pgfqpoint{3.793695in}{0.853674in}}%
\pgfpathcurveto{\pgfqpoint{3.801508in}{0.845861in}}{\pgfqpoint{3.812107in}{0.841470in}}{\pgfqpoint{3.823157in}{0.841470in}}%
\pgfpathlineto{\pgfqpoint{3.823157in}{0.841470in}}%
\pgfpathclose%
\pgfusepath{stroke}%
\end{pgfscope}%
\begin{pgfscope}%
\pgfpathrectangle{\pgfqpoint{0.847223in}{0.554012in}}{\pgfqpoint{6.200000in}{4.620000in}}%
\pgfusepath{clip}%
\pgfsetbuttcap%
\pgfsetroundjoin%
\pgfsetlinewidth{1.003750pt}%
\definecolor{currentstroke}{rgb}{1.000000,0.000000,0.000000}%
\pgfsetstrokecolor{currentstroke}%
\pgfsetdash{}{0pt}%
\pgfpathmoveto{\pgfqpoint{3.828491in}{0.840322in}}%
\pgfpathcurveto{\pgfqpoint{3.839541in}{0.840322in}}{\pgfqpoint{3.850140in}{0.844712in}}{\pgfqpoint{3.857953in}{0.852526in}}%
\pgfpathcurveto{\pgfqpoint{3.865767in}{0.860339in}}{\pgfqpoint{3.870157in}{0.870938in}}{\pgfqpoint{3.870157in}{0.881989in}}%
\pgfpathcurveto{\pgfqpoint{3.870157in}{0.893039in}}{\pgfqpoint{3.865767in}{0.903638in}}{\pgfqpoint{3.857953in}{0.911451in}}%
\pgfpathcurveto{\pgfqpoint{3.850140in}{0.919265in}}{\pgfqpoint{3.839541in}{0.923655in}}{\pgfqpoint{3.828491in}{0.923655in}}%
\pgfpathcurveto{\pgfqpoint{3.817440in}{0.923655in}}{\pgfqpoint{3.806841in}{0.919265in}}{\pgfqpoint{3.799028in}{0.911451in}}%
\pgfpathcurveto{\pgfqpoint{3.791214in}{0.903638in}}{\pgfqpoint{3.786824in}{0.893039in}}{\pgfqpoint{3.786824in}{0.881989in}}%
\pgfpathcurveto{\pgfqpoint{3.786824in}{0.870938in}}{\pgfqpoint{3.791214in}{0.860339in}}{\pgfqpoint{3.799028in}{0.852526in}}%
\pgfpathcurveto{\pgfqpoint{3.806841in}{0.844712in}}{\pgfqpoint{3.817440in}{0.840322in}}{\pgfqpoint{3.828491in}{0.840322in}}%
\pgfpathlineto{\pgfqpoint{3.828491in}{0.840322in}}%
\pgfpathclose%
\pgfusepath{stroke}%
\end{pgfscope}%
\begin{pgfscope}%
\pgfpathrectangle{\pgfqpoint{0.847223in}{0.554012in}}{\pgfqpoint{6.200000in}{4.620000in}}%
\pgfusepath{clip}%
\pgfsetbuttcap%
\pgfsetroundjoin%
\pgfsetlinewidth{1.003750pt}%
\definecolor{currentstroke}{rgb}{1.000000,0.000000,0.000000}%
\pgfsetstrokecolor{currentstroke}%
\pgfsetdash{}{0pt}%
\pgfpathmoveto{\pgfqpoint{3.833824in}{0.839177in}}%
\pgfpathcurveto{\pgfqpoint{3.844874in}{0.839177in}}{\pgfqpoint{3.855473in}{0.843567in}}{\pgfqpoint{3.863287in}{0.851381in}}%
\pgfpathcurveto{\pgfqpoint{3.871100in}{0.859195in}}{\pgfqpoint{3.875490in}{0.869794in}}{\pgfqpoint{3.875490in}{0.880844in}}%
\pgfpathcurveto{\pgfqpoint{3.875490in}{0.891894in}}{\pgfqpoint{3.871100in}{0.902493in}}{\pgfqpoint{3.863287in}{0.910307in}}%
\pgfpathcurveto{\pgfqpoint{3.855473in}{0.918120in}}{\pgfqpoint{3.844874in}{0.922510in}}{\pgfqpoint{3.833824in}{0.922510in}}%
\pgfpathcurveto{\pgfqpoint{3.822774in}{0.922510in}}{\pgfqpoint{3.812175in}{0.918120in}}{\pgfqpoint{3.804361in}{0.910307in}}%
\pgfpathcurveto{\pgfqpoint{3.796547in}{0.902493in}}{\pgfqpoint{3.792157in}{0.891894in}}{\pgfqpoint{3.792157in}{0.880844in}}%
\pgfpathcurveto{\pgfqpoint{3.792157in}{0.869794in}}{\pgfqpoint{3.796547in}{0.859195in}}{\pgfqpoint{3.804361in}{0.851381in}}%
\pgfpathcurveto{\pgfqpoint{3.812175in}{0.843567in}}{\pgfqpoint{3.822774in}{0.839177in}}{\pgfqpoint{3.833824in}{0.839177in}}%
\pgfpathlineto{\pgfqpoint{3.833824in}{0.839177in}}%
\pgfpathclose%
\pgfusepath{stroke}%
\end{pgfscope}%
\begin{pgfscope}%
\pgfpathrectangle{\pgfqpoint{0.847223in}{0.554012in}}{\pgfqpoint{6.200000in}{4.620000in}}%
\pgfusepath{clip}%
\pgfsetbuttcap%
\pgfsetroundjoin%
\pgfsetlinewidth{1.003750pt}%
\definecolor{currentstroke}{rgb}{1.000000,0.000000,0.000000}%
\pgfsetstrokecolor{currentstroke}%
\pgfsetdash{}{0pt}%
\pgfpathmoveto{\pgfqpoint{3.839157in}{0.838036in}}%
\pgfpathcurveto{\pgfqpoint{3.850207in}{0.838036in}}{\pgfqpoint{3.860806in}{0.842426in}}{\pgfqpoint{3.868620in}{0.850240in}}%
\pgfpathcurveto{\pgfqpoint{3.876433in}{0.858053in}}{\pgfqpoint{3.880824in}{0.868652in}}{\pgfqpoint{3.880824in}{0.879702in}}%
\pgfpathcurveto{\pgfqpoint{3.880824in}{0.890753in}}{\pgfqpoint{3.876433in}{0.901352in}}{\pgfqpoint{3.868620in}{0.909165in}}%
\pgfpathcurveto{\pgfqpoint{3.860806in}{0.916979in}}{\pgfqpoint{3.850207in}{0.921369in}}{\pgfqpoint{3.839157in}{0.921369in}}%
\pgfpathcurveto{\pgfqpoint{3.828107in}{0.921369in}}{\pgfqpoint{3.817508in}{0.916979in}}{\pgfqpoint{3.809694in}{0.909165in}}%
\pgfpathcurveto{\pgfqpoint{3.801881in}{0.901352in}}{\pgfqpoint{3.797490in}{0.890753in}}{\pgfqpoint{3.797490in}{0.879702in}}%
\pgfpathcurveto{\pgfqpoint{3.797490in}{0.868652in}}{\pgfqpoint{3.801881in}{0.858053in}}{\pgfqpoint{3.809694in}{0.850240in}}%
\pgfpathcurveto{\pgfqpoint{3.817508in}{0.842426in}}{\pgfqpoint{3.828107in}{0.838036in}}{\pgfqpoint{3.839157in}{0.838036in}}%
\pgfpathlineto{\pgfqpoint{3.839157in}{0.838036in}}%
\pgfpathclose%
\pgfusepath{stroke}%
\end{pgfscope}%
\begin{pgfscope}%
\pgfpathrectangle{\pgfqpoint{0.847223in}{0.554012in}}{\pgfqpoint{6.200000in}{4.620000in}}%
\pgfusepath{clip}%
\pgfsetbuttcap%
\pgfsetroundjoin%
\pgfsetlinewidth{1.003750pt}%
\definecolor{currentstroke}{rgb}{1.000000,0.000000,0.000000}%
\pgfsetstrokecolor{currentstroke}%
\pgfsetdash{}{0pt}%
\pgfpathmoveto{\pgfqpoint{3.844490in}{0.836898in}}%
\pgfpathcurveto{\pgfqpoint{3.855540in}{0.836898in}}{\pgfqpoint{3.866139in}{0.841288in}}{\pgfqpoint{3.873953in}{0.849102in}}%
\pgfpathcurveto{\pgfqpoint{3.881767in}{0.856915in}}{\pgfqpoint{3.886157in}{0.867514in}}{\pgfqpoint{3.886157in}{0.878565in}}%
\pgfpathcurveto{\pgfqpoint{3.886157in}{0.889615in}}{\pgfqpoint{3.881767in}{0.900214in}}{\pgfqpoint{3.873953in}{0.908027in}}%
\pgfpathcurveto{\pgfqpoint{3.866139in}{0.915841in}}{\pgfqpoint{3.855540in}{0.920231in}}{\pgfqpoint{3.844490in}{0.920231in}}%
\pgfpathcurveto{\pgfqpoint{3.833440in}{0.920231in}}{\pgfqpoint{3.822841in}{0.915841in}}{\pgfqpoint{3.815027in}{0.908027in}}%
\pgfpathcurveto{\pgfqpoint{3.807214in}{0.900214in}}{\pgfqpoint{3.802824in}{0.889615in}}{\pgfqpoint{3.802824in}{0.878565in}}%
\pgfpathcurveto{\pgfqpoint{3.802824in}{0.867514in}}{\pgfqpoint{3.807214in}{0.856915in}}{\pgfqpoint{3.815027in}{0.849102in}}%
\pgfpathcurveto{\pgfqpoint{3.822841in}{0.841288in}}{\pgfqpoint{3.833440in}{0.836898in}}{\pgfqpoint{3.844490in}{0.836898in}}%
\pgfpathlineto{\pgfqpoint{3.844490in}{0.836898in}}%
\pgfpathclose%
\pgfusepath{stroke}%
\end{pgfscope}%
\begin{pgfscope}%
\pgfpathrectangle{\pgfqpoint{0.847223in}{0.554012in}}{\pgfqpoint{6.200000in}{4.620000in}}%
\pgfusepath{clip}%
\pgfsetbuttcap%
\pgfsetroundjoin%
\pgfsetlinewidth{1.003750pt}%
\definecolor{currentstroke}{rgb}{1.000000,0.000000,0.000000}%
\pgfsetstrokecolor{currentstroke}%
\pgfsetdash{}{0pt}%
\pgfpathmoveto{\pgfqpoint{3.849823in}{0.835763in}}%
\pgfpathcurveto{\pgfqpoint{3.860874in}{0.835763in}}{\pgfqpoint{3.871473in}{0.840154in}}{\pgfqpoint{3.879286in}{0.847967in}}%
\pgfpathcurveto{\pgfqpoint{3.887100in}{0.855781in}}{\pgfqpoint{3.891490in}{0.866380in}}{\pgfqpoint{3.891490in}{0.877430in}}%
\pgfpathcurveto{\pgfqpoint{3.891490in}{0.888480in}}{\pgfqpoint{3.887100in}{0.899079in}}{\pgfqpoint{3.879286in}{0.906893in}}%
\pgfpathcurveto{\pgfqpoint{3.871473in}{0.914706in}}{\pgfqpoint{3.860874in}{0.919097in}}{\pgfqpoint{3.849823in}{0.919097in}}%
\pgfpathcurveto{\pgfqpoint{3.838773in}{0.919097in}}{\pgfqpoint{3.828174in}{0.914706in}}{\pgfqpoint{3.820361in}{0.906893in}}%
\pgfpathcurveto{\pgfqpoint{3.812547in}{0.899079in}}{\pgfqpoint{3.808157in}{0.888480in}}{\pgfqpoint{3.808157in}{0.877430in}}%
\pgfpathcurveto{\pgfqpoint{3.808157in}{0.866380in}}{\pgfqpoint{3.812547in}{0.855781in}}{\pgfqpoint{3.820361in}{0.847967in}}%
\pgfpathcurveto{\pgfqpoint{3.828174in}{0.840154in}}{\pgfqpoint{3.838773in}{0.835763in}}{\pgfqpoint{3.849823in}{0.835763in}}%
\pgfpathlineto{\pgfqpoint{3.849823in}{0.835763in}}%
\pgfpathclose%
\pgfusepath{stroke}%
\end{pgfscope}%
\begin{pgfscope}%
\pgfpathrectangle{\pgfqpoint{0.847223in}{0.554012in}}{\pgfqpoint{6.200000in}{4.620000in}}%
\pgfusepath{clip}%
\pgfsetbuttcap%
\pgfsetroundjoin%
\pgfsetlinewidth{1.003750pt}%
\definecolor{currentstroke}{rgb}{1.000000,0.000000,0.000000}%
\pgfsetstrokecolor{currentstroke}%
\pgfsetdash{}{0pt}%
\pgfpathmoveto{\pgfqpoint{3.855157in}{0.834632in}}%
\pgfpathcurveto{\pgfqpoint{3.866207in}{0.834632in}}{\pgfqpoint{3.876806in}{0.839023in}}{\pgfqpoint{3.884619in}{0.846836in}}%
\pgfpathcurveto{\pgfqpoint{3.892433in}{0.854650in}}{\pgfqpoint{3.896823in}{0.865249in}}{\pgfqpoint{3.896823in}{0.876299in}}%
\pgfpathcurveto{\pgfqpoint{3.896823in}{0.887349in}}{\pgfqpoint{3.892433in}{0.897948in}}{\pgfqpoint{3.884619in}{0.905762in}}%
\pgfpathcurveto{\pgfqpoint{3.876806in}{0.913575in}}{\pgfqpoint{3.866207in}{0.917966in}}{\pgfqpoint{3.855157in}{0.917966in}}%
\pgfpathcurveto{\pgfqpoint{3.844107in}{0.917966in}}{\pgfqpoint{3.833508in}{0.913575in}}{\pgfqpoint{3.825694in}{0.905762in}}%
\pgfpathcurveto{\pgfqpoint{3.817880in}{0.897948in}}{\pgfqpoint{3.813490in}{0.887349in}}{\pgfqpoint{3.813490in}{0.876299in}}%
\pgfpathcurveto{\pgfqpoint{3.813490in}{0.865249in}}{\pgfqpoint{3.817880in}{0.854650in}}{\pgfqpoint{3.825694in}{0.846836in}}%
\pgfpathcurveto{\pgfqpoint{3.833508in}{0.839023in}}{\pgfqpoint{3.844107in}{0.834632in}}{\pgfqpoint{3.855157in}{0.834632in}}%
\pgfpathlineto{\pgfqpoint{3.855157in}{0.834632in}}%
\pgfpathclose%
\pgfusepath{stroke}%
\end{pgfscope}%
\begin{pgfscope}%
\pgfpathrectangle{\pgfqpoint{0.847223in}{0.554012in}}{\pgfqpoint{6.200000in}{4.620000in}}%
\pgfusepath{clip}%
\pgfsetbuttcap%
\pgfsetroundjoin%
\pgfsetlinewidth{1.003750pt}%
\definecolor{currentstroke}{rgb}{1.000000,0.000000,0.000000}%
\pgfsetstrokecolor{currentstroke}%
\pgfsetdash{}{0pt}%
\pgfpathmoveto{\pgfqpoint{3.860490in}{0.833505in}}%
\pgfpathcurveto{\pgfqpoint{3.871540in}{0.833505in}}{\pgfqpoint{3.882139in}{0.837895in}}{\pgfqpoint{3.889953in}{0.845709in}}%
\pgfpathcurveto{\pgfqpoint{3.897766in}{0.853522in}}{\pgfqpoint{3.902157in}{0.864121in}}{\pgfqpoint{3.902157in}{0.875171in}}%
\pgfpathcurveto{\pgfqpoint{3.902157in}{0.886222in}}{\pgfqpoint{3.897766in}{0.896821in}}{\pgfqpoint{3.889953in}{0.904634in}}%
\pgfpathcurveto{\pgfqpoint{3.882139in}{0.912448in}}{\pgfqpoint{3.871540in}{0.916838in}}{\pgfqpoint{3.860490in}{0.916838in}}%
\pgfpathcurveto{\pgfqpoint{3.849440in}{0.916838in}}{\pgfqpoint{3.838841in}{0.912448in}}{\pgfqpoint{3.831027in}{0.904634in}}%
\pgfpathcurveto{\pgfqpoint{3.823213in}{0.896821in}}{\pgfqpoint{3.818823in}{0.886222in}}{\pgfqpoint{3.818823in}{0.875171in}}%
\pgfpathcurveto{\pgfqpoint{3.818823in}{0.864121in}}{\pgfqpoint{3.823213in}{0.853522in}}{\pgfqpoint{3.831027in}{0.845709in}}%
\pgfpathcurveto{\pgfqpoint{3.838841in}{0.837895in}}{\pgfqpoint{3.849440in}{0.833505in}}{\pgfqpoint{3.860490in}{0.833505in}}%
\pgfpathlineto{\pgfqpoint{3.860490in}{0.833505in}}%
\pgfpathclose%
\pgfusepath{stroke}%
\end{pgfscope}%
\begin{pgfscope}%
\pgfpathrectangle{\pgfqpoint{0.847223in}{0.554012in}}{\pgfqpoint{6.200000in}{4.620000in}}%
\pgfusepath{clip}%
\pgfsetbuttcap%
\pgfsetroundjoin%
\pgfsetlinewidth{1.003750pt}%
\definecolor{currentstroke}{rgb}{1.000000,0.000000,0.000000}%
\pgfsetstrokecolor{currentstroke}%
\pgfsetdash{}{0pt}%
\pgfpathmoveto{\pgfqpoint{3.865823in}{0.832381in}}%
\pgfpathcurveto{\pgfqpoint{3.876873in}{0.832381in}}{\pgfqpoint{3.887472in}{0.836771in}}{\pgfqpoint{3.895286in}{0.844584in}}%
\pgfpathcurveto{\pgfqpoint{3.903100in}{0.852398in}}{\pgfqpoint{3.907490in}{0.862997in}}{\pgfqpoint{3.907490in}{0.874047in}}%
\pgfpathcurveto{\pgfqpoint{3.907490in}{0.885097in}}{\pgfqpoint{3.903100in}{0.895696in}}{\pgfqpoint{3.895286in}{0.903510in}}%
\pgfpathcurveto{\pgfqpoint{3.887472in}{0.911324in}}{\pgfqpoint{3.876873in}{0.915714in}}{\pgfqpoint{3.865823in}{0.915714in}}%
\pgfpathcurveto{\pgfqpoint{3.854773in}{0.915714in}}{\pgfqpoint{3.844174in}{0.911324in}}{\pgfqpoint{3.836360in}{0.903510in}}%
\pgfpathcurveto{\pgfqpoint{3.828547in}{0.895696in}}{\pgfqpoint{3.824156in}{0.885097in}}{\pgfqpoint{3.824156in}{0.874047in}}%
\pgfpathcurveto{\pgfqpoint{3.824156in}{0.862997in}}{\pgfqpoint{3.828547in}{0.852398in}}{\pgfqpoint{3.836360in}{0.844584in}}%
\pgfpathcurveto{\pgfqpoint{3.844174in}{0.836771in}}{\pgfqpoint{3.854773in}{0.832381in}}{\pgfqpoint{3.865823in}{0.832381in}}%
\pgfpathlineto{\pgfqpoint{3.865823in}{0.832381in}}%
\pgfpathclose%
\pgfusepath{stroke}%
\end{pgfscope}%
\begin{pgfscope}%
\pgfpathrectangle{\pgfqpoint{0.847223in}{0.554012in}}{\pgfqpoint{6.200000in}{4.620000in}}%
\pgfusepath{clip}%
\pgfsetbuttcap%
\pgfsetroundjoin%
\pgfsetlinewidth{1.003750pt}%
\definecolor{currentstroke}{rgb}{1.000000,0.000000,0.000000}%
\pgfsetstrokecolor{currentstroke}%
\pgfsetdash{}{0pt}%
\pgfpathmoveto{\pgfqpoint{3.871156in}{0.831260in}}%
\pgfpathcurveto{\pgfqpoint{3.882206in}{0.831260in}}{\pgfqpoint{3.892805in}{0.835650in}}{\pgfqpoint{3.900619in}{0.843464in}}%
\pgfpathcurveto{\pgfqpoint{3.908433in}{0.851277in}}{\pgfqpoint{3.912823in}{0.861876in}}{\pgfqpoint{3.912823in}{0.872926in}}%
\pgfpathcurveto{\pgfqpoint{3.912823in}{0.883976in}}{\pgfqpoint{3.908433in}{0.894575in}}{\pgfqpoint{3.900619in}{0.902389in}}%
\pgfpathcurveto{\pgfqpoint{3.892805in}{0.910203in}}{\pgfqpoint{3.882206in}{0.914593in}}{\pgfqpoint{3.871156in}{0.914593in}}%
\pgfpathcurveto{\pgfqpoint{3.860106in}{0.914593in}}{\pgfqpoint{3.849507in}{0.910203in}}{\pgfqpoint{3.841694in}{0.902389in}}%
\pgfpathcurveto{\pgfqpoint{3.833880in}{0.894575in}}{\pgfqpoint{3.829490in}{0.883976in}}{\pgfqpoint{3.829490in}{0.872926in}}%
\pgfpathcurveto{\pgfqpoint{3.829490in}{0.861876in}}{\pgfqpoint{3.833880in}{0.851277in}}{\pgfqpoint{3.841694in}{0.843464in}}%
\pgfpathcurveto{\pgfqpoint{3.849507in}{0.835650in}}{\pgfqpoint{3.860106in}{0.831260in}}{\pgfqpoint{3.871156in}{0.831260in}}%
\pgfpathlineto{\pgfqpoint{3.871156in}{0.831260in}}%
\pgfpathclose%
\pgfusepath{stroke}%
\end{pgfscope}%
\begin{pgfscope}%
\pgfpathrectangle{\pgfqpoint{0.847223in}{0.554012in}}{\pgfqpoint{6.200000in}{4.620000in}}%
\pgfusepath{clip}%
\pgfsetbuttcap%
\pgfsetroundjoin%
\pgfsetlinewidth{1.003750pt}%
\definecolor{currentstroke}{rgb}{1.000000,0.000000,0.000000}%
\pgfsetstrokecolor{currentstroke}%
\pgfsetdash{}{0pt}%
\pgfpathmoveto{\pgfqpoint{3.876490in}{0.830142in}}%
\pgfpathcurveto{\pgfqpoint{3.887540in}{0.830142in}}{\pgfqpoint{3.898139in}{0.834532in}}{\pgfqpoint{3.905952in}{0.842346in}}%
\pgfpathcurveto{\pgfqpoint{3.913766in}{0.850160in}}{\pgfqpoint{3.918156in}{0.860759in}}{\pgfqpoint{3.918156in}{0.871809in}}%
\pgfpathcurveto{\pgfqpoint{3.918156in}{0.882859in}}{\pgfqpoint{3.913766in}{0.893458in}}{\pgfqpoint{3.905952in}{0.901272in}}%
\pgfpathcurveto{\pgfqpoint{3.898139in}{0.909085in}}{\pgfqpoint{3.887540in}{0.913476in}}{\pgfqpoint{3.876490in}{0.913476in}}%
\pgfpathcurveto{\pgfqpoint{3.865439in}{0.913476in}}{\pgfqpoint{3.854840in}{0.909085in}}{\pgfqpoint{3.847027in}{0.901272in}}%
\pgfpathcurveto{\pgfqpoint{3.839213in}{0.893458in}}{\pgfqpoint{3.834823in}{0.882859in}}{\pgfqpoint{3.834823in}{0.871809in}}%
\pgfpathcurveto{\pgfqpoint{3.834823in}{0.860759in}}{\pgfqpoint{3.839213in}{0.850160in}}{\pgfqpoint{3.847027in}{0.842346in}}%
\pgfpathcurveto{\pgfqpoint{3.854840in}{0.834532in}}{\pgfqpoint{3.865439in}{0.830142in}}{\pgfqpoint{3.876490in}{0.830142in}}%
\pgfpathlineto{\pgfqpoint{3.876490in}{0.830142in}}%
\pgfpathclose%
\pgfusepath{stroke}%
\end{pgfscope}%
\begin{pgfscope}%
\pgfpathrectangle{\pgfqpoint{0.847223in}{0.554012in}}{\pgfqpoint{6.200000in}{4.620000in}}%
\pgfusepath{clip}%
\pgfsetbuttcap%
\pgfsetroundjoin%
\pgfsetlinewidth{1.003750pt}%
\definecolor{currentstroke}{rgb}{1.000000,0.000000,0.000000}%
\pgfsetstrokecolor{currentstroke}%
\pgfsetdash{}{0pt}%
\pgfpathmoveto{\pgfqpoint{3.881823in}{0.829028in}}%
\pgfpathcurveto{\pgfqpoint{3.892873in}{0.829028in}}{\pgfqpoint{3.903472in}{0.833418in}}{\pgfqpoint{3.911286in}{0.841232in}}%
\pgfpathcurveto{\pgfqpoint{3.919099in}{0.849046in}}{\pgfqpoint{3.923489in}{0.859645in}}{\pgfqpoint{3.923489in}{0.870695in}}%
\pgfpathcurveto{\pgfqpoint{3.923489in}{0.881745in}}{\pgfqpoint{3.919099in}{0.892344in}}{\pgfqpoint{3.911286in}{0.900158in}}%
\pgfpathcurveto{\pgfqpoint{3.903472in}{0.907971in}}{\pgfqpoint{3.892873in}{0.912361in}}{\pgfqpoint{3.881823in}{0.912361in}}%
\pgfpathcurveto{\pgfqpoint{3.870773in}{0.912361in}}{\pgfqpoint{3.860174in}{0.907971in}}{\pgfqpoint{3.852360in}{0.900158in}}%
\pgfpathcurveto{\pgfqpoint{3.844546in}{0.892344in}}{\pgfqpoint{3.840156in}{0.881745in}}{\pgfqpoint{3.840156in}{0.870695in}}%
\pgfpathcurveto{\pgfqpoint{3.840156in}{0.859645in}}{\pgfqpoint{3.844546in}{0.849046in}}{\pgfqpoint{3.852360in}{0.841232in}}%
\pgfpathcurveto{\pgfqpoint{3.860174in}{0.833418in}}{\pgfqpoint{3.870773in}{0.829028in}}{\pgfqpoint{3.881823in}{0.829028in}}%
\pgfpathlineto{\pgfqpoint{3.881823in}{0.829028in}}%
\pgfpathclose%
\pgfusepath{stroke}%
\end{pgfscope}%
\begin{pgfscope}%
\pgfpathrectangle{\pgfqpoint{0.847223in}{0.554012in}}{\pgfqpoint{6.200000in}{4.620000in}}%
\pgfusepath{clip}%
\pgfsetbuttcap%
\pgfsetroundjoin%
\pgfsetlinewidth{1.003750pt}%
\definecolor{currentstroke}{rgb}{1.000000,0.000000,0.000000}%
\pgfsetstrokecolor{currentstroke}%
\pgfsetdash{}{0pt}%
\pgfpathmoveto{\pgfqpoint{3.887156in}{0.827917in}}%
\pgfpathcurveto{\pgfqpoint{3.898206in}{0.827917in}}{\pgfqpoint{3.908805in}{0.832308in}}{\pgfqpoint{3.916619in}{0.840121in}}%
\pgfpathcurveto{\pgfqpoint{3.924432in}{0.847935in}}{\pgfqpoint{3.928823in}{0.858534in}}{\pgfqpoint{3.928823in}{0.869584in}}%
\pgfpathcurveto{\pgfqpoint{3.928823in}{0.880634in}}{\pgfqpoint{3.924432in}{0.891233in}}{\pgfqpoint{3.916619in}{0.899047in}}%
\pgfpathcurveto{\pgfqpoint{3.908805in}{0.906860in}}{\pgfqpoint{3.898206in}{0.911251in}}{\pgfqpoint{3.887156in}{0.911251in}}%
\pgfpathcurveto{\pgfqpoint{3.876106in}{0.911251in}}{\pgfqpoint{3.865507in}{0.906860in}}{\pgfqpoint{3.857693in}{0.899047in}}%
\pgfpathcurveto{\pgfqpoint{3.849880in}{0.891233in}}{\pgfqpoint{3.845489in}{0.880634in}}{\pgfqpoint{3.845489in}{0.869584in}}%
\pgfpathcurveto{\pgfqpoint{3.845489in}{0.858534in}}{\pgfqpoint{3.849880in}{0.847935in}}{\pgfqpoint{3.857693in}{0.840121in}}%
\pgfpathcurveto{\pgfqpoint{3.865507in}{0.832308in}}{\pgfqpoint{3.876106in}{0.827917in}}{\pgfqpoint{3.887156in}{0.827917in}}%
\pgfpathlineto{\pgfqpoint{3.887156in}{0.827917in}}%
\pgfpathclose%
\pgfusepath{stroke}%
\end{pgfscope}%
\begin{pgfscope}%
\pgfpathrectangle{\pgfqpoint{0.847223in}{0.554012in}}{\pgfqpoint{6.200000in}{4.620000in}}%
\pgfusepath{clip}%
\pgfsetbuttcap%
\pgfsetroundjoin%
\pgfsetlinewidth{1.003750pt}%
\definecolor{currentstroke}{rgb}{1.000000,0.000000,0.000000}%
\pgfsetstrokecolor{currentstroke}%
\pgfsetdash{}{0pt}%
\pgfpathmoveto{\pgfqpoint{3.892489in}{0.826810in}}%
\pgfpathcurveto{\pgfqpoint{3.903539in}{0.826810in}}{\pgfqpoint{3.914138in}{0.831200in}}{\pgfqpoint{3.921952in}{0.839014in}}%
\pgfpathcurveto{\pgfqpoint{3.929766in}{0.846827in}}{\pgfqpoint{3.934156in}{0.857426in}}{\pgfqpoint{3.934156in}{0.868476in}}%
\pgfpathcurveto{\pgfqpoint{3.934156in}{0.879527in}}{\pgfqpoint{3.929766in}{0.890126in}}{\pgfqpoint{3.921952in}{0.897939in}}%
\pgfpathcurveto{\pgfqpoint{3.914138in}{0.905753in}}{\pgfqpoint{3.903539in}{0.910143in}}{\pgfqpoint{3.892489in}{0.910143in}}%
\pgfpathcurveto{\pgfqpoint{3.881439in}{0.910143in}}{\pgfqpoint{3.870840in}{0.905753in}}{\pgfqpoint{3.863026in}{0.897939in}}%
\pgfpathcurveto{\pgfqpoint{3.855213in}{0.890126in}}{\pgfqpoint{3.850823in}{0.879527in}}{\pgfqpoint{3.850823in}{0.868476in}}%
\pgfpathcurveto{\pgfqpoint{3.850823in}{0.857426in}}{\pgfqpoint{3.855213in}{0.846827in}}{\pgfqpoint{3.863026in}{0.839014in}}%
\pgfpathcurveto{\pgfqpoint{3.870840in}{0.831200in}}{\pgfqpoint{3.881439in}{0.826810in}}{\pgfqpoint{3.892489in}{0.826810in}}%
\pgfpathlineto{\pgfqpoint{3.892489in}{0.826810in}}%
\pgfpathclose%
\pgfusepath{stroke}%
\end{pgfscope}%
\begin{pgfscope}%
\pgfpathrectangle{\pgfqpoint{0.847223in}{0.554012in}}{\pgfqpoint{6.200000in}{4.620000in}}%
\pgfusepath{clip}%
\pgfsetbuttcap%
\pgfsetroundjoin%
\pgfsetlinewidth{1.003750pt}%
\definecolor{currentstroke}{rgb}{1.000000,0.000000,0.000000}%
\pgfsetstrokecolor{currentstroke}%
\pgfsetdash{}{0pt}%
\pgfpathmoveto{\pgfqpoint{3.897822in}{0.825706in}}%
\pgfpathcurveto{\pgfqpoint{3.908873in}{0.825706in}}{\pgfqpoint{3.919472in}{0.830096in}}{\pgfqpoint{3.927285in}{0.837910in}}%
\pgfpathcurveto{\pgfqpoint{3.935099in}{0.845723in}}{\pgfqpoint{3.939489in}{0.856322in}}{\pgfqpoint{3.939489in}{0.867372in}}%
\pgfpathcurveto{\pgfqpoint{3.939489in}{0.878422in}}{\pgfqpoint{3.935099in}{0.889021in}}{\pgfqpoint{3.927285in}{0.896835in}}%
\pgfpathcurveto{\pgfqpoint{3.919472in}{0.904649in}}{\pgfqpoint{3.908873in}{0.909039in}}{\pgfqpoint{3.897822in}{0.909039in}}%
\pgfpathcurveto{\pgfqpoint{3.886772in}{0.909039in}}{\pgfqpoint{3.876173in}{0.904649in}}{\pgfqpoint{3.868360in}{0.896835in}}%
\pgfpathcurveto{\pgfqpoint{3.860546in}{0.889021in}}{\pgfqpoint{3.856156in}{0.878422in}}{\pgfqpoint{3.856156in}{0.867372in}}%
\pgfpathcurveto{\pgfqpoint{3.856156in}{0.856322in}}{\pgfqpoint{3.860546in}{0.845723in}}{\pgfqpoint{3.868360in}{0.837910in}}%
\pgfpathcurveto{\pgfqpoint{3.876173in}{0.830096in}}{\pgfqpoint{3.886772in}{0.825706in}}{\pgfqpoint{3.897822in}{0.825706in}}%
\pgfpathlineto{\pgfqpoint{3.897822in}{0.825706in}}%
\pgfpathclose%
\pgfusepath{stroke}%
\end{pgfscope}%
\begin{pgfscope}%
\pgfpathrectangle{\pgfqpoint{0.847223in}{0.554012in}}{\pgfqpoint{6.200000in}{4.620000in}}%
\pgfusepath{clip}%
\pgfsetbuttcap%
\pgfsetroundjoin%
\pgfsetlinewidth{1.003750pt}%
\definecolor{currentstroke}{rgb}{1.000000,0.000000,0.000000}%
\pgfsetstrokecolor{currentstroke}%
\pgfsetdash{}{0pt}%
\pgfpathmoveto{\pgfqpoint{3.903156in}{0.824605in}}%
\pgfpathcurveto{\pgfqpoint{3.914206in}{0.824605in}}{\pgfqpoint{3.924805in}{0.828995in}}{\pgfqpoint{3.932618in}{0.836809in}}%
\pgfpathcurveto{\pgfqpoint{3.940432in}{0.844622in}}{\pgfqpoint{3.944822in}{0.855221in}}{\pgfqpoint{3.944822in}{0.866271in}}%
\pgfpathcurveto{\pgfqpoint{3.944822in}{0.877322in}}{\pgfqpoint{3.940432in}{0.887921in}}{\pgfqpoint{3.932618in}{0.895734in}}%
\pgfpathcurveto{\pgfqpoint{3.924805in}{0.903548in}}{\pgfqpoint{3.914206in}{0.907938in}}{\pgfqpoint{3.903156in}{0.907938in}}%
\pgfpathcurveto{\pgfqpoint{3.892105in}{0.907938in}}{\pgfqpoint{3.881506in}{0.903548in}}{\pgfqpoint{3.873693in}{0.895734in}}%
\pgfpathcurveto{\pgfqpoint{3.865879in}{0.887921in}}{\pgfqpoint{3.861489in}{0.877322in}}{\pgfqpoint{3.861489in}{0.866271in}}%
\pgfpathcurveto{\pgfqpoint{3.861489in}{0.855221in}}{\pgfqpoint{3.865879in}{0.844622in}}{\pgfqpoint{3.873693in}{0.836809in}}%
\pgfpathcurveto{\pgfqpoint{3.881506in}{0.828995in}}{\pgfqpoint{3.892105in}{0.824605in}}{\pgfqpoint{3.903156in}{0.824605in}}%
\pgfpathlineto{\pgfqpoint{3.903156in}{0.824605in}}%
\pgfpathclose%
\pgfusepath{stroke}%
\end{pgfscope}%
\begin{pgfscope}%
\pgfpathrectangle{\pgfqpoint{0.847223in}{0.554012in}}{\pgfqpoint{6.200000in}{4.620000in}}%
\pgfusepath{clip}%
\pgfsetbuttcap%
\pgfsetroundjoin%
\pgfsetlinewidth{1.003750pt}%
\definecolor{currentstroke}{rgb}{1.000000,0.000000,0.000000}%
\pgfsetstrokecolor{currentstroke}%
\pgfsetdash{}{0pt}%
\pgfpathmoveto{\pgfqpoint{3.908489in}{0.823507in}}%
\pgfpathcurveto{\pgfqpoint{3.919539in}{0.823507in}}{\pgfqpoint{3.930138in}{0.827897in}}{\pgfqpoint{3.937952in}{0.835711in}}%
\pgfpathcurveto{\pgfqpoint{3.945765in}{0.843525in}}{\pgfqpoint{3.950156in}{0.854124in}}{\pgfqpoint{3.950156in}{0.865174in}}%
\pgfpathcurveto{\pgfqpoint{3.950156in}{0.876224in}}{\pgfqpoint{3.945765in}{0.886823in}}{\pgfqpoint{3.937952in}{0.894637in}}%
\pgfpathcurveto{\pgfqpoint{3.930138in}{0.902450in}}{\pgfqpoint{3.919539in}{0.906840in}}{\pgfqpoint{3.908489in}{0.906840in}}%
\pgfpathcurveto{\pgfqpoint{3.897439in}{0.906840in}}{\pgfqpoint{3.886840in}{0.902450in}}{\pgfqpoint{3.879026in}{0.894637in}}%
\pgfpathcurveto{\pgfqpoint{3.871212in}{0.886823in}}{\pgfqpoint{3.866822in}{0.876224in}}{\pgfqpoint{3.866822in}{0.865174in}}%
\pgfpathcurveto{\pgfqpoint{3.866822in}{0.854124in}}{\pgfqpoint{3.871212in}{0.843525in}}{\pgfqpoint{3.879026in}{0.835711in}}%
\pgfpathcurveto{\pgfqpoint{3.886840in}{0.827897in}}{\pgfqpoint{3.897439in}{0.823507in}}{\pgfqpoint{3.908489in}{0.823507in}}%
\pgfpathlineto{\pgfqpoint{3.908489in}{0.823507in}}%
\pgfpathclose%
\pgfusepath{stroke}%
\end{pgfscope}%
\begin{pgfscope}%
\pgfpathrectangle{\pgfqpoint{0.847223in}{0.554012in}}{\pgfqpoint{6.200000in}{4.620000in}}%
\pgfusepath{clip}%
\pgfsetbuttcap%
\pgfsetroundjoin%
\pgfsetlinewidth{1.003750pt}%
\definecolor{currentstroke}{rgb}{1.000000,0.000000,0.000000}%
\pgfsetstrokecolor{currentstroke}%
\pgfsetdash{}{0pt}%
\pgfpathmoveto{\pgfqpoint{3.913822in}{0.822413in}}%
\pgfpathcurveto{\pgfqpoint{3.924872in}{0.822413in}}{\pgfqpoint{3.935471in}{0.826803in}}{\pgfqpoint{3.943285in}{0.834617in}}%
\pgfpathcurveto{\pgfqpoint{3.951098in}{0.842430in}}{\pgfqpoint{3.955489in}{0.853029in}}{\pgfqpoint{3.955489in}{0.864079in}}%
\pgfpathcurveto{\pgfqpoint{3.955489in}{0.875130in}}{\pgfqpoint{3.951098in}{0.885729in}}{\pgfqpoint{3.943285in}{0.893542in}}%
\pgfpathcurveto{\pgfqpoint{3.935471in}{0.901356in}}{\pgfqpoint{3.924872in}{0.905746in}}{\pgfqpoint{3.913822in}{0.905746in}}%
\pgfpathcurveto{\pgfqpoint{3.902772in}{0.905746in}}{\pgfqpoint{3.892173in}{0.901356in}}{\pgfqpoint{3.884359in}{0.893542in}}%
\pgfpathcurveto{\pgfqpoint{3.876546in}{0.885729in}}{\pgfqpoint{3.872155in}{0.875130in}}{\pgfqpoint{3.872155in}{0.864079in}}%
\pgfpathcurveto{\pgfqpoint{3.872155in}{0.853029in}}{\pgfqpoint{3.876546in}{0.842430in}}{\pgfqpoint{3.884359in}{0.834617in}}%
\pgfpathcurveto{\pgfqpoint{3.892173in}{0.826803in}}{\pgfqpoint{3.902772in}{0.822413in}}{\pgfqpoint{3.913822in}{0.822413in}}%
\pgfpathlineto{\pgfqpoint{3.913822in}{0.822413in}}%
\pgfpathclose%
\pgfusepath{stroke}%
\end{pgfscope}%
\begin{pgfscope}%
\pgfpathrectangle{\pgfqpoint{0.847223in}{0.554012in}}{\pgfqpoint{6.200000in}{4.620000in}}%
\pgfusepath{clip}%
\pgfsetbuttcap%
\pgfsetroundjoin%
\pgfsetlinewidth{1.003750pt}%
\definecolor{currentstroke}{rgb}{1.000000,0.000000,0.000000}%
\pgfsetstrokecolor{currentstroke}%
\pgfsetdash{}{0pt}%
\pgfpathmoveto{\pgfqpoint{3.919155in}{0.821322in}}%
\pgfpathcurveto{\pgfqpoint{3.930205in}{0.821322in}}{\pgfqpoint{3.940804in}{0.825712in}}{\pgfqpoint{3.948618in}{0.833526in}}%
\pgfpathcurveto{\pgfqpoint{3.956432in}{0.841339in}}{\pgfqpoint{3.960822in}{0.851938in}}{\pgfqpoint{3.960822in}{0.862988in}}%
\pgfpathcurveto{\pgfqpoint{3.960822in}{0.874039in}}{\pgfqpoint{3.956432in}{0.884638in}}{\pgfqpoint{3.948618in}{0.892451in}}%
\pgfpathcurveto{\pgfqpoint{3.940804in}{0.900265in}}{\pgfqpoint{3.930205in}{0.904655in}}{\pgfqpoint{3.919155in}{0.904655in}}%
\pgfpathcurveto{\pgfqpoint{3.908105in}{0.904655in}}{\pgfqpoint{3.897506in}{0.900265in}}{\pgfqpoint{3.889692in}{0.892451in}}%
\pgfpathcurveto{\pgfqpoint{3.881879in}{0.884638in}}{\pgfqpoint{3.877489in}{0.874039in}}{\pgfqpoint{3.877489in}{0.862988in}}%
\pgfpathcurveto{\pgfqpoint{3.877489in}{0.851938in}}{\pgfqpoint{3.881879in}{0.841339in}}{\pgfqpoint{3.889692in}{0.833526in}}%
\pgfpathcurveto{\pgfqpoint{3.897506in}{0.825712in}}{\pgfqpoint{3.908105in}{0.821322in}}{\pgfqpoint{3.919155in}{0.821322in}}%
\pgfpathlineto{\pgfqpoint{3.919155in}{0.821322in}}%
\pgfpathclose%
\pgfusepath{stroke}%
\end{pgfscope}%
\begin{pgfscope}%
\pgfpathrectangle{\pgfqpoint{0.847223in}{0.554012in}}{\pgfqpoint{6.200000in}{4.620000in}}%
\pgfusepath{clip}%
\pgfsetbuttcap%
\pgfsetroundjoin%
\pgfsetlinewidth{1.003750pt}%
\definecolor{currentstroke}{rgb}{1.000000,0.000000,0.000000}%
\pgfsetstrokecolor{currentstroke}%
\pgfsetdash{}{0pt}%
\pgfpathmoveto{\pgfqpoint{3.924488in}{0.820234in}}%
\pgfpathcurveto{\pgfqpoint{3.935539in}{0.820234in}}{\pgfqpoint{3.946138in}{0.824624in}}{\pgfqpoint{3.953951in}{0.832438in}}%
\pgfpathcurveto{\pgfqpoint{3.961765in}{0.840251in}}{\pgfqpoint{3.966155in}{0.850850in}}{\pgfqpoint{3.966155in}{0.861901in}}%
\pgfpathcurveto{\pgfqpoint{3.966155in}{0.872951in}}{\pgfqpoint{3.961765in}{0.883550in}}{\pgfqpoint{3.953951in}{0.891363in}}%
\pgfpathcurveto{\pgfqpoint{3.946138in}{0.899177in}}{\pgfqpoint{3.935539in}{0.903567in}}{\pgfqpoint{3.924488in}{0.903567in}}%
\pgfpathcurveto{\pgfqpoint{3.913438in}{0.903567in}}{\pgfqpoint{3.902839in}{0.899177in}}{\pgfqpoint{3.895026in}{0.891363in}}%
\pgfpathcurveto{\pgfqpoint{3.887212in}{0.883550in}}{\pgfqpoint{3.882822in}{0.872951in}}{\pgfqpoint{3.882822in}{0.861901in}}%
\pgfpathcurveto{\pgfqpoint{3.882822in}{0.850850in}}{\pgfqpoint{3.887212in}{0.840251in}}{\pgfqpoint{3.895026in}{0.832438in}}%
\pgfpathcurveto{\pgfqpoint{3.902839in}{0.824624in}}{\pgfqpoint{3.913438in}{0.820234in}}{\pgfqpoint{3.924488in}{0.820234in}}%
\pgfpathlineto{\pgfqpoint{3.924488in}{0.820234in}}%
\pgfpathclose%
\pgfusepath{stroke}%
\end{pgfscope}%
\begin{pgfscope}%
\pgfpathrectangle{\pgfqpoint{0.847223in}{0.554012in}}{\pgfqpoint{6.200000in}{4.620000in}}%
\pgfusepath{clip}%
\pgfsetbuttcap%
\pgfsetroundjoin%
\pgfsetlinewidth{1.003750pt}%
\definecolor{currentstroke}{rgb}{1.000000,0.000000,0.000000}%
\pgfsetstrokecolor{currentstroke}%
\pgfsetdash{}{0pt}%
\pgfpathmoveto{\pgfqpoint{3.929822in}{0.819149in}}%
\pgfpathcurveto{\pgfqpoint{3.940872in}{0.819149in}}{\pgfqpoint{3.951471in}{0.823539in}}{\pgfqpoint{3.959284in}{0.831353in}}%
\pgfpathcurveto{\pgfqpoint{3.967098in}{0.839167in}}{\pgfqpoint{3.971488in}{0.849766in}}{\pgfqpoint{3.971488in}{0.860816in}}%
\pgfpathcurveto{\pgfqpoint{3.971488in}{0.871866in}}{\pgfqpoint{3.967098in}{0.882465in}}{\pgfqpoint{3.959284in}{0.890279in}}%
\pgfpathcurveto{\pgfqpoint{3.951471in}{0.898092in}}{\pgfqpoint{3.940872in}{0.902483in}}{\pgfqpoint{3.929822in}{0.902483in}}%
\pgfpathcurveto{\pgfqpoint{3.918772in}{0.902483in}}{\pgfqpoint{3.908173in}{0.898092in}}{\pgfqpoint{3.900359in}{0.890279in}}%
\pgfpathcurveto{\pgfqpoint{3.892545in}{0.882465in}}{\pgfqpoint{3.888155in}{0.871866in}}{\pgfqpoint{3.888155in}{0.860816in}}%
\pgfpathcurveto{\pgfqpoint{3.888155in}{0.849766in}}{\pgfqpoint{3.892545in}{0.839167in}}{\pgfqpoint{3.900359in}{0.831353in}}%
\pgfpathcurveto{\pgfqpoint{3.908173in}{0.823539in}}{\pgfqpoint{3.918772in}{0.819149in}}{\pgfqpoint{3.929822in}{0.819149in}}%
\pgfpathlineto{\pgfqpoint{3.929822in}{0.819149in}}%
\pgfpathclose%
\pgfusepath{stroke}%
\end{pgfscope}%
\begin{pgfscope}%
\pgfpathrectangle{\pgfqpoint{0.847223in}{0.554012in}}{\pgfqpoint{6.200000in}{4.620000in}}%
\pgfusepath{clip}%
\pgfsetbuttcap%
\pgfsetroundjoin%
\pgfsetlinewidth{1.003750pt}%
\definecolor{currentstroke}{rgb}{1.000000,0.000000,0.000000}%
\pgfsetstrokecolor{currentstroke}%
\pgfsetdash{}{0pt}%
\pgfpathmoveto{\pgfqpoint{3.935155in}{0.818068in}}%
\pgfpathcurveto{\pgfqpoint{3.946205in}{0.818068in}}{\pgfqpoint{3.956804in}{0.822458in}}{\pgfqpoint{3.964618in}{0.830272in}}%
\pgfpathcurveto{\pgfqpoint{3.972431in}{0.838085in}}{\pgfqpoint{3.976822in}{0.848684in}}{\pgfqpoint{3.976822in}{0.859734in}}%
\pgfpathcurveto{\pgfqpoint{3.976822in}{0.870785in}}{\pgfqpoint{3.972431in}{0.881384in}}{\pgfqpoint{3.964618in}{0.889197in}}%
\pgfpathcurveto{\pgfqpoint{3.956804in}{0.897011in}}{\pgfqpoint{3.946205in}{0.901401in}}{\pgfqpoint{3.935155in}{0.901401in}}%
\pgfpathcurveto{\pgfqpoint{3.924105in}{0.901401in}}{\pgfqpoint{3.913506in}{0.897011in}}{\pgfqpoint{3.905692in}{0.889197in}}%
\pgfpathcurveto{\pgfqpoint{3.897879in}{0.881384in}}{\pgfqpoint{3.893488in}{0.870785in}}{\pgfqpoint{3.893488in}{0.859734in}}%
\pgfpathcurveto{\pgfqpoint{3.893488in}{0.848684in}}{\pgfqpoint{3.897879in}{0.838085in}}{\pgfqpoint{3.905692in}{0.830272in}}%
\pgfpathcurveto{\pgfqpoint{3.913506in}{0.822458in}}{\pgfqpoint{3.924105in}{0.818068in}}{\pgfqpoint{3.935155in}{0.818068in}}%
\pgfpathlineto{\pgfqpoint{3.935155in}{0.818068in}}%
\pgfpathclose%
\pgfusepath{stroke}%
\end{pgfscope}%
\begin{pgfscope}%
\pgfpathrectangle{\pgfqpoint{0.847223in}{0.554012in}}{\pgfqpoint{6.200000in}{4.620000in}}%
\pgfusepath{clip}%
\pgfsetbuttcap%
\pgfsetroundjoin%
\pgfsetlinewidth{1.003750pt}%
\definecolor{currentstroke}{rgb}{1.000000,0.000000,0.000000}%
\pgfsetstrokecolor{currentstroke}%
\pgfsetdash{}{0pt}%
\pgfpathmoveto{\pgfqpoint{3.940488in}{0.816990in}}%
\pgfpathcurveto{\pgfqpoint{3.951538in}{0.816990in}}{\pgfqpoint{3.962137in}{0.821380in}}{\pgfqpoint{3.969951in}{0.829193in}}%
\pgfpathcurveto{\pgfqpoint{3.977765in}{0.837007in}}{\pgfqpoint{3.982155in}{0.847606in}}{\pgfqpoint{3.982155in}{0.858656in}}%
\pgfpathcurveto{\pgfqpoint{3.982155in}{0.869706in}}{\pgfqpoint{3.977765in}{0.880305in}}{\pgfqpoint{3.969951in}{0.888119in}}%
\pgfpathcurveto{\pgfqpoint{3.962137in}{0.895933in}}{\pgfqpoint{3.951538in}{0.900323in}}{\pgfqpoint{3.940488in}{0.900323in}}%
\pgfpathcurveto{\pgfqpoint{3.929438in}{0.900323in}}{\pgfqpoint{3.918839in}{0.895933in}}{\pgfqpoint{3.911025in}{0.888119in}}%
\pgfpathcurveto{\pgfqpoint{3.903212in}{0.880305in}}{\pgfqpoint{3.898821in}{0.869706in}}{\pgfqpoint{3.898821in}{0.858656in}}%
\pgfpathcurveto{\pgfqpoint{3.898821in}{0.847606in}}{\pgfqpoint{3.903212in}{0.837007in}}{\pgfqpoint{3.911025in}{0.829193in}}%
\pgfpathcurveto{\pgfqpoint{3.918839in}{0.821380in}}{\pgfqpoint{3.929438in}{0.816990in}}{\pgfqpoint{3.940488in}{0.816990in}}%
\pgfpathlineto{\pgfqpoint{3.940488in}{0.816990in}}%
\pgfpathclose%
\pgfusepath{stroke}%
\end{pgfscope}%
\begin{pgfscope}%
\pgfpathrectangle{\pgfqpoint{0.847223in}{0.554012in}}{\pgfqpoint{6.200000in}{4.620000in}}%
\pgfusepath{clip}%
\pgfsetbuttcap%
\pgfsetroundjoin%
\pgfsetlinewidth{1.003750pt}%
\definecolor{currentstroke}{rgb}{1.000000,0.000000,0.000000}%
\pgfsetstrokecolor{currentstroke}%
\pgfsetdash{}{0pt}%
\pgfpathmoveto{\pgfqpoint{3.945821in}{0.815914in}}%
\pgfpathcurveto{\pgfqpoint{3.956871in}{0.815914in}}{\pgfqpoint{3.967471in}{0.820305in}}{\pgfqpoint{3.975284in}{0.828118in}}%
\pgfpathcurveto{\pgfqpoint{3.983098in}{0.835932in}}{\pgfqpoint{3.987488in}{0.846531in}}{\pgfqpoint{3.987488in}{0.857581in}}%
\pgfpathcurveto{\pgfqpoint{3.987488in}{0.868631in}}{\pgfqpoint{3.983098in}{0.879230in}}{\pgfqpoint{3.975284in}{0.887044in}}%
\pgfpathcurveto{\pgfqpoint{3.967471in}{0.894858in}}{\pgfqpoint{3.956871in}{0.899248in}}{\pgfqpoint{3.945821in}{0.899248in}}%
\pgfpathcurveto{\pgfqpoint{3.934771in}{0.899248in}}{\pgfqpoint{3.924172in}{0.894858in}}{\pgfqpoint{3.916359in}{0.887044in}}%
\pgfpathcurveto{\pgfqpoint{3.908545in}{0.879230in}}{\pgfqpoint{3.904155in}{0.868631in}}{\pgfqpoint{3.904155in}{0.857581in}}%
\pgfpathcurveto{\pgfqpoint{3.904155in}{0.846531in}}{\pgfqpoint{3.908545in}{0.835932in}}{\pgfqpoint{3.916359in}{0.828118in}}%
\pgfpathcurveto{\pgfqpoint{3.924172in}{0.820305in}}{\pgfqpoint{3.934771in}{0.815914in}}{\pgfqpoint{3.945821in}{0.815914in}}%
\pgfpathlineto{\pgfqpoint{3.945821in}{0.815914in}}%
\pgfpathclose%
\pgfusepath{stroke}%
\end{pgfscope}%
\begin{pgfscope}%
\pgfpathrectangle{\pgfqpoint{0.847223in}{0.554012in}}{\pgfqpoint{6.200000in}{4.620000in}}%
\pgfusepath{clip}%
\pgfsetbuttcap%
\pgfsetroundjoin%
\pgfsetlinewidth{1.003750pt}%
\definecolor{currentstroke}{rgb}{1.000000,0.000000,0.000000}%
\pgfsetstrokecolor{currentstroke}%
\pgfsetdash{}{0pt}%
\pgfpathmoveto{\pgfqpoint{3.951155in}{0.814843in}}%
\pgfpathcurveto{\pgfqpoint{3.962205in}{0.814843in}}{\pgfqpoint{3.972804in}{0.819233in}}{\pgfqpoint{3.980617in}{0.827046in}}%
\pgfpathcurveto{\pgfqpoint{3.988431in}{0.834860in}}{\pgfqpoint{3.992821in}{0.845459in}}{\pgfqpoint{3.992821in}{0.856509in}}%
\pgfpathcurveto{\pgfqpoint{3.992821in}{0.867559in}}{\pgfqpoint{3.988431in}{0.878158in}}{\pgfqpoint{3.980617in}{0.885972in}}%
\pgfpathcurveto{\pgfqpoint{3.972804in}{0.893786in}}{\pgfqpoint{3.962205in}{0.898176in}}{\pgfqpoint{3.951155in}{0.898176in}}%
\pgfpathcurveto{\pgfqpoint{3.940104in}{0.898176in}}{\pgfqpoint{3.929505in}{0.893786in}}{\pgfqpoint{3.921692in}{0.885972in}}%
\pgfpathcurveto{\pgfqpoint{3.913878in}{0.878158in}}{\pgfqpoint{3.909488in}{0.867559in}}{\pgfqpoint{3.909488in}{0.856509in}}%
\pgfpathcurveto{\pgfqpoint{3.909488in}{0.845459in}}{\pgfqpoint{3.913878in}{0.834860in}}{\pgfqpoint{3.921692in}{0.827046in}}%
\pgfpathcurveto{\pgfqpoint{3.929505in}{0.819233in}}{\pgfqpoint{3.940104in}{0.814843in}}{\pgfqpoint{3.951155in}{0.814843in}}%
\pgfpathlineto{\pgfqpoint{3.951155in}{0.814843in}}%
\pgfpathclose%
\pgfusepath{stroke}%
\end{pgfscope}%
\begin{pgfscope}%
\pgfpathrectangle{\pgfqpoint{0.847223in}{0.554012in}}{\pgfqpoint{6.200000in}{4.620000in}}%
\pgfusepath{clip}%
\pgfsetbuttcap%
\pgfsetroundjoin%
\pgfsetlinewidth{1.003750pt}%
\definecolor{currentstroke}{rgb}{1.000000,0.000000,0.000000}%
\pgfsetstrokecolor{currentstroke}%
\pgfsetdash{}{0pt}%
\pgfpathmoveto{\pgfqpoint{3.956488in}{0.813774in}}%
\pgfpathcurveto{\pgfqpoint{3.967538in}{0.813774in}}{\pgfqpoint{3.978137in}{0.818164in}}{\pgfqpoint{3.985951in}{0.825978in}}%
\pgfpathcurveto{\pgfqpoint{3.993764in}{0.833791in}}{\pgfqpoint{3.998154in}{0.844390in}}{\pgfqpoint{3.998154in}{0.855440in}}%
\pgfpathcurveto{\pgfqpoint{3.998154in}{0.866491in}}{\pgfqpoint{3.993764in}{0.877090in}}{\pgfqpoint{3.985951in}{0.884903in}}%
\pgfpathcurveto{\pgfqpoint{3.978137in}{0.892717in}}{\pgfqpoint{3.967538in}{0.897107in}}{\pgfqpoint{3.956488in}{0.897107in}}%
\pgfpathcurveto{\pgfqpoint{3.945438in}{0.897107in}}{\pgfqpoint{3.934839in}{0.892717in}}{\pgfqpoint{3.927025in}{0.884903in}}%
\pgfpathcurveto{\pgfqpoint{3.919211in}{0.877090in}}{\pgfqpoint{3.914821in}{0.866491in}}{\pgfqpoint{3.914821in}{0.855440in}}%
\pgfpathcurveto{\pgfqpoint{3.914821in}{0.844390in}}{\pgfqpoint{3.919211in}{0.833791in}}{\pgfqpoint{3.927025in}{0.825978in}}%
\pgfpathcurveto{\pgfqpoint{3.934839in}{0.818164in}}{\pgfqpoint{3.945438in}{0.813774in}}{\pgfqpoint{3.956488in}{0.813774in}}%
\pgfpathlineto{\pgfqpoint{3.956488in}{0.813774in}}%
\pgfpathclose%
\pgfusepath{stroke}%
\end{pgfscope}%
\begin{pgfscope}%
\pgfpathrectangle{\pgfqpoint{0.847223in}{0.554012in}}{\pgfqpoint{6.200000in}{4.620000in}}%
\pgfusepath{clip}%
\pgfsetbuttcap%
\pgfsetroundjoin%
\pgfsetlinewidth{1.003750pt}%
\definecolor{currentstroke}{rgb}{1.000000,0.000000,0.000000}%
\pgfsetstrokecolor{currentstroke}%
\pgfsetdash{}{0pt}%
\pgfpathmoveto{\pgfqpoint{3.961821in}{0.812708in}}%
\pgfpathcurveto{\pgfqpoint{3.972871in}{0.812708in}}{\pgfqpoint{3.983470in}{0.817098in}}{\pgfqpoint{3.991284in}{0.824912in}}%
\pgfpathcurveto{\pgfqpoint{3.999097in}{0.832726in}}{\pgfqpoint{4.003488in}{0.843325in}}{\pgfqpoint{4.003488in}{0.854375in}}%
\pgfpathcurveto{\pgfqpoint{4.003488in}{0.865425in}}{\pgfqpoint{3.999097in}{0.876024in}}{\pgfqpoint{3.991284in}{0.883838in}}%
\pgfpathcurveto{\pgfqpoint{3.983470in}{0.891651in}}{\pgfqpoint{3.972871in}{0.896041in}}{\pgfqpoint{3.961821in}{0.896041in}}%
\pgfpathcurveto{\pgfqpoint{3.950771in}{0.896041in}}{\pgfqpoint{3.940172in}{0.891651in}}{\pgfqpoint{3.932358in}{0.883838in}}%
\pgfpathcurveto{\pgfqpoint{3.924545in}{0.876024in}}{\pgfqpoint{3.920154in}{0.865425in}}{\pgfqpoint{3.920154in}{0.854375in}}%
\pgfpathcurveto{\pgfqpoint{3.920154in}{0.843325in}}{\pgfqpoint{3.924545in}{0.832726in}}{\pgfqpoint{3.932358in}{0.824912in}}%
\pgfpathcurveto{\pgfqpoint{3.940172in}{0.817098in}}{\pgfqpoint{3.950771in}{0.812708in}}{\pgfqpoint{3.961821in}{0.812708in}}%
\pgfpathlineto{\pgfqpoint{3.961821in}{0.812708in}}%
\pgfpathclose%
\pgfusepath{stroke}%
\end{pgfscope}%
\begin{pgfscope}%
\pgfpathrectangle{\pgfqpoint{0.847223in}{0.554012in}}{\pgfqpoint{6.200000in}{4.620000in}}%
\pgfusepath{clip}%
\pgfsetbuttcap%
\pgfsetroundjoin%
\pgfsetlinewidth{1.003750pt}%
\definecolor{currentstroke}{rgb}{1.000000,0.000000,0.000000}%
\pgfsetstrokecolor{currentstroke}%
\pgfsetdash{}{0pt}%
\pgfpathmoveto{\pgfqpoint{3.967154in}{0.811646in}}%
\pgfpathcurveto{\pgfqpoint{3.978204in}{0.811646in}}{\pgfqpoint{3.988803in}{0.816036in}}{\pgfqpoint{3.996617in}{0.823850in}}%
\pgfpathcurveto{\pgfqpoint{4.004431in}{0.831663in}}{\pgfqpoint{4.008821in}{0.842262in}}{\pgfqpoint{4.008821in}{0.853312in}}%
\pgfpathcurveto{\pgfqpoint{4.008821in}{0.864362in}}{\pgfqpoint{4.004431in}{0.874961in}}{\pgfqpoint{3.996617in}{0.882775in}}%
\pgfpathcurveto{\pgfqpoint{3.988803in}{0.890589in}}{\pgfqpoint{3.978204in}{0.894979in}}{\pgfqpoint{3.967154in}{0.894979in}}%
\pgfpathcurveto{\pgfqpoint{3.956104in}{0.894979in}}{\pgfqpoint{3.945505in}{0.890589in}}{\pgfqpoint{3.937691in}{0.882775in}}%
\pgfpathcurveto{\pgfqpoint{3.929878in}{0.874961in}}{\pgfqpoint{3.925488in}{0.864362in}}{\pgfqpoint{3.925488in}{0.853312in}}%
\pgfpathcurveto{\pgfqpoint{3.925488in}{0.842262in}}{\pgfqpoint{3.929878in}{0.831663in}}{\pgfqpoint{3.937691in}{0.823850in}}%
\pgfpathcurveto{\pgfqpoint{3.945505in}{0.816036in}}{\pgfqpoint{3.956104in}{0.811646in}}{\pgfqpoint{3.967154in}{0.811646in}}%
\pgfpathlineto{\pgfqpoint{3.967154in}{0.811646in}}%
\pgfpathclose%
\pgfusepath{stroke}%
\end{pgfscope}%
\begin{pgfscope}%
\pgfpathrectangle{\pgfqpoint{0.847223in}{0.554012in}}{\pgfqpoint{6.200000in}{4.620000in}}%
\pgfusepath{clip}%
\pgfsetbuttcap%
\pgfsetroundjoin%
\pgfsetlinewidth{1.003750pt}%
\definecolor{currentstroke}{rgb}{1.000000,0.000000,0.000000}%
\pgfsetstrokecolor{currentstroke}%
\pgfsetdash{}{0pt}%
\pgfpathmoveto{\pgfqpoint{3.972487in}{0.810586in}}%
\pgfpathcurveto{\pgfqpoint{3.983538in}{0.810586in}}{\pgfqpoint{3.994137in}{0.814976in}}{\pgfqpoint{4.001950in}{0.822790in}}%
\pgfpathcurveto{\pgfqpoint{4.009764in}{0.830604in}}{\pgfqpoint{4.014154in}{0.841203in}}{\pgfqpoint{4.014154in}{0.852253in}}%
\pgfpathcurveto{\pgfqpoint{4.014154in}{0.863303in}}{\pgfqpoint{4.009764in}{0.873902in}}{\pgfqpoint{4.001950in}{0.881716in}}%
\pgfpathcurveto{\pgfqpoint{3.994137in}{0.889529in}}{\pgfqpoint{3.983538in}{0.893920in}}{\pgfqpoint{3.972487in}{0.893920in}}%
\pgfpathcurveto{\pgfqpoint{3.961437in}{0.893920in}}{\pgfqpoint{3.950838in}{0.889529in}}{\pgfqpoint{3.943025in}{0.881716in}}%
\pgfpathcurveto{\pgfqpoint{3.935211in}{0.873902in}}{\pgfqpoint{3.930821in}{0.863303in}}{\pgfqpoint{3.930821in}{0.852253in}}%
\pgfpathcurveto{\pgfqpoint{3.930821in}{0.841203in}}{\pgfqpoint{3.935211in}{0.830604in}}{\pgfqpoint{3.943025in}{0.822790in}}%
\pgfpathcurveto{\pgfqpoint{3.950838in}{0.814976in}}{\pgfqpoint{3.961437in}{0.810586in}}{\pgfqpoint{3.972487in}{0.810586in}}%
\pgfpathlineto{\pgfqpoint{3.972487in}{0.810586in}}%
\pgfpathclose%
\pgfusepath{stroke}%
\end{pgfscope}%
\begin{pgfscope}%
\pgfpathrectangle{\pgfqpoint{0.847223in}{0.554012in}}{\pgfqpoint{6.200000in}{4.620000in}}%
\pgfusepath{clip}%
\pgfsetbuttcap%
\pgfsetroundjoin%
\pgfsetlinewidth{1.003750pt}%
\definecolor{currentstroke}{rgb}{1.000000,0.000000,0.000000}%
\pgfsetstrokecolor{currentstroke}%
\pgfsetdash{}{0pt}%
\pgfpathmoveto{\pgfqpoint{3.977821in}{0.809530in}}%
\pgfpathcurveto{\pgfqpoint{3.988871in}{0.809530in}}{\pgfqpoint{3.999470in}{0.813920in}}{\pgfqpoint{4.007283in}{0.821734in}}%
\pgfpathcurveto{\pgfqpoint{4.015097in}{0.829547in}}{\pgfqpoint{4.019487in}{0.840146in}}{\pgfqpoint{4.019487in}{0.851197in}}%
\pgfpathcurveto{\pgfqpoint{4.019487in}{0.862247in}}{\pgfqpoint{4.015097in}{0.872846in}}{\pgfqpoint{4.007283in}{0.880659in}}%
\pgfpathcurveto{\pgfqpoint{3.999470in}{0.888473in}}{\pgfqpoint{3.988871in}{0.892863in}}{\pgfqpoint{3.977821in}{0.892863in}}%
\pgfpathcurveto{\pgfqpoint{3.966771in}{0.892863in}}{\pgfqpoint{3.956171in}{0.888473in}}{\pgfqpoint{3.948358in}{0.880659in}}%
\pgfpathcurveto{\pgfqpoint{3.940544in}{0.872846in}}{\pgfqpoint{3.936154in}{0.862247in}}{\pgfqpoint{3.936154in}{0.851197in}}%
\pgfpathcurveto{\pgfqpoint{3.936154in}{0.840146in}}{\pgfqpoint{3.940544in}{0.829547in}}{\pgfqpoint{3.948358in}{0.821734in}}%
\pgfpathcurveto{\pgfqpoint{3.956171in}{0.813920in}}{\pgfqpoint{3.966771in}{0.809530in}}{\pgfqpoint{3.977821in}{0.809530in}}%
\pgfpathlineto{\pgfqpoint{3.977821in}{0.809530in}}%
\pgfpathclose%
\pgfusepath{stroke}%
\end{pgfscope}%
\begin{pgfscope}%
\pgfpathrectangle{\pgfqpoint{0.847223in}{0.554012in}}{\pgfqpoint{6.200000in}{4.620000in}}%
\pgfusepath{clip}%
\pgfsetbuttcap%
\pgfsetroundjoin%
\pgfsetlinewidth{1.003750pt}%
\definecolor{currentstroke}{rgb}{1.000000,0.000000,0.000000}%
\pgfsetstrokecolor{currentstroke}%
\pgfsetdash{}{0pt}%
\pgfpathmoveto{\pgfqpoint{3.983154in}{0.808477in}}%
\pgfpathcurveto{\pgfqpoint{3.994204in}{0.808477in}}{\pgfqpoint{4.004803in}{0.812867in}}{\pgfqpoint{4.012617in}{0.820681in}}%
\pgfpathcurveto{\pgfqpoint{4.020430in}{0.828494in}}{\pgfqpoint{4.024821in}{0.839093in}}{\pgfqpoint{4.024821in}{0.850143in}}%
\pgfpathcurveto{\pgfqpoint{4.024821in}{0.861193in}}{\pgfqpoint{4.020430in}{0.871792in}}{\pgfqpoint{4.012617in}{0.879606in}}%
\pgfpathcurveto{\pgfqpoint{4.004803in}{0.887420in}}{\pgfqpoint{3.994204in}{0.891810in}}{\pgfqpoint{3.983154in}{0.891810in}}%
\pgfpathcurveto{\pgfqpoint{3.972104in}{0.891810in}}{\pgfqpoint{3.961505in}{0.887420in}}{\pgfqpoint{3.953691in}{0.879606in}}%
\pgfpathcurveto{\pgfqpoint{3.945877in}{0.871792in}}{\pgfqpoint{3.941487in}{0.861193in}}{\pgfqpoint{3.941487in}{0.850143in}}%
\pgfpathcurveto{\pgfqpoint{3.941487in}{0.839093in}}{\pgfqpoint{3.945877in}{0.828494in}}{\pgfqpoint{3.953691in}{0.820681in}}%
\pgfpathcurveto{\pgfqpoint{3.961505in}{0.812867in}}{\pgfqpoint{3.972104in}{0.808477in}}{\pgfqpoint{3.983154in}{0.808477in}}%
\pgfpathlineto{\pgfqpoint{3.983154in}{0.808477in}}%
\pgfpathclose%
\pgfusepath{stroke}%
\end{pgfscope}%
\begin{pgfscope}%
\pgfpathrectangle{\pgfqpoint{0.847223in}{0.554012in}}{\pgfqpoint{6.200000in}{4.620000in}}%
\pgfusepath{clip}%
\pgfsetbuttcap%
\pgfsetroundjoin%
\pgfsetlinewidth{1.003750pt}%
\definecolor{currentstroke}{rgb}{1.000000,0.000000,0.000000}%
\pgfsetstrokecolor{currentstroke}%
\pgfsetdash{}{0pt}%
\pgfpathmoveto{\pgfqpoint{3.988487in}{0.807427in}}%
\pgfpathcurveto{\pgfqpoint{3.999537in}{0.807427in}}{\pgfqpoint{4.010136in}{0.811817in}}{\pgfqpoint{4.017950in}{0.819630in}}%
\pgfpathcurveto{\pgfqpoint{4.025763in}{0.827444in}}{\pgfqpoint{4.030154in}{0.838043in}}{\pgfqpoint{4.030154in}{0.849093in}}%
\pgfpathcurveto{\pgfqpoint{4.030154in}{0.860143in}}{\pgfqpoint{4.025763in}{0.870742in}}{\pgfqpoint{4.017950in}{0.878556in}}%
\pgfpathcurveto{\pgfqpoint{4.010136in}{0.886370in}}{\pgfqpoint{3.999537in}{0.890760in}}{\pgfqpoint{3.988487in}{0.890760in}}%
\pgfpathcurveto{\pgfqpoint{3.977437in}{0.890760in}}{\pgfqpoint{3.966838in}{0.886370in}}{\pgfqpoint{3.959024in}{0.878556in}}%
\pgfpathcurveto{\pgfqpoint{3.951211in}{0.870742in}}{\pgfqpoint{3.946820in}{0.860143in}}{\pgfqpoint{3.946820in}{0.849093in}}%
\pgfpathcurveto{\pgfqpoint{3.946820in}{0.838043in}}{\pgfqpoint{3.951211in}{0.827444in}}{\pgfqpoint{3.959024in}{0.819630in}}%
\pgfpathcurveto{\pgfqpoint{3.966838in}{0.811817in}}{\pgfqpoint{3.977437in}{0.807427in}}{\pgfqpoint{3.988487in}{0.807427in}}%
\pgfpathlineto{\pgfqpoint{3.988487in}{0.807427in}}%
\pgfpathclose%
\pgfusepath{stroke}%
\end{pgfscope}%
\begin{pgfscope}%
\pgfpathrectangle{\pgfqpoint{0.847223in}{0.554012in}}{\pgfqpoint{6.200000in}{4.620000in}}%
\pgfusepath{clip}%
\pgfsetbuttcap%
\pgfsetroundjoin%
\pgfsetlinewidth{1.003750pt}%
\definecolor{currentstroke}{rgb}{1.000000,0.000000,0.000000}%
\pgfsetstrokecolor{currentstroke}%
\pgfsetdash{}{0pt}%
\pgfpathmoveto{\pgfqpoint{3.993820in}{0.806379in}}%
\pgfpathcurveto{\pgfqpoint{4.004870in}{0.806379in}}{\pgfqpoint{4.015469in}{0.810770in}}{\pgfqpoint{4.023283in}{0.818583in}}%
\pgfpathcurveto{\pgfqpoint{4.031097in}{0.826397in}}{\pgfqpoint{4.035487in}{0.836996in}}{\pgfqpoint{4.035487in}{0.848046in}}%
\pgfpathcurveto{\pgfqpoint{4.035487in}{0.859096in}}{\pgfqpoint{4.031097in}{0.869695in}}{\pgfqpoint{4.023283in}{0.877509in}}%
\pgfpathcurveto{\pgfqpoint{4.015469in}{0.885322in}}{\pgfqpoint{4.004870in}{0.889713in}}{\pgfqpoint{3.993820in}{0.889713in}}%
\pgfpathcurveto{\pgfqpoint{3.982770in}{0.889713in}}{\pgfqpoint{3.972171in}{0.885322in}}{\pgfqpoint{3.964358in}{0.877509in}}%
\pgfpathcurveto{\pgfqpoint{3.956544in}{0.869695in}}{\pgfqpoint{3.952154in}{0.859096in}}{\pgfqpoint{3.952154in}{0.848046in}}%
\pgfpathcurveto{\pgfqpoint{3.952154in}{0.836996in}}{\pgfqpoint{3.956544in}{0.826397in}}{\pgfqpoint{3.964358in}{0.818583in}}%
\pgfpathcurveto{\pgfqpoint{3.972171in}{0.810770in}}{\pgfqpoint{3.982770in}{0.806379in}}{\pgfqpoint{3.993820in}{0.806379in}}%
\pgfpathlineto{\pgfqpoint{3.993820in}{0.806379in}}%
\pgfpathclose%
\pgfusepath{stroke}%
\end{pgfscope}%
\begin{pgfscope}%
\pgfpathrectangle{\pgfqpoint{0.847223in}{0.554012in}}{\pgfqpoint{6.200000in}{4.620000in}}%
\pgfusepath{clip}%
\pgfsetbuttcap%
\pgfsetroundjoin%
\pgfsetlinewidth{1.003750pt}%
\definecolor{currentstroke}{rgb}{1.000000,0.000000,0.000000}%
\pgfsetstrokecolor{currentstroke}%
\pgfsetdash{}{0pt}%
\pgfpathmoveto{\pgfqpoint{3.999154in}{0.805335in}}%
\pgfpathcurveto{\pgfqpoint{4.010204in}{0.805335in}}{\pgfqpoint{4.020803in}{0.809726in}}{\pgfqpoint{4.028616in}{0.817539in}}%
\pgfpathcurveto{\pgfqpoint{4.036430in}{0.825353in}}{\pgfqpoint{4.040820in}{0.835952in}}{\pgfqpoint{4.040820in}{0.847002in}}%
\pgfpathcurveto{\pgfqpoint{4.040820in}{0.858052in}}{\pgfqpoint{4.036430in}{0.868651in}}{\pgfqpoint{4.028616in}{0.876465in}}%
\pgfpathcurveto{\pgfqpoint{4.020803in}{0.884278in}}{\pgfqpoint{4.010204in}{0.888669in}}{\pgfqpoint{3.999154in}{0.888669in}}%
\pgfpathcurveto{\pgfqpoint{3.988103in}{0.888669in}}{\pgfqpoint{3.977504in}{0.884278in}}{\pgfqpoint{3.969691in}{0.876465in}}%
\pgfpathcurveto{\pgfqpoint{3.961877in}{0.868651in}}{\pgfqpoint{3.957487in}{0.858052in}}{\pgfqpoint{3.957487in}{0.847002in}}%
\pgfpathcurveto{\pgfqpoint{3.957487in}{0.835952in}}{\pgfqpoint{3.961877in}{0.825353in}}{\pgfqpoint{3.969691in}{0.817539in}}%
\pgfpathcurveto{\pgfqpoint{3.977504in}{0.809726in}}{\pgfqpoint{3.988103in}{0.805335in}}{\pgfqpoint{3.999154in}{0.805335in}}%
\pgfpathlineto{\pgfqpoint{3.999154in}{0.805335in}}%
\pgfpathclose%
\pgfusepath{stroke}%
\end{pgfscope}%
\begin{pgfscope}%
\pgfpathrectangle{\pgfqpoint{0.847223in}{0.554012in}}{\pgfqpoint{6.200000in}{4.620000in}}%
\pgfusepath{clip}%
\pgfsetbuttcap%
\pgfsetroundjoin%
\pgfsetlinewidth{1.003750pt}%
\definecolor{currentstroke}{rgb}{1.000000,0.000000,0.000000}%
\pgfsetstrokecolor{currentstroke}%
\pgfsetdash{}{0pt}%
\pgfpathmoveto{\pgfqpoint{4.004487in}{0.804294in}}%
\pgfpathcurveto{\pgfqpoint{4.015537in}{0.804294in}}{\pgfqpoint{4.026136in}{0.808685in}}{\pgfqpoint{4.033950in}{0.816498in}}%
\pgfpathcurveto{\pgfqpoint{4.041763in}{0.824312in}}{\pgfqpoint{4.046153in}{0.834911in}}{\pgfqpoint{4.046153in}{0.845961in}}%
\pgfpathcurveto{\pgfqpoint{4.046153in}{0.857011in}}{\pgfqpoint{4.041763in}{0.867610in}}{\pgfqpoint{4.033950in}{0.875424in}}%
\pgfpathcurveto{\pgfqpoint{4.026136in}{0.883237in}}{\pgfqpoint{4.015537in}{0.887628in}}{\pgfqpoint{4.004487in}{0.887628in}}%
\pgfpathcurveto{\pgfqpoint{3.993437in}{0.887628in}}{\pgfqpoint{3.982838in}{0.883237in}}{\pgfqpoint{3.975024in}{0.875424in}}%
\pgfpathcurveto{\pgfqpoint{3.967210in}{0.867610in}}{\pgfqpoint{3.962820in}{0.857011in}}{\pgfqpoint{3.962820in}{0.845961in}}%
\pgfpathcurveto{\pgfqpoint{3.962820in}{0.834911in}}{\pgfqpoint{3.967210in}{0.824312in}}{\pgfqpoint{3.975024in}{0.816498in}}%
\pgfpathcurveto{\pgfqpoint{3.982838in}{0.808685in}}{\pgfqpoint{3.993437in}{0.804294in}}{\pgfqpoint{4.004487in}{0.804294in}}%
\pgfpathlineto{\pgfqpoint{4.004487in}{0.804294in}}%
\pgfpathclose%
\pgfusepath{stroke}%
\end{pgfscope}%
\begin{pgfscope}%
\pgfpathrectangle{\pgfqpoint{0.847223in}{0.554012in}}{\pgfqpoint{6.200000in}{4.620000in}}%
\pgfusepath{clip}%
\pgfsetbuttcap%
\pgfsetroundjoin%
\pgfsetlinewidth{1.003750pt}%
\definecolor{currentstroke}{rgb}{1.000000,0.000000,0.000000}%
\pgfsetstrokecolor{currentstroke}%
\pgfsetdash{}{0pt}%
\pgfpathmoveto{\pgfqpoint{4.009820in}{0.803256in}}%
\pgfpathcurveto{\pgfqpoint{4.020870in}{0.803256in}}{\pgfqpoint{4.031469in}{0.807646in}}{\pgfqpoint{4.039283in}{0.815460in}}%
\pgfpathcurveto{\pgfqpoint{4.047096in}{0.823274in}}{\pgfqpoint{4.051487in}{0.833873in}}{\pgfqpoint{4.051487in}{0.844923in}}%
\pgfpathcurveto{\pgfqpoint{4.051487in}{0.855973in}}{\pgfqpoint{4.047096in}{0.866572in}}{\pgfqpoint{4.039283in}{0.874386in}}%
\pgfpathcurveto{\pgfqpoint{4.031469in}{0.882199in}}{\pgfqpoint{4.020870in}{0.886590in}}{\pgfqpoint{4.009820in}{0.886590in}}%
\pgfpathcurveto{\pgfqpoint{3.998770in}{0.886590in}}{\pgfqpoint{3.988171in}{0.882199in}}{\pgfqpoint{3.980357in}{0.874386in}}%
\pgfpathcurveto{\pgfqpoint{3.972544in}{0.866572in}}{\pgfqpoint{3.968153in}{0.855973in}}{\pgfqpoint{3.968153in}{0.844923in}}%
\pgfpathcurveto{\pgfqpoint{3.968153in}{0.833873in}}{\pgfqpoint{3.972544in}{0.823274in}}{\pgfqpoint{3.980357in}{0.815460in}}%
\pgfpathcurveto{\pgfqpoint{3.988171in}{0.807646in}}{\pgfqpoint{3.998770in}{0.803256in}}{\pgfqpoint{4.009820in}{0.803256in}}%
\pgfpathlineto{\pgfqpoint{4.009820in}{0.803256in}}%
\pgfpathclose%
\pgfusepath{stroke}%
\end{pgfscope}%
\begin{pgfscope}%
\pgfpathrectangle{\pgfqpoint{0.847223in}{0.554012in}}{\pgfqpoint{6.200000in}{4.620000in}}%
\pgfusepath{clip}%
\pgfsetbuttcap%
\pgfsetroundjoin%
\pgfsetlinewidth{1.003750pt}%
\definecolor{currentstroke}{rgb}{1.000000,0.000000,0.000000}%
\pgfsetstrokecolor{currentstroke}%
\pgfsetdash{}{0pt}%
\pgfpathmoveto{\pgfqpoint{4.015153in}{0.802221in}}%
\pgfpathcurveto{\pgfqpoint{4.026203in}{0.802221in}}{\pgfqpoint{4.036802in}{0.806611in}}{\pgfqpoint{4.044616in}{0.814425in}}%
\pgfpathcurveto{\pgfqpoint{4.052430in}{0.822239in}}{\pgfqpoint{4.056820in}{0.832838in}}{\pgfqpoint{4.056820in}{0.843888in}}%
\pgfpathcurveto{\pgfqpoint{4.056820in}{0.854938in}}{\pgfqpoint{4.052430in}{0.865537in}}{\pgfqpoint{4.044616in}{0.873351in}}%
\pgfpathcurveto{\pgfqpoint{4.036802in}{0.881164in}}{\pgfqpoint{4.026203in}{0.885555in}}{\pgfqpoint{4.015153in}{0.885555in}}%
\pgfpathcurveto{\pgfqpoint{4.004103in}{0.885555in}}{\pgfqpoint{3.993504in}{0.881164in}}{\pgfqpoint{3.985690in}{0.873351in}}%
\pgfpathcurveto{\pgfqpoint{3.977877in}{0.865537in}}{\pgfqpoint{3.973486in}{0.854938in}}{\pgfqpoint{3.973486in}{0.843888in}}%
\pgfpathcurveto{\pgfqpoint{3.973486in}{0.832838in}}{\pgfqpoint{3.977877in}{0.822239in}}{\pgfqpoint{3.985690in}{0.814425in}}%
\pgfpathcurveto{\pgfqpoint{3.993504in}{0.806611in}}{\pgfqpoint{4.004103in}{0.802221in}}{\pgfqpoint{4.015153in}{0.802221in}}%
\pgfpathlineto{\pgfqpoint{4.015153in}{0.802221in}}%
\pgfpathclose%
\pgfusepath{stroke}%
\end{pgfscope}%
\begin{pgfscope}%
\pgfpathrectangle{\pgfqpoint{0.847223in}{0.554012in}}{\pgfqpoint{6.200000in}{4.620000in}}%
\pgfusepath{clip}%
\pgfsetbuttcap%
\pgfsetroundjoin%
\pgfsetlinewidth{1.003750pt}%
\definecolor{currentstroke}{rgb}{1.000000,0.000000,0.000000}%
\pgfsetstrokecolor{currentstroke}%
\pgfsetdash{}{0pt}%
\pgfpathmoveto{\pgfqpoint{4.020486in}{0.801189in}}%
\pgfpathcurveto{\pgfqpoint{4.031537in}{0.801189in}}{\pgfqpoint{4.042136in}{0.805579in}}{\pgfqpoint{4.049949in}{0.813393in}}%
\pgfpathcurveto{\pgfqpoint{4.057763in}{0.821207in}}{\pgfqpoint{4.062153in}{0.831806in}}{\pgfqpoint{4.062153in}{0.842856in}}%
\pgfpathcurveto{\pgfqpoint{4.062153in}{0.853906in}}{\pgfqpoint{4.057763in}{0.864505in}}{\pgfqpoint{4.049949in}{0.872319in}}%
\pgfpathcurveto{\pgfqpoint{4.042136in}{0.880132in}}{\pgfqpoint{4.031537in}{0.884522in}}{\pgfqpoint{4.020486in}{0.884522in}}%
\pgfpathcurveto{\pgfqpoint{4.009436in}{0.884522in}}{\pgfqpoint{3.998837in}{0.880132in}}{\pgfqpoint{3.991024in}{0.872319in}}%
\pgfpathcurveto{\pgfqpoint{3.983210in}{0.864505in}}{\pgfqpoint{3.978820in}{0.853906in}}{\pgfqpoint{3.978820in}{0.842856in}}%
\pgfpathcurveto{\pgfqpoint{3.978820in}{0.831806in}}{\pgfqpoint{3.983210in}{0.821207in}}{\pgfqpoint{3.991024in}{0.813393in}}%
\pgfpathcurveto{\pgfqpoint{3.998837in}{0.805579in}}{\pgfqpoint{4.009436in}{0.801189in}}{\pgfqpoint{4.020486in}{0.801189in}}%
\pgfpathlineto{\pgfqpoint{4.020486in}{0.801189in}}%
\pgfpathclose%
\pgfusepath{stroke}%
\end{pgfscope}%
\begin{pgfscope}%
\pgfpathrectangle{\pgfqpoint{0.847223in}{0.554012in}}{\pgfqpoint{6.200000in}{4.620000in}}%
\pgfusepath{clip}%
\pgfsetbuttcap%
\pgfsetroundjoin%
\pgfsetlinewidth{1.003750pt}%
\definecolor{currentstroke}{rgb}{1.000000,0.000000,0.000000}%
\pgfsetstrokecolor{currentstroke}%
\pgfsetdash{}{0pt}%
\pgfpathmoveto{\pgfqpoint{4.025820in}{0.800160in}}%
\pgfpathcurveto{\pgfqpoint{4.036870in}{0.800160in}}{\pgfqpoint{4.047469in}{0.804550in}}{\pgfqpoint{4.055282in}{0.812364in}}%
\pgfpathcurveto{\pgfqpoint{4.063096in}{0.820178in}}{\pgfqpoint{4.067486in}{0.830777in}}{\pgfqpoint{4.067486in}{0.841827in}}%
\pgfpathcurveto{\pgfqpoint{4.067486in}{0.852877in}}{\pgfqpoint{4.063096in}{0.863476in}}{\pgfqpoint{4.055282in}{0.871289in}}%
\pgfpathcurveto{\pgfqpoint{4.047469in}{0.879103in}}{\pgfqpoint{4.036870in}{0.883493in}}{\pgfqpoint{4.025820in}{0.883493in}}%
\pgfpathcurveto{\pgfqpoint{4.014769in}{0.883493in}}{\pgfqpoint{4.004170in}{0.879103in}}{\pgfqpoint{3.996357in}{0.871289in}}%
\pgfpathcurveto{\pgfqpoint{3.988543in}{0.863476in}}{\pgfqpoint{3.984153in}{0.852877in}}{\pgfqpoint{3.984153in}{0.841827in}}%
\pgfpathcurveto{\pgfqpoint{3.984153in}{0.830777in}}{\pgfqpoint{3.988543in}{0.820178in}}{\pgfqpoint{3.996357in}{0.812364in}}%
\pgfpathcurveto{\pgfqpoint{4.004170in}{0.804550in}}{\pgfqpoint{4.014769in}{0.800160in}}{\pgfqpoint{4.025820in}{0.800160in}}%
\pgfpathlineto{\pgfqpoint{4.025820in}{0.800160in}}%
\pgfpathclose%
\pgfusepath{stroke}%
\end{pgfscope}%
\begin{pgfscope}%
\pgfpathrectangle{\pgfqpoint{0.847223in}{0.554012in}}{\pgfqpoint{6.200000in}{4.620000in}}%
\pgfusepath{clip}%
\pgfsetbuttcap%
\pgfsetroundjoin%
\pgfsetlinewidth{1.003750pt}%
\definecolor{currentstroke}{rgb}{1.000000,0.000000,0.000000}%
\pgfsetstrokecolor{currentstroke}%
\pgfsetdash{}{0pt}%
\pgfpathmoveto{\pgfqpoint{4.031153in}{0.799134in}}%
\pgfpathcurveto{\pgfqpoint{4.042203in}{0.799134in}}{\pgfqpoint{4.052802in}{0.803524in}}{\pgfqpoint{4.060616in}{0.811338in}}%
\pgfpathcurveto{\pgfqpoint{4.068429in}{0.819151in}}{\pgfqpoint{4.072819in}{0.829750in}}{\pgfqpoint{4.072819in}{0.840801in}}%
\pgfpathcurveto{\pgfqpoint{4.072819in}{0.851851in}}{\pgfqpoint{4.068429in}{0.862450in}}{\pgfqpoint{4.060616in}{0.870263in}}%
\pgfpathcurveto{\pgfqpoint{4.052802in}{0.878077in}}{\pgfqpoint{4.042203in}{0.882467in}}{\pgfqpoint{4.031153in}{0.882467in}}%
\pgfpathcurveto{\pgfqpoint{4.020103in}{0.882467in}}{\pgfqpoint{4.009504in}{0.878077in}}{\pgfqpoint{4.001690in}{0.870263in}}%
\pgfpathcurveto{\pgfqpoint{3.993876in}{0.862450in}}{\pgfqpoint{3.989486in}{0.851851in}}{\pgfqpoint{3.989486in}{0.840801in}}%
\pgfpathcurveto{\pgfqpoint{3.989486in}{0.829750in}}{\pgfqpoint{3.993876in}{0.819151in}}{\pgfqpoint{4.001690in}{0.811338in}}%
\pgfpathcurveto{\pgfqpoint{4.009504in}{0.803524in}}{\pgfqpoint{4.020103in}{0.799134in}}{\pgfqpoint{4.031153in}{0.799134in}}%
\pgfpathlineto{\pgfqpoint{4.031153in}{0.799134in}}%
\pgfpathclose%
\pgfusepath{stroke}%
\end{pgfscope}%
\begin{pgfscope}%
\pgfpathrectangle{\pgfqpoint{0.847223in}{0.554012in}}{\pgfqpoint{6.200000in}{4.620000in}}%
\pgfusepath{clip}%
\pgfsetbuttcap%
\pgfsetroundjoin%
\pgfsetlinewidth{1.003750pt}%
\definecolor{currentstroke}{rgb}{1.000000,0.000000,0.000000}%
\pgfsetstrokecolor{currentstroke}%
\pgfsetdash{}{0pt}%
\pgfpathmoveto{\pgfqpoint{4.036486in}{0.798111in}}%
\pgfpathcurveto{\pgfqpoint{4.047536in}{0.798111in}}{\pgfqpoint{4.058135in}{0.802501in}}{\pgfqpoint{4.065949in}{0.810315in}}%
\pgfpathcurveto{\pgfqpoint{4.073762in}{0.818128in}}{\pgfqpoint{4.078153in}{0.828727in}}{\pgfqpoint{4.078153in}{0.839777in}}%
\pgfpathcurveto{\pgfqpoint{4.078153in}{0.850828in}}{\pgfqpoint{4.073762in}{0.861427in}}{\pgfqpoint{4.065949in}{0.869240in}}%
\pgfpathcurveto{\pgfqpoint{4.058135in}{0.877054in}}{\pgfqpoint{4.047536in}{0.881444in}}{\pgfqpoint{4.036486in}{0.881444in}}%
\pgfpathcurveto{\pgfqpoint{4.025436in}{0.881444in}}{\pgfqpoint{4.014837in}{0.877054in}}{\pgfqpoint{4.007023in}{0.869240in}}%
\pgfpathcurveto{\pgfqpoint{3.999210in}{0.861427in}}{\pgfqpoint{3.994819in}{0.850828in}}{\pgfqpoint{3.994819in}{0.839777in}}%
\pgfpathcurveto{\pgfqpoint{3.994819in}{0.828727in}}{\pgfqpoint{3.999210in}{0.818128in}}{\pgfqpoint{4.007023in}{0.810315in}}%
\pgfpathcurveto{\pgfqpoint{4.014837in}{0.802501in}}{\pgfqpoint{4.025436in}{0.798111in}}{\pgfqpoint{4.036486in}{0.798111in}}%
\pgfpathlineto{\pgfqpoint{4.036486in}{0.798111in}}%
\pgfpathclose%
\pgfusepath{stroke}%
\end{pgfscope}%
\begin{pgfscope}%
\pgfpathrectangle{\pgfqpoint{0.847223in}{0.554012in}}{\pgfqpoint{6.200000in}{4.620000in}}%
\pgfusepath{clip}%
\pgfsetbuttcap%
\pgfsetroundjoin%
\pgfsetlinewidth{1.003750pt}%
\definecolor{currentstroke}{rgb}{1.000000,0.000000,0.000000}%
\pgfsetstrokecolor{currentstroke}%
\pgfsetdash{}{0pt}%
\pgfpathmoveto{\pgfqpoint{4.041819in}{0.797091in}}%
\pgfpathcurveto{\pgfqpoint{4.052869in}{0.797091in}}{\pgfqpoint{4.063468in}{0.801481in}}{\pgfqpoint{4.071282in}{0.809294in}}%
\pgfpathcurveto{\pgfqpoint{4.079096in}{0.817108in}}{\pgfqpoint{4.083486in}{0.827707in}}{\pgfqpoint{4.083486in}{0.838757in}}%
\pgfpathcurveto{\pgfqpoint{4.083486in}{0.849807in}}{\pgfqpoint{4.079096in}{0.860406in}}{\pgfqpoint{4.071282in}{0.868220in}}%
\pgfpathcurveto{\pgfqpoint{4.063468in}{0.876034in}}{\pgfqpoint{4.052869in}{0.880424in}}{\pgfqpoint{4.041819in}{0.880424in}}%
\pgfpathcurveto{\pgfqpoint{4.030769in}{0.880424in}}{\pgfqpoint{4.020170in}{0.876034in}}{\pgfqpoint{4.012356in}{0.868220in}}%
\pgfpathcurveto{\pgfqpoint{4.004543in}{0.860406in}}{\pgfqpoint{4.000153in}{0.849807in}}{\pgfqpoint{4.000153in}{0.838757in}}%
\pgfpathcurveto{\pgfqpoint{4.000153in}{0.827707in}}{\pgfqpoint{4.004543in}{0.817108in}}{\pgfqpoint{4.012356in}{0.809294in}}%
\pgfpathcurveto{\pgfqpoint{4.020170in}{0.801481in}}{\pgfqpoint{4.030769in}{0.797091in}}{\pgfqpoint{4.041819in}{0.797091in}}%
\pgfpathlineto{\pgfqpoint{4.041819in}{0.797091in}}%
\pgfpathclose%
\pgfusepath{stroke}%
\end{pgfscope}%
\begin{pgfscope}%
\pgfpathrectangle{\pgfqpoint{0.847223in}{0.554012in}}{\pgfqpoint{6.200000in}{4.620000in}}%
\pgfusepath{clip}%
\pgfsetbuttcap%
\pgfsetroundjoin%
\pgfsetlinewidth{1.003750pt}%
\definecolor{currentstroke}{rgb}{1.000000,0.000000,0.000000}%
\pgfsetstrokecolor{currentstroke}%
\pgfsetdash{}{0pt}%
\pgfpathmoveto{\pgfqpoint{4.047152in}{0.796073in}}%
\pgfpathcurveto{\pgfqpoint{4.058203in}{0.796073in}}{\pgfqpoint{4.068802in}{0.800463in}}{\pgfqpoint{4.076615in}{0.808277in}}%
\pgfpathcurveto{\pgfqpoint{4.084429in}{0.816091in}}{\pgfqpoint{4.088819in}{0.826690in}}{\pgfqpoint{4.088819in}{0.837740in}}%
\pgfpathcurveto{\pgfqpoint{4.088819in}{0.848790in}}{\pgfqpoint{4.084429in}{0.859389in}}{\pgfqpoint{4.076615in}{0.867203in}}%
\pgfpathcurveto{\pgfqpoint{4.068802in}{0.875016in}}{\pgfqpoint{4.058203in}{0.879407in}}{\pgfqpoint{4.047152in}{0.879407in}}%
\pgfpathcurveto{\pgfqpoint{4.036102in}{0.879407in}}{\pgfqpoint{4.025503in}{0.875016in}}{\pgfqpoint{4.017690in}{0.867203in}}%
\pgfpathcurveto{\pgfqpoint{4.009876in}{0.859389in}}{\pgfqpoint{4.005486in}{0.848790in}}{\pgfqpoint{4.005486in}{0.837740in}}%
\pgfpathcurveto{\pgfqpoint{4.005486in}{0.826690in}}{\pgfqpoint{4.009876in}{0.816091in}}{\pgfqpoint{4.017690in}{0.808277in}}%
\pgfpathcurveto{\pgfqpoint{4.025503in}{0.800463in}}{\pgfqpoint{4.036102in}{0.796073in}}{\pgfqpoint{4.047152in}{0.796073in}}%
\pgfpathlineto{\pgfqpoint{4.047152in}{0.796073in}}%
\pgfpathclose%
\pgfusepath{stroke}%
\end{pgfscope}%
\begin{pgfscope}%
\pgfpathrectangle{\pgfqpoint{0.847223in}{0.554012in}}{\pgfqpoint{6.200000in}{4.620000in}}%
\pgfusepath{clip}%
\pgfsetbuttcap%
\pgfsetroundjoin%
\pgfsetlinewidth{1.003750pt}%
\definecolor{currentstroke}{rgb}{1.000000,0.000000,0.000000}%
\pgfsetstrokecolor{currentstroke}%
\pgfsetdash{}{0pt}%
\pgfpathmoveto{\pgfqpoint{4.052486in}{0.795059in}}%
\pgfpathcurveto{\pgfqpoint{4.063536in}{0.795059in}}{\pgfqpoint{4.074135in}{0.799449in}}{\pgfqpoint{4.081948in}{0.807263in}}%
\pgfpathcurveto{\pgfqpoint{4.089762in}{0.815076in}}{\pgfqpoint{4.094152in}{0.825675in}}{\pgfqpoint{4.094152in}{0.836725in}}%
\pgfpathcurveto{\pgfqpoint{4.094152in}{0.847776in}}{\pgfqpoint{4.089762in}{0.858375in}}{\pgfqpoint{4.081948in}{0.866188in}}%
\pgfpathcurveto{\pgfqpoint{4.074135in}{0.874002in}}{\pgfqpoint{4.063536in}{0.878392in}}{\pgfqpoint{4.052486in}{0.878392in}}%
\pgfpathcurveto{\pgfqpoint{4.041436in}{0.878392in}}{\pgfqpoint{4.030837in}{0.874002in}}{\pgfqpoint{4.023023in}{0.866188in}}%
\pgfpathcurveto{\pgfqpoint{4.015209in}{0.858375in}}{\pgfqpoint{4.010819in}{0.847776in}}{\pgfqpoint{4.010819in}{0.836725in}}%
\pgfpathcurveto{\pgfqpoint{4.010819in}{0.825675in}}{\pgfqpoint{4.015209in}{0.815076in}}{\pgfqpoint{4.023023in}{0.807263in}}%
\pgfpathcurveto{\pgfqpoint{4.030837in}{0.799449in}}{\pgfqpoint{4.041436in}{0.795059in}}{\pgfqpoint{4.052486in}{0.795059in}}%
\pgfpathlineto{\pgfqpoint{4.052486in}{0.795059in}}%
\pgfpathclose%
\pgfusepath{stroke}%
\end{pgfscope}%
\begin{pgfscope}%
\pgfpathrectangle{\pgfqpoint{0.847223in}{0.554012in}}{\pgfqpoint{6.200000in}{4.620000in}}%
\pgfusepath{clip}%
\pgfsetbuttcap%
\pgfsetroundjoin%
\pgfsetlinewidth{1.003750pt}%
\definecolor{currentstroke}{rgb}{1.000000,0.000000,0.000000}%
\pgfsetstrokecolor{currentstroke}%
\pgfsetdash{}{0pt}%
\pgfpathmoveto{\pgfqpoint{4.057819in}{0.794047in}}%
\pgfpathcurveto{\pgfqpoint{4.068869in}{0.794047in}}{\pgfqpoint{4.079468in}{0.798438in}}{\pgfqpoint{4.087282in}{0.806251in}}%
\pgfpathcurveto{\pgfqpoint{4.095095in}{0.814065in}}{\pgfqpoint{4.099486in}{0.824664in}}{\pgfqpoint{4.099486in}{0.835714in}}%
\pgfpathcurveto{\pgfqpoint{4.099486in}{0.846764in}}{\pgfqpoint{4.095095in}{0.857363in}}{\pgfqpoint{4.087282in}{0.865177in}}%
\pgfpathcurveto{\pgfqpoint{4.079468in}{0.872990in}}{\pgfqpoint{4.068869in}{0.877381in}}{\pgfqpoint{4.057819in}{0.877381in}}%
\pgfpathcurveto{\pgfqpoint{4.046769in}{0.877381in}}{\pgfqpoint{4.036170in}{0.872990in}}{\pgfqpoint{4.028356in}{0.865177in}}%
\pgfpathcurveto{\pgfqpoint{4.020542in}{0.857363in}}{\pgfqpoint{4.016152in}{0.846764in}}{\pgfqpoint{4.016152in}{0.835714in}}%
\pgfpathcurveto{\pgfqpoint{4.016152in}{0.824664in}}{\pgfqpoint{4.020542in}{0.814065in}}{\pgfqpoint{4.028356in}{0.806251in}}%
\pgfpathcurveto{\pgfqpoint{4.036170in}{0.798438in}}{\pgfqpoint{4.046769in}{0.794047in}}{\pgfqpoint{4.057819in}{0.794047in}}%
\pgfpathlineto{\pgfqpoint{4.057819in}{0.794047in}}%
\pgfpathclose%
\pgfusepath{stroke}%
\end{pgfscope}%
\begin{pgfscope}%
\pgfpathrectangle{\pgfqpoint{0.847223in}{0.554012in}}{\pgfqpoint{6.200000in}{4.620000in}}%
\pgfusepath{clip}%
\pgfsetbuttcap%
\pgfsetroundjoin%
\pgfsetlinewidth{1.003750pt}%
\definecolor{currentstroke}{rgb}{1.000000,0.000000,0.000000}%
\pgfsetstrokecolor{currentstroke}%
\pgfsetdash{}{0pt}%
\pgfpathmoveto{\pgfqpoint{4.063152in}{0.793039in}}%
\pgfpathcurveto{\pgfqpoint{4.074202in}{0.793039in}}{\pgfqpoint{4.084801in}{0.797429in}}{\pgfqpoint{4.092615in}{0.805243in}}%
\pgfpathcurveto{\pgfqpoint{4.100429in}{0.813056in}}{\pgfqpoint{4.104819in}{0.823655in}}{\pgfqpoint{4.104819in}{0.834705in}}%
\pgfpathcurveto{\pgfqpoint{4.104819in}{0.845755in}}{\pgfqpoint{4.100429in}{0.856355in}}{\pgfqpoint{4.092615in}{0.864168in}}%
\pgfpathcurveto{\pgfqpoint{4.084801in}{0.871982in}}{\pgfqpoint{4.074202in}{0.876372in}}{\pgfqpoint{4.063152in}{0.876372in}}%
\pgfpathcurveto{\pgfqpoint{4.052102in}{0.876372in}}{\pgfqpoint{4.041503in}{0.871982in}}{\pgfqpoint{4.033689in}{0.864168in}}%
\pgfpathcurveto{\pgfqpoint{4.025876in}{0.856355in}}{\pgfqpoint{4.021485in}{0.845755in}}{\pgfqpoint{4.021485in}{0.834705in}}%
\pgfpathcurveto{\pgfqpoint{4.021485in}{0.823655in}}{\pgfqpoint{4.025876in}{0.813056in}}{\pgfqpoint{4.033689in}{0.805243in}}%
\pgfpathcurveto{\pgfqpoint{4.041503in}{0.797429in}}{\pgfqpoint{4.052102in}{0.793039in}}{\pgfqpoint{4.063152in}{0.793039in}}%
\pgfpathlineto{\pgfqpoint{4.063152in}{0.793039in}}%
\pgfpathclose%
\pgfusepath{stroke}%
\end{pgfscope}%
\begin{pgfscope}%
\pgfpathrectangle{\pgfqpoint{0.847223in}{0.554012in}}{\pgfqpoint{6.200000in}{4.620000in}}%
\pgfusepath{clip}%
\pgfsetbuttcap%
\pgfsetroundjoin%
\pgfsetlinewidth{1.003750pt}%
\definecolor{currentstroke}{rgb}{1.000000,0.000000,0.000000}%
\pgfsetstrokecolor{currentstroke}%
\pgfsetdash{}{0pt}%
\pgfpathmoveto{\pgfqpoint{4.068485in}{0.792033in}}%
\pgfpathcurveto{\pgfqpoint{4.079535in}{0.792033in}}{\pgfqpoint{4.090134in}{0.796423in}}{\pgfqpoint{4.097948in}{0.804237in}}%
\pgfpathcurveto{\pgfqpoint{4.105762in}{0.812050in}}{\pgfqpoint{4.110152in}{0.822649in}}{\pgfqpoint{4.110152in}{0.833700in}}%
\pgfpathcurveto{\pgfqpoint{4.110152in}{0.844750in}}{\pgfqpoint{4.105762in}{0.855349in}}{\pgfqpoint{4.097948in}{0.863162in}}%
\pgfpathcurveto{\pgfqpoint{4.090134in}{0.870976in}}{\pgfqpoint{4.079535in}{0.875366in}}{\pgfqpoint{4.068485in}{0.875366in}}%
\pgfpathcurveto{\pgfqpoint{4.057435in}{0.875366in}}{\pgfqpoint{4.046836in}{0.870976in}}{\pgfqpoint{4.039023in}{0.863162in}}%
\pgfpathcurveto{\pgfqpoint{4.031209in}{0.855349in}}{\pgfqpoint{4.026819in}{0.844750in}}{\pgfqpoint{4.026819in}{0.833700in}}%
\pgfpathcurveto{\pgfqpoint{4.026819in}{0.822649in}}{\pgfqpoint{4.031209in}{0.812050in}}{\pgfqpoint{4.039023in}{0.804237in}}%
\pgfpathcurveto{\pgfqpoint{4.046836in}{0.796423in}}{\pgfqpoint{4.057435in}{0.792033in}}{\pgfqpoint{4.068485in}{0.792033in}}%
\pgfpathlineto{\pgfqpoint{4.068485in}{0.792033in}}%
\pgfpathclose%
\pgfusepath{stroke}%
\end{pgfscope}%
\begin{pgfscope}%
\pgfpathrectangle{\pgfqpoint{0.847223in}{0.554012in}}{\pgfqpoint{6.200000in}{4.620000in}}%
\pgfusepath{clip}%
\pgfsetbuttcap%
\pgfsetroundjoin%
\pgfsetlinewidth{1.003750pt}%
\definecolor{currentstroke}{rgb}{1.000000,0.000000,0.000000}%
\pgfsetstrokecolor{currentstroke}%
\pgfsetdash{}{0pt}%
\pgfpathmoveto{\pgfqpoint{4.073819in}{0.791030in}}%
\pgfpathcurveto{\pgfqpoint{4.084869in}{0.791030in}}{\pgfqpoint{4.095468in}{0.795420in}}{\pgfqpoint{4.103281in}{0.803234in}}%
\pgfpathcurveto{\pgfqpoint{4.111095in}{0.811048in}}{\pgfqpoint{4.115485in}{0.821647in}}{\pgfqpoint{4.115485in}{0.832697in}}%
\pgfpathcurveto{\pgfqpoint{4.115485in}{0.843747in}}{\pgfqpoint{4.111095in}{0.854346in}}{\pgfqpoint{4.103281in}{0.862159in}}%
\pgfpathcurveto{\pgfqpoint{4.095468in}{0.869973in}}{\pgfqpoint{4.084869in}{0.874363in}}{\pgfqpoint{4.073819in}{0.874363in}}%
\pgfpathcurveto{\pgfqpoint{4.062768in}{0.874363in}}{\pgfqpoint{4.052169in}{0.869973in}}{\pgfqpoint{4.044356in}{0.862159in}}%
\pgfpathcurveto{\pgfqpoint{4.036542in}{0.854346in}}{\pgfqpoint{4.032152in}{0.843747in}}{\pgfqpoint{4.032152in}{0.832697in}}%
\pgfpathcurveto{\pgfqpoint{4.032152in}{0.821647in}}{\pgfqpoint{4.036542in}{0.811048in}}{\pgfqpoint{4.044356in}{0.803234in}}%
\pgfpathcurveto{\pgfqpoint{4.052169in}{0.795420in}}{\pgfqpoint{4.062768in}{0.791030in}}{\pgfqpoint{4.073819in}{0.791030in}}%
\pgfpathlineto{\pgfqpoint{4.073819in}{0.791030in}}%
\pgfpathclose%
\pgfusepath{stroke}%
\end{pgfscope}%
\begin{pgfscope}%
\pgfpathrectangle{\pgfqpoint{0.847223in}{0.554012in}}{\pgfqpoint{6.200000in}{4.620000in}}%
\pgfusepath{clip}%
\pgfsetbuttcap%
\pgfsetroundjoin%
\pgfsetlinewidth{1.003750pt}%
\definecolor{currentstroke}{rgb}{1.000000,0.000000,0.000000}%
\pgfsetstrokecolor{currentstroke}%
\pgfsetdash{}{0pt}%
\pgfpathmoveto{\pgfqpoint{4.079152in}{0.790030in}}%
\pgfpathcurveto{\pgfqpoint{4.090202in}{0.790030in}}{\pgfqpoint{4.100801in}{0.794420in}}{\pgfqpoint{4.108615in}{0.802234in}}%
\pgfpathcurveto{\pgfqpoint{4.116428in}{0.810048in}}{\pgfqpoint{4.120818in}{0.820647in}}{\pgfqpoint{4.120818in}{0.831697in}}%
\pgfpathcurveto{\pgfqpoint{4.120818in}{0.842747in}}{\pgfqpoint{4.116428in}{0.853346in}}{\pgfqpoint{4.108615in}{0.861159in}}%
\pgfpathcurveto{\pgfqpoint{4.100801in}{0.868973in}}{\pgfqpoint{4.090202in}{0.873363in}}{\pgfqpoint{4.079152in}{0.873363in}}%
\pgfpathcurveto{\pgfqpoint{4.068102in}{0.873363in}}{\pgfqpoint{4.057503in}{0.868973in}}{\pgfqpoint{4.049689in}{0.861159in}}%
\pgfpathcurveto{\pgfqpoint{4.041875in}{0.853346in}}{\pgfqpoint{4.037485in}{0.842747in}}{\pgfqpoint{4.037485in}{0.831697in}}%
\pgfpathcurveto{\pgfqpoint{4.037485in}{0.820647in}}{\pgfqpoint{4.041875in}{0.810048in}}{\pgfqpoint{4.049689in}{0.802234in}}%
\pgfpathcurveto{\pgfqpoint{4.057503in}{0.794420in}}{\pgfqpoint{4.068102in}{0.790030in}}{\pgfqpoint{4.079152in}{0.790030in}}%
\pgfpathlineto{\pgfqpoint{4.079152in}{0.790030in}}%
\pgfpathclose%
\pgfusepath{stroke}%
\end{pgfscope}%
\begin{pgfscope}%
\pgfpathrectangle{\pgfqpoint{0.847223in}{0.554012in}}{\pgfqpoint{6.200000in}{4.620000in}}%
\pgfusepath{clip}%
\pgfsetbuttcap%
\pgfsetroundjoin%
\pgfsetlinewidth{1.003750pt}%
\definecolor{currentstroke}{rgb}{1.000000,0.000000,0.000000}%
\pgfsetstrokecolor{currentstroke}%
\pgfsetdash{}{0pt}%
\pgfpathmoveto{\pgfqpoint{4.084485in}{0.789033in}}%
\pgfpathcurveto{\pgfqpoint{4.095535in}{0.789033in}}{\pgfqpoint{4.106134in}{0.793423in}}{\pgfqpoint{4.113948in}{0.801237in}}%
\pgfpathcurveto{\pgfqpoint{4.121761in}{0.809050in}}{\pgfqpoint{4.126152in}{0.819649in}}{\pgfqpoint{4.126152in}{0.830699in}}%
\pgfpathcurveto{\pgfqpoint{4.126152in}{0.841750in}}{\pgfqpoint{4.121761in}{0.852349in}}{\pgfqpoint{4.113948in}{0.860162in}}%
\pgfpathcurveto{\pgfqpoint{4.106134in}{0.867976in}}{\pgfqpoint{4.095535in}{0.872366in}}{\pgfqpoint{4.084485in}{0.872366in}}%
\pgfpathcurveto{\pgfqpoint{4.073435in}{0.872366in}}{\pgfqpoint{4.062836in}{0.867976in}}{\pgfqpoint{4.055022in}{0.860162in}}%
\pgfpathcurveto{\pgfqpoint{4.047209in}{0.852349in}}{\pgfqpoint{4.042818in}{0.841750in}}{\pgfqpoint{4.042818in}{0.830699in}}%
\pgfpathcurveto{\pgfqpoint{4.042818in}{0.819649in}}{\pgfqpoint{4.047209in}{0.809050in}}{\pgfqpoint{4.055022in}{0.801237in}}%
\pgfpathcurveto{\pgfqpoint{4.062836in}{0.793423in}}{\pgfqpoint{4.073435in}{0.789033in}}{\pgfqpoint{4.084485in}{0.789033in}}%
\pgfpathlineto{\pgfqpoint{4.084485in}{0.789033in}}%
\pgfpathclose%
\pgfusepath{stroke}%
\end{pgfscope}%
\begin{pgfscope}%
\pgfpathrectangle{\pgfqpoint{0.847223in}{0.554012in}}{\pgfqpoint{6.200000in}{4.620000in}}%
\pgfusepath{clip}%
\pgfsetbuttcap%
\pgfsetroundjoin%
\pgfsetlinewidth{1.003750pt}%
\definecolor{currentstroke}{rgb}{1.000000,0.000000,0.000000}%
\pgfsetstrokecolor{currentstroke}%
\pgfsetdash{}{0pt}%
\pgfpathmoveto{\pgfqpoint{4.089818in}{0.788038in}}%
\pgfpathcurveto{\pgfqpoint{4.100868in}{0.788038in}}{\pgfqpoint{4.111467in}{0.792429in}}{\pgfqpoint{4.119281in}{0.800242in}}%
\pgfpathcurveto{\pgfqpoint{4.127095in}{0.808056in}}{\pgfqpoint{4.131485in}{0.818655in}}{\pgfqpoint{4.131485in}{0.829705in}}%
\pgfpathcurveto{\pgfqpoint{4.131485in}{0.840755in}}{\pgfqpoint{4.127095in}{0.851354in}}{\pgfqpoint{4.119281in}{0.859168in}}%
\pgfpathcurveto{\pgfqpoint{4.111467in}{0.866982in}}{\pgfqpoint{4.100868in}{0.871372in}}{\pgfqpoint{4.089818in}{0.871372in}}%
\pgfpathcurveto{\pgfqpoint{4.078768in}{0.871372in}}{\pgfqpoint{4.068169in}{0.866982in}}{\pgfqpoint{4.060355in}{0.859168in}}%
\pgfpathcurveto{\pgfqpoint{4.052542in}{0.851354in}}{\pgfqpoint{4.048152in}{0.840755in}}{\pgfqpoint{4.048152in}{0.829705in}}%
\pgfpathcurveto{\pgfqpoint{4.048152in}{0.818655in}}{\pgfqpoint{4.052542in}{0.808056in}}{\pgfqpoint{4.060355in}{0.800242in}}%
\pgfpathcurveto{\pgfqpoint{4.068169in}{0.792429in}}{\pgfqpoint{4.078768in}{0.788038in}}{\pgfqpoint{4.089818in}{0.788038in}}%
\pgfpathlineto{\pgfqpoint{4.089818in}{0.788038in}}%
\pgfpathclose%
\pgfusepath{stroke}%
\end{pgfscope}%
\begin{pgfscope}%
\pgfpathrectangle{\pgfqpoint{0.847223in}{0.554012in}}{\pgfqpoint{6.200000in}{4.620000in}}%
\pgfusepath{clip}%
\pgfsetbuttcap%
\pgfsetroundjoin%
\pgfsetlinewidth{1.003750pt}%
\definecolor{currentstroke}{rgb}{1.000000,0.000000,0.000000}%
\pgfsetstrokecolor{currentstroke}%
\pgfsetdash{}{0pt}%
\pgfpathmoveto{\pgfqpoint{4.095151in}{0.787047in}}%
\pgfpathcurveto{\pgfqpoint{4.106202in}{0.787047in}}{\pgfqpoint{4.116801in}{0.791437in}}{\pgfqpoint{4.124614in}{0.799251in}}%
\pgfpathcurveto{\pgfqpoint{4.132428in}{0.807064in}}{\pgfqpoint{4.136818in}{0.817663in}}{\pgfqpoint{4.136818in}{0.828714in}}%
\pgfpathcurveto{\pgfqpoint{4.136818in}{0.839764in}}{\pgfqpoint{4.132428in}{0.850363in}}{\pgfqpoint{4.124614in}{0.858176in}}%
\pgfpathcurveto{\pgfqpoint{4.116801in}{0.865990in}}{\pgfqpoint{4.106202in}{0.870380in}}{\pgfqpoint{4.095151in}{0.870380in}}%
\pgfpathcurveto{\pgfqpoint{4.084101in}{0.870380in}}{\pgfqpoint{4.073502in}{0.865990in}}{\pgfqpoint{4.065689in}{0.858176in}}%
\pgfpathcurveto{\pgfqpoint{4.057875in}{0.850363in}}{\pgfqpoint{4.053485in}{0.839764in}}{\pgfqpoint{4.053485in}{0.828714in}}%
\pgfpathcurveto{\pgfqpoint{4.053485in}{0.817663in}}{\pgfqpoint{4.057875in}{0.807064in}}{\pgfqpoint{4.065689in}{0.799251in}}%
\pgfpathcurveto{\pgfqpoint{4.073502in}{0.791437in}}{\pgfqpoint{4.084101in}{0.787047in}}{\pgfqpoint{4.095151in}{0.787047in}}%
\pgfpathlineto{\pgfqpoint{4.095151in}{0.787047in}}%
\pgfpathclose%
\pgfusepath{stroke}%
\end{pgfscope}%
\begin{pgfscope}%
\pgfpathrectangle{\pgfqpoint{0.847223in}{0.554012in}}{\pgfqpoint{6.200000in}{4.620000in}}%
\pgfusepath{clip}%
\pgfsetbuttcap%
\pgfsetroundjoin%
\pgfsetlinewidth{1.003750pt}%
\definecolor{currentstroke}{rgb}{1.000000,0.000000,0.000000}%
\pgfsetstrokecolor{currentstroke}%
\pgfsetdash{}{0pt}%
\pgfpathmoveto{\pgfqpoint{4.100485in}{0.786058in}}%
\pgfpathcurveto{\pgfqpoint{4.111535in}{0.786058in}}{\pgfqpoint{4.122134in}{0.790448in}}{\pgfqpoint{4.129947in}{0.798262in}}%
\pgfpathcurveto{\pgfqpoint{4.137761in}{0.806076in}}{\pgfqpoint{4.142151in}{0.816675in}}{\pgfqpoint{4.142151in}{0.827725in}}%
\pgfpathcurveto{\pgfqpoint{4.142151in}{0.838775in}}{\pgfqpoint{4.137761in}{0.849374in}}{\pgfqpoint{4.129947in}{0.857188in}}%
\pgfpathcurveto{\pgfqpoint{4.122134in}{0.865001in}}{\pgfqpoint{4.111535in}{0.869391in}}{\pgfqpoint{4.100485in}{0.869391in}}%
\pgfpathcurveto{\pgfqpoint{4.089434in}{0.869391in}}{\pgfqpoint{4.078835in}{0.865001in}}{\pgfqpoint{4.071022in}{0.857188in}}%
\pgfpathcurveto{\pgfqpoint{4.063208in}{0.849374in}}{\pgfqpoint{4.058818in}{0.838775in}}{\pgfqpoint{4.058818in}{0.827725in}}%
\pgfpathcurveto{\pgfqpoint{4.058818in}{0.816675in}}{\pgfqpoint{4.063208in}{0.806076in}}{\pgfqpoint{4.071022in}{0.798262in}}%
\pgfpathcurveto{\pgfqpoint{4.078835in}{0.790448in}}{\pgfqpoint{4.089434in}{0.786058in}}{\pgfqpoint{4.100485in}{0.786058in}}%
\pgfpathlineto{\pgfqpoint{4.100485in}{0.786058in}}%
\pgfpathclose%
\pgfusepath{stroke}%
\end{pgfscope}%
\begin{pgfscope}%
\pgfpathrectangle{\pgfqpoint{0.847223in}{0.554012in}}{\pgfqpoint{6.200000in}{4.620000in}}%
\pgfusepath{clip}%
\pgfsetbuttcap%
\pgfsetroundjoin%
\pgfsetlinewidth{1.003750pt}%
\definecolor{currentstroke}{rgb}{1.000000,0.000000,0.000000}%
\pgfsetstrokecolor{currentstroke}%
\pgfsetdash{}{0pt}%
\pgfpathmoveto{\pgfqpoint{4.105818in}{0.785072in}}%
\pgfpathcurveto{\pgfqpoint{4.116868in}{0.785072in}}{\pgfqpoint{4.127467in}{0.789462in}}{\pgfqpoint{4.135281in}{0.797276in}}%
\pgfpathcurveto{\pgfqpoint{4.143094in}{0.805090in}}{\pgfqpoint{4.147484in}{0.815689in}}{\pgfqpoint{4.147484in}{0.826739in}}%
\pgfpathcurveto{\pgfqpoint{4.147484in}{0.837789in}}{\pgfqpoint{4.143094in}{0.848388in}}{\pgfqpoint{4.135281in}{0.856202in}}%
\pgfpathcurveto{\pgfqpoint{4.127467in}{0.864015in}}{\pgfqpoint{4.116868in}{0.868405in}}{\pgfqpoint{4.105818in}{0.868405in}}%
\pgfpathcurveto{\pgfqpoint{4.094768in}{0.868405in}}{\pgfqpoint{4.084169in}{0.864015in}}{\pgfqpoint{4.076355in}{0.856202in}}%
\pgfpathcurveto{\pgfqpoint{4.068541in}{0.848388in}}{\pgfqpoint{4.064151in}{0.837789in}}{\pgfqpoint{4.064151in}{0.826739in}}%
\pgfpathcurveto{\pgfqpoint{4.064151in}{0.815689in}}{\pgfqpoint{4.068541in}{0.805090in}}{\pgfqpoint{4.076355in}{0.797276in}}%
\pgfpathcurveto{\pgfqpoint{4.084169in}{0.789462in}}{\pgfqpoint{4.094768in}{0.785072in}}{\pgfqpoint{4.105818in}{0.785072in}}%
\pgfpathlineto{\pgfqpoint{4.105818in}{0.785072in}}%
\pgfpathclose%
\pgfusepath{stroke}%
\end{pgfscope}%
\begin{pgfscope}%
\pgfpathrectangle{\pgfqpoint{0.847223in}{0.554012in}}{\pgfqpoint{6.200000in}{4.620000in}}%
\pgfusepath{clip}%
\pgfsetbuttcap%
\pgfsetroundjoin%
\pgfsetlinewidth{1.003750pt}%
\definecolor{currentstroke}{rgb}{1.000000,0.000000,0.000000}%
\pgfsetstrokecolor{currentstroke}%
\pgfsetdash{}{0pt}%
\pgfpathmoveto{\pgfqpoint{4.111151in}{0.784089in}}%
\pgfpathcurveto{\pgfqpoint{4.122201in}{0.784089in}}{\pgfqpoint{4.132800in}{0.788479in}}{\pgfqpoint{4.140614in}{0.796293in}}%
\pgfpathcurveto{\pgfqpoint{4.148427in}{0.804106in}}{\pgfqpoint{4.152818in}{0.814705in}}{\pgfqpoint{4.152818in}{0.825756in}}%
\pgfpathcurveto{\pgfqpoint{4.152818in}{0.836806in}}{\pgfqpoint{4.148427in}{0.847405in}}{\pgfqpoint{4.140614in}{0.855218in}}%
\pgfpathcurveto{\pgfqpoint{4.132800in}{0.863032in}}{\pgfqpoint{4.122201in}{0.867422in}}{\pgfqpoint{4.111151in}{0.867422in}}%
\pgfpathcurveto{\pgfqpoint{4.100101in}{0.867422in}}{\pgfqpoint{4.089502in}{0.863032in}}{\pgfqpoint{4.081688in}{0.855218in}}%
\pgfpathcurveto{\pgfqpoint{4.073875in}{0.847405in}}{\pgfqpoint{4.069484in}{0.836806in}}{\pgfqpoint{4.069484in}{0.825756in}}%
\pgfpathcurveto{\pgfqpoint{4.069484in}{0.814705in}}{\pgfqpoint{4.073875in}{0.804106in}}{\pgfqpoint{4.081688in}{0.796293in}}%
\pgfpathcurveto{\pgfqpoint{4.089502in}{0.788479in}}{\pgfqpoint{4.100101in}{0.784089in}}{\pgfqpoint{4.111151in}{0.784089in}}%
\pgfpathlineto{\pgfqpoint{4.111151in}{0.784089in}}%
\pgfpathclose%
\pgfusepath{stroke}%
\end{pgfscope}%
\begin{pgfscope}%
\pgfpathrectangle{\pgfqpoint{0.847223in}{0.554012in}}{\pgfqpoint{6.200000in}{4.620000in}}%
\pgfusepath{clip}%
\pgfsetbuttcap%
\pgfsetroundjoin%
\pgfsetlinewidth{1.003750pt}%
\definecolor{currentstroke}{rgb}{1.000000,0.000000,0.000000}%
\pgfsetstrokecolor{currentstroke}%
\pgfsetdash{}{0pt}%
\pgfpathmoveto{\pgfqpoint{4.116484in}{0.783108in}}%
\pgfpathcurveto{\pgfqpoint{4.127534in}{0.783108in}}{\pgfqpoint{4.138133in}{0.787499in}}{\pgfqpoint{4.145947in}{0.795312in}}%
\pgfpathcurveto{\pgfqpoint{4.153761in}{0.803126in}}{\pgfqpoint{4.158151in}{0.813725in}}{\pgfqpoint{4.158151in}{0.824775in}}%
\pgfpathcurveto{\pgfqpoint{4.158151in}{0.835825in}}{\pgfqpoint{4.153761in}{0.846424in}}{\pgfqpoint{4.145947in}{0.854238in}}%
\pgfpathcurveto{\pgfqpoint{4.138133in}{0.862052in}}{\pgfqpoint{4.127534in}{0.866442in}}{\pgfqpoint{4.116484in}{0.866442in}}%
\pgfpathcurveto{\pgfqpoint{4.105434in}{0.866442in}}{\pgfqpoint{4.094835in}{0.862052in}}{\pgfqpoint{4.087021in}{0.854238in}}%
\pgfpathcurveto{\pgfqpoint{4.079208in}{0.846424in}}{\pgfqpoint{4.074818in}{0.835825in}}{\pgfqpoint{4.074818in}{0.824775in}}%
\pgfpathcurveto{\pgfqpoint{4.074818in}{0.813725in}}{\pgfqpoint{4.079208in}{0.803126in}}{\pgfqpoint{4.087021in}{0.795312in}}%
\pgfpathcurveto{\pgfqpoint{4.094835in}{0.787499in}}{\pgfqpoint{4.105434in}{0.783108in}}{\pgfqpoint{4.116484in}{0.783108in}}%
\pgfpathlineto{\pgfqpoint{4.116484in}{0.783108in}}%
\pgfpathclose%
\pgfusepath{stroke}%
\end{pgfscope}%
\begin{pgfscope}%
\pgfpathrectangle{\pgfqpoint{0.847223in}{0.554012in}}{\pgfqpoint{6.200000in}{4.620000in}}%
\pgfusepath{clip}%
\pgfsetbuttcap%
\pgfsetroundjoin%
\pgfsetlinewidth{1.003750pt}%
\definecolor{currentstroke}{rgb}{1.000000,0.000000,0.000000}%
\pgfsetstrokecolor{currentstroke}%
\pgfsetdash{}{0pt}%
\pgfpathmoveto{\pgfqpoint{4.121817in}{0.782131in}}%
\pgfpathcurveto{\pgfqpoint{4.132868in}{0.782131in}}{\pgfqpoint{4.143467in}{0.786521in}}{\pgfqpoint{4.151280in}{0.794335in}}%
\pgfpathcurveto{\pgfqpoint{4.159094in}{0.802148in}}{\pgfqpoint{4.163484in}{0.812747in}}{\pgfqpoint{4.163484in}{0.823797in}}%
\pgfpathcurveto{\pgfqpoint{4.163484in}{0.834848in}}{\pgfqpoint{4.159094in}{0.845447in}}{\pgfqpoint{4.151280in}{0.853260in}}%
\pgfpathcurveto{\pgfqpoint{4.143467in}{0.861074in}}{\pgfqpoint{4.132868in}{0.865464in}}{\pgfqpoint{4.121817in}{0.865464in}}%
\pgfpathcurveto{\pgfqpoint{4.110767in}{0.865464in}}{\pgfqpoint{4.100168in}{0.861074in}}{\pgfqpoint{4.092355in}{0.853260in}}%
\pgfpathcurveto{\pgfqpoint{4.084541in}{0.845447in}}{\pgfqpoint{4.080151in}{0.834848in}}{\pgfqpoint{4.080151in}{0.823797in}}%
\pgfpathcurveto{\pgfqpoint{4.080151in}{0.812747in}}{\pgfqpoint{4.084541in}{0.802148in}}{\pgfqpoint{4.092355in}{0.794335in}}%
\pgfpathcurveto{\pgfqpoint{4.100168in}{0.786521in}}{\pgfqpoint{4.110767in}{0.782131in}}{\pgfqpoint{4.121817in}{0.782131in}}%
\pgfpathlineto{\pgfqpoint{4.121817in}{0.782131in}}%
\pgfpathclose%
\pgfusepath{stroke}%
\end{pgfscope}%
\begin{pgfscope}%
\pgfpathrectangle{\pgfqpoint{0.847223in}{0.554012in}}{\pgfqpoint{6.200000in}{4.620000in}}%
\pgfusepath{clip}%
\pgfsetbuttcap%
\pgfsetroundjoin%
\pgfsetlinewidth{1.003750pt}%
\definecolor{currentstroke}{rgb}{1.000000,0.000000,0.000000}%
\pgfsetstrokecolor{currentstroke}%
\pgfsetdash{}{0pt}%
\pgfpathmoveto{\pgfqpoint{4.127151in}{0.781156in}}%
\pgfpathcurveto{\pgfqpoint{4.138201in}{0.781156in}}{\pgfqpoint{4.148800in}{0.785546in}}{\pgfqpoint{4.156613in}{0.793360in}}%
\pgfpathcurveto{\pgfqpoint{4.164427in}{0.801173in}}{\pgfqpoint{4.168817in}{0.811772in}}{\pgfqpoint{4.168817in}{0.822823in}}%
\pgfpathcurveto{\pgfqpoint{4.168817in}{0.833873in}}{\pgfqpoint{4.164427in}{0.844472in}}{\pgfqpoint{4.156613in}{0.852285in}}%
\pgfpathcurveto{\pgfqpoint{4.148800in}{0.860099in}}{\pgfqpoint{4.138201in}{0.864489in}}{\pgfqpoint{4.127151in}{0.864489in}}%
\pgfpathcurveto{\pgfqpoint{4.116101in}{0.864489in}}{\pgfqpoint{4.105502in}{0.860099in}}{\pgfqpoint{4.097688in}{0.852285in}}%
\pgfpathcurveto{\pgfqpoint{4.089874in}{0.844472in}}{\pgfqpoint{4.085484in}{0.833873in}}{\pgfqpoint{4.085484in}{0.822823in}}%
\pgfpathcurveto{\pgfqpoint{4.085484in}{0.811772in}}{\pgfqpoint{4.089874in}{0.801173in}}{\pgfqpoint{4.097688in}{0.793360in}}%
\pgfpathcurveto{\pgfqpoint{4.105502in}{0.785546in}}{\pgfqpoint{4.116101in}{0.781156in}}{\pgfqpoint{4.127151in}{0.781156in}}%
\pgfpathlineto{\pgfqpoint{4.127151in}{0.781156in}}%
\pgfpathclose%
\pgfusepath{stroke}%
\end{pgfscope}%
\begin{pgfscope}%
\pgfpathrectangle{\pgfqpoint{0.847223in}{0.554012in}}{\pgfqpoint{6.200000in}{4.620000in}}%
\pgfusepath{clip}%
\pgfsetbuttcap%
\pgfsetroundjoin%
\pgfsetlinewidth{1.003750pt}%
\definecolor{currentstroke}{rgb}{1.000000,0.000000,0.000000}%
\pgfsetstrokecolor{currentstroke}%
\pgfsetdash{}{0pt}%
\pgfpathmoveto{\pgfqpoint{4.132484in}{0.780184in}}%
\pgfpathcurveto{\pgfqpoint{4.143534in}{0.780184in}}{\pgfqpoint{4.154133in}{0.784574in}}{\pgfqpoint{4.161947in}{0.792388in}}%
\pgfpathcurveto{\pgfqpoint{4.169760in}{0.800201in}}{\pgfqpoint{4.174151in}{0.810800in}}{\pgfqpoint{4.174151in}{0.821850in}}%
\pgfpathcurveto{\pgfqpoint{4.174151in}{0.832900in}}{\pgfqpoint{4.169760in}{0.843500in}}{\pgfqpoint{4.161947in}{0.851313in}}%
\pgfpathcurveto{\pgfqpoint{4.154133in}{0.859127in}}{\pgfqpoint{4.143534in}{0.863517in}}{\pgfqpoint{4.132484in}{0.863517in}}%
\pgfpathcurveto{\pgfqpoint{4.121434in}{0.863517in}}{\pgfqpoint{4.110835in}{0.859127in}}{\pgfqpoint{4.103021in}{0.851313in}}%
\pgfpathcurveto{\pgfqpoint{4.095207in}{0.843500in}}{\pgfqpoint{4.090817in}{0.832900in}}{\pgfqpoint{4.090817in}{0.821850in}}%
\pgfpathcurveto{\pgfqpoint{4.090817in}{0.810800in}}{\pgfqpoint{4.095207in}{0.800201in}}{\pgfqpoint{4.103021in}{0.792388in}}%
\pgfpathcurveto{\pgfqpoint{4.110835in}{0.784574in}}{\pgfqpoint{4.121434in}{0.780184in}}{\pgfqpoint{4.132484in}{0.780184in}}%
\pgfpathlineto{\pgfqpoint{4.132484in}{0.780184in}}%
\pgfpathclose%
\pgfusepath{stroke}%
\end{pgfscope}%
\begin{pgfscope}%
\pgfpathrectangle{\pgfqpoint{0.847223in}{0.554012in}}{\pgfqpoint{6.200000in}{4.620000in}}%
\pgfusepath{clip}%
\pgfsetbuttcap%
\pgfsetroundjoin%
\pgfsetlinewidth{1.003750pt}%
\definecolor{currentstroke}{rgb}{1.000000,0.000000,0.000000}%
\pgfsetstrokecolor{currentstroke}%
\pgfsetdash{}{0pt}%
\pgfpathmoveto{\pgfqpoint{4.137817in}{0.779214in}}%
\pgfpathcurveto{\pgfqpoint{4.148867in}{0.779214in}}{\pgfqpoint{4.159466in}{0.783604in}}{\pgfqpoint{4.167280in}{0.791418in}}%
\pgfpathcurveto{\pgfqpoint{4.175094in}{0.799232in}}{\pgfqpoint{4.179484in}{0.809831in}}{\pgfqpoint{4.179484in}{0.820881in}}%
\pgfpathcurveto{\pgfqpoint{4.179484in}{0.831931in}}{\pgfqpoint{4.175094in}{0.842530in}}{\pgfqpoint{4.167280in}{0.850344in}}%
\pgfpathcurveto{\pgfqpoint{4.159466in}{0.858157in}}{\pgfqpoint{4.148867in}{0.862548in}}{\pgfqpoint{4.137817in}{0.862548in}}%
\pgfpathcurveto{\pgfqpoint{4.126767in}{0.862548in}}{\pgfqpoint{4.116168in}{0.858157in}}{\pgfqpoint{4.108354in}{0.850344in}}%
\pgfpathcurveto{\pgfqpoint{4.100541in}{0.842530in}}{\pgfqpoint{4.096150in}{0.831931in}}{\pgfqpoint{4.096150in}{0.820881in}}%
\pgfpathcurveto{\pgfqpoint{4.096150in}{0.809831in}}{\pgfqpoint{4.100541in}{0.799232in}}{\pgfqpoint{4.108354in}{0.791418in}}%
\pgfpathcurveto{\pgfqpoint{4.116168in}{0.783604in}}{\pgfqpoint{4.126767in}{0.779214in}}{\pgfqpoint{4.137817in}{0.779214in}}%
\pgfpathlineto{\pgfqpoint{4.137817in}{0.779214in}}%
\pgfpathclose%
\pgfusepath{stroke}%
\end{pgfscope}%
\begin{pgfscope}%
\pgfpathrectangle{\pgfqpoint{0.847223in}{0.554012in}}{\pgfqpoint{6.200000in}{4.620000in}}%
\pgfusepath{clip}%
\pgfsetbuttcap%
\pgfsetroundjoin%
\pgfsetlinewidth{1.003750pt}%
\definecolor{currentstroke}{rgb}{1.000000,0.000000,0.000000}%
\pgfsetstrokecolor{currentstroke}%
\pgfsetdash{}{0pt}%
\pgfpathmoveto{\pgfqpoint{4.143150in}{0.778247in}}%
\pgfpathcurveto{\pgfqpoint{4.154200in}{0.778247in}}{\pgfqpoint{4.164799in}{0.782638in}}{\pgfqpoint{4.172613in}{0.790451in}}%
\pgfpathcurveto{\pgfqpoint{4.180427in}{0.798265in}}{\pgfqpoint{4.184817in}{0.808864in}}{\pgfqpoint{4.184817in}{0.819914in}}%
\pgfpathcurveto{\pgfqpoint{4.184817in}{0.830964in}}{\pgfqpoint{4.180427in}{0.841563in}}{\pgfqpoint{4.172613in}{0.849377in}}%
\pgfpathcurveto{\pgfqpoint{4.164799in}{0.857190in}}{\pgfqpoint{4.154200in}{0.861581in}}{\pgfqpoint{4.143150in}{0.861581in}}%
\pgfpathcurveto{\pgfqpoint{4.132100in}{0.861581in}}{\pgfqpoint{4.121501in}{0.857190in}}{\pgfqpoint{4.113688in}{0.849377in}}%
\pgfpathcurveto{\pgfqpoint{4.105874in}{0.841563in}}{\pgfqpoint{4.101484in}{0.830964in}}{\pgfqpoint{4.101484in}{0.819914in}}%
\pgfpathcurveto{\pgfqpoint{4.101484in}{0.808864in}}{\pgfqpoint{4.105874in}{0.798265in}}{\pgfqpoint{4.113688in}{0.790451in}}%
\pgfpathcurveto{\pgfqpoint{4.121501in}{0.782638in}}{\pgfqpoint{4.132100in}{0.778247in}}{\pgfqpoint{4.143150in}{0.778247in}}%
\pgfpathlineto{\pgfqpoint{4.143150in}{0.778247in}}%
\pgfpathclose%
\pgfusepath{stroke}%
\end{pgfscope}%
\begin{pgfscope}%
\pgfpathrectangle{\pgfqpoint{0.847223in}{0.554012in}}{\pgfqpoint{6.200000in}{4.620000in}}%
\pgfusepath{clip}%
\pgfsetbuttcap%
\pgfsetroundjoin%
\pgfsetlinewidth{1.003750pt}%
\definecolor{currentstroke}{rgb}{1.000000,0.000000,0.000000}%
\pgfsetstrokecolor{currentstroke}%
\pgfsetdash{}{0pt}%
\pgfpathmoveto{\pgfqpoint{4.148484in}{0.777283in}}%
\pgfpathcurveto{\pgfqpoint{4.159534in}{0.777283in}}{\pgfqpoint{4.170133in}{0.781674in}}{\pgfqpoint{4.177946in}{0.789487in}}%
\pgfpathcurveto{\pgfqpoint{4.185760in}{0.797301in}}{\pgfqpoint{4.190150in}{0.807900in}}{\pgfqpoint{4.190150in}{0.818950in}}%
\pgfpathcurveto{\pgfqpoint{4.190150in}{0.830000in}}{\pgfqpoint{4.185760in}{0.840599in}}{\pgfqpoint{4.177946in}{0.848413in}}%
\pgfpathcurveto{\pgfqpoint{4.170133in}{0.856226in}}{\pgfqpoint{4.159534in}{0.860617in}}{\pgfqpoint{4.148484in}{0.860617in}}%
\pgfpathcurveto{\pgfqpoint{4.137433in}{0.860617in}}{\pgfqpoint{4.126834in}{0.856226in}}{\pgfqpoint{4.119021in}{0.848413in}}%
\pgfpathcurveto{\pgfqpoint{4.111207in}{0.840599in}}{\pgfqpoint{4.106817in}{0.830000in}}{\pgfqpoint{4.106817in}{0.818950in}}%
\pgfpathcurveto{\pgfqpoint{4.106817in}{0.807900in}}{\pgfqpoint{4.111207in}{0.797301in}}{\pgfqpoint{4.119021in}{0.789487in}}%
\pgfpathcurveto{\pgfqpoint{4.126834in}{0.781674in}}{\pgfqpoint{4.137433in}{0.777283in}}{\pgfqpoint{4.148484in}{0.777283in}}%
\pgfpathlineto{\pgfqpoint{4.148484in}{0.777283in}}%
\pgfpathclose%
\pgfusepath{stroke}%
\end{pgfscope}%
\begin{pgfscope}%
\pgfpathrectangle{\pgfqpoint{0.847223in}{0.554012in}}{\pgfqpoint{6.200000in}{4.620000in}}%
\pgfusepath{clip}%
\pgfsetbuttcap%
\pgfsetroundjoin%
\pgfsetlinewidth{1.003750pt}%
\definecolor{currentstroke}{rgb}{1.000000,0.000000,0.000000}%
\pgfsetstrokecolor{currentstroke}%
\pgfsetdash{}{0pt}%
\pgfpathmoveto{\pgfqpoint{4.153817in}{0.776322in}}%
\pgfpathcurveto{\pgfqpoint{4.164867in}{0.776322in}}{\pgfqpoint{4.175466in}{0.780712in}}{\pgfqpoint{4.183280in}{0.788526in}}%
\pgfpathcurveto{\pgfqpoint{4.191093in}{0.796339in}}{\pgfqpoint{4.195483in}{0.806938in}}{\pgfqpoint{4.195483in}{0.817989in}}%
\pgfpathcurveto{\pgfqpoint{4.195483in}{0.829039in}}{\pgfqpoint{4.191093in}{0.839638in}}{\pgfqpoint{4.183280in}{0.847451in}}%
\pgfpathcurveto{\pgfqpoint{4.175466in}{0.855265in}}{\pgfqpoint{4.164867in}{0.859655in}}{\pgfqpoint{4.153817in}{0.859655in}}%
\pgfpathcurveto{\pgfqpoint{4.142767in}{0.859655in}}{\pgfqpoint{4.132168in}{0.855265in}}{\pgfqpoint{4.124354in}{0.847451in}}%
\pgfpathcurveto{\pgfqpoint{4.116540in}{0.839638in}}{\pgfqpoint{4.112150in}{0.829039in}}{\pgfqpoint{4.112150in}{0.817989in}}%
\pgfpathcurveto{\pgfqpoint{4.112150in}{0.806938in}}{\pgfqpoint{4.116540in}{0.796339in}}{\pgfqpoint{4.124354in}{0.788526in}}%
\pgfpathcurveto{\pgfqpoint{4.132168in}{0.780712in}}{\pgfqpoint{4.142767in}{0.776322in}}{\pgfqpoint{4.153817in}{0.776322in}}%
\pgfpathlineto{\pgfqpoint{4.153817in}{0.776322in}}%
\pgfpathclose%
\pgfusepath{stroke}%
\end{pgfscope}%
\begin{pgfscope}%
\pgfpathrectangle{\pgfqpoint{0.847223in}{0.554012in}}{\pgfqpoint{6.200000in}{4.620000in}}%
\pgfusepath{clip}%
\pgfsetbuttcap%
\pgfsetroundjoin%
\pgfsetlinewidth{1.003750pt}%
\definecolor{currentstroke}{rgb}{1.000000,0.000000,0.000000}%
\pgfsetstrokecolor{currentstroke}%
\pgfsetdash{}{0pt}%
\pgfpathmoveto{\pgfqpoint{4.159150in}{0.775363in}}%
\pgfpathcurveto{\pgfqpoint{4.170200in}{0.775363in}}{\pgfqpoint{4.180799in}{0.779753in}}{\pgfqpoint{4.188613in}{0.787567in}}%
\pgfpathcurveto{\pgfqpoint{4.196426in}{0.795381in}}{\pgfqpoint{4.200817in}{0.805980in}}{\pgfqpoint{4.200817in}{0.817030in}}%
\pgfpathcurveto{\pgfqpoint{4.200817in}{0.828080in}}{\pgfqpoint{4.196426in}{0.838679in}}{\pgfqpoint{4.188613in}{0.846493in}}%
\pgfpathcurveto{\pgfqpoint{4.180799in}{0.854306in}}{\pgfqpoint{4.170200in}{0.858697in}}{\pgfqpoint{4.159150in}{0.858697in}}%
\pgfpathcurveto{\pgfqpoint{4.148100in}{0.858697in}}{\pgfqpoint{4.137501in}{0.854306in}}{\pgfqpoint{4.129687in}{0.846493in}}%
\pgfpathcurveto{\pgfqpoint{4.121874in}{0.838679in}}{\pgfqpoint{4.117483in}{0.828080in}}{\pgfqpoint{4.117483in}{0.817030in}}%
\pgfpathcurveto{\pgfqpoint{4.117483in}{0.805980in}}{\pgfqpoint{4.121874in}{0.795381in}}{\pgfqpoint{4.129687in}{0.787567in}}%
\pgfpathcurveto{\pgfqpoint{4.137501in}{0.779753in}}{\pgfqpoint{4.148100in}{0.775363in}}{\pgfqpoint{4.159150in}{0.775363in}}%
\pgfpathlineto{\pgfqpoint{4.159150in}{0.775363in}}%
\pgfpathclose%
\pgfusepath{stroke}%
\end{pgfscope}%
\begin{pgfscope}%
\pgfpathrectangle{\pgfqpoint{0.847223in}{0.554012in}}{\pgfqpoint{6.200000in}{4.620000in}}%
\pgfusepath{clip}%
\pgfsetbuttcap%
\pgfsetroundjoin%
\pgfsetlinewidth{1.003750pt}%
\definecolor{currentstroke}{rgb}{1.000000,0.000000,0.000000}%
\pgfsetstrokecolor{currentstroke}%
\pgfsetdash{}{0pt}%
\pgfpathmoveto{\pgfqpoint{4.164483in}{0.774407in}}%
\pgfpathcurveto{\pgfqpoint{4.175533in}{0.774407in}}{\pgfqpoint{4.186132in}{0.778797in}}{\pgfqpoint{4.193946in}{0.786611in}}%
\pgfpathcurveto{\pgfqpoint{4.201760in}{0.794425in}}{\pgfqpoint{4.206150in}{0.805024in}}{\pgfqpoint{4.206150in}{0.816074in}}%
\pgfpathcurveto{\pgfqpoint{4.206150in}{0.827124in}}{\pgfqpoint{4.201760in}{0.837723in}}{\pgfqpoint{4.193946in}{0.845537in}}%
\pgfpathcurveto{\pgfqpoint{4.186132in}{0.853350in}}{\pgfqpoint{4.175533in}{0.857740in}}{\pgfqpoint{4.164483in}{0.857740in}}%
\pgfpathcurveto{\pgfqpoint{4.153433in}{0.857740in}}{\pgfqpoint{4.142834in}{0.853350in}}{\pgfqpoint{4.135020in}{0.845537in}}%
\pgfpathcurveto{\pgfqpoint{4.127207in}{0.837723in}}{\pgfqpoint{4.122817in}{0.827124in}}{\pgfqpoint{4.122817in}{0.816074in}}%
\pgfpathcurveto{\pgfqpoint{4.122817in}{0.805024in}}{\pgfqpoint{4.127207in}{0.794425in}}{\pgfqpoint{4.135020in}{0.786611in}}%
\pgfpathcurveto{\pgfqpoint{4.142834in}{0.778797in}}{\pgfqpoint{4.153433in}{0.774407in}}{\pgfqpoint{4.164483in}{0.774407in}}%
\pgfpathlineto{\pgfqpoint{4.164483in}{0.774407in}}%
\pgfpathclose%
\pgfusepath{stroke}%
\end{pgfscope}%
\begin{pgfscope}%
\pgfpathrectangle{\pgfqpoint{0.847223in}{0.554012in}}{\pgfqpoint{6.200000in}{4.620000in}}%
\pgfusepath{clip}%
\pgfsetbuttcap%
\pgfsetroundjoin%
\pgfsetlinewidth{1.003750pt}%
\definecolor{currentstroke}{rgb}{1.000000,0.000000,0.000000}%
\pgfsetstrokecolor{currentstroke}%
\pgfsetdash{}{0pt}%
\pgfpathmoveto{\pgfqpoint{4.169816in}{0.773454in}}%
\pgfpathcurveto{\pgfqpoint{4.180867in}{0.773454in}}{\pgfqpoint{4.191466in}{0.777844in}}{\pgfqpoint{4.199279in}{0.785658in}}%
\pgfpathcurveto{\pgfqpoint{4.207093in}{0.793471in}}{\pgfqpoint{4.211483in}{0.804070in}}{\pgfqpoint{4.211483in}{0.815120in}}%
\pgfpathcurveto{\pgfqpoint{4.211483in}{0.826171in}}{\pgfqpoint{4.207093in}{0.836770in}}{\pgfqpoint{4.199279in}{0.844583in}}%
\pgfpathcurveto{\pgfqpoint{4.191466in}{0.852397in}}{\pgfqpoint{4.180867in}{0.856787in}}{\pgfqpoint{4.169816in}{0.856787in}}%
\pgfpathcurveto{\pgfqpoint{4.158766in}{0.856787in}}{\pgfqpoint{4.148167in}{0.852397in}}{\pgfqpoint{4.140354in}{0.844583in}}%
\pgfpathcurveto{\pgfqpoint{4.132540in}{0.836770in}}{\pgfqpoint{4.128150in}{0.826171in}}{\pgfqpoint{4.128150in}{0.815120in}}%
\pgfpathcurveto{\pgfqpoint{4.128150in}{0.804070in}}{\pgfqpoint{4.132540in}{0.793471in}}{\pgfqpoint{4.140354in}{0.785658in}}%
\pgfpathcurveto{\pgfqpoint{4.148167in}{0.777844in}}{\pgfqpoint{4.158766in}{0.773454in}}{\pgfqpoint{4.169816in}{0.773454in}}%
\pgfpathlineto{\pgfqpoint{4.169816in}{0.773454in}}%
\pgfpathclose%
\pgfusepath{stroke}%
\end{pgfscope}%
\begin{pgfscope}%
\pgfpathrectangle{\pgfqpoint{0.847223in}{0.554012in}}{\pgfqpoint{6.200000in}{4.620000in}}%
\pgfusepath{clip}%
\pgfsetbuttcap%
\pgfsetroundjoin%
\pgfsetlinewidth{1.003750pt}%
\definecolor{currentstroke}{rgb}{1.000000,0.000000,0.000000}%
\pgfsetstrokecolor{currentstroke}%
\pgfsetdash{}{0pt}%
\pgfpathmoveto{\pgfqpoint{4.175150in}{0.772503in}}%
\pgfpathcurveto{\pgfqpoint{4.186200in}{0.772503in}}{\pgfqpoint{4.196799in}{0.776893in}}{\pgfqpoint{4.204612in}{0.784707in}}%
\pgfpathcurveto{\pgfqpoint{4.212426in}{0.792520in}}{\pgfqpoint{4.216816in}{0.803119in}}{\pgfqpoint{4.216816in}{0.814170in}}%
\pgfpathcurveto{\pgfqpoint{4.216816in}{0.825220in}}{\pgfqpoint{4.212426in}{0.835819in}}{\pgfqpoint{4.204612in}{0.843632in}}%
\pgfpathcurveto{\pgfqpoint{4.196799in}{0.851446in}}{\pgfqpoint{4.186200in}{0.855836in}}{\pgfqpoint{4.175150in}{0.855836in}}%
\pgfpathcurveto{\pgfqpoint{4.164099in}{0.855836in}}{\pgfqpoint{4.153500in}{0.851446in}}{\pgfqpoint{4.145687in}{0.843632in}}%
\pgfpathcurveto{\pgfqpoint{4.137873in}{0.835819in}}{\pgfqpoint{4.133483in}{0.825220in}}{\pgfqpoint{4.133483in}{0.814170in}}%
\pgfpathcurveto{\pgfqpoint{4.133483in}{0.803119in}}{\pgfqpoint{4.137873in}{0.792520in}}{\pgfqpoint{4.145687in}{0.784707in}}%
\pgfpathcurveto{\pgfqpoint{4.153500in}{0.776893in}}{\pgfqpoint{4.164099in}{0.772503in}}{\pgfqpoint{4.175150in}{0.772503in}}%
\pgfpathlineto{\pgfqpoint{4.175150in}{0.772503in}}%
\pgfpathclose%
\pgfusepath{stroke}%
\end{pgfscope}%
\begin{pgfscope}%
\pgfpathrectangle{\pgfqpoint{0.847223in}{0.554012in}}{\pgfqpoint{6.200000in}{4.620000in}}%
\pgfusepath{clip}%
\pgfsetbuttcap%
\pgfsetroundjoin%
\pgfsetlinewidth{1.003750pt}%
\definecolor{currentstroke}{rgb}{1.000000,0.000000,0.000000}%
\pgfsetstrokecolor{currentstroke}%
\pgfsetdash{}{0pt}%
\pgfpathmoveto{\pgfqpoint{4.180483in}{0.771555in}}%
\pgfpathcurveto{\pgfqpoint{4.191533in}{0.771555in}}{\pgfqpoint{4.202132in}{0.775945in}}{\pgfqpoint{4.209946in}{0.783759in}}%
\pgfpathcurveto{\pgfqpoint{4.217759in}{0.791572in}}{\pgfqpoint{4.222150in}{0.802171in}}{\pgfqpoint{4.222150in}{0.813221in}}%
\pgfpathcurveto{\pgfqpoint{4.222150in}{0.824272in}}{\pgfqpoint{4.217759in}{0.834871in}}{\pgfqpoint{4.209946in}{0.842684in}}%
\pgfpathcurveto{\pgfqpoint{4.202132in}{0.850498in}}{\pgfqpoint{4.191533in}{0.854888in}}{\pgfqpoint{4.180483in}{0.854888in}}%
\pgfpathcurveto{\pgfqpoint{4.169433in}{0.854888in}}{\pgfqpoint{4.158834in}{0.850498in}}{\pgfqpoint{4.151020in}{0.842684in}}%
\pgfpathcurveto{\pgfqpoint{4.143206in}{0.834871in}}{\pgfqpoint{4.138816in}{0.824272in}}{\pgfqpoint{4.138816in}{0.813221in}}%
\pgfpathcurveto{\pgfqpoint{4.138816in}{0.802171in}}{\pgfqpoint{4.143206in}{0.791572in}}{\pgfqpoint{4.151020in}{0.783759in}}%
\pgfpathcurveto{\pgfqpoint{4.158834in}{0.775945in}}{\pgfqpoint{4.169433in}{0.771555in}}{\pgfqpoint{4.180483in}{0.771555in}}%
\pgfpathlineto{\pgfqpoint{4.180483in}{0.771555in}}%
\pgfpathclose%
\pgfusepath{stroke}%
\end{pgfscope}%
\begin{pgfscope}%
\pgfpathrectangle{\pgfqpoint{0.847223in}{0.554012in}}{\pgfqpoint{6.200000in}{4.620000in}}%
\pgfusepath{clip}%
\pgfsetbuttcap%
\pgfsetroundjoin%
\pgfsetlinewidth{1.003750pt}%
\definecolor{currentstroke}{rgb}{1.000000,0.000000,0.000000}%
\pgfsetstrokecolor{currentstroke}%
\pgfsetdash{}{0pt}%
\pgfpathmoveto{\pgfqpoint{4.185816in}{0.770609in}}%
\pgfpathcurveto{\pgfqpoint{4.196866in}{0.770609in}}{\pgfqpoint{4.207465in}{0.775000in}}{\pgfqpoint{4.215279in}{0.782813in}}%
\pgfpathcurveto{\pgfqpoint{4.223092in}{0.790627in}}{\pgfqpoint{4.227483in}{0.801226in}}{\pgfqpoint{4.227483in}{0.812276in}}%
\pgfpathcurveto{\pgfqpoint{4.227483in}{0.823326in}}{\pgfqpoint{4.223092in}{0.833925in}}{\pgfqpoint{4.215279in}{0.841739in}}%
\pgfpathcurveto{\pgfqpoint{4.207465in}{0.849552in}}{\pgfqpoint{4.196866in}{0.853943in}}{\pgfqpoint{4.185816in}{0.853943in}}%
\pgfpathcurveto{\pgfqpoint{4.174766in}{0.853943in}}{\pgfqpoint{4.164167in}{0.849552in}}{\pgfqpoint{4.156353in}{0.841739in}}%
\pgfpathcurveto{\pgfqpoint{4.148540in}{0.833925in}}{\pgfqpoint{4.144149in}{0.823326in}}{\pgfqpoint{4.144149in}{0.812276in}}%
\pgfpathcurveto{\pgfqpoint{4.144149in}{0.801226in}}{\pgfqpoint{4.148540in}{0.790627in}}{\pgfqpoint{4.156353in}{0.782813in}}%
\pgfpathcurveto{\pgfqpoint{4.164167in}{0.775000in}}{\pgfqpoint{4.174766in}{0.770609in}}{\pgfqpoint{4.185816in}{0.770609in}}%
\pgfpathlineto{\pgfqpoint{4.185816in}{0.770609in}}%
\pgfpathclose%
\pgfusepath{stroke}%
\end{pgfscope}%
\begin{pgfscope}%
\pgfpathrectangle{\pgfqpoint{0.847223in}{0.554012in}}{\pgfqpoint{6.200000in}{4.620000in}}%
\pgfusepath{clip}%
\pgfsetbuttcap%
\pgfsetroundjoin%
\pgfsetlinewidth{1.003750pt}%
\definecolor{currentstroke}{rgb}{1.000000,0.000000,0.000000}%
\pgfsetstrokecolor{currentstroke}%
\pgfsetdash{}{0pt}%
\pgfpathmoveto{\pgfqpoint{4.191149in}{0.769666in}}%
\pgfpathcurveto{\pgfqpoint{4.202199in}{0.769666in}}{\pgfqpoint{4.212798in}{0.774057in}}{\pgfqpoint{4.220612in}{0.781870in}}%
\pgfpathcurveto{\pgfqpoint{4.228426in}{0.789684in}}{\pgfqpoint{4.232816in}{0.800283in}}{\pgfqpoint{4.232816in}{0.811333in}}%
\pgfpathcurveto{\pgfqpoint{4.232816in}{0.822383in}}{\pgfqpoint{4.228426in}{0.832982in}}{\pgfqpoint{4.220612in}{0.840796in}}%
\pgfpathcurveto{\pgfqpoint{4.212798in}{0.848609in}}{\pgfqpoint{4.202199in}{0.853000in}}{\pgfqpoint{4.191149in}{0.853000in}}%
\pgfpathcurveto{\pgfqpoint{4.180099in}{0.853000in}}{\pgfqpoint{4.169500in}{0.848609in}}{\pgfqpoint{4.161686in}{0.840796in}}%
\pgfpathcurveto{\pgfqpoint{4.153873in}{0.832982in}}{\pgfqpoint{4.149483in}{0.822383in}}{\pgfqpoint{4.149483in}{0.811333in}}%
\pgfpathcurveto{\pgfqpoint{4.149483in}{0.800283in}}{\pgfqpoint{4.153873in}{0.789684in}}{\pgfqpoint{4.161686in}{0.781870in}}%
\pgfpathcurveto{\pgfqpoint{4.169500in}{0.774057in}}{\pgfqpoint{4.180099in}{0.769666in}}{\pgfqpoint{4.191149in}{0.769666in}}%
\pgfpathlineto{\pgfqpoint{4.191149in}{0.769666in}}%
\pgfpathclose%
\pgfusepath{stroke}%
\end{pgfscope}%
\begin{pgfscope}%
\pgfpathrectangle{\pgfqpoint{0.847223in}{0.554012in}}{\pgfqpoint{6.200000in}{4.620000in}}%
\pgfusepath{clip}%
\pgfsetbuttcap%
\pgfsetroundjoin%
\pgfsetlinewidth{1.003750pt}%
\definecolor{currentstroke}{rgb}{1.000000,0.000000,0.000000}%
\pgfsetstrokecolor{currentstroke}%
\pgfsetdash{}{0pt}%
\pgfpathmoveto{\pgfqpoint{4.196482in}{0.768726in}}%
\pgfpathcurveto{\pgfqpoint{4.207533in}{0.768726in}}{\pgfqpoint{4.218132in}{0.773116in}}{\pgfqpoint{4.225945in}{0.780930in}}%
\pgfpathcurveto{\pgfqpoint{4.233759in}{0.788744in}}{\pgfqpoint{4.238149in}{0.799343in}}{\pgfqpoint{4.238149in}{0.810393in}}%
\pgfpathcurveto{\pgfqpoint{4.238149in}{0.821443in}}{\pgfqpoint{4.233759in}{0.832042in}}{\pgfqpoint{4.225945in}{0.839856in}}%
\pgfpathcurveto{\pgfqpoint{4.218132in}{0.847669in}}{\pgfqpoint{4.207533in}{0.852059in}}{\pgfqpoint{4.196482in}{0.852059in}}%
\pgfpathcurveto{\pgfqpoint{4.185432in}{0.852059in}}{\pgfqpoint{4.174833in}{0.847669in}}{\pgfqpoint{4.167020in}{0.839856in}}%
\pgfpathcurveto{\pgfqpoint{4.159206in}{0.832042in}}{\pgfqpoint{4.154816in}{0.821443in}}{\pgfqpoint{4.154816in}{0.810393in}}%
\pgfpathcurveto{\pgfqpoint{4.154816in}{0.799343in}}{\pgfqpoint{4.159206in}{0.788744in}}{\pgfqpoint{4.167020in}{0.780930in}}%
\pgfpathcurveto{\pgfqpoint{4.174833in}{0.773116in}}{\pgfqpoint{4.185432in}{0.768726in}}{\pgfqpoint{4.196482in}{0.768726in}}%
\pgfpathlineto{\pgfqpoint{4.196482in}{0.768726in}}%
\pgfpathclose%
\pgfusepath{stroke}%
\end{pgfscope}%
\begin{pgfscope}%
\pgfpathrectangle{\pgfqpoint{0.847223in}{0.554012in}}{\pgfqpoint{6.200000in}{4.620000in}}%
\pgfusepath{clip}%
\pgfsetbuttcap%
\pgfsetroundjoin%
\pgfsetlinewidth{1.003750pt}%
\definecolor{currentstroke}{rgb}{1.000000,0.000000,0.000000}%
\pgfsetstrokecolor{currentstroke}%
\pgfsetdash{}{0pt}%
\pgfpathmoveto{\pgfqpoint{4.201816in}{0.767788in}}%
\pgfpathcurveto{\pgfqpoint{4.212866in}{0.767788in}}{\pgfqpoint{4.223465in}{0.772179in}}{\pgfqpoint{4.231278in}{0.779992in}}%
\pgfpathcurveto{\pgfqpoint{4.239092in}{0.787806in}}{\pgfqpoint{4.243482in}{0.798405in}}{\pgfqpoint{4.243482in}{0.809455in}}%
\pgfpathcurveto{\pgfqpoint{4.243482in}{0.820505in}}{\pgfqpoint{4.239092in}{0.831104in}}{\pgfqpoint{4.231278in}{0.838918in}}%
\pgfpathcurveto{\pgfqpoint{4.223465in}{0.846731in}}{\pgfqpoint{4.212866in}{0.851122in}}{\pgfqpoint{4.201816in}{0.851122in}}%
\pgfpathcurveto{\pgfqpoint{4.190766in}{0.851122in}}{\pgfqpoint{4.180167in}{0.846731in}}{\pgfqpoint{4.172353in}{0.838918in}}%
\pgfpathcurveto{\pgfqpoint{4.164539in}{0.831104in}}{\pgfqpoint{4.160149in}{0.820505in}}{\pgfqpoint{4.160149in}{0.809455in}}%
\pgfpathcurveto{\pgfqpoint{4.160149in}{0.798405in}}{\pgfqpoint{4.164539in}{0.787806in}}{\pgfqpoint{4.172353in}{0.779992in}}%
\pgfpathcurveto{\pgfqpoint{4.180167in}{0.772179in}}{\pgfqpoint{4.190766in}{0.767788in}}{\pgfqpoint{4.201816in}{0.767788in}}%
\pgfpathlineto{\pgfqpoint{4.201816in}{0.767788in}}%
\pgfpathclose%
\pgfusepath{stroke}%
\end{pgfscope}%
\begin{pgfscope}%
\pgfpathrectangle{\pgfqpoint{0.847223in}{0.554012in}}{\pgfqpoint{6.200000in}{4.620000in}}%
\pgfusepath{clip}%
\pgfsetbuttcap%
\pgfsetroundjoin%
\pgfsetlinewidth{1.003750pt}%
\definecolor{currentstroke}{rgb}{1.000000,0.000000,0.000000}%
\pgfsetstrokecolor{currentstroke}%
\pgfsetdash{}{0pt}%
\pgfpathmoveto{\pgfqpoint{4.207149in}{0.766853in}}%
\pgfpathcurveto{\pgfqpoint{4.218199in}{0.766853in}}{\pgfqpoint{4.228798in}{0.771243in}}{\pgfqpoint{4.236612in}{0.779057in}}%
\pgfpathcurveto{\pgfqpoint{4.244425in}{0.786871in}}{\pgfqpoint{4.248816in}{0.797470in}}{\pgfqpoint{4.248816in}{0.808520in}}%
\pgfpathcurveto{\pgfqpoint{4.248816in}{0.819570in}}{\pgfqpoint{4.244425in}{0.830169in}}{\pgfqpoint{4.236612in}{0.837983in}}%
\pgfpathcurveto{\pgfqpoint{4.228798in}{0.845796in}}{\pgfqpoint{4.218199in}{0.850187in}}{\pgfqpoint{4.207149in}{0.850187in}}%
\pgfpathcurveto{\pgfqpoint{4.196099in}{0.850187in}}{\pgfqpoint{4.185500in}{0.845796in}}{\pgfqpoint{4.177686in}{0.837983in}}%
\pgfpathcurveto{\pgfqpoint{4.169873in}{0.830169in}}{\pgfqpoint{4.165482in}{0.819570in}}{\pgfqpoint{4.165482in}{0.808520in}}%
\pgfpathcurveto{\pgfqpoint{4.165482in}{0.797470in}}{\pgfqpoint{4.169873in}{0.786871in}}{\pgfqpoint{4.177686in}{0.779057in}}%
\pgfpathcurveto{\pgfqpoint{4.185500in}{0.771243in}}{\pgfqpoint{4.196099in}{0.766853in}}{\pgfqpoint{4.207149in}{0.766853in}}%
\pgfpathlineto{\pgfqpoint{4.207149in}{0.766853in}}%
\pgfpathclose%
\pgfusepath{stroke}%
\end{pgfscope}%
\begin{pgfscope}%
\pgfpathrectangle{\pgfqpoint{0.847223in}{0.554012in}}{\pgfqpoint{6.200000in}{4.620000in}}%
\pgfusepath{clip}%
\pgfsetbuttcap%
\pgfsetroundjoin%
\pgfsetlinewidth{1.003750pt}%
\definecolor{currentstroke}{rgb}{1.000000,0.000000,0.000000}%
\pgfsetstrokecolor{currentstroke}%
\pgfsetdash{}{0pt}%
\pgfpathmoveto{\pgfqpoint{4.212482in}{0.765921in}}%
\pgfpathcurveto{\pgfqpoint{4.223532in}{0.765921in}}{\pgfqpoint{4.234131in}{0.770311in}}{\pgfqpoint{4.241945in}{0.778124in}}%
\pgfpathcurveto{\pgfqpoint{4.249759in}{0.785938in}}{\pgfqpoint{4.254149in}{0.796537in}}{\pgfqpoint{4.254149in}{0.807587in}}%
\pgfpathcurveto{\pgfqpoint{4.254149in}{0.818637in}}{\pgfqpoint{4.249759in}{0.829236in}}{\pgfqpoint{4.241945in}{0.837050in}}%
\pgfpathcurveto{\pgfqpoint{4.234131in}{0.844864in}}{\pgfqpoint{4.223532in}{0.849254in}}{\pgfqpoint{4.212482in}{0.849254in}}%
\pgfpathcurveto{\pgfqpoint{4.201432in}{0.849254in}}{\pgfqpoint{4.190833in}{0.844864in}}{\pgfqpoint{4.183019in}{0.837050in}}%
\pgfpathcurveto{\pgfqpoint{4.175206in}{0.829236in}}{\pgfqpoint{4.170815in}{0.818637in}}{\pgfqpoint{4.170815in}{0.807587in}}%
\pgfpathcurveto{\pgfqpoint{4.170815in}{0.796537in}}{\pgfqpoint{4.175206in}{0.785938in}}{\pgfqpoint{4.183019in}{0.778124in}}%
\pgfpathcurveto{\pgfqpoint{4.190833in}{0.770311in}}{\pgfqpoint{4.201432in}{0.765921in}}{\pgfqpoint{4.212482in}{0.765921in}}%
\pgfpathlineto{\pgfqpoint{4.212482in}{0.765921in}}%
\pgfpathclose%
\pgfusepath{stroke}%
\end{pgfscope}%
\begin{pgfscope}%
\pgfpathrectangle{\pgfqpoint{0.847223in}{0.554012in}}{\pgfqpoint{6.200000in}{4.620000in}}%
\pgfusepath{clip}%
\pgfsetbuttcap%
\pgfsetroundjoin%
\pgfsetlinewidth{1.003750pt}%
\definecolor{currentstroke}{rgb}{1.000000,0.000000,0.000000}%
\pgfsetstrokecolor{currentstroke}%
\pgfsetdash{}{0pt}%
\pgfpathmoveto{\pgfqpoint{4.217815in}{0.764991in}}%
\pgfpathcurveto{\pgfqpoint{4.228865in}{0.764991in}}{\pgfqpoint{4.239465in}{0.769381in}}{\pgfqpoint{4.247278in}{0.777194in}}%
\pgfpathcurveto{\pgfqpoint{4.255092in}{0.785008in}}{\pgfqpoint{4.259482in}{0.795607in}}{\pgfqpoint{4.259482in}{0.806657in}}%
\pgfpathcurveto{\pgfqpoint{4.259482in}{0.817707in}}{\pgfqpoint{4.255092in}{0.828306in}}{\pgfqpoint{4.247278in}{0.836120in}}%
\pgfpathcurveto{\pgfqpoint{4.239465in}{0.843934in}}{\pgfqpoint{4.228865in}{0.848324in}}{\pgfqpoint{4.217815in}{0.848324in}}%
\pgfpathcurveto{\pgfqpoint{4.206765in}{0.848324in}}{\pgfqpoint{4.196166in}{0.843934in}}{\pgfqpoint{4.188353in}{0.836120in}}%
\pgfpathcurveto{\pgfqpoint{4.180539in}{0.828306in}}{\pgfqpoint{4.176149in}{0.817707in}}{\pgfqpoint{4.176149in}{0.806657in}}%
\pgfpathcurveto{\pgfqpoint{4.176149in}{0.795607in}}{\pgfqpoint{4.180539in}{0.785008in}}{\pgfqpoint{4.188353in}{0.777194in}}%
\pgfpathcurveto{\pgfqpoint{4.196166in}{0.769381in}}{\pgfqpoint{4.206765in}{0.764991in}}{\pgfqpoint{4.217815in}{0.764991in}}%
\pgfpathlineto{\pgfqpoint{4.217815in}{0.764991in}}%
\pgfpathclose%
\pgfusepath{stroke}%
\end{pgfscope}%
\begin{pgfscope}%
\pgfpathrectangle{\pgfqpoint{0.847223in}{0.554012in}}{\pgfqpoint{6.200000in}{4.620000in}}%
\pgfusepath{clip}%
\pgfsetbuttcap%
\pgfsetroundjoin%
\pgfsetlinewidth{1.003750pt}%
\definecolor{currentstroke}{rgb}{1.000000,0.000000,0.000000}%
\pgfsetstrokecolor{currentstroke}%
\pgfsetdash{}{0pt}%
\pgfpathmoveto{\pgfqpoint{4.223149in}{0.764063in}}%
\pgfpathcurveto{\pgfqpoint{4.234199in}{0.764063in}}{\pgfqpoint{4.244798in}{0.768453in}}{\pgfqpoint{4.252611in}{0.776267in}}%
\pgfpathcurveto{\pgfqpoint{4.260425in}{0.784081in}}{\pgfqpoint{4.264815in}{0.794680in}}{\pgfqpoint{4.264815in}{0.805730in}}%
\pgfpathcurveto{\pgfqpoint{4.264815in}{0.816780in}}{\pgfqpoint{4.260425in}{0.827379in}}{\pgfqpoint{4.252611in}{0.835193in}}%
\pgfpathcurveto{\pgfqpoint{4.244798in}{0.843006in}}{\pgfqpoint{4.234199in}{0.847396in}}{\pgfqpoint{4.223149in}{0.847396in}}%
\pgfpathcurveto{\pgfqpoint{4.212098in}{0.847396in}}{\pgfqpoint{4.201499in}{0.843006in}}{\pgfqpoint{4.193686in}{0.835193in}}%
\pgfpathcurveto{\pgfqpoint{4.185872in}{0.827379in}}{\pgfqpoint{4.181482in}{0.816780in}}{\pgfqpoint{4.181482in}{0.805730in}}%
\pgfpathcurveto{\pgfqpoint{4.181482in}{0.794680in}}{\pgfqpoint{4.185872in}{0.784081in}}{\pgfqpoint{4.193686in}{0.776267in}}%
\pgfpathcurveto{\pgfqpoint{4.201499in}{0.768453in}}{\pgfqpoint{4.212098in}{0.764063in}}{\pgfqpoint{4.223149in}{0.764063in}}%
\pgfpathlineto{\pgfqpoint{4.223149in}{0.764063in}}%
\pgfpathclose%
\pgfusepath{stroke}%
\end{pgfscope}%
\begin{pgfscope}%
\pgfpathrectangle{\pgfqpoint{0.847223in}{0.554012in}}{\pgfqpoint{6.200000in}{4.620000in}}%
\pgfusepath{clip}%
\pgfsetbuttcap%
\pgfsetroundjoin%
\pgfsetlinewidth{1.003750pt}%
\definecolor{currentstroke}{rgb}{1.000000,0.000000,0.000000}%
\pgfsetstrokecolor{currentstroke}%
\pgfsetdash{}{0pt}%
\pgfpathmoveto{\pgfqpoint{4.228482in}{0.763138in}}%
\pgfpathcurveto{\pgfqpoint{4.239532in}{0.763138in}}{\pgfqpoint{4.250131in}{0.767528in}}{\pgfqpoint{4.257945in}{0.775342in}}%
\pgfpathcurveto{\pgfqpoint{4.265758in}{0.783156in}}{\pgfqpoint{4.270148in}{0.793755in}}{\pgfqpoint{4.270148in}{0.804805in}}%
\pgfpathcurveto{\pgfqpoint{4.270148in}{0.815855in}}{\pgfqpoint{4.265758in}{0.826454in}}{\pgfqpoint{4.257945in}{0.834268in}}%
\pgfpathcurveto{\pgfqpoint{4.250131in}{0.842081in}}{\pgfqpoint{4.239532in}{0.846471in}}{\pgfqpoint{4.228482in}{0.846471in}}%
\pgfpathcurveto{\pgfqpoint{4.217432in}{0.846471in}}{\pgfqpoint{4.206833in}{0.842081in}}{\pgfqpoint{4.199019in}{0.834268in}}%
\pgfpathcurveto{\pgfqpoint{4.191205in}{0.826454in}}{\pgfqpoint{4.186815in}{0.815855in}}{\pgfqpoint{4.186815in}{0.804805in}}%
\pgfpathcurveto{\pgfqpoint{4.186815in}{0.793755in}}{\pgfqpoint{4.191205in}{0.783156in}}{\pgfqpoint{4.199019in}{0.775342in}}%
\pgfpathcurveto{\pgfqpoint{4.206833in}{0.767528in}}{\pgfqpoint{4.217432in}{0.763138in}}{\pgfqpoint{4.228482in}{0.763138in}}%
\pgfpathlineto{\pgfqpoint{4.228482in}{0.763138in}}%
\pgfpathclose%
\pgfusepath{stroke}%
\end{pgfscope}%
\begin{pgfscope}%
\pgfpathrectangle{\pgfqpoint{0.847223in}{0.554012in}}{\pgfqpoint{6.200000in}{4.620000in}}%
\pgfusepath{clip}%
\pgfsetbuttcap%
\pgfsetroundjoin%
\pgfsetlinewidth{1.003750pt}%
\definecolor{currentstroke}{rgb}{1.000000,0.000000,0.000000}%
\pgfsetstrokecolor{currentstroke}%
\pgfsetdash{}{0pt}%
\pgfpathmoveto{\pgfqpoint{4.233815in}{0.762216in}}%
\pgfpathcurveto{\pgfqpoint{4.244865in}{0.762216in}}{\pgfqpoint{4.255464in}{0.766606in}}{\pgfqpoint{4.263278in}{0.774420in}}%
\pgfpathcurveto{\pgfqpoint{4.271091in}{0.782233in}}{\pgfqpoint{4.275482in}{0.792832in}}{\pgfqpoint{4.275482in}{0.803882in}}%
\pgfpathcurveto{\pgfqpoint{4.275482in}{0.814932in}}{\pgfqpoint{4.271091in}{0.825531in}}{\pgfqpoint{4.263278in}{0.833345in}}%
\pgfpathcurveto{\pgfqpoint{4.255464in}{0.841159in}}{\pgfqpoint{4.244865in}{0.845549in}}{\pgfqpoint{4.233815in}{0.845549in}}%
\pgfpathcurveto{\pgfqpoint{4.222765in}{0.845549in}}{\pgfqpoint{4.212166in}{0.841159in}}{\pgfqpoint{4.204352in}{0.833345in}}%
\pgfpathcurveto{\pgfqpoint{4.196539in}{0.825531in}}{\pgfqpoint{4.192148in}{0.814932in}}{\pgfqpoint{4.192148in}{0.803882in}}%
\pgfpathcurveto{\pgfqpoint{4.192148in}{0.792832in}}{\pgfqpoint{4.196539in}{0.782233in}}{\pgfqpoint{4.204352in}{0.774420in}}%
\pgfpathcurveto{\pgfqpoint{4.212166in}{0.766606in}}{\pgfqpoint{4.222765in}{0.762216in}}{\pgfqpoint{4.233815in}{0.762216in}}%
\pgfpathlineto{\pgfqpoint{4.233815in}{0.762216in}}%
\pgfpathclose%
\pgfusepath{stroke}%
\end{pgfscope}%
\begin{pgfscope}%
\pgfpathrectangle{\pgfqpoint{0.847223in}{0.554012in}}{\pgfqpoint{6.200000in}{4.620000in}}%
\pgfusepath{clip}%
\pgfsetbuttcap%
\pgfsetroundjoin%
\pgfsetlinewidth{1.003750pt}%
\definecolor{currentstroke}{rgb}{1.000000,0.000000,0.000000}%
\pgfsetstrokecolor{currentstroke}%
\pgfsetdash{}{0pt}%
\pgfpathmoveto{\pgfqpoint{4.239148in}{0.761296in}}%
\pgfpathcurveto{\pgfqpoint{4.250198in}{0.761296in}}{\pgfqpoint{4.260797in}{0.765686in}}{\pgfqpoint{4.268611in}{0.773500in}}%
\pgfpathcurveto{\pgfqpoint{4.276425in}{0.781313in}}{\pgfqpoint{4.280815in}{0.791912in}}{\pgfqpoint{4.280815in}{0.802962in}}%
\pgfpathcurveto{\pgfqpoint{4.280815in}{0.814013in}}{\pgfqpoint{4.276425in}{0.824612in}}{\pgfqpoint{4.268611in}{0.832425in}}%
\pgfpathcurveto{\pgfqpoint{4.260797in}{0.840239in}}{\pgfqpoint{4.250198in}{0.844629in}}{\pgfqpoint{4.239148in}{0.844629in}}%
\pgfpathcurveto{\pgfqpoint{4.228098in}{0.844629in}}{\pgfqpoint{4.217499in}{0.840239in}}{\pgfqpoint{4.209685in}{0.832425in}}%
\pgfpathcurveto{\pgfqpoint{4.201872in}{0.824612in}}{\pgfqpoint{4.197482in}{0.814013in}}{\pgfqpoint{4.197482in}{0.802962in}}%
\pgfpathcurveto{\pgfqpoint{4.197482in}{0.791912in}}{\pgfqpoint{4.201872in}{0.781313in}}{\pgfqpoint{4.209685in}{0.773500in}}%
\pgfpathcurveto{\pgfqpoint{4.217499in}{0.765686in}}{\pgfqpoint{4.228098in}{0.761296in}}{\pgfqpoint{4.239148in}{0.761296in}}%
\pgfpathlineto{\pgfqpoint{4.239148in}{0.761296in}}%
\pgfpathclose%
\pgfusepath{stroke}%
\end{pgfscope}%
\begin{pgfscope}%
\pgfpathrectangle{\pgfqpoint{0.847223in}{0.554012in}}{\pgfqpoint{6.200000in}{4.620000in}}%
\pgfusepath{clip}%
\pgfsetbuttcap%
\pgfsetroundjoin%
\pgfsetlinewidth{1.003750pt}%
\definecolor{currentstroke}{rgb}{1.000000,0.000000,0.000000}%
\pgfsetstrokecolor{currentstroke}%
\pgfsetdash{}{0pt}%
\pgfpathmoveto{\pgfqpoint{4.244481in}{0.760378in}}%
\pgfpathcurveto{\pgfqpoint{4.255532in}{0.760378in}}{\pgfqpoint{4.266131in}{0.764769in}}{\pgfqpoint{4.273944in}{0.772582in}}%
\pgfpathcurveto{\pgfqpoint{4.281758in}{0.780396in}}{\pgfqpoint{4.286148in}{0.790995in}}{\pgfqpoint{4.286148in}{0.802045in}}%
\pgfpathcurveto{\pgfqpoint{4.286148in}{0.813095in}}{\pgfqpoint{4.281758in}{0.823694in}}{\pgfqpoint{4.273944in}{0.831508in}}%
\pgfpathcurveto{\pgfqpoint{4.266131in}{0.839321in}}{\pgfqpoint{4.255532in}{0.843712in}}{\pgfqpoint{4.244481in}{0.843712in}}%
\pgfpathcurveto{\pgfqpoint{4.233431in}{0.843712in}}{\pgfqpoint{4.222832in}{0.839321in}}{\pgfqpoint{4.215019in}{0.831508in}}%
\pgfpathcurveto{\pgfqpoint{4.207205in}{0.823694in}}{\pgfqpoint{4.202815in}{0.813095in}}{\pgfqpoint{4.202815in}{0.802045in}}%
\pgfpathcurveto{\pgfqpoint{4.202815in}{0.790995in}}{\pgfqpoint{4.207205in}{0.780396in}}{\pgfqpoint{4.215019in}{0.772582in}}%
\pgfpathcurveto{\pgfqpoint{4.222832in}{0.764769in}}{\pgfqpoint{4.233431in}{0.760378in}}{\pgfqpoint{4.244481in}{0.760378in}}%
\pgfpathlineto{\pgfqpoint{4.244481in}{0.760378in}}%
\pgfpathclose%
\pgfusepath{stroke}%
\end{pgfscope}%
\begin{pgfscope}%
\pgfpathrectangle{\pgfqpoint{0.847223in}{0.554012in}}{\pgfqpoint{6.200000in}{4.620000in}}%
\pgfusepath{clip}%
\pgfsetbuttcap%
\pgfsetroundjoin%
\pgfsetlinewidth{1.003750pt}%
\definecolor{currentstroke}{rgb}{1.000000,0.000000,0.000000}%
\pgfsetstrokecolor{currentstroke}%
\pgfsetdash{}{0pt}%
\pgfpathmoveto{\pgfqpoint{4.249815in}{0.759463in}}%
\pgfpathcurveto{\pgfqpoint{4.260865in}{0.759463in}}{\pgfqpoint{4.271464in}{0.763854in}}{\pgfqpoint{4.279277in}{0.771667in}}%
\pgfpathcurveto{\pgfqpoint{4.287091in}{0.779481in}}{\pgfqpoint{4.291481in}{0.790080in}}{\pgfqpoint{4.291481in}{0.801130in}}%
\pgfpathcurveto{\pgfqpoint{4.291481in}{0.812180in}}{\pgfqpoint{4.287091in}{0.822779in}}{\pgfqpoint{4.279277in}{0.830593in}}%
\pgfpathcurveto{\pgfqpoint{4.271464in}{0.838406in}}{\pgfqpoint{4.260865in}{0.842797in}}{\pgfqpoint{4.249815in}{0.842797in}}%
\pgfpathcurveto{\pgfqpoint{4.238765in}{0.842797in}}{\pgfqpoint{4.228165in}{0.838406in}}{\pgfqpoint{4.220352in}{0.830593in}}%
\pgfpathcurveto{\pgfqpoint{4.212538in}{0.822779in}}{\pgfqpoint{4.208148in}{0.812180in}}{\pgfqpoint{4.208148in}{0.801130in}}%
\pgfpathcurveto{\pgfqpoint{4.208148in}{0.790080in}}{\pgfqpoint{4.212538in}{0.779481in}}{\pgfqpoint{4.220352in}{0.771667in}}%
\pgfpathcurveto{\pgfqpoint{4.228165in}{0.763854in}}{\pgfqpoint{4.238765in}{0.759463in}}{\pgfqpoint{4.249815in}{0.759463in}}%
\pgfpathlineto{\pgfqpoint{4.249815in}{0.759463in}}%
\pgfpathclose%
\pgfusepath{stroke}%
\end{pgfscope}%
\begin{pgfscope}%
\pgfpathrectangle{\pgfqpoint{0.847223in}{0.554012in}}{\pgfqpoint{6.200000in}{4.620000in}}%
\pgfusepath{clip}%
\pgfsetbuttcap%
\pgfsetroundjoin%
\pgfsetlinewidth{1.003750pt}%
\definecolor{currentstroke}{rgb}{1.000000,0.000000,0.000000}%
\pgfsetstrokecolor{currentstroke}%
\pgfsetdash{}{0pt}%
\pgfpathmoveto{\pgfqpoint{4.255148in}{0.758551in}}%
\pgfpathcurveto{\pgfqpoint{4.266198in}{0.758551in}}{\pgfqpoint{4.276797in}{0.762941in}}{\pgfqpoint{4.284611in}{0.770755in}}%
\pgfpathcurveto{\pgfqpoint{4.292424in}{0.778568in}}{\pgfqpoint{4.296815in}{0.789167in}}{\pgfqpoint{4.296815in}{0.800218in}}%
\pgfpathcurveto{\pgfqpoint{4.296815in}{0.811268in}}{\pgfqpoint{4.292424in}{0.821867in}}{\pgfqpoint{4.284611in}{0.829680in}}%
\pgfpathcurveto{\pgfqpoint{4.276797in}{0.837494in}}{\pgfqpoint{4.266198in}{0.841884in}}{\pgfqpoint{4.255148in}{0.841884in}}%
\pgfpathcurveto{\pgfqpoint{4.244098in}{0.841884in}}{\pgfqpoint{4.233499in}{0.837494in}}{\pgfqpoint{4.225685in}{0.829680in}}%
\pgfpathcurveto{\pgfqpoint{4.217871in}{0.821867in}}{\pgfqpoint{4.213481in}{0.811268in}}{\pgfqpoint{4.213481in}{0.800218in}}%
\pgfpathcurveto{\pgfqpoint{4.213481in}{0.789167in}}{\pgfqpoint{4.217871in}{0.778568in}}{\pgfqpoint{4.225685in}{0.770755in}}%
\pgfpathcurveto{\pgfqpoint{4.233499in}{0.762941in}}{\pgfqpoint{4.244098in}{0.758551in}}{\pgfqpoint{4.255148in}{0.758551in}}%
\pgfpathlineto{\pgfqpoint{4.255148in}{0.758551in}}%
\pgfpathclose%
\pgfusepath{stroke}%
\end{pgfscope}%
\begin{pgfscope}%
\pgfpathrectangle{\pgfqpoint{0.847223in}{0.554012in}}{\pgfqpoint{6.200000in}{4.620000in}}%
\pgfusepath{clip}%
\pgfsetbuttcap%
\pgfsetroundjoin%
\pgfsetlinewidth{1.003750pt}%
\definecolor{currentstroke}{rgb}{1.000000,0.000000,0.000000}%
\pgfsetstrokecolor{currentstroke}%
\pgfsetdash{}{0pt}%
\pgfpathmoveto{\pgfqpoint{4.260481in}{0.757641in}}%
\pgfpathcurveto{\pgfqpoint{4.271531in}{0.757641in}}{\pgfqpoint{4.282130in}{0.762031in}}{\pgfqpoint{4.289944in}{0.769845in}}%
\pgfpathcurveto{\pgfqpoint{4.297757in}{0.777658in}}{\pgfqpoint{4.302148in}{0.788258in}}{\pgfqpoint{4.302148in}{0.799308in}}%
\pgfpathcurveto{\pgfqpoint{4.302148in}{0.810358in}}{\pgfqpoint{4.297757in}{0.820957in}}{\pgfqpoint{4.289944in}{0.828770in}}%
\pgfpathcurveto{\pgfqpoint{4.282130in}{0.836584in}}{\pgfqpoint{4.271531in}{0.840974in}}{\pgfqpoint{4.260481in}{0.840974in}}%
\pgfpathcurveto{\pgfqpoint{4.249431in}{0.840974in}}{\pgfqpoint{4.238832in}{0.836584in}}{\pgfqpoint{4.231018in}{0.828770in}}%
\pgfpathcurveto{\pgfqpoint{4.223205in}{0.820957in}}{\pgfqpoint{4.218814in}{0.810358in}}{\pgfqpoint{4.218814in}{0.799308in}}%
\pgfpathcurveto{\pgfqpoint{4.218814in}{0.788258in}}{\pgfqpoint{4.223205in}{0.777658in}}{\pgfqpoint{4.231018in}{0.769845in}}%
\pgfpathcurveto{\pgfqpoint{4.238832in}{0.762031in}}{\pgfqpoint{4.249431in}{0.757641in}}{\pgfqpoint{4.260481in}{0.757641in}}%
\pgfpathlineto{\pgfqpoint{4.260481in}{0.757641in}}%
\pgfpathclose%
\pgfusepath{stroke}%
\end{pgfscope}%
\begin{pgfscope}%
\pgfpathrectangle{\pgfqpoint{0.847223in}{0.554012in}}{\pgfqpoint{6.200000in}{4.620000in}}%
\pgfusepath{clip}%
\pgfsetbuttcap%
\pgfsetroundjoin%
\pgfsetlinewidth{1.003750pt}%
\definecolor{currentstroke}{rgb}{1.000000,0.000000,0.000000}%
\pgfsetstrokecolor{currentstroke}%
\pgfsetdash{}{0pt}%
\pgfpathmoveto{\pgfqpoint{4.265814in}{0.756733in}}%
\pgfpathcurveto{\pgfqpoint{4.276864in}{0.756733in}}{\pgfqpoint{4.287463in}{0.761124in}}{\pgfqpoint{4.295277in}{0.768937in}}%
\pgfpathcurveto{\pgfqpoint{4.303091in}{0.776751in}}{\pgfqpoint{4.307481in}{0.787350in}}{\pgfqpoint{4.307481in}{0.798400in}}%
\pgfpathcurveto{\pgfqpoint{4.307481in}{0.809450in}}{\pgfqpoint{4.303091in}{0.820049in}}{\pgfqpoint{4.295277in}{0.827863in}}%
\pgfpathcurveto{\pgfqpoint{4.287463in}{0.835677in}}{\pgfqpoint{4.276864in}{0.840067in}}{\pgfqpoint{4.265814in}{0.840067in}}%
\pgfpathcurveto{\pgfqpoint{4.254764in}{0.840067in}}{\pgfqpoint{4.244165in}{0.835677in}}{\pgfqpoint{4.236352in}{0.827863in}}%
\pgfpathcurveto{\pgfqpoint{4.228538in}{0.820049in}}{\pgfqpoint{4.224148in}{0.809450in}}{\pgfqpoint{4.224148in}{0.798400in}}%
\pgfpathcurveto{\pgfqpoint{4.224148in}{0.787350in}}{\pgfqpoint{4.228538in}{0.776751in}}{\pgfqpoint{4.236352in}{0.768937in}}%
\pgfpathcurveto{\pgfqpoint{4.244165in}{0.761124in}}{\pgfqpoint{4.254764in}{0.756733in}}{\pgfqpoint{4.265814in}{0.756733in}}%
\pgfpathlineto{\pgfqpoint{4.265814in}{0.756733in}}%
\pgfpathclose%
\pgfusepath{stroke}%
\end{pgfscope}%
\begin{pgfscope}%
\pgfpathrectangle{\pgfqpoint{0.847223in}{0.554012in}}{\pgfqpoint{6.200000in}{4.620000in}}%
\pgfusepath{clip}%
\pgfsetbuttcap%
\pgfsetroundjoin%
\pgfsetlinewidth{1.003750pt}%
\definecolor{currentstroke}{rgb}{1.000000,0.000000,0.000000}%
\pgfsetstrokecolor{currentstroke}%
\pgfsetdash{}{0pt}%
\pgfpathmoveto{\pgfqpoint{4.271148in}{0.755828in}}%
\pgfpathcurveto{\pgfqpoint{4.282198in}{0.755828in}}{\pgfqpoint{4.292797in}{0.760219in}}{\pgfqpoint{4.300610in}{0.768032in}}%
\pgfpathcurveto{\pgfqpoint{4.308424in}{0.775846in}}{\pgfqpoint{4.312814in}{0.786445in}}{\pgfqpoint{4.312814in}{0.797495in}}%
\pgfpathcurveto{\pgfqpoint{4.312814in}{0.808545in}}{\pgfqpoint{4.308424in}{0.819144in}}{\pgfqpoint{4.300610in}{0.826958in}}%
\pgfpathcurveto{\pgfqpoint{4.292797in}{0.834771in}}{\pgfqpoint{4.282198in}{0.839162in}}{\pgfqpoint{4.271148in}{0.839162in}}%
\pgfpathcurveto{\pgfqpoint{4.260097in}{0.839162in}}{\pgfqpoint{4.249498in}{0.834771in}}{\pgfqpoint{4.241685in}{0.826958in}}%
\pgfpathcurveto{\pgfqpoint{4.233871in}{0.819144in}}{\pgfqpoint{4.229481in}{0.808545in}}{\pgfqpoint{4.229481in}{0.797495in}}%
\pgfpathcurveto{\pgfqpoint{4.229481in}{0.786445in}}{\pgfqpoint{4.233871in}{0.775846in}}{\pgfqpoint{4.241685in}{0.768032in}}%
\pgfpathcurveto{\pgfqpoint{4.249498in}{0.760219in}}{\pgfqpoint{4.260097in}{0.755828in}}{\pgfqpoint{4.271148in}{0.755828in}}%
\pgfpathlineto{\pgfqpoint{4.271148in}{0.755828in}}%
\pgfpathclose%
\pgfusepath{stroke}%
\end{pgfscope}%
\begin{pgfscope}%
\pgfpathrectangle{\pgfqpoint{0.847223in}{0.554012in}}{\pgfqpoint{6.200000in}{4.620000in}}%
\pgfusepath{clip}%
\pgfsetbuttcap%
\pgfsetroundjoin%
\pgfsetlinewidth{1.003750pt}%
\definecolor{currentstroke}{rgb}{1.000000,0.000000,0.000000}%
\pgfsetstrokecolor{currentstroke}%
\pgfsetdash{}{0pt}%
\pgfpathmoveto{\pgfqpoint{4.276481in}{0.754926in}}%
\pgfpathcurveto{\pgfqpoint{4.287531in}{0.754926in}}{\pgfqpoint{4.298130in}{0.759316in}}{\pgfqpoint{4.305944in}{0.767130in}}%
\pgfpathcurveto{\pgfqpoint{4.313757in}{0.774943in}}{\pgfqpoint{4.318147in}{0.785542in}}{\pgfqpoint{4.318147in}{0.796592in}}%
\pgfpathcurveto{\pgfqpoint{4.318147in}{0.807643in}}{\pgfqpoint{4.313757in}{0.818242in}}{\pgfqpoint{4.305944in}{0.826055in}}%
\pgfpathcurveto{\pgfqpoint{4.298130in}{0.833869in}}{\pgfqpoint{4.287531in}{0.838259in}}{\pgfqpoint{4.276481in}{0.838259in}}%
\pgfpathcurveto{\pgfqpoint{4.265431in}{0.838259in}}{\pgfqpoint{4.254832in}{0.833869in}}{\pgfqpoint{4.247018in}{0.826055in}}%
\pgfpathcurveto{\pgfqpoint{4.239204in}{0.818242in}}{\pgfqpoint{4.234814in}{0.807643in}}{\pgfqpoint{4.234814in}{0.796592in}}%
\pgfpathcurveto{\pgfqpoint{4.234814in}{0.785542in}}{\pgfqpoint{4.239204in}{0.774943in}}{\pgfqpoint{4.247018in}{0.767130in}}%
\pgfpathcurveto{\pgfqpoint{4.254832in}{0.759316in}}{\pgfqpoint{4.265431in}{0.754926in}}{\pgfqpoint{4.276481in}{0.754926in}}%
\pgfpathlineto{\pgfqpoint{4.276481in}{0.754926in}}%
\pgfpathclose%
\pgfusepath{stroke}%
\end{pgfscope}%
\begin{pgfscope}%
\pgfpathrectangle{\pgfqpoint{0.847223in}{0.554012in}}{\pgfqpoint{6.200000in}{4.620000in}}%
\pgfusepath{clip}%
\pgfsetbuttcap%
\pgfsetroundjoin%
\pgfsetlinewidth{1.003750pt}%
\definecolor{currentstroke}{rgb}{1.000000,0.000000,0.000000}%
\pgfsetstrokecolor{currentstroke}%
\pgfsetdash{}{0pt}%
\pgfpathmoveto{\pgfqpoint{4.281814in}{0.754026in}}%
\pgfpathcurveto{\pgfqpoint{4.292864in}{0.754026in}}{\pgfqpoint{4.303463in}{0.758416in}}{\pgfqpoint{4.311277in}{0.766230in}}%
\pgfpathcurveto{\pgfqpoint{4.319090in}{0.774043in}}{\pgfqpoint{4.323481in}{0.784642in}}{\pgfqpoint{4.323481in}{0.795692in}}%
\pgfpathcurveto{\pgfqpoint{4.323481in}{0.806742in}}{\pgfqpoint{4.319090in}{0.817341in}}{\pgfqpoint{4.311277in}{0.825155in}}%
\pgfpathcurveto{\pgfqpoint{4.303463in}{0.832969in}}{\pgfqpoint{4.292864in}{0.837359in}}{\pgfqpoint{4.281814in}{0.837359in}}%
\pgfpathcurveto{\pgfqpoint{4.270764in}{0.837359in}}{\pgfqpoint{4.260165in}{0.832969in}}{\pgfqpoint{4.252351in}{0.825155in}}%
\pgfpathcurveto{\pgfqpoint{4.244538in}{0.817341in}}{\pgfqpoint{4.240147in}{0.806742in}}{\pgfqpoint{4.240147in}{0.795692in}}%
\pgfpathcurveto{\pgfqpoint{4.240147in}{0.784642in}}{\pgfqpoint{4.244538in}{0.774043in}}{\pgfqpoint{4.252351in}{0.766230in}}%
\pgfpathcurveto{\pgfqpoint{4.260165in}{0.758416in}}{\pgfqpoint{4.270764in}{0.754026in}}{\pgfqpoint{4.281814in}{0.754026in}}%
\pgfpathlineto{\pgfqpoint{4.281814in}{0.754026in}}%
\pgfpathclose%
\pgfusepath{stroke}%
\end{pgfscope}%
\begin{pgfscope}%
\pgfpathrectangle{\pgfqpoint{0.847223in}{0.554012in}}{\pgfqpoint{6.200000in}{4.620000in}}%
\pgfusepath{clip}%
\pgfsetbuttcap%
\pgfsetroundjoin%
\pgfsetlinewidth{1.003750pt}%
\definecolor{currentstroke}{rgb}{1.000000,0.000000,0.000000}%
\pgfsetstrokecolor{currentstroke}%
\pgfsetdash{}{0pt}%
\pgfpathmoveto{\pgfqpoint{4.287147in}{0.753128in}}%
\pgfpathcurveto{\pgfqpoint{4.298197in}{0.753128in}}{\pgfqpoint{4.308796in}{0.757518in}}{\pgfqpoint{4.316610in}{0.765332in}}%
\pgfpathcurveto{\pgfqpoint{4.324424in}{0.773145in}}{\pgfqpoint{4.328814in}{0.783744in}}{\pgfqpoint{4.328814in}{0.794795in}}%
\pgfpathcurveto{\pgfqpoint{4.328814in}{0.805845in}}{\pgfqpoint{4.324424in}{0.816444in}}{\pgfqpoint{4.316610in}{0.824257in}}%
\pgfpathcurveto{\pgfqpoint{4.308796in}{0.832071in}}{\pgfqpoint{4.298197in}{0.836461in}}{\pgfqpoint{4.287147in}{0.836461in}}%
\pgfpathcurveto{\pgfqpoint{4.276097in}{0.836461in}}{\pgfqpoint{4.265498in}{0.832071in}}{\pgfqpoint{4.257684in}{0.824257in}}%
\pgfpathcurveto{\pgfqpoint{4.249871in}{0.816444in}}{\pgfqpoint{4.245480in}{0.805845in}}{\pgfqpoint{4.245480in}{0.794795in}}%
\pgfpathcurveto{\pgfqpoint{4.245480in}{0.783744in}}{\pgfqpoint{4.249871in}{0.773145in}}{\pgfqpoint{4.257684in}{0.765332in}}%
\pgfpathcurveto{\pgfqpoint{4.265498in}{0.757518in}}{\pgfqpoint{4.276097in}{0.753128in}}{\pgfqpoint{4.287147in}{0.753128in}}%
\pgfpathlineto{\pgfqpoint{4.287147in}{0.753128in}}%
\pgfpathclose%
\pgfusepath{stroke}%
\end{pgfscope}%
\begin{pgfscope}%
\pgfpathrectangle{\pgfqpoint{0.847223in}{0.554012in}}{\pgfqpoint{6.200000in}{4.620000in}}%
\pgfusepath{clip}%
\pgfsetbuttcap%
\pgfsetroundjoin%
\pgfsetlinewidth{1.003750pt}%
\definecolor{currentstroke}{rgb}{1.000000,0.000000,0.000000}%
\pgfsetstrokecolor{currentstroke}%
\pgfsetdash{}{0pt}%
\pgfpathmoveto{\pgfqpoint{4.292480in}{0.752233in}}%
\pgfpathcurveto{\pgfqpoint{4.303531in}{0.752233in}}{\pgfqpoint{4.314130in}{0.756623in}}{\pgfqpoint{4.321943in}{0.764436in}}%
\pgfpathcurveto{\pgfqpoint{4.329757in}{0.772250in}}{\pgfqpoint{4.334147in}{0.782849in}}{\pgfqpoint{4.334147in}{0.793899in}}%
\pgfpathcurveto{\pgfqpoint{4.334147in}{0.804949in}}{\pgfqpoint{4.329757in}{0.815548in}}{\pgfqpoint{4.321943in}{0.823362in}}%
\pgfpathcurveto{\pgfqpoint{4.314130in}{0.831176in}}{\pgfqpoint{4.303531in}{0.835566in}}{\pgfqpoint{4.292480in}{0.835566in}}%
\pgfpathcurveto{\pgfqpoint{4.281430in}{0.835566in}}{\pgfqpoint{4.270831in}{0.831176in}}{\pgfqpoint{4.263018in}{0.823362in}}%
\pgfpathcurveto{\pgfqpoint{4.255204in}{0.815548in}}{\pgfqpoint{4.250814in}{0.804949in}}{\pgfqpoint{4.250814in}{0.793899in}}%
\pgfpathcurveto{\pgfqpoint{4.250814in}{0.782849in}}{\pgfqpoint{4.255204in}{0.772250in}}{\pgfqpoint{4.263018in}{0.764436in}}%
\pgfpathcurveto{\pgfqpoint{4.270831in}{0.756623in}}{\pgfqpoint{4.281430in}{0.752233in}}{\pgfqpoint{4.292480in}{0.752233in}}%
\pgfpathlineto{\pgfqpoint{4.292480in}{0.752233in}}%
\pgfpathclose%
\pgfusepath{stroke}%
\end{pgfscope}%
\begin{pgfscope}%
\pgfpathrectangle{\pgfqpoint{0.847223in}{0.554012in}}{\pgfqpoint{6.200000in}{4.620000in}}%
\pgfusepath{clip}%
\pgfsetbuttcap%
\pgfsetroundjoin%
\pgfsetlinewidth{1.003750pt}%
\definecolor{currentstroke}{rgb}{1.000000,0.000000,0.000000}%
\pgfsetstrokecolor{currentstroke}%
\pgfsetdash{}{0pt}%
\pgfpathmoveto{\pgfqpoint{4.297814in}{0.751340in}}%
\pgfpathcurveto{\pgfqpoint{4.308864in}{0.751340in}}{\pgfqpoint{4.319463in}{0.755730in}}{\pgfqpoint{4.327276in}{0.763543in}}%
\pgfpathcurveto{\pgfqpoint{4.335090in}{0.771357in}}{\pgfqpoint{4.339480in}{0.781956in}}{\pgfqpoint{4.339480in}{0.793006in}}%
\pgfpathcurveto{\pgfqpoint{4.339480in}{0.804056in}}{\pgfqpoint{4.335090in}{0.814655in}}{\pgfqpoint{4.327276in}{0.822469in}}%
\pgfpathcurveto{\pgfqpoint{4.319463in}{0.830283in}}{\pgfqpoint{4.308864in}{0.834673in}}{\pgfqpoint{4.297814in}{0.834673in}}%
\pgfpathcurveto{\pgfqpoint{4.286763in}{0.834673in}}{\pgfqpoint{4.276164in}{0.830283in}}{\pgfqpoint{4.268351in}{0.822469in}}%
\pgfpathcurveto{\pgfqpoint{4.260537in}{0.814655in}}{\pgfqpoint{4.256147in}{0.804056in}}{\pgfqpoint{4.256147in}{0.793006in}}%
\pgfpathcurveto{\pgfqpoint{4.256147in}{0.781956in}}{\pgfqpoint{4.260537in}{0.771357in}}{\pgfqpoint{4.268351in}{0.763543in}}%
\pgfpathcurveto{\pgfqpoint{4.276164in}{0.755730in}}{\pgfqpoint{4.286763in}{0.751340in}}{\pgfqpoint{4.297814in}{0.751340in}}%
\pgfpathlineto{\pgfqpoint{4.297814in}{0.751340in}}%
\pgfpathclose%
\pgfusepath{stroke}%
\end{pgfscope}%
\begin{pgfscope}%
\pgfpathrectangle{\pgfqpoint{0.847223in}{0.554012in}}{\pgfqpoint{6.200000in}{4.620000in}}%
\pgfusepath{clip}%
\pgfsetbuttcap%
\pgfsetroundjoin%
\pgfsetlinewidth{1.003750pt}%
\definecolor{currentstroke}{rgb}{1.000000,0.000000,0.000000}%
\pgfsetstrokecolor{currentstroke}%
\pgfsetdash{}{0pt}%
\pgfpathmoveto{\pgfqpoint{4.303147in}{0.750449in}}%
\pgfpathcurveto{\pgfqpoint{4.314197in}{0.750449in}}{\pgfqpoint{4.324796in}{0.754839in}}{\pgfqpoint{4.332610in}{0.762653in}}%
\pgfpathcurveto{\pgfqpoint{4.340423in}{0.770467in}}{\pgfqpoint{4.344813in}{0.781066in}}{\pgfqpoint{4.344813in}{0.792116in}}%
\pgfpathcurveto{\pgfqpoint{4.344813in}{0.803166in}}{\pgfqpoint{4.340423in}{0.813765in}}{\pgfqpoint{4.332610in}{0.821578in}}%
\pgfpathcurveto{\pgfqpoint{4.324796in}{0.829392in}}{\pgfqpoint{4.314197in}{0.833782in}}{\pgfqpoint{4.303147in}{0.833782in}}%
\pgfpathcurveto{\pgfqpoint{4.292097in}{0.833782in}}{\pgfqpoint{4.281498in}{0.829392in}}{\pgfqpoint{4.273684in}{0.821578in}}%
\pgfpathcurveto{\pgfqpoint{4.265870in}{0.813765in}}{\pgfqpoint{4.261480in}{0.803166in}}{\pgfqpoint{4.261480in}{0.792116in}}%
\pgfpathcurveto{\pgfqpoint{4.261480in}{0.781066in}}{\pgfqpoint{4.265870in}{0.770467in}}{\pgfqpoint{4.273684in}{0.762653in}}%
\pgfpathcurveto{\pgfqpoint{4.281498in}{0.754839in}}{\pgfqpoint{4.292097in}{0.750449in}}{\pgfqpoint{4.303147in}{0.750449in}}%
\pgfpathlineto{\pgfqpoint{4.303147in}{0.750449in}}%
\pgfpathclose%
\pgfusepath{stroke}%
\end{pgfscope}%
\begin{pgfscope}%
\pgfpathrectangle{\pgfqpoint{0.847223in}{0.554012in}}{\pgfqpoint{6.200000in}{4.620000in}}%
\pgfusepath{clip}%
\pgfsetbuttcap%
\pgfsetroundjoin%
\pgfsetlinewidth{1.003750pt}%
\definecolor{currentstroke}{rgb}{1.000000,0.000000,0.000000}%
\pgfsetstrokecolor{currentstroke}%
\pgfsetdash{}{0pt}%
\pgfpathmoveto{\pgfqpoint{4.308480in}{0.749561in}}%
\pgfpathcurveto{\pgfqpoint{4.319530in}{0.749561in}}{\pgfqpoint{4.330129in}{0.753951in}}{\pgfqpoint{4.337943in}{0.761765in}}%
\pgfpathcurveto{\pgfqpoint{4.345756in}{0.769578in}}{\pgfqpoint{4.350147in}{0.780177in}}{\pgfqpoint{4.350147in}{0.791228in}}%
\pgfpathcurveto{\pgfqpoint{4.350147in}{0.802278in}}{\pgfqpoint{4.345756in}{0.812877in}}{\pgfqpoint{4.337943in}{0.820690in}}%
\pgfpathcurveto{\pgfqpoint{4.330129in}{0.828504in}}{\pgfqpoint{4.319530in}{0.832894in}}{\pgfqpoint{4.308480in}{0.832894in}}%
\pgfpathcurveto{\pgfqpoint{4.297430in}{0.832894in}}{\pgfqpoint{4.286831in}{0.828504in}}{\pgfqpoint{4.279017in}{0.820690in}}%
\pgfpathcurveto{\pgfqpoint{4.271204in}{0.812877in}}{\pgfqpoint{4.266813in}{0.802278in}}{\pgfqpoint{4.266813in}{0.791228in}}%
\pgfpathcurveto{\pgfqpoint{4.266813in}{0.780177in}}{\pgfqpoint{4.271204in}{0.769578in}}{\pgfqpoint{4.279017in}{0.761765in}}%
\pgfpathcurveto{\pgfqpoint{4.286831in}{0.753951in}}{\pgfqpoint{4.297430in}{0.749561in}}{\pgfqpoint{4.308480in}{0.749561in}}%
\pgfpathlineto{\pgfqpoint{4.308480in}{0.749561in}}%
\pgfpathclose%
\pgfusepath{stroke}%
\end{pgfscope}%
\begin{pgfscope}%
\pgfpathrectangle{\pgfqpoint{0.847223in}{0.554012in}}{\pgfqpoint{6.200000in}{4.620000in}}%
\pgfusepath{clip}%
\pgfsetbuttcap%
\pgfsetroundjoin%
\pgfsetlinewidth{1.003750pt}%
\definecolor{currentstroke}{rgb}{1.000000,0.000000,0.000000}%
\pgfsetstrokecolor{currentstroke}%
\pgfsetdash{}{0pt}%
\pgfpathmoveto{\pgfqpoint{4.313813in}{0.748675in}}%
\pgfpathcurveto{\pgfqpoint{4.324863in}{0.748675in}}{\pgfqpoint{4.335462in}{0.753065in}}{\pgfqpoint{4.343276in}{0.760879in}}%
\pgfpathcurveto{\pgfqpoint{4.351090in}{0.768693in}}{\pgfqpoint{4.355480in}{0.779292in}}{\pgfqpoint{4.355480in}{0.790342in}}%
\pgfpathcurveto{\pgfqpoint{4.355480in}{0.801392in}}{\pgfqpoint{4.351090in}{0.811991in}}{\pgfqpoint{4.343276in}{0.819805in}}%
\pgfpathcurveto{\pgfqpoint{4.335462in}{0.827618in}}{\pgfqpoint{4.324863in}{0.832008in}}{\pgfqpoint{4.313813in}{0.832008in}}%
\pgfpathcurveto{\pgfqpoint{4.302763in}{0.832008in}}{\pgfqpoint{4.292164in}{0.827618in}}{\pgfqpoint{4.284350in}{0.819805in}}%
\pgfpathcurveto{\pgfqpoint{4.276537in}{0.811991in}}{\pgfqpoint{4.272147in}{0.801392in}}{\pgfqpoint{4.272147in}{0.790342in}}%
\pgfpathcurveto{\pgfqpoint{4.272147in}{0.779292in}}{\pgfqpoint{4.276537in}{0.768693in}}{\pgfqpoint{4.284350in}{0.760879in}}%
\pgfpathcurveto{\pgfqpoint{4.292164in}{0.753065in}}{\pgfqpoint{4.302763in}{0.748675in}}{\pgfqpoint{4.313813in}{0.748675in}}%
\pgfpathlineto{\pgfqpoint{4.313813in}{0.748675in}}%
\pgfpathclose%
\pgfusepath{stroke}%
\end{pgfscope}%
\begin{pgfscope}%
\pgfpathrectangle{\pgfqpoint{0.847223in}{0.554012in}}{\pgfqpoint{6.200000in}{4.620000in}}%
\pgfusepath{clip}%
\pgfsetbuttcap%
\pgfsetroundjoin%
\pgfsetlinewidth{1.003750pt}%
\definecolor{currentstroke}{rgb}{1.000000,0.000000,0.000000}%
\pgfsetstrokecolor{currentstroke}%
\pgfsetdash{}{0pt}%
\pgfpathmoveto{\pgfqpoint{4.319146in}{0.747792in}}%
\pgfpathcurveto{\pgfqpoint{4.330197in}{0.747792in}}{\pgfqpoint{4.340796in}{0.752182in}}{\pgfqpoint{4.348609in}{0.759996in}}%
\pgfpathcurveto{\pgfqpoint{4.356423in}{0.767809in}}{\pgfqpoint{4.360813in}{0.778408in}}{\pgfqpoint{4.360813in}{0.789458in}}%
\pgfpathcurveto{\pgfqpoint{4.360813in}{0.800508in}}{\pgfqpoint{4.356423in}{0.811107in}}{\pgfqpoint{4.348609in}{0.818921in}}%
\pgfpathcurveto{\pgfqpoint{4.340796in}{0.826735in}}{\pgfqpoint{4.330197in}{0.831125in}}{\pgfqpoint{4.319146in}{0.831125in}}%
\pgfpathcurveto{\pgfqpoint{4.308096in}{0.831125in}}{\pgfqpoint{4.297497in}{0.826735in}}{\pgfqpoint{4.289684in}{0.818921in}}%
\pgfpathcurveto{\pgfqpoint{4.281870in}{0.811107in}}{\pgfqpoint{4.277480in}{0.800508in}}{\pgfqpoint{4.277480in}{0.789458in}}%
\pgfpathcurveto{\pgfqpoint{4.277480in}{0.778408in}}{\pgfqpoint{4.281870in}{0.767809in}}{\pgfqpoint{4.289684in}{0.759996in}}%
\pgfpathcurveto{\pgfqpoint{4.297497in}{0.752182in}}{\pgfqpoint{4.308096in}{0.747792in}}{\pgfqpoint{4.319146in}{0.747792in}}%
\pgfpathlineto{\pgfqpoint{4.319146in}{0.747792in}}%
\pgfpathclose%
\pgfusepath{stroke}%
\end{pgfscope}%
\begin{pgfscope}%
\pgfpathrectangle{\pgfqpoint{0.847223in}{0.554012in}}{\pgfqpoint{6.200000in}{4.620000in}}%
\pgfusepath{clip}%
\pgfsetbuttcap%
\pgfsetroundjoin%
\pgfsetlinewidth{1.003750pt}%
\definecolor{currentstroke}{rgb}{1.000000,0.000000,0.000000}%
\pgfsetstrokecolor{currentstroke}%
\pgfsetdash{}{0pt}%
\pgfpathmoveto{\pgfqpoint{4.324480in}{0.746911in}}%
\pgfpathcurveto{\pgfqpoint{4.335530in}{0.746911in}}{\pgfqpoint{4.346129in}{0.751301in}}{\pgfqpoint{4.353942in}{0.759114in}}%
\pgfpathcurveto{\pgfqpoint{4.361756in}{0.766928in}}{\pgfqpoint{4.366146in}{0.777527in}}{\pgfqpoint{4.366146in}{0.788577in}}%
\pgfpathcurveto{\pgfqpoint{4.366146in}{0.799627in}}{\pgfqpoint{4.361756in}{0.810226in}}{\pgfqpoint{4.353942in}{0.818040in}}%
\pgfpathcurveto{\pgfqpoint{4.346129in}{0.825854in}}{\pgfqpoint{4.335530in}{0.830244in}}{\pgfqpoint{4.324480in}{0.830244in}}%
\pgfpathcurveto{\pgfqpoint{4.313430in}{0.830244in}}{\pgfqpoint{4.302831in}{0.825854in}}{\pgfqpoint{4.295017in}{0.818040in}}%
\pgfpathcurveto{\pgfqpoint{4.287203in}{0.810226in}}{\pgfqpoint{4.282813in}{0.799627in}}{\pgfqpoint{4.282813in}{0.788577in}}%
\pgfpathcurveto{\pgfqpoint{4.282813in}{0.777527in}}{\pgfqpoint{4.287203in}{0.766928in}}{\pgfqpoint{4.295017in}{0.759114in}}%
\pgfpathcurveto{\pgfqpoint{4.302831in}{0.751301in}}{\pgfqpoint{4.313430in}{0.746911in}}{\pgfqpoint{4.324480in}{0.746911in}}%
\pgfpathlineto{\pgfqpoint{4.324480in}{0.746911in}}%
\pgfpathclose%
\pgfusepath{stroke}%
\end{pgfscope}%
\begin{pgfscope}%
\pgfpathrectangle{\pgfqpoint{0.847223in}{0.554012in}}{\pgfqpoint{6.200000in}{4.620000in}}%
\pgfusepath{clip}%
\pgfsetbuttcap%
\pgfsetroundjoin%
\pgfsetlinewidth{1.003750pt}%
\definecolor{currentstroke}{rgb}{1.000000,0.000000,0.000000}%
\pgfsetstrokecolor{currentstroke}%
\pgfsetdash{}{0pt}%
\pgfpathmoveto{\pgfqpoint{4.329813in}{0.746032in}}%
\pgfpathcurveto{\pgfqpoint{4.340863in}{0.746032in}}{\pgfqpoint{4.351462in}{0.750422in}}{\pgfqpoint{4.359276in}{0.758236in}}%
\pgfpathcurveto{\pgfqpoint{4.367089in}{0.766049in}}{\pgfqpoint{4.371480in}{0.776648in}}{\pgfqpoint{4.371480in}{0.787699in}}%
\pgfpathcurveto{\pgfqpoint{4.371480in}{0.798749in}}{\pgfqpoint{4.367089in}{0.809348in}}{\pgfqpoint{4.359276in}{0.817161in}}%
\pgfpathcurveto{\pgfqpoint{4.351462in}{0.824975in}}{\pgfqpoint{4.340863in}{0.829365in}}{\pgfqpoint{4.329813in}{0.829365in}}%
\pgfpathcurveto{\pgfqpoint{4.318763in}{0.829365in}}{\pgfqpoint{4.308164in}{0.824975in}}{\pgfqpoint{4.300350in}{0.817161in}}%
\pgfpathcurveto{\pgfqpoint{4.292536in}{0.809348in}}{\pgfqpoint{4.288146in}{0.798749in}}{\pgfqpoint{4.288146in}{0.787699in}}%
\pgfpathcurveto{\pgfqpoint{4.288146in}{0.776648in}}{\pgfqpoint{4.292536in}{0.766049in}}{\pgfqpoint{4.300350in}{0.758236in}}%
\pgfpathcurveto{\pgfqpoint{4.308164in}{0.750422in}}{\pgfqpoint{4.318763in}{0.746032in}}{\pgfqpoint{4.329813in}{0.746032in}}%
\pgfpathlineto{\pgfqpoint{4.329813in}{0.746032in}}%
\pgfpathclose%
\pgfusepath{stroke}%
\end{pgfscope}%
\begin{pgfscope}%
\pgfpathrectangle{\pgfqpoint{0.847223in}{0.554012in}}{\pgfqpoint{6.200000in}{4.620000in}}%
\pgfusepath{clip}%
\pgfsetbuttcap%
\pgfsetroundjoin%
\pgfsetlinewidth{1.003750pt}%
\definecolor{currentstroke}{rgb}{1.000000,0.000000,0.000000}%
\pgfsetstrokecolor{currentstroke}%
\pgfsetdash{}{0pt}%
\pgfpathmoveto{\pgfqpoint{4.335146in}{0.745155in}}%
\pgfpathcurveto{\pgfqpoint{4.346196in}{0.745155in}}{\pgfqpoint{4.356795in}{0.749546in}}{\pgfqpoint{4.364609in}{0.757359in}}%
\pgfpathcurveto{\pgfqpoint{4.372423in}{0.765173in}}{\pgfqpoint{4.376813in}{0.775772in}}{\pgfqpoint{4.376813in}{0.786822in}}%
\pgfpathcurveto{\pgfqpoint{4.376813in}{0.797872in}}{\pgfqpoint{4.372423in}{0.808471in}}{\pgfqpoint{4.364609in}{0.816285in}}%
\pgfpathcurveto{\pgfqpoint{4.356795in}{0.824099in}}{\pgfqpoint{4.346196in}{0.828489in}}{\pgfqpoint{4.335146in}{0.828489in}}%
\pgfpathcurveto{\pgfqpoint{4.324096in}{0.828489in}}{\pgfqpoint{4.313497in}{0.824099in}}{\pgfqpoint{4.305683in}{0.816285in}}%
\pgfpathcurveto{\pgfqpoint{4.297870in}{0.808471in}}{\pgfqpoint{4.293479in}{0.797872in}}{\pgfqpoint{4.293479in}{0.786822in}}%
\pgfpathcurveto{\pgfqpoint{4.293479in}{0.775772in}}{\pgfqpoint{4.297870in}{0.765173in}}{\pgfqpoint{4.305683in}{0.757359in}}%
\pgfpathcurveto{\pgfqpoint{4.313497in}{0.749546in}}{\pgfqpoint{4.324096in}{0.745155in}}{\pgfqpoint{4.335146in}{0.745155in}}%
\pgfpathlineto{\pgfqpoint{4.335146in}{0.745155in}}%
\pgfpathclose%
\pgfusepath{stroke}%
\end{pgfscope}%
\begin{pgfscope}%
\pgfpathrectangle{\pgfqpoint{0.847223in}{0.554012in}}{\pgfqpoint{6.200000in}{4.620000in}}%
\pgfusepath{clip}%
\pgfsetbuttcap%
\pgfsetroundjoin%
\pgfsetlinewidth{1.003750pt}%
\definecolor{currentstroke}{rgb}{1.000000,0.000000,0.000000}%
\pgfsetstrokecolor{currentstroke}%
\pgfsetdash{}{0pt}%
\pgfpathmoveto{\pgfqpoint{4.340479in}{0.744281in}}%
\pgfpathcurveto{\pgfqpoint{4.351529in}{0.744281in}}{\pgfqpoint{4.362128in}{0.748672in}}{\pgfqpoint{4.369942in}{0.756485in}}%
\pgfpathcurveto{\pgfqpoint{4.377756in}{0.764299in}}{\pgfqpoint{4.382146in}{0.774898in}}{\pgfqpoint{4.382146in}{0.785948in}}%
\pgfpathcurveto{\pgfqpoint{4.382146in}{0.796998in}}{\pgfqpoint{4.377756in}{0.807597in}}{\pgfqpoint{4.369942in}{0.815411in}}%
\pgfpathcurveto{\pgfqpoint{4.362128in}{0.823224in}}{\pgfqpoint{4.351529in}{0.827615in}}{\pgfqpoint{4.340479in}{0.827615in}}%
\pgfpathcurveto{\pgfqpoint{4.329429in}{0.827615in}}{\pgfqpoint{4.318830in}{0.823224in}}{\pgfqpoint{4.311017in}{0.815411in}}%
\pgfpathcurveto{\pgfqpoint{4.303203in}{0.807597in}}{\pgfqpoint{4.298813in}{0.796998in}}{\pgfqpoint{4.298813in}{0.785948in}}%
\pgfpathcurveto{\pgfqpoint{4.298813in}{0.774898in}}{\pgfqpoint{4.303203in}{0.764299in}}{\pgfqpoint{4.311017in}{0.756485in}}%
\pgfpathcurveto{\pgfqpoint{4.318830in}{0.748672in}}{\pgfqpoint{4.329429in}{0.744281in}}{\pgfqpoint{4.340479in}{0.744281in}}%
\pgfpathlineto{\pgfqpoint{4.340479in}{0.744281in}}%
\pgfpathclose%
\pgfusepath{stroke}%
\end{pgfscope}%
\begin{pgfscope}%
\pgfpathrectangle{\pgfqpoint{0.847223in}{0.554012in}}{\pgfqpoint{6.200000in}{4.620000in}}%
\pgfusepath{clip}%
\pgfsetbuttcap%
\pgfsetroundjoin%
\pgfsetlinewidth{1.003750pt}%
\definecolor{currentstroke}{rgb}{1.000000,0.000000,0.000000}%
\pgfsetstrokecolor{currentstroke}%
\pgfsetdash{}{0pt}%
\pgfpathmoveto{\pgfqpoint{4.345813in}{0.743410in}}%
\pgfpathcurveto{\pgfqpoint{4.356863in}{0.743410in}}{\pgfqpoint{4.367462in}{0.747800in}}{\pgfqpoint{4.375275in}{0.755614in}}%
\pgfpathcurveto{\pgfqpoint{4.383089in}{0.763427in}}{\pgfqpoint{4.387479in}{0.774026in}}{\pgfqpoint{4.387479in}{0.785076in}}%
\pgfpathcurveto{\pgfqpoint{4.387479in}{0.796126in}}{\pgfqpoint{4.383089in}{0.806725in}}{\pgfqpoint{4.375275in}{0.814539in}}%
\pgfpathcurveto{\pgfqpoint{4.367462in}{0.822353in}}{\pgfqpoint{4.356863in}{0.826743in}}{\pgfqpoint{4.345813in}{0.826743in}}%
\pgfpathcurveto{\pgfqpoint{4.334762in}{0.826743in}}{\pgfqpoint{4.324163in}{0.822353in}}{\pgfqpoint{4.316350in}{0.814539in}}%
\pgfpathcurveto{\pgfqpoint{4.308536in}{0.806725in}}{\pgfqpoint{4.304146in}{0.796126in}}{\pgfqpoint{4.304146in}{0.785076in}}%
\pgfpathcurveto{\pgfqpoint{4.304146in}{0.774026in}}{\pgfqpoint{4.308536in}{0.763427in}}{\pgfqpoint{4.316350in}{0.755614in}}%
\pgfpathcurveto{\pgfqpoint{4.324163in}{0.747800in}}{\pgfqpoint{4.334762in}{0.743410in}}{\pgfqpoint{4.345813in}{0.743410in}}%
\pgfpathlineto{\pgfqpoint{4.345813in}{0.743410in}}%
\pgfpathclose%
\pgfusepath{stroke}%
\end{pgfscope}%
\begin{pgfscope}%
\pgfpathrectangle{\pgfqpoint{0.847223in}{0.554012in}}{\pgfqpoint{6.200000in}{4.620000in}}%
\pgfusepath{clip}%
\pgfsetbuttcap%
\pgfsetroundjoin%
\pgfsetlinewidth{1.003750pt}%
\definecolor{currentstroke}{rgb}{1.000000,0.000000,0.000000}%
\pgfsetstrokecolor{currentstroke}%
\pgfsetdash{}{0pt}%
\pgfpathmoveto{\pgfqpoint{4.351146in}{0.742540in}}%
\pgfpathcurveto{\pgfqpoint{4.362196in}{0.742540in}}{\pgfqpoint{4.372795in}{0.746930in}}{\pgfqpoint{4.380609in}{0.754744in}}%
\pgfpathcurveto{\pgfqpoint{4.388422in}{0.762558in}}{\pgfqpoint{4.392812in}{0.773157in}}{\pgfqpoint{4.392812in}{0.784207in}}%
\pgfpathcurveto{\pgfqpoint{4.392812in}{0.795257in}}{\pgfqpoint{4.388422in}{0.805856in}}{\pgfqpoint{4.380609in}{0.813670in}}%
\pgfpathcurveto{\pgfqpoint{4.372795in}{0.821483in}}{\pgfqpoint{4.362196in}{0.825874in}}{\pgfqpoint{4.351146in}{0.825874in}}%
\pgfpathcurveto{\pgfqpoint{4.340096in}{0.825874in}}{\pgfqpoint{4.329497in}{0.821483in}}{\pgfqpoint{4.321683in}{0.813670in}}%
\pgfpathcurveto{\pgfqpoint{4.313869in}{0.805856in}}{\pgfqpoint{4.309479in}{0.795257in}}{\pgfqpoint{4.309479in}{0.784207in}}%
\pgfpathcurveto{\pgfqpoint{4.309479in}{0.773157in}}{\pgfqpoint{4.313869in}{0.762558in}}{\pgfqpoint{4.321683in}{0.754744in}}%
\pgfpathcurveto{\pgfqpoint{4.329497in}{0.746930in}}{\pgfqpoint{4.340096in}{0.742540in}}{\pgfqpoint{4.351146in}{0.742540in}}%
\pgfpathlineto{\pgfqpoint{4.351146in}{0.742540in}}%
\pgfpathclose%
\pgfusepath{stroke}%
\end{pgfscope}%
\begin{pgfscope}%
\pgfpathrectangle{\pgfqpoint{0.847223in}{0.554012in}}{\pgfqpoint{6.200000in}{4.620000in}}%
\pgfusepath{clip}%
\pgfsetbuttcap%
\pgfsetroundjoin%
\pgfsetlinewidth{1.003750pt}%
\definecolor{currentstroke}{rgb}{1.000000,0.000000,0.000000}%
\pgfsetstrokecolor{currentstroke}%
\pgfsetdash{}{0pt}%
\pgfpathmoveto{\pgfqpoint{4.356479in}{0.741673in}}%
\pgfpathcurveto{\pgfqpoint{4.367529in}{0.741673in}}{\pgfqpoint{4.378128in}{0.746063in}}{\pgfqpoint{4.385942in}{0.753877in}}%
\pgfpathcurveto{\pgfqpoint{4.393755in}{0.761691in}}{\pgfqpoint{4.398146in}{0.772290in}}{\pgfqpoint{4.398146in}{0.783340in}}%
\pgfpathcurveto{\pgfqpoint{4.398146in}{0.794390in}}{\pgfqpoint{4.393755in}{0.804989in}}{\pgfqpoint{4.385942in}{0.812803in}}%
\pgfpathcurveto{\pgfqpoint{4.378128in}{0.820616in}}{\pgfqpoint{4.367529in}{0.825006in}}{\pgfqpoint{4.356479in}{0.825006in}}%
\pgfpathcurveto{\pgfqpoint{4.345429in}{0.825006in}}{\pgfqpoint{4.334830in}{0.820616in}}{\pgfqpoint{4.327016in}{0.812803in}}%
\pgfpathcurveto{\pgfqpoint{4.319203in}{0.804989in}}{\pgfqpoint{4.314812in}{0.794390in}}{\pgfqpoint{4.314812in}{0.783340in}}%
\pgfpathcurveto{\pgfqpoint{4.314812in}{0.772290in}}{\pgfqpoint{4.319203in}{0.761691in}}{\pgfqpoint{4.327016in}{0.753877in}}%
\pgfpathcurveto{\pgfqpoint{4.334830in}{0.746063in}}{\pgfqpoint{4.345429in}{0.741673in}}{\pgfqpoint{4.356479in}{0.741673in}}%
\pgfpathlineto{\pgfqpoint{4.356479in}{0.741673in}}%
\pgfpathclose%
\pgfusepath{stroke}%
\end{pgfscope}%
\begin{pgfscope}%
\pgfpathrectangle{\pgfqpoint{0.847223in}{0.554012in}}{\pgfqpoint{6.200000in}{4.620000in}}%
\pgfusepath{clip}%
\pgfsetbuttcap%
\pgfsetroundjoin%
\pgfsetlinewidth{1.003750pt}%
\definecolor{currentstroke}{rgb}{1.000000,0.000000,0.000000}%
\pgfsetstrokecolor{currentstroke}%
\pgfsetdash{}{0pt}%
\pgfpathmoveto{\pgfqpoint{4.361812in}{0.740808in}}%
\pgfpathcurveto{\pgfqpoint{4.372862in}{0.740808in}}{\pgfqpoint{4.383461in}{0.745199in}}{\pgfqpoint{4.391275in}{0.753012in}}%
\pgfpathcurveto{\pgfqpoint{4.399089in}{0.760826in}}{\pgfqpoint{4.403479in}{0.771425in}}{\pgfqpoint{4.403479in}{0.782475in}}%
\pgfpathcurveto{\pgfqpoint{4.403479in}{0.793525in}}{\pgfqpoint{4.399089in}{0.804124in}}{\pgfqpoint{4.391275in}{0.811938in}}%
\pgfpathcurveto{\pgfqpoint{4.383461in}{0.819751in}}{\pgfqpoint{4.372862in}{0.824142in}}{\pgfqpoint{4.361812in}{0.824142in}}%
\pgfpathcurveto{\pgfqpoint{4.350762in}{0.824142in}}{\pgfqpoint{4.340163in}{0.819751in}}{\pgfqpoint{4.332349in}{0.811938in}}%
\pgfpathcurveto{\pgfqpoint{4.324536in}{0.804124in}}{\pgfqpoint{4.320146in}{0.793525in}}{\pgfqpoint{4.320146in}{0.782475in}}%
\pgfpathcurveto{\pgfqpoint{4.320146in}{0.771425in}}{\pgfqpoint{4.324536in}{0.760826in}}{\pgfqpoint{4.332349in}{0.753012in}}%
\pgfpathcurveto{\pgfqpoint{4.340163in}{0.745199in}}{\pgfqpoint{4.350762in}{0.740808in}}{\pgfqpoint{4.361812in}{0.740808in}}%
\pgfpathlineto{\pgfqpoint{4.361812in}{0.740808in}}%
\pgfpathclose%
\pgfusepath{stroke}%
\end{pgfscope}%
\begin{pgfscope}%
\pgfpathrectangle{\pgfqpoint{0.847223in}{0.554012in}}{\pgfqpoint{6.200000in}{4.620000in}}%
\pgfusepath{clip}%
\pgfsetbuttcap%
\pgfsetroundjoin%
\pgfsetlinewidth{1.003750pt}%
\definecolor{currentstroke}{rgb}{1.000000,0.000000,0.000000}%
\pgfsetstrokecolor{currentstroke}%
\pgfsetdash{}{0pt}%
\pgfpathmoveto{\pgfqpoint{4.367145in}{0.739946in}}%
\pgfpathcurveto{\pgfqpoint{4.378196in}{0.739946in}}{\pgfqpoint{4.388795in}{0.744336in}}{\pgfqpoint{4.396608in}{0.752150in}}%
\pgfpathcurveto{\pgfqpoint{4.404422in}{0.759963in}}{\pgfqpoint{4.408812in}{0.770562in}}{\pgfqpoint{4.408812in}{0.781612in}}%
\pgfpathcurveto{\pgfqpoint{4.408812in}{0.792662in}}{\pgfqpoint{4.404422in}{0.803262in}}{\pgfqpoint{4.396608in}{0.811075in}}%
\pgfpathcurveto{\pgfqpoint{4.388795in}{0.818889in}}{\pgfqpoint{4.378196in}{0.823279in}}{\pgfqpoint{4.367145in}{0.823279in}}%
\pgfpathcurveto{\pgfqpoint{4.356095in}{0.823279in}}{\pgfqpoint{4.345496in}{0.818889in}}{\pgfqpoint{4.337683in}{0.811075in}}%
\pgfpathcurveto{\pgfqpoint{4.329869in}{0.803262in}}{\pgfqpoint{4.325479in}{0.792662in}}{\pgfqpoint{4.325479in}{0.781612in}}%
\pgfpathcurveto{\pgfqpoint{4.325479in}{0.770562in}}{\pgfqpoint{4.329869in}{0.759963in}}{\pgfqpoint{4.337683in}{0.752150in}}%
\pgfpathcurveto{\pgfqpoint{4.345496in}{0.744336in}}{\pgfqpoint{4.356095in}{0.739946in}}{\pgfqpoint{4.367145in}{0.739946in}}%
\pgfpathlineto{\pgfqpoint{4.367145in}{0.739946in}}%
\pgfpathclose%
\pgfusepath{stroke}%
\end{pgfscope}%
\begin{pgfscope}%
\pgfpathrectangle{\pgfqpoint{0.847223in}{0.554012in}}{\pgfqpoint{6.200000in}{4.620000in}}%
\pgfusepath{clip}%
\pgfsetbuttcap%
\pgfsetroundjoin%
\pgfsetlinewidth{1.003750pt}%
\definecolor{currentstroke}{rgb}{1.000000,0.000000,0.000000}%
\pgfsetstrokecolor{currentstroke}%
\pgfsetdash{}{0pt}%
\pgfpathmoveto{\pgfqpoint{4.372479in}{0.739085in}}%
\pgfpathcurveto{\pgfqpoint{4.383529in}{0.739085in}}{\pgfqpoint{4.394128in}{0.743476in}}{\pgfqpoint{4.401941in}{0.751289in}}%
\pgfpathcurveto{\pgfqpoint{4.409755in}{0.759103in}}{\pgfqpoint{4.414145in}{0.769702in}}{\pgfqpoint{4.414145in}{0.780752in}}%
\pgfpathcurveto{\pgfqpoint{4.414145in}{0.791802in}}{\pgfqpoint{4.409755in}{0.802401in}}{\pgfqpoint{4.401941in}{0.810215in}}%
\pgfpathcurveto{\pgfqpoint{4.394128in}{0.818028in}}{\pgfqpoint{4.383529in}{0.822419in}}{\pgfqpoint{4.372479in}{0.822419in}}%
\pgfpathcurveto{\pgfqpoint{4.361428in}{0.822419in}}{\pgfqpoint{4.350829in}{0.818028in}}{\pgfqpoint{4.343016in}{0.810215in}}%
\pgfpathcurveto{\pgfqpoint{4.335202in}{0.802401in}}{\pgfqpoint{4.330812in}{0.791802in}}{\pgfqpoint{4.330812in}{0.780752in}}%
\pgfpathcurveto{\pgfqpoint{4.330812in}{0.769702in}}{\pgfqpoint{4.335202in}{0.759103in}}{\pgfqpoint{4.343016in}{0.751289in}}%
\pgfpathcurveto{\pgfqpoint{4.350829in}{0.743476in}}{\pgfqpoint{4.361428in}{0.739085in}}{\pgfqpoint{4.372479in}{0.739085in}}%
\pgfpathlineto{\pgfqpoint{4.372479in}{0.739085in}}%
\pgfpathclose%
\pgfusepath{stroke}%
\end{pgfscope}%
\begin{pgfscope}%
\pgfpathrectangle{\pgfqpoint{0.847223in}{0.554012in}}{\pgfqpoint{6.200000in}{4.620000in}}%
\pgfusepath{clip}%
\pgfsetbuttcap%
\pgfsetroundjoin%
\pgfsetlinewidth{1.003750pt}%
\definecolor{currentstroke}{rgb}{1.000000,0.000000,0.000000}%
\pgfsetstrokecolor{currentstroke}%
\pgfsetdash{}{0pt}%
\pgfpathmoveto{\pgfqpoint{4.377812in}{0.738227in}}%
\pgfpathcurveto{\pgfqpoint{4.388862in}{0.738227in}}{\pgfqpoint{4.399461in}{0.742618in}}{\pgfqpoint{4.407275in}{0.750431in}}%
\pgfpathcurveto{\pgfqpoint{4.415088in}{0.758245in}}{\pgfqpoint{4.419478in}{0.768844in}}{\pgfqpoint{4.419478in}{0.779894in}}%
\pgfpathcurveto{\pgfqpoint{4.419478in}{0.790944in}}{\pgfqpoint{4.415088in}{0.801543in}}{\pgfqpoint{4.407275in}{0.809357in}}%
\pgfpathcurveto{\pgfqpoint{4.399461in}{0.817170in}}{\pgfqpoint{4.388862in}{0.821561in}}{\pgfqpoint{4.377812in}{0.821561in}}%
\pgfpathcurveto{\pgfqpoint{4.366762in}{0.821561in}}{\pgfqpoint{4.356163in}{0.817170in}}{\pgfqpoint{4.348349in}{0.809357in}}%
\pgfpathcurveto{\pgfqpoint{4.340535in}{0.801543in}}{\pgfqpoint{4.336145in}{0.790944in}}{\pgfqpoint{4.336145in}{0.779894in}}%
\pgfpathcurveto{\pgfqpoint{4.336145in}{0.768844in}}{\pgfqpoint{4.340535in}{0.758245in}}{\pgfqpoint{4.348349in}{0.750431in}}%
\pgfpathcurveto{\pgfqpoint{4.356163in}{0.742618in}}{\pgfqpoint{4.366762in}{0.738227in}}{\pgfqpoint{4.377812in}{0.738227in}}%
\pgfpathlineto{\pgfqpoint{4.377812in}{0.738227in}}%
\pgfpathclose%
\pgfusepath{stroke}%
\end{pgfscope}%
\begin{pgfscope}%
\pgfpathrectangle{\pgfqpoint{0.847223in}{0.554012in}}{\pgfqpoint{6.200000in}{4.620000in}}%
\pgfusepath{clip}%
\pgfsetbuttcap%
\pgfsetroundjoin%
\pgfsetlinewidth{1.003750pt}%
\definecolor{currentstroke}{rgb}{1.000000,0.000000,0.000000}%
\pgfsetstrokecolor{currentstroke}%
\pgfsetdash{}{0pt}%
\pgfpathmoveto{\pgfqpoint{4.383145in}{0.737372in}}%
\pgfpathcurveto{\pgfqpoint{4.394195in}{0.737372in}}{\pgfqpoint{4.404794in}{0.741762in}}{\pgfqpoint{4.412608in}{0.749575in}}%
\pgfpathcurveto{\pgfqpoint{4.420421in}{0.757389in}}{\pgfqpoint{4.424812in}{0.767988in}}{\pgfqpoint{4.424812in}{0.779038in}}%
\pgfpathcurveto{\pgfqpoint{4.424812in}{0.790088in}}{\pgfqpoint{4.420421in}{0.800687in}}{\pgfqpoint{4.412608in}{0.808501in}}%
\pgfpathcurveto{\pgfqpoint{4.404794in}{0.816315in}}{\pgfqpoint{4.394195in}{0.820705in}}{\pgfqpoint{4.383145in}{0.820705in}}%
\pgfpathcurveto{\pgfqpoint{4.372095in}{0.820705in}}{\pgfqpoint{4.361496in}{0.816315in}}{\pgfqpoint{4.353682in}{0.808501in}}%
\pgfpathcurveto{\pgfqpoint{4.345869in}{0.800687in}}{\pgfqpoint{4.341478in}{0.790088in}}{\pgfqpoint{4.341478in}{0.779038in}}%
\pgfpathcurveto{\pgfqpoint{4.341478in}{0.767988in}}{\pgfqpoint{4.345869in}{0.757389in}}{\pgfqpoint{4.353682in}{0.749575in}}%
\pgfpathcurveto{\pgfqpoint{4.361496in}{0.741762in}}{\pgfqpoint{4.372095in}{0.737372in}}{\pgfqpoint{4.383145in}{0.737372in}}%
\pgfpathlineto{\pgfqpoint{4.383145in}{0.737372in}}%
\pgfpathclose%
\pgfusepath{stroke}%
\end{pgfscope}%
\begin{pgfscope}%
\pgfpathrectangle{\pgfqpoint{0.847223in}{0.554012in}}{\pgfqpoint{6.200000in}{4.620000in}}%
\pgfusepath{clip}%
\pgfsetbuttcap%
\pgfsetroundjoin%
\pgfsetlinewidth{1.003750pt}%
\definecolor{currentstroke}{rgb}{1.000000,0.000000,0.000000}%
\pgfsetstrokecolor{currentstroke}%
\pgfsetdash{}{0pt}%
\pgfpathmoveto{\pgfqpoint{4.388478in}{0.736518in}}%
\pgfpathcurveto{\pgfqpoint{4.399528in}{0.736518in}}{\pgfqpoint{4.410127in}{0.740908in}}{\pgfqpoint{4.417941in}{0.748722in}}%
\pgfpathcurveto{\pgfqpoint{4.425755in}{0.756536in}}{\pgfqpoint{4.430145in}{0.767135in}}{\pgfqpoint{4.430145in}{0.778185in}}%
\pgfpathcurveto{\pgfqpoint{4.430145in}{0.789235in}}{\pgfqpoint{4.425755in}{0.799834in}}{\pgfqpoint{4.417941in}{0.807648in}}%
\pgfpathcurveto{\pgfqpoint{4.410127in}{0.815461in}}{\pgfqpoint{4.399528in}{0.819851in}}{\pgfqpoint{4.388478in}{0.819851in}}%
\pgfpathcurveto{\pgfqpoint{4.377428in}{0.819851in}}{\pgfqpoint{4.366829in}{0.815461in}}{\pgfqpoint{4.359015in}{0.807648in}}%
\pgfpathcurveto{\pgfqpoint{4.351202in}{0.799834in}}{\pgfqpoint{4.346812in}{0.789235in}}{\pgfqpoint{4.346812in}{0.778185in}}%
\pgfpathcurveto{\pgfqpoint{4.346812in}{0.767135in}}{\pgfqpoint{4.351202in}{0.756536in}}{\pgfqpoint{4.359015in}{0.748722in}}%
\pgfpathcurveto{\pgfqpoint{4.366829in}{0.740908in}}{\pgfqpoint{4.377428in}{0.736518in}}{\pgfqpoint{4.388478in}{0.736518in}}%
\pgfpathlineto{\pgfqpoint{4.388478in}{0.736518in}}%
\pgfpathclose%
\pgfusepath{stroke}%
\end{pgfscope}%
\begin{pgfscope}%
\pgfpathrectangle{\pgfqpoint{0.847223in}{0.554012in}}{\pgfqpoint{6.200000in}{4.620000in}}%
\pgfusepath{clip}%
\pgfsetbuttcap%
\pgfsetroundjoin%
\pgfsetlinewidth{1.003750pt}%
\definecolor{currentstroke}{rgb}{1.000000,0.000000,0.000000}%
\pgfsetstrokecolor{currentstroke}%
\pgfsetdash{}{0pt}%
\pgfpathmoveto{\pgfqpoint{4.393811in}{0.735667in}}%
\pgfpathcurveto{\pgfqpoint{4.404862in}{0.735667in}}{\pgfqpoint{4.415461in}{0.740057in}}{\pgfqpoint{4.423274in}{0.747871in}}%
\pgfpathcurveto{\pgfqpoint{4.431088in}{0.755684in}}{\pgfqpoint{4.435478in}{0.766283in}}{\pgfqpoint{4.435478in}{0.777333in}}%
\pgfpathcurveto{\pgfqpoint{4.435478in}{0.788384in}}{\pgfqpoint{4.431088in}{0.798983in}}{\pgfqpoint{4.423274in}{0.806796in}}%
\pgfpathcurveto{\pgfqpoint{4.415461in}{0.814610in}}{\pgfqpoint{4.404862in}{0.819000in}}{\pgfqpoint{4.393811in}{0.819000in}}%
\pgfpathcurveto{\pgfqpoint{4.382761in}{0.819000in}}{\pgfqpoint{4.372162in}{0.814610in}}{\pgfqpoint{4.364349in}{0.806796in}}%
\pgfpathcurveto{\pgfqpoint{4.356535in}{0.798983in}}{\pgfqpoint{4.352145in}{0.788384in}}{\pgfqpoint{4.352145in}{0.777333in}}%
\pgfpathcurveto{\pgfqpoint{4.352145in}{0.766283in}}{\pgfqpoint{4.356535in}{0.755684in}}{\pgfqpoint{4.364349in}{0.747871in}}%
\pgfpathcurveto{\pgfqpoint{4.372162in}{0.740057in}}{\pgfqpoint{4.382761in}{0.735667in}}{\pgfqpoint{4.393811in}{0.735667in}}%
\pgfpathlineto{\pgfqpoint{4.393811in}{0.735667in}}%
\pgfpathclose%
\pgfusepath{stroke}%
\end{pgfscope}%
\begin{pgfscope}%
\pgfpathrectangle{\pgfqpoint{0.847223in}{0.554012in}}{\pgfqpoint{6.200000in}{4.620000in}}%
\pgfusepath{clip}%
\pgfsetbuttcap%
\pgfsetroundjoin%
\pgfsetlinewidth{1.003750pt}%
\definecolor{currentstroke}{rgb}{1.000000,0.000000,0.000000}%
\pgfsetstrokecolor{currentstroke}%
\pgfsetdash{}{0pt}%
\pgfpathmoveto{\pgfqpoint{4.399145in}{0.734818in}}%
\pgfpathcurveto{\pgfqpoint{4.410195in}{0.734818in}}{\pgfqpoint{4.420794in}{0.739208in}}{\pgfqpoint{4.428607in}{0.747022in}}%
\pgfpathcurveto{\pgfqpoint{4.436421in}{0.754835in}}{\pgfqpoint{4.440811in}{0.765434in}}{\pgfqpoint{4.440811in}{0.776484in}}%
\pgfpathcurveto{\pgfqpoint{4.440811in}{0.787535in}}{\pgfqpoint{4.436421in}{0.798134in}}{\pgfqpoint{4.428607in}{0.805947in}}%
\pgfpathcurveto{\pgfqpoint{4.420794in}{0.813761in}}{\pgfqpoint{4.410195in}{0.818151in}}{\pgfqpoint{4.399145in}{0.818151in}}%
\pgfpathcurveto{\pgfqpoint{4.388095in}{0.818151in}}{\pgfqpoint{4.377496in}{0.813761in}}{\pgfqpoint{4.369682in}{0.805947in}}%
\pgfpathcurveto{\pgfqpoint{4.361868in}{0.798134in}}{\pgfqpoint{4.357478in}{0.787535in}}{\pgfqpoint{4.357478in}{0.776484in}}%
\pgfpathcurveto{\pgfqpoint{4.357478in}{0.765434in}}{\pgfqpoint{4.361868in}{0.754835in}}{\pgfqpoint{4.369682in}{0.747022in}}%
\pgfpathcurveto{\pgfqpoint{4.377496in}{0.739208in}}{\pgfqpoint{4.388095in}{0.734818in}}{\pgfqpoint{4.399145in}{0.734818in}}%
\pgfpathlineto{\pgfqpoint{4.399145in}{0.734818in}}%
\pgfpathclose%
\pgfusepath{stroke}%
\end{pgfscope}%
\begin{pgfscope}%
\pgfpathrectangle{\pgfqpoint{0.847223in}{0.554012in}}{\pgfqpoint{6.200000in}{4.620000in}}%
\pgfusepath{clip}%
\pgfsetbuttcap%
\pgfsetroundjoin%
\pgfsetlinewidth{1.003750pt}%
\definecolor{currentstroke}{rgb}{1.000000,0.000000,0.000000}%
\pgfsetstrokecolor{currentstroke}%
\pgfsetdash{}{0pt}%
\pgfpathmoveto{\pgfqpoint{4.404478in}{0.733971in}}%
\pgfpathcurveto{\pgfqpoint{4.415528in}{0.733971in}}{\pgfqpoint{4.426127in}{0.738361in}}{\pgfqpoint{4.433941in}{0.746175in}}%
\pgfpathcurveto{\pgfqpoint{4.441754in}{0.753988in}}{\pgfqpoint{4.446145in}{0.764587in}}{\pgfqpoint{4.446145in}{0.775638in}}%
\pgfpathcurveto{\pgfqpoint{4.446145in}{0.786688in}}{\pgfqpoint{4.441754in}{0.797287in}}{\pgfqpoint{4.433941in}{0.805100in}}%
\pgfpathcurveto{\pgfqpoint{4.426127in}{0.812914in}}{\pgfqpoint{4.415528in}{0.817304in}}{\pgfqpoint{4.404478in}{0.817304in}}%
\pgfpathcurveto{\pgfqpoint{4.393428in}{0.817304in}}{\pgfqpoint{4.382829in}{0.812914in}}{\pgfqpoint{4.375015in}{0.805100in}}%
\pgfpathcurveto{\pgfqpoint{4.367201in}{0.797287in}}{\pgfqpoint{4.362811in}{0.786688in}}{\pgfqpoint{4.362811in}{0.775638in}}%
\pgfpathcurveto{\pgfqpoint{4.362811in}{0.764587in}}{\pgfqpoint{4.367201in}{0.753988in}}{\pgfqpoint{4.375015in}{0.746175in}}%
\pgfpathcurveto{\pgfqpoint{4.382829in}{0.738361in}}{\pgfqpoint{4.393428in}{0.733971in}}{\pgfqpoint{4.404478in}{0.733971in}}%
\pgfpathlineto{\pgfqpoint{4.404478in}{0.733971in}}%
\pgfpathclose%
\pgfusepath{stroke}%
\end{pgfscope}%
\begin{pgfscope}%
\pgfpathrectangle{\pgfqpoint{0.847223in}{0.554012in}}{\pgfqpoint{6.200000in}{4.620000in}}%
\pgfusepath{clip}%
\pgfsetbuttcap%
\pgfsetroundjoin%
\pgfsetlinewidth{1.003750pt}%
\definecolor{currentstroke}{rgb}{1.000000,0.000000,0.000000}%
\pgfsetstrokecolor{currentstroke}%
\pgfsetdash{}{0pt}%
\pgfpathmoveto{\pgfqpoint{4.409811in}{0.733126in}}%
\pgfpathcurveto{\pgfqpoint{4.420861in}{0.733126in}}{\pgfqpoint{4.431460in}{0.737517in}}{\pgfqpoint{4.439274in}{0.745330in}}%
\pgfpathcurveto{\pgfqpoint{4.447088in}{0.753144in}}{\pgfqpoint{4.451478in}{0.763743in}}{\pgfqpoint{4.451478in}{0.774793in}}%
\pgfpathcurveto{\pgfqpoint{4.451478in}{0.785843in}}{\pgfqpoint{4.447088in}{0.796442in}}{\pgfqpoint{4.439274in}{0.804256in}}%
\pgfpathcurveto{\pgfqpoint{4.431460in}{0.812069in}}{\pgfqpoint{4.420861in}{0.816460in}}{\pgfqpoint{4.409811in}{0.816460in}}%
\pgfpathcurveto{\pgfqpoint{4.398761in}{0.816460in}}{\pgfqpoint{4.388162in}{0.812069in}}{\pgfqpoint{4.380348in}{0.804256in}}%
\pgfpathcurveto{\pgfqpoint{4.372535in}{0.796442in}}{\pgfqpoint{4.368144in}{0.785843in}}{\pgfqpoint{4.368144in}{0.774793in}}%
\pgfpathcurveto{\pgfqpoint{4.368144in}{0.763743in}}{\pgfqpoint{4.372535in}{0.753144in}}{\pgfqpoint{4.380348in}{0.745330in}}%
\pgfpathcurveto{\pgfqpoint{4.388162in}{0.737517in}}{\pgfqpoint{4.398761in}{0.733126in}}{\pgfqpoint{4.409811in}{0.733126in}}%
\pgfpathlineto{\pgfqpoint{4.409811in}{0.733126in}}%
\pgfpathclose%
\pgfusepath{stroke}%
\end{pgfscope}%
\begin{pgfscope}%
\pgfpathrectangle{\pgfqpoint{0.847223in}{0.554012in}}{\pgfqpoint{6.200000in}{4.620000in}}%
\pgfusepath{clip}%
\pgfsetbuttcap%
\pgfsetroundjoin%
\pgfsetlinewidth{1.003750pt}%
\definecolor{currentstroke}{rgb}{1.000000,0.000000,0.000000}%
\pgfsetstrokecolor{currentstroke}%
\pgfsetdash{}{0pt}%
\pgfpathmoveto{\pgfqpoint{4.415144in}{0.732284in}}%
\pgfpathcurveto{\pgfqpoint{4.426194in}{0.732284in}}{\pgfqpoint{4.436793in}{0.736674in}}{\pgfqpoint{4.444607in}{0.744488in}}%
\pgfpathcurveto{\pgfqpoint{4.452421in}{0.752301in}}{\pgfqpoint{4.456811in}{0.762900in}}{\pgfqpoint{4.456811in}{0.773950in}}%
\pgfpathcurveto{\pgfqpoint{4.456811in}{0.785001in}}{\pgfqpoint{4.452421in}{0.795600in}}{\pgfqpoint{4.444607in}{0.803413in}}%
\pgfpathcurveto{\pgfqpoint{4.436793in}{0.811227in}}{\pgfqpoint{4.426194in}{0.815617in}}{\pgfqpoint{4.415144in}{0.815617in}}%
\pgfpathcurveto{\pgfqpoint{4.404094in}{0.815617in}}{\pgfqpoint{4.393495in}{0.811227in}}{\pgfqpoint{4.385682in}{0.803413in}}%
\pgfpathcurveto{\pgfqpoint{4.377868in}{0.795600in}}{\pgfqpoint{4.373478in}{0.785001in}}{\pgfqpoint{4.373478in}{0.773950in}}%
\pgfpathcurveto{\pgfqpoint{4.373478in}{0.762900in}}{\pgfqpoint{4.377868in}{0.752301in}}{\pgfqpoint{4.385682in}{0.744488in}}%
\pgfpathcurveto{\pgfqpoint{4.393495in}{0.736674in}}{\pgfqpoint{4.404094in}{0.732284in}}{\pgfqpoint{4.415144in}{0.732284in}}%
\pgfpathlineto{\pgfqpoint{4.415144in}{0.732284in}}%
\pgfpathclose%
\pgfusepath{stroke}%
\end{pgfscope}%
\begin{pgfscope}%
\pgfpathrectangle{\pgfqpoint{0.847223in}{0.554012in}}{\pgfqpoint{6.200000in}{4.620000in}}%
\pgfusepath{clip}%
\pgfsetbuttcap%
\pgfsetroundjoin%
\pgfsetlinewidth{1.003750pt}%
\definecolor{currentstroke}{rgb}{1.000000,0.000000,0.000000}%
\pgfsetstrokecolor{currentstroke}%
\pgfsetdash{}{0pt}%
\pgfpathmoveto{\pgfqpoint{4.420478in}{0.731444in}}%
\pgfpathcurveto{\pgfqpoint{4.431528in}{0.731444in}}{\pgfqpoint{4.442127in}{0.735834in}}{\pgfqpoint{4.449940in}{0.743647in}}%
\pgfpathcurveto{\pgfqpoint{4.457754in}{0.751461in}}{\pgfqpoint{4.462144in}{0.762060in}}{\pgfqpoint{4.462144in}{0.773110in}}%
\pgfpathcurveto{\pgfqpoint{4.462144in}{0.784160in}}{\pgfqpoint{4.457754in}{0.794759in}}{\pgfqpoint{4.449940in}{0.802573in}}%
\pgfpathcurveto{\pgfqpoint{4.442127in}{0.810387in}}{\pgfqpoint{4.431528in}{0.814777in}}{\pgfqpoint{4.420478in}{0.814777in}}%
\pgfpathcurveto{\pgfqpoint{4.409427in}{0.814777in}}{\pgfqpoint{4.398828in}{0.810387in}}{\pgfqpoint{4.391015in}{0.802573in}}%
\pgfpathcurveto{\pgfqpoint{4.383201in}{0.794759in}}{\pgfqpoint{4.378811in}{0.784160in}}{\pgfqpoint{4.378811in}{0.773110in}}%
\pgfpathcurveto{\pgfqpoint{4.378811in}{0.762060in}}{\pgfqpoint{4.383201in}{0.751461in}}{\pgfqpoint{4.391015in}{0.743647in}}%
\pgfpathcurveto{\pgfqpoint{4.398828in}{0.735834in}}{\pgfqpoint{4.409427in}{0.731444in}}{\pgfqpoint{4.420478in}{0.731444in}}%
\pgfpathlineto{\pgfqpoint{4.420478in}{0.731444in}}%
\pgfpathclose%
\pgfusepath{stroke}%
\end{pgfscope}%
\begin{pgfscope}%
\pgfpathrectangle{\pgfqpoint{0.847223in}{0.554012in}}{\pgfqpoint{6.200000in}{4.620000in}}%
\pgfusepath{clip}%
\pgfsetbuttcap%
\pgfsetroundjoin%
\pgfsetlinewidth{1.003750pt}%
\definecolor{currentstroke}{rgb}{1.000000,0.000000,0.000000}%
\pgfsetstrokecolor{currentstroke}%
\pgfsetdash{}{0pt}%
\pgfpathmoveto{\pgfqpoint{4.425811in}{0.730606in}}%
\pgfpathcurveto{\pgfqpoint{4.436861in}{0.730606in}}{\pgfqpoint{4.447460in}{0.734996in}}{\pgfqpoint{4.455274in}{0.742809in}}%
\pgfpathcurveto{\pgfqpoint{4.463087in}{0.750623in}}{\pgfqpoint{4.467477in}{0.761222in}}{\pgfqpoint{4.467477in}{0.772272in}}%
\pgfpathcurveto{\pgfqpoint{4.467477in}{0.783322in}}{\pgfqpoint{4.463087in}{0.793921in}}{\pgfqpoint{4.455274in}{0.801735in}}%
\pgfpathcurveto{\pgfqpoint{4.447460in}{0.809549in}}{\pgfqpoint{4.436861in}{0.813939in}}{\pgfqpoint{4.425811in}{0.813939in}}%
\pgfpathcurveto{\pgfqpoint{4.414761in}{0.813939in}}{\pgfqpoint{4.404162in}{0.809549in}}{\pgfqpoint{4.396348in}{0.801735in}}%
\pgfpathcurveto{\pgfqpoint{4.388534in}{0.793921in}}{\pgfqpoint{4.384144in}{0.783322in}}{\pgfqpoint{4.384144in}{0.772272in}}%
\pgfpathcurveto{\pgfqpoint{4.384144in}{0.761222in}}{\pgfqpoint{4.388534in}{0.750623in}}{\pgfqpoint{4.396348in}{0.742809in}}%
\pgfpathcurveto{\pgfqpoint{4.404162in}{0.734996in}}{\pgfqpoint{4.414761in}{0.730606in}}{\pgfqpoint{4.425811in}{0.730606in}}%
\pgfpathlineto{\pgfqpoint{4.425811in}{0.730606in}}%
\pgfpathclose%
\pgfusepath{stroke}%
\end{pgfscope}%
\begin{pgfscope}%
\pgfpathrectangle{\pgfqpoint{0.847223in}{0.554012in}}{\pgfqpoint{6.200000in}{4.620000in}}%
\pgfusepath{clip}%
\pgfsetbuttcap%
\pgfsetroundjoin%
\pgfsetlinewidth{1.003750pt}%
\definecolor{currentstroke}{rgb}{1.000000,0.000000,0.000000}%
\pgfsetstrokecolor{currentstroke}%
\pgfsetdash{}{0pt}%
\pgfpathmoveto{\pgfqpoint{4.431144in}{0.729770in}}%
\pgfpathcurveto{\pgfqpoint{4.442194in}{0.729770in}}{\pgfqpoint{4.452793in}{0.734160in}}{\pgfqpoint{4.460607in}{0.741974in}}%
\pgfpathcurveto{\pgfqpoint{4.468420in}{0.749787in}}{\pgfqpoint{4.472811in}{0.760386in}}{\pgfqpoint{4.472811in}{0.771436in}}%
\pgfpathcurveto{\pgfqpoint{4.472811in}{0.782486in}}{\pgfqpoint{4.468420in}{0.793085in}}{\pgfqpoint{4.460607in}{0.800899in}}%
\pgfpathcurveto{\pgfqpoint{4.452793in}{0.808713in}}{\pgfqpoint{4.442194in}{0.813103in}}{\pgfqpoint{4.431144in}{0.813103in}}%
\pgfpathcurveto{\pgfqpoint{4.420094in}{0.813103in}}{\pgfqpoint{4.409495in}{0.808713in}}{\pgfqpoint{4.401681in}{0.800899in}}%
\pgfpathcurveto{\pgfqpoint{4.393868in}{0.793085in}}{\pgfqpoint{4.389477in}{0.782486in}}{\pgfqpoint{4.389477in}{0.771436in}}%
\pgfpathcurveto{\pgfqpoint{4.389477in}{0.760386in}}{\pgfqpoint{4.393868in}{0.749787in}}{\pgfqpoint{4.401681in}{0.741974in}}%
\pgfpathcurveto{\pgfqpoint{4.409495in}{0.734160in}}{\pgfqpoint{4.420094in}{0.729770in}}{\pgfqpoint{4.431144in}{0.729770in}}%
\pgfpathlineto{\pgfqpoint{4.431144in}{0.729770in}}%
\pgfpathclose%
\pgfusepath{stroke}%
\end{pgfscope}%
\begin{pgfscope}%
\pgfpathrectangle{\pgfqpoint{0.847223in}{0.554012in}}{\pgfqpoint{6.200000in}{4.620000in}}%
\pgfusepath{clip}%
\pgfsetbuttcap%
\pgfsetroundjoin%
\pgfsetlinewidth{1.003750pt}%
\definecolor{currentstroke}{rgb}{1.000000,0.000000,0.000000}%
\pgfsetstrokecolor{currentstroke}%
\pgfsetdash{}{0pt}%
\pgfpathmoveto{\pgfqpoint{4.436477in}{0.728936in}}%
\pgfpathcurveto{\pgfqpoint{4.447527in}{0.728936in}}{\pgfqpoint{4.458126in}{0.733326in}}{\pgfqpoint{4.465940in}{0.741140in}}%
\pgfpathcurveto{\pgfqpoint{4.473754in}{0.748953in}}{\pgfqpoint{4.478144in}{0.759552in}}{\pgfqpoint{4.478144in}{0.770603in}}%
\pgfpathcurveto{\pgfqpoint{4.478144in}{0.781653in}}{\pgfqpoint{4.473754in}{0.792252in}}{\pgfqpoint{4.465940in}{0.800065in}}%
\pgfpathcurveto{\pgfqpoint{4.458126in}{0.807879in}}{\pgfqpoint{4.447527in}{0.812269in}}{\pgfqpoint{4.436477in}{0.812269in}}%
\pgfpathcurveto{\pgfqpoint{4.425427in}{0.812269in}}{\pgfqpoint{4.414828in}{0.807879in}}{\pgfqpoint{4.407014in}{0.800065in}}%
\pgfpathcurveto{\pgfqpoint{4.399201in}{0.792252in}}{\pgfqpoint{4.394811in}{0.781653in}}{\pgfqpoint{4.394811in}{0.770603in}}%
\pgfpathcurveto{\pgfqpoint{4.394811in}{0.759552in}}{\pgfqpoint{4.399201in}{0.748953in}}{\pgfqpoint{4.407014in}{0.741140in}}%
\pgfpathcurveto{\pgfqpoint{4.414828in}{0.733326in}}{\pgfqpoint{4.425427in}{0.728936in}}{\pgfqpoint{4.436477in}{0.728936in}}%
\pgfpathlineto{\pgfqpoint{4.436477in}{0.728936in}}%
\pgfpathclose%
\pgfusepath{stroke}%
\end{pgfscope}%
\begin{pgfscope}%
\pgfpathrectangle{\pgfqpoint{0.847223in}{0.554012in}}{\pgfqpoint{6.200000in}{4.620000in}}%
\pgfusepath{clip}%
\pgfsetbuttcap%
\pgfsetroundjoin%
\pgfsetlinewidth{1.003750pt}%
\definecolor{currentstroke}{rgb}{1.000000,0.000000,0.000000}%
\pgfsetstrokecolor{currentstroke}%
\pgfsetdash{}{0pt}%
\pgfpathmoveto{\pgfqpoint{4.441810in}{0.728104in}}%
\pgfpathcurveto{\pgfqpoint{4.452861in}{0.728104in}}{\pgfqpoint{4.463460in}{0.732495in}}{\pgfqpoint{4.471273in}{0.740308in}}%
\pgfpathcurveto{\pgfqpoint{4.479087in}{0.748122in}}{\pgfqpoint{4.483477in}{0.758721in}}{\pgfqpoint{4.483477in}{0.769771in}}%
\pgfpathcurveto{\pgfqpoint{4.483477in}{0.780821in}}{\pgfqpoint{4.479087in}{0.791420in}}{\pgfqpoint{4.471273in}{0.799234in}}%
\pgfpathcurveto{\pgfqpoint{4.463460in}{0.807047in}}{\pgfqpoint{4.452861in}{0.811438in}}{\pgfqpoint{4.441810in}{0.811438in}}%
\pgfpathcurveto{\pgfqpoint{4.430760in}{0.811438in}}{\pgfqpoint{4.420161in}{0.807047in}}{\pgfqpoint{4.412348in}{0.799234in}}%
\pgfpathcurveto{\pgfqpoint{4.404534in}{0.791420in}}{\pgfqpoint{4.400144in}{0.780821in}}{\pgfqpoint{4.400144in}{0.769771in}}%
\pgfpathcurveto{\pgfqpoint{4.400144in}{0.758721in}}{\pgfqpoint{4.404534in}{0.748122in}}{\pgfqpoint{4.412348in}{0.740308in}}%
\pgfpathcurveto{\pgfqpoint{4.420161in}{0.732495in}}{\pgfqpoint{4.430760in}{0.728104in}}{\pgfqpoint{4.441810in}{0.728104in}}%
\pgfpathlineto{\pgfqpoint{4.441810in}{0.728104in}}%
\pgfpathclose%
\pgfusepath{stroke}%
\end{pgfscope}%
\begin{pgfscope}%
\pgfpathrectangle{\pgfqpoint{0.847223in}{0.554012in}}{\pgfqpoint{6.200000in}{4.620000in}}%
\pgfusepath{clip}%
\pgfsetbuttcap%
\pgfsetroundjoin%
\pgfsetlinewidth{1.003750pt}%
\definecolor{currentstroke}{rgb}{1.000000,0.000000,0.000000}%
\pgfsetstrokecolor{currentstroke}%
\pgfsetdash{}{0pt}%
\pgfpathmoveto{\pgfqpoint{4.447144in}{0.727275in}}%
\pgfpathcurveto{\pgfqpoint{4.458194in}{0.727275in}}{\pgfqpoint{4.468793in}{0.731665in}}{\pgfqpoint{4.476606in}{0.739479in}}%
\pgfpathcurveto{\pgfqpoint{4.484420in}{0.747293in}}{\pgfqpoint{4.488810in}{0.757892in}}{\pgfqpoint{4.488810in}{0.768942in}}%
\pgfpathcurveto{\pgfqpoint{4.488810in}{0.779992in}}{\pgfqpoint{4.484420in}{0.790591in}}{\pgfqpoint{4.476606in}{0.798404in}}%
\pgfpathcurveto{\pgfqpoint{4.468793in}{0.806218in}}{\pgfqpoint{4.458194in}{0.810608in}}{\pgfqpoint{4.447144in}{0.810608in}}%
\pgfpathcurveto{\pgfqpoint{4.436093in}{0.810608in}}{\pgfqpoint{4.425494in}{0.806218in}}{\pgfqpoint{4.417681in}{0.798404in}}%
\pgfpathcurveto{\pgfqpoint{4.409867in}{0.790591in}}{\pgfqpoint{4.405477in}{0.779992in}}{\pgfqpoint{4.405477in}{0.768942in}}%
\pgfpathcurveto{\pgfqpoint{4.405477in}{0.757892in}}{\pgfqpoint{4.409867in}{0.747293in}}{\pgfqpoint{4.417681in}{0.739479in}}%
\pgfpathcurveto{\pgfqpoint{4.425494in}{0.731665in}}{\pgfqpoint{4.436093in}{0.727275in}}{\pgfqpoint{4.447144in}{0.727275in}}%
\pgfpathlineto{\pgfqpoint{4.447144in}{0.727275in}}%
\pgfpathclose%
\pgfusepath{stroke}%
\end{pgfscope}%
\begin{pgfscope}%
\pgfpathrectangle{\pgfqpoint{0.847223in}{0.554012in}}{\pgfqpoint{6.200000in}{4.620000in}}%
\pgfusepath{clip}%
\pgfsetbuttcap%
\pgfsetroundjoin%
\pgfsetlinewidth{1.003750pt}%
\definecolor{currentstroke}{rgb}{1.000000,0.000000,0.000000}%
\pgfsetstrokecolor{currentstroke}%
\pgfsetdash{}{0pt}%
\pgfpathmoveto{\pgfqpoint{4.452477in}{0.726448in}}%
\pgfpathcurveto{\pgfqpoint{4.463527in}{0.726448in}}{\pgfqpoint{4.474126in}{0.730838in}}{\pgfqpoint{4.481940in}{0.738652in}}%
\pgfpathcurveto{\pgfqpoint{4.489753in}{0.746465in}}{\pgfqpoint{4.494144in}{0.757064in}}{\pgfqpoint{4.494144in}{0.768114in}}%
\pgfpathcurveto{\pgfqpoint{4.494144in}{0.779165in}}{\pgfqpoint{4.489753in}{0.789764in}}{\pgfqpoint{4.481940in}{0.797577in}}%
\pgfpathcurveto{\pgfqpoint{4.474126in}{0.805391in}}{\pgfqpoint{4.463527in}{0.809781in}}{\pgfqpoint{4.452477in}{0.809781in}}%
\pgfpathcurveto{\pgfqpoint{4.441427in}{0.809781in}}{\pgfqpoint{4.430828in}{0.805391in}}{\pgfqpoint{4.423014in}{0.797577in}}%
\pgfpathcurveto{\pgfqpoint{4.415200in}{0.789764in}}{\pgfqpoint{4.410810in}{0.779165in}}{\pgfqpoint{4.410810in}{0.768114in}}%
\pgfpathcurveto{\pgfqpoint{4.410810in}{0.757064in}}{\pgfqpoint{4.415200in}{0.746465in}}{\pgfqpoint{4.423014in}{0.738652in}}%
\pgfpathcurveto{\pgfqpoint{4.430828in}{0.730838in}}{\pgfqpoint{4.441427in}{0.726448in}}{\pgfqpoint{4.452477in}{0.726448in}}%
\pgfpathlineto{\pgfqpoint{4.452477in}{0.726448in}}%
\pgfpathclose%
\pgfusepath{stroke}%
\end{pgfscope}%
\begin{pgfscope}%
\pgfpathrectangle{\pgfqpoint{0.847223in}{0.554012in}}{\pgfqpoint{6.200000in}{4.620000in}}%
\pgfusepath{clip}%
\pgfsetbuttcap%
\pgfsetroundjoin%
\pgfsetlinewidth{1.003750pt}%
\definecolor{currentstroke}{rgb}{1.000000,0.000000,0.000000}%
\pgfsetstrokecolor{currentstroke}%
\pgfsetdash{}{0pt}%
\pgfpathmoveto{\pgfqpoint{4.457810in}{0.725623in}}%
\pgfpathcurveto{\pgfqpoint{4.468860in}{0.725623in}}{\pgfqpoint{4.479459in}{0.730013in}}{\pgfqpoint{4.487273in}{0.737827in}}%
\pgfpathcurveto{\pgfqpoint{4.495086in}{0.745640in}}{\pgfqpoint{4.499477in}{0.756239in}}{\pgfqpoint{4.499477in}{0.767289in}}%
\pgfpathcurveto{\pgfqpoint{4.499477in}{0.778339in}}{\pgfqpoint{4.495086in}{0.788938in}}{\pgfqpoint{4.487273in}{0.796752in}}%
\pgfpathcurveto{\pgfqpoint{4.479459in}{0.804566in}}{\pgfqpoint{4.468860in}{0.808956in}}{\pgfqpoint{4.457810in}{0.808956in}}%
\pgfpathcurveto{\pgfqpoint{4.446760in}{0.808956in}}{\pgfqpoint{4.436161in}{0.804566in}}{\pgfqpoint{4.428347in}{0.796752in}}%
\pgfpathcurveto{\pgfqpoint{4.420534in}{0.788938in}}{\pgfqpoint{4.416143in}{0.778339in}}{\pgfqpoint{4.416143in}{0.767289in}}%
\pgfpathcurveto{\pgfqpoint{4.416143in}{0.756239in}}{\pgfqpoint{4.420534in}{0.745640in}}{\pgfqpoint{4.428347in}{0.737827in}}%
\pgfpathcurveto{\pgfqpoint{4.436161in}{0.730013in}}{\pgfqpoint{4.446760in}{0.725623in}}{\pgfqpoint{4.457810in}{0.725623in}}%
\pgfpathlineto{\pgfqpoint{4.457810in}{0.725623in}}%
\pgfpathclose%
\pgfusepath{stroke}%
\end{pgfscope}%
\begin{pgfscope}%
\pgfpathrectangle{\pgfqpoint{0.847223in}{0.554012in}}{\pgfqpoint{6.200000in}{4.620000in}}%
\pgfusepath{clip}%
\pgfsetbuttcap%
\pgfsetroundjoin%
\pgfsetlinewidth{1.003750pt}%
\definecolor{currentstroke}{rgb}{1.000000,0.000000,0.000000}%
\pgfsetstrokecolor{currentstroke}%
\pgfsetdash{}{0pt}%
\pgfpathmoveto{\pgfqpoint{4.463143in}{0.724800in}}%
\pgfpathcurveto{\pgfqpoint{4.474193in}{0.724800in}}{\pgfqpoint{4.484792in}{0.729190in}}{\pgfqpoint{4.492606in}{0.737004in}}%
\pgfpathcurveto{\pgfqpoint{4.500420in}{0.744817in}}{\pgfqpoint{4.504810in}{0.755416in}}{\pgfqpoint{4.504810in}{0.766466in}}%
\pgfpathcurveto{\pgfqpoint{4.504810in}{0.777516in}}{\pgfqpoint{4.500420in}{0.788115in}}{\pgfqpoint{4.492606in}{0.795929in}}%
\pgfpathcurveto{\pgfqpoint{4.484792in}{0.803743in}}{\pgfqpoint{4.474193in}{0.808133in}}{\pgfqpoint{4.463143in}{0.808133in}}%
\pgfpathcurveto{\pgfqpoint{4.452093in}{0.808133in}}{\pgfqpoint{4.441494in}{0.803743in}}{\pgfqpoint{4.433680in}{0.795929in}}%
\pgfpathcurveto{\pgfqpoint{4.425867in}{0.788115in}}{\pgfqpoint{4.421477in}{0.777516in}}{\pgfqpoint{4.421477in}{0.766466in}}%
\pgfpathcurveto{\pgfqpoint{4.421477in}{0.755416in}}{\pgfqpoint{4.425867in}{0.744817in}}{\pgfqpoint{4.433680in}{0.737004in}}%
\pgfpathcurveto{\pgfqpoint{4.441494in}{0.729190in}}{\pgfqpoint{4.452093in}{0.724800in}}{\pgfqpoint{4.463143in}{0.724800in}}%
\pgfpathlineto{\pgfqpoint{4.463143in}{0.724800in}}%
\pgfpathclose%
\pgfusepath{stroke}%
\end{pgfscope}%
\begin{pgfscope}%
\pgfpathrectangle{\pgfqpoint{0.847223in}{0.554012in}}{\pgfqpoint{6.200000in}{4.620000in}}%
\pgfusepath{clip}%
\pgfsetbuttcap%
\pgfsetroundjoin%
\pgfsetlinewidth{1.003750pt}%
\definecolor{currentstroke}{rgb}{1.000000,0.000000,0.000000}%
\pgfsetstrokecolor{currentstroke}%
\pgfsetdash{}{0pt}%
\pgfpathmoveto{\pgfqpoint{4.468476in}{0.723979in}}%
\pgfpathcurveto{\pgfqpoint{4.479527in}{0.723979in}}{\pgfqpoint{4.490126in}{0.728369in}}{\pgfqpoint{4.497939in}{0.736183in}}%
\pgfpathcurveto{\pgfqpoint{4.505753in}{0.743996in}}{\pgfqpoint{4.510143in}{0.754595in}}{\pgfqpoint{4.510143in}{0.765645in}}%
\pgfpathcurveto{\pgfqpoint{4.510143in}{0.776696in}}{\pgfqpoint{4.505753in}{0.787295in}}{\pgfqpoint{4.497939in}{0.795108in}}%
\pgfpathcurveto{\pgfqpoint{4.490126in}{0.802922in}}{\pgfqpoint{4.479527in}{0.807312in}}{\pgfqpoint{4.468476in}{0.807312in}}%
\pgfpathcurveto{\pgfqpoint{4.457426in}{0.807312in}}{\pgfqpoint{4.446827in}{0.802922in}}{\pgfqpoint{4.439014in}{0.795108in}}%
\pgfpathcurveto{\pgfqpoint{4.431200in}{0.787295in}}{\pgfqpoint{4.426810in}{0.776696in}}{\pgfqpoint{4.426810in}{0.765645in}}%
\pgfpathcurveto{\pgfqpoint{4.426810in}{0.754595in}}{\pgfqpoint{4.431200in}{0.743996in}}{\pgfqpoint{4.439014in}{0.736183in}}%
\pgfpathcurveto{\pgfqpoint{4.446827in}{0.728369in}}{\pgfqpoint{4.457426in}{0.723979in}}{\pgfqpoint{4.468476in}{0.723979in}}%
\pgfpathlineto{\pgfqpoint{4.468476in}{0.723979in}}%
\pgfpathclose%
\pgfusepath{stroke}%
\end{pgfscope}%
\begin{pgfscope}%
\pgfpathrectangle{\pgfqpoint{0.847223in}{0.554012in}}{\pgfqpoint{6.200000in}{4.620000in}}%
\pgfusepath{clip}%
\pgfsetbuttcap%
\pgfsetroundjoin%
\pgfsetlinewidth{1.003750pt}%
\definecolor{currentstroke}{rgb}{1.000000,0.000000,0.000000}%
\pgfsetstrokecolor{currentstroke}%
\pgfsetdash{}{0pt}%
\pgfpathmoveto{\pgfqpoint{4.473810in}{0.723160in}}%
\pgfpathcurveto{\pgfqpoint{4.484860in}{0.723160in}}{\pgfqpoint{4.495459in}{0.727550in}}{\pgfqpoint{4.503272in}{0.735364in}}%
\pgfpathcurveto{\pgfqpoint{4.511086in}{0.743178in}}{\pgfqpoint{4.515476in}{0.753777in}}{\pgfqpoint{4.515476in}{0.764827in}}%
\pgfpathcurveto{\pgfqpoint{4.515476in}{0.775877in}}{\pgfqpoint{4.511086in}{0.786476in}}{\pgfqpoint{4.503272in}{0.794289in}}%
\pgfpathcurveto{\pgfqpoint{4.495459in}{0.802103in}}{\pgfqpoint{4.484860in}{0.806493in}}{\pgfqpoint{4.473810in}{0.806493in}}%
\pgfpathcurveto{\pgfqpoint{4.462760in}{0.806493in}}{\pgfqpoint{4.452161in}{0.802103in}}{\pgfqpoint{4.444347in}{0.794289in}}%
\pgfpathcurveto{\pgfqpoint{4.436533in}{0.786476in}}{\pgfqpoint{4.432143in}{0.775877in}}{\pgfqpoint{4.432143in}{0.764827in}}%
\pgfpathcurveto{\pgfqpoint{4.432143in}{0.753777in}}{\pgfqpoint{4.436533in}{0.743178in}}{\pgfqpoint{4.444347in}{0.735364in}}%
\pgfpathcurveto{\pgfqpoint{4.452161in}{0.727550in}}{\pgfqpoint{4.462760in}{0.723160in}}{\pgfqpoint{4.473810in}{0.723160in}}%
\pgfpathlineto{\pgfqpoint{4.473810in}{0.723160in}}%
\pgfpathclose%
\pgfusepath{stroke}%
\end{pgfscope}%
\begin{pgfscope}%
\pgfpathrectangle{\pgfqpoint{0.847223in}{0.554012in}}{\pgfqpoint{6.200000in}{4.620000in}}%
\pgfusepath{clip}%
\pgfsetbuttcap%
\pgfsetroundjoin%
\pgfsetlinewidth{1.003750pt}%
\definecolor{currentstroke}{rgb}{1.000000,0.000000,0.000000}%
\pgfsetstrokecolor{currentstroke}%
\pgfsetdash{}{0pt}%
\pgfpathmoveto{\pgfqpoint{4.479143in}{0.722343in}}%
\pgfpathcurveto{\pgfqpoint{4.490193in}{0.722343in}}{\pgfqpoint{4.500792in}{0.726734in}}{\pgfqpoint{4.508606in}{0.734547in}}%
\pgfpathcurveto{\pgfqpoint{4.516419in}{0.742361in}}{\pgfqpoint{4.520810in}{0.752960in}}{\pgfqpoint{4.520810in}{0.764010in}}%
\pgfpathcurveto{\pgfqpoint{4.520810in}{0.775060in}}{\pgfqpoint{4.516419in}{0.785659in}}{\pgfqpoint{4.508606in}{0.793473in}}%
\pgfpathcurveto{\pgfqpoint{4.500792in}{0.801286in}}{\pgfqpoint{4.490193in}{0.805677in}}{\pgfqpoint{4.479143in}{0.805677in}}%
\pgfpathcurveto{\pgfqpoint{4.468093in}{0.805677in}}{\pgfqpoint{4.457494in}{0.801286in}}{\pgfqpoint{4.449680in}{0.793473in}}%
\pgfpathcurveto{\pgfqpoint{4.441867in}{0.785659in}}{\pgfqpoint{4.437476in}{0.775060in}}{\pgfqpoint{4.437476in}{0.764010in}}%
\pgfpathcurveto{\pgfqpoint{4.437476in}{0.752960in}}{\pgfqpoint{4.441867in}{0.742361in}}{\pgfqpoint{4.449680in}{0.734547in}}%
\pgfpathcurveto{\pgfqpoint{4.457494in}{0.726734in}}{\pgfqpoint{4.468093in}{0.722343in}}{\pgfqpoint{4.479143in}{0.722343in}}%
\pgfpathlineto{\pgfqpoint{4.479143in}{0.722343in}}%
\pgfpathclose%
\pgfusepath{stroke}%
\end{pgfscope}%
\begin{pgfscope}%
\pgfpathrectangle{\pgfqpoint{0.847223in}{0.554012in}}{\pgfqpoint{6.200000in}{4.620000in}}%
\pgfusepath{clip}%
\pgfsetbuttcap%
\pgfsetroundjoin%
\pgfsetlinewidth{1.003750pt}%
\definecolor{currentstroke}{rgb}{1.000000,0.000000,0.000000}%
\pgfsetstrokecolor{currentstroke}%
\pgfsetdash{}{0pt}%
\pgfpathmoveto{\pgfqpoint{4.484476in}{0.721529in}}%
\pgfpathcurveto{\pgfqpoint{4.495526in}{0.721529in}}{\pgfqpoint{4.506125in}{0.725919in}}{\pgfqpoint{4.513939in}{0.733733in}}%
\pgfpathcurveto{\pgfqpoint{4.521753in}{0.741546in}}{\pgfqpoint{4.526143in}{0.752145in}}{\pgfqpoint{4.526143in}{0.763195in}}%
\pgfpathcurveto{\pgfqpoint{4.526143in}{0.774246in}}{\pgfqpoint{4.521753in}{0.784845in}}{\pgfqpoint{4.513939in}{0.792658in}}%
\pgfpathcurveto{\pgfqpoint{4.506125in}{0.800472in}}{\pgfqpoint{4.495526in}{0.804862in}}{\pgfqpoint{4.484476in}{0.804862in}}%
\pgfpathcurveto{\pgfqpoint{4.473426in}{0.804862in}}{\pgfqpoint{4.462827in}{0.800472in}}{\pgfqpoint{4.455013in}{0.792658in}}%
\pgfpathcurveto{\pgfqpoint{4.447200in}{0.784845in}}{\pgfqpoint{4.442809in}{0.774246in}}{\pgfqpoint{4.442809in}{0.763195in}}%
\pgfpathcurveto{\pgfqpoint{4.442809in}{0.752145in}}{\pgfqpoint{4.447200in}{0.741546in}}{\pgfqpoint{4.455013in}{0.733733in}}%
\pgfpathcurveto{\pgfqpoint{4.462827in}{0.725919in}}{\pgfqpoint{4.473426in}{0.721529in}}{\pgfqpoint{4.484476in}{0.721529in}}%
\pgfpathlineto{\pgfqpoint{4.484476in}{0.721529in}}%
\pgfpathclose%
\pgfusepath{stroke}%
\end{pgfscope}%
\begin{pgfscope}%
\pgfpathrectangle{\pgfqpoint{0.847223in}{0.554012in}}{\pgfqpoint{6.200000in}{4.620000in}}%
\pgfusepath{clip}%
\pgfsetbuttcap%
\pgfsetroundjoin%
\pgfsetlinewidth{1.003750pt}%
\definecolor{currentstroke}{rgb}{1.000000,0.000000,0.000000}%
\pgfsetstrokecolor{currentstroke}%
\pgfsetdash{}{0pt}%
\pgfpathmoveto{\pgfqpoint{4.489809in}{0.720716in}}%
\pgfpathcurveto{\pgfqpoint{4.500859in}{0.720716in}}{\pgfqpoint{4.511459in}{0.725107in}}{\pgfqpoint{4.519272in}{0.732920in}}%
\pgfpathcurveto{\pgfqpoint{4.527086in}{0.740734in}}{\pgfqpoint{4.531476in}{0.751333in}}{\pgfqpoint{4.531476in}{0.762383in}}%
\pgfpathcurveto{\pgfqpoint{4.531476in}{0.773433in}}{\pgfqpoint{4.527086in}{0.784032in}}{\pgfqpoint{4.519272in}{0.791846in}}%
\pgfpathcurveto{\pgfqpoint{4.511459in}{0.799659in}}{\pgfqpoint{4.500859in}{0.804050in}}{\pgfqpoint{4.489809in}{0.804050in}}%
\pgfpathcurveto{\pgfqpoint{4.478759in}{0.804050in}}{\pgfqpoint{4.468160in}{0.799659in}}{\pgfqpoint{4.460347in}{0.791846in}}%
\pgfpathcurveto{\pgfqpoint{4.452533in}{0.784032in}}{\pgfqpoint{4.448143in}{0.773433in}}{\pgfqpoint{4.448143in}{0.762383in}}%
\pgfpathcurveto{\pgfqpoint{4.448143in}{0.751333in}}{\pgfqpoint{4.452533in}{0.740734in}}{\pgfqpoint{4.460347in}{0.732920in}}%
\pgfpathcurveto{\pgfqpoint{4.468160in}{0.725107in}}{\pgfqpoint{4.478759in}{0.720716in}}{\pgfqpoint{4.489809in}{0.720716in}}%
\pgfpathlineto{\pgfqpoint{4.489809in}{0.720716in}}%
\pgfpathclose%
\pgfusepath{stroke}%
\end{pgfscope}%
\begin{pgfscope}%
\pgfpathrectangle{\pgfqpoint{0.847223in}{0.554012in}}{\pgfqpoint{6.200000in}{4.620000in}}%
\pgfusepath{clip}%
\pgfsetbuttcap%
\pgfsetroundjoin%
\pgfsetlinewidth{1.003750pt}%
\definecolor{currentstroke}{rgb}{1.000000,0.000000,0.000000}%
\pgfsetstrokecolor{currentstroke}%
\pgfsetdash{}{0pt}%
\pgfpathmoveto{\pgfqpoint{4.495143in}{0.719906in}}%
\pgfpathcurveto{\pgfqpoint{4.506193in}{0.719906in}}{\pgfqpoint{4.516792in}{0.724296in}}{\pgfqpoint{4.524605in}{0.732110in}}%
\pgfpathcurveto{\pgfqpoint{4.532419in}{0.739923in}}{\pgfqpoint{4.536809in}{0.750522in}}{\pgfqpoint{4.536809in}{0.761573in}}%
\pgfpathcurveto{\pgfqpoint{4.536809in}{0.772623in}}{\pgfqpoint{4.532419in}{0.783222in}}{\pgfqpoint{4.524605in}{0.791035in}}%
\pgfpathcurveto{\pgfqpoint{4.516792in}{0.798849in}}{\pgfqpoint{4.506193in}{0.803239in}}{\pgfqpoint{4.495143in}{0.803239in}}%
\pgfpathcurveto{\pgfqpoint{4.484092in}{0.803239in}}{\pgfqpoint{4.473493in}{0.798849in}}{\pgfqpoint{4.465680in}{0.791035in}}%
\pgfpathcurveto{\pgfqpoint{4.457866in}{0.783222in}}{\pgfqpoint{4.453476in}{0.772623in}}{\pgfqpoint{4.453476in}{0.761573in}}%
\pgfpathcurveto{\pgfqpoint{4.453476in}{0.750522in}}{\pgfqpoint{4.457866in}{0.739923in}}{\pgfqpoint{4.465680in}{0.732110in}}%
\pgfpathcurveto{\pgfqpoint{4.473493in}{0.724296in}}{\pgfqpoint{4.484092in}{0.719906in}}{\pgfqpoint{4.495143in}{0.719906in}}%
\pgfpathlineto{\pgfqpoint{4.495143in}{0.719906in}}%
\pgfpathclose%
\pgfusepath{stroke}%
\end{pgfscope}%
\begin{pgfscope}%
\pgfpathrectangle{\pgfqpoint{0.847223in}{0.554012in}}{\pgfqpoint{6.200000in}{4.620000in}}%
\pgfusepath{clip}%
\pgfsetbuttcap%
\pgfsetroundjoin%
\pgfsetlinewidth{1.003750pt}%
\definecolor{currentstroke}{rgb}{1.000000,0.000000,0.000000}%
\pgfsetstrokecolor{currentstroke}%
\pgfsetdash{}{0pt}%
\pgfpathmoveto{\pgfqpoint{4.500476in}{0.719098in}}%
\pgfpathcurveto{\pgfqpoint{4.511526in}{0.719098in}}{\pgfqpoint{4.522125in}{0.723488in}}{\pgfqpoint{4.529939in}{0.731301in}}%
\pgfpathcurveto{\pgfqpoint{4.537752in}{0.739115in}}{\pgfqpoint{4.542142in}{0.749714in}}{\pgfqpoint{4.542142in}{0.760764in}}%
\pgfpathcurveto{\pgfqpoint{4.542142in}{0.771814in}}{\pgfqpoint{4.537752in}{0.782413in}}{\pgfqpoint{4.529939in}{0.790227in}}%
\pgfpathcurveto{\pgfqpoint{4.522125in}{0.798041in}}{\pgfqpoint{4.511526in}{0.802431in}}{\pgfqpoint{4.500476in}{0.802431in}}%
\pgfpathcurveto{\pgfqpoint{4.489426in}{0.802431in}}{\pgfqpoint{4.478827in}{0.798041in}}{\pgfqpoint{4.471013in}{0.790227in}}%
\pgfpathcurveto{\pgfqpoint{4.463199in}{0.782413in}}{\pgfqpoint{4.458809in}{0.771814in}}{\pgfqpoint{4.458809in}{0.760764in}}%
\pgfpathcurveto{\pgfqpoint{4.458809in}{0.749714in}}{\pgfqpoint{4.463199in}{0.739115in}}{\pgfqpoint{4.471013in}{0.731301in}}%
\pgfpathcurveto{\pgfqpoint{4.478827in}{0.723488in}}{\pgfqpoint{4.489426in}{0.719098in}}{\pgfqpoint{4.500476in}{0.719098in}}%
\pgfpathlineto{\pgfqpoint{4.500476in}{0.719098in}}%
\pgfpathclose%
\pgfusepath{stroke}%
\end{pgfscope}%
\begin{pgfscope}%
\pgfpathrectangle{\pgfqpoint{0.847223in}{0.554012in}}{\pgfqpoint{6.200000in}{4.620000in}}%
\pgfusepath{clip}%
\pgfsetbuttcap%
\pgfsetroundjoin%
\pgfsetlinewidth{1.003750pt}%
\definecolor{currentstroke}{rgb}{1.000000,0.000000,0.000000}%
\pgfsetstrokecolor{currentstroke}%
\pgfsetdash{}{0pt}%
\pgfpathmoveto{\pgfqpoint{4.505809in}{0.718291in}}%
\pgfpathcurveto{\pgfqpoint{4.516859in}{0.718291in}}{\pgfqpoint{4.527458in}{0.722682in}}{\pgfqpoint{4.535272in}{0.730495in}}%
\pgfpathcurveto{\pgfqpoint{4.543085in}{0.738309in}}{\pgfqpoint{4.547476in}{0.748908in}}{\pgfqpoint{4.547476in}{0.759958in}}%
\pgfpathcurveto{\pgfqpoint{4.547476in}{0.771008in}}{\pgfqpoint{4.543085in}{0.781607in}}{\pgfqpoint{4.535272in}{0.789421in}}%
\pgfpathcurveto{\pgfqpoint{4.527458in}{0.797234in}}{\pgfqpoint{4.516859in}{0.801625in}}{\pgfqpoint{4.505809in}{0.801625in}}%
\pgfpathcurveto{\pgfqpoint{4.494759in}{0.801625in}}{\pgfqpoint{4.484160in}{0.797234in}}{\pgfqpoint{4.476346in}{0.789421in}}%
\pgfpathcurveto{\pgfqpoint{4.468533in}{0.781607in}}{\pgfqpoint{4.464142in}{0.771008in}}{\pgfqpoint{4.464142in}{0.759958in}}%
\pgfpathcurveto{\pgfqpoint{4.464142in}{0.748908in}}{\pgfqpoint{4.468533in}{0.738309in}}{\pgfqpoint{4.476346in}{0.730495in}}%
\pgfpathcurveto{\pgfqpoint{4.484160in}{0.722682in}}{\pgfqpoint{4.494759in}{0.718291in}}{\pgfqpoint{4.505809in}{0.718291in}}%
\pgfpathlineto{\pgfqpoint{4.505809in}{0.718291in}}%
\pgfpathclose%
\pgfusepath{stroke}%
\end{pgfscope}%
\begin{pgfscope}%
\pgfpathrectangle{\pgfqpoint{0.847223in}{0.554012in}}{\pgfqpoint{6.200000in}{4.620000in}}%
\pgfusepath{clip}%
\pgfsetbuttcap%
\pgfsetroundjoin%
\pgfsetlinewidth{1.003750pt}%
\definecolor{currentstroke}{rgb}{1.000000,0.000000,0.000000}%
\pgfsetstrokecolor{currentstroke}%
\pgfsetdash{}{0pt}%
\pgfpathmoveto{\pgfqpoint{4.511142in}{0.717487in}}%
\pgfpathcurveto{\pgfqpoint{4.522192in}{0.717487in}}{\pgfqpoint{4.532791in}{0.721877in}}{\pgfqpoint{4.540605in}{0.729691in}}%
\pgfpathcurveto{\pgfqpoint{4.548419in}{0.737505in}}{\pgfqpoint{4.552809in}{0.748104in}}{\pgfqpoint{4.552809in}{0.759154in}}%
\pgfpathcurveto{\pgfqpoint{4.552809in}{0.770204in}}{\pgfqpoint{4.548419in}{0.780803in}}{\pgfqpoint{4.540605in}{0.788617in}}%
\pgfpathcurveto{\pgfqpoint{4.532791in}{0.796430in}}{\pgfqpoint{4.522192in}{0.800820in}}{\pgfqpoint{4.511142in}{0.800820in}}%
\pgfpathcurveto{\pgfqpoint{4.500092in}{0.800820in}}{\pgfqpoint{4.489493in}{0.796430in}}{\pgfqpoint{4.481679in}{0.788617in}}%
\pgfpathcurveto{\pgfqpoint{4.473866in}{0.780803in}}{\pgfqpoint{4.469476in}{0.770204in}}{\pgfqpoint{4.469476in}{0.759154in}}%
\pgfpathcurveto{\pgfqpoint{4.469476in}{0.748104in}}{\pgfqpoint{4.473866in}{0.737505in}}{\pgfqpoint{4.481679in}{0.729691in}}%
\pgfpathcurveto{\pgfqpoint{4.489493in}{0.721877in}}{\pgfqpoint{4.500092in}{0.717487in}}{\pgfqpoint{4.511142in}{0.717487in}}%
\pgfpathlineto{\pgfqpoint{4.511142in}{0.717487in}}%
\pgfpathclose%
\pgfusepath{stroke}%
\end{pgfscope}%
\begin{pgfscope}%
\pgfpathrectangle{\pgfqpoint{0.847223in}{0.554012in}}{\pgfqpoint{6.200000in}{4.620000in}}%
\pgfusepath{clip}%
\pgfsetbuttcap%
\pgfsetroundjoin%
\pgfsetlinewidth{1.003750pt}%
\definecolor{currentstroke}{rgb}{1.000000,0.000000,0.000000}%
\pgfsetstrokecolor{currentstroke}%
\pgfsetdash{}{0pt}%
\pgfpathmoveto{\pgfqpoint{4.516475in}{0.716685in}}%
\pgfpathcurveto{\pgfqpoint{4.527526in}{0.716685in}}{\pgfqpoint{4.538125in}{0.721075in}}{\pgfqpoint{4.545938in}{0.728889in}}%
\pgfpathcurveto{\pgfqpoint{4.553752in}{0.736702in}}{\pgfqpoint{4.558142in}{0.747301in}}{\pgfqpoint{4.558142in}{0.758352in}}%
\pgfpathcurveto{\pgfqpoint{4.558142in}{0.769402in}}{\pgfqpoint{4.553752in}{0.780001in}}{\pgfqpoint{4.545938in}{0.787814in}}%
\pgfpathcurveto{\pgfqpoint{4.538125in}{0.795628in}}{\pgfqpoint{4.527526in}{0.800018in}}{\pgfqpoint{4.516475in}{0.800018in}}%
\pgfpathcurveto{\pgfqpoint{4.505425in}{0.800018in}}{\pgfqpoint{4.494826in}{0.795628in}}{\pgfqpoint{4.487013in}{0.787814in}}%
\pgfpathcurveto{\pgfqpoint{4.479199in}{0.780001in}}{\pgfqpoint{4.474809in}{0.769402in}}{\pgfqpoint{4.474809in}{0.758352in}}%
\pgfpathcurveto{\pgfqpoint{4.474809in}{0.747301in}}{\pgfqpoint{4.479199in}{0.736702in}}{\pgfqpoint{4.487013in}{0.728889in}}%
\pgfpathcurveto{\pgfqpoint{4.494826in}{0.721075in}}{\pgfqpoint{4.505425in}{0.716685in}}{\pgfqpoint{4.516475in}{0.716685in}}%
\pgfpathlineto{\pgfqpoint{4.516475in}{0.716685in}}%
\pgfpathclose%
\pgfusepath{stroke}%
\end{pgfscope}%
\begin{pgfscope}%
\pgfpathrectangle{\pgfqpoint{0.847223in}{0.554012in}}{\pgfqpoint{6.200000in}{4.620000in}}%
\pgfusepath{clip}%
\pgfsetbuttcap%
\pgfsetroundjoin%
\pgfsetlinewidth{1.003750pt}%
\definecolor{currentstroke}{rgb}{1.000000,0.000000,0.000000}%
\pgfsetstrokecolor{currentstroke}%
\pgfsetdash{}{0pt}%
\pgfpathmoveto{\pgfqpoint{4.521809in}{0.715885in}}%
\pgfpathcurveto{\pgfqpoint{4.532859in}{0.715885in}}{\pgfqpoint{4.543458in}{0.720275in}}{\pgfqpoint{4.551271in}{0.728089in}}%
\pgfpathcurveto{\pgfqpoint{4.559085in}{0.735902in}}{\pgfqpoint{4.563475in}{0.746501in}}{\pgfqpoint{4.563475in}{0.757551in}}%
\pgfpathcurveto{\pgfqpoint{4.563475in}{0.768602in}}{\pgfqpoint{4.559085in}{0.779201in}}{\pgfqpoint{4.551271in}{0.787014in}}%
\pgfpathcurveto{\pgfqpoint{4.543458in}{0.794828in}}{\pgfqpoint{4.532859in}{0.799218in}}{\pgfqpoint{4.521809in}{0.799218in}}%
\pgfpathcurveto{\pgfqpoint{4.510759in}{0.799218in}}{\pgfqpoint{4.500159in}{0.794828in}}{\pgfqpoint{4.492346in}{0.787014in}}%
\pgfpathcurveto{\pgfqpoint{4.484532in}{0.779201in}}{\pgfqpoint{4.480142in}{0.768602in}}{\pgfqpoint{4.480142in}{0.757551in}}%
\pgfpathcurveto{\pgfqpoint{4.480142in}{0.746501in}}{\pgfqpoint{4.484532in}{0.735902in}}{\pgfqpoint{4.492346in}{0.728089in}}%
\pgfpathcurveto{\pgfqpoint{4.500159in}{0.720275in}}{\pgfqpoint{4.510759in}{0.715885in}}{\pgfqpoint{4.521809in}{0.715885in}}%
\pgfpathlineto{\pgfqpoint{4.521809in}{0.715885in}}%
\pgfpathclose%
\pgfusepath{stroke}%
\end{pgfscope}%
\begin{pgfscope}%
\pgfpathrectangle{\pgfqpoint{0.847223in}{0.554012in}}{\pgfqpoint{6.200000in}{4.620000in}}%
\pgfusepath{clip}%
\pgfsetbuttcap%
\pgfsetroundjoin%
\pgfsetlinewidth{1.003750pt}%
\definecolor{currentstroke}{rgb}{1.000000,0.000000,0.000000}%
\pgfsetstrokecolor{currentstroke}%
\pgfsetdash{}{0pt}%
\pgfpathmoveto{\pgfqpoint{4.527142in}{0.715087in}}%
\pgfpathcurveto{\pgfqpoint{4.538192in}{0.715087in}}{\pgfqpoint{4.548791in}{0.719477in}}{\pgfqpoint{4.556605in}{0.727291in}}%
\pgfpathcurveto{\pgfqpoint{4.564418in}{0.735104in}}{\pgfqpoint{4.568809in}{0.745703in}}{\pgfqpoint{4.568809in}{0.756753in}}%
\pgfpathcurveto{\pgfqpoint{4.568809in}{0.767803in}}{\pgfqpoint{4.564418in}{0.778402in}}{\pgfqpoint{4.556605in}{0.786216in}}%
\pgfpathcurveto{\pgfqpoint{4.548791in}{0.794030in}}{\pgfqpoint{4.538192in}{0.798420in}}{\pgfqpoint{4.527142in}{0.798420in}}%
\pgfpathcurveto{\pgfqpoint{4.516092in}{0.798420in}}{\pgfqpoint{4.505493in}{0.794030in}}{\pgfqpoint{4.497679in}{0.786216in}}%
\pgfpathcurveto{\pgfqpoint{4.489865in}{0.778402in}}{\pgfqpoint{4.485475in}{0.767803in}}{\pgfqpoint{4.485475in}{0.756753in}}%
\pgfpathcurveto{\pgfqpoint{4.485475in}{0.745703in}}{\pgfqpoint{4.489865in}{0.735104in}}{\pgfqpoint{4.497679in}{0.727291in}}%
\pgfpathcurveto{\pgfqpoint{4.505493in}{0.719477in}}{\pgfqpoint{4.516092in}{0.715087in}}{\pgfqpoint{4.527142in}{0.715087in}}%
\pgfpathlineto{\pgfqpoint{4.527142in}{0.715087in}}%
\pgfpathclose%
\pgfusepath{stroke}%
\end{pgfscope}%
\begin{pgfscope}%
\pgfpathrectangle{\pgfqpoint{0.847223in}{0.554012in}}{\pgfqpoint{6.200000in}{4.620000in}}%
\pgfusepath{clip}%
\pgfsetbuttcap%
\pgfsetroundjoin%
\pgfsetlinewidth{1.003750pt}%
\definecolor{currentstroke}{rgb}{1.000000,0.000000,0.000000}%
\pgfsetstrokecolor{currentstroke}%
\pgfsetdash{}{0pt}%
\pgfpathmoveto{\pgfqpoint{4.532475in}{0.714291in}}%
\pgfpathcurveto{\pgfqpoint{4.543525in}{0.714291in}}{\pgfqpoint{4.554124in}{0.718681in}}{\pgfqpoint{4.561938in}{0.726494in}}%
\pgfpathcurveto{\pgfqpoint{4.569751in}{0.734308in}}{\pgfqpoint{4.574142in}{0.744907in}}{\pgfqpoint{4.574142in}{0.755957in}}%
\pgfpathcurveto{\pgfqpoint{4.574142in}{0.767007in}}{\pgfqpoint{4.569751in}{0.777606in}}{\pgfqpoint{4.561938in}{0.785420in}}%
\pgfpathcurveto{\pgfqpoint{4.554124in}{0.793234in}}{\pgfqpoint{4.543525in}{0.797624in}}{\pgfqpoint{4.532475in}{0.797624in}}%
\pgfpathcurveto{\pgfqpoint{4.521425in}{0.797624in}}{\pgfqpoint{4.510826in}{0.793234in}}{\pgfqpoint{4.503012in}{0.785420in}}%
\pgfpathcurveto{\pgfqpoint{4.495199in}{0.777606in}}{\pgfqpoint{4.490808in}{0.767007in}}{\pgfqpoint{4.490808in}{0.755957in}}%
\pgfpathcurveto{\pgfqpoint{4.490808in}{0.744907in}}{\pgfqpoint{4.495199in}{0.734308in}}{\pgfqpoint{4.503012in}{0.726494in}}%
\pgfpathcurveto{\pgfqpoint{4.510826in}{0.718681in}}{\pgfqpoint{4.521425in}{0.714291in}}{\pgfqpoint{4.532475in}{0.714291in}}%
\pgfpathlineto{\pgfqpoint{4.532475in}{0.714291in}}%
\pgfpathclose%
\pgfusepath{stroke}%
\end{pgfscope}%
\begin{pgfscope}%
\pgfpathrectangle{\pgfqpoint{0.847223in}{0.554012in}}{\pgfqpoint{6.200000in}{4.620000in}}%
\pgfusepath{clip}%
\pgfsetbuttcap%
\pgfsetroundjoin%
\pgfsetlinewidth{1.003750pt}%
\definecolor{currentstroke}{rgb}{1.000000,0.000000,0.000000}%
\pgfsetstrokecolor{currentstroke}%
\pgfsetdash{}{0pt}%
\pgfpathmoveto{\pgfqpoint{4.537808in}{0.713496in}}%
\pgfpathcurveto{\pgfqpoint{4.548858in}{0.713496in}}{\pgfqpoint{4.559457in}{0.717887in}}{\pgfqpoint{4.567271in}{0.725700in}}%
\pgfpathcurveto{\pgfqpoint{4.575085in}{0.733514in}}{\pgfqpoint{4.579475in}{0.744113in}}{\pgfqpoint{4.579475in}{0.755163in}}%
\pgfpathcurveto{\pgfqpoint{4.579475in}{0.766213in}}{\pgfqpoint{4.575085in}{0.776812in}}{\pgfqpoint{4.567271in}{0.784626in}}%
\pgfpathcurveto{\pgfqpoint{4.559457in}{0.792440in}}{\pgfqpoint{4.548858in}{0.796830in}}{\pgfqpoint{4.537808in}{0.796830in}}%
\pgfpathcurveto{\pgfqpoint{4.526758in}{0.796830in}}{\pgfqpoint{4.516159in}{0.792440in}}{\pgfqpoint{4.508346in}{0.784626in}}%
\pgfpathcurveto{\pgfqpoint{4.500532in}{0.776812in}}{\pgfqpoint{4.496142in}{0.766213in}}{\pgfqpoint{4.496142in}{0.755163in}}%
\pgfpathcurveto{\pgfqpoint{4.496142in}{0.744113in}}{\pgfqpoint{4.500532in}{0.733514in}}{\pgfqpoint{4.508346in}{0.725700in}}%
\pgfpathcurveto{\pgfqpoint{4.516159in}{0.717887in}}{\pgfqpoint{4.526758in}{0.713496in}}{\pgfqpoint{4.537808in}{0.713496in}}%
\pgfpathlineto{\pgfqpoint{4.537808in}{0.713496in}}%
\pgfpathclose%
\pgfusepath{stroke}%
\end{pgfscope}%
\begin{pgfscope}%
\pgfpathrectangle{\pgfqpoint{0.847223in}{0.554012in}}{\pgfqpoint{6.200000in}{4.620000in}}%
\pgfusepath{clip}%
\pgfsetbuttcap%
\pgfsetroundjoin%
\pgfsetlinewidth{1.003750pt}%
\definecolor{currentstroke}{rgb}{1.000000,0.000000,0.000000}%
\pgfsetstrokecolor{currentstroke}%
\pgfsetdash{}{0pt}%
\pgfpathmoveto{\pgfqpoint{4.543142in}{0.712704in}}%
\pgfpathcurveto{\pgfqpoint{4.554192in}{0.712704in}}{\pgfqpoint{4.564791in}{0.717095in}}{\pgfqpoint{4.572604in}{0.724908in}}%
\pgfpathcurveto{\pgfqpoint{4.580418in}{0.732722in}}{\pgfqpoint{4.584808in}{0.743321in}}{\pgfqpoint{4.584808in}{0.754371in}}%
\pgfpathcurveto{\pgfqpoint{4.584808in}{0.765421in}}{\pgfqpoint{4.580418in}{0.776020in}}{\pgfqpoint{4.572604in}{0.783834in}}%
\pgfpathcurveto{\pgfqpoint{4.564791in}{0.791648in}}{\pgfqpoint{4.554192in}{0.796038in}}{\pgfqpoint{4.543142in}{0.796038in}}%
\pgfpathcurveto{\pgfqpoint{4.532091in}{0.796038in}}{\pgfqpoint{4.521492in}{0.791648in}}{\pgfqpoint{4.513679in}{0.783834in}}%
\pgfpathcurveto{\pgfqpoint{4.505865in}{0.776020in}}{\pgfqpoint{4.501475in}{0.765421in}}{\pgfqpoint{4.501475in}{0.754371in}}%
\pgfpathcurveto{\pgfqpoint{4.501475in}{0.743321in}}{\pgfqpoint{4.505865in}{0.732722in}}{\pgfqpoint{4.513679in}{0.724908in}}%
\pgfpathcurveto{\pgfqpoint{4.521492in}{0.717095in}}{\pgfqpoint{4.532091in}{0.712704in}}{\pgfqpoint{4.543142in}{0.712704in}}%
\pgfpathlineto{\pgfqpoint{4.543142in}{0.712704in}}%
\pgfpathclose%
\pgfusepath{stroke}%
\end{pgfscope}%
\begin{pgfscope}%
\pgfpathrectangle{\pgfqpoint{0.847223in}{0.554012in}}{\pgfqpoint{6.200000in}{4.620000in}}%
\pgfusepath{clip}%
\pgfsetbuttcap%
\pgfsetroundjoin%
\pgfsetlinewidth{1.003750pt}%
\definecolor{currentstroke}{rgb}{1.000000,0.000000,0.000000}%
\pgfsetstrokecolor{currentstroke}%
\pgfsetdash{}{0pt}%
\pgfpathmoveto{\pgfqpoint{4.548475in}{0.711914in}}%
\pgfpathcurveto{\pgfqpoint{4.559525in}{0.711914in}}{\pgfqpoint{4.570124in}{0.716305in}}{\pgfqpoint{4.577938in}{0.724118in}}%
\pgfpathcurveto{\pgfqpoint{4.585751in}{0.731932in}}{\pgfqpoint{4.590141in}{0.742531in}}{\pgfqpoint{4.590141in}{0.753581in}}%
\pgfpathcurveto{\pgfqpoint{4.590141in}{0.764631in}}{\pgfqpoint{4.585751in}{0.775230in}}{\pgfqpoint{4.577938in}{0.783044in}}%
\pgfpathcurveto{\pgfqpoint{4.570124in}{0.790857in}}{\pgfqpoint{4.559525in}{0.795248in}}{\pgfqpoint{4.548475in}{0.795248in}}%
\pgfpathcurveto{\pgfqpoint{4.537425in}{0.795248in}}{\pgfqpoint{4.526826in}{0.790857in}}{\pgfqpoint{4.519012in}{0.783044in}}%
\pgfpathcurveto{\pgfqpoint{4.511198in}{0.775230in}}{\pgfqpoint{4.506808in}{0.764631in}}{\pgfqpoint{4.506808in}{0.753581in}}%
\pgfpathcurveto{\pgfqpoint{4.506808in}{0.742531in}}{\pgfqpoint{4.511198in}{0.731932in}}{\pgfqpoint{4.519012in}{0.724118in}}%
\pgfpathcurveto{\pgfqpoint{4.526826in}{0.716305in}}{\pgfqpoint{4.537425in}{0.711914in}}{\pgfqpoint{4.548475in}{0.711914in}}%
\pgfpathlineto{\pgfqpoint{4.548475in}{0.711914in}}%
\pgfpathclose%
\pgfusepath{stroke}%
\end{pgfscope}%
\begin{pgfscope}%
\pgfpathrectangle{\pgfqpoint{0.847223in}{0.554012in}}{\pgfqpoint{6.200000in}{4.620000in}}%
\pgfusepath{clip}%
\pgfsetbuttcap%
\pgfsetroundjoin%
\pgfsetlinewidth{1.003750pt}%
\definecolor{currentstroke}{rgb}{1.000000,0.000000,0.000000}%
\pgfsetstrokecolor{currentstroke}%
\pgfsetdash{}{0pt}%
\pgfpathmoveto{\pgfqpoint{4.553808in}{0.711126in}}%
\pgfpathcurveto{\pgfqpoint{4.564858in}{0.711126in}}{\pgfqpoint{4.575457in}{0.715517in}}{\pgfqpoint{4.583271in}{0.723330in}}%
\pgfpathcurveto{\pgfqpoint{4.591084in}{0.731144in}}{\pgfqpoint{4.595475in}{0.741743in}}{\pgfqpoint{4.595475in}{0.752793in}}%
\pgfpathcurveto{\pgfqpoint{4.595475in}{0.763843in}}{\pgfqpoint{4.591084in}{0.774442in}}{\pgfqpoint{4.583271in}{0.782256in}}%
\pgfpathcurveto{\pgfqpoint{4.575457in}{0.790069in}}{\pgfqpoint{4.564858in}{0.794460in}}{\pgfqpoint{4.553808in}{0.794460in}}%
\pgfpathcurveto{\pgfqpoint{4.542758in}{0.794460in}}{\pgfqpoint{4.532159in}{0.790069in}}{\pgfqpoint{4.524345in}{0.782256in}}%
\pgfpathcurveto{\pgfqpoint{4.516532in}{0.774442in}}{\pgfqpoint{4.512141in}{0.763843in}}{\pgfqpoint{4.512141in}{0.752793in}}%
\pgfpathcurveto{\pgfqpoint{4.512141in}{0.741743in}}{\pgfqpoint{4.516532in}{0.731144in}}{\pgfqpoint{4.524345in}{0.723330in}}%
\pgfpathcurveto{\pgfqpoint{4.532159in}{0.715517in}}{\pgfqpoint{4.542758in}{0.711126in}}{\pgfqpoint{4.553808in}{0.711126in}}%
\pgfpathlineto{\pgfqpoint{4.553808in}{0.711126in}}%
\pgfpathclose%
\pgfusepath{stroke}%
\end{pgfscope}%
\begin{pgfscope}%
\pgfpathrectangle{\pgfqpoint{0.847223in}{0.554012in}}{\pgfqpoint{6.200000in}{4.620000in}}%
\pgfusepath{clip}%
\pgfsetbuttcap%
\pgfsetroundjoin%
\pgfsetlinewidth{1.003750pt}%
\definecolor{currentstroke}{rgb}{1.000000,0.000000,0.000000}%
\pgfsetstrokecolor{currentstroke}%
\pgfsetdash{}{0pt}%
\pgfpathmoveto{\pgfqpoint{4.559141in}{0.710340in}}%
\pgfpathcurveto{\pgfqpoint{4.570191in}{0.710340in}}{\pgfqpoint{4.580790in}{0.714730in}}{\pgfqpoint{4.588604in}{0.722544in}}%
\pgfpathcurveto{\pgfqpoint{4.596418in}{0.730358in}}{\pgfqpoint{4.600808in}{0.740957in}}{\pgfqpoint{4.600808in}{0.752007in}}%
\pgfpathcurveto{\pgfqpoint{4.600808in}{0.763057in}}{\pgfqpoint{4.596418in}{0.773656in}}{\pgfqpoint{4.588604in}{0.781470in}}%
\pgfpathcurveto{\pgfqpoint{4.580790in}{0.789283in}}{\pgfqpoint{4.570191in}{0.793674in}}{\pgfqpoint{4.559141in}{0.793674in}}%
\pgfpathcurveto{\pgfqpoint{4.548091in}{0.793674in}}{\pgfqpoint{4.537492in}{0.789283in}}{\pgfqpoint{4.529678in}{0.781470in}}%
\pgfpathcurveto{\pgfqpoint{4.521865in}{0.773656in}}{\pgfqpoint{4.517474in}{0.763057in}}{\pgfqpoint{4.517474in}{0.752007in}}%
\pgfpathcurveto{\pgfqpoint{4.517474in}{0.740957in}}{\pgfqpoint{4.521865in}{0.730358in}}{\pgfqpoint{4.529678in}{0.722544in}}%
\pgfpathcurveto{\pgfqpoint{4.537492in}{0.714730in}}{\pgfqpoint{4.548091in}{0.710340in}}{\pgfqpoint{4.559141in}{0.710340in}}%
\pgfpathlineto{\pgfqpoint{4.559141in}{0.710340in}}%
\pgfpathclose%
\pgfusepath{stroke}%
\end{pgfscope}%
\begin{pgfscope}%
\pgfpathrectangle{\pgfqpoint{0.847223in}{0.554012in}}{\pgfqpoint{6.200000in}{4.620000in}}%
\pgfusepath{clip}%
\pgfsetbuttcap%
\pgfsetroundjoin%
\pgfsetlinewidth{1.003750pt}%
\definecolor{currentstroke}{rgb}{1.000000,0.000000,0.000000}%
\pgfsetstrokecolor{currentstroke}%
\pgfsetdash{}{0pt}%
\pgfpathmoveto{\pgfqpoint{4.564474in}{0.709556in}}%
\pgfpathcurveto{\pgfqpoint{4.575524in}{0.709556in}}{\pgfqpoint{4.586124in}{0.713946in}}{\pgfqpoint{4.593937in}{0.721760in}}%
\pgfpathcurveto{\pgfqpoint{4.601751in}{0.729574in}}{\pgfqpoint{4.606141in}{0.740173in}}{\pgfqpoint{4.606141in}{0.751223in}}%
\pgfpathcurveto{\pgfqpoint{4.606141in}{0.762273in}}{\pgfqpoint{4.601751in}{0.772872in}}{\pgfqpoint{4.593937in}{0.780686in}}%
\pgfpathcurveto{\pgfqpoint{4.586124in}{0.788499in}}{\pgfqpoint{4.575524in}{0.792889in}}{\pgfqpoint{4.564474in}{0.792889in}}%
\pgfpathcurveto{\pgfqpoint{4.553424in}{0.792889in}}{\pgfqpoint{4.542825in}{0.788499in}}{\pgfqpoint{4.535012in}{0.780686in}}%
\pgfpathcurveto{\pgfqpoint{4.527198in}{0.772872in}}{\pgfqpoint{4.522808in}{0.762273in}}{\pgfqpoint{4.522808in}{0.751223in}}%
\pgfpathcurveto{\pgfqpoint{4.522808in}{0.740173in}}{\pgfqpoint{4.527198in}{0.729574in}}{\pgfqpoint{4.535012in}{0.721760in}}%
\pgfpathcurveto{\pgfqpoint{4.542825in}{0.713946in}}{\pgfqpoint{4.553424in}{0.709556in}}{\pgfqpoint{4.564474in}{0.709556in}}%
\pgfpathlineto{\pgfqpoint{4.564474in}{0.709556in}}%
\pgfpathclose%
\pgfusepath{stroke}%
\end{pgfscope}%
\begin{pgfscope}%
\pgfpathrectangle{\pgfqpoint{0.847223in}{0.554012in}}{\pgfqpoint{6.200000in}{4.620000in}}%
\pgfusepath{clip}%
\pgfsetbuttcap%
\pgfsetroundjoin%
\pgfsetlinewidth{1.003750pt}%
\definecolor{currentstroke}{rgb}{1.000000,0.000000,0.000000}%
\pgfsetstrokecolor{currentstroke}%
\pgfsetdash{}{0pt}%
\pgfpathmoveto{\pgfqpoint{4.569808in}{0.708774in}}%
\pgfpathcurveto{\pgfqpoint{4.580858in}{0.708774in}}{\pgfqpoint{4.591457in}{0.713164in}}{\pgfqpoint{4.599270in}{0.720978in}}%
\pgfpathcurveto{\pgfqpoint{4.607084in}{0.728791in}}{\pgfqpoint{4.611474in}{0.739391in}}{\pgfqpoint{4.611474in}{0.750441in}}%
\pgfpathcurveto{\pgfqpoint{4.611474in}{0.761491in}}{\pgfqpoint{4.607084in}{0.772090in}}{\pgfqpoint{4.599270in}{0.779903in}}%
\pgfpathcurveto{\pgfqpoint{4.591457in}{0.787717in}}{\pgfqpoint{4.580858in}{0.792107in}}{\pgfqpoint{4.569808in}{0.792107in}}%
\pgfpathcurveto{\pgfqpoint{4.558757in}{0.792107in}}{\pgfqpoint{4.548158in}{0.787717in}}{\pgfqpoint{4.540345in}{0.779903in}}%
\pgfpathcurveto{\pgfqpoint{4.532531in}{0.772090in}}{\pgfqpoint{4.528141in}{0.761491in}}{\pgfqpoint{4.528141in}{0.750441in}}%
\pgfpathcurveto{\pgfqpoint{4.528141in}{0.739391in}}{\pgfqpoint{4.532531in}{0.728791in}}{\pgfqpoint{4.540345in}{0.720978in}}%
\pgfpathcurveto{\pgfqpoint{4.548158in}{0.713164in}}{\pgfqpoint{4.558757in}{0.708774in}}{\pgfqpoint{4.569808in}{0.708774in}}%
\pgfpathlineto{\pgfqpoint{4.569808in}{0.708774in}}%
\pgfpathclose%
\pgfusepath{stroke}%
\end{pgfscope}%
\begin{pgfscope}%
\pgfpathrectangle{\pgfqpoint{0.847223in}{0.554012in}}{\pgfqpoint{6.200000in}{4.620000in}}%
\pgfusepath{clip}%
\pgfsetbuttcap%
\pgfsetroundjoin%
\pgfsetlinewidth{1.003750pt}%
\definecolor{currentstroke}{rgb}{1.000000,0.000000,0.000000}%
\pgfsetstrokecolor{currentstroke}%
\pgfsetdash{}{0pt}%
\pgfpathmoveto{\pgfqpoint{4.575141in}{0.707994in}}%
\pgfpathcurveto{\pgfqpoint{4.586191in}{0.707994in}}{\pgfqpoint{4.596790in}{0.712384in}}{\pgfqpoint{4.604604in}{0.720198in}}%
\pgfpathcurveto{\pgfqpoint{4.612417in}{0.728011in}}{\pgfqpoint{4.616807in}{0.738610in}}{\pgfqpoint{4.616807in}{0.749660in}}%
\pgfpathcurveto{\pgfqpoint{4.616807in}{0.760711in}}{\pgfqpoint{4.612417in}{0.771310in}}{\pgfqpoint{4.604604in}{0.779123in}}%
\pgfpathcurveto{\pgfqpoint{4.596790in}{0.786937in}}{\pgfqpoint{4.586191in}{0.791327in}}{\pgfqpoint{4.575141in}{0.791327in}}%
\pgfpathcurveto{\pgfqpoint{4.564091in}{0.791327in}}{\pgfqpoint{4.553492in}{0.786937in}}{\pgfqpoint{4.545678in}{0.779123in}}%
\pgfpathcurveto{\pgfqpoint{4.537864in}{0.771310in}}{\pgfqpoint{4.533474in}{0.760711in}}{\pgfqpoint{4.533474in}{0.749660in}}%
\pgfpathcurveto{\pgfqpoint{4.533474in}{0.738610in}}{\pgfqpoint{4.537864in}{0.728011in}}{\pgfqpoint{4.545678in}{0.720198in}}%
\pgfpathcurveto{\pgfqpoint{4.553492in}{0.712384in}}{\pgfqpoint{4.564091in}{0.707994in}}{\pgfqpoint{4.575141in}{0.707994in}}%
\pgfpathlineto{\pgfqpoint{4.575141in}{0.707994in}}%
\pgfpathclose%
\pgfusepath{stroke}%
\end{pgfscope}%
\begin{pgfscope}%
\pgfpathrectangle{\pgfqpoint{0.847223in}{0.554012in}}{\pgfqpoint{6.200000in}{4.620000in}}%
\pgfusepath{clip}%
\pgfsetbuttcap%
\pgfsetroundjoin%
\pgfsetlinewidth{1.003750pt}%
\definecolor{currentstroke}{rgb}{1.000000,0.000000,0.000000}%
\pgfsetstrokecolor{currentstroke}%
\pgfsetdash{}{0pt}%
\pgfpathmoveto{\pgfqpoint{4.580474in}{0.707216in}}%
\pgfpathcurveto{\pgfqpoint{4.591524in}{0.707216in}}{\pgfqpoint{4.602123in}{0.711606in}}{\pgfqpoint{4.609937in}{0.719419in}}%
\pgfpathcurveto{\pgfqpoint{4.617750in}{0.727233in}}{\pgfqpoint{4.622141in}{0.737832in}}{\pgfqpoint{4.622141in}{0.748882in}}%
\pgfpathcurveto{\pgfqpoint{4.622141in}{0.759932in}}{\pgfqpoint{4.617750in}{0.770531in}}{\pgfqpoint{4.609937in}{0.778345in}}%
\pgfpathcurveto{\pgfqpoint{4.602123in}{0.786159in}}{\pgfqpoint{4.591524in}{0.790549in}}{\pgfqpoint{4.580474in}{0.790549in}}%
\pgfpathcurveto{\pgfqpoint{4.569424in}{0.790549in}}{\pgfqpoint{4.558825in}{0.786159in}}{\pgfqpoint{4.551011in}{0.778345in}}%
\pgfpathcurveto{\pgfqpoint{4.543198in}{0.770531in}}{\pgfqpoint{4.538807in}{0.759932in}}{\pgfqpoint{4.538807in}{0.748882in}}%
\pgfpathcurveto{\pgfqpoint{4.538807in}{0.737832in}}{\pgfqpoint{4.543198in}{0.727233in}}{\pgfqpoint{4.551011in}{0.719419in}}%
\pgfpathcurveto{\pgfqpoint{4.558825in}{0.711606in}}{\pgfqpoint{4.569424in}{0.707216in}}{\pgfqpoint{4.580474in}{0.707216in}}%
\pgfpathlineto{\pgfqpoint{4.580474in}{0.707216in}}%
\pgfpathclose%
\pgfusepath{stroke}%
\end{pgfscope}%
\begin{pgfscope}%
\pgfpathrectangle{\pgfqpoint{0.847223in}{0.554012in}}{\pgfqpoint{6.200000in}{4.620000in}}%
\pgfusepath{clip}%
\pgfsetbuttcap%
\pgfsetroundjoin%
\pgfsetlinewidth{1.003750pt}%
\definecolor{currentstroke}{rgb}{1.000000,0.000000,0.000000}%
\pgfsetstrokecolor{currentstroke}%
\pgfsetdash{}{0pt}%
\pgfpathmoveto{\pgfqpoint{4.585807in}{0.706439in}}%
\pgfpathcurveto{\pgfqpoint{4.596857in}{0.706439in}}{\pgfqpoint{4.607456in}{0.710830in}}{\pgfqpoint{4.615270in}{0.718643in}}%
\pgfpathcurveto{\pgfqpoint{4.623084in}{0.726457in}}{\pgfqpoint{4.627474in}{0.737056in}}{\pgfqpoint{4.627474in}{0.748106in}}%
\pgfpathcurveto{\pgfqpoint{4.627474in}{0.759156in}}{\pgfqpoint{4.623084in}{0.769755in}}{\pgfqpoint{4.615270in}{0.777569in}}%
\pgfpathcurveto{\pgfqpoint{4.607456in}{0.785382in}}{\pgfqpoint{4.596857in}{0.789773in}}{\pgfqpoint{4.585807in}{0.789773in}}%
\pgfpathcurveto{\pgfqpoint{4.574757in}{0.789773in}}{\pgfqpoint{4.564158in}{0.785382in}}{\pgfqpoint{4.556344in}{0.777569in}}%
\pgfpathcurveto{\pgfqpoint{4.548531in}{0.769755in}}{\pgfqpoint{4.544141in}{0.759156in}}{\pgfqpoint{4.544141in}{0.748106in}}%
\pgfpathcurveto{\pgfqpoint{4.544141in}{0.737056in}}{\pgfqpoint{4.548531in}{0.726457in}}{\pgfqpoint{4.556344in}{0.718643in}}%
\pgfpathcurveto{\pgfqpoint{4.564158in}{0.710830in}}{\pgfqpoint{4.574757in}{0.706439in}}{\pgfqpoint{4.585807in}{0.706439in}}%
\pgfpathlineto{\pgfqpoint{4.585807in}{0.706439in}}%
\pgfpathclose%
\pgfusepath{stroke}%
\end{pgfscope}%
\begin{pgfscope}%
\pgfpathrectangle{\pgfqpoint{0.847223in}{0.554012in}}{\pgfqpoint{6.200000in}{4.620000in}}%
\pgfusepath{clip}%
\pgfsetbuttcap%
\pgfsetroundjoin%
\pgfsetlinewidth{1.003750pt}%
\definecolor{currentstroke}{rgb}{1.000000,0.000000,0.000000}%
\pgfsetstrokecolor{currentstroke}%
\pgfsetdash{}{0pt}%
\pgfpathmoveto{\pgfqpoint{4.591140in}{0.705665in}}%
\pgfpathcurveto{\pgfqpoint{4.602191in}{0.705665in}}{\pgfqpoint{4.612790in}{0.710055in}}{\pgfqpoint{4.620603in}{0.717869in}}%
\pgfpathcurveto{\pgfqpoint{4.628417in}{0.725682in}}{\pgfqpoint{4.632807in}{0.736282in}}{\pgfqpoint{4.632807in}{0.747332in}}%
\pgfpathcurveto{\pgfqpoint{4.632807in}{0.758382in}}{\pgfqpoint{4.628417in}{0.768981in}}{\pgfqpoint{4.620603in}{0.776794in}}%
\pgfpathcurveto{\pgfqpoint{4.612790in}{0.784608in}}{\pgfqpoint{4.602191in}{0.788998in}}{\pgfqpoint{4.591140in}{0.788998in}}%
\pgfpathcurveto{\pgfqpoint{4.580090in}{0.788998in}}{\pgfqpoint{4.569491in}{0.784608in}}{\pgfqpoint{4.561678in}{0.776794in}}%
\pgfpathcurveto{\pgfqpoint{4.553864in}{0.768981in}}{\pgfqpoint{4.549474in}{0.758382in}}{\pgfqpoint{4.549474in}{0.747332in}}%
\pgfpathcurveto{\pgfqpoint{4.549474in}{0.736282in}}{\pgfqpoint{4.553864in}{0.725682in}}{\pgfqpoint{4.561678in}{0.717869in}}%
\pgfpathcurveto{\pgfqpoint{4.569491in}{0.710055in}}{\pgfqpoint{4.580090in}{0.705665in}}{\pgfqpoint{4.591140in}{0.705665in}}%
\pgfpathlineto{\pgfqpoint{4.591140in}{0.705665in}}%
\pgfpathclose%
\pgfusepath{stroke}%
\end{pgfscope}%
\begin{pgfscope}%
\pgfpathrectangle{\pgfqpoint{0.847223in}{0.554012in}}{\pgfqpoint{6.200000in}{4.620000in}}%
\pgfusepath{clip}%
\pgfsetbuttcap%
\pgfsetroundjoin%
\pgfsetlinewidth{1.003750pt}%
\definecolor{currentstroke}{rgb}{1.000000,0.000000,0.000000}%
\pgfsetstrokecolor{currentstroke}%
\pgfsetdash{}{0pt}%
\pgfpathmoveto{\pgfqpoint{4.596474in}{0.704893in}}%
\pgfpathcurveto{\pgfqpoint{4.607524in}{0.704893in}}{\pgfqpoint{4.618123in}{0.709283in}}{\pgfqpoint{4.625936in}{0.717096in}}%
\pgfpathcurveto{\pgfqpoint{4.633750in}{0.724910in}}{\pgfqpoint{4.638140in}{0.735509in}}{\pgfqpoint{4.638140in}{0.746559in}}%
\pgfpathcurveto{\pgfqpoint{4.638140in}{0.757609in}}{\pgfqpoint{4.633750in}{0.768208in}}{\pgfqpoint{4.625936in}{0.776022in}}%
\pgfpathcurveto{\pgfqpoint{4.618123in}{0.783836in}}{\pgfqpoint{4.607524in}{0.788226in}}{\pgfqpoint{4.596474in}{0.788226in}}%
\pgfpathcurveto{\pgfqpoint{4.585424in}{0.788226in}}{\pgfqpoint{4.574825in}{0.783836in}}{\pgfqpoint{4.567011in}{0.776022in}}%
\pgfpathcurveto{\pgfqpoint{4.559197in}{0.768208in}}{\pgfqpoint{4.554807in}{0.757609in}}{\pgfqpoint{4.554807in}{0.746559in}}%
\pgfpathcurveto{\pgfqpoint{4.554807in}{0.735509in}}{\pgfqpoint{4.559197in}{0.724910in}}{\pgfqpoint{4.567011in}{0.717096in}}%
\pgfpathcurveto{\pgfqpoint{4.574825in}{0.709283in}}{\pgfqpoint{4.585424in}{0.704893in}}{\pgfqpoint{4.596474in}{0.704893in}}%
\pgfpathlineto{\pgfqpoint{4.596474in}{0.704893in}}%
\pgfpathclose%
\pgfusepath{stroke}%
\end{pgfscope}%
\begin{pgfscope}%
\pgfpathrectangle{\pgfqpoint{0.847223in}{0.554012in}}{\pgfqpoint{6.200000in}{4.620000in}}%
\pgfusepath{clip}%
\pgfsetbuttcap%
\pgfsetroundjoin%
\pgfsetlinewidth{1.003750pt}%
\definecolor{currentstroke}{rgb}{1.000000,0.000000,0.000000}%
\pgfsetstrokecolor{currentstroke}%
\pgfsetdash{}{0pt}%
\pgfpathmoveto{\pgfqpoint{4.601807in}{0.704122in}}%
\pgfpathcurveto{\pgfqpoint{4.612857in}{0.704122in}}{\pgfqpoint{4.623456in}{0.708512in}}{\pgfqpoint{4.631270in}{0.716326in}}%
\pgfpathcurveto{\pgfqpoint{4.639083in}{0.724140in}}{\pgfqpoint{4.643474in}{0.734739in}}{\pgfqpoint{4.643474in}{0.745789in}}%
\pgfpathcurveto{\pgfqpoint{4.643474in}{0.756839in}}{\pgfqpoint{4.639083in}{0.767438in}}{\pgfqpoint{4.631270in}{0.775252in}}%
\pgfpathcurveto{\pgfqpoint{4.623456in}{0.783065in}}{\pgfqpoint{4.612857in}{0.787455in}}{\pgfqpoint{4.601807in}{0.787455in}}%
\pgfpathcurveto{\pgfqpoint{4.590757in}{0.787455in}}{\pgfqpoint{4.580158in}{0.783065in}}{\pgfqpoint{4.572344in}{0.775252in}}%
\pgfpathcurveto{\pgfqpoint{4.564530in}{0.767438in}}{\pgfqpoint{4.560140in}{0.756839in}}{\pgfqpoint{4.560140in}{0.745789in}}%
\pgfpathcurveto{\pgfqpoint{4.560140in}{0.734739in}}{\pgfqpoint{4.564530in}{0.724140in}}{\pgfqpoint{4.572344in}{0.716326in}}%
\pgfpathcurveto{\pgfqpoint{4.580158in}{0.708512in}}{\pgfqpoint{4.590757in}{0.704122in}}{\pgfqpoint{4.601807in}{0.704122in}}%
\pgfpathlineto{\pgfqpoint{4.601807in}{0.704122in}}%
\pgfpathclose%
\pgfusepath{stroke}%
\end{pgfscope}%
\begin{pgfscope}%
\pgfpathrectangle{\pgfqpoint{0.847223in}{0.554012in}}{\pgfqpoint{6.200000in}{4.620000in}}%
\pgfusepath{clip}%
\pgfsetbuttcap%
\pgfsetroundjoin%
\pgfsetlinewidth{1.003750pt}%
\definecolor{currentstroke}{rgb}{1.000000,0.000000,0.000000}%
\pgfsetstrokecolor{currentstroke}%
\pgfsetdash{}{0pt}%
\pgfpathmoveto{\pgfqpoint{4.607140in}{0.703354in}}%
\pgfpathcurveto{\pgfqpoint{4.618190in}{0.703354in}}{\pgfqpoint{4.628789in}{0.707744in}}{\pgfqpoint{4.636603in}{0.715557in}}%
\pgfpathcurveto{\pgfqpoint{4.644416in}{0.723371in}}{\pgfqpoint{4.648807in}{0.733970in}}{\pgfqpoint{4.648807in}{0.745020in}}%
\pgfpathcurveto{\pgfqpoint{4.648807in}{0.756070in}}{\pgfqpoint{4.644416in}{0.766669in}}{\pgfqpoint{4.636603in}{0.774483in}}%
\pgfpathcurveto{\pgfqpoint{4.628789in}{0.782297in}}{\pgfqpoint{4.618190in}{0.786687in}}{\pgfqpoint{4.607140in}{0.786687in}}%
\pgfpathcurveto{\pgfqpoint{4.596090in}{0.786687in}}{\pgfqpoint{4.585491in}{0.782297in}}{\pgfqpoint{4.577677in}{0.774483in}}%
\pgfpathcurveto{\pgfqpoint{4.569864in}{0.766669in}}{\pgfqpoint{4.565473in}{0.756070in}}{\pgfqpoint{4.565473in}{0.745020in}}%
\pgfpathcurveto{\pgfqpoint{4.565473in}{0.733970in}}{\pgfqpoint{4.569864in}{0.723371in}}{\pgfqpoint{4.577677in}{0.715557in}}%
\pgfpathcurveto{\pgfqpoint{4.585491in}{0.707744in}}{\pgfqpoint{4.596090in}{0.703354in}}{\pgfqpoint{4.607140in}{0.703354in}}%
\pgfpathlineto{\pgfqpoint{4.607140in}{0.703354in}}%
\pgfpathclose%
\pgfusepath{stroke}%
\end{pgfscope}%
\begin{pgfscope}%
\pgfpathrectangle{\pgfqpoint{0.847223in}{0.554012in}}{\pgfqpoint{6.200000in}{4.620000in}}%
\pgfusepath{clip}%
\pgfsetbuttcap%
\pgfsetroundjoin%
\pgfsetlinewidth{1.003750pt}%
\definecolor{currentstroke}{rgb}{1.000000,0.000000,0.000000}%
\pgfsetstrokecolor{currentstroke}%
\pgfsetdash{}{0pt}%
\pgfpathmoveto{\pgfqpoint{4.612473in}{0.702587in}}%
\pgfpathcurveto{\pgfqpoint{4.623523in}{0.702587in}}{\pgfqpoint{4.634122in}{0.706977in}}{\pgfqpoint{4.641936in}{0.714791in}}%
\pgfpathcurveto{\pgfqpoint{4.649750in}{0.722604in}}{\pgfqpoint{4.654140in}{0.733203in}}{\pgfqpoint{4.654140in}{0.744254in}}%
\pgfpathcurveto{\pgfqpoint{4.654140in}{0.755304in}}{\pgfqpoint{4.649750in}{0.765903in}}{\pgfqpoint{4.641936in}{0.773716in}}%
\pgfpathcurveto{\pgfqpoint{4.634122in}{0.781530in}}{\pgfqpoint{4.623523in}{0.785920in}}{\pgfqpoint{4.612473in}{0.785920in}}%
\pgfpathcurveto{\pgfqpoint{4.601423in}{0.785920in}}{\pgfqpoint{4.590824in}{0.781530in}}{\pgfqpoint{4.583011in}{0.773716in}}%
\pgfpathcurveto{\pgfqpoint{4.575197in}{0.765903in}}{\pgfqpoint{4.570807in}{0.755304in}}{\pgfqpoint{4.570807in}{0.744254in}}%
\pgfpathcurveto{\pgfqpoint{4.570807in}{0.733203in}}{\pgfqpoint{4.575197in}{0.722604in}}{\pgfqpoint{4.583011in}{0.714791in}}%
\pgfpathcurveto{\pgfqpoint{4.590824in}{0.706977in}}{\pgfqpoint{4.601423in}{0.702587in}}{\pgfqpoint{4.612473in}{0.702587in}}%
\pgfpathlineto{\pgfqpoint{4.612473in}{0.702587in}}%
\pgfpathclose%
\pgfusepath{stroke}%
\end{pgfscope}%
\begin{pgfscope}%
\pgfpathrectangle{\pgfqpoint{0.847223in}{0.554012in}}{\pgfqpoint{6.200000in}{4.620000in}}%
\pgfusepath{clip}%
\pgfsetbuttcap%
\pgfsetroundjoin%
\pgfsetlinewidth{1.003750pt}%
\definecolor{currentstroke}{rgb}{1.000000,0.000000,0.000000}%
\pgfsetstrokecolor{currentstroke}%
\pgfsetdash{}{0pt}%
\pgfpathmoveto{\pgfqpoint{4.617807in}{0.701822in}}%
\pgfpathcurveto{\pgfqpoint{4.628857in}{0.701822in}}{\pgfqpoint{4.639456in}{0.706212in}}{\pgfqpoint{4.647269in}{0.714026in}}%
\pgfpathcurveto{\pgfqpoint{4.655083in}{0.721840in}}{\pgfqpoint{4.659473in}{0.732439in}}{\pgfqpoint{4.659473in}{0.743489in}}%
\pgfpathcurveto{\pgfqpoint{4.659473in}{0.754539in}}{\pgfqpoint{4.655083in}{0.765138in}}{\pgfqpoint{4.647269in}{0.772952in}}%
\pgfpathcurveto{\pgfqpoint{4.639456in}{0.780765in}}{\pgfqpoint{4.628857in}{0.785156in}}{\pgfqpoint{4.617807in}{0.785156in}}%
\pgfpathcurveto{\pgfqpoint{4.606756in}{0.785156in}}{\pgfqpoint{4.596157in}{0.780765in}}{\pgfqpoint{4.588344in}{0.772952in}}%
\pgfpathcurveto{\pgfqpoint{4.580530in}{0.765138in}}{\pgfqpoint{4.576140in}{0.754539in}}{\pgfqpoint{4.576140in}{0.743489in}}%
\pgfpathcurveto{\pgfqpoint{4.576140in}{0.732439in}}{\pgfqpoint{4.580530in}{0.721840in}}{\pgfqpoint{4.588344in}{0.714026in}}%
\pgfpathcurveto{\pgfqpoint{4.596157in}{0.706212in}}{\pgfqpoint{4.606756in}{0.701822in}}{\pgfqpoint{4.617807in}{0.701822in}}%
\pgfpathlineto{\pgfqpoint{4.617807in}{0.701822in}}%
\pgfpathclose%
\pgfusepath{stroke}%
\end{pgfscope}%
\begin{pgfscope}%
\pgfpathrectangle{\pgfqpoint{0.847223in}{0.554012in}}{\pgfqpoint{6.200000in}{4.620000in}}%
\pgfusepath{clip}%
\pgfsetbuttcap%
\pgfsetroundjoin%
\pgfsetlinewidth{1.003750pt}%
\definecolor{currentstroke}{rgb}{1.000000,0.000000,0.000000}%
\pgfsetstrokecolor{currentstroke}%
\pgfsetdash{}{0pt}%
\pgfpathmoveto{\pgfqpoint{4.623140in}{0.701059in}}%
\pgfpathcurveto{\pgfqpoint{4.634190in}{0.701059in}}{\pgfqpoint{4.644789in}{0.705450in}}{\pgfqpoint{4.652603in}{0.713263in}}%
\pgfpathcurveto{\pgfqpoint{4.660416in}{0.721077in}}{\pgfqpoint{4.664806in}{0.731676in}}{\pgfqpoint{4.664806in}{0.742726in}}%
\pgfpathcurveto{\pgfqpoint{4.664806in}{0.753776in}}{\pgfqpoint{4.660416in}{0.764375in}}{\pgfqpoint{4.652603in}{0.772189in}}%
\pgfpathcurveto{\pgfqpoint{4.644789in}{0.780002in}}{\pgfqpoint{4.634190in}{0.784393in}}{\pgfqpoint{4.623140in}{0.784393in}}%
\pgfpathcurveto{\pgfqpoint{4.612090in}{0.784393in}}{\pgfqpoint{4.601491in}{0.780002in}}{\pgfqpoint{4.593677in}{0.772189in}}%
\pgfpathcurveto{\pgfqpoint{4.585863in}{0.764375in}}{\pgfqpoint{4.581473in}{0.753776in}}{\pgfqpoint{4.581473in}{0.742726in}}%
\pgfpathcurveto{\pgfqpoint{4.581473in}{0.731676in}}{\pgfqpoint{4.585863in}{0.721077in}}{\pgfqpoint{4.593677in}{0.713263in}}%
\pgfpathcurveto{\pgfqpoint{4.601491in}{0.705450in}}{\pgfqpoint{4.612090in}{0.701059in}}{\pgfqpoint{4.623140in}{0.701059in}}%
\pgfpathlineto{\pgfqpoint{4.623140in}{0.701059in}}%
\pgfpathclose%
\pgfusepath{stroke}%
\end{pgfscope}%
\begin{pgfscope}%
\pgfpathrectangle{\pgfqpoint{0.847223in}{0.554012in}}{\pgfqpoint{6.200000in}{4.620000in}}%
\pgfusepath{clip}%
\pgfsetbuttcap%
\pgfsetroundjoin%
\pgfsetlinewidth{1.003750pt}%
\definecolor{currentstroke}{rgb}{1.000000,0.000000,0.000000}%
\pgfsetstrokecolor{currentstroke}%
\pgfsetdash{}{0pt}%
\pgfpathmoveto{\pgfqpoint{4.628473in}{0.700298in}}%
\pgfpathcurveto{\pgfqpoint{4.639523in}{0.700298in}}{\pgfqpoint{4.650122in}{0.704689in}}{\pgfqpoint{4.657936in}{0.712502in}}%
\pgfpathcurveto{\pgfqpoint{4.665749in}{0.720316in}}{\pgfqpoint{4.670140in}{0.730915in}}{\pgfqpoint{4.670140in}{0.741965in}}%
\pgfpathcurveto{\pgfqpoint{4.670140in}{0.753015in}}{\pgfqpoint{4.665749in}{0.763614in}}{\pgfqpoint{4.657936in}{0.771428in}}%
\pgfpathcurveto{\pgfqpoint{4.650122in}{0.779241in}}{\pgfqpoint{4.639523in}{0.783632in}}{\pgfqpoint{4.628473in}{0.783632in}}%
\pgfpathcurveto{\pgfqpoint{4.617423in}{0.783632in}}{\pgfqpoint{4.606824in}{0.779241in}}{\pgfqpoint{4.599010in}{0.771428in}}%
\pgfpathcurveto{\pgfqpoint{4.591197in}{0.763614in}}{\pgfqpoint{4.586806in}{0.753015in}}{\pgfqpoint{4.586806in}{0.741965in}}%
\pgfpathcurveto{\pgfqpoint{4.586806in}{0.730915in}}{\pgfqpoint{4.591197in}{0.720316in}}{\pgfqpoint{4.599010in}{0.712502in}}%
\pgfpathcurveto{\pgfqpoint{4.606824in}{0.704689in}}{\pgfqpoint{4.617423in}{0.700298in}}{\pgfqpoint{4.628473in}{0.700298in}}%
\pgfpathlineto{\pgfqpoint{4.628473in}{0.700298in}}%
\pgfpathclose%
\pgfusepath{stroke}%
\end{pgfscope}%
\begin{pgfscope}%
\pgfpathrectangle{\pgfqpoint{0.847223in}{0.554012in}}{\pgfqpoint{6.200000in}{4.620000in}}%
\pgfusepath{clip}%
\pgfsetbuttcap%
\pgfsetroundjoin%
\pgfsetlinewidth{1.003750pt}%
\definecolor{currentstroke}{rgb}{1.000000,0.000000,0.000000}%
\pgfsetstrokecolor{currentstroke}%
\pgfsetdash{}{0pt}%
\pgfpathmoveto{\pgfqpoint{4.633806in}{0.699539in}}%
\pgfpathcurveto{\pgfqpoint{4.644856in}{0.699539in}}{\pgfqpoint{4.655455in}{0.703930in}}{\pgfqpoint{4.663269in}{0.711743in}}%
\pgfpathcurveto{\pgfqpoint{4.671083in}{0.719557in}}{\pgfqpoint{4.675473in}{0.730156in}}{\pgfqpoint{4.675473in}{0.741206in}}%
\pgfpathcurveto{\pgfqpoint{4.675473in}{0.752256in}}{\pgfqpoint{4.671083in}{0.762855in}}{\pgfqpoint{4.663269in}{0.770669in}}%
\pgfpathcurveto{\pgfqpoint{4.655455in}{0.778482in}}{\pgfqpoint{4.644856in}{0.782873in}}{\pgfqpoint{4.633806in}{0.782873in}}%
\pgfpathcurveto{\pgfqpoint{4.622756in}{0.782873in}}{\pgfqpoint{4.612157in}{0.778482in}}{\pgfqpoint{4.604343in}{0.770669in}}%
\pgfpathcurveto{\pgfqpoint{4.596530in}{0.762855in}}{\pgfqpoint{4.592140in}{0.752256in}}{\pgfqpoint{4.592140in}{0.741206in}}%
\pgfpathcurveto{\pgfqpoint{4.592140in}{0.730156in}}{\pgfqpoint{4.596530in}{0.719557in}}{\pgfqpoint{4.604343in}{0.711743in}}%
\pgfpathcurveto{\pgfqpoint{4.612157in}{0.703930in}}{\pgfqpoint{4.622756in}{0.699539in}}{\pgfqpoint{4.633806in}{0.699539in}}%
\pgfpathlineto{\pgfqpoint{4.633806in}{0.699539in}}%
\pgfpathclose%
\pgfusepath{stroke}%
\end{pgfscope}%
\begin{pgfscope}%
\pgfpathrectangle{\pgfqpoint{0.847223in}{0.554012in}}{\pgfqpoint{6.200000in}{4.620000in}}%
\pgfusepath{clip}%
\pgfsetbuttcap%
\pgfsetroundjoin%
\pgfsetlinewidth{1.003750pt}%
\definecolor{currentstroke}{rgb}{1.000000,0.000000,0.000000}%
\pgfsetstrokecolor{currentstroke}%
\pgfsetdash{}{0pt}%
\pgfpathmoveto{\pgfqpoint{4.639139in}{0.698782in}}%
\pgfpathcurveto{\pgfqpoint{4.650190in}{0.698782in}}{\pgfqpoint{4.660789in}{0.703172in}}{\pgfqpoint{4.668602in}{0.710986in}}%
\pgfpathcurveto{\pgfqpoint{4.676416in}{0.718800in}}{\pgfqpoint{4.680806in}{0.729399in}}{\pgfqpoint{4.680806in}{0.740449in}}%
\pgfpathcurveto{\pgfqpoint{4.680806in}{0.751499in}}{\pgfqpoint{4.676416in}{0.762098in}}{\pgfqpoint{4.668602in}{0.769912in}}%
\pgfpathcurveto{\pgfqpoint{4.660789in}{0.777725in}}{\pgfqpoint{4.650190in}{0.782115in}}{\pgfqpoint{4.639139in}{0.782115in}}%
\pgfpathcurveto{\pgfqpoint{4.628089in}{0.782115in}}{\pgfqpoint{4.617490in}{0.777725in}}{\pgfqpoint{4.609677in}{0.769912in}}%
\pgfpathcurveto{\pgfqpoint{4.601863in}{0.762098in}}{\pgfqpoint{4.597473in}{0.751499in}}{\pgfqpoint{4.597473in}{0.740449in}}%
\pgfpathcurveto{\pgfqpoint{4.597473in}{0.729399in}}{\pgfqpoint{4.601863in}{0.718800in}}{\pgfqpoint{4.609677in}{0.710986in}}%
\pgfpathcurveto{\pgfqpoint{4.617490in}{0.703172in}}{\pgfqpoint{4.628089in}{0.698782in}}{\pgfqpoint{4.639139in}{0.698782in}}%
\pgfpathlineto{\pgfqpoint{4.639139in}{0.698782in}}%
\pgfpathclose%
\pgfusepath{stroke}%
\end{pgfscope}%
\begin{pgfscope}%
\pgfpathrectangle{\pgfqpoint{0.847223in}{0.554012in}}{\pgfqpoint{6.200000in}{4.620000in}}%
\pgfusepath{clip}%
\pgfsetbuttcap%
\pgfsetroundjoin%
\pgfsetlinewidth{1.003750pt}%
\definecolor{currentstroke}{rgb}{1.000000,0.000000,0.000000}%
\pgfsetstrokecolor{currentstroke}%
\pgfsetdash{}{0pt}%
\pgfpathmoveto{\pgfqpoint{4.644473in}{0.698027in}}%
\pgfpathcurveto{\pgfqpoint{4.655523in}{0.698027in}}{\pgfqpoint{4.666122in}{0.702417in}}{\pgfqpoint{4.673935in}{0.710231in}}%
\pgfpathcurveto{\pgfqpoint{4.681749in}{0.718044in}}{\pgfqpoint{4.686139in}{0.728643in}}{\pgfqpoint{4.686139in}{0.739693in}}%
\pgfpathcurveto{\pgfqpoint{4.686139in}{0.750744in}}{\pgfqpoint{4.681749in}{0.761343in}}{\pgfqpoint{4.673935in}{0.769156in}}%
\pgfpathcurveto{\pgfqpoint{4.666122in}{0.776970in}}{\pgfqpoint{4.655523in}{0.781360in}}{\pgfqpoint{4.644473in}{0.781360in}}%
\pgfpathcurveto{\pgfqpoint{4.633422in}{0.781360in}}{\pgfqpoint{4.622823in}{0.776970in}}{\pgfqpoint{4.615010in}{0.769156in}}%
\pgfpathcurveto{\pgfqpoint{4.607196in}{0.761343in}}{\pgfqpoint{4.602806in}{0.750744in}}{\pgfqpoint{4.602806in}{0.739693in}}%
\pgfpathcurveto{\pgfqpoint{4.602806in}{0.728643in}}{\pgfqpoint{4.607196in}{0.718044in}}{\pgfqpoint{4.615010in}{0.710231in}}%
\pgfpathcurveto{\pgfqpoint{4.622823in}{0.702417in}}{\pgfqpoint{4.633422in}{0.698027in}}{\pgfqpoint{4.644473in}{0.698027in}}%
\pgfpathlineto{\pgfqpoint{4.644473in}{0.698027in}}%
\pgfpathclose%
\pgfusepath{stroke}%
\end{pgfscope}%
\begin{pgfscope}%
\pgfpathrectangle{\pgfqpoint{0.847223in}{0.554012in}}{\pgfqpoint{6.200000in}{4.620000in}}%
\pgfusepath{clip}%
\pgfsetbuttcap%
\pgfsetroundjoin%
\pgfsetlinewidth{1.003750pt}%
\definecolor{currentstroke}{rgb}{1.000000,0.000000,0.000000}%
\pgfsetstrokecolor{currentstroke}%
\pgfsetdash{}{0pt}%
\pgfpathmoveto{\pgfqpoint{4.649806in}{0.697273in}}%
\pgfpathcurveto{\pgfqpoint{4.660856in}{0.697273in}}{\pgfqpoint{4.671455in}{0.701664in}}{\pgfqpoint{4.679269in}{0.709477in}}%
\pgfpathcurveto{\pgfqpoint{4.687082in}{0.717291in}}{\pgfqpoint{4.691472in}{0.727890in}}{\pgfqpoint{4.691472in}{0.738940in}}%
\pgfpathcurveto{\pgfqpoint{4.691472in}{0.749990in}}{\pgfqpoint{4.687082in}{0.760589in}}{\pgfqpoint{4.679269in}{0.768403in}}%
\pgfpathcurveto{\pgfqpoint{4.671455in}{0.776216in}}{\pgfqpoint{4.660856in}{0.780607in}}{\pgfqpoint{4.649806in}{0.780607in}}%
\pgfpathcurveto{\pgfqpoint{4.638756in}{0.780607in}}{\pgfqpoint{4.628157in}{0.776216in}}{\pgfqpoint{4.620343in}{0.768403in}}%
\pgfpathcurveto{\pgfqpoint{4.612529in}{0.760589in}}{\pgfqpoint{4.608139in}{0.749990in}}{\pgfqpoint{4.608139in}{0.738940in}}%
\pgfpathcurveto{\pgfqpoint{4.608139in}{0.727890in}}{\pgfqpoint{4.612529in}{0.717291in}}{\pgfqpoint{4.620343in}{0.709477in}}%
\pgfpathcurveto{\pgfqpoint{4.628157in}{0.701664in}}{\pgfqpoint{4.638756in}{0.697273in}}{\pgfqpoint{4.649806in}{0.697273in}}%
\pgfpathlineto{\pgfqpoint{4.649806in}{0.697273in}}%
\pgfpathclose%
\pgfusepath{stroke}%
\end{pgfscope}%
\begin{pgfscope}%
\pgfpathrectangle{\pgfqpoint{0.847223in}{0.554012in}}{\pgfqpoint{6.200000in}{4.620000in}}%
\pgfusepath{clip}%
\pgfsetbuttcap%
\pgfsetroundjoin%
\pgfsetlinewidth{1.003750pt}%
\definecolor{currentstroke}{rgb}{1.000000,0.000000,0.000000}%
\pgfsetstrokecolor{currentstroke}%
\pgfsetdash{}{0pt}%
\pgfpathmoveto{\pgfqpoint{4.655139in}{0.696522in}}%
\pgfpathcurveto{\pgfqpoint{4.666189in}{0.696522in}}{\pgfqpoint{4.676788in}{0.700912in}}{\pgfqpoint{4.684602in}{0.708726in}}%
\pgfpathcurveto{\pgfqpoint{4.692415in}{0.716539in}}{\pgfqpoint{4.696806in}{0.727138in}}{\pgfqpoint{4.696806in}{0.738188in}}%
\pgfpathcurveto{\pgfqpoint{4.696806in}{0.749239in}}{\pgfqpoint{4.692415in}{0.759838in}}{\pgfqpoint{4.684602in}{0.767651in}}%
\pgfpathcurveto{\pgfqpoint{4.676788in}{0.775465in}}{\pgfqpoint{4.666189in}{0.779855in}}{\pgfqpoint{4.655139in}{0.779855in}}%
\pgfpathcurveto{\pgfqpoint{4.644089in}{0.779855in}}{\pgfqpoint{4.633490in}{0.775465in}}{\pgfqpoint{4.625676in}{0.767651in}}%
\pgfpathcurveto{\pgfqpoint{4.617863in}{0.759838in}}{\pgfqpoint{4.613472in}{0.749239in}}{\pgfqpoint{4.613472in}{0.738188in}}%
\pgfpathcurveto{\pgfqpoint{4.613472in}{0.727138in}}{\pgfqpoint{4.617863in}{0.716539in}}{\pgfqpoint{4.625676in}{0.708726in}}%
\pgfpathcurveto{\pgfqpoint{4.633490in}{0.700912in}}{\pgfqpoint{4.644089in}{0.696522in}}{\pgfqpoint{4.655139in}{0.696522in}}%
\pgfpathlineto{\pgfqpoint{4.655139in}{0.696522in}}%
\pgfpathclose%
\pgfusepath{stroke}%
\end{pgfscope}%
\begin{pgfscope}%
\pgfpathrectangle{\pgfqpoint{0.847223in}{0.554012in}}{\pgfqpoint{6.200000in}{4.620000in}}%
\pgfusepath{clip}%
\pgfsetbuttcap%
\pgfsetroundjoin%
\pgfsetlinewidth{1.003750pt}%
\definecolor{currentstroke}{rgb}{1.000000,0.000000,0.000000}%
\pgfsetstrokecolor{currentstroke}%
\pgfsetdash{}{0pt}%
\pgfpathmoveto{\pgfqpoint{4.660472in}{0.695772in}}%
\pgfpathcurveto{\pgfqpoint{4.671522in}{0.695772in}}{\pgfqpoint{4.682121in}{0.700162in}}{\pgfqpoint{4.689935in}{0.707976in}}%
\pgfpathcurveto{\pgfqpoint{4.697749in}{0.715790in}}{\pgfqpoint{4.702139in}{0.726389in}}{\pgfqpoint{4.702139in}{0.737439in}}%
\pgfpathcurveto{\pgfqpoint{4.702139in}{0.748489in}}{\pgfqpoint{4.697749in}{0.759088in}}{\pgfqpoint{4.689935in}{0.766901in}}%
\pgfpathcurveto{\pgfqpoint{4.682121in}{0.774715in}}{\pgfqpoint{4.671522in}{0.779105in}}{\pgfqpoint{4.660472in}{0.779105in}}%
\pgfpathcurveto{\pgfqpoint{4.649422in}{0.779105in}}{\pgfqpoint{4.638823in}{0.774715in}}{\pgfqpoint{4.631009in}{0.766901in}}%
\pgfpathcurveto{\pgfqpoint{4.623196in}{0.759088in}}{\pgfqpoint{4.618806in}{0.748489in}}{\pgfqpoint{4.618806in}{0.737439in}}%
\pgfpathcurveto{\pgfqpoint{4.618806in}{0.726389in}}{\pgfqpoint{4.623196in}{0.715790in}}{\pgfqpoint{4.631009in}{0.707976in}}%
\pgfpathcurveto{\pgfqpoint{4.638823in}{0.700162in}}{\pgfqpoint{4.649422in}{0.695772in}}{\pgfqpoint{4.660472in}{0.695772in}}%
\pgfpathlineto{\pgfqpoint{4.660472in}{0.695772in}}%
\pgfpathclose%
\pgfusepath{stroke}%
\end{pgfscope}%
\begin{pgfscope}%
\pgfpathrectangle{\pgfqpoint{0.847223in}{0.554012in}}{\pgfqpoint{6.200000in}{4.620000in}}%
\pgfusepath{clip}%
\pgfsetbuttcap%
\pgfsetroundjoin%
\pgfsetlinewidth{1.003750pt}%
\definecolor{currentstroke}{rgb}{1.000000,0.000000,0.000000}%
\pgfsetstrokecolor{currentstroke}%
\pgfsetdash{}{0pt}%
\pgfpathmoveto{\pgfqpoint{4.665805in}{0.695024in}}%
\pgfpathcurveto{\pgfqpoint{4.676856in}{0.695024in}}{\pgfqpoint{4.687455in}{0.699414in}}{\pgfqpoint{4.695268in}{0.707228in}}%
\pgfpathcurveto{\pgfqpoint{4.703082in}{0.715042in}}{\pgfqpoint{4.707472in}{0.725641in}}{\pgfqpoint{4.707472in}{0.736691in}}%
\pgfpathcurveto{\pgfqpoint{4.707472in}{0.747741in}}{\pgfqpoint{4.703082in}{0.758340in}}{\pgfqpoint{4.695268in}{0.766154in}}%
\pgfpathcurveto{\pgfqpoint{4.687455in}{0.773967in}}{\pgfqpoint{4.676856in}{0.778357in}}{\pgfqpoint{4.665805in}{0.778357in}}%
\pgfpathcurveto{\pgfqpoint{4.654755in}{0.778357in}}{\pgfqpoint{4.644156in}{0.773967in}}{\pgfqpoint{4.636343in}{0.766154in}}%
\pgfpathcurveto{\pgfqpoint{4.628529in}{0.758340in}}{\pgfqpoint{4.624139in}{0.747741in}}{\pgfqpoint{4.624139in}{0.736691in}}%
\pgfpathcurveto{\pgfqpoint{4.624139in}{0.725641in}}{\pgfqpoint{4.628529in}{0.715042in}}{\pgfqpoint{4.636343in}{0.707228in}}%
\pgfpathcurveto{\pgfqpoint{4.644156in}{0.699414in}}{\pgfqpoint{4.654755in}{0.695024in}}{\pgfqpoint{4.665805in}{0.695024in}}%
\pgfpathlineto{\pgfqpoint{4.665805in}{0.695024in}}%
\pgfpathclose%
\pgfusepath{stroke}%
\end{pgfscope}%
\begin{pgfscope}%
\pgfpathrectangle{\pgfqpoint{0.847223in}{0.554012in}}{\pgfqpoint{6.200000in}{4.620000in}}%
\pgfusepath{clip}%
\pgfsetbuttcap%
\pgfsetroundjoin%
\pgfsetlinewidth{1.003750pt}%
\definecolor{currentstroke}{rgb}{1.000000,0.000000,0.000000}%
\pgfsetstrokecolor{currentstroke}%
\pgfsetdash{}{0pt}%
\pgfpathmoveto{\pgfqpoint{4.671139in}{0.694278in}}%
\pgfpathcurveto{\pgfqpoint{4.682189in}{0.694278in}}{\pgfqpoint{4.692788in}{0.698668in}}{\pgfqpoint{4.700601in}{0.706482in}}%
\pgfpathcurveto{\pgfqpoint{4.708415in}{0.714296in}}{\pgfqpoint{4.712805in}{0.724895in}}{\pgfqpoint{4.712805in}{0.735945in}}%
\pgfpathcurveto{\pgfqpoint{4.712805in}{0.746995in}}{\pgfqpoint{4.708415in}{0.757594in}}{\pgfqpoint{4.700601in}{0.765407in}}%
\pgfpathcurveto{\pgfqpoint{4.692788in}{0.773221in}}{\pgfqpoint{4.682189in}{0.777611in}}{\pgfqpoint{4.671139in}{0.777611in}}%
\pgfpathcurveto{\pgfqpoint{4.660089in}{0.777611in}}{\pgfqpoint{4.649490in}{0.773221in}}{\pgfqpoint{4.641676in}{0.765407in}}%
\pgfpathcurveto{\pgfqpoint{4.633862in}{0.757594in}}{\pgfqpoint{4.629472in}{0.746995in}}{\pgfqpoint{4.629472in}{0.735945in}}%
\pgfpathcurveto{\pgfqpoint{4.629472in}{0.724895in}}{\pgfqpoint{4.633862in}{0.714296in}}{\pgfqpoint{4.641676in}{0.706482in}}%
\pgfpathcurveto{\pgfqpoint{4.649490in}{0.698668in}}{\pgfqpoint{4.660089in}{0.694278in}}{\pgfqpoint{4.671139in}{0.694278in}}%
\pgfpathlineto{\pgfqpoint{4.671139in}{0.694278in}}%
\pgfpathclose%
\pgfusepath{stroke}%
\end{pgfscope}%
\begin{pgfscope}%
\pgfpathrectangle{\pgfqpoint{0.847223in}{0.554012in}}{\pgfqpoint{6.200000in}{4.620000in}}%
\pgfusepath{clip}%
\pgfsetbuttcap%
\pgfsetroundjoin%
\pgfsetlinewidth{1.003750pt}%
\definecolor{currentstroke}{rgb}{1.000000,0.000000,0.000000}%
\pgfsetstrokecolor{currentstroke}%
\pgfsetdash{}{0pt}%
\pgfpathmoveto{\pgfqpoint{4.676472in}{0.693534in}}%
\pgfpathcurveto{\pgfqpoint{4.687522in}{0.693534in}}{\pgfqpoint{4.698121in}{0.697924in}}{\pgfqpoint{4.705935in}{0.705738in}}%
\pgfpathcurveto{\pgfqpoint{4.713748in}{0.713551in}}{\pgfqpoint{4.718139in}{0.724150in}}{\pgfqpoint{4.718139in}{0.735200in}}%
\pgfpathcurveto{\pgfqpoint{4.718139in}{0.746251in}}{\pgfqpoint{4.713748in}{0.756850in}}{\pgfqpoint{4.705935in}{0.764663in}}%
\pgfpathcurveto{\pgfqpoint{4.698121in}{0.772477in}}{\pgfqpoint{4.687522in}{0.776867in}}{\pgfqpoint{4.676472in}{0.776867in}}%
\pgfpathcurveto{\pgfqpoint{4.665422in}{0.776867in}}{\pgfqpoint{4.654823in}{0.772477in}}{\pgfqpoint{4.647009in}{0.764663in}}%
\pgfpathcurveto{\pgfqpoint{4.639195in}{0.756850in}}{\pgfqpoint{4.634805in}{0.746251in}}{\pgfqpoint{4.634805in}{0.735200in}}%
\pgfpathcurveto{\pgfqpoint{4.634805in}{0.724150in}}{\pgfqpoint{4.639195in}{0.713551in}}{\pgfqpoint{4.647009in}{0.705738in}}%
\pgfpathcurveto{\pgfqpoint{4.654823in}{0.697924in}}{\pgfqpoint{4.665422in}{0.693534in}}{\pgfqpoint{4.676472in}{0.693534in}}%
\pgfpathlineto{\pgfqpoint{4.676472in}{0.693534in}}%
\pgfpathclose%
\pgfusepath{stroke}%
\end{pgfscope}%
\begin{pgfscope}%
\pgfpathrectangle{\pgfqpoint{0.847223in}{0.554012in}}{\pgfqpoint{6.200000in}{4.620000in}}%
\pgfusepath{clip}%
\pgfsetbuttcap%
\pgfsetroundjoin%
\pgfsetlinewidth{1.003750pt}%
\definecolor{currentstroke}{rgb}{1.000000,0.000000,0.000000}%
\pgfsetstrokecolor{currentstroke}%
\pgfsetdash{}{0pt}%
\pgfpathmoveto{\pgfqpoint{4.681805in}{0.692791in}}%
\pgfpathcurveto{\pgfqpoint{4.692855in}{0.692791in}}{\pgfqpoint{4.703454in}{0.697182in}}{\pgfqpoint{4.711268in}{0.704995in}}%
\pgfpathcurveto{\pgfqpoint{4.719082in}{0.712809in}}{\pgfqpoint{4.723472in}{0.723408in}}{\pgfqpoint{4.723472in}{0.734458in}}%
\pgfpathcurveto{\pgfqpoint{4.723472in}{0.745508in}}{\pgfqpoint{4.719082in}{0.756107in}}{\pgfqpoint{4.711268in}{0.763921in}}%
\pgfpathcurveto{\pgfqpoint{4.703454in}{0.771734in}}{\pgfqpoint{4.692855in}{0.776125in}}{\pgfqpoint{4.681805in}{0.776125in}}%
\pgfpathcurveto{\pgfqpoint{4.670755in}{0.776125in}}{\pgfqpoint{4.660156in}{0.771734in}}{\pgfqpoint{4.652342in}{0.763921in}}%
\pgfpathcurveto{\pgfqpoint{4.644529in}{0.756107in}}{\pgfqpoint{4.640138in}{0.745508in}}{\pgfqpoint{4.640138in}{0.734458in}}%
\pgfpathcurveto{\pgfqpoint{4.640138in}{0.723408in}}{\pgfqpoint{4.644529in}{0.712809in}}{\pgfqpoint{4.652342in}{0.704995in}}%
\pgfpathcurveto{\pgfqpoint{4.660156in}{0.697182in}}{\pgfqpoint{4.670755in}{0.692791in}}{\pgfqpoint{4.681805in}{0.692791in}}%
\pgfpathlineto{\pgfqpoint{4.681805in}{0.692791in}}%
\pgfpathclose%
\pgfusepath{stroke}%
\end{pgfscope}%
\begin{pgfscope}%
\pgfpathrectangle{\pgfqpoint{0.847223in}{0.554012in}}{\pgfqpoint{6.200000in}{4.620000in}}%
\pgfusepath{clip}%
\pgfsetbuttcap%
\pgfsetroundjoin%
\pgfsetlinewidth{1.003750pt}%
\definecolor{currentstroke}{rgb}{1.000000,0.000000,0.000000}%
\pgfsetstrokecolor{currentstroke}%
\pgfsetdash{}{0pt}%
\pgfpathmoveto{\pgfqpoint{4.687138in}{0.692051in}}%
\pgfpathcurveto{\pgfqpoint{4.698188in}{0.692051in}}{\pgfqpoint{4.708787in}{0.696441in}}{\pgfqpoint{4.716601in}{0.704255in}}%
\pgfpathcurveto{\pgfqpoint{4.724415in}{0.712068in}}{\pgfqpoint{4.728805in}{0.722667in}}{\pgfqpoint{4.728805in}{0.733717in}}%
\pgfpathcurveto{\pgfqpoint{4.728805in}{0.744768in}}{\pgfqpoint{4.724415in}{0.755367in}}{\pgfqpoint{4.716601in}{0.763180in}}%
\pgfpathcurveto{\pgfqpoint{4.708787in}{0.770994in}}{\pgfqpoint{4.698188in}{0.775384in}}{\pgfqpoint{4.687138in}{0.775384in}}%
\pgfpathcurveto{\pgfqpoint{4.676088in}{0.775384in}}{\pgfqpoint{4.665489in}{0.770994in}}{\pgfqpoint{4.657676in}{0.763180in}}%
\pgfpathcurveto{\pgfqpoint{4.649862in}{0.755367in}}{\pgfqpoint{4.645472in}{0.744768in}}{\pgfqpoint{4.645472in}{0.733717in}}%
\pgfpathcurveto{\pgfqpoint{4.645472in}{0.722667in}}{\pgfqpoint{4.649862in}{0.712068in}}{\pgfqpoint{4.657676in}{0.704255in}}%
\pgfpathcurveto{\pgfqpoint{4.665489in}{0.696441in}}{\pgfqpoint{4.676088in}{0.692051in}}{\pgfqpoint{4.687138in}{0.692051in}}%
\pgfpathlineto{\pgfqpoint{4.687138in}{0.692051in}}%
\pgfpathclose%
\pgfusepath{stroke}%
\end{pgfscope}%
\begin{pgfscope}%
\pgfpathrectangle{\pgfqpoint{0.847223in}{0.554012in}}{\pgfqpoint{6.200000in}{4.620000in}}%
\pgfusepath{clip}%
\pgfsetbuttcap%
\pgfsetroundjoin%
\pgfsetlinewidth{1.003750pt}%
\definecolor{currentstroke}{rgb}{1.000000,0.000000,0.000000}%
\pgfsetstrokecolor{currentstroke}%
\pgfsetdash{}{0pt}%
\pgfpathmoveto{\pgfqpoint{4.692472in}{0.691312in}}%
\pgfpathcurveto{\pgfqpoint{4.703522in}{0.691312in}}{\pgfqpoint{4.714121in}{0.695702in}}{\pgfqpoint{4.721934in}{0.703516in}}%
\pgfpathcurveto{\pgfqpoint{4.729748in}{0.711329in}}{\pgfqpoint{4.734138in}{0.721928in}}{\pgfqpoint{4.734138in}{0.732979in}}%
\pgfpathcurveto{\pgfqpoint{4.734138in}{0.744029in}}{\pgfqpoint{4.729748in}{0.754628in}}{\pgfqpoint{4.721934in}{0.762441in}}%
\pgfpathcurveto{\pgfqpoint{4.714121in}{0.770255in}}{\pgfqpoint{4.703522in}{0.774645in}}{\pgfqpoint{4.692472in}{0.774645in}}%
\pgfpathcurveto{\pgfqpoint{4.681421in}{0.774645in}}{\pgfqpoint{4.670822in}{0.770255in}}{\pgfqpoint{4.663009in}{0.762441in}}%
\pgfpathcurveto{\pgfqpoint{4.655195in}{0.754628in}}{\pgfqpoint{4.650805in}{0.744029in}}{\pgfqpoint{4.650805in}{0.732979in}}%
\pgfpathcurveto{\pgfqpoint{4.650805in}{0.721928in}}{\pgfqpoint{4.655195in}{0.711329in}}{\pgfqpoint{4.663009in}{0.703516in}}%
\pgfpathcurveto{\pgfqpoint{4.670822in}{0.695702in}}{\pgfqpoint{4.681421in}{0.691312in}}{\pgfqpoint{4.692472in}{0.691312in}}%
\pgfpathlineto{\pgfqpoint{4.692472in}{0.691312in}}%
\pgfpathclose%
\pgfusepath{stroke}%
\end{pgfscope}%
\begin{pgfscope}%
\pgfpathrectangle{\pgfqpoint{0.847223in}{0.554012in}}{\pgfqpoint{6.200000in}{4.620000in}}%
\pgfusepath{clip}%
\pgfsetbuttcap%
\pgfsetroundjoin%
\pgfsetlinewidth{1.003750pt}%
\definecolor{currentstroke}{rgb}{1.000000,0.000000,0.000000}%
\pgfsetstrokecolor{currentstroke}%
\pgfsetdash{}{0pt}%
\pgfpathmoveto{\pgfqpoint{4.697805in}{0.690575in}}%
\pgfpathcurveto{\pgfqpoint{4.708855in}{0.690575in}}{\pgfqpoint{4.719454in}{0.694965in}}{\pgfqpoint{4.727268in}{0.702779in}}%
\pgfpathcurveto{\pgfqpoint{4.735081in}{0.710592in}}{\pgfqpoint{4.739471in}{0.721191in}}{\pgfqpoint{4.739471in}{0.732242in}}%
\pgfpathcurveto{\pgfqpoint{4.739471in}{0.743292in}}{\pgfqpoint{4.735081in}{0.753891in}}{\pgfqpoint{4.727268in}{0.761704in}}%
\pgfpathcurveto{\pgfqpoint{4.719454in}{0.769518in}}{\pgfqpoint{4.708855in}{0.773908in}}{\pgfqpoint{4.697805in}{0.773908in}}%
\pgfpathcurveto{\pgfqpoint{4.686755in}{0.773908in}}{\pgfqpoint{4.676156in}{0.769518in}}{\pgfqpoint{4.668342in}{0.761704in}}%
\pgfpathcurveto{\pgfqpoint{4.660528in}{0.753891in}}{\pgfqpoint{4.656138in}{0.743292in}}{\pgfqpoint{4.656138in}{0.732242in}}%
\pgfpathcurveto{\pgfqpoint{4.656138in}{0.721191in}}{\pgfqpoint{4.660528in}{0.710592in}}{\pgfqpoint{4.668342in}{0.702779in}}%
\pgfpathcurveto{\pgfqpoint{4.676156in}{0.694965in}}{\pgfqpoint{4.686755in}{0.690575in}}{\pgfqpoint{4.697805in}{0.690575in}}%
\pgfpathlineto{\pgfqpoint{4.697805in}{0.690575in}}%
\pgfpathclose%
\pgfusepath{stroke}%
\end{pgfscope}%
\begin{pgfscope}%
\pgfpathrectangle{\pgfqpoint{0.847223in}{0.554012in}}{\pgfqpoint{6.200000in}{4.620000in}}%
\pgfusepath{clip}%
\pgfsetbuttcap%
\pgfsetroundjoin%
\pgfsetlinewidth{1.003750pt}%
\definecolor{currentstroke}{rgb}{1.000000,0.000000,0.000000}%
\pgfsetstrokecolor{currentstroke}%
\pgfsetdash{}{0pt}%
\pgfpathmoveto{\pgfqpoint{4.703138in}{0.689840in}}%
\pgfpathcurveto{\pgfqpoint{4.714188in}{0.689840in}}{\pgfqpoint{4.724787in}{0.694230in}}{\pgfqpoint{4.732601in}{0.702044in}}%
\pgfpathcurveto{\pgfqpoint{4.740414in}{0.709857in}}{\pgfqpoint{4.744805in}{0.720456in}}{\pgfqpoint{4.744805in}{0.731506in}}%
\pgfpathcurveto{\pgfqpoint{4.744805in}{0.742556in}}{\pgfqpoint{4.740414in}{0.753156in}}{\pgfqpoint{4.732601in}{0.760969in}}%
\pgfpathcurveto{\pgfqpoint{4.724787in}{0.768783in}}{\pgfqpoint{4.714188in}{0.773173in}}{\pgfqpoint{4.703138in}{0.773173in}}%
\pgfpathcurveto{\pgfqpoint{4.692088in}{0.773173in}}{\pgfqpoint{4.681489in}{0.768783in}}{\pgfqpoint{4.673675in}{0.760969in}}%
\pgfpathcurveto{\pgfqpoint{4.665862in}{0.753156in}}{\pgfqpoint{4.661471in}{0.742556in}}{\pgfqpoint{4.661471in}{0.731506in}}%
\pgfpathcurveto{\pgfqpoint{4.661471in}{0.720456in}}{\pgfqpoint{4.665862in}{0.709857in}}{\pgfqpoint{4.673675in}{0.702044in}}%
\pgfpathcurveto{\pgfqpoint{4.681489in}{0.694230in}}{\pgfqpoint{4.692088in}{0.689840in}}{\pgfqpoint{4.703138in}{0.689840in}}%
\pgfpathlineto{\pgfqpoint{4.703138in}{0.689840in}}%
\pgfpathclose%
\pgfusepath{stroke}%
\end{pgfscope}%
\begin{pgfscope}%
\pgfpathrectangle{\pgfqpoint{0.847223in}{0.554012in}}{\pgfqpoint{6.200000in}{4.620000in}}%
\pgfusepath{clip}%
\pgfsetbuttcap%
\pgfsetroundjoin%
\pgfsetlinewidth{1.003750pt}%
\definecolor{currentstroke}{rgb}{1.000000,0.000000,0.000000}%
\pgfsetstrokecolor{currentstroke}%
\pgfsetdash{}{0pt}%
\pgfpathmoveto{\pgfqpoint{4.708471in}{0.689106in}}%
\pgfpathcurveto{\pgfqpoint{4.719521in}{0.689106in}}{\pgfqpoint{4.730120in}{0.693497in}}{\pgfqpoint{4.737934in}{0.701310in}}%
\pgfpathcurveto{\pgfqpoint{4.745748in}{0.709124in}}{\pgfqpoint{4.750138in}{0.719723in}}{\pgfqpoint{4.750138in}{0.730773in}}%
\pgfpathcurveto{\pgfqpoint{4.750138in}{0.741823in}}{\pgfqpoint{4.745748in}{0.752422in}}{\pgfqpoint{4.737934in}{0.760236in}}%
\pgfpathcurveto{\pgfqpoint{4.730120in}{0.768049in}}{\pgfqpoint{4.719521in}{0.772440in}}{\pgfqpoint{4.708471in}{0.772440in}}%
\pgfpathcurveto{\pgfqpoint{4.697421in}{0.772440in}}{\pgfqpoint{4.686822in}{0.768049in}}{\pgfqpoint{4.679008in}{0.760236in}}%
\pgfpathcurveto{\pgfqpoint{4.671195in}{0.752422in}}{\pgfqpoint{4.666805in}{0.741823in}}{\pgfqpoint{4.666805in}{0.730773in}}%
\pgfpathcurveto{\pgfqpoint{4.666805in}{0.719723in}}{\pgfqpoint{4.671195in}{0.709124in}}{\pgfqpoint{4.679008in}{0.701310in}}%
\pgfpathcurveto{\pgfqpoint{4.686822in}{0.693497in}}{\pgfqpoint{4.697421in}{0.689106in}}{\pgfqpoint{4.708471in}{0.689106in}}%
\pgfpathlineto{\pgfqpoint{4.708471in}{0.689106in}}%
\pgfpathclose%
\pgfusepath{stroke}%
\end{pgfscope}%
\begin{pgfscope}%
\pgfpathrectangle{\pgfqpoint{0.847223in}{0.554012in}}{\pgfqpoint{6.200000in}{4.620000in}}%
\pgfusepath{clip}%
\pgfsetbuttcap%
\pgfsetroundjoin%
\pgfsetlinewidth{1.003750pt}%
\definecolor{currentstroke}{rgb}{1.000000,0.000000,0.000000}%
\pgfsetstrokecolor{currentstroke}%
\pgfsetdash{}{0pt}%
\pgfpathmoveto{\pgfqpoint{4.713804in}{0.688375in}}%
\pgfpathcurveto{\pgfqpoint{4.724855in}{0.688375in}}{\pgfqpoint{4.735454in}{0.692765in}}{\pgfqpoint{4.743267in}{0.700579in}}%
\pgfpathcurveto{\pgfqpoint{4.751081in}{0.708392in}}{\pgfqpoint{4.755471in}{0.718991in}}{\pgfqpoint{4.755471in}{0.730041in}}%
\pgfpathcurveto{\pgfqpoint{4.755471in}{0.741091in}}{\pgfqpoint{4.751081in}{0.751690in}}{\pgfqpoint{4.743267in}{0.759504in}}%
\pgfpathcurveto{\pgfqpoint{4.735454in}{0.767318in}}{\pgfqpoint{4.724855in}{0.771708in}}{\pgfqpoint{4.713804in}{0.771708in}}%
\pgfpathcurveto{\pgfqpoint{4.702754in}{0.771708in}}{\pgfqpoint{4.692155in}{0.767318in}}{\pgfqpoint{4.684342in}{0.759504in}}%
\pgfpathcurveto{\pgfqpoint{4.676528in}{0.751690in}}{\pgfqpoint{4.672138in}{0.741091in}}{\pgfqpoint{4.672138in}{0.730041in}}%
\pgfpathcurveto{\pgfqpoint{4.672138in}{0.718991in}}{\pgfqpoint{4.676528in}{0.708392in}}{\pgfqpoint{4.684342in}{0.700579in}}%
\pgfpathcurveto{\pgfqpoint{4.692155in}{0.692765in}}{\pgfqpoint{4.702754in}{0.688375in}}{\pgfqpoint{4.713804in}{0.688375in}}%
\pgfpathlineto{\pgfqpoint{4.713804in}{0.688375in}}%
\pgfpathclose%
\pgfusepath{stroke}%
\end{pgfscope}%
\begin{pgfscope}%
\pgfpathrectangle{\pgfqpoint{0.847223in}{0.554012in}}{\pgfqpoint{6.200000in}{4.620000in}}%
\pgfusepath{clip}%
\pgfsetbuttcap%
\pgfsetroundjoin%
\pgfsetlinewidth{1.003750pt}%
\definecolor{currentstroke}{rgb}{1.000000,0.000000,0.000000}%
\pgfsetstrokecolor{currentstroke}%
\pgfsetdash{}{0pt}%
\pgfpathmoveto{\pgfqpoint{4.719138in}{0.687645in}}%
\pgfpathcurveto{\pgfqpoint{4.730188in}{0.687645in}}{\pgfqpoint{4.740787in}{0.692035in}}{\pgfqpoint{4.748600in}{0.699849in}}%
\pgfpathcurveto{\pgfqpoint{4.756414in}{0.707662in}}{\pgfqpoint{4.760804in}{0.718261in}}{\pgfqpoint{4.760804in}{0.729311in}}%
\pgfpathcurveto{\pgfqpoint{4.760804in}{0.740362in}}{\pgfqpoint{4.756414in}{0.750961in}}{\pgfqpoint{4.748600in}{0.758774in}}%
\pgfpathcurveto{\pgfqpoint{4.740787in}{0.766588in}}{\pgfqpoint{4.730188in}{0.770978in}}{\pgfqpoint{4.719138in}{0.770978in}}%
\pgfpathcurveto{\pgfqpoint{4.708087in}{0.770978in}}{\pgfqpoint{4.697488in}{0.766588in}}{\pgfqpoint{4.689675in}{0.758774in}}%
\pgfpathcurveto{\pgfqpoint{4.681861in}{0.750961in}}{\pgfqpoint{4.677471in}{0.740362in}}{\pgfqpoint{4.677471in}{0.729311in}}%
\pgfpathcurveto{\pgfqpoint{4.677471in}{0.718261in}}{\pgfqpoint{4.681861in}{0.707662in}}{\pgfqpoint{4.689675in}{0.699849in}}%
\pgfpathcurveto{\pgfqpoint{4.697488in}{0.692035in}}{\pgfqpoint{4.708087in}{0.687645in}}{\pgfqpoint{4.719138in}{0.687645in}}%
\pgfpathlineto{\pgfqpoint{4.719138in}{0.687645in}}%
\pgfpathclose%
\pgfusepath{stroke}%
\end{pgfscope}%
\begin{pgfscope}%
\pgfpathrectangle{\pgfqpoint{0.847223in}{0.554012in}}{\pgfqpoint{6.200000in}{4.620000in}}%
\pgfusepath{clip}%
\pgfsetbuttcap%
\pgfsetroundjoin%
\pgfsetlinewidth{1.003750pt}%
\definecolor{currentstroke}{rgb}{1.000000,0.000000,0.000000}%
\pgfsetstrokecolor{currentstroke}%
\pgfsetdash{}{0pt}%
\pgfpathmoveto{\pgfqpoint{4.724471in}{0.686917in}}%
\pgfpathcurveto{\pgfqpoint{4.735521in}{0.686917in}}{\pgfqpoint{4.746120in}{0.691307in}}{\pgfqpoint{4.753934in}{0.699121in}}%
\pgfpathcurveto{\pgfqpoint{4.761747in}{0.706934in}}{\pgfqpoint{4.766138in}{0.717533in}}{\pgfqpoint{4.766138in}{0.728583in}}%
\pgfpathcurveto{\pgfqpoint{4.766138in}{0.739633in}}{\pgfqpoint{4.761747in}{0.750233in}}{\pgfqpoint{4.753934in}{0.758046in}}%
\pgfpathcurveto{\pgfqpoint{4.746120in}{0.765860in}}{\pgfqpoint{4.735521in}{0.770250in}}{\pgfqpoint{4.724471in}{0.770250in}}%
\pgfpathcurveto{\pgfqpoint{4.713421in}{0.770250in}}{\pgfqpoint{4.702822in}{0.765860in}}{\pgfqpoint{4.695008in}{0.758046in}}%
\pgfpathcurveto{\pgfqpoint{4.687194in}{0.750233in}}{\pgfqpoint{4.682804in}{0.739633in}}{\pgfqpoint{4.682804in}{0.728583in}}%
\pgfpathcurveto{\pgfqpoint{4.682804in}{0.717533in}}{\pgfqpoint{4.687194in}{0.706934in}}{\pgfqpoint{4.695008in}{0.699121in}}%
\pgfpathcurveto{\pgfqpoint{4.702822in}{0.691307in}}{\pgfqpoint{4.713421in}{0.686917in}}{\pgfqpoint{4.724471in}{0.686917in}}%
\pgfpathlineto{\pgfqpoint{4.724471in}{0.686917in}}%
\pgfpathclose%
\pgfusepath{stroke}%
\end{pgfscope}%
\begin{pgfscope}%
\pgfpathrectangle{\pgfqpoint{0.847223in}{0.554012in}}{\pgfqpoint{6.200000in}{4.620000in}}%
\pgfusepath{clip}%
\pgfsetbuttcap%
\pgfsetroundjoin%
\pgfsetlinewidth{1.003750pt}%
\definecolor{currentstroke}{rgb}{1.000000,0.000000,0.000000}%
\pgfsetstrokecolor{currentstroke}%
\pgfsetdash{}{0pt}%
\pgfpathmoveto{\pgfqpoint{4.729804in}{0.686190in}}%
\pgfpathcurveto{\pgfqpoint{4.740854in}{0.686190in}}{\pgfqpoint{4.751453in}{0.690581in}}{\pgfqpoint{4.759267in}{0.698394in}}%
\pgfpathcurveto{\pgfqpoint{4.767080in}{0.706208in}}{\pgfqpoint{4.771471in}{0.716807in}}{\pgfqpoint{4.771471in}{0.727857in}}%
\pgfpathcurveto{\pgfqpoint{4.771471in}{0.738907in}}{\pgfqpoint{4.767080in}{0.749506in}}{\pgfqpoint{4.759267in}{0.757320in}}%
\pgfpathcurveto{\pgfqpoint{4.751453in}{0.765133in}}{\pgfqpoint{4.740854in}{0.769524in}}{\pgfqpoint{4.729804in}{0.769524in}}%
\pgfpathcurveto{\pgfqpoint{4.718754in}{0.769524in}}{\pgfqpoint{4.708155in}{0.765133in}}{\pgfqpoint{4.700341in}{0.757320in}}%
\pgfpathcurveto{\pgfqpoint{4.692528in}{0.749506in}}{\pgfqpoint{4.688137in}{0.738907in}}{\pgfqpoint{4.688137in}{0.727857in}}%
\pgfpathcurveto{\pgfqpoint{4.688137in}{0.716807in}}{\pgfqpoint{4.692528in}{0.706208in}}{\pgfqpoint{4.700341in}{0.698394in}}%
\pgfpathcurveto{\pgfqpoint{4.708155in}{0.690581in}}{\pgfqpoint{4.718754in}{0.686190in}}{\pgfqpoint{4.729804in}{0.686190in}}%
\pgfpathlineto{\pgfqpoint{4.729804in}{0.686190in}}%
\pgfpathclose%
\pgfusepath{stroke}%
\end{pgfscope}%
\begin{pgfscope}%
\pgfpathrectangle{\pgfqpoint{0.847223in}{0.554012in}}{\pgfqpoint{6.200000in}{4.620000in}}%
\pgfusepath{clip}%
\pgfsetbuttcap%
\pgfsetroundjoin%
\pgfsetlinewidth{1.003750pt}%
\definecolor{currentstroke}{rgb}{1.000000,0.000000,0.000000}%
\pgfsetstrokecolor{currentstroke}%
\pgfsetdash{}{0pt}%
\pgfpathmoveto{\pgfqpoint{4.735137in}{0.685466in}}%
\pgfpathcurveto{\pgfqpoint{4.746187in}{0.685466in}}{\pgfqpoint{4.756786in}{0.689856in}}{\pgfqpoint{4.764600in}{0.697670in}}%
\pgfpathcurveto{\pgfqpoint{4.772414in}{0.705483in}}{\pgfqpoint{4.776804in}{0.716082in}}{\pgfqpoint{4.776804in}{0.727132in}}%
\pgfpathcurveto{\pgfqpoint{4.776804in}{0.738183in}}{\pgfqpoint{4.772414in}{0.748782in}}{\pgfqpoint{4.764600in}{0.756595in}}%
\pgfpathcurveto{\pgfqpoint{4.756786in}{0.764409in}}{\pgfqpoint{4.746187in}{0.768799in}}{\pgfqpoint{4.735137in}{0.768799in}}%
\pgfpathcurveto{\pgfqpoint{4.724087in}{0.768799in}}{\pgfqpoint{4.713488in}{0.764409in}}{\pgfqpoint{4.705674in}{0.756595in}}%
\pgfpathcurveto{\pgfqpoint{4.697861in}{0.748782in}}{\pgfqpoint{4.693471in}{0.738183in}}{\pgfqpoint{4.693471in}{0.727132in}}%
\pgfpathcurveto{\pgfqpoint{4.693471in}{0.716082in}}{\pgfqpoint{4.697861in}{0.705483in}}{\pgfqpoint{4.705674in}{0.697670in}}%
\pgfpathcurveto{\pgfqpoint{4.713488in}{0.689856in}}{\pgfqpoint{4.724087in}{0.685466in}}{\pgfqpoint{4.735137in}{0.685466in}}%
\pgfpathlineto{\pgfqpoint{4.735137in}{0.685466in}}%
\pgfpathclose%
\pgfusepath{stroke}%
\end{pgfscope}%
\begin{pgfscope}%
\pgfpathrectangle{\pgfqpoint{0.847223in}{0.554012in}}{\pgfqpoint{6.200000in}{4.620000in}}%
\pgfusepath{clip}%
\pgfsetbuttcap%
\pgfsetroundjoin%
\pgfsetlinewidth{1.003750pt}%
\definecolor{currentstroke}{rgb}{1.000000,0.000000,0.000000}%
\pgfsetstrokecolor{currentstroke}%
\pgfsetdash{}{0pt}%
\pgfpathmoveto{\pgfqpoint{4.740470in}{0.684743in}}%
\pgfpathcurveto{\pgfqpoint{4.751521in}{0.684743in}}{\pgfqpoint{4.762120in}{0.689133in}}{\pgfqpoint{4.769933in}{0.696947in}}%
\pgfpathcurveto{\pgfqpoint{4.777747in}{0.704760in}}{\pgfqpoint{4.782137in}{0.715360in}}{\pgfqpoint{4.782137in}{0.726410in}}%
\pgfpathcurveto{\pgfqpoint{4.782137in}{0.737460in}}{\pgfqpoint{4.777747in}{0.748059in}}{\pgfqpoint{4.769933in}{0.755872in}}%
\pgfpathcurveto{\pgfqpoint{4.762120in}{0.763686in}}{\pgfqpoint{4.751521in}{0.768076in}}{\pgfqpoint{4.740470in}{0.768076in}}%
\pgfpathcurveto{\pgfqpoint{4.729420in}{0.768076in}}{\pgfqpoint{4.718821in}{0.763686in}}{\pgfqpoint{4.711008in}{0.755872in}}%
\pgfpathcurveto{\pgfqpoint{4.703194in}{0.748059in}}{\pgfqpoint{4.698804in}{0.737460in}}{\pgfqpoint{4.698804in}{0.726410in}}%
\pgfpathcurveto{\pgfqpoint{4.698804in}{0.715360in}}{\pgfqpoint{4.703194in}{0.704760in}}{\pgfqpoint{4.711008in}{0.696947in}}%
\pgfpathcurveto{\pgfqpoint{4.718821in}{0.689133in}}{\pgfqpoint{4.729420in}{0.684743in}}{\pgfqpoint{4.740470in}{0.684743in}}%
\pgfpathlineto{\pgfqpoint{4.740470in}{0.684743in}}%
\pgfpathclose%
\pgfusepath{stroke}%
\end{pgfscope}%
\begin{pgfscope}%
\pgfpathrectangle{\pgfqpoint{0.847223in}{0.554012in}}{\pgfqpoint{6.200000in}{4.620000in}}%
\pgfusepath{clip}%
\pgfsetbuttcap%
\pgfsetroundjoin%
\pgfsetlinewidth{1.003750pt}%
\definecolor{currentstroke}{rgb}{1.000000,0.000000,0.000000}%
\pgfsetstrokecolor{currentstroke}%
\pgfsetdash{}{0pt}%
\pgfpathmoveto{\pgfqpoint{4.745804in}{0.684022in}}%
\pgfpathcurveto{\pgfqpoint{4.756854in}{0.684022in}}{\pgfqpoint{4.767453in}{0.688412in}}{\pgfqpoint{4.775266in}{0.696226in}}%
\pgfpathcurveto{\pgfqpoint{4.783080in}{0.704039in}}{\pgfqpoint{4.787470in}{0.714638in}}{\pgfqpoint{4.787470in}{0.725689in}}%
\pgfpathcurveto{\pgfqpoint{4.787470in}{0.736739in}}{\pgfqpoint{4.783080in}{0.747338in}}{\pgfqpoint{4.775266in}{0.755151in}}%
\pgfpathcurveto{\pgfqpoint{4.767453in}{0.762965in}}{\pgfqpoint{4.756854in}{0.767355in}}{\pgfqpoint{4.745804in}{0.767355in}}%
\pgfpathcurveto{\pgfqpoint{4.734754in}{0.767355in}}{\pgfqpoint{4.724155in}{0.762965in}}{\pgfqpoint{4.716341in}{0.755151in}}%
\pgfpathcurveto{\pgfqpoint{4.708527in}{0.747338in}}{\pgfqpoint{4.704137in}{0.736739in}}{\pgfqpoint{4.704137in}{0.725689in}}%
\pgfpathcurveto{\pgfqpoint{4.704137in}{0.714638in}}{\pgfqpoint{4.708527in}{0.704039in}}{\pgfqpoint{4.716341in}{0.696226in}}%
\pgfpathcurveto{\pgfqpoint{4.724155in}{0.688412in}}{\pgfqpoint{4.734754in}{0.684022in}}{\pgfqpoint{4.745804in}{0.684022in}}%
\pgfpathlineto{\pgfqpoint{4.745804in}{0.684022in}}%
\pgfpathclose%
\pgfusepath{stroke}%
\end{pgfscope}%
\begin{pgfscope}%
\pgfpathrectangle{\pgfqpoint{0.847223in}{0.554012in}}{\pgfqpoint{6.200000in}{4.620000in}}%
\pgfusepath{clip}%
\pgfsetbuttcap%
\pgfsetroundjoin%
\pgfsetlinewidth{1.003750pt}%
\definecolor{currentstroke}{rgb}{1.000000,0.000000,0.000000}%
\pgfsetstrokecolor{currentstroke}%
\pgfsetdash{}{0pt}%
\pgfpathmoveto{\pgfqpoint{4.751137in}{0.683303in}}%
\pgfpathcurveto{\pgfqpoint{4.762187in}{0.683303in}}{\pgfqpoint{4.772786in}{0.687693in}}{\pgfqpoint{4.780600in}{0.695506in}}%
\pgfpathcurveto{\pgfqpoint{4.788413in}{0.703320in}}{\pgfqpoint{4.792804in}{0.713919in}}{\pgfqpoint{4.792804in}{0.724969in}}%
\pgfpathcurveto{\pgfqpoint{4.792804in}{0.736019in}}{\pgfqpoint{4.788413in}{0.746618in}}{\pgfqpoint{4.780600in}{0.754432in}}%
\pgfpathcurveto{\pgfqpoint{4.772786in}{0.762246in}}{\pgfqpoint{4.762187in}{0.766636in}}{\pgfqpoint{4.751137in}{0.766636in}}%
\pgfpathcurveto{\pgfqpoint{4.740087in}{0.766636in}}{\pgfqpoint{4.729488in}{0.762246in}}{\pgfqpoint{4.721674in}{0.754432in}}%
\pgfpathcurveto{\pgfqpoint{4.713861in}{0.746618in}}{\pgfqpoint{4.709470in}{0.736019in}}{\pgfqpoint{4.709470in}{0.724969in}}%
\pgfpathcurveto{\pgfqpoint{4.709470in}{0.713919in}}{\pgfqpoint{4.713861in}{0.703320in}}{\pgfqpoint{4.721674in}{0.695506in}}%
\pgfpathcurveto{\pgfqpoint{4.729488in}{0.687693in}}{\pgfqpoint{4.740087in}{0.683303in}}{\pgfqpoint{4.751137in}{0.683303in}}%
\pgfpathlineto{\pgfqpoint{4.751137in}{0.683303in}}%
\pgfpathclose%
\pgfusepath{stroke}%
\end{pgfscope}%
\begin{pgfscope}%
\pgfpathrectangle{\pgfqpoint{0.847223in}{0.554012in}}{\pgfqpoint{6.200000in}{4.620000in}}%
\pgfusepath{clip}%
\pgfsetbuttcap%
\pgfsetroundjoin%
\pgfsetlinewidth{1.003750pt}%
\definecolor{currentstroke}{rgb}{1.000000,0.000000,0.000000}%
\pgfsetstrokecolor{currentstroke}%
\pgfsetdash{}{0pt}%
\pgfpathmoveto{\pgfqpoint{4.756470in}{0.682585in}}%
\pgfpathcurveto{\pgfqpoint{4.767520in}{0.682585in}}{\pgfqpoint{4.778119in}{0.686975in}}{\pgfqpoint{4.785933in}{0.694789in}}%
\pgfpathcurveto{\pgfqpoint{4.793747in}{0.702602in}}{\pgfqpoint{4.798137in}{0.713201in}}{\pgfqpoint{4.798137in}{0.724252in}}%
\pgfpathcurveto{\pgfqpoint{4.798137in}{0.735302in}}{\pgfqpoint{4.793747in}{0.745901in}}{\pgfqpoint{4.785933in}{0.753714in}}%
\pgfpathcurveto{\pgfqpoint{4.778119in}{0.761528in}}{\pgfqpoint{4.767520in}{0.765918in}}{\pgfqpoint{4.756470in}{0.765918in}}%
\pgfpathcurveto{\pgfqpoint{4.745420in}{0.765918in}}{\pgfqpoint{4.734821in}{0.761528in}}{\pgfqpoint{4.727007in}{0.753714in}}%
\pgfpathcurveto{\pgfqpoint{4.719194in}{0.745901in}}{\pgfqpoint{4.714803in}{0.735302in}}{\pgfqpoint{4.714803in}{0.724252in}}%
\pgfpathcurveto{\pgfqpoint{4.714803in}{0.713201in}}{\pgfqpoint{4.719194in}{0.702602in}}{\pgfqpoint{4.727007in}{0.694789in}}%
\pgfpathcurveto{\pgfqpoint{4.734821in}{0.686975in}}{\pgfqpoint{4.745420in}{0.682585in}}{\pgfqpoint{4.756470in}{0.682585in}}%
\pgfpathlineto{\pgfqpoint{4.756470in}{0.682585in}}%
\pgfpathclose%
\pgfusepath{stroke}%
\end{pgfscope}%
\begin{pgfscope}%
\pgfpathrectangle{\pgfqpoint{0.847223in}{0.554012in}}{\pgfqpoint{6.200000in}{4.620000in}}%
\pgfusepath{clip}%
\pgfsetbuttcap%
\pgfsetroundjoin%
\pgfsetlinewidth{1.003750pt}%
\definecolor{currentstroke}{rgb}{1.000000,0.000000,0.000000}%
\pgfsetstrokecolor{currentstroke}%
\pgfsetdash{}{0pt}%
\pgfpathmoveto{\pgfqpoint{4.761803in}{0.681869in}}%
\pgfpathcurveto{\pgfqpoint{4.772853in}{0.681869in}}{\pgfqpoint{4.783453in}{0.686259in}}{\pgfqpoint{4.791266in}{0.694073in}}%
\pgfpathcurveto{\pgfqpoint{4.799080in}{0.701887in}}{\pgfqpoint{4.803470in}{0.712486in}}{\pgfqpoint{4.803470in}{0.723536in}}%
\pgfpathcurveto{\pgfqpoint{4.803470in}{0.734586in}}{\pgfqpoint{4.799080in}{0.745185in}}{\pgfqpoint{4.791266in}{0.752998in}}%
\pgfpathcurveto{\pgfqpoint{4.783453in}{0.760812in}}{\pgfqpoint{4.772853in}{0.765202in}}{\pgfqpoint{4.761803in}{0.765202in}}%
\pgfpathcurveto{\pgfqpoint{4.750753in}{0.765202in}}{\pgfqpoint{4.740154in}{0.760812in}}{\pgfqpoint{4.732341in}{0.752998in}}%
\pgfpathcurveto{\pgfqpoint{4.724527in}{0.745185in}}{\pgfqpoint{4.720137in}{0.734586in}}{\pgfqpoint{4.720137in}{0.723536in}}%
\pgfpathcurveto{\pgfqpoint{4.720137in}{0.712486in}}{\pgfqpoint{4.724527in}{0.701887in}}{\pgfqpoint{4.732341in}{0.694073in}}%
\pgfpathcurveto{\pgfqpoint{4.740154in}{0.686259in}}{\pgfqpoint{4.750753in}{0.681869in}}{\pgfqpoint{4.761803in}{0.681869in}}%
\pgfpathlineto{\pgfqpoint{4.761803in}{0.681869in}}%
\pgfpathclose%
\pgfusepath{stroke}%
\end{pgfscope}%
\begin{pgfscope}%
\pgfpathrectangle{\pgfqpoint{0.847223in}{0.554012in}}{\pgfqpoint{6.200000in}{4.620000in}}%
\pgfusepath{clip}%
\pgfsetbuttcap%
\pgfsetroundjoin%
\pgfsetlinewidth{1.003750pt}%
\definecolor{currentstroke}{rgb}{1.000000,0.000000,0.000000}%
\pgfsetstrokecolor{currentstroke}%
\pgfsetdash{}{0pt}%
\pgfpathmoveto{\pgfqpoint{4.767137in}{0.681155in}}%
\pgfpathcurveto{\pgfqpoint{4.778187in}{0.681155in}}{\pgfqpoint{4.788786in}{0.685545in}}{\pgfqpoint{4.796599in}{0.693359in}}%
\pgfpathcurveto{\pgfqpoint{4.804413in}{0.701172in}}{\pgfqpoint{4.808803in}{0.711771in}}{\pgfqpoint{4.808803in}{0.722822in}}%
\pgfpathcurveto{\pgfqpoint{4.808803in}{0.733872in}}{\pgfqpoint{4.804413in}{0.744471in}}{\pgfqpoint{4.796599in}{0.752284in}}%
\pgfpathcurveto{\pgfqpoint{4.788786in}{0.760098in}}{\pgfqpoint{4.778187in}{0.764488in}}{\pgfqpoint{4.767137in}{0.764488in}}%
\pgfpathcurveto{\pgfqpoint{4.756086in}{0.764488in}}{\pgfqpoint{4.745487in}{0.760098in}}{\pgfqpoint{4.737674in}{0.752284in}}%
\pgfpathcurveto{\pgfqpoint{4.729860in}{0.744471in}}{\pgfqpoint{4.725470in}{0.733872in}}{\pgfqpoint{4.725470in}{0.722822in}}%
\pgfpathcurveto{\pgfqpoint{4.725470in}{0.711771in}}{\pgfqpoint{4.729860in}{0.701172in}}{\pgfqpoint{4.737674in}{0.693359in}}%
\pgfpathcurveto{\pgfqpoint{4.745487in}{0.685545in}}{\pgfqpoint{4.756086in}{0.681155in}}{\pgfqpoint{4.767137in}{0.681155in}}%
\pgfpathlineto{\pgfqpoint{4.767137in}{0.681155in}}%
\pgfpathclose%
\pgfusepath{stroke}%
\end{pgfscope}%
\begin{pgfscope}%
\pgfpathrectangle{\pgfqpoint{0.847223in}{0.554012in}}{\pgfqpoint{6.200000in}{4.620000in}}%
\pgfusepath{clip}%
\pgfsetbuttcap%
\pgfsetroundjoin%
\pgfsetlinewidth{1.003750pt}%
\definecolor{currentstroke}{rgb}{1.000000,0.000000,0.000000}%
\pgfsetstrokecolor{currentstroke}%
\pgfsetdash{}{0pt}%
\pgfpathmoveto{\pgfqpoint{4.772470in}{0.680442in}}%
\pgfpathcurveto{\pgfqpoint{4.783520in}{0.680442in}}{\pgfqpoint{4.794119in}{0.684833in}}{\pgfqpoint{4.801933in}{0.692646in}}%
\pgfpathcurveto{\pgfqpoint{4.809746in}{0.700460in}}{\pgfqpoint{4.814136in}{0.711059in}}{\pgfqpoint{4.814136in}{0.722109in}}%
\pgfpathcurveto{\pgfqpoint{4.814136in}{0.733159in}}{\pgfqpoint{4.809746in}{0.743758in}}{\pgfqpoint{4.801933in}{0.751572in}}%
\pgfpathcurveto{\pgfqpoint{4.794119in}{0.759385in}}{\pgfqpoint{4.783520in}{0.763776in}}{\pgfqpoint{4.772470in}{0.763776in}}%
\pgfpathcurveto{\pgfqpoint{4.761420in}{0.763776in}}{\pgfqpoint{4.750821in}{0.759385in}}{\pgfqpoint{4.743007in}{0.751572in}}%
\pgfpathcurveto{\pgfqpoint{4.735193in}{0.743758in}}{\pgfqpoint{4.730803in}{0.733159in}}{\pgfqpoint{4.730803in}{0.722109in}}%
\pgfpathcurveto{\pgfqpoint{4.730803in}{0.711059in}}{\pgfqpoint{4.735193in}{0.700460in}}{\pgfqpoint{4.743007in}{0.692646in}}%
\pgfpathcurveto{\pgfqpoint{4.750821in}{0.684833in}}{\pgfqpoint{4.761420in}{0.680442in}}{\pgfqpoint{4.772470in}{0.680442in}}%
\pgfpathlineto{\pgfqpoint{4.772470in}{0.680442in}}%
\pgfpathclose%
\pgfusepath{stroke}%
\end{pgfscope}%
\begin{pgfscope}%
\pgfpathrectangle{\pgfqpoint{0.847223in}{0.554012in}}{\pgfqpoint{6.200000in}{4.620000in}}%
\pgfusepath{clip}%
\pgfsetbuttcap%
\pgfsetroundjoin%
\pgfsetlinewidth{1.003750pt}%
\definecolor{currentstroke}{rgb}{1.000000,0.000000,0.000000}%
\pgfsetstrokecolor{currentstroke}%
\pgfsetdash{}{0pt}%
\pgfpathmoveto{\pgfqpoint{4.777803in}{0.679732in}}%
\pgfpathcurveto{\pgfqpoint{4.788853in}{0.679732in}}{\pgfqpoint{4.799452in}{0.684122in}}{\pgfqpoint{4.807266in}{0.691936in}}%
\pgfpathcurveto{\pgfqpoint{4.815079in}{0.699749in}}{\pgfqpoint{4.819470in}{0.710348in}}{\pgfqpoint{4.819470in}{0.721398in}}%
\pgfpathcurveto{\pgfqpoint{4.819470in}{0.732448in}}{\pgfqpoint{4.815079in}{0.743047in}}{\pgfqpoint{4.807266in}{0.750861in}}%
\pgfpathcurveto{\pgfqpoint{4.799452in}{0.758675in}}{\pgfqpoint{4.788853in}{0.763065in}}{\pgfqpoint{4.777803in}{0.763065in}}%
\pgfpathcurveto{\pgfqpoint{4.766753in}{0.763065in}}{\pgfqpoint{4.756154in}{0.758675in}}{\pgfqpoint{4.748340in}{0.750861in}}%
\pgfpathcurveto{\pgfqpoint{4.740527in}{0.743047in}}{\pgfqpoint{4.736136in}{0.732448in}}{\pgfqpoint{4.736136in}{0.721398in}}%
\pgfpathcurveto{\pgfqpoint{4.736136in}{0.710348in}}{\pgfqpoint{4.740527in}{0.699749in}}{\pgfqpoint{4.748340in}{0.691936in}}%
\pgfpathcurveto{\pgfqpoint{4.756154in}{0.684122in}}{\pgfqpoint{4.766753in}{0.679732in}}{\pgfqpoint{4.777803in}{0.679732in}}%
\pgfpathlineto{\pgfqpoint{4.777803in}{0.679732in}}%
\pgfpathclose%
\pgfusepath{stroke}%
\end{pgfscope}%
\begin{pgfscope}%
\pgfpathrectangle{\pgfqpoint{0.847223in}{0.554012in}}{\pgfqpoint{6.200000in}{4.620000in}}%
\pgfusepath{clip}%
\pgfsetbuttcap%
\pgfsetroundjoin%
\pgfsetlinewidth{1.003750pt}%
\definecolor{currentstroke}{rgb}{1.000000,0.000000,0.000000}%
\pgfsetstrokecolor{currentstroke}%
\pgfsetdash{}{0pt}%
\pgfpathmoveto{\pgfqpoint{4.783136in}{0.679023in}}%
\pgfpathcurveto{\pgfqpoint{4.794186in}{0.679023in}}{\pgfqpoint{4.804785in}{0.683413in}}{\pgfqpoint{4.812599in}{0.691227in}}%
\pgfpathcurveto{\pgfqpoint{4.820413in}{0.699040in}}{\pgfqpoint{4.824803in}{0.709639in}}{\pgfqpoint{4.824803in}{0.720689in}}%
\pgfpathcurveto{\pgfqpoint{4.824803in}{0.731739in}}{\pgfqpoint{4.820413in}{0.742338in}}{\pgfqpoint{4.812599in}{0.750152in}}%
\pgfpathcurveto{\pgfqpoint{4.804785in}{0.757966in}}{\pgfqpoint{4.794186in}{0.762356in}}{\pgfqpoint{4.783136in}{0.762356in}}%
\pgfpathcurveto{\pgfqpoint{4.772086in}{0.762356in}}{\pgfqpoint{4.761487in}{0.757966in}}{\pgfqpoint{4.753673in}{0.750152in}}%
\pgfpathcurveto{\pgfqpoint{4.745860in}{0.742338in}}{\pgfqpoint{4.741470in}{0.731739in}}{\pgfqpoint{4.741470in}{0.720689in}}%
\pgfpathcurveto{\pgfqpoint{4.741470in}{0.709639in}}{\pgfqpoint{4.745860in}{0.699040in}}{\pgfqpoint{4.753673in}{0.691227in}}%
\pgfpathcurveto{\pgfqpoint{4.761487in}{0.683413in}}{\pgfqpoint{4.772086in}{0.679023in}}{\pgfqpoint{4.783136in}{0.679023in}}%
\pgfpathlineto{\pgfqpoint{4.783136in}{0.679023in}}%
\pgfpathclose%
\pgfusepath{stroke}%
\end{pgfscope}%
\begin{pgfscope}%
\pgfpathrectangle{\pgfqpoint{0.847223in}{0.554012in}}{\pgfqpoint{6.200000in}{4.620000in}}%
\pgfusepath{clip}%
\pgfsetbuttcap%
\pgfsetroundjoin%
\pgfsetlinewidth{1.003750pt}%
\definecolor{currentstroke}{rgb}{1.000000,0.000000,0.000000}%
\pgfsetstrokecolor{currentstroke}%
\pgfsetdash{}{0pt}%
\pgfpathmoveto{\pgfqpoint{4.788469in}{0.678315in}}%
\pgfpathcurveto{\pgfqpoint{4.799520in}{0.678315in}}{\pgfqpoint{4.810119in}{0.682706in}}{\pgfqpoint{4.817932in}{0.690519in}}%
\pgfpathcurveto{\pgfqpoint{4.825746in}{0.698333in}}{\pgfqpoint{4.830136in}{0.708932in}}{\pgfqpoint{4.830136in}{0.719982in}}%
\pgfpathcurveto{\pgfqpoint{4.830136in}{0.731032in}}{\pgfqpoint{4.825746in}{0.741631in}}{\pgfqpoint{4.817932in}{0.749445in}}%
\pgfpathcurveto{\pgfqpoint{4.810119in}{0.757258in}}{\pgfqpoint{4.799520in}{0.761649in}}{\pgfqpoint{4.788469in}{0.761649in}}%
\pgfpathcurveto{\pgfqpoint{4.777419in}{0.761649in}}{\pgfqpoint{4.766820in}{0.757258in}}{\pgfqpoint{4.759007in}{0.749445in}}%
\pgfpathcurveto{\pgfqpoint{4.751193in}{0.741631in}}{\pgfqpoint{4.746803in}{0.731032in}}{\pgfqpoint{4.746803in}{0.719982in}}%
\pgfpathcurveto{\pgfqpoint{4.746803in}{0.708932in}}{\pgfqpoint{4.751193in}{0.698333in}}{\pgfqpoint{4.759007in}{0.690519in}}%
\pgfpathcurveto{\pgfqpoint{4.766820in}{0.682706in}}{\pgfqpoint{4.777419in}{0.678315in}}{\pgfqpoint{4.788469in}{0.678315in}}%
\pgfpathlineto{\pgfqpoint{4.788469in}{0.678315in}}%
\pgfpathclose%
\pgfusepath{stroke}%
\end{pgfscope}%
\begin{pgfscope}%
\pgfpathrectangle{\pgfqpoint{0.847223in}{0.554012in}}{\pgfqpoint{6.200000in}{4.620000in}}%
\pgfusepath{clip}%
\pgfsetbuttcap%
\pgfsetroundjoin%
\pgfsetlinewidth{1.003750pt}%
\definecolor{currentstroke}{rgb}{1.000000,0.000000,0.000000}%
\pgfsetstrokecolor{currentstroke}%
\pgfsetdash{}{0pt}%
\pgfpathmoveto{\pgfqpoint{4.793803in}{0.677610in}}%
\pgfpathcurveto{\pgfqpoint{4.804853in}{0.677610in}}{\pgfqpoint{4.815452in}{0.682000in}}{\pgfqpoint{4.823265in}{0.689813in}}%
\pgfpathcurveto{\pgfqpoint{4.831079in}{0.697627in}}{\pgfqpoint{4.835469in}{0.708226in}}{\pgfqpoint{4.835469in}{0.719276in}}%
\pgfpathcurveto{\pgfqpoint{4.835469in}{0.730326in}}{\pgfqpoint{4.831079in}{0.740925in}}{\pgfqpoint{4.823265in}{0.748739in}}%
\pgfpathcurveto{\pgfqpoint{4.815452in}{0.756553in}}{\pgfqpoint{4.804853in}{0.760943in}}{\pgfqpoint{4.793803in}{0.760943in}}%
\pgfpathcurveto{\pgfqpoint{4.782753in}{0.760943in}}{\pgfqpoint{4.772153in}{0.756553in}}{\pgfqpoint{4.764340in}{0.748739in}}%
\pgfpathcurveto{\pgfqpoint{4.756526in}{0.740925in}}{\pgfqpoint{4.752136in}{0.730326in}}{\pgfqpoint{4.752136in}{0.719276in}}%
\pgfpathcurveto{\pgfqpoint{4.752136in}{0.708226in}}{\pgfqpoint{4.756526in}{0.697627in}}{\pgfqpoint{4.764340in}{0.689813in}}%
\pgfpathcurveto{\pgfqpoint{4.772153in}{0.682000in}}{\pgfqpoint{4.782753in}{0.677610in}}{\pgfqpoint{4.793803in}{0.677610in}}%
\pgfpathlineto{\pgfqpoint{4.793803in}{0.677610in}}%
\pgfpathclose%
\pgfusepath{stroke}%
\end{pgfscope}%
\begin{pgfscope}%
\pgfpathrectangle{\pgfqpoint{0.847223in}{0.554012in}}{\pgfqpoint{6.200000in}{4.620000in}}%
\pgfusepath{clip}%
\pgfsetbuttcap%
\pgfsetroundjoin%
\pgfsetlinewidth{1.003750pt}%
\definecolor{currentstroke}{rgb}{1.000000,0.000000,0.000000}%
\pgfsetstrokecolor{currentstroke}%
\pgfsetdash{}{0pt}%
\pgfpathmoveto{\pgfqpoint{4.799136in}{0.676906in}}%
\pgfpathcurveto{\pgfqpoint{4.810186in}{0.676906in}}{\pgfqpoint{4.820785in}{0.681296in}}{\pgfqpoint{4.828599in}{0.689109in}}%
\pgfpathcurveto{\pgfqpoint{4.836412in}{0.696923in}}{\pgfqpoint{4.840803in}{0.707522in}}{\pgfqpoint{4.840803in}{0.718572in}}%
\pgfpathcurveto{\pgfqpoint{4.840803in}{0.729622in}}{\pgfqpoint{4.836412in}{0.740221in}}{\pgfqpoint{4.828599in}{0.748035in}}%
\pgfpathcurveto{\pgfqpoint{4.820785in}{0.755849in}}{\pgfqpoint{4.810186in}{0.760239in}}{\pgfqpoint{4.799136in}{0.760239in}}%
\pgfpathcurveto{\pgfqpoint{4.788086in}{0.760239in}}{\pgfqpoint{4.777487in}{0.755849in}}{\pgfqpoint{4.769673in}{0.748035in}}%
\pgfpathcurveto{\pgfqpoint{4.761859in}{0.740221in}}{\pgfqpoint{4.757469in}{0.729622in}}{\pgfqpoint{4.757469in}{0.718572in}}%
\pgfpathcurveto{\pgfqpoint{4.757469in}{0.707522in}}{\pgfqpoint{4.761859in}{0.696923in}}{\pgfqpoint{4.769673in}{0.689109in}}%
\pgfpathcurveto{\pgfqpoint{4.777487in}{0.681296in}}{\pgfqpoint{4.788086in}{0.676906in}}{\pgfqpoint{4.799136in}{0.676906in}}%
\pgfpathlineto{\pgfqpoint{4.799136in}{0.676906in}}%
\pgfpathclose%
\pgfusepath{stroke}%
\end{pgfscope}%
\begin{pgfscope}%
\pgfpathrectangle{\pgfqpoint{0.847223in}{0.554012in}}{\pgfqpoint{6.200000in}{4.620000in}}%
\pgfusepath{clip}%
\pgfsetbuttcap%
\pgfsetroundjoin%
\pgfsetlinewidth{1.003750pt}%
\definecolor{currentstroke}{rgb}{1.000000,0.000000,0.000000}%
\pgfsetstrokecolor{currentstroke}%
\pgfsetdash{}{0pt}%
\pgfpathmoveto{\pgfqpoint{4.804469in}{0.676203in}}%
\pgfpathcurveto{\pgfqpoint{4.815519in}{0.676203in}}{\pgfqpoint{4.826118in}{0.680594in}}{\pgfqpoint{4.833932in}{0.688407in}}%
\pgfpathcurveto{\pgfqpoint{4.841745in}{0.696221in}}{\pgfqpoint{4.846136in}{0.706820in}}{\pgfqpoint{4.846136in}{0.717870in}}%
\pgfpathcurveto{\pgfqpoint{4.846136in}{0.728920in}}{\pgfqpoint{4.841745in}{0.739519in}}{\pgfqpoint{4.833932in}{0.747333in}}%
\pgfpathcurveto{\pgfqpoint{4.826118in}{0.755146in}}{\pgfqpoint{4.815519in}{0.759537in}}{\pgfqpoint{4.804469in}{0.759537in}}%
\pgfpathcurveto{\pgfqpoint{4.793419in}{0.759537in}}{\pgfqpoint{4.782820in}{0.755146in}}{\pgfqpoint{4.775006in}{0.747333in}}%
\pgfpathcurveto{\pgfqpoint{4.767193in}{0.739519in}}{\pgfqpoint{4.762802in}{0.728920in}}{\pgfqpoint{4.762802in}{0.717870in}}%
\pgfpathcurveto{\pgfqpoint{4.762802in}{0.706820in}}{\pgfqpoint{4.767193in}{0.696221in}}{\pgfqpoint{4.775006in}{0.688407in}}%
\pgfpathcurveto{\pgfqpoint{4.782820in}{0.680594in}}{\pgfqpoint{4.793419in}{0.676203in}}{\pgfqpoint{4.804469in}{0.676203in}}%
\pgfpathlineto{\pgfqpoint{4.804469in}{0.676203in}}%
\pgfpathclose%
\pgfusepath{stroke}%
\end{pgfscope}%
\begin{pgfscope}%
\pgfpathrectangle{\pgfqpoint{0.847223in}{0.554012in}}{\pgfqpoint{6.200000in}{4.620000in}}%
\pgfusepath{clip}%
\pgfsetbuttcap%
\pgfsetroundjoin%
\pgfsetlinewidth{1.003750pt}%
\definecolor{currentstroke}{rgb}{1.000000,0.000000,0.000000}%
\pgfsetstrokecolor{currentstroke}%
\pgfsetdash{}{0pt}%
\pgfpathmoveto{\pgfqpoint{4.809802in}{0.675503in}}%
\pgfpathcurveto{\pgfqpoint{4.820852in}{0.675503in}}{\pgfqpoint{4.831451in}{0.679893in}}{\pgfqpoint{4.839265in}{0.687707in}}%
\pgfpathcurveto{\pgfqpoint{4.847079in}{0.695520in}}{\pgfqpoint{4.851469in}{0.706119in}}{\pgfqpoint{4.851469in}{0.717169in}}%
\pgfpathcurveto{\pgfqpoint{4.851469in}{0.728219in}}{\pgfqpoint{4.847079in}{0.738818in}}{\pgfqpoint{4.839265in}{0.746632in}}%
\pgfpathcurveto{\pgfqpoint{4.831451in}{0.754446in}}{\pgfqpoint{4.820852in}{0.758836in}}{\pgfqpoint{4.809802in}{0.758836in}}%
\pgfpathcurveto{\pgfqpoint{4.798752in}{0.758836in}}{\pgfqpoint{4.788153in}{0.754446in}}{\pgfqpoint{4.780340in}{0.746632in}}%
\pgfpathcurveto{\pgfqpoint{4.772526in}{0.738818in}}{\pgfqpoint{4.768136in}{0.728219in}}{\pgfqpoint{4.768136in}{0.717169in}}%
\pgfpathcurveto{\pgfqpoint{4.768136in}{0.706119in}}{\pgfqpoint{4.772526in}{0.695520in}}{\pgfqpoint{4.780340in}{0.687707in}}%
\pgfpathcurveto{\pgfqpoint{4.788153in}{0.679893in}}{\pgfqpoint{4.798752in}{0.675503in}}{\pgfqpoint{4.809802in}{0.675503in}}%
\pgfpathlineto{\pgfqpoint{4.809802in}{0.675503in}}%
\pgfpathclose%
\pgfusepath{stroke}%
\end{pgfscope}%
\begin{pgfscope}%
\pgfpathrectangle{\pgfqpoint{0.847223in}{0.554012in}}{\pgfqpoint{6.200000in}{4.620000in}}%
\pgfusepath{clip}%
\pgfsetbuttcap%
\pgfsetroundjoin%
\pgfsetlinewidth{1.003750pt}%
\definecolor{currentstroke}{rgb}{1.000000,0.000000,0.000000}%
\pgfsetstrokecolor{currentstroke}%
\pgfsetdash{}{0pt}%
\pgfpathmoveto{\pgfqpoint{4.815136in}{0.674804in}}%
\pgfpathcurveto{\pgfqpoint{4.826186in}{0.674804in}}{\pgfqpoint{4.836785in}{0.679194in}}{\pgfqpoint{4.844598in}{0.687008in}}%
\pgfpathcurveto{\pgfqpoint{4.852412in}{0.694821in}}{\pgfqpoint{4.856802in}{0.705420in}}{\pgfqpoint{4.856802in}{0.716470in}}%
\pgfpathcurveto{\pgfqpoint{4.856802in}{0.727520in}}{\pgfqpoint{4.852412in}{0.738120in}}{\pgfqpoint{4.844598in}{0.745933in}}%
\pgfpathcurveto{\pgfqpoint{4.836785in}{0.753747in}}{\pgfqpoint{4.826186in}{0.758137in}}{\pgfqpoint{4.815136in}{0.758137in}}%
\pgfpathcurveto{\pgfqpoint{4.804085in}{0.758137in}}{\pgfqpoint{4.793486in}{0.753747in}}{\pgfqpoint{4.785673in}{0.745933in}}%
\pgfpathcurveto{\pgfqpoint{4.777859in}{0.738120in}}{\pgfqpoint{4.773469in}{0.727520in}}{\pgfqpoint{4.773469in}{0.716470in}}%
\pgfpathcurveto{\pgfqpoint{4.773469in}{0.705420in}}{\pgfqpoint{4.777859in}{0.694821in}}{\pgfqpoint{4.785673in}{0.687008in}}%
\pgfpathcurveto{\pgfqpoint{4.793486in}{0.679194in}}{\pgfqpoint{4.804085in}{0.674804in}}{\pgfqpoint{4.815136in}{0.674804in}}%
\pgfpathlineto{\pgfqpoint{4.815136in}{0.674804in}}%
\pgfpathclose%
\pgfusepath{stroke}%
\end{pgfscope}%
\begin{pgfscope}%
\pgfpathrectangle{\pgfqpoint{0.847223in}{0.554012in}}{\pgfqpoint{6.200000in}{4.620000in}}%
\pgfusepath{clip}%
\pgfsetbuttcap%
\pgfsetroundjoin%
\pgfsetlinewidth{1.003750pt}%
\definecolor{currentstroke}{rgb}{1.000000,0.000000,0.000000}%
\pgfsetstrokecolor{currentstroke}%
\pgfsetdash{}{0pt}%
\pgfpathmoveto{\pgfqpoint{4.820469in}{0.674106in}}%
\pgfpathcurveto{\pgfqpoint{4.831519in}{0.674106in}}{\pgfqpoint{4.842118in}{0.678497in}}{\pgfqpoint{4.849932in}{0.686310in}}%
\pgfpathcurveto{\pgfqpoint{4.857745in}{0.694124in}}{\pgfqpoint{4.862135in}{0.704723in}}{\pgfqpoint{4.862135in}{0.715773in}}%
\pgfpathcurveto{\pgfqpoint{4.862135in}{0.726823in}}{\pgfqpoint{4.857745in}{0.737422in}}{\pgfqpoint{4.849932in}{0.745236in}}%
\pgfpathcurveto{\pgfqpoint{4.842118in}{0.753049in}}{\pgfqpoint{4.831519in}{0.757440in}}{\pgfqpoint{4.820469in}{0.757440in}}%
\pgfpathcurveto{\pgfqpoint{4.809419in}{0.757440in}}{\pgfqpoint{4.798820in}{0.753049in}}{\pgfqpoint{4.791006in}{0.745236in}}%
\pgfpathcurveto{\pgfqpoint{4.783192in}{0.737422in}}{\pgfqpoint{4.778802in}{0.726823in}}{\pgfqpoint{4.778802in}{0.715773in}}%
\pgfpathcurveto{\pgfqpoint{4.778802in}{0.704723in}}{\pgfqpoint{4.783192in}{0.694124in}}{\pgfqpoint{4.791006in}{0.686310in}}%
\pgfpathcurveto{\pgfqpoint{4.798820in}{0.678497in}}{\pgfqpoint{4.809419in}{0.674106in}}{\pgfqpoint{4.820469in}{0.674106in}}%
\pgfpathlineto{\pgfqpoint{4.820469in}{0.674106in}}%
\pgfpathclose%
\pgfusepath{stroke}%
\end{pgfscope}%
\begin{pgfscope}%
\pgfpathrectangle{\pgfqpoint{0.847223in}{0.554012in}}{\pgfqpoint{6.200000in}{4.620000in}}%
\pgfusepath{clip}%
\pgfsetbuttcap%
\pgfsetroundjoin%
\pgfsetlinewidth{1.003750pt}%
\definecolor{currentstroke}{rgb}{1.000000,0.000000,0.000000}%
\pgfsetstrokecolor{currentstroke}%
\pgfsetdash{}{0pt}%
\pgfpathmoveto{\pgfqpoint{4.825802in}{0.673411in}}%
\pgfpathcurveto{\pgfqpoint{4.836852in}{0.673411in}}{\pgfqpoint{4.847451in}{0.677801in}}{\pgfqpoint{4.855265in}{0.685615in}}%
\pgfpathcurveto{\pgfqpoint{4.863078in}{0.693428in}}{\pgfqpoint{4.867469in}{0.704027in}}{\pgfqpoint{4.867469in}{0.715077in}}%
\pgfpathcurveto{\pgfqpoint{4.867469in}{0.726127in}}{\pgfqpoint{4.863078in}{0.736727in}}{\pgfqpoint{4.855265in}{0.744540in}}%
\pgfpathcurveto{\pgfqpoint{4.847451in}{0.752354in}}{\pgfqpoint{4.836852in}{0.756744in}}{\pgfqpoint{4.825802in}{0.756744in}}%
\pgfpathcurveto{\pgfqpoint{4.814752in}{0.756744in}}{\pgfqpoint{4.804153in}{0.752354in}}{\pgfqpoint{4.796339in}{0.744540in}}%
\pgfpathcurveto{\pgfqpoint{4.788526in}{0.736727in}}{\pgfqpoint{4.784135in}{0.726127in}}{\pgfqpoint{4.784135in}{0.715077in}}%
\pgfpathcurveto{\pgfqpoint{4.784135in}{0.704027in}}{\pgfqpoint{4.788526in}{0.693428in}}{\pgfqpoint{4.796339in}{0.685615in}}%
\pgfpathcurveto{\pgfqpoint{4.804153in}{0.677801in}}{\pgfqpoint{4.814752in}{0.673411in}}{\pgfqpoint{4.825802in}{0.673411in}}%
\pgfpathlineto{\pgfqpoint{4.825802in}{0.673411in}}%
\pgfpathclose%
\pgfusepath{stroke}%
\end{pgfscope}%
\begin{pgfscope}%
\pgfpathrectangle{\pgfqpoint{0.847223in}{0.554012in}}{\pgfqpoint{6.200000in}{4.620000in}}%
\pgfusepath{clip}%
\pgfsetbuttcap%
\pgfsetroundjoin%
\pgfsetlinewidth{1.003750pt}%
\definecolor{currentstroke}{rgb}{1.000000,0.000000,0.000000}%
\pgfsetstrokecolor{currentstroke}%
\pgfsetdash{}{0pt}%
\pgfpathmoveto{\pgfqpoint{4.831135in}{0.672717in}}%
\pgfpathcurveto{\pgfqpoint{4.842185in}{0.672717in}}{\pgfqpoint{4.852784in}{0.677107in}}{\pgfqpoint{4.860598in}{0.684921in}}%
\pgfpathcurveto{\pgfqpoint{4.868412in}{0.692734in}}{\pgfqpoint{4.872802in}{0.703333in}}{\pgfqpoint{4.872802in}{0.714383in}}%
\pgfpathcurveto{\pgfqpoint{4.872802in}{0.725433in}}{\pgfqpoint{4.868412in}{0.736033in}}{\pgfqpoint{4.860598in}{0.743846in}}%
\pgfpathcurveto{\pgfqpoint{4.852784in}{0.751660in}}{\pgfqpoint{4.842185in}{0.756050in}}{\pgfqpoint{4.831135in}{0.756050in}}%
\pgfpathcurveto{\pgfqpoint{4.820085in}{0.756050in}}{\pgfqpoint{4.809486in}{0.751660in}}{\pgfqpoint{4.801672in}{0.743846in}}%
\pgfpathcurveto{\pgfqpoint{4.793859in}{0.736033in}}{\pgfqpoint{4.789468in}{0.725433in}}{\pgfqpoint{4.789468in}{0.714383in}}%
\pgfpathcurveto{\pgfqpoint{4.789468in}{0.703333in}}{\pgfqpoint{4.793859in}{0.692734in}}{\pgfqpoint{4.801672in}{0.684921in}}%
\pgfpathcurveto{\pgfqpoint{4.809486in}{0.677107in}}{\pgfqpoint{4.820085in}{0.672717in}}{\pgfqpoint{4.831135in}{0.672717in}}%
\pgfpathlineto{\pgfqpoint{4.831135in}{0.672717in}}%
\pgfpathclose%
\pgfusepath{stroke}%
\end{pgfscope}%
\begin{pgfscope}%
\pgfpathrectangle{\pgfqpoint{0.847223in}{0.554012in}}{\pgfqpoint{6.200000in}{4.620000in}}%
\pgfusepath{clip}%
\pgfsetbuttcap%
\pgfsetroundjoin%
\pgfsetlinewidth{1.003750pt}%
\definecolor{currentstroke}{rgb}{1.000000,0.000000,0.000000}%
\pgfsetstrokecolor{currentstroke}%
\pgfsetdash{}{0pt}%
\pgfpathmoveto{\pgfqpoint{4.836468in}{0.672024in}}%
\pgfpathcurveto{\pgfqpoint{4.847518in}{0.672024in}}{\pgfqpoint{4.858118in}{0.676415in}}{\pgfqpoint{4.865931in}{0.684228in}}%
\pgfpathcurveto{\pgfqpoint{4.873745in}{0.692042in}}{\pgfqpoint{4.878135in}{0.702641in}}{\pgfqpoint{4.878135in}{0.713691in}}%
\pgfpathcurveto{\pgfqpoint{4.878135in}{0.724741in}}{\pgfqpoint{4.873745in}{0.735340in}}{\pgfqpoint{4.865931in}{0.743154in}}%
\pgfpathcurveto{\pgfqpoint{4.858118in}{0.750967in}}{\pgfqpoint{4.847518in}{0.755358in}}{\pgfqpoint{4.836468in}{0.755358in}}%
\pgfpathcurveto{\pgfqpoint{4.825418in}{0.755358in}}{\pgfqpoint{4.814819in}{0.750967in}}{\pgfqpoint{4.807006in}{0.743154in}}%
\pgfpathcurveto{\pgfqpoint{4.799192in}{0.735340in}}{\pgfqpoint{4.794802in}{0.724741in}}{\pgfqpoint{4.794802in}{0.713691in}}%
\pgfpathcurveto{\pgfqpoint{4.794802in}{0.702641in}}{\pgfqpoint{4.799192in}{0.692042in}}{\pgfqpoint{4.807006in}{0.684228in}}%
\pgfpathcurveto{\pgfqpoint{4.814819in}{0.676415in}}{\pgfqpoint{4.825418in}{0.672024in}}{\pgfqpoint{4.836468in}{0.672024in}}%
\pgfpathlineto{\pgfqpoint{4.836468in}{0.672024in}}%
\pgfpathclose%
\pgfusepath{stroke}%
\end{pgfscope}%
\begin{pgfscope}%
\pgfpathrectangle{\pgfqpoint{0.847223in}{0.554012in}}{\pgfqpoint{6.200000in}{4.620000in}}%
\pgfusepath{clip}%
\pgfsetbuttcap%
\pgfsetroundjoin%
\pgfsetlinewidth{1.003750pt}%
\definecolor{currentstroke}{rgb}{1.000000,0.000000,0.000000}%
\pgfsetstrokecolor{currentstroke}%
\pgfsetdash{}{0pt}%
\pgfpathmoveto{\pgfqpoint{4.841802in}{0.671334in}}%
\pgfpathcurveto{\pgfqpoint{4.852852in}{0.671334in}}{\pgfqpoint{4.863451in}{0.675724in}}{\pgfqpoint{4.871264in}{0.683537in}}%
\pgfpathcurveto{\pgfqpoint{4.879078in}{0.691351in}}{\pgfqpoint{4.883468in}{0.701950in}}{\pgfqpoint{4.883468in}{0.713000in}}%
\pgfpathcurveto{\pgfqpoint{4.883468in}{0.724050in}}{\pgfqpoint{4.879078in}{0.734649in}}{\pgfqpoint{4.871264in}{0.742463in}}%
\pgfpathcurveto{\pgfqpoint{4.863451in}{0.750277in}}{\pgfqpoint{4.852852in}{0.754667in}}{\pgfqpoint{4.841802in}{0.754667in}}%
\pgfpathcurveto{\pgfqpoint{4.830751in}{0.754667in}}{\pgfqpoint{4.820152in}{0.750277in}}{\pgfqpoint{4.812339in}{0.742463in}}%
\pgfpathcurveto{\pgfqpoint{4.804525in}{0.734649in}}{\pgfqpoint{4.800135in}{0.724050in}}{\pgfqpoint{4.800135in}{0.713000in}}%
\pgfpathcurveto{\pgfqpoint{4.800135in}{0.701950in}}{\pgfqpoint{4.804525in}{0.691351in}}{\pgfqpoint{4.812339in}{0.683537in}}%
\pgfpathcurveto{\pgfqpoint{4.820152in}{0.675724in}}{\pgfqpoint{4.830751in}{0.671334in}}{\pgfqpoint{4.841802in}{0.671334in}}%
\pgfpathlineto{\pgfqpoint{4.841802in}{0.671334in}}%
\pgfpathclose%
\pgfusepath{stroke}%
\end{pgfscope}%
\begin{pgfscope}%
\pgfpathrectangle{\pgfqpoint{0.847223in}{0.554012in}}{\pgfqpoint{6.200000in}{4.620000in}}%
\pgfusepath{clip}%
\pgfsetbuttcap%
\pgfsetroundjoin%
\pgfsetlinewidth{1.003750pt}%
\definecolor{currentstroke}{rgb}{1.000000,0.000000,0.000000}%
\pgfsetstrokecolor{currentstroke}%
\pgfsetdash{}{0pt}%
\pgfpathmoveto{\pgfqpoint{4.847135in}{0.670644in}}%
\pgfpathcurveto{\pgfqpoint{4.858185in}{0.670644in}}{\pgfqpoint{4.868784in}{0.675035in}}{\pgfqpoint{4.876598in}{0.682848in}}%
\pgfpathcurveto{\pgfqpoint{4.884411in}{0.690662in}}{\pgfqpoint{4.888801in}{0.701261in}}{\pgfqpoint{4.888801in}{0.712311in}}%
\pgfpathcurveto{\pgfqpoint{4.888801in}{0.723361in}}{\pgfqpoint{4.884411in}{0.733960in}}{\pgfqpoint{4.876598in}{0.741774in}}%
\pgfpathcurveto{\pgfqpoint{4.868784in}{0.749588in}}{\pgfqpoint{4.858185in}{0.753978in}}{\pgfqpoint{4.847135in}{0.753978in}}%
\pgfpathcurveto{\pgfqpoint{4.836085in}{0.753978in}}{\pgfqpoint{4.825486in}{0.749588in}}{\pgfqpoint{4.817672in}{0.741774in}}%
\pgfpathcurveto{\pgfqpoint{4.809858in}{0.733960in}}{\pgfqpoint{4.805468in}{0.723361in}}{\pgfqpoint{4.805468in}{0.712311in}}%
\pgfpathcurveto{\pgfqpoint{4.805468in}{0.701261in}}{\pgfqpoint{4.809858in}{0.690662in}}{\pgfqpoint{4.817672in}{0.682848in}}%
\pgfpathcurveto{\pgfqpoint{4.825486in}{0.675035in}}{\pgfqpoint{4.836085in}{0.670644in}}{\pgfqpoint{4.847135in}{0.670644in}}%
\pgfpathlineto{\pgfqpoint{4.847135in}{0.670644in}}%
\pgfpathclose%
\pgfusepath{stroke}%
\end{pgfscope}%
\begin{pgfscope}%
\pgfpathrectangle{\pgfqpoint{0.847223in}{0.554012in}}{\pgfqpoint{6.200000in}{4.620000in}}%
\pgfusepath{clip}%
\pgfsetbuttcap%
\pgfsetroundjoin%
\pgfsetlinewidth{1.003750pt}%
\definecolor{currentstroke}{rgb}{1.000000,0.000000,0.000000}%
\pgfsetstrokecolor{currentstroke}%
\pgfsetdash{}{0pt}%
\pgfpathmoveto{\pgfqpoint{4.852468in}{0.669957in}}%
\pgfpathcurveto{\pgfqpoint{4.863518in}{0.669957in}}{\pgfqpoint{4.874117in}{0.674347in}}{\pgfqpoint{4.881931in}{0.682161in}}%
\pgfpathcurveto{\pgfqpoint{4.889744in}{0.689974in}}{\pgfqpoint{4.894135in}{0.700573in}}{\pgfqpoint{4.894135in}{0.711624in}}%
\pgfpathcurveto{\pgfqpoint{4.894135in}{0.722674in}}{\pgfqpoint{4.889744in}{0.733273in}}{\pgfqpoint{4.881931in}{0.741086in}}%
\pgfpathcurveto{\pgfqpoint{4.874117in}{0.748900in}}{\pgfqpoint{4.863518in}{0.753290in}}{\pgfqpoint{4.852468in}{0.753290in}}%
\pgfpathcurveto{\pgfqpoint{4.841418in}{0.753290in}}{\pgfqpoint{4.830819in}{0.748900in}}{\pgfqpoint{4.823005in}{0.741086in}}%
\pgfpathcurveto{\pgfqpoint{4.815192in}{0.733273in}}{\pgfqpoint{4.810801in}{0.722674in}}{\pgfqpoint{4.810801in}{0.711624in}}%
\pgfpathcurveto{\pgfqpoint{4.810801in}{0.700573in}}{\pgfqpoint{4.815192in}{0.689974in}}{\pgfqpoint{4.823005in}{0.682161in}}%
\pgfpathcurveto{\pgfqpoint{4.830819in}{0.674347in}}{\pgfqpoint{4.841418in}{0.669957in}}{\pgfqpoint{4.852468in}{0.669957in}}%
\pgfpathlineto{\pgfqpoint{4.852468in}{0.669957in}}%
\pgfpathclose%
\pgfusepath{stroke}%
\end{pgfscope}%
\begin{pgfscope}%
\pgfpathrectangle{\pgfqpoint{0.847223in}{0.554012in}}{\pgfqpoint{6.200000in}{4.620000in}}%
\pgfusepath{clip}%
\pgfsetbuttcap%
\pgfsetroundjoin%
\pgfsetlinewidth{1.003750pt}%
\definecolor{currentstroke}{rgb}{1.000000,0.000000,0.000000}%
\pgfsetstrokecolor{currentstroke}%
\pgfsetdash{}{0pt}%
\pgfpathmoveto{\pgfqpoint{4.857801in}{0.669271in}}%
\pgfpathcurveto{\pgfqpoint{4.868851in}{0.669271in}}{\pgfqpoint{4.879450in}{0.673661in}}{\pgfqpoint{4.887264in}{0.681475in}}%
\pgfpathcurveto{\pgfqpoint{4.895078in}{0.689289in}}{\pgfqpoint{4.899468in}{0.699888in}}{\pgfqpoint{4.899468in}{0.710938in}}%
\pgfpathcurveto{\pgfqpoint{4.899468in}{0.721988in}}{\pgfqpoint{4.895078in}{0.732587in}}{\pgfqpoint{4.887264in}{0.740401in}}%
\pgfpathcurveto{\pgfqpoint{4.879450in}{0.748214in}}{\pgfqpoint{4.868851in}{0.752604in}}{\pgfqpoint{4.857801in}{0.752604in}}%
\pgfpathcurveto{\pgfqpoint{4.846751in}{0.752604in}}{\pgfqpoint{4.836152in}{0.748214in}}{\pgfqpoint{4.828338in}{0.740401in}}%
\pgfpathcurveto{\pgfqpoint{4.820525in}{0.732587in}}{\pgfqpoint{4.816135in}{0.721988in}}{\pgfqpoint{4.816135in}{0.710938in}}%
\pgfpathcurveto{\pgfqpoint{4.816135in}{0.699888in}}{\pgfqpoint{4.820525in}{0.689289in}}{\pgfqpoint{4.828338in}{0.681475in}}%
\pgfpathcurveto{\pgfqpoint{4.836152in}{0.673661in}}{\pgfqpoint{4.846751in}{0.669271in}}{\pgfqpoint{4.857801in}{0.669271in}}%
\pgfpathlineto{\pgfqpoint{4.857801in}{0.669271in}}%
\pgfpathclose%
\pgfusepath{stroke}%
\end{pgfscope}%
\begin{pgfscope}%
\pgfpathrectangle{\pgfqpoint{0.847223in}{0.554012in}}{\pgfqpoint{6.200000in}{4.620000in}}%
\pgfusepath{clip}%
\pgfsetbuttcap%
\pgfsetroundjoin%
\pgfsetlinewidth{1.003750pt}%
\definecolor{currentstroke}{rgb}{1.000000,0.000000,0.000000}%
\pgfsetstrokecolor{currentstroke}%
\pgfsetdash{}{0pt}%
\pgfpathmoveto{\pgfqpoint{4.863134in}{0.668587in}}%
\pgfpathcurveto{\pgfqpoint{4.874185in}{0.668587in}}{\pgfqpoint{4.884784in}{0.672977in}}{\pgfqpoint{4.892597in}{0.680791in}}%
\pgfpathcurveto{\pgfqpoint{4.900411in}{0.688604in}}{\pgfqpoint{4.904801in}{0.699203in}}{\pgfqpoint{4.904801in}{0.710253in}}%
\pgfpathcurveto{\pgfqpoint{4.904801in}{0.721304in}}{\pgfqpoint{4.900411in}{0.731903in}}{\pgfqpoint{4.892597in}{0.739716in}}%
\pgfpathcurveto{\pgfqpoint{4.884784in}{0.747530in}}{\pgfqpoint{4.874185in}{0.751920in}}{\pgfqpoint{4.863134in}{0.751920in}}%
\pgfpathcurveto{\pgfqpoint{4.852084in}{0.751920in}}{\pgfqpoint{4.841485in}{0.747530in}}{\pgfqpoint{4.833672in}{0.739716in}}%
\pgfpathcurveto{\pgfqpoint{4.825858in}{0.731903in}}{\pgfqpoint{4.821468in}{0.721304in}}{\pgfqpoint{4.821468in}{0.710253in}}%
\pgfpathcurveto{\pgfqpoint{4.821468in}{0.699203in}}{\pgfqpoint{4.825858in}{0.688604in}}{\pgfqpoint{4.833672in}{0.680791in}}%
\pgfpathcurveto{\pgfqpoint{4.841485in}{0.672977in}}{\pgfqpoint{4.852084in}{0.668587in}}{\pgfqpoint{4.863134in}{0.668587in}}%
\pgfpathlineto{\pgfqpoint{4.863134in}{0.668587in}}%
\pgfpathclose%
\pgfusepath{stroke}%
\end{pgfscope}%
\begin{pgfscope}%
\pgfpathrectangle{\pgfqpoint{0.847223in}{0.554012in}}{\pgfqpoint{6.200000in}{4.620000in}}%
\pgfusepath{clip}%
\pgfsetbuttcap%
\pgfsetroundjoin%
\pgfsetlinewidth{1.003750pt}%
\definecolor{currentstroke}{rgb}{1.000000,0.000000,0.000000}%
\pgfsetstrokecolor{currentstroke}%
\pgfsetdash{}{0pt}%
\pgfpathmoveto{\pgfqpoint{4.868468in}{0.667904in}}%
\pgfpathcurveto{\pgfqpoint{4.879518in}{0.667904in}}{\pgfqpoint{4.890117in}{0.672294in}}{\pgfqpoint{4.897930in}{0.680108in}}%
\pgfpathcurveto{\pgfqpoint{4.905744in}{0.687922in}}{\pgfqpoint{4.910134in}{0.698521in}}{\pgfqpoint{4.910134in}{0.709571in}}%
\pgfpathcurveto{\pgfqpoint{4.910134in}{0.720621in}}{\pgfqpoint{4.905744in}{0.731220in}}{\pgfqpoint{4.897930in}{0.739034in}}%
\pgfpathcurveto{\pgfqpoint{4.890117in}{0.746847in}}{\pgfqpoint{4.879518in}{0.751237in}}{\pgfqpoint{4.868468in}{0.751237in}}%
\pgfpathcurveto{\pgfqpoint{4.857418in}{0.751237in}}{\pgfqpoint{4.846819in}{0.746847in}}{\pgfqpoint{4.839005in}{0.739034in}}%
\pgfpathcurveto{\pgfqpoint{4.831191in}{0.731220in}}{\pgfqpoint{4.826801in}{0.720621in}}{\pgfqpoint{4.826801in}{0.709571in}}%
\pgfpathcurveto{\pgfqpoint{4.826801in}{0.698521in}}{\pgfqpoint{4.831191in}{0.687922in}}{\pgfqpoint{4.839005in}{0.680108in}}%
\pgfpathcurveto{\pgfqpoint{4.846819in}{0.672294in}}{\pgfqpoint{4.857418in}{0.667904in}}{\pgfqpoint{4.868468in}{0.667904in}}%
\pgfpathlineto{\pgfqpoint{4.868468in}{0.667904in}}%
\pgfpathclose%
\pgfusepath{stroke}%
\end{pgfscope}%
\begin{pgfscope}%
\pgfpathrectangle{\pgfqpoint{0.847223in}{0.554012in}}{\pgfqpoint{6.200000in}{4.620000in}}%
\pgfusepath{clip}%
\pgfsetbuttcap%
\pgfsetroundjoin%
\pgfsetlinewidth{1.003750pt}%
\definecolor{currentstroke}{rgb}{1.000000,0.000000,0.000000}%
\pgfsetstrokecolor{currentstroke}%
\pgfsetdash{}{0pt}%
\pgfpathmoveto{\pgfqpoint{4.873801in}{0.667223in}}%
\pgfpathcurveto{\pgfqpoint{4.884851in}{0.667223in}}{\pgfqpoint{4.895450in}{0.671613in}}{\pgfqpoint{4.903264in}{0.679427in}}%
\pgfpathcurveto{\pgfqpoint{4.911077in}{0.687241in}}{\pgfqpoint{4.915468in}{0.697840in}}{\pgfqpoint{4.915468in}{0.708890in}}%
\pgfpathcurveto{\pgfqpoint{4.915468in}{0.719940in}}{\pgfqpoint{4.911077in}{0.730539in}}{\pgfqpoint{4.903264in}{0.738353in}}%
\pgfpathcurveto{\pgfqpoint{4.895450in}{0.746166in}}{\pgfqpoint{4.884851in}{0.750556in}}{\pgfqpoint{4.873801in}{0.750556in}}%
\pgfpathcurveto{\pgfqpoint{4.862751in}{0.750556in}}{\pgfqpoint{4.852152in}{0.746166in}}{\pgfqpoint{4.844338in}{0.738353in}}%
\pgfpathcurveto{\pgfqpoint{4.836524in}{0.730539in}}{\pgfqpoint{4.832134in}{0.719940in}}{\pgfqpoint{4.832134in}{0.708890in}}%
\pgfpathcurveto{\pgfqpoint{4.832134in}{0.697840in}}{\pgfqpoint{4.836524in}{0.687241in}}{\pgfqpoint{4.844338in}{0.679427in}}%
\pgfpathcurveto{\pgfqpoint{4.852152in}{0.671613in}}{\pgfqpoint{4.862751in}{0.667223in}}{\pgfqpoint{4.873801in}{0.667223in}}%
\pgfpathlineto{\pgfqpoint{4.873801in}{0.667223in}}%
\pgfpathclose%
\pgfusepath{stroke}%
\end{pgfscope}%
\begin{pgfscope}%
\pgfpathrectangle{\pgfqpoint{0.847223in}{0.554012in}}{\pgfqpoint{6.200000in}{4.620000in}}%
\pgfusepath{clip}%
\pgfsetbuttcap%
\pgfsetroundjoin%
\pgfsetlinewidth{1.003750pt}%
\definecolor{currentstroke}{rgb}{1.000000,0.000000,0.000000}%
\pgfsetstrokecolor{currentstroke}%
\pgfsetdash{}{0pt}%
\pgfpathmoveto{\pgfqpoint{4.879134in}{0.666544in}}%
\pgfpathcurveto{\pgfqpoint{4.890184in}{0.666544in}}{\pgfqpoint{4.900783in}{0.670934in}}{\pgfqpoint{4.908597in}{0.678748in}}%
\pgfpathcurveto{\pgfqpoint{4.916410in}{0.686561in}}{\pgfqpoint{4.920801in}{0.697160in}}{\pgfqpoint{4.920801in}{0.708210in}}%
\pgfpathcurveto{\pgfqpoint{4.920801in}{0.719260in}}{\pgfqpoint{4.916410in}{0.729859in}}{\pgfqpoint{4.908597in}{0.737673in}}%
\pgfpathcurveto{\pgfqpoint{4.900783in}{0.745487in}}{\pgfqpoint{4.890184in}{0.749877in}}{\pgfqpoint{4.879134in}{0.749877in}}%
\pgfpathcurveto{\pgfqpoint{4.868084in}{0.749877in}}{\pgfqpoint{4.857485in}{0.745487in}}{\pgfqpoint{4.849671in}{0.737673in}}%
\pgfpathcurveto{\pgfqpoint{4.841858in}{0.729859in}}{\pgfqpoint{4.837467in}{0.719260in}}{\pgfqpoint{4.837467in}{0.708210in}}%
\pgfpathcurveto{\pgfqpoint{4.837467in}{0.697160in}}{\pgfqpoint{4.841858in}{0.686561in}}{\pgfqpoint{4.849671in}{0.678748in}}%
\pgfpathcurveto{\pgfqpoint{4.857485in}{0.670934in}}{\pgfqpoint{4.868084in}{0.666544in}}{\pgfqpoint{4.879134in}{0.666544in}}%
\pgfpathlineto{\pgfqpoint{4.879134in}{0.666544in}}%
\pgfpathclose%
\pgfusepath{stroke}%
\end{pgfscope}%
\begin{pgfscope}%
\pgfpathrectangle{\pgfqpoint{0.847223in}{0.554012in}}{\pgfqpoint{6.200000in}{4.620000in}}%
\pgfusepath{clip}%
\pgfsetbuttcap%
\pgfsetroundjoin%
\pgfsetlinewidth{1.003750pt}%
\definecolor{currentstroke}{rgb}{1.000000,0.000000,0.000000}%
\pgfsetstrokecolor{currentstroke}%
\pgfsetdash{}{0pt}%
\pgfpathmoveto{\pgfqpoint{4.884467in}{0.665866in}}%
\pgfpathcurveto{\pgfqpoint{4.895517in}{0.665866in}}{\pgfqpoint{4.906116in}{0.670256in}}{\pgfqpoint{4.913930in}{0.678070in}}%
\pgfpathcurveto{\pgfqpoint{4.921744in}{0.685883in}}{\pgfqpoint{4.926134in}{0.696482in}}{\pgfqpoint{4.926134in}{0.707532in}}%
\pgfpathcurveto{\pgfqpoint{4.926134in}{0.718583in}}{\pgfqpoint{4.921744in}{0.729182in}}{\pgfqpoint{4.913930in}{0.736995in}}%
\pgfpathcurveto{\pgfqpoint{4.906116in}{0.744809in}}{\pgfqpoint{4.895517in}{0.749199in}}{\pgfqpoint{4.884467in}{0.749199in}}%
\pgfpathcurveto{\pgfqpoint{4.873417in}{0.749199in}}{\pgfqpoint{4.862818in}{0.744809in}}{\pgfqpoint{4.855005in}{0.736995in}}%
\pgfpathcurveto{\pgfqpoint{4.847191in}{0.729182in}}{\pgfqpoint{4.842801in}{0.718583in}}{\pgfqpoint{4.842801in}{0.707532in}}%
\pgfpathcurveto{\pgfqpoint{4.842801in}{0.696482in}}{\pgfqpoint{4.847191in}{0.685883in}}{\pgfqpoint{4.855005in}{0.678070in}}%
\pgfpathcurveto{\pgfqpoint{4.862818in}{0.670256in}}{\pgfqpoint{4.873417in}{0.665866in}}{\pgfqpoint{4.884467in}{0.665866in}}%
\pgfpathlineto{\pgfqpoint{4.884467in}{0.665866in}}%
\pgfpathclose%
\pgfusepath{stroke}%
\end{pgfscope}%
\begin{pgfscope}%
\pgfpathrectangle{\pgfqpoint{0.847223in}{0.554012in}}{\pgfqpoint{6.200000in}{4.620000in}}%
\pgfusepath{clip}%
\pgfsetbuttcap%
\pgfsetroundjoin%
\pgfsetlinewidth{1.003750pt}%
\definecolor{currentstroke}{rgb}{1.000000,0.000000,0.000000}%
\pgfsetstrokecolor{currentstroke}%
\pgfsetdash{}{0pt}%
\pgfpathmoveto{\pgfqpoint{4.889801in}{0.665189in}}%
\pgfpathcurveto{\pgfqpoint{4.900851in}{0.665189in}}{\pgfqpoint{4.911450in}{0.669580in}}{\pgfqpoint{4.919263in}{0.677393in}}%
\pgfpathcurveto{\pgfqpoint{4.927077in}{0.685207in}}{\pgfqpoint{4.931467in}{0.695806in}}{\pgfqpoint{4.931467in}{0.706856in}}%
\pgfpathcurveto{\pgfqpoint{4.931467in}{0.717906in}}{\pgfqpoint{4.927077in}{0.728505in}}{\pgfqpoint{4.919263in}{0.736319in}}%
\pgfpathcurveto{\pgfqpoint{4.911450in}{0.744133in}}{\pgfqpoint{4.900851in}{0.748523in}}{\pgfqpoint{4.889801in}{0.748523in}}%
\pgfpathcurveto{\pgfqpoint{4.878750in}{0.748523in}}{\pgfqpoint{4.868151in}{0.744133in}}{\pgfqpoint{4.860338in}{0.736319in}}%
\pgfpathcurveto{\pgfqpoint{4.852524in}{0.728505in}}{\pgfqpoint{4.848134in}{0.717906in}}{\pgfqpoint{4.848134in}{0.706856in}}%
\pgfpathcurveto{\pgfqpoint{4.848134in}{0.695806in}}{\pgfqpoint{4.852524in}{0.685207in}}{\pgfqpoint{4.860338in}{0.677393in}}%
\pgfpathcurveto{\pgfqpoint{4.868151in}{0.669580in}}{\pgfqpoint{4.878750in}{0.665189in}}{\pgfqpoint{4.889801in}{0.665189in}}%
\pgfpathlineto{\pgfqpoint{4.889801in}{0.665189in}}%
\pgfpathclose%
\pgfusepath{stroke}%
\end{pgfscope}%
\begin{pgfscope}%
\pgfpathrectangle{\pgfqpoint{0.847223in}{0.554012in}}{\pgfqpoint{6.200000in}{4.620000in}}%
\pgfusepath{clip}%
\pgfsetbuttcap%
\pgfsetroundjoin%
\pgfsetlinewidth{1.003750pt}%
\definecolor{currentstroke}{rgb}{1.000000,0.000000,0.000000}%
\pgfsetstrokecolor{currentstroke}%
\pgfsetdash{}{0pt}%
\pgfpathmoveto{\pgfqpoint{4.895134in}{0.664515in}}%
\pgfpathcurveto{\pgfqpoint{4.906184in}{0.664515in}}{\pgfqpoint{4.916783in}{0.668905in}}{\pgfqpoint{4.924597in}{0.676719in}}%
\pgfpathcurveto{\pgfqpoint{4.932410in}{0.684532in}}{\pgfqpoint{4.936800in}{0.695131in}}{\pgfqpoint{4.936800in}{0.706181in}}%
\pgfpathcurveto{\pgfqpoint{4.936800in}{0.717232in}}{\pgfqpoint{4.932410in}{0.727831in}}{\pgfqpoint{4.924597in}{0.735644in}}%
\pgfpathcurveto{\pgfqpoint{4.916783in}{0.743458in}}{\pgfqpoint{4.906184in}{0.747848in}}{\pgfqpoint{4.895134in}{0.747848in}}%
\pgfpathcurveto{\pgfqpoint{4.884084in}{0.747848in}}{\pgfqpoint{4.873485in}{0.743458in}}{\pgfqpoint{4.865671in}{0.735644in}}%
\pgfpathcurveto{\pgfqpoint{4.857857in}{0.727831in}}{\pgfqpoint{4.853467in}{0.717232in}}{\pgfqpoint{4.853467in}{0.706181in}}%
\pgfpathcurveto{\pgfqpoint{4.853467in}{0.695131in}}{\pgfqpoint{4.857857in}{0.684532in}}{\pgfqpoint{4.865671in}{0.676719in}}%
\pgfpathcurveto{\pgfqpoint{4.873485in}{0.668905in}}{\pgfqpoint{4.884084in}{0.664515in}}{\pgfqpoint{4.895134in}{0.664515in}}%
\pgfpathlineto{\pgfqpoint{4.895134in}{0.664515in}}%
\pgfpathclose%
\pgfusepath{stroke}%
\end{pgfscope}%
\begin{pgfscope}%
\pgfpathrectangle{\pgfqpoint{0.847223in}{0.554012in}}{\pgfqpoint{6.200000in}{4.620000in}}%
\pgfusepath{clip}%
\pgfsetbuttcap%
\pgfsetroundjoin%
\pgfsetlinewidth{1.003750pt}%
\definecolor{currentstroke}{rgb}{1.000000,0.000000,0.000000}%
\pgfsetstrokecolor{currentstroke}%
\pgfsetdash{}{0pt}%
\pgfpathmoveto{\pgfqpoint{4.900467in}{0.663842in}}%
\pgfpathcurveto{\pgfqpoint{4.911517in}{0.663842in}}{\pgfqpoint{4.922116in}{0.668232in}}{\pgfqpoint{4.929930in}{0.676046in}}%
\pgfpathcurveto{\pgfqpoint{4.937743in}{0.683859in}}{\pgfqpoint{4.942134in}{0.694458in}}{\pgfqpoint{4.942134in}{0.705508in}}%
\pgfpathcurveto{\pgfqpoint{4.942134in}{0.716558in}}{\pgfqpoint{4.937743in}{0.727157in}}{\pgfqpoint{4.929930in}{0.734971in}}%
\pgfpathcurveto{\pgfqpoint{4.922116in}{0.742785in}}{\pgfqpoint{4.911517in}{0.747175in}}{\pgfqpoint{4.900467in}{0.747175in}}%
\pgfpathcurveto{\pgfqpoint{4.889417in}{0.747175in}}{\pgfqpoint{4.878818in}{0.742785in}}{\pgfqpoint{4.871004in}{0.734971in}}%
\pgfpathcurveto{\pgfqpoint{4.863191in}{0.727157in}}{\pgfqpoint{4.858800in}{0.716558in}}{\pgfqpoint{4.858800in}{0.705508in}}%
\pgfpathcurveto{\pgfqpoint{4.858800in}{0.694458in}}{\pgfqpoint{4.863191in}{0.683859in}}{\pgfqpoint{4.871004in}{0.676046in}}%
\pgfpathcurveto{\pgfqpoint{4.878818in}{0.668232in}}{\pgfqpoint{4.889417in}{0.663842in}}{\pgfqpoint{4.900467in}{0.663842in}}%
\pgfpathlineto{\pgfqpoint{4.900467in}{0.663842in}}%
\pgfpathclose%
\pgfusepath{stroke}%
\end{pgfscope}%
\begin{pgfscope}%
\pgfpathrectangle{\pgfqpoint{0.847223in}{0.554012in}}{\pgfqpoint{6.200000in}{4.620000in}}%
\pgfusepath{clip}%
\pgfsetbuttcap%
\pgfsetroundjoin%
\pgfsetlinewidth{1.003750pt}%
\definecolor{currentstroke}{rgb}{1.000000,0.000000,0.000000}%
\pgfsetstrokecolor{currentstroke}%
\pgfsetdash{}{0pt}%
\pgfpathmoveto{\pgfqpoint{4.905800in}{0.663170in}}%
\pgfpathcurveto{\pgfqpoint{4.916850in}{0.663170in}}{\pgfqpoint{4.927449in}{0.667560in}}{\pgfqpoint{4.935263in}{0.675374in}}%
\pgfpathcurveto{\pgfqpoint{4.943077in}{0.683188in}}{\pgfqpoint{4.947467in}{0.693787in}}{\pgfqpoint{4.947467in}{0.704837in}}%
\pgfpathcurveto{\pgfqpoint{4.947467in}{0.715887in}}{\pgfqpoint{4.943077in}{0.726486in}}{\pgfqpoint{4.935263in}{0.734299in}}%
\pgfpathcurveto{\pgfqpoint{4.927449in}{0.742113in}}{\pgfqpoint{4.916850in}{0.746503in}}{\pgfqpoint{4.905800in}{0.746503in}}%
\pgfpathcurveto{\pgfqpoint{4.894750in}{0.746503in}}{\pgfqpoint{4.884151in}{0.742113in}}{\pgfqpoint{4.876337in}{0.734299in}}%
\pgfpathcurveto{\pgfqpoint{4.868524in}{0.726486in}}{\pgfqpoint{4.864134in}{0.715887in}}{\pgfqpoint{4.864134in}{0.704837in}}%
\pgfpathcurveto{\pgfqpoint{4.864134in}{0.693787in}}{\pgfqpoint{4.868524in}{0.683188in}}{\pgfqpoint{4.876337in}{0.675374in}}%
\pgfpathcurveto{\pgfqpoint{4.884151in}{0.667560in}}{\pgfqpoint{4.894750in}{0.663170in}}{\pgfqpoint{4.905800in}{0.663170in}}%
\pgfpathlineto{\pgfqpoint{4.905800in}{0.663170in}}%
\pgfpathclose%
\pgfusepath{stroke}%
\end{pgfscope}%
\begin{pgfscope}%
\pgfpathrectangle{\pgfqpoint{0.847223in}{0.554012in}}{\pgfqpoint{6.200000in}{4.620000in}}%
\pgfusepath{clip}%
\pgfsetbuttcap%
\pgfsetroundjoin%
\pgfsetlinewidth{1.003750pt}%
\definecolor{currentstroke}{rgb}{1.000000,0.000000,0.000000}%
\pgfsetstrokecolor{currentstroke}%
\pgfsetdash{}{0pt}%
\pgfpathmoveto{\pgfqpoint{4.911133in}{0.662500in}}%
\pgfpathcurveto{\pgfqpoint{4.922184in}{0.662500in}}{\pgfqpoint{4.932783in}{0.666890in}}{\pgfqpoint{4.940596in}{0.674704in}}%
\pgfpathcurveto{\pgfqpoint{4.948410in}{0.682518in}}{\pgfqpoint{4.952800in}{0.693117in}}{\pgfqpoint{4.952800in}{0.704167in}}%
\pgfpathcurveto{\pgfqpoint{4.952800in}{0.715217in}}{\pgfqpoint{4.948410in}{0.725816in}}{\pgfqpoint{4.940596in}{0.733629in}}%
\pgfpathcurveto{\pgfqpoint{4.932783in}{0.741443in}}{\pgfqpoint{4.922184in}{0.745833in}}{\pgfqpoint{4.911133in}{0.745833in}}%
\pgfpathcurveto{\pgfqpoint{4.900083in}{0.745833in}}{\pgfqpoint{4.889484in}{0.741443in}}{\pgfqpoint{4.881671in}{0.733629in}}%
\pgfpathcurveto{\pgfqpoint{4.873857in}{0.725816in}}{\pgfqpoint{4.869467in}{0.715217in}}{\pgfqpoint{4.869467in}{0.704167in}}%
\pgfpathcurveto{\pgfqpoint{4.869467in}{0.693117in}}{\pgfqpoint{4.873857in}{0.682518in}}{\pgfqpoint{4.881671in}{0.674704in}}%
\pgfpathcurveto{\pgfqpoint{4.889484in}{0.666890in}}{\pgfqpoint{4.900083in}{0.662500in}}{\pgfqpoint{4.911133in}{0.662500in}}%
\pgfpathlineto{\pgfqpoint{4.911133in}{0.662500in}}%
\pgfpathclose%
\pgfusepath{stroke}%
\end{pgfscope}%
\begin{pgfscope}%
\pgfpathrectangle{\pgfqpoint{0.847223in}{0.554012in}}{\pgfqpoint{6.200000in}{4.620000in}}%
\pgfusepath{clip}%
\pgfsetbuttcap%
\pgfsetroundjoin%
\pgfsetlinewidth{1.003750pt}%
\definecolor{currentstroke}{rgb}{1.000000,0.000000,0.000000}%
\pgfsetstrokecolor{currentstroke}%
\pgfsetdash{}{0pt}%
\pgfpathmoveto{\pgfqpoint{4.916467in}{0.661832in}}%
\pgfpathcurveto{\pgfqpoint{4.927517in}{0.661832in}}{\pgfqpoint{4.938116in}{0.666222in}}{\pgfqpoint{4.945929in}{0.674035in}}%
\pgfpathcurveto{\pgfqpoint{4.953743in}{0.681849in}}{\pgfqpoint{4.958133in}{0.692448in}}{\pgfqpoint{4.958133in}{0.703498in}}%
\pgfpathcurveto{\pgfqpoint{4.958133in}{0.714548in}}{\pgfqpoint{4.953743in}{0.725147in}}{\pgfqpoint{4.945929in}{0.732961in}}%
\pgfpathcurveto{\pgfqpoint{4.938116in}{0.740775in}}{\pgfqpoint{4.927517in}{0.745165in}}{\pgfqpoint{4.916467in}{0.745165in}}%
\pgfpathcurveto{\pgfqpoint{4.905416in}{0.745165in}}{\pgfqpoint{4.894817in}{0.740775in}}{\pgfqpoint{4.887004in}{0.732961in}}%
\pgfpathcurveto{\pgfqpoint{4.879190in}{0.725147in}}{\pgfqpoint{4.874800in}{0.714548in}}{\pgfqpoint{4.874800in}{0.703498in}}%
\pgfpathcurveto{\pgfqpoint{4.874800in}{0.692448in}}{\pgfqpoint{4.879190in}{0.681849in}}{\pgfqpoint{4.887004in}{0.674035in}}%
\pgfpathcurveto{\pgfqpoint{4.894817in}{0.666222in}}{\pgfqpoint{4.905416in}{0.661832in}}{\pgfqpoint{4.916467in}{0.661832in}}%
\pgfpathlineto{\pgfqpoint{4.916467in}{0.661832in}}%
\pgfpathclose%
\pgfusepath{stroke}%
\end{pgfscope}%
\begin{pgfscope}%
\pgfpathrectangle{\pgfqpoint{0.847223in}{0.554012in}}{\pgfqpoint{6.200000in}{4.620000in}}%
\pgfusepath{clip}%
\pgfsetbuttcap%
\pgfsetroundjoin%
\pgfsetlinewidth{1.003750pt}%
\definecolor{currentstroke}{rgb}{1.000000,0.000000,0.000000}%
\pgfsetstrokecolor{currentstroke}%
\pgfsetdash{}{0pt}%
\pgfpathmoveto{\pgfqpoint{4.921800in}{0.661165in}}%
\pgfpathcurveto{\pgfqpoint{4.932850in}{0.661165in}}{\pgfqpoint{4.943449in}{0.665555in}}{\pgfqpoint{4.951263in}{0.673369in}}%
\pgfpathcurveto{\pgfqpoint{4.959076in}{0.681182in}}{\pgfqpoint{4.963466in}{0.691781in}}{\pgfqpoint{4.963466in}{0.702831in}}%
\pgfpathcurveto{\pgfqpoint{4.963466in}{0.713881in}}{\pgfqpoint{4.959076in}{0.724481in}}{\pgfqpoint{4.951263in}{0.732294in}}%
\pgfpathcurveto{\pgfqpoint{4.943449in}{0.740108in}}{\pgfqpoint{4.932850in}{0.744498in}}{\pgfqpoint{4.921800in}{0.744498in}}%
\pgfpathcurveto{\pgfqpoint{4.910750in}{0.744498in}}{\pgfqpoint{4.900151in}{0.740108in}}{\pgfqpoint{4.892337in}{0.732294in}}%
\pgfpathcurveto{\pgfqpoint{4.884523in}{0.724481in}}{\pgfqpoint{4.880133in}{0.713881in}}{\pgfqpoint{4.880133in}{0.702831in}}%
\pgfpathcurveto{\pgfqpoint{4.880133in}{0.691781in}}{\pgfqpoint{4.884523in}{0.681182in}}{\pgfqpoint{4.892337in}{0.673369in}}%
\pgfpathcurveto{\pgfqpoint{4.900151in}{0.665555in}}{\pgfqpoint{4.910750in}{0.661165in}}{\pgfqpoint{4.921800in}{0.661165in}}%
\pgfpathlineto{\pgfqpoint{4.921800in}{0.661165in}}%
\pgfpathclose%
\pgfusepath{stroke}%
\end{pgfscope}%
\begin{pgfscope}%
\pgfpathrectangle{\pgfqpoint{0.847223in}{0.554012in}}{\pgfqpoint{6.200000in}{4.620000in}}%
\pgfusepath{clip}%
\pgfsetbuttcap%
\pgfsetroundjoin%
\pgfsetlinewidth{1.003750pt}%
\definecolor{currentstroke}{rgb}{1.000000,0.000000,0.000000}%
\pgfsetstrokecolor{currentstroke}%
\pgfsetdash{}{0pt}%
\pgfpathmoveto{\pgfqpoint{4.927133in}{0.660499in}}%
\pgfpathcurveto{\pgfqpoint{4.938183in}{0.660499in}}{\pgfqpoint{4.948782in}{0.664890in}}{\pgfqpoint{4.956596in}{0.672703in}}%
\pgfpathcurveto{\pgfqpoint{4.964409in}{0.680517in}}{\pgfqpoint{4.968800in}{0.691116in}}{\pgfqpoint{4.968800in}{0.702166in}}%
\pgfpathcurveto{\pgfqpoint{4.968800in}{0.713216in}}{\pgfqpoint{4.964409in}{0.723815in}}{\pgfqpoint{4.956596in}{0.731629in}}%
\pgfpathcurveto{\pgfqpoint{4.948782in}{0.739442in}}{\pgfqpoint{4.938183in}{0.743833in}}{\pgfqpoint{4.927133in}{0.743833in}}%
\pgfpathcurveto{\pgfqpoint{4.916083in}{0.743833in}}{\pgfqpoint{4.905484in}{0.739442in}}{\pgfqpoint{4.897670in}{0.731629in}}%
\pgfpathcurveto{\pgfqpoint{4.889857in}{0.723815in}}{\pgfqpoint{4.885466in}{0.713216in}}{\pgfqpoint{4.885466in}{0.702166in}}%
\pgfpathcurveto{\pgfqpoint{4.885466in}{0.691116in}}{\pgfqpoint{4.889857in}{0.680517in}}{\pgfqpoint{4.897670in}{0.672703in}}%
\pgfpathcurveto{\pgfqpoint{4.905484in}{0.664890in}}{\pgfqpoint{4.916083in}{0.660499in}}{\pgfqpoint{4.927133in}{0.660499in}}%
\pgfpathlineto{\pgfqpoint{4.927133in}{0.660499in}}%
\pgfpathclose%
\pgfusepath{stroke}%
\end{pgfscope}%
\begin{pgfscope}%
\pgfpathrectangle{\pgfqpoint{0.847223in}{0.554012in}}{\pgfqpoint{6.200000in}{4.620000in}}%
\pgfusepath{clip}%
\pgfsetbuttcap%
\pgfsetroundjoin%
\pgfsetlinewidth{1.003750pt}%
\definecolor{currentstroke}{rgb}{1.000000,0.000000,0.000000}%
\pgfsetstrokecolor{currentstroke}%
\pgfsetdash{}{0pt}%
\pgfpathmoveto{\pgfqpoint{4.932466in}{0.659835in}}%
\pgfpathcurveto{\pgfqpoint{4.943516in}{0.659835in}}{\pgfqpoint{4.954115in}{0.664226in}}{\pgfqpoint{4.961929in}{0.672039in}}%
\pgfpathcurveto{\pgfqpoint{4.969743in}{0.679853in}}{\pgfqpoint{4.974133in}{0.690452in}}{\pgfqpoint{4.974133in}{0.701502in}}%
\pgfpathcurveto{\pgfqpoint{4.974133in}{0.712552in}}{\pgfqpoint{4.969743in}{0.723151in}}{\pgfqpoint{4.961929in}{0.730965in}}%
\pgfpathcurveto{\pgfqpoint{4.954115in}{0.738779in}}{\pgfqpoint{4.943516in}{0.743169in}}{\pgfqpoint{4.932466in}{0.743169in}}%
\pgfpathcurveto{\pgfqpoint{4.921416in}{0.743169in}}{\pgfqpoint{4.910817in}{0.738779in}}{\pgfqpoint{4.903003in}{0.730965in}}%
\pgfpathcurveto{\pgfqpoint{4.895190in}{0.723151in}}{\pgfqpoint{4.890800in}{0.712552in}}{\pgfqpoint{4.890800in}{0.701502in}}%
\pgfpathcurveto{\pgfqpoint{4.890800in}{0.690452in}}{\pgfqpoint{4.895190in}{0.679853in}}{\pgfqpoint{4.903003in}{0.672039in}}%
\pgfpathcurveto{\pgfqpoint{4.910817in}{0.664226in}}{\pgfqpoint{4.921416in}{0.659835in}}{\pgfqpoint{4.932466in}{0.659835in}}%
\pgfpathlineto{\pgfqpoint{4.932466in}{0.659835in}}%
\pgfpathclose%
\pgfusepath{stroke}%
\end{pgfscope}%
\begin{pgfscope}%
\pgfpathrectangle{\pgfqpoint{0.847223in}{0.554012in}}{\pgfqpoint{6.200000in}{4.620000in}}%
\pgfusepath{clip}%
\pgfsetbuttcap%
\pgfsetroundjoin%
\pgfsetlinewidth{1.003750pt}%
\definecolor{currentstroke}{rgb}{1.000000,0.000000,0.000000}%
\pgfsetstrokecolor{currentstroke}%
\pgfsetdash{}{0pt}%
\pgfpathmoveto{\pgfqpoint{4.937799in}{0.659173in}}%
\pgfpathcurveto{\pgfqpoint{4.948850in}{0.659173in}}{\pgfqpoint{4.959449in}{0.663563in}}{\pgfqpoint{4.967262in}{0.671377in}}%
\pgfpathcurveto{\pgfqpoint{4.975076in}{0.679191in}}{\pgfqpoint{4.979466in}{0.689790in}}{\pgfqpoint{4.979466in}{0.700840in}}%
\pgfpathcurveto{\pgfqpoint{4.979466in}{0.711890in}}{\pgfqpoint{4.975076in}{0.722489in}}{\pgfqpoint{4.967262in}{0.730303in}}%
\pgfpathcurveto{\pgfqpoint{4.959449in}{0.738116in}}{\pgfqpoint{4.948850in}{0.742507in}}{\pgfqpoint{4.937799in}{0.742507in}}%
\pgfpathcurveto{\pgfqpoint{4.926749in}{0.742507in}}{\pgfqpoint{4.916150in}{0.738116in}}{\pgfqpoint{4.908337in}{0.730303in}}%
\pgfpathcurveto{\pgfqpoint{4.900523in}{0.722489in}}{\pgfqpoint{4.896133in}{0.711890in}}{\pgfqpoint{4.896133in}{0.700840in}}%
\pgfpathcurveto{\pgfqpoint{4.896133in}{0.689790in}}{\pgfqpoint{4.900523in}{0.679191in}}{\pgfqpoint{4.908337in}{0.671377in}}%
\pgfpathcurveto{\pgfqpoint{4.916150in}{0.663563in}}{\pgfqpoint{4.926749in}{0.659173in}}{\pgfqpoint{4.937799in}{0.659173in}}%
\pgfpathlineto{\pgfqpoint{4.937799in}{0.659173in}}%
\pgfpathclose%
\pgfusepath{stroke}%
\end{pgfscope}%
\begin{pgfscope}%
\pgfpathrectangle{\pgfqpoint{0.847223in}{0.554012in}}{\pgfqpoint{6.200000in}{4.620000in}}%
\pgfusepath{clip}%
\pgfsetbuttcap%
\pgfsetroundjoin%
\pgfsetlinewidth{1.003750pt}%
\definecolor{currentstroke}{rgb}{1.000000,0.000000,0.000000}%
\pgfsetstrokecolor{currentstroke}%
\pgfsetdash{}{0pt}%
\pgfpathmoveto{\pgfqpoint{4.943133in}{0.658512in}}%
\pgfpathcurveto{\pgfqpoint{4.954183in}{0.658512in}}{\pgfqpoint{4.964782in}{0.662903in}}{\pgfqpoint{4.972595in}{0.670716in}}%
\pgfpathcurveto{\pgfqpoint{4.980409in}{0.678530in}}{\pgfqpoint{4.984799in}{0.689129in}}{\pgfqpoint{4.984799in}{0.700179in}}%
\pgfpathcurveto{\pgfqpoint{4.984799in}{0.711229in}}{\pgfqpoint{4.980409in}{0.721828in}}{\pgfqpoint{4.972595in}{0.729642in}}%
\pgfpathcurveto{\pgfqpoint{4.964782in}{0.737456in}}{\pgfqpoint{4.954183in}{0.741846in}}{\pgfqpoint{4.943133in}{0.741846in}}%
\pgfpathcurveto{\pgfqpoint{4.932083in}{0.741846in}}{\pgfqpoint{4.921484in}{0.737456in}}{\pgfqpoint{4.913670in}{0.729642in}}%
\pgfpathcurveto{\pgfqpoint{4.905856in}{0.721828in}}{\pgfqpoint{4.901466in}{0.711229in}}{\pgfqpoint{4.901466in}{0.700179in}}%
\pgfpathcurveto{\pgfqpoint{4.901466in}{0.689129in}}{\pgfqpoint{4.905856in}{0.678530in}}{\pgfqpoint{4.913670in}{0.670716in}}%
\pgfpathcurveto{\pgfqpoint{4.921484in}{0.662903in}}{\pgfqpoint{4.932083in}{0.658512in}}{\pgfqpoint{4.943133in}{0.658512in}}%
\pgfpathlineto{\pgfqpoint{4.943133in}{0.658512in}}%
\pgfpathclose%
\pgfusepath{stroke}%
\end{pgfscope}%
\begin{pgfscope}%
\pgfpathrectangle{\pgfqpoint{0.847223in}{0.554012in}}{\pgfqpoint{6.200000in}{4.620000in}}%
\pgfusepath{clip}%
\pgfsetbuttcap%
\pgfsetroundjoin%
\pgfsetlinewidth{1.003750pt}%
\definecolor{currentstroke}{rgb}{1.000000,0.000000,0.000000}%
\pgfsetstrokecolor{currentstroke}%
\pgfsetdash{}{0pt}%
\pgfpathmoveto{\pgfqpoint{4.948466in}{0.657853in}}%
\pgfpathcurveto{\pgfqpoint{4.959516in}{0.657853in}}{\pgfqpoint{4.970115in}{0.662243in}}{\pgfqpoint{4.977929in}{0.670057in}}%
\pgfpathcurveto{\pgfqpoint{4.985742in}{0.677871in}}{\pgfqpoint{4.990133in}{0.688470in}}{\pgfqpoint{4.990133in}{0.699520in}}%
\pgfpathcurveto{\pgfqpoint{4.990133in}{0.710570in}}{\pgfqpoint{4.985742in}{0.721169in}}{\pgfqpoint{4.977929in}{0.728983in}}%
\pgfpathcurveto{\pgfqpoint{4.970115in}{0.736796in}}{\pgfqpoint{4.959516in}{0.741187in}}{\pgfqpoint{4.948466in}{0.741187in}}%
\pgfpathcurveto{\pgfqpoint{4.937416in}{0.741187in}}{\pgfqpoint{4.926817in}{0.736796in}}{\pgfqpoint{4.919003in}{0.728983in}}%
\pgfpathcurveto{\pgfqpoint{4.911189in}{0.721169in}}{\pgfqpoint{4.906799in}{0.710570in}}{\pgfqpoint{4.906799in}{0.699520in}}%
\pgfpathcurveto{\pgfqpoint{4.906799in}{0.688470in}}{\pgfqpoint{4.911189in}{0.677871in}}{\pgfqpoint{4.919003in}{0.670057in}}%
\pgfpathcurveto{\pgfqpoint{4.926817in}{0.662243in}}{\pgfqpoint{4.937416in}{0.657853in}}{\pgfqpoint{4.948466in}{0.657853in}}%
\pgfpathlineto{\pgfqpoint{4.948466in}{0.657853in}}%
\pgfpathclose%
\pgfusepath{stroke}%
\end{pgfscope}%
\begin{pgfscope}%
\pgfpathrectangle{\pgfqpoint{0.847223in}{0.554012in}}{\pgfqpoint{6.200000in}{4.620000in}}%
\pgfusepath{clip}%
\pgfsetbuttcap%
\pgfsetroundjoin%
\pgfsetlinewidth{1.003750pt}%
\definecolor{currentstroke}{rgb}{1.000000,0.000000,0.000000}%
\pgfsetstrokecolor{currentstroke}%
\pgfsetdash{}{0pt}%
\pgfpathmoveto{\pgfqpoint{4.953799in}{0.657196in}}%
\pgfpathcurveto{\pgfqpoint{4.964849in}{0.657196in}}{\pgfqpoint{4.975448in}{0.661586in}}{\pgfqpoint{4.983262in}{0.669399in}}%
\pgfpathcurveto{\pgfqpoint{4.991076in}{0.677213in}}{\pgfqpoint{4.995466in}{0.687812in}}{\pgfqpoint{4.995466in}{0.698862in}}%
\pgfpathcurveto{\pgfqpoint{4.995466in}{0.709912in}}{\pgfqpoint{4.991076in}{0.720511in}}{\pgfqpoint{4.983262in}{0.728325in}}%
\pgfpathcurveto{\pgfqpoint{4.975448in}{0.736139in}}{\pgfqpoint{4.964849in}{0.740529in}}{\pgfqpoint{4.953799in}{0.740529in}}%
\pgfpathcurveto{\pgfqpoint{4.942749in}{0.740529in}}{\pgfqpoint{4.932150in}{0.736139in}}{\pgfqpoint{4.924336in}{0.728325in}}%
\pgfpathcurveto{\pgfqpoint{4.916523in}{0.720511in}}{\pgfqpoint{4.912132in}{0.709912in}}{\pgfqpoint{4.912132in}{0.698862in}}%
\pgfpathcurveto{\pgfqpoint{4.912132in}{0.687812in}}{\pgfqpoint{4.916523in}{0.677213in}}{\pgfqpoint{4.924336in}{0.669399in}}%
\pgfpathcurveto{\pgfqpoint{4.932150in}{0.661586in}}{\pgfqpoint{4.942749in}{0.657196in}}{\pgfqpoint{4.953799in}{0.657196in}}%
\pgfpathlineto{\pgfqpoint{4.953799in}{0.657196in}}%
\pgfpathclose%
\pgfusepath{stroke}%
\end{pgfscope}%
\begin{pgfscope}%
\pgfpathrectangle{\pgfqpoint{0.847223in}{0.554012in}}{\pgfqpoint{6.200000in}{4.620000in}}%
\pgfusepath{clip}%
\pgfsetbuttcap%
\pgfsetroundjoin%
\pgfsetlinewidth{1.003750pt}%
\definecolor{currentstroke}{rgb}{1.000000,0.000000,0.000000}%
\pgfsetstrokecolor{currentstroke}%
\pgfsetdash{}{0pt}%
\pgfpathmoveto{\pgfqpoint{4.959132in}{0.656539in}}%
\pgfpathcurveto{\pgfqpoint{4.970182in}{0.656539in}}{\pgfqpoint{4.980781in}{0.660930in}}{\pgfqpoint{4.988595in}{0.668743in}}%
\pgfpathcurveto{\pgfqpoint{4.996409in}{0.676557in}}{\pgfqpoint{5.000799in}{0.687156in}}{\pgfqpoint{5.000799in}{0.698206in}}%
\pgfpathcurveto{\pgfqpoint{5.000799in}{0.709256in}}{\pgfqpoint{4.996409in}{0.719855in}}{\pgfqpoint{4.988595in}{0.727669in}}%
\pgfpathcurveto{\pgfqpoint{4.980781in}{0.735482in}}{\pgfqpoint{4.970182in}{0.739873in}}{\pgfqpoint{4.959132in}{0.739873in}}%
\pgfpathcurveto{\pgfqpoint{4.948082in}{0.739873in}}{\pgfqpoint{4.937483in}{0.735482in}}{\pgfqpoint{4.929670in}{0.727669in}}%
\pgfpathcurveto{\pgfqpoint{4.921856in}{0.719855in}}{\pgfqpoint{4.917466in}{0.709256in}}{\pgfqpoint{4.917466in}{0.698206in}}%
\pgfpathcurveto{\pgfqpoint{4.917466in}{0.687156in}}{\pgfqpoint{4.921856in}{0.676557in}}{\pgfqpoint{4.929670in}{0.668743in}}%
\pgfpathcurveto{\pgfqpoint{4.937483in}{0.660930in}}{\pgfqpoint{4.948082in}{0.656539in}}{\pgfqpoint{4.959132in}{0.656539in}}%
\pgfpathlineto{\pgfqpoint{4.959132in}{0.656539in}}%
\pgfpathclose%
\pgfusepath{stroke}%
\end{pgfscope}%
\begin{pgfscope}%
\pgfpathrectangle{\pgfqpoint{0.847223in}{0.554012in}}{\pgfqpoint{6.200000in}{4.620000in}}%
\pgfusepath{clip}%
\pgfsetbuttcap%
\pgfsetroundjoin%
\pgfsetlinewidth{1.003750pt}%
\definecolor{currentstroke}{rgb}{1.000000,0.000000,0.000000}%
\pgfsetstrokecolor{currentstroke}%
\pgfsetdash{}{0pt}%
\pgfpathmoveto{\pgfqpoint{4.964466in}{0.655885in}}%
\pgfpathcurveto{\pgfqpoint{4.975516in}{0.655885in}}{\pgfqpoint{4.986115in}{0.660275in}}{\pgfqpoint{4.993928in}{0.668089in}}%
\pgfpathcurveto{\pgfqpoint{5.001742in}{0.675902in}}{\pgfqpoint{5.006132in}{0.686501in}}{\pgfqpoint{5.006132in}{0.697551in}}%
\pgfpathcurveto{\pgfqpoint{5.006132in}{0.708601in}}{\pgfqpoint{5.001742in}{0.719200in}}{\pgfqpoint{4.993928in}{0.727014in}}%
\pgfpathcurveto{\pgfqpoint{4.986115in}{0.734828in}}{\pgfqpoint{4.975516in}{0.739218in}}{\pgfqpoint{4.964466in}{0.739218in}}%
\pgfpathcurveto{\pgfqpoint{4.953415in}{0.739218in}}{\pgfqpoint{4.942816in}{0.734828in}}{\pgfqpoint{4.935003in}{0.727014in}}%
\pgfpathcurveto{\pgfqpoint{4.927189in}{0.719200in}}{\pgfqpoint{4.922799in}{0.708601in}}{\pgfqpoint{4.922799in}{0.697551in}}%
\pgfpathcurveto{\pgfqpoint{4.922799in}{0.686501in}}{\pgfqpoint{4.927189in}{0.675902in}}{\pgfqpoint{4.935003in}{0.668089in}}%
\pgfpathcurveto{\pgfqpoint{4.942816in}{0.660275in}}{\pgfqpoint{4.953415in}{0.655885in}}{\pgfqpoint{4.964466in}{0.655885in}}%
\pgfpathlineto{\pgfqpoint{4.964466in}{0.655885in}}%
\pgfpathclose%
\pgfusepath{stroke}%
\end{pgfscope}%
\begin{pgfscope}%
\pgfpathrectangle{\pgfqpoint{0.847223in}{0.554012in}}{\pgfqpoint{6.200000in}{4.620000in}}%
\pgfusepath{clip}%
\pgfsetbuttcap%
\pgfsetroundjoin%
\pgfsetlinewidth{1.003750pt}%
\definecolor{currentstroke}{rgb}{1.000000,0.000000,0.000000}%
\pgfsetstrokecolor{currentstroke}%
\pgfsetdash{}{0pt}%
\pgfpathmoveto{\pgfqpoint{4.969799in}{0.655231in}}%
\pgfpathcurveto{\pgfqpoint{4.980849in}{0.655231in}}{\pgfqpoint{4.991448in}{0.659622in}}{\pgfqpoint{4.999262in}{0.667435in}}%
\pgfpathcurveto{\pgfqpoint{5.007075in}{0.675249in}}{\pgfqpoint{5.011465in}{0.685848in}}{\pgfqpoint{5.011465in}{0.696898in}}%
\pgfpathcurveto{\pgfqpoint{5.011465in}{0.707948in}}{\pgfqpoint{5.007075in}{0.718547in}}{\pgfqpoint{4.999262in}{0.726361in}}%
\pgfpathcurveto{\pgfqpoint{4.991448in}{0.734175in}}{\pgfqpoint{4.980849in}{0.738565in}}{\pgfqpoint{4.969799in}{0.738565in}}%
\pgfpathcurveto{\pgfqpoint{4.958749in}{0.738565in}}{\pgfqpoint{4.948150in}{0.734175in}}{\pgfqpoint{4.940336in}{0.726361in}}%
\pgfpathcurveto{\pgfqpoint{4.932522in}{0.718547in}}{\pgfqpoint{4.928132in}{0.707948in}}{\pgfqpoint{4.928132in}{0.696898in}}%
\pgfpathcurveto{\pgfqpoint{4.928132in}{0.685848in}}{\pgfqpoint{4.932522in}{0.675249in}}{\pgfqpoint{4.940336in}{0.667435in}}%
\pgfpathcurveto{\pgfqpoint{4.948150in}{0.659622in}}{\pgfqpoint{4.958749in}{0.655231in}}{\pgfqpoint{4.969799in}{0.655231in}}%
\pgfpathlineto{\pgfqpoint{4.969799in}{0.655231in}}%
\pgfpathclose%
\pgfusepath{stroke}%
\end{pgfscope}%
\begin{pgfscope}%
\pgfpathrectangle{\pgfqpoint{0.847223in}{0.554012in}}{\pgfqpoint{6.200000in}{4.620000in}}%
\pgfusepath{clip}%
\pgfsetbuttcap%
\pgfsetroundjoin%
\pgfsetlinewidth{1.003750pt}%
\definecolor{currentstroke}{rgb}{1.000000,0.000000,0.000000}%
\pgfsetstrokecolor{currentstroke}%
\pgfsetdash{}{0pt}%
\pgfpathmoveto{\pgfqpoint{4.975132in}{0.654580in}}%
\pgfpathcurveto{\pgfqpoint{4.986182in}{0.654580in}}{\pgfqpoint{4.996781in}{0.658970in}}{\pgfqpoint{5.004595in}{0.666784in}}%
\pgfpathcurveto{\pgfqpoint{5.012408in}{0.674597in}}{\pgfqpoint{5.016799in}{0.685196in}}{\pgfqpoint{5.016799in}{0.696246in}}%
\pgfpathcurveto{\pgfqpoint{5.016799in}{0.707297in}}{\pgfqpoint{5.012408in}{0.717896in}}{\pgfqpoint{5.004595in}{0.725709in}}%
\pgfpathcurveto{\pgfqpoint{4.996781in}{0.733523in}}{\pgfqpoint{4.986182in}{0.737913in}}{\pgfqpoint{4.975132in}{0.737913in}}%
\pgfpathcurveto{\pgfqpoint{4.964082in}{0.737913in}}{\pgfqpoint{4.953483in}{0.733523in}}{\pgfqpoint{4.945669in}{0.725709in}}%
\pgfpathcurveto{\pgfqpoint{4.937856in}{0.717896in}}{\pgfqpoint{4.933465in}{0.707297in}}{\pgfqpoint{4.933465in}{0.696246in}}%
\pgfpathcurveto{\pgfqpoint{4.933465in}{0.685196in}}{\pgfqpoint{4.937856in}{0.674597in}}{\pgfqpoint{4.945669in}{0.666784in}}%
\pgfpathcurveto{\pgfqpoint{4.953483in}{0.658970in}}{\pgfqpoint{4.964082in}{0.654580in}}{\pgfqpoint{4.975132in}{0.654580in}}%
\pgfpathlineto{\pgfqpoint{4.975132in}{0.654580in}}%
\pgfpathclose%
\pgfusepath{stroke}%
\end{pgfscope}%
\begin{pgfscope}%
\pgfpathrectangle{\pgfqpoint{0.847223in}{0.554012in}}{\pgfqpoint{6.200000in}{4.620000in}}%
\pgfusepath{clip}%
\pgfsetbuttcap%
\pgfsetroundjoin%
\pgfsetlinewidth{1.003750pt}%
\definecolor{currentstroke}{rgb}{1.000000,0.000000,0.000000}%
\pgfsetstrokecolor{currentstroke}%
\pgfsetdash{}{0pt}%
\pgfpathmoveto{\pgfqpoint{4.980465in}{0.653930in}}%
\pgfpathcurveto{\pgfqpoint{4.991515in}{0.653930in}}{\pgfqpoint{5.002114in}{0.658320in}}{\pgfqpoint{5.009928in}{0.666133in}}%
\pgfpathcurveto{\pgfqpoint{5.017742in}{0.673947in}}{\pgfqpoint{5.022132in}{0.684546in}}{\pgfqpoint{5.022132in}{0.695596in}}%
\pgfpathcurveto{\pgfqpoint{5.022132in}{0.706646in}}{\pgfqpoint{5.017742in}{0.717245in}}{\pgfqpoint{5.009928in}{0.725059in}}%
\pgfpathcurveto{\pgfqpoint{5.002114in}{0.732873in}}{\pgfqpoint{4.991515in}{0.737263in}}{\pgfqpoint{4.980465in}{0.737263in}}%
\pgfpathcurveto{\pgfqpoint{4.969415in}{0.737263in}}{\pgfqpoint{4.958816in}{0.732873in}}{\pgfqpoint{4.951002in}{0.725059in}}%
\pgfpathcurveto{\pgfqpoint{4.943189in}{0.717245in}}{\pgfqpoint{4.938799in}{0.706646in}}{\pgfqpoint{4.938799in}{0.695596in}}%
\pgfpathcurveto{\pgfqpoint{4.938799in}{0.684546in}}{\pgfqpoint{4.943189in}{0.673947in}}{\pgfqpoint{4.951002in}{0.666133in}}%
\pgfpathcurveto{\pgfqpoint{4.958816in}{0.658320in}}{\pgfqpoint{4.969415in}{0.653930in}}{\pgfqpoint{4.980465in}{0.653930in}}%
\pgfpathlineto{\pgfqpoint{4.980465in}{0.653930in}}%
\pgfpathclose%
\pgfusepath{stroke}%
\end{pgfscope}%
\begin{pgfscope}%
\pgfpathrectangle{\pgfqpoint{0.847223in}{0.554012in}}{\pgfqpoint{6.200000in}{4.620000in}}%
\pgfusepath{clip}%
\pgfsetbuttcap%
\pgfsetroundjoin%
\pgfsetlinewidth{1.003750pt}%
\definecolor{currentstroke}{rgb}{1.000000,0.000000,0.000000}%
\pgfsetstrokecolor{currentstroke}%
\pgfsetdash{}{0pt}%
\pgfpathmoveto{\pgfqpoint{4.985798in}{0.653281in}}%
\pgfpathcurveto{\pgfqpoint{4.996849in}{0.653281in}}{\pgfqpoint{5.007448in}{0.657671in}}{\pgfqpoint{5.015261in}{0.665485in}}%
\pgfpathcurveto{\pgfqpoint{5.023075in}{0.673298in}}{\pgfqpoint{5.027465in}{0.683897in}}{\pgfqpoint{5.027465in}{0.694948in}}%
\pgfpathcurveto{\pgfqpoint{5.027465in}{0.705998in}}{\pgfqpoint{5.023075in}{0.716597in}}{\pgfqpoint{5.015261in}{0.724410in}}%
\pgfpathcurveto{\pgfqpoint{5.007448in}{0.732224in}}{\pgfqpoint{4.996849in}{0.736614in}}{\pgfqpoint{4.985798in}{0.736614in}}%
\pgfpathcurveto{\pgfqpoint{4.974748in}{0.736614in}}{\pgfqpoint{4.964149in}{0.732224in}}{\pgfqpoint{4.956336in}{0.724410in}}%
\pgfpathcurveto{\pgfqpoint{4.948522in}{0.716597in}}{\pgfqpoint{4.944132in}{0.705998in}}{\pgfqpoint{4.944132in}{0.694948in}}%
\pgfpathcurveto{\pgfqpoint{4.944132in}{0.683897in}}{\pgfqpoint{4.948522in}{0.673298in}}{\pgfqpoint{4.956336in}{0.665485in}}%
\pgfpathcurveto{\pgfqpoint{4.964149in}{0.657671in}}{\pgfqpoint{4.974748in}{0.653281in}}{\pgfqpoint{4.985798in}{0.653281in}}%
\pgfpathlineto{\pgfqpoint{4.985798in}{0.653281in}}%
\pgfpathclose%
\pgfusepath{stroke}%
\end{pgfscope}%
\begin{pgfscope}%
\pgfpathrectangle{\pgfqpoint{0.847223in}{0.554012in}}{\pgfqpoint{6.200000in}{4.620000in}}%
\pgfusepath{clip}%
\pgfsetbuttcap%
\pgfsetroundjoin%
\pgfsetlinewidth{1.003750pt}%
\definecolor{currentstroke}{rgb}{1.000000,0.000000,0.000000}%
\pgfsetstrokecolor{currentstroke}%
\pgfsetdash{}{0pt}%
\pgfpathmoveto{\pgfqpoint{4.991132in}{0.652634in}}%
\pgfpathcurveto{\pgfqpoint{5.002182in}{0.652634in}}{\pgfqpoint{5.012781in}{0.657024in}}{\pgfqpoint{5.020594in}{0.664838in}}%
\pgfpathcurveto{\pgfqpoint{5.028408in}{0.672651in}}{\pgfqpoint{5.032798in}{0.683250in}}{\pgfqpoint{5.032798in}{0.694300in}}%
\pgfpathcurveto{\pgfqpoint{5.032798in}{0.705350in}}{\pgfqpoint{5.028408in}{0.715949in}}{\pgfqpoint{5.020594in}{0.723763in}}%
\pgfpathcurveto{\pgfqpoint{5.012781in}{0.731577in}}{\pgfqpoint{5.002182in}{0.735967in}}{\pgfqpoint{4.991132in}{0.735967in}}%
\pgfpathcurveto{\pgfqpoint{4.980081in}{0.735967in}}{\pgfqpoint{4.969482in}{0.731577in}}{\pgfqpoint{4.961669in}{0.723763in}}%
\pgfpathcurveto{\pgfqpoint{4.953855in}{0.715949in}}{\pgfqpoint{4.949465in}{0.705350in}}{\pgfqpoint{4.949465in}{0.694300in}}%
\pgfpathcurveto{\pgfqpoint{4.949465in}{0.683250in}}{\pgfqpoint{4.953855in}{0.672651in}}{\pgfqpoint{4.961669in}{0.664838in}}%
\pgfpathcurveto{\pgfqpoint{4.969482in}{0.657024in}}{\pgfqpoint{4.980081in}{0.652634in}}{\pgfqpoint{4.991132in}{0.652634in}}%
\pgfpathlineto{\pgfqpoint{4.991132in}{0.652634in}}%
\pgfpathclose%
\pgfusepath{stroke}%
\end{pgfscope}%
\begin{pgfscope}%
\pgfpathrectangle{\pgfqpoint{0.847223in}{0.554012in}}{\pgfqpoint{6.200000in}{4.620000in}}%
\pgfusepath{clip}%
\pgfsetbuttcap%
\pgfsetroundjoin%
\pgfsetlinewidth{1.003750pt}%
\definecolor{currentstroke}{rgb}{1.000000,0.000000,0.000000}%
\pgfsetstrokecolor{currentstroke}%
\pgfsetdash{}{0pt}%
\pgfpathmoveto{\pgfqpoint{4.996465in}{0.651988in}}%
\pgfpathcurveto{\pgfqpoint{5.007515in}{0.651988in}}{\pgfqpoint{5.018114in}{0.656378in}}{\pgfqpoint{5.025928in}{0.664192in}}%
\pgfpathcurveto{\pgfqpoint{5.033741in}{0.672005in}}{\pgfqpoint{5.038132in}{0.682604in}}{\pgfqpoint{5.038132in}{0.693655in}}%
\pgfpathcurveto{\pgfqpoint{5.038132in}{0.704705in}}{\pgfqpoint{5.033741in}{0.715304in}}{\pgfqpoint{5.025928in}{0.723117in}}%
\pgfpathcurveto{\pgfqpoint{5.018114in}{0.730931in}}{\pgfqpoint{5.007515in}{0.735321in}}{\pgfqpoint{4.996465in}{0.735321in}}%
\pgfpathcurveto{\pgfqpoint{4.985415in}{0.735321in}}{\pgfqpoint{4.974816in}{0.730931in}}{\pgfqpoint{4.967002in}{0.723117in}}%
\pgfpathcurveto{\pgfqpoint{4.959188in}{0.715304in}}{\pgfqpoint{4.954798in}{0.704705in}}{\pgfqpoint{4.954798in}{0.693655in}}%
\pgfpathcurveto{\pgfqpoint{4.954798in}{0.682604in}}{\pgfqpoint{4.959188in}{0.672005in}}{\pgfqpoint{4.967002in}{0.664192in}}%
\pgfpathcurveto{\pgfqpoint{4.974816in}{0.656378in}}{\pgfqpoint{4.985415in}{0.651988in}}{\pgfqpoint{4.996465in}{0.651988in}}%
\pgfpathlineto{\pgfqpoint{4.996465in}{0.651988in}}%
\pgfpathclose%
\pgfusepath{stroke}%
\end{pgfscope}%
\begin{pgfscope}%
\pgfpathrectangle{\pgfqpoint{0.847223in}{0.554012in}}{\pgfqpoint{6.200000in}{4.620000in}}%
\pgfusepath{clip}%
\pgfsetbuttcap%
\pgfsetroundjoin%
\pgfsetlinewidth{1.003750pt}%
\definecolor{currentstroke}{rgb}{1.000000,0.000000,0.000000}%
\pgfsetstrokecolor{currentstroke}%
\pgfsetdash{}{0pt}%
\pgfpathmoveto{\pgfqpoint{5.001798in}{0.651344in}}%
\pgfpathcurveto{\pgfqpoint{5.012848in}{0.651344in}}{\pgfqpoint{5.023447in}{0.655734in}}{\pgfqpoint{5.031261in}{0.663547in}}%
\pgfpathcurveto{\pgfqpoint{5.039074in}{0.671361in}}{\pgfqpoint{5.043465in}{0.681960in}}{\pgfqpoint{5.043465in}{0.693010in}}%
\pgfpathcurveto{\pgfqpoint{5.043465in}{0.704060in}}{\pgfqpoint{5.039074in}{0.714659in}}{\pgfqpoint{5.031261in}{0.722473in}}%
\pgfpathcurveto{\pgfqpoint{5.023447in}{0.730287in}}{\pgfqpoint{5.012848in}{0.734677in}}{\pgfqpoint{5.001798in}{0.734677in}}%
\pgfpathcurveto{\pgfqpoint{4.990748in}{0.734677in}}{\pgfqpoint{4.980149in}{0.730287in}}{\pgfqpoint{4.972335in}{0.722473in}}%
\pgfpathcurveto{\pgfqpoint{4.964522in}{0.714659in}}{\pgfqpoint{4.960131in}{0.704060in}}{\pgfqpoint{4.960131in}{0.693010in}}%
\pgfpathcurveto{\pgfqpoint{4.960131in}{0.681960in}}{\pgfqpoint{4.964522in}{0.671361in}}{\pgfqpoint{4.972335in}{0.663547in}}%
\pgfpathcurveto{\pgfqpoint{4.980149in}{0.655734in}}{\pgfqpoint{4.990748in}{0.651344in}}{\pgfqpoint{5.001798in}{0.651344in}}%
\pgfpathlineto{\pgfqpoint{5.001798in}{0.651344in}}%
\pgfpathclose%
\pgfusepath{stroke}%
\end{pgfscope}%
\begin{pgfscope}%
\pgfpathrectangle{\pgfqpoint{0.847223in}{0.554012in}}{\pgfqpoint{6.200000in}{4.620000in}}%
\pgfusepath{clip}%
\pgfsetbuttcap%
\pgfsetroundjoin%
\pgfsetlinewidth{1.003750pt}%
\definecolor{currentstroke}{rgb}{1.000000,0.000000,0.000000}%
\pgfsetstrokecolor{currentstroke}%
\pgfsetdash{}{0pt}%
\pgfpathmoveto{\pgfqpoint{5.007131in}{0.650701in}}%
\pgfpathcurveto{\pgfqpoint{5.018181in}{0.650701in}}{\pgfqpoint{5.028780in}{0.655091in}}{\pgfqpoint{5.036594in}{0.662905in}}%
\pgfpathcurveto{\pgfqpoint{5.044408in}{0.670718in}}{\pgfqpoint{5.048798in}{0.681317in}}{\pgfqpoint{5.048798in}{0.692367in}}%
\pgfpathcurveto{\pgfqpoint{5.048798in}{0.703418in}}{\pgfqpoint{5.044408in}{0.714017in}}{\pgfqpoint{5.036594in}{0.721830in}}%
\pgfpathcurveto{\pgfqpoint{5.028780in}{0.729644in}}{\pgfqpoint{5.018181in}{0.734034in}}{\pgfqpoint{5.007131in}{0.734034in}}%
\pgfpathcurveto{\pgfqpoint{4.996081in}{0.734034in}}{\pgfqpoint{4.985482in}{0.729644in}}{\pgfqpoint{4.977668in}{0.721830in}}%
\pgfpathcurveto{\pgfqpoint{4.969855in}{0.714017in}}{\pgfqpoint{4.965465in}{0.703418in}}{\pgfqpoint{4.965465in}{0.692367in}}%
\pgfpathcurveto{\pgfqpoint{4.965465in}{0.681317in}}{\pgfqpoint{4.969855in}{0.670718in}}{\pgfqpoint{4.977668in}{0.662905in}}%
\pgfpathcurveto{\pgfqpoint{4.985482in}{0.655091in}}{\pgfqpoint{4.996081in}{0.650701in}}{\pgfqpoint{5.007131in}{0.650701in}}%
\pgfpathlineto{\pgfqpoint{5.007131in}{0.650701in}}%
\pgfpathclose%
\pgfusepath{stroke}%
\end{pgfscope}%
\begin{pgfscope}%
\pgfpathrectangle{\pgfqpoint{0.847223in}{0.554012in}}{\pgfqpoint{6.200000in}{4.620000in}}%
\pgfusepath{clip}%
\pgfsetbuttcap%
\pgfsetroundjoin%
\pgfsetlinewidth{1.003750pt}%
\definecolor{currentstroke}{rgb}{1.000000,0.000000,0.000000}%
\pgfsetstrokecolor{currentstroke}%
\pgfsetdash{}{0pt}%
\pgfpathmoveto{\pgfqpoint{5.012464in}{0.650059in}}%
\pgfpathcurveto{\pgfqpoint{5.023515in}{0.650059in}}{\pgfqpoint{5.034114in}{0.654450in}}{\pgfqpoint{5.041927in}{0.662263in}}%
\pgfpathcurveto{\pgfqpoint{5.049741in}{0.670077in}}{\pgfqpoint{5.054131in}{0.680676in}}{\pgfqpoint{5.054131in}{0.691726in}}%
\pgfpathcurveto{\pgfqpoint{5.054131in}{0.702776in}}{\pgfqpoint{5.049741in}{0.713375in}}{\pgfqpoint{5.041927in}{0.721189in}}%
\pgfpathcurveto{\pgfqpoint{5.034114in}{0.729003in}}{\pgfqpoint{5.023515in}{0.733393in}}{\pgfqpoint{5.012464in}{0.733393in}}%
\pgfpathcurveto{\pgfqpoint{5.001414in}{0.733393in}}{\pgfqpoint{4.990815in}{0.729003in}}{\pgfqpoint{4.983002in}{0.721189in}}%
\pgfpathcurveto{\pgfqpoint{4.975188in}{0.713375in}}{\pgfqpoint{4.970798in}{0.702776in}}{\pgfqpoint{4.970798in}{0.691726in}}%
\pgfpathcurveto{\pgfqpoint{4.970798in}{0.680676in}}{\pgfqpoint{4.975188in}{0.670077in}}{\pgfqpoint{4.983002in}{0.662263in}}%
\pgfpathcurveto{\pgfqpoint{4.990815in}{0.654450in}}{\pgfqpoint{5.001414in}{0.650059in}}{\pgfqpoint{5.012464in}{0.650059in}}%
\pgfpathlineto{\pgfqpoint{5.012464in}{0.650059in}}%
\pgfpathclose%
\pgfusepath{stroke}%
\end{pgfscope}%
\begin{pgfscope}%
\pgfpathrectangle{\pgfqpoint{0.847223in}{0.554012in}}{\pgfqpoint{6.200000in}{4.620000in}}%
\pgfusepath{clip}%
\pgfsetbuttcap%
\pgfsetroundjoin%
\pgfsetlinewidth{1.003750pt}%
\definecolor{currentstroke}{rgb}{1.000000,0.000000,0.000000}%
\pgfsetstrokecolor{currentstroke}%
\pgfsetdash{}{0pt}%
\pgfpathmoveto{\pgfqpoint{5.017798in}{0.649420in}}%
\pgfpathcurveto{\pgfqpoint{5.028848in}{0.649420in}}{\pgfqpoint{5.039447in}{0.653810in}}{\pgfqpoint{5.047260in}{0.661623in}}%
\pgfpathcurveto{\pgfqpoint{5.055074in}{0.669437in}}{\pgfqpoint{5.059464in}{0.680036in}}{\pgfqpoint{5.059464in}{0.691086in}}%
\pgfpathcurveto{\pgfqpoint{5.059464in}{0.702136in}}{\pgfqpoint{5.055074in}{0.712735in}}{\pgfqpoint{5.047260in}{0.720549in}}%
\pgfpathcurveto{\pgfqpoint{5.039447in}{0.728363in}}{\pgfqpoint{5.028848in}{0.732753in}}{\pgfqpoint{5.017798in}{0.732753in}}%
\pgfpathcurveto{\pgfqpoint{5.006748in}{0.732753in}}{\pgfqpoint{4.996149in}{0.728363in}}{\pgfqpoint{4.988335in}{0.720549in}}%
\pgfpathcurveto{\pgfqpoint{4.980521in}{0.712735in}}{\pgfqpoint{4.976131in}{0.702136in}}{\pgfqpoint{4.976131in}{0.691086in}}%
\pgfpathcurveto{\pgfqpoint{4.976131in}{0.680036in}}{\pgfqpoint{4.980521in}{0.669437in}}{\pgfqpoint{4.988335in}{0.661623in}}%
\pgfpathcurveto{\pgfqpoint{4.996149in}{0.653810in}}{\pgfqpoint{5.006748in}{0.649420in}}{\pgfqpoint{5.017798in}{0.649420in}}%
\pgfpathlineto{\pgfqpoint{5.017798in}{0.649420in}}%
\pgfpathclose%
\pgfusepath{stroke}%
\end{pgfscope}%
\begin{pgfscope}%
\pgfpathrectangle{\pgfqpoint{0.847223in}{0.554012in}}{\pgfqpoint{6.200000in}{4.620000in}}%
\pgfusepath{clip}%
\pgfsetbuttcap%
\pgfsetroundjoin%
\pgfsetlinewidth{1.003750pt}%
\definecolor{currentstroke}{rgb}{1.000000,0.000000,0.000000}%
\pgfsetstrokecolor{currentstroke}%
\pgfsetdash{}{0pt}%
\pgfpathmoveto{\pgfqpoint{5.023131in}{0.648781in}}%
\pgfpathcurveto{\pgfqpoint{5.034181in}{0.648781in}}{\pgfqpoint{5.044780in}{0.653171in}}{\pgfqpoint{5.052594in}{0.660985in}}%
\pgfpathcurveto{\pgfqpoint{5.060407in}{0.668799in}}{\pgfqpoint{5.064798in}{0.679398in}}{\pgfqpoint{5.064798in}{0.690448in}}%
\pgfpathcurveto{\pgfqpoint{5.064798in}{0.701498in}}{\pgfqpoint{5.060407in}{0.712097in}}{\pgfqpoint{5.052594in}{0.719911in}}%
\pgfpathcurveto{\pgfqpoint{5.044780in}{0.727724in}}{\pgfqpoint{5.034181in}{0.732114in}}{\pgfqpoint{5.023131in}{0.732114in}}%
\pgfpathcurveto{\pgfqpoint{5.012081in}{0.732114in}}{\pgfqpoint{5.001482in}{0.727724in}}{\pgfqpoint{4.993668in}{0.719911in}}%
\pgfpathcurveto{\pgfqpoint{4.985855in}{0.712097in}}{\pgfqpoint{4.981464in}{0.701498in}}{\pgfqpoint{4.981464in}{0.690448in}}%
\pgfpathcurveto{\pgfqpoint{4.981464in}{0.679398in}}{\pgfqpoint{4.985855in}{0.668799in}}{\pgfqpoint{4.993668in}{0.660985in}}%
\pgfpathcurveto{\pgfqpoint{5.001482in}{0.653171in}}{\pgfqpoint{5.012081in}{0.648781in}}{\pgfqpoint{5.023131in}{0.648781in}}%
\pgfpathlineto{\pgfqpoint{5.023131in}{0.648781in}}%
\pgfpathclose%
\pgfusepath{stroke}%
\end{pgfscope}%
\begin{pgfscope}%
\pgfpathrectangle{\pgfqpoint{0.847223in}{0.554012in}}{\pgfqpoint{6.200000in}{4.620000in}}%
\pgfusepath{clip}%
\pgfsetbuttcap%
\pgfsetroundjoin%
\pgfsetlinewidth{1.003750pt}%
\definecolor{currentstroke}{rgb}{1.000000,0.000000,0.000000}%
\pgfsetstrokecolor{currentstroke}%
\pgfsetdash{}{0pt}%
\pgfpathmoveto{\pgfqpoint{5.028464in}{0.648144in}}%
\pgfpathcurveto{\pgfqpoint{5.039514in}{0.648144in}}{\pgfqpoint{5.050113in}{0.652534in}}{\pgfqpoint{5.057927in}{0.660348in}}%
\pgfpathcurveto{\pgfqpoint{5.065741in}{0.668162in}}{\pgfqpoint{5.070131in}{0.678761in}}{\pgfqpoint{5.070131in}{0.689811in}}%
\pgfpathcurveto{\pgfqpoint{5.070131in}{0.700861in}}{\pgfqpoint{5.065741in}{0.711460in}}{\pgfqpoint{5.057927in}{0.719274in}}%
\pgfpathcurveto{\pgfqpoint{5.050113in}{0.727087in}}{\pgfqpoint{5.039514in}{0.731477in}}{\pgfqpoint{5.028464in}{0.731477in}}%
\pgfpathcurveto{\pgfqpoint{5.017414in}{0.731477in}}{\pgfqpoint{5.006815in}{0.727087in}}{\pgfqpoint{4.999001in}{0.719274in}}%
\pgfpathcurveto{\pgfqpoint{4.991188in}{0.711460in}}{\pgfqpoint{4.986797in}{0.700861in}}{\pgfqpoint{4.986797in}{0.689811in}}%
\pgfpathcurveto{\pgfqpoint{4.986797in}{0.678761in}}{\pgfqpoint{4.991188in}{0.668162in}}{\pgfqpoint{4.999001in}{0.660348in}}%
\pgfpathcurveto{\pgfqpoint{5.006815in}{0.652534in}}{\pgfqpoint{5.017414in}{0.648144in}}{\pgfqpoint{5.028464in}{0.648144in}}%
\pgfpathlineto{\pgfqpoint{5.028464in}{0.648144in}}%
\pgfpathclose%
\pgfusepath{stroke}%
\end{pgfscope}%
\begin{pgfscope}%
\pgfpathrectangle{\pgfqpoint{0.847223in}{0.554012in}}{\pgfqpoint{6.200000in}{4.620000in}}%
\pgfusepath{clip}%
\pgfsetbuttcap%
\pgfsetroundjoin%
\pgfsetlinewidth{1.003750pt}%
\definecolor{currentstroke}{rgb}{1.000000,0.000000,0.000000}%
\pgfsetstrokecolor{currentstroke}%
\pgfsetdash{}{0pt}%
\pgfpathmoveto{\pgfqpoint{5.033797in}{0.647509in}}%
\pgfpathcurveto{\pgfqpoint{5.044847in}{0.647509in}}{\pgfqpoint{5.055447in}{0.651899in}}{\pgfqpoint{5.063260in}{0.659712in}}%
\pgfpathcurveto{\pgfqpoint{5.071074in}{0.667526in}}{\pgfqpoint{5.075464in}{0.678125in}}{\pgfqpoint{5.075464in}{0.689175in}}%
\pgfpathcurveto{\pgfqpoint{5.075464in}{0.700225in}}{\pgfqpoint{5.071074in}{0.710824in}}{\pgfqpoint{5.063260in}{0.718638in}}%
\pgfpathcurveto{\pgfqpoint{5.055447in}{0.726452in}}{\pgfqpoint{5.044847in}{0.730842in}}{\pgfqpoint{5.033797in}{0.730842in}}%
\pgfpathcurveto{\pgfqpoint{5.022747in}{0.730842in}}{\pgfqpoint{5.012148in}{0.726452in}}{\pgfqpoint{5.004335in}{0.718638in}}%
\pgfpathcurveto{\pgfqpoint{4.996521in}{0.710824in}}{\pgfqpoint{4.992131in}{0.700225in}}{\pgfqpoint{4.992131in}{0.689175in}}%
\pgfpathcurveto{\pgfqpoint{4.992131in}{0.678125in}}{\pgfqpoint{4.996521in}{0.667526in}}{\pgfqpoint{5.004335in}{0.659712in}}%
\pgfpathcurveto{\pgfqpoint{5.012148in}{0.651899in}}{\pgfqpoint{5.022747in}{0.647509in}}{\pgfqpoint{5.033797in}{0.647509in}}%
\pgfpathlineto{\pgfqpoint{5.033797in}{0.647509in}}%
\pgfpathclose%
\pgfusepath{stroke}%
\end{pgfscope}%
\begin{pgfscope}%
\pgfpathrectangle{\pgfqpoint{0.847223in}{0.554012in}}{\pgfqpoint{6.200000in}{4.620000in}}%
\pgfusepath{clip}%
\pgfsetbuttcap%
\pgfsetroundjoin%
\pgfsetlinewidth{1.003750pt}%
\definecolor{currentstroke}{rgb}{1.000000,0.000000,0.000000}%
\pgfsetstrokecolor{currentstroke}%
\pgfsetdash{}{0pt}%
\pgfpathmoveto{\pgfqpoint{5.039131in}{0.646874in}}%
\pgfpathcurveto{\pgfqpoint{5.050181in}{0.646874in}}{\pgfqpoint{5.060780in}{0.651265in}}{\pgfqpoint{5.068593in}{0.659078in}}%
\pgfpathcurveto{\pgfqpoint{5.076407in}{0.666892in}}{\pgfqpoint{5.080797in}{0.677491in}}{\pgfqpoint{5.080797in}{0.688541in}}%
\pgfpathcurveto{\pgfqpoint{5.080797in}{0.699591in}}{\pgfqpoint{5.076407in}{0.710190in}}{\pgfqpoint{5.068593in}{0.718004in}}%
\pgfpathcurveto{\pgfqpoint{5.060780in}{0.725818in}}{\pgfqpoint{5.050181in}{0.730208in}}{\pgfqpoint{5.039131in}{0.730208in}}%
\pgfpathcurveto{\pgfqpoint{5.028080in}{0.730208in}}{\pgfqpoint{5.017481in}{0.725818in}}{\pgfqpoint{5.009668in}{0.718004in}}%
\pgfpathcurveto{\pgfqpoint{5.001854in}{0.710190in}}{\pgfqpoint{4.997464in}{0.699591in}}{\pgfqpoint{4.997464in}{0.688541in}}%
\pgfpathcurveto{\pgfqpoint{4.997464in}{0.677491in}}{\pgfqpoint{5.001854in}{0.666892in}}{\pgfqpoint{5.009668in}{0.659078in}}%
\pgfpathcurveto{\pgfqpoint{5.017481in}{0.651265in}}{\pgfqpoint{5.028080in}{0.646874in}}{\pgfqpoint{5.039131in}{0.646874in}}%
\pgfpathlineto{\pgfqpoint{5.039131in}{0.646874in}}%
\pgfpathclose%
\pgfusepath{stroke}%
\end{pgfscope}%
\begin{pgfscope}%
\pgfpathrectangle{\pgfqpoint{0.847223in}{0.554012in}}{\pgfqpoint{6.200000in}{4.620000in}}%
\pgfusepath{clip}%
\pgfsetbuttcap%
\pgfsetroundjoin%
\pgfsetlinewidth{1.003750pt}%
\definecolor{currentstroke}{rgb}{1.000000,0.000000,0.000000}%
\pgfsetstrokecolor{currentstroke}%
\pgfsetdash{}{0pt}%
\pgfpathmoveto{\pgfqpoint{5.044464in}{0.646242in}}%
\pgfpathcurveto{\pgfqpoint{5.055514in}{0.646242in}}{\pgfqpoint{5.066113in}{0.650632in}}{\pgfqpoint{5.073927in}{0.658446in}}%
\pgfpathcurveto{\pgfqpoint{5.081740in}{0.666259in}}{\pgfqpoint{5.086130in}{0.676858in}}{\pgfqpoint{5.086130in}{0.687908in}}%
\pgfpathcurveto{\pgfqpoint{5.086130in}{0.698959in}}{\pgfqpoint{5.081740in}{0.709558in}}{\pgfqpoint{5.073927in}{0.717371in}}%
\pgfpathcurveto{\pgfqpoint{5.066113in}{0.725185in}}{\pgfqpoint{5.055514in}{0.729575in}}{\pgfqpoint{5.044464in}{0.729575in}}%
\pgfpathcurveto{\pgfqpoint{5.033414in}{0.729575in}}{\pgfqpoint{5.022815in}{0.725185in}}{\pgfqpoint{5.015001in}{0.717371in}}%
\pgfpathcurveto{\pgfqpoint{5.007187in}{0.709558in}}{\pgfqpoint{5.002797in}{0.698959in}}{\pgfqpoint{5.002797in}{0.687908in}}%
\pgfpathcurveto{\pgfqpoint{5.002797in}{0.676858in}}{\pgfqpoint{5.007187in}{0.666259in}}{\pgfqpoint{5.015001in}{0.658446in}}%
\pgfpathcurveto{\pgfqpoint{5.022815in}{0.650632in}}{\pgfqpoint{5.033414in}{0.646242in}}{\pgfqpoint{5.044464in}{0.646242in}}%
\pgfpathlineto{\pgfqpoint{5.044464in}{0.646242in}}%
\pgfpathclose%
\pgfusepath{stroke}%
\end{pgfscope}%
\begin{pgfscope}%
\pgfpathrectangle{\pgfqpoint{0.847223in}{0.554012in}}{\pgfqpoint{6.200000in}{4.620000in}}%
\pgfusepath{clip}%
\pgfsetbuttcap%
\pgfsetroundjoin%
\pgfsetlinewidth{1.003750pt}%
\definecolor{currentstroke}{rgb}{1.000000,0.000000,0.000000}%
\pgfsetstrokecolor{currentstroke}%
\pgfsetdash{}{0pt}%
\pgfpathmoveto{\pgfqpoint{5.049797in}{0.645611in}}%
\pgfpathcurveto{\pgfqpoint{5.060847in}{0.645611in}}{\pgfqpoint{5.071446in}{0.650001in}}{\pgfqpoint{5.079260in}{0.657814in}}%
\pgfpathcurveto{\pgfqpoint{5.087073in}{0.665628in}}{\pgfqpoint{5.091464in}{0.676227in}}{\pgfqpoint{5.091464in}{0.687277in}}%
\pgfpathcurveto{\pgfqpoint{5.091464in}{0.698327in}}{\pgfqpoint{5.087073in}{0.708926in}}{\pgfqpoint{5.079260in}{0.716740in}}%
\pgfpathcurveto{\pgfqpoint{5.071446in}{0.724554in}}{\pgfqpoint{5.060847in}{0.728944in}}{\pgfqpoint{5.049797in}{0.728944in}}%
\pgfpathcurveto{\pgfqpoint{5.038747in}{0.728944in}}{\pgfqpoint{5.028148in}{0.724554in}}{\pgfqpoint{5.020334in}{0.716740in}}%
\pgfpathcurveto{\pgfqpoint{5.012521in}{0.708926in}}{\pgfqpoint{5.008130in}{0.698327in}}{\pgfqpoint{5.008130in}{0.687277in}}%
\pgfpathcurveto{\pgfqpoint{5.008130in}{0.676227in}}{\pgfqpoint{5.012521in}{0.665628in}}{\pgfqpoint{5.020334in}{0.657814in}}%
\pgfpathcurveto{\pgfqpoint{5.028148in}{0.650001in}}{\pgfqpoint{5.038747in}{0.645611in}}{\pgfqpoint{5.049797in}{0.645611in}}%
\pgfpathlineto{\pgfqpoint{5.049797in}{0.645611in}}%
\pgfpathclose%
\pgfusepath{stroke}%
\end{pgfscope}%
\begin{pgfscope}%
\pgfpathrectangle{\pgfqpoint{0.847223in}{0.554012in}}{\pgfqpoint{6.200000in}{4.620000in}}%
\pgfusepath{clip}%
\pgfsetbuttcap%
\pgfsetroundjoin%
\pgfsetlinewidth{1.003750pt}%
\definecolor{currentstroke}{rgb}{1.000000,0.000000,0.000000}%
\pgfsetstrokecolor{currentstroke}%
\pgfsetdash{}{0pt}%
\pgfpathmoveto{\pgfqpoint{5.055130in}{0.644981in}}%
\pgfpathcurveto{\pgfqpoint{5.066180in}{0.644981in}}{\pgfqpoint{5.076779in}{0.649371in}}{\pgfqpoint{5.084593in}{0.657185in}}%
\pgfpathcurveto{\pgfqpoint{5.092407in}{0.664998in}}{\pgfqpoint{5.096797in}{0.675597in}}{\pgfqpoint{5.096797in}{0.686647in}}%
\pgfpathcurveto{\pgfqpoint{5.096797in}{0.697698in}}{\pgfqpoint{5.092407in}{0.708297in}}{\pgfqpoint{5.084593in}{0.716110in}}%
\pgfpathcurveto{\pgfqpoint{5.076779in}{0.723924in}}{\pgfqpoint{5.066180in}{0.728314in}}{\pgfqpoint{5.055130in}{0.728314in}}%
\pgfpathcurveto{\pgfqpoint{5.044080in}{0.728314in}}{\pgfqpoint{5.033481in}{0.723924in}}{\pgfqpoint{5.025667in}{0.716110in}}%
\pgfpathcurveto{\pgfqpoint{5.017854in}{0.708297in}}{\pgfqpoint{5.013464in}{0.697698in}}{\pgfqpoint{5.013464in}{0.686647in}}%
\pgfpathcurveto{\pgfqpoint{5.013464in}{0.675597in}}{\pgfqpoint{5.017854in}{0.664998in}}{\pgfqpoint{5.025667in}{0.657185in}}%
\pgfpathcurveto{\pgfqpoint{5.033481in}{0.649371in}}{\pgfqpoint{5.044080in}{0.644981in}}{\pgfqpoint{5.055130in}{0.644981in}}%
\pgfpathlineto{\pgfqpoint{5.055130in}{0.644981in}}%
\pgfpathclose%
\pgfusepath{stroke}%
\end{pgfscope}%
\begin{pgfscope}%
\pgfpathrectangle{\pgfqpoint{0.847223in}{0.554012in}}{\pgfqpoint{6.200000in}{4.620000in}}%
\pgfusepath{clip}%
\pgfsetbuttcap%
\pgfsetroundjoin%
\pgfsetlinewidth{1.003750pt}%
\definecolor{currentstroke}{rgb}{1.000000,0.000000,0.000000}%
\pgfsetstrokecolor{currentstroke}%
\pgfsetdash{}{0pt}%
\pgfpathmoveto{\pgfqpoint{5.060463in}{0.644352in}}%
\pgfpathcurveto{\pgfqpoint{5.071514in}{0.644352in}}{\pgfqpoint{5.082113in}{0.648743in}}{\pgfqpoint{5.089926in}{0.656556in}}%
\pgfpathcurveto{\pgfqpoint{5.097740in}{0.664370in}}{\pgfqpoint{5.102130in}{0.674969in}}{\pgfqpoint{5.102130in}{0.686019in}}%
\pgfpathcurveto{\pgfqpoint{5.102130in}{0.697069in}}{\pgfqpoint{5.097740in}{0.707668in}}{\pgfqpoint{5.089926in}{0.715482in}}%
\pgfpathcurveto{\pgfqpoint{5.082113in}{0.723295in}}{\pgfqpoint{5.071514in}{0.727686in}}{\pgfqpoint{5.060463in}{0.727686in}}%
\pgfpathcurveto{\pgfqpoint{5.049413in}{0.727686in}}{\pgfqpoint{5.038814in}{0.723295in}}{\pgfqpoint{5.031001in}{0.715482in}}%
\pgfpathcurveto{\pgfqpoint{5.023187in}{0.707668in}}{\pgfqpoint{5.018797in}{0.697069in}}{\pgfqpoint{5.018797in}{0.686019in}}%
\pgfpathcurveto{\pgfqpoint{5.018797in}{0.674969in}}{\pgfqpoint{5.023187in}{0.664370in}}{\pgfqpoint{5.031001in}{0.656556in}}%
\pgfpathcurveto{\pgfqpoint{5.038814in}{0.648743in}}{\pgfqpoint{5.049413in}{0.644352in}}{\pgfqpoint{5.060463in}{0.644352in}}%
\pgfpathlineto{\pgfqpoint{5.060463in}{0.644352in}}%
\pgfpathclose%
\pgfusepath{stroke}%
\end{pgfscope}%
\begin{pgfscope}%
\pgfpathrectangle{\pgfqpoint{0.847223in}{0.554012in}}{\pgfqpoint{6.200000in}{4.620000in}}%
\pgfusepath{clip}%
\pgfsetbuttcap%
\pgfsetroundjoin%
\pgfsetlinewidth{1.003750pt}%
\definecolor{currentstroke}{rgb}{1.000000,0.000000,0.000000}%
\pgfsetstrokecolor{currentstroke}%
\pgfsetdash{}{0pt}%
\pgfpathmoveto{\pgfqpoint{5.065797in}{0.643725in}}%
\pgfpathcurveto{\pgfqpoint{5.076847in}{0.643725in}}{\pgfqpoint{5.087446in}{0.648116in}}{\pgfqpoint{5.095259in}{0.655929in}}%
\pgfpathcurveto{\pgfqpoint{5.103073in}{0.663743in}}{\pgfqpoint{5.107463in}{0.674342in}}{\pgfqpoint{5.107463in}{0.685392in}}%
\pgfpathcurveto{\pgfqpoint{5.107463in}{0.696442in}}{\pgfqpoint{5.103073in}{0.707041in}}{\pgfqpoint{5.095259in}{0.714855in}}%
\pgfpathcurveto{\pgfqpoint{5.087446in}{0.722668in}}{\pgfqpoint{5.076847in}{0.727059in}}{\pgfqpoint{5.065797in}{0.727059in}}%
\pgfpathcurveto{\pgfqpoint{5.054747in}{0.727059in}}{\pgfqpoint{5.044147in}{0.722668in}}{\pgfqpoint{5.036334in}{0.714855in}}%
\pgfpathcurveto{\pgfqpoint{5.028520in}{0.707041in}}{\pgfqpoint{5.024130in}{0.696442in}}{\pgfqpoint{5.024130in}{0.685392in}}%
\pgfpathcurveto{\pgfqpoint{5.024130in}{0.674342in}}{\pgfqpoint{5.028520in}{0.663743in}}{\pgfqpoint{5.036334in}{0.655929in}}%
\pgfpathcurveto{\pgfqpoint{5.044147in}{0.648116in}}{\pgfqpoint{5.054747in}{0.643725in}}{\pgfqpoint{5.065797in}{0.643725in}}%
\pgfpathlineto{\pgfqpoint{5.065797in}{0.643725in}}%
\pgfpathclose%
\pgfusepath{stroke}%
\end{pgfscope}%
\begin{pgfscope}%
\pgfpathrectangle{\pgfqpoint{0.847223in}{0.554012in}}{\pgfqpoint{6.200000in}{4.620000in}}%
\pgfusepath{clip}%
\pgfsetbuttcap%
\pgfsetroundjoin%
\pgfsetlinewidth{1.003750pt}%
\definecolor{currentstroke}{rgb}{1.000000,0.000000,0.000000}%
\pgfsetstrokecolor{currentstroke}%
\pgfsetdash{}{0pt}%
\pgfpathmoveto{\pgfqpoint{5.071130in}{0.643100in}}%
\pgfpathcurveto{\pgfqpoint{5.082180in}{0.643100in}}{\pgfqpoint{5.092779in}{0.647490in}}{\pgfqpoint{5.100593in}{0.655304in}}%
\pgfpathcurveto{\pgfqpoint{5.108406in}{0.663117in}}{\pgfqpoint{5.112797in}{0.673716in}}{\pgfqpoint{5.112797in}{0.684766in}}%
\pgfpathcurveto{\pgfqpoint{5.112797in}{0.695817in}}{\pgfqpoint{5.108406in}{0.706416in}}{\pgfqpoint{5.100593in}{0.714229in}}%
\pgfpathcurveto{\pgfqpoint{5.092779in}{0.722043in}}{\pgfqpoint{5.082180in}{0.726433in}}{\pgfqpoint{5.071130in}{0.726433in}}%
\pgfpathcurveto{\pgfqpoint{5.060080in}{0.726433in}}{\pgfqpoint{5.049481in}{0.722043in}}{\pgfqpoint{5.041667in}{0.714229in}}%
\pgfpathcurveto{\pgfqpoint{5.033853in}{0.706416in}}{\pgfqpoint{5.029463in}{0.695817in}}{\pgfqpoint{5.029463in}{0.684766in}}%
\pgfpathcurveto{\pgfqpoint{5.029463in}{0.673716in}}{\pgfqpoint{5.033853in}{0.663117in}}{\pgfqpoint{5.041667in}{0.655304in}}%
\pgfpathcurveto{\pgfqpoint{5.049481in}{0.647490in}}{\pgfqpoint{5.060080in}{0.643100in}}{\pgfqpoint{5.071130in}{0.643100in}}%
\pgfpathlineto{\pgfqpoint{5.071130in}{0.643100in}}%
\pgfpathclose%
\pgfusepath{stroke}%
\end{pgfscope}%
\begin{pgfscope}%
\pgfpathrectangle{\pgfqpoint{0.847223in}{0.554012in}}{\pgfqpoint{6.200000in}{4.620000in}}%
\pgfusepath{clip}%
\pgfsetbuttcap%
\pgfsetroundjoin%
\pgfsetlinewidth{1.003750pt}%
\definecolor{currentstroke}{rgb}{1.000000,0.000000,0.000000}%
\pgfsetstrokecolor{currentstroke}%
\pgfsetdash{}{0pt}%
\pgfpathmoveto{\pgfqpoint{5.076463in}{0.642476in}}%
\pgfpathcurveto{\pgfqpoint{5.087513in}{0.642476in}}{\pgfqpoint{5.098112in}{0.646866in}}{\pgfqpoint{5.105926in}{0.654679in}}%
\pgfpathcurveto{\pgfqpoint{5.113739in}{0.662493in}}{\pgfqpoint{5.118130in}{0.673092in}}{\pgfqpoint{5.118130in}{0.684142in}}%
\pgfpathcurveto{\pgfqpoint{5.118130in}{0.695192in}}{\pgfqpoint{5.113739in}{0.705791in}}{\pgfqpoint{5.105926in}{0.713605in}}%
\pgfpathcurveto{\pgfqpoint{5.098112in}{0.721419in}}{\pgfqpoint{5.087513in}{0.725809in}}{\pgfqpoint{5.076463in}{0.725809in}}%
\pgfpathcurveto{\pgfqpoint{5.065413in}{0.725809in}}{\pgfqpoint{5.054814in}{0.721419in}}{\pgfqpoint{5.047000in}{0.713605in}}%
\pgfpathcurveto{\pgfqpoint{5.039187in}{0.705791in}}{\pgfqpoint{5.034796in}{0.695192in}}{\pgfqpoint{5.034796in}{0.684142in}}%
\pgfpathcurveto{\pgfqpoint{5.034796in}{0.673092in}}{\pgfqpoint{5.039187in}{0.662493in}}{\pgfqpoint{5.047000in}{0.654679in}}%
\pgfpathcurveto{\pgfqpoint{5.054814in}{0.646866in}}{\pgfqpoint{5.065413in}{0.642476in}}{\pgfqpoint{5.076463in}{0.642476in}}%
\pgfpathlineto{\pgfqpoint{5.076463in}{0.642476in}}%
\pgfpathclose%
\pgfusepath{stroke}%
\end{pgfscope}%
\begin{pgfscope}%
\pgfpathrectangle{\pgfqpoint{0.847223in}{0.554012in}}{\pgfqpoint{6.200000in}{4.620000in}}%
\pgfusepath{clip}%
\pgfsetbuttcap%
\pgfsetroundjoin%
\pgfsetlinewidth{1.003750pt}%
\definecolor{currentstroke}{rgb}{1.000000,0.000000,0.000000}%
\pgfsetstrokecolor{currentstroke}%
\pgfsetdash{}{0pt}%
\pgfpathmoveto{\pgfqpoint{5.081796in}{0.641853in}}%
\pgfpathcurveto{\pgfqpoint{5.092846in}{0.641853in}}{\pgfqpoint{5.103445in}{0.646243in}}{\pgfqpoint{5.111259in}{0.654057in}}%
\pgfpathcurveto{\pgfqpoint{5.119073in}{0.661870in}}{\pgfqpoint{5.123463in}{0.672469in}}{\pgfqpoint{5.123463in}{0.683519in}}%
\pgfpathcurveto{\pgfqpoint{5.123463in}{0.694570in}}{\pgfqpoint{5.119073in}{0.705169in}}{\pgfqpoint{5.111259in}{0.712982in}}%
\pgfpathcurveto{\pgfqpoint{5.103445in}{0.720796in}}{\pgfqpoint{5.092846in}{0.725186in}}{\pgfqpoint{5.081796in}{0.725186in}}%
\pgfpathcurveto{\pgfqpoint{5.070746in}{0.725186in}}{\pgfqpoint{5.060147in}{0.720796in}}{\pgfqpoint{5.052334in}{0.712982in}}%
\pgfpathcurveto{\pgfqpoint{5.044520in}{0.705169in}}{\pgfqpoint{5.040130in}{0.694570in}}{\pgfqpoint{5.040130in}{0.683519in}}%
\pgfpathcurveto{\pgfqpoint{5.040130in}{0.672469in}}{\pgfqpoint{5.044520in}{0.661870in}}{\pgfqpoint{5.052334in}{0.654057in}}%
\pgfpathcurveto{\pgfqpoint{5.060147in}{0.646243in}}{\pgfqpoint{5.070746in}{0.641853in}}{\pgfqpoint{5.081796in}{0.641853in}}%
\pgfpathlineto{\pgfqpoint{5.081796in}{0.641853in}}%
\pgfpathclose%
\pgfusepath{stroke}%
\end{pgfscope}%
\begin{pgfscope}%
\pgfpathrectangle{\pgfqpoint{0.847223in}{0.554012in}}{\pgfqpoint{6.200000in}{4.620000in}}%
\pgfusepath{clip}%
\pgfsetbuttcap%
\pgfsetroundjoin%
\pgfsetlinewidth{1.003750pt}%
\definecolor{currentstroke}{rgb}{1.000000,0.000000,0.000000}%
\pgfsetstrokecolor{currentstroke}%
\pgfsetdash{}{0pt}%
\pgfpathmoveto{\pgfqpoint{5.087130in}{0.641231in}}%
\pgfpathcurveto{\pgfqpoint{5.098180in}{0.641231in}}{\pgfqpoint{5.108779in}{0.645622in}}{\pgfqpoint{5.116592in}{0.653435in}}%
\pgfpathcurveto{\pgfqpoint{5.124406in}{0.661249in}}{\pgfqpoint{5.128796in}{0.671848in}}{\pgfqpoint{5.128796in}{0.682898in}}%
\pgfpathcurveto{\pgfqpoint{5.128796in}{0.693948in}}{\pgfqpoint{5.124406in}{0.704547in}}{\pgfqpoint{5.116592in}{0.712361in}}%
\pgfpathcurveto{\pgfqpoint{5.108779in}{0.720174in}}{\pgfqpoint{5.098180in}{0.724565in}}{\pgfqpoint{5.087130in}{0.724565in}}%
\pgfpathcurveto{\pgfqpoint{5.076079in}{0.724565in}}{\pgfqpoint{5.065480in}{0.720174in}}{\pgfqpoint{5.057667in}{0.712361in}}%
\pgfpathcurveto{\pgfqpoint{5.049853in}{0.704547in}}{\pgfqpoint{5.045463in}{0.693948in}}{\pgfqpoint{5.045463in}{0.682898in}}%
\pgfpathcurveto{\pgfqpoint{5.045463in}{0.671848in}}{\pgfqpoint{5.049853in}{0.661249in}}{\pgfqpoint{5.057667in}{0.653435in}}%
\pgfpathcurveto{\pgfqpoint{5.065480in}{0.645622in}}{\pgfqpoint{5.076079in}{0.641231in}}{\pgfqpoint{5.087130in}{0.641231in}}%
\pgfpathlineto{\pgfqpoint{5.087130in}{0.641231in}}%
\pgfpathclose%
\pgfusepath{stroke}%
\end{pgfscope}%
\begin{pgfscope}%
\pgfpathrectangle{\pgfqpoint{0.847223in}{0.554012in}}{\pgfqpoint{6.200000in}{4.620000in}}%
\pgfusepath{clip}%
\pgfsetbuttcap%
\pgfsetroundjoin%
\pgfsetlinewidth{1.003750pt}%
\definecolor{currentstroke}{rgb}{1.000000,0.000000,0.000000}%
\pgfsetstrokecolor{currentstroke}%
\pgfsetdash{}{0pt}%
\pgfpathmoveto{\pgfqpoint{5.092463in}{0.640611in}}%
\pgfpathcurveto{\pgfqpoint{5.103513in}{0.640611in}}{\pgfqpoint{5.114112in}{0.645002in}}{\pgfqpoint{5.121926in}{0.652815in}}%
\pgfpathcurveto{\pgfqpoint{5.129739in}{0.660629in}}{\pgfqpoint{5.134129in}{0.671228in}}{\pgfqpoint{5.134129in}{0.682278in}}%
\pgfpathcurveto{\pgfqpoint{5.134129in}{0.693328in}}{\pgfqpoint{5.129739in}{0.703927in}}{\pgfqpoint{5.121926in}{0.711741in}}%
\pgfpathcurveto{\pgfqpoint{5.114112in}{0.719554in}}{\pgfqpoint{5.103513in}{0.723945in}}{\pgfqpoint{5.092463in}{0.723945in}}%
\pgfpathcurveto{\pgfqpoint{5.081413in}{0.723945in}}{\pgfqpoint{5.070814in}{0.719554in}}{\pgfqpoint{5.063000in}{0.711741in}}%
\pgfpathcurveto{\pgfqpoint{5.055186in}{0.703927in}}{\pgfqpoint{5.050796in}{0.693328in}}{\pgfqpoint{5.050796in}{0.682278in}}%
\pgfpathcurveto{\pgfqpoint{5.050796in}{0.671228in}}{\pgfqpoint{5.055186in}{0.660629in}}{\pgfqpoint{5.063000in}{0.652815in}}%
\pgfpathcurveto{\pgfqpoint{5.070814in}{0.645002in}}{\pgfqpoint{5.081413in}{0.640611in}}{\pgfqpoint{5.092463in}{0.640611in}}%
\pgfpathlineto{\pgfqpoint{5.092463in}{0.640611in}}%
\pgfpathclose%
\pgfusepath{stroke}%
\end{pgfscope}%
\begin{pgfscope}%
\pgfpathrectangle{\pgfqpoint{0.847223in}{0.554012in}}{\pgfqpoint{6.200000in}{4.620000in}}%
\pgfusepath{clip}%
\pgfsetbuttcap%
\pgfsetroundjoin%
\pgfsetlinewidth{1.003750pt}%
\definecolor{currentstroke}{rgb}{1.000000,0.000000,0.000000}%
\pgfsetstrokecolor{currentstroke}%
\pgfsetdash{}{0pt}%
\pgfpathmoveto{\pgfqpoint{5.097796in}{0.639993in}}%
\pgfpathcurveto{\pgfqpoint{5.108846in}{0.639993in}}{\pgfqpoint{5.119445in}{0.644383in}}{\pgfqpoint{5.127259in}{0.652197in}}%
\pgfpathcurveto{\pgfqpoint{5.135072in}{0.660010in}}{\pgfqpoint{5.139463in}{0.670609in}}{\pgfqpoint{5.139463in}{0.681659in}}%
\pgfpathcurveto{\pgfqpoint{5.139463in}{0.692710in}}{\pgfqpoint{5.135072in}{0.703309in}}{\pgfqpoint{5.127259in}{0.711122in}}%
\pgfpathcurveto{\pgfqpoint{5.119445in}{0.718936in}}{\pgfqpoint{5.108846in}{0.723326in}}{\pgfqpoint{5.097796in}{0.723326in}}%
\pgfpathcurveto{\pgfqpoint{5.086746in}{0.723326in}}{\pgfqpoint{5.076147in}{0.718936in}}{\pgfqpoint{5.068333in}{0.711122in}}%
\pgfpathcurveto{\pgfqpoint{5.060520in}{0.703309in}}{\pgfqpoint{5.056129in}{0.692710in}}{\pgfqpoint{5.056129in}{0.681659in}}%
\pgfpathcurveto{\pgfqpoint{5.056129in}{0.670609in}}{\pgfqpoint{5.060520in}{0.660010in}}{\pgfqpoint{5.068333in}{0.652197in}}%
\pgfpathcurveto{\pgfqpoint{5.076147in}{0.644383in}}{\pgfqpoint{5.086746in}{0.639993in}}{\pgfqpoint{5.097796in}{0.639993in}}%
\pgfpathlineto{\pgfqpoint{5.097796in}{0.639993in}}%
\pgfpathclose%
\pgfusepath{stroke}%
\end{pgfscope}%
\begin{pgfscope}%
\pgfpathrectangle{\pgfqpoint{0.847223in}{0.554012in}}{\pgfqpoint{6.200000in}{4.620000in}}%
\pgfusepath{clip}%
\pgfsetbuttcap%
\pgfsetroundjoin%
\pgfsetlinewidth{1.003750pt}%
\definecolor{currentstroke}{rgb}{1.000000,0.000000,0.000000}%
\pgfsetstrokecolor{currentstroke}%
\pgfsetdash{}{0pt}%
\pgfpathmoveto{\pgfqpoint{5.103129in}{0.639376in}}%
\pgfpathcurveto{\pgfqpoint{5.114179in}{0.639376in}}{\pgfqpoint{5.124778in}{0.643766in}}{\pgfqpoint{5.132592in}{0.651579in}}%
\pgfpathcurveto{\pgfqpoint{5.140406in}{0.659393in}}{\pgfqpoint{5.144796in}{0.669992in}}{\pgfqpoint{5.144796in}{0.681042in}}%
\pgfpathcurveto{\pgfqpoint{5.144796in}{0.692092in}}{\pgfqpoint{5.140406in}{0.702691in}}{\pgfqpoint{5.132592in}{0.710505in}}%
\pgfpathcurveto{\pgfqpoint{5.124778in}{0.718319in}}{\pgfqpoint{5.114179in}{0.722709in}}{\pgfqpoint{5.103129in}{0.722709in}}%
\pgfpathcurveto{\pgfqpoint{5.092079in}{0.722709in}}{\pgfqpoint{5.081480in}{0.718319in}}{\pgfqpoint{5.073666in}{0.710505in}}%
\pgfpathcurveto{\pgfqpoint{5.065853in}{0.702691in}}{\pgfqpoint{5.061462in}{0.692092in}}{\pgfqpoint{5.061462in}{0.681042in}}%
\pgfpathcurveto{\pgfqpoint{5.061462in}{0.669992in}}{\pgfqpoint{5.065853in}{0.659393in}}{\pgfqpoint{5.073666in}{0.651579in}}%
\pgfpathcurveto{\pgfqpoint{5.081480in}{0.643766in}}{\pgfqpoint{5.092079in}{0.639376in}}{\pgfqpoint{5.103129in}{0.639376in}}%
\pgfpathlineto{\pgfqpoint{5.103129in}{0.639376in}}%
\pgfpathclose%
\pgfusepath{stroke}%
\end{pgfscope}%
\begin{pgfscope}%
\pgfpathrectangle{\pgfqpoint{0.847223in}{0.554012in}}{\pgfqpoint{6.200000in}{4.620000in}}%
\pgfusepath{clip}%
\pgfsetbuttcap%
\pgfsetroundjoin%
\pgfsetlinewidth{1.003750pt}%
\definecolor{currentstroke}{rgb}{1.000000,0.000000,0.000000}%
\pgfsetstrokecolor{currentstroke}%
\pgfsetdash{}{0pt}%
\pgfpathmoveto{\pgfqpoint{5.108462in}{0.638760in}}%
\pgfpathcurveto{\pgfqpoint{5.119512in}{0.638760in}}{\pgfqpoint{5.130112in}{0.643150in}}{\pgfqpoint{5.137925in}{0.650964in}}%
\pgfpathcurveto{\pgfqpoint{5.145739in}{0.658777in}}{\pgfqpoint{5.150129in}{0.669376in}}{\pgfqpoint{5.150129in}{0.680426in}}%
\pgfpathcurveto{\pgfqpoint{5.150129in}{0.691476in}}{\pgfqpoint{5.145739in}{0.702076in}}{\pgfqpoint{5.137925in}{0.709889in}}%
\pgfpathcurveto{\pgfqpoint{5.130112in}{0.717703in}}{\pgfqpoint{5.119512in}{0.722093in}}{\pgfqpoint{5.108462in}{0.722093in}}%
\pgfpathcurveto{\pgfqpoint{5.097412in}{0.722093in}}{\pgfqpoint{5.086813in}{0.717703in}}{\pgfqpoint{5.079000in}{0.709889in}}%
\pgfpathcurveto{\pgfqpoint{5.071186in}{0.702076in}}{\pgfqpoint{5.066796in}{0.691476in}}{\pgfqpoint{5.066796in}{0.680426in}}%
\pgfpathcurveto{\pgfqpoint{5.066796in}{0.669376in}}{\pgfqpoint{5.071186in}{0.658777in}}{\pgfqpoint{5.079000in}{0.650964in}}%
\pgfpathcurveto{\pgfqpoint{5.086813in}{0.643150in}}{\pgfqpoint{5.097412in}{0.638760in}}{\pgfqpoint{5.108462in}{0.638760in}}%
\pgfpathlineto{\pgfqpoint{5.108462in}{0.638760in}}%
\pgfpathclose%
\pgfusepath{stroke}%
\end{pgfscope}%
\begin{pgfscope}%
\pgfpathrectangle{\pgfqpoint{0.847223in}{0.554012in}}{\pgfqpoint{6.200000in}{4.620000in}}%
\pgfusepath{clip}%
\pgfsetbuttcap%
\pgfsetroundjoin%
\pgfsetlinewidth{1.003750pt}%
\definecolor{currentstroke}{rgb}{1.000000,0.000000,0.000000}%
\pgfsetstrokecolor{currentstroke}%
\pgfsetdash{}{0pt}%
\pgfpathmoveto{\pgfqpoint{5.113796in}{0.638145in}}%
\pgfpathcurveto{\pgfqpoint{5.124846in}{0.638145in}}{\pgfqpoint{5.135445in}{0.642535in}}{\pgfqpoint{5.143258in}{0.650349in}}%
\pgfpathcurveto{\pgfqpoint{5.151072in}{0.658163in}}{\pgfqpoint{5.155462in}{0.668762in}}{\pgfqpoint{5.155462in}{0.679812in}}%
\pgfpathcurveto{\pgfqpoint{5.155462in}{0.690862in}}{\pgfqpoint{5.151072in}{0.701461in}}{\pgfqpoint{5.143258in}{0.709275in}}%
\pgfpathcurveto{\pgfqpoint{5.135445in}{0.717088in}}{\pgfqpoint{5.124846in}{0.721479in}}{\pgfqpoint{5.113796in}{0.721479in}}%
\pgfpathcurveto{\pgfqpoint{5.102745in}{0.721479in}}{\pgfqpoint{5.092146in}{0.717088in}}{\pgfqpoint{5.084333in}{0.709275in}}%
\pgfpathcurveto{\pgfqpoint{5.076519in}{0.701461in}}{\pgfqpoint{5.072129in}{0.690862in}}{\pgfqpoint{5.072129in}{0.679812in}}%
\pgfpathcurveto{\pgfqpoint{5.072129in}{0.668762in}}{\pgfqpoint{5.076519in}{0.658163in}}{\pgfqpoint{5.084333in}{0.650349in}}%
\pgfpathcurveto{\pgfqpoint{5.092146in}{0.642535in}}{\pgfqpoint{5.102745in}{0.638145in}}{\pgfqpoint{5.113796in}{0.638145in}}%
\pgfpathlineto{\pgfqpoint{5.113796in}{0.638145in}}%
\pgfpathclose%
\pgfusepath{stroke}%
\end{pgfscope}%
\begin{pgfscope}%
\pgfpathrectangle{\pgfqpoint{0.847223in}{0.554012in}}{\pgfqpoint{6.200000in}{4.620000in}}%
\pgfusepath{clip}%
\pgfsetbuttcap%
\pgfsetroundjoin%
\pgfsetlinewidth{1.003750pt}%
\definecolor{currentstroke}{rgb}{1.000000,0.000000,0.000000}%
\pgfsetstrokecolor{currentstroke}%
\pgfsetdash{}{0pt}%
\pgfpathmoveto{\pgfqpoint{5.119129in}{0.637532in}}%
\pgfpathcurveto{\pgfqpoint{5.130179in}{0.637532in}}{\pgfqpoint{5.140778in}{0.641922in}}{\pgfqpoint{5.148592in}{0.649736in}}%
\pgfpathcurveto{\pgfqpoint{5.156405in}{0.657550in}}{\pgfqpoint{5.160795in}{0.668149in}}{\pgfqpoint{5.160795in}{0.679199in}}%
\pgfpathcurveto{\pgfqpoint{5.160795in}{0.690249in}}{\pgfqpoint{5.156405in}{0.700848in}}{\pgfqpoint{5.148592in}{0.708662in}}%
\pgfpathcurveto{\pgfqpoint{5.140778in}{0.716475in}}{\pgfqpoint{5.130179in}{0.720865in}}{\pgfqpoint{5.119129in}{0.720865in}}%
\pgfpathcurveto{\pgfqpoint{5.108079in}{0.720865in}}{\pgfqpoint{5.097480in}{0.716475in}}{\pgfqpoint{5.089666in}{0.708662in}}%
\pgfpathcurveto{\pgfqpoint{5.081852in}{0.700848in}}{\pgfqpoint{5.077462in}{0.690249in}}{\pgfqpoint{5.077462in}{0.679199in}}%
\pgfpathcurveto{\pgfqpoint{5.077462in}{0.668149in}}{\pgfqpoint{5.081852in}{0.657550in}}{\pgfqpoint{5.089666in}{0.649736in}}%
\pgfpathcurveto{\pgfqpoint{5.097480in}{0.641922in}}{\pgfqpoint{5.108079in}{0.637532in}}{\pgfqpoint{5.119129in}{0.637532in}}%
\pgfpathlineto{\pgfqpoint{5.119129in}{0.637532in}}%
\pgfpathclose%
\pgfusepath{stroke}%
\end{pgfscope}%
\begin{pgfscope}%
\pgfpathrectangle{\pgfqpoint{0.847223in}{0.554012in}}{\pgfqpoint{6.200000in}{4.620000in}}%
\pgfusepath{clip}%
\pgfsetbuttcap%
\pgfsetroundjoin%
\pgfsetlinewidth{1.003750pt}%
\definecolor{currentstroke}{rgb}{1.000000,0.000000,0.000000}%
\pgfsetstrokecolor{currentstroke}%
\pgfsetdash{}{0pt}%
\pgfpathmoveto{\pgfqpoint{5.124462in}{0.636920in}}%
\pgfpathcurveto{\pgfqpoint{5.135512in}{0.636920in}}{\pgfqpoint{5.146111in}{0.641311in}}{\pgfqpoint{5.153925in}{0.649124in}}%
\pgfpathcurveto{\pgfqpoint{5.161738in}{0.656938in}}{\pgfqpoint{5.166129in}{0.667537in}}{\pgfqpoint{5.166129in}{0.678587in}}%
\pgfpathcurveto{\pgfqpoint{5.166129in}{0.689637in}}{\pgfqpoint{5.161738in}{0.700236in}}{\pgfqpoint{5.153925in}{0.708050in}}%
\pgfpathcurveto{\pgfqpoint{5.146111in}{0.715863in}}{\pgfqpoint{5.135512in}{0.720254in}}{\pgfqpoint{5.124462in}{0.720254in}}%
\pgfpathcurveto{\pgfqpoint{5.113412in}{0.720254in}}{\pgfqpoint{5.102813in}{0.715863in}}{\pgfqpoint{5.094999in}{0.708050in}}%
\pgfpathcurveto{\pgfqpoint{5.087186in}{0.700236in}}{\pgfqpoint{5.082795in}{0.689637in}}{\pgfqpoint{5.082795in}{0.678587in}}%
\pgfpathcurveto{\pgfqpoint{5.082795in}{0.667537in}}{\pgfqpoint{5.087186in}{0.656938in}}{\pgfqpoint{5.094999in}{0.649124in}}%
\pgfpathcurveto{\pgfqpoint{5.102813in}{0.641311in}}{\pgfqpoint{5.113412in}{0.636920in}}{\pgfqpoint{5.124462in}{0.636920in}}%
\pgfpathlineto{\pgfqpoint{5.124462in}{0.636920in}}%
\pgfpathclose%
\pgfusepath{stroke}%
\end{pgfscope}%
\begin{pgfscope}%
\pgfpathrectangle{\pgfqpoint{0.847223in}{0.554012in}}{\pgfqpoint{6.200000in}{4.620000in}}%
\pgfusepath{clip}%
\pgfsetbuttcap%
\pgfsetroundjoin%
\pgfsetlinewidth{1.003750pt}%
\definecolor{currentstroke}{rgb}{1.000000,0.000000,0.000000}%
\pgfsetstrokecolor{currentstroke}%
\pgfsetdash{}{0pt}%
\pgfpathmoveto{\pgfqpoint{5.129795in}{0.636310in}}%
\pgfpathcurveto{\pgfqpoint{5.140845in}{0.636310in}}{\pgfqpoint{5.151444in}{0.640700in}}{\pgfqpoint{5.159258in}{0.648514in}}%
\pgfpathcurveto{\pgfqpoint{5.167072in}{0.656327in}}{\pgfqpoint{5.171462in}{0.666926in}}{\pgfqpoint{5.171462in}{0.677977in}}%
\pgfpathcurveto{\pgfqpoint{5.171462in}{0.689027in}}{\pgfqpoint{5.167072in}{0.699626in}}{\pgfqpoint{5.159258in}{0.707439in}}%
\pgfpathcurveto{\pgfqpoint{5.151444in}{0.715253in}}{\pgfqpoint{5.140845in}{0.719643in}}{\pgfqpoint{5.129795in}{0.719643in}}%
\pgfpathcurveto{\pgfqpoint{5.118745in}{0.719643in}}{\pgfqpoint{5.108146in}{0.715253in}}{\pgfqpoint{5.100332in}{0.707439in}}%
\pgfpathcurveto{\pgfqpoint{5.092519in}{0.699626in}}{\pgfqpoint{5.088129in}{0.689027in}}{\pgfqpoint{5.088129in}{0.677977in}}%
\pgfpathcurveto{\pgfqpoint{5.088129in}{0.666926in}}{\pgfqpoint{5.092519in}{0.656327in}}{\pgfqpoint{5.100332in}{0.648514in}}%
\pgfpathcurveto{\pgfqpoint{5.108146in}{0.640700in}}{\pgfqpoint{5.118745in}{0.636310in}}{\pgfqpoint{5.129795in}{0.636310in}}%
\pgfpathlineto{\pgfqpoint{5.129795in}{0.636310in}}%
\pgfpathclose%
\pgfusepath{stroke}%
\end{pgfscope}%
\begin{pgfscope}%
\pgfpathrectangle{\pgfqpoint{0.847223in}{0.554012in}}{\pgfqpoint{6.200000in}{4.620000in}}%
\pgfusepath{clip}%
\pgfsetbuttcap%
\pgfsetroundjoin%
\pgfsetlinewidth{1.003750pt}%
\definecolor{currentstroke}{rgb}{1.000000,0.000000,0.000000}%
\pgfsetstrokecolor{currentstroke}%
\pgfsetdash{}{0pt}%
\pgfpathmoveto{\pgfqpoint{5.135128in}{0.635701in}}%
\pgfpathcurveto{\pgfqpoint{5.146179in}{0.635701in}}{\pgfqpoint{5.156778in}{0.640091in}}{\pgfqpoint{5.164591in}{0.647905in}}%
\pgfpathcurveto{\pgfqpoint{5.172405in}{0.655718in}}{\pgfqpoint{5.176795in}{0.666317in}}{\pgfqpoint{5.176795in}{0.677368in}}%
\pgfpathcurveto{\pgfqpoint{5.176795in}{0.688418in}}{\pgfqpoint{5.172405in}{0.699017in}}{\pgfqpoint{5.164591in}{0.706830in}}%
\pgfpathcurveto{\pgfqpoint{5.156778in}{0.714644in}}{\pgfqpoint{5.146179in}{0.719034in}}{\pgfqpoint{5.135128in}{0.719034in}}%
\pgfpathcurveto{\pgfqpoint{5.124078in}{0.719034in}}{\pgfqpoint{5.113479in}{0.714644in}}{\pgfqpoint{5.105666in}{0.706830in}}%
\pgfpathcurveto{\pgfqpoint{5.097852in}{0.699017in}}{\pgfqpoint{5.093462in}{0.688418in}}{\pgfqpoint{5.093462in}{0.677368in}}%
\pgfpathcurveto{\pgfqpoint{5.093462in}{0.666317in}}{\pgfqpoint{5.097852in}{0.655718in}}{\pgfqpoint{5.105666in}{0.647905in}}%
\pgfpathcurveto{\pgfqpoint{5.113479in}{0.640091in}}{\pgfqpoint{5.124078in}{0.635701in}}{\pgfqpoint{5.135128in}{0.635701in}}%
\pgfpathlineto{\pgfqpoint{5.135128in}{0.635701in}}%
\pgfpathclose%
\pgfusepath{stroke}%
\end{pgfscope}%
\begin{pgfscope}%
\pgfpathrectangle{\pgfqpoint{0.847223in}{0.554012in}}{\pgfqpoint{6.200000in}{4.620000in}}%
\pgfusepath{clip}%
\pgfsetbuttcap%
\pgfsetroundjoin%
\pgfsetlinewidth{1.003750pt}%
\definecolor{currentstroke}{rgb}{1.000000,0.000000,0.000000}%
\pgfsetstrokecolor{currentstroke}%
\pgfsetdash{}{0pt}%
\pgfpathmoveto{\pgfqpoint{5.140462in}{0.635093in}}%
\pgfpathcurveto{\pgfqpoint{5.151512in}{0.635093in}}{\pgfqpoint{5.162111in}{0.639483in}}{\pgfqpoint{5.169924in}{0.647297in}}%
\pgfpathcurveto{\pgfqpoint{5.177738in}{0.655111in}}{\pgfqpoint{5.182128in}{0.665710in}}{\pgfqpoint{5.182128in}{0.676760in}}%
\pgfpathcurveto{\pgfqpoint{5.182128in}{0.687810in}}{\pgfqpoint{5.177738in}{0.698409in}}{\pgfqpoint{5.169924in}{0.706223in}}%
\pgfpathcurveto{\pgfqpoint{5.162111in}{0.714036in}}{\pgfqpoint{5.151512in}{0.718427in}}{\pgfqpoint{5.140462in}{0.718427in}}%
\pgfpathcurveto{\pgfqpoint{5.129412in}{0.718427in}}{\pgfqpoint{5.118813in}{0.714036in}}{\pgfqpoint{5.110999in}{0.706223in}}%
\pgfpathcurveto{\pgfqpoint{5.103185in}{0.698409in}}{\pgfqpoint{5.098795in}{0.687810in}}{\pgfqpoint{5.098795in}{0.676760in}}%
\pgfpathcurveto{\pgfqpoint{5.098795in}{0.665710in}}{\pgfqpoint{5.103185in}{0.655111in}}{\pgfqpoint{5.110999in}{0.647297in}}%
\pgfpathcurveto{\pgfqpoint{5.118813in}{0.639483in}}{\pgfqpoint{5.129412in}{0.635093in}}{\pgfqpoint{5.140462in}{0.635093in}}%
\pgfpathlineto{\pgfqpoint{5.140462in}{0.635093in}}%
\pgfpathclose%
\pgfusepath{stroke}%
\end{pgfscope}%
\begin{pgfscope}%
\pgfpathrectangle{\pgfqpoint{0.847223in}{0.554012in}}{\pgfqpoint{6.200000in}{4.620000in}}%
\pgfusepath{clip}%
\pgfsetbuttcap%
\pgfsetroundjoin%
\pgfsetlinewidth{1.003750pt}%
\definecolor{currentstroke}{rgb}{1.000000,0.000000,0.000000}%
\pgfsetstrokecolor{currentstroke}%
\pgfsetdash{}{0pt}%
\pgfpathmoveto{\pgfqpoint{5.145795in}{0.634487in}}%
\pgfpathcurveto{\pgfqpoint{5.156845in}{0.634487in}}{\pgfqpoint{5.167444in}{0.638877in}}{\pgfqpoint{5.175258in}{0.646691in}}%
\pgfpathcurveto{\pgfqpoint{5.183071in}{0.654504in}}{\pgfqpoint{5.187462in}{0.665103in}}{\pgfqpoint{5.187462in}{0.676154in}}%
\pgfpathcurveto{\pgfqpoint{5.187462in}{0.687204in}}{\pgfqpoint{5.183071in}{0.697803in}}{\pgfqpoint{5.175258in}{0.705616in}}%
\pgfpathcurveto{\pgfqpoint{5.167444in}{0.713430in}}{\pgfqpoint{5.156845in}{0.717820in}}{\pgfqpoint{5.145795in}{0.717820in}}%
\pgfpathcurveto{\pgfqpoint{5.134745in}{0.717820in}}{\pgfqpoint{5.124146in}{0.713430in}}{\pgfqpoint{5.116332in}{0.705616in}}%
\pgfpathcurveto{\pgfqpoint{5.108518in}{0.697803in}}{\pgfqpoint{5.104128in}{0.687204in}}{\pgfqpoint{5.104128in}{0.676154in}}%
\pgfpathcurveto{\pgfqpoint{5.104128in}{0.665103in}}{\pgfqpoint{5.108518in}{0.654504in}}{\pgfqpoint{5.116332in}{0.646691in}}%
\pgfpathcurveto{\pgfqpoint{5.124146in}{0.638877in}}{\pgfqpoint{5.134745in}{0.634487in}}{\pgfqpoint{5.145795in}{0.634487in}}%
\pgfpathlineto{\pgfqpoint{5.145795in}{0.634487in}}%
\pgfpathclose%
\pgfusepath{stroke}%
\end{pgfscope}%
\begin{pgfscope}%
\pgfpathrectangle{\pgfqpoint{0.847223in}{0.554012in}}{\pgfqpoint{6.200000in}{4.620000in}}%
\pgfusepath{clip}%
\pgfsetbuttcap%
\pgfsetroundjoin%
\pgfsetlinewidth{1.003750pt}%
\definecolor{currentstroke}{rgb}{1.000000,0.000000,0.000000}%
\pgfsetstrokecolor{currentstroke}%
\pgfsetdash{}{0pt}%
\pgfpathmoveto{\pgfqpoint{5.151128in}{0.633882in}}%
\pgfpathcurveto{\pgfqpoint{5.162178in}{0.633882in}}{\pgfqpoint{5.172777in}{0.638272in}}{\pgfqpoint{5.180591in}{0.646086in}}%
\pgfpathcurveto{\pgfqpoint{5.188404in}{0.653899in}}{\pgfqpoint{5.192795in}{0.664498in}}{\pgfqpoint{5.192795in}{0.675549in}}%
\pgfpathcurveto{\pgfqpoint{5.192795in}{0.686599in}}{\pgfqpoint{5.188404in}{0.697198in}}{\pgfqpoint{5.180591in}{0.705011in}}%
\pgfpathcurveto{\pgfqpoint{5.172777in}{0.712825in}}{\pgfqpoint{5.162178in}{0.717215in}}{\pgfqpoint{5.151128in}{0.717215in}}%
\pgfpathcurveto{\pgfqpoint{5.140078in}{0.717215in}}{\pgfqpoint{5.129479in}{0.712825in}}{\pgfqpoint{5.121665in}{0.705011in}}%
\pgfpathcurveto{\pgfqpoint{5.113852in}{0.697198in}}{\pgfqpoint{5.109461in}{0.686599in}}{\pgfqpoint{5.109461in}{0.675549in}}%
\pgfpathcurveto{\pgfqpoint{5.109461in}{0.664498in}}{\pgfqpoint{5.113852in}{0.653899in}}{\pgfqpoint{5.121665in}{0.646086in}}%
\pgfpathcurveto{\pgfqpoint{5.129479in}{0.638272in}}{\pgfqpoint{5.140078in}{0.633882in}}{\pgfqpoint{5.151128in}{0.633882in}}%
\pgfpathlineto{\pgfqpoint{5.151128in}{0.633882in}}%
\pgfpathclose%
\pgfusepath{stroke}%
\end{pgfscope}%
\begin{pgfscope}%
\pgfpathrectangle{\pgfqpoint{0.847223in}{0.554012in}}{\pgfqpoint{6.200000in}{4.620000in}}%
\pgfusepath{clip}%
\pgfsetbuttcap%
\pgfsetroundjoin%
\pgfsetlinewidth{1.003750pt}%
\definecolor{currentstroke}{rgb}{1.000000,0.000000,0.000000}%
\pgfsetstrokecolor{currentstroke}%
\pgfsetdash{}{0pt}%
\pgfpathmoveto{\pgfqpoint{5.156461in}{0.633278in}}%
\pgfpathcurveto{\pgfqpoint{5.167511in}{0.633278in}}{\pgfqpoint{5.178110in}{0.637668in}}{\pgfqpoint{5.185924in}{0.645482in}}%
\pgfpathcurveto{\pgfqpoint{5.193738in}{0.653296in}}{\pgfqpoint{5.198128in}{0.663895in}}{\pgfqpoint{5.198128in}{0.674945in}}%
\pgfpathcurveto{\pgfqpoint{5.198128in}{0.685995in}}{\pgfqpoint{5.193738in}{0.696594in}}{\pgfqpoint{5.185924in}{0.704408in}}%
\pgfpathcurveto{\pgfqpoint{5.178110in}{0.712221in}}{\pgfqpoint{5.167511in}{0.716612in}}{\pgfqpoint{5.156461in}{0.716612in}}%
\pgfpathcurveto{\pgfqpoint{5.145411in}{0.716612in}}{\pgfqpoint{5.134812in}{0.712221in}}{\pgfqpoint{5.126999in}{0.704408in}}%
\pgfpathcurveto{\pgfqpoint{5.119185in}{0.696594in}}{\pgfqpoint{5.114795in}{0.685995in}}{\pgfqpoint{5.114795in}{0.674945in}}%
\pgfpathcurveto{\pgfqpoint{5.114795in}{0.663895in}}{\pgfqpoint{5.119185in}{0.653296in}}{\pgfqpoint{5.126999in}{0.645482in}}%
\pgfpathcurveto{\pgfqpoint{5.134812in}{0.637668in}}{\pgfqpoint{5.145411in}{0.633278in}}{\pgfqpoint{5.156461in}{0.633278in}}%
\pgfpathlineto{\pgfqpoint{5.156461in}{0.633278in}}%
\pgfpathclose%
\pgfusepath{stroke}%
\end{pgfscope}%
\begin{pgfscope}%
\pgfpathrectangle{\pgfqpoint{0.847223in}{0.554012in}}{\pgfqpoint{6.200000in}{4.620000in}}%
\pgfusepath{clip}%
\pgfsetbuttcap%
\pgfsetroundjoin%
\pgfsetlinewidth{1.003750pt}%
\definecolor{currentstroke}{rgb}{1.000000,0.000000,0.000000}%
\pgfsetstrokecolor{currentstroke}%
\pgfsetdash{}{0pt}%
\pgfpathmoveto{\pgfqpoint{5.161795in}{0.632676in}}%
\pgfpathcurveto{\pgfqpoint{5.172845in}{0.632676in}}{\pgfqpoint{5.183444in}{0.637066in}}{\pgfqpoint{5.191257in}{0.644880in}}%
\pgfpathcurveto{\pgfqpoint{5.199071in}{0.652693in}}{\pgfqpoint{5.203461in}{0.663292in}}{\pgfqpoint{5.203461in}{0.674342in}}%
\pgfpathcurveto{\pgfqpoint{5.203461in}{0.685393in}}{\pgfqpoint{5.199071in}{0.695992in}}{\pgfqpoint{5.191257in}{0.703805in}}%
\pgfpathcurveto{\pgfqpoint{5.183444in}{0.711619in}}{\pgfqpoint{5.172845in}{0.716009in}}{\pgfqpoint{5.161795in}{0.716009in}}%
\pgfpathcurveto{\pgfqpoint{5.150744in}{0.716009in}}{\pgfqpoint{5.140145in}{0.711619in}}{\pgfqpoint{5.132332in}{0.703805in}}%
\pgfpathcurveto{\pgfqpoint{5.124518in}{0.695992in}}{\pgfqpoint{5.120128in}{0.685393in}}{\pgfqpoint{5.120128in}{0.674342in}}%
\pgfpathcurveto{\pgfqpoint{5.120128in}{0.663292in}}{\pgfqpoint{5.124518in}{0.652693in}}{\pgfqpoint{5.132332in}{0.644880in}}%
\pgfpathcurveto{\pgfqpoint{5.140145in}{0.637066in}}{\pgfqpoint{5.150744in}{0.632676in}}{\pgfqpoint{5.161795in}{0.632676in}}%
\pgfpathlineto{\pgfqpoint{5.161795in}{0.632676in}}%
\pgfpathclose%
\pgfusepath{stroke}%
\end{pgfscope}%
\begin{pgfscope}%
\pgfpathrectangle{\pgfqpoint{0.847223in}{0.554012in}}{\pgfqpoint{6.200000in}{4.620000in}}%
\pgfusepath{clip}%
\pgfsetbuttcap%
\pgfsetroundjoin%
\pgfsetlinewidth{1.003750pt}%
\definecolor{currentstroke}{rgb}{1.000000,0.000000,0.000000}%
\pgfsetstrokecolor{currentstroke}%
\pgfsetdash{}{0pt}%
\pgfpathmoveto{\pgfqpoint{5.167128in}{0.632075in}}%
\pgfpathcurveto{\pgfqpoint{5.178178in}{0.632075in}}{\pgfqpoint{5.188777in}{0.636465in}}{\pgfqpoint{5.196591in}{0.644279in}}%
\pgfpathcurveto{\pgfqpoint{5.204404in}{0.652092in}}{\pgfqpoint{5.208794in}{0.662691in}}{\pgfqpoint{5.208794in}{0.673741in}}%
\pgfpathcurveto{\pgfqpoint{5.208794in}{0.684792in}}{\pgfqpoint{5.204404in}{0.695391in}}{\pgfqpoint{5.196591in}{0.703204in}}%
\pgfpathcurveto{\pgfqpoint{5.188777in}{0.711018in}}{\pgfqpoint{5.178178in}{0.715408in}}{\pgfqpoint{5.167128in}{0.715408in}}%
\pgfpathcurveto{\pgfqpoint{5.156078in}{0.715408in}}{\pgfqpoint{5.145479in}{0.711018in}}{\pgfqpoint{5.137665in}{0.703204in}}%
\pgfpathcurveto{\pgfqpoint{5.129851in}{0.695391in}}{\pgfqpoint{5.125461in}{0.684792in}}{\pgfqpoint{5.125461in}{0.673741in}}%
\pgfpathcurveto{\pgfqpoint{5.125461in}{0.662691in}}{\pgfqpoint{5.129851in}{0.652092in}}{\pgfqpoint{5.137665in}{0.644279in}}%
\pgfpathcurveto{\pgfqpoint{5.145479in}{0.636465in}}{\pgfqpoint{5.156078in}{0.632075in}}{\pgfqpoint{5.167128in}{0.632075in}}%
\pgfpathlineto{\pgfqpoint{5.167128in}{0.632075in}}%
\pgfpathclose%
\pgfusepath{stroke}%
\end{pgfscope}%
\begin{pgfscope}%
\pgfpathrectangle{\pgfqpoint{0.847223in}{0.554012in}}{\pgfqpoint{6.200000in}{4.620000in}}%
\pgfusepath{clip}%
\pgfsetbuttcap%
\pgfsetroundjoin%
\pgfsetlinewidth{1.003750pt}%
\definecolor{currentstroke}{rgb}{1.000000,0.000000,0.000000}%
\pgfsetstrokecolor{currentstroke}%
\pgfsetdash{}{0pt}%
\pgfpathmoveto{\pgfqpoint{5.172461in}{0.631475in}}%
\pgfpathcurveto{\pgfqpoint{5.183511in}{0.631475in}}{\pgfqpoint{5.194110in}{0.635865in}}{\pgfqpoint{5.201924in}{0.643679in}}%
\pgfpathcurveto{\pgfqpoint{5.209737in}{0.651493in}}{\pgfqpoint{5.214128in}{0.662092in}}{\pgfqpoint{5.214128in}{0.673142in}}%
\pgfpathcurveto{\pgfqpoint{5.214128in}{0.684192in}}{\pgfqpoint{5.209737in}{0.694791in}}{\pgfqpoint{5.201924in}{0.702605in}}%
\pgfpathcurveto{\pgfqpoint{5.194110in}{0.710418in}}{\pgfqpoint{5.183511in}{0.714808in}}{\pgfqpoint{5.172461in}{0.714808in}}%
\pgfpathcurveto{\pgfqpoint{5.161411in}{0.714808in}}{\pgfqpoint{5.150812in}{0.710418in}}{\pgfqpoint{5.142998in}{0.702605in}}%
\pgfpathcurveto{\pgfqpoint{5.135185in}{0.694791in}}{\pgfqpoint{5.130794in}{0.684192in}}{\pgfqpoint{5.130794in}{0.673142in}}%
\pgfpathcurveto{\pgfqpoint{5.130794in}{0.662092in}}{\pgfqpoint{5.135185in}{0.651493in}}{\pgfqpoint{5.142998in}{0.643679in}}%
\pgfpathcurveto{\pgfqpoint{5.150812in}{0.635865in}}{\pgfqpoint{5.161411in}{0.631475in}}{\pgfqpoint{5.172461in}{0.631475in}}%
\pgfpathlineto{\pgfqpoint{5.172461in}{0.631475in}}%
\pgfpathclose%
\pgfusepath{stroke}%
\end{pgfscope}%
\begin{pgfscope}%
\pgfpathrectangle{\pgfqpoint{0.847223in}{0.554012in}}{\pgfqpoint{6.200000in}{4.620000in}}%
\pgfusepath{clip}%
\pgfsetbuttcap%
\pgfsetroundjoin%
\pgfsetlinewidth{1.003750pt}%
\definecolor{currentstroke}{rgb}{1.000000,0.000000,0.000000}%
\pgfsetstrokecolor{currentstroke}%
\pgfsetdash{}{0pt}%
\pgfpathmoveto{\pgfqpoint{5.177794in}{0.630877in}}%
\pgfpathcurveto{\pgfqpoint{5.188844in}{0.630877in}}{\pgfqpoint{5.199443in}{0.635267in}}{\pgfqpoint{5.207257in}{0.643081in}}%
\pgfpathcurveto{\pgfqpoint{5.215071in}{0.650894in}}{\pgfqpoint{5.219461in}{0.661493in}}{\pgfqpoint{5.219461in}{0.672543in}}%
\pgfpathcurveto{\pgfqpoint{5.219461in}{0.683594in}}{\pgfqpoint{5.215071in}{0.694193in}}{\pgfqpoint{5.207257in}{0.702006in}}%
\pgfpathcurveto{\pgfqpoint{5.199443in}{0.709820in}}{\pgfqpoint{5.188844in}{0.714210in}}{\pgfqpoint{5.177794in}{0.714210in}}%
\pgfpathcurveto{\pgfqpoint{5.166744in}{0.714210in}}{\pgfqpoint{5.156145in}{0.709820in}}{\pgfqpoint{5.148331in}{0.702006in}}%
\pgfpathcurveto{\pgfqpoint{5.140518in}{0.694193in}}{\pgfqpoint{5.136128in}{0.683594in}}{\pgfqpoint{5.136128in}{0.672543in}}%
\pgfpathcurveto{\pgfqpoint{5.136128in}{0.661493in}}{\pgfqpoint{5.140518in}{0.650894in}}{\pgfqpoint{5.148331in}{0.643081in}}%
\pgfpathcurveto{\pgfqpoint{5.156145in}{0.635267in}}{\pgfqpoint{5.166744in}{0.630877in}}{\pgfqpoint{5.177794in}{0.630877in}}%
\pgfpathlineto{\pgfqpoint{5.177794in}{0.630877in}}%
\pgfpathclose%
\pgfusepath{stroke}%
\end{pgfscope}%
\begin{pgfscope}%
\pgfpathrectangle{\pgfqpoint{0.847223in}{0.554012in}}{\pgfqpoint{6.200000in}{4.620000in}}%
\pgfusepath{clip}%
\pgfsetbuttcap%
\pgfsetroundjoin%
\pgfsetlinewidth{1.003750pt}%
\definecolor{currentstroke}{rgb}{1.000000,0.000000,0.000000}%
\pgfsetstrokecolor{currentstroke}%
\pgfsetdash{}{0pt}%
\pgfpathmoveto{\pgfqpoint{5.183127in}{0.630280in}}%
\pgfpathcurveto{\pgfqpoint{5.194178in}{0.630280in}}{\pgfqpoint{5.204777in}{0.634670in}}{\pgfqpoint{5.212590in}{0.642484in}}%
\pgfpathcurveto{\pgfqpoint{5.220404in}{0.650297in}}{\pgfqpoint{5.224794in}{0.660896in}}{\pgfqpoint{5.224794in}{0.671946in}}%
\pgfpathcurveto{\pgfqpoint{5.224794in}{0.682996in}}{\pgfqpoint{5.220404in}{0.693596in}}{\pgfqpoint{5.212590in}{0.701409in}}%
\pgfpathcurveto{\pgfqpoint{5.204777in}{0.709223in}}{\pgfqpoint{5.194178in}{0.713613in}}{\pgfqpoint{5.183127in}{0.713613in}}%
\pgfpathcurveto{\pgfqpoint{5.172077in}{0.713613in}}{\pgfqpoint{5.161478in}{0.709223in}}{\pgfqpoint{5.153665in}{0.701409in}}%
\pgfpathcurveto{\pgfqpoint{5.145851in}{0.693596in}}{\pgfqpoint{5.141461in}{0.682996in}}{\pgfqpoint{5.141461in}{0.671946in}}%
\pgfpathcurveto{\pgfqpoint{5.141461in}{0.660896in}}{\pgfqpoint{5.145851in}{0.650297in}}{\pgfqpoint{5.153665in}{0.642484in}}%
\pgfpathcurveto{\pgfqpoint{5.161478in}{0.634670in}}{\pgfqpoint{5.172077in}{0.630280in}}{\pgfqpoint{5.183127in}{0.630280in}}%
\pgfpathlineto{\pgfqpoint{5.183127in}{0.630280in}}%
\pgfpathclose%
\pgfusepath{stroke}%
\end{pgfscope}%
\begin{pgfscope}%
\pgfpathrectangle{\pgfqpoint{0.847223in}{0.554012in}}{\pgfqpoint{6.200000in}{4.620000in}}%
\pgfusepath{clip}%
\pgfsetbuttcap%
\pgfsetroundjoin%
\pgfsetlinewidth{1.003750pt}%
\definecolor{currentstroke}{rgb}{1.000000,0.000000,0.000000}%
\pgfsetstrokecolor{currentstroke}%
\pgfsetdash{}{0pt}%
\pgfpathmoveto{\pgfqpoint{5.188461in}{0.629684in}}%
\pgfpathcurveto{\pgfqpoint{5.199511in}{0.629684in}}{\pgfqpoint{5.210110in}{0.634074in}}{\pgfqpoint{5.217923in}{0.641888in}}%
\pgfpathcurveto{\pgfqpoint{5.225737in}{0.649701in}}{\pgfqpoint{5.230127in}{0.660300in}}{\pgfqpoint{5.230127in}{0.671351in}}%
\pgfpathcurveto{\pgfqpoint{5.230127in}{0.682401in}}{\pgfqpoint{5.225737in}{0.693000in}}{\pgfqpoint{5.217923in}{0.700813in}}%
\pgfpathcurveto{\pgfqpoint{5.210110in}{0.708627in}}{\pgfqpoint{5.199511in}{0.713017in}}{\pgfqpoint{5.188461in}{0.713017in}}%
\pgfpathcurveto{\pgfqpoint{5.177410in}{0.713017in}}{\pgfqpoint{5.166811in}{0.708627in}}{\pgfqpoint{5.158998in}{0.700813in}}%
\pgfpathcurveto{\pgfqpoint{5.151184in}{0.693000in}}{\pgfqpoint{5.146794in}{0.682401in}}{\pgfqpoint{5.146794in}{0.671351in}}%
\pgfpathcurveto{\pgfqpoint{5.146794in}{0.660300in}}{\pgfqpoint{5.151184in}{0.649701in}}{\pgfqpoint{5.158998in}{0.641888in}}%
\pgfpathcurveto{\pgfqpoint{5.166811in}{0.634074in}}{\pgfqpoint{5.177410in}{0.629684in}}{\pgfqpoint{5.188461in}{0.629684in}}%
\pgfpathlineto{\pgfqpoint{5.188461in}{0.629684in}}%
\pgfpathclose%
\pgfusepath{stroke}%
\end{pgfscope}%
\begin{pgfscope}%
\pgfpathrectangle{\pgfqpoint{0.847223in}{0.554012in}}{\pgfqpoint{6.200000in}{4.620000in}}%
\pgfusepath{clip}%
\pgfsetbuttcap%
\pgfsetroundjoin%
\pgfsetlinewidth{1.003750pt}%
\definecolor{currentstroke}{rgb}{1.000000,0.000000,0.000000}%
\pgfsetstrokecolor{currentstroke}%
\pgfsetdash{}{0pt}%
\pgfpathmoveto{\pgfqpoint{5.193794in}{0.629089in}}%
\pgfpathcurveto{\pgfqpoint{5.204844in}{0.629089in}}{\pgfqpoint{5.215443in}{0.633480in}}{\pgfqpoint{5.223257in}{0.641293in}}%
\pgfpathcurveto{\pgfqpoint{5.231070in}{0.649107in}}{\pgfqpoint{5.235460in}{0.659706in}}{\pgfqpoint{5.235460in}{0.670756in}}%
\pgfpathcurveto{\pgfqpoint{5.235460in}{0.681806in}}{\pgfqpoint{5.231070in}{0.692405in}}{\pgfqpoint{5.223257in}{0.700219in}}%
\pgfpathcurveto{\pgfqpoint{5.215443in}{0.708033in}}{\pgfqpoint{5.204844in}{0.712423in}}{\pgfqpoint{5.193794in}{0.712423in}}%
\pgfpathcurveto{\pgfqpoint{5.182744in}{0.712423in}}{\pgfqpoint{5.172145in}{0.708033in}}{\pgfqpoint{5.164331in}{0.700219in}}%
\pgfpathcurveto{\pgfqpoint{5.156517in}{0.692405in}}{\pgfqpoint{5.152127in}{0.681806in}}{\pgfqpoint{5.152127in}{0.670756in}}%
\pgfpathcurveto{\pgfqpoint{5.152127in}{0.659706in}}{\pgfqpoint{5.156517in}{0.649107in}}{\pgfqpoint{5.164331in}{0.641293in}}%
\pgfpathcurveto{\pgfqpoint{5.172145in}{0.633480in}}{\pgfqpoint{5.182744in}{0.629089in}}{\pgfqpoint{5.193794in}{0.629089in}}%
\pgfpathlineto{\pgfqpoint{5.193794in}{0.629089in}}%
\pgfpathclose%
\pgfusepath{stroke}%
\end{pgfscope}%
\begin{pgfscope}%
\pgfpathrectangle{\pgfqpoint{0.847223in}{0.554012in}}{\pgfqpoint{6.200000in}{4.620000in}}%
\pgfusepath{clip}%
\pgfsetbuttcap%
\pgfsetroundjoin%
\pgfsetlinewidth{1.003750pt}%
\definecolor{currentstroke}{rgb}{1.000000,0.000000,0.000000}%
\pgfsetstrokecolor{currentstroke}%
\pgfsetdash{}{0pt}%
\pgfpathmoveto{\pgfqpoint{5.199127in}{0.628496in}}%
\pgfpathcurveto{\pgfqpoint{5.210177in}{0.628496in}}{\pgfqpoint{5.220776in}{0.632887in}}{\pgfqpoint{5.228590in}{0.640700in}}%
\pgfpathcurveto{\pgfqpoint{5.236403in}{0.648514in}}{\pgfqpoint{5.240794in}{0.659113in}}{\pgfqpoint{5.240794in}{0.670163in}}%
\pgfpathcurveto{\pgfqpoint{5.240794in}{0.681213in}}{\pgfqpoint{5.236403in}{0.691812in}}{\pgfqpoint{5.228590in}{0.699626in}}%
\pgfpathcurveto{\pgfqpoint{5.220776in}{0.707439in}}{\pgfqpoint{5.210177in}{0.711830in}}{\pgfqpoint{5.199127in}{0.711830in}}%
\pgfpathcurveto{\pgfqpoint{5.188077in}{0.711830in}}{\pgfqpoint{5.177478in}{0.707439in}}{\pgfqpoint{5.169664in}{0.699626in}}%
\pgfpathcurveto{\pgfqpoint{5.161851in}{0.691812in}}{\pgfqpoint{5.157460in}{0.681213in}}{\pgfqpoint{5.157460in}{0.670163in}}%
\pgfpathcurveto{\pgfqpoint{5.157460in}{0.659113in}}{\pgfqpoint{5.161851in}{0.648514in}}{\pgfqpoint{5.169664in}{0.640700in}}%
\pgfpathcurveto{\pgfqpoint{5.177478in}{0.632887in}}{\pgfqpoint{5.188077in}{0.628496in}}{\pgfqpoint{5.199127in}{0.628496in}}%
\pgfpathlineto{\pgfqpoint{5.199127in}{0.628496in}}%
\pgfpathclose%
\pgfusepath{stroke}%
\end{pgfscope}%
\begin{pgfscope}%
\pgfpathrectangle{\pgfqpoint{0.847223in}{0.554012in}}{\pgfqpoint{6.200000in}{4.620000in}}%
\pgfusepath{clip}%
\pgfsetbuttcap%
\pgfsetroundjoin%
\pgfsetlinewidth{1.003750pt}%
\definecolor{currentstroke}{rgb}{1.000000,0.000000,0.000000}%
\pgfsetstrokecolor{currentstroke}%
\pgfsetdash{}{0pt}%
\pgfpathmoveto{\pgfqpoint{5.204460in}{0.627904in}}%
\pgfpathcurveto{\pgfqpoint{5.215510in}{0.627904in}}{\pgfqpoint{5.226109in}{0.632295in}}{\pgfqpoint{5.233923in}{0.640108in}}%
\pgfpathcurveto{\pgfqpoint{5.241737in}{0.647922in}}{\pgfqpoint{5.246127in}{0.658521in}}{\pgfqpoint{5.246127in}{0.669571in}}%
\pgfpathcurveto{\pgfqpoint{5.246127in}{0.680621in}}{\pgfqpoint{5.241737in}{0.691220in}}{\pgfqpoint{5.233923in}{0.699034in}}%
\pgfpathcurveto{\pgfqpoint{5.226109in}{0.706848in}}{\pgfqpoint{5.215510in}{0.711238in}}{\pgfqpoint{5.204460in}{0.711238in}}%
\pgfpathcurveto{\pgfqpoint{5.193410in}{0.711238in}}{\pgfqpoint{5.182811in}{0.706848in}}{\pgfqpoint{5.174997in}{0.699034in}}%
\pgfpathcurveto{\pgfqpoint{5.167184in}{0.691220in}}{\pgfqpoint{5.162794in}{0.680621in}}{\pgfqpoint{5.162794in}{0.669571in}}%
\pgfpathcurveto{\pgfqpoint{5.162794in}{0.658521in}}{\pgfqpoint{5.167184in}{0.647922in}}{\pgfqpoint{5.174997in}{0.640108in}}%
\pgfpathcurveto{\pgfqpoint{5.182811in}{0.632295in}}{\pgfqpoint{5.193410in}{0.627904in}}{\pgfqpoint{5.204460in}{0.627904in}}%
\pgfpathlineto{\pgfqpoint{5.204460in}{0.627904in}}%
\pgfpathclose%
\pgfusepath{stroke}%
\end{pgfscope}%
\begin{pgfscope}%
\pgfpathrectangle{\pgfqpoint{0.847223in}{0.554012in}}{\pgfqpoint{6.200000in}{4.620000in}}%
\pgfusepath{clip}%
\pgfsetbuttcap%
\pgfsetroundjoin%
\pgfsetlinewidth{1.003750pt}%
\definecolor{currentstroke}{rgb}{1.000000,0.000000,0.000000}%
\pgfsetstrokecolor{currentstroke}%
\pgfsetdash{}{0pt}%
\pgfpathmoveto{\pgfqpoint{5.209793in}{0.627314in}}%
\pgfpathcurveto{\pgfqpoint{5.220844in}{0.627314in}}{\pgfqpoint{5.231443in}{0.631704in}}{\pgfqpoint{5.239256in}{0.639518in}}%
\pgfpathcurveto{\pgfqpoint{5.247070in}{0.647331in}}{\pgfqpoint{5.251460in}{0.657930in}}{\pgfqpoint{5.251460in}{0.668981in}}%
\pgfpathcurveto{\pgfqpoint{5.251460in}{0.680031in}}{\pgfqpoint{5.247070in}{0.690630in}}{\pgfqpoint{5.239256in}{0.698443in}}%
\pgfpathcurveto{\pgfqpoint{5.231443in}{0.706257in}}{\pgfqpoint{5.220844in}{0.710647in}}{\pgfqpoint{5.209793in}{0.710647in}}%
\pgfpathcurveto{\pgfqpoint{5.198743in}{0.710647in}}{\pgfqpoint{5.188144in}{0.706257in}}{\pgfqpoint{5.180331in}{0.698443in}}%
\pgfpathcurveto{\pgfqpoint{5.172517in}{0.690630in}}{\pgfqpoint{5.168127in}{0.680031in}}{\pgfqpoint{5.168127in}{0.668981in}}%
\pgfpathcurveto{\pgfqpoint{5.168127in}{0.657930in}}{\pgfqpoint{5.172517in}{0.647331in}}{\pgfqpoint{5.180331in}{0.639518in}}%
\pgfpathcurveto{\pgfqpoint{5.188144in}{0.631704in}}{\pgfqpoint{5.198743in}{0.627314in}}{\pgfqpoint{5.209793in}{0.627314in}}%
\pgfpathlineto{\pgfqpoint{5.209793in}{0.627314in}}%
\pgfpathclose%
\pgfusepath{stroke}%
\end{pgfscope}%
\begin{pgfscope}%
\pgfpathrectangle{\pgfqpoint{0.847223in}{0.554012in}}{\pgfqpoint{6.200000in}{4.620000in}}%
\pgfusepath{clip}%
\pgfsetbuttcap%
\pgfsetroundjoin%
\pgfsetlinewidth{1.003750pt}%
\definecolor{currentstroke}{rgb}{1.000000,0.000000,0.000000}%
\pgfsetstrokecolor{currentstroke}%
\pgfsetdash{}{0pt}%
\pgfpathmoveto{\pgfqpoint{5.215127in}{0.626725in}}%
\pgfpathcurveto{\pgfqpoint{5.226177in}{0.626725in}}{\pgfqpoint{5.236776in}{0.631115in}}{\pgfqpoint{5.244589in}{0.638929in}}%
\pgfpathcurveto{\pgfqpoint{5.252403in}{0.646742in}}{\pgfqpoint{5.256793in}{0.657341in}}{\pgfqpoint{5.256793in}{0.668391in}}%
\pgfpathcurveto{\pgfqpoint{5.256793in}{0.679441in}}{\pgfqpoint{5.252403in}{0.690040in}}{\pgfqpoint{5.244589in}{0.697854in}}%
\pgfpathcurveto{\pgfqpoint{5.236776in}{0.705668in}}{\pgfqpoint{5.226177in}{0.710058in}}{\pgfqpoint{5.215127in}{0.710058in}}%
\pgfpathcurveto{\pgfqpoint{5.204077in}{0.710058in}}{\pgfqpoint{5.193478in}{0.705668in}}{\pgfqpoint{5.185664in}{0.697854in}}%
\pgfpathcurveto{\pgfqpoint{5.177850in}{0.690040in}}{\pgfqpoint{5.173460in}{0.679441in}}{\pgfqpoint{5.173460in}{0.668391in}}%
\pgfpathcurveto{\pgfqpoint{5.173460in}{0.657341in}}{\pgfqpoint{5.177850in}{0.646742in}}{\pgfqpoint{5.185664in}{0.638929in}}%
\pgfpathcurveto{\pgfqpoint{5.193478in}{0.631115in}}{\pgfqpoint{5.204077in}{0.626725in}}{\pgfqpoint{5.215127in}{0.626725in}}%
\pgfpathlineto{\pgfqpoint{5.215127in}{0.626725in}}%
\pgfpathclose%
\pgfusepath{stroke}%
\end{pgfscope}%
\begin{pgfscope}%
\pgfpathrectangle{\pgfqpoint{0.847223in}{0.554012in}}{\pgfqpoint{6.200000in}{4.620000in}}%
\pgfusepath{clip}%
\pgfsetbuttcap%
\pgfsetroundjoin%
\pgfsetlinewidth{1.003750pt}%
\definecolor{currentstroke}{rgb}{1.000000,0.000000,0.000000}%
\pgfsetstrokecolor{currentstroke}%
\pgfsetdash{}{0pt}%
\pgfpathmoveto{\pgfqpoint{5.220460in}{0.626137in}}%
\pgfpathcurveto{\pgfqpoint{5.231510in}{0.626137in}}{\pgfqpoint{5.242109in}{0.630527in}}{\pgfqpoint{5.249923in}{0.638341in}}%
\pgfpathcurveto{\pgfqpoint{5.257736in}{0.646154in}}{\pgfqpoint{5.262127in}{0.656753in}}{\pgfqpoint{5.262127in}{0.667803in}}%
\pgfpathcurveto{\pgfqpoint{5.262127in}{0.678853in}}{\pgfqpoint{5.257736in}{0.689452in}}{\pgfqpoint{5.249923in}{0.697266in}}%
\pgfpathcurveto{\pgfqpoint{5.242109in}{0.705080in}}{\pgfqpoint{5.231510in}{0.709470in}}{\pgfqpoint{5.220460in}{0.709470in}}%
\pgfpathcurveto{\pgfqpoint{5.209410in}{0.709470in}}{\pgfqpoint{5.198811in}{0.705080in}}{\pgfqpoint{5.190997in}{0.697266in}}%
\pgfpathcurveto{\pgfqpoint{5.183183in}{0.689452in}}{\pgfqpoint{5.178793in}{0.678853in}}{\pgfqpoint{5.178793in}{0.667803in}}%
\pgfpathcurveto{\pgfqpoint{5.178793in}{0.656753in}}{\pgfqpoint{5.183183in}{0.646154in}}{\pgfqpoint{5.190997in}{0.638341in}}%
\pgfpathcurveto{\pgfqpoint{5.198811in}{0.630527in}}{\pgfqpoint{5.209410in}{0.626137in}}{\pgfqpoint{5.220460in}{0.626137in}}%
\pgfpathlineto{\pgfqpoint{5.220460in}{0.626137in}}%
\pgfpathclose%
\pgfusepath{stroke}%
\end{pgfscope}%
\begin{pgfscope}%
\pgfpathrectangle{\pgfqpoint{0.847223in}{0.554012in}}{\pgfqpoint{6.200000in}{4.620000in}}%
\pgfusepath{clip}%
\pgfsetbuttcap%
\pgfsetroundjoin%
\pgfsetlinewidth{1.003750pt}%
\definecolor{currentstroke}{rgb}{1.000000,0.000000,0.000000}%
\pgfsetstrokecolor{currentstroke}%
\pgfsetdash{}{0pt}%
\pgfpathmoveto{\pgfqpoint{5.225793in}{0.625550in}}%
\pgfpathcurveto{\pgfqpoint{5.236843in}{0.625550in}}{\pgfqpoint{5.247442in}{0.629940in}}{\pgfqpoint{5.255256in}{0.637754in}}%
\pgfpathcurveto{\pgfqpoint{5.263070in}{0.645567in}}{\pgfqpoint{5.267460in}{0.656166in}}{\pgfqpoint{5.267460in}{0.667217in}}%
\pgfpathcurveto{\pgfqpoint{5.267460in}{0.678267in}}{\pgfqpoint{5.263070in}{0.688866in}}{\pgfqpoint{5.255256in}{0.696679in}}%
\pgfpathcurveto{\pgfqpoint{5.247442in}{0.704493in}}{\pgfqpoint{5.236843in}{0.708883in}}{\pgfqpoint{5.225793in}{0.708883in}}%
\pgfpathcurveto{\pgfqpoint{5.214743in}{0.708883in}}{\pgfqpoint{5.204144in}{0.704493in}}{\pgfqpoint{5.196330in}{0.696679in}}%
\pgfpathcurveto{\pgfqpoint{5.188517in}{0.688866in}}{\pgfqpoint{5.184126in}{0.678267in}}{\pgfqpoint{5.184126in}{0.667217in}}%
\pgfpathcurveto{\pgfqpoint{5.184126in}{0.656166in}}{\pgfqpoint{5.188517in}{0.645567in}}{\pgfqpoint{5.196330in}{0.637754in}}%
\pgfpathcurveto{\pgfqpoint{5.204144in}{0.629940in}}{\pgfqpoint{5.214743in}{0.625550in}}{\pgfqpoint{5.225793in}{0.625550in}}%
\pgfpathlineto{\pgfqpoint{5.225793in}{0.625550in}}%
\pgfpathclose%
\pgfusepath{stroke}%
\end{pgfscope}%
\begin{pgfscope}%
\pgfpathrectangle{\pgfqpoint{0.847223in}{0.554012in}}{\pgfqpoint{6.200000in}{4.620000in}}%
\pgfusepath{clip}%
\pgfsetbuttcap%
\pgfsetroundjoin%
\pgfsetlinewidth{1.003750pt}%
\definecolor{currentstroke}{rgb}{1.000000,0.000000,0.000000}%
\pgfsetstrokecolor{currentstroke}%
\pgfsetdash{}{0pt}%
\pgfpathmoveto{\pgfqpoint{5.231126in}{0.624964in}}%
\pgfpathcurveto{\pgfqpoint{5.242176in}{0.624964in}}{\pgfqpoint{5.252775in}{0.629355in}}{\pgfqpoint{5.260589in}{0.637168in}}%
\pgfpathcurveto{\pgfqpoint{5.268403in}{0.644982in}}{\pgfqpoint{5.272793in}{0.655581in}}{\pgfqpoint{5.272793in}{0.666631in}}%
\pgfpathcurveto{\pgfqpoint{5.272793in}{0.677681in}}{\pgfqpoint{5.268403in}{0.688280in}}{\pgfqpoint{5.260589in}{0.696094in}}%
\pgfpathcurveto{\pgfqpoint{5.252775in}{0.703908in}}{\pgfqpoint{5.242176in}{0.708298in}}{\pgfqpoint{5.231126in}{0.708298in}}%
\pgfpathcurveto{\pgfqpoint{5.220076in}{0.708298in}}{\pgfqpoint{5.209477in}{0.703908in}}{\pgfqpoint{5.201664in}{0.696094in}}%
\pgfpathcurveto{\pgfqpoint{5.193850in}{0.688280in}}{\pgfqpoint{5.189460in}{0.677681in}}{\pgfqpoint{5.189460in}{0.666631in}}%
\pgfpathcurveto{\pgfqpoint{5.189460in}{0.655581in}}{\pgfqpoint{5.193850in}{0.644982in}}{\pgfqpoint{5.201664in}{0.637168in}}%
\pgfpathcurveto{\pgfqpoint{5.209477in}{0.629355in}}{\pgfqpoint{5.220076in}{0.624964in}}{\pgfqpoint{5.231126in}{0.624964in}}%
\pgfpathlineto{\pgfqpoint{5.231126in}{0.624964in}}%
\pgfpathclose%
\pgfusepath{stroke}%
\end{pgfscope}%
\begin{pgfscope}%
\pgfpathrectangle{\pgfqpoint{0.847223in}{0.554012in}}{\pgfqpoint{6.200000in}{4.620000in}}%
\pgfusepath{clip}%
\pgfsetbuttcap%
\pgfsetroundjoin%
\pgfsetlinewidth{1.003750pt}%
\definecolor{currentstroke}{rgb}{1.000000,0.000000,0.000000}%
\pgfsetstrokecolor{currentstroke}%
\pgfsetdash{}{0pt}%
\pgfpathmoveto{\pgfqpoint{5.236460in}{0.624380in}}%
\pgfpathcurveto{\pgfqpoint{5.247510in}{0.624380in}}{\pgfqpoint{5.258109in}{0.628771in}}{\pgfqpoint{5.265922in}{0.636584in}}%
\pgfpathcurveto{\pgfqpoint{5.273736in}{0.644398in}}{\pgfqpoint{5.278126in}{0.654997in}}{\pgfqpoint{5.278126in}{0.666047in}}%
\pgfpathcurveto{\pgfqpoint{5.278126in}{0.677097in}}{\pgfqpoint{5.273736in}{0.687696in}}{\pgfqpoint{5.265922in}{0.695510in}}%
\pgfpathcurveto{\pgfqpoint{5.258109in}{0.703323in}}{\pgfqpoint{5.247510in}{0.707714in}}{\pgfqpoint{5.236460in}{0.707714in}}%
\pgfpathcurveto{\pgfqpoint{5.225409in}{0.707714in}}{\pgfqpoint{5.214810in}{0.703323in}}{\pgfqpoint{5.206997in}{0.695510in}}%
\pgfpathcurveto{\pgfqpoint{5.199183in}{0.687696in}}{\pgfqpoint{5.194793in}{0.677097in}}{\pgfqpoint{5.194793in}{0.666047in}}%
\pgfpathcurveto{\pgfqpoint{5.194793in}{0.654997in}}{\pgfqpoint{5.199183in}{0.644398in}}{\pgfqpoint{5.206997in}{0.636584in}}%
\pgfpathcurveto{\pgfqpoint{5.214810in}{0.628771in}}{\pgfqpoint{5.225409in}{0.624380in}}{\pgfqpoint{5.236460in}{0.624380in}}%
\pgfpathlineto{\pgfqpoint{5.236460in}{0.624380in}}%
\pgfpathclose%
\pgfusepath{stroke}%
\end{pgfscope}%
\begin{pgfscope}%
\pgfpathrectangle{\pgfqpoint{0.847223in}{0.554012in}}{\pgfqpoint{6.200000in}{4.620000in}}%
\pgfusepath{clip}%
\pgfsetbuttcap%
\pgfsetroundjoin%
\pgfsetlinewidth{1.003750pt}%
\definecolor{currentstroke}{rgb}{1.000000,0.000000,0.000000}%
\pgfsetstrokecolor{currentstroke}%
\pgfsetdash{}{0pt}%
\pgfpathmoveto{\pgfqpoint{5.241793in}{0.623797in}}%
\pgfpathcurveto{\pgfqpoint{5.252843in}{0.623797in}}{\pgfqpoint{5.263442in}{0.628188in}}{\pgfqpoint{5.271256in}{0.636001in}}%
\pgfpathcurveto{\pgfqpoint{5.279069in}{0.643815in}}{\pgfqpoint{5.283459in}{0.654414in}}{\pgfqpoint{5.283459in}{0.665464in}}%
\pgfpathcurveto{\pgfqpoint{5.283459in}{0.676514in}}{\pgfqpoint{5.279069in}{0.687113in}}{\pgfqpoint{5.271256in}{0.694927in}}%
\pgfpathcurveto{\pgfqpoint{5.263442in}{0.702741in}}{\pgfqpoint{5.252843in}{0.707131in}}{\pgfqpoint{5.241793in}{0.707131in}}%
\pgfpathcurveto{\pgfqpoint{5.230743in}{0.707131in}}{\pgfqpoint{5.220144in}{0.702741in}}{\pgfqpoint{5.212330in}{0.694927in}}%
\pgfpathcurveto{\pgfqpoint{5.204516in}{0.687113in}}{\pgfqpoint{5.200126in}{0.676514in}}{\pgfqpoint{5.200126in}{0.665464in}}%
\pgfpathcurveto{\pgfqpoint{5.200126in}{0.654414in}}{\pgfqpoint{5.204516in}{0.643815in}}{\pgfqpoint{5.212330in}{0.636001in}}%
\pgfpathcurveto{\pgfqpoint{5.220144in}{0.628188in}}{\pgfqpoint{5.230743in}{0.623797in}}{\pgfqpoint{5.241793in}{0.623797in}}%
\pgfpathlineto{\pgfqpoint{5.241793in}{0.623797in}}%
\pgfpathclose%
\pgfusepath{stroke}%
\end{pgfscope}%
\begin{pgfscope}%
\pgfpathrectangle{\pgfqpoint{0.847223in}{0.554012in}}{\pgfqpoint{6.200000in}{4.620000in}}%
\pgfusepath{clip}%
\pgfsetbuttcap%
\pgfsetroundjoin%
\pgfsetlinewidth{1.003750pt}%
\definecolor{currentstroke}{rgb}{1.000000,0.000000,0.000000}%
\pgfsetstrokecolor{currentstroke}%
\pgfsetdash{}{0pt}%
\pgfpathmoveto{\pgfqpoint{5.247126in}{0.623216in}}%
\pgfpathcurveto{\pgfqpoint{5.258176in}{0.623216in}}{\pgfqpoint{5.268775in}{0.627606in}}{\pgfqpoint{5.276589in}{0.635420in}}%
\pgfpathcurveto{\pgfqpoint{5.284402in}{0.643233in}}{\pgfqpoint{5.288793in}{0.653832in}}{\pgfqpoint{5.288793in}{0.664882in}}%
\pgfpathcurveto{\pgfqpoint{5.288793in}{0.675933in}}{\pgfqpoint{5.284402in}{0.686532in}}{\pgfqpoint{5.276589in}{0.694345in}}%
\pgfpathcurveto{\pgfqpoint{5.268775in}{0.702159in}}{\pgfqpoint{5.258176in}{0.706549in}}{\pgfqpoint{5.247126in}{0.706549in}}%
\pgfpathcurveto{\pgfqpoint{5.236076in}{0.706549in}}{\pgfqpoint{5.225477in}{0.702159in}}{\pgfqpoint{5.217663in}{0.694345in}}%
\pgfpathcurveto{\pgfqpoint{5.209850in}{0.686532in}}{\pgfqpoint{5.205459in}{0.675933in}}{\pgfqpoint{5.205459in}{0.664882in}}%
\pgfpathcurveto{\pgfqpoint{5.205459in}{0.653832in}}{\pgfqpoint{5.209850in}{0.643233in}}{\pgfqpoint{5.217663in}{0.635420in}}%
\pgfpathcurveto{\pgfqpoint{5.225477in}{0.627606in}}{\pgfqpoint{5.236076in}{0.623216in}}{\pgfqpoint{5.247126in}{0.623216in}}%
\pgfpathlineto{\pgfqpoint{5.247126in}{0.623216in}}%
\pgfpathclose%
\pgfusepath{stroke}%
\end{pgfscope}%
\begin{pgfscope}%
\pgfpathrectangle{\pgfqpoint{0.847223in}{0.554012in}}{\pgfqpoint{6.200000in}{4.620000in}}%
\pgfusepath{clip}%
\pgfsetbuttcap%
\pgfsetroundjoin%
\pgfsetlinewidth{1.003750pt}%
\definecolor{currentstroke}{rgb}{1.000000,0.000000,0.000000}%
\pgfsetstrokecolor{currentstroke}%
\pgfsetdash{}{0pt}%
\pgfpathmoveto{\pgfqpoint{5.252459in}{0.622635in}}%
\pgfpathcurveto{\pgfqpoint{5.263509in}{0.622635in}}{\pgfqpoint{5.274108in}{0.627026in}}{\pgfqpoint{5.281922in}{0.634839in}}%
\pgfpathcurveto{\pgfqpoint{5.289736in}{0.642653in}}{\pgfqpoint{5.294126in}{0.653252in}}{\pgfqpoint{5.294126in}{0.664302in}}%
\pgfpathcurveto{\pgfqpoint{5.294126in}{0.675352in}}{\pgfqpoint{5.289736in}{0.685951in}}{\pgfqpoint{5.281922in}{0.693765in}}%
\pgfpathcurveto{\pgfqpoint{5.274108in}{0.701578in}}{\pgfqpoint{5.263509in}{0.705969in}}{\pgfqpoint{5.252459in}{0.705969in}}%
\pgfpathcurveto{\pgfqpoint{5.241409in}{0.705969in}}{\pgfqpoint{5.230810in}{0.701578in}}{\pgfqpoint{5.222996in}{0.693765in}}%
\pgfpathcurveto{\pgfqpoint{5.215183in}{0.685951in}}{\pgfqpoint{5.210793in}{0.675352in}}{\pgfqpoint{5.210793in}{0.664302in}}%
\pgfpathcurveto{\pgfqpoint{5.210793in}{0.653252in}}{\pgfqpoint{5.215183in}{0.642653in}}{\pgfqpoint{5.222996in}{0.634839in}}%
\pgfpathcurveto{\pgfqpoint{5.230810in}{0.627026in}}{\pgfqpoint{5.241409in}{0.622635in}}{\pgfqpoint{5.252459in}{0.622635in}}%
\pgfpathlineto{\pgfqpoint{5.252459in}{0.622635in}}%
\pgfpathclose%
\pgfusepath{stroke}%
\end{pgfscope}%
\begin{pgfscope}%
\pgfpathrectangle{\pgfqpoint{0.847223in}{0.554012in}}{\pgfqpoint{6.200000in}{4.620000in}}%
\pgfusepath{clip}%
\pgfsetbuttcap%
\pgfsetroundjoin%
\pgfsetlinewidth{1.003750pt}%
\definecolor{currentstroke}{rgb}{1.000000,0.000000,0.000000}%
\pgfsetstrokecolor{currentstroke}%
\pgfsetdash{}{0pt}%
\pgfpathmoveto{\pgfqpoint{5.257792in}{0.622056in}}%
\pgfpathcurveto{\pgfqpoint{5.268843in}{0.622056in}}{\pgfqpoint{5.279442in}{0.626447in}}{\pgfqpoint{5.287255in}{0.634260in}}%
\pgfpathcurveto{\pgfqpoint{5.295069in}{0.642074in}}{\pgfqpoint{5.299459in}{0.652673in}}{\pgfqpoint{5.299459in}{0.663723in}}%
\pgfpathcurveto{\pgfqpoint{5.299459in}{0.674773in}}{\pgfqpoint{5.295069in}{0.685372in}}{\pgfqpoint{5.287255in}{0.693186in}}%
\pgfpathcurveto{\pgfqpoint{5.279442in}{0.700999in}}{\pgfqpoint{5.268843in}{0.705390in}}{\pgfqpoint{5.257792in}{0.705390in}}%
\pgfpathcurveto{\pgfqpoint{5.246742in}{0.705390in}}{\pgfqpoint{5.236143in}{0.700999in}}{\pgfqpoint{5.228330in}{0.693186in}}%
\pgfpathcurveto{\pgfqpoint{5.220516in}{0.685372in}}{\pgfqpoint{5.216126in}{0.674773in}}{\pgfqpoint{5.216126in}{0.663723in}}%
\pgfpathcurveto{\pgfqpoint{5.216126in}{0.652673in}}{\pgfqpoint{5.220516in}{0.642074in}}{\pgfqpoint{5.228330in}{0.634260in}}%
\pgfpathcurveto{\pgfqpoint{5.236143in}{0.626447in}}{\pgfqpoint{5.246742in}{0.622056in}}{\pgfqpoint{5.257792in}{0.622056in}}%
\pgfpathlineto{\pgfqpoint{5.257792in}{0.622056in}}%
\pgfpathclose%
\pgfusepath{stroke}%
\end{pgfscope}%
\begin{pgfscope}%
\pgfpathrectangle{\pgfqpoint{0.847223in}{0.554012in}}{\pgfqpoint{6.200000in}{4.620000in}}%
\pgfusepath{clip}%
\pgfsetbuttcap%
\pgfsetroundjoin%
\pgfsetlinewidth{1.003750pt}%
\definecolor{currentstroke}{rgb}{1.000000,0.000000,0.000000}%
\pgfsetstrokecolor{currentstroke}%
\pgfsetdash{}{0pt}%
\pgfpathmoveto{\pgfqpoint{5.263126in}{0.621478in}}%
\pgfpathcurveto{\pgfqpoint{5.274176in}{0.621478in}}{\pgfqpoint{5.284775in}{0.625869in}}{\pgfqpoint{5.292588in}{0.633682in}}%
\pgfpathcurveto{\pgfqpoint{5.300402in}{0.641496in}}{\pgfqpoint{5.304792in}{0.652095in}}{\pgfqpoint{5.304792in}{0.663145in}}%
\pgfpathcurveto{\pgfqpoint{5.304792in}{0.674195in}}{\pgfqpoint{5.300402in}{0.684794in}}{\pgfqpoint{5.292588in}{0.692608in}}%
\pgfpathcurveto{\pgfqpoint{5.284775in}{0.700422in}}{\pgfqpoint{5.274176in}{0.704812in}}{\pgfqpoint{5.263126in}{0.704812in}}%
\pgfpathcurveto{\pgfqpoint{5.252075in}{0.704812in}}{\pgfqpoint{5.241476in}{0.700422in}}{\pgfqpoint{5.233663in}{0.692608in}}%
\pgfpathcurveto{\pgfqpoint{5.225849in}{0.684794in}}{\pgfqpoint{5.221459in}{0.674195in}}{\pgfqpoint{5.221459in}{0.663145in}}%
\pgfpathcurveto{\pgfqpoint{5.221459in}{0.652095in}}{\pgfqpoint{5.225849in}{0.641496in}}{\pgfqpoint{5.233663in}{0.633682in}}%
\pgfpathcurveto{\pgfqpoint{5.241476in}{0.625869in}}{\pgfqpoint{5.252075in}{0.621478in}}{\pgfqpoint{5.263126in}{0.621478in}}%
\pgfpathlineto{\pgfqpoint{5.263126in}{0.621478in}}%
\pgfpathclose%
\pgfusepath{stroke}%
\end{pgfscope}%
\begin{pgfscope}%
\pgfpathrectangle{\pgfqpoint{0.847223in}{0.554012in}}{\pgfqpoint{6.200000in}{4.620000in}}%
\pgfusepath{clip}%
\pgfsetbuttcap%
\pgfsetroundjoin%
\pgfsetlinewidth{1.003750pt}%
\definecolor{currentstroke}{rgb}{1.000000,0.000000,0.000000}%
\pgfsetstrokecolor{currentstroke}%
\pgfsetdash{}{0pt}%
\pgfpathmoveto{\pgfqpoint{5.268459in}{0.620902in}}%
\pgfpathcurveto{\pgfqpoint{5.279509in}{0.620902in}}{\pgfqpoint{5.290108in}{0.625292in}}{\pgfqpoint{5.297922in}{0.633106in}}%
\pgfpathcurveto{\pgfqpoint{5.305735in}{0.640919in}}{\pgfqpoint{5.310126in}{0.651518in}}{\pgfqpoint{5.310126in}{0.662568in}}%
\pgfpathcurveto{\pgfqpoint{5.310126in}{0.673619in}}{\pgfqpoint{5.305735in}{0.684218in}}{\pgfqpoint{5.297922in}{0.692031in}}%
\pgfpathcurveto{\pgfqpoint{5.290108in}{0.699845in}}{\pgfqpoint{5.279509in}{0.704235in}}{\pgfqpoint{5.268459in}{0.704235in}}%
\pgfpathcurveto{\pgfqpoint{5.257409in}{0.704235in}}{\pgfqpoint{5.246810in}{0.699845in}}{\pgfqpoint{5.238996in}{0.692031in}}%
\pgfpathcurveto{\pgfqpoint{5.231182in}{0.684218in}}{\pgfqpoint{5.226792in}{0.673619in}}{\pgfqpoint{5.226792in}{0.662568in}}%
\pgfpathcurveto{\pgfqpoint{5.226792in}{0.651518in}}{\pgfqpoint{5.231182in}{0.640919in}}{\pgfqpoint{5.238996in}{0.633106in}}%
\pgfpathcurveto{\pgfqpoint{5.246810in}{0.625292in}}{\pgfqpoint{5.257409in}{0.620902in}}{\pgfqpoint{5.268459in}{0.620902in}}%
\pgfpathlineto{\pgfqpoint{5.268459in}{0.620902in}}%
\pgfpathclose%
\pgfusepath{stroke}%
\end{pgfscope}%
\begin{pgfscope}%
\pgfpathrectangle{\pgfqpoint{0.847223in}{0.554012in}}{\pgfqpoint{6.200000in}{4.620000in}}%
\pgfusepath{clip}%
\pgfsetbuttcap%
\pgfsetroundjoin%
\pgfsetlinewidth{1.003750pt}%
\definecolor{currentstroke}{rgb}{1.000000,0.000000,0.000000}%
\pgfsetstrokecolor{currentstroke}%
\pgfsetdash{}{0pt}%
\pgfpathmoveto{\pgfqpoint{5.273792in}{0.620326in}}%
\pgfpathcurveto{\pgfqpoint{5.284842in}{0.620326in}}{\pgfqpoint{5.295441in}{0.624717in}}{\pgfqpoint{5.303255in}{0.632530in}}%
\pgfpathcurveto{\pgfqpoint{5.311068in}{0.640344in}}{\pgfqpoint{5.315459in}{0.650943in}}{\pgfqpoint{5.315459in}{0.661993in}}%
\pgfpathcurveto{\pgfqpoint{5.315459in}{0.673043in}}{\pgfqpoint{5.311068in}{0.683642in}}{\pgfqpoint{5.303255in}{0.691456in}}%
\pgfpathcurveto{\pgfqpoint{5.295441in}{0.699270in}}{\pgfqpoint{5.284842in}{0.703660in}}{\pgfqpoint{5.273792in}{0.703660in}}%
\pgfpathcurveto{\pgfqpoint{5.262742in}{0.703660in}}{\pgfqpoint{5.252143in}{0.699270in}}{\pgfqpoint{5.244329in}{0.691456in}}%
\pgfpathcurveto{\pgfqpoint{5.236516in}{0.683642in}}{\pgfqpoint{5.232125in}{0.673043in}}{\pgfqpoint{5.232125in}{0.661993in}}%
\pgfpathcurveto{\pgfqpoint{5.232125in}{0.650943in}}{\pgfqpoint{5.236516in}{0.640344in}}{\pgfqpoint{5.244329in}{0.632530in}}%
\pgfpathcurveto{\pgfqpoint{5.252143in}{0.624717in}}{\pgfqpoint{5.262742in}{0.620326in}}{\pgfqpoint{5.273792in}{0.620326in}}%
\pgfpathlineto{\pgfqpoint{5.273792in}{0.620326in}}%
\pgfpathclose%
\pgfusepath{stroke}%
\end{pgfscope}%
\begin{pgfscope}%
\pgfpathrectangle{\pgfqpoint{0.847223in}{0.554012in}}{\pgfqpoint{6.200000in}{4.620000in}}%
\pgfusepath{clip}%
\pgfsetbuttcap%
\pgfsetroundjoin%
\pgfsetlinewidth{1.003750pt}%
\definecolor{currentstroke}{rgb}{1.000000,0.000000,0.000000}%
\pgfsetstrokecolor{currentstroke}%
\pgfsetdash{}{0pt}%
\pgfpathmoveto{\pgfqpoint{5.279125in}{0.619752in}}%
\pgfpathcurveto{\pgfqpoint{5.290175in}{0.619752in}}{\pgfqpoint{5.300774in}{0.624143in}}{\pgfqpoint{5.308588in}{0.631956in}}%
\pgfpathcurveto{\pgfqpoint{5.316402in}{0.639770in}}{\pgfqpoint{5.320792in}{0.650369in}}{\pgfqpoint{5.320792in}{0.661419in}}%
\pgfpathcurveto{\pgfqpoint{5.320792in}{0.672469in}}{\pgfqpoint{5.316402in}{0.683068in}}{\pgfqpoint{5.308588in}{0.690882in}}%
\pgfpathcurveto{\pgfqpoint{5.300774in}{0.698695in}}{\pgfqpoint{5.290175in}{0.703086in}}{\pgfqpoint{5.279125in}{0.703086in}}%
\pgfpathcurveto{\pgfqpoint{5.268075in}{0.703086in}}{\pgfqpoint{5.257476in}{0.698695in}}{\pgfqpoint{5.249662in}{0.690882in}}%
\pgfpathcurveto{\pgfqpoint{5.241849in}{0.683068in}}{\pgfqpoint{5.237459in}{0.672469in}}{\pgfqpoint{5.237459in}{0.661419in}}%
\pgfpathcurveto{\pgfqpoint{5.237459in}{0.650369in}}{\pgfqpoint{5.241849in}{0.639770in}}{\pgfqpoint{5.249662in}{0.631956in}}%
\pgfpathcurveto{\pgfqpoint{5.257476in}{0.624143in}}{\pgfqpoint{5.268075in}{0.619752in}}{\pgfqpoint{5.279125in}{0.619752in}}%
\pgfpathlineto{\pgfqpoint{5.279125in}{0.619752in}}%
\pgfpathclose%
\pgfusepath{stroke}%
\end{pgfscope}%
\begin{pgfscope}%
\pgfpathrectangle{\pgfqpoint{0.847223in}{0.554012in}}{\pgfqpoint{6.200000in}{4.620000in}}%
\pgfusepath{clip}%
\pgfsetbuttcap%
\pgfsetroundjoin%
\pgfsetlinewidth{1.003750pt}%
\definecolor{currentstroke}{rgb}{1.000000,0.000000,0.000000}%
\pgfsetstrokecolor{currentstroke}%
\pgfsetdash{}{0pt}%
\pgfpathmoveto{\pgfqpoint{5.284458in}{0.619179in}}%
\pgfpathcurveto{\pgfqpoint{5.295509in}{0.619179in}}{\pgfqpoint{5.306108in}{0.623570in}}{\pgfqpoint{5.313921in}{0.631383in}}%
\pgfpathcurveto{\pgfqpoint{5.321735in}{0.639197in}}{\pgfqpoint{5.326125in}{0.649796in}}{\pgfqpoint{5.326125in}{0.660846in}}%
\pgfpathcurveto{\pgfqpoint{5.326125in}{0.671896in}}{\pgfqpoint{5.321735in}{0.682495in}}{\pgfqpoint{5.313921in}{0.690309in}}%
\pgfpathcurveto{\pgfqpoint{5.306108in}{0.698122in}}{\pgfqpoint{5.295509in}{0.702513in}}{\pgfqpoint{5.284458in}{0.702513in}}%
\pgfpathcurveto{\pgfqpoint{5.273408in}{0.702513in}}{\pgfqpoint{5.262809in}{0.698122in}}{\pgfqpoint{5.254996in}{0.690309in}}%
\pgfpathcurveto{\pgfqpoint{5.247182in}{0.682495in}}{\pgfqpoint{5.242792in}{0.671896in}}{\pgfqpoint{5.242792in}{0.660846in}}%
\pgfpathcurveto{\pgfqpoint{5.242792in}{0.649796in}}{\pgfqpoint{5.247182in}{0.639197in}}{\pgfqpoint{5.254996in}{0.631383in}}%
\pgfpathcurveto{\pgfqpoint{5.262809in}{0.623570in}}{\pgfqpoint{5.273408in}{0.619179in}}{\pgfqpoint{5.284458in}{0.619179in}}%
\pgfpathlineto{\pgfqpoint{5.284458in}{0.619179in}}%
\pgfpathclose%
\pgfusepath{stroke}%
\end{pgfscope}%
\begin{pgfscope}%
\pgfpathrectangle{\pgfqpoint{0.847223in}{0.554012in}}{\pgfqpoint{6.200000in}{4.620000in}}%
\pgfusepath{clip}%
\pgfsetbuttcap%
\pgfsetroundjoin%
\pgfsetlinewidth{1.003750pt}%
\definecolor{currentstroke}{rgb}{1.000000,0.000000,0.000000}%
\pgfsetstrokecolor{currentstroke}%
\pgfsetdash{}{0pt}%
\pgfpathmoveto{\pgfqpoint{5.289792in}{0.618608in}}%
\pgfpathcurveto{\pgfqpoint{5.300842in}{0.618608in}}{\pgfqpoint{5.311441in}{0.622998in}}{\pgfqpoint{5.319254in}{0.630812in}}%
\pgfpathcurveto{\pgfqpoint{5.327068in}{0.638625in}}{\pgfqpoint{5.331458in}{0.649224in}}{\pgfqpoint{5.331458in}{0.660274in}}%
\pgfpathcurveto{\pgfqpoint{5.331458in}{0.671325in}}{\pgfqpoint{5.327068in}{0.681924in}}{\pgfqpoint{5.319254in}{0.689737in}}%
\pgfpathcurveto{\pgfqpoint{5.311441in}{0.697551in}}{\pgfqpoint{5.300842in}{0.701941in}}{\pgfqpoint{5.289792in}{0.701941in}}%
\pgfpathcurveto{\pgfqpoint{5.278742in}{0.701941in}}{\pgfqpoint{5.268143in}{0.697551in}}{\pgfqpoint{5.260329in}{0.689737in}}%
\pgfpathcurveto{\pgfqpoint{5.252515in}{0.681924in}}{\pgfqpoint{5.248125in}{0.671325in}}{\pgfqpoint{5.248125in}{0.660274in}}%
\pgfpathcurveto{\pgfqpoint{5.248125in}{0.649224in}}{\pgfqpoint{5.252515in}{0.638625in}}{\pgfqpoint{5.260329in}{0.630812in}}%
\pgfpathcurveto{\pgfqpoint{5.268143in}{0.622998in}}{\pgfqpoint{5.278742in}{0.618608in}}{\pgfqpoint{5.289792in}{0.618608in}}%
\pgfpathlineto{\pgfqpoint{5.289792in}{0.618608in}}%
\pgfpathclose%
\pgfusepath{stroke}%
\end{pgfscope}%
\begin{pgfscope}%
\pgfpathrectangle{\pgfqpoint{0.847223in}{0.554012in}}{\pgfqpoint{6.200000in}{4.620000in}}%
\pgfusepath{clip}%
\pgfsetbuttcap%
\pgfsetroundjoin%
\pgfsetlinewidth{1.003750pt}%
\definecolor{currentstroke}{rgb}{1.000000,0.000000,0.000000}%
\pgfsetstrokecolor{currentstroke}%
\pgfsetdash{}{0pt}%
\pgfpathmoveto{\pgfqpoint{5.295125in}{0.618037in}}%
\pgfpathcurveto{\pgfqpoint{5.306175in}{0.618037in}}{\pgfqpoint{5.316774in}{0.622428in}}{\pgfqpoint{5.324588in}{0.630241in}}%
\pgfpathcurveto{\pgfqpoint{5.332401in}{0.638055in}}{\pgfqpoint{5.336792in}{0.648654in}}{\pgfqpoint{5.336792in}{0.659704in}}%
\pgfpathcurveto{\pgfqpoint{5.336792in}{0.670754in}}{\pgfqpoint{5.332401in}{0.681353in}}{\pgfqpoint{5.324588in}{0.689167in}}%
\pgfpathcurveto{\pgfqpoint{5.316774in}{0.696980in}}{\pgfqpoint{5.306175in}{0.701371in}}{\pgfqpoint{5.295125in}{0.701371in}}%
\pgfpathcurveto{\pgfqpoint{5.284075in}{0.701371in}}{\pgfqpoint{5.273476in}{0.696980in}}{\pgfqpoint{5.265662in}{0.689167in}}%
\pgfpathcurveto{\pgfqpoint{5.257849in}{0.681353in}}{\pgfqpoint{5.253458in}{0.670754in}}{\pgfqpoint{5.253458in}{0.659704in}}%
\pgfpathcurveto{\pgfqpoint{5.253458in}{0.648654in}}{\pgfqpoint{5.257849in}{0.638055in}}{\pgfqpoint{5.265662in}{0.630241in}}%
\pgfpathcurveto{\pgfqpoint{5.273476in}{0.622428in}}{\pgfqpoint{5.284075in}{0.618037in}}{\pgfqpoint{5.295125in}{0.618037in}}%
\pgfpathlineto{\pgfqpoint{5.295125in}{0.618037in}}%
\pgfpathclose%
\pgfusepath{stroke}%
\end{pgfscope}%
\begin{pgfscope}%
\pgfpathrectangle{\pgfqpoint{0.847223in}{0.554012in}}{\pgfqpoint{6.200000in}{4.620000in}}%
\pgfusepath{clip}%
\pgfsetbuttcap%
\pgfsetroundjoin%
\pgfsetlinewidth{1.003750pt}%
\definecolor{currentstroke}{rgb}{1.000000,0.000000,0.000000}%
\pgfsetstrokecolor{currentstroke}%
\pgfsetdash{}{0pt}%
\pgfpathmoveto{\pgfqpoint{5.300458in}{0.617468in}}%
\pgfpathcurveto{\pgfqpoint{5.311508in}{0.617468in}}{\pgfqpoint{5.322107in}{0.621858in}}{\pgfqpoint{5.329921in}{0.629672in}}%
\pgfpathcurveto{\pgfqpoint{5.337735in}{0.637486in}}{\pgfqpoint{5.342125in}{0.648085in}}{\pgfqpoint{5.342125in}{0.659135in}}%
\pgfpathcurveto{\pgfqpoint{5.342125in}{0.670185in}}{\pgfqpoint{5.337735in}{0.680784in}}{\pgfqpoint{5.329921in}{0.688597in}}%
\pgfpathcurveto{\pgfqpoint{5.322107in}{0.696411in}}{\pgfqpoint{5.311508in}{0.700801in}}{\pgfqpoint{5.300458in}{0.700801in}}%
\pgfpathcurveto{\pgfqpoint{5.289408in}{0.700801in}}{\pgfqpoint{5.278809in}{0.696411in}}{\pgfqpoint{5.270995in}{0.688597in}}%
\pgfpathcurveto{\pgfqpoint{5.263182in}{0.680784in}}{\pgfqpoint{5.258791in}{0.670185in}}{\pgfqpoint{5.258791in}{0.659135in}}%
\pgfpathcurveto{\pgfqpoint{5.258791in}{0.648085in}}{\pgfqpoint{5.263182in}{0.637486in}}{\pgfqpoint{5.270995in}{0.629672in}}%
\pgfpathcurveto{\pgfqpoint{5.278809in}{0.621858in}}{\pgfqpoint{5.289408in}{0.617468in}}{\pgfqpoint{5.300458in}{0.617468in}}%
\pgfpathlineto{\pgfqpoint{5.300458in}{0.617468in}}%
\pgfpathclose%
\pgfusepath{stroke}%
\end{pgfscope}%
\begin{pgfscope}%
\pgfpathrectangle{\pgfqpoint{0.847223in}{0.554012in}}{\pgfqpoint{6.200000in}{4.620000in}}%
\pgfusepath{clip}%
\pgfsetbuttcap%
\pgfsetroundjoin%
\pgfsetlinewidth{1.003750pt}%
\definecolor{currentstroke}{rgb}{1.000000,0.000000,0.000000}%
\pgfsetstrokecolor{currentstroke}%
\pgfsetdash{}{0pt}%
\pgfpathmoveto{\pgfqpoint{5.305791in}{0.616900in}}%
\pgfpathcurveto{\pgfqpoint{5.316841in}{0.616900in}}{\pgfqpoint{5.327441in}{0.621290in}}{\pgfqpoint{5.335254in}{0.629104in}}%
\pgfpathcurveto{\pgfqpoint{5.343068in}{0.636918in}}{\pgfqpoint{5.347458in}{0.647517in}}{\pgfqpoint{5.347458in}{0.658567in}}%
\pgfpathcurveto{\pgfqpoint{5.347458in}{0.669617in}}{\pgfqpoint{5.343068in}{0.680216in}}{\pgfqpoint{5.335254in}{0.688029in}}%
\pgfpathcurveto{\pgfqpoint{5.327441in}{0.695843in}}{\pgfqpoint{5.316841in}{0.700233in}}{\pgfqpoint{5.305791in}{0.700233in}}%
\pgfpathcurveto{\pgfqpoint{5.294741in}{0.700233in}}{\pgfqpoint{5.284142in}{0.695843in}}{\pgfqpoint{5.276329in}{0.688029in}}%
\pgfpathcurveto{\pgfqpoint{5.268515in}{0.680216in}}{\pgfqpoint{5.264125in}{0.669617in}}{\pgfqpoint{5.264125in}{0.658567in}}%
\pgfpathcurveto{\pgfqpoint{5.264125in}{0.647517in}}{\pgfqpoint{5.268515in}{0.636918in}}{\pgfqpoint{5.276329in}{0.629104in}}%
\pgfpathcurveto{\pgfqpoint{5.284142in}{0.621290in}}{\pgfqpoint{5.294741in}{0.616900in}}{\pgfqpoint{5.305791in}{0.616900in}}%
\pgfpathlineto{\pgfqpoint{5.305791in}{0.616900in}}%
\pgfpathclose%
\pgfusepath{stroke}%
\end{pgfscope}%
\begin{pgfscope}%
\pgfpathrectangle{\pgfqpoint{0.847223in}{0.554012in}}{\pgfqpoint{6.200000in}{4.620000in}}%
\pgfusepath{clip}%
\pgfsetbuttcap%
\pgfsetroundjoin%
\pgfsetlinewidth{1.003750pt}%
\definecolor{currentstroke}{rgb}{1.000000,0.000000,0.000000}%
\pgfsetstrokecolor{currentstroke}%
\pgfsetdash{}{0pt}%
\pgfpathmoveto{\pgfqpoint{5.311125in}{0.616333in}}%
\pgfpathcurveto{\pgfqpoint{5.322175in}{0.616333in}}{\pgfqpoint{5.332774in}{0.620723in}}{\pgfqpoint{5.340587in}{0.628537in}}%
\pgfpathcurveto{\pgfqpoint{5.348401in}{0.636351in}}{\pgfqpoint{5.352791in}{0.646950in}}{\pgfqpoint{5.352791in}{0.658000in}}%
\pgfpathcurveto{\pgfqpoint{5.352791in}{0.669050in}}{\pgfqpoint{5.348401in}{0.679649in}}{\pgfqpoint{5.340587in}{0.687463in}}%
\pgfpathcurveto{\pgfqpoint{5.332774in}{0.695276in}}{\pgfqpoint{5.322175in}{0.699667in}}{\pgfqpoint{5.311125in}{0.699667in}}%
\pgfpathcurveto{\pgfqpoint{5.300074in}{0.699667in}}{\pgfqpoint{5.289475in}{0.695276in}}{\pgfqpoint{5.281662in}{0.687463in}}%
\pgfpathcurveto{\pgfqpoint{5.273848in}{0.679649in}}{\pgfqpoint{5.269458in}{0.669050in}}{\pgfqpoint{5.269458in}{0.658000in}}%
\pgfpathcurveto{\pgfqpoint{5.269458in}{0.646950in}}{\pgfqpoint{5.273848in}{0.636351in}}{\pgfqpoint{5.281662in}{0.628537in}}%
\pgfpathcurveto{\pgfqpoint{5.289475in}{0.620723in}}{\pgfqpoint{5.300074in}{0.616333in}}{\pgfqpoint{5.311125in}{0.616333in}}%
\pgfpathlineto{\pgfqpoint{5.311125in}{0.616333in}}%
\pgfpathclose%
\pgfusepath{stroke}%
\end{pgfscope}%
\begin{pgfscope}%
\pgfpathrectangle{\pgfqpoint{0.847223in}{0.554012in}}{\pgfqpoint{6.200000in}{4.620000in}}%
\pgfusepath{clip}%
\pgfsetbuttcap%
\pgfsetroundjoin%
\pgfsetlinewidth{1.003750pt}%
\definecolor{currentstroke}{rgb}{1.000000,0.000000,0.000000}%
\pgfsetstrokecolor{currentstroke}%
\pgfsetdash{}{0pt}%
\pgfpathmoveto{\pgfqpoint{5.316458in}{0.615768in}}%
\pgfpathcurveto{\pgfqpoint{5.327508in}{0.615768in}}{\pgfqpoint{5.338107in}{0.620158in}}{\pgfqpoint{5.345921in}{0.627972in}}%
\pgfpathcurveto{\pgfqpoint{5.353734in}{0.635785in}}{\pgfqpoint{5.358124in}{0.646384in}}{\pgfqpoint{5.358124in}{0.657434in}}%
\pgfpathcurveto{\pgfqpoint{5.358124in}{0.668484in}}{\pgfqpoint{5.353734in}{0.679083in}}{\pgfqpoint{5.345921in}{0.686897in}}%
\pgfpathcurveto{\pgfqpoint{5.338107in}{0.694711in}}{\pgfqpoint{5.327508in}{0.699101in}}{\pgfqpoint{5.316458in}{0.699101in}}%
\pgfpathcurveto{\pgfqpoint{5.305408in}{0.699101in}}{\pgfqpoint{5.294809in}{0.694711in}}{\pgfqpoint{5.286995in}{0.686897in}}%
\pgfpathcurveto{\pgfqpoint{5.279181in}{0.679083in}}{\pgfqpoint{5.274791in}{0.668484in}}{\pgfqpoint{5.274791in}{0.657434in}}%
\pgfpathcurveto{\pgfqpoint{5.274791in}{0.646384in}}{\pgfqpoint{5.279181in}{0.635785in}}{\pgfqpoint{5.286995in}{0.627972in}}%
\pgfpathcurveto{\pgfqpoint{5.294809in}{0.620158in}}{\pgfqpoint{5.305408in}{0.615768in}}{\pgfqpoint{5.316458in}{0.615768in}}%
\pgfpathlineto{\pgfqpoint{5.316458in}{0.615768in}}%
\pgfpathclose%
\pgfusepath{stroke}%
\end{pgfscope}%
\begin{pgfscope}%
\pgfpathrectangle{\pgfqpoint{0.847223in}{0.554012in}}{\pgfqpoint{6.200000in}{4.620000in}}%
\pgfusepath{clip}%
\pgfsetbuttcap%
\pgfsetroundjoin%
\pgfsetlinewidth{1.003750pt}%
\definecolor{currentstroke}{rgb}{1.000000,0.000000,0.000000}%
\pgfsetstrokecolor{currentstroke}%
\pgfsetdash{}{0pt}%
\pgfpathmoveto{\pgfqpoint{5.321791in}{0.615203in}}%
\pgfpathcurveto{\pgfqpoint{5.332841in}{0.615203in}}{\pgfqpoint{5.343440in}{0.619594in}}{\pgfqpoint{5.351254in}{0.627407in}}%
\pgfpathcurveto{\pgfqpoint{5.359067in}{0.635221in}}{\pgfqpoint{5.363458in}{0.645820in}}{\pgfqpoint{5.363458in}{0.656870in}}%
\pgfpathcurveto{\pgfqpoint{5.363458in}{0.667920in}}{\pgfqpoint{5.359067in}{0.678519in}}{\pgfqpoint{5.351254in}{0.686333in}}%
\pgfpathcurveto{\pgfqpoint{5.343440in}{0.694146in}}{\pgfqpoint{5.332841in}{0.698537in}}{\pgfqpoint{5.321791in}{0.698537in}}%
\pgfpathcurveto{\pgfqpoint{5.310741in}{0.698537in}}{\pgfqpoint{5.300142in}{0.694146in}}{\pgfqpoint{5.292328in}{0.686333in}}%
\pgfpathcurveto{\pgfqpoint{5.284515in}{0.678519in}}{\pgfqpoint{5.280124in}{0.667920in}}{\pgfqpoint{5.280124in}{0.656870in}}%
\pgfpathcurveto{\pgfqpoint{5.280124in}{0.645820in}}{\pgfqpoint{5.284515in}{0.635221in}}{\pgfqpoint{5.292328in}{0.627407in}}%
\pgfpathcurveto{\pgfqpoint{5.300142in}{0.619594in}}{\pgfqpoint{5.310741in}{0.615203in}}{\pgfqpoint{5.321791in}{0.615203in}}%
\pgfpathlineto{\pgfqpoint{5.321791in}{0.615203in}}%
\pgfpathclose%
\pgfusepath{stroke}%
\end{pgfscope}%
\begin{pgfscope}%
\pgfpathrectangle{\pgfqpoint{0.847223in}{0.554012in}}{\pgfqpoint{6.200000in}{4.620000in}}%
\pgfusepath{clip}%
\pgfsetbuttcap%
\pgfsetroundjoin%
\pgfsetlinewidth{1.003750pt}%
\definecolor{currentstroke}{rgb}{1.000000,0.000000,0.000000}%
\pgfsetstrokecolor{currentstroke}%
\pgfsetdash{}{0pt}%
\pgfpathmoveto{\pgfqpoint{5.327124in}{0.614640in}}%
\pgfpathcurveto{\pgfqpoint{5.338174in}{0.614640in}}{\pgfqpoint{5.348773in}{0.619030in}}{\pgfqpoint{5.356587in}{0.626844in}}%
\pgfpathcurveto{\pgfqpoint{5.364401in}{0.634658in}}{\pgfqpoint{5.368791in}{0.645257in}}{\pgfqpoint{5.368791in}{0.656307in}}%
\pgfpathcurveto{\pgfqpoint{5.368791in}{0.667357in}}{\pgfqpoint{5.364401in}{0.677956in}}{\pgfqpoint{5.356587in}{0.685770in}}%
\pgfpathcurveto{\pgfqpoint{5.348773in}{0.693583in}}{\pgfqpoint{5.338174in}{0.697973in}}{\pgfqpoint{5.327124in}{0.697973in}}%
\pgfpathcurveto{\pgfqpoint{5.316074in}{0.697973in}}{\pgfqpoint{5.305475in}{0.693583in}}{\pgfqpoint{5.297661in}{0.685770in}}%
\pgfpathcurveto{\pgfqpoint{5.289848in}{0.677956in}}{\pgfqpoint{5.285458in}{0.667357in}}{\pgfqpoint{5.285458in}{0.656307in}}%
\pgfpathcurveto{\pgfqpoint{5.285458in}{0.645257in}}{\pgfqpoint{5.289848in}{0.634658in}}{\pgfqpoint{5.297661in}{0.626844in}}%
\pgfpathcurveto{\pgfqpoint{5.305475in}{0.619030in}}{\pgfqpoint{5.316074in}{0.614640in}}{\pgfqpoint{5.327124in}{0.614640in}}%
\pgfpathlineto{\pgfqpoint{5.327124in}{0.614640in}}%
\pgfpathclose%
\pgfusepath{stroke}%
\end{pgfscope}%
\begin{pgfscope}%
\pgfpathrectangle{\pgfqpoint{0.847223in}{0.554012in}}{\pgfqpoint{6.200000in}{4.620000in}}%
\pgfusepath{clip}%
\pgfsetbuttcap%
\pgfsetroundjoin%
\pgfsetlinewidth{1.003750pt}%
\definecolor{currentstroke}{rgb}{1.000000,0.000000,0.000000}%
\pgfsetstrokecolor{currentstroke}%
\pgfsetdash{}{0pt}%
\pgfpathmoveto{\pgfqpoint{5.332457in}{0.614078in}}%
\pgfpathcurveto{\pgfqpoint{5.343508in}{0.614078in}}{\pgfqpoint{5.354107in}{0.618468in}}{\pgfqpoint{5.361920in}{0.626282in}}%
\pgfpathcurveto{\pgfqpoint{5.369734in}{0.634096in}}{\pgfqpoint{5.374124in}{0.644695in}}{\pgfqpoint{5.374124in}{0.655745in}}%
\pgfpathcurveto{\pgfqpoint{5.374124in}{0.666795in}}{\pgfqpoint{5.369734in}{0.677394in}}{\pgfqpoint{5.361920in}{0.685208in}}%
\pgfpathcurveto{\pgfqpoint{5.354107in}{0.693021in}}{\pgfqpoint{5.343508in}{0.697411in}}{\pgfqpoint{5.332457in}{0.697411in}}%
\pgfpathcurveto{\pgfqpoint{5.321407in}{0.697411in}}{\pgfqpoint{5.310808in}{0.693021in}}{\pgfqpoint{5.302995in}{0.685208in}}%
\pgfpathcurveto{\pgfqpoint{5.295181in}{0.677394in}}{\pgfqpoint{5.290791in}{0.666795in}}{\pgfqpoint{5.290791in}{0.655745in}}%
\pgfpathcurveto{\pgfqpoint{5.290791in}{0.644695in}}{\pgfqpoint{5.295181in}{0.634096in}}{\pgfqpoint{5.302995in}{0.626282in}}%
\pgfpathcurveto{\pgfqpoint{5.310808in}{0.618468in}}{\pgfqpoint{5.321407in}{0.614078in}}{\pgfqpoint{5.332457in}{0.614078in}}%
\pgfpathlineto{\pgfqpoint{5.332457in}{0.614078in}}%
\pgfpathclose%
\pgfusepath{stroke}%
\end{pgfscope}%
\begin{pgfscope}%
\pgfpathrectangle{\pgfqpoint{0.847223in}{0.554012in}}{\pgfqpoint{6.200000in}{4.620000in}}%
\pgfusepath{clip}%
\pgfsetbuttcap%
\pgfsetroundjoin%
\pgfsetlinewidth{1.003750pt}%
\definecolor{currentstroke}{rgb}{1.000000,0.000000,0.000000}%
\pgfsetstrokecolor{currentstroke}%
\pgfsetdash{}{0pt}%
\pgfpathmoveto{\pgfqpoint{5.337791in}{0.613517in}}%
\pgfpathcurveto{\pgfqpoint{5.348841in}{0.613517in}}{\pgfqpoint{5.359440in}{0.617908in}}{\pgfqpoint{5.367253in}{0.625721in}}%
\pgfpathcurveto{\pgfqpoint{5.375067in}{0.633535in}}{\pgfqpoint{5.379457in}{0.644134in}}{\pgfqpoint{5.379457in}{0.655184in}}%
\pgfpathcurveto{\pgfqpoint{5.379457in}{0.666234in}}{\pgfqpoint{5.375067in}{0.676833in}}{\pgfqpoint{5.367253in}{0.684647in}}%
\pgfpathcurveto{\pgfqpoint{5.359440in}{0.692460in}}{\pgfqpoint{5.348841in}{0.696851in}}{\pgfqpoint{5.337791in}{0.696851in}}%
\pgfpathcurveto{\pgfqpoint{5.326741in}{0.696851in}}{\pgfqpoint{5.316141in}{0.692460in}}{\pgfqpoint{5.308328in}{0.684647in}}%
\pgfpathcurveto{\pgfqpoint{5.300514in}{0.676833in}}{\pgfqpoint{5.296124in}{0.666234in}}{\pgfqpoint{5.296124in}{0.655184in}}%
\pgfpathcurveto{\pgfqpoint{5.296124in}{0.644134in}}{\pgfqpoint{5.300514in}{0.633535in}}{\pgfqpoint{5.308328in}{0.625721in}}%
\pgfpathcurveto{\pgfqpoint{5.316141in}{0.617908in}}{\pgfqpoint{5.326741in}{0.613517in}}{\pgfqpoint{5.337791in}{0.613517in}}%
\pgfpathlineto{\pgfqpoint{5.337791in}{0.613517in}}%
\pgfpathclose%
\pgfusepath{stroke}%
\end{pgfscope}%
\begin{pgfscope}%
\pgfpathrectangle{\pgfqpoint{0.847223in}{0.554012in}}{\pgfqpoint{6.200000in}{4.620000in}}%
\pgfusepath{clip}%
\pgfsetbuttcap%
\pgfsetroundjoin%
\pgfsetlinewidth{1.003750pt}%
\definecolor{currentstroke}{rgb}{1.000000,0.000000,0.000000}%
\pgfsetstrokecolor{currentstroke}%
\pgfsetdash{}{0pt}%
\pgfpathmoveto{\pgfqpoint{5.343124in}{0.612958in}}%
\pgfpathcurveto{\pgfqpoint{5.354174in}{0.612958in}}{\pgfqpoint{5.364773in}{0.617348in}}{\pgfqpoint{5.372587in}{0.625162in}}%
\pgfpathcurveto{\pgfqpoint{5.380400in}{0.632975in}}{\pgfqpoint{5.384791in}{0.643574in}}{\pgfqpoint{5.384791in}{0.654624in}}%
\pgfpathcurveto{\pgfqpoint{5.384791in}{0.665675in}}{\pgfqpoint{5.380400in}{0.676274in}}{\pgfqpoint{5.372587in}{0.684087in}}%
\pgfpathcurveto{\pgfqpoint{5.364773in}{0.691901in}}{\pgfqpoint{5.354174in}{0.696291in}}{\pgfqpoint{5.343124in}{0.696291in}}%
\pgfpathcurveto{\pgfqpoint{5.332074in}{0.696291in}}{\pgfqpoint{5.321475in}{0.691901in}}{\pgfqpoint{5.313661in}{0.684087in}}%
\pgfpathcurveto{\pgfqpoint{5.305847in}{0.676274in}}{\pgfqpoint{5.301457in}{0.665675in}}{\pgfqpoint{5.301457in}{0.654624in}}%
\pgfpathcurveto{\pgfqpoint{5.301457in}{0.643574in}}{\pgfqpoint{5.305847in}{0.632975in}}{\pgfqpoint{5.313661in}{0.625162in}}%
\pgfpathcurveto{\pgfqpoint{5.321475in}{0.617348in}}{\pgfqpoint{5.332074in}{0.612958in}}{\pgfqpoint{5.343124in}{0.612958in}}%
\pgfpathlineto{\pgfqpoint{5.343124in}{0.612958in}}%
\pgfpathclose%
\pgfusepath{stroke}%
\end{pgfscope}%
\begin{pgfscope}%
\pgfpathrectangle{\pgfqpoint{0.847223in}{0.554012in}}{\pgfqpoint{6.200000in}{4.620000in}}%
\pgfusepath{clip}%
\pgfsetbuttcap%
\pgfsetroundjoin%
\pgfsetlinewidth{1.003750pt}%
\definecolor{currentstroke}{rgb}{1.000000,0.000000,0.000000}%
\pgfsetstrokecolor{currentstroke}%
\pgfsetdash{}{0pt}%
\pgfpathmoveto{\pgfqpoint{5.348457in}{0.612399in}}%
\pgfpathcurveto{\pgfqpoint{5.359507in}{0.612399in}}{\pgfqpoint{5.370106in}{0.616790in}}{\pgfqpoint{5.377920in}{0.624603in}}%
\pgfpathcurveto{\pgfqpoint{5.385733in}{0.632417in}}{\pgfqpoint{5.390124in}{0.643016in}}{\pgfqpoint{5.390124in}{0.654066in}}%
\pgfpathcurveto{\pgfqpoint{5.390124in}{0.665116in}}{\pgfqpoint{5.385733in}{0.675715in}}{\pgfqpoint{5.377920in}{0.683529in}}%
\pgfpathcurveto{\pgfqpoint{5.370106in}{0.691342in}}{\pgfqpoint{5.359507in}{0.695733in}}{\pgfqpoint{5.348457in}{0.695733in}}%
\pgfpathcurveto{\pgfqpoint{5.337407in}{0.695733in}}{\pgfqpoint{5.326808in}{0.691342in}}{\pgfqpoint{5.318994in}{0.683529in}}%
\pgfpathcurveto{\pgfqpoint{5.311181in}{0.675715in}}{\pgfqpoint{5.306790in}{0.665116in}}{\pgfqpoint{5.306790in}{0.654066in}}%
\pgfpathcurveto{\pgfqpoint{5.306790in}{0.643016in}}{\pgfqpoint{5.311181in}{0.632417in}}{\pgfqpoint{5.318994in}{0.624603in}}%
\pgfpathcurveto{\pgfqpoint{5.326808in}{0.616790in}}{\pgfqpoint{5.337407in}{0.612399in}}{\pgfqpoint{5.348457in}{0.612399in}}%
\pgfpathlineto{\pgfqpoint{5.348457in}{0.612399in}}%
\pgfpathclose%
\pgfusepath{stroke}%
\end{pgfscope}%
\begin{pgfscope}%
\pgfpathrectangle{\pgfqpoint{0.847223in}{0.554012in}}{\pgfqpoint{6.200000in}{4.620000in}}%
\pgfusepath{clip}%
\pgfsetbuttcap%
\pgfsetroundjoin%
\pgfsetlinewidth{1.003750pt}%
\definecolor{currentstroke}{rgb}{1.000000,0.000000,0.000000}%
\pgfsetstrokecolor{currentstroke}%
\pgfsetdash{}{0pt}%
\pgfpathmoveto{\pgfqpoint{5.353790in}{0.611842in}}%
\pgfpathcurveto{\pgfqpoint{5.364840in}{0.611842in}}{\pgfqpoint{5.375439in}{0.616232in}}{\pgfqpoint{5.383253in}{0.624046in}}%
\pgfpathcurveto{\pgfqpoint{5.391067in}{0.631860in}}{\pgfqpoint{5.395457in}{0.642459in}}{\pgfqpoint{5.395457in}{0.653509in}}%
\pgfpathcurveto{\pgfqpoint{5.395457in}{0.664559in}}{\pgfqpoint{5.391067in}{0.675158in}}{\pgfqpoint{5.383253in}{0.682972in}}%
\pgfpathcurveto{\pgfqpoint{5.375439in}{0.690785in}}{\pgfqpoint{5.364840in}{0.695175in}}{\pgfqpoint{5.353790in}{0.695175in}}%
\pgfpathcurveto{\pgfqpoint{5.342740in}{0.695175in}}{\pgfqpoint{5.332141in}{0.690785in}}{\pgfqpoint{5.324328in}{0.682972in}}%
\pgfpathcurveto{\pgfqpoint{5.316514in}{0.675158in}}{\pgfqpoint{5.312124in}{0.664559in}}{\pgfqpoint{5.312124in}{0.653509in}}%
\pgfpathcurveto{\pgfqpoint{5.312124in}{0.642459in}}{\pgfqpoint{5.316514in}{0.631860in}}{\pgfqpoint{5.324328in}{0.624046in}}%
\pgfpathcurveto{\pgfqpoint{5.332141in}{0.616232in}}{\pgfqpoint{5.342740in}{0.611842in}}{\pgfqpoint{5.353790in}{0.611842in}}%
\pgfpathlineto{\pgfqpoint{5.353790in}{0.611842in}}%
\pgfpathclose%
\pgfusepath{stroke}%
\end{pgfscope}%
\begin{pgfscope}%
\pgfpathrectangle{\pgfqpoint{0.847223in}{0.554012in}}{\pgfqpoint{6.200000in}{4.620000in}}%
\pgfusepath{clip}%
\pgfsetbuttcap%
\pgfsetroundjoin%
\pgfsetlinewidth{1.003750pt}%
\definecolor{currentstroke}{rgb}{1.000000,0.000000,0.000000}%
\pgfsetstrokecolor{currentstroke}%
\pgfsetdash{}{0pt}%
\pgfpathmoveto{\pgfqpoint{5.359124in}{0.611286in}}%
\pgfpathcurveto{\pgfqpoint{5.370174in}{0.611286in}}{\pgfqpoint{5.380773in}{0.615676in}}{\pgfqpoint{5.388586in}{0.623490in}}%
\pgfpathcurveto{\pgfqpoint{5.396400in}{0.631304in}}{\pgfqpoint{5.400790in}{0.641903in}}{\pgfqpoint{5.400790in}{0.652953in}}%
\pgfpathcurveto{\pgfqpoint{5.400790in}{0.664003in}}{\pgfqpoint{5.396400in}{0.674602in}}{\pgfqpoint{5.388586in}{0.682416in}}%
\pgfpathcurveto{\pgfqpoint{5.380773in}{0.690229in}}{\pgfqpoint{5.370174in}{0.694619in}}{\pgfqpoint{5.359124in}{0.694619in}}%
\pgfpathcurveto{\pgfqpoint{5.348073in}{0.694619in}}{\pgfqpoint{5.337474in}{0.690229in}}{\pgfqpoint{5.329661in}{0.682416in}}%
\pgfpathcurveto{\pgfqpoint{5.321847in}{0.674602in}}{\pgfqpoint{5.317457in}{0.664003in}}{\pgfqpoint{5.317457in}{0.652953in}}%
\pgfpathcurveto{\pgfqpoint{5.317457in}{0.641903in}}{\pgfqpoint{5.321847in}{0.631304in}}{\pgfqpoint{5.329661in}{0.623490in}}%
\pgfpathcurveto{\pgfqpoint{5.337474in}{0.615676in}}{\pgfqpoint{5.348073in}{0.611286in}}{\pgfqpoint{5.359124in}{0.611286in}}%
\pgfpathlineto{\pgfqpoint{5.359124in}{0.611286in}}%
\pgfpathclose%
\pgfusepath{stroke}%
\end{pgfscope}%
\begin{pgfscope}%
\pgfpathrectangle{\pgfqpoint{0.847223in}{0.554012in}}{\pgfqpoint{6.200000in}{4.620000in}}%
\pgfusepath{clip}%
\pgfsetbuttcap%
\pgfsetroundjoin%
\pgfsetlinewidth{1.003750pt}%
\definecolor{currentstroke}{rgb}{1.000000,0.000000,0.000000}%
\pgfsetstrokecolor{currentstroke}%
\pgfsetdash{}{0pt}%
\pgfpathmoveto{\pgfqpoint{5.364457in}{0.610731in}}%
\pgfpathcurveto{\pgfqpoint{5.375507in}{0.610731in}}{\pgfqpoint{5.386106in}{0.615122in}}{\pgfqpoint{5.393920in}{0.622935in}}%
\pgfpathcurveto{\pgfqpoint{5.401733in}{0.630749in}}{\pgfqpoint{5.406123in}{0.641348in}}{\pgfqpoint{5.406123in}{0.652398in}}%
\pgfpathcurveto{\pgfqpoint{5.406123in}{0.663448in}}{\pgfqpoint{5.401733in}{0.674047in}}{\pgfqpoint{5.393920in}{0.681861in}}%
\pgfpathcurveto{\pgfqpoint{5.386106in}{0.689674in}}{\pgfqpoint{5.375507in}{0.694065in}}{\pgfqpoint{5.364457in}{0.694065in}}%
\pgfpathcurveto{\pgfqpoint{5.353407in}{0.694065in}}{\pgfqpoint{5.342808in}{0.689674in}}{\pgfqpoint{5.334994in}{0.681861in}}%
\pgfpathcurveto{\pgfqpoint{5.327180in}{0.674047in}}{\pgfqpoint{5.322790in}{0.663448in}}{\pgfqpoint{5.322790in}{0.652398in}}%
\pgfpathcurveto{\pgfqpoint{5.322790in}{0.641348in}}{\pgfqpoint{5.327180in}{0.630749in}}{\pgfqpoint{5.334994in}{0.622935in}}%
\pgfpathcurveto{\pgfqpoint{5.342808in}{0.615122in}}{\pgfqpoint{5.353407in}{0.610731in}}{\pgfqpoint{5.364457in}{0.610731in}}%
\pgfpathlineto{\pgfqpoint{5.364457in}{0.610731in}}%
\pgfpathclose%
\pgfusepath{stroke}%
\end{pgfscope}%
\begin{pgfscope}%
\pgfpathrectangle{\pgfqpoint{0.847223in}{0.554012in}}{\pgfqpoint{6.200000in}{4.620000in}}%
\pgfusepath{clip}%
\pgfsetbuttcap%
\pgfsetroundjoin%
\pgfsetlinewidth{1.003750pt}%
\definecolor{currentstroke}{rgb}{1.000000,0.000000,0.000000}%
\pgfsetstrokecolor{currentstroke}%
\pgfsetdash{}{0pt}%
\pgfpathmoveto{\pgfqpoint{5.369790in}{0.610178in}}%
\pgfpathcurveto{\pgfqpoint{5.380840in}{0.610178in}}{\pgfqpoint{5.391439in}{0.614568in}}{\pgfqpoint{5.399253in}{0.622381in}}%
\pgfpathcurveto{\pgfqpoint{5.407066in}{0.630195in}}{\pgfqpoint{5.411457in}{0.640794in}}{\pgfqpoint{5.411457in}{0.651844in}}%
\pgfpathcurveto{\pgfqpoint{5.411457in}{0.662894in}}{\pgfqpoint{5.407066in}{0.673493in}}{\pgfqpoint{5.399253in}{0.681307in}}%
\pgfpathcurveto{\pgfqpoint{5.391439in}{0.689121in}}{\pgfqpoint{5.380840in}{0.693511in}}{\pgfqpoint{5.369790in}{0.693511in}}%
\pgfpathcurveto{\pgfqpoint{5.358740in}{0.693511in}}{\pgfqpoint{5.348141in}{0.689121in}}{\pgfqpoint{5.340327in}{0.681307in}}%
\pgfpathcurveto{\pgfqpoint{5.332514in}{0.673493in}}{\pgfqpoint{5.328123in}{0.662894in}}{\pgfqpoint{5.328123in}{0.651844in}}%
\pgfpathcurveto{\pgfqpoint{5.328123in}{0.640794in}}{\pgfqpoint{5.332514in}{0.630195in}}{\pgfqpoint{5.340327in}{0.622381in}}%
\pgfpathcurveto{\pgfqpoint{5.348141in}{0.614568in}}{\pgfqpoint{5.358740in}{0.610178in}}{\pgfqpoint{5.369790in}{0.610178in}}%
\pgfpathlineto{\pgfqpoint{5.369790in}{0.610178in}}%
\pgfpathclose%
\pgfusepath{stroke}%
\end{pgfscope}%
\begin{pgfscope}%
\pgfpathrectangle{\pgfqpoint{0.847223in}{0.554012in}}{\pgfqpoint{6.200000in}{4.620000in}}%
\pgfusepath{clip}%
\pgfsetbuttcap%
\pgfsetroundjoin%
\pgfsetlinewidth{1.003750pt}%
\definecolor{currentstroke}{rgb}{1.000000,0.000000,0.000000}%
\pgfsetstrokecolor{currentstroke}%
\pgfsetdash{}{0pt}%
\pgfpathmoveto{\pgfqpoint{5.375123in}{0.609625in}}%
\pgfpathcurveto{\pgfqpoint{5.386173in}{0.609625in}}{\pgfqpoint{5.396772in}{0.614015in}}{\pgfqpoint{5.404586in}{0.621829in}}%
\pgfpathcurveto{\pgfqpoint{5.412400in}{0.629643in}}{\pgfqpoint{5.416790in}{0.640242in}}{\pgfqpoint{5.416790in}{0.651292in}}%
\pgfpathcurveto{\pgfqpoint{5.416790in}{0.662342in}}{\pgfqpoint{5.412400in}{0.672941in}}{\pgfqpoint{5.404586in}{0.680755in}}%
\pgfpathcurveto{\pgfqpoint{5.396772in}{0.688568in}}{\pgfqpoint{5.386173in}{0.692958in}}{\pgfqpoint{5.375123in}{0.692958in}}%
\pgfpathcurveto{\pgfqpoint{5.364073in}{0.692958in}}{\pgfqpoint{5.353474in}{0.688568in}}{\pgfqpoint{5.345660in}{0.680755in}}%
\pgfpathcurveto{\pgfqpoint{5.337847in}{0.672941in}}{\pgfqpoint{5.333456in}{0.662342in}}{\pgfqpoint{5.333456in}{0.651292in}}%
\pgfpathcurveto{\pgfqpoint{5.333456in}{0.640242in}}{\pgfqpoint{5.337847in}{0.629643in}}{\pgfqpoint{5.345660in}{0.621829in}}%
\pgfpathcurveto{\pgfqpoint{5.353474in}{0.614015in}}{\pgfqpoint{5.364073in}{0.609625in}}{\pgfqpoint{5.375123in}{0.609625in}}%
\pgfpathlineto{\pgfqpoint{5.375123in}{0.609625in}}%
\pgfpathclose%
\pgfusepath{stroke}%
\end{pgfscope}%
\begin{pgfscope}%
\pgfpathrectangle{\pgfqpoint{0.847223in}{0.554012in}}{\pgfqpoint{6.200000in}{4.620000in}}%
\pgfusepath{clip}%
\pgfsetbuttcap%
\pgfsetroundjoin%
\pgfsetlinewidth{1.003750pt}%
\definecolor{currentstroke}{rgb}{1.000000,0.000000,0.000000}%
\pgfsetstrokecolor{currentstroke}%
\pgfsetdash{}{0pt}%
\pgfpathmoveto{\pgfqpoint{5.380456in}{0.609074in}}%
\pgfpathcurveto{\pgfqpoint{5.391506in}{0.609074in}}{\pgfqpoint{5.402106in}{0.613464in}}{\pgfqpoint{5.409919in}{0.621278in}}%
\pgfpathcurveto{\pgfqpoint{5.417733in}{0.629091in}}{\pgfqpoint{5.422123in}{0.639690in}}{\pgfqpoint{5.422123in}{0.650740in}}%
\pgfpathcurveto{\pgfqpoint{5.422123in}{0.661790in}}{\pgfqpoint{5.417733in}{0.672390in}}{\pgfqpoint{5.409919in}{0.680203in}}%
\pgfpathcurveto{\pgfqpoint{5.402106in}{0.688017in}}{\pgfqpoint{5.391506in}{0.692407in}}{\pgfqpoint{5.380456in}{0.692407in}}%
\pgfpathcurveto{\pgfqpoint{5.369406in}{0.692407in}}{\pgfqpoint{5.358807in}{0.688017in}}{\pgfqpoint{5.350994in}{0.680203in}}%
\pgfpathcurveto{\pgfqpoint{5.343180in}{0.672390in}}{\pgfqpoint{5.338790in}{0.661790in}}{\pgfqpoint{5.338790in}{0.650740in}}%
\pgfpathcurveto{\pgfqpoint{5.338790in}{0.639690in}}{\pgfqpoint{5.343180in}{0.629091in}}{\pgfqpoint{5.350994in}{0.621278in}}%
\pgfpathcurveto{\pgfqpoint{5.358807in}{0.613464in}}{\pgfqpoint{5.369406in}{0.609074in}}{\pgfqpoint{5.380456in}{0.609074in}}%
\pgfpathlineto{\pgfqpoint{5.380456in}{0.609074in}}%
\pgfpathclose%
\pgfusepath{stroke}%
\end{pgfscope}%
\begin{pgfscope}%
\pgfpathrectangle{\pgfqpoint{0.847223in}{0.554012in}}{\pgfqpoint{6.200000in}{4.620000in}}%
\pgfusepath{clip}%
\pgfsetbuttcap%
\pgfsetroundjoin%
\pgfsetlinewidth{1.003750pt}%
\definecolor{currentstroke}{rgb}{1.000000,0.000000,0.000000}%
\pgfsetstrokecolor{currentstroke}%
\pgfsetdash{}{0pt}%
\pgfpathmoveto{\pgfqpoint{5.385790in}{0.608524in}}%
\pgfpathcurveto{\pgfqpoint{5.396840in}{0.608524in}}{\pgfqpoint{5.407439in}{0.612914in}}{\pgfqpoint{5.415252in}{0.620727in}}%
\pgfpathcurveto{\pgfqpoint{5.423066in}{0.628541in}}{\pgfqpoint{5.427456in}{0.639140in}}{\pgfqpoint{5.427456in}{0.650190in}}%
\pgfpathcurveto{\pgfqpoint{5.427456in}{0.661240in}}{\pgfqpoint{5.423066in}{0.671839in}}{\pgfqpoint{5.415252in}{0.679653in}}%
\pgfpathcurveto{\pgfqpoint{5.407439in}{0.687467in}}{\pgfqpoint{5.396840in}{0.691857in}}{\pgfqpoint{5.385790in}{0.691857in}}%
\pgfpathcurveto{\pgfqpoint{5.374739in}{0.691857in}}{\pgfqpoint{5.364140in}{0.687467in}}{\pgfqpoint{5.356327in}{0.679653in}}%
\pgfpathcurveto{\pgfqpoint{5.348513in}{0.671839in}}{\pgfqpoint{5.344123in}{0.661240in}}{\pgfqpoint{5.344123in}{0.650190in}}%
\pgfpathcurveto{\pgfqpoint{5.344123in}{0.639140in}}{\pgfqpoint{5.348513in}{0.628541in}}{\pgfqpoint{5.356327in}{0.620727in}}%
\pgfpathcurveto{\pgfqpoint{5.364140in}{0.612914in}}{\pgfqpoint{5.374739in}{0.608524in}}{\pgfqpoint{5.385790in}{0.608524in}}%
\pgfpathlineto{\pgfqpoint{5.385790in}{0.608524in}}%
\pgfpathclose%
\pgfusepath{stroke}%
\end{pgfscope}%
\begin{pgfscope}%
\pgfpathrectangle{\pgfqpoint{0.847223in}{0.554012in}}{\pgfqpoint{6.200000in}{4.620000in}}%
\pgfusepath{clip}%
\pgfsetbuttcap%
\pgfsetroundjoin%
\pgfsetlinewidth{1.003750pt}%
\definecolor{currentstroke}{rgb}{1.000000,0.000000,0.000000}%
\pgfsetstrokecolor{currentstroke}%
\pgfsetdash{}{0pt}%
\pgfpathmoveto{\pgfqpoint{5.391123in}{0.607974in}}%
\pgfpathcurveto{\pgfqpoint{5.402173in}{0.607974in}}{\pgfqpoint{5.412772in}{0.612365in}}{\pgfqpoint{5.420586in}{0.620178in}}%
\pgfpathcurveto{\pgfqpoint{5.428399in}{0.627992in}}{\pgfqpoint{5.432789in}{0.638591in}}{\pgfqpoint{5.432789in}{0.649641in}}%
\pgfpathcurveto{\pgfqpoint{5.432789in}{0.660691in}}{\pgfqpoint{5.428399in}{0.671290in}}{\pgfqpoint{5.420586in}{0.679104in}}%
\pgfpathcurveto{\pgfqpoint{5.412772in}{0.686918in}}{\pgfqpoint{5.402173in}{0.691308in}}{\pgfqpoint{5.391123in}{0.691308in}}%
\pgfpathcurveto{\pgfqpoint{5.380073in}{0.691308in}}{\pgfqpoint{5.369474in}{0.686918in}}{\pgfqpoint{5.361660in}{0.679104in}}%
\pgfpathcurveto{\pgfqpoint{5.353846in}{0.671290in}}{\pgfqpoint{5.349456in}{0.660691in}}{\pgfqpoint{5.349456in}{0.649641in}}%
\pgfpathcurveto{\pgfqpoint{5.349456in}{0.638591in}}{\pgfqpoint{5.353846in}{0.627992in}}{\pgfqpoint{5.361660in}{0.620178in}}%
\pgfpathcurveto{\pgfqpoint{5.369474in}{0.612365in}}{\pgfqpoint{5.380073in}{0.607974in}}{\pgfqpoint{5.391123in}{0.607974in}}%
\pgfpathlineto{\pgfqpoint{5.391123in}{0.607974in}}%
\pgfpathclose%
\pgfusepath{stroke}%
\end{pgfscope}%
\begin{pgfscope}%
\pgfpathrectangle{\pgfqpoint{0.847223in}{0.554012in}}{\pgfqpoint{6.200000in}{4.620000in}}%
\pgfusepath{clip}%
\pgfsetbuttcap%
\pgfsetroundjoin%
\pgfsetlinewidth{1.003750pt}%
\definecolor{currentstroke}{rgb}{1.000000,0.000000,0.000000}%
\pgfsetstrokecolor{currentstroke}%
\pgfsetdash{}{0pt}%
\pgfpathmoveto{\pgfqpoint{5.396456in}{0.607427in}}%
\pgfpathcurveto{\pgfqpoint{5.407506in}{0.607427in}}{\pgfqpoint{5.418105in}{0.611817in}}{\pgfqpoint{5.425919in}{0.619630in}}%
\pgfpathcurveto{\pgfqpoint{5.433732in}{0.627444in}}{\pgfqpoint{5.438123in}{0.638043in}}{\pgfqpoint{5.438123in}{0.649093in}}%
\pgfpathcurveto{\pgfqpoint{5.438123in}{0.660143in}}{\pgfqpoint{5.433732in}{0.670742in}}{\pgfqpoint{5.425919in}{0.678556in}}%
\pgfpathcurveto{\pgfqpoint{5.418105in}{0.686370in}}{\pgfqpoint{5.407506in}{0.690760in}}{\pgfqpoint{5.396456in}{0.690760in}}%
\pgfpathcurveto{\pgfqpoint{5.385406in}{0.690760in}}{\pgfqpoint{5.374807in}{0.686370in}}{\pgfqpoint{5.366993in}{0.678556in}}%
\pgfpathcurveto{\pgfqpoint{5.359180in}{0.670742in}}{\pgfqpoint{5.354789in}{0.660143in}}{\pgfqpoint{5.354789in}{0.649093in}}%
\pgfpathcurveto{\pgfqpoint{5.354789in}{0.638043in}}{\pgfqpoint{5.359180in}{0.627444in}}{\pgfqpoint{5.366993in}{0.619630in}}%
\pgfpathcurveto{\pgfqpoint{5.374807in}{0.611817in}}{\pgfqpoint{5.385406in}{0.607427in}}{\pgfqpoint{5.396456in}{0.607427in}}%
\pgfpathlineto{\pgfqpoint{5.396456in}{0.607427in}}%
\pgfpathclose%
\pgfusepath{stroke}%
\end{pgfscope}%
\begin{pgfscope}%
\pgfpathrectangle{\pgfqpoint{0.847223in}{0.554012in}}{\pgfqpoint{6.200000in}{4.620000in}}%
\pgfusepath{clip}%
\pgfsetbuttcap%
\pgfsetroundjoin%
\pgfsetlinewidth{1.003750pt}%
\definecolor{currentstroke}{rgb}{1.000000,0.000000,0.000000}%
\pgfsetstrokecolor{currentstroke}%
\pgfsetdash{}{0pt}%
\pgfpathmoveto{\pgfqpoint{5.401789in}{0.606880in}}%
\pgfpathcurveto{\pgfqpoint{5.412839in}{0.606880in}}{\pgfqpoint{5.423438in}{0.611270in}}{\pgfqpoint{5.431252in}{0.619084in}}%
\pgfpathcurveto{\pgfqpoint{5.439066in}{0.626897in}}{\pgfqpoint{5.443456in}{0.637496in}}{\pgfqpoint{5.443456in}{0.648547in}}%
\pgfpathcurveto{\pgfqpoint{5.443456in}{0.659597in}}{\pgfqpoint{5.439066in}{0.670196in}}{\pgfqpoint{5.431252in}{0.678009in}}%
\pgfpathcurveto{\pgfqpoint{5.423438in}{0.685823in}}{\pgfqpoint{5.412839in}{0.690213in}}{\pgfqpoint{5.401789in}{0.690213in}}%
\pgfpathcurveto{\pgfqpoint{5.390739in}{0.690213in}}{\pgfqpoint{5.380140in}{0.685823in}}{\pgfqpoint{5.372326in}{0.678009in}}%
\pgfpathcurveto{\pgfqpoint{5.364513in}{0.670196in}}{\pgfqpoint{5.360123in}{0.659597in}}{\pgfqpoint{5.360123in}{0.648547in}}%
\pgfpathcurveto{\pgfqpoint{5.360123in}{0.637496in}}{\pgfqpoint{5.364513in}{0.626897in}}{\pgfqpoint{5.372326in}{0.619084in}}%
\pgfpathcurveto{\pgfqpoint{5.380140in}{0.611270in}}{\pgfqpoint{5.390739in}{0.606880in}}{\pgfqpoint{5.401789in}{0.606880in}}%
\pgfpathlineto{\pgfqpoint{5.401789in}{0.606880in}}%
\pgfpathclose%
\pgfusepath{stroke}%
\end{pgfscope}%
\begin{pgfscope}%
\pgfpathrectangle{\pgfqpoint{0.847223in}{0.554012in}}{\pgfqpoint{6.200000in}{4.620000in}}%
\pgfusepath{clip}%
\pgfsetbuttcap%
\pgfsetroundjoin%
\pgfsetlinewidth{1.003750pt}%
\definecolor{currentstroke}{rgb}{1.000000,0.000000,0.000000}%
\pgfsetstrokecolor{currentstroke}%
\pgfsetdash{}{0pt}%
\pgfpathmoveto{\pgfqpoint{5.407122in}{0.606334in}}%
\pgfpathcurveto{\pgfqpoint{5.418173in}{0.606334in}}{\pgfqpoint{5.428772in}{0.610725in}}{\pgfqpoint{5.436585in}{0.618538in}}%
\pgfpathcurveto{\pgfqpoint{5.444399in}{0.626352in}}{\pgfqpoint{5.448789in}{0.636951in}}{\pgfqpoint{5.448789in}{0.648001in}}%
\pgfpathcurveto{\pgfqpoint{5.448789in}{0.659051in}}{\pgfqpoint{5.444399in}{0.669650in}}{\pgfqpoint{5.436585in}{0.677464in}}%
\pgfpathcurveto{\pgfqpoint{5.428772in}{0.685277in}}{\pgfqpoint{5.418173in}{0.689668in}}{\pgfqpoint{5.407122in}{0.689668in}}%
\pgfpathcurveto{\pgfqpoint{5.396072in}{0.689668in}}{\pgfqpoint{5.385473in}{0.685277in}}{\pgfqpoint{5.377660in}{0.677464in}}%
\pgfpathcurveto{\pgfqpoint{5.369846in}{0.669650in}}{\pgfqpoint{5.365456in}{0.659051in}}{\pgfqpoint{5.365456in}{0.648001in}}%
\pgfpathcurveto{\pgfqpoint{5.365456in}{0.636951in}}{\pgfqpoint{5.369846in}{0.626352in}}{\pgfqpoint{5.377660in}{0.618538in}}%
\pgfpathcurveto{\pgfqpoint{5.385473in}{0.610725in}}{\pgfqpoint{5.396072in}{0.606334in}}{\pgfqpoint{5.407122in}{0.606334in}}%
\pgfpathlineto{\pgfqpoint{5.407122in}{0.606334in}}%
\pgfpathclose%
\pgfusepath{stroke}%
\end{pgfscope}%
\begin{pgfscope}%
\pgfpathrectangle{\pgfqpoint{0.847223in}{0.554012in}}{\pgfqpoint{6.200000in}{4.620000in}}%
\pgfusepath{clip}%
\pgfsetbuttcap%
\pgfsetroundjoin%
\pgfsetlinewidth{1.003750pt}%
\definecolor{currentstroke}{rgb}{1.000000,0.000000,0.000000}%
\pgfsetstrokecolor{currentstroke}%
\pgfsetdash{}{0pt}%
\pgfpathmoveto{\pgfqpoint{5.412456in}{0.605790in}}%
\pgfpathcurveto{\pgfqpoint{5.423506in}{0.605790in}}{\pgfqpoint{5.434105in}{0.610180in}}{\pgfqpoint{5.441918in}{0.617994in}}%
\pgfpathcurveto{\pgfqpoint{5.449732in}{0.625807in}}{\pgfqpoint{5.454122in}{0.636406in}}{\pgfqpoint{5.454122in}{0.647457in}}%
\pgfpathcurveto{\pgfqpoint{5.454122in}{0.658507in}}{\pgfqpoint{5.449732in}{0.669106in}}{\pgfqpoint{5.441918in}{0.676919in}}%
\pgfpathcurveto{\pgfqpoint{5.434105in}{0.684733in}}{\pgfqpoint{5.423506in}{0.689123in}}{\pgfqpoint{5.412456in}{0.689123in}}%
\pgfpathcurveto{\pgfqpoint{5.401406in}{0.689123in}}{\pgfqpoint{5.390806in}{0.684733in}}{\pgfqpoint{5.382993in}{0.676919in}}%
\pgfpathcurveto{\pgfqpoint{5.375179in}{0.669106in}}{\pgfqpoint{5.370789in}{0.658507in}}{\pgfqpoint{5.370789in}{0.647457in}}%
\pgfpathcurveto{\pgfqpoint{5.370789in}{0.636406in}}{\pgfqpoint{5.375179in}{0.625807in}}{\pgfqpoint{5.382993in}{0.617994in}}%
\pgfpathcurveto{\pgfqpoint{5.390806in}{0.610180in}}{\pgfqpoint{5.401406in}{0.605790in}}{\pgfqpoint{5.412456in}{0.605790in}}%
\pgfpathlineto{\pgfqpoint{5.412456in}{0.605790in}}%
\pgfpathclose%
\pgfusepath{stroke}%
\end{pgfscope}%
\begin{pgfscope}%
\pgfpathrectangle{\pgfqpoint{0.847223in}{0.554012in}}{\pgfqpoint{6.200000in}{4.620000in}}%
\pgfusepath{clip}%
\pgfsetbuttcap%
\pgfsetroundjoin%
\pgfsetlinewidth{1.003750pt}%
\definecolor{currentstroke}{rgb}{1.000000,0.000000,0.000000}%
\pgfsetstrokecolor{currentstroke}%
\pgfsetdash{}{0pt}%
\pgfpathmoveto{\pgfqpoint{5.417789in}{0.605247in}}%
\pgfpathcurveto{\pgfqpoint{5.428839in}{0.605247in}}{\pgfqpoint{5.439438in}{0.609637in}}{\pgfqpoint{5.447252in}{0.617450in}}%
\pgfpathcurveto{\pgfqpoint{5.455065in}{0.625264in}}{\pgfqpoint{5.459456in}{0.635863in}}{\pgfqpoint{5.459456in}{0.646913in}}%
\pgfpathcurveto{\pgfqpoint{5.459456in}{0.657963in}}{\pgfqpoint{5.455065in}{0.668562in}}{\pgfqpoint{5.447252in}{0.676376in}}%
\pgfpathcurveto{\pgfqpoint{5.439438in}{0.684190in}}{\pgfqpoint{5.428839in}{0.688580in}}{\pgfqpoint{5.417789in}{0.688580in}}%
\pgfpathcurveto{\pgfqpoint{5.406739in}{0.688580in}}{\pgfqpoint{5.396140in}{0.684190in}}{\pgfqpoint{5.388326in}{0.676376in}}%
\pgfpathcurveto{\pgfqpoint{5.380512in}{0.668562in}}{\pgfqpoint{5.376122in}{0.657963in}}{\pgfqpoint{5.376122in}{0.646913in}}%
\pgfpathcurveto{\pgfqpoint{5.376122in}{0.635863in}}{\pgfqpoint{5.380512in}{0.625264in}}{\pgfqpoint{5.388326in}{0.617450in}}%
\pgfpathcurveto{\pgfqpoint{5.396140in}{0.609637in}}{\pgfqpoint{5.406739in}{0.605247in}}{\pgfqpoint{5.417789in}{0.605247in}}%
\pgfpathlineto{\pgfqpoint{5.417789in}{0.605247in}}%
\pgfpathclose%
\pgfusepath{stroke}%
\end{pgfscope}%
\begin{pgfscope}%
\pgfpathrectangle{\pgfqpoint{0.847223in}{0.554012in}}{\pgfqpoint{6.200000in}{4.620000in}}%
\pgfusepath{clip}%
\pgfsetbuttcap%
\pgfsetroundjoin%
\pgfsetlinewidth{1.003750pt}%
\definecolor{currentstroke}{rgb}{1.000000,0.000000,0.000000}%
\pgfsetstrokecolor{currentstroke}%
\pgfsetdash{}{0pt}%
\pgfpathmoveto{\pgfqpoint{5.423122in}{0.604704in}}%
\pgfpathcurveto{\pgfqpoint{5.434172in}{0.604704in}}{\pgfqpoint{5.444771in}{0.609095in}}{\pgfqpoint{5.452585in}{0.616908in}}%
\pgfpathcurveto{\pgfqpoint{5.460398in}{0.624722in}}{\pgfqpoint{5.464789in}{0.635321in}}{\pgfqpoint{5.464789in}{0.646371in}}%
\pgfpathcurveto{\pgfqpoint{5.464789in}{0.657421in}}{\pgfqpoint{5.460398in}{0.668020in}}{\pgfqpoint{5.452585in}{0.675834in}}%
\pgfpathcurveto{\pgfqpoint{5.444771in}{0.683647in}}{\pgfqpoint{5.434172in}{0.688038in}}{\pgfqpoint{5.423122in}{0.688038in}}%
\pgfpathcurveto{\pgfqpoint{5.412072in}{0.688038in}}{\pgfqpoint{5.401473in}{0.683647in}}{\pgfqpoint{5.393659in}{0.675834in}}%
\pgfpathcurveto{\pgfqpoint{5.385846in}{0.668020in}}{\pgfqpoint{5.381455in}{0.657421in}}{\pgfqpoint{5.381455in}{0.646371in}}%
\pgfpathcurveto{\pgfqpoint{5.381455in}{0.635321in}}{\pgfqpoint{5.385846in}{0.624722in}}{\pgfqpoint{5.393659in}{0.616908in}}%
\pgfpathcurveto{\pgfqpoint{5.401473in}{0.609095in}}{\pgfqpoint{5.412072in}{0.604704in}}{\pgfqpoint{5.423122in}{0.604704in}}%
\pgfpathlineto{\pgfqpoint{5.423122in}{0.604704in}}%
\pgfpathclose%
\pgfusepath{stroke}%
\end{pgfscope}%
\begin{pgfscope}%
\pgfpathrectangle{\pgfqpoint{0.847223in}{0.554012in}}{\pgfqpoint{6.200000in}{4.620000in}}%
\pgfusepath{clip}%
\pgfsetbuttcap%
\pgfsetroundjoin%
\pgfsetlinewidth{1.003750pt}%
\definecolor{currentstroke}{rgb}{1.000000,0.000000,0.000000}%
\pgfsetstrokecolor{currentstroke}%
\pgfsetdash{}{0pt}%
\pgfpathmoveto{\pgfqpoint{5.428455in}{0.604163in}}%
\pgfpathcurveto{\pgfqpoint{5.439505in}{0.604163in}}{\pgfqpoint{5.450104in}{0.608554in}}{\pgfqpoint{5.457918in}{0.616367in}}%
\pgfpathcurveto{\pgfqpoint{5.465732in}{0.624181in}}{\pgfqpoint{5.470122in}{0.634780in}}{\pgfqpoint{5.470122in}{0.645830in}}%
\pgfpathcurveto{\pgfqpoint{5.470122in}{0.656880in}}{\pgfqpoint{5.465732in}{0.667479in}}{\pgfqpoint{5.457918in}{0.675293in}}%
\pgfpathcurveto{\pgfqpoint{5.450104in}{0.683106in}}{\pgfqpoint{5.439505in}{0.687497in}}{\pgfqpoint{5.428455in}{0.687497in}}%
\pgfpathcurveto{\pgfqpoint{5.417405in}{0.687497in}}{\pgfqpoint{5.406806in}{0.683106in}}{\pgfqpoint{5.398993in}{0.675293in}}%
\pgfpathcurveto{\pgfqpoint{5.391179in}{0.667479in}}{\pgfqpoint{5.386789in}{0.656880in}}{\pgfqpoint{5.386789in}{0.645830in}}%
\pgfpathcurveto{\pgfqpoint{5.386789in}{0.634780in}}{\pgfqpoint{5.391179in}{0.624181in}}{\pgfqpoint{5.398993in}{0.616367in}}%
\pgfpathcurveto{\pgfqpoint{5.406806in}{0.608554in}}{\pgfqpoint{5.417405in}{0.604163in}}{\pgfqpoint{5.428455in}{0.604163in}}%
\pgfpathlineto{\pgfqpoint{5.428455in}{0.604163in}}%
\pgfpathclose%
\pgfusepath{stroke}%
\end{pgfscope}%
\begin{pgfscope}%
\pgfpathrectangle{\pgfqpoint{0.847223in}{0.554012in}}{\pgfqpoint{6.200000in}{4.620000in}}%
\pgfusepath{clip}%
\pgfsetbuttcap%
\pgfsetroundjoin%
\pgfsetlinewidth{1.003750pt}%
\definecolor{currentstroke}{rgb}{1.000000,0.000000,0.000000}%
\pgfsetstrokecolor{currentstroke}%
\pgfsetdash{}{0pt}%
\pgfpathmoveto{\pgfqpoint{5.433789in}{0.603624in}}%
\pgfpathcurveto{\pgfqpoint{5.444839in}{0.603624in}}{\pgfqpoint{5.455438in}{0.608014in}}{\pgfqpoint{5.463251in}{0.615827in}}%
\pgfpathcurveto{\pgfqpoint{5.471065in}{0.623641in}}{\pgfqpoint{5.475455in}{0.634240in}}{\pgfqpoint{5.475455in}{0.645290in}}%
\pgfpathcurveto{\pgfqpoint{5.475455in}{0.656340in}}{\pgfqpoint{5.471065in}{0.666939in}}{\pgfqpoint{5.463251in}{0.674753in}}%
\pgfpathcurveto{\pgfqpoint{5.455438in}{0.682567in}}{\pgfqpoint{5.444839in}{0.686957in}}{\pgfqpoint{5.433789in}{0.686957in}}%
\pgfpathcurveto{\pgfqpoint{5.422738in}{0.686957in}}{\pgfqpoint{5.412139in}{0.682567in}}{\pgfqpoint{5.404326in}{0.674753in}}%
\pgfpathcurveto{\pgfqpoint{5.396512in}{0.666939in}}{\pgfqpoint{5.392122in}{0.656340in}}{\pgfqpoint{5.392122in}{0.645290in}}%
\pgfpathcurveto{\pgfqpoint{5.392122in}{0.634240in}}{\pgfqpoint{5.396512in}{0.623641in}}{\pgfqpoint{5.404326in}{0.615827in}}%
\pgfpathcurveto{\pgfqpoint{5.412139in}{0.608014in}}{\pgfqpoint{5.422738in}{0.603624in}}{\pgfqpoint{5.433789in}{0.603624in}}%
\pgfpathlineto{\pgfqpoint{5.433789in}{0.603624in}}%
\pgfpathclose%
\pgfusepath{stroke}%
\end{pgfscope}%
\begin{pgfscope}%
\pgfpathrectangle{\pgfqpoint{0.847223in}{0.554012in}}{\pgfqpoint{6.200000in}{4.620000in}}%
\pgfusepath{clip}%
\pgfsetbuttcap%
\pgfsetroundjoin%
\pgfsetlinewidth{1.003750pt}%
\definecolor{currentstroke}{rgb}{1.000000,0.000000,0.000000}%
\pgfsetstrokecolor{currentstroke}%
\pgfsetdash{}{0pt}%
\pgfpathmoveto{\pgfqpoint{5.439122in}{0.603085in}}%
\pgfpathcurveto{\pgfqpoint{5.450172in}{0.603085in}}{\pgfqpoint{5.460771in}{0.607475in}}{\pgfqpoint{5.468585in}{0.615289in}}%
\pgfpathcurveto{\pgfqpoint{5.476398in}{0.623102in}}{\pgfqpoint{5.480788in}{0.633701in}}{\pgfqpoint{5.480788in}{0.644751in}}%
\pgfpathcurveto{\pgfqpoint{5.480788in}{0.655802in}}{\pgfqpoint{5.476398in}{0.666401in}}{\pgfqpoint{5.468585in}{0.674214in}}%
\pgfpathcurveto{\pgfqpoint{5.460771in}{0.682028in}}{\pgfqpoint{5.450172in}{0.686418in}}{\pgfqpoint{5.439122in}{0.686418in}}%
\pgfpathcurveto{\pgfqpoint{5.428072in}{0.686418in}}{\pgfqpoint{5.417473in}{0.682028in}}{\pgfqpoint{5.409659in}{0.674214in}}%
\pgfpathcurveto{\pgfqpoint{5.401845in}{0.666401in}}{\pgfqpoint{5.397455in}{0.655802in}}{\pgfqpoint{5.397455in}{0.644751in}}%
\pgfpathcurveto{\pgfqpoint{5.397455in}{0.633701in}}{\pgfqpoint{5.401845in}{0.623102in}}{\pgfqpoint{5.409659in}{0.615289in}}%
\pgfpathcurveto{\pgfqpoint{5.417473in}{0.607475in}}{\pgfqpoint{5.428072in}{0.603085in}}{\pgfqpoint{5.439122in}{0.603085in}}%
\pgfpathlineto{\pgfqpoint{5.439122in}{0.603085in}}%
\pgfpathclose%
\pgfusepath{stroke}%
\end{pgfscope}%
\begin{pgfscope}%
\pgfpathrectangle{\pgfqpoint{0.847223in}{0.554012in}}{\pgfqpoint{6.200000in}{4.620000in}}%
\pgfusepath{clip}%
\pgfsetbuttcap%
\pgfsetroundjoin%
\pgfsetlinewidth{1.003750pt}%
\definecolor{currentstroke}{rgb}{1.000000,0.000000,0.000000}%
\pgfsetstrokecolor{currentstroke}%
\pgfsetdash{}{0pt}%
\pgfpathmoveto{\pgfqpoint{5.444455in}{0.602547in}}%
\pgfpathcurveto{\pgfqpoint{5.455505in}{0.602547in}}{\pgfqpoint{5.466104in}{0.606937in}}{\pgfqpoint{5.473918in}{0.614751in}}%
\pgfpathcurveto{\pgfqpoint{5.481731in}{0.622565in}}{\pgfqpoint{5.486122in}{0.633164in}}{\pgfqpoint{5.486122in}{0.644214in}}%
\pgfpathcurveto{\pgfqpoint{5.486122in}{0.655264in}}{\pgfqpoint{5.481731in}{0.665863in}}{\pgfqpoint{5.473918in}{0.673677in}}%
\pgfpathcurveto{\pgfqpoint{5.466104in}{0.681490in}}{\pgfqpoint{5.455505in}{0.685880in}}{\pgfqpoint{5.444455in}{0.685880in}}%
\pgfpathcurveto{\pgfqpoint{5.433405in}{0.685880in}}{\pgfqpoint{5.422806in}{0.681490in}}{\pgfqpoint{5.414992in}{0.673677in}}%
\pgfpathcurveto{\pgfqpoint{5.407179in}{0.665863in}}{\pgfqpoint{5.402788in}{0.655264in}}{\pgfqpoint{5.402788in}{0.644214in}}%
\pgfpathcurveto{\pgfqpoint{5.402788in}{0.633164in}}{\pgfqpoint{5.407179in}{0.622565in}}{\pgfqpoint{5.414992in}{0.614751in}}%
\pgfpathcurveto{\pgfqpoint{5.422806in}{0.606937in}}{\pgfqpoint{5.433405in}{0.602547in}}{\pgfqpoint{5.444455in}{0.602547in}}%
\pgfpathlineto{\pgfqpoint{5.444455in}{0.602547in}}%
\pgfpathclose%
\pgfusepath{stroke}%
\end{pgfscope}%
\begin{pgfscope}%
\pgfpathrectangle{\pgfqpoint{0.847223in}{0.554012in}}{\pgfqpoint{6.200000in}{4.620000in}}%
\pgfusepath{clip}%
\pgfsetbuttcap%
\pgfsetroundjoin%
\pgfsetlinewidth{1.003750pt}%
\definecolor{currentstroke}{rgb}{1.000000,0.000000,0.000000}%
\pgfsetstrokecolor{currentstroke}%
\pgfsetdash{}{0pt}%
\pgfpathmoveto{\pgfqpoint{5.449788in}{0.602011in}}%
\pgfpathcurveto{\pgfqpoint{5.460838in}{0.602011in}}{\pgfqpoint{5.471437in}{0.606401in}}{\pgfqpoint{5.479251in}{0.614214in}}%
\pgfpathcurveto{\pgfqpoint{5.487065in}{0.622028in}}{\pgfqpoint{5.491455in}{0.632627in}}{\pgfqpoint{5.491455in}{0.643677in}}%
\pgfpathcurveto{\pgfqpoint{5.491455in}{0.654727in}}{\pgfqpoint{5.487065in}{0.665326in}}{\pgfqpoint{5.479251in}{0.673140in}}%
\pgfpathcurveto{\pgfqpoint{5.471437in}{0.680954in}}{\pgfqpoint{5.460838in}{0.685344in}}{\pgfqpoint{5.449788in}{0.685344in}}%
\pgfpathcurveto{\pgfqpoint{5.438738in}{0.685344in}}{\pgfqpoint{5.428139in}{0.680954in}}{\pgfqpoint{5.420325in}{0.673140in}}%
\pgfpathcurveto{\pgfqpoint{5.412512in}{0.665326in}}{\pgfqpoint{5.408122in}{0.654727in}}{\pgfqpoint{5.408122in}{0.643677in}}%
\pgfpathcurveto{\pgfqpoint{5.408122in}{0.632627in}}{\pgfqpoint{5.412512in}{0.622028in}}{\pgfqpoint{5.420325in}{0.614214in}}%
\pgfpathcurveto{\pgfqpoint{5.428139in}{0.606401in}}{\pgfqpoint{5.438738in}{0.602011in}}{\pgfqpoint{5.449788in}{0.602011in}}%
\pgfpathlineto{\pgfqpoint{5.449788in}{0.602011in}}%
\pgfpathclose%
\pgfusepath{stroke}%
\end{pgfscope}%
\begin{pgfscope}%
\pgfpathrectangle{\pgfqpoint{0.847223in}{0.554012in}}{\pgfqpoint{6.200000in}{4.620000in}}%
\pgfusepath{clip}%
\pgfsetbuttcap%
\pgfsetroundjoin%
\pgfsetlinewidth{1.003750pt}%
\definecolor{currentstroke}{rgb}{1.000000,0.000000,0.000000}%
\pgfsetstrokecolor{currentstroke}%
\pgfsetdash{}{0pt}%
\pgfpathmoveto{\pgfqpoint{5.455121in}{0.601475in}}%
\pgfpathcurveto{\pgfqpoint{5.466172in}{0.601475in}}{\pgfqpoint{5.476771in}{0.605865in}}{\pgfqpoint{5.484584in}{0.613679in}}%
\pgfpathcurveto{\pgfqpoint{5.492398in}{0.621493in}}{\pgfqpoint{5.496788in}{0.632092in}}{\pgfqpoint{5.496788in}{0.643142in}}%
\pgfpathcurveto{\pgfqpoint{5.496788in}{0.654192in}}{\pgfqpoint{5.492398in}{0.664791in}}{\pgfqpoint{5.484584in}{0.672605in}}%
\pgfpathcurveto{\pgfqpoint{5.476771in}{0.680418in}}{\pgfqpoint{5.466172in}{0.684808in}}{\pgfqpoint{5.455121in}{0.684808in}}%
\pgfpathcurveto{\pgfqpoint{5.444071in}{0.684808in}}{\pgfqpoint{5.433472in}{0.680418in}}{\pgfqpoint{5.425659in}{0.672605in}}%
\pgfpathcurveto{\pgfqpoint{5.417845in}{0.664791in}}{\pgfqpoint{5.413455in}{0.654192in}}{\pgfqpoint{5.413455in}{0.643142in}}%
\pgfpathcurveto{\pgfqpoint{5.413455in}{0.632092in}}{\pgfqpoint{5.417845in}{0.621493in}}{\pgfqpoint{5.425659in}{0.613679in}}%
\pgfpathcurveto{\pgfqpoint{5.433472in}{0.605865in}}{\pgfqpoint{5.444071in}{0.601475in}}{\pgfqpoint{5.455121in}{0.601475in}}%
\pgfpathlineto{\pgfqpoint{5.455121in}{0.601475in}}%
\pgfpathclose%
\pgfusepath{stroke}%
\end{pgfscope}%
\begin{pgfscope}%
\pgfpathrectangle{\pgfqpoint{0.847223in}{0.554012in}}{\pgfqpoint{6.200000in}{4.620000in}}%
\pgfusepath{clip}%
\pgfsetbuttcap%
\pgfsetroundjoin%
\pgfsetlinewidth{1.003750pt}%
\definecolor{currentstroke}{rgb}{1.000000,0.000000,0.000000}%
\pgfsetstrokecolor{currentstroke}%
\pgfsetdash{}{0pt}%
\pgfpathmoveto{\pgfqpoint{5.460455in}{0.600941in}}%
\pgfpathcurveto{\pgfqpoint{5.471505in}{0.600941in}}{\pgfqpoint{5.482104in}{0.605331in}}{\pgfqpoint{5.489917in}{0.613145in}}%
\pgfpathcurveto{\pgfqpoint{5.497731in}{0.620958in}}{\pgfqpoint{5.502121in}{0.631557in}}{\pgfqpoint{5.502121in}{0.642608in}}%
\pgfpathcurveto{\pgfqpoint{5.502121in}{0.653658in}}{\pgfqpoint{5.497731in}{0.664257in}}{\pgfqpoint{5.489917in}{0.672070in}}%
\pgfpathcurveto{\pgfqpoint{5.482104in}{0.679884in}}{\pgfqpoint{5.471505in}{0.684274in}}{\pgfqpoint{5.460455in}{0.684274in}}%
\pgfpathcurveto{\pgfqpoint{5.449404in}{0.684274in}}{\pgfqpoint{5.438805in}{0.679884in}}{\pgfqpoint{5.430992in}{0.672070in}}%
\pgfpathcurveto{\pgfqpoint{5.423178in}{0.664257in}}{\pgfqpoint{5.418788in}{0.653658in}}{\pgfqpoint{5.418788in}{0.642608in}}%
\pgfpathcurveto{\pgfqpoint{5.418788in}{0.631557in}}{\pgfqpoint{5.423178in}{0.620958in}}{\pgfqpoint{5.430992in}{0.613145in}}%
\pgfpathcurveto{\pgfqpoint{5.438805in}{0.605331in}}{\pgfqpoint{5.449404in}{0.600941in}}{\pgfqpoint{5.460455in}{0.600941in}}%
\pgfpathlineto{\pgfqpoint{5.460455in}{0.600941in}}%
\pgfpathclose%
\pgfusepath{stroke}%
\end{pgfscope}%
\begin{pgfscope}%
\pgfpathrectangle{\pgfqpoint{0.847223in}{0.554012in}}{\pgfqpoint{6.200000in}{4.620000in}}%
\pgfusepath{clip}%
\pgfsetbuttcap%
\pgfsetroundjoin%
\pgfsetlinewidth{1.003750pt}%
\definecolor{currentstroke}{rgb}{1.000000,0.000000,0.000000}%
\pgfsetstrokecolor{currentstroke}%
\pgfsetdash{}{0pt}%
\pgfpathmoveto{\pgfqpoint{5.465788in}{0.600408in}}%
\pgfpathcurveto{\pgfqpoint{5.476838in}{0.600408in}}{\pgfqpoint{5.487437in}{0.604798in}}{\pgfqpoint{5.495251in}{0.612612in}}%
\pgfpathcurveto{\pgfqpoint{5.503064in}{0.620425in}}{\pgfqpoint{5.507454in}{0.631024in}}{\pgfqpoint{5.507454in}{0.642074in}}%
\pgfpathcurveto{\pgfqpoint{5.507454in}{0.653124in}}{\pgfqpoint{5.503064in}{0.663723in}}{\pgfqpoint{5.495251in}{0.671537in}}%
\pgfpathcurveto{\pgfqpoint{5.487437in}{0.679351in}}{\pgfqpoint{5.476838in}{0.683741in}}{\pgfqpoint{5.465788in}{0.683741in}}%
\pgfpathcurveto{\pgfqpoint{5.454738in}{0.683741in}}{\pgfqpoint{5.444139in}{0.679351in}}{\pgfqpoint{5.436325in}{0.671537in}}%
\pgfpathcurveto{\pgfqpoint{5.428511in}{0.663723in}}{\pgfqpoint{5.424121in}{0.653124in}}{\pgfqpoint{5.424121in}{0.642074in}}%
\pgfpathcurveto{\pgfqpoint{5.424121in}{0.631024in}}{\pgfqpoint{5.428511in}{0.620425in}}{\pgfqpoint{5.436325in}{0.612612in}}%
\pgfpathcurveto{\pgfqpoint{5.444139in}{0.604798in}}{\pgfqpoint{5.454738in}{0.600408in}}{\pgfqpoint{5.465788in}{0.600408in}}%
\pgfpathlineto{\pgfqpoint{5.465788in}{0.600408in}}%
\pgfpathclose%
\pgfusepath{stroke}%
\end{pgfscope}%
\begin{pgfscope}%
\pgfpathrectangle{\pgfqpoint{0.847223in}{0.554012in}}{\pgfqpoint{6.200000in}{4.620000in}}%
\pgfusepath{clip}%
\pgfsetbuttcap%
\pgfsetroundjoin%
\pgfsetlinewidth{1.003750pt}%
\definecolor{currentstroke}{rgb}{1.000000,0.000000,0.000000}%
\pgfsetstrokecolor{currentstroke}%
\pgfsetdash{}{0pt}%
\pgfpathmoveto{\pgfqpoint{5.471121in}{0.599876in}}%
\pgfpathcurveto{\pgfqpoint{5.482171in}{0.599876in}}{\pgfqpoint{5.492770in}{0.604266in}}{\pgfqpoint{5.500584in}{0.612079in}}%
\pgfpathcurveto{\pgfqpoint{5.508397in}{0.619893in}}{\pgfqpoint{5.512788in}{0.630492in}}{\pgfqpoint{5.512788in}{0.641542in}}%
\pgfpathcurveto{\pgfqpoint{5.512788in}{0.652592in}}{\pgfqpoint{5.508397in}{0.663191in}}{\pgfqpoint{5.500584in}{0.671005in}}%
\pgfpathcurveto{\pgfqpoint{5.492770in}{0.678819in}}{\pgfqpoint{5.482171in}{0.683209in}}{\pgfqpoint{5.471121in}{0.683209in}}%
\pgfpathcurveto{\pgfqpoint{5.460071in}{0.683209in}}{\pgfqpoint{5.449472in}{0.678819in}}{\pgfqpoint{5.441658in}{0.671005in}}%
\pgfpathcurveto{\pgfqpoint{5.433845in}{0.663191in}}{\pgfqpoint{5.429454in}{0.652592in}}{\pgfqpoint{5.429454in}{0.641542in}}%
\pgfpathcurveto{\pgfqpoint{5.429454in}{0.630492in}}{\pgfqpoint{5.433845in}{0.619893in}}{\pgfqpoint{5.441658in}{0.612079in}}%
\pgfpathcurveto{\pgfqpoint{5.449472in}{0.604266in}}{\pgfqpoint{5.460071in}{0.599876in}}{\pgfqpoint{5.471121in}{0.599876in}}%
\pgfpathlineto{\pgfqpoint{5.471121in}{0.599876in}}%
\pgfpathclose%
\pgfusepath{stroke}%
\end{pgfscope}%
\begin{pgfscope}%
\pgfpathrectangle{\pgfqpoint{0.847223in}{0.554012in}}{\pgfqpoint{6.200000in}{4.620000in}}%
\pgfusepath{clip}%
\pgfsetbuttcap%
\pgfsetroundjoin%
\pgfsetlinewidth{1.003750pt}%
\definecolor{currentstroke}{rgb}{1.000000,0.000000,0.000000}%
\pgfsetstrokecolor{currentstroke}%
\pgfsetdash{}{0pt}%
\pgfpathmoveto{\pgfqpoint{5.476454in}{0.599345in}}%
\pgfpathcurveto{\pgfqpoint{5.487504in}{0.599345in}}{\pgfqpoint{5.498103in}{0.603735in}}{\pgfqpoint{5.505917in}{0.611548in}}%
\pgfpathcurveto{\pgfqpoint{5.513731in}{0.619362in}}{\pgfqpoint{5.518121in}{0.629961in}}{\pgfqpoint{5.518121in}{0.641011in}}%
\pgfpathcurveto{\pgfqpoint{5.518121in}{0.652061in}}{\pgfqpoint{5.513731in}{0.662660in}}{\pgfqpoint{5.505917in}{0.670474in}}%
\pgfpathcurveto{\pgfqpoint{5.498103in}{0.678288in}}{\pgfqpoint{5.487504in}{0.682678in}}{\pgfqpoint{5.476454in}{0.682678in}}%
\pgfpathcurveto{\pgfqpoint{5.465404in}{0.682678in}}{\pgfqpoint{5.454805in}{0.678288in}}{\pgfqpoint{5.446991in}{0.670474in}}%
\pgfpathcurveto{\pgfqpoint{5.439178in}{0.662660in}}{\pgfqpoint{5.434788in}{0.652061in}}{\pgfqpoint{5.434788in}{0.641011in}}%
\pgfpathcurveto{\pgfqpoint{5.434788in}{0.629961in}}{\pgfqpoint{5.439178in}{0.619362in}}{\pgfqpoint{5.446991in}{0.611548in}}%
\pgfpathcurveto{\pgfqpoint{5.454805in}{0.603735in}}{\pgfqpoint{5.465404in}{0.599345in}}{\pgfqpoint{5.476454in}{0.599345in}}%
\pgfpathlineto{\pgfqpoint{5.476454in}{0.599345in}}%
\pgfpathclose%
\pgfusepath{stroke}%
\end{pgfscope}%
\begin{pgfscope}%
\pgfpathrectangle{\pgfqpoint{0.847223in}{0.554012in}}{\pgfqpoint{6.200000in}{4.620000in}}%
\pgfusepath{clip}%
\pgfsetbuttcap%
\pgfsetroundjoin%
\pgfsetlinewidth{1.003750pt}%
\definecolor{currentstroke}{rgb}{1.000000,0.000000,0.000000}%
\pgfsetstrokecolor{currentstroke}%
\pgfsetdash{}{0pt}%
\pgfpathmoveto{\pgfqpoint{5.481787in}{0.598815in}}%
\pgfpathcurveto{\pgfqpoint{5.492838in}{0.598815in}}{\pgfqpoint{5.503437in}{0.603205in}}{\pgfqpoint{5.511250in}{0.611019in}}%
\pgfpathcurveto{\pgfqpoint{5.519064in}{0.618832in}}{\pgfqpoint{5.523454in}{0.629431in}}{\pgfqpoint{5.523454in}{0.640481in}}%
\pgfpathcurveto{\pgfqpoint{5.523454in}{0.651531in}}{\pgfqpoint{5.519064in}{0.662131in}}{\pgfqpoint{5.511250in}{0.669944in}}%
\pgfpathcurveto{\pgfqpoint{5.503437in}{0.677758in}}{\pgfqpoint{5.492838in}{0.682148in}}{\pgfqpoint{5.481787in}{0.682148in}}%
\pgfpathcurveto{\pgfqpoint{5.470737in}{0.682148in}}{\pgfqpoint{5.460138in}{0.677758in}}{\pgfqpoint{5.452325in}{0.669944in}}%
\pgfpathcurveto{\pgfqpoint{5.444511in}{0.662131in}}{\pgfqpoint{5.440121in}{0.651531in}}{\pgfqpoint{5.440121in}{0.640481in}}%
\pgfpathcurveto{\pgfqpoint{5.440121in}{0.629431in}}{\pgfqpoint{5.444511in}{0.618832in}}{\pgfqpoint{5.452325in}{0.611019in}}%
\pgfpathcurveto{\pgfqpoint{5.460138in}{0.603205in}}{\pgfqpoint{5.470737in}{0.598815in}}{\pgfqpoint{5.481787in}{0.598815in}}%
\pgfpathlineto{\pgfqpoint{5.481787in}{0.598815in}}%
\pgfpathclose%
\pgfusepath{stroke}%
\end{pgfscope}%
\begin{pgfscope}%
\pgfpathrectangle{\pgfqpoint{0.847223in}{0.554012in}}{\pgfqpoint{6.200000in}{4.620000in}}%
\pgfusepath{clip}%
\pgfsetbuttcap%
\pgfsetroundjoin%
\pgfsetlinewidth{1.003750pt}%
\definecolor{currentstroke}{rgb}{1.000000,0.000000,0.000000}%
\pgfsetstrokecolor{currentstroke}%
\pgfsetdash{}{0pt}%
\pgfpathmoveto{\pgfqpoint{5.487121in}{0.598286in}}%
\pgfpathcurveto{\pgfqpoint{5.498171in}{0.598286in}}{\pgfqpoint{5.508770in}{0.602676in}}{\pgfqpoint{5.516583in}{0.610490in}}%
\pgfpathcurveto{\pgfqpoint{5.524397in}{0.618303in}}{\pgfqpoint{5.528787in}{0.628902in}}{\pgfqpoint{5.528787in}{0.639953in}}%
\pgfpathcurveto{\pgfqpoint{5.528787in}{0.651003in}}{\pgfqpoint{5.524397in}{0.661602in}}{\pgfqpoint{5.516583in}{0.669415in}}%
\pgfpathcurveto{\pgfqpoint{5.508770in}{0.677229in}}{\pgfqpoint{5.498171in}{0.681619in}}{\pgfqpoint{5.487121in}{0.681619in}}%
\pgfpathcurveto{\pgfqpoint{5.476071in}{0.681619in}}{\pgfqpoint{5.465472in}{0.677229in}}{\pgfqpoint{5.457658in}{0.669415in}}%
\pgfpathcurveto{\pgfqpoint{5.449844in}{0.661602in}}{\pgfqpoint{5.445454in}{0.651003in}}{\pgfqpoint{5.445454in}{0.639953in}}%
\pgfpathcurveto{\pgfqpoint{5.445454in}{0.628902in}}{\pgfqpoint{5.449844in}{0.618303in}}{\pgfqpoint{5.457658in}{0.610490in}}%
\pgfpathcurveto{\pgfqpoint{5.465472in}{0.602676in}}{\pgfqpoint{5.476071in}{0.598286in}}{\pgfqpoint{5.487121in}{0.598286in}}%
\pgfpathlineto{\pgfqpoint{5.487121in}{0.598286in}}%
\pgfpathclose%
\pgfusepath{stroke}%
\end{pgfscope}%
\begin{pgfscope}%
\pgfpathrectangle{\pgfqpoint{0.847223in}{0.554012in}}{\pgfqpoint{6.200000in}{4.620000in}}%
\pgfusepath{clip}%
\pgfsetbuttcap%
\pgfsetroundjoin%
\pgfsetlinewidth{1.003750pt}%
\definecolor{currentstroke}{rgb}{1.000000,0.000000,0.000000}%
\pgfsetstrokecolor{currentstroke}%
\pgfsetdash{}{0pt}%
\pgfpathmoveto{\pgfqpoint{5.492454in}{0.597758in}}%
\pgfpathcurveto{\pgfqpoint{5.503504in}{0.597758in}}{\pgfqpoint{5.514103in}{0.602148in}}{\pgfqpoint{5.521917in}{0.609962in}}%
\pgfpathcurveto{\pgfqpoint{5.529730in}{0.617776in}}{\pgfqpoint{5.534121in}{0.628375in}}{\pgfqpoint{5.534121in}{0.639425in}}%
\pgfpathcurveto{\pgfqpoint{5.534121in}{0.650475in}}{\pgfqpoint{5.529730in}{0.661074in}}{\pgfqpoint{5.521917in}{0.668888in}}%
\pgfpathcurveto{\pgfqpoint{5.514103in}{0.676701in}}{\pgfqpoint{5.503504in}{0.681092in}}{\pgfqpoint{5.492454in}{0.681092in}}%
\pgfpathcurveto{\pgfqpoint{5.481404in}{0.681092in}}{\pgfqpoint{5.470805in}{0.676701in}}{\pgfqpoint{5.462991in}{0.668888in}}%
\pgfpathcurveto{\pgfqpoint{5.455177in}{0.661074in}}{\pgfqpoint{5.450787in}{0.650475in}}{\pgfqpoint{5.450787in}{0.639425in}}%
\pgfpathcurveto{\pgfqpoint{5.450787in}{0.628375in}}{\pgfqpoint{5.455177in}{0.617776in}}{\pgfqpoint{5.462991in}{0.609962in}}%
\pgfpathcurveto{\pgfqpoint{5.470805in}{0.602148in}}{\pgfqpoint{5.481404in}{0.597758in}}{\pgfqpoint{5.492454in}{0.597758in}}%
\pgfpathlineto{\pgfqpoint{5.492454in}{0.597758in}}%
\pgfpathclose%
\pgfusepath{stroke}%
\end{pgfscope}%
\begin{pgfscope}%
\pgfpathrectangle{\pgfqpoint{0.847223in}{0.554012in}}{\pgfqpoint{6.200000in}{4.620000in}}%
\pgfusepath{clip}%
\pgfsetbuttcap%
\pgfsetroundjoin%
\pgfsetlinewidth{1.003750pt}%
\definecolor{currentstroke}{rgb}{1.000000,0.000000,0.000000}%
\pgfsetstrokecolor{currentstroke}%
\pgfsetdash{}{0pt}%
\pgfpathmoveto{\pgfqpoint{5.497787in}{0.597232in}}%
\pgfpathcurveto{\pgfqpoint{5.508837in}{0.597232in}}{\pgfqpoint{5.519436in}{0.601622in}}{\pgfqpoint{5.527250in}{0.609435in}}%
\pgfpathcurveto{\pgfqpoint{5.535064in}{0.617249in}}{\pgfqpoint{5.539454in}{0.627848in}}{\pgfqpoint{5.539454in}{0.638898in}}%
\pgfpathcurveto{\pgfqpoint{5.539454in}{0.649948in}}{\pgfqpoint{5.535064in}{0.660547in}}{\pgfqpoint{5.527250in}{0.668361in}}%
\pgfpathcurveto{\pgfqpoint{5.519436in}{0.676175in}}{\pgfqpoint{5.508837in}{0.680565in}}{\pgfqpoint{5.497787in}{0.680565in}}%
\pgfpathcurveto{\pgfqpoint{5.486737in}{0.680565in}}{\pgfqpoint{5.476138in}{0.676175in}}{\pgfqpoint{5.468324in}{0.668361in}}%
\pgfpathcurveto{\pgfqpoint{5.460511in}{0.660547in}}{\pgfqpoint{5.456120in}{0.649948in}}{\pgfqpoint{5.456120in}{0.638898in}}%
\pgfpathcurveto{\pgfqpoint{5.456120in}{0.627848in}}{\pgfqpoint{5.460511in}{0.617249in}}{\pgfqpoint{5.468324in}{0.609435in}}%
\pgfpathcurveto{\pgfqpoint{5.476138in}{0.601622in}}{\pgfqpoint{5.486737in}{0.597232in}}{\pgfqpoint{5.497787in}{0.597232in}}%
\pgfpathlineto{\pgfqpoint{5.497787in}{0.597232in}}%
\pgfpathclose%
\pgfusepath{stroke}%
\end{pgfscope}%
\begin{pgfscope}%
\pgfpathrectangle{\pgfqpoint{0.847223in}{0.554012in}}{\pgfqpoint{6.200000in}{4.620000in}}%
\pgfusepath{clip}%
\pgfsetbuttcap%
\pgfsetroundjoin%
\pgfsetlinewidth{1.003750pt}%
\definecolor{currentstroke}{rgb}{1.000000,0.000000,0.000000}%
\pgfsetstrokecolor{currentstroke}%
\pgfsetdash{}{0pt}%
\pgfpathmoveto{\pgfqpoint{5.503120in}{0.596706in}}%
\pgfpathcurveto{\pgfqpoint{5.514170in}{0.596706in}}{\pgfqpoint{5.524769in}{0.601096in}}{\pgfqpoint{5.532583in}{0.608910in}}%
\pgfpathcurveto{\pgfqpoint{5.540397in}{0.616724in}}{\pgfqpoint{5.544787in}{0.627323in}}{\pgfqpoint{5.544787in}{0.638373in}}%
\pgfpathcurveto{\pgfqpoint{5.544787in}{0.649423in}}{\pgfqpoint{5.540397in}{0.660022in}}{\pgfqpoint{5.532583in}{0.667836in}}%
\pgfpathcurveto{\pgfqpoint{5.524769in}{0.675649in}}{\pgfqpoint{5.514170in}{0.680039in}}{\pgfqpoint{5.503120in}{0.680039in}}%
\pgfpathcurveto{\pgfqpoint{5.492070in}{0.680039in}}{\pgfqpoint{5.481471in}{0.675649in}}{\pgfqpoint{5.473658in}{0.667836in}}%
\pgfpathcurveto{\pgfqpoint{5.465844in}{0.660022in}}{\pgfqpoint{5.461454in}{0.649423in}}{\pgfqpoint{5.461454in}{0.638373in}}%
\pgfpathcurveto{\pgfqpoint{5.461454in}{0.627323in}}{\pgfqpoint{5.465844in}{0.616724in}}{\pgfqpoint{5.473658in}{0.608910in}}%
\pgfpathcurveto{\pgfqpoint{5.481471in}{0.601096in}}{\pgfqpoint{5.492070in}{0.596706in}}{\pgfqpoint{5.503120in}{0.596706in}}%
\pgfpathlineto{\pgfqpoint{5.503120in}{0.596706in}}%
\pgfpathclose%
\pgfusepath{stroke}%
\end{pgfscope}%
\begin{pgfscope}%
\pgfpathrectangle{\pgfqpoint{0.847223in}{0.554012in}}{\pgfqpoint{6.200000in}{4.620000in}}%
\pgfusepath{clip}%
\pgfsetbuttcap%
\pgfsetroundjoin%
\pgfsetlinewidth{1.003750pt}%
\definecolor{currentstroke}{rgb}{1.000000,0.000000,0.000000}%
\pgfsetstrokecolor{currentstroke}%
\pgfsetdash{}{0pt}%
\pgfpathmoveto{\pgfqpoint{5.508454in}{0.596182in}}%
\pgfpathcurveto{\pgfqpoint{5.519504in}{0.596182in}}{\pgfqpoint{5.530103in}{0.600572in}}{\pgfqpoint{5.537916in}{0.608385in}}%
\pgfpathcurveto{\pgfqpoint{5.545730in}{0.616199in}}{\pgfqpoint{5.550120in}{0.626798in}}{\pgfqpoint{5.550120in}{0.637848in}}%
\pgfpathcurveto{\pgfqpoint{5.550120in}{0.648898in}}{\pgfqpoint{5.545730in}{0.659497in}}{\pgfqpoint{5.537916in}{0.667311in}}%
\pgfpathcurveto{\pgfqpoint{5.530103in}{0.675125in}}{\pgfqpoint{5.519504in}{0.679515in}}{\pgfqpoint{5.508454in}{0.679515in}}%
\pgfpathcurveto{\pgfqpoint{5.497403in}{0.679515in}}{\pgfqpoint{5.486804in}{0.675125in}}{\pgfqpoint{5.478991in}{0.667311in}}%
\pgfpathcurveto{\pgfqpoint{5.471177in}{0.659497in}}{\pgfqpoint{5.466787in}{0.648898in}}{\pgfqpoint{5.466787in}{0.637848in}}%
\pgfpathcurveto{\pgfqpoint{5.466787in}{0.626798in}}{\pgfqpoint{5.471177in}{0.616199in}}{\pgfqpoint{5.478991in}{0.608385in}}%
\pgfpathcurveto{\pgfqpoint{5.486804in}{0.600572in}}{\pgfqpoint{5.497403in}{0.596182in}}{\pgfqpoint{5.508454in}{0.596182in}}%
\pgfpathlineto{\pgfqpoint{5.508454in}{0.596182in}}%
\pgfpathclose%
\pgfusepath{stroke}%
\end{pgfscope}%
\begin{pgfscope}%
\pgfpathrectangle{\pgfqpoint{0.847223in}{0.554012in}}{\pgfqpoint{6.200000in}{4.620000in}}%
\pgfusepath{clip}%
\pgfsetbuttcap%
\pgfsetroundjoin%
\pgfsetlinewidth{1.003750pt}%
\definecolor{currentstroke}{rgb}{1.000000,0.000000,0.000000}%
\pgfsetstrokecolor{currentstroke}%
\pgfsetdash{}{0pt}%
\pgfpathmoveto{\pgfqpoint{5.513787in}{0.595658in}}%
\pgfpathcurveto{\pgfqpoint{5.524837in}{0.595658in}}{\pgfqpoint{5.535436in}{0.600048in}}{\pgfqpoint{5.543250in}{0.607862in}}%
\pgfpathcurveto{\pgfqpoint{5.551063in}{0.615676in}}{\pgfqpoint{5.555453in}{0.626275in}}{\pgfqpoint{5.555453in}{0.637325in}}%
\pgfpathcurveto{\pgfqpoint{5.555453in}{0.648375in}}{\pgfqpoint{5.551063in}{0.658974in}}{\pgfqpoint{5.543250in}{0.666788in}}%
\pgfpathcurveto{\pgfqpoint{5.535436in}{0.674601in}}{\pgfqpoint{5.524837in}{0.678992in}}{\pgfqpoint{5.513787in}{0.678992in}}%
\pgfpathcurveto{\pgfqpoint{5.502737in}{0.678992in}}{\pgfqpoint{5.492138in}{0.674601in}}{\pgfqpoint{5.484324in}{0.666788in}}%
\pgfpathcurveto{\pgfqpoint{5.476510in}{0.658974in}}{\pgfqpoint{5.472120in}{0.648375in}}{\pgfqpoint{5.472120in}{0.637325in}}%
\pgfpathcurveto{\pgfqpoint{5.472120in}{0.626275in}}{\pgfqpoint{5.476510in}{0.615676in}}{\pgfqpoint{5.484324in}{0.607862in}}%
\pgfpathcurveto{\pgfqpoint{5.492138in}{0.600048in}}{\pgfqpoint{5.502737in}{0.595658in}}{\pgfqpoint{5.513787in}{0.595658in}}%
\pgfpathlineto{\pgfqpoint{5.513787in}{0.595658in}}%
\pgfpathclose%
\pgfusepath{stroke}%
\end{pgfscope}%
\begin{pgfscope}%
\pgfpathrectangle{\pgfqpoint{0.847223in}{0.554012in}}{\pgfqpoint{6.200000in}{4.620000in}}%
\pgfusepath{clip}%
\pgfsetbuttcap%
\pgfsetroundjoin%
\pgfsetlinewidth{1.003750pt}%
\definecolor{currentstroke}{rgb}{1.000000,0.000000,0.000000}%
\pgfsetstrokecolor{currentstroke}%
\pgfsetdash{}{0pt}%
\pgfpathmoveto{\pgfqpoint{5.519120in}{0.595136in}}%
\pgfpathcurveto{\pgfqpoint{5.530170in}{0.595136in}}{\pgfqpoint{5.540769in}{0.599526in}}{\pgfqpoint{5.548583in}{0.607340in}}%
\pgfpathcurveto{\pgfqpoint{5.556396in}{0.615153in}}{\pgfqpoint{5.560787in}{0.625752in}}{\pgfqpoint{5.560787in}{0.636803in}}%
\pgfpathcurveto{\pgfqpoint{5.560787in}{0.647853in}}{\pgfqpoint{5.556396in}{0.658452in}}{\pgfqpoint{5.548583in}{0.666265in}}%
\pgfpathcurveto{\pgfqpoint{5.540769in}{0.674079in}}{\pgfqpoint{5.530170in}{0.678469in}}{\pgfqpoint{5.519120in}{0.678469in}}%
\pgfpathcurveto{\pgfqpoint{5.508070in}{0.678469in}}{\pgfqpoint{5.497471in}{0.674079in}}{\pgfqpoint{5.489657in}{0.666265in}}%
\pgfpathcurveto{\pgfqpoint{5.481844in}{0.658452in}}{\pgfqpoint{5.477453in}{0.647853in}}{\pgfqpoint{5.477453in}{0.636803in}}%
\pgfpathcurveto{\pgfqpoint{5.477453in}{0.625752in}}{\pgfqpoint{5.481844in}{0.615153in}}{\pgfqpoint{5.489657in}{0.607340in}}%
\pgfpathcurveto{\pgfqpoint{5.497471in}{0.599526in}}{\pgfqpoint{5.508070in}{0.595136in}}{\pgfqpoint{5.519120in}{0.595136in}}%
\pgfpathlineto{\pgfqpoint{5.519120in}{0.595136in}}%
\pgfpathclose%
\pgfusepath{stroke}%
\end{pgfscope}%
\begin{pgfscope}%
\pgfpathrectangle{\pgfqpoint{0.847223in}{0.554012in}}{\pgfqpoint{6.200000in}{4.620000in}}%
\pgfusepath{clip}%
\pgfsetbuttcap%
\pgfsetroundjoin%
\pgfsetlinewidth{1.003750pt}%
\definecolor{currentstroke}{rgb}{1.000000,0.000000,0.000000}%
\pgfsetstrokecolor{currentstroke}%
\pgfsetdash{}{0pt}%
\pgfpathmoveto{\pgfqpoint{5.524453in}{0.594615in}}%
\pgfpathcurveto{\pgfqpoint{5.535503in}{0.594615in}}{\pgfqpoint{5.546102in}{0.599005in}}{\pgfqpoint{5.553916in}{0.606819in}}%
\pgfpathcurveto{\pgfqpoint{5.561730in}{0.614632in}}{\pgfqpoint{5.566120in}{0.625231in}}{\pgfqpoint{5.566120in}{0.636281in}}%
\pgfpathcurveto{\pgfqpoint{5.566120in}{0.647332in}}{\pgfqpoint{5.561730in}{0.657931in}}{\pgfqpoint{5.553916in}{0.665744in}}%
\pgfpathcurveto{\pgfqpoint{5.546102in}{0.673558in}}{\pgfqpoint{5.535503in}{0.677948in}}{\pgfqpoint{5.524453in}{0.677948in}}%
\pgfpathcurveto{\pgfqpoint{5.513403in}{0.677948in}}{\pgfqpoint{5.502804in}{0.673558in}}{\pgfqpoint{5.494990in}{0.665744in}}%
\pgfpathcurveto{\pgfqpoint{5.487177in}{0.657931in}}{\pgfqpoint{5.482787in}{0.647332in}}{\pgfqpoint{5.482787in}{0.636281in}}%
\pgfpathcurveto{\pgfqpoint{5.482787in}{0.625231in}}{\pgfqpoint{5.487177in}{0.614632in}}{\pgfqpoint{5.494990in}{0.606819in}}%
\pgfpathcurveto{\pgfqpoint{5.502804in}{0.599005in}}{\pgfqpoint{5.513403in}{0.594615in}}{\pgfqpoint{5.524453in}{0.594615in}}%
\pgfpathlineto{\pgfqpoint{5.524453in}{0.594615in}}%
\pgfpathclose%
\pgfusepath{stroke}%
\end{pgfscope}%
\begin{pgfscope}%
\pgfpathrectangle{\pgfqpoint{0.847223in}{0.554012in}}{\pgfqpoint{6.200000in}{4.620000in}}%
\pgfusepath{clip}%
\pgfsetbuttcap%
\pgfsetroundjoin%
\pgfsetlinewidth{1.003750pt}%
\definecolor{currentstroke}{rgb}{1.000000,0.000000,0.000000}%
\pgfsetstrokecolor{currentstroke}%
\pgfsetdash{}{0pt}%
\pgfpathmoveto{\pgfqpoint{5.529786in}{0.594095in}}%
\pgfpathcurveto{\pgfqpoint{5.540837in}{0.594095in}}{\pgfqpoint{5.551436in}{0.598485in}}{\pgfqpoint{5.559249in}{0.606298in}}%
\pgfpathcurveto{\pgfqpoint{5.567063in}{0.614112in}}{\pgfqpoint{5.571453in}{0.624711in}}{\pgfqpoint{5.571453in}{0.635761in}}%
\pgfpathcurveto{\pgfqpoint{5.571453in}{0.646811in}}{\pgfqpoint{5.567063in}{0.657410in}}{\pgfqpoint{5.559249in}{0.665224in}}%
\pgfpathcurveto{\pgfqpoint{5.551436in}{0.673038in}}{\pgfqpoint{5.540837in}{0.677428in}}{\pgfqpoint{5.529786in}{0.677428in}}%
\pgfpathcurveto{\pgfqpoint{5.518736in}{0.677428in}}{\pgfqpoint{5.508137in}{0.673038in}}{\pgfqpoint{5.500324in}{0.665224in}}%
\pgfpathcurveto{\pgfqpoint{5.492510in}{0.657410in}}{\pgfqpoint{5.488120in}{0.646811in}}{\pgfqpoint{5.488120in}{0.635761in}}%
\pgfpathcurveto{\pgfqpoint{5.488120in}{0.624711in}}{\pgfqpoint{5.492510in}{0.614112in}}{\pgfqpoint{5.500324in}{0.606298in}}%
\pgfpathcurveto{\pgfqpoint{5.508137in}{0.598485in}}{\pgfqpoint{5.518736in}{0.594095in}}{\pgfqpoint{5.529786in}{0.594095in}}%
\pgfpathlineto{\pgfqpoint{5.529786in}{0.594095in}}%
\pgfpathclose%
\pgfusepath{stroke}%
\end{pgfscope}%
\begin{pgfscope}%
\pgfpathrectangle{\pgfqpoint{0.847223in}{0.554012in}}{\pgfqpoint{6.200000in}{4.620000in}}%
\pgfusepath{clip}%
\pgfsetbuttcap%
\pgfsetroundjoin%
\pgfsetlinewidth{1.003750pt}%
\definecolor{currentstroke}{rgb}{1.000000,0.000000,0.000000}%
\pgfsetstrokecolor{currentstroke}%
\pgfsetdash{}{0pt}%
\pgfpathmoveto{\pgfqpoint{5.535120in}{0.593575in}}%
\pgfpathcurveto{\pgfqpoint{5.546170in}{0.593575in}}{\pgfqpoint{5.556769in}{0.597966in}}{\pgfqpoint{5.564582in}{0.605779in}}%
\pgfpathcurveto{\pgfqpoint{5.572396in}{0.613593in}}{\pgfqpoint{5.576786in}{0.624192in}}{\pgfqpoint{5.576786in}{0.635242in}}%
\pgfpathcurveto{\pgfqpoint{5.576786in}{0.646292in}}{\pgfqpoint{5.572396in}{0.656891in}}{\pgfqpoint{5.564582in}{0.664705in}}%
\pgfpathcurveto{\pgfqpoint{5.556769in}{0.672519in}}{\pgfqpoint{5.546170in}{0.676909in}}{\pgfqpoint{5.535120in}{0.676909in}}%
\pgfpathcurveto{\pgfqpoint{5.524069in}{0.676909in}}{\pgfqpoint{5.513470in}{0.672519in}}{\pgfqpoint{5.505657in}{0.664705in}}%
\pgfpathcurveto{\pgfqpoint{5.497843in}{0.656891in}}{\pgfqpoint{5.493453in}{0.646292in}}{\pgfqpoint{5.493453in}{0.635242in}}%
\pgfpathcurveto{\pgfqpoint{5.493453in}{0.624192in}}{\pgfqpoint{5.497843in}{0.613593in}}{\pgfqpoint{5.505657in}{0.605779in}}%
\pgfpathcurveto{\pgfqpoint{5.513470in}{0.597966in}}{\pgfqpoint{5.524069in}{0.593575in}}{\pgfqpoint{5.535120in}{0.593575in}}%
\pgfpathlineto{\pgfqpoint{5.535120in}{0.593575in}}%
\pgfpathclose%
\pgfusepath{stroke}%
\end{pgfscope}%
\begin{pgfscope}%
\pgfpathrectangle{\pgfqpoint{0.847223in}{0.554012in}}{\pgfqpoint{6.200000in}{4.620000in}}%
\pgfusepath{clip}%
\pgfsetbuttcap%
\pgfsetroundjoin%
\pgfsetlinewidth{1.003750pt}%
\definecolor{currentstroke}{rgb}{1.000000,0.000000,0.000000}%
\pgfsetstrokecolor{currentstroke}%
\pgfsetdash{}{0pt}%
\pgfpathmoveto{\pgfqpoint{5.540453in}{0.593057in}}%
\pgfpathcurveto{\pgfqpoint{5.551503in}{0.593057in}}{\pgfqpoint{5.562102in}{0.597448in}}{\pgfqpoint{5.569916in}{0.605261in}}%
\pgfpathcurveto{\pgfqpoint{5.577729in}{0.613075in}}{\pgfqpoint{5.582120in}{0.623674in}}{\pgfqpoint{5.582120in}{0.634724in}}%
\pgfpathcurveto{\pgfqpoint{5.582120in}{0.645774in}}{\pgfqpoint{5.577729in}{0.656373in}}{\pgfqpoint{5.569916in}{0.664187in}}%
\pgfpathcurveto{\pgfqpoint{5.562102in}{0.672001in}}{\pgfqpoint{5.551503in}{0.676391in}}{\pgfqpoint{5.540453in}{0.676391in}}%
\pgfpathcurveto{\pgfqpoint{5.529403in}{0.676391in}}{\pgfqpoint{5.518804in}{0.672001in}}{\pgfqpoint{5.510990in}{0.664187in}}%
\pgfpathcurveto{\pgfqpoint{5.503176in}{0.656373in}}{\pgfqpoint{5.498786in}{0.645774in}}{\pgfqpoint{5.498786in}{0.634724in}}%
\pgfpathcurveto{\pgfqpoint{5.498786in}{0.623674in}}{\pgfqpoint{5.503176in}{0.613075in}}{\pgfqpoint{5.510990in}{0.605261in}}%
\pgfpathcurveto{\pgfqpoint{5.518804in}{0.597448in}}{\pgfqpoint{5.529403in}{0.593057in}}{\pgfqpoint{5.540453in}{0.593057in}}%
\pgfpathlineto{\pgfqpoint{5.540453in}{0.593057in}}%
\pgfpathclose%
\pgfusepath{stroke}%
\end{pgfscope}%
\begin{pgfscope}%
\pgfpathrectangle{\pgfqpoint{0.847223in}{0.554012in}}{\pgfqpoint{6.200000in}{4.620000in}}%
\pgfusepath{clip}%
\pgfsetbuttcap%
\pgfsetroundjoin%
\pgfsetlinewidth{1.003750pt}%
\definecolor{currentstroke}{rgb}{1.000000,0.000000,0.000000}%
\pgfsetstrokecolor{currentstroke}%
\pgfsetdash{}{0pt}%
\pgfpathmoveto{\pgfqpoint{5.545786in}{0.592540in}}%
\pgfpathcurveto{\pgfqpoint{5.556836in}{0.592540in}}{\pgfqpoint{5.567435in}{0.596931in}}{\pgfqpoint{5.575249in}{0.604744in}}%
\pgfpathcurveto{\pgfqpoint{5.583062in}{0.612558in}}{\pgfqpoint{5.587453in}{0.623157in}}{\pgfqpoint{5.587453in}{0.634207in}}%
\pgfpathcurveto{\pgfqpoint{5.587453in}{0.645257in}}{\pgfqpoint{5.583062in}{0.655856in}}{\pgfqpoint{5.575249in}{0.663670in}}%
\pgfpathcurveto{\pgfqpoint{5.567435in}{0.671484in}}{\pgfqpoint{5.556836in}{0.675874in}}{\pgfqpoint{5.545786in}{0.675874in}}%
\pgfpathcurveto{\pgfqpoint{5.534736in}{0.675874in}}{\pgfqpoint{5.524137in}{0.671484in}}{\pgfqpoint{5.516323in}{0.663670in}}%
\pgfpathcurveto{\pgfqpoint{5.508510in}{0.655856in}}{\pgfqpoint{5.504119in}{0.645257in}}{\pgfqpoint{5.504119in}{0.634207in}}%
\pgfpathcurveto{\pgfqpoint{5.504119in}{0.623157in}}{\pgfqpoint{5.508510in}{0.612558in}}{\pgfqpoint{5.516323in}{0.604744in}}%
\pgfpathcurveto{\pgfqpoint{5.524137in}{0.596931in}}{\pgfqpoint{5.534736in}{0.592540in}}{\pgfqpoint{5.545786in}{0.592540in}}%
\pgfpathlineto{\pgfqpoint{5.545786in}{0.592540in}}%
\pgfpathclose%
\pgfusepath{stroke}%
\end{pgfscope}%
\begin{pgfscope}%
\pgfpathrectangle{\pgfqpoint{0.847223in}{0.554012in}}{\pgfqpoint{6.200000in}{4.620000in}}%
\pgfusepath{clip}%
\pgfsetbuttcap%
\pgfsetroundjoin%
\pgfsetlinewidth{1.003750pt}%
\definecolor{currentstroke}{rgb}{1.000000,0.000000,0.000000}%
\pgfsetstrokecolor{currentstroke}%
\pgfsetdash{}{0pt}%
\pgfpathmoveto{\pgfqpoint{5.551119in}{0.592025in}}%
\pgfpathcurveto{\pgfqpoint{5.562169in}{0.592025in}}{\pgfqpoint{5.572768in}{0.596415in}}{\pgfqpoint{5.580582in}{0.604228in}}%
\pgfpathcurveto{\pgfqpoint{5.588396in}{0.612042in}}{\pgfqpoint{5.592786in}{0.622641in}}{\pgfqpoint{5.592786in}{0.633691in}}%
\pgfpathcurveto{\pgfqpoint{5.592786in}{0.644741in}}{\pgfqpoint{5.588396in}{0.655340in}}{\pgfqpoint{5.580582in}{0.663154in}}%
\pgfpathcurveto{\pgfqpoint{5.572768in}{0.670968in}}{\pgfqpoint{5.562169in}{0.675358in}}{\pgfqpoint{5.551119in}{0.675358in}}%
\pgfpathcurveto{\pgfqpoint{5.540069in}{0.675358in}}{\pgfqpoint{5.529470in}{0.670968in}}{\pgfqpoint{5.521656in}{0.663154in}}%
\pgfpathcurveto{\pgfqpoint{5.513843in}{0.655340in}}{\pgfqpoint{5.509453in}{0.644741in}}{\pgfqpoint{5.509453in}{0.633691in}}%
\pgfpathcurveto{\pgfqpoint{5.509453in}{0.622641in}}{\pgfqpoint{5.513843in}{0.612042in}}{\pgfqpoint{5.521656in}{0.604228in}}%
\pgfpathcurveto{\pgfqpoint{5.529470in}{0.596415in}}{\pgfqpoint{5.540069in}{0.592025in}}{\pgfqpoint{5.551119in}{0.592025in}}%
\pgfpathlineto{\pgfqpoint{5.551119in}{0.592025in}}%
\pgfpathclose%
\pgfusepath{stroke}%
\end{pgfscope}%
\begin{pgfscope}%
\pgfpathrectangle{\pgfqpoint{0.847223in}{0.554012in}}{\pgfqpoint{6.200000in}{4.620000in}}%
\pgfusepath{clip}%
\pgfsetbuttcap%
\pgfsetroundjoin%
\pgfsetlinewidth{1.003750pt}%
\definecolor{currentstroke}{rgb}{1.000000,0.000000,0.000000}%
\pgfsetstrokecolor{currentstroke}%
\pgfsetdash{}{0pt}%
\pgfpathmoveto{\pgfqpoint{5.556452in}{0.591510in}}%
\pgfpathcurveto{\pgfqpoint{5.567503in}{0.591510in}}{\pgfqpoint{5.578102in}{0.595900in}}{\pgfqpoint{5.585915in}{0.603714in}}%
\pgfpathcurveto{\pgfqpoint{5.593729in}{0.611527in}}{\pgfqpoint{5.598119in}{0.622126in}}{\pgfqpoint{5.598119in}{0.633176in}}%
\pgfpathcurveto{\pgfqpoint{5.598119in}{0.644226in}}{\pgfqpoint{5.593729in}{0.654826in}}{\pgfqpoint{5.585915in}{0.662639in}}%
\pgfpathcurveto{\pgfqpoint{5.578102in}{0.670453in}}{\pgfqpoint{5.567503in}{0.674843in}}{\pgfqpoint{5.556452in}{0.674843in}}%
\pgfpathcurveto{\pgfqpoint{5.545402in}{0.674843in}}{\pgfqpoint{5.534803in}{0.670453in}}{\pgfqpoint{5.526990in}{0.662639in}}%
\pgfpathcurveto{\pgfqpoint{5.519176in}{0.654826in}}{\pgfqpoint{5.514786in}{0.644226in}}{\pgfqpoint{5.514786in}{0.633176in}}%
\pgfpathcurveto{\pgfqpoint{5.514786in}{0.622126in}}{\pgfqpoint{5.519176in}{0.611527in}}{\pgfqpoint{5.526990in}{0.603714in}}%
\pgfpathcurveto{\pgfqpoint{5.534803in}{0.595900in}}{\pgfqpoint{5.545402in}{0.591510in}}{\pgfqpoint{5.556452in}{0.591510in}}%
\pgfpathlineto{\pgfqpoint{5.556452in}{0.591510in}}%
\pgfpathclose%
\pgfusepath{stroke}%
\end{pgfscope}%
\begin{pgfscope}%
\pgfpathrectangle{\pgfqpoint{0.847223in}{0.554012in}}{\pgfqpoint{6.200000in}{4.620000in}}%
\pgfusepath{clip}%
\pgfsetbuttcap%
\pgfsetroundjoin%
\pgfsetlinewidth{1.003750pt}%
\definecolor{currentstroke}{rgb}{1.000000,0.000000,0.000000}%
\pgfsetstrokecolor{currentstroke}%
\pgfsetdash{}{0pt}%
\pgfpathmoveto{\pgfqpoint{5.561786in}{0.590996in}}%
\pgfpathcurveto{\pgfqpoint{5.572836in}{0.590996in}}{\pgfqpoint{5.583435in}{0.595386in}}{\pgfqpoint{5.591248in}{0.603200in}}%
\pgfpathcurveto{\pgfqpoint{5.599062in}{0.611013in}}{\pgfqpoint{5.603452in}{0.621612in}}{\pgfqpoint{5.603452in}{0.632663in}}%
\pgfpathcurveto{\pgfqpoint{5.603452in}{0.643713in}}{\pgfqpoint{5.599062in}{0.654312in}}{\pgfqpoint{5.591248in}{0.662125in}}%
\pgfpathcurveto{\pgfqpoint{5.583435in}{0.669939in}}{\pgfqpoint{5.572836in}{0.674329in}}{\pgfqpoint{5.561786in}{0.674329in}}%
\pgfpathcurveto{\pgfqpoint{5.550736in}{0.674329in}}{\pgfqpoint{5.540137in}{0.669939in}}{\pgfqpoint{5.532323in}{0.662125in}}%
\pgfpathcurveto{\pgfqpoint{5.524509in}{0.654312in}}{\pgfqpoint{5.520119in}{0.643713in}}{\pgfqpoint{5.520119in}{0.632663in}}%
\pgfpathcurveto{\pgfqpoint{5.520119in}{0.621612in}}{\pgfqpoint{5.524509in}{0.611013in}}{\pgfqpoint{5.532323in}{0.603200in}}%
\pgfpathcurveto{\pgfqpoint{5.540137in}{0.595386in}}{\pgfqpoint{5.550736in}{0.590996in}}{\pgfqpoint{5.561786in}{0.590996in}}%
\pgfpathlineto{\pgfqpoint{5.561786in}{0.590996in}}%
\pgfpathclose%
\pgfusepath{stroke}%
\end{pgfscope}%
\begin{pgfscope}%
\pgfpathrectangle{\pgfqpoint{0.847223in}{0.554012in}}{\pgfqpoint{6.200000in}{4.620000in}}%
\pgfusepath{clip}%
\pgfsetbuttcap%
\pgfsetroundjoin%
\pgfsetlinewidth{1.003750pt}%
\definecolor{currentstroke}{rgb}{1.000000,0.000000,0.000000}%
\pgfsetstrokecolor{currentstroke}%
\pgfsetdash{}{0pt}%
\pgfpathmoveto{\pgfqpoint{5.567119in}{0.590483in}}%
\pgfpathcurveto{\pgfqpoint{5.578169in}{0.590483in}}{\pgfqpoint{5.588768in}{0.594873in}}{\pgfqpoint{5.596582in}{0.602687in}}%
\pgfpathcurveto{\pgfqpoint{5.604395in}{0.610501in}}{\pgfqpoint{5.608786in}{0.621100in}}{\pgfqpoint{5.608786in}{0.632150in}}%
\pgfpathcurveto{\pgfqpoint{5.608786in}{0.643200in}}{\pgfqpoint{5.604395in}{0.653799in}}{\pgfqpoint{5.596582in}{0.661613in}}%
\pgfpathcurveto{\pgfqpoint{5.588768in}{0.669426in}}{\pgfqpoint{5.578169in}{0.673816in}}{\pgfqpoint{5.567119in}{0.673816in}}%
\pgfpathcurveto{\pgfqpoint{5.556069in}{0.673816in}}{\pgfqpoint{5.545470in}{0.669426in}}{\pgfqpoint{5.537656in}{0.661613in}}%
\pgfpathcurveto{\pgfqpoint{5.529843in}{0.653799in}}{\pgfqpoint{5.525452in}{0.643200in}}{\pgfqpoint{5.525452in}{0.632150in}}%
\pgfpathcurveto{\pgfqpoint{5.525452in}{0.621100in}}{\pgfqpoint{5.529843in}{0.610501in}}{\pgfqpoint{5.537656in}{0.602687in}}%
\pgfpathcurveto{\pgfqpoint{5.545470in}{0.594873in}}{\pgfqpoint{5.556069in}{0.590483in}}{\pgfqpoint{5.567119in}{0.590483in}}%
\pgfpathlineto{\pgfqpoint{5.567119in}{0.590483in}}%
\pgfpathclose%
\pgfusepath{stroke}%
\end{pgfscope}%
\begin{pgfscope}%
\pgfpathrectangle{\pgfqpoint{0.847223in}{0.554012in}}{\pgfqpoint{6.200000in}{4.620000in}}%
\pgfusepath{clip}%
\pgfsetbuttcap%
\pgfsetroundjoin%
\pgfsetlinewidth{1.003750pt}%
\definecolor{currentstroke}{rgb}{1.000000,0.000000,0.000000}%
\pgfsetstrokecolor{currentstroke}%
\pgfsetdash{}{0pt}%
\pgfpathmoveto{\pgfqpoint{5.572452in}{0.589971in}}%
\pgfpathcurveto{\pgfqpoint{5.583502in}{0.589971in}}{\pgfqpoint{5.594101in}{0.594362in}}{\pgfqpoint{5.601915in}{0.602175in}}%
\pgfpathcurveto{\pgfqpoint{5.609729in}{0.609989in}}{\pgfqpoint{5.614119in}{0.620588in}}{\pgfqpoint{5.614119in}{0.631638in}}%
\pgfpathcurveto{\pgfqpoint{5.614119in}{0.642688in}}{\pgfqpoint{5.609729in}{0.653287in}}{\pgfqpoint{5.601915in}{0.661101in}}%
\pgfpathcurveto{\pgfqpoint{5.594101in}{0.668914in}}{\pgfqpoint{5.583502in}{0.673305in}}{\pgfqpoint{5.572452in}{0.673305in}}%
\pgfpathcurveto{\pgfqpoint{5.561402in}{0.673305in}}{\pgfqpoint{5.550803in}{0.668914in}}{\pgfqpoint{5.542989in}{0.661101in}}%
\pgfpathcurveto{\pgfqpoint{5.535176in}{0.653287in}}{\pgfqpoint{5.530785in}{0.642688in}}{\pgfqpoint{5.530785in}{0.631638in}}%
\pgfpathcurveto{\pgfqpoint{5.530785in}{0.620588in}}{\pgfqpoint{5.535176in}{0.609989in}}{\pgfqpoint{5.542989in}{0.602175in}}%
\pgfpathcurveto{\pgfqpoint{5.550803in}{0.594362in}}{\pgfqpoint{5.561402in}{0.589971in}}{\pgfqpoint{5.572452in}{0.589971in}}%
\pgfpathlineto{\pgfqpoint{5.572452in}{0.589971in}}%
\pgfpathclose%
\pgfusepath{stroke}%
\end{pgfscope}%
\begin{pgfscope}%
\pgfpathrectangle{\pgfqpoint{0.847223in}{0.554012in}}{\pgfqpoint{6.200000in}{4.620000in}}%
\pgfusepath{clip}%
\pgfsetbuttcap%
\pgfsetroundjoin%
\pgfsetlinewidth{1.003750pt}%
\definecolor{currentstroke}{rgb}{1.000000,0.000000,0.000000}%
\pgfsetstrokecolor{currentstroke}%
\pgfsetdash{}{0pt}%
\pgfpathmoveto{\pgfqpoint{5.577785in}{0.589461in}}%
\pgfpathcurveto{\pgfqpoint{5.588835in}{0.589461in}}{\pgfqpoint{5.599435in}{0.593851in}}{\pgfqpoint{5.607248in}{0.601665in}}%
\pgfpathcurveto{\pgfqpoint{5.615062in}{0.609478in}}{\pgfqpoint{5.619452in}{0.620077in}}{\pgfqpoint{5.619452in}{0.631127in}}%
\pgfpathcurveto{\pgfqpoint{5.619452in}{0.642177in}}{\pgfqpoint{5.615062in}{0.652776in}}{\pgfqpoint{5.607248in}{0.660590in}}%
\pgfpathcurveto{\pgfqpoint{5.599435in}{0.668404in}}{\pgfqpoint{5.588835in}{0.672794in}}{\pgfqpoint{5.577785in}{0.672794in}}%
\pgfpathcurveto{\pgfqpoint{5.566735in}{0.672794in}}{\pgfqpoint{5.556136in}{0.668404in}}{\pgfqpoint{5.548323in}{0.660590in}}%
\pgfpathcurveto{\pgfqpoint{5.540509in}{0.652776in}}{\pgfqpoint{5.536119in}{0.642177in}}{\pgfqpoint{5.536119in}{0.631127in}}%
\pgfpathcurveto{\pgfqpoint{5.536119in}{0.620077in}}{\pgfqpoint{5.540509in}{0.609478in}}{\pgfqpoint{5.548323in}{0.601665in}}%
\pgfpathcurveto{\pgfqpoint{5.556136in}{0.593851in}}{\pgfqpoint{5.566735in}{0.589461in}}{\pgfqpoint{5.577785in}{0.589461in}}%
\pgfpathlineto{\pgfqpoint{5.577785in}{0.589461in}}%
\pgfpathclose%
\pgfusepath{stroke}%
\end{pgfscope}%
\begin{pgfscope}%
\pgfpathrectangle{\pgfqpoint{0.847223in}{0.554012in}}{\pgfqpoint{6.200000in}{4.620000in}}%
\pgfusepath{clip}%
\pgfsetbuttcap%
\pgfsetroundjoin%
\pgfsetlinewidth{1.003750pt}%
\definecolor{currentstroke}{rgb}{1.000000,0.000000,0.000000}%
\pgfsetstrokecolor{currentstroke}%
\pgfsetdash{}{0pt}%
\pgfpathmoveto{\pgfqpoint{5.583119in}{0.588951in}}%
\pgfpathcurveto{\pgfqpoint{5.594169in}{0.588951in}}{\pgfqpoint{5.604768in}{0.593341in}}{\pgfqpoint{5.612581in}{0.601155in}}%
\pgfpathcurveto{\pgfqpoint{5.620395in}{0.608969in}}{\pgfqpoint{5.624785in}{0.619568in}}{\pgfqpoint{5.624785in}{0.630618in}}%
\pgfpathcurveto{\pgfqpoint{5.624785in}{0.641668in}}{\pgfqpoint{5.620395in}{0.652267in}}{\pgfqpoint{5.612581in}{0.660080in}}%
\pgfpathcurveto{\pgfqpoint{5.604768in}{0.667894in}}{\pgfqpoint{5.594169in}{0.672284in}}{\pgfqpoint{5.583119in}{0.672284in}}%
\pgfpathcurveto{\pgfqpoint{5.572068in}{0.672284in}}{\pgfqpoint{5.561469in}{0.667894in}}{\pgfqpoint{5.553656in}{0.660080in}}%
\pgfpathcurveto{\pgfqpoint{5.545842in}{0.652267in}}{\pgfqpoint{5.541452in}{0.641668in}}{\pgfqpoint{5.541452in}{0.630618in}}%
\pgfpathcurveto{\pgfqpoint{5.541452in}{0.619568in}}{\pgfqpoint{5.545842in}{0.608969in}}{\pgfqpoint{5.553656in}{0.601155in}}%
\pgfpathcurveto{\pgfqpoint{5.561469in}{0.593341in}}{\pgfqpoint{5.572068in}{0.588951in}}{\pgfqpoint{5.583119in}{0.588951in}}%
\pgfpathlineto{\pgfqpoint{5.583119in}{0.588951in}}%
\pgfpathclose%
\pgfusepath{stroke}%
\end{pgfscope}%
\begin{pgfscope}%
\pgfpathrectangle{\pgfqpoint{0.847223in}{0.554012in}}{\pgfqpoint{6.200000in}{4.620000in}}%
\pgfusepath{clip}%
\pgfsetbuttcap%
\pgfsetroundjoin%
\pgfsetlinewidth{1.003750pt}%
\definecolor{currentstroke}{rgb}{1.000000,0.000000,0.000000}%
\pgfsetstrokecolor{currentstroke}%
\pgfsetdash{}{0pt}%
\pgfpathmoveto{\pgfqpoint{5.588452in}{0.588442in}}%
\pgfpathcurveto{\pgfqpoint{5.599502in}{0.588442in}}{\pgfqpoint{5.610101in}{0.592833in}}{\pgfqpoint{5.617915in}{0.600646in}}%
\pgfpathcurveto{\pgfqpoint{5.625728in}{0.608460in}}{\pgfqpoint{5.630118in}{0.619059in}}{\pgfqpoint{5.630118in}{0.630109in}}%
\pgfpathcurveto{\pgfqpoint{5.630118in}{0.641159in}}{\pgfqpoint{5.625728in}{0.651758in}}{\pgfqpoint{5.617915in}{0.659572in}}%
\pgfpathcurveto{\pgfqpoint{5.610101in}{0.667385in}}{\pgfqpoint{5.599502in}{0.671776in}}{\pgfqpoint{5.588452in}{0.671776in}}%
\pgfpathcurveto{\pgfqpoint{5.577402in}{0.671776in}}{\pgfqpoint{5.566803in}{0.667385in}}{\pgfqpoint{5.558989in}{0.659572in}}%
\pgfpathcurveto{\pgfqpoint{5.551175in}{0.651758in}}{\pgfqpoint{5.546785in}{0.641159in}}{\pgfqpoint{5.546785in}{0.630109in}}%
\pgfpathcurveto{\pgfqpoint{5.546785in}{0.619059in}}{\pgfqpoint{5.551175in}{0.608460in}}{\pgfqpoint{5.558989in}{0.600646in}}%
\pgfpathcurveto{\pgfqpoint{5.566803in}{0.592833in}}{\pgfqpoint{5.577402in}{0.588442in}}{\pgfqpoint{5.588452in}{0.588442in}}%
\pgfpathlineto{\pgfqpoint{5.588452in}{0.588442in}}%
\pgfpathclose%
\pgfusepath{stroke}%
\end{pgfscope}%
\begin{pgfscope}%
\pgfpathrectangle{\pgfqpoint{0.847223in}{0.554012in}}{\pgfqpoint{6.200000in}{4.620000in}}%
\pgfusepath{clip}%
\pgfsetbuttcap%
\pgfsetroundjoin%
\pgfsetlinewidth{1.003750pt}%
\definecolor{currentstroke}{rgb}{1.000000,0.000000,0.000000}%
\pgfsetstrokecolor{currentstroke}%
\pgfsetdash{}{0pt}%
\pgfpathmoveto{\pgfqpoint{5.593785in}{0.587935in}}%
\pgfpathcurveto{\pgfqpoint{5.604835in}{0.587935in}}{\pgfqpoint{5.615434in}{0.592325in}}{\pgfqpoint{5.623248in}{0.600139in}}%
\pgfpathcurveto{\pgfqpoint{5.631061in}{0.607952in}}{\pgfqpoint{5.635452in}{0.618551in}}{\pgfqpoint{5.635452in}{0.629601in}}%
\pgfpathcurveto{\pgfqpoint{5.635452in}{0.640652in}}{\pgfqpoint{5.631061in}{0.651251in}}{\pgfqpoint{5.623248in}{0.659064in}}%
\pgfpathcurveto{\pgfqpoint{5.615434in}{0.666878in}}{\pgfqpoint{5.604835in}{0.671268in}}{\pgfqpoint{5.593785in}{0.671268in}}%
\pgfpathcurveto{\pgfqpoint{5.582735in}{0.671268in}}{\pgfqpoint{5.572136in}{0.666878in}}{\pgfqpoint{5.564322in}{0.659064in}}%
\pgfpathcurveto{\pgfqpoint{5.556509in}{0.651251in}}{\pgfqpoint{5.552118in}{0.640652in}}{\pgfqpoint{5.552118in}{0.629601in}}%
\pgfpathcurveto{\pgfqpoint{5.552118in}{0.618551in}}{\pgfqpoint{5.556509in}{0.607952in}}{\pgfqpoint{5.564322in}{0.600139in}}%
\pgfpathcurveto{\pgfqpoint{5.572136in}{0.592325in}}{\pgfqpoint{5.582735in}{0.587935in}}{\pgfqpoint{5.593785in}{0.587935in}}%
\pgfpathlineto{\pgfqpoint{5.593785in}{0.587935in}}%
\pgfpathclose%
\pgfusepath{stroke}%
\end{pgfscope}%
\begin{pgfscope}%
\pgfpathrectangle{\pgfqpoint{0.847223in}{0.554012in}}{\pgfqpoint{6.200000in}{4.620000in}}%
\pgfusepath{clip}%
\pgfsetbuttcap%
\pgfsetroundjoin%
\pgfsetlinewidth{1.003750pt}%
\definecolor{currentstroke}{rgb}{1.000000,0.000000,0.000000}%
\pgfsetstrokecolor{currentstroke}%
\pgfsetdash{}{0pt}%
\pgfpathmoveto{\pgfqpoint{5.599118in}{0.587428in}}%
\pgfpathcurveto{\pgfqpoint{5.610168in}{0.587428in}}{\pgfqpoint{5.620767in}{0.591818in}}{\pgfqpoint{5.628581in}{0.599632in}}%
\pgfpathcurveto{\pgfqpoint{5.636395in}{0.607446in}}{\pgfqpoint{5.640785in}{0.618045in}}{\pgfqpoint{5.640785in}{0.629095in}}%
\pgfpathcurveto{\pgfqpoint{5.640785in}{0.640145in}}{\pgfqpoint{5.636395in}{0.650744in}}{\pgfqpoint{5.628581in}{0.658558in}}%
\pgfpathcurveto{\pgfqpoint{5.620767in}{0.666371in}}{\pgfqpoint{5.610168in}{0.670762in}}{\pgfqpoint{5.599118in}{0.670762in}}%
\pgfpathcurveto{\pgfqpoint{5.588068in}{0.670762in}}{\pgfqpoint{5.577469in}{0.666371in}}{\pgfqpoint{5.569655in}{0.658558in}}%
\pgfpathcurveto{\pgfqpoint{5.561842in}{0.650744in}}{\pgfqpoint{5.557452in}{0.640145in}}{\pgfqpoint{5.557452in}{0.629095in}}%
\pgfpathcurveto{\pgfqpoint{5.557452in}{0.618045in}}{\pgfqpoint{5.561842in}{0.607446in}}{\pgfqpoint{5.569655in}{0.599632in}}%
\pgfpathcurveto{\pgfqpoint{5.577469in}{0.591818in}}{\pgfqpoint{5.588068in}{0.587428in}}{\pgfqpoint{5.599118in}{0.587428in}}%
\pgfpathlineto{\pgfqpoint{5.599118in}{0.587428in}}%
\pgfpathclose%
\pgfusepath{stroke}%
\end{pgfscope}%
\begin{pgfscope}%
\pgfpathrectangle{\pgfqpoint{0.847223in}{0.554012in}}{\pgfqpoint{6.200000in}{4.620000in}}%
\pgfusepath{clip}%
\pgfsetbuttcap%
\pgfsetroundjoin%
\pgfsetlinewidth{1.003750pt}%
\definecolor{currentstroke}{rgb}{1.000000,0.000000,0.000000}%
\pgfsetstrokecolor{currentstroke}%
\pgfsetdash{}{0pt}%
\pgfpathmoveto{\pgfqpoint{5.604451in}{0.586923in}}%
\pgfpathcurveto{\pgfqpoint{5.615502in}{0.586923in}}{\pgfqpoint{5.626101in}{0.591313in}}{\pgfqpoint{5.633914in}{0.599127in}}%
\pgfpathcurveto{\pgfqpoint{5.641728in}{0.606940in}}{\pgfqpoint{5.646118in}{0.617539in}}{\pgfqpoint{5.646118in}{0.628589in}}%
\pgfpathcurveto{\pgfqpoint{5.646118in}{0.639639in}}{\pgfqpoint{5.641728in}{0.650239in}}{\pgfqpoint{5.633914in}{0.658052in}}%
\pgfpathcurveto{\pgfqpoint{5.626101in}{0.665866in}}{\pgfqpoint{5.615502in}{0.670256in}}{\pgfqpoint{5.604451in}{0.670256in}}%
\pgfpathcurveto{\pgfqpoint{5.593401in}{0.670256in}}{\pgfqpoint{5.582802in}{0.665866in}}{\pgfqpoint{5.574989in}{0.658052in}}%
\pgfpathcurveto{\pgfqpoint{5.567175in}{0.650239in}}{\pgfqpoint{5.562785in}{0.639639in}}{\pgfqpoint{5.562785in}{0.628589in}}%
\pgfpathcurveto{\pgfqpoint{5.562785in}{0.617539in}}{\pgfqpoint{5.567175in}{0.606940in}}{\pgfqpoint{5.574989in}{0.599127in}}%
\pgfpathcurveto{\pgfqpoint{5.582802in}{0.591313in}}{\pgfqpoint{5.593401in}{0.586923in}}{\pgfqpoint{5.604451in}{0.586923in}}%
\pgfpathlineto{\pgfqpoint{5.604451in}{0.586923in}}%
\pgfpathclose%
\pgfusepath{stroke}%
\end{pgfscope}%
\begin{pgfscope}%
\pgfpathrectangle{\pgfqpoint{0.847223in}{0.554012in}}{\pgfqpoint{6.200000in}{4.620000in}}%
\pgfusepath{clip}%
\pgfsetbuttcap%
\pgfsetroundjoin%
\pgfsetlinewidth{1.003750pt}%
\definecolor{currentstroke}{rgb}{1.000000,0.000000,0.000000}%
\pgfsetstrokecolor{currentstroke}%
\pgfsetdash{}{0pt}%
\pgfpathmoveto{\pgfqpoint{5.609785in}{0.586418in}}%
\pgfpathcurveto{\pgfqpoint{5.620835in}{0.586418in}}{\pgfqpoint{5.631434in}{0.590808in}}{\pgfqpoint{5.639247in}{0.598622in}}%
\pgfpathcurveto{\pgfqpoint{5.647061in}{0.606436in}}{\pgfqpoint{5.651451in}{0.617035in}}{\pgfqpoint{5.651451in}{0.628085in}}%
\pgfpathcurveto{\pgfqpoint{5.651451in}{0.639135in}}{\pgfqpoint{5.647061in}{0.649734in}}{\pgfqpoint{5.639247in}{0.657548in}}%
\pgfpathcurveto{\pgfqpoint{5.631434in}{0.665361in}}{\pgfqpoint{5.620835in}{0.669751in}}{\pgfqpoint{5.609785in}{0.669751in}}%
\pgfpathcurveto{\pgfqpoint{5.598735in}{0.669751in}}{\pgfqpoint{5.588135in}{0.665361in}}{\pgfqpoint{5.580322in}{0.657548in}}%
\pgfpathcurveto{\pgfqpoint{5.572508in}{0.649734in}}{\pgfqpoint{5.568118in}{0.639135in}}{\pgfqpoint{5.568118in}{0.628085in}}%
\pgfpathcurveto{\pgfqpoint{5.568118in}{0.617035in}}{\pgfqpoint{5.572508in}{0.606436in}}{\pgfqpoint{5.580322in}{0.598622in}}%
\pgfpathcurveto{\pgfqpoint{5.588135in}{0.590808in}}{\pgfqpoint{5.598735in}{0.586418in}}{\pgfqpoint{5.609785in}{0.586418in}}%
\pgfpathlineto{\pgfqpoint{5.609785in}{0.586418in}}%
\pgfpathclose%
\pgfusepath{stroke}%
\end{pgfscope}%
\begin{pgfscope}%
\pgfpathrectangle{\pgfqpoint{0.847223in}{0.554012in}}{\pgfqpoint{6.200000in}{4.620000in}}%
\pgfusepath{clip}%
\pgfsetbuttcap%
\pgfsetroundjoin%
\pgfsetlinewidth{1.003750pt}%
\definecolor{currentstroke}{rgb}{1.000000,0.000000,0.000000}%
\pgfsetstrokecolor{currentstroke}%
\pgfsetdash{}{0pt}%
\pgfpathmoveto{\pgfqpoint{5.615118in}{0.585915in}}%
\pgfpathcurveto{\pgfqpoint{5.626168in}{0.585915in}}{\pgfqpoint{5.636767in}{0.590305in}}{\pgfqpoint{5.644581in}{0.598119in}}%
\pgfpathcurveto{\pgfqpoint{5.652394in}{0.605932in}}{\pgfqpoint{5.656785in}{0.616531in}}{\pgfqpoint{5.656785in}{0.627581in}}%
\pgfpathcurveto{\pgfqpoint{5.656785in}{0.638631in}}{\pgfqpoint{5.652394in}{0.649230in}}{\pgfqpoint{5.644581in}{0.657044in}}%
\pgfpathcurveto{\pgfqpoint{5.636767in}{0.664858in}}{\pgfqpoint{5.626168in}{0.669248in}}{\pgfqpoint{5.615118in}{0.669248in}}%
\pgfpathcurveto{\pgfqpoint{5.604068in}{0.669248in}}{\pgfqpoint{5.593469in}{0.664858in}}{\pgfqpoint{5.585655in}{0.657044in}}%
\pgfpathcurveto{\pgfqpoint{5.577841in}{0.649230in}}{\pgfqpoint{5.573451in}{0.638631in}}{\pgfqpoint{5.573451in}{0.627581in}}%
\pgfpathcurveto{\pgfqpoint{5.573451in}{0.616531in}}{\pgfqpoint{5.577841in}{0.605932in}}{\pgfqpoint{5.585655in}{0.598119in}}%
\pgfpathcurveto{\pgfqpoint{5.593469in}{0.590305in}}{\pgfqpoint{5.604068in}{0.585915in}}{\pgfqpoint{5.615118in}{0.585915in}}%
\pgfpathlineto{\pgfqpoint{5.615118in}{0.585915in}}%
\pgfpathclose%
\pgfusepath{stroke}%
\end{pgfscope}%
\begin{pgfscope}%
\pgfpathrectangle{\pgfqpoint{0.847223in}{0.554012in}}{\pgfqpoint{6.200000in}{4.620000in}}%
\pgfusepath{clip}%
\pgfsetbuttcap%
\pgfsetroundjoin%
\pgfsetlinewidth{1.003750pt}%
\definecolor{currentstroke}{rgb}{1.000000,0.000000,0.000000}%
\pgfsetstrokecolor{currentstroke}%
\pgfsetdash{}{0pt}%
\pgfpathmoveto{\pgfqpoint{5.620451in}{0.585412in}}%
\pgfpathcurveto{\pgfqpoint{5.631501in}{0.585412in}}{\pgfqpoint{5.642100in}{0.589802in}}{\pgfqpoint{5.649914in}{0.597616in}}%
\pgfpathcurveto{\pgfqpoint{5.657727in}{0.605430in}}{\pgfqpoint{5.662118in}{0.616029in}}{\pgfqpoint{5.662118in}{0.627079in}}%
\pgfpathcurveto{\pgfqpoint{5.662118in}{0.638129in}}{\pgfqpoint{5.657727in}{0.648728in}}{\pgfqpoint{5.649914in}{0.656542in}}%
\pgfpathcurveto{\pgfqpoint{5.642100in}{0.664355in}}{\pgfqpoint{5.631501in}{0.668745in}}{\pgfqpoint{5.620451in}{0.668745in}}%
\pgfpathcurveto{\pgfqpoint{5.609401in}{0.668745in}}{\pgfqpoint{5.598802in}{0.664355in}}{\pgfqpoint{5.590988in}{0.656542in}}%
\pgfpathcurveto{\pgfqpoint{5.583175in}{0.648728in}}{\pgfqpoint{5.578784in}{0.638129in}}{\pgfqpoint{5.578784in}{0.627079in}}%
\pgfpathcurveto{\pgfqpoint{5.578784in}{0.616029in}}{\pgfqpoint{5.583175in}{0.605430in}}{\pgfqpoint{5.590988in}{0.597616in}}%
\pgfpathcurveto{\pgfqpoint{5.598802in}{0.589802in}}{\pgfqpoint{5.609401in}{0.585412in}}{\pgfqpoint{5.620451in}{0.585412in}}%
\pgfpathlineto{\pgfqpoint{5.620451in}{0.585412in}}%
\pgfpathclose%
\pgfusepath{stroke}%
\end{pgfscope}%
\begin{pgfscope}%
\pgfpathrectangle{\pgfqpoint{0.847223in}{0.554012in}}{\pgfqpoint{6.200000in}{4.620000in}}%
\pgfusepath{clip}%
\pgfsetbuttcap%
\pgfsetroundjoin%
\pgfsetlinewidth{1.003750pt}%
\definecolor{currentstroke}{rgb}{1.000000,0.000000,0.000000}%
\pgfsetstrokecolor{currentstroke}%
\pgfsetdash{}{0pt}%
\pgfpathmoveto{\pgfqpoint{5.625784in}{0.584911in}}%
\pgfpathcurveto{\pgfqpoint{5.636834in}{0.584911in}}{\pgfqpoint{5.647433in}{0.589301in}}{\pgfqpoint{5.655247in}{0.597115in}}%
\pgfpathcurveto{\pgfqpoint{5.663061in}{0.604928in}}{\pgfqpoint{5.667451in}{0.615527in}}{\pgfqpoint{5.667451in}{0.626577in}}%
\pgfpathcurveto{\pgfqpoint{5.667451in}{0.637627in}}{\pgfqpoint{5.663061in}{0.648226in}}{\pgfqpoint{5.655247in}{0.656040in}}%
\pgfpathcurveto{\pgfqpoint{5.647433in}{0.663854in}}{\pgfqpoint{5.636834in}{0.668244in}}{\pgfqpoint{5.625784in}{0.668244in}}%
\pgfpathcurveto{\pgfqpoint{5.614734in}{0.668244in}}{\pgfqpoint{5.604135in}{0.663854in}}{\pgfqpoint{5.596322in}{0.656040in}}%
\pgfpathcurveto{\pgfqpoint{5.588508in}{0.648226in}}{\pgfqpoint{5.584118in}{0.637627in}}{\pgfqpoint{5.584118in}{0.626577in}}%
\pgfpathcurveto{\pgfqpoint{5.584118in}{0.615527in}}{\pgfqpoint{5.588508in}{0.604928in}}{\pgfqpoint{5.596322in}{0.597115in}}%
\pgfpathcurveto{\pgfqpoint{5.604135in}{0.589301in}}{\pgfqpoint{5.614734in}{0.584911in}}{\pgfqpoint{5.625784in}{0.584911in}}%
\pgfpathlineto{\pgfqpoint{5.625784in}{0.584911in}}%
\pgfpathclose%
\pgfusepath{stroke}%
\end{pgfscope}%
\begin{pgfscope}%
\pgfpathrectangle{\pgfqpoint{0.847223in}{0.554012in}}{\pgfqpoint{6.200000in}{4.620000in}}%
\pgfusepath{clip}%
\pgfsetbuttcap%
\pgfsetroundjoin%
\pgfsetlinewidth{1.003750pt}%
\definecolor{currentstroke}{rgb}{1.000000,0.000000,0.000000}%
\pgfsetstrokecolor{currentstroke}%
\pgfsetdash{}{0pt}%
\pgfpathmoveto{\pgfqpoint{5.631118in}{0.584410in}}%
\pgfpathcurveto{\pgfqpoint{5.642168in}{0.584410in}}{\pgfqpoint{5.652767in}{0.588800in}}{\pgfqpoint{5.660580in}{0.596614in}}%
\pgfpathcurveto{\pgfqpoint{5.668394in}{0.604428in}}{\pgfqpoint{5.672784in}{0.615027in}}{\pgfqpoint{5.672784in}{0.626077in}}%
\pgfpathcurveto{\pgfqpoint{5.672784in}{0.637127in}}{\pgfqpoint{5.668394in}{0.647726in}}{\pgfqpoint{5.660580in}{0.655540in}}%
\pgfpathcurveto{\pgfqpoint{5.652767in}{0.663353in}}{\pgfqpoint{5.642168in}{0.667744in}}{\pgfqpoint{5.631118in}{0.667744in}}%
\pgfpathcurveto{\pgfqpoint{5.620067in}{0.667744in}}{\pgfqpoint{5.609468in}{0.663353in}}{\pgfqpoint{5.601655in}{0.655540in}}%
\pgfpathcurveto{\pgfqpoint{5.593841in}{0.647726in}}{\pgfqpoint{5.589451in}{0.637127in}}{\pgfqpoint{5.589451in}{0.626077in}}%
\pgfpathcurveto{\pgfqpoint{5.589451in}{0.615027in}}{\pgfqpoint{5.593841in}{0.604428in}}{\pgfqpoint{5.601655in}{0.596614in}}%
\pgfpathcurveto{\pgfqpoint{5.609468in}{0.588800in}}{\pgfqpoint{5.620067in}{0.584410in}}{\pgfqpoint{5.631118in}{0.584410in}}%
\pgfpathlineto{\pgfqpoint{5.631118in}{0.584410in}}%
\pgfpathclose%
\pgfusepath{stroke}%
\end{pgfscope}%
\begin{pgfscope}%
\pgfpathrectangle{\pgfqpoint{0.847223in}{0.554012in}}{\pgfqpoint{6.200000in}{4.620000in}}%
\pgfusepath{clip}%
\pgfsetbuttcap%
\pgfsetroundjoin%
\pgfsetlinewidth{1.003750pt}%
\definecolor{currentstroke}{rgb}{1.000000,0.000000,0.000000}%
\pgfsetstrokecolor{currentstroke}%
\pgfsetdash{}{0pt}%
\pgfpathmoveto{\pgfqpoint{5.636451in}{0.583911in}}%
\pgfpathcurveto{\pgfqpoint{5.647501in}{0.583911in}}{\pgfqpoint{5.658100in}{0.588301in}}{\pgfqpoint{5.665914in}{0.596115in}}%
\pgfpathcurveto{\pgfqpoint{5.673727in}{0.603928in}}{\pgfqpoint{5.678117in}{0.614527in}}{\pgfqpoint{5.678117in}{0.625577in}}%
\pgfpathcurveto{\pgfqpoint{5.678117in}{0.636628in}}{\pgfqpoint{5.673727in}{0.647227in}}{\pgfqpoint{5.665914in}{0.655040in}}%
\pgfpathcurveto{\pgfqpoint{5.658100in}{0.662854in}}{\pgfqpoint{5.647501in}{0.667244in}}{\pgfqpoint{5.636451in}{0.667244in}}%
\pgfpathcurveto{\pgfqpoint{5.625401in}{0.667244in}}{\pgfqpoint{5.614802in}{0.662854in}}{\pgfqpoint{5.606988in}{0.655040in}}%
\pgfpathcurveto{\pgfqpoint{5.599174in}{0.647227in}}{\pgfqpoint{5.594784in}{0.636628in}}{\pgfqpoint{5.594784in}{0.625577in}}%
\pgfpathcurveto{\pgfqpoint{5.594784in}{0.614527in}}{\pgfqpoint{5.599174in}{0.603928in}}{\pgfqpoint{5.606988in}{0.596115in}}%
\pgfpathcurveto{\pgfqpoint{5.614802in}{0.588301in}}{\pgfqpoint{5.625401in}{0.583911in}}{\pgfqpoint{5.636451in}{0.583911in}}%
\pgfpathlineto{\pgfqpoint{5.636451in}{0.583911in}}%
\pgfpathclose%
\pgfusepath{stroke}%
\end{pgfscope}%
\begin{pgfscope}%
\pgfpathrectangle{\pgfqpoint{0.847223in}{0.554012in}}{\pgfqpoint{6.200000in}{4.620000in}}%
\pgfusepath{clip}%
\pgfsetbuttcap%
\pgfsetroundjoin%
\pgfsetlinewidth{1.003750pt}%
\definecolor{currentstroke}{rgb}{1.000000,0.000000,0.000000}%
\pgfsetstrokecolor{currentstroke}%
\pgfsetdash{}{0pt}%
\pgfpathmoveto{\pgfqpoint{5.641784in}{0.583412in}}%
\pgfpathcurveto{\pgfqpoint{5.652834in}{0.583412in}}{\pgfqpoint{5.663433in}{0.587803in}}{\pgfqpoint{5.671247in}{0.595616in}}%
\pgfpathcurveto{\pgfqpoint{5.679060in}{0.603430in}}{\pgfqpoint{5.683451in}{0.614029in}}{\pgfqpoint{5.683451in}{0.625079in}}%
\pgfpathcurveto{\pgfqpoint{5.683451in}{0.636129in}}{\pgfqpoint{5.679060in}{0.646728in}}{\pgfqpoint{5.671247in}{0.654542in}}%
\pgfpathcurveto{\pgfqpoint{5.663433in}{0.662355in}}{\pgfqpoint{5.652834in}{0.666746in}}{\pgfqpoint{5.641784in}{0.666746in}}%
\pgfpathcurveto{\pgfqpoint{5.630734in}{0.666746in}}{\pgfqpoint{5.620135in}{0.662355in}}{\pgfqpoint{5.612321in}{0.654542in}}%
\pgfpathcurveto{\pgfqpoint{5.604508in}{0.646728in}}{\pgfqpoint{5.600117in}{0.636129in}}{\pgfqpoint{5.600117in}{0.625079in}}%
\pgfpathcurveto{\pgfqpoint{5.600117in}{0.614029in}}{\pgfqpoint{5.604508in}{0.603430in}}{\pgfqpoint{5.612321in}{0.595616in}}%
\pgfpathcurveto{\pgfqpoint{5.620135in}{0.587803in}}{\pgfqpoint{5.630734in}{0.583412in}}{\pgfqpoint{5.641784in}{0.583412in}}%
\pgfpathlineto{\pgfqpoint{5.641784in}{0.583412in}}%
\pgfpathclose%
\pgfusepath{stroke}%
\end{pgfscope}%
\begin{pgfscope}%
\pgfpathrectangle{\pgfqpoint{0.847223in}{0.554012in}}{\pgfqpoint{6.200000in}{4.620000in}}%
\pgfusepath{clip}%
\pgfsetbuttcap%
\pgfsetroundjoin%
\pgfsetlinewidth{1.003750pt}%
\definecolor{currentstroke}{rgb}{1.000000,0.000000,0.000000}%
\pgfsetstrokecolor{currentstroke}%
\pgfsetdash{}{0pt}%
\pgfpathmoveto{\pgfqpoint{5.647117in}{0.582915in}}%
\pgfpathcurveto{\pgfqpoint{5.658167in}{0.582915in}}{\pgfqpoint{5.668766in}{0.587305in}}{\pgfqpoint{5.676580in}{0.595119in}}%
\pgfpathcurveto{\pgfqpoint{5.684394in}{0.602932in}}{\pgfqpoint{5.688784in}{0.613531in}}{\pgfqpoint{5.688784in}{0.624581in}}%
\pgfpathcurveto{\pgfqpoint{5.688784in}{0.635632in}}{\pgfqpoint{5.684394in}{0.646231in}}{\pgfqpoint{5.676580in}{0.654044in}}%
\pgfpathcurveto{\pgfqpoint{5.668766in}{0.661858in}}{\pgfqpoint{5.658167in}{0.666248in}}{\pgfqpoint{5.647117in}{0.666248in}}%
\pgfpathcurveto{\pgfqpoint{5.636067in}{0.666248in}}{\pgfqpoint{5.625468in}{0.661858in}}{\pgfqpoint{5.617654in}{0.654044in}}%
\pgfpathcurveto{\pgfqpoint{5.609841in}{0.646231in}}{\pgfqpoint{5.605450in}{0.635632in}}{\pgfqpoint{5.605450in}{0.624581in}}%
\pgfpathcurveto{\pgfqpoint{5.605450in}{0.613531in}}{\pgfqpoint{5.609841in}{0.602932in}}{\pgfqpoint{5.617654in}{0.595119in}}%
\pgfpathcurveto{\pgfqpoint{5.625468in}{0.587305in}}{\pgfqpoint{5.636067in}{0.582915in}}{\pgfqpoint{5.647117in}{0.582915in}}%
\pgfpathlineto{\pgfqpoint{5.647117in}{0.582915in}}%
\pgfpathclose%
\pgfusepath{stroke}%
\end{pgfscope}%
\begin{pgfscope}%
\pgfpathrectangle{\pgfqpoint{0.847223in}{0.554012in}}{\pgfqpoint{6.200000in}{4.620000in}}%
\pgfusepath{clip}%
\pgfsetbuttcap%
\pgfsetroundjoin%
\pgfsetlinewidth{1.003750pt}%
\definecolor{currentstroke}{rgb}{1.000000,0.000000,0.000000}%
\pgfsetstrokecolor{currentstroke}%
\pgfsetdash{}{0pt}%
\pgfpathmoveto{\pgfqpoint{5.652450in}{0.582418in}}%
\pgfpathcurveto{\pgfqpoint{5.663500in}{0.582418in}}{\pgfqpoint{5.674100in}{0.586809in}}{\pgfqpoint{5.681913in}{0.594622in}}%
\pgfpathcurveto{\pgfqpoint{5.689727in}{0.602436in}}{\pgfqpoint{5.694117in}{0.613035in}}{\pgfqpoint{5.694117in}{0.624085in}}%
\pgfpathcurveto{\pgfqpoint{5.694117in}{0.635135in}}{\pgfqpoint{5.689727in}{0.645734in}}{\pgfqpoint{5.681913in}{0.653548in}}%
\pgfpathcurveto{\pgfqpoint{5.674100in}{0.661361in}}{\pgfqpoint{5.663500in}{0.665752in}}{\pgfqpoint{5.652450in}{0.665752in}}%
\pgfpathcurveto{\pgfqpoint{5.641400in}{0.665752in}}{\pgfqpoint{5.630801in}{0.661361in}}{\pgfqpoint{5.622988in}{0.653548in}}%
\pgfpathcurveto{\pgfqpoint{5.615174in}{0.645734in}}{\pgfqpoint{5.610784in}{0.635135in}}{\pgfqpoint{5.610784in}{0.624085in}}%
\pgfpathcurveto{\pgfqpoint{5.610784in}{0.613035in}}{\pgfqpoint{5.615174in}{0.602436in}}{\pgfqpoint{5.622988in}{0.594622in}}%
\pgfpathcurveto{\pgfqpoint{5.630801in}{0.586809in}}{\pgfqpoint{5.641400in}{0.582418in}}{\pgfqpoint{5.652450in}{0.582418in}}%
\pgfpathlineto{\pgfqpoint{5.652450in}{0.582418in}}%
\pgfpathclose%
\pgfusepath{stroke}%
\end{pgfscope}%
\begin{pgfscope}%
\pgfpathrectangle{\pgfqpoint{0.847223in}{0.554012in}}{\pgfqpoint{6.200000in}{4.620000in}}%
\pgfusepath{clip}%
\pgfsetbuttcap%
\pgfsetroundjoin%
\pgfsetlinewidth{1.003750pt}%
\definecolor{currentstroke}{rgb}{1.000000,0.000000,0.000000}%
\pgfsetstrokecolor{currentstroke}%
\pgfsetdash{}{0pt}%
\pgfpathmoveto{\pgfqpoint{5.657784in}{0.581923in}}%
\pgfpathcurveto{\pgfqpoint{5.668834in}{0.581923in}}{\pgfqpoint{5.679433in}{0.586313in}}{\pgfqpoint{5.687246in}{0.594127in}}%
\pgfpathcurveto{\pgfqpoint{5.695060in}{0.601940in}}{\pgfqpoint{5.699450in}{0.612539in}}{\pgfqpoint{5.699450in}{0.623590in}}%
\pgfpathcurveto{\pgfqpoint{5.699450in}{0.634640in}}{\pgfqpoint{5.695060in}{0.645239in}}{\pgfqpoint{5.687246in}{0.653052in}}%
\pgfpathcurveto{\pgfqpoint{5.679433in}{0.660866in}}{\pgfqpoint{5.668834in}{0.665256in}}{\pgfqpoint{5.657784in}{0.665256in}}%
\pgfpathcurveto{\pgfqpoint{5.646733in}{0.665256in}}{\pgfqpoint{5.636134in}{0.660866in}}{\pgfqpoint{5.628321in}{0.653052in}}%
\pgfpathcurveto{\pgfqpoint{5.620507in}{0.645239in}}{\pgfqpoint{5.616117in}{0.634640in}}{\pgfqpoint{5.616117in}{0.623590in}}%
\pgfpathcurveto{\pgfqpoint{5.616117in}{0.612539in}}{\pgfqpoint{5.620507in}{0.601940in}}{\pgfqpoint{5.628321in}{0.594127in}}%
\pgfpathcurveto{\pgfqpoint{5.636134in}{0.586313in}}{\pgfqpoint{5.646733in}{0.581923in}}{\pgfqpoint{5.657784in}{0.581923in}}%
\pgfpathlineto{\pgfqpoint{5.657784in}{0.581923in}}%
\pgfpathclose%
\pgfusepath{stroke}%
\end{pgfscope}%
\begin{pgfscope}%
\pgfpathrectangle{\pgfqpoint{0.847223in}{0.554012in}}{\pgfqpoint{6.200000in}{4.620000in}}%
\pgfusepath{clip}%
\pgfsetbuttcap%
\pgfsetroundjoin%
\pgfsetlinewidth{1.003750pt}%
\definecolor{currentstroke}{rgb}{1.000000,0.000000,0.000000}%
\pgfsetstrokecolor{currentstroke}%
\pgfsetdash{}{0pt}%
\pgfpathmoveto{\pgfqpoint{5.663117in}{0.581428in}}%
\pgfpathcurveto{\pgfqpoint{5.674167in}{0.581428in}}{\pgfqpoint{5.684766in}{0.585819in}}{\pgfqpoint{5.692580in}{0.593632in}}%
\pgfpathcurveto{\pgfqpoint{5.700393in}{0.601446in}}{\pgfqpoint{5.704783in}{0.612045in}}{\pgfqpoint{5.704783in}{0.623095in}}%
\pgfpathcurveto{\pgfqpoint{5.704783in}{0.634145in}}{\pgfqpoint{5.700393in}{0.644744in}}{\pgfqpoint{5.692580in}{0.652558in}}%
\pgfpathcurveto{\pgfqpoint{5.684766in}{0.660371in}}{\pgfqpoint{5.674167in}{0.664762in}}{\pgfqpoint{5.663117in}{0.664762in}}%
\pgfpathcurveto{\pgfqpoint{5.652067in}{0.664762in}}{\pgfqpoint{5.641468in}{0.660371in}}{\pgfqpoint{5.633654in}{0.652558in}}%
\pgfpathcurveto{\pgfqpoint{5.625840in}{0.644744in}}{\pgfqpoint{5.621450in}{0.634145in}}{\pgfqpoint{5.621450in}{0.623095in}}%
\pgfpathcurveto{\pgfqpoint{5.621450in}{0.612045in}}{\pgfqpoint{5.625840in}{0.601446in}}{\pgfqpoint{5.633654in}{0.593632in}}%
\pgfpathcurveto{\pgfqpoint{5.641468in}{0.585819in}}{\pgfqpoint{5.652067in}{0.581428in}}{\pgfqpoint{5.663117in}{0.581428in}}%
\pgfpathlineto{\pgfqpoint{5.663117in}{0.581428in}}%
\pgfpathclose%
\pgfusepath{stroke}%
\end{pgfscope}%
\begin{pgfscope}%
\pgfpathrectangle{\pgfqpoint{0.847223in}{0.554012in}}{\pgfqpoint{6.200000in}{4.620000in}}%
\pgfusepath{clip}%
\pgfsetbuttcap%
\pgfsetroundjoin%
\pgfsetlinewidth{1.003750pt}%
\definecolor{currentstroke}{rgb}{1.000000,0.000000,0.000000}%
\pgfsetstrokecolor{currentstroke}%
\pgfsetdash{}{0pt}%
\pgfpathmoveto{\pgfqpoint{5.668450in}{0.580935in}}%
\pgfpathcurveto{\pgfqpoint{5.679500in}{0.580935in}}{\pgfqpoint{5.690099in}{0.585325in}}{\pgfqpoint{5.697913in}{0.593139in}}%
\pgfpathcurveto{\pgfqpoint{5.705726in}{0.600952in}}{\pgfqpoint{5.710117in}{0.611551in}}{\pgfqpoint{5.710117in}{0.622602in}}%
\pgfpathcurveto{\pgfqpoint{5.710117in}{0.633652in}}{\pgfqpoint{5.705726in}{0.644251in}}{\pgfqpoint{5.697913in}{0.652064in}}%
\pgfpathcurveto{\pgfqpoint{5.690099in}{0.659878in}}{\pgfqpoint{5.679500in}{0.664268in}}{\pgfqpoint{5.668450in}{0.664268in}}%
\pgfpathcurveto{\pgfqpoint{5.657400in}{0.664268in}}{\pgfqpoint{5.646801in}{0.659878in}}{\pgfqpoint{5.638987in}{0.652064in}}%
\pgfpathcurveto{\pgfqpoint{5.631174in}{0.644251in}}{\pgfqpoint{5.626783in}{0.633652in}}{\pgfqpoint{5.626783in}{0.622602in}}%
\pgfpathcurveto{\pgfqpoint{5.626783in}{0.611551in}}{\pgfqpoint{5.631174in}{0.600952in}}{\pgfqpoint{5.638987in}{0.593139in}}%
\pgfpathcurveto{\pgfqpoint{5.646801in}{0.585325in}}{\pgfqpoint{5.657400in}{0.580935in}}{\pgfqpoint{5.668450in}{0.580935in}}%
\pgfpathlineto{\pgfqpoint{5.668450in}{0.580935in}}%
\pgfpathclose%
\pgfusepath{stroke}%
\end{pgfscope}%
\begin{pgfscope}%
\pgfpathrectangle{\pgfqpoint{0.847223in}{0.554012in}}{\pgfqpoint{6.200000in}{4.620000in}}%
\pgfusepath{clip}%
\pgfsetbuttcap%
\pgfsetroundjoin%
\pgfsetlinewidth{1.003750pt}%
\definecolor{currentstroke}{rgb}{1.000000,0.000000,0.000000}%
\pgfsetstrokecolor{currentstroke}%
\pgfsetdash{}{0pt}%
\pgfpathmoveto{\pgfqpoint{5.673783in}{0.580442in}}%
\pgfpathcurveto{\pgfqpoint{5.684833in}{0.580442in}}{\pgfqpoint{5.695432in}{0.584833in}}{\pgfqpoint{5.703246in}{0.592646in}}%
\pgfpathcurveto{\pgfqpoint{5.711060in}{0.600460in}}{\pgfqpoint{5.715450in}{0.611059in}}{\pgfqpoint{5.715450in}{0.622109in}}%
\pgfpathcurveto{\pgfqpoint{5.715450in}{0.633159in}}{\pgfqpoint{5.711060in}{0.643758in}}{\pgfqpoint{5.703246in}{0.651572in}}%
\pgfpathcurveto{\pgfqpoint{5.695432in}{0.659385in}}{\pgfqpoint{5.684833in}{0.663776in}}{\pgfqpoint{5.673783in}{0.663776in}}%
\pgfpathcurveto{\pgfqpoint{5.662733in}{0.663776in}}{\pgfqpoint{5.652134in}{0.659385in}}{\pgfqpoint{5.644320in}{0.651572in}}%
\pgfpathcurveto{\pgfqpoint{5.636507in}{0.643758in}}{\pgfqpoint{5.632117in}{0.633159in}}{\pgfqpoint{5.632117in}{0.622109in}}%
\pgfpathcurveto{\pgfqpoint{5.632117in}{0.611059in}}{\pgfqpoint{5.636507in}{0.600460in}}{\pgfqpoint{5.644320in}{0.592646in}}%
\pgfpathcurveto{\pgfqpoint{5.652134in}{0.584833in}}{\pgfqpoint{5.662733in}{0.580442in}}{\pgfqpoint{5.673783in}{0.580442in}}%
\pgfpathlineto{\pgfqpoint{5.673783in}{0.580442in}}%
\pgfpathclose%
\pgfusepath{stroke}%
\end{pgfscope}%
\begin{pgfscope}%
\pgfpathrectangle{\pgfqpoint{0.847223in}{0.554012in}}{\pgfqpoint{6.200000in}{4.620000in}}%
\pgfusepath{clip}%
\pgfsetbuttcap%
\pgfsetroundjoin%
\pgfsetlinewidth{1.003750pt}%
\definecolor{currentstroke}{rgb}{1.000000,0.000000,0.000000}%
\pgfsetstrokecolor{currentstroke}%
\pgfsetdash{}{0pt}%
\pgfpathmoveto{\pgfqpoint{5.679116in}{0.579951in}}%
\pgfpathcurveto{\pgfqpoint{5.690167in}{0.579951in}}{\pgfqpoint{5.700766in}{0.584341in}}{\pgfqpoint{5.708579in}{0.592155in}}%
\pgfpathcurveto{\pgfqpoint{5.716393in}{0.599968in}}{\pgfqpoint{5.720783in}{0.610567in}}{\pgfqpoint{5.720783in}{0.621617in}}%
\pgfpathcurveto{\pgfqpoint{5.720783in}{0.632668in}}{\pgfqpoint{5.716393in}{0.643267in}}{\pgfqpoint{5.708579in}{0.651080in}}%
\pgfpathcurveto{\pgfqpoint{5.700766in}{0.658894in}}{\pgfqpoint{5.690167in}{0.663284in}}{\pgfqpoint{5.679116in}{0.663284in}}%
\pgfpathcurveto{\pgfqpoint{5.668066in}{0.663284in}}{\pgfqpoint{5.657467in}{0.658894in}}{\pgfqpoint{5.649654in}{0.651080in}}%
\pgfpathcurveto{\pgfqpoint{5.641840in}{0.643267in}}{\pgfqpoint{5.637450in}{0.632668in}}{\pgfqpoint{5.637450in}{0.621617in}}%
\pgfpathcurveto{\pgfqpoint{5.637450in}{0.610567in}}{\pgfqpoint{5.641840in}{0.599968in}}{\pgfqpoint{5.649654in}{0.592155in}}%
\pgfpathcurveto{\pgfqpoint{5.657467in}{0.584341in}}{\pgfqpoint{5.668066in}{0.579951in}}{\pgfqpoint{5.679116in}{0.579951in}}%
\pgfpathlineto{\pgfqpoint{5.679116in}{0.579951in}}%
\pgfpathclose%
\pgfusepath{stroke}%
\end{pgfscope}%
\begin{pgfscope}%
\pgfpathrectangle{\pgfqpoint{0.847223in}{0.554012in}}{\pgfqpoint{6.200000in}{4.620000in}}%
\pgfusepath{clip}%
\pgfsetbuttcap%
\pgfsetroundjoin%
\pgfsetlinewidth{1.003750pt}%
\definecolor{currentstroke}{rgb}{1.000000,0.000000,0.000000}%
\pgfsetstrokecolor{currentstroke}%
\pgfsetdash{}{0pt}%
\pgfpathmoveto{\pgfqpoint{5.684450in}{0.579460in}}%
\pgfpathcurveto{\pgfqpoint{5.695500in}{0.579460in}}{\pgfqpoint{5.706099in}{0.583850in}}{\pgfqpoint{5.713912in}{0.591664in}}%
\pgfpathcurveto{\pgfqpoint{5.721726in}{0.599478in}}{\pgfqpoint{5.726116in}{0.610077in}}{\pgfqpoint{5.726116in}{0.621127in}}%
\pgfpathcurveto{\pgfqpoint{5.726116in}{0.632177in}}{\pgfqpoint{5.721726in}{0.642776in}}{\pgfqpoint{5.713912in}{0.650590in}}%
\pgfpathcurveto{\pgfqpoint{5.706099in}{0.658403in}}{\pgfqpoint{5.695500in}{0.662794in}}{\pgfqpoint{5.684450in}{0.662794in}}%
\pgfpathcurveto{\pgfqpoint{5.673400in}{0.662794in}}{\pgfqpoint{5.662800in}{0.658403in}}{\pgfqpoint{5.654987in}{0.650590in}}%
\pgfpathcurveto{\pgfqpoint{5.647173in}{0.642776in}}{\pgfqpoint{5.642783in}{0.632177in}}{\pgfqpoint{5.642783in}{0.621127in}}%
\pgfpathcurveto{\pgfqpoint{5.642783in}{0.610077in}}{\pgfqpoint{5.647173in}{0.599478in}}{\pgfqpoint{5.654987in}{0.591664in}}%
\pgfpathcurveto{\pgfqpoint{5.662800in}{0.583850in}}{\pgfqpoint{5.673400in}{0.579460in}}{\pgfqpoint{5.684450in}{0.579460in}}%
\pgfpathlineto{\pgfqpoint{5.684450in}{0.579460in}}%
\pgfpathclose%
\pgfusepath{stroke}%
\end{pgfscope}%
\begin{pgfscope}%
\pgfpathrectangle{\pgfqpoint{0.847223in}{0.554012in}}{\pgfqpoint{6.200000in}{4.620000in}}%
\pgfusepath{clip}%
\pgfsetbuttcap%
\pgfsetroundjoin%
\pgfsetlinewidth{1.003750pt}%
\definecolor{currentstroke}{rgb}{1.000000,0.000000,0.000000}%
\pgfsetstrokecolor{currentstroke}%
\pgfsetdash{}{0pt}%
\pgfpathmoveto{\pgfqpoint{5.689783in}{0.578971in}}%
\pgfpathcurveto{\pgfqpoint{5.700833in}{0.578971in}}{\pgfqpoint{5.711432in}{0.583361in}}{\pgfqpoint{5.719246in}{0.591175in}}%
\pgfpathcurveto{\pgfqpoint{5.727059in}{0.598988in}}{\pgfqpoint{5.731450in}{0.609587in}}{\pgfqpoint{5.731450in}{0.620637in}}%
\pgfpathcurveto{\pgfqpoint{5.731450in}{0.631687in}}{\pgfqpoint{5.727059in}{0.642286in}}{\pgfqpoint{5.719246in}{0.650100in}}%
\pgfpathcurveto{\pgfqpoint{5.711432in}{0.657914in}}{\pgfqpoint{5.700833in}{0.662304in}}{\pgfqpoint{5.689783in}{0.662304in}}%
\pgfpathcurveto{\pgfqpoint{5.678733in}{0.662304in}}{\pgfqpoint{5.668134in}{0.657914in}}{\pgfqpoint{5.660320in}{0.650100in}}%
\pgfpathcurveto{\pgfqpoint{5.652506in}{0.642286in}}{\pgfqpoint{5.648116in}{0.631687in}}{\pgfqpoint{5.648116in}{0.620637in}}%
\pgfpathcurveto{\pgfqpoint{5.648116in}{0.609587in}}{\pgfqpoint{5.652506in}{0.598988in}}{\pgfqpoint{5.660320in}{0.591175in}}%
\pgfpathcurveto{\pgfqpoint{5.668134in}{0.583361in}}{\pgfqpoint{5.678733in}{0.578971in}}{\pgfqpoint{5.689783in}{0.578971in}}%
\pgfpathlineto{\pgfqpoint{5.689783in}{0.578971in}}%
\pgfpathclose%
\pgfusepath{stroke}%
\end{pgfscope}%
\begin{pgfscope}%
\pgfpathrectangle{\pgfqpoint{0.847223in}{0.554012in}}{\pgfqpoint{6.200000in}{4.620000in}}%
\pgfusepath{clip}%
\pgfsetbuttcap%
\pgfsetroundjoin%
\pgfsetlinewidth{1.003750pt}%
\definecolor{currentstroke}{rgb}{1.000000,0.000000,0.000000}%
\pgfsetstrokecolor{currentstroke}%
\pgfsetdash{}{0pt}%
\pgfpathmoveto{\pgfqpoint{5.695116in}{0.578482in}}%
\pgfpathcurveto{\pgfqpoint{5.706166in}{0.578482in}}{\pgfqpoint{5.716765in}{0.582872in}}{\pgfqpoint{5.724579in}{0.590686in}}%
\pgfpathcurveto{\pgfqpoint{5.732392in}{0.598500in}}{\pgfqpoint{5.736783in}{0.609099in}}{\pgfqpoint{5.736783in}{0.620149in}}%
\pgfpathcurveto{\pgfqpoint{5.736783in}{0.631199in}}{\pgfqpoint{5.732392in}{0.641798in}}{\pgfqpoint{5.724579in}{0.649611in}}%
\pgfpathcurveto{\pgfqpoint{5.716765in}{0.657425in}}{\pgfqpoint{5.706166in}{0.661815in}}{\pgfqpoint{5.695116in}{0.661815in}}%
\pgfpathcurveto{\pgfqpoint{5.684066in}{0.661815in}}{\pgfqpoint{5.673467in}{0.657425in}}{\pgfqpoint{5.665653in}{0.649611in}}%
\pgfpathcurveto{\pgfqpoint{5.657840in}{0.641798in}}{\pgfqpoint{5.653449in}{0.631199in}}{\pgfqpoint{5.653449in}{0.620149in}}%
\pgfpathcurveto{\pgfqpoint{5.653449in}{0.609099in}}{\pgfqpoint{5.657840in}{0.598500in}}{\pgfqpoint{5.665653in}{0.590686in}}%
\pgfpathcurveto{\pgfqpoint{5.673467in}{0.582872in}}{\pgfqpoint{5.684066in}{0.578482in}}{\pgfqpoint{5.695116in}{0.578482in}}%
\pgfpathlineto{\pgfqpoint{5.695116in}{0.578482in}}%
\pgfpathclose%
\pgfusepath{stroke}%
\end{pgfscope}%
\begin{pgfscope}%
\pgfpathrectangle{\pgfqpoint{0.847223in}{0.554012in}}{\pgfqpoint{6.200000in}{4.620000in}}%
\pgfusepath{clip}%
\pgfsetbuttcap%
\pgfsetroundjoin%
\pgfsetlinewidth{1.003750pt}%
\definecolor{currentstroke}{rgb}{1.000000,0.000000,0.000000}%
\pgfsetstrokecolor{currentstroke}%
\pgfsetdash{}{0pt}%
\pgfpathmoveto{\pgfqpoint{5.700449in}{0.577994in}}%
\pgfpathcurveto{\pgfqpoint{5.711499in}{0.577994in}}{\pgfqpoint{5.722098in}{0.582385in}}{\pgfqpoint{5.729912in}{0.590198in}}%
\pgfpathcurveto{\pgfqpoint{5.737726in}{0.598012in}}{\pgfqpoint{5.742116in}{0.608611in}}{\pgfqpoint{5.742116in}{0.619661in}}%
\pgfpathcurveto{\pgfqpoint{5.742116in}{0.630711in}}{\pgfqpoint{5.737726in}{0.641310in}}{\pgfqpoint{5.729912in}{0.649124in}}%
\pgfpathcurveto{\pgfqpoint{5.722098in}{0.656937in}}{\pgfqpoint{5.711499in}{0.661328in}}{\pgfqpoint{5.700449in}{0.661328in}}%
\pgfpathcurveto{\pgfqpoint{5.689399in}{0.661328in}}{\pgfqpoint{5.678800in}{0.656937in}}{\pgfqpoint{5.670987in}{0.649124in}}%
\pgfpathcurveto{\pgfqpoint{5.663173in}{0.641310in}}{\pgfqpoint{5.658783in}{0.630711in}}{\pgfqpoint{5.658783in}{0.619661in}}%
\pgfpathcurveto{\pgfqpoint{5.658783in}{0.608611in}}{\pgfqpoint{5.663173in}{0.598012in}}{\pgfqpoint{5.670987in}{0.590198in}}%
\pgfpathcurveto{\pgfqpoint{5.678800in}{0.582385in}}{\pgfqpoint{5.689399in}{0.577994in}}{\pgfqpoint{5.700449in}{0.577994in}}%
\pgfpathlineto{\pgfqpoint{5.700449in}{0.577994in}}%
\pgfpathclose%
\pgfusepath{stroke}%
\end{pgfscope}%
\begin{pgfscope}%
\pgfpathrectangle{\pgfqpoint{0.847223in}{0.554012in}}{\pgfqpoint{6.200000in}{4.620000in}}%
\pgfusepath{clip}%
\pgfsetbuttcap%
\pgfsetroundjoin%
\pgfsetlinewidth{1.003750pt}%
\definecolor{currentstroke}{rgb}{1.000000,0.000000,0.000000}%
\pgfsetstrokecolor{currentstroke}%
\pgfsetdash{}{0pt}%
\pgfpathmoveto{\pgfqpoint{5.705783in}{0.577508in}}%
\pgfpathcurveto{\pgfqpoint{5.716833in}{0.577508in}}{\pgfqpoint{5.727432in}{0.581898in}}{\pgfqpoint{5.735245in}{0.589712in}}%
\pgfpathcurveto{\pgfqpoint{5.743059in}{0.597525in}}{\pgfqpoint{5.747449in}{0.608124in}}{\pgfqpoint{5.747449in}{0.619174in}}%
\pgfpathcurveto{\pgfqpoint{5.747449in}{0.630225in}}{\pgfqpoint{5.743059in}{0.640824in}}{\pgfqpoint{5.735245in}{0.648637in}}%
\pgfpathcurveto{\pgfqpoint{5.727432in}{0.656451in}}{\pgfqpoint{5.716833in}{0.660841in}}{\pgfqpoint{5.705783in}{0.660841in}}%
\pgfpathcurveto{\pgfqpoint{5.694732in}{0.660841in}}{\pgfqpoint{5.684133in}{0.656451in}}{\pgfqpoint{5.676320in}{0.648637in}}%
\pgfpathcurveto{\pgfqpoint{5.668506in}{0.640824in}}{\pgfqpoint{5.664116in}{0.630225in}}{\pgfqpoint{5.664116in}{0.619174in}}%
\pgfpathcurveto{\pgfqpoint{5.664116in}{0.608124in}}{\pgfqpoint{5.668506in}{0.597525in}}{\pgfqpoint{5.676320in}{0.589712in}}%
\pgfpathcurveto{\pgfqpoint{5.684133in}{0.581898in}}{\pgfqpoint{5.694732in}{0.577508in}}{\pgfqpoint{5.705783in}{0.577508in}}%
\pgfpathlineto{\pgfqpoint{5.705783in}{0.577508in}}%
\pgfpathclose%
\pgfusepath{stroke}%
\end{pgfscope}%
\begin{pgfscope}%
\pgfpathrectangle{\pgfqpoint{0.847223in}{0.554012in}}{\pgfqpoint{6.200000in}{4.620000in}}%
\pgfusepath{clip}%
\pgfsetbuttcap%
\pgfsetroundjoin%
\pgfsetlinewidth{1.003750pt}%
\definecolor{currentstroke}{rgb}{1.000000,0.000000,0.000000}%
\pgfsetstrokecolor{currentstroke}%
\pgfsetdash{}{0pt}%
\pgfpathmoveto{\pgfqpoint{5.711116in}{0.577022in}}%
\pgfpathcurveto{\pgfqpoint{5.722166in}{0.577022in}}{\pgfqpoint{5.732765in}{0.581412in}}{\pgfqpoint{5.740579in}{0.589226in}}%
\pgfpathcurveto{\pgfqpoint{5.748392in}{0.597039in}}{\pgfqpoint{5.752782in}{0.607639in}}{\pgfqpoint{5.752782in}{0.618689in}}%
\pgfpathcurveto{\pgfqpoint{5.752782in}{0.629739in}}{\pgfqpoint{5.748392in}{0.640338in}}{\pgfqpoint{5.740579in}{0.648151in}}%
\pgfpathcurveto{\pgfqpoint{5.732765in}{0.655965in}}{\pgfqpoint{5.722166in}{0.660355in}}{\pgfqpoint{5.711116in}{0.660355in}}%
\pgfpathcurveto{\pgfqpoint{5.700066in}{0.660355in}}{\pgfqpoint{5.689467in}{0.655965in}}{\pgfqpoint{5.681653in}{0.648151in}}%
\pgfpathcurveto{\pgfqpoint{5.673839in}{0.640338in}}{\pgfqpoint{5.669449in}{0.629739in}}{\pgfqpoint{5.669449in}{0.618689in}}%
\pgfpathcurveto{\pgfqpoint{5.669449in}{0.607639in}}{\pgfqpoint{5.673839in}{0.597039in}}{\pgfqpoint{5.681653in}{0.589226in}}%
\pgfpathcurveto{\pgfqpoint{5.689467in}{0.581412in}}{\pgfqpoint{5.700066in}{0.577022in}}{\pgfqpoint{5.711116in}{0.577022in}}%
\pgfpathlineto{\pgfqpoint{5.711116in}{0.577022in}}%
\pgfpathclose%
\pgfusepath{stroke}%
\end{pgfscope}%
\begin{pgfscope}%
\pgfpathrectangle{\pgfqpoint{0.847223in}{0.554012in}}{\pgfqpoint{6.200000in}{4.620000in}}%
\pgfusepath{clip}%
\pgfsetbuttcap%
\pgfsetroundjoin%
\pgfsetlinewidth{1.003750pt}%
\definecolor{currentstroke}{rgb}{1.000000,0.000000,0.000000}%
\pgfsetstrokecolor{currentstroke}%
\pgfsetdash{}{0pt}%
\pgfpathmoveto{\pgfqpoint{5.716449in}{0.576537in}}%
\pgfpathcurveto{\pgfqpoint{5.727499in}{0.576537in}}{\pgfqpoint{5.738098in}{0.580928in}}{\pgfqpoint{5.745912in}{0.588741in}}%
\pgfpathcurveto{\pgfqpoint{5.753725in}{0.596555in}}{\pgfqpoint{5.758116in}{0.607154in}}{\pgfqpoint{5.758116in}{0.618204in}}%
\pgfpathcurveto{\pgfqpoint{5.758116in}{0.629254in}}{\pgfqpoint{5.753725in}{0.639853in}}{\pgfqpoint{5.745912in}{0.647667in}}%
\pgfpathcurveto{\pgfqpoint{5.738098in}{0.655480in}}{\pgfqpoint{5.727499in}{0.659871in}}{\pgfqpoint{5.716449in}{0.659871in}}%
\pgfpathcurveto{\pgfqpoint{5.705399in}{0.659871in}}{\pgfqpoint{5.694800in}{0.655480in}}{\pgfqpoint{5.686986in}{0.647667in}}%
\pgfpathcurveto{\pgfqpoint{5.679173in}{0.639853in}}{\pgfqpoint{5.674782in}{0.629254in}}{\pgfqpoint{5.674782in}{0.618204in}}%
\pgfpathcurveto{\pgfqpoint{5.674782in}{0.607154in}}{\pgfqpoint{5.679173in}{0.596555in}}{\pgfqpoint{5.686986in}{0.588741in}}%
\pgfpathcurveto{\pgfqpoint{5.694800in}{0.580928in}}{\pgfqpoint{5.705399in}{0.576537in}}{\pgfqpoint{5.716449in}{0.576537in}}%
\pgfpathlineto{\pgfqpoint{5.716449in}{0.576537in}}%
\pgfpathclose%
\pgfusepath{stroke}%
\end{pgfscope}%
\begin{pgfscope}%
\pgfpathrectangle{\pgfqpoint{0.847223in}{0.554012in}}{\pgfqpoint{6.200000in}{4.620000in}}%
\pgfusepath{clip}%
\pgfsetbuttcap%
\pgfsetroundjoin%
\pgfsetlinewidth{1.003750pt}%
\definecolor{currentstroke}{rgb}{1.000000,0.000000,0.000000}%
\pgfsetstrokecolor{currentstroke}%
\pgfsetdash{}{0pt}%
\pgfpathmoveto{\pgfqpoint{5.721782in}{0.576053in}}%
\pgfpathcurveto{\pgfqpoint{5.732832in}{0.576053in}}{\pgfqpoint{5.743431in}{0.580444in}}{\pgfqpoint{5.751245in}{0.588257in}}%
\pgfpathcurveto{\pgfqpoint{5.759059in}{0.596071in}}{\pgfqpoint{5.763449in}{0.606670in}}{\pgfqpoint{5.763449in}{0.617720in}}%
\pgfpathcurveto{\pgfqpoint{5.763449in}{0.628770in}}{\pgfqpoint{5.759059in}{0.639369in}}{\pgfqpoint{5.751245in}{0.647183in}}%
\pgfpathcurveto{\pgfqpoint{5.743431in}{0.654997in}}{\pgfqpoint{5.732832in}{0.659387in}}{\pgfqpoint{5.721782in}{0.659387in}}%
\pgfpathcurveto{\pgfqpoint{5.710732in}{0.659387in}}{\pgfqpoint{5.700133in}{0.654997in}}{\pgfqpoint{5.692319in}{0.647183in}}%
\pgfpathcurveto{\pgfqpoint{5.684506in}{0.639369in}}{\pgfqpoint{5.680116in}{0.628770in}}{\pgfqpoint{5.680116in}{0.617720in}}%
\pgfpathcurveto{\pgfqpoint{5.680116in}{0.606670in}}{\pgfqpoint{5.684506in}{0.596071in}}{\pgfqpoint{5.692319in}{0.588257in}}%
\pgfpathcurveto{\pgfqpoint{5.700133in}{0.580444in}}{\pgfqpoint{5.710732in}{0.576053in}}{\pgfqpoint{5.721782in}{0.576053in}}%
\pgfpathlineto{\pgfqpoint{5.721782in}{0.576053in}}%
\pgfpathclose%
\pgfusepath{stroke}%
\end{pgfscope}%
\begin{pgfscope}%
\pgfpathrectangle{\pgfqpoint{0.847223in}{0.554012in}}{\pgfqpoint{6.200000in}{4.620000in}}%
\pgfusepath{clip}%
\pgfsetbuttcap%
\pgfsetroundjoin%
\pgfsetlinewidth{1.003750pt}%
\definecolor{currentstroke}{rgb}{1.000000,0.000000,0.000000}%
\pgfsetstrokecolor{currentstroke}%
\pgfsetdash{}{0pt}%
\pgfpathmoveto{\pgfqpoint{5.727115in}{0.575571in}}%
\pgfpathcurveto{\pgfqpoint{5.738166in}{0.575571in}}{\pgfqpoint{5.748765in}{0.579961in}}{\pgfqpoint{5.756578in}{0.587774in}}%
\pgfpathcurveto{\pgfqpoint{5.764392in}{0.595588in}}{\pgfqpoint{5.768782in}{0.606187in}}{\pgfqpoint{5.768782in}{0.617237in}}%
\pgfpathcurveto{\pgfqpoint{5.768782in}{0.628287in}}{\pgfqpoint{5.764392in}{0.638886in}}{\pgfqpoint{5.756578in}{0.646700in}}%
\pgfpathcurveto{\pgfqpoint{5.748765in}{0.654514in}}{\pgfqpoint{5.738166in}{0.658904in}}{\pgfqpoint{5.727115in}{0.658904in}}%
\pgfpathcurveto{\pgfqpoint{5.716065in}{0.658904in}}{\pgfqpoint{5.705466in}{0.654514in}}{\pgfqpoint{5.697653in}{0.646700in}}%
\pgfpathcurveto{\pgfqpoint{5.689839in}{0.638886in}}{\pgfqpoint{5.685449in}{0.628287in}}{\pgfqpoint{5.685449in}{0.617237in}}%
\pgfpathcurveto{\pgfqpoint{5.685449in}{0.606187in}}{\pgfqpoint{5.689839in}{0.595588in}}{\pgfqpoint{5.697653in}{0.587774in}}%
\pgfpathcurveto{\pgfqpoint{5.705466in}{0.579961in}}{\pgfqpoint{5.716065in}{0.575571in}}{\pgfqpoint{5.727115in}{0.575571in}}%
\pgfpathlineto{\pgfqpoint{5.727115in}{0.575571in}}%
\pgfpathclose%
\pgfusepath{stroke}%
\end{pgfscope}%
\begin{pgfscope}%
\pgfpathrectangle{\pgfqpoint{0.847223in}{0.554012in}}{\pgfqpoint{6.200000in}{4.620000in}}%
\pgfusepath{clip}%
\pgfsetbuttcap%
\pgfsetroundjoin%
\pgfsetlinewidth{1.003750pt}%
\definecolor{currentstroke}{rgb}{1.000000,0.000000,0.000000}%
\pgfsetstrokecolor{currentstroke}%
\pgfsetdash{}{0pt}%
\pgfpathmoveto{\pgfqpoint{5.732449in}{0.575089in}}%
\pgfpathcurveto{\pgfqpoint{5.743499in}{0.575089in}}{\pgfqpoint{5.754098in}{0.579479in}}{\pgfqpoint{5.761911in}{0.587293in}}%
\pgfpathcurveto{\pgfqpoint{5.769725in}{0.595106in}}{\pgfqpoint{5.774115in}{0.605705in}}{\pgfqpoint{5.774115in}{0.616755in}}%
\pgfpathcurveto{\pgfqpoint{5.774115in}{0.627806in}}{\pgfqpoint{5.769725in}{0.638405in}}{\pgfqpoint{5.761911in}{0.646218in}}%
\pgfpathcurveto{\pgfqpoint{5.754098in}{0.654032in}}{\pgfqpoint{5.743499in}{0.658422in}}{\pgfqpoint{5.732449in}{0.658422in}}%
\pgfpathcurveto{\pgfqpoint{5.721398in}{0.658422in}}{\pgfqpoint{5.710799in}{0.654032in}}{\pgfqpoint{5.702986in}{0.646218in}}%
\pgfpathcurveto{\pgfqpoint{5.695172in}{0.638405in}}{\pgfqpoint{5.690782in}{0.627806in}}{\pgfqpoint{5.690782in}{0.616755in}}%
\pgfpathcurveto{\pgfqpoint{5.690782in}{0.605705in}}{\pgfqpoint{5.695172in}{0.595106in}}{\pgfqpoint{5.702986in}{0.587293in}}%
\pgfpathcurveto{\pgfqpoint{5.710799in}{0.579479in}}{\pgfqpoint{5.721398in}{0.575089in}}{\pgfqpoint{5.732449in}{0.575089in}}%
\pgfpathlineto{\pgfqpoint{5.732449in}{0.575089in}}%
\pgfpathclose%
\pgfusepath{stroke}%
\end{pgfscope}%
\begin{pgfscope}%
\pgfpathrectangle{\pgfqpoint{0.847223in}{0.554012in}}{\pgfqpoint{6.200000in}{4.620000in}}%
\pgfusepath{clip}%
\pgfsetbuttcap%
\pgfsetroundjoin%
\pgfsetlinewidth{1.003750pt}%
\definecolor{currentstroke}{rgb}{1.000000,0.000000,0.000000}%
\pgfsetstrokecolor{currentstroke}%
\pgfsetdash{}{0pt}%
\pgfpathmoveto{\pgfqpoint{5.737782in}{0.574608in}}%
\pgfpathcurveto{\pgfqpoint{5.748832in}{0.574608in}}{\pgfqpoint{5.759431in}{0.578998in}}{\pgfqpoint{5.767245in}{0.586812in}}%
\pgfpathcurveto{\pgfqpoint{5.775058in}{0.594625in}}{\pgfqpoint{5.779448in}{0.605224in}}{\pgfqpoint{5.779448in}{0.616274in}}%
\pgfpathcurveto{\pgfqpoint{5.779448in}{0.627325in}}{\pgfqpoint{5.775058in}{0.637924in}}{\pgfqpoint{5.767245in}{0.645737in}}%
\pgfpathcurveto{\pgfqpoint{5.759431in}{0.653551in}}{\pgfqpoint{5.748832in}{0.657941in}}{\pgfqpoint{5.737782in}{0.657941in}}%
\pgfpathcurveto{\pgfqpoint{5.726732in}{0.657941in}}{\pgfqpoint{5.716133in}{0.653551in}}{\pgfqpoint{5.708319in}{0.645737in}}%
\pgfpathcurveto{\pgfqpoint{5.700505in}{0.637924in}}{\pgfqpoint{5.696115in}{0.627325in}}{\pgfqpoint{5.696115in}{0.616274in}}%
\pgfpathcurveto{\pgfqpoint{5.696115in}{0.605224in}}{\pgfqpoint{5.700505in}{0.594625in}}{\pgfqpoint{5.708319in}{0.586812in}}%
\pgfpathcurveto{\pgfqpoint{5.716133in}{0.578998in}}{\pgfqpoint{5.726732in}{0.574608in}}{\pgfqpoint{5.737782in}{0.574608in}}%
\pgfpathlineto{\pgfqpoint{5.737782in}{0.574608in}}%
\pgfpathclose%
\pgfusepath{stroke}%
\end{pgfscope}%
\begin{pgfscope}%
\pgfpathrectangle{\pgfqpoint{0.847223in}{0.554012in}}{\pgfqpoint{6.200000in}{4.620000in}}%
\pgfusepath{clip}%
\pgfsetbuttcap%
\pgfsetroundjoin%
\pgfsetlinewidth{1.003750pt}%
\definecolor{currentstroke}{rgb}{1.000000,0.000000,0.000000}%
\pgfsetstrokecolor{currentstroke}%
\pgfsetdash{}{0pt}%
\pgfpathmoveto{\pgfqpoint{5.743115in}{0.574128in}}%
\pgfpathcurveto{\pgfqpoint{5.754165in}{0.574128in}}{\pgfqpoint{5.764764in}{0.578518in}}{\pgfqpoint{5.772578in}{0.586332in}}%
\pgfpathcurveto{\pgfqpoint{5.780391in}{0.594145in}}{\pgfqpoint{5.784782in}{0.604744in}}{\pgfqpoint{5.784782in}{0.615794in}}%
\pgfpathcurveto{\pgfqpoint{5.784782in}{0.626845in}}{\pgfqpoint{5.780391in}{0.637444in}}{\pgfqpoint{5.772578in}{0.645257in}}%
\pgfpathcurveto{\pgfqpoint{5.764764in}{0.653071in}}{\pgfqpoint{5.754165in}{0.657461in}}{\pgfqpoint{5.743115in}{0.657461in}}%
\pgfpathcurveto{\pgfqpoint{5.732065in}{0.657461in}}{\pgfqpoint{5.721466in}{0.653071in}}{\pgfqpoint{5.713652in}{0.645257in}}%
\pgfpathcurveto{\pgfqpoint{5.705839in}{0.637444in}}{\pgfqpoint{5.701448in}{0.626845in}}{\pgfqpoint{5.701448in}{0.615794in}}%
\pgfpathcurveto{\pgfqpoint{5.701448in}{0.604744in}}{\pgfqpoint{5.705839in}{0.594145in}}{\pgfqpoint{5.713652in}{0.586332in}}%
\pgfpathcurveto{\pgfqpoint{5.721466in}{0.578518in}}{\pgfqpoint{5.732065in}{0.574128in}}{\pgfqpoint{5.743115in}{0.574128in}}%
\pgfpathlineto{\pgfqpoint{5.743115in}{0.574128in}}%
\pgfpathclose%
\pgfusepath{stroke}%
\end{pgfscope}%
\begin{pgfscope}%
\pgfpathrectangle{\pgfqpoint{0.847223in}{0.554012in}}{\pgfqpoint{6.200000in}{4.620000in}}%
\pgfusepath{clip}%
\pgfsetbuttcap%
\pgfsetroundjoin%
\pgfsetlinewidth{1.003750pt}%
\definecolor{currentstroke}{rgb}{1.000000,0.000000,0.000000}%
\pgfsetstrokecolor{currentstroke}%
\pgfsetdash{}{0pt}%
\pgfpathmoveto{\pgfqpoint{5.748448in}{0.573649in}}%
\pgfpathcurveto{\pgfqpoint{5.759498in}{0.573649in}}{\pgfqpoint{5.770097in}{0.578039in}}{\pgfqpoint{5.777911in}{0.585853in}}%
\pgfpathcurveto{\pgfqpoint{5.785725in}{0.593666in}}{\pgfqpoint{5.790115in}{0.604265in}}{\pgfqpoint{5.790115in}{0.615315in}}%
\pgfpathcurveto{\pgfqpoint{5.790115in}{0.626366in}}{\pgfqpoint{5.785725in}{0.636965in}}{\pgfqpoint{5.777911in}{0.644778in}}%
\pgfpathcurveto{\pgfqpoint{5.770097in}{0.652592in}}{\pgfqpoint{5.759498in}{0.656982in}}{\pgfqpoint{5.748448in}{0.656982in}}%
\pgfpathcurveto{\pgfqpoint{5.737398in}{0.656982in}}{\pgfqpoint{5.726799in}{0.652592in}}{\pgfqpoint{5.718985in}{0.644778in}}%
\pgfpathcurveto{\pgfqpoint{5.711172in}{0.636965in}}{\pgfqpoint{5.706782in}{0.626366in}}{\pgfqpoint{5.706782in}{0.615315in}}%
\pgfpathcurveto{\pgfqpoint{5.706782in}{0.604265in}}{\pgfqpoint{5.711172in}{0.593666in}}{\pgfqpoint{5.718985in}{0.585853in}}%
\pgfpathcurveto{\pgfqpoint{5.726799in}{0.578039in}}{\pgfqpoint{5.737398in}{0.573649in}}{\pgfqpoint{5.748448in}{0.573649in}}%
\pgfpathlineto{\pgfqpoint{5.748448in}{0.573649in}}%
\pgfpathclose%
\pgfusepath{stroke}%
\end{pgfscope}%
\begin{pgfscope}%
\pgfpathrectangle{\pgfqpoint{0.847223in}{0.554012in}}{\pgfqpoint{6.200000in}{4.620000in}}%
\pgfusepath{clip}%
\pgfsetbuttcap%
\pgfsetroundjoin%
\pgfsetlinewidth{1.003750pt}%
\definecolor{currentstroke}{rgb}{1.000000,0.000000,0.000000}%
\pgfsetstrokecolor{currentstroke}%
\pgfsetdash{}{0pt}%
\pgfpathmoveto{\pgfqpoint{5.753781in}{0.573171in}}%
\pgfpathcurveto{\pgfqpoint{5.764832in}{0.573171in}}{\pgfqpoint{5.775431in}{0.577561in}}{\pgfqpoint{5.783244in}{0.585375in}}%
\pgfpathcurveto{\pgfqpoint{5.791058in}{0.593188in}}{\pgfqpoint{5.795448in}{0.603787in}}{\pgfqpoint{5.795448in}{0.614837in}}%
\pgfpathcurveto{\pgfqpoint{5.795448in}{0.625887in}}{\pgfqpoint{5.791058in}{0.636486in}}{\pgfqpoint{5.783244in}{0.644300in}}%
\pgfpathcurveto{\pgfqpoint{5.775431in}{0.652114in}}{\pgfqpoint{5.764832in}{0.656504in}}{\pgfqpoint{5.753781in}{0.656504in}}%
\pgfpathcurveto{\pgfqpoint{5.742731in}{0.656504in}}{\pgfqpoint{5.732132in}{0.652114in}}{\pgfqpoint{5.724319in}{0.644300in}}%
\pgfpathcurveto{\pgfqpoint{5.716505in}{0.636486in}}{\pgfqpoint{5.712115in}{0.625887in}}{\pgfqpoint{5.712115in}{0.614837in}}%
\pgfpathcurveto{\pgfqpoint{5.712115in}{0.603787in}}{\pgfqpoint{5.716505in}{0.593188in}}{\pgfqpoint{5.724319in}{0.585375in}}%
\pgfpathcurveto{\pgfqpoint{5.732132in}{0.577561in}}{\pgfqpoint{5.742731in}{0.573171in}}{\pgfqpoint{5.753781in}{0.573171in}}%
\pgfpathlineto{\pgfqpoint{5.753781in}{0.573171in}}%
\pgfpathclose%
\pgfusepath{stroke}%
\end{pgfscope}%
\begin{pgfscope}%
\pgfpathrectangle{\pgfqpoint{0.847223in}{0.554012in}}{\pgfqpoint{6.200000in}{4.620000in}}%
\pgfusepath{clip}%
\pgfsetbuttcap%
\pgfsetroundjoin%
\pgfsetlinewidth{1.003750pt}%
\definecolor{currentstroke}{rgb}{1.000000,0.000000,0.000000}%
\pgfsetstrokecolor{currentstroke}%
\pgfsetdash{}{0pt}%
\pgfpathmoveto{\pgfqpoint{5.759115in}{0.572693in}}%
\pgfpathcurveto{\pgfqpoint{5.770165in}{0.572693in}}{\pgfqpoint{5.780764in}{0.577084in}}{\pgfqpoint{5.788577in}{0.584897in}}%
\pgfpathcurveto{\pgfqpoint{5.796391in}{0.592711in}}{\pgfqpoint{5.800781in}{0.603310in}}{\pgfqpoint{5.800781in}{0.614360in}}%
\pgfpathcurveto{\pgfqpoint{5.800781in}{0.625410in}}{\pgfqpoint{5.796391in}{0.636009in}}{\pgfqpoint{5.788577in}{0.643823in}}%
\pgfpathcurveto{\pgfqpoint{5.780764in}{0.651637in}}{\pgfqpoint{5.770165in}{0.656027in}}{\pgfqpoint{5.759115in}{0.656027in}}%
\pgfpathcurveto{\pgfqpoint{5.748065in}{0.656027in}}{\pgfqpoint{5.737466in}{0.651637in}}{\pgfqpoint{5.729652in}{0.643823in}}%
\pgfpathcurveto{\pgfqpoint{5.721838in}{0.636009in}}{\pgfqpoint{5.717448in}{0.625410in}}{\pgfqpoint{5.717448in}{0.614360in}}%
\pgfpathcurveto{\pgfqpoint{5.717448in}{0.603310in}}{\pgfqpoint{5.721838in}{0.592711in}}{\pgfqpoint{5.729652in}{0.584897in}}%
\pgfpathcurveto{\pgfqpoint{5.737466in}{0.577084in}}{\pgfqpoint{5.748065in}{0.572693in}}{\pgfqpoint{5.759115in}{0.572693in}}%
\pgfpathlineto{\pgfqpoint{5.759115in}{0.572693in}}%
\pgfpathclose%
\pgfusepath{stroke}%
\end{pgfscope}%
\begin{pgfscope}%
\pgfpathrectangle{\pgfqpoint{0.847223in}{0.554012in}}{\pgfqpoint{6.200000in}{4.620000in}}%
\pgfusepath{clip}%
\pgfsetbuttcap%
\pgfsetroundjoin%
\pgfsetlinewidth{1.003750pt}%
\definecolor{currentstroke}{rgb}{1.000000,0.000000,0.000000}%
\pgfsetstrokecolor{currentstroke}%
\pgfsetdash{}{0pt}%
\pgfpathmoveto{\pgfqpoint{5.764448in}{0.572217in}}%
\pgfpathcurveto{\pgfqpoint{5.775498in}{0.572217in}}{\pgfqpoint{5.786097in}{0.576607in}}{\pgfqpoint{5.793911in}{0.584421in}}%
\pgfpathcurveto{\pgfqpoint{5.801724in}{0.592235in}}{\pgfqpoint{5.806115in}{0.602834in}}{\pgfqpoint{5.806115in}{0.613884in}}%
\pgfpathcurveto{\pgfqpoint{5.806115in}{0.624934in}}{\pgfqpoint{5.801724in}{0.635533in}}{\pgfqpoint{5.793911in}{0.643347in}}%
\pgfpathcurveto{\pgfqpoint{5.786097in}{0.651160in}}{\pgfqpoint{5.775498in}{0.655551in}}{\pgfqpoint{5.764448in}{0.655551in}}%
\pgfpathcurveto{\pgfqpoint{5.753398in}{0.655551in}}{\pgfqpoint{5.742799in}{0.651160in}}{\pgfqpoint{5.734985in}{0.643347in}}%
\pgfpathcurveto{\pgfqpoint{5.727171in}{0.635533in}}{\pgfqpoint{5.722781in}{0.624934in}}{\pgfqpoint{5.722781in}{0.613884in}}%
\pgfpathcurveto{\pgfqpoint{5.722781in}{0.602834in}}{\pgfqpoint{5.727171in}{0.592235in}}{\pgfqpoint{5.734985in}{0.584421in}}%
\pgfpathcurveto{\pgfqpoint{5.742799in}{0.576607in}}{\pgfqpoint{5.753398in}{0.572217in}}{\pgfqpoint{5.764448in}{0.572217in}}%
\pgfpathlineto{\pgfqpoint{5.764448in}{0.572217in}}%
\pgfpathclose%
\pgfusepath{stroke}%
\end{pgfscope}%
\begin{pgfscope}%
\pgfpathrectangle{\pgfqpoint{0.847223in}{0.554012in}}{\pgfqpoint{6.200000in}{4.620000in}}%
\pgfusepath{clip}%
\pgfsetbuttcap%
\pgfsetroundjoin%
\pgfsetlinewidth{1.003750pt}%
\definecolor{currentstroke}{rgb}{1.000000,0.000000,0.000000}%
\pgfsetstrokecolor{currentstroke}%
\pgfsetdash{}{0pt}%
\pgfpathmoveto{\pgfqpoint{5.769781in}{0.571742in}}%
\pgfpathcurveto{\pgfqpoint{5.780831in}{0.571742in}}{\pgfqpoint{5.791430in}{0.576132in}}{\pgfqpoint{5.799244in}{0.583946in}}%
\pgfpathcurveto{\pgfqpoint{5.807058in}{0.591759in}}{\pgfqpoint{5.811448in}{0.602358in}}{\pgfqpoint{5.811448in}{0.613409in}}%
\pgfpathcurveto{\pgfqpoint{5.811448in}{0.624459in}}{\pgfqpoint{5.807058in}{0.635058in}}{\pgfqpoint{5.799244in}{0.642871in}}%
\pgfpathcurveto{\pgfqpoint{5.791430in}{0.650685in}}{\pgfqpoint{5.780831in}{0.655075in}}{\pgfqpoint{5.769781in}{0.655075in}}%
\pgfpathcurveto{\pgfqpoint{5.758731in}{0.655075in}}{\pgfqpoint{5.748132in}{0.650685in}}{\pgfqpoint{5.740318in}{0.642871in}}%
\pgfpathcurveto{\pgfqpoint{5.732505in}{0.635058in}}{\pgfqpoint{5.728114in}{0.624459in}}{\pgfqpoint{5.728114in}{0.613409in}}%
\pgfpathcurveto{\pgfqpoint{5.728114in}{0.602358in}}{\pgfqpoint{5.732505in}{0.591759in}}{\pgfqpoint{5.740318in}{0.583946in}}%
\pgfpathcurveto{\pgfqpoint{5.748132in}{0.576132in}}{\pgfqpoint{5.758731in}{0.571742in}}{\pgfqpoint{5.769781in}{0.571742in}}%
\pgfpathlineto{\pgfqpoint{5.769781in}{0.571742in}}%
\pgfpathclose%
\pgfusepath{stroke}%
\end{pgfscope}%
\begin{pgfscope}%
\pgfpathrectangle{\pgfqpoint{0.847223in}{0.554012in}}{\pgfqpoint{6.200000in}{4.620000in}}%
\pgfusepath{clip}%
\pgfsetbuttcap%
\pgfsetroundjoin%
\pgfsetlinewidth{1.003750pt}%
\definecolor{currentstroke}{rgb}{1.000000,0.000000,0.000000}%
\pgfsetstrokecolor{currentstroke}%
\pgfsetdash{}{0pt}%
\pgfpathmoveto{\pgfqpoint{5.775114in}{0.571268in}}%
\pgfpathcurveto{\pgfqpoint{5.786164in}{0.571268in}}{\pgfqpoint{5.796763in}{0.575658in}}{\pgfqpoint{5.804577in}{0.583471in}}%
\pgfpathcurveto{\pgfqpoint{5.812391in}{0.591285in}}{\pgfqpoint{5.816781in}{0.601884in}}{\pgfqpoint{5.816781in}{0.612934in}}%
\pgfpathcurveto{\pgfqpoint{5.816781in}{0.623984in}}{\pgfqpoint{5.812391in}{0.634583in}}{\pgfqpoint{5.804577in}{0.642397in}}%
\pgfpathcurveto{\pgfqpoint{5.796763in}{0.650211in}}{\pgfqpoint{5.786164in}{0.654601in}}{\pgfqpoint{5.775114in}{0.654601in}}%
\pgfpathcurveto{\pgfqpoint{5.764064in}{0.654601in}}{\pgfqpoint{5.753465in}{0.650211in}}{\pgfqpoint{5.745652in}{0.642397in}}%
\pgfpathcurveto{\pgfqpoint{5.737838in}{0.634583in}}{\pgfqpoint{5.733448in}{0.623984in}}{\pgfqpoint{5.733448in}{0.612934in}}%
\pgfpathcurveto{\pgfqpoint{5.733448in}{0.601884in}}{\pgfqpoint{5.737838in}{0.591285in}}{\pgfqpoint{5.745652in}{0.583471in}}%
\pgfpathcurveto{\pgfqpoint{5.753465in}{0.575658in}}{\pgfqpoint{5.764064in}{0.571268in}}{\pgfqpoint{5.775114in}{0.571268in}}%
\pgfpathlineto{\pgfqpoint{5.775114in}{0.571268in}}%
\pgfpathclose%
\pgfusepath{stroke}%
\end{pgfscope}%
\begin{pgfscope}%
\pgfpathrectangle{\pgfqpoint{0.847223in}{0.554012in}}{\pgfqpoint{6.200000in}{4.620000in}}%
\pgfusepath{clip}%
\pgfsetbuttcap%
\pgfsetroundjoin%
\pgfsetlinewidth{1.003750pt}%
\definecolor{currentstroke}{rgb}{1.000000,0.000000,0.000000}%
\pgfsetstrokecolor{currentstroke}%
\pgfsetdash{}{0pt}%
\pgfpathmoveto{\pgfqpoint{5.780448in}{0.570794in}}%
\pgfpathcurveto{\pgfqpoint{5.791498in}{0.570794in}}{\pgfqpoint{5.802097in}{0.575184in}}{\pgfqpoint{5.809910in}{0.582998in}}%
\pgfpathcurveto{\pgfqpoint{5.817724in}{0.590812in}}{\pgfqpoint{5.822114in}{0.601411in}}{\pgfqpoint{5.822114in}{0.612461in}}%
\pgfpathcurveto{\pgfqpoint{5.822114in}{0.623511in}}{\pgfqpoint{5.817724in}{0.634110in}}{\pgfqpoint{5.809910in}{0.641924in}}%
\pgfpathcurveto{\pgfqpoint{5.802097in}{0.649737in}}{\pgfqpoint{5.791498in}{0.654127in}}{\pgfqpoint{5.780448in}{0.654127in}}%
\pgfpathcurveto{\pgfqpoint{5.769397in}{0.654127in}}{\pgfqpoint{5.758798in}{0.649737in}}{\pgfqpoint{5.750985in}{0.641924in}}%
\pgfpathcurveto{\pgfqpoint{5.743171in}{0.634110in}}{\pgfqpoint{5.738781in}{0.623511in}}{\pgfqpoint{5.738781in}{0.612461in}}%
\pgfpathcurveto{\pgfqpoint{5.738781in}{0.601411in}}{\pgfqpoint{5.743171in}{0.590812in}}{\pgfqpoint{5.750985in}{0.582998in}}%
\pgfpathcurveto{\pgfqpoint{5.758798in}{0.575184in}}{\pgfqpoint{5.769397in}{0.570794in}}{\pgfqpoint{5.780448in}{0.570794in}}%
\pgfpathlineto{\pgfqpoint{5.780448in}{0.570794in}}%
\pgfpathclose%
\pgfusepath{stroke}%
\end{pgfscope}%
\begin{pgfscope}%
\pgfpathrectangle{\pgfqpoint{0.847223in}{0.554012in}}{\pgfqpoint{6.200000in}{4.620000in}}%
\pgfusepath{clip}%
\pgfsetbuttcap%
\pgfsetroundjoin%
\pgfsetlinewidth{1.003750pt}%
\definecolor{currentstroke}{rgb}{1.000000,0.000000,0.000000}%
\pgfsetstrokecolor{currentstroke}%
\pgfsetdash{}{0pt}%
\pgfpathmoveto{\pgfqpoint{5.785781in}{0.570322in}}%
\pgfpathcurveto{\pgfqpoint{5.796831in}{0.570322in}}{\pgfqpoint{5.807430in}{0.574712in}}{\pgfqpoint{5.815244in}{0.582525in}}%
\pgfpathcurveto{\pgfqpoint{5.823057in}{0.590339in}}{\pgfqpoint{5.827447in}{0.600938in}}{\pgfqpoint{5.827447in}{0.611988in}}%
\pgfpathcurveto{\pgfqpoint{5.827447in}{0.623038in}}{\pgfqpoint{5.823057in}{0.633637in}}{\pgfqpoint{5.815244in}{0.641451in}}%
\pgfpathcurveto{\pgfqpoint{5.807430in}{0.649265in}}{\pgfqpoint{5.796831in}{0.653655in}}{\pgfqpoint{5.785781in}{0.653655in}}%
\pgfpathcurveto{\pgfqpoint{5.774731in}{0.653655in}}{\pgfqpoint{5.764132in}{0.649265in}}{\pgfqpoint{5.756318in}{0.641451in}}%
\pgfpathcurveto{\pgfqpoint{5.748504in}{0.633637in}}{\pgfqpoint{5.744114in}{0.623038in}}{\pgfqpoint{5.744114in}{0.611988in}}%
\pgfpathcurveto{\pgfqpoint{5.744114in}{0.600938in}}{\pgfqpoint{5.748504in}{0.590339in}}{\pgfqpoint{5.756318in}{0.582525in}}%
\pgfpathcurveto{\pgfqpoint{5.764132in}{0.574712in}}{\pgfqpoint{5.774731in}{0.570322in}}{\pgfqpoint{5.785781in}{0.570322in}}%
\pgfpathlineto{\pgfqpoint{5.785781in}{0.570322in}}%
\pgfpathclose%
\pgfusepath{stroke}%
\end{pgfscope}%
\begin{pgfscope}%
\pgfpathrectangle{\pgfqpoint{0.847223in}{0.554012in}}{\pgfqpoint{6.200000in}{4.620000in}}%
\pgfusepath{clip}%
\pgfsetbuttcap%
\pgfsetroundjoin%
\pgfsetlinewidth{1.003750pt}%
\definecolor{currentstroke}{rgb}{1.000000,0.000000,0.000000}%
\pgfsetstrokecolor{currentstroke}%
\pgfsetdash{}{0pt}%
\pgfpathmoveto{\pgfqpoint{5.791114in}{0.569850in}}%
\pgfpathcurveto{\pgfqpoint{5.802164in}{0.569850in}}{\pgfqpoint{5.812763in}{0.574240in}}{\pgfqpoint{5.820577in}{0.582054in}}%
\pgfpathcurveto{\pgfqpoint{5.828390in}{0.589868in}}{\pgfqpoint{5.832781in}{0.600467in}}{\pgfqpoint{5.832781in}{0.611517in}}%
\pgfpathcurveto{\pgfqpoint{5.832781in}{0.622567in}}{\pgfqpoint{5.828390in}{0.633166in}}{\pgfqpoint{5.820577in}{0.640979in}}%
\pgfpathcurveto{\pgfqpoint{5.812763in}{0.648793in}}{\pgfqpoint{5.802164in}{0.653183in}}{\pgfqpoint{5.791114in}{0.653183in}}%
\pgfpathcurveto{\pgfqpoint{5.780064in}{0.653183in}}{\pgfqpoint{5.769465in}{0.648793in}}{\pgfqpoint{5.761651in}{0.640979in}}%
\pgfpathcurveto{\pgfqpoint{5.753838in}{0.633166in}}{\pgfqpoint{5.749447in}{0.622567in}}{\pgfqpoint{5.749447in}{0.611517in}}%
\pgfpathcurveto{\pgfqpoint{5.749447in}{0.600467in}}{\pgfqpoint{5.753838in}{0.589868in}}{\pgfqpoint{5.761651in}{0.582054in}}%
\pgfpathcurveto{\pgfqpoint{5.769465in}{0.574240in}}{\pgfqpoint{5.780064in}{0.569850in}}{\pgfqpoint{5.791114in}{0.569850in}}%
\pgfpathlineto{\pgfqpoint{5.791114in}{0.569850in}}%
\pgfpathclose%
\pgfusepath{stroke}%
\end{pgfscope}%
\begin{pgfscope}%
\pgfpathrectangle{\pgfqpoint{0.847223in}{0.554012in}}{\pgfqpoint{6.200000in}{4.620000in}}%
\pgfusepath{clip}%
\pgfsetbuttcap%
\pgfsetroundjoin%
\pgfsetlinewidth{1.003750pt}%
\definecolor{currentstroke}{rgb}{1.000000,0.000000,0.000000}%
\pgfsetstrokecolor{currentstroke}%
\pgfsetdash{}{0pt}%
\pgfpathmoveto{\pgfqpoint{5.796447in}{0.569379in}}%
\pgfpathcurveto{\pgfqpoint{5.807497in}{0.569379in}}{\pgfqpoint{5.818096in}{0.573770in}}{\pgfqpoint{5.825910in}{0.581583in}}%
\pgfpathcurveto{\pgfqpoint{5.833724in}{0.589397in}}{\pgfqpoint{5.838114in}{0.599996in}}{\pgfqpoint{5.838114in}{0.611046in}}%
\pgfpathcurveto{\pgfqpoint{5.838114in}{0.622096in}}{\pgfqpoint{5.833724in}{0.632695in}}{\pgfqpoint{5.825910in}{0.640509in}}%
\pgfpathcurveto{\pgfqpoint{5.818096in}{0.648322in}}{\pgfqpoint{5.807497in}{0.652713in}}{\pgfqpoint{5.796447in}{0.652713in}}%
\pgfpathcurveto{\pgfqpoint{5.785397in}{0.652713in}}{\pgfqpoint{5.774798in}{0.648322in}}{\pgfqpoint{5.766984in}{0.640509in}}%
\pgfpathcurveto{\pgfqpoint{5.759171in}{0.632695in}}{\pgfqpoint{5.754781in}{0.622096in}}{\pgfqpoint{5.754781in}{0.611046in}}%
\pgfpathcurveto{\pgfqpoint{5.754781in}{0.599996in}}{\pgfqpoint{5.759171in}{0.589397in}}{\pgfqpoint{5.766984in}{0.581583in}}%
\pgfpathcurveto{\pgfqpoint{5.774798in}{0.573770in}}{\pgfqpoint{5.785397in}{0.569379in}}{\pgfqpoint{5.796447in}{0.569379in}}%
\pgfpathlineto{\pgfqpoint{5.796447in}{0.569379in}}%
\pgfpathclose%
\pgfusepath{stroke}%
\end{pgfscope}%
\begin{pgfscope}%
\pgfpathrectangle{\pgfqpoint{0.847223in}{0.554012in}}{\pgfqpoint{6.200000in}{4.620000in}}%
\pgfusepath{clip}%
\pgfsetbuttcap%
\pgfsetroundjoin%
\pgfsetlinewidth{1.003750pt}%
\definecolor{currentstroke}{rgb}{1.000000,0.000000,0.000000}%
\pgfsetstrokecolor{currentstroke}%
\pgfsetdash{}{0pt}%
\pgfpathmoveto{\pgfqpoint{5.801780in}{0.568910in}}%
\pgfpathcurveto{\pgfqpoint{5.812831in}{0.568910in}}{\pgfqpoint{5.823430in}{0.573300in}}{\pgfqpoint{5.831243in}{0.581113in}}%
\pgfpathcurveto{\pgfqpoint{5.839057in}{0.588927in}}{\pgfqpoint{5.843447in}{0.599526in}}{\pgfqpoint{5.843447in}{0.610576in}}%
\pgfpathcurveto{\pgfqpoint{5.843447in}{0.621626in}}{\pgfqpoint{5.839057in}{0.632225in}}{\pgfqpoint{5.831243in}{0.640039in}}%
\pgfpathcurveto{\pgfqpoint{5.823430in}{0.647853in}}{\pgfqpoint{5.812831in}{0.652243in}}{\pgfqpoint{5.801780in}{0.652243in}}%
\pgfpathcurveto{\pgfqpoint{5.790730in}{0.652243in}}{\pgfqpoint{5.780131in}{0.647853in}}{\pgfqpoint{5.772318in}{0.640039in}}%
\pgfpathcurveto{\pgfqpoint{5.764504in}{0.632225in}}{\pgfqpoint{5.760114in}{0.621626in}}{\pgfqpoint{5.760114in}{0.610576in}}%
\pgfpathcurveto{\pgfqpoint{5.760114in}{0.599526in}}{\pgfqpoint{5.764504in}{0.588927in}}{\pgfqpoint{5.772318in}{0.581113in}}%
\pgfpathcurveto{\pgfqpoint{5.780131in}{0.573300in}}{\pgfqpoint{5.790730in}{0.568910in}}{\pgfqpoint{5.801780in}{0.568910in}}%
\pgfpathlineto{\pgfqpoint{5.801780in}{0.568910in}}%
\pgfpathclose%
\pgfusepath{stroke}%
\end{pgfscope}%
\begin{pgfscope}%
\pgfpathrectangle{\pgfqpoint{0.847223in}{0.554012in}}{\pgfqpoint{6.200000in}{4.620000in}}%
\pgfusepath{clip}%
\pgfsetbuttcap%
\pgfsetroundjoin%
\pgfsetlinewidth{1.003750pt}%
\definecolor{currentstroke}{rgb}{1.000000,0.000000,0.000000}%
\pgfsetstrokecolor{currentstroke}%
\pgfsetdash{}{0pt}%
\pgfpathmoveto{\pgfqpoint{5.807114in}{0.568441in}}%
\pgfpathcurveto{\pgfqpoint{5.818164in}{0.568441in}}{\pgfqpoint{5.828763in}{0.572831in}}{\pgfqpoint{5.836576in}{0.580645in}}%
\pgfpathcurveto{\pgfqpoint{5.844390in}{0.588458in}}{\pgfqpoint{5.848780in}{0.599057in}}{\pgfqpoint{5.848780in}{0.610107in}}%
\pgfpathcurveto{\pgfqpoint{5.848780in}{0.621158in}}{\pgfqpoint{5.844390in}{0.631757in}}{\pgfqpoint{5.836576in}{0.639570in}}%
\pgfpathcurveto{\pgfqpoint{5.828763in}{0.647384in}}{\pgfqpoint{5.818164in}{0.651774in}}{\pgfqpoint{5.807114in}{0.651774in}}%
\pgfpathcurveto{\pgfqpoint{5.796063in}{0.651774in}}{\pgfqpoint{5.785464in}{0.647384in}}{\pgfqpoint{5.777651in}{0.639570in}}%
\pgfpathcurveto{\pgfqpoint{5.769837in}{0.631757in}}{\pgfqpoint{5.765447in}{0.621158in}}{\pgfqpoint{5.765447in}{0.610107in}}%
\pgfpathcurveto{\pgfqpoint{5.765447in}{0.599057in}}{\pgfqpoint{5.769837in}{0.588458in}}{\pgfqpoint{5.777651in}{0.580645in}}%
\pgfpathcurveto{\pgfqpoint{5.785464in}{0.572831in}}{\pgfqpoint{5.796063in}{0.568441in}}{\pgfqpoint{5.807114in}{0.568441in}}%
\pgfpathlineto{\pgfqpoint{5.807114in}{0.568441in}}%
\pgfpathclose%
\pgfusepath{stroke}%
\end{pgfscope}%
\begin{pgfscope}%
\pgfpathrectangle{\pgfqpoint{0.847223in}{0.554012in}}{\pgfqpoint{6.200000in}{4.620000in}}%
\pgfusepath{clip}%
\pgfsetbuttcap%
\pgfsetroundjoin%
\pgfsetlinewidth{1.003750pt}%
\definecolor{currentstroke}{rgb}{1.000000,0.000000,0.000000}%
\pgfsetstrokecolor{currentstroke}%
\pgfsetdash{}{0pt}%
\pgfpathmoveto{\pgfqpoint{5.812447in}{0.567973in}}%
\pgfpathcurveto{\pgfqpoint{5.823497in}{0.567973in}}{\pgfqpoint{5.834096in}{0.572363in}}{\pgfqpoint{5.841910in}{0.580177in}}%
\pgfpathcurveto{\pgfqpoint{5.849723in}{0.587990in}}{\pgfqpoint{5.854114in}{0.598589in}}{\pgfqpoint{5.854114in}{0.609639in}}%
\pgfpathcurveto{\pgfqpoint{5.854114in}{0.620690in}}{\pgfqpoint{5.849723in}{0.631289in}}{\pgfqpoint{5.841910in}{0.639102in}}%
\pgfpathcurveto{\pgfqpoint{5.834096in}{0.646916in}}{\pgfqpoint{5.823497in}{0.651306in}}{\pgfqpoint{5.812447in}{0.651306in}}%
\pgfpathcurveto{\pgfqpoint{5.801397in}{0.651306in}}{\pgfqpoint{5.790798in}{0.646916in}}{\pgfqpoint{5.782984in}{0.639102in}}%
\pgfpathcurveto{\pgfqpoint{5.775170in}{0.631289in}}{\pgfqpoint{5.770780in}{0.620690in}}{\pgfqpoint{5.770780in}{0.609639in}}%
\pgfpathcurveto{\pgfqpoint{5.770780in}{0.598589in}}{\pgfqpoint{5.775170in}{0.587990in}}{\pgfqpoint{5.782984in}{0.580177in}}%
\pgfpathcurveto{\pgfqpoint{5.790798in}{0.572363in}}{\pgfqpoint{5.801397in}{0.567973in}}{\pgfqpoint{5.812447in}{0.567973in}}%
\pgfpathlineto{\pgfqpoint{5.812447in}{0.567973in}}%
\pgfpathclose%
\pgfusepath{stroke}%
\end{pgfscope}%
\begin{pgfscope}%
\pgfpathrectangle{\pgfqpoint{0.847223in}{0.554012in}}{\pgfqpoint{6.200000in}{4.620000in}}%
\pgfusepath{clip}%
\pgfsetbuttcap%
\pgfsetroundjoin%
\pgfsetlinewidth{1.003750pt}%
\definecolor{currentstroke}{rgb}{1.000000,0.000000,0.000000}%
\pgfsetstrokecolor{currentstroke}%
\pgfsetdash{}{0pt}%
\pgfpathmoveto{\pgfqpoint{5.817780in}{0.567506in}}%
\pgfpathcurveto{\pgfqpoint{5.828830in}{0.567506in}}{\pgfqpoint{5.839429in}{0.571896in}}{\pgfqpoint{5.847243in}{0.579710in}}%
\pgfpathcurveto{\pgfqpoint{5.855056in}{0.587523in}}{\pgfqpoint{5.859447in}{0.598122in}}{\pgfqpoint{5.859447in}{0.609172in}}%
\pgfpathcurveto{\pgfqpoint{5.859447in}{0.620223in}}{\pgfqpoint{5.855056in}{0.630822in}}{\pgfqpoint{5.847243in}{0.638635in}}%
\pgfpathcurveto{\pgfqpoint{5.839429in}{0.646449in}}{\pgfqpoint{5.828830in}{0.650839in}}{\pgfqpoint{5.817780in}{0.650839in}}%
\pgfpathcurveto{\pgfqpoint{5.806730in}{0.650839in}}{\pgfqpoint{5.796131in}{0.646449in}}{\pgfqpoint{5.788317in}{0.638635in}}%
\pgfpathcurveto{\pgfqpoint{5.780504in}{0.630822in}}{\pgfqpoint{5.776113in}{0.620223in}}{\pgfqpoint{5.776113in}{0.609172in}}%
\pgfpathcurveto{\pgfqpoint{5.776113in}{0.598122in}}{\pgfqpoint{5.780504in}{0.587523in}}{\pgfqpoint{5.788317in}{0.579710in}}%
\pgfpathcurveto{\pgfqpoint{5.796131in}{0.571896in}}{\pgfqpoint{5.806730in}{0.567506in}}{\pgfqpoint{5.817780in}{0.567506in}}%
\pgfpathlineto{\pgfqpoint{5.817780in}{0.567506in}}%
\pgfpathclose%
\pgfusepath{stroke}%
\end{pgfscope}%
\begin{pgfscope}%
\pgfpathrectangle{\pgfqpoint{0.847223in}{0.554012in}}{\pgfqpoint{6.200000in}{4.620000in}}%
\pgfusepath{clip}%
\pgfsetbuttcap%
\pgfsetroundjoin%
\pgfsetlinewidth{1.003750pt}%
\definecolor{currentstroke}{rgb}{1.000000,0.000000,0.000000}%
\pgfsetstrokecolor{currentstroke}%
\pgfsetdash{}{0pt}%
\pgfpathmoveto{\pgfqpoint{5.823113in}{0.567040in}}%
\pgfpathcurveto{\pgfqpoint{5.834163in}{0.567040in}}{\pgfqpoint{5.844762in}{0.571430in}}{\pgfqpoint{5.852576in}{0.579244in}}%
\pgfpathcurveto{\pgfqpoint{5.860390in}{0.587057in}}{\pgfqpoint{5.864780in}{0.597656in}}{\pgfqpoint{5.864780in}{0.608706in}}%
\pgfpathcurveto{\pgfqpoint{5.864780in}{0.619756in}}{\pgfqpoint{5.860390in}{0.630355in}}{\pgfqpoint{5.852576in}{0.638169in}}%
\pgfpathcurveto{\pgfqpoint{5.844762in}{0.645983in}}{\pgfqpoint{5.834163in}{0.650373in}}{\pgfqpoint{5.823113in}{0.650373in}}%
\pgfpathcurveto{\pgfqpoint{5.812063in}{0.650373in}}{\pgfqpoint{5.801464in}{0.645983in}}{\pgfqpoint{5.793650in}{0.638169in}}%
\pgfpathcurveto{\pgfqpoint{5.785837in}{0.630355in}}{\pgfqpoint{5.781447in}{0.619756in}}{\pgfqpoint{5.781447in}{0.608706in}}%
\pgfpathcurveto{\pgfqpoint{5.781447in}{0.597656in}}{\pgfqpoint{5.785837in}{0.587057in}}{\pgfqpoint{5.793650in}{0.579244in}}%
\pgfpathcurveto{\pgfqpoint{5.801464in}{0.571430in}}{\pgfqpoint{5.812063in}{0.567040in}}{\pgfqpoint{5.823113in}{0.567040in}}%
\pgfpathlineto{\pgfqpoint{5.823113in}{0.567040in}}%
\pgfpathclose%
\pgfusepath{stroke}%
\end{pgfscope}%
\begin{pgfscope}%
\pgfpathrectangle{\pgfqpoint{0.847223in}{0.554012in}}{\pgfqpoint{6.200000in}{4.620000in}}%
\pgfusepath{clip}%
\pgfsetbuttcap%
\pgfsetroundjoin%
\pgfsetlinewidth{1.003750pt}%
\definecolor{currentstroke}{rgb}{1.000000,0.000000,0.000000}%
\pgfsetstrokecolor{currentstroke}%
\pgfsetdash{}{0pt}%
\pgfpathmoveto{\pgfqpoint{5.828446in}{0.566574in}}%
\pgfpathcurveto{\pgfqpoint{5.839497in}{0.566574in}}{\pgfqpoint{5.850096in}{0.570965in}}{\pgfqpoint{5.857909in}{0.578778in}}%
\pgfpathcurveto{\pgfqpoint{5.865723in}{0.586592in}}{\pgfqpoint{5.870113in}{0.597191in}}{\pgfqpoint{5.870113in}{0.608241in}}%
\pgfpathcurveto{\pgfqpoint{5.870113in}{0.619291in}}{\pgfqpoint{5.865723in}{0.629890in}}{\pgfqpoint{5.857909in}{0.637704in}}%
\pgfpathcurveto{\pgfqpoint{5.850096in}{0.645517in}}{\pgfqpoint{5.839497in}{0.649908in}}{\pgfqpoint{5.828446in}{0.649908in}}%
\pgfpathcurveto{\pgfqpoint{5.817396in}{0.649908in}}{\pgfqpoint{5.806797in}{0.645517in}}{\pgfqpoint{5.798984in}{0.637704in}}%
\pgfpathcurveto{\pgfqpoint{5.791170in}{0.629890in}}{\pgfqpoint{5.786780in}{0.619291in}}{\pgfqpoint{5.786780in}{0.608241in}}%
\pgfpathcurveto{\pgfqpoint{5.786780in}{0.597191in}}{\pgfqpoint{5.791170in}{0.586592in}}{\pgfqpoint{5.798984in}{0.578778in}}%
\pgfpathcurveto{\pgfqpoint{5.806797in}{0.570965in}}{\pgfqpoint{5.817396in}{0.566574in}}{\pgfqpoint{5.828446in}{0.566574in}}%
\pgfpathlineto{\pgfqpoint{5.828446in}{0.566574in}}%
\pgfpathclose%
\pgfusepath{stroke}%
\end{pgfscope}%
\begin{pgfscope}%
\pgfpathrectangle{\pgfqpoint{0.847223in}{0.554012in}}{\pgfqpoint{6.200000in}{4.620000in}}%
\pgfusepath{clip}%
\pgfsetbuttcap%
\pgfsetroundjoin%
\pgfsetlinewidth{1.003750pt}%
\definecolor{currentstroke}{rgb}{1.000000,0.000000,0.000000}%
\pgfsetstrokecolor{currentstroke}%
\pgfsetdash{}{0pt}%
\pgfpathmoveto{\pgfqpoint{5.833780in}{0.566110in}}%
\pgfpathcurveto{\pgfqpoint{5.844830in}{0.566110in}}{\pgfqpoint{5.855429in}{0.570500in}}{\pgfqpoint{5.863242in}{0.578314in}}%
\pgfpathcurveto{\pgfqpoint{5.871056in}{0.586128in}}{\pgfqpoint{5.875446in}{0.596727in}}{\pgfqpoint{5.875446in}{0.607777in}}%
\pgfpathcurveto{\pgfqpoint{5.875446in}{0.618827in}}{\pgfqpoint{5.871056in}{0.629426in}}{\pgfqpoint{5.863242in}{0.637240in}}%
\pgfpathcurveto{\pgfqpoint{5.855429in}{0.645053in}}{\pgfqpoint{5.844830in}{0.649443in}}{\pgfqpoint{5.833780in}{0.649443in}}%
\pgfpathcurveto{\pgfqpoint{5.822730in}{0.649443in}}{\pgfqpoint{5.812131in}{0.645053in}}{\pgfqpoint{5.804317in}{0.637240in}}%
\pgfpathcurveto{\pgfqpoint{5.796503in}{0.629426in}}{\pgfqpoint{5.792113in}{0.618827in}}{\pgfqpoint{5.792113in}{0.607777in}}%
\pgfpathcurveto{\pgfqpoint{5.792113in}{0.596727in}}{\pgfqpoint{5.796503in}{0.586128in}}{\pgfqpoint{5.804317in}{0.578314in}}%
\pgfpathcurveto{\pgfqpoint{5.812131in}{0.570500in}}{\pgfqpoint{5.822730in}{0.566110in}}{\pgfqpoint{5.833780in}{0.566110in}}%
\pgfpathlineto{\pgfqpoint{5.833780in}{0.566110in}}%
\pgfpathclose%
\pgfusepath{stroke}%
\end{pgfscope}%
\begin{pgfscope}%
\pgfpathrectangle{\pgfqpoint{0.847223in}{0.554012in}}{\pgfqpoint{6.200000in}{4.620000in}}%
\pgfusepath{clip}%
\pgfsetbuttcap%
\pgfsetroundjoin%
\pgfsetlinewidth{1.003750pt}%
\definecolor{currentstroke}{rgb}{1.000000,0.000000,0.000000}%
\pgfsetstrokecolor{currentstroke}%
\pgfsetdash{}{0pt}%
\pgfpathmoveto{\pgfqpoint{5.839113in}{0.565647in}}%
\pgfpathcurveto{\pgfqpoint{5.850163in}{0.565647in}}{\pgfqpoint{5.860762in}{0.570037in}}{\pgfqpoint{5.868576in}{0.577851in}}%
\pgfpathcurveto{\pgfqpoint{5.876389in}{0.585664in}}{\pgfqpoint{5.880780in}{0.596263in}}{\pgfqpoint{5.880780in}{0.607313in}}%
\pgfpathcurveto{\pgfqpoint{5.880780in}{0.618363in}}{\pgfqpoint{5.876389in}{0.628963in}}{\pgfqpoint{5.868576in}{0.636776in}}%
\pgfpathcurveto{\pgfqpoint{5.860762in}{0.644590in}}{\pgfqpoint{5.850163in}{0.648980in}}{\pgfqpoint{5.839113in}{0.648980in}}%
\pgfpathcurveto{\pgfqpoint{5.828063in}{0.648980in}}{\pgfqpoint{5.817464in}{0.644590in}}{\pgfqpoint{5.809650in}{0.636776in}}%
\pgfpathcurveto{\pgfqpoint{5.801837in}{0.628963in}}{\pgfqpoint{5.797446in}{0.618363in}}{\pgfqpoint{5.797446in}{0.607313in}}%
\pgfpathcurveto{\pgfqpoint{5.797446in}{0.596263in}}{\pgfqpoint{5.801837in}{0.585664in}}{\pgfqpoint{5.809650in}{0.577851in}}%
\pgfpathcurveto{\pgfqpoint{5.817464in}{0.570037in}}{\pgfqpoint{5.828063in}{0.565647in}}{\pgfqpoint{5.839113in}{0.565647in}}%
\pgfpathlineto{\pgfqpoint{5.839113in}{0.565647in}}%
\pgfpathclose%
\pgfusepath{stroke}%
\end{pgfscope}%
\begin{pgfscope}%
\pgfpathrectangle{\pgfqpoint{0.847223in}{0.554012in}}{\pgfqpoint{6.200000in}{4.620000in}}%
\pgfusepath{clip}%
\pgfsetbuttcap%
\pgfsetroundjoin%
\pgfsetlinewidth{1.003750pt}%
\definecolor{currentstroke}{rgb}{1.000000,0.000000,0.000000}%
\pgfsetstrokecolor{currentstroke}%
\pgfsetdash{}{0pt}%
\pgfpathmoveto{\pgfqpoint{5.844446in}{0.565184in}}%
\pgfpathcurveto{\pgfqpoint{5.855496in}{0.565184in}}{\pgfqpoint{5.866095in}{0.569574in}}{\pgfqpoint{5.873909in}{0.577388in}}%
\pgfpathcurveto{\pgfqpoint{5.881723in}{0.585202in}}{\pgfqpoint{5.886113in}{0.595801in}}{\pgfqpoint{5.886113in}{0.606851in}}%
\pgfpathcurveto{\pgfqpoint{5.886113in}{0.617901in}}{\pgfqpoint{5.881723in}{0.628500in}}{\pgfqpoint{5.873909in}{0.636314in}}%
\pgfpathcurveto{\pgfqpoint{5.866095in}{0.644127in}}{\pgfqpoint{5.855496in}{0.648517in}}{\pgfqpoint{5.844446in}{0.648517in}}%
\pgfpathcurveto{\pgfqpoint{5.833396in}{0.648517in}}{\pgfqpoint{5.822797in}{0.644127in}}{\pgfqpoint{5.814983in}{0.636314in}}%
\pgfpathcurveto{\pgfqpoint{5.807170in}{0.628500in}}{\pgfqpoint{5.802779in}{0.617901in}}{\pgfqpoint{5.802779in}{0.606851in}}%
\pgfpathcurveto{\pgfqpoint{5.802779in}{0.595801in}}{\pgfqpoint{5.807170in}{0.585202in}}{\pgfqpoint{5.814983in}{0.577388in}}%
\pgfpathcurveto{\pgfqpoint{5.822797in}{0.569574in}}{\pgfqpoint{5.833396in}{0.565184in}}{\pgfqpoint{5.844446in}{0.565184in}}%
\pgfpathlineto{\pgfqpoint{5.844446in}{0.565184in}}%
\pgfpathclose%
\pgfusepath{stroke}%
\end{pgfscope}%
\begin{pgfscope}%
\pgfpathrectangle{\pgfqpoint{0.847223in}{0.554012in}}{\pgfqpoint{6.200000in}{4.620000in}}%
\pgfusepath{clip}%
\pgfsetbuttcap%
\pgfsetroundjoin%
\pgfsetlinewidth{1.003750pt}%
\definecolor{currentstroke}{rgb}{1.000000,0.000000,0.000000}%
\pgfsetstrokecolor{currentstroke}%
\pgfsetdash{}{0pt}%
\pgfpathmoveto{\pgfqpoint{5.849779in}{0.564723in}}%
\pgfpathcurveto{\pgfqpoint{5.860829in}{0.564723in}}{\pgfqpoint{5.871429in}{0.569113in}}{\pgfqpoint{5.879242in}{0.576926in}}%
\pgfpathcurveto{\pgfqpoint{5.887056in}{0.584740in}}{\pgfqpoint{5.891446in}{0.595339in}}{\pgfqpoint{5.891446in}{0.606389in}}%
\pgfpathcurveto{\pgfqpoint{5.891446in}{0.617439in}}{\pgfqpoint{5.887056in}{0.628038in}}{\pgfqpoint{5.879242in}{0.635852in}}%
\pgfpathcurveto{\pgfqpoint{5.871429in}{0.643666in}}{\pgfqpoint{5.860829in}{0.648056in}}{\pgfqpoint{5.849779in}{0.648056in}}%
\pgfpathcurveto{\pgfqpoint{5.838729in}{0.648056in}}{\pgfqpoint{5.828130in}{0.643666in}}{\pgfqpoint{5.820317in}{0.635852in}}%
\pgfpathcurveto{\pgfqpoint{5.812503in}{0.628038in}}{\pgfqpoint{5.808113in}{0.617439in}}{\pgfqpoint{5.808113in}{0.606389in}}%
\pgfpathcurveto{\pgfqpoint{5.808113in}{0.595339in}}{\pgfqpoint{5.812503in}{0.584740in}}{\pgfqpoint{5.820317in}{0.576926in}}%
\pgfpathcurveto{\pgfqpoint{5.828130in}{0.569113in}}{\pgfqpoint{5.838729in}{0.564723in}}{\pgfqpoint{5.849779in}{0.564723in}}%
\pgfpathlineto{\pgfqpoint{5.849779in}{0.564723in}}%
\pgfpathclose%
\pgfusepath{stroke}%
\end{pgfscope}%
\begin{pgfscope}%
\pgfpathrectangle{\pgfqpoint{0.847223in}{0.554012in}}{\pgfqpoint{6.200000in}{4.620000in}}%
\pgfusepath{clip}%
\pgfsetbuttcap%
\pgfsetroundjoin%
\pgfsetlinewidth{1.003750pt}%
\definecolor{currentstroke}{rgb}{1.000000,0.000000,0.000000}%
\pgfsetstrokecolor{currentstroke}%
\pgfsetdash{}{0pt}%
\pgfpathmoveto{\pgfqpoint{5.855113in}{0.564262in}}%
\pgfpathcurveto{\pgfqpoint{5.866163in}{0.564262in}}{\pgfqpoint{5.876762in}{0.568652in}}{\pgfqpoint{5.884575in}{0.576466in}}%
\pgfpathcurveto{\pgfqpoint{5.892389in}{0.584279in}}{\pgfqpoint{5.896779in}{0.594878in}}{\pgfqpoint{5.896779in}{0.605928in}}%
\pgfpathcurveto{\pgfqpoint{5.896779in}{0.616979in}}{\pgfqpoint{5.892389in}{0.627578in}}{\pgfqpoint{5.884575in}{0.635391in}}%
\pgfpathcurveto{\pgfqpoint{5.876762in}{0.643205in}}{\pgfqpoint{5.866163in}{0.647595in}}{\pgfqpoint{5.855113in}{0.647595in}}%
\pgfpathcurveto{\pgfqpoint{5.844062in}{0.647595in}}{\pgfqpoint{5.833463in}{0.643205in}}{\pgfqpoint{5.825650in}{0.635391in}}%
\pgfpathcurveto{\pgfqpoint{5.817836in}{0.627578in}}{\pgfqpoint{5.813446in}{0.616979in}}{\pgfqpoint{5.813446in}{0.605928in}}%
\pgfpathcurveto{\pgfqpoint{5.813446in}{0.594878in}}{\pgfqpoint{5.817836in}{0.584279in}}{\pgfqpoint{5.825650in}{0.576466in}}%
\pgfpathcurveto{\pgfqpoint{5.833463in}{0.568652in}}{\pgfqpoint{5.844062in}{0.564262in}}{\pgfqpoint{5.855113in}{0.564262in}}%
\pgfpathlineto{\pgfqpoint{5.855113in}{0.564262in}}%
\pgfpathclose%
\pgfusepath{stroke}%
\end{pgfscope}%
\begin{pgfscope}%
\pgfpathrectangle{\pgfqpoint{0.847223in}{0.554012in}}{\pgfqpoint{6.200000in}{4.620000in}}%
\pgfusepath{clip}%
\pgfsetbuttcap%
\pgfsetroundjoin%
\pgfsetlinewidth{1.003750pt}%
\definecolor{currentstroke}{rgb}{1.000000,0.000000,0.000000}%
\pgfsetstrokecolor{currentstroke}%
\pgfsetdash{}{0pt}%
\pgfpathmoveto{\pgfqpoint{5.860446in}{0.563802in}}%
\pgfpathcurveto{\pgfqpoint{5.871496in}{0.563802in}}{\pgfqpoint{5.882095in}{0.568192in}}{\pgfqpoint{5.889909in}{0.576006in}}%
\pgfpathcurveto{\pgfqpoint{5.897722in}{0.583819in}}{\pgfqpoint{5.902112in}{0.594418in}}{\pgfqpoint{5.902112in}{0.605469in}}%
\pgfpathcurveto{\pgfqpoint{5.902112in}{0.616519in}}{\pgfqpoint{5.897722in}{0.627118in}}{\pgfqpoint{5.889909in}{0.634931in}}%
\pgfpathcurveto{\pgfqpoint{5.882095in}{0.642745in}}{\pgfqpoint{5.871496in}{0.647135in}}{\pgfqpoint{5.860446in}{0.647135in}}%
\pgfpathcurveto{\pgfqpoint{5.849396in}{0.647135in}}{\pgfqpoint{5.838797in}{0.642745in}}{\pgfqpoint{5.830983in}{0.634931in}}%
\pgfpathcurveto{\pgfqpoint{5.823169in}{0.627118in}}{\pgfqpoint{5.818779in}{0.616519in}}{\pgfqpoint{5.818779in}{0.605469in}}%
\pgfpathcurveto{\pgfqpoint{5.818779in}{0.594418in}}{\pgfqpoint{5.823169in}{0.583819in}}{\pgfqpoint{5.830983in}{0.576006in}}%
\pgfpathcurveto{\pgfqpoint{5.838797in}{0.568192in}}{\pgfqpoint{5.849396in}{0.563802in}}{\pgfqpoint{5.860446in}{0.563802in}}%
\pgfpathlineto{\pgfqpoint{5.860446in}{0.563802in}}%
\pgfpathclose%
\pgfusepath{stroke}%
\end{pgfscope}%
\begin{pgfscope}%
\pgfpathrectangle{\pgfqpoint{0.847223in}{0.554012in}}{\pgfqpoint{6.200000in}{4.620000in}}%
\pgfusepath{clip}%
\pgfsetbuttcap%
\pgfsetroundjoin%
\pgfsetlinewidth{1.003750pt}%
\definecolor{currentstroke}{rgb}{1.000000,0.000000,0.000000}%
\pgfsetstrokecolor{currentstroke}%
\pgfsetdash{}{0pt}%
\pgfpathmoveto{\pgfqpoint{5.865779in}{0.563343in}}%
\pgfpathcurveto{\pgfqpoint{5.876829in}{0.563343in}}{\pgfqpoint{5.887428in}{0.567733in}}{\pgfqpoint{5.895242in}{0.575547in}}%
\pgfpathcurveto{\pgfqpoint{5.903055in}{0.583360in}}{\pgfqpoint{5.907446in}{0.593959in}}{\pgfqpoint{5.907446in}{0.605010in}}%
\pgfpathcurveto{\pgfqpoint{5.907446in}{0.616060in}}{\pgfqpoint{5.903055in}{0.626659in}}{\pgfqpoint{5.895242in}{0.634472in}}%
\pgfpathcurveto{\pgfqpoint{5.887428in}{0.642286in}}{\pgfqpoint{5.876829in}{0.646676in}}{\pgfqpoint{5.865779in}{0.646676in}}%
\pgfpathcurveto{\pgfqpoint{5.854729in}{0.646676in}}{\pgfqpoint{5.844130in}{0.642286in}}{\pgfqpoint{5.836316in}{0.634472in}}%
\pgfpathcurveto{\pgfqpoint{5.828503in}{0.626659in}}{\pgfqpoint{5.824112in}{0.616060in}}{\pgfqpoint{5.824112in}{0.605010in}}%
\pgfpathcurveto{\pgfqpoint{5.824112in}{0.593959in}}{\pgfqpoint{5.828503in}{0.583360in}}{\pgfqpoint{5.836316in}{0.575547in}}%
\pgfpathcurveto{\pgfqpoint{5.844130in}{0.567733in}}{\pgfqpoint{5.854729in}{0.563343in}}{\pgfqpoint{5.865779in}{0.563343in}}%
\pgfpathlineto{\pgfqpoint{5.865779in}{0.563343in}}%
\pgfpathclose%
\pgfusepath{stroke}%
\end{pgfscope}%
\begin{pgfscope}%
\pgfpathrectangle{\pgfqpoint{0.847223in}{0.554012in}}{\pgfqpoint{6.200000in}{4.620000in}}%
\pgfusepath{clip}%
\pgfsetbuttcap%
\pgfsetroundjoin%
\pgfsetlinewidth{1.003750pt}%
\definecolor{currentstroke}{rgb}{1.000000,0.000000,0.000000}%
\pgfsetstrokecolor{currentstroke}%
\pgfsetdash{}{0pt}%
\pgfpathmoveto{\pgfqpoint{5.871112in}{0.562885in}}%
\pgfpathcurveto{\pgfqpoint{5.882162in}{0.562885in}}{\pgfqpoint{5.892761in}{0.567275in}}{\pgfqpoint{5.900575in}{0.575089in}}%
\pgfpathcurveto{\pgfqpoint{5.908389in}{0.582902in}}{\pgfqpoint{5.912779in}{0.593501in}}{\pgfqpoint{5.912779in}{0.604552in}}%
\pgfpathcurveto{\pgfqpoint{5.912779in}{0.615602in}}{\pgfqpoint{5.908389in}{0.626201in}}{\pgfqpoint{5.900575in}{0.634014in}}%
\pgfpathcurveto{\pgfqpoint{5.892761in}{0.641828in}}{\pgfqpoint{5.882162in}{0.646218in}}{\pgfqpoint{5.871112in}{0.646218in}}%
\pgfpathcurveto{\pgfqpoint{5.860062in}{0.646218in}}{\pgfqpoint{5.849463in}{0.641828in}}{\pgfqpoint{5.841649in}{0.634014in}}%
\pgfpathcurveto{\pgfqpoint{5.833836in}{0.626201in}}{\pgfqpoint{5.829446in}{0.615602in}}{\pgfqpoint{5.829446in}{0.604552in}}%
\pgfpathcurveto{\pgfqpoint{5.829446in}{0.593501in}}{\pgfqpoint{5.833836in}{0.582902in}}{\pgfqpoint{5.841649in}{0.575089in}}%
\pgfpathcurveto{\pgfqpoint{5.849463in}{0.567275in}}{\pgfqpoint{5.860062in}{0.562885in}}{\pgfqpoint{5.871112in}{0.562885in}}%
\pgfpathlineto{\pgfqpoint{5.871112in}{0.562885in}}%
\pgfpathclose%
\pgfusepath{stroke}%
\end{pgfscope}%
\begin{pgfscope}%
\pgfpathrectangle{\pgfqpoint{0.847223in}{0.554012in}}{\pgfqpoint{6.200000in}{4.620000in}}%
\pgfusepath{clip}%
\pgfsetbuttcap%
\pgfsetroundjoin%
\pgfsetlinewidth{1.003750pt}%
\definecolor{currentstroke}{rgb}{1.000000,0.000000,0.000000}%
\pgfsetstrokecolor{currentstroke}%
\pgfsetdash{}{0pt}%
\pgfpathmoveto{\pgfqpoint{5.876445in}{0.562428in}}%
\pgfpathcurveto{\pgfqpoint{5.887496in}{0.562428in}}{\pgfqpoint{5.898095in}{0.566818in}}{\pgfqpoint{5.905908in}{0.574632in}}%
\pgfpathcurveto{\pgfqpoint{5.913722in}{0.582445in}}{\pgfqpoint{5.918112in}{0.593044in}}{\pgfqpoint{5.918112in}{0.604094in}}%
\pgfpathcurveto{\pgfqpoint{5.918112in}{0.615144in}}{\pgfqpoint{5.913722in}{0.625743in}}{\pgfqpoint{5.905908in}{0.633557in}}%
\pgfpathcurveto{\pgfqpoint{5.898095in}{0.641371in}}{\pgfqpoint{5.887496in}{0.645761in}}{\pgfqpoint{5.876445in}{0.645761in}}%
\pgfpathcurveto{\pgfqpoint{5.865395in}{0.645761in}}{\pgfqpoint{5.854796in}{0.641371in}}{\pgfqpoint{5.846983in}{0.633557in}}%
\pgfpathcurveto{\pgfqpoint{5.839169in}{0.625743in}}{\pgfqpoint{5.834779in}{0.615144in}}{\pgfqpoint{5.834779in}{0.604094in}}%
\pgfpathcurveto{\pgfqpoint{5.834779in}{0.593044in}}{\pgfqpoint{5.839169in}{0.582445in}}{\pgfqpoint{5.846983in}{0.574632in}}%
\pgfpathcurveto{\pgfqpoint{5.854796in}{0.566818in}}{\pgfqpoint{5.865395in}{0.562428in}}{\pgfqpoint{5.876445in}{0.562428in}}%
\pgfpathlineto{\pgfqpoint{5.876445in}{0.562428in}}%
\pgfpathclose%
\pgfusepath{stroke}%
\end{pgfscope}%
\begin{pgfscope}%
\pgfpathrectangle{\pgfqpoint{0.847223in}{0.554012in}}{\pgfqpoint{6.200000in}{4.620000in}}%
\pgfusepath{clip}%
\pgfsetbuttcap%
\pgfsetroundjoin%
\pgfsetlinewidth{1.003750pt}%
\definecolor{currentstroke}{rgb}{1.000000,0.000000,0.000000}%
\pgfsetstrokecolor{currentstroke}%
\pgfsetdash{}{0pt}%
\pgfpathmoveto{\pgfqpoint{5.881779in}{0.561971in}}%
\pgfpathcurveto{\pgfqpoint{5.892829in}{0.561971in}}{\pgfqpoint{5.903428in}{0.566362in}}{\pgfqpoint{5.911241in}{0.574175in}}%
\pgfpathcurveto{\pgfqpoint{5.919055in}{0.581989in}}{\pgfqpoint{5.923445in}{0.592588in}}{\pgfqpoint{5.923445in}{0.603638in}}%
\pgfpathcurveto{\pgfqpoint{5.923445in}{0.614688in}}{\pgfqpoint{5.919055in}{0.625287in}}{\pgfqpoint{5.911241in}{0.633101in}}%
\pgfpathcurveto{\pgfqpoint{5.903428in}{0.640914in}}{\pgfqpoint{5.892829in}{0.645305in}}{\pgfqpoint{5.881779in}{0.645305in}}%
\pgfpathcurveto{\pgfqpoint{5.870729in}{0.645305in}}{\pgfqpoint{5.860129in}{0.640914in}}{\pgfqpoint{5.852316in}{0.633101in}}%
\pgfpathcurveto{\pgfqpoint{5.844502in}{0.625287in}}{\pgfqpoint{5.840112in}{0.614688in}}{\pgfqpoint{5.840112in}{0.603638in}}%
\pgfpathcurveto{\pgfqpoint{5.840112in}{0.592588in}}{\pgfqpoint{5.844502in}{0.581989in}}{\pgfqpoint{5.852316in}{0.574175in}}%
\pgfpathcurveto{\pgfqpoint{5.860129in}{0.566362in}}{\pgfqpoint{5.870729in}{0.561971in}}{\pgfqpoint{5.881779in}{0.561971in}}%
\pgfpathlineto{\pgfqpoint{5.881779in}{0.561971in}}%
\pgfpathclose%
\pgfusepath{stroke}%
\end{pgfscope}%
\begin{pgfscope}%
\pgfpathrectangle{\pgfqpoint{0.847223in}{0.554012in}}{\pgfqpoint{6.200000in}{4.620000in}}%
\pgfusepath{clip}%
\pgfsetbuttcap%
\pgfsetroundjoin%
\pgfsetlinewidth{1.003750pt}%
\definecolor{currentstroke}{rgb}{1.000000,0.000000,0.000000}%
\pgfsetstrokecolor{currentstroke}%
\pgfsetdash{}{0pt}%
\pgfpathmoveto{\pgfqpoint{5.887112in}{0.561516in}}%
\pgfpathcurveto{\pgfqpoint{5.898162in}{0.561516in}}{\pgfqpoint{5.908761in}{0.565906in}}{\pgfqpoint{5.916575in}{0.573720in}}%
\pgfpathcurveto{\pgfqpoint{5.924388in}{0.581533in}}{\pgfqpoint{5.928779in}{0.592132in}}{\pgfqpoint{5.928779in}{0.603183in}}%
\pgfpathcurveto{\pgfqpoint{5.928779in}{0.614233in}}{\pgfqpoint{5.924388in}{0.624832in}}{\pgfqpoint{5.916575in}{0.632645in}}%
\pgfpathcurveto{\pgfqpoint{5.908761in}{0.640459in}}{\pgfqpoint{5.898162in}{0.644849in}}{\pgfqpoint{5.887112in}{0.644849in}}%
\pgfpathcurveto{\pgfqpoint{5.876062in}{0.644849in}}{\pgfqpoint{5.865463in}{0.640459in}}{\pgfqpoint{5.857649in}{0.632645in}}%
\pgfpathcurveto{\pgfqpoint{5.849835in}{0.624832in}}{\pgfqpoint{5.845445in}{0.614233in}}{\pgfqpoint{5.845445in}{0.603183in}}%
\pgfpathcurveto{\pgfqpoint{5.845445in}{0.592132in}}{\pgfqpoint{5.849835in}{0.581533in}}{\pgfqpoint{5.857649in}{0.573720in}}%
\pgfpathcurveto{\pgfqpoint{5.865463in}{0.565906in}}{\pgfqpoint{5.876062in}{0.561516in}}{\pgfqpoint{5.887112in}{0.561516in}}%
\pgfpathlineto{\pgfqpoint{5.887112in}{0.561516in}}%
\pgfpathclose%
\pgfusepath{stroke}%
\end{pgfscope}%
\begin{pgfscope}%
\pgfpathrectangle{\pgfqpoint{0.847223in}{0.554012in}}{\pgfqpoint{6.200000in}{4.620000in}}%
\pgfusepath{clip}%
\pgfsetbuttcap%
\pgfsetroundjoin%
\pgfsetlinewidth{1.003750pt}%
\definecolor{currentstroke}{rgb}{1.000000,0.000000,0.000000}%
\pgfsetstrokecolor{currentstroke}%
\pgfsetdash{}{0pt}%
\pgfpathmoveto{\pgfqpoint{5.892445in}{0.561061in}}%
\pgfpathcurveto{\pgfqpoint{5.903495in}{0.561061in}}{\pgfqpoint{5.914094in}{0.565452in}}{\pgfqpoint{5.921908in}{0.573265in}}%
\pgfpathcurveto{\pgfqpoint{5.929721in}{0.581079in}}{\pgfqpoint{5.934112in}{0.591678in}}{\pgfqpoint{5.934112in}{0.602728in}}%
\pgfpathcurveto{\pgfqpoint{5.934112in}{0.613778in}}{\pgfqpoint{5.929721in}{0.624377in}}{\pgfqpoint{5.921908in}{0.632191in}}%
\pgfpathcurveto{\pgfqpoint{5.914094in}{0.640004in}}{\pgfqpoint{5.903495in}{0.644395in}}{\pgfqpoint{5.892445in}{0.644395in}}%
\pgfpathcurveto{\pgfqpoint{5.881395in}{0.644395in}}{\pgfqpoint{5.870796in}{0.640004in}}{\pgfqpoint{5.862982in}{0.632191in}}%
\pgfpathcurveto{\pgfqpoint{5.855169in}{0.624377in}}{\pgfqpoint{5.850778in}{0.613778in}}{\pgfqpoint{5.850778in}{0.602728in}}%
\pgfpathcurveto{\pgfqpoint{5.850778in}{0.591678in}}{\pgfqpoint{5.855169in}{0.581079in}}{\pgfqpoint{5.862982in}{0.573265in}}%
\pgfpathcurveto{\pgfqpoint{5.870796in}{0.565452in}}{\pgfqpoint{5.881395in}{0.561061in}}{\pgfqpoint{5.892445in}{0.561061in}}%
\pgfpathlineto{\pgfqpoint{5.892445in}{0.561061in}}%
\pgfpathclose%
\pgfusepath{stroke}%
\end{pgfscope}%
\begin{pgfscope}%
\pgfpathrectangle{\pgfqpoint{0.847223in}{0.554012in}}{\pgfqpoint{6.200000in}{4.620000in}}%
\pgfusepath{clip}%
\pgfsetbuttcap%
\pgfsetroundjoin%
\pgfsetlinewidth{1.003750pt}%
\definecolor{currentstroke}{rgb}{1.000000,0.000000,0.000000}%
\pgfsetstrokecolor{currentstroke}%
\pgfsetdash{}{0pt}%
\pgfpathmoveto{\pgfqpoint{5.897778in}{0.560608in}}%
\pgfpathcurveto{\pgfqpoint{5.908828in}{0.560608in}}{\pgfqpoint{5.919427in}{0.564998in}}{\pgfqpoint{5.927241in}{0.572811in}}%
\pgfpathcurveto{\pgfqpoint{5.935055in}{0.580625in}}{\pgfqpoint{5.939445in}{0.591224in}}{\pgfqpoint{5.939445in}{0.602274in}}%
\pgfpathcurveto{\pgfqpoint{5.939445in}{0.613324in}}{\pgfqpoint{5.935055in}{0.623923in}}{\pgfqpoint{5.927241in}{0.631737in}}%
\pgfpathcurveto{\pgfqpoint{5.919427in}{0.639551in}}{\pgfqpoint{5.908828in}{0.643941in}}{\pgfqpoint{5.897778in}{0.643941in}}%
\pgfpathcurveto{\pgfqpoint{5.886728in}{0.643941in}}{\pgfqpoint{5.876129in}{0.639551in}}{\pgfqpoint{5.868316in}{0.631737in}}%
\pgfpathcurveto{\pgfqpoint{5.860502in}{0.623923in}}{\pgfqpoint{5.856112in}{0.613324in}}{\pgfqpoint{5.856112in}{0.602274in}}%
\pgfpathcurveto{\pgfqpoint{5.856112in}{0.591224in}}{\pgfqpoint{5.860502in}{0.580625in}}{\pgfqpoint{5.868316in}{0.572811in}}%
\pgfpathcurveto{\pgfqpoint{5.876129in}{0.564998in}}{\pgfqpoint{5.886728in}{0.560608in}}{\pgfqpoint{5.897778in}{0.560608in}}%
\pgfpathlineto{\pgfqpoint{5.897778in}{0.560608in}}%
\pgfpathclose%
\pgfusepath{stroke}%
\end{pgfscope}%
\begin{pgfscope}%
\pgfpathrectangle{\pgfqpoint{0.847223in}{0.554012in}}{\pgfqpoint{6.200000in}{4.620000in}}%
\pgfusepath{clip}%
\pgfsetbuttcap%
\pgfsetroundjoin%
\pgfsetlinewidth{1.003750pt}%
\definecolor{currentstroke}{rgb}{1.000000,0.000000,0.000000}%
\pgfsetstrokecolor{currentstroke}%
\pgfsetdash{}{0pt}%
\pgfpathmoveto{\pgfqpoint{5.903112in}{0.560155in}}%
\pgfpathcurveto{\pgfqpoint{5.914162in}{0.560155in}}{\pgfqpoint{5.924761in}{0.564545in}}{\pgfqpoint{5.932574in}{0.572359in}}%
\pgfpathcurveto{\pgfqpoint{5.940388in}{0.580172in}}{\pgfqpoint{5.944778in}{0.590771in}}{\pgfqpoint{5.944778in}{0.601821in}}%
\pgfpathcurveto{\pgfqpoint{5.944778in}{0.612872in}}{\pgfqpoint{5.940388in}{0.623471in}}{\pgfqpoint{5.932574in}{0.631284in}}%
\pgfpathcurveto{\pgfqpoint{5.924761in}{0.639098in}}{\pgfqpoint{5.914162in}{0.643488in}}{\pgfqpoint{5.903112in}{0.643488in}}%
\pgfpathcurveto{\pgfqpoint{5.892061in}{0.643488in}}{\pgfqpoint{5.881462in}{0.639098in}}{\pgfqpoint{5.873649in}{0.631284in}}%
\pgfpathcurveto{\pgfqpoint{5.865835in}{0.623471in}}{\pgfqpoint{5.861445in}{0.612872in}}{\pgfqpoint{5.861445in}{0.601821in}}%
\pgfpathcurveto{\pgfqpoint{5.861445in}{0.590771in}}{\pgfqpoint{5.865835in}{0.580172in}}{\pgfqpoint{5.873649in}{0.572359in}}%
\pgfpathcurveto{\pgfqpoint{5.881462in}{0.564545in}}{\pgfqpoint{5.892061in}{0.560155in}}{\pgfqpoint{5.903112in}{0.560155in}}%
\pgfpathlineto{\pgfqpoint{5.903112in}{0.560155in}}%
\pgfpathclose%
\pgfusepath{stroke}%
\end{pgfscope}%
\begin{pgfscope}%
\pgfpathrectangle{\pgfqpoint{0.847223in}{0.554012in}}{\pgfqpoint{6.200000in}{4.620000in}}%
\pgfusepath{clip}%
\pgfsetbuttcap%
\pgfsetroundjoin%
\pgfsetlinewidth{1.003750pt}%
\definecolor{currentstroke}{rgb}{1.000000,0.000000,0.000000}%
\pgfsetstrokecolor{currentstroke}%
\pgfsetdash{}{0pt}%
\pgfpathmoveto{\pgfqpoint{5.908445in}{0.559703in}}%
\pgfpathcurveto{\pgfqpoint{5.919495in}{0.559703in}}{\pgfqpoint{5.930094in}{0.564093in}}{\pgfqpoint{5.937907in}{0.571907in}}%
\pgfpathcurveto{\pgfqpoint{5.945721in}{0.579720in}}{\pgfqpoint{5.950111in}{0.590319in}}{\pgfqpoint{5.950111in}{0.601369in}}%
\pgfpathcurveto{\pgfqpoint{5.950111in}{0.612420in}}{\pgfqpoint{5.945721in}{0.623019in}}{\pgfqpoint{5.937907in}{0.630832in}}%
\pgfpathcurveto{\pgfqpoint{5.930094in}{0.638646in}}{\pgfqpoint{5.919495in}{0.643036in}}{\pgfqpoint{5.908445in}{0.643036in}}%
\pgfpathcurveto{\pgfqpoint{5.897395in}{0.643036in}}{\pgfqpoint{5.886796in}{0.638646in}}{\pgfqpoint{5.878982in}{0.630832in}}%
\pgfpathcurveto{\pgfqpoint{5.871168in}{0.623019in}}{\pgfqpoint{5.866778in}{0.612420in}}{\pgfqpoint{5.866778in}{0.601369in}}%
\pgfpathcurveto{\pgfqpoint{5.866778in}{0.590319in}}{\pgfqpoint{5.871168in}{0.579720in}}{\pgfqpoint{5.878982in}{0.571907in}}%
\pgfpathcurveto{\pgfqpoint{5.886796in}{0.564093in}}{\pgfqpoint{5.897395in}{0.559703in}}{\pgfqpoint{5.908445in}{0.559703in}}%
\pgfpathlineto{\pgfqpoint{5.908445in}{0.559703in}}%
\pgfpathclose%
\pgfusepath{stroke}%
\end{pgfscope}%
\begin{pgfscope}%
\pgfpathrectangle{\pgfqpoint{0.847223in}{0.554012in}}{\pgfqpoint{6.200000in}{4.620000in}}%
\pgfusepath{clip}%
\pgfsetbuttcap%
\pgfsetroundjoin%
\pgfsetlinewidth{1.003750pt}%
\definecolor{currentstroke}{rgb}{1.000000,0.000000,0.000000}%
\pgfsetstrokecolor{currentstroke}%
\pgfsetdash{}{0pt}%
\pgfpathmoveto{\pgfqpoint{5.913778in}{0.559252in}}%
\pgfpathcurveto{\pgfqpoint{5.924828in}{0.559252in}}{\pgfqpoint{5.935427in}{0.563642in}}{\pgfqpoint{5.943241in}{0.571456in}}%
\pgfpathcurveto{\pgfqpoint{5.951054in}{0.579269in}}{\pgfqpoint{5.955445in}{0.589868in}}{\pgfqpoint{5.955445in}{0.600918in}}%
\pgfpathcurveto{\pgfqpoint{5.955445in}{0.611968in}}{\pgfqpoint{5.951054in}{0.622567in}}{\pgfqpoint{5.943241in}{0.630381in}}%
\pgfpathcurveto{\pgfqpoint{5.935427in}{0.638195in}}{\pgfqpoint{5.924828in}{0.642585in}}{\pgfqpoint{5.913778in}{0.642585in}}%
\pgfpathcurveto{\pgfqpoint{5.902728in}{0.642585in}}{\pgfqpoint{5.892129in}{0.638195in}}{\pgfqpoint{5.884315in}{0.630381in}}%
\pgfpathcurveto{\pgfqpoint{5.876502in}{0.622567in}}{\pgfqpoint{5.872111in}{0.611968in}}{\pgfqpoint{5.872111in}{0.600918in}}%
\pgfpathcurveto{\pgfqpoint{5.872111in}{0.589868in}}{\pgfqpoint{5.876502in}{0.579269in}}{\pgfqpoint{5.884315in}{0.571456in}}%
\pgfpathcurveto{\pgfqpoint{5.892129in}{0.563642in}}{\pgfqpoint{5.902728in}{0.559252in}}{\pgfqpoint{5.913778in}{0.559252in}}%
\pgfpathlineto{\pgfqpoint{5.913778in}{0.559252in}}%
\pgfpathclose%
\pgfusepath{stroke}%
\end{pgfscope}%
\begin{pgfscope}%
\pgfpathrectangle{\pgfqpoint{0.847223in}{0.554012in}}{\pgfqpoint{6.200000in}{4.620000in}}%
\pgfusepath{clip}%
\pgfsetbuttcap%
\pgfsetroundjoin%
\pgfsetlinewidth{1.003750pt}%
\definecolor{currentstroke}{rgb}{1.000000,0.000000,0.000000}%
\pgfsetstrokecolor{currentstroke}%
\pgfsetdash{}{0pt}%
\pgfpathmoveto{\pgfqpoint{5.919111in}{0.558801in}}%
\pgfpathcurveto{\pgfqpoint{5.930161in}{0.558801in}}{\pgfqpoint{5.940760in}{0.563192in}}{\pgfqpoint{5.948574in}{0.571005in}}%
\pgfpathcurveto{\pgfqpoint{5.956388in}{0.578819in}}{\pgfqpoint{5.960778in}{0.589418in}}{\pgfqpoint{5.960778in}{0.600468in}}%
\pgfpathcurveto{\pgfqpoint{5.960778in}{0.611518in}}{\pgfqpoint{5.956388in}{0.622117in}}{\pgfqpoint{5.948574in}{0.629931in}}%
\pgfpathcurveto{\pgfqpoint{5.940760in}{0.637744in}}{\pgfqpoint{5.930161in}{0.642135in}}{\pgfqpoint{5.919111in}{0.642135in}}%
\pgfpathcurveto{\pgfqpoint{5.908061in}{0.642135in}}{\pgfqpoint{5.897462in}{0.637744in}}{\pgfqpoint{5.889648in}{0.629931in}}%
\pgfpathcurveto{\pgfqpoint{5.881835in}{0.622117in}}{\pgfqpoint{5.877444in}{0.611518in}}{\pgfqpoint{5.877444in}{0.600468in}}%
\pgfpathcurveto{\pgfqpoint{5.877444in}{0.589418in}}{\pgfqpoint{5.881835in}{0.578819in}}{\pgfqpoint{5.889648in}{0.571005in}}%
\pgfpathcurveto{\pgfqpoint{5.897462in}{0.563192in}}{\pgfqpoint{5.908061in}{0.558801in}}{\pgfqpoint{5.919111in}{0.558801in}}%
\pgfpathlineto{\pgfqpoint{5.919111in}{0.558801in}}%
\pgfpathclose%
\pgfusepath{stroke}%
\end{pgfscope}%
\begin{pgfscope}%
\pgfpathrectangle{\pgfqpoint{0.847223in}{0.554012in}}{\pgfqpoint{6.200000in}{4.620000in}}%
\pgfusepath{clip}%
\pgfsetbuttcap%
\pgfsetroundjoin%
\pgfsetlinewidth{1.003750pt}%
\definecolor{currentstroke}{rgb}{1.000000,0.000000,0.000000}%
\pgfsetstrokecolor{currentstroke}%
\pgfsetdash{}{0pt}%
\pgfpathmoveto{\pgfqpoint{5.924444in}{0.558352in}}%
\pgfpathcurveto{\pgfqpoint{5.935494in}{0.558352in}}{\pgfqpoint{5.946094in}{0.562742in}}{\pgfqpoint{5.953907in}{0.570556in}}%
\pgfpathcurveto{\pgfqpoint{5.961721in}{0.578369in}}{\pgfqpoint{5.966111in}{0.588969in}}{\pgfqpoint{5.966111in}{0.600019in}}%
\pgfpathcurveto{\pgfqpoint{5.966111in}{0.611069in}}{\pgfqpoint{5.961721in}{0.621668in}}{\pgfqpoint{5.953907in}{0.629481in}}%
\pgfpathcurveto{\pgfqpoint{5.946094in}{0.637295in}}{\pgfqpoint{5.935494in}{0.641685in}}{\pgfqpoint{5.924444in}{0.641685in}}%
\pgfpathcurveto{\pgfqpoint{5.913394in}{0.641685in}}{\pgfqpoint{5.902795in}{0.637295in}}{\pgfqpoint{5.894982in}{0.629481in}}%
\pgfpathcurveto{\pgfqpoint{5.887168in}{0.621668in}}{\pgfqpoint{5.882778in}{0.611069in}}{\pgfqpoint{5.882778in}{0.600019in}}%
\pgfpathcurveto{\pgfqpoint{5.882778in}{0.588969in}}{\pgfqpoint{5.887168in}{0.578369in}}{\pgfqpoint{5.894982in}{0.570556in}}%
\pgfpathcurveto{\pgfqpoint{5.902795in}{0.562742in}}{\pgfqpoint{5.913394in}{0.558352in}}{\pgfqpoint{5.924444in}{0.558352in}}%
\pgfpathlineto{\pgfqpoint{5.924444in}{0.558352in}}%
\pgfpathclose%
\pgfusepath{stroke}%
\end{pgfscope}%
\begin{pgfscope}%
\pgfpathrectangle{\pgfqpoint{0.847223in}{0.554012in}}{\pgfqpoint{6.200000in}{4.620000in}}%
\pgfusepath{clip}%
\pgfsetbuttcap%
\pgfsetroundjoin%
\pgfsetlinewidth{1.003750pt}%
\definecolor{currentstroke}{rgb}{1.000000,0.000000,0.000000}%
\pgfsetstrokecolor{currentstroke}%
\pgfsetdash{}{0pt}%
\pgfpathmoveto{\pgfqpoint{5.929778in}{0.557903in}}%
\pgfpathcurveto{\pgfqpoint{5.940828in}{0.557903in}}{\pgfqpoint{5.951427in}{0.562294in}}{\pgfqpoint{5.959240in}{0.570107in}}%
\pgfpathcurveto{\pgfqpoint{5.967054in}{0.577921in}}{\pgfqpoint{5.971444in}{0.588520in}}{\pgfqpoint{5.971444in}{0.599570in}}%
\pgfpathcurveto{\pgfqpoint{5.971444in}{0.610620in}}{\pgfqpoint{5.967054in}{0.621219in}}{\pgfqpoint{5.959240in}{0.629033in}}%
\pgfpathcurveto{\pgfqpoint{5.951427in}{0.636847in}}{\pgfqpoint{5.940828in}{0.641237in}}{\pgfqpoint{5.929778in}{0.641237in}}%
\pgfpathcurveto{\pgfqpoint{5.918727in}{0.641237in}}{\pgfqpoint{5.908128in}{0.636847in}}{\pgfqpoint{5.900315in}{0.629033in}}%
\pgfpathcurveto{\pgfqpoint{5.892501in}{0.621219in}}{\pgfqpoint{5.888111in}{0.610620in}}{\pgfqpoint{5.888111in}{0.599570in}}%
\pgfpathcurveto{\pgfqpoint{5.888111in}{0.588520in}}{\pgfqpoint{5.892501in}{0.577921in}}{\pgfqpoint{5.900315in}{0.570107in}}%
\pgfpathcurveto{\pgfqpoint{5.908128in}{0.562294in}}{\pgfqpoint{5.918727in}{0.557903in}}{\pgfqpoint{5.929778in}{0.557903in}}%
\pgfpathlineto{\pgfqpoint{5.929778in}{0.557903in}}%
\pgfpathclose%
\pgfusepath{stroke}%
\end{pgfscope}%
\begin{pgfscope}%
\pgfpathrectangle{\pgfqpoint{0.847223in}{0.554012in}}{\pgfqpoint{6.200000in}{4.620000in}}%
\pgfusepath{clip}%
\pgfsetbuttcap%
\pgfsetroundjoin%
\pgfsetlinewidth{1.003750pt}%
\definecolor{currentstroke}{rgb}{1.000000,0.000000,0.000000}%
\pgfsetstrokecolor{currentstroke}%
\pgfsetdash{}{0pt}%
\pgfpathmoveto{\pgfqpoint{5.935111in}{0.557456in}}%
\pgfpathcurveto{\pgfqpoint{5.946161in}{0.557456in}}{\pgfqpoint{5.956760in}{0.561846in}}{\pgfqpoint{5.964574in}{0.569660in}}%
\pgfpathcurveto{\pgfqpoint{5.972387in}{0.577473in}}{\pgfqpoint{5.976777in}{0.588072in}}{\pgfqpoint{5.976777in}{0.599122in}}%
\pgfpathcurveto{\pgfqpoint{5.976777in}{0.610173in}}{\pgfqpoint{5.972387in}{0.620772in}}{\pgfqpoint{5.964574in}{0.628585in}}%
\pgfpathcurveto{\pgfqpoint{5.956760in}{0.636399in}}{\pgfqpoint{5.946161in}{0.640789in}}{\pgfqpoint{5.935111in}{0.640789in}}%
\pgfpathcurveto{\pgfqpoint{5.924061in}{0.640789in}}{\pgfqpoint{5.913462in}{0.636399in}}{\pgfqpoint{5.905648in}{0.628585in}}%
\pgfpathcurveto{\pgfqpoint{5.897834in}{0.620772in}}{\pgfqpoint{5.893444in}{0.610173in}}{\pgfqpoint{5.893444in}{0.599122in}}%
\pgfpathcurveto{\pgfqpoint{5.893444in}{0.588072in}}{\pgfqpoint{5.897834in}{0.577473in}}{\pgfqpoint{5.905648in}{0.569660in}}%
\pgfpathcurveto{\pgfqpoint{5.913462in}{0.561846in}}{\pgfqpoint{5.924061in}{0.557456in}}{\pgfqpoint{5.935111in}{0.557456in}}%
\pgfpathlineto{\pgfqpoint{5.935111in}{0.557456in}}%
\pgfpathclose%
\pgfusepath{stroke}%
\end{pgfscope}%
\begin{pgfscope}%
\pgfpathrectangle{\pgfqpoint{0.847223in}{0.554012in}}{\pgfqpoint{6.200000in}{4.620000in}}%
\pgfusepath{clip}%
\pgfsetbuttcap%
\pgfsetroundjoin%
\pgfsetlinewidth{1.003750pt}%
\definecolor{currentstroke}{rgb}{1.000000,0.000000,0.000000}%
\pgfsetstrokecolor{currentstroke}%
\pgfsetdash{}{0pt}%
\pgfpathmoveto{\pgfqpoint{5.940444in}{0.557009in}}%
\pgfpathcurveto{\pgfqpoint{5.951494in}{0.557009in}}{\pgfqpoint{5.962093in}{0.561399in}}{\pgfqpoint{5.969907in}{0.569213in}}%
\pgfpathcurveto{\pgfqpoint{5.977720in}{0.577026in}}{\pgfqpoint{5.982111in}{0.587625in}}{\pgfqpoint{5.982111in}{0.598676in}}%
\pgfpathcurveto{\pgfqpoint{5.982111in}{0.609726in}}{\pgfqpoint{5.977720in}{0.620325in}}{\pgfqpoint{5.969907in}{0.628138in}}%
\pgfpathcurveto{\pgfqpoint{5.962093in}{0.635952in}}{\pgfqpoint{5.951494in}{0.640342in}}{\pgfqpoint{5.940444in}{0.640342in}}%
\pgfpathcurveto{\pgfqpoint{5.929394in}{0.640342in}}{\pgfqpoint{5.918795in}{0.635952in}}{\pgfqpoint{5.910981in}{0.628138in}}%
\pgfpathcurveto{\pgfqpoint{5.903168in}{0.620325in}}{\pgfqpoint{5.898777in}{0.609726in}}{\pgfqpoint{5.898777in}{0.598676in}}%
\pgfpathcurveto{\pgfqpoint{5.898777in}{0.587625in}}{\pgfqpoint{5.903168in}{0.577026in}}{\pgfqpoint{5.910981in}{0.569213in}}%
\pgfpathcurveto{\pgfqpoint{5.918795in}{0.561399in}}{\pgfqpoint{5.929394in}{0.557009in}}{\pgfqpoint{5.940444in}{0.557009in}}%
\pgfpathlineto{\pgfqpoint{5.940444in}{0.557009in}}%
\pgfpathclose%
\pgfusepath{stroke}%
\end{pgfscope}%
\begin{pgfscope}%
\pgfpathrectangle{\pgfqpoint{0.847223in}{0.554012in}}{\pgfqpoint{6.200000in}{4.620000in}}%
\pgfusepath{clip}%
\pgfsetbuttcap%
\pgfsetroundjoin%
\pgfsetlinewidth{1.003750pt}%
\definecolor{currentstroke}{rgb}{1.000000,0.000000,0.000000}%
\pgfsetstrokecolor{currentstroke}%
\pgfsetdash{}{0pt}%
\pgfpathmoveto{\pgfqpoint{5.945777in}{0.556563in}}%
\pgfpathcurveto{\pgfqpoint{5.956827in}{0.556563in}}{\pgfqpoint{5.967426in}{0.560953in}}{\pgfqpoint{5.975240in}{0.568767in}}%
\pgfpathcurveto{\pgfqpoint{5.983054in}{0.576580in}}{\pgfqpoint{5.987444in}{0.587179in}}{\pgfqpoint{5.987444in}{0.598230in}}%
\pgfpathcurveto{\pgfqpoint{5.987444in}{0.609280in}}{\pgfqpoint{5.983054in}{0.619879in}}{\pgfqpoint{5.975240in}{0.627692in}}%
\pgfpathcurveto{\pgfqpoint{5.967426in}{0.635506in}}{\pgfqpoint{5.956827in}{0.639896in}}{\pgfqpoint{5.945777in}{0.639896in}}%
\pgfpathcurveto{\pgfqpoint{5.934727in}{0.639896in}}{\pgfqpoint{5.924128in}{0.635506in}}{\pgfqpoint{5.916314in}{0.627692in}}%
\pgfpathcurveto{\pgfqpoint{5.908501in}{0.619879in}}{\pgfqpoint{5.904111in}{0.609280in}}{\pgfqpoint{5.904111in}{0.598230in}}%
\pgfpathcurveto{\pgfqpoint{5.904111in}{0.587179in}}{\pgfqpoint{5.908501in}{0.576580in}}{\pgfqpoint{5.916314in}{0.568767in}}%
\pgfpathcurveto{\pgfqpoint{5.924128in}{0.560953in}}{\pgfqpoint{5.934727in}{0.556563in}}{\pgfqpoint{5.945777in}{0.556563in}}%
\pgfpathlineto{\pgfqpoint{5.945777in}{0.556563in}}%
\pgfpathclose%
\pgfusepath{stroke}%
\end{pgfscope}%
\begin{pgfscope}%
\pgfpathrectangle{\pgfqpoint{0.847223in}{0.554012in}}{\pgfqpoint{6.200000in}{4.620000in}}%
\pgfusepath{clip}%
\pgfsetbuttcap%
\pgfsetroundjoin%
\pgfsetlinewidth{1.003750pt}%
\definecolor{currentstroke}{rgb}{1.000000,0.000000,0.000000}%
\pgfsetstrokecolor{currentstroke}%
\pgfsetdash{}{0pt}%
\pgfpathmoveto{\pgfqpoint{5.951110in}{0.556118in}}%
\pgfpathcurveto{\pgfqpoint{5.962161in}{0.556118in}}{\pgfqpoint{5.972760in}{0.560508in}}{\pgfqpoint{5.980573in}{0.568322in}}%
\pgfpathcurveto{\pgfqpoint{5.988387in}{0.576135in}}{\pgfqpoint{5.992777in}{0.586734in}}{\pgfqpoint{5.992777in}{0.597784in}}%
\pgfpathcurveto{\pgfqpoint{5.992777in}{0.608835in}}{\pgfqpoint{5.988387in}{0.619434in}}{\pgfqpoint{5.980573in}{0.627247in}}%
\pgfpathcurveto{\pgfqpoint{5.972760in}{0.635061in}}{\pgfqpoint{5.962161in}{0.639451in}}{\pgfqpoint{5.951110in}{0.639451in}}%
\pgfpathcurveto{\pgfqpoint{5.940060in}{0.639451in}}{\pgfqpoint{5.929461in}{0.635061in}}{\pgfqpoint{5.921648in}{0.627247in}}%
\pgfpathcurveto{\pgfqpoint{5.913834in}{0.619434in}}{\pgfqpoint{5.909444in}{0.608835in}}{\pgfqpoint{5.909444in}{0.597784in}}%
\pgfpathcurveto{\pgfqpoint{5.909444in}{0.586734in}}{\pgfqpoint{5.913834in}{0.576135in}}{\pgfqpoint{5.921648in}{0.568322in}}%
\pgfpathcurveto{\pgfqpoint{5.929461in}{0.560508in}}{\pgfqpoint{5.940060in}{0.556118in}}{\pgfqpoint{5.951110in}{0.556118in}}%
\pgfpathlineto{\pgfqpoint{5.951110in}{0.556118in}}%
\pgfpathclose%
\pgfusepath{stroke}%
\end{pgfscope}%
\begin{pgfscope}%
\pgfpathrectangle{\pgfqpoint{0.847223in}{0.554012in}}{\pgfqpoint{6.200000in}{4.620000in}}%
\pgfusepath{clip}%
\pgfsetbuttcap%
\pgfsetroundjoin%
\pgfsetlinewidth{1.003750pt}%
\definecolor{currentstroke}{rgb}{1.000000,0.000000,0.000000}%
\pgfsetstrokecolor{currentstroke}%
\pgfsetdash{}{0pt}%
\pgfpathmoveto{\pgfqpoint{5.956444in}{0.555673in}}%
\pgfpathcurveto{\pgfqpoint{5.967494in}{0.555673in}}{\pgfqpoint{5.978093in}{0.560064in}}{\pgfqpoint{5.985906in}{0.567877in}}%
\pgfpathcurveto{\pgfqpoint{5.993720in}{0.575691in}}{\pgfqpoint{5.998110in}{0.586290in}}{\pgfqpoint{5.998110in}{0.597340in}}%
\pgfpathcurveto{\pgfqpoint{5.998110in}{0.608390in}}{\pgfqpoint{5.993720in}{0.618989in}}{\pgfqpoint{5.985906in}{0.626803in}}%
\pgfpathcurveto{\pgfqpoint{5.978093in}{0.634617in}}{\pgfqpoint{5.967494in}{0.639007in}}{\pgfqpoint{5.956444in}{0.639007in}}%
\pgfpathcurveto{\pgfqpoint{5.945394in}{0.639007in}}{\pgfqpoint{5.934794in}{0.634617in}}{\pgfqpoint{5.926981in}{0.626803in}}%
\pgfpathcurveto{\pgfqpoint{5.919167in}{0.618989in}}{\pgfqpoint{5.914777in}{0.608390in}}{\pgfqpoint{5.914777in}{0.597340in}}%
\pgfpathcurveto{\pgfqpoint{5.914777in}{0.586290in}}{\pgfqpoint{5.919167in}{0.575691in}}{\pgfqpoint{5.926981in}{0.567877in}}%
\pgfpathcurveto{\pgfqpoint{5.934794in}{0.560064in}}{\pgfqpoint{5.945394in}{0.555673in}}{\pgfqpoint{5.956444in}{0.555673in}}%
\pgfpathlineto{\pgfqpoint{5.956444in}{0.555673in}}%
\pgfpathclose%
\pgfusepath{stroke}%
\end{pgfscope}%
\begin{pgfscope}%
\pgfpathrectangle{\pgfqpoint{0.847223in}{0.554012in}}{\pgfqpoint{6.200000in}{4.620000in}}%
\pgfusepath{clip}%
\pgfsetbuttcap%
\pgfsetroundjoin%
\pgfsetlinewidth{1.003750pt}%
\definecolor{currentstroke}{rgb}{1.000000,0.000000,0.000000}%
\pgfsetstrokecolor{currentstroke}%
\pgfsetdash{}{0pt}%
\pgfpathmoveto{\pgfqpoint{5.961777in}{0.555230in}}%
\pgfpathcurveto{\pgfqpoint{5.972827in}{0.555230in}}{\pgfqpoint{5.983426in}{0.559620in}}{\pgfqpoint{5.991240in}{0.567434in}}%
\pgfpathcurveto{\pgfqpoint{5.999053in}{0.575248in}}{\pgfqpoint{6.003444in}{0.585847in}}{\pgfqpoint{6.003444in}{0.596897in}}%
\pgfpathcurveto{\pgfqpoint{6.003444in}{0.607947in}}{\pgfqpoint{5.999053in}{0.618546in}}{\pgfqpoint{5.991240in}{0.626359in}}%
\pgfpathcurveto{\pgfqpoint{5.983426in}{0.634173in}}{\pgfqpoint{5.972827in}{0.638563in}}{\pgfqpoint{5.961777in}{0.638563in}}%
\pgfpathcurveto{\pgfqpoint{5.950727in}{0.638563in}}{\pgfqpoint{5.940128in}{0.634173in}}{\pgfqpoint{5.932314in}{0.626359in}}%
\pgfpathcurveto{\pgfqpoint{5.924500in}{0.618546in}}{\pgfqpoint{5.920110in}{0.607947in}}{\pgfqpoint{5.920110in}{0.596897in}}%
\pgfpathcurveto{\pgfqpoint{5.920110in}{0.585847in}}{\pgfqpoint{5.924500in}{0.575248in}}{\pgfqpoint{5.932314in}{0.567434in}}%
\pgfpathcurveto{\pgfqpoint{5.940128in}{0.559620in}}{\pgfqpoint{5.950727in}{0.555230in}}{\pgfqpoint{5.961777in}{0.555230in}}%
\pgfpathlineto{\pgfqpoint{5.961777in}{0.555230in}}%
\pgfpathclose%
\pgfusepath{stroke}%
\end{pgfscope}%
\begin{pgfscope}%
\pgfpathrectangle{\pgfqpoint{0.847223in}{0.554012in}}{\pgfqpoint{6.200000in}{4.620000in}}%
\pgfusepath{clip}%
\pgfsetbuttcap%
\pgfsetroundjoin%
\pgfsetlinewidth{1.003750pt}%
\definecolor{currentstroke}{rgb}{1.000000,0.000000,0.000000}%
\pgfsetstrokecolor{currentstroke}%
\pgfsetdash{}{0pt}%
\pgfpathmoveto{\pgfqpoint{5.967110in}{0.554787in}}%
\pgfpathcurveto{\pgfqpoint{5.978160in}{0.554787in}}{\pgfqpoint{5.988759in}{0.559178in}}{\pgfqpoint{5.996573in}{0.566991in}}%
\pgfpathcurveto{\pgfqpoint{6.004386in}{0.574805in}}{\pgfqpoint{6.008777in}{0.585404in}}{\pgfqpoint{6.008777in}{0.596454in}}%
\pgfpathcurveto{\pgfqpoint{6.008777in}{0.607504in}}{\pgfqpoint{6.004386in}{0.618103in}}{\pgfqpoint{5.996573in}{0.625917in}}%
\pgfpathcurveto{\pgfqpoint{5.988759in}{0.633730in}}{\pgfqpoint{5.978160in}{0.638121in}}{\pgfqpoint{5.967110in}{0.638121in}}%
\pgfpathcurveto{\pgfqpoint{5.956060in}{0.638121in}}{\pgfqpoint{5.945461in}{0.633730in}}{\pgfqpoint{5.937647in}{0.625917in}}%
\pgfpathcurveto{\pgfqpoint{5.929834in}{0.618103in}}{\pgfqpoint{5.925443in}{0.607504in}}{\pgfqpoint{5.925443in}{0.596454in}}%
\pgfpathcurveto{\pgfqpoint{5.925443in}{0.585404in}}{\pgfqpoint{5.929834in}{0.574805in}}{\pgfqpoint{5.937647in}{0.566991in}}%
\pgfpathcurveto{\pgfqpoint{5.945461in}{0.559178in}}{\pgfqpoint{5.956060in}{0.554787in}}{\pgfqpoint{5.967110in}{0.554787in}}%
\pgfpathlineto{\pgfqpoint{5.967110in}{0.554787in}}%
\pgfpathclose%
\pgfusepath{stroke}%
\end{pgfscope}%
\begin{pgfscope}%
\pgfpathrectangle{\pgfqpoint{0.847223in}{0.554012in}}{\pgfqpoint{6.200000in}{4.620000in}}%
\pgfusepath{clip}%
\pgfsetbuttcap%
\pgfsetroundjoin%
\pgfsetlinewidth{1.003750pt}%
\definecolor{currentstroke}{rgb}{1.000000,0.000000,0.000000}%
\pgfsetstrokecolor{currentstroke}%
\pgfsetdash{}{0pt}%
\pgfpathmoveto{\pgfqpoint{5.972443in}{0.554346in}}%
\pgfpathcurveto{\pgfqpoint{5.983493in}{0.554346in}}{\pgfqpoint{5.994092in}{0.558736in}}{\pgfqpoint{6.001906in}{0.566549in}}%
\pgfpathcurveto{\pgfqpoint{6.009720in}{0.574363in}}{\pgfqpoint{6.014110in}{0.584962in}}{\pgfqpoint{6.014110in}{0.596012in}}%
\pgfpathcurveto{\pgfqpoint{6.014110in}{0.607062in}}{\pgfqpoint{6.009720in}{0.617661in}}{\pgfqpoint{6.001906in}{0.625475in}}%
\pgfpathcurveto{\pgfqpoint{5.994092in}{0.633289in}}{\pgfqpoint{5.983493in}{0.637679in}}{\pgfqpoint{5.972443in}{0.637679in}}%
\pgfpathcurveto{\pgfqpoint{5.961393in}{0.637679in}}{\pgfqpoint{5.950794in}{0.633289in}}{\pgfqpoint{5.942981in}{0.625475in}}%
\pgfpathcurveto{\pgfqpoint{5.935167in}{0.617661in}}{\pgfqpoint{5.930777in}{0.607062in}}{\pgfqpoint{5.930777in}{0.596012in}}%
\pgfpathcurveto{\pgfqpoint{5.930777in}{0.584962in}}{\pgfqpoint{5.935167in}{0.574363in}}{\pgfqpoint{5.942981in}{0.566549in}}%
\pgfpathcurveto{\pgfqpoint{5.950794in}{0.558736in}}{\pgfqpoint{5.961393in}{0.554346in}}{\pgfqpoint{5.972443in}{0.554346in}}%
\pgfpathlineto{\pgfqpoint{5.972443in}{0.554346in}}%
\pgfpathclose%
\pgfusepath{stroke}%
\end{pgfscope}%
\begin{pgfscope}%
\pgfpathrectangle{\pgfqpoint{0.847223in}{0.554012in}}{\pgfqpoint{6.200000in}{4.620000in}}%
\pgfusepath{clip}%
\pgfsetbuttcap%
\pgfsetroundjoin%
\pgfsetlinewidth{1.003750pt}%
\definecolor{currentstroke}{rgb}{1.000000,0.000000,0.000000}%
\pgfsetstrokecolor{currentstroke}%
\pgfsetdash{}{0pt}%
\pgfpathmoveto{\pgfqpoint{5.977777in}{0.553905in}}%
\pgfpathcurveto{\pgfqpoint{5.988827in}{0.553905in}}{\pgfqpoint{5.999426in}{0.558295in}}{\pgfqpoint{6.007239in}{0.566109in}}%
\pgfpathcurveto{\pgfqpoint{6.015053in}{0.573922in}}{\pgfqpoint{6.019443in}{0.584521in}}{\pgfqpoint{6.019443in}{0.595571in}}%
\pgfpathcurveto{\pgfqpoint{6.019443in}{0.606621in}}{\pgfqpoint{6.015053in}{0.617220in}}{\pgfqpoint{6.007239in}{0.625034in}}%
\pgfpathcurveto{\pgfqpoint{5.999426in}{0.632848in}}{\pgfqpoint{5.988827in}{0.637238in}}{\pgfqpoint{5.977777in}{0.637238in}}%
\pgfpathcurveto{\pgfqpoint{5.966726in}{0.637238in}}{\pgfqpoint{5.956127in}{0.632848in}}{\pgfqpoint{5.948314in}{0.625034in}}%
\pgfpathcurveto{\pgfqpoint{5.940500in}{0.617220in}}{\pgfqpoint{5.936110in}{0.606621in}}{\pgfqpoint{5.936110in}{0.595571in}}%
\pgfpathcurveto{\pgfqpoint{5.936110in}{0.584521in}}{\pgfqpoint{5.940500in}{0.573922in}}{\pgfqpoint{5.948314in}{0.566109in}}%
\pgfpathcurveto{\pgfqpoint{5.956127in}{0.558295in}}{\pgfqpoint{5.966726in}{0.553905in}}{\pgfqpoint{5.977777in}{0.553905in}}%
\pgfpathlineto{\pgfqpoint{5.977777in}{0.553905in}}%
\pgfpathclose%
\pgfusepath{stroke}%
\end{pgfscope}%
\begin{pgfscope}%
\pgfpathrectangle{\pgfqpoint{0.847223in}{0.554012in}}{\pgfqpoint{6.200000in}{4.620000in}}%
\pgfusepath{clip}%
\pgfsetbuttcap%
\pgfsetroundjoin%
\pgfsetlinewidth{1.003750pt}%
\definecolor{currentstroke}{rgb}{1.000000,0.000000,0.000000}%
\pgfsetstrokecolor{currentstroke}%
\pgfsetdash{}{0pt}%
\pgfpathmoveto{\pgfqpoint{5.983110in}{0.553465in}}%
\pgfpathcurveto{\pgfqpoint{5.994160in}{0.553465in}}{\pgfqpoint{6.004759in}{0.557855in}}{\pgfqpoint{6.012573in}{0.565668in}}%
\pgfpathcurveto{\pgfqpoint{6.020386in}{0.573482in}}{\pgfqpoint{6.024776in}{0.584081in}}{\pgfqpoint{6.024776in}{0.595131in}}%
\pgfpathcurveto{\pgfqpoint{6.024776in}{0.606181in}}{\pgfqpoint{6.020386in}{0.616780in}}{\pgfqpoint{6.012573in}{0.624594in}}%
\pgfpathcurveto{\pgfqpoint{6.004759in}{0.632408in}}{\pgfqpoint{5.994160in}{0.636798in}}{\pgfqpoint{5.983110in}{0.636798in}}%
\pgfpathcurveto{\pgfqpoint{5.972060in}{0.636798in}}{\pgfqpoint{5.961461in}{0.632408in}}{\pgfqpoint{5.953647in}{0.624594in}}%
\pgfpathcurveto{\pgfqpoint{5.945833in}{0.616780in}}{\pgfqpoint{5.941443in}{0.606181in}}{\pgfqpoint{5.941443in}{0.595131in}}%
\pgfpathcurveto{\pgfqpoint{5.941443in}{0.584081in}}{\pgfqpoint{5.945833in}{0.573482in}}{\pgfqpoint{5.953647in}{0.565668in}}%
\pgfpathcurveto{\pgfqpoint{5.961461in}{0.557855in}}{\pgfqpoint{5.972060in}{0.553465in}}{\pgfqpoint{5.983110in}{0.553465in}}%
\pgfpathlineto{\pgfqpoint{5.983110in}{0.553465in}}%
\pgfpathclose%
\pgfusepath{stroke}%
\end{pgfscope}%
\begin{pgfscope}%
\pgfpathrectangle{\pgfqpoint{0.847223in}{0.554012in}}{\pgfqpoint{6.200000in}{4.620000in}}%
\pgfusepath{clip}%
\pgfsetbuttcap%
\pgfsetroundjoin%
\pgfsetlinewidth{1.003750pt}%
\definecolor{currentstroke}{rgb}{1.000000,0.000000,0.000000}%
\pgfsetstrokecolor{currentstroke}%
\pgfsetdash{}{0pt}%
\pgfpathmoveto{\pgfqpoint{5.988443in}{0.553025in}}%
\pgfpathcurveto{\pgfqpoint{5.999493in}{0.553025in}}{\pgfqpoint{6.010092in}{0.557415in}}{\pgfqpoint{6.017906in}{0.565229in}}%
\pgfpathcurveto{\pgfqpoint{6.025719in}{0.573043in}}{\pgfqpoint{6.030110in}{0.583642in}}{\pgfqpoint{6.030110in}{0.594692in}}%
\pgfpathcurveto{\pgfqpoint{6.030110in}{0.605742in}}{\pgfqpoint{6.025719in}{0.616341in}}{\pgfqpoint{6.017906in}{0.624155in}}%
\pgfpathcurveto{\pgfqpoint{6.010092in}{0.631968in}}{\pgfqpoint{5.999493in}{0.636359in}}{\pgfqpoint{5.988443in}{0.636359in}}%
\pgfpathcurveto{\pgfqpoint{5.977393in}{0.636359in}}{\pgfqpoint{5.966794in}{0.631968in}}{\pgfqpoint{5.958980in}{0.624155in}}%
\pgfpathcurveto{\pgfqpoint{5.951167in}{0.616341in}}{\pgfqpoint{5.946776in}{0.605742in}}{\pgfqpoint{5.946776in}{0.594692in}}%
\pgfpathcurveto{\pgfqpoint{5.946776in}{0.583642in}}{\pgfqpoint{5.951167in}{0.573043in}}{\pgfqpoint{5.958980in}{0.565229in}}%
\pgfpathcurveto{\pgfqpoint{5.966794in}{0.557415in}}{\pgfqpoint{5.977393in}{0.553025in}}{\pgfqpoint{5.988443in}{0.553025in}}%
\pgfpathlineto{\pgfqpoint{5.988443in}{0.553025in}}%
\pgfpathclose%
\pgfusepath{stroke}%
\end{pgfscope}%
\begin{pgfscope}%
\pgfpathrectangle{\pgfqpoint{0.847223in}{0.554012in}}{\pgfqpoint{6.200000in}{4.620000in}}%
\pgfusepath{clip}%
\pgfsetbuttcap%
\pgfsetroundjoin%
\pgfsetlinewidth{1.003750pt}%
\definecolor{currentstroke}{rgb}{1.000000,0.000000,0.000000}%
\pgfsetstrokecolor{currentstroke}%
\pgfsetdash{}{0pt}%
\pgfpathmoveto{\pgfqpoint{5.993776in}{0.552587in}}%
\pgfpathcurveto{\pgfqpoint{6.004826in}{0.552587in}}{\pgfqpoint{6.015425in}{0.556977in}}{\pgfqpoint{6.023239in}{0.564791in}}%
\pgfpathcurveto{\pgfqpoint{6.031053in}{0.572604in}}{\pgfqpoint{6.035443in}{0.583203in}}{\pgfqpoint{6.035443in}{0.594253in}}%
\pgfpathcurveto{\pgfqpoint{6.035443in}{0.605304in}}{\pgfqpoint{6.031053in}{0.615903in}}{\pgfqpoint{6.023239in}{0.623716in}}%
\pgfpathcurveto{\pgfqpoint{6.015425in}{0.631530in}}{\pgfqpoint{6.004826in}{0.635920in}}{\pgfqpoint{5.993776in}{0.635920in}}%
\pgfpathcurveto{\pgfqpoint{5.982726in}{0.635920in}}{\pgfqpoint{5.972127in}{0.631530in}}{\pgfqpoint{5.964313in}{0.623716in}}%
\pgfpathcurveto{\pgfqpoint{5.956500in}{0.615903in}}{\pgfqpoint{5.952110in}{0.605304in}}{\pgfqpoint{5.952110in}{0.594253in}}%
\pgfpathcurveto{\pgfqpoint{5.952110in}{0.583203in}}{\pgfqpoint{5.956500in}{0.572604in}}{\pgfqpoint{5.964313in}{0.564791in}}%
\pgfpathcurveto{\pgfqpoint{5.972127in}{0.556977in}}{\pgfqpoint{5.982726in}{0.552587in}}{\pgfqpoint{5.993776in}{0.552587in}}%
\pgfpathlineto{\pgfqpoint{5.993776in}{0.552587in}}%
\pgfpathclose%
\pgfusepath{stroke}%
\end{pgfscope}%
\begin{pgfscope}%
\pgfpathrectangle{\pgfqpoint{0.847223in}{0.554012in}}{\pgfqpoint{6.200000in}{4.620000in}}%
\pgfusepath{clip}%
\pgfsetbuttcap%
\pgfsetroundjoin%
\pgfsetlinewidth{1.003750pt}%
\definecolor{currentstroke}{rgb}{1.000000,0.000000,0.000000}%
\pgfsetstrokecolor{currentstroke}%
\pgfsetdash{}{0pt}%
\pgfpathmoveto{\pgfqpoint{5.999109in}{0.552149in}}%
\pgfpathcurveto{\pgfqpoint{6.010160in}{0.552149in}}{\pgfqpoint{6.020759in}{0.556539in}}{\pgfqpoint{6.028572in}{0.564353in}}%
\pgfpathcurveto{\pgfqpoint{6.036386in}{0.572167in}}{\pgfqpoint{6.040776in}{0.582766in}}{\pgfqpoint{6.040776in}{0.593816in}}%
\pgfpathcurveto{\pgfqpoint{6.040776in}{0.604866in}}{\pgfqpoint{6.036386in}{0.615465in}}{\pgfqpoint{6.028572in}{0.623279in}}%
\pgfpathcurveto{\pgfqpoint{6.020759in}{0.631092in}}{\pgfqpoint{6.010160in}{0.635482in}}{\pgfqpoint{5.999109in}{0.635482in}}%
\pgfpathcurveto{\pgfqpoint{5.988059in}{0.635482in}}{\pgfqpoint{5.977460in}{0.631092in}}{\pgfqpoint{5.969647in}{0.623279in}}%
\pgfpathcurveto{\pgfqpoint{5.961833in}{0.615465in}}{\pgfqpoint{5.957443in}{0.604866in}}{\pgfqpoint{5.957443in}{0.593816in}}%
\pgfpathcurveto{\pgfqpoint{5.957443in}{0.582766in}}{\pgfqpoint{5.961833in}{0.572167in}}{\pgfqpoint{5.969647in}{0.564353in}}%
\pgfpathcurveto{\pgfqpoint{5.977460in}{0.556539in}}{\pgfqpoint{5.988059in}{0.552149in}}{\pgfqpoint{5.999109in}{0.552149in}}%
\pgfusepath{stroke}%
\end{pgfscope}%
\begin{pgfscope}%
\pgfpathrectangle{\pgfqpoint{0.847223in}{0.554012in}}{\pgfqpoint{6.200000in}{4.620000in}}%
\pgfusepath{clip}%
\pgfsetbuttcap%
\pgfsetroundjoin%
\pgfsetlinewidth{1.003750pt}%
\definecolor{currentstroke}{rgb}{1.000000,0.000000,0.000000}%
\pgfsetstrokecolor{currentstroke}%
\pgfsetdash{}{0pt}%
\pgfpathmoveto{\pgfqpoint{6.004443in}{0.551712in}}%
\pgfpathcurveto{\pgfqpoint{6.015493in}{0.551712in}}{\pgfqpoint{6.026092in}{0.556103in}}{\pgfqpoint{6.033905in}{0.563916in}}%
\pgfpathcurveto{\pgfqpoint{6.041719in}{0.571730in}}{\pgfqpoint{6.046109in}{0.582329in}}{\pgfqpoint{6.046109in}{0.593379in}}%
\pgfpathcurveto{\pgfqpoint{6.046109in}{0.604429in}}{\pgfqpoint{6.041719in}{0.615028in}}{\pgfqpoint{6.033905in}{0.622842in}}%
\pgfpathcurveto{\pgfqpoint{6.026092in}{0.630655in}}{\pgfqpoint{6.015493in}{0.635046in}}{\pgfqpoint{6.004443in}{0.635046in}}%
\pgfpathcurveto{\pgfqpoint{5.993392in}{0.635046in}}{\pgfqpoint{5.982793in}{0.630655in}}{\pgfqpoint{5.974980in}{0.622842in}}%
\pgfpathcurveto{\pgfqpoint{5.967166in}{0.615028in}}{\pgfqpoint{5.962776in}{0.604429in}}{\pgfqpoint{5.962776in}{0.593379in}}%
\pgfpathcurveto{\pgfqpoint{5.962776in}{0.582329in}}{\pgfqpoint{5.967166in}{0.571730in}}{\pgfqpoint{5.974980in}{0.563916in}}%
\pgfpathcurveto{\pgfqpoint{5.982793in}{0.556103in}}{\pgfqpoint{5.993392in}{0.551712in}}{\pgfqpoint{6.004443in}{0.551712in}}%
\pgfusepath{stroke}%
\end{pgfscope}%
\begin{pgfscope}%
\pgfpathrectangle{\pgfqpoint{0.847223in}{0.554012in}}{\pgfqpoint{6.200000in}{4.620000in}}%
\pgfusepath{clip}%
\pgfsetbuttcap%
\pgfsetroundjoin%
\pgfsetlinewidth{1.003750pt}%
\definecolor{currentstroke}{rgb}{1.000000,0.000000,0.000000}%
\pgfsetstrokecolor{currentstroke}%
\pgfsetdash{}{0pt}%
\pgfpathmoveto{\pgfqpoint{6.009776in}{0.551276in}}%
\pgfpathcurveto{\pgfqpoint{6.020826in}{0.551276in}}{\pgfqpoint{6.031425in}{0.555667in}}{\pgfqpoint{6.039239in}{0.563480in}}%
\pgfpathcurveto{\pgfqpoint{6.047052in}{0.571294in}}{\pgfqpoint{6.051442in}{0.581893in}}{\pgfqpoint{6.051442in}{0.592943in}}%
\pgfpathcurveto{\pgfqpoint{6.051442in}{0.603993in}}{\pgfqpoint{6.047052in}{0.614592in}}{\pgfqpoint{6.039239in}{0.622406in}}%
\pgfpathcurveto{\pgfqpoint{6.031425in}{0.630219in}}{\pgfqpoint{6.020826in}{0.634610in}}{\pgfqpoint{6.009776in}{0.634610in}}%
\pgfpathcurveto{\pgfqpoint{5.998726in}{0.634610in}}{\pgfqpoint{5.988127in}{0.630219in}}{\pgfqpoint{5.980313in}{0.622406in}}%
\pgfpathcurveto{\pgfqpoint{5.972499in}{0.614592in}}{\pgfqpoint{5.968109in}{0.603993in}}{\pgfqpoint{5.968109in}{0.592943in}}%
\pgfpathcurveto{\pgfqpoint{5.968109in}{0.581893in}}{\pgfqpoint{5.972499in}{0.571294in}}{\pgfqpoint{5.980313in}{0.563480in}}%
\pgfpathcurveto{\pgfqpoint{5.988127in}{0.555667in}}{\pgfqpoint{5.998726in}{0.551276in}}{\pgfqpoint{6.009776in}{0.551276in}}%
\pgfusepath{stroke}%
\end{pgfscope}%
\begin{pgfscope}%
\pgfpathrectangle{\pgfqpoint{0.847223in}{0.554012in}}{\pgfqpoint{6.200000in}{4.620000in}}%
\pgfusepath{clip}%
\pgfsetbuttcap%
\pgfsetroundjoin%
\pgfsetlinewidth{1.003750pt}%
\definecolor{currentstroke}{rgb}{1.000000,0.000000,0.000000}%
\pgfsetstrokecolor{currentstroke}%
\pgfsetdash{}{0pt}%
\pgfpathmoveto{\pgfqpoint{6.015109in}{0.550841in}}%
\pgfpathcurveto{\pgfqpoint{6.026159in}{0.550841in}}{\pgfqpoint{6.036758in}{0.555231in}}{\pgfqpoint{6.044572in}{0.563045in}}%
\pgfpathcurveto{\pgfqpoint{6.052385in}{0.570859in}}{\pgfqpoint{6.056776in}{0.581458in}}{\pgfqpoint{6.056776in}{0.592508in}}%
\pgfpathcurveto{\pgfqpoint{6.056776in}{0.603558in}}{\pgfqpoint{6.052385in}{0.614157in}}{\pgfqpoint{6.044572in}{0.621971in}}%
\pgfpathcurveto{\pgfqpoint{6.036758in}{0.629784in}}{\pgfqpoint{6.026159in}{0.634174in}}{\pgfqpoint{6.015109in}{0.634174in}}%
\pgfpathcurveto{\pgfqpoint{6.004059in}{0.634174in}}{\pgfqpoint{5.993460in}{0.629784in}}{\pgfqpoint{5.985646in}{0.621971in}}%
\pgfpathcurveto{\pgfqpoint{5.977833in}{0.614157in}}{\pgfqpoint{5.973442in}{0.603558in}}{\pgfqpoint{5.973442in}{0.592508in}}%
\pgfpathcurveto{\pgfqpoint{5.973442in}{0.581458in}}{\pgfqpoint{5.977833in}{0.570859in}}{\pgfqpoint{5.985646in}{0.563045in}}%
\pgfpathcurveto{\pgfqpoint{5.993460in}{0.555231in}}{\pgfqpoint{6.004059in}{0.550841in}}{\pgfqpoint{6.015109in}{0.550841in}}%
\pgfusepath{stroke}%
\end{pgfscope}%
\begin{pgfscope}%
\pgfpathrectangle{\pgfqpoint{0.847223in}{0.554012in}}{\pgfqpoint{6.200000in}{4.620000in}}%
\pgfusepath{clip}%
\pgfsetbuttcap%
\pgfsetroundjoin%
\pgfsetlinewidth{1.003750pt}%
\definecolor{currentstroke}{rgb}{1.000000,0.000000,0.000000}%
\pgfsetstrokecolor{currentstroke}%
\pgfsetdash{}{0pt}%
\pgfpathmoveto{\pgfqpoint{6.020442in}{0.550407in}}%
\pgfpathcurveto{\pgfqpoint{6.031492in}{0.550407in}}{\pgfqpoint{6.042091in}{0.554797in}}{\pgfqpoint{6.049905in}{0.562611in}}%
\pgfpathcurveto{\pgfqpoint{6.057719in}{0.570424in}}{\pgfqpoint{6.062109in}{0.581023in}}{\pgfqpoint{6.062109in}{0.592073in}}%
\pgfpathcurveto{\pgfqpoint{6.062109in}{0.603123in}}{\pgfqpoint{6.057719in}{0.613723in}}{\pgfqpoint{6.049905in}{0.621536in}}%
\pgfpathcurveto{\pgfqpoint{6.042091in}{0.629350in}}{\pgfqpoint{6.031492in}{0.633740in}}{\pgfqpoint{6.020442in}{0.633740in}}%
\pgfpathcurveto{\pgfqpoint{6.009392in}{0.633740in}}{\pgfqpoint{5.998793in}{0.629350in}}{\pgfqpoint{5.990979in}{0.621536in}}%
\pgfpathcurveto{\pgfqpoint{5.983166in}{0.613723in}}{\pgfqpoint{5.978776in}{0.603123in}}{\pgfqpoint{5.978776in}{0.592073in}}%
\pgfpathcurveto{\pgfqpoint{5.978776in}{0.581023in}}{\pgfqpoint{5.983166in}{0.570424in}}{\pgfqpoint{5.990979in}{0.562611in}}%
\pgfpathcurveto{\pgfqpoint{5.998793in}{0.554797in}}{\pgfqpoint{6.009392in}{0.550407in}}{\pgfqpoint{6.020442in}{0.550407in}}%
\pgfusepath{stroke}%
\end{pgfscope}%
\begin{pgfscope}%
\pgfpathrectangle{\pgfqpoint{0.847223in}{0.554012in}}{\pgfqpoint{6.200000in}{4.620000in}}%
\pgfusepath{clip}%
\pgfsetbuttcap%
\pgfsetroundjoin%
\pgfsetlinewidth{1.003750pt}%
\definecolor{currentstroke}{rgb}{1.000000,0.000000,0.000000}%
\pgfsetstrokecolor{currentstroke}%
\pgfsetdash{}{0pt}%
\pgfpathmoveto{\pgfqpoint{6.025775in}{0.549973in}}%
\pgfpathcurveto{\pgfqpoint{6.036826in}{0.549973in}}{\pgfqpoint{6.047425in}{0.554363in}}{\pgfqpoint{6.055238in}{0.562177in}}%
\pgfpathcurveto{\pgfqpoint{6.063052in}{0.569991in}}{\pgfqpoint{6.067442in}{0.580590in}}{\pgfqpoint{6.067442in}{0.591640in}}%
\pgfpathcurveto{\pgfqpoint{6.067442in}{0.602690in}}{\pgfqpoint{6.063052in}{0.613289in}}{\pgfqpoint{6.055238in}{0.621103in}}%
\pgfpathcurveto{\pgfqpoint{6.047425in}{0.628916in}}{\pgfqpoint{6.036826in}{0.633306in}}{\pgfqpoint{6.025775in}{0.633306in}}%
\pgfpathcurveto{\pgfqpoint{6.014725in}{0.633306in}}{\pgfqpoint{6.004126in}{0.628916in}}{\pgfqpoint{5.996313in}{0.621103in}}%
\pgfpathcurveto{\pgfqpoint{5.988499in}{0.613289in}}{\pgfqpoint{5.984109in}{0.602690in}}{\pgfqpoint{5.984109in}{0.591640in}}%
\pgfpathcurveto{\pgfqpoint{5.984109in}{0.580590in}}{\pgfqpoint{5.988499in}{0.569991in}}{\pgfqpoint{5.996313in}{0.562177in}}%
\pgfpathcurveto{\pgfqpoint{6.004126in}{0.554363in}}{\pgfqpoint{6.014725in}{0.549973in}}{\pgfqpoint{6.025775in}{0.549973in}}%
\pgfusepath{stroke}%
\end{pgfscope}%
\begin{pgfscope}%
\pgfpathrectangle{\pgfqpoint{0.847223in}{0.554012in}}{\pgfqpoint{6.200000in}{4.620000in}}%
\pgfusepath{clip}%
\pgfsetbuttcap%
\pgfsetroundjoin%
\pgfsetlinewidth{1.003750pt}%
\definecolor{currentstroke}{rgb}{1.000000,0.000000,0.000000}%
\pgfsetstrokecolor{currentstroke}%
\pgfsetdash{}{0pt}%
\pgfpathmoveto{\pgfqpoint{6.031109in}{0.549540in}}%
\pgfpathcurveto{\pgfqpoint{6.042159in}{0.549540in}}{\pgfqpoint{6.052758in}{0.553931in}}{\pgfqpoint{6.060571in}{0.561744in}}%
\pgfpathcurveto{\pgfqpoint{6.068385in}{0.569558in}}{\pgfqpoint{6.072775in}{0.580157in}}{\pgfqpoint{6.072775in}{0.591207in}}%
\pgfpathcurveto{\pgfqpoint{6.072775in}{0.602257in}}{\pgfqpoint{6.068385in}{0.612856in}}{\pgfqpoint{6.060571in}{0.620670in}}%
\pgfpathcurveto{\pgfqpoint{6.052758in}{0.628483in}}{\pgfqpoint{6.042159in}{0.632874in}}{\pgfqpoint{6.031109in}{0.632874in}}%
\pgfpathcurveto{\pgfqpoint{6.020059in}{0.632874in}}{\pgfqpoint{6.009460in}{0.628483in}}{\pgfqpoint{6.001646in}{0.620670in}}%
\pgfpathcurveto{\pgfqpoint{5.993832in}{0.612856in}}{\pgfqpoint{5.989442in}{0.602257in}}{\pgfqpoint{5.989442in}{0.591207in}}%
\pgfpathcurveto{\pgfqpoint{5.989442in}{0.580157in}}{\pgfqpoint{5.993832in}{0.569558in}}{\pgfqpoint{6.001646in}{0.561744in}}%
\pgfpathcurveto{\pgfqpoint{6.009460in}{0.553931in}}{\pgfqpoint{6.020059in}{0.549540in}}{\pgfqpoint{6.031109in}{0.549540in}}%
\pgfusepath{stroke}%
\end{pgfscope}%
\begin{pgfscope}%
\pgfpathrectangle{\pgfqpoint{0.847223in}{0.554012in}}{\pgfqpoint{6.200000in}{4.620000in}}%
\pgfusepath{clip}%
\pgfsetbuttcap%
\pgfsetroundjoin%
\pgfsetlinewidth{1.003750pt}%
\definecolor{currentstroke}{rgb}{1.000000,0.000000,0.000000}%
\pgfsetstrokecolor{currentstroke}%
\pgfsetdash{}{0pt}%
\pgfpathmoveto{\pgfqpoint{6.036442in}{0.549108in}}%
\pgfpathcurveto{\pgfqpoint{6.047492in}{0.549108in}}{\pgfqpoint{6.058091in}{0.553499in}}{\pgfqpoint{6.065905in}{0.561312in}}%
\pgfpathcurveto{\pgfqpoint{6.073718in}{0.569126in}}{\pgfqpoint{6.078109in}{0.579725in}}{\pgfqpoint{6.078109in}{0.590775in}}%
\pgfpathcurveto{\pgfqpoint{6.078109in}{0.601825in}}{\pgfqpoint{6.073718in}{0.612424in}}{\pgfqpoint{6.065905in}{0.620238in}}%
\pgfpathcurveto{\pgfqpoint{6.058091in}{0.628052in}}{\pgfqpoint{6.047492in}{0.632442in}}{\pgfqpoint{6.036442in}{0.632442in}}%
\pgfpathcurveto{\pgfqpoint{6.025392in}{0.632442in}}{\pgfqpoint{6.014793in}{0.628052in}}{\pgfqpoint{6.006979in}{0.620238in}}%
\pgfpathcurveto{\pgfqpoint{5.999165in}{0.612424in}}{\pgfqpoint{5.994775in}{0.601825in}}{\pgfqpoint{5.994775in}{0.590775in}}%
\pgfpathcurveto{\pgfqpoint{5.994775in}{0.579725in}}{\pgfqpoint{5.999165in}{0.569126in}}{\pgfqpoint{6.006979in}{0.561312in}}%
\pgfpathcurveto{\pgfqpoint{6.014793in}{0.553499in}}{\pgfqpoint{6.025392in}{0.549108in}}{\pgfqpoint{6.036442in}{0.549108in}}%
\pgfusepath{stroke}%
\end{pgfscope}%
\begin{pgfscope}%
\pgfpathrectangle{\pgfqpoint{0.847223in}{0.554012in}}{\pgfqpoint{6.200000in}{4.620000in}}%
\pgfusepath{clip}%
\pgfsetbuttcap%
\pgfsetroundjoin%
\pgfsetlinewidth{1.003750pt}%
\definecolor{currentstroke}{rgb}{1.000000,0.000000,0.000000}%
\pgfsetstrokecolor{currentstroke}%
\pgfsetdash{}{0pt}%
\pgfpathmoveto{\pgfqpoint{6.041775in}{0.548677in}}%
\pgfpathcurveto{\pgfqpoint{6.052825in}{0.548677in}}{\pgfqpoint{6.063424in}{0.553068in}}{\pgfqpoint{6.071238in}{0.560881in}}%
\pgfpathcurveto{\pgfqpoint{6.079052in}{0.568695in}}{\pgfqpoint{6.083442in}{0.579294in}}{\pgfqpoint{6.083442in}{0.590344in}}%
\pgfpathcurveto{\pgfqpoint{6.083442in}{0.601394in}}{\pgfqpoint{6.079052in}{0.611993in}}{\pgfqpoint{6.071238in}{0.619807in}}%
\pgfpathcurveto{\pgfqpoint{6.063424in}{0.627620in}}{\pgfqpoint{6.052825in}{0.632011in}}{\pgfqpoint{6.041775in}{0.632011in}}%
\pgfpathcurveto{\pgfqpoint{6.030725in}{0.632011in}}{\pgfqpoint{6.020126in}{0.627620in}}{\pgfqpoint{6.012312in}{0.619807in}}%
\pgfpathcurveto{\pgfqpoint{6.004499in}{0.611993in}}{\pgfqpoint{6.000108in}{0.601394in}}{\pgfqpoint{6.000108in}{0.590344in}}%
\pgfpathcurveto{\pgfqpoint{6.000108in}{0.579294in}}{\pgfqpoint{6.004499in}{0.568695in}}{\pgfqpoint{6.012312in}{0.560881in}}%
\pgfpathcurveto{\pgfqpoint{6.020126in}{0.553068in}}{\pgfqpoint{6.030725in}{0.548677in}}{\pgfqpoint{6.041775in}{0.548677in}}%
\pgfusepath{stroke}%
\end{pgfscope}%
\begin{pgfscope}%
\pgfpathrectangle{\pgfqpoint{0.847223in}{0.554012in}}{\pgfqpoint{6.200000in}{4.620000in}}%
\pgfusepath{clip}%
\pgfsetbuttcap%
\pgfsetroundjoin%
\pgfsetlinewidth{1.003750pt}%
\definecolor{currentstroke}{rgb}{1.000000,0.000000,0.000000}%
\pgfsetstrokecolor{currentstroke}%
\pgfsetdash{}{0pt}%
\pgfpathmoveto{\pgfqpoint{6.047108in}{0.548247in}}%
\pgfpathcurveto{\pgfqpoint{6.058158in}{0.548247in}}{\pgfqpoint{6.068757in}{0.552637in}}{\pgfqpoint{6.076571in}{0.560451in}}%
\pgfpathcurveto{\pgfqpoint{6.084385in}{0.568264in}}{\pgfqpoint{6.088775in}{0.578864in}}{\pgfqpoint{6.088775in}{0.589914in}}%
\pgfpathcurveto{\pgfqpoint{6.088775in}{0.600964in}}{\pgfqpoint{6.084385in}{0.611563in}}{\pgfqpoint{6.076571in}{0.619376in}}%
\pgfpathcurveto{\pgfqpoint{6.068757in}{0.627190in}}{\pgfqpoint{6.058158in}{0.631580in}}{\pgfqpoint{6.047108in}{0.631580in}}%
\pgfpathcurveto{\pgfqpoint{6.036058in}{0.631580in}}{\pgfqpoint{6.025459in}{0.627190in}}{\pgfqpoint{6.017646in}{0.619376in}}%
\pgfpathcurveto{\pgfqpoint{6.009832in}{0.611563in}}{\pgfqpoint{6.005442in}{0.600964in}}{\pgfqpoint{6.005442in}{0.589914in}}%
\pgfpathcurveto{\pgfqpoint{6.005442in}{0.578864in}}{\pgfqpoint{6.009832in}{0.568264in}}{\pgfqpoint{6.017646in}{0.560451in}}%
\pgfpathcurveto{\pgfqpoint{6.025459in}{0.552637in}}{\pgfqpoint{6.036058in}{0.548247in}}{\pgfqpoint{6.047108in}{0.548247in}}%
\pgfusepath{stroke}%
\end{pgfscope}%
\begin{pgfscope}%
\pgfpathrectangle{\pgfqpoint{0.847223in}{0.554012in}}{\pgfqpoint{6.200000in}{4.620000in}}%
\pgfusepath{clip}%
\pgfsetbuttcap%
\pgfsetroundjoin%
\pgfsetlinewidth{1.003750pt}%
\definecolor{currentstroke}{rgb}{1.000000,0.000000,0.000000}%
\pgfsetstrokecolor{currentstroke}%
\pgfsetdash{}{0pt}%
\pgfpathmoveto{\pgfqpoint{6.052442in}{0.547817in}}%
\pgfpathcurveto{\pgfqpoint{6.063492in}{0.547817in}}{\pgfqpoint{6.074091in}{0.552208in}}{\pgfqpoint{6.081904in}{0.560021in}}%
\pgfpathcurveto{\pgfqpoint{6.089718in}{0.567835in}}{\pgfqpoint{6.094108in}{0.578434in}}{\pgfqpoint{6.094108in}{0.589484in}}%
\pgfpathcurveto{\pgfqpoint{6.094108in}{0.600534in}}{\pgfqpoint{6.089718in}{0.611133in}}{\pgfqpoint{6.081904in}{0.618947in}}%
\pgfpathcurveto{\pgfqpoint{6.074091in}{0.626761in}}{\pgfqpoint{6.063492in}{0.631151in}}{\pgfqpoint{6.052442in}{0.631151in}}%
\pgfpathcurveto{\pgfqpoint{6.041391in}{0.631151in}}{\pgfqpoint{6.030792in}{0.626761in}}{\pgfqpoint{6.022979in}{0.618947in}}%
\pgfpathcurveto{\pgfqpoint{6.015165in}{0.611133in}}{\pgfqpoint{6.010775in}{0.600534in}}{\pgfqpoint{6.010775in}{0.589484in}}%
\pgfpathcurveto{\pgfqpoint{6.010775in}{0.578434in}}{\pgfqpoint{6.015165in}{0.567835in}}{\pgfqpoint{6.022979in}{0.560021in}}%
\pgfpathcurveto{\pgfqpoint{6.030792in}{0.552208in}}{\pgfqpoint{6.041391in}{0.547817in}}{\pgfqpoint{6.052442in}{0.547817in}}%
\pgfusepath{stroke}%
\end{pgfscope}%
\begin{pgfscope}%
\pgfpathrectangle{\pgfqpoint{0.847223in}{0.554012in}}{\pgfqpoint{6.200000in}{4.620000in}}%
\pgfusepath{clip}%
\pgfsetbuttcap%
\pgfsetroundjoin%
\pgfsetlinewidth{1.003750pt}%
\definecolor{currentstroke}{rgb}{1.000000,0.000000,0.000000}%
\pgfsetstrokecolor{currentstroke}%
\pgfsetdash{}{0pt}%
\pgfpathmoveto{\pgfqpoint{6.057775in}{0.547389in}}%
\pgfpathcurveto{\pgfqpoint{6.068825in}{0.547389in}}{\pgfqpoint{6.079424in}{0.551779in}}{\pgfqpoint{6.087238in}{0.559593in}}%
\pgfpathcurveto{\pgfqpoint{6.095051in}{0.567406in}}{\pgfqpoint{6.099441in}{0.578005in}}{\pgfqpoint{6.099441in}{0.589055in}}%
\pgfpathcurveto{\pgfqpoint{6.099441in}{0.600106in}}{\pgfqpoint{6.095051in}{0.610705in}}{\pgfqpoint{6.087238in}{0.618518in}}%
\pgfpathcurveto{\pgfqpoint{6.079424in}{0.626332in}}{\pgfqpoint{6.068825in}{0.630722in}}{\pgfqpoint{6.057775in}{0.630722in}}%
\pgfpathcurveto{\pgfqpoint{6.046725in}{0.630722in}}{\pgfqpoint{6.036126in}{0.626332in}}{\pgfqpoint{6.028312in}{0.618518in}}%
\pgfpathcurveto{\pgfqpoint{6.020498in}{0.610705in}}{\pgfqpoint{6.016108in}{0.600106in}}{\pgfqpoint{6.016108in}{0.589055in}}%
\pgfpathcurveto{\pgfqpoint{6.016108in}{0.578005in}}{\pgfqpoint{6.020498in}{0.567406in}}{\pgfqpoint{6.028312in}{0.559593in}}%
\pgfpathcurveto{\pgfqpoint{6.036126in}{0.551779in}}{\pgfqpoint{6.046725in}{0.547389in}}{\pgfqpoint{6.057775in}{0.547389in}}%
\pgfusepath{stroke}%
\end{pgfscope}%
\begin{pgfscope}%
\pgfpathrectangle{\pgfqpoint{0.847223in}{0.554012in}}{\pgfqpoint{6.200000in}{4.620000in}}%
\pgfusepath{clip}%
\pgfsetbuttcap%
\pgfsetroundjoin%
\pgfsetlinewidth{1.003750pt}%
\definecolor{currentstroke}{rgb}{1.000000,0.000000,0.000000}%
\pgfsetstrokecolor{currentstroke}%
\pgfsetdash{}{0pt}%
\pgfpathmoveto{\pgfqpoint{6.063108in}{0.546961in}}%
\pgfpathcurveto{\pgfqpoint{6.074158in}{0.546961in}}{\pgfqpoint{6.084757in}{0.551351in}}{\pgfqpoint{6.092571in}{0.559165in}}%
\pgfpathcurveto{\pgfqpoint{6.100384in}{0.566978in}}{\pgfqpoint{6.104775in}{0.577577in}}{\pgfqpoint{6.104775in}{0.588627in}}%
\pgfpathcurveto{\pgfqpoint{6.104775in}{0.599678in}}{\pgfqpoint{6.100384in}{0.610277in}}{\pgfqpoint{6.092571in}{0.618090in}}%
\pgfpathcurveto{\pgfqpoint{6.084757in}{0.625904in}}{\pgfqpoint{6.074158in}{0.630294in}}{\pgfqpoint{6.063108in}{0.630294in}}%
\pgfpathcurveto{\pgfqpoint{6.052058in}{0.630294in}}{\pgfqpoint{6.041459in}{0.625904in}}{\pgfqpoint{6.033645in}{0.618090in}}%
\pgfpathcurveto{\pgfqpoint{6.025832in}{0.610277in}}{\pgfqpoint{6.021441in}{0.599678in}}{\pgfqpoint{6.021441in}{0.588627in}}%
\pgfpathcurveto{\pgfqpoint{6.021441in}{0.577577in}}{\pgfqpoint{6.025832in}{0.566978in}}{\pgfqpoint{6.033645in}{0.559165in}}%
\pgfpathcurveto{\pgfqpoint{6.041459in}{0.551351in}}{\pgfqpoint{6.052058in}{0.546961in}}{\pgfqpoint{6.063108in}{0.546961in}}%
\pgfusepath{stroke}%
\end{pgfscope}%
\begin{pgfscope}%
\pgfpathrectangle{\pgfqpoint{0.847223in}{0.554012in}}{\pgfqpoint{6.200000in}{4.620000in}}%
\pgfusepath{clip}%
\pgfsetbuttcap%
\pgfsetroundjoin%
\pgfsetlinewidth{1.003750pt}%
\definecolor{currentstroke}{rgb}{1.000000,0.000000,0.000000}%
\pgfsetstrokecolor{currentstroke}%
\pgfsetdash{}{0pt}%
\pgfpathmoveto{\pgfqpoint{6.068441in}{0.546534in}}%
\pgfpathcurveto{\pgfqpoint{6.079491in}{0.546534in}}{\pgfqpoint{6.090090in}{0.550924in}}{\pgfqpoint{6.097904in}{0.558738in}}%
\pgfpathcurveto{\pgfqpoint{6.105718in}{0.566551in}}{\pgfqpoint{6.110108in}{0.577150in}}{\pgfqpoint{6.110108in}{0.588200in}}%
\pgfpathcurveto{\pgfqpoint{6.110108in}{0.599250in}}{\pgfqpoint{6.105718in}{0.609849in}}{\pgfqpoint{6.097904in}{0.617663in}}%
\pgfpathcurveto{\pgfqpoint{6.090090in}{0.625477in}}{\pgfqpoint{6.079491in}{0.629867in}}{\pgfqpoint{6.068441in}{0.629867in}}%
\pgfpathcurveto{\pgfqpoint{6.057391in}{0.629867in}}{\pgfqpoint{6.046792in}{0.625477in}}{\pgfqpoint{6.038978in}{0.617663in}}%
\pgfpathcurveto{\pgfqpoint{6.031165in}{0.609849in}}{\pgfqpoint{6.026775in}{0.599250in}}{\pgfqpoint{6.026775in}{0.588200in}}%
\pgfpathcurveto{\pgfqpoint{6.026775in}{0.577150in}}{\pgfqpoint{6.031165in}{0.566551in}}{\pgfqpoint{6.038978in}{0.558738in}}%
\pgfpathcurveto{\pgfqpoint{6.046792in}{0.550924in}}{\pgfqpoint{6.057391in}{0.546534in}}{\pgfqpoint{6.068441in}{0.546534in}}%
\pgfusepath{stroke}%
\end{pgfscope}%
\begin{pgfscope}%
\pgfpathrectangle{\pgfqpoint{0.847223in}{0.554012in}}{\pgfqpoint{6.200000in}{4.620000in}}%
\pgfusepath{clip}%
\pgfsetbuttcap%
\pgfsetroundjoin%
\pgfsetlinewidth{1.003750pt}%
\definecolor{currentstroke}{rgb}{1.000000,0.000000,0.000000}%
\pgfsetstrokecolor{currentstroke}%
\pgfsetdash{}{0pt}%
\pgfpathmoveto{\pgfqpoint{6.073774in}{0.546107in}}%
\pgfpathcurveto{\pgfqpoint{6.084825in}{0.546107in}}{\pgfqpoint{6.095424in}{0.550498in}}{\pgfqpoint{6.103237in}{0.558311in}}%
\pgfpathcurveto{\pgfqpoint{6.111051in}{0.566125in}}{\pgfqpoint{6.115441in}{0.576724in}}{\pgfqpoint{6.115441in}{0.587774in}}%
\pgfpathcurveto{\pgfqpoint{6.115441in}{0.598824in}}{\pgfqpoint{6.111051in}{0.609423in}}{\pgfqpoint{6.103237in}{0.617237in}}%
\pgfpathcurveto{\pgfqpoint{6.095424in}{0.625050in}}{\pgfqpoint{6.084825in}{0.629441in}}{\pgfqpoint{6.073774in}{0.629441in}}%
\pgfpathcurveto{\pgfqpoint{6.062724in}{0.629441in}}{\pgfqpoint{6.052125in}{0.625050in}}{\pgfqpoint{6.044312in}{0.617237in}}%
\pgfpathcurveto{\pgfqpoint{6.036498in}{0.609423in}}{\pgfqpoint{6.032108in}{0.598824in}}{\pgfqpoint{6.032108in}{0.587774in}}%
\pgfpathcurveto{\pgfqpoint{6.032108in}{0.576724in}}{\pgfqpoint{6.036498in}{0.566125in}}{\pgfqpoint{6.044312in}{0.558311in}}%
\pgfpathcurveto{\pgfqpoint{6.052125in}{0.550498in}}{\pgfqpoint{6.062724in}{0.546107in}}{\pgfqpoint{6.073774in}{0.546107in}}%
\pgfusepath{stroke}%
\end{pgfscope}%
\begin{pgfscope}%
\pgfpathrectangle{\pgfqpoint{0.847223in}{0.554012in}}{\pgfqpoint{6.200000in}{4.620000in}}%
\pgfusepath{clip}%
\pgfsetbuttcap%
\pgfsetroundjoin%
\pgfsetlinewidth{1.003750pt}%
\definecolor{currentstroke}{rgb}{1.000000,0.000000,0.000000}%
\pgfsetstrokecolor{currentstroke}%
\pgfsetdash{}{0pt}%
\pgfpathmoveto{\pgfqpoint{6.079108in}{0.545682in}}%
\pgfpathcurveto{\pgfqpoint{6.090158in}{0.545682in}}{\pgfqpoint{6.100757in}{0.550072in}}{\pgfqpoint{6.108570in}{0.557886in}}%
\pgfpathcurveto{\pgfqpoint{6.116384in}{0.565699in}}{\pgfqpoint{6.120774in}{0.576298in}}{\pgfqpoint{6.120774in}{0.587348in}}%
\pgfpathcurveto{\pgfqpoint{6.120774in}{0.598399in}}{\pgfqpoint{6.116384in}{0.608998in}}{\pgfqpoint{6.108570in}{0.616811in}}%
\pgfpathcurveto{\pgfqpoint{6.100757in}{0.624625in}}{\pgfqpoint{6.090158in}{0.629015in}}{\pgfqpoint{6.079108in}{0.629015in}}%
\pgfpathcurveto{\pgfqpoint{6.068057in}{0.629015in}}{\pgfqpoint{6.057458in}{0.624625in}}{\pgfqpoint{6.049645in}{0.616811in}}%
\pgfpathcurveto{\pgfqpoint{6.041831in}{0.608998in}}{\pgfqpoint{6.037441in}{0.598399in}}{\pgfqpoint{6.037441in}{0.587348in}}%
\pgfpathcurveto{\pgfqpoint{6.037441in}{0.576298in}}{\pgfqpoint{6.041831in}{0.565699in}}{\pgfqpoint{6.049645in}{0.557886in}}%
\pgfpathcurveto{\pgfqpoint{6.057458in}{0.550072in}}{\pgfqpoint{6.068057in}{0.545682in}}{\pgfqpoint{6.079108in}{0.545682in}}%
\pgfusepath{stroke}%
\end{pgfscope}%
\begin{pgfscope}%
\pgfpathrectangle{\pgfqpoint{0.847223in}{0.554012in}}{\pgfqpoint{6.200000in}{4.620000in}}%
\pgfusepath{clip}%
\pgfsetbuttcap%
\pgfsetroundjoin%
\pgfsetlinewidth{1.003750pt}%
\definecolor{currentstroke}{rgb}{1.000000,0.000000,0.000000}%
\pgfsetstrokecolor{currentstroke}%
\pgfsetdash{}{0pt}%
\pgfpathmoveto{\pgfqpoint{6.084441in}{0.545257in}}%
\pgfpathcurveto{\pgfqpoint{6.095491in}{0.545257in}}{\pgfqpoint{6.106090in}{0.549647in}}{\pgfqpoint{6.113904in}{0.557461in}}%
\pgfpathcurveto{\pgfqpoint{6.121717in}{0.565274in}}{\pgfqpoint{6.126108in}{0.575873in}}{\pgfqpoint{6.126108in}{0.586924in}}%
\pgfpathcurveto{\pgfqpoint{6.126108in}{0.597974in}}{\pgfqpoint{6.121717in}{0.608573in}}{\pgfqpoint{6.113904in}{0.616386in}}%
\pgfpathcurveto{\pgfqpoint{6.106090in}{0.624200in}}{\pgfqpoint{6.095491in}{0.628590in}}{\pgfqpoint{6.084441in}{0.628590in}}%
\pgfpathcurveto{\pgfqpoint{6.073391in}{0.628590in}}{\pgfqpoint{6.062792in}{0.624200in}}{\pgfqpoint{6.054978in}{0.616386in}}%
\pgfpathcurveto{\pgfqpoint{6.047164in}{0.608573in}}{\pgfqpoint{6.042774in}{0.597974in}}{\pgfqpoint{6.042774in}{0.586924in}}%
\pgfpathcurveto{\pgfqpoint{6.042774in}{0.575873in}}{\pgfqpoint{6.047164in}{0.565274in}}{\pgfqpoint{6.054978in}{0.557461in}}%
\pgfpathcurveto{\pgfqpoint{6.062792in}{0.549647in}}{\pgfqpoint{6.073391in}{0.545257in}}{\pgfqpoint{6.084441in}{0.545257in}}%
\pgfusepath{stroke}%
\end{pgfscope}%
\begin{pgfscope}%
\pgfpathrectangle{\pgfqpoint{0.847223in}{0.554012in}}{\pgfqpoint{6.200000in}{4.620000in}}%
\pgfusepath{clip}%
\pgfsetbuttcap%
\pgfsetroundjoin%
\pgfsetlinewidth{1.003750pt}%
\definecolor{currentstroke}{rgb}{1.000000,0.000000,0.000000}%
\pgfsetstrokecolor{currentstroke}%
\pgfsetdash{}{0pt}%
\pgfpathmoveto{\pgfqpoint{6.089774in}{0.544833in}}%
\pgfpathcurveto{\pgfqpoint{6.100824in}{0.544833in}}{\pgfqpoint{6.111423in}{0.549223in}}{\pgfqpoint{6.119237in}{0.557037in}}%
\pgfpathcurveto{\pgfqpoint{6.127050in}{0.564850in}}{\pgfqpoint{6.131441in}{0.575450in}}{\pgfqpoint{6.131441in}{0.586500in}}%
\pgfpathcurveto{\pgfqpoint{6.131441in}{0.597550in}}{\pgfqpoint{6.127050in}{0.608149in}}{\pgfqpoint{6.119237in}{0.615962in}}%
\pgfpathcurveto{\pgfqpoint{6.111423in}{0.623776in}}{\pgfqpoint{6.100824in}{0.628166in}}{\pgfqpoint{6.089774in}{0.628166in}}%
\pgfpathcurveto{\pgfqpoint{6.078724in}{0.628166in}}{\pgfqpoint{6.068125in}{0.623776in}}{\pgfqpoint{6.060311in}{0.615962in}}%
\pgfpathcurveto{\pgfqpoint{6.052498in}{0.608149in}}{\pgfqpoint{6.048107in}{0.597550in}}{\pgfqpoint{6.048107in}{0.586500in}}%
\pgfpathcurveto{\pgfqpoint{6.048107in}{0.575450in}}{\pgfqpoint{6.052498in}{0.564850in}}{\pgfqpoint{6.060311in}{0.557037in}}%
\pgfpathcurveto{\pgfqpoint{6.068125in}{0.549223in}}{\pgfqpoint{6.078724in}{0.544833in}}{\pgfqpoint{6.089774in}{0.544833in}}%
\pgfusepath{stroke}%
\end{pgfscope}%
\begin{pgfscope}%
\pgfpathrectangle{\pgfqpoint{0.847223in}{0.554012in}}{\pgfqpoint{6.200000in}{4.620000in}}%
\pgfusepath{clip}%
\pgfsetbuttcap%
\pgfsetroundjoin%
\pgfsetlinewidth{1.003750pt}%
\definecolor{currentstroke}{rgb}{1.000000,0.000000,0.000000}%
\pgfsetstrokecolor{currentstroke}%
\pgfsetdash{}{0pt}%
\pgfpathmoveto{\pgfqpoint{6.095107in}{0.544410in}}%
\pgfpathcurveto{\pgfqpoint{6.106157in}{0.544410in}}{\pgfqpoint{6.116756in}{0.548800in}}{\pgfqpoint{6.124570in}{0.556614in}}%
\pgfpathcurveto{\pgfqpoint{6.132384in}{0.564427in}}{\pgfqpoint{6.136774in}{0.575026in}}{\pgfqpoint{6.136774in}{0.586076in}}%
\pgfpathcurveto{\pgfqpoint{6.136774in}{0.597127in}}{\pgfqpoint{6.132384in}{0.607726in}}{\pgfqpoint{6.124570in}{0.615539in}}%
\pgfpathcurveto{\pgfqpoint{6.116756in}{0.623353in}}{\pgfqpoint{6.106157in}{0.627743in}}{\pgfqpoint{6.095107in}{0.627743in}}%
\pgfpathcurveto{\pgfqpoint{6.084057in}{0.627743in}}{\pgfqpoint{6.073458in}{0.623353in}}{\pgfqpoint{6.065644in}{0.615539in}}%
\pgfpathcurveto{\pgfqpoint{6.057831in}{0.607726in}}{\pgfqpoint{6.053441in}{0.597127in}}{\pgfqpoint{6.053441in}{0.586076in}}%
\pgfpathcurveto{\pgfqpoint{6.053441in}{0.575026in}}{\pgfqpoint{6.057831in}{0.564427in}}{\pgfqpoint{6.065644in}{0.556614in}}%
\pgfpathcurveto{\pgfqpoint{6.073458in}{0.548800in}}{\pgfqpoint{6.084057in}{0.544410in}}{\pgfqpoint{6.095107in}{0.544410in}}%
\pgfusepath{stroke}%
\end{pgfscope}%
\begin{pgfscope}%
\pgfpathrectangle{\pgfqpoint{0.847223in}{0.554012in}}{\pgfqpoint{6.200000in}{4.620000in}}%
\pgfusepath{clip}%
\pgfsetbuttcap%
\pgfsetroundjoin%
\pgfsetlinewidth{1.003750pt}%
\definecolor{currentstroke}{rgb}{1.000000,0.000000,0.000000}%
\pgfsetstrokecolor{currentstroke}%
\pgfsetdash{}{0pt}%
\pgfpathmoveto{\pgfqpoint{6.100440in}{0.543987in}}%
\pgfpathcurveto{\pgfqpoint{6.111491in}{0.543987in}}{\pgfqpoint{6.122090in}{0.548378in}}{\pgfqpoint{6.129903in}{0.556191in}}%
\pgfpathcurveto{\pgfqpoint{6.137717in}{0.564005in}}{\pgfqpoint{6.142107in}{0.574604in}}{\pgfqpoint{6.142107in}{0.585654in}}%
\pgfpathcurveto{\pgfqpoint{6.142107in}{0.596704in}}{\pgfqpoint{6.137717in}{0.607303in}}{\pgfqpoint{6.129903in}{0.615117in}}%
\pgfpathcurveto{\pgfqpoint{6.122090in}{0.622930in}}{\pgfqpoint{6.111491in}{0.627321in}}{\pgfqpoint{6.100440in}{0.627321in}}%
\pgfpathcurveto{\pgfqpoint{6.089390in}{0.627321in}}{\pgfqpoint{6.078791in}{0.622930in}}{\pgfqpoint{6.070978in}{0.615117in}}%
\pgfpathcurveto{\pgfqpoint{6.063164in}{0.607303in}}{\pgfqpoint{6.058774in}{0.596704in}}{\pgfqpoint{6.058774in}{0.585654in}}%
\pgfpathcurveto{\pgfqpoint{6.058774in}{0.574604in}}{\pgfqpoint{6.063164in}{0.564005in}}{\pgfqpoint{6.070978in}{0.556191in}}%
\pgfpathcurveto{\pgfqpoint{6.078791in}{0.548378in}}{\pgfqpoint{6.089390in}{0.543987in}}{\pgfqpoint{6.100440in}{0.543987in}}%
\pgfusepath{stroke}%
\end{pgfscope}%
\begin{pgfscope}%
\pgfpathrectangle{\pgfqpoint{0.847223in}{0.554012in}}{\pgfqpoint{6.200000in}{4.620000in}}%
\pgfusepath{clip}%
\pgfsetbuttcap%
\pgfsetroundjoin%
\pgfsetlinewidth{1.003750pt}%
\definecolor{currentstroke}{rgb}{1.000000,0.000000,0.000000}%
\pgfsetstrokecolor{currentstroke}%
\pgfsetdash{}{0pt}%
\pgfpathmoveto{\pgfqpoint{6.105774in}{0.543566in}}%
\pgfpathcurveto{\pgfqpoint{6.116824in}{0.543566in}}{\pgfqpoint{6.127423in}{0.547956in}}{\pgfqpoint{6.135236in}{0.555770in}}%
\pgfpathcurveto{\pgfqpoint{6.143050in}{0.563583in}}{\pgfqpoint{6.147440in}{0.574182in}}{\pgfqpoint{6.147440in}{0.585232in}}%
\pgfpathcurveto{\pgfqpoint{6.147440in}{0.596282in}}{\pgfqpoint{6.143050in}{0.606882in}}{\pgfqpoint{6.135236in}{0.614695in}}%
\pgfpathcurveto{\pgfqpoint{6.127423in}{0.622509in}}{\pgfqpoint{6.116824in}{0.626899in}}{\pgfqpoint{6.105774in}{0.626899in}}%
\pgfpathcurveto{\pgfqpoint{6.094724in}{0.626899in}}{\pgfqpoint{6.084125in}{0.622509in}}{\pgfqpoint{6.076311in}{0.614695in}}%
\pgfpathcurveto{\pgfqpoint{6.068497in}{0.606882in}}{\pgfqpoint{6.064107in}{0.596282in}}{\pgfqpoint{6.064107in}{0.585232in}}%
\pgfpathcurveto{\pgfqpoint{6.064107in}{0.574182in}}{\pgfqpoint{6.068497in}{0.563583in}}{\pgfqpoint{6.076311in}{0.555770in}}%
\pgfpathcurveto{\pgfqpoint{6.084125in}{0.547956in}}{\pgfqpoint{6.094724in}{0.543566in}}{\pgfqpoint{6.105774in}{0.543566in}}%
\pgfusepath{stroke}%
\end{pgfscope}%
\begin{pgfscope}%
\pgfpathrectangle{\pgfqpoint{0.847223in}{0.554012in}}{\pgfqpoint{6.200000in}{4.620000in}}%
\pgfusepath{clip}%
\pgfsetbuttcap%
\pgfsetroundjoin%
\pgfsetlinewidth{1.003750pt}%
\definecolor{currentstroke}{rgb}{1.000000,0.000000,0.000000}%
\pgfsetstrokecolor{currentstroke}%
\pgfsetdash{}{0pt}%
\pgfpathmoveto{\pgfqpoint{6.111107in}{0.543145in}}%
\pgfpathcurveto{\pgfqpoint{6.122157in}{0.543145in}}{\pgfqpoint{6.132756in}{0.547535in}}{\pgfqpoint{6.140570in}{0.555349in}}%
\pgfpathcurveto{\pgfqpoint{6.148383in}{0.563162in}}{\pgfqpoint{6.152774in}{0.573761in}}{\pgfqpoint{6.152774in}{0.584812in}}%
\pgfpathcurveto{\pgfqpoint{6.152774in}{0.595862in}}{\pgfqpoint{6.148383in}{0.606461in}}{\pgfqpoint{6.140570in}{0.614274in}}%
\pgfpathcurveto{\pgfqpoint{6.132756in}{0.622088in}}{\pgfqpoint{6.122157in}{0.626478in}}{\pgfqpoint{6.111107in}{0.626478in}}%
\pgfpathcurveto{\pgfqpoint{6.100057in}{0.626478in}}{\pgfqpoint{6.089458in}{0.622088in}}{\pgfqpoint{6.081644in}{0.614274in}}%
\pgfpathcurveto{\pgfqpoint{6.073831in}{0.606461in}}{\pgfqpoint{6.069440in}{0.595862in}}{\pgfqpoint{6.069440in}{0.584812in}}%
\pgfpathcurveto{\pgfqpoint{6.069440in}{0.573761in}}{\pgfqpoint{6.073831in}{0.563162in}}{\pgfqpoint{6.081644in}{0.555349in}}%
\pgfpathcurveto{\pgfqpoint{6.089458in}{0.547535in}}{\pgfqpoint{6.100057in}{0.543145in}}{\pgfqpoint{6.111107in}{0.543145in}}%
\pgfusepath{stroke}%
\end{pgfscope}%
\begin{pgfscope}%
\pgfpathrectangle{\pgfqpoint{0.847223in}{0.554012in}}{\pgfqpoint{6.200000in}{4.620000in}}%
\pgfusepath{clip}%
\pgfsetbuttcap%
\pgfsetroundjoin%
\pgfsetlinewidth{1.003750pt}%
\definecolor{currentstroke}{rgb}{1.000000,0.000000,0.000000}%
\pgfsetstrokecolor{currentstroke}%
\pgfsetdash{}{0pt}%
\pgfpathmoveto{\pgfqpoint{6.116440in}{0.542725in}}%
\pgfpathcurveto{\pgfqpoint{6.127490in}{0.542725in}}{\pgfqpoint{6.138089in}{0.547115in}}{\pgfqpoint{6.145903in}{0.554929in}}%
\pgfpathcurveto{\pgfqpoint{6.153717in}{0.562742in}}{\pgfqpoint{6.158107in}{0.573341in}}{\pgfqpoint{6.158107in}{0.584391in}}%
\pgfpathcurveto{\pgfqpoint{6.158107in}{0.595442in}}{\pgfqpoint{6.153717in}{0.606041in}}{\pgfqpoint{6.145903in}{0.613854in}}%
\pgfpathcurveto{\pgfqpoint{6.138089in}{0.621668in}}{\pgfqpoint{6.127490in}{0.626058in}}{\pgfqpoint{6.116440in}{0.626058in}}%
\pgfpathcurveto{\pgfqpoint{6.105390in}{0.626058in}}{\pgfqpoint{6.094791in}{0.621668in}}{\pgfqpoint{6.086977in}{0.613854in}}%
\pgfpathcurveto{\pgfqpoint{6.079164in}{0.606041in}}{\pgfqpoint{6.074773in}{0.595442in}}{\pgfqpoint{6.074773in}{0.584391in}}%
\pgfpathcurveto{\pgfqpoint{6.074773in}{0.573341in}}{\pgfqpoint{6.079164in}{0.562742in}}{\pgfqpoint{6.086977in}{0.554929in}}%
\pgfpathcurveto{\pgfqpoint{6.094791in}{0.547115in}}{\pgfqpoint{6.105390in}{0.542725in}}{\pgfqpoint{6.116440in}{0.542725in}}%
\pgfusepath{stroke}%
\end{pgfscope}%
\begin{pgfscope}%
\pgfpathrectangle{\pgfqpoint{0.847223in}{0.554012in}}{\pgfqpoint{6.200000in}{4.620000in}}%
\pgfusepath{clip}%
\pgfsetbuttcap%
\pgfsetroundjoin%
\pgfsetlinewidth{1.003750pt}%
\definecolor{currentstroke}{rgb}{1.000000,0.000000,0.000000}%
\pgfsetstrokecolor{currentstroke}%
\pgfsetdash{}{0pt}%
\pgfpathmoveto{\pgfqpoint{6.121773in}{0.542305in}}%
\pgfpathcurveto{\pgfqpoint{6.132823in}{0.542305in}}{\pgfqpoint{6.143423in}{0.546696in}}{\pgfqpoint{6.151236in}{0.554509in}}%
\pgfpathcurveto{\pgfqpoint{6.159050in}{0.562323in}}{\pgfqpoint{6.163440in}{0.572922in}}{\pgfqpoint{6.163440in}{0.583972in}}%
\pgfpathcurveto{\pgfqpoint{6.163440in}{0.595022in}}{\pgfqpoint{6.159050in}{0.605621in}}{\pgfqpoint{6.151236in}{0.613435in}}%
\pgfpathcurveto{\pgfqpoint{6.143423in}{0.621249in}}{\pgfqpoint{6.132823in}{0.625639in}}{\pgfqpoint{6.121773in}{0.625639in}}%
\pgfpathcurveto{\pgfqpoint{6.110723in}{0.625639in}}{\pgfqpoint{6.100124in}{0.621249in}}{\pgfqpoint{6.092311in}{0.613435in}}%
\pgfpathcurveto{\pgfqpoint{6.084497in}{0.605621in}}{\pgfqpoint{6.080107in}{0.595022in}}{\pgfqpoint{6.080107in}{0.583972in}}%
\pgfpathcurveto{\pgfqpoint{6.080107in}{0.572922in}}{\pgfqpoint{6.084497in}{0.562323in}}{\pgfqpoint{6.092311in}{0.554509in}}%
\pgfpathcurveto{\pgfqpoint{6.100124in}{0.546696in}}{\pgfqpoint{6.110723in}{0.542305in}}{\pgfqpoint{6.121773in}{0.542305in}}%
\pgfusepath{stroke}%
\end{pgfscope}%
\begin{pgfscope}%
\pgfpathrectangle{\pgfqpoint{0.847223in}{0.554012in}}{\pgfqpoint{6.200000in}{4.620000in}}%
\pgfusepath{clip}%
\pgfsetbuttcap%
\pgfsetroundjoin%
\pgfsetlinewidth{1.003750pt}%
\definecolor{currentstroke}{rgb}{1.000000,0.000000,0.000000}%
\pgfsetstrokecolor{currentstroke}%
\pgfsetdash{}{0pt}%
\pgfpathmoveto{\pgfqpoint{6.127107in}{0.541887in}}%
\pgfpathcurveto{\pgfqpoint{6.138157in}{0.541887in}}{\pgfqpoint{6.148756in}{0.546277in}}{\pgfqpoint{6.156569in}{0.554091in}}%
\pgfpathcurveto{\pgfqpoint{6.164383in}{0.561904in}}{\pgfqpoint{6.168773in}{0.572503in}}{\pgfqpoint{6.168773in}{0.583554in}}%
\pgfpathcurveto{\pgfqpoint{6.168773in}{0.594604in}}{\pgfqpoint{6.164383in}{0.605203in}}{\pgfqpoint{6.156569in}{0.613016in}}%
\pgfpathcurveto{\pgfqpoint{6.148756in}{0.620830in}}{\pgfqpoint{6.138157in}{0.625220in}}{\pgfqpoint{6.127107in}{0.625220in}}%
\pgfpathcurveto{\pgfqpoint{6.116056in}{0.625220in}}{\pgfqpoint{6.105457in}{0.620830in}}{\pgfqpoint{6.097644in}{0.613016in}}%
\pgfpathcurveto{\pgfqpoint{6.089830in}{0.605203in}}{\pgfqpoint{6.085440in}{0.594604in}}{\pgfqpoint{6.085440in}{0.583554in}}%
\pgfpathcurveto{\pgfqpoint{6.085440in}{0.572503in}}{\pgfqpoint{6.089830in}{0.561904in}}{\pgfqpoint{6.097644in}{0.554091in}}%
\pgfpathcurveto{\pgfqpoint{6.105457in}{0.546277in}}{\pgfqpoint{6.116056in}{0.541887in}}{\pgfqpoint{6.127107in}{0.541887in}}%
\pgfusepath{stroke}%
\end{pgfscope}%
\begin{pgfscope}%
\pgfpathrectangle{\pgfqpoint{0.847223in}{0.554012in}}{\pgfqpoint{6.200000in}{4.620000in}}%
\pgfusepath{clip}%
\pgfsetbuttcap%
\pgfsetroundjoin%
\pgfsetlinewidth{1.003750pt}%
\definecolor{currentstroke}{rgb}{1.000000,0.000000,0.000000}%
\pgfsetstrokecolor{currentstroke}%
\pgfsetdash{}{0pt}%
\pgfpathmoveto{\pgfqpoint{6.132440in}{0.541469in}}%
\pgfpathcurveto{\pgfqpoint{6.143490in}{0.541469in}}{\pgfqpoint{6.154089in}{0.545859in}}{\pgfqpoint{6.161903in}{0.553673in}}%
\pgfpathcurveto{\pgfqpoint{6.169716in}{0.561487in}}{\pgfqpoint{6.174106in}{0.572086in}}{\pgfqpoint{6.174106in}{0.583136in}}%
\pgfpathcurveto{\pgfqpoint{6.174106in}{0.594186in}}{\pgfqpoint{6.169716in}{0.604785in}}{\pgfqpoint{6.161903in}{0.612599in}}%
\pgfpathcurveto{\pgfqpoint{6.154089in}{0.620412in}}{\pgfqpoint{6.143490in}{0.624802in}}{\pgfqpoint{6.132440in}{0.624802in}}%
\pgfpathcurveto{\pgfqpoint{6.121390in}{0.624802in}}{\pgfqpoint{6.110791in}{0.620412in}}{\pgfqpoint{6.102977in}{0.612599in}}%
\pgfpathcurveto{\pgfqpoint{6.095163in}{0.604785in}}{\pgfqpoint{6.090773in}{0.594186in}}{\pgfqpoint{6.090773in}{0.583136in}}%
\pgfpathcurveto{\pgfqpoint{6.090773in}{0.572086in}}{\pgfqpoint{6.095163in}{0.561487in}}{\pgfqpoint{6.102977in}{0.553673in}}%
\pgfpathcurveto{\pgfqpoint{6.110791in}{0.545859in}}{\pgfqpoint{6.121390in}{0.541469in}}{\pgfqpoint{6.132440in}{0.541469in}}%
\pgfusepath{stroke}%
\end{pgfscope}%
\begin{pgfscope}%
\pgfpathrectangle{\pgfqpoint{0.847223in}{0.554012in}}{\pgfqpoint{6.200000in}{4.620000in}}%
\pgfusepath{clip}%
\pgfsetbuttcap%
\pgfsetroundjoin%
\pgfsetlinewidth{1.003750pt}%
\definecolor{currentstroke}{rgb}{1.000000,0.000000,0.000000}%
\pgfsetstrokecolor{currentstroke}%
\pgfsetdash{}{0pt}%
\pgfpathmoveto{\pgfqpoint{6.137773in}{0.541052in}}%
\pgfpathcurveto{\pgfqpoint{6.148823in}{0.541052in}}{\pgfqpoint{6.159422in}{0.545442in}}{\pgfqpoint{6.167236in}{0.553256in}}%
\pgfpathcurveto{\pgfqpoint{6.175049in}{0.561070in}}{\pgfqpoint{6.179440in}{0.571669in}}{\pgfqpoint{6.179440in}{0.582719in}}%
\pgfpathcurveto{\pgfqpoint{6.179440in}{0.593769in}}{\pgfqpoint{6.175049in}{0.604368in}}{\pgfqpoint{6.167236in}{0.612182in}}%
\pgfpathcurveto{\pgfqpoint{6.159422in}{0.619995in}}{\pgfqpoint{6.148823in}{0.624385in}}{\pgfqpoint{6.137773in}{0.624385in}}%
\pgfpathcurveto{\pgfqpoint{6.126723in}{0.624385in}}{\pgfqpoint{6.116124in}{0.619995in}}{\pgfqpoint{6.108310in}{0.612182in}}%
\pgfpathcurveto{\pgfqpoint{6.100497in}{0.604368in}}{\pgfqpoint{6.096106in}{0.593769in}}{\pgfqpoint{6.096106in}{0.582719in}}%
\pgfpathcurveto{\pgfqpoint{6.096106in}{0.571669in}}{\pgfqpoint{6.100497in}{0.561070in}}{\pgfqpoint{6.108310in}{0.553256in}}%
\pgfpathcurveto{\pgfqpoint{6.116124in}{0.545442in}}{\pgfqpoint{6.126723in}{0.541052in}}{\pgfqpoint{6.137773in}{0.541052in}}%
\pgfusepath{stroke}%
\end{pgfscope}%
\begin{pgfscope}%
\pgfpathrectangle{\pgfqpoint{0.847223in}{0.554012in}}{\pgfqpoint{6.200000in}{4.620000in}}%
\pgfusepath{clip}%
\pgfsetbuttcap%
\pgfsetroundjoin%
\pgfsetlinewidth{1.003750pt}%
\definecolor{currentstroke}{rgb}{1.000000,0.000000,0.000000}%
\pgfsetstrokecolor{currentstroke}%
\pgfsetdash{}{0pt}%
\pgfpathmoveto{\pgfqpoint{6.143106in}{0.540636in}}%
\pgfpathcurveto{\pgfqpoint{6.154156in}{0.540636in}}{\pgfqpoint{6.164755in}{0.545026in}}{\pgfqpoint{6.172569in}{0.552840in}}%
\pgfpathcurveto{\pgfqpoint{6.180383in}{0.560653in}}{\pgfqpoint{6.184773in}{0.571252in}}{\pgfqpoint{6.184773in}{0.582303in}}%
\pgfpathcurveto{\pgfqpoint{6.184773in}{0.593353in}}{\pgfqpoint{6.180383in}{0.603952in}}{\pgfqpoint{6.172569in}{0.611765in}}%
\pgfpathcurveto{\pgfqpoint{6.164755in}{0.619579in}}{\pgfqpoint{6.154156in}{0.623969in}}{\pgfqpoint{6.143106in}{0.623969in}}%
\pgfpathcurveto{\pgfqpoint{6.132056in}{0.623969in}}{\pgfqpoint{6.121457in}{0.619579in}}{\pgfqpoint{6.113643in}{0.611765in}}%
\pgfpathcurveto{\pgfqpoint{6.105830in}{0.603952in}}{\pgfqpoint{6.101440in}{0.593353in}}{\pgfqpoint{6.101440in}{0.582303in}}%
\pgfpathcurveto{\pgfqpoint{6.101440in}{0.571252in}}{\pgfqpoint{6.105830in}{0.560653in}}{\pgfqpoint{6.113643in}{0.552840in}}%
\pgfpathcurveto{\pgfqpoint{6.121457in}{0.545026in}}{\pgfqpoint{6.132056in}{0.540636in}}{\pgfqpoint{6.143106in}{0.540636in}}%
\pgfusepath{stroke}%
\end{pgfscope}%
\begin{pgfscope}%
\pgfpathrectangle{\pgfqpoint{0.847223in}{0.554012in}}{\pgfqpoint{6.200000in}{4.620000in}}%
\pgfusepath{clip}%
\pgfsetbuttcap%
\pgfsetroundjoin%
\pgfsetlinewidth{1.003750pt}%
\definecolor{currentstroke}{rgb}{1.000000,0.000000,0.000000}%
\pgfsetstrokecolor{currentstroke}%
\pgfsetdash{}{0pt}%
\pgfpathmoveto{\pgfqpoint{6.148439in}{0.540220in}}%
\pgfpathcurveto{\pgfqpoint{6.159490in}{0.540220in}}{\pgfqpoint{6.170089in}{0.544611in}}{\pgfqpoint{6.177902in}{0.552424in}}%
\pgfpathcurveto{\pgfqpoint{6.185716in}{0.560238in}}{\pgfqpoint{6.190106in}{0.570837in}}{\pgfqpoint{6.190106in}{0.581887in}}%
\pgfpathcurveto{\pgfqpoint{6.190106in}{0.592937in}}{\pgfqpoint{6.185716in}{0.603536in}}{\pgfqpoint{6.177902in}{0.611350in}}%
\pgfpathcurveto{\pgfqpoint{6.170089in}{0.619163in}}{\pgfqpoint{6.159490in}{0.623554in}}{\pgfqpoint{6.148439in}{0.623554in}}%
\pgfpathcurveto{\pgfqpoint{6.137389in}{0.623554in}}{\pgfqpoint{6.126790in}{0.619163in}}{\pgfqpoint{6.118977in}{0.611350in}}%
\pgfpathcurveto{\pgfqpoint{6.111163in}{0.603536in}}{\pgfqpoint{6.106773in}{0.592937in}}{\pgfqpoint{6.106773in}{0.581887in}}%
\pgfpathcurveto{\pgfqpoint{6.106773in}{0.570837in}}{\pgfqpoint{6.111163in}{0.560238in}}{\pgfqpoint{6.118977in}{0.552424in}}%
\pgfpathcurveto{\pgfqpoint{6.126790in}{0.544611in}}{\pgfqpoint{6.137389in}{0.540220in}}{\pgfqpoint{6.148439in}{0.540220in}}%
\pgfusepath{stroke}%
\end{pgfscope}%
\begin{pgfscope}%
\pgfpathrectangle{\pgfqpoint{0.847223in}{0.554012in}}{\pgfqpoint{6.200000in}{4.620000in}}%
\pgfusepath{clip}%
\pgfsetbuttcap%
\pgfsetroundjoin%
\pgfsetlinewidth{1.003750pt}%
\definecolor{currentstroke}{rgb}{1.000000,0.000000,0.000000}%
\pgfsetstrokecolor{currentstroke}%
\pgfsetdash{}{0pt}%
\pgfpathmoveto{\pgfqpoint{6.153773in}{0.539806in}}%
\pgfpathcurveto{\pgfqpoint{6.164823in}{0.539806in}}{\pgfqpoint{6.175422in}{0.544196in}}{\pgfqpoint{6.183235in}{0.552010in}}%
\pgfpathcurveto{\pgfqpoint{6.191049in}{0.559823in}}{\pgfqpoint{6.195439in}{0.570422in}}{\pgfqpoint{6.195439in}{0.581472in}}%
\pgfpathcurveto{\pgfqpoint{6.195439in}{0.592522in}}{\pgfqpoint{6.191049in}{0.603121in}}{\pgfqpoint{6.183235in}{0.610935in}}%
\pgfpathcurveto{\pgfqpoint{6.175422in}{0.618749in}}{\pgfqpoint{6.164823in}{0.623139in}}{\pgfqpoint{6.153773in}{0.623139in}}%
\pgfpathcurveto{\pgfqpoint{6.142723in}{0.623139in}}{\pgfqpoint{6.132123in}{0.618749in}}{\pgfqpoint{6.124310in}{0.610935in}}%
\pgfpathcurveto{\pgfqpoint{6.116496in}{0.603121in}}{\pgfqpoint{6.112106in}{0.592522in}}{\pgfqpoint{6.112106in}{0.581472in}}%
\pgfpathcurveto{\pgfqpoint{6.112106in}{0.570422in}}{\pgfqpoint{6.116496in}{0.559823in}}{\pgfqpoint{6.124310in}{0.552010in}}%
\pgfpathcurveto{\pgfqpoint{6.132123in}{0.544196in}}{\pgfqpoint{6.142723in}{0.539806in}}{\pgfqpoint{6.153773in}{0.539806in}}%
\pgfusepath{stroke}%
\end{pgfscope}%
\begin{pgfscope}%
\pgfpathrectangle{\pgfqpoint{0.847223in}{0.554012in}}{\pgfqpoint{6.200000in}{4.620000in}}%
\pgfusepath{clip}%
\pgfsetbuttcap%
\pgfsetroundjoin%
\pgfsetlinewidth{1.003750pt}%
\definecolor{currentstroke}{rgb}{1.000000,0.000000,0.000000}%
\pgfsetstrokecolor{currentstroke}%
\pgfsetdash{}{0pt}%
\pgfpathmoveto{\pgfqpoint{6.159106in}{0.539392in}}%
\pgfpathcurveto{\pgfqpoint{6.170156in}{0.539392in}}{\pgfqpoint{6.180755in}{0.543782in}}{\pgfqpoint{6.188569in}{0.551596in}}%
\pgfpathcurveto{\pgfqpoint{6.196382in}{0.559409in}}{\pgfqpoint{6.200773in}{0.570008in}}{\pgfqpoint{6.200773in}{0.581058in}}%
\pgfpathcurveto{\pgfqpoint{6.200773in}{0.592108in}}{\pgfqpoint{6.196382in}{0.602708in}}{\pgfqpoint{6.188569in}{0.610521in}}%
\pgfpathcurveto{\pgfqpoint{6.180755in}{0.618335in}}{\pgfqpoint{6.170156in}{0.622725in}}{\pgfqpoint{6.159106in}{0.622725in}}%
\pgfpathcurveto{\pgfqpoint{6.148056in}{0.622725in}}{\pgfqpoint{6.137457in}{0.618335in}}{\pgfqpoint{6.129643in}{0.610521in}}%
\pgfpathcurveto{\pgfqpoint{6.121829in}{0.602708in}}{\pgfqpoint{6.117439in}{0.592108in}}{\pgfqpoint{6.117439in}{0.581058in}}%
\pgfpathcurveto{\pgfqpoint{6.117439in}{0.570008in}}{\pgfqpoint{6.121829in}{0.559409in}}{\pgfqpoint{6.129643in}{0.551596in}}%
\pgfpathcurveto{\pgfqpoint{6.137457in}{0.543782in}}{\pgfqpoint{6.148056in}{0.539392in}}{\pgfqpoint{6.159106in}{0.539392in}}%
\pgfusepath{stroke}%
\end{pgfscope}%
\begin{pgfscope}%
\pgfpathrectangle{\pgfqpoint{0.847223in}{0.554012in}}{\pgfqpoint{6.200000in}{4.620000in}}%
\pgfusepath{clip}%
\pgfsetbuttcap%
\pgfsetroundjoin%
\pgfsetlinewidth{1.003750pt}%
\definecolor{currentstroke}{rgb}{1.000000,0.000000,0.000000}%
\pgfsetstrokecolor{currentstroke}%
\pgfsetdash{}{0pt}%
\pgfpathmoveto{\pgfqpoint{6.164439in}{0.538978in}}%
\pgfpathcurveto{\pgfqpoint{6.175489in}{0.538978in}}{\pgfqpoint{6.186088in}{0.543369in}}{\pgfqpoint{6.193902in}{0.551182in}}%
\pgfpathcurveto{\pgfqpoint{6.201715in}{0.558996in}}{\pgfqpoint{6.206106in}{0.569595in}}{\pgfqpoint{6.206106in}{0.580645in}}%
\pgfpathcurveto{\pgfqpoint{6.206106in}{0.591695in}}{\pgfqpoint{6.201715in}{0.602294in}}{\pgfqpoint{6.193902in}{0.610108in}}%
\pgfpathcurveto{\pgfqpoint{6.186088in}{0.617922in}}{\pgfqpoint{6.175489in}{0.622312in}}{\pgfqpoint{6.164439in}{0.622312in}}%
\pgfpathcurveto{\pgfqpoint{6.153389in}{0.622312in}}{\pgfqpoint{6.142790in}{0.617922in}}{\pgfqpoint{6.134976in}{0.610108in}}%
\pgfpathcurveto{\pgfqpoint{6.127163in}{0.602294in}}{\pgfqpoint{6.122772in}{0.591695in}}{\pgfqpoint{6.122772in}{0.580645in}}%
\pgfpathcurveto{\pgfqpoint{6.122772in}{0.569595in}}{\pgfqpoint{6.127163in}{0.558996in}}{\pgfqpoint{6.134976in}{0.551182in}}%
\pgfpathcurveto{\pgfqpoint{6.142790in}{0.543369in}}{\pgfqpoint{6.153389in}{0.538978in}}{\pgfqpoint{6.164439in}{0.538978in}}%
\pgfusepath{stroke}%
\end{pgfscope}%
\begin{pgfscope}%
\pgfpathrectangle{\pgfqpoint{0.847223in}{0.554012in}}{\pgfqpoint{6.200000in}{4.620000in}}%
\pgfusepath{clip}%
\pgfsetbuttcap%
\pgfsetroundjoin%
\pgfsetlinewidth{1.003750pt}%
\definecolor{currentstroke}{rgb}{1.000000,0.000000,0.000000}%
\pgfsetstrokecolor{currentstroke}%
\pgfsetdash{}{0pt}%
\pgfpathmoveto{\pgfqpoint{6.169772in}{0.538566in}}%
\pgfpathcurveto{\pgfqpoint{6.180822in}{0.538566in}}{\pgfqpoint{6.191421in}{0.542956in}}{\pgfqpoint{6.199235in}{0.550770in}}%
\pgfpathcurveto{\pgfqpoint{6.207049in}{0.558584in}}{\pgfqpoint{6.211439in}{0.569183in}}{\pgfqpoint{6.211439in}{0.580233in}}%
\pgfpathcurveto{\pgfqpoint{6.211439in}{0.591283in}}{\pgfqpoint{6.207049in}{0.601882in}}{\pgfqpoint{6.199235in}{0.609695in}}%
\pgfpathcurveto{\pgfqpoint{6.191421in}{0.617509in}}{\pgfqpoint{6.180822in}{0.621899in}}{\pgfqpoint{6.169772in}{0.621899in}}%
\pgfpathcurveto{\pgfqpoint{6.158722in}{0.621899in}}{\pgfqpoint{6.148123in}{0.617509in}}{\pgfqpoint{6.140310in}{0.609695in}}%
\pgfpathcurveto{\pgfqpoint{6.132496in}{0.601882in}}{\pgfqpoint{6.128106in}{0.591283in}}{\pgfqpoint{6.128106in}{0.580233in}}%
\pgfpathcurveto{\pgfqpoint{6.128106in}{0.569183in}}{\pgfqpoint{6.132496in}{0.558584in}}{\pgfqpoint{6.140310in}{0.550770in}}%
\pgfpathcurveto{\pgfqpoint{6.148123in}{0.542956in}}{\pgfqpoint{6.158722in}{0.538566in}}{\pgfqpoint{6.169772in}{0.538566in}}%
\pgfusepath{stroke}%
\end{pgfscope}%
\begin{pgfscope}%
\pgfpathrectangle{\pgfqpoint{0.847223in}{0.554012in}}{\pgfqpoint{6.200000in}{4.620000in}}%
\pgfusepath{clip}%
\pgfsetbuttcap%
\pgfsetroundjoin%
\pgfsetlinewidth{1.003750pt}%
\definecolor{currentstroke}{rgb}{1.000000,0.000000,0.000000}%
\pgfsetstrokecolor{currentstroke}%
\pgfsetdash{}{0pt}%
\pgfpathmoveto{\pgfqpoint{6.175105in}{0.538154in}}%
\pgfpathcurveto{\pgfqpoint{6.186156in}{0.538154in}}{\pgfqpoint{6.196755in}{0.542545in}}{\pgfqpoint{6.204568in}{0.550358in}}%
\pgfpathcurveto{\pgfqpoint{6.212382in}{0.558172in}}{\pgfqpoint{6.216772in}{0.568771in}}{\pgfqpoint{6.216772in}{0.579821in}}%
\pgfpathcurveto{\pgfqpoint{6.216772in}{0.590871in}}{\pgfqpoint{6.212382in}{0.601470in}}{\pgfqpoint{6.204568in}{0.609284in}}%
\pgfpathcurveto{\pgfqpoint{6.196755in}{0.617097in}}{\pgfqpoint{6.186156in}{0.621488in}}{\pgfqpoint{6.175105in}{0.621488in}}%
\pgfpathcurveto{\pgfqpoint{6.164055in}{0.621488in}}{\pgfqpoint{6.153456in}{0.617097in}}{\pgfqpoint{6.145643in}{0.609284in}}%
\pgfpathcurveto{\pgfqpoint{6.137829in}{0.601470in}}{\pgfqpoint{6.133439in}{0.590871in}}{\pgfqpoint{6.133439in}{0.579821in}}%
\pgfpathcurveto{\pgfqpoint{6.133439in}{0.568771in}}{\pgfqpoint{6.137829in}{0.558172in}}{\pgfqpoint{6.145643in}{0.550358in}}%
\pgfpathcurveto{\pgfqpoint{6.153456in}{0.542545in}}{\pgfqpoint{6.164055in}{0.538154in}}{\pgfqpoint{6.175105in}{0.538154in}}%
\pgfusepath{stroke}%
\end{pgfscope}%
\begin{pgfscope}%
\pgfpathrectangle{\pgfqpoint{0.847223in}{0.554012in}}{\pgfqpoint{6.200000in}{4.620000in}}%
\pgfusepath{clip}%
\pgfsetbuttcap%
\pgfsetroundjoin%
\pgfsetlinewidth{1.003750pt}%
\definecolor{currentstroke}{rgb}{1.000000,0.000000,0.000000}%
\pgfsetstrokecolor{currentstroke}%
\pgfsetdash{}{0pt}%
\pgfpathmoveto{\pgfqpoint{6.175105in}{0.538154in}}%
\pgfpathcurveto{\pgfqpoint{6.186156in}{0.538154in}}{\pgfqpoint{6.196755in}{0.542545in}}{\pgfqpoint{6.204568in}{0.550358in}}%
\pgfpathcurveto{\pgfqpoint{6.212382in}{0.558172in}}{\pgfqpoint{6.216772in}{0.568771in}}{\pgfqpoint{6.216772in}{0.579821in}}%
\pgfpathcurveto{\pgfqpoint{6.216772in}{0.590871in}}{\pgfqpoint{6.212382in}{0.601470in}}{\pgfqpoint{6.204568in}{0.609284in}}%
\pgfpathcurveto{\pgfqpoint{6.196755in}{0.617097in}}{\pgfqpoint{6.186156in}{0.621488in}}{\pgfqpoint{6.175105in}{0.621488in}}%
\pgfpathcurveto{\pgfqpoint{6.164055in}{0.621488in}}{\pgfqpoint{6.153456in}{0.617097in}}{\pgfqpoint{6.145643in}{0.609284in}}%
\pgfpathcurveto{\pgfqpoint{6.137829in}{0.601470in}}{\pgfqpoint{6.133439in}{0.590871in}}{\pgfqpoint{6.133439in}{0.579821in}}%
\pgfpathcurveto{\pgfqpoint{6.133439in}{0.568771in}}{\pgfqpoint{6.137829in}{0.558172in}}{\pgfqpoint{6.145643in}{0.550358in}}%
\pgfpathcurveto{\pgfqpoint{6.153456in}{0.542545in}}{\pgfqpoint{6.164055in}{0.538154in}}{\pgfqpoint{6.175105in}{0.538154in}}%
\pgfusepath{stroke}%
\end{pgfscope}%
\begin{pgfscope}%
\pgfpathrectangle{\pgfqpoint{0.847223in}{0.554012in}}{\pgfqpoint{6.200000in}{4.620000in}}%
\pgfusepath{clip}%
\pgfsetbuttcap%
\pgfsetroundjoin%
\pgfsetlinewidth{1.003750pt}%
\definecolor{currentstroke}{rgb}{1.000000,0.000000,0.000000}%
\pgfsetstrokecolor{currentstroke}%
\pgfsetdash{}{0pt}%
\pgfpathmoveto{\pgfqpoint{6.183915in}{0.537488in}}%
\pgfpathcurveto{\pgfqpoint{6.194965in}{0.537488in}}{\pgfqpoint{6.205564in}{0.541879in}}{\pgfqpoint{6.213378in}{0.549692in}}%
\pgfpathcurveto{\pgfqpoint{6.221191in}{0.557506in}}{\pgfqpoint{6.225581in}{0.568105in}}{\pgfqpoint{6.225581in}{0.579155in}}%
\pgfpathcurveto{\pgfqpoint{6.225581in}{0.590205in}}{\pgfqpoint{6.221191in}{0.600804in}}{\pgfqpoint{6.213378in}{0.608618in}}%
\pgfpathcurveto{\pgfqpoint{6.205564in}{0.616431in}}{\pgfqpoint{6.194965in}{0.620822in}}{\pgfqpoint{6.183915in}{0.620822in}}%
\pgfpathcurveto{\pgfqpoint{6.172865in}{0.620822in}}{\pgfqpoint{6.162266in}{0.616431in}}{\pgfqpoint{6.154452in}{0.608618in}}%
\pgfpathcurveto{\pgfqpoint{6.146638in}{0.600804in}}{\pgfqpoint{6.142248in}{0.590205in}}{\pgfqpoint{6.142248in}{0.579155in}}%
\pgfpathcurveto{\pgfqpoint{6.142248in}{0.568105in}}{\pgfqpoint{6.146638in}{0.557506in}}{\pgfqpoint{6.154452in}{0.549692in}}%
\pgfpathcurveto{\pgfqpoint{6.162266in}{0.541879in}}{\pgfqpoint{6.172865in}{0.537488in}}{\pgfqpoint{6.183915in}{0.537488in}}%
\pgfusepath{stroke}%
\end{pgfscope}%
\begin{pgfscope}%
\pgfpathrectangle{\pgfqpoint{0.847223in}{0.554012in}}{\pgfqpoint{6.200000in}{4.620000in}}%
\pgfusepath{clip}%
\pgfsetbuttcap%
\pgfsetroundjoin%
\pgfsetlinewidth{1.003750pt}%
\definecolor{currentstroke}{rgb}{1.000000,0.000000,0.000000}%
\pgfsetstrokecolor{currentstroke}%
\pgfsetdash{}{0pt}%
\pgfpathmoveto{\pgfqpoint{6.192724in}{0.536848in}}%
\pgfpathcurveto{\pgfqpoint{6.203774in}{0.536848in}}{\pgfqpoint{6.214373in}{0.541238in}}{\pgfqpoint{6.222187in}{0.549052in}}%
\pgfpathcurveto{\pgfqpoint{6.230000in}{0.556865in}}{\pgfqpoint{6.234391in}{0.567464in}}{\pgfqpoint{6.234391in}{0.578515in}}%
\pgfpathcurveto{\pgfqpoint{6.234391in}{0.589565in}}{\pgfqpoint{6.230000in}{0.600164in}}{\pgfqpoint{6.222187in}{0.607977in}}%
\pgfpathcurveto{\pgfqpoint{6.214373in}{0.615791in}}{\pgfqpoint{6.203774in}{0.620181in}}{\pgfqpoint{6.192724in}{0.620181in}}%
\pgfpathcurveto{\pgfqpoint{6.181674in}{0.620181in}}{\pgfqpoint{6.171075in}{0.615791in}}{\pgfqpoint{6.163261in}{0.607977in}}%
\pgfpathcurveto{\pgfqpoint{6.155448in}{0.600164in}}{\pgfqpoint{6.151057in}{0.589565in}}{\pgfqpoint{6.151057in}{0.578515in}}%
\pgfpathcurveto{\pgfqpoint{6.151057in}{0.567464in}}{\pgfqpoint{6.155448in}{0.556865in}}{\pgfqpoint{6.163261in}{0.549052in}}%
\pgfpathcurveto{\pgfqpoint{6.171075in}{0.541238in}}{\pgfqpoint{6.181674in}{0.536848in}}{\pgfqpoint{6.192724in}{0.536848in}}%
\pgfusepath{stroke}%
\end{pgfscope}%
\begin{pgfscope}%
\pgfpathrectangle{\pgfqpoint{0.847223in}{0.554012in}}{\pgfqpoint{6.200000in}{4.620000in}}%
\pgfusepath{clip}%
\pgfsetbuttcap%
\pgfsetroundjoin%
\pgfsetlinewidth{1.003750pt}%
\definecolor{currentstroke}{rgb}{1.000000,0.000000,0.000000}%
\pgfsetstrokecolor{currentstroke}%
\pgfsetdash{}{0pt}%
\pgfpathmoveto{\pgfqpoint{6.201533in}{0.536231in}}%
\pgfpathcurveto{\pgfqpoint{6.212583in}{0.536231in}}{\pgfqpoint{6.223182in}{0.540622in}}{\pgfqpoint{6.230996in}{0.548435in}}%
\pgfpathcurveto{\pgfqpoint{6.238810in}{0.556249in}}{\pgfqpoint{6.243200in}{0.566848in}}{\pgfqpoint{6.243200in}{0.577898in}}%
\pgfpathcurveto{\pgfqpoint{6.243200in}{0.588948in}}{\pgfqpoint{6.238810in}{0.599547in}}{\pgfqpoint{6.230996in}{0.607361in}}%
\pgfpathcurveto{\pgfqpoint{6.223182in}{0.615174in}}{\pgfqpoint{6.212583in}{0.619565in}}{\pgfqpoint{6.201533in}{0.619565in}}%
\pgfpathcurveto{\pgfqpoint{6.190483in}{0.619565in}}{\pgfqpoint{6.179884in}{0.615174in}}{\pgfqpoint{6.172071in}{0.607361in}}%
\pgfpathcurveto{\pgfqpoint{6.164257in}{0.599547in}}{\pgfqpoint{6.159867in}{0.588948in}}{\pgfqpoint{6.159867in}{0.577898in}}%
\pgfpathcurveto{\pgfqpoint{6.159867in}{0.566848in}}{\pgfqpoint{6.164257in}{0.556249in}}{\pgfqpoint{6.172071in}{0.548435in}}%
\pgfpathcurveto{\pgfqpoint{6.179884in}{0.540622in}}{\pgfqpoint{6.190483in}{0.536231in}}{\pgfqpoint{6.201533in}{0.536231in}}%
\pgfusepath{stroke}%
\end{pgfscope}%
\begin{pgfscope}%
\pgfpathrectangle{\pgfqpoint{0.847223in}{0.554012in}}{\pgfqpoint{6.200000in}{4.620000in}}%
\pgfusepath{clip}%
\pgfsetbuttcap%
\pgfsetroundjoin%
\pgfsetlinewidth{1.003750pt}%
\definecolor{currentstroke}{rgb}{1.000000,0.000000,0.000000}%
\pgfsetstrokecolor{currentstroke}%
\pgfsetdash{}{0pt}%
\pgfpathmoveto{\pgfqpoint{6.210343in}{0.535637in}}%
\pgfpathcurveto{\pgfqpoint{6.221393in}{0.535637in}}{\pgfqpoint{6.231992in}{0.540027in}}{\pgfqpoint{6.239805in}{0.547841in}}%
\pgfpathcurveto{\pgfqpoint{6.247619in}{0.555654in}}{\pgfqpoint{6.252009in}{0.566253in}}{\pgfqpoint{6.252009in}{0.577303in}}%
\pgfpathcurveto{\pgfqpoint{6.252009in}{0.588353in}}{\pgfqpoint{6.247619in}{0.598952in}}{\pgfqpoint{6.239805in}{0.606766in}}%
\pgfpathcurveto{\pgfqpoint{6.231992in}{0.614580in}}{\pgfqpoint{6.221393in}{0.618970in}}{\pgfqpoint{6.210343in}{0.618970in}}%
\pgfpathcurveto{\pgfqpoint{6.199292in}{0.618970in}}{\pgfqpoint{6.188693in}{0.614580in}}{\pgfqpoint{6.180880in}{0.606766in}}%
\pgfpathcurveto{\pgfqpoint{6.173066in}{0.598952in}}{\pgfqpoint{6.168676in}{0.588353in}}{\pgfqpoint{6.168676in}{0.577303in}}%
\pgfpathcurveto{\pgfqpoint{6.168676in}{0.566253in}}{\pgfqpoint{6.173066in}{0.555654in}}{\pgfqpoint{6.180880in}{0.547841in}}%
\pgfpathcurveto{\pgfqpoint{6.188693in}{0.540027in}}{\pgfqpoint{6.199292in}{0.535637in}}{\pgfqpoint{6.210343in}{0.535637in}}%
\pgfusepath{stroke}%
\end{pgfscope}%
\begin{pgfscope}%
\pgfpathrectangle{\pgfqpoint{0.847223in}{0.554012in}}{\pgfqpoint{6.200000in}{4.620000in}}%
\pgfusepath{clip}%
\pgfsetbuttcap%
\pgfsetroundjoin%
\pgfsetlinewidth{1.003750pt}%
\definecolor{currentstroke}{rgb}{1.000000,0.000000,0.000000}%
\pgfsetstrokecolor{currentstroke}%
\pgfsetdash{}{0pt}%
\pgfpathmoveto{\pgfqpoint{6.219152in}{0.535063in}}%
\pgfpathcurveto{\pgfqpoint{6.230202in}{0.535063in}}{\pgfqpoint{6.240801in}{0.539453in}}{\pgfqpoint{6.248615in}{0.547266in}}%
\pgfpathcurveto{\pgfqpoint{6.256428in}{0.555080in}}{\pgfqpoint{6.260819in}{0.565679in}}{\pgfqpoint{6.260819in}{0.576729in}}%
\pgfpathcurveto{\pgfqpoint{6.260819in}{0.587779in}}{\pgfqpoint{6.256428in}{0.598378in}}{\pgfqpoint{6.248615in}{0.606192in}}%
\pgfpathcurveto{\pgfqpoint{6.240801in}{0.614006in}}{\pgfqpoint{6.230202in}{0.618396in}}{\pgfqpoint{6.219152in}{0.618396in}}%
\pgfpathcurveto{\pgfqpoint{6.208102in}{0.618396in}}{\pgfqpoint{6.197503in}{0.614006in}}{\pgfqpoint{6.189689in}{0.606192in}}%
\pgfpathcurveto{\pgfqpoint{6.181875in}{0.598378in}}{\pgfqpoint{6.177485in}{0.587779in}}{\pgfqpoint{6.177485in}{0.576729in}}%
\pgfpathcurveto{\pgfqpoint{6.177485in}{0.565679in}}{\pgfqpoint{6.181875in}{0.555080in}}{\pgfqpoint{6.189689in}{0.547266in}}%
\pgfpathcurveto{\pgfqpoint{6.197503in}{0.539453in}}{\pgfqpoint{6.208102in}{0.535063in}}{\pgfqpoint{6.219152in}{0.535063in}}%
\pgfusepath{stroke}%
\end{pgfscope}%
\begin{pgfscope}%
\pgfpathrectangle{\pgfqpoint{0.847223in}{0.554012in}}{\pgfqpoint{6.200000in}{4.620000in}}%
\pgfusepath{clip}%
\pgfsetbuttcap%
\pgfsetroundjoin%
\pgfsetlinewidth{1.003750pt}%
\definecolor{currentstroke}{rgb}{1.000000,0.000000,0.000000}%
\pgfsetstrokecolor{currentstroke}%
\pgfsetdash{}{0pt}%
\pgfpathmoveto{\pgfqpoint{6.227961in}{0.534508in}}%
\pgfpathcurveto{\pgfqpoint{6.239011in}{0.534508in}}{\pgfqpoint{6.249610in}{0.538898in}}{\pgfqpoint{6.257424in}{0.546712in}}%
\pgfpathcurveto{\pgfqpoint{6.265238in}{0.554525in}}{\pgfqpoint{6.269628in}{0.565124in}}{\pgfqpoint{6.269628in}{0.576175in}}%
\pgfpathcurveto{\pgfqpoint{6.269628in}{0.587225in}}{\pgfqpoint{6.265238in}{0.597824in}}{\pgfqpoint{6.257424in}{0.605637in}}%
\pgfpathcurveto{\pgfqpoint{6.249610in}{0.613451in}}{\pgfqpoint{6.239011in}{0.617841in}}{\pgfqpoint{6.227961in}{0.617841in}}%
\pgfpathcurveto{\pgfqpoint{6.216911in}{0.617841in}}{\pgfqpoint{6.206312in}{0.613451in}}{\pgfqpoint{6.198498in}{0.605637in}}%
\pgfpathcurveto{\pgfqpoint{6.190685in}{0.597824in}}{\pgfqpoint{6.186294in}{0.587225in}}{\pgfqpoint{6.186294in}{0.576175in}}%
\pgfpathcurveto{\pgfqpoint{6.186294in}{0.565124in}}{\pgfqpoint{6.190685in}{0.554525in}}{\pgfqpoint{6.198498in}{0.546712in}}%
\pgfpathcurveto{\pgfqpoint{6.206312in}{0.538898in}}{\pgfqpoint{6.216911in}{0.534508in}}{\pgfqpoint{6.227961in}{0.534508in}}%
\pgfusepath{stroke}%
\end{pgfscope}%
\begin{pgfscope}%
\pgfpathrectangle{\pgfqpoint{0.847223in}{0.554012in}}{\pgfqpoint{6.200000in}{4.620000in}}%
\pgfusepath{clip}%
\pgfsetbuttcap%
\pgfsetroundjoin%
\pgfsetlinewidth{1.003750pt}%
\definecolor{currentstroke}{rgb}{1.000000,0.000000,0.000000}%
\pgfsetstrokecolor{currentstroke}%
\pgfsetdash{}{0pt}%
\pgfpathmoveto{\pgfqpoint{6.236770in}{0.533971in}}%
\pgfpathcurveto{\pgfqpoint{6.247821in}{0.533971in}}{\pgfqpoint{6.258420in}{0.538361in}}{\pgfqpoint{6.266233in}{0.546175in}}%
\pgfpathcurveto{\pgfqpoint{6.274047in}{0.553989in}}{\pgfqpoint{6.278437in}{0.564588in}}{\pgfqpoint{6.278437in}{0.575638in}}%
\pgfpathcurveto{\pgfqpoint{6.278437in}{0.586688in}}{\pgfqpoint{6.274047in}{0.597287in}}{\pgfqpoint{6.266233in}{0.605101in}}%
\pgfpathcurveto{\pgfqpoint{6.258420in}{0.612914in}}{\pgfqpoint{6.247821in}{0.617304in}}{\pgfqpoint{6.236770in}{0.617304in}}%
\pgfpathcurveto{\pgfqpoint{6.225720in}{0.617304in}}{\pgfqpoint{6.215121in}{0.612914in}}{\pgfqpoint{6.207308in}{0.605101in}}%
\pgfpathcurveto{\pgfqpoint{6.199494in}{0.597287in}}{\pgfqpoint{6.195104in}{0.586688in}}{\pgfqpoint{6.195104in}{0.575638in}}%
\pgfpathcurveto{\pgfqpoint{6.195104in}{0.564588in}}{\pgfqpoint{6.199494in}{0.553989in}}{\pgfqpoint{6.207308in}{0.546175in}}%
\pgfpathcurveto{\pgfqpoint{6.215121in}{0.538361in}}{\pgfqpoint{6.225720in}{0.533971in}}{\pgfqpoint{6.236770in}{0.533971in}}%
\pgfusepath{stroke}%
\end{pgfscope}%
\begin{pgfscope}%
\pgfpathrectangle{\pgfqpoint{0.847223in}{0.554012in}}{\pgfqpoint{6.200000in}{4.620000in}}%
\pgfusepath{clip}%
\pgfsetbuttcap%
\pgfsetroundjoin%
\pgfsetlinewidth{1.003750pt}%
\definecolor{currentstroke}{rgb}{1.000000,0.000000,0.000000}%
\pgfsetstrokecolor{currentstroke}%
\pgfsetdash{}{0pt}%
\pgfpathmoveto{\pgfqpoint{6.245580in}{0.533451in}}%
\pgfpathcurveto{\pgfqpoint{6.256630in}{0.533451in}}{\pgfqpoint{6.267229in}{0.537842in}}{\pgfqpoint{6.275042in}{0.545655in}}%
\pgfpathcurveto{\pgfqpoint{6.282856in}{0.553469in}}{\pgfqpoint{6.287246in}{0.564068in}}{\pgfqpoint{6.287246in}{0.575118in}}%
\pgfpathcurveto{\pgfqpoint{6.287246in}{0.586168in}}{\pgfqpoint{6.282856in}{0.596767in}}{\pgfqpoint{6.275042in}{0.604581in}}%
\pgfpathcurveto{\pgfqpoint{6.267229in}{0.612395in}}{\pgfqpoint{6.256630in}{0.616785in}}{\pgfqpoint{6.245580in}{0.616785in}}%
\pgfpathcurveto{\pgfqpoint{6.234530in}{0.616785in}}{\pgfqpoint{6.223930in}{0.612395in}}{\pgfqpoint{6.216117in}{0.604581in}}%
\pgfpathcurveto{\pgfqpoint{6.208303in}{0.596767in}}{\pgfqpoint{6.203913in}{0.586168in}}{\pgfqpoint{6.203913in}{0.575118in}}%
\pgfpathcurveto{\pgfqpoint{6.203913in}{0.564068in}}{\pgfqpoint{6.208303in}{0.553469in}}{\pgfqpoint{6.216117in}{0.545655in}}%
\pgfpathcurveto{\pgfqpoint{6.223930in}{0.537842in}}{\pgfqpoint{6.234530in}{0.533451in}}{\pgfqpoint{6.245580in}{0.533451in}}%
\pgfusepath{stroke}%
\end{pgfscope}%
\begin{pgfscope}%
\pgfpathrectangle{\pgfqpoint{0.847223in}{0.554012in}}{\pgfqpoint{6.200000in}{4.620000in}}%
\pgfusepath{clip}%
\pgfsetbuttcap%
\pgfsetroundjoin%
\pgfsetlinewidth{1.003750pt}%
\definecolor{currentstroke}{rgb}{1.000000,0.000000,0.000000}%
\pgfsetstrokecolor{currentstroke}%
\pgfsetdash{}{0pt}%
\pgfpathmoveto{\pgfqpoint{6.254389in}{0.532948in}}%
\pgfpathcurveto{\pgfqpoint{6.265439in}{0.532948in}}{\pgfqpoint{6.276038in}{0.537338in}}{\pgfqpoint{6.283852in}{0.545152in}}%
\pgfpathcurveto{\pgfqpoint{6.291665in}{0.552965in}}{\pgfqpoint{6.296056in}{0.563564in}}{\pgfqpoint{6.296056in}{0.574614in}}%
\pgfpathcurveto{\pgfqpoint{6.296056in}{0.585665in}}{\pgfqpoint{6.291665in}{0.596264in}}{\pgfqpoint{6.283852in}{0.604077in}}%
\pgfpathcurveto{\pgfqpoint{6.276038in}{0.611891in}}{\pgfqpoint{6.265439in}{0.616281in}}{\pgfqpoint{6.254389in}{0.616281in}}%
\pgfpathcurveto{\pgfqpoint{6.243339in}{0.616281in}}{\pgfqpoint{6.232740in}{0.611891in}}{\pgfqpoint{6.224926in}{0.604077in}}%
\pgfpathcurveto{\pgfqpoint{6.217113in}{0.596264in}}{\pgfqpoint{6.212722in}{0.585665in}}{\pgfqpoint{6.212722in}{0.574614in}}%
\pgfpathcurveto{\pgfqpoint{6.212722in}{0.563564in}}{\pgfqpoint{6.217113in}{0.552965in}}{\pgfqpoint{6.224926in}{0.545152in}}%
\pgfpathcurveto{\pgfqpoint{6.232740in}{0.537338in}}{\pgfqpoint{6.243339in}{0.532948in}}{\pgfqpoint{6.254389in}{0.532948in}}%
\pgfusepath{stroke}%
\end{pgfscope}%
\begin{pgfscope}%
\pgfpathrectangle{\pgfqpoint{0.847223in}{0.554012in}}{\pgfqpoint{6.200000in}{4.620000in}}%
\pgfusepath{clip}%
\pgfsetbuttcap%
\pgfsetroundjoin%
\pgfsetlinewidth{1.003750pt}%
\definecolor{currentstroke}{rgb}{1.000000,0.000000,0.000000}%
\pgfsetstrokecolor{currentstroke}%
\pgfsetdash{}{0pt}%
\pgfpathmoveto{\pgfqpoint{6.263198in}{0.532459in}}%
\pgfpathcurveto{\pgfqpoint{6.274248in}{0.532459in}}{\pgfqpoint{6.284847in}{0.536849in}}{\pgfqpoint{6.292661in}{0.544663in}}%
\pgfpathcurveto{\pgfqpoint{6.300475in}{0.552477in}}{\pgfqpoint{6.304865in}{0.563076in}}{\pgfqpoint{6.304865in}{0.574126in}}%
\pgfpathcurveto{\pgfqpoint{6.304865in}{0.585176in}}{\pgfqpoint{6.300475in}{0.595775in}}{\pgfqpoint{6.292661in}{0.603589in}}%
\pgfpathcurveto{\pgfqpoint{6.284847in}{0.611402in}}{\pgfqpoint{6.274248in}{0.615792in}}{\pgfqpoint{6.263198in}{0.615792in}}%
\pgfpathcurveto{\pgfqpoint{6.252148in}{0.615792in}}{\pgfqpoint{6.241549in}{0.611402in}}{\pgfqpoint{6.233735in}{0.603589in}}%
\pgfpathcurveto{\pgfqpoint{6.225922in}{0.595775in}}{\pgfqpoint{6.221532in}{0.585176in}}{\pgfqpoint{6.221532in}{0.574126in}}%
\pgfpathcurveto{\pgfqpoint{6.221532in}{0.563076in}}{\pgfqpoint{6.225922in}{0.552477in}}{\pgfqpoint{6.233735in}{0.544663in}}%
\pgfpathcurveto{\pgfqpoint{6.241549in}{0.536849in}}{\pgfqpoint{6.252148in}{0.532459in}}{\pgfqpoint{6.263198in}{0.532459in}}%
\pgfusepath{stroke}%
\end{pgfscope}%
\begin{pgfscope}%
\pgfpathrectangle{\pgfqpoint{0.847223in}{0.554012in}}{\pgfqpoint{6.200000in}{4.620000in}}%
\pgfusepath{clip}%
\pgfsetbuttcap%
\pgfsetroundjoin%
\pgfsetlinewidth{1.003750pt}%
\definecolor{currentstroke}{rgb}{1.000000,0.000000,0.000000}%
\pgfsetstrokecolor{currentstroke}%
\pgfsetdash{}{0pt}%
\pgfpathmoveto{\pgfqpoint{6.272007in}{0.531985in}}%
\pgfpathcurveto{\pgfqpoint{6.283058in}{0.531985in}}{\pgfqpoint{6.293657in}{0.536375in}}{\pgfqpoint{6.301470in}{0.544189in}}%
\pgfpathcurveto{\pgfqpoint{6.309284in}{0.552002in}}{\pgfqpoint{6.313674in}{0.562601in}}{\pgfqpoint{6.313674in}{0.573651in}}%
\pgfpathcurveto{\pgfqpoint{6.313674in}{0.584701in}}{\pgfqpoint{6.309284in}{0.595300in}}{\pgfqpoint{6.301470in}{0.603114in}}%
\pgfpathcurveto{\pgfqpoint{6.293657in}{0.610928in}}{\pgfqpoint{6.283058in}{0.615318in}}{\pgfqpoint{6.272007in}{0.615318in}}%
\pgfpathcurveto{\pgfqpoint{6.260957in}{0.615318in}}{\pgfqpoint{6.250358in}{0.610928in}}{\pgfqpoint{6.242545in}{0.603114in}}%
\pgfpathcurveto{\pgfqpoint{6.234731in}{0.595300in}}{\pgfqpoint{6.230341in}{0.584701in}}{\pgfqpoint{6.230341in}{0.573651in}}%
\pgfpathcurveto{\pgfqpoint{6.230341in}{0.562601in}}{\pgfqpoint{6.234731in}{0.552002in}}{\pgfqpoint{6.242545in}{0.544189in}}%
\pgfpathcurveto{\pgfqpoint{6.250358in}{0.536375in}}{\pgfqpoint{6.260957in}{0.531985in}}{\pgfqpoint{6.272007in}{0.531985in}}%
\pgfusepath{stroke}%
\end{pgfscope}%
\begin{pgfscope}%
\pgfpathrectangle{\pgfqpoint{0.847223in}{0.554012in}}{\pgfqpoint{6.200000in}{4.620000in}}%
\pgfusepath{clip}%
\pgfsetbuttcap%
\pgfsetroundjoin%
\pgfsetlinewidth{1.003750pt}%
\definecolor{currentstroke}{rgb}{1.000000,0.000000,0.000000}%
\pgfsetstrokecolor{currentstroke}%
\pgfsetdash{}{0pt}%
\pgfpathmoveto{\pgfqpoint{6.280817in}{0.531524in}}%
\pgfpathcurveto{\pgfqpoint{6.291867in}{0.531524in}}{\pgfqpoint{6.302466in}{0.535914in}}{\pgfqpoint{6.310280in}{0.543728in}}%
\pgfpathcurveto{\pgfqpoint{6.318093in}{0.551541in}}{\pgfqpoint{6.322483in}{0.562140in}}{\pgfqpoint{6.322483in}{0.573190in}}%
\pgfpathcurveto{\pgfqpoint{6.322483in}{0.584241in}}{\pgfqpoint{6.318093in}{0.594840in}}{\pgfqpoint{6.310280in}{0.602653in}}%
\pgfpathcurveto{\pgfqpoint{6.302466in}{0.610467in}}{\pgfqpoint{6.291867in}{0.614857in}}{\pgfqpoint{6.280817in}{0.614857in}}%
\pgfpathcurveto{\pgfqpoint{6.269767in}{0.614857in}}{\pgfqpoint{6.259168in}{0.610467in}}{\pgfqpoint{6.251354in}{0.602653in}}%
\pgfpathcurveto{\pgfqpoint{6.243540in}{0.594840in}}{\pgfqpoint{6.239150in}{0.584241in}}{\pgfqpoint{6.239150in}{0.573190in}}%
\pgfpathcurveto{\pgfqpoint{6.239150in}{0.562140in}}{\pgfqpoint{6.243540in}{0.551541in}}{\pgfqpoint{6.251354in}{0.543728in}}%
\pgfpathcurveto{\pgfqpoint{6.259168in}{0.535914in}}{\pgfqpoint{6.269767in}{0.531524in}}{\pgfqpoint{6.280817in}{0.531524in}}%
\pgfusepath{stroke}%
\end{pgfscope}%
\begin{pgfscope}%
\pgfpathrectangle{\pgfqpoint{0.847223in}{0.554012in}}{\pgfqpoint{6.200000in}{4.620000in}}%
\pgfusepath{clip}%
\pgfsetbuttcap%
\pgfsetroundjoin%
\pgfsetlinewidth{1.003750pt}%
\definecolor{currentstroke}{rgb}{1.000000,0.000000,0.000000}%
\pgfsetstrokecolor{currentstroke}%
\pgfsetdash{}{0pt}%
\pgfpathmoveto{\pgfqpoint{6.289626in}{0.531076in}}%
\pgfpathcurveto{\pgfqpoint{6.300676in}{0.531076in}}{\pgfqpoint{6.311275in}{0.535466in}}{\pgfqpoint{6.319089in}{0.543279in}}%
\pgfpathcurveto{\pgfqpoint{6.326902in}{0.551093in}}{\pgfqpoint{6.331293in}{0.561692in}}{\pgfqpoint{6.331293in}{0.572742in}}%
\pgfpathcurveto{\pgfqpoint{6.331293in}{0.583792in}}{\pgfqpoint{6.326902in}{0.594391in}}{\pgfqpoint{6.319089in}{0.602205in}}%
\pgfpathcurveto{\pgfqpoint{6.311275in}{0.610019in}}{\pgfqpoint{6.300676in}{0.614409in}}{\pgfqpoint{6.289626in}{0.614409in}}%
\pgfpathcurveto{\pgfqpoint{6.278576in}{0.614409in}}{\pgfqpoint{6.267977in}{0.610019in}}{\pgfqpoint{6.260163in}{0.602205in}}%
\pgfpathcurveto{\pgfqpoint{6.252350in}{0.594391in}}{\pgfqpoint{6.247959in}{0.583792in}}{\pgfqpoint{6.247959in}{0.572742in}}%
\pgfpathcurveto{\pgfqpoint{6.247959in}{0.561692in}}{\pgfqpoint{6.252350in}{0.551093in}}{\pgfqpoint{6.260163in}{0.543279in}}%
\pgfpathcurveto{\pgfqpoint{6.267977in}{0.535466in}}{\pgfqpoint{6.278576in}{0.531076in}}{\pgfqpoint{6.289626in}{0.531076in}}%
\pgfusepath{stroke}%
\end{pgfscope}%
\begin{pgfscope}%
\pgfpathrectangle{\pgfqpoint{0.847223in}{0.554012in}}{\pgfqpoint{6.200000in}{4.620000in}}%
\pgfusepath{clip}%
\pgfsetbuttcap%
\pgfsetroundjoin%
\pgfsetlinewidth{1.003750pt}%
\definecolor{currentstroke}{rgb}{1.000000,0.000000,0.000000}%
\pgfsetstrokecolor{currentstroke}%
\pgfsetdash{}{0pt}%
\pgfpathmoveto{\pgfqpoint{6.298435in}{0.530640in}}%
\pgfpathcurveto{\pgfqpoint{6.309485in}{0.530640in}}{\pgfqpoint{6.320084in}{0.535030in}}{\pgfqpoint{6.327898in}{0.542843in}}%
\pgfpathcurveto{\pgfqpoint{6.335712in}{0.550657in}}{\pgfqpoint{6.340102in}{0.561256in}}{\pgfqpoint{6.340102in}{0.572306in}}%
\pgfpathcurveto{\pgfqpoint{6.340102in}{0.583356in}}{\pgfqpoint{6.335712in}{0.593955in}}{\pgfqpoint{6.327898in}{0.601769in}}%
\pgfpathcurveto{\pgfqpoint{6.320084in}{0.609583in}}{\pgfqpoint{6.309485in}{0.613973in}}{\pgfqpoint{6.298435in}{0.613973in}}%
\pgfpathcurveto{\pgfqpoint{6.287385in}{0.613973in}}{\pgfqpoint{6.276786in}{0.609583in}}{\pgfqpoint{6.268972in}{0.601769in}}%
\pgfpathcurveto{\pgfqpoint{6.261159in}{0.593955in}}{\pgfqpoint{6.256769in}{0.583356in}}{\pgfqpoint{6.256769in}{0.572306in}}%
\pgfpathcurveto{\pgfqpoint{6.256769in}{0.561256in}}{\pgfqpoint{6.261159in}{0.550657in}}{\pgfqpoint{6.268972in}{0.542843in}}%
\pgfpathcurveto{\pgfqpoint{6.276786in}{0.535030in}}{\pgfqpoint{6.287385in}{0.530640in}}{\pgfqpoint{6.298435in}{0.530640in}}%
\pgfusepath{stroke}%
\end{pgfscope}%
\begin{pgfscope}%
\pgfpathrectangle{\pgfqpoint{0.847223in}{0.554012in}}{\pgfqpoint{6.200000in}{4.620000in}}%
\pgfusepath{clip}%
\pgfsetbuttcap%
\pgfsetroundjoin%
\pgfsetlinewidth{1.003750pt}%
\definecolor{currentstroke}{rgb}{1.000000,0.000000,0.000000}%
\pgfsetstrokecolor{currentstroke}%
\pgfsetdash{}{0pt}%
\pgfpathmoveto{\pgfqpoint{6.307245in}{0.530215in}}%
\pgfpathcurveto{\pgfqpoint{6.318295in}{0.530215in}}{\pgfqpoint{6.328894in}{0.534605in}}{\pgfqpoint{6.336707in}{0.542419in}}%
\pgfpathcurveto{\pgfqpoint{6.344521in}{0.550233in}}{\pgfqpoint{6.348911in}{0.560832in}}{\pgfqpoint{6.348911in}{0.571882in}}%
\pgfpathcurveto{\pgfqpoint{6.348911in}{0.582932in}}{\pgfqpoint{6.344521in}{0.593531in}}{\pgfqpoint{6.336707in}{0.601345in}}%
\pgfpathcurveto{\pgfqpoint{6.328894in}{0.609158in}}{\pgfqpoint{6.318295in}{0.613548in}}{\pgfqpoint{6.307245in}{0.613548in}}%
\pgfpathcurveto{\pgfqpoint{6.296194in}{0.613548in}}{\pgfqpoint{6.285595in}{0.609158in}}{\pgfqpoint{6.277782in}{0.601345in}}%
\pgfpathcurveto{\pgfqpoint{6.269968in}{0.593531in}}{\pgfqpoint{6.265578in}{0.582932in}}{\pgfqpoint{6.265578in}{0.571882in}}%
\pgfpathcurveto{\pgfqpoint{6.265578in}{0.560832in}}{\pgfqpoint{6.269968in}{0.550233in}}{\pgfqpoint{6.277782in}{0.542419in}}%
\pgfpathcurveto{\pgfqpoint{6.285595in}{0.534605in}}{\pgfqpoint{6.296194in}{0.530215in}}{\pgfqpoint{6.307245in}{0.530215in}}%
\pgfusepath{stroke}%
\end{pgfscope}%
\begin{pgfscope}%
\pgfpathrectangle{\pgfqpoint{0.847223in}{0.554012in}}{\pgfqpoint{6.200000in}{4.620000in}}%
\pgfusepath{clip}%
\pgfsetbuttcap%
\pgfsetroundjoin%
\pgfsetlinewidth{1.003750pt}%
\definecolor{currentstroke}{rgb}{1.000000,0.000000,0.000000}%
\pgfsetstrokecolor{currentstroke}%
\pgfsetdash{}{0pt}%
\pgfpathmoveto{\pgfqpoint{6.316054in}{0.529802in}}%
\pgfpathcurveto{\pgfqpoint{6.327104in}{0.529802in}}{\pgfqpoint{6.337703in}{0.534192in}}{\pgfqpoint{6.345517in}{0.542006in}}%
\pgfpathcurveto{\pgfqpoint{6.353330in}{0.549819in}}{\pgfqpoint{6.357720in}{0.560418in}}{\pgfqpoint{6.357720in}{0.571468in}}%
\pgfpathcurveto{\pgfqpoint{6.357720in}{0.582518in}}{\pgfqpoint{6.353330in}{0.593117in}}{\pgfqpoint{6.345517in}{0.600931in}}%
\pgfpathcurveto{\pgfqpoint{6.337703in}{0.608745in}}{\pgfqpoint{6.327104in}{0.613135in}}{\pgfqpoint{6.316054in}{0.613135in}}%
\pgfpathcurveto{\pgfqpoint{6.305004in}{0.613135in}}{\pgfqpoint{6.294405in}{0.608745in}}{\pgfqpoint{6.286591in}{0.600931in}}%
\pgfpathcurveto{\pgfqpoint{6.278777in}{0.593117in}}{\pgfqpoint{6.274387in}{0.582518in}}{\pgfqpoint{6.274387in}{0.571468in}}%
\pgfpathcurveto{\pgfqpoint{6.274387in}{0.560418in}}{\pgfqpoint{6.278777in}{0.549819in}}{\pgfqpoint{6.286591in}{0.542006in}}%
\pgfpathcurveto{\pgfqpoint{6.294405in}{0.534192in}}{\pgfqpoint{6.305004in}{0.529802in}}{\pgfqpoint{6.316054in}{0.529802in}}%
\pgfusepath{stroke}%
\end{pgfscope}%
\begin{pgfscope}%
\pgfpathrectangle{\pgfqpoint{0.847223in}{0.554012in}}{\pgfqpoint{6.200000in}{4.620000in}}%
\pgfusepath{clip}%
\pgfsetbuttcap%
\pgfsetroundjoin%
\pgfsetlinewidth{1.003750pt}%
\definecolor{currentstroke}{rgb}{1.000000,0.000000,0.000000}%
\pgfsetstrokecolor{currentstroke}%
\pgfsetdash{}{0pt}%
\pgfpathmoveto{\pgfqpoint{6.324863in}{0.529399in}}%
\pgfpathcurveto{\pgfqpoint{6.335913in}{0.529399in}}{\pgfqpoint{6.346512in}{0.533789in}}{\pgfqpoint{6.354326in}{0.541603in}}%
\pgfpathcurveto{\pgfqpoint{6.362139in}{0.549416in}}{\pgfqpoint{6.366530in}{0.560015in}}{\pgfqpoint{6.366530in}{0.571065in}}%
\pgfpathcurveto{\pgfqpoint{6.366530in}{0.582115in}}{\pgfqpoint{6.362139in}{0.592714in}}{\pgfqpoint{6.354326in}{0.600528in}}%
\pgfpathcurveto{\pgfqpoint{6.346512in}{0.608342in}}{\pgfqpoint{6.335913in}{0.612732in}}{\pgfqpoint{6.324863in}{0.612732in}}%
\pgfpathcurveto{\pgfqpoint{6.313813in}{0.612732in}}{\pgfqpoint{6.303214in}{0.608342in}}{\pgfqpoint{6.295400in}{0.600528in}}%
\pgfpathcurveto{\pgfqpoint{6.287587in}{0.592714in}}{\pgfqpoint{6.283196in}{0.582115in}}{\pgfqpoint{6.283196in}{0.571065in}}%
\pgfpathcurveto{\pgfqpoint{6.283196in}{0.560015in}}{\pgfqpoint{6.287587in}{0.549416in}}{\pgfqpoint{6.295400in}{0.541603in}}%
\pgfpathcurveto{\pgfqpoint{6.303214in}{0.533789in}}{\pgfqpoint{6.313813in}{0.529399in}}{\pgfqpoint{6.324863in}{0.529399in}}%
\pgfusepath{stroke}%
\end{pgfscope}%
\begin{pgfscope}%
\pgfpathrectangle{\pgfqpoint{0.847223in}{0.554012in}}{\pgfqpoint{6.200000in}{4.620000in}}%
\pgfusepath{clip}%
\pgfsetbuttcap%
\pgfsetroundjoin%
\pgfsetlinewidth{1.003750pt}%
\definecolor{currentstroke}{rgb}{1.000000,0.000000,0.000000}%
\pgfsetstrokecolor{currentstroke}%
\pgfsetdash{}{0pt}%
\pgfpathmoveto{\pgfqpoint{6.333672in}{0.529006in}}%
\pgfpathcurveto{\pgfqpoint{6.344722in}{0.529006in}}{\pgfqpoint{6.355321in}{0.533396in}}{\pgfqpoint{6.363135in}{0.541210in}}%
\pgfpathcurveto{\pgfqpoint{6.370949in}{0.549023in}}{\pgfqpoint{6.375339in}{0.559622in}}{\pgfqpoint{6.375339in}{0.570672in}}%
\pgfpathcurveto{\pgfqpoint{6.375339in}{0.581722in}}{\pgfqpoint{6.370949in}{0.592321in}}{\pgfqpoint{6.363135in}{0.600135in}}%
\pgfpathcurveto{\pgfqpoint{6.355321in}{0.607949in}}{\pgfqpoint{6.344722in}{0.612339in}}{\pgfqpoint{6.333672in}{0.612339in}}%
\pgfpathcurveto{\pgfqpoint{6.322622in}{0.612339in}}{\pgfqpoint{6.312023in}{0.607949in}}{\pgfqpoint{6.304210in}{0.600135in}}%
\pgfpathcurveto{\pgfqpoint{6.296396in}{0.592321in}}{\pgfqpoint{6.292006in}{0.581722in}}{\pgfqpoint{6.292006in}{0.570672in}}%
\pgfpathcurveto{\pgfqpoint{6.292006in}{0.559622in}}{\pgfqpoint{6.296396in}{0.549023in}}{\pgfqpoint{6.304210in}{0.541210in}}%
\pgfpathcurveto{\pgfqpoint{6.312023in}{0.533396in}}{\pgfqpoint{6.322622in}{0.529006in}}{\pgfqpoint{6.333672in}{0.529006in}}%
\pgfusepath{stroke}%
\end{pgfscope}%
\begin{pgfscope}%
\pgfpathrectangle{\pgfqpoint{0.847223in}{0.554012in}}{\pgfqpoint{6.200000in}{4.620000in}}%
\pgfusepath{clip}%
\pgfsetbuttcap%
\pgfsetroundjoin%
\pgfsetlinewidth{1.003750pt}%
\definecolor{currentstroke}{rgb}{1.000000,0.000000,0.000000}%
\pgfsetstrokecolor{currentstroke}%
\pgfsetdash{}{0pt}%
\pgfpathmoveto{\pgfqpoint{6.342482in}{0.528622in}}%
\pgfpathcurveto{\pgfqpoint{6.353532in}{0.528622in}}{\pgfqpoint{6.364131in}{0.533012in}}{\pgfqpoint{6.371944in}{0.540826in}}%
\pgfpathcurveto{\pgfqpoint{6.379758in}{0.548640in}}{\pgfqpoint{6.384148in}{0.559239in}}{\pgfqpoint{6.384148in}{0.570289in}}%
\pgfpathcurveto{\pgfqpoint{6.384148in}{0.581339in}}{\pgfqpoint{6.379758in}{0.591938in}}{\pgfqpoint{6.371944in}{0.599752in}}%
\pgfpathcurveto{\pgfqpoint{6.364131in}{0.607565in}}{\pgfqpoint{6.353532in}{0.611956in}}{\pgfqpoint{6.342482in}{0.611956in}}%
\pgfpathcurveto{\pgfqpoint{6.331431in}{0.611956in}}{\pgfqpoint{6.320832in}{0.607565in}}{\pgfqpoint{6.313019in}{0.599752in}}%
\pgfpathcurveto{\pgfqpoint{6.305205in}{0.591938in}}{\pgfqpoint{6.300815in}{0.581339in}}{\pgfqpoint{6.300815in}{0.570289in}}%
\pgfpathcurveto{\pgfqpoint{6.300815in}{0.559239in}}{\pgfqpoint{6.305205in}{0.548640in}}{\pgfqpoint{6.313019in}{0.540826in}}%
\pgfpathcurveto{\pgfqpoint{6.320832in}{0.533012in}}{\pgfqpoint{6.331431in}{0.528622in}}{\pgfqpoint{6.342482in}{0.528622in}}%
\pgfusepath{stroke}%
\end{pgfscope}%
\begin{pgfscope}%
\pgfpathrectangle{\pgfqpoint{0.847223in}{0.554012in}}{\pgfqpoint{6.200000in}{4.620000in}}%
\pgfusepath{clip}%
\pgfsetbuttcap%
\pgfsetroundjoin%
\pgfsetlinewidth{1.003750pt}%
\definecolor{currentstroke}{rgb}{1.000000,0.000000,0.000000}%
\pgfsetstrokecolor{currentstroke}%
\pgfsetdash{}{0pt}%
\pgfpathmoveto{\pgfqpoint{6.351291in}{0.528248in}}%
\pgfpathcurveto{\pgfqpoint{6.362341in}{0.528248in}}{\pgfqpoint{6.372940in}{0.532638in}}{\pgfqpoint{6.380754in}{0.540452in}}%
\pgfpathcurveto{\pgfqpoint{6.388567in}{0.548265in}}{\pgfqpoint{6.392958in}{0.558864in}}{\pgfqpoint{6.392958in}{0.569915in}}%
\pgfpathcurveto{\pgfqpoint{6.392958in}{0.580965in}}{\pgfqpoint{6.388567in}{0.591564in}}{\pgfqpoint{6.380754in}{0.599377in}}%
\pgfpathcurveto{\pgfqpoint{6.372940in}{0.607191in}}{\pgfqpoint{6.362341in}{0.611581in}}{\pgfqpoint{6.351291in}{0.611581in}}%
\pgfpathcurveto{\pgfqpoint{6.340241in}{0.611581in}}{\pgfqpoint{6.329642in}{0.607191in}}{\pgfqpoint{6.321828in}{0.599377in}}%
\pgfpathcurveto{\pgfqpoint{6.314014in}{0.591564in}}{\pgfqpoint{6.309624in}{0.580965in}}{\pgfqpoint{6.309624in}{0.569915in}}%
\pgfpathcurveto{\pgfqpoint{6.309624in}{0.558864in}}{\pgfqpoint{6.314014in}{0.548265in}}{\pgfqpoint{6.321828in}{0.540452in}}%
\pgfpathcurveto{\pgfqpoint{6.329642in}{0.532638in}}{\pgfqpoint{6.340241in}{0.528248in}}{\pgfqpoint{6.351291in}{0.528248in}}%
\pgfusepath{stroke}%
\end{pgfscope}%
\begin{pgfscope}%
\pgfpathrectangle{\pgfqpoint{0.847223in}{0.554012in}}{\pgfqpoint{6.200000in}{4.620000in}}%
\pgfusepath{clip}%
\pgfsetbuttcap%
\pgfsetroundjoin%
\pgfsetlinewidth{1.003750pt}%
\definecolor{currentstroke}{rgb}{1.000000,0.000000,0.000000}%
\pgfsetstrokecolor{currentstroke}%
\pgfsetdash{}{0pt}%
\pgfpathmoveto{\pgfqpoint{6.360100in}{0.527882in}}%
\pgfpathcurveto{\pgfqpoint{6.371150in}{0.527882in}}{\pgfqpoint{6.381749in}{0.532273in}}{\pgfqpoint{6.389563in}{0.540086in}}%
\pgfpathcurveto{\pgfqpoint{6.397377in}{0.547900in}}{\pgfqpoint{6.401767in}{0.558499in}}{\pgfqpoint{6.401767in}{0.569549in}}%
\pgfpathcurveto{\pgfqpoint{6.401767in}{0.580599in}}{\pgfqpoint{6.397377in}{0.591198in}}{\pgfqpoint{6.389563in}{0.599012in}}%
\pgfpathcurveto{\pgfqpoint{6.381749in}{0.606825in}}{\pgfqpoint{6.371150in}{0.611216in}}{\pgfqpoint{6.360100in}{0.611216in}}%
\pgfpathcurveto{\pgfqpoint{6.349050in}{0.611216in}}{\pgfqpoint{6.338451in}{0.606825in}}{\pgfqpoint{6.330637in}{0.599012in}}%
\pgfpathcurveto{\pgfqpoint{6.322824in}{0.591198in}}{\pgfqpoint{6.318433in}{0.580599in}}{\pgfqpoint{6.318433in}{0.569549in}}%
\pgfpathcurveto{\pgfqpoint{6.318433in}{0.558499in}}{\pgfqpoint{6.322824in}{0.547900in}}{\pgfqpoint{6.330637in}{0.540086in}}%
\pgfpathcurveto{\pgfqpoint{6.338451in}{0.532273in}}{\pgfqpoint{6.349050in}{0.527882in}}{\pgfqpoint{6.360100in}{0.527882in}}%
\pgfusepath{stroke}%
\end{pgfscope}%
\begin{pgfscope}%
\pgfpathrectangle{\pgfqpoint{0.847223in}{0.554012in}}{\pgfqpoint{6.200000in}{4.620000in}}%
\pgfusepath{clip}%
\pgfsetbuttcap%
\pgfsetroundjoin%
\pgfsetlinewidth{1.003750pt}%
\definecolor{currentstroke}{rgb}{1.000000,0.000000,0.000000}%
\pgfsetstrokecolor{currentstroke}%
\pgfsetdash{}{0pt}%
\pgfpathmoveto{\pgfqpoint{6.368909in}{0.527525in}}%
\pgfpathcurveto{\pgfqpoint{6.379960in}{0.527525in}}{\pgfqpoint{6.390559in}{0.531915in}}{\pgfqpoint{6.398372in}{0.539729in}}%
\pgfpathcurveto{\pgfqpoint{6.406186in}{0.547543in}}{\pgfqpoint{6.410576in}{0.558142in}}{\pgfqpoint{6.410576in}{0.569192in}}%
\pgfpathcurveto{\pgfqpoint{6.410576in}{0.580242in}}{\pgfqpoint{6.406186in}{0.590841in}}{\pgfqpoint{6.398372in}{0.598654in}}%
\pgfpathcurveto{\pgfqpoint{6.390559in}{0.606468in}}{\pgfqpoint{6.379960in}{0.610858in}}{\pgfqpoint{6.368909in}{0.610858in}}%
\pgfpathcurveto{\pgfqpoint{6.357859in}{0.610858in}}{\pgfqpoint{6.347260in}{0.606468in}}{\pgfqpoint{6.339447in}{0.598654in}}%
\pgfpathcurveto{\pgfqpoint{6.331633in}{0.590841in}}{\pgfqpoint{6.327243in}{0.580242in}}{\pgfqpoint{6.327243in}{0.569192in}}%
\pgfpathcurveto{\pgfqpoint{6.327243in}{0.558142in}}{\pgfqpoint{6.331633in}{0.547543in}}{\pgfqpoint{6.339447in}{0.539729in}}%
\pgfpathcurveto{\pgfqpoint{6.347260in}{0.531915in}}{\pgfqpoint{6.357859in}{0.527525in}}{\pgfqpoint{6.368909in}{0.527525in}}%
\pgfusepath{stroke}%
\end{pgfscope}%
\begin{pgfscope}%
\pgfpathrectangle{\pgfqpoint{0.847223in}{0.554012in}}{\pgfqpoint{6.200000in}{4.620000in}}%
\pgfusepath{clip}%
\pgfsetbuttcap%
\pgfsetroundjoin%
\pgfsetlinewidth{1.003750pt}%
\definecolor{currentstroke}{rgb}{1.000000,0.000000,0.000000}%
\pgfsetstrokecolor{currentstroke}%
\pgfsetdash{}{0pt}%
\pgfpathmoveto{\pgfqpoint{6.377719in}{0.527176in}}%
\pgfpathcurveto{\pgfqpoint{6.388769in}{0.527176in}}{\pgfqpoint{6.399368in}{0.531566in}}{\pgfqpoint{6.407181in}{0.539380in}}%
\pgfpathcurveto{\pgfqpoint{6.414995in}{0.547193in}}{\pgfqpoint{6.419385in}{0.557792in}}{\pgfqpoint{6.419385in}{0.568843in}}%
\pgfpathcurveto{\pgfqpoint{6.419385in}{0.579893in}}{\pgfqpoint{6.414995in}{0.590492in}}{\pgfqpoint{6.407181in}{0.598305in}}%
\pgfpathcurveto{\pgfqpoint{6.399368in}{0.606119in}}{\pgfqpoint{6.388769in}{0.610509in}}{\pgfqpoint{6.377719in}{0.610509in}}%
\pgfpathcurveto{\pgfqpoint{6.366669in}{0.610509in}}{\pgfqpoint{6.356070in}{0.606119in}}{\pgfqpoint{6.348256in}{0.598305in}}%
\pgfpathcurveto{\pgfqpoint{6.340442in}{0.590492in}}{\pgfqpoint{6.336052in}{0.579893in}}{\pgfqpoint{6.336052in}{0.568843in}}%
\pgfpathcurveto{\pgfqpoint{6.336052in}{0.557792in}}{\pgfqpoint{6.340442in}{0.547193in}}{\pgfqpoint{6.348256in}{0.539380in}}%
\pgfpathcurveto{\pgfqpoint{6.356070in}{0.531566in}}{\pgfqpoint{6.366669in}{0.527176in}}{\pgfqpoint{6.377719in}{0.527176in}}%
\pgfusepath{stroke}%
\end{pgfscope}%
\begin{pgfscope}%
\pgfpathrectangle{\pgfqpoint{0.847223in}{0.554012in}}{\pgfqpoint{6.200000in}{4.620000in}}%
\pgfusepath{clip}%
\pgfsetbuttcap%
\pgfsetroundjoin%
\pgfsetlinewidth{1.003750pt}%
\definecolor{currentstroke}{rgb}{1.000000,0.000000,0.000000}%
\pgfsetstrokecolor{currentstroke}%
\pgfsetdash{}{0pt}%
\pgfpathmoveto{\pgfqpoint{6.386528in}{0.526834in}}%
\pgfpathcurveto{\pgfqpoint{6.397578in}{0.526834in}}{\pgfqpoint{6.408177in}{0.531225in}}{\pgfqpoint{6.415991in}{0.539038in}}%
\pgfpathcurveto{\pgfqpoint{6.423804in}{0.546852in}}{\pgfqpoint{6.428195in}{0.557451in}}{\pgfqpoint{6.428195in}{0.568501in}}%
\pgfpathcurveto{\pgfqpoint{6.428195in}{0.579551in}}{\pgfqpoint{6.423804in}{0.590150in}}{\pgfqpoint{6.415991in}{0.597964in}}%
\pgfpathcurveto{\pgfqpoint{6.408177in}{0.605777in}}{\pgfqpoint{6.397578in}{0.610168in}}{\pgfqpoint{6.386528in}{0.610168in}}%
\pgfpathcurveto{\pgfqpoint{6.375478in}{0.610168in}}{\pgfqpoint{6.364879in}{0.605777in}}{\pgfqpoint{6.357065in}{0.597964in}}%
\pgfpathcurveto{\pgfqpoint{6.349252in}{0.590150in}}{\pgfqpoint{6.344861in}{0.579551in}}{\pgfqpoint{6.344861in}{0.568501in}}%
\pgfpathcurveto{\pgfqpoint{6.344861in}{0.557451in}}{\pgfqpoint{6.349252in}{0.546852in}}{\pgfqpoint{6.357065in}{0.539038in}}%
\pgfpathcurveto{\pgfqpoint{6.364879in}{0.531225in}}{\pgfqpoint{6.375478in}{0.526834in}}{\pgfqpoint{6.386528in}{0.526834in}}%
\pgfusepath{stroke}%
\end{pgfscope}%
\begin{pgfscope}%
\pgfpathrectangle{\pgfqpoint{0.847223in}{0.554012in}}{\pgfqpoint{6.200000in}{4.620000in}}%
\pgfusepath{clip}%
\pgfsetbuttcap%
\pgfsetroundjoin%
\pgfsetlinewidth{1.003750pt}%
\definecolor{currentstroke}{rgb}{1.000000,0.000000,0.000000}%
\pgfsetstrokecolor{currentstroke}%
\pgfsetdash{}{0pt}%
\pgfpathmoveto{\pgfqpoint{6.395337in}{0.526500in}}%
\pgfpathcurveto{\pgfqpoint{6.406387in}{0.526500in}}{\pgfqpoint{6.416986in}{0.530891in}}{\pgfqpoint{6.424800in}{0.538704in}}%
\pgfpathcurveto{\pgfqpoint{6.432614in}{0.546518in}}{\pgfqpoint{6.437004in}{0.557117in}}{\pgfqpoint{6.437004in}{0.568167in}}%
\pgfpathcurveto{\pgfqpoint{6.437004in}{0.579217in}}{\pgfqpoint{6.432614in}{0.589816in}}{\pgfqpoint{6.424800in}{0.597630in}}%
\pgfpathcurveto{\pgfqpoint{6.416986in}{0.605443in}}{\pgfqpoint{6.406387in}{0.609834in}}{\pgfqpoint{6.395337in}{0.609834in}}%
\pgfpathcurveto{\pgfqpoint{6.384287in}{0.609834in}}{\pgfqpoint{6.373688in}{0.605443in}}{\pgfqpoint{6.365874in}{0.597630in}}%
\pgfpathcurveto{\pgfqpoint{6.358061in}{0.589816in}}{\pgfqpoint{6.353671in}{0.579217in}}{\pgfqpoint{6.353671in}{0.568167in}}%
\pgfpathcurveto{\pgfqpoint{6.353671in}{0.557117in}}{\pgfqpoint{6.358061in}{0.546518in}}{\pgfqpoint{6.365874in}{0.538704in}}%
\pgfpathcurveto{\pgfqpoint{6.373688in}{0.530891in}}{\pgfqpoint{6.384287in}{0.526500in}}{\pgfqpoint{6.395337in}{0.526500in}}%
\pgfusepath{stroke}%
\end{pgfscope}%
\begin{pgfscope}%
\pgfpathrectangle{\pgfqpoint{0.847223in}{0.554012in}}{\pgfqpoint{6.200000in}{4.620000in}}%
\pgfusepath{clip}%
\pgfsetbuttcap%
\pgfsetroundjoin%
\pgfsetlinewidth{1.003750pt}%
\definecolor{currentstroke}{rgb}{1.000000,0.000000,0.000000}%
\pgfsetstrokecolor{currentstroke}%
\pgfsetdash{}{0pt}%
\pgfpathmoveto{\pgfqpoint{6.404146in}{0.526173in}}%
\pgfpathcurveto{\pgfqpoint{6.415197in}{0.526173in}}{\pgfqpoint{6.425796in}{0.530564in}}{\pgfqpoint{6.433609in}{0.538377in}}%
\pgfpathcurveto{\pgfqpoint{6.441423in}{0.546191in}}{\pgfqpoint{6.445813in}{0.556790in}}{\pgfqpoint{6.445813in}{0.567840in}}%
\pgfpathcurveto{\pgfqpoint{6.445813in}{0.578890in}}{\pgfqpoint{6.441423in}{0.589489in}}{\pgfqpoint{6.433609in}{0.597303in}}%
\pgfpathcurveto{\pgfqpoint{6.425796in}{0.605116in}}{\pgfqpoint{6.415197in}{0.609507in}}{\pgfqpoint{6.404146in}{0.609507in}}%
\pgfpathcurveto{\pgfqpoint{6.393096in}{0.609507in}}{\pgfqpoint{6.382497in}{0.605116in}}{\pgfqpoint{6.374684in}{0.597303in}}%
\pgfpathcurveto{\pgfqpoint{6.366870in}{0.589489in}}{\pgfqpoint{6.362480in}{0.578890in}}{\pgfqpoint{6.362480in}{0.567840in}}%
\pgfpathcurveto{\pgfqpoint{6.362480in}{0.556790in}}{\pgfqpoint{6.366870in}{0.546191in}}{\pgfqpoint{6.374684in}{0.538377in}}%
\pgfpathcurveto{\pgfqpoint{6.382497in}{0.530564in}}{\pgfqpoint{6.393096in}{0.526173in}}{\pgfqpoint{6.404146in}{0.526173in}}%
\pgfusepath{stroke}%
\end{pgfscope}%
\begin{pgfscope}%
\pgfpathrectangle{\pgfqpoint{0.847223in}{0.554012in}}{\pgfqpoint{6.200000in}{4.620000in}}%
\pgfusepath{clip}%
\pgfsetbuttcap%
\pgfsetroundjoin%
\pgfsetlinewidth{1.003750pt}%
\definecolor{currentstroke}{rgb}{1.000000,0.000000,0.000000}%
\pgfsetstrokecolor{currentstroke}%
\pgfsetdash{}{0pt}%
\pgfpathmoveto{\pgfqpoint{6.412956in}{0.525853in}}%
\pgfpathcurveto{\pgfqpoint{6.424006in}{0.525853in}}{\pgfqpoint{6.434605in}{0.530244in}}{\pgfqpoint{6.442419in}{0.538057in}}%
\pgfpathcurveto{\pgfqpoint{6.450232in}{0.545871in}}{\pgfqpoint{6.454622in}{0.556470in}}{\pgfqpoint{6.454622in}{0.567520in}}%
\pgfpathcurveto{\pgfqpoint{6.454622in}{0.578570in}}{\pgfqpoint{6.450232in}{0.589169in}}{\pgfqpoint{6.442419in}{0.596983in}}%
\pgfpathcurveto{\pgfqpoint{6.434605in}{0.604796in}}{\pgfqpoint{6.424006in}{0.609187in}}{\pgfqpoint{6.412956in}{0.609187in}}%
\pgfpathcurveto{\pgfqpoint{6.401906in}{0.609187in}}{\pgfqpoint{6.391307in}{0.604796in}}{\pgfqpoint{6.383493in}{0.596983in}}%
\pgfpathcurveto{\pgfqpoint{6.375679in}{0.589169in}}{\pgfqpoint{6.371289in}{0.578570in}}{\pgfqpoint{6.371289in}{0.567520in}}%
\pgfpathcurveto{\pgfqpoint{6.371289in}{0.556470in}}{\pgfqpoint{6.375679in}{0.545871in}}{\pgfqpoint{6.383493in}{0.538057in}}%
\pgfpathcurveto{\pgfqpoint{6.391307in}{0.530244in}}{\pgfqpoint{6.401906in}{0.525853in}}{\pgfqpoint{6.412956in}{0.525853in}}%
\pgfusepath{stroke}%
\end{pgfscope}%
\begin{pgfscope}%
\pgfpathrectangle{\pgfqpoint{0.847223in}{0.554012in}}{\pgfqpoint{6.200000in}{4.620000in}}%
\pgfusepath{clip}%
\pgfsetbuttcap%
\pgfsetroundjoin%
\pgfsetlinewidth{1.003750pt}%
\definecolor{currentstroke}{rgb}{1.000000,0.000000,0.000000}%
\pgfsetstrokecolor{currentstroke}%
\pgfsetdash{}{0pt}%
\pgfpathmoveto{\pgfqpoint{6.421765in}{0.525540in}}%
\pgfpathcurveto{\pgfqpoint{6.432815in}{0.525540in}}{\pgfqpoint{6.443414in}{0.529930in}}{\pgfqpoint{6.451228in}{0.537744in}}%
\pgfpathcurveto{\pgfqpoint{6.459041in}{0.545557in}}{\pgfqpoint{6.463432in}{0.556156in}}{\pgfqpoint{6.463432in}{0.567206in}}%
\pgfpathcurveto{\pgfqpoint{6.463432in}{0.578257in}}{\pgfqpoint{6.459041in}{0.588856in}}{\pgfqpoint{6.451228in}{0.596669in}}%
\pgfpathcurveto{\pgfqpoint{6.443414in}{0.604483in}}{\pgfqpoint{6.432815in}{0.608873in}}{\pgfqpoint{6.421765in}{0.608873in}}%
\pgfpathcurveto{\pgfqpoint{6.410715in}{0.608873in}}{\pgfqpoint{6.400116in}{0.604483in}}{\pgfqpoint{6.392302in}{0.596669in}}%
\pgfpathcurveto{\pgfqpoint{6.384489in}{0.588856in}}{\pgfqpoint{6.380098in}{0.578257in}}{\pgfqpoint{6.380098in}{0.567206in}}%
\pgfpathcurveto{\pgfqpoint{6.380098in}{0.556156in}}{\pgfqpoint{6.384489in}{0.545557in}}{\pgfqpoint{6.392302in}{0.537744in}}%
\pgfpathcurveto{\pgfqpoint{6.400116in}{0.529930in}}{\pgfqpoint{6.410715in}{0.525540in}}{\pgfqpoint{6.421765in}{0.525540in}}%
\pgfusepath{stroke}%
\end{pgfscope}%
\begin{pgfscope}%
\pgfpathrectangle{\pgfqpoint{0.847223in}{0.554012in}}{\pgfqpoint{6.200000in}{4.620000in}}%
\pgfusepath{clip}%
\pgfsetbuttcap%
\pgfsetroundjoin%
\pgfsetlinewidth{1.003750pt}%
\definecolor{currentstroke}{rgb}{1.000000,0.000000,0.000000}%
\pgfsetstrokecolor{currentstroke}%
\pgfsetdash{}{0pt}%
\pgfpathmoveto{\pgfqpoint{6.430574in}{0.525233in}}%
\pgfpathcurveto{\pgfqpoint{6.441624in}{0.525233in}}{\pgfqpoint{6.452223in}{0.529623in}}{\pgfqpoint{6.460037in}{0.537437in}}%
\pgfpathcurveto{\pgfqpoint{6.467851in}{0.545250in}}{\pgfqpoint{6.472241in}{0.555849in}}{\pgfqpoint{6.472241in}{0.566899in}}%
\pgfpathcurveto{\pgfqpoint{6.472241in}{0.577949in}}{\pgfqpoint{6.467851in}{0.588548in}}{\pgfqpoint{6.460037in}{0.596362in}}%
\pgfpathcurveto{\pgfqpoint{6.452223in}{0.604176in}}{\pgfqpoint{6.441624in}{0.608566in}}{\pgfqpoint{6.430574in}{0.608566in}}%
\pgfpathcurveto{\pgfqpoint{6.419524in}{0.608566in}}{\pgfqpoint{6.408925in}{0.604176in}}{\pgfqpoint{6.401111in}{0.596362in}}%
\pgfpathcurveto{\pgfqpoint{6.393298in}{0.588548in}}{\pgfqpoint{6.388908in}{0.577949in}}{\pgfqpoint{6.388908in}{0.566899in}}%
\pgfpathcurveto{\pgfqpoint{6.388908in}{0.555849in}}{\pgfqpoint{6.393298in}{0.545250in}}{\pgfqpoint{6.401111in}{0.537437in}}%
\pgfpathcurveto{\pgfqpoint{6.408925in}{0.529623in}}{\pgfqpoint{6.419524in}{0.525233in}}{\pgfqpoint{6.430574in}{0.525233in}}%
\pgfusepath{stroke}%
\end{pgfscope}%
\begin{pgfscope}%
\pgfpathrectangle{\pgfqpoint{0.847223in}{0.554012in}}{\pgfqpoint{6.200000in}{4.620000in}}%
\pgfusepath{clip}%
\pgfsetbuttcap%
\pgfsetroundjoin%
\pgfsetlinewidth{1.003750pt}%
\definecolor{currentstroke}{rgb}{1.000000,0.000000,0.000000}%
\pgfsetstrokecolor{currentstroke}%
\pgfsetdash{}{0pt}%
\pgfpathmoveto{\pgfqpoint{6.439384in}{0.524932in}}%
\pgfpathcurveto{\pgfqpoint{6.450434in}{0.524932in}}{\pgfqpoint{6.461033in}{0.529322in}}{\pgfqpoint{6.468846in}{0.537136in}}%
\pgfpathcurveto{\pgfqpoint{6.476660in}{0.544949in}}{\pgfqpoint{6.481050in}{0.555548in}}{\pgfqpoint{6.481050in}{0.566598in}}%
\pgfpathcurveto{\pgfqpoint{6.481050in}{0.577648in}}{\pgfqpoint{6.476660in}{0.588247in}}{\pgfqpoint{6.468846in}{0.596061in}}%
\pgfpathcurveto{\pgfqpoint{6.461033in}{0.603875in}}{\pgfqpoint{6.450434in}{0.608265in}}{\pgfqpoint{6.439384in}{0.608265in}}%
\pgfpathcurveto{\pgfqpoint{6.428333in}{0.608265in}}{\pgfqpoint{6.417734in}{0.603875in}}{\pgfqpoint{6.409921in}{0.596061in}}%
\pgfpathcurveto{\pgfqpoint{6.402107in}{0.588247in}}{\pgfqpoint{6.397717in}{0.577648in}}{\pgfqpoint{6.397717in}{0.566598in}}%
\pgfpathcurveto{\pgfqpoint{6.397717in}{0.555548in}}{\pgfqpoint{6.402107in}{0.544949in}}{\pgfqpoint{6.409921in}{0.537136in}}%
\pgfpathcurveto{\pgfqpoint{6.417734in}{0.529322in}}{\pgfqpoint{6.428333in}{0.524932in}}{\pgfqpoint{6.439384in}{0.524932in}}%
\pgfusepath{stroke}%
\end{pgfscope}%
\begin{pgfscope}%
\pgfpathrectangle{\pgfqpoint{0.847223in}{0.554012in}}{\pgfqpoint{6.200000in}{4.620000in}}%
\pgfusepath{clip}%
\pgfsetbuttcap%
\pgfsetroundjoin%
\pgfsetlinewidth{1.003750pt}%
\definecolor{currentstroke}{rgb}{1.000000,0.000000,0.000000}%
\pgfsetstrokecolor{currentstroke}%
\pgfsetdash{}{0pt}%
\pgfpathmoveto{\pgfqpoint{6.448193in}{0.524637in}}%
\pgfpathcurveto{\pgfqpoint{6.459243in}{0.524637in}}{\pgfqpoint{6.469842in}{0.529027in}}{\pgfqpoint{6.477656in}{0.536840in}}%
\pgfpathcurveto{\pgfqpoint{6.485469in}{0.544654in}}{\pgfqpoint{6.489859in}{0.555253in}}{\pgfqpoint{6.489859in}{0.566303in}}%
\pgfpathcurveto{\pgfqpoint{6.489859in}{0.577353in}}{\pgfqpoint{6.485469in}{0.587952in}}{\pgfqpoint{6.477656in}{0.595766in}}%
\pgfpathcurveto{\pgfqpoint{6.469842in}{0.603580in}}{\pgfqpoint{6.459243in}{0.607970in}}{\pgfqpoint{6.448193in}{0.607970in}}%
\pgfpathcurveto{\pgfqpoint{6.437143in}{0.607970in}}{\pgfqpoint{6.426544in}{0.603580in}}{\pgfqpoint{6.418730in}{0.595766in}}%
\pgfpathcurveto{\pgfqpoint{6.410916in}{0.587952in}}{\pgfqpoint{6.406526in}{0.577353in}}{\pgfqpoint{6.406526in}{0.566303in}}%
\pgfpathcurveto{\pgfqpoint{6.406526in}{0.555253in}}{\pgfqpoint{6.410916in}{0.544654in}}{\pgfqpoint{6.418730in}{0.536840in}}%
\pgfpathcurveto{\pgfqpoint{6.426544in}{0.529027in}}{\pgfqpoint{6.437143in}{0.524637in}}{\pgfqpoint{6.448193in}{0.524637in}}%
\pgfusepath{stroke}%
\end{pgfscope}%
\begin{pgfscope}%
\pgfpathrectangle{\pgfqpoint{0.847223in}{0.554012in}}{\pgfqpoint{6.200000in}{4.620000in}}%
\pgfusepath{clip}%
\pgfsetbuttcap%
\pgfsetroundjoin%
\pgfsetlinewidth{1.003750pt}%
\definecolor{currentstroke}{rgb}{1.000000,0.000000,0.000000}%
\pgfsetstrokecolor{currentstroke}%
\pgfsetdash{}{0pt}%
\pgfpathmoveto{\pgfqpoint{6.457002in}{0.524347in}}%
\pgfpathcurveto{\pgfqpoint{6.468052in}{0.524347in}}{\pgfqpoint{6.478651in}{0.528737in}}{\pgfqpoint{6.486465in}{0.536551in}}%
\pgfpathcurveto{\pgfqpoint{6.494278in}{0.544365in}}{\pgfqpoint{6.498669in}{0.554964in}}{\pgfqpoint{6.498669in}{0.566014in}}%
\pgfpathcurveto{\pgfqpoint{6.498669in}{0.577064in}}{\pgfqpoint{6.494278in}{0.587663in}}{\pgfqpoint{6.486465in}{0.595477in}}%
\pgfpathcurveto{\pgfqpoint{6.478651in}{0.603290in}}{\pgfqpoint{6.468052in}{0.607681in}}{\pgfqpoint{6.457002in}{0.607681in}}%
\pgfpathcurveto{\pgfqpoint{6.445952in}{0.607681in}}{\pgfqpoint{6.435353in}{0.603290in}}{\pgfqpoint{6.427539in}{0.595477in}}%
\pgfpathcurveto{\pgfqpoint{6.419726in}{0.587663in}}{\pgfqpoint{6.415335in}{0.577064in}}{\pgfqpoint{6.415335in}{0.566014in}}%
\pgfpathcurveto{\pgfqpoint{6.415335in}{0.554964in}}{\pgfqpoint{6.419726in}{0.544365in}}{\pgfqpoint{6.427539in}{0.536551in}}%
\pgfpathcurveto{\pgfqpoint{6.435353in}{0.528737in}}{\pgfqpoint{6.445952in}{0.524347in}}{\pgfqpoint{6.457002in}{0.524347in}}%
\pgfusepath{stroke}%
\end{pgfscope}%
\begin{pgfscope}%
\pgfpathrectangle{\pgfqpoint{0.847223in}{0.554012in}}{\pgfqpoint{6.200000in}{4.620000in}}%
\pgfusepath{clip}%
\pgfsetbuttcap%
\pgfsetroundjoin%
\pgfsetlinewidth{1.003750pt}%
\definecolor{currentstroke}{rgb}{1.000000,0.000000,0.000000}%
\pgfsetstrokecolor{currentstroke}%
\pgfsetdash{}{0pt}%
\pgfpathmoveto{\pgfqpoint{6.465811in}{0.524063in}}%
\pgfpathcurveto{\pgfqpoint{6.476861in}{0.524063in}}{\pgfqpoint{6.487461in}{0.528454in}}{\pgfqpoint{6.495274in}{0.536267in}}%
\pgfpathcurveto{\pgfqpoint{6.503088in}{0.544081in}}{\pgfqpoint{6.507478in}{0.554680in}}{\pgfqpoint{6.507478in}{0.565730in}}%
\pgfpathcurveto{\pgfqpoint{6.507478in}{0.576780in}}{\pgfqpoint{6.503088in}{0.587379in}}{\pgfqpoint{6.495274in}{0.595193in}}%
\pgfpathcurveto{\pgfqpoint{6.487461in}{0.603006in}}{\pgfqpoint{6.476861in}{0.607397in}}{\pgfqpoint{6.465811in}{0.607397in}}%
\pgfpathcurveto{\pgfqpoint{6.454761in}{0.607397in}}{\pgfqpoint{6.444162in}{0.603006in}}{\pgfqpoint{6.436349in}{0.595193in}}%
\pgfpathcurveto{\pgfqpoint{6.428535in}{0.587379in}}{\pgfqpoint{6.424145in}{0.576780in}}{\pgfqpoint{6.424145in}{0.565730in}}%
\pgfpathcurveto{\pgfqpoint{6.424145in}{0.554680in}}{\pgfqpoint{6.428535in}{0.544081in}}{\pgfqpoint{6.436349in}{0.536267in}}%
\pgfpathcurveto{\pgfqpoint{6.444162in}{0.528454in}}{\pgfqpoint{6.454761in}{0.524063in}}{\pgfqpoint{6.465811in}{0.524063in}}%
\pgfusepath{stroke}%
\end{pgfscope}%
\begin{pgfscope}%
\pgfpathrectangle{\pgfqpoint{0.847223in}{0.554012in}}{\pgfqpoint{6.200000in}{4.620000in}}%
\pgfusepath{clip}%
\pgfsetbuttcap%
\pgfsetroundjoin%
\pgfsetlinewidth{1.003750pt}%
\definecolor{currentstroke}{rgb}{1.000000,0.000000,0.000000}%
\pgfsetstrokecolor{currentstroke}%
\pgfsetdash{}{0pt}%
\pgfpathmoveto{\pgfqpoint{6.474621in}{0.523785in}}%
\pgfpathcurveto{\pgfqpoint{6.485671in}{0.523785in}}{\pgfqpoint{6.496270in}{0.528175in}}{\pgfqpoint{6.504083in}{0.535989in}}%
\pgfpathcurveto{\pgfqpoint{6.511897in}{0.543802in}}{\pgfqpoint{6.516287in}{0.554401in}}{\pgfqpoint{6.516287in}{0.565452in}}%
\pgfpathcurveto{\pgfqpoint{6.516287in}{0.576502in}}{\pgfqpoint{6.511897in}{0.587101in}}{\pgfqpoint{6.504083in}{0.594914in}}%
\pgfpathcurveto{\pgfqpoint{6.496270in}{0.602728in}}{\pgfqpoint{6.485671in}{0.607118in}}{\pgfqpoint{6.474621in}{0.607118in}}%
\pgfpathcurveto{\pgfqpoint{6.463570in}{0.607118in}}{\pgfqpoint{6.452971in}{0.602728in}}{\pgfqpoint{6.445158in}{0.594914in}}%
\pgfpathcurveto{\pgfqpoint{6.437344in}{0.587101in}}{\pgfqpoint{6.432954in}{0.576502in}}{\pgfqpoint{6.432954in}{0.565452in}}%
\pgfpathcurveto{\pgfqpoint{6.432954in}{0.554401in}}{\pgfqpoint{6.437344in}{0.543802in}}{\pgfqpoint{6.445158in}{0.535989in}}%
\pgfpathcurveto{\pgfqpoint{6.452971in}{0.528175in}}{\pgfqpoint{6.463570in}{0.523785in}}{\pgfqpoint{6.474621in}{0.523785in}}%
\pgfusepath{stroke}%
\end{pgfscope}%
\begin{pgfscope}%
\pgfpathrectangle{\pgfqpoint{0.847223in}{0.554012in}}{\pgfqpoint{6.200000in}{4.620000in}}%
\pgfusepath{clip}%
\pgfsetbuttcap%
\pgfsetroundjoin%
\pgfsetlinewidth{1.003750pt}%
\definecolor{currentstroke}{rgb}{1.000000,0.000000,0.000000}%
\pgfsetstrokecolor{currentstroke}%
\pgfsetdash{}{0pt}%
\pgfpathmoveto{\pgfqpoint{6.483430in}{0.523512in}}%
\pgfpathcurveto{\pgfqpoint{6.494480in}{0.523512in}}{\pgfqpoint{6.505079in}{0.527902in}}{\pgfqpoint{6.512893in}{0.535715in}}%
\pgfpathcurveto{\pgfqpoint{6.520706in}{0.543529in}}{\pgfqpoint{6.525097in}{0.554128in}}{\pgfqpoint{6.525097in}{0.565178in}}%
\pgfpathcurveto{\pgfqpoint{6.525097in}{0.576228in}}{\pgfqpoint{6.520706in}{0.586827in}}{\pgfqpoint{6.512893in}{0.594641in}}%
\pgfpathcurveto{\pgfqpoint{6.505079in}{0.602455in}}{\pgfqpoint{6.494480in}{0.606845in}}{\pgfqpoint{6.483430in}{0.606845in}}%
\pgfpathcurveto{\pgfqpoint{6.472380in}{0.606845in}}{\pgfqpoint{6.461781in}{0.602455in}}{\pgfqpoint{6.453967in}{0.594641in}}%
\pgfpathcurveto{\pgfqpoint{6.446153in}{0.586827in}}{\pgfqpoint{6.441763in}{0.576228in}}{\pgfqpoint{6.441763in}{0.565178in}}%
\pgfpathcurveto{\pgfqpoint{6.441763in}{0.554128in}}{\pgfqpoint{6.446153in}{0.543529in}}{\pgfqpoint{6.453967in}{0.535715in}}%
\pgfpathcurveto{\pgfqpoint{6.461781in}{0.527902in}}{\pgfqpoint{6.472380in}{0.523512in}}{\pgfqpoint{6.483430in}{0.523512in}}%
\pgfusepath{stroke}%
\end{pgfscope}%
\begin{pgfscope}%
\pgfpathrectangle{\pgfqpoint{0.847223in}{0.554012in}}{\pgfqpoint{6.200000in}{4.620000in}}%
\pgfusepath{clip}%
\pgfsetbuttcap%
\pgfsetroundjoin%
\pgfsetlinewidth{1.003750pt}%
\definecolor{currentstroke}{rgb}{1.000000,0.000000,0.000000}%
\pgfsetstrokecolor{currentstroke}%
\pgfsetdash{}{0pt}%
\pgfpathmoveto{\pgfqpoint{6.492239in}{0.523243in}}%
\pgfpathcurveto{\pgfqpoint{6.503289in}{0.523243in}}{\pgfqpoint{6.513888in}{0.527633in}}{\pgfqpoint{6.521702in}{0.535447in}}%
\pgfpathcurveto{\pgfqpoint{6.529516in}{0.543261in}}{\pgfqpoint{6.533906in}{0.553860in}}{\pgfqpoint{6.533906in}{0.564910in}}%
\pgfpathcurveto{\pgfqpoint{6.533906in}{0.575960in}}{\pgfqpoint{6.529516in}{0.586559in}}{\pgfqpoint{6.521702in}{0.594373in}}%
\pgfpathcurveto{\pgfqpoint{6.513888in}{0.602186in}}{\pgfqpoint{6.503289in}{0.606576in}}{\pgfqpoint{6.492239in}{0.606576in}}%
\pgfpathcurveto{\pgfqpoint{6.481189in}{0.606576in}}{\pgfqpoint{6.470590in}{0.602186in}}{\pgfqpoint{6.462776in}{0.594373in}}%
\pgfpathcurveto{\pgfqpoint{6.454963in}{0.586559in}}{\pgfqpoint{6.450572in}{0.575960in}}{\pgfqpoint{6.450572in}{0.564910in}}%
\pgfpathcurveto{\pgfqpoint{6.450572in}{0.553860in}}{\pgfqpoint{6.454963in}{0.543261in}}{\pgfqpoint{6.462776in}{0.535447in}}%
\pgfpathcurveto{\pgfqpoint{6.470590in}{0.527633in}}{\pgfqpoint{6.481189in}{0.523243in}}{\pgfqpoint{6.492239in}{0.523243in}}%
\pgfusepath{stroke}%
\end{pgfscope}%
\begin{pgfscope}%
\pgfpathrectangle{\pgfqpoint{0.847223in}{0.554012in}}{\pgfqpoint{6.200000in}{4.620000in}}%
\pgfusepath{clip}%
\pgfsetbuttcap%
\pgfsetroundjoin%
\pgfsetlinewidth{1.003750pt}%
\definecolor{currentstroke}{rgb}{1.000000,0.000000,0.000000}%
\pgfsetstrokecolor{currentstroke}%
\pgfsetdash{}{0pt}%
\pgfpathmoveto{\pgfqpoint{6.501048in}{0.522980in}}%
\pgfpathcurveto{\pgfqpoint{6.512099in}{0.522980in}}{\pgfqpoint{6.522698in}{0.527370in}}{\pgfqpoint{6.530511in}{0.535184in}}%
\pgfpathcurveto{\pgfqpoint{6.538325in}{0.542997in}}{\pgfqpoint{6.542715in}{0.553596in}}{\pgfqpoint{6.542715in}{0.564646in}}%
\pgfpathcurveto{\pgfqpoint{6.542715in}{0.575696in}}{\pgfqpoint{6.538325in}{0.586295in}}{\pgfqpoint{6.530511in}{0.594109in}}%
\pgfpathcurveto{\pgfqpoint{6.522698in}{0.601923in}}{\pgfqpoint{6.512099in}{0.606313in}}{\pgfqpoint{6.501048in}{0.606313in}}%
\pgfpathcurveto{\pgfqpoint{6.489998in}{0.606313in}}{\pgfqpoint{6.479399in}{0.601923in}}{\pgfqpoint{6.471586in}{0.594109in}}%
\pgfpathcurveto{\pgfqpoint{6.463772in}{0.586295in}}{\pgfqpoint{6.459382in}{0.575696in}}{\pgfqpoint{6.459382in}{0.564646in}}%
\pgfpathcurveto{\pgfqpoint{6.459382in}{0.553596in}}{\pgfqpoint{6.463772in}{0.542997in}}{\pgfqpoint{6.471586in}{0.535184in}}%
\pgfpathcurveto{\pgfqpoint{6.479399in}{0.527370in}}{\pgfqpoint{6.489998in}{0.522980in}}{\pgfqpoint{6.501048in}{0.522980in}}%
\pgfusepath{stroke}%
\end{pgfscope}%
\begin{pgfscope}%
\pgfpathrectangle{\pgfqpoint{0.847223in}{0.554012in}}{\pgfqpoint{6.200000in}{4.620000in}}%
\pgfusepath{clip}%
\pgfsetbuttcap%
\pgfsetroundjoin%
\pgfsetlinewidth{1.003750pt}%
\definecolor{currentstroke}{rgb}{1.000000,0.000000,0.000000}%
\pgfsetstrokecolor{currentstroke}%
\pgfsetdash{}{0pt}%
\pgfpathmoveto{\pgfqpoint{6.509858in}{0.522721in}}%
\pgfpathcurveto{\pgfqpoint{6.520908in}{0.522721in}}{\pgfqpoint{6.531507in}{0.527111in}}{\pgfqpoint{6.539320in}{0.534925in}}%
\pgfpathcurveto{\pgfqpoint{6.547134in}{0.542738in}}{\pgfqpoint{6.551524in}{0.553337in}}{\pgfqpoint{6.551524in}{0.564387in}}%
\pgfpathcurveto{\pgfqpoint{6.551524in}{0.575438in}}{\pgfqpoint{6.547134in}{0.586037in}}{\pgfqpoint{6.539320in}{0.593850in}}%
\pgfpathcurveto{\pgfqpoint{6.531507in}{0.601664in}}{\pgfqpoint{6.520908in}{0.606054in}}{\pgfqpoint{6.509858in}{0.606054in}}%
\pgfpathcurveto{\pgfqpoint{6.498808in}{0.606054in}}{\pgfqpoint{6.488209in}{0.601664in}}{\pgfqpoint{6.480395in}{0.593850in}}%
\pgfpathcurveto{\pgfqpoint{6.472581in}{0.586037in}}{\pgfqpoint{6.468191in}{0.575438in}}{\pgfqpoint{6.468191in}{0.564387in}}%
\pgfpathcurveto{\pgfqpoint{6.468191in}{0.553337in}}{\pgfqpoint{6.472581in}{0.542738in}}{\pgfqpoint{6.480395in}{0.534925in}}%
\pgfpathcurveto{\pgfqpoint{6.488209in}{0.527111in}}{\pgfqpoint{6.498808in}{0.522721in}}{\pgfqpoint{6.509858in}{0.522721in}}%
\pgfusepath{stroke}%
\end{pgfscope}%
\begin{pgfscope}%
\pgfpathrectangle{\pgfqpoint{0.847223in}{0.554012in}}{\pgfqpoint{6.200000in}{4.620000in}}%
\pgfusepath{clip}%
\pgfsetbuttcap%
\pgfsetroundjoin%
\pgfsetlinewidth{1.003750pt}%
\definecolor{currentstroke}{rgb}{1.000000,0.000000,0.000000}%
\pgfsetstrokecolor{currentstroke}%
\pgfsetdash{}{0pt}%
\pgfpathmoveto{\pgfqpoint{6.518667in}{0.522466in}}%
\pgfpathcurveto{\pgfqpoint{6.529717in}{0.522466in}}{\pgfqpoint{6.540316in}{0.526857in}}{\pgfqpoint{6.548130in}{0.534670in}}%
\pgfpathcurveto{\pgfqpoint{6.555943in}{0.542484in}}{\pgfqpoint{6.560334in}{0.553083in}}{\pgfqpoint{6.560334in}{0.564133in}}%
\pgfpathcurveto{\pgfqpoint{6.560334in}{0.575183in}}{\pgfqpoint{6.555943in}{0.585782in}}{\pgfqpoint{6.548130in}{0.593596in}}%
\pgfpathcurveto{\pgfqpoint{6.540316in}{0.601410in}}{\pgfqpoint{6.529717in}{0.605800in}}{\pgfqpoint{6.518667in}{0.605800in}}%
\pgfpathcurveto{\pgfqpoint{6.507617in}{0.605800in}}{\pgfqpoint{6.497018in}{0.601410in}}{\pgfqpoint{6.489204in}{0.593596in}}%
\pgfpathcurveto{\pgfqpoint{6.481391in}{0.585782in}}{\pgfqpoint{6.477000in}{0.575183in}}{\pgfqpoint{6.477000in}{0.564133in}}%
\pgfpathcurveto{\pgfqpoint{6.477000in}{0.553083in}}{\pgfqpoint{6.481391in}{0.542484in}}{\pgfqpoint{6.489204in}{0.534670in}}%
\pgfpathcurveto{\pgfqpoint{6.497018in}{0.526857in}}{\pgfqpoint{6.507617in}{0.522466in}}{\pgfqpoint{6.518667in}{0.522466in}}%
\pgfusepath{stroke}%
\end{pgfscope}%
\begin{pgfscope}%
\pgfpathrectangle{\pgfqpoint{0.847223in}{0.554012in}}{\pgfqpoint{6.200000in}{4.620000in}}%
\pgfusepath{clip}%
\pgfsetbuttcap%
\pgfsetroundjoin%
\pgfsetlinewidth{1.003750pt}%
\definecolor{currentstroke}{rgb}{1.000000,0.000000,0.000000}%
\pgfsetstrokecolor{currentstroke}%
\pgfsetdash{}{0pt}%
\pgfpathmoveto{\pgfqpoint{6.527476in}{0.522217in}}%
\pgfpathcurveto{\pgfqpoint{6.538526in}{0.522217in}}{\pgfqpoint{6.549125in}{0.526607in}}{\pgfqpoint{6.556939in}{0.534420in}}%
\pgfpathcurveto{\pgfqpoint{6.564753in}{0.542234in}}{\pgfqpoint{6.569143in}{0.552833in}}{\pgfqpoint{6.569143in}{0.563883in}}%
\pgfpathcurveto{\pgfqpoint{6.569143in}{0.574933in}}{\pgfqpoint{6.564753in}{0.585532in}}{\pgfqpoint{6.556939in}{0.593346in}}%
\pgfpathcurveto{\pgfqpoint{6.549125in}{0.601160in}}{\pgfqpoint{6.538526in}{0.605550in}}{\pgfqpoint{6.527476in}{0.605550in}}%
\pgfpathcurveto{\pgfqpoint{6.516426in}{0.605550in}}{\pgfqpoint{6.505827in}{0.601160in}}{\pgfqpoint{6.498013in}{0.593346in}}%
\pgfpathcurveto{\pgfqpoint{6.490200in}{0.585532in}}{\pgfqpoint{6.485810in}{0.574933in}}{\pgfqpoint{6.485810in}{0.563883in}}%
\pgfpathcurveto{\pgfqpoint{6.485810in}{0.552833in}}{\pgfqpoint{6.490200in}{0.542234in}}{\pgfqpoint{6.498013in}{0.534420in}}%
\pgfpathcurveto{\pgfqpoint{6.505827in}{0.526607in}}{\pgfqpoint{6.516426in}{0.522217in}}{\pgfqpoint{6.527476in}{0.522217in}}%
\pgfusepath{stroke}%
\end{pgfscope}%
\begin{pgfscope}%
\pgfpathrectangle{\pgfqpoint{0.847223in}{0.554012in}}{\pgfqpoint{6.200000in}{4.620000in}}%
\pgfusepath{clip}%
\pgfsetbuttcap%
\pgfsetroundjoin%
\pgfsetlinewidth{1.003750pt}%
\definecolor{currentstroke}{rgb}{1.000000,0.000000,0.000000}%
\pgfsetstrokecolor{currentstroke}%
\pgfsetdash{}{0pt}%
\pgfpathmoveto{\pgfqpoint{6.536285in}{0.521971in}}%
\pgfpathcurveto{\pgfqpoint{6.547336in}{0.521971in}}{\pgfqpoint{6.557935in}{0.526361in}}{\pgfqpoint{6.565748in}{0.534175in}}%
\pgfpathcurveto{\pgfqpoint{6.573562in}{0.541988in}}{\pgfqpoint{6.577952in}{0.552587in}}{\pgfqpoint{6.577952in}{0.563638in}}%
\pgfpathcurveto{\pgfqpoint{6.577952in}{0.574688in}}{\pgfqpoint{6.573562in}{0.585287in}}{\pgfqpoint{6.565748in}{0.593100in}}%
\pgfpathcurveto{\pgfqpoint{6.557935in}{0.600914in}}{\pgfqpoint{6.547336in}{0.605304in}}{\pgfqpoint{6.536285in}{0.605304in}}%
\pgfpathcurveto{\pgfqpoint{6.525235in}{0.605304in}}{\pgfqpoint{6.514636in}{0.600914in}}{\pgfqpoint{6.506823in}{0.593100in}}%
\pgfpathcurveto{\pgfqpoint{6.499009in}{0.585287in}}{\pgfqpoint{6.494619in}{0.574688in}}{\pgfqpoint{6.494619in}{0.563638in}}%
\pgfpathcurveto{\pgfqpoint{6.494619in}{0.552587in}}{\pgfqpoint{6.499009in}{0.541988in}}{\pgfqpoint{6.506823in}{0.534175in}}%
\pgfpathcurveto{\pgfqpoint{6.514636in}{0.526361in}}{\pgfqpoint{6.525235in}{0.521971in}}{\pgfqpoint{6.536285in}{0.521971in}}%
\pgfusepath{stroke}%
\end{pgfscope}%
\begin{pgfscope}%
\pgfpathrectangle{\pgfqpoint{0.847223in}{0.554012in}}{\pgfqpoint{6.200000in}{4.620000in}}%
\pgfusepath{clip}%
\pgfsetbuttcap%
\pgfsetroundjoin%
\pgfsetlinewidth{1.003750pt}%
\definecolor{currentstroke}{rgb}{1.000000,0.000000,0.000000}%
\pgfsetstrokecolor{currentstroke}%
\pgfsetdash{}{0pt}%
\pgfpathmoveto{\pgfqpoint{6.545095in}{0.521729in}}%
\pgfpathcurveto{\pgfqpoint{6.556145in}{0.521729in}}{\pgfqpoint{6.566744in}{0.526120in}}{\pgfqpoint{6.574558in}{0.533933in}}%
\pgfpathcurveto{\pgfqpoint{6.582371in}{0.541747in}}{\pgfqpoint{6.586761in}{0.552346in}}{\pgfqpoint{6.586761in}{0.563396in}}%
\pgfpathcurveto{\pgfqpoint{6.586761in}{0.574446in}}{\pgfqpoint{6.582371in}{0.585045in}}{\pgfqpoint{6.574558in}{0.592859in}}%
\pgfpathcurveto{\pgfqpoint{6.566744in}{0.600673in}}{\pgfqpoint{6.556145in}{0.605063in}}{\pgfqpoint{6.545095in}{0.605063in}}%
\pgfpathcurveto{\pgfqpoint{6.534045in}{0.605063in}}{\pgfqpoint{6.523446in}{0.600673in}}{\pgfqpoint{6.515632in}{0.592859in}}%
\pgfpathcurveto{\pgfqpoint{6.507818in}{0.585045in}}{\pgfqpoint{6.503428in}{0.574446in}}{\pgfqpoint{6.503428in}{0.563396in}}%
\pgfpathcurveto{\pgfqpoint{6.503428in}{0.552346in}}{\pgfqpoint{6.507818in}{0.541747in}}{\pgfqpoint{6.515632in}{0.533933in}}%
\pgfpathcurveto{\pgfqpoint{6.523446in}{0.526120in}}{\pgfqpoint{6.534045in}{0.521729in}}{\pgfqpoint{6.545095in}{0.521729in}}%
\pgfusepath{stroke}%
\end{pgfscope}%
\begin{pgfscope}%
\pgfpathrectangle{\pgfqpoint{0.847223in}{0.554012in}}{\pgfqpoint{6.200000in}{4.620000in}}%
\pgfusepath{clip}%
\pgfsetbuttcap%
\pgfsetroundjoin%
\pgfsetlinewidth{1.003750pt}%
\definecolor{currentstroke}{rgb}{1.000000,0.000000,0.000000}%
\pgfsetstrokecolor{currentstroke}%
\pgfsetdash{}{0pt}%
\pgfpathmoveto{\pgfqpoint{6.553904in}{0.521492in}}%
\pgfpathcurveto{\pgfqpoint{6.564954in}{0.521492in}}{\pgfqpoint{6.575553in}{0.525882in}}{\pgfqpoint{6.583367in}{0.533696in}}%
\pgfpathcurveto{\pgfqpoint{6.591180in}{0.541510in}}{\pgfqpoint{6.595571in}{0.552109in}}{\pgfqpoint{6.595571in}{0.563159in}}%
\pgfpathcurveto{\pgfqpoint{6.595571in}{0.574209in}}{\pgfqpoint{6.591180in}{0.584808in}}{\pgfqpoint{6.583367in}{0.592621in}}%
\pgfpathcurveto{\pgfqpoint{6.575553in}{0.600435in}}{\pgfqpoint{6.564954in}{0.604825in}}{\pgfqpoint{6.553904in}{0.604825in}}%
\pgfpathcurveto{\pgfqpoint{6.542854in}{0.604825in}}{\pgfqpoint{6.532255in}{0.600435in}}{\pgfqpoint{6.524441in}{0.592621in}}%
\pgfpathcurveto{\pgfqpoint{6.516628in}{0.584808in}}{\pgfqpoint{6.512237in}{0.574209in}}{\pgfqpoint{6.512237in}{0.563159in}}%
\pgfpathcurveto{\pgfqpoint{6.512237in}{0.552109in}}{\pgfqpoint{6.516628in}{0.541510in}}{\pgfqpoint{6.524441in}{0.533696in}}%
\pgfpathcurveto{\pgfqpoint{6.532255in}{0.525882in}}{\pgfqpoint{6.542854in}{0.521492in}}{\pgfqpoint{6.553904in}{0.521492in}}%
\pgfusepath{stroke}%
\end{pgfscope}%
\begin{pgfscope}%
\pgfpathrectangle{\pgfqpoint{0.847223in}{0.554012in}}{\pgfqpoint{6.200000in}{4.620000in}}%
\pgfusepath{clip}%
\pgfsetbuttcap%
\pgfsetroundjoin%
\pgfsetlinewidth{1.003750pt}%
\definecolor{currentstroke}{rgb}{1.000000,0.000000,0.000000}%
\pgfsetstrokecolor{currentstroke}%
\pgfsetdash{}{0pt}%
\pgfpathmoveto{\pgfqpoint{6.562713in}{0.521258in}}%
\pgfpathcurveto{\pgfqpoint{6.573763in}{0.521258in}}{\pgfqpoint{6.584362in}{0.525649in}}{\pgfqpoint{6.592176in}{0.533462in}}%
\pgfpathcurveto{\pgfqpoint{6.599990in}{0.541276in}}{\pgfqpoint{6.604380in}{0.551875in}}{\pgfqpoint{6.604380in}{0.562925in}}%
\pgfpathcurveto{\pgfqpoint{6.604380in}{0.573975in}}{\pgfqpoint{6.599990in}{0.584574in}}{\pgfqpoint{6.592176in}{0.592388in}}%
\pgfpathcurveto{\pgfqpoint{6.584362in}{0.600202in}}{\pgfqpoint{6.573763in}{0.604592in}}{\pgfqpoint{6.562713in}{0.604592in}}%
\pgfpathcurveto{\pgfqpoint{6.551663in}{0.604592in}}{\pgfqpoint{6.541064in}{0.600202in}}{\pgfqpoint{6.533251in}{0.592388in}}%
\pgfpathcurveto{\pgfqpoint{6.525437in}{0.584574in}}{\pgfqpoint{6.521047in}{0.573975in}}{\pgfqpoint{6.521047in}{0.562925in}}%
\pgfpathcurveto{\pgfqpoint{6.521047in}{0.551875in}}{\pgfqpoint{6.525437in}{0.541276in}}{\pgfqpoint{6.533251in}{0.533462in}}%
\pgfpathcurveto{\pgfqpoint{6.541064in}{0.525649in}}{\pgfqpoint{6.551663in}{0.521258in}}{\pgfqpoint{6.562713in}{0.521258in}}%
\pgfusepath{stroke}%
\end{pgfscope}%
\begin{pgfscope}%
\pgfpathrectangle{\pgfqpoint{0.847223in}{0.554012in}}{\pgfqpoint{6.200000in}{4.620000in}}%
\pgfusepath{clip}%
\pgfsetbuttcap%
\pgfsetroundjoin%
\pgfsetlinewidth{1.003750pt}%
\definecolor{currentstroke}{rgb}{1.000000,0.000000,0.000000}%
\pgfsetstrokecolor{currentstroke}%
\pgfsetdash{}{0pt}%
\pgfpathmoveto{\pgfqpoint{6.571523in}{0.521029in}}%
\pgfpathcurveto{\pgfqpoint{6.582573in}{0.521029in}}{\pgfqpoint{6.593172in}{0.525419in}}{\pgfqpoint{6.600985in}{0.533233in}}%
\pgfpathcurveto{\pgfqpoint{6.608799in}{0.541046in}}{\pgfqpoint{6.613189in}{0.551645in}}{\pgfqpoint{6.613189in}{0.562695in}}%
\pgfpathcurveto{\pgfqpoint{6.613189in}{0.573745in}}{\pgfqpoint{6.608799in}{0.584345in}}{\pgfqpoint{6.600985in}{0.592158in}}%
\pgfpathcurveto{\pgfqpoint{6.593172in}{0.599972in}}{\pgfqpoint{6.582573in}{0.604362in}}{\pgfqpoint{6.571523in}{0.604362in}}%
\pgfpathcurveto{\pgfqpoint{6.560472in}{0.604362in}}{\pgfqpoint{6.549873in}{0.599972in}}{\pgfqpoint{6.542060in}{0.592158in}}%
\pgfpathcurveto{\pgfqpoint{6.534246in}{0.584345in}}{\pgfqpoint{6.529856in}{0.573745in}}{\pgfqpoint{6.529856in}{0.562695in}}%
\pgfpathcurveto{\pgfqpoint{6.529856in}{0.551645in}}{\pgfqpoint{6.534246in}{0.541046in}}{\pgfqpoint{6.542060in}{0.533233in}}%
\pgfpathcurveto{\pgfqpoint{6.549873in}{0.525419in}}{\pgfqpoint{6.560472in}{0.521029in}}{\pgfqpoint{6.571523in}{0.521029in}}%
\pgfusepath{stroke}%
\end{pgfscope}%
\begin{pgfscope}%
\pgfpathrectangle{\pgfqpoint{0.847223in}{0.554012in}}{\pgfqpoint{6.200000in}{4.620000in}}%
\pgfusepath{clip}%
\pgfsetbuttcap%
\pgfsetroundjoin%
\pgfsetlinewidth{1.003750pt}%
\definecolor{currentstroke}{rgb}{1.000000,0.000000,0.000000}%
\pgfsetstrokecolor{currentstroke}%
\pgfsetdash{}{0pt}%
\pgfpathmoveto{\pgfqpoint{6.580332in}{0.520803in}}%
\pgfpathcurveto{\pgfqpoint{6.591382in}{0.520803in}}{\pgfqpoint{6.601981in}{0.525193in}}{\pgfqpoint{6.609795in}{0.533007in}}%
\pgfpathcurveto{\pgfqpoint{6.617608in}{0.540820in}}{\pgfqpoint{6.621999in}{0.551419in}}{\pgfqpoint{6.621999in}{0.562469in}}%
\pgfpathcurveto{\pgfqpoint{6.621999in}{0.573519in}}{\pgfqpoint{6.617608in}{0.584118in}}{\pgfqpoint{6.609795in}{0.591932in}}%
\pgfpathcurveto{\pgfqpoint{6.601981in}{0.599746in}}{\pgfqpoint{6.591382in}{0.604136in}}{\pgfqpoint{6.580332in}{0.604136in}}%
\pgfpathcurveto{\pgfqpoint{6.569282in}{0.604136in}}{\pgfqpoint{6.558683in}{0.599746in}}{\pgfqpoint{6.550869in}{0.591932in}}%
\pgfpathcurveto{\pgfqpoint{6.543055in}{0.584118in}}{\pgfqpoint{6.538665in}{0.573519in}}{\pgfqpoint{6.538665in}{0.562469in}}%
\pgfpathcurveto{\pgfqpoint{6.538665in}{0.551419in}}{\pgfqpoint{6.543055in}{0.540820in}}{\pgfqpoint{6.550869in}{0.533007in}}%
\pgfpathcurveto{\pgfqpoint{6.558683in}{0.525193in}}{\pgfqpoint{6.569282in}{0.520803in}}{\pgfqpoint{6.580332in}{0.520803in}}%
\pgfusepath{stroke}%
\end{pgfscope}%
\begin{pgfscope}%
\pgfpathrectangle{\pgfqpoint{0.847223in}{0.554012in}}{\pgfqpoint{6.200000in}{4.620000in}}%
\pgfusepath{clip}%
\pgfsetbuttcap%
\pgfsetroundjoin%
\pgfsetlinewidth{1.003750pt}%
\definecolor{currentstroke}{rgb}{1.000000,0.000000,0.000000}%
\pgfsetstrokecolor{currentstroke}%
\pgfsetdash{}{0pt}%
\pgfpathmoveto{\pgfqpoint{6.589141in}{0.520580in}}%
\pgfpathcurveto{\pgfqpoint{6.600191in}{0.520580in}}{\pgfqpoint{6.610790in}{0.524970in}}{\pgfqpoint{6.618604in}{0.532784in}}%
\pgfpathcurveto{\pgfqpoint{6.626418in}{0.540598in}}{\pgfqpoint{6.630808in}{0.551197in}}{\pgfqpoint{6.630808in}{0.562247in}}%
\pgfpathcurveto{\pgfqpoint{6.630808in}{0.573297in}}{\pgfqpoint{6.626418in}{0.583896in}}{\pgfqpoint{6.618604in}{0.591710in}}%
\pgfpathcurveto{\pgfqpoint{6.610790in}{0.599523in}}{\pgfqpoint{6.600191in}{0.603913in}}{\pgfqpoint{6.589141in}{0.603913in}}%
\pgfpathcurveto{\pgfqpoint{6.578091in}{0.603913in}}{\pgfqpoint{6.567492in}{0.599523in}}{\pgfqpoint{6.559678in}{0.591710in}}%
\pgfpathcurveto{\pgfqpoint{6.551865in}{0.583896in}}{\pgfqpoint{6.547474in}{0.573297in}}{\pgfqpoint{6.547474in}{0.562247in}}%
\pgfpathcurveto{\pgfqpoint{6.547474in}{0.551197in}}{\pgfqpoint{6.551865in}{0.540598in}}{\pgfqpoint{6.559678in}{0.532784in}}%
\pgfpathcurveto{\pgfqpoint{6.567492in}{0.524970in}}{\pgfqpoint{6.578091in}{0.520580in}}{\pgfqpoint{6.589141in}{0.520580in}}%
\pgfusepath{stroke}%
\end{pgfscope}%
\begin{pgfscope}%
\pgfpathrectangle{\pgfqpoint{0.847223in}{0.554012in}}{\pgfqpoint{6.200000in}{4.620000in}}%
\pgfusepath{clip}%
\pgfsetbuttcap%
\pgfsetroundjoin%
\pgfsetlinewidth{1.003750pt}%
\definecolor{currentstroke}{rgb}{1.000000,0.000000,0.000000}%
\pgfsetstrokecolor{currentstroke}%
\pgfsetdash{}{0pt}%
\pgfpathmoveto{\pgfqpoint{6.597950in}{0.520361in}}%
\pgfpathcurveto{\pgfqpoint{6.609001in}{0.520361in}}{\pgfqpoint{6.619600in}{0.524751in}}{\pgfqpoint{6.627413in}{0.532565in}}%
\pgfpathcurveto{\pgfqpoint{6.635227in}{0.540379in}}{\pgfqpoint{6.639617in}{0.550978in}}{\pgfqpoint{6.639617in}{0.562028in}}%
\pgfpathcurveto{\pgfqpoint{6.639617in}{0.573078in}}{\pgfqpoint{6.635227in}{0.583677in}}{\pgfqpoint{6.627413in}{0.591491in}}%
\pgfpathcurveto{\pgfqpoint{6.619600in}{0.599304in}}{\pgfqpoint{6.609001in}{0.603695in}}{\pgfqpoint{6.597950in}{0.603695in}}%
\pgfpathcurveto{\pgfqpoint{6.586900in}{0.603695in}}{\pgfqpoint{6.576301in}{0.599304in}}{\pgfqpoint{6.568488in}{0.591491in}}%
\pgfpathcurveto{\pgfqpoint{6.560674in}{0.583677in}}{\pgfqpoint{6.556284in}{0.573078in}}{\pgfqpoint{6.556284in}{0.562028in}}%
\pgfpathcurveto{\pgfqpoint{6.556284in}{0.550978in}}{\pgfqpoint{6.560674in}{0.540379in}}{\pgfqpoint{6.568488in}{0.532565in}}%
\pgfpathcurveto{\pgfqpoint{6.576301in}{0.524751in}}{\pgfqpoint{6.586900in}{0.520361in}}{\pgfqpoint{6.597950in}{0.520361in}}%
\pgfusepath{stroke}%
\end{pgfscope}%
\begin{pgfscope}%
\pgfpathrectangle{\pgfqpoint{0.847223in}{0.554012in}}{\pgfqpoint{6.200000in}{4.620000in}}%
\pgfusepath{clip}%
\pgfsetbuttcap%
\pgfsetroundjoin%
\pgfsetlinewidth{1.003750pt}%
\definecolor{currentstroke}{rgb}{1.000000,0.000000,0.000000}%
\pgfsetstrokecolor{currentstroke}%
\pgfsetdash{}{0pt}%
\pgfpathmoveto{\pgfqpoint{6.606760in}{0.520146in}}%
\pgfpathcurveto{\pgfqpoint{6.617810in}{0.520146in}}{\pgfqpoint{6.628409in}{0.524536in}}{\pgfqpoint{6.636222in}{0.532349in}}%
\pgfpathcurveto{\pgfqpoint{6.644036in}{0.540163in}}{\pgfqpoint{6.648426in}{0.550762in}}{\pgfqpoint{6.648426in}{0.561812in}}%
\pgfpathcurveto{\pgfqpoint{6.648426in}{0.572862in}}{\pgfqpoint{6.644036in}{0.583461in}}{\pgfqpoint{6.636222in}{0.591275in}}%
\pgfpathcurveto{\pgfqpoint{6.628409in}{0.599089in}}{\pgfqpoint{6.617810in}{0.603479in}}{\pgfqpoint{6.606760in}{0.603479in}}%
\pgfpathcurveto{\pgfqpoint{6.595710in}{0.603479in}}{\pgfqpoint{6.585110in}{0.599089in}}{\pgfqpoint{6.577297in}{0.591275in}}%
\pgfpathcurveto{\pgfqpoint{6.569483in}{0.583461in}}{\pgfqpoint{6.565093in}{0.572862in}}{\pgfqpoint{6.565093in}{0.561812in}}%
\pgfpathcurveto{\pgfqpoint{6.565093in}{0.550762in}}{\pgfqpoint{6.569483in}{0.540163in}}{\pgfqpoint{6.577297in}{0.532349in}}%
\pgfpathcurveto{\pgfqpoint{6.585110in}{0.524536in}}{\pgfqpoint{6.595710in}{0.520146in}}{\pgfqpoint{6.606760in}{0.520146in}}%
\pgfusepath{stroke}%
\end{pgfscope}%
\begin{pgfscope}%
\pgfpathrectangle{\pgfqpoint{0.847223in}{0.554012in}}{\pgfqpoint{6.200000in}{4.620000in}}%
\pgfusepath{clip}%
\pgfsetbuttcap%
\pgfsetroundjoin%
\pgfsetlinewidth{1.003750pt}%
\definecolor{currentstroke}{rgb}{1.000000,0.000000,0.000000}%
\pgfsetstrokecolor{currentstroke}%
\pgfsetdash{}{0pt}%
\pgfpathmoveto{\pgfqpoint{6.615569in}{0.519933in}}%
\pgfpathcurveto{\pgfqpoint{6.626619in}{0.519933in}}{\pgfqpoint{6.637218in}{0.524324in}}{\pgfqpoint{6.645032in}{0.532137in}}%
\pgfpathcurveto{\pgfqpoint{6.652845in}{0.539951in}}{\pgfqpoint{6.657236in}{0.550550in}}{\pgfqpoint{6.657236in}{0.561600in}}%
\pgfpathcurveto{\pgfqpoint{6.657236in}{0.572650in}}{\pgfqpoint{6.652845in}{0.583249in}}{\pgfqpoint{6.645032in}{0.591063in}}%
\pgfpathcurveto{\pgfqpoint{6.637218in}{0.598876in}}{\pgfqpoint{6.626619in}{0.603267in}}{\pgfqpoint{6.615569in}{0.603267in}}%
\pgfpathcurveto{\pgfqpoint{6.604519in}{0.603267in}}{\pgfqpoint{6.593920in}{0.598876in}}{\pgfqpoint{6.586106in}{0.591063in}}%
\pgfpathcurveto{\pgfqpoint{6.578293in}{0.583249in}}{\pgfqpoint{6.573902in}{0.572650in}}{\pgfqpoint{6.573902in}{0.561600in}}%
\pgfpathcurveto{\pgfqpoint{6.573902in}{0.550550in}}{\pgfqpoint{6.578293in}{0.539951in}}{\pgfqpoint{6.586106in}{0.532137in}}%
\pgfpathcurveto{\pgfqpoint{6.593920in}{0.524324in}}{\pgfqpoint{6.604519in}{0.519933in}}{\pgfqpoint{6.615569in}{0.519933in}}%
\pgfusepath{stroke}%
\end{pgfscope}%
\begin{pgfscope}%
\pgfpathrectangle{\pgfqpoint{0.847223in}{0.554012in}}{\pgfqpoint{6.200000in}{4.620000in}}%
\pgfusepath{clip}%
\pgfsetbuttcap%
\pgfsetroundjoin%
\pgfsetlinewidth{1.003750pt}%
\definecolor{currentstroke}{rgb}{1.000000,0.000000,0.000000}%
\pgfsetstrokecolor{currentstroke}%
\pgfsetdash{}{0pt}%
\pgfpathmoveto{\pgfqpoint{6.624378in}{0.519724in}}%
\pgfpathcurveto{\pgfqpoint{6.635428in}{0.519724in}}{\pgfqpoint{6.646027in}{0.524115in}}{\pgfqpoint{6.653841in}{0.531928in}}%
\pgfpathcurveto{\pgfqpoint{6.661655in}{0.539742in}}{\pgfqpoint{6.666045in}{0.550341in}}{\pgfqpoint{6.666045in}{0.561391in}}%
\pgfpathcurveto{\pgfqpoint{6.666045in}{0.572441in}}{\pgfqpoint{6.661655in}{0.583040in}}{\pgfqpoint{6.653841in}{0.590854in}}%
\pgfpathcurveto{\pgfqpoint{6.646027in}{0.598667in}}{\pgfqpoint{6.635428in}{0.603058in}}{\pgfqpoint{6.624378in}{0.603058in}}%
\pgfpathcurveto{\pgfqpoint{6.613328in}{0.603058in}}{\pgfqpoint{6.602729in}{0.598667in}}{\pgfqpoint{6.594915in}{0.590854in}}%
\pgfpathcurveto{\pgfqpoint{6.587102in}{0.583040in}}{\pgfqpoint{6.582712in}{0.572441in}}{\pgfqpoint{6.582712in}{0.561391in}}%
\pgfpathcurveto{\pgfqpoint{6.582712in}{0.550341in}}{\pgfqpoint{6.587102in}{0.539742in}}{\pgfqpoint{6.594915in}{0.531928in}}%
\pgfpathcurveto{\pgfqpoint{6.602729in}{0.524115in}}{\pgfqpoint{6.613328in}{0.519724in}}{\pgfqpoint{6.624378in}{0.519724in}}%
\pgfusepath{stroke}%
\end{pgfscope}%
\begin{pgfscope}%
\pgfpathrectangle{\pgfqpoint{0.847223in}{0.554012in}}{\pgfqpoint{6.200000in}{4.620000in}}%
\pgfusepath{clip}%
\pgfsetbuttcap%
\pgfsetroundjoin%
\pgfsetlinewidth{1.003750pt}%
\definecolor{currentstroke}{rgb}{1.000000,0.000000,0.000000}%
\pgfsetstrokecolor{currentstroke}%
\pgfsetdash{}{0pt}%
\pgfpathmoveto{\pgfqpoint{6.633187in}{0.519518in}}%
\pgfpathcurveto{\pgfqpoint{6.644238in}{0.519518in}}{\pgfqpoint{6.654837in}{0.523909in}}{\pgfqpoint{6.662650in}{0.531722in}}%
\pgfpathcurveto{\pgfqpoint{6.670464in}{0.539536in}}{\pgfqpoint{6.674854in}{0.550135in}}{\pgfqpoint{6.674854in}{0.561185in}}%
\pgfpathcurveto{\pgfqpoint{6.674854in}{0.572235in}}{\pgfqpoint{6.670464in}{0.582834in}}{\pgfqpoint{6.662650in}{0.590648in}}%
\pgfpathcurveto{\pgfqpoint{6.654837in}{0.598461in}}{\pgfqpoint{6.644238in}{0.602852in}}{\pgfqpoint{6.633187in}{0.602852in}}%
\pgfpathcurveto{\pgfqpoint{6.622137in}{0.602852in}}{\pgfqpoint{6.611538in}{0.598461in}}{\pgfqpoint{6.603725in}{0.590648in}}%
\pgfpathcurveto{\pgfqpoint{6.595911in}{0.582834in}}{\pgfqpoint{6.591521in}{0.572235in}}{\pgfqpoint{6.591521in}{0.561185in}}%
\pgfpathcurveto{\pgfqpoint{6.591521in}{0.550135in}}{\pgfqpoint{6.595911in}{0.539536in}}{\pgfqpoint{6.603725in}{0.531722in}}%
\pgfpathcurveto{\pgfqpoint{6.611538in}{0.523909in}}{\pgfqpoint{6.622137in}{0.519518in}}{\pgfqpoint{6.633187in}{0.519518in}}%
\pgfusepath{stroke}%
\end{pgfscope}%
\begin{pgfscope}%
\pgfpathrectangle{\pgfqpoint{0.847223in}{0.554012in}}{\pgfqpoint{6.200000in}{4.620000in}}%
\pgfusepath{clip}%
\pgfsetbuttcap%
\pgfsetroundjoin%
\pgfsetlinewidth{1.003750pt}%
\definecolor{currentstroke}{rgb}{1.000000,0.000000,0.000000}%
\pgfsetstrokecolor{currentstroke}%
\pgfsetdash{}{0pt}%
\pgfpathmoveto{\pgfqpoint{6.641997in}{0.519316in}}%
\pgfpathcurveto{\pgfqpoint{6.653047in}{0.519316in}}{\pgfqpoint{6.663646in}{0.523706in}}{\pgfqpoint{6.671459in}{0.531519in}}%
\pgfpathcurveto{\pgfqpoint{6.679273in}{0.539333in}}{\pgfqpoint{6.683663in}{0.549932in}}{\pgfqpoint{6.683663in}{0.560982in}}%
\pgfpathcurveto{\pgfqpoint{6.683663in}{0.572032in}}{\pgfqpoint{6.679273in}{0.582631in}}{\pgfqpoint{6.671459in}{0.590445in}}%
\pgfpathcurveto{\pgfqpoint{6.663646in}{0.598259in}}{\pgfqpoint{6.653047in}{0.602649in}}{\pgfqpoint{6.641997in}{0.602649in}}%
\pgfpathcurveto{\pgfqpoint{6.630947in}{0.602649in}}{\pgfqpoint{6.620348in}{0.598259in}}{\pgfqpoint{6.612534in}{0.590445in}}%
\pgfpathcurveto{\pgfqpoint{6.604720in}{0.582631in}}{\pgfqpoint{6.600330in}{0.572032in}}{\pgfqpoint{6.600330in}{0.560982in}}%
\pgfpathcurveto{\pgfqpoint{6.600330in}{0.549932in}}{\pgfqpoint{6.604720in}{0.539333in}}{\pgfqpoint{6.612534in}{0.531519in}}%
\pgfpathcurveto{\pgfqpoint{6.620348in}{0.523706in}}{\pgfqpoint{6.630947in}{0.519316in}}{\pgfqpoint{6.641997in}{0.519316in}}%
\pgfusepath{stroke}%
\end{pgfscope}%
\begin{pgfscope}%
\pgfpathrectangle{\pgfqpoint{0.847223in}{0.554012in}}{\pgfqpoint{6.200000in}{4.620000in}}%
\pgfusepath{clip}%
\pgfsetbuttcap%
\pgfsetroundjoin%
\pgfsetlinewidth{1.003750pt}%
\definecolor{currentstroke}{rgb}{1.000000,0.000000,0.000000}%
\pgfsetstrokecolor{currentstroke}%
\pgfsetdash{}{0pt}%
\pgfpathmoveto{\pgfqpoint{6.650806in}{0.519116in}}%
\pgfpathcurveto{\pgfqpoint{6.661856in}{0.519116in}}{\pgfqpoint{6.672455in}{0.523506in}}{\pgfqpoint{6.680269in}{0.531320in}}%
\pgfpathcurveto{\pgfqpoint{6.688082in}{0.539133in}}{\pgfqpoint{6.692473in}{0.549732in}}{\pgfqpoint{6.692473in}{0.560782in}}%
\pgfpathcurveto{\pgfqpoint{6.692473in}{0.571832in}}{\pgfqpoint{6.688082in}{0.582431in}}{\pgfqpoint{6.680269in}{0.590245in}}%
\pgfpathcurveto{\pgfqpoint{6.672455in}{0.598059in}}{\pgfqpoint{6.661856in}{0.602449in}}{\pgfqpoint{6.650806in}{0.602449in}}%
\pgfpathcurveto{\pgfqpoint{6.639756in}{0.602449in}}{\pgfqpoint{6.629157in}{0.598059in}}{\pgfqpoint{6.621343in}{0.590245in}}%
\pgfpathcurveto{\pgfqpoint{6.613530in}{0.582431in}}{\pgfqpoint{6.609139in}{0.571832in}}{\pgfqpoint{6.609139in}{0.560782in}}%
\pgfpathcurveto{\pgfqpoint{6.609139in}{0.549732in}}{\pgfqpoint{6.613530in}{0.539133in}}{\pgfqpoint{6.621343in}{0.531320in}}%
\pgfpathcurveto{\pgfqpoint{6.629157in}{0.523506in}}{\pgfqpoint{6.639756in}{0.519116in}}{\pgfqpoint{6.650806in}{0.519116in}}%
\pgfusepath{stroke}%
\end{pgfscope}%
\begin{pgfscope}%
\pgfpathrectangle{\pgfqpoint{0.847223in}{0.554012in}}{\pgfqpoint{6.200000in}{4.620000in}}%
\pgfusepath{clip}%
\pgfsetbuttcap%
\pgfsetroundjoin%
\pgfsetlinewidth{1.003750pt}%
\definecolor{currentstroke}{rgb}{1.000000,0.000000,0.000000}%
\pgfsetstrokecolor{currentstroke}%
\pgfsetdash{}{0pt}%
\pgfpathmoveto{\pgfqpoint{6.659615in}{0.518919in}}%
\pgfpathcurveto{\pgfqpoint{6.670665in}{0.518919in}}{\pgfqpoint{6.681264in}{0.523309in}}{\pgfqpoint{6.689078in}{0.531123in}}%
\pgfpathcurveto{\pgfqpoint{6.696892in}{0.538936in}}{\pgfqpoint{6.701282in}{0.549535in}}{\pgfqpoint{6.701282in}{0.560585in}}%
\pgfpathcurveto{\pgfqpoint{6.701282in}{0.571635in}}{\pgfqpoint{6.696892in}{0.582235in}}{\pgfqpoint{6.689078in}{0.590048in}}%
\pgfpathcurveto{\pgfqpoint{6.681264in}{0.597862in}}{\pgfqpoint{6.670665in}{0.602252in}}{\pgfqpoint{6.659615in}{0.602252in}}%
\pgfpathcurveto{\pgfqpoint{6.648565in}{0.602252in}}{\pgfqpoint{6.637966in}{0.597862in}}{\pgfqpoint{6.630152in}{0.590048in}}%
\pgfpathcurveto{\pgfqpoint{6.622339in}{0.582235in}}{\pgfqpoint{6.617949in}{0.571635in}}{\pgfqpoint{6.617949in}{0.560585in}}%
\pgfpathcurveto{\pgfqpoint{6.617949in}{0.549535in}}{\pgfqpoint{6.622339in}{0.538936in}}{\pgfqpoint{6.630152in}{0.531123in}}%
\pgfpathcurveto{\pgfqpoint{6.637966in}{0.523309in}}{\pgfqpoint{6.648565in}{0.518919in}}{\pgfqpoint{6.659615in}{0.518919in}}%
\pgfusepath{stroke}%
\end{pgfscope}%
\begin{pgfscope}%
\pgfpathrectangle{\pgfqpoint{0.847223in}{0.554012in}}{\pgfqpoint{6.200000in}{4.620000in}}%
\pgfusepath{clip}%
\pgfsetbuttcap%
\pgfsetroundjoin%
\pgfsetlinewidth{1.003750pt}%
\definecolor{currentstroke}{rgb}{1.000000,0.000000,0.000000}%
\pgfsetstrokecolor{currentstroke}%
\pgfsetdash{}{0pt}%
\pgfpathmoveto{\pgfqpoint{6.668425in}{0.518725in}}%
\pgfpathcurveto{\pgfqpoint{6.679475in}{0.518725in}}{\pgfqpoint{6.690074in}{0.523115in}}{\pgfqpoint{6.697887in}{0.530928in}}%
\pgfpathcurveto{\pgfqpoint{6.705701in}{0.538742in}}{\pgfqpoint{6.710091in}{0.549341in}}{\pgfqpoint{6.710091in}{0.560391in}}%
\pgfpathcurveto{\pgfqpoint{6.710091in}{0.571441in}}{\pgfqpoint{6.705701in}{0.582040in}}{\pgfqpoint{6.697887in}{0.589854in}}%
\pgfpathcurveto{\pgfqpoint{6.690074in}{0.597668in}}{\pgfqpoint{6.679475in}{0.602058in}}{\pgfqpoint{6.668425in}{0.602058in}}%
\pgfpathcurveto{\pgfqpoint{6.657374in}{0.602058in}}{\pgfqpoint{6.646775in}{0.597668in}}{\pgfqpoint{6.638962in}{0.589854in}}%
\pgfpathcurveto{\pgfqpoint{6.631148in}{0.582040in}}{\pgfqpoint{6.626758in}{0.571441in}}{\pgfqpoint{6.626758in}{0.560391in}}%
\pgfpathcurveto{\pgfqpoint{6.626758in}{0.549341in}}{\pgfqpoint{6.631148in}{0.538742in}}{\pgfqpoint{6.638962in}{0.530928in}}%
\pgfpathcurveto{\pgfqpoint{6.646775in}{0.523115in}}{\pgfqpoint{6.657374in}{0.518725in}}{\pgfqpoint{6.668425in}{0.518725in}}%
\pgfusepath{stroke}%
\end{pgfscope}%
\begin{pgfscope}%
\pgfpathrectangle{\pgfqpoint{0.847223in}{0.554012in}}{\pgfqpoint{6.200000in}{4.620000in}}%
\pgfusepath{clip}%
\pgfsetbuttcap%
\pgfsetroundjoin%
\pgfsetlinewidth{1.003750pt}%
\definecolor{currentstroke}{rgb}{1.000000,0.000000,0.000000}%
\pgfsetstrokecolor{currentstroke}%
\pgfsetdash{}{0pt}%
\pgfpathmoveto{\pgfqpoint{6.677234in}{0.518533in}}%
\pgfpathcurveto{\pgfqpoint{6.688284in}{0.518533in}}{\pgfqpoint{6.698883in}{0.522923in}}{\pgfqpoint{6.706697in}{0.530737in}}%
\pgfpathcurveto{\pgfqpoint{6.714510in}{0.538551in}}{\pgfqpoint{6.718900in}{0.549150in}}{\pgfqpoint{6.718900in}{0.560200in}}%
\pgfpathcurveto{\pgfqpoint{6.718900in}{0.571250in}}{\pgfqpoint{6.714510in}{0.581849in}}{\pgfqpoint{6.706697in}{0.589663in}}%
\pgfpathcurveto{\pgfqpoint{6.698883in}{0.597476in}}{\pgfqpoint{6.688284in}{0.601867in}}{\pgfqpoint{6.677234in}{0.601867in}}%
\pgfpathcurveto{\pgfqpoint{6.666184in}{0.601867in}}{\pgfqpoint{6.655585in}{0.597476in}}{\pgfqpoint{6.647771in}{0.589663in}}%
\pgfpathcurveto{\pgfqpoint{6.639957in}{0.581849in}}{\pgfqpoint{6.635567in}{0.571250in}}{\pgfqpoint{6.635567in}{0.560200in}}%
\pgfpathcurveto{\pgfqpoint{6.635567in}{0.549150in}}{\pgfqpoint{6.639957in}{0.538551in}}{\pgfqpoint{6.647771in}{0.530737in}}%
\pgfpathcurveto{\pgfqpoint{6.655585in}{0.522923in}}{\pgfqpoint{6.666184in}{0.518533in}}{\pgfqpoint{6.677234in}{0.518533in}}%
\pgfusepath{stroke}%
\end{pgfscope}%
\begin{pgfscope}%
\pgfpathrectangle{\pgfqpoint{0.847223in}{0.554012in}}{\pgfqpoint{6.200000in}{4.620000in}}%
\pgfusepath{clip}%
\pgfsetbuttcap%
\pgfsetroundjoin%
\pgfsetlinewidth{1.003750pt}%
\definecolor{currentstroke}{rgb}{1.000000,0.000000,0.000000}%
\pgfsetstrokecolor{currentstroke}%
\pgfsetdash{}{0pt}%
\pgfpathmoveto{\pgfqpoint{6.686043in}{0.518345in}}%
\pgfpathcurveto{\pgfqpoint{6.697093in}{0.518345in}}{\pgfqpoint{6.707692in}{0.522735in}}{\pgfqpoint{6.715506in}{0.530548in}}%
\pgfpathcurveto{\pgfqpoint{6.723319in}{0.538362in}}{\pgfqpoint{6.727710in}{0.548961in}}{\pgfqpoint{6.727710in}{0.560011in}}%
\pgfpathcurveto{\pgfqpoint{6.727710in}{0.571061in}}{\pgfqpoint{6.723319in}{0.581660in}}{\pgfqpoint{6.715506in}{0.589474in}}%
\pgfpathcurveto{\pgfqpoint{6.707692in}{0.597288in}}{\pgfqpoint{6.697093in}{0.601678in}}{\pgfqpoint{6.686043in}{0.601678in}}%
\pgfpathcurveto{\pgfqpoint{6.674993in}{0.601678in}}{\pgfqpoint{6.664394in}{0.597288in}}{\pgfqpoint{6.656580in}{0.589474in}}%
\pgfpathcurveto{\pgfqpoint{6.648767in}{0.581660in}}{\pgfqpoint{6.644376in}{0.571061in}}{\pgfqpoint{6.644376in}{0.560011in}}%
\pgfpathcurveto{\pgfqpoint{6.644376in}{0.548961in}}{\pgfqpoint{6.648767in}{0.538362in}}{\pgfqpoint{6.656580in}{0.530548in}}%
\pgfpathcurveto{\pgfqpoint{6.664394in}{0.522735in}}{\pgfqpoint{6.674993in}{0.518345in}}{\pgfqpoint{6.686043in}{0.518345in}}%
\pgfusepath{stroke}%
\end{pgfscope}%
\begin{pgfscope}%
\pgfpathrectangle{\pgfqpoint{0.847223in}{0.554012in}}{\pgfqpoint{6.200000in}{4.620000in}}%
\pgfusepath{clip}%
\pgfsetbuttcap%
\pgfsetroundjoin%
\pgfsetlinewidth{1.003750pt}%
\definecolor{currentstroke}{rgb}{1.000000,0.000000,0.000000}%
\pgfsetstrokecolor{currentstroke}%
\pgfsetdash{}{0pt}%
\pgfpathmoveto{\pgfqpoint{6.694852in}{0.518159in}}%
\pgfpathcurveto{\pgfqpoint{6.705902in}{0.518159in}}{\pgfqpoint{6.716501in}{0.522549in}}{\pgfqpoint{6.724315in}{0.530362in}}%
\pgfpathcurveto{\pgfqpoint{6.732129in}{0.538176in}}{\pgfqpoint{6.736519in}{0.548775in}}{\pgfqpoint{6.736519in}{0.559825in}}%
\pgfpathcurveto{\pgfqpoint{6.736519in}{0.570875in}}{\pgfqpoint{6.732129in}{0.581474in}}{\pgfqpoint{6.724315in}{0.589288in}}%
\pgfpathcurveto{\pgfqpoint{6.716501in}{0.597102in}}{\pgfqpoint{6.705902in}{0.601492in}}{\pgfqpoint{6.694852in}{0.601492in}}%
\pgfpathcurveto{\pgfqpoint{6.683802in}{0.601492in}}{\pgfqpoint{6.673203in}{0.597102in}}{\pgfqpoint{6.665390in}{0.589288in}}%
\pgfpathcurveto{\pgfqpoint{6.657576in}{0.581474in}}{\pgfqpoint{6.653186in}{0.570875in}}{\pgfqpoint{6.653186in}{0.559825in}}%
\pgfpathcurveto{\pgfqpoint{6.653186in}{0.548775in}}{\pgfqpoint{6.657576in}{0.538176in}}{\pgfqpoint{6.665390in}{0.530362in}}%
\pgfpathcurveto{\pgfqpoint{6.673203in}{0.522549in}}{\pgfqpoint{6.683802in}{0.518159in}}{\pgfqpoint{6.694852in}{0.518159in}}%
\pgfusepath{stroke}%
\end{pgfscope}%
\begin{pgfscope}%
\pgfpathrectangle{\pgfqpoint{0.847223in}{0.554012in}}{\pgfqpoint{6.200000in}{4.620000in}}%
\pgfusepath{clip}%
\pgfsetbuttcap%
\pgfsetroundjoin%
\pgfsetlinewidth{1.003750pt}%
\definecolor{currentstroke}{rgb}{1.000000,0.000000,0.000000}%
\pgfsetstrokecolor{currentstroke}%
\pgfsetdash{}{0pt}%
\pgfpathmoveto{\pgfqpoint{6.703662in}{0.517975in}}%
\pgfpathcurveto{\pgfqpoint{6.714712in}{0.517975in}}{\pgfqpoint{6.725311in}{0.522365in}}{\pgfqpoint{6.733124in}{0.530179in}}%
\pgfpathcurveto{\pgfqpoint{6.740938in}{0.537993in}}{\pgfqpoint{6.745328in}{0.548592in}}{\pgfqpoint{6.745328in}{0.559642in}}%
\pgfpathcurveto{\pgfqpoint{6.745328in}{0.570692in}}{\pgfqpoint{6.740938in}{0.581291in}}{\pgfqpoint{6.733124in}{0.589105in}}%
\pgfpathcurveto{\pgfqpoint{6.725311in}{0.596918in}}{\pgfqpoint{6.714712in}{0.601308in}}{\pgfqpoint{6.703662in}{0.601308in}}%
\pgfpathcurveto{\pgfqpoint{6.692611in}{0.601308in}}{\pgfqpoint{6.682012in}{0.596918in}}{\pgfqpoint{6.674199in}{0.589105in}}%
\pgfpathcurveto{\pgfqpoint{6.666385in}{0.581291in}}{\pgfqpoint{6.661995in}{0.570692in}}{\pgfqpoint{6.661995in}{0.559642in}}%
\pgfpathcurveto{\pgfqpoint{6.661995in}{0.548592in}}{\pgfqpoint{6.666385in}{0.537993in}}{\pgfqpoint{6.674199in}{0.530179in}}%
\pgfpathcurveto{\pgfqpoint{6.682012in}{0.522365in}}{\pgfqpoint{6.692611in}{0.517975in}}{\pgfqpoint{6.703662in}{0.517975in}}%
\pgfusepath{stroke}%
\end{pgfscope}%
\begin{pgfscope}%
\pgfpathrectangle{\pgfqpoint{0.847223in}{0.554012in}}{\pgfqpoint{6.200000in}{4.620000in}}%
\pgfusepath{clip}%
\pgfsetbuttcap%
\pgfsetroundjoin%
\pgfsetlinewidth{1.003750pt}%
\definecolor{currentstroke}{rgb}{1.000000,0.000000,0.000000}%
\pgfsetstrokecolor{currentstroke}%
\pgfsetdash{}{0pt}%
\pgfpathmoveto{\pgfqpoint{6.712471in}{0.517794in}}%
\pgfpathcurveto{\pgfqpoint{6.723521in}{0.517794in}}{\pgfqpoint{6.734120in}{0.522184in}}{\pgfqpoint{6.741934in}{0.529998in}}%
\pgfpathcurveto{\pgfqpoint{6.749747in}{0.537812in}}{\pgfqpoint{6.754138in}{0.548411in}}{\pgfqpoint{6.754138in}{0.559461in}}%
\pgfpathcurveto{\pgfqpoint{6.754138in}{0.570511in}}{\pgfqpoint{6.749747in}{0.581110in}}{\pgfqpoint{6.741934in}{0.588924in}}%
\pgfpathcurveto{\pgfqpoint{6.734120in}{0.596737in}}{\pgfqpoint{6.723521in}{0.601128in}}{\pgfqpoint{6.712471in}{0.601128in}}%
\pgfpathcurveto{\pgfqpoint{6.701421in}{0.601128in}}{\pgfqpoint{6.690822in}{0.596737in}}{\pgfqpoint{6.683008in}{0.588924in}}%
\pgfpathcurveto{\pgfqpoint{6.675194in}{0.581110in}}{\pgfqpoint{6.670804in}{0.570511in}}{\pgfqpoint{6.670804in}{0.559461in}}%
\pgfpathcurveto{\pgfqpoint{6.670804in}{0.548411in}}{\pgfqpoint{6.675194in}{0.537812in}}{\pgfqpoint{6.683008in}{0.529998in}}%
\pgfpathcurveto{\pgfqpoint{6.690822in}{0.522184in}}{\pgfqpoint{6.701421in}{0.517794in}}{\pgfqpoint{6.712471in}{0.517794in}}%
\pgfusepath{stroke}%
\end{pgfscope}%
\begin{pgfscope}%
\pgfpathrectangle{\pgfqpoint{0.847223in}{0.554012in}}{\pgfqpoint{6.200000in}{4.620000in}}%
\pgfusepath{clip}%
\pgfsetbuttcap%
\pgfsetroundjoin%
\pgfsetlinewidth{1.003750pt}%
\definecolor{currentstroke}{rgb}{1.000000,0.000000,0.000000}%
\pgfsetstrokecolor{currentstroke}%
\pgfsetdash{}{0pt}%
\pgfpathmoveto{\pgfqpoint{6.721280in}{0.517616in}}%
\pgfpathcurveto{\pgfqpoint{6.732330in}{0.517616in}}{\pgfqpoint{6.742929in}{0.522006in}}{\pgfqpoint{6.750743in}{0.529820in}}%
\pgfpathcurveto{\pgfqpoint{6.758557in}{0.537633in}}{\pgfqpoint{6.762947in}{0.548232in}}{\pgfqpoint{6.762947in}{0.559282in}}%
\pgfpathcurveto{\pgfqpoint{6.762947in}{0.570333in}}{\pgfqpoint{6.758557in}{0.580932in}}{\pgfqpoint{6.750743in}{0.588745in}}%
\pgfpathcurveto{\pgfqpoint{6.742929in}{0.596559in}}{\pgfqpoint{6.732330in}{0.600949in}}{\pgfqpoint{6.721280in}{0.600949in}}%
\pgfpathcurveto{\pgfqpoint{6.710230in}{0.600949in}}{\pgfqpoint{6.699631in}{0.596559in}}{\pgfqpoint{6.691817in}{0.588745in}}%
\pgfpathcurveto{\pgfqpoint{6.684004in}{0.580932in}}{\pgfqpoint{6.679613in}{0.570333in}}{\pgfqpoint{6.679613in}{0.559282in}}%
\pgfpathcurveto{\pgfqpoint{6.679613in}{0.548232in}}{\pgfqpoint{6.684004in}{0.537633in}}{\pgfqpoint{6.691817in}{0.529820in}}%
\pgfpathcurveto{\pgfqpoint{6.699631in}{0.522006in}}{\pgfqpoint{6.710230in}{0.517616in}}{\pgfqpoint{6.721280in}{0.517616in}}%
\pgfusepath{stroke}%
\end{pgfscope}%
\begin{pgfscope}%
\pgfpathrectangle{\pgfqpoint{0.847223in}{0.554012in}}{\pgfqpoint{6.200000in}{4.620000in}}%
\pgfusepath{clip}%
\pgfsetbuttcap%
\pgfsetroundjoin%
\pgfsetlinewidth{1.003750pt}%
\definecolor{currentstroke}{rgb}{1.000000,0.000000,0.000000}%
\pgfsetstrokecolor{currentstroke}%
\pgfsetdash{}{0pt}%
\pgfpathmoveto{\pgfqpoint{6.730089in}{0.517440in}}%
\pgfpathcurveto{\pgfqpoint{6.741140in}{0.517440in}}{\pgfqpoint{6.751739in}{0.521830in}}{\pgfqpoint{6.759552in}{0.529644in}}%
\pgfpathcurveto{\pgfqpoint{6.767366in}{0.537457in}}{\pgfqpoint{6.771756in}{0.548056in}}{\pgfqpoint{6.771756in}{0.559106in}}%
\pgfpathcurveto{\pgfqpoint{6.771756in}{0.570156in}}{\pgfqpoint{6.767366in}{0.580755in}}{\pgfqpoint{6.759552in}{0.588569in}}%
\pgfpathcurveto{\pgfqpoint{6.751739in}{0.596383in}}{\pgfqpoint{6.741140in}{0.600773in}}{\pgfqpoint{6.730089in}{0.600773in}}%
\pgfpathcurveto{\pgfqpoint{6.719039in}{0.600773in}}{\pgfqpoint{6.708440in}{0.596383in}}{\pgfqpoint{6.700627in}{0.588569in}}%
\pgfpathcurveto{\pgfqpoint{6.692813in}{0.580755in}}{\pgfqpoint{6.688423in}{0.570156in}}{\pgfqpoint{6.688423in}{0.559106in}}%
\pgfpathcurveto{\pgfqpoint{6.688423in}{0.548056in}}{\pgfqpoint{6.692813in}{0.537457in}}{\pgfqpoint{6.700627in}{0.529644in}}%
\pgfpathcurveto{\pgfqpoint{6.708440in}{0.521830in}}{\pgfqpoint{6.719039in}{0.517440in}}{\pgfqpoint{6.730089in}{0.517440in}}%
\pgfusepath{stroke}%
\end{pgfscope}%
\begin{pgfscope}%
\pgfpathrectangle{\pgfqpoint{0.847223in}{0.554012in}}{\pgfqpoint{6.200000in}{4.620000in}}%
\pgfusepath{clip}%
\pgfsetbuttcap%
\pgfsetroundjoin%
\pgfsetlinewidth{1.003750pt}%
\definecolor{currentstroke}{rgb}{1.000000,0.000000,0.000000}%
\pgfsetstrokecolor{currentstroke}%
\pgfsetdash{}{0pt}%
\pgfpathmoveto{\pgfqpoint{6.738899in}{0.517266in}}%
\pgfpathcurveto{\pgfqpoint{6.749949in}{0.517266in}}{\pgfqpoint{6.760548in}{0.521656in}}{\pgfqpoint{6.768361in}{0.529470in}}%
\pgfpathcurveto{\pgfqpoint{6.776175in}{0.537283in}}{\pgfqpoint{6.780565in}{0.547882in}}{\pgfqpoint{6.780565in}{0.558933in}}%
\pgfpathcurveto{\pgfqpoint{6.780565in}{0.569983in}}{\pgfqpoint{6.776175in}{0.580582in}}{\pgfqpoint{6.768361in}{0.588395in}}%
\pgfpathcurveto{\pgfqpoint{6.760548in}{0.596209in}}{\pgfqpoint{6.749949in}{0.600599in}}{\pgfqpoint{6.738899in}{0.600599in}}%
\pgfpathcurveto{\pgfqpoint{6.727849in}{0.600599in}}{\pgfqpoint{6.717250in}{0.596209in}}{\pgfqpoint{6.709436in}{0.588395in}}%
\pgfpathcurveto{\pgfqpoint{6.701622in}{0.580582in}}{\pgfqpoint{6.697232in}{0.569983in}}{\pgfqpoint{6.697232in}{0.558933in}}%
\pgfpathcurveto{\pgfqpoint{6.697232in}{0.547882in}}{\pgfqpoint{6.701622in}{0.537283in}}{\pgfqpoint{6.709436in}{0.529470in}}%
\pgfpathcurveto{\pgfqpoint{6.717250in}{0.521656in}}{\pgfqpoint{6.727849in}{0.517266in}}{\pgfqpoint{6.738899in}{0.517266in}}%
\pgfusepath{stroke}%
\end{pgfscope}%
\begin{pgfscope}%
\pgfpathrectangle{\pgfqpoint{0.847223in}{0.554012in}}{\pgfqpoint{6.200000in}{4.620000in}}%
\pgfusepath{clip}%
\pgfsetbuttcap%
\pgfsetroundjoin%
\pgfsetlinewidth{1.003750pt}%
\definecolor{currentstroke}{rgb}{1.000000,0.000000,0.000000}%
\pgfsetstrokecolor{currentstroke}%
\pgfsetdash{}{0pt}%
\pgfpathmoveto{\pgfqpoint{6.747708in}{0.517095in}}%
\pgfpathcurveto{\pgfqpoint{6.758758in}{0.517095in}}{\pgfqpoint{6.769357in}{0.521485in}}{\pgfqpoint{6.777171in}{0.529298in}}%
\pgfpathcurveto{\pgfqpoint{6.784984in}{0.537112in}}{\pgfqpoint{6.789375in}{0.547711in}}{\pgfqpoint{6.789375in}{0.558761in}}%
\pgfpathcurveto{\pgfqpoint{6.789375in}{0.569811in}}{\pgfqpoint{6.784984in}{0.580410in}}{\pgfqpoint{6.777171in}{0.588224in}}%
\pgfpathcurveto{\pgfqpoint{6.769357in}{0.596038in}}{\pgfqpoint{6.758758in}{0.600428in}}{\pgfqpoint{6.747708in}{0.600428in}}%
\pgfpathcurveto{\pgfqpoint{6.736658in}{0.600428in}}{\pgfqpoint{6.726059in}{0.596038in}}{\pgfqpoint{6.718245in}{0.588224in}}%
\pgfpathcurveto{\pgfqpoint{6.710432in}{0.580410in}}{\pgfqpoint{6.706041in}{0.569811in}}{\pgfqpoint{6.706041in}{0.558761in}}%
\pgfpathcurveto{\pgfqpoint{6.706041in}{0.547711in}}{\pgfqpoint{6.710432in}{0.537112in}}{\pgfqpoint{6.718245in}{0.529298in}}%
\pgfpathcurveto{\pgfqpoint{6.726059in}{0.521485in}}{\pgfqpoint{6.736658in}{0.517095in}}{\pgfqpoint{6.747708in}{0.517095in}}%
\pgfusepath{stroke}%
\end{pgfscope}%
\begin{pgfscope}%
\pgfpathrectangle{\pgfqpoint{0.847223in}{0.554012in}}{\pgfqpoint{6.200000in}{4.620000in}}%
\pgfusepath{clip}%
\pgfsetbuttcap%
\pgfsetroundjoin%
\pgfsetlinewidth{1.003750pt}%
\definecolor{currentstroke}{rgb}{1.000000,0.000000,0.000000}%
\pgfsetstrokecolor{currentstroke}%
\pgfsetdash{}{0pt}%
\pgfpathmoveto{\pgfqpoint{6.756517in}{0.516925in}}%
\pgfpathcurveto{\pgfqpoint{6.767567in}{0.516925in}}{\pgfqpoint{6.778166in}{0.521316in}}{\pgfqpoint{6.785980in}{0.529129in}}%
\pgfpathcurveto{\pgfqpoint{6.793794in}{0.536943in}}{\pgfqpoint{6.798184in}{0.547542in}}{\pgfqpoint{6.798184in}{0.558592in}}%
\pgfpathcurveto{\pgfqpoint{6.798184in}{0.569642in}}{\pgfqpoint{6.793794in}{0.580241in}}{\pgfqpoint{6.785980in}{0.588055in}}%
\pgfpathcurveto{\pgfqpoint{6.778166in}{0.595868in}}{\pgfqpoint{6.767567in}{0.600259in}}{\pgfqpoint{6.756517in}{0.600259in}}%
\pgfpathcurveto{\pgfqpoint{6.745467in}{0.600259in}}{\pgfqpoint{6.734868in}{0.595868in}}{\pgfqpoint{6.727054in}{0.588055in}}%
\pgfpathcurveto{\pgfqpoint{6.719241in}{0.580241in}}{\pgfqpoint{6.714851in}{0.569642in}}{\pgfqpoint{6.714851in}{0.558592in}}%
\pgfpathcurveto{\pgfqpoint{6.714851in}{0.547542in}}{\pgfqpoint{6.719241in}{0.536943in}}{\pgfqpoint{6.727054in}{0.529129in}}%
\pgfpathcurveto{\pgfqpoint{6.734868in}{0.521316in}}{\pgfqpoint{6.745467in}{0.516925in}}{\pgfqpoint{6.756517in}{0.516925in}}%
\pgfusepath{stroke}%
\end{pgfscope}%
\begin{pgfscope}%
\pgfpathrectangle{\pgfqpoint{0.847223in}{0.554012in}}{\pgfqpoint{6.200000in}{4.620000in}}%
\pgfusepath{clip}%
\pgfsetbuttcap%
\pgfsetroundjoin%
\pgfsetlinewidth{1.003750pt}%
\definecolor{currentstroke}{rgb}{1.000000,0.000000,0.000000}%
\pgfsetstrokecolor{currentstroke}%
\pgfsetdash{}{0pt}%
\pgfpathmoveto{\pgfqpoint{6.765326in}{0.516758in}}%
\pgfpathcurveto{\pgfqpoint{6.776377in}{0.516758in}}{\pgfqpoint{6.786976in}{0.521149in}}{\pgfqpoint{6.794789in}{0.528962in}}%
\pgfpathcurveto{\pgfqpoint{6.802603in}{0.536776in}}{\pgfqpoint{6.806993in}{0.547375in}}{\pgfqpoint{6.806993in}{0.558425in}}%
\pgfpathcurveto{\pgfqpoint{6.806993in}{0.569475in}}{\pgfqpoint{6.802603in}{0.580074in}}{\pgfqpoint{6.794789in}{0.587888in}}%
\pgfpathcurveto{\pgfqpoint{6.786976in}{0.595701in}}{\pgfqpoint{6.776377in}{0.600092in}}{\pgfqpoint{6.765326in}{0.600092in}}%
\pgfpathcurveto{\pgfqpoint{6.754276in}{0.600092in}}{\pgfqpoint{6.743677in}{0.595701in}}{\pgfqpoint{6.735864in}{0.587888in}}%
\pgfpathcurveto{\pgfqpoint{6.728050in}{0.580074in}}{\pgfqpoint{6.723660in}{0.569475in}}{\pgfqpoint{6.723660in}{0.558425in}}%
\pgfpathcurveto{\pgfqpoint{6.723660in}{0.547375in}}{\pgfqpoint{6.728050in}{0.536776in}}{\pgfqpoint{6.735864in}{0.528962in}}%
\pgfpathcurveto{\pgfqpoint{6.743677in}{0.521149in}}{\pgfqpoint{6.754276in}{0.516758in}}{\pgfqpoint{6.765326in}{0.516758in}}%
\pgfusepath{stroke}%
\end{pgfscope}%
\begin{pgfscope}%
\pgfpathrectangle{\pgfqpoint{0.847223in}{0.554012in}}{\pgfqpoint{6.200000in}{4.620000in}}%
\pgfusepath{clip}%
\pgfsetbuttcap%
\pgfsetroundjoin%
\pgfsetlinewidth{1.003750pt}%
\definecolor{currentstroke}{rgb}{1.000000,0.000000,0.000000}%
\pgfsetstrokecolor{currentstroke}%
\pgfsetdash{}{0pt}%
\pgfpathmoveto{\pgfqpoint{6.774136in}{0.516594in}}%
\pgfpathcurveto{\pgfqpoint{6.785186in}{0.516594in}}{\pgfqpoint{6.795785in}{0.520984in}}{\pgfqpoint{6.803599in}{0.528797in}}%
\pgfpathcurveto{\pgfqpoint{6.811412in}{0.536611in}}{\pgfqpoint{6.815802in}{0.547210in}}{\pgfqpoint{6.815802in}{0.558260in}}%
\pgfpathcurveto{\pgfqpoint{6.815802in}{0.569310in}}{\pgfqpoint{6.811412in}{0.579909in}}{\pgfqpoint{6.803599in}{0.587723in}}%
\pgfpathcurveto{\pgfqpoint{6.795785in}{0.595537in}}{\pgfqpoint{6.785186in}{0.599927in}}{\pgfqpoint{6.774136in}{0.599927in}}%
\pgfpathcurveto{\pgfqpoint{6.763086in}{0.599927in}}{\pgfqpoint{6.752487in}{0.595537in}}{\pgfqpoint{6.744673in}{0.587723in}}%
\pgfpathcurveto{\pgfqpoint{6.736859in}{0.579909in}}{\pgfqpoint{6.732469in}{0.569310in}}{\pgfqpoint{6.732469in}{0.558260in}}%
\pgfpathcurveto{\pgfqpoint{6.732469in}{0.547210in}}{\pgfqpoint{6.736859in}{0.536611in}}{\pgfqpoint{6.744673in}{0.528797in}}%
\pgfpathcurveto{\pgfqpoint{6.752487in}{0.520984in}}{\pgfqpoint{6.763086in}{0.516594in}}{\pgfqpoint{6.774136in}{0.516594in}}%
\pgfusepath{stroke}%
\end{pgfscope}%
\begin{pgfscope}%
\pgfpathrectangle{\pgfqpoint{0.847223in}{0.554012in}}{\pgfqpoint{6.200000in}{4.620000in}}%
\pgfusepath{clip}%
\pgfsetbuttcap%
\pgfsetroundjoin%
\pgfsetlinewidth{1.003750pt}%
\definecolor{currentstroke}{rgb}{1.000000,0.000000,0.000000}%
\pgfsetstrokecolor{currentstroke}%
\pgfsetdash{}{0pt}%
\pgfpathmoveto{\pgfqpoint{6.782945in}{0.516431in}}%
\pgfpathcurveto{\pgfqpoint{6.793995in}{0.516431in}}{\pgfqpoint{6.804594in}{0.520821in}}{\pgfqpoint{6.812408in}{0.528635in}}%
\pgfpathcurveto{\pgfqpoint{6.820221in}{0.536448in}}{\pgfqpoint{6.824612in}{0.547047in}}{\pgfqpoint{6.824612in}{0.558097in}}%
\pgfpathcurveto{\pgfqpoint{6.824612in}{0.569148in}}{\pgfqpoint{6.820221in}{0.579747in}}{\pgfqpoint{6.812408in}{0.587560in}}%
\pgfpathcurveto{\pgfqpoint{6.804594in}{0.595374in}}{\pgfqpoint{6.793995in}{0.599764in}}{\pgfqpoint{6.782945in}{0.599764in}}%
\pgfpathcurveto{\pgfqpoint{6.771895in}{0.599764in}}{\pgfqpoint{6.761296in}{0.595374in}}{\pgfqpoint{6.753482in}{0.587560in}}%
\pgfpathcurveto{\pgfqpoint{6.745669in}{0.579747in}}{\pgfqpoint{6.741278in}{0.569148in}}{\pgfqpoint{6.741278in}{0.558097in}}%
\pgfpathcurveto{\pgfqpoint{6.741278in}{0.547047in}}{\pgfqpoint{6.745669in}{0.536448in}}{\pgfqpoint{6.753482in}{0.528635in}}%
\pgfpathcurveto{\pgfqpoint{6.761296in}{0.520821in}}{\pgfqpoint{6.771895in}{0.516431in}}{\pgfqpoint{6.782945in}{0.516431in}}%
\pgfusepath{stroke}%
\end{pgfscope}%
\begin{pgfscope}%
\pgfpathrectangle{\pgfqpoint{0.847223in}{0.554012in}}{\pgfqpoint{6.200000in}{4.620000in}}%
\pgfusepath{clip}%
\pgfsetbuttcap%
\pgfsetroundjoin%
\pgfsetlinewidth{1.003750pt}%
\definecolor{currentstroke}{rgb}{1.000000,0.000000,0.000000}%
\pgfsetstrokecolor{currentstroke}%
\pgfsetdash{}{0pt}%
\pgfpathmoveto{\pgfqpoint{6.791754in}{0.516270in}}%
\pgfpathcurveto{\pgfqpoint{6.802804in}{0.516270in}}{\pgfqpoint{6.813403in}{0.520660in}}{\pgfqpoint{6.821217in}{0.528474in}}%
\pgfpathcurveto{\pgfqpoint{6.829031in}{0.536288in}}{\pgfqpoint{6.833421in}{0.546887in}}{\pgfqpoint{6.833421in}{0.557937in}}%
\pgfpathcurveto{\pgfqpoint{6.833421in}{0.568987in}}{\pgfqpoint{6.829031in}{0.579586in}}{\pgfqpoint{6.821217in}{0.587400in}}%
\pgfpathcurveto{\pgfqpoint{6.813403in}{0.595213in}}{\pgfqpoint{6.802804in}{0.599603in}}{\pgfqpoint{6.791754in}{0.599603in}}%
\pgfpathcurveto{\pgfqpoint{6.780704in}{0.599603in}}{\pgfqpoint{6.770105in}{0.595213in}}{\pgfqpoint{6.762291in}{0.587400in}}%
\pgfpathcurveto{\pgfqpoint{6.754478in}{0.579586in}}{\pgfqpoint{6.750088in}{0.568987in}}{\pgfqpoint{6.750088in}{0.557937in}}%
\pgfpathcurveto{\pgfqpoint{6.750088in}{0.546887in}}{\pgfqpoint{6.754478in}{0.536288in}}{\pgfqpoint{6.762291in}{0.528474in}}%
\pgfpathcurveto{\pgfqpoint{6.770105in}{0.520660in}}{\pgfqpoint{6.780704in}{0.516270in}}{\pgfqpoint{6.791754in}{0.516270in}}%
\pgfusepath{stroke}%
\end{pgfscope}%
\begin{pgfscope}%
\pgfpathrectangle{\pgfqpoint{0.847223in}{0.554012in}}{\pgfqpoint{6.200000in}{4.620000in}}%
\pgfusepath{clip}%
\pgfsetbuttcap%
\pgfsetroundjoin%
\pgfsetlinewidth{1.003750pt}%
\definecolor{currentstroke}{rgb}{1.000000,0.000000,0.000000}%
\pgfsetstrokecolor{currentstroke}%
\pgfsetdash{}{0pt}%
\pgfpathmoveto{\pgfqpoint{6.800564in}{0.516111in}}%
\pgfpathcurveto{\pgfqpoint{6.811614in}{0.516111in}}{\pgfqpoint{6.822213in}{0.520502in}}{\pgfqpoint{6.830026in}{0.528315in}}%
\pgfpathcurveto{\pgfqpoint{6.837840in}{0.536129in}}{\pgfqpoint{6.842230in}{0.546728in}}{\pgfqpoint{6.842230in}{0.557778in}}%
\pgfpathcurveto{\pgfqpoint{6.842230in}{0.568828in}}{\pgfqpoint{6.837840in}{0.579427in}}{\pgfqpoint{6.830026in}{0.587241in}}%
\pgfpathcurveto{\pgfqpoint{6.822213in}{0.595055in}}{\pgfqpoint{6.811614in}{0.599445in}}{\pgfqpoint{6.800564in}{0.599445in}}%
\pgfpathcurveto{\pgfqpoint{6.789513in}{0.599445in}}{\pgfqpoint{6.778914in}{0.595055in}}{\pgfqpoint{6.771101in}{0.587241in}}%
\pgfpathcurveto{\pgfqpoint{6.763287in}{0.579427in}}{\pgfqpoint{6.758897in}{0.568828in}}{\pgfqpoint{6.758897in}{0.557778in}}%
\pgfpathcurveto{\pgfqpoint{6.758897in}{0.546728in}}{\pgfqpoint{6.763287in}{0.536129in}}{\pgfqpoint{6.771101in}{0.528315in}}%
\pgfpathcurveto{\pgfqpoint{6.778914in}{0.520502in}}{\pgfqpoint{6.789513in}{0.516111in}}{\pgfqpoint{6.800564in}{0.516111in}}%
\pgfusepath{stroke}%
\end{pgfscope}%
\begin{pgfscope}%
\pgfpathrectangle{\pgfqpoint{0.847223in}{0.554012in}}{\pgfqpoint{6.200000in}{4.620000in}}%
\pgfusepath{clip}%
\pgfsetbuttcap%
\pgfsetroundjoin%
\pgfsetlinewidth{1.003750pt}%
\definecolor{currentstroke}{rgb}{1.000000,0.000000,0.000000}%
\pgfsetstrokecolor{currentstroke}%
\pgfsetdash{}{0pt}%
\pgfpathmoveto{\pgfqpoint{6.809373in}{0.515955in}}%
\pgfpathcurveto{\pgfqpoint{6.820423in}{0.515955in}}{\pgfqpoint{6.831022in}{0.520345in}}{\pgfqpoint{6.838836in}{0.528159in}}%
\pgfpathcurveto{\pgfqpoint{6.846649in}{0.535972in}}{\pgfqpoint{6.851039in}{0.546571in}}{\pgfqpoint{6.851039in}{0.557621in}}%
\pgfpathcurveto{\pgfqpoint{6.851039in}{0.568672in}}{\pgfqpoint{6.846649in}{0.579271in}}{\pgfqpoint{6.838836in}{0.587084in}}%
\pgfpathcurveto{\pgfqpoint{6.831022in}{0.594898in}}{\pgfqpoint{6.820423in}{0.599288in}}{\pgfqpoint{6.809373in}{0.599288in}}%
\pgfpathcurveto{\pgfqpoint{6.798323in}{0.599288in}}{\pgfqpoint{6.787724in}{0.594898in}}{\pgfqpoint{6.779910in}{0.587084in}}%
\pgfpathcurveto{\pgfqpoint{6.772096in}{0.579271in}}{\pgfqpoint{6.767706in}{0.568672in}}{\pgfqpoint{6.767706in}{0.557621in}}%
\pgfpathcurveto{\pgfqpoint{6.767706in}{0.546571in}}{\pgfqpoint{6.772096in}{0.535972in}}{\pgfqpoint{6.779910in}{0.528159in}}%
\pgfpathcurveto{\pgfqpoint{6.787724in}{0.520345in}}{\pgfqpoint{6.798323in}{0.515955in}}{\pgfqpoint{6.809373in}{0.515955in}}%
\pgfusepath{stroke}%
\end{pgfscope}%
\begin{pgfscope}%
\pgfpathrectangle{\pgfqpoint{0.847223in}{0.554012in}}{\pgfqpoint{6.200000in}{4.620000in}}%
\pgfusepath{clip}%
\pgfsetbuttcap%
\pgfsetroundjoin%
\pgfsetlinewidth{1.003750pt}%
\definecolor{currentstroke}{rgb}{1.000000,0.000000,0.000000}%
\pgfsetstrokecolor{currentstroke}%
\pgfsetdash{}{0pt}%
\pgfpathmoveto{\pgfqpoint{6.818182in}{0.515800in}}%
\pgfpathcurveto{\pgfqpoint{6.829232in}{0.515800in}}{\pgfqpoint{6.839831in}{0.520190in}}{\pgfqpoint{6.847645in}{0.528004in}}%
\pgfpathcurveto{\pgfqpoint{6.855458in}{0.535817in}}{\pgfqpoint{6.859849in}{0.546417in}}{\pgfqpoint{6.859849in}{0.557467in}}%
\pgfpathcurveto{\pgfqpoint{6.859849in}{0.568517in}}{\pgfqpoint{6.855458in}{0.579116in}}{\pgfqpoint{6.847645in}{0.586929in}}%
\pgfpathcurveto{\pgfqpoint{6.839831in}{0.594743in}}{\pgfqpoint{6.829232in}{0.599133in}}{\pgfqpoint{6.818182in}{0.599133in}}%
\pgfpathcurveto{\pgfqpoint{6.807132in}{0.599133in}}{\pgfqpoint{6.796533in}{0.594743in}}{\pgfqpoint{6.788719in}{0.586929in}}%
\pgfpathcurveto{\pgfqpoint{6.780906in}{0.579116in}}{\pgfqpoint{6.776515in}{0.568517in}}{\pgfqpoint{6.776515in}{0.557467in}}%
\pgfpathcurveto{\pgfqpoint{6.776515in}{0.546417in}}{\pgfqpoint{6.780906in}{0.535817in}}{\pgfqpoint{6.788719in}{0.528004in}}%
\pgfpathcurveto{\pgfqpoint{6.796533in}{0.520190in}}{\pgfqpoint{6.807132in}{0.515800in}}{\pgfqpoint{6.818182in}{0.515800in}}%
\pgfusepath{stroke}%
\end{pgfscope}%
\begin{pgfscope}%
\pgfpathrectangle{\pgfqpoint{0.847223in}{0.554012in}}{\pgfqpoint{6.200000in}{4.620000in}}%
\pgfusepath{clip}%
\pgfsetbuttcap%
\pgfsetroundjoin%
\pgfsetlinewidth{1.003750pt}%
\definecolor{currentstroke}{rgb}{1.000000,0.000000,0.000000}%
\pgfsetstrokecolor{currentstroke}%
\pgfsetdash{}{0pt}%
\pgfpathmoveto{\pgfqpoint{6.826991in}{0.515647in}}%
\pgfpathcurveto{\pgfqpoint{6.838041in}{0.515647in}}{\pgfqpoint{6.848641in}{0.520037in}}{\pgfqpoint{6.856454in}{0.527851in}}%
\pgfpathcurveto{\pgfqpoint{6.864268in}{0.535665in}}{\pgfqpoint{6.868658in}{0.546264in}}{\pgfqpoint{6.868658in}{0.557314in}}%
\pgfpathcurveto{\pgfqpoint{6.868658in}{0.568364in}}{\pgfqpoint{6.864268in}{0.578963in}}{\pgfqpoint{6.856454in}{0.586777in}}%
\pgfpathcurveto{\pgfqpoint{6.848641in}{0.594590in}}{\pgfqpoint{6.838041in}{0.598980in}}{\pgfqpoint{6.826991in}{0.598980in}}%
\pgfpathcurveto{\pgfqpoint{6.815941in}{0.598980in}}{\pgfqpoint{6.805342in}{0.594590in}}{\pgfqpoint{6.797529in}{0.586777in}}%
\pgfpathcurveto{\pgfqpoint{6.789715in}{0.578963in}}{\pgfqpoint{6.785325in}{0.568364in}}{\pgfqpoint{6.785325in}{0.557314in}}%
\pgfpathcurveto{\pgfqpoint{6.785325in}{0.546264in}}{\pgfqpoint{6.789715in}{0.535665in}}{\pgfqpoint{6.797529in}{0.527851in}}%
\pgfpathcurveto{\pgfqpoint{6.805342in}{0.520037in}}{\pgfqpoint{6.815941in}{0.515647in}}{\pgfqpoint{6.826991in}{0.515647in}}%
\pgfusepath{stroke}%
\end{pgfscope}%
\begin{pgfscope}%
\pgfpathrectangle{\pgfqpoint{0.847223in}{0.554012in}}{\pgfqpoint{6.200000in}{4.620000in}}%
\pgfusepath{clip}%
\pgfsetbuttcap%
\pgfsetroundjoin%
\pgfsetlinewidth{1.003750pt}%
\definecolor{currentstroke}{rgb}{1.000000,0.000000,0.000000}%
\pgfsetstrokecolor{currentstroke}%
\pgfsetdash{}{0pt}%
\pgfpathmoveto{\pgfqpoint{6.835801in}{0.515496in}}%
\pgfpathcurveto{\pgfqpoint{6.846851in}{0.515496in}}{\pgfqpoint{6.857450in}{0.519886in}}{\pgfqpoint{6.865263in}{0.527700in}}%
\pgfpathcurveto{\pgfqpoint{6.873077in}{0.535514in}}{\pgfqpoint{6.877467in}{0.546113in}}{\pgfqpoint{6.877467in}{0.557163in}}%
\pgfpathcurveto{\pgfqpoint{6.877467in}{0.568213in}}{\pgfqpoint{6.873077in}{0.578812in}}{\pgfqpoint{6.865263in}{0.586626in}}%
\pgfpathcurveto{\pgfqpoint{6.857450in}{0.594439in}}{\pgfqpoint{6.846851in}{0.598829in}}{\pgfqpoint{6.835801in}{0.598829in}}%
\pgfpathcurveto{\pgfqpoint{6.824750in}{0.598829in}}{\pgfqpoint{6.814151in}{0.594439in}}{\pgfqpoint{6.806338in}{0.586626in}}%
\pgfpathcurveto{\pgfqpoint{6.798524in}{0.578812in}}{\pgfqpoint{6.794134in}{0.568213in}}{\pgfqpoint{6.794134in}{0.557163in}}%
\pgfpathcurveto{\pgfqpoint{6.794134in}{0.546113in}}{\pgfqpoint{6.798524in}{0.535514in}}{\pgfqpoint{6.806338in}{0.527700in}}%
\pgfpathcurveto{\pgfqpoint{6.814151in}{0.519886in}}{\pgfqpoint{6.824750in}{0.515496in}}{\pgfqpoint{6.835801in}{0.515496in}}%
\pgfusepath{stroke}%
\end{pgfscope}%
\begin{pgfscope}%
\pgfpathrectangle{\pgfqpoint{0.847223in}{0.554012in}}{\pgfqpoint{6.200000in}{4.620000in}}%
\pgfusepath{clip}%
\pgfsetbuttcap%
\pgfsetroundjoin%
\pgfsetlinewidth{1.003750pt}%
\definecolor{currentstroke}{rgb}{1.000000,0.000000,0.000000}%
\pgfsetstrokecolor{currentstroke}%
\pgfsetdash{}{0pt}%
\pgfpathmoveto{\pgfqpoint{6.844610in}{0.515347in}}%
\pgfpathcurveto{\pgfqpoint{6.855660in}{0.515347in}}{\pgfqpoint{6.866259in}{0.519737in}}{\pgfqpoint{6.874073in}{0.527551in}}%
\pgfpathcurveto{\pgfqpoint{6.881886in}{0.535364in}}{\pgfqpoint{6.886277in}{0.545963in}}{\pgfqpoint{6.886277in}{0.557014in}}%
\pgfpathcurveto{\pgfqpoint{6.886277in}{0.568064in}}{\pgfqpoint{6.881886in}{0.578663in}}{\pgfqpoint{6.874073in}{0.586476in}}%
\pgfpathcurveto{\pgfqpoint{6.866259in}{0.594290in}}{\pgfqpoint{6.855660in}{0.598680in}}{\pgfqpoint{6.844610in}{0.598680in}}%
\pgfpathcurveto{\pgfqpoint{6.833560in}{0.598680in}}{\pgfqpoint{6.822961in}{0.594290in}}{\pgfqpoint{6.815147in}{0.586476in}}%
\pgfpathcurveto{\pgfqpoint{6.807333in}{0.578663in}}{\pgfqpoint{6.802943in}{0.568064in}}{\pgfqpoint{6.802943in}{0.557014in}}%
\pgfpathcurveto{\pgfqpoint{6.802943in}{0.545963in}}{\pgfqpoint{6.807333in}{0.535364in}}{\pgfqpoint{6.815147in}{0.527551in}}%
\pgfpathcurveto{\pgfqpoint{6.822961in}{0.519737in}}{\pgfqpoint{6.833560in}{0.515347in}}{\pgfqpoint{6.844610in}{0.515347in}}%
\pgfusepath{stroke}%
\end{pgfscope}%
\begin{pgfscope}%
\pgfpathrectangle{\pgfqpoint{0.847223in}{0.554012in}}{\pgfqpoint{6.200000in}{4.620000in}}%
\pgfusepath{clip}%
\pgfsetbuttcap%
\pgfsetroundjoin%
\pgfsetlinewidth{1.003750pt}%
\definecolor{currentstroke}{rgb}{1.000000,0.000000,0.000000}%
\pgfsetstrokecolor{currentstroke}%
\pgfsetdash{}{0pt}%
\pgfpathmoveto{\pgfqpoint{6.853419in}{0.515199in}}%
\pgfpathcurveto{\pgfqpoint{6.864469in}{0.515199in}}{\pgfqpoint{6.875068in}{0.519590in}}{\pgfqpoint{6.882882in}{0.527403in}}%
\pgfpathcurveto{\pgfqpoint{6.890696in}{0.535217in}}{\pgfqpoint{6.895086in}{0.545816in}}{\pgfqpoint{6.895086in}{0.556866in}}%
\pgfpathcurveto{\pgfqpoint{6.895086in}{0.567916in}}{\pgfqpoint{6.890696in}{0.578515in}}{\pgfqpoint{6.882882in}{0.586329in}}%
\pgfpathcurveto{\pgfqpoint{6.875068in}{0.594143in}}{\pgfqpoint{6.864469in}{0.598533in}}{\pgfqpoint{6.853419in}{0.598533in}}%
\pgfpathcurveto{\pgfqpoint{6.842369in}{0.598533in}}{\pgfqpoint{6.831770in}{0.594143in}}{\pgfqpoint{6.823956in}{0.586329in}}%
\pgfpathcurveto{\pgfqpoint{6.816143in}{0.578515in}}{\pgfqpoint{6.811752in}{0.567916in}}{\pgfqpoint{6.811752in}{0.556866in}}%
\pgfpathcurveto{\pgfqpoint{6.811752in}{0.545816in}}{\pgfqpoint{6.816143in}{0.535217in}}{\pgfqpoint{6.823956in}{0.527403in}}%
\pgfpathcurveto{\pgfqpoint{6.831770in}{0.519590in}}{\pgfqpoint{6.842369in}{0.515199in}}{\pgfqpoint{6.853419in}{0.515199in}}%
\pgfusepath{stroke}%
\end{pgfscope}%
\begin{pgfscope}%
\pgfpathrectangle{\pgfqpoint{0.847223in}{0.554012in}}{\pgfqpoint{6.200000in}{4.620000in}}%
\pgfusepath{clip}%
\pgfsetbuttcap%
\pgfsetroundjoin%
\pgfsetlinewidth{1.003750pt}%
\definecolor{currentstroke}{rgb}{1.000000,0.000000,0.000000}%
\pgfsetstrokecolor{currentstroke}%
\pgfsetdash{}{0pt}%
\pgfpathmoveto{\pgfqpoint{6.862228in}{0.515054in}}%
\pgfpathcurveto{\pgfqpoint{6.873279in}{0.515054in}}{\pgfqpoint{6.883878in}{0.519444in}}{\pgfqpoint{6.891691in}{0.527258in}}%
\pgfpathcurveto{\pgfqpoint{6.899505in}{0.535071in}}{\pgfqpoint{6.903895in}{0.545670in}}{\pgfqpoint{6.903895in}{0.556720in}}%
\pgfpathcurveto{\pgfqpoint{6.903895in}{0.567771in}}{\pgfqpoint{6.899505in}{0.578370in}}{\pgfqpoint{6.891691in}{0.586183in}}%
\pgfpathcurveto{\pgfqpoint{6.883878in}{0.593997in}}{\pgfqpoint{6.873279in}{0.598387in}}{\pgfqpoint{6.862228in}{0.598387in}}%
\pgfpathcurveto{\pgfqpoint{6.851178in}{0.598387in}}{\pgfqpoint{6.840579in}{0.593997in}}{\pgfqpoint{6.832766in}{0.586183in}}%
\pgfpathcurveto{\pgfqpoint{6.824952in}{0.578370in}}{\pgfqpoint{6.820562in}{0.567771in}}{\pgfqpoint{6.820562in}{0.556720in}}%
\pgfpathcurveto{\pgfqpoint{6.820562in}{0.545670in}}{\pgfqpoint{6.824952in}{0.535071in}}{\pgfqpoint{6.832766in}{0.527258in}}%
\pgfpathcurveto{\pgfqpoint{6.840579in}{0.519444in}}{\pgfqpoint{6.851178in}{0.515054in}}{\pgfqpoint{6.862228in}{0.515054in}}%
\pgfusepath{stroke}%
\end{pgfscope}%
\begin{pgfscope}%
\pgfpathrectangle{\pgfqpoint{0.847223in}{0.554012in}}{\pgfqpoint{6.200000in}{4.620000in}}%
\pgfusepath{clip}%
\pgfsetbuttcap%
\pgfsetroundjoin%
\pgfsetlinewidth{1.003750pt}%
\definecolor{currentstroke}{rgb}{1.000000,0.000000,0.000000}%
\pgfsetstrokecolor{currentstroke}%
\pgfsetdash{}{0pt}%
\pgfpathmoveto{\pgfqpoint{6.871038in}{0.514910in}}%
\pgfpathcurveto{\pgfqpoint{6.882088in}{0.514910in}}{\pgfqpoint{6.892687in}{0.519300in}}{\pgfqpoint{6.900500in}{0.527114in}}%
\pgfpathcurveto{\pgfqpoint{6.908314in}{0.534927in}}{\pgfqpoint{6.912704in}{0.545526in}}{\pgfqpoint{6.912704in}{0.556577in}}%
\pgfpathcurveto{\pgfqpoint{6.912704in}{0.567627in}}{\pgfqpoint{6.908314in}{0.578226in}}{\pgfqpoint{6.900500in}{0.586039in}}%
\pgfpathcurveto{\pgfqpoint{6.892687in}{0.593853in}}{\pgfqpoint{6.882088in}{0.598243in}}{\pgfqpoint{6.871038in}{0.598243in}}%
\pgfpathcurveto{\pgfqpoint{6.859988in}{0.598243in}}{\pgfqpoint{6.849389in}{0.593853in}}{\pgfqpoint{6.841575in}{0.586039in}}%
\pgfpathcurveto{\pgfqpoint{6.833761in}{0.578226in}}{\pgfqpoint{6.829371in}{0.567627in}}{\pgfqpoint{6.829371in}{0.556577in}}%
\pgfpathcurveto{\pgfqpoint{6.829371in}{0.545526in}}{\pgfqpoint{6.833761in}{0.534927in}}{\pgfqpoint{6.841575in}{0.527114in}}%
\pgfpathcurveto{\pgfqpoint{6.849389in}{0.519300in}}{\pgfqpoint{6.859988in}{0.514910in}}{\pgfqpoint{6.871038in}{0.514910in}}%
\pgfusepath{stroke}%
\end{pgfscope}%
\begin{pgfscope}%
\pgfpathrectangle{\pgfqpoint{0.847223in}{0.554012in}}{\pgfqpoint{6.200000in}{4.620000in}}%
\pgfusepath{clip}%
\pgfsetbuttcap%
\pgfsetroundjoin%
\pgfsetlinewidth{1.003750pt}%
\definecolor{currentstroke}{rgb}{1.000000,0.000000,0.000000}%
\pgfsetstrokecolor{currentstroke}%
\pgfsetdash{}{0pt}%
\pgfpathmoveto{\pgfqpoint{6.879847in}{0.514768in}}%
\pgfpathcurveto{\pgfqpoint{6.890897in}{0.514768in}}{\pgfqpoint{6.901496in}{0.519158in}}{\pgfqpoint{6.909310in}{0.526971in}}%
\pgfpathcurveto{\pgfqpoint{6.917123in}{0.534785in}}{\pgfqpoint{6.921514in}{0.545384in}}{\pgfqpoint{6.921514in}{0.556434in}}%
\pgfpathcurveto{\pgfqpoint{6.921514in}{0.567484in}}{\pgfqpoint{6.917123in}{0.578083in}}{\pgfqpoint{6.909310in}{0.585897in}}%
\pgfpathcurveto{\pgfqpoint{6.901496in}{0.593711in}}{\pgfqpoint{6.890897in}{0.598101in}}{\pgfqpoint{6.879847in}{0.598101in}}%
\pgfpathcurveto{\pgfqpoint{6.868797in}{0.598101in}}{\pgfqpoint{6.858198in}{0.593711in}}{\pgfqpoint{6.850384in}{0.585897in}}%
\pgfpathcurveto{\pgfqpoint{6.842571in}{0.578083in}}{\pgfqpoint{6.838180in}{0.567484in}}{\pgfqpoint{6.838180in}{0.556434in}}%
\pgfpathcurveto{\pgfqpoint{6.838180in}{0.545384in}}{\pgfqpoint{6.842571in}{0.534785in}}{\pgfqpoint{6.850384in}{0.526971in}}%
\pgfpathcurveto{\pgfqpoint{6.858198in}{0.519158in}}{\pgfqpoint{6.868797in}{0.514768in}}{\pgfqpoint{6.879847in}{0.514768in}}%
\pgfusepath{stroke}%
\end{pgfscope}%
\begin{pgfscope}%
\pgfpathrectangle{\pgfqpoint{0.847223in}{0.554012in}}{\pgfqpoint{6.200000in}{4.620000in}}%
\pgfusepath{clip}%
\pgfsetbuttcap%
\pgfsetroundjoin%
\pgfsetlinewidth{1.003750pt}%
\definecolor{currentstroke}{rgb}{1.000000,0.000000,0.000000}%
\pgfsetstrokecolor{currentstroke}%
\pgfsetdash{}{0pt}%
\pgfpathmoveto{\pgfqpoint{6.888656in}{0.514627in}}%
\pgfpathcurveto{\pgfqpoint{6.899706in}{0.514627in}}{\pgfqpoint{6.910305in}{0.519017in}}{\pgfqpoint{6.918119in}{0.526831in}}%
\pgfpathcurveto{\pgfqpoint{6.925933in}{0.534644in}}{\pgfqpoint{6.930323in}{0.545243in}}{\pgfqpoint{6.930323in}{0.556294in}}%
\pgfpathcurveto{\pgfqpoint{6.930323in}{0.567344in}}{\pgfqpoint{6.925933in}{0.577943in}}{\pgfqpoint{6.918119in}{0.585756in}}%
\pgfpathcurveto{\pgfqpoint{6.910305in}{0.593570in}}{\pgfqpoint{6.899706in}{0.597960in}}{\pgfqpoint{6.888656in}{0.597960in}}%
\pgfpathcurveto{\pgfqpoint{6.877606in}{0.597960in}}{\pgfqpoint{6.867007in}{0.593570in}}{\pgfqpoint{6.859193in}{0.585756in}}%
\pgfpathcurveto{\pgfqpoint{6.851380in}{0.577943in}}{\pgfqpoint{6.846990in}{0.567344in}}{\pgfqpoint{6.846990in}{0.556294in}}%
\pgfpathcurveto{\pgfqpoint{6.846990in}{0.545243in}}{\pgfqpoint{6.851380in}{0.534644in}}{\pgfqpoint{6.859193in}{0.526831in}}%
\pgfpathcurveto{\pgfqpoint{6.867007in}{0.519017in}}{\pgfqpoint{6.877606in}{0.514627in}}{\pgfqpoint{6.888656in}{0.514627in}}%
\pgfusepath{stroke}%
\end{pgfscope}%
\begin{pgfscope}%
\pgfpathrectangle{\pgfqpoint{0.847223in}{0.554012in}}{\pgfqpoint{6.200000in}{4.620000in}}%
\pgfusepath{clip}%
\pgfsetbuttcap%
\pgfsetroundjoin%
\pgfsetlinewidth{1.003750pt}%
\definecolor{currentstroke}{rgb}{1.000000,0.000000,0.000000}%
\pgfsetstrokecolor{currentstroke}%
\pgfsetdash{}{0pt}%
\pgfpathmoveto{\pgfqpoint{6.897465in}{0.514488in}}%
\pgfpathcurveto{\pgfqpoint{6.908516in}{0.514488in}}{\pgfqpoint{6.919115in}{0.518878in}}{\pgfqpoint{6.926928in}{0.526692in}}%
\pgfpathcurveto{\pgfqpoint{6.934742in}{0.534505in}}{\pgfqpoint{6.939132in}{0.545104in}}{\pgfqpoint{6.939132in}{0.556155in}}%
\pgfpathcurveto{\pgfqpoint{6.939132in}{0.567205in}}{\pgfqpoint{6.934742in}{0.577804in}}{\pgfqpoint{6.926928in}{0.585617in}}%
\pgfpathcurveto{\pgfqpoint{6.919115in}{0.593431in}}{\pgfqpoint{6.908516in}{0.597821in}}{\pgfqpoint{6.897465in}{0.597821in}}%
\pgfpathcurveto{\pgfqpoint{6.886415in}{0.597821in}}{\pgfqpoint{6.875816in}{0.593431in}}{\pgfqpoint{6.868003in}{0.585617in}}%
\pgfpathcurveto{\pgfqpoint{6.860189in}{0.577804in}}{\pgfqpoint{6.855799in}{0.567205in}}{\pgfqpoint{6.855799in}{0.556155in}}%
\pgfpathcurveto{\pgfqpoint{6.855799in}{0.545104in}}{\pgfqpoint{6.860189in}{0.534505in}}{\pgfqpoint{6.868003in}{0.526692in}}%
\pgfpathcurveto{\pgfqpoint{6.875816in}{0.518878in}}{\pgfqpoint{6.886415in}{0.514488in}}{\pgfqpoint{6.897465in}{0.514488in}}%
\pgfusepath{stroke}%
\end{pgfscope}%
\begin{pgfscope}%
\pgfpathrectangle{\pgfqpoint{0.847223in}{0.554012in}}{\pgfqpoint{6.200000in}{4.620000in}}%
\pgfusepath{clip}%
\pgfsetbuttcap%
\pgfsetroundjoin%
\pgfsetlinewidth{1.003750pt}%
\definecolor{currentstroke}{rgb}{1.000000,0.000000,0.000000}%
\pgfsetstrokecolor{currentstroke}%
\pgfsetdash{}{0pt}%
\pgfpathmoveto{\pgfqpoint{6.906275in}{0.514351in}}%
\pgfpathcurveto{\pgfqpoint{6.917325in}{0.514351in}}{\pgfqpoint{6.927924in}{0.518741in}}{\pgfqpoint{6.935738in}{0.526554in}}%
\pgfpathcurveto{\pgfqpoint{6.943551in}{0.534368in}}{\pgfqpoint{6.947941in}{0.544967in}}{\pgfqpoint{6.947941in}{0.556017in}}%
\pgfpathcurveto{\pgfqpoint{6.947941in}{0.567067in}}{\pgfqpoint{6.943551in}{0.577666in}}{\pgfqpoint{6.935738in}{0.585480in}}%
\pgfpathcurveto{\pgfqpoint{6.927924in}{0.593294in}}{\pgfqpoint{6.917325in}{0.597684in}}{\pgfqpoint{6.906275in}{0.597684in}}%
\pgfpathcurveto{\pgfqpoint{6.895225in}{0.597684in}}{\pgfqpoint{6.884626in}{0.593294in}}{\pgfqpoint{6.876812in}{0.585480in}}%
\pgfpathcurveto{\pgfqpoint{6.868998in}{0.577666in}}{\pgfqpoint{6.864608in}{0.567067in}}{\pgfqpoint{6.864608in}{0.556017in}}%
\pgfpathcurveto{\pgfqpoint{6.864608in}{0.544967in}}{\pgfqpoint{6.868998in}{0.534368in}}{\pgfqpoint{6.876812in}{0.526554in}}%
\pgfpathcurveto{\pgfqpoint{6.884626in}{0.518741in}}{\pgfqpoint{6.895225in}{0.514351in}}{\pgfqpoint{6.906275in}{0.514351in}}%
\pgfusepath{stroke}%
\end{pgfscope}%
\begin{pgfscope}%
\pgfpathrectangle{\pgfqpoint{0.847223in}{0.554012in}}{\pgfqpoint{6.200000in}{4.620000in}}%
\pgfusepath{clip}%
\pgfsetbuttcap%
\pgfsetroundjoin%
\pgfsetlinewidth{1.003750pt}%
\definecolor{currentstroke}{rgb}{1.000000,0.000000,0.000000}%
\pgfsetstrokecolor{currentstroke}%
\pgfsetdash{}{0pt}%
\pgfpathmoveto{\pgfqpoint{6.915084in}{0.514215in}}%
\pgfpathcurveto{\pgfqpoint{6.926134in}{0.514215in}}{\pgfqpoint{6.936733in}{0.518605in}}{\pgfqpoint{6.944547in}{0.526419in}}%
\pgfpathcurveto{\pgfqpoint{6.952360in}{0.534232in}}{\pgfqpoint{6.956751in}{0.544831in}}{\pgfqpoint{6.956751in}{0.555881in}}%
\pgfpathcurveto{\pgfqpoint{6.956751in}{0.566931in}}{\pgfqpoint{6.952360in}{0.577530in}}{\pgfqpoint{6.944547in}{0.585344in}}%
\pgfpathcurveto{\pgfqpoint{6.936733in}{0.593158in}}{\pgfqpoint{6.926134in}{0.597548in}}{\pgfqpoint{6.915084in}{0.597548in}}%
\pgfpathcurveto{\pgfqpoint{6.904034in}{0.597548in}}{\pgfqpoint{6.893435in}{0.593158in}}{\pgfqpoint{6.885621in}{0.585344in}}%
\pgfpathcurveto{\pgfqpoint{6.877808in}{0.577530in}}{\pgfqpoint{6.873417in}{0.566931in}}{\pgfqpoint{6.873417in}{0.555881in}}%
\pgfpathcurveto{\pgfqpoint{6.873417in}{0.544831in}}{\pgfqpoint{6.877808in}{0.534232in}}{\pgfqpoint{6.885621in}{0.526419in}}%
\pgfpathcurveto{\pgfqpoint{6.893435in}{0.518605in}}{\pgfqpoint{6.904034in}{0.514215in}}{\pgfqpoint{6.915084in}{0.514215in}}%
\pgfusepath{stroke}%
\end{pgfscope}%
\begin{pgfscope}%
\pgfpathrectangle{\pgfqpoint{0.847223in}{0.554012in}}{\pgfqpoint{6.200000in}{4.620000in}}%
\pgfusepath{clip}%
\pgfsetbuttcap%
\pgfsetroundjoin%
\pgfsetlinewidth{1.003750pt}%
\definecolor{currentstroke}{rgb}{1.000000,0.000000,0.000000}%
\pgfsetstrokecolor{currentstroke}%
\pgfsetdash{}{0pt}%
\pgfpathmoveto{\pgfqpoint{6.923893in}{0.514080in}}%
\pgfpathcurveto{\pgfqpoint{6.934943in}{0.514080in}}{\pgfqpoint{6.945542in}{0.518471in}}{\pgfqpoint{6.953356in}{0.526284in}}%
\pgfpathcurveto{\pgfqpoint{6.961170in}{0.534098in}}{\pgfqpoint{6.965560in}{0.544697in}}{\pgfqpoint{6.965560in}{0.555747in}}%
\pgfpathcurveto{\pgfqpoint{6.965560in}{0.566797in}}{\pgfqpoint{6.961170in}{0.577396in}}{\pgfqpoint{6.953356in}{0.585210in}}%
\pgfpathcurveto{\pgfqpoint{6.945542in}{0.593023in}}{\pgfqpoint{6.934943in}{0.597414in}}{\pgfqpoint{6.923893in}{0.597414in}}%
\pgfpathcurveto{\pgfqpoint{6.912843in}{0.597414in}}{\pgfqpoint{6.902244in}{0.593023in}}{\pgfqpoint{6.894431in}{0.585210in}}%
\pgfpathcurveto{\pgfqpoint{6.886617in}{0.577396in}}{\pgfqpoint{6.882227in}{0.566797in}}{\pgfqpoint{6.882227in}{0.555747in}}%
\pgfpathcurveto{\pgfqpoint{6.882227in}{0.544697in}}{\pgfqpoint{6.886617in}{0.534098in}}{\pgfqpoint{6.894431in}{0.526284in}}%
\pgfpathcurveto{\pgfqpoint{6.902244in}{0.518471in}}{\pgfqpoint{6.912843in}{0.514080in}}{\pgfqpoint{6.923893in}{0.514080in}}%
\pgfusepath{stroke}%
\end{pgfscope}%
\begin{pgfscope}%
\pgfpathrectangle{\pgfqpoint{0.847223in}{0.554012in}}{\pgfqpoint{6.200000in}{4.620000in}}%
\pgfusepath{clip}%
\pgfsetbuttcap%
\pgfsetroundjoin%
\pgfsetlinewidth{1.003750pt}%
\definecolor{currentstroke}{rgb}{1.000000,0.000000,0.000000}%
\pgfsetstrokecolor{currentstroke}%
\pgfsetdash{}{0pt}%
\pgfpathmoveto{\pgfqpoint{6.932703in}{0.513947in}}%
\pgfpathcurveto{\pgfqpoint{6.943753in}{0.513947in}}{\pgfqpoint{6.954352in}{0.518338in}}{\pgfqpoint{6.962165in}{0.526151in}}%
\pgfpathcurveto{\pgfqpoint{6.969979in}{0.533965in}}{\pgfqpoint{6.974369in}{0.544564in}}{\pgfqpoint{6.974369in}{0.555614in}}%
\pgfpathcurveto{\pgfqpoint{6.974369in}{0.566664in}}{\pgfqpoint{6.969979in}{0.577263in}}{\pgfqpoint{6.962165in}{0.585077in}}%
\pgfpathcurveto{\pgfqpoint{6.954352in}{0.592890in}}{\pgfqpoint{6.943753in}{0.597281in}}{\pgfqpoint{6.932703in}{0.597281in}}%
\pgfpathcurveto{\pgfqpoint{6.921652in}{0.597281in}}{\pgfqpoint{6.911053in}{0.592890in}}{\pgfqpoint{6.903240in}{0.585077in}}%
\pgfpathcurveto{\pgfqpoint{6.895426in}{0.577263in}}{\pgfqpoint{6.891036in}{0.566664in}}{\pgfqpoint{6.891036in}{0.555614in}}%
\pgfpathcurveto{\pgfqpoint{6.891036in}{0.544564in}}{\pgfqpoint{6.895426in}{0.533965in}}{\pgfqpoint{6.903240in}{0.526151in}}%
\pgfpathcurveto{\pgfqpoint{6.911053in}{0.518338in}}{\pgfqpoint{6.921652in}{0.513947in}}{\pgfqpoint{6.932703in}{0.513947in}}%
\pgfusepath{stroke}%
\end{pgfscope}%
\begin{pgfscope}%
\pgfpathrectangle{\pgfqpoint{0.847223in}{0.554012in}}{\pgfqpoint{6.200000in}{4.620000in}}%
\pgfusepath{clip}%
\pgfsetbuttcap%
\pgfsetroundjoin%
\pgfsetlinewidth{1.003750pt}%
\definecolor{currentstroke}{rgb}{1.000000,0.000000,0.000000}%
\pgfsetstrokecolor{currentstroke}%
\pgfsetdash{}{0pt}%
\pgfpathmoveto{\pgfqpoint{6.941512in}{0.513816in}}%
\pgfpathcurveto{\pgfqpoint{6.952562in}{0.513816in}}{\pgfqpoint{6.963161in}{0.518206in}}{\pgfqpoint{6.970975in}{0.526020in}}%
\pgfpathcurveto{\pgfqpoint{6.978788in}{0.533834in}}{\pgfqpoint{6.983179in}{0.544433in}}{\pgfqpoint{6.983179in}{0.555483in}}%
\pgfpathcurveto{\pgfqpoint{6.983179in}{0.566533in}}{\pgfqpoint{6.978788in}{0.577132in}}{\pgfqpoint{6.970975in}{0.584945in}}%
\pgfpathcurveto{\pgfqpoint{6.963161in}{0.592759in}}{\pgfqpoint{6.952562in}{0.597149in}}{\pgfqpoint{6.941512in}{0.597149in}}%
\pgfpathcurveto{\pgfqpoint{6.930462in}{0.597149in}}{\pgfqpoint{6.919863in}{0.592759in}}{\pgfqpoint{6.912049in}{0.584945in}}%
\pgfpathcurveto{\pgfqpoint{6.904235in}{0.577132in}}{\pgfqpoint{6.899845in}{0.566533in}}{\pgfqpoint{6.899845in}{0.555483in}}%
\pgfpathcurveto{\pgfqpoint{6.899845in}{0.544433in}}{\pgfqpoint{6.904235in}{0.533834in}}{\pgfqpoint{6.912049in}{0.526020in}}%
\pgfpathcurveto{\pgfqpoint{6.919863in}{0.518206in}}{\pgfqpoint{6.930462in}{0.513816in}}{\pgfqpoint{6.941512in}{0.513816in}}%
\pgfusepath{stroke}%
\end{pgfscope}%
\begin{pgfscope}%
\pgfpathrectangle{\pgfqpoint{0.847223in}{0.554012in}}{\pgfqpoint{6.200000in}{4.620000in}}%
\pgfusepath{clip}%
\pgfsetbuttcap%
\pgfsetroundjoin%
\pgfsetlinewidth{1.003750pt}%
\definecolor{currentstroke}{rgb}{1.000000,0.000000,0.000000}%
\pgfsetstrokecolor{currentstroke}%
\pgfsetdash{}{0pt}%
\pgfpathmoveto{\pgfqpoint{6.950321in}{0.513686in}}%
\pgfpathcurveto{\pgfqpoint{6.961371in}{0.513686in}}{\pgfqpoint{6.971970in}{0.518076in}}{\pgfqpoint{6.979784in}{0.525890in}}%
\pgfpathcurveto{\pgfqpoint{6.987598in}{0.533704in}}{\pgfqpoint{6.991988in}{0.544303in}}{\pgfqpoint{6.991988in}{0.555353in}}%
\pgfpathcurveto{\pgfqpoint{6.991988in}{0.566403in}}{\pgfqpoint{6.987598in}{0.577002in}}{\pgfqpoint{6.979784in}{0.584816in}}%
\pgfpathcurveto{\pgfqpoint{6.971970in}{0.592629in}}{\pgfqpoint{6.961371in}{0.597019in}}{\pgfqpoint{6.950321in}{0.597019in}}%
\pgfpathcurveto{\pgfqpoint{6.939271in}{0.597019in}}{\pgfqpoint{6.928672in}{0.592629in}}{\pgfqpoint{6.920858in}{0.584816in}}%
\pgfpathcurveto{\pgfqpoint{6.913045in}{0.577002in}}{\pgfqpoint{6.908654in}{0.566403in}}{\pgfqpoint{6.908654in}{0.555353in}}%
\pgfpathcurveto{\pgfqpoint{6.908654in}{0.544303in}}{\pgfqpoint{6.913045in}{0.533704in}}{\pgfqpoint{6.920858in}{0.525890in}}%
\pgfpathcurveto{\pgfqpoint{6.928672in}{0.518076in}}{\pgfqpoint{6.939271in}{0.513686in}}{\pgfqpoint{6.950321in}{0.513686in}}%
\pgfusepath{stroke}%
\end{pgfscope}%
\begin{pgfscope}%
\pgfpathrectangle{\pgfqpoint{0.847223in}{0.554012in}}{\pgfqpoint{6.200000in}{4.620000in}}%
\pgfusepath{clip}%
\pgfsetbuttcap%
\pgfsetroundjoin%
\pgfsetlinewidth{1.003750pt}%
\definecolor{currentstroke}{rgb}{1.000000,0.000000,0.000000}%
\pgfsetstrokecolor{currentstroke}%
\pgfsetdash{}{0pt}%
\pgfpathmoveto{\pgfqpoint{6.959130in}{0.513558in}}%
\pgfpathcurveto{\pgfqpoint{6.970181in}{0.513558in}}{\pgfqpoint{6.980780in}{0.517948in}}{\pgfqpoint{6.988593in}{0.525761in}}%
\pgfpathcurveto{\pgfqpoint{6.996407in}{0.533575in}}{\pgfqpoint{7.000797in}{0.544174in}}{\pgfqpoint{7.000797in}{0.555224in}}%
\pgfpathcurveto{\pgfqpoint{7.000797in}{0.566274in}}{\pgfqpoint{6.996407in}{0.576873in}}{\pgfqpoint{6.988593in}{0.584687in}}%
\pgfpathcurveto{\pgfqpoint{6.980780in}{0.592501in}}{\pgfqpoint{6.970181in}{0.596891in}}{\pgfqpoint{6.959130in}{0.596891in}}%
\pgfpathcurveto{\pgfqpoint{6.948080in}{0.596891in}}{\pgfqpoint{6.937481in}{0.592501in}}{\pgfqpoint{6.929668in}{0.584687in}}%
\pgfpathcurveto{\pgfqpoint{6.921854in}{0.576873in}}{\pgfqpoint{6.917464in}{0.566274in}}{\pgfqpoint{6.917464in}{0.555224in}}%
\pgfpathcurveto{\pgfqpoint{6.917464in}{0.544174in}}{\pgfqpoint{6.921854in}{0.533575in}}{\pgfqpoint{6.929668in}{0.525761in}}%
\pgfpathcurveto{\pgfqpoint{6.937481in}{0.517948in}}{\pgfqpoint{6.948080in}{0.513558in}}{\pgfqpoint{6.959130in}{0.513558in}}%
\pgfusepath{stroke}%
\end{pgfscope}%
\begin{pgfscope}%
\pgfpathrectangle{\pgfqpoint{0.847223in}{0.554012in}}{\pgfqpoint{6.200000in}{4.620000in}}%
\pgfusepath{clip}%
\pgfsetbuttcap%
\pgfsetroundjoin%
\pgfsetlinewidth{1.003750pt}%
\definecolor{currentstroke}{rgb}{1.000000,0.000000,0.000000}%
\pgfsetstrokecolor{currentstroke}%
\pgfsetdash{}{0pt}%
\pgfpathmoveto{\pgfqpoint{6.967940in}{0.513430in}}%
\pgfpathcurveto{\pgfqpoint{6.978990in}{0.513430in}}{\pgfqpoint{6.989589in}{0.517821in}}{\pgfqpoint{6.997402in}{0.525634in}}%
\pgfpathcurveto{\pgfqpoint{7.005216in}{0.533448in}}{\pgfqpoint{7.009606in}{0.544047in}}{\pgfqpoint{7.009606in}{0.555097in}}%
\pgfpathcurveto{\pgfqpoint{7.009606in}{0.566147in}}{\pgfqpoint{7.005216in}{0.576746in}}{\pgfqpoint{6.997402in}{0.584560in}}%
\pgfpathcurveto{\pgfqpoint{6.989589in}{0.592374in}}{\pgfqpoint{6.978990in}{0.596764in}}{\pgfqpoint{6.967940in}{0.596764in}}%
\pgfpathcurveto{\pgfqpoint{6.956890in}{0.596764in}}{\pgfqpoint{6.946290in}{0.592374in}}{\pgfqpoint{6.938477in}{0.584560in}}%
\pgfpathcurveto{\pgfqpoint{6.930663in}{0.576746in}}{\pgfqpoint{6.926273in}{0.566147in}}{\pgfqpoint{6.926273in}{0.555097in}}%
\pgfpathcurveto{\pgfqpoint{6.926273in}{0.544047in}}{\pgfqpoint{6.930663in}{0.533448in}}{\pgfqpoint{6.938477in}{0.525634in}}%
\pgfpathcurveto{\pgfqpoint{6.946290in}{0.517821in}}{\pgfqpoint{6.956890in}{0.513430in}}{\pgfqpoint{6.967940in}{0.513430in}}%
\pgfusepath{stroke}%
\end{pgfscope}%
\begin{pgfscope}%
\pgfpathrectangle{\pgfqpoint{0.847223in}{0.554012in}}{\pgfqpoint{6.200000in}{4.620000in}}%
\pgfusepath{clip}%
\pgfsetbuttcap%
\pgfsetroundjoin%
\pgfsetlinewidth{1.003750pt}%
\definecolor{currentstroke}{rgb}{1.000000,0.000000,0.000000}%
\pgfsetstrokecolor{currentstroke}%
\pgfsetdash{}{0pt}%
\pgfpathmoveto{\pgfqpoint{6.976749in}{0.513305in}}%
\pgfpathcurveto{\pgfqpoint{6.987799in}{0.513305in}}{\pgfqpoint{6.998398in}{0.517695in}}{\pgfqpoint{7.006212in}{0.525509in}}%
\pgfpathcurveto{\pgfqpoint{7.014025in}{0.533322in}}{\pgfqpoint{7.018416in}{0.543921in}}{\pgfqpoint{7.018416in}{0.554971in}}%
\pgfpathcurveto{\pgfqpoint{7.018416in}{0.566021in}}{\pgfqpoint{7.014025in}{0.576621in}}{\pgfqpoint{7.006212in}{0.584434in}}%
\pgfpathcurveto{\pgfqpoint{6.998398in}{0.592248in}}{\pgfqpoint{6.987799in}{0.596638in}}{\pgfqpoint{6.976749in}{0.596638in}}%
\pgfpathcurveto{\pgfqpoint{6.965699in}{0.596638in}}{\pgfqpoint{6.955100in}{0.592248in}}{\pgfqpoint{6.947286in}{0.584434in}}%
\pgfpathcurveto{\pgfqpoint{6.939473in}{0.576621in}}{\pgfqpoint{6.935082in}{0.566021in}}{\pgfqpoint{6.935082in}{0.554971in}}%
\pgfpathcurveto{\pgfqpoint{6.935082in}{0.543921in}}{\pgfqpoint{6.939473in}{0.533322in}}{\pgfqpoint{6.947286in}{0.525509in}}%
\pgfpathcurveto{\pgfqpoint{6.955100in}{0.517695in}}{\pgfqpoint{6.965699in}{0.513305in}}{\pgfqpoint{6.976749in}{0.513305in}}%
\pgfusepath{stroke}%
\end{pgfscope}%
\begin{pgfscope}%
\pgfpathrectangle{\pgfqpoint{0.847223in}{0.554012in}}{\pgfqpoint{6.200000in}{4.620000in}}%
\pgfusepath{clip}%
\pgfsetbuttcap%
\pgfsetroundjoin%
\pgfsetlinewidth{1.003750pt}%
\definecolor{currentstroke}{rgb}{1.000000,0.000000,0.000000}%
\pgfsetstrokecolor{currentstroke}%
\pgfsetdash{}{0pt}%
\pgfpathmoveto{\pgfqpoint{6.985558in}{0.513180in}}%
\pgfpathcurveto{\pgfqpoint{6.996608in}{0.513180in}}{\pgfqpoint{7.007207in}{0.517571in}}{\pgfqpoint{7.015021in}{0.525384in}}%
\pgfpathcurveto{\pgfqpoint{7.022835in}{0.533198in}}{\pgfqpoint{7.027225in}{0.543797in}}{\pgfqpoint{7.027225in}{0.554847in}}%
\pgfpathcurveto{\pgfqpoint{7.027225in}{0.565897in}}{\pgfqpoint{7.022835in}{0.576496in}}{\pgfqpoint{7.015021in}{0.584310in}}%
\pgfpathcurveto{\pgfqpoint{7.007207in}{0.592123in}}{\pgfqpoint{6.996608in}{0.596514in}}{\pgfqpoint{6.985558in}{0.596514in}}%
\pgfpathcurveto{\pgfqpoint{6.974508in}{0.596514in}}{\pgfqpoint{6.963909in}{0.592123in}}{\pgfqpoint{6.956095in}{0.584310in}}%
\pgfpathcurveto{\pgfqpoint{6.948282in}{0.576496in}}{\pgfqpoint{6.943892in}{0.565897in}}{\pgfqpoint{6.943892in}{0.554847in}}%
\pgfpathcurveto{\pgfqpoint{6.943892in}{0.543797in}}{\pgfqpoint{6.948282in}{0.533198in}}{\pgfqpoint{6.956095in}{0.525384in}}%
\pgfpathcurveto{\pgfqpoint{6.963909in}{0.517571in}}{\pgfqpoint{6.974508in}{0.513180in}}{\pgfqpoint{6.985558in}{0.513180in}}%
\pgfusepath{stroke}%
\end{pgfscope}%
\begin{pgfscope}%
\pgfpathrectangle{\pgfqpoint{0.847223in}{0.554012in}}{\pgfqpoint{6.200000in}{4.620000in}}%
\pgfusepath{clip}%
\pgfsetbuttcap%
\pgfsetroundjoin%
\pgfsetlinewidth{1.003750pt}%
\definecolor{currentstroke}{rgb}{1.000000,0.000000,0.000000}%
\pgfsetstrokecolor{currentstroke}%
\pgfsetdash{}{0pt}%
\pgfpathmoveto{\pgfqpoint{6.994367in}{0.513057in}}%
\pgfpathcurveto{\pgfqpoint{7.005418in}{0.513057in}}{\pgfqpoint{7.016017in}{0.517447in}}{\pgfqpoint{7.023830in}{0.525261in}}%
\pgfpathcurveto{\pgfqpoint{7.031644in}{0.533075in}}{\pgfqpoint{7.036034in}{0.543674in}}{\pgfqpoint{7.036034in}{0.554724in}}%
\pgfpathcurveto{\pgfqpoint{7.036034in}{0.565774in}}{\pgfqpoint{7.031644in}{0.576373in}}{\pgfqpoint{7.023830in}{0.584187in}}%
\pgfpathcurveto{\pgfqpoint{7.016017in}{0.592000in}}{\pgfqpoint{7.005418in}{0.596391in}}{\pgfqpoint{6.994367in}{0.596391in}}%
\pgfpathcurveto{\pgfqpoint{6.983317in}{0.596391in}}{\pgfqpoint{6.972718in}{0.592000in}}{\pgfqpoint{6.964905in}{0.584187in}}%
\pgfpathcurveto{\pgfqpoint{6.957091in}{0.576373in}}{\pgfqpoint{6.952701in}{0.565774in}}{\pgfqpoint{6.952701in}{0.554724in}}%
\pgfpathcurveto{\pgfqpoint{6.952701in}{0.543674in}}{\pgfqpoint{6.957091in}{0.533075in}}{\pgfqpoint{6.964905in}{0.525261in}}%
\pgfpathcurveto{\pgfqpoint{6.972718in}{0.517447in}}{\pgfqpoint{6.983317in}{0.513057in}}{\pgfqpoint{6.994367in}{0.513057in}}%
\pgfusepath{stroke}%
\end{pgfscope}%
\begin{pgfscope}%
\pgfpathrectangle{\pgfqpoint{0.847223in}{0.554012in}}{\pgfqpoint{6.200000in}{4.620000in}}%
\pgfusepath{clip}%
\pgfsetbuttcap%
\pgfsetroundjoin%
\pgfsetlinewidth{1.003750pt}%
\definecolor{currentstroke}{rgb}{1.000000,0.000000,0.000000}%
\pgfsetstrokecolor{currentstroke}%
\pgfsetdash{}{0pt}%
\pgfpathmoveto{\pgfqpoint{7.003177in}{0.512935in}}%
\pgfpathcurveto{\pgfqpoint{7.014227in}{0.512935in}}{\pgfqpoint{7.024826in}{0.517326in}}{\pgfqpoint{7.032639in}{0.525139in}}%
\pgfpathcurveto{\pgfqpoint{7.040453in}{0.532953in}}{\pgfqpoint{7.044843in}{0.543552in}}{\pgfqpoint{7.044843in}{0.554602in}}%
\pgfpathcurveto{\pgfqpoint{7.044843in}{0.565652in}}{\pgfqpoint{7.040453in}{0.576251in}}{\pgfqpoint{7.032639in}{0.584065in}}%
\pgfpathcurveto{\pgfqpoint{7.024826in}{0.591878in}}{\pgfqpoint{7.014227in}{0.596269in}}{\pgfqpoint{7.003177in}{0.596269in}}%
\pgfpathcurveto{\pgfqpoint{6.992127in}{0.596269in}}{\pgfqpoint{6.981528in}{0.591878in}}{\pgfqpoint{6.973714in}{0.584065in}}%
\pgfpathcurveto{\pgfqpoint{6.965900in}{0.576251in}}{\pgfqpoint{6.961510in}{0.565652in}}{\pgfqpoint{6.961510in}{0.554602in}}%
\pgfpathcurveto{\pgfqpoint{6.961510in}{0.543552in}}{\pgfqpoint{6.965900in}{0.532953in}}{\pgfqpoint{6.973714in}{0.525139in}}%
\pgfpathcurveto{\pgfqpoint{6.981528in}{0.517326in}}{\pgfqpoint{6.992127in}{0.512935in}}{\pgfqpoint{7.003177in}{0.512935in}}%
\pgfusepath{stroke}%
\end{pgfscope}%
\begin{pgfscope}%
\pgfpathrectangle{\pgfqpoint{0.847223in}{0.554012in}}{\pgfqpoint{6.200000in}{4.620000in}}%
\pgfusepath{clip}%
\pgfsetbuttcap%
\pgfsetroundjoin%
\pgfsetlinewidth{1.003750pt}%
\definecolor{currentstroke}{rgb}{1.000000,0.000000,0.000000}%
\pgfsetstrokecolor{currentstroke}%
\pgfsetdash{}{0pt}%
\pgfpathmoveto{\pgfqpoint{7.011986in}{0.512815in}}%
\pgfpathcurveto{\pgfqpoint{7.023036in}{0.512815in}}{\pgfqpoint{7.033635in}{0.517205in}}{\pgfqpoint{7.041449in}{0.525019in}}%
\pgfpathcurveto{\pgfqpoint{7.049262in}{0.532832in}}{\pgfqpoint{7.053653in}{0.543431in}}{\pgfqpoint{7.053653in}{0.554482in}}%
\pgfpathcurveto{\pgfqpoint{7.053653in}{0.565532in}}{\pgfqpoint{7.049262in}{0.576131in}}{\pgfqpoint{7.041449in}{0.583944in}}%
\pgfpathcurveto{\pgfqpoint{7.033635in}{0.591758in}}{\pgfqpoint{7.023036in}{0.596148in}}{\pgfqpoint{7.011986in}{0.596148in}}%
\pgfpathcurveto{\pgfqpoint{7.000936in}{0.596148in}}{\pgfqpoint{6.990337in}{0.591758in}}{\pgfqpoint{6.982523in}{0.583944in}}%
\pgfpathcurveto{\pgfqpoint{6.974710in}{0.576131in}}{\pgfqpoint{6.970319in}{0.565532in}}{\pgfqpoint{6.970319in}{0.554482in}}%
\pgfpathcurveto{\pgfqpoint{6.970319in}{0.543431in}}{\pgfqpoint{6.974710in}{0.532832in}}{\pgfqpoint{6.982523in}{0.525019in}}%
\pgfpathcurveto{\pgfqpoint{6.990337in}{0.517205in}}{\pgfqpoint{7.000936in}{0.512815in}}{\pgfqpoint{7.011986in}{0.512815in}}%
\pgfusepath{stroke}%
\end{pgfscope}%
\begin{pgfscope}%
\pgfpathrectangle{\pgfqpoint{0.847223in}{0.554012in}}{\pgfqpoint{6.200000in}{4.620000in}}%
\pgfusepath{clip}%
\pgfsetbuttcap%
\pgfsetroundjoin%
\pgfsetlinewidth{1.003750pt}%
\definecolor{currentstroke}{rgb}{1.000000,0.000000,0.000000}%
\pgfsetstrokecolor{currentstroke}%
\pgfsetdash{}{0pt}%
\pgfpathmoveto{\pgfqpoint{7.020795in}{0.512696in}}%
\pgfpathcurveto{\pgfqpoint{7.031845in}{0.512696in}}{\pgfqpoint{7.042444in}{0.517086in}}{\pgfqpoint{7.050258in}{0.524899in}}%
\pgfpathcurveto{\pgfqpoint{7.058072in}{0.532713in}}{\pgfqpoint{7.062462in}{0.543312in}}{\pgfqpoint{7.062462in}{0.554362in}}%
\pgfpathcurveto{\pgfqpoint{7.062462in}{0.565412in}}{\pgfqpoint{7.058072in}{0.576011in}}{\pgfqpoint{7.050258in}{0.583825in}}%
\pgfpathcurveto{\pgfqpoint{7.042444in}{0.591639in}}{\pgfqpoint{7.031845in}{0.596029in}}{\pgfqpoint{7.020795in}{0.596029in}}%
\pgfpathcurveto{\pgfqpoint{7.009745in}{0.596029in}}{\pgfqpoint{6.999146in}{0.591639in}}{\pgfqpoint{6.991332in}{0.583825in}}%
\pgfpathcurveto{\pgfqpoint{6.983519in}{0.576011in}}{\pgfqpoint{6.979129in}{0.565412in}}{\pgfqpoint{6.979129in}{0.554362in}}%
\pgfpathcurveto{\pgfqpoint{6.979129in}{0.543312in}}{\pgfqpoint{6.983519in}{0.532713in}}{\pgfqpoint{6.991332in}{0.524899in}}%
\pgfpathcurveto{\pgfqpoint{6.999146in}{0.517086in}}{\pgfqpoint{7.009745in}{0.512696in}}{\pgfqpoint{7.020795in}{0.512696in}}%
\pgfusepath{stroke}%
\end{pgfscope}%
\begin{pgfscope}%
\pgfpathrectangle{\pgfqpoint{0.847223in}{0.554012in}}{\pgfqpoint{6.200000in}{4.620000in}}%
\pgfusepath{clip}%
\pgfsetbuttcap%
\pgfsetroundjoin%
\pgfsetlinewidth{1.003750pt}%
\definecolor{currentstroke}{rgb}{1.000000,0.000000,0.000000}%
\pgfsetstrokecolor{currentstroke}%
\pgfsetdash{}{0pt}%
\pgfpathmoveto{\pgfqpoint{7.029605in}{0.512578in}}%
\pgfpathcurveto{\pgfqpoint{7.040655in}{0.512578in}}{\pgfqpoint{7.051254in}{0.516968in}}{\pgfqpoint{7.059067in}{0.524781in}}%
\pgfpathcurveto{\pgfqpoint{7.066881in}{0.532595in}}{\pgfqpoint{7.071271in}{0.543194in}}{\pgfqpoint{7.071271in}{0.554244in}}%
\pgfpathcurveto{\pgfqpoint{7.071271in}{0.565294in}}{\pgfqpoint{7.066881in}{0.575893in}}{\pgfqpoint{7.059067in}{0.583707in}}%
\pgfpathcurveto{\pgfqpoint{7.051254in}{0.591521in}}{\pgfqpoint{7.040655in}{0.595911in}}{\pgfqpoint{7.029605in}{0.595911in}}%
\pgfpathcurveto{\pgfqpoint{7.018554in}{0.595911in}}{\pgfqpoint{7.007955in}{0.591521in}}{\pgfqpoint{7.000142in}{0.583707in}}%
\pgfpathcurveto{\pgfqpoint{6.992328in}{0.575893in}}{\pgfqpoint{6.987938in}{0.565294in}}{\pgfqpoint{6.987938in}{0.554244in}}%
\pgfpathcurveto{\pgfqpoint{6.987938in}{0.543194in}}{\pgfqpoint{6.992328in}{0.532595in}}{\pgfqpoint{7.000142in}{0.524781in}}%
\pgfpathcurveto{\pgfqpoint{7.007955in}{0.516968in}}{\pgfqpoint{7.018554in}{0.512578in}}{\pgfqpoint{7.029605in}{0.512578in}}%
\pgfusepath{stroke}%
\end{pgfscope}%
\begin{pgfscope}%
\pgfpathrectangle{\pgfqpoint{0.847223in}{0.554012in}}{\pgfqpoint{6.200000in}{4.620000in}}%
\pgfusepath{clip}%
\pgfsetbuttcap%
\pgfsetroundjoin%
\pgfsetlinewidth{1.003750pt}%
\definecolor{currentstroke}{rgb}{1.000000,0.000000,0.000000}%
\pgfsetstrokecolor{currentstroke}%
\pgfsetdash{}{0pt}%
\pgfpathmoveto{\pgfqpoint{7.038414in}{0.512461in}}%
\pgfpathcurveto{\pgfqpoint{7.049464in}{0.512461in}}{\pgfqpoint{7.060063in}{0.516851in}}{\pgfqpoint{7.067877in}{0.524665in}}%
\pgfpathcurveto{\pgfqpoint{7.075690in}{0.532478in}}{\pgfqpoint{7.080080in}{0.543077in}}{\pgfqpoint{7.080080in}{0.554127in}}%
\pgfpathcurveto{\pgfqpoint{7.080080in}{0.565178in}}{\pgfqpoint{7.075690in}{0.575777in}}{\pgfqpoint{7.067877in}{0.583590in}}%
\pgfpathcurveto{\pgfqpoint{7.060063in}{0.591404in}}{\pgfqpoint{7.049464in}{0.595794in}}{\pgfqpoint{7.038414in}{0.595794in}}%
\pgfpathcurveto{\pgfqpoint{7.027364in}{0.595794in}}{\pgfqpoint{7.016765in}{0.591404in}}{\pgfqpoint{7.008951in}{0.583590in}}%
\pgfpathcurveto{\pgfqpoint{7.001137in}{0.575777in}}{\pgfqpoint{6.996747in}{0.565178in}}{\pgfqpoint{6.996747in}{0.554127in}}%
\pgfpathcurveto{\pgfqpoint{6.996747in}{0.543077in}}{\pgfqpoint{7.001137in}{0.532478in}}{\pgfqpoint{7.008951in}{0.524665in}}%
\pgfpathcurveto{\pgfqpoint{7.016765in}{0.516851in}}{\pgfqpoint{7.027364in}{0.512461in}}{\pgfqpoint{7.038414in}{0.512461in}}%
\pgfusepath{stroke}%
\end{pgfscope}%
\begin{pgfscope}%
\pgfpathrectangle{\pgfqpoint{0.847223in}{0.554012in}}{\pgfqpoint{6.200000in}{4.620000in}}%
\pgfusepath{clip}%
\pgfsetbuttcap%
\pgfsetroundjoin%
\pgfsetlinewidth{1.003750pt}%
\definecolor{currentstroke}{rgb}{1.000000,0.000000,0.000000}%
\pgfsetstrokecolor{currentstroke}%
\pgfsetdash{}{0pt}%
\pgfpathmoveto{\pgfqpoint{7.047223in}{0.512345in}}%
\pgfpathcurveto{\pgfqpoint{7.058273in}{0.512345in}}{\pgfqpoint{7.068872in}{0.516735in}}{\pgfqpoint{7.076686in}{0.524549in}}%
\pgfpathcurveto{\pgfqpoint{7.084499in}{0.532363in}}{\pgfqpoint{7.088890in}{0.542962in}}{\pgfqpoint{7.088890in}{0.554012in}}%
\pgfpathcurveto{\pgfqpoint{7.088890in}{0.565062in}}{\pgfqpoint{7.084499in}{0.575661in}}{\pgfqpoint{7.076686in}{0.583475in}}%
\pgfpathcurveto{\pgfqpoint{7.068872in}{0.591288in}}{\pgfqpoint{7.058273in}{0.595678in}}{\pgfqpoint{7.047223in}{0.595678in}}%
\pgfpathcurveto{\pgfqpoint{7.036173in}{0.595678in}}{\pgfqpoint{7.025574in}{0.591288in}}{\pgfqpoint{7.017760in}{0.583475in}}%
\pgfpathcurveto{\pgfqpoint{7.009947in}{0.575661in}}{\pgfqpoint{7.005556in}{0.565062in}}{\pgfqpoint{7.005556in}{0.554012in}}%
\pgfpathcurveto{\pgfqpoint{7.005556in}{0.542962in}}{\pgfqpoint{7.009947in}{0.532363in}}{\pgfqpoint{7.017760in}{0.524549in}}%
\pgfpathcurveto{\pgfqpoint{7.025574in}{0.516735in}}{\pgfqpoint{7.036173in}{0.512345in}}{\pgfqpoint{7.047223in}{0.512345in}}%
\pgfusepath{stroke}%
\end{pgfscope}%
\begin{pgfscope}%
\pgfpathrectangle{\pgfqpoint{0.847223in}{0.554012in}}{\pgfqpoint{6.200000in}{4.620000in}}%
\pgfusepath{clip}%
\pgfsetbuttcap%
\pgfsetroundjoin%
\definecolor{currentfill}{rgb}{0.501961,0.501961,0.501961}%
\pgfsetfillcolor{currentfill}%
\pgfsetfillopacity{0.200000}%
\pgfsetlinewidth{1.003750pt}%
\definecolor{currentstroke}{rgb}{0.501961,0.501961,0.501961}%
\pgfsetstrokecolor{currentstroke}%
\pgfsetstrokeopacity{0.200000}%
\pgfsetdash{}{0pt}%
\pgfsys@defobject{currentmarker}{\pgfqpoint{0.847223in}{0.128540in}}{\pgfqpoint{7.047223in}{5.174012in}}{%
\pgfpathmoveto{\pgfqpoint{0.847223in}{0.128540in}}%
\pgfpathlineto{\pgfqpoint{0.847223in}{5.174012in}}%
\pgfpathlineto{\pgfqpoint{0.852556in}{5.123114in}}%
\pgfpathlineto{\pgfqpoint{0.857889in}{5.073234in}}%
\pgfpathlineto{\pgfqpoint{0.863223in}{5.024339in}}%
\pgfpathlineto{\pgfqpoint{0.868556in}{4.976403in}}%
\pgfpathlineto{\pgfqpoint{0.873889in}{4.929396in}}%
\pgfpathlineto{\pgfqpoint{0.879222in}{4.883291in}}%
\pgfpathlineto{\pgfqpoint{0.884556in}{4.838064in}}%
\pgfpathlineto{\pgfqpoint{0.889889in}{4.793689in}}%
\pgfpathlineto{\pgfqpoint{0.895222in}{4.750143in}}%
\pgfpathlineto{\pgfqpoint{0.900555in}{4.707402in}}%
\pgfpathlineto{\pgfqpoint{0.905888in}{4.665444in}}%
\pgfpathlineto{\pgfqpoint{0.911222in}{4.624248in}}%
\pgfpathlineto{\pgfqpoint{0.916555in}{4.583794in}}%
\pgfpathlineto{\pgfqpoint{0.921888in}{4.544061in}}%
\pgfpathlineto{\pgfqpoint{0.927221in}{4.505030in}}%
\pgfpathlineto{\pgfqpoint{0.932555in}{4.466684in}}%
\pgfpathlineto{\pgfqpoint{0.937888in}{4.429004in}}%
\pgfpathlineto{\pgfqpoint{0.943221in}{4.391972in}}%
\pgfpathlineto{\pgfqpoint{0.948554in}{4.355573in}}%
\pgfpathlineto{\pgfqpoint{0.953887in}{4.319791in}}%
\pgfpathlineto{\pgfqpoint{0.959221in}{4.284608in}}%
\pgfpathlineto{\pgfqpoint{0.964554in}{4.250012in}}%
\pgfpathlineto{\pgfqpoint{0.969887in}{4.215987in}}%
\pgfpathlineto{\pgfqpoint{0.975220in}{4.182519in}}%
\pgfpathlineto{\pgfqpoint{0.980553in}{4.149595in}}%
\pgfpathlineto{\pgfqpoint{0.985887in}{4.117201in}}%
\pgfpathlineto{\pgfqpoint{0.991220in}{4.085325in}}%
\pgfpathlineto{\pgfqpoint{0.996553in}{4.053954in}}%
\pgfpathlineto{\pgfqpoint{1.001886in}{4.023077in}}%
\pgfpathlineto{\pgfqpoint{1.007220in}{3.992682in}}%
\pgfpathlineto{\pgfqpoint{1.012553in}{3.962758in}}%
\pgfpathlineto{\pgfqpoint{1.017886in}{3.933293in}}%
\pgfpathlineto{\pgfqpoint{1.023219in}{3.904278in}}%
\pgfpathlineto{\pgfqpoint{1.028552in}{3.875702in}}%
\pgfpathlineto{\pgfqpoint{1.033886in}{3.847556in}}%
\pgfpathlineto{\pgfqpoint{1.039219in}{3.819829in}}%
\pgfpathlineto{\pgfqpoint{1.044552in}{3.792512in}}%
\pgfpathlineto{\pgfqpoint{1.049885in}{3.765597in}}%
\pgfpathlineto{\pgfqpoint{1.055218in}{3.739075in}}%
\pgfpathlineto{\pgfqpoint{1.060552in}{3.712936in}}%
\pgfpathlineto{\pgfqpoint{1.065885in}{3.687173in}}%
\pgfpathlineto{\pgfqpoint{1.071218in}{3.661778in}}%
\pgfpathlineto{\pgfqpoint{1.076551in}{3.636743in}}%
\pgfpathlineto{\pgfqpoint{1.081885in}{3.612060in}}%
\pgfpathlineto{\pgfqpoint{1.087218in}{3.587722in}}%
\pgfpathlineto{\pgfqpoint{1.092551in}{3.563721in}}%
\pgfpathlineto{\pgfqpoint{1.097884in}{3.540052in}}%
\pgfpathlineto{\pgfqpoint{1.103217in}{3.516706in}}%
\pgfpathlineto{\pgfqpoint{1.108551in}{3.493678in}}%
\pgfpathlineto{\pgfqpoint{1.113884in}{3.470960in}}%
\pgfpathlineto{\pgfqpoint{1.119217in}{3.448547in}}%
\pgfpathlineto{\pgfqpoint{1.124550in}{3.426433in}}%
\pgfpathlineto{\pgfqpoint{1.129883in}{3.404612in}}%
\pgfpathlineto{\pgfqpoint{1.135217in}{3.383077in}}%
\pgfpathlineto{\pgfqpoint{1.140550in}{3.361824in}}%
\pgfpathlineto{\pgfqpoint{1.145883in}{3.340846in}}%
\pgfpathlineto{\pgfqpoint{1.151216in}{3.320139in}}%
\pgfpathlineto{\pgfqpoint{1.156550in}{3.299697in}}%
\pgfpathlineto{\pgfqpoint{1.161883in}{3.279515in}}%
\pgfpathlineto{\pgfqpoint{1.167216in}{3.259589in}}%
\pgfpathlineto{\pgfqpoint{1.172549in}{3.239913in}}%
\pgfpathlineto{\pgfqpoint{1.177882in}{3.220483in}}%
\pgfpathlineto{\pgfqpoint{1.183216in}{3.201294in}}%
\pgfpathlineto{\pgfqpoint{1.188549in}{3.182341in}}%
\pgfpathlineto{\pgfqpoint{1.193882in}{3.163621in}}%
\pgfpathlineto{\pgfqpoint{1.199215in}{3.145129in}}%
\pgfpathlineto{\pgfqpoint{1.204549in}{3.126861in}}%
\pgfpathlineto{\pgfqpoint{1.209882in}{3.108813in}}%
\pgfpathlineto{\pgfqpoint{1.215215in}{3.090981in}}%
\pgfpathlineto{\pgfqpoint{1.220548in}{3.073361in}}%
\pgfpathlineto{\pgfqpoint{1.225881in}{3.055950in}}%
\pgfpathlineto{\pgfqpoint{1.231215in}{3.038743in}}%
\pgfpathlineto{\pgfqpoint{1.236548in}{3.021737in}}%
\pgfpathlineto{\pgfqpoint{1.241881in}{3.004929in}}%
\pgfpathlineto{\pgfqpoint{1.247214in}{2.988315in}}%
\pgfpathlineto{\pgfqpoint{1.252547in}{2.971892in}}%
\pgfpathlineto{\pgfqpoint{1.257881in}{2.955656in}}%
\pgfpathlineto{\pgfqpoint{1.263214in}{2.939605in}}%
\pgfpathlineto{\pgfqpoint{1.268547in}{2.923735in}}%
\pgfpathlineto{\pgfqpoint{1.273880in}{2.908043in}}%
\pgfpathlineto{\pgfqpoint{1.279214in}{2.892526in}}%
\pgfpathlineto{\pgfqpoint{1.284547in}{2.877182in}}%
\pgfpathlineto{\pgfqpoint{1.289880in}{2.862007in}}%
\pgfpathlineto{\pgfqpoint{1.295213in}{2.846999in}}%
\pgfpathlineto{\pgfqpoint{1.300546in}{2.832154in}}%
\pgfpathlineto{\pgfqpoint{1.305880in}{2.817471in}}%
\pgfpathlineto{\pgfqpoint{1.311213in}{2.802947in}}%
\pgfpathlineto{\pgfqpoint{1.316546in}{2.788578in}}%
\pgfpathlineto{\pgfqpoint{1.321879in}{2.774363in}}%
\pgfpathlineto{\pgfqpoint{1.327212in}{2.760300in}}%
\pgfpathlineto{\pgfqpoint{1.332546in}{2.746385in}}%
\pgfpathlineto{\pgfqpoint{1.337879in}{2.732616in}}%
\pgfpathlineto{\pgfqpoint{1.343212in}{2.718991in}}%
\pgfpathlineto{\pgfqpoint{1.348545in}{2.705509in}}%
\pgfpathlineto{\pgfqpoint{1.353879in}{2.692165in}}%
\pgfpathlineto{\pgfqpoint{1.359212in}{2.678960in}}%
\pgfpathlineto{\pgfqpoint{1.364545in}{2.665889in}}%
\pgfpathlineto{\pgfqpoint{1.369878in}{2.652952in}}%
\pgfpathlineto{\pgfqpoint{1.375211in}{2.640146in}}%
\pgfpathlineto{\pgfqpoint{1.380545in}{2.627470in}}%
\pgfpathlineto{\pgfqpoint{1.385878in}{2.614921in}}%
\pgfpathlineto{\pgfqpoint{1.391211in}{2.602497in}}%
\pgfpathlineto{\pgfqpoint{1.396544in}{2.590197in}}%
\pgfpathlineto{\pgfqpoint{1.401877in}{2.578018in}}%
\pgfpathlineto{\pgfqpoint{1.407211in}{2.565959in}}%
\pgfpathlineto{\pgfqpoint{1.412544in}{2.554019in}}%
\pgfpathlineto{\pgfqpoint{1.417877in}{2.542195in}}%
\pgfpathlineto{\pgfqpoint{1.423210in}{2.530485in}}%
\pgfpathlineto{\pgfqpoint{1.428544in}{2.518889in}}%
\pgfpathlineto{\pgfqpoint{1.433877in}{2.507404in}}%
\pgfpathlineto{\pgfqpoint{1.439210in}{2.496029in}}%
\pgfpathlineto{\pgfqpoint{1.444543in}{2.484763in}}%
\pgfpathlineto{\pgfqpoint{1.449876in}{2.473602in}}%
\pgfpathlineto{\pgfqpoint{1.455210in}{2.462548in}}%
\pgfpathlineto{\pgfqpoint{1.460543in}{2.451597in}}%
\pgfpathlineto{\pgfqpoint{1.465876in}{2.440748in}}%
\pgfpathlineto{\pgfqpoint{1.471209in}{2.430000in}}%
\pgfpathlineto{\pgfqpoint{1.476543in}{2.419351in}}%
\pgfpathlineto{\pgfqpoint{1.481876in}{2.408801in}}%
\pgfpathlineto{\pgfqpoint{1.487209in}{2.398347in}}%
\pgfpathlineto{\pgfqpoint{1.492542in}{2.387989in}}%
\pgfpathlineto{\pgfqpoint{1.497875in}{2.377725in}}%
\pgfpathlineto{\pgfqpoint{1.503209in}{2.367554in}}%
\pgfpathlineto{\pgfqpoint{1.508542in}{2.357474in}}%
\pgfpathlineto{\pgfqpoint{1.513875in}{2.347484in}}%
\pgfpathlineto{\pgfqpoint{1.519208in}{2.337584in}}%
\pgfpathlineto{\pgfqpoint{1.524541in}{2.327772in}}%
\pgfpathlineto{\pgfqpoint{1.529875in}{2.318046in}}%
\pgfpathlineto{\pgfqpoint{1.535208in}{2.308407in}}%
\pgfpathlineto{\pgfqpoint{1.540541in}{2.298851in}}%
\pgfpathlineto{\pgfqpoint{1.545874in}{2.289379in}}%
\pgfpathlineto{\pgfqpoint{1.551208in}{2.279990in}}%
\pgfpathlineto{\pgfqpoint{1.556541in}{2.270681in}}%
\pgfpathlineto{\pgfqpoint{1.561874in}{2.261453in}}%
\pgfpathlineto{\pgfqpoint{1.567207in}{2.252304in}}%
\pgfpathlineto{\pgfqpoint{1.572540in}{2.243233in}}%
\pgfpathlineto{\pgfqpoint{1.577874in}{2.234240in}}%
\pgfpathlineto{\pgfqpoint{1.583207in}{2.225322in}}%
\pgfpathlineto{\pgfqpoint{1.588540in}{2.216480in}}%
\pgfpathlineto{\pgfqpoint{1.593873in}{2.207712in}}%
\pgfpathlineto{\pgfqpoint{1.599206in}{2.199017in}}%
\pgfpathlineto{\pgfqpoint{1.604540in}{2.190395in}}%
\pgfpathlineto{\pgfqpoint{1.609873in}{2.181844in}}%
\pgfpathlineto{\pgfqpoint{1.615206in}{2.173364in}}%
\pgfpathlineto{\pgfqpoint{1.620539in}{2.164953in}}%
\pgfpathlineto{\pgfqpoint{1.625873in}{2.156612in}}%
\pgfpathlineto{\pgfqpoint{1.631206in}{2.148338in}}%
\pgfpathlineto{\pgfqpoint{1.636539in}{2.140132in}}%
\pgfpathlineto{\pgfqpoint{1.641872in}{2.131992in}}%
\pgfpathlineto{\pgfqpoint{1.647205in}{2.123918in}}%
\pgfpathlineto{\pgfqpoint{1.652539in}{2.115909in}}%
\pgfpathlineto{\pgfqpoint{1.657872in}{2.107963in}}%
\pgfpathlineto{\pgfqpoint{1.663205in}{2.100081in}}%
\pgfpathlineto{\pgfqpoint{1.668538in}{2.092262in}}%
\pgfpathlineto{\pgfqpoint{1.673871in}{2.084504in}}%
\pgfpathlineto{\pgfqpoint{1.679205in}{2.076807in}}%
\pgfpathlineto{\pgfqpoint{1.684538in}{2.069171in}}%
\pgfpathlineto{\pgfqpoint{1.689871in}{2.061594in}}%
\pgfpathlineto{\pgfqpoint{1.695204in}{2.054076in}}%
\pgfpathlineto{\pgfqpoint{1.700538in}{2.046617in}}%
\pgfpathlineto{\pgfqpoint{1.705871in}{2.039215in}}%
\pgfpathlineto{\pgfqpoint{1.711204in}{2.031870in}}%
\pgfpathlineto{\pgfqpoint{1.716537in}{2.024581in}}%
\pgfpathlineto{\pgfqpoint{1.721870in}{2.017348in}}%
\pgfpathlineto{\pgfqpoint{1.727204in}{2.010170in}}%
\pgfpathlineto{\pgfqpoint{1.732537in}{2.003046in}}%
\pgfpathlineto{\pgfqpoint{1.737870in}{1.995976in}}%
\pgfpathlineto{\pgfqpoint{1.743203in}{1.988959in}}%
\pgfpathlineto{\pgfqpoint{1.748537in}{1.981994in}}%
\pgfpathlineto{\pgfqpoint{1.753870in}{1.975082in}}%
\pgfpathlineto{\pgfqpoint{1.759203in}{1.968221in}}%
\pgfpathlineto{\pgfqpoint{1.764536in}{1.961410in}}%
\pgfpathlineto{\pgfqpoint{1.769869in}{1.954650in}}%
\pgfpathlineto{\pgfqpoint{1.775203in}{1.947940in}}%
\pgfpathlineto{\pgfqpoint{1.780536in}{1.941278in}}%
\pgfpathlineto{\pgfqpoint{1.785869in}{1.934666in}}%
\pgfpathlineto{\pgfqpoint{1.791202in}{1.928101in}}%
\pgfpathlineto{\pgfqpoint{1.796535in}{1.921584in}}%
\pgfpathlineto{\pgfqpoint{1.801869in}{1.915114in}}%
\pgfpathlineto{\pgfqpoint{1.807202in}{1.908690in}}%
\pgfpathlineto{\pgfqpoint{1.812535in}{1.902313in}}%
\pgfpathlineto{\pgfqpoint{1.817868in}{1.895981in}}%
\pgfpathlineto{\pgfqpoint{1.823202in}{1.889694in}}%
\pgfpathlineto{\pgfqpoint{1.828535in}{1.883452in}}%
\pgfpathlineto{\pgfqpoint{1.833868in}{1.877253in}}%
\pgfpathlineto{\pgfqpoint{1.839201in}{1.871099in}}%
\pgfpathlineto{\pgfqpoint{1.844534in}{1.864987in}}%
\pgfpathlineto{\pgfqpoint{1.849868in}{1.858919in}}%
\pgfpathlineto{\pgfqpoint{1.855201in}{1.852892in}}%
\pgfpathlineto{\pgfqpoint{1.860534in}{1.846908in}}%
\pgfpathlineto{\pgfqpoint{1.865867in}{1.840964in}}%
\pgfpathlineto{\pgfqpoint{1.871200in}{1.835062in}}%
\pgfpathlineto{\pgfqpoint{1.876534in}{1.829200in}}%
\pgfpathlineto{\pgfqpoint{1.881867in}{1.823379in}}%
\pgfpathlineto{\pgfqpoint{1.887200in}{1.817597in}}%
\pgfpathlineto{\pgfqpoint{1.892533in}{1.811854in}}%
\pgfpathlineto{\pgfqpoint{1.897867in}{1.806151in}}%
\pgfpathlineto{\pgfqpoint{1.903200in}{1.800486in}}%
\pgfpathlineto{\pgfqpoint{1.908533in}{1.794859in}}%
\pgfpathlineto{\pgfqpoint{1.913866in}{1.789269in}}%
\pgfpathlineto{\pgfqpoint{1.919199in}{1.783718in}}%
\pgfpathlineto{\pgfqpoint{1.924533in}{1.778203in}}%
\pgfpathlineto{\pgfqpoint{1.929866in}{1.772725in}}%
\pgfpathlineto{\pgfqpoint{1.935199in}{1.767283in}}%
\pgfpathlineto{\pgfqpoint{1.940532in}{1.761877in}}%
\pgfpathlineto{\pgfqpoint{1.945865in}{1.756506in}}%
\pgfpathlineto{\pgfqpoint{1.951199in}{1.751171in}}%
\pgfpathlineto{\pgfqpoint{1.956532in}{1.745870in}}%
\pgfpathlineto{\pgfqpoint{1.961865in}{1.740604in}}%
\pgfpathlineto{\pgfqpoint{1.967198in}{1.735373in}}%
\pgfpathlineto{\pgfqpoint{1.972532in}{1.730175in}}%
\pgfpathlineto{\pgfqpoint{1.977865in}{1.725010in}}%
\pgfpathlineto{\pgfqpoint{1.983198in}{1.719879in}}%
\pgfpathlineto{\pgfqpoint{1.988531in}{1.714781in}}%
\pgfpathlineto{\pgfqpoint{1.993864in}{1.709715in}}%
\pgfpathlineto{\pgfqpoint{1.999198in}{1.704681in}}%
\pgfpathlineto{\pgfqpoint{2.004531in}{1.699680in}}%
\pgfpathlineto{\pgfqpoint{2.009864in}{1.694710in}}%
\pgfpathlineto{\pgfqpoint{2.015197in}{1.689771in}}%
\pgfpathlineto{\pgfqpoint{2.020531in}{1.684864in}}%
\pgfpathlineto{\pgfqpoint{2.025864in}{1.679987in}}%
\pgfpathlineto{\pgfqpoint{2.031197in}{1.675141in}}%
\pgfpathlineto{\pgfqpoint{2.036530in}{1.670325in}}%
\pgfpathlineto{\pgfqpoint{2.041863in}{1.665539in}}%
\pgfpathlineto{\pgfqpoint{2.047197in}{1.660782in}}%
\pgfpathlineto{\pgfqpoint{2.052530in}{1.656055in}}%
\pgfpathlineto{\pgfqpoint{2.057863in}{1.651357in}}%
\pgfpathlineto{\pgfqpoint{2.063196in}{1.646687in}}%
\pgfpathlineto{\pgfqpoint{2.068529in}{1.642046in}}%
\pgfpathlineto{\pgfqpoint{2.073863in}{1.637434in}}%
\pgfpathlineto{\pgfqpoint{2.079196in}{1.632849in}}%
\pgfpathlineto{\pgfqpoint{2.084529in}{1.628293in}}%
\pgfpathlineto{\pgfqpoint{2.089862in}{1.623763in}}%
\pgfpathlineto{\pgfqpoint{2.095196in}{1.619262in}}%
\pgfpathlineto{\pgfqpoint{2.100529in}{1.614787in}}%
\pgfpathlineto{\pgfqpoint{2.105862in}{1.610339in}}%
\pgfpathlineto{\pgfqpoint{2.111195in}{1.605917in}}%
\pgfpathlineto{\pgfqpoint{2.116528in}{1.601522in}}%
\pgfpathlineto{\pgfqpoint{2.121862in}{1.597153in}}%
\pgfpathlineto{\pgfqpoint{2.127195in}{1.592809in}}%
\pgfpathlineto{\pgfqpoint{2.132528in}{1.588492in}}%
\pgfpathlineto{\pgfqpoint{2.137861in}{1.584199in}}%
\pgfpathlineto{\pgfqpoint{2.143194in}{1.579932in}}%
\pgfpathlineto{\pgfqpoint{2.148528in}{1.575690in}}%
\pgfpathlineto{\pgfqpoint{2.153861in}{1.571472in}}%
\pgfpathlineto{\pgfqpoint{2.159194in}{1.567279in}}%
\pgfpathlineto{\pgfqpoint{2.164527in}{1.563111in}}%
\pgfpathlineto{\pgfqpoint{2.169861in}{1.558966in}}%
\pgfpathlineto{\pgfqpoint{2.175194in}{1.554845in}}%
\pgfpathlineto{\pgfqpoint{2.180527in}{1.550748in}}%
\pgfpathlineto{\pgfqpoint{2.185860in}{1.546675in}}%
\pgfpathlineto{\pgfqpoint{2.191193in}{1.542624in}}%
\pgfpathlineto{\pgfqpoint{2.196527in}{1.538597in}}%
\pgfpathlineto{\pgfqpoint{2.201860in}{1.534593in}}%
\pgfpathlineto{\pgfqpoint{2.207193in}{1.530611in}}%
\pgfpathlineto{\pgfqpoint{2.212526in}{1.526652in}}%
\pgfpathlineto{\pgfqpoint{2.217859in}{1.522715in}}%
\pgfpathlineto{\pgfqpoint{2.223193in}{1.518800in}}%
\pgfpathlineto{\pgfqpoint{2.228526in}{1.514907in}}%
\pgfpathlineto{\pgfqpoint{2.233859in}{1.511036in}}%
\pgfpathlineto{\pgfqpoint{2.239192in}{1.507187in}}%
\pgfpathlineto{\pgfqpoint{2.244526in}{1.503359in}}%
\pgfpathlineto{\pgfqpoint{2.249859in}{1.499552in}}%
\pgfpathlineto{\pgfqpoint{2.255192in}{1.495766in}}%
\pgfpathlineto{\pgfqpoint{2.260525in}{1.492000in}}%
\pgfpathlineto{\pgfqpoint{2.265858in}{1.488256in}}%
\pgfpathlineto{\pgfqpoint{2.271192in}{1.484532in}}%
\pgfpathlineto{\pgfqpoint{2.276525in}{1.480829in}}%
\pgfpathlineto{\pgfqpoint{2.281858in}{1.477145in}}%
\pgfpathlineto{\pgfqpoint{2.287191in}{1.473482in}}%
\pgfpathlineto{\pgfqpoint{2.292524in}{1.469838in}}%
\pgfpathlineto{\pgfqpoint{2.297858in}{1.466214in}}%
\pgfpathlineto{\pgfqpoint{2.303191in}{1.462610in}}%
\pgfpathlineto{\pgfqpoint{2.308524in}{1.459025in}}%
\pgfpathlineto{\pgfqpoint{2.313857in}{1.455459in}}%
\pgfpathlineto{\pgfqpoint{2.319191in}{1.451913in}}%
\pgfpathlineto{\pgfqpoint{2.324524in}{1.448385in}}%
\pgfpathlineto{\pgfqpoint{2.329857in}{1.444876in}}%
\pgfpathlineto{\pgfqpoint{2.335190in}{1.441385in}}%
\pgfpathlineto{\pgfqpoint{2.340523in}{1.437914in}}%
\pgfpathlineto{\pgfqpoint{2.345857in}{1.434460in}}%
\pgfpathlineto{\pgfqpoint{2.351190in}{1.431024in}}%
\pgfpathlineto{\pgfqpoint{2.356523in}{1.427607in}}%
\pgfpathlineto{\pgfqpoint{2.361856in}{1.424208in}}%
\pgfpathlineto{\pgfqpoint{2.367190in}{1.420826in}}%
\pgfpathlineto{\pgfqpoint{2.372523in}{1.417462in}}%
\pgfpathlineto{\pgfqpoint{2.377856in}{1.414115in}}%
\pgfpathlineto{\pgfqpoint{2.383189in}{1.410785in}}%
\pgfpathlineto{\pgfqpoint{2.388522in}{1.407473in}}%
\pgfpathlineto{\pgfqpoint{2.393856in}{1.404178in}}%
\pgfpathlineto{\pgfqpoint{2.399189in}{1.400900in}}%
\pgfpathlineto{\pgfqpoint{2.404522in}{1.397639in}}%
\pgfpathlineto{\pgfqpoint{2.409855in}{1.394394in}}%
\pgfpathlineto{\pgfqpoint{2.415188in}{1.391166in}}%
\pgfpathlineto{\pgfqpoint{2.420522in}{1.387954in}}%
\pgfpathlineto{\pgfqpoint{2.425855in}{1.384759in}}%
\pgfpathlineto{\pgfqpoint{2.431188in}{1.381579in}}%
\pgfpathlineto{\pgfqpoint{2.436521in}{1.378416in}}%
\pgfpathlineto{\pgfqpoint{2.441855in}{1.375269in}}%
\pgfpathlineto{\pgfqpoint{2.447188in}{1.372137in}}%
\pgfpathlineto{\pgfqpoint{2.452521in}{1.369022in}}%
\pgfpathlineto{\pgfqpoint{2.457854in}{1.365921in}}%
\pgfpathlineto{\pgfqpoint{2.463187in}{1.362837in}}%
\pgfpathlineto{\pgfqpoint{2.468521in}{1.359767in}}%
\pgfpathlineto{\pgfqpoint{2.473854in}{1.356713in}}%
\pgfpathlineto{\pgfqpoint{2.479187in}{1.353674in}}%
\pgfpathlineto{\pgfqpoint{2.484520in}{1.350650in}}%
\pgfpathlineto{\pgfqpoint{2.489853in}{1.347641in}}%
\pgfpathlineto{\pgfqpoint{2.495187in}{1.344647in}}%
\pgfpathlineto{\pgfqpoint{2.500520in}{1.341667in}}%
\pgfpathlineto{\pgfqpoint{2.505853in}{1.338702in}}%
\pgfpathlineto{\pgfqpoint{2.511186in}{1.335751in}}%
\pgfpathlineto{\pgfqpoint{2.516520in}{1.332815in}}%
\pgfpathlineto{\pgfqpoint{2.521853in}{1.329893in}}%
\pgfpathlineto{\pgfqpoint{2.527186in}{1.326985in}}%
\pgfpathlineto{\pgfqpoint{2.532519in}{1.324091in}}%
\pgfpathlineto{\pgfqpoint{2.537852in}{1.321211in}}%
\pgfpathlineto{\pgfqpoint{2.543186in}{1.318345in}}%
\pgfpathlineto{\pgfqpoint{2.548519in}{1.315492in}}%
\pgfpathlineto{\pgfqpoint{2.553852in}{1.312654in}}%
\pgfpathlineto{\pgfqpoint{2.559185in}{1.309829in}}%
\pgfpathlineto{\pgfqpoint{2.564518in}{1.307017in}}%
\pgfpathlineto{\pgfqpoint{2.569852in}{1.304218in}}%
\pgfpathlineto{\pgfqpoint{2.575185in}{1.301433in}}%
\pgfpathlineto{\pgfqpoint{2.580518in}{1.298661in}}%
\pgfpathlineto{\pgfqpoint{2.585851in}{1.295903in}}%
\pgfpathlineto{\pgfqpoint{2.591185in}{1.293157in}}%
\pgfpathlineto{\pgfqpoint{2.596518in}{1.290424in}}%
\pgfpathlineto{\pgfqpoint{2.601851in}{1.287703in}}%
\pgfpathlineto{\pgfqpoint{2.607184in}{1.284996in}}%
\pgfpathlineto{\pgfqpoint{2.612517in}{1.282301in}}%
\pgfpathlineto{\pgfqpoint{2.617851in}{1.279619in}}%
\pgfpathlineto{\pgfqpoint{2.623184in}{1.276949in}}%
\pgfpathlineto{\pgfqpoint{2.628517in}{1.274291in}}%
\pgfpathlineto{\pgfqpoint{2.633850in}{1.271646in}}%
\pgfpathlineto{\pgfqpoint{2.639184in}{1.269013in}}%
\pgfpathlineto{\pgfqpoint{2.644517in}{1.266392in}}%
\pgfpathlineto{\pgfqpoint{2.649850in}{1.263783in}}%
\pgfpathlineto{\pgfqpoint{2.655183in}{1.261186in}}%
\pgfpathlineto{\pgfqpoint{2.660516in}{1.258601in}}%
\pgfpathlineto{\pgfqpoint{2.665850in}{1.256027in}}%
\pgfpathlineto{\pgfqpoint{2.671183in}{1.253466in}}%
\pgfpathlineto{\pgfqpoint{2.676516in}{1.250915in}}%
\pgfpathlineto{\pgfqpoint{2.681849in}{1.248377in}}%
\pgfpathlineto{\pgfqpoint{2.687182in}{1.245850in}}%
\pgfpathlineto{\pgfqpoint{2.692516in}{1.243334in}}%
\pgfpathlineto{\pgfqpoint{2.697849in}{1.240830in}}%
\pgfpathlineto{\pgfqpoint{2.703182in}{1.238336in}}%
\pgfpathlineto{\pgfqpoint{2.708515in}{1.235854in}}%
\pgfpathlineto{\pgfqpoint{2.713849in}{1.233383in}}%
\pgfpathlineto{\pgfqpoint{2.719182in}{1.230924in}}%
\pgfpathlineto{\pgfqpoint{2.724515in}{1.228474in}}%
\pgfpathlineto{\pgfqpoint{2.729848in}{1.226036in}}%
\pgfpathlineto{\pgfqpoint{2.735181in}{1.223609in}}%
\pgfpathlineto{\pgfqpoint{2.740515in}{1.221192in}}%
\pgfpathlineto{\pgfqpoint{2.745848in}{1.218786in}}%
\pgfpathlineto{\pgfqpoint{2.751181in}{1.216391in}}%
\pgfpathlineto{\pgfqpoint{2.756514in}{1.214006in}}%
\pgfpathlineto{\pgfqpoint{2.761847in}{1.211631in}}%
\pgfpathlineto{\pgfqpoint{2.767181in}{1.209267in}}%
\pgfpathlineto{\pgfqpoint{2.772514in}{1.206913in}}%
\pgfpathlineto{\pgfqpoint{2.777847in}{1.204570in}}%
\pgfpathlineto{\pgfqpoint{2.783180in}{1.202236in}}%
\pgfpathlineto{\pgfqpoint{2.788514in}{1.199913in}}%
\pgfpathlineto{\pgfqpoint{2.793847in}{1.197599in}}%
\pgfpathlineto{\pgfqpoint{2.799180in}{1.195296in}}%
\pgfpathlineto{\pgfqpoint{2.804513in}{1.193003in}}%
\pgfpathlineto{\pgfqpoint{2.809846in}{1.190719in}}%
\pgfpathlineto{\pgfqpoint{2.815180in}{1.188445in}}%
\pgfpathlineto{\pgfqpoint{2.820513in}{1.186181in}}%
\pgfpathlineto{\pgfqpoint{2.825846in}{1.183927in}}%
\pgfpathlineto{\pgfqpoint{2.831179in}{1.181682in}}%
\pgfpathlineto{\pgfqpoint{2.836512in}{1.179446in}}%
\pgfpathlineto{\pgfqpoint{2.841846in}{1.177221in}}%
\pgfpathlineto{\pgfqpoint{2.847179in}{1.175004in}}%
\pgfpathlineto{\pgfqpoint{2.852512in}{1.172797in}}%
\pgfpathlineto{\pgfqpoint{2.857845in}{1.170599in}}%
\pgfpathlineto{\pgfqpoint{2.863179in}{1.168410in}}%
\pgfpathlineto{\pgfqpoint{2.868512in}{1.166231in}}%
\pgfpathlineto{\pgfqpoint{2.873845in}{1.164061in}}%
\pgfpathlineto{\pgfqpoint{2.879178in}{1.161900in}}%
\pgfpathlineto{\pgfqpoint{2.884511in}{1.159747in}}%
\pgfpathlineto{\pgfqpoint{2.889845in}{1.157604in}}%
\pgfpathlineto{\pgfqpoint{2.895178in}{1.155470in}}%
\pgfpathlineto{\pgfqpoint{2.900511in}{1.153344in}}%
\pgfpathlineto{\pgfqpoint{2.905844in}{1.151227in}}%
\pgfpathlineto{\pgfqpoint{2.911178in}{1.149119in}}%
\pgfpathlineto{\pgfqpoint{2.916511in}{1.147020in}}%
\pgfpathlineto{\pgfqpoint{2.921844in}{1.144929in}}%
\pgfpathlineto{\pgfqpoint{2.927177in}{1.142847in}}%
\pgfpathlineto{\pgfqpoint{2.932510in}{1.140773in}}%
\pgfpathlineto{\pgfqpoint{2.937844in}{1.138708in}}%
\pgfpathlineto{\pgfqpoint{2.943177in}{1.136651in}}%
\pgfpathlineto{\pgfqpoint{2.948510in}{1.134603in}}%
\pgfpathlineto{\pgfqpoint{2.953843in}{1.132562in}}%
\pgfpathlineto{\pgfqpoint{2.959176in}{1.130530in}}%
\pgfpathlineto{\pgfqpoint{2.964510in}{1.128507in}}%
\pgfpathlineto{\pgfqpoint{2.969843in}{1.126491in}}%
\pgfpathlineto{\pgfqpoint{2.975176in}{1.124484in}}%
\pgfpathlineto{\pgfqpoint{2.980509in}{1.122484in}}%
\pgfpathlineto{\pgfqpoint{2.985843in}{1.120493in}}%
\pgfpathlineto{\pgfqpoint{2.991176in}{1.118510in}}%
\pgfpathlineto{\pgfqpoint{2.996509in}{1.116534in}}%
\pgfpathlineto{\pgfqpoint{3.001842in}{1.114567in}}%
\pgfpathlineto{\pgfqpoint{3.007175in}{1.112607in}}%
\pgfpathlineto{\pgfqpoint{3.012509in}{1.110655in}}%
\pgfpathlineto{\pgfqpoint{3.017842in}{1.108711in}}%
\pgfpathlineto{\pgfqpoint{3.023175in}{1.106774in}}%
\pgfpathlineto{\pgfqpoint{3.028508in}{1.104845in}}%
\pgfpathlineto{\pgfqpoint{3.033841in}{1.102924in}}%
\pgfpathlineto{\pgfqpoint{3.039175in}{1.101010in}}%
\pgfpathlineto{\pgfqpoint{3.044508in}{1.099104in}}%
\pgfpathlineto{\pgfqpoint{3.049841in}{1.097205in}}%
\pgfpathlineto{\pgfqpoint{3.055174in}{1.095313in}}%
\pgfpathlineto{\pgfqpoint{3.060508in}{1.093429in}}%
\pgfpathlineto{\pgfqpoint{3.065841in}{1.091552in}}%
\pgfpathlineto{\pgfqpoint{3.071174in}{1.089683in}}%
\pgfpathlineto{\pgfqpoint{3.076507in}{1.087821in}}%
\pgfpathlineto{\pgfqpoint{3.081840in}{1.085966in}}%
\pgfpathlineto{\pgfqpoint{3.087174in}{1.084118in}}%
\pgfpathlineto{\pgfqpoint{3.092507in}{1.082277in}}%
\pgfpathlineto{\pgfqpoint{3.097840in}{1.080444in}}%
\pgfpathlineto{\pgfqpoint{3.103173in}{1.078617in}}%
\pgfpathlineto{\pgfqpoint{3.108506in}{1.076797in}}%
\pgfpathlineto{\pgfqpoint{3.113840in}{1.074985in}}%
\pgfpathlineto{\pgfqpoint{3.119173in}{1.073179in}}%
\pgfpathlineto{\pgfqpoint{3.124506in}{1.071380in}}%
\pgfpathlineto{\pgfqpoint{3.129839in}{1.069588in}}%
\pgfpathlineto{\pgfqpoint{3.135173in}{1.067803in}}%
\pgfpathlineto{\pgfqpoint{3.140506in}{1.066024in}}%
\pgfpathlineto{\pgfqpoint{3.145839in}{1.064253in}}%
\pgfpathlineto{\pgfqpoint{3.151172in}{1.062487in}}%
\pgfpathlineto{\pgfqpoint{3.156505in}{1.060729in}}%
\pgfpathlineto{\pgfqpoint{3.161839in}{1.058977in}}%
\pgfpathlineto{\pgfqpoint{3.167172in}{1.057232in}}%
\pgfpathlineto{\pgfqpoint{3.172505in}{1.055493in}}%
\pgfpathlineto{\pgfqpoint{3.177838in}{1.053761in}}%
\pgfpathlineto{\pgfqpoint{3.183172in}{1.052035in}}%
\pgfpathlineto{\pgfqpoint{3.188505in}{1.050316in}}%
\pgfpathlineto{\pgfqpoint{3.193838in}{1.048603in}}%
\pgfpathlineto{\pgfqpoint{3.199171in}{1.046897in}}%
\pgfpathlineto{\pgfqpoint{3.204504in}{1.045196in}}%
\pgfpathlineto{\pgfqpoint{3.209838in}{1.043502in}}%
\pgfpathlineto{\pgfqpoint{3.215171in}{1.041815in}}%
\pgfpathlineto{\pgfqpoint{3.220504in}{1.040133in}}%
\pgfpathlineto{\pgfqpoint{3.225837in}{1.038458in}}%
\pgfpathlineto{\pgfqpoint{3.231170in}{1.036789in}}%
\pgfpathlineto{\pgfqpoint{3.236504in}{1.035126in}}%
\pgfpathlineto{\pgfqpoint{3.241837in}{1.033469in}}%
\pgfpathlineto{\pgfqpoint{3.247170in}{1.031818in}}%
\pgfpathlineto{\pgfqpoint{3.252503in}{1.030173in}}%
\pgfpathlineto{\pgfqpoint{3.257837in}{1.028534in}}%
\pgfpathlineto{\pgfqpoint{3.263170in}{1.026901in}}%
\pgfpathlineto{\pgfqpoint{3.268503in}{1.025274in}}%
\pgfpathlineto{\pgfqpoint{3.273836in}{1.023653in}}%
\pgfpathlineto{\pgfqpoint{3.279169in}{1.022037in}}%
\pgfpathlineto{\pgfqpoint{3.284503in}{1.020428in}}%
\pgfpathlineto{\pgfqpoint{3.289836in}{1.018824in}}%
\pgfpathlineto{\pgfqpoint{3.295169in}{1.017226in}}%
\pgfpathlineto{\pgfqpoint{3.300502in}{1.015634in}}%
\pgfpathlineto{\pgfqpoint{3.305835in}{1.014047in}}%
\pgfpathlineto{\pgfqpoint{3.311169in}{1.012466in}}%
\pgfpathlineto{\pgfqpoint{3.316502in}{1.010891in}}%
\pgfpathlineto{\pgfqpoint{3.321835in}{1.009321in}}%
\pgfpathlineto{\pgfqpoint{3.327168in}{1.007757in}}%
\pgfpathlineto{\pgfqpoint{3.332502in}{1.006199in}}%
\pgfpathlineto{\pgfqpoint{3.337835in}{1.004646in}}%
\pgfpathlineto{\pgfqpoint{3.343168in}{1.003098in}}%
\pgfpathlineto{\pgfqpoint{3.348501in}{1.001556in}}%
\pgfpathlineto{\pgfqpoint{3.353834in}{1.000019in}}%
\pgfpathlineto{\pgfqpoint{3.359168in}{0.998488in}}%
\pgfpathlineto{\pgfqpoint{3.364501in}{0.996962in}}%
\pgfpathlineto{\pgfqpoint{3.369834in}{0.995442in}}%
\pgfpathlineto{\pgfqpoint{3.375167in}{0.993927in}}%
\pgfpathlineto{\pgfqpoint{3.380500in}{0.992417in}}%
\pgfpathlineto{\pgfqpoint{3.385834in}{0.990912in}}%
\pgfpathlineto{\pgfqpoint{3.391167in}{0.989413in}}%
\pgfpathlineto{\pgfqpoint{3.396500in}{0.987918in}}%
\pgfpathlineto{\pgfqpoint{3.401833in}{0.986429in}}%
\pgfpathlineto{\pgfqpoint{3.407167in}{0.984945in}}%
\pgfpathlineto{\pgfqpoint{3.412500in}{0.983467in}}%
\pgfpathlineto{\pgfqpoint{3.417833in}{0.981993in}}%
\pgfpathlineto{\pgfqpoint{3.423166in}{0.980524in}}%
\pgfpathlineto{\pgfqpoint{3.428499in}{0.979061in}}%
\pgfpathlineto{\pgfqpoint{3.433833in}{0.977602in}}%
\pgfpathlineto{\pgfqpoint{3.439166in}{0.976149in}}%
\pgfpathlineto{\pgfqpoint{3.444499in}{0.974700in}}%
\pgfpathlineto{\pgfqpoint{3.449832in}{0.973256in}}%
\pgfpathlineto{\pgfqpoint{3.455166in}{0.971818in}}%
\pgfpathlineto{\pgfqpoint{3.460499in}{0.970384in}}%
\pgfpathlineto{\pgfqpoint{3.465832in}{0.968955in}}%
\pgfpathlineto{\pgfqpoint{3.471165in}{0.967531in}}%
\pgfpathlineto{\pgfqpoint{3.476498in}{0.966111in}}%
\pgfpathlineto{\pgfqpoint{3.481832in}{0.964697in}}%
\pgfpathlineto{\pgfqpoint{3.487165in}{0.963287in}}%
\pgfpathlineto{\pgfqpoint{3.492498in}{0.961882in}}%
\pgfpathlineto{\pgfqpoint{3.497831in}{0.960482in}}%
\pgfpathlineto{\pgfqpoint{3.503164in}{0.959086in}}%
\pgfpathlineto{\pgfqpoint{3.508498in}{0.957696in}}%
\pgfpathlineto{\pgfqpoint{3.513831in}{0.956309in}}%
\pgfpathlineto{\pgfqpoint{3.519164in}{0.954928in}}%
\pgfpathlineto{\pgfqpoint{3.524497in}{0.953551in}}%
\pgfpathlineto{\pgfqpoint{3.529831in}{0.952178in}}%
\pgfpathlineto{\pgfqpoint{3.535164in}{0.950810in}}%
\pgfpathlineto{\pgfqpoint{3.540497in}{0.949447in}}%
\pgfpathlineto{\pgfqpoint{3.545830in}{0.948088in}}%
\pgfpathlineto{\pgfqpoint{3.551163in}{0.946734in}}%
\pgfpathlineto{\pgfqpoint{3.556497in}{0.945384in}}%
\pgfpathlineto{\pgfqpoint{3.561830in}{0.944039in}}%
\pgfpathlineto{\pgfqpoint{3.567163in}{0.942698in}}%
\pgfpathlineto{\pgfqpoint{3.572496in}{0.941361in}}%
\pgfpathlineto{\pgfqpoint{3.577829in}{0.940029in}}%
\pgfpathlineto{\pgfqpoint{3.583163in}{0.938701in}}%
\pgfpathlineto{\pgfqpoint{3.588496in}{0.937377in}}%
\pgfpathlineto{\pgfqpoint{3.593829in}{0.936058in}}%
\pgfpathlineto{\pgfqpoint{3.599162in}{0.934743in}}%
\pgfpathlineto{\pgfqpoint{3.604496in}{0.933433in}}%
\pgfpathlineto{\pgfqpoint{3.609829in}{0.932126in}}%
\pgfpathlineto{\pgfqpoint{3.615162in}{0.930824in}}%
\pgfpathlineto{\pgfqpoint{3.620495in}{0.929526in}}%
\pgfpathlineto{\pgfqpoint{3.625828in}{0.928233in}}%
\pgfpathlineto{\pgfqpoint{3.631162in}{0.926943in}}%
\pgfpathlineto{\pgfqpoint{3.636495in}{0.925658in}}%
\pgfpathlineto{\pgfqpoint{3.641828in}{0.924376in}}%
\pgfpathlineto{\pgfqpoint{3.647161in}{0.923099in}}%
\pgfpathlineto{\pgfqpoint{3.652494in}{0.921826in}}%
\pgfpathlineto{\pgfqpoint{3.657828in}{0.920557in}}%
\pgfpathlineto{\pgfqpoint{3.663161in}{0.919292in}}%
\pgfpathlineto{\pgfqpoint{3.668494in}{0.918031in}}%
\pgfpathlineto{\pgfqpoint{3.673827in}{0.916774in}}%
\pgfpathlineto{\pgfqpoint{3.679161in}{0.915521in}}%
\pgfpathlineto{\pgfqpoint{3.684494in}{0.914273in}}%
\pgfpathlineto{\pgfqpoint{3.689827in}{0.913028in}}%
\pgfpathlineto{\pgfqpoint{3.695160in}{0.911787in}}%
\pgfpathlineto{\pgfqpoint{3.700493in}{0.910549in}}%
\pgfpathlineto{\pgfqpoint{3.705827in}{0.909316in}}%
\pgfpathlineto{\pgfqpoint{3.711160in}{0.908087in}}%
\pgfpathlineto{\pgfqpoint{3.716493in}{0.906862in}}%
\pgfpathlineto{\pgfqpoint{3.721826in}{0.905640in}}%
\pgfpathlineto{\pgfqpoint{3.727160in}{0.904422in}}%
\pgfpathlineto{\pgfqpoint{3.732493in}{0.903208in}}%
\pgfpathlineto{\pgfqpoint{3.737826in}{0.901998in}}%
\pgfpathlineto{\pgfqpoint{3.743159in}{0.900792in}}%
\pgfpathlineto{\pgfqpoint{3.748492in}{0.899589in}}%
\pgfpathlineto{\pgfqpoint{3.753826in}{0.898390in}}%
\pgfpathlineto{\pgfqpoint{3.759159in}{0.897195in}}%
\pgfpathlineto{\pgfqpoint{3.764492in}{0.896003in}}%
\pgfpathlineto{\pgfqpoint{3.769825in}{0.894816in}}%
\pgfpathlineto{\pgfqpoint{3.775158in}{0.893632in}}%
\pgfpathlineto{\pgfqpoint{3.780492in}{0.892451in}}%
\pgfpathlineto{\pgfqpoint{3.785825in}{0.891274in}}%
\pgfpathlineto{\pgfqpoint{3.791158in}{0.890101in}}%
\pgfpathlineto{\pgfqpoint{3.796491in}{0.888931in}}%
\pgfpathlineto{\pgfqpoint{3.801825in}{0.887765in}}%
\pgfpathlineto{\pgfqpoint{3.807158in}{0.886603in}}%
\pgfpathlineto{\pgfqpoint{3.812491in}{0.885444in}}%
\pgfpathlineto{\pgfqpoint{3.817824in}{0.884289in}}%
\pgfpathlineto{\pgfqpoint{3.823157in}{0.883137in}}%
\pgfpathlineto{\pgfqpoint{3.828491in}{0.881989in}}%
\pgfpathlineto{\pgfqpoint{3.833824in}{0.880844in}}%
\pgfpathlineto{\pgfqpoint{3.839157in}{0.879702in}}%
\pgfpathlineto{\pgfqpoint{3.844490in}{0.878565in}}%
\pgfpathlineto{\pgfqpoint{3.849823in}{0.877430in}}%
\pgfpathlineto{\pgfqpoint{3.855157in}{0.876299in}}%
\pgfpathlineto{\pgfqpoint{3.860490in}{0.875171in}}%
\pgfpathlineto{\pgfqpoint{3.865823in}{0.874047in}}%
\pgfpathlineto{\pgfqpoint{3.871156in}{0.872926in}}%
\pgfpathlineto{\pgfqpoint{3.876490in}{0.871809in}}%
\pgfpathlineto{\pgfqpoint{3.881823in}{0.870695in}}%
\pgfpathlineto{\pgfqpoint{3.887156in}{0.869584in}}%
\pgfpathlineto{\pgfqpoint{3.892489in}{0.868476in}}%
\pgfpathlineto{\pgfqpoint{3.897822in}{0.867372in}}%
\pgfpathlineto{\pgfqpoint{3.903156in}{0.866271in}}%
\pgfpathlineto{\pgfqpoint{3.908489in}{0.865174in}}%
\pgfpathlineto{\pgfqpoint{3.913822in}{0.864079in}}%
\pgfpathlineto{\pgfqpoint{3.919155in}{0.862988in}}%
\pgfpathlineto{\pgfqpoint{3.924488in}{0.861901in}}%
\pgfpathlineto{\pgfqpoint{3.929822in}{0.860816in}}%
\pgfpathlineto{\pgfqpoint{3.935155in}{0.859734in}}%
\pgfpathlineto{\pgfqpoint{3.940488in}{0.858656in}}%
\pgfpathlineto{\pgfqpoint{3.945821in}{0.857581in}}%
\pgfpathlineto{\pgfqpoint{3.951155in}{0.856509in}}%
\pgfpathlineto{\pgfqpoint{3.956488in}{0.855440in}}%
\pgfpathlineto{\pgfqpoint{3.961821in}{0.854375in}}%
\pgfpathlineto{\pgfqpoint{3.967154in}{0.853312in}}%
\pgfpathlineto{\pgfqpoint{3.972487in}{0.852253in}}%
\pgfpathlineto{\pgfqpoint{3.977821in}{0.851197in}}%
\pgfpathlineto{\pgfqpoint{3.983154in}{0.850143in}}%
\pgfpathlineto{\pgfqpoint{3.988487in}{0.849093in}}%
\pgfpathlineto{\pgfqpoint{3.993820in}{0.848046in}}%
\pgfpathlineto{\pgfqpoint{3.999154in}{0.847002in}}%
\pgfpathlineto{\pgfqpoint{4.004487in}{0.845961in}}%
\pgfpathlineto{\pgfqpoint{4.009820in}{0.844923in}}%
\pgfpathlineto{\pgfqpoint{4.015153in}{0.843888in}}%
\pgfpathlineto{\pgfqpoint{4.020486in}{0.842856in}}%
\pgfpathlineto{\pgfqpoint{4.025820in}{0.841827in}}%
\pgfpathlineto{\pgfqpoint{4.031153in}{0.840801in}}%
\pgfpathlineto{\pgfqpoint{4.036486in}{0.839777in}}%
\pgfpathlineto{\pgfqpoint{4.041819in}{0.838757in}}%
\pgfpathlineto{\pgfqpoint{4.047152in}{0.837740in}}%
\pgfpathlineto{\pgfqpoint{4.052486in}{0.836725in}}%
\pgfpathlineto{\pgfqpoint{4.057819in}{0.835714in}}%
\pgfpathlineto{\pgfqpoint{4.063152in}{0.834705in}}%
\pgfpathlineto{\pgfqpoint{4.068485in}{0.833700in}}%
\pgfpathlineto{\pgfqpoint{4.073819in}{0.832697in}}%
\pgfpathlineto{\pgfqpoint{4.079152in}{0.831697in}}%
\pgfpathlineto{\pgfqpoint{4.084485in}{0.830699in}}%
\pgfpathlineto{\pgfqpoint{4.089818in}{0.829705in}}%
\pgfpathlineto{\pgfqpoint{4.095151in}{0.828714in}}%
\pgfpathlineto{\pgfqpoint{4.100485in}{0.827725in}}%
\pgfpathlineto{\pgfqpoint{4.105818in}{0.826739in}}%
\pgfpathlineto{\pgfqpoint{4.111151in}{0.825756in}}%
\pgfpathlineto{\pgfqpoint{4.116484in}{0.824775in}}%
\pgfpathlineto{\pgfqpoint{4.121817in}{0.823797in}}%
\pgfpathlineto{\pgfqpoint{4.127151in}{0.822823in}}%
\pgfpathlineto{\pgfqpoint{4.132484in}{0.821850in}}%
\pgfpathlineto{\pgfqpoint{4.137817in}{0.820881in}}%
\pgfpathlineto{\pgfqpoint{4.143150in}{0.819914in}}%
\pgfpathlineto{\pgfqpoint{4.148484in}{0.818950in}}%
\pgfpathlineto{\pgfqpoint{4.153817in}{0.817989in}}%
\pgfpathlineto{\pgfqpoint{4.159150in}{0.817030in}}%
\pgfpathlineto{\pgfqpoint{4.164483in}{0.816074in}}%
\pgfpathlineto{\pgfqpoint{4.169816in}{0.815120in}}%
\pgfpathlineto{\pgfqpoint{4.175150in}{0.814170in}}%
\pgfpathlineto{\pgfqpoint{4.180483in}{0.813221in}}%
\pgfpathlineto{\pgfqpoint{4.185816in}{0.812276in}}%
\pgfpathlineto{\pgfqpoint{4.191149in}{0.811333in}}%
\pgfpathlineto{\pgfqpoint{4.196482in}{0.810393in}}%
\pgfpathlineto{\pgfqpoint{4.201816in}{0.809455in}}%
\pgfpathlineto{\pgfqpoint{4.207149in}{0.808520in}}%
\pgfpathlineto{\pgfqpoint{4.212482in}{0.807587in}}%
\pgfpathlineto{\pgfqpoint{4.217815in}{0.806657in}}%
\pgfpathlineto{\pgfqpoint{4.223149in}{0.805730in}}%
\pgfpathlineto{\pgfqpoint{4.228482in}{0.804805in}}%
\pgfpathlineto{\pgfqpoint{4.233815in}{0.803882in}}%
\pgfpathlineto{\pgfqpoint{4.239148in}{0.802962in}}%
\pgfpathlineto{\pgfqpoint{4.244481in}{0.802045in}}%
\pgfpathlineto{\pgfqpoint{4.249815in}{0.801130in}}%
\pgfpathlineto{\pgfqpoint{4.255148in}{0.800218in}}%
\pgfpathlineto{\pgfqpoint{4.260481in}{0.799308in}}%
\pgfpathlineto{\pgfqpoint{4.265814in}{0.798400in}}%
\pgfpathlineto{\pgfqpoint{4.271148in}{0.797495in}}%
\pgfpathlineto{\pgfqpoint{4.276481in}{0.796592in}}%
\pgfpathlineto{\pgfqpoint{4.281814in}{0.795692in}}%
\pgfpathlineto{\pgfqpoint{4.287147in}{0.794795in}}%
\pgfpathlineto{\pgfqpoint{4.292480in}{0.793899in}}%
\pgfpathlineto{\pgfqpoint{4.297814in}{0.793006in}}%
\pgfpathlineto{\pgfqpoint{4.303147in}{0.792116in}}%
\pgfpathlineto{\pgfqpoint{4.308480in}{0.791228in}}%
\pgfpathlineto{\pgfqpoint{4.313813in}{0.790342in}}%
\pgfpathlineto{\pgfqpoint{4.319146in}{0.789458in}}%
\pgfpathlineto{\pgfqpoint{4.324480in}{0.788577in}}%
\pgfpathlineto{\pgfqpoint{4.329813in}{0.787699in}}%
\pgfpathlineto{\pgfqpoint{4.335146in}{0.786822in}}%
\pgfpathlineto{\pgfqpoint{4.340479in}{0.785948in}}%
\pgfpathlineto{\pgfqpoint{4.345813in}{0.785076in}}%
\pgfpathlineto{\pgfqpoint{4.351146in}{0.784207in}}%
\pgfpathlineto{\pgfqpoint{4.356479in}{0.783340in}}%
\pgfpathlineto{\pgfqpoint{4.361812in}{0.782475in}}%
\pgfpathlineto{\pgfqpoint{4.367145in}{0.781612in}}%
\pgfpathlineto{\pgfqpoint{4.372479in}{0.780752in}}%
\pgfpathlineto{\pgfqpoint{4.377812in}{0.779894in}}%
\pgfpathlineto{\pgfqpoint{4.383145in}{0.779038in}}%
\pgfpathlineto{\pgfqpoint{4.388478in}{0.778185in}}%
\pgfpathlineto{\pgfqpoint{4.393811in}{0.777333in}}%
\pgfpathlineto{\pgfqpoint{4.399145in}{0.776484in}}%
\pgfpathlineto{\pgfqpoint{4.404478in}{0.775638in}}%
\pgfpathlineto{\pgfqpoint{4.409811in}{0.774793in}}%
\pgfpathlineto{\pgfqpoint{4.415144in}{0.773950in}}%
\pgfpathlineto{\pgfqpoint{4.420478in}{0.773110in}}%
\pgfpathlineto{\pgfqpoint{4.425811in}{0.772272in}}%
\pgfpathlineto{\pgfqpoint{4.431144in}{0.771436in}}%
\pgfpathlineto{\pgfqpoint{4.436477in}{0.770603in}}%
\pgfpathlineto{\pgfqpoint{4.441810in}{0.769771in}}%
\pgfpathlineto{\pgfqpoint{4.447144in}{0.768942in}}%
\pgfpathlineto{\pgfqpoint{4.452477in}{0.768114in}}%
\pgfpathlineto{\pgfqpoint{4.457810in}{0.767289in}}%
\pgfpathlineto{\pgfqpoint{4.463143in}{0.766466in}}%
\pgfpathlineto{\pgfqpoint{4.468476in}{0.765645in}}%
\pgfpathlineto{\pgfqpoint{4.473810in}{0.764827in}}%
\pgfpathlineto{\pgfqpoint{4.479143in}{0.764010in}}%
\pgfpathlineto{\pgfqpoint{4.484476in}{0.763195in}}%
\pgfpathlineto{\pgfqpoint{4.489809in}{0.762383in}}%
\pgfpathlineto{\pgfqpoint{4.495143in}{0.761573in}}%
\pgfpathlineto{\pgfqpoint{4.500476in}{0.760764in}}%
\pgfpathlineto{\pgfqpoint{4.505809in}{0.759958in}}%
\pgfpathlineto{\pgfqpoint{4.511142in}{0.759154in}}%
\pgfpathlineto{\pgfqpoint{4.516475in}{0.758352in}}%
\pgfpathlineto{\pgfqpoint{4.521809in}{0.757551in}}%
\pgfpathlineto{\pgfqpoint{4.527142in}{0.756753in}}%
\pgfpathlineto{\pgfqpoint{4.532475in}{0.755957in}}%
\pgfpathlineto{\pgfqpoint{4.537808in}{0.755163in}}%
\pgfpathlineto{\pgfqpoint{4.543142in}{0.754371in}}%
\pgfpathlineto{\pgfqpoint{4.548475in}{0.753581in}}%
\pgfpathlineto{\pgfqpoint{4.553808in}{0.752793in}}%
\pgfpathlineto{\pgfqpoint{4.559141in}{0.752007in}}%
\pgfpathlineto{\pgfqpoint{4.564474in}{0.751223in}}%
\pgfpathlineto{\pgfqpoint{4.569808in}{0.750441in}}%
\pgfpathlineto{\pgfqpoint{4.575141in}{0.749660in}}%
\pgfpathlineto{\pgfqpoint{4.580474in}{0.748882in}}%
\pgfpathlineto{\pgfqpoint{4.585807in}{0.748106in}}%
\pgfpathlineto{\pgfqpoint{4.591140in}{0.747332in}}%
\pgfpathlineto{\pgfqpoint{4.596474in}{0.746559in}}%
\pgfpathlineto{\pgfqpoint{4.601807in}{0.745789in}}%
\pgfpathlineto{\pgfqpoint{4.607140in}{0.745020in}}%
\pgfpathlineto{\pgfqpoint{4.612473in}{0.744254in}}%
\pgfpathlineto{\pgfqpoint{4.617807in}{0.743489in}}%
\pgfpathlineto{\pgfqpoint{4.623140in}{0.742726in}}%
\pgfpathlineto{\pgfqpoint{4.628473in}{0.741965in}}%
\pgfpathlineto{\pgfqpoint{4.633806in}{0.741206in}}%
\pgfpathlineto{\pgfqpoint{4.639139in}{0.740449in}}%
\pgfpathlineto{\pgfqpoint{4.644473in}{0.739693in}}%
\pgfpathlineto{\pgfqpoint{4.649806in}{0.738940in}}%
\pgfpathlineto{\pgfqpoint{4.655139in}{0.738188in}}%
\pgfpathlineto{\pgfqpoint{4.660472in}{0.737439in}}%
\pgfpathlineto{\pgfqpoint{4.665805in}{0.736691in}}%
\pgfpathlineto{\pgfqpoint{4.671139in}{0.735945in}}%
\pgfpathlineto{\pgfqpoint{4.676472in}{0.735200in}}%
\pgfpathlineto{\pgfqpoint{4.681805in}{0.734458in}}%
\pgfpathlineto{\pgfqpoint{4.687138in}{0.733717in}}%
\pgfpathlineto{\pgfqpoint{4.692472in}{0.732979in}}%
\pgfpathlineto{\pgfqpoint{4.697805in}{0.732242in}}%
\pgfpathlineto{\pgfqpoint{4.703138in}{0.731506in}}%
\pgfpathlineto{\pgfqpoint{4.708471in}{0.730773in}}%
\pgfpathlineto{\pgfqpoint{4.713804in}{0.730041in}}%
\pgfpathlineto{\pgfqpoint{4.719138in}{0.729311in}}%
\pgfpathlineto{\pgfqpoint{4.724471in}{0.728583in}}%
\pgfpathlineto{\pgfqpoint{4.729804in}{0.727857in}}%
\pgfpathlineto{\pgfqpoint{4.735137in}{0.727132in}}%
\pgfpathlineto{\pgfqpoint{4.740470in}{0.726410in}}%
\pgfpathlineto{\pgfqpoint{4.745804in}{0.725689in}}%
\pgfpathlineto{\pgfqpoint{4.751137in}{0.724969in}}%
\pgfpathlineto{\pgfqpoint{4.756470in}{0.724252in}}%
\pgfpathlineto{\pgfqpoint{4.761803in}{0.723536in}}%
\pgfpathlineto{\pgfqpoint{4.767137in}{0.722822in}}%
\pgfpathlineto{\pgfqpoint{4.772470in}{0.722109in}}%
\pgfpathlineto{\pgfqpoint{4.777803in}{0.721398in}}%
\pgfpathlineto{\pgfqpoint{4.783136in}{0.720689in}}%
\pgfpathlineto{\pgfqpoint{4.788469in}{0.719982in}}%
\pgfpathlineto{\pgfqpoint{4.793803in}{0.719276in}}%
\pgfpathlineto{\pgfqpoint{4.799136in}{0.718572in}}%
\pgfpathlineto{\pgfqpoint{4.804469in}{0.717870in}}%
\pgfpathlineto{\pgfqpoint{4.809802in}{0.717169in}}%
\pgfpathlineto{\pgfqpoint{4.815136in}{0.716470in}}%
\pgfpathlineto{\pgfqpoint{4.820469in}{0.715773in}}%
\pgfpathlineto{\pgfqpoint{4.825802in}{0.715077in}}%
\pgfpathlineto{\pgfqpoint{4.831135in}{0.714383in}}%
\pgfpathlineto{\pgfqpoint{4.836468in}{0.713691in}}%
\pgfpathlineto{\pgfqpoint{4.841802in}{0.713000in}}%
\pgfpathlineto{\pgfqpoint{4.847135in}{0.712311in}}%
\pgfpathlineto{\pgfqpoint{4.852468in}{0.711624in}}%
\pgfpathlineto{\pgfqpoint{4.857801in}{0.710938in}}%
\pgfpathlineto{\pgfqpoint{4.863134in}{0.710253in}}%
\pgfpathlineto{\pgfqpoint{4.868468in}{0.709571in}}%
\pgfpathlineto{\pgfqpoint{4.873801in}{0.708890in}}%
\pgfpathlineto{\pgfqpoint{4.879134in}{0.708210in}}%
\pgfpathlineto{\pgfqpoint{4.884467in}{0.707532in}}%
\pgfpathlineto{\pgfqpoint{4.889801in}{0.706856in}}%
\pgfpathlineto{\pgfqpoint{4.895134in}{0.706181in}}%
\pgfpathlineto{\pgfqpoint{4.900467in}{0.705508in}}%
\pgfpathlineto{\pgfqpoint{4.905800in}{0.704837in}}%
\pgfpathlineto{\pgfqpoint{4.911133in}{0.704167in}}%
\pgfpathlineto{\pgfqpoint{4.916467in}{0.703498in}}%
\pgfpathlineto{\pgfqpoint{4.921800in}{0.702831in}}%
\pgfpathlineto{\pgfqpoint{4.927133in}{0.702166in}}%
\pgfpathlineto{\pgfqpoint{4.932466in}{0.701502in}}%
\pgfpathlineto{\pgfqpoint{4.937799in}{0.700840in}}%
\pgfpathlineto{\pgfqpoint{4.943133in}{0.700179in}}%
\pgfpathlineto{\pgfqpoint{4.948466in}{0.699520in}}%
\pgfpathlineto{\pgfqpoint{4.953799in}{0.698862in}}%
\pgfpathlineto{\pgfqpoint{4.959132in}{0.698206in}}%
\pgfpathlineto{\pgfqpoint{4.964466in}{0.697551in}}%
\pgfpathlineto{\pgfqpoint{4.969799in}{0.696898in}}%
\pgfpathlineto{\pgfqpoint{4.975132in}{0.696246in}}%
\pgfpathlineto{\pgfqpoint{4.980465in}{0.695596in}}%
\pgfpathlineto{\pgfqpoint{4.985798in}{0.694948in}}%
\pgfpathlineto{\pgfqpoint{4.991132in}{0.694300in}}%
\pgfpathlineto{\pgfqpoint{4.996465in}{0.693655in}}%
\pgfpathlineto{\pgfqpoint{5.001798in}{0.693010in}}%
\pgfpathlineto{\pgfqpoint{5.007131in}{0.692367in}}%
\pgfpathlineto{\pgfqpoint{5.012464in}{0.691726in}}%
\pgfpathlineto{\pgfqpoint{5.017798in}{0.691086in}}%
\pgfpathlineto{\pgfqpoint{5.023131in}{0.690448in}}%
\pgfpathlineto{\pgfqpoint{5.028464in}{0.689811in}}%
\pgfpathlineto{\pgfqpoint{5.033797in}{0.689175in}}%
\pgfpathlineto{\pgfqpoint{5.039131in}{0.688541in}}%
\pgfpathlineto{\pgfqpoint{5.044464in}{0.687908in}}%
\pgfpathlineto{\pgfqpoint{5.049797in}{0.687277in}}%
\pgfpathlineto{\pgfqpoint{5.055130in}{0.686647in}}%
\pgfpathlineto{\pgfqpoint{5.060463in}{0.686019in}}%
\pgfpathlineto{\pgfqpoint{5.065797in}{0.685392in}}%
\pgfpathlineto{\pgfqpoint{5.071130in}{0.684766in}}%
\pgfpathlineto{\pgfqpoint{5.076463in}{0.684142in}}%
\pgfpathlineto{\pgfqpoint{5.081796in}{0.683519in}}%
\pgfpathlineto{\pgfqpoint{5.087130in}{0.682898in}}%
\pgfpathlineto{\pgfqpoint{5.092463in}{0.682278in}}%
\pgfpathlineto{\pgfqpoint{5.097796in}{0.681659in}}%
\pgfpathlineto{\pgfqpoint{5.103129in}{0.681042in}}%
\pgfpathlineto{\pgfqpoint{5.108462in}{0.680426in}}%
\pgfpathlineto{\pgfqpoint{5.113796in}{0.679812in}}%
\pgfpathlineto{\pgfqpoint{5.119129in}{0.679199in}}%
\pgfpathlineto{\pgfqpoint{5.124462in}{0.678587in}}%
\pgfpathlineto{\pgfqpoint{5.129795in}{0.677977in}}%
\pgfpathlineto{\pgfqpoint{5.135128in}{0.677368in}}%
\pgfpathlineto{\pgfqpoint{5.140462in}{0.676760in}}%
\pgfpathlineto{\pgfqpoint{5.145795in}{0.676154in}}%
\pgfpathlineto{\pgfqpoint{5.151128in}{0.675549in}}%
\pgfpathlineto{\pgfqpoint{5.156461in}{0.674945in}}%
\pgfpathlineto{\pgfqpoint{5.161795in}{0.674342in}}%
\pgfpathlineto{\pgfqpoint{5.167128in}{0.673741in}}%
\pgfpathlineto{\pgfqpoint{5.172461in}{0.673142in}}%
\pgfpathlineto{\pgfqpoint{5.177794in}{0.672543in}}%
\pgfpathlineto{\pgfqpoint{5.183127in}{0.671946in}}%
\pgfpathlineto{\pgfqpoint{5.188461in}{0.671351in}}%
\pgfpathlineto{\pgfqpoint{5.193794in}{0.670756in}}%
\pgfpathlineto{\pgfqpoint{5.199127in}{0.670163in}}%
\pgfpathlineto{\pgfqpoint{5.204460in}{0.669571in}}%
\pgfpathlineto{\pgfqpoint{5.209793in}{0.668981in}}%
\pgfpathlineto{\pgfqpoint{5.215127in}{0.668391in}}%
\pgfpathlineto{\pgfqpoint{5.220460in}{0.667803in}}%
\pgfpathlineto{\pgfqpoint{5.225793in}{0.667217in}}%
\pgfpathlineto{\pgfqpoint{5.231126in}{0.666631in}}%
\pgfpathlineto{\pgfqpoint{5.236460in}{0.666047in}}%
\pgfpathlineto{\pgfqpoint{5.241793in}{0.665464in}}%
\pgfpathlineto{\pgfqpoint{5.247126in}{0.664882in}}%
\pgfpathlineto{\pgfqpoint{5.252459in}{0.664302in}}%
\pgfpathlineto{\pgfqpoint{5.257792in}{0.663723in}}%
\pgfpathlineto{\pgfqpoint{5.263126in}{0.663145in}}%
\pgfpathlineto{\pgfqpoint{5.268459in}{0.662568in}}%
\pgfpathlineto{\pgfqpoint{5.273792in}{0.661993in}}%
\pgfpathlineto{\pgfqpoint{5.279125in}{0.661419in}}%
\pgfpathlineto{\pgfqpoint{5.284458in}{0.660846in}}%
\pgfpathlineto{\pgfqpoint{5.289792in}{0.660274in}}%
\pgfpathlineto{\pgfqpoint{5.295125in}{0.659704in}}%
\pgfpathlineto{\pgfqpoint{5.300458in}{0.659135in}}%
\pgfpathlineto{\pgfqpoint{5.305791in}{0.658567in}}%
\pgfpathlineto{\pgfqpoint{5.311125in}{0.658000in}}%
\pgfpathlineto{\pgfqpoint{5.316458in}{0.657434in}}%
\pgfpathlineto{\pgfqpoint{5.321791in}{0.656870in}}%
\pgfpathlineto{\pgfqpoint{5.327124in}{0.656307in}}%
\pgfpathlineto{\pgfqpoint{5.332457in}{0.655745in}}%
\pgfpathlineto{\pgfqpoint{5.337791in}{0.655184in}}%
\pgfpathlineto{\pgfqpoint{5.343124in}{0.654624in}}%
\pgfpathlineto{\pgfqpoint{5.348457in}{0.654066in}}%
\pgfpathlineto{\pgfqpoint{5.353790in}{0.653509in}}%
\pgfpathlineto{\pgfqpoint{5.359124in}{0.652953in}}%
\pgfpathlineto{\pgfqpoint{5.364457in}{0.652398in}}%
\pgfpathlineto{\pgfqpoint{5.369790in}{0.651844in}}%
\pgfpathlineto{\pgfqpoint{5.375123in}{0.651292in}}%
\pgfpathlineto{\pgfqpoint{5.380456in}{0.650740in}}%
\pgfpathlineto{\pgfqpoint{5.385790in}{0.650190in}}%
\pgfpathlineto{\pgfqpoint{5.391123in}{0.649641in}}%
\pgfpathlineto{\pgfqpoint{5.396456in}{0.649093in}}%
\pgfpathlineto{\pgfqpoint{5.401789in}{0.648547in}}%
\pgfpathlineto{\pgfqpoint{5.407122in}{0.648001in}}%
\pgfpathlineto{\pgfqpoint{5.412456in}{0.647457in}}%
\pgfpathlineto{\pgfqpoint{5.417789in}{0.646913in}}%
\pgfpathlineto{\pgfqpoint{5.423122in}{0.646371in}}%
\pgfpathlineto{\pgfqpoint{5.428455in}{0.645830in}}%
\pgfpathlineto{\pgfqpoint{5.433789in}{0.645290in}}%
\pgfpathlineto{\pgfqpoint{5.439122in}{0.644751in}}%
\pgfpathlineto{\pgfqpoint{5.444455in}{0.644214in}}%
\pgfpathlineto{\pgfqpoint{5.449788in}{0.643677in}}%
\pgfpathlineto{\pgfqpoint{5.455121in}{0.643142in}}%
\pgfpathlineto{\pgfqpoint{5.460455in}{0.642608in}}%
\pgfpathlineto{\pgfqpoint{5.465788in}{0.642074in}}%
\pgfpathlineto{\pgfqpoint{5.471121in}{0.641542in}}%
\pgfpathlineto{\pgfqpoint{5.476454in}{0.641011in}}%
\pgfpathlineto{\pgfqpoint{5.481787in}{0.640481in}}%
\pgfpathlineto{\pgfqpoint{5.487121in}{0.639953in}}%
\pgfpathlineto{\pgfqpoint{5.492454in}{0.639425in}}%
\pgfpathlineto{\pgfqpoint{5.497787in}{0.638898in}}%
\pgfpathlineto{\pgfqpoint{5.503120in}{0.638373in}}%
\pgfpathlineto{\pgfqpoint{5.508454in}{0.637848in}}%
\pgfpathlineto{\pgfqpoint{5.513787in}{0.637325in}}%
\pgfpathlineto{\pgfqpoint{5.519120in}{0.636803in}}%
\pgfpathlineto{\pgfqpoint{5.524453in}{0.636281in}}%
\pgfpathlineto{\pgfqpoint{5.529786in}{0.635761in}}%
\pgfpathlineto{\pgfqpoint{5.535120in}{0.635242in}}%
\pgfpathlineto{\pgfqpoint{5.540453in}{0.634724in}}%
\pgfpathlineto{\pgfqpoint{5.545786in}{0.634207in}}%
\pgfpathlineto{\pgfqpoint{5.551119in}{0.633691in}}%
\pgfpathlineto{\pgfqpoint{5.556452in}{0.633176in}}%
\pgfpathlineto{\pgfqpoint{5.561786in}{0.632663in}}%
\pgfpathlineto{\pgfqpoint{5.567119in}{0.632150in}}%
\pgfpathlineto{\pgfqpoint{5.572452in}{0.631638in}}%
\pgfpathlineto{\pgfqpoint{5.577785in}{0.631127in}}%
\pgfpathlineto{\pgfqpoint{5.583119in}{0.630618in}}%
\pgfpathlineto{\pgfqpoint{5.588452in}{0.630109in}}%
\pgfpathlineto{\pgfqpoint{5.593785in}{0.629601in}}%
\pgfpathlineto{\pgfqpoint{5.599118in}{0.629095in}}%
\pgfpathlineto{\pgfqpoint{5.604451in}{0.628589in}}%
\pgfpathlineto{\pgfqpoint{5.609785in}{0.628085in}}%
\pgfpathlineto{\pgfqpoint{5.615118in}{0.627581in}}%
\pgfpathlineto{\pgfqpoint{5.620451in}{0.627079in}}%
\pgfpathlineto{\pgfqpoint{5.625784in}{0.626577in}}%
\pgfpathlineto{\pgfqpoint{5.631118in}{0.626077in}}%
\pgfpathlineto{\pgfqpoint{5.636451in}{0.625577in}}%
\pgfpathlineto{\pgfqpoint{5.641784in}{0.625079in}}%
\pgfpathlineto{\pgfqpoint{5.647117in}{0.624581in}}%
\pgfpathlineto{\pgfqpoint{5.652450in}{0.624085in}}%
\pgfpathlineto{\pgfqpoint{5.657784in}{0.623590in}}%
\pgfpathlineto{\pgfqpoint{5.663117in}{0.623095in}}%
\pgfpathlineto{\pgfqpoint{5.668450in}{0.622602in}}%
\pgfpathlineto{\pgfqpoint{5.673783in}{0.622109in}}%
\pgfpathlineto{\pgfqpoint{5.679116in}{0.621617in}}%
\pgfpathlineto{\pgfqpoint{5.684450in}{0.621127in}}%
\pgfpathlineto{\pgfqpoint{5.689783in}{0.620637in}}%
\pgfpathlineto{\pgfqpoint{5.695116in}{0.620149in}}%
\pgfpathlineto{\pgfqpoint{5.700449in}{0.619661in}}%
\pgfpathlineto{\pgfqpoint{5.705783in}{0.619174in}}%
\pgfpathlineto{\pgfqpoint{5.711116in}{0.618689in}}%
\pgfpathlineto{\pgfqpoint{5.716449in}{0.618204in}}%
\pgfpathlineto{\pgfqpoint{5.721782in}{0.617720in}}%
\pgfpathlineto{\pgfqpoint{5.727115in}{0.617237in}}%
\pgfpathlineto{\pgfqpoint{5.732449in}{0.616755in}}%
\pgfpathlineto{\pgfqpoint{5.737782in}{0.616274in}}%
\pgfpathlineto{\pgfqpoint{5.743115in}{0.615794in}}%
\pgfpathlineto{\pgfqpoint{5.748448in}{0.615315in}}%
\pgfpathlineto{\pgfqpoint{5.753781in}{0.614837in}}%
\pgfpathlineto{\pgfqpoint{5.759115in}{0.614360in}}%
\pgfpathlineto{\pgfqpoint{5.764448in}{0.613884in}}%
\pgfpathlineto{\pgfqpoint{5.769781in}{0.613409in}}%
\pgfpathlineto{\pgfqpoint{5.775114in}{0.612934in}}%
\pgfpathlineto{\pgfqpoint{5.780448in}{0.612461in}}%
\pgfpathlineto{\pgfqpoint{5.785781in}{0.611988in}}%
\pgfpathlineto{\pgfqpoint{5.791114in}{0.611517in}}%
\pgfpathlineto{\pgfqpoint{5.796447in}{0.611046in}}%
\pgfpathlineto{\pgfqpoint{5.801780in}{0.610576in}}%
\pgfpathlineto{\pgfqpoint{5.807114in}{0.610107in}}%
\pgfpathlineto{\pgfqpoint{5.812447in}{0.609639in}}%
\pgfpathlineto{\pgfqpoint{5.817780in}{0.609172in}}%
\pgfpathlineto{\pgfqpoint{5.823113in}{0.608706in}}%
\pgfpathlineto{\pgfqpoint{5.828446in}{0.608241in}}%
\pgfpathlineto{\pgfqpoint{5.833780in}{0.607777in}}%
\pgfpathlineto{\pgfqpoint{5.839113in}{0.607313in}}%
\pgfpathlineto{\pgfqpoint{5.844446in}{0.606851in}}%
\pgfpathlineto{\pgfqpoint{5.849779in}{0.606389in}}%
\pgfpathlineto{\pgfqpoint{5.855113in}{0.605928in}}%
\pgfpathlineto{\pgfqpoint{5.860446in}{0.605469in}}%
\pgfpathlineto{\pgfqpoint{5.865779in}{0.605010in}}%
\pgfpathlineto{\pgfqpoint{5.871112in}{0.604552in}}%
\pgfpathlineto{\pgfqpoint{5.876445in}{0.604094in}}%
\pgfpathlineto{\pgfqpoint{5.881779in}{0.603638in}}%
\pgfpathlineto{\pgfqpoint{5.887112in}{0.603183in}}%
\pgfpathlineto{\pgfqpoint{5.892445in}{0.602728in}}%
\pgfpathlineto{\pgfqpoint{5.897778in}{0.602274in}}%
\pgfpathlineto{\pgfqpoint{5.903112in}{0.601821in}}%
\pgfpathlineto{\pgfqpoint{5.908445in}{0.601369in}}%
\pgfpathlineto{\pgfqpoint{5.913778in}{0.600918in}}%
\pgfpathlineto{\pgfqpoint{5.919111in}{0.600468in}}%
\pgfpathlineto{\pgfqpoint{5.924444in}{0.600019in}}%
\pgfpathlineto{\pgfqpoint{5.929778in}{0.599570in}}%
\pgfpathlineto{\pgfqpoint{5.935111in}{0.599122in}}%
\pgfpathlineto{\pgfqpoint{5.940444in}{0.598676in}}%
\pgfpathlineto{\pgfqpoint{5.945777in}{0.598230in}}%
\pgfpathlineto{\pgfqpoint{5.951110in}{0.597784in}}%
\pgfpathlineto{\pgfqpoint{5.956444in}{0.597340in}}%
\pgfpathlineto{\pgfqpoint{5.961777in}{0.596897in}}%
\pgfpathlineto{\pgfqpoint{5.967110in}{0.596454in}}%
\pgfpathlineto{\pgfqpoint{5.972443in}{0.596012in}}%
\pgfpathlineto{\pgfqpoint{5.977777in}{0.595571in}}%
\pgfpathlineto{\pgfqpoint{5.983110in}{0.595131in}}%
\pgfpathlineto{\pgfqpoint{5.988443in}{0.594692in}}%
\pgfpathlineto{\pgfqpoint{5.993776in}{0.594253in}}%
\pgfpathlineto{\pgfqpoint{5.999109in}{0.593816in}}%
\pgfpathlineto{\pgfqpoint{6.004443in}{0.593379in}}%
\pgfpathlineto{\pgfqpoint{6.009776in}{0.592943in}}%
\pgfpathlineto{\pgfqpoint{6.015109in}{0.592508in}}%
\pgfpathlineto{\pgfqpoint{6.020442in}{0.592073in}}%
\pgfpathlineto{\pgfqpoint{6.025775in}{0.591640in}}%
\pgfpathlineto{\pgfqpoint{6.031109in}{0.591207in}}%
\pgfpathlineto{\pgfqpoint{6.036442in}{0.590775in}}%
\pgfpathlineto{\pgfqpoint{6.041775in}{0.590344in}}%
\pgfpathlineto{\pgfqpoint{6.047108in}{0.589914in}}%
\pgfpathlineto{\pgfqpoint{6.052442in}{0.589484in}}%
\pgfpathlineto{\pgfqpoint{6.057775in}{0.589055in}}%
\pgfpathlineto{\pgfqpoint{6.063108in}{0.588627in}}%
\pgfpathlineto{\pgfqpoint{6.068441in}{0.588200in}}%
\pgfpathlineto{\pgfqpoint{6.073774in}{0.587774in}}%
\pgfpathlineto{\pgfqpoint{6.079108in}{0.587348in}}%
\pgfpathlineto{\pgfqpoint{6.084441in}{0.586924in}}%
\pgfpathlineto{\pgfqpoint{6.089774in}{0.586500in}}%
\pgfpathlineto{\pgfqpoint{6.095107in}{0.586076in}}%
\pgfpathlineto{\pgfqpoint{6.100440in}{0.585654in}}%
\pgfpathlineto{\pgfqpoint{6.105774in}{0.585232in}}%
\pgfpathlineto{\pgfqpoint{6.111107in}{0.584812in}}%
\pgfpathlineto{\pgfqpoint{6.116440in}{0.584391in}}%
\pgfpathlineto{\pgfqpoint{6.121773in}{0.583972in}}%
\pgfpathlineto{\pgfqpoint{6.127107in}{0.583554in}}%
\pgfpathlineto{\pgfqpoint{6.132440in}{0.583136in}}%
\pgfpathlineto{\pgfqpoint{6.137773in}{0.582719in}}%
\pgfpathlineto{\pgfqpoint{6.143106in}{0.582303in}}%
\pgfpathlineto{\pgfqpoint{6.148439in}{0.581887in}}%
\pgfpathlineto{\pgfqpoint{6.153773in}{0.581472in}}%
\pgfpathlineto{\pgfqpoint{6.159106in}{0.581058in}}%
\pgfpathlineto{\pgfqpoint{6.164439in}{0.580645in}}%
\pgfpathlineto{\pgfqpoint{6.169772in}{0.580233in}}%
\pgfpathlineto{\pgfqpoint{6.175105in}{0.579821in}}%
\pgfpathlineto{\pgfqpoint{6.175105in}{0.579821in}}%
\pgfpathlineto{\pgfqpoint{6.183915in}{0.579155in}}%
\pgfpathlineto{\pgfqpoint{6.192724in}{0.578515in}}%
\pgfpathlineto{\pgfqpoint{6.201533in}{0.577898in}}%
\pgfpathlineto{\pgfqpoint{6.210343in}{0.577303in}}%
\pgfpathlineto{\pgfqpoint{6.219152in}{0.576729in}}%
\pgfpathlineto{\pgfqpoint{6.227961in}{0.576175in}}%
\pgfpathlineto{\pgfqpoint{6.236770in}{0.575638in}}%
\pgfpathlineto{\pgfqpoint{6.245580in}{0.575118in}}%
\pgfpathlineto{\pgfqpoint{6.254389in}{0.574614in}}%
\pgfpathlineto{\pgfqpoint{6.263198in}{0.574126in}}%
\pgfpathlineto{\pgfqpoint{6.272007in}{0.573651in}}%
\pgfpathlineto{\pgfqpoint{6.280817in}{0.573190in}}%
\pgfpathlineto{\pgfqpoint{6.289626in}{0.572742in}}%
\pgfpathlineto{\pgfqpoint{6.298435in}{0.572306in}}%
\pgfpathlineto{\pgfqpoint{6.307245in}{0.571882in}}%
\pgfpathlineto{\pgfqpoint{6.316054in}{0.571468in}}%
\pgfpathlineto{\pgfqpoint{6.324863in}{0.571065in}}%
\pgfpathlineto{\pgfqpoint{6.333672in}{0.570672in}}%
\pgfpathlineto{\pgfqpoint{6.342482in}{0.570289in}}%
\pgfpathlineto{\pgfqpoint{6.351291in}{0.569915in}}%
\pgfpathlineto{\pgfqpoint{6.360100in}{0.569549in}}%
\pgfpathlineto{\pgfqpoint{6.368909in}{0.569192in}}%
\pgfpathlineto{\pgfqpoint{6.377719in}{0.568843in}}%
\pgfpathlineto{\pgfqpoint{6.386528in}{0.568501in}}%
\pgfpathlineto{\pgfqpoint{6.395337in}{0.568167in}}%
\pgfpathlineto{\pgfqpoint{6.404146in}{0.567840in}}%
\pgfpathlineto{\pgfqpoint{6.412956in}{0.567520in}}%
\pgfpathlineto{\pgfqpoint{6.421765in}{0.567206in}}%
\pgfpathlineto{\pgfqpoint{6.430574in}{0.566899in}}%
\pgfpathlineto{\pgfqpoint{6.439384in}{0.566598in}}%
\pgfpathlineto{\pgfqpoint{6.448193in}{0.566303in}}%
\pgfpathlineto{\pgfqpoint{6.457002in}{0.566014in}}%
\pgfpathlineto{\pgfqpoint{6.465811in}{0.565730in}}%
\pgfpathlineto{\pgfqpoint{6.474621in}{0.565452in}}%
\pgfpathlineto{\pgfqpoint{6.483430in}{0.565178in}}%
\pgfpathlineto{\pgfqpoint{6.492239in}{0.564910in}}%
\pgfpathlineto{\pgfqpoint{6.501048in}{0.564646in}}%
\pgfpathlineto{\pgfqpoint{6.509858in}{0.564387in}}%
\pgfpathlineto{\pgfqpoint{6.518667in}{0.564133in}}%
\pgfpathlineto{\pgfqpoint{6.527476in}{0.563883in}}%
\pgfpathlineto{\pgfqpoint{6.536285in}{0.563638in}}%
\pgfpathlineto{\pgfqpoint{6.545095in}{0.563396in}}%
\pgfpathlineto{\pgfqpoint{6.553904in}{0.563159in}}%
\pgfpathlineto{\pgfqpoint{6.562713in}{0.562925in}}%
\pgfpathlineto{\pgfqpoint{6.571523in}{0.562695in}}%
\pgfpathlineto{\pgfqpoint{6.580332in}{0.562469in}}%
\pgfpathlineto{\pgfqpoint{6.589141in}{0.562247in}}%
\pgfpathlineto{\pgfqpoint{6.597950in}{0.562028in}}%
\pgfpathlineto{\pgfqpoint{6.606760in}{0.561812in}}%
\pgfpathlineto{\pgfqpoint{6.615569in}{0.561600in}}%
\pgfpathlineto{\pgfqpoint{6.624378in}{0.561391in}}%
\pgfpathlineto{\pgfqpoint{6.633187in}{0.561185in}}%
\pgfpathlineto{\pgfqpoint{6.641997in}{0.560982in}}%
\pgfpathlineto{\pgfqpoint{6.650806in}{0.560782in}}%
\pgfpathlineto{\pgfqpoint{6.659615in}{0.560585in}}%
\pgfpathlineto{\pgfqpoint{6.668425in}{0.560391in}}%
\pgfpathlineto{\pgfqpoint{6.677234in}{0.560200in}}%
\pgfpathlineto{\pgfqpoint{6.686043in}{0.560011in}}%
\pgfpathlineto{\pgfqpoint{6.694852in}{0.559825in}}%
\pgfpathlineto{\pgfqpoint{6.703662in}{0.559642in}}%
\pgfpathlineto{\pgfqpoint{6.712471in}{0.559461in}}%
\pgfpathlineto{\pgfqpoint{6.721280in}{0.559282in}}%
\pgfpathlineto{\pgfqpoint{6.730089in}{0.559106in}}%
\pgfpathlineto{\pgfqpoint{6.738899in}{0.558933in}}%
\pgfpathlineto{\pgfqpoint{6.747708in}{0.558761in}}%
\pgfpathlineto{\pgfqpoint{6.756517in}{0.558592in}}%
\pgfpathlineto{\pgfqpoint{6.765326in}{0.558425in}}%
\pgfpathlineto{\pgfqpoint{6.774136in}{0.558260in}}%
\pgfpathlineto{\pgfqpoint{6.782945in}{0.558097in}}%
\pgfpathlineto{\pgfqpoint{6.791754in}{0.557937in}}%
\pgfpathlineto{\pgfqpoint{6.800564in}{0.557778in}}%
\pgfpathlineto{\pgfqpoint{6.809373in}{0.557621in}}%
\pgfpathlineto{\pgfqpoint{6.818182in}{0.557467in}}%
\pgfpathlineto{\pgfqpoint{6.826991in}{0.557314in}}%
\pgfpathlineto{\pgfqpoint{6.835801in}{0.557163in}}%
\pgfpathlineto{\pgfqpoint{6.844610in}{0.557014in}}%
\pgfpathlineto{\pgfqpoint{6.853419in}{0.556866in}}%
\pgfpathlineto{\pgfqpoint{6.862228in}{0.556720in}}%
\pgfpathlineto{\pgfqpoint{6.871038in}{0.556577in}}%
\pgfpathlineto{\pgfqpoint{6.879847in}{0.556434in}}%
\pgfpathlineto{\pgfqpoint{6.888656in}{0.556294in}}%
\pgfpathlineto{\pgfqpoint{6.897465in}{0.556155in}}%
\pgfpathlineto{\pgfqpoint{6.906275in}{0.556017in}}%
\pgfpathlineto{\pgfqpoint{6.915084in}{0.555881in}}%
\pgfpathlineto{\pgfqpoint{6.923893in}{0.555747in}}%
\pgfpathlineto{\pgfqpoint{6.932703in}{0.555614in}}%
\pgfpathlineto{\pgfqpoint{6.941512in}{0.555483in}}%
\pgfpathlineto{\pgfqpoint{6.950321in}{0.555353in}}%
\pgfpathlineto{\pgfqpoint{6.959130in}{0.555224in}}%
\pgfpathlineto{\pgfqpoint{6.967940in}{0.555097in}}%
\pgfpathlineto{\pgfqpoint{6.976749in}{0.554971in}}%
\pgfpathlineto{\pgfqpoint{6.985558in}{0.554847in}}%
\pgfpathlineto{\pgfqpoint{6.994367in}{0.554724in}}%
\pgfpathlineto{\pgfqpoint{7.003177in}{0.554602in}}%
\pgfpathlineto{\pgfqpoint{7.011986in}{0.554482in}}%
\pgfpathlineto{\pgfqpoint{7.020795in}{0.554362in}}%
\pgfpathlineto{\pgfqpoint{7.029605in}{0.554244in}}%
\pgfpathlineto{\pgfqpoint{7.038414in}{0.554127in}}%
\pgfpathlineto{\pgfqpoint{7.047223in}{0.554012in}}%
\pgfpathlineto{\pgfqpoint{7.047223in}{0.128540in}}%
\pgfpathlineto{\pgfqpoint{7.047223in}{0.128540in}}%
\pgfpathlineto{\pgfqpoint{7.038414in}{0.128540in}}%
\pgfpathlineto{\pgfqpoint{7.029605in}{0.128540in}}%
\pgfpathlineto{\pgfqpoint{7.020795in}{0.128540in}}%
\pgfpathlineto{\pgfqpoint{7.011986in}{0.128540in}}%
\pgfpathlineto{\pgfqpoint{7.003177in}{0.128540in}}%
\pgfpathlineto{\pgfqpoint{6.994367in}{0.128540in}}%
\pgfpathlineto{\pgfqpoint{6.985558in}{0.128540in}}%
\pgfpathlineto{\pgfqpoint{6.976749in}{0.128540in}}%
\pgfpathlineto{\pgfqpoint{6.967940in}{0.128540in}}%
\pgfpathlineto{\pgfqpoint{6.959130in}{0.128540in}}%
\pgfpathlineto{\pgfqpoint{6.950321in}{0.128540in}}%
\pgfpathlineto{\pgfqpoint{6.941512in}{0.128540in}}%
\pgfpathlineto{\pgfqpoint{6.932703in}{0.128540in}}%
\pgfpathlineto{\pgfqpoint{6.923893in}{0.128540in}}%
\pgfpathlineto{\pgfqpoint{6.915084in}{0.128540in}}%
\pgfpathlineto{\pgfqpoint{6.906275in}{0.128540in}}%
\pgfpathlineto{\pgfqpoint{6.897465in}{0.128540in}}%
\pgfpathlineto{\pgfqpoint{6.888656in}{0.128540in}}%
\pgfpathlineto{\pgfqpoint{6.879847in}{0.128540in}}%
\pgfpathlineto{\pgfqpoint{6.871038in}{0.128540in}}%
\pgfpathlineto{\pgfqpoint{6.862228in}{0.128540in}}%
\pgfpathlineto{\pgfqpoint{6.853419in}{0.128540in}}%
\pgfpathlineto{\pgfqpoint{6.844610in}{0.128540in}}%
\pgfpathlineto{\pgfqpoint{6.835801in}{0.128540in}}%
\pgfpathlineto{\pgfqpoint{6.826991in}{0.128540in}}%
\pgfpathlineto{\pgfqpoint{6.818182in}{0.128540in}}%
\pgfpathlineto{\pgfqpoint{6.809373in}{0.128540in}}%
\pgfpathlineto{\pgfqpoint{6.800564in}{0.128540in}}%
\pgfpathlineto{\pgfqpoint{6.791754in}{0.128540in}}%
\pgfpathlineto{\pgfqpoint{6.782945in}{0.128540in}}%
\pgfpathlineto{\pgfqpoint{6.774136in}{0.128540in}}%
\pgfpathlineto{\pgfqpoint{6.765326in}{0.128540in}}%
\pgfpathlineto{\pgfqpoint{6.756517in}{0.128540in}}%
\pgfpathlineto{\pgfqpoint{6.747708in}{0.128540in}}%
\pgfpathlineto{\pgfqpoint{6.738899in}{0.128540in}}%
\pgfpathlineto{\pgfqpoint{6.730089in}{0.128540in}}%
\pgfpathlineto{\pgfqpoint{6.721280in}{0.128540in}}%
\pgfpathlineto{\pgfqpoint{6.712471in}{0.128540in}}%
\pgfpathlineto{\pgfqpoint{6.703662in}{0.128540in}}%
\pgfpathlineto{\pgfqpoint{6.694852in}{0.128540in}}%
\pgfpathlineto{\pgfqpoint{6.686043in}{0.128540in}}%
\pgfpathlineto{\pgfqpoint{6.677234in}{0.128540in}}%
\pgfpathlineto{\pgfqpoint{6.668425in}{0.128540in}}%
\pgfpathlineto{\pgfqpoint{6.659615in}{0.128540in}}%
\pgfpathlineto{\pgfqpoint{6.650806in}{0.128540in}}%
\pgfpathlineto{\pgfqpoint{6.641997in}{0.128540in}}%
\pgfpathlineto{\pgfqpoint{6.633187in}{0.128540in}}%
\pgfpathlineto{\pgfqpoint{6.624378in}{0.128540in}}%
\pgfpathlineto{\pgfqpoint{6.615569in}{0.128540in}}%
\pgfpathlineto{\pgfqpoint{6.606760in}{0.128540in}}%
\pgfpathlineto{\pgfqpoint{6.597950in}{0.128540in}}%
\pgfpathlineto{\pgfqpoint{6.589141in}{0.128540in}}%
\pgfpathlineto{\pgfqpoint{6.580332in}{0.128540in}}%
\pgfpathlineto{\pgfqpoint{6.571523in}{0.128540in}}%
\pgfpathlineto{\pgfqpoint{6.562713in}{0.128540in}}%
\pgfpathlineto{\pgfqpoint{6.553904in}{0.128540in}}%
\pgfpathlineto{\pgfqpoint{6.545095in}{0.128540in}}%
\pgfpathlineto{\pgfqpoint{6.536285in}{0.128540in}}%
\pgfpathlineto{\pgfqpoint{6.527476in}{0.128540in}}%
\pgfpathlineto{\pgfqpoint{6.518667in}{0.128540in}}%
\pgfpathlineto{\pgfqpoint{6.509858in}{0.128540in}}%
\pgfpathlineto{\pgfqpoint{6.501048in}{0.128540in}}%
\pgfpathlineto{\pgfqpoint{6.492239in}{0.128540in}}%
\pgfpathlineto{\pgfqpoint{6.483430in}{0.128540in}}%
\pgfpathlineto{\pgfqpoint{6.474621in}{0.128540in}}%
\pgfpathlineto{\pgfqpoint{6.465811in}{0.128540in}}%
\pgfpathlineto{\pgfqpoint{6.457002in}{0.128540in}}%
\pgfpathlineto{\pgfqpoint{6.448193in}{0.128540in}}%
\pgfpathlineto{\pgfqpoint{6.439384in}{0.128540in}}%
\pgfpathlineto{\pgfqpoint{6.430574in}{0.128540in}}%
\pgfpathlineto{\pgfqpoint{6.421765in}{0.128540in}}%
\pgfpathlineto{\pgfqpoint{6.412956in}{0.128540in}}%
\pgfpathlineto{\pgfqpoint{6.404146in}{0.128540in}}%
\pgfpathlineto{\pgfqpoint{6.395337in}{0.128540in}}%
\pgfpathlineto{\pgfqpoint{6.386528in}{0.128540in}}%
\pgfpathlineto{\pgfqpoint{6.377719in}{0.128540in}}%
\pgfpathlineto{\pgfqpoint{6.368909in}{0.128540in}}%
\pgfpathlineto{\pgfqpoint{6.360100in}{0.128540in}}%
\pgfpathlineto{\pgfqpoint{6.351291in}{0.128540in}}%
\pgfpathlineto{\pgfqpoint{6.342482in}{0.128540in}}%
\pgfpathlineto{\pgfqpoint{6.333672in}{0.128540in}}%
\pgfpathlineto{\pgfqpoint{6.324863in}{0.128540in}}%
\pgfpathlineto{\pgfqpoint{6.316054in}{0.128540in}}%
\pgfpathlineto{\pgfqpoint{6.307245in}{0.128540in}}%
\pgfpathlineto{\pgfqpoint{6.298435in}{0.128540in}}%
\pgfpathlineto{\pgfqpoint{6.289626in}{0.128540in}}%
\pgfpathlineto{\pgfqpoint{6.280817in}{0.128540in}}%
\pgfpathlineto{\pgfqpoint{6.272007in}{0.128540in}}%
\pgfpathlineto{\pgfqpoint{6.263198in}{0.128540in}}%
\pgfpathlineto{\pgfqpoint{6.254389in}{0.128540in}}%
\pgfpathlineto{\pgfqpoint{6.245580in}{0.128540in}}%
\pgfpathlineto{\pgfqpoint{6.236770in}{0.128540in}}%
\pgfpathlineto{\pgfqpoint{6.227961in}{0.128540in}}%
\pgfpathlineto{\pgfqpoint{6.219152in}{0.128540in}}%
\pgfpathlineto{\pgfqpoint{6.210343in}{0.128540in}}%
\pgfpathlineto{\pgfqpoint{6.201533in}{0.128540in}}%
\pgfpathlineto{\pgfqpoint{6.192724in}{0.128540in}}%
\pgfpathlineto{\pgfqpoint{6.183915in}{0.128540in}}%
\pgfpathlineto{\pgfqpoint{6.175105in}{0.128540in}}%
\pgfpathlineto{\pgfqpoint{6.175105in}{0.128540in}}%
\pgfpathlineto{\pgfqpoint{6.169772in}{0.128540in}}%
\pgfpathlineto{\pgfqpoint{6.164439in}{0.128540in}}%
\pgfpathlineto{\pgfqpoint{6.159106in}{0.128540in}}%
\pgfpathlineto{\pgfqpoint{6.153773in}{0.128540in}}%
\pgfpathlineto{\pgfqpoint{6.148439in}{0.128540in}}%
\pgfpathlineto{\pgfqpoint{6.143106in}{0.128540in}}%
\pgfpathlineto{\pgfqpoint{6.137773in}{0.128540in}}%
\pgfpathlineto{\pgfqpoint{6.132440in}{0.128540in}}%
\pgfpathlineto{\pgfqpoint{6.127107in}{0.128540in}}%
\pgfpathlineto{\pgfqpoint{6.121773in}{0.128540in}}%
\pgfpathlineto{\pgfqpoint{6.116440in}{0.128540in}}%
\pgfpathlineto{\pgfqpoint{6.111107in}{0.128540in}}%
\pgfpathlineto{\pgfqpoint{6.105774in}{0.128540in}}%
\pgfpathlineto{\pgfqpoint{6.100440in}{0.128540in}}%
\pgfpathlineto{\pgfqpoint{6.095107in}{0.128540in}}%
\pgfpathlineto{\pgfqpoint{6.089774in}{0.128540in}}%
\pgfpathlineto{\pgfqpoint{6.084441in}{0.128540in}}%
\pgfpathlineto{\pgfqpoint{6.079108in}{0.128540in}}%
\pgfpathlineto{\pgfqpoint{6.073774in}{0.128540in}}%
\pgfpathlineto{\pgfqpoint{6.068441in}{0.128540in}}%
\pgfpathlineto{\pgfqpoint{6.063108in}{0.128540in}}%
\pgfpathlineto{\pgfqpoint{6.057775in}{0.128540in}}%
\pgfpathlineto{\pgfqpoint{6.052442in}{0.128540in}}%
\pgfpathlineto{\pgfqpoint{6.047108in}{0.128540in}}%
\pgfpathlineto{\pgfqpoint{6.041775in}{0.128540in}}%
\pgfpathlineto{\pgfqpoint{6.036442in}{0.128540in}}%
\pgfpathlineto{\pgfqpoint{6.031109in}{0.128540in}}%
\pgfpathlineto{\pgfqpoint{6.025775in}{0.128540in}}%
\pgfpathlineto{\pgfqpoint{6.020442in}{0.128540in}}%
\pgfpathlineto{\pgfqpoint{6.015109in}{0.128540in}}%
\pgfpathlineto{\pgfqpoint{6.009776in}{0.128540in}}%
\pgfpathlineto{\pgfqpoint{6.004443in}{0.128540in}}%
\pgfpathlineto{\pgfqpoint{5.999109in}{0.128540in}}%
\pgfpathlineto{\pgfqpoint{5.993776in}{0.128540in}}%
\pgfpathlineto{\pgfqpoint{5.988443in}{0.128540in}}%
\pgfpathlineto{\pgfqpoint{5.983110in}{0.128540in}}%
\pgfpathlineto{\pgfqpoint{5.977777in}{0.128540in}}%
\pgfpathlineto{\pgfqpoint{5.972443in}{0.128540in}}%
\pgfpathlineto{\pgfqpoint{5.967110in}{0.128540in}}%
\pgfpathlineto{\pgfqpoint{5.961777in}{0.128540in}}%
\pgfpathlineto{\pgfqpoint{5.956444in}{0.128540in}}%
\pgfpathlineto{\pgfqpoint{5.951110in}{0.128540in}}%
\pgfpathlineto{\pgfqpoint{5.945777in}{0.128540in}}%
\pgfpathlineto{\pgfqpoint{5.940444in}{0.128540in}}%
\pgfpathlineto{\pgfqpoint{5.935111in}{0.128540in}}%
\pgfpathlineto{\pgfqpoint{5.929778in}{0.128540in}}%
\pgfpathlineto{\pgfqpoint{5.924444in}{0.128540in}}%
\pgfpathlineto{\pgfqpoint{5.919111in}{0.128540in}}%
\pgfpathlineto{\pgfqpoint{5.913778in}{0.128540in}}%
\pgfpathlineto{\pgfqpoint{5.908445in}{0.128540in}}%
\pgfpathlineto{\pgfqpoint{5.903112in}{0.128540in}}%
\pgfpathlineto{\pgfqpoint{5.897778in}{0.128540in}}%
\pgfpathlineto{\pgfqpoint{5.892445in}{0.128540in}}%
\pgfpathlineto{\pgfqpoint{5.887112in}{0.128540in}}%
\pgfpathlineto{\pgfqpoint{5.881779in}{0.128540in}}%
\pgfpathlineto{\pgfqpoint{5.876445in}{0.128540in}}%
\pgfpathlineto{\pgfqpoint{5.871112in}{0.128540in}}%
\pgfpathlineto{\pgfqpoint{5.865779in}{0.128540in}}%
\pgfpathlineto{\pgfqpoint{5.860446in}{0.128540in}}%
\pgfpathlineto{\pgfqpoint{5.855113in}{0.128540in}}%
\pgfpathlineto{\pgfqpoint{5.849779in}{0.128540in}}%
\pgfpathlineto{\pgfqpoint{5.844446in}{0.128540in}}%
\pgfpathlineto{\pgfqpoint{5.839113in}{0.128540in}}%
\pgfpathlineto{\pgfqpoint{5.833780in}{0.128540in}}%
\pgfpathlineto{\pgfqpoint{5.828446in}{0.128540in}}%
\pgfpathlineto{\pgfqpoint{5.823113in}{0.128540in}}%
\pgfpathlineto{\pgfqpoint{5.817780in}{0.128540in}}%
\pgfpathlineto{\pgfqpoint{5.812447in}{0.128540in}}%
\pgfpathlineto{\pgfqpoint{5.807114in}{0.128540in}}%
\pgfpathlineto{\pgfqpoint{5.801780in}{0.128540in}}%
\pgfpathlineto{\pgfqpoint{5.796447in}{0.128540in}}%
\pgfpathlineto{\pgfqpoint{5.791114in}{0.128540in}}%
\pgfpathlineto{\pgfqpoint{5.785781in}{0.128540in}}%
\pgfpathlineto{\pgfqpoint{5.780448in}{0.128540in}}%
\pgfpathlineto{\pgfqpoint{5.775114in}{0.128540in}}%
\pgfpathlineto{\pgfqpoint{5.769781in}{0.128540in}}%
\pgfpathlineto{\pgfqpoint{5.764448in}{0.128540in}}%
\pgfpathlineto{\pgfqpoint{5.759115in}{0.128540in}}%
\pgfpathlineto{\pgfqpoint{5.753781in}{0.128540in}}%
\pgfpathlineto{\pgfqpoint{5.748448in}{0.128540in}}%
\pgfpathlineto{\pgfqpoint{5.743115in}{0.128540in}}%
\pgfpathlineto{\pgfqpoint{5.737782in}{0.128540in}}%
\pgfpathlineto{\pgfqpoint{5.732449in}{0.128540in}}%
\pgfpathlineto{\pgfqpoint{5.727115in}{0.128540in}}%
\pgfpathlineto{\pgfqpoint{5.721782in}{0.128540in}}%
\pgfpathlineto{\pgfqpoint{5.716449in}{0.128540in}}%
\pgfpathlineto{\pgfqpoint{5.711116in}{0.128540in}}%
\pgfpathlineto{\pgfqpoint{5.705783in}{0.128540in}}%
\pgfpathlineto{\pgfqpoint{5.700449in}{0.128540in}}%
\pgfpathlineto{\pgfqpoint{5.695116in}{0.128540in}}%
\pgfpathlineto{\pgfqpoint{5.689783in}{0.128540in}}%
\pgfpathlineto{\pgfqpoint{5.684450in}{0.128540in}}%
\pgfpathlineto{\pgfqpoint{5.679116in}{0.128540in}}%
\pgfpathlineto{\pgfqpoint{5.673783in}{0.128540in}}%
\pgfpathlineto{\pgfqpoint{5.668450in}{0.128540in}}%
\pgfpathlineto{\pgfqpoint{5.663117in}{0.128540in}}%
\pgfpathlineto{\pgfqpoint{5.657784in}{0.128540in}}%
\pgfpathlineto{\pgfqpoint{5.652450in}{0.128540in}}%
\pgfpathlineto{\pgfqpoint{5.647117in}{0.128540in}}%
\pgfpathlineto{\pgfqpoint{5.641784in}{0.128540in}}%
\pgfpathlineto{\pgfqpoint{5.636451in}{0.128540in}}%
\pgfpathlineto{\pgfqpoint{5.631118in}{0.128540in}}%
\pgfpathlineto{\pgfqpoint{5.625784in}{0.128540in}}%
\pgfpathlineto{\pgfqpoint{5.620451in}{0.128540in}}%
\pgfpathlineto{\pgfqpoint{5.615118in}{0.128540in}}%
\pgfpathlineto{\pgfqpoint{5.609785in}{0.128540in}}%
\pgfpathlineto{\pgfqpoint{5.604451in}{0.128540in}}%
\pgfpathlineto{\pgfqpoint{5.599118in}{0.128540in}}%
\pgfpathlineto{\pgfqpoint{5.593785in}{0.128540in}}%
\pgfpathlineto{\pgfqpoint{5.588452in}{0.128540in}}%
\pgfpathlineto{\pgfqpoint{5.583119in}{0.128540in}}%
\pgfpathlineto{\pgfqpoint{5.577785in}{0.128540in}}%
\pgfpathlineto{\pgfqpoint{5.572452in}{0.128540in}}%
\pgfpathlineto{\pgfqpoint{5.567119in}{0.128540in}}%
\pgfpathlineto{\pgfqpoint{5.561786in}{0.128540in}}%
\pgfpathlineto{\pgfqpoint{5.556452in}{0.128540in}}%
\pgfpathlineto{\pgfqpoint{5.551119in}{0.128540in}}%
\pgfpathlineto{\pgfqpoint{5.545786in}{0.128540in}}%
\pgfpathlineto{\pgfqpoint{5.540453in}{0.128540in}}%
\pgfpathlineto{\pgfqpoint{5.535120in}{0.128540in}}%
\pgfpathlineto{\pgfqpoint{5.529786in}{0.128540in}}%
\pgfpathlineto{\pgfqpoint{5.524453in}{0.128540in}}%
\pgfpathlineto{\pgfqpoint{5.519120in}{0.128540in}}%
\pgfpathlineto{\pgfqpoint{5.513787in}{0.128540in}}%
\pgfpathlineto{\pgfqpoint{5.508454in}{0.128540in}}%
\pgfpathlineto{\pgfqpoint{5.503120in}{0.128540in}}%
\pgfpathlineto{\pgfqpoint{5.497787in}{0.128540in}}%
\pgfpathlineto{\pgfqpoint{5.492454in}{0.128540in}}%
\pgfpathlineto{\pgfqpoint{5.487121in}{0.128540in}}%
\pgfpathlineto{\pgfqpoint{5.481787in}{0.128540in}}%
\pgfpathlineto{\pgfqpoint{5.476454in}{0.128540in}}%
\pgfpathlineto{\pgfqpoint{5.471121in}{0.128540in}}%
\pgfpathlineto{\pgfqpoint{5.465788in}{0.128540in}}%
\pgfpathlineto{\pgfqpoint{5.460455in}{0.128540in}}%
\pgfpathlineto{\pgfqpoint{5.455121in}{0.128540in}}%
\pgfpathlineto{\pgfqpoint{5.449788in}{0.128540in}}%
\pgfpathlineto{\pgfqpoint{5.444455in}{0.128540in}}%
\pgfpathlineto{\pgfqpoint{5.439122in}{0.128540in}}%
\pgfpathlineto{\pgfqpoint{5.433789in}{0.128540in}}%
\pgfpathlineto{\pgfqpoint{5.428455in}{0.128540in}}%
\pgfpathlineto{\pgfqpoint{5.423122in}{0.128540in}}%
\pgfpathlineto{\pgfqpoint{5.417789in}{0.128540in}}%
\pgfpathlineto{\pgfqpoint{5.412456in}{0.128540in}}%
\pgfpathlineto{\pgfqpoint{5.407122in}{0.128540in}}%
\pgfpathlineto{\pgfqpoint{5.401789in}{0.128540in}}%
\pgfpathlineto{\pgfqpoint{5.396456in}{0.128540in}}%
\pgfpathlineto{\pgfqpoint{5.391123in}{0.128540in}}%
\pgfpathlineto{\pgfqpoint{5.385790in}{0.128540in}}%
\pgfpathlineto{\pgfqpoint{5.380456in}{0.128540in}}%
\pgfpathlineto{\pgfqpoint{5.375123in}{0.128540in}}%
\pgfpathlineto{\pgfqpoint{5.369790in}{0.128540in}}%
\pgfpathlineto{\pgfqpoint{5.364457in}{0.128540in}}%
\pgfpathlineto{\pgfqpoint{5.359124in}{0.128540in}}%
\pgfpathlineto{\pgfqpoint{5.353790in}{0.128540in}}%
\pgfpathlineto{\pgfqpoint{5.348457in}{0.128540in}}%
\pgfpathlineto{\pgfqpoint{5.343124in}{0.128540in}}%
\pgfpathlineto{\pgfqpoint{5.337791in}{0.128540in}}%
\pgfpathlineto{\pgfqpoint{5.332457in}{0.128540in}}%
\pgfpathlineto{\pgfqpoint{5.327124in}{0.128540in}}%
\pgfpathlineto{\pgfqpoint{5.321791in}{0.128540in}}%
\pgfpathlineto{\pgfqpoint{5.316458in}{0.128540in}}%
\pgfpathlineto{\pgfqpoint{5.311125in}{0.128540in}}%
\pgfpathlineto{\pgfqpoint{5.305791in}{0.128540in}}%
\pgfpathlineto{\pgfqpoint{5.300458in}{0.128540in}}%
\pgfpathlineto{\pgfqpoint{5.295125in}{0.128540in}}%
\pgfpathlineto{\pgfqpoint{5.289792in}{0.128540in}}%
\pgfpathlineto{\pgfqpoint{5.284458in}{0.128540in}}%
\pgfpathlineto{\pgfqpoint{5.279125in}{0.128540in}}%
\pgfpathlineto{\pgfqpoint{5.273792in}{0.128540in}}%
\pgfpathlineto{\pgfqpoint{5.268459in}{0.128540in}}%
\pgfpathlineto{\pgfqpoint{5.263126in}{0.128540in}}%
\pgfpathlineto{\pgfqpoint{5.257792in}{0.128540in}}%
\pgfpathlineto{\pgfqpoint{5.252459in}{0.128540in}}%
\pgfpathlineto{\pgfqpoint{5.247126in}{0.128540in}}%
\pgfpathlineto{\pgfqpoint{5.241793in}{0.128540in}}%
\pgfpathlineto{\pgfqpoint{5.236460in}{0.128540in}}%
\pgfpathlineto{\pgfqpoint{5.231126in}{0.128540in}}%
\pgfpathlineto{\pgfqpoint{5.225793in}{0.128540in}}%
\pgfpathlineto{\pgfqpoint{5.220460in}{0.128540in}}%
\pgfpathlineto{\pgfqpoint{5.215127in}{0.128540in}}%
\pgfpathlineto{\pgfqpoint{5.209793in}{0.128540in}}%
\pgfpathlineto{\pgfqpoint{5.204460in}{0.128540in}}%
\pgfpathlineto{\pgfqpoint{5.199127in}{0.128540in}}%
\pgfpathlineto{\pgfqpoint{5.193794in}{0.128540in}}%
\pgfpathlineto{\pgfqpoint{5.188461in}{0.128540in}}%
\pgfpathlineto{\pgfqpoint{5.183127in}{0.128540in}}%
\pgfpathlineto{\pgfqpoint{5.177794in}{0.128540in}}%
\pgfpathlineto{\pgfqpoint{5.172461in}{0.128540in}}%
\pgfpathlineto{\pgfqpoint{5.167128in}{0.128540in}}%
\pgfpathlineto{\pgfqpoint{5.161795in}{0.128540in}}%
\pgfpathlineto{\pgfqpoint{5.156461in}{0.128540in}}%
\pgfpathlineto{\pgfqpoint{5.151128in}{0.128540in}}%
\pgfpathlineto{\pgfqpoint{5.145795in}{0.128540in}}%
\pgfpathlineto{\pgfqpoint{5.140462in}{0.128540in}}%
\pgfpathlineto{\pgfqpoint{5.135128in}{0.128540in}}%
\pgfpathlineto{\pgfqpoint{5.129795in}{0.128540in}}%
\pgfpathlineto{\pgfqpoint{5.124462in}{0.128540in}}%
\pgfpathlineto{\pgfqpoint{5.119129in}{0.128540in}}%
\pgfpathlineto{\pgfqpoint{5.113796in}{0.128540in}}%
\pgfpathlineto{\pgfqpoint{5.108462in}{0.128540in}}%
\pgfpathlineto{\pgfqpoint{5.103129in}{0.128540in}}%
\pgfpathlineto{\pgfqpoint{5.097796in}{0.128540in}}%
\pgfpathlineto{\pgfqpoint{5.092463in}{0.128540in}}%
\pgfpathlineto{\pgfqpoint{5.087130in}{0.128540in}}%
\pgfpathlineto{\pgfqpoint{5.081796in}{0.128540in}}%
\pgfpathlineto{\pgfqpoint{5.076463in}{0.128540in}}%
\pgfpathlineto{\pgfqpoint{5.071130in}{0.128540in}}%
\pgfpathlineto{\pgfqpoint{5.065797in}{0.128540in}}%
\pgfpathlineto{\pgfqpoint{5.060463in}{0.128540in}}%
\pgfpathlineto{\pgfqpoint{5.055130in}{0.128540in}}%
\pgfpathlineto{\pgfqpoint{5.049797in}{0.128540in}}%
\pgfpathlineto{\pgfqpoint{5.044464in}{0.128540in}}%
\pgfpathlineto{\pgfqpoint{5.039131in}{0.128540in}}%
\pgfpathlineto{\pgfqpoint{5.033797in}{0.128540in}}%
\pgfpathlineto{\pgfqpoint{5.028464in}{0.128540in}}%
\pgfpathlineto{\pgfqpoint{5.023131in}{0.128540in}}%
\pgfpathlineto{\pgfqpoint{5.017798in}{0.128540in}}%
\pgfpathlineto{\pgfqpoint{5.012464in}{0.128540in}}%
\pgfpathlineto{\pgfqpoint{5.007131in}{0.128540in}}%
\pgfpathlineto{\pgfqpoint{5.001798in}{0.128540in}}%
\pgfpathlineto{\pgfqpoint{4.996465in}{0.128540in}}%
\pgfpathlineto{\pgfqpoint{4.991132in}{0.128540in}}%
\pgfpathlineto{\pgfqpoint{4.985798in}{0.128540in}}%
\pgfpathlineto{\pgfqpoint{4.980465in}{0.128540in}}%
\pgfpathlineto{\pgfqpoint{4.975132in}{0.128540in}}%
\pgfpathlineto{\pgfqpoint{4.969799in}{0.128540in}}%
\pgfpathlineto{\pgfqpoint{4.964466in}{0.128540in}}%
\pgfpathlineto{\pgfqpoint{4.959132in}{0.128540in}}%
\pgfpathlineto{\pgfqpoint{4.953799in}{0.128540in}}%
\pgfpathlineto{\pgfqpoint{4.948466in}{0.128540in}}%
\pgfpathlineto{\pgfqpoint{4.943133in}{0.128540in}}%
\pgfpathlineto{\pgfqpoint{4.937799in}{0.128540in}}%
\pgfpathlineto{\pgfqpoint{4.932466in}{0.128540in}}%
\pgfpathlineto{\pgfqpoint{4.927133in}{0.128540in}}%
\pgfpathlineto{\pgfqpoint{4.921800in}{0.128540in}}%
\pgfpathlineto{\pgfqpoint{4.916467in}{0.128540in}}%
\pgfpathlineto{\pgfqpoint{4.911133in}{0.128540in}}%
\pgfpathlineto{\pgfqpoint{4.905800in}{0.128540in}}%
\pgfpathlineto{\pgfqpoint{4.900467in}{0.128540in}}%
\pgfpathlineto{\pgfqpoint{4.895134in}{0.128540in}}%
\pgfpathlineto{\pgfqpoint{4.889801in}{0.128540in}}%
\pgfpathlineto{\pgfqpoint{4.884467in}{0.128540in}}%
\pgfpathlineto{\pgfqpoint{4.879134in}{0.128540in}}%
\pgfpathlineto{\pgfqpoint{4.873801in}{0.128540in}}%
\pgfpathlineto{\pgfqpoint{4.868468in}{0.128540in}}%
\pgfpathlineto{\pgfqpoint{4.863134in}{0.128540in}}%
\pgfpathlineto{\pgfqpoint{4.857801in}{0.128540in}}%
\pgfpathlineto{\pgfqpoint{4.852468in}{0.128540in}}%
\pgfpathlineto{\pgfqpoint{4.847135in}{0.128540in}}%
\pgfpathlineto{\pgfqpoint{4.841802in}{0.128540in}}%
\pgfpathlineto{\pgfqpoint{4.836468in}{0.128540in}}%
\pgfpathlineto{\pgfqpoint{4.831135in}{0.128540in}}%
\pgfpathlineto{\pgfqpoint{4.825802in}{0.128540in}}%
\pgfpathlineto{\pgfqpoint{4.820469in}{0.128540in}}%
\pgfpathlineto{\pgfqpoint{4.815136in}{0.128540in}}%
\pgfpathlineto{\pgfqpoint{4.809802in}{0.128540in}}%
\pgfpathlineto{\pgfqpoint{4.804469in}{0.128540in}}%
\pgfpathlineto{\pgfqpoint{4.799136in}{0.128540in}}%
\pgfpathlineto{\pgfqpoint{4.793803in}{0.128540in}}%
\pgfpathlineto{\pgfqpoint{4.788469in}{0.128540in}}%
\pgfpathlineto{\pgfqpoint{4.783136in}{0.128540in}}%
\pgfpathlineto{\pgfqpoint{4.777803in}{0.128540in}}%
\pgfpathlineto{\pgfqpoint{4.772470in}{0.128540in}}%
\pgfpathlineto{\pgfqpoint{4.767137in}{0.128540in}}%
\pgfpathlineto{\pgfqpoint{4.761803in}{0.128540in}}%
\pgfpathlineto{\pgfqpoint{4.756470in}{0.128540in}}%
\pgfpathlineto{\pgfqpoint{4.751137in}{0.128540in}}%
\pgfpathlineto{\pgfqpoint{4.745804in}{0.128540in}}%
\pgfpathlineto{\pgfqpoint{4.740470in}{0.128540in}}%
\pgfpathlineto{\pgfqpoint{4.735137in}{0.128540in}}%
\pgfpathlineto{\pgfqpoint{4.729804in}{0.128540in}}%
\pgfpathlineto{\pgfqpoint{4.724471in}{0.128540in}}%
\pgfpathlineto{\pgfqpoint{4.719138in}{0.128540in}}%
\pgfpathlineto{\pgfqpoint{4.713804in}{0.128540in}}%
\pgfpathlineto{\pgfqpoint{4.708471in}{0.128540in}}%
\pgfpathlineto{\pgfqpoint{4.703138in}{0.128540in}}%
\pgfpathlineto{\pgfqpoint{4.697805in}{0.128540in}}%
\pgfpathlineto{\pgfqpoint{4.692472in}{0.128540in}}%
\pgfpathlineto{\pgfqpoint{4.687138in}{0.128540in}}%
\pgfpathlineto{\pgfqpoint{4.681805in}{0.128540in}}%
\pgfpathlineto{\pgfqpoint{4.676472in}{0.128540in}}%
\pgfpathlineto{\pgfqpoint{4.671139in}{0.128540in}}%
\pgfpathlineto{\pgfqpoint{4.665805in}{0.128540in}}%
\pgfpathlineto{\pgfqpoint{4.660472in}{0.128540in}}%
\pgfpathlineto{\pgfqpoint{4.655139in}{0.128540in}}%
\pgfpathlineto{\pgfqpoint{4.649806in}{0.128540in}}%
\pgfpathlineto{\pgfqpoint{4.644473in}{0.128540in}}%
\pgfpathlineto{\pgfqpoint{4.639139in}{0.128540in}}%
\pgfpathlineto{\pgfqpoint{4.633806in}{0.128540in}}%
\pgfpathlineto{\pgfqpoint{4.628473in}{0.128540in}}%
\pgfpathlineto{\pgfqpoint{4.623140in}{0.128540in}}%
\pgfpathlineto{\pgfqpoint{4.617807in}{0.128540in}}%
\pgfpathlineto{\pgfqpoint{4.612473in}{0.128540in}}%
\pgfpathlineto{\pgfqpoint{4.607140in}{0.128540in}}%
\pgfpathlineto{\pgfqpoint{4.601807in}{0.128540in}}%
\pgfpathlineto{\pgfqpoint{4.596474in}{0.128540in}}%
\pgfpathlineto{\pgfqpoint{4.591140in}{0.128540in}}%
\pgfpathlineto{\pgfqpoint{4.585807in}{0.128540in}}%
\pgfpathlineto{\pgfqpoint{4.580474in}{0.128540in}}%
\pgfpathlineto{\pgfqpoint{4.575141in}{0.128540in}}%
\pgfpathlineto{\pgfqpoint{4.569808in}{0.128540in}}%
\pgfpathlineto{\pgfqpoint{4.564474in}{0.128540in}}%
\pgfpathlineto{\pgfqpoint{4.559141in}{0.128540in}}%
\pgfpathlineto{\pgfqpoint{4.553808in}{0.128540in}}%
\pgfpathlineto{\pgfqpoint{4.548475in}{0.128540in}}%
\pgfpathlineto{\pgfqpoint{4.543142in}{0.128540in}}%
\pgfpathlineto{\pgfqpoint{4.537808in}{0.128540in}}%
\pgfpathlineto{\pgfqpoint{4.532475in}{0.128540in}}%
\pgfpathlineto{\pgfqpoint{4.527142in}{0.128540in}}%
\pgfpathlineto{\pgfqpoint{4.521809in}{0.128540in}}%
\pgfpathlineto{\pgfqpoint{4.516475in}{0.128540in}}%
\pgfpathlineto{\pgfqpoint{4.511142in}{0.128540in}}%
\pgfpathlineto{\pgfqpoint{4.505809in}{0.128540in}}%
\pgfpathlineto{\pgfqpoint{4.500476in}{0.128540in}}%
\pgfpathlineto{\pgfqpoint{4.495143in}{0.128540in}}%
\pgfpathlineto{\pgfqpoint{4.489809in}{0.128540in}}%
\pgfpathlineto{\pgfqpoint{4.484476in}{0.128540in}}%
\pgfpathlineto{\pgfqpoint{4.479143in}{0.128540in}}%
\pgfpathlineto{\pgfqpoint{4.473810in}{0.128540in}}%
\pgfpathlineto{\pgfqpoint{4.468476in}{0.128540in}}%
\pgfpathlineto{\pgfqpoint{4.463143in}{0.128540in}}%
\pgfpathlineto{\pgfqpoint{4.457810in}{0.128540in}}%
\pgfpathlineto{\pgfqpoint{4.452477in}{0.128540in}}%
\pgfpathlineto{\pgfqpoint{4.447144in}{0.128540in}}%
\pgfpathlineto{\pgfqpoint{4.441810in}{0.128540in}}%
\pgfpathlineto{\pgfqpoint{4.436477in}{0.128540in}}%
\pgfpathlineto{\pgfqpoint{4.431144in}{0.128540in}}%
\pgfpathlineto{\pgfqpoint{4.425811in}{0.128540in}}%
\pgfpathlineto{\pgfqpoint{4.420478in}{0.128540in}}%
\pgfpathlineto{\pgfqpoint{4.415144in}{0.128540in}}%
\pgfpathlineto{\pgfqpoint{4.409811in}{0.128540in}}%
\pgfpathlineto{\pgfqpoint{4.404478in}{0.128540in}}%
\pgfpathlineto{\pgfqpoint{4.399145in}{0.128540in}}%
\pgfpathlineto{\pgfqpoint{4.393811in}{0.128540in}}%
\pgfpathlineto{\pgfqpoint{4.388478in}{0.128540in}}%
\pgfpathlineto{\pgfqpoint{4.383145in}{0.128540in}}%
\pgfpathlineto{\pgfqpoint{4.377812in}{0.128540in}}%
\pgfpathlineto{\pgfqpoint{4.372479in}{0.128540in}}%
\pgfpathlineto{\pgfqpoint{4.367145in}{0.128540in}}%
\pgfpathlineto{\pgfqpoint{4.361812in}{0.128540in}}%
\pgfpathlineto{\pgfqpoint{4.356479in}{0.128540in}}%
\pgfpathlineto{\pgfqpoint{4.351146in}{0.128540in}}%
\pgfpathlineto{\pgfqpoint{4.345813in}{0.128540in}}%
\pgfpathlineto{\pgfqpoint{4.340479in}{0.128540in}}%
\pgfpathlineto{\pgfqpoint{4.335146in}{0.128540in}}%
\pgfpathlineto{\pgfqpoint{4.329813in}{0.128540in}}%
\pgfpathlineto{\pgfqpoint{4.324480in}{0.128540in}}%
\pgfpathlineto{\pgfqpoint{4.319146in}{0.128540in}}%
\pgfpathlineto{\pgfqpoint{4.313813in}{0.128540in}}%
\pgfpathlineto{\pgfqpoint{4.308480in}{0.128540in}}%
\pgfpathlineto{\pgfqpoint{4.303147in}{0.128540in}}%
\pgfpathlineto{\pgfqpoint{4.297814in}{0.128540in}}%
\pgfpathlineto{\pgfqpoint{4.292480in}{0.128540in}}%
\pgfpathlineto{\pgfqpoint{4.287147in}{0.128540in}}%
\pgfpathlineto{\pgfqpoint{4.281814in}{0.128540in}}%
\pgfpathlineto{\pgfqpoint{4.276481in}{0.128540in}}%
\pgfpathlineto{\pgfqpoint{4.271148in}{0.128540in}}%
\pgfpathlineto{\pgfqpoint{4.265814in}{0.128540in}}%
\pgfpathlineto{\pgfqpoint{4.260481in}{0.128540in}}%
\pgfpathlineto{\pgfqpoint{4.255148in}{0.128540in}}%
\pgfpathlineto{\pgfqpoint{4.249815in}{0.128540in}}%
\pgfpathlineto{\pgfqpoint{4.244481in}{0.128540in}}%
\pgfpathlineto{\pgfqpoint{4.239148in}{0.128540in}}%
\pgfpathlineto{\pgfqpoint{4.233815in}{0.128540in}}%
\pgfpathlineto{\pgfqpoint{4.228482in}{0.128540in}}%
\pgfpathlineto{\pgfqpoint{4.223149in}{0.128540in}}%
\pgfpathlineto{\pgfqpoint{4.217815in}{0.128540in}}%
\pgfpathlineto{\pgfqpoint{4.212482in}{0.128540in}}%
\pgfpathlineto{\pgfqpoint{4.207149in}{0.128540in}}%
\pgfpathlineto{\pgfqpoint{4.201816in}{0.128540in}}%
\pgfpathlineto{\pgfqpoint{4.196482in}{0.128540in}}%
\pgfpathlineto{\pgfqpoint{4.191149in}{0.128540in}}%
\pgfpathlineto{\pgfqpoint{4.185816in}{0.128540in}}%
\pgfpathlineto{\pgfqpoint{4.180483in}{0.128540in}}%
\pgfpathlineto{\pgfqpoint{4.175150in}{0.128540in}}%
\pgfpathlineto{\pgfqpoint{4.169816in}{0.128540in}}%
\pgfpathlineto{\pgfqpoint{4.164483in}{0.128540in}}%
\pgfpathlineto{\pgfqpoint{4.159150in}{0.128540in}}%
\pgfpathlineto{\pgfqpoint{4.153817in}{0.128540in}}%
\pgfpathlineto{\pgfqpoint{4.148484in}{0.128540in}}%
\pgfpathlineto{\pgfqpoint{4.143150in}{0.128540in}}%
\pgfpathlineto{\pgfqpoint{4.137817in}{0.128540in}}%
\pgfpathlineto{\pgfqpoint{4.132484in}{0.128540in}}%
\pgfpathlineto{\pgfqpoint{4.127151in}{0.128540in}}%
\pgfpathlineto{\pgfqpoint{4.121817in}{0.128540in}}%
\pgfpathlineto{\pgfqpoint{4.116484in}{0.128540in}}%
\pgfpathlineto{\pgfqpoint{4.111151in}{0.128540in}}%
\pgfpathlineto{\pgfqpoint{4.105818in}{0.128540in}}%
\pgfpathlineto{\pgfqpoint{4.100485in}{0.128540in}}%
\pgfpathlineto{\pgfqpoint{4.095151in}{0.128540in}}%
\pgfpathlineto{\pgfqpoint{4.089818in}{0.128540in}}%
\pgfpathlineto{\pgfqpoint{4.084485in}{0.128540in}}%
\pgfpathlineto{\pgfqpoint{4.079152in}{0.128540in}}%
\pgfpathlineto{\pgfqpoint{4.073819in}{0.128540in}}%
\pgfpathlineto{\pgfqpoint{4.068485in}{0.128540in}}%
\pgfpathlineto{\pgfqpoint{4.063152in}{0.128540in}}%
\pgfpathlineto{\pgfqpoint{4.057819in}{0.128540in}}%
\pgfpathlineto{\pgfqpoint{4.052486in}{0.128540in}}%
\pgfpathlineto{\pgfqpoint{4.047152in}{0.128540in}}%
\pgfpathlineto{\pgfqpoint{4.041819in}{0.128540in}}%
\pgfpathlineto{\pgfqpoint{4.036486in}{0.128540in}}%
\pgfpathlineto{\pgfqpoint{4.031153in}{0.128540in}}%
\pgfpathlineto{\pgfqpoint{4.025820in}{0.128540in}}%
\pgfpathlineto{\pgfqpoint{4.020486in}{0.128540in}}%
\pgfpathlineto{\pgfqpoint{4.015153in}{0.128540in}}%
\pgfpathlineto{\pgfqpoint{4.009820in}{0.128540in}}%
\pgfpathlineto{\pgfqpoint{4.004487in}{0.128540in}}%
\pgfpathlineto{\pgfqpoint{3.999154in}{0.128540in}}%
\pgfpathlineto{\pgfqpoint{3.993820in}{0.128540in}}%
\pgfpathlineto{\pgfqpoint{3.988487in}{0.128540in}}%
\pgfpathlineto{\pgfqpoint{3.983154in}{0.128540in}}%
\pgfpathlineto{\pgfqpoint{3.977821in}{0.128540in}}%
\pgfpathlineto{\pgfqpoint{3.972487in}{0.128540in}}%
\pgfpathlineto{\pgfqpoint{3.967154in}{0.128540in}}%
\pgfpathlineto{\pgfqpoint{3.961821in}{0.128540in}}%
\pgfpathlineto{\pgfqpoint{3.956488in}{0.128540in}}%
\pgfpathlineto{\pgfqpoint{3.951155in}{0.128540in}}%
\pgfpathlineto{\pgfqpoint{3.945821in}{0.128540in}}%
\pgfpathlineto{\pgfqpoint{3.940488in}{0.128540in}}%
\pgfpathlineto{\pgfqpoint{3.935155in}{0.128540in}}%
\pgfpathlineto{\pgfqpoint{3.929822in}{0.128540in}}%
\pgfpathlineto{\pgfqpoint{3.924488in}{0.128540in}}%
\pgfpathlineto{\pgfqpoint{3.919155in}{0.128540in}}%
\pgfpathlineto{\pgfqpoint{3.913822in}{0.128540in}}%
\pgfpathlineto{\pgfqpoint{3.908489in}{0.128540in}}%
\pgfpathlineto{\pgfqpoint{3.903156in}{0.128540in}}%
\pgfpathlineto{\pgfqpoint{3.897822in}{0.128540in}}%
\pgfpathlineto{\pgfqpoint{3.892489in}{0.128540in}}%
\pgfpathlineto{\pgfqpoint{3.887156in}{0.128540in}}%
\pgfpathlineto{\pgfqpoint{3.881823in}{0.128540in}}%
\pgfpathlineto{\pgfqpoint{3.876490in}{0.128540in}}%
\pgfpathlineto{\pgfqpoint{3.871156in}{0.128540in}}%
\pgfpathlineto{\pgfqpoint{3.865823in}{0.128540in}}%
\pgfpathlineto{\pgfqpoint{3.860490in}{0.128540in}}%
\pgfpathlineto{\pgfqpoint{3.855157in}{0.128540in}}%
\pgfpathlineto{\pgfqpoint{3.849823in}{0.128540in}}%
\pgfpathlineto{\pgfqpoint{3.844490in}{0.128540in}}%
\pgfpathlineto{\pgfqpoint{3.839157in}{0.128540in}}%
\pgfpathlineto{\pgfqpoint{3.833824in}{0.128540in}}%
\pgfpathlineto{\pgfqpoint{3.828491in}{0.128540in}}%
\pgfpathlineto{\pgfqpoint{3.823157in}{0.128540in}}%
\pgfpathlineto{\pgfqpoint{3.817824in}{0.128540in}}%
\pgfpathlineto{\pgfqpoint{3.812491in}{0.128540in}}%
\pgfpathlineto{\pgfqpoint{3.807158in}{0.128540in}}%
\pgfpathlineto{\pgfqpoint{3.801825in}{0.128540in}}%
\pgfpathlineto{\pgfqpoint{3.796491in}{0.128540in}}%
\pgfpathlineto{\pgfqpoint{3.791158in}{0.128540in}}%
\pgfpathlineto{\pgfqpoint{3.785825in}{0.128540in}}%
\pgfpathlineto{\pgfqpoint{3.780492in}{0.128540in}}%
\pgfpathlineto{\pgfqpoint{3.775158in}{0.128540in}}%
\pgfpathlineto{\pgfqpoint{3.769825in}{0.128540in}}%
\pgfpathlineto{\pgfqpoint{3.764492in}{0.128540in}}%
\pgfpathlineto{\pgfqpoint{3.759159in}{0.128540in}}%
\pgfpathlineto{\pgfqpoint{3.753826in}{0.128540in}}%
\pgfpathlineto{\pgfqpoint{3.748492in}{0.128540in}}%
\pgfpathlineto{\pgfqpoint{3.743159in}{0.128540in}}%
\pgfpathlineto{\pgfqpoint{3.737826in}{0.128540in}}%
\pgfpathlineto{\pgfqpoint{3.732493in}{0.128540in}}%
\pgfpathlineto{\pgfqpoint{3.727160in}{0.128540in}}%
\pgfpathlineto{\pgfqpoint{3.721826in}{0.128540in}}%
\pgfpathlineto{\pgfqpoint{3.716493in}{0.128540in}}%
\pgfpathlineto{\pgfqpoint{3.711160in}{0.128540in}}%
\pgfpathlineto{\pgfqpoint{3.705827in}{0.128540in}}%
\pgfpathlineto{\pgfqpoint{3.700493in}{0.128540in}}%
\pgfpathlineto{\pgfqpoint{3.695160in}{0.128540in}}%
\pgfpathlineto{\pgfqpoint{3.689827in}{0.128540in}}%
\pgfpathlineto{\pgfqpoint{3.684494in}{0.128540in}}%
\pgfpathlineto{\pgfqpoint{3.679161in}{0.128540in}}%
\pgfpathlineto{\pgfqpoint{3.673827in}{0.128540in}}%
\pgfpathlineto{\pgfqpoint{3.668494in}{0.128540in}}%
\pgfpathlineto{\pgfqpoint{3.663161in}{0.128540in}}%
\pgfpathlineto{\pgfqpoint{3.657828in}{0.128540in}}%
\pgfpathlineto{\pgfqpoint{3.652494in}{0.128540in}}%
\pgfpathlineto{\pgfqpoint{3.647161in}{0.128540in}}%
\pgfpathlineto{\pgfqpoint{3.641828in}{0.128540in}}%
\pgfpathlineto{\pgfqpoint{3.636495in}{0.128540in}}%
\pgfpathlineto{\pgfqpoint{3.631162in}{0.128540in}}%
\pgfpathlineto{\pgfqpoint{3.625828in}{0.128540in}}%
\pgfpathlineto{\pgfqpoint{3.620495in}{0.128540in}}%
\pgfpathlineto{\pgfqpoint{3.615162in}{0.128540in}}%
\pgfpathlineto{\pgfqpoint{3.609829in}{0.128540in}}%
\pgfpathlineto{\pgfqpoint{3.604496in}{0.128540in}}%
\pgfpathlineto{\pgfqpoint{3.599162in}{0.128540in}}%
\pgfpathlineto{\pgfqpoint{3.593829in}{0.128540in}}%
\pgfpathlineto{\pgfqpoint{3.588496in}{0.128540in}}%
\pgfpathlineto{\pgfqpoint{3.583163in}{0.128540in}}%
\pgfpathlineto{\pgfqpoint{3.577829in}{0.128540in}}%
\pgfpathlineto{\pgfqpoint{3.572496in}{0.128540in}}%
\pgfpathlineto{\pgfqpoint{3.567163in}{0.128540in}}%
\pgfpathlineto{\pgfqpoint{3.561830in}{0.128540in}}%
\pgfpathlineto{\pgfqpoint{3.556497in}{0.128540in}}%
\pgfpathlineto{\pgfqpoint{3.551163in}{0.128540in}}%
\pgfpathlineto{\pgfqpoint{3.545830in}{0.128540in}}%
\pgfpathlineto{\pgfqpoint{3.540497in}{0.128540in}}%
\pgfpathlineto{\pgfqpoint{3.535164in}{0.128540in}}%
\pgfpathlineto{\pgfqpoint{3.529831in}{0.128540in}}%
\pgfpathlineto{\pgfqpoint{3.524497in}{0.128540in}}%
\pgfpathlineto{\pgfqpoint{3.519164in}{0.128540in}}%
\pgfpathlineto{\pgfqpoint{3.513831in}{0.128540in}}%
\pgfpathlineto{\pgfqpoint{3.508498in}{0.128540in}}%
\pgfpathlineto{\pgfqpoint{3.503164in}{0.128540in}}%
\pgfpathlineto{\pgfqpoint{3.497831in}{0.128540in}}%
\pgfpathlineto{\pgfqpoint{3.492498in}{0.128540in}}%
\pgfpathlineto{\pgfqpoint{3.487165in}{0.128540in}}%
\pgfpathlineto{\pgfqpoint{3.481832in}{0.128540in}}%
\pgfpathlineto{\pgfqpoint{3.476498in}{0.128540in}}%
\pgfpathlineto{\pgfqpoint{3.471165in}{0.128540in}}%
\pgfpathlineto{\pgfqpoint{3.465832in}{0.128540in}}%
\pgfpathlineto{\pgfqpoint{3.460499in}{0.128540in}}%
\pgfpathlineto{\pgfqpoint{3.455166in}{0.128540in}}%
\pgfpathlineto{\pgfqpoint{3.449832in}{0.128540in}}%
\pgfpathlineto{\pgfqpoint{3.444499in}{0.128540in}}%
\pgfpathlineto{\pgfqpoint{3.439166in}{0.128540in}}%
\pgfpathlineto{\pgfqpoint{3.433833in}{0.128540in}}%
\pgfpathlineto{\pgfqpoint{3.428499in}{0.128540in}}%
\pgfpathlineto{\pgfqpoint{3.423166in}{0.128540in}}%
\pgfpathlineto{\pgfqpoint{3.417833in}{0.128540in}}%
\pgfpathlineto{\pgfqpoint{3.412500in}{0.128540in}}%
\pgfpathlineto{\pgfqpoint{3.407167in}{0.128540in}}%
\pgfpathlineto{\pgfqpoint{3.401833in}{0.128540in}}%
\pgfpathlineto{\pgfqpoint{3.396500in}{0.128540in}}%
\pgfpathlineto{\pgfqpoint{3.391167in}{0.128540in}}%
\pgfpathlineto{\pgfqpoint{3.385834in}{0.128540in}}%
\pgfpathlineto{\pgfqpoint{3.380500in}{0.128540in}}%
\pgfpathlineto{\pgfqpoint{3.375167in}{0.128540in}}%
\pgfpathlineto{\pgfqpoint{3.369834in}{0.128540in}}%
\pgfpathlineto{\pgfqpoint{3.364501in}{0.128540in}}%
\pgfpathlineto{\pgfqpoint{3.359168in}{0.128540in}}%
\pgfpathlineto{\pgfqpoint{3.353834in}{0.128540in}}%
\pgfpathlineto{\pgfqpoint{3.348501in}{0.128540in}}%
\pgfpathlineto{\pgfqpoint{3.343168in}{0.128540in}}%
\pgfpathlineto{\pgfqpoint{3.337835in}{0.128540in}}%
\pgfpathlineto{\pgfqpoint{3.332502in}{0.128540in}}%
\pgfpathlineto{\pgfqpoint{3.327168in}{0.128540in}}%
\pgfpathlineto{\pgfqpoint{3.321835in}{0.128540in}}%
\pgfpathlineto{\pgfqpoint{3.316502in}{0.128540in}}%
\pgfpathlineto{\pgfqpoint{3.311169in}{0.128540in}}%
\pgfpathlineto{\pgfqpoint{3.305835in}{0.128540in}}%
\pgfpathlineto{\pgfqpoint{3.300502in}{0.128540in}}%
\pgfpathlineto{\pgfqpoint{3.295169in}{0.128540in}}%
\pgfpathlineto{\pgfqpoint{3.289836in}{0.128540in}}%
\pgfpathlineto{\pgfqpoint{3.284503in}{0.128540in}}%
\pgfpathlineto{\pgfqpoint{3.279169in}{0.128540in}}%
\pgfpathlineto{\pgfqpoint{3.273836in}{0.128540in}}%
\pgfpathlineto{\pgfqpoint{3.268503in}{0.128540in}}%
\pgfpathlineto{\pgfqpoint{3.263170in}{0.128540in}}%
\pgfpathlineto{\pgfqpoint{3.257837in}{0.128540in}}%
\pgfpathlineto{\pgfqpoint{3.252503in}{0.128540in}}%
\pgfpathlineto{\pgfqpoint{3.247170in}{0.128540in}}%
\pgfpathlineto{\pgfqpoint{3.241837in}{0.128540in}}%
\pgfpathlineto{\pgfqpoint{3.236504in}{0.128540in}}%
\pgfpathlineto{\pgfqpoint{3.231170in}{0.128540in}}%
\pgfpathlineto{\pgfqpoint{3.225837in}{0.128540in}}%
\pgfpathlineto{\pgfqpoint{3.220504in}{0.128540in}}%
\pgfpathlineto{\pgfqpoint{3.215171in}{0.128540in}}%
\pgfpathlineto{\pgfqpoint{3.209838in}{0.128540in}}%
\pgfpathlineto{\pgfqpoint{3.204504in}{0.128540in}}%
\pgfpathlineto{\pgfqpoint{3.199171in}{0.128540in}}%
\pgfpathlineto{\pgfqpoint{3.193838in}{0.128540in}}%
\pgfpathlineto{\pgfqpoint{3.188505in}{0.128540in}}%
\pgfpathlineto{\pgfqpoint{3.183172in}{0.128540in}}%
\pgfpathlineto{\pgfqpoint{3.177838in}{0.128540in}}%
\pgfpathlineto{\pgfqpoint{3.172505in}{0.128540in}}%
\pgfpathlineto{\pgfqpoint{3.167172in}{0.128540in}}%
\pgfpathlineto{\pgfqpoint{3.161839in}{0.128540in}}%
\pgfpathlineto{\pgfqpoint{3.156505in}{0.128540in}}%
\pgfpathlineto{\pgfqpoint{3.151172in}{0.128540in}}%
\pgfpathlineto{\pgfqpoint{3.145839in}{0.128540in}}%
\pgfpathlineto{\pgfqpoint{3.140506in}{0.128540in}}%
\pgfpathlineto{\pgfqpoint{3.135173in}{0.128540in}}%
\pgfpathlineto{\pgfqpoint{3.129839in}{0.128540in}}%
\pgfpathlineto{\pgfqpoint{3.124506in}{0.128540in}}%
\pgfpathlineto{\pgfqpoint{3.119173in}{0.128540in}}%
\pgfpathlineto{\pgfqpoint{3.113840in}{0.128540in}}%
\pgfpathlineto{\pgfqpoint{3.108506in}{0.128540in}}%
\pgfpathlineto{\pgfqpoint{3.103173in}{0.128540in}}%
\pgfpathlineto{\pgfqpoint{3.097840in}{0.128540in}}%
\pgfpathlineto{\pgfqpoint{3.092507in}{0.128540in}}%
\pgfpathlineto{\pgfqpoint{3.087174in}{0.128540in}}%
\pgfpathlineto{\pgfqpoint{3.081840in}{0.128540in}}%
\pgfpathlineto{\pgfqpoint{3.076507in}{0.128540in}}%
\pgfpathlineto{\pgfqpoint{3.071174in}{0.128540in}}%
\pgfpathlineto{\pgfqpoint{3.065841in}{0.128540in}}%
\pgfpathlineto{\pgfqpoint{3.060508in}{0.128540in}}%
\pgfpathlineto{\pgfqpoint{3.055174in}{0.128540in}}%
\pgfpathlineto{\pgfqpoint{3.049841in}{0.128540in}}%
\pgfpathlineto{\pgfqpoint{3.044508in}{0.128540in}}%
\pgfpathlineto{\pgfqpoint{3.039175in}{0.128540in}}%
\pgfpathlineto{\pgfqpoint{3.033841in}{0.128540in}}%
\pgfpathlineto{\pgfqpoint{3.028508in}{0.128540in}}%
\pgfpathlineto{\pgfqpoint{3.023175in}{0.128540in}}%
\pgfpathlineto{\pgfqpoint{3.017842in}{0.128540in}}%
\pgfpathlineto{\pgfqpoint{3.012509in}{0.128540in}}%
\pgfpathlineto{\pgfqpoint{3.007175in}{0.128540in}}%
\pgfpathlineto{\pgfqpoint{3.001842in}{0.128540in}}%
\pgfpathlineto{\pgfqpoint{2.996509in}{0.128540in}}%
\pgfpathlineto{\pgfqpoint{2.991176in}{0.128540in}}%
\pgfpathlineto{\pgfqpoint{2.985843in}{0.128540in}}%
\pgfpathlineto{\pgfqpoint{2.980509in}{0.128540in}}%
\pgfpathlineto{\pgfqpoint{2.975176in}{0.128540in}}%
\pgfpathlineto{\pgfqpoint{2.969843in}{0.128540in}}%
\pgfpathlineto{\pgfqpoint{2.964510in}{0.128540in}}%
\pgfpathlineto{\pgfqpoint{2.959176in}{0.128540in}}%
\pgfpathlineto{\pgfqpoint{2.953843in}{0.128540in}}%
\pgfpathlineto{\pgfqpoint{2.948510in}{0.128540in}}%
\pgfpathlineto{\pgfqpoint{2.943177in}{0.128540in}}%
\pgfpathlineto{\pgfqpoint{2.937844in}{0.128540in}}%
\pgfpathlineto{\pgfqpoint{2.932510in}{0.128540in}}%
\pgfpathlineto{\pgfqpoint{2.927177in}{0.128540in}}%
\pgfpathlineto{\pgfqpoint{2.921844in}{0.128540in}}%
\pgfpathlineto{\pgfqpoint{2.916511in}{0.128540in}}%
\pgfpathlineto{\pgfqpoint{2.911178in}{0.128540in}}%
\pgfpathlineto{\pgfqpoint{2.905844in}{0.128540in}}%
\pgfpathlineto{\pgfqpoint{2.900511in}{0.128540in}}%
\pgfpathlineto{\pgfqpoint{2.895178in}{0.128540in}}%
\pgfpathlineto{\pgfqpoint{2.889845in}{0.128540in}}%
\pgfpathlineto{\pgfqpoint{2.884511in}{0.128540in}}%
\pgfpathlineto{\pgfqpoint{2.879178in}{0.128540in}}%
\pgfpathlineto{\pgfqpoint{2.873845in}{0.128540in}}%
\pgfpathlineto{\pgfqpoint{2.868512in}{0.128540in}}%
\pgfpathlineto{\pgfqpoint{2.863179in}{0.128540in}}%
\pgfpathlineto{\pgfqpoint{2.857845in}{0.128540in}}%
\pgfpathlineto{\pgfqpoint{2.852512in}{0.128540in}}%
\pgfpathlineto{\pgfqpoint{2.847179in}{0.128540in}}%
\pgfpathlineto{\pgfqpoint{2.841846in}{0.128540in}}%
\pgfpathlineto{\pgfqpoint{2.836512in}{0.128540in}}%
\pgfpathlineto{\pgfqpoint{2.831179in}{0.128540in}}%
\pgfpathlineto{\pgfqpoint{2.825846in}{0.128540in}}%
\pgfpathlineto{\pgfqpoint{2.820513in}{0.128540in}}%
\pgfpathlineto{\pgfqpoint{2.815180in}{0.128540in}}%
\pgfpathlineto{\pgfqpoint{2.809846in}{0.128540in}}%
\pgfpathlineto{\pgfqpoint{2.804513in}{0.128540in}}%
\pgfpathlineto{\pgfqpoint{2.799180in}{0.128540in}}%
\pgfpathlineto{\pgfqpoint{2.793847in}{0.128540in}}%
\pgfpathlineto{\pgfqpoint{2.788514in}{0.128540in}}%
\pgfpathlineto{\pgfqpoint{2.783180in}{0.128540in}}%
\pgfpathlineto{\pgfqpoint{2.777847in}{0.128540in}}%
\pgfpathlineto{\pgfqpoint{2.772514in}{0.128540in}}%
\pgfpathlineto{\pgfqpoint{2.767181in}{0.128540in}}%
\pgfpathlineto{\pgfqpoint{2.761847in}{0.128540in}}%
\pgfpathlineto{\pgfqpoint{2.756514in}{0.128540in}}%
\pgfpathlineto{\pgfqpoint{2.751181in}{0.128540in}}%
\pgfpathlineto{\pgfqpoint{2.745848in}{0.128540in}}%
\pgfpathlineto{\pgfqpoint{2.740515in}{0.128540in}}%
\pgfpathlineto{\pgfqpoint{2.735181in}{0.128540in}}%
\pgfpathlineto{\pgfqpoint{2.729848in}{0.128540in}}%
\pgfpathlineto{\pgfqpoint{2.724515in}{0.128540in}}%
\pgfpathlineto{\pgfqpoint{2.719182in}{0.128540in}}%
\pgfpathlineto{\pgfqpoint{2.713849in}{0.128540in}}%
\pgfpathlineto{\pgfqpoint{2.708515in}{0.128540in}}%
\pgfpathlineto{\pgfqpoint{2.703182in}{0.128540in}}%
\pgfpathlineto{\pgfqpoint{2.697849in}{0.128540in}}%
\pgfpathlineto{\pgfqpoint{2.692516in}{0.128540in}}%
\pgfpathlineto{\pgfqpoint{2.687182in}{0.128540in}}%
\pgfpathlineto{\pgfqpoint{2.681849in}{0.128540in}}%
\pgfpathlineto{\pgfqpoint{2.676516in}{0.128540in}}%
\pgfpathlineto{\pgfqpoint{2.671183in}{0.128540in}}%
\pgfpathlineto{\pgfqpoint{2.665850in}{0.128540in}}%
\pgfpathlineto{\pgfqpoint{2.660516in}{0.128540in}}%
\pgfpathlineto{\pgfqpoint{2.655183in}{0.128540in}}%
\pgfpathlineto{\pgfqpoint{2.649850in}{0.128540in}}%
\pgfpathlineto{\pgfqpoint{2.644517in}{0.128540in}}%
\pgfpathlineto{\pgfqpoint{2.639184in}{0.128540in}}%
\pgfpathlineto{\pgfqpoint{2.633850in}{0.128540in}}%
\pgfpathlineto{\pgfqpoint{2.628517in}{0.128540in}}%
\pgfpathlineto{\pgfqpoint{2.623184in}{0.128540in}}%
\pgfpathlineto{\pgfqpoint{2.617851in}{0.128540in}}%
\pgfpathlineto{\pgfqpoint{2.612517in}{0.128540in}}%
\pgfpathlineto{\pgfqpoint{2.607184in}{0.128540in}}%
\pgfpathlineto{\pgfqpoint{2.601851in}{0.128540in}}%
\pgfpathlineto{\pgfqpoint{2.596518in}{0.128540in}}%
\pgfpathlineto{\pgfqpoint{2.591185in}{0.128540in}}%
\pgfpathlineto{\pgfqpoint{2.585851in}{0.128540in}}%
\pgfpathlineto{\pgfqpoint{2.580518in}{0.128540in}}%
\pgfpathlineto{\pgfqpoint{2.575185in}{0.128540in}}%
\pgfpathlineto{\pgfqpoint{2.569852in}{0.128540in}}%
\pgfpathlineto{\pgfqpoint{2.564518in}{0.128540in}}%
\pgfpathlineto{\pgfqpoint{2.559185in}{0.128540in}}%
\pgfpathlineto{\pgfqpoint{2.553852in}{0.128540in}}%
\pgfpathlineto{\pgfqpoint{2.548519in}{0.128540in}}%
\pgfpathlineto{\pgfqpoint{2.543186in}{0.128540in}}%
\pgfpathlineto{\pgfqpoint{2.537852in}{0.128540in}}%
\pgfpathlineto{\pgfqpoint{2.532519in}{0.128540in}}%
\pgfpathlineto{\pgfqpoint{2.527186in}{0.128540in}}%
\pgfpathlineto{\pgfqpoint{2.521853in}{0.128540in}}%
\pgfpathlineto{\pgfqpoint{2.516520in}{0.128540in}}%
\pgfpathlineto{\pgfqpoint{2.511186in}{0.128540in}}%
\pgfpathlineto{\pgfqpoint{2.505853in}{0.128540in}}%
\pgfpathlineto{\pgfqpoint{2.500520in}{0.128540in}}%
\pgfpathlineto{\pgfqpoint{2.495187in}{0.128540in}}%
\pgfpathlineto{\pgfqpoint{2.489853in}{0.128540in}}%
\pgfpathlineto{\pgfqpoint{2.484520in}{0.128540in}}%
\pgfpathlineto{\pgfqpoint{2.479187in}{0.128540in}}%
\pgfpathlineto{\pgfqpoint{2.473854in}{0.128540in}}%
\pgfpathlineto{\pgfqpoint{2.468521in}{0.128540in}}%
\pgfpathlineto{\pgfqpoint{2.463187in}{0.128540in}}%
\pgfpathlineto{\pgfqpoint{2.457854in}{0.128540in}}%
\pgfpathlineto{\pgfqpoint{2.452521in}{0.128540in}}%
\pgfpathlineto{\pgfqpoint{2.447188in}{0.128540in}}%
\pgfpathlineto{\pgfqpoint{2.441855in}{0.128540in}}%
\pgfpathlineto{\pgfqpoint{2.436521in}{0.128540in}}%
\pgfpathlineto{\pgfqpoint{2.431188in}{0.128540in}}%
\pgfpathlineto{\pgfqpoint{2.425855in}{0.128540in}}%
\pgfpathlineto{\pgfqpoint{2.420522in}{0.128540in}}%
\pgfpathlineto{\pgfqpoint{2.415188in}{0.128540in}}%
\pgfpathlineto{\pgfqpoint{2.409855in}{0.128540in}}%
\pgfpathlineto{\pgfqpoint{2.404522in}{0.128540in}}%
\pgfpathlineto{\pgfqpoint{2.399189in}{0.128540in}}%
\pgfpathlineto{\pgfqpoint{2.393856in}{0.128540in}}%
\pgfpathlineto{\pgfqpoint{2.388522in}{0.128540in}}%
\pgfpathlineto{\pgfqpoint{2.383189in}{0.128540in}}%
\pgfpathlineto{\pgfqpoint{2.377856in}{0.128540in}}%
\pgfpathlineto{\pgfqpoint{2.372523in}{0.128540in}}%
\pgfpathlineto{\pgfqpoint{2.367190in}{0.128540in}}%
\pgfpathlineto{\pgfqpoint{2.361856in}{0.128540in}}%
\pgfpathlineto{\pgfqpoint{2.356523in}{0.128540in}}%
\pgfpathlineto{\pgfqpoint{2.351190in}{0.128540in}}%
\pgfpathlineto{\pgfqpoint{2.345857in}{0.128540in}}%
\pgfpathlineto{\pgfqpoint{2.340523in}{0.128540in}}%
\pgfpathlineto{\pgfqpoint{2.335190in}{0.128540in}}%
\pgfpathlineto{\pgfqpoint{2.329857in}{0.128540in}}%
\pgfpathlineto{\pgfqpoint{2.324524in}{0.128540in}}%
\pgfpathlineto{\pgfqpoint{2.319191in}{0.128540in}}%
\pgfpathlineto{\pgfqpoint{2.313857in}{0.128540in}}%
\pgfpathlineto{\pgfqpoint{2.308524in}{0.128540in}}%
\pgfpathlineto{\pgfqpoint{2.303191in}{0.128540in}}%
\pgfpathlineto{\pgfqpoint{2.297858in}{0.128540in}}%
\pgfpathlineto{\pgfqpoint{2.292524in}{0.128540in}}%
\pgfpathlineto{\pgfqpoint{2.287191in}{0.128540in}}%
\pgfpathlineto{\pgfqpoint{2.281858in}{0.128540in}}%
\pgfpathlineto{\pgfqpoint{2.276525in}{0.128540in}}%
\pgfpathlineto{\pgfqpoint{2.271192in}{0.128540in}}%
\pgfpathlineto{\pgfqpoint{2.265858in}{0.128540in}}%
\pgfpathlineto{\pgfqpoint{2.260525in}{0.128540in}}%
\pgfpathlineto{\pgfqpoint{2.255192in}{0.128540in}}%
\pgfpathlineto{\pgfqpoint{2.249859in}{0.128540in}}%
\pgfpathlineto{\pgfqpoint{2.244526in}{0.128540in}}%
\pgfpathlineto{\pgfqpoint{2.239192in}{0.128540in}}%
\pgfpathlineto{\pgfqpoint{2.233859in}{0.128540in}}%
\pgfpathlineto{\pgfqpoint{2.228526in}{0.128540in}}%
\pgfpathlineto{\pgfqpoint{2.223193in}{0.128540in}}%
\pgfpathlineto{\pgfqpoint{2.217859in}{0.128540in}}%
\pgfpathlineto{\pgfqpoint{2.212526in}{0.128540in}}%
\pgfpathlineto{\pgfqpoint{2.207193in}{0.128540in}}%
\pgfpathlineto{\pgfqpoint{2.201860in}{0.128540in}}%
\pgfpathlineto{\pgfqpoint{2.196527in}{0.128540in}}%
\pgfpathlineto{\pgfqpoint{2.191193in}{0.128540in}}%
\pgfpathlineto{\pgfqpoint{2.185860in}{0.128540in}}%
\pgfpathlineto{\pgfqpoint{2.180527in}{0.128540in}}%
\pgfpathlineto{\pgfqpoint{2.175194in}{0.128540in}}%
\pgfpathlineto{\pgfqpoint{2.169861in}{0.128540in}}%
\pgfpathlineto{\pgfqpoint{2.164527in}{0.128540in}}%
\pgfpathlineto{\pgfqpoint{2.159194in}{0.128540in}}%
\pgfpathlineto{\pgfqpoint{2.153861in}{0.128540in}}%
\pgfpathlineto{\pgfqpoint{2.148528in}{0.128540in}}%
\pgfpathlineto{\pgfqpoint{2.143194in}{0.128540in}}%
\pgfpathlineto{\pgfqpoint{2.137861in}{0.128540in}}%
\pgfpathlineto{\pgfqpoint{2.132528in}{0.128540in}}%
\pgfpathlineto{\pgfqpoint{2.127195in}{0.128540in}}%
\pgfpathlineto{\pgfqpoint{2.121862in}{0.128540in}}%
\pgfpathlineto{\pgfqpoint{2.116528in}{0.128540in}}%
\pgfpathlineto{\pgfqpoint{2.111195in}{0.128540in}}%
\pgfpathlineto{\pgfqpoint{2.105862in}{0.128540in}}%
\pgfpathlineto{\pgfqpoint{2.100529in}{0.128540in}}%
\pgfpathlineto{\pgfqpoint{2.095196in}{0.128540in}}%
\pgfpathlineto{\pgfqpoint{2.089862in}{0.128540in}}%
\pgfpathlineto{\pgfqpoint{2.084529in}{0.128540in}}%
\pgfpathlineto{\pgfqpoint{2.079196in}{0.128540in}}%
\pgfpathlineto{\pgfqpoint{2.073863in}{0.128540in}}%
\pgfpathlineto{\pgfqpoint{2.068529in}{0.128540in}}%
\pgfpathlineto{\pgfqpoint{2.063196in}{0.128540in}}%
\pgfpathlineto{\pgfqpoint{2.057863in}{0.128540in}}%
\pgfpathlineto{\pgfqpoint{2.052530in}{0.128540in}}%
\pgfpathlineto{\pgfqpoint{2.047197in}{0.128540in}}%
\pgfpathlineto{\pgfqpoint{2.041863in}{0.128540in}}%
\pgfpathlineto{\pgfqpoint{2.036530in}{0.128540in}}%
\pgfpathlineto{\pgfqpoint{2.031197in}{0.128540in}}%
\pgfpathlineto{\pgfqpoint{2.025864in}{0.128540in}}%
\pgfpathlineto{\pgfqpoint{2.020531in}{0.128540in}}%
\pgfpathlineto{\pgfqpoint{2.015197in}{0.128540in}}%
\pgfpathlineto{\pgfqpoint{2.009864in}{0.128540in}}%
\pgfpathlineto{\pgfqpoint{2.004531in}{0.128540in}}%
\pgfpathlineto{\pgfqpoint{1.999198in}{0.128540in}}%
\pgfpathlineto{\pgfqpoint{1.993864in}{0.128540in}}%
\pgfpathlineto{\pgfqpoint{1.988531in}{0.128540in}}%
\pgfpathlineto{\pgfqpoint{1.983198in}{0.128540in}}%
\pgfpathlineto{\pgfqpoint{1.977865in}{0.128540in}}%
\pgfpathlineto{\pgfqpoint{1.972532in}{0.128540in}}%
\pgfpathlineto{\pgfqpoint{1.967198in}{0.128540in}}%
\pgfpathlineto{\pgfqpoint{1.961865in}{0.128540in}}%
\pgfpathlineto{\pgfqpoint{1.956532in}{0.128540in}}%
\pgfpathlineto{\pgfqpoint{1.951199in}{0.128540in}}%
\pgfpathlineto{\pgfqpoint{1.945865in}{0.128540in}}%
\pgfpathlineto{\pgfqpoint{1.940532in}{0.128540in}}%
\pgfpathlineto{\pgfqpoint{1.935199in}{0.128540in}}%
\pgfpathlineto{\pgfqpoint{1.929866in}{0.128540in}}%
\pgfpathlineto{\pgfqpoint{1.924533in}{0.128540in}}%
\pgfpathlineto{\pgfqpoint{1.919199in}{0.128540in}}%
\pgfpathlineto{\pgfqpoint{1.913866in}{0.128540in}}%
\pgfpathlineto{\pgfqpoint{1.908533in}{0.128540in}}%
\pgfpathlineto{\pgfqpoint{1.903200in}{0.128540in}}%
\pgfpathlineto{\pgfqpoint{1.897867in}{0.128540in}}%
\pgfpathlineto{\pgfqpoint{1.892533in}{0.128540in}}%
\pgfpathlineto{\pgfqpoint{1.887200in}{0.128540in}}%
\pgfpathlineto{\pgfqpoint{1.881867in}{0.128540in}}%
\pgfpathlineto{\pgfqpoint{1.876534in}{0.128540in}}%
\pgfpathlineto{\pgfqpoint{1.871200in}{0.128540in}}%
\pgfpathlineto{\pgfqpoint{1.865867in}{0.128540in}}%
\pgfpathlineto{\pgfqpoint{1.860534in}{0.128540in}}%
\pgfpathlineto{\pgfqpoint{1.855201in}{0.128540in}}%
\pgfpathlineto{\pgfqpoint{1.849868in}{0.128540in}}%
\pgfpathlineto{\pgfqpoint{1.844534in}{0.128540in}}%
\pgfpathlineto{\pgfqpoint{1.839201in}{0.128540in}}%
\pgfpathlineto{\pgfqpoint{1.833868in}{0.128540in}}%
\pgfpathlineto{\pgfqpoint{1.828535in}{0.128540in}}%
\pgfpathlineto{\pgfqpoint{1.823202in}{0.128540in}}%
\pgfpathlineto{\pgfqpoint{1.817868in}{0.128540in}}%
\pgfpathlineto{\pgfqpoint{1.812535in}{0.128540in}}%
\pgfpathlineto{\pgfqpoint{1.807202in}{0.128540in}}%
\pgfpathlineto{\pgfqpoint{1.801869in}{0.128540in}}%
\pgfpathlineto{\pgfqpoint{1.796535in}{0.128540in}}%
\pgfpathlineto{\pgfqpoint{1.791202in}{0.128540in}}%
\pgfpathlineto{\pgfqpoint{1.785869in}{0.128540in}}%
\pgfpathlineto{\pgfqpoint{1.780536in}{0.128540in}}%
\pgfpathlineto{\pgfqpoint{1.775203in}{0.128540in}}%
\pgfpathlineto{\pgfqpoint{1.769869in}{0.128540in}}%
\pgfpathlineto{\pgfqpoint{1.764536in}{0.128540in}}%
\pgfpathlineto{\pgfqpoint{1.759203in}{0.128540in}}%
\pgfpathlineto{\pgfqpoint{1.753870in}{0.128540in}}%
\pgfpathlineto{\pgfqpoint{1.748537in}{0.128540in}}%
\pgfpathlineto{\pgfqpoint{1.743203in}{0.128540in}}%
\pgfpathlineto{\pgfqpoint{1.737870in}{0.128540in}}%
\pgfpathlineto{\pgfqpoint{1.732537in}{0.128540in}}%
\pgfpathlineto{\pgfqpoint{1.727204in}{0.128540in}}%
\pgfpathlineto{\pgfqpoint{1.721870in}{0.128540in}}%
\pgfpathlineto{\pgfqpoint{1.716537in}{0.128540in}}%
\pgfpathlineto{\pgfqpoint{1.711204in}{0.128540in}}%
\pgfpathlineto{\pgfqpoint{1.705871in}{0.128540in}}%
\pgfpathlineto{\pgfqpoint{1.700538in}{0.128540in}}%
\pgfpathlineto{\pgfqpoint{1.695204in}{0.128540in}}%
\pgfpathlineto{\pgfqpoint{1.689871in}{0.128540in}}%
\pgfpathlineto{\pgfqpoint{1.684538in}{0.128540in}}%
\pgfpathlineto{\pgfqpoint{1.679205in}{0.128540in}}%
\pgfpathlineto{\pgfqpoint{1.673871in}{0.128540in}}%
\pgfpathlineto{\pgfqpoint{1.668538in}{0.128540in}}%
\pgfpathlineto{\pgfqpoint{1.663205in}{0.128540in}}%
\pgfpathlineto{\pgfqpoint{1.657872in}{0.128540in}}%
\pgfpathlineto{\pgfqpoint{1.652539in}{0.128540in}}%
\pgfpathlineto{\pgfqpoint{1.647205in}{0.128540in}}%
\pgfpathlineto{\pgfqpoint{1.641872in}{0.128540in}}%
\pgfpathlineto{\pgfqpoint{1.636539in}{0.128540in}}%
\pgfpathlineto{\pgfqpoint{1.631206in}{0.128540in}}%
\pgfpathlineto{\pgfqpoint{1.625873in}{0.128540in}}%
\pgfpathlineto{\pgfqpoint{1.620539in}{0.128540in}}%
\pgfpathlineto{\pgfqpoint{1.615206in}{0.128540in}}%
\pgfpathlineto{\pgfqpoint{1.609873in}{0.128540in}}%
\pgfpathlineto{\pgfqpoint{1.604540in}{0.128540in}}%
\pgfpathlineto{\pgfqpoint{1.599206in}{0.128540in}}%
\pgfpathlineto{\pgfqpoint{1.593873in}{0.128540in}}%
\pgfpathlineto{\pgfqpoint{1.588540in}{0.128540in}}%
\pgfpathlineto{\pgfqpoint{1.583207in}{0.128540in}}%
\pgfpathlineto{\pgfqpoint{1.577874in}{0.128540in}}%
\pgfpathlineto{\pgfqpoint{1.572540in}{0.128540in}}%
\pgfpathlineto{\pgfqpoint{1.567207in}{0.128540in}}%
\pgfpathlineto{\pgfqpoint{1.561874in}{0.128540in}}%
\pgfpathlineto{\pgfqpoint{1.556541in}{0.128540in}}%
\pgfpathlineto{\pgfqpoint{1.551208in}{0.128540in}}%
\pgfpathlineto{\pgfqpoint{1.545874in}{0.128540in}}%
\pgfpathlineto{\pgfqpoint{1.540541in}{0.128540in}}%
\pgfpathlineto{\pgfqpoint{1.535208in}{0.128540in}}%
\pgfpathlineto{\pgfqpoint{1.529875in}{0.128540in}}%
\pgfpathlineto{\pgfqpoint{1.524541in}{0.128540in}}%
\pgfpathlineto{\pgfqpoint{1.519208in}{0.128540in}}%
\pgfpathlineto{\pgfqpoint{1.513875in}{0.128540in}}%
\pgfpathlineto{\pgfqpoint{1.508542in}{0.128540in}}%
\pgfpathlineto{\pgfqpoint{1.503209in}{0.128540in}}%
\pgfpathlineto{\pgfqpoint{1.497875in}{0.128540in}}%
\pgfpathlineto{\pgfqpoint{1.492542in}{0.128540in}}%
\pgfpathlineto{\pgfqpoint{1.487209in}{0.128540in}}%
\pgfpathlineto{\pgfqpoint{1.481876in}{0.128540in}}%
\pgfpathlineto{\pgfqpoint{1.476543in}{0.128540in}}%
\pgfpathlineto{\pgfqpoint{1.471209in}{0.128540in}}%
\pgfpathlineto{\pgfqpoint{1.465876in}{0.128540in}}%
\pgfpathlineto{\pgfqpoint{1.460543in}{0.128540in}}%
\pgfpathlineto{\pgfqpoint{1.455210in}{0.128540in}}%
\pgfpathlineto{\pgfqpoint{1.449876in}{0.128540in}}%
\pgfpathlineto{\pgfqpoint{1.444543in}{0.128540in}}%
\pgfpathlineto{\pgfqpoint{1.439210in}{0.128540in}}%
\pgfpathlineto{\pgfqpoint{1.433877in}{0.128540in}}%
\pgfpathlineto{\pgfqpoint{1.428544in}{0.128540in}}%
\pgfpathlineto{\pgfqpoint{1.423210in}{0.128540in}}%
\pgfpathlineto{\pgfqpoint{1.417877in}{0.128540in}}%
\pgfpathlineto{\pgfqpoint{1.412544in}{0.128540in}}%
\pgfpathlineto{\pgfqpoint{1.407211in}{0.128540in}}%
\pgfpathlineto{\pgfqpoint{1.401877in}{0.128540in}}%
\pgfpathlineto{\pgfqpoint{1.396544in}{0.128540in}}%
\pgfpathlineto{\pgfqpoint{1.391211in}{0.128540in}}%
\pgfpathlineto{\pgfqpoint{1.385878in}{0.128540in}}%
\pgfpathlineto{\pgfqpoint{1.380545in}{0.128540in}}%
\pgfpathlineto{\pgfqpoint{1.375211in}{0.128540in}}%
\pgfpathlineto{\pgfqpoint{1.369878in}{0.128540in}}%
\pgfpathlineto{\pgfqpoint{1.364545in}{0.128540in}}%
\pgfpathlineto{\pgfqpoint{1.359212in}{0.128540in}}%
\pgfpathlineto{\pgfqpoint{1.353879in}{0.128540in}}%
\pgfpathlineto{\pgfqpoint{1.348545in}{0.128540in}}%
\pgfpathlineto{\pgfqpoint{1.343212in}{0.128540in}}%
\pgfpathlineto{\pgfqpoint{1.337879in}{0.128540in}}%
\pgfpathlineto{\pgfqpoint{1.332546in}{0.128540in}}%
\pgfpathlineto{\pgfqpoint{1.327212in}{0.128540in}}%
\pgfpathlineto{\pgfqpoint{1.321879in}{0.128540in}}%
\pgfpathlineto{\pgfqpoint{1.316546in}{0.128540in}}%
\pgfpathlineto{\pgfqpoint{1.311213in}{0.128540in}}%
\pgfpathlineto{\pgfqpoint{1.305880in}{0.128540in}}%
\pgfpathlineto{\pgfqpoint{1.300546in}{0.128540in}}%
\pgfpathlineto{\pgfqpoint{1.295213in}{0.128540in}}%
\pgfpathlineto{\pgfqpoint{1.289880in}{0.128540in}}%
\pgfpathlineto{\pgfqpoint{1.284547in}{0.128540in}}%
\pgfpathlineto{\pgfqpoint{1.279214in}{0.128540in}}%
\pgfpathlineto{\pgfqpoint{1.273880in}{0.128540in}}%
\pgfpathlineto{\pgfqpoint{1.268547in}{0.128540in}}%
\pgfpathlineto{\pgfqpoint{1.263214in}{0.128540in}}%
\pgfpathlineto{\pgfqpoint{1.257881in}{0.128540in}}%
\pgfpathlineto{\pgfqpoint{1.252547in}{0.128540in}}%
\pgfpathlineto{\pgfqpoint{1.247214in}{0.128540in}}%
\pgfpathlineto{\pgfqpoint{1.241881in}{0.128540in}}%
\pgfpathlineto{\pgfqpoint{1.236548in}{0.128540in}}%
\pgfpathlineto{\pgfqpoint{1.231215in}{0.128540in}}%
\pgfpathlineto{\pgfqpoint{1.225881in}{0.128540in}}%
\pgfpathlineto{\pgfqpoint{1.220548in}{0.128540in}}%
\pgfpathlineto{\pgfqpoint{1.215215in}{0.128540in}}%
\pgfpathlineto{\pgfqpoint{1.209882in}{0.128540in}}%
\pgfpathlineto{\pgfqpoint{1.204549in}{0.128540in}}%
\pgfpathlineto{\pgfqpoint{1.199215in}{0.128540in}}%
\pgfpathlineto{\pgfqpoint{1.193882in}{0.128540in}}%
\pgfpathlineto{\pgfqpoint{1.188549in}{0.128540in}}%
\pgfpathlineto{\pgfqpoint{1.183216in}{0.128540in}}%
\pgfpathlineto{\pgfqpoint{1.177882in}{0.128540in}}%
\pgfpathlineto{\pgfqpoint{1.172549in}{0.128540in}}%
\pgfpathlineto{\pgfqpoint{1.167216in}{0.128540in}}%
\pgfpathlineto{\pgfqpoint{1.161883in}{0.128540in}}%
\pgfpathlineto{\pgfqpoint{1.156550in}{0.128540in}}%
\pgfpathlineto{\pgfqpoint{1.151216in}{0.128540in}}%
\pgfpathlineto{\pgfqpoint{1.145883in}{0.128540in}}%
\pgfpathlineto{\pgfqpoint{1.140550in}{0.128540in}}%
\pgfpathlineto{\pgfqpoint{1.135217in}{0.128540in}}%
\pgfpathlineto{\pgfqpoint{1.129883in}{0.128540in}}%
\pgfpathlineto{\pgfqpoint{1.124550in}{0.128540in}}%
\pgfpathlineto{\pgfqpoint{1.119217in}{0.128540in}}%
\pgfpathlineto{\pgfqpoint{1.113884in}{0.128540in}}%
\pgfpathlineto{\pgfqpoint{1.108551in}{0.128540in}}%
\pgfpathlineto{\pgfqpoint{1.103217in}{0.128540in}}%
\pgfpathlineto{\pgfqpoint{1.097884in}{0.128540in}}%
\pgfpathlineto{\pgfqpoint{1.092551in}{0.128540in}}%
\pgfpathlineto{\pgfqpoint{1.087218in}{0.128540in}}%
\pgfpathlineto{\pgfqpoint{1.081885in}{0.128540in}}%
\pgfpathlineto{\pgfqpoint{1.076551in}{0.128540in}}%
\pgfpathlineto{\pgfqpoint{1.071218in}{0.128540in}}%
\pgfpathlineto{\pgfqpoint{1.065885in}{0.128540in}}%
\pgfpathlineto{\pgfqpoint{1.060552in}{0.128540in}}%
\pgfpathlineto{\pgfqpoint{1.055218in}{0.128540in}}%
\pgfpathlineto{\pgfqpoint{1.049885in}{0.128540in}}%
\pgfpathlineto{\pgfqpoint{1.044552in}{0.128540in}}%
\pgfpathlineto{\pgfqpoint{1.039219in}{0.128540in}}%
\pgfpathlineto{\pgfqpoint{1.033886in}{0.128540in}}%
\pgfpathlineto{\pgfqpoint{1.028552in}{0.128540in}}%
\pgfpathlineto{\pgfqpoint{1.023219in}{0.128540in}}%
\pgfpathlineto{\pgfqpoint{1.017886in}{0.128540in}}%
\pgfpathlineto{\pgfqpoint{1.012553in}{0.128540in}}%
\pgfpathlineto{\pgfqpoint{1.007220in}{0.128540in}}%
\pgfpathlineto{\pgfqpoint{1.001886in}{0.128540in}}%
\pgfpathlineto{\pgfqpoint{0.996553in}{0.128540in}}%
\pgfpathlineto{\pgfqpoint{0.991220in}{0.128540in}}%
\pgfpathlineto{\pgfqpoint{0.985887in}{0.128540in}}%
\pgfpathlineto{\pgfqpoint{0.980553in}{0.128540in}}%
\pgfpathlineto{\pgfqpoint{0.975220in}{0.128540in}}%
\pgfpathlineto{\pgfqpoint{0.969887in}{0.128540in}}%
\pgfpathlineto{\pgfqpoint{0.964554in}{0.128540in}}%
\pgfpathlineto{\pgfqpoint{0.959221in}{0.128540in}}%
\pgfpathlineto{\pgfqpoint{0.953887in}{0.128540in}}%
\pgfpathlineto{\pgfqpoint{0.948554in}{0.128540in}}%
\pgfpathlineto{\pgfqpoint{0.943221in}{0.128540in}}%
\pgfpathlineto{\pgfqpoint{0.937888in}{0.128540in}}%
\pgfpathlineto{\pgfqpoint{0.932555in}{0.128540in}}%
\pgfpathlineto{\pgfqpoint{0.927221in}{0.128540in}}%
\pgfpathlineto{\pgfqpoint{0.921888in}{0.128540in}}%
\pgfpathlineto{\pgfqpoint{0.916555in}{0.128540in}}%
\pgfpathlineto{\pgfqpoint{0.911222in}{0.128540in}}%
\pgfpathlineto{\pgfqpoint{0.905888in}{0.128540in}}%
\pgfpathlineto{\pgfqpoint{0.900555in}{0.128540in}}%
\pgfpathlineto{\pgfqpoint{0.895222in}{0.128540in}}%
\pgfpathlineto{\pgfqpoint{0.889889in}{0.128540in}}%
\pgfpathlineto{\pgfqpoint{0.884556in}{0.128540in}}%
\pgfpathlineto{\pgfqpoint{0.879222in}{0.128540in}}%
\pgfpathlineto{\pgfqpoint{0.873889in}{0.128540in}}%
\pgfpathlineto{\pgfqpoint{0.868556in}{0.128540in}}%
\pgfpathlineto{\pgfqpoint{0.863223in}{0.128540in}}%
\pgfpathlineto{\pgfqpoint{0.857889in}{0.128540in}}%
\pgfpathlineto{\pgfqpoint{0.852556in}{0.128540in}}%
\pgfpathlineto{\pgfqpoint{0.847223in}{0.128540in}}%
\pgfpathlineto{\pgfqpoint{0.847223in}{0.128540in}}%
\pgfpathclose%
\pgfusepath{stroke,fill}%
}%
\begin{pgfscope}%
\pgfsys@transformshift{0.000000in}{0.000000in}%
\pgfsys@useobject{currentmarker}{}%
\end{pgfscope}%
\end{pgfscope}%
\begin{pgfscope}%
\pgfpathrectangle{\pgfqpoint{0.847223in}{0.554012in}}{\pgfqpoint{6.200000in}{4.620000in}}%
\pgfusepath{clip}%
\pgfsetbuttcap%
\pgfsetroundjoin%
\definecolor{currentfill}{rgb}{0.121569,0.466667,0.705882}%
\pgfsetfillcolor{currentfill}%
\pgfsetfillopacity{0.200000}%
\pgfsetlinewidth{1.003750pt}%
\definecolor{currentstroke}{rgb}{0.121569,0.466667,0.705882}%
\pgfsetstrokecolor{currentstroke}%
\pgfsetstrokeopacity{0.200000}%
\pgfsetdash{}{0pt}%
\pgfsys@defobject{currentmarker}{\pgfqpoint{0.847223in}{0.554012in}}{\pgfqpoint{7.047223in}{5.174012in}}{%
\pgfpathmoveto{\pgfqpoint{0.847223in}{5.174012in}}%
\pgfpathlineto{\pgfqpoint{0.847223in}{5.174012in}}%
\pgfpathlineto{\pgfqpoint{0.852556in}{5.123114in}}%
\pgfpathlineto{\pgfqpoint{0.857889in}{5.073234in}}%
\pgfpathlineto{\pgfqpoint{0.863223in}{5.024339in}}%
\pgfpathlineto{\pgfqpoint{0.868556in}{4.976403in}}%
\pgfpathlineto{\pgfqpoint{0.873889in}{4.929396in}}%
\pgfpathlineto{\pgfqpoint{0.879222in}{4.883291in}}%
\pgfpathlineto{\pgfqpoint{0.884556in}{4.838064in}}%
\pgfpathlineto{\pgfqpoint{0.889889in}{4.793689in}}%
\pgfpathlineto{\pgfqpoint{0.895222in}{4.750143in}}%
\pgfpathlineto{\pgfqpoint{0.900555in}{4.707402in}}%
\pgfpathlineto{\pgfqpoint{0.905888in}{4.665444in}}%
\pgfpathlineto{\pgfqpoint{0.911222in}{4.624248in}}%
\pgfpathlineto{\pgfqpoint{0.916555in}{4.583794in}}%
\pgfpathlineto{\pgfqpoint{0.921888in}{4.544061in}}%
\pgfpathlineto{\pgfqpoint{0.927221in}{4.505030in}}%
\pgfpathlineto{\pgfqpoint{0.932555in}{4.466684in}}%
\pgfpathlineto{\pgfqpoint{0.937888in}{4.429004in}}%
\pgfpathlineto{\pgfqpoint{0.943221in}{4.391972in}}%
\pgfpathlineto{\pgfqpoint{0.948554in}{4.355573in}}%
\pgfpathlineto{\pgfqpoint{0.953887in}{4.319791in}}%
\pgfpathlineto{\pgfqpoint{0.959221in}{4.284608in}}%
\pgfpathlineto{\pgfqpoint{0.964554in}{4.250012in}}%
\pgfpathlineto{\pgfqpoint{0.969887in}{4.215987in}}%
\pgfpathlineto{\pgfqpoint{0.975220in}{4.182519in}}%
\pgfpathlineto{\pgfqpoint{0.980553in}{4.149595in}}%
\pgfpathlineto{\pgfqpoint{0.985887in}{4.117201in}}%
\pgfpathlineto{\pgfqpoint{0.991220in}{4.085325in}}%
\pgfpathlineto{\pgfqpoint{0.996553in}{4.053954in}}%
\pgfpathlineto{\pgfqpoint{1.001886in}{4.023077in}}%
\pgfpathlineto{\pgfqpoint{1.007220in}{3.992682in}}%
\pgfpathlineto{\pgfqpoint{1.012553in}{3.962758in}}%
\pgfpathlineto{\pgfqpoint{1.017886in}{3.933293in}}%
\pgfpathlineto{\pgfqpoint{1.023219in}{3.904278in}}%
\pgfpathlineto{\pgfqpoint{1.028552in}{3.875702in}}%
\pgfpathlineto{\pgfqpoint{1.033886in}{3.847556in}}%
\pgfpathlineto{\pgfqpoint{1.039219in}{3.819829in}}%
\pgfpathlineto{\pgfqpoint{1.044552in}{3.792512in}}%
\pgfpathlineto{\pgfqpoint{1.049885in}{3.765597in}}%
\pgfpathlineto{\pgfqpoint{1.055218in}{3.739075in}}%
\pgfpathlineto{\pgfqpoint{1.060552in}{3.712936in}}%
\pgfpathlineto{\pgfqpoint{1.065885in}{3.687173in}}%
\pgfpathlineto{\pgfqpoint{1.071218in}{3.661778in}}%
\pgfpathlineto{\pgfqpoint{1.076551in}{3.636743in}}%
\pgfpathlineto{\pgfqpoint{1.081885in}{3.612060in}}%
\pgfpathlineto{\pgfqpoint{1.087218in}{3.587722in}}%
\pgfpathlineto{\pgfqpoint{1.092551in}{3.563721in}}%
\pgfpathlineto{\pgfqpoint{1.097884in}{3.540052in}}%
\pgfpathlineto{\pgfqpoint{1.103217in}{3.516706in}}%
\pgfpathlineto{\pgfqpoint{1.108551in}{3.493678in}}%
\pgfpathlineto{\pgfqpoint{1.113884in}{3.470960in}}%
\pgfpathlineto{\pgfqpoint{1.119217in}{3.448547in}}%
\pgfpathlineto{\pgfqpoint{1.124550in}{3.426433in}}%
\pgfpathlineto{\pgfqpoint{1.129883in}{3.404612in}}%
\pgfpathlineto{\pgfqpoint{1.135217in}{3.383077in}}%
\pgfpathlineto{\pgfqpoint{1.140550in}{3.361824in}}%
\pgfpathlineto{\pgfqpoint{1.145883in}{3.340846in}}%
\pgfpathlineto{\pgfqpoint{1.151216in}{3.320139in}}%
\pgfpathlineto{\pgfqpoint{1.156550in}{3.299697in}}%
\pgfpathlineto{\pgfqpoint{1.161883in}{3.279515in}}%
\pgfpathlineto{\pgfqpoint{1.167216in}{3.259589in}}%
\pgfpathlineto{\pgfqpoint{1.172549in}{3.239913in}}%
\pgfpathlineto{\pgfqpoint{1.177882in}{3.220483in}}%
\pgfpathlineto{\pgfqpoint{1.183216in}{3.201294in}}%
\pgfpathlineto{\pgfqpoint{1.188549in}{3.182341in}}%
\pgfpathlineto{\pgfqpoint{1.193882in}{3.163621in}}%
\pgfpathlineto{\pgfqpoint{1.199215in}{3.145129in}}%
\pgfpathlineto{\pgfqpoint{1.204549in}{3.126861in}}%
\pgfpathlineto{\pgfqpoint{1.209882in}{3.108813in}}%
\pgfpathlineto{\pgfqpoint{1.215215in}{3.090981in}}%
\pgfpathlineto{\pgfqpoint{1.220548in}{3.073361in}}%
\pgfpathlineto{\pgfqpoint{1.225881in}{3.055950in}}%
\pgfpathlineto{\pgfqpoint{1.231215in}{3.038743in}}%
\pgfpathlineto{\pgfqpoint{1.236548in}{3.021737in}}%
\pgfpathlineto{\pgfqpoint{1.241881in}{3.004929in}}%
\pgfpathlineto{\pgfqpoint{1.247214in}{2.988315in}}%
\pgfpathlineto{\pgfqpoint{1.252547in}{2.971892in}}%
\pgfpathlineto{\pgfqpoint{1.257881in}{2.955656in}}%
\pgfpathlineto{\pgfqpoint{1.263214in}{2.939605in}}%
\pgfpathlineto{\pgfqpoint{1.268547in}{2.923735in}}%
\pgfpathlineto{\pgfqpoint{1.273880in}{2.908043in}}%
\pgfpathlineto{\pgfqpoint{1.279214in}{2.892526in}}%
\pgfpathlineto{\pgfqpoint{1.284547in}{2.877182in}}%
\pgfpathlineto{\pgfqpoint{1.289880in}{2.862007in}}%
\pgfpathlineto{\pgfqpoint{1.295213in}{2.846999in}}%
\pgfpathlineto{\pgfqpoint{1.300546in}{2.832154in}}%
\pgfpathlineto{\pgfqpoint{1.305880in}{2.817471in}}%
\pgfpathlineto{\pgfqpoint{1.311213in}{2.802947in}}%
\pgfpathlineto{\pgfqpoint{1.316546in}{2.788578in}}%
\pgfpathlineto{\pgfqpoint{1.321879in}{2.774363in}}%
\pgfpathlineto{\pgfqpoint{1.327212in}{2.760300in}}%
\pgfpathlineto{\pgfqpoint{1.332546in}{2.746385in}}%
\pgfpathlineto{\pgfqpoint{1.337879in}{2.732616in}}%
\pgfpathlineto{\pgfqpoint{1.343212in}{2.718991in}}%
\pgfpathlineto{\pgfqpoint{1.348545in}{2.705509in}}%
\pgfpathlineto{\pgfqpoint{1.353879in}{2.692165in}}%
\pgfpathlineto{\pgfqpoint{1.359212in}{2.678960in}}%
\pgfpathlineto{\pgfqpoint{1.364545in}{2.665889in}}%
\pgfpathlineto{\pgfqpoint{1.369878in}{2.652952in}}%
\pgfpathlineto{\pgfqpoint{1.375211in}{2.640146in}}%
\pgfpathlineto{\pgfqpoint{1.380545in}{2.627470in}}%
\pgfpathlineto{\pgfqpoint{1.385878in}{2.614921in}}%
\pgfpathlineto{\pgfqpoint{1.391211in}{2.602497in}}%
\pgfpathlineto{\pgfqpoint{1.396544in}{2.590197in}}%
\pgfpathlineto{\pgfqpoint{1.401877in}{2.578018in}}%
\pgfpathlineto{\pgfqpoint{1.407211in}{2.565959in}}%
\pgfpathlineto{\pgfqpoint{1.412544in}{2.554019in}}%
\pgfpathlineto{\pgfqpoint{1.417877in}{2.542195in}}%
\pgfpathlineto{\pgfqpoint{1.423210in}{2.530485in}}%
\pgfpathlineto{\pgfqpoint{1.428544in}{2.518889in}}%
\pgfpathlineto{\pgfqpoint{1.433877in}{2.507404in}}%
\pgfpathlineto{\pgfqpoint{1.439210in}{2.496029in}}%
\pgfpathlineto{\pgfqpoint{1.444543in}{2.484763in}}%
\pgfpathlineto{\pgfqpoint{1.449876in}{2.473602in}}%
\pgfpathlineto{\pgfqpoint{1.455210in}{2.462548in}}%
\pgfpathlineto{\pgfqpoint{1.460543in}{2.451597in}}%
\pgfpathlineto{\pgfqpoint{1.465876in}{2.440748in}}%
\pgfpathlineto{\pgfqpoint{1.471209in}{2.430000in}}%
\pgfpathlineto{\pgfqpoint{1.476543in}{2.419351in}}%
\pgfpathlineto{\pgfqpoint{1.481876in}{2.408801in}}%
\pgfpathlineto{\pgfqpoint{1.487209in}{2.398347in}}%
\pgfpathlineto{\pgfqpoint{1.492542in}{2.387989in}}%
\pgfpathlineto{\pgfqpoint{1.497875in}{2.377725in}}%
\pgfpathlineto{\pgfqpoint{1.503209in}{2.367554in}}%
\pgfpathlineto{\pgfqpoint{1.508542in}{2.357474in}}%
\pgfpathlineto{\pgfqpoint{1.513875in}{2.347484in}}%
\pgfpathlineto{\pgfqpoint{1.519208in}{2.337584in}}%
\pgfpathlineto{\pgfqpoint{1.524541in}{2.327772in}}%
\pgfpathlineto{\pgfqpoint{1.529875in}{2.318046in}}%
\pgfpathlineto{\pgfqpoint{1.535208in}{2.308407in}}%
\pgfpathlineto{\pgfqpoint{1.540541in}{2.298851in}}%
\pgfpathlineto{\pgfqpoint{1.545874in}{2.289379in}}%
\pgfpathlineto{\pgfqpoint{1.551208in}{2.279990in}}%
\pgfpathlineto{\pgfqpoint{1.556541in}{2.270681in}}%
\pgfpathlineto{\pgfqpoint{1.561874in}{2.261453in}}%
\pgfpathlineto{\pgfqpoint{1.567207in}{2.252304in}}%
\pgfpathlineto{\pgfqpoint{1.572540in}{2.243233in}}%
\pgfpathlineto{\pgfqpoint{1.577874in}{2.234240in}}%
\pgfpathlineto{\pgfqpoint{1.583207in}{2.225322in}}%
\pgfpathlineto{\pgfqpoint{1.588540in}{2.216480in}}%
\pgfpathlineto{\pgfqpoint{1.593873in}{2.207712in}}%
\pgfpathlineto{\pgfqpoint{1.599206in}{2.199017in}}%
\pgfpathlineto{\pgfqpoint{1.604540in}{2.190395in}}%
\pgfpathlineto{\pgfqpoint{1.609873in}{2.181844in}}%
\pgfpathlineto{\pgfqpoint{1.615206in}{2.173364in}}%
\pgfpathlineto{\pgfqpoint{1.620539in}{2.164953in}}%
\pgfpathlineto{\pgfqpoint{1.625873in}{2.156612in}}%
\pgfpathlineto{\pgfqpoint{1.631206in}{2.148338in}}%
\pgfpathlineto{\pgfqpoint{1.636539in}{2.140132in}}%
\pgfpathlineto{\pgfqpoint{1.641872in}{2.131992in}}%
\pgfpathlineto{\pgfqpoint{1.647205in}{2.123918in}}%
\pgfpathlineto{\pgfqpoint{1.652539in}{2.115909in}}%
\pgfpathlineto{\pgfqpoint{1.657872in}{2.107963in}}%
\pgfpathlineto{\pgfqpoint{1.663205in}{2.100081in}}%
\pgfpathlineto{\pgfqpoint{1.668538in}{2.092262in}}%
\pgfpathlineto{\pgfqpoint{1.673871in}{2.084504in}}%
\pgfpathlineto{\pgfqpoint{1.679205in}{2.076807in}}%
\pgfpathlineto{\pgfqpoint{1.684538in}{2.069171in}}%
\pgfpathlineto{\pgfqpoint{1.689871in}{2.061594in}}%
\pgfpathlineto{\pgfqpoint{1.695204in}{2.054076in}}%
\pgfpathlineto{\pgfqpoint{1.700538in}{2.046617in}}%
\pgfpathlineto{\pgfqpoint{1.705871in}{2.039215in}}%
\pgfpathlineto{\pgfqpoint{1.711204in}{2.031870in}}%
\pgfpathlineto{\pgfqpoint{1.716537in}{2.024581in}}%
\pgfpathlineto{\pgfqpoint{1.721870in}{2.017348in}}%
\pgfpathlineto{\pgfqpoint{1.727204in}{2.010170in}}%
\pgfpathlineto{\pgfqpoint{1.732537in}{2.003046in}}%
\pgfpathlineto{\pgfqpoint{1.737870in}{1.995976in}}%
\pgfpathlineto{\pgfqpoint{1.743203in}{1.988959in}}%
\pgfpathlineto{\pgfqpoint{1.748537in}{1.981994in}}%
\pgfpathlineto{\pgfqpoint{1.753870in}{1.975082in}}%
\pgfpathlineto{\pgfqpoint{1.759203in}{1.968221in}}%
\pgfpathlineto{\pgfqpoint{1.764536in}{1.961410in}}%
\pgfpathlineto{\pgfqpoint{1.769869in}{1.954650in}}%
\pgfpathlineto{\pgfqpoint{1.775203in}{1.947940in}}%
\pgfpathlineto{\pgfqpoint{1.780536in}{1.941278in}}%
\pgfpathlineto{\pgfqpoint{1.785869in}{1.934666in}}%
\pgfpathlineto{\pgfqpoint{1.791202in}{1.928101in}}%
\pgfpathlineto{\pgfqpoint{1.796535in}{1.921584in}}%
\pgfpathlineto{\pgfqpoint{1.801869in}{1.915114in}}%
\pgfpathlineto{\pgfqpoint{1.807202in}{1.908690in}}%
\pgfpathlineto{\pgfqpoint{1.812535in}{1.902313in}}%
\pgfpathlineto{\pgfqpoint{1.817868in}{1.895981in}}%
\pgfpathlineto{\pgfqpoint{1.823202in}{1.889694in}}%
\pgfpathlineto{\pgfqpoint{1.828535in}{1.883452in}}%
\pgfpathlineto{\pgfqpoint{1.833868in}{1.877253in}}%
\pgfpathlineto{\pgfqpoint{1.839201in}{1.871099in}}%
\pgfpathlineto{\pgfqpoint{1.844534in}{1.864987in}}%
\pgfpathlineto{\pgfqpoint{1.849868in}{1.858919in}}%
\pgfpathlineto{\pgfqpoint{1.855201in}{1.852892in}}%
\pgfpathlineto{\pgfqpoint{1.860534in}{1.846908in}}%
\pgfpathlineto{\pgfqpoint{1.865867in}{1.840964in}}%
\pgfpathlineto{\pgfqpoint{1.871200in}{1.835062in}}%
\pgfpathlineto{\pgfqpoint{1.876534in}{1.829200in}}%
\pgfpathlineto{\pgfqpoint{1.881867in}{1.823379in}}%
\pgfpathlineto{\pgfqpoint{1.887200in}{1.817597in}}%
\pgfpathlineto{\pgfqpoint{1.892533in}{1.811854in}}%
\pgfpathlineto{\pgfqpoint{1.897867in}{1.806151in}}%
\pgfpathlineto{\pgfqpoint{1.903200in}{1.800486in}}%
\pgfpathlineto{\pgfqpoint{1.908533in}{1.794859in}}%
\pgfpathlineto{\pgfqpoint{1.913866in}{1.789269in}}%
\pgfpathlineto{\pgfqpoint{1.919199in}{1.783718in}}%
\pgfpathlineto{\pgfqpoint{1.924533in}{1.778203in}}%
\pgfpathlineto{\pgfqpoint{1.929866in}{1.772725in}}%
\pgfpathlineto{\pgfqpoint{1.935199in}{1.767283in}}%
\pgfpathlineto{\pgfqpoint{1.940532in}{1.761877in}}%
\pgfpathlineto{\pgfqpoint{1.945865in}{1.756506in}}%
\pgfpathlineto{\pgfqpoint{1.951199in}{1.751171in}}%
\pgfpathlineto{\pgfqpoint{1.956532in}{1.745870in}}%
\pgfpathlineto{\pgfqpoint{1.961865in}{1.740604in}}%
\pgfpathlineto{\pgfqpoint{1.967198in}{1.735373in}}%
\pgfpathlineto{\pgfqpoint{1.972532in}{1.730175in}}%
\pgfpathlineto{\pgfqpoint{1.977865in}{1.725010in}}%
\pgfpathlineto{\pgfqpoint{1.983198in}{1.719879in}}%
\pgfpathlineto{\pgfqpoint{1.988531in}{1.714781in}}%
\pgfpathlineto{\pgfqpoint{1.993864in}{1.709715in}}%
\pgfpathlineto{\pgfqpoint{1.999198in}{1.704681in}}%
\pgfpathlineto{\pgfqpoint{2.004531in}{1.699680in}}%
\pgfpathlineto{\pgfqpoint{2.009864in}{1.694710in}}%
\pgfpathlineto{\pgfqpoint{2.015197in}{1.689771in}}%
\pgfpathlineto{\pgfqpoint{2.020531in}{1.684864in}}%
\pgfpathlineto{\pgfqpoint{2.025864in}{1.679987in}}%
\pgfpathlineto{\pgfqpoint{2.031197in}{1.675141in}}%
\pgfpathlineto{\pgfqpoint{2.036530in}{1.670325in}}%
\pgfpathlineto{\pgfqpoint{2.041863in}{1.665539in}}%
\pgfpathlineto{\pgfqpoint{2.047197in}{1.660782in}}%
\pgfpathlineto{\pgfqpoint{2.052530in}{1.656055in}}%
\pgfpathlineto{\pgfqpoint{2.057863in}{1.651357in}}%
\pgfpathlineto{\pgfqpoint{2.063196in}{1.646687in}}%
\pgfpathlineto{\pgfqpoint{2.068529in}{1.642046in}}%
\pgfpathlineto{\pgfqpoint{2.073863in}{1.637434in}}%
\pgfpathlineto{\pgfqpoint{2.079196in}{1.632849in}}%
\pgfpathlineto{\pgfqpoint{2.084529in}{1.628293in}}%
\pgfpathlineto{\pgfqpoint{2.089862in}{1.623763in}}%
\pgfpathlineto{\pgfqpoint{2.095196in}{1.619262in}}%
\pgfpathlineto{\pgfqpoint{2.100529in}{1.614787in}}%
\pgfpathlineto{\pgfqpoint{2.105862in}{1.610339in}}%
\pgfpathlineto{\pgfqpoint{2.111195in}{1.605917in}}%
\pgfpathlineto{\pgfqpoint{2.116528in}{1.601522in}}%
\pgfpathlineto{\pgfqpoint{2.121862in}{1.597153in}}%
\pgfpathlineto{\pgfqpoint{2.127195in}{1.592809in}}%
\pgfpathlineto{\pgfqpoint{2.132528in}{1.588492in}}%
\pgfpathlineto{\pgfqpoint{2.137861in}{1.584199in}}%
\pgfpathlineto{\pgfqpoint{2.143194in}{1.579932in}}%
\pgfpathlineto{\pgfqpoint{2.148528in}{1.575690in}}%
\pgfpathlineto{\pgfqpoint{2.153861in}{1.571472in}}%
\pgfpathlineto{\pgfqpoint{2.159194in}{1.567279in}}%
\pgfpathlineto{\pgfqpoint{2.164527in}{1.563111in}}%
\pgfpathlineto{\pgfqpoint{2.169861in}{1.558966in}}%
\pgfpathlineto{\pgfqpoint{2.175194in}{1.554845in}}%
\pgfpathlineto{\pgfqpoint{2.180527in}{1.550748in}}%
\pgfpathlineto{\pgfqpoint{2.185860in}{1.546675in}}%
\pgfpathlineto{\pgfqpoint{2.191193in}{1.542624in}}%
\pgfpathlineto{\pgfqpoint{2.196527in}{1.538597in}}%
\pgfpathlineto{\pgfqpoint{2.201860in}{1.534593in}}%
\pgfpathlineto{\pgfqpoint{2.207193in}{1.530611in}}%
\pgfpathlineto{\pgfqpoint{2.212526in}{1.526652in}}%
\pgfpathlineto{\pgfqpoint{2.217859in}{1.522715in}}%
\pgfpathlineto{\pgfqpoint{2.223193in}{1.518800in}}%
\pgfpathlineto{\pgfqpoint{2.228526in}{1.514907in}}%
\pgfpathlineto{\pgfqpoint{2.233859in}{1.511036in}}%
\pgfpathlineto{\pgfqpoint{2.239192in}{1.507187in}}%
\pgfpathlineto{\pgfqpoint{2.244526in}{1.503359in}}%
\pgfpathlineto{\pgfqpoint{2.249859in}{1.499552in}}%
\pgfpathlineto{\pgfqpoint{2.255192in}{1.495766in}}%
\pgfpathlineto{\pgfqpoint{2.260525in}{1.492000in}}%
\pgfpathlineto{\pgfqpoint{2.265858in}{1.488256in}}%
\pgfpathlineto{\pgfqpoint{2.271192in}{1.484532in}}%
\pgfpathlineto{\pgfqpoint{2.276525in}{1.480829in}}%
\pgfpathlineto{\pgfqpoint{2.281858in}{1.477145in}}%
\pgfpathlineto{\pgfqpoint{2.287191in}{1.473482in}}%
\pgfpathlineto{\pgfqpoint{2.292524in}{1.469838in}}%
\pgfpathlineto{\pgfqpoint{2.297858in}{1.466214in}}%
\pgfpathlineto{\pgfqpoint{2.303191in}{1.462610in}}%
\pgfpathlineto{\pgfqpoint{2.308524in}{1.459025in}}%
\pgfpathlineto{\pgfqpoint{2.313857in}{1.455459in}}%
\pgfpathlineto{\pgfqpoint{2.319191in}{1.451913in}}%
\pgfpathlineto{\pgfqpoint{2.324524in}{1.448385in}}%
\pgfpathlineto{\pgfqpoint{2.329857in}{1.444876in}}%
\pgfpathlineto{\pgfqpoint{2.335190in}{1.441385in}}%
\pgfpathlineto{\pgfqpoint{2.340523in}{1.437914in}}%
\pgfpathlineto{\pgfqpoint{2.345857in}{1.434460in}}%
\pgfpathlineto{\pgfqpoint{2.351190in}{1.431024in}}%
\pgfpathlineto{\pgfqpoint{2.356523in}{1.427607in}}%
\pgfpathlineto{\pgfqpoint{2.361856in}{1.424208in}}%
\pgfpathlineto{\pgfqpoint{2.367190in}{1.420826in}}%
\pgfpathlineto{\pgfqpoint{2.372523in}{1.417462in}}%
\pgfpathlineto{\pgfqpoint{2.377856in}{1.414115in}}%
\pgfpathlineto{\pgfqpoint{2.383189in}{1.410785in}}%
\pgfpathlineto{\pgfqpoint{2.388522in}{1.407473in}}%
\pgfpathlineto{\pgfqpoint{2.393856in}{1.404178in}}%
\pgfpathlineto{\pgfqpoint{2.399189in}{1.400900in}}%
\pgfpathlineto{\pgfqpoint{2.404522in}{1.397639in}}%
\pgfpathlineto{\pgfqpoint{2.409855in}{1.394394in}}%
\pgfpathlineto{\pgfqpoint{2.415188in}{1.391166in}}%
\pgfpathlineto{\pgfqpoint{2.420522in}{1.387954in}}%
\pgfpathlineto{\pgfqpoint{2.425855in}{1.384759in}}%
\pgfpathlineto{\pgfqpoint{2.431188in}{1.381579in}}%
\pgfpathlineto{\pgfqpoint{2.436521in}{1.378416in}}%
\pgfpathlineto{\pgfqpoint{2.441855in}{1.375269in}}%
\pgfpathlineto{\pgfqpoint{2.447188in}{1.372137in}}%
\pgfpathlineto{\pgfqpoint{2.452521in}{1.369022in}}%
\pgfpathlineto{\pgfqpoint{2.457854in}{1.365921in}}%
\pgfpathlineto{\pgfqpoint{2.463187in}{1.362837in}}%
\pgfpathlineto{\pgfqpoint{2.468521in}{1.359767in}}%
\pgfpathlineto{\pgfqpoint{2.473854in}{1.356713in}}%
\pgfpathlineto{\pgfqpoint{2.479187in}{1.353674in}}%
\pgfpathlineto{\pgfqpoint{2.484520in}{1.350650in}}%
\pgfpathlineto{\pgfqpoint{2.489853in}{1.347641in}}%
\pgfpathlineto{\pgfqpoint{2.495187in}{1.344647in}}%
\pgfpathlineto{\pgfqpoint{2.500520in}{1.341667in}}%
\pgfpathlineto{\pgfqpoint{2.505853in}{1.338702in}}%
\pgfpathlineto{\pgfqpoint{2.511186in}{1.335751in}}%
\pgfpathlineto{\pgfqpoint{2.516520in}{1.332815in}}%
\pgfpathlineto{\pgfqpoint{2.521853in}{1.329893in}}%
\pgfpathlineto{\pgfqpoint{2.527186in}{1.326985in}}%
\pgfpathlineto{\pgfqpoint{2.532519in}{1.324091in}}%
\pgfpathlineto{\pgfqpoint{2.537852in}{1.321211in}}%
\pgfpathlineto{\pgfqpoint{2.543186in}{1.318345in}}%
\pgfpathlineto{\pgfqpoint{2.548519in}{1.315492in}}%
\pgfpathlineto{\pgfqpoint{2.553852in}{1.312654in}}%
\pgfpathlineto{\pgfqpoint{2.559185in}{1.309829in}}%
\pgfpathlineto{\pgfqpoint{2.564518in}{1.307017in}}%
\pgfpathlineto{\pgfqpoint{2.569852in}{1.304218in}}%
\pgfpathlineto{\pgfqpoint{2.575185in}{1.301433in}}%
\pgfpathlineto{\pgfqpoint{2.580518in}{1.298661in}}%
\pgfpathlineto{\pgfqpoint{2.585851in}{1.295903in}}%
\pgfpathlineto{\pgfqpoint{2.591185in}{1.293157in}}%
\pgfpathlineto{\pgfqpoint{2.596518in}{1.290424in}}%
\pgfpathlineto{\pgfqpoint{2.601851in}{1.287703in}}%
\pgfpathlineto{\pgfqpoint{2.607184in}{1.284996in}}%
\pgfpathlineto{\pgfqpoint{2.612517in}{1.282301in}}%
\pgfpathlineto{\pgfqpoint{2.617851in}{1.279619in}}%
\pgfpathlineto{\pgfqpoint{2.623184in}{1.276949in}}%
\pgfpathlineto{\pgfqpoint{2.628517in}{1.274291in}}%
\pgfpathlineto{\pgfqpoint{2.633850in}{1.271646in}}%
\pgfpathlineto{\pgfqpoint{2.639184in}{1.269013in}}%
\pgfpathlineto{\pgfqpoint{2.644517in}{1.266392in}}%
\pgfpathlineto{\pgfqpoint{2.649850in}{1.263783in}}%
\pgfpathlineto{\pgfqpoint{2.655183in}{1.261186in}}%
\pgfpathlineto{\pgfqpoint{2.660516in}{1.258601in}}%
\pgfpathlineto{\pgfqpoint{2.665850in}{1.256027in}}%
\pgfpathlineto{\pgfqpoint{2.671183in}{1.253466in}}%
\pgfpathlineto{\pgfqpoint{2.676516in}{1.250915in}}%
\pgfpathlineto{\pgfqpoint{2.681849in}{1.248377in}}%
\pgfpathlineto{\pgfqpoint{2.687182in}{1.245850in}}%
\pgfpathlineto{\pgfqpoint{2.692516in}{1.243334in}}%
\pgfpathlineto{\pgfqpoint{2.697849in}{1.240830in}}%
\pgfpathlineto{\pgfqpoint{2.703182in}{1.238336in}}%
\pgfpathlineto{\pgfqpoint{2.708515in}{1.235854in}}%
\pgfpathlineto{\pgfqpoint{2.713849in}{1.233383in}}%
\pgfpathlineto{\pgfqpoint{2.719182in}{1.230924in}}%
\pgfpathlineto{\pgfqpoint{2.724515in}{1.228474in}}%
\pgfpathlineto{\pgfqpoint{2.729848in}{1.226036in}}%
\pgfpathlineto{\pgfqpoint{2.735181in}{1.223609in}}%
\pgfpathlineto{\pgfqpoint{2.740515in}{1.221192in}}%
\pgfpathlineto{\pgfqpoint{2.745848in}{1.218786in}}%
\pgfpathlineto{\pgfqpoint{2.751181in}{1.216391in}}%
\pgfpathlineto{\pgfqpoint{2.756514in}{1.214006in}}%
\pgfpathlineto{\pgfqpoint{2.761847in}{1.211631in}}%
\pgfpathlineto{\pgfqpoint{2.767181in}{1.209267in}}%
\pgfpathlineto{\pgfqpoint{2.772514in}{1.206913in}}%
\pgfpathlineto{\pgfqpoint{2.777847in}{1.204570in}}%
\pgfpathlineto{\pgfqpoint{2.783180in}{1.202236in}}%
\pgfpathlineto{\pgfqpoint{2.788514in}{1.199913in}}%
\pgfpathlineto{\pgfqpoint{2.793847in}{1.197599in}}%
\pgfpathlineto{\pgfqpoint{2.799180in}{1.195296in}}%
\pgfpathlineto{\pgfqpoint{2.804513in}{1.193003in}}%
\pgfpathlineto{\pgfqpoint{2.809846in}{1.190719in}}%
\pgfpathlineto{\pgfqpoint{2.815180in}{1.188445in}}%
\pgfpathlineto{\pgfqpoint{2.820513in}{1.186181in}}%
\pgfpathlineto{\pgfqpoint{2.825846in}{1.183927in}}%
\pgfpathlineto{\pgfqpoint{2.831179in}{1.181682in}}%
\pgfpathlineto{\pgfqpoint{2.836512in}{1.179446in}}%
\pgfpathlineto{\pgfqpoint{2.841846in}{1.177221in}}%
\pgfpathlineto{\pgfqpoint{2.847179in}{1.175004in}}%
\pgfpathlineto{\pgfqpoint{2.852512in}{1.172797in}}%
\pgfpathlineto{\pgfqpoint{2.857845in}{1.170599in}}%
\pgfpathlineto{\pgfqpoint{2.863179in}{1.168410in}}%
\pgfpathlineto{\pgfqpoint{2.868512in}{1.166231in}}%
\pgfpathlineto{\pgfqpoint{2.873845in}{1.164061in}}%
\pgfpathlineto{\pgfqpoint{2.879178in}{1.161900in}}%
\pgfpathlineto{\pgfqpoint{2.884511in}{1.159747in}}%
\pgfpathlineto{\pgfqpoint{2.889845in}{1.157604in}}%
\pgfpathlineto{\pgfqpoint{2.895178in}{1.155470in}}%
\pgfpathlineto{\pgfqpoint{2.900511in}{1.153344in}}%
\pgfpathlineto{\pgfqpoint{2.905844in}{1.151227in}}%
\pgfpathlineto{\pgfqpoint{2.911178in}{1.149119in}}%
\pgfpathlineto{\pgfqpoint{2.916511in}{1.147020in}}%
\pgfpathlineto{\pgfqpoint{2.921844in}{1.144929in}}%
\pgfpathlineto{\pgfqpoint{2.927177in}{1.142847in}}%
\pgfpathlineto{\pgfqpoint{2.932510in}{1.140773in}}%
\pgfpathlineto{\pgfqpoint{2.937844in}{1.138708in}}%
\pgfpathlineto{\pgfqpoint{2.943177in}{1.136651in}}%
\pgfpathlineto{\pgfqpoint{2.948510in}{1.134603in}}%
\pgfpathlineto{\pgfqpoint{2.953843in}{1.132562in}}%
\pgfpathlineto{\pgfqpoint{2.959176in}{1.130530in}}%
\pgfpathlineto{\pgfqpoint{2.964510in}{1.128507in}}%
\pgfpathlineto{\pgfqpoint{2.969843in}{1.126491in}}%
\pgfpathlineto{\pgfqpoint{2.975176in}{1.124484in}}%
\pgfpathlineto{\pgfqpoint{2.980509in}{1.122484in}}%
\pgfpathlineto{\pgfqpoint{2.985843in}{1.120493in}}%
\pgfpathlineto{\pgfqpoint{2.991176in}{1.118510in}}%
\pgfpathlineto{\pgfqpoint{2.996509in}{1.116534in}}%
\pgfpathlineto{\pgfqpoint{3.001842in}{1.114567in}}%
\pgfpathlineto{\pgfqpoint{3.007175in}{1.112607in}}%
\pgfpathlineto{\pgfqpoint{3.012509in}{1.110655in}}%
\pgfpathlineto{\pgfqpoint{3.017842in}{1.108711in}}%
\pgfpathlineto{\pgfqpoint{3.023175in}{1.106774in}}%
\pgfpathlineto{\pgfqpoint{3.028508in}{1.104845in}}%
\pgfpathlineto{\pgfqpoint{3.033841in}{1.102924in}}%
\pgfpathlineto{\pgfqpoint{3.039175in}{1.101010in}}%
\pgfpathlineto{\pgfqpoint{3.044508in}{1.099104in}}%
\pgfpathlineto{\pgfqpoint{3.049841in}{1.097205in}}%
\pgfpathlineto{\pgfqpoint{3.055174in}{1.095313in}}%
\pgfpathlineto{\pgfqpoint{3.060508in}{1.093429in}}%
\pgfpathlineto{\pgfqpoint{3.065841in}{1.091552in}}%
\pgfpathlineto{\pgfqpoint{3.071174in}{1.089683in}}%
\pgfpathlineto{\pgfqpoint{3.076507in}{1.087821in}}%
\pgfpathlineto{\pgfqpoint{3.081840in}{1.085966in}}%
\pgfpathlineto{\pgfqpoint{3.087174in}{1.084118in}}%
\pgfpathlineto{\pgfqpoint{3.092507in}{1.082277in}}%
\pgfpathlineto{\pgfqpoint{3.097840in}{1.080444in}}%
\pgfpathlineto{\pgfqpoint{3.103173in}{1.078617in}}%
\pgfpathlineto{\pgfqpoint{3.108506in}{1.076797in}}%
\pgfpathlineto{\pgfqpoint{3.113840in}{1.074985in}}%
\pgfpathlineto{\pgfqpoint{3.119173in}{1.073179in}}%
\pgfpathlineto{\pgfqpoint{3.124506in}{1.071380in}}%
\pgfpathlineto{\pgfqpoint{3.129839in}{1.069588in}}%
\pgfpathlineto{\pgfqpoint{3.135173in}{1.067803in}}%
\pgfpathlineto{\pgfqpoint{3.140506in}{1.066024in}}%
\pgfpathlineto{\pgfqpoint{3.145839in}{1.064253in}}%
\pgfpathlineto{\pgfqpoint{3.151172in}{1.062487in}}%
\pgfpathlineto{\pgfqpoint{3.156505in}{1.060729in}}%
\pgfpathlineto{\pgfqpoint{3.161839in}{1.058977in}}%
\pgfpathlineto{\pgfqpoint{3.167172in}{1.057232in}}%
\pgfpathlineto{\pgfqpoint{3.172505in}{1.055493in}}%
\pgfpathlineto{\pgfqpoint{3.177838in}{1.053761in}}%
\pgfpathlineto{\pgfqpoint{3.183172in}{1.052035in}}%
\pgfpathlineto{\pgfqpoint{3.188505in}{1.050316in}}%
\pgfpathlineto{\pgfqpoint{3.193838in}{1.048603in}}%
\pgfpathlineto{\pgfqpoint{3.199171in}{1.046897in}}%
\pgfpathlineto{\pgfqpoint{3.204504in}{1.045196in}}%
\pgfpathlineto{\pgfqpoint{3.209838in}{1.043502in}}%
\pgfpathlineto{\pgfqpoint{3.215171in}{1.041815in}}%
\pgfpathlineto{\pgfqpoint{3.220504in}{1.040133in}}%
\pgfpathlineto{\pgfqpoint{3.225837in}{1.038458in}}%
\pgfpathlineto{\pgfqpoint{3.231170in}{1.036789in}}%
\pgfpathlineto{\pgfqpoint{3.236504in}{1.035126in}}%
\pgfpathlineto{\pgfqpoint{3.241837in}{1.033469in}}%
\pgfpathlineto{\pgfqpoint{3.247170in}{1.031818in}}%
\pgfpathlineto{\pgfqpoint{3.252503in}{1.030173in}}%
\pgfpathlineto{\pgfqpoint{3.257837in}{1.028534in}}%
\pgfpathlineto{\pgfqpoint{3.263170in}{1.026901in}}%
\pgfpathlineto{\pgfqpoint{3.268503in}{1.025274in}}%
\pgfpathlineto{\pgfqpoint{3.273836in}{1.023653in}}%
\pgfpathlineto{\pgfqpoint{3.279169in}{1.022037in}}%
\pgfpathlineto{\pgfqpoint{3.284503in}{1.020428in}}%
\pgfpathlineto{\pgfqpoint{3.289836in}{1.018824in}}%
\pgfpathlineto{\pgfqpoint{3.295169in}{1.017226in}}%
\pgfpathlineto{\pgfqpoint{3.300502in}{1.015634in}}%
\pgfpathlineto{\pgfqpoint{3.305835in}{1.014047in}}%
\pgfpathlineto{\pgfqpoint{3.311169in}{1.012466in}}%
\pgfpathlineto{\pgfqpoint{3.316502in}{1.010891in}}%
\pgfpathlineto{\pgfqpoint{3.321835in}{1.009321in}}%
\pgfpathlineto{\pgfqpoint{3.327168in}{1.007757in}}%
\pgfpathlineto{\pgfqpoint{3.332502in}{1.006199in}}%
\pgfpathlineto{\pgfqpoint{3.337835in}{1.004646in}}%
\pgfpathlineto{\pgfqpoint{3.343168in}{1.003098in}}%
\pgfpathlineto{\pgfqpoint{3.348501in}{1.001556in}}%
\pgfpathlineto{\pgfqpoint{3.353834in}{1.000019in}}%
\pgfpathlineto{\pgfqpoint{3.359168in}{0.998488in}}%
\pgfpathlineto{\pgfqpoint{3.364501in}{0.996962in}}%
\pgfpathlineto{\pgfqpoint{3.369834in}{0.995442in}}%
\pgfpathlineto{\pgfqpoint{3.375167in}{0.993927in}}%
\pgfpathlineto{\pgfqpoint{3.380500in}{0.992417in}}%
\pgfpathlineto{\pgfqpoint{3.385834in}{0.990912in}}%
\pgfpathlineto{\pgfqpoint{3.391167in}{0.989413in}}%
\pgfpathlineto{\pgfqpoint{3.396500in}{0.987918in}}%
\pgfpathlineto{\pgfqpoint{3.401833in}{0.986429in}}%
\pgfpathlineto{\pgfqpoint{3.407167in}{0.984945in}}%
\pgfpathlineto{\pgfqpoint{3.412500in}{0.983467in}}%
\pgfpathlineto{\pgfqpoint{3.417833in}{0.981993in}}%
\pgfpathlineto{\pgfqpoint{3.423166in}{0.980524in}}%
\pgfpathlineto{\pgfqpoint{3.428499in}{0.979061in}}%
\pgfpathlineto{\pgfqpoint{3.433833in}{0.977602in}}%
\pgfpathlineto{\pgfqpoint{3.439166in}{0.976149in}}%
\pgfpathlineto{\pgfqpoint{3.444499in}{0.974700in}}%
\pgfpathlineto{\pgfqpoint{3.449832in}{0.973256in}}%
\pgfpathlineto{\pgfqpoint{3.455166in}{0.971818in}}%
\pgfpathlineto{\pgfqpoint{3.460499in}{0.970384in}}%
\pgfpathlineto{\pgfqpoint{3.465832in}{0.968955in}}%
\pgfpathlineto{\pgfqpoint{3.471165in}{0.967531in}}%
\pgfpathlineto{\pgfqpoint{3.476498in}{0.966111in}}%
\pgfpathlineto{\pgfqpoint{3.481832in}{0.964697in}}%
\pgfpathlineto{\pgfqpoint{3.487165in}{0.963287in}}%
\pgfpathlineto{\pgfqpoint{3.492498in}{0.961882in}}%
\pgfpathlineto{\pgfqpoint{3.497831in}{0.960482in}}%
\pgfpathlineto{\pgfqpoint{3.503164in}{0.959086in}}%
\pgfpathlineto{\pgfqpoint{3.508498in}{0.957696in}}%
\pgfpathlineto{\pgfqpoint{3.513831in}{0.956309in}}%
\pgfpathlineto{\pgfqpoint{3.519164in}{0.954928in}}%
\pgfpathlineto{\pgfqpoint{3.524497in}{0.953551in}}%
\pgfpathlineto{\pgfqpoint{3.529831in}{0.952178in}}%
\pgfpathlineto{\pgfqpoint{3.535164in}{0.950810in}}%
\pgfpathlineto{\pgfqpoint{3.540497in}{0.949447in}}%
\pgfpathlineto{\pgfqpoint{3.545830in}{0.948088in}}%
\pgfpathlineto{\pgfqpoint{3.551163in}{0.946734in}}%
\pgfpathlineto{\pgfqpoint{3.556497in}{0.945384in}}%
\pgfpathlineto{\pgfqpoint{3.561830in}{0.944039in}}%
\pgfpathlineto{\pgfqpoint{3.567163in}{0.942698in}}%
\pgfpathlineto{\pgfqpoint{3.572496in}{0.941361in}}%
\pgfpathlineto{\pgfqpoint{3.577829in}{0.940029in}}%
\pgfpathlineto{\pgfqpoint{3.583163in}{0.938701in}}%
\pgfpathlineto{\pgfqpoint{3.588496in}{0.937377in}}%
\pgfpathlineto{\pgfqpoint{3.593829in}{0.936058in}}%
\pgfpathlineto{\pgfqpoint{3.599162in}{0.934743in}}%
\pgfpathlineto{\pgfqpoint{3.604496in}{0.933433in}}%
\pgfpathlineto{\pgfqpoint{3.609829in}{0.932126in}}%
\pgfpathlineto{\pgfqpoint{3.615162in}{0.930824in}}%
\pgfpathlineto{\pgfqpoint{3.620495in}{0.929526in}}%
\pgfpathlineto{\pgfqpoint{3.625828in}{0.928233in}}%
\pgfpathlineto{\pgfqpoint{3.631162in}{0.926943in}}%
\pgfpathlineto{\pgfqpoint{3.636495in}{0.925658in}}%
\pgfpathlineto{\pgfqpoint{3.641828in}{0.924376in}}%
\pgfpathlineto{\pgfqpoint{3.647161in}{0.923099in}}%
\pgfpathlineto{\pgfqpoint{3.652494in}{0.921826in}}%
\pgfpathlineto{\pgfqpoint{3.657828in}{0.920557in}}%
\pgfpathlineto{\pgfqpoint{3.663161in}{0.919292in}}%
\pgfpathlineto{\pgfqpoint{3.668494in}{0.918031in}}%
\pgfpathlineto{\pgfqpoint{3.673827in}{0.916774in}}%
\pgfpathlineto{\pgfqpoint{3.679161in}{0.915521in}}%
\pgfpathlineto{\pgfqpoint{3.684494in}{0.914273in}}%
\pgfpathlineto{\pgfqpoint{3.689827in}{0.913028in}}%
\pgfpathlineto{\pgfqpoint{3.695160in}{0.911787in}}%
\pgfpathlineto{\pgfqpoint{3.700493in}{0.910549in}}%
\pgfpathlineto{\pgfqpoint{3.705827in}{0.909316in}}%
\pgfpathlineto{\pgfqpoint{3.711160in}{0.908087in}}%
\pgfpathlineto{\pgfqpoint{3.716493in}{0.906862in}}%
\pgfpathlineto{\pgfqpoint{3.721826in}{0.905640in}}%
\pgfpathlineto{\pgfqpoint{3.727160in}{0.904422in}}%
\pgfpathlineto{\pgfqpoint{3.732493in}{0.903208in}}%
\pgfpathlineto{\pgfqpoint{3.737826in}{0.901998in}}%
\pgfpathlineto{\pgfqpoint{3.743159in}{0.900792in}}%
\pgfpathlineto{\pgfqpoint{3.748492in}{0.899589in}}%
\pgfpathlineto{\pgfqpoint{3.753826in}{0.898390in}}%
\pgfpathlineto{\pgfqpoint{3.759159in}{0.897195in}}%
\pgfpathlineto{\pgfqpoint{3.764492in}{0.896003in}}%
\pgfpathlineto{\pgfqpoint{3.769825in}{0.894816in}}%
\pgfpathlineto{\pgfqpoint{3.775158in}{0.893632in}}%
\pgfpathlineto{\pgfqpoint{3.780492in}{0.892451in}}%
\pgfpathlineto{\pgfqpoint{3.785825in}{0.891274in}}%
\pgfpathlineto{\pgfqpoint{3.791158in}{0.890101in}}%
\pgfpathlineto{\pgfqpoint{3.796491in}{0.888931in}}%
\pgfpathlineto{\pgfqpoint{3.801825in}{0.887765in}}%
\pgfpathlineto{\pgfqpoint{3.807158in}{0.886603in}}%
\pgfpathlineto{\pgfqpoint{3.812491in}{0.885444in}}%
\pgfpathlineto{\pgfqpoint{3.817824in}{0.884289in}}%
\pgfpathlineto{\pgfqpoint{3.823157in}{0.883137in}}%
\pgfpathlineto{\pgfqpoint{3.828491in}{0.881989in}}%
\pgfpathlineto{\pgfqpoint{3.833824in}{0.880844in}}%
\pgfpathlineto{\pgfqpoint{3.839157in}{0.879702in}}%
\pgfpathlineto{\pgfqpoint{3.844490in}{0.878565in}}%
\pgfpathlineto{\pgfqpoint{3.849823in}{0.877430in}}%
\pgfpathlineto{\pgfqpoint{3.855157in}{0.876299in}}%
\pgfpathlineto{\pgfqpoint{3.860490in}{0.875171in}}%
\pgfpathlineto{\pgfqpoint{3.865823in}{0.874047in}}%
\pgfpathlineto{\pgfqpoint{3.871156in}{0.872926in}}%
\pgfpathlineto{\pgfqpoint{3.876490in}{0.871809in}}%
\pgfpathlineto{\pgfqpoint{3.881823in}{0.870695in}}%
\pgfpathlineto{\pgfqpoint{3.887156in}{0.869584in}}%
\pgfpathlineto{\pgfqpoint{3.892489in}{0.868476in}}%
\pgfpathlineto{\pgfqpoint{3.897822in}{0.867372in}}%
\pgfpathlineto{\pgfqpoint{3.903156in}{0.866271in}}%
\pgfpathlineto{\pgfqpoint{3.908489in}{0.865174in}}%
\pgfpathlineto{\pgfqpoint{3.913822in}{0.864079in}}%
\pgfpathlineto{\pgfqpoint{3.919155in}{0.862988in}}%
\pgfpathlineto{\pgfqpoint{3.924488in}{0.861901in}}%
\pgfpathlineto{\pgfqpoint{3.929822in}{0.860816in}}%
\pgfpathlineto{\pgfqpoint{3.935155in}{0.859734in}}%
\pgfpathlineto{\pgfqpoint{3.940488in}{0.858656in}}%
\pgfpathlineto{\pgfqpoint{3.945821in}{0.857581in}}%
\pgfpathlineto{\pgfqpoint{3.951155in}{0.856509in}}%
\pgfpathlineto{\pgfqpoint{3.956488in}{0.855440in}}%
\pgfpathlineto{\pgfqpoint{3.961821in}{0.854375in}}%
\pgfpathlineto{\pgfqpoint{3.967154in}{0.853312in}}%
\pgfpathlineto{\pgfqpoint{3.972487in}{0.852253in}}%
\pgfpathlineto{\pgfqpoint{3.977821in}{0.851197in}}%
\pgfpathlineto{\pgfqpoint{3.983154in}{0.850143in}}%
\pgfpathlineto{\pgfqpoint{3.988487in}{0.849093in}}%
\pgfpathlineto{\pgfqpoint{3.993820in}{0.848046in}}%
\pgfpathlineto{\pgfqpoint{3.999154in}{0.847002in}}%
\pgfpathlineto{\pgfqpoint{4.004487in}{0.845961in}}%
\pgfpathlineto{\pgfqpoint{4.009820in}{0.844923in}}%
\pgfpathlineto{\pgfqpoint{4.015153in}{0.843888in}}%
\pgfpathlineto{\pgfqpoint{4.020486in}{0.842856in}}%
\pgfpathlineto{\pgfqpoint{4.025820in}{0.841827in}}%
\pgfpathlineto{\pgfqpoint{4.031153in}{0.840801in}}%
\pgfpathlineto{\pgfqpoint{4.036486in}{0.839777in}}%
\pgfpathlineto{\pgfqpoint{4.041819in}{0.838757in}}%
\pgfpathlineto{\pgfqpoint{4.047152in}{0.837740in}}%
\pgfpathlineto{\pgfqpoint{4.052486in}{0.836725in}}%
\pgfpathlineto{\pgfqpoint{4.057819in}{0.835714in}}%
\pgfpathlineto{\pgfqpoint{4.063152in}{0.834705in}}%
\pgfpathlineto{\pgfqpoint{4.068485in}{0.833700in}}%
\pgfpathlineto{\pgfqpoint{4.073819in}{0.832697in}}%
\pgfpathlineto{\pgfqpoint{4.079152in}{0.831697in}}%
\pgfpathlineto{\pgfqpoint{4.084485in}{0.830699in}}%
\pgfpathlineto{\pgfqpoint{4.089818in}{0.829705in}}%
\pgfpathlineto{\pgfqpoint{4.095151in}{0.828714in}}%
\pgfpathlineto{\pgfqpoint{4.100485in}{0.827725in}}%
\pgfpathlineto{\pgfqpoint{4.105818in}{0.826739in}}%
\pgfpathlineto{\pgfqpoint{4.111151in}{0.825756in}}%
\pgfpathlineto{\pgfqpoint{4.116484in}{0.824775in}}%
\pgfpathlineto{\pgfqpoint{4.121817in}{0.823797in}}%
\pgfpathlineto{\pgfqpoint{4.127151in}{0.822823in}}%
\pgfpathlineto{\pgfqpoint{4.132484in}{0.821850in}}%
\pgfpathlineto{\pgfqpoint{4.137817in}{0.820881in}}%
\pgfpathlineto{\pgfqpoint{4.143150in}{0.819914in}}%
\pgfpathlineto{\pgfqpoint{4.148484in}{0.818950in}}%
\pgfpathlineto{\pgfqpoint{4.153817in}{0.817989in}}%
\pgfpathlineto{\pgfqpoint{4.159150in}{0.817030in}}%
\pgfpathlineto{\pgfqpoint{4.164483in}{0.816074in}}%
\pgfpathlineto{\pgfqpoint{4.169816in}{0.815120in}}%
\pgfpathlineto{\pgfqpoint{4.175150in}{0.814170in}}%
\pgfpathlineto{\pgfqpoint{4.180483in}{0.813221in}}%
\pgfpathlineto{\pgfqpoint{4.185816in}{0.812276in}}%
\pgfpathlineto{\pgfqpoint{4.191149in}{0.811333in}}%
\pgfpathlineto{\pgfqpoint{4.196482in}{0.810393in}}%
\pgfpathlineto{\pgfqpoint{4.201816in}{0.809455in}}%
\pgfpathlineto{\pgfqpoint{4.207149in}{0.808520in}}%
\pgfpathlineto{\pgfqpoint{4.212482in}{0.807587in}}%
\pgfpathlineto{\pgfqpoint{4.217815in}{0.806657in}}%
\pgfpathlineto{\pgfqpoint{4.223149in}{0.805730in}}%
\pgfpathlineto{\pgfqpoint{4.228482in}{0.804805in}}%
\pgfpathlineto{\pgfqpoint{4.233815in}{0.803882in}}%
\pgfpathlineto{\pgfqpoint{4.239148in}{0.802962in}}%
\pgfpathlineto{\pgfqpoint{4.244481in}{0.802045in}}%
\pgfpathlineto{\pgfqpoint{4.249815in}{0.801130in}}%
\pgfpathlineto{\pgfqpoint{4.255148in}{0.800218in}}%
\pgfpathlineto{\pgfqpoint{4.260481in}{0.799308in}}%
\pgfpathlineto{\pgfqpoint{4.265814in}{0.798400in}}%
\pgfpathlineto{\pgfqpoint{4.271148in}{0.797495in}}%
\pgfpathlineto{\pgfqpoint{4.276481in}{0.796592in}}%
\pgfpathlineto{\pgfqpoint{4.281814in}{0.795692in}}%
\pgfpathlineto{\pgfqpoint{4.287147in}{0.794795in}}%
\pgfpathlineto{\pgfqpoint{4.292480in}{0.793899in}}%
\pgfpathlineto{\pgfqpoint{4.297814in}{0.793006in}}%
\pgfpathlineto{\pgfqpoint{4.303147in}{0.792116in}}%
\pgfpathlineto{\pgfqpoint{4.308480in}{0.791228in}}%
\pgfpathlineto{\pgfqpoint{4.313813in}{0.790342in}}%
\pgfpathlineto{\pgfqpoint{4.319146in}{0.789458in}}%
\pgfpathlineto{\pgfqpoint{4.324480in}{0.788577in}}%
\pgfpathlineto{\pgfqpoint{4.329813in}{0.787699in}}%
\pgfpathlineto{\pgfqpoint{4.335146in}{0.786822in}}%
\pgfpathlineto{\pgfqpoint{4.340479in}{0.785948in}}%
\pgfpathlineto{\pgfqpoint{4.345813in}{0.785076in}}%
\pgfpathlineto{\pgfqpoint{4.351146in}{0.784207in}}%
\pgfpathlineto{\pgfqpoint{4.356479in}{0.783340in}}%
\pgfpathlineto{\pgfqpoint{4.361812in}{0.782475in}}%
\pgfpathlineto{\pgfqpoint{4.367145in}{0.781612in}}%
\pgfpathlineto{\pgfqpoint{4.372479in}{0.780752in}}%
\pgfpathlineto{\pgfqpoint{4.377812in}{0.779894in}}%
\pgfpathlineto{\pgfqpoint{4.383145in}{0.779038in}}%
\pgfpathlineto{\pgfqpoint{4.388478in}{0.778185in}}%
\pgfpathlineto{\pgfqpoint{4.393811in}{0.777333in}}%
\pgfpathlineto{\pgfqpoint{4.399145in}{0.776484in}}%
\pgfpathlineto{\pgfqpoint{4.404478in}{0.775638in}}%
\pgfpathlineto{\pgfqpoint{4.409811in}{0.774793in}}%
\pgfpathlineto{\pgfqpoint{4.415144in}{0.773950in}}%
\pgfpathlineto{\pgfqpoint{4.420478in}{0.773110in}}%
\pgfpathlineto{\pgfqpoint{4.425811in}{0.772272in}}%
\pgfpathlineto{\pgfqpoint{4.431144in}{0.771436in}}%
\pgfpathlineto{\pgfqpoint{4.436477in}{0.770603in}}%
\pgfpathlineto{\pgfqpoint{4.441810in}{0.769771in}}%
\pgfpathlineto{\pgfqpoint{4.447144in}{0.768942in}}%
\pgfpathlineto{\pgfqpoint{4.452477in}{0.768114in}}%
\pgfpathlineto{\pgfqpoint{4.457810in}{0.767289in}}%
\pgfpathlineto{\pgfqpoint{4.463143in}{0.766466in}}%
\pgfpathlineto{\pgfqpoint{4.468476in}{0.765645in}}%
\pgfpathlineto{\pgfqpoint{4.473810in}{0.764827in}}%
\pgfpathlineto{\pgfqpoint{4.479143in}{0.764010in}}%
\pgfpathlineto{\pgfqpoint{4.484476in}{0.763195in}}%
\pgfpathlineto{\pgfqpoint{4.489809in}{0.762383in}}%
\pgfpathlineto{\pgfqpoint{4.495143in}{0.761573in}}%
\pgfpathlineto{\pgfqpoint{4.500476in}{0.760764in}}%
\pgfpathlineto{\pgfqpoint{4.505809in}{0.759958in}}%
\pgfpathlineto{\pgfqpoint{4.511142in}{0.759154in}}%
\pgfpathlineto{\pgfqpoint{4.516475in}{0.758352in}}%
\pgfpathlineto{\pgfqpoint{4.521809in}{0.757551in}}%
\pgfpathlineto{\pgfqpoint{4.527142in}{0.756753in}}%
\pgfpathlineto{\pgfqpoint{4.532475in}{0.755957in}}%
\pgfpathlineto{\pgfqpoint{4.537808in}{0.755163in}}%
\pgfpathlineto{\pgfqpoint{4.543142in}{0.754371in}}%
\pgfpathlineto{\pgfqpoint{4.548475in}{0.753581in}}%
\pgfpathlineto{\pgfqpoint{4.553808in}{0.752793in}}%
\pgfpathlineto{\pgfqpoint{4.559141in}{0.752007in}}%
\pgfpathlineto{\pgfqpoint{4.564474in}{0.751223in}}%
\pgfpathlineto{\pgfqpoint{4.569808in}{0.750441in}}%
\pgfpathlineto{\pgfqpoint{4.575141in}{0.749660in}}%
\pgfpathlineto{\pgfqpoint{4.580474in}{0.748882in}}%
\pgfpathlineto{\pgfqpoint{4.585807in}{0.748106in}}%
\pgfpathlineto{\pgfqpoint{4.591140in}{0.747332in}}%
\pgfpathlineto{\pgfqpoint{4.596474in}{0.746559in}}%
\pgfpathlineto{\pgfqpoint{4.601807in}{0.745789in}}%
\pgfpathlineto{\pgfqpoint{4.607140in}{0.745020in}}%
\pgfpathlineto{\pgfqpoint{4.612473in}{0.744254in}}%
\pgfpathlineto{\pgfqpoint{4.617807in}{0.743489in}}%
\pgfpathlineto{\pgfqpoint{4.623140in}{0.742726in}}%
\pgfpathlineto{\pgfqpoint{4.628473in}{0.741965in}}%
\pgfpathlineto{\pgfqpoint{4.633806in}{0.741206in}}%
\pgfpathlineto{\pgfqpoint{4.639139in}{0.740449in}}%
\pgfpathlineto{\pgfqpoint{4.644473in}{0.739693in}}%
\pgfpathlineto{\pgfqpoint{4.649806in}{0.738940in}}%
\pgfpathlineto{\pgfqpoint{4.655139in}{0.738188in}}%
\pgfpathlineto{\pgfqpoint{4.660472in}{0.737439in}}%
\pgfpathlineto{\pgfqpoint{4.665805in}{0.736691in}}%
\pgfpathlineto{\pgfqpoint{4.671139in}{0.735945in}}%
\pgfpathlineto{\pgfqpoint{4.676472in}{0.735200in}}%
\pgfpathlineto{\pgfqpoint{4.681805in}{0.734458in}}%
\pgfpathlineto{\pgfqpoint{4.687138in}{0.733717in}}%
\pgfpathlineto{\pgfqpoint{4.692472in}{0.732979in}}%
\pgfpathlineto{\pgfqpoint{4.697805in}{0.732242in}}%
\pgfpathlineto{\pgfqpoint{4.703138in}{0.731506in}}%
\pgfpathlineto{\pgfqpoint{4.708471in}{0.730773in}}%
\pgfpathlineto{\pgfqpoint{4.713804in}{0.730041in}}%
\pgfpathlineto{\pgfqpoint{4.719138in}{0.729311in}}%
\pgfpathlineto{\pgfqpoint{4.724471in}{0.728583in}}%
\pgfpathlineto{\pgfqpoint{4.729804in}{0.727857in}}%
\pgfpathlineto{\pgfqpoint{4.735137in}{0.727132in}}%
\pgfpathlineto{\pgfqpoint{4.740470in}{0.726410in}}%
\pgfpathlineto{\pgfqpoint{4.745804in}{0.725689in}}%
\pgfpathlineto{\pgfqpoint{4.751137in}{0.724969in}}%
\pgfpathlineto{\pgfqpoint{4.756470in}{0.724252in}}%
\pgfpathlineto{\pgfqpoint{4.761803in}{0.723536in}}%
\pgfpathlineto{\pgfqpoint{4.767137in}{0.722822in}}%
\pgfpathlineto{\pgfqpoint{4.772470in}{0.722109in}}%
\pgfpathlineto{\pgfqpoint{4.777803in}{0.721398in}}%
\pgfpathlineto{\pgfqpoint{4.783136in}{0.720689in}}%
\pgfpathlineto{\pgfqpoint{4.788469in}{0.719982in}}%
\pgfpathlineto{\pgfqpoint{4.793803in}{0.719276in}}%
\pgfpathlineto{\pgfqpoint{4.799136in}{0.718572in}}%
\pgfpathlineto{\pgfqpoint{4.804469in}{0.717870in}}%
\pgfpathlineto{\pgfqpoint{4.809802in}{0.717169in}}%
\pgfpathlineto{\pgfqpoint{4.815136in}{0.716470in}}%
\pgfpathlineto{\pgfqpoint{4.820469in}{0.715773in}}%
\pgfpathlineto{\pgfqpoint{4.825802in}{0.715077in}}%
\pgfpathlineto{\pgfqpoint{4.831135in}{0.714383in}}%
\pgfpathlineto{\pgfqpoint{4.836468in}{0.713691in}}%
\pgfpathlineto{\pgfqpoint{4.841802in}{0.713000in}}%
\pgfpathlineto{\pgfqpoint{4.847135in}{0.712311in}}%
\pgfpathlineto{\pgfqpoint{4.852468in}{0.711624in}}%
\pgfpathlineto{\pgfqpoint{4.857801in}{0.710938in}}%
\pgfpathlineto{\pgfqpoint{4.863134in}{0.710253in}}%
\pgfpathlineto{\pgfqpoint{4.868468in}{0.709571in}}%
\pgfpathlineto{\pgfqpoint{4.873801in}{0.708890in}}%
\pgfpathlineto{\pgfqpoint{4.879134in}{0.708210in}}%
\pgfpathlineto{\pgfqpoint{4.884467in}{0.707532in}}%
\pgfpathlineto{\pgfqpoint{4.889801in}{0.706856in}}%
\pgfpathlineto{\pgfqpoint{4.895134in}{0.706181in}}%
\pgfpathlineto{\pgfqpoint{4.900467in}{0.705508in}}%
\pgfpathlineto{\pgfqpoint{4.905800in}{0.704837in}}%
\pgfpathlineto{\pgfqpoint{4.911133in}{0.704167in}}%
\pgfpathlineto{\pgfqpoint{4.916467in}{0.703498in}}%
\pgfpathlineto{\pgfqpoint{4.921800in}{0.702831in}}%
\pgfpathlineto{\pgfqpoint{4.927133in}{0.702166in}}%
\pgfpathlineto{\pgfqpoint{4.932466in}{0.701502in}}%
\pgfpathlineto{\pgfqpoint{4.937799in}{0.700840in}}%
\pgfpathlineto{\pgfqpoint{4.943133in}{0.700179in}}%
\pgfpathlineto{\pgfqpoint{4.948466in}{0.699520in}}%
\pgfpathlineto{\pgfqpoint{4.953799in}{0.698862in}}%
\pgfpathlineto{\pgfqpoint{4.959132in}{0.698206in}}%
\pgfpathlineto{\pgfqpoint{4.964466in}{0.697551in}}%
\pgfpathlineto{\pgfqpoint{4.969799in}{0.696898in}}%
\pgfpathlineto{\pgfqpoint{4.975132in}{0.696246in}}%
\pgfpathlineto{\pgfqpoint{4.980465in}{0.695596in}}%
\pgfpathlineto{\pgfqpoint{4.985798in}{0.694948in}}%
\pgfpathlineto{\pgfqpoint{4.991132in}{0.694300in}}%
\pgfpathlineto{\pgfqpoint{4.996465in}{0.693655in}}%
\pgfpathlineto{\pgfqpoint{5.001798in}{0.693010in}}%
\pgfpathlineto{\pgfqpoint{5.007131in}{0.692367in}}%
\pgfpathlineto{\pgfqpoint{5.012464in}{0.691726in}}%
\pgfpathlineto{\pgfqpoint{5.017798in}{0.691086in}}%
\pgfpathlineto{\pgfqpoint{5.023131in}{0.690448in}}%
\pgfpathlineto{\pgfqpoint{5.028464in}{0.689811in}}%
\pgfpathlineto{\pgfqpoint{5.033797in}{0.689175in}}%
\pgfpathlineto{\pgfqpoint{5.039131in}{0.688541in}}%
\pgfpathlineto{\pgfqpoint{5.044464in}{0.687908in}}%
\pgfpathlineto{\pgfqpoint{5.049797in}{0.687277in}}%
\pgfpathlineto{\pgfqpoint{5.055130in}{0.686647in}}%
\pgfpathlineto{\pgfqpoint{5.060463in}{0.686019in}}%
\pgfpathlineto{\pgfqpoint{5.065797in}{0.685392in}}%
\pgfpathlineto{\pgfqpoint{5.071130in}{0.684766in}}%
\pgfpathlineto{\pgfqpoint{5.076463in}{0.684142in}}%
\pgfpathlineto{\pgfqpoint{5.081796in}{0.683519in}}%
\pgfpathlineto{\pgfqpoint{5.087130in}{0.682898in}}%
\pgfpathlineto{\pgfqpoint{5.092463in}{0.682278in}}%
\pgfpathlineto{\pgfqpoint{5.097796in}{0.681659in}}%
\pgfpathlineto{\pgfqpoint{5.103129in}{0.681042in}}%
\pgfpathlineto{\pgfqpoint{5.108462in}{0.680426in}}%
\pgfpathlineto{\pgfqpoint{5.113796in}{0.679812in}}%
\pgfpathlineto{\pgfqpoint{5.119129in}{0.679199in}}%
\pgfpathlineto{\pgfqpoint{5.124462in}{0.678587in}}%
\pgfpathlineto{\pgfqpoint{5.129795in}{0.677977in}}%
\pgfpathlineto{\pgfqpoint{5.135128in}{0.677368in}}%
\pgfpathlineto{\pgfqpoint{5.140462in}{0.676760in}}%
\pgfpathlineto{\pgfqpoint{5.145795in}{0.676154in}}%
\pgfpathlineto{\pgfqpoint{5.151128in}{0.675549in}}%
\pgfpathlineto{\pgfqpoint{5.156461in}{0.674945in}}%
\pgfpathlineto{\pgfqpoint{5.161795in}{0.674342in}}%
\pgfpathlineto{\pgfqpoint{5.167128in}{0.673741in}}%
\pgfpathlineto{\pgfqpoint{5.172461in}{0.673142in}}%
\pgfpathlineto{\pgfqpoint{5.177794in}{0.672543in}}%
\pgfpathlineto{\pgfqpoint{5.183127in}{0.671946in}}%
\pgfpathlineto{\pgfqpoint{5.188461in}{0.671351in}}%
\pgfpathlineto{\pgfqpoint{5.193794in}{0.670756in}}%
\pgfpathlineto{\pgfqpoint{5.199127in}{0.670163in}}%
\pgfpathlineto{\pgfqpoint{5.204460in}{0.669571in}}%
\pgfpathlineto{\pgfqpoint{5.209793in}{0.668981in}}%
\pgfpathlineto{\pgfqpoint{5.215127in}{0.668391in}}%
\pgfpathlineto{\pgfqpoint{5.220460in}{0.667803in}}%
\pgfpathlineto{\pgfqpoint{5.225793in}{0.667217in}}%
\pgfpathlineto{\pgfqpoint{5.231126in}{0.666631in}}%
\pgfpathlineto{\pgfqpoint{5.236460in}{0.666047in}}%
\pgfpathlineto{\pgfqpoint{5.241793in}{0.665464in}}%
\pgfpathlineto{\pgfqpoint{5.247126in}{0.664882in}}%
\pgfpathlineto{\pgfqpoint{5.252459in}{0.664302in}}%
\pgfpathlineto{\pgfqpoint{5.257792in}{0.663723in}}%
\pgfpathlineto{\pgfqpoint{5.263126in}{0.663145in}}%
\pgfpathlineto{\pgfqpoint{5.268459in}{0.662568in}}%
\pgfpathlineto{\pgfqpoint{5.273792in}{0.661993in}}%
\pgfpathlineto{\pgfqpoint{5.279125in}{0.661419in}}%
\pgfpathlineto{\pgfqpoint{5.284458in}{0.660846in}}%
\pgfpathlineto{\pgfqpoint{5.289792in}{0.660274in}}%
\pgfpathlineto{\pgfqpoint{5.295125in}{0.659704in}}%
\pgfpathlineto{\pgfqpoint{5.300458in}{0.659135in}}%
\pgfpathlineto{\pgfqpoint{5.305791in}{0.658567in}}%
\pgfpathlineto{\pgfqpoint{5.311125in}{0.658000in}}%
\pgfpathlineto{\pgfqpoint{5.316458in}{0.657434in}}%
\pgfpathlineto{\pgfqpoint{5.321791in}{0.656870in}}%
\pgfpathlineto{\pgfqpoint{5.327124in}{0.656307in}}%
\pgfpathlineto{\pgfqpoint{5.332457in}{0.655745in}}%
\pgfpathlineto{\pgfqpoint{5.337791in}{0.655184in}}%
\pgfpathlineto{\pgfqpoint{5.343124in}{0.654624in}}%
\pgfpathlineto{\pgfqpoint{5.348457in}{0.654066in}}%
\pgfpathlineto{\pgfqpoint{5.353790in}{0.653509in}}%
\pgfpathlineto{\pgfqpoint{5.359124in}{0.652953in}}%
\pgfpathlineto{\pgfqpoint{5.364457in}{0.652398in}}%
\pgfpathlineto{\pgfqpoint{5.369790in}{0.651844in}}%
\pgfpathlineto{\pgfqpoint{5.375123in}{0.651292in}}%
\pgfpathlineto{\pgfqpoint{5.380456in}{0.650740in}}%
\pgfpathlineto{\pgfqpoint{5.385790in}{0.650190in}}%
\pgfpathlineto{\pgfqpoint{5.391123in}{0.649641in}}%
\pgfpathlineto{\pgfqpoint{5.396456in}{0.649093in}}%
\pgfpathlineto{\pgfqpoint{5.401789in}{0.648547in}}%
\pgfpathlineto{\pgfqpoint{5.407122in}{0.648001in}}%
\pgfpathlineto{\pgfqpoint{5.412456in}{0.647457in}}%
\pgfpathlineto{\pgfqpoint{5.417789in}{0.646913in}}%
\pgfpathlineto{\pgfqpoint{5.423122in}{0.646371in}}%
\pgfpathlineto{\pgfqpoint{5.428455in}{0.645830in}}%
\pgfpathlineto{\pgfqpoint{5.433789in}{0.645290in}}%
\pgfpathlineto{\pgfqpoint{5.439122in}{0.644751in}}%
\pgfpathlineto{\pgfqpoint{5.444455in}{0.644214in}}%
\pgfpathlineto{\pgfqpoint{5.449788in}{0.643677in}}%
\pgfpathlineto{\pgfqpoint{5.455121in}{0.643142in}}%
\pgfpathlineto{\pgfqpoint{5.460455in}{0.642608in}}%
\pgfpathlineto{\pgfqpoint{5.465788in}{0.642074in}}%
\pgfpathlineto{\pgfqpoint{5.471121in}{0.641542in}}%
\pgfpathlineto{\pgfqpoint{5.476454in}{0.641011in}}%
\pgfpathlineto{\pgfqpoint{5.481787in}{0.640481in}}%
\pgfpathlineto{\pgfqpoint{5.487121in}{0.639953in}}%
\pgfpathlineto{\pgfqpoint{5.492454in}{0.639425in}}%
\pgfpathlineto{\pgfqpoint{5.497787in}{0.638898in}}%
\pgfpathlineto{\pgfqpoint{5.503120in}{0.638373in}}%
\pgfpathlineto{\pgfqpoint{5.508454in}{0.637848in}}%
\pgfpathlineto{\pgfqpoint{5.513787in}{0.637325in}}%
\pgfpathlineto{\pgfqpoint{5.519120in}{0.636803in}}%
\pgfpathlineto{\pgfqpoint{5.524453in}{0.636281in}}%
\pgfpathlineto{\pgfqpoint{5.529786in}{0.635761in}}%
\pgfpathlineto{\pgfqpoint{5.535120in}{0.635242in}}%
\pgfpathlineto{\pgfqpoint{5.540453in}{0.634724in}}%
\pgfpathlineto{\pgfqpoint{5.545786in}{0.634207in}}%
\pgfpathlineto{\pgfqpoint{5.551119in}{0.633691in}}%
\pgfpathlineto{\pgfqpoint{5.556452in}{0.633176in}}%
\pgfpathlineto{\pgfqpoint{5.561786in}{0.632663in}}%
\pgfpathlineto{\pgfqpoint{5.567119in}{0.632150in}}%
\pgfpathlineto{\pgfqpoint{5.572452in}{0.631638in}}%
\pgfpathlineto{\pgfqpoint{5.577785in}{0.631127in}}%
\pgfpathlineto{\pgfqpoint{5.583119in}{0.630618in}}%
\pgfpathlineto{\pgfqpoint{5.588452in}{0.630109in}}%
\pgfpathlineto{\pgfqpoint{5.593785in}{0.629601in}}%
\pgfpathlineto{\pgfqpoint{5.599118in}{0.629095in}}%
\pgfpathlineto{\pgfqpoint{5.604451in}{0.628589in}}%
\pgfpathlineto{\pgfqpoint{5.609785in}{0.628085in}}%
\pgfpathlineto{\pgfqpoint{5.615118in}{0.627581in}}%
\pgfpathlineto{\pgfqpoint{5.620451in}{0.627079in}}%
\pgfpathlineto{\pgfqpoint{5.625784in}{0.626577in}}%
\pgfpathlineto{\pgfqpoint{5.631118in}{0.626077in}}%
\pgfpathlineto{\pgfqpoint{5.636451in}{0.625577in}}%
\pgfpathlineto{\pgfqpoint{5.641784in}{0.625079in}}%
\pgfpathlineto{\pgfqpoint{5.647117in}{0.624581in}}%
\pgfpathlineto{\pgfqpoint{5.652450in}{0.624085in}}%
\pgfpathlineto{\pgfqpoint{5.657784in}{0.623590in}}%
\pgfpathlineto{\pgfqpoint{5.663117in}{0.623095in}}%
\pgfpathlineto{\pgfqpoint{5.668450in}{0.622602in}}%
\pgfpathlineto{\pgfqpoint{5.673783in}{0.622109in}}%
\pgfpathlineto{\pgfqpoint{5.679116in}{0.621617in}}%
\pgfpathlineto{\pgfqpoint{5.684450in}{0.621127in}}%
\pgfpathlineto{\pgfqpoint{5.689783in}{0.620637in}}%
\pgfpathlineto{\pgfqpoint{5.695116in}{0.620149in}}%
\pgfpathlineto{\pgfqpoint{5.700449in}{0.619661in}}%
\pgfpathlineto{\pgfqpoint{5.705783in}{0.619174in}}%
\pgfpathlineto{\pgfqpoint{5.711116in}{0.618689in}}%
\pgfpathlineto{\pgfqpoint{5.716449in}{0.618204in}}%
\pgfpathlineto{\pgfqpoint{5.721782in}{0.617720in}}%
\pgfpathlineto{\pgfqpoint{5.727115in}{0.617237in}}%
\pgfpathlineto{\pgfqpoint{5.732449in}{0.616755in}}%
\pgfpathlineto{\pgfqpoint{5.737782in}{0.616274in}}%
\pgfpathlineto{\pgfqpoint{5.743115in}{0.615794in}}%
\pgfpathlineto{\pgfqpoint{5.748448in}{0.615315in}}%
\pgfpathlineto{\pgfqpoint{5.753781in}{0.614837in}}%
\pgfpathlineto{\pgfqpoint{5.759115in}{0.614360in}}%
\pgfpathlineto{\pgfqpoint{5.764448in}{0.613884in}}%
\pgfpathlineto{\pgfqpoint{5.769781in}{0.613409in}}%
\pgfpathlineto{\pgfqpoint{5.775114in}{0.612934in}}%
\pgfpathlineto{\pgfqpoint{5.780448in}{0.612461in}}%
\pgfpathlineto{\pgfqpoint{5.785781in}{0.611988in}}%
\pgfpathlineto{\pgfqpoint{5.791114in}{0.611517in}}%
\pgfpathlineto{\pgfqpoint{5.796447in}{0.611046in}}%
\pgfpathlineto{\pgfqpoint{5.801780in}{0.610576in}}%
\pgfpathlineto{\pgfqpoint{5.807114in}{0.610107in}}%
\pgfpathlineto{\pgfqpoint{5.812447in}{0.609639in}}%
\pgfpathlineto{\pgfqpoint{5.817780in}{0.609172in}}%
\pgfpathlineto{\pgfqpoint{5.823113in}{0.608706in}}%
\pgfpathlineto{\pgfqpoint{5.828446in}{0.608241in}}%
\pgfpathlineto{\pgfqpoint{5.833780in}{0.607777in}}%
\pgfpathlineto{\pgfqpoint{5.839113in}{0.607313in}}%
\pgfpathlineto{\pgfqpoint{5.844446in}{0.606851in}}%
\pgfpathlineto{\pgfqpoint{5.849779in}{0.606389in}}%
\pgfpathlineto{\pgfqpoint{5.855113in}{0.605928in}}%
\pgfpathlineto{\pgfqpoint{5.860446in}{0.605469in}}%
\pgfpathlineto{\pgfqpoint{5.865779in}{0.605010in}}%
\pgfpathlineto{\pgfqpoint{5.871112in}{0.604552in}}%
\pgfpathlineto{\pgfqpoint{5.876445in}{0.604094in}}%
\pgfpathlineto{\pgfqpoint{5.881779in}{0.603638in}}%
\pgfpathlineto{\pgfqpoint{5.887112in}{0.603183in}}%
\pgfpathlineto{\pgfqpoint{5.892445in}{0.602728in}}%
\pgfpathlineto{\pgfqpoint{5.897778in}{0.602274in}}%
\pgfpathlineto{\pgfqpoint{5.903112in}{0.601821in}}%
\pgfpathlineto{\pgfqpoint{5.908445in}{0.601369in}}%
\pgfpathlineto{\pgfqpoint{5.913778in}{0.600918in}}%
\pgfpathlineto{\pgfqpoint{5.919111in}{0.600468in}}%
\pgfpathlineto{\pgfqpoint{5.924444in}{0.600019in}}%
\pgfpathlineto{\pgfqpoint{5.929778in}{0.599570in}}%
\pgfpathlineto{\pgfqpoint{5.935111in}{0.599122in}}%
\pgfpathlineto{\pgfqpoint{5.940444in}{0.598676in}}%
\pgfpathlineto{\pgfqpoint{5.945777in}{0.598230in}}%
\pgfpathlineto{\pgfqpoint{5.951110in}{0.597784in}}%
\pgfpathlineto{\pgfqpoint{5.956444in}{0.597340in}}%
\pgfpathlineto{\pgfqpoint{5.961777in}{0.596897in}}%
\pgfpathlineto{\pgfqpoint{5.967110in}{0.596454in}}%
\pgfpathlineto{\pgfqpoint{5.972443in}{0.596012in}}%
\pgfpathlineto{\pgfqpoint{5.977777in}{0.595571in}}%
\pgfpathlineto{\pgfqpoint{5.983110in}{0.595131in}}%
\pgfpathlineto{\pgfqpoint{5.988443in}{0.594692in}}%
\pgfpathlineto{\pgfqpoint{5.993776in}{0.594253in}}%
\pgfpathlineto{\pgfqpoint{5.999109in}{0.593816in}}%
\pgfpathlineto{\pgfqpoint{6.004443in}{0.593379in}}%
\pgfpathlineto{\pgfqpoint{6.009776in}{0.592943in}}%
\pgfpathlineto{\pgfqpoint{6.015109in}{0.592508in}}%
\pgfpathlineto{\pgfqpoint{6.020442in}{0.592073in}}%
\pgfpathlineto{\pgfqpoint{6.025775in}{0.591640in}}%
\pgfpathlineto{\pgfqpoint{6.031109in}{0.591207in}}%
\pgfpathlineto{\pgfqpoint{6.036442in}{0.590775in}}%
\pgfpathlineto{\pgfqpoint{6.041775in}{0.590344in}}%
\pgfpathlineto{\pgfqpoint{6.047108in}{0.589914in}}%
\pgfpathlineto{\pgfqpoint{6.052442in}{0.589484in}}%
\pgfpathlineto{\pgfqpoint{6.057775in}{0.589055in}}%
\pgfpathlineto{\pgfqpoint{6.063108in}{0.588627in}}%
\pgfpathlineto{\pgfqpoint{6.068441in}{0.588200in}}%
\pgfpathlineto{\pgfqpoint{6.073774in}{0.587774in}}%
\pgfpathlineto{\pgfqpoint{6.079108in}{0.587348in}}%
\pgfpathlineto{\pgfqpoint{6.084441in}{0.586924in}}%
\pgfpathlineto{\pgfqpoint{6.089774in}{0.586500in}}%
\pgfpathlineto{\pgfqpoint{6.095107in}{0.586076in}}%
\pgfpathlineto{\pgfqpoint{6.100440in}{0.585654in}}%
\pgfpathlineto{\pgfqpoint{6.105774in}{0.585232in}}%
\pgfpathlineto{\pgfqpoint{6.111107in}{0.584812in}}%
\pgfpathlineto{\pgfqpoint{6.116440in}{0.584391in}}%
\pgfpathlineto{\pgfqpoint{6.121773in}{0.583972in}}%
\pgfpathlineto{\pgfqpoint{6.127107in}{0.583554in}}%
\pgfpathlineto{\pgfqpoint{6.132440in}{0.583136in}}%
\pgfpathlineto{\pgfqpoint{6.137773in}{0.582719in}}%
\pgfpathlineto{\pgfqpoint{6.143106in}{0.582303in}}%
\pgfpathlineto{\pgfqpoint{6.148439in}{0.581887in}}%
\pgfpathlineto{\pgfqpoint{6.153773in}{0.581472in}}%
\pgfpathlineto{\pgfqpoint{6.159106in}{0.581058in}}%
\pgfpathlineto{\pgfqpoint{6.164439in}{0.580645in}}%
\pgfpathlineto{\pgfqpoint{6.169772in}{0.580233in}}%
\pgfpathlineto{\pgfqpoint{6.175105in}{0.579821in}}%
\pgfpathlineto{\pgfqpoint{6.175105in}{0.579821in}}%
\pgfpathlineto{\pgfqpoint{6.183915in}{0.579155in}}%
\pgfpathlineto{\pgfqpoint{6.192724in}{0.578515in}}%
\pgfpathlineto{\pgfqpoint{6.201533in}{0.577898in}}%
\pgfpathlineto{\pgfqpoint{6.210343in}{0.577303in}}%
\pgfpathlineto{\pgfqpoint{6.219152in}{0.576729in}}%
\pgfpathlineto{\pgfqpoint{6.227961in}{0.576175in}}%
\pgfpathlineto{\pgfqpoint{6.236770in}{0.575638in}}%
\pgfpathlineto{\pgfqpoint{6.245580in}{0.575118in}}%
\pgfpathlineto{\pgfqpoint{6.254389in}{0.574614in}}%
\pgfpathlineto{\pgfqpoint{6.263198in}{0.574126in}}%
\pgfpathlineto{\pgfqpoint{6.272007in}{0.573651in}}%
\pgfpathlineto{\pgfqpoint{6.280817in}{0.573190in}}%
\pgfpathlineto{\pgfqpoint{6.289626in}{0.572742in}}%
\pgfpathlineto{\pgfqpoint{6.298435in}{0.572306in}}%
\pgfpathlineto{\pgfqpoint{6.307245in}{0.571882in}}%
\pgfpathlineto{\pgfqpoint{6.316054in}{0.571468in}}%
\pgfpathlineto{\pgfqpoint{6.324863in}{0.571065in}}%
\pgfpathlineto{\pgfqpoint{6.333672in}{0.570672in}}%
\pgfpathlineto{\pgfqpoint{6.342482in}{0.570289in}}%
\pgfpathlineto{\pgfqpoint{6.351291in}{0.569915in}}%
\pgfpathlineto{\pgfqpoint{6.360100in}{0.569549in}}%
\pgfpathlineto{\pgfqpoint{6.368909in}{0.569192in}}%
\pgfpathlineto{\pgfqpoint{6.377719in}{0.568843in}}%
\pgfpathlineto{\pgfqpoint{6.386528in}{0.568501in}}%
\pgfpathlineto{\pgfqpoint{6.395337in}{0.568167in}}%
\pgfpathlineto{\pgfqpoint{6.404146in}{0.567840in}}%
\pgfpathlineto{\pgfqpoint{6.412956in}{0.567520in}}%
\pgfpathlineto{\pgfqpoint{6.421765in}{0.567206in}}%
\pgfpathlineto{\pgfqpoint{6.430574in}{0.566899in}}%
\pgfpathlineto{\pgfqpoint{6.439384in}{0.566598in}}%
\pgfpathlineto{\pgfqpoint{6.448193in}{0.566303in}}%
\pgfpathlineto{\pgfqpoint{6.457002in}{0.566014in}}%
\pgfpathlineto{\pgfqpoint{6.465811in}{0.565730in}}%
\pgfpathlineto{\pgfqpoint{6.474621in}{0.565452in}}%
\pgfpathlineto{\pgfqpoint{6.483430in}{0.565178in}}%
\pgfpathlineto{\pgfqpoint{6.492239in}{0.564910in}}%
\pgfpathlineto{\pgfqpoint{6.501048in}{0.564646in}}%
\pgfpathlineto{\pgfqpoint{6.509858in}{0.564387in}}%
\pgfpathlineto{\pgfqpoint{6.518667in}{0.564133in}}%
\pgfpathlineto{\pgfqpoint{6.527476in}{0.563883in}}%
\pgfpathlineto{\pgfqpoint{6.536285in}{0.563638in}}%
\pgfpathlineto{\pgfqpoint{6.545095in}{0.563396in}}%
\pgfpathlineto{\pgfqpoint{6.553904in}{0.563159in}}%
\pgfpathlineto{\pgfqpoint{6.562713in}{0.562925in}}%
\pgfpathlineto{\pgfqpoint{6.571523in}{0.562695in}}%
\pgfpathlineto{\pgfqpoint{6.580332in}{0.562469in}}%
\pgfpathlineto{\pgfqpoint{6.589141in}{0.562247in}}%
\pgfpathlineto{\pgfqpoint{6.597950in}{0.562028in}}%
\pgfpathlineto{\pgfqpoint{6.606760in}{0.561812in}}%
\pgfpathlineto{\pgfqpoint{6.615569in}{0.561600in}}%
\pgfpathlineto{\pgfqpoint{6.624378in}{0.561391in}}%
\pgfpathlineto{\pgfqpoint{6.633187in}{0.561185in}}%
\pgfpathlineto{\pgfqpoint{6.641997in}{0.560982in}}%
\pgfpathlineto{\pgfqpoint{6.650806in}{0.560782in}}%
\pgfpathlineto{\pgfqpoint{6.659615in}{0.560585in}}%
\pgfpathlineto{\pgfqpoint{6.668425in}{0.560391in}}%
\pgfpathlineto{\pgfqpoint{6.677234in}{0.560200in}}%
\pgfpathlineto{\pgfqpoint{6.686043in}{0.560011in}}%
\pgfpathlineto{\pgfqpoint{6.694852in}{0.559825in}}%
\pgfpathlineto{\pgfqpoint{6.703662in}{0.559642in}}%
\pgfpathlineto{\pgfqpoint{6.712471in}{0.559461in}}%
\pgfpathlineto{\pgfqpoint{6.721280in}{0.559282in}}%
\pgfpathlineto{\pgfqpoint{6.730089in}{0.559106in}}%
\pgfpathlineto{\pgfqpoint{6.738899in}{0.558933in}}%
\pgfpathlineto{\pgfqpoint{6.747708in}{0.558761in}}%
\pgfpathlineto{\pgfqpoint{6.756517in}{0.558592in}}%
\pgfpathlineto{\pgfqpoint{6.765326in}{0.558425in}}%
\pgfpathlineto{\pgfqpoint{6.774136in}{0.558260in}}%
\pgfpathlineto{\pgfqpoint{6.782945in}{0.558097in}}%
\pgfpathlineto{\pgfqpoint{6.791754in}{0.557937in}}%
\pgfpathlineto{\pgfqpoint{6.800564in}{0.557778in}}%
\pgfpathlineto{\pgfqpoint{6.809373in}{0.557621in}}%
\pgfpathlineto{\pgfqpoint{6.818182in}{0.557467in}}%
\pgfpathlineto{\pgfqpoint{6.826991in}{0.557314in}}%
\pgfpathlineto{\pgfqpoint{6.835801in}{0.557163in}}%
\pgfpathlineto{\pgfqpoint{6.844610in}{0.557014in}}%
\pgfpathlineto{\pgfqpoint{6.853419in}{0.556866in}}%
\pgfpathlineto{\pgfqpoint{6.862228in}{0.556720in}}%
\pgfpathlineto{\pgfqpoint{6.871038in}{0.556577in}}%
\pgfpathlineto{\pgfqpoint{6.879847in}{0.556434in}}%
\pgfpathlineto{\pgfqpoint{6.888656in}{0.556294in}}%
\pgfpathlineto{\pgfqpoint{6.897465in}{0.556155in}}%
\pgfpathlineto{\pgfqpoint{6.906275in}{0.556017in}}%
\pgfpathlineto{\pgfqpoint{6.915084in}{0.555881in}}%
\pgfpathlineto{\pgfqpoint{6.923893in}{0.555747in}}%
\pgfpathlineto{\pgfqpoint{6.932703in}{0.555614in}}%
\pgfpathlineto{\pgfqpoint{6.941512in}{0.555483in}}%
\pgfpathlineto{\pgfqpoint{6.950321in}{0.555353in}}%
\pgfpathlineto{\pgfqpoint{6.959130in}{0.555224in}}%
\pgfpathlineto{\pgfqpoint{6.967940in}{0.555097in}}%
\pgfpathlineto{\pgfqpoint{6.976749in}{0.554971in}}%
\pgfpathlineto{\pgfqpoint{6.985558in}{0.554847in}}%
\pgfpathlineto{\pgfqpoint{6.994367in}{0.554724in}}%
\pgfpathlineto{\pgfqpoint{7.003177in}{0.554602in}}%
\pgfpathlineto{\pgfqpoint{7.011986in}{0.554482in}}%
\pgfpathlineto{\pgfqpoint{7.020795in}{0.554362in}}%
\pgfpathlineto{\pgfqpoint{7.029605in}{0.554244in}}%
\pgfpathlineto{\pgfqpoint{7.038414in}{0.554127in}}%
\pgfpathlineto{\pgfqpoint{7.047223in}{0.554012in}}%
\pgfpathlineto{\pgfqpoint{7.047223in}{5.174012in}}%
\pgfpathlineto{\pgfqpoint{7.047223in}{5.174012in}}%
\pgfpathlineto{\pgfqpoint{7.038414in}{5.174012in}}%
\pgfpathlineto{\pgfqpoint{7.029605in}{5.174012in}}%
\pgfpathlineto{\pgfqpoint{7.020795in}{5.174012in}}%
\pgfpathlineto{\pgfqpoint{7.011986in}{5.174012in}}%
\pgfpathlineto{\pgfqpoint{7.003177in}{5.174012in}}%
\pgfpathlineto{\pgfqpoint{6.994367in}{5.174012in}}%
\pgfpathlineto{\pgfqpoint{6.985558in}{5.174012in}}%
\pgfpathlineto{\pgfqpoint{6.976749in}{5.174012in}}%
\pgfpathlineto{\pgfqpoint{6.967940in}{5.174012in}}%
\pgfpathlineto{\pgfqpoint{6.959130in}{5.174012in}}%
\pgfpathlineto{\pgfqpoint{6.950321in}{5.174012in}}%
\pgfpathlineto{\pgfqpoint{6.941512in}{5.174012in}}%
\pgfpathlineto{\pgfqpoint{6.932703in}{5.174012in}}%
\pgfpathlineto{\pgfqpoint{6.923893in}{5.174012in}}%
\pgfpathlineto{\pgfqpoint{6.915084in}{5.174012in}}%
\pgfpathlineto{\pgfqpoint{6.906275in}{5.174012in}}%
\pgfpathlineto{\pgfqpoint{6.897465in}{5.174012in}}%
\pgfpathlineto{\pgfqpoint{6.888656in}{5.174012in}}%
\pgfpathlineto{\pgfqpoint{6.879847in}{5.174012in}}%
\pgfpathlineto{\pgfqpoint{6.871038in}{5.174012in}}%
\pgfpathlineto{\pgfqpoint{6.862228in}{5.174012in}}%
\pgfpathlineto{\pgfqpoint{6.853419in}{5.174012in}}%
\pgfpathlineto{\pgfqpoint{6.844610in}{5.174012in}}%
\pgfpathlineto{\pgfqpoint{6.835801in}{5.174012in}}%
\pgfpathlineto{\pgfqpoint{6.826991in}{5.174012in}}%
\pgfpathlineto{\pgfqpoint{6.818182in}{5.174012in}}%
\pgfpathlineto{\pgfqpoint{6.809373in}{5.174012in}}%
\pgfpathlineto{\pgfqpoint{6.800564in}{5.174012in}}%
\pgfpathlineto{\pgfqpoint{6.791754in}{5.174012in}}%
\pgfpathlineto{\pgfqpoint{6.782945in}{5.174012in}}%
\pgfpathlineto{\pgfqpoint{6.774136in}{5.174012in}}%
\pgfpathlineto{\pgfqpoint{6.765326in}{5.174012in}}%
\pgfpathlineto{\pgfqpoint{6.756517in}{5.174012in}}%
\pgfpathlineto{\pgfqpoint{6.747708in}{5.174012in}}%
\pgfpathlineto{\pgfqpoint{6.738899in}{5.174012in}}%
\pgfpathlineto{\pgfqpoint{6.730089in}{5.174012in}}%
\pgfpathlineto{\pgfqpoint{6.721280in}{5.174012in}}%
\pgfpathlineto{\pgfqpoint{6.712471in}{5.174012in}}%
\pgfpathlineto{\pgfqpoint{6.703662in}{5.174012in}}%
\pgfpathlineto{\pgfqpoint{6.694852in}{5.174012in}}%
\pgfpathlineto{\pgfqpoint{6.686043in}{5.174012in}}%
\pgfpathlineto{\pgfqpoint{6.677234in}{5.174012in}}%
\pgfpathlineto{\pgfqpoint{6.668425in}{5.174012in}}%
\pgfpathlineto{\pgfqpoint{6.659615in}{5.174012in}}%
\pgfpathlineto{\pgfqpoint{6.650806in}{5.174012in}}%
\pgfpathlineto{\pgfqpoint{6.641997in}{5.174012in}}%
\pgfpathlineto{\pgfqpoint{6.633187in}{5.174012in}}%
\pgfpathlineto{\pgfqpoint{6.624378in}{5.174012in}}%
\pgfpathlineto{\pgfqpoint{6.615569in}{5.174012in}}%
\pgfpathlineto{\pgfqpoint{6.606760in}{5.174012in}}%
\pgfpathlineto{\pgfqpoint{6.597950in}{5.174012in}}%
\pgfpathlineto{\pgfqpoint{6.589141in}{5.174012in}}%
\pgfpathlineto{\pgfqpoint{6.580332in}{5.174012in}}%
\pgfpathlineto{\pgfqpoint{6.571523in}{5.174012in}}%
\pgfpathlineto{\pgfqpoint{6.562713in}{5.174012in}}%
\pgfpathlineto{\pgfqpoint{6.553904in}{5.174012in}}%
\pgfpathlineto{\pgfqpoint{6.545095in}{5.174012in}}%
\pgfpathlineto{\pgfqpoint{6.536285in}{5.174012in}}%
\pgfpathlineto{\pgfqpoint{6.527476in}{5.174012in}}%
\pgfpathlineto{\pgfqpoint{6.518667in}{5.174012in}}%
\pgfpathlineto{\pgfqpoint{6.509858in}{5.174012in}}%
\pgfpathlineto{\pgfqpoint{6.501048in}{5.174012in}}%
\pgfpathlineto{\pgfqpoint{6.492239in}{5.174012in}}%
\pgfpathlineto{\pgfqpoint{6.483430in}{5.174012in}}%
\pgfpathlineto{\pgfqpoint{6.474621in}{5.174012in}}%
\pgfpathlineto{\pgfqpoint{6.465811in}{5.174012in}}%
\pgfpathlineto{\pgfqpoint{6.457002in}{5.174012in}}%
\pgfpathlineto{\pgfqpoint{6.448193in}{5.174012in}}%
\pgfpathlineto{\pgfqpoint{6.439384in}{5.174012in}}%
\pgfpathlineto{\pgfqpoint{6.430574in}{5.174012in}}%
\pgfpathlineto{\pgfqpoint{6.421765in}{5.174012in}}%
\pgfpathlineto{\pgfqpoint{6.412956in}{5.174012in}}%
\pgfpathlineto{\pgfqpoint{6.404146in}{5.174012in}}%
\pgfpathlineto{\pgfqpoint{6.395337in}{5.174012in}}%
\pgfpathlineto{\pgfqpoint{6.386528in}{5.174012in}}%
\pgfpathlineto{\pgfqpoint{6.377719in}{5.174012in}}%
\pgfpathlineto{\pgfqpoint{6.368909in}{5.174012in}}%
\pgfpathlineto{\pgfqpoint{6.360100in}{5.174012in}}%
\pgfpathlineto{\pgfqpoint{6.351291in}{5.174012in}}%
\pgfpathlineto{\pgfqpoint{6.342482in}{5.174012in}}%
\pgfpathlineto{\pgfqpoint{6.333672in}{5.174012in}}%
\pgfpathlineto{\pgfqpoint{6.324863in}{5.174012in}}%
\pgfpathlineto{\pgfqpoint{6.316054in}{5.174012in}}%
\pgfpathlineto{\pgfqpoint{6.307245in}{5.174012in}}%
\pgfpathlineto{\pgfqpoint{6.298435in}{5.174012in}}%
\pgfpathlineto{\pgfqpoint{6.289626in}{5.174012in}}%
\pgfpathlineto{\pgfqpoint{6.280817in}{5.174012in}}%
\pgfpathlineto{\pgfqpoint{6.272007in}{5.174012in}}%
\pgfpathlineto{\pgfqpoint{6.263198in}{5.174012in}}%
\pgfpathlineto{\pgfqpoint{6.254389in}{5.174012in}}%
\pgfpathlineto{\pgfqpoint{6.245580in}{5.174012in}}%
\pgfpathlineto{\pgfqpoint{6.236770in}{5.174012in}}%
\pgfpathlineto{\pgfqpoint{6.227961in}{5.174012in}}%
\pgfpathlineto{\pgfqpoint{6.219152in}{5.174012in}}%
\pgfpathlineto{\pgfqpoint{6.210343in}{5.174012in}}%
\pgfpathlineto{\pgfqpoint{6.201533in}{5.174012in}}%
\pgfpathlineto{\pgfqpoint{6.192724in}{5.174012in}}%
\pgfpathlineto{\pgfqpoint{6.183915in}{5.174012in}}%
\pgfpathlineto{\pgfqpoint{6.175105in}{5.174012in}}%
\pgfpathlineto{\pgfqpoint{6.175105in}{5.174012in}}%
\pgfpathlineto{\pgfqpoint{6.169772in}{5.174012in}}%
\pgfpathlineto{\pgfqpoint{6.164439in}{5.174012in}}%
\pgfpathlineto{\pgfqpoint{6.159106in}{5.174012in}}%
\pgfpathlineto{\pgfqpoint{6.153773in}{5.174012in}}%
\pgfpathlineto{\pgfqpoint{6.148439in}{5.174012in}}%
\pgfpathlineto{\pgfqpoint{6.143106in}{5.174012in}}%
\pgfpathlineto{\pgfqpoint{6.137773in}{5.174012in}}%
\pgfpathlineto{\pgfqpoint{6.132440in}{5.174012in}}%
\pgfpathlineto{\pgfqpoint{6.127107in}{5.174012in}}%
\pgfpathlineto{\pgfqpoint{6.121773in}{5.174012in}}%
\pgfpathlineto{\pgfqpoint{6.116440in}{5.174012in}}%
\pgfpathlineto{\pgfqpoint{6.111107in}{5.174012in}}%
\pgfpathlineto{\pgfqpoint{6.105774in}{5.174012in}}%
\pgfpathlineto{\pgfqpoint{6.100440in}{5.174012in}}%
\pgfpathlineto{\pgfqpoint{6.095107in}{5.174012in}}%
\pgfpathlineto{\pgfqpoint{6.089774in}{5.174012in}}%
\pgfpathlineto{\pgfqpoint{6.084441in}{5.174012in}}%
\pgfpathlineto{\pgfqpoint{6.079108in}{5.174012in}}%
\pgfpathlineto{\pgfqpoint{6.073774in}{5.174012in}}%
\pgfpathlineto{\pgfqpoint{6.068441in}{5.174012in}}%
\pgfpathlineto{\pgfqpoint{6.063108in}{5.174012in}}%
\pgfpathlineto{\pgfqpoint{6.057775in}{5.174012in}}%
\pgfpathlineto{\pgfqpoint{6.052442in}{5.174012in}}%
\pgfpathlineto{\pgfqpoint{6.047108in}{5.174012in}}%
\pgfpathlineto{\pgfqpoint{6.041775in}{5.174012in}}%
\pgfpathlineto{\pgfqpoint{6.036442in}{5.174012in}}%
\pgfpathlineto{\pgfqpoint{6.031109in}{5.174012in}}%
\pgfpathlineto{\pgfqpoint{6.025775in}{5.174012in}}%
\pgfpathlineto{\pgfqpoint{6.020442in}{5.174012in}}%
\pgfpathlineto{\pgfqpoint{6.015109in}{5.174012in}}%
\pgfpathlineto{\pgfqpoint{6.009776in}{5.174012in}}%
\pgfpathlineto{\pgfqpoint{6.004443in}{5.174012in}}%
\pgfpathlineto{\pgfqpoint{5.999109in}{5.174012in}}%
\pgfpathlineto{\pgfqpoint{5.993776in}{5.174012in}}%
\pgfpathlineto{\pgfqpoint{5.988443in}{5.174012in}}%
\pgfpathlineto{\pgfqpoint{5.983110in}{5.174012in}}%
\pgfpathlineto{\pgfqpoint{5.977777in}{5.174012in}}%
\pgfpathlineto{\pgfqpoint{5.972443in}{5.174012in}}%
\pgfpathlineto{\pgfqpoint{5.967110in}{5.174012in}}%
\pgfpathlineto{\pgfqpoint{5.961777in}{5.174012in}}%
\pgfpathlineto{\pgfqpoint{5.956444in}{5.174012in}}%
\pgfpathlineto{\pgfqpoint{5.951110in}{5.174012in}}%
\pgfpathlineto{\pgfqpoint{5.945777in}{5.174012in}}%
\pgfpathlineto{\pgfqpoint{5.940444in}{5.174012in}}%
\pgfpathlineto{\pgfqpoint{5.935111in}{5.174012in}}%
\pgfpathlineto{\pgfqpoint{5.929778in}{5.174012in}}%
\pgfpathlineto{\pgfqpoint{5.924444in}{5.174012in}}%
\pgfpathlineto{\pgfqpoint{5.919111in}{5.174012in}}%
\pgfpathlineto{\pgfqpoint{5.913778in}{5.174012in}}%
\pgfpathlineto{\pgfqpoint{5.908445in}{5.174012in}}%
\pgfpathlineto{\pgfqpoint{5.903112in}{5.174012in}}%
\pgfpathlineto{\pgfqpoint{5.897778in}{5.174012in}}%
\pgfpathlineto{\pgfqpoint{5.892445in}{5.174012in}}%
\pgfpathlineto{\pgfqpoint{5.887112in}{5.174012in}}%
\pgfpathlineto{\pgfqpoint{5.881779in}{5.174012in}}%
\pgfpathlineto{\pgfqpoint{5.876445in}{5.174012in}}%
\pgfpathlineto{\pgfqpoint{5.871112in}{5.174012in}}%
\pgfpathlineto{\pgfqpoint{5.865779in}{5.174012in}}%
\pgfpathlineto{\pgfqpoint{5.860446in}{5.174012in}}%
\pgfpathlineto{\pgfqpoint{5.855113in}{5.174012in}}%
\pgfpathlineto{\pgfqpoint{5.849779in}{5.174012in}}%
\pgfpathlineto{\pgfqpoint{5.844446in}{5.174012in}}%
\pgfpathlineto{\pgfqpoint{5.839113in}{5.174012in}}%
\pgfpathlineto{\pgfqpoint{5.833780in}{5.174012in}}%
\pgfpathlineto{\pgfqpoint{5.828446in}{5.174012in}}%
\pgfpathlineto{\pgfqpoint{5.823113in}{5.174012in}}%
\pgfpathlineto{\pgfqpoint{5.817780in}{5.174012in}}%
\pgfpathlineto{\pgfqpoint{5.812447in}{5.174012in}}%
\pgfpathlineto{\pgfqpoint{5.807114in}{5.174012in}}%
\pgfpathlineto{\pgfqpoint{5.801780in}{5.174012in}}%
\pgfpathlineto{\pgfqpoint{5.796447in}{5.174012in}}%
\pgfpathlineto{\pgfqpoint{5.791114in}{5.174012in}}%
\pgfpathlineto{\pgfqpoint{5.785781in}{5.174012in}}%
\pgfpathlineto{\pgfqpoint{5.780448in}{5.174012in}}%
\pgfpathlineto{\pgfqpoint{5.775114in}{5.174012in}}%
\pgfpathlineto{\pgfqpoint{5.769781in}{5.174012in}}%
\pgfpathlineto{\pgfqpoint{5.764448in}{5.174012in}}%
\pgfpathlineto{\pgfqpoint{5.759115in}{5.174012in}}%
\pgfpathlineto{\pgfqpoint{5.753781in}{5.174012in}}%
\pgfpathlineto{\pgfqpoint{5.748448in}{5.174012in}}%
\pgfpathlineto{\pgfqpoint{5.743115in}{5.174012in}}%
\pgfpathlineto{\pgfqpoint{5.737782in}{5.174012in}}%
\pgfpathlineto{\pgfqpoint{5.732449in}{5.174012in}}%
\pgfpathlineto{\pgfqpoint{5.727115in}{5.174012in}}%
\pgfpathlineto{\pgfqpoint{5.721782in}{5.174012in}}%
\pgfpathlineto{\pgfqpoint{5.716449in}{5.174012in}}%
\pgfpathlineto{\pgfqpoint{5.711116in}{5.174012in}}%
\pgfpathlineto{\pgfqpoint{5.705783in}{5.174012in}}%
\pgfpathlineto{\pgfqpoint{5.700449in}{5.174012in}}%
\pgfpathlineto{\pgfqpoint{5.695116in}{5.174012in}}%
\pgfpathlineto{\pgfqpoint{5.689783in}{5.174012in}}%
\pgfpathlineto{\pgfqpoint{5.684450in}{5.174012in}}%
\pgfpathlineto{\pgfqpoint{5.679116in}{5.174012in}}%
\pgfpathlineto{\pgfqpoint{5.673783in}{5.174012in}}%
\pgfpathlineto{\pgfqpoint{5.668450in}{5.174012in}}%
\pgfpathlineto{\pgfqpoint{5.663117in}{5.174012in}}%
\pgfpathlineto{\pgfqpoint{5.657784in}{5.174012in}}%
\pgfpathlineto{\pgfqpoint{5.652450in}{5.174012in}}%
\pgfpathlineto{\pgfqpoint{5.647117in}{5.174012in}}%
\pgfpathlineto{\pgfqpoint{5.641784in}{5.174012in}}%
\pgfpathlineto{\pgfqpoint{5.636451in}{5.174012in}}%
\pgfpathlineto{\pgfqpoint{5.631118in}{5.174012in}}%
\pgfpathlineto{\pgfqpoint{5.625784in}{5.174012in}}%
\pgfpathlineto{\pgfqpoint{5.620451in}{5.174012in}}%
\pgfpathlineto{\pgfqpoint{5.615118in}{5.174012in}}%
\pgfpathlineto{\pgfqpoint{5.609785in}{5.174012in}}%
\pgfpathlineto{\pgfqpoint{5.604451in}{5.174012in}}%
\pgfpathlineto{\pgfqpoint{5.599118in}{5.174012in}}%
\pgfpathlineto{\pgfqpoint{5.593785in}{5.174012in}}%
\pgfpathlineto{\pgfqpoint{5.588452in}{5.174012in}}%
\pgfpathlineto{\pgfqpoint{5.583119in}{5.174012in}}%
\pgfpathlineto{\pgfqpoint{5.577785in}{5.174012in}}%
\pgfpathlineto{\pgfqpoint{5.572452in}{5.174012in}}%
\pgfpathlineto{\pgfqpoint{5.567119in}{5.174012in}}%
\pgfpathlineto{\pgfqpoint{5.561786in}{5.174012in}}%
\pgfpathlineto{\pgfqpoint{5.556452in}{5.174012in}}%
\pgfpathlineto{\pgfqpoint{5.551119in}{5.174012in}}%
\pgfpathlineto{\pgfqpoint{5.545786in}{5.174012in}}%
\pgfpathlineto{\pgfqpoint{5.540453in}{5.174012in}}%
\pgfpathlineto{\pgfqpoint{5.535120in}{5.174012in}}%
\pgfpathlineto{\pgfqpoint{5.529786in}{5.174012in}}%
\pgfpathlineto{\pgfqpoint{5.524453in}{5.174012in}}%
\pgfpathlineto{\pgfqpoint{5.519120in}{5.174012in}}%
\pgfpathlineto{\pgfqpoint{5.513787in}{5.174012in}}%
\pgfpathlineto{\pgfqpoint{5.508454in}{5.174012in}}%
\pgfpathlineto{\pgfqpoint{5.503120in}{5.174012in}}%
\pgfpathlineto{\pgfqpoint{5.497787in}{5.174012in}}%
\pgfpathlineto{\pgfqpoint{5.492454in}{5.174012in}}%
\pgfpathlineto{\pgfqpoint{5.487121in}{5.174012in}}%
\pgfpathlineto{\pgfqpoint{5.481787in}{5.174012in}}%
\pgfpathlineto{\pgfqpoint{5.476454in}{5.174012in}}%
\pgfpathlineto{\pgfqpoint{5.471121in}{5.174012in}}%
\pgfpathlineto{\pgfqpoint{5.465788in}{5.174012in}}%
\pgfpathlineto{\pgfqpoint{5.460455in}{5.174012in}}%
\pgfpathlineto{\pgfqpoint{5.455121in}{5.174012in}}%
\pgfpathlineto{\pgfqpoint{5.449788in}{5.174012in}}%
\pgfpathlineto{\pgfqpoint{5.444455in}{5.174012in}}%
\pgfpathlineto{\pgfqpoint{5.439122in}{5.174012in}}%
\pgfpathlineto{\pgfqpoint{5.433789in}{5.174012in}}%
\pgfpathlineto{\pgfqpoint{5.428455in}{5.174012in}}%
\pgfpathlineto{\pgfqpoint{5.423122in}{5.174012in}}%
\pgfpathlineto{\pgfqpoint{5.417789in}{5.174012in}}%
\pgfpathlineto{\pgfqpoint{5.412456in}{5.174012in}}%
\pgfpathlineto{\pgfqpoint{5.407122in}{5.174012in}}%
\pgfpathlineto{\pgfqpoint{5.401789in}{5.174012in}}%
\pgfpathlineto{\pgfqpoint{5.396456in}{5.174012in}}%
\pgfpathlineto{\pgfqpoint{5.391123in}{5.174012in}}%
\pgfpathlineto{\pgfqpoint{5.385790in}{5.174012in}}%
\pgfpathlineto{\pgfqpoint{5.380456in}{5.174012in}}%
\pgfpathlineto{\pgfqpoint{5.375123in}{5.174012in}}%
\pgfpathlineto{\pgfqpoint{5.369790in}{5.174012in}}%
\pgfpathlineto{\pgfqpoint{5.364457in}{5.174012in}}%
\pgfpathlineto{\pgfqpoint{5.359124in}{5.174012in}}%
\pgfpathlineto{\pgfqpoint{5.353790in}{5.174012in}}%
\pgfpathlineto{\pgfqpoint{5.348457in}{5.174012in}}%
\pgfpathlineto{\pgfqpoint{5.343124in}{5.174012in}}%
\pgfpathlineto{\pgfqpoint{5.337791in}{5.174012in}}%
\pgfpathlineto{\pgfqpoint{5.332457in}{5.174012in}}%
\pgfpathlineto{\pgfqpoint{5.327124in}{5.174012in}}%
\pgfpathlineto{\pgfqpoint{5.321791in}{5.174012in}}%
\pgfpathlineto{\pgfqpoint{5.316458in}{5.174012in}}%
\pgfpathlineto{\pgfqpoint{5.311125in}{5.174012in}}%
\pgfpathlineto{\pgfqpoint{5.305791in}{5.174012in}}%
\pgfpathlineto{\pgfqpoint{5.300458in}{5.174012in}}%
\pgfpathlineto{\pgfqpoint{5.295125in}{5.174012in}}%
\pgfpathlineto{\pgfqpoint{5.289792in}{5.174012in}}%
\pgfpathlineto{\pgfqpoint{5.284458in}{5.174012in}}%
\pgfpathlineto{\pgfqpoint{5.279125in}{5.174012in}}%
\pgfpathlineto{\pgfqpoint{5.273792in}{5.174012in}}%
\pgfpathlineto{\pgfqpoint{5.268459in}{5.174012in}}%
\pgfpathlineto{\pgfqpoint{5.263126in}{5.174012in}}%
\pgfpathlineto{\pgfqpoint{5.257792in}{5.174012in}}%
\pgfpathlineto{\pgfqpoint{5.252459in}{5.174012in}}%
\pgfpathlineto{\pgfqpoint{5.247126in}{5.174012in}}%
\pgfpathlineto{\pgfqpoint{5.241793in}{5.174012in}}%
\pgfpathlineto{\pgfqpoint{5.236460in}{5.174012in}}%
\pgfpathlineto{\pgfqpoint{5.231126in}{5.174012in}}%
\pgfpathlineto{\pgfqpoint{5.225793in}{5.174012in}}%
\pgfpathlineto{\pgfqpoint{5.220460in}{5.174012in}}%
\pgfpathlineto{\pgfqpoint{5.215127in}{5.174012in}}%
\pgfpathlineto{\pgfqpoint{5.209793in}{5.174012in}}%
\pgfpathlineto{\pgfqpoint{5.204460in}{5.174012in}}%
\pgfpathlineto{\pgfqpoint{5.199127in}{5.174012in}}%
\pgfpathlineto{\pgfqpoint{5.193794in}{5.174012in}}%
\pgfpathlineto{\pgfqpoint{5.188461in}{5.174012in}}%
\pgfpathlineto{\pgfqpoint{5.183127in}{5.174012in}}%
\pgfpathlineto{\pgfqpoint{5.177794in}{5.174012in}}%
\pgfpathlineto{\pgfqpoint{5.172461in}{5.174012in}}%
\pgfpathlineto{\pgfqpoint{5.167128in}{5.174012in}}%
\pgfpathlineto{\pgfqpoint{5.161795in}{5.174012in}}%
\pgfpathlineto{\pgfqpoint{5.156461in}{5.174012in}}%
\pgfpathlineto{\pgfqpoint{5.151128in}{5.174012in}}%
\pgfpathlineto{\pgfqpoint{5.145795in}{5.174012in}}%
\pgfpathlineto{\pgfqpoint{5.140462in}{5.174012in}}%
\pgfpathlineto{\pgfqpoint{5.135128in}{5.174012in}}%
\pgfpathlineto{\pgfqpoint{5.129795in}{5.174012in}}%
\pgfpathlineto{\pgfqpoint{5.124462in}{5.174012in}}%
\pgfpathlineto{\pgfqpoint{5.119129in}{5.174012in}}%
\pgfpathlineto{\pgfqpoint{5.113796in}{5.174012in}}%
\pgfpathlineto{\pgfqpoint{5.108462in}{5.174012in}}%
\pgfpathlineto{\pgfqpoint{5.103129in}{5.174012in}}%
\pgfpathlineto{\pgfqpoint{5.097796in}{5.174012in}}%
\pgfpathlineto{\pgfqpoint{5.092463in}{5.174012in}}%
\pgfpathlineto{\pgfqpoint{5.087130in}{5.174012in}}%
\pgfpathlineto{\pgfqpoint{5.081796in}{5.174012in}}%
\pgfpathlineto{\pgfqpoint{5.076463in}{5.174012in}}%
\pgfpathlineto{\pgfqpoint{5.071130in}{5.174012in}}%
\pgfpathlineto{\pgfqpoint{5.065797in}{5.174012in}}%
\pgfpathlineto{\pgfqpoint{5.060463in}{5.174012in}}%
\pgfpathlineto{\pgfqpoint{5.055130in}{5.174012in}}%
\pgfpathlineto{\pgfqpoint{5.049797in}{5.174012in}}%
\pgfpathlineto{\pgfqpoint{5.044464in}{5.174012in}}%
\pgfpathlineto{\pgfqpoint{5.039131in}{5.174012in}}%
\pgfpathlineto{\pgfqpoint{5.033797in}{5.174012in}}%
\pgfpathlineto{\pgfqpoint{5.028464in}{5.174012in}}%
\pgfpathlineto{\pgfqpoint{5.023131in}{5.174012in}}%
\pgfpathlineto{\pgfqpoint{5.017798in}{5.174012in}}%
\pgfpathlineto{\pgfqpoint{5.012464in}{5.174012in}}%
\pgfpathlineto{\pgfqpoint{5.007131in}{5.174012in}}%
\pgfpathlineto{\pgfqpoint{5.001798in}{5.174012in}}%
\pgfpathlineto{\pgfqpoint{4.996465in}{5.174012in}}%
\pgfpathlineto{\pgfqpoint{4.991132in}{5.174012in}}%
\pgfpathlineto{\pgfqpoint{4.985798in}{5.174012in}}%
\pgfpathlineto{\pgfqpoint{4.980465in}{5.174012in}}%
\pgfpathlineto{\pgfqpoint{4.975132in}{5.174012in}}%
\pgfpathlineto{\pgfqpoint{4.969799in}{5.174012in}}%
\pgfpathlineto{\pgfqpoint{4.964466in}{5.174012in}}%
\pgfpathlineto{\pgfqpoint{4.959132in}{5.174012in}}%
\pgfpathlineto{\pgfqpoint{4.953799in}{5.174012in}}%
\pgfpathlineto{\pgfqpoint{4.948466in}{5.174012in}}%
\pgfpathlineto{\pgfqpoint{4.943133in}{5.174012in}}%
\pgfpathlineto{\pgfqpoint{4.937799in}{5.174012in}}%
\pgfpathlineto{\pgfqpoint{4.932466in}{5.174012in}}%
\pgfpathlineto{\pgfqpoint{4.927133in}{5.174012in}}%
\pgfpathlineto{\pgfqpoint{4.921800in}{5.174012in}}%
\pgfpathlineto{\pgfqpoint{4.916467in}{5.174012in}}%
\pgfpathlineto{\pgfqpoint{4.911133in}{5.174012in}}%
\pgfpathlineto{\pgfqpoint{4.905800in}{5.174012in}}%
\pgfpathlineto{\pgfqpoint{4.900467in}{5.174012in}}%
\pgfpathlineto{\pgfqpoint{4.895134in}{5.174012in}}%
\pgfpathlineto{\pgfqpoint{4.889801in}{5.174012in}}%
\pgfpathlineto{\pgfqpoint{4.884467in}{5.174012in}}%
\pgfpathlineto{\pgfqpoint{4.879134in}{5.174012in}}%
\pgfpathlineto{\pgfqpoint{4.873801in}{5.174012in}}%
\pgfpathlineto{\pgfqpoint{4.868468in}{5.174012in}}%
\pgfpathlineto{\pgfqpoint{4.863134in}{5.174012in}}%
\pgfpathlineto{\pgfqpoint{4.857801in}{5.174012in}}%
\pgfpathlineto{\pgfqpoint{4.852468in}{5.174012in}}%
\pgfpathlineto{\pgfqpoint{4.847135in}{5.174012in}}%
\pgfpathlineto{\pgfqpoint{4.841802in}{5.174012in}}%
\pgfpathlineto{\pgfqpoint{4.836468in}{5.174012in}}%
\pgfpathlineto{\pgfqpoint{4.831135in}{5.174012in}}%
\pgfpathlineto{\pgfqpoint{4.825802in}{5.174012in}}%
\pgfpathlineto{\pgfqpoint{4.820469in}{5.174012in}}%
\pgfpathlineto{\pgfqpoint{4.815136in}{5.174012in}}%
\pgfpathlineto{\pgfqpoint{4.809802in}{5.174012in}}%
\pgfpathlineto{\pgfqpoint{4.804469in}{5.174012in}}%
\pgfpathlineto{\pgfqpoint{4.799136in}{5.174012in}}%
\pgfpathlineto{\pgfqpoint{4.793803in}{5.174012in}}%
\pgfpathlineto{\pgfqpoint{4.788469in}{5.174012in}}%
\pgfpathlineto{\pgfqpoint{4.783136in}{5.174012in}}%
\pgfpathlineto{\pgfqpoint{4.777803in}{5.174012in}}%
\pgfpathlineto{\pgfqpoint{4.772470in}{5.174012in}}%
\pgfpathlineto{\pgfqpoint{4.767137in}{5.174012in}}%
\pgfpathlineto{\pgfqpoint{4.761803in}{5.174012in}}%
\pgfpathlineto{\pgfqpoint{4.756470in}{5.174012in}}%
\pgfpathlineto{\pgfqpoint{4.751137in}{5.174012in}}%
\pgfpathlineto{\pgfqpoint{4.745804in}{5.174012in}}%
\pgfpathlineto{\pgfqpoint{4.740470in}{5.174012in}}%
\pgfpathlineto{\pgfqpoint{4.735137in}{5.174012in}}%
\pgfpathlineto{\pgfqpoint{4.729804in}{5.174012in}}%
\pgfpathlineto{\pgfqpoint{4.724471in}{5.174012in}}%
\pgfpathlineto{\pgfqpoint{4.719138in}{5.174012in}}%
\pgfpathlineto{\pgfqpoint{4.713804in}{5.174012in}}%
\pgfpathlineto{\pgfqpoint{4.708471in}{5.174012in}}%
\pgfpathlineto{\pgfqpoint{4.703138in}{5.174012in}}%
\pgfpathlineto{\pgfqpoint{4.697805in}{5.174012in}}%
\pgfpathlineto{\pgfqpoint{4.692472in}{5.174012in}}%
\pgfpathlineto{\pgfqpoint{4.687138in}{5.174012in}}%
\pgfpathlineto{\pgfqpoint{4.681805in}{5.174012in}}%
\pgfpathlineto{\pgfqpoint{4.676472in}{5.174012in}}%
\pgfpathlineto{\pgfqpoint{4.671139in}{5.174012in}}%
\pgfpathlineto{\pgfqpoint{4.665805in}{5.174012in}}%
\pgfpathlineto{\pgfqpoint{4.660472in}{5.174012in}}%
\pgfpathlineto{\pgfqpoint{4.655139in}{5.174012in}}%
\pgfpathlineto{\pgfqpoint{4.649806in}{5.174012in}}%
\pgfpathlineto{\pgfqpoint{4.644473in}{5.174012in}}%
\pgfpathlineto{\pgfqpoint{4.639139in}{5.174012in}}%
\pgfpathlineto{\pgfqpoint{4.633806in}{5.174012in}}%
\pgfpathlineto{\pgfqpoint{4.628473in}{5.174012in}}%
\pgfpathlineto{\pgfqpoint{4.623140in}{5.174012in}}%
\pgfpathlineto{\pgfqpoint{4.617807in}{5.174012in}}%
\pgfpathlineto{\pgfqpoint{4.612473in}{5.174012in}}%
\pgfpathlineto{\pgfqpoint{4.607140in}{5.174012in}}%
\pgfpathlineto{\pgfqpoint{4.601807in}{5.174012in}}%
\pgfpathlineto{\pgfqpoint{4.596474in}{5.174012in}}%
\pgfpathlineto{\pgfqpoint{4.591140in}{5.174012in}}%
\pgfpathlineto{\pgfqpoint{4.585807in}{5.174012in}}%
\pgfpathlineto{\pgfqpoint{4.580474in}{5.174012in}}%
\pgfpathlineto{\pgfqpoint{4.575141in}{5.174012in}}%
\pgfpathlineto{\pgfqpoint{4.569808in}{5.174012in}}%
\pgfpathlineto{\pgfqpoint{4.564474in}{5.174012in}}%
\pgfpathlineto{\pgfqpoint{4.559141in}{5.174012in}}%
\pgfpathlineto{\pgfqpoint{4.553808in}{5.174012in}}%
\pgfpathlineto{\pgfqpoint{4.548475in}{5.174012in}}%
\pgfpathlineto{\pgfqpoint{4.543142in}{5.174012in}}%
\pgfpathlineto{\pgfqpoint{4.537808in}{5.174012in}}%
\pgfpathlineto{\pgfqpoint{4.532475in}{5.174012in}}%
\pgfpathlineto{\pgfqpoint{4.527142in}{5.174012in}}%
\pgfpathlineto{\pgfqpoint{4.521809in}{5.174012in}}%
\pgfpathlineto{\pgfqpoint{4.516475in}{5.174012in}}%
\pgfpathlineto{\pgfqpoint{4.511142in}{5.174012in}}%
\pgfpathlineto{\pgfqpoint{4.505809in}{5.174012in}}%
\pgfpathlineto{\pgfqpoint{4.500476in}{5.174012in}}%
\pgfpathlineto{\pgfqpoint{4.495143in}{5.174012in}}%
\pgfpathlineto{\pgfqpoint{4.489809in}{5.174012in}}%
\pgfpathlineto{\pgfqpoint{4.484476in}{5.174012in}}%
\pgfpathlineto{\pgfqpoint{4.479143in}{5.174012in}}%
\pgfpathlineto{\pgfqpoint{4.473810in}{5.174012in}}%
\pgfpathlineto{\pgfqpoint{4.468476in}{5.174012in}}%
\pgfpathlineto{\pgfqpoint{4.463143in}{5.174012in}}%
\pgfpathlineto{\pgfqpoint{4.457810in}{5.174012in}}%
\pgfpathlineto{\pgfqpoint{4.452477in}{5.174012in}}%
\pgfpathlineto{\pgfqpoint{4.447144in}{5.174012in}}%
\pgfpathlineto{\pgfqpoint{4.441810in}{5.174012in}}%
\pgfpathlineto{\pgfqpoint{4.436477in}{5.174012in}}%
\pgfpathlineto{\pgfqpoint{4.431144in}{5.174012in}}%
\pgfpathlineto{\pgfqpoint{4.425811in}{5.174012in}}%
\pgfpathlineto{\pgfqpoint{4.420478in}{5.174012in}}%
\pgfpathlineto{\pgfqpoint{4.415144in}{5.174012in}}%
\pgfpathlineto{\pgfqpoint{4.409811in}{5.174012in}}%
\pgfpathlineto{\pgfqpoint{4.404478in}{5.174012in}}%
\pgfpathlineto{\pgfqpoint{4.399145in}{5.174012in}}%
\pgfpathlineto{\pgfqpoint{4.393811in}{5.174012in}}%
\pgfpathlineto{\pgfqpoint{4.388478in}{5.174012in}}%
\pgfpathlineto{\pgfqpoint{4.383145in}{5.174012in}}%
\pgfpathlineto{\pgfqpoint{4.377812in}{5.174012in}}%
\pgfpathlineto{\pgfqpoint{4.372479in}{5.174012in}}%
\pgfpathlineto{\pgfqpoint{4.367145in}{5.174012in}}%
\pgfpathlineto{\pgfqpoint{4.361812in}{5.174012in}}%
\pgfpathlineto{\pgfqpoint{4.356479in}{5.174012in}}%
\pgfpathlineto{\pgfqpoint{4.351146in}{5.174012in}}%
\pgfpathlineto{\pgfqpoint{4.345813in}{5.174012in}}%
\pgfpathlineto{\pgfqpoint{4.340479in}{5.174012in}}%
\pgfpathlineto{\pgfqpoint{4.335146in}{5.174012in}}%
\pgfpathlineto{\pgfqpoint{4.329813in}{5.174012in}}%
\pgfpathlineto{\pgfqpoint{4.324480in}{5.174012in}}%
\pgfpathlineto{\pgfqpoint{4.319146in}{5.174012in}}%
\pgfpathlineto{\pgfqpoint{4.313813in}{5.174012in}}%
\pgfpathlineto{\pgfqpoint{4.308480in}{5.174012in}}%
\pgfpathlineto{\pgfqpoint{4.303147in}{5.174012in}}%
\pgfpathlineto{\pgfqpoint{4.297814in}{5.174012in}}%
\pgfpathlineto{\pgfqpoint{4.292480in}{5.174012in}}%
\pgfpathlineto{\pgfqpoint{4.287147in}{5.174012in}}%
\pgfpathlineto{\pgfqpoint{4.281814in}{5.174012in}}%
\pgfpathlineto{\pgfqpoint{4.276481in}{5.174012in}}%
\pgfpathlineto{\pgfqpoint{4.271148in}{5.174012in}}%
\pgfpathlineto{\pgfqpoint{4.265814in}{5.174012in}}%
\pgfpathlineto{\pgfqpoint{4.260481in}{5.174012in}}%
\pgfpathlineto{\pgfqpoint{4.255148in}{5.174012in}}%
\pgfpathlineto{\pgfqpoint{4.249815in}{5.174012in}}%
\pgfpathlineto{\pgfqpoint{4.244481in}{5.174012in}}%
\pgfpathlineto{\pgfqpoint{4.239148in}{5.174012in}}%
\pgfpathlineto{\pgfqpoint{4.233815in}{5.174012in}}%
\pgfpathlineto{\pgfqpoint{4.228482in}{5.174012in}}%
\pgfpathlineto{\pgfqpoint{4.223149in}{5.174012in}}%
\pgfpathlineto{\pgfqpoint{4.217815in}{5.174012in}}%
\pgfpathlineto{\pgfqpoint{4.212482in}{5.174012in}}%
\pgfpathlineto{\pgfqpoint{4.207149in}{5.174012in}}%
\pgfpathlineto{\pgfqpoint{4.201816in}{5.174012in}}%
\pgfpathlineto{\pgfqpoint{4.196482in}{5.174012in}}%
\pgfpathlineto{\pgfqpoint{4.191149in}{5.174012in}}%
\pgfpathlineto{\pgfqpoint{4.185816in}{5.174012in}}%
\pgfpathlineto{\pgfqpoint{4.180483in}{5.174012in}}%
\pgfpathlineto{\pgfqpoint{4.175150in}{5.174012in}}%
\pgfpathlineto{\pgfqpoint{4.169816in}{5.174012in}}%
\pgfpathlineto{\pgfqpoint{4.164483in}{5.174012in}}%
\pgfpathlineto{\pgfqpoint{4.159150in}{5.174012in}}%
\pgfpathlineto{\pgfqpoint{4.153817in}{5.174012in}}%
\pgfpathlineto{\pgfqpoint{4.148484in}{5.174012in}}%
\pgfpathlineto{\pgfqpoint{4.143150in}{5.174012in}}%
\pgfpathlineto{\pgfqpoint{4.137817in}{5.174012in}}%
\pgfpathlineto{\pgfqpoint{4.132484in}{5.174012in}}%
\pgfpathlineto{\pgfqpoint{4.127151in}{5.174012in}}%
\pgfpathlineto{\pgfqpoint{4.121817in}{5.174012in}}%
\pgfpathlineto{\pgfqpoint{4.116484in}{5.174012in}}%
\pgfpathlineto{\pgfqpoint{4.111151in}{5.174012in}}%
\pgfpathlineto{\pgfqpoint{4.105818in}{5.174012in}}%
\pgfpathlineto{\pgfqpoint{4.100485in}{5.174012in}}%
\pgfpathlineto{\pgfqpoint{4.095151in}{5.174012in}}%
\pgfpathlineto{\pgfqpoint{4.089818in}{5.174012in}}%
\pgfpathlineto{\pgfqpoint{4.084485in}{5.174012in}}%
\pgfpathlineto{\pgfqpoint{4.079152in}{5.174012in}}%
\pgfpathlineto{\pgfqpoint{4.073819in}{5.174012in}}%
\pgfpathlineto{\pgfqpoint{4.068485in}{5.174012in}}%
\pgfpathlineto{\pgfqpoint{4.063152in}{5.174012in}}%
\pgfpathlineto{\pgfqpoint{4.057819in}{5.174012in}}%
\pgfpathlineto{\pgfqpoint{4.052486in}{5.174012in}}%
\pgfpathlineto{\pgfqpoint{4.047152in}{5.174012in}}%
\pgfpathlineto{\pgfqpoint{4.041819in}{5.174012in}}%
\pgfpathlineto{\pgfqpoint{4.036486in}{5.174012in}}%
\pgfpathlineto{\pgfqpoint{4.031153in}{5.174012in}}%
\pgfpathlineto{\pgfqpoint{4.025820in}{5.174012in}}%
\pgfpathlineto{\pgfqpoint{4.020486in}{5.174012in}}%
\pgfpathlineto{\pgfqpoint{4.015153in}{5.174012in}}%
\pgfpathlineto{\pgfqpoint{4.009820in}{5.174012in}}%
\pgfpathlineto{\pgfqpoint{4.004487in}{5.174012in}}%
\pgfpathlineto{\pgfqpoint{3.999154in}{5.174012in}}%
\pgfpathlineto{\pgfqpoint{3.993820in}{5.174012in}}%
\pgfpathlineto{\pgfqpoint{3.988487in}{5.174012in}}%
\pgfpathlineto{\pgfqpoint{3.983154in}{5.174012in}}%
\pgfpathlineto{\pgfqpoint{3.977821in}{5.174012in}}%
\pgfpathlineto{\pgfqpoint{3.972487in}{5.174012in}}%
\pgfpathlineto{\pgfqpoint{3.967154in}{5.174012in}}%
\pgfpathlineto{\pgfqpoint{3.961821in}{5.174012in}}%
\pgfpathlineto{\pgfqpoint{3.956488in}{5.174012in}}%
\pgfpathlineto{\pgfqpoint{3.951155in}{5.174012in}}%
\pgfpathlineto{\pgfqpoint{3.945821in}{5.174012in}}%
\pgfpathlineto{\pgfqpoint{3.940488in}{5.174012in}}%
\pgfpathlineto{\pgfqpoint{3.935155in}{5.174012in}}%
\pgfpathlineto{\pgfqpoint{3.929822in}{5.174012in}}%
\pgfpathlineto{\pgfqpoint{3.924488in}{5.174012in}}%
\pgfpathlineto{\pgfqpoint{3.919155in}{5.174012in}}%
\pgfpathlineto{\pgfqpoint{3.913822in}{5.174012in}}%
\pgfpathlineto{\pgfqpoint{3.908489in}{5.174012in}}%
\pgfpathlineto{\pgfqpoint{3.903156in}{5.174012in}}%
\pgfpathlineto{\pgfqpoint{3.897822in}{5.174012in}}%
\pgfpathlineto{\pgfqpoint{3.892489in}{5.174012in}}%
\pgfpathlineto{\pgfqpoint{3.887156in}{5.174012in}}%
\pgfpathlineto{\pgfqpoint{3.881823in}{5.174012in}}%
\pgfpathlineto{\pgfqpoint{3.876490in}{5.174012in}}%
\pgfpathlineto{\pgfqpoint{3.871156in}{5.174012in}}%
\pgfpathlineto{\pgfqpoint{3.865823in}{5.174012in}}%
\pgfpathlineto{\pgfqpoint{3.860490in}{5.174012in}}%
\pgfpathlineto{\pgfqpoint{3.855157in}{5.174012in}}%
\pgfpathlineto{\pgfqpoint{3.849823in}{5.174012in}}%
\pgfpathlineto{\pgfqpoint{3.844490in}{5.174012in}}%
\pgfpathlineto{\pgfqpoint{3.839157in}{5.174012in}}%
\pgfpathlineto{\pgfqpoint{3.833824in}{5.174012in}}%
\pgfpathlineto{\pgfqpoint{3.828491in}{5.174012in}}%
\pgfpathlineto{\pgfqpoint{3.823157in}{5.174012in}}%
\pgfpathlineto{\pgfqpoint{3.817824in}{5.174012in}}%
\pgfpathlineto{\pgfqpoint{3.812491in}{5.174012in}}%
\pgfpathlineto{\pgfqpoint{3.807158in}{5.174012in}}%
\pgfpathlineto{\pgfqpoint{3.801825in}{5.174012in}}%
\pgfpathlineto{\pgfqpoint{3.796491in}{5.174012in}}%
\pgfpathlineto{\pgfqpoint{3.791158in}{5.174012in}}%
\pgfpathlineto{\pgfqpoint{3.785825in}{5.174012in}}%
\pgfpathlineto{\pgfqpoint{3.780492in}{5.174012in}}%
\pgfpathlineto{\pgfqpoint{3.775158in}{5.174012in}}%
\pgfpathlineto{\pgfqpoint{3.769825in}{5.174012in}}%
\pgfpathlineto{\pgfqpoint{3.764492in}{5.174012in}}%
\pgfpathlineto{\pgfqpoint{3.759159in}{5.174012in}}%
\pgfpathlineto{\pgfqpoint{3.753826in}{5.174012in}}%
\pgfpathlineto{\pgfqpoint{3.748492in}{5.174012in}}%
\pgfpathlineto{\pgfqpoint{3.743159in}{5.174012in}}%
\pgfpathlineto{\pgfqpoint{3.737826in}{5.174012in}}%
\pgfpathlineto{\pgfqpoint{3.732493in}{5.174012in}}%
\pgfpathlineto{\pgfqpoint{3.727160in}{5.174012in}}%
\pgfpathlineto{\pgfqpoint{3.721826in}{5.174012in}}%
\pgfpathlineto{\pgfqpoint{3.716493in}{5.174012in}}%
\pgfpathlineto{\pgfqpoint{3.711160in}{5.174012in}}%
\pgfpathlineto{\pgfqpoint{3.705827in}{5.174012in}}%
\pgfpathlineto{\pgfqpoint{3.700493in}{5.174012in}}%
\pgfpathlineto{\pgfqpoint{3.695160in}{5.174012in}}%
\pgfpathlineto{\pgfqpoint{3.689827in}{5.174012in}}%
\pgfpathlineto{\pgfqpoint{3.684494in}{5.174012in}}%
\pgfpathlineto{\pgfqpoint{3.679161in}{5.174012in}}%
\pgfpathlineto{\pgfqpoint{3.673827in}{5.174012in}}%
\pgfpathlineto{\pgfqpoint{3.668494in}{5.174012in}}%
\pgfpathlineto{\pgfqpoint{3.663161in}{5.174012in}}%
\pgfpathlineto{\pgfqpoint{3.657828in}{5.174012in}}%
\pgfpathlineto{\pgfqpoint{3.652494in}{5.174012in}}%
\pgfpathlineto{\pgfqpoint{3.647161in}{5.174012in}}%
\pgfpathlineto{\pgfqpoint{3.641828in}{5.174012in}}%
\pgfpathlineto{\pgfqpoint{3.636495in}{5.174012in}}%
\pgfpathlineto{\pgfqpoint{3.631162in}{5.174012in}}%
\pgfpathlineto{\pgfqpoint{3.625828in}{5.174012in}}%
\pgfpathlineto{\pgfqpoint{3.620495in}{5.174012in}}%
\pgfpathlineto{\pgfqpoint{3.615162in}{5.174012in}}%
\pgfpathlineto{\pgfqpoint{3.609829in}{5.174012in}}%
\pgfpathlineto{\pgfqpoint{3.604496in}{5.174012in}}%
\pgfpathlineto{\pgfqpoint{3.599162in}{5.174012in}}%
\pgfpathlineto{\pgfqpoint{3.593829in}{5.174012in}}%
\pgfpathlineto{\pgfqpoint{3.588496in}{5.174012in}}%
\pgfpathlineto{\pgfqpoint{3.583163in}{5.174012in}}%
\pgfpathlineto{\pgfqpoint{3.577829in}{5.174012in}}%
\pgfpathlineto{\pgfqpoint{3.572496in}{5.174012in}}%
\pgfpathlineto{\pgfqpoint{3.567163in}{5.174012in}}%
\pgfpathlineto{\pgfqpoint{3.561830in}{5.174012in}}%
\pgfpathlineto{\pgfqpoint{3.556497in}{5.174012in}}%
\pgfpathlineto{\pgfqpoint{3.551163in}{5.174012in}}%
\pgfpathlineto{\pgfqpoint{3.545830in}{5.174012in}}%
\pgfpathlineto{\pgfqpoint{3.540497in}{5.174012in}}%
\pgfpathlineto{\pgfqpoint{3.535164in}{5.174012in}}%
\pgfpathlineto{\pgfqpoint{3.529831in}{5.174012in}}%
\pgfpathlineto{\pgfqpoint{3.524497in}{5.174012in}}%
\pgfpathlineto{\pgfqpoint{3.519164in}{5.174012in}}%
\pgfpathlineto{\pgfqpoint{3.513831in}{5.174012in}}%
\pgfpathlineto{\pgfqpoint{3.508498in}{5.174012in}}%
\pgfpathlineto{\pgfqpoint{3.503164in}{5.174012in}}%
\pgfpathlineto{\pgfqpoint{3.497831in}{5.174012in}}%
\pgfpathlineto{\pgfqpoint{3.492498in}{5.174012in}}%
\pgfpathlineto{\pgfqpoint{3.487165in}{5.174012in}}%
\pgfpathlineto{\pgfqpoint{3.481832in}{5.174012in}}%
\pgfpathlineto{\pgfqpoint{3.476498in}{5.174012in}}%
\pgfpathlineto{\pgfqpoint{3.471165in}{5.174012in}}%
\pgfpathlineto{\pgfqpoint{3.465832in}{5.174012in}}%
\pgfpathlineto{\pgfqpoint{3.460499in}{5.174012in}}%
\pgfpathlineto{\pgfqpoint{3.455166in}{5.174012in}}%
\pgfpathlineto{\pgfqpoint{3.449832in}{5.174012in}}%
\pgfpathlineto{\pgfqpoint{3.444499in}{5.174012in}}%
\pgfpathlineto{\pgfqpoint{3.439166in}{5.174012in}}%
\pgfpathlineto{\pgfqpoint{3.433833in}{5.174012in}}%
\pgfpathlineto{\pgfqpoint{3.428499in}{5.174012in}}%
\pgfpathlineto{\pgfqpoint{3.423166in}{5.174012in}}%
\pgfpathlineto{\pgfqpoint{3.417833in}{5.174012in}}%
\pgfpathlineto{\pgfqpoint{3.412500in}{5.174012in}}%
\pgfpathlineto{\pgfqpoint{3.407167in}{5.174012in}}%
\pgfpathlineto{\pgfqpoint{3.401833in}{5.174012in}}%
\pgfpathlineto{\pgfqpoint{3.396500in}{5.174012in}}%
\pgfpathlineto{\pgfqpoint{3.391167in}{5.174012in}}%
\pgfpathlineto{\pgfqpoint{3.385834in}{5.174012in}}%
\pgfpathlineto{\pgfqpoint{3.380500in}{5.174012in}}%
\pgfpathlineto{\pgfqpoint{3.375167in}{5.174012in}}%
\pgfpathlineto{\pgfqpoint{3.369834in}{5.174012in}}%
\pgfpathlineto{\pgfqpoint{3.364501in}{5.174012in}}%
\pgfpathlineto{\pgfqpoint{3.359168in}{5.174012in}}%
\pgfpathlineto{\pgfqpoint{3.353834in}{5.174012in}}%
\pgfpathlineto{\pgfqpoint{3.348501in}{5.174012in}}%
\pgfpathlineto{\pgfqpoint{3.343168in}{5.174012in}}%
\pgfpathlineto{\pgfqpoint{3.337835in}{5.174012in}}%
\pgfpathlineto{\pgfqpoint{3.332502in}{5.174012in}}%
\pgfpathlineto{\pgfqpoint{3.327168in}{5.174012in}}%
\pgfpathlineto{\pgfqpoint{3.321835in}{5.174012in}}%
\pgfpathlineto{\pgfqpoint{3.316502in}{5.174012in}}%
\pgfpathlineto{\pgfqpoint{3.311169in}{5.174012in}}%
\pgfpathlineto{\pgfqpoint{3.305835in}{5.174012in}}%
\pgfpathlineto{\pgfqpoint{3.300502in}{5.174012in}}%
\pgfpathlineto{\pgfqpoint{3.295169in}{5.174012in}}%
\pgfpathlineto{\pgfqpoint{3.289836in}{5.174012in}}%
\pgfpathlineto{\pgfqpoint{3.284503in}{5.174012in}}%
\pgfpathlineto{\pgfqpoint{3.279169in}{5.174012in}}%
\pgfpathlineto{\pgfqpoint{3.273836in}{5.174012in}}%
\pgfpathlineto{\pgfqpoint{3.268503in}{5.174012in}}%
\pgfpathlineto{\pgfqpoint{3.263170in}{5.174012in}}%
\pgfpathlineto{\pgfqpoint{3.257837in}{5.174012in}}%
\pgfpathlineto{\pgfqpoint{3.252503in}{5.174012in}}%
\pgfpathlineto{\pgfqpoint{3.247170in}{5.174012in}}%
\pgfpathlineto{\pgfqpoint{3.241837in}{5.174012in}}%
\pgfpathlineto{\pgfqpoint{3.236504in}{5.174012in}}%
\pgfpathlineto{\pgfqpoint{3.231170in}{5.174012in}}%
\pgfpathlineto{\pgfqpoint{3.225837in}{5.174012in}}%
\pgfpathlineto{\pgfqpoint{3.220504in}{5.174012in}}%
\pgfpathlineto{\pgfqpoint{3.215171in}{5.174012in}}%
\pgfpathlineto{\pgfqpoint{3.209838in}{5.174012in}}%
\pgfpathlineto{\pgfqpoint{3.204504in}{5.174012in}}%
\pgfpathlineto{\pgfqpoint{3.199171in}{5.174012in}}%
\pgfpathlineto{\pgfqpoint{3.193838in}{5.174012in}}%
\pgfpathlineto{\pgfqpoint{3.188505in}{5.174012in}}%
\pgfpathlineto{\pgfqpoint{3.183172in}{5.174012in}}%
\pgfpathlineto{\pgfqpoint{3.177838in}{5.174012in}}%
\pgfpathlineto{\pgfqpoint{3.172505in}{5.174012in}}%
\pgfpathlineto{\pgfqpoint{3.167172in}{5.174012in}}%
\pgfpathlineto{\pgfqpoint{3.161839in}{5.174012in}}%
\pgfpathlineto{\pgfqpoint{3.156505in}{5.174012in}}%
\pgfpathlineto{\pgfqpoint{3.151172in}{5.174012in}}%
\pgfpathlineto{\pgfqpoint{3.145839in}{5.174012in}}%
\pgfpathlineto{\pgfqpoint{3.140506in}{5.174012in}}%
\pgfpathlineto{\pgfqpoint{3.135173in}{5.174012in}}%
\pgfpathlineto{\pgfqpoint{3.129839in}{5.174012in}}%
\pgfpathlineto{\pgfqpoint{3.124506in}{5.174012in}}%
\pgfpathlineto{\pgfqpoint{3.119173in}{5.174012in}}%
\pgfpathlineto{\pgfqpoint{3.113840in}{5.174012in}}%
\pgfpathlineto{\pgfqpoint{3.108506in}{5.174012in}}%
\pgfpathlineto{\pgfqpoint{3.103173in}{5.174012in}}%
\pgfpathlineto{\pgfqpoint{3.097840in}{5.174012in}}%
\pgfpathlineto{\pgfqpoint{3.092507in}{5.174012in}}%
\pgfpathlineto{\pgfqpoint{3.087174in}{5.174012in}}%
\pgfpathlineto{\pgfqpoint{3.081840in}{5.174012in}}%
\pgfpathlineto{\pgfqpoint{3.076507in}{5.174012in}}%
\pgfpathlineto{\pgfqpoint{3.071174in}{5.174012in}}%
\pgfpathlineto{\pgfqpoint{3.065841in}{5.174012in}}%
\pgfpathlineto{\pgfqpoint{3.060508in}{5.174012in}}%
\pgfpathlineto{\pgfqpoint{3.055174in}{5.174012in}}%
\pgfpathlineto{\pgfqpoint{3.049841in}{5.174012in}}%
\pgfpathlineto{\pgfqpoint{3.044508in}{5.174012in}}%
\pgfpathlineto{\pgfqpoint{3.039175in}{5.174012in}}%
\pgfpathlineto{\pgfqpoint{3.033841in}{5.174012in}}%
\pgfpathlineto{\pgfqpoint{3.028508in}{5.174012in}}%
\pgfpathlineto{\pgfqpoint{3.023175in}{5.174012in}}%
\pgfpathlineto{\pgfqpoint{3.017842in}{5.174012in}}%
\pgfpathlineto{\pgfqpoint{3.012509in}{5.174012in}}%
\pgfpathlineto{\pgfqpoint{3.007175in}{5.174012in}}%
\pgfpathlineto{\pgfqpoint{3.001842in}{5.174012in}}%
\pgfpathlineto{\pgfqpoint{2.996509in}{5.174012in}}%
\pgfpathlineto{\pgfqpoint{2.991176in}{5.174012in}}%
\pgfpathlineto{\pgfqpoint{2.985843in}{5.174012in}}%
\pgfpathlineto{\pgfqpoint{2.980509in}{5.174012in}}%
\pgfpathlineto{\pgfqpoint{2.975176in}{5.174012in}}%
\pgfpathlineto{\pgfqpoint{2.969843in}{5.174012in}}%
\pgfpathlineto{\pgfqpoint{2.964510in}{5.174012in}}%
\pgfpathlineto{\pgfqpoint{2.959176in}{5.174012in}}%
\pgfpathlineto{\pgfqpoint{2.953843in}{5.174012in}}%
\pgfpathlineto{\pgfqpoint{2.948510in}{5.174012in}}%
\pgfpathlineto{\pgfqpoint{2.943177in}{5.174012in}}%
\pgfpathlineto{\pgfqpoint{2.937844in}{5.174012in}}%
\pgfpathlineto{\pgfqpoint{2.932510in}{5.174012in}}%
\pgfpathlineto{\pgfqpoint{2.927177in}{5.174012in}}%
\pgfpathlineto{\pgfqpoint{2.921844in}{5.174012in}}%
\pgfpathlineto{\pgfqpoint{2.916511in}{5.174012in}}%
\pgfpathlineto{\pgfqpoint{2.911178in}{5.174012in}}%
\pgfpathlineto{\pgfqpoint{2.905844in}{5.174012in}}%
\pgfpathlineto{\pgfqpoint{2.900511in}{5.174012in}}%
\pgfpathlineto{\pgfqpoint{2.895178in}{5.174012in}}%
\pgfpathlineto{\pgfqpoint{2.889845in}{5.174012in}}%
\pgfpathlineto{\pgfqpoint{2.884511in}{5.174012in}}%
\pgfpathlineto{\pgfqpoint{2.879178in}{5.174012in}}%
\pgfpathlineto{\pgfqpoint{2.873845in}{5.174012in}}%
\pgfpathlineto{\pgfqpoint{2.868512in}{5.174012in}}%
\pgfpathlineto{\pgfqpoint{2.863179in}{5.174012in}}%
\pgfpathlineto{\pgfqpoint{2.857845in}{5.174012in}}%
\pgfpathlineto{\pgfqpoint{2.852512in}{5.174012in}}%
\pgfpathlineto{\pgfqpoint{2.847179in}{5.174012in}}%
\pgfpathlineto{\pgfqpoint{2.841846in}{5.174012in}}%
\pgfpathlineto{\pgfqpoint{2.836512in}{5.174012in}}%
\pgfpathlineto{\pgfqpoint{2.831179in}{5.174012in}}%
\pgfpathlineto{\pgfqpoint{2.825846in}{5.174012in}}%
\pgfpathlineto{\pgfqpoint{2.820513in}{5.174012in}}%
\pgfpathlineto{\pgfqpoint{2.815180in}{5.174012in}}%
\pgfpathlineto{\pgfqpoint{2.809846in}{5.174012in}}%
\pgfpathlineto{\pgfqpoint{2.804513in}{5.174012in}}%
\pgfpathlineto{\pgfqpoint{2.799180in}{5.174012in}}%
\pgfpathlineto{\pgfqpoint{2.793847in}{5.174012in}}%
\pgfpathlineto{\pgfqpoint{2.788514in}{5.174012in}}%
\pgfpathlineto{\pgfqpoint{2.783180in}{5.174012in}}%
\pgfpathlineto{\pgfqpoint{2.777847in}{5.174012in}}%
\pgfpathlineto{\pgfqpoint{2.772514in}{5.174012in}}%
\pgfpathlineto{\pgfqpoint{2.767181in}{5.174012in}}%
\pgfpathlineto{\pgfqpoint{2.761847in}{5.174012in}}%
\pgfpathlineto{\pgfqpoint{2.756514in}{5.174012in}}%
\pgfpathlineto{\pgfqpoint{2.751181in}{5.174012in}}%
\pgfpathlineto{\pgfqpoint{2.745848in}{5.174012in}}%
\pgfpathlineto{\pgfqpoint{2.740515in}{5.174012in}}%
\pgfpathlineto{\pgfqpoint{2.735181in}{5.174012in}}%
\pgfpathlineto{\pgfqpoint{2.729848in}{5.174012in}}%
\pgfpathlineto{\pgfqpoint{2.724515in}{5.174012in}}%
\pgfpathlineto{\pgfqpoint{2.719182in}{5.174012in}}%
\pgfpathlineto{\pgfqpoint{2.713849in}{5.174012in}}%
\pgfpathlineto{\pgfqpoint{2.708515in}{5.174012in}}%
\pgfpathlineto{\pgfqpoint{2.703182in}{5.174012in}}%
\pgfpathlineto{\pgfqpoint{2.697849in}{5.174012in}}%
\pgfpathlineto{\pgfqpoint{2.692516in}{5.174012in}}%
\pgfpathlineto{\pgfqpoint{2.687182in}{5.174012in}}%
\pgfpathlineto{\pgfqpoint{2.681849in}{5.174012in}}%
\pgfpathlineto{\pgfqpoint{2.676516in}{5.174012in}}%
\pgfpathlineto{\pgfqpoint{2.671183in}{5.174012in}}%
\pgfpathlineto{\pgfqpoint{2.665850in}{5.174012in}}%
\pgfpathlineto{\pgfqpoint{2.660516in}{5.174012in}}%
\pgfpathlineto{\pgfqpoint{2.655183in}{5.174012in}}%
\pgfpathlineto{\pgfqpoint{2.649850in}{5.174012in}}%
\pgfpathlineto{\pgfqpoint{2.644517in}{5.174012in}}%
\pgfpathlineto{\pgfqpoint{2.639184in}{5.174012in}}%
\pgfpathlineto{\pgfqpoint{2.633850in}{5.174012in}}%
\pgfpathlineto{\pgfqpoint{2.628517in}{5.174012in}}%
\pgfpathlineto{\pgfqpoint{2.623184in}{5.174012in}}%
\pgfpathlineto{\pgfqpoint{2.617851in}{5.174012in}}%
\pgfpathlineto{\pgfqpoint{2.612517in}{5.174012in}}%
\pgfpathlineto{\pgfqpoint{2.607184in}{5.174012in}}%
\pgfpathlineto{\pgfqpoint{2.601851in}{5.174012in}}%
\pgfpathlineto{\pgfqpoint{2.596518in}{5.174012in}}%
\pgfpathlineto{\pgfqpoint{2.591185in}{5.174012in}}%
\pgfpathlineto{\pgfqpoint{2.585851in}{5.174012in}}%
\pgfpathlineto{\pgfqpoint{2.580518in}{5.174012in}}%
\pgfpathlineto{\pgfqpoint{2.575185in}{5.174012in}}%
\pgfpathlineto{\pgfqpoint{2.569852in}{5.174012in}}%
\pgfpathlineto{\pgfqpoint{2.564518in}{5.174012in}}%
\pgfpathlineto{\pgfqpoint{2.559185in}{5.174012in}}%
\pgfpathlineto{\pgfqpoint{2.553852in}{5.174012in}}%
\pgfpathlineto{\pgfqpoint{2.548519in}{5.174012in}}%
\pgfpathlineto{\pgfqpoint{2.543186in}{5.174012in}}%
\pgfpathlineto{\pgfqpoint{2.537852in}{5.174012in}}%
\pgfpathlineto{\pgfqpoint{2.532519in}{5.174012in}}%
\pgfpathlineto{\pgfqpoint{2.527186in}{5.174012in}}%
\pgfpathlineto{\pgfqpoint{2.521853in}{5.174012in}}%
\pgfpathlineto{\pgfqpoint{2.516520in}{5.174012in}}%
\pgfpathlineto{\pgfqpoint{2.511186in}{5.174012in}}%
\pgfpathlineto{\pgfqpoint{2.505853in}{5.174012in}}%
\pgfpathlineto{\pgfqpoint{2.500520in}{5.174012in}}%
\pgfpathlineto{\pgfqpoint{2.495187in}{5.174012in}}%
\pgfpathlineto{\pgfqpoint{2.489853in}{5.174012in}}%
\pgfpathlineto{\pgfqpoint{2.484520in}{5.174012in}}%
\pgfpathlineto{\pgfqpoint{2.479187in}{5.174012in}}%
\pgfpathlineto{\pgfqpoint{2.473854in}{5.174012in}}%
\pgfpathlineto{\pgfqpoint{2.468521in}{5.174012in}}%
\pgfpathlineto{\pgfqpoint{2.463187in}{5.174012in}}%
\pgfpathlineto{\pgfqpoint{2.457854in}{5.174012in}}%
\pgfpathlineto{\pgfqpoint{2.452521in}{5.174012in}}%
\pgfpathlineto{\pgfqpoint{2.447188in}{5.174012in}}%
\pgfpathlineto{\pgfqpoint{2.441855in}{5.174012in}}%
\pgfpathlineto{\pgfqpoint{2.436521in}{5.174012in}}%
\pgfpathlineto{\pgfqpoint{2.431188in}{5.174012in}}%
\pgfpathlineto{\pgfqpoint{2.425855in}{5.174012in}}%
\pgfpathlineto{\pgfqpoint{2.420522in}{5.174012in}}%
\pgfpathlineto{\pgfqpoint{2.415188in}{5.174012in}}%
\pgfpathlineto{\pgfqpoint{2.409855in}{5.174012in}}%
\pgfpathlineto{\pgfqpoint{2.404522in}{5.174012in}}%
\pgfpathlineto{\pgfqpoint{2.399189in}{5.174012in}}%
\pgfpathlineto{\pgfqpoint{2.393856in}{5.174012in}}%
\pgfpathlineto{\pgfqpoint{2.388522in}{5.174012in}}%
\pgfpathlineto{\pgfqpoint{2.383189in}{5.174012in}}%
\pgfpathlineto{\pgfqpoint{2.377856in}{5.174012in}}%
\pgfpathlineto{\pgfqpoint{2.372523in}{5.174012in}}%
\pgfpathlineto{\pgfqpoint{2.367190in}{5.174012in}}%
\pgfpathlineto{\pgfqpoint{2.361856in}{5.174012in}}%
\pgfpathlineto{\pgfqpoint{2.356523in}{5.174012in}}%
\pgfpathlineto{\pgfqpoint{2.351190in}{5.174012in}}%
\pgfpathlineto{\pgfqpoint{2.345857in}{5.174012in}}%
\pgfpathlineto{\pgfqpoint{2.340523in}{5.174012in}}%
\pgfpathlineto{\pgfqpoint{2.335190in}{5.174012in}}%
\pgfpathlineto{\pgfqpoint{2.329857in}{5.174012in}}%
\pgfpathlineto{\pgfqpoint{2.324524in}{5.174012in}}%
\pgfpathlineto{\pgfqpoint{2.319191in}{5.174012in}}%
\pgfpathlineto{\pgfqpoint{2.313857in}{5.174012in}}%
\pgfpathlineto{\pgfqpoint{2.308524in}{5.174012in}}%
\pgfpathlineto{\pgfqpoint{2.303191in}{5.174012in}}%
\pgfpathlineto{\pgfqpoint{2.297858in}{5.174012in}}%
\pgfpathlineto{\pgfqpoint{2.292524in}{5.174012in}}%
\pgfpathlineto{\pgfqpoint{2.287191in}{5.174012in}}%
\pgfpathlineto{\pgfqpoint{2.281858in}{5.174012in}}%
\pgfpathlineto{\pgfqpoint{2.276525in}{5.174012in}}%
\pgfpathlineto{\pgfqpoint{2.271192in}{5.174012in}}%
\pgfpathlineto{\pgfqpoint{2.265858in}{5.174012in}}%
\pgfpathlineto{\pgfqpoint{2.260525in}{5.174012in}}%
\pgfpathlineto{\pgfqpoint{2.255192in}{5.174012in}}%
\pgfpathlineto{\pgfqpoint{2.249859in}{5.174012in}}%
\pgfpathlineto{\pgfqpoint{2.244526in}{5.174012in}}%
\pgfpathlineto{\pgfqpoint{2.239192in}{5.174012in}}%
\pgfpathlineto{\pgfqpoint{2.233859in}{5.174012in}}%
\pgfpathlineto{\pgfqpoint{2.228526in}{5.174012in}}%
\pgfpathlineto{\pgfqpoint{2.223193in}{5.174012in}}%
\pgfpathlineto{\pgfqpoint{2.217859in}{5.174012in}}%
\pgfpathlineto{\pgfqpoint{2.212526in}{5.174012in}}%
\pgfpathlineto{\pgfqpoint{2.207193in}{5.174012in}}%
\pgfpathlineto{\pgfqpoint{2.201860in}{5.174012in}}%
\pgfpathlineto{\pgfqpoint{2.196527in}{5.174012in}}%
\pgfpathlineto{\pgfqpoint{2.191193in}{5.174012in}}%
\pgfpathlineto{\pgfqpoint{2.185860in}{5.174012in}}%
\pgfpathlineto{\pgfqpoint{2.180527in}{5.174012in}}%
\pgfpathlineto{\pgfqpoint{2.175194in}{5.174012in}}%
\pgfpathlineto{\pgfqpoint{2.169861in}{5.174012in}}%
\pgfpathlineto{\pgfqpoint{2.164527in}{5.174012in}}%
\pgfpathlineto{\pgfqpoint{2.159194in}{5.174012in}}%
\pgfpathlineto{\pgfqpoint{2.153861in}{5.174012in}}%
\pgfpathlineto{\pgfqpoint{2.148528in}{5.174012in}}%
\pgfpathlineto{\pgfqpoint{2.143194in}{5.174012in}}%
\pgfpathlineto{\pgfqpoint{2.137861in}{5.174012in}}%
\pgfpathlineto{\pgfqpoint{2.132528in}{5.174012in}}%
\pgfpathlineto{\pgfqpoint{2.127195in}{5.174012in}}%
\pgfpathlineto{\pgfqpoint{2.121862in}{5.174012in}}%
\pgfpathlineto{\pgfqpoint{2.116528in}{5.174012in}}%
\pgfpathlineto{\pgfqpoint{2.111195in}{5.174012in}}%
\pgfpathlineto{\pgfqpoint{2.105862in}{5.174012in}}%
\pgfpathlineto{\pgfqpoint{2.100529in}{5.174012in}}%
\pgfpathlineto{\pgfqpoint{2.095196in}{5.174012in}}%
\pgfpathlineto{\pgfqpoint{2.089862in}{5.174012in}}%
\pgfpathlineto{\pgfqpoint{2.084529in}{5.174012in}}%
\pgfpathlineto{\pgfqpoint{2.079196in}{5.174012in}}%
\pgfpathlineto{\pgfqpoint{2.073863in}{5.174012in}}%
\pgfpathlineto{\pgfqpoint{2.068529in}{5.174012in}}%
\pgfpathlineto{\pgfqpoint{2.063196in}{5.174012in}}%
\pgfpathlineto{\pgfqpoint{2.057863in}{5.174012in}}%
\pgfpathlineto{\pgfqpoint{2.052530in}{5.174012in}}%
\pgfpathlineto{\pgfqpoint{2.047197in}{5.174012in}}%
\pgfpathlineto{\pgfqpoint{2.041863in}{5.174012in}}%
\pgfpathlineto{\pgfqpoint{2.036530in}{5.174012in}}%
\pgfpathlineto{\pgfqpoint{2.031197in}{5.174012in}}%
\pgfpathlineto{\pgfqpoint{2.025864in}{5.174012in}}%
\pgfpathlineto{\pgfqpoint{2.020531in}{5.174012in}}%
\pgfpathlineto{\pgfqpoint{2.015197in}{5.174012in}}%
\pgfpathlineto{\pgfqpoint{2.009864in}{5.174012in}}%
\pgfpathlineto{\pgfqpoint{2.004531in}{5.174012in}}%
\pgfpathlineto{\pgfqpoint{1.999198in}{5.174012in}}%
\pgfpathlineto{\pgfqpoint{1.993864in}{5.174012in}}%
\pgfpathlineto{\pgfqpoint{1.988531in}{5.174012in}}%
\pgfpathlineto{\pgfqpoint{1.983198in}{5.174012in}}%
\pgfpathlineto{\pgfqpoint{1.977865in}{5.174012in}}%
\pgfpathlineto{\pgfqpoint{1.972532in}{5.174012in}}%
\pgfpathlineto{\pgfqpoint{1.967198in}{5.174012in}}%
\pgfpathlineto{\pgfqpoint{1.961865in}{5.174012in}}%
\pgfpathlineto{\pgfqpoint{1.956532in}{5.174012in}}%
\pgfpathlineto{\pgfqpoint{1.951199in}{5.174012in}}%
\pgfpathlineto{\pgfqpoint{1.945865in}{5.174012in}}%
\pgfpathlineto{\pgfqpoint{1.940532in}{5.174012in}}%
\pgfpathlineto{\pgfqpoint{1.935199in}{5.174012in}}%
\pgfpathlineto{\pgfqpoint{1.929866in}{5.174012in}}%
\pgfpathlineto{\pgfqpoint{1.924533in}{5.174012in}}%
\pgfpathlineto{\pgfqpoint{1.919199in}{5.174012in}}%
\pgfpathlineto{\pgfqpoint{1.913866in}{5.174012in}}%
\pgfpathlineto{\pgfqpoint{1.908533in}{5.174012in}}%
\pgfpathlineto{\pgfqpoint{1.903200in}{5.174012in}}%
\pgfpathlineto{\pgfqpoint{1.897867in}{5.174012in}}%
\pgfpathlineto{\pgfqpoint{1.892533in}{5.174012in}}%
\pgfpathlineto{\pgfqpoint{1.887200in}{5.174012in}}%
\pgfpathlineto{\pgfqpoint{1.881867in}{5.174012in}}%
\pgfpathlineto{\pgfqpoint{1.876534in}{5.174012in}}%
\pgfpathlineto{\pgfqpoint{1.871200in}{5.174012in}}%
\pgfpathlineto{\pgfqpoint{1.865867in}{5.174012in}}%
\pgfpathlineto{\pgfqpoint{1.860534in}{5.174012in}}%
\pgfpathlineto{\pgfqpoint{1.855201in}{5.174012in}}%
\pgfpathlineto{\pgfqpoint{1.849868in}{5.174012in}}%
\pgfpathlineto{\pgfqpoint{1.844534in}{5.174012in}}%
\pgfpathlineto{\pgfqpoint{1.839201in}{5.174012in}}%
\pgfpathlineto{\pgfqpoint{1.833868in}{5.174012in}}%
\pgfpathlineto{\pgfqpoint{1.828535in}{5.174012in}}%
\pgfpathlineto{\pgfqpoint{1.823202in}{5.174012in}}%
\pgfpathlineto{\pgfqpoint{1.817868in}{5.174012in}}%
\pgfpathlineto{\pgfqpoint{1.812535in}{5.174012in}}%
\pgfpathlineto{\pgfqpoint{1.807202in}{5.174012in}}%
\pgfpathlineto{\pgfqpoint{1.801869in}{5.174012in}}%
\pgfpathlineto{\pgfqpoint{1.796535in}{5.174012in}}%
\pgfpathlineto{\pgfqpoint{1.791202in}{5.174012in}}%
\pgfpathlineto{\pgfqpoint{1.785869in}{5.174012in}}%
\pgfpathlineto{\pgfqpoint{1.780536in}{5.174012in}}%
\pgfpathlineto{\pgfqpoint{1.775203in}{5.174012in}}%
\pgfpathlineto{\pgfqpoint{1.769869in}{5.174012in}}%
\pgfpathlineto{\pgfqpoint{1.764536in}{5.174012in}}%
\pgfpathlineto{\pgfqpoint{1.759203in}{5.174012in}}%
\pgfpathlineto{\pgfqpoint{1.753870in}{5.174012in}}%
\pgfpathlineto{\pgfqpoint{1.748537in}{5.174012in}}%
\pgfpathlineto{\pgfqpoint{1.743203in}{5.174012in}}%
\pgfpathlineto{\pgfqpoint{1.737870in}{5.174012in}}%
\pgfpathlineto{\pgfqpoint{1.732537in}{5.174012in}}%
\pgfpathlineto{\pgfqpoint{1.727204in}{5.174012in}}%
\pgfpathlineto{\pgfqpoint{1.721870in}{5.174012in}}%
\pgfpathlineto{\pgfqpoint{1.716537in}{5.174012in}}%
\pgfpathlineto{\pgfqpoint{1.711204in}{5.174012in}}%
\pgfpathlineto{\pgfqpoint{1.705871in}{5.174012in}}%
\pgfpathlineto{\pgfqpoint{1.700538in}{5.174012in}}%
\pgfpathlineto{\pgfqpoint{1.695204in}{5.174012in}}%
\pgfpathlineto{\pgfqpoint{1.689871in}{5.174012in}}%
\pgfpathlineto{\pgfqpoint{1.684538in}{5.174012in}}%
\pgfpathlineto{\pgfqpoint{1.679205in}{5.174012in}}%
\pgfpathlineto{\pgfqpoint{1.673871in}{5.174012in}}%
\pgfpathlineto{\pgfqpoint{1.668538in}{5.174012in}}%
\pgfpathlineto{\pgfqpoint{1.663205in}{5.174012in}}%
\pgfpathlineto{\pgfqpoint{1.657872in}{5.174012in}}%
\pgfpathlineto{\pgfqpoint{1.652539in}{5.174012in}}%
\pgfpathlineto{\pgfqpoint{1.647205in}{5.174012in}}%
\pgfpathlineto{\pgfqpoint{1.641872in}{5.174012in}}%
\pgfpathlineto{\pgfqpoint{1.636539in}{5.174012in}}%
\pgfpathlineto{\pgfqpoint{1.631206in}{5.174012in}}%
\pgfpathlineto{\pgfqpoint{1.625873in}{5.174012in}}%
\pgfpathlineto{\pgfqpoint{1.620539in}{5.174012in}}%
\pgfpathlineto{\pgfqpoint{1.615206in}{5.174012in}}%
\pgfpathlineto{\pgfqpoint{1.609873in}{5.174012in}}%
\pgfpathlineto{\pgfqpoint{1.604540in}{5.174012in}}%
\pgfpathlineto{\pgfqpoint{1.599206in}{5.174012in}}%
\pgfpathlineto{\pgfqpoint{1.593873in}{5.174012in}}%
\pgfpathlineto{\pgfqpoint{1.588540in}{5.174012in}}%
\pgfpathlineto{\pgfqpoint{1.583207in}{5.174012in}}%
\pgfpathlineto{\pgfqpoint{1.577874in}{5.174012in}}%
\pgfpathlineto{\pgfqpoint{1.572540in}{5.174012in}}%
\pgfpathlineto{\pgfqpoint{1.567207in}{5.174012in}}%
\pgfpathlineto{\pgfqpoint{1.561874in}{5.174012in}}%
\pgfpathlineto{\pgfqpoint{1.556541in}{5.174012in}}%
\pgfpathlineto{\pgfqpoint{1.551208in}{5.174012in}}%
\pgfpathlineto{\pgfqpoint{1.545874in}{5.174012in}}%
\pgfpathlineto{\pgfqpoint{1.540541in}{5.174012in}}%
\pgfpathlineto{\pgfqpoint{1.535208in}{5.174012in}}%
\pgfpathlineto{\pgfqpoint{1.529875in}{5.174012in}}%
\pgfpathlineto{\pgfqpoint{1.524541in}{5.174012in}}%
\pgfpathlineto{\pgfqpoint{1.519208in}{5.174012in}}%
\pgfpathlineto{\pgfqpoint{1.513875in}{5.174012in}}%
\pgfpathlineto{\pgfqpoint{1.508542in}{5.174012in}}%
\pgfpathlineto{\pgfqpoint{1.503209in}{5.174012in}}%
\pgfpathlineto{\pgfqpoint{1.497875in}{5.174012in}}%
\pgfpathlineto{\pgfqpoint{1.492542in}{5.174012in}}%
\pgfpathlineto{\pgfqpoint{1.487209in}{5.174012in}}%
\pgfpathlineto{\pgfqpoint{1.481876in}{5.174012in}}%
\pgfpathlineto{\pgfqpoint{1.476543in}{5.174012in}}%
\pgfpathlineto{\pgfqpoint{1.471209in}{5.174012in}}%
\pgfpathlineto{\pgfqpoint{1.465876in}{5.174012in}}%
\pgfpathlineto{\pgfqpoint{1.460543in}{5.174012in}}%
\pgfpathlineto{\pgfqpoint{1.455210in}{5.174012in}}%
\pgfpathlineto{\pgfqpoint{1.449876in}{5.174012in}}%
\pgfpathlineto{\pgfqpoint{1.444543in}{5.174012in}}%
\pgfpathlineto{\pgfqpoint{1.439210in}{5.174012in}}%
\pgfpathlineto{\pgfqpoint{1.433877in}{5.174012in}}%
\pgfpathlineto{\pgfqpoint{1.428544in}{5.174012in}}%
\pgfpathlineto{\pgfqpoint{1.423210in}{5.174012in}}%
\pgfpathlineto{\pgfqpoint{1.417877in}{5.174012in}}%
\pgfpathlineto{\pgfqpoint{1.412544in}{5.174012in}}%
\pgfpathlineto{\pgfqpoint{1.407211in}{5.174012in}}%
\pgfpathlineto{\pgfqpoint{1.401877in}{5.174012in}}%
\pgfpathlineto{\pgfqpoint{1.396544in}{5.174012in}}%
\pgfpathlineto{\pgfqpoint{1.391211in}{5.174012in}}%
\pgfpathlineto{\pgfqpoint{1.385878in}{5.174012in}}%
\pgfpathlineto{\pgfqpoint{1.380545in}{5.174012in}}%
\pgfpathlineto{\pgfqpoint{1.375211in}{5.174012in}}%
\pgfpathlineto{\pgfqpoint{1.369878in}{5.174012in}}%
\pgfpathlineto{\pgfqpoint{1.364545in}{5.174012in}}%
\pgfpathlineto{\pgfqpoint{1.359212in}{5.174012in}}%
\pgfpathlineto{\pgfqpoint{1.353879in}{5.174012in}}%
\pgfpathlineto{\pgfqpoint{1.348545in}{5.174012in}}%
\pgfpathlineto{\pgfqpoint{1.343212in}{5.174012in}}%
\pgfpathlineto{\pgfqpoint{1.337879in}{5.174012in}}%
\pgfpathlineto{\pgfqpoint{1.332546in}{5.174012in}}%
\pgfpathlineto{\pgfqpoint{1.327212in}{5.174012in}}%
\pgfpathlineto{\pgfqpoint{1.321879in}{5.174012in}}%
\pgfpathlineto{\pgfqpoint{1.316546in}{5.174012in}}%
\pgfpathlineto{\pgfqpoint{1.311213in}{5.174012in}}%
\pgfpathlineto{\pgfqpoint{1.305880in}{5.174012in}}%
\pgfpathlineto{\pgfqpoint{1.300546in}{5.174012in}}%
\pgfpathlineto{\pgfqpoint{1.295213in}{5.174012in}}%
\pgfpathlineto{\pgfqpoint{1.289880in}{5.174012in}}%
\pgfpathlineto{\pgfqpoint{1.284547in}{5.174012in}}%
\pgfpathlineto{\pgfqpoint{1.279214in}{5.174012in}}%
\pgfpathlineto{\pgfqpoint{1.273880in}{5.174012in}}%
\pgfpathlineto{\pgfqpoint{1.268547in}{5.174012in}}%
\pgfpathlineto{\pgfqpoint{1.263214in}{5.174012in}}%
\pgfpathlineto{\pgfqpoint{1.257881in}{5.174012in}}%
\pgfpathlineto{\pgfqpoint{1.252547in}{5.174012in}}%
\pgfpathlineto{\pgfqpoint{1.247214in}{5.174012in}}%
\pgfpathlineto{\pgfqpoint{1.241881in}{5.174012in}}%
\pgfpathlineto{\pgfqpoint{1.236548in}{5.174012in}}%
\pgfpathlineto{\pgfqpoint{1.231215in}{5.174012in}}%
\pgfpathlineto{\pgfqpoint{1.225881in}{5.174012in}}%
\pgfpathlineto{\pgfqpoint{1.220548in}{5.174012in}}%
\pgfpathlineto{\pgfqpoint{1.215215in}{5.174012in}}%
\pgfpathlineto{\pgfqpoint{1.209882in}{5.174012in}}%
\pgfpathlineto{\pgfqpoint{1.204549in}{5.174012in}}%
\pgfpathlineto{\pgfqpoint{1.199215in}{5.174012in}}%
\pgfpathlineto{\pgfqpoint{1.193882in}{5.174012in}}%
\pgfpathlineto{\pgfqpoint{1.188549in}{5.174012in}}%
\pgfpathlineto{\pgfqpoint{1.183216in}{5.174012in}}%
\pgfpathlineto{\pgfqpoint{1.177882in}{5.174012in}}%
\pgfpathlineto{\pgfqpoint{1.172549in}{5.174012in}}%
\pgfpathlineto{\pgfqpoint{1.167216in}{5.174012in}}%
\pgfpathlineto{\pgfqpoint{1.161883in}{5.174012in}}%
\pgfpathlineto{\pgfqpoint{1.156550in}{5.174012in}}%
\pgfpathlineto{\pgfqpoint{1.151216in}{5.174012in}}%
\pgfpathlineto{\pgfqpoint{1.145883in}{5.174012in}}%
\pgfpathlineto{\pgfqpoint{1.140550in}{5.174012in}}%
\pgfpathlineto{\pgfqpoint{1.135217in}{5.174012in}}%
\pgfpathlineto{\pgfqpoint{1.129883in}{5.174012in}}%
\pgfpathlineto{\pgfqpoint{1.124550in}{5.174012in}}%
\pgfpathlineto{\pgfqpoint{1.119217in}{5.174012in}}%
\pgfpathlineto{\pgfqpoint{1.113884in}{5.174012in}}%
\pgfpathlineto{\pgfqpoint{1.108551in}{5.174012in}}%
\pgfpathlineto{\pgfqpoint{1.103217in}{5.174012in}}%
\pgfpathlineto{\pgfqpoint{1.097884in}{5.174012in}}%
\pgfpathlineto{\pgfqpoint{1.092551in}{5.174012in}}%
\pgfpathlineto{\pgfqpoint{1.087218in}{5.174012in}}%
\pgfpathlineto{\pgfqpoint{1.081885in}{5.174012in}}%
\pgfpathlineto{\pgfqpoint{1.076551in}{5.174012in}}%
\pgfpathlineto{\pgfqpoint{1.071218in}{5.174012in}}%
\pgfpathlineto{\pgfqpoint{1.065885in}{5.174012in}}%
\pgfpathlineto{\pgfqpoint{1.060552in}{5.174012in}}%
\pgfpathlineto{\pgfqpoint{1.055218in}{5.174012in}}%
\pgfpathlineto{\pgfqpoint{1.049885in}{5.174012in}}%
\pgfpathlineto{\pgfqpoint{1.044552in}{5.174012in}}%
\pgfpathlineto{\pgfqpoint{1.039219in}{5.174012in}}%
\pgfpathlineto{\pgfqpoint{1.033886in}{5.174012in}}%
\pgfpathlineto{\pgfqpoint{1.028552in}{5.174012in}}%
\pgfpathlineto{\pgfqpoint{1.023219in}{5.174012in}}%
\pgfpathlineto{\pgfqpoint{1.017886in}{5.174012in}}%
\pgfpathlineto{\pgfqpoint{1.012553in}{5.174012in}}%
\pgfpathlineto{\pgfqpoint{1.007220in}{5.174012in}}%
\pgfpathlineto{\pgfqpoint{1.001886in}{5.174012in}}%
\pgfpathlineto{\pgfqpoint{0.996553in}{5.174012in}}%
\pgfpathlineto{\pgfqpoint{0.991220in}{5.174012in}}%
\pgfpathlineto{\pgfqpoint{0.985887in}{5.174012in}}%
\pgfpathlineto{\pgfqpoint{0.980553in}{5.174012in}}%
\pgfpathlineto{\pgfqpoint{0.975220in}{5.174012in}}%
\pgfpathlineto{\pgfqpoint{0.969887in}{5.174012in}}%
\pgfpathlineto{\pgfqpoint{0.964554in}{5.174012in}}%
\pgfpathlineto{\pgfqpoint{0.959221in}{5.174012in}}%
\pgfpathlineto{\pgfqpoint{0.953887in}{5.174012in}}%
\pgfpathlineto{\pgfqpoint{0.948554in}{5.174012in}}%
\pgfpathlineto{\pgfqpoint{0.943221in}{5.174012in}}%
\pgfpathlineto{\pgfqpoint{0.937888in}{5.174012in}}%
\pgfpathlineto{\pgfqpoint{0.932555in}{5.174012in}}%
\pgfpathlineto{\pgfqpoint{0.927221in}{5.174012in}}%
\pgfpathlineto{\pgfqpoint{0.921888in}{5.174012in}}%
\pgfpathlineto{\pgfqpoint{0.916555in}{5.174012in}}%
\pgfpathlineto{\pgfqpoint{0.911222in}{5.174012in}}%
\pgfpathlineto{\pgfqpoint{0.905888in}{5.174012in}}%
\pgfpathlineto{\pgfqpoint{0.900555in}{5.174012in}}%
\pgfpathlineto{\pgfqpoint{0.895222in}{5.174012in}}%
\pgfpathlineto{\pgfqpoint{0.889889in}{5.174012in}}%
\pgfpathlineto{\pgfqpoint{0.884556in}{5.174012in}}%
\pgfpathlineto{\pgfqpoint{0.879222in}{5.174012in}}%
\pgfpathlineto{\pgfqpoint{0.873889in}{5.174012in}}%
\pgfpathlineto{\pgfqpoint{0.868556in}{5.174012in}}%
\pgfpathlineto{\pgfqpoint{0.863223in}{5.174012in}}%
\pgfpathlineto{\pgfqpoint{0.857889in}{5.174012in}}%
\pgfpathlineto{\pgfqpoint{0.852556in}{5.174012in}}%
\pgfpathlineto{\pgfqpoint{0.847223in}{5.174012in}}%
\pgfpathlineto{\pgfqpoint{0.847223in}{5.174012in}}%
\pgfpathclose%
\pgfusepath{stroke,fill}%
}%
\begin{pgfscope}%
\pgfsys@transformshift{0.000000in}{0.000000in}%
\pgfsys@useobject{currentmarker}{}%
\end{pgfscope}%
\end{pgfscope}%
\begin{pgfscope}%
\pgfpathrectangle{\pgfqpoint{0.847223in}{0.554012in}}{\pgfqpoint{6.200000in}{4.620000in}}%
\pgfusepath{clip}%
\pgfsetbuttcap%
\pgfsetroundjoin%
\definecolor{currentfill}{rgb}{0.121569,0.466667,0.705882}%
\pgfsetfillcolor{currentfill}%
\pgfsetlinewidth{1.003750pt}%
\definecolor{currentstroke}{rgb}{0.121569,0.466667,0.705882}%
\pgfsetstrokecolor{currentstroke}%
\pgfsetdash{}{0pt}%
\pgfsys@defobject{currentmarker}{\pgfqpoint{-0.114394in}{-0.097310in}}{\pgfqpoint{0.114394in}{0.120281in}}{%
\pgfpathmoveto{\pgfqpoint{0.000000in}{0.120281in}}%
\pgfpathlineto{\pgfqpoint{-0.027005in}{0.037169in}}%
\pgfpathlineto{\pgfqpoint{-0.114394in}{0.037169in}}%
\pgfpathlineto{\pgfqpoint{-0.043695in}{-0.014197in}}%
\pgfpathlineto{\pgfqpoint{-0.070700in}{-0.097310in}}%
\pgfpathlineto{\pgfqpoint{-0.000000in}{-0.045943in}}%
\pgfpathlineto{\pgfqpoint{0.070700in}{-0.097310in}}%
\pgfpathlineto{\pgfqpoint{0.043695in}{-0.014197in}}%
\pgfpathlineto{\pgfqpoint{0.114394in}{0.037169in}}%
\pgfpathlineto{\pgfqpoint{0.027005in}{0.037169in}}%
\pgfpathlineto{\pgfqpoint{0.000000in}{0.120281in}}%
\pgfpathclose%
\pgfusepath{stroke,fill}%
}%
\begin{pgfscope}%
\pgfsys@transformshift{0.978061in}{0.660380in}%
\pgfsys@useobject{currentmarker}{}%
\end{pgfscope}%
\end{pgfscope}%
\begin{pgfscope}%
\pgfpathrectangle{\pgfqpoint{0.847223in}{0.554012in}}{\pgfqpoint{6.200000in}{4.620000in}}%
\pgfusepath{clip}%
\pgfsetbuttcap%
\pgfsetroundjoin%
\definecolor{currentfill}{rgb}{1.000000,0.498039,0.054902}%
\pgfsetfillcolor{currentfill}%
\pgfsetlinewidth{1.003750pt}%
\definecolor{currentstroke}{rgb}{1.000000,0.498039,0.054902}%
\pgfsetstrokecolor{currentstroke}%
\pgfsetdash{}{0pt}%
\pgfsys@defobject{currentmarker}{\pgfqpoint{-0.098209in}{-0.098209in}}{\pgfqpoint{0.098209in}{0.098209in}}{%
\pgfpathmoveto{\pgfqpoint{-0.098209in}{-0.098209in}}%
\pgfpathlineto{\pgfqpoint{0.098209in}{-0.098209in}}%
\pgfpathlineto{\pgfqpoint{0.098209in}{0.098209in}}%
\pgfpathlineto{\pgfqpoint{-0.098209in}{0.098209in}}%
\pgfpathlineto{\pgfqpoint{-0.098209in}{-0.098209in}}%
\pgfpathclose%
\pgfusepath{stroke,fill}%
}%
\begin{pgfscope}%
\pgfsys@transformshift{6.711056in}{4.921738in}%
\pgfsys@useobject{currentmarker}{}%
\end{pgfscope}%
\end{pgfscope}%
\begin{pgfscope}%
\pgfsetbuttcap%
\pgfsetroundjoin%
\definecolor{currentfill}{rgb}{0.000000,0.000000,0.000000}%
\pgfsetfillcolor{currentfill}%
\pgfsetlinewidth{0.803000pt}%
\definecolor{currentstroke}{rgb}{0.000000,0.000000,0.000000}%
\pgfsetstrokecolor{currentstroke}%
\pgfsetdash{}{0pt}%
\pgfsys@defobject{currentmarker}{\pgfqpoint{0.000000in}{-0.048611in}}{\pgfqpoint{0.000000in}{0.000000in}}{%
\pgfpathmoveto{\pgfqpoint{0.000000in}{0.000000in}}%
\pgfpathlineto{\pgfqpoint{0.000000in}{-0.048611in}}%
\pgfusepath{stroke,fill}%
}%
\begin{pgfscope}%
\pgfsys@transformshift{1.632248in}{0.554012in}%
\pgfsys@useobject{currentmarker}{}%
\end{pgfscope}%
\end{pgfscope}%
\begin{pgfscope}%
\definecolor{textcolor}{rgb}{0.000000,0.000000,0.000000}%
\pgfsetstrokecolor{textcolor}%
\pgfsetfillcolor{textcolor}%
\pgftext[x=1.632248in,y=0.456790in,,top]{\color{textcolor}{\rmfamily\fontsize{10.000000}{12.000000}\selectfont\catcode`\^=\active\def^{\ifmmode\sp\else\^{}\fi}\catcode`\%=\active\def%{\%}$\mathdefault{0.01}$}}%
\end{pgfscope}%
\begin{pgfscope}%
\pgfsetbuttcap%
\pgfsetroundjoin%
\definecolor{currentfill}{rgb}{0.000000,0.000000,0.000000}%
\pgfsetfillcolor{currentfill}%
\pgfsetlinewidth{0.803000pt}%
\definecolor{currentstroke}{rgb}{0.000000,0.000000,0.000000}%
\pgfsetstrokecolor{currentstroke}%
\pgfsetdash{}{0pt}%
\pgfsys@defobject{currentmarker}{\pgfqpoint{0.000000in}{-0.048611in}}{\pgfqpoint{0.000000in}{0.000000in}}{%
\pgfpathmoveto{\pgfqpoint{0.000000in}{0.000000in}}%
\pgfpathlineto{\pgfqpoint{0.000000in}{-0.048611in}}%
\pgfusepath{stroke,fill}%
}%
\begin{pgfscope}%
\pgfsys@transformshift{2.940624in}{0.554012in}%
\pgfsys@useobject{currentmarker}{}%
\end{pgfscope}%
\end{pgfscope}%
\begin{pgfscope}%
\definecolor{textcolor}{rgb}{0.000000,0.000000,0.000000}%
\pgfsetstrokecolor{textcolor}%
\pgfsetfillcolor{textcolor}%
\pgftext[x=2.940624in,y=0.456790in,,top]{\color{textcolor}{\rmfamily\fontsize{10.000000}{12.000000}\selectfont\catcode`\^=\active\def^{\ifmmode\sp\else\^{}\fi}\catcode`\%=\active\def%{\%}$\mathdefault{0.02}$}}%
\end{pgfscope}%
\begin{pgfscope}%
\pgfsetbuttcap%
\pgfsetroundjoin%
\definecolor{currentfill}{rgb}{0.000000,0.000000,0.000000}%
\pgfsetfillcolor{currentfill}%
\pgfsetlinewidth{0.803000pt}%
\definecolor{currentstroke}{rgb}{0.000000,0.000000,0.000000}%
\pgfsetstrokecolor{currentstroke}%
\pgfsetdash{}{0pt}%
\pgfsys@defobject{currentmarker}{\pgfqpoint{0.000000in}{-0.048611in}}{\pgfqpoint{0.000000in}{0.000000in}}{%
\pgfpathmoveto{\pgfqpoint{0.000000in}{0.000000in}}%
\pgfpathlineto{\pgfqpoint{0.000000in}{-0.048611in}}%
\pgfusepath{stroke,fill}%
}%
\begin{pgfscope}%
\pgfsys@transformshift{4.248999in}{0.554012in}%
\pgfsys@useobject{currentmarker}{}%
\end{pgfscope}%
\end{pgfscope}%
\begin{pgfscope}%
\definecolor{textcolor}{rgb}{0.000000,0.000000,0.000000}%
\pgfsetstrokecolor{textcolor}%
\pgfsetfillcolor{textcolor}%
\pgftext[x=4.248999in,y=0.456790in,,top]{\color{textcolor}{\rmfamily\fontsize{10.000000}{12.000000}\selectfont\catcode`\^=\active\def^{\ifmmode\sp\else\^{}\fi}\catcode`\%=\active\def%{\%}$\mathdefault{0.03}$}}%
\end{pgfscope}%
\begin{pgfscope}%
\pgfsetbuttcap%
\pgfsetroundjoin%
\definecolor{currentfill}{rgb}{0.000000,0.000000,0.000000}%
\pgfsetfillcolor{currentfill}%
\pgfsetlinewidth{0.803000pt}%
\definecolor{currentstroke}{rgb}{0.000000,0.000000,0.000000}%
\pgfsetstrokecolor{currentstroke}%
\pgfsetdash{}{0pt}%
\pgfsys@defobject{currentmarker}{\pgfqpoint{0.000000in}{-0.048611in}}{\pgfqpoint{0.000000in}{0.000000in}}{%
\pgfpathmoveto{\pgfqpoint{0.000000in}{0.000000in}}%
\pgfpathlineto{\pgfqpoint{0.000000in}{-0.048611in}}%
\pgfusepath{stroke,fill}%
}%
\begin{pgfscope}%
\pgfsys@transformshift{5.557374in}{0.554012in}%
\pgfsys@useobject{currentmarker}{}%
\end{pgfscope}%
\end{pgfscope}%
\begin{pgfscope}%
\definecolor{textcolor}{rgb}{0.000000,0.000000,0.000000}%
\pgfsetstrokecolor{textcolor}%
\pgfsetfillcolor{textcolor}%
\pgftext[x=5.557374in,y=0.456790in,,top]{\color{textcolor}{\rmfamily\fontsize{10.000000}{12.000000}\selectfont\catcode`\^=\active\def^{\ifmmode\sp\else\^{}\fi}\catcode`\%=\active\def%{\%}$\mathdefault{0.04}$}}%
\end{pgfscope}%
\begin{pgfscope}%
\pgfsetbuttcap%
\pgfsetroundjoin%
\definecolor{currentfill}{rgb}{0.000000,0.000000,0.000000}%
\pgfsetfillcolor{currentfill}%
\pgfsetlinewidth{0.803000pt}%
\definecolor{currentstroke}{rgb}{0.000000,0.000000,0.000000}%
\pgfsetstrokecolor{currentstroke}%
\pgfsetdash{}{0pt}%
\pgfsys@defobject{currentmarker}{\pgfqpoint{0.000000in}{-0.048611in}}{\pgfqpoint{0.000000in}{0.000000in}}{%
\pgfpathmoveto{\pgfqpoint{0.000000in}{0.000000in}}%
\pgfpathlineto{\pgfqpoint{0.000000in}{-0.048611in}}%
\pgfusepath{stroke,fill}%
}%
\begin{pgfscope}%
\pgfsys@transformshift{6.865750in}{0.554012in}%
\pgfsys@useobject{currentmarker}{}%
\end{pgfscope}%
\end{pgfscope}%
\begin{pgfscope}%
\definecolor{textcolor}{rgb}{0.000000,0.000000,0.000000}%
\pgfsetstrokecolor{textcolor}%
\pgfsetfillcolor{textcolor}%
\pgftext[x=6.865750in,y=0.456790in,,top]{\color{textcolor}{\rmfamily\fontsize{10.000000}{12.000000}\selectfont\catcode`\^=\active\def^{\ifmmode\sp\else\^{}\fi}\catcode`\%=\active\def%{\%}$\mathdefault{0.05}$}}%
\end{pgfscope}%
\begin{pgfscope}%
\definecolor{textcolor}{rgb}{0.000000,0.000000,0.000000}%
\pgfsetstrokecolor{textcolor}%
\pgfsetfillcolor{textcolor}%
\pgftext[x=3.947223in,y=0.277777in,,top]{\color{textcolor}{\rmfamily\fontsize{14.000000}{16.800000}\selectfont\catcode`\^=\active\def^{\ifmmode\sp\else\^{}\fi}\catcode`\%=\active\def%{\%}f1}}%
\end{pgfscope}%
\begin{pgfscope}%
\pgfsetbuttcap%
\pgfsetroundjoin%
\definecolor{currentfill}{rgb}{0.000000,0.000000,0.000000}%
\pgfsetfillcolor{currentfill}%
\pgfsetlinewidth{0.803000pt}%
\definecolor{currentstroke}{rgb}{0.000000,0.000000,0.000000}%
\pgfsetstrokecolor{currentstroke}%
\pgfsetdash{}{0pt}%
\pgfsys@defobject{currentmarker}{\pgfqpoint{-0.048611in}{0.000000in}}{\pgfqpoint{-0.000000in}{0.000000in}}{%
\pgfpathmoveto{\pgfqpoint{-0.000000in}{0.000000in}}%
\pgfpathlineto{\pgfqpoint{-0.048611in}{0.000000in}}%
\pgfusepath{stroke,fill}%
}%
\begin{pgfscope}%
\pgfsys@transformshift{0.847223in}{1.137635in}%
\pgfsys@useobject{currentmarker}{}%
\end{pgfscope}%
\end{pgfscope}%
\begin{pgfscope}%
\definecolor{textcolor}{rgb}{0.000000,0.000000,0.000000}%
\pgfsetstrokecolor{textcolor}%
\pgfsetfillcolor{textcolor}%
\pgftext[x=0.402777in, y=1.089409in, left, base]{\color{textcolor}{\rmfamily\fontsize{10.000000}{12.000000}\selectfont\catcode`\^=\active\def^{\ifmmode\sp\else\^{}\fi}\catcode`\%=\active\def%{\%}$\mathdefault{20000}$}}%
\end{pgfscope}%
\begin{pgfscope}%
\pgfsetbuttcap%
\pgfsetroundjoin%
\definecolor{currentfill}{rgb}{0.000000,0.000000,0.000000}%
\pgfsetfillcolor{currentfill}%
\pgfsetlinewidth{0.803000pt}%
\definecolor{currentstroke}{rgb}{0.000000,0.000000,0.000000}%
\pgfsetstrokecolor{currentstroke}%
\pgfsetdash{}{0pt}%
\pgfsys@defobject{currentmarker}{\pgfqpoint{-0.048611in}{0.000000in}}{\pgfqpoint{-0.000000in}{0.000000in}}{%
\pgfpathmoveto{\pgfqpoint{-0.000000in}{0.000000in}}%
\pgfpathlineto{\pgfqpoint{-0.048611in}{0.000000in}}%
\pgfusepath{stroke,fill}%
}%
\begin{pgfscope}%
\pgfsys@transformshift{0.847223in}{2.146729in}%
\pgfsys@useobject{currentmarker}{}%
\end{pgfscope}%
\end{pgfscope}%
\begin{pgfscope}%
\definecolor{textcolor}{rgb}{0.000000,0.000000,0.000000}%
\pgfsetstrokecolor{textcolor}%
\pgfsetfillcolor{textcolor}%
\pgftext[x=0.402777in, y=2.098504in, left, base]{\color{textcolor}{\rmfamily\fontsize{10.000000}{12.000000}\selectfont\catcode`\^=\active\def^{\ifmmode\sp\else\^{}\fi}\catcode`\%=\active\def%{\%}$\mathdefault{40000}$}}%
\end{pgfscope}%
\begin{pgfscope}%
\pgfsetbuttcap%
\pgfsetroundjoin%
\definecolor{currentfill}{rgb}{0.000000,0.000000,0.000000}%
\pgfsetfillcolor{currentfill}%
\pgfsetlinewidth{0.803000pt}%
\definecolor{currentstroke}{rgb}{0.000000,0.000000,0.000000}%
\pgfsetstrokecolor{currentstroke}%
\pgfsetdash{}{0pt}%
\pgfsys@defobject{currentmarker}{\pgfqpoint{-0.048611in}{0.000000in}}{\pgfqpoint{-0.000000in}{0.000000in}}{%
\pgfpathmoveto{\pgfqpoint{-0.000000in}{0.000000in}}%
\pgfpathlineto{\pgfqpoint{-0.048611in}{0.000000in}}%
\pgfusepath{stroke,fill}%
}%
\begin{pgfscope}%
\pgfsys@transformshift{0.847223in}{3.155823in}%
\pgfsys@useobject{currentmarker}{}%
\end{pgfscope}%
\end{pgfscope}%
\begin{pgfscope}%
\definecolor{textcolor}{rgb}{0.000000,0.000000,0.000000}%
\pgfsetstrokecolor{textcolor}%
\pgfsetfillcolor{textcolor}%
\pgftext[x=0.402777in, y=3.107598in, left, base]{\color{textcolor}{\rmfamily\fontsize{10.000000}{12.000000}\selectfont\catcode`\^=\active\def^{\ifmmode\sp\else\^{}\fi}\catcode`\%=\active\def%{\%}$\mathdefault{60000}$}}%
\end{pgfscope}%
\begin{pgfscope}%
\pgfsetbuttcap%
\pgfsetroundjoin%
\definecolor{currentfill}{rgb}{0.000000,0.000000,0.000000}%
\pgfsetfillcolor{currentfill}%
\pgfsetlinewidth{0.803000pt}%
\definecolor{currentstroke}{rgb}{0.000000,0.000000,0.000000}%
\pgfsetstrokecolor{currentstroke}%
\pgfsetdash{}{0pt}%
\pgfsys@defobject{currentmarker}{\pgfqpoint{-0.048611in}{0.000000in}}{\pgfqpoint{-0.000000in}{0.000000in}}{%
\pgfpathmoveto{\pgfqpoint{-0.000000in}{0.000000in}}%
\pgfpathlineto{\pgfqpoint{-0.048611in}{0.000000in}}%
\pgfusepath{stroke,fill}%
}%
\begin{pgfscope}%
\pgfsys@transformshift{0.847223in}{4.164918in}%
\pgfsys@useobject{currentmarker}{}%
\end{pgfscope}%
\end{pgfscope}%
\begin{pgfscope}%
\definecolor{textcolor}{rgb}{0.000000,0.000000,0.000000}%
\pgfsetstrokecolor{textcolor}%
\pgfsetfillcolor{textcolor}%
\pgftext[x=0.402777in, y=4.116692in, left, base]{\color{textcolor}{\rmfamily\fontsize{10.000000}{12.000000}\selectfont\catcode`\^=\active\def^{\ifmmode\sp\else\^{}\fi}\catcode`\%=\active\def%{\%}$\mathdefault{80000}$}}%
\end{pgfscope}%
\begin{pgfscope}%
\pgfsetbuttcap%
\pgfsetroundjoin%
\definecolor{currentfill}{rgb}{0.000000,0.000000,0.000000}%
\pgfsetfillcolor{currentfill}%
\pgfsetlinewidth{0.803000pt}%
\definecolor{currentstroke}{rgb}{0.000000,0.000000,0.000000}%
\pgfsetstrokecolor{currentstroke}%
\pgfsetdash{}{0pt}%
\pgfsys@defobject{currentmarker}{\pgfqpoint{-0.048611in}{0.000000in}}{\pgfqpoint{-0.000000in}{0.000000in}}{%
\pgfpathmoveto{\pgfqpoint{-0.000000in}{0.000000in}}%
\pgfpathlineto{\pgfqpoint{-0.048611in}{0.000000in}}%
\pgfusepath{stroke,fill}%
}%
\begin{pgfscope}%
\pgfsys@transformshift{0.847223in}{5.174012in}%
\pgfsys@useobject{currentmarker}{}%
\end{pgfscope}%
\end{pgfscope}%
\begin{pgfscope}%
\definecolor{textcolor}{rgb}{0.000000,0.000000,0.000000}%
\pgfsetstrokecolor{textcolor}%
\pgfsetfillcolor{textcolor}%
\pgftext[x=0.333333in, y=5.125787in, left, base]{\color{textcolor}{\rmfamily\fontsize{10.000000}{12.000000}\selectfont\catcode`\^=\active\def^{\ifmmode\sp\else\^{}\fi}\catcode`\%=\active\def%{\%}$\mathdefault{100000}$}}%
\end{pgfscope}%
\begin{pgfscope}%
\definecolor{textcolor}{rgb}{0.000000,0.000000,0.000000}%
\pgfsetstrokecolor{textcolor}%
\pgfsetfillcolor{textcolor}%
\pgftext[x=0.277777in,y=2.864012in,,bottom,rotate=90.000000]{\color{textcolor}{\rmfamily\fontsize{14.000000}{16.800000}\selectfont\catcode`\^=\active\def^{\ifmmode\sp\else\^{}\fi}\catcode`\%=\active\def%{\%}f2}}%
\end{pgfscope}%
\begin{pgfscope}%
\pgfpathrectangle{\pgfqpoint{0.847223in}{0.554012in}}{\pgfqpoint{6.200000in}{4.620000in}}%
\pgfusepath{clip}%
\pgfsetrectcap%
\pgfsetroundjoin%
\pgfsetlinewidth{2.509375pt}%
\definecolor{currentstroke}{rgb}{0.000000,0.000000,0.000000}%
\pgfsetstrokecolor{currentstroke}%
\pgfsetdash{}{0pt}%
\pgfpathmoveto{\pgfqpoint{0.847223in}{5.174012in}}%
\pgfpathlineto{\pgfqpoint{0.857889in}{5.073234in}}%
\pgfpathlineto{\pgfqpoint{0.868556in}{4.976403in}}%
\pgfpathlineto{\pgfqpoint{0.879222in}{4.883291in}}%
\pgfpathlineto{\pgfqpoint{0.889889in}{4.793689in}}%
\pgfpathlineto{\pgfqpoint{0.900555in}{4.707402in}}%
\pgfpathlineto{\pgfqpoint{0.911222in}{4.624248in}}%
\pgfpathlineto{\pgfqpoint{0.921888in}{4.544061in}}%
\pgfpathlineto{\pgfqpoint{0.932555in}{4.466684in}}%
\pgfpathlineto{\pgfqpoint{0.943221in}{4.391972in}}%
\pgfpathlineto{\pgfqpoint{0.953887in}{4.319791in}}%
\pgfpathlineto{\pgfqpoint{0.964554in}{4.250012in}}%
\pgfpathlineto{\pgfqpoint{0.975220in}{4.182519in}}%
\pgfpathlineto{\pgfqpoint{0.985887in}{4.117201in}}%
\pgfpathlineto{\pgfqpoint{0.996553in}{4.053954in}}%
\pgfpathlineto{\pgfqpoint{1.007220in}{3.992682in}}%
\pgfpathlineto{\pgfqpoint{1.017886in}{3.933293in}}%
\pgfpathlineto{\pgfqpoint{1.028552in}{3.875702in}}%
\pgfpathlineto{\pgfqpoint{1.039219in}{3.819829in}}%
\pgfpathlineto{\pgfqpoint{1.049885in}{3.765597in}}%
\pgfpathlineto{\pgfqpoint{1.060552in}{3.712936in}}%
\pgfpathlineto{\pgfqpoint{1.071218in}{3.661778in}}%
\pgfpathlineto{\pgfqpoint{1.081885in}{3.612060in}}%
\pgfpathlineto{\pgfqpoint{1.092551in}{3.563721in}}%
\pgfpathlineto{\pgfqpoint{1.103217in}{3.516706in}}%
\pgfpathlineto{\pgfqpoint{1.113884in}{3.470960in}}%
\pgfpathlineto{\pgfqpoint{1.124550in}{3.426433in}}%
\pgfpathlineto{\pgfqpoint{1.135217in}{3.383077in}}%
\pgfpathlineto{\pgfqpoint{1.145883in}{3.340846in}}%
\pgfpathlineto{\pgfqpoint{1.156550in}{3.299697in}}%
\pgfpathlineto{\pgfqpoint{1.167216in}{3.259589in}}%
\pgfpathlineto{\pgfqpoint{1.177882in}{3.220483in}}%
\pgfpathlineto{\pgfqpoint{1.188549in}{3.182341in}}%
\pgfpathlineto{\pgfqpoint{1.199215in}{3.145129in}}%
\pgfpathlineto{\pgfqpoint{1.215215in}{3.090981in}}%
\pgfpathlineto{\pgfqpoint{1.231215in}{3.038743in}}%
\pgfpathlineto{\pgfqpoint{1.247214in}{2.988315in}}%
\pgfpathlineto{\pgfqpoint{1.263214in}{2.939605in}}%
\pgfpathlineto{\pgfqpoint{1.279214in}{2.892526in}}%
\pgfpathlineto{\pgfqpoint{1.295213in}{2.846999in}}%
\pgfpathlineto{\pgfqpoint{1.311213in}{2.802947in}}%
\pgfpathlineto{\pgfqpoint{1.327212in}{2.760300in}}%
\pgfpathlineto{\pgfqpoint{1.343212in}{2.718991in}}%
\pgfpathlineto{\pgfqpoint{1.359212in}{2.678960in}}%
\pgfpathlineto{\pgfqpoint{1.375211in}{2.640146in}}%
\pgfpathlineto{\pgfqpoint{1.391211in}{2.602497in}}%
\pgfpathlineto{\pgfqpoint{1.407211in}{2.565959in}}%
\pgfpathlineto{\pgfqpoint{1.423210in}{2.530485in}}%
\pgfpathlineto{\pgfqpoint{1.439210in}{2.496029in}}%
\pgfpathlineto{\pgfqpoint{1.455210in}{2.462548in}}%
\pgfpathlineto{\pgfqpoint{1.471209in}{2.430000in}}%
\pgfpathlineto{\pgfqpoint{1.487209in}{2.398347in}}%
\pgfpathlineto{\pgfqpoint{1.503209in}{2.367554in}}%
\pgfpathlineto{\pgfqpoint{1.519208in}{2.337584in}}%
\pgfpathlineto{\pgfqpoint{1.535208in}{2.308407in}}%
\pgfpathlineto{\pgfqpoint{1.551208in}{2.279990in}}%
\pgfpathlineto{\pgfqpoint{1.567207in}{2.252304in}}%
\pgfpathlineto{\pgfqpoint{1.583207in}{2.225322in}}%
\pgfpathlineto{\pgfqpoint{1.599206in}{2.199017in}}%
\pgfpathlineto{\pgfqpoint{1.615206in}{2.173364in}}%
\pgfpathlineto{\pgfqpoint{1.631206in}{2.148338in}}%
\pgfpathlineto{\pgfqpoint{1.647205in}{2.123918in}}%
\pgfpathlineto{\pgfqpoint{1.663205in}{2.100081in}}%
\pgfpathlineto{\pgfqpoint{1.679205in}{2.076807in}}%
\pgfpathlineto{\pgfqpoint{1.695204in}{2.054076in}}%
\pgfpathlineto{\pgfqpoint{1.711204in}{2.031870in}}%
\pgfpathlineto{\pgfqpoint{1.727204in}{2.010170in}}%
\pgfpathlineto{\pgfqpoint{1.743203in}{1.988959in}}%
\pgfpathlineto{\pgfqpoint{1.759203in}{1.968221in}}%
\pgfpathlineto{\pgfqpoint{1.780536in}{1.941278in}}%
\pgfpathlineto{\pgfqpoint{1.801869in}{1.915114in}}%
\pgfpathlineto{\pgfqpoint{1.823202in}{1.889694in}}%
\pgfpathlineto{\pgfqpoint{1.844534in}{1.864987in}}%
\pgfpathlineto{\pgfqpoint{1.865867in}{1.840964in}}%
\pgfpathlineto{\pgfqpoint{1.887200in}{1.817597in}}%
\pgfpathlineto{\pgfqpoint{1.908533in}{1.794859in}}%
\pgfpathlineto{\pgfqpoint{1.929866in}{1.772725in}}%
\pgfpathlineto{\pgfqpoint{1.951199in}{1.751171in}}%
\pgfpathlineto{\pgfqpoint{1.972532in}{1.730175in}}%
\pgfpathlineto{\pgfqpoint{1.993864in}{1.709715in}}%
\pgfpathlineto{\pgfqpoint{2.015197in}{1.689771in}}%
\pgfpathlineto{\pgfqpoint{2.036530in}{1.670325in}}%
\pgfpathlineto{\pgfqpoint{2.057863in}{1.651357in}}%
\pgfpathlineto{\pgfqpoint{2.079196in}{1.632849in}}%
\pgfpathlineto{\pgfqpoint{2.100529in}{1.614787in}}%
\pgfpathlineto{\pgfqpoint{2.121862in}{1.597153in}}%
\pgfpathlineto{\pgfqpoint{2.143194in}{1.579932in}}%
\pgfpathlineto{\pgfqpoint{2.164527in}{1.563111in}}%
\pgfpathlineto{\pgfqpoint{2.185860in}{1.546675in}}%
\pgfpathlineto{\pgfqpoint{2.212526in}{1.526652in}}%
\pgfpathlineto{\pgfqpoint{2.239192in}{1.507187in}}%
\pgfpathlineto{\pgfqpoint{2.265858in}{1.488256in}}%
\pgfpathlineto{\pgfqpoint{2.292524in}{1.469838in}}%
\pgfpathlineto{\pgfqpoint{2.319191in}{1.451913in}}%
\pgfpathlineto{\pgfqpoint{2.345857in}{1.434460in}}%
\pgfpathlineto{\pgfqpoint{2.372523in}{1.417462in}}%
\pgfpathlineto{\pgfqpoint{2.399189in}{1.400900in}}%
\pgfpathlineto{\pgfqpoint{2.425855in}{1.384759in}}%
\pgfpathlineto{\pgfqpoint{2.452521in}{1.369022in}}%
\pgfpathlineto{\pgfqpoint{2.479187in}{1.353674in}}%
\pgfpathlineto{\pgfqpoint{2.505853in}{1.338702in}}%
\pgfpathlineto{\pgfqpoint{2.532519in}{1.324091in}}%
\pgfpathlineto{\pgfqpoint{2.564518in}{1.307017in}}%
\pgfpathlineto{\pgfqpoint{2.596518in}{1.290424in}}%
\pgfpathlineto{\pgfqpoint{2.628517in}{1.274291in}}%
\pgfpathlineto{\pgfqpoint{2.660516in}{1.258601in}}%
\pgfpathlineto{\pgfqpoint{2.692516in}{1.243334in}}%
\pgfpathlineto{\pgfqpoint{2.724515in}{1.228474in}}%
\pgfpathlineto{\pgfqpoint{2.756514in}{1.214006in}}%
\pgfpathlineto{\pgfqpoint{2.788514in}{1.199913in}}%
\pgfpathlineto{\pgfqpoint{2.820513in}{1.186181in}}%
\pgfpathlineto{\pgfqpoint{2.857845in}{1.170599in}}%
\pgfpathlineto{\pgfqpoint{2.895178in}{1.155470in}}%
\pgfpathlineto{\pgfqpoint{2.932510in}{1.140773in}}%
\pgfpathlineto{\pgfqpoint{2.969843in}{1.126491in}}%
\pgfpathlineto{\pgfqpoint{3.007175in}{1.112607in}}%
\pgfpathlineto{\pgfqpoint{3.044508in}{1.099104in}}%
\pgfpathlineto{\pgfqpoint{3.081840in}{1.085966in}}%
\pgfpathlineto{\pgfqpoint{3.119173in}{1.073179in}}%
\pgfpathlineto{\pgfqpoint{3.161839in}{1.058977in}}%
\pgfpathlineto{\pgfqpoint{3.204504in}{1.045196in}}%
\pgfpathlineto{\pgfqpoint{3.247170in}{1.031818in}}%
\pgfpathlineto{\pgfqpoint{3.289836in}{1.018824in}}%
\pgfpathlineto{\pgfqpoint{3.332502in}{1.006199in}}%
\pgfpathlineto{\pgfqpoint{3.375167in}{0.993927in}}%
\pgfpathlineto{\pgfqpoint{3.423166in}{0.980524in}}%
\pgfpathlineto{\pgfqpoint{3.471165in}{0.967531in}}%
\pgfpathlineto{\pgfqpoint{3.519164in}{0.954928in}}%
\pgfpathlineto{\pgfqpoint{3.567163in}{0.942698in}}%
\pgfpathlineto{\pgfqpoint{3.615162in}{0.930824in}}%
\pgfpathlineto{\pgfqpoint{3.668494in}{0.918031in}}%
\pgfpathlineto{\pgfqpoint{3.721826in}{0.905640in}}%
\pgfpathlineto{\pgfqpoint{3.775158in}{0.893632in}}%
\pgfpathlineto{\pgfqpoint{3.828491in}{0.881989in}}%
\pgfpathlineto{\pgfqpoint{3.887156in}{0.869584in}}%
\pgfpathlineto{\pgfqpoint{3.945821in}{0.857581in}}%
\pgfpathlineto{\pgfqpoint{4.004487in}{0.845961in}}%
\pgfpathlineto{\pgfqpoint{4.063152in}{0.834705in}}%
\pgfpathlineto{\pgfqpoint{4.127151in}{0.822823in}}%
\pgfpathlineto{\pgfqpoint{4.191149in}{0.811333in}}%
\pgfpathlineto{\pgfqpoint{4.255148in}{0.800218in}}%
\pgfpathlineto{\pgfqpoint{4.319146in}{0.789458in}}%
\pgfpathlineto{\pgfqpoint{4.388478in}{0.778185in}}%
\pgfpathlineto{\pgfqpoint{4.457810in}{0.767289in}}%
\pgfpathlineto{\pgfqpoint{4.527142in}{0.756753in}}%
\pgfpathlineto{\pgfqpoint{4.601807in}{0.745789in}}%
\pgfpathlineto{\pgfqpoint{4.676472in}{0.735200in}}%
\pgfpathlineto{\pgfqpoint{4.751137in}{0.724969in}}%
\pgfpathlineto{\pgfqpoint{4.831135in}{0.714383in}}%
\pgfpathlineto{\pgfqpoint{4.911133in}{0.704167in}}%
\pgfpathlineto{\pgfqpoint{4.996465in}{0.693655in}}%
\pgfpathlineto{\pgfqpoint{5.081796in}{0.683519in}}%
\pgfpathlineto{\pgfqpoint{5.172461in}{0.673142in}}%
\pgfpathlineto{\pgfqpoint{5.263126in}{0.663145in}}%
\pgfpathlineto{\pgfqpoint{5.353790in}{0.653509in}}%
\pgfpathlineto{\pgfqpoint{5.449788in}{0.643677in}}%
\pgfpathlineto{\pgfqpoint{5.545786in}{0.634207in}}%
\pgfpathlineto{\pgfqpoint{5.647117in}{0.624581in}}%
\pgfpathlineto{\pgfqpoint{5.753781in}{0.614837in}}%
\pgfpathlineto{\pgfqpoint{5.860446in}{0.605469in}}%
\pgfpathlineto{\pgfqpoint{5.972443in}{0.596012in}}%
\pgfpathlineto{\pgfqpoint{6.084441in}{0.586924in}}%
\pgfpathlineto{\pgfqpoint{6.183915in}{0.579155in}}%
\pgfpathlineto{\pgfqpoint{6.219152in}{0.576729in}}%
\pgfpathlineto{\pgfqpoint{6.263198in}{0.574126in}}%
\pgfpathlineto{\pgfqpoint{6.307245in}{0.571882in}}%
\pgfpathlineto{\pgfqpoint{6.360100in}{0.569549in}}%
\pgfpathlineto{\pgfqpoint{6.421765in}{0.567206in}}%
\pgfpathlineto{\pgfqpoint{6.492239in}{0.564910in}}%
\pgfpathlineto{\pgfqpoint{6.571523in}{0.562695in}}%
\pgfpathlineto{\pgfqpoint{6.668425in}{0.560391in}}%
\pgfpathlineto{\pgfqpoint{6.774136in}{0.558260in}}%
\pgfpathlineto{\pgfqpoint{6.897465in}{0.556155in}}%
\pgfpathlineto{\pgfqpoint{7.038414in}{0.554127in}}%
\pgfpathlineto{\pgfqpoint{7.047223in}{0.554012in}}%
\pgfpathlineto{\pgfqpoint{7.047223in}{0.554012in}}%
\pgfusepath{stroke}%
\end{pgfscope}%
\begin{pgfscope}%
\pgfsetrectcap%
\pgfsetmiterjoin%
\pgfsetlinewidth{0.803000pt}%
\definecolor{currentstroke}{rgb}{0.000000,0.000000,0.000000}%
\pgfsetstrokecolor{currentstroke}%
\pgfsetdash{}{0pt}%
\pgfpathmoveto{\pgfqpoint{0.847223in}{0.554012in}}%
\pgfpathlineto{\pgfqpoint{0.847223in}{5.174012in}}%
\pgfusepath{stroke}%
\end{pgfscope}%
\begin{pgfscope}%
\pgfsetrectcap%
\pgfsetmiterjoin%
\pgfsetlinewidth{0.803000pt}%
\definecolor{currentstroke}{rgb}{0.000000,0.000000,0.000000}%
\pgfsetstrokecolor{currentstroke}%
\pgfsetdash{}{0pt}%
\pgfpathmoveto{\pgfqpoint{7.047223in}{0.554012in}}%
\pgfpathlineto{\pgfqpoint{7.047223in}{5.174012in}}%
\pgfusepath{stroke}%
\end{pgfscope}%
\begin{pgfscope}%
\pgfsetrectcap%
\pgfsetmiterjoin%
\pgfsetlinewidth{0.803000pt}%
\definecolor{currentstroke}{rgb}{0.000000,0.000000,0.000000}%
\pgfsetstrokecolor{currentstroke}%
\pgfsetdash{}{0pt}%
\pgfpathmoveto{\pgfqpoint{0.847223in}{0.554012in}}%
\pgfpathlineto{\pgfqpoint{7.047223in}{0.554012in}}%
\pgfusepath{stroke}%
\end{pgfscope}%
\begin{pgfscope}%
\pgfsetrectcap%
\pgfsetmiterjoin%
\pgfsetlinewidth{0.803000pt}%
\definecolor{currentstroke}{rgb}{0.000000,0.000000,0.000000}%
\pgfsetstrokecolor{currentstroke}%
\pgfsetdash{}{0pt}%
\pgfpathmoveto{\pgfqpoint{0.847223in}{5.174012in}}%
\pgfpathlineto{\pgfqpoint{7.047223in}{5.174012in}}%
\pgfusepath{stroke}%
\end{pgfscope}%
\begin{pgfscope}%
\pgfsetbuttcap%
\pgfsetmiterjoin%
\definecolor{currentfill}{rgb}{1.000000,1.000000,1.000000}%
\pgfsetfillcolor{currentfill}%
\pgfsetfillopacity{0.800000}%
\pgfsetlinewidth{1.003750pt}%
\definecolor{currentstroke}{rgb}{0.800000,0.800000,0.800000}%
\pgfsetstrokecolor{currentstroke}%
\pgfsetstrokeopacity{0.800000}%
\pgfsetdash{}{0pt}%
\pgfpathmoveto{\pgfqpoint{4.927935in}{2.009846in}}%
\pgfpathlineto{\pgfqpoint{6.911112in}{2.009846in}}%
\pgfpathquadraticcurveto{\pgfqpoint{6.950001in}{2.009846in}}{\pgfqpoint{6.950001in}{2.048735in}}%
\pgfpathlineto{\pgfqpoint{6.950001in}{3.679288in}}%
\pgfpathquadraticcurveto{\pgfqpoint{6.950001in}{3.718177in}}{\pgfqpoint{6.911112in}{3.718177in}}%
\pgfpathlineto{\pgfqpoint{4.927935in}{3.718177in}}%
\pgfpathquadraticcurveto{\pgfqpoint{4.889046in}{3.718177in}}{\pgfqpoint{4.889046in}{3.679288in}}%
\pgfpathlineto{\pgfqpoint{4.889046in}{2.048735in}}%
\pgfpathquadraticcurveto{\pgfqpoint{4.889046in}{2.009846in}}{\pgfqpoint{4.927935in}{2.009846in}}%
\pgfpathlineto{\pgfqpoint{4.927935in}{2.009846in}}%
\pgfpathclose%
\pgfusepath{stroke,fill}%
\end{pgfscope}%
\begin{pgfscope}%
\pgfsetbuttcap%
\pgfsetroundjoin%
\pgfsetlinewidth{1.003750pt}%
\definecolor{currentstroke}{rgb}{1.000000,0.000000,0.000000}%
\pgfsetstrokecolor{currentstroke}%
\pgfsetdash{}{0pt}%
\pgfpathmoveto{\pgfqpoint{5.161268in}{3.510886in}}%
\pgfpathcurveto{\pgfqpoint{5.172318in}{3.510886in}}{\pgfqpoint{5.182917in}{3.515276in}}{\pgfqpoint{5.190731in}{3.523090in}}%
\pgfpathcurveto{\pgfqpoint{5.198544in}{3.530903in}}{\pgfqpoint{5.202935in}{3.541502in}}{\pgfqpoint{5.202935in}{3.552553in}}%
\pgfpathcurveto{\pgfqpoint{5.202935in}{3.563603in}}{\pgfqpoint{5.198544in}{3.574202in}}{\pgfqpoint{5.190731in}{3.582015in}}%
\pgfpathcurveto{\pgfqpoint{5.182917in}{3.589829in}}{\pgfqpoint{5.172318in}{3.594219in}}{\pgfqpoint{5.161268in}{3.594219in}}%
\pgfpathcurveto{\pgfqpoint{5.150218in}{3.594219in}}{\pgfqpoint{5.139619in}{3.589829in}}{\pgfqpoint{5.131805in}{3.582015in}}%
\pgfpathcurveto{\pgfqpoint{5.123992in}{3.574202in}}{\pgfqpoint{5.119601in}{3.563603in}}{\pgfqpoint{5.119601in}{3.552553in}}%
\pgfpathcurveto{\pgfqpoint{5.119601in}{3.541502in}}{\pgfqpoint{5.123992in}{3.530903in}}{\pgfqpoint{5.131805in}{3.523090in}}%
\pgfpathcurveto{\pgfqpoint{5.139619in}{3.515276in}}{\pgfqpoint{5.150218in}{3.510886in}}{\pgfqpoint{5.161268in}{3.510886in}}%
\pgfpathlineto{\pgfqpoint{5.161268in}{3.510886in}}%
\pgfpathclose%
\pgfusepath{stroke}%
\end{pgfscope}%
\begin{pgfscope}%
\definecolor{textcolor}{rgb}{0.000000,0.000000,0.000000}%
\pgfsetstrokecolor{textcolor}%
\pgfsetfillcolor{textcolor}%
\pgftext[x=5.511268in,y=3.501511in,left,base]{\color{textcolor}{\rmfamily\fontsize{14.000000}{16.800000}\selectfont\catcode`\^=\active\def^{\ifmmode\sp\else\^{}\fi}\catcode`\%=\active\def%{\%}Solutions}}%
\end{pgfscope}%
\begin{pgfscope}%
\pgfsetrectcap%
\pgfsetroundjoin%
\pgfsetlinewidth{2.509375pt}%
\definecolor{currentstroke}{rgb}{0.000000,0.000000,0.000000}%
\pgfsetstrokecolor{currentstroke}%
\pgfsetdash{}{0pt}%
\pgfpathmoveto{\pgfqpoint{4.966823in}{3.294567in}}%
\pgfpathlineto{\pgfqpoint{5.161268in}{3.294567in}}%
\pgfpathlineto{\pgfqpoint{5.355712in}{3.294567in}}%
\pgfusepath{stroke}%
\end{pgfscope}%
\begin{pgfscope}%
\definecolor{textcolor}{rgb}{0.000000,0.000000,0.000000}%
\pgfsetstrokecolor{textcolor}%
\pgfsetfillcolor{textcolor}%
\pgftext[x=5.511268in,y=3.226511in,left,base]{\color{textcolor}{\rmfamily\fontsize{14.000000}{16.800000}\selectfont\catcode`\^=\active\def^{\ifmmode\sp\else\^{}\fi}\catcode`\%=\active\def%{\%}Pareto-front}}%
\end{pgfscope}%
\begin{pgfscope}%
\pgfsetbuttcap%
\pgfsetmiterjoin%
\definecolor{currentfill}{rgb}{0.501961,0.501961,0.501961}%
\pgfsetfillcolor{currentfill}%
\pgfsetfillopacity{0.200000}%
\pgfsetlinewidth{1.003750pt}%
\definecolor{currentstroke}{rgb}{0.501961,0.501961,0.501961}%
\pgfsetstrokecolor{currentstroke}%
\pgfsetstrokeopacity{0.200000}%
\pgfsetdash{}{0pt}%
\pgfpathmoveto{\pgfqpoint{4.966823in}{2.951512in}}%
\pgfpathlineto{\pgfqpoint{5.355712in}{2.951512in}}%
\pgfpathlineto{\pgfqpoint{5.355712in}{3.087623in}}%
\pgfpathlineto{\pgfqpoint{4.966823in}{3.087623in}}%
\pgfpathlineto{\pgfqpoint{4.966823in}{2.951512in}}%
\pgfpathclose%
\pgfusepath{stroke,fill}%
\end{pgfscope}%
\begin{pgfscope}%
\definecolor{textcolor}{rgb}{0.000000,0.000000,0.000000}%
\pgfsetstrokecolor{textcolor}%
\pgfsetfillcolor{textcolor}%
\pgftext[x=5.511268in,y=2.951512in,left,base]{\color{textcolor}{\rmfamily\fontsize{14.000000}{16.800000}\selectfont\catcode`\^=\active\def^{\ifmmode\sp\else\^{}\fi}\catcode`\%=\active\def%{\%}Infeasible Space}}%
\end{pgfscope}%
\begin{pgfscope}%
\pgfsetbuttcap%
\pgfsetmiterjoin%
\definecolor{currentfill}{rgb}{0.121569,0.466667,0.705882}%
\pgfsetfillcolor{currentfill}%
\pgfsetfillopacity{0.200000}%
\pgfsetlinewidth{1.003750pt}%
\definecolor{currentstroke}{rgb}{0.121569,0.466667,0.705882}%
\pgfsetstrokecolor{currentstroke}%
\pgfsetstrokeopacity{0.200000}%
\pgfsetdash{}{0pt}%
\pgfpathmoveto{\pgfqpoint{4.966823in}{2.676512in}}%
\pgfpathlineto{\pgfqpoint{5.355712in}{2.676512in}}%
\pgfpathlineto{\pgfqpoint{5.355712in}{2.812623in}}%
\pgfpathlineto{\pgfqpoint{4.966823in}{2.812623in}}%
\pgfpathlineto{\pgfqpoint{4.966823in}{2.676512in}}%
\pgfpathclose%
\pgfusepath{stroke,fill}%
\end{pgfscope}%
\begin{pgfscope}%
\definecolor{textcolor}{rgb}{0.000000,0.000000,0.000000}%
\pgfsetstrokecolor{textcolor}%
\pgfsetfillcolor{textcolor}%
\pgftext[x=5.511268in,y=2.676512in,left,base]{\color{textcolor}{\rmfamily\fontsize{14.000000}{16.800000}\selectfont\catcode`\^=\active\def^{\ifmmode\sp\else\^{}\fi}\catcode`\%=\active\def%{\%}Feasible Space}}%
\end{pgfscope}%
\begin{pgfscope}%
\pgfsetbuttcap%
\pgfsetroundjoin%
\definecolor{currentfill}{rgb}{0.121569,0.466667,0.705882}%
\pgfsetfillcolor{currentfill}%
\pgfsetlinewidth{1.003750pt}%
\definecolor{currentstroke}{rgb}{0.121569,0.466667,0.705882}%
\pgfsetstrokecolor{currentstroke}%
\pgfsetdash{}{0pt}%
\pgfsys@defobject{currentmarker}{\pgfqpoint{-0.114394in}{-0.097310in}}{\pgfqpoint{0.114394in}{0.120281in}}{%
\pgfpathmoveto{\pgfqpoint{0.000000in}{0.120281in}}%
\pgfpathlineto{\pgfqpoint{-0.027005in}{0.037169in}}%
\pgfpathlineto{\pgfqpoint{-0.114394in}{0.037169in}}%
\pgfpathlineto{\pgfqpoint{-0.043695in}{-0.014197in}}%
\pgfpathlineto{\pgfqpoint{-0.070700in}{-0.097310in}}%
\pgfpathlineto{\pgfqpoint{-0.000000in}{-0.045943in}}%
\pgfpathlineto{\pgfqpoint{0.070700in}{-0.097310in}}%
\pgfpathlineto{\pgfqpoint{0.043695in}{-0.014197in}}%
\pgfpathlineto{\pgfqpoint{0.114394in}{0.037169in}}%
\pgfpathlineto{\pgfqpoint{0.027005in}{0.037169in}}%
\pgfpathlineto{\pgfqpoint{0.000000in}{0.120281in}}%
\pgfpathclose%
\pgfusepath{stroke,fill}%
}%
\begin{pgfscope}%
\pgfsys@transformshift{5.161268in}{2.452554in}%
\pgfsys@useobject{currentmarker}{}%
\end{pgfscope}%
\end{pgfscope}%
\begin{pgfscope}%
\definecolor{textcolor}{rgb}{0.000000,0.000000,0.000000}%
\pgfsetstrokecolor{textcolor}%
\pgfsetfillcolor{textcolor}%
\pgftext[x=5.511268in,y=2.401512in,left,base]{\color{textcolor}{\rmfamily\fontsize{14.000000}{16.800000}\selectfont\catcode`\^=\active\def^{\ifmmode\sp\else\^{}\fi}\catcode`\%=\active\def%{\%}Ideal}}%
\end{pgfscope}%
\begin{pgfscope}%
\pgfsetbuttcap%
\pgfsetroundjoin%
\definecolor{currentfill}{rgb}{1.000000,0.498039,0.054902}%
\pgfsetfillcolor{currentfill}%
\pgfsetlinewidth{1.003750pt}%
\definecolor{currentstroke}{rgb}{1.000000,0.498039,0.054902}%
\pgfsetstrokecolor{currentstroke}%
\pgfsetdash{}{0pt}%
\pgfsys@defobject{currentmarker}{\pgfqpoint{-0.098209in}{-0.098209in}}{\pgfqpoint{0.098209in}{0.098209in}}{%
\pgfpathmoveto{\pgfqpoint{-0.098209in}{-0.098209in}}%
\pgfpathlineto{\pgfqpoint{0.098209in}{-0.098209in}}%
\pgfpathlineto{\pgfqpoint{0.098209in}{0.098209in}}%
\pgfpathlineto{\pgfqpoint{-0.098209in}{0.098209in}}%
\pgfpathlineto{\pgfqpoint{-0.098209in}{-0.098209in}}%
\pgfpathclose%
\pgfusepath{stroke,fill}%
}%
\begin{pgfscope}%
\pgfsys@transformshift{5.161268in}{2.177555in}%
\pgfsys@useobject{currentmarker}{}%
\end{pgfscope}%
\end{pgfscope}%
\begin{pgfscope}%
\definecolor{textcolor}{rgb}{0.000000,0.000000,0.000000}%
\pgfsetstrokecolor{textcolor}%
\pgfsetfillcolor{textcolor}%
\pgftext[x=5.511268in,y=2.126513in,left,base]{\color{textcolor}{\rmfamily\fontsize{14.000000}{16.800000}\selectfont\catcode`\^=\active\def^{\ifmmode\sp\else\^{}\fi}\catcode`\%=\active\def%{\%}Nadir}}%
\end{pgfscope}%
\end{pgfpicture}%
\makeatother%
\endgroup%
}
        \caption{\gls{osier} generates a set of \boldorange{co-optimal solutions} called a \boldorange{Pareto front}.}
    \end{figure}
\end{frame}

\begin{frame}
    \frametitle{Dispatch Modeling}
    \begin{figure}
        \centering
        \includegraphics[width=\columnwidth]{images/osier_dispatch_example.png}
        \caption{\gls{osier} dispatch for an energy system with natural gas plus battery storage.}
    \end{figure}
\end{frame}

\begin{frame}[fragile]
    \frametitle{User defined objectives}
    % \scriptsize
    \begin{minted}[ frame=lines, framesep=2mm, baselinestretch=1.2, bgcolor=LightGray,
fontsize=\footnotesize, linenos ]{python} 

    nuclear.readiness = 9
    fusion.readiness = 3

    technology_list = [nuclear, fusion]

    def osier_objective(technology_list, solved_dispatch_model): 
        """ 
            Calculate the capacity-weighted technology readiness 
            score for this energy mix. 
        """

        total_capacity = np.array([t.capacity for t in technology_list]).sum()
        
        objective_value = np.array([t.readiness*t.capacity 
                                    for t in technology_list]).sum()

        return objective_value / total_capacity
    \end{minted}
\end{frame}

\begin{frame}
    \frametitle{Constrained optimization}

        \begin{figure}
            \centering
            \includegraphics[width=0.85\columnwidth]{images/osier_constraint.png}
            \caption{Osier Pareto front with and without emissions constraint.}
        \end{figure}
\end{frame}

\begin{frame}
    \frametitle{High-dimensional Pareto fronts}
    \begin{figure}
        \centering
        \resizebox{\columnwidth}{!}{\input{../docs/figures/04_benchmark_chapter/4_obj_objective_space_MGA.pgf}}
        \caption{Objective space for a four-objective problem.}
        \label{fig:4-obj-space}
    \end{figure}
\end{frame}


\begin{frame}
    \frametitle{Documenation as a feature}
    \url{https://osier.readthedocs.io}
\end{frame}


\begin{frame}
    \frametitle{Goals from Preliminary Exam}

    \begin{block}{Prelim Goals}
        \begin{enumerate}[<+->]
            \item \boldorange{Improve} \gls{osier} by
            \begin{enumerate}
                \item Adding a new, faster, dispatch algorithm
                \item Developing an enhanced \gls{mga} algorithm
            \end{enumerate}
            \item \boldorange{Validate} \gls{osier}'s usefulness through a 
            multiple-case study of energy decision-making in Illinois.
        \end{enumerate}
    \end{block}
    \pause
    \begin{block}{Presentation Objectives}
        \frametitle{Objectives of this presentation}
        \begin{enumerate}[<+->]
            \item Show the technical improvements to \gls{osier}.
            \item Demonstrate how \gls{osier} can be used to evaluate nuclear fuel cycle options.
            \item Share the results from the case study.
        \end{enumerate}
    \end{block}
\end{frame}