\section{\gls{osier} for Nuclear Fuel Cycle Analysis}

\subsection{Framing Questions}
\begin{frame}
    \begin{block}{Current challenges}
    \begin{enumerate}[<+->]
        \item Addressing climate change and data center demand will require a
        lot of new infrastructure.
        \item Nuclear infrastructure is needed
        \cite{julie_kozeracki_pathways_2024}.
        \item \textbf{How can we determine the ``best'' mix of new
        infrastructure?}
        \item \textbf{How can we decide on a fuel cycle to support new
        reactors?}
    \end{enumerate}    
    \end{block}
    \pause
    \begin{block}{Current practice}
        Use energy system optimization models (ESOMs) to identify the mix of
        resources that minimizes total system cost.
    \end{block}
\end{frame}

\subsection{\gls{set} History and Methodology}
\begin{frame}
    \frametitle{Identifying promising fuel cycles}
    \begin{block}{Which fuel cycle should the U.S. pursue?}
        In 2011, the U.S. \Gls{DOE} commissioned the Nuclear Fuel Cycle
        Evaluation and Screening Study (``the Study'').
    \end{block}
    \pause
    \begin{block}{How were recommendations made?}
        % \begin{itemize}
        \begin{itemize}[<+->]
            \item Thousands of fuel cycle options were filtered into 40
            ``evaluation groups'' (e.g., ``EG01'').
            \item These evaluation groups were measured against nine metrics,
            using the \gls{set}.
        \end{itemize}
    \end{block}
\end{frame}

\begin{frame}
    \frametitle{\gls{set} Methodology}
    \begin{columns}
        \column[t]{3.5cm}
        \begin{block}{Metrics in the \gls{set}}
            \begin{enumerate}
                \item Nuclear waste management
                \item Proliferation risk
                \item Nuclear Material security risk
                \item Safety
                \item Environmental impacts
                \item Resource Utilization
                \item Development and deployment risk
                \item Institutional issues
                \item Economics
            \end{enumerate}        
        \end{block}
        \column[t]{6.5cm}
          \begin{figure}[htbp!]
            \begin{center}
            % \resizebox{\columnwidth}{!}{\input{../docs/figures/set_bin_plot.pgf}}
            \resizebox{\columnwidth}{!}{\input{../docs/figures/05_examples_chapter/set_bin_plot.pgf}}
            \end{center}
                \caption{Binned groups for SNF+HLW activity at 100 years.
                Reproduced from \cite{wigeland_nuclear_2014}.}
            \label{fig:set_activity_bin}
        \end{figure}
    \end{columns}
\end{frame}

\subsection{Results and Discussion}
\begin{frame}
    \frametitle{Analyzing fuel cycles through Pareto optimality}
        
  \begin{figure}[htbp!]
    \begin{center}
      \resizebox{0.9\columnwidth}{!}{\input{../docs/figures/05_examples_chapter/full_set_plot.pgf}}
    \end{center}
          \caption{Pareto optimal solutions for the \gls{set}.}
    \label{fig:full-set-pcp}
  \end{figure}
\end{frame}

\begin{frame}
  \frametitle{High Tradeoff (``Knee'') Solution}
  % a comment
        
  \begin{figure}[htbp!]
    \begin{center}
      \resizebox{0.9\columnwidth}{!}{\input{../docs/figures/05_examples_chapter/single-eg_set_plot.pgf}}
    \end{center}
          \caption{EG04 minimizes tradeoff across all objectives and evaluation groups.}
    \label{fig:single-eg-pcp}
  \end{figure}
\end{frame}

\begin{frame}
  \frametitle{Differences between \gls{set} and \gls{osier}}
    \begin{table}
        \centering
        \caption{Summary of non-optimal solutions and disagreement. Highlighted
        rows indicate disagreement between \Gls{osier} and \Gls{set} results.}
        \label{tab:non-optimal-subset}
        \resizebox{\columnwidth}{!}{\input{../docs/tables/non-optimal-subset.tex}}
    \end{table}
\end{frame}

\begin{frame}
  \frametitle{Sources of disagreement between \gls{osier} and \gls{set}}
    \begin{columns}
        \column[t]{3.5cm}
            \begin{enumerate}
                \item Data binning is lossy.
                \item Reliance on expert judgement for metric values and
                weights.
            \end{enumerate}
        \pause        
        \column[t]{6.5cm}
          \begin{figure}[htbp!]
            \begin{center}
            \resizebox{\columnwidth}{!}{\input{../docs/figures/05_examples_chapter/set_bin_plot.pgf}}
            % \resizebox{\columnwidth}{!}{\input{./figures/set_bin_plot.pgf}}
            \end{center}
                \caption{Binned groups for SNF+HLW activity at 100 years.
                Reproduced from \cite{wigeland_nuclear_2014}.}
            \label{fig:set_activity_bin2}
        \end{figure}
    \end{columns}
\end{frame}

\begin{frame}
  \frametitle{How does this enhance justice?}
  \begin{enumerate}[<+->]
    \item The recommendations from the \gls{set} were made based on expert
    technical input.
    \item Public stakeholders can express preferences about metrics.
    \item Including diverse perspectives to analyze tradeoffs enhances
    procedural and recognition justice.
  \end{enumerate}  
\end{frame}