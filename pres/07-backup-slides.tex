\begin{frame}
    Backup Slides
\end{frame}


% Types of Justice

\begin{frame}
    \frametitle{Distributional}
    \begin{columns}
        \column[t]{3cm}
        \begin{figure}
            \centering
            \resizebox{0.7\columnwidth}{!}{
                \begin{tikzpicture}[nodes={text depth=0.25ex,text height=1.25ex distance=1.7cm}]
                    \tikzstyle{every node}=[font=\small] \tikzstyle{vertex} =
                    [circle, draw=black, fill=trueilliniorange]
                    \tikzstyle{unfocus} = [circle, draw=gray, fill=illiniorange]
                    \tikzstyle{hidden} = [draw=none] \tikzstyle{edge} = [<->,
                    very thick]
                    
                    \node[vertex](v3) at (0,2)
                    {\textcolor{black}{\textbf{Distribution}}};
                    \node[unfocus](v2) at (0,0)
                    {\textcolor{gray}{\textbf{Procedural}}}; \node[unfocus](v1)
                    at (0,-2) {\textcolor{gray}{\textbf{Recognition}}};
        
            \end{tikzpicture}}
        \end{figure}
        \column[t]{7cm}
        \begin{block}{Distributional Justice}
            Related to the distribution of burdens and benefits.
        \end{block}
        \begin{block}{Normative Question}
            What is the fairest way to distribute benefits and burdens?
        \end{block}
        \begin{block}{Examples of injustice}
            \begin{itemize}
                \item Dispossession of land and benefits
                \cite{yenneti_spatial_2016,sovacool_dispossessed_2021}. 
                \item Poorer air quality around fossil fuel plants --- primarily
                located in poorer communities \cite{mohai_which_2015}.
                \item Solar panel subsidies and installations benefitting
                wealthier communities \cite{reames_distributional_2020}.
            \end{itemize}
        \end{block}
    \end{columns}
    
\end{frame}

\begin{frame}
    \frametitle{Procedural}
    \begin{columns}
        \column[t]{3cm}
        \begin{figure}
            \centering
            \resizebox{0.7\columnwidth}{!}{
                \begin{tikzpicture}[nodes={text depth=0.25ex,text height=1.25ex distance=1.7cm}]
                    \tikzstyle{every node}=[font=\small] \tikzstyle{vertex} =
                    [circle, draw=black, fill=trueilliniorange]
                    \tikzstyle{unfocus} = [circle, draw=gray, fill=illiniorange]
                    \tikzstyle{hidden} = [draw=none] \tikzstyle{edge} = [<->,
                    very thick]
                    
                    \node[unfocus](v3) at (0,2)
                    {\textcolor{gray}{\textbf{Distribution}}}; \node[vertex](v2)
                    at (0,0) {\textcolor{black}{\textbf{Procedural}}};
                    \node[unfocus](v1) at (0,-2)
                    {\textcolor{gray}{\textbf{Recognition}}};
        
            \end{tikzpicture}}
        \end{figure}
        \column[t]{7cm}
        \begin{block}{Procedural Justice}
            Related to decision-making processes --- method and inclusion.
        \end{block}
        \begin{block}{Normative Question}
            What is the fairest way to make decisions affecting specific groups
            of people?
        \end{block}
        \begin{block}{Examples of injustice}
            \begin{itemize}
                \item Dismissal of testimony for its lack of technical expertise
                \cite{johnson_dakota_2021}. 
                \item Lack of transparency in decision making.
            \end{itemize}
        \end{block}
    \end{columns}
    
\end{frame}

\begin{frame}
    \frametitle{Recognitional}
    \begin{columns}
        \column[t]{3cm}
        \begin{figure}
            \centering
            \resizebox{0.7\columnwidth}{!}{
                \begin{tikzpicture}[nodes={text depth=0.25ex,text height=1.25ex distance=1.7cm}]
                    \tikzstyle{every node}=[font=\small] \tikzstyle{vertex} =
                    [circle, draw=black, fill=trueilliniorange]
                    \tikzstyle{unfocus} = [circle, draw=gray, fill=illiniorange]
                    \tikzstyle{hidden} = [draw=none] \tikzstyle{edge} = [<->,
                    very thick]
                    
                    \node[unfocus](v3) at (0,2)
                    {\textcolor{gray}{\textbf{Distribution}}};
                    \node[unfocus](v2) at (0,0)
                    {\textcolor{gray}{\textbf{Procedural}}}; \node[vertex](v1)
                    at (0,-2) {\textcolor{black}{\textbf{Recognition}}};
        
            \end{tikzpicture}}
        \end{figure}
        \column[t]{7cm}
        \begin{block}{Recognitional Justice}
            Related to social value of people or groups derived from
            relationships, laws, and cultural standing.
        \end{block}
        \begin{block}{Normative Question}
            How much and in what ways should a person or group of people be
            valued?
        \end{block}
        \begin{block}{Examples of injustice}
            \begin{itemize}
                \item Energy policies that interfere with loving relationships
                (e.g., stress from energy
                insecurity) \cite{van_uffelen_revisiting_2022}.
                \item Lack of labor protections for
                workers \cite{van_uffelen_revisiting_2022}.
                \item Exclusion from a policy process\cite{van_uffelen_revisiting_2022}.
            \end{itemize}
        \end{block}
    \end{columns}
    
\end{frame}